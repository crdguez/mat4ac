
        \documentclass[spanish, 11pt]{exam}

        %These tell TeX which packages to use.
        \usepackage{array,epsfig}
        \usepackage{amsmath, textcomp}
        \usepackage{amsfonts}
        \usepackage{amssymb}
        \usepackage{amsxtra}
        \usepackage{amsthm}
        \usepackage{mathrsfs}
        \usepackage{color}
        \usepackage{multicol, xparse}
        \usepackage{verbatim}


        \usepackage[utf8]{inputenc}
        \usepackage[spanish]{babel}
        \usepackage{eurosym}

        \usepackage{graphicx}
        \graphicspath{{../img/}}



        \printanswers
        \nopointsinmargin
        \pointformat{}

        %Pagination stuff.
        %\setlength{\topmargin}{-.3 in}
        %\setlength{\oddsidemargin}{0in}
        %\setlength{\evensidemargin}{0in}
        %\setlength{\textheight}{9.in}
        %\setlength{\textwidth}{6.5in}
        %\pagestyle{empty}

        \let\multicolmulticols\multicols
        \let\endmulticolmulticols\endmulticols
        \RenewDocumentEnvironment{multicols}{mO{}}
         {%
          \ifnum#1=1
            #2%
          \else % More than 1 column
            \multicolmulticols{#1}[#2]
          \fi
         }
         {%
          \ifnum#1=1
          \else % More than 1 column
            \endmulticolmulticols
          \fi
         }
        \renewcommand{\solutiontitle}{\noindent\textbf{Sol:}\enspace}

        \newcommand{\samedir}{\mathbin{\!/\mkern-5mu/\!}}

        \newcommand{\class}{4º ESO}
        \newcommand{\examdate}{\today}

        \newcommand{\tipo}{A}


        \newcommand{\timelimit}{50 minutos}



        \pagestyle{head}
        \firstpageheader{\includegraphics[width=0.2\columnwidth]{header_left}}{\textbf{Departamento de Matemáticas\linebreak \class}\linebreak \examnum}{\includegraphics[width=0.1\columnwidth]{header_right}}
        \runningheader{\class}{\examnum}{Página \thepage\ of \numpages}
        \runningheadrule

        \newcommand{\examnum}{Autoevaluación - Tema 1}
        \begin{document}
        \begin{questions}
        
        \question a01e00 - Calcula la Unión y la Intersección de los siguientes conjuntos (da el resultado en forma de intervalo y de desigualdad):
        \begin{multicols}{1} 
        \begin{parts} \part[1] $A=\left[2, 10\right)$ y $B=\left(7, 12\right)$  \begin{solution}   Unión: \\ $A\cup B=\left[2, 12\right)$ ó $A\cup B= \{x|2 \leq x  < 12\}$ \\Intersección: \\ $A\cap B =\left(7, 10\right)$ ó $A\cap B=\{x|7 < x < 10\}$ \end{solution} 
        \end{parts}
        \end{multicols}
        \question a01e01 - Opera las siguientes potencias con variables::
        \begin{multicols}{2} 
        \begin{parts} \part[1]  $(\frac{8p^5d^2}{3q})^3\cdot(\frac{12p^4q^3}{32d})^4$  \begin{solution}   $\frac{3 d^{2} p^{31} q^{9}}{8}$  \end{solution} 
        \part[1]  $(\frac{12a^6b^4}{36c})^2\cdot(\frac{4a^6b^3}{12c})^3$  \begin{solution}  $\frac{a^{30} b^{17}}{243 c^{5}}$  \end{solution} 
        \end{parts}
        \end{multicols}
        \question a01e02 - Operar en notación científica:
        \begin{multicols}{2} 
        \begin{parts} \part[1]  $\frac{50000000\cdot12000\cdot0.00002\cdot0.0001}{400000\cdot0.00003}$  \begin{solution}  $1\cdot 10^{2}$  \end{solution} \part[1] $\frac{36000000\cdot1000000\cdot0.00002\cdot0.0001}{900000\cdot0.00004}$  \begin{solution}  $2\cdot 10^{3}$  \end{solution} 
        \end{parts}
        \end{multicols}
        \question a01e03 - Opera y simplifica:
        \begin{multicols}{2} 
        \begin{parts} \part[1]  $\sqrt[4]{\frac{49}{4}\sqrt[3]{\frac{4}{49}}}$  \begin{solution}  $\sqrt[3]{\frac{7}{2}}$  \end{solution} \part[1]  $ 3\sqrt{20}-3\sqrt{80}+7\sqrt{125}-2\sqrt{5}$  \begin{solution}  $27 \sqrt{5}$  \end{solution}
        \part[1]  $\sqrt{148}\cdot2\sqrt{16}$  \begin{solution}  $16 \sqrt{37}$  \end{solution} \part[1]  $ \frac{1}{4}\sqrt {3125}  - 4\sqrt {20}  - \frac{3}{4}\sqrt {45} $  \begin{solution}  $ - 4 \sqrt{5} $  \end{solution}
        \part[1]  $ 2\sqrt {3125}  + 3\sqrt {20}  - 12\sqrt {45} $  \begin{solution}  $ 20 \sqrt{5} $  \end{solution} \part[1] $\sqrt{\frac{1}{2}}+\sqrt{8}$   \begin{solution} $\frac{5 \sqrt{2}}{2}$   \end{solution}
        \part[1] $(2+x)\cdot\sqrt{\frac{2-x}{2+x}}$   \begin{solution} $\sqrt{4-x^2}$   \end{solution}
        \part[1]  $\sqrt[3]{\frac{x}{2}\sqrt{\frac{2}{x}}}$   \begin{solution} $\sqrt[6]{\frac{x}{2}}$   \end{solution}
        \end{parts}
        \end{multicols}
        \question a01e04 - Calcula aplicando la definición de logaritmo:
        \begin{multicols}{2} 
        \begin{parts} \part[1]  $\log_3{81}$  \begin{solution}  $4$  \end{solution} \part[1]  $\log_2{0.25}$  \begin{solution}  $-2$  \end{solution} \part[1] $\log_3{1/27}$  \begin{solution}  $-3$  \end{solution} \part[1]  $\log_9{3}$  \begin{solution}  $\frac{1}{2}$  \end{solution} 
        \part[1]   $\log_{0.5}{1/16}$  \begin{solution}  $4$  \end{solution} 
        \part[1]  $\log_{0.5}{16}$  \begin{solution}  $-4$  \end{solution} 
        \part[1] $\log_2 \frac{1}{4} -  \log_5 0.2 +\log_4 \frac{1}{16} - \log_2 0.5$  \begin{solution}  $\log_2 \frac{1}{4} -  \log_5 0.2 +\log_4 \frac{1}{16} - \log_2 0.5= (-2.0 )+( 1.0 )+( -2.0 )+( 1.0)=-2.0$  \end{solution} 
        
        \end{parts}
        \end{multicols}
        \question a01e05 - Utilizando las propiedades de los logaritmos:
        \begin{multicols}{1} 
        \begin{parts} \part[1]  Sabiendo que $\log x=2$ y $\log y=-1$, calcula $$\log (\frac{100\cdot x^2}{\sqrt{x\cdot y}})$$  \begin{solution}  $\log (\frac{100\cdot x^2}{\sqrt{x\cdot y}})=\frac{3 \log{\left (x \right )}}{2} - \frac{\log{\left (y \right )}}{2} + 2=2 - \frac{-1}{2} + \frac{3 \cdot 2}{2}=\frac{11}{2}$  \end{solution}         \end{parts}
        \end{multicols}

    \end{questions}
    \end{document}
    