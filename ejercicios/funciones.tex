\documentclass[spanish, 11pt]{exam}

%These tell TeX which packages to use.
\usepackage{array,epsfig}
\usepackage{amsmath, textcomp}
\usepackage{amsfonts}
\usepackage{amssymb}
\usepackage{amsxtra}
\usepackage{amsthm}
\usepackage{mathrsfs}
\usepackage{color}
\usepackage{multicol}
\usepackage{verbatim}
\usepackage{svg}

\usepackage{pgf,tikz}
\usetikzlibrary{shapes, calc, shapes, arrows, math, babel}


\usepackage[utf8]{inputenc}
\usepackage[spanish]{babel}
\usepackage{eurosym}

\usepackage{graphicx}
\graphicspath{{../img/}}



\printanswers
\nopointsinmargin
\pointformat{}

%Pagination stuff.
%\setlength{\topmargin}{-.3 in}
%\setlength{\oddsidemargin}{0in}
%\setlength{\evensidemargin}{0in}
%\setlength{\textheight}{9.in}
%\setlength{\textwidth}{6.5in}
%\pagestyle{empty}

\renewcommand{\solutiontitle}{\noindent\textbf{Sol:}\enspace}

\newcommand{\samedir}{\mathbin{\!/\mkern-5mu/\!}}

\newcommand{\class}{4º Académicas}
\newcommand{\examdate}{\today}
\newcommand{\examnum}{Funciones}
\newcommand{\tipo}{A}


\newcommand{\timelimit}{50 minutos}

\newcommand\xa{3} %tamaño ejes par tikz

\pagestyle{head}
\firstpageheader{\includegraphics[width=0.2\columnwidth]{header_left}}{\textbf{Departamento de Matemáticas\linebreak \class}\linebreak \examnum}{\includegraphics[width=0.1\columnwidth]{header_right}}
\runningheader{\class}{\examnum}{Página \thepage\ of \numpages}
\runningheadrule

\DeclareUnicodeCharacter{2212}{-}

\begin{document}

\begin{questions}

\question Calcula el dominio de las siguientes funciones:
\begin{multicols}{2}
\begin{parts} \part[1] $f(x)=\dfrac{x+13}{x^4+x^3-3x^2-3x}$\begin{solution} $\left(-\infty, - \sqrt{3}\right) \cup \left(- \sqrt{3}, -1\right) \cup \left(-1, 0\right) \cup \left(0, \sqrt{3}\right) \cup \left(\sqrt{3}, \infty\right)$\end{solution} \part[1] $f(x)=x^6+x^2-2$\begin{solution} $\mathbb{R}$\end{solution} \part[1] $f(x)=\dfrac{7x+9}{x^3+8}$\begin{solution} $\left(-\infty, -2\right) \cup \left(-2, \infty\right)$\end{solution} \part[1] $f(x)=\sqrt{\dfrac{x-1}{x}}$\begin{solution} $\left(-\infty, 0\right) \cup \left[1, \infty\right)$\end{solution} \part[1] $f(x)=\sqrt[3]{\dfrac{x-1}{x}}$\begin{solution} $\left(-\infty, 0\right) \cup \left(0, \infty\right)$\end{solution} \part[1] $f(x)=\sqrt[4]{\dfrac{x(x+7)}{x^2+5x+6}}$\begin{solution} $\left(-\infty, -3\right) \cup \left(-3, -2\right) \cup \left(-2, \infty\right)$\end{solution} \part[1] $f(x)=\dfrac{x^3-6x^2+4x+8}{x^3-x^2-9x+9}$\begin{solution} $\left(-\infty, -3\right) \cup \left(-3, 1\right) \cup \left(1, 3\right) \cup \left(3, \infty\right)$\end{solution} \part[1] $f(x)=\dfrac{1}{4x^2-1}$\begin{solution} $\left(-\infty, - \dfrac{1}{2}\right) \cup \left(- \dfrac{1}{2}, \dfrac{1}{2}\right) \cup \left(\dfrac{1}{2}, \infty\right)$\end{solution} \part[1] $f(x)=\dfrac{1}{\sqrt[4]{9-x^2}}$\begin{solution} $\left(-\infty, -3\right) \cup \left(-3, 3\right) \cup \left(3, \infty\right)$\end{solution} \part[1] $f(x)=\dfrac{2x+7}{\sqrt[3]{9-x}}$\begin{solution} $\left(-\infty, 9\right) \cup \left(9, \infty\right)$\end{solution} \part[1] $f(x)=\dfrac{x^2-5x+6}{\sqrt{x^4-1}}$\begin{solution} $\left(-\infty, -1\right) \cup \left(1, \infty\right)$\end{solution} \part[1] $f(x)=\sqrt{-2x^2+5x-3}$\begin{solution} $\left[1, \dfrac{3}{2}\right]$\end{solution} \part[1] $f(x)=\dfrac{x^2-3}{x^3-2x^2-x+2}$\begin{solution} $\left(-\infty, -1\right) \cup \left(-1, 1\right) \cup \left(1, 2\right) \cup \left(2, \infty\right)$\end{solution} \part[1] $f(x)=\dfrac{5x^3-8}{1+x+x^2}$\begin{solution} $\mathbb{R}$\end{solution} \part[1] $f(x)=\dfrac{x-1}{x^4-7x^2-144}$\begin{solution} $\left(-\infty, -4\right) \cup \left(-4, 4\right) \cup \left(4, \infty\right)$\end{solution} \part[1] $f(x)=\dfrac{7x+9}{81x^4-16}$\begin{solution} $\left(-\infty, - \dfrac{2}{3}\right) \cup \left(- \dfrac{2}{3}, \dfrac{2}{3}\right) \cup \left(\dfrac{2}{3}, \infty\right)$\end{solution} \part[1] $f(x)=\sqrt[3]{\dfrac{x^6-5x+1}{x^2-4x+4}}$\begin{solution} $\left(-\infty, 2\right) \cup \left(2, \infty\right)$\end{solution} \part[1] $f(x)=\dfrac{\sqrt{x^2-4x-5}}{x^2+2x+1}$\begin{solution} $\left(-\infty, -1\right) \cup \left[5, \infty\right)$\end{solution} \end{parts} 
\end{multicols}

\question Calcular el dominio de las siguientes funciones:
\begin{multicols}{2}
\begin{parts}
\part[] $f(x)=\dfrac{x+13}{x^4+x^3-3x^2-3x} $ 
%g = (x+13)/(x**4+x**3-3*x**2-3*x)
%singularities(g, x)
\begin{solution}$\mathbb{R}-\left\{-1, 0, - \sqrt{3}, \sqrt{3}\right\}$ \\
\includegraphics[width=1\columnwidth]{funcion1a}
 \end{solution}

\part[] $f(x)=x^6+x^2-2$ 
\begin{solution} $\mathbb{R}-\emptyset$ \end{solution}

\part[] $f(x)=\dfrac{7x+9}{x^3+8} $ 
%g = (7*x+9)/(x**3+8)
%singularities(g,x)	
\begin{solution} $ \mathbb{R}-\left\{-2, 1 - \sqrt{3} i, 1 + \sqrt{3} i\right\}=\mathbb{R}-\left\{-2\right\}$\\
\includegraphics[width=1\columnwidth]{funcion1c}
 \end{solution}
 
%from sympy.solvers.inequalities import reduce_rational_inequalities 
%Complement(S.Reals, reduce_rational_inequalities([[(x-1)/(x) < 0]], x,relational=0))

\part[] $ f(x)=\sqrt{\dfrac{x-1}{x}}$ 
\begin{solution} $\mathbb{R}-\left[0,1\right)=\left(-\infty, 0\right) \cup \left[1, \infty\right)$ \\
\includegraphics[width=1\columnwidth]{funcion1d}
\end{solution}

\part[] $ f(x)=\sqrt[3]{\dfrac{x-1}{x}} $ 
\begin{solution} $\mathbb{R}-\left\{0\right\}$\\
\includegraphics[width=1\columnwidth]{funcion1e}\end{solution}

\part[] $f(x)=\sqrt[4]{\dfrac{x(x+7)}{x^2+5x+6}} $ 
\begin{solution} \end{solution}

\part[] $f(x)=\dfrac{x^3-6x^2+4x+8}{x^3-x^2-9x+9}$ 
% g = (x**3-6*x**2+4*x+8)/(x**3-x**2-9*x+9)
% singularities(g, x)
\begin{solution} $\left\{-3, 1, 3\right\}$ \\

\includegraphics[width=1\columnwidth]{funcion1g}
\end{solution}

\part[] $ f(x)=\dfrac{1}{4x^2-1}$ 
\begin{solution} \end{solution}

\part[] $f(x)=\dfrac{1}{\sqrt[4]{9-x^2}} $ 
\begin{solution} \end{solution}

\part[] $ f(x)=\dfrac{2x+7}{\sqrt[3]{9-x}}$ 
\begin{solution} \end{solution}

\part[] $ f(x)=\dfrac{x^2-5x+6}{\sqrt{x^4-1}}$ 
\begin{solution} \end{solution}

\part[] $f(x)=\sqrt{-2x^2+5x-3} $ 
\begin{solution} \end{solution}

\part[] $f(x)=\dfrac{x^2-3}{x^3-2x^2-x+2} $ 
\begin{solution} \end{solution}

\part[] $ f(x)=\dfrac{5x^3-8}{1+x+x^2}$ 
\begin{solution} \end{solution}

\part[] $ f(x)=\dfrac{x-1}{x^4-7x^2-144}$ 
\begin{solution} \end{solution}

\part[] $f(x)=\dfrac{7x+9}{81x^4-16} $ 
\begin{solution} \end{solution}

\part[] $f(x)=\sqrt[3]{\dfrac{x^6-5x+1}{x^2-4x+4}} $ 
\begin{solution} \end{solution}

\part[] $f(x)=\dfrac{\sqrt{x^2-4x-5}}{x^2+2x+1} $ 
\begin{solution} \end{solution}

\end{parts}
\end{multicols}

\question Representa las siguientes funciones e indica sus propiedades\begin{parts} \part[1] $f(x)=x^2-4x-5$\begin{solution} \scalebox{.6}{%% Creator: Matplotlib, PGF backend
%%
%% To include the figure in your LaTeX document, write
%%   \input{<filename>.pgf}
%%
%% Make sure the required packages are loaded in your preamble
%%   \usepackage{pgf}
%%
%% and, on pdftex
%%   \usepackage[utf8]{inputenc}\DeclareUnicodeCharacter{2212}{-}
%%
%% or, on luatex and xetex
%%   \usepackage{unicode-math}
%%
%% Figures using additional raster images can only be included by \input if
%% they are in the same directory as the main LaTeX file. For loading figures
%% from other directories you can use the `import` package
%%   \usepackage{import}
%%
%% and then include the figures with
%%   \import{<path to file>}{<filename>.pgf}
%%
%% Matplotlib used the following preamble
%%   \usepackage{fontspec}
%%   \setmainfont{DejaVuSerif.ttf}[Path=/home/hp/Mis_aplicaciones/anaconda3/lib/python3.6/site-packages/matplotlib/mpl-data/fonts/ttf/]
%%   \setsansfont{DejaVuSans.ttf}[Path=/home/hp/Mis_aplicaciones/anaconda3/lib/python3.6/site-packages/matplotlib/mpl-data/fonts/ttf/]
%%   \setmonofont{DejaVuSansMono.ttf}[Path=/home/hp/Mis_aplicaciones/anaconda3/lib/python3.6/site-packages/matplotlib/mpl-data/fonts/ttf/]
%%
\begingroup%
\makeatletter%
\begin{pgfpicture}%
\pgfpathrectangle{\pgfpointorigin}{\pgfqpoint{6.400000in}{4.800000in}}%
\pgfusepath{use as bounding box, clip}%
\begin{pgfscope}%
\pgfsetbuttcap%
\pgfsetmiterjoin%
\definecolor{currentfill}{rgb}{1.000000,1.000000,1.000000}%
\pgfsetfillcolor{currentfill}%
\pgfsetlinewidth{0.000000pt}%
\definecolor{currentstroke}{rgb}{1.000000,1.000000,1.000000}%
\pgfsetstrokecolor{currentstroke}%
\pgfsetdash{}{0pt}%
\pgfpathmoveto{\pgfqpoint{0.000000in}{0.000000in}}%
\pgfpathlineto{\pgfqpoint{6.400000in}{0.000000in}}%
\pgfpathlineto{\pgfqpoint{6.400000in}{4.800000in}}%
\pgfpathlineto{\pgfqpoint{0.000000in}{4.800000in}}%
\pgfpathclose%
\pgfusepath{fill}%
\end{pgfscope}%
\begin{pgfscope}%
\pgfsetbuttcap%
\pgfsetmiterjoin%
\definecolor{currentfill}{rgb}{1.000000,1.000000,1.000000}%
\pgfsetfillcolor{currentfill}%
\pgfsetlinewidth{0.000000pt}%
\definecolor{currentstroke}{rgb}{0.000000,0.000000,0.000000}%
\pgfsetstrokecolor{currentstroke}%
\pgfsetstrokeopacity{0.000000}%
\pgfsetdash{}{0pt}%
\pgfpathmoveto{\pgfqpoint{1.049063in}{0.235000in}}%
\pgfpathlineto{\pgfqpoint{5.409063in}{0.235000in}}%
\pgfpathlineto{\pgfqpoint{5.409063in}{4.595000in}}%
\pgfpathlineto{\pgfqpoint{1.049063in}{4.595000in}}%
\pgfpathclose%
\pgfusepath{fill}%
\end{pgfscope}%
\begin{pgfscope}%
\pgfpathrectangle{\pgfqpoint{1.049063in}{0.235000in}}{\pgfqpoint{4.360000in}{4.360000in}}%
\pgfusepath{clip}%
\pgfsetbuttcap%
\pgfsetmiterjoin%
\definecolor{currentfill}{rgb}{0.000000,0.000000,1.000000}%
\pgfsetfillcolor{currentfill}%
\pgfsetlinewidth{0.000000pt}%
\definecolor{currentstroke}{rgb}{0.000000,0.000000,0.000000}%
\pgfsetstrokecolor{currentstroke}%
\pgfsetstrokeopacity{0.000000}%
\pgfsetdash{}{0pt}%
\pgfpathmoveto{\pgfqpoint{2.952304in}{3.636995in}}%
\pgfpathlineto{\pgfqpoint{2.952304in}{3.641253in}}%
\pgfpathlineto{\pgfqpoint{2.956562in}{3.641253in}}%
\pgfpathlineto{\pgfqpoint{2.956562in}{3.636995in}}%
\pgfpathmoveto{\pgfqpoint{2.952304in}{3.641253in}}%
\pgfpathlineto{\pgfqpoint{2.952304in}{3.641253in}}%
\pgfpathlineto{\pgfqpoint{2.952304in}{3.645511in}}%
\pgfpathlineto{\pgfqpoint{2.956562in}{3.645511in}}%
\pgfpathlineto{\pgfqpoint{2.956562in}{3.641253in}}%
\pgfpathmoveto{\pgfqpoint{2.952304in}{3.645511in}}%
\pgfpathlineto{\pgfqpoint{2.952304in}{3.645511in}}%
\pgfpathlineto{\pgfqpoint{2.952304in}{3.649769in}}%
\pgfpathlineto{\pgfqpoint{2.956562in}{3.649769in}}%
\pgfpathlineto{\pgfqpoint{2.956562in}{3.645511in}}%
\pgfpathmoveto{\pgfqpoint{2.952304in}{3.649769in}}%
\pgfpathlineto{\pgfqpoint{2.952304in}{3.649769in}}%
\pgfpathlineto{\pgfqpoint{2.952304in}{3.654027in}}%
\pgfpathlineto{\pgfqpoint{2.956562in}{3.654027in}}%
\pgfpathlineto{\pgfqpoint{2.956562in}{3.649769in}}%
\pgfpathmoveto{\pgfqpoint{2.952304in}{3.654027in}}%
\pgfpathlineto{\pgfqpoint{2.952304in}{3.654027in}}%
\pgfpathlineto{\pgfqpoint{2.952304in}{3.658284in}}%
\pgfpathlineto{\pgfqpoint{2.956562in}{3.658284in}}%
\pgfpathlineto{\pgfqpoint{2.956562in}{3.654027in}}%
\pgfpathmoveto{\pgfqpoint{2.952304in}{3.658284in}}%
\pgfpathlineto{\pgfqpoint{2.952304in}{3.658284in}}%
\pgfpathlineto{\pgfqpoint{2.952304in}{3.662542in}}%
\pgfpathlineto{\pgfqpoint{2.956562in}{3.662542in}}%
\pgfpathlineto{\pgfqpoint{2.956562in}{3.658284in}}%
\pgfpathmoveto{\pgfqpoint{2.952304in}{3.662542in}}%
\pgfpathlineto{\pgfqpoint{2.952304in}{3.662542in}}%
\pgfpathlineto{\pgfqpoint{2.952304in}{3.666800in}}%
\pgfpathlineto{\pgfqpoint{2.956562in}{3.666800in}}%
\pgfpathlineto{\pgfqpoint{2.956562in}{3.662542in}}%
\pgfpathmoveto{\pgfqpoint{2.952304in}{3.666800in}}%
\pgfpathlineto{\pgfqpoint{2.952304in}{3.666800in}}%
\pgfpathlineto{\pgfqpoint{2.952304in}{3.671058in}}%
\pgfpathlineto{\pgfqpoint{2.956562in}{3.671058in}}%
\pgfpathlineto{\pgfqpoint{2.956562in}{3.666800in}}%
\pgfpathmoveto{\pgfqpoint{2.952304in}{3.671058in}}%
\pgfpathlineto{\pgfqpoint{2.952304in}{3.671058in}}%
\pgfpathlineto{\pgfqpoint{2.952304in}{3.675316in}}%
\pgfpathlineto{\pgfqpoint{2.956562in}{3.675316in}}%
\pgfpathlineto{\pgfqpoint{2.956562in}{3.671058in}}%
\pgfpathmoveto{\pgfqpoint{2.948046in}{3.679573in}}%
\pgfpathlineto{\pgfqpoint{2.948046in}{3.679573in}}%
\pgfpathlineto{\pgfqpoint{2.948046in}{3.683831in}}%
\pgfpathlineto{\pgfqpoint{2.952304in}{3.683831in}}%
\pgfpathlineto{\pgfqpoint{2.952304in}{3.679573in}}%
\pgfpathmoveto{\pgfqpoint{2.952304in}{3.675316in}}%
\pgfpathlineto{\pgfqpoint{2.952304in}{3.675316in}}%
\pgfpathlineto{\pgfqpoint{2.952304in}{3.679573in}}%
\pgfpathlineto{\pgfqpoint{2.956562in}{3.679573in}}%
\pgfpathlineto{\pgfqpoint{2.956562in}{3.675316in}}%
\pgfpathmoveto{\pgfqpoint{2.952304in}{3.679573in}}%
\pgfpathlineto{\pgfqpoint{2.952304in}{3.679573in}}%
\pgfpathlineto{\pgfqpoint{2.952304in}{3.683831in}}%
\pgfpathlineto{\pgfqpoint{2.956562in}{3.683831in}}%
\pgfpathlineto{\pgfqpoint{2.956562in}{3.679573in}}%
\pgfpathmoveto{\pgfqpoint{2.948046in}{3.683831in}}%
\pgfpathlineto{\pgfqpoint{2.948046in}{3.683831in}}%
\pgfpathlineto{\pgfqpoint{2.948046in}{3.688089in}}%
\pgfpathlineto{\pgfqpoint{2.952304in}{3.688089in}}%
\pgfpathlineto{\pgfqpoint{2.952304in}{3.683831in}}%
\pgfpathmoveto{\pgfqpoint{2.948046in}{3.688089in}}%
\pgfpathlineto{\pgfqpoint{2.948046in}{3.688089in}}%
\pgfpathlineto{\pgfqpoint{2.948046in}{3.692347in}}%
\pgfpathlineto{\pgfqpoint{2.952304in}{3.692347in}}%
\pgfpathlineto{\pgfqpoint{2.952304in}{3.688089in}}%
\pgfpathmoveto{\pgfqpoint{2.948046in}{3.692347in}}%
\pgfpathlineto{\pgfqpoint{2.948046in}{3.692347in}}%
\pgfpathlineto{\pgfqpoint{2.948046in}{3.696604in}}%
\pgfpathlineto{\pgfqpoint{2.952304in}{3.696604in}}%
\pgfpathlineto{\pgfqpoint{2.952304in}{3.692347in}}%
\pgfpathmoveto{\pgfqpoint{2.948046in}{3.696604in}}%
\pgfpathlineto{\pgfqpoint{2.948046in}{3.696604in}}%
\pgfpathlineto{\pgfqpoint{2.948046in}{3.700862in}}%
\pgfpathlineto{\pgfqpoint{2.952304in}{3.700862in}}%
\pgfpathlineto{\pgfqpoint{2.952304in}{3.696604in}}%
\pgfpathmoveto{\pgfqpoint{2.948046in}{3.700862in}}%
\pgfpathlineto{\pgfqpoint{2.948046in}{3.700862in}}%
\pgfpathlineto{\pgfqpoint{2.948046in}{3.705120in}}%
\pgfpathlineto{\pgfqpoint{2.952304in}{3.705120in}}%
\pgfpathlineto{\pgfqpoint{2.952304in}{3.700862in}}%
\pgfpathmoveto{\pgfqpoint{2.948046in}{3.705120in}}%
\pgfpathlineto{\pgfqpoint{2.948046in}{3.705120in}}%
\pgfpathlineto{\pgfqpoint{2.948046in}{3.709378in}}%
\pgfpathlineto{\pgfqpoint{2.952304in}{3.709378in}}%
\pgfpathlineto{\pgfqpoint{2.952304in}{3.705120in}}%
\pgfpathmoveto{\pgfqpoint{2.943788in}{3.717893in}}%
\pgfpathlineto{\pgfqpoint{2.943788in}{3.717893in}}%
\pgfpathlineto{\pgfqpoint{2.943788in}{3.722151in}}%
\pgfpathlineto{\pgfqpoint{2.948046in}{3.722151in}}%
\pgfpathlineto{\pgfqpoint{2.948046in}{3.717893in}}%
\pgfpathmoveto{\pgfqpoint{2.943788in}{3.722151in}}%
\pgfpathlineto{\pgfqpoint{2.943788in}{3.722151in}}%
\pgfpathlineto{\pgfqpoint{2.943788in}{3.726409in}}%
\pgfpathlineto{\pgfqpoint{2.948046in}{3.726409in}}%
\pgfpathlineto{\pgfqpoint{2.948046in}{3.722151in}}%
\pgfpathmoveto{\pgfqpoint{2.948046in}{3.709378in}}%
\pgfpathlineto{\pgfqpoint{2.948046in}{3.709378in}}%
\pgfpathlineto{\pgfqpoint{2.948046in}{3.713636in}}%
\pgfpathlineto{\pgfqpoint{2.952304in}{3.713636in}}%
\pgfpathlineto{\pgfqpoint{2.952304in}{3.709378in}}%
\pgfpathmoveto{\pgfqpoint{2.948046in}{3.713636in}}%
\pgfpathlineto{\pgfqpoint{2.948046in}{3.713636in}}%
\pgfpathlineto{\pgfqpoint{2.948046in}{3.717893in}}%
\pgfpathlineto{\pgfqpoint{2.952304in}{3.717893in}}%
\pgfpathlineto{\pgfqpoint{2.952304in}{3.713636in}}%
\pgfpathmoveto{\pgfqpoint{2.948046in}{3.717893in}}%
\pgfpathlineto{\pgfqpoint{2.948046in}{3.717893in}}%
\pgfpathlineto{\pgfqpoint{2.948046in}{3.722151in}}%
\pgfpathlineto{\pgfqpoint{2.952304in}{3.722151in}}%
\pgfpathlineto{\pgfqpoint{2.952304in}{3.717893in}}%
\pgfpathmoveto{\pgfqpoint{2.943788in}{3.726409in}}%
\pgfpathlineto{\pgfqpoint{2.943788in}{3.726409in}}%
\pgfpathlineto{\pgfqpoint{2.943788in}{3.730667in}}%
\pgfpathlineto{\pgfqpoint{2.948046in}{3.730667in}}%
\pgfpathlineto{\pgfqpoint{2.948046in}{3.726409in}}%
\pgfpathmoveto{\pgfqpoint{2.943788in}{3.730667in}}%
\pgfpathlineto{\pgfqpoint{2.943788in}{3.730667in}}%
\pgfpathlineto{\pgfqpoint{2.943788in}{3.734924in}}%
\pgfpathlineto{\pgfqpoint{2.948046in}{3.734924in}}%
\pgfpathlineto{\pgfqpoint{2.948046in}{3.730667in}}%
\pgfpathmoveto{\pgfqpoint{2.943788in}{3.734924in}}%
\pgfpathlineto{\pgfqpoint{2.943788in}{3.734924in}}%
\pgfpathlineto{\pgfqpoint{2.943788in}{3.739182in}}%
\pgfpathlineto{\pgfqpoint{2.948046in}{3.739182in}}%
\pgfpathlineto{\pgfqpoint{2.948046in}{3.734924in}}%
\pgfpathmoveto{\pgfqpoint{2.943788in}{3.739182in}}%
\pgfpathlineto{\pgfqpoint{2.943788in}{3.739182in}}%
\pgfpathlineto{\pgfqpoint{2.943788in}{3.743440in}}%
\pgfpathlineto{\pgfqpoint{2.948046in}{3.743440in}}%
\pgfpathlineto{\pgfqpoint{2.948046in}{3.739182in}}%
\pgfpathmoveto{\pgfqpoint{2.943788in}{3.743440in}}%
\pgfpathlineto{\pgfqpoint{2.943788in}{3.743440in}}%
\pgfpathlineto{\pgfqpoint{2.943788in}{3.747698in}}%
\pgfpathlineto{\pgfqpoint{2.948046in}{3.747698in}}%
\pgfpathlineto{\pgfqpoint{2.948046in}{3.743440in}}%
\pgfpathmoveto{\pgfqpoint{2.943788in}{3.747698in}}%
\pgfpathlineto{\pgfqpoint{2.943788in}{3.747698in}}%
\pgfpathlineto{\pgfqpoint{2.943788in}{3.751955in}}%
\pgfpathlineto{\pgfqpoint{2.948046in}{3.751955in}}%
\pgfpathlineto{\pgfqpoint{2.948046in}{3.747698in}}%
\pgfpathmoveto{\pgfqpoint{2.939530in}{3.756213in}}%
\pgfpathlineto{\pgfqpoint{2.939530in}{3.756213in}}%
\pgfpathlineto{\pgfqpoint{2.939530in}{3.760471in}}%
\pgfpathlineto{\pgfqpoint{2.943788in}{3.760471in}}%
\pgfpathlineto{\pgfqpoint{2.943788in}{3.756213in}}%
\pgfpathmoveto{\pgfqpoint{2.943788in}{3.751955in}}%
\pgfpathlineto{\pgfqpoint{2.943788in}{3.751955in}}%
\pgfpathlineto{\pgfqpoint{2.943788in}{3.756213in}}%
\pgfpathlineto{\pgfqpoint{2.948046in}{3.756213in}}%
\pgfpathlineto{\pgfqpoint{2.948046in}{3.751955in}}%
\pgfpathmoveto{\pgfqpoint{2.943788in}{3.756213in}}%
\pgfpathlineto{\pgfqpoint{2.943788in}{3.756213in}}%
\pgfpathlineto{\pgfqpoint{2.943788in}{3.760471in}}%
\pgfpathlineto{\pgfqpoint{2.948046in}{3.760471in}}%
\pgfpathlineto{\pgfqpoint{2.948046in}{3.756213in}}%
\pgfpathmoveto{\pgfqpoint{2.939530in}{3.760471in}}%
\pgfpathlineto{\pgfqpoint{2.939530in}{3.760471in}}%
\pgfpathlineto{\pgfqpoint{2.939530in}{3.764729in}}%
\pgfpathlineto{\pgfqpoint{2.943788in}{3.764729in}}%
\pgfpathlineto{\pgfqpoint{2.943788in}{3.760471in}}%
\pgfpathmoveto{\pgfqpoint{2.939530in}{3.764729in}}%
\pgfpathlineto{\pgfqpoint{2.939530in}{3.764729in}}%
\pgfpathlineto{\pgfqpoint{2.939530in}{3.768987in}}%
\pgfpathlineto{\pgfqpoint{2.943788in}{3.768987in}}%
\pgfpathlineto{\pgfqpoint{2.943788in}{3.764729in}}%
\pgfpathmoveto{\pgfqpoint{2.939530in}{3.768987in}}%
\pgfpathlineto{\pgfqpoint{2.939530in}{3.768987in}}%
\pgfpathlineto{\pgfqpoint{2.939530in}{3.773244in}}%
\pgfpathlineto{\pgfqpoint{2.943788in}{3.773244in}}%
\pgfpathlineto{\pgfqpoint{2.943788in}{3.768987in}}%
\pgfpathmoveto{\pgfqpoint{2.939530in}{3.773244in}}%
\pgfpathlineto{\pgfqpoint{2.939530in}{3.773244in}}%
\pgfpathlineto{\pgfqpoint{2.939530in}{3.777502in}}%
\pgfpathlineto{\pgfqpoint{2.943788in}{3.777502in}}%
\pgfpathlineto{\pgfqpoint{2.943788in}{3.773244in}}%
\pgfpathmoveto{\pgfqpoint{2.935273in}{3.794533in}}%
\pgfpathlineto{\pgfqpoint{2.935273in}{3.794533in}}%
\pgfpathlineto{\pgfqpoint{2.935273in}{3.798791in}}%
\pgfpathlineto{\pgfqpoint{2.939530in}{3.798791in}}%
\pgfpathlineto{\pgfqpoint{2.939530in}{3.794533in}}%
\pgfpathmoveto{\pgfqpoint{2.935273in}{3.798791in}}%
\pgfpathlineto{\pgfqpoint{2.935273in}{3.798791in}}%
\pgfpathlineto{\pgfqpoint{2.935273in}{3.803049in}}%
\pgfpathlineto{\pgfqpoint{2.939530in}{3.803049in}}%
\pgfpathlineto{\pgfqpoint{2.939530in}{3.798791in}}%
\pgfpathmoveto{\pgfqpoint{2.935273in}{3.803049in}}%
\pgfpathlineto{\pgfqpoint{2.935273in}{3.803049in}}%
\pgfpathlineto{\pgfqpoint{2.935273in}{3.807306in}}%
\pgfpathlineto{\pgfqpoint{2.939530in}{3.807306in}}%
\pgfpathlineto{\pgfqpoint{2.939530in}{3.803049in}}%
\pgfpathmoveto{\pgfqpoint{2.935273in}{3.807306in}}%
\pgfpathlineto{\pgfqpoint{2.935273in}{3.807306in}}%
\pgfpathlineto{\pgfqpoint{2.935273in}{3.811564in}}%
\pgfpathlineto{\pgfqpoint{2.939530in}{3.811564in}}%
\pgfpathlineto{\pgfqpoint{2.939530in}{3.807306in}}%
\pgfpathmoveto{\pgfqpoint{2.939530in}{3.777502in}}%
\pgfpathlineto{\pgfqpoint{2.939530in}{3.777502in}}%
\pgfpathlineto{\pgfqpoint{2.939530in}{3.781760in}}%
\pgfpathlineto{\pgfqpoint{2.943788in}{3.781760in}}%
\pgfpathlineto{\pgfqpoint{2.943788in}{3.777502in}}%
\pgfpathmoveto{\pgfqpoint{2.939530in}{3.781760in}}%
\pgfpathlineto{\pgfqpoint{2.939530in}{3.781760in}}%
\pgfpathlineto{\pgfqpoint{2.939530in}{3.786018in}}%
\pgfpathlineto{\pgfqpoint{2.943788in}{3.786018in}}%
\pgfpathlineto{\pgfqpoint{2.943788in}{3.781760in}}%
\pgfpathmoveto{\pgfqpoint{2.939530in}{3.786018in}}%
\pgfpathlineto{\pgfqpoint{2.939530in}{3.786018in}}%
\pgfpathlineto{\pgfqpoint{2.939530in}{3.790275in}}%
\pgfpathlineto{\pgfqpoint{2.943788in}{3.790275in}}%
\pgfpathlineto{\pgfqpoint{2.943788in}{3.786018in}}%
\pgfpathmoveto{\pgfqpoint{2.939530in}{3.790275in}}%
\pgfpathlineto{\pgfqpoint{2.939530in}{3.790275in}}%
\pgfpathlineto{\pgfqpoint{2.939530in}{3.794533in}}%
\pgfpathlineto{\pgfqpoint{2.943788in}{3.794533in}}%
\pgfpathlineto{\pgfqpoint{2.943788in}{3.790275in}}%
\pgfpathmoveto{\pgfqpoint{2.939530in}{3.794533in}}%
\pgfpathlineto{\pgfqpoint{2.939530in}{3.794533in}}%
\pgfpathlineto{\pgfqpoint{2.939530in}{3.798791in}}%
\pgfpathlineto{\pgfqpoint{2.943788in}{3.798791in}}%
\pgfpathlineto{\pgfqpoint{2.943788in}{3.794533in}}%
\pgfpathmoveto{\pgfqpoint{2.935273in}{3.811564in}}%
\pgfpathlineto{\pgfqpoint{2.935273in}{3.811564in}}%
\pgfpathlineto{\pgfqpoint{2.935273in}{3.815822in}}%
\pgfpathlineto{\pgfqpoint{2.939530in}{3.815822in}}%
\pgfpathlineto{\pgfqpoint{2.939530in}{3.811564in}}%
\pgfpathmoveto{\pgfqpoint{2.935273in}{3.815822in}}%
\pgfpathlineto{\pgfqpoint{2.935273in}{3.815822in}}%
\pgfpathlineto{\pgfqpoint{2.935273in}{3.820080in}}%
\pgfpathlineto{\pgfqpoint{2.939530in}{3.820080in}}%
\pgfpathlineto{\pgfqpoint{2.939530in}{3.815822in}}%
\pgfpathmoveto{\pgfqpoint{2.935273in}{3.820080in}}%
\pgfpathlineto{\pgfqpoint{2.935273in}{3.820080in}}%
\pgfpathlineto{\pgfqpoint{2.935273in}{3.824337in}}%
\pgfpathlineto{\pgfqpoint{2.939530in}{3.824337in}}%
\pgfpathlineto{\pgfqpoint{2.939530in}{3.820080in}}%
\pgfpathmoveto{\pgfqpoint{2.935273in}{3.824337in}}%
\pgfpathlineto{\pgfqpoint{2.935273in}{3.824337in}}%
\pgfpathlineto{\pgfqpoint{2.935273in}{3.828595in}}%
\pgfpathlineto{\pgfqpoint{2.939530in}{3.828595in}}%
\pgfpathlineto{\pgfqpoint{2.939530in}{3.824337in}}%
\pgfpathmoveto{\pgfqpoint{2.931015in}{3.832853in}}%
\pgfpathlineto{\pgfqpoint{2.931015in}{3.832853in}}%
\pgfpathlineto{\pgfqpoint{2.931015in}{3.837111in}}%
\pgfpathlineto{\pgfqpoint{2.935273in}{3.837111in}}%
\pgfpathlineto{\pgfqpoint{2.935273in}{3.832853in}}%
\pgfpathmoveto{\pgfqpoint{2.935273in}{3.828595in}}%
\pgfpathlineto{\pgfqpoint{2.935273in}{3.828595in}}%
\pgfpathlineto{\pgfqpoint{2.935273in}{3.832853in}}%
\pgfpathlineto{\pgfqpoint{2.939530in}{3.832853in}}%
\pgfpathlineto{\pgfqpoint{2.939530in}{3.828595in}}%
\pgfpathmoveto{\pgfqpoint{2.935273in}{3.832853in}}%
\pgfpathlineto{\pgfqpoint{2.935273in}{3.832853in}}%
\pgfpathlineto{\pgfqpoint{2.935273in}{3.837111in}}%
\pgfpathlineto{\pgfqpoint{2.939530in}{3.837111in}}%
\pgfpathlineto{\pgfqpoint{2.939530in}{3.832853in}}%
\pgfpathmoveto{\pgfqpoint{2.931015in}{3.837111in}}%
\pgfpathlineto{\pgfqpoint{2.931015in}{3.837111in}}%
\pgfpathlineto{\pgfqpoint{2.931015in}{3.841368in}}%
\pgfpathlineto{\pgfqpoint{2.935273in}{3.841368in}}%
\pgfpathlineto{\pgfqpoint{2.935273in}{3.837111in}}%
\pgfpathmoveto{\pgfqpoint{2.931015in}{3.841368in}}%
\pgfpathlineto{\pgfqpoint{2.931015in}{3.841368in}}%
\pgfpathlineto{\pgfqpoint{2.931015in}{3.845626in}}%
\pgfpathlineto{\pgfqpoint{2.935273in}{3.845626in}}%
\pgfpathlineto{\pgfqpoint{2.935273in}{3.841368in}}%
\pgfpathmoveto{\pgfqpoint{2.931015in}{3.845626in}}%
\pgfpathlineto{\pgfqpoint{2.931015in}{3.845626in}}%
\pgfpathlineto{\pgfqpoint{2.931015in}{3.849884in}}%
\pgfpathlineto{\pgfqpoint{2.935273in}{3.849884in}}%
\pgfpathlineto{\pgfqpoint{2.935273in}{3.845626in}}%
\pgfpathmoveto{\pgfqpoint{2.931015in}{3.849884in}}%
\pgfpathlineto{\pgfqpoint{2.931015in}{3.849884in}}%
\pgfpathlineto{\pgfqpoint{2.931015in}{3.854142in}}%
\pgfpathlineto{\pgfqpoint{2.935273in}{3.854142in}}%
\pgfpathlineto{\pgfqpoint{2.935273in}{3.849884in}}%
\pgfpathmoveto{\pgfqpoint{2.931015in}{3.854142in}}%
\pgfpathlineto{\pgfqpoint{2.931015in}{3.854142in}}%
\pgfpathlineto{\pgfqpoint{2.931015in}{3.858399in}}%
\pgfpathlineto{\pgfqpoint{2.935273in}{3.858399in}}%
\pgfpathlineto{\pgfqpoint{2.935273in}{3.854142in}}%
\pgfpathmoveto{\pgfqpoint{2.931015in}{3.858399in}}%
\pgfpathlineto{\pgfqpoint{2.931015in}{3.858399in}}%
\pgfpathlineto{\pgfqpoint{2.931015in}{3.862657in}}%
\pgfpathlineto{\pgfqpoint{2.935273in}{3.862657in}}%
\pgfpathlineto{\pgfqpoint{2.935273in}{3.858399in}}%
\pgfpathmoveto{\pgfqpoint{2.926757in}{3.875430in}}%
\pgfpathlineto{\pgfqpoint{2.926757in}{3.875430in}}%
\pgfpathlineto{\pgfqpoint{2.926757in}{3.879688in}}%
\pgfpathlineto{\pgfqpoint{2.931015in}{3.879688in}}%
\pgfpathlineto{\pgfqpoint{2.931015in}{3.875430in}}%
\pgfpathmoveto{\pgfqpoint{2.931015in}{3.862657in}}%
\pgfpathlineto{\pgfqpoint{2.931015in}{3.862657in}}%
\pgfpathlineto{\pgfqpoint{2.931015in}{3.866915in}}%
\pgfpathlineto{\pgfqpoint{2.935273in}{3.866915in}}%
\pgfpathlineto{\pgfqpoint{2.935273in}{3.862657in}}%
\pgfpathmoveto{\pgfqpoint{2.931015in}{3.866915in}}%
\pgfpathlineto{\pgfqpoint{2.931015in}{3.866915in}}%
\pgfpathlineto{\pgfqpoint{2.931015in}{3.871173in}}%
\pgfpathlineto{\pgfqpoint{2.935273in}{3.871173in}}%
\pgfpathlineto{\pgfqpoint{2.935273in}{3.866915in}}%
\pgfpathmoveto{\pgfqpoint{2.931015in}{3.871173in}}%
\pgfpathlineto{\pgfqpoint{2.931015in}{3.871173in}}%
\pgfpathlineto{\pgfqpoint{2.931015in}{3.875430in}}%
\pgfpathlineto{\pgfqpoint{2.935273in}{3.875430in}}%
\pgfpathlineto{\pgfqpoint{2.935273in}{3.871173in}}%
\pgfpathmoveto{\pgfqpoint{2.931015in}{3.875430in}}%
\pgfpathlineto{\pgfqpoint{2.931015in}{3.875430in}}%
\pgfpathlineto{\pgfqpoint{2.931015in}{3.879688in}}%
\pgfpathlineto{\pgfqpoint{2.935273in}{3.879688in}}%
\pgfpathlineto{\pgfqpoint{2.935273in}{3.875430in}}%
\pgfpathmoveto{\pgfqpoint{2.926757in}{3.879688in}}%
\pgfpathlineto{\pgfqpoint{2.926757in}{3.879688in}}%
\pgfpathlineto{\pgfqpoint{2.926757in}{3.883946in}}%
\pgfpathlineto{\pgfqpoint{2.931015in}{3.883946in}}%
\pgfpathlineto{\pgfqpoint{2.931015in}{3.879688in}}%
\pgfpathmoveto{\pgfqpoint{2.926757in}{3.883946in}}%
\pgfpathlineto{\pgfqpoint{2.926757in}{3.883946in}}%
\pgfpathlineto{\pgfqpoint{2.926757in}{3.888204in}}%
\pgfpathlineto{\pgfqpoint{2.931015in}{3.888204in}}%
\pgfpathlineto{\pgfqpoint{2.931015in}{3.883946in}}%
\pgfpathmoveto{\pgfqpoint{2.926757in}{3.888204in}}%
\pgfpathlineto{\pgfqpoint{2.926757in}{3.888204in}}%
\pgfpathlineto{\pgfqpoint{2.926757in}{3.892461in}}%
\pgfpathlineto{\pgfqpoint{2.931015in}{3.892461in}}%
\pgfpathlineto{\pgfqpoint{2.931015in}{3.888204in}}%
\pgfpathmoveto{\pgfqpoint{2.926757in}{3.892461in}}%
\pgfpathlineto{\pgfqpoint{2.926757in}{3.892461in}}%
\pgfpathlineto{\pgfqpoint{2.926757in}{3.896719in}}%
\pgfpathlineto{\pgfqpoint{2.931015in}{3.896719in}}%
\pgfpathlineto{\pgfqpoint{2.931015in}{3.892461in}}%
\pgfpathmoveto{\pgfqpoint{2.926757in}{3.896719in}}%
\pgfpathlineto{\pgfqpoint{2.926757in}{3.896719in}}%
\pgfpathlineto{\pgfqpoint{2.926757in}{3.900977in}}%
\pgfpathlineto{\pgfqpoint{2.931015in}{3.900977in}}%
\pgfpathlineto{\pgfqpoint{2.931015in}{3.896719in}}%
\pgfpathmoveto{\pgfqpoint{2.926757in}{3.900977in}}%
\pgfpathlineto{\pgfqpoint{2.926757in}{3.900977in}}%
\pgfpathlineto{\pgfqpoint{2.926757in}{3.905235in}}%
\pgfpathlineto{\pgfqpoint{2.931015in}{3.905235in}}%
\pgfpathlineto{\pgfqpoint{2.931015in}{3.900977in}}%
\pgfpathmoveto{\pgfqpoint{2.926757in}{3.905235in}}%
\pgfpathlineto{\pgfqpoint{2.926757in}{3.905235in}}%
\pgfpathlineto{\pgfqpoint{2.926757in}{3.909492in}}%
\pgfpathlineto{\pgfqpoint{2.931015in}{3.909492in}}%
\pgfpathlineto{\pgfqpoint{2.931015in}{3.905235in}}%
\pgfpathmoveto{\pgfqpoint{2.926757in}{3.909492in}}%
\pgfpathlineto{\pgfqpoint{2.926757in}{3.909492in}}%
\pgfpathlineto{\pgfqpoint{2.926757in}{3.913750in}}%
\pgfpathlineto{\pgfqpoint{2.931015in}{3.913750in}}%
\pgfpathlineto{\pgfqpoint{2.931015in}{3.909492in}}%
\pgfpathmoveto{\pgfqpoint{2.918241in}{3.956327in}}%
\pgfpathlineto{\pgfqpoint{2.918241in}{3.956327in}}%
\pgfpathlineto{\pgfqpoint{2.918241in}{3.960585in}}%
\pgfpathlineto{\pgfqpoint{2.922499in}{3.960585in}}%
\pgfpathlineto{\pgfqpoint{2.922499in}{3.956327in}}%
\pgfpathmoveto{\pgfqpoint{2.918241in}{3.960585in}}%
\pgfpathlineto{\pgfqpoint{2.918241in}{3.960585in}}%
\pgfpathlineto{\pgfqpoint{2.918241in}{3.964843in}}%
\pgfpathlineto{\pgfqpoint{2.922499in}{3.964843in}}%
\pgfpathlineto{\pgfqpoint{2.922499in}{3.960585in}}%
\pgfpathmoveto{\pgfqpoint{2.918241in}{3.964843in}}%
\pgfpathlineto{\pgfqpoint{2.918241in}{3.964843in}}%
\pgfpathlineto{\pgfqpoint{2.918241in}{3.969100in}}%
\pgfpathlineto{\pgfqpoint{2.922499in}{3.969100in}}%
\pgfpathlineto{\pgfqpoint{2.922499in}{3.964843in}}%
\pgfpathmoveto{\pgfqpoint{2.918241in}{3.969100in}}%
\pgfpathlineto{\pgfqpoint{2.918241in}{3.969100in}}%
\pgfpathlineto{\pgfqpoint{2.918241in}{3.973358in}}%
\pgfpathlineto{\pgfqpoint{2.922499in}{3.973358in}}%
\pgfpathlineto{\pgfqpoint{2.922499in}{3.969100in}}%
\pgfpathmoveto{\pgfqpoint{2.918241in}{3.973358in}}%
\pgfpathlineto{\pgfqpoint{2.918241in}{3.973358in}}%
\pgfpathlineto{\pgfqpoint{2.918241in}{3.977616in}}%
\pgfpathlineto{\pgfqpoint{2.922499in}{3.977616in}}%
\pgfpathlineto{\pgfqpoint{2.922499in}{3.973358in}}%
\pgfpathmoveto{\pgfqpoint{2.918241in}{3.977616in}}%
\pgfpathlineto{\pgfqpoint{2.918241in}{3.977616in}}%
\pgfpathlineto{\pgfqpoint{2.918241in}{3.981873in}}%
\pgfpathlineto{\pgfqpoint{2.922499in}{3.981873in}}%
\pgfpathlineto{\pgfqpoint{2.922499in}{3.977616in}}%
\pgfpathmoveto{\pgfqpoint{2.922499in}{3.913750in}}%
\pgfpathlineto{\pgfqpoint{2.922499in}{3.913750in}}%
\pgfpathlineto{\pgfqpoint{2.922499in}{3.918008in}}%
\pgfpathlineto{\pgfqpoint{2.926757in}{3.918008in}}%
\pgfpathlineto{\pgfqpoint{2.926757in}{3.913750in}}%
\pgfpathmoveto{\pgfqpoint{2.922499in}{3.918008in}}%
\pgfpathlineto{\pgfqpoint{2.922499in}{3.918008in}}%
\pgfpathlineto{\pgfqpoint{2.922499in}{3.922266in}}%
\pgfpathlineto{\pgfqpoint{2.926757in}{3.922266in}}%
\pgfpathlineto{\pgfqpoint{2.926757in}{3.918008in}}%
\pgfpathmoveto{\pgfqpoint{2.926757in}{3.913750in}}%
\pgfpathlineto{\pgfqpoint{2.926757in}{3.913750in}}%
\pgfpathlineto{\pgfqpoint{2.926757in}{3.918008in}}%
\pgfpathlineto{\pgfqpoint{2.931015in}{3.918008in}}%
\pgfpathlineto{\pgfqpoint{2.931015in}{3.913750in}}%
\pgfpathmoveto{\pgfqpoint{2.922499in}{3.922266in}}%
\pgfpathlineto{\pgfqpoint{2.922499in}{3.922266in}}%
\pgfpathlineto{\pgfqpoint{2.922499in}{3.926523in}}%
\pgfpathlineto{\pgfqpoint{2.926757in}{3.926523in}}%
\pgfpathlineto{\pgfqpoint{2.926757in}{3.922266in}}%
\pgfpathmoveto{\pgfqpoint{2.922499in}{3.926523in}}%
\pgfpathlineto{\pgfqpoint{2.922499in}{3.926523in}}%
\pgfpathlineto{\pgfqpoint{2.922499in}{3.930781in}}%
\pgfpathlineto{\pgfqpoint{2.926757in}{3.930781in}}%
\pgfpathlineto{\pgfqpoint{2.926757in}{3.926523in}}%
\pgfpathmoveto{\pgfqpoint{2.922499in}{3.930781in}}%
\pgfpathlineto{\pgfqpoint{2.922499in}{3.930781in}}%
\pgfpathlineto{\pgfqpoint{2.922499in}{3.935039in}}%
\pgfpathlineto{\pgfqpoint{2.926757in}{3.935039in}}%
\pgfpathlineto{\pgfqpoint{2.926757in}{3.930781in}}%
\pgfpathmoveto{\pgfqpoint{2.922499in}{3.935039in}}%
\pgfpathlineto{\pgfqpoint{2.922499in}{3.935039in}}%
\pgfpathlineto{\pgfqpoint{2.922499in}{3.939296in}}%
\pgfpathlineto{\pgfqpoint{2.926757in}{3.939296in}}%
\pgfpathlineto{\pgfqpoint{2.926757in}{3.935039in}}%
\pgfpathmoveto{\pgfqpoint{2.922499in}{3.939296in}}%
\pgfpathlineto{\pgfqpoint{2.922499in}{3.939296in}}%
\pgfpathlineto{\pgfqpoint{2.922499in}{3.943554in}}%
\pgfpathlineto{\pgfqpoint{2.926757in}{3.943554in}}%
\pgfpathlineto{\pgfqpoint{2.926757in}{3.939296in}}%
\pgfpathmoveto{\pgfqpoint{2.922499in}{3.943554in}}%
\pgfpathlineto{\pgfqpoint{2.922499in}{3.943554in}}%
\pgfpathlineto{\pgfqpoint{2.922499in}{3.947812in}}%
\pgfpathlineto{\pgfqpoint{2.926757in}{3.947812in}}%
\pgfpathlineto{\pgfqpoint{2.926757in}{3.943554in}}%
\pgfpathmoveto{\pgfqpoint{2.922499in}{3.947812in}}%
\pgfpathlineto{\pgfqpoint{2.922499in}{3.947812in}}%
\pgfpathlineto{\pgfqpoint{2.922499in}{3.952069in}}%
\pgfpathlineto{\pgfqpoint{2.926757in}{3.952069in}}%
\pgfpathlineto{\pgfqpoint{2.926757in}{3.947812in}}%
\pgfpathmoveto{\pgfqpoint{2.922499in}{3.952069in}}%
\pgfpathlineto{\pgfqpoint{2.922499in}{3.952069in}}%
\pgfpathlineto{\pgfqpoint{2.922499in}{3.956327in}}%
\pgfpathlineto{\pgfqpoint{2.926757in}{3.956327in}}%
\pgfpathlineto{\pgfqpoint{2.926757in}{3.952069in}}%
\pgfpathmoveto{\pgfqpoint{2.922499in}{3.956327in}}%
\pgfpathlineto{\pgfqpoint{2.922499in}{3.956327in}}%
\pgfpathlineto{\pgfqpoint{2.922499in}{3.960585in}}%
\pgfpathlineto{\pgfqpoint{2.926757in}{3.960585in}}%
\pgfpathlineto{\pgfqpoint{2.926757in}{3.956327in}}%
\pgfpathmoveto{\pgfqpoint{2.918241in}{3.981873in}}%
\pgfpathlineto{\pgfqpoint{2.918241in}{3.981873in}}%
\pgfpathlineto{\pgfqpoint{2.918241in}{3.986131in}}%
\pgfpathlineto{\pgfqpoint{2.922499in}{3.986131in}}%
\pgfpathlineto{\pgfqpoint{2.922499in}{3.981873in}}%
\pgfpathmoveto{\pgfqpoint{2.918241in}{3.986131in}}%
\pgfpathlineto{\pgfqpoint{2.918241in}{3.986131in}}%
\pgfpathlineto{\pgfqpoint{2.918241in}{3.990389in}}%
\pgfpathlineto{\pgfqpoint{2.922499in}{3.990389in}}%
\pgfpathlineto{\pgfqpoint{2.922499in}{3.986131in}}%
\pgfpathmoveto{\pgfqpoint{2.918241in}{3.990389in}}%
\pgfpathlineto{\pgfqpoint{2.918241in}{3.990389in}}%
\pgfpathlineto{\pgfqpoint{2.918241in}{3.994647in}}%
\pgfpathlineto{\pgfqpoint{2.922499in}{3.994647in}}%
\pgfpathlineto{\pgfqpoint{2.922499in}{3.990389in}}%
\pgfpathmoveto{\pgfqpoint{2.918241in}{3.994647in}}%
\pgfpathlineto{\pgfqpoint{2.918241in}{3.994647in}}%
\pgfpathlineto{\pgfqpoint{2.918241in}{3.998904in}}%
\pgfpathlineto{\pgfqpoint{2.922499in}{3.998904in}}%
\pgfpathlineto{\pgfqpoint{2.922499in}{3.994647in}}%
\pgfpathmoveto{\pgfqpoint{2.913983in}{3.998904in}}%
\pgfpathlineto{\pgfqpoint{2.913983in}{3.998904in}}%
\pgfpathlineto{\pgfqpoint{2.913983in}{4.003162in}}%
\pgfpathlineto{\pgfqpoint{2.918241in}{4.003162in}}%
\pgfpathlineto{\pgfqpoint{2.918241in}{3.998904in}}%
\pgfpathmoveto{\pgfqpoint{2.913983in}{4.003162in}}%
\pgfpathlineto{\pgfqpoint{2.913983in}{4.003162in}}%
\pgfpathlineto{\pgfqpoint{2.913983in}{4.007420in}}%
\pgfpathlineto{\pgfqpoint{2.918241in}{4.007420in}}%
\pgfpathlineto{\pgfqpoint{2.918241in}{4.003162in}}%
\pgfpathmoveto{\pgfqpoint{2.918241in}{3.998904in}}%
\pgfpathlineto{\pgfqpoint{2.918241in}{3.998904in}}%
\pgfpathlineto{\pgfqpoint{2.918241in}{4.003162in}}%
\pgfpathlineto{\pgfqpoint{2.922499in}{4.003162in}}%
\pgfpathlineto{\pgfqpoint{2.922499in}{3.998904in}}%
\pgfpathmoveto{\pgfqpoint{2.913983in}{4.007420in}}%
\pgfpathlineto{\pgfqpoint{2.913983in}{4.007420in}}%
\pgfpathlineto{\pgfqpoint{2.913983in}{4.011677in}}%
\pgfpathlineto{\pgfqpoint{2.918241in}{4.011677in}}%
\pgfpathlineto{\pgfqpoint{2.918241in}{4.007420in}}%
\pgfpathmoveto{\pgfqpoint{2.913983in}{4.011677in}}%
\pgfpathlineto{\pgfqpoint{2.913983in}{4.011677in}}%
\pgfpathlineto{\pgfqpoint{2.913983in}{4.015935in}}%
\pgfpathlineto{\pgfqpoint{2.918241in}{4.015935in}}%
\pgfpathlineto{\pgfqpoint{2.918241in}{4.011677in}}%
\pgfpathmoveto{\pgfqpoint{2.913983in}{4.015935in}}%
\pgfpathlineto{\pgfqpoint{2.913983in}{4.015935in}}%
\pgfpathlineto{\pgfqpoint{2.913983in}{4.020193in}}%
\pgfpathlineto{\pgfqpoint{2.918241in}{4.020193in}}%
\pgfpathlineto{\pgfqpoint{2.918241in}{4.015935in}}%
\pgfpathmoveto{\pgfqpoint{2.913983in}{4.020193in}}%
\pgfpathlineto{\pgfqpoint{2.913983in}{4.020193in}}%
\pgfpathlineto{\pgfqpoint{2.913983in}{4.024451in}}%
\pgfpathlineto{\pgfqpoint{2.918241in}{4.024451in}}%
\pgfpathlineto{\pgfqpoint{2.918241in}{4.020193in}}%
\pgfpathmoveto{\pgfqpoint{2.913983in}{4.024451in}}%
\pgfpathlineto{\pgfqpoint{2.913983in}{4.024451in}}%
\pgfpathlineto{\pgfqpoint{2.913983in}{4.028708in}}%
\pgfpathlineto{\pgfqpoint{2.918241in}{4.028708in}}%
\pgfpathlineto{\pgfqpoint{2.918241in}{4.024451in}}%
\pgfpathmoveto{\pgfqpoint{2.913983in}{4.028708in}}%
\pgfpathlineto{\pgfqpoint{2.913983in}{4.028708in}}%
\pgfpathlineto{\pgfqpoint{2.913983in}{4.032966in}}%
\pgfpathlineto{\pgfqpoint{2.918241in}{4.032966in}}%
\pgfpathlineto{\pgfqpoint{2.918241in}{4.028708in}}%
\pgfpathmoveto{\pgfqpoint{2.909725in}{4.037224in}}%
\pgfpathlineto{\pgfqpoint{2.909725in}{4.037224in}}%
\pgfpathlineto{\pgfqpoint{2.909725in}{4.041481in}}%
\pgfpathlineto{\pgfqpoint{2.913983in}{4.041481in}}%
\pgfpathlineto{\pgfqpoint{2.913983in}{4.037224in}}%
\pgfpathmoveto{\pgfqpoint{2.909725in}{4.041481in}}%
\pgfpathlineto{\pgfqpoint{2.909725in}{4.041481in}}%
\pgfpathlineto{\pgfqpoint{2.909725in}{4.045739in}}%
\pgfpathlineto{\pgfqpoint{2.913983in}{4.045739in}}%
\pgfpathlineto{\pgfqpoint{2.913983in}{4.041481in}}%
\pgfpathmoveto{\pgfqpoint{2.909725in}{4.045739in}}%
\pgfpathlineto{\pgfqpoint{2.909725in}{4.045739in}}%
\pgfpathlineto{\pgfqpoint{2.909725in}{4.049997in}}%
\pgfpathlineto{\pgfqpoint{2.913983in}{4.049997in}}%
\pgfpathlineto{\pgfqpoint{2.913983in}{4.045739in}}%
\pgfpathmoveto{\pgfqpoint{2.913983in}{4.032966in}}%
\pgfpathlineto{\pgfqpoint{2.913983in}{4.032966in}}%
\pgfpathlineto{\pgfqpoint{2.913983in}{4.037224in}}%
\pgfpathlineto{\pgfqpoint{2.918241in}{4.037224in}}%
\pgfpathlineto{\pgfqpoint{2.918241in}{4.032966in}}%
\pgfpathmoveto{\pgfqpoint{2.913983in}{4.037224in}}%
\pgfpathlineto{\pgfqpoint{2.913983in}{4.037224in}}%
\pgfpathlineto{\pgfqpoint{2.913983in}{4.041481in}}%
\pgfpathlineto{\pgfqpoint{2.918241in}{4.041481in}}%
\pgfpathlineto{\pgfqpoint{2.918241in}{4.037224in}}%
\pgfpathmoveto{\pgfqpoint{2.909725in}{4.049997in}}%
\pgfpathlineto{\pgfqpoint{2.909725in}{4.049997in}}%
\pgfpathlineto{\pgfqpoint{2.909725in}{4.054255in}}%
\pgfpathlineto{\pgfqpoint{2.913983in}{4.054255in}}%
\pgfpathlineto{\pgfqpoint{2.913983in}{4.049997in}}%
\pgfpathmoveto{\pgfqpoint{2.909725in}{4.054255in}}%
\pgfpathlineto{\pgfqpoint{2.909725in}{4.054255in}}%
\pgfpathlineto{\pgfqpoint{2.909725in}{4.058512in}}%
\pgfpathlineto{\pgfqpoint{2.913983in}{4.058512in}}%
\pgfpathlineto{\pgfqpoint{2.913983in}{4.054255in}}%
\pgfpathmoveto{\pgfqpoint{2.909725in}{4.058512in}}%
\pgfpathlineto{\pgfqpoint{2.909725in}{4.058512in}}%
\pgfpathlineto{\pgfqpoint{2.909725in}{4.062770in}}%
\pgfpathlineto{\pgfqpoint{2.913983in}{4.062770in}}%
\pgfpathlineto{\pgfqpoint{2.913983in}{4.058512in}}%
\pgfpathmoveto{\pgfqpoint{2.909725in}{4.062770in}}%
\pgfpathlineto{\pgfqpoint{2.909725in}{4.062770in}}%
\pgfpathlineto{\pgfqpoint{2.909725in}{4.067028in}}%
\pgfpathlineto{\pgfqpoint{2.913983in}{4.067028in}}%
\pgfpathlineto{\pgfqpoint{2.913983in}{4.062770in}}%
\pgfpathmoveto{\pgfqpoint{2.909725in}{4.067028in}}%
\pgfpathlineto{\pgfqpoint{2.909725in}{4.067028in}}%
\pgfpathlineto{\pgfqpoint{2.909725in}{4.071286in}}%
\pgfpathlineto{\pgfqpoint{2.913983in}{4.071286in}}%
\pgfpathlineto{\pgfqpoint{2.913983in}{4.067028in}}%
\pgfpathmoveto{\pgfqpoint{2.909725in}{4.071286in}}%
\pgfpathlineto{\pgfqpoint{2.909725in}{4.071286in}}%
\pgfpathlineto{\pgfqpoint{2.909725in}{4.075544in}}%
\pgfpathlineto{\pgfqpoint{2.913983in}{4.075544in}}%
\pgfpathlineto{\pgfqpoint{2.913983in}{4.071286in}}%
\pgfpathmoveto{\pgfqpoint{2.905467in}{4.079801in}}%
\pgfpathlineto{\pgfqpoint{2.905467in}{4.079801in}}%
\pgfpathlineto{\pgfqpoint{2.905467in}{4.084059in}}%
\pgfpathlineto{\pgfqpoint{2.909725in}{4.084059in}}%
\pgfpathlineto{\pgfqpoint{2.909725in}{4.079801in}}%
\pgfpathmoveto{\pgfqpoint{2.909725in}{4.075544in}}%
\pgfpathlineto{\pgfqpoint{2.909725in}{4.075544in}}%
\pgfpathlineto{\pgfqpoint{2.909725in}{4.079801in}}%
\pgfpathlineto{\pgfqpoint{2.913983in}{4.079801in}}%
\pgfpathlineto{\pgfqpoint{2.913983in}{4.075544in}}%
\pgfpathmoveto{\pgfqpoint{2.909725in}{4.079801in}}%
\pgfpathlineto{\pgfqpoint{2.909725in}{4.079801in}}%
\pgfpathlineto{\pgfqpoint{2.909725in}{4.084059in}}%
\pgfpathlineto{\pgfqpoint{2.913983in}{4.084059in}}%
\pgfpathlineto{\pgfqpoint{2.913983in}{4.079801in}}%
\pgfpathmoveto{\pgfqpoint{2.905467in}{4.084059in}}%
\pgfpathlineto{\pgfqpoint{2.905467in}{4.084059in}}%
\pgfpathlineto{\pgfqpoint{2.905467in}{4.088317in}}%
\pgfpathlineto{\pgfqpoint{2.909725in}{4.088317in}}%
\pgfpathlineto{\pgfqpoint{2.909725in}{4.084059in}}%
\pgfpathmoveto{\pgfqpoint{2.905467in}{4.088317in}}%
\pgfpathlineto{\pgfqpoint{2.905467in}{4.088317in}}%
\pgfpathlineto{\pgfqpoint{2.905467in}{4.092575in}}%
\pgfpathlineto{\pgfqpoint{2.909725in}{4.092575in}}%
\pgfpathlineto{\pgfqpoint{2.909725in}{4.088317in}}%
\pgfpathmoveto{\pgfqpoint{2.905467in}{4.092575in}}%
\pgfpathlineto{\pgfqpoint{2.905467in}{4.092575in}}%
\pgfpathlineto{\pgfqpoint{2.905467in}{4.096833in}}%
\pgfpathlineto{\pgfqpoint{2.909725in}{4.096833in}}%
\pgfpathlineto{\pgfqpoint{2.909725in}{4.092575in}}%
\pgfpathmoveto{\pgfqpoint{2.905467in}{4.096833in}}%
\pgfpathlineto{\pgfqpoint{2.905467in}{4.096833in}}%
\pgfpathlineto{\pgfqpoint{2.905467in}{4.101090in}}%
\pgfpathlineto{\pgfqpoint{2.909725in}{4.101090in}}%
\pgfpathlineto{\pgfqpoint{2.909725in}{4.096833in}}%
\pgfpathmoveto{\pgfqpoint{2.905467in}{4.101090in}}%
\pgfpathlineto{\pgfqpoint{2.905467in}{4.101090in}}%
\pgfpathlineto{\pgfqpoint{2.905467in}{4.105348in}}%
\pgfpathlineto{\pgfqpoint{2.909725in}{4.105348in}}%
\pgfpathlineto{\pgfqpoint{2.909725in}{4.101090in}}%
\pgfpathmoveto{\pgfqpoint{2.905467in}{4.105348in}}%
\pgfpathlineto{\pgfqpoint{2.905467in}{4.105348in}}%
\pgfpathlineto{\pgfqpoint{2.905467in}{4.109606in}}%
\pgfpathlineto{\pgfqpoint{2.909725in}{4.109606in}}%
\pgfpathlineto{\pgfqpoint{2.909725in}{4.105348in}}%
\pgfpathmoveto{\pgfqpoint{2.905467in}{4.109606in}}%
\pgfpathlineto{\pgfqpoint{2.905467in}{4.109606in}}%
\pgfpathlineto{\pgfqpoint{2.905467in}{4.113864in}}%
\pgfpathlineto{\pgfqpoint{2.909725in}{4.113864in}}%
\pgfpathlineto{\pgfqpoint{2.909725in}{4.109606in}}%
\pgfpathmoveto{\pgfqpoint{2.905467in}{4.113864in}}%
\pgfpathlineto{\pgfqpoint{2.905467in}{4.113864in}}%
\pgfpathlineto{\pgfqpoint{2.905467in}{4.118122in}}%
\pgfpathlineto{\pgfqpoint{2.909725in}{4.118122in}}%
\pgfpathlineto{\pgfqpoint{2.909725in}{4.113864in}}%
\pgfpathmoveto{\pgfqpoint{2.901209in}{4.122380in}}%
\pgfpathlineto{\pgfqpoint{2.901209in}{4.122380in}}%
\pgfpathlineto{\pgfqpoint{2.901209in}{4.126637in}}%
\pgfpathlineto{\pgfqpoint{2.905467in}{4.126637in}}%
\pgfpathlineto{\pgfqpoint{2.905467in}{4.122380in}}%
\pgfpathmoveto{\pgfqpoint{2.901209in}{4.126637in}}%
\pgfpathlineto{\pgfqpoint{2.901209in}{4.126637in}}%
\pgfpathlineto{\pgfqpoint{2.901209in}{4.130895in}}%
\pgfpathlineto{\pgfqpoint{2.905467in}{4.130895in}}%
\pgfpathlineto{\pgfqpoint{2.905467in}{4.126637in}}%
\pgfpathmoveto{\pgfqpoint{2.901209in}{4.130895in}}%
\pgfpathlineto{\pgfqpoint{2.901209in}{4.130895in}}%
\pgfpathlineto{\pgfqpoint{2.901209in}{4.135153in}}%
\pgfpathlineto{\pgfqpoint{2.905467in}{4.135153in}}%
\pgfpathlineto{\pgfqpoint{2.905467in}{4.130895in}}%
\pgfpathmoveto{\pgfqpoint{2.901209in}{4.135153in}}%
\pgfpathlineto{\pgfqpoint{2.901209in}{4.135153in}}%
\pgfpathlineto{\pgfqpoint{2.901209in}{4.139411in}}%
\pgfpathlineto{\pgfqpoint{2.905467in}{4.139411in}}%
\pgfpathlineto{\pgfqpoint{2.905467in}{4.135153in}}%
\pgfpathmoveto{\pgfqpoint{2.901209in}{4.139411in}}%
\pgfpathlineto{\pgfqpoint{2.901209in}{4.139411in}}%
\pgfpathlineto{\pgfqpoint{2.901209in}{4.143669in}}%
\pgfpathlineto{\pgfqpoint{2.905467in}{4.143669in}}%
\pgfpathlineto{\pgfqpoint{2.905467in}{4.139411in}}%
\pgfpathmoveto{\pgfqpoint{2.901209in}{4.143669in}}%
\pgfpathlineto{\pgfqpoint{2.901209in}{4.143669in}}%
\pgfpathlineto{\pgfqpoint{2.901209in}{4.147926in}}%
\pgfpathlineto{\pgfqpoint{2.905467in}{4.147926in}}%
\pgfpathlineto{\pgfqpoint{2.905467in}{4.143669in}}%
\pgfpathmoveto{\pgfqpoint{2.901209in}{4.147926in}}%
\pgfpathlineto{\pgfqpoint{2.901209in}{4.147926in}}%
\pgfpathlineto{\pgfqpoint{2.901209in}{4.152184in}}%
\pgfpathlineto{\pgfqpoint{2.905467in}{4.152184in}}%
\pgfpathlineto{\pgfqpoint{2.905467in}{4.147926in}}%
\pgfpathmoveto{\pgfqpoint{2.905467in}{4.118122in}}%
\pgfpathlineto{\pgfqpoint{2.905467in}{4.118122in}}%
\pgfpathlineto{\pgfqpoint{2.905467in}{4.122380in}}%
\pgfpathlineto{\pgfqpoint{2.909725in}{4.122380in}}%
\pgfpathlineto{\pgfqpoint{2.909725in}{4.118122in}}%
\pgfpathmoveto{\pgfqpoint{2.905467in}{4.122380in}}%
\pgfpathlineto{\pgfqpoint{2.905467in}{4.122380in}}%
\pgfpathlineto{\pgfqpoint{2.905467in}{4.126637in}}%
\pgfpathlineto{\pgfqpoint{2.909725in}{4.126637in}}%
\pgfpathlineto{\pgfqpoint{2.909725in}{4.122380in}}%
\pgfpathmoveto{\pgfqpoint{2.901209in}{4.152184in}}%
\pgfpathlineto{\pgfqpoint{2.901209in}{4.152184in}}%
\pgfpathlineto{\pgfqpoint{2.901209in}{4.156442in}}%
\pgfpathlineto{\pgfqpoint{2.905467in}{4.156442in}}%
\pgfpathlineto{\pgfqpoint{2.905467in}{4.152184in}}%
\pgfpathmoveto{\pgfqpoint{2.901209in}{4.156442in}}%
\pgfpathlineto{\pgfqpoint{2.901209in}{4.156442in}}%
\pgfpathlineto{\pgfqpoint{2.901209in}{4.160700in}}%
\pgfpathlineto{\pgfqpoint{2.905467in}{4.160700in}}%
\pgfpathlineto{\pgfqpoint{2.905467in}{4.156442in}}%
\pgfpathmoveto{\pgfqpoint{2.896951in}{4.164958in}}%
\pgfpathlineto{\pgfqpoint{2.896951in}{4.164958in}}%
\pgfpathlineto{\pgfqpoint{2.896951in}{4.169215in}}%
\pgfpathlineto{\pgfqpoint{2.901209in}{4.169215in}}%
\pgfpathlineto{\pgfqpoint{2.901209in}{4.164958in}}%
\pgfpathmoveto{\pgfqpoint{2.901209in}{4.160700in}}%
\pgfpathlineto{\pgfqpoint{2.901209in}{4.160700in}}%
\pgfpathlineto{\pgfqpoint{2.901209in}{4.164958in}}%
\pgfpathlineto{\pgfqpoint{2.905467in}{4.164958in}}%
\pgfpathlineto{\pgfqpoint{2.905467in}{4.160700in}}%
\pgfpathmoveto{\pgfqpoint{2.901209in}{4.164958in}}%
\pgfpathlineto{\pgfqpoint{2.901209in}{4.164958in}}%
\pgfpathlineto{\pgfqpoint{2.901209in}{4.169215in}}%
\pgfpathlineto{\pgfqpoint{2.905467in}{4.169215in}}%
\pgfpathlineto{\pgfqpoint{2.905467in}{4.164958in}}%
\pgfpathmoveto{\pgfqpoint{2.896951in}{4.169215in}}%
\pgfpathlineto{\pgfqpoint{2.896951in}{4.169215in}}%
\pgfpathlineto{\pgfqpoint{2.896951in}{4.173473in}}%
\pgfpathlineto{\pgfqpoint{2.901209in}{4.173473in}}%
\pgfpathlineto{\pgfqpoint{2.901209in}{4.169215in}}%
\pgfpathmoveto{\pgfqpoint{2.896951in}{4.173473in}}%
\pgfpathlineto{\pgfqpoint{2.896951in}{4.173473in}}%
\pgfpathlineto{\pgfqpoint{2.896951in}{4.177731in}}%
\pgfpathlineto{\pgfqpoint{2.901209in}{4.177731in}}%
\pgfpathlineto{\pgfqpoint{2.901209in}{4.173473in}}%
\pgfpathmoveto{\pgfqpoint{2.896951in}{4.177731in}}%
\pgfpathlineto{\pgfqpoint{2.896951in}{4.177731in}}%
\pgfpathlineto{\pgfqpoint{2.896951in}{4.181989in}}%
\pgfpathlineto{\pgfqpoint{2.901209in}{4.181989in}}%
\pgfpathlineto{\pgfqpoint{2.901209in}{4.177731in}}%
\pgfpathmoveto{\pgfqpoint{2.896951in}{4.181989in}}%
\pgfpathlineto{\pgfqpoint{2.896951in}{4.181989in}}%
\pgfpathlineto{\pgfqpoint{2.896951in}{4.186247in}}%
\pgfpathlineto{\pgfqpoint{2.901209in}{4.186247in}}%
\pgfpathlineto{\pgfqpoint{2.901209in}{4.181989in}}%
\pgfpathmoveto{\pgfqpoint{2.884178in}{4.296950in}}%
\pgfpathlineto{\pgfqpoint{2.884178in}{4.296950in}}%
\pgfpathlineto{\pgfqpoint{2.884178in}{4.301208in}}%
\pgfpathlineto{\pgfqpoint{2.888435in}{4.301208in}}%
\pgfpathlineto{\pgfqpoint{2.888435in}{4.296950in}}%
\pgfpathmoveto{\pgfqpoint{2.884178in}{4.301208in}}%
\pgfpathlineto{\pgfqpoint{2.884178in}{4.301208in}}%
\pgfpathlineto{\pgfqpoint{2.884178in}{4.305466in}}%
\pgfpathlineto{\pgfqpoint{2.888435in}{4.305466in}}%
\pgfpathlineto{\pgfqpoint{2.888435in}{4.301208in}}%
\pgfpathmoveto{\pgfqpoint{2.884178in}{4.305466in}}%
\pgfpathlineto{\pgfqpoint{2.884178in}{4.305466in}}%
\pgfpathlineto{\pgfqpoint{2.884178in}{4.309724in}}%
\pgfpathlineto{\pgfqpoint{2.888435in}{4.309724in}}%
\pgfpathlineto{\pgfqpoint{2.888435in}{4.305466in}}%
\pgfpathmoveto{\pgfqpoint{2.884178in}{4.309724in}}%
\pgfpathlineto{\pgfqpoint{2.884178in}{4.309724in}}%
\pgfpathlineto{\pgfqpoint{2.884178in}{4.313982in}}%
\pgfpathlineto{\pgfqpoint{2.888435in}{4.313982in}}%
\pgfpathlineto{\pgfqpoint{2.888435in}{4.309724in}}%
\pgfpathmoveto{\pgfqpoint{2.884178in}{4.313982in}}%
\pgfpathlineto{\pgfqpoint{2.884178in}{4.313982in}}%
\pgfpathlineto{\pgfqpoint{2.884178in}{4.318239in}}%
\pgfpathlineto{\pgfqpoint{2.888435in}{4.318239in}}%
\pgfpathlineto{\pgfqpoint{2.888435in}{4.313982in}}%
\pgfpathmoveto{\pgfqpoint{2.884178in}{4.318239in}}%
\pgfpathlineto{\pgfqpoint{2.884178in}{4.318239in}}%
\pgfpathlineto{\pgfqpoint{2.884178in}{4.322497in}}%
\pgfpathlineto{\pgfqpoint{2.888435in}{4.322497in}}%
\pgfpathlineto{\pgfqpoint{2.888435in}{4.318239in}}%
\pgfpathmoveto{\pgfqpoint{2.896951in}{4.186247in}}%
\pgfpathlineto{\pgfqpoint{2.896951in}{4.186247in}}%
\pgfpathlineto{\pgfqpoint{2.896951in}{4.190504in}}%
\pgfpathlineto{\pgfqpoint{2.901209in}{4.190504in}}%
\pgfpathlineto{\pgfqpoint{2.901209in}{4.186247in}}%
\pgfpathmoveto{\pgfqpoint{2.896951in}{4.190504in}}%
\pgfpathlineto{\pgfqpoint{2.896951in}{4.190504in}}%
\pgfpathlineto{\pgfqpoint{2.896951in}{4.194762in}}%
\pgfpathlineto{\pgfqpoint{2.901209in}{4.194762in}}%
\pgfpathlineto{\pgfqpoint{2.901209in}{4.190504in}}%
\pgfpathmoveto{\pgfqpoint{2.896951in}{4.194762in}}%
\pgfpathlineto{\pgfqpoint{2.896951in}{4.194762in}}%
\pgfpathlineto{\pgfqpoint{2.896951in}{4.199020in}}%
\pgfpathlineto{\pgfqpoint{2.901209in}{4.199020in}}%
\pgfpathlineto{\pgfqpoint{2.901209in}{4.194762in}}%
\pgfpathmoveto{\pgfqpoint{2.896951in}{4.199020in}}%
\pgfpathlineto{\pgfqpoint{2.896951in}{4.199020in}}%
\pgfpathlineto{\pgfqpoint{2.896951in}{4.203278in}}%
\pgfpathlineto{\pgfqpoint{2.901209in}{4.203278in}}%
\pgfpathlineto{\pgfqpoint{2.901209in}{4.199020in}}%
\pgfpathmoveto{\pgfqpoint{2.892693in}{4.207536in}}%
\pgfpathlineto{\pgfqpoint{2.892693in}{4.207536in}}%
\pgfpathlineto{\pgfqpoint{2.892693in}{4.211794in}}%
\pgfpathlineto{\pgfqpoint{2.896951in}{4.211794in}}%
\pgfpathlineto{\pgfqpoint{2.896951in}{4.207536in}}%
\pgfpathmoveto{\pgfqpoint{2.892693in}{4.211794in}}%
\pgfpathlineto{\pgfqpoint{2.892693in}{4.211794in}}%
\pgfpathlineto{\pgfqpoint{2.892693in}{4.216051in}}%
\pgfpathlineto{\pgfqpoint{2.896951in}{4.216051in}}%
\pgfpathlineto{\pgfqpoint{2.896951in}{4.211794in}}%
\pgfpathmoveto{\pgfqpoint{2.892693in}{4.216051in}}%
\pgfpathlineto{\pgfqpoint{2.892693in}{4.216051in}}%
\pgfpathlineto{\pgfqpoint{2.892693in}{4.220309in}}%
\pgfpathlineto{\pgfqpoint{2.896951in}{4.220309in}}%
\pgfpathlineto{\pgfqpoint{2.896951in}{4.216051in}}%
\pgfpathmoveto{\pgfqpoint{2.896951in}{4.203278in}}%
\pgfpathlineto{\pgfqpoint{2.896951in}{4.203278in}}%
\pgfpathlineto{\pgfqpoint{2.896951in}{4.207536in}}%
\pgfpathlineto{\pgfqpoint{2.901209in}{4.207536in}}%
\pgfpathlineto{\pgfqpoint{2.901209in}{4.203278in}}%
\pgfpathmoveto{\pgfqpoint{2.896951in}{4.207536in}}%
\pgfpathlineto{\pgfqpoint{2.896951in}{4.207536in}}%
\pgfpathlineto{\pgfqpoint{2.896951in}{4.211794in}}%
\pgfpathlineto{\pgfqpoint{2.901209in}{4.211794in}}%
\pgfpathlineto{\pgfqpoint{2.901209in}{4.207536in}}%
\pgfpathmoveto{\pgfqpoint{2.892693in}{4.220309in}}%
\pgfpathlineto{\pgfqpoint{2.892693in}{4.220309in}}%
\pgfpathlineto{\pgfqpoint{2.892693in}{4.224567in}}%
\pgfpathlineto{\pgfqpoint{2.896951in}{4.224567in}}%
\pgfpathlineto{\pgfqpoint{2.896951in}{4.220309in}}%
\pgfpathmoveto{\pgfqpoint{2.892693in}{4.224567in}}%
\pgfpathlineto{\pgfqpoint{2.892693in}{4.224567in}}%
\pgfpathlineto{\pgfqpoint{2.892693in}{4.228825in}}%
\pgfpathlineto{\pgfqpoint{2.896951in}{4.228825in}}%
\pgfpathlineto{\pgfqpoint{2.896951in}{4.224567in}}%
\pgfpathmoveto{\pgfqpoint{2.892693in}{4.228825in}}%
\pgfpathlineto{\pgfqpoint{2.892693in}{4.228825in}}%
\pgfpathlineto{\pgfqpoint{2.892693in}{4.233083in}}%
\pgfpathlineto{\pgfqpoint{2.896951in}{4.233083in}}%
\pgfpathlineto{\pgfqpoint{2.896951in}{4.228825in}}%
\pgfpathmoveto{\pgfqpoint{2.892693in}{4.233083in}}%
\pgfpathlineto{\pgfqpoint{2.892693in}{4.233083in}}%
\pgfpathlineto{\pgfqpoint{2.892693in}{4.237341in}}%
\pgfpathlineto{\pgfqpoint{2.896951in}{4.237341in}}%
\pgfpathlineto{\pgfqpoint{2.896951in}{4.233083in}}%
\pgfpathmoveto{\pgfqpoint{2.892693in}{4.237341in}}%
\pgfpathlineto{\pgfqpoint{2.892693in}{4.237341in}}%
\pgfpathlineto{\pgfqpoint{2.892693in}{4.241598in}}%
\pgfpathlineto{\pgfqpoint{2.896951in}{4.241598in}}%
\pgfpathlineto{\pgfqpoint{2.896951in}{4.237341in}}%
\pgfpathmoveto{\pgfqpoint{2.892693in}{4.241598in}}%
\pgfpathlineto{\pgfqpoint{2.892693in}{4.241598in}}%
\pgfpathlineto{\pgfqpoint{2.892693in}{4.245856in}}%
\pgfpathlineto{\pgfqpoint{2.896951in}{4.245856in}}%
\pgfpathlineto{\pgfqpoint{2.896951in}{4.241598in}}%
\pgfpathmoveto{\pgfqpoint{2.888435in}{4.250114in}}%
\pgfpathlineto{\pgfqpoint{2.888435in}{4.250114in}}%
\pgfpathlineto{\pgfqpoint{2.888435in}{4.254372in}}%
\pgfpathlineto{\pgfqpoint{2.892693in}{4.254372in}}%
\pgfpathlineto{\pgfqpoint{2.892693in}{4.250114in}}%
\pgfpathmoveto{\pgfqpoint{2.892693in}{4.245856in}}%
\pgfpathlineto{\pgfqpoint{2.892693in}{4.245856in}}%
\pgfpathlineto{\pgfqpoint{2.892693in}{4.250114in}}%
\pgfpathlineto{\pgfqpoint{2.896951in}{4.250114in}}%
\pgfpathlineto{\pgfqpoint{2.896951in}{4.245856in}}%
\pgfpathmoveto{\pgfqpoint{2.892693in}{4.250114in}}%
\pgfpathlineto{\pgfqpoint{2.892693in}{4.250114in}}%
\pgfpathlineto{\pgfqpoint{2.892693in}{4.254372in}}%
\pgfpathlineto{\pgfqpoint{2.896951in}{4.254372in}}%
\pgfpathlineto{\pgfqpoint{2.896951in}{4.250114in}}%
\pgfpathmoveto{\pgfqpoint{2.888435in}{4.254372in}}%
\pgfpathlineto{\pgfqpoint{2.888435in}{4.254372in}}%
\pgfpathlineto{\pgfqpoint{2.888435in}{4.258630in}}%
\pgfpathlineto{\pgfqpoint{2.892693in}{4.258630in}}%
\pgfpathlineto{\pgfqpoint{2.892693in}{4.254372in}}%
\pgfpathmoveto{\pgfqpoint{2.888435in}{4.258630in}}%
\pgfpathlineto{\pgfqpoint{2.888435in}{4.258630in}}%
\pgfpathlineto{\pgfqpoint{2.888435in}{4.262888in}}%
\pgfpathlineto{\pgfqpoint{2.892693in}{4.262888in}}%
\pgfpathlineto{\pgfqpoint{2.892693in}{4.258630in}}%
\pgfpathmoveto{\pgfqpoint{2.888435in}{4.262888in}}%
\pgfpathlineto{\pgfqpoint{2.888435in}{4.262888in}}%
\pgfpathlineto{\pgfqpoint{2.888435in}{4.267145in}}%
\pgfpathlineto{\pgfqpoint{2.892693in}{4.267145in}}%
\pgfpathlineto{\pgfqpoint{2.892693in}{4.262888in}}%
\pgfpathmoveto{\pgfqpoint{2.888435in}{4.267145in}}%
\pgfpathlineto{\pgfqpoint{2.888435in}{4.267145in}}%
\pgfpathlineto{\pgfqpoint{2.888435in}{4.271403in}}%
\pgfpathlineto{\pgfqpoint{2.892693in}{4.271403in}}%
\pgfpathlineto{\pgfqpoint{2.892693in}{4.267145in}}%
\pgfpathmoveto{\pgfqpoint{2.888435in}{4.271403in}}%
\pgfpathlineto{\pgfqpoint{2.888435in}{4.271403in}}%
\pgfpathlineto{\pgfqpoint{2.888435in}{4.275661in}}%
\pgfpathlineto{\pgfqpoint{2.892693in}{4.275661in}}%
\pgfpathlineto{\pgfqpoint{2.892693in}{4.271403in}}%
\pgfpathmoveto{\pgfqpoint{2.888435in}{4.275661in}}%
\pgfpathlineto{\pgfqpoint{2.888435in}{4.275661in}}%
\pgfpathlineto{\pgfqpoint{2.888435in}{4.279919in}}%
\pgfpathlineto{\pgfqpoint{2.892693in}{4.279919in}}%
\pgfpathlineto{\pgfqpoint{2.892693in}{4.275661in}}%
\pgfpathmoveto{\pgfqpoint{2.888435in}{4.279919in}}%
\pgfpathlineto{\pgfqpoint{2.888435in}{4.279919in}}%
\pgfpathlineto{\pgfqpoint{2.888435in}{4.284177in}}%
\pgfpathlineto{\pgfqpoint{2.892693in}{4.284177in}}%
\pgfpathlineto{\pgfqpoint{2.892693in}{4.279919in}}%
\pgfpathmoveto{\pgfqpoint{2.888435in}{4.284177in}}%
\pgfpathlineto{\pgfqpoint{2.888435in}{4.284177in}}%
\pgfpathlineto{\pgfqpoint{2.888435in}{4.288435in}}%
\pgfpathlineto{\pgfqpoint{2.892693in}{4.288435in}}%
\pgfpathlineto{\pgfqpoint{2.892693in}{4.284177in}}%
\pgfpathmoveto{\pgfqpoint{2.888435in}{4.288435in}}%
\pgfpathlineto{\pgfqpoint{2.888435in}{4.288435in}}%
\pgfpathlineto{\pgfqpoint{2.888435in}{4.292692in}}%
\pgfpathlineto{\pgfqpoint{2.892693in}{4.292692in}}%
\pgfpathlineto{\pgfqpoint{2.892693in}{4.288435in}}%
\pgfpathmoveto{\pgfqpoint{2.888435in}{4.292692in}}%
\pgfpathlineto{\pgfqpoint{2.888435in}{4.292692in}}%
\pgfpathlineto{\pgfqpoint{2.888435in}{4.296950in}}%
\pgfpathlineto{\pgfqpoint{2.892693in}{4.296950in}}%
\pgfpathlineto{\pgfqpoint{2.892693in}{4.292692in}}%
\pgfpathmoveto{\pgfqpoint{2.888435in}{4.296950in}}%
\pgfpathlineto{\pgfqpoint{2.888435in}{4.296950in}}%
\pgfpathlineto{\pgfqpoint{2.888435in}{4.301208in}}%
\pgfpathlineto{\pgfqpoint{2.892693in}{4.301208in}}%
\pgfpathlineto{\pgfqpoint{2.892693in}{4.296950in}}%
\pgfpathmoveto{\pgfqpoint{2.884178in}{4.322497in}}%
\pgfpathlineto{\pgfqpoint{2.884178in}{4.322497in}}%
\pgfpathlineto{\pgfqpoint{2.884178in}{4.326755in}}%
\pgfpathlineto{\pgfqpoint{2.888435in}{4.326755in}}%
\pgfpathlineto{\pgfqpoint{2.888435in}{4.322497in}}%
\pgfpathmoveto{\pgfqpoint{2.884178in}{4.326755in}}%
\pgfpathlineto{\pgfqpoint{2.884178in}{4.326755in}}%
\pgfpathlineto{\pgfqpoint{2.884178in}{4.331013in}}%
\pgfpathlineto{\pgfqpoint{2.888435in}{4.331013in}}%
\pgfpathlineto{\pgfqpoint{2.888435in}{4.326755in}}%
\pgfpathmoveto{\pgfqpoint{2.884178in}{4.331013in}}%
\pgfpathlineto{\pgfqpoint{2.884178in}{4.331013in}}%
\pgfpathlineto{\pgfqpoint{2.884178in}{4.335271in}}%
\pgfpathlineto{\pgfqpoint{2.888435in}{4.335271in}}%
\pgfpathlineto{\pgfqpoint{2.888435in}{4.331013in}}%
\pgfpathmoveto{\pgfqpoint{2.884178in}{4.335271in}}%
\pgfpathlineto{\pgfqpoint{2.884178in}{4.335271in}}%
\pgfpathlineto{\pgfqpoint{2.884178in}{4.339529in}}%
\pgfpathlineto{\pgfqpoint{2.888435in}{4.339529in}}%
\pgfpathlineto{\pgfqpoint{2.888435in}{4.335271in}}%
\pgfpathmoveto{\pgfqpoint{2.879920in}{4.339529in}}%
\pgfpathlineto{\pgfqpoint{2.879920in}{4.339529in}}%
\pgfpathlineto{\pgfqpoint{2.879920in}{4.343787in}}%
\pgfpathlineto{\pgfqpoint{2.884178in}{4.343787in}}%
\pgfpathlineto{\pgfqpoint{2.884178in}{4.339529in}}%
\pgfpathmoveto{\pgfqpoint{2.879920in}{4.343787in}}%
\pgfpathlineto{\pgfqpoint{2.879920in}{4.343787in}}%
\pgfpathlineto{\pgfqpoint{2.879920in}{4.348045in}}%
\pgfpathlineto{\pgfqpoint{2.884178in}{4.348045in}}%
\pgfpathlineto{\pgfqpoint{2.884178in}{4.343787in}}%
\pgfpathmoveto{\pgfqpoint{2.884178in}{4.339529in}}%
\pgfpathlineto{\pgfqpoint{2.884178in}{4.339529in}}%
\pgfpathlineto{\pgfqpoint{2.884178in}{4.343787in}}%
\pgfpathlineto{\pgfqpoint{2.888435in}{4.343787in}}%
\pgfpathlineto{\pgfqpoint{2.888435in}{4.339529in}}%
\pgfpathmoveto{\pgfqpoint{2.879920in}{4.348045in}}%
\pgfpathlineto{\pgfqpoint{2.879920in}{4.348045in}}%
\pgfpathlineto{\pgfqpoint{2.879920in}{4.352303in}}%
\pgfpathlineto{\pgfqpoint{2.884178in}{4.352303in}}%
\pgfpathlineto{\pgfqpoint{2.884178in}{4.348045in}}%
\pgfpathmoveto{\pgfqpoint{2.879920in}{4.352303in}}%
\pgfpathlineto{\pgfqpoint{2.879920in}{4.352303in}}%
\pgfpathlineto{\pgfqpoint{2.879920in}{4.356561in}}%
\pgfpathlineto{\pgfqpoint{2.884178in}{4.356561in}}%
\pgfpathlineto{\pgfqpoint{2.884178in}{4.352303in}}%
\pgfpathmoveto{\pgfqpoint{2.879920in}{4.356561in}}%
\pgfpathlineto{\pgfqpoint{2.879920in}{4.356561in}}%
\pgfpathlineto{\pgfqpoint{2.879920in}{4.360819in}}%
\pgfpathlineto{\pgfqpoint{2.884178in}{4.360819in}}%
\pgfpathlineto{\pgfqpoint{2.884178in}{4.356561in}}%
\pgfpathmoveto{\pgfqpoint{2.879920in}{4.360819in}}%
\pgfpathlineto{\pgfqpoint{2.879920in}{4.360819in}}%
\pgfpathlineto{\pgfqpoint{2.879920in}{4.365077in}}%
\pgfpathlineto{\pgfqpoint{2.884178in}{4.365077in}}%
\pgfpathlineto{\pgfqpoint{2.884178in}{4.360819in}}%
\pgfpathmoveto{\pgfqpoint{2.879920in}{4.365077in}}%
\pgfpathlineto{\pgfqpoint{2.879920in}{4.365077in}}%
\pgfpathlineto{\pgfqpoint{2.879920in}{4.369335in}}%
\pgfpathlineto{\pgfqpoint{2.884178in}{4.369335in}}%
\pgfpathlineto{\pgfqpoint{2.884178in}{4.365077in}}%
\pgfpathmoveto{\pgfqpoint{2.879920in}{4.369335in}}%
\pgfpathlineto{\pgfqpoint{2.879920in}{4.369335in}}%
\pgfpathlineto{\pgfqpoint{2.879920in}{4.373593in}}%
\pgfpathlineto{\pgfqpoint{2.884178in}{4.373593in}}%
\pgfpathlineto{\pgfqpoint{2.884178in}{4.369335in}}%
\pgfpathmoveto{\pgfqpoint{2.875662in}{4.382109in}}%
\pgfpathlineto{\pgfqpoint{2.875662in}{4.382109in}}%
\pgfpathlineto{\pgfqpoint{2.875662in}{4.386367in}}%
\pgfpathlineto{\pgfqpoint{2.879920in}{4.386367in}}%
\pgfpathlineto{\pgfqpoint{2.879920in}{4.382109in}}%
\pgfpathmoveto{\pgfqpoint{2.875662in}{4.386367in}}%
\pgfpathlineto{\pgfqpoint{2.875662in}{4.386367in}}%
\pgfpathlineto{\pgfqpoint{2.875662in}{4.390625in}}%
\pgfpathlineto{\pgfqpoint{2.879920in}{4.390625in}}%
\pgfpathlineto{\pgfqpoint{2.879920in}{4.386367in}}%
\pgfpathmoveto{\pgfqpoint{2.879920in}{4.373593in}}%
\pgfpathlineto{\pgfqpoint{2.879920in}{4.373593in}}%
\pgfpathlineto{\pgfqpoint{2.879920in}{4.377851in}}%
\pgfpathlineto{\pgfqpoint{2.884178in}{4.377851in}}%
\pgfpathlineto{\pgfqpoint{2.884178in}{4.373593in}}%
\pgfpathmoveto{\pgfqpoint{2.879920in}{4.377851in}}%
\pgfpathlineto{\pgfqpoint{2.879920in}{4.377851in}}%
\pgfpathlineto{\pgfqpoint{2.879920in}{4.382109in}}%
\pgfpathlineto{\pgfqpoint{2.884178in}{4.382109in}}%
\pgfpathlineto{\pgfqpoint{2.884178in}{4.377851in}}%
\pgfpathmoveto{\pgfqpoint{2.879920in}{4.382109in}}%
\pgfpathlineto{\pgfqpoint{2.879920in}{4.382109in}}%
\pgfpathlineto{\pgfqpoint{2.879920in}{4.386367in}}%
\pgfpathlineto{\pgfqpoint{2.884178in}{4.386367in}}%
\pgfpathlineto{\pgfqpoint{2.884178in}{4.382109in}}%
\pgfpathmoveto{\pgfqpoint{2.875662in}{4.390625in}}%
\pgfpathlineto{\pgfqpoint{2.875662in}{4.390625in}}%
\pgfpathlineto{\pgfqpoint{2.875662in}{4.394883in}}%
\pgfpathlineto{\pgfqpoint{2.879920in}{4.394883in}}%
\pgfpathlineto{\pgfqpoint{2.879920in}{4.390625in}}%
\pgfpathmoveto{\pgfqpoint{2.875662in}{4.394883in}}%
\pgfpathlineto{\pgfqpoint{2.875662in}{4.394883in}}%
\pgfpathlineto{\pgfqpoint{2.875662in}{4.399141in}}%
\pgfpathlineto{\pgfqpoint{2.879920in}{4.399141in}}%
\pgfpathlineto{\pgfqpoint{2.879920in}{4.394883in}}%
\pgfpathmoveto{\pgfqpoint{2.875662in}{4.399141in}}%
\pgfpathlineto{\pgfqpoint{2.875662in}{4.399141in}}%
\pgfpathlineto{\pgfqpoint{2.875662in}{4.403399in}}%
\pgfpathlineto{\pgfqpoint{2.879920in}{4.403399in}}%
\pgfpathlineto{\pgfqpoint{2.879920in}{4.399141in}}%
\pgfpathmoveto{\pgfqpoint{2.875662in}{4.403399in}}%
\pgfpathlineto{\pgfqpoint{2.875662in}{4.403399in}}%
\pgfpathlineto{\pgfqpoint{2.875662in}{4.407657in}}%
\pgfpathlineto{\pgfqpoint{2.879920in}{4.407657in}}%
\pgfpathlineto{\pgfqpoint{2.879920in}{4.403399in}}%
\pgfpathmoveto{\pgfqpoint{2.875662in}{4.407657in}}%
\pgfpathlineto{\pgfqpoint{2.875662in}{4.407657in}}%
\pgfpathlineto{\pgfqpoint{2.875662in}{4.411915in}}%
\pgfpathlineto{\pgfqpoint{2.879920in}{4.411915in}}%
\pgfpathlineto{\pgfqpoint{2.879920in}{4.407657in}}%
\pgfpathmoveto{\pgfqpoint{2.875662in}{4.411915in}}%
\pgfpathlineto{\pgfqpoint{2.875662in}{4.411915in}}%
\pgfpathlineto{\pgfqpoint{2.875662in}{4.416173in}}%
\pgfpathlineto{\pgfqpoint{2.879920in}{4.416173in}}%
\pgfpathlineto{\pgfqpoint{2.879920in}{4.411915in}}%
\pgfpathmoveto{\pgfqpoint{2.875662in}{4.416173in}}%
\pgfpathlineto{\pgfqpoint{2.875662in}{4.416173in}}%
\pgfpathlineto{\pgfqpoint{2.875662in}{4.420431in}}%
\pgfpathlineto{\pgfqpoint{2.879920in}{4.420431in}}%
\pgfpathlineto{\pgfqpoint{2.879920in}{4.416173in}}%
\pgfpathmoveto{\pgfqpoint{2.875662in}{4.420431in}}%
\pgfpathlineto{\pgfqpoint{2.875662in}{4.420431in}}%
\pgfpathlineto{\pgfqpoint{2.875662in}{4.424689in}}%
\pgfpathlineto{\pgfqpoint{2.879920in}{4.424689in}}%
\pgfpathlineto{\pgfqpoint{2.879920in}{4.420431in}}%
\pgfpathmoveto{\pgfqpoint{2.871404in}{4.428947in}}%
\pgfpathlineto{\pgfqpoint{2.871404in}{4.428947in}}%
\pgfpathlineto{\pgfqpoint{2.871404in}{4.433205in}}%
\pgfpathlineto{\pgfqpoint{2.875662in}{4.433205in}}%
\pgfpathlineto{\pgfqpoint{2.875662in}{4.428947in}}%
\pgfpathmoveto{\pgfqpoint{2.875662in}{4.424689in}}%
\pgfpathlineto{\pgfqpoint{2.875662in}{4.424689in}}%
\pgfpathlineto{\pgfqpoint{2.875662in}{4.428947in}}%
\pgfpathlineto{\pgfqpoint{2.879920in}{4.428947in}}%
\pgfpathlineto{\pgfqpoint{2.879920in}{4.424689in}}%
\pgfpathmoveto{\pgfqpoint{2.875662in}{4.428947in}}%
\pgfpathlineto{\pgfqpoint{2.875662in}{4.428947in}}%
\pgfpathlineto{\pgfqpoint{2.875662in}{4.433205in}}%
\pgfpathlineto{\pgfqpoint{2.879920in}{4.433205in}}%
\pgfpathlineto{\pgfqpoint{2.879920in}{4.428947in}}%
\pgfpathmoveto{\pgfqpoint{2.871404in}{4.433205in}}%
\pgfpathlineto{\pgfqpoint{2.871404in}{4.433205in}}%
\pgfpathlineto{\pgfqpoint{2.871404in}{4.437462in}}%
\pgfpathlineto{\pgfqpoint{2.875662in}{4.437462in}}%
\pgfpathlineto{\pgfqpoint{2.875662in}{4.433205in}}%
\pgfpathmoveto{\pgfqpoint{2.871404in}{4.437462in}}%
\pgfpathlineto{\pgfqpoint{2.871404in}{4.437462in}}%
\pgfpathlineto{\pgfqpoint{2.871404in}{4.441720in}}%
\pgfpathlineto{\pgfqpoint{2.875662in}{4.441720in}}%
\pgfpathlineto{\pgfqpoint{2.875662in}{4.437462in}}%
\pgfpathmoveto{\pgfqpoint{2.871404in}{4.441720in}}%
\pgfpathlineto{\pgfqpoint{2.871404in}{4.441720in}}%
\pgfpathlineto{\pgfqpoint{2.871404in}{4.445978in}}%
\pgfpathlineto{\pgfqpoint{2.875662in}{4.445978in}}%
\pgfpathlineto{\pgfqpoint{2.875662in}{4.441720in}}%
\pgfpathmoveto{\pgfqpoint{2.871404in}{4.445978in}}%
\pgfpathlineto{\pgfqpoint{2.871404in}{4.445978in}}%
\pgfpathlineto{\pgfqpoint{2.871404in}{4.450236in}}%
\pgfpathlineto{\pgfqpoint{2.875662in}{4.450236in}}%
\pgfpathlineto{\pgfqpoint{2.875662in}{4.445978in}}%
\pgfpathmoveto{\pgfqpoint{2.871404in}{4.450236in}}%
\pgfpathlineto{\pgfqpoint{2.871404in}{4.450236in}}%
\pgfpathlineto{\pgfqpoint{2.871404in}{4.454494in}}%
\pgfpathlineto{\pgfqpoint{2.875662in}{4.454494in}}%
\pgfpathlineto{\pgfqpoint{2.875662in}{4.450236in}}%
\pgfpathmoveto{\pgfqpoint{2.871404in}{4.454494in}}%
\pgfpathlineto{\pgfqpoint{2.871404in}{4.454494in}}%
\pgfpathlineto{\pgfqpoint{2.871404in}{4.458752in}}%
\pgfpathlineto{\pgfqpoint{2.875662in}{4.458752in}}%
\pgfpathlineto{\pgfqpoint{2.875662in}{4.454494in}}%
\pgfpathmoveto{\pgfqpoint{2.867146in}{4.471525in}}%
\pgfpathlineto{\pgfqpoint{2.867146in}{4.471525in}}%
\pgfpathlineto{\pgfqpoint{2.867146in}{4.475783in}}%
\pgfpathlineto{\pgfqpoint{2.871404in}{4.475783in}}%
\pgfpathlineto{\pgfqpoint{2.871404in}{4.471525in}}%
\pgfpathmoveto{\pgfqpoint{2.867146in}{4.475783in}}%
\pgfpathlineto{\pgfqpoint{2.867146in}{4.475783in}}%
\pgfpathlineto{\pgfqpoint{2.867146in}{4.480041in}}%
\pgfpathlineto{\pgfqpoint{2.871404in}{4.480041in}}%
\pgfpathlineto{\pgfqpoint{2.871404in}{4.475783in}}%
\pgfpathmoveto{\pgfqpoint{2.867146in}{4.480041in}}%
\pgfpathlineto{\pgfqpoint{2.867146in}{4.480041in}}%
\pgfpathlineto{\pgfqpoint{2.867146in}{4.484298in}}%
\pgfpathlineto{\pgfqpoint{2.871404in}{4.484298in}}%
\pgfpathlineto{\pgfqpoint{2.871404in}{4.480041in}}%
\pgfpathmoveto{\pgfqpoint{2.867146in}{4.484298in}}%
\pgfpathlineto{\pgfqpoint{2.867146in}{4.484298in}}%
\pgfpathlineto{\pgfqpoint{2.867146in}{4.488556in}}%
\pgfpathlineto{\pgfqpoint{2.871404in}{4.488556in}}%
\pgfpathlineto{\pgfqpoint{2.871404in}{4.484298in}}%
\pgfpathmoveto{\pgfqpoint{2.867146in}{4.488556in}}%
\pgfpathlineto{\pgfqpoint{2.867146in}{4.488556in}}%
\pgfpathlineto{\pgfqpoint{2.867146in}{4.492814in}}%
\pgfpathlineto{\pgfqpoint{2.871404in}{4.492814in}}%
\pgfpathlineto{\pgfqpoint{2.871404in}{4.488556in}}%
\pgfpathmoveto{\pgfqpoint{2.871404in}{4.458752in}}%
\pgfpathlineto{\pgfqpoint{2.871404in}{4.458752in}}%
\pgfpathlineto{\pgfqpoint{2.871404in}{4.463010in}}%
\pgfpathlineto{\pgfqpoint{2.875662in}{4.463010in}}%
\pgfpathlineto{\pgfqpoint{2.875662in}{4.458752in}}%
\pgfpathmoveto{\pgfqpoint{2.871404in}{4.463010in}}%
\pgfpathlineto{\pgfqpoint{2.871404in}{4.463010in}}%
\pgfpathlineto{\pgfqpoint{2.871404in}{4.467268in}}%
\pgfpathlineto{\pgfqpoint{2.875662in}{4.467268in}}%
\pgfpathlineto{\pgfqpoint{2.875662in}{4.463010in}}%
\pgfpathmoveto{\pgfqpoint{2.871404in}{4.467268in}}%
\pgfpathlineto{\pgfqpoint{2.871404in}{4.467268in}}%
\pgfpathlineto{\pgfqpoint{2.871404in}{4.471525in}}%
\pgfpathlineto{\pgfqpoint{2.875662in}{4.471525in}}%
\pgfpathlineto{\pgfqpoint{2.875662in}{4.467268in}}%
\pgfpathmoveto{\pgfqpoint{2.871404in}{4.471525in}}%
\pgfpathlineto{\pgfqpoint{2.871404in}{4.471525in}}%
\pgfpathlineto{\pgfqpoint{2.871404in}{4.475783in}}%
\pgfpathlineto{\pgfqpoint{2.875662in}{4.475783in}}%
\pgfpathlineto{\pgfqpoint{2.875662in}{4.471525in}}%
\pgfpathmoveto{\pgfqpoint{2.867146in}{4.492814in}}%
\pgfpathlineto{\pgfqpoint{2.867146in}{4.492814in}}%
\pgfpathlineto{\pgfqpoint{2.867146in}{4.497071in}}%
\pgfpathlineto{\pgfqpoint{2.871404in}{4.497071in}}%
\pgfpathlineto{\pgfqpoint{2.871404in}{4.492814in}}%
\pgfpathmoveto{\pgfqpoint{2.867146in}{4.497071in}}%
\pgfpathlineto{\pgfqpoint{2.867146in}{4.497071in}}%
\pgfpathlineto{\pgfqpoint{2.867146in}{4.501329in}}%
\pgfpathlineto{\pgfqpoint{2.871404in}{4.501329in}}%
\pgfpathlineto{\pgfqpoint{2.871404in}{4.497071in}}%
\pgfpathmoveto{\pgfqpoint{2.867146in}{4.501329in}}%
\pgfpathlineto{\pgfqpoint{2.867146in}{4.501329in}}%
\pgfpathlineto{\pgfqpoint{2.867146in}{4.505587in}}%
\pgfpathlineto{\pgfqpoint{2.871404in}{4.505587in}}%
\pgfpathlineto{\pgfqpoint{2.871404in}{4.501329in}}%
\pgfpathmoveto{\pgfqpoint{2.867146in}{4.505587in}}%
\pgfpathlineto{\pgfqpoint{2.867146in}{4.505587in}}%
\pgfpathlineto{\pgfqpoint{2.867146in}{4.509845in}}%
\pgfpathlineto{\pgfqpoint{2.871404in}{4.509845in}}%
\pgfpathlineto{\pgfqpoint{2.871404in}{4.505587in}}%
\pgfpathmoveto{\pgfqpoint{2.867146in}{4.509845in}}%
\pgfpathlineto{\pgfqpoint{2.867146in}{4.509845in}}%
\pgfpathlineto{\pgfqpoint{2.867146in}{4.514102in}}%
\pgfpathlineto{\pgfqpoint{2.871404in}{4.514102in}}%
\pgfpathlineto{\pgfqpoint{2.871404in}{4.509845in}}%
\pgfpathmoveto{\pgfqpoint{2.867146in}{4.514102in}}%
\pgfpathlineto{\pgfqpoint{2.867146in}{4.514102in}}%
\pgfpathlineto{\pgfqpoint{2.867146in}{4.518360in}}%
\pgfpathlineto{\pgfqpoint{2.871404in}{4.518360in}}%
\pgfpathlineto{\pgfqpoint{2.871404in}{4.514102in}}%
\pgfpathmoveto{\pgfqpoint{2.862888in}{4.518360in}}%
\pgfpathlineto{\pgfqpoint{2.862888in}{4.518360in}}%
\pgfpathlineto{\pgfqpoint{2.862888in}{4.522618in}}%
\pgfpathlineto{\pgfqpoint{2.867146in}{4.522618in}}%
\pgfpathlineto{\pgfqpoint{2.867146in}{4.518360in}}%
\pgfpathmoveto{\pgfqpoint{2.862888in}{4.522618in}}%
\pgfpathlineto{\pgfqpoint{2.862888in}{4.522618in}}%
\pgfpathlineto{\pgfqpoint{2.862888in}{4.526875in}}%
\pgfpathlineto{\pgfqpoint{2.867146in}{4.526875in}}%
\pgfpathlineto{\pgfqpoint{2.867146in}{4.522618in}}%
\pgfpathmoveto{\pgfqpoint{2.867146in}{4.518360in}}%
\pgfpathlineto{\pgfqpoint{2.867146in}{4.518360in}}%
\pgfpathlineto{\pgfqpoint{2.867146in}{4.522618in}}%
\pgfpathlineto{\pgfqpoint{2.871404in}{4.522618in}}%
\pgfpathlineto{\pgfqpoint{2.871404in}{4.518360in}}%
\pgfpathmoveto{\pgfqpoint{2.862888in}{4.526875in}}%
\pgfpathlineto{\pgfqpoint{2.862888in}{4.526875in}}%
\pgfpathlineto{\pgfqpoint{2.862888in}{4.531133in}}%
\pgfpathlineto{\pgfqpoint{2.867146in}{4.531133in}}%
\pgfpathlineto{\pgfqpoint{2.867146in}{4.526875in}}%
\pgfpathmoveto{\pgfqpoint{2.862888in}{4.531133in}}%
\pgfpathlineto{\pgfqpoint{2.862888in}{4.531133in}}%
\pgfpathlineto{\pgfqpoint{2.862888in}{4.535391in}}%
\pgfpathlineto{\pgfqpoint{2.867146in}{4.535391in}}%
\pgfpathlineto{\pgfqpoint{2.867146in}{4.531133in}}%
\pgfpathmoveto{\pgfqpoint{2.862888in}{4.535391in}}%
\pgfpathlineto{\pgfqpoint{2.862888in}{4.535391in}}%
\pgfpathlineto{\pgfqpoint{2.862888in}{4.539648in}}%
\pgfpathlineto{\pgfqpoint{2.867146in}{4.539648in}}%
\pgfpathlineto{\pgfqpoint{2.867146in}{4.535391in}}%
\pgfpathmoveto{\pgfqpoint{2.862888in}{4.539648in}}%
\pgfpathlineto{\pgfqpoint{2.862888in}{4.539648in}}%
\pgfpathlineto{\pgfqpoint{2.862888in}{4.543906in}}%
\pgfpathlineto{\pgfqpoint{2.867146in}{4.543906in}}%
\pgfpathlineto{\pgfqpoint{2.867146in}{4.539648in}}%
\pgfpathmoveto{\pgfqpoint{2.862888in}{4.543906in}}%
\pgfpathlineto{\pgfqpoint{2.862888in}{4.543906in}}%
\pgfpathlineto{\pgfqpoint{2.862888in}{4.548164in}}%
\pgfpathlineto{\pgfqpoint{2.867146in}{4.548164in}}%
\pgfpathlineto{\pgfqpoint{2.867146in}{4.543906in}}%
\pgfpathmoveto{\pgfqpoint{2.862888in}{4.548164in}}%
\pgfpathlineto{\pgfqpoint{2.862888in}{4.548164in}}%
\pgfpathlineto{\pgfqpoint{2.862888in}{4.552421in}}%
\pgfpathlineto{\pgfqpoint{2.867146in}{4.552421in}}%
\pgfpathlineto{\pgfqpoint{2.867146in}{4.548164in}}%
\pgfpathmoveto{\pgfqpoint{2.862888in}{4.552421in}}%
\pgfpathlineto{\pgfqpoint{2.862888in}{4.552421in}}%
\pgfpathlineto{\pgfqpoint{2.862888in}{4.556679in}}%
\pgfpathlineto{\pgfqpoint{2.867146in}{4.556679in}}%
\pgfpathlineto{\pgfqpoint{2.867146in}{4.552421in}}%
\pgfpathmoveto{\pgfqpoint{2.862888in}{4.556679in}}%
\pgfpathlineto{\pgfqpoint{2.862888in}{4.556679in}}%
\pgfpathlineto{\pgfqpoint{2.862888in}{4.560937in}}%
\pgfpathlineto{\pgfqpoint{2.867146in}{4.560937in}}%
\pgfpathlineto{\pgfqpoint{2.867146in}{4.556679in}}%
\pgfpathmoveto{\pgfqpoint{2.858630in}{4.560937in}}%
\pgfpathlineto{\pgfqpoint{2.858630in}{4.560937in}}%
\pgfpathlineto{\pgfqpoint{2.858630in}{4.565194in}}%
\pgfpathlineto{\pgfqpoint{2.862888in}{4.565194in}}%
\pgfpathlineto{\pgfqpoint{2.862888in}{4.560937in}}%
\pgfpathmoveto{\pgfqpoint{2.858630in}{4.565194in}}%
\pgfpathlineto{\pgfqpoint{2.858630in}{4.565194in}}%
\pgfpathlineto{\pgfqpoint{2.858630in}{4.569452in}}%
\pgfpathlineto{\pgfqpoint{2.862888in}{4.569452in}}%
\pgfpathlineto{\pgfqpoint{2.862888in}{4.565194in}}%
\pgfpathmoveto{\pgfqpoint{2.858630in}{4.569452in}}%
\pgfpathlineto{\pgfqpoint{2.858630in}{4.569452in}}%
\pgfpathlineto{\pgfqpoint{2.858630in}{4.573710in}}%
\pgfpathlineto{\pgfqpoint{2.862888in}{4.573710in}}%
\pgfpathlineto{\pgfqpoint{2.862888in}{4.569452in}}%
\pgfpathmoveto{\pgfqpoint{2.858630in}{4.573710in}}%
\pgfpathlineto{\pgfqpoint{2.858630in}{4.573710in}}%
\pgfpathlineto{\pgfqpoint{2.858630in}{4.577967in}}%
\pgfpathlineto{\pgfqpoint{2.862888in}{4.577967in}}%
\pgfpathlineto{\pgfqpoint{2.862888in}{4.573710in}}%
\pgfpathmoveto{\pgfqpoint{2.862888in}{4.560937in}}%
\pgfpathlineto{\pgfqpoint{2.862888in}{4.560937in}}%
\pgfpathlineto{\pgfqpoint{2.862888in}{4.565194in}}%
\pgfpathlineto{\pgfqpoint{2.867146in}{4.565194in}}%
\pgfpathlineto{\pgfqpoint{2.867146in}{4.560937in}}%
\pgfpathmoveto{\pgfqpoint{2.858630in}{4.577967in}}%
\pgfpathlineto{\pgfqpoint{2.858630in}{4.577967in}}%
\pgfpathlineto{\pgfqpoint{2.858630in}{4.582225in}}%
\pgfpathlineto{\pgfqpoint{2.862888in}{4.582225in}}%
\pgfpathlineto{\pgfqpoint{2.862888in}{4.577967in}}%
\pgfpathmoveto{\pgfqpoint{2.858630in}{4.582225in}}%
\pgfpathlineto{\pgfqpoint{2.858630in}{4.582225in}}%
\pgfpathlineto{\pgfqpoint{2.858630in}{4.586483in}}%
\pgfpathlineto{\pgfqpoint{2.862888in}{4.586483in}}%
\pgfpathlineto{\pgfqpoint{2.862888in}{4.582225in}}%
\pgfpathmoveto{\pgfqpoint{2.858630in}{4.586483in}}%
\pgfpathlineto{\pgfqpoint{2.858630in}{4.586483in}}%
\pgfpathlineto{\pgfqpoint{2.858630in}{4.590741in}}%
\pgfpathlineto{\pgfqpoint{2.862888in}{4.590741in}}%
\pgfpathlineto{\pgfqpoint{2.862888in}{4.586483in}}%
\pgfpathmoveto{\pgfqpoint{2.858630in}{4.590741in}}%
\pgfpathlineto{\pgfqpoint{2.858630in}{4.590741in}}%
\pgfpathlineto{\pgfqpoint{2.858630in}{4.594998in}}%
\pgfpathlineto{\pgfqpoint{2.862888in}{4.594998in}}%
\pgfpathlineto{\pgfqpoint{2.862888in}{4.590741in}}%
\pgfpathmoveto{\pgfqpoint{3.088555in}{2.581052in}}%
\pgfpathlineto{\pgfqpoint{3.088555in}{2.581052in}}%
\pgfpathlineto{\pgfqpoint{3.088555in}{2.585310in}}%
\pgfpathlineto{\pgfqpoint{3.092813in}{2.585310in}}%
\pgfpathlineto{\pgfqpoint{3.092813in}{2.581052in}}%
\pgfpathmoveto{\pgfqpoint{3.088555in}{2.585310in}}%
\pgfpathlineto{\pgfqpoint{3.088555in}{2.585310in}}%
\pgfpathlineto{\pgfqpoint{3.088555in}{2.589568in}}%
\pgfpathlineto{\pgfqpoint{3.092813in}{2.589568in}}%
\pgfpathlineto{\pgfqpoint{3.092813in}{2.585310in}}%
\pgfpathmoveto{\pgfqpoint{3.088555in}{2.589568in}}%
\pgfpathlineto{\pgfqpoint{3.088555in}{2.589568in}}%
\pgfpathlineto{\pgfqpoint{3.088555in}{2.593826in}}%
\pgfpathlineto{\pgfqpoint{3.092813in}{2.593826in}}%
\pgfpathlineto{\pgfqpoint{3.092813in}{2.589568in}}%
\pgfpathmoveto{\pgfqpoint{3.088555in}{2.593826in}}%
\pgfpathlineto{\pgfqpoint{3.088555in}{2.593826in}}%
\pgfpathlineto{\pgfqpoint{3.088555in}{2.598084in}}%
\pgfpathlineto{\pgfqpoint{3.092813in}{2.598084in}}%
\pgfpathlineto{\pgfqpoint{3.092813in}{2.593826in}}%
\pgfpathmoveto{\pgfqpoint{3.088555in}{2.598084in}}%
\pgfpathlineto{\pgfqpoint{3.088555in}{2.598084in}}%
\pgfpathlineto{\pgfqpoint{3.088555in}{2.602341in}}%
\pgfpathlineto{\pgfqpoint{3.092813in}{2.602341in}}%
\pgfpathlineto{\pgfqpoint{3.092813in}{2.598084in}}%
\pgfpathmoveto{\pgfqpoint{3.088555in}{2.602341in}}%
\pgfpathlineto{\pgfqpoint{3.088555in}{2.602341in}}%
\pgfpathlineto{\pgfqpoint{3.088555in}{2.606599in}}%
\pgfpathlineto{\pgfqpoint{3.092813in}{2.606599in}}%
\pgfpathlineto{\pgfqpoint{3.092813in}{2.602341in}}%
\pgfpathmoveto{\pgfqpoint{3.088555in}{2.606599in}}%
\pgfpathlineto{\pgfqpoint{3.088555in}{2.606599in}}%
\pgfpathlineto{\pgfqpoint{3.088555in}{2.610857in}}%
\pgfpathlineto{\pgfqpoint{3.092813in}{2.610857in}}%
\pgfpathlineto{\pgfqpoint{3.092813in}{2.606599in}}%
\pgfpathmoveto{\pgfqpoint{3.084297in}{2.610857in}}%
\pgfpathlineto{\pgfqpoint{3.084297in}{2.610857in}}%
\pgfpathlineto{\pgfqpoint{3.084297in}{2.615115in}}%
\pgfpathlineto{\pgfqpoint{3.088555in}{2.615115in}}%
\pgfpathlineto{\pgfqpoint{3.088555in}{2.610857in}}%
\pgfpathmoveto{\pgfqpoint{3.084297in}{2.615115in}}%
\pgfpathlineto{\pgfqpoint{3.084297in}{2.615115in}}%
\pgfpathlineto{\pgfqpoint{3.084297in}{2.619373in}}%
\pgfpathlineto{\pgfqpoint{3.088555in}{2.619373in}}%
\pgfpathlineto{\pgfqpoint{3.088555in}{2.615115in}}%
\pgfpathmoveto{\pgfqpoint{3.088555in}{2.610857in}}%
\pgfpathlineto{\pgfqpoint{3.088555in}{2.610857in}}%
\pgfpathlineto{\pgfqpoint{3.088555in}{2.615115in}}%
\pgfpathlineto{\pgfqpoint{3.092813in}{2.615115in}}%
\pgfpathlineto{\pgfqpoint{3.092813in}{2.610857in}}%
\pgfpathmoveto{\pgfqpoint{3.084297in}{2.619373in}}%
\pgfpathlineto{\pgfqpoint{3.084297in}{2.619373in}}%
\pgfpathlineto{\pgfqpoint{3.084297in}{2.623631in}}%
\pgfpathlineto{\pgfqpoint{3.088555in}{2.623631in}}%
\pgfpathlineto{\pgfqpoint{3.088555in}{2.619373in}}%
\pgfpathmoveto{\pgfqpoint{3.084297in}{2.623631in}}%
\pgfpathlineto{\pgfqpoint{3.084297in}{2.623631in}}%
\pgfpathlineto{\pgfqpoint{3.084297in}{2.627889in}}%
\pgfpathlineto{\pgfqpoint{3.088555in}{2.627889in}}%
\pgfpathlineto{\pgfqpoint{3.088555in}{2.623631in}}%
\pgfpathmoveto{\pgfqpoint{3.084297in}{2.627889in}}%
\pgfpathlineto{\pgfqpoint{3.084297in}{2.627889in}}%
\pgfpathlineto{\pgfqpoint{3.084297in}{2.632147in}}%
\pgfpathlineto{\pgfqpoint{3.088555in}{2.632147in}}%
\pgfpathlineto{\pgfqpoint{3.088555in}{2.627889in}}%
\pgfpathmoveto{\pgfqpoint{3.084297in}{2.632147in}}%
\pgfpathlineto{\pgfqpoint{3.084297in}{2.632147in}}%
\pgfpathlineto{\pgfqpoint{3.084297in}{2.636404in}}%
\pgfpathlineto{\pgfqpoint{3.088555in}{2.636404in}}%
\pgfpathlineto{\pgfqpoint{3.088555in}{2.632147in}}%
\pgfpathmoveto{\pgfqpoint{3.080040in}{2.640662in}}%
\pgfpathlineto{\pgfqpoint{3.080040in}{2.640662in}}%
\pgfpathlineto{\pgfqpoint{3.080040in}{2.644920in}}%
\pgfpathlineto{\pgfqpoint{3.084297in}{2.644920in}}%
\pgfpathlineto{\pgfqpoint{3.084297in}{2.640662in}}%
\pgfpathmoveto{\pgfqpoint{3.080040in}{2.644920in}}%
\pgfpathlineto{\pgfqpoint{3.080040in}{2.644920in}}%
\pgfpathlineto{\pgfqpoint{3.080040in}{2.649178in}}%
\pgfpathlineto{\pgfqpoint{3.084297in}{2.649178in}}%
\pgfpathlineto{\pgfqpoint{3.084297in}{2.644920in}}%
\pgfpathmoveto{\pgfqpoint{3.080040in}{2.649178in}}%
\pgfpathlineto{\pgfqpoint{3.080040in}{2.649178in}}%
\pgfpathlineto{\pgfqpoint{3.080040in}{2.653436in}}%
\pgfpathlineto{\pgfqpoint{3.084297in}{2.653436in}}%
\pgfpathlineto{\pgfqpoint{3.084297in}{2.649178in}}%
\pgfpathmoveto{\pgfqpoint{3.084297in}{2.636404in}}%
\pgfpathlineto{\pgfqpoint{3.084297in}{2.636404in}}%
\pgfpathlineto{\pgfqpoint{3.084297in}{2.640662in}}%
\pgfpathlineto{\pgfqpoint{3.088555in}{2.640662in}}%
\pgfpathlineto{\pgfqpoint{3.088555in}{2.636404in}}%
\pgfpathmoveto{\pgfqpoint{3.084297in}{2.640662in}}%
\pgfpathlineto{\pgfqpoint{3.084297in}{2.640662in}}%
\pgfpathlineto{\pgfqpoint{3.084297in}{2.644920in}}%
\pgfpathlineto{\pgfqpoint{3.088555in}{2.644920in}}%
\pgfpathlineto{\pgfqpoint{3.088555in}{2.640662in}}%
\pgfpathmoveto{\pgfqpoint{3.080040in}{2.653436in}}%
\pgfpathlineto{\pgfqpoint{3.080040in}{2.653436in}}%
\pgfpathlineto{\pgfqpoint{3.080040in}{2.657694in}}%
\pgfpathlineto{\pgfqpoint{3.084297in}{2.657694in}}%
\pgfpathlineto{\pgfqpoint{3.084297in}{2.653436in}}%
\pgfpathmoveto{\pgfqpoint{3.080040in}{2.657694in}}%
\pgfpathlineto{\pgfqpoint{3.080040in}{2.657694in}}%
\pgfpathlineto{\pgfqpoint{3.080040in}{2.661952in}}%
\pgfpathlineto{\pgfqpoint{3.084297in}{2.661952in}}%
\pgfpathlineto{\pgfqpoint{3.084297in}{2.657694in}}%
\pgfpathmoveto{\pgfqpoint{3.075782in}{2.666209in}}%
\pgfpathlineto{\pgfqpoint{3.075782in}{2.666209in}}%
\pgfpathlineto{\pgfqpoint{3.075782in}{2.670467in}}%
\pgfpathlineto{\pgfqpoint{3.080040in}{2.670467in}}%
\pgfpathlineto{\pgfqpoint{3.080040in}{2.666209in}}%
\pgfpathmoveto{\pgfqpoint{3.080040in}{2.661952in}}%
\pgfpathlineto{\pgfqpoint{3.080040in}{2.661952in}}%
\pgfpathlineto{\pgfqpoint{3.080040in}{2.666209in}}%
\pgfpathlineto{\pgfqpoint{3.084297in}{2.666209in}}%
\pgfpathlineto{\pgfqpoint{3.084297in}{2.661952in}}%
\pgfpathmoveto{\pgfqpoint{3.080040in}{2.666209in}}%
\pgfpathlineto{\pgfqpoint{3.080040in}{2.666209in}}%
\pgfpathlineto{\pgfqpoint{3.080040in}{2.670467in}}%
\pgfpathlineto{\pgfqpoint{3.084297in}{2.670467in}}%
\pgfpathlineto{\pgfqpoint{3.084297in}{2.666209in}}%
\pgfpathmoveto{\pgfqpoint{3.075782in}{2.670467in}}%
\pgfpathlineto{\pgfqpoint{3.075782in}{2.670467in}}%
\pgfpathlineto{\pgfqpoint{3.075782in}{2.674725in}}%
\pgfpathlineto{\pgfqpoint{3.080040in}{2.674725in}}%
\pgfpathlineto{\pgfqpoint{3.080040in}{2.670467in}}%
\pgfpathmoveto{\pgfqpoint{3.075782in}{2.674725in}}%
\pgfpathlineto{\pgfqpoint{3.075782in}{2.674725in}}%
\pgfpathlineto{\pgfqpoint{3.075782in}{2.678983in}}%
\pgfpathlineto{\pgfqpoint{3.080040in}{2.678983in}}%
\pgfpathlineto{\pgfqpoint{3.080040in}{2.674725in}}%
\pgfpathmoveto{\pgfqpoint{3.075782in}{2.678983in}}%
\pgfpathlineto{\pgfqpoint{3.075782in}{2.678983in}}%
\pgfpathlineto{\pgfqpoint{3.075782in}{2.683241in}}%
\pgfpathlineto{\pgfqpoint{3.080040in}{2.683241in}}%
\pgfpathlineto{\pgfqpoint{3.080040in}{2.678983in}}%
\pgfpathmoveto{\pgfqpoint{3.075782in}{2.683241in}}%
\pgfpathlineto{\pgfqpoint{3.075782in}{2.683241in}}%
\pgfpathlineto{\pgfqpoint{3.075782in}{2.687499in}}%
\pgfpathlineto{\pgfqpoint{3.080040in}{2.687499in}}%
\pgfpathlineto{\pgfqpoint{3.080040in}{2.683241in}}%
\pgfpathmoveto{\pgfqpoint{3.071524in}{2.696015in}}%
\pgfpathlineto{\pgfqpoint{3.071524in}{2.696015in}}%
\pgfpathlineto{\pgfqpoint{3.071524in}{2.700273in}}%
\pgfpathlineto{\pgfqpoint{3.075782in}{2.700273in}}%
\pgfpathlineto{\pgfqpoint{3.075782in}{2.696015in}}%
\pgfpathmoveto{\pgfqpoint{3.071524in}{2.700273in}}%
\pgfpathlineto{\pgfqpoint{3.071524in}{2.700273in}}%
\pgfpathlineto{\pgfqpoint{3.071524in}{2.704531in}}%
\pgfpathlineto{\pgfqpoint{3.075782in}{2.704531in}}%
\pgfpathlineto{\pgfqpoint{3.075782in}{2.700273in}}%
\pgfpathmoveto{\pgfqpoint{3.071524in}{2.704531in}}%
\pgfpathlineto{\pgfqpoint{3.071524in}{2.704531in}}%
\pgfpathlineto{\pgfqpoint{3.071524in}{2.708789in}}%
\pgfpathlineto{\pgfqpoint{3.075782in}{2.708789in}}%
\pgfpathlineto{\pgfqpoint{3.075782in}{2.704531in}}%
\pgfpathmoveto{\pgfqpoint{3.071524in}{2.708789in}}%
\pgfpathlineto{\pgfqpoint{3.071524in}{2.708789in}}%
\pgfpathlineto{\pgfqpoint{3.071524in}{2.713046in}}%
\pgfpathlineto{\pgfqpoint{3.075782in}{2.713046in}}%
\pgfpathlineto{\pgfqpoint{3.075782in}{2.708789in}}%
\pgfpathmoveto{\pgfqpoint{3.071524in}{2.713046in}}%
\pgfpathlineto{\pgfqpoint{3.071524in}{2.713046in}}%
\pgfpathlineto{\pgfqpoint{3.071524in}{2.717304in}}%
\pgfpathlineto{\pgfqpoint{3.075782in}{2.717304in}}%
\pgfpathlineto{\pgfqpoint{3.075782in}{2.713046in}}%
\pgfpathmoveto{\pgfqpoint{3.071524in}{2.717304in}}%
\pgfpathlineto{\pgfqpoint{3.071524in}{2.717304in}}%
\pgfpathlineto{\pgfqpoint{3.071524in}{2.721562in}}%
\pgfpathlineto{\pgfqpoint{3.075782in}{2.721562in}}%
\pgfpathlineto{\pgfqpoint{3.075782in}{2.717304in}}%
\pgfpathmoveto{\pgfqpoint{3.075782in}{2.687499in}}%
\pgfpathlineto{\pgfqpoint{3.075782in}{2.687499in}}%
\pgfpathlineto{\pgfqpoint{3.075782in}{2.691757in}}%
\pgfpathlineto{\pgfqpoint{3.080040in}{2.691757in}}%
\pgfpathlineto{\pgfqpoint{3.080040in}{2.687499in}}%
\pgfpathmoveto{\pgfqpoint{3.075782in}{2.691757in}}%
\pgfpathlineto{\pgfqpoint{3.075782in}{2.691757in}}%
\pgfpathlineto{\pgfqpoint{3.075782in}{2.696015in}}%
\pgfpathlineto{\pgfqpoint{3.080040in}{2.696015in}}%
\pgfpathlineto{\pgfqpoint{3.080040in}{2.691757in}}%
\pgfpathmoveto{\pgfqpoint{3.075782in}{2.696015in}}%
\pgfpathlineto{\pgfqpoint{3.075782in}{2.696015in}}%
\pgfpathlineto{\pgfqpoint{3.075782in}{2.700273in}}%
\pgfpathlineto{\pgfqpoint{3.080040in}{2.700273in}}%
\pgfpathlineto{\pgfqpoint{3.080040in}{2.696015in}}%
\pgfpathmoveto{\pgfqpoint{3.067266in}{2.725820in}}%
\pgfpathlineto{\pgfqpoint{3.067266in}{2.725820in}}%
\pgfpathlineto{\pgfqpoint{3.067266in}{2.730078in}}%
\pgfpathlineto{\pgfqpoint{3.071524in}{2.730078in}}%
\pgfpathlineto{\pgfqpoint{3.071524in}{2.725820in}}%
\pgfpathmoveto{\pgfqpoint{3.071524in}{2.721562in}}%
\pgfpathlineto{\pgfqpoint{3.071524in}{2.721562in}}%
\pgfpathlineto{\pgfqpoint{3.071524in}{2.725820in}}%
\pgfpathlineto{\pgfqpoint{3.075782in}{2.725820in}}%
\pgfpathlineto{\pgfqpoint{3.075782in}{2.721562in}}%
\pgfpathmoveto{\pgfqpoint{3.071524in}{2.725820in}}%
\pgfpathlineto{\pgfqpoint{3.071524in}{2.725820in}}%
\pgfpathlineto{\pgfqpoint{3.071524in}{2.730078in}}%
\pgfpathlineto{\pgfqpoint{3.075782in}{2.730078in}}%
\pgfpathlineto{\pgfqpoint{3.075782in}{2.725820in}}%
\pgfpathmoveto{\pgfqpoint{3.067266in}{2.730078in}}%
\pgfpathlineto{\pgfqpoint{3.067266in}{2.730078in}}%
\pgfpathlineto{\pgfqpoint{3.067266in}{2.734336in}}%
\pgfpathlineto{\pgfqpoint{3.071524in}{2.734336in}}%
\pgfpathlineto{\pgfqpoint{3.071524in}{2.730078in}}%
\pgfpathmoveto{\pgfqpoint{3.067266in}{2.734336in}}%
\pgfpathlineto{\pgfqpoint{3.067266in}{2.734336in}}%
\pgfpathlineto{\pgfqpoint{3.067266in}{2.738594in}}%
\pgfpathlineto{\pgfqpoint{3.071524in}{2.738594in}}%
\pgfpathlineto{\pgfqpoint{3.071524in}{2.734336in}}%
\pgfpathmoveto{\pgfqpoint{3.067266in}{2.738594in}}%
\pgfpathlineto{\pgfqpoint{3.067266in}{2.738594in}}%
\pgfpathlineto{\pgfqpoint{3.067266in}{2.742852in}}%
\pgfpathlineto{\pgfqpoint{3.071524in}{2.742852in}}%
\pgfpathlineto{\pgfqpoint{3.071524in}{2.738594in}}%
\pgfpathmoveto{\pgfqpoint{3.067266in}{2.742852in}}%
\pgfpathlineto{\pgfqpoint{3.067266in}{2.742852in}}%
\pgfpathlineto{\pgfqpoint{3.067266in}{2.747110in}}%
\pgfpathlineto{\pgfqpoint{3.071524in}{2.747110in}}%
\pgfpathlineto{\pgfqpoint{3.071524in}{2.742852in}}%
\pgfpathmoveto{\pgfqpoint{3.067266in}{2.747110in}}%
\pgfpathlineto{\pgfqpoint{3.067266in}{2.747110in}}%
\pgfpathlineto{\pgfqpoint{3.067266in}{2.751368in}}%
\pgfpathlineto{\pgfqpoint{3.071524in}{2.751368in}}%
\pgfpathlineto{\pgfqpoint{3.071524in}{2.747110in}}%
\pgfpathmoveto{\pgfqpoint{3.067266in}{2.751368in}}%
\pgfpathlineto{\pgfqpoint{3.067266in}{2.751368in}}%
\pgfpathlineto{\pgfqpoint{3.067266in}{2.755626in}}%
\pgfpathlineto{\pgfqpoint{3.071524in}{2.755626in}}%
\pgfpathlineto{\pgfqpoint{3.071524in}{2.751368in}}%
\pgfpathmoveto{\pgfqpoint{3.054492in}{2.815238in}}%
\pgfpathlineto{\pgfqpoint{3.054492in}{2.815238in}}%
\pgfpathlineto{\pgfqpoint{3.054492in}{2.819496in}}%
\pgfpathlineto{\pgfqpoint{3.058750in}{2.819496in}}%
\pgfpathlineto{\pgfqpoint{3.058750in}{2.815238in}}%
\pgfpathmoveto{\pgfqpoint{3.054492in}{2.819496in}}%
\pgfpathlineto{\pgfqpoint{3.054492in}{2.819496in}}%
\pgfpathlineto{\pgfqpoint{3.054492in}{2.823754in}}%
\pgfpathlineto{\pgfqpoint{3.058750in}{2.823754in}}%
\pgfpathlineto{\pgfqpoint{3.058750in}{2.819496in}}%
\pgfpathmoveto{\pgfqpoint{3.063008in}{2.755626in}}%
\pgfpathlineto{\pgfqpoint{3.063008in}{2.755626in}}%
\pgfpathlineto{\pgfqpoint{3.063008in}{2.759884in}}%
\pgfpathlineto{\pgfqpoint{3.067266in}{2.759884in}}%
\pgfpathlineto{\pgfqpoint{3.067266in}{2.755626in}}%
\pgfpathmoveto{\pgfqpoint{3.063008in}{2.759884in}}%
\pgfpathlineto{\pgfqpoint{3.063008in}{2.759884in}}%
\pgfpathlineto{\pgfqpoint{3.063008in}{2.764142in}}%
\pgfpathlineto{\pgfqpoint{3.067266in}{2.764142in}}%
\pgfpathlineto{\pgfqpoint{3.067266in}{2.759884in}}%
\pgfpathmoveto{\pgfqpoint{3.063008in}{2.764142in}}%
\pgfpathlineto{\pgfqpoint{3.063008in}{2.764142in}}%
\pgfpathlineto{\pgfqpoint{3.063008in}{2.768400in}}%
\pgfpathlineto{\pgfqpoint{3.067266in}{2.768400in}}%
\pgfpathlineto{\pgfqpoint{3.067266in}{2.764142in}}%
\pgfpathmoveto{\pgfqpoint{3.063008in}{2.768400in}}%
\pgfpathlineto{\pgfqpoint{3.063008in}{2.768400in}}%
\pgfpathlineto{\pgfqpoint{3.063008in}{2.772658in}}%
\pgfpathlineto{\pgfqpoint{3.067266in}{2.772658in}}%
\pgfpathlineto{\pgfqpoint{3.067266in}{2.768400in}}%
\pgfpathmoveto{\pgfqpoint{3.067266in}{2.755626in}}%
\pgfpathlineto{\pgfqpoint{3.067266in}{2.755626in}}%
\pgfpathlineto{\pgfqpoint{3.067266in}{2.759884in}}%
\pgfpathlineto{\pgfqpoint{3.071524in}{2.759884in}}%
\pgfpathlineto{\pgfqpoint{3.071524in}{2.755626in}}%
\pgfpathmoveto{\pgfqpoint{3.063008in}{2.772658in}}%
\pgfpathlineto{\pgfqpoint{3.063008in}{2.772658in}}%
\pgfpathlineto{\pgfqpoint{3.063008in}{2.776916in}}%
\pgfpathlineto{\pgfqpoint{3.067266in}{2.776916in}}%
\pgfpathlineto{\pgfqpoint{3.067266in}{2.772658in}}%
\pgfpathmoveto{\pgfqpoint{3.063008in}{2.776916in}}%
\pgfpathlineto{\pgfqpoint{3.063008in}{2.776916in}}%
\pgfpathlineto{\pgfqpoint{3.063008in}{2.781174in}}%
\pgfpathlineto{\pgfqpoint{3.067266in}{2.781174in}}%
\pgfpathlineto{\pgfqpoint{3.067266in}{2.776916in}}%
\pgfpathmoveto{\pgfqpoint{3.058750in}{2.785432in}}%
\pgfpathlineto{\pgfqpoint{3.058750in}{2.785432in}}%
\pgfpathlineto{\pgfqpoint{3.058750in}{2.789690in}}%
\pgfpathlineto{\pgfqpoint{3.063008in}{2.789690in}}%
\pgfpathlineto{\pgfqpoint{3.063008in}{2.785432in}}%
\pgfpathmoveto{\pgfqpoint{3.063008in}{2.781174in}}%
\pgfpathlineto{\pgfqpoint{3.063008in}{2.781174in}}%
\pgfpathlineto{\pgfqpoint{3.063008in}{2.785432in}}%
\pgfpathlineto{\pgfqpoint{3.067266in}{2.785432in}}%
\pgfpathlineto{\pgfqpoint{3.067266in}{2.781174in}}%
\pgfpathmoveto{\pgfqpoint{3.063008in}{2.785432in}}%
\pgfpathlineto{\pgfqpoint{3.063008in}{2.785432in}}%
\pgfpathlineto{\pgfqpoint{3.063008in}{2.789690in}}%
\pgfpathlineto{\pgfqpoint{3.067266in}{2.789690in}}%
\pgfpathlineto{\pgfqpoint{3.067266in}{2.785432in}}%
\pgfpathmoveto{\pgfqpoint{3.058750in}{2.789690in}}%
\pgfpathlineto{\pgfqpoint{3.058750in}{2.789690in}}%
\pgfpathlineto{\pgfqpoint{3.058750in}{2.793948in}}%
\pgfpathlineto{\pgfqpoint{3.063008in}{2.793948in}}%
\pgfpathlineto{\pgfqpoint{3.063008in}{2.789690in}}%
\pgfpathmoveto{\pgfqpoint{3.058750in}{2.793948in}}%
\pgfpathlineto{\pgfqpoint{3.058750in}{2.793948in}}%
\pgfpathlineto{\pgfqpoint{3.058750in}{2.798206in}}%
\pgfpathlineto{\pgfqpoint{3.063008in}{2.798206in}}%
\pgfpathlineto{\pgfqpoint{3.063008in}{2.793948in}}%
\pgfpathmoveto{\pgfqpoint{3.058750in}{2.798206in}}%
\pgfpathlineto{\pgfqpoint{3.058750in}{2.798206in}}%
\pgfpathlineto{\pgfqpoint{3.058750in}{2.802464in}}%
\pgfpathlineto{\pgfqpoint{3.063008in}{2.802464in}}%
\pgfpathlineto{\pgfqpoint{3.063008in}{2.798206in}}%
\pgfpathmoveto{\pgfqpoint{3.058750in}{2.802464in}}%
\pgfpathlineto{\pgfqpoint{3.058750in}{2.802464in}}%
\pgfpathlineto{\pgfqpoint{3.058750in}{2.806722in}}%
\pgfpathlineto{\pgfqpoint{3.063008in}{2.806722in}}%
\pgfpathlineto{\pgfqpoint{3.063008in}{2.802464in}}%
\pgfpathmoveto{\pgfqpoint{3.058750in}{2.806722in}}%
\pgfpathlineto{\pgfqpoint{3.058750in}{2.806722in}}%
\pgfpathlineto{\pgfqpoint{3.058750in}{2.810980in}}%
\pgfpathlineto{\pgfqpoint{3.063008in}{2.810980in}}%
\pgfpathlineto{\pgfqpoint{3.063008in}{2.806722in}}%
\pgfpathmoveto{\pgfqpoint{3.058750in}{2.810980in}}%
\pgfpathlineto{\pgfqpoint{3.058750in}{2.810980in}}%
\pgfpathlineto{\pgfqpoint{3.058750in}{2.815238in}}%
\pgfpathlineto{\pgfqpoint{3.063008in}{2.815238in}}%
\pgfpathlineto{\pgfqpoint{3.063008in}{2.810980in}}%
\pgfpathmoveto{\pgfqpoint{3.058750in}{2.815238in}}%
\pgfpathlineto{\pgfqpoint{3.058750in}{2.815238in}}%
\pgfpathlineto{\pgfqpoint{3.058750in}{2.819496in}}%
\pgfpathlineto{\pgfqpoint{3.063008in}{2.819496in}}%
\pgfpathlineto{\pgfqpoint{3.063008in}{2.815238in}}%
\pgfpathmoveto{\pgfqpoint{3.054492in}{2.823754in}}%
\pgfpathlineto{\pgfqpoint{3.054492in}{2.823754in}}%
\pgfpathlineto{\pgfqpoint{3.054492in}{2.828011in}}%
\pgfpathlineto{\pgfqpoint{3.058750in}{2.828011in}}%
\pgfpathlineto{\pgfqpoint{3.058750in}{2.823754in}}%
\pgfpathmoveto{\pgfqpoint{3.054492in}{2.828011in}}%
\pgfpathlineto{\pgfqpoint{3.054492in}{2.828011in}}%
\pgfpathlineto{\pgfqpoint{3.054492in}{2.832269in}}%
\pgfpathlineto{\pgfqpoint{3.058750in}{2.832269in}}%
\pgfpathlineto{\pgfqpoint{3.058750in}{2.828011in}}%
\pgfpathmoveto{\pgfqpoint{3.054492in}{2.832269in}}%
\pgfpathlineto{\pgfqpoint{3.054492in}{2.832269in}}%
\pgfpathlineto{\pgfqpoint{3.054492in}{2.836526in}}%
\pgfpathlineto{\pgfqpoint{3.058750in}{2.836526in}}%
\pgfpathlineto{\pgfqpoint{3.058750in}{2.832269in}}%
\pgfpathmoveto{\pgfqpoint{3.054492in}{2.836526in}}%
\pgfpathlineto{\pgfqpoint{3.054492in}{2.836526in}}%
\pgfpathlineto{\pgfqpoint{3.054492in}{2.840784in}}%
\pgfpathlineto{\pgfqpoint{3.058750in}{2.840784in}}%
\pgfpathlineto{\pgfqpoint{3.058750in}{2.836526in}}%
\pgfpathmoveto{\pgfqpoint{3.050235in}{2.845042in}}%
\pgfpathlineto{\pgfqpoint{3.050235in}{2.845042in}}%
\pgfpathlineto{\pgfqpoint{3.050235in}{2.849299in}}%
\pgfpathlineto{\pgfqpoint{3.054492in}{2.849299in}}%
\pgfpathlineto{\pgfqpoint{3.054492in}{2.845042in}}%
\pgfpathmoveto{\pgfqpoint{3.054492in}{2.840784in}}%
\pgfpathlineto{\pgfqpoint{3.054492in}{2.840784in}}%
\pgfpathlineto{\pgfqpoint{3.054492in}{2.845042in}}%
\pgfpathlineto{\pgfqpoint{3.058750in}{2.845042in}}%
\pgfpathlineto{\pgfqpoint{3.058750in}{2.840784in}}%
\pgfpathmoveto{\pgfqpoint{3.054492in}{2.845042in}}%
\pgfpathlineto{\pgfqpoint{3.054492in}{2.845042in}}%
\pgfpathlineto{\pgfqpoint{3.054492in}{2.849299in}}%
\pgfpathlineto{\pgfqpoint{3.058750in}{2.849299in}}%
\pgfpathlineto{\pgfqpoint{3.058750in}{2.845042in}}%
\pgfpathmoveto{\pgfqpoint{3.050235in}{2.849299in}}%
\pgfpathlineto{\pgfqpoint{3.050235in}{2.849299in}}%
\pgfpathlineto{\pgfqpoint{3.050235in}{2.853557in}}%
\pgfpathlineto{\pgfqpoint{3.054492in}{2.853557in}}%
\pgfpathlineto{\pgfqpoint{3.054492in}{2.849299in}}%
\pgfpathmoveto{\pgfqpoint{3.050235in}{2.853557in}}%
\pgfpathlineto{\pgfqpoint{3.050235in}{2.853557in}}%
\pgfpathlineto{\pgfqpoint{3.050235in}{2.857814in}}%
\pgfpathlineto{\pgfqpoint{3.054492in}{2.857814in}}%
\pgfpathlineto{\pgfqpoint{3.054492in}{2.853557in}}%
\pgfpathmoveto{\pgfqpoint{3.050235in}{2.857814in}}%
\pgfpathlineto{\pgfqpoint{3.050235in}{2.857814in}}%
\pgfpathlineto{\pgfqpoint{3.050235in}{2.862072in}}%
\pgfpathlineto{\pgfqpoint{3.054492in}{2.862072in}}%
\pgfpathlineto{\pgfqpoint{3.054492in}{2.857814in}}%
\pgfpathmoveto{\pgfqpoint{3.050235in}{2.862072in}}%
\pgfpathlineto{\pgfqpoint{3.050235in}{2.862072in}}%
\pgfpathlineto{\pgfqpoint{3.050235in}{2.866329in}}%
\pgfpathlineto{\pgfqpoint{3.054492in}{2.866329in}}%
\pgfpathlineto{\pgfqpoint{3.054492in}{2.862072in}}%
\pgfpathmoveto{\pgfqpoint{3.050235in}{2.866329in}}%
\pgfpathlineto{\pgfqpoint{3.050235in}{2.866329in}}%
\pgfpathlineto{\pgfqpoint{3.050235in}{2.870587in}}%
\pgfpathlineto{\pgfqpoint{3.054492in}{2.870587in}}%
\pgfpathlineto{\pgfqpoint{3.054492in}{2.866329in}}%
\pgfpathmoveto{\pgfqpoint{3.050235in}{2.870587in}}%
\pgfpathlineto{\pgfqpoint{3.050235in}{2.870587in}}%
\pgfpathlineto{\pgfqpoint{3.050235in}{2.874845in}}%
\pgfpathlineto{\pgfqpoint{3.054492in}{2.874845in}}%
\pgfpathlineto{\pgfqpoint{3.054492in}{2.870587in}}%
\pgfpathmoveto{\pgfqpoint{3.045977in}{2.874845in}}%
\pgfpathlineto{\pgfqpoint{3.045977in}{2.874845in}}%
\pgfpathlineto{\pgfqpoint{3.045977in}{2.879102in}}%
\pgfpathlineto{\pgfqpoint{3.050235in}{2.879102in}}%
\pgfpathlineto{\pgfqpoint{3.050235in}{2.874845in}}%
\pgfpathmoveto{\pgfqpoint{3.045977in}{2.879102in}}%
\pgfpathlineto{\pgfqpoint{3.045977in}{2.879102in}}%
\pgfpathlineto{\pgfqpoint{3.045977in}{2.883360in}}%
\pgfpathlineto{\pgfqpoint{3.050235in}{2.883360in}}%
\pgfpathlineto{\pgfqpoint{3.050235in}{2.879102in}}%
\pgfpathmoveto{\pgfqpoint{3.045977in}{2.883360in}}%
\pgfpathlineto{\pgfqpoint{3.045977in}{2.883360in}}%
\pgfpathlineto{\pgfqpoint{3.045977in}{2.887617in}}%
\pgfpathlineto{\pgfqpoint{3.050235in}{2.887617in}}%
\pgfpathlineto{\pgfqpoint{3.050235in}{2.883360in}}%
\pgfpathmoveto{\pgfqpoint{3.045977in}{2.887617in}}%
\pgfpathlineto{\pgfqpoint{3.045977in}{2.887617in}}%
\pgfpathlineto{\pgfqpoint{3.045977in}{2.891875in}}%
\pgfpathlineto{\pgfqpoint{3.050235in}{2.891875in}}%
\pgfpathlineto{\pgfqpoint{3.050235in}{2.887617in}}%
\pgfpathmoveto{\pgfqpoint{3.050235in}{2.874845in}}%
\pgfpathlineto{\pgfqpoint{3.050235in}{2.874845in}}%
\pgfpathlineto{\pgfqpoint{3.050235in}{2.879102in}}%
\pgfpathlineto{\pgfqpoint{3.054492in}{2.879102in}}%
\pgfpathlineto{\pgfqpoint{3.054492in}{2.874845in}}%
\pgfpathmoveto{\pgfqpoint{3.045977in}{2.891875in}}%
\pgfpathlineto{\pgfqpoint{3.045977in}{2.891875in}}%
\pgfpathlineto{\pgfqpoint{3.045977in}{2.896133in}}%
\pgfpathlineto{\pgfqpoint{3.050235in}{2.896133in}}%
\pgfpathlineto{\pgfqpoint{3.050235in}{2.891875in}}%
\pgfpathmoveto{\pgfqpoint{3.045977in}{2.896133in}}%
\pgfpathlineto{\pgfqpoint{3.045977in}{2.896133in}}%
\pgfpathlineto{\pgfqpoint{3.045977in}{2.900390in}}%
\pgfpathlineto{\pgfqpoint{3.050235in}{2.900390in}}%
\pgfpathlineto{\pgfqpoint{3.050235in}{2.896133in}}%
\pgfpathmoveto{\pgfqpoint{3.045977in}{2.900390in}}%
\pgfpathlineto{\pgfqpoint{3.045977in}{2.900390in}}%
\pgfpathlineto{\pgfqpoint{3.045977in}{2.904648in}}%
\pgfpathlineto{\pgfqpoint{3.050235in}{2.904648in}}%
\pgfpathlineto{\pgfqpoint{3.050235in}{2.900390in}}%
\pgfpathmoveto{\pgfqpoint{3.045977in}{2.904648in}}%
\pgfpathlineto{\pgfqpoint{3.045977in}{2.904648in}}%
\pgfpathlineto{\pgfqpoint{3.045977in}{2.908905in}}%
\pgfpathlineto{\pgfqpoint{3.050235in}{2.908905in}}%
\pgfpathlineto{\pgfqpoint{3.050235in}{2.904648in}}%
\pgfpathmoveto{\pgfqpoint{3.041719in}{2.908905in}}%
\pgfpathlineto{\pgfqpoint{3.041719in}{2.908905in}}%
\pgfpathlineto{\pgfqpoint{3.041719in}{2.913163in}}%
\pgfpathlineto{\pgfqpoint{3.045977in}{2.913163in}}%
\pgfpathlineto{\pgfqpoint{3.045977in}{2.908905in}}%
\pgfpathmoveto{\pgfqpoint{3.041719in}{2.913163in}}%
\pgfpathlineto{\pgfqpoint{3.041719in}{2.913163in}}%
\pgfpathlineto{\pgfqpoint{3.041719in}{2.917421in}}%
\pgfpathlineto{\pgfqpoint{3.045977in}{2.917421in}}%
\pgfpathlineto{\pgfqpoint{3.045977in}{2.913163in}}%
\pgfpathmoveto{\pgfqpoint{3.045977in}{2.908905in}}%
\pgfpathlineto{\pgfqpoint{3.045977in}{2.908905in}}%
\pgfpathlineto{\pgfqpoint{3.045977in}{2.913163in}}%
\pgfpathlineto{\pgfqpoint{3.050235in}{2.913163in}}%
\pgfpathlineto{\pgfqpoint{3.050235in}{2.908905in}}%
\pgfpathmoveto{\pgfqpoint{3.041719in}{2.917421in}}%
\pgfpathlineto{\pgfqpoint{3.041719in}{2.917421in}}%
\pgfpathlineto{\pgfqpoint{3.041719in}{2.921678in}}%
\pgfpathlineto{\pgfqpoint{3.045977in}{2.921678in}}%
\pgfpathlineto{\pgfqpoint{3.045977in}{2.917421in}}%
\pgfpathmoveto{\pgfqpoint{3.041719in}{2.921678in}}%
\pgfpathlineto{\pgfqpoint{3.041719in}{2.921678in}}%
\pgfpathlineto{\pgfqpoint{3.041719in}{2.925936in}}%
\pgfpathlineto{\pgfqpoint{3.045977in}{2.925936in}}%
\pgfpathlineto{\pgfqpoint{3.045977in}{2.921678in}}%
\pgfpathmoveto{\pgfqpoint{3.037461in}{2.938708in}}%
\pgfpathlineto{\pgfqpoint{3.037461in}{2.938708in}}%
\pgfpathlineto{\pgfqpoint{3.037461in}{2.942966in}}%
\pgfpathlineto{\pgfqpoint{3.041719in}{2.942966in}}%
\pgfpathlineto{\pgfqpoint{3.041719in}{2.938708in}}%
\pgfpathmoveto{\pgfqpoint{3.037461in}{2.942966in}}%
\pgfpathlineto{\pgfqpoint{3.037461in}{2.942966in}}%
\pgfpathlineto{\pgfqpoint{3.037461in}{2.947224in}}%
\pgfpathlineto{\pgfqpoint{3.041719in}{2.947224in}}%
\pgfpathlineto{\pgfqpoint{3.041719in}{2.942966in}}%
\pgfpathmoveto{\pgfqpoint{3.037461in}{2.947224in}}%
\pgfpathlineto{\pgfqpoint{3.037461in}{2.947224in}}%
\pgfpathlineto{\pgfqpoint{3.037461in}{2.951481in}}%
\pgfpathlineto{\pgfqpoint{3.041719in}{2.951481in}}%
\pgfpathlineto{\pgfqpoint{3.041719in}{2.947224in}}%
\pgfpathmoveto{\pgfqpoint{3.037461in}{2.951481in}}%
\pgfpathlineto{\pgfqpoint{3.037461in}{2.951481in}}%
\pgfpathlineto{\pgfqpoint{3.037461in}{2.955739in}}%
\pgfpathlineto{\pgfqpoint{3.041719in}{2.955739in}}%
\pgfpathlineto{\pgfqpoint{3.041719in}{2.951481in}}%
\pgfpathmoveto{\pgfqpoint{3.037461in}{2.955739in}}%
\pgfpathlineto{\pgfqpoint{3.037461in}{2.955739in}}%
\pgfpathlineto{\pgfqpoint{3.037461in}{2.959996in}}%
\pgfpathlineto{\pgfqpoint{3.041719in}{2.959996in}}%
\pgfpathlineto{\pgfqpoint{3.041719in}{2.955739in}}%
\pgfpathmoveto{\pgfqpoint{3.041719in}{2.925936in}}%
\pgfpathlineto{\pgfqpoint{3.041719in}{2.925936in}}%
\pgfpathlineto{\pgfqpoint{3.041719in}{2.930193in}}%
\pgfpathlineto{\pgfqpoint{3.045977in}{2.930193in}}%
\pgfpathlineto{\pgfqpoint{3.045977in}{2.925936in}}%
\pgfpathmoveto{\pgfqpoint{3.041719in}{2.930193in}}%
\pgfpathlineto{\pgfqpoint{3.041719in}{2.930193in}}%
\pgfpathlineto{\pgfqpoint{3.041719in}{2.934451in}}%
\pgfpathlineto{\pgfqpoint{3.045977in}{2.934451in}}%
\pgfpathlineto{\pgfqpoint{3.045977in}{2.930193in}}%
\pgfpathmoveto{\pgfqpoint{3.041719in}{2.934451in}}%
\pgfpathlineto{\pgfqpoint{3.041719in}{2.934451in}}%
\pgfpathlineto{\pgfqpoint{3.041719in}{2.938708in}}%
\pgfpathlineto{\pgfqpoint{3.045977in}{2.938708in}}%
\pgfpathlineto{\pgfqpoint{3.045977in}{2.934451in}}%
\pgfpathmoveto{\pgfqpoint{3.041719in}{2.938708in}}%
\pgfpathlineto{\pgfqpoint{3.041719in}{2.938708in}}%
\pgfpathlineto{\pgfqpoint{3.041719in}{2.942966in}}%
\pgfpathlineto{\pgfqpoint{3.045977in}{2.942966in}}%
\pgfpathlineto{\pgfqpoint{3.045977in}{2.938708in}}%
\pgfpathmoveto{\pgfqpoint{3.020430in}{3.070700in}}%
\pgfpathlineto{\pgfqpoint{3.020430in}{3.070700in}}%
\pgfpathlineto{\pgfqpoint{3.020430in}{3.074958in}}%
\pgfpathlineto{\pgfqpoint{3.024688in}{3.074958in}}%
\pgfpathlineto{\pgfqpoint{3.024688in}{3.070700in}}%
\pgfpathmoveto{\pgfqpoint{3.020430in}{3.074958in}}%
\pgfpathlineto{\pgfqpoint{3.020430in}{3.074958in}}%
\pgfpathlineto{\pgfqpoint{3.020430in}{3.079216in}}%
\pgfpathlineto{\pgfqpoint{3.024688in}{3.079216in}}%
\pgfpathlineto{\pgfqpoint{3.024688in}{3.074958in}}%
\pgfpathmoveto{\pgfqpoint{3.020430in}{3.079216in}}%
\pgfpathlineto{\pgfqpoint{3.020430in}{3.079216in}}%
\pgfpathlineto{\pgfqpoint{3.020430in}{3.083474in}}%
\pgfpathlineto{\pgfqpoint{3.024688in}{3.083474in}}%
\pgfpathlineto{\pgfqpoint{3.024688in}{3.079216in}}%
\pgfpathmoveto{\pgfqpoint{3.020430in}{3.083474in}}%
\pgfpathlineto{\pgfqpoint{3.020430in}{3.083474in}}%
\pgfpathlineto{\pgfqpoint{3.020430in}{3.087731in}}%
\pgfpathlineto{\pgfqpoint{3.024688in}{3.087731in}}%
\pgfpathlineto{\pgfqpoint{3.024688in}{3.083474in}}%
\pgfpathmoveto{\pgfqpoint{3.020430in}{3.087731in}}%
\pgfpathlineto{\pgfqpoint{3.020430in}{3.087731in}}%
\pgfpathlineto{\pgfqpoint{3.020430in}{3.091989in}}%
\pgfpathlineto{\pgfqpoint{3.024688in}{3.091989in}}%
\pgfpathlineto{\pgfqpoint{3.024688in}{3.087731in}}%
\pgfpathmoveto{\pgfqpoint{3.020430in}{3.091989in}}%
\pgfpathlineto{\pgfqpoint{3.020430in}{3.091989in}}%
\pgfpathlineto{\pgfqpoint{3.020430in}{3.096247in}}%
\pgfpathlineto{\pgfqpoint{3.024688in}{3.096247in}}%
\pgfpathlineto{\pgfqpoint{3.024688in}{3.091989in}}%
\pgfpathmoveto{\pgfqpoint{3.037461in}{2.959996in}}%
\pgfpathlineto{\pgfqpoint{3.037461in}{2.959996in}}%
\pgfpathlineto{\pgfqpoint{3.037461in}{2.964254in}}%
\pgfpathlineto{\pgfqpoint{3.041719in}{2.964254in}}%
\pgfpathlineto{\pgfqpoint{3.041719in}{2.959996in}}%
\pgfpathmoveto{\pgfqpoint{3.037461in}{2.964254in}}%
\pgfpathlineto{\pgfqpoint{3.037461in}{2.964254in}}%
\pgfpathlineto{\pgfqpoint{3.037461in}{2.968512in}}%
\pgfpathlineto{\pgfqpoint{3.041719in}{2.968512in}}%
\pgfpathlineto{\pgfqpoint{3.041719in}{2.964254in}}%
\pgfpathmoveto{\pgfqpoint{3.033203in}{2.972770in}}%
\pgfpathlineto{\pgfqpoint{3.033203in}{2.972770in}}%
\pgfpathlineto{\pgfqpoint{3.033203in}{2.977028in}}%
\pgfpathlineto{\pgfqpoint{3.037461in}{2.977028in}}%
\pgfpathlineto{\pgfqpoint{3.037461in}{2.972770in}}%
\pgfpathmoveto{\pgfqpoint{3.037461in}{2.968512in}}%
\pgfpathlineto{\pgfqpoint{3.037461in}{2.968512in}}%
\pgfpathlineto{\pgfqpoint{3.037461in}{2.972770in}}%
\pgfpathlineto{\pgfqpoint{3.041719in}{2.972770in}}%
\pgfpathlineto{\pgfqpoint{3.041719in}{2.968512in}}%
\pgfpathmoveto{\pgfqpoint{3.037461in}{2.972770in}}%
\pgfpathlineto{\pgfqpoint{3.037461in}{2.972770in}}%
\pgfpathlineto{\pgfqpoint{3.037461in}{2.977028in}}%
\pgfpathlineto{\pgfqpoint{3.041719in}{2.977028in}}%
\pgfpathlineto{\pgfqpoint{3.041719in}{2.972770in}}%
\pgfpathmoveto{\pgfqpoint{3.033203in}{2.977028in}}%
\pgfpathlineto{\pgfqpoint{3.033203in}{2.977028in}}%
\pgfpathlineto{\pgfqpoint{3.033203in}{2.981286in}}%
\pgfpathlineto{\pgfqpoint{3.037461in}{2.981286in}}%
\pgfpathlineto{\pgfqpoint{3.037461in}{2.977028in}}%
\pgfpathmoveto{\pgfqpoint{3.033203in}{2.981286in}}%
\pgfpathlineto{\pgfqpoint{3.033203in}{2.981286in}}%
\pgfpathlineto{\pgfqpoint{3.033203in}{2.985543in}}%
\pgfpathlineto{\pgfqpoint{3.037461in}{2.985543in}}%
\pgfpathlineto{\pgfqpoint{3.037461in}{2.981286in}}%
\pgfpathmoveto{\pgfqpoint{3.033203in}{2.985543in}}%
\pgfpathlineto{\pgfqpoint{3.033203in}{2.985543in}}%
\pgfpathlineto{\pgfqpoint{3.033203in}{2.989801in}}%
\pgfpathlineto{\pgfqpoint{3.037461in}{2.989801in}}%
\pgfpathlineto{\pgfqpoint{3.037461in}{2.985543in}}%
\pgfpathmoveto{\pgfqpoint{3.033203in}{2.989801in}}%
\pgfpathlineto{\pgfqpoint{3.033203in}{2.989801in}}%
\pgfpathlineto{\pgfqpoint{3.033203in}{2.994059in}}%
\pgfpathlineto{\pgfqpoint{3.037461in}{2.994059in}}%
\pgfpathlineto{\pgfqpoint{3.037461in}{2.989801in}}%
\pgfpathmoveto{\pgfqpoint{3.028945in}{3.002575in}}%
\pgfpathlineto{\pgfqpoint{3.028945in}{3.002575in}}%
\pgfpathlineto{\pgfqpoint{3.028945in}{3.006833in}}%
\pgfpathlineto{\pgfqpoint{3.033203in}{3.006833in}}%
\pgfpathlineto{\pgfqpoint{3.033203in}{3.002575in}}%
\pgfpathmoveto{\pgfqpoint{3.028945in}{3.006833in}}%
\pgfpathlineto{\pgfqpoint{3.028945in}{3.006833in}}%
\pgfpathlineto{\pgfqpoint{3.028945in}{3.011090in}}%
\pgfpathlineto{\pgfqpoint{3.033203in}{3.011090in}}%
\pgfpathlineto{\pgfqpoint{3.033203in}{3.006833in}}%
\pgfpathmoveto{\pgfqpoint{3.033203in}{2.994059in}}%
\pgfpathlineto{\pgfqpoint{3.033203in}{2.994059in}}%
\pgfpathlineto{\pgfqpoint{3.033203in}{2.998317in}}%
\pgfpathlineto{\pgfqpoint{3.037461in}{2.998317in}}%
\pgfpathlineto{\pgfqpoint{3.037461in}{2.994059in}}%
\pgfpathmoveto{\pgfqpoint{3.033203in}{2.998317in}}%
\pgfpathlineto{\pgfqpoint{3.033203in}{2.998317in}}%
\pgfpathlineto{\pgfqpoint{3.033203in}{3.002575in}}%
\pgfpathlineto{\pgfqpoint{3.037461in}{3.002575in}}%
\pgfpathlineto{\pgfqpoint{3.037461in}{2.998317in}}%
\pgfpathmoveto{\pgfqpoint{3.033203in}{3.002575in}}%
\pgfpathlineto{\pgfqpoint{3.033203in}{3.002575in}}%
\pgfpathlineto{\pgfqpoint{3.033203in}{3.006833in}}%
\pgfpathlineto{\pgfqpoint{3.037461in}{3.006833in}}%
\pgfpathlineto{\pgfqpoint{3.037461in}{3.002575in}}%
\pgfpathmoveto{\pgfqpoint{3.028945in}{3.011090in}}%
\pgfpathlineto{\pgfqpoint{3.028945in}{3.011090in}}%
\pgfpathlineto{\pgfqpoint{3.028945in}{3.015348in}}%
\pgfpathlineto{\pgfqpoint{3.033203in}{3.015348in}}%
\pgfpathlineto{\pgfqpoint{3.033203in}{3.011090in}}%
\pgfpathmoveto{\pgfqpoint{3.028945in}{3.015348in}}%
\pgfpathlineto{\pgfqpoint{3.028945in}{3.015348in}}%
\pgfpathlineto{\pgfqpoint{3.028945in}{3.019606in}}%
\pgfpathlineto{\pgfqpoint{3.033203in}{3.019606in}}%
\pgfpathlineto{\pgfqpoint{3.033203in}{3.015348in}}%
\pgfpathmoveto{\pgfqpoint{3.028945in}{3.019606in}}%
\pgfpathlineto{\pgfqpoint{3.028945in}{3.019606in}}%
\pgfpathlineto{\pgfqpoint{3.028945in}{3.023864in}}%
\pgfpathlineto{\pgfqpoint{3.033203in}{3.023864in}}%
\pgfpathlineto{\pgfqpoint{3.033203in}{3.019606in}}%
\pgfpathmoveto{\pgfqpoint{3.028945in}{3.023864in}}%
\pgfpathlineto{\pgfqpoint{3.028945in}{3.023864in}}%
\pgfpathlineto{\pgfqpoint{3.028945in}{3.028122in}}%
\pgfpathlineto{\pgfqpoint{3.033203in}{3.028122in}}%
\pgfpathlineto{\pgfqpoint{3.033203in}{3.023864in}}%
\pgfpathmoveto{\pgfqpoint{3.028945in}{3.028122in}}%
\pgfpathlineto{\pgfqpoint{3.028945in}{3.028122in}}%
\pgfpathlineto{\pgfqpoint{3.028945in}{3.032380in}}%
\pgfpathlineto{\pgfqpoint{3.033203in}{3.032380in}}%
\pgfpathlineto{\pgfqpoint{3.033203in}{3.028122in}}%
\pgfpathmoveto{\pgfqpoint{3.028945in}{3.032380in}}%
\pgfpathlineto{\pgfqpoint{3.028945in}{3.032380in}}%
\pgfpathlineto{\pgfqpoint{3.028945in}{3.036637in}}%
\pgfpathlineto{\pgfqpoint{3.033203in}{3.036637in}}%
\pgfpathlineto{\pgfqpoint{3.033203in}{3.032380in}}%
\pgfpathmoveto{\pgfqpoint{3.024688in}{3.036637in}}%
\pgfpathlineto{\pgfqpoint{3.024688in}{3.036637in}}%
\pgfpathlineto{\pgfqpoint{3.024688in}{3.040895in}}%
\pgfpathlineto{\pgfqpoint{3.028945in}{3.040895in}}%
\pgfpathlineto{\pgfqpoint{3.028945in}{3.036637in}}%
\pgfpathmoveto{\pgfqpoint{3.024688in}{3.040895in}}%
\pgfpathlineto{\pgfqpoint{3.024688in}{3.040895in}}%
\pgfpathlineto{\pgfqpoint{3.024688in}{3.045153in}}%
\pgfpathlineto{\pgfqpoint{3.028945in}{3.045153in}}%
\pgfpathlineto{\pgfqpoint{3.028945in}{3.040895in}}%
\pgfpathmoveto{\pgfqpoint{3.028945in}{3.036637in}}%
\pgfpathlineto{\pgfqpoint{3.028945in}{3.036637in}}%
\pgfpathlineto{\pgfqpoint{3.028945in}{3.040895in}}%
\pgfpathlineto{\pgfqpoint{3.033203in}{3.040895in}}%
\pgfpathlineto{\pgfqpoint{3.033203in}{3.036637in}}%
\pgfpathmoveto{\pgfqpoint{3.024688in}{3.045153in}}%
\pgfpathlineto{\pgfqpoint{3.024688in}{3.045153in}}%
\pgfpathlineto{\pgfqpoint{3.024688in}{3.049411in}}%
\pgfpathlineto{\pgfqpoint{3.028945in}{3.049411in}}%
\pgfpathlineto{\pgfqpoint{3.028945in}{3.045153in}}%
\pgfpathmoveto{\pgfqpoint{3.024688in}{3.049411in}}%
\pgfpathlineto{\pgfqpoint{3.024688in}{3.049411in}}%
\pgfpathlineto{\pgfqpoint{3.024688in}{3.053669in}}%
\pgfpathlineto{\pgfqpoint{3.028945in}{3.053669in}}%
\pgfpathlineto{\pgfqpoint{3.028945in}{3.049411in}}%
\pgfpathmoveto{\pgfqpoint{3.024688in}{3.053669in}}%
\pgfpathlineto{\pgfqpoint{3.024688in}{3.053669in}}%
\pgfpathlineto{\pgfqpoint{3.024688in}{3.057927in}}%
\pgfpathlineto{\pgfqpoint{3.028945in}{3.057927in}}%
\pgfpathlineto{\pgfqpoint{3.028945in}{3.053669in}}%
\pgfpathmoveto{\pgfqpoint{3.024688in}{3.057927in}}%
\pgfpathlineto{\pgfqpoint{3.024688in}{3.057927in}}%
\pgfpathlineto{\pgfqpoint{3.024688in}{3.062184in}}%
\pgfpathlineto{\pgfqpoint{3.028945in}{3.062184in}}%
\pgfpathlineto{\pgfqpoint{3.028945in}{3.057927in}}%
\pgfpathmoveto{\pgfqpoint{3.024688in}{3.062184in}}%
\pgfpathlineto{\pgfqpoint{3.024688in}{3.062184in}}%
\pgfpathlineto{\pgfqpoint{3.024688in}{3.066442in}}%
\pgfpathlineto{\pgfqpoint{3.028945in}{3.066442in}}%
\pgfpathlineto{\pgfqpoint{3.028945in}{3.062184in}}%
\pgfpathmoveto{\pgfqpoint{3.024688in}{3.066442in}}%
\pgfpathlineto{\pgfqpoint{3.024688in}{3.066442in}}%
\pgfpathlineto{\pgfqpoint{3.024688in}{3.070700in}}%
\pgfpathlineto{\pgfqpoint{3.028945in}{3.070700in}}%
\pgfpathlineto{\pgfqpoint{3.028945in}{3.066442in}}%
\pgfpathmoveto{\pgfqpoint{3.024688in}{3.070700in}}%
\pgfpathlineto{\pgfqpoint{3.024688in}{3.070700in}}%
\pgfpathlineto{\pgfqpoint{3.024688in}{3.074958in}}%
\pgfpathlineto{\pgfqpoint{3.028945in}{3.074958in}}%
\pgfpathlineto{\pgfqpoint{3.028945in}{3.070700in}}%
\pgfpathmoveto{\pgfqpoint{3.016172in}{3.100505in}}%
\pgfpathlineto{\pgfqpoint{3.016172in}{3.100505in}}%
\pgfpathlineto{\pgfqpoint{3.016172in}{3.104763in}}%
\pgfpathlineto{\pgfqpoint{3.020430in}{3.104763in}}%
\pgfpathlineto{\pgfqpoint{3.020430in}{3.100505in}}%
\pgfpathmoveto{\pgfqpoint{3.020430in}{3.096247in}}%
\pgfpathlineto{\pgfqpoint{3.020430in}{3.096247in}}%
\pgfpathlineto{\pgfqpoint{3.020430in}{3.100505in}}%
\pgfpathlineto{\pgfqpoint{3.024688in}{3.100505in}}%
\pgfpathlineto{\pgfqpoint{3.024688in}{3.096247in}}%
\pgfpathmoveto{\pgfqpoint{3.020430in}{3.100505in}}%
\pgfpathlineto{\pgfqpoint{3.020430in}{3.100505in}}%
\pgfpathlineto{\pgfqpoint{3.020430in}{3.104763in}}%
\pgfpathlineto{\pgfqpoint{3.024688in}{3.104763in}}%
\pgfpathlineto{\pgfqpoint{3.024688in}{3.100505in}}%
\pgfpathmoveto{\pgfqpoint{3.016172in}{3.104763in}}%
\pgfpathlineto{\pgfqpoint{3.016172in}{3.104763in}}%
\pgfpathlineto{\pgfqpoint{3.016172in}{3.109021in}}%
\pgfpathlineto{\pgfqpoint{3.020430in}{3.109021in}}%
\pgfpathlineto{\pgfqpoint{3.020430in}{3.104763in}}%
\pgfpathmoveto{\pgfqpoint{3.016172in}{3.109021in}}%
\pgfpathlineto{\pgfqpoint{3.016172in}{3.109021in}}%
\pgfpathlineto{\pgfqpoint{3.016172in}{3.113278in}}%
\pgfpathlineto{\pgfqpoint{3.020430in}{3.113278in}}%
\pgfpathlineto{\pgfqpoint{3.020430in}{3.109021in}}%
\pgfpathmoveto{\pgfqpoint{3.016172in}{3.113278in}}%
\pgfpathlineto{\pgfqpoint{3.016172in}{3.113278in}}%
\pgfpathlineto{\pgfqpoint{3.016172in}{3.117536in}}%
\pgfpathlineto{\pgfqpoint{3.020430in}{3.117536in}}%
\pgfpathlineto{\pgfqpoint{3.020430in}{3.113278in}}%
\pgfpathmoveto{\pgfqpoint{3.016172in}{3.117536in}}%
\pgfpathlineto{\pgfqpoint{3.016172in}{3.117536in}}%
\pgfpathlineto{\pgfqpoint{3.016172in}{3.121794in}}%
\pgfpathlineto{\pgfqpoint{3.020430in}{3.121794in}}%
\pgfpathlineto{\pgfqpoint{3.020430in}{3.117536in}}%
\pgfpathmoveto{\pgfqpoint{3.016172in}{3.121794in}}%
\pgfpathlineto{\pgfqpoint{3.016172in}{3.121794in}}%
\pgfpathlineto{\pgfqpoint{3.016172in}{3.126052in}}%
\pgfpathlineto{\pgfqpoint{3.020430in}{3.126052in}}%
\pgfpathlineto{\pgfqpoint{3.020430in}{3.121794in}}%
\pgfpathmoveto{\pgfqpoint{3.016172in}{3.126052in}}%
\pgfpathlineto{\pgfqpoint{3.016172in}{3.126052in}}%
\pgfpathlineto{\pgfqpoint{3.016172in}{3.130310in}}%
\pgfpathlineto{\pgfqpoint{3.020430in}{3.130310in}}%
\pgfpathlineto{\pgfqpoint{3.020430in}{3.126052in}}%
\pgfpathmoveto{\pgfqpoint{3.011914in}{3.134567in}}%
\pgfpathlineto{\pgfqpoint{3.011914in}{3.134567in}}%
\pgfpathlineto{\pgfqpoint{3.011914in}{3.138825in}}%
\pgfpathlineto{\pgfqpoint{3.016172in}{3.138825in}}%
\pgfpathlineto{\pgfqpoint{3.016172in}{3.134567in}}%
\pgfpathmoveto{\pgfqpoint{3.011914in}{3.138825in}}%
\pgfpathlineto{\pgfqpoint{3.011914in}{3.138825in}}%
\pgfpathlineto{\pgfqpoint{3.011914in}{3.143083in}}%
\pgfpathlineto{\pgfqpoint{3.016172in}{3.143083in}}%
\pgfpathlineto{\pgfqpoint{3.016172in}{3.138825in}}%
\pgfpathmoveto{\pgfqpoint{3.011914in}{3.143083in}}%
\pgfpathlineto{\pgfqpoint{3.011914in}{3.143083in}}%
\pgfpathlineto{\pgfqpoint{3.011914in}{3.147341in}}%
\pgfpathlineto{\pgfqpoint{3.016172in}{3.147341in}}%
\pgfpathlineto{\pgfqpoint{3.016172in}{3.143083in}}%
\pgfpathmoveto{\pgfqpoint{3.016172in}{3.130310in}}%
\pgfpathlineto{\pgfqpoint{3.016172in}{3.130310in}}%
\pgfpathlineto{\pgfqpoint{3.016172in}{3.134567in}}%
\pgfpathlineto{\pgfqpoint{3.020430in}{3.134567in}}%
\pgfpathlineto{\pgfqpoint{3.020430in}{3.130310in}}%
\pgfpathmoveto{\pgfqpoint{3.016172in}{3.134567in}}%
\pgfpathlineto{\pgfqpoint{3.016172in}{3.134567in}}%
\pgfpathlineto{\pgfqpoint{3.016172in}{3.138825in}}%
\pgfpathlineto{\pgfqpoint{3.020430in}{3.138825in}}%
\pgfpathlineto{\pgfqpoint{3.020430in}{3.134567in}}%
\pgfpathmoveto{\pgfqpoint{3.011914in}{3.147341in}}%
\pgfpathlineto{\pgfqpoint{3.011914in}{3.147341in}}%
\pgfpathlineto{\pgfqpoint{3.011914in}{3.151599in}}%
\pgfpathlineto{\pgfqpoint{3.016172in}{3.151599in}}%
\pgfpathlineto{\pgfqpoint{3.016172in}{3.147341in}}%
\pgfpathmoveto{\pgfqpoint{3.011914in}{3.151599in}}%
\pgfpathlineto{\pgfqpoint{3.011914in}{3.151599in}}%
\pgfpathlineto{\pgfqpoint{3.011914in}{3.155856in}}%
\pgfpathlineto{\pgfqpoint{3.016172in}{3.155856in}}%
\pgfpathlineto{\pgfqpoint{3.016172in}{3.151599in}}%
\pgfpathmoveto{\pgfqpoint{3.011914in}{3.155856in}}%
\pgfpathlineto{\pgfqpoint{3.011914in}{3.155856in}}%
\pgfpathlineto{\pgfqpoint{3.011914in}{3.160114in}}%
\pgfpathlineto{\pgfqpoint{3.016172in}{3.160114in}}%
\pgfpathlineto{\pgfqpoint{3.016172in}{3.155856in}}%
\pgfpathmoveto{\pgfqpoint{3.011914in}{3.160114in}}%
\pgfpathlineto{\pgfqpoint{3.011914in}{3.160114in}}%
\pgfpathlineto{\pgfqpoint{3.011914in}{3.164372in}}%
\pgfpathlineto{\pgfqpoint{3.016172in}{3.164372in}}%
\pgfpathlineto{\pgfqpoint{3.016172in}{3.160114in}}%
\pgfpathmoveto{\pgfqpoint{3.007656in}{3.168630in}}%
\pgfpathlineto{\pgfqpoint{3.007656in}{3.168630in}}%
\pgfpathlineto{\pgfqpoint{3.007656in}{3.172888in}}%
\pgfpathlineto{\pgfqpoint{3.011914in}{3.172888in}}%
\pgfpathlineto{\pgfqpoint{3.011914in}{3.168630in}}%
\pgfpathmoveto{\pgfqpoint{3.011914in}{3.164372in}}%
\pgfpathlineto{\pgfqpoint{3.011914in}{3.164372in}}%
\pgfpathlineto{\pgfqpoint{3.011914in}{3.168630in}}%
\pgfpathlineto{\pgfqpoint{3.016172in}{3.168630in}}%
\pgfpathlineto{\pgfqpoint{3.016172in}{3.164372in}}%
\pgfpathmoveto{\pgfqpoint{3.011914in}{3.168630in}}%
\pgfpathlineto{\pgfqpoint{3.011914in}{3.168630in}}%
\pgfpathlineto{\pgfqpoint{3.011914in}{3.172888in}}%
\pgfpathlineto{\pgfqpoint{3.016172in}{3.172888in}}%
\pgfpathlineto{\pgfqpoint{3.016172in}{3.168630in}}%
\pgfpathmoveto{\pgfqpoint{3.007656in}{3.172888in}}%
\pgfpathlineto{\pgfqpoint{3.007656in}{3.172888in}}%
\pgfpathlineto{\pgfqpoint{3.007656in}{3.177146in}}%
\pgfpathlineto{\pgfqpoint{3.011914in}{3.177146in}}%
\pgfpathlineto{\pgfqpoint{3.011914in}{3.172888in}}%
\pgfpathmoveto{\pgfqpoint{3.007656in}{3.177146in}}%
\pgfpathlineto{\pgfqpoint{3.007656in}{3.177146in}}%
\pgfpathlineto{\pgfqpoint{3.007656in}{3.181403in}}%
\pgfpathlineto{\pgfqpoint{3.011914in}{3.181403in}}%
\pgfpathlineto{\pgfqpoint{3.011914in}{3.177146in}}%
\pgfpathmoveto{\pgfqpoint{3.007656in}{3.181403in}}%
\pgfpathlineto{\pgfqpoint{3.007656in}{3.181403in}}%
\pgfpathlineto{\pgfqpoint{3.007656in}{3.185661in}}%
\pgfpathlineto{\pgfqpoint{3.011914in}{3.185661in}}%
\pgfpathlineto{\pgfqpoint{3.011914in}{3.181403in}}%
\pgfpathmoveto{\pgfqpoint{3.007656in}{3.185661in}}%
\pgfpathlineto{\pgfqpoint{3.007656in}{3.185661in}}%
\pgfpathlineto{\pgfqpoint{3.007656in}{3.189919in}}%
\pgfpathlineto{\pgfqpoint{3.011914in}{3.189919in}}%
\pgfpathlineto{\pgfqpoint{3.011914in}{3.185661in}}%
\pgfpathmoveto{\pgfqpoint{3.007656in}{3.189919in}}%
\pgfpathlineto{\pgfqpoint{3.007656in}{3.189919in}}%
\pgfpathlineto{\pgfqpoint{3.007656in}{3.194177in}}%
\pgfpathlineto{\pgfqpoint{3.011914in}{3.194177in}}%
\pgfpathlineto{\pgfqpoint{3.011914in}{3.189919in}}%
\pgfpathmoveto{\pgfqpoint{3.007656in}{3.194177in}}%
\pgfpathlineto{\pgfqpoint{3.007656in}{3.194177in}}%
\pgfpathlineto{\pgfqpoint{3.007656in}{3.198435in}}%
\pgfpathlineto{\pgfqpoint{3.011914in}{3.198435in}}%
\pgfpathlineto{\pgfqpoint{3.011914in}{3.194177in}}%
\pgfpathmoveto{\pgfqpoint{3.003398in}{3.202692in}}%
\pgfpathlineto{\pgfqpoint{3.003398in}{3.202692in}}%
\pgfpathlineto{\pgfqpoint{3.003398in}{3.206950in}}%
\pgfpathlineto{\pgfqpoint{3.007656in}{3.206950in}}%
\pgfpathlineto{\pgfqpoint{3.007656in}{3.202692in}}%
\pgfpathmoveto{\pgfqpoint{3.003398in}{3.206950in}}%
\pgfpathlineto{\pgfqpoint{3.003398in}{3.206950in}}%
\pgfpathlineto{\pgfqpoint{3.003398in}{3.211208in}}%
\pgfpathlineto{\pgfqpoint{3.007656in}{3.211208in}}%
\pgfpathlineto{\pgfqpoint{3.007656in}{3.206950in}}%
\pgfpathmoveto{\pgfqpoint{3.003398in}{3.211208in}}%
\pgfpathlineto{\pgfqpoint{3.003398in}{3.211208in}}%
\pgfpathlineto{\pgfqpoint{3.003398in}{3.215466in}}%
\pgfpathlineto{\pgfqpoint{3.007656in}{3.215466in}}%
\pgfpathlineto{\pgfqpoint{3.007656in}{3.211208in}}%
\pgfpathmoveto{\pgfqpoint{3.003398in}{3.215466in}}%
\pgfpathlineto{\pgfqpoint{3.003398in}{3.215466in}}%
\pgfpathlineto{\pgfqpoint{3.003398in}{3.219724in}}%
\pgfpathlineto{\pgfqpoint{3.007656in}{3.219724in}}%
\pgfpathlineto{\pgfqpoint{3.007656in}{3.215466in}}%
\pgfpathmoveto{\pgfqpoint{3.003398in}{3.219724in}}%
\pgfpathlineto{\pgfqpoint{3.003398in}{3.219724in}}%
\pgfpathlineto{\pgfqpoint{3.003398in}{3.223981in}}%
\pgfpathlineto{\pgfqpoint{3.007656in}{3.223981in}}%
\pgfpathlineto{\pgfqpoint{3.007656in}{3.219724in}}%
\pgfpathmoveto{\pgfqpoint{3.003398in}{3.223981in}}%
\pgfpathlineto{\pgfqpoint{3.003398in}{3.223981in}}%
\pgfpathlineto{\pgfqpoint{3.003398in}{3.228239in}}%
\pgfpathlineto{\pgfqpoint{3.007656in}{3.228239in}}%
\pgfpathlineto{\pgfqpoint{3.007656in}{3.223981in}}%
\pgfpathmoveto{\pgfqpoint{3.003398in}{3.228239in}}%
\pgfpathlineto{\pgfqpoint{3.003398in}{3.228239in}}%
\pgfpathlineto{\pgfqpoint{3.003398in}{3.232497in}}%
\pgfpathlineto{\pgfqpoint{3.007656in}{3.232497in}}%
\pgfpathlineto{\pgfqpoint{3.007656in}{3.228239in}}%
\pgfpathmoveto{\pgfqpoint{3.007656in}{3.198435in}}%
\pgfpathlineto{\pgfqpoint{3.007656in}{3.198435in}}%
\pgfpathlineto{\pgfqpoint{3.007656in}{3.202692in}}%
\pgfpathlineto{\pgfqpoint{3.011914in}{3.202692in}}%
\pgfpathlineto{\pgfqpoint{3.011914in}{3.198435in}}%
\pgfpathmoveto{\pgfqpoint{3.007656in}{3.202692in}}%
\pgfpathlineto{\pgfqpoint{3.007656in}{3.202692in}}%
\pgfpathlineto{\pgfqpoint{3.007656in}{3.206950in}}%
\pgfpathlineto{\pgfqpoint{3.011914in}{3.206950in}}%
\pgfpathlineto{\pgfqpoint{3.011914in}{3.202692in}}%
\pgfpathmoveto{\pgfqpoint{2.999141in}{3.236755in}}%
\pgfpathlineto{\pgfqpoint{2.999141in}{3.236755in}}%
\pgfpathlineto{\pgfqpoint{2.999141in}{3.241013in}}%
\pgfpathlineto{\pgfqpoint{3.003398in}{3.241013in}}%
\pgfpathlineto{\pgfqpoint{3.003398in}{3.236755in}}%
\pgfpathmoveto{\pgfqpoint{3.003398in}{3.232497in}}%
\pgfpathlineto{\pgfqpoint{3.003398in}{3.232497in}}%
\pgfpathlineto{\pgfqpoint{3.003398in}{3.236755in}}%
\pgfpathlineto{\pgfqpoint{3.007656in}{3.236755in}}%
\pgfpathlineto{\pgfqpoint{3.007656in}{3.232497in}}%
\pgfpathmoveto{\pgfqpoint{3.003398in}{3.236755in}}%
\pgfpathlineto{\pgfqpoint{3.003398in}{3.236755in}}%
\pgfpathlineto{\pgfqpoint{3.003398in}{3.241013in}}%
\pgfpathlineto{\pgfqpoint{3.007656in}{3.241013in}}%
\pgfpathlineto{\pgfqpoint{3.007656in}{3.236755in}}%
\pgfpathmoveto{\pgfqpoint{2.999141in}{3.241013in}}%
\pgfpathlineto{\pgfqpoint{2.999141in}{3.241013in}}%
\pgfpathlineto{\pgfqpoint{2.999141in}{3.245270in}}%
\pgfpathlineto{\pgfqpoint{3.003398in}{3.245270in}}%
\pgfpathlineto{\pgfqpoint{3.003398in}{3.241013in}}%
\pgfpathmoveto{\pgfqpoint{2.999141in}{3.245270in}}%
\pgfpathlineto{\pgfqpoint{2.999141in}{3.245270in}}%
\pgfpathlineto{\pgfqpoint{2.999141in}{3.249528in}}%
\pgfpathlineto{\pgfqpoint{3.003398in}{3.249528in}}%
\pgfpathlineto{\pgfqpoint{3.003398in}{3.245270in}}%
\pgfpathmoveto{\pgfqpoint{2.999141in}{3.249528in}}%
\pgfpathlineto{\pgfqpoint{2.999141in}{3.249528in}}%
\pgfpathlineto{\pgfqpoint{2.999141in}{3.253786in}}%
\pgfpathlineto{\pgfqpoint{3.003398in}{3.253786in}}%
\pgfpathlineto{\pgfqpoint{3.003398in}{3.249528in}}%
\pgfpathmoveto{\pgfqpoint{2.999141in}{3.253786in}}%
\pgfpathlineto{\pgfqpoint{2.999141in}{3.253786in}}%
\pgfpathlineto{\pgfqpoint{2.999141in}{3.258044in}}%
\pgfpathlineto{\pgfqpoint{3.003398in}{3.258044in}}%
\pgfpathlineto{\pgfqpoint{3.003398in}{3.253786in}}%
\pgfpathmoveto{\pgfqpoint{2.999141in}{3.258044in}}%
\pgfpathlineto{\pgfqpoint{2.999141in}{3.258044in}}%
\pgfpathlineto{\pgfqpoint{2.999141in}{3.262302in}}%
\pgfpathlineto{\pgfqpoint{3.003398in}{3.262302in}}%
\pgfpathlineto{\pgfqpoint{3.003398in}{3.258044in}}%
\pgfpathmoveto{\pgfqpoint{2.999141in}{3.262302in}}%
\pgfpathlineto{\pgfqpoint{2.999141in}{3.262302in}}%
\pgfpathlineto{\pgfqpoint{2.999141in}{3.266559in}}%
\pgfpathlineto{\pgfqpoint{3.003398in}{3.266559in}}%
\pgfpathlineto{\pgfqpoint{3.003398in}{3.262302in}}%
\pgfpathmoveto{\pgfqpoint{2.994883in}{3.270817in}}%
\pgfpathlineto{\pgfqpoint{2.994883in}{3.270817in}}%
\pgfpathlineto{\pgfqpoint{2.994883in}{3.275075in}}%
\pgfpathlineto{\pgfqpoint{2.999141in}{3.275075in}}%
\pgfpathlineto{\pgfqpoint{2.999141in}{3.270817in}}%
\pgfpathmoveto{\pgfqpoint{2.994883in}{3.275075in}}%
\pgfpathlineto{\pgfqpoint{2.994883in}{3.275075in}}%
\pgfpathlineto{\pgfqpoint{2.994883in}{3.279333in}}%
\pgfpathlineto{\pgfqpoint{2.999141in}{3.279333in}}%
\pgfpathlineto{\pgfqpoint{2.999141in}{3.275075in}}%
\pgfpathmoveto{\pgfqpoint{2.994883in}{3.279333in}}%
\pgfpathlineto{\pgfqpoint{2.994883in}{3.279333in}}%
\pgfpathlineto{\pgfqpoint{2.994883in}{3.283591in}}%
\pgfpathlineto{\pgfqpoint{2.999141in}{3.283591in}}%
\pgfpathlineto{\pgfqpoint{2.999141in}{3.279333in}}%
\pgfpathmoveto{\pgfqpoint{2.999141in}{3.266559in}}%
\pgfpathlineto{\pgfqpoint{2.999141in}{3.266559in}}%
\pgfpathlineto{\pgfqpoint{2.999141in}{3.270817in}}%
\pgfpathlineto{\pgfqpoint{3.003398in}{3.270817in}}%
\pgfpathlineto{\pgfqpoint{3.003398in}{3.266559in}}%
\pgfpathmoveto{\pgfqpoint{2.999141in}{3.270817in}}%
\pgfpathlineto{\pgfqpoint{2.999141in}{3.270817in}}%
\pgfpathlineto{\pgfqpoint{2.999141in}{3.275075in}}%
\pgfpathlineto{\pgfqpoint{3.003398in}{3.275075in}}%
\pgfpathlineto{\pgfqpoint{3.003398in}{3.270817in}}%
\pgfpathmoveto{\pgfqpoint{2.994883in}{3.283591in}}%
\pgfpathlineto{\pgfqpoint{2.994883in}{3.283591in}}%
\pgfpathlineto{\pgfqpoint{2.994883in}{3.287848in}}%
\pgfpathlineto{\pgfqpoint{2.999141in}{3.287848in}}%
\pgfpathlineto{\pgfqpoint{2.999141in}{3.283591in}}%
\pgfpathmoveto{\pgfqpoint{2.994883in}{3.287848in}}%
\pgfpathlineto{\pgfqpoint{2.994883in}{3.287848in}}%
\pgfpathlineto{\pgfqpoint{2.994883in}{3.292106in}}%
\pgfpathlineto{\pgfqpoint{2.999141in}{3.292106in}}%
\pgfpathlineto{\pgfqpoint{2.999141in}{3.287848in}}%
\pgfpathmoveto{\pgfqpoint{2.994883in}{3.292106in}}%
\pgfpathlineto{\pgfqpoint{2.994883in}{3.292106in}}%
\pgfpathlineto{\pgfqpoint{2.994883in}{3.296364in}}%
\pgfpathlineto{\pgfqpoint{2.999141in}{3.296364in}}%
\pgfpathlineto{\pgfqpoint{2.999141in}{3.292106in}}%
\pgfpathmoveto{\pgfqpoint{2.994883in}{3.296364in}}%
\pgfpathlineto{\pgfqpoint{2.994883in}{3.296364in}}%
\pgfpathlineto{\pgfqpoint{2.994883in}{3.300622in}}%
\pgfpathlineto{\pgfqpoint{2.999141in}{3.300622in}}%
\pgfpathlineto{\pgfqpoint{2.999141in}{3.296364in}}%
\pgfpathmoveto{\pgfqpoint{2.986367in}{3.343200in}}%
\pgfpathlineto{\pgfqpoint{2.986367in}{3.343200in}}%
\pgfpathlineto{\pgfqpoint{2.986367in}{3.347457in}}%
\pgfpathlineto{\pgfqpoint{2.990625in}{3.347457in}}%
\pgfpathlineto{\pgfqpoint{2.990625in}{3.343200in}}%
\pgfpathmoveto{\pgfqpoint{2.986367in}{3.347457in}}%
\pgfpathlineto{\pgfqpoint{2.986367in}{3.347457in}}%
\pgfpathlineto{\pgfqpoint{2.986367in}{3.351715in}}%
\pgfpathlineto{\pgfqpoint{2.990625in}{3.351715in}}%
\pgfpathlineto{\pgfqpoint{2.990625in}{3.347457in}}%
\pgfpathmoveto{\pgfqpoint{2.986367in}{3.351715in}}%
\pgfpathlineto{\pgfqpoint{2.986367in}{3.351715in}}%
\pgfpathlineto{\pgfqpoint{2.986367in}{3.355973in}}%
\pgfpathlineto{\pgfqpoint{2.990625in}{3.355973in}}%
\pgfpathlineto{\pgfqpoint{2.990625in}{3.351715in}}%
\pgfpathmoveto{\pgfqpoint{2.986367in}{3.355973in}}%
\pgfpathlineto{\pgfqpoint{2.986367in}{3.355973in}}%
\pgfpathlineto{\pgfqpoint{2.986367in}{3.360231in}}%
\pgfpathlineto{\pgfqpoint{2.990625in}{3.360231in}}%
\pgfpathlineto{\pgfqpoint{2.990625in}{3.355973in}}%
\pgfpathmoveto{\pgfqpoint{2.986367in}{3.360231in}}%
\pgfpathlineto{\pgfqpoint{2.986367in}{3.360231in}}%
\pgfpathlineto{\pgfqpoint{2.986367in}{3.364489in}}%
\pgfpathlineto{\pgfqpoint{2.990625in}{3.364489in}}%
\pgfpathlineto{\pgfqpoint{2.990625in}{3.360231in}}%
\pgfpathmoveto{\pgfqpoint{2.986367in}{3.364489in}}%
\pgfpathlineto{\pgfqpoint{2.986367in}{3.364489in}}%
\pgfpathlineto{\pgfqpoint{2.986367in}{3.368746in}}%
\pgfpathlineto{\pgfqpoint{2.990625in}{3.368746in}}%
\pgfpathlineto{\pgfqpoint{2.990625in}{3.364489in}}%
\pgfpathmoveto{\pgfqpoint{2.994883in}{3.300622in}}%
\pgfpathlineto{\pgfqpoint{2.994883in}{3.300622in}}%
\pgfpathlineto{\pgfqpoint{2.994883in}{3.304880in}}%
\pgfpathlineto{\pgfqpoint{2.999141in}{3.304880in}}%
\pgfpathlineto{\pgfqpoint{2.999141in}{3.300622in}}%
\pgfpathmoveto{\pgfqpoint{2.994883in}{3.304880in}}%
\pgfpathlineto{\pgfqpoint{2.994883in}{3.304880in}}%
\pgfpathlineto{\pgfqpoint{2.994883in}{3.309137in}}%
\pgfpathlineto{\pgfqpoint{2.999141in}{3.309137in}}%
\pgfpathlineto{\pgfqpoint{2.999141in}{3.304880in}}%
\pgfpathmoveto{\pgfqpoint{2.990625in}{3.309137in}}%
\pgfpathlineto{\pgfqpoint{2.990625in}{3.309137in}}%
\pgfpathlineto{\pgfqpoint{2.990625in}{3.313395in}}%
\pgfpathlineto{\pgfqpoint{2.994883in}{3.313395in}}%
\pgfpathlineto{\pgfqpoint{2.994883in}{3.309137in}}%
\pgfpathmoveto{\pgfqpoint{2.990625in}{3.313395in}}%
\pgfpathlineto{\pgfqpoint{2.990625in}{3.313395in}}%
\pgfpathlineto{\pgfqpoint{2.990625in}{3.317653in}}%
\pgfpathlineto{\pgfqpoint{2.994883in}{3.317653in}}%
\pgfpathlineto{\pgfqpoint{2.994883in}{3.313395in}}%
\pgfpathmoveto{\pgfqpoint{2.994883in}{3.309137in}}%
\pgfpathlineto{\pgfqpoint{2.994883in}{3.309137in}}%
\pgfpathlineto{\pgfqpoint{2.994883in}{3.313395in}}%
\pgfpathlineto{\pgfqpoint{2.999141in}{3.313395in}}%
\pgfpathlineto{\pgfqpoint{2.999141in}{3.309137in}}%
\pgfpathmoveto{\pgfqpoint{2.990625in}{3.317653in}}%
\pgfpathlineto{\pgfqpoint{2.990625in}{3.317653in}}%
\pgfpathlineto{\pgfqpoint{2.990625in}{3.321911in}}%
\pgfpathlineto{\pgfqpoint{2.994883in}{3.321911in}}%
\pgfpathlineto{\pgfqpoint{2.994883in}{3.317653in}}%
\pgfpathmoveto{\pgfqpoint{2.990625in}{3.321911in}}%
\pgfpathlineto{\pgfqpoint{2.990625in}{3.321911in}}%
\pgfpathlineto{\pgfqpoint{2.990625in}{3.326168in}}%
\pgfpathlineto{\pgfqpoint{2.994883in}{3.326168in}}%
\pgfpathlineto{\pgfqpoint{2.994883in}{3.321911in}}%
\pgfpathmoveto{\pgfqpoint{2.990625in}{3.326168in}}%
\pgfpathlineto{\pgfqpoint{2.990625in}{3.326168in}}%
\pgfpathlineto{\pgfqpoint{2.990625in}{3.330426in}}%
\pgfpathlineto{\pgfqpoint{2.994883in}{3.330426in}}%
\pgfpathlineto{\pgfqpoint{2.994883in}{3.326168in}}%
\pgfpathmoveto{\pgfqpoint{2.990625in}{3.330426in}}%
\pgfpathlineto{\pgfqpoint{2.990625in}{3.330426in}}%
\pgfpathlineto{\pgfqpoint{2.990625in}{3.334684in}}%
\pgfpathlineto{\pgfqpoint{2.994883in}{3.334684in}}%
\pgfpathlineto{\pgfqpoint{2.994883in}{3.330426in}}%
\pgfpathmoveto{\pgfqpoint{2.990625in}{3.334684in}}%
\pgfpathlineto{\pgfqpoint{2.990625in}{3.334684in}}%
\pgfpathlineto{\pgfqpoint{2.990625in}{3.338942in}}%
\pgfpathlineto{\pgfqpoint{2.994883in}{3.338942in}}%
\pgfpathlineto{\pgfqpoint{2.994883in}{3.334684in}}%
\pgfpathmoveto{\pgfqpoint{2.990625in}{3.338942in}}%
\pgfpathlineto{\pgfqpoint{2.990625in}{3.338942in}}%
\pgfpathlineto{\pgfqpoint{2.990625in}{3.343200in}}%
\pgfpathlineto{\pgfqpoint{2.994883in}{3.343200in}}%
\pgfpathlineto{\pgfqpoint{2.994883in}{3.338942in}}%
\pgfpathmoveto{\pgfqpoint{2.990625in}{3.343200in}}%
\pgfpathlineto{\pgfqpoint{2.990625in}{3.343200in}}%
\pgfpathlineto{\pgfqpoint{2.990625in}{3.347457in}}%
\pgfpathlineto{\pgfqpoint{2.994883in}{3.347457in}}%
\pgfpathlineto{\pgfqpoint{2.994883in}{3.343200in}}%
\pgfpathmoveto{\pgfqpoint{2.986367in}{3.368746in}}%
\pgfpathlineto{\pgfqpoint{2.986367in}{3.368746in}}%
\pgfpathlineto{\pgfqpoint{2.986367in}{3.373004in}}%
\pgfpathlineto{\pgfqpoint{2.990625in}{3.373004in}}%
\pgfpathlineto{\pgfqpoint{2.990625in}{3.368746in}}%
\pgfpathmoveto{\pgfqpoint{2.986367in}{3.373004in}}%
\pgfpathlineto{\pgfqpoint{2.986367in}{3.373004in}}%
\pgfpathlineto{\pgfqpoint{2.986367in}{3.377262in}}%
\pgfpathlineto{\pgfqpoint{2.990625in}{3.377262in}}%
\pgfpathlineto{\pgfqpoint{2.990625in}{3.373004in}}%
\pgfpathmoveto{\pgfqpoint{2.982109in}{3.377262in}}%
\pgfpathlineto{\pgfqpoint{2.982109in}{3.377262in}}%
\pgfpathlineto{\pgfqpoint{2.982109in}{3.381520in}}%
\pgfpathlineto{\pgfqpoint{2.986367in}{3.381520in}}%
\pgfpathlineto{\pgfqpoint{2.986367in}{3.377262in}}%
\pgfpathmoveto{\pgfqpoint{2.982109in}{3.381520in}}%
\pgfpathlineto{\pgfqpoint{2.982109in}{3.381520in}}%
\pgfpathlineto{\pgfqpoint{2.982109in}{3.385778in}}%
\pgfpathlineto{\pgfqpoint{2.986367in}{3.385778in}}%
\pgfpathlineto{\pgfqpoint{2.986367in}{3.381520in}}%
\pgfpathmoveto{\pgfqpoint{2.986367in}{3.377262in}}%
\pgfpathlineto{\pgfqpoint{2.986367in}{3.377262in}}%
\pgfpathlineto{\pgfqpoint{2.986367in}{3.381520in}}%
\pgfpathlineto{\pgfqpoint{2.990625in}{3.381520in}}%
\pgfpathlineto{\pgfqpoint{2.990625in}{3.377262in}}%
\pgfpathmoveto{\pgfqpoint{2.982109in}{3.385778in}}%
\pgfpathlineto{\pgfqpoint{2.982109in}{3.385778in}}%
\pgfpathlineto{\pgfqpoint{2.982109in}{3.390036in}}%
\pgfpathlineto{\pgfqpoint{2.986367in}{3.390036in}}%
\pgfpathlineto{\pgfqpoint{2.986367in}{3.385778in}}%
\pgfpathmoveto{\pgfqpoint{2.982109in}{3.390036in}}%
\pgfpathlineto{\pgfqpoint{2.982109in}{3.390036in}}%
\pgfpathlineto{\pgfqpoint{2.982109in}{3.394294in}}%
\pgfpathlineto{\pgfqpoint{2.986367in}{3.394294in}}%
\pgfpathlineto{\pgfqpoint{2.986367in}{3.390036in}}%
\pgfpathmoveto{\pgfqpoint{2.982109in}{3.394294in}}%
\pgfpathlineto{\pgfqpoint{2.982109in}{3.394294in}}%
\pgfpathlineto{\pgfqpoint{2.982109in}{3.398552in}}%
\pgfpathlineto{\pgfqpoint{2.986367in}{3.398552in}}%
\pgfpathlineto{\pgfqpoint{2.986367in}{3.394294in}}%
\pgfpathmoveto{\pgfqpoint{2.982109in}{3.398552in}}%
\pgfpathlineto{\pgfqpoint{2.982109in}{3.398552in}}%
\pgfpathlineto{\pgfqpoint{2.982109in}{3.402810in}}%
\pgfpathlineto{\pgfqpoint{2.986367in}{3.402810in}}%
\pgfpathlineto{\pgfqpoint{2.986367in}{3.398552in}}%
\pgfpathmoveto{\pgfqpoint{2.977851in}{3.415584in}}%
\pgfpathlineto{\pgfqpoint{2.977851in}{3.415584in}}%
\pgfpathlineto{\pgfqpoint{2.977851in}{3.419842in}}%
\pgfpathlineto{\pgfqpoint{2.982109in}{3.419842in}}%
\pgfpathlineto{\pgfqpoint{2.982109in}{3.415584in}}%
\pgfpathmoveto{\pgfqpoint{2.982109in}{3.402810in}}%
\pgfpathlineto{\pgfqpoint{2.982109in}{3.402810in}}%
\pgfpathlineto{\pgfqpoint{2.982109in}{3.407068in}}%
\pgfpathlineto{\pgfqpoint{2.986367in}{3.407068in}}%
\pgfpathlineto{\pgfqpoint{2.986367in}{3.402810in}}%
\pgfpathmoveto{\pgfqpoint{2.982109in}{3.407068in}}%
\pgfpathlineto{\pgfqpoint{2.982109in}{3.407068in}}%
\pgfpathlineto{\pgfqpoint{2.982109in}{3.411326in}}%
\pgfpathlineto{\pgfqpoint{2.986367in}{3.411326in}}%
\pgfpathlineto{\pgfqpoint{2.986367in}{3.407068in}}%
\pgfpathmoveto{\pgfqpoint{2.982109in}{3.411326in}}%
\pgfpathlineto{\pgfqpoint{2.982109in}{3.411326in}}%
\pgfpathlineto{\pgfqpoint{2.982109in}{3.415584in}}%
\pgfpathlineto{\pgfqpoint{2.986367in}{3.415584in}}%
\pgfpathlineto{\pgfqpoint{2.986367in}{3.411326in}}%
\pgfpathmoveto{\pgfqpoint{2.982109in}{3.415584in}}%
\pgfpathlineto{\pgfqpoint{2.982109in}{3.415584in}}%
\pgfpathlineto{\pgfqpoint{2.982109in}{3.419842in}}%
\pgfpathlineto{\pgfqpoint{2.986367in}{3.419842in}}%
\pgfpathlineto{\pgfqpoint{2.986367in}{3.415584in}}%
\pgfpathmoveto{\pgfqpoint{2.977851in}{3.419842in}}%
\pgfpathlineto{\pgfqpoint{2.977851in}{3.419842in}}%
\pgfpathlineto{\pgfqpoint{2.977851in}{3.424100in}}%
\pgfpathlineto{\pgfqpoint{2.982109in}{3.424100in}}%
\pgfpathlineto{\pgfqpoint{2.982109in}{3.419842in}}%
\pgfpathmoveto{\pgfqpoint{2.977851in}{3.424100in}}%
\pgfpathlineto{\pgfqpoint{2.977851in}{3.424100in}}%
\pgfpathlineto{\pgfqpoint{2.977851in}{3.428358in}}%
\pgfpathlineto{\pgfqpoint{2.982109in}{3.428358in}}%
\pgfpathlineto{\pgfqpoint{2.982109in}{3.424100in}}%
\pgfpathmoveto{\pgfqpoint{2.977851in}{3.428358in}}%
\pgfpathlineto{\pgfqpoint{2.977851in}{3.428358in}}%
\pgfpathlineto{\pgfqpoint{2.977851in}{3.432616in}}%
\pgfpathlineto{\pgfqpoint{2.982109in}{3.432616in}}%
\pgfpathlineto{\pgfqpoint{2.982109in}{3.428358in}}%
\pgfpathmoveto{\pgfqpoint{2.977851in}{3.432616in}}%
\pgfpathlineto{\pgfqpoint{2.977851in}{3.432616in}}%
\pgfpathlineto{\pgfqpoint{2.977851in}{3.436874in}}%
\pgfpathlineto{\pgfqpoint{2.982109in}{3.436874in}}%
\pgfpathlineto{\pgfqpoint{2.982109in}{3.432616in}}%
\pgfpathmoveto{\pgfqpoint{2.977851in}{3.436874in}}%
\pgfpathlineto{\pgfqpoint{2.977851in}{3.436874in}}%
\pgfpathlineto{\pgfqpoint{2.977851in}{3.441132in}}%
\pgfpathlineto{\pgfqpoint{2.982109in}{3.441132in}}%
\pgfpathlineto{\pgfqpoint{2.982109in}{3.436874in}}%
\pgfpathmoveto{\pgfqpoint{2.977851in}{3.441132in}}%
\pgfpathlineto{\pgfqpoint{2.977851in}{3.441132in}}%
\pgfpathlineto{\pgfqpoint{2.977851in}{3.445389in}}%
\pgfpathlineto{\pgfqpoint{2.982109in}{3.445389in}}%
\pgfpathlineto{\pgfqpoint{2.982109in}{3.441132in}}%
\pgfpathmoveto{\pgfqpoint{2.973594in}{3.449647in}}%
\pgfpathlineto{\pgfqpoint{2.973594in}{3.449647in}}%
\pgfpathlineto{\pgfqpoint{2.973594in}{3.453905in}}%
\pgfpathlineto{\pgfqpoint{2.977851in}{3.453905in}}%
\pgfpathlineto{\pgfqpoint{2.977851in}{3.449647in}}%
\pgfpathmoveto{\pgfqpoint{2.977851in}{3.445389in}}%
\pgfpathlineto{\pgfqpoint{2.977851in}{3.445389in}}%
\pgfpathlineto{\pgfqpoint{2.977851in}{3.449647in}}%
\pgfpathlineto{\pgfqpoint{2.982109in}{3.449647in}}%
\pgfpathlineto{\pgfqpoint{2.982109in}{3.445389in}}%
\pgfpathmoveto{\pgfqpoint{2.977851in}{3.449647in}}%
\pgfpathlineto{\pgfqpoint{2.977851in}{3.449647in}}%
\pgfpathlineto{\pgfqpoint{2.977851in}{3.453905in}}%
\pgfpathlineto{\pgfqpoint{2.982109in}{3.453905in}}%
\pgfpathlineto{\pgfqpoint{2.982109in}{3.449647in}}%
\pgfpathmoveto{\pgfqpoint{2.973594in}{3.453905in}}%
\pgfpathlineto{\pgfqpoint{2.973594in}{3.453905in}}%
\pgfpathlineto{\pgfqpoint{2.973594in}{3.458163in}}%
\pgfpathlineto{\pgfqpoint{2.977851in}{3.458163in}}%
\pgfpathlineto{\pgfqpoint{2.977851in}{3.453905in}}%
\pgfpathmoveto{\pgfqpoint{2.973594in}{3.458163in}}%
\pgfpathlineto{\pgfqpoint{2.973594in}{3.458163in}}%
\pgfpathlineto{\pgfqpoint{2.973594in}{3.462421in}}%
\pgfpathlineto{\pgfqpoint{2.977851in}{3.462421in}}%
\pgfpathlineto{\pgfqpoint{2.977851in}{3.458163in}}%
\pgfpathmoveto{\pgfqpoint{2.973594in}{3.462421in}}%
\pgfpathlineto{\pgfqpoint{2.973594in}{3.462421in}}%
\pgfpathlineto{\pgfqpoint{2.973594in}{3.466679in}}%
\pgfpathlineto{\pgfqpoint{2.977851in}{3.466679in}}%
\pgfpathlineto{\pgfqpoint{2.977851in}{3.462421in}}%
\pgfpathmoveto{\pgfqpoint{2.973594in}{3.466679in}}%
\pgfpathlineto{\pgfqpoint{2.973594in}{3.466679in}}%
\pgfpathlineto{\pgfqpoint{2.973594in}{3.470937in}}%
\pgfpathlineto{\pgfqpoint{2.977851in}{3.470937in}}%
\pgfpathlineto{\pgfqpoint{2.977851in}{3.466679in}}%
\pgfpathmoveto{\pgfqpoint{2.969336in}{3.487969in}}%
\pgfpathlineto{\pgfqpoint{2.969336in}{3.487969in}}%
\pgfpathlineto{\pgfqpoint{2.969336in}{3.492227in}}%
\pgfpathlineto{\pgfqpoint{2.973594in}{3.492227in}}%
\pgfpathlineto{\pgfqpoint{2.973594in}{3.487969in}}%
\pgfpathmoveto{\pgfqpoint{2.969336in}{3.492227in}}%
\pgfpathlineto{\pgfqpoint{2.969336in}{3.492227in}}%
\pgfpathlineto{\pgfqpoint{2.969336in}{3.496485in}}%
\pgfpathlineto{\pgfqpoint{2.973594in}{3.496485in}}%
\pgfpathlineto{\pgfqpoint{2.973594in}{3.492227in}}%
\pgfpathmoveto{\pgfqpoint{2.969336in}{3.496485in}}%
\pgfpathlineto{\pgfqpoint{2.969336in}{3.496485in}}%
\pgfpathlineto{\pgfqpoint{2.969336in}{3.500743in}}%
\pgfpathlineto{\pgfqpoint{2.973594in}{3.500743in}}%
\pgfpathlineto{\pgfqpoint{2.973594in}{3.496485in}}%
\pgfpathmoveto{\pgfqpoint{2.969336in}{3.500743in}}%
\pgfpathlineto{\pgfqpoint{2.969336in}{3.500743in}}%
\pgfpathlineto{\pgfqpoint{2.969336in}{3.505001in}}%
\pgfpathlineto{\pgfqpoint{2.973594in}{3.505001in}}%
\pgfpathlineto{\pgfqpoint{2.973594in}{3.500743in}}%
\pgfpathmoveto{\pgfqpoint{2.973594in}{3.470937in}}%
\pgfpathlineto{\pgfqpoint{2.973594in}{3.470937in}}%
\pgfpathlineto{\pgfqpoint{2.973594in}{3.475195in}}%
\pgfpathlineto{\pgfqpoint{2.977851in}{3.475195in}}%
\pgfpathlineto{\pgfqpoint{2.977851in}{3.470937in}}%
\pgfpathmoveto{\pgfqpoint{2.973594in}{3.475195in}}%
\pgfpathlineto{\pgfqpoint{2.973594in}{3.475195in}}%
\pgfpathlineto{\pgfqpoint{2.973594in}{3.479453in}}%
\pgfpathlineto{\pgfqpoint{2.977851in}{3.479453in}}%
\pgfpathlineto{\pgfqpoint{2.977851in}{3.475195in}}%
\pgfpathmoveto{\pgfqpoint{2.973594in}{3.479453in}}%
\pgfpathlineto{\pgfqpoint{2.973594in}{3.479453in}}%
\pgfpathlineto{\pgfqpoint{2.973594in}{3.483711in}}%
\pgfpathlineto{\pgfqpoint{2.977851in}{3.483711in}}%
\pgfpathlineto{\pgfqpoint{2.977851in}{3.479453in}}%
\pgfpathmoveto{\pgfqpoint{2.973594in}{3.483711in}}%
\pgfpathlineto{\pgfqpoint{2.973594in}{3.483711in}}%
\pgfpathlineto{\pgfqpoint{2.973594in}{3.487969in}}%
\pgfpathlineto{\pgfqpoint{2.977851in}{3.487969in}}%
\pgfpathlineto{\pgfqpoint{2.977851in}{3.483711in}}%
\pgfpathmoveto{\pgfqpoint{2.973594in}{3.487969in}}%
\pgfpathlineto{\pgfqpoint{2.973594in}{3.487969in}}%
\pgfpathlineto{\pgfqpoint{2.973594in}{3.492227in}}%
\pgfpathlineto{\pgfqpoint{2.977851in}{3.492227in}}%
\pgfpathlineto{\pgfqpoint{2.977851in}{3.487969in}}%
\pgfpathmoveto{\pgfqpoint{2.969336in}{3.505001in}}%
\pgfpathlineto{\pgfqpoint{2.969336in}{3.505001in}}%
\pgfpathlineto{\pgfqpoint{2.969336in}{3.509259in}}%
\pgfpathlineto{\pgfqpoint{2.973594in}{3.509259in}}%
\pgfpathlineto{\pgfqpoint{2.973594in}{3.505001in}}%
\pgfpathmoveto{\pgfqpoint{2.969336in}{3.509259in}}%
\pgfpathlineto{\pgfqpoint{2.969336in}{3.509259in}}%
\pgfpathlineto{\pgfqpoint{2.969336in}{3.513517in}}%
\pgfpathlineto{\pgfqpoint{2.973594in}{3.513517in}}%
\pgfpathlineto{\pgfqpoint{2.973594in}{3.509259in}}%
\pgfpathmoveto{\pgfqpoint{2.969336in}{3.513517in}}%
\pgfpathlineto{\pgfqpoint{2.969336in}{3.513517in}}%
\pgfpathlineto{\pgfqpoint{2.969336in}{3.517774in}}%
\pgfpathlineto{\pgfqpoint{2.973594in}{3.517774in}}%
\pgfpathlineto{\pgfqpoint{2.973594in}{3.513517in}}%
\pgfpathmoveto{\pgfqpoint{2.969336in}{3.517774in}}%
\pgfpathlineto{\pgfqpoint{2.969336in}{3.517774in}}%
\pgfpathlineto{\pgfqpoint{2.969336in}{3.522032in}}%
\pgfpathlineto{\pgfqpoint{2.973594in}{3.522032in}}%
\pgfpathlineto{\pgfqpoint{2.973594in}{3.517774in}}%
\pgfpathmoveto{\pgfqpoint{2.965078in}{3.526290in}}%
\pgfpathlineto{\pgfqpoint{2.965078in}{3.526290in}}%
\pgfpathlineto{\pgfqpoint{2.965078in}{3.530548in}}%
\pgfpathlineto{\pgfqpoint{2.969336in}{3.530548in}}%
\pgfpathlineto{\pgfqpoint{2.969336in}{3.526290in}}%
\pgfpathmoveto{\pgfqpoint{2.969336in}{3.522032in}}%
\pgfpathlineto{\pgfqpoint{2.969336in}{3.522032in}}%
\pgfpathlineto{\pgfqpoint{2.969336in}{3.526290in}}%
\pgfpathlineto{\pgfqpoint{2.973594in}{3.526290in}}%
\pgfpathlineto{\pgfqpoint{2.973594in}{3.522032in}}%
\pgfpathmoveto{\pgfqpoint{2.969336in}{3.526290in}}%
\pgfpathlineto{\pgfqpoint{2.969336in}{3.526290in}}%
\pgfpathlineto{\pgfqpoint{2.969336in}{3.530548in}}%
\pgfpathlineto{\pgfqpoint{2.973594in}{3.530548in}}%
\pgfpathlineto{\pgfqpoint{2.973594in}{3.526290in}}%
\pgfpathmoveto{\pgfqpoint{2.965078in}{3.530548in}}%
\pgfpathlineto{\pgfqpoint{2.965078in}{3.530548in}}%
\pgfpathlineto{\pgfqpoint{2.965078in}{3.534806in}}%
\pgfpathlineto{\pgfqpoint{2.969336in}{3.534806in}}%
\pgfpathlineto{\pgfqpoint{2.969336in}{3.530548in}}%
\pgfpathmoveto{\pgfqpoint{2.965078in}{3.534806in}}%
\pgfpathlineto{\pgfqpoint{2.965078in}{3.534806in}}%
\pgfpathlineto{\pgfqpoint{2.965078in}{3.539064in}}%
\pgfpathlineto{\pgfqpoint{2.969336in}{3.539064in}}%
\pgfpathlineto{\pgfqpoint{2.969336in}{3.534806in}}%
\pgfpathmoveto{\pgfqpoint{2.965078in}{3.539064in}}%
\pgfpathlineto{\pgfqpoint{2.965078in}{3.539064in}}%
\pgfpathlineto{\pgfqpoint{2.965078in}{3.543322in}}%
\pgfpathlineto{\pgfqpoint{2.969336in}{3.543322in}}%
\pgfpathlineto{\pgfqpoint{2.969336in}{3.539064in}}%
\pgfpathmoveto{\pgfqpoint{2.965078in}{3.543322in}}%
\pgfpathlineto{\pgfqpoint{2.965078in}{3.543322in}}%
\pgfpathlineto{\pgfqpoint{2.965078in}{3.547580in}}%
\pgfpathlineto{\pgfqpoint{2.969336in}{3.547580in}}%
\pgfpathlineto{\pgfqpoint{2.969336in}{3.543322in}}%
\pgfpathmoveto{\pgfqpoint{2.965078in}{3.547580in}}%
\pgfpathlineto{\pgfqpoint{2.965078in}{3.547580in}}%
\pgfpathlineto{\pgfqpoint{2.965078in}{3.551838in}}%
\pgfpathlineto{\pgfqpoint{2.969336in}{3.551838in}}%
\pgfpathlineto{\pgfqpoint{2.969336in}{3.547580in}}%
\pgfpathmoveto{\pgfqpoint{2.965078in}{3.551838in}}%
\pgfpathlineto{\pgfqpoint{2.965078in}{3.551838in}}%
\pgfpathlineto{\pgfqpoint{2.965078in}{3.556095in}}%
\pgfpathlineto{\pgfqpoint{2.969336in}{3.556095in}}%
\pgfpathlineto{\pgfqpoint{2.969336in}{3.551838in}}%
\pgfpathmoveto{\pgfqpoint{2.960820in}{3.564611in}}%
\pgfpathlineto{\pgfqpoint{2.960820in}{3.564611in}}%
\pgfpathlineto{\pgfqpoint{2.960820in}{3.568869in}}%
\pgfpathlineto{\pgfqpoint{2.965078in}{3.568869in}}%
\pgfpathlineto{\pgfqpoint{2.965078in}{3.564611in}}%
\pgfpathmoveto{\pgfqpoint{2.960820in}{3.568869in}}%
\pgfpathlineto{\pgfqpoint{2.960820in}{3.568869in}}%
\pgfpathlineto{\pgfqpoint{2.960820in}{3.573127in}}%
\pgfpathlineto{\pgfqpoint{2.965078in}{3.573127in}}%
\pgfpathlineto{\pgfqpoint{2.965078in}{3.568869in}}%
\pgfpathmoveto{\pgfqpoint{2.965078in}{3.556095in}}%
\pgfpathlineto{\pgfqpoint{2.965078in}{3.556095in}}%
\pgfpathlineto{\pgfqpoint{2.965078in}{3.560353in}}%
\pgfpathlineto{\pgfqpoint{2.969336in}{3.560353in}}%
\pgfpathlineto{\pgfqpoint{2.969336in}{3.556095in}}%
\pgfpathmoveto{\pgfqpoint{2.965078in}{3.560353in}}%
\pgfpathlineto{\pgfqpoint{2.965078in}{3.560353in}}%
\pgfpathlineto{\pgfqpoint{2.965078in}{3.564611in}}%
\pgfpathlineto{\pgfqpoint{2.969336in}{3.564611in}}%
\pgfpathlineto{\pgfqpoint{2.969336in}{3.560353in}}%
\pgfpathmoveto{\pgfqpoint{2.965078in}{3.564611in}}%
\pgfpathlineto{\pgfqpoint{2.965078in}{3.564611in}}%
\pgfpathlineto{\pgfqpoint{2.965078in}{3.568869in}}%
\pgfpathlineto{\pgfqpoint{2.969336in}{3.568869in}}%
\pgfpathlineto{\pgfqpoint{2.969336in}{3.564611in}}%
\pgfpathmoveto{\pgfqpoint{2.960820in}{3.573127in}}%
\pgfpathlineto{\pgfqpoint{2.960820in}{3.573127in}}%
\pgfpathlineto{\pgfqpoint{2.960820in}{3.577385in}}%
\pgfpathlineto{\pgfqpoint{2.965078in}{3.577385in}}%
\pgfpathlineto{\pgfqpoint{2.965078in}{3.573127in}}%
\pgfpathmoveto{\pgfqpoint{2.960820in}{3.577385in}}%
\pgfpathlineto{\pgfqpoint{2.960820in}{3.577385in}}%
\pgfpathlineto{\pgfqpoint{2.960820in}{3.581643in}}%
\pgfpathlineto{\pgfqpoint{2.965078in}{3.581643in}}%
\pgfpathlineto{\pgfqpoint{2.965078in}{3.577385in}}%
\pgfpathmoveto{\pgfqpoint{2.960820in}{3.581643in}}%
\pgfpathlineto{\pgfqpoint{2.960820in}{3.581643in}}%
\pgfpathlineto{\pgfqpoint{2.960820in}{3.585901in}}%
\pgfpathlineto{\pgfqpoint{2.965078in}{3.585901in}}%
\pgfpathlineto{\pgfqpoint{2.965078in}{3.581643in}}%
\pgfpathmoveto{\pgfqpoint{2.960820in}{3.585901in}}%
\pgfpathlineto{\pgfqpoint{2.960820in}{3.585901in}}%
\pgfpathlineto{\pgfqpoint{2.960820in}{3.590159in}}%
\pgfpathlineto{\pgfqpoint{2.965078in}{3.590159in}}%
\pgfpathlineto{\pgfqpoint{2.965078in}{3.585901in}}%
\pgfpathmoveto{\pgfqpoint{2.960820in}{3.590159in}}%
\pgfpathlineto{\pgfqpoint{2.960820in}{3.590159in}}%
\pgfpathlineto{\pgfqpoint{2.960820in}{3.594417in}}%
\pgfpathlineto{\pgfqpoint{2.965078in}{3.594417in}}%
\pgfpathlineto{\pgfqpoint{2.965078in}{3.590159in}}%
\pgfpathmoveto{\pgfqpoint{2.960820in}{3.594417in}}%
\pgfpathlineto{\pgfqpoint{2.960820in}{3.594417in}}%
\pgfpathlineto{\pgfqpoint{2.960820in}{3.598674in}}%
\pgfpathlineto{\pgfqpoint{2.965078in}{3.598674in}}%
\pgfpathlineto{\pgfqpoint{2.965078in}{3.594417in}}%
\pgfpathmoveto{\pgfqpoint{2.956562in}{3.602932in}}%
\pgfpathlineto{\pgfqpoint{2.956562in}{3.602932in}}%
\pgfpathlineto{\pgfqpoint{2.956562in}{3.607190in}}%
\pgfpathlineto{\pgfqpoint{2.960820in}{3.607190in}}%
\pgfpathlineto{\pgfqpoint{2.960820in}{3.602932in}}%
\pgfpathmoveto{\pgfqpoint{2.960820in}{3.598674in}}%
\pgfpathlineto{\pgfqpoint{2.960820in}{3.598674in}}%
\pgfpathlineto{\pgfqpoint{2.960820in}{3.602932in}}%
\pgfpathlineto{\pgfqpoint{2.965078in}{3.602932in}}%
\pgfpathlineto{\pgfqpoint{2.965078in}{3.598674in}}%
\pgfpathmoveto{\pgfqpoint{2.960820in}{3.602932in}}%
\pgfpathlineto{\pgfqpoint{2.960820in}{3.602932in}}%
\pgfpathlineto{\pgfqpoint{2.960820in}{3.607190in}}%
\pgfpathlineto{\pgfqpoint{2.965078in}{3.607190in}}%
\pgfpathlineto{\pgfqpoint{2.965078in}{3.602932in}}%
\pgfpathmoveto{\pgfqpoint{2.956562in}{3.607190in}}%
\pgfpathlineto{\pgfqpoint{2.956562in}{3.607190in}}%
\pgfpathlineto{\pgfqpoint{2.956562in}{3.611448in}}%
\pgfpathlineto{\pgfqpoint{2.960820in}{3.611448in}}%
\pgfpathlineto{\pgfqpoint{2.960820in}{3.607190in}}%
\pgfpathmoveto{\pgfqpoint{2.956562in}{3.611448in}}%
\pgfpathlineto{\pgfqpoint{2.956562in}{3.611448in}}%
\pgfpathlineto{\pgfqpoint{2.956562in}{3.615706in}}%
\pgfpathlineto{\pgfqpoint{2.960820in}{3.615706in}}%
\pgfpathlineto{\pgfqpoint{2.960820in}{3.611448in}}%
\pgfpathmoveto{\pgfqpoint{2.956562in}{3.615706in}}%
\pgfpathlineto{\pgfqpoint{2.956562in}{3.615706in}}%
\pgfpathlineto{\pgfqpoint{2.956562in}{3.619964in}}%
\pgfpathlineto{\pgfqpoint{2.960820in}{3.619964in}}%
\pgfpathlineto{\pgfqpoint{2.960820in}{3.615706in}}%
\pgfpathmoveto{\pgfqpoint{2.956562in}{3.619964in}}%
\pgfpathlineto{\pgfqpoint{2.956562in}{3.619964in}}%
\pgfpathlineto{\pgfqpoint{2.956562in}{3.624222in}}%
\pgfpathlineto{\pgfqpoint{2.960820in}{3.624222in}}%
\pgfpathlineto{\pgfqpoint{2.960820in}{3.619964in}}%
\pgfpathmoveto{\pgfqpoint{2.956562in}{3.624222in}}%
\pgfpathlineto{\pgfqpoint{2.956562in}{3.624222in}}%
\pgfpathlineto{\pgfqpoint{2.956562in}{3.628480in}}%
\pgfpathlineto{\pgfqpoint{2.960820in}{3.628480in}}%
\pgfpathlineto{\pgfqpoint{2.960820in}{3.624222in}}%
\pgfpathmoveto{\pgfqpoint{2.956562in}{3.628480in}}%
\pgfpathlineto{\pgfqpoint{2.956562in}{3.628480in}}%
\pgfpathlineto{\pgfqpoint{2.956562in}{3.632738in}}%
\pgfpathlineto{\pgfqpoint{2.960820in}{3.632738in}}%
\pgfpathlineto{\pgfqpoint{2.960820in}{3.628480in}}%
\pgfpathmoveto{\pgfqpoint{2.956562in}{3.632738in}}%
\pgfpathlineto{\pgfqpoint{2.956562in}{3.632738in}}%
\pgfpathlineto{\pgfqpoint{2.956562in}{3.636995in}}%
\pgfpathlineto{\pgfqpoint{2.960820in}{3.636995in}}%
\pgfpathlineto{\pgfqpoint{2.960820in}{3.632738in}}%
\pgfpathmoveto{\pgfqpoint{2.956562in}{3.636995in}}%
\pgfpathlineto{\pgfqpoint{2.956562in}{3.636995in}}%
\pgfpathlineto{\pgfqpoint{2.956562in}{3.641253in}}%
\pgfpathlineto{\pgfqpoint{2.960820in}{3.641253in}}%
\pgfpathlineto{\pgfqpoint{2.960820in}{3.636995in}}%
\pgfpathmoveto{\pgfqpoint{3.224806in}{1.865740in}}%
\pgfpathlineto{\pgfqpoint{3.224806in}{1.865740in}}%
\pgfpathlineto{\pgfqpoint{3.224806in}{1.869998in}}%
\pgfpathlineto{\pgfqpoint{3.229064in}{1.869998in}}%
\pgfpathlineto{\pgfqpoint{3.229064in}{1.865740in}}%
\pgfpathmoveto{\pgfqpoint{3.224806in}{1.869998in}}%
\pgfpathlineto{\pgfqpoint{3.224806in}{1.869998in}}%
\pgfpathlineto{\pgfqpoint{3.224806in}{1.874255in}}%
\pgfpathlineto{\pgfqpoint{3.229064in}{1.874255in}}%
\pgfpathlineto{\pgfqpoint{3.229064in}{1.869998in}}%
\pgfpathmoveto{\pgfqpoint{3.224806in}{1.874255in}}%
\pgfpathlineto{\pgfqpoint{3.224806in}{1.874255in}}%
\pgfpathlineto{\pgfqpoint{3.224806in}{1.878513in}}%
\pgfpathlineto{\pgfqpoint{3.229064in}{1.878513in}}%
\pgfpathlineto{\pgfqpoint{3.229064in}{1.874255in}}%
\pgfpathmoveto{\pgfqpoint{3.224806in}{1.878513in}}%
\pgfpathlineto{\pgfqpoint{3.224806in}{1.878513in}}%
\pgfpathlineto{\pgfqpoint{3.224806in}{1.882771in}}%
\pgfpathlineto{\pgfqpoint{3.229064in}{1.882771in}}%
\pgfpathlineto{\pgfqpoint{3.229064in}{1.878513in}}%
\pgfpathmoveto{\pgfqpoint{3.224806in}{1.882771in}}%
\pgfpathlineto{\pgfqpoint{3.224806in}{1.882771in}}%
\pgfpathlineto{\pgfqpoint{3.224806in}{1.887029in}}%
\pgfpathlineto{\pgfqpoint{3.229064in}{1.887029in}}%
\pgfpathlineto{\pgfqpoint{3.229064in}{1.882771in}}%
\pgfpathmoveto{\pgfqpoint{3.220548in}{1.887029in}}%
\pgfpathlineto{\pgfqpoint{3.220548in}{1.887029in}}%
\pgfpathlineto{\pgfqpoint{3.220548in}{1.891287in}}%
\pgfpathlineto{\pgfqpoint{3.224806in}{1.891287in}}%
\pgfpathlineto{\pgfqpoint{3.224806in}{1.887029in}}%
\pgfpathmoveto{\pgfqpoint{3.220548in}{1.891287in}}%
\pgfpathlineto{\pgfqpoint{3.220548in}{1.891287in}}%
\pgfpathlineto{\pgfqpoint{3.220548in}{1.895545in}}%
\pgfpathlineto{\pgfqpoint{3.224806in}{1.895545in}}%
\pgfpathlineto{\pgfqpoint{3.224806in}{1.891287in}}%
\pgfpathmoveto{\pgfqpoint{3.224806in}{1.887029in}}%
\pgfpathlineto{\pgfqpoint{3.224806in}{1.887029in}}%
\pgfpathlineto{\pgfqpoint{3.224806in}{1.891287in}}%
\pgfpathlineto{\pgfqpoint{3.229064in}{1.891287in}}%
\pgfpathlineto{\pgfqpoint{3.229064in}{1.887029in}}%
\pgfpathmoveto{\pgfqpoint{3.220548in}{1.895545in}}%
\pgfpathlineto{\pgfqpoint{3.220548in}{1.895545in}}%
\pgfpathlineto{\pgfqpoint{3.220548in}{1.899802in}}%
\pgfpathlineto{\pgfqpoint{3.224806in}{1.899802in}}%
\pgfpathlineto{\pgfqpoint{3.224806in}{1.895545in}}%
\pgfpathmoveto{\pgfqpoint{3.220548in}{1.899802in}}%
\pgfpathlineto{\pgfqpoint{3.220548in}{1.899802in}}%
\pgfpathlineto{\pgfqpoint{3.220548in}{1.904060in}}%
\pgfpathlineto{\pgfqpoint{3.224806in}{1.904060in}}%
\pgfpathlineto{\pgfqpoint{3.224806in}{1.899802in}}%
\pgfpathmoveto{\pgfqpoint{3.216290in}{1.904060in}}%
\pgfpathlineto{\pgfqpoint{3.216290in}{1.904060in}}%
\pgfpathlineto{\pgfqpoint{3.216290in}{1.908318in}}%
\pgfpathlineto{\pgfqpoint{3.220548in}{1.908318in}}%
\pgfpathlineto{\pgfqpoint{3.220548in}{1.904060in}}%
\pgfpathmoveto{\pgfqpoint{3.216290in}{1.908318in}}%
\pgfpathlineto{\pgfqpoint{3.216290in}{1.908318in}}%
\pgfpathlineto{\pgfqpoint{3.216290in}{1.912576in}}%
\pgfpathlineto{\pgfqpoint{3.220548in}{1.912576in}}%
\pgfpathlineto{\pgfqpoint{3.220548in}{1.908318in}}%
\pgfpathmoveto{\pgfqpoint{3.216290in}{1.912576in}}%
\pgfpathlineto{\pgfqpoint{3.216290in}{1.912576in}}%
\pgfpathlineto{\pgfqpoint{3.216290in}{1.916834in}}%
\pgfpathlineto{\pgfqpoint{3.220548in}{1.916834in}}%
\pgfpathlineto{\pgfqpoint{3.220548in}{1.912576in}}%
\pgfpathmoveto{\pgfqpoint{3.216290in}{1.916834in}}%
\pgfpathlineto{\pgfqpoint{3.216290in}{1.916834in}}%
\pgfpathlineto{\pgfqpoint{3.216290in}{1.921092in}}%
\pgfpathlineto{\pgfqpoint{3.220548in}{1.921092in}}%
\pgfpathlineto{\pgfqpoint{3.220548in}{1.916834in}}%
\pgfpathmoveto{\pgfqpoint{3.220548in}{1.904060in}}%
\pgfpathlineto{\pgfqpoint{3.220548in}{1.904060in}}%
\pgfpathlineto{\pgfqpoint{3.220548in}{1.908318in}}%
\pgfpathlineto{\pgfqpoint{3.224806in}{1.908318in}}%
\pgfpathlineto{\pgfqpoint{3.224806in}{1.904060in}}%
\pgfpathmoveto{\pgfqpoint{3.212032in}{1.921092in}}%
\pgfpathlineto{\pgfqpoint{3.212032in}{1.921092in}}%
\pgfpathlineto{\pgfqpoint{3.212032in}{1.925349in}}%
\pgfpathlineto{\pgfqpoint{3.216290in}{1.925349in}}%
\pgfpathlineto{\pgfqpoint{3.216290in}{1.921092in}}%
\pgfpathmoveto{\pgfqpoint{3.212032in}{1.925349in}}%
\pgfpathlineto{\pgfqpoint{3.212032in}{1.925349in}}%
\pgfpathlineto{\pgfqpoint{3.212032in}{1.929607in}}%
\pgfpathlineto{\pgfqpoint{3.216290in}{1.929607in}}%
\pgfpathlineto{\pgfqpoint{3.216290in}{1.925349in}}%
\pgfpathmoveto{\pgfqpoint{3.216290in}{1.921092in}}%
\pgfpathlineto{\pgfqpoint{3.216290in}{1.921092in}}%
\pgfpathlineto{\pgfqpoint{3.216290in}{1.925349in}}%
\pgfpathlineto{\pgfqpoint{3.220548in}{1.925349in}}%
\pgfpathlineto{\pgfqpoint{3.220548in}{1.921092in}}%
\pgfpathmoveto{\pgfqpoint{3.212032in}{1.929607in}}%
\pgfpathlineto{\pgfqpoint{3.212032in}{1.929607in}}%
\pgfpathlineto{\pgfqpoint{3.212032in}{1.933865in}}%
\pgfpathlineto{\pgfqpoint{3.216290in}{1.933865in}}%
\pgfpathlineto{\pgfqpoint{3.216290in}{1.929607in}}%
\pgfpathmoveto{\pgfqpoint{3.212032in}{1.933865in}}%
\pgfpathlineto{\pgfqpoint{3.212032in}{1.933865in}}%
\pgfpathlineto{\pgfqpoint{3.212032in}{1.938123in}}%
\pgfpathlineto{\pgfqpoint{3.216290in}{1.938123in}}%
\pgfpathlineto{\pgfqpoint{3.216290in}{1.933865in}}%
\pgfpathmoveto{\pgfqpoint{3.207775in}{1.938123in}}%
\pgfpathlineto{\pgfqpoint{3.207775in}{1.938123in}}%
\pgfpathlineto{\pgfqpoint{3.207775in}{1.942381in}}%
\pgfpathlineto{\pgfqpoint{3.212032in}{1.942381in}}%
\pgfpathlineto{\pgfqpoint{3.212032in}{1.938123in}}%
\pgfpathmoveto{\pgfqpoint{3.207775in}{1.942381in}}%
\pgfpathlineto{\pgfqpoint{3.207775in}{1.942381in}}%
\pgfpathlineto{\pgfqpoint{3.207775in}{1.946638in}}%
\pgfpathlineto{\pgfqpoint{3.212032in}{1.946638in}}%
\pgfpathlineto{\pgfqpoint{3.212032in}{1.942381in}}%
\pgfpathmoveto{\pgfqpoint{3.207775in}{1.946638in}}%
\pgfpathlineto{\pgfqpoint{3.207775in}{1.946638in}}%
\pgfpathlineto{\pgfqpoint{3.207775in}{1.950896in}}%
\pgfpathlineto{\pgfqpoint{3.212032in}{1.950896in}}%
\pgfpathlineto{\pgfqpoint{3.212032in}{1.946638in}}%
\pgfpathmoveto{\pgfqpoint{3.207775in}{1.950896in}}%
\pgfpathlineto{\pgfqpoint{3.207775in}{1.950896in}}%
\pgfpathlineto{\pgfqpoint{3.207775in}{1.955154in}}%
\pgfpathlineto{\pgfqpoint{3.212032in}{1.955154in}}%
\pgfpathlineto{\pgfqpoint{3.212032in}{1.950896in}}%
\pgfpathmoveto{\pgfqpoint{3.203517in}{1.955154in}}%
\pgfpathlineto{\pgfqpoint{3.203517in}{1.955154in}}%
\pgfpathlineto{\pgfqpoint{3.203517in}{1.959412in}}%
\pgfpathlineto{\pgfqpoint{3.207775in}{1.959412in}}%
\pgfpathlineto{\pgfqpoint{3.207775in}{1.955154in}}%
\pgfpathmoveto{\pgfqpoint{3.203517in}{1.959412in}}%
\pgfpathlineto{\pgfqpoint{3.203517in}{1.959412in}}%
\pgfpathlineto{\pgfqpoint{3.203517in}{1.963670in}}%
\pgfpathlineto{\pgfqpoint{3.207775in}{1.963670in}}%
\pgfpathlineto{\pgfqpoint{3.207775in}{1.959412in}}%
\pgfpathmoveto{\pgfqpoint{3.207775in}{1.955154in}}%
\pgfpathlineto{\pgfqpoint{3.207775in}{1.955154in}}%
\pgfpathlineto{\pgfqpoint{3.207775in}{1.959412in}}%
\pgfpathlineto{\pgfqpoint{3.212032in}{1.959412in}}%
\pgfpathlineto{\pgfqpoint{3.212032in}{1.955154in}}%
\pgfpathmoveto{\pgfqpoint{3.203517in}{1.963670in}}%
\pgfpathlineto{\pgfqpoint{3.203517in}{1.963670in}}%
\pgfpathlineto{\pgfqpoint{3.203517in}{1.967928in}}%
\pgfpathlineto{\pgfqpoint{3.207775in}{1.967928in}}%
\pgfpathlineto{\pgfqpoint{3.207775in}{1.963670in}}%
\pgfpathmoveto{\pgfqpoint{3.203517in}{1.967928in}}%
\pgfpathlineto{\pgfqpoint{3.203517in}{1.967928in}}%
\pgfpathlineto{\pgfqpoint{3.203517in}{1.972185in}}%
\pgfpathlineto{\pgfqpoint{3.207775in}{1.972185in}}%
\pgfpathlineto{\pgfqpoint{3.207775in}{1.967928in}}%
\pgfpathmoveto{\pgfqpoint{3.212032in}{1.938123in}}%
\pgfpathlineto{\pgfqpoint{3.212032in}{1.938123in}}%
\pgfpathlineto{\pgfqpoint{3.212032in}{1.942381in}}%
\pgfpathlineto{\pgfqpoint{3.216290in}{1.942381in}}%
\pgfpathlineto{\pgfqpoint{3.216290in}{1.938123in}}%
\pgfpathmoveto{\pgfqpoint{3.199259in}{1.976443in}}%
\pgfpathlineto{\pgfqpoint{3.199259in}{1.976443in}}%
\pgfpathlineto{\pgfqpoint{3.199259in}{1.980701in}}%
\pgfpathlineto{\pgfqpoint{3.203517in}{1.980701in}}%
\pgfpathlineto{\pgfqpoint{3.203517in}{1.976443in}}%
\pgfpathmoveto{\pgfqpoint{3.199259in}{1.980701in}}%
\pgfpathlineto{\pgfqpoint{3.199259in}{1.980701in}}%
\pgfpathlineto{\pgfqpoint{3.199259in}{1.984959in}}%
\pgfpathlineto{\pgfqpoint{3.203517in}{1.984959in}}%
\pgfpathlineto{\pgfqpoint{3.203517in}{1.980701in}}%
\pgfpathmoveto{\pgfqpoint{3.199259in}{1.984959in}}%
\pgfpathlineto{\pgfqpoint{3.199259in}{1.984959in}}%
\pgfpathlineto{\pgfqpoint{3.199259in}{1.989217in}}%
\pgfpathlineto{\pgfqpoint{3.203517in}{1.989217in}}%
\pgfpathlineto{\pgfqpoint{3.203517in}{1.984959in}}%
\pgfpathmoveto{\pgfqpoint{3.203517in}{1.972185in}}%
\pgfpathlineto{\pgfqpoint{3.203517in}{1.972185in}}%
\pgfpathlineto{\pgfqpoint{3.203517in}{1.976443in}}%
\pgfpathlineto{\pgfqpoint{3.207775in}{1.976443in}}%
\pgfpathlineto{\pgfqpoint{3.207775in}{1.972185in}}%
\pgfpathmoveto{\pgfqpoint{3.203517in}{1.976443in}}%
\pgfpathlineto{\pgfqpoint{3.203517in}{1.976443in}}%
\pgfpathlineto{\pgfqpoint{3.203517in}{1.980701in}}%
\pgfpathlineto{\pgfqpoint{3.207775in}{1.980701in}}%
\pgfpathlineto{\pgfqpoint{3.207775in}{1.976443in}}%
\pgfpathmoveto{\pgfqpoint{3.195001in}{1.993475in}}%
\pgfpathlineto{\pgfqpoint{3.195001in}{1.993475in}}%
\pgfpathlineto{\pgfqpoint{3.195001in}{1.997732in}}%
\pgfpathlineto{\pgfqpoint{3.199259in}{1.997732in}}%
\pgfpathlineto{\pgfqpoint{3.199259in}{1.993475in}}%
\pgfpathmoveto{\pgfqpoint{3.199259in}{1.989217in}}%
\pgfpathlineto{\pgfqpoint{3.199259in}{1.989217in}}%
\pgfpathlineto{\pgfqpoint{3.199259in}{1.993475in}}%
\pgfpathlineto{\pgfqpoint{3.203517in}{1.993475in}}%
\pgfpathlineto{\pgfqpoint{3.203517in}{1.989217in}}%
\pgfpathmoveto{\pgfqpoint{3.199259in}{1.993475in}}%
\pgfpathlineto{\pgfqpoint{3.199259in}{1.993475in}}%
\pgfpathlineto{\pgfqpoint{3.199259in}{1.997732in}}%
\pgfpathlineto{\pgfqpoint{3.203517in}{1.997732in}}%
\pgfpathlineto{\pgfqpoint{3.203517in}{1.993475in}}%
\pgfpathmoveto{\pgfqpoint{3.195001in}{1.997732in}}%
\pgfpathlineto{\pgfqpoint{3.195001in}{1.997732in}}%
\pgfpathlineto{\pgfqpoint{3.195001in}{2.001990in}}%
\pgfpathlineto{\pgfqpoint{3.199259in}{2.001990in}}%
\pgfpathlineto{\pgfqpoint{3.199259in}{1.997732in}}%
\pgfpathmoveto{\pgfqpoint{3.195001in}{2.001990in}}%
\pgfpathlineto{\pgfqpoint{3.195001in}{2.001990in}}%
\pgfpathlineto{\pgfqpoint{3.195001in}{2.006248in}}%
\pgfpathlineto{\pgfqpoint{3.199259in}{2.006248in}}%
\pgfpathlineto{\pgfqpoint{3.199259in}{2.001990in}}%
\pgfpathmoveto{\pgfqpoint{3.190743in}{2.014764in}}%
\pgfpathlineto{\pgfqpoint{3.190743in}{2.014764in}}%
\pgfpathlineto{\pgfqpoint{3.190743in}{2.019022in}}%
\pgfpathlineto{\pgfqpoint{3.195001in}{2.019022in}}%
\pgfpathlineto{\pgfqpoint{3.195001in}{2.014764in}}%
\pgfpathmoveto{\pgfqpoint{3.190743in}{2.019022in}}%
\pgfpathlineto{\pgfqpoint{3.190743in}{2.019022in}}%
\pgfpathlineto{\pgfqpoint{3.190743in}{2.023279in}}%
\pgfpathlineto{\pgfqpoint{3.195001in}{2.023279in}}%
\pgfpathlineto{\pgfqpoint{3.195001in}{2.019022in}}%
\pgfpathmoveto{\pgfqpoint{3.190743in}{2.023279in}}%
\pgfpathlineto{\pgfqpoint{3.190743in}{2.023279in}}%
\pgfpathlineto{\pgfqpoint{3.190743in}{2.027537in}}%
\pgfpathlineto{\pgfqpoint{3.195001in}{2.027537in}}%
\pgfpathlineto{\pgfqpoint{3.195001in}{2.023279in}}%
\pgfpathmoveto{\pgfqpoint{3.190743in}{2.027537in}}%
\pgfpathlineto{\pgfqpoint{3.190743in}{2.027537in}}%
\pgfpathlineto{\pgfqpoint{3.190743in}{2.031795in}}%
\pgfpathlineto{\pgfqpoint{3.195001in}{2.031795in}}%
\pgfpathlineto{\pgfqpoint{3.195001in}{2.027537in}}%
\pgfpathmoveto{\pgfqpoint{3.186485in}{2.036053in}}%
\pgfpathlineto{\pgfqpoint{3.186485in}{2.036053in}}%
\pgfpathlineto{\pgfqpoint{3.186485in}{2.040311in}}%
\pgfpathlineto{\pgfqpoint{3.190743in}{2.040311in}}%
\pgfpathlineto{\pgfqpoint{3.190743in}{2.036053in}}%
\pgfpathmoveto{\pgfqpoint{3.190743in}{2.031795in}}%
\pgfpathlineto{\pgfqpoint{3.190743in}{2.031795in}}%
\pgfpathlineto{\pgfqpoint{3.190743in}{2.036053in}}%
\pgfpathlineto{\pgfqpoint{3.195001in}{2.036053in}}%
\pgfpathlineto{\pgfqpoint{3.195001in}{2.031795in}}%
\pgfpathmoveto{\pgfqpoint{3.190743in}{2.036053in}}%
\pgfpathlineto{\pgfqpoint{3.190743in}{2.036053in}}%
\pgfpathlineto{\pgfqpoint{3.190743in}{2.040311in}}%
\pgfpathlineto{\pgfqpoint{3.195001in}{2.040311in}}%
\pgfpathlineto{\pgfqpoint{3.195001in}{2.036053in}}%
\pgfpathmoveto{\pgfqpoint{3.182228in}{2.053084in}}%
\pgfpathlineto{\pgfqpoint{3.182228in}{2.053084in}}%
\pgfpathlineto{\pgfqpoint{3.182228in}{2.057342in}}%
\pgfpathlineto{\pgfqpoint{3.186485in}{2.057342in}}%
\pgfpathlineto{\pgfqpoint{3.186485in}{2.053084in}}%
\pgfpathmoveto{\pgfqpoint{3.186485in}{2.040311in}}%
\pgfpathlineto{\pgfqpoint{3.186485in}{2.040311in}}%
\pgfpathlineto{\pgfqpoint{3.186485in}{2.044569in}}%
\pgfpathlineto{\pgfqpoint{3.190743in}{2.044569in}}%
\pgfpathlineto{\pgfqpoint{3.190743in}{2.040311in}}%
\pgfpathmoveto{\pgfqpoint{3.186485in}{2.044569in}}%
\pgfpathlineto{\pgfqpoint{3.186485in}{2.044569in}}%
\pgfpathlineto{\pgfqpoint{3.186485in}{2.048827in}}%
\pgfpathlineto{\pgfqpoint{3.190743in}{2.048827in}}%
\pgfpathlineto{\pgfqpoint{3.190743in}{2.044569in}}%
\pgfpathmoveto{\pgfqpoint{3.186485in}{2.048827in}}%
\pgfpathlineto{\pgfqpoint{3.186485in}{2.048827in}}%
\pgfpathlineto{\pgfqpoint{3.186485in}{2.053084in}}%
\pgfpathlineto{\pgfqpoint{3.190743in}{2.053084in}}%
\pgfpathlineto{\pgfqpoint{3.190743in}{2.048827in}}%
\pgfpathmoveto{\pgfqpoint{3.186485in}{2.053084in}}%
\pgfpathlineto{\pgfqpoint{3.186485in}{2.053084in}}%
\pgfpathlineto{\pgfqpoint{3.186485in}{2.057342in}}%
\pgfpathlineto{\pgfqpoint{3.190743in}{2.057342in}}%
\pgfpathlineto{\pgfqpoint{3.190743in}{2.053084in}}%
\pgfpathmoveto{\pgfqpoint{3.182228in}{2.057342in}}%
\pgfpathlineto{\pgfqpoint{3.182228in}{2.057342in}}%
\pgfpathlineto{\pgfqpoint{3.182228in}{2.061600in}}%
\pgfpathlineto{\pgfqpoint{3.186485in}{2.061600in}}%
\pgfpathlineto{\pgfqpoint{3.186485in}{2.057342in}}%
\pgfpathmoveto{\pgfqpoint{3.182228in}{2.061600in}}%
\pgfpathlineto{\pgfqpoint{3.182228in}{2.061600in}}%
\pgfpathlineto{\pgfqpoint{3.182228in}{2.065858in}}%
\pgfpathlineto{\pgfqpoint{3.186485in}{2.065858in}}%
\pgfpathlineto{\pgfqpoint{3.186485in}{2.061600in}}%
\pgfpathmoveto{\pgfqpoint{3.182228in}{2.065858in}}%
\pgfpathlineto{\pgfqpoint{3.182228in}{2.065858in}}%
\pgfpathlineto{\pgfqpoint{3.182228in}{2.070116in}}%
\pgfpathlineto{\pgfqpoint{3.186485in}{2.070116in}}%
\pgfpathlineto{\pgfqpoint{3.186485in}{2.065858in}}%
\pgfpathmoveto{\pgfqpoint{3.182228in}{2.070116in}}%
\pgfpathlineto{\pgfqpoint{3.182228in}{2.070116in}}%
\pgfpathlineto{\pgfqpoint{3.182228in}{2.074374in}}%
\pgfpathlineto{\pgfqpoint{3.186485in}{2.074374in}}%
\pgfpathlineto{\pgfqpoint{3.186485in}{2.070116in}}%
\pgfpathmoveto{\pgfqpoint{3.195001in}{2.006248in}}%
\pgfpathlineto{\pgfqpoint{3.195001in}{2.006248in}}%
\pgfpathlineto{\pgfqpoint{3.195001in}{2.010506in}}%
\pgfpathlineto{\pgfqpoint{3.199259in}{2.010506in}}%
\pgfpathlineto{\pgfqpoint{3.199259in}{2.006248in}}%
\pgfpathmoveto{\pgfqpoint{3.195001in}{2.010506in}}%
\pgfpathlineto{\pgfqpoint{3.195001in}{2.010506in}}%
\pgfpathlineto{\pgfqpoint{3.195001in}{2.014764in}}%
\pgfpathlineto{\pgfqpoint{3.199259in}{2.014764in}}%
\pgfpathlineto{\pgfqpoint{3.199259in}{2.010506in}}%
\pgfpathmoveto{\pgfqpoint{3.195001in}{2.014764in}}%
\pgfpathlineto{\pgfqpoint{3.195001in}{2.014764in}}%
\pgfpathlineto{\pgfqpoint{3.195001in}{2.019022in}}%
\pgfpathlineto{\pgfqpoint{3.199259in}{2.019022in}}%
\pgfpathlineto{\pgfqpoint{3.199259in}{2.014764in}}%
\pgfpathmoveto{\pgfqpoint{3.173712in}{2.095663in}}%
\pgfpathlineto{\pgfqpoint{3.173712in}{2.095663in}}%
\pgfpathlineto{\pgfqpoint{3.173712in}{2.099921in}}%
\pgfpathlineto{\pgfqpoint{3.177970in}{2.099921in}}%
\pgfpathlineto{\pgfqpoint{3.177970in}{2.095663in}}%
\pgfpathmoveto{\pgfqpoint{3.173712in}{2.099921in}}%
\pgfpathlineto{\pgfqpoint{3.173712in}{2.099921in}}%
\pgfpathlineto{\pgfqpoint{3.173712in}{2.104179in}}%
\pgfpathlineto{\pgfqpoint{3.177970in}{2.104179in}}%
\pgfpathlineto{\pgfqpoint{3.177970in}{2.099921in}}%
\pgfpathmoveto{\pgfqpoint{3.173712in}{2.104179in}}%
\pgfpathlineto{\pgfqpoint{3.173712in}{2.104179in}}%
\pgfpathlineto{\pgfqpoint{3.173712in}{2.108436in}}%
\pgfpathlineto{\pgfqpoint{3.177970in}{2.108436in}}%
\pgfpathlineto{\pgfqpoint{3.177970in}{2.104179in}}%
\pgfpathmoveto{\pgfqpoint{3.177970in}{2.074374in}}%
\pgfpathlineto{\pgfqpoint{3.177970in}{2.074374in}}%
\pgfpathlineto{\pgfqpoint{3.177970in}{2.078631in}}%
\pgfpathlineto{\pgfqpoint{3.182228in}{2.078631in}}%
\pgfpathlineto{\pgfqpoint{3.182228in}{2.074374in}}%
\pgfpathmoveto{\pgfqpoint{3.177970in}{2.078631in}}%
\pgfpathlineto{\pgfqpoint{3.177970in}{2.078631in}}%
\pgfpathlineto{\pgfqpoint{3.177970in}{2.082889in}}%
\pgfpathlineto{\pgfqpoint{3.182228in}{2.082889in}}%
\pgfpathlineto{\pgfqpoint{3.182228in}{2.078631in}}%
\pgfpathmoveto{\pgfqpoint{3.182228in}{2.074374in}}%
\pgfpathlineto{\pgfqpoint{3.182228in}{2.074374in}}%
\pgfpathlineto{\pgfqpoint{3.182228in}{2.078631in}}%
\pgfpathlineto{\pgfqpoint{3.186485in}{2.078631in}}%
\pgfpathlineto{\pgfqpoint{3.186485in}{2.074374in}}%
\pgfpathmoveto{\pgfqpoint{3.177970in}{2.082889in}}%
\pgfpathlineto{\pgfqpoint{3.177970in}{2.082889in}}%
\pgfpathlineto{\pgfqpoint{3.177970in}{2.087147in}}%
\pgfpathlineto{\pgfqpoint{3.182228in}{2.087147in}}%
\pgfpathlineto{\pgfqpoint{3.182228in}{2.082889in}}%
\pgfpathmoveto{\pgfqpoint{3.177970in}{2.087147in}}%
\pgfpathlineto{\pgfqpoint{3.177970in}{2.087147in}}%
\pgfpathlineto{\pgfqpoint{3.177970in}{2.091405in}}%
\pgfpathlineto{\pgfqpoint{3.182228in}{2.091405in}}%
\pgfpathlineto{\pgfqpoint{3.182228in}{2.087147in}}%
\pgfpathmoveto{\pgfqpoint{3.177970in}{2.091405in}}%
\pgfpathlineto{\pgfqpoint{3.177970in}{2.091405in}}%
\pgfpathlineto{\pgfqpoint{3.177970in}{2.095663in}}%
\pgfpathlineto{\pgfqpoint{3.182228in}{2.095663in}}%
\pgfpathlineto{\pgfqpoint{3.182228in}{2.091405in}}%
\pgfpathmoveto{\pgfqpoint{3.177970in}{2.095663in}}%
\pgfpathlineto{\pgfqpoint{3.177970in}{2.095663in}}%
\pgfpathlineto{\pgfqpoint{3.177970in}{2.099921in}}%
\pgfpathlineto{\pgfqpoint{3.182228in}{2.099921in}}%
\pgfpathlineto{\pgfqpoint{3.182228in}{2.095663in}}%
\pgfpathmoveto{\pgfqpoint{3.173712in}{2.108436in}}%
\pgfpathlineto{\pgfqpoint{3.173712in}{2.108436in}}%
\pgfpathlineto{\pgfqpoint{3.173712in}{2.112694in}}%
\pgfpathlineto{\pgfqpoint{3.177970in}{2.112694in}}%
\pgfpathlineto{\pgfqpoint{3.177970in}{2.108436in}}%
\pgfpathmoveto{\pgfqpoint{3.173712in}{2.112694in}}%
\pgfpathlineto{\pgfqpoint{3.173712in}{2.112694in}}%
\pgfpathlineto{\pgfqpoint{3.173712in}{2.116952in}}%
\pgfpathlineto{\pgfqpoint{3.177970in}{2.116952in}}%
\pgfpathlineto{\pgfqpoint{3.177970in}{2.112694in}}%
\pgfpathmoveto{\pgfqpoint{3.169454in}{2.116952in}}%
\pgfpathlineto{\pgfqpoint{3.169454in}{2.116952in}}%
\pgfpathlineto{\pgfqpoint{3.169454in}{2.121210in}}%
\pgfpathlineto{\pgfqpoint{3.173712in}{2.121210in}}%
\pgfpathlineto{\pgfqpoint{3.173712in}{2.116952in}}%
\pgfpathmoveto{\pgfqpoint{3.169454in}{2.121210in}}%
\pgfpathlineto{\pgfqpoint{3.169454in}{2.121210in}}%
\pgfpathlineto{\pgfqpoint{3.169454in}{2.125468in}}%
\pgfpathlineto{\pgfqpoint{3.173712in}{2.125468in}}%
\pgfpathlineto{\pgfqpoint{3.173712in}{2.121210in}}%
\pgfpathmoveto{\pgfqpoint{3.173712in}{2.116952in}}%
\pgfpathlineto{\pgfqpoint{3.173712in}{2.116952in}}%
\pgfpathlineto{\pgfqpoint{3.173712in}{2.121210in}}%
\pgfpathlineto{\pgfqpoint{3.177970in}{2.121210in}}%
\pgfpathlineto{\pgfqpoint{3.177970in}{2.116952in}}%
\pgfpathmoveto{\pgfqpoint{3.165196in}{2.138241in}}%
\pgfpathlineto{\pgfqpoint{3.165196in}{2.138241in}}%
\pgfpathlineto{\pgfqpoint{3.165196in}{2.142499in}}%
\pgfpathlineto{\pgfqpoint{3.169454in}{2.142499in}}%
\pgfpathlineto{\pgfqpoint{3.169454in}{2.138241in}}%
\pgfpathmoveto{\pgfqpoint{3.169454in}{2.125468in}}%
\pgfpathlineto{\pgfqpoint{3.169454in}{2.125468in}}%
\pgfpathlineto{\pgfqpoint{3.169454in}{2.129726in}}%
\pgfpathlineto{\pgfqpoint{3.173712in}{2.129726in}}%
\pgfpathlineto{\pgfqpoint{3.173712in}{2.125468in}}%
\pgfpathmoveto{\pgfqpoint{3.169454in}{2.129726in}}%
\pgfpathlineto{\pgfqpoint{3.169454in}{2.129726in}}%
\pgfpathlineto{\pgfqpoint{3.169454in}{2.133983in}}%
\pgfpathlineto{\pgfqpoint{3.173712in}{2.133983in}}%
\pgfpathlineto{\pgfqpoint{3.173712in}{2.129726in}}%
\pgfpathmoveto{\pgfqpoint{3.169454in}{2.133983in}}%
\pgfpathlineto{\pgfqpoint{3.169454in}{2.133983in}}%
\pgfpathlineto{\pgfqpoint{3.169454in}{2.138241in}}%
\pgfpathlineto{\pgfqpoint{3.173712in}{2.138241in}}%
\pgfpathlineto{\pgfqpoint{3.173712in}{2.133983in}}%
\pgfpathmoveto{\pgfqpoint{3.169454in}{2.138241in}}%
\pgfpathlineto{\pgfqpoint{3.169454in}{2.138241in}}%
\pgfpathlineto{\pgfqpoint{3.169454in}{2.142499in}}%
\pgfpathlineto{\pgfqpoint{3.173712in}{2.142499in}}%
\pgfpathlineto{\pgfqpoint{3.173712in}{2.138241in}}%
\pgfpathmoveto{\pgfqpoint{3.156681in}{2.180819in}}%
\pgfpathlineto{\pgfqpoint{3.156681in}{2.180819in}}%
\pgfpathlineto{\pgfqpoint{3.156681in}{2.185077in}}%
\pgfpathlineto{\pgfqpoint{3.160938in}{2.185077in}}%
\pgfpathlineto{\pgfqpoint{3.160938in}{2.180819in}}%
\pgfpathmoveto{\pgfqpoint{3.156681in}{2.185077in}}%
\pgfpathlineto{\pgfqpoint{3.156681in}{2.185077in}}%
\pgfpathlineto{\pgfqpoint{3.156681in}{2.189334in}}%
\pgfpathlineto{\pgfqpoint{3.160938in}{2.189334in}}%
\pgfpathlineto{\pgfqpoint{3.160938in}{2.185077in}}%
\pgfpathmoveto{\pgfqpoint{3.156681in}{2.189334in}}%
\pgfpathlineto{\pgfqpoint{3.156681in}{2.189334in}}%
\pgfpathlineto{\pgfqpoint{3.156681in}{2.193592in}}%
\pgfpathlineto{\pgfqpoint{3.160938in}{2.193592in}}%
\pgfpathlineto{\pgfqpoint{3.160938in}{2.189334in}}%
\pgfpathmoveto{\pgfqpoint{3.156681in}{2.193592in}}%
\pgfpathlineto{\pgfqpoint{3.156681in}{2.193592in}}%
\pgfpathlineto{\pgfqpoint{3.156681in}{2.197850in}}%
\pgfpathlineto{\pgfqpoint{3.160938in}{2.197850in}}%
\pgfpathlineto{\pgfqpoint{3.160938in}{2.193592in}}%
\pgfpathmoveto{\pgfqpoint{3.156681in}{2.197850in}}%
\pgfpathlineto{\pgfqpoint{3.156681in}{2.197850in}}%
\pgfpathlineto{\pgfqpoint{3.156681in}{2.202108in}}%
\pgfpathlineto{\pgfqpoint{3.160938in}{2.202108in}}%
\pgfpathlineto{\pgfqpoint{3.160938in}{2.197850in}}%
\pgfpathmoveto{\pgfqpoint{3.152423in}{2.206365in}}%
\pgfpathlineto{\pgfqpoint{3.152423in}{2.206365in}}%
\pgfpathlineto{\pgfqpoint{3.152423in}{2.210623in}}%
\pgfpathlineto{\pgfqpoint{3.156681in}{2.210623in}}%
\pgfpathlineto{\pgfqpoint{3.156681in}{2.206365in}}%
\pgfpathmoveto{\pgfqpoint{3.156681in}{2.202108in}}%
\pgfpathlineto{\pgfqpoint{3.156681in}{2.202108in}}%
\pgfpathlineto{\pgfqpoint{3.156681in}{2.206365in}}%
\pgfpathlineto{\pgfqpoint{3.160938in}{2.206365in}}%
\pgfpathlineto{\pgfqpoint{3.160938in}{2.202108in}}%
\pgfpathmoveto{\pgfqpoint{3.156681in}{2.206365in}}%
\pgfpathlineto{\pgfqpoint{3.156681in}{2.206365in}}%
\pgfpathlineto{\pgfqpoint{3.156681in}{2.210623in}}%
\pgfpathlineto{\pgfqpoint{3.160938in}{2.210623in}}%
\pgfpathlineto{\pgfqpoint{3.160938in}{2.206365in}}%
\pgfpathmoveto{\pgfqpoint{3.152423in}{2.210623in}}%
\pgfpathlineto{\pgfqpoint{3.152423in}{2.210623in}}%
\pgfpathlineto{\pgfqpoint{3.152423in}{2.214881in}}%
\pgfpathlineto{\pgfqpoint{3.156681in}{2.214881in}}%
\pgfpathlineto{\pgfqpoint{3.156681in}{2.210623in}}%
\pgfpathmoveto{\pgfqpoint{3.152423in}{2.214881in}}%
\pgfpathlineto{\pgfqpoint{3.152423in}{2.214881in}}%
\pgfpathlineto{\pgfqpoint{3.152423in}{2.219139in}}%
\pgfpathlineto{\pgfqpoint{3.156681in}{2.219139in}}%
\pgfpathlineto{\pgfqpoint{3.156681in}{2.214881in}}%
\pgfpathmoveto{\pgfqpoint{3.152423in}{2.219139in}}%
\pgfpathlineto{\pgfqpoint{3.152423in}{2.219139in}}%
\pgfpathlineto{\pgfqpoint{3.152423in}{2.223396in}}%
\pgfpathlineto{\pgfqpoint{3.156681in}{2.223396in}}%
\pgfpathlineto{\pgfqpoint{3.156681in}{2.219139in}}%
\pgfpathmoveto{\pgfqpoint{3.152423in}{2.223396in}}%
\pgfpathlineto{\pgfqpoint{3.152423in}{2.223396in}}%
\pgfpathlineto{\pgfqpoint{3.152423in}{2.227654in}}%
\pgfpathlineto{\pgfqpoint{3.156681in}{2.227654in}}%
\pgfpathlineto{\pgfqpoint{3.156681in}{2.223396in}}%
\pgfpathmoveto{\pgfqpoint{3.148165in}{2.227654in}}%
\pgfpathlineto{\pgfqpoint{3.148165in}{2.227654in}}%
\pgfpathlineto{\pgfqpoint{3.148165in}{2.231912in}}%
\pgfpathlineto{\pgfqpoint{3.152423in}{2.231912in}}%
\pgfpathlineto{\pgfqpoint{3.152423in}{2.227654in}}%
\pgfpathmoveto{\pgfqpoint{3.148165in}{2.231912in}}%
\pgfpathlineto{\pgfqpoint{3.148165in}{2.231912in}}%
\pgfpathlineto{\pgfqpoint{3.148165in}{2.236170in}}%
\pgfpathlineto{\pgfqpoint{3.152423in}{2.236170in}}%
\pgfpathlineto{\pgfqpoint{3.152423in}{2.231912in}}%
\pgfpathmoveto{\pgfqpoint{3.148165in}{2.236170in}}%
\pgfpathlineto{\pgfqpoint{3.148165in}{2.236170in}}%
\pgfpathlineto{\pgfqpoint{3.148165in}{2.240427in}}%
\pgfpathlineto{\pgfqpoint{3.152423in}{2.240427in}}%
\pgfpathlineto{\pgfqpoint{3.152423in}{2.236170in}}%
\pgfpathmoveto{\pgfqpoint{3.148165in}{2.240427in}}%
\pgfpathlineto{\pgfqpoint{3.148165in}{2.240427in}}%
\pgfpathlineto{\pgfqpoint{3.148165in}{2.244685in}}%
\pgfpathlineto{\pgfqpoint{3.152423in}{2.244685in}}%
\pgfpathlineto{\pgfqpoint{3.152423in}{2.240427in}}%
\pgfpathmoveto{\pgfqpoint{3.152423in}{2.227654in}}%
\pgfpathlineto{\pgfqpoint{3.152423in}{2.227654in}}%
\pgfpathlineto{\pgfqpoint{3.152423in}{2.231912in}}%
\pgfpathlineto{\pgfqpoint{3.156681in}{2.231912in}}%
\pgfpathlineto{\pgfqpoint{3.156681in}{2.227654in}}%
\pgfpathmoveto{\pgfqpoint{3.139649in}{2.274489in}}%
\pgfpathlineto{\pgfqpoint{3.139649in}{2.274489in}}%
\pgfpathlineto{\pgfqpoint{3.139649in}{2.278747in}}%
\pgfpathlineto{\pgfqpoint{3.143907in}{2.278747in}}%
\pgfpathlineto{\pgfqpoint{3.143907in}{2.274489in}}%
\pgfpathmoveto{\pgfqpoint{3.148165in}{2.244685in}}%
\pgfpathlineto{\pgfqpoint{3.148165in}{2.244685in}}%
\pgfpathlineto{\pgfqpoint{3.148165in}{2.248943in}}%
\pgfpathlineto{\pgfqpoint{3.152423in}{2.248943in}}%
\pgfpathlineto{\pgfqpoint{3.152423in}{2.244685in}}%
\pgfpathmoveto{\pgfqpoint{3.148165in}{2.248943in}}%
\pgfpathlineto{\pgfqpoint{3.148165in}{2.248943in}}%
\pgfpathlineto{\pgfqpoint{3.148165in}{2.253201in}}%
\pgfpathlineto{\pgfqpoint{3.152423in}{2.253201in}}%
\pgfpathlineto{\pgfqpoint{3.152423in}{2.248943in}}%
\pgfpathmoveto{\pgfqpoint{3.143907in}{2.253201in}}%
\pgfpathlineto{\pgfqpoint{3.143907in}{2.253201in}}%
\pgfpathlineto{\pgfqpoint{3.143907in}{2.257458in}}%
\pgfpathlineto{\pgfqpoint{3.148165in}{2.257458in}}%
\pgfpathlineto{\pgfqpoint{3.148165in}{2.253201in}}%
\pgfpathmoveto{\pgfqpoint{3.143907in}{2.257458in}}%
\pgfpathlineto{\pgfqpoint{3.143907in}{2.257458in}}%
\pgfpathlineto{\pgfqpoint{3.143907in}{2.261716in}}%
\pgfpathlineto{\pgfqpoint{3.148165in}{2.261716in}}%
\pgfpathlineto{\pgfqpoint{3.148165in}{2.257458in}}%
\pgfpathmoveto{\pgfqpoint{3.148165in}{2.253201in}}%
\pgfpathlineto{\pgfqpoint{3.148165in}{2.253201in}}%
\pgfpathlineto{\pgfqpoint{3.148165in}{2.257458in}}%
\pgfpathlineto{\pgfqpoint{3.152423in}{2.257458in}}%
\pgfpathlineto{\pgfqpoint{3.152423in}{2.253201in}}%
\pgfpathmoveto{\pgfqpoint{3.143907in}{2.261716in}}%
\pgfpathlineto{\pgfqpoint{3.143907in}{2.261716in}}%
\pgfpathlineto{\pgfqpoint{3.143907in}{2.265974in}}%
\pgfpathlineto{\pgfqpoint{3.148165in}{2.265974in}}%
\pgfpathlineto{\pgfqpoint{3.148165in}{2.261716in}}%
\pgfpathmoveto{\pgfqpoint{3.143907in}{2.265974in}}%
\pgfpathlineto{\pgfqpoint{3.143907in}{2.265974in}}%
\pgfpathlineto{\pgfqpoint{3.143907in}{2.270232in}}%
\pgfpathlineto{\pgfqpoint{3.148165in}{2.270232in}}%
\pgfpathlineto{\pgfqpoint{3.148165in}{2.265974in}}%
\pgfpathmoveto{\pgfqpoint{3.143907in}{2.270232in}}%
\pgfpathlineto{\pgfqpoint{3.143907in}{2.270232in}}%
\pgfpathlineto{\pgfqpoint{3.143907in}{2.274489in}}%
\pgfpathlineto{\pgfqpoint{3.148165in}{2.274489in}}%
\pgfpathlineto{\pgfqpoint{3.148165in}{2.270232in}}%
\pgfpathmoveto{\pgfqpoint{3.143907in}{2.274489in}}%
\pgfpathlineto{\pgfqpoint{3.143907in}{2.274489in}}%
\pgfpathlineto{\pgfqpoint{3.143907in}{2.278747in}}%
\pgfpathlineto{\pgfqpoint{3.148165in}{2.278747in}}%
\pgfpathlineto{\pgfqpoint{3.148165in}{2.274489in}}%
\pgfpathmoveto{\pgfqpoint{3.165196in}{2.142499in}}%
\pgfpathlineto{\pgfqpoint{3.165196in}{2.142499in}}%
\pgfpathlineto{\pgfqpoint{3.165196in}{2.146757in}}%
\pgfpathlineto{\pgfqpoint{3.169454in}{2.146757in}}%
\pgfpathlineto{\pgfqpoint{3.169454in}{2.142499in}}%
\pgfpathmoveto{\pgfqpoint{3.165196in}{2.146757in}}%
\pgfpathlineto{\pgfqpoint{3.165196in}{2.146757in}}%
\pgfpathlineto{\pgfqpoint{3.165196in}{2.151015in}}%
\pgfpathlineto{\pgfqpoint{3.169454in}{2.151015in}}%
\pgfpathlineto{\pgfqpoint{3.169454in}{2.146757in}}%
\pgfpathmoveto{\pgfqpoint{3.165196in}{2.151015in}}%
\pgfpathlineto{\pgfqpoint{3.165196in}{2.151015in}}%
\pgfpathlineto{\pgfqpoint{3.165196in}{2.155272in}}%
\pgfpathlineto{\pgfqpoint{3.169454in}{2.155272in}}%
\pgfpathlineto{\pgfqpoint{3.169454in}{2.151015in}}%
\pgfpathmoveto{\pgfqpoint{3.165196in}{2.155272in}}%
\pgfpathlineto{\pgfqpoint{3.165196in}{2.155272in}}%
\pgfpathlineto{\pgfqpoint{3.165196in}{2.159530in}}%
\pgfpathlineto{\pgfqpoint{3.169454in}{2.159530in}}%
\pgfpathlineto{\pgfqpoint{3.169454in}{2.155272in}}%
\pgfpathmoveto{\pgfqpoint{3.160938in}{2.159530in}}%
\pgfpathlineto{\pgfqpoint{3.160938in}{2.159530in}}%
\pgfpathlineto{\pgfqpoint{3.160938in}{2.163788in}}%
\pgfpathlineto{\pgfqpoint{3.165196in}{2.163788in}}%
\pgfpathlineto{\pgfqpoint{3.165196in}{2.159530in}}%
\pgfpathmoveto{\pgfqpoint{3.160938in}{2.163788in}}%
\pgfpathlineto{\pgfqpoint{3.160938in}{2.163788in}}%
\pgfpathlineto{\pgfqpoint{3.160938in}{2.168046in}}%
\pgfpathlineto{\pgfqpoint{3.165196in}{2.168046in}}%
\pgfpathlineto{\pgfqpoint{3.165196in}{2.163788in}}%
\pgfpathmoveto{\pgfqpoint{3.165196in}{2.159530in}}%
\pgfpathlineto{\pgfqpoint{3.165196in}{2.159530in}}%
\pgfpathlineto{\pgfqpoint{3.165196in}{2.163788in}}%
\pgfpathlineto{\pgfqpoint{3.169454in}{2.163788in}}%
\pgfpathlineto{\pgfqpoint{3.169454in}{2.159530in}}%
\pgfpathmoveto{\pgfqpoint{3.160938in}{2.168046in}}%
\pgfpathlineto{\pgfqpoint{3.160938in}{2.168046in}}%
\pgfpathlineto{\pgfqpoint{3.160938in}{2.172303in}}%
\pgfpathlineto{\pgfqpoint{3.165196in}{2.172303in}}%
\pgfpathlineto{\pgfqpoint{3.165196in}{2.168046in}}%
\pgfpathmoveto{\pgfqpoint{3.160938in}{2.172303in}}%
\pgfpathlineto{\pgfqpoint{3.160938in}{2.172303in}}%
\pgfpathlineto{\pgfqpoint{3.160938in}{2.176561in}}%
\pgfpathlineto{\pgfqpoint{3.165196in}{2.176561in}}%
\pgfpathlineto{\pgfqpoint{3.165196in}{2.172303in}}%
\pgfpathmoveto{\pgfqpoint{3.160938in}{2.176561in}}%
\pgfpathlineto{\pgfqpoint{3.160938in}{2.176561in}}%
\pgfpathlineto{\pgfqpoint{3.160938in}{2.180819in}}%
\pgfpathlineto{\pgfqpoint{3.165196in}{2.180819in}}%
\pgfpathlineto{\pgfqpoint{3.165196in}{2.176561in}}%
\pgfpathmoveto{\pgfqpoint{3.160938in}{2.180819in}}%
\pgfpathlineto{\pgfqpoint{3.160938in}{2.180819in}}%
\pgfpathlineto{\pgfqpoint{3.160938in}{2.185077in}}%
\pgfpathlineto{\pgfqpoint{3.165196in}{2.185077in}}%
\pgfpathlineto{\pgfqpoint{3.165196in}{2.180819in}}%
\pgfpathmoveto{\pgfqpoint{3.139649in}{2.278747in}}%
\pgfpathlineto{\pgfqpoint{3.139649in}{2.278747in}}%
\pgfpathlineto{\pgfqpoint{3.139649in}{2.283005in}}%
\pgfpathlineto{\pgfqpoint{3.143907in}{2.283005in}}%
\pgfpathlineto{\pgfqpoint{3.143907in}{2.278747in}}%
\pgfpathmoveto{\pgfqpoint{3.139649in}{2.283005in}}%
\pgfpathlineto{\pgfqpoint{3.139649in}{2.283005in}}%
\pgfpathlineto{\pgfqpoint{3.139649in}{2.287263in}}%
\pgfpathlineto{\pgfqpoint{3.143907in}{2.287263in}}%
\pgfpathlineto{\pgfqpoint{3.143907in}{2.283005in}}%
\pgfpathmoveto{\pgfqpoint{3.139649in}{2.287263in}}%
\pgfpathlineto{\pgfqpoint{3.139649in}{2.287263in}}%
\pgfpathlineto{\pgfqpoint{3.139649in}{2.291521in}}%
\pgfpathlineto{\pgfqpoint{3.143907in}{2.291521in}}%
\pgfpathlineto{\pgfqpoint{3.143907in}{2.287263in}}%
\pgfpathmoveto{\pgfqpoint{3.139649in}{2.291521in}}%
\pgfpathlineto{\pgfqpoint{3.139649in}{2.291521in}}%
\pgfpathlineto{\pgfqpoint{3.139649in}{2.295779in}}%
\pgfpathlineto{\pgfqpoint{3.143907in}{2.295779in}}%
\pgfpathlineto{\pgfqpoint{3.143907in}{2.291521in}}%
\pgfpathmoveto{\pgfqpoint{3.135391in}{2.300037in}}%
\pgfpathlineto{\pgfqpoint{3.135391in}{2.300037in}}%
\pgfpathlineto{\pgfqpoint{3.135391in}{2.304295in}}%
\pgfpathlineto{\pgfqpoint{3.139649in}{2.304295in}}%
\pgfpathlineto{\pgfqpoint{3.139649in}{2.300037in}}%
\pgfpathmoveto{\pgfqpoint{3.139649in}{2.295779in}}%
\pgfpathlineto{\pgfqpoint{3.139649in}{2.295779in}}%
\pgfpathlineto{\pgfqpoint{3.139649in}{2.300037in}}%
\pgfpathlineto{\pgfqpoint{3.143907in}{2.300037in}}%
\pgfpathlineto{\pgfqpoint{3.143907in}{2.295779in}}%
\pgfpathmoveto{\pgfqpoint{3.139649in}{2.300037in}}%
\pgfpathlineto{\pgfqpoint{3.139649in}{2.300037in}}%
\pgfpathlineto{\pgfqpoint{3.139649in}{2.304295in}}%
\pgfpathlineto{\pgfqpoint{3.143907in}{2.304295in}}%
\pgfpathlineto{\pgfqpoint{3.143907in}{2.300037in}}%
\pgfpathmoveto{\pgfqpoint{3.135391in}{2.304295in}}%
\pgfpathlineto{\pgfqpoint{3.135391in}{2.304295in}}%
\pgfpathlineto{\pgfqpoint{3.135391in}{2.308552in}}%
\pgfpathlineto{\pgfqpoint{3.139649in}{2.308552in}}%
\pgfpathlineto{\pgfqpoint{3.139649in}{2.304295in}}%
\pgfpathmoveto{\pgfqpoint{3.135391in}{2.308552in}}%
\pgfpathlineto{\pgfqpoint{3.135391in}{2.308552in}}%
\pgfpathlineto{\pgfqpoint{3.135391in}{2.312810in}}%
\pgfpathlineto{\pgfqpoint{3.139649in}{2.312810in}}%
\pgfpathlineto{\pgfqpoint{3.139649in}{2.308552in}}%
\pgfpathmoveto{\pgfqpoint{3.131134in}{2.321326in}}%
\pgfpathlineto{\pgfqpoint{3.131134in}{2.321326in}}%
\pgfpathlineto{\pgfqpoint{3.131134in}{2.325584in}}%
\pgfpathlineto{\pgfqpoint{3.135391in}{2.325584in}}%
\pgfpathlineto{\pgfqpoint{3.135391in}{2.321326in}}%
\pgfpathmoveto{\pgfqpoint{3.131134in}{2.325584in}}%
\pgfpathlineto{\pgfqpoint{3.131134in}{2.325584in}}%
\pgfpathlineto{\pgfqpoint{3.131134in}{2.329842in}}%
\pgfpathlineto{\pgfqpoint{3.135391in}{2.329842in}}%
\pgfpathlineto{\pgfqpoint{3.135391in}{2.325584in}}%
\pgfpathmoveto{\pgfqpoint{3.135391in}{2.312810in}}%
\pgfpathlineto{\pgfqpoint{3.135391in}{2.312810in}}%
\pgfpathlineto{\pgfqpoint{3.135391in}{2.317068in}}%
\pgfpathlineto{\pgfqpoint{3.139649in}{2.317068in}}%
\pgfpathlineto{\pgfqpoint{3.139649in}{2.312810in}}%
\pgfpathmoveto{\pgfqpoint{3.135391in}{2.317068in}}%
\pgfpathlineto{\pgfqpoint{3.135391in}{2.317068in}}%
\pgfpathlineto{\pgfqpoint{3.135391in}{2.321326in}}%
\pgfpathlineto{\pgfqpoint{3.139649in}{2.321326in}}%
\pgfpathlineto{\pgfqpoint{3.139649in}{2.317068in}}%
\pgfpathmoveto{\pgfqpoint{3.135391in}{2.321326in}}%
\pgfpathlineto{\pgfqpoint{3.135391in}{2.321326in}}%
\pgfpathlineto{\pgfqpoint{3.135391in}{2.325584in}}%
\pgfpathlineto{\pgfqpoint{3.139649in}{2.325584in}}%
\pgfpathlineto{\pgfqpoint{3.139649in}{2.321326in}}%
\pgfpathmoveto{\pgfqpoint{3.131134in}{2.329842in}}%
\pgfpathlineto{\pgfqpoint{3.131134in}{2.329842in}}%
\pgfpathlineto{\pgfqpoint{3.131134in}{2.334100in}}%
\pgfpathlineto{\pgfqpoint{3.135391in}{2.334100in}}%
\pgfpathlineto{\pgfqpoint{3.135391in}{2.329842in}}%
\pgfpathmoveto{\pgfqpoint{3.131134in}{2.334100in}}%
\pgfpathlineto{\pgfqpoint{3.131134in}{2.334100in}}%
\pgfpathlineto{\pgfqpoint{3.131134in}{2.338358in}}%
\pgfpathlineto{\pgfqpoint{3.135391in}{2.338358in}}%
\pgfpathlineto{\pgfqpoint{3.135391in}{2.334100in}}%
\pgfpathmoveto{\pgfqpoint{3.131134in}{2.338358in}}%
\pgfpathlineto{\pgfqpoint{3.131134in}{2.338358in}}%
\pgfpathlineto{\pgfqpoint{3.131134in}{2.342616in}}%
\pgfpathlineto{\pgfqpoint{3.135391in}{2.342616in}}%
\pgfpathlineto{\pgfqpoint{3.135391in}{2.338358in}}%
\pgfpathmoveto{\pgfqpoint{3.131134in}{2.342616in}}%
\pgfpathlineto{\pgfqpoint{3.131134in}{2.342616in}}%
\pgfpathlineto{\pgfqpoint{3.131134in}{2.346874in}}%
\pgfpathlineto{\pgfqpoint{3.135391in}{2.346874in}}%
\pgfpathlineto{\pgfqpoint{3.135391in}{2.342616in}}%
\pgfpathmoveto{\pgfqpoint{3.122618in}{2.372421in}}%
\pgfpathlineto{\pgfqpoint{3.122618in}{2.372421in}}%
\pgfpathlineto{\pgfqpoint{3.122618in}{2.376679in}}%
\pgfpathlineto{\pgfqpoint{3.126876in}{2.376679in}}%
\pgfpathlineto{\pgfqpoint{3.126876in}{2.372421in}}%
\pgfpathmoveto{\pgfqpoint{3.122618in}{2.376679in}}%
\pgfpathlineto{\pgfqpoint{3.122618in}{2.376679in}}%
\pgfpathlineto{\pgfqpoint{3.122618in}{2.380937in}}%
\pgfpathlineto{\pgfqpoint{3.126876in}{2.380937in}}%
\pgfpathlineto{\pgfqpoint{3.126876in}{2.376679in}}%
\pgfpathmoveto{\pgfqpoint{3.122618in}{2.380937in}}%
\pgfpathlineto{\pgfqpoint{3.122618in}{2.380937in}}%
\pgfpathlineto{\pgfqpoint{3.122618in}{2.385195in}}%
\pgfpathlineto{\pgfqpoint{3.126876in}{2.385195in}}%
\pgfpathlineto{\pgfqpoint{3.126876in}{2.380937in}}%
\pgfpathmoveto{\pgfqpoint{3.122618in}{2.385195in}}%
\pgfpathlineto{\pgfqpoint{3.122618in}{2.385195in}}%
\pgfpathlineto{\pgfqpoint{3.122618in}{2.389453in}}%
\pgfpathlineto{\pgfqpoint{3.126876in}{2.389453in}}%
\pgfpathlineto{\pgfqpoint{3.126876in}{2.385195in}}%
\pgfpathmoveto{\pgfqpoint{3.122618in}{2.389453in}}%
\pgfpathlineto{\pgfqpoint{3.122618in}{2.389453in}}%
\pgfpathlineto{\pgfqpoint{3.122618in}{2.393711in}}%
\pgfpathlineto{\pgfqpoint{3.126876in}{2.393711in}}%
\pgfpathlineto{\pgfqpoint{3.126876in}{2.389453in}}%
\pgfpathmoveto{\pgfqpoint{3.122618in}{2.393711in}}%
\pgfpathlineto{\pgfqpoint{3.122618in}{2.393711in}}%
\pgfpathlineto{\pgfqpoint{3.122618in}{2.397969in}}%
\pgfpathlineto{\pgfqpoint{3.126876in}{2.397969in}}%
\pgfpathlineto{\pgfqpoint{3.126876in}{2.393711in}}%
\pgfpathmoveto{\pgfqpoint{3.118360in}{2.397969in}}%
\pgfpathlineto{\pgfqpoint{3.118360in}{2.397969in}}%
\pgfpathlineto{\pgfqpoint{3.118360in}{2.402226in}}%
\pgfpathlineto{\pgfqpoint{3.122618in}{2.402226in}}%
\pgfpathlineto{\pgfqpoint{3.122618in}{2.397969in}}%
\pgfpathmoveto{\pgfqpoint{3.118360in}{2.402226in}}%
\pgfpathlineto{\pgfqpoint{3.118360in}{2.402226in}}%
\pgfpathlineto{\pgfqpoint{3.118360in}{2.406484in}}%
\pgfpathlineto{\pgfqpoint{3.122618in}{2.406484in}}%
\pgfpathlineto{\pgfqpoint{3.122618in}{2.402226in}}%
\pgfpathmoveto{\pgfqpoint{3.122618in}{2.397969in}}%
\pgfpathlineto{\pgfqpoint{3.122618in}{2.397969in}}%
\pgfpathlineto{\pgfqpoint{3.122618in}{2.402226in}}%
\pgfpathlineto{\pgfqpoint{3.126876in}{2.402226in}}%
\pgfpathlineto{\pgfqpoint{3.126876in}{2.397969in}}%
\pgfpathmoveto{\pgfqpoint{3.118360in}{2.406484in}}%
\pgfpathlineto{\pgfqpoint{3.118360in}{2.406484in}}%
\pgfpathlineto{\pgfqpoint{3.118360in}{2.410742in}}%
\pgfpathlineto{\pgfqpoint{3.122618in}{2.410742in}}%
\pgfpathlineto{\pgfqpoint{3.122618in}{2.406484in}}%
\pgfpathmoveto{\pgfqpoint{3.118360in}{2.410742in}}%
\pgfpathlineto{\pgfqpoint{3.118360in}{2.410742in}}%
\pgfpathlineto{\pgfqpoint{3.118360in}{2.415000in}}%
\pgfpathlineto{\pgfqpoint{3.122618in}{2.415000in}}%
\pgfpathlineto{\pgfqpoint{3.122618in}{2.410742in}}%
\pgfpathmoveto{\pgfqpoint{3.126876in}{2.346874in}}%
\pgfpathlineto{\pgfqpoint{3.126876in}{2.346874in}}%
\pgfpathlineto{\pgfqpoint{3.126876in}{2.351132in}}%
\pgfpathlineto{\pgfqpoint{3.131134in}{2.351132in}}%
\pgfpathlineto{\pgfqpoint{3.131134in}{2.346874in}}%
\pgfpathmoveto{\pgfqpoint{3.126876in}{2.351132in}}%
\pgfpathlineto{\pgfqpoint{3.126876in}{2.351132in}}%
\pgfpathlineto{\pgfqpoint{3.126876in}{2.355389in}}%
\pgfpathlineto{\pgfqpoint{3.131134in}{2.355389in}}%
\pgfpathlineto{\pgfqpoint{3.131134in}{2.351132in}}%
\pgfpathmoveto{\pgfqpoint{3.131134in}{2.346874in}}%
\pgfpathlineto{\pgfqpoint{3.131134in}{2.346874in}}%
\pgfpathlineto{\pgfqpoint{3.131134in}{2.351132in}}%
\pgfpathlineto{\pgfqpoint{3.135391in}{2.351132in}}%
\pgfpathlineto{\pgfqpoint{3.135391in}{2.346874in}}%
\pgfpathmoveto{\pgfqpoint{3.126876in}{2.355389in}}%
\pgfpathlineto{\pgfqpoint{3.126876in}{2.355389in}}%
\pgfpathlineto{\pgfqpoint{3.126876in}{2.359647in}}%
\pgfpathlineto{\pgfqpoint{3.131134in}{2.359647in}}%
\pgfpathlineto{\pgfqpoint{3.131134in}{2.355389in}}%
\pgfpathmoveto{\pgfqpoint{3.126876in}{2.359647in}}%
\pgfpathlineto{\pgfqpoint{3.126876in}{2.359647in}}%
\pgfpathlineto{\pgfqpoint{3.126876in}{2.363905in}}%
\pgfpathlineto{\pgfqpoint{3.131134in}{2.363905in}}%
\pgfpathlineto{\pgfqpoint{3.131134in}{2.359647in}}%
\pgfpathmoveto{\pgfqpoint{3.126876in}{2.363905in}}%
\pgfpathlineto{\pgfqpoint{3.126876in}{2.363905in}}%
\pgfpathlineto{\pgfqpoint{3.126876in}{2.368163in}}%
\pgfpathlineto{\pgfqpoint{3.131134in}{2.368163in}}%
\pgfpathlineto{\pgfqpoint{3.131134in}{2.363905in}}%
\pgfpathmoveto{\pgfqpoint{3.126876in}{2.368163in}}%
\pgfpathlineto{\pgfqpoint{3.126876in}{2.368163in}}%
\pgfpathlineto{\pgfqpoint{3.126876in}{2.372421in}}%
\pgfpathlineto{\pgfqpoint{3.131134in}{2.372421in}}%
\pgfpathlineto{\pgfqpoint{3.131134in}{2.368163in}}%
\pgfpathmoveto{\pgfqpoint{3.126876in}{2.372421in}}%
\pgfpathlineto{\pgfqpoint{3.126876in}{2.372421in}}%
\pgfpathlineto{\pgfqpoint{3.126876in}{2.376679in}}%
\pgfpathlineto{\pgfqpoint{3.131134in}{2.376679in}}%
\pgfpathlineto{\pgfqpoint{3.131134in}{2.372421in}}%
\pgfpathmoveto{\pgfqpoint{3.114102in}{2.423516in}}%
\pgfpathlineto{\pgfqpoint{3.114102in}{2.423516in}}%
\pgfpathlineto{\pgfqpoint{3.114102in}{2.427773in}}%
\pgfpathlineto{\pgfqpoint{3.118360in}{2.427773in}}%
\pgfpathlineto{\pgfqpoint{3.118360in}{2.423516in}}%
\pgfpathmoveto{\pgfqpoint{3.114102in}{2.427773in}}%
\pgfpathlineto{\pgfqpoint{3.114102in}{2.427773in}}%
\pgfpathlineto{\pgfqpoint{3.114102in}{2.432031in}}%
\pgfpathlineto{\pgfqpoint{3.118360in}{2.432031in}}%
\pgfpathlineto{\pgfqpoint{3.118360in}{2.427773in}}%
\pgfpathmoveto{\pgfqpoint{3.118360in}{2.415000in}}%
\pgfpathlineto{\pgfqpoint{3.118360in}{2.415000in}}%
\pgfpathlineto{\pgfqpoint{3.118360in}{2.419258in}}%
\pgfpathlineto{\pgfqpoint{3.122618in}{2.419258in}}%
\pgfpathlineto{\pgfqpoint{3.122618in}{2.415000in}}%
\pgfpathmoveto{\pgfqpoint{3.118360in}{2.419258in}}%
\pgfpathlineto{\pgfqpoint{3.118360in}{2.419258in}}%
\pgfpathlineto{\pgfqpoint{3.118360in}{2.423516in}}%
\pgfpathlineto{\pgfqpoint{3.122618in}{2.423516in}}%
\pgfpathlineto{\pgfqpoint{3.122618in}{2.419258in}}%
\pgfpathmoveto{\pgfqpoint{3.118360in}{2.423516in}}%
\pgfpathlineto{\pgfqpoint{3.118360in}{2.423516in}}%
\pgfpathlineto{\pgfqpoint{3.118360in}{2.427773in}}%
\pgfpathlineto{\pgfqpoint{3.122618in}{2.427773in}}%
\pgfpathlineto{\pgfqpoint{3.122618in}{2.423516in}}%
\pgfpathmoveto{\pgfqpoint{3.114102in}{2.432031in}}%
\pgfpathlineto{\pgfqpoint{3.114102in}{2.432031in}}%
\pgfpathlineto{\pgfqpoint{3.114102in}{2.436289in}}%
\pgfpathlineto{\pgfqpoint{3.118360in}{2.436289in}}%
\pgfpathlineto{\pgfqpoint{3.118360in}{2.432031in}}%
\pgfpathmoveto{\pgfqpoint{3.114102in}{2.436289in}}%
\pgfpathlineto{\pgfqpoint{3.114102in}{2.436289in}}%
\pgfpathlineto{\pgfqpoint{3.114102in}{2.440547in}}%
\pgfpathlineto{\pgfqpoint{3.118360in}{2.440547in}}%
\pgfpathlineto{\pgfqpoint{3.118360in}{2.436289in}}%
\pgfpathmoveto{\pgfqpoint{3.114102in}{2.440547in}}%
\pgfpathlineto{\pgfqpoint{3.114102in}{2.440547in}}%
\pgfpathlineto{\pgfqpoint{3.114102in}{2.444804in}}%
\pgfpathlineto{\pgfqpoint{3.118360in}{2.444804in}}%
\pgfpathlineto{\pgfqpoint{3.118360in}{2.440547in}}%
\pgfpathmoveto{\pgfqpoint{3.114102in}{2.444804in}}%
\pgfpathlineto{\pgfqpoint{3.114102in}{2.444804in}}%
\pgfpathlineto{\pgfqpoint{3.114102in}{2.449062in}}%
\pgfpathlineto{\pgfqpoint{3.118360in}{2.449062in}}%
\pgfpathlineto{\pgfqpoint{3.118360in}{2.444804in}}%
\pgfpathmoveto{\pgfqpoint{3.105587in}{2.474608in}}%
\pgfpathlineto{\pgfqpoint{3.105587in}{2.474608in}}%
\pgfpathlineto{\pgfqpoint{3.105587in}{2.478866in}}%
\pgfpathlineto{\pgfqpoint{3.109844in}{2.478866in}}%
\pgfpathlineto{\pgfqpoint{3.109844in}{2.474608in}}%
\pgfpathmoveto{\pgfqpoint{3.105587in}{2.478866in}}%
\pgfpathlineto{\pgfqpoint{3.105587in}{2.478866in}}%
\pgfpathlineto{\pgfqpoint{3.105587in}{2.483124in}}%
\pgfpathlineto{\pgfqpoint{3.109844in}{2.483124in}}%
\pgfpathlineto{\pgfqpoint{3.109844in}{2.478866in}}%
\pgfpathmoveto{\pgfqpoint{3.109844in}{2.449062in}}%
\pgfpathlineto{\pgfqpoint{3.109844in}{2.449062in}}%
\pgfpathlineto{\pgfqpoint{3.109844in}{2.453320in}}%
\pgfpathlineto{\pgfqpoint{3.114102in}{2.453320in}}%
\pgfpathlineto{\pgfqpoint{3.114102in}{2.449062in}}%
\pgfpathmoveto{\pgfqpoint{3.109844in}{2.453320in}}%
\pgfpathlineto{\pgfqpoint{3.109844in}{2.453320in}}%
\pgfpathlineto{\pgfqpoint{3.109844in}{2.457577in}}%
\pgfpathlineto{\pgfqpoint{3.114102in}{2.457577in}}%
\pgfpathlineto{\pgfqpoint{3.114102in}{2.453320in}}%
\pgfpathmoveto{\pgfqpoint{3.114102in}{2.449062in}}%
\pgfpathlineto{\pgfqpoint{3.114102in}{2.449062in}}%
\pgfpathlineto{\pgfqpoint{3.114102in}{2.453320in}}%
\pgfpathlineto{\pgfqpoint{3.118360in}{2.453320in}}%
\pgfpathlineto{\pgfqpoint{3.118360in}{2.449062in}}%
\pgfpathmoveto{\pgfqpoint{3.109844in}{2.457577in}}%
\pgfpathlineto{\pgfqpoint{3.109844in}{2.457577in}}%
\pgfpathlineto{\pgfqpoint{3.109844in}{2.461835in}}%
\pgfpathlineto{\pgfqpoint{3.114102in}{2.461835in}}%
\pgfpathlineto{\pgfqpoint{3.114102in}{2.457577in}}%
\pgfpathmoveto{\pgfqpoint{3.109844in}{2.461835in}}%
\pgfpathlineto{\pgfqpoint{3.109844in}{2.461835in}}%
\pgfpathlineto{\pgfqpoint{3.109844in}{2.466093in}}%
\pgfpathlineto{\pgfqpoint{3.114102in}{2.466093in}}%
\pgfpathlineto{\pgfqpoint{3.114102in}{2.461835in}}%
\pgfpathmoveto{\pgfqpoint{3.109844in}{2.466093in}}%
\pgfpathlineto{\pgfqpoint{3.109844in}{2.466093in}}%
\pgfpathlineto{\pgfqpoint{3.109844in}{2.470351in}}%
\pgfpathlineto{\pgfqpoint{3.114102in}{2.470351in}}%
\pgfpathlineto{\pgfqpoint{3.114102in}{2.466093in}}%
\pgfpathmoveto{\pgfqpoint{3.109844in}{2.470351in}}%
\pgfpathlineto{\pgfqpoint{3.109844in}{2.470351in}}%
\pgfpathlineto{\pgfqpoint{3.109844in}{2.474608in}}%
\pgfpathlineto{\pgfqpoint{3.114102in}{2.474608in}}%
\pgfpathlineto{\pgfqpoint{3.114102in}{2.470351in}}%
\pgfpathmoveto{\pgfqpoint{3.109844in}{2.474608in}}%
\pgfpathlineto{\pgfqpoint{3.109844in}{2.474608in}}%
\pgfpathlineto{\pgfqpoint{3.109844in}{2.478866in}}%
\pgfpathlineto{\pgfqpoint{3.114102in}{2.478866in}}%
\pgfpathlineto{\pgfqpoint{3.114102in}{2.474608in}}%
\pgfpathmoveto{\pgfqpoint{3.105587in}{2.483124in}}%
\pgfpathlineto{\pgfqpoint{3.105587in}{2.483124in}}%
\pgfpathlineto{\pgfqpoint{3.105587in}{2.487381in}}%
\pgfpathlineto{\pgfqpoint{3.109844in}{2.487381in}}%
\pgfpathlineto{\pgfqpoint{3.109844in}{2.483124in}}%
\pgfpathmoveto{\pgfqpoint{3.105587in}{2.487381in}}%
\pgfpathlineto{\pgfqpoint{3.105587in}{2.487381in}}%
\pgfpathlineto{\pgfqpoint{3.105587in}{2.491639in}}%
\pgfpathlineto{\pgfqpoint{3.109844in}{2.491639in}}%
\pgfpathlineto{\pgfqpoint{3.109844in}{2.487381in}}%
\pgfpathmoveto{\pgfqpoint{3.105587in}{2.491639in}}%
\pgfpathlineto{\pgfqpoint{3.105587in}{2.491639in}}%
\pgfpathlineto{\pgfqpoint{3.105587in}{2.495897in}}%
\pgfpathlineto{\pgfqpoint{3.109844in}{2.495897in}}%
\pgfpathlineto{\pgfqpoint{3.109844in}{2.491639in}}%
\pgfpathmoveto{\pgfqpoint{3.105587in}{2.495897in}}%
\pgfpathlineto{\pgfqpoint{3.105587in}{2.495897in}}%
\pgfpathlineto{\pgfqpoint{3.105587in}{2.500155in}}%
\pgfpathlineto{\pgfqpoint{3.109844in}{2.500155in}}%
\pgfpathlineto{\pgfqpoint{3.109844in}{2.495897in}}%
\pgfpathmoveto{\pgfqpoint{3.101329in}{2.500155in}}%
\pgfpathlineto{\pgfqpoint{3.101329in}{2.500155in}}%
\pgfpathlineto{\pgfqpoint{3.101329in}{2.504412in}}%
\pgfpathlineto{\pgfqpoint{3.105587in}{2.504412in}}%
\pgfpathlineto{\pgfqpoint{3.105587in}{2.500155in}}%
\pgfpathmoveto{\pgfqpoint{3.101329in}{2.504412in}}%
\pgfpathlineto{\pgfqpoint{3.101329in}{2.504412in}}%
\pgfpathlineto{\pgfqpoint{3.101329in}{2.508670in}}%
\pgfpathlineto{\pgfqpoint{3.105587in}{2.508670in}}%
\pgfpathlineto{\pgfqpoint{3.105587in}{2.504412in}}%
\pgfpathmoveto{\pgfqpoint{3.105587in}{2.500155in}}%
\pgfpathlineto{\pgfqpoint{3.105587in}{2.500155in}}%
\pgfpathlineto{\pgfqpoint{3.105587in}{2.504412in}}%
\pgfpathlineto{\pgfqpoint{3.109844in}{2.504412in}}%
\pgfpathlineto{\pgfqpoint{3.109844in}{2.500155in}}%
\pgfpathmoveto{\pgfqpoint{3.101329in}{2.508670in}}%
\pgfpathlineto{\pgfqpoint{3.101329in}{2.508670in}}%
\pgfpathlineto{\pgfqpoint{3.101329in}{2.512928in}}%
\pgfpathlineto{\pgfqpoint{3.105587in}{2.512928in}}%
\pgfpathlineto{\pgfqpoint{3.105587in}{2.508670in}}%
\pgfpathmoveto{\pgfqpoint{3.101329in}{2.512928in}}%
\pgfpathlineto{\pgfqpoint{3.101329in}{2.512928in}}%
\pgfpathlineto{\pgfqpoint{3.101329in}{2.517185in}}%
\pgfpathlineto{\pgfqpoint{3.105587in}{2.517185in}}%
\pgfpathlineto{\pgfqpoint{3.105587in}{2.512928in}}%
\pgfpathmoveto{\pgfqpoint{3.097071in}{2.529959in}}%
\pgfpathlineto{\pgfqpoint{3.097071in}{2.529959in}}%
\pgfpathlineto{\pgfqpoint{3.097071in}{2.534216in}}%
\pgfpathlineto{\pgfqpoint{3.101329in}{2.534216in}}%
\pgfpathlineto{\pgfqpoint{3.101329in}{2.529959in}}%
\pgfpathmoveto{\pgfqpoint{3.101329in}{2.517185in}}%
\pgfpathlineto{\pgfqpoint{3.101329in}{2.517185in}}%
\pgfpathlineto{\pgfqpoint{3.101329in}{2.521443in}}%
\pgfpathlineto{\pgfqpoint{3.105587in}{2.521443in}}%
\pgfpathlineto{\pgfqpoint{3.105587in}{2.517185in}}%
\pgfpathmoveto{\pgfqpoint{3.101329in}{2.521443in}}%
\pgfpathlineto{\pgfqpoint{3.101329in}{2.521443in}}%
\pgfpathlineto{\pgfqpoint{3.101329in}{2.525701in}}%
\pgfpathlineto{\pgfqpoint{3.105587in}{2.525701in}}%
\pgfpathlineto{\pgfqpoint{3.105587in}{2.521443in}}%
\pgfpathmoveto{\pgfqpoint{3.101329in}{2.525701in}}%
\pgfpathlineto{\pgfqpoint{3.101329in}{2.525701in}}%
\pgfpathlineto{\pgfqpoint{3.101329in}{2.529959in}}%
\pgfpathlineto{\pgfqpoint{3.105587in}{2.529959in}}%
\pgfpathlineto{\pgfqpoint{3.105587in}{2.525701in}}%
\pgfpathmoveto{\pgfqpoint{3.101329in}{2.529959in}}%
\pgfpathlineto{\pgfqpoint{3.101329in}{2.529959in}}%
\pgfpathlineto{\pgfqpoint{3.101329in}{2.534216in}}%
\pgfpathlineto{\pgfqpoint{3.105587in}{2.534216in}}%
\pgfpathlineto{\pgfqpoint{3.105587in}{2.529959in}}%
\pgfpathmoveto{\pgfqpoint{3.097071in}{2.534216in}}%
\pgfpathlineto{\pgfqpoint{3.097071in}{2.534216in}}%
\pgfpathlineto{\pgfqpoint{3.097071in}{2.538474in}}%
\pgfpathlineto{\pgfqpoint{3.101329in}{2.538474in}}%
\pgfpathlineto{\pgfqpoint{3.101329in}{2.534216in}}%
\pgfpathmoveto{\pgfqpoint{3.097071in}{2.538474in}}%
\pgfpathlineto{\pgfqpoint{3.097071in}{2.538474in}}%
\pgfpathlineto{\pgfqpoint{3.097071in}{2.542732in}}%
\pgfpathlineto{\pgfqpoint{3.101329in}{2.542732in}}%
\pgfpathlineto{\pgfqpoint{3.101329in}{2.538474in}}%
\pgfpathmoveto{\pgfqpoint{3.097071in}{2.542732in}}%
\pgfpathlineto{\pgfqpoint{3.097071in}{2.542732in}}%
\pgfpathlineto{\pgfqpoint{3.097071in}{2.546989in}}%
\pgfpathlineto{\pgfqpoint{3.101329in}{2.546989in}}%
\pgfpathlineto{\pgfqpoint{3.101329in}{2.542732in}}%
\pgfpathmoveto{\pgfqpoint{3.097071in}{2.546989in}}%
\pgfpathlineto{\pgfqpoint{3.097071in}{2.546989in}}%
\pgfpathlineto{\pgfqpoint{3.097071in}{2.551247in}}%
\pgfpathlineto{\pgfqpoint{3.101329in}{2.551247in}}%
\pgfpathlineto{\pgfqpoint{3.101329in}{2.546989in}}%
\pgfpathmoveto{\pgfqpoint{3.092813in}{2.555505in}}%
\pgfpathlineto{\pgfqpoint{3.092813in}{2.555505in}}%
\pgfpathlineto{\pgfqpoint{3.092813in}{2.559763in}}%
\pgfpathlineto{\pgfqpoint{3.097071in}{2.559763in}}%
\pgfpathlineto{\pgfqpoint{3.097071in}{2.555505in}}%
\pgfpathmoveto{\pgfqpoint{3.097071in}{2.551247in}}%
\pgfpathlineto{\pgfqpoint{3.097071in}{2.551247in}}%
\pgfpathlineto{\pgfqpoint{3.097071in}{2.555505in}}%
\pgfpathlineto{\pgfqpoint{3.101329in}{2.555505in}}%
\pgfpathlineto{\pgfqpoint{3.101329in}{2.551247in}}%
\pgfpathmoveto{\pgfqpoint{3.097071in}{2.555505in}}%
\pgfpathlineto{\pgfqpoint{3.097071in}{2.555505in}}%
\pgfpathlineto{\pgfqpoint{3.097071in}{2.559763in}}%
\pgfpathlineto{\pgfqpoint{3.101329in}{2.559763in}}%
\pgfpathlineto{\pgfqpoint{3.101329in}{2.555505in}}%
\pgfpathmoveto{\pgfqpoint{3.092813in}{2.559763in}}%
\pgfpathlineto{\pgfqpoint{3.092813in}{2.559763in}}%
\pgfpathlineto{\pgfqpoint{3.092813in}{2.564021in}}%
\pgfpathlineto{\pgfqpoint{3.097071in}{2.564021in}}%
\pgfpathlineto{\pgfqpoint{3.097071in}{2.559763in}}%
\pgfpathmoveto{\pgfqpoint{3.092813in}{2.564021in}}%
\pgfpathlineto{\pgfqpoint{3.092813in}{2.564021in}}%
\pgfpathlineto{\pgfqpoint{3.092813in}{2.568279in}}%
\pgfpathlineto{\pgfqpoint{3.097071in}{2.568279in}}%
\pgfpathlineto{\pgfqpoint{3.097071in}{2.564021in}}%
\pgfpathmoveto{\pgfqpoint{3.092813in}{2.568279in}}%
\pgfpathlineto{\pgfqpoint{3.092813in}{2.568279in}}%
\pgfpathlineto{\pgfqpoint{3.092813in}{2.572536in}}%
\pgfpathlineto{\pgfqpoint{3.097071in}{2.572536in}}%
\pgfpathlineto{\pgfqpoint{3.097071in}{2.568279in}}%
\pgfpathmoveto{\pgfqpoint{3.092813in}{2.572536in}}%
\pgfpathlineto{\pgfqpoint{3.092813in}{2.572536in}}%
\pgfpathlineto{\pgfqpoint{3.092813in}{2.576794in}}%
\pgfpathlineto{\pgfqpoint{3.097071in}{2.576794in}}%
\pgfpathlineto{\pgfqpoint{3.097071in}{2.572536in}}%
\pgfpathmoveto{\pgfqpoint{3.092813in}{2.576794in}}%
\pgfpathlineto{\pgfqpoint{3.092813in}{2.576794in}}%
\pgfpathlineto{\pgfqpoint{3.092813in}{2.581052in}}%
\pgfpathlineto{\pgfqpoint{3.097071in}{2.581052in}}%
\pgfpathlineto{\pgfqpoint{3.097071in}{2.576794in}}%
\pgfpathmoveto{\pgfqpoint{3.092813in}{2.581052in}}%
\pgfpathlineto{\pgfqpoint{3.092813in}{2.581052in}}%
\pgfpathlineto{\pgfqpoint{3.092813in}{2.585310in}}%
\pgfpathlineto{\pgfqpoint{3.097071in}{2.585310in}}%
\pgfpathlineto{\pgfqpoint{3.097071in}{2.581052in}}%
\pgfpathmoveto{\pgfqpoint{3.361053in}{1.482538in}}%
\pgfpathlineto{\pgfqpoint{3.361053in}{1.482538in}}%
\pgfpathlineto{\pgfqpoint{3.361053in}{1.486796in}}%
\pgfpathlineto{\pgfqpoint{3.365311in}{1.486796in}}%
\pgfpathlineto{\pgfqpoint{3.365311in}{1.482538in}}%
\pgfpathmoveto{\pgfqpoint{3.356795in}{1.491054in}}%
\pgfpathlineto{\pgfqpoint{3.356795in}{1.491054in}}%
\pgfpathlineto{\pgfqpoint{3.356795in}{1.495312in}}%
\pgfpathlineto{\pgfqpoint{3.361053in}{1.495312in}}%
\pgfpathlineto{\pgfqpoint{3.361053in}{1.491054in}}%
\pgfpathmoveto{\pgfqpoint{3.361053in}{1.486796in}}%
\pgfpathlineto{\pgfqpoint{3.361053in}{1.486796in}}%
\pgfpathlineto{\pgfqpoint{3.361053in}{1.491054in}}%
\pgfpathlineto{\pgfqpoint{3.365311in}{1.491054in}}%
\pgfpathlineto{\pgfqpoint{3.365311in}{1.486796in}}%
\pgfpathmoveto{\pgfqpoint{3.361053in}{1.491054in}}%
\pgfpathlineto{\pgfqpoint{3.361053in}{1.491054in}}%
\pgfpathlineto{\pgfqpoint{3.361053in}{1.495312in}}%
\pgfpathlineto{\pgfqpoint{3.365311in}{1.495312in}}%
\pgfpathlineto{\pgfqpoint{3.365311in}{1.491054in}}%
\pgfpathmoveto{\pgfqpoint{3.344022in}{1.512344in}}%
\pgfpathlineto{\pgfqpoint{3.344022in}{1.512344in}}%
\pgfpathlineto{\pgfqpoint{3.344022in}{1.516602in}}%
\pgfpathlineto{\pgfqpoint{3.348280in}{1.516602in}}%
\pgfpathlineto{\pgfqpoint{3.348280in}{1.512344in}}%
\pgfpathmoveto{\pgfqpoint{3.344022in}{1.516602in}}%
\pgfpathlineto{\pgfqpoint{3.344022in}{1.516602in}}%
\pgfpathlineto{\pgfqpoint{3.344022in}{1.520860in}}%
\pgfpathlineto{\pgfqpoint{3.348280in}{1.520860in}}%
\pgfpathlineto{\pgfqpoint{3.348280in}{1.516602in}}%
\pgfpathmoveto{\pgfqpoint{3.339764in}{1.520860in}}%
\pgfpathlineto{\pgfqpoint{3.339764in}{1.520860in}}%
\pgfpathlineto{\pgfqpoint{3.339764in}{1.525118in}}%
\pgfpathlineto{\pgfqpoint{3.344022in}{1.525118in}}%
\pgfpathlineto{\pgfqpoint{3.344022in}{1.520860in}}%
\pgfpathmoveto{\pgfqpoint{3.339764in}{1.525118in}}%
\pgfpathlineto{\pgfqpoint{3.339764in}{1.525118in}}%
\pgfpathlineto{\pgfqpoint{3.339764in}{1.529376in}}%
\pgfpathlineto{\pgfqpoint{3.344022in}{1.529376in}}%
\pgfpathlineto{\pgfqpoint{3.344022in}{1.525118in}}%
\pgfpathmoveto{\pgfqpoint{3.344022in}{1.520860in}}%
\pgfpathlineto{\pgfqpoint{3.344022in}{1.520860in}}%
\pgfpathlineto{\pgfqpoint{3.344022in}{1.525118in}}%
\pgfpathlineto{\pgfqpoint{3.348280in}{1.525118in}}%
\pgfpathlineto{\pgfqpoint{3.348280in}{1.520860in}}%
\pgfpathmoveto{\pgfqpoint{3.344022in}{1.525118in}}%
\pgfpathlineto{\pgfqpoint{3.344022in}{1.525118in}}%
\pgfpathlineto{\pgfqpoint{3.344022in}{1.529376in}}%
\pgfpathlineto{\pgfqpoint{3.348280in}{1.529376in}}%
\pgfpathlineto{\pgfqpoint{3.348280in}{1.525118in}}%
\pgfpathmoveto{\pgfqpoint{3.352537in}{1.495312in}}%
\pgfpathlineto{\pgfqpoint{3.352537in}{1.495312in}}%
\pgfpathlineto{\pgfqpoint{3.352537in}{1.499570in}}%
\pgfpathlineto{\pgfqpoint{3.356795in}{1.499570in}}%
\pgfpathlineto{\pgfqpoint{3.356795in}{1.495312in}}%
\pgfpathmoveto{\pgfqpoint{3.352537in}{1.499570in}}%
\pgfpathlineto{\pgfqpoint{3.352537in}{1.499570in}}%
\pgfpathlineto{\pgfqpoint{3.352537in}{1.503828in}}%
\pgfpathlineto{\pgfqpoint{3.356795in}{1.503828in}}%
\pgfpathlineto{\pgfqpoint{3.356795in}{1.499570in}}%
\pgfpathmoveto{\pgfqpoint{3.348280in}{1.503828in}}%
\pgfpathlineto{\pgfqpoint{3.348280in}{1.503828in}}%
\pgfpathlineto{\pgfqpoint{3.348280in}{1.508086in}}%
\pgfpathlineto{\pgfqpoint{3.352537in}{1.508086in}}%
\pgfpathlineto{\pgfqpoint{3.352537in}{1.503828in}}%
\pgfpathmoveto{\pgfqpoint{3.348280in}{1.508086in}}%
\pgfpathlineto{\pgfqpoint{3.348280in}{1.508086in}}%
\pgfpathlineto{\pgfqpoint{3.348280in}{1.512344in}}%
\pgfpathlineto{\pgfqpoint{3.352537in}{1.512344in}}%
\pgfpathlineto{\pgfqpoint{3.352537in}{1.508086in}}%
\pgfpathmoveto{\pgfqpoint{3.352537in}{1.503828in}}%
\pgfpathlineto{\pgfqpoint{3.352537in}{1.503828in}}%
\pgfpathlineto{\pgfqpoint{3.352537in}{1.508086in}}%
\pgfpathlineto{\pgfqpoint{3.356795in}{1.508086in}}%
\pgfpathlineto{\pgfqpoint{3.356795in}{1.503828in}}%
\pgfpathmoveto{\pgfqpoint{3.352537in}{1.508086in}}%
\pgfpathlineto{\pgfqpoint{3.352537in}{1.508086in}}%
\pgfpathlineto{\pgfqpoint{3.352537in}{1.512344in}}%
\pgfpathlineto{\pgfqpoint{3.356795in}{1.512344in}}%
\pgfpathlineto{\pgfqpoint{3.356795in}{1.508086in}}%
\pgfpathmoveto{\pgfqpoint{3.356795in}{1.495312in}}%
\pgfpathlineto{\pgfqpoint{3.356795in}{1.495312in}}%
\pgfpathlineto{\pgfqpoint{3.356795in}{1.499570in}}%
\pgfpathlineto{\pgfqpoint{3.361053in}{1.499570in}}%
\pgfpathlineto{\pgfqpoint{3.361053in}{1.495312in}}%
\pgfpathmoveto{\pgfqpoint{3.356795in}{1.499570in}}%
\pgfpathlineto{\pgfqpoint{3.356795in}{1.499570in}}%
\pgfpathlineto{\pgfqpoint{3.356795in}{1.503828in}}%
\pgfpathlineto{\pgfqpoint{3.361053in}{1.503828in}}%
\pgfpathlineto{\pgfqpoint{3.361053in}{1.499570in}}%
\pgfpathmoveto{\pgfqpoint{3.361053in}{1.495312in}}%
\pgfpathlineto{\pgfqpoint{3.361053in}{1.495312in}}%
\pgfpathlineto{\pgfqpoint{3.361053in}{1.499570in}}%
\pgfpathlineto{\pgfqpoint{3.365311in}{1.499570in}}%
\pgfpathlineto{\pgfqpoint{3.365311in}{1.495312in}}%
\pgfpathmoveto{\pgfqpoint{3.361053in}{1.499570in}}%
\pgfpathlineto{\pgfqpoint{3.361053in}{1.499570in}}%
\pgfpathlineto{\pgfqpoint{3.361053in}{1.503828in}}%
\pgfpathlineto{\pgfqpoint{3.365311in}{1.503828in}}%
\pgfpathlineto{\pgfqpoint{3.365311in}{1.499570in}}%
\pgfpathmoveto{\pgfqpoint{3.356795in}{1.503828in}}%
\pgfpathlineto{\pgfqpoint{3.356795in}{1.503828in}}%
\pgfpathlineto{\pgfqpoint{3.356795in}{1.508086in}}%
\pgfpathlineto{\pgfqpoint{3.361053in}{1.508086in}}%
\pgfpathlineto{\pgfqpoint{3.361053in}{1.503828in}}%
\pgfpathmoveto{\pgfqpoint{3.356795in}{1.508086in}}%
\pgfpathlineto{\pgfqpoint{3.356795in}{1.508086in}}%
\pgfpathlineto{\pgfqpoint{3.356795in}{1.512344in}}%
\pgfpathlineto{\pgfqpoint{3.361053in}{1.512344in}}%
\pgfpathlineto{\pgfqpoint{3.361053in}{1.508086in}}%
\pgfpathmoveto{\pgfqpoint{3.361053in}{1.503828in}}%
\pgfpathlineto{\pgfqpoint{3.361053in}{1.503828in}}%
\pgfpathlineto{\pgfqpoint{3.361053in}{1.508086in}}%
\pgfpathlineto{\pgfqpoint{3.365311in}{1.508086in}}%
\pgfpathlineto{\pgfqpoint{3.365311in}{1.503828in}}%
\pgfpathmoveto{\pgfqpoint{3.361053in}{1.508086in}}%
\pgfpathlineto{\pgfqpoint{3.361053in}{1.508086in}}%
\pgfpathlineto{\pgfqpoint{3.361053in}{1.512344in}}%
\pgfpathlineto{\pgfqpoint{3.365311in}{1.512344in}}%
\pgfpathlineto{\pgfqpoint{3.365311in}{1.508086in}}%
\pgfpathmoveto{\pgfqpoint{3.348280in}{1.512344in}}%
\pgfpathlineto{\pgfqpoint{3.348280in}{1.512344in}}%
\pgfpathlineto{\pgfqpoint{3.348280in}{1.516602in}}%
\pgfpathlineto{\pgfqpoint{3.352537in}{1.516602in}}%
\pgfpathlineto{\pgfqpoint{3.352537in}{1.512344in}}%
\pgfpathmoveto{\pgfqpoint{3.348280in}{1.516602in}}%
\pgfpathlineto{\pgfqpoint{3.348280in}{1.516602in}}%
\pgfpathlineto{\pgfqpoint{3.348280in}{1.520860in}}%
\pgfpathlineto{\pgfqpoint{3.352537in}{1.520860in}}%
\pgfpathlineto{\pgfqpoint{3.352537in}{1.516602in}}%
\pgfpathmoveto{\pgfqpoint{3.352537in}{1.512344in}}%
\pgfpathlineto{\pgfqpoint{3.352537in}{1.512344in}}%
\pgfpathlineto{\pgfqpoint{3.352537in}{1.516602in}}%
\pgfpathlineto{\pgfqpoint{3.356795in}{1.516602in}}%
\pgfpathlineto{\pgfqpoint{3.356795in}{1.512344in}}%
\pgfpathmoveto{\pgfqpoint{3.352537in}{1.516602in}}%
\pgfpathlineto{\pgfqpoint{3.352537in}{1.516602in}}%
\pgfpathlineto{\pgfqpoint{3.352537in}{1.520860in}}%
\pgfpathlineto{\pgfqpoint{3.356795in}{1.520860in}}%
\pgfpathlineto{\pgfqpoint{3.356795in}{1.516602in}}%
\pgfpathmoveto{\pgfqpoint{3.348280in}{1.520860in}}%
\pgfpathlineto{\pgfqpoint{3.348280in}{1.520860in}}%
\pgfpathlineto{\pgfqpoint{3.348280in}{1.525118in}}%
\pgfpathlineto{\pgfqpoint{3.352537in}{1.525118in}}%
\pgfpathlineto{\pgfqpoint{3.352537in}{1.520860in}}%
\pgfpathmoveto{\pgfqpoint{3.348280in}{1.525118in}}%
\pgfpathlineto{\pgfqpoint{3.348280in}{1.525118in}}%
\pgfpathlineto{\pgfqpoint{3.348280in}{1.529376in}}%
\pgfpathlineto{\pgfqpoint{3.352537in}{1.529376in}}%
\pgfpathlineto{\pgfqpoint{3.352537in}{1.525118in}}%
\pgfpathmoveto{\pgfqpoint{3.352537in}{1.520860in}}%
\pgfpathlineto{\pgfqpoint{3.352537in}{1.520860in}}%
\pgfpathlineto{\pgfqpoint{3.352537in}{1.525118in}}%
\pgfpathlineto{\pgfqpoint{3.356795in}{1.525118in}}%
\pgfpathlineto{\pgfqpoint{3.356795in}{1.520860in}}%
\pgfpathmoveto{\pgfqpoint{3.352537in}{1.525118in}}%
\pgfpathlineto{\pgfqpoint{3.352537in}{1.525118in}}%
\pgfpathlineto{\pgfqpoint{3.352537in}{1.529376in}}%
\pgfpathlineto{\pgfqpoint{3.356795in}{1.529376in}}%
\pgfpathlineto{\pgfqpoint{3.356795in}{1.525118in}}%
\pgfpathmoveto{\pgfqpoint{3.356795in}{1.512344in}}%
\pgfpathlineto{\pgfqpoint{3.356795in}{1.512344in}}%
\pgfpathlineto{\pgfqpoint{3.356795in}{1.516602in}}%
\pgfpathlineto{\pgfqpoint{3.361053in}{1.516602in}}%
\pgfpathlineto{\pgfqpoint{3.361053in}{1.512344in}}%
\pgfpathmoveto{\pgfqpoint{3.356795in}{1.516602in}}%
\pgfpathlineto{\pgfqpoint{3.356795in}{1.516602in}}%
\pgfpathlineto{\pgfqpoint{3.356795in}{1.520860in}}%
\pgfpathlineto{\pgfqpoint{3.361053in}{1.520860in}}%
\pgfpathlineto{\pgfqpoint{3.361053in}{1.516602in}}%
\pgfpathmoveto{\pgfqpoint{3.361053in}{1.512344in}}%
\pgfpathlineto{\pgfqpoint{3.361053in}{1.512344in}}%
\pgfpathlineto{\pgfqpoint{3.361053in}{1.516602in}}%
\pgfpathlineto{\pgfqpoint{3.365311in}{1.516602in}}%
\pgfpathlineto{\pgfqpoint{3.365311in}{1.512344in}}%
\pgfpathmoveto{\pgfqpoint{3.326991in}{1.546408in}}%
\pgfpathlineto{\pgfqpoint{3.326991in}{1.546408in}}%
\pgfpathlineto{\pgfqpoint{3.326991in}{1.550666in}}%
\pgfpathlineto{\pgfqpoint{3.331249in}{1.550666in}}%
\pgfpathlineto{\pgfqpoint{3.331249in}{1.546408in}}%
\pgfpathmoveto{\pgfqpoint{3.326991in}{1.550666in}}%
\pgfpathlineto{\pgfqpoint{3.326991in}{1.550666in}}%
\pgfpathlineto{\pgfqpoint{3.326991in}{1.554924in}}%
\pgfpathlineto{\pgfqpoint{3.331249in}{1.554924in}}%
\pgfpathlineto{\pgfqpoint{3.331249in}{1.550666in}}%
\pgfpathmoveto{\pgfqpoint{3.322733in}{1.554924in}}%
\pgfpathlineto{\pgfqpoint{3.322733in}{1.554924in}}%
\pgfpathlineto{\pgfqpoint{3.322733in}{1.559182in}}%
\pgfpathlineto{\pgfqpoint{3.326991in}{1.559182in}}%
\pgfpathlineto{\pgfqpoint{3.326991in}{1.554924in}}%
\pgfpathmoveto{\pgfqpoint{3.322733in}{1.559182in}}%
\pgfpathlineto{\pgfqpoint{3.322733in}{1.559182in}}%
\pgfpathlineto{\pgfqpoint{3.322733in}{1.563440in}}%
\pgfpathlineto{\pgfqpoint{3.326991in}{1.563440in}}%
\pgfpathlineto{\pgfqpoint{3.326991in}{1.559182in}}%
\pgfpathmoveto{\pgfqpoint{3.326991in}{1.554924in}}%
\pgfpathlineto{\pgfqpoint{3.326991in}{1.554924in}}%
\pgfpathlineto{\pgfqpoint{3.326991in}{1.559182in}}%
\pgfpathlineto{\pgfqpoint{3.331249in}{1.559182in}}%
\pgfpathlineto{\pgfqpoint{3.331249in}{1.554924in}}%
\pgfpathmoveto{\pgfqpoint{3.326991in}{1.559182in}}%
\pgfpathlineto{\pgfqpoint{3.326991in}{1.559182in}}%
\pgfpathlineto{\pgfqpoint{3.326991in}{1.563440in}}%
\pgfpathlineto{\pgfqpoint{3.331249in}{1.563440in}}%
\pgfpathlineto{\pgfqpoint{3.331249in}{1.559182in}}%
\pgfpathmoveto{\pgfqpoint{3.309960in}{1.588988in}}%
\pgfpathlineto{\pgfqpoint{3.309960in}{1.588988in}}%
\pgfpathlineto{\pgfqpoint{3.309960in}{1.593246in}}%
\pgfpathlineto{\pgfqpoint{3.314218in}{1.593246in}}%
\pgfpathlineto{\pgfqpoint{3.314218in}{1.588988in}}%
\pgfpathmoveto{\pgfqpoint{3.309960in}{1.593246in}}%
\pgfpathlineto{\pgfqpoint{3.309960in}{1.593246in}}%
\pgfpathlineto{\pgfqpoint{3.309960in}{1.597504in}}%
\pgfpathlineto{\pgfqpoint{3.314218in}{1.597504in}}%
\pgfpathlineto{\pgfqpoint{3.314218in}{1.593246in}}%
\pgfpathmoveto{\pgfqpoint{3.318476in}{1.567698in}}%
\pgfpathlineto{\pgfqpoint{3.318476in}{1.567698in}}%
\pgfpathlineto{\pgfqpoint{3.318476in}{1.571956in}}%
\pgfpathlineto{\pgfqpoint{3.322733in}{1.571956in}}%
\pgfpathlineto{\pgfqpoint{3.322733in}{1.567698in}}%
\pgfpathmoveto{\pgfqpoint{3.314218in}{1.576214in}}%
\pgfpathlineto{\pgfqpoint{3.314218in}{1.576214in}}%
\pgfpathlineto{\pgfqpoint{3.314218in}{1.580472in}}%
\pgfpathlineto{\pgfqpoint{3.318476in}{1.580472in}}%
\pgfpathlineto{\pgfqpoint{3.318476in}{1.576214in}}%
\pgfpathmoveto{\pgfqpoint{3.318476in}{1.571956in}}%
\pgfpathlineto{\pgfqpoint{3.318476in}{1.571956in}}%
\pgfpathlineto{\pgfqpoint{3.318476in}{1.576214in}}%
\pgfpathlineto{\pgfqpoint{3.322733in}{1.576214in}}%
\pgfpathlineto{\pgfqpoint{3.322733in}{1.571956in}}%
\pgfpathmoveto{\pgfqpoint{3.318476in}{1.576214in}}%
\pgfpathlineto{\pgfqpoint{3.318476in}{1.576214in}}%
\pgfpathlineto{\pgfqpoint{3.318476in}{1.580472in}}%
\pgfpathlineto{\pgfqpoint{3.322733in}{1.580472in}}%
\pgfpathlineto{\pgfqpoint{3.322733in}{1.576214in}}%
\pgfpathmoveto{\pgfqpoint{3.322733in}{1.563440in}}%
\pgfpathlineto{\pgfqpoint{3.322733in}{1.563440in}}%
\pgfpathlineto{\pgfqpoint{3.322733in}{1.567698in}}%
\pgfpathlineto{\pgfqpoint{3.326991in}{1.567698in}}%
\pgfpathlineto{\pgfqpoint{3.326991in}{1.563440in}}%
\pgfpathmoveto{\pgfqpoint{3.322733in}{1.567698in}}%
\pgfpathlineto{\pgfqpoint{3.322733in}{1.567698in}}%
\pgfpathlineto{\pgfqpoint{3.322733in}{1.571956in}}%
\pgfpathlineto{\pgfqpoint{3.326991in}{1.571956in}}%
\pgfpathlineto{\pgfqpoint{3.326991in}{1.567698in}}%
\pgfpathmoveto{\pgfqpoint{3.326991in}{1.563440in}}%
\pgfpathlineto{\pgfqpoint{3.326991in}{1.563440in}}%
\pgfpathlineto{\pgfqpoint{3.326991in}{1.567698in}}%
\pgfpathlineto{\pgfqpoint{3.331249in}{1.567698in}}%
\pgfpathlineto{\pgfqpoint{3.331249in}{1.563440in}}%
\pgfpathmoveto{\pgfqpoint{3.326991in}{1.567698in}}%
\pgfpathlineto{\pgfqpoint{3.326991in}{1.567698in}}%
\pgfpathlineto{\pgfqpoint{3.326991in}{1.571956in}}%
\pgfpathlineto{\pgfqpoint{3.331249in}{1.571956in}}%
\pgfpathlineto{\pgfqpoint{3.331249in}{1.567698in}}%
\pgfpathmoveto{\pgfqpoint{3.322733in}{1.571956in}}%
\pgfpathlineto{\pgfqpoint{3.322733in}{1.571956in}}%
\pgfpathlineto{\pgfqpoint{3.322733in}{1.576214in}}%
\pgfpathlineto{\pgfqpoint{3.326991in}{1.576214in}}%
\pgfpathlineto{\pgfqpoint{3.326991in}{1.571956in}}%
\pgfpathmoveto{\pgfqpoint{3.322733in}{1.576214in}}%
\pgfpathlineto{\pgfqpoint{3.322733in}{1.576214in}}%
\pgfpathlineto{\pgfqpoint{3.322733in}{1.580472in}}%
\pgfpathlineto{\pgfqpoint{3.326991in}{1.580472in}}%
\pgfpathlineto{\pgfqpoint{3.326991in}{1.576214in}}%
\pgfpathmoveto{\pgfqpoint{3.326991in}{1.571956in}}%
\pgfpathlineto{\pgfqpoint{3.326991in}{1.571956in}}%
\pgfpathlineto{\pgfqpoint{3.326991in}{1.576214in}}%
\pgfpathlineto{\pgfqpoint{3.331249in}{1.576214in}}%
\pgfpathlineto{\pgfqpoint{3.331249in}{1.571956in}}%
\pgfpathmoveto{\pgfqpoint{3.314218in}{1.580472in}}%
\pgfpathlineto{\pgfqpoint{3.314218in}{1.580472in}}%
\pgfpathlineto{\pgfqpoint{3.314218in}{1.584730in}}%
\pgfpathlineto{\pgfqpoint{3.318476in}{1.584730in}}%
\pgfpathlineto{\pgfqpoint{3.318476in}{1.580472in}}%
\pgfpathmoveto{\pgfqpoint{3.314218in}{1.584730in}}%
\pgfpathlineto{\pgfqpoint{3.314218in}{1.584730in}}%
\pgfpathlineto{\pgfqpoint{3.314218in}{1.588988in}}%
\pgfpathlineto{\pgfqpoint{3.318476in}{1.588988in}}%
\pgfpathlineto{\pgfqpoint{3.318476in}{1.584730in}}%
\pgfpathmoveto{\pgfqpoint{3.318476in}{1.580472in}}%
\pgfpathlineto{\pgfqpoint{3.318476in}{1.580472in}}%
\pgfpathlineto{\pgfqpoint{3.318476in}{1.584730in}}%
\pgfpathlineto{\pgfqpoint{3.322733in}{1.584730in}}%
\pgfpathlineto{\pgfqpoint{3.322733in}{1.580472in}}%
\pgfpathmoveto{\pgfqpoint{3.318476in}{1.584730in}}%
\pgfpathlineto{\pgfqpoint{3.318476in}{1.584730in}}%
\pgfpathlineto{\pgfqpoint{3.318476in}{1.588988in}}%
\pgfpathlineto{\pgfqpoint{3.322733in}{1.588988in}}%
\pgfpathlineto{\pgfqpoint{3.322733in}{1.584730in}}%
\pgfpathmoveto{\pgfqpoint{3.314218in}{1.588988in}}%
\pgfpathlineto{\pgfqpoint{3.314218in}{1.588988in}}%
\pgfpathlineto{\pgfqpoint{3.314218in}{1.593246in}}%
\pgfpathlineto{\pgfqpoint{3.318476in}{1.593246in}}%
\pgfpathlineto{\pgfqpoint{3.318476in}{1.588988in}}%
\pgfpathmoveto{\pgfqpoint{3.314218in}{1.593246in}}%
\pgfpathlineto{\pgfqpoint{3.314218in}{1.593246in}}%
\pgfpathlineto{\pgfqpoint{3.314218in}{1.597504in}}%
\pgfpathlineto{\pgfqpoint{3.318476in}{1.597504in}}%
\pgfpathlineto{\pgfqpoint{3.318476in}{1.593246in}}%
\pgfpathmoveto{\pgfqpoint{3.318476in}{1.588988in}}%
\pgfpathlineto{\pgfqpoint{3.318476in}{1.588988in}}%
\pgfpathlineto{\pgfqpoint{3.318476in}{1.593246in}}%
\pgfpathlineto{\pgfqpoint{3.322733in}{1.593246in}}%
\pgfpathlineto{\pgfqpoint{3.322733in}{1.588988in}}%
\pgfpathmoveto{\pgfqpoint{3.322733in}{1.580472in}}%
\pgfpathlineto{\pgfqpoint{3.322733in}{1.580472in}}%
\pgfpathlineto{\pgfqpoint{3.322733in}{1.584730in}}%
\pgfpathlineto{\pgfqpoint{3.326991in}{1.584730in}}%
\pgfpathlineto{\pgfqpoint{3.326991in}{1.580472in}}%
\pgfpathmoveto{\pgfqpoint{3.335507in}{1.529376in}}%
\pgfpathlineto{\pgfqpoint{3.335507in}{1.529376in}}%
\pgfpathlineto{\pgfqpoint{3.335507in}{1.533634in}}%
\pgfpathlineto{\pgfqpoint{3.339764in}{1.533634in}}%
\pgfpathlineto{\pgfqpoint{3.339764in}{1.529376in}}%
\pgfpathmoveto{\pgfqpoint{3.335507in}{1.533634in}}%
\pgfpathlineto{\pgfqpoint{3.335507in}{1.533634in}}%
\pgfpathlineto{\pgfqpoint{3.335507in}{1.537892in}}%
\pgfpathlineto{\pgfqpoint{3.339764in}{1.537892in}}%
\pgfpathlineto{\pgfqpoint{3.339764in}{1.533634in}}%
\pgfpathmoveto{\pgfqpoint{3.331249in}{1.537892in}}%
\pgfpathlineto{\pgfqpoint{3.331249in}{1.537892in}}%
\pgfpathlineto{\pgfqpoint{3.331249in}{1.542150in}}%
\pgfpathlineto{\pgfqpoint{3.335507in}{1.542150in}}%
\pgfpathlineto{\pgfqpoint{3.335507in}{1.537892in}}%
\pgfpathmoveto{\pgfqpoint{3.331249in}{1.542150in}}%
\pgfpathlineto{\pgfqpoint{3.331249in}{1.542150in}}%
\pgfpathlineto{\pgfqpoint{3.331249in}{1.546408in}}%
\pgfpathlineto{\pgfqpoint{3.335507in}{1.546408in}}%
\pgfpathlineto{\pgfqpoint{3.335507in}{1.542150in}}%
\pgfpathmoveto{\pgfqpoint{3.335507in}{1.537892in}}%
\pgfpathlineto{\pgfqpoint{3.335507in}{1.537892in}}%
\pgfpathlineto{\pgfqpoint{3.335507in}{1.542150in}}%
\pgfpathlineto{\pgfqpoint{3.339764in}{1.542150in}}%
\pgfpathlineto{\pgfqpoint{3.339764in}{1.537892in}}%
\pgfpathmoveto{\pgfqpoint{3.335507in}{1.542150in}}%
\pgfpathlineto{\pgfqpoint{3.335507in}{1.542150in}}%
\pgfpathlineto{\pgfqpoint{3.335507in}{1.546408in}}%
\pgfpathlineto{\pgfqpoint{3.339764in}{1.546408in}}%
\pgfpathlineto{\pgfqpoint{3.339764in}{1.542150in}}%
\pgfpathmoveto{\pgfqpoint{3.339764in}{1.529376in}}%
\pgfpathlineto{\pgfqpoint{3.339764in}{1.529376in}}%
\pgfpathlineto{\pgfqpoint{3.339764in}{1.533634in}}%
\pgfpathlineto{\pgfqpoint{3.344022in}{1.533634in}}%
\pgfpathlineto{\pgfqpoint{3.344022in}{1.529376in}}%
\pgfpathmoveto{\pgfqpoint{3.339764in}{1.533634in}}%
\pgfpathlineto{\pgfqpoint{3.339764in}{1.533634in}}%
\pgfpathlineto{\pgfqpoint{3.339764in}{1.537892in}}%
\pgfpathlineto{\pgfqpoint{3.344022in}{1.537892in}}%
\pgfpathlineto{\pgfqpoint{3.344022in}{1.533634in}}%
\pgfpathmoveto{\pgfqpoint{3.344022in}{1.529376in}}%
\pgfpathlineto{\pgfqpoint{3.344022in}{1.529376in}}%
\pgfpathlineto{\pgfqpoint{3.344022in}{1.533634in}}%
\pgfpathlineto{\pgfqpoint{3.348280in}{1.533634in}}%
\pgfpathlineto{\pgfqpoint{3.348280in}{1.529376in}}%
\pgfpathmoveto{\pgfqpoint{3.344022in}{1.533634in}}%
\pgfpathlineto{\pgfqpoint{3.344022in}{1.533634in}}%
\pgfpathlineto{\pgfqpoint{3.344022in}{1.537892in}}%
\pgfpathlineto{\pgfqpoint{3.348280in}{1.537892in}}%
\pgfpathlineto{\pgfqpoint{3.348280in}{1.533634in}}%
\pgfpathmoveto{\pgfqpoint{3.339764in}{1.537892in}}%
\pgfpathlineto{\pgfqpoint{3.339764in}{1.537892in}}%
\pgfpathlineto{\pgfqpoint{3.339764in}{1.542150in}}%
\pgfpathlineto{\pgfqpoint{3.344022in}{1.542150in}}%
\pgfpathlineto{\pgfqpoint{3.344022in}{1.537892in}}%
\pgfpathmoveto{\pgfqpoint{3.339764in}{1.542150in}}%
\pgfpathlineto{\pgfqpoint{3.339764in}{1.542150in}}%
\pgfpathlineto{\pgfqpoint{3.339764in}{1.546408in}}%
\pgfpathlineto{\pgfqpoint{3.344022in}{1.546408in}}%
\pgfpathlineto{\pgfqpoint{3.344022in}{1.542150in}}%
\pgfpathmoveto{\pgfqpoint{3.344022in}{1.537892in}}%
\pgfpathlineto{\pgfqpoint{3.344022in}{1.537892in}}%
\pgfpathlineto{\pgfqpoint{3.344022in}{1.542150in}}%
\pgfpathlineto{\pgfqpoint{3.348280in}{1.542150in}}%
\pgfpathlineto{\pgfqpoint{3.348280in}{1.537892in}}%
\pgfpathmoveto{\pgfqpoint{3.331249in}{1.546408in}}%
\pgfpathlineto{\pgfqpoint{3.331249in}{1.546408in}}%
\pgfpathlineto{\pgfqpoint{3.331249in}{1.550666in}}%
\pgfpathlineto{\pgfqpoint{3.335507in}{1.550666in}}%
\pgfpathlineto{\pgfqpoint{3.335507in}{1.546408in}}%
\pgfpathmoveto{\pgfqpoint{3.331249in}{1.550666in}}%
\pgfpathlineto{\pgfqpoint{3.331249in}{1.550666in}}%
\pgfpathlineto{\pgfqpoint{3.331249in}{1.554924in}}%
\pgfpathlineto{\pgfqpoint{3.335507in}{1.554924in}}%
\pgfpathlineto{\pgfqpoint{3.335507in}{1.550666in}}%
\pgfpathmoveto{\pgfqpoint{3.335507in}{1.546408in}}%
\pgfpathlineto{\pgfqpoint{3.335507in}{1.546408in}}%
\pgfpathlineto{\pgfqpoint{3.335507in}{1.550666in}}%
\pgfpathlineto{\pgfqpoint{3.339764in}{1.550666in}}%
\pgfpathlineto{\pgfqpoint{3.339764in}{1.546408in}}%
\pgfpathmoveto{\pgfqpoint{3.335507in}{1.550666in}}%
\pgfpathlineto{\pgfqpoint{3.335507in}{1.550666in}}%
\pgfpathlineto{\pgfqpoint{3.335507in}{1.554924in}}%
\pgfpathlineto{\pgfqpoint{3.339764in}{1.554924in}}%
\pgfpathlineto{\pgfqpoint{3.339764in}{1.550666in}}%
\pgfpathmoveto{\pgfqpoint{3.331249in}{1.554924in}}%
\pgfpathlineto{\pgfqpoint{3.331249in}{1.554924in}}%
\pgfpathlineto{\pgfqpoint{3.331249in}{1.559182in}}%
\pgfpathlineto{\pgfqpoint{3.335507in}{1.559182in}}%
\pgfpathlineto{\pgfqpoint{3.335507in}{1.554924in}}%
\pgfpathmoveto{\pgfqpoint{3.331249in}{1.559182in}}%
\pgfpathlineto{\pgfqpoint{3.331249in}{1.559182in}}%
\pgfpathlineto{\pgfqpoint{3.331249in}{1.563440in}}%
\pgfpathlineto{\pgfqpoint{3.335507in}{1.563440in}}%
\pgfpathlineto{\pgfqpoint{3.335507in}{1.559182in}}%
\pgfpathmoveto{\pgfqpoint{3.335507in}{1.554924in}}%
\pgfpathlineto{\pgfqpoint{3.335507in}{1.554924in}}%
\pgfpathlineto{\pgfqpoint{3.335507in}{1.559182in}}%
\pgfpathlineto{\pgfqpoint{3.339764in}{1.559182in}}%
\pgfpathlineto{\pgfqpoint{3.339764in}{1.554924in}}%
\pgfpathmoveto{\pgfqpoint{3.339764in}{1.546408in}}%
\pgfpathlineto{\pgfqpoint{3.339764in}{1.546408in}}%
\pgfpathlineto{\pgfqpoint{3.339764in}{1.550666in}}%
\pgfpathlineto{\pgfqpoint{3.344022in}{1.550666in}}%
\pgfpathlineto{\pgfqpoint{3.344022in}{1.546408in}}%
\pgfpathmoveto{\pgfqpoint{3.348280in}{1.529376in}}%
\pgfpathlineto{\pgfqpoint{3.348280in}{1.529376in}}%
\pgfpathlineto{\pgfqpoint{3.348280in}{1.533634in}}%
\pgfpathlineto{\pgfqpoint{3.352537in}{1.533634in}}%
\pgfpathlineto{\pgfqpoint{3.352537in}{1.529376in}}%
\pgfpathmoveto{\pgfqpoint{3.331249in}{1.563440in}}%
\pgfpathlineto{\pgfqpoint{3.331249in}{1.563440in}}%
\pgfpathlineto{\pgfqpoint{3.331249in}{1.567698in}}%
\pgfpathlineto{\pgfqpoint{3.335507in}{1.567698in}}%
\pgfpathlineto{\pgfqpoint{3.335507in}{1.563440in}}%
\pgfpathmoveto{\pgfqpoint{3.292929in}{1.631566in}}%
\pgfpathlineto{\pgfqpoint{3.292929in}{1.631566in}}%
\pgfpathlineto{\pgfqpoint{3.292929in}{1.635823in}}%
\pgfpathlineto{\pgfqpoint{3.297187in}{1.635823in}}%
\pgfpathlineto{\pgfqpoint{3.297187in}{1.631566in}}%
\pgfpathmoveto{\pgfqpoint{3.292929in}{1.635823in}}%
\pgfpathlineto{\pgfqpoint{3.292929in}{1.635823in}}%
\pgfpathlineto{\pgfqpoint{3.292929in}{1.640081in}}%
\pgfpathlineto{\pgfqpoint{3.297187in}{1.640081in}}%
\pgfpathlineto{\pgfqpoint{3.297187in}{1.635823in}}%
\pgfpathmoveto{\pgfqpoint{3.288672in}{1.644339in}}%
\pgfpathlineto{\pgfqpoint{3.288672in}{1.644339in}}%
\pgfpathlineto{\pgfqpoint{3.288672in}{1.648597in}}%
\pgfpathlineto{\pgfqpoint{3.292929in}{1.648597in}}%
\pgfpathlineto{\pgfqpoint{3.292929in}{1.644339in}}%
\pgfpathmoveto{\pgfqpoint{3.292929in}{1.640081in}}%
\pgfpathlineto{\pgfqpoint{3.292929in}{1.640081in}}%
\pgfpathlineto{\pgfqpoint{3.292929in}{1.644339in}}%
\pgfpathlineto{\pgfqpoint{3.297187in}{1.644339in}}%
\pgfpathlineto{\pgfqpoint{3.297187in}{1.640081in}}%
\pgfpathmoveto{\pgfqpoint{3.292929in}{1.644339in}}%
\pgfpathlineto{\pgfqpoint{3.292929in}{1.644339in}}%
\pgfpathlineto{\pgfqpoint{3.292929in}{1.648597in}}%
\pgfpathlineto{\pgfqpoint{3.297187in}{1.648597in}}%
\pgfpathlineto{\pgfqpoint{3.297187in}{1.644339in}}%
\pgfpathmoveto{\pgfqpoint{3.284414in}{1.657112in}}%
\pgfpathlineto{\pgfqpoint{3.284414in}{1.657112in}}%
\pgfpathlineto{\pgfqpoint{3.284414in}{1.661370in}}%
\pgfpathlineto{\pgfqpoint{3.288672in}{1.661370in}}%
\pgfpathlineto{\pgfqpoint{3.288672in}{1.657112in}}%
\pgfpathmoveto{\pgfqpoint{3.284414in}{1.661370in}}%
\pgfpathlineto{\pgfqpoint{3.284414in}{1.661370in}}%
\pgfpathlineto{\pgfqpoint{3.284414in}{1.665627in}}%
\pgfpathlineto{\pgfqpoint{3.288672in}{1.665627in}}%
\pgfpathlineto{\pgfqpoint{3.288672in}{1.661370in}}%
\pgfpathmoveto{\pgfqpoint{3.288672in}{1.648597in}}%
\pgfpathlineto{\pgfqpoint{3.288672in}{1.648597in}}%
\pgfpathlineto{\pgfqpoint{3.288672in}{1.652854in}}%
\pgfpathlineto{\pgfqpoint{3.292929in}{1.652854in}}%
\pgfpathlineto{\pgfqpoint{3.292929in}{1.648597in}}%
\pgfpathmoveto{\pgfqpoint{3.288672in}{1.652854in}}%
\pgfpathlineto{\pgfqpoint{3.288672in}{1.652854in}}%
\pgfpathlineto{\pgfqpoint{3.288672in}{1.657112in}}%
\pgfpathlineto{\pgfqpoint{3.292929in}{1.657112in}}%
\pgfpathlineto{\pgfqpoint{3.292929in}{1.652854in}}%
\pgfpathmoveto{\pgfqpoint{3.292929in}{1.648597in}}%
\pgfpathlineto{\pgfqpoint{3.292929in}{1.648597in}}%
\pgfpathlineto{\pgfqpoint{3.292929in}{1.652854in}}%
\pgfpathlineto{\pgfqpoint{3.297187in}{1.652854in}}%
\pgfpathlineto{\pgfqpoint{3.297187in}{1.648597in}}%
\pgfpathmoveto{\pgfqpoint{3.292929in}{1.652854in}}%
\pgfpathlineto{\pgfqpoint{3.292929in}{1.652854in}}%
\pgfpathlineto{\pgfqpoint{3.292929in}{1.657112in}}%
\pgfpathlineto{\pgfqpoint{3.297187in}{1.657112in}}%
\pgfpathlineto{\pgfqpoint{3.297187in}{1.652854in}}%
\pgfpathmoveto{\pgfqpoint{3.288672in}{1.657112in}}%
\pgfpathlineto{\pgfqpoint{3.288672in}{1.657112in}}%
\pgfpathlineto{\pgfqpoint{3.288672in}{1.661370in}}%
\pgfpathlineto{\pgfqpoint{3.292929in}{1.661370in}}%
\pgfpathlineto{\pgfqpoint{3.292929in}{1.657112in}}%
\pgfpathmoveto{\pgfqpoint{3.288672in}{1.661370in}}%
\pgfpathlineto{\pgfqpoint{3.288672in}{1.661370in}}%
\pgfpathlineto{\pgfqpoint{3.288672in}{1.665627in}}%
\pgfpathlineto{\pgfqpoint{3.292929in}{1.665627in}}%
\pgfpathlineto{\pgfqpoint{3.292929in}{1.661370in}}%
\pgfpathmoveto{\pgfqpoint{3.275899in}{1.682658in}}%
\pgfpathlineto{\pgfqpoint{3.275899in}{1.682658in}}%
\pgfpathlineto{\pgfqpoint{3.275899in}{1.686916in}}%
\pgfpathlineto{\pgfqpoint{3.280156in}{1.686916in}}%
\pgfpathlineto{\pgfqpoint{3.280156in}{1.682658in}}%
\pgfpathmoveto{\pgfqpoint{3.275899in}{1.686916in}}%
\pgfpathlineto{\pgfqpoint{3.275899in}{1.686916in}}%
\pgfpathlineto{\pgfqpoint{3.275899in}{1.691174in}}%
\pgfpathlineto{\pgfqpoint{3.280156in}{1.691174in}}%
\pgfpathlineto{\pgfqpoint{3.280156in}{1.686916in}}%
\pgfpathmoveto{\pgfqpoint{3.271641in}{1.695431in}}%
\pgfpathlineto{\pgfqpoint{3.271641in}{1.695431in}}%
\pgfpathlineto{\pgfqpoint{3.271641in}{1.699689in}}%
\pgfpathlineto{\pgfqpoint{3.275899in}{1.699689in}}%
\pgfpathlineto{\pgfqpoint{3.275899in}{1.695431in}}%
\pgfpathmoveto{\pgfqpoint{3.275899in}{1.691174in}}%
\pgfpathlineto{\pgfqpoint{3.275899in}{1.691174in}}%
\pgfpathlineto{\pgfqpoint{3.275899in}{1.695431in}}%
\pgfpathlineto{\pgfqpoint{3.280156in}{1.695431in}}%
\pgfpathlineto{\pgfqpoint{3.280156in}{1.691174in}}%
\pgfpathmoveto{\pgfqpoint{3.275899in}{1.695431in}}%
\pgfpathlineto{\pgfqpoint{3.275899in}{1.695431in}}%
\pgfpathlineto{\pgfqpoint{3.275899in}{1.699689in}}%
\pgfpathlineto{\pgfqpoint{3.280156in}{1.699689in}}%
\pgfpathlineto{\pgfqpoint{3.280156in}{1.695431in}}%
\pgfpathmoveto{\pgfqpoint{3.280156in}{1.669885in}}%
\pgfpathlineto{\pgfqpoint{3.280156in}{1.669885in}}%
\pgfpathlineto{\pgfqpoint{3.280156in}{1.674143in}}%
\pgfpathlineto{\pgfqpoint{3.284414in}{1.674143in}}%
\pgfpathlineto{\pgfqpoint{3.284414in}{1.669885in}}%
\pgfpathmoveto{\pgfqpoint{3.284414in}{1.665627in}}%
\pgfpathlineto{\pgfqpoint{3.284414in}{1.665627in}}%
\pgfpathlineto{\pgfqpoint{3.284414in}{1.669885in}}%
\pgfpathlineto{\pgfqpoint{3.288672in}{1.669885in}}%
\pgfpathlineto{\pgfqpoint{3.288672in}{1.665627in}}%
\pgfpathmoveto{\pgfqpoint{3.284414in}{1.669885in}}%
\pgfpathlineto{\pgfqpoint{3.284414in}{1.669885in}}%
\pgfpathlineto{\pgfqpoint{3.284414in}{1.674143in}}%
\pgfpathlineto{\pgfqpoint{3.288672in}{1.674143in}}%
\pgfpathlineto{\pgfqpoint{3.288672in}{1.669885in}}%
\pgfpathmoveto{\pgfqpoint{3.280156in}{1.674143in}}%
\pgfpathlineto{\pgfqpoint{3.280156in}{1.674143in}}%
\pgfpathlineto{\pgfqpoint{3.280156in}{1.678401in}}%
\pgfpathlineto{\pgfqpoint{3.284414in}{1.678401in}}%
\pgfpathlineto{\pgfqpoint{3.284414in}{1.674143in}}%
\pgfpathmoveto{\pgfqpoint{3.280156in}{1.678401in}}%
\pgfpathlineto{\pgfqpoint{3.280156in}{1.678401in}}%
\pgfpathlineto{\pgfqpoint{3.280156in}{1.682658in}}%
\pgfpathlineto{\pgfqpoint{3.284414in}{1.682658in}}%
\pgfpathlineto{\pgfqpoint{3.284414in}{1.678401in}}%
\pgfpathmoveto{\pgfqpoint{3.284414in}{1.674143in}}%
\pgfpathlineto{\pgfqpoint{3.284414in}{1.674143in}}%
\pgfpathlineto{\pgfqpoint{3.284414in}{1.678401in}}%
\pgfpathlineto{\pgfqpoint{3.288672in}{1.678401in}}%
\pgfpathlineto{\pgfqpoint{3.288672in}{1.674143in}}%
\pgfpathmoveto{\pgfqpoint{3.284414in}{1.678401in}}%
\pgfpathlineto{\pgfqpoint{3.284414in}{1.678401in}}%
\pgfpathlineto{\pgfqpoint{3.284414in}{1.682658in}}%
\pgfpathlineto{\pgfqpoint{3.288672in}{1.682658in}}%
\pgfpathlineto{\pgfqpoint{3.288672in}{1.678401in}}%
\pgfpathmoveto{\pgfqpoint{3.288672in}{1.665627in}}%
\pgfpathlineto{\pgfqpoint{3.288672in}{1.665627in}}%
\pgfpathlineto{\pgfqpoint{3.288672in}{1.669885in}}%
\pgfpathlineto{\pgfqpoint{3.292929in}{1.669885in}}%
\pgfpathlineto{\pgfqpoint{3.292929in}{1.665627in}}%
\pgfpathmoveto{\pgfqpoint{3.280156in}{1.682658in}}%
\pgfpathlineto{\pgfqpoint{3.280156in}{1.682658in}}%
\pgfpathlineto{\pgfqpoint{3.280156in}{1.686916in}}%
\pgfpathlineto{\pgfqpoint{3.284414in}{1.686916in}}%
\pgfpathlineto{\pgfqpoint{3.284414in}{1.682658in}}%
\pgfpathmoveto{\pgfqpoint{3.280156in}{1.686916in}}%
\pgfpathlineto{\pgfqpoint{3.280156in}{1.686916in}}%
\pgfpathlineto{\pgfqpoint{3.280156in}{1.691174in}}%
\pgfpathlineto{\pgfqpoint{3.284414in}{1.691174in}}%
\pgfpathlineto{\pgfqpoint{3.284414in}{1.686916in}}%
\pgfpathmoveto{\pgfqpoint{3.280156in}{1.691174in}}%
\pgfpathlineto{\pgfqpoint{3.280156in}{1.691174in}}%
\pgfpathlineto{\pgfqpoint{3.280156in}{1.695431in}}%
\pgfpathlineto{\pgfqpoint{3.284414in}{1.695431in}}%
\pgfpathlineto{\pgfqpoint{3.284414in}{1.691174in}}%
\pgfpathmoveto{\pgfqpoint{3.267383in}{1.712462in}}%
\pgfpathlineto{\pgfqpoint{3.267383in}{1.712462in}}%
\pgfpathlineto{\pgfqpoint{3.267383in}{1.716720in}}%
\pgfpathlineto{\pgfqpoint{3.271641in}{1.716720in}}%
\pgfpathlineto{\pgfqpoint{3.271641in}{1.712462in}}%
\pgfpathmoveto{\pgfqpoint{3.271641in}{1.699689in}}%
\pgfpathlineto{\pgfqpoint{3.271641in}{1.699689in}}%
\pgfpathlineto{\pgfqpoint{3.271641in}{1.703947in}}%
\pgfpathlineto{\pgfqpoint{3.275899in}{1.703947in}}%
\pgfpathlineto{\pgfqpoint{3.275899in}{1.699689in}}%
\pgfpathmoveto{\pgfqpoint{3.271641in}{1.703947in}}%
\pgfpathlineto{\pgfqpoint{3.271641in}{1.703947in}}%
\pgfpathlineto{\pgfqpoint{3.271641in}{1.708205in}}%
\pgfpathlineto{\pgfqpoint{3.275899in}{1.708205in}}%
\pgfpathlineto{\pgfqpoint{3.275899in}{1.703947in}}%
\pgfpathmoveto{\pgfqpoint{3.275899in}{1.699689in}}%
\pgfpathlineto{\pgfqpoint{3.275899in}{1.699689in}}%
\pgfpathlineto{\pgfqpoint{3.275899in}{1.703947in}}%
\pgfpathlineto{\pgfqpoint{3.280156in}{1.703947in}}%
\pgfpathlineto{\pgfqpoint{3.280156in}{1.699689in}}%
\pgfpathmoveto{\pgfqpoint{3.275899in}{1.703947in}}%
\pgfpathlineto{\pgfqpoint{3.275899in}{1.703947in}}%
\pgfpathlineto{\pgfqpoint{3.275899in}{1.708205in}}%
\pgfpathlineto{\pgfqpoint{3.280156in}{1.708205in}}%
\pgfpathlineto{\pgfqpoint{3.280156in}{1.703947in}}%
\pgfpathmoveto{\pgfqpoint{3.271641in}{1.708205in}}%
\pgfpathlineto{\pgfqpoint{3.271641in}{1.708205in}}%
\pgfpathlineto{\pgfqpoint{3.271641in}{1.712462in}}%
\pgfpathlineto{\pgfqpoint{3.275899in}{1.712462in}}%
\pgfpathlineto{\pgfqpoint{3.275899in}{1.708205in}}%
\pgfpathmoveto{\pgfqpoint{3.271641in}{1.712462in}}%
\pgfpathlineto{\pgfqpoint{3.271641in}{1.712462in}}%
\pgfpathlineto{\pgfqpoint{3.271641in}{1.716720in}}%
\pgfpathlineto{\pgfqpoint{3.275899in}{1.716720in}}%
\pgfpathlineto{\pgfqpoint{3.275899in}{1.712462in}}%
\pgfpathmoveto{\pgfqpoint{3.267383in}{1.716720in}}%
\pgfpathlineto{\pgfqpoint{3.267383in}{1.716720in}}%
\pgfpathlineto{\pgfqpoint{3.267383in}{1.720978in}}%
\pgfpathlineto{\pgfqpoint{3.271641in}{1.720978in}}%
\pgfpathlineto{\pgfqpoint{3.271641in}{1.716720in}}%
\pgfpathmoveto{\pgfqpoint{3.267383in}{1.720978in}}%
\pgfpathlineto{\pgfqpoint{3.267383in}{1.720978in}}%
\pgfpathlineto{\pgfqpoint{3.267383in}{1.725235in}}%
\pgfpathlineto{\pgfqpoint{3.271641in}{1.725235in}}%
\pgfpathlineto{\pgfqpoint{3.271641in}{1.720978in}}%
\pgfpathmoveto{\pgfqpoint{3.263125in}{1.725235in}}%
\pgfpathlineto{\pgfqpoint{3.263125in}{1.725235in}}%
\pgfpathlineto{\pgfqpoint{3.263125in}{1.729493in}}%
\pgfpathlineto{\pgfqpoint{3.267383in}{1.729493in}}%
\pgfpathlineto{\pgfqpoint{3.267383in}{1.725235in}}%
\pgfpathmoveto{\pgfqpoint{3.263125in}{1.729493in}}%
\pgfpathlineto{\pgfqpoint{3.263125in}{1.729493in}}%
\pgfpathlineto{\pgfqpoint{3.263125in}{1.733751in}}%
\pgfpathlineto{\pgfqpoint{3.267383in}{1.733751in}}%
\pgfpathlineto{\pgfqpoint{3.267383in}{1.729493in}}%
\pgfpathmoveto{\pgfqpoint{3.267383in}{1.725235in}}%
\pgfpathlineto{\pgfqpoint{3.267383in}{1.725235in}}%
\pgfpathlineto{\pgfqpoint{3.267383in}{1.729493in}}%
\pgfpathlineto{\pgfqpoint{3.271641in}{1.729493in}}%
\pgfpathlineto{\pgfqpoint{3.271641in}{1.725235in}}%
\pgfpathmoveto{\pgfqpoint{3.267383in}{1.729493in}}%
\pgfpathlineto{\pgfqpoint{3.267383in}{1.729493in}}%
\pgfpathlineto{\pgfqpoint{3.267383in}{1.733751in}}%
\pgfpathlineto{\pgfqpoint{3.271641in}{1.733751in}}%
\pgfpathlineto{\pgfqpoint{3.271641in}{1.729493in}}%
\pgfpathmoveto{\pgfqpoint{3.271641in}{1.716720in}}%
\pgfpathlineto{\pgfqpoint{3.271641in}{1.716720in}}%
\pgfpathlineto{\pgfqpoint{3.271641in}{1.720978in}}%
\pgfpathlineto{\pgfqpoint{3.275899in}{1.720978in}}%
\pgfpathlineto{\pgfqpoint{3.275899in}{1.716720in}}%
\pgfpathmoveto{\pgfqpoint{3.301445in}{1.610277in}}%
\pgfpathlineto{\pgfqpoint{3.301445in}{1.610277in}}%
\pgfpathlineto{\pgfqpoint{3.301445in}{1.614535in}}%
\pgfpathlineto{\pgfqpoint{3.305703in}{1.614535in}}%
\pgfpathlineto{\pgfqpoint{3.305703in}{1.610277in}}%
\pgfpathmoveto{\pgfqpoint{3.305703in}{1.597504in}}%
\pgfpathlineto{\pgfqpoint{3.305703in}{1.597504in}}%
\pgfpathlineto{\pgfqpoint{3.305703in}{1.601762in}}%
\pgfpathlineto{\pgfqpoint{3.309960in}{1.601762in}}%
\pgfpathlineto{\pgfqpoint{3.309960in}{1.597504in}}%
\pgfpathmoveto{\pgfqpoint{3.305703in}{1.601762in}}%
\pgfpathlineto{\pgfqpoint{3.305703in}{1.601762in}}%
\pgfpathlineto{\pgfqpoint{3.305703in}{1.606019in}}%
\pgfpathlineto{\pgfqpoint{3.309960in}{1.606019in}}%
\pgfpathlineto{\pgfqpoint{3.309960in}{1.601762in}}%
\pgfpathmoveto{\pgfqpoint{3.309960in}{1.597504in}}%
\pgfpathlineto{\pgfqpoint{3.309960in}{1.597504in}}%
\pgfpathlineto{\pgfqpoint{3.309960in}{1.601762in}}%
\pgfpathlineto{\pgfqpoint{3.314218in}{1.601762in}}%
\pgfpathlineto{\pgfqpoint{3.314218in}{1.597504in}}%
\pgfpathmoveto{\pgfqpoint{3.309960in}{1.601762in}}%
\pgfpathlineto{\pgfqpoint{3.309960in}{1.601762in}}%
\pgfpathlineto{\pgfqpoint{3.309960in}{1.606019in}}%
\pgfpathlineto{\pgfqpoint{3.314218in}{1.606019in}}%
\pgfpathlineto{\pgfqpoint{3.314218in}{1.601762in}}%
\pgfpathmoveto{\pgfqpoint{3.305703in}{1.606019in}}%
\pgfpathlineto{\pgfqpoint{3.305703in}{1.606019in}}%
\pgfpathlineto{\pgfqpoint{3.305703in}{1.610277in}}%
\pgfpathlineto{\pgfqpoint{3.309960in}{1.610277in}}%
\pgfpathlineto{\pgfqpoint{3.309960in}{1.606019in}}%
\pgfpathmoveto{\pgfqpoint{3.305703in}{1.610277in}}%
\pgfpathlineto{\pgfqpoint{3.305703in}{1.610277in}}%
\pgfpathlineto{\pgfqpoint{3.305703in}{1.614535in}}%
\pgfpathlineto{\pgfqpoint{3.309960in}{1.614535in}}%
\pgfpathlineto{\pgfqpoint{3.309960in}{1.610277in}}%
\pgfpathmoveto{\pgfqpoint{3.309960in}{1.606019in}}%
\pgfpathlineto{\pgfqpoint{3.309960in}{1.606019in}}%
\pgfpathlineto{\pgfqpoint{3.309960in}{1.610277in}}%
\pgfpathlineto{\pgfqpoint{3.314218in}{1.610277in}}%
\pgfpathlineto{\pgfqpoint{3.314218in}{1.606019in}}%
\pgfpathmoveto{\pgfqpoint{3.309960in}{1.610277in}}%
\pgfpathlineto{\pgfqpoint{3.309960in}{1.610277in}}%
\pgfpathlineto{\pgfqpoint{3.309960in}{1.614535in}}%
\pgfpathlineto{\pgfqpoint{3.314218in}{1.614535in}}%
\pgfpathlineto{\pgfqpoint{3.314218in}{1.610277in}}%
\pgfpathmoveto{\pgfqpoint{3.297187in}{1.618793in}}%
\pgfpathlineto{\pgfqpoint{3.297187in}{1.618793in}}%
\pgfpathlineto{\pgfqpoint{3.297187in}{1.623050in}}%
\pgfpathlineto{\pgfqpoint{3.301445in}{1.623050in}}%
\pgfpathlineto{\pgfqpoint{3.301445in}{1.618793in}}%
\pgfpathmoveto{\pgfqpoint{3.301445in}{1.614535in}}%
\pgfpathlineto{\pgfqpoint{3.301445in}{1.614535in}}%
\pgfpathlineto{\pgfqpoint{3.301445in}{1.618793in}}%
\pgfpathlineto{\pgfqpoint{3.305703in}{1.618793in}}%
\pgfpathlineto{\pgfqpoint{3.305703in}{1.614535in}}%
\pgfpathmoveto{\pgfqpoint{3.301445in}{1.618793in}}%
\pgfpathlineto{\pgfqpoint{3.301445in}{1.618793in}}%
\pgfpathlineto{\pgfqpoint{3.301445in}{1.623050in}}%
\pgfpathlineto{\pgfqpoint{3.305703in}{1.623050in}}%
\pgfpathlineto{\pgfqpoint{3.305703in}{1.618793in}}%
\pgfpathmoveto{\pgfqpoint{3.297187in}{1.623050in}}%
\pgfpathlineto{\pgfqpoint{3.297187in}{1.623050in}}%
\pgfpathlineto{\pgfqpoint{3.297187in}{1.627308in}}%
\pgfpathlineto{\pgfqpoint{3.301445in}{1.627308in}}%
\pgfpathlineto{\pgfqpoint{3.301445in}{1.623050in}}%
\pgfpathmoveto{\pgfqpoint{3.297187in}{1.627308in}}%
\pgfpathlineto{\pgfqpoint{3.297187in}{1.627308in}}%
\pgfpathlineto{\pgfqpoint{3.297187in}{1.631566in}}%
\pgfpathlineto{\pgfqpoint{3.301445in}{1.631566in}}%
\pgfpathlineto{\pgfqpoint{3.301445in}{1.627308in}}%
\pgfpathmoveto{\pgfqpoint{3.301445in}{1.623050in}}%
\pgfpathlineto{\pgfqpoint{3.301445in}{1.623050in}}%
\pgfpathlineto{\pgfqpoint{3.301445in}{1.627308in}}%
\pgfpathlineto{\pgfqpoint{3.305703in}{1.627308in}}%
\pgfpathlineto{\pgfqpoint{3.305703in}{1.623050in}}%
\pgfpathmoveto{\pgfqpoint{3.301445in}{1.627308in}}%
\pgfpathlineto{\pgfqpoint{3.301445in}{1.627308in}}%
\pgfpathlineto{\pgfqpoint{3.301445in}{1.631566in}}%
\pgfpathlineto{\pgfqpoint{3.305703in}{1.631566in}}%
\pgfpathlineto{\pgfqpoint{3.305703in}{1.627308in}}%
\pgfpathmoveto{\pgfqpoint{3.305703in}{1.614535in}}%
\pgfpathlineto{\pgfqpoint{3.305703in}{1.614535in}}%
\pgfpathlineto{\pgfqpoint{3.305703in}{1.618793in}}%
\pgfpathlineto{\pgfqpoint{3.309960in}{1.618793in}}%
\pgfpathlineto{\pgfqpoint{3.309960in}{1.614535in}}%
\pgfpathmoveto{\pgfqpoint{3.305703in}{1.618793in}}%
\pgfpathlineto{\pgfqpoint{3.305703in}{1.618793in}}%
\pgfpathlineto{\pgfqpoint{3.305703in}{1.623050in}}%
\pgfpathlineto{\pgfqpoint{3.309960in}{1.623050in}}%
\pgfpathlineto{\pgfqpoint{3.309960in}{1.618793in}}%
\pgfpathmoveto{\pgfqpoint{3.305703in}{1.623050in}}%
\pgfpathlineto{\pgfqpoint{3.305703in}{1.623050in}}%
\pgfpathlineto{\pgfqpoint{3.305703in}{1.627308in}}%
\pgfpathlineto{\pgfqpoint{3.309960in}{1.627308in}}%
\pgfpathlineto{\pgfqpoint{3.309960in}{1.623050in}}%
\pgfpathmoveto{\pgfqpoint{3.314218in}{1.597504in}}%
\pgfpathlineto{\pgfqpoint{3.314218in}{1.597504in}}%
\pgfpathlineto{\pgfqpoint{3.314218in}{1.601762in}}%
\pgfpathlineto{\pgfqpoint{3.318476in}{1.601762in}}%
\pgfpathlineto{\pgfqpoint{3.318476in}{1.597504in}}%
\pgfpathmoveto{\pgfqpoint{3.314218in}{1.601762in}}%
\pgfpathlineto{\pgfqpoint{3.314218in}{1.601762in}}%
\pgfpathlineto{\pgfqpoint{3.314218in}{1.606019in}}%
\pgfpathlineto{\pgfqpoint{3.318476in}{1.606019in}}%
\pgfpathlineto{\pgfqpoint{3.318476in}{1.601762in}}%
\pgfpathmoveto{\pgfqpoint{3.297187in}{1.631566in}}%
\pgfpathlineto{\pgfqpoint{3.297187in}{1.631566in}}%
\pgfpathlineto{\pgfqpoint{3.297187in}{1.635823in}}%
\pgfpathlineto{\pgfqpoint{3.301445in}{1.635823in}}%
\pgfpathlineto{\pgfqpoint{3.301445in}{1.631566in}}%
\pgfpathmoveto{\pgfqpoint{3.297187in}{1.635823in}}%
\pgfpathlineto{\pgfqpoint{3.297187in}{1.635823in}}%
\pgfpathlineto{\pgfqpoint{3.297187in}{1.640081in}}%
\pgfpathlineto{\pgfqpoint{3.301445in}{1.640081in}}%
\pgfpathlineto{\pgfqpoint{3.301445in}{1.635823in}}%
\pgfpathmoveto{\pgfqpoint{3.301445in}{1.631566in}}%
\pgfpathlineto{\pgfqpoint{3.301445in}{1.631566in}}%
\pgfpathlineto{\pgfqpoint{3.301445in}{1.635823in}}%
\pgfpathlineto{\pgfqpoint{3.305703in}{1.635823in}}%
\pgfpathlineto{\pgfqpoint{3.305703in}{1.631566in}}%
\pgfpathmoveto{\pgfqpoint{3.297187in}{1.640081in}}%
\pgfpathlineto{\pgfqpoint{3.297187in}{1.640081in}}%
\pgfpathlineto{\pgfqpoint{3.297187in}{1.644339in}}%
\pgfpathlineto{\pgfqpoint{3.301445in}{1.644339in}}%
\pgfpathlineto{\pgfqpoint{3.301445in}{1.640081in}}%
\pgfpathmoveto{\pgfqpoint{3.297187in}{1.644339in}}%
\pgfpathlineto{\pgfqpoint{3.297187in}{1.644339in}}%
\pgfpathlineto{\pgfqpoint{3.297187in}{1.648597in}}%
\pgfpathlineto{\pgfqpoint{3.301445in}{1.648597in}}%
\pgfpathlineto{\pgfqpoint{3.301445in}{1.644339in}}%
\pgfpathmoveto{\pgfqpoint{3.258868in}{1.738009in}}%
\pgfpathlineto{\pgfqpoint{3.258868in}{1.738009in}}%
\pgfpathlineto{\pgfqpoint{3.258868in}{1.742266in}}%
\pgfpathlineto{\pgfqpoint{3.263125in}{1.742266in}}%
\pgfpathlineto{\pgfqpoint{3.263125in}{1.738009in}}%
\pgfpathmoveto{\pgfqpoint{3.258868in}{1.742266in}}%
\pgfpathlineto{\pgfqpoint{3.258868in}{1.742266in}}%
\pgfpathlineto{\pgfqpoint{3.258868in}{1.746524in}}%
\pgfpathlineto{\pgfqpoint{3.263125in}{1.746524in}}%
\pgfpathlineto{\pgfqpoint{3.263125in}{1.742266in}}%
\pgfpathmoveto{\pgfqpoint{3.258868in}{1.746524in}}%
\pgfpathlineto{\pgfqpoint{3.258868in}{1.746524in}}%
\pgfpathlineto{\pgfqpoint{3.258868in}{1.750782in}}%
\pgfpathlineto{\pgfqpoint{3.263125in}{1.750782in}}%
\pgfpathlineto{\pgfqpoint{3.263125in}{1.746524in}}%
\pgfpathmoveto{\pgfqpoint{3.254610in}{1.755039in}}%
\pgfpathlineto{\pgfqpoint{3.254610in}{1.755039in}}%
\pgfpathlineto{\pgfqpoint{3.254610in}{1.759297in}}%
\pgfpathlineto{\pgfqpoint{3.258868in}{1.759297in}}%
\pgfpathlineto{\pgfqpoint{3.258868in}{1.755039in}}%
\pgfpathmoveto{\pgfqpoint{3.258868in}{1.750782in}}%
\pgfpathlineto{\pgfqpoint{3.258868in}{1.750782in}}%
\pgfpathlineto{\pgfqpoint{3.258868in}{1.755039in}}%
\pgfpathlineto{\pgfqpoint{3.263125in}{1.755039in}}%
\pgfpathlineto{\pgfqpoint{3.263125in}{1.750782in}}%
\pgfpathmoveto{\pgfqpoint{3.258868in}{1.755039in}}%
\pgfpathlineto{\pgfqpoint{3.258868in}{1.755039in}}%
\pgfpathlineto{\pgfqpoint{3.258868in}{1.759297in}}%
\pgfpathlineto{\pgfqpoint{3.263125in}{1.759297in}}%
\pgfpathlineto{\pgfqpoint{3.263125in}{1.755039in}}%
\pgfpathmoveto{\pgfqpoint{3.254610in}{1.759297in}}%
\pgfpathlineto{\pgfqpoint{3.254610in}{1.759297in}}%
\pgfpathlineto{\pgfqpoint{3.254610in}{1.763555in}}%
\pgfpathlineto{\pgfqpoint{3.258868in}{1.763555in}}%
\pgfpathlineto{\pgfqpoint{3.258868in}{1.759297in}}%
\pgfpathmoveto{\pgfqpoint{3.254610in}{1.763555in}}%
\pgfpathlineto{\pgfqpoint{3.254610in}{1.763555in}}%
\pgfpathlineto{\pgfqpoint{3.254610in}{1.767813in}}%
\pgfpathlineto{\pgfqpoint{3.258868in}{1.767813in}}%
\pgfpathlineto{\pgfqpoint{3.258868in}{1.763555in}}%
\pgfpathmoveto{\pgfqpoint{3.258868in}{1.759297in}}%
\pgfpathlineto{\pgfqpoint{3.258868in}{1.759297in}}%
\pgfpathlineto{\pgfqpoint{3.258868in}{1.763555in}}%
\pgfpathlineto{\pgfqpoint{3.263125in}{1.763555in}}%
\pgfpathlineto{\pgfqpoint{3.263125in}{1.759297in}}%
\pgfpathmoveto{\pgfqpoint{3.250352in}{1.767813in}}%
\pgfpathlineto{\pgfqpoint{3.250352in}{1.767813in}}%
\pgfpathlineto{\pgfqpoint{3.250352in}{1.772070in}}%
\pgfpathlineto{\pgfqpoint{3.254610in}{1.772070in}}%
\pgfpathlineto{\pgfqpoint{3.254610in}{1.767813in}}%
\pgfpathmoveto{\pgfqpoint{3.250352in}{1.772070in}}%
\pgfpathlineto{\pgfqpoint{3.250352in}{1.772070in}}%
\pgfpathlineto{\pgfqpoint{3.250352in}{1.776328in}}%
\pgfpathlineto{\pgfqpoint{3.254610in}{1.776328in}}%
\pgfpathlineto{\pgfqpoint{3.254610in}{1.772070in}}%
\pgfpathmoveto{\pgfqpoint{3.250352in}{1.776328in}}%
\pgfpathlineto{\pgfqpoint{3.250352in}{1.776328in}}%
\pgfpathlineto{\pgfqpoint{3.250352in}{1.780586in}}%
\pgfpathlineto{\pgfqpoint{3.254610in}{1.780586in}}%
\pgfpathlineto{\pgfqpoint{3.254610in}{1.776328in}}%
\pgfpathmoveto{\pgfqpoint{3.250352in}{1.780586in}}%
\pgfpathlineto{\pgfqpoint{3.250352in}{1.780586in}}%
\pgfpathlineto{\pgfqpoint{3.250352in}{1.784843in}}%
\pgfpathlineto{\pgfqpoint{3.254610in}{1.784843in}}%
\pgfpathlineto{\pgfqpoint{3.254610in}{1.780586in}}%
\pgfpathmoveto{\pgfqpoint{3.254610in}{1.767813in}}%
\pgfpathlineto{\pgfqpoint{3.254610in}{1.767813in}}%
\pgfpathlineto{\pgfqpoint{3.254610in}{1.772070in}}%
\pgfpathlineto{\pgfqpoint{3.258868in}{1.772070in}}%
\pgfpathlineto{\pgfqpoint{3.258868in}{1.767813in}}%
\pgfpathmoveto{\pgfqpoint{3.254610in}{1.772070in}}%
\pgfpathlineto{\pgfqpoint{3.254610in}{1.772070in}}%
\pgfpathlineto{\pgfqpoint{3.254610in}{1.776328in}}%
\pgfpathlineto{\pgfqpoint{3.258868in}{1.776328in}}%
\pgfpathlineto{\pgfqpoint{3.258868in}{1.772070in}}%
\pgfpathmoveto{\pgfqpoint{3.246095in}{1.784843in}}%
\pgfpathlineto{\pgfqpoint{3.246095in}{1.784843in}}%
\pgfpathlineto{\pgfqpoint{3.246095in}{1.789101in}}%
\pgfpathlineto{\pgfqpoint{3.250352in}{1.789101in}}%
\pgfpathlineto{\pgfqpoint{3.250352in}{1.784843in}}%
\pgfpathmoveto{\pgfqpoint{3.246095in}{1.789101in}}%
\pgfpathlineto{\pgfqpoint{3.246095in}{1.789101in}}%
\pgfpathlineto{\pgfqpoint{3.246095in}{1.793359in}}%
\pgfpathlineto{\pgfqpoint{3.250352in}{1.793359in}}%
\pgfpathlineto{\pgfqpoint{3.250352in}{1.789101in}}%
\pgfpathmoveto{\pgfqpoint{3.250352in}{1.784843in}}%
\pgfpathlineto{\pgfqpoint{3.250352in}{1.784843in}}%
\pgfpathlineto{\pgfqpoint{3.250352in}{1.789101in}}%
\pgfpathlineto{\pgfqpoint{3.254610in}{1.789101in}}%
\pgfpathlineto{\pgfqpoint{3.254610in}{1.784843in}}%
\pgfpathmoveto{\pgfqpoint{3.250352in}{1.789101in}}%
\pgfpathlineto{\pgfqpoint{3.250352in}{1.789101in}}%
\pgfpathlineto{\pgfqpoint{3.250352in}{1.793359in}}%
\pgfpathlineto{\pgfqpoint{3.254610in}{1.793359in}}%
\pgfpathlineto{\pgfqpoint{3.254610in}{1.789101in}}%
\pgfpathmoveto{\pgfqpoint{3.246095in}{1.793359in}}%
\pgfpathlineto{\pgfqpoint{3.246095in}{1.793359in}}%
\pgfpathlineto{\pgfqpoint{3.246095in}{1.797617in}}%
\pgfpathlineto{\pgfqpoint{3.250352in}{1.797617in}}%
\pgfpathlineto{\pgfqpoint{3.250352in}{1.793359in}}%
\pgfpathmoveto{\pgfqpoint{3.246095in}{1.797617in}}%
\pgfpathlineto{\pgfqpoint{3.246095in}{1.797617in}}%
\pgfpathlineto{\pgfqpoint{3.246095in}{1.801874in}}%
\pgfpathlineto{\pgfqpoint{3.250352in}{1.801874in}}%
\pgfpathlineto{\pgfqpoint{3.250352in}{1.797617in}}%
\pgfpathmoveto{\pgfqpoint{3.263125in}{1.733751in}}%
\pgfpathlineto{\pgfqpoint{3.263125in}{1.733751in}}%
\pgfpathlineto{\pgfqpoint{3.263125in}{1.738009in}}%
\pgfpathlineto{\pgfqpoint{3.267383in}{1.738009in}}%
\pgfpathlineto{\pgfqpoint{3.267383in}{1.733751in}}%
\pgfpathmoveto{\pgfqpoint{3.263125in}{1.738009in}}%
\pgfpathlineto{\pgfqpoint{3.263125in}{1.738009in}}%
\pgfpathlineto{\pgfqpoint{3.263125in}{1.742266in}}%
\pgfpathlineto{\pgfqpoint{3.267383in}{1.742266in}}%
\pgfpathlineto{\pgfqpoint{3.267383in}{1.738009in}}%
\pgfpathmoveto{\pgfqpoint{3.263125in}{1.742266in}}%
\pgfpathlineto{\pgfqpoint{3.263125in}{1.742266in}}%
\pgfpathlineto{\pgfqpoint{3.263125in}{1.746524in}}%
\pgfpathlineto{\pgfqpoint{3.267383in}{1.746524in}}%
\pgfpathlineto{\pgfqpoint{3.267383in}{1.742266in}}%
\pgfpathmoveto{\pgfqpoint{3.263125in}{1.746524in}}%
\pgfpathlineto{\pgfqpoint{3.263125in}{1.746524in}}%
\pgfpathlineto{\pgfqpoint{3.263125in}{1.750782in}}%
\pgfpathlineto{\pgfqpoint{3.267383in}{1.750782in}}%
\pgfpathlineto{\pgfqpoint{3.267383in}{1.746524in}}%
\pgfpathmoveto{\pgfqpoint{3.241837in}{1.801874in}}%
\pgfpathlineto{\pgfqpoint{3.241837in}{1.801874in}}%
\pgfpathlineto{\pgfqpoint{3.241837in}{1.806132in}}%
\pgfpathlineto{\pgfqpoint{3.246095in}{1.806132in}}%
\pgfpathlineto{\pgfqpoint{3.246095in}{1.801874in}}%
\pgfpathmoveto{\pgfqpoint{3.241837in}{1.806132in}}%
\pgfpathlineto{\pgfqpoint{3.241837in}{1.806132in}}%
\pgfpathlineto{\pgfqpoint{3.241837in}{1.810390in}}%
\pgfpathlineto{\pgfqpoint{3.246095in}{1.810390in}}%
\pgfpathlineto{\pgfqpoint{3.246095in}{1.806132in}}%
\pgfpathmoveto{\pgfqpoint{3.241837in}{1.810390in}}%
\pgfpathlineto{\pgfqpoint{3.241837in}{1.810390in}}%
\pgfpathlineto{\pgfqpoint{3.241837in}{1.814647in}}%
\pgfpathlineto{\pgfqpoint{3.246095in}{1.814647in}}%
\pgfpathlineto{\pgfqpoint{3.246095in}{1.810390in}}%
\pgfpathmoveto{\pgfqpoint{3.241837in}{1.814647in}}%
\pgfpathlineto{\pgfqpoint{3.241837in}{1.814647in}}%
\pgfpathlineto{\pgfqpoint{3.241837in}{1.818905in}}%
\pgfpathlineto{\pgfqpoint{3.246095in}{1.818905in}}%
\pgfpathlineto{\pgfqpoint{3.246095in}{1.814647in}}%
\pgfpathmoveto{\pgfqpoint{3.237579in}{1.818905in}}%
\pgfpathlineto{\pgfqpoint{3.237579in}{1.818905in}}%
\pgfpathlineto{\pgfqpoint{3.237579in}{1.823163in}}%
\pgfpathlineto{\pgfqpoint{3.241837in}{1.823163in}}%
\pgfpathlineto{\pgfqpoint{3.241837in}{1.818905in}}%
\pgfpathmoveto{\pgfqpoint{3.237579in}{1.823163in}}%
\pgfpathlineto{\pgfqpoint{3.237579in}{1.823163in}}%
\pgfpathlineto{\pgfqpoint{3.237579in}{1.827420in}}%
\pgfpathlineto{\pgfqpoint{3.241837in}{1.827420in}}%
\pgfpathlineto{\pgfqpoint{3.241837in}{1.823163in}}%
\pgfpathmoveto{\pgfqpoint{3.241837in}{1.818905in}}%
\pgfpathlineto{\pgfqpoint{3.241837in}{1.818905in}}%
\pgfpathlineto{\pgfqpoint{3.241837in}{1.823163in}}%
\pgfpathlineto{\pgfqpoint{3.246095in}{1.823163in}}%
\pgfpathlineto{\pgfqpoint{3.246095in}{1.818905in}}%
\pgfpathmoveto{\pgfqpoint{3.237579in}{1.827420in}}%
\pgfpathlineto{\pgfqpoint{3.237579in}{1.827420in}}%
\pgfpathlineto{\pgfqpoint{3.237579in}{1.831678in}}%
\pgfpathlineto{\pgfqpoint{3.241837in}{1.831678in}}%
\pgfpathlineto{\pgfqpoint{3.241837in}{1.827420in}}%
\pgfpathmoveto{\pgfqpoint{3.237579in}{1.831678in}}%
\pgfpathlineto{\pgfqpoint{3.237579in}{1.831678in}}%
\pgfpathlineto{\pgfqpoint{3.237579in}{1.835936in}}%
\pgfpathlineto{\pgfqpoint{3.241837in}{1.835936in}}%
\pgfpathlineto{\pgfqpoint{3.241837in}{1.831678in}}%
\pgfpathmoveto{\pgfqpoint{3.246095in}{1.801874in}}%
\pgfpathlineto{\pgfqpoint{3.246095in}{1.801874in}}%
\pgfpathlineto{\pgfqpoint{3.246095in}{1.806132in}}%
\pgfpathlineto{\pgfqpoint{3.250352in}{1.806132in}}%
\pgfpathlineto{\pgfqpoint{3.250352in}{1.801874in}}%
\pgfpathmoveto{\pgfqpoint{3.233322in}{1.835936in}}%
\pgfpathlineto{\pgfqpoint{3.233322in}{1.835936in}}%
\pgfpathlineto{\pgfqpoint{3.233322in}{1.840194in}}%
\pgfpathlineto{\pgfqpoint{3.237579in}{1.840194in}}%
\pgfpathlineto{\pgfqpoint{3.237579in}{1.835936in}}%
\pgfpathmoveto{\pgfqpoint{3.233322in}{1.840194in}}%
\pgfpathlineto{\pgfqpoint{3.233322in}{1.840194in}}%
\pgfpathlineto{\pgfqpoint{3.233322in}{1.844451in}}%
\pgfpathlineto{\pgfqpoint{3.237579in}{1.844451in}}%
\pgfpathlineto{\pgfqpoint{3.237579in}{1.840194in}}%
\pgfpathmoveto{\pgfqpoint{3.229064in}{1.848709in}}%
\pgfpathlineto{\pgfqpoint{3.229064in}{1.848709in}}%
\pgfpathlineto{\pgfqpoint{3.229064in}{1.852967in}}%
\pgfpathlineto{\pgfqpoint{3.233322in}{1.852967in}}%
\pgfpathlineto{\pgfqpoint{3.233322in}{1.848709in}}%
\pgfpathmoveto{\pgfqpoint{3.233322in}{1.844451in}}%
\pgfpathlineto{\pgfqpoint{3.233322in}{1.844451in}}%
\pgfpathlineto{\pgfqpoint{3.233322in}{1.848709in}}%
\pgfpathlineto{\pgfqpoint{3.237579in}{1.848709in}}%
\pgfpathlineto{\pgfqpoint{3.237579in}{1.844451in}}%
\pgfpathmoveto{\pgfqpoint{3.233322in}{1.848709in}}%
\pgfpathlineto{\pgfqpoint{3.233322in}{1.848709in}}%
\pgfpathlineto{\pgfqpoint{3.233322in}{1.852967in}}%
\pgfpathlineto{\pgfqpoint{3.237579in}{1.852967in}}%
\pgfpathlineto{\pgfqpoint{3.237579in}{1.848709in}}%
\pgfpathmoveto{\pgfqpoint{3.237579in}{1.835936in}}%
\pgfpathlineto{\pgfqpoint{3.237579in}{1.835936in}}%
\pgfpathlineto{\pgfqpoint{3.237579in}{1.840194in}}%
\pgfpathlineto{\pgfqpoint{3.241837in}{1.840194in}}%
\pgfpathlineto{\pgfqpoint{3.241837in}{1.835936in}}%
\pgfpathmoveto{\pgfqpoint{3.229064in}{1.852967in}}%
\pgfpathlineto{\pgfqpoint{3.229064in}{1.852967in}}%
\pgfpathlineto{\pgfqpoint{3.229064in}{1.857224in}}%
\pgfpathlineto{\pgfqpoint{3.233322in}{1.857224in}}%
\pgfpathlineto{\pgfqpoint{3.233322in}{1.852967in}}%
\pgfpathmoveto{\pgfqpoint{3.229064in}{1.857224in}}%
\pgfpathlineto{\pgfqpoint{3.229064in}{1.857224in}}%
\pgfpathlineto{\pgfqpoint{3.229064in}{1.861482in}}%
\pgfpathlineto{\pgfqpoint{3.233322in}{1.861482in}}%
\pgfpathlineto{\pgfqpoint{3.233322in}{1.857224in}}%
\pgfpathmoveto{\pgfqpoint{3.233322in}{1.852967in}}%
\pgfpathlineto{\pgfqpoint{3.233322in}{1.852967in}}%
\pgfpathlineto{\pgfqpoint{3.233322in}{1.857224in}}%
\pgfpathlineto{\pgfqpoint{3.237579in}{1.857224in}}%
\pgfpathlineto{\pgfqpoint{3.237579in}{1.852967in}}%
\pgfpathmoveto{\pgfqpoint{3.229064in}{1.861482in}}%
\pgfpathlineto{\pgfqpoint{3.229064in}{1.861482in}}%
\pgfpathlineto{\pgfqpoint{3.229064in}{1.865740in}}%
\pgfpathlineto{\pgfqpoint{3.233322in}{1.865740in}}%
\pgfpathlineto{\pgfqpoint{3.233322in}{1.861482in}}%
\pgfpathmoveto{\pgfqpoint{3.229064in}{1.865740in}}%
\pgfpathlineto{\pgfqpoint{3.229064in}{1.865740in}}%
\pgfpathlineto{\pgfqpoint{3.229064in}{1.869998in}}%
\pgfpathlineto{\pgfqpoint{3.233322in}{1.869998in}}%
\pgfpathlineto{\pgfqpoint{3.233322in}{1.865740in}}%
\pgfpathmoveto{\pgfqpoint{3.229064in}{1.869998in}}%
\pgfpathlineto{\pgfqpoint{3.229064in}{1.869998in}}%
\pgfpathlineto{\pgfqpoint{3.229064in}{1.874255in}}%
\pgfpathlineto{\pgfqpoint{3.233322in}{1.874255in}}%
\pgfpathlineto{\pgfqpoint{3.233322in}{1.869998in}}%
\pgfpathmoveto{\pgfqpoint{3.378084in}{1.456990in}}%
\pgfpathlineto{\pgfqpoint{3.378084in}{1.456990in}}%
\pgfpathlineto{\pgfqpoint{3.378084in}{1.461248in}}%
\pgfpathlineto{\pgfqpoint{3.382342in}{1.461248in}}%
\pgfpathlineto{\pgfqpoint{3.382342in}{1.456990in}}%
\pgfpathmoveto{\pgfqpoint{3.395115in}{1.439958in}}%
\pgfpathlineto{\pgfqpoint{3.395115in}{1.439958in}}%
\pgfpathlineto{\pgfqpoint{3.395115in}{1.444216in}}%
\pgfpathlineto{\pgfqpoint{3.399373in}{1.444216in}}%
\pgfpathlineto{\pgfqpoint{3.399373in}{1.439958in}}%
\pgfpathmoveto{\pgfqpoint{3.386600in}{1.448474in}}%
\pgfpathlineto{\pgfqpoint{3.386600in}{1.448474in}}%
\pgfpathlineto{\pgfqpoint{3.386600in}{1.452732in}}%
\pgfpathlineto{\pgfqpoint{3.390858in}{1.452732in}}%
\pgfpathlineto{\pgfqpoint{3.390858in}{1.448474in}}%
\pgfpathmoveto{\pgfqpoint{3.382342in}{1.452732in}}%
\pgfpathlineto{\pgfqpoint{3.382342in}{1.452732in}}%
\pgfpathlineto{\pgfqpoint{3.382342in}{1.456990in}}%
\pgfpathlineto{\pgfqpoint{3.386600in}{1.456990in}}%
\pgfpathlineto{\pgfqpoint{3.386600in}{1.452732in}}%
\pgfpathmoveto{\pgfqpoint{3.382342in}{1.456990in}}%
\pgfpathlineto{\pgfqpoint{3.382342in}{1.456990in}}%
\pgfpathlineto{\pgfqpoint{3.382342in}{1.461248in}}%
\pgfpathlineto{\pgfqpoint{3.386600in}{1.461248in}}%
\pgfpathlineto{\pgfqpoint{3.386600in}{1.456990in}}%
\pgfpathmoveto{\pgfqpoint{3.386600in}{1.452732in}}%
\pgfpathlineto{\pgfqpoint{3.386600in}{1.452732in}}%
\pgfpathlineto{\pgfqpoint{3.386600in}{1.456990in}}%
\pgfpathlineto{\pgfqpoint{3.390858in}{1.456990in}}%
\pgfpathlineto{\pgfqpoint{3.390858in}{1.452732in}}%
\pgfpathmoveto{\pgfqpoint{3.386600in}{1.456990in}}%
\pgfpathlineto{\pgfqpoint{3.386600in}{1.456990in}}%
\pgfpathlineto{\pgfqpoint{3.386600in}{1.461248in}}%
\pgfpathlineto{\pgfqpoint{3.390858in}{1.461248in}}%
\pgfpathlineto{\pgfqpoint{3.390858in}{1.456990in}}%
\pgfpathmoveto{\pgfqpoint{3.390858in}{1.444216in}}%
\pgfpathlineto{\pgfqpoint{3.390858in}{1.444216in}}%
\pgfpathlineto{\pgfqpoint{3.390858in}{1.448474in}}%
\pgfpathlineto{\pgfqpoint{3.395115in}{1.448474in}}%
\pgfpathlineto{\pgfqpoint{3.395115in}{1.444216in}}%
\pgfpathmoveto{\pgfqpoint{3.390858in}{1.448474in}}%
\pgfpathlineto{\pgfqpoint{3.390858in}{1.448474in}}%
\pgfpathlineto{\pgfqpoint{3.390858in}{1.452732in}}%
\pgfpathlineto{\pgfqpoint{3.395115in}{1.452732in}}%
\pgfpathlineto{\pgfqpoint{3.395115in}{1.448474in}}%
\pgfpathmoveto{\pgfqpoint{3.395115in}{1.444216in}}%
\pgfpathlineto{\pgfqpoint{3.395115in}{1.444216in}}%
\pgfpathlineto{\pgfqpoint{3.395115in}{1.448474in}}%
\pgfpathlineto{\pgfqpoint{3.399373in}{1.448474in}}%
\pgfpathlineto{\pgfqpoint{3.399373in}{1.444216in}}%
\pgfpathmoveto{\pgfqpoint{3.395115in}{1.448474in}}%
\pgfpathlineto{\pgfqpoint{3.395115in}{1.448474in}}%
\pgfpathlineto{\pgfqpoint{3.395115in}{1.452732in}}%
\pgfpathlineto{\pgfqpoint{3.399373in}{1.452732in}}%
\pgfpathlineto{\pgfqpoint{3.399373in}{1.448474in}}%
\pgfpathmoveto{\pgfqpoint{3.390858in}{1.452732in}}%
\pgfpathlineto{\pgfqpoint{3.390858in}{1.452732in}}%
\pgfpathlineto{\pgfqpoint{3.390858in}{1.456990in}}%
\pgfpathlineto{\pgfqpoint{3.395115in}{1.456990in}}%
\pgfpathlineto{\pgfqpoint{3.395115in}{1.452732in}}%
\pgfpathmoveto{\pgfqpoint{3.390858in}{1.456990in}}%
\pgfpathlineto{\pgfqpoint{3.390858in}{1.456990in}}%
\pgfpathlineto{\pgfqpoint{3.390858in}{1.461248in}}%
\pgfpathlineto{\pgfqpoint{3.395115in}{1.461248in}}%
\pgfpathlineto{\pgfqpoint{3.395115in}{1.456990in}}%
\pgfpathmoveto{\pgfqpoint{3.395115in}{1.452732in}}%
\pgfpathlineto{\pgfqpoint{3.395115in}{1.452732in}}%
\pgfpathlineto{\pgfqpoint{3.395115in}{1.456990in}}%
\pgfpathlineto{\pgfqpoint{3.399373in}{1.456990in}}%
\pgfpathlineto{\pgfqpoint{3.399373in}{1.452732in}}%
\pgfpathmoveto{\pgfqpoint{3.395115in}{1.456990in}}%
\pgfpathlineto{\pgfqpoint{3.395115in}{1.456990in}}%
\pgfpathlineto{\pgfqpoint{3.395115in}{1.461248in}}%
\pgfpathlineto{\pgfqpoint{3.399373in}{1.461248in}}%
\pgfpathlineto{\pgfqpoint{3.399373in}{1.456990in}}%
\pgfpathmoveto{\pgfqpoint{3.416404in}{1.422927in}}%
\pgfpathlineto{\pgfqpoint{3.416404in}{1.422927in}}%
\pgfpathlineto{\pgfqpoint{3.416404in}{1.427185in}}%
\pgfpathlineto{\pgfqpoint{3.420662in}{1.427185in}}%
\pgfpathlineto{\pgfqpoint{3.420662in}{1.422927in}}%
\pgfpathmoveto{\pgfqpoint{3.420662in}{1.422927in}}%
\pgfpathlineto{\pgfqpoint{3.420662in}{1.422927in}}%
\pgfpathlineto{\pgfqpoint{3.420662in}{1.427185in}}%
\pgfpathlineto{\pgfqpoint{3.424920in}{1.427185in}}%
\pgfpathlineto{\pgfqpoint{3.424920in}{1.422927in}}%
\pgfpathmoveto{\pgfqpoint{3.424920in}{1.418669in}}%
\pgfpathlineto{\pgfqpoint{3.424920in}{1.418669in}}%
\pgfpathlineto{\pgfqpoint{3.424920in}{1.422927in}}%
\pgfpathlineto{\pgfqpoint{3.429178in}{1.422927in}}%
\pgfpathlineto{\pgfqpoint{3.429178in}{1.418669in}}%
\pgfpathmoveto{\pgfqpoint{3.424920in}{1.422927in}}%
\pgfpathlineto{\pgfqpoint{3.424920in}{1.422927in}}%
\pgfpathlineto{\pgfqpoint{3.424920in}{1.427185in}}%
\pgfpathlineto{\pgfqpoint{3.429178in}{1.427185in}}%
\pgfpathlineto{\pgfqpoint{3.429178in}{1.422927in}}%
\pgfpathmoveto{\pgfqpoint{3.429178in}{1.418669in}}%
\pgfpathlineto{\pgfqpoint{3.429178in}{1.418669in}}%
\pgfpathlineto{\pgfqpoint{3.429178in}{1.422927in}}%
\pgfpathlineto{\pgfqpoint{3.433436in}{1.422927in}}%
\pgfpathlineto{\pgfqpoint{3.433436in}{1.418669in}}%
\pgfpathmoveto{\pgfqpoint{3.429178in}{1.422927in}}%
\pgfpathlineto{\pgfqpoint{3.429178in}{1.422927in}}%
\pgfpathlineto{\pgfqpoint{3.429178in}{1.427185in}}%
\pgfpathlineto{\pgfqpoint{3.433436in}{1.427185in}}%
\pgfpathlineto{\pgfqpoint{3.433436in}{1.422927in}}%
\pgfpathmoveto{\pgfqpoint{3.403631in}{1.431443in}}%
\pgfpathlineto{\pgfqpoint{3.403631in}{1.431443in}}%
\pgfpathlineto{\pgfqpoint{3.403631in}{1.435701in}}%
\pgfpathlineto{\pgfqpoint{3.407889in}{1.435701in}}%
\pgfpathlineto{\pgfqpoint{3.407889in}{1.431443in}}%
\pgfpathmoveto{\pgfqpoint{3.399373in}{1.435701in}}%
\pgfpathlineto{\pgfqpoint{3.399373in}{1.435701in}}%
\pgfpathlineto{\pgfqpoint{3.399373in}{1.439958in}}%
\pgfpathlineto{\pgfqpoint{3.403631in}{1.439958in}}%
\pgfpathlineto{\pgfqpoint{3.403631in}{1.435701in}}%
\pgfpathmoveto{\pgfqpoint{3.399373in}{1.439958in}}%
\pgfpathlineto{\pgfqpoint{3.399373in}{1.439958in}}%
\pgfpathlineto{\pgfqpoint{3.399373in}{1.444216in}}%
\pgfpathlineto{\pgfqpoint{3.403631in}{1.444216in}}%
\pgfpathlineto{\pgfqpoint{3.403631in}{1.439958in}}%
\pgfpathmoveto{\pgfqpoint{3.403631in}{1.435701in}}%
\pgfpathlineto{\pgfqpoint{3.403631in}{1.435701in}}%
\pgfpathlineto{\pgfqpoint{3.403631in}{1.439958in}}%
\pgfpathlineto{\pgfqpoint{3.407889in}{1.439958in}}%
\pgfpathlineto{\pgfqpoint{3.407889in}{1.435701in}}%
\pgfpathmoveto{\pgfqpoint{3.403631in}{1.439958in}}%
\pgfpathlineto{\pgfqpoint{3.403631in}{1.439958in}}%
\pgfpathlineto{\pgfqpoint{3.403631in}{1.444216in}}%
\pgfpathlineto{\pgfqpoint{3.407889in}{1.444216in}}%
\pgfpathlineto{\pgfqpoint{3.407889in}{1.439958in}}%
\pgfpathmoveto{\pgfqpoint{3.407889in}{1.427185in}}%
\pgfpathlineto{\pgfqpoint{3.407889in}{1.427185in}}%
\pgfpathlineto{\pgfqpoint{3.407889in}{1.431443in}}%
\pgfpathlineto{\pgfqpoint{3.412147in}{1.431443in}}%
\pgfpathlineto{\pgfqpoint{3.412147in}{1.427185in}}%
\pgfpathmoveto{\pgfqpoint{3.407889in}{1.431443in}}%
\pgfpathlineto{\pgfqpoint{3.407889in}{1.431443in}}%
\pgfpathlineto{\pgfqpoint{3.407889in}{1.435701in}}%
\pgfpathlineto{\pgfqpoint{3.412147in}{1.435701in}}%
\pgfpathlineto{\pgfqpoint{3.412147in}{1.431443in}}%
\pgfpathmoveto{\pgfqpoint{3.412147in}{1.427185in}}%
\pgfpathlineto{\pgfqpoint{3.412147in}{1.427185in}}%
\pgfpathlineto{\pgfqpoint{3.412147in}{1.431443in}}%
\pgfpathlineto{\pgfqpoint{3.416404in}{1.431443in}}%
\pgfpathlineto{\pgfqpoint{3.416404in}{1.427185in}}%
\pgfpathmoveto{\pgfqpoint{3.412147in}{1.431443in}}%
\pgfpathlineto{\pgfqpoint{3.412147in}{1.431443in}}%
\pgfpathlineto{\pgfqpoint{3.412147in}{1.435701in}}%
\pgfpathlineto{\pgfqpoint{3.416404in}{1.435701in}}%
\pgfpathlineto{\pgfqpoint{3.416404in}{1.431443in}}%
\pgfpathmoveto{\pgfqpoint{3.407889in}{1.435701in}}%
\pgfpathlineto{\pgfqpoint{3.407889in}{1.435701in}}%
\pgfpathlineto{\pgfqpoint{3.407889in}{1.439958in}}%
\pgfpathlineto{\pgfqpoint{3.412147in}{1.439958in}}%
\pgfpathlineto{\pgfqpoint{3.412147in}{1.435701in}}%
\pgfpathmoveto{\pgfqpoint{3.407889in}{1.439958in}}%
\pgfpathlineto{\pgfqpoint{3.407889in}{1.439958in}}%
\pgfpathlineto{\pgfqpoint{3.407889in}{1.444216in}}%
\pgfpathlineto{\pgfqpoint{3.412147in}{1.444216in}}%
\pgfpathlineto{\pgfqpoint{3.412147in}{1.439958in}}%
\pgfpathmoveto{\pgfqpoint{3.412147in}{1.435701in}}%
\pgfpathlineto{\pgfqpoint{3.412147in}{1.435701in}}%
\pgfpathlineto{\pgfqpoint{3.412147in}{1.439958in}}%
\pgfpathlineto{\pgfqpoint{3.416404in}{1.439958in}}%
\pgfpathlineto{\pgfqpoint{3.416404in}{1.435701in}}%
\pgfpathmoveto{\pgfqpoint{3.412147in}{1.439958in}}%
\pgfpathlineto{\pgfqpoint{3.412147in}{1.439958in}}%
\pgfpathlineto{\pgfqpoint{3.412147in}{1.444216in}}%
\pgfpathlineto{\pgfqpoint{3.416404in}{1.444216in}}%
\pgfpathlineto{\pgfqpoint{3.416404in}{1.439958in}}%
\pgfpathmoveto{\pgfqpoint{3.399373in}{1.444216in}}%
\pgfpathlineto{\pgfqpoint{3.399373in}{1.444216in}}%
\pgfpathlineto{\pgfqpoint{3.399373in}{1.448474in}}%
\pgfpathlineto{\pgfqpoint{3.403631in}{1.448474in}}%
\pgfpathlineto{\pgfqpoint{3.403631in}{1.444216in}}%
\pgfpathmoveto{\pgfqpoint{3.399373in}{1.448474in}}%
\pgfpathlineto{\pgfqpoint{3.399373in}{1.448474in}}%
\pgfpathlineto{\pgfqpoint{3.399373in}{1.452732in}}%
\pgfpathlineto{\pgfqpoint{3.403631in}{1.452732in}}%
\pgfpathlineto{\pgfqpoint{3.403631in}{1.448474in}}%
\pgfpathmoveto{\pgfqpoint{3.403631in}{1.444216in}}%
\pgfpathlineto{\pgfqpoint{3.403631in}{1.444216in}}%
\pgfpathlineto{\pgfqpoint{3.403631in}{1.448474in}}%
\pgfpathlineto{\pgfqpoint{3.407889in}{1.448474in}}%
\pgfpathlineto{\pgfqpoint{3.407889in}{1.444216in}}%
\pgfpathmoveto{\pgfqpoint{3.403631in}{1.448474in}}%
\pgfpathlineto{\pgfqpoint{3.403631in}{1.448474in}}%
\pgfpathlineto{\pgfqpoint{3.403631in}{1.452732in}}%
\pgfpathlineto{\pgfqpoint{3.407889in}{1.452732in}}%
\pgfpathlineto{\pgfqpoint{3.407889in}{1.448474in}}%
\pgfpathmoveto{\pgfqpoint{3.399373in}{1.452732in}}%
\pgfpathlineto{\pgfqpoint{3.399373in}{1.452732in}}%
\pgfpathlineto{\pgfqpoint{3.399373in}{1.456990in}}%
\pgfpathlineto{\pgfqpoint{3.403631in}{1.456990in}}%
\pgfpathlineto{\pgfqpoint{3.403631in}{1.452732in}}%
\pgfpathmoveto{\pgfqpoint{3.399373in}{1.456990in}}%
\pgfpathlineto{\pgfqpoint{3.399373in}{1.456990in}}%
\pgfpathlineto{\pgfqpoint{3.399373in}{1.461248in}}%
\pgfpathlineto{\pgfqpoint{3.403631in}{1.461248in}}%
\pgfpathlineto{\pgfqpoint{3.403631in}{1.456990in}}%
\pgfpathmoveto{\pgfqpoint{3.403631in}{1.452732in}}%
\pgfpathlineto{\pgfqpoint{3.403631in}{1.452732in}}%
\pgfpathlineto{\pgfqpoint{3.403631in}{1.456990in}}%
\pgfpathlineto{\pgfqpoint{3.407889in}{1.456990in}}%
\pgfpathlineto{\pgfqpoint{3.407889in}{1.452732in}}%
\pgfpathmoveto{\pgfqpoint{3.403631in}{1.456990in}}%
\pgfpathlineto{\pgfqpoint{3.403631in}{1.456990in}}%
\pgfpathlineto{\pgfqpoint{3.403631in}{1.461248in}}%
\pgfpathlineto{\pgfqpoint{3.407889in}{1.461248in}}%
\pgfpathlineto{\pgfqpoint{3.407889in}{1.456990in}}%
\pgfpathmoveto{\pgfqpoint{3.407889in}{1.444216in}}%
\pgfpathlineto{\pgfqpoint{3.407889in}{1.444216in}}%
\pgfpathlineto{\pgfqpoint{3.407889in}{1.448474in}}%
\pgfpathlineto{\pgfqpoint{3.412147in}{1.448474in}}%
\pgfpathlineto{\pgfqpoint{3.412147in}{1.444216in}}%
\pgfpathmoveto{\pgfqpoint{3.407889in}{1.448474in}}%
\pgfpathlineto{\pgfqpoint{3.407889in}{1.448474in}}%
\pgfpathlineto{\pgfqpoint{3.407889in}{1.452732in}}%
\pgfpathlineto{\pgfqpoint{3.412147in}{1.452732in}}%
\pgfpathlineto{\pgfqpoint{3.412147in}{1.448474in}}%
\pgfpathmoveto{\pgfqpoint{3.412147in}{1.444216in}}%
\pgfpathlineto{\pgfqpoint{3.412147in}{1.444216in}}%
\pgfpathlineto{\pgfqpoint{3.412147in}{1.448474in}}%
\pgfpathlineto{\pgfqpoint{3.416404in}{1.448474in}}%
\pgfpathlineto{\pgfqpoint{3.416404in}{1.444216in}}%
\pgfpathmoveto{\pgfqpoint{3.412147in}{1.448474in}}%
\pgfpathlineto{\pgfqpoint{3.412147in}{1.448474in}}%
\pgfpathlineto{\pgfqpoint{3.412147in}{1.452732in}}%
\pgfpathlineto{\pgfqpoint{3.416404in}{1.452732in}}%
\pgfpathlineto{\pgfqpoint{3.416404in}{1.448474in}}%
\pgfpathmoveto{\pgfqpoint{3.407889in}{1.452732in}}%
\pgfpathlineto{\pgfqpoint{3.407889in}{1.452732in}}%
\pgfpathlineto{\pgfqpoint{3.407889in}{1.456990in}}%
\pgfpathlineto{\pgfqpoint{3.412147in}{1.456990in}}%
\pgfpathlineto{\pgfqpoint{3.412147in}{1.452732in}}%
\pgfpathmoveto{\pgfqpoint{3.407889in}{1.456990in}}%
\pgfpathlineto{\pgfqpoint{3.407889in}{1.456990in}}%
\pgfpathlineto{\pgfqpoint{3.407889in}{1.461248in}}%
\pgfpathlineto{\pgfqpoint{3.412147in}{1.461248in}}%
\pgfpathlineto{\pgfqpoint{3.412147in}{1.456990in}}%
\pgfpathmoveto{\pgfqpoint{3.412147in}{1.452732in}}%
\pgfpathlineto{\pgfqpoint{3.412147in}{1.452732in}}%
\pgfpathlineto{\pgfqpoint{3.412147in}{1.456990in}}%
\pgfpathlineto{\pgfqpoint{3.416404in}{1.456990in}}%
\pgfpathlineto{\pgfqpoint{3.416404in}{1.452732in}}%
\pgfpathmoveto{\pgfqpoint{3.412147in}{1.456990in}}%
\pgfpathlineto{\pgfqpoint{3.412147in}{1.456990in}}%
\pgfpathlineto{\pgfqpoint{3.412147in}{1.461248in}}%
\pgfpathlineto{\pgfqpoint{3.416404in}{1.461248in}}%
\pgfpathlineto{\pgfqpoint{3.416404in}{1.456990in}}%
\pgfpathmoveto{\pgfqpoint{3.416404in}{1.427185in}}%
\pgfpathlineto{\pgfqpoint{3.416404in}{1.427185in}}%
\pgfpathlineto{\pgfqpoint{3.416404in}{1.431443in}}%
\pgfpathlineto{\pgfqpoint{3.420662in}{1.431443in}}%
\pgfpathlineto{\pgfqpoint{3.420662in}{1.427185in}}%
\pgfpathmoveto{\pgfqpoint{3.416404in}{1.431443in}}%
\pgfpathlineto{\pgfqpoint{3.416404in}{1.431443in}}%
\pgfpathlineto{\pgfqpoint{3.416404in}{1.435701in}}%
\pgfpathlineto{\pgfqpoint{3.420662in}{1.435701in}}%
\pgfpathlineto{\pgfqpoint{3.420662in}{1.431443in}}%
\pgfpathmoveto{\pgfqpoint{3.420662in}{1.427185in}}%
\pgfpathlineto{\pgfqpoint{3.420662in}{1.427185in}}%
\pgfpathlineto{\pgfqpoint{3.420662in}{1.431443in}}%
\pgfpathlineto{\pgfqpoint{3.424920in}{1.431443in}}%
\pgfpathlineto{\pgfqpoint{3.424920in}{1.427185in}}%
\pgfpathmoveto{\pgfqpoint{3.420662in}{1.431443in}}%
\pgfpathlineto{\pgfqpoint{3.420662in}{1.431443in}}%
\pgfpathlineto{\pgfqpoint{3.420662in}{1.435701in}}%
\pgfpathlineto{\pgfqpoint{3.424920in}{1.435701in}}%
\pgfpathlineto{\pgfqpoint{3.424920in}{1.431443in}}%
\pgfpathmoveto{\pgfqpoint{3.416404in}{1.435701in}}%
\pgfpathlineto{\pgfqpoint{3.416404in}{1.435701in}}%
\pgfpathlineto{\pgfqpoint{3.416404in}{1.439958in}}%
\pgfpathlineto{\pgfqpoint{3.420662in}{1.439958in}}%
\pgfpathlineto{\pgfqpoint{3.420662in}{1.435701in}}%
\pgfpathmoveto{\pgfqpoint{3.416404in}{1.439958in}}%
\pgfpathlineto{\pgfqpoint{3.416404in}{1.439958in}}%
\pgfpathlineto{\pgfqpoint{3.416404in}{1.444216in}}%
\pgfpathlineto{\pgfqpoint{3.420662in}{1.444216in}}%
\pgfpathlineto{\pgfqpoint{3.420662in}{1.439958in}}%
\pgfpathmoveto{\pgfqpoint{3.420662in}{1.435701in}}%
\pgfpathlineto{\pgfqpoint{3.420662in}{1.435701in}}%
\pgfpathlineto{\pgfqpoint{3.420662in}{1.439958in}}%
\pgfpathlineto{\pgfqpoint{3.424920in}{1.439958in}}%
\pgfpathlineto{\pgfqpoint{3.424920in}{1.435701in}}%
\pgfpathmoveto{\pgfqpoint{3.420662in}{1.439958in}}%
\pgfpathlineto{\pgfqpoint{3.420662in}{1.439958in}}%
\pgfpathlineto{\pgfqpoint{3.420662in}{1.444216in}}%
\pgfpathlineto{\pgfqpoint{3.424920in}{1.444216in}}%
\pgfpathlineto{\pgfqpoint{3.424920in}{1.439958in}}%
\pgfpathmoveto{\pgfqpoint{3.424920in}{1.427185in}}%
\pgfpathlineto{\pgfqpoint{3.424920in}{1.427185in}}%
\pgfpathlineto{\pgfqpoint{3.424920in}{1.431443in}}%
\pgfpathlineto{\pgfqpoint{3.429178in}{1.431443in}}%
\pgfpathlineto{\pgfqpoint{3.429178in}{1.427185in}}%
\pgfpathmoveto{\pgfqpoint{3.424920in}{1.431443in}}%
\pgfpathlineto{\pgfqpoint{3.424920in}{1.431443in}}%
\pgfpathlineto{\pgfqpoint{3.424920in}{1.435701in}}%
\pgfpathlineto{\pgfqpoint{3.429178in}{1.435701in}}%
\pgfpathlineto{\pgfqpoint{3.429178in}{1.431443in}}%
\pgfpathmoveto{\pgfqpoint{3.429178in}{1.427185in}}%
\pgfpathlineto{\pgfqpoint{3.429178in}{1.427185in}}%
\pgfpathlineto{\pgfqpoint{3.429178in}{1.431443in}}%
\pgfpathlineto{\pgfqpoint{3.433436in}{1.431443in}}%
\pgfpathlineto{\pgfqpoint{3.433436in}{1.427185in}}%
\pgfpathmoveto{\pgfqpoint{3.429178in}{1.431443in}}%
\pgfpathlineto{\pgfqpoint{3.429178in}{1.431443in}}%
\pgfpathlineto{\pgfqpoint{3.429178in}{1.435701in}}%
\pgfpathlineto{\pgfqpoint{3.433436in}{1.435701in}}%
\pgfpathlineto{\pgfqpoint{3.433436in}{1.431443in}}%
\pgfpathmoveto{\pgfqpoint{3.424920in}{1.435701in}}%
\pgfpathlineto{\pgfqpoint{3.424920in}{1.435701in}}%
\pgfpathlineto{\pgfqpoint{3.424920in}{1.439958in}}%
\pgfpathlineto{\pgfqpoint{3.429178in}{1.439958in}}%
\pgfpathlineto{\pgfqpoint{3.429178in}{1.435701in}}%
\pgfpathmoveto{\pgfqpoint{3.424920in}{1.439958in}}%
\pgfpathlineto{\pgfqpoint{3.424920in}{1.439958in}}%
\pgfpathlineto{\pgfqpoint{3.424920in}{1.444216in}}%
\pgfpathlineto{\pgfqpoint{3.429178in}{1.444216in}}%
\pgfpathlineto{\pgfqpoint{3.429178in}{1.439958in}}%
\pgfpathmoveto{\pgfqpoint{3.429178in}{1.435701in}}%
\pgfpathlineto{\pgfqpoint{3.429178in}{1.435701in}}%
\pgfpathlineto{\pgfqpoint{3.429178in}{1.439958in}}%
\pgfpathlineto{\pgfqpoint{3.433436in}{1.439958in}}%
\pgfpathlineto{\pgfqpoint{3.433436in}{1.435701in}}%
\pgfpathmoveto{\pgfqpoint{3.429178in}{1.439958in}}%
\pgfpathlineto{\pgfqpoint{3.429178in}{1.439958in}}%
\pgfpathlineto{\pgfqpoint{3.429178in}{1.444216in}}%
\pgfpathlineto{\pgfqpoint{3.433436in}{1.444216in}}%
\pgfpathlineto{\pgfqpoint{3.433436in}{1.439958in}}%
\pgfpathmoveto{\pgfqpoint{3.416404in}{1.444216in}}%
\pgfpathlineto{\pgfqpoint{3.416404in}{1.444216in}}%
\pgfpathlineto{\pgfqpoint{3.416404in}{1.448474in}}%
\pgfpathlineto{\pgfqpoint{3.420662in}{1.448474in}}%
\pgfpathlineto{\pgfqpoint{3.420662in}{1.444216in}}%
\pgfpathmoveto{\pgfqpoint{3.416404in}{1.448474in}}%
\pgfpathlineto{\pgfqpoint{3.416404in}{1.448474in}}%
\pgfpathlineto{\pgfqpoint{3.416404in}{1.452732in}}%
\pgfpathlineto{\pgfqpoint{3.420662in}{1.452732in}}%
\pgfpathlineto{\pgfqpoint{3.420662in}{1.448474in}}%
\pgfpathmoveto{\pgfqpoint{3.420662in}{1.444216in}}%
\pgfpathlineto{\pgfqpoint{3.420662in}{1.444216in}}%
\pgfpathlineto{\pgfqpoint{3.420662in}{1.448474in}}%
\pgfpathlineto{\pgfqpoint{3.424920in}{1.448474in}}%
\pgfpathlineto{\pgfqpoint{3.424920in}{1.444216in}}%
\pgfpathmoveto{\pgfqpoint{3.420662in}{1.448474in}}%
\pgfpathlineto{\pgfqpoint{3.420662in}{1.448474in}}%
\pgfpathlineto{\pgfqpoint{3.420662in}{1.452732in}}%
\pgfpathlineto{\pgfqpoint{3.424920in}{1.452732in}}%
\pgfpathlineto{\pgfqpoint{3.424920in}{1.448474in}}%
\pgfpathmoveto{\pgfqpoint{3.416404in}{1.452732in}}%
\pgfpathlineto{\pgfqpoint{3.416404in}{1.452732in}}%
\pgfpathlineto{\pgfqpoint{3.416404in}{1.456990in}}%
\pgfpathlineto{\pgfqpoint{3.420662in}{1.456990in}}%
\pgfpathlineto{\pgfqpoint{3.420662in}{1.452732in}}%
\pgfpathmoveto{\pgfqpoint{3.416404in}{1.456990in}}%
\pgfpathlineto{\pgfqpoint{3.416404in}{1.456990in}}%
\pgfpathlineto{\pgfqpoint{3.416404in}{1.461248in}}%
\pgfpathlineto{\pgfqpoint{3.420662in}{1.461248in}}%
\pgfpathlineto{\pgfqpoint{3.420662in}{1.456990in}}%
\pgfpathmoveto{\pgfqpoint{3.420662in}{1.452732in}}%
\pgfpathlineto{\pgfqpoint{3.420662in}{1.452732in}}%
\pgfpathlineto{\pgfqpoint{3.420662in}{1.456990in}}%
\pgfpathlineto{\pgfqpoint{3.424920in}{1.456990in}}%
\pgfpathlineto{\pgfqpoint{3.424920in}{1.452732in}}%
\pgfpathmoveto{\pgfqpoint{3.424920in}{1.444216in}}%
\pgfpathlineto{\pgfqpoint{3.424920in}{1.444216in}}%
\pgfpathlineto{\pgfqpoint{3.424920in}{1.448474in}}%
\pgfpathlineto{\pgfqpoint{3.429178in}{1.448474in}}%
\pgfpathlineto{\pgfqpoint{3.429178in}{1.444216in}}%
\pgfpathmoveto{\pgfqpoint{3.424920in}{1.448474in}}%
\pgfpathlineto{\pgfqpoint{3.424920in}{1.448474in}}%
\pgfpathlineto{\pgfqpoint{3.424920in}{1.452732in}}%
\pgfpathlineto{\pgfqpoint{3.429178in}{1.452732in}}%
\pgfpathlineto{\pgfqpoint{3.429178in}{1.448474in}}%
\pgfpathmoveto{\pgfqpoint{3.429178in}{1.444216in}}%
\pgfpathlineto{\pgfqpoint{3.429178in}{1.444216in}}%
\pgfpathlineto{\pgfqpoint{3.429178in}{1.448474in}}%
\pgfpathlineto{\pgfqpoint{3.433436in}{1.448474in}}%
\pgfpathlineto{\pgfqpoint{3.433436in}{1.444216in}}%
\pgfpathmoveto{\pgfqpoint{3.429178in}{1.448474in}}%
\pgfpathlineto{\pgfqpoint{3.429178in}{1.448474in}}%
\pgfpathlineto{\pgfqpoint{3.429178in}{1.452732in}}%
\pgfpathlineto{\pgfqpoint{3.433436in}{1.452732in}}%
\pgfpathlineto{\pgfqpoint{3.433436in}{1.448474in}}%
\pgfpathmoveto{\pgfqpoint{3.424920in}{1.452732in}}%
\pgfpathlineto{\pgfqpoint{3.424920in}{1.452732in}}%
\pgfpathlineto{\pgfqpoint{3.424920in}{1.456990in}}%
\pgfpathlineto{\pgfqpoint{3.429178in}{1.456990in}}%
\pgfpathlineto{\pgfqpoint{3.429178in}{1.452732in}}%
\pgfpathmoveto{\pgfqpoint{3.429178in}{1.452732in}}%
\pgfpathlineto{\pgfqpoint{3.429178in}{1.452732in}}%
\pgfpathlineto{\pgfqpoint{3.429178in}{1.456990in}}%
\pgfpathlineto{\pgfqpoint{3.433436in}{1.456990in}}%
\pgfpathlineto{\pgfqpoint{3.433436in}{1.452732in}}%
\pgfpathmoveto{\pgfqpoint{3.437694in}{1.414411in}}%
\pgfpathlineto{\pgfqpoint{3.437694in}{1.414411in}}%
\pgfpathlineto{\pgfqpoint{3.437694in}{1.418669in}}%
\pgfpathlineto{\pgfqpoint{3.441951in}{1.418669in}}%
\pgfpathlineto{\pgfqpoint{3.441951in}{1.414411in}}%
\pgfpathmoveto{\pgfqpoint{3.433436in}{1.418669in}}%
\pgfpathlineto{\pgfqpoint{3.433436in}{1.418669in}}%
\pgfpathlineto{\pgfqpoint{3.433436in}{1.422927in}}%
\pgfpathlineto{\pgfqpoint{3.437694in}{1.422927in}}%
\pgfpathlineto{\pgfqpoint{3.437694in}{1.418669in}}%
\pgfpathmoveto{\pgfqpoint{3.433436in}{1.422927in}}%
\pgfpathlineto{\pgfqpoint{3.433436in}{1.422927in}}%
\pgfpathlineto{\pgfqpoint{3.433436in}{1.427185in}}%
\pgfpathlineto{\pgfqpoint{3.437694in}{1.427185in}}%
\pgfpathlineto{\pgfqpoint{3.437694in}{1.422927in}}%
\pgfpathmoveto{\pgfqpoint{3.437694in}{1.418669in}}%
\pgfpathlineto{\pgfqpoint{3.437694in}{1.418669in}}%
\pgfpathlineto{\pgfqpoint{3.437694in}{1.422927in}}%
\pgfpathlineto{\pgfqpoint{3.441951in}{1.422927in}}%
\pgfpathlineto{\pgfqpoint{3.441951in}{1.418669in}}%
\pgfpathmoveto{\pgfqpoint{3.437694in}{1.422927in}}%
\pgfpathlineto{\pgfqpoint{3.437694in}{1.422927in}}%
\pgfpathlineto{\pgfqpoint{3.437694in}{1.427185in}}%
\pgfpathlineto{\pgfqpoint{3.441951in}{1.427185in}}%
\pgfpathlineto{\pgfqpoint{3.441951in}{1.422927in}}%
\pgfpathmoveto{\pgfqpoint{3.441951in}{1.414411in}}%
\pgfpathlineto{\pgfqpoint{3.441951in}{1.414411in}}%
\pgfpathlineto{\pgfqpoint{3.441951in}{1.418669in}}%
\pgfpathlineto{\pgfqpoint{3.446209in}{1.418669in}}%
\pgfpathlineto{\pgfqpoint{3.446209in}{1.414411in}}%
\pgfpathmoveto{\pgfqpoint{3.446209in}{1.414411in}}%
\pgfpathlineto{\pgfqpoint{3.446209in}{1.414411in}}%
\pgfpathlineto{\pgfqpoint{3.446209in}{1.418669in}}%
\pgfpathlineto{\pgfqpoint{3.450467in}{1.418669in}}%
\pgfpathlineto{\pgfqpoint{3.450467in}{1.414411in}}%
\pgfpathmoveto{\pgfqpoint{3.441951in}{1.418669in}}%
\pgfpathlineto{\pgfqpoint{3.441951in}{1.418669in}}%
\pgfpathlineto{\pgfqpoint{3.441951in}{1.422927in}}%
\pgfpathlineto{\pgfqpoint{3.446209in}{1.422927in}}%
\pgfpathlineto{\pgfqpoint{3.446209in}{1.418669in}}%
\pgfpathmoveto{\pgfqpoint{3.441951in}{1.422927in}}%
\pgfpathlineto{\pgfqpoint{3.441951in}{1.422927in}}%
\pgfpathlineto{\pgfqpoint{3.441951in}{1.427185in}}%
\pgfpathlineto{\pgfqpoint{3.446209in}{1.427185in}}%
\pgfpathlineto{\pgfqpoint{3.446209in}{1.422927in}}%
\pgfpathmoveto{\pgfqpoint{3.446209in}{1.418669in}}%
\pgfpathlineto{\pgfqpoint{3.446209in}{1.418669in}}%
\pgfpathlineto{\pgfqpoint{3.446209in}{1.422927in}}%
\pgfpathlineto{\pgfqpoint{3.450467in}{1.422927in}}%
\pgfpathlineto{\pgfqpoint{3.450467in}{1.418669in}}%
\pgfpathmoveto{\pgfqpoint{3.446209in}{1.422927in}}%
\pgfpathlineto{\pgfqpoint{3.446209in}{1.422927in}}%
\pgfpathlineto{\pgfqpoint{3.446209in}{1.427185in}}%
\pgfpathlineto{\pgfqpoint{3.450467in}{1.427185in}}%
\pgfpathlineto{\pgfqpoint{3.450467in}{1.422927in}}%
\pgfpathmoveto{\pgfqpoint{3.450467in}{1.414411in}}%
\pgfpathlineto{\pgfqpoint{3.450467in}{1.414411in}}%
\pgfpathlineto{\pgfqpoint{3.450467in}{1.418669in}}%
\pgfpathlineto{\pgfqpoint{3.454725in}{1.418669in}}%
\pgfpathlineto{\pgfqpoint{3.454725in}{1.414411in}}%
\pgfpathmoveto{\pgfqpoint{3.454725in}{1.414411in}}%
\pgfpathlineto{\pgfqpoint{3.454725in}{1.414411in}}%
\pgfpathlineto{\pgfqpoint{3.454725in}{1.418669in}}%
\pgfpathlineto{\pgfqpoint{3.458983in}{1.418669in}}%
\pgfpathlineto{\pgfqpoint{3.458983in}{1.414411in}}%
\pgfpathmoveto{\pgfqpoint{3.450467in}{1.418669in}}%
\pgfpathlineto{\pgfqpoint{3.450467in}{1.418669in}}%
\pgfpathlineto{\pgfqpoint{3.450467in}{1.422927in}}%
\pgfpathlineto{\pgfqpoint{3.454725in}{1.422927in}}%
\pgfpathlineto{\pgfqpoint{3.454725in}{1.418669in}}%
\pgfpathmoveto{\pgfqpoint{3.450467in}{1.422927in}}%
\pgfpathlineto{\pgfqpoint{3.450467in}{1.422927in}}%
\pgfpathlineto{\pgfqpoint{3.450467in}{1.427185in}}%
\pgfpathlineto{\pgfqpoint{3.454725in}{1.427185in}}%
\pgfpathlineto{\pgfqpoint{3.454725in}{1.422927in}}%
\pgfpathmoveto{\pgfqpoint{3.454725in}{1.418669in}}%
\pgfpathlineto{\pgfqpoint{3.454725in}{1.418669in}}%
\pgfpathlineto{\pgfqpoint{3.454725in}{1.422927in}}%
\pgfpathlineto{\pgfqpoint{3.458983in}{1.422927in}}%
\pgfpathlineto{\pgfqpoint{3.458983in}{1.418669in}}%
\pgfpathmoveto{\pgfqpoint{3.454725in}{1.422927in}}%
\pgfpathlineto{\pgfqpoint{3.454725in}{1.422927in}}%
\pgfpathlineto{\pgfqpoint{3.454725in}{1.427185in}}%
\pgfpathlineto{\pgfqpoint{3.458983in}{1.427185in}}%
\pgfpathlineto{\pgfqpoint{3.458983in}{1.422927in}}%
\pgfpathmoveto{\pgfqpoint{3.458983in}{1.414411in}}%
\pgfpathlineto{\pgfqpoint{3.458983in}{1.414411in}}%
\pgfpathlineto{\pgfqpoint{3.458983in}{1.418669in}}%
\pgfpathlineto{\pgfqpoint{3.463241in}{1.418669in}}%
\pgfpathlineto{\pgfqpoint{3.463241in}{1.414411in}}%
\pgfpathmoveto{\pgfqpoint{3.458983in}{1.418669in}}%
\pgfpathlineto{\pgfqpoint{3.458983in}{1.418669in}}%
\pgfpathlineto{\pgfqpoint{3.458983in}{1.422927in}}%
\pgfpathlineto{\pgfqpoint{3.463241in}{1.422927in}}%
\pgfpathlineto{\pgfqpoint{3.463241in}{1.418669in}}%
\pgfpathmoveto{\pgfqpoint{3.458983in}{1.422927in}}%
\pgfpathlineto{\pgfqpoint{3.458983in}{1.422927in}}%
\pgfpathlineto{\pgfqpoint{3.458983in}{1.427185in}}%
\pgfpathlineto{\pgfqpoint{3.463241in}{1.427185in}}%
\pgfpathlineto{\pgfqpoint{3.463241in}{1.422927in}}%
\pgfpathmoveto{\pgfqpoint{3.463241in}{1.418669in}}%
\pgfpathlineto{\pgfqpoint{3.463241in}{1.418669in}}%
\pgfpathlineto{\pgfqpoint{3.463241in}{1.422927in}}%
\pgfpathlineto{\pgfqpoint{3.467498in}{1.422927in}}%
\pgfpathlineto{\pgfqpoint{3.467498in}{1.418669in}}%
\pgfpathmoveto{\pgfqpoint{3.463241in}{1.422927in}}%
\pgfpathlineto{\pgfqpoint{3.463241in}{1.422927in}}%
\pgfpathlineto{\pgfqpoint{3.463241in}{1.427185in}}%
\pgfpathlineto{\pgfqpoint{3.467498in}{1.427185in}}%
\pgfpathlineto{\pgfqpoint{3.467498in}{1.422927in}}%
\pgfpathmoveto{\pgfqpoint{3.433436in}{1.427185in}}%
\pgfpathlineto{\pgfqpoint{3.433436in}{1.427185in}}%
\pgfpathlineto{\pgfqpoint{3.433436in}{1.431443in}}%
\pgfpathlineto{\pgfqpoint{3.437694in}{1.431443in}}%
\pgfpathlineto{\pgfqpoint{3.437694in}{1.427185in}}%
\pgfpathmoveto{\pgfqpoint{3.433436in}{1.431443in}}%
\pgfpathlineto{\pgfqpoint{3.433436in}{1.431443in}}%
\pgfpathlineto{\pgfqpoint{3.433436in}{1.435701in}}%
\pgfpathlineto{\pgfqpoint{3.437694in}{1.435701in}}%
\pgfpathlineto{\pgfqpoint{3.437694in}{1.431443in}}%
\pgfpathmoveto{\pgfqpoint{3.437694in}{1.427185in}}%
\pgfpathlineto{\pgfqpoint{3.437694in}{1.427185in}}%
\pgfpathlineto{\pgfqpoint{3.437694in}{1.431443in}}%
\pgfpathlineto{\pgfqpoint{3.441951in}{1.431443in}}%
\pgfpathlineto{\pgfqpoint{3.441951in}{1.427185in}}%
\pgfpathmoveto{\pgfqpoint{3.437694in}{1.431443in}}%
\pgfpathlineto{\pgfqpoint{3.437694in}{1.431443in}}%
\pgfpathlineto{\pgfqpoint{3.437694in}{1.435701in}}%
\pgfpathlineto{\pgfqpoint{3.441951in}{1.435701in}}%
\pgfpathlineto{\pgfqpoint{3.441951in}{1.431443in}}%
\pgfpathmoveto{\pgfqpoint{3.433436in}{1.435701in}}%
\pgfpathlineto{\pgfqpoint{3.433436in}{1.435701in}}%
\pgfpathlineto{\pgfqpoint{3.433436in}{1.439958in}}%
\pgfpathlineto{\pgfqpoint{3.437694in}{1.439958in}}%
\pgfpathlineto{\pgfqpoint{3.437694in}{1.435701in}}%
\pgfpathmoveto{\pgfqpoint{3.433436in}{1.439958in}}%
\pgfpathlineto{\pgfqpoint{3.433436in}{1.439958in}}%
\pgfpathlineto{\pgfqpoint{3.433436in}{1.444216in}}%
\pgfpathlineto{\pgfqpoint{3.437694in}{1.444216in}}%
\pgfpathlineto{\pgfqpoint{3.437694in}{1.439958in}}%
\pgfpathmoveto{\pgfqpoint{3.437694in}{1.435701in}}%
\pgfpathlineto{\pgfqpoint{3.437694in}{1.435701in}}%
\pgfpathlineto{\pgfqpoint{3.437694in}{1.439958in}}%
\pgfpathlineto{\pgfqpoint{3.441951in}{1.439958in}}%
\pgfpathlineto{\pgfqpoint{3.441951in}{1.435701in}}%
\pgfpathmoveto{\pgfqpoint{3.437694in}{1.439958in}}%
\pgfpathlineto{\pgfqpoint{3.437694in}{1.439958in}}%
\pgfpathlineto{\pgfqpoint{3.437694in}{1.444216in}}%
\pgfpathlineto{\pgfqpoint{3.441951in}{1.444216in}}%
\pgfpathlineto{\pgfqpoint{3.441951in}{1.439958in}}%
\pgfpathmoveto{\pgfqpoint{3.441951in}{1.427185in}}%
\pgfpathlineto{\pgfqpoint{3.441951in}{1.427185in}}%
\pgfpathlineto{\pgfqpoint{3.441951in}{1.431443in}}%
\pgfpathlineto{\pgfqpoint{3.446209in}{1.431443in}}%
\pgfpathlineto{\pgfqpoint{3.446209in}{1.427185in}}%
\pgfpathmoveto{\pgfqpoint{3.441951in}{1.431443in}}%
\pgfpathlineto{\pgfqpoint{3.441951in}{1.431443in}}%
\pgfpathlineto{\pgfqpoint{3.441951in}{1.435701in}}%
\pgfpathlineto{\pgfqpoint{3.446209in}{1.435701in}}%
\pgfpathlineto{\pgfqpoint{3.446209in}{1.431443in}}%
\pgfpathmoveto{\pgfqpoint{3.446209in}{1.427185in}}%
\pgfpathlineto{\pgfqpoint{3.446209in}{1.427185in}}%
\pgfpathlineto{\pgfqpoint{3.446209in}{1.431443in}}%
\pgfpathlineto{\pgfqpoint{3.450467in}{1.431443in}}%
\pgfpathlineto{\pgfqpoint{3.450467in}{1.427185in}}%
\pgfpathmoveto{\pgfqpoint{3.446209in}{1.431443in}}%
\pgfpathlineto{\pgfqpoint{3.446209in}{1.431443in}}%
\pgfpathlineto{\pgfqpoint{3.446209in}{1.435701in}}%
\pgfpathlineto{\pgfqpoint{3.450467in}{1.435701in}}%
\pgfpathlineto{\pgfqpoint{3.450467in}{1.431443in}}%
\pgfpathmoveto{\pgfqpoint{3.441951in}{1.435701in}}%
\pgfpathlineto{\pgfqpoint{3.441951in}{1.435701in}}%
\pgfpathlineto{\pgfqpoint{3.441951in}{1.439958in}}%
\pgfpathlineto{\pgfqpoint{3.446209in}{1.439958in}}%
\pgfpathlineto{\pgfqpoint{3.446209in}{1.435701in}}%
\pgfpathmoveto{\pgfqpoint{3.441951in}{1.439958in}}%
\pgfpathlineto{\pgfqpoint{3.441951in}{1.439958in}}%
\pgfpathlineto{\pgfqpoint{3.441951in}{1.444216in}}%
\pgfpathlineto{\pgfqpoint{3.446209in}{1.444216in}}%
\pgfpathlineto{\pgfqpoint{3.446209in}{1.439958in}}%
\pgfpathmoveto{\pgfqpoint{3.446209in}{1.435701in}}%
\pgfpathlineto{\pgfqpoint{3.446209in}{1.435701in}}%
\pgfpathlineto{\pgfqpoint{3.446209in}{1.439958in}}%
\pgfpathlineto{\pgfqpoint{3.450467in}{1.439958in}}%
\pgfpathlineto{\pgfqpoint{3.450467in}{1.435701in}}%
\pgfpathmoveto{\pgfqpoint{3.446209in}{1.439958in}}%
\pgfpathlineto{\pgfqpoint{3.446209in}{1.439958in}}%
\pgfpathlineto{\pgfqpoint{3.446209in}{1.444216in}}%
\pgfpathlineto{\pgfqpoint{3.450467in}{1.444216in}}%
\pgfpathlineto{\pgfqpoint{3.450467in}{1.439958in}}%
\pgfpathmoveto{\pgfqpoint{3.433436in}{1.444216in}}%
\pgfpathlineto{\pgfqpoint{3.433436in}{1.444216in}}%
\pgfpathlineto{\pgfqpoint{3.433436in}{1.448474in}}%
\pgfpathlineto{\pgfqpoint{3.437694in}{1.448474in}}%
\pgfpathlineto{\pgfqpoint{3.437694in}{1.444216in}}%
\pgfpathmoveto{\pgfqpoint{3.433436in}{1.448474in}}%
\pgfpathlineto{\pgfqpoint{3.433436in}{1.448474in}}%
\pgfpathlineto{\pgfqpoint{3.433436in}{1.452732in}}%
\pgfpathlineto{\pgfqpoint{3.437694in}{1.452732in}}%
\pgfpathlineto{\pgfqpoint{3.437694in}{1.448474in}}%
\pgfpathmoveto{\pgfqpoint{3.437694in}{1.444216in}}%
\pgfpathlineto{\pgfqpoint{3.437694in}{1.444216in}}%
\pgfpathlineto{\pgfqpoint{3.437694in}{1.448474in}}%
\pgfpathlineto{\pgfqpoint{3.441951in}{1.448474in}}%
\pgfpathlineto{\pgfqpoint{3.441951in}{1.444216in}}%
\pgfpathmoveto{\pgfqpoint{3.437694in}{1.448474in}}%
\pgfpathlineto{\pgfqpoint{3.437694in}{1.448474in}}%
\pgfpathlineto{\pgfqpoint{3.437694in}{1.452732in}}%
\pgfpathlineto{\pgfqpoint{3.441951in}{1.452732in}}%
\pgfpathlineto{\pgfqpoint{3.441951in}{1.448474in}}%
\pgfpathmoveto{\pgfqpoint{3.441951in}{1.444216in}}%
\pgfpathlineto{\pgfqpoint{3.441951in}{1.444216in}}%
\pgfpathlineto{\pgfqpoint{3.441951in}{1.448474in}}%
\pgfpathlineto{\pgfqpoint{3.446209in}{1.448474in}}%
\pgfpathlineto{\pgfqpoint{3.446209in}{1.444216in}}%
\pgfpathmoveto{\pgfqpoint{3.441951in}{1.448474in}}%
\pgfpathlineto{\pgfqpoint{3.441951in}{1.448474in}}%
\pgfpathlineto{\pgfqpoint{3.441951in}{1.452732in}}%
\pgfpathlineto{\pgfqpoint{3.446209in}{1.452732in}}%
\pgfpathlineto{\pgfqpoint{3.446209in}{1.448474in}}%
\pgfpathmoveto{\pgfqpoint{3.446209in}{1.444216in}}%
\pgfpathlineto{\pgfqpoint{3.446209in}{1.444216in}}%
\pgfpathlineto{\pgfqpoint{3.446209in}{1.448474in}}%
\pgfpathlineto{\pgfqpoint{3.450467in}{1.448474in}}%
\pgfpathlineto{\pgfqpoint{3.450467in}{1.444216in}}%
\pgfpathmoveto{\pgfqpoint{3.446209in}{1.448474in}}%
\pgfpathlineto{\pgfqpoint{3.446209in}{1.448474in}}%
\pgfpathlineto{\pgfqpoint{3.446209in}{1.452732in}}%
\pgfpathlineto{\pgfqpoint{3.450467in}{1.452732in}}%
\pgfpathlineto{\pgfqpoint{3.450467in}{1.448474in}}%
\pgfpathmoveto{\pgfqpoint{3.450467in}{1.427185in}}%
\pgfpathlineto{\pgfqpoint{3.450467in}{1.427185in}}%
\pgfpathlineto{\pgfqpoint{3.450467in}{1.431443in}}%
\pgfpathlineto{\pgfqpoint{3.454725in}{1.431443in}}%
\pgfpathlineto{\pgfqpoint{3.454725in}{1.427185in}}%
\pgfpathmoveto{\pgfqpoint{3.450467in}{1.431443in}}%
\pgfpathlineto{\pgfqpoint{3.450467in}{1.431443in}}%
\pgfpathlineto{\pgfqpoint{3.450467in}{1.435701in}}%
\pgfpathlineto{\pgfqpoint{3.454725in}{1.435701in}}%
\pgfpathlineto{\pgfqpoint{3.454725in}{1.431443in}}%
\pgfpathmoveto{\pgfqpoint{3.454725in}{1.427185in}}%
\pgfpathlineto{\pgfqpoint{3.454725in}{1.427185in}}%
\pgfpathlineto{\pgfqpoint{3.454725in}{1.431443in}}%
\pgfpathlineto{\pgfqpoint{3.458983in}{1.431443in}}%
\pgfpathlineto{\pgfqpoint{3.458983in}{1.427185in}}%
\pgfpathmoveto{\pgfqpoint{3.454725in}{1.431443in}}%
\pgfpathlineto{\pgfqpoint{3.454725in}{1.431443in}}%
\pgfpathlineto{\pgfqpoint{3.454725in}{1.435701in}}%
\pgfpathlineto{\pgfqpoint{3.458983in}{1.435701in}}%
\pgfpathlineto{\pgfqpoint{3.458983in}{1.431443in}}%
\pgfpathmoveto{\pgfqpoint{3.450467in}{1.435701in}}%
\pgfpathlineto{\pgfqpoint{3.450467in}{1.435701in}}%
\pgfpathlineto{\pgfqpoint{3.450467in}{1.439958in}}%
\pgfpathlineto{\pgfqpoint{3.454725in}{1.439958in}}%
\pgfpathlineto{\pgfqpoint{3.454725in}{1.435701in}}%
\pgfpathmoveto{\pgfqpoint{3.450467in}{1.439958in}}%
\pgfpathlineto{\pgfqpoint{3.450467in}{1.439958in}}%
\pgfpathlineto{\pgfqpoint{3.450467in}{1.444216in}}%
\pgfpathlineto{\pgfqpoint{3.454725in}{1.444216in}}%
\pgfpathlineto{\pgfqpoint{3.454725in}{1.439958in}}%
\pgfpathmoveto{\pgfqpoint{3.454725in}{1.435701in}}%
\pgfpathlineto{\pgfqpoint{3.454725in}{1.435701in}}%
\pgfpathlineto{\pgfqpoint{3.454725in}{1.439958in}}%
\pgfpathlineto{\pgfqpoint{3.458983in}{1.439958in}}%
\pgfpathlineto{\pgfqpoint{3.458983in}{1.435701in}}%
\pgfpathmoveto{\pgfqpoint{3.454725in}{1.439958in}}%
\pgfpathlineto{\pgfqpoint{3.454725in}{1.439958in}}%
\pgfpathlineto{\pgfqpoint{3.454725in}{1.444216in}}%
\pgfpathlineto{\pgfqpoint{3.458983in}{1.444216in}}%
\pgfpathlineto{\pgfqpoint{3.458983in}{1.439958in}}%
\pgfpathmoveto{\pgfqpoint{3.458983in}{1.427185in}}%
\pgfpathlineto{\pgfqpoint{3.458983in}{1.427185in}}%
\pgfpathlineto{\pgfqpoint{3.458983in}{1.431443in}}%
\pgfpathlineto{\pgfqpoint{3.463241in}{1.431443in}}%
\pgfpathlineto{\pgfqpoint{3.463241in}{1.427185in}}%
\pgfpathmoveto{\pgfqpoint{3.458983in}{1.431443in}}%
\pgfpathlineto{\pgfqpoint{3.458983in}{1.431443in}}%
\pgfpathlineto{\pgfqpoint{3.458983in}{1.435701in}}%
\pgfpathlineto{\pgfqpoint{3.463241in}{1.435701in}}%
\pgfpathlineto{\pgfqpoint{3.463241in}{1.431443in}}%
\pgfpathmoveto{\pgfqpoint{3.463241in}{1.427185in}}%
\pgfpathlineto{\pgfqpoint{3.463241in}{1.427185in}}%
\pgfpathlineto{\pgfqpoint{3.463241in}{1.431443in}}%
\pgfpathlineto{\pgfqpoint{3.467498in}{1.431443in}}%
\pgfpathlineto{\pgfqpoint{3.467498in}{1.427185in}}%
\pgfpathmoveto{\pgfqpoint{3.463241in}{1.431443in}}%
\pgfpathlineto{\pgfqpoint{3.463241in}{1.431443in}}%
\pgfpathlineto{\pgfqpoint{3.463241in}{1.435701in}}%
\pgfpathlineto{\pgfqpoint{3.467498in}{1.435701in}}%
\pgfpathlineto{\pgfqpoint{3.467498in}{1.431443in}}%
\pgfpathmoveto{\pgfqpoint{3.458983in}{1.435701in}}%
\pgfpathlineto{\pgfqpoint{3.458983in}{1.435701in}}%
\pgfpathlineto{\pgfqpoint{3.458983in}{1.439958in}}%
\pgfpathlineto{\pgfqpoint{3.463241in}{1.439958in}}%
\pgfpathlineto{\pgfqpoint{3.463241in}{1.435701in}}%
\pgfpathmoveto{\pgfqpoint{3.458983in}{1.439958in}}%
\pgfpathlineto{\pgfqpoint{3.458983in}{1.439958in}}%
\pgfpathlineto{\pgfqpoint{3.458983in}{1.444216in}}%
\pgfpathlineto{\pgfqpoint{3.463241in}{1.444216in}}%
\pgfpathlineto{\pgfqpoint{3.463241in}{1.439958in}}%
\pgfpathmoveto{\pgfqpoint{3.463241in}{1.435701in}}%
\pgfpathlineto{\pgfqpoint{3.463241in}{1.435701in}}%
\pgfpathlineto{\pgfqpoint{3.463241in}{1.439958in}}%
\pgfpathlineto{\pgfqpoint{3.467498in}{1.439958in}}%
\pgfpathlineto{\pgfqpoint{3.467498in}{1.435701in}}%
\pgfpathmoveto{\pgfqpoint{3.463241in}{1.439958in}}%
\pgfpathlineto{\pgfqpoint{3.463241in}{1.439958in}}%
\pgfpathlineto{\pgfqpoint{3.463241in}{1.444216in}}%
\pgfpathlineto{\pgfqpoint{3.467498in}{1.444216in}}%
\pgfpathlineto{\pgfqpoint{3.467498in}{1.439958in}}%
\pgfpathmoveto{\pgfqpoint{3.450467in}{1.444216in}}%
\pgfpathlineto{\pgfqpoint{3.450467in}{1.444216in}}%
\pgfpathlineto{\pgfqpoint{3.450467in}{1.448474in}}%
\pgfpathlineto{\pgfqpoint{3.454725in}{1.448474in}}%
\pgfpathlineto{\pgfqpoint{3.454725in}{1.444216in}}%
\pgfpathmoveto{\pgfqpoint{3.450467in}{1.448474in}}%
\pgfpathlineto{\pgfqpoint{3.450467in}{1.448474in}}%
\pgfpathlineto{\pgfqpoint{3.450467in}{1.452732in}}%
\pgfpathlineto{\pgfqpoint{3.454725in}{1.452732in}}%
\pgfpathlineto{\pgfqpoint{3.454725in}{1.448474in}}%
\pgfpathmoveto{\pgfqpoint{3.454725in}{1.444216in}}%
\pgfpathlineto{\pgfqpoint{3.454725in}{1.444216in}}%
\pgfpathlineto{\pgfqpoint{3.454725in}{1.448474in}}%
\pgfpathlineto{\pgfqpoint{3.458983in}{1.448474in}}%
\pgfpathlineto{\pgfqpoint{3.458983in}{1.444216in}}%
\pgfpathmoveto{\pgfqpoint{3.454725in}{1.448474in}}%
\pgfpathlineto{\pgfqpoint{3.454725in}{1.448474in}}%
\pgfpathlineto{\pgfqpoint{3.454725in}{1.452732in}}%
\pgfpathlineto{\pgfqpoint{3.458983in}{1.452732in}}%
\pgfpathlineto{\pgfqpoint{3.458983in}{1.448474in}}%
\pgfpathmoveto{\pgfqpoint{3.458983in}{1.444216in}}%
\pgfpathlineto{\pgfqpoint{3.458983in}{1.444216in}}%
\pgfpathlineto{\pgfqpoint{3.458983in}{1.448474in}}%
\pgfpathlineto{\pgfqpoint{3.463241in}{1.448474in}}%
\pgfpathlineto{\pgfqpoint{3.463241in}{1.444216in}}%
\pgfpathmoveto{\pgfqpoint{3.458983in}{1.448474in}}%
\pgfpathlineto{\pgfqpoint{3.458983in}{1.448474in}}%
\pgfpathlineto{\pgfqpoint{3.458983in}{1.452732in}}%
\pgfpathlineto{\pgfqpoint{3.463241in}{1.452732in}}%
\pgfpathlineto{\pgfqpoint{3.463241in}{1.448474in}}%
\pgfpathmoveto{\pgfqpoint{3.463241in}{1.444216in}}%
\pgfpathlineto{\pgfqpoint{3.463241in}{1.444216in}}%
\pgfpathlineto{\pgfqpoint{3.463241in}{1.448474in}}%
\pgfpathlineto{\pgfqpoint{3.467498in}{1.448474in}}%
\pgfpathlineto{\pgfqpoint{3.467498in}{1.444216in}}%
\pgfpathmoveto{\pgfqpoint{3.463241in}{1.448474in}}%
\pgfpathlineto{\pgfqpoint{3.463241in}{1.448474in}}%
\pgfpathlineto{\pgfqpoint{3.463241in}{1.452732in}}%
\pgfpathlineto{\pgfqpoint{3.467498in}{1.452732in}}%
\pgfpathlineto{\pgfqpoint{3.467498in}{1.448474in}}%
\pgfpathmoveto{\pgfqpoint{3.458983in}{1.452732in}}%
\pgfpathlineto{\pgfqpoint{3.458983in}{1.452732in}}%
\pgfpathlineto{\pgfqpoint{3.458983in}{1.456990in}}%
\pgfpathlineto{\pgfqpoint{3.463241in}{1.456990in}}%
\pgfpathlineto{\pgfqpoint{3.463241in}{1.452732in}}%
\pgfpathmoveto{\pgfqpoint{3.463241in}{1.452732in}}%
\pgfpathlineto{\pgfqpoint{3.463241in}{1.452732in}}%
\pgfpathlineto{\pgfqpoint{3.463241in}{1.456990in}}%
\pgfpathlineto{\pgfqpoint{3.467498in}{1.456990in}}%
\pgfpathlineto{\pgfqpoint{3.467498in}{1.452732in}}%
\pgfpathmoveto{\pgfqpoint{3.467498in}{1.418669in}}%
\pgfpathlineto{\pgfqpoint{3.467498in}{1.418669in}}%
\pgfpathlineto{\pgfqpoint{3.467498in}{1.422927in}}%
\pgfpathlineto{\pgfqpoint{3.471756in}{1.422927in}}%
\pgfpathlineto{\pgfqpoint{3.471756in}{1.418669in}}%
\pgfpathmoveto{\pgfqpoint{3.467498in}{1.422927in}}%
\pgfpathlineto{\pgfqpoint{3.467498in}{1.422927in}}%
\pgfpathlineto{\pgfqpoint{3.467498in}{1.427185in}}%
\pgfpathlineto{\pgfqpoint{3.471756in}{1.427185in}}%
\pgfpathlineto{\pgfqpoint{3.471756in}{1.422927in}}%
\pgfpathmoveto{\pgfqpoint{3.471756in}{1.418669in}}%
\pgfpathlineto{\pgfqpoint{3.471756in}{1.418669in}}%
\pgfpathlineto{\pgfqpoint{3.471756in}{1.422927in}}%
\pgfpathlineto{\pgfqpoint{3.476014in}{1.422927in}}%
\pgfpathlineto{\pgfqpoint{3.476014in}{1.418669in}}%
\pgfpathmoveto{\pgfqpoint{3.471756in}{1.422927in}}%
\pgfpathlineto{\pgfqpoint{3.471756in}{1.422927in}}%
\pgfpathlineto{\pgfqpoint{3.471756in}{1.427185in}}%
\pgfpathlineto{\pgfqpoint{3.476014in}{1.427185in}}%
\pgfpathlineto{\pgfqpoint{3.476014in}{1.422927in}}%
\pgfpathmoveto{\pgfqpoint{3.476014in}{1.422927in}}%
\pgfpathlineto{\pgfqpoint{3.476014in}{1.422927in}}%
\pgfpathlineto{\pgfqpoint{3.476014in}{1.427185in}}%
\pgfpathlineto{\pgfqpoint{3.480272in}{1.427185in}}%
\pgfpathlineto{\pgfqpoint{3.480272in}{1.422927in}}%
\pgfpathmoveto{\pgfqpoint{3.480272in}{1.422927in}}%
\pgfpathlineto{\pgfqpoint{3.480272in}{1.422927in}}%
\pgfpathlineto{\pgfqpoint{3.480272in}{1.427185in}}%
\pgfpathlineto{\pgfqpoint{3.484530in}{1.427185in}}%
\pgfpathlineto{\pgfqpoint{3.484530in}{1.422927in}}%
\pgfpathmoveto{\pgfqpoint{3.467498in}{1.427185in}}%
\pgfpathlineto{\pgfqpoint{3.467498in}{1.427185in}}%
\pgfpathlineto{\pgfqpoint{3.467498in}{1.431443in}}%
\pgfpathlineto{\pgfqpoint{3.471756in}{1.431443in}}%
\pgfpathlineto{\pgfqpoint{3.471756in}{1.427185in}}%
\pgfpathmoveto{\pgfqpoint{3.467498in}{1.431443in}}%
\pgfpathlineto{\pgfqpoint{3.467498in}{1.431443in}}%
\pgfpathlineto{\pgfqpoint{3.467498in}{1.435701in}}%
\pgfpathlineto{\pgfqpoint{3.471756in}{1.435701in}}%
\pgfpathlineto{\pgfqpoint{3.471756in}{1.431443in}}%
\pgfpathmoveto{\pgfqpoint{3.471756in}{1.427185in}}%
\pgfpathlineto{\pgfqpoint{3.471756in}{1.427185in}}%
\pgfpathlineto{\pgfqpoint{3.471756in}{1.431443in}}%
\pgfpathlineto{\pgfqpoint{3.476014in}{1.431443in}}%
\pgfpathlineto{\pgfqpoint{3.476014in}{1.427185in}}%
\pgfpathmoveto{\pgfqpoint{3.471756in}{1.431443in}}%
\pgfpathlineto{\pgfqpoint{3.471756in}{1.431443in}}%
\pgfpathlineto{\pgfqpoint{3.471756in}{1.435701in}}%
\pgfpathlineto{\pgfqpoint{3.476014in}{1.435701in}}%
\pgfpathlineto{\pgfqpoint{3.476014in}{1.431443in}}%
\pgfpathmoveto{\pgfqpoint{3.467498in}{1.435701in}}%
\pgfpathlineto{\pgfqpoint{3.467498in}{1.435701in}}%
\pgfpathlineto{\pgfqpoint{3.467498in}{1.439958in}}%
\pgfpathlineto{\pgfqpoint{3.471756in}{1.439958in}}%
\pgfpathlineto{\pgfqpoint{3.471756in}{1.435701in}}%
\pgfpathmoveto{\pgfqpoint{3.467498in}{1.439958in}}%
\pgfpathlineto{\pgfqpoint{3.467498in}{1.439958in}}%
\pgfpathlineto{\pgfqpoint{3.467498in}{1.444216in}}%
\pgfpathlineto{\pgfqpoint{3.471756in}{1.444216in}}%
\pgfpathlineto{\pgfqpoint{3.471756in}{1.439958in}}%
\pgfpathmoveto{\pgfqpoint{3.471756in}{1.435701in}}%
\pgfpathlineto{\pgfqpoint{3.471756in}{1.435701in}}%
\pgfpathlineto{\pgfqpoint{3.471756in}{1.439958in}}%
\pgfpathlineto{\pgfqpoint{3.476014in}{1.439958in}}%
\pgfpathlineto{\pgfqpoint{3.476014in}{1.435701in}}%
\pgfpathmoveto{\pgfqpoint{3.471756in}{1.439958in}}%
\pgfpathlineto{\pgfqpoint{3.471756in}{1.439958in}}%
\pgfpathlineto{\pgfqpoint{3.471756in}{1.444216in}}%
\pgfpathlineto{\pgfqpoint{3.476014in}{1.444216in}}%
\pgfpathlineto{\pgfqpoint{3.476014in}{1.439958in}}%
\pgfpathmoveto{\pgfqpoint{3.476014in}{1.427185in}}%
\pgfpathlineto{\pgfqpoint{3.476014in}{1.427185in}}%
\pgfpathlineto{\pgfqpoint{3.476014in}{1.431443in}}%
\pgfpathlineto{\pgfqpoint{3.480272in}{1.431443in}}%
\pgfpathlineto{\pgfqpoint{3.480272in}{1.427185in}}%
\pgfpathmoveto{\pgfqpoint{3.476014in}{1.431443in}}%
\pgfpathlineto{\pgfqpoint{3.476014in}{1.431443in}}%
\pgfpathlineto{\pgfqpoint{3.476014in}{1.435701in}}%
\pgfpathlineto{\pgfqpoint{3.480272in}{1.435701in}}%
\pgfpathlineto{\pgfqpoint{3.480272in}{1.431443in}}%
\pgfpathmoveto{\pgfqpoint{3.480272in}{1.427185in}}%
\pgfpathlineto{\pgfqpoint{3.480272in}{1.427185in}}%
\pgfpathlineto{\pgfqpoint{3.480272in}{1.431443in}}%
\pgfpathlineto{\pgfqpoint{3.484530in}{1.431443in}}%
\pgfpathlineto{\pgfqpoint{3.484530in}{1.427185in}}%
\pgfpathmoveto{\pgfqpoint{3.480272in}{1.431443in}}%
\pgfpathlineto{\pgfqpoint{3.480272in}{1.431443in}}%
\pgfpathlineto{\pgfqpoint{3.480272in}{1.435701in}}%
\pgfpathlineto{\pgfqpoint{3.484530in}{1.435701in}}%
\pgfpathlineto{\pgfqpoint{3.484530in}{1.431443in}}%
\pgfpathmoveto{\pgfqpoint{3.476014in}{1.435701in}}%
\pgfpathlineto{\pgfqpoint{3.476014in}{1.435701in}}%
\pgfpathlineto{\pgfqpoint{3.476014in}{1.439958in}}%
\pgfpathlineto{\pgfqpoint{3.480272in}{1.439958in}}%
\pgfpathlineto{\pgfqpoint{3.480272in}{1.435701in}}%
\pgfpathmoveto{\pgfqpoint{3.476014in}{1.439958in}}%
\pgfpathlineto{\pgfqpoint{3.476014in}{1.439958in}}%
\pgfpathlineto{\pgfqpoint{3.476014in}{1.444216in}}%
\pgfpathlineto{\pgfqpoint{3.480272in}{1.444216in}}%
\pgfpathlineto{\pgfqpoint{3.480272in}{1.439958in}}%
\pgfpathmoveto{\pgfqpoint{3.480272in}{1.435701in}}%
\pgfpathlineto{\pgfqpoint{3.480272in}{1.435701in}}%
\pgfpathlineto{\pgfqpoint{3.480272in}{1.439958in}}%
\pgfpathlineto{\pgfqpoint{3.484530in}{1.439958in}}%
\pgfpathlineto{\pgfqpoint{3.484530in}{1.435701in}}%
\pgfpathmoveto{\pgfqpoint{3.480272in}{1.439958in}}%
\pgfpathlineto{\pgfqpoint{3.480272in}{1.439958in}}%
\pgfpathlineto{\pgfqpoint{3.480272in}{1.444216in}}%
\pgfpathlineto{\pgfqpoint{3.484530in}{1.444216in}}%
\pgfpathlineto{\pgfqpoint{3.484530in}{1.439958in}}%
\pgfpathmoveto{\pgfqpoint{3.467498in}{1.444216in}}%
\pgfpathlineto{\pgfqpoint{3.467498in}{1.444216in}}%
\pgfpathlineto{\pgfqpoint{3.467498in}{1.448474in}}%
\pgfpathlineto{\pgfqpoint{3.471756in}{1.448474in}}%
\pgfpathlineto{\pgfqpoint{3.471756in}{1.444216in}}%
\pgfpathmoveto{\pgfqpoint{3.467498in}{1.448474in}}%
\pgfpathlineto{\pgfqpoint{3.467498in}{1.448474in}}%
\pgfpathlineto{\pgfqpoint{3.467498in}{1.452732in}}%
\pgfpathlineto{\pgfqpoint{3.471756in}{1.452732in}}%
\pgfpathlineto{\pgfqpoint{3.471756in}{1.448474in}}%
\pgfpathmoveto{\pgfqpoint{3.471756in}{1.444216in}}%
\pgfpathlineto{\pgfqpoint{3.471756in}{1.444216in}}%
\pgfpathlineto{\pgfqpoint{3.471756in}{1.448474in}}%
\pgfpathlineto{\pgfqpoint{3.476014in}{1.448474in}}%
\pgfpathlineto{\pgfqpoint{3.476014in}{1.444216in}}%
\pgfpathmoveto{\pgfqpoint{3.471756in}{1.448474in}}%
\pgfpathlineto{\pgfqpoint{3.471756in}{1.448474in}}%
\pgfpathlineto{\pgfqpoint{3.471756in}{1.452732in}}%
\pgfpathlineto{\pgfqpoint{3.476014in}{1.452732in}}%
\pgfpathlineto{\pgfqpoint{3.476014in}{1.448474in}}%
\pgfpathmoveto{\pgfqpoint{3.467498in}{1.452732in}}%
\pgfpathlineto{\pgfqpoint{3.467498in}{1.452732in}}%
\pgfpathlineto{\pgfqpoint{3.467498in}{1.456990in}}%
\pgfpathlineto{\pgfqpoint{3.471756in}{1.456990in}}%
\pgfpathlineto{\pgfqpoint{3.471756in}{1.452732in}}%
\pgfpathmoveto{\pgfqpoint{3.471756in}{1.452732in}}%
\pgfpathlineto{\pgfqpoint{3.471756in}{1.452732in}}%
\pgfpathlineto{\pgfqpoint{3.471756in}{1.456990in}}%
\pgfpathlineto{\pgfqpoint{3.476014in}{1.456990in}}%
\pgfpathlineto{\pgfqpoint{3.476014in}{1.452732in}}%
\pgfpathmoveto{\pgfqpoint{3.471756in}{1.456990in}}%
\pgfpathlineto{\pgfqpoint{3.471756in}{1.456990in}}%
\pgfpathlineto{\pgfqpoint{3.471756in}{1.461248in}}%
\pgfpathlineto{\pgfqpoint{3.476014in}{1.461248in}}%
\pgfpathlineto{\pgfqpoint{3.476014in}{1.456990in}}%
\pgfpathmoveto{\pgfqpoint{3.476014in}{1.444216in}}%
\pgfpathlineto{\pgfqpoint{3.476014in}{1.444216in}}%
\pgfpathlineto{\pgfqpoint{3.476014in}{1.448474in}}%
\pgfpathlineto{\pgfqpoint{3.480272in}{1.448474in}}%
\pgfpathlineto{\pgfqpoint{3.480272in}{1.444216in}}%
\pgfpathmoveto{\pgfqpoint{3.476014in}{1.448474in}}%
\pgfpathlineto{\pgfqpoint{3.476014in}{1.448474in}}%
\pgfpathlineto{\pgfqpoint{3.476014in}{1.452732in}}%
\pgfpathlineto{\pgfqpoint{3.480272in}{1.452732in}}%
\pgfpathlineto{\pgfqpoint{3.480272in}{1.448474in}}%
\pgfpathmoveto{\pgfqpoint{3.480272in}{1.444216in}}%
\pgfpathlineto{\pgfqpoint{3.480272in}{1.444216in}}%
\pgfpathlineto{\pgfqpoint{3.480272in}{1.448474in}}%
\pgfpathlineto{\pgfqpoint{3.484530in}{1.448474in}}%
\pgfpathlineto{\pgfqpoint{3.484530in}{1.444216in}}%
\pgfpathmoveto{\pgfqpoint{3.480272in}{1.448474in}}%
\pgfpathlineto{\pgfqpoint{3.480272in}{1.448474in}}%
\pgfpathlineto{\pgfqpoint{3.480272in}{1.452732in}}%
\pgfpathlineto{\pgfqpoint{3.484530in}{1.452732in}}%
\pgfpathlineto{\pgfqpoint{3.484530in}{1.448474in}}%
\pgfpathmoveto{\pgfqpoint{3.476014in}{1.452732in}}%
\pgfpathlineto{\pgfqpoint{3.476014in}{1.452732in}}%
\pgfpathlineto{\pgfqpoint{3.476014in}{1.456990in}}%
\pgfpathlineto{\pgfqpoint{3.480272in}{1.456990in}}%
\pgfpathlineto{\pgfqpoint{3.480272in}{1.452732in}}%
\pgfpathmoveto{\pgfqpoint{3.476014in}{1.456990in}}%
\pgfpathlineto{\pgfqpoint{3.476014in}{1.456990in}}%
\pgfpathlineto{\pgfqpoint{3.476014in}{1.461248in}}%
\pgfpathlineto{\pgfqpoint{3.480272in}{1.461248in}}%
\pgfpathlineto{\pgfqpoint{3.480272in}{1.456990in}}%
\pgfpathmoveto{\pgfqpoint{3.480272in}{1.452732in}}%
\pgfpathlineto{\pgfqpoint{3.480272in}{1.452732in}}%
\pgfpathlineto{\pgfqpoint{3.480272in}{1.456990in}}%
\pgfpathlineto{\pgfqpoint{3.484530in}{1.456990in}}%
\pgfpathlineto{\pgfqpoint{3.484530in}{1.452732in}}%
\pgfpathmoveto{\pgfqpoint{3.480272in}{1.456990in}}%
\pgfpathlineto{\pgfqpoint{3.480272in}{1.456990in}}%
\pgfpathlineto{\pgfqpoint{3.480272in}{1.461248in}}%
\pgfpathlineto{\pgfqpoint{3.484530in}{1.461248in}}%
\pgfpathlineto{\pgfqpoint{3.484530in}{1.456990in}}%
\pgfpathmoveto{\pgfqpoint{3.484530in}{1.427185in}}%
\pgfpathlineto{\pgfqpoint{3.484530in}{1.427185in}}%
\pgfpathlineto{\pgfqpoint{3.484530in}{1.431443in}}%
\pgfpathlineto{\pgfqpoint{3.488788in}{1.431443in}}%
\pgfpathlineto{\pgfqpoint{3.488788in}{1.427185in}}%
\pgfpathmoveto{\pgfqpoint{3.484530in}{1.431443in}}%
\pgfpathlineto{\pgfqpoint{3.484530in}{1.431443in}}%
\pgfpathlineto{\pgfqpoint{3.484530in}{1.435701in}}%
\pgfpathlineto{\pgfqpoint{3.488788in}{1.435701in}}%
\pgfpathlineto{\pgfqpoint{3.488788in}{1.431443in}}%
\pgfpathmoveto{\pgfqpoint{3.488788in}{1.431443in}}%
\pgfpathlineto{\pgfqpoint{3.488788in}{1.431443in}}%
\pgfpathlineto{\pgfqpoint{3.488788in}{1.435701in}}%
\pgfpathlineto{\pgfqpoint{3.493045in}{1.435701in}}%
\pgfpathlineto{\pgfqpoint{3.493045in}{1.431443in}}%
\pgfpathmoveto{\pgfqpoint{3.484530in}{1.435701in}}%
\pgfpathlineto{\pgfqpoint{3.484530in}{1.435701in}}%
\pgfpathlineto{\pgfqpoint{3.484530in}{1.439958in}}%
\pgfpathlineto{\pgfqpoint{3.488788in}{1.439958in}}%
\pgfpathlineto{\pgfqpoint{3.488788in}{1.435701in}}%
\pgfpathmoveto{\pgfqpoint{3.484530in}{1.439958in}}%
\pgfpathlineto{\pgfqpoint{3.484530in}{1.439958in}}%
\pgfpathlineto{\pgfqpoint{3.484530in}{1.444216in}}%
\pgfpathlineto{\pgfqpoint{3.488788in}{1.444216in}}%
\pgfpathlineto{\pgfqpoint{3.488788in}{1.439958in}}%
\pgfpathmoveto{\pgfqpoint{3.488788in}{1.435701in}}%
\pgfpathlineto{\pgfqpoint{3.488788in}{1.435701in}}%
\pgfpathlineto{\pgfqpoint{3.488788in}{1.439958in}}%
\pgfpathlineto{\pgfqpoint{3.493045in}{1.439958in}}%
\pgfpathlineto{\pgfqpoint{3.493045in}{1.435701in}}%
\pgfpathmoveto{\pgfqpoint{3.488788in}{1.439958in}}%
\pgfpathlineto{\pgfqpoint{3.488788in}{1.439958in}}%
\pgfpathlineto{\pgfqpoint{3.488788in}{1.444216in}}%
\pgfpathlineto{\pgfqpoint{3.493045in}{1.444216in}}%
\pgfpathlineto{\pgfqpoint{3.493045in}{1.439958in}}%
\pgfpathmoveto{\pgfqpoint{3.493045in}{1.435701in}}%
\pgfpathlineto{\pgfqpoint{3.493045in}{1.435701in}}%
\pgfpathlineto{\pgfqpoint{3.493045in}{1.439958in}}%
\pgfpathlineto{\pgfqpoint{3.497303in}{1.439958in}}%
\pgfpathlineto{\pgfqpoint{3.497303in}{1.435701in}}%
\pgfpathmoveto{\pgfqpoint{3.493045in}{1.439958in}}%
\pgfpathlineto{\pgfqpoint{3.493045in}{1.439958in}}%
\pgfpathlineto{\pgfqpoint{3.493045in}{1.444216in}}%
\pgfpathlineto{\pgfqpoint{3.497303in}{1.444216in}}%
\pgfpathlineto{\pgfqpoint{3.497303in}{1.439958in}}%
\pgfpathmoveto{\pgfqpoint{3.497303in}{1.439958in}}%
\pgfpathlineto{\pgfqpoint{3.497303in}{1.439958in}}%
\pgfpathlineto{\pgfqpoint{3.497303in}{1.444216in}}%
\pgfpathlineto{\pgfqpoint{3.501561in}{1.444216in}}%
\pgfpathlineto{\pgfqpoint{3.501561in}{1.439958in}}%
\pgfpathmoveto{\pgfqpoint{3.484530in}{1.444216in}}%
\pgfpathlineto{\pgfqpoint{3.484530in}{1.444216in}}%
\pgfpathlineto{\pgfqpoint{3.484530in}{1.448474in}}%
\pgfpathlineto{\pgfqpoint{3.488788in}{1.448474in}}%
\pgfpathlineto{\pgfqpoint{3.488788in}{1.444216in}}%
\pgfpathmoveto{\pgfqpoint{3.484530in}{1.448474in}}%
\pgfpathlineto{\pgfqpoint{3.484530in}{1.448474in}}%
\pgfpathlineto{\pgfqpoint{3.484530in}{1.452732in}}%
\pgfpathlineto{\pgfqpoint{3.488788in}{1.452732in}}%
\pgfpathlineto{\pgfqpoint{3.488788in}{1.448474in}}%
\pgfpathmoveto{\pgfqpoint{3.488788in}{1.444216in}}%
\pgfpathlineto{\pgfqpoint{3.488788in}{1.444216in}}%
\pgfpathlineto{\pgfqpoint{3.488788in}{1.448474in}}%
\pgfpathlineto{\pgfqpoint{3.493045in}{1.448474in}}%
\pgfpathlineto{\pgfqpoint{3.493045in}{1.444216in}}%
\pgfpathmoveto{\pgfqpoint{3.488788in}{1.448474in}}%
\pgfpathlineto{\pgfqpoint{3.488788in}{1.448474in}}%
\pgfpathlineto{\pgfqpoint{3.488788in}{1.452732in}}%
\pgfpathlineto{\pgfqpoint{3.493045in}{1.452732in}}%
\pgfpathlineto{\pgfqpoint{3.493045in}{1.448474in}}%
\pgfpathmoveto{\pgfqpoint{3.484530in}{1.452732in}}%
\pgfpathlineto{\pgfqpoint{3.484530in}{1.452732in}}%
\pgfpathlineto{\pgfqpoint{3.484530in}{1.456990in}}%
\pgfpathlineto{\pgfqpoint{3.488788in}{1.456990in}}%
\pgfpathlineto{\pgfqpoint{3.488788in}{1.452732in}}%
\pgfpathmoveto{\pgfqpoint{3.484530in}{1.456990in}}%
\pgfpathlineto{\pgfqpoint{3.484530in}{1.456990in}}%
\pgfpathlineto{\pgfqpoint{3.484530in}{1.461248in}}%
\pgfpathlineto{\pgfqpoint{3.488788in}{1.461248in}}%
\pgfpathlineto{\pgfqpoint{3.488788in}{1.456990in}}%
\pgfpathmoveto{\pgfqpoint{3.488788in}{1.452732in}}%
\pgfpathlineto{\pgfqpoint{3.488788in}{1.452732in}}%
\pgfpathlineto{\pgfqpoint{3.488788in}{1.456990in}}%
\pgfpathlineto{\pgfqpoint{3.493045in}{1.456990in}}%
\pgfpathlineto{\pgfqpoint{3.493045in}{1.452732in}}%
\pgfpathmoveto{\pgfqpoint{3.488788in}{1.456990in}}%
\pgfpathlineto{\pgfqpoint{3.488788in}{1.456990in}}%
\pgfpathlineto{\pgfqpoint{3.488788in}{1.461248in}}%
\pgfpathlineto{\pgfqpoint{3.493045in}{1.461248in}}%
\pgfpathlineto{\pgfqpoint{3.493045in}{1.456990in}}%
\pgfpathmoveto{\pgfqpoint{3.493045in}{1.444216in}}%
\pgfpathlineto{\pgfqpoint{3.493045in}{1.444216in}}%
\pgfpathlineto{\pgfqpoint{3.493045in}{1.448474in}}%
\pgfpathlineto{\pgfqpoint{3.497303in}{1.448474in}}%
\pgfpathlineto{\pgfqpoint{3.497303in}{1.444216in}}%
\pgfpathmoveto{\pgfqpoint{3.493045in}{1.448474in}}%
\pgfpathlineto{\pgfqpoint{3.493045in}{1.448474in}}%
\pgfpathlineto{\pgfqpoint{3.493045in}{1.452732in}}%
\pgfpathlineto{\pgfqpoint{3.497303in}{1.452732in}}%
\pgfpathlineto{\pgfqpoint{3.497303in}{1.448474in}}%
\pgfpathmoveto{\pgfqpoint{3.497303in}{1.444216in}}%
\pgfpathlineto{\pgfqpoint{3.497303in}{1.444216in}}%
\pgfpathlineto{\pgfqpoint{3.497303in}{1.448474in}}%
\pgfpathlineto{\pgfqpoint{3.501561in}{1.448474in}}%
\pgfpathlineto{\pgfqpoint{3.501561in}{1.444216in}}%
\pgfpathmoveto{\pgfqpoint{3.497303in}{1.448474in}}%
\pgfpathlineto{\pgfqpoint{3.497303in}{1.448474in}}%
\pgfpathlineto{\pgfqpoint{3.497303in}{1.452732in}}%
\pgfpathlineto{\pgfqpoint{3.501561in}{1.452732in}}%
\pgfpathlineto{\pgfqpoint{3.501561in}{1.448474in}}%
\pgfpathmoveto{\pgfqpoint{3.493045in}{1.452732in}}%
\pgfpathlineto{\pgfqpoint{3.493045in}{1.452732in}}%
\pgfpathlineto{\pgfqpoint{3.493045in}{1.456990in}}%
\pgfpathlineto{\pgfqpoint{3.497303in}{1.456990in}}%
\pgfpathlineto{\pgfqpoint{3.497303in}{1.452732in}}%
\pgfpathmoveto{\pgfqpoint{3.493045in}{1.456990in}}%
\pgfpathlineto{\pgfqpoint{3.493045in}{1.456990in}}%
\pgfpathlineto{\pgfqpoint{3.493045in}{1.461248in}}%
\pgfpathlineto{\pgfqpoint{3.497303in}{1.461248in}}%
\pgfpathlineto{\pgfqpoint{3.497303in}{1.456990in}}%
\pgfpathmoveto{\pgfqpoint{3.497303in}{1.452732in}}%
\pgfpathlineto{\pgfqpoint{3.497303in}{1.452732in}}%
\pgfpathlineto{\pgfqpoint{3.497303in}{1.456990in}}%
\pgfpathlineto{\pgfqpoint{3.501561in}{1.456990in}}%
\pgfpathlineto{\pgfqpoint{3.501561in}{1.452732in}}%
\pgfpathmoveto{\pgfqpoint{3.497303in}{1.456990in}}%
\pgfpathlineto{\pgfqpoint{3.497303in}{1.456990in}}%
\pgfpathlineto{\pgfqpoint{3.497303in}{1.461248in}}%
\pgfpathlineto{\pgfqpoint{3.501561in}{1.461248in}}%
\pgfpathlineto{\pgfqpoint{3.501561in}{1.456990in}}%
\pgfpathmoveto{\pgfqpoint{3.369568in}{1.469764in}}%
\pgfpathlineto{\pgfqpoint{3.369568in}{1.469764in}}%
\pgfpathlineto{\pgfqpoint{3.369568in}{1.474022in}}%
\pgfpathlineto{\pgfqpoint{3.373826in}{1.474022in}}%
\pgfpathlineto{\pgfqpoint{3.373826in}{1.469764in}}%
\pgfpathmoveto{\pgfqpoint{3.369568in}{1.474022in}}%
\pgfpathlineto{\pgfqpoint{3.369568in}{1.474022in}}%
\pgfpathlineto{\pgfqpoint{3.369568in}{1.478280in}}%
\pgfpathlineto{\pgfqpoint{3.373826in}{1.478280in}}%
\pgfpathlineto{\pgfqpoint{3.373826in}{1.474022in}}%
\pgfpathmoveto{\pgfqpoint{3.373826in}{1.465506in}}%
\pgfpathlineto{\pgfqpoint{3.373826in}{1.465506in}}%
\pgfpathlineto{\pgfqpoint{3.373826in}{1.469764in}}%
\pgfpathlineto{\pgfqpoint{3.378084in}{1.469764in}}%
\pgfpathlineto{\pgfqpoint{3.378084in}{1.465506in}}%
\pgfpathmoveto{\pgfqpoint{3.378084in}{1.461248in}}%
\pgfpathlineto{\pgfqpoint{3.378084in}{1.461248in}}%
\pgfpathlineto{\pgfqpoint{3.378084in}{1.465506in}}%
\pgfpathlineto{\pgfqpoint{3.382342in}{1.465506in}}%
\pgfpathlineto{\pgfqpoint{3.382342in}{1.461248in}}%
\pgfpathmoveto{\pgfqpoint{3.378084in}{1.465506in}}%
\pgfpathlineto{\pgfqpoint{3.378084in}{1.465506in}}%
\pgfpathlineto{\pgfqpoint{3.378084in}{1.469764in}}%
\pgfpathlineto{\pgfqpoint{3.382342in}{1.469764in}}%
\pgfpathlineto{\pgfqpoint{3.382342in}{1.465506in}}%
\pgfpathmoveto{\pgfqpoint{3.373826in}{1.469764in}}%
\pgfpathlineto{\pgfqpoint{3.373826in}{1.469764in}}%
\pgfpathlineto{\pgfqpoint{3.373826in}{1.474022in}}%
\pgfpathlineto{\pgfqpoint{3.378084in}{1.474022in}}%
\pgfpathlineto{\pgfqpoint{3.378084in}{1.469764in}}%
\pgfpathmoveto{\pgfqpoint{3.373826in}{1.474022in}}%
\pgfpathlineto{\pgfqpoint{3.373826in}{1.474022in}}%
\pgfpathlineto{\pgfqpoint{3.373826in}{1.478280in}}%
\pgfpathlineto{\pgfqpoint{3.378084in}{1.478280in}}%
\pgfpathlineto{\pgfqpoint{3.378084in}{1.474022in}}%
\pgfpathmoveto{\pgfqpoint{3.378084in}{1.469764in}}%
\pgfpathlineto{\pgfqpoint{3.378084in}{1.469764in}}%
\pgfpathlineto{\pgfqpoint{3.378084in}{1.474022in}}%
\pgfpathlineto{\pgfqpoint{3.382342in}{1.474022in}}%
\pgfpathlineto{\pgfqpoint{3.382342in}{1.469764in}}%
\pgfpathmoveto{\pgfqpoint{3.378084in}{1.474022in}}%
\pgfpathlineto{\pgfqpoint{3.378084in}{1.474022in}}%
\pgfpathlineto{\pgfqpoint{3.378084in}{1.478280in}}%
\pgfpathlineto{\pgfqpoint{3.382342in}{1.478280in}}%
\pgfpathlineto{\pgfqpoint{3.382342in}{1.474022in}}%
\pgfpathmoveto{\pgfqpoint{3.365311in}{1.478280in}}%
\pgfpathlineto{\pgfqpoint{3.365311in}{1.478280in}}%
\pgfpathlineto{\pgfqpoint{3.365311in}{1.482538in}}%
\pgfpathlineto{\pgfqpoint{3.369568in}{1.482538in}}%
\pgfpathlineto{\pgfqpoint{3.369568in}{1.478280in}}%
\pgfpathmoveto{\pgfqpoint{3.365311in}{1.482538in}}%
\pgfpathlineto{\pgfqpoint{3.365311in}{1.482538in}}%
\pgfpathlineto{\pgfqpoint{3.365311in}{1.486796in}}%
\pgfpathlineto{\pgfqpoint{3.369568in}{1.486796in}}%
\pgfpathlineto{\pgfqpoint{3.369568in}{1.482538in}}%
\pgfpathmoveto{\pgfqpoint{3.369568in}{1.478280in}}%
\pgfpathlineto{\pgfqpoint{3.369568in}{1.478280in}}%
\pgfpathlineto{\pgfqpoint{3.369568in}{1.482538in}}%
\pgfpathlineto{\pgfqpoint{3.373826in}{1.482538in}}%
\pgfpathlineto{\pgfqpoint{3.373826in}{1.478280in}}%
\pgfpathmoveto{\pgfqpoint{3.369568in}{1.482538in}}%
\pgfpathlineto{\pgfqpoint{3.369568in}{1.482538in}}%
\pgfpathlineto{\pgfqpoint{3.369568in}{1.486796in}}%
\pgfpathlineto{\pgfqpoint{3.373826in}{1.486796in}}%
\pgfpathlineto{\pgfqpoint{3.373826in}{1.482538in}}%
\pgfpathmoveto{\pgfqpoint{3.365311in}{1.486796in}}%
\pgfpathlineto{\pgfqpoint{3.365311in}{1.486796in}}%
\pgfpathlineto{\pgfqpoint{3.365311in}{1.491054in}}%
\pgfpathlineto{\pgfqpoint{3.369568in}{1.491054in}}%
\pgfpathlineto{\pgfqpoint{3.369568in}{1.486796in}}%
\pgfpathmoveto{\pgfqpoint{3.365311in}{1.491054in}}%
\pgfpathlineto{\pgfqpoint{3.365311in}{1.491054in}}%
\pgfpathlineto{\pgfqpoint{3.365311in}{1.495312in}}%
\pgfpathlineto{\pgfqpoint{3.369568in}{1.495312in}}%
\pgfpathlineto{\pgfqpoint{3.369568in}{1.491054in}}%
\pgfpathmoveto{\pgfqpoint{3.369568in}{1.486796in}}%
\pgfpathlineto{\pgfqpoint{3.369568in}{1.486796in}}%
\pgfpathlineto{\pgfqpoint{3.369568in}{1.491054in}}%
\pgfpathlineto{\pgfqpoint{3.373826in}{1.491054in}}%
\pgfpathlineto{\pgfqpoint{3.373826in}{1.486796in}}%
\pgfpathmoveto{\pgfqpoint{3.369568in}{1.491054in}}%
\pgfpathlineto{\pgfqpoint{3.369568in}{1.491054in}}%
\pgfpathlineto{\pgfqpoint{3.369568in}{1.495312in}}%
\pgfpathlineto{\pgfqpoint{3.373826in}{1.495312in}}%
\pgfpathlineto{\pgfqpoint{3.373826in}{1.491054in}}%
\pgfpathmoveto{\pgfqpoint{3.373826in}{1.478280in}}%
\pgfpathlineto{\pgfqpoint{3.373826in}{1.478280in}}%
\pgfpathlineto{\pgfqpoint{3.373826in}{1.482538in}}%
\pgfpathlineto{\pgfqpoint{3.378084in}{1.482538in}}%
\pgfpathlineto{\pgfqpoint{3.378084in}{1.478280in}}%
\pgfpathmoveto{\pgfqpoint{3.373826in}{1.482538in}}%
\pgfpathlineto{\pgfqpoint{3.373826in}{1.482538in}}%
\pgfpathlineto{\pgfqpoint{3.373826in}{1.486796in}}%
\pgfpathlineto{\pgfqpoint{3.378084in}{1.486796in}}%
\pgfpathlineto{\pgfqpoint{3.378084in}{1.482538in}}%
\pgfpathmoveto{\pgfqpoint{3.378084in}{1.478280in}}%
\pgfpathlineto{\pgfqpoint{3.378084in}{1.478280in}}%
\pgfpathlineto{\pgfqpoint{3.378084in}{1.482538in}}%
\pgfpathlineto{\pgfqpoint{3.382342in}{1.482538in}}%
\pgfpathlineto{\pgfqpoint{3.382342in}{1.478280in}}%
\pgfpathmoveto{\pgfqpoint{3.378084in}{1.482538in}}%
\pgfpathlineto{\pgfqpoint{3.378084in}{1.482538in}}%
\pgfpathlineto{\pgfqpoint{3.378084in}{1.486796in}}%
\pgfpathlineto{\pgfqpoint{3.382342in}{1.486796in}}%
\pgfpathlineto{\pgfqpoint{3.382342in}{1.482538in}}%
\pgfpathmoveto{\pgfqpoint{3.373826in}{1.486796in}}%
\pgfpathlineto{\pgfqpoint{3.373826in}{1.486796in}}%
\pgfpathlineto{\pgfqpoint{3.373826in}{1.491054in}}%
\pgfpathlineto{\pgfqpoint{3.378084in}{1.491054in}}%
\pgfpathlineto{\pgfqpoint{3.378084in}{1.486796in}}%
\pgfpathmoveto{\pgfqpoint{3.373826in}{1.491054in}}%
\pgfpathlineto{\pgfqpoint{3.373826in}{1.491054in}}%
\pgfpathlineto{\pgfqpoint{3.373826in}{1.495312in}}%
\pgfpathlineto{\pgfqpoint{3.378084in}{1.495312in}}%
\pgfpathlineto{\pgfqpoint{3.378084in}{1.491054in}}%
\pgfpathmoveto{\pgfqpoint{3.378084in}{1.486796in}}%
\pgfpathlineto{\pgfqpoint{3.378084in}{1.486796in}}%
\pgfpathlineto{\pgfqpoint{3.378084in}{1.491054in}}%
\pgfpathlineto{\pgfqpoint{3.382342in}{1.491054in}}%
\pgfpathlineto{\pgfqpoint{3.382342in}{1.486796in}}%
\pgfpathmoveto{\pgfqpoint{3.382342in}{1.461248in}}%
\pgfpathlineto{\pgfqpoint{3.382342in}{1.461248in}}%
\pgfpathlineto{\pgfqpoint{3.382342in}{1.465506in}}%
\pgfpathlineto{\pgfqpoint{3.386600in}{1.465506in}}%
\pgfpathlineto{\pgfqpoint{3.386600in}{1.461248in}}%
\pgfpathmoveto{\pgfqpoint{3.382342in}{1.465506in}}%
\pgfpathlineto{\pgfqpoint{3.382342in}{1.465506in}}%
\pgfpathlineto{\pgfqpoint{3.382342in}{1.469764in}}%
\pgfpathlineto{\pgfqpoint{3.386600in}{1.469764in}}%
\pgfpathlineto{\pgfqpoint{3.386600in}{1.465506in}}%
\pgfpathmoveto{\pgfqpoint{3.386600in}{1.461248in}}%
\pgfpathlineto{\pgfqpoint{3.386600in}{1.461248in}}%
\pgfpathlineto{\pgfqpoint{3.386600in}{1.465506in}}%
\pgfpathlineto{\pgfqpoint{3.390858in}{1.465506in}}%
\pgfpathlineto{\pgfqpoint{3.390858in}{1.461248in}}%
\pgfpathmoveto{\pgfqpoint{3.386600in}{1.465506in}}%
\pgfpathlineto{\pgfqpoint{3.386600in}{1.465506in}}%
\pgfpathlineto{\pgfqpoint{3.386600in}{1.469764in}}%
\pgfpathlineto{\pgfqpoint{3.390858in}{1.469764in}}%
\pgfpathlineto{\pgfqpoint{3.390858in}{1.465506in}}%
\pgfpathmoveto{\pgfqpoint{3.382342in}{1.469764in}}%
\pgfpathlineto{\pgfqpoint{3.382342in}{1.469764in}}%
\pgfpathlineto{\pgfqpoint{3.382342in}{1.474022in}}%
\pgfpathlineto{\pgfqpoint{3.386600in}{1.474022in}}%
\pgfpathlineto{\pgfqpoint{3.386600in}{1.469764in}}%
\pgfpathmoveto{\pgfqpoint{3.382342in}{1.474022in}}%
\pgfpathlineto{\pgfqpoint{3.382342in}{1.474022in}}%
\pgfpathlineto{\pgfqpoint{3.382342in}{1.478280in}}%
\pgfpathlineto{\pgfqpoint{3.386600in}{1.478280in}}%
\pgfpathlineto{\pgfqpoint{3.386600in}{1.474022in}}%
\pgfpathmoveto{\pgfqpoint{3.386600in}{1.469764in}}%
\pgfpathlineto{\pgfqpoint{3.386600in}{1.469764in}}%
\pgfpathlineto{\pgfqpoint{3.386600in}{1.474022in}}%
\pgfpathlineto{\pgfqpoint{3.390858in}{1.474022in}}%
\pgfpathlineto{\pgfqpoint{3.390858in}{1.469764in}}%
\pgfpathmoveto{\pgfqpoint{3.386600in}{1.474022in}}%
\pgfpathlineto{\pgfqpoint{3.386600in}{1.474022in}}%
\pgfpathlineto{\pgfqpoint{3.386600in}{1.478280in}}%
\pgfpathlineto{\pgfqpoint{3.390858in}{1.478280in}}%
\pgfpathlineto{\pgfqpoint{3.390858in}{1.474022in}}%
\pgfpathmoveto{\pgfqpoint{3.390858in}{1.461248in}}%
\pgfpathlineto{\pgfqpoint{3.390858in}{1.461248in}}%
\pgfpathlineto{\pgfqpoint{3.390858in}{1.465506in}}%
\pgfpathlineto{\pgfqpoint{3.395115in}{1.465506in}}%
\pgfpathlineto{\pgfqpoint{3.395115in}{1.461248in}}%
\pgfpathmoveto{\pgfqpoint{3.390858in}{1.465506in}}%
\pgfpathlineto{\pgfqpoint{3.390858in}{1.465506in}}%
\pgfpathlineto{\pgfqpoint{3.390858in}{1.469764in}}%
\pgfpathlineto{\pgfqpoint{3.395115in}{1.469764in}}%
\pgfpathlineto{\pgfqpoint{3.395115in}{1.465506in}}%
\pgfpathmoveto{\pgfqpoint{3.395115in}{1.461248in}}%
\pgfpathlineto{\pgfqpoint{3.395115in}{1.461248in}}%
\pgfpathlineto{\pgfqpoint{3.395115in}{1.465506in}}%
\pgfpathlineto{\pgfqpoint{3.399373in}{1.465506in}}%
\pgfpathlineto{\pgfqpoint{3.399373in}{1.461248in}}%
\pgfpathmoveto{\pgfqpoint{3.395115in}{1.465506in}}%
\pgfpathlineto{\pgfqpoint{3.395115in}{1.465506in}}%
\pgfpathlineto{\pgfqpoint{3.395115in}{1.469764in}}%
\pgfpathlineto{\pgfqpoint{3.399373in}{1.469764in}}%
\pgfpathlineto{\pgfqpoint{3.399373in}{1.465506in}}%
\pgfpathmoveto{\pgfqpoint{3.390858in}{1.469764in}}%
\pgfpathlineto{\pgfqpoint{3.390858in}{1.469764in}}%
\pgfpathlineto{\pgfqpoint{3.390858in}{1.474022in}}%
\pgfpathlineto{\pgfqpoint{3.395115in}{1.474022in}}%
\pgfpathlineto{\pgfqpoint{3.395115in}{1.469764in}}%
\pgfpathmoveto{\pgfqpoint{3.390858in}{1.474022in}}%
\pgfpathlineto{\pgfqpoint{3.390858in}{1.474022in}}%
\pgfpathlineto{\pgfqpoint{3.390858in}{1.478280in}}%
\pgfpathlineto{\pgfqpoint{3.395115in}{1.478280in}}%
\pgfpathlineto{\pgfqpoint{3.395115in}{1.474022in}}%
\pgfpathmoveto{\pgfqpoint{3.395115in}{1.469764in}}%
\pgfpathlineto{\pgfqpoint{3.395115in}{1.469764in}}%
\pgfpathlineto{\pgfqpoint{3.395115in}{1.474022in}}%
\pgfpathlineto{\pgfqpoint{3.399373in}{1.474022in}}%
\pgfpathlineto{\pgfqpoint{3.399373in}{1.469764in}}%
\pgfpathmoveto{\pgfqpoint{3.382342in}{1.478280in}}%
\pgfpathlineto{\pgfqpoint{3.382342in}{1.478280in}}%
\pgfpathlineto{\pgfqpoint{3.382342in}{1.482538in}}%
\pgfpathlineto{\pgfqpoint{3.386600in}{1.482538in}}%
\pgfpathlineto{\pgfqpoint{3.386600in}{1.478280in}}%
\pgfpathmoveto{\pgfqpoint{3.382342in}{1.482538in}}%
\pgfpathlineto{\pgfqpoint{3.382342in}{1.482538in}}%
\pgfpathlineto{\pgfqpoint{3.382342in}{1.486796in}}%
\pgfpathlineto{\pgfqpoint{3.386600in}{1.486796in}}%
\pgfpathlineto{\pgfqpoint{3.386600in}{1.482538in}}%
\pgfpathmoveto{\pgfqpoint{3.386600in}{1.478280in}}%
\pgfpathlineto{\pgfqpoint{3.386600in}{1.478280in}}%
\pgfpathlineto{\pgfqpoint{3.386600in}{1.482538in}}%
\pgfpathlineto{\pgfqpoint{3.390858in}{1.482538in}}%
\pgfpathlineto{\pgfqpoint{3.390858in}{1.478280in}}%
\pgfpathmoveto{\pgfqpoint{3.365311in}{1.495312in}}%
\pgfpathlineto{\pgfqpoint{3.365311in}{1.495312in}}%
\pgfpathlineto{\pgfqpoint{3.365311in}{1.499570in}}%
\pgfpathlineto{\pgfqpoint{3.369568in}{1.499570in}}%
\pgfpathlineto{\pgfqpoint{3.369568in}{1.495312in}}%
\pgfpathmoveto{\pgfqpoint{3.365311in}{1.499570in}}%
\pgfpathlineto{\pgfqpoint{3.365311in}{1.499570in}}%
\pgfpathlineto{\pgfqpoint{3.365311in}{1.503828in}}%
\pgfpathlineto{\pgfqpoint{3.369568in}{1.503828in}}%
\pgfpathlineto{\pgfqpoint{3.369568in}{1.499570in}}%
\pgfpathmoveto{\pgfqpoint{3.369568in}{1.495312in}}%
\pgfpathlineto{\pgfqpoint{3.369568in}{1.495312in}}%
\pgfpathlineto{\pgfqpoint{3.369568in}{1.499570in}}%
\pgfpathlineto{\pgfqpoint{3.373826in}{1.499570in}}%
\pgfpathlineto{\pgfqpoint{3.373826in}{1.495312in}}%
\pgfpathmoveto{\pgfqpoint{3.369568in}{1.499570in}}%
\pgfpathlineto{\pgfqpoint{3.369568in}{1.499570in}}%
\pgfpathlineto{\pgfqpoint{3.369568in}{1.503828in}}%
\pgfpathlineto{\pgfqpoint{3.373826in}{1.503828in}}%
\pgfpathlineto{\pgfqpoint{3.373826in}{1.499570in}}%
\pgfpathmoveto{\pgfqpoint{3.365311in}{1.503828in}}%
\pgfpathlineto{\pgfqpoint{3.365311in}{1.503828in}}%
\pgfpathlineto{\pgfqpoint{3.365311in}{1.508086in}}%
\pgfpathlineto{\pgfqpoint{3.369568in}{1.508086in}}%
\pgfpathlineto{\pgfqpoint{3.369568in}{1.503828in}}%
\pgfpathmoveto{\pgfqpoint{3.399373in}{1.461248in}}%
\pgfpathlineto{\pgfqpoint{3.399373in}{1.461248in}}%
\pgfpathlineto{\pgfqpoint{3.399373in}{1.465506in}}%
\pgfpathlineto{\pgfqpoint{3.403631in}{1.465506in}}%
\pgfpathlineto{\pgfqpoint{3.403631in}{1.461248in}}%
\pgfpathmoveto{\pgfqpoint{3.399373in}{1.465506in}}%
\pgfpathlineto{\pgfqpoint{3.399373in}{1.465506in}}%
\pgfpathlineto{\pgfqpoint{3.399373in}{1.469764in}}%
\pgfpathlineto{\pgfqpoint{3.403631in}{1.469764in}}%
\pgfpathlineto{\pgfqpoint{3.403631in}{1.465506in}}%
\pgfpathmoveto{\pgfqpoint{3.403631in}{1.461248in}}%
\pgfpathlineto{\pgfqpoint{3.403631in}{1.461248in}}%
\pgfpathlineto{\pgfqpoint{3.403631in}{1.465506in}}%
\pgfpathlineto{\pgfqpoint{3.407889in}{1.465506in}}%
\pgfpathlineto{\pgfqpoint{3.407889in}{1.461248in}}%
\pgfpathmoveto{\pgfqpoint{3.407889in}{1.461248in}}%
\pgfpathlineto{\pgfqpoint{3.407889in}{1.461248in}}%
\pgfpathlineto{\pgfqpoint{3.407889in}{1.465506in}}%
\pgfpathlineto{\pgfqpoint{3.412147in}{1.465506in}}%
\pgfpathlineto{\pgfqpoint{3.412147in}{1.461248in}}%
\pgfpathmoveto{\pgfqpoint{3.480272in}{1.461248in}}%
\pgfpathlineto{\pgfqpoint{3.480272in}{1.461248in}}%
\pgfpathlineto{\pgfqpoint{3.480272in}{1.465506in}}%
\pgfpathlineto{\pgfqpoint{3.484530in}{1.465506in}}%
\pgfpathlineto{\pgfqpoint{3.484530in}{1.461248in}}%
\pgfpathmoveto{\pgfqpoint{3.484530in}{1.461248in}}%
\pgfpathlineto{\pgfqpoint{3.484530in}{1.461248in}}%
\pgfpathlineto{\pgfqpoint{3.484530in}{1.465506in}}%
\pgfpathlineto{\pgfqpoint{3.488788in}{1.465506in}}%
\pgfpathlineto{\pgfqpoint{3.488788in}{1.461248in}}%
\pgfpathmoveto{\pgfqpoint{3.484530in}{1.465506in}}%
\pgfpathlineto{\pgfqpoint{3.484530in}{1.465506in}}%
\pgfpathlineto{\pgfqpoint{3.484530in}{1.469764in}}%
\pgfpathlineto{\pgfqpoint{3.488788in}{1.469764in}}%
\pgfpathlineto{\pgfqpoint{3.488788in}{1.465506in}}%
\pgfpathmoveto{\pgfqpoint{3.488788in}{1.461248in}}%
\pgfpathlineto{\pgfqpoint{3.488788in}{1.461248in}}%
\pgfpathlineto{\pgfqpoint{3.488788in}{1.465506in}}%
\pgfpathlineto{\pgfqpoint{3.493045in}{1.465506in}}%
\pgfpathlineto{\pgfqpoint{3.493045in}{1.461248in}}%
\pgfpathmoveto{\pgfqpoint{3.488788in}{1.465506in}}%
\pgfpathlineto{\pgfqpoint{3.488788in}{1.465506in}}%
\pgfpathlineto{\pgfqpoint{3.488788in}{1.469764in}}%
\pgfpathlineto{\pgfqpoint{3.493045in}{1.469764in}}%
\pgfpathlineto{\pgfqpoint{3.493045in}{1.465506in}}%
\pgfpathmoveto{\pgfqpoint{3.488788in}{1.469764in}}%
\pgfpathlineto{\pgfqpoint{3.488788in}{1.469764in}}%
\pgfpathlineto{\pgfqpoint{3.488788in}{1.474022in}}%
\pgfpathlineto{\pgfqpoint{3.493045in}{1.474022in}}%
\pgfpathlineto{\pgfqpoint{3.493045in}{1.469764in}}%
\pgfpathmoveto{\pgfqpoint{3.493045in}{1.461248in}}%
\pgfpathlineto{\pgfqpoint{3.493045in}{1.461248in}}%
\pgfpathlineto{\pgfqpoint{3.493045in}{1.465506in}}%
\pgfpathlineto{\pgfqpoint{3.497303in}{1.465506in}}%
\pgfpathlineto{\pgfqpoint{3.497303in}{1.461248in}}%
\pgfpathmoveto{\pgfqpoint{3.493045in}{1.465506in}}%
\pgfpathlineto{\pgfqpoint{3.493045in}{1.465506in}}%
\pgfpathlineto{\pgfqpoint{3.493045in}{1.469764in}}%
\pgfpathlineto{\pgfqpoint{3.497303in}{1.469764in}}%
\pgfpathlineto{\pgfqpoint{3.497303in}{1.465506in}}%
\pgfpathmoveto{\pgfqpoint{3.497303in}{1.461248in}}%
\pgfpathlineto{\pgfqpoint{3.497303in}{1.461248in}}%
\pgfpathlineto{\pgfqpoint{3.497303in}{1.465506in}}%
\pgfpathlineto{\pgfqpoint{3.501561in}{1.465506in}}%
\pgfpathlineto{\pgfqpoint{3.501561in}{1.461248in}}%
\pgfpathmoveto{\pgfqpoint{3.497303in}{1.465506in}}%
\pgfpathlineto{\pgfqpoint{3.497303in}{1.465506in}}%
\pgfpathlineto{\pgfqpoint{3.497303in}{1.469764in}}%
\pgfpathlineto{\pgfqpoint{3.501561in}{1.469764in}}%
\pgfpathlineto{\pgfqpoint{3.501561in}{1.465506in}}%
\pgfpathmoveto{\pgfqpoint{3.493045in}{1.469764in}}%
\pgfpathlineto{\pgfqpoint{3.493045in}{1.469764in}}%
\pgfpathlineto{\pgfqpoint{3.493045in}{1.474022in}}%
\pgfpathlineto{\pgfqpoint{3.497303in}{1.474022in}}%
\pgfpathlineto{\pgfqpoint{3.497303in}{1.469764in}}%
\pgfpathmoveto{\pgfqpoint{3.493045in}{1.474022in}}%
\pgfpathlineto{\pgfqpoint{3.493045in}{1.474022in}}%
\pgfpathlineto{\pgfqpoint{3.493045in}{1.478280in}}%
\pgfpathlineto{\pgfqpoint{3.497303in}{1.478280in}}%
\pgfpathlineto{\pgfqpoint{3.497303in}{1.474022in}}%
\pgfpathmoveto{\pgfqpoint{3.497303in}{1.469764in}}%
\pgfpathlineto{\pgfqpoint{3.497303in}{1.469764in}}%
\pgfpathlineto{\pgfqpoint{3.497303in}{1.474022in}}%
\pgfpathlineto{\pgfqpoint{3.501561in}{1.474022in}}%
\pgfpathlineto{\pgfqpoint{3.501561in}{1.469764in}}%
\pgfpathmoveto{\pgfqpoint{3.497303in}{1.474022in}}%
\pgfpathlineto{\pgfqpoint{3.497303in}{1.474022in}}%
\pgfpathlineto{\pgfqpoint{3.497303in}{1.478280in}}%
\pgfpathlineto{\pgfqpoint{3.501561in}{1.478280in}}%
\pgfpathlineto{\pgfqpoint{3.501561in}{1.474022in}}%
\pgfpathmoveto{\pgfqpoint{3.497303in}{1.478280in}}%
\pgfpathlineto{\pgfqpoint{3.497303in}{1.478280in}}%
\pgfpathlineto{\pgfqpoint{3.497303in}{1.482538in}}%
\pgfpathlineto{\pgfqpoint{3.501561in}{1.482538in}}%
\pgfpathlineto{\pgfqpoint{3.501561in}{1.478280in}}%
\pgfpathmoveto{\pgfqpoint{3.501561in}{1.444216in}}%
\pgfpathlineto{\pgfqpoint{3.501561in}{1.444216in}}%
\pgfpathlineto{\pgfqpoint{3.501561in}{1.448474in}}%
\pgfpathlineto{\pgfqpoint{3.505819in}{1.448474in}}%
\pgfpathlineto{\pgfqpoint{3.505819in}{1.444216in}}%
\pgfpathmoveto{\pgfqpoint{3.501561in}{1.448474in}}%
\pgfpathlineto{\pgfqpoint{3.501561in}{1.448474in}}%
\pgfpathlineto{\pgfqpoint{3.501561in}{1.452732in}}%
\pgfpathlineto{\pgfqpoint{3.505819in}{1.452732in}}%
\pgfpathlineto{\pgfqpoint{3.505819in}{1.448474in}}%
\pgfpathmoveto{\pgfqpoint{3.505819in}{1.448474in}}%
\pgfpathlineto{\pgfqpoint{3.505819in}{1.448474in}}%
\pgfpathlineto{\pgfqpoint{3.505819in}{1.452732in}}%
\pgfpathlineto{\pgfqpoint{3.510077in}{1.452732in}}%
\pgfpathlineto{\pgfqpoint{3.510077in}{1.448474in}}%
\pgfpathmoveto{\pgfqpoint{3.501561in}{1.452732in}}%
\pgfpathlineto{\pgfqpoint{3.501561in}{1.452732in}}%
\pgfpathlineto{\pgfqpoint{3.501561in}{1.456990in}}%
\pgfpathlineto{\pgfqpoint{3.505819in}{1.456990in}}%
\pgfpathlineto{\pgfqpoint{3.505819in}{1.452732in}}%
\pgfpathmoveto{\pgfqpoint{3.501561in}{1.456990in}}%
\pgfpathlineto{\pgfqpoint{3.501561in}{1.456990in}}%
\pgfpathlineto{\pgfqpoint{3.501561in}{1.461248in}}%
\pgfpathlineto{\pgfqpoint{3.505819in}{1.461248in}}%
\pgfpathlineto{\pgfqpoint{3.505819in}{1.456990in}}%
\pgfpathmoveto{\pgfqpoint{3.505819in}{1.452732in}}%
\pgfpathlineto{\pgfqpoint{3.505819in}{1.452732in}}%
\pgfpathlineto{\pgfqpoint{3.505819in}{1.456990in}}%
\pgfpathlineto{\pgfqpoint{3.510077in}{1.456990in}}%
\pgfpathlineto{\pgfqpoint{3.510077in}{1.452732in}}%
\pgfpathmoveto{\pgfqpoint{3.505819in}{1.456990in}}%
\pgfpathlineto{\pgfqpoint{3.505819in}{1.456990in}}%
\pgfpathlineto{\pgfqpoint{3.505819in}{1.461248in}}%
\pgfpathlineto{\pgfqpoint{3.510077in}{1.461248in}}%
\pgfpathlineto{\pgfqpoint{3.510077in}{1.456990in}}%
\pgfpathmoveto{\pgfqpoint{3.510077in}{1.452732in}}%
\pgfpathlineto{\pgfqpoint{3.510077in}{1.452732in}}%
\pgfpathlineto{\pgfqpoint{3.510077in}{1.456990in}}%
\pgfpathlineto{\pgfqpoint{3.514334in}{1.456990in}}%
\pgfpathlineto{\pgfqpoint{3.514334in}{1.452732in}}%
\pgfpathmoveto{\pgfqpoint{3.510077in}{1.456990in}}%
\pgfpathlineto{\pgfqpoint{3.510077in}{1.456990in}}%
\pgfpathlineto{\pgfqpoint{3.510077in}{1.461248in}}%
\pgfpathlineto{\pgfqpoint{3.514334in}{1.461248in}}%
\pgfpathlineto{\pgfqpoint{3.514334in}{1.456990in}}%
\pgfpathmoveto{\pgfqpoint{3.514334in}{1.456990in}}%
\pgfpathlineto{\pgfqpoint{3.514334in}{1.456990in}}%
\pgfpathlineto{\pgfqpoint{3.514334in}{1.461248in}}%
\pgfpathlineto{\pgfqpoint{3.518592in}{1.461248in}}%
\pgfpathlineto{\pgfqpoint{3.518592in}{1.456990in}}%
\pgfpathmoveto{\pgfqpoint{3.501561in}{1.461248in}}%
\pgfpathlineto{\pgfqpoint{3.501561in}{1.461248in}}%
\pgfpathlineto{\pgfqpoint{3.501561in}{1.465506in}}%
\pgfpathlineto{\pgfqpoint{3.505819in}{1.465506in}}%
\pgfpathlineto{\pgfqpoint{3.505819in}{1.461248in}}%
\pgfpathmoveto{\pgfqpoint{3.501561in}{1.465506in}}%
\pgfpathlineto{\pgfqpoint{3.501561in}{1.465506in}}%
\pgfpathlineto{\pgfqpoint{3.501561in}{1.469764in}}%
\pgfpathlineto{\pgfqpoint{3.505819in}{1.469764in}}%
\pgfpathlineto{\pgfqpoint{3.505819in}{1.465506in}}%
\pgfpathmoveto{\pgfqpoint{3.505819in}{1.461248in}}%
\pgfpathlineto{\pgfqpoint{3.505819in}{1.461248in}}%
\pgfpathlineto{\pgfqpoint{3.505819in}{1.465506in}}%
\pgfpathlineto{\pgfqpoint{3.510077in}{1.465506in}}%
\pgfpathlineto{\pgfqpoint{3.510077in}{1.461248in}}%
\pgfpathmoveto{\pgfqpoint{3.505819in}{1.465506in}}%
\pgfpathlineto{\pgfqpoint{3.505819in}{1.465506in}}%
\pgfpathlineto{\pgfqpoint{3.505819in}{1.469764in}}%
\pgfpathlineto{\pgfqpoint{3.510077in}{1.469764in}}%
\pgfpathlineto{\pgfqpoint{3.510077in}{1.465506in}}%
\pgfpathmoveto{\pgfqpoint{3.501561in}{1.469764in}}%
\pgfpathlineto{\pgfqpoint{3.501561in}{1.469764in}}%
\pgfpathlineto{\pgfqpoint{3.501561in}{1.474022in}}%
\pgfpathlineto{\pgfqpoint{3.505819in}{1.474022in}}%
\pgfpathlineto{\pgfqpoint{3.505819in}{1.469764in}}%
\pgfpathmoveto{\pgfqpoint{3.501561in}{1.474022in}}%
\pgfpathlineto{\pgfqpoint{3.501561in}{1.474022in}}%
\pgfpathlineto{\pgfqpoint{3.501561in}{1.478280in}}%
\pgfpathlineto{\pgfqpoint{3.505819in}{1.478280in}}%
\pgfpathlineto{\pgfqpoint{3.505819in}{1.474022in}}%
\pgfpathmoveto{\pgfqpoint{3.505819in}{1.469764in}}%
\pgfpathlineto{\pgfqpoint{3.505819in}{1.469764in}}%
\pgfpathlineto{\pgfqpoint{3.505819in}{1.474022in}}%
\pgfpathlineto{\pgfqpoint{3.510077in}{1.474022in}}%
\pgfpathlineto{\pgfqpoint{3.510077in}{1.469764in}}%
\pgfpathmoveto{\pgfqpoint{3.505819in}{1.474022in}}%
\pgfpathlineto{\pgfqpoint{3.505819in}{1.474022in}}%
\pgfpathlineto{\pgfqpoint{3.505819in}{1.478280in}}%
\pgfpathlineto{\pgfqpoint{3.510077in}{1.478280in}}%
\pgfpathlineto{\pgfqpoint{3.510077in}{1.474022in}}%
\pgfpathmoveto{\pgfqpoint{3.510077in}{1.461248in}}%
\pgfpathlineto{\pgfqpoint{3.510077in}{1.461248in}}%
\pgfpathlineto{\pgfqpoint{3.510077in}{1.465506in}}%
\pgfpathlineto{\pgfqpoint{3.514334in}{1.465506in}}%
\pgfpathlineto{\pgfqpoint{3.514334in}{1.461248in}}%
\pgfpathmoveto{\pgfqpoint{3.510077in}{1.465506in}}%
\pgfpathlineto{\pgfqpoint{3.510077in}{1.465506in}}%
\pgfpathlineto{\pgfqpoint{3.510077in}{1.469764in}}%
\pgfpathlineto{\pgfqpoint{3.514334in}{1.469764in}}%
\pgfpathlineto{\pgfqpoint{3.514334in}{1.465506in}}%
\pgfpathmoveto{\pgfqpoint{3.514334in}{1.461248in}}%
\pgfpathlineto{\pgfqpoint{3.514334in}{1.461248in}}%
\pgfpathlineto{\pgfqpoint{3.514334in}{1.465506in}}%
\pgfpathlineto{\pgfqpoint{3.518592in}{1.465506in}}%
\pgfpathlineto{\pgfqpoint{3.518592in}{1.461248in}}%
\pgfpathmoveto{\pgfqpoint{3.514334in}{1.465506in}}%
\pgfpathlineto{\pgfqpoint{3.514334in}{1.465506in}}%
\pgfpathlineto{\pgfqpoint{3.514334in}{1.469764in}}%
\pgfpathlineto{\pgfqpoint{3.518592in}{1.469764in}}%
\pgfpathlineto{\pgfqpoint{3.518592in}{1.465506in}}%
\pgfpathmoveto{\pgfqpoint{3.510077in}{1.469764in}}%
\pgfpathlineto{\pgfqpoint{3.510077in}{1.469764in}}%
\pgfpathlineto{\pgfqpoint{3.510077in}{1.474022in}}%
\pgfpathlineto{\pgfqpoint{3.514334in}{1.474022in}}%
\pgfpathlineto{\pgfqpoint{3.514334in}{1.469764in}}%
\pgfpathmoveto{\pgfqpoint{3.510077in}{1.474022in}}%
\pgfpathlineto{\pgfqpoint{3.510077in}{1.474022in}}%
\pgfpathlineto{\pgfqpoint{3.510077in}{1.478280in}}%
\pgfpathlineto{\pgfqpoint{3.514334in}{1.478280in}}%
\pgfpathlineto{\pgfqpoint{3.514334in}{1.474022in}}%
\pgfpathmoveto{\pgfqpoint{3.514334in}{1.469764in}}%
\pgfpathlineto{\pgfqpoint{3.514334in}{1.469764in}}%
\pgfpathlineto{\pgfqpoint{3.514334in}{1.474022in}}%
\pgfpathlineto{\pgfqpoint{3.518592in}{1.474022in}}%
\pgfpathlineto{\pgfqpoint{3.518592in}{1.469764in}}%
\pgfpathmoveto{\pgfqpoint{3.514334in}{1.474022in}}%
\pgfpathlineto{\pgfqpoint{3.514334in}{1.474022in}}%
\pgfpathlineto{\pgfqpoint{3.514334in}{1.478280in}}%
\pgfpathlineto{\pgfqpoint{3.518592in}{1.478280in}}%
\pgfpathlineto{\pgfqpoint{3.518592in}{1.474022in}}%
\pgfpathmoveto{\pgfqpoint{3.501561in}{1.478280in}}%
\pgfpathlineto{\pgfqpoint{3.501561in}{1.478280in}}%
\pgfpathlineto{\pgfqpoint{3.501561in}{1.482538in}}%
\pgfpathlineto{\pgfqpoint{3.505819in}{1.482538in}}%
\pgfpathlineto{\pgfqpoint{3.505819in}{1.478280in}}%
\pgfpathmoveto{\pgfqpoint{3.501561in}{1.482538in}}%
\pgfpathlineto{\pgfqpoint{3.501561in}{1.482538in}}%
\pgfpathlineto{\pgfqpoint{3.501561in}{1.486796in}}%
\pgfpathlineto{\pgfqpoint{3.505819in}{1.486796in}}%
\pgfpathlineto{\pgfqpoint{3.505819in}{1.482538in}}%
\pgfpathmoveto{\pgfqpoint{3.505819in}{1.478280in}}%
\pgfpathlineto{\pgfqpoint{3.505819in}{1.478280in}}%
\pgfpathlineto{\pgfqpoint{3.505819in}{1.482538in}}%
\pgfpathlineto{\pgfqpoint{3.510077in}{1.482538in}}%
\pgfpathlineto{\pgfqpoint{3.510077in}{1.478280in}}%
\pgfpathmoveto{\pgfqpoint{3.505819in}{1.482538in}}%
\pgfpathlineto{\pgfqpoint{3.505819in}{1.482538in}}%
\pgfpathlineto{\pgfqpoint{3.505819in}{1.486796in}}%
\pgfpathlineto{\pgfqpoint{3.510077in}{1.486796in}}%
\pgfpathlineto{\pgfqpoint{3.510077in}{1.482538in}}%
\pgfpathmoveto{\pgfqpoint{3.505819in}{1.486796in}}%
\pgfpathlineto{\pgfqpoint{3.505819in}{1.486796in}}%
\pgfpathlineto{\pgfqpoint{3.505819in}{1.491054in}}%
\pgfpathlineto{\pgfqpoint{3.510077in}{1.491054in}}%
\pgfpathlineto{\pgfqpoint{3.510077in}{1.486796in}}%
\pgfpathmoveto{\pgfqpoint{3.510077in}{1.478280in}}%
\pgfpathlineto{\pgfqpoint{3.510077in}{1.478280in}}%
\pgfpathlineto{\pgfqpoint{3.510077in}{1.482538in}}%
\pgfpathlineto{\pgfqpoint{3.514334in}{1.482538in}}%
\pgfpathlineto{\pgfqpoint{3.514334in}{1.478280in}}%
\pgfpathmoveto{\pgfqpoint{3.510077in}{1.482538in}}%
\pgfpathlineto{\pgfqpoint{3.510077in}{1.482538in}}%
\pgfpathlineto{\pgfqpoint{3.510077in}{1.486796in}}%
\pgfpathlineto{\pgfqpoint{3.514334in}{1.486796in}}%
\pgfpathlineto{\pgfqpoint{3.514334in}{1.482538in}}%
\pgfpathmoveto{\pgfqpoint{3.514334in}{1.478280in}}%
\pgfpathlineto{\pgfqpoint{3.514334in}{1.478280in}}%
\pgfpathlineto{\pgfqpoint{3.514334in}{1.482538in}}%
\pgfpathlineto{\pgfqpoint{3.518592in}{1.482538in}}%
\pgfpathlineto{\pgfqpoint{3.518592in}{1.478280in}}%
\pgfpathmoveto{\pgfqpoint{3.514334in}{1.482538in}}%
\pgfpathlineto{\pgfqpoint{3.514334in}{1.482538in}}%
\pgfpathlineto{\pgfqpoint{3.514334in}{1.486796in}}%
\pgfpathlineto{\pgfqpoint{3.518592in}{1.486796in}}%
\pgfpathlineto{\pgfqpoint{3.518592in}{1.482538in}}%
\pgfpathmoveto{\pgfqpoint{3.510077in}{1.486796in}}%
\pgfpathlineto{\pgfqpoint{3.510077in}{1.486796in}}%
\pgfpathlineto{\pgfqpoint{3.510077in}{1.491054in}}%
\pgfpathlineto{\pgfqpoint{3.514334in}{1.491054in}}%
\pgfpathlineto{\pgfqpoint{3.514334in}{1.486796in}}%
\pgfpathmoveto{\pgfqpoint{3.510077in}{1.491054in}}%
\pgfpathlineto{\pgfqpoint{3.510077in}{1.491054in}}%
\pgfpathlineto{\pgfqpoint{3.510077in}{1.495312in}}%
\pgfpathlineto{\pgfqpoint{3.514334in}{1.495312in}}%
\pgfpathlineto{\pgfqpoint{3.514334in}{1.491054in}}%
\pgfpathmoveto{\pgfqpoint{3.514334in}{1.486796in}}%
\pgfpathlineto{\pgfqpoint{3.514334in}{1.486796in}}%
\pgfpathlineto{\pgfqpoint{3.514334in}{1.491054in}}%
\pgfpathlineto{\pgfqpoint{3.518592in}{1.491054in}}%
\pgfpathlineto{\pgfqpoint{3.518592in}{1.486796in}}%
\pgfpathmoveto{\pgfqpoint{3.514334in}{1.491054in}}%
\pgfpathlineto{\pgfqpoint{3.514334in}{1.491054in}}%
\pgfpathlineto{\pgfqpoint{3.514334in}{1.495312in}}%
\pgfpathlineto{\pgfqpoint{3.518592in}{1.495312in}}%
\pgfpathlineto{\pgfqpoint{3.518592in}{1.491054in}}%
\pgfpathmoveto{\pgfqpoint{3.518592in}{1.461248in}}%
\pgfpathlineto{\pgfqpoint{3.518592in}{1.461248in}}%
\pgfpathlineto{\pgfqpoint{3.518592in}{1.465506in}}%
\pgfpathlineto{\pgfqpoint{3.522850in}{1.465506in}}%
\pgfpathlineto{\pgfqpoint{3.522850in}{1.461248in}}%
\pgfpathmoveto{\pgfqpoint{3.518592in}{1.465506in}}%
\pgfpathlineto{\pgfqpoint{3.518592in}{1.465506in}}%
\pgfpathlineto{\pgfqpoint{3.518592in}{1.469764in}}%
\pgfpathlineto{\pgfqpoint{3.522850in}{1.469764in}}%
\pgfpathlineto{\pgfqpoint{3.522850in}{1.465506in}}%
\pgfpathmoveto{\pgfqpoint{3.522850in}{1.465506in}}%
\pgfpathlineto{\pgfqpoint{3.522850in}{1.465506in}}%
\pgfpathlineto{\pgfqpoint{3.522850in}{1.469764in}}%
\pgfpathlineto{\pgfqpoint{3.527108in}{1.469764in}}%
\pgfpathlineto{\pgfqpoint{3.527108in}{1.465506in}}%
\pgfpathmoveto{\pgfqpoint{3.518592in}{1.469764in}}%
\pgfpathlineto{\pgfqpoint{3.518592in}{1.469764in}}%
\pgfpathlineto{\pgfqpoint{3.518592in}{1.474022in}}%
\pgfpathlineto{\pgfqpoint{3.522850in}{1.474022in}}%
\pgfpathlineto{\pgfqpoint{3.522850in}{1.469764in}}%
\pgfpathmoveto{\pgfqpoint{3.518592in}{1.474022in}}%
\pgfpathlineto{\pgfqpoint{3.518592in}{1.474022in}}%
\pgfpathlineto{\pgfqpoint{3.518592in}{1.478280in}}%
\pgfpathlineto{\pgfqpoint{3.522850in}{1.478280in}}%
\pgfpathlineto{\pgfqpoint{3.522850in}{1.474022in}}%
\pgfpathmoveto{\pgfqpoint{3.522850in}{1.469764in}}%
\pgfpathlineto{\pgfqpoint{3.522850in}{1.469764in}}%
\pgfpathlineto{\pgfqpoint{3.522850in}{1.474022in}}%
\pgfpathlineto{\pgfqpoint{3.527108in}{1.474022in}}%
\pgfpathlineto{\pgfqpoint{3.527108in}{1.469764in}}%
\pgfpathmoveto{\pgfqpoint{3.522850in}{1.474022in}}%
\pgfpathlineto{\pgfqpoint{3.522850in}{1.474022in}}%
\pgfpathlineto{\pgfqpoint{3.522850in}{1.478280in}}%
\pgfpathlineto{\pgfqpoint{3.527108in}{1.478280in}}%
\pgfpathlineto{\pgfqpoint{3.527108in}{1.474022in}}%
\pgfpathmoveto{\pgfqpoint{3.527108in}{1.474022in}}%
\pgfpathlineto{\pgfqpoint{3.527108in}{1.474022in}}%
\pgfpathlineto{\pgfqpoint{3.527108in}{1.478280in}}%
\pgfpathlineto{\pgfqpoint{3.531365in}{1.478280in}}%
\pgfpathlineto{\pgfqpoint{3.531365in}{1.474022in}}%
\pgfpathmoveto{\pgfqpoint{3.518592in}{1.478280in}}%
\pgfpathlineto{\pgfqpoint{3.518592in}{1.478280in}}%
\pgfpathlineto{\pgfqpoint{3.518592in}{1.482538in}}%
\pgfpathlineto{\pgfqpoint{3.522850in}{1.482538in}}%
\pgfpathlineto{\pgfqpoint{3.522850in}{1.478280in}}%
\pgfpathmoveto{\pgfqpoint{3.518592in}{1.482538in}}%
\pgfpathlineto{\pgfqpoint{3.518592in}{1.482538in}}%
\pgfpathlineto{\pgfqpoint{3.518592in}{1.486796in}}%
\pgfpathlineto{\pgfqpoint{3.522850in}{1.486796in}}%
\pgfpathlineto{\pgfqpoint{3.522850in}{1.482538in}}%
\pgfpathmoveto{\pgfqpoint{3.522850in}{1.478280in}}%
\pgfpathlineto{\pgfqpoint{3.522850in}{1.478280in}}%
\pgfpathlineto{\pgfqpoint{3.522850in}{1.482538in}}%
\pgfpathlineto{\pgfqpoint{3.527108in}{1.482538in}}%
\pgfpathlineto{\pgfqpoint{3.527108in}{1.478280in}}%
\pgfpathmoveto{\pgfqpoint{3.522850in}{1.482538in}}%
\pgfpathlineto{\pgfqpoint{3.522850in}{1.482538in}}%
\pgfpathlineto{\pgfqpoint{3.522850in}{1.486796in}}%
\pgfpathlineto{\pgfqpoint{3.527108in}{1.486796in}}%
\pgfpathlineto{\pgfqpoint{3.527108in}{1.482538in}}%
\pgfpathmoveto{\pgfqpoint{3.518592in}{1.486796in}}%
\pgfpathlineto{\pgfqpoint{3.518592in}{1.486796in}}%
\pgfpathlineto{\pgfqpoint{3.518592in}{1.491054in}}%
\pgfpathlineto{\pgfqpoint{3.522850in}{1.491054in}}%
\pgfpathlineto{\pgfqpoint{3.522850in}{1.486796in}}%
\pgfpathmoveto{\pgfqpoint{3.518592in}{1.491054in}}%
\pgfpathlineto{\pgfqpoint{3.518592in}{1.491054in}}%
\pgfpathlineto{\pgfqpoint{3.518592in}{1.495312in}}%
\pgfpathlineto{\pgfqpoint{3.522850in}{1.495312in}}%
\pgfpathlineto{\pgfqpoint{3.522850in}{1.491054in}}%
\pgfpathmoveto{\pgfqpoint{3.522850in}{1.486796in}}%
\pgfpathlineto{\pgfqpoint{3.522850in}{1.486796in}}%
\pgfpathlineto{\pgfqpoint{3.522850in}{1.491054in}}%
\pgfpathlineto{\pgfqpoint{3.527108in}{1.491054in}}%
\pgfpathlineto{\pgfqpoint{3.527108in}{1.486796in}}%
\pgfpathmoveto{\pgfqpoint{3.522850in}{1.491054in}}%
\pgfpathlineto{\pgfqpoint{3.522850in}{1.491054in}}%
\pgfpathlineto{\pgfqpoint{3.522850in}{1.495312in}}%
\pgfpathlineto{\pgfqpoint{3.527108in}{1.495312in}}%
\pgfpathlineto{\pgfqpoint{3.527108in}{1.491054in}}%
\pgfpathmoveto{\pgfqpoint{3.527108in}{1.478280in}}%
\pgfpathlineto{\pgfqpoint{3.527108in}{1.478280in}}%
\pgfpathlineto{\pgfqpoint{3.527108in}{1.482538in}}%
\pgfpathlineto{\pgfqpoint{3.531365in}{1.482538in}}%
\pgfpathlineto{\pgfqpoint{3.531365in}{1.478280in}}%
\pgfpathmoveto{\pgfqpoint{3.527108in}{1.482538in}}%
\pgfpathlineto{\pgfqpoint{3.527108in}{1.482538in}}%
\pgfpathlineto{\pgfqpoint{3.527108in}{1.486796in}}%
\pgfpathlineto{\pgfqpoint{3.531365in}{1.486796in}}%
\pgfpathlineto{\pgfqpoint{3.531365in}{1.482538in}}%
\pgfpathmoveto{\pgfqpoint{3.531365in}{1.478280in}}%
\pgfpathlineto{\pgfqpoint{3.531365in}{1.478280in}}%
\pgfpathlineto{\pgfqpoint{3.531365in}{1.482538in}}%
\pgfpathlineto{\pgfqpoint{3.535623in}{1.482538in}}%
\pgfpathlineto{\pgfqpoint{3.535623in}{1.478280in}}%
\pgfpathmoveto{\pgfqpoint{3.531365in}{1.482538in}}%
\pgfpathlineto{\pgfqpoint{3.531365in}{1.482538in}}%
\pgfpathlineto{\pgfqpoint{3.531365in}{1.486796in}}%
\pgfpathlineto{\pgfqpoint{3.535623in}{1.486796in}}%
\pgfpathlineto{\pgfqpoint{3.535623in}{1.482538in}}%
\pgfpathmoveto{\pgfqpoint{3.527108in}{1.486796in}}%
\pgfpathlineto{\pgfqpoint{3.527108in}{1.486796in}}%
\pgfpathlineto{\pgfqpoint{3.527108in}{1.491054in}}%
\pgfpathlineto{\pgfqpoint{3.531365in}{1.491054in}}%
\pgfpathlineto{\pgfqpoint{3.531365in}{1.486796in}}%
\pgfpathmoveto{\pgfqpoint{3.527108in}{1.491054in}}%
\pgfpathlineto{\pgfqpoint{3.527108in}{1.491054in}}%
\pgfpathlineto{\pgfqpoint{3.527108in}{1.495312in}}%
\pgfpathlineto{\pgfqpoint{3.531365in}{1.495312in}}%
\pgfpathlineto{\pgfqpoint{3.531365in}{1.491054in}}%
\pgfpathmoveto{\pgfqpoint{3.531365in}{1.486796in}}%
\pgfpathlineto{\pgfqpoint{3.531365in}{1.486796in}}%
\pgfpathlineto{\pgfqpoint{3.531365in}{1.491054in}}%
\pgfpathlineto{\pgfqpoint{3.535623in}{1.491054in}}%
\pgfpathlineto{\pgfqpoint{3.535623in}{1.486796in}}%
\pgfpathmoveto{\pgfqpoint{3.531365in}{1.491054in}}%
\pgfpathlineto{\pgfqpoint{3.531365in}{1.491054in}}%
\pgfpathlineto{\pgfqpoint{3.531365in}{1.495312in}}%
\pgfpathlineto{\pgfqpoint{3.535623in}{1.495312in}}%
\pgfpathlineto{\pgfqpoint{3.535623in}{1.491054in}}%
\pgfpathmoveto{\pgfqpoint{3.514334in}{1.495312in}}%
\pgfpathlineto{\pgfqpoint{3.514334in}{1.495312in}}%
\pgfpathlineto{\pgfqpoint{3.514334in}{1.499570in}}%
\pgfpathlineto{\pgfqpoint{3.518592in}{1.499570in}}%
\pgfpathlineto{\pgfqpoint{3.518592in}{1.495312in}}%
\pgfpathmoveto{\pgfqpoint{3.518592in}{1.495312in}}%
\pgfpathlineto{\pgfqpoint{3.518592in}{1.495312in}}%
\pgfpathlineto{\pgfqpoint{3.518592in}{1.499570in}}%
\pgfpathlineto{\pgfqpoint{3.522850in}{1.499570in}}%
\pgfpathlineto{\pgfqpoint{3.522850in}{1.495312in}}%
\pgfpathmoveto{\pgfqpoint{3.518592in}{1.499570in}}%
\pgfpathlineto{\pgfqpoint{3.518592in}{1.499570in}}%
\pgfpathlineto{\pgfqpoint{3.518592in}{1.503828in}}%
\pgfpathlineto{\pgfqpoint{3.522850in}{1.503828in}}%
\pgfpathlineto{\pgfqpoint{3.522850in}{1.499570in}}%
\pgfpathmoveto{\pgfqpoint{3.522850in}{1.495312in}}%
\pgfpathlineto{\pgfqpoint{3.522850in}{1.495312in}}%
\pgfpathlineto{\pgfqpoint{3.522850in}{1.499570in}}%
\pgfpathlineto{\pgfqpoint{3.527108in}{1.499570in}}%
\pgfpathlineto{\pgfqpoint{3.527108in}{1.495312in}}%
\pgfpathmoveto{\pgfqpoint{3.522850in}{1.499570in}}%
\pgfpathlineto{\pgfqpoint{3.522850in}{1.499570in}}%
\pgfpathlineto{\pgfqpoint{3.522850in}{1.503828in}}%
\pgfpathlineto{\pgfqpoint{3.527108in}{1.503828in}}%
\pgfpathlineto{\pgfqpoint{3.527108in}{1.499570in}}%
\pgfpathmoveto{\pgfqpoint{3.522850in}{1.503828in}}%
\pgfpathlineto{\pgfqpoint{3.522850in}{1.503828in}}%
\pgfpathlineto{\pgfqpoint{3.522850in}{1.508086in}}%
\pgfpathlineto{\pgfqpoint{3.527108in}{1.508086in}}%
\pgfpathlineto{\pgfqpoint{3.527108in}{1.503828in}}%
\pgfpathmoveto{\pgfqpoint{3.522850in}{1.508086in}}%
\pgfpathlineto{\pgfqpoint{3.522850in}{1.508086in}}%
\pgfpathlineto{\pgfqpoint{3.522850in}{1.512344in}}%
\pgfpathlineto{\pgfqpoint{3.527108in}{1.512344in}}%
\pgfpathlineto{\pgfqpoint{3.527108in}{1.508086in}}%
\pgfpathmoveto{\pgfqpoint{3.527108in}{1.495312in}}%
\pgfpathlineto{\pgfqpoint{3.527108in}{1.495312in}}%
\pgfpathlineto{\pgfqpoint{3.527108in}{1.499570in}}%
\pgfpathlineto{\pgfqpoint{3.531365in}{1.499570in}}%
\pgfpathlineto{\pgfqpoint{3.531365in}{1.495312in}}%
\pgfpathmoveto{\pgfqpoint{3.527108in}{1.499570in}}%
\pgfpathlineto{\pgfqpoint{3.527108in}{1.499570in}}%
\pgfpathlineto{\pgfqpoint{3.527108in}{1.503828in}}%
\pgfpathlineto{\pgfqpoint{3.531365in}{1.503828in}}%
\pgfpathlineto{\pgfqpoint{3.531365in}{1.499570in}}%
\pgfpathmoveto{\pgfqpoint{3.531365in}{1.495312in}}%
\pgfpathlineto{\pgfqpoint{3.531365in}{1.495312in}}%
\pgfpathlineto{\pgfqpoint{3.531365in}{1.499570in}}%
\pgfpathlineto{\pgfqpoint{3.535623in}{1.499570in}}%
\pgfpathlineto{\pgfqpoint{3.535623in}{1.495312in}}%
\pgfpathmoveto{\pgfqpoint{3.531365in}{1.499570in}}%
\pgfpathlineto{\pgfqpoint{3.531365in}{1.499570in}}%
\pgfpathlineto{\pgfqpoint{3.531365in}{1.503828in}}%
\pgfpathlineto{\pgfqpoint{3.535623in}{1.503828in}}%
\pgfpathlineto{\pgfqpoint{3.535623in}{1.499570in}}%
\pgfpathmoveto{\pgfqpoint{3.527108in}{1.503828in}}%
\pgfpathlineto{\pgfqpoint{3.527108in}{1.503828in}}%
\pgfpathlineto{\pgfqpoint{3.527108in}{1.508086in}}%
\pgfpathlineto{\pgfqpoint{3.531365in}{1.508086in}}%
\pgfpathlineto{\pgfqpoint{3.531365in}{1.503828in}}%
\pgfpathmoveto{\pgfqpoint{3.527108in}{1.508086in}}%
\pgfpathlineto{\pgfqpoint{3.527108in}{1.508086in}}%
\pgfpathlineto{\pgfqpoint{3.527108in}{1.512344in}}%
\pgfpathlineto{\pgfqpoint{3.531365in}{1.512344in}}%
\pgfpathlineto{\pgfqpoint{3.531365in}{1.508086in}}%
\pgfpathmoveto{\pgfqpoint{3.531365in}{1.503828in}}%
\pgfpathlineto{\pgfqpoint{3.531365in}{1.503828in}}%
\pgfpathlineto{\pgfqpoint{3.531365in}{1.508086in}}%
\pgfpathlineto{\pgfqpoint{3.535623in}{1.508086in}}%
\pgfpathlineto{\pgfqpoint{3.535623in}{1.503828in}}%
\pgfpathmoveto{\pgfqpoint{3.531365in}{1.508086in}}%
\pgfpathlineto{\pgfqpoint{3.531365in}{1.508086in}}%
\pgfpathlineto{\pgfqpoint{3.531365in}{1.512344in}}%
\pgfpathlineto{\pgfqpoint{3.535623in}{1.512344in}}%
\pgfpathlineto{\pgfqpoint{3.535623in}{1.508086in}}%
\pgfpathmoveto{\pgfqpoint{3.527108in}{1.512344in}}%
\pgfpathlineto{\pgfqpoint{3.527108in}{1.512344in}}%
\pgfpathlineto{\pgfqpoint{3.527108in}{1.516602in}}%
\pgfpathlineto{\pgfqpoint{3.531365in}{1.516602in}}%
\pgfpathlineto{\pgfqpoint{3.531365in}{1.512344in}}%
\pgfpathmoveto{\pgfqpoint{3.531365in}{1.512344in}}%
\pgfpathlineto{\pgfqpoint{3.531365in}{1.512344in}}%
\pgfpathlineto{\pgfqpoint{3.531365in}{1.516602in}}%
\pgfpathlineto{\pgfqpoint{3.535623in}{1.516602in}}%
\pgfpathlineto{\pgfqpoint{3.535623in}{1.512344in}}%
\pgfpathmoveto{\pgfqpoint{3.531365in}{1.516602in}}%
\pgfpathlineto{\pgfqpoint{3.531365in}{1.516602in}}%
\pgfpathlineto{\pgfqpoint{3.531365in}{1.520860in}}%
\pgfpathlineto{\pgfqpoint{3.535623in}{1.520860in}}%
\pgfpathlineto{\pgfqpoint{3.535623in}{1.516602in}}%
\pgfpathmoveto{\pgfqpoint{3.531365in}{1.520860in}}%
\pgfpathlineto{\pgfqpoint{3.531365in}{1.520860in}}%
\pgfpathlineto{\pgfqpoint{3.531365in}{1.525118in}}%
\pgfpathlineto{\pgfqpoint{3.535623in}{1.525118in}}%
\pgfpathlineto{\pgfqpoint{3.535623in}{1.520860in}}%
\pgfpathmoveto{\pgfqpoint{3.535623in}{1.486796in}}%
\pgfpathlineto{\pgfqpoint{3.535623in}{1.486796in}}%
\pgfpathlineto{\pgfqpoint{3.535623in}{1.491054in}}%
\pgfpathlineto{\pgfqpoint{3.539881in}{1.491054in}}%
\pgfpathlineto{\pgfqpoint{3.539881in}{1.486796in}}%
\pgfpathmoveto{\pgfqpoint{3.535623in}{1.491054in}}%
\pgfpathlineto{\pgfqpoint{3.535623in}{1.491054in}}%
\pgfpathlineto{\pgfqpoint{3.535623in}{1.495312in}}%
\pgfpathlineto{\pgfqpoint{3.539881in}{1.495312in}}%
\pgfpathlineto{\pgfqpoint{3.539881in}{1.491054in}}%
\pgfpathmoveto{\pgfqpoint{3.535623in}{1.495312in}}%
\pgfpathlineto{\pgfqpoint{3.535623in}{1.495312in}}%
\pgfpathlineto{\pgfqpoint{3.535623in}{1.499570in}}%
\pgfpathlineto{\pgfqpoint{3.539881in}{1.499570in}}%
\pgfpathlineto{\pgfqpoint{3.539881in}{1.495312in}}%
\pgfpathmoveto{\pgfqpoint{3.535623in}{1.499570in}}%
\pgfpathlineto{\pgfqpoint{3.535623in}{1.499570in}}%
\pgfpathlineto{\pgfqpoint{3.535623in}{1.503828in}}%
\pgfpathlineto{\pgfqpoint{3.539881in}{1.503828in}}%
\pgfpathlineto{\pgfqpoint{3.539881in}{1.499570in}}%
\pgfpathmoveto{\pgfqpoint{3.539881in}{1.495312in}}%
\pgfpathlineto{\pgfqpoint{3.539881in}{1.495312in}}%
\pgfpathlineto{\pgfqpoint{3.539881in}{1.499570in}}%
\pgfpathlineto{\pgfqpoint{3.544138in}{1.499570in}}%
\pgfpathlineto{\pgfqpoint{3.544138in}{1.495312in}}%
\pgfpathmoveto{\pgfqpoint{3.539881in}{1.499570in}}%
\pgfpathlineto{\pgfqpoint{3.539881in}{1.499570in}}%
\pgfpathlineto{\pgfqpoint{3.539881in}{1.503828in}}%
\pgfpathlineto{\pgfqpoint{3.544138in}{1.503828in}}%
\pgfpathlineto{\pgfqpoint{3.544138in}{1.499570in}}%
\pgfpathmoveto{\pgfqpoint{3.535623in}{1.503828in}}%
\pgfpathlineto{\pgfqpoint{3.535623in}{1.503828in}}%
\pgfpathlineto{\pgfqpoint{3.535623in}{1.508086in}}%
\pgfpathlineto{\pgfqpoint{3.539881in}{1.508086in}}%
\pgfpathlineto{\pgfqpoint{3.539881in}{1.503828in}}%
\pgfpathmoveto{\pgfqpoint{3.535623in}{1.508086in}}%
\pgfpathlineto{\pgfqpoint{3.535623in}{1.508086in}}%
\pgfpathlineto{\pgfqpoint{3.535623in}{1.512344in}}%
\pgfpathlineto{\pgfqpoint{3.539881in}{1.512344in}}%
\pgfpathlineto{\pgfqpoint{3.539881in}{1.508086in}}%
\pgfpathmoveto{\pgfqpoint{3.539881in}{1.503828in}}%
\pgfpathlineto{\pgfqpoint{3.539881in}{1.503828in}}%
\pgfpathlineto{\pgfqpoint{3.539881in}{1.508086in}}%
\pgfpathlineto{\pgfqpoint{3.544138in}{1.508086in}}%
\pgfpathlineto{\pgfqpoint{3.544138in}{1.503828in}}%
\pgfpathmoveto{\pgfqpoint{3.539881in}{1.508086in}}%
\pgfpathlineto{\pgfqpoint{3.539881in}{1.508086in}}%
\pgfpathlineto{\pgfqpoint{3.539881in}{1.512344in}}%
\pgfpathlineto{\pgfqpoint{3.544138in}{1.512344in}}%
\pgfpathlineto{\pgfqpoint{3.544138in}{1.508086in}}%
\pgfpathmoveto{\pgfqpoint{3.544138in}{1.499570in}}%
\pgfpathlineto{\pgfqpoint{3.544138in}{1.499570in}}%
\pgfpathlineto{\pgfqpoint{3.544138in}{1.503828in}}%
\pgfpathlineto{\pgfqpoint{3.548396in}{1.503828in}}%
\pgfpathlineto{\pgfqpoint{3.548396in}{1.499570in}}%
\pgfpathmoveto{\pgfqpoint{3.544138in}{1.503828in}}%
\pgfpathlineto{\pgfqpoint{3.544138in}{1.503828in}}%
\pgfpathlineto{\pgfqpoint{3.544138in}{1.508086in}}%
\pgfpathlineto{\pgfqpoint{3.548396in}{1.508086in}}%
\pgfpathlineto{\pgfqpoint{3.548396in}{1.503828in}}%
\pgfpathmoveto{\pgfqpoint{3.544138in}{1.508086in}}%
\pgfpathlineto{\pgfqpoint{3.544138in}{1.508086in}}%
\pgfpathlineto{\pgfqpoint{3.544138in}{1.512344in}}%
\pgfpathlineto{\pgfqpoint{3.548396in}{1.512344in}}%
\pgfpathlineto{\pgfqpoint{3.548396in}{1.508086in}}%
\pgfpathmoveto{\pgfqpoint{3.548396in}{1.508086in}}%
\pgfpathlineto{\pgfqpoint{3.548396in}{1.508086in}}%
\pgfpathlineto{\pgfqpoint{3.548396in}{1.512344in}}%
\pgfpathlineto{\pgfqpoint{3.552654in}{1.512344in}}%
\pgfpathlineto{\pgfqpoint{3.552654in}{1.508086in}}%
\pgfpathmoveto{\pgfqpoint{3.535623in}{1.512344in}}%
\pgfpathlineto{\pgfqpoint{3.535623in}{1.512344in}}%
\pgfpathlineto{\pgfqpoint{3.535623in}{1.516602in}}%
\pgfpathlineto{\pgfqpoint{3.539881in}{1.516602in}}%
\pgfpathlineto{\pgfqpoint{3.539881in}{1.512344in}}%
\pgfpathmoveto{\pgfqpoint{3.535623in}{1.516602in}}%
\pgfpathlineto{\pgfqpoint{3.535623in}{1.516602in}}%
\pgfpathlineto{\pgfqpoint{3.535623in}{1.520860in}}%
\pgfpathlineto{\pgfqpoint{3.539881in}{1.520860in}}%
\pgfpathlineto{\pgfqpoint{3.539881in}{1.516602in}}%
\pgfpathmoveto{\pgfqpoint{3.539881in}{1.512344in}}%
\pgfpathlineto{\pgfqpoint{3.539881in}{1.512344in}}%
\pgfpathlineto{\pgfqpoint{3.539881in}{1.516602in}}%
\pgfpathlineto{\pgfqpoint{3.544138in}{1.516602in}}%
\pgfpathlineto{\pgfqpoint{3.544138in}{1.512344in}}%
\pgfpathmoveto{\pgfqpoint{3.539881in}{1.516602in}}%
\pgfpathlineto{\pgfqpoint{3.539881in}{1.516602in}}%
\pgfpathlineto{\pgfqpoint{3.539881in}{1.520860in}}%
\pgfpathlineto{\pgfqpoint{3.544138in}{1.520860in}}%
\pgfpathlineto{\pgfqpoint{3.544138in}{1.516602in}}%
\pgfpathmoveto{\pgfqpoint{3.535623in}{1.520860in}}%
\pgfpathlineto{\pgfqpoint{3.535623in}{1.520860in}}%
\pgfpathlineto{\pgfqpoint{3.535623in}{1.525118in}}%
\pgfpathlineto{\pgfqpoint{3.539881in}{1.525118in}}%
\pgfpathlineto{\pgfqpoint{3.539881in}{1.520860in}}%
\pgfpathmoveto{\pgfqpoint{3.535623in}{1.525118in}}%
\pgfpathlineto{\pgfqpoint{3.535623in}{1.525118in}}%
\pgfpathlineto{\pgfqpoint{3.535623in}{1.529376in}}%
\pgfpathlineto{\pgfqpoint{3.539881in}{1.529376in}}%
\pgfpathlineto{\pgfqpoint{3.539881in}{1.525118in}}%
\pgfpathmoveto{\pgfqpoint{3.539881in}{1.520860in}}%
\pgfpathlineto{\pgfqpoint{3.539881in}{1.520860in}}%
\pgfpathlineto{\pgfqpoint{3.539881in}{1.525118in}}%
\pgfpathlineto{\pgfqpoint{3.544138in}{1.525118in}}%
\pgfpathlineto{\pgfqpoint{3.544138in}{1.520860in}}%
\pgfpathmoveto{\pgfqpoint{3.539881in}{1.525118in}}%
\pgfpathlineto{\pgfqpoint{3.539881in}{1.525118in}}%
\pgfpathlineto{\pgfqpoint{3.539881in}{1.529376in}}%
\pgfpathlineto{\pgfqpoint{3.544138in}{1.529376in}}%
\pgfpathlineto{\pgfqpoint{3.544138in}{1.525118in}}%
\pgfpathmoveto{\pgfqpoint{3.544138in}{1.512344in}}%
\pgfpathlineto{\pgfqpoint{3.544138in}{1.512344in}}%
\pgfpathlineto{\pgfqpoint{3.544138in}{1.516602in}}%
\pgfpathlineto{\pgfqpoint{3.548396in}{1.516602in}}%
\pgfpathlineto{\pgfqpoint{3.548396in}{1.512344in}}%
\pgfpathmoveto{\pgfqpoint{3.544138in}{1.516602in}}%
\pgfpathlineto{\pgfqpoint{3.544138in}{1.516602in}}%
\pgfpathlineto{\pgfqpoint{3.544138in}{1.520860in}}%
\pgfpathlineto{\pgfqpoint{3.548396in}{1.520860in}}%
\pgfpathlineto{\pgfqpoint{3.548396in}{1.516602in}}%
\pgfpathmoveto{\pgfqpoint{3.548396in}{1.512344in}}%
\pgfpathlineto{\pgfqpoint{3.548396in}{1.512344in}}%
\pgfpathlineto{\pgfqpoint{3.548396in}{1.516602in}}%
\pgfpathlineto{\pgfqpoint{3.552654in}{1.516602in}}%
\pgfpathlineto{\pgfqpoint{3.552654in}{1.512344in}}%
\pgfpathmoveto{\pgfqpoint{3.548396in}{1.516602in}}%
\pgfpathlineto{\pgfqpoint{3.548396in}{1.516602in}}%
\pgfpathlineto{\pgfqpoint{3.548396in}{1.520860in}}%
\pgfpathlineto{\pgfqpoint{3.552654in}{1.520860in}}%
\pgfpathlineto{\pgfqpoint{3.552654in}{1.516602in}}%
\pgfpathmoveto{\pgfqpoint{3.544138in}{1.520860in}}%
\pgfpathlineto{\pgfqpoint{3.544138in}{1.520860in}}%
\pgfpathlineto{\pgfqpoint{3.544138in}{1.525118in}}%
\pgfpathlineto{\pgfqpoint{3.548396in}{1.525118in}}%
\pgfpathlineto{\pgfqpoint{3.548396in}{1.520860in}}%
\pgfpathmoveto{\pgfqpoint{3.544138in}{1.525118in}}%
\pgfpathlineto{\pgfqpoint{3.544138in}{1.525118in}}%
\pgfpathlineto{\pgfqpoint{3.544138in}{1.529376in}}%
\pgfpathlineto{\pgfqpoint{3.548396in}{1.529376in}}%
\pgfpathlineto{\pgfqpoint{3.548396in}{1.525118in}}%
\pgfpathmoveto{\pgfqpoint{3.548396in}{1.520860in}}%
\pgfpathlineto{\pgfqpoint{3.548396in}{1.520860in}}%
\pgfpathlineto{\pgfqpoint{3.548396in}{1.525118in}}%
\pgfpathlineto{\pgfqpoint{3.552654in}{1.525118in}}%
\pgfpathlineto{\pgfqpoint{3.552654in}{1.520860in}}%
\pgfpathmoveto{\pgfqpoint{3.548396in}{1.525118in}}%
\pgfpathlineto{\pgfqpoint{3.548396in}{1.525118in}}%
\pgfpathlineto{\pgfqpoint{3.548396in}{1.529376in}}%
\pgfpathlineto{\pgfqpoint{3.552654in}{1.529376in}}%
\pgfpathlineto{\pgfqpoint{3.552654in}{1.525118in}}%
\pgfpathmoveto{\pgfqpoint{3.552654in}{1.516602in}}%
\pgfpathlineto{\pgfqpoint{3.552654in}{1.516602in}}%
\pgfpathlineto{\pgfqpoint{3.552654in}{1.520860in}}%
\pgfpathlineto{\pgfqpoint{3.556912in}{1.520860in}}%
\pgfpathlineto{\pgfqpoint{3.556912in}{1.516602in}}%
\pgfpathmoveto{\pgfqpoint{3.552654in}{1.520860in}}%
\pgfpathlineto{\pgfqpoint{3.552654in}{1.520860in}}%
\pgfpathlineto{\pgfqpoint{3.552654in}{1.525118in}}%
\pgfpathlineto{\pgfqpoint{3.556912in}{1.525118in}}%
\pgfpathlineto{\pgfqpoint{3.556912in}{1.520860in}}%
\pgfpathmoveto{\pgfqpoint{3.552654in}{1.525118in}}%
\pgfpathlineto{\pgfqpoint{3.552654in}{1.525118in}}%
\pgfpathlineto{\pgfqpoint{3.552654in}{1.529376in}}%
\pgfpathlineto{\pgfqpoint{3.556912in}{1.529376in}}%
\pgfpathlineto{\pgfqpoint{3.556912in}{1.525118in}}%
\pgfpathmoveto{\pgfqpoint{3.556912in}{1.525118in}}%
\pgfpathlineto{\pgfqpoint{3.556912in}{1.525118in}}%
\pgfpathlineto{\pgfqpoint{3.556912in}{1.529376in}}%
\pgfpathlineto{\pgfqpoint{3.561169in}{1.529376in}}%
\pgfpathlineto{\pgfqpoint{3.561169in}{1.525118in}}%
\pgfpathmoveto{\pgfqpoint{3.535623in}{1.529376in}}%
\pgfpathlineto{\pgfqpoint{3.535623in}{1.529376in}}%
\pgfpathlineto{\pgfqpoint{3.535623in}{1.533634in}}%
\pgfpathlineto{\pgfqpoint{3.539881in}{1.533634in}}%
\pgfpathlineto{\pgfqpoint{3.539881in}{1.529376in}}%
\pgfpathmoveto{\pgfqpoint{3.539881in}{1.529376in}}%
\pgfpathlineto{\pgfqpoint{3.539881in}{1.529376in}}%
\pgfpathlineto{\pgfqpoint{3.539881in}{1.533634in}}%
\pgfpathlineto{\pgfqpoint{3.544138in}{1.533634in}}%
\pgfpathlineto{\pgfqpoint{3.544138in}{1.529376in}}%
\pgfpathmoveto{\pgfqpoint{3.539881in}{1.533634in}}%
\pgfpathlineto{\pgfqpoint{3.539881in}{1.533634in}}%
\pgfpathlineto{\pgfqpoint{3.539881in}{1.537892in}}%
\pgfpathlineto{\pgfqpoint{3.544138in}{1.537892in}}%
\pgfpathlineto{\pgfqpoint{3.544138in}{1.533634in}}%
\pgfpathmoveto{\pgfqpoint{3.544138in}{1.529376in}}%
\pgfpathlineto{\pgfqpoint{3.544138in}{1.529376in}}%
\pgfpathlineto{\pgfqpoint{3.544138in}{1.533634in}}%
\pgfpathlineto{\pgfqpoint{3.548396in}{1.533634in}}%
\pgfpathlineto{\pgfqpoint{3.548396in}{1.529376in}}%
\pgfpathmoveto{\pgfqpoint{3.544138in}{1.533634in}}%
\pgfpathlineto{\pgfqpoint{3.544138in}{1.533634in}}%
\pgfpathlineto{\pgfqpoint{3.544138in}{1.537892in}}%
\pgfpathlineto{\pgfqpoint{3.548396in}{1.537892in}}%
\pgfpathlineto{\pgfqpoint{3.548396in}{1.533634in}}%
\pgfpathmoveto{\pgfqpoint{3.548396in}{1.529376in}}%
\pgfpathlineto{\pgfqpoint{3.548396in}{1.529376in}}%
\pgfpathlineto{\pgfqpoint{3.548396in}{1.533634in}}%
\pgfpathlineto{\pgfqpoint{3.552654in}{1.533634in}}%
\pgfpathlineto{\pgfqpoint{3.552654in}{1.529376in}}%
\pgfpathmoveto{\pgfqpoint{3.548396in}{1.533634in}}%
\pgfpathlineto{\pgfqpoint{3.548396in}{1.533634in}}%
\pgfpathlineto{\pgfqpoint{3.548396in}{1.537892in}}%
\pgfpathlineto{\pgfqpoint{3.552654in}{1.537892in}}%
\pgfpathlineto{\pgfqpoint{3.552654in}{1.533634in}}%
\pgfpathmoveto{\pgfqpoint{3.544138in}{1.537892in}}%
\pgfpathlineto{\pgfqpoint{3.544138in}{1.537892in}}%
\pgfpathlineto{\pgfqpoint{3.544138in}{1.542150in}}%
\pgfpathlineto{\pgfqpoint{3.548396in}{1.542150in}}%
\pgfpathlineto{\pgfqpoint{3.548396in}{1.537892in}}%
\pgfpathmoveto{\pgfqpoint{3.544138in}{1.542150in}}%
\pgfpathlineto{\pgfqpoint{3.544138in}{1.542150in}}%
\pgfpathlineto{\pgfqpoint{3.544138in}{1.546408in}}%
\pgfpathlineto{\pgfqpoint{3.548396in}{1.546408in}}%
\pgfpathlineto{\pgfqpoint{3.548396in}{1.542150in}}%
\pgfpathmoveto{\pgfqpoint{3.548396in}{1.537892in}}%
\pgfpathlineto{\pgfqpoint{3.548396in}{1.537892in}}%
\pgfpathlineto{\pgfqpoint{3.548396in}{1.542150in}}%
\pgfpathlineto{\pgfqpoint{3.552654in}{1.542150in}}%
\pgfpathlineto{\pgfqpoint{3.552654in}{1.537892in}}%
\pgfpathmoveto{\pgfqpoint{3.548396in}{1.542150in}}%
\pgfpathlineto{\pgfqpoint{3.548396in}{1.542150in}}%
\pgfpathlineto{\pgfqpoint{3.548396in}{1.546408in}}%
\pgfpathlineto{\pgfqpoint{3.552654in}{1.546408in}}%
\pgfpathlineto{\pgfqpoint{3.552654in}{1.542150in}}%
\pgfpathmoveto{\pgfqpoint{3.548396in}{1.546408in}}%
\pgfpathlineto{\pgfqpoint{3.548396in}{1.546408in}}%
\pgfpathlineto{\pgfqpoint{3.548396in}{1.550666in}}%
\pgfpathlineto{\pgfqpoint{3.552654in}{1.550666in}}%
\pgfpathlineto{\pgfqpoint{3.552654in}{1.546408in}}%
\pgfpathmoveto{\pgfqpoint{3.548396in}{1.550666in}}%
\pgfpathlineto{\pgfqpoint{3.548396in}{1.550666in}}%
\pgfpathlineto{\pgfqpoint{3.548396in}{1.554924in}}%
\pgfpathlineto{\pgfqpoint{3.552654in}{1.554924in}}%
\pgfpathlineto{\pgfqpoint{3.552654in}{1.550666in}}%
\pgfpathmoveto{\pgfqpoint{3.552654in}{1.529376in}}%
\pgfpathlineto{\pgfqpoint{3.552654in}{1.529376in}}%
\pgfpathlineto{\pgfqpoint{3.552654in}{1.533634in}}%
\pgfpathlineto{\pgfqpoint{3.556912in}{1.533634in}}%
\pgfpathlineto{\pgfqpoint{3.556912in}{1.529376in}}%
\pgfpathmoveto{\pgfqpoint{3.552654in}{1.533634in}}%
\pgfpathlineto{\pgfqpoint{3.552654in}{1.533634in}}%
\pgfpathlineto{\pgfqpoint{3.552654in}{1.537892in}}%
\pgfpathlineto{\pgfqpoint{3.556912in}{1.537892in}}%
\pgfpathlineto{\pgfqpoint{3.556912in}{1.533634in}}%
\pgfpathmoveto{\pgfqpoint{3.556912in}{1.529376in}}%
\pgfpathlineto{\pgfqpoint{3.556912in}{1.529376in}}%
\pgfpathlineto{\pgfqpoint{3.556912in}{1.533634in}}%
\pgfpathlineto{\pgfqpoint{3.561169in}{1.533634in}}%
\pgfpathlineto{\pgfqpoint{3.561169in}{1.529376in}}%
\pgfpathmoveto{\pgfqpoint{3.556912in}{1.533634in}}%
\pgfpathlineto{\pgfqpoint{3.556912in}{1.533634in}}%
\pgfpathlineto{\pgfqpoint{3.556912in}{1.537892in}}%
\pgfpathlineto{\pgfqpoint{3.561169in}{1.537892in}}%
\pgfpathlineto{\pgfqpoint{3.561169in}{1.533634in}}%
\pgfpathmoveto{\pgfqpoint{3.552654in}{1.537892in}}%
\pgfpathlineto{\pgfqpoint{3.552654in}{1.537892in}}%
\pgfpathlineto{\pgfqpoint{3.552654in}{1.542150in}}%
\pgfpathlineto{\pgfqpoint{3.556912in}{1.542150in}}%
\pgfpathlineto{\pgfqpoint{3.556912in}{1.537892in}}%
\pgfpathmoveto{\pgfqpoint{3.552654in}{1.542150in}}%
\pgfpathlineto{\pgfqpoint{3.552654in}{1.542150in}}%
\pgfpathlineto{\pgfqpoint{3.552654in}{1.546408in}}%
\pgfpathlineto{\pgfqpoint{3.556912in}{1.546408in}}%
\pgfpathlineto{\pgfqpoint{3.556912in}{1.542150in}}%
\pgfpathmoveto{\pgfqpoint{3.556912in}{1.537892in}}%
\pgfpathlineto{\pgfqpoint{3.556912in}{1.537892in}}%
\pgfpathlineto{\pgfqpoint{3.556912in}{1.542150in}}%
\pgfpathlineto{\pgfqpoint{3.561169in}{1.542150in}}%
\pgfpathlineto{\pgfqpoint{3.561169in}{1.537892in}}%
\pgfpathmoveto{\pgfqpoint{3.556912in}{1.542150in}}%
\pgfpathlineto{\pgfqpoint{3.556912in}{1.542150in}}%
\pgfpathlineto{\pgfqpoint{3.556912in}{1.546408in}}%
\pgfpathlineto{\pgfqpoint{3.561169in}{1.546408in}}%
\pgfpathlineto{\pgfqpoint{3.561169in}{1.542150in}}%
\pgfpathmoveto{\pgfqpoint{3.561169in}{1.533634in}}%
\pgfpathlineto{\pgfqpoint{3.561169in}{1.533634in}}%
\pgfpathlineto{\pgfqpoint{3.561169in}{1.537892in}}%
\pgfpathlineto{\pgfqpoint{3.565427in}{1.537892in}}%
\pgfpathlineto{\pgfqpoint{3.565427in}{1.533634in}}%
\pgfpathmoveto{\pgfqpoint{3.561169in}{1.537892in}}%
\pgfpathlineto{\pgfqpoint{3.561169in}{1.537892in}}%
\pgfpathlineto{\pgfqpoint{3.561169in}{1.542150in}}%
\pgfpathlineto{\pgfqpoint{3.565427in}{1.542150in}}%
\pgfpathlineto{\pgfqpoint{3.565427in}{1.537892in}}%
\pgfpathmoveto{\pgfqpoint{3.561169in}{1.542150in}}%
\pgfpathlineto{\pgfqpoint{3.561169in}{1.542150in}}%
\pgfpathlineto{\pgfqpoint{3.561169in}{1.546408in}}%
\pgfpathlineto{\pgfqpoint{3.565427in}{1.546408in}}%
\pgfpathlineto{\pgfqpoint{3.565427in}{1.542150in}}%
\pgfpathmoveto{\pgfqpoint{3.565427in}{1.542150in}}%
\pgfpathlineto{\pgfqpoint{3.565427in}{1.542150in}}%
\pgfpathlineto{\pgfqpoint{3.565427in}{1.546408in}}%
\pgfpathlineto{\pgfqpoint{3.569685in}{1.546408in}}%
\pgfpathlineto{\pgfqpoint{3.569685in}{1.542150in}}%
\pgfpathmoveto{\pgfqpoint{3.552654in}{1.546408in}}%
\pgfpathlineto{\pgfqpoint{3.552654in}{1.546408in}}%
\pgfpathlineto{\pgfqpoint{3.552654in}{1.550666in}}%
\pgfpathlineto{\pgfqpoint{3.556912in}{1.550666in}}%
\pgfpathlineto{\pgfqpoint{3.556912in}{1.546408in}}%
\pgfpathmoveto{\pgfqpoint{3.552654in}{1.550666in}}%
\pgfpathlineto{\pgfqpoint{3.552654in}{1.550666in}}%
\pgfpathlineto{\pgfqpoint{3.552654in}{1.554924in}}%
\pgfpathlineto{\pgfqpoint{3.556912in}{1.554924in}}%
\pgfpathlineto{\pgfqpoint{3.556912in}{1.550666in}}%
\pgfpathmoveto{\pgfqpoint{3.556912in}{1.546408in}}%
\pgfpathlineto{\pgfqpoint{3.556912in}{1.546408in}}%
\pgfpathlineto{\pgfqpoint{3.556912in}{1.550666in}}%
\pgfpathlineto{\pgfqpoint{3.561169in}{1.550666in}}%
\pgfpathlineto{\pgfqpoint{3.561169in}{1.546408in}}%
\pgfpathmoveto{\pgfqpoint{3.556912in}{1.550666in}}%
\pgfpathlineto{\pgfqpoint{3.556912in}{1.550666in}}%
\pgfpathlineto{\pgfqpoint{3.556912in}{1.554924in}}%
\pgfpathlineto{\pgfqpoint{3.561169in}{1.554924in}}%
\pgfpathlineto{\pgfqpoint{3.561169in}{1.550666in}}%
\pgfpathmoveto{\pgfqpoint{3.552654in}{1.554924in}}%
\pgfpathlineto{\pgfqpoint{3.552654in}{1.554924in}}%
\pgfpathlineto{\pgfqpoint{3.552654in}{1.559182in}}%
\pgfpathlineto{\pgfqpoint{3.556912in}{1.559182in}}%
\pgfpathlineto{\pgfqpoint{3.556912in}{1.554924in}}%
\pgfpathmoveto{\pgfqpoint{3.552654in}{1.559182in}}%
\pgfpathlineto{\pgfqpoint{3.552654in}{1.559182in}}%
\pgfpathlineto{\pgfqpoint{3.552654in}{1.563440in}}%
\pgfpathlineto{\pgfqpoint{3.556912in}{1.563440in}}%
\pgfpathlineto{\pgfqpoint{3.556912in}{1.559182in}}%
\pgfpathmoveto{\pgfqpoint{3.556912in}{1.554924in}}%
\pgfpathlineto{\pgfqpoint{3.556912in}{1.554924in}}%
\pgfpathlineto{\pgfqpoint{3.556912in}{1.559182in}}%
\pgfpathlineto{\pgfqpoint{3.561169in}{1.559182in}}%
\pgfpathlineto{\pgfqpoint{3.561169in}{1.554924in}}%
\pgfpathmoveto{\pgfqpoint{3.556912in}{1.559182in}}%
\pgfpathlineto{\pgfqpoint{3.556912in}{1.559182in}}%
\pgfpathlineto{\pgfqpoint{3.556912in}{1.563440in}}%
\pgfpathlineto{\pgfqpoint{3.561169in}{1.563440in}}%
\pgfpathlineto{\pgfqpoint{3.561169in}{1.559182in}}%
\pgfpathmoveto{\pgfqpoint{3.561169in}{1.546408in}}%
\pgfpathlineto{\pgfqpoint{3.561169in}{1.546408in}}%
\pgfpathlineto{\pgfqpoint{3.561169in}{1.550666in}}%
\pgfpathlineto{\pgfqpoint{3.565427in}{1.550666in}}%
\pgfpathlineto{\pgfqpoint{3.565427in}{1.546408in}}%
\pgfpathmoveto{\pgfqpoint{3.561169in}{1.550666in}}%
\pgfpathlineto{\pgfqpoint{3.561169in}{1.550666in}}%
\pgfpathlineto{\pgfqpoint{3.561169in}{1.554924in}}%
\pgfpathlineto{\pgfqpoint{3.565427in}{1.554924in}}%
\pgfpathlineto{\pgfqpoint{3.565427in}{1.550666in}}%
\pgfpathmoveto{\pgfqpoint{3.565427in}{1.546408in}}%
\pgfpathlineto{\pgfqpoint{3.565427in}{1.546408in}}%
\pgfpathlineto{\pgfqpoint{3.565427in}{1.550666in}}%
\pgfpathlineto{\pgfqpoint{3.569685in}{1.550666in}}%
\pgfpathlineto{\pgfqpoint{3.569685in}{1.546408in}}%
\pgfpathmoveto{\pgfqpoint{3.565427in}{1.550666in}}%
\pgfpathlineto{\pgfqpoint{3.565427in}{1.550666in}}%
\pgfpathlineto{\pgfqpoint{3.565427in}{1.554924in}}%
\pgfpathlineto{\pgfqpoint{3.569685in}{1.554924in}}%
\pgfpathlineto{\pgfqpoint{3.569685in}{1.550666in}}%
\pgfpathmoveto{\pgfqpoint{3.561169in}{1.554924in}}%
\pgfpathlineto{\pgfqpoint{3.561169in}{1.554924in}}%
\pgfpathlineto{\pgfqpoint{3.561169in}{1.559182in}}%
\pgfpathlineto{\pgfqpoint{3.565427in}{1.559182in}}%
\pgfpathlineto{\pgfqpoint{3.565427in}{1.554924in}}%
\pgfpathmoveto{\pgfqpoint{3.561169in}{1.559182in}}%
\pgfpathlineto{\pgfqpoint{3.561169in}{1.559182in}}%
\pgfpathlineto{\pgfqpoint{3.561169in}{1.563440in}}%
\pgfpathlineto{\pgfqpoint{3.565427in}{1.563440in}}%
\pgfpathlineto{\pgfqpoint{3.565427in}{1.559182in}}%
\pgfpathmoveto{\pgfqpoint{3.565427in}{1.554924in}}%
\pgfpathlineto{\pgfqpoint{3.565427in}{1.554924in}}%
\pgfpathlineto{\pgfqpoint{3.565427in}{1.559182in}}%
\pgfpathlineto{\pgfqpoint{3.569685in}{1.559182in}}%
\pgfpathlineto{\pgfqpoint{3.569685in}{1.554924in}}%
\pgfpathmoveto{\pgfqpoint{3.565427in}{1.559182in}}%
\pgfpathlineto{\pgfqpoint{3.565427in}{1.559182in}}%
\pgfpathlineto{\pgfqpoint{3.565427in}{1.563440in}}%
\pgfpathlineto{\pgfqpoint{3.569685in}{1.563440in}}%
\pgfpathlineto{\pgfqpoint{3.569685in}{1.559182in}}%
\pgfpathmoveto{\pgfqpoint{3.556912in}{1.563440in}}%
\pgfpathlineto{\pgfqpoint{3.556912in}{1.563440in}}%
\pgfpathlineto{\pgfqpoint{3.556912in}{1.567698in}}%
\pgfpathlineto{\pgfqpoint{3.561169in}{1.567698in}}%
\pgfpathlineto{\pgfqpoint{3.561169in}{1.563440in}}%
\pgfpathmoveto{\pgfqpoint{3.556912in}{1.567698in}}%
\pgfpathlineto{\pgfqpoint{3.556912in}{1.567698in}}%
\pgfpathlineto{\pgfqpoint{3.556912in}{1.571956in}}%
\pgfpathlineto{\pgfqpoint{3.561169in}{1.571956in}}%
\pgfpathlineto{\pgfqpoint{3.561169in}{1.567698in}}%
\pgfpathmoveto{\pgfqpoint{3.561169in}{1.563440in}}%
\pgfpathlineto{\pgfqpoint{3.561169in}{1.563440in}}%
\pgfpathlineto{\pgfqpoint{3.561169in}{1.567698in}}%
\pgfpathlineto{\pgfqpoint{3.565427in}{1.567698in}}%
\pgfpathlineto{\pgfqpoint{3.565427in}{1.563440in}}%
\pgfpathmoveto{\pgfqpoint{3.561169in}{1.567698in}}%
\pgfpathlineto{\pgfqpoint{3.561169in}{1.567698in}}%
\pgfpathlineto{\pgfqpoint{3.561169in}{1.571956in}}%
\pgfpathlineto{\pgfqpoint{3.565427in}{1.571956in}}%
\pgfpathlineto{\pgfqpoint{3.565427in}{1.567698in}}%
\pgfpathmoveto{\pgfqpoint{3.565427in}{1.563440in}}%
\pgfpathlineto{\pgfqpoint{3.565427in}{1.563440in}}%
\pgfpathlineto{\pgfqpoint{3.565427in}{1.567698in}}%
\pgfpathlineto{\pgfqpoint{3.569685in}{1.567698in}}%
\pgfpathlineto{\pgfqpoint{3.569685in}{1.563440in}}%
\pgfpathmoveto{\pgfqpoint{3.565427in}{1.567698in}}%
\pgfpathlineto{\pgfqpoint{3.565427in}{1.567698in}}%
\pgfpathlineto{\pgfqpoint{3.565427in}{1.571956in}}%
\pgfpathlineto{\pgfqpoint{3.569685in}{1.571956in}}%
\pgfpathlineto{\pgfqpoint{3.569685in}{1.567698in}}%
\pgfpathmoveto{\pgfqpoint{3.561169in}{1.571956in}}%
\pgfpathlineto{\pgfqpoint{3.561169in}{1.571956in}}%
\pgfpathlineto{\pgfqpoint{3.561169in}{1.576214in}}%
\pgfpathlineto{\pgfqpoint{3.565427in}{1.576214in}}%
\pgfpathlineto{\pgfqpoint{3.565427in}{1.571956in}}%
\pgfpathmoveto{\pgfqpoint{3.561169in}{1.576214in}}%
\pgfpathlineto{\pgfqpoint{3.561169in}{1.576214in}}%
\pgfpathlineto{\pgfqpoint{3.561169in}{1.580472in}}%
\pgfpathlineto{\pgfqpoint{3.565427in}{1.580472in}}%
\pgfpathlineto{\pgfqpoint{3.565427in}{1.576214in}}%
\pgfpathmoveto{\pgfqpoint{3.565427in}{1.571956in}}%
\pgfpathlineto{\pgfqpoint{3.565427in}{1.571956in}}%
\pgfpathlineto{\pgfqpoint{3.565427in}{1.576214in}}%
\pgfpathlineto{\pgfqpoint{3.569685in}{1.576214in}}%
\pgfpathlineto{\pgfqpoint{3.569685in}{1.571956in}}%
\pgfpathmoveto{\pgfqpoint{3.565427in}{1.576214in}}%
\pgfpathlineto{\pgfqpoint{3.565427in}{1.576214in}}%
\pgfpathlineto{\pgfqpoint{3.565427in}{1.580472in}}%
\pgfpathlineto{\pgfqpoint{3.569685in}{1.580472in}}%
\pgfpathlineto{\pgfqpoint{3.569685in}{1.576214in}}%
\pgfpathmoveto{\pgfqpoint{3.565427in}{1.580472in}}%
\pgfpathlineto{\pgfqpoint{3.565427in}{1.580472in}}%
\pgfpathlineto{\pgfqpoint{3.565427in}{1.584730in}}%
\pgfpathlineto{\pgfqpoint{3.569685in}{1.584730in}}%
\pgfpathlineto{\pgfqpoint{3.569685in}{1.580472in}}%
\pgfpathmoveto{\pgfqpoint{3.565427in}{1.584730in}}%
\pgfpathlineto{\pgfqpoint{3.565427in}{1.584730in}}%
\pgfpathlineto{\pgfqpoint{3.565427in}{1.588988in}}%
\pgfpathlineto{\pgfqpoint{3.569685in}{1.588988in}}%
\pgfpathlineto{\pgfqpoint{3.569685in}{1.584730in}}%
\pgfpathmoveto{\pgfqpoint{3.569685in}{1.550666in}}%
\pgfpathlineto{\pgfqpoint{3.569685in}{1.550666in}}%
\pgfpathlineto{\pgfqpoint{3.569685in}{1.554924in}}%
\pgfpathlineto{\pgfqpoint{3.573943in}{1.554924in}}%
\pgfpathlineto{\pgfqpoint{3.573943in}{1.550666in}}%
\pgfpathmoveto{\pgfqpoint{3.569685in}{1.554924in}}%
\pgfpathlineto{\pgfqpoint{3.569685in}{1.554924in}}%
\pgfpathlineto{\pgfqpoint{3.569685in}{1.559182in}}%
\pgfpathlineto{\pgfqpoint{3.573943in}{1.559182in}}%
\pgfpathlineto{\pgfqpoint{3.573943in}{1.554924in}}%
\pgfpathmoveto{\pgfqpoint{3.569685in}{1.559182in}}%
\pgfpathlineto{\pgfqpoint{3.569685in}{1.559182in}}%
\pgfpathlineto{\pgfqpoint{3.569685in}{1.563440in}}%
\pgfpathlineto{\pgfqpoint{3.573943in}{1.563440in}}%
\pgfpathlineto{\pgfqpoint{3.573943in}{1.559182in}}%
\pgfpathmoveto{\pgfqpoint{3.569685in}{1.563440in}}%
\pgfpathlineto{\pgfqpoint{3.569685in}{1.563440in}}%
\pgfpathlineto{\pgfqpoint{3.569685in}{1.567698in}}%
\pgfpathlineto{\pgfqpoint{3.573943in}{1.567698in}}%
\pgfpathlineto{\pgfqpoint{3.573943in}{1.563440in}}%
\pgfpathmoveto{\pgfqpoint{3.569685in}{1.567698in}}%
\pgfpathlineto{\pgfqpoint{3.569685in}{1.567698in}}%
\pgfpathlineto{\pgfqpoint{3.569685in}{1.571956in}}%
\pgfpathlineto{\pgfqpoint{3.573943in}{1.571956in}}%
\pgfpathlineto{\pgfqpoint{3.573943in}{1.567698in}}%
\pgfpathmoveto{\pgfqpoint{3.573943in}{1.563440in}}%
\pgfpathlineto{\pgfqpoint{3.573943in}{1.563440in}}%
\pgfpathlineto{\pgfqpoint{3.573943in}{1.567698in}}%
\pgfpathlineto{\pgfqpoint{3.578200in}{1.567698in}}%
\pgfpathlineto{\pgfqpoint{3.578200in}{1.563440in}}%
\pgfpathmoveto{\pgfqpoint{3.573943in}{1.567698in}}%
\pgfpathlineto{\pgfqpoint{3.573943in}{1.567698in}}%
\pgfpathlineto{\pgfqpoint{3.573943in}{1.571956in}}%
\pgfpathlineto{\pgfqpoint{3.578200in}{1.571956in}}%
\pgfpathlineto{\pgfqpoint{3.578200in}{1.567698in}}%
\pgfpathmoveto{\pgfqpoint{3.569685in}{1.571956in}}%
\pgfpathlineto{\pgfqpoint{3.569685in}{1.571956in}}%
\pgfpathlineto{\pgfqpoint{3.569685in}{1.576214in}}%
\pgfpathlineto{\pgfqpoint{3.573943in}{1.576214in}}%
\pgfpathlineto{\pgfqpoint{3.573943in}{1.571956in}}%
\pgfpathmoveto{\pgfqpoint{3.569685in}{1.576214in}}%
\pgfpathlineto{\pgfqpoint{3.569685in}{1.576214in}}%
\pgfpathlineto{\pgfqpoint{3.569685in}{1.580472in}}%
\pgfpathlineto{\pgfqpoint{3.573943in}{1.580472in}}%
\pgfpathlineto{\pgfqpoint{3.573943in}{1.576214in}}%
\pgfpathmoveto{\pgfqpoint{3.573943in}{1.571956in}}%
\pgfpathlineto{\pgfqpoint{3.573943in}{1.571956in}}%
\pgfpathlineto{\pgfqpoint{3.573943in}{1.576214in}}%
\pgfpathlineto{\pgfqpoint{3.578200in}{1.576214in}}%
\pgfpathlineto{\pgfqpoint{3.578200in}{1.571956in}}%
\pgfpathmoveto{\pgfqpoint{3.573943in}{1.576214in}}%
\pgfpathlineto{\pgfqpoint{3.573943in}{1.576214in}}%
\pgfpathlineto{\pgfqpoint{3.573943in}{1.580472in}}%
\pgfpathlineto{\pgfqpoint{3.578200in}{1.580472in}}%
\pgfpathlineto{\pgfqpoint{3.578200in}{1.576214in}}%
\pgfpathmoveto{\pgfqpoint{3.578200in}{1.571956in}}%
\pgfpathlineto{\pgfqpoint{3.578200in}{1.571956in}}%
\pgfpathlineto{\pgfqpoint{3.578200in}{1.576214in}}%
\pgfpathlineto{\pgfqpoint{3.582458in}{1.576214in}}%
\pgfpathlineto{\pgfqpoint{3.582458in}{1.571956in}}%
\pgfpathmoveto{\pgfqpoint{3.578200in}{1.576214in}}%
\pgfpathlineto{\pgfqpoint{3.578200in}{1.576214in}}%
\pgfpathlineto{\pgfqpoint{3.578200in}{1.580472in}}%
\pgfpathlineto{\pgfqpoint{3.582458in}{1.580472in}}%
\pgfpathlineto{\pgfqpoint{3.582458in}{1.576214in}}%
\pgfpathmoveto{\pgfqpoint{3.569685in}{1.580472in}}%
\pgfpathlineto{\pgfqpoint{3.569685in}{1.580472in}}%
\pgfpathlineto{\pgfqpoint{3.569685in}{1.584730in}}%
\pgfpathlineto{\pgfqpoint{3.573943in}{1.584730in}}%
\pgfpathlineto{\pgfqpoint{3.573943in}{1.580472in}}%
\pgfpathmoveto{\pgfqpoint{3.569685in}{1.584730in}}%
\pgfpathlineto{\pgfqpoint{3.569685in}{1.584730in}}%
\pgfpathlineto{\pgfqpoint{3.569685in}{1.588988in}}%
\pgfpathlineto{\pgfqpoint{3.573943in}{1.588988in}}%
\pgfpathlineto{\pgfqpoint{3.573943in}{1.584730in}}%
\pgfpathmoveto{\pgfqpoint{3.573943in}{1.580472in}}%
\pgfpathlineto{\pgfqpoint{3.573943in}{1.580472in}}%
\pgfpathlineto{\pgfqpoint{3.573943in}{1.584730in}}%
\pgfpathlineto{\pgfqpoint{3.578200in}{1.584730in}}%
\pgfpathlineto{\pgfqpoint{3.578200in}{1.580472in}}%
\pgfpathmoveto{\pgfqpoint{3.573943in}{1.584730in}}%
\pgfpathlineto{\pgfqpoint{3.573943in}{1.584730in}}%
\pgfpathlineto{\pgfqpoint{3.573943in}{1.588988in}}%
\pgfpathlineto{\pgfqpoint{3.578200in}{1.588988in}}%
\pgfpathlineto{\pgfqpoint{3.578200in}{1.584730in}}%
\pgfpathmoveto{\pgfqpoint{3.569685in}{1.588988in}}%
\pgfpathlineto{\pgfqpoint{3.569685in}{1.588988in}}%
\pgfpathlineto{\pgfqpoint{3.569685in}{1.593246in}}%
\pgfpathlineto{\pgfqpoint{3.573943in}{1.593246in}}%
\pgfpathlineto{\pgfqpoint{3.573943in}{1.588988in}}%
\pgfpathmoveto{\pgfqpoint{3.569685in}{1.593246in}}%
\pgfpathlineto{\pgfqpoint{3.569685in}{1.593246in}}%
\pgfpathlineto{\pgfqpoint{3.569685in}{1.597504in}}%
\pgfpathlineto{\pgfqpoint{3.573943in}{1.597504in}}%
\pgfpathlineto{\pgfqpoint{3.573943in}{1.593246in}}%
\pgfpathmoveto{\pgfqpoint{3.573943in}{1.588988in}}%
\pgfpathlineto{\pgfqpoint{3.573943in}{1.588988in}}%
\pgfpathlineto{\pgfqpoint{3.573943in}{1.593246in}}%
\pgfpathlineto{\pgfqpoint{3.578200in}{1.593246in}}%
\pgfpathlineto{\pgfqpoint{3.578200in}{1.588988in}}%
\pgfpathmoveto{\pgfqpoint{3.573943in}{1.593246in}}%
\pgfpathlineto{\pgfqpoint{3.573943in}{1.593246in}}%
\pgfpathlineto{\pgfqpoint{3.573943in}{1.597504in}}%
\pgfpathlineto{\pgfqpoint{3.578200in}{1.597504in}}%
\pgfpathlineto{\pgfqpoint{3.578200in}{1.593246in}}%
\pgfpathmoveto{\pgfqpoint{3.578200in}{1.580472in}}%
\pgfpathlineto{\pgfqpoint{3.578200in}{1.580472in}}%
\pgfpathlineto{\pgfqpoint{3.578200in}{1.584730in}}%
\pgfpathlineto{\pgfqpoint{3.582458in}{1.584730in}}%
\pgfpathlineto{\pgfqpoint{3.582458in}{1.580472in}}%
\pgfpathmoveto{\pgfqpoint{3.578200in}{1.584730in}}%
\pgfpathlineto{\pgfqpoint{3.578200in}{1.584730in}}%
\pgfpathlineto{\pgfqpoint{3.578200in}{1.588988in}}%
\pgfpathlineto{\pgfqpoint{3.582458in}{1.588988in}}%
\pgfpathlineto{\pgfqpoint{3.582458in}{1.584730in}}%
\pgfpathmoveto{\pgfqpoint{3.582458in}{1.584730in}}%
\pgfpathlineto{\pgfqpoint{3.582458in}{1.584730in}}%
\pgfpathlineto{\pgfqpoint{3.582458in}{1.588988in}}%
\pgfpathlineto{\pgfqpoint{3.586716in}{1.588988in}}%
\pgfpathlineto{\pgfqpoint{3.586716in}{1.584730in}}%
\pgfpathmoveto{\pgfqpoint{3.578200in}{1.588988in}}%
\pgfpathlineto{\pgfqpoint{3.578200in}{1.588988in}}%
\pgfpathlineto{\pgfqpoint{3.578200in}{1.593246in}}%
\pgfpathlineto{\pgfqpoint{3.582458in}{1.593246in}}%
\pgfpathlineto{\pgfqpoint{3.582458in}{1.588988in}}%
\pgfpathmoveto{\pgfqpoint{3.578200in}{1.593246in}}%
\pgfpathlineto{\pgfqpoint{3.578200in}{1.593246in}}%
\pgfpathlineto{\pgfqpoint{3.578200in}{1.597504in}}%
\pgfpathlineto{\pgfqpoint{3.582458in}{1.597504in}}%
\pgfpathlineto{\pgfqpoint{3.582458in}{1.593246in}}%
\pgfpathmoveto{\pgfqpoint{3.582458in}{1.588988in}}%
\pgfpathlineto{\pgfqpoint{3.582458in}{1.588988in}}%
\pgfpathlineto{\pgfqpoint{3.582458in}{1.593246in}}%
\pgfpathlineto{\pgfqpoint{3.586716in}{1.593246in}}%
\pgfpathlineto{\pgfqpoint{3.586716in}{1.588988in}}%
\pgfpathmoveto{\pgfqpoint{3.582458in}{1.593246in}}%
\pgfpathlineto{\pgfqpoint{3.582458in}{1.593246in}}%
\pgfpathlineto{\pgfqpoint{3.582458in}{1.597504in}}%
\pgfpathlineto{\pgfqpoint{3.586716in}{1.597504in}}%
\pgfpathlineto{\pgfqpoint{3.586716in}{1.593246in}}%
\pgfpathmoveto{\pgfqpoint{3.586716in}{1.593246in}}%
\pgfpathlineto{\pgfqpoint{3.586716in}{1.593246in}}%
\pgfpathlineto{\pgfqpoint{3.586716in}{1.597504in}}%
\pgfpathlineto{\pgfqpoint{3.590974in}{1.597504in}}%
\pgfpathlineto{\pgfqpoint{3.590974in}{1.593246in}}%
\pgfpathmoveto{\pgfqpoint{3.569685in}{1.597504in}}%
\pgfpathlineto{\pgfqpoint{3.569685in}{1.597504in}}%
\pgfpathlineto{\pgfqpoint{3.569685in}{1.601762in}}%
\pgfpathlineto{\pgfqpoint{3.573943in}{1.601762in}}%
\pgfpathlineto{\pgfqpoint{3.573943in}{1.597504in}}%
\pgfpathmoveto{\pgfqpoint{3.573943in}{1.597504in}}%
\pgfpathlineto{\pgfqpoint{3.573943in}{1.597504in}}%
\pgfpathlineto{\pgfqpoint{3.573943in}{1.601762in}}%
\pgfpathlineto{\pgfqpoint{3.578200in}{1.601762in}}%
\pgfpathlineto{\pgfqpoint{3.578200in}{1.597504in}}%
\pgfpathmoveto{\pgfqpoint{3.573943in}{1.601762in}}%
\pgfpathlineto{\pgfqpoint{3.573943in}{1.601762in}}%
\pgfpathlineto{\pgfqpoint{3.573943in}{1.606019in}}%
\pgfpathlineto{\pgfqpoint{3.578200in}{1.606019in}}%
\pgfpathlineto{\pgfqpoint{3.578200in}{1.601762in}}%
\pgfpathmoveto{\pgfqpoint{3.573943in}{1.606019in}}%
\pgfpathlineto{\pgfqpoint{3.573943in}{1.606019in}}%
\pgfpathlineto{\pgfqpoint{3.573943in}{1.610277in}}%
\pgfpathlineto{\pgfqpoint{3.578200in}{1.610277in}}%
\pgfpathlineto{\pgfqpoint{3.578200in}{1.606019in}}%
\pgfpathmoveto{\pgfqpoint{3.578200in}{1.597504in}}%
\pgfpathlineto{\pgfqpoint{3.578200in}{1.597504in}}%
\pgfpathlineto{\pgfqpoint{3.578200in}{1.601762in}}%
\pgfpathlineto{\pgfqpoint{3.582458in}{1.601762in}}%
\pgfpathlineto{\pgfqpoint{3.582458in}{1.597504in}}%
\pgfpathmoveto{\pgfqpoint{3.578200in}{1.601762in}}%
\pgfpathlineto{\pgfqpoint{3.578200in}{1.601762in}}%
\pgfpathlineto{\pgfqpoint{3.578200in}{1.606019in}}%
\pgfpathlineto{\pgfqpoint{3.582458in}{1.606019in}}%
\pgfpathlineto{\pgfqpoint{3.582458in}{1.601762in}}%
\pgfpathmoveto{\pgfqpoint{3.582458in}{1.597504in}}%
\pgfpathlineto{\pgfqpoint{3.582458in}{1.597504in}}%
\pgfpathlineto{\pgfqpoint{3.582458in}{1.601762in}}%
\pgfpathlineto{\pgfqpoint{3.586716in}{1.601762in}}%
\pgfpathlineto{\pgfqpoint{3.586716in}{1.597504in}}%
\pgfpathmoveto{\pgfqpoint{3.582458in}{1.601762in}}%
\pgfpathlineto{\pgfqpoint{3.582458in}{1.601762in}}%
\pgfpathlineto{\pgfqpoint{3.582458in}{1.606019in}}%
\pgfpathlineto{\pgfqpoint{3.586716in}{1.606019in}}%
\pgfpathlineto{\pgfqpoint{3.586716in}{1.601762in}}%
\pgfpathmoveto{\pgfqpoint{3.578200in}{1.606019in}}%
\pgfpathlineto{\pgfqpoint{3.578200in}{1.606019in}}%
\pgfpathlineto{\pgfqpoint{3.578200in}{1.610277in}}%
\pgfpathlineto{\pgfqpoint{3.582458in}{1.610277in}}%
\pgfpathlineto{\pgfqpoint{3.582458in}{1.606019in}}%
\pgfpathmoveto{\pgfqpoint{3.578200in}{1.610277in}}%
\pgfpathlineto{\pgfqpoint{3.578200in}{1.610277in}}%
\pgfpathlineto{\pgfqpoint{3.578200in}{1.614535in}}%
\pgfpathlineto{\pgfqpoint{3.582458in}{1.614535in}}%
\pgfpathlineto{\pgfqpoint{3.582458in}{1.610277in}}%
\pgfpathmoveto{\pgfqpoint{3.582458in}{1.606019in}}%
\pgfpathlineto{\pgfqpoint{3.582458in}{1.606019in}}%
\pgfpathlineto{\pgfqpoint{3.582458in}{1.610277in}}%
\pgfpathlineto{\pgfqpoint{3.586716in}{1.610277in}}%
\pgfpathlineto{\pgfqpoint{3.586716in}{1.606019in}}%
\pgfpathmoveto{\pgfqpoint{3.582458in}{1.610277in}}%
\pgfpathlineto{\pgfqpoint{3.582458in}{1.610277in}}%
\pgfpathlineto{\pgfqpoint{3.582458in}{1.614535in}}%
\pgfpathlineto{\pgfqpoint{3.586716in}{1.614535in}}%
\pgfpathlineto{\pgfqpoint{3.586716in}{1.610277in}}%
\pgfpathmoveto{\pgfqpoint{3.578200in}{1.614535in}}%
\pgfpathlineto{\pgfqpoint{3.578200in}{1.614535in}}%
\pgfpathlineto{\pgfqpoint{3.578200in}{1.618793in}}%
\pgfpathlineto{\pgfqpoint{3.582458in}{1.618793in}}%
\pgfpathlineto{\pgfqpoint{3.582458in}{1.614535in}}%
\pgfpathmoveto{\pgfqpoint{3.578200in}{1.618793in}}%
\pgfpathlineto{\pgfqpoint{3.578200in}{1.618793in}}%
\pgfpathlineto{\pgfqpoint{3.578200in}{1.623050in}}%
\pgfpathlineto{\pgfqpoint{3.582458in}{1.623050in}}%
\pgfpathlineto{\pgfqpoint{3.582458in}{1.618793in}}%
\pgfpathmoveto{\pgfqpoint{3.582458in}{1.614535in}}%
\pgfpathlineto{\pgfqpoint{3.582458in}{1.614535in}}%
\pgfpathlineto{\pgfqpoint{3.582458in}{1.618793in}}%
\pgfpathlineto{\pgfqpoint{3.586716in}{1.618793in}}%
\pgfpathlineto{\pgfqpoint{3.586716in}{1.614535in}}%
\pgfpathmoveto{\pgfqpoint{3.582458in}{1.618793in}}%
\pgfpathlineto{\pgfqpoint{3.582458in}{1.618793in}}%
\pgfpathlineto{\pgfqpoint{3.582458in}{1.623050in}}%
\pgfpathlineto{\pgfqpoint{3.586716in}{1.623050in}}%
\pgfpathlineto{\pgfqpoint{3.586716in}{1.618793in}}%
\pgfpathmoveto{\pgfqpoint{3.582458in}{1.623050in}}%
\pgfpathlineto{\pgfqpoint{3.582458in}{1.623050in}}%
\pgfpathlineto{\pgfqpoint{3.582458in}{1.627308in}}%
\pgfpathlineto{\pgfqpoint{3.586716in}{1.627308in}}%
\pgfpathlineto{\pgfqpoint{3.586716in}{1.623050in}}%
\pgfpathmoveto{\pgfqpoint{3.582458in}{1.627308in}}%
\pgfpathlineto{\pgfqpoint{3.582458in}{1.627308in}}%
\pgfpathlineto{\pgfqpoint{3.582458in}{1.631566in}}%
\pgfpathlineto{\pgfqpoint{3.586716in}{1.631566in}}%
\pgfpathlineto{\pgfqpoint{3.586716in}{1.627308in}}%
\pgfpathmoveto{\pgfqpoint{3.586716in}{1.597504in}}%
\pgfpathlineto{\pgfqpoint{3.586716in}{1.597504in}}%
\pgfpathlineto{\pgfqpoint{3.586716in}{1.601762in}}%
\pgfpathlineto{\pgfqpoint{3.590974in}{1.601762in}}%
\pgfpathlineto{\pgfqpoint{3.590974in}{1.597504in}}%
\pgfpathmoveto{\pgfqpoint{3.586716in}{1.601762in}}%
\pgfpathlineto{\pgfqpoint{3.586716in}{1.601762in}}%
\pgfpathlineto{\pgfqpoint{3.586716in}{1.606019in}}%
\pgfpathlineto{\pgfqpoint{3.590974in}{1.606019in}}%
\pgfpathlineto{\pgfqpoint{3.590974in}{1.601762in}}%
\pgfpathmoveto{\pgfqpoint{3.586716in}{1.606019in}}%
\pgfpathlineto{\pgfqpoint{3.586716in}{1.606019in}}%
\pgfpathlineto{\pgfqpoint{3.586716in}{1.610277in}}%
\pgfpathlineto{\pgfqpoint{3.590974in}{1.610277in}}%
\pgfpathlineto{\pgfqpoint{3.590974in}{1.606019in}}%
\pgfpathmoveto{\pgfqpoint{3.586716in}{1.610277in}}%
\pgfpathlineto{\pgfqpoint{3.586716in}{1.610277in}}%
\pgfpathlineto{\pgfqpoint{3.586716in}{1.614535in}}%
\pgfpathlineto{\pgfqpoint{3.590974in}{1.614535in}}%
\pgfpathlineto{\pgfqpoint{3.590974in}{1.610277in}}%
\pgfpathmoveto{\pgfqpoint{3.590974in}{1.606019in}}%
\pgfpathlineto{\pgfqpoint{3.590974in}{1.606019in}}%
\pgfpathlineto{\pgfqpoint{3.590974in}{1.610277in}}%
\pgfpathlineto{\pgfqpoint{3.595231in}{1.610277in}}%
\pgfpathlineto{\pgfqpoint{3.595231in}{1.606019in}}%
\pgfpathmoveto{\pgfqpoint{3.590974in}{1.610277in}}%
\pgfpathlineto{\pgfqpoint{3.590974in}{1.610277in}}%
\pgfpathlineto{\pgfqpoint{3.590974in}{1.614535in}}%
\pgfpathlineto{\pgfqpoint{3.595231in}{1.614535in}}%
\pgfpathlineto{\pgfqpoint{3.595231in}{1.610277in}}%
\pgfpathmoveto{\pgfqpoint{3.586716in}{1.614535in}}%
\pgfpathlineto{\pgfqpoint{3.586716in}{1.614535in}}%
\pgfpathlineto{\pgfqpoint{3.586716in}{1.618793in}}%
\pgfpathlineto{\pgfqpoint{3.590974in}{1.618793in}}%
\pgfpathlineto{\pgfqpoint{3.590974in}{1.614535in}}%
\pgfpathmoveto{\pgfqpoint{3.586716in}{1.618793in}}%
\pgfpathlineto{\pgfqpoint{3.586716in}{1.618793in}}%
\pgfpathlineto{\pgfqpoint{3.586716in}{1.623050in}}%
\pgfpathlineto{\pgfqpoint{3.590974in}{1.623050in}}%
\pgfpathlineto{\pgfqpoint{3.590974in}{1.618793in}}%
\pgfpathmoveto{\pgfqpoint{3.590974in}{1.614535in}}%
\pgfpathlineto{\pgfqpoint{3.590974in}{1.614535in}}%
\pgfpathlineto{\pgfqpoint{3.590974in}{1.618793in}}%
\pgfpathlineto{\pgfqpoint{3.595231in}{1.618793in}}%
\pgfpathlineto{\pgfqpoint{3.595231in}{1.614535in}}%
\pgfpathmoveto{\pgfqpoint{3.590974in}{1.618793in}}%
\pgfpathlineto{\pgfqpoint{3.590974in}{1.618793in}}%
\pgfpathlineto{\pgfqpoint{3.590974in}{1.623050in}}%
\pgfpathlineto{\pgfqpoint{3.595231in}{1.623050in}}%
\pgfpathlineto{\pgfqpoint{3.595231in}{1.618793in}}%
\pgfpathmoveto{\pgfqpoint{3.586716in}{1.623050in}}%
\pgfpathlineto{\pgfqpoint{3.586716in}{1.623050in}}%
\pgfpathlineto{\pgfqpoint{3.586716in}{1.627308in}}%
\pgfpathlineto{\pgfqpoint{3.590974in}{1.627308in}}%
\pgfpathlineto{\pgfqpoint{3.590974in}{1.623050in}}%
\pgfpathmoveto{\pgfqpoint{3.586716in}{1.627308in}}%
\pgfpathlineto{\pgfqpoint{3.586716in}{1.627308in}}%
\pgfpathlineto{\pgfqpoint{3.586716in}{1.631566in}}%
\pgfpathlineto{\pgfqpoint{3.590974in}{1.631566in}}%
\pgfpathlineto{\pgfqpoint{3.590974in}{1.627308in}}%
\pgfpathmoveto{\pgfqpoint{3.590974in}{1.623050in}}%
\pgfpathlineto{\pgfqpoint{3.590974in}{1.623050in}}%
\pgfpathlineto{\pgfqpoint{3.590974in}{1.627308in}}%
\pgfpathlineto{\pgfqpoint{3.595231in}{1.627308in}}%
\pgfpathlineto{\pgfqpoint{3.595231in}{1.623050in}}%
\pgfpathmoveto{\pgfqpoint{3.590974in}{1.627308in}}%
\pgfpathlineto{\pgfqpoint{3.590974in}{1.627308in}}%
\pgfpathlineto{\pgfqpoint{3.590974in}{1.631566in}}%
\pgfpathlineto{\pgfqpoint{3.595231in}{1.631566in}}%
\pgfpathlineto{\pgfqpoint{3.595231in}{1.627308in}}%
\pgfpathmoveto{\pgfqpoint{3.595231in}{1.614535in}}%
\pgfpathlineto{\pgfqpoint{3.595231in}{1.614535in}}%
\pgfpathlineto{\pgfqpoint{3.595231in}{1.618793in}}%
\pgfpathlineto{\pgfqpoint{3.599489in}{1.618793in}}%
\pgfpathlineto{\pgfqpoint{3.599489in}{1.614535in}}%
\pgfpathmoveto{\pgfqpoint{3.595231in}{1.618793in}}%
\pgfpathlineto{\pgfqpoint{3.595231in}{1.618793in}}%
\pgfpathlineto{\pgfqpoint{3.595231in}{1.623050in}}%
\pgfpathlineto{\pgfqpoint{3.599489in}{1.623050in}}%
\pgfpathlineto{\pgfqpoint{3.599489in}{1.618793in}}%
\pgfpathmoveto{\pgfqpoint{3.595231in}{1.623050in}}%
\pgfpathlineto{\pgfqpoint{3.595231in}{1.623050in}}%
\pgfpathlineto{\pgfqpoint{3.595231in}{1.627308in}}%
\pgfpathlineto{\pgfqpoint{3.599489in}{1.627308in}}%
\pgfpathlineto{\pgfqpoint{3.599489in}{1.623050in}}%
\pgfpathmoveto{\pgfqpoint{3.595231in}{1.627308in}}%
\pgfpathlineto{\pgfqpoint{3.595231in}{1.627308in}}%
\pgfpathlineto{\pgfqpoint{3.595231in}{1.631566in}}%
\pgfpathlineto{\pgfqpoint{3.599489in}{1.631566in}}%
\pgfpathlineto{\pgfqpoint{3.599489in}{1.627308in}}%
\pgfpathmoveto{\pgfqpoint{3.599489in}{1.627308in}}%
\pgfpathlineto{\pgfqpoint{3.599489in}{1.627308in}}%
\pgfpathlineto{\pgfqpoint{3.599489in}{1.631566in}}%
\pgfpathlineto{\pgfqpoint{3.603747in}{1.631566in}}%
\pgfpathlineto{\pgfqpoint{3.603747in}{1.627308in}}%
\pgfpathmoveto{\pgfqpoint{3.586716in}{1.631566in}}%
\pgfpathlineto{\pgfqpoint{3.586716in}{1.631566in}}%
\pgfpathlineto{\pgfqpoint{3.586716in}{1.635823in}}%
\pgfpathlineto{\pgfqpoint{3.590974in}{1.635823in}}%
\pgfpathlineto{\pgfqpoint{3.590974in}{1.631566in}}%
\pgfpathmoveto{\pgfqpoint{3.586716in}{1.635823in}}%
\pgfpathlineto{\pgfqpoint{3.586716in}{1.635823in}}%
\pgfpathlineto{\pgfqpoint{3.586716in}{1.640081in}}%
\pgfpathlineto{\pgfqpoint{3.590974in}{1.640081in}}%
\pgfpathlineto{\pgfqpoint{3.590974in}{1.635823in}}%
\pgfpathmoveto{\pgfqpoint{3.590974in}{1.631566in}}%
\pgfpathlineto{\pgfqpoint{3.590974in}{1.631566in}}%
\pgfpathlineto{\pgfqpoint{3.590974in}{1.635823in}}%
\pgfpathlineto{\pgfqpoint{3.595231in}{1.635823in}}%
\pgfpathlineto{\pgfqpoint{3.595231in}{1.631566in}}%
\pgfpathmoveto{\pgfqpoint{3.590974in}{1.635823in}}%
\pgfpathlineto{\pgfqpoint{3.590974in}{1.635823in}}%
\pgfpathlineto{\pgfqpoint{3.590974in}{1.640081in}}%
\pgfpathlineto{\pgfqpoint{3.595231in}{1.640081in}}%
\pgfpathlineto{\pgfqpoint{3.595231in}{1.635823in}}%
\pgfpathmoveto{\pgfqpoint{3.586716in}{1.640081in}}%
\pgfpathlineto{\pgfqpoint{3.586716in}{1.640081in}}%
\pgfpathlineto{\pgfqpoint{3.586716in}{1.644339in}}%
\pgfpathlineto{\pgfqpoint{3.590974in}{1.644339in}}%
\pgfpathlineto{\pgfqpoint{3.590974in}{1.640081in}}%
\pgfpathmoveto{\pgfqpoint{3.590974in}{1.640081in}}%
\pgfpathlineto{\pgfqpoint{3.590974in}{1.640081in}}%
\pgfpathlineto{\pgfqpoint{3.590974in}{1.644339in}}%
\pgfpathlineto{\pgfqpoint{3.595231in}{1.644339in}}%
\pgfpathlineto{\pgfqpoint{3.595231in}{1.640081in}}%
\pgfpathmoveto{\pgfqpoint{3.590974in}{1.644339in}}%
\pgfpathlineto{\pgfqpoint{3.590974in}{1.644339in}}%
\pgfpathlineto{\pgfqpoint{3.590974in}{1.648597in}}%
\pgfpathlineto{\pgfqpoint{3.595231in}{1.648597in}}%
\pgfpathlineto{\pgfqpoint{3.595231in}{1.644339in}}%
\pgfpathmoveto{\pgfqpoint{3.595231in}{1.631566in}}%
\pgfpathlineto{\pgfqpoint{3.595231in}{1.631566in}}%
\pgfpathlineto{\pgfqpoint{3.595231in}{1.635823in}}%
\pgfpathlineto{\pgfqpoint{3.599489in}{1.635823in}}%
\pgfpathlineto{\pgfqpoint{3.599489in}{1.631566in}}%
\pgfpathmoveto{\pgfqpoint{3.595231in}{1.635823in}}%
\pgfpathlineto{\pgfqpoint{3.595231in}{1.635823in}}%
\pgfpathlineto{\pgfqpoint{3.595231in}{1.640081in}}%
\pgfpathlineto{\pgfqpoint{3.599489in}{1.640081in}}%
\pgfpathlineto{\pgfqpoint{3.599489in}{1.635823in}}%
\pgfpathmoveto{\pgfqpoint{3.599489in}{1.631566in}}%
\pgfpathlineto{\pgfqpoint{3.599489in}{1.631566in}}%
\pgfpathlineto{\pgfqpoint{3.599489in}{1.635823in}}%
\pgfpathlineto{\pgfqpoint{3.603747in}{1.635823in}}%
\pgfpathlineto{\pgfqpoint{3.603747in}{1.631566in}}%
\pgfpathmoveto{\pgfqpoint{3.599489in}{1.635823in}}%
\pgfpathlineto{\pgfqpoint{3.599489in}{1.635823in}}%
\pgfpathlineto{\pgfqpoint{3.599489in}{1.640081in}}%
\pgfpathlineto{\pgfqpoint{3.603747in}{1.640081in}}%
\pgfpathlineto{\pgfqpoint{3.603747in}{1.635823in}}%
\pgfpathmoveto{\pgfqpoint{3.595231in}{1.640081in}}%
\pgfpathlineto{\pgfqpoint{3.595231in}{1.640081in}}%
\pgfpathlineto{\pgfqpoint{3.595231in}{1.644339in}}%
\pgfpathlineto{\pgfqpoint{3.599489in}{1.644339in}}%
\pgfpathlineto{\pgfqpoint{3.599489in}{1.640081in}}%
\pgfpathmoveto{\pgfqpoint{3.595231in}{1.644339in}}%
\pgfpathlineto{\pgfqpoint{3.595231in}{1.644339in}}%
\pgfpathlineto{\pgfqpoint{3.595231in}{1.648597in}}%
\pgfpathlineto{\pgfqpoint{3.599489in}{1.648597in}}%
\pgfpathlineto{\pgfqpoint{3.599489in}{1.644339in}}%
\pgfpathmoveto{\pgfqpoint{3.599489in}{1.640081in}}%
\pgfpathlineto{\pgfqpoint{3.599489in}{1.640081in}}%
\pgfpathlineto{\pgfqpoint{3.599489in}{1.644339in}}%
\pgfpathlineto{\pgfqpoint{3.603747in}{1.644339in}}%
\pgfpathlineto{\pgfqpoint{3.603747in}{1.640081in}}%
\pgfpathmoveto{\pgfqpoint{3.599489in}{1.644339in}}%
\pgfpathlineto{\pgfqpoint{3.599489in}{1.644339in}}%
\pgfpathlineto{\pgfqpoint{3.599489in}{1.648597in}}%
\pgfpathlineto{\pgfqpoint{3.603747in}{1.648597in}}%
\pgfpathlineto{\pgfqpoint{3.603747in}{1.644339in}}%
\pgfpathmoveto{\pgfqpoint{3.590974in}{1.648597in}}%
\pgfpathlineto{\pgfqpoint{3.590974in}{1.648597in}}%
\pgfpathlineto{\pgfqpoint{3.590974in}{1.652854in}}%
\pgfpathlineto{\pgfqpoint{3.595231in}{1.652854in}}%
\pgfpathlineto{\pgfqpoint{3.595231in}{1.648597in}}%
\pgfpathmoveto{\pgfqpoint{3.595231in}{1.648597in}}%
\pgfpathlineto{\pgfqpoint{3.595231in}{1.648597in}}%
\pgfpathlineto{\pgfqpoint{3.595231in}{1.652854in}}%
\pgfpathlineto{\pgfqpoint{3.599489in}{1.652854in}}%
\pgfpathlineto{\pgfqpoint{3.599489in}{1.648597in}}%
\pgfpathmoveto{\pgfqpoint{3.595231in}{1.652854in}}%
\pgfpathlineto{\pgfqpoint{3.595231in}{1.652854in}}%
\pgfpathlineto{\pgfqpoint{3.595231in}{1.657112in}}%
\pgfpathlineto{\pgfqpoint{3.599489in}{1.657112in}}%
\pgfpathlineto{\pgfqpoint{3.599489in}{1.652854in}}%
\pgfpathmoveto{\pgfqpoint{3.599489in}{1.648597in}}%
\pgfpathlineto{\pgfqpoint{3.599489in}{1.648597in}}%
\pgfpathlineto{\pgfqpoint{3.599489in}{1.652854in}}%
\pgfpathlineto{\pgfqpoint{3.603747in}{1.652854in}}%
\pgfpathlineto{\pgfqpoint{3.603747in}{1.648597in}}%
\pgfpathmoveto{\pgfqpoint{3.599489in}{1.652854in}}%
\pgfpathlineto{\pgfqpoint{3.599489in}{1.652854in}}%
\pgfpathlineto{\pgfqpoint{3.599489in}{1.657112in}}%
\pgfpathlineto{\pgfqpoint{3.603747in}{1.657112in}}%
\pgfpathlineto{\pgfqpoint{3.603747in}{1.652854in}}%
\pgfpathmoveto{\pgfqpoint{3.595231in}{1.657112in}}%
\pgfpathlineto{\pgfqpoint{3.595231in}{1.657112in}}%
\pgfpathlineto{\pgfqpoint{3.595231in}{1.661370in}}%
\pgfpathlineto{\pgfqpoint{3.599489in}{1.661370in}}%
\pgfpathlineto{\pgfqpoint{3.599489in}{1.657112in}}%
\pgfpathmoveto{\pgfqpoint{3.595231in}{1.661370in}}%
\pgfpathlineto{\pgfqpoint{3.595231in}{1.661370in}}%
\pgfpathlineto{\pgfqpoint{3.595231in}{1.665627in}}%
\pgfpathlineto{\pgfqpoint{3.599489in}{1.665627in}}%
\pgfpathlineto{\pgfqpoint{3.599489in}{1.661370in}}%
\pgfpathmoveto{\pgfqpoint{3.599489in}{1.657112in}}%
\pgfpathlineto{\pgfqpoint{3.599489in}{1.657112in}}%
\pgfpathlineto{\pgfqpoint{3.599489in}{1.661370in}}%
\pgfpathlineto{\pgfqpoint{3.603747in}{1.661370in}}%
\pgfpathlineto{\pgfqpoint{3.603747in}{1.657112in}}%
\pgfpathmoveto{\pgfqpoint{3.599489in}{1.661370in}}%
\pgfpathlineto{\pgfqpoint{3.599489in}{1.661370in}}%
\pgfpathlineto{\pgfqpoint{3.599489in}{1.665627in}}%
\pgfpathlineto{\pgfqpoint{3.603747in}{1.665627in}}%
\pgfpathlineto{\pgfqpoint{3.603747in}{1.661370in}}%
\pgfpathmoveto{\pgfqpoint{3.603747in}{1.640081in}}%
\pgfpathlineto{\pgfqpoint{3.603747in}{1.640081in}}%
\pgfpathlineto{\pgfqpoint{3.603747in}{1.644339in}}%
\pgfpathlineto{\pgfqpoint{3.608005in}{1.644339in}}%
\pgfpathlineto{\pgfqpoint{3.608005in}{1.640081in}}%
\pgfpathmoveto{\pgfqpoint{3.603747in}{1.644339in}}%
\pgfpathlineto{\pgfqpoint{3.603747in}{1.644339in}}%
\pgfpathlineto{\pgfqpoint{3.603747in}{1.648597in}}%
\pgfpathlineto{\pgfqpoint{3.608005in}{1.648597in}}%
\pgfpathlineto{\pgfqpoint{3.608005in}{1.644339in}}%
\pgfpathmoveto{\pgfqpoint{3.603747in}{1.648597in}}%
\pgfpathlineto{\pgfqpoint{3.603747in}{1.648597in}}%
\pgfpathlineto{\pgfqpoint{3.603747in}{1.652854in}}%
\pgfpathlineto{\pgfqpoint{3.608005in}{1.652854in}}%
\pgfpathlineto{\pgfqpoint{3.608005in}{1.648597in}}%
\pgfpathmoveto{\pgfqpoint{3.603747in}{1.652854in}}%
\pgfpathlineto{\pgfqpoint{3.603747in}{1.652854in}}%
\pgfpathlineto{\pgfqpoint{3.603747in}{1.657112in}}%
\pgfpathlineto{\pgfqpoint{3.608005in}{1.657112in}}%
\pgfpathlineto{\pgfqpoint{3.608005in}{1.652854in}}%
\pgfpathmoveto{\pgfqpoint{3.608005in}{1.652854in}}%
\pgfpathlineto{\pgfqpoint{3.608005in}{1.652854in}}%
\pgfpathlineto{\pgfqpoint{3.608005in}{1.657112in}}%
\pgfpathlineto{\pgfqpoint{3.612262in}{1.657112in}}%
\pgfpathlineto{\pgfqpoint{3.612262in}{1.652854in}}%
\pgfpathmoveto{\pgfqpoint{3.603747in}{1.657112in}}%
\pgfpathlineto{\pgfqpoint{3.603747in}{1.657112in}}%
\pgfpathlineto{\pgfqpoint{3.603747in}{1.661370in}}%
\pgfpathlineto{\pgfqpoint{3.608005in}{1.661370in}}%
\pgfpathlineto{\pgfqpoint{3.608005in}{1.657112in}}%
\pgfpathmoveto{\pgfqpoint{3.603747in}{1.661370in}}%
\pgfpathlineto{\pgfqpoint{3.603747in}{1.661370in}}%
\pgfpathlineto{\pgfqpoint{3.603747in}{1.665627in}}%
\pgfpathlineto{\pgfqpoint{3.608005in}{1.665627in}}%
\pgfpathlineto{\pgfqpoint{3.608005in}{1.661370in}}%
\pgfpathmoveto{\pgfqpoint{3.608005in}{1.657112in}}%
\pgfpathlineto{\pgfqpoint{3.608005in}{1.657112in}}%
\pgfpathlineto{\pgfqpoint{3.608005in}{1.661370in}}%
\pgfpathlineto{\pgfqpoint{3.612262in}{1.661370in}}%
\pgfpathlineto{\pgfqpoint{3.612262in}{1.657112in}}%
\pgfpathmoveto{\pgfqpoint{3.608005in}{1.661370in}}%
\pgfpathlineto{\pgfqpoint{3.608005in}{1.661370in}}%
\pgfpathlineto{\pgfqpoint{3.608005in}{1.665627in}}%
\pgfpathlineto{\pgfqpoint{3.612262in}{1.665627in}}%
\pgfpathlineto{\pgfqpoint{3.612262in}{1.661370in}}%
\pgfpathmoveto{\pgfqpoint{3.599489in}{1.665627in}}%
\pgfpathlineto{\pgfqpoint{3.599489in}{1.665627in}}%
\pgfpathlineto{\pgfqpoint{3.599489in}{1.669885in}}%
\pgfpathlineto{\pgfqpoint{3.603747in}{1.669885in}}%
\pgfpathlineto{\pgfqpoint{3.603747in}{1.665627in}}%
\pgfpathmoveto{\pgfqpoint{3.599489in}{1.669885in}}%
\pgfpathlineto{\pgfqpoint{3.599489in}{1.669885in}}%
\pgfpathlineto{\pgfqpoint{3.599489in}{1.674143in}}%
\pgfpathlineto{\pgfqpoint{3.603747in}{1.674143in}}%
\pgfpathlineto{\pgfqpoint{3.603747in}{1.669885in}}%
\pgfpathmoveto{\pgfqpoint{3.599489in}{1.674143in}}%
\pgfpathlineto{\pgfqpoint{3.599489in}{1.674143in}}%
\pgfpathlineto{\pgfqpoint{3.599489in}{1.678401in}}%
\pgfpathlineto{\pgfqpoint{3.603747in}{1.678401in}}%
\pgfpathlineto{\pgfqpoint{3.603747in}{1.674143in}}%
\pgfpathmoveto{\pgfqpoint{3.603747in}{1.665627in}}%
\pgfpathlineto{\pgfqpoint{3.603747in}{1.665627in}}%
\pgfpathlineto{\pgfqpoint{3.603747in}{1.669885in}}%
\pgfpathlineto{\pgfqpoint{3.608005in}{1.669885in}}%
\pgfpathlineto{\pgfqpoint{3.608005in}{1.665627in}}%
\pgfpathmoveto{\pgfqpoint{3.603747in}{1.669885in}}%
\pgfpathlineto{\pgfqpoint{3.603747in}{1.669885in}}%
\pgfpathlineto{\pgfqpoint{3.603747in}{1.674143in}}%
\pgfpathlineto{\pgfqpoint{3.608005in}{1.674143in}}%
\pgfpathlineto{\pgfqpoint{3.608005in}{1.669885in}}%
\pgfpathmoveto{\pgfqpoint{3.608005in}{1.665627in}}%
\pgfpathlineto{\pgfqpoint{3.608005in}{1.665627in}}%
\pgfpathlineto{\pgfqpoint{3.608005in}{1.669885in}}%
\pgfpathlineto{\pgfqpoint{3.612262in}{1.669885in}}%
\pgfpathlineto{\pgfqpoint{3.612262in}{1.665627in}}%
\pgfpathmoveto{\pgfqpoint{3.608005in}{1.669885in}}%
\pgfpathlineto{\pgfqpoint{3.608005in}{1.669885in}}%
\pgfpathlineto{\pgfqpoint{3.608005in}{1.674143in}}%
\pgfpathlineto{\pgfqpoint{3.612262in}{1.674143in}}%
\pgfpathlineto{\pgfqpoint{3.612262in}{1.669885in}}%
\pgfpathmoveto{\pgfqpoint{3.603747in}{1.674143in}}%
\pgfpathlineto{\pgfqpoint{3.603747in}{1.674143in}}%
\pgfpathlineto{\pgfqpoint{3.603747in}{1.678401in}}%
\pgfpathlineto{\pgfqpoint{3.608005in}{1.678401in}}%
\pgfpathlineto{\pgfqpoint{3.608005in}{1.674143in}}%
\pgfpathmoveto{\pgfqpoint{3.603747in}{1.678401in}}%
\pgfpathlineto{\pgfqpoint{3.603747in}{1.678401in}}%
\pgfpathlineto{\pgfqpoint{3.603747in}{1.682658in}}%
\pgfpathlineto{\pgfqpoint{3.608005in}{1.682658in}}%
\pgfpathlineto{\pgfqpoint{3.608005in}{1.678401in}}%
\pgfpathmoveto{\pgfqpoint{3.608005in}{1.674143in}}%
\pgfpathlineto{\pgfqpoint{3.608005in}{1.674143in}}%
\pgfpathlineto{\pgfqpoint{3.608005in}{1.678401in}}%
\pgfpathlineto{\pgfqpoint{3.612262in}{1.678401in}}%
\pgfpathlineto{\pgfqpoint{3.612262in}{1.674143in}}%
\pgfpathmoveto{\pgfqpoint{3.608005in}{1.678401in}}%
\pgfpathlineto{\pgfqpoint{3.608005in}{1.678401in}}%
\pgfpathlineto{\pgfqpoint{3.608005in}{1.682658in}}%
\pgfpathlineto{\pgfqpoint{3.612262in}{1.682658in}}%
\pgfpathlineto{\pgfqpoint{3.612262in}{1.678401in}}%
\pgfpathmoveto{\pgfqpoint{3.612262in}{1.665627in}}%
\pgfpathlineto{\pgfqpoint{3.612262in}{1.665627in}}%
\pgfpathlineto{\pgfqpoint{3.612262in}{1.669885in}}%
\pgfpathlineto{\pgfqpoint{3.616520in}{1.669885in}}%
\pgfpathlineto{\pgfqpoint{3.616520in}{1.665627in}}%
\pgfpathmoveto{\pgfqpoint{3.612262in}{1.669885in}}%
\pgfpathlineto{\pgfqpoint{3.612262in}{1.669885in}}%
\pgfpathlineto{\pgfqpoint{3.612262in}{1.674143in}}%
\pgfpathlineto{\pgfqpoint{3.616520in}{1.674143in}}%
\pgfpathlineto{\pgfqpoint{3.616520in}{1.669885in}}%
\pgfpathmoveto{\pgfqpoint{3.612262in}{1.674143in}}%
\pgfpathlineto{\pgfqpoint{3.612262in}{1.674143in}}%
\pgfpathlineto{\pgfqpoint{3.612262in}{1.678401in}}%
\pgfpathlineto{\pgfqpoint{3.616520in}{1.678401in}}%
\pgfpathlineto{\pgfqpoint{3.616520in}{1.674143in}}%
\pgfpathmoveto{\pgfqpoint{3.612262in}{1.678401in}}%
\pgfpathlineto{\pgfqpoint{3.612262in}{1.678401in}}%
\pgfpathlineto{\pgfqpoint{3.612262in}{1.682658in}}%
\pgfpathlineto{\pgfqpoint{3.616520in}{1.682658in}}%
\pgfpathlineto{\pgfqpoint{3.616520in}{1.678401in}}%
\pgfpathmoveto{\pgfqpoint{3.616520in}{1.678401in}}%
\pgfpathlineto{\pgfqpoint{3.616520in}{1.678401in}}%
\pgfpathlineto{\pgfqpoint{3.616520in}{1.682658in}}%
\pgfpathlineto{\pgfqpoint{3.620778in}{1.682658in}}%
\pgfpathlineto{\pgfqpoint{3.620778in}{1.678401in}}%
\pgfpathmoveto{\pgfqpoint{3.603747in}{1.682658in}}%
\pgfpathlineto{\pgfqpoint{3.603747in}{1.682658in}}%
\pgfpathlineto{\pgfqpoint{3.603747in}{1.686916in}}%
\pgfpathlineto{\pgfqpoint{3.608005in}{1.686916in}}%
\pgfpathlineto{\pgfqpoint{3.608005in}{1.682658in}}%
\pgfpathmoveto{\pgfqpoint{3.603747in}{1.686916in}}%
\pgfpathlineto{\pgfqpoint{3.603747in}{1.686916in}}%
\pgfpathlineto{\pgfqpoint{3.603747in}{1.691174in}}%
\pgfpathlineto{\pgfqpoint{3.608005in}{1.691174in}}%
\pgfpathlineto{\pgfqpoint{3.608005in}{1.686916in}}%
\pgfpathmoveto{\pgfqpoint{3.608005in}{1.682658in}}%
\pgfpathlineto{\pgfqpoint{3.608005in}{1.682658in}}%
\pgfpathlineto{\pgfqpoint{3.608005in}{1.686916in}}%
\pgfpathlineto{\pgfqpoint{3.612262in}{1.686916in}}%
\pgfpathlineto{\pgfqpoint{3.612262in}{1.682658in}}%
\pgfpathmoveto{\pgfqpoint{3.608005in}{1.686916in}}%
\pgfpathlineto{\pgfqpoint{3.608005in}{1.686916in}}%
\pgfpathlineto{\pgfqpoint{3.608005in}{1.691174in}}%
\pgfpathlineto{\pgfqpoint{3.612262in}{1.691174in}}%
\pgfpathlineto{\pgfqpoint{3.612262in}{1.686916in}}%
\pgfpathmoveto{\pgfqpoint{3.608005in}{1.691174in}}%
\pgfpathlineto{\pgfqpoint{3.608005in}{1.691174in}}%
\pgfpathlineto{\pgfqpoint{3.608005in}{1.695431in}}%
\pgfpathlineto{\pgfqpoint{3.612262in}{1.695431in}}%
\pgfpathlineto{\pgfqpoint{3.612262in}{1.691174in}}%
\pgfpathmoveto{\pgfqpoint{3.608005in}{1.695431in}}%
\pgfpathlineto{\pgfqpoint{3.608005in}{1.695431in}}%
\pgfpathlineto{\pgfqpoint{3.608005in}{1.699689in}}%
\pgfpathlineto{\pgfqpoint{3.612262in}{1.699689in}}%
\pgfpathlineto{\pgfqpoint{3.612262in}{1.695431in}}%
\pgfpathmoveto{\pgfqpoint{3.612262in}{1.682658in}}%
\pgfpathlineto{\pgfqpoint{3.612262in}{1.682658in}}%
\pgfpathlineto{\pgfqpoint{3.612262in}{1.686916in}}%
\pgfpathlineto{\pgfqpoint{3.616520in}{1.686916in}}%
\pgfpathlineto{\pgfqpoint{3.616520in}{1.682658in}}%
\pgfpathmoveto{\pgfqpoint{3.612262in}{1.686916in}}%
\pgfpathlineto{\pgfqpoint{3.612262in}{1.686916in}}%
\pgfpathlineto{\pgfqpoint{3.612262in}{1.691174in}}%
\pgfpathlineto{\pgfqpoint{3.616520in}{1.691174in}}%
\pgfpathlineto{\pgfqpoint{3.616520in}{1.686916in}}%
\pgfpathmoveto{\pgfqpoint{3.616520in}{1.682658in}}%
\pgfpathlineto{\pgfqpoint{3.616520in}{1.682658in}}%
\pgfpathlineto{\pgfqpoint{3.616520in}{1.686916in}}%
\pgfpathlineto{\pgfqpoint{3.620778in}{1.686916in}}%
\pgfpathlineto{\pgfqpoint{3.620778in}{1.682658in}}%
\pgfpathmoveto{\pgfqpoint{3.616520in}{1.686916in}}%
\pgfpathlineto{\pgfqpoint{3.616520in}{1.686916in}}%
\pgfpathlineto{\pgfqpoint{3.616520in}{1.691174in}}%
\pgfpathlineto{\pgfqpoint{3.620778in}{1.691174in}}%
\pgfpathlineto{\pgfqpoint{3.620778in}{1.686916in}}%
\pgfpathmoveto{\pgfqpoint{3.612262in}{1.691174in}}%
\pgfpathlineto{\pgfqpoint{3.612262in}{1.691174in}}%
\pgfpathlineto{\pgfqpoint{3.612262in}{1.695431in}}%
\pgfpathlineto{\pgfqpoint{3.616520in}{1.695431in}}%
\pgfpathlineto{\pgfqpoint{3.616520in}{1.691174in}}%
\pgfpathmoveto{\pgfqpoint{3.612262in}{1.695431in}}%
\pgfpathlineto{\pgfqpoint{3.612262in}{1.695431in}}%
\pgfpathlineto{\pgfqpoint{3.612262in}{1.699689in}}%
\pgfpathlineto{\pgfqpoint{3.616520in}{1.699689in}}%
\pgfpathlineto{\pgfqpoint{3.616520in}{1.695431in}}%
\pgfpathmoveto{\pgfqpoint{3.616520in}{1.691174in}}%
\pgfpathlineto{\pgfqpoint{3.616520in}{1.691174in}}%
\pgfpathlineto{\pgfqpoint{3.616520in}{1.695431in}}%
\pgfpathlineto{\pgfqpoint{3.620778in}{1.695431in}}%
\pgfpathlineto{\pgfqpoint{3.620778in}{1.691174in}}%
\pgfpathmoveto{\pgfqpoint{3.616520in}{1.695431in}}%
\pgfpathlineto{\pgfqpoint{3.616520in}{1.695431in}}%
\pgfpathlineto{\pgfqpoint{3.616520in}{1.699689in}}%
\pgfpathlineto{\pgfqpoint{3.620778in}{1.699689in}}%
\pgfpathlineto{\pgfqpoint{3.620778in}{1.695431in}}%
\pgfpathmoveto{\pgfqpoint{3.620778in}{1.691174in}}%
\pgfpathlineto{\pgfqpoint{3.620778in}{1.691174in}}%
\pgfpathlineto{\pgfqpoint{3.620778in}{1.695431in}}%
\pgfpathlineto{\pgfqpoint{3.625036in}{1.695431in}}%
\pgfpathlineto{\pgfqpoint{3.625036in}{1.691174in}}%
\pgfpathmoveto{\pgfqpoint{3.620778in}{1.695431in}}%
\pgfpathlineto{\pgfqpoint{3.620778in}{1.695431in}}%
\pgfpathlineto{\pgfqpoint{3.620778in}{1.699689in}}%
\pgfpathlineto{\pgfqpoint{3.625036in}{1.699689in}}%
\pgfpathlineto{\pgfqpoint{3.625036in}{1.695431in}}%
\pgfpathmoveto{\pgfqpoint{3.608005in}{1.699689in}}%
\pgfpathlineto{\pgfqpoint{3.608005in}{1.699689in}}%
\pgfpathlineto{\pgfqpoint{3.608005in}{1.703947in}}%
\pgfpathlineto{\pgfqpoint{3.612262in}{1.703947in}}%
\pgfpathlineto{\pgfqpoint{3.612262in}{1.699689in}}%
\pgfpathmoveto{\pgfqpoint{3.612262in}{1.699689in}}%
\pgfpathlineto{\pgfqpoint{3.612262in}{1.699689in}}%
\pgfpathlineto{\pgfqpoint{3.612262in}{1.703947in}}%
\pgfpathlineto{\pgfqpoint{3.616520in}{1.703947in}}%
\pgfpathlineto{\pgfqpoint{3.616520in}{1.699689in}}%
\pgfpathmoveto{\pgfqpoint{3.612262in}{1.703947in}}%
\pgfpathlineto{\pgfqpoint{3.612262in}{1.703947in}}%
\pgfpathlineto{\pgfqpoint{3.612262in}{1.708205in}}%
\pgfpathlineto{\pgfqpoint{3.616520in}{1.708205in}}%
\pgfpathlineto{\pgfqpoint{3.616520in}{1.703947in}}%
\pgfpathmoveto{\pgfqpoint{3.616520in}{1.699689in}}%
\pgfpathlineto{\pgfqpoint{3.616520in}{1.699689in}}%
\pgfpathlineto{\pgfqpoint{3.616520in}{1.703947in}}%
\pgfpathlineto{\pgfqpoint{3.620778in}{1.703947in}}%
\pgfpathlineto{\pgfqpoint{3.620778in}{1.699689in}}%
\pgfpathmoveto{\pgfqpoint{3.616520in}{1.703947in}}%
\pgfpathlineto{\pgfqpoint{3.616520in}{1.703947in}}%
\pgfpathlineto{\pgfqpoint{3.616520in}{1.708205in}}%
\pgfpathlineto{\pgfqpoint{3.620778in}{1.708205in}}%
\pgfpathlineto{\pgfqpoint{3.620778in}{1.703947in}}%
\pgfpathmoveto{\pgfqpoint{3.612262in}{1.708205in}}%
\pgfpathlineto{\pgfqpoint{3.612262in}{1.708205in}}%
\pgfpathlineto{\pgfqpoint{3.612262in}{1.712462in}}%
\pgfpathlineto{\pgfqpoint{3.616520in}{1.712462in}}%
\pgfpathlineto{\pgfqpoint{3.616520in}{1.708205in}}%
\pgfpathmoveto{\pgfqpoint{3.612262in}{1.712462in}}%
\pgfpathlineto{\pgfqpoint{3.612262in}{1.712462in}}%
\pgfpathlineto{\pgfqpoint{3.612262in}{1.716720in}}%
\pgfpathlineto{\pgfqpoint{3.616520in}{1.716720in}}%
\pgfpathlineto{\pgfqpoint{3.616520in}{1.712462in}}%
\pgfpathmoveto{\pgfqpoint{3.616520in}{1.708205in}}%
\pgfpathlineto{\pgfqpoint{3.616520in}{1.708205in}}%
\pgfpathlineto{\pgfqpoint{3.616520in}{1.712462in}}%
\pgfpathlineto{\pgfqpoint{3.620778in}{1.712462in}}%
\pgfpathlineto{\pgfqpoint{3.620778in}{1.708205in}}%
\pgfpathmoveto{\pgfqpoint{3.616520in}{1.712462in}}%
\pgfpathlineto{\pgfqpoint{3.616520in}{1.712462in}}%
\pgfpathlineto{\pgfqpoint{3.616520in}{1.716720in}}%
\pgfpathlineto{\pgfqpoint{3.620778in}{1.716720in}}%
\pgfpathlineto{\pgfqpoint{3.620778in}{1.712462in}}%
\pgfpathmoveto{\pgfqpoint{3.616520in}{1.716720in}}%
\pgfpathlineto{\pgfqpoint{3.616520in}{1.716720in}}%
\pgfpathlineto{\pgfqpoint{3.616520in}{1.720978in}}%
\pgfpathlineto{\pgfqpoint{3.620778in}{1.720978in}}%
\pgfpathlineto{\pgfqpoint{3.620778in}{1.716720in}}%
\pgfpathmoveto{\pgfqpoint{3.616520in}{1.720978in}}%
\pgfpathlineto{\pgfqpoint{3.616520in}{1.720978in}}%
\pgfpathlineto{\pgfqpoint{3.616520in}{1.725235in}}%
\pgfpathlineto{\pgfqpoint{3.620778in}{1.725235in}}%
\pgfpathlineto{\pgfqpoint{3.620778in}{1.720978in}}%
\pgfpathmoveto{\pgfqpoint{3.616520in}{1.725235in}}%
\pgfpathlineto{\pgfqpoint{3.616520in}{1.725235in}}%
\pgfpathlineto{\pgfqpoint{3.616520in}{1.729493in}}%
\pgfpathlineto{\pgfqpoint{3.620778in}{1.729493in}}%
\pgfpathlineto{\pgfqpoint{3.620778in}{1.725235in}}%
\pgfpathmoveto{\pgfqpoint{3.620778in}{1.699689in}}%
\pgfpathlineto{\pgfqpoint{3.620778in}{1.699689in}}%
\pgfpathlineto{\pgfqpoint{3.620778in}{1.703947in}}%
\pgfpathlineto{\pgfqpoint{3.625036in}{1.703947in}}%
\pgfpathlineto{\pgfqpoint{3.625036in}{1.699689in}}%
\pgfpathmoveto{\pgfqpoint{3.620778in}{1.703947in}}%
\pgfpathlineto{\pgfqpoint{3.620778in}{1.703947in}}%
\pgfpathlineto{\pgfqpoint{3.620778in}{1.708205in}}%
\pgfpathlineto{\pgfqpoint{3.625036in}{1.708205in}}%
\pgfpathlineto{\pgfqpoint{3.625036in}{1.703947in}}%
\pgfpathmoveto{\pgfqpoint{3.625036in}{1.703947in}}%
\pgfpathlineto{\pgfqpoint{3.625036in}{1.703947in}}%
\pgfpathlineto{\pgfqpoint{3.625036in}{1.708205in}}%
\pgfpathlineto{\pgfqpoint{3.629293in}{1.708205in}}%
\pgfpathlineto{\pgfqpoint{3.629293in}{1.703947in}}%
\pgfpathmoveto{\pgfqpoint{3.620778in}{1.708205in}}%
\pgfpathlineto{\pgfqpoint{3.620778in}{1.708205in}}%
\pgfpathlineto{\pgfqpoint{3.620778in}{1.712462in}}%
\pgfpathlineto{\pgfqpoint{3.625036in}{1.712462in}}%
\pgfpathlineto{\pgfqpoint{3.625036in}{1.708205in}}%
\pgfpathmoveto{\pgfqpoint{3.620778in}{1.712462in}}%
\pgfpathlineto{\pgfqpoint{3.620778in}{1.712462in}}%
\pgfpathlineto{\pgfqpoint{3.620778in}{1.716720in}}%
\pgfpathlineto{\pgfqpoint{3.625036in}{1.716720in}}%
\pgfpathlineto{\pgfqpoint{3.625036in}{1.712462in}}%
\pgfpathmoveto{\pgfqpoint{3.625036in}{1.708205in}}%
\pgfpathlineto{\pgfqpoint{3.625036in}{1.708205in}}%
\pgfpathlineto{\pgfqpoint{3.625036in}{1.712462in}}%
\pgfpathlineto{\pgfqpoint{3.629293in}{1.712462in}}%
\pgfpathlineto{\pgfqpoint{3.629293in}{1.708205in}}%
\pgfpathmoveto{\pgfqpoint{3.625036in}{1.712462in}}%
\pgfpathlineto{\pgfqpoint{3.625036in}{1.712462in}}%
\pgfpathlineto{\pgfqpoint{3.625036in}{1.716720in}}%
\pgfpathlineto{\pgfqpoint{3.629293in}{1.716720in}}%
\pgfpathlineto{\pgfqpoint{3.629293in}{1.712462in}}%
\pgfpathmoveto{\pgfqpoint{3.620778in}{1.716720in}}%
\pgfpathlineto{\pgfqpoint{3.620778in}{1.716720in}}%
\pgfpathlineto{\pgfqpoint{3.620778in}{1.720978in}}%
\pgfpathlineto{\pgfqpoint{3.625036in}{1.720978in}}%
\pgfpathlineto{\pgfqpoint{3.625036in}{1.716720in}}%
\pgfpathmoveto{\pgfqpoint{3.620778in}{1.720978in}}%
\pgfpathlineto{\pgfqpoint{3.620778in}{1.720978in}}%
\pgfpathlineto{\pgfqpoint{3.620778in}{1.725235in}}%
\pgfpathlineto{\pgfqpoint{3.625036in}{1.725235in}}%
\pgfpathlineto{\pgfqpoint{3.625036in}{1.720978in}}%
\pgfpathmoveto{\pgfqpoint{3.625036in}{1.716720in}}%
\pgfpathlineto{\pgfqpoint{3.625036in}{1.716720in}}%
\pgfpathlineto{\pgfqpoint{3.625036in}{1.720978in}}%
\pgfpathlineto{\pgfqpoint{3.629293in}{1.720978in}}%
\pgfpathlineto{\pgfqpoint{3.629293in}{1.716720in}}%
\pgfpathmoveto{\pgfqpoint{3.625036in}{1.720978in}}%
\pgfpathlineto{\pgfqpoint{3.625036in}{1.720978in}}%
\pgfpathlineto{\pgfqpoint{3.625036in}{1.725235in}}%
\pgfpathlineto{\pgfqpoint{3.629293in}{1.725235in}}%
\pgfpathlineto{\pgfqpoint{3.629293in}{1.720978in}}%
\pgfpathmoveto{\pgfqpoint{3.620778in}{1.725235in}}%
\pgfpathlineto{\pgfqpoint{3.620778in}{1.725235in}}%
\pgfpathlineto{\pgfqpoint{3.620778in}{1.729493in}}%
\pgfpathlineto{\pgfqpoint{3.625036in}{1.729493in}}%
\pgfpathlineto{\pgfqpoint{3.625036in}{1.725235in}}%
\pgfpathmoveto{\pgfqpoint{3.620778in}{1.729493in}}%
\pgfpathlineto{\pgfqpoint{3.620778in}{1.729493in}}%
\pgfpathlineto{\pgfqpoint{3.620778in}{1.733751in}}%
\pgfpathlineto{\pgfqpoint{3.625036in}{1.733751in}}%
\pgfpathlineto{\pgfqpoint{3.625036in}{1.729493in}}%
\pgfpathmoveto{\pgfqpoint{3.625036in}{1.725235in}}%
\pgfpathlineto{\pgfqpoint{3.625036in}{1.725235in}}%
\pgfpathlineto{\pgfqpoint{3.625036in}{1.729493in}}%
\pgfpathlineto{\pgfqpoint{3.629293in}{1.729493in}}%
\pgfpathlineto{\pgfqpoint{3.629293in}{1.725235in}}%
\pgfpathmoveto{\pgfqpoint{3.625036in}{1.729493in}}%
\pgfpathlineto{\pgfqpoint{3.625036in}{1.729493in}}%
\pgfpathlineto{\pgfqpoint{3.625036in}{1.733751in}}%
\pgfpathlineto{\pgfqpoint{3.629293in}{1.733751in}}%
\pgfpathlineto{\pgfqpoint{3.629293in}{1.729493in}}%
\pgfpathmoveto{\pgfqpoint{3.629293in}{1.720978in}}%
\pgfpathlineto{\pgfqpoint{3.629293in}{1.720978in}}%
\pgfpathlineto{\pgfqpoint{3.629293in}{1.725235in}}%
\pgfpathlineto{\pgfqpoint{3.633551in}{1.725235in}}%
\pgfpathlineto{\pgfqpoint{3.633551in}{1.720978in}}%
\pgfpathmoveto{\pgfqpoint{3.629293in}{1.725235in}}%
\pgfpathlineto{\pgfqpoint{3.629293in}{1.725235in}}%
\pgfpathlineto{\pgfqpoint{3.629293in}{1.729493in}}%
\pgfpathlineto{\pgfqpoint{3.633551in}{1.729493in}}%
\pgfpathlineto{\pgfqpoint{3.633551in}{1.725235in}}%
\pgfpathmoveto{\pgfqpoint{3.629293in}{1.729493in}}%
\pgfpathlineto{\pgfqpoint{3.629293in}{1.729493in}}%
\pgfpathlineto{\pgfqpoint{3.629293in}{1.733751in}}%
\pgfpathlineto{\pgfqpoint{3.633551in}{1.733751in}}%
\pgfpathlineto{\pgfqpoint{3.633551in}{1.729493in}}%
\pgfpathmoveto{\pgfqpoint{3.620778in}{1.733751in}}%
\pgfpathlineto{\pgfqpoint{3.620778in}{1.733751in}}%
\pgfpathlineto{\pgfqpoint{3.620778in}{1.738009in}}%
\pgfpathlineto{\pgfqpoint{3.625036in}{1.738009in}}%
\pgfpathlineto{\pgfqpoint{3.625036in}{1.733751in}}%
\pgfpathmoveto{\pgfqpoint{3.620778in}{1.738009in}}%
\pgfpathlineto{\pgfqpoint{3.620778in}{1.738009in}}%
\pgfpathlineto{\pgfqpoint{3.620778in}{1.742266in}}%
\pgfpathlineto{\pgfqpoint{3.625036in}{1.742266in}}%
\pgfpathlineto{\pgfqpoint{3.625036in}{1.738009in}}%
\pgfpathmoveto{\pgfqpoint{3.625036in}{1.733751in}}%
\pgfpathlineto{\pgfqpoint{3.625036in}{1.733751in}}%
\pgfpathlineto{\pgfqpoint{3.625036in}{1.738009in}}%
\pgfpathlineto{\pgfqpoint{3.629293in}{1.738009in}}%
\pgfpathlineto{\pgfqpoint{3.629293in}{1.733751in}}%
\pgfpathmoveto{\pgfqpoint{3.625036in}{1.738009in}}%
\pgfpathlineto{\pgfqpoint{3.625036in}{1.738009in}}%
\pgfpathlineto{\pgfqpoint{3.625036in}{1.742266in}}%
\pgfpathlineto{\pgfqpoint{3.629293in}{1.742266in}}%
\pgfpathlineto{\pgfqpoint{3.629293in}{1.738009in}}%
\pgfpathmoveto{\pgfqpoint{3.625036in}{1.742266in}}%
\pgfpathlineto{\pgfqpoint{3.625036in}{1.742266in}}%
\pgfpathlineto{\pgfqpoint{3.625036in}{1.746524in}}%
\pgfpathlineto{\pgfqpoint{3.629293in}{1.746524in}}%
\pgfpathlineto{\pgfqpoint{3.629293in}{1.742266in}}%
\pgfpathmoveto{\pgfqpoint{3.625036in}{1.746524in}}%
\pgfpathlineto{\pgfqpoint{3.625036in}{1.746524in}}%
\pgfpathlineto{\pgfqpoint{3.625036in}{1.750782in}}%
\pgfpathlineto{\pgfqpoint{3.629293in}{1.750782in}}%
\pgfpathlineto{\pgfqpoint{3.629293in}{1.746524in}}%
\pgfpathmoveto{\pgfqpoint{3.629293in}{1.733751in}}%
\pgfpathlineto{\pgfqpoint{3.629293in}{1.733751in}}%
\pgfpathlineto{\pgfqpoint{3.629293in}{1.738009in}}%
\pgfpathlineto{\pgfqpoint{3.633551in}{1.738009in}}%
\pgfpathlineto{\pgfqpoint{3.633551in}{1.733751in}}%
\pgfpathmoveto{\pgfqpoint{3.629293in}{1.738009in}}%
\pgfpathlineto{\pgfqpoint{3.629293in}{1.738009in}}%
\pgfpathlineto{\pgfqpoint{3.629293in}{1.742266in}}%
\pgfpathlineto{\pgfqpoint{3.633551in}{1.742266in}}%
\pgfpathlineto{\pgfqpoint{3.633551in}{1.738009in}}%
\pgfpathmoveto{\pgfqpoint{3.633551in}{1.733751in}}%
\pgfpathlineto{\pgfqpoint{3.633551in}{1.733751in}}%
\pgfpathlineto{\pgfqpoint{3.633551in}{1.738009in}}%
\pgfpathlineto{\pgfqpoint{3.637809in}{1.738009in}}%
\pgfpathlineto{\pgfqpoint{3.637809in}{1.733751in}}%
\pgfpathmoveto{\pgfqpoint{3.633551in}{1.738009in}}%
\pgfpathlineto{\pgfqpoint{3.633551in}{1.738009in}}%
\pgfpathlineto{\pgfqpoint{3.633551in}{1.742266in}}%
\pgfpathlineto{\pgfqpoint{3.637809in}{1.742266in}}%
\pgfpathlineto{\pgfqpoint{3.637809in}{1.738009in}}%
\pgfpathmoveto{\pgfqpoint{3.629293in}{1.742266in}}%
\pgfpathlineto{\pgfqpoint{3.629293in}{1.742266in}}%
\pgfpathlineto{\pgfqpoint{3.629293in}{1.746524in}}%
\pgfpathlineto{\pgfqpoint{3.633551in}{1.746524in}}%
\pgfpathlineto{\pgfqpoint{3.633551in}{1.742266in}}%
\pgfpathmoveto{\pgfqpoint{3.629293in}{1.746524in}}%
\pgfpathlineto{\pgfqpoint{3.629293in}{1.746524in}}%
\pgfpathlineto{\pgfqpoint{3.629293in}{1.750782in}}%
\pgfpathlineto{\pgfqpoint{3.633551in}{1.750782in}}%
\pgfpathlineto{\pgfqpoint{3.633551in}{1.746524in}}%
\pgfpathmoveto{\pgfqpoint{3.633551in}{1.742266in}}%
\pgfpathlineto{\pgfqpoint{3.633551in}{1.742266in}}%
\pgfpathlineto{\pgfqpoint{3.633551in}{1.746524in}}%
\pgfpathlineto{\pgfqpoint{3.637809in}{1.746524in}}%
\pgfpathlineto{\pgfqpoint{3.637809in}{1.742266in}}%
\pgfpathmoveto{\pgfqpoint{3.633551in}{1.746524in}}%
\pgfpathlineto{\pgfqpoint{3.633551in}{1.746524in}}%
\pgfpathlineto{\pgfqpoint{3.633551in}{1.750782in}}%
\pgfpathlineto{\pgfqpoint{3.637809in}{1.750782in}}%
\pgfpathlineto{\pgfqpoint{3.637809in}{1.746524in}}%
\pgfpathmoveto{\pgfqpoint{3.625036in}{1.750782in}}%
\pgfpathlineto{\pgfqpoint{3.625036in}{1.750782in}}%
\pgfpathlineto{\pgfqpoint{3.625036in}{1.755039in}}%
\pgfpathlineto{\pgfqpoint{3.629293in}{1.755039in}}%
\pgfpathlineto{\pgfqpoint{3.629293in}{1.750782in}}%
\pgfpathmoveto{\pgfqpoint{3.625036in}{1.755039in}}%
\pgfpathlineto{\pgfqpoint{3.625036in}{1.755039in}}%
\pgfpathlineto{\pgfqpoint{3.625036in}{1.759297in}}%
\pgfpathlineto{\pgfqpoint{3.629293in}{1.759297in}}%
\pgfpathlineto{\pgfqpoint{3.629293in}{1.755039in}}%
\pgfpathmoveto{\pgfqpoint{3.629293in}{1.750782in}}%
\pgfpathlineto{\pgfqpoint{3.629293in}{1.750782in}}%
\pgfpathlineto{\pgfqpoint{3.629293in}{1.755039in}}%
\pgfpathlineto{\pgfqpoint{3.633551in}{1.755039in}}%
\pgfpathlineto{\pgfqpoint{3.633551in}{1.750782in}}%
\pgfpathmoveto{\pgfqpoint{3.629293in}{1.755039in}}%
\pgfpathlineto{\pgfqpoint{3.629293in}{1.755039in}}%
\pgfpathlineto{\pgfqpoint{3.629293in}{1.759297in}}%
\pgfpathlineto{\pgfqpoint{3.633551in}{1.759297in}}%
\pgfpathlineto{\pgfqpoint{3.633551in}{1.755039in}}%
\pgfpathmoveto{\pgfqpoint{3.633551in}{1.750782in}}%
\pgfpathlineto{\pgfqpoint{3.633551in}{1.750782in}}%
\pgfpathlineto{\pgfqpoint{3.633551in}{1.755039in}}%
\pgfpathlineto{\pgfqpoint{3.637809in}{1.755039in}}%
\pgfpathlineto{\pgfqpoint{3.637809in}{1.750782in}}%
\pgfpathmoveto{\pgfqpoint{3.633551in}{1.755039in}}%
\pgfpathlineto{\pgfqpoint{3.633551in}{1.755039in}}%
\pgfpathlineto{\pgfqpoint{3.633551in}{1.759297in}}%
\pgfpathlineto{\pgfqpoint{3.637809in}{1.759297in}}%
\pgfpathlineto{\pgfqpoint{3.637809in}{1.755039in}}%
\pgfpathmoveto{\pgfqpoint{3.629293in}{1.759297in}}%
\pgfpathlineto{\pgfqpoint{3.629293in}{1.759297in}}%
\pgfpathlineto{\pgfqpoint{3.629293in}{1.763555in}}%
\pgfpathlineto{\pgfqpoint{3.633551in}{1.763555in}}%
\pgfpathlineto{\pgfqpoint{3.633551in}{1.759297in}}%
\pgfpathmoveto{\pgfqpoint{3.629293in}{1.763555in}}%
\pgfpathlineto{\pgfqpoint{3.629293in}{1.763555in}}%
\pgfpathlineto{\pgfqpoint{3.629293in}{1.767813in}}%
\pgfpathlineto{\pgfqpoint{3.633551in}{1.767813in}}%
\pgfpathlineto{\pgfqpoint{3.633551in}{1.763555in}}%
\pgfpathmoveto{\pgfqpoint{3.633551in}{1.759297in}}%
\pgfpathlineto{\pgfqpoint{3.633551in}{1.759297in}}%
\pgfpathlineto{\pgfqpoint{3.633551in}{1.763555in}}%
\pgfpathlineto{\pgfqpoint{3.637809in}{1.763555in}}%
\pgfpathlineto{\pgfqpoint{3.637809in}{1.759297in}}%
\pgfpathmoveto{\pgfqpoint{3.633551in}{1.763555in}}%
\pgfpathlineto{\pgfqpoint{3.633551in}{1.763555in}}%
\pgfpathlineto{\pgfqpoint{3.633551in}{1.767813in}}%
\pgfpathlineto{\pgfqpoint{3.637809in}{1.767813in}}%
\pgfpathlineto{\pgfqpoint{3.637809in}{1.763555in}}%
\pgfpathmoveto{\pgfqpoint{3.629293in}{1.767813in}}%
\pgfpathlineto{\pgfqpoint{3.629293in}{1.767813in}}%
\pgfpathlineto{\pgfqpoint{3.629293in}{1.772070in}}%
\pgfpathlineto{\pgfqpoint{3.633551in}{1.772070in}}%
\pgfpathlineto{\pgfqpoint{3.633551in}{1.767813in}}%
\pgfpathmoveto{\pgfqpoint{3.633551in}{1.767813in}}%
\pgfpathlineto{\pgfqpoint{3.633551in}{1.767813in}}%
\pgfpathlineto{\pgfqpoint{3.633551in}{1.772070in}}%
\pgfpathlineto{\pgfqpoint{3.637809in}{1.772070in}}%
\pgfpathlineto{\pgfqpoint{3.637809in}{1.767813in}}%
\pgfpathmoveto{\pgfqpoint{3.633551in}{1.772070in}}%
\pgfpathlineto{\pgfqpoint{3.633551in}{1.772070in}}%
\pgfpathlineto{\pgfqpoint{3.633551in}{1.776328in}}%
\pgfpathlineto{\pgfqpoint{3.637809in}{1.776328in}}%
\pgfpathlineto{\pgfqpoint{3.637809in}{1.772070in}}%
\pgfpathmoveto{\pgfqpoint{3.633551in}{1.776328in}}%
\pgfpathlineto{\pgfqpoint{3.633551in}{1.776328in}}%
\pgfpathlineto{\pgfqpoint{3.633551in}{1.780586in}}%
\pgfpathlineto{\pgfqpoint{3.637809in}{1.780586in}}%
\pgfpathlineto{\pgfqpoint{3.637809in}{1.776328in}}%
\pgfpathmoveto{\pgfqpoint{3.633551in}{1.780586in}}%
\pgfpathlineto{\pgfqpoint{3.633551in}{1.780586in}}%
\pgfpathlineto{\pgfqpoint{3.633551in}{1.784843in}}%
\pgfpathlineto{\pgfqpoint{3.637809in}{1.784843in}}%
\pgfpathlineto{\pgfqpoint{3.637809in}{1.780586in}}%
\pgfpathmoveto{\pgfqpoint{3.637809in}{1.746524in}}%
\pgfpathlineto{\pgfqpoint{3.637809in}{1.746524in}}%
\pgfpathlineto{\pgfqpoint{3.637809in}{1.750782in}}%
\pgfpathlineto{\pgfqpoint{3.642067in}{1.750782in}}%
\pgfpathlineto{\pgfqpoint{3.642067in}{1.746524in}}%
\pgfpathmoveto{\pgfqpoint{3.637809in}{1.750782in}}%
\pgfpathlineto{\pgfqpoint{3.637809in}{1.750782in}}%
\pgfpathlineto{\pgfqpoint{3.637809in}{1.755039in}}%
\pgfpathlineto{\pgfqpoint{3.642067in}{1.755039in}}%
\pgfpathlineto{\pgfqpoint{3.642067in}{1.750782in}}%
\pgfpathmoveto{\pgfqpoint{3.637809in}{1.755039in}}%
\pgfpathlineto{\pgfqpoint{3.637809in}{1.755039in}}%
\pgfpathlineto{\pgfqpoint{3.637809in}{1.759297in}}%
\pgfpathlineto{\pgfqpoint{3.642067in}{1.759297in}}%
\pgfpathlineto{\pgfqpoint{3.642067in}{1.755039in}}%
\pgfpathmoveto{\pgfqpoint{3.637809in}{1.759297in}}%
\pgfpathlineto{\pgfqpoint{3.637809in}{1.759297in}}%
\pgfpathlineto{\pgfqpoint{3.637809in}{1.763555in}}%
\pgfpathlineto{\pgfqpoint{3.642067in}{1.763555in}}%
\pgfpathlineto{\pgfqpoint{3.642067in}{1.759297in}}%
\pgfpathmoveto{\pgfqpoint{3.637809in}{1.763555in}}%
\pgfpathlineto{\pgfqpoint{3.637809in}{1.763555in}}%
\pgfpathlineto{\pgfqpoint{3.637809in}{1.767813in}}%
\pgfpathlineto{\pgfqpoint{3.642067in}{1.767813in}}%
\pgfpathlineto{\pgfqpoint{3.642067in}{1.763555in}}%
\pgfpathmoveto{\pgfqpoint{3.642067in}{1.763555in}}%
\pgfpathlineto{\pgfqpoint{3.642067in}{1.763555in}}%
\pgfpathlineto{\pgfqpoint{3.642067in}{1.767813in}}%
\pgfpathlineto{\pgfqpoint{3.646325in}{1.767813in}}%
\pgfpathlineto{\pgfqpoint{3.646325in}{1.763555in}}%
\pgfpathmoveto{\pgfqpoint{3.637809in}{1.767813in}}%
\pgfpathlineto{\pgfqpoint{3.637809in}{1.767813in}}%
\pgfpathlineto{\pgfqpoint{3.637809in}{1.772070in}}%
\pgfpathlineto{\pgfqpoint{3.642067in}{1.772070in}}%
\pgfpathlineto{\pgfqpoint{3.642067in}{1.767813in}}%
\pgfpathmoveto{\pgfqpoint{3.637809in}{1.772070in}}%
\pgfpathlineto{\pgfqpoint{3.637809in}{1.772070in}}%
\pgfpathlineto{\pgfqpoint{3.637809in}{1.776328in}}%
\pgfpathlineto{\pgfqpoint{3.642067in}{1.776328in}}%
\pgfpathlineto{\pgfqpoint{3.642067in}{1.772070in}}%
\pgfpathmoveto{\pgfqpoint{3.642067in}{1.767813in}}%
\pgfpathlineto{\pgfqpoint{3.642067in}{1.767813in}}%
\pgfpathlineto{\pgfqpoint{3.642067in}{1.772070in}}%
\pgfpathlineto{\pgfqpoint{3.646325in}{1.772070in}}%
\pgfpathlineto{\pgfqpoint{3.646325in}{1.767813in}}%
\pgfpathmoveto{\pgfqpoint{3.642067in}{1.772070in}}%
\pgfpathlineto{\pgfqpoint{3.642067in}{1.772070in}}%
\pgfpathlineto{\pgfqpoint{3.642067in}{1.776328in}}%
\pgfpathlineto{\pgfqpoint{3.646325in}{1.776328in}}%
\pgfpathlineto{\pgfqpoint{3.646325in}{1.772070in}}%
\pgfpathmoveto{\pgfqpoint{3.637809in}{1.776328in}}%
\pgfpathlineto{\pgfqpoint{3.637809in}{1.776328in}}%
\pgfpathlineto{\pgfqpoint{3.637809in}{1.780586in}}%
\pgfpathlineto{\pgfqpoint{3.642067in}{1.780586in}}%
\pgfpathlineto{\pgfqpoint{3.642067in}{1.776328in}}%
\pgfpathmoveto{\pgfqpoint{3.637809in}{1.780586in}}%
\pgfpathlineto{\pgfqpoint{3.637809in}{1.780586in}}%
\pgfpathlineto{\pgfqpoint{3.637809in}{1.784843in}}%
\pgfpathlineto{\pgfqpoint{3.642067in}{1.784843in}}%
\pgfpathlineto{\pgfqpoint{3.642067in}{1.780586in}}%
\pgfpathmoveto{\pgfqpoint{3.642067in}{1.776328in}}%
\pgfpathlineto{\pgfqpoint{3.642067in}{1.776328in}}%
\pgfpathlineto{\pgfqpoint{3.642067in}{1.780586in}}%
\pgfpathlineto{\pgfqpoint{3.646325in}{1.780586in}}%
\pgfpathlineto{\pgfqpoint{3.646325in}{1.776328in}}%
\pgfpathmoveto{\pgfqpoint{3.642067in}{1.780586in}}%
\pgfpathlineto{\pgfqpoint{3.642067in}{1.780586in}}%
\pgfpathlineto{\pgfqpoint{3.642067in}{1.784843in}}%
\pgfpathlineto{\pgfqpoint{3.646325in}{1.784843in}}%
\pgfpathlineto{\pgfqpoint{3.646325in}{1.780586in}}%
\pgfpathmoveto{\pgfqpoint{3.646325in}{1.780586in}}%
\pgfpathlineto{\pgfqpoint{3.646325in}{1.780586in}}%
\pgfpathlineto{\pgfqpoint{3.646325in}{1.784843in}}%
\pgfpathlineto{\pgfqpoint{3.650583in}{1.784843in}}%
\pgfpathlineto{\pgfqpoint{3.650583in}{1.780586in}}%
\pgfpathmoveto{\pgfqpoint{3.637809in}{1.784843in}}%
\pgfpathlineto{\pgfqpoint{3.637809in}{1.784843in}}%
\pgfpathlineto{\pgfqpoint{3.637809in}{1.789101in}}%
\pgfpathlineto{\pgfqpoint{3.642067in}{1.789101in}}%
\pgfpathlineto{\pgfqpoint{3.642067in}{1.784843in}}%
\pgfpathmoveto{\pgfqpoint{3.637809in}{1.789101in}}%
\pgfpathlineto{\pgfqpoint{3.637809in}{1.789101in}}%
\pgfpathlineto{\pgfqpoint{3.637809in}{1.793359in}}%
\pgfpathlineto{\pgfqpoint{3.642067in}{1.793359in}}%
\pgfpathlineto{\pgfqpoint{3.642067in}{1.789101in}}%
\pgfpathmoveto{\pgfqpoint{3.642067in}{1.784843in}}%
\pgfpathlineto{\pgfqpoint{3.642067in}{1.784843in}}%
\pgfpathlineto{\pgfqpoint{3.642067in}{1.789101in}}%
\pgfpathlineto{\pgfqpoint{3.646325in}{1.789101in}}%
\pgfpathlineto{\pgfqpoint{3.646325in}{1.784843in}}%
\pgfpathmoveto{\pgfqpoint{3.642067in}{1.789101in}}%
\pgfpathlineto{\pgfqpoint{3.642067in}{1.789101in}}%
\pgfpathlineto{\pgfqpoint{3.642067in}{1.793359in}}%
\pgfpathlineto{\pgfqpoint{3.646325in}{1.793359in}}%
\pgfpathlineto{\pgfqpoint{3.646325in}{1.789101in}}%
\pgfpathmoveto{\pgfqpoint{3.637809in}{1.793359in}}%
\pgfpathlineto{\pgfqpoint{3.637809in}{1.793359in}}%
\pgfpathlineto{\pgfqpoint{3.637809in}{1.797617in}}%
\pgfpathlineto{\pgfqpoint{3.642067in}{1.797617in}}%
\pgfpathlineto{\pgfqpoint{3.642067in}{1.793359in}}%
\pgfpathmoveto{\pgfqpoint{3.637809in}{1.797617in}}%
\pgfpathlineto{\pgfqpoint{3.637809in}{1.797617in}}%
\pgfpathlineto{\pgfqpoint{3.637809in}{1.801874in}}%
\pgfpathlineto{\pgfqpoint{3.642067in}{1.801874in}}%
\pgfpathlineto{\pgfqpoint{3.642067in}{1.797617in}}%
\pgfpathmoveto{\pgfqpoint{3.642067in}{1.793359in}}%
\pgfpathlineto{\pgfqpoint{3.642067in}{1.793359in}}%
\pgfpathlineto{\pgfqpoint{3.642067in}{1.797617in}}%
\pgfpathlineto{\pgfqpoint{3.646325in}{1.797617in}}%
\pgfpathlineto{\pgfqpoint{3.646325in}{1.793359in}}%
\pgfpathmoveto{\pgfqpoint{3.642067in}{1.797617in}}%
\pgfpathlineto{\pgfqpoint{3.642067in}{1.797617in}}%
\pgfpathlineto{\pgfqpoint{3.642067in}{1.801874in}}%
\pgfpathlineto{\pgfqpoint{3.646325in}{1.801874in}}%
\pgfpathlineto{\pgfqpoint{3.646325in}{1.797617in}}%
\pgfpathmoveto{\pgfqpoint{3.646325in}{1.784843in}}%
\pgfpathlineto{\pgfqpoint{3.646325in}{1.784843in}}%
\pgfpathlineto{\pgfqpoint{3.646325in}{1.789101in}}%
\pgfpathlineto{\pgfqpoint{3.650583in}{1.789101in}}%
\pgfpathlineto{\pgfqpoint{3.650583in}{1.784843in}}%
\pgfpathmoveto{\pgfqpoint{3.646325in}{1.789101in}}%
\pgfpathlineto{\pgfqpoint{3.646325in}{1.789101in}}%
\pgfpathlineto{\pgfqpoint{3.646325in}{1.793359in}}%
\pgfpathlineto{\pgfqpoint{3.650583in}{1.793359in}}%
\pgfpathlineto{\pgfqpoint{3.650583in}{1.789101in}}%
\pgfpathmoveto{\pgfqpoint{3.646325in}{1.793359in}}%
\pgfpathlineto{\pgfqpoint{3.646325in}{1.793359in}}%
\pgfpathlineto{\pgfqpoint{3.646325in}{1.797617in}}%
\pgfpathlineto{\pgfqpoint{3.650583in}{1.797617in}}%
\pgfpathlineto{\pgfqpoint{3.650583in}{1.793359in}}%
\pgfpathmoveto{\pgfqpoint{3.646325in}{1.797617in}}%
\pgfpathlineto{\pgfqpoint{3.646325in}{1.797617in}}%
\pgfpathlineto{\pgfqpoint{3.646325in}{1.801874in}}%
\pgfpathlineto{\pgfqpoint{3.650583in}{1.801874in}}%
\pgfpathlineto{\pgfqpoint{3.650583in}{1.797617in}}%
\pgfpathmoveto{\pgfqpoint{3.650583in}{1.793359in}}%
\pgfpathlineto{\pgfqpoint{3.650583in}{1.793359in}}%
\pgfpathlineto{\pgfqpoint{3.650583in}{1.797617in}}%
\pgfpathlineto{\pgfqpoint{3.654841in}{1.797617in}}%
\pgfpathlineto{\pgfqpoint{3.654841in}{1.793359in}}%
\pgfpathmoveto{\pgfqpoint{3.650583in}{1.797617in}}%
\pgfpathlineto{\pgfqpoint{3.650583in}{1.797617in}}%
\pgfpathlineto{\pgfqpoint{3.650583in}{1.801874in}}%
\pgfpathlineto{\pgfqpoint{3.654841in}{1.801874in}}%
\pgfpathlineto{\pgfqpoint{3.654841in}{1.797617in}}%
\pgfpathmoveto{\pgfqpoint{3.642067in}{1.801874in}}%
\pgfpathlineto{\pgfqpoint{3.642067in}{1.801874in}}%
\pgfpathlineto{\pgfqpoint{3.642067in}{1.806132in}}%
\pgfpathlineto{\pgfqpoint{3.646325in}{1.806132in}}%
\pgfpathlineto{\pgfqpoint{3.646325in}{1.801874in}}%
\pgfpathmoveto{\pgfqpoint{3.642067in}{1.806132in}}%
\pgfpathlineto{\pgfqpoint{3.642067in}{1.806132in}}%
\pgfpathlineto{\pgfqpoint{3.642067in}{1.810390in}}%
\pgfpathlineto{\pgfqpoint{3.646325in}{1.810390in}}%
\pgfpathlineto{\pgfqpoint{3.646325in}{1.806132in}}%
\pgfpathmoveto{\pgfqpoint{3.642067in}{1.810390in}}%
\pgfpathlineto{\pgfqpoint{3.642067in}{1.810390in}}%
\pgfpathlineto{\pgfqpoint{3.642067in}{1.814647in}}%
\pgfpathlineto{\pgfqpoint{3.646325in}{1.814647in}}%
\pgfpathlineto{\pgfqpoint{3.646325in}{1.810390in}}%
\pgfpathmoveto{\pgfqpoint{3.642067in}{1.814647in}}%
\pgfpathlineto{\pgfqpoint{3.642067in}{1.814647in}}%
\pgfpathlineto{\pgfqpoint{3.642067in}{1.818905in}}%
\pgfpathlineto{\pgfqpoint{3.646325in}{1.818905in}}%
\pgfpathlineto{\pgfqpoint{3.646325in}{1.814647in}}%
\pgfpathmoveto{\pgfqpoint{3.646325in}{1.801874in}}%
\pgfpathlineto{\pgfqpoint{3.646325in}{1.801874in}}%
\pgfpathlineto{\pgfqpoint{3.646325in}{1.806132in}}%
\pgfpathlineto{\pgfqpoint{3.650583in}{1.806132in}}%
\pgfpathlineto{\pgfqpoint{3.650583in}{1.801874in}}%
\pgfpathmoveto{\pgfqpoint{3.646325in}{1.806132in}}%
\pgfpathlineto{\pgfqpoint{3.646325in}{1.806132in}}%
\pgfpathlineto{\pgfqpoint{3.646325in}{1.810390in}}%
\pgfpathlineto{\pgfqpoint{3.650583in}{1.810390in}}%
\pgfpathlineto{\pgfqpoint{3.650583in}{1.806132in}}%
\pgfpathmoveto{\pgfqpoint{3.650583in}{1.801874in}}%
\pgfpathlineto{\pgfqpoint{3.650583in}{1.801874in}}%
\pgfpathlineto{\pgfqpoint{3.650583in}{1.806132in}}%
\pgfpathlineto{\pgfqpoint{3.654841in}{1.806132in}}%
\pgfpathlineto{\pgfqpoint{3.654841in}{1.801874in}}%
\pgfpathmoveto{\pgfqpoint{3.650583in}{1.806132in}}%
\pgfpathlineto{\pgfqpoint{3.650583in}{1.806132in}}%
\pgfpathlineto{\pgfqpoint{3.650583in}{1.810390in}}%
\pgfpathlineto{\pgfqpoint{3.654841in}{1.810390in}}%
\pgfpathlineto{\pgfqpoint{3.654841in}{1.806132in}}%
\pgfpathmoveto{\pgfqpoint{3.646325in}{1.810390in}}%
\pgfpathlineto{\pgfqpoint{3.646325in}{1.810390in}}%
\pgfpathlineto{\pgfqpoint{3.646325in}{1.814647in}}%
\pgfpathlineto{\pgfqpoint{3.650583in}{1.814647in}}%
\pgfpathlineto{\pgfqpoint{3.650583in}{1.810390in}}%
\pgfpathmoveto{\pgfqpoint{3.646325in}{1.814647in}}%
\pgfpathlineto{\pgfqpoint{3.646325in}{1.814647in}}%
\pgfpathlineto{\pgfqpoint{3.646325in}{1.818905in}}%
\pgfpathlineto{\pgfqpoint{3.650583in}{1.818905in}}%
\pgfpathlineto{\pgfqpoint{3.650583in}{1.814647in}}%
\pgfpathmoveto{\pgfqpoint{3.650583in}{1.810390in}}%
\pgfpathlineto{\pgfqpoint{3.650583in}{1.810390in}}%
\pgfpathlineto{\pgfqpoint{3.650583in}{1.814647in}}%
\pgfpathlineto{\pgfqpoint{3.654841in}{1.814647in}}%
\pgfpathlineto{\pgfqpoint{3.654841in}{1.810390in}}%
\pgfpathmoveto{\pgfqpoint{3.650583in}{1.814647in}}%
\pgfpathlineto{\pgfqpoint{3.650583in}{1.814647in}}%
\pgfpathlineto{\pgfqpoint{3.650583in}{1.818905in}}%
\pgfpathlineto{\pgfqpoint{3.654841in}{1.818905in}}%
\pgfpathlineto{\pgfqpoint{3.654841in}{1.814647in}}%
\pgfpathmoveto{\pgfqpoint{3.646325in}{1.818905in}}%
\pgfpathlineto{\pgfqpoint{3.646325in}{1.818905in}}%
\pgfpathlineto{\pgfqpoint{3.646325in}{1.823163in}}%
\pgfpathlineto{\pgfqpoint{3.650583in}{1.823163in}}%
\pgfpathlineto{\pgfqpoint{3.650583in}{1.818905in}}%
\pgfpathmoveto{\pgfqpoint{3.646325in}{1.823163in}}%
\pgfpathlineto{\pgfqpoint{3.646325in}{1.823163in}}%
\pgfpathlineto{\pgfqpoint{3.646325in}{1.827420in}}%
\pgfpathlineto{\pgfqpoint{3.650583in}{1.827420in}}%
\pgfpathlineto{\pgfqpoint{3.650583in}{1.823163in}}%
\pgfpathmoveto{\pgfqpoint{3.650583in}{1.818905in}}%
\pgfpathlineto{\pgfqpoint{3.650583in}{1.818905in}}%
\pgfpathlineto{\pgfqpoint{3.650583in}{1.823163in}}%
\pgfpathlineto{\pgfqpoint{3.654841in}{1.823163in}}%
\pgfpathlineto{\pgfqpoint{3.654841in}{1.818905in}}%
\pgfpathmoveto{\pgfqpoint{3.650583in}{1.823163in}}%
\pgfpathlineto{\pgfqpoint{3.650583in}{1.823163in}}%
\pgfpathlineto{\pgfqpoint{3.650583in}{1.827420in}}%
\pgfpathlineto{\pgfqpoint{3.654841in}{1.827420in}}%
\pgfpathlineto{\pgfqpoint{3.654841in}{1.823163in}}%
\pgfpathmoveto{\pgfqpoint{3.646325in}{1.827420in}}%
\pgfpathlineto{\pgfqpoint{3.646325in}{1.827420in}}%
\pgfpathlineto{\pgfqpoint{3.646325in}{1.831678in}}%
\pgfpathlineto{\pgfqpoint{3.650583in}{1.831678in}}%
\pgfpathlineto{\pgfqpoint{3.650583in}{1.827420in}}%
\pgfpathmoveto{\pgfqpoint{3.650583in}{1.827420in}}%
\pgfpathlineto{\pgfqpoint{3.650583in}{1.827420in}}%
\pgfpathlineto{\pgfqpoint{3.650583in}{1.831678in}}%
\pgfpathlineto{\pgfqpoint{3.654841in}{1.831678in}}%
\pgfpathlineto{\pgfqpoint{3.654841in}{1.827420in}}%
\pgfpathmoveto{\pgfqpoint{3.650583in}{1.831678in}}%
\pgfpathlineto{\pgfqpoint{3.650583in}{1.831678in}}%
\pgfpathlineto{\pgfqpoint{3.650583in}{1.835936in}}%
\pgfpathlineto{\pgfqpoint{3.654841in}{1.835936in}}%
\pgfpathlineto{\pgfqpoint{3.654841in}{1.831678in}}%
\pgfpathmoveto{\pgfqpoint{3.654841in}{1.810390in}}%
\pgfpathlineto{\pgfqpoint{3.654841in}{1.810390in}}%
\pgfpathlineto{\pgfqpoint{3.654841in}{1.814647in}}%
\pgfpathlineto{\pgfqpoint{3.659098in}{1.814647in}}%
\pgfpathlineto{\pgfqpoint{3.659098in}{1.810390in}}%
\pgfpathmoveto{\pgfqpoint{3.654841in}{1.814647in}}%
\pgfpathlineto{\pgfqpoint{3.654841in}{1.814647in}}%
\pgfpathlineto{\pgfqpoint{3.654841in}{1.818905in}}%
\pgfpathlineto{\pgfqpoint{3.659098in}{1.818905in}}%
\pgfpathlineto{\pgfqpoint{3.659098in}{1.814647in}}%
\pgfpathmoveto{\pgfqpoint{3.654841in}{1.818905in}}%
\pgfpathlineto{\pgfqpoint{3.654841in}{1.818905in}}%
\pgfpathlineto{\pgfqpoint{3.654841in}{1.823163in}}%
\pgfpathlineto{\pgfqpoint{3.659098in}{1.823163in}}%
\pgfpathlineto{\pgfqpoint{3.659098in}{1.818905in}}%
\pgfpathmoveto{\pgfqpoint{3.654841in}{1.823163in}}%
\pgfpathlineto{\pgfqpoint{3.654841in}{1.823163in}}%
\pgfpathlineto{\pgfqpoint{3.654841in}{1.827420in}}%
\pgfpathlineto{\pgfqpoint{3.659098in}{1.827420in}}%
\pgfpathlineto{\pgfqpoint{3.659098in}{1.823163in}}%
\pgfpathmoveto{\pgfqpoint{3.654841in}{1.827420in}}%
\pgfpathlineto{\pgfqpoint{3.654841in}{1.827420in}}%
\pgfpathlineto{\pgfqpoint{3.654841in}{1.831678in}}%
\pgfpathlineto{\pgfqpoint{3.659098in}{1.831678in}}%
\pgfpathlineto{\pgfqpoint{3.659098in}{1.827420in}}%
\pgfpathmoveto{\pgfqpoint{3.654841in}{1.831678in}}%
\pgfpathlineto{\pgfqpoint{3.654841in}{1.831678in}}%
\pgfpathlineto{\pgfqpoint{3.654841in}{1.835936in}}%
\pgfpathlineto{\pgfqpoint{3.659098in}{1.835936in}}%
\pgfpathlineto{\pgfqpoint{3.659098in}{1.831678in}}%
\pgfpathmoveto{\pgfqpoint{3.659098in}{1.827420in}}%
\pgfpathlineto{\pgfqpoint{3.659098in}{1.827420in}}%
\pgfpathlineto{\pgfqpoint{3.659098in}{1.831678in}}%
\pgfpathlineto{\pgfqpoint{3.663356in}{1.831678in}}%
\pgfpathlineto{\pgfqpoint{3.663356in}{1.827420in}}%
\pgfpathmoveto{\pgfqpoint{3.659098in}{1.831678in}}%
\pgfpathlineto{\pgfqpoint{3.659098in}{1.831678in}}%
\pgfpathlineto{\pgfqpoint{3.659098in}{1.835936in}}%
\pgfpathlineto{\pgfqpoint{3.663356in}{1.835936in}}%
\pgfpathlineto{\pgfqpoint{3.663356in}{1.831678in}}%
\pgfpathmoveto{\pgfqpoint{3.650583in}{1.835936in}}%
\pgfpathlineto{\pgfqpoint{3.650583in}{1.835936in}}%
\pgfpathlineto{\pgfqpoint{3.650583in}{1.840194in}}%
\pgfpathlineto{\pgfqpoint{3.654841in}{1.840194in}}%
\pgfpathlineto{\pgfqpoint{3.654841in}{1.835936in}}%
\pgfpathmoveto{\pgfqpoint{3.650583in}{1.840194in}}%
\pgfpathlineto{\pgfqpoint{3.650583in}{1.840194in}}%
\pgfpathlineto{\pgfqpoint{3.650583in}{1.844451in}}%
\pgfpathlineto{\pgfqpoint{3.654841in}{1.844451in}}%
\pgfpathlineto{\pgfqpoint{3.654841in}{1.840194in}}%
\pgfpathmoveto{\pgfqpoint{3.650583in}{1.844451in}}%
\pgfpathlineto{\pgfqpoint{3.650583in}{1.844451in}}%
\pgfpathlineto{\pgfqpoint{3.650583in}{1.848709in}}%
\pgfpathlineto{\pgfqpoint{3.654841in}{1.848709in}}%
\pgfpathlineto{\pgfqpoint{3.654841in}{1.844451in}}%
\pgfpathmoveto{\pgfqpoint{3.654841in}{1.835936in}}%
\pgfpathlineto{\pgfqpoint{3.654841in}{1.835936in}}%
\pgfpathlineto{\pgfqpoint{3.654841in}{1.840194in}}%
\pgfpathlineto{\pgfqpoint{3.659098in}{1.840194in}}%
\pgfpathlineto{\pgfqpoint{3.659098in}{1.835936in}}%
\pgfpathmoveto{\pgfqpoint{3.654841in}{1.840194in}}%
\pgfpathlineto{\pgfqpoint{3.654841in}{1.840194in}}%
\pgfpathlineto{\pgfqpoint{3.654841in}{1.844451in}}%
\pgfpathlineto{\pgfqpoint{3.659098in}{1.844451in}}%
\pgfpathlineto{\pgfqpoint{3.659098in}{1.840194in}}%
\pgfpathmoveto{\pgfqpoint{3.659098in}{1.835936in}}%
\pgfpathlineto{\pgfqpoint{3.659098in}{1.835936in}}%
\pgfpathlineto{\pgfqpoint{3.659098in}{1.840194in}}%
\pgfpathlineto{\pgfqpoint{3.663356in}{1.840194in}}%
\pgfpathlineto{\pgfqpoint{3.663356in}{1.835936in}}%
\pgfpathmoveto{\pgfqpoint{3.659098in}{1.840194in}}%
\pgfpathlineto{\pgfqpoint{3.659098in}{1.840194in}}%
\pgfpathlineto{\pgfqpoint{3.659098in}{1.844451in}}%
\pgfpathlineto{\pgfqpoint{3.663356in}{1.844451in}}%
\pgfpathlineto{\pgfqpoint{3.663356in}{1.840194in}}%
\pgfpathmoveto{\pgfqpoint{3.654841in}{1.844451in}}%
\pgfpathlineto{\pgfqpoint{3.654841in}{1.844451in}}%
\pgfpathlineto{\pgfqpoint{3.654841in}{1.848709in}}%
\pgfpathlineto{\pgfqpoint{3.659098in}{1.848709in}}%
\pgfpathlineto{\pgfqpoint{3.659098in}{1.844451in}}%
\pgfpathmoveto{\pgfqpoint{3.654841in}{1.848709in}}%
\pgfpathlineto{\pgfqpoint{3.654841in}{1.848709in}}%
\pgfpathlineto{\pgfqpoint{3.654841in}{1.852967in}}%
\pgfpathlineto{\pgfqpoint{3.659098in}{1.852967in}}%
\pgfpathlineto{\pgfqpoint{3.659098in}{1.848709in}}%
\pgfpathmoveto{\pgfqpoint{3.659098in}{1.844451in}}%
\pgfpathlineto{\pgfqpoint{3.659098in}{1.844451in}}%
\pgfpathlineto{\pgfqpoint{3.659098in}{1.848709in}}%
\pgfpathlineto{\pgfqpoint{3.663356in}{1.848709in}}%
\pgfpathlineto{\pgfqpoint{3.663356in}{1.844451in}}%
\pgfpathmoveto{\pgfqpoint{3.659098in}{1.848709in}}%
\pgfpathlineto{\pgfqpoint{3.659098in}{1.848709in}}%
\pgfpathlineto{\pgfqpoint{3.659098in}{1.852967in}}%
\pgfpathlineto{\pgfqpoint{3.663356in}{1.852967in}}%
\pgfpathlineto{\pgfqpoint{3.663356in}{1.848709in}}%
\pgfpathmoveto{\pgfqpoint{3.663356in}{1.844451in}}%
\pgfpathlineto{\pgfqpoint{3.663356in}{1.844451in}}%
\pgfpathlineto{\pgfqpoint{3.663356in}{1.848709in}}%
\pgfpathlineto{\pgfqpoint{3.667614in}{1.848709in}}%
\pgfpathlineto{\pgfqpoint{3.667614in}{1.844451in}}%
\pgfpathmoveto{\pgfqpoint{3.663356in}{1.848709in}}%
\pgfpathlineto{\pgfqpoint{3.663356in}{1.848709in}}%
\pgfpathlineto{\pgfqpoint{3.663356in}{1.852967in}}%
\pgfpathlineto{\pgfqpoint{3.667614in}{1.852967in}}%
\pgfpathlineto{\pgfqpoint{3.667614in}{1.848709in}}%
\pgfpathmoveto{\pgfqpoint{3.654841in}{1.852967in}}%
\pgfpathlineto{\pgfqpoint{3.654841in}{1.852967in}}%
\pgfpathlineto{\pgfqpoint{3.654841in}{1.857224in}}%
\pgfpathlineto{\pgfqpoint{3.659098in}{1.857224in}}%
\pgfpathlineto{\pgfqpoint{3.659098in}{1.852967in}}%
\pgfpathmoveto{\pgfqpoint{3.654841in}{1.857224in}}%
\pgfpathlineto{\pgfqpoint{3.654841in}{1.857224in}}%
\pgfpathlineto{\pgfqpoint{3.654841in}{1.861482in}}%
\pgfpathlineto{\pgfqpoint{3.659098in}{1.861482in}}%
\pgfpathlineto{\pgfqpoint{3.659098in}{1.857224in}}%
\pgfpathmoveto{\pgfqpoint{3.659098in}{1.852967in}}%
\pgfpathlineto{\pgfqpoint{3.659098in}{1.852967in}}%
\pgfpathlineto{\pgfqpoint{3.659098in}{1.857224in}}%
\pgfpathlineto{\pgfqpoint{3.663356in}{1.857224in}}%
\pgfpathlineto{\pgfqpoint{3.663356in}{1.852967in}}%
\pgfpathmoveto{\pgfqpoint{3.659098in}{1.857224in}}%
\pgfpathlineto{\pgfqpoint{3.659098in}{1.857224in}}%
\pgfpathlineto{\pgfqpoint{3.659098in}{1.861482in}}%
\pgfpathlineto{\pgfqpoint{3.663356in}{1.861482in}}%
\pgfpathlineto{\pgfqpoint{3.663356in}{1.857224in}}%
\pgfpathmoveto{\pgfqpoint{3.654841in}{1.861482in}}%
\pgfpathlineto{\pgfqpoint{3.654841in}{1.861482in}}%
\pgfpathlineto{\pgfqpoint{3.654841in}{1.865740in}}%
\pgfpathlineto{\pgfqpoint{3.659098in}{1.865740in}}%
\pgfpathlineto{\pgfqpoint{3.659098in}{1.861482in}}%
\pgfpathmoveto{\pgfqpoint{3.659098in}{1.861482in}}%
\pgfpathlineto{\pgfqpoint{3.659098in}{1.861482in}}%
\pgfpathlineto{\pgfqpoint{3.659098in}{1.865740in}}%
\pgfpathlineto{\pgfqpoint{3.663356in}{1.865740in}}%
\pgfpathlineto{\pgfqpoint{3.663356in}{1.861482in}}%
\pgfpathmoveto{\pgfqpoint{3.659098in}{1.865740in}}%
\pgfpathlineto{\pgfqpoint{3.659098in}{1.865740in}}%
\pgfpathlineto{\pgfqpoint{3.659098in}{1.869998in}}%
\pgfpathlineto{\pgfqpoint{3.663356in}{1.869998in}}%
\pgfpathlineto{\pgfqpoint{3.663356in}{1.865740in}}%
\pgfpathmoveto{\pgfqpoint{3.663356in}{1.852967in}}%
\pgfpathlineto{\pgfqpoint{3.663356in}{1.852967in}}%
\pgfpathlineto{\pgfqpoint{3.663356in}{1.857224in}}%
\pgfpathlineto{\pgfqpoint{3.667614in}{1.857224in}}%
\pgfpathlineto{\pgfqpoint{3.667614in}{1.852967in}}%
\pgfpathmoveto{\pgfqpoint{3.663356in}{1.857224in}}%
\pgfpathlineto{\pgfqpoint{3.663356in}{1.857224in}}%
\pgfpathlineto{\pgfqpoint{3.663356in}{1.861482in}}%
\pgfpathlineto{\pgfqpoint{3.667614in}{1.861482in}}%
\pgfpathlineto{\pgfqpoint{3.667614in}{1.857224in}}%
\pgfpathmoveto{\pgfqpoint{3.663356in}{1.861482in}}%
\pgfpathlineto{\pgfqpoint{3.663356in}{1.861482in}}%
\pgfpathlineto{\pgfqpoint{3.663356in}{1.865740in}}%
\pgfpathlineto{\pgfqpoint{3.667614in}{1.865740in}}%
\pgfpathlineto{\pgfqpoint{3.667614in}{1.861482in}}%
\pgfpathmoveto{\pgfqpoint{3.663356in}{1.865740in}}%
\pgfpathlineto{\pgfqpoint{3.663356in}{1.865740in}}%
\pgfpathlineto{\pgfqpoint{3.663356in}{1.869998in}}%
\pgfpathlineto{\pgfqpoint{3.667614in}{1.869998in}}%
\pgfpathlineto{\pgfqpoint{3.667614in}{1.865740in}}%
\pgfpathmoveto{\pgfqpoint{3.667614in}{1.861482in}}%
\pgfpathlineto{\pgfqpoint{3.667614in}{1.861482in}}%
\pgfpathlineto{\pgfqpoint{3.667614in}{1.865740in}}%
\pgfpathlineto{\pgfqpoint{3.671872in}{1.865740in}}%
\pgfpathlineto{\pgfqpoint{3.671872in}{1.861482in}}%
\pgfpathmoveto{\pgfqpoint{3.667614in}{1.865740in}}%
\pgfpathlineto{\pgfqpoint{3.667614in}{1.865740in}}%
\pgfpathlineto{\pgfqpoint{3.667614in}{1.869998in}}%
\pgfpathlineto{\pgfqpoint{3.671872in}{1.869998in}}%
\pgfpathlineto{\pgfqpoint{3.671872in}{1.865740in}}%
\pgfpathmoveto{\pgfqpoint{3.659098in}{1.869998in}}%
\pgfpathlineto{\pgfqpoint{3.659098in}{1.869998in}}%
\pgfpathlineto{\pgfqpoint{3.659098in}{1.874255in}}%
\pgfpathlineto{\pgfqpoint{3.663356in}{1.874255in}}%
\pgfpathlineto{\pgfqpoint{3.663356in}{1.869998in}}%
\pgfpathmoveto{\pgfqpoint{3.659098in}{1.874255in}}%
\pgfpathlineto{\pgfqpoint{3.659098in}{1.874255in}}%
\pgfpathlineto{\pgfqpoint{3.659098in}{1.878513in}}%
\pgfpathlineto{\pgfqpoint{3.663356in}{1.878513in}}%
\pgfpathlineto{\pgfqpoint{3.663356in}{1.874255in}}%
\pgfpathmoveto{\pgfqpoint{3.659098in}{1.878513in}}%
\pgfpathlineto{\pgfqpoint{3.659098in}{1.878513in}}%
\pgfpathlineto{\pgfqpoint{3.659098in}{1.882771in}}%
\pgfpathlineto{\pgfqpoint{3.663356in}{1.882771in}}%
\pgfpathlineto{\pgfqpoint{3.663356in}{1.878513in}}%
\pgfpathmoveto{\pgfqpoint{3.663356in}{1.869998in}}%
\pgfpathlineto{\pgfqpoint{3.663356in}{1.869998in}}%
\pgfpathlineto{\pgfqpoint{3.663356in}{1.874255in}}%
\pgfpathlineto{\pgfqpoint{3.667614in}{1.874255in}}%
\pgfpathlineto{\pgfqpoint{3.667614in}{1.869998in}}%
\pgfpathmoveto{\pgfqpoint{3.663356in}{1.874255in}}%
\pgfpathlineto{\pgfqpoint{3.663356in}{1.874255in}}%
\pgfpathlineto{\pgfqpoint{3.663356in}{1.878513in}}%
\pgfpathlineto{\pgfqpoint{3.667614in}{1.878513in}}%
\pgfpathlineto{\pgfqpoint{3.667614in}{1.874255in}}%
\pgfpathmoveto{\pgfqpoint{3.667614in}{1.869998in}}%
\pgfpathlineto{\pgfqpoint{3.667614in}{1.869998in}}%
\pgfpathlineto{\pgfqpoint{3.667614in}{1.874255in}}%
\pgfpathlineto{\pgfqpoint{3.671872in}{1.874255in}}%
\pgfpathlineto{\pgfqpoint{3.671872in}{1.869998in}}%
\pgfpathmoveto{\pgfqpoint{3.667614in}{1.874255in}}%
\pgfpathlineto{\pgfqpoint{3.667614in}{1.874255in}}%
\pgfpathlineto{\pgfqpoint{3.667614in}{1.878513in}}%
\pgfpathlineto{\pgfqpoint{3.671872in}{1.878513in}}%
\pgfpathlineto{\pgfqpoint{3.671872in}{1.874255in}}%
\pgfpathmoveto{\pgfqpoint{3.663356in}{1.878513in}}%
\pgfpathlineto{\pgfqpoint{3.663356in}{1.878513in}}%
\pgfpathlineto{\pgfqpoint{3.663356in}{1.882771in}}%
\pgfpathlineto{\pgfqpoint{3.667614in}{1.882771in}}%
\pgfpathlineto{\pgfqpoint{3.667614in}{1.878513in}}%
\pgfpathmoveto{\pgfqpoint{3.663356in}{1.882771in}}%
\pgfpathlineto{\pgfqpoint{3.663356in}{1.882771in}}%
\pgfpathlineto{\pgfqpoint{3.663356in}{1.887029in}}%
\pgfpathlineto{\pgfqpoint{3.667614in}{1.887029in}}%
\pgfpathlineto{\pgfqpoint{3.667614in}{1.882771in}}%
\pgfpathmoveto{\pgfqpoint{3.667614in}{1.878513in}}%
\pgfpathlineto{\pgfqpoint{3.667614in}{1.878513in}}%
\pgfpathlineto{\pgfqpoint{3.667614in}{1.882771in}}%
\pgfpathlineto{\pgfqpoint{3.671872in}{1.882771in}}%
\pgfpathlineto{\pgfqpoint{3.671872in}{1.878513in}}%
\pgfpathmoveto{\pgfqpoint{3.667614in}{1.882771in}}%
\pgfpathlineto{\pgfqpoint{3.667614in}{1.882771in}}%
\pgfpathlineto{\pgfqpoint{3.667614in}{1.887029in}}%
\pgfpathlineto{\pgfqpoint{3.671872in}{1.887029in}}%
\pgfpathlineto{\pgfqpoint{3.671872in}{1.882771in}}%
\pgfpathmoveto{\pgfqpoint{3.663356in}{1.887029in}}%
\pgfpathlineto{\pgfqpoint{3.663356in}{1.887029in}}%
\pgfpathlineto{\pgfqpoint{3.663356in}{1.891287in}}%
\pgfpathlineto{\pgfqpoint{3.667614in}{1.891287in}}%
\pgfpathlineto{\pgfqpoint{3.667614in}{1.887029in}}%
\pgfpathmoveto{\pgfqpoint{3.663356in}{1.891287in}}%
\pgfpathlineto{\pgfqpoint{3.663356in}{1.891287in}}%
\pgfpathlineto{\pgfqpoint{3.663356in}{1.895545in}}%
\pgfpathlineto{\pgfqpoint{3.667614in}{1.895545in}}%
\pgfpathlineto{\pgfqpoint{3.667614in}{1.891287in}}%
\pgfpathmoveto{\pgfqpoint{3.667614in}{1.887029in}}%
\pgfpathlineto{\pgfqpoint{3.667614in}{1.887029in}}%
\pgfpathlineto{\pgfqpoint{3.667614in}{1.891287in}}%
\pgfpathlineto{\pgfqpoint{3.671872in}{1.891287in}}%
\pgfpathlineto{\pgfqpoint{3.671872in}{1.887029in}}%
\pgfpathmoveto{\pgfqpoint{3.667614in}{1.891287in}}%
\pgfpathlineto{\pgfqpoint{3.667614in}{1.891287in}}%
\pgfpathlineto{\pgfqpoint{3.667614in}{1.895545in}}%
\pgfpathlineto{\pgfqpoint{3.671872in}{1.895545in}}%
\pgfpathlineto{\pgfqpoint{3.671872in}{1.891287in}}%
\pgfpathmoveto{\pgfqpoint{3.663356in}{1.895545in}}%
\pgfpathlineto{\pgfqpoint{3.663356in}{1.895545in}}%
\pgfpathlineto{\pgfqpoint{3.663356in}{1.899802in}}%
\pgfpathlineto{\pgfqpoint{3.667614in}{1.899802in}}%
\pgfpathlineto{\pgfqpoint{3.667614in}{1.895545in}}%
\pgfpathmoveto{\pgfqpoint{3.667614in}{1.895545in}}%
\pgfpathlineto{\pgfqpoint{3.667614in}{1.895545in}}%
\pgfpathlineto{\pgfqpoint{3.667614in}{1.899802in}}%
\pgfpathlineto{\pgfqpoint{3.671872in}{1.899802in}}%
\pgfpathlineto{\pgfqpoint{3.671872in}{1.895545in}}%
\pgfpathmoveto{\pgfqpoint{3.667614in}{1.899802in}}%
\pgfpathlineto{\pgfqpoint{3.667614in}{1.899802in}}%
\pgfpathlineto{\pgfqpoint{3.667614in}{1.904060in}}%
\pgfpathlineto{\pgfqpoint{3.671872in}{1.904060in}}%
\pgfpathlineto{\pgfqpoint{3.671872in}{1.899802in}}%
\pgfpathmoveto{\pgfqpoint{3.667614in}{1.904060in}}%
\pgfpathlineto{\pgfqpoint{3.667614in}{1.904060in}}%
\pgfpathlineto{\pgfqpoint{3.667614in}{1.908318in}}%
\pgfpathlineto{\pgfqpoint{3.671872in}{1.908318in}}%
\pgfpathlineto{\pgfqpoint{3.671872in}{1.904060in}}%
\pgfpathmoveto{\pgfqpoint{3.667614in}{1.908318in}}%
\pgfpathlineto{\pgfqpoint{3.667614in}{1.908318in}}%
\pgfpathlineto{\pgfqpoint{3.667614in}{1.912576in}}%
\pgfpathlineto{\pgfqpoint{3.671872in}{1.912576in}}%
\pgfpathlineto{\pgfqpoint{3.671872in}{1.908318in}}%
\pgfpathmoveto{\pgfqpoint{3.667614in}{1.912576in}}%
\pgfpathlineto{\pgfqpoint{3.667614in}{1.912576in}}%
\pgfpathlineto{\pgfqpoint{3.667614in}{1.916834in}}%
\pgfpathlineto{\pgfqpoint{3.671872in}{1.916834in}}%
\pgfpathlineto{\pgfqpoint{3.671872in}{1.912576in}}%
\pgfpathmoveto{\pgfqpoint{3.671872in}{1.878513in}}%
\pgfpathlineto{\pgfqpoint{3.671872in}{1.878513in}}%
\pgfpathlineto{\pgfqpoint{3.671872in}{1.882771in}}%
\pgfpathlineto{\pgfqpoint{3.676130in}{1.882771in}}%
\pgfpathlineto{\pgfqpoint{3.676130in}{1.878513in}}%
\pgfpathmoveto{\pgfqpoint{3.671872in}{1.882771in}}%
\pgfpathlineto{\pgfqpoint{3.671872in}{1.882771in}}%
\pgfpathlineto{\pgfqpoint{3.671872in}{1.887029in}}%
\pgfpathlineto{\pgfqpoint{3.676130in}{1.887029in}}%
\pgfpathlineto{\pgfqpoint{3.676130in}{1.882771in}}%
\pgfpathmoveto{\pgfqpoint{3.671872in}{1.887029in}}%
\pgfpathlineto{\pgfqpoint{3.671872in}{1.887029in}}%
\pgfpathlineto{\pgfqpoint{3.671872in}{1.891287in}}%
\pgfpathlineto{\pgfqpoint{3.676130in}{1.891287in}}%
\pgfpathlineto{\pgfqpoint{3.676130in}{1.887029in}}%
\pgfpathmoveto{\pgfqpoint{3.671872in}{1.891287in}}%
\pgfpathlineto{\pgfqpoint{3.671872in}{1.891287in}}%
\pgfpathlineto{\pgfqpoint{3.671872in}{1.895545in}}%
\pgfpathlineto{\pgfqpoint{3.676130in}{1.895545in}}%
\pgfpathlineto{\pgfqpoint{3.676130in}{1.891287in}}%
\pgfpathmoveto{\pgfqpoint{3.671872in}{1.895545in}}%
\pgfpathlineto{\pgfqpoint{3.671872in}{1.895545in}}%
\pgfpathlineto{\pgfqpoint{3.671872in}{1.899802in}}%
\pgfpathlineto{\pgfqpoint{3.676130in}{1.899802in}}%
\pgfpathlineto{\pgfqpoint{3.676130in}{1.895545in}}%
\pgfpathmoveto{\pgfqpoint{3.671872in}{1.899802in}}%
\pgfpathlineto{\pgfqpoint{3.671872in}{1.899802in}}%
\pgfpathlineto{\pgfqpoint{3.671872in}{1.904060in}}%
\pgfpathlineto{\pgfqpoint{3.676130in}{1.904060in}}%
\pgfpathlineto{\pgfqpoint{3.676130in}{1.899802in}}%
\pgfpathmoveto{\pgfqpoint{3.676130in}{1.895545in}}%
\pgfpathlineto{\pgfqpoint{3.676130in}{1.895545in}}%
\pgfpathlineto{\pgfqpoint{3.676130in}{1.899802in}}%
\pgfpathlineto{\pgfqpoint{3.680388in}{1.899802in}}%
\pgfpathlineto{\pgfqpoint{3.680388in}{1.895545in}}%
\pgfpathmoveto{\pgfqpoint{3.676130in}{1.899802in}}%
\pgfpathlineto{\pgfqpoint{3.676130in}{1.899802in}}%
\pgfpathlineto{\pgfqpoint{3.676130in}{1.904060in}}%
\pgfpathlineto{\pgfqpoint{3.680388in}{1.904060in}}%
\pgfpathlineto{\pgfqpoint{3.680388in}{1.899802in}}%
\pgfpathmoveto{\pgfqpoint{3.671872in}{1.904060in}}%
\pgfpathlineto{\pgfqpoint{3.671872in}{1.904060in}}%
\pgfpathlineto{\pgfqpoint{3.671872in}{1.908318in}}%
\pgfpathlineto{\pgfqpoint{3.676130in}{1.908318in}}%
\pgfpathlineto{\pgfqpoint{3.676130in}{1.904060in}}%
\pgfpathmoveto{\pgfqpoint{3.671872in}{1.908318in}}%
\pgfpathlineto{\pgfqpoint{3.671872in}{1.908318in}}%
\pgfpathlineto{\pgfqpoint{3.671872in}{1.912576in}}%
\pgfpathlineto{\pgfqpoint{3.676130in}{1.912576in}}%
\pgfpathlineto{\pgfqpoint{3.676130in}{1.908318in}}%
\pgfpathmoveto{\pgfqpoint{3.676130in}{1.904060in}}%
\pgfpathlineto{\pgfqpoint{3.676130in}{1.904060in}}%
\pgfpathlineto{\pgfqpoint{3.676130in}{1.908318in}}%
\pgfpathlineto{\pgfqpoint{3.680388in}{1.908318in}}%
\pgfpathlineto{\pgfqpoint{3.680388in}{1.904060in}}%
\pgfpathmoveto{\pgfqpoint{3.676130in}{1.908318in}}%
\pgfpathlineto{\pgfqpoint{3.676130in}{1.908318in}}%
\pgfpathlineto{\pgfqpoint{3.676130in}{1.912576in}}%
\pgfpathlineto{\pgfqpoint{3.680388in}{1.912576in}}%
\pgfpathlineto{\pgfqpoint{3.680388in}{1.908318in}}%
\pgfpathmoveto{\pgfqpoint{3.671872in}{1.912576in}}%
\pgfpathlineto{\pgfqpoint{3.671872in}{1.912576in}}%
\pgfpathlineto{\pgfqpoint{3.671872in}{1.916834in}}%
\pgfpathlineto{\pgfqpoint{3.676130in}{1.916834in}}%
\pgfpathlineto{\pgfqpoint{3.676130in}{1.912576in}}%
\pgfpathmoveto{\pgfqpoint{3.671872in}{1.916834in}}%
\pgfpathlineto{\pgfqpoint{3.671872in}{1.916834in}}%
\pgfpathlineto{\pgfqpoint{3.671872in}{1.921092in}}%
\pgfpathlineto{\pgfqpoint{3.676130in}{1.921092in}}%
\pgfpathlineto{\pgfqpoint{3.676130in}{1.916834in}}%
\pgfpathmoveto{\pgfqpoint{3.676130in}{1.912576in}}%
\pgfpathlineto{\pgfqpoint{3.676130in}{1.912576in}}%
\pgfpathlineto{\pgfqpoint{3.676130in}{1.916834in}}%
\pgfpathlineto{\pgfqpoint{3.680388in}{1.916834in}}%
\pgfpathlineto{\pgfqpoint{3.680388in}{1.912576in}}%
\pgfpathmoveto{\pgfqpoint{3.676130in}{1.916834in}}%
\pgfpathlineto{\pgfqpoint{3.676130in}{1.916834in}}%
\pgfpathlineto{\pgfqpoint{3.676130in}{1.921092in}}%
\pgfpathlineto{\pgfqpoint{3.680388in}{1.921092in}}%
\pgfpathlineto{\pgfqpoint{3.680388in}{1.916834in}}%
\pgfpathmoveto{\pgfqpoint{3.680388in}{1.912576in}}%
\pgfpathlineto{\pgfqpoint{3.680388in}{1.912576in}}%
\pgfpathlineto{\pgfqpoint{3.680388in}{1.916834in}}%
\pgfpathlineto{\pgfqpoint{3.684646in}{1.916834in}}%
\pgfpathlineto{\pgfqpoint{3.684646in}{1.912576in}}%
\pgfpathmoveto{\pgfqpoint{3.680388in}{1.916834in}}%
\pgfpathlineto{\pgfqpoint{3.680388in}{1.916834in}}%
\pgfpathlineto{\pgfqpoint{3.680388in}{1.921092in}}%
\pgfpathlineto{\pgfqpoint{3.684646in}{1.921092in}}%
\pgfpathlineto{\pgfqpoint{3.684646in}{1.916834in}}%
\pgfpathmoveto{\pgfqpoint{3.671872in}{1.921092in}}%
\pgfpathlineto{\pgfqpoint{3.671872in}{1.921092in}}%
\pgfpathlineto{\pgfqpoint{3.671872in}{1.925349in}}%
\pgfpathlineto{\pgfqpoint{3.676130in}{1.925349in}}%
\pgfpathlineto{\pgfqpoint{3.676130in}{1.921092in}}%
\pgfpathmoveto{\pgfqpoint{3.671872in}{1.925349in}}%
\pgfpathlineto{\pgfqpoint{3.671872in}{1.925349in}}%
\pgfpathlineto{\pgfqpoint{3.671872in}{1.929607in}}%
\pgfpathlineto{\pgfqpoint{3.676130in}{1.929607in}}%
\pgfpathlineto{\pgfqpoint{3.676130in}{1.925349in}}%
\pgfpathmoveto{\pgfqpoint{3.676130in}{1.921092in}}%
\pgfpathlineto{\pgfqpoint{3.676130in}{1.921092in}}%
\pgfpathlineto{\pgfqpoint{3.676130in}{1.925349in}}%
\pgfpathlineto{\pgfqpoint{3.680388in}{1.925349in}}%
\pgfpathlineto{\pgfqpoint{3.680388in}{1.921092in}}%
\pgfpathmoveto{\pgfqpoint{3.676130in}{1.925349in}}%
\pgfpathlineto{\pgfqpoint{3.676130in}{1.925349in}}%
\pgfpathlineto{\pgfqpoint{3.676130in}{1.929607in}}%
\pgfpathlineto{\pgfqpoint{3.680388in}{1.929607in}}%
\pgfpathlineto{\pgfqpoint{3.680388in}{1.925349in}}%
\pgfpathmoveto{\pgfqpoint{3.671872in}{1.929607in}}%
\pgfpathlineto{\pgfqpoint{3.671872in}{1.929607in}}%
\pgfpathlineto{\pgfqpoint{3.671872in}{1.933865in}}%
\pgfpathlineto{\pgfqpoint{3.676130in}{1.933865in}}%
\pgfpathlineto{\pgfqpoint{3.676130in}{1.929607in}}%
\pgfpathmoveto{\pgfqpoint{3.676130in}{1.929607in}}%
\pgfpathlineto{\pgfqpoint{3.676130in}{1.929607in}}%
\pgfpathlineto{\pgfqpoint{3.676130in}{1.933865in}}%
\pgfpathlineto{\pgfqpoint{3.680388in}{1.933865in}}%
\pgfpathlineto{\pgfqpoint{3.680388in}{1.929607in}}%
\pgfpathmoveto{\pgfqpoint{3.676130in}{1.933865in}}%
\pgfpathlineto{\pgfqpoint{3.676130in}{1.933865in}}%
\pgfpathlineto{\pgfqpoint{3.676130in}{1.938123in}}%
\pgfpathlineto{\pgfqpoint{3.680388in}{1.938123in}}%
\pgfpathlineto{\pgfqpoint{3.680388in}{1.933865in}}%
\pgfpathmoveto{\pgfqpoint{3.680388in}{1.921092in}}%
\pgfpathlineto{\pgfqpoint{3.680388in}{1.921092in}}%
\pgfpathlineto{\pgfqpoint{3.680388in}{1.925349in}}%
\pgfpathlineto{\pgfqpoint{3.684646in}{1.925349in}}%
\pgfpathlineto{\pgfqpoint{3.684646in}{1.921092in}}%
\pgfpathmoveto{\pgfqpoint{3.680388in}{1.925349in}}%
\pgfpathlineto{\pgfqpoint{3.680388in}{1.925349in}}%
\pgfpathlineto{\pgfqpoint{3.680388in}{1.929607in}}%
\pgfpathlineto{\pgfqpoint{3.684646in}{1.929607in}}%
\pgfpathlineto{\pgfqpoint{3.684646in}{1.925349in}}%
\pgfpathmoveto{\pgfqpoint{3.680388in}{1.929607in}}%
\pgfpathlineto{\pgfqpoint{3.680388in}{1.929607in}}%
\pgfpathlineto{\pgfqpoint{3.680388in}{1.933865in}}%
\pgfpathlineto{\pgfqpoint{3.684646in}{1.933865in}}%
\pgfpathlineto{\pgfqpoint{3.684646in}{1.929607in}}%
\pgfpathmoveto{\pgfqpoint{3.680388in}{1.933865in}}%
\pgfpathlineto{\pgfqpoint{3.680388in}{1.933865in}}%
\pgfpathlineto{\pgfqpoint{3.680388in}{1.938123in}}%
\pgfpathlineto{\pgfqpoint{3.684646in}{1.938123in}}%
\pgfpathlineto{\pgfqpoint{3.684646in}{1.933865in}}%
\pgfpathmoveto{\pgfqpoint{3.684646in}{1.933865in}}%
\pgfpathlineto{\pgfqpoint{3.684646in}{1.933865in}}%
\pgfpathlineto{\pgfqpoint{3.684646in}{1.938123in}}%
\pgfpathlineto{\pgfqpoint{3.688904in}{1.938123in}}%
\pgfpathlineto{\pgfqpoint{3.688904in}{1.933865in}}%
\pgfpathmoveto{\pgfqpoint{3.676130in}{1.938123in}}%
\pgfpathlineto{\pgfqpoint{3.676130in}{1.938123in}}%
\pgfpathlineto{\pgfqpoint{3.676130in}{1.942381in}}%
\pgfpathlineto{\pgfqpoint{3.680388in}{1.942381in}}%
\pgfpathlineto{\pgfqpoint{3.680388in}{1.938123in}}%
\pgfpathmoveto{\pgfqpoint{3.676130in}{1.942381in}}%
\pgfpathlineto{\pgfqpoint{3.676130in}{1.942381in}}%
\pgfpathlineto{\pgfqpoint{3.676130in}{1.946638in}}%
\pgfpathlineto{\pgfqpoint{3.680388in}{1.946638in}}%
\pgfpathlineto{\pgfqpoint{3.680388in}{1.942381in}}%
\pgfpathmoveto{\pgfqpoint{3.676130in}{1.946638in}}%
\pgfpathlineto{\pgfqpoint{3.676130in}{1.946638in}}%
\pgfpathlineto{\pgfqpoint{3.676130in}{1.950896in}}%
\pgfpathlineto{\pgfqpoint{3.680388in}{1.950896in}}%
\pgfpathlineto{\pgfqpoint{3.680388in}{1.946638in}}%
\pgfpathmoveto{\pgfqpoint{3.680388in}{1.938123in}}%
\pgfpathlineto{\pgfqpoint{3.680388in}{1.938123in}}%
\pgfpathlineto{\pgfqpoint{3.680388in}{1.942381in}}%
\pgfpathlineto{\pgfqpoint{3.684646in}{1.942381in}}%
\pgfpathlineto{\pgfqpoint{3.684646in}{1.938123in}}%
\pgfpathmoveto{\pgfqpoint{3.680388in}{1.942381in}}%
\pgfpathlineto{\pgfqpoint{3.680388in}{1.942381in}}%
\pgfpathlineto{\pgfqpoint{3.680388in}{1.946638in}}%
\pgfpathlineto{\pgfqpoint{3.684646in}{1.946638in}}%
\pgfpathlineto{\pgfqpoint{3.684646in}{1.942381in}}%
\pgfpathmoveto{\pgfqpoint{3.684646in}{1.938123in}}%
\pgfpathlineto{\pgfqpoint{3.684646in}{1.938123in}}%
\pgfpathlineto{\pgfqpoint{3.684646in}{1.942381in}}%
\pgfpathlineto{\pgfqpoint{3.688904in}{1.942381in}}%
\pgfpathlineto{\pgfqpoint{3.688904in}{1.938123in}}%
\pgfpathmoveto{\pgfqpoint{3.684646in}{1.942381in}}%
\pgfpathlineto{\pgfqpoint{3.684646in}{1.942381in}}%
\pgfpathlineto{\pgfqpoint{3.684646in}{1.946638in}}%
\pgfpathlineto{\pgfqpoint{3.688904in}{1.946638in}}%
\pgfpathlineto{\pgfqpoint{3.688904in}{1.942381in}}%
\pgfpathmoveto{\pgfqpoint{3.680388in}{1.946638in}}%
\pgfpathlineto{\pgfqpoint{3.680388in}{1.946638in}}%
\pgfpathlineto{\pgfqpoint{3.680388in}{1.950896in}}%
\pgfpathlineto{\pgfqpoint{3.684646in}{1.950896in}}%
\pgfpathlineto{\pgfqpoint{3.684646in}{1.946638in}}%
\pgfpathmoveto{\pgfqpoint{3.680388in}{1.950896in}}%
\pgfpathlineto{\pgfqpoint{3.680388in}{1.950896in}}%
\pgfpathlineto{\pgfqpoint{3.680388in}{1.955154in}}%
\pgfpathlineto{\pgfqpoint{3.684646in}{1.955154in}}%
\pgfpathlineto{\pgfqpoint{3.684646in}{1.950896in}}%
\pgfpathmoveto{\pgfqpoint{3.684646in}{1.946638in}}%
\pgfpathlineto{\pgfqpoint{3.684646in}{1.946638in}}%
\pgfpathlineto{\pgfqpoint{3.684646in}{1.950896in}}%
\pgfpathlineto{\pgfqpoint{3.688904in}{1.950896in}}%
\pgfpathlineto{\pgfqpoint{3.688904in}{1.946638in}}%
\pgfpathmoveto{\pgfqpoint{3.684646in}{1.950896in}}%
\pgfpathlineto{\pgfqpoint{3.684646in}{1.950896in}}%
\pgfpathlineto{\pgfqpoint{3.684646in}{1.955154in}}%
\pgfpathlineto{\pgfqpoint{3.688904in}{1.955154in}}%
\pgfpathlineto{\pgfqpoint{3.688904in}{1.950896in}}%
\pgfpathmoveto{\pgfqpoint{3.680388in}{1.955154in}}%
\pgfpathlineto{\pgfqpoint{3.680388in}{1.955154in}}%
\pgfpathlineto{\pgfqpoint{3.680388in}{1.959412in}}%
\pgfpathlineto{\pgfqpoint{3.684646in}{1.959412in}}%
\pgfpathlineto{\pgfqpoint{3.684646in}{1.955154in}}%
\pgfpathmoveto{\pgfqpoint{3.680388in}{1.959412in}}%
\pgfpathlineto{\pgfqpoint{3.680388in}{1.959412in}}%
\pgfpathlineto{\pgfqpoint{3.680388in}{1.963670in}}%
\pgfpathlineto{\pgfqpoint{3.684646in}{1.963670in}}%
\pgfpathlineto{\pgfqpoint{3.684646in}{1.959412in}}%
\pgfpathmoveto{\pgfqpoint{3.684646in}{1.955154in}}%
\pgfpathlineto{\pgfqpoint{3.684646in}{1.955154in}}%
\pgfpathlineto{\pgfqpoint{3.684646in}{1.959412in}}%
\pgfpathlineto{\pgfqpoint{3.688904in}{1.959412in}}%
\pgfpathlineto{\pgfqpoint{3.688904in}{1.955154in}}%
\pgfpathmoveto{\pgfqpoint{3.684646in}{1.959412in}}%
\pgfpathlineto{\pgfqpoint{3.684646in}{1.959412in}}%
\pgfpathlineto{\pgfqpoint{3.684646in}{1.963670in}}%
\pgfpathlineto{\pgfqpoint{3.688904in}{1.963670in}}%
\pgfpathlineto{\pgfqpoint{3.688904in}{1.959412in}}%
\pgfpathmoveto{\pgfqpoint{3.680388in}{1.963670in}}%
\pgfpathlineto{\pgfqpoint{3.680388in}{1.963670in}}%
\pgfpathlineto{\pgfqpoint{3.680388in}{1.967928in}}%
\pgfpathlineto{\pgfqpoint{3.684646in}{1.967928in}}%
\pgfpathlineto{\pgfqpoint{3.684646in}{1.963670in}}%
\pgfpathmoveto{\pgfqpoint{3.680388in}{1.967928in}}%
\pgfpathlineto{\pgfqpoint{3.680388in}{1.967928in}}%
\pgfpathlineto{\pgfqpoint{3.680388in}{1.972185in}}%
\pgfpathlineto{\pgfqpoint{3.684646in}{1.972185in}}%
\pgfpathlineto{\pgfqpoint{3.684646in}{1.967928in}}%
\pgfpathmoveto{\pgfqpoint{3.684646in}{1.963670in}}%
\pgfpathlineto{\pgfqpoint{3.684646in}{1.963670in}}%
\pgfpathlineto{\pgfqpoint{3.684646in}{1.967928in}}%
\pgfpathlineto{\pgfqpoint{3.688904in}{1.967928in}}%
\pgfpathlineto{\pgfqpoint{3.688904in}{1.963670in}}%
\pgfpathmoveto{\pgfqpoint{3.684646in}{1.967928in}}%
\pgfpathlineto{\pgfqpoint{3.684646in}{1.967928in}}%
\pgfpathlineto{\pgfqpoint{3.684646in}{1.972185in}}%
\pgfpathlineto{\pgfqpoint{3.688904in}{1.972185in}}%
\pgfpathlineto{\pgfqpoint{3.688904in}{1.967928in}}%
\pgfpathmoveto{\pgfqpoint{3.688904in}{1.950896in}}%
\pgfpathlineto{\pgfqpoint{3.688904in}{1.950896in}}%
\pgfpathlineto{\pgfqpoint{3.688904in}{1.955154in}}%
\pgfpathlineto{\pgfqpoint{3.693162in}{1.955154in}}%
\pgfpathlineto{\pgfqpoint{3.693162in}{1.950896in}}%
\pgfpathmoveto{\pgfqpoint{3.688904in}{1.955154in}}%
\pgfpathlineto{\pgfqpoint{3.688904in}{1.955154in}}%
\pgfpathlineto{\pgfqpoint{3.688904in}{1.959412in}}%
\pgfpathlineto{\pgfqpoint{3.693162in}{1.959412in}}%
\pgfpathlineto{\pgfqpoint{3.693162in}{1.955154in}}%
\pgfpathmoveto{\pgfqpoint{3.688904in}{1.959412in}}%
\pgfpathlineto{\pgfqpoint{3.688904in}{1.959412in}}%
\pgfpathlineto{\pgfqpoint{3.688904in}{1.963670in}}%
\pgfpathlineto{\pgfqpoint{3.693162in}{1.963670in}}%
\pgfpathlineto{\pgfqpoint{3.693162in}{1.959412in}}%
\pgfpathmoveto{\pgfqpoint{3.688904in}{1.963670in}}%
\pgfpathlineto{\pgfqpoint{3.688904in}{1.963670in}}%
\pgfpathlineto{\pgfqpoint{3.688904in}{1.967928in}}%
\pgfpathlineto{\pgfqpoint{3.693162in}{1.967928in}}%
\pgfpathlineto{\pgfqpoint{3.693162in}{1.963670in}}%
\pgfpathmoveto{\pgfqpoint{3.688904in}{1.967928in}}%
\pgfpathlineto{\pgfqpoint{3.688904in}{1.967928in}}%
\pgfpathlineto{\pgfqpoint{3.688904in}{1.972185in}}%
\pgfpathlineto{\pgfqpoint{3.693162in}{1.972185in}}%
\pgfpathlineto{\pgfqpoint{3.693162in}{1.967928in}}%
\pgfpathmoveto{\pgfqpoint{3.684646in}{1.972185in}}%
\pgfpathlineto{\pgfqpoint{3.684646in}{1.972185in}}%
\pgfpathlineto{\pgfqpoint{3.684646in}{1.976443in}}%
\pgfpathlineto{\pgfqpoint{3.688904in}{1.976443in}}%
\pgfpathlineto{\pgfqpoint{3.688904in}{1.972185in}}%
\pgfpathmoveto{\pgfqpoint{3.684646in}{1.976443in}}%
\pgfpathlineto{\pgfqpoint{3.684646in}{1.976443in}}%
\pgfpathlineto{\pgfqpoint{3.684646in}{1.980701in}}%
\pgfpathlineto{\pgfqpoint{3.688904in}{1.980701in}}%
\pgfpathlineto{\pgfqpoint{3.688904in}{1.976443in}}%
\pgfpathmoveto{\pgfqpoint{3.684646in}{1.980701in}}%
\pgfpathlineto{\pgfqpoint{3.684646in}{1.980701in}}%
\pgfpathlineto{\pgfqpoint{3.684646in}{1.984959in}}%
\pgfpathlineto{\pgfqpoint{3.688904in}{1.984959in}}%
\pgfpathlineto{\pgfqpoint{3.688904in}{1.980701in}}%
\pgfpathmoveto{\pgfqpoint{3.684646in}{1.984959in}}%
\pgfpathlineto{\pgfqpoint{3.684646in}{1.984959in}}%
\pgfpathlineto{\pgfqpoint{3.684646in}{1.989217in}}%
\pgfpathlineto{\pgfqpoint{3.688904in}{1.989217in}}%
\pgfpathlineto{\pgfqpoint{3.688904in}{1.984959in}}%
\pgfpathmoveto{\pgfqpoint{3.688904in}{1.972185in}}%
\pgfpathlineto{\pgfqpoint{3.688904in}{1.972185in}}%
\pgfpathlineto{\pgfqpoint{3.688904in}{1.976443in}}%
\pgfpathlineto{\pgfqpoint{3.693162in}{1.976443in}}%
\pgfpathlineto{\pgfqpoint{3.693162in}{1.972185in}}%
\pgfpathmoveto{\pgfqpoint{3.688904in}{1.976443in}}%
\pgfpathlineto{\pgfqpoint{3.688904in}{1.976443in}}%
\pgfpathlineto{\pgfqpoint{3.688904in}{1.980701in}}%
\pgfpathlineto{\pgfqpoint{3.693162in}{1.980701in}}%
\pgfpathlineto{\pgfqpoint{3.693162in}{1.976443in}}%
\pgfpathmoveto{\pgfqpoint{3.693162in}{1.972185in}}%
\pgfpathlineto{\pgfqpoint{3.693162in}{1.972185in}}%
\pgfpathlineto{\pgfqpoint{3.693162in}{1.976443in}}%
\pgfpathlineto{\pgfqpoint{3.697420in}{1.976443in}}%
\pgfpathlineto{\pgfqpoint{3.697420in}{1.972185in}}%
\pgfpathmoveto{\pgfqpoint{3.693162in}{1.976443in}}%
\pgfpathlineto{\pgfqpoint{3.693162in}{1.976443in}}%
\pgfpathlineto{\pgfqpoint{3.693162in}{1.980701in}}%
\pgfpathlineto{\pgfqpoint{3.697420in}{1.980701in}}%
\pgfpathlineto{\pgfqpoint{3.697420in}{1.976443in}}%
\pgfpathmoveto{\pgfqpoint{3.688904in}{1.980701in}}%
\pgfpathlineto{\pgfqpoint{3.688904in}{1.980701in}}%
\pgfpathlineto{\pgfqpoint{3.688904in}{1.984959in}}%
\pgfpathlineto{\pgfqpoint{3.693162in}{1.984959in}}%
\pgfpathlineto{\pgfqpoint{3.693162in}{1.980701in}}%
\pgfpathmoveto{\pgfqpoint{3.688904in}{1.984959in}}%
\pgfpathlineto{\pgfqpoint{3.688904in}{1.984959in}}%
\pgfpathlineto{\pgfqpoint{3.688904in}{1.989217in}}%
\pgfpathlineto{\pgfqpoint{3.693162in}{1.989217in}}%
\pgfpathlineto{\pgfqpoint{3.693162in}{1.984959in}}%
\pgfpathmoveto{\pgfqpoint{3.693162in}{1.980701in}}%
\pgfpathlineto{\pgfqpoint{3.693162in}{1.980701in}}%
\pgfpathlineto{\pgfqpoint{3.693162in}{1.984959in}}%
\pgfpathlineto{\pgfqpoint{3.697420in}{1.984959in}}%
\pgfpathlineto{\pgfqpoint{3.697420in}{1.980701in}}%
\pgfpathmoveto{\pgfqpoint{3.693162in}{1.984959in}}%
\pgfpathlineto{\pgfqpoint{3.693162in}{1.984959in}}%
\pgfpathlineto{\pgfqpoint{3.693162in}{1.989217in}}%
\pgfpathlineto{\pgfqpoint{3.697420in}{1.989217in}}%
\pgfpathlineto{\pgfqpoint{3.697420in}{1.984959in}}%
\pgfpathmoveto{\pgfqpoint{3.688904in}{1.989217in}}%
\pgfpathlineto{\pgfqpoint{3.688904in}{1.989217in}}%
\pgfpathlineto{\pgfqpoint{3.688904in}{1.993475in}}%
\pgfpathlineto{\pgfqpoint{3.693162in}{1.993475in}}%
\pgfpathlineto{\pgfqpoint{3.693162in}{1.989217in}}%
\pgfpathmoveto{\pgfqpoint{3.688904in}{1.993475in}}%
\pgfpathlineto{\pgfqpoint{3.688904in}{1.993475in}}%
\pgfpathlineto{\pgfqpoint{3.688904in}{1.997732in}}%
\pgfpathlineto{\pgfqpoint{3.693162in}{1.997732in}}%
\pgfpathlineto{\pgfqpoint{3.693162in}{1.993475in}}%
\pgfpathmoveto{\pgfqpoint{3.693162in}{1.989217in}}%
\pgfpathlineto{\pgfqpoint{3.693162in}{1.989217in}}%
\pgfpathlineto{\pgfqpoint{3.693162in}{1.993475in}}%
\pgfpathlineto{\pgfqpoint{3.697420in}{1.993475in}}%
\pgfpathlineto{\pgfqpoint{3.697420in}{1.989217in}}%
\pgfpathmoveto{\pgfqpoint{3.693162in}{1.993475in}}%
\pgfpathlineto{\pgfqpoint{3.693162in}{1.993475in}}%
\pgfpathlineto{\pgfqpoint{3.693162in}{1.997732in}}%
\pgfpathlineto{\pgfqpoint{3.697420in}{1.997732in}}%
\pgfpathlineto{\pgfqpoint{3.697420in}{1.993475in}}%
\pgfpathmoveto{\pgfqpoint{3.688904in}{1.997732in}}%
\pgfpathlineto{\pgfqpoint{3.688904in}{1.997732in}}%
\pgfpathlineto{\pgfqpoint{3.688904in}{2.001990in}}%
\pgfpathlineto{\pgfqpoint{3.693162in}{2.001990in}}%
\pgfpathlineto{\pgfqpoint{3.693162in}{1.997732in}}%
\pgfpathmoveto{\pgfqpoint{3.688904in}{2.001990in}}%
\pgfpathlineto{\pgfqpoint{3.688904in}{2.001990in}}%
\pgfpathlineto{\pgfqpoint{3.688904in}{2.006248in}}%
\pgfpathlineto{\pgfqpoint{3.693162in}{2.006248in}}%
\pgfpathlineto{\pgfqpoint{3.693162in}{2.001990in}}%
\pgfpathmoveto{\pgfqpoint{3.693162in}{1.997732in}}%
\pgfpathlineto{\pgfqpoint{3.693162in}{1.997732in}}%
\pgfpathlineto{\pgfqpoint{3.693162in}{2.001990in}}%
\pgfpathlineto{\pgfqpoint{3.697420in}{2.001990in}}%
\pgfpathlineto{\pgfqpoint{3.697420in}{1.997732in}}%
\pgfpathmoveto{\pgfqpoint{3.693162in}{2.001990in}}%
\pgfpathlineto{\pgfqpoint{3.693162in}{2.001990in}}%
\pgfpathlineto{\pgfqpoint{3.693162in}{2.006248in}}%
\pgfpathlineto{\pgfqpoint{3.697420in}{2.006248in}}%
\pgfpathlineto{\pgfqpoint{3.697420in}{2.001990in}}%
\pgfpathmoveto{\pgfqpoint{3.697420in}{1.989217in}}%
\pgfpathlineto{\pgfqpoint{3.697420in}{1.989217in}}%
\pgfpathlineto{\pgfqpoint{3.697420in}{1.993475in}}%
\pgfpathlineto{\pgfqpoint{3.701678in}{1.993475in}}%
\pgfpathlineto{\pgfqpoint{3.701678in}{1.989217in}}%
\pgfpathmoveto{\pgfqpoint{3.697420in}{1.993475in}}%
\pgfpathlineto{\pgfqpoint{3.697420in}{1.993475in}}%
\pgfpathlineto{\pgfqpoint{3.697420in}{1.997732in}}%
\pgfpathlineto{\pgfqpoint{3.701678in}{1.997732in}}%
\pgfpathlineto{\pgfqpoint{3.701678in}{1.993475in}}%
\pgfpathmoveto{\pgfqpoint{3.697420in}{1.997732in}}%
\pgfpathlineto{\pgfqpoint{3.697420in}{1.997732in}}%
\pgfpathlineto{\pgfqpoint{3.697420in}{2.001990in}}%
\pgfpathlineto{\pgfqpoint{3.701678in}{2.001990in}}%
\pgfpathlineto{\pgfqpoint{3.701678in}{1.997732in}}%
\pgfpathmoveto{\pgfqpoint{3.697420in}{2.001990in}}%
\pgfpathlineto{\pgfqpoint{3.697420in}{2.001990in}}%
\pgfpathlineto{\pgfqpoint{3.697420in}{2.006248in}}%
\pgfpathlineto{\pgfqpoint{3.701678in}{2.006248in}}%
\pgfpathlineto{\pgfqpoint{3.701678in}{2.001990in}}%
\pgfpathmoveto{\pgfqpoint{3.688904in}{2.006248in}}%
\pgfpathlineto{\pgfqpoint{3.688904in}{2.006248in}}%
\pgfpathlineto{\pgfqpoint{3.688904in}{2.010506in}}%
\pgfpathlineto{\pgfqpoint{3.693162in}{2.010506in}}%
\pgfpathlineto{\pgfqpoint{3.693162in}{2.006248in}}%
\pgfpathmoveto{\pgfqpoint{3.693162in}{2.006248in}}%
\pgfpathlineto{\pgfqpoint{3.693162in}{2.006248in}}%
\pgfpathlineto{\pgfqpoint{3.693162in}{2.010506in}}%
\pgfpathlineto{\pgfqpoint{3.697420in}{2.010506in}}%
\pgfpathlineto{\pgfqpoint{3.697420in}{2.006248in}}%
\pgfpathmoveto{\pgfqpoint{3.693162in}{2.010506in}}%
\pgfpathlineto{\pgfqpoint{3.693162in}{2.010506in}}%
\pgfpathlineto{\pgfqpoint{3.693162in}{2.014764in}}%
\pgfpathlineto{\pgfqpoint{3.697420in}{2.014764in}}%
\pgfpathlineto{\pgfqpoint{3.697420in}{2.010506in}}%
\pgfpathmoveto{\pgfqpoint{3.693162in}{2.014764in}}%
\pgfpathlineto{\pgfqpoint{3.693162in}{2.014764in}}%
\pgfpathlineto{\pgfqpoint{3.693162in}{2.019022in}}%
\pgfpathlineto{\pgfqpoint{3.697420in}{2.019022in}}%
\pgfpathlineto{\pgfqpoint{3.697420in}{2.014764in}}%
\pgfpathmoveto{\pgfqpoint{3.693162in}{2.019022in}}%
\pgfpathlineto{\pgfqpoint{3.693162in}{2.019022in}}%
\pgfpathlineto{\pgfqpoint{3.693162in}{2.023279in}}%
\pgfpathlineto{\pgfqpoint{3.697420in}{2.023279in}}%
\pgfpathlineto{\pgfqpoint{3.697420in}{2.019022in}}%
\pgfpathmoveto{\pgfqpoint{3.697420in}{2.006248in}}%
\pgfpathlineto{\pgfqpoint{3.697420in}{2.006248in}}%
\pgfpathlineto{\pgfqpoint{3.697420in}{2.010506in}}%
\pgfpathlineto{\pgfqpoint{3.701678in}{2.010506in}}%
\pgfpathlineto{\pgfqpoint{3.701678in}{2.006248in}}%
\pgfpathmoveto{\pgfqpoint{3.697420in}{2.010506in}}%
\pgfpathlineto{\pgfqpoint{3.697420in}{2.010506in}}%
\pgfpathlineto{\pgfqpoint{3.697420in}{2.014764in}}%
\pgfpathlineto{\pgfqpoint{3.701678in}{2.014764in}}%
\pgfpathlineto{\pgfqpoint{3.701678in}{2.010506in}}%
\pgfpathmoveto{\pgfqpoint{3.701678in}{2.010506in}}%
\pgfpathlineto{\pgfqpoint{3.701678in}{2.010506in}}%
\pgfpathlineto{\pgfqpoint{3.701678in}{2.014764in}}%
\pgfpathlineto{\pgfqpoint{3.705936in}{2.014764in}}%
\pgfpathlineto{\pgfqpoint{3.705936in}{2.010506in}}%
\pgfpathmoveto{\pgfqpoint{3.697420in}{2.014764in}}%
\pgfpathlineto{\pgfqpoint{3.697420in}{2.014764in}}%
\pgfpathlineto{\pgfqpoint{3.697420in}{2.019022in}}%
\pgfpathlineto{\pgfqpoint{3.701678in}{2.019022in}}%
\pgfpathlineto{\pgfqpoint{3.701678in}{2.014764in}}%
\pgfpathmoveto{\pgfqpoint{3.697420in}{2.019022in}}%
\pgfpathlineto{\pgfqpoint{3.697420in}{2.019022in}}%
\pgfpathlineto{\pgfqpoint{3.697420in}{2.023279in}}%
\pgfpathlineto{\pgfqpoint{3.701678in}{2.023279in}}%
\pgfpathlineto{\pgfqpoint{3.701678in}{2.019022in}}%
\pgfpathmoveto{\pgfqpoint{3.701678in}{2.014764in}}%
\pgfpathlineto{\pgfqpoint{3.701678in}{2.014764in}}%
\pgfpathlineto{\pgfqpoint{3.701678in}{2.019022in}}%
\pgfpathlineto{\pgfqpoint{3.705936in}{2.019022in}}%
\pgfpathlineto{\pgfqpoint{3.705936in}{2.014764in}}%
\pgfpathmoveto{\pgfqpoint{3.701678in}{2.019022in}}%
\pgfpathlineto{\pgfqpoint{3.701678in}{2.019022in}}%
\pgfpathlineto{\pgfqpoint{3.701678in}{2.023279in}}%
\pgfpathlineto{\pgfqpoint{3.705936in}{2.023279in}}%
\pgfpathlineto{\pgfqpoint{3.705936in}{2.019022in}}%
\pgfpathmoveto{\pgfqpoint{3.693162in}{2.023279in}}%
\pgfpathlineto{\pgfqpoint{3.693162in}{2.023279in}}%
\pgfpathlineto{\pgfqpoint{3.693162in}{2.027537in}}%
\pgfpathlineto{\pgfqpoint{3.697420in}{2.027537in}}%
\pgfpathlineto{\pgfqpoint{3.697420in}{2.023279in}}%
\pgfpathmoveto{\pgfqpoint{3.697420in}{2.023279in}}%
\pgfpathlineto{\pgfqpoint{3.697420in}{2.023279in}}%
\pgfpathlineto{\pgfqpoint{3.697420in}{2.027537in}}%
\pgfpathlineto{\pgfqpoint{3.701678in}{2.027537in}}%
\pgfpathlineto{\pgfqpoint{3.701678in}{2.023279in}}%
\pgfpathmoveto{\pgfqpoint{3.697420in}{2.027537in}}%
\pgfpathlineto{\pgfqpoint{3.697420in}{2.027537in}}%
\pgfpathlineto{\pgfqpoint{3.697420in}{2.031795in}}%
\pgfpathlineto{\pgfqpoint{3.701678in}{2.031795in}}%
\pgfpathlineto{\pgfqpoint{3.701678in}{2.027537in}}%
\pgfpathmoveto{\pgfqpoint{3.701678in}{2.023279in}}%
\pgfpathlineto{\pgfqpoint{3.701678in}{2.023279in}}%
\pgfpathlineto{\pgfqpoint{3.701678in}{2.027537in}}%
\pgfpathlineto{\pgfqpoint{3.705936in}{2.027537in}}%
\pgfpathlineto{\pgfqpoint{3.705936in}{2.023279in}}%
\pgfpathmoveto{\pgfqpoint{3.701678in}{2.027537in}}%
\pgfpathlineto{\pgfqpoint{3.701678in}{2.027537in}}%
\pgfpathlineto{\pgfqpoint{3.701678in}{2.031795in}}%
\pgfpathlineto{\pgfqpoint{3.705936in}{2.031795in}}%
\pgfpathlineto{\pgfqpoint{3.705936in}{2.027537in}}%
\pgfpathmoveto{\pgfqpoint{3.697420in}{2.031795in}}%
\pgfpathlineto{\pgfqpoint{3.697420in}{2.031795in}}%
\pgfpathlineto{\pgfqpoint{3.697420in}{2.036053in}}%
\pgfpathlineto{\pgfqpoint{3.701678in}{2.036053in}}%
\pgfpathlineto{\pgfqpoint{3.701678in}{2.031795in}}%
\pgfpathmoveto{\pgfqpoint{3.697420in}{2.036053in}}%
\pgfpathlineto{\pgfqpoint{3.697420in}{2.036053in}}%
\pgfpathlineto{\pgfqpoint{3.697420in}{2.040311in}}%
\pgfpathlineto{\pgfqpoint{3.701678in}{2.040311in}}%
\pgfpathlineto{\pgfqpoint{3.701678in}{2.036053in}}%
\pgfpathmoveto{\pgfqpoint{3.701678in}{2.031795in}}%
\pgfpathlineto{\pgfqpoint{3.701678in}{2.031795in}}%
\pgfpathlineto{\pgfqpoint{3.701678in}{2.036053in}}%
\pgfpathlineto{\pgfqpoint{3.705936in}{2.036053in}}%
\pgfpathlineto{\pgfqpoint{3.705936in}{2.031795in}}%
\pgfpathmoveto{\pgfqpoint{3.701678in}{2.036053in}}%
\pgfpathlineto{\pgfqpoint{3.701678in}{2.036053in}}%
\pgfpathlineto{\pgfqpoint{3.701678in}{2.040311in}}%
\pgfpathlineto{\pgfqpoint{3.705936in}{2.040311in}}%
\pgfpathlineto{\pgfqpoint{3.705936in}{2.036053in}}%
\pgfpathmoveto{\pgfqpoint{3.697420in}{2.040311in}}%
\pgfpathlineto{\pgfqpoint{3.697420in}{2.040311in}}%
\pgfpathlineto{\pgfqpoint{3.697420in}{2.044569in}}%
\pgfpathlineto{\pgfqpoint{3.701678in}{2.044569in}}%
\pgfpathlineto{\pgfqpoint{3.701678in}{2.040311in}}%
\pgfpathmoveto{\pgfqpoint{3.697420in}{2.044569in}}%
\pgfpathlineto{\pgfqpoint{3.697420in}{2.044569in}}%
\pgfpathlineto{\pgfqpoint{3.697420in}{2.048827in}}%
\pgfpathlineto{\pgfqpoint{3.701678in}{2.048827in}}%
\pgfpathlineto{\pgfqpoint{3.701678in}{2.044569in}}%
\pgfpathmoveto{\pgfqpoint{3.701678in}{2.040311in}}%
\pgfpathlineto{\pgfqpoint{3.701678in}{2.040311in}}%
\pgfpathlineto{\pgfqpoint{3.701678in}{2.044569in}}%
\pgfpathlineto{\pgfqpoint{3.705936in}{2.044569in}}%
\pgfpathlineto{\pgfqpoint{3.705936in}{2.040311in}}%
\pgfpathmoveto{\pgfqpoint{3.701678in}{2.044569in}}%
\pgfpathlineto{\pgfqpoint{3.701678in}{2.044569in}}%
\pgfpathlineto{\pgfqpoint{3.701678in}{2.048827in}}%
\pgfpathlineto{\pgfqpoint{3.705936in}{2.048827in}}%
\pgfpathlineto{\pgfqpoint{3.705936in}{2.044569in}}%
\pgfpathmoveto{\pgfqpoint{3.701678in}{2.048827in}}%
\pgfpathlineto{\pgfqpoint{3.701678in}{2.048827in}}%
\pgfpathlineto{\pgfqpoint{3.701678in}{2.053084in}}%
\pgfpathlineto{\pgfqpoint{3.705936in}{2.053084in}}%
\pgfpathlineto{\pgfqpoint{3.705936in}{2.048827in}}%
\pgfpathmoveto{\pgfqpoint{3.701678in}{2.053084in}}%
\pgfpathlineto{\pgfqpoint{3.701678in}{2.053084in}}%
\pgfpathlineto{\pgfqpoint{3.701678in}{2.057342in}}%
\pgfpathlineto{\pgfqpoint{3.705936in}{2.057342in}}%
\pgfpathlineto{\pgfqpoint{3.705936in}{2.053084in}}%
\pgfpathmoveto{\pgfqpoint{3.701678in}{2.057342in}}%
\pgfpathlineto{\pgfqpoint{3.701678in}{2.057342in}}%
\pgfpathlineto{\pgfqpoint{3.701678in}{2.061600in}}%
\pgfpathlineto{\pgfqpoint{3.705936in}{2.061600in}}%
\pgfpathlineto{\pgfqpoint{3.705936in}{2.057342in}}%
\pgfpathmoveto{\pgfqpoint{3.701678in}{2.061600in}}%
\pgfpathlineto{\pgfqpoint{3.701678in}{2.061600in}}%
\pgfpathlineto{\pgfqpoint{3.701678in}{2.065858in}}%
\pgfpathlineto{\pgfqpoint{3.705936in}{2.065858in}}%
\pgfpathlineto{\pgfqpoint{3.705936in}{2.061600in}}%
\pgfpathmoveto{\pgfqpoint{3.705936in}{2.027537in}}%
\pgfpathlineto{\pgfqpoint{3.705936in}{2.027537in}}%
\pgfpathlineto{\pgfqpoint{3.705936in}{2.031795in}}%
\pgfpathlineto{\pgfqpoint{3.710194in}{2.031795in}}%
\pgfpathlineto{\pgfqpoint{3.710194in}{2.027537in}}%
\pgfpathmoveto{\pgfqpoint{3.705936in}{2.031795in}}%
\pgfpathlineto{\pgfqpoint{3.705936in}{2.031795in}}%
\pgfpathlineto{\pgfqpoint{3.705936in}{2.036053in}}%
\pgfpathlineto{\pgfqpoint{3.710194in}{2.036053in}}%
\pgfpathlineto{\pgfqpoint{3.710194in}{2.031795in}}%
\pgfpathmoveto{\pgfqpoint{3.705936in}{2.036053in}}%
\pgfpathlineto{\pgfqpoint{3.705936in}{2.036053in}}%
\pgfpathlineto{\pgfqpoint{3.705936in}{2.040311in}}%
\pgfpathlineto{\pgfqpoint{3.710194in}{2.040311in}}%
\pgfpathlineto{\pgfqpoint{3.710194in}{2.036053in}}%
\pgfpathmoveto{\pgfqpoint{3.705936in}{2.040311in}}%
\pgfpathlineto{\pgfqpoint{3.705936in}{2.040311in}}%
\pgfpathlineto{\pgfqpoint{3.705936in}{2.044569in}}%
\pgfpathlineto{\pgfqpoint{3.710194in}{2.044569in}}%
\pgfpathlineto{\pgfqpoint{3.710194in}{2.040311in}}%
\pgfpathmoveto{\pgfqpoint{3.705936in}{2.044569in}}%
\pgfpathlineto{\pgfqpoint{3.705936in}{2.044569in}}%
\pgfpathlineto{\pgfqpoint{3.705936in}{2.048827in}}%
\pgfpathlineto{\pgfqpoint{3.710194in}{2.048827in}}%
\pgfpathlineto{\pgfqpoint{3.710194in}{2.044569in}}%
\pgfpathmoveto{\pgfqpoint{3.705936in}{2.048827in}}%
\pgfpathlineto{\pgfqpoint{3.705936in}{2.048827in}}%
\pgfpathlineto{\pgfqpoint{3.705936in}{2.053084in}}%
\pgfpathlineto{\pgfqpoint{3.710194in}{2.053084in}}%
\pgfpathlineto{\pgfqpoint{3.710194in}{2.048827in}}%
\pgfpathmoveto{\pgfqpoint{3.705936in}{2.053084in}}%
\pgfpathlineto{\pgfqpoint{3.705936in}{2.053084in}}%
\pgfpathlineto{\pgfqpoint{3.705936in}{2.057342in}}%
\pgfpathlineto{\pgfqpoint{3.710194in}{2.057342in}}%
\pgfpathlineto{\pgfqpoint{3.710194in}{2.053084in}}%
\pgfpathmoveto{\pgfqpoint{3.710194in}{2.048827in}}%
\pgfpathlineto{\pgfqpoint{3.710194in}{2.048827in}}%
\pgfpathlineto{\pgfqpoint{3.710194in}{2.053084in}}%
\pgfpathlineto{\pgfqpoint{3.714452in}{2.053084in}}%
\pgfpathlineto{\pgfqpoint{3.714452in}{2.048827in}}%
\pgfpathmoveto{\pgfqpoint{3.710194in}{2.053084in}}%
\pgfpathlineto{\pgfqpoint{3.710194in}{2.053084in}}%
\pgfpathlineto{\pgfqpoint{3.710194in}{2.057342in}}%
\pgfpathlineto{\pgfqpoint{3.714452in}{2.057342in}}%
\pgfpathlineto{\pgfqpoint{3.714452in}{2.053084in}}%
\pgfpathmoveto{\pgfqpoint{3.705936in}{2.057342in}}%
\pgfpathlineto{\pgfqpoint{3.705936in}{2.057342in}}%
\pgfpathlineto{\pgfqpoint{3.705936in}{2.061600in}}%
\pgfpathlineto{\pgfqpoint{3.710194in}{2.061600in}}%
\pgfpathlineto{\pgfqpoint{3.710194in}{2.057342in}}%
\pgfpathmoveto{\pgfqpoint{3.705936in}{2.061600in}}%
\pgfpathlineto{\pgfqpoint{3.705936in}{2.061600in}}%
\pgfpathlineto{\pgfqpoint{3.705936in}{2.065858in}}%
\pgfpathlineto{\pgfqpoint{3.710194in}{2.065858in}}%
\pgfpathlineto{\pgfqpoint{3.710194in}{2.061600in}}%
\pgfpathmoveto{\pgfqpoint{3.710194in}{2.057342in}}%
\pgfpathlineto{\pgfqpoint{3.710194in}{2.057342in}}%
\pgfpathlineto{\pgfqpoint{3.710194in}{2.061600in}}%
\pgfpathlineto{\pgfqpoint{3.714452in}{2.061600in}}%
\pgfpathlineto{\pgfqpoint{3.714452in}{2.057342in}}%
\pgfpathmoveto{\pgfqpoint{3.710194in}{2.061600in}}%
\pgfpathlineto{\pgfqpoint{3.710194in}{2.061600in}}%
\pgfpathlineto{\pgfqpoint{3.710194in}{2.065858in}}%
\pgfpathlineto{\pgfqpoint{3.714452in}{2.065858in}}%
\pgfpathlineto{\pgfqpoint{3.714452in}{2.061600in}}%
\pgfpathmoveto{\pgfqpoint{3.705936in}{2.065858in}}%
\pgfpathlineto{\pgfqpoint{3.705936in}{2.065858in}}%
\pgfpathlineto{\pgfqpoint{3.705936in}{2.070116in}}%
\pgfpathlineto{\pgfqpoint{3.710194in}{2.070116in}}%
\pgfpathlineto{\pgfqpoint{3.710194in}{2.065858in}}%
\pgfpathmoveto{\pgfqpoint{3.705936in}{2.070116in}}%
\pgfpathlineto{\pgfqpoint{3.705936in}{2.070116in}}%
\pgfpathlineto{\pgfqpoint{3.705936in}{2.074374in}}%
\pgfpathlineto{\pgfqpoint{3.710194in}{2.074374in}}%
\pgfpathlineto{\pgfqpoint{3.710194in}{2.070116in}}%
\pgfpathmoveto{\pgfqpoint{3.710194in}{2.065858in}}%
\pgfpathlineto{\pgfqpoint{3.710194in}{2.065858in}}%
\pgfpathlineto{\pgfqpoint{3.710194in}{2.070116in}}%
\pgfpathlineto{\pgfqpoint{3.714452in}{2.070116in}}%
\pgfpathlineto{\pgfqpoint{3.714452in}{2.065858in}}%
\pgfpathmoveto{\pgfqpoint{3.710194in}{2.070116in}}%
\pgfpathlineto{\pgfqpoint{3.710194in}{2.070116in}}%
\pgfpathlineto{\pgfqpoint{3.710194in}{2.074374in}}%
\pgfpathlineto{\pgfqpoint{3.714452in}{2.074374in}}%
\pgfpathlineto{\pgfqpoint{3.714452in}{2.070116in}}%
\pgfpathmoveto{\pgfqpoint{3.714452in}{2.070116in}}%
\pgfpathlineto{\pgfqpoint{3.714452in}{2.070116in}}%
\pgfpathlineto{\pgfqpoint{3.714452in}{2.074374in}}%
\pgfpathlineto{\pgfqpoint{3.718709in}{2.074374in}}%
\pgfpathlineto{\pgfqpoint{3.718709in}{2.070116in}}%
\pgfpathmoveto{\pgfqpoint{3.705936in}{2.074374in}}%
\pgfpathlineto{\pgfqpoint{3.705936in}{2.074374in}}%
\pgfpathlineto{\pgfqpoint{3.705936in}{2.078631in}}%
\pgfpathlineto{\pgfqpoint{3.710194in}{2.078631in}}%
\pgfpathlineto{\pgfqpoint{3.710194in}{2.074374in}}%
\pgfpathmoveto{\pgfqpoint{3.705936in}{2.078631in}}%
\pgfpathlineto{\pgfqpoint{3.705936in}{2.078631in}}%
\pgfpathlineto{\pgfqpoint{3.705936in}{2.082889in}}%
\pgfpathlineto{\pgfqpoint{3.710194in}{2.082889in}}%
\pgfpathlineto{\pgfqpoint{3.710194in}{2.078631in}}%
\pgfpathmoveto{\pgfqpoint{3.710194in}{2.074374in}}%
\pgfpathlineto{\pgfqpoint{3.710194in}{2.074374in}}%
\pgfpathlineto{\pgfqpoint{3.710194in}{2.078631in}}%
\pgfpathlineto{\pgfqpoint{3.714452in}{2.078631in}}%
\pgfpathlineto{\pgfqpoint{3.714452in}{2.074374in}}%
\pgfpathmoveto{\pgfqpoint{3.710194in}{2.078631in}}%
\pgfpathlineto{\pgfqpoint{3.710194in}{2.078631in}}%
\pgfpathlineto{\pgfqpoint{3.710194in}{2.082889in}}%
\pgfpathlineto{\pgfqpoint{3.714452in}{2.082889in}}%
\pgfpathlineto{\pgfqpoint{3.714452in}{2.078631in}}%
\pgfpathmoveto{\pgfqpoint{3.705936in}{2.082889in}}%
\pgfpathlineto{\pgfqpoint{3.705936in}{2.082889in}}%
\pgfpathlineto{\pgfqpoint{3.705936in}{2.087147in}}%
\pgfpathlineto{\pgfqpoint{3.710194in}{2.087147in}}%
\pgfpathlineto{\pgfqpoint{3.710194in}{2.082889in}}%
\pgfpathmoveto{\pgfqpoint{3.710194in}{2.082889in}}%
\pgfpathlineto{\pgfqpoint{3.710194in}{2.082889in}}%
\pgfpathlineto{\pgfqpoint{3.710194in}{2.087147in}}%
\pgfpathlineto{\pgfqpoint{3.714452in}{2.087147in}}%
\pgfpathlineto{\pgfqpoint{3.714452in}{2.082889in}}%
\pgfpathmoveto{\pgfqpoint{3.710194in}{2.087147in}}%
\pgfpathlineto{\pgfqpoint{3.710194in}{2.087147in}}%
\pgfpathlineto{\pgfqpoint{3.710194in}{2.091405in}}%
\pgfpathlineto{\pgfqpoint{3.714452in}{2.091405in}}%
\pgfpathlineto{\pgfqpoint{3.714452in}{2.087147in}}%
\pgfpathmoveto{\pgfqpoint{3.714452in}{2.074374in}}%
\pgfpathlineto{\pgfqpoint{3.714452in}{2.074374in}}%
\pgfpathlineto{\pgfqpoint{3.714452in}{2.078631in}}%
\pgfpathlineto{\pgfqpoint{3.718709in}{2.078631in}}%
\pgfpathlineto{\pgfqpoint{3.718709in}{2.074374in}}%
\pgfpathmoveto{\pgfqpoint{3.714452in}{2.078631in}}%
\pgfpathlineto{\pgfqpoint{3.714452in}{2.078631in}}%
\pgfpathlineto{\pgfqpoint{3.714452in}{2.082889in}}%
\pgfpathlineto{\pgfqpoint{3.718709in}{2.082889in}}%
\pgfpathlineto{\pgfqpoint{3.718709in}{2.078631in}}%
\pgfpathmoveto{\pgfqpoint{3.714452in}{2.082889in}}%
\pgfpathlineto{\pgfqpoint{3.714452in}{2.082889in}}%
\pgfpathlineto{\pgfqpoint{3.714452in}{2.087147in}}%
\pgfpathlineto{\pgfqpoint{3.718709in}{2.087147in}}%
\pgfpathlineto{\pgfqpoint{3.718709in}{2.082889in}}%
\pgfpathmoveto{\pgfqpoint{3.714452in}{2.087147in}}%
\pgfpathlineto{\pgfqpoint{3.714452in}{2.087147in}}%
\pgfpathlineto{\pgfqpoint{3.714452in}{2.091405in}}%
\pgfpathlineto{\pgfqpoint{3.718709in}{2.091405in}}%
\pgfpathlineto{\pgfqpoint{3.718709in}{2.087147in}}%
\pgfpathmoveto{\pgfqpoint{3.710194in}{2.091405in}}%
\pgfpathlineto{\pgfqpoint{3.710194in}{2.091405in}}%
\pgfpathlineto{\pgfqpoint{3.710194in}{2.095663in}}%
\pgfpathlineto{\pgfqpoint{3.714452in}{2.095663in}}%
\pgfpathlineto{\pgfqpoint{3.714452in}{2.091405in}}%
\pgfpathmoveto{\pgfqpoint{3.710194in}{2.095663in}}%
\pgfpathlineto{\pgfqpoint{3.710194in}{2.095663in}}%
\pgfpathlineto{\pgfqpoint{3.710194in}{2.099921in}}%
\pgfpathlineto{\pgfqpoint{3.714452in}{2.099921in}}%
\pgfpathlineto{\pgfqpoint{3.714452in}{2.095663in}}%
\pgfpathmoveto{\pgfqpoint{3.710194in}{2.099921in}}%
\pgfpathlineto{\pgfqpoint{3.710194in}{2.099921in}}%
\pgfpathlineto{\pgfqpoint{3.710194in}{2.104179in}}%
\pgfpathlineto{\pgfqpoint{3.714452in}{2.104179in}}%
\pgfpathlineto{\pgfqpoint{3.714452in}{2.099921in}}%
\pgfpathmoveto{\pgfqpoint{3.710194in}{2.104179in}}%
\pgfpathlineto{\pgfqpoint{3.710194in}{2.104179in}}%
\pgfpathlineto{\pgfqpoint{3.710194in}{2.108436in}}%
\pgfpathlineto{\pgfqpoint{3.714452in}{2.108436in}}%
\pgfpathlineto{\pgfqpoint{3.714452in}{2.104179in}}%
\pgfpathmoveto{\pgfqpoint{3.714452in}{2.091405in}}%
\pgfpathlineto{\pgfqpoint{3.714452in}{2.091405in}}%
\pgfpathlineto{\pgfqpoint{3.714452in}{2.095663in}}%
\pgfpathlineto{\pgfqpoint{3.718709in}{2.095663in}}%
\pgfpathlineto{\pgfqpoint{3.718709in}{2.091405in}}%
\pgfpathmoveto{\pgfqpoint{3.714452in}{2.095663in}}%
\pgfpathlineto{\pgfqpoint{3.714452in}{2.095663in}}%
\pgfpathlineto{\pgfqpoint{3.714452in}{2.099921in}}%
\pgfpathlineto{\pgfqpoint{3.718709in}{2.099921in}}%
\pgfpathlineto{\pgfqpoint{3.718709in}{2.095663in}}%
\pgfpathmoveto{\pgfqpoint{3.718709in}{2.091405in}}%
\pgfpathlineto{\pgfqpoint{3.718709in}{2.091405in}}%
\pgfpathlineto{\pgfqpoint{3.718709in}{2.095663in}}%
\pgfpathlineto{\pgfqpoint{3.722967in}{2.095663in}}%
\pgfpathlineto{\pgfqpoint{3.722967in}{2.091405in}}%
\pgfpathmoveto{\pgfqpoint{3.718709in}{2.095663in}}%
\pgfpathlineto{\pgfqpoint{3.718709in}{2.095663in}}%
\pgfpathlineto{\pgfqpoint{3.718709in}{2.099921in}}%
\pgfpathlineto{\pgfqpoint{3.722967in}{2.099921in}}%
\pgfpathlineto{\pgfqpoint{3.722967in}{2.095663in}}%
\pgfpathmoveto{\pgfqpoint{3.714452in}{2.099921in}}%
\pgfpathlineto{\pgfqpoint{3.714452in}{2.099921in}}%
\pgfpathlineto{\pgfqpoint{3.714452in}{2.104179in}}%
\pgfpathlineto{\pgfqpoint{3.718709in}{2.104179in}}%
\pgfpathlineto{\pgfqpoint{3.718709in}{2.099921in}}%
\pgfpathmoveto{\pgfqpoint{3.714452in}{2.104179in}}%
\pgfpathlineto{\pgfqpoint{3.714452in}{2.104179in}}%
\pgfpathlineto{\pgfqpoint{3.714452in}{2.108436in}}%
\pgfpathlineto{\pgfqpoint{3.718709in}{2.108436in}}%
\pgfpathlineto{\pgfqpoint{3.718709in}{2.104179in}}%
\pgfpathmoveto{\pgfqpoint{3.718709in}{2.099921in}}%
\pgfpathlineto{\pgfqpoint{3.718709in}{2.099921in}}%
\pgfpathlineto{\pgfqpoint{3.718709in}{2.104179in}}%
\pgfpathlineto{\pgfqpoint{3.722967in}{2.104179in}}%
\pgfpathlineto{\pgfqpoint{3.722967in}{2.099921in}}%
\pgfpathmoveto{\pgfqpoint{3.718709in}{2.104179in}}%
\pgfpathlineto{\pgfqpoint{3.718709in}{2.104179in}}%
\pgfpathlineto{\pgfqpoint{3.718709in}{2.108436in}}%
\pgfpathlineto{\pgfqpoint{3.722967in}{2.108436in}}%
\pgfpathlineto{\pgfqpoint{3.722967in}{2.104179in}}%
\pgfpathmoveto{\pgfqpoint{3.714452in}{2.108436in}}%
\pgfpathlineto{\pgfqpoint{3.714452in}{2.108436in}}%
\pgfpathlineto{\pgfqpoint{3.714452in}{2.112694in}}%
\pgfpathlineto{\pgfqpoint{3.718709in}{2.112694in}}%
\pgfpathlineto{\pgfqpoint{3.718709in}{2.108436in}}%
\pgfpathmoveto{\pgfqpoint{3.714452in}{2.112694in}}%
\pgfpathlineto{\pgfqpoint{3.714452in}{2.112694in}}%
\pgfpathlineto{\pgfqpoint{3.714452in}{2.116952in}}%
\pgfpathlineto{\pgfqpoint{3.718709in}{2.116952in}}%
\pgfpathlineto{\pgfqpoint{3.718709in}{2.112694in}}%
\pgfpathmoveto{\pgfqpoint{3.718709in}{2.108436in}}%
\pgfpathlineto{\pgfqpoint{3.718709in}{2.108436in}}%
\pgfpathlineto{\pgfqpoint{3.718709in}{2.112694in}}%
\pgfpathlineto{\pgfqpoint{3.722967in}{2.112694in}}%
\pgfpathlineto{\pgfqpoint{3.722967in}{2.108436in}}%
\pgfpathmoveto{\pgfqpoint{3.718709in}{2.112694in}}%
\pgfpathlineto{\pgfqpoint{3.718709in}{2.112694in}}%
\pgfpathlineto{\pgfqpoint{3.718709in}{2.116952in}}%
\pgfpathlineto{\pgfqpoint{3.722967in}{2.116952in}}%
\pgfpathlineto{\pgfqpoint{3.722967in}{2.112694in}}%
\pgfpathmoveto{\pgfqpoint{3.714452in}{2.116952in}}%
\pgfpathlineto{\pgfqpoint{3.714452in}{2.116952in}}%
\pgfpathlineto{\pgfqpoint{3.714452in}{2.121210in}}%
\pgfpathlineto{\pgfqpoint{3.718709in}{2.121210in}}%
\pgfpathlineto{\pgfqpoint{3.718709in}{2.116952in}}%
\pgfpathmoveto{\pgfqpoint{3.714452in}{2.121210in}}%
\pgfpathlineto{\pgfqpoint{3.714452in}{2.121210in}}%
\pgfpathlineto{\pgfqpoint{3.714452in}{2.125468in}}%
\pgfpathlineto{\pgfqpoint{3.718709in}{2.125468in}}%
\pgfpathlineto{\pgfqpoint{3.718709in}{2.121210in}}%
\pgfpathmoveto{\pgfqpoint{3.718709in}{2.116952in}}%
\pgfpathlineto{\pgfqpoint{3.718709in}{2.116952in}}%
\pgfpathlineto{\pgfqpoint{3.718709in}{2.121210in}}%
\pgfpathlineto{\pgfqpoint{3.722967in}{2.121210in}}%
\pgfpathlineto{\pgfqpoint{3.722967in}{2.116952in}}%
\pgfpathmoveto{\pgfqpoint{3.718709in}{2.121210in}}%
\pgfpathlineto{\pgfqpoint{3.718709in}{2.121210in}}%
\pgfpathlineto{\pgfqpoint{3.718709in}{2.125468in}}%
\pgfpathlineto{\pgfqpoint{3.722967in}{2.125468in}}%
\pgfpathlineto{\pgfqpoint{3.722967in}{2.121210in}}%
\pgfpathmoveto{\pgfqpoint{3.714452in}{2.125468in}}%
\pgfpathlineto{\pgfqpoint{3.714452in}{2.125468in}}%
\pgfpathlineto{\pgfqpoint{3.714452in}{2.129726in}}%
\pgfpathlineto{\pgfqpoint{3.718709in}{2.129726in}}%
\pgfpathlineto{\pgfqpoint{3.718709in}{2.125468in}}%
\pgfpathmoveto{\pgfqpoint{3.718709in}{2.125468in}}%
\pgfpathlineto{\pgfqpoint{3.718709in}{2.125468in}}%
\pgfpathlineto{\pgfqpoint{3.718709in}{2.129726in}}%
\pgfpathlineto{\pgfqpoint{3.722967in}{2.129726in}}%
\pgfpathlineto{\pgfqpoint{3.722967in}{2.125468in}}%
\pgfpathmoveto{\pgfqpoint{3.718709in}{2.129726in}}%
\pgfpathlineto{\pgfqpoint{3.718709in}{2.129726in}}%
\pgfpathlineto{\pgfqpoint{3.718709in}{2.133983in}}%
\pgfpathlineto{\pgfqpoint{3.722967in}{2.133983in}}%
\pgfpathlineto{\pgfqpoint{3.722967in}{2.129726in}}%
\pgfpathmoveto{\pgfqpoint{3.718709in}{2.133983in}}%
\pgfpathlineto{\pgfqpoint{3.718709in}{2.133983in}}%
\pgfpathlineto{\pgfqpoint{3.718709in}{2.138241in}}%
\pgfpathlineto{\pgfqpoint{3.722967in}{2.138241in}}%
\pgfpathlineto{\pgfqpoint{3.722967in}{2.133983in}}%
\pgfpathmoveto{\pgfqpoint{3.718709in}{2.138241in}}%
\pgfpathlineto{\pgfqpoint{3.718709in}{2.138241in}}%
\pgfpathlineto{\pgfqpoint{3.718709in}{2.142499in}}%
\pgfpathlineto{\pgfqpoint{3.722967in}{2.142499in}}%
\pgfpathlineto{\pgfqpoint{3.722967in}{2.138241in}}%
\pgfpathmoveto{\pgfqpoint{3.722967in}{2.112694in}}%
\pgfpathlineto{\pgfqpoint{3.722967in}{2.112694in}}%
\pgfpathlineto{\pgfqpoint{3.722967in}{2.116952in}}%
\pgfpathlineto{\pgfqpoint{3.727225in}{2.116952in}}%
\pgfpathlineto{\pgfqpoint{3.727225in}{2.112694in}}%
\pgfpathmoveto{\pgfqpoint{3.722967in}{2.116952in}}%
\pgfpathlineto{\pgfqpoint{3.722967in}{2.116952in}}%
\pgfpathlineto{\pgfqpoint{3.722967in}{2.121210in}}%
\pgfpathlineto{\pgfqpoint{3.727225in}{2.121210in}}%
\pgfpathlineto{\pgfqpoint{3.727225in}{2.116952in}}%
\pgfpathmoveto{\pgfqpoint{3.722967in}{2.121210in}}%
\pgfpathlineto{\pgfqpoint{3.722967in}{2.121210in}}%
\pgfpathlineto{\pgfqpoint{3.722967in}{2.125468in}}%
\pgfpathlineto{\pgfqpoint{3.727225in}{2.125468in}}%
\pgfpathlineto{\pgfqpoint{3.727225in}{2.121210in}}%
\pgfpathmoveto{\pgfqpoint{3.722967in}{2.125468in}}%
\pgfpathlineto{\pgfqpoint{3.722967in}{2.125468in}}%
\pgfpathlineto{\pgfqpoint{3.722967in}{2.129726in}}%
\pgfpathlineto{\pgfqpoint{3.727225in}{2.129726in}}%
\pgfpathlineto{\pgfqpoint{3.727225in}{2.125468in}}%
\pgfpathmoveto{\pgfqpoint{3.722967in}{2.129726in}}%
\pgfpathlineto{\pgfqpoint{3.722967in}{2.129726in}}%
\pgfpathlineto{\pgfqpoint{3.722967in}{2.133983in}}%
\pgfpathlineto{\pgfqpoint{3.727225in}{2.133983in}}%
\pgfpathlineto{\pgfqpoint{3.727225in}{2.129726in}}%
\pgfpathmoveto{\pgfqpoint{3.722967in}{2.133983in}}%
\pgfpathlineto{\pgfqpoint{3.722967in}{2.133983in}}%
\pgfpathlineto{\pgfqpoint{3.722967in}{2.138241in}}%
\pgfpathlineto{\pgfqpoint{3.727225in}{2.138241in}}%
\pgfpathlineto{\pgfqpoint{3.727225in}{2.133983in}}%
\pgfpathmoveto{\pgfqpoint{3.722967in}{2.138241in}}%
\pgfpathlineto{\pgfqpoint{3.722967in}{2.138241in}}%
\pgfpathlineto{\pgfqpoint{3.722967in}{2.142499in}}%
\pgfpathlineto{\pgfqpoint{3.727225in}{2.142499in}}%
\pgfpathlineto{\pgfqpoint{3.727225in}{2.138241in}}%
\pgfpathmoveto{\pgfqpoint{3.727225in}{2.133983in}}%
\pgfpathlineto{\pgfqpoint{3.727225in}{2.133983in}}%
\pgfpathlineto{\pgfqpoint{3.727225in}{2.138241in}}%
\pgfpathlineto{\pgfqpoint{3.731483in}{2.138241in}}%
\pgfpathlineto{\pgfqpoint{3.731483in}{2.133983in}}%
\pgfpathmoveto{\pgfqpoint{3.727225in}{2.138241in}}%
\pgfpathlineto{\pgfqpoint{3.727225in}{2.138241in}}%
\pgfpathlineto{\pgfqpoint{3.727225in}{2.142499in}}%
\pgfpathlineto{\pgfqpoint{3.731483in}{2.142499in}}%
\pgfpathlineto{\pgfqpoint{3.731483in}{2.138241in}}%
\pgfpathmoveto{\pgfqpoint{3.718709in}{2.142499in}}%
\pgfpathlineto{\pgfqpoint{3.718709in}{2.142499in}}%
\pgfpathlineto{\pgfqpoint{3.718709in}{2.146757in}}%
\pgfpathlineto{\pgfqpoint{3.722967in}{2.146757in}}%
\pgfpathlineto{\pgfqpoint{3.722967in}{2.142499in}}%
\pgfpathmoveto{\pgfqpoint{3.718709in}{2.146757in}}%
\pgfpathlineto{\pgfqpoint{3.718709in}{2.146757in}}%
\pgfpathlineto{\pgfqpoint{3.718709in}{2.151015in}}%
\pgfpathlineto{\pgfqpoint{3.722967in}{2.151015in}}%
\pgfpathlineto{\pgfqpoint{3.722967in}{2.146757in}}%
\pgfpathmoveto{\pgfqpoint{3.722967in}{2.142499in}}%
\pgfpathlineto{\pgfqpoint{3.722967in}{2.142499in}}%
\pgfpathlineto{\pgfqpoint{3.722967in}{2.146757in}}%
\pgfpathlineto{\pgfqpoint{3.727225in}{2.146757in}}%
\pgfpathlineto{\pgfqpoint{3.727225in}{2.142499in}}%
\pgfpathmoveto{\pgfqpoint{3.722967in}{2.146757in}}%
\pgfpathlineto{\pgfqpoint{3.722967in}{2.146757in}}%
\pgfpathlineto{\pgfqpoint{3.722967in}{2.151015in}}%
\pgfpathlineto{\pgfqpoint{3.727225in}{2.151015in}}%
\pgfpathlineto{\pgfqpoint{3.727225in}{2.146757in}}%
\pgfpathmoveto{\pgfqpoint{3.727225in}{2.142499in}}%
\pgfpathlineto{\pgfqpoint{3.727225in}{2.142499in}}%
\pgfpathlineto{\pgfqpoint{3.727225in}{2.146757in}}%
\pgfpathlineto{\pgfqpoint{3.731483in}{2.146757in}}%
\pgfpathlineto{\pgfqpoint{3.731483in}{2.142499in}}%
\pgfpathmoveto{\pgfqpoint{3.727225in}{2.146757in}}%
\pgfpathlineto{\pgfqpoint{3.727225in}{2.146757in}}%
\pgfpathlineto{\pgfqpoint{3.727225in}{2.151015in}}%
\pgfpathlineto{\pgfqpoint{3.731483in}{2.151015in}}%
\pgfpathlineto{\pgfqpoint{3.731483in}{2.146757in}}%
\pgfpathmoveto{\pgfqpoint{3.722967in}{2.151015in}}%
\pgfpathlineto{\pgfqpoint{3.722967in}{2.151015in}}%
\pgfpathlineto{\pgfqpoint{3.722967in}{2.155272in}}%
\pgfpathlineto{\pgfqpoint{3.727225in}{2.155272in}}%
\pgfpathlineto{\pgfqpoint{3.727225in}{2.151015in}}%
\pgfpathmoveto{\pgfqpoint{3.722967in}{2.155272in}}%
\pgfpathlineto{\pgfqpoint{3.722967in}{2.155272in}}%
\pgfpathlineto{\pgfqpoint{3.722967in}{2.159530in}}%
\pgfpathlineto{\pgfqpoint{3.727225in}{2.159530in}}%
\pgfpathlineto{\pgfqpoint{3.727225in}{2.155272in}}%
\pgfpathmoveto{\pgfqpoint{3.727225in}{2.151015in}}%
\pgfpathlineto{\pgfqpoint{3.727225in}{2.151015in}}%
\pgfpathlineto{\pgfqpoint{3.727225in}{2.155272in}}%
\pgfpathlineto{\pgfqpoint{3.731483in}{2.155272in}}%
\pgfpathlineto{\pgfqpoint{3.731483in}{2.151015in}}%
\pgfpathmoveto{\pgfqpoint{3.727225in}{2.155272in}}%
\pgfpathlineto{\pgfqpoint{3.727225in}{2.155272in}}%
\pgfpathlineto{\pgfqpoint{3.727225in}{2.159530in}}%
\pgfpathlineto{\pgfqpoint{3.731483in}{2.159530in}}%
\pgfpathlineto{\pgfqpoint{3.731483in}{2.155272in}}%
\pgfpathmoveto{\pgfqpoint{3.731483in}{2.155272in}}%
\pgfpathlineto{\pgfqpoint{3.731483in}{2.155272in}}%
\pgfpathlineto{\pgfqpoint{3.731483in}{2.159530in}}%
\pgfpathlineto{\pgfqpoint{3.735741in}{2.159530in}}%
\pgfpathlineto{\pgfqpoint{3.735741in}{2.155272in}}%
\pgfpathmoveto{\pgfqpoint{3.722967in}{2.159530in}}%
\pgfpathlineto{\pgfqpoint{3.722967in}{2.159530in}}%
\pgfpathlineto{\pgfqpoint{3.722967in}{2.163788in}}%
\pgfpathlineto{\pgfqpoint{3.727225in}{2.163788in}}%
\pgfpathlineto{\pgfqpoint{3.727225in}{2.159530in}}%
\pgfpathmoveto{\pgfqpoint{3.722967in}{2.163788in}}%
\pgfpathlineto{\pgfqpoint{3.722967in}{2.163788in}}%
\pgfpathlineto{\pgfqpoint{3.722967in}{2.168046in}}%
\pgfpathlineto{\pgfqpoint{3.727225in}{2.168046in}}%
\pgfpathlineto{\pgfqpoint{3.727225in}{2.163788in}}%
\pgfpathmoveto{\pgfqpoint{3.727225in}{2.159530in}}%
\pgfpathlineto{\pgfqpoint{3.727225in}{2.159530in}}%
\pgfpathlineto{\pgfqpoint{3.727225in}{2.163788in}}%
\pgfpathlineto{\pgfqpoint{3.731483in}{2.163788in}}%
\pgfpathlineto{\pgfqpoint{3.731483in}{2.159530in}}%
\pgfpathmoveto{\pgfqpoint{3.727225in}{2.163788in}}%
\pgfpathlineto{\pgfqpoint{3.727225in}{2.163788in}}%
\pgfpathlineto{\pgfqpoint{3.727225in}{2.168046in}}%
\pgfpathlineto{\pgfqpoint{3.731483in}{2.168046in}}%
\pgfpathlineto{\pgfqpoint{3.731483in}{2.163788in}}%
\pgfpathmoveto{\pgfqpoint{3.722967in}{2.168046in}}%
\pgfpathlineto{\pgfqpoint{3.722967in}{2.168046in}}%
\pgfpathlineto{\pgfqpoint{3.722967in}{2.172303in}}%
\pgfpathlineto{\pgfqpoint{3.727225in}{2.172303in}}%
\pgfpathlineto{\pgfqpoint{3.727225in}{2.168046in}}%
\pgfpathmoveto{\pgfqpoint{3.727225in}{2.168046in}}%
\pgfpathlineto{\pgfqpoint{3.727225in}{2.168046in}}%
\pgfpathlineto{\pgfqpoint{3.727225in}{2.172303in}}%
\pgfpathlineto{\pgfqpoint{3.731483in}{2.172303in}}%
\pgfpathlineto{\pgfqpoint{3.731483in}{2.168046in}}%
\pgfpathmoveto{\pgfqpoint{3.727225in}{2.172303in}}%
\pgfpathlineto{\pgfqpoint{3.727225in}{2.172303in}}%
\pgfpathlineto{\pgfqpoint{3.727225in}{2.176561in}}%
\pgfpathlineto{\pgfqpoint{3.731483in}{2.176561in}}%
\pgfpathlineto{\pgfqpoint{3.731483in}{2.172303in}}%
\pgfpathmoveto{\pgfqpoint{3.731483in}{2.159530in}}%
\pgfpathlineto{\pgfqpoint{3.731483in}{2.159530in}}%
\pgfpathlineto{\pgfqpoint{3.731483in}{2.163788in}}%
\pgfpathlineto{\pgfqpoint{3.735741in}{2.163788in}}%
\pgfpathlineto{\pgfqpoint{3.735741in}{2.159530in}}%
\pgfpathmoveto{\pgfqpoint{3.731483in}{2.163788in}}%
\pgfpathlineto{\pgfqpoint{3.731483in}{2.163788in}}%
\pgfpathlineto{\pgfqpoint{3.731483in}{2.168046in}}%
\pgfpathlineto{\pgfqpoint{3.735741in}{2.168046in}}%
\pgfpathlineto{\pgfqpoint{3.735741in}{2.163788in}}%
\pgfpathmoveto{\pgfqpoint{3.731483in}{2.168046in}}%
\pgfpathlineto{\pgfqpoint{3.731483in}{2.168046in}}%
\pgfpathlineto{\pgfqpoint{3.731483in}{2.172303in}}%
\pgfpathlineto{\pgfqpoint{3.735741in}{2.172303in}}%
\pgfpathlineto{\pgfqpoint{3.735741in}{2.168046in}}%
\pgfpathmoveto{\pgfqpoint{3.731483in}{2.172303in}}%
\pgfpathlineto{\pgfqpoint{3.731483in}{2.172303in}}%
\pgfpathlineto{\pgfqpoint{3.731483in}{2.176561in}}%
\pgfpathlineto{\pgfqpoint{3.735741in}{2.176561in}}%
\pgfpathlineto{\pgfqpoint{3.735741in}{2.172303in}}%
\pgfpathmoveto{\pgfqpoint{3.727225in}{2.176561in}}%
\pgfpathlineto{\pgfqpoint{3.727225in}{2.176561in}}%
\pgfpathlineto{\pgfqpoint{3.727225in}{2.180819in}}%
\pgfpathlineto{\pgfqpoint{3.731483in}{2.180819in}}%
\pgfpathlineto{\pgfqpoint{3.731483in}{2.176561in}}%
\pgfpathmoveto{\pgfqpoint{3.727225in}{2.180819in}}%
\pgfpathlineto{\pgfqpoint{3.727225in}{2.180819in}}%
\pgfpathlineto{\pgfqpoint{3.727225in}{2.185077in}}%
\pgfpathlineto{\pgfqpoint{3.731483in}{2.185077in}}%
\pgfpathlineto{\pgfqpoint{3.731483in}{2.180819in}}%
\pgfpathmoveto{\pgfqpoint{3.727225in}{2.185077in}}%
\pgfpathlineto{\pgfqpoint{3.727225in}{2.185077in}}%
\pgfpathlineto{\pgfqpoint{3.727225in}{2.189334in}}%
\pgfpathlineto{\pgfqpoint{3.731483in}{2.189334in}}%
\pgfpathlineto{\pgfqpoint{3.731483in}{2.185077in}}%
\pgfpathmoveto{\pgfqpoint{3.727225in}{2.189334in}}%
\pgfpathlineto{\pgfqpoint{3.727225in}{2.189334in}}%
\pgfpathlineto{\pgfqpoint{3.727225in}{2.193592in}}%
\pgfpathlineto{\pgfqpoint{3.731483in}{2.193592in}}%
\pgfpathlineto{\pgfqpoint{3.731483in}{2.189334in}}%
\pgfpathmoveto{\pgfqpoint{3.731483in}{2.176561in}}%
\pgfpathlineto{\pgfqpoint{3.731483in}{2.176561in}}%
\pgfpathlineto{\pgfqpoint{3.731483in}{2.180819in}}%
\pgfpathlineto{\pgfqpoint{3.735741in}{2.180819in}}%
\pgfpathlineto{\pgfqpoint{3.735741in}{2.176561in}}%
\pgfpathmoveto{\pgfqpoint{3.731483in}{2.180819in}}%
\pgfpathlineto{\pgfqpoint{3.731483in}{2.180819in}}%
\pgfpathlineto{\pgfqpoint{3.731483in}{2.185077in}}%
\pgfpathlineto{\pgfqpoint{3.735741in}{2.185077in}}%
\pgfpathlineto{\pgfqpoint{3.735741in}{2.180819in}}%
\pgfpathmoveto{\pgfqpoint{3.735741in}{2.180819in}}%
\pgfpathlineto{\pgfqpoint{3.735741in}{2.180819in}}%
\pgfpathlineto{\pgfqpoint{3.735741in}{2.185077in}}%
\pgfpathlineto{\pgfqpoint{3.739999in}{2.185077in}}%
\pgfpathlineto{\pgfqpoint{3.739999in}{2.180819in}}%
\pgfpathmoveto{\pgfqpoint{3.731483in}{2.185077in}}%
\pgfpathlineto{\pgfqpoint{3.731483in}{2.185077in}}%
\pgfpathlineto{\pgfqpoint{3.731483in}{2.189334in}}%
\pgfpathlineto{\pgfqpoint{3.735741in}{2.189334in}}%
\pgfpathlineto{\pgfqpoint{3.735741in}{2.185077in}}%
\pgfpathmoveto{\pgfqpoint{3.731483in}{2.189334in}}%
\pgfpathlineto{\pgfqpoint{3.731483in}{2.189334in}}%
\pgfpathlineto{\pgfqpoint{3.731483in}{2.193592in}}%
\pgfpathlineto{\pgfqpoint{3.735741in}{2.193592in}}%
\pgfpathlineto{\pgfqpoint{3.735741in}{2.189334in}}%
\pgfpathmoveto{\pgfqpoint{3.735741in}{2.185077in}}%
\pgfpathlineto{\pgfqpoint{3.735741in}{2.185077in}}%
\pgfpathlineto{\pgfqpoint{3.735741in}{2.189334in}}%
\pgfpathlineto{\pgfqpoint{3.739999in}{2.189334in}}%
\pgfpathlineto{\pgfqpoint{3.739999in}{2.185077in}}%
\pgfpathmoveto{\pgfqpoint{3.735741in}{2.189334in}}%
\pgfpathlineto{\pgfqpoint{3.735741in}{2.189334in}}%
\pgfpathlineto{\pgfqpoint{3.735741in}{2.193592in}}%
\pgfpathlineto{\pgfqpoint{3.739999in}{2.193592in}}%
\pgfpathlineto{\pgfqpoint{3.739999in}{2.189334in}}%
\pgfpathmoveto{\pgfqpoint{3.731483in}{2.193592in}}%
\pgfpathlineto{\pgfqpoint{3.731483in}{2.193592in}}%
\pgfpathlineto{\pgfqpoint{3.731483in}{2.197850in}}%
\pgfpathlineto{\pgfqpoint{3.735741in}{2.197850in}}%
\pgfpathlineto{\pgfqpoint{3.735741in}{2.193592in}}%
\pgfpathmoveto{\pgfqpoint{3.731483in}{2.197850in}}%
\pgfpathlineto{\pgfqpoint{3.731483in}{2.197850in}}%
\pgfpathlineto{\pgfqpoint{3.731483in}{2.202108in}}%
\pgfpathlineto{\pgfqpoint{3.735741in}{2.202108in}}%
\pgfpathlineto{\pgfqpoint{3.735741in}{2.197850in}}%
\pgfpathmoveto{\pgfqpoint{3.735741in}{2.193592in}}%
\pgfpathlineto{\pgfqpoint{3.735741in}{2.193592in}}%
\pgfpathlineto{\pgfqpoint{3.735741in}{2.197850in}}%
\pgfpathlineto{\pgfqpoint{3.739999in}{2.197850in}}%
\pgfpathlineto{\pgfqpoint{3.739999in}{2.193592in}}%
\pgfpathmoveto{\pgfqpoint{3.735741in}{2.197850in}}%
\pgfpathlineto{\pgfqpoint{3.735741in}{2.197850in}}%
\pgfpathlineto{\pgfqpoint{3.735741in}{2.202108in}}%
\pgfpathlineto{\pgfqpoint{3.739999in}{2.202108in}}%
\pgfpathlineto{\pgfqpoint{3.739999in}{2.197850in}}%
\pgfpathmoveto{\pgfqpoint{3.731483in}{2.202108in}}%
\pgfpathlineto{\pgfqpoint{3.731483in}{2.202108in}}%
\pgfpathlineto{\pgfqpoint{3.731483in}{2.206365in}}%
\pgfpathlineto{\pgfqpoint{3.735741in}{2.206365in}}%
\pgfpathlineto{\pgfqpoint{3.735741in}{2.202108in}}%
\pgfpathmoveto{\pgfqpoint{3.731483in}{2.206365in}}%
\pgfpathlineto{\pgfqpoint{3.731483in}{2.206365in}}%
\pgfpathlineto{\pgfqpoint{3.731483in}{2.210623in}}%
\pgfpathlineto{\pgfqpoint{3.735741in}{2.210623in}}%
\pgfpathlineto{\pgfqpoint{3.735741in}{2.206365in}}%
\pgfpathmoveto{\pgfqpoint{3.735741in}{2.202108in}}%
\pgfpathlineto{\pgfqpoint{3.735741in}{2.202108in}}%
\pgfpathlineto{\pgfqpoint{3.735741in}{2.206365in}}%
\pgfpathlineto{\pgfqpoint{3.739999in}{2.206365in}}%
\pgfpathlineto{\pgfqpoint{3.739999in}{2.202108in}}%
\pgfpathmoveto{\pgfqpoint{3.735741in}{2.206365in}}%
\pgfpathlineto{\pgfqpoint{3.735741in}{2.206365in}}%
\pgfpathlineto{\pgfqpoint{3.735741in}{2.210623in}}%
\pgfpathlineto{\pgfqpoint{3.739999in}{2.210623in}}%
\pgfpathlineto{\pgfqpoint{3.739999in}{2.206365in}}%
\pgfpathmoveto{\pgfqpoint{3.739999in}{2.202108in}}%
\pgfpathlineto{\pgfqpoint{3.739999in}{2.202108in}}%
\pgfpathlineto{\pgfqpoint{3.739999in}{2.206365in}}%
\pgfpathlineto{\pgfqpoint{3.744257in}{2.206365in}}%
\pgfpathlineto{\pgfqpoint{3.744257in}{2.202108in}}%
\pgfpathmoveto{\pgfqpoint{3.739999in}{2.206365in}}%
\pgfpathlineto{\pgfqpoint{3.739999in}{2.206365in}}%
\pgfpathlineto{\pgfqpoint{3.739999in}{2.210623in}}%
\pgfpathlineto{\pgfqpoint{3.744257in}{2.210623in}}%
\pgfpathlineto{\pgfqpoint{3.744257in}{2.206365in}}%
\pgfpathmoveto{\pgfqpoint{3.731483in}{2.210623in}}%
\pgfpathlineto{\pgfqpoint{3.731483in}{2.210623in}}%
\pgfpathlineto{\pgfqpoint{3.731483in}{2.214881in}}%
\pgfpathlineto{\pgfqpoint{3.735741in}{2.214881in}}%
\pgfpathlineto{\pgfqpoint{3.735741in}{2.210623in}}%
\pgfpathmoveto{\pgfqpoint{3.731483in}{2.214881in}}%
\pgfpathlineto{\pgfqpoint{3.731483in}{2.214881in}}%
\pgfpathlineto{\pgfqpoint{3.731483in}{2.219139in}}%
\pgfpathlineto{\pgfqpoint{3.735741in}{2.219139in}}%
\pgfpathlineto{\pgfqpoint{3.735741in}{2.214881in}}%
\pgfpathmoveto{\pgfqpoint{3.735741in}{2.210623in}}%
\pgfpathlineto{\pgfqpoint{3.735741in}{2.210623in}}%
\pgfpathlineto{\pgfqpoint{3.735741in}{2.214881in}}%
\pgfpathlineto{\pgfqpoint{3.739999in}{2.214881in}}%
\pgfpathlineto{\pgfqpoint{3.739999in}{2.210623in}}%
\pgfpathmoveto{\pgfqpoint{3.735741in}{2.214881in}}%
\pgfpathlineto{\pgfqpoint{3.735741in}{2.214881in}}%
\pgfpathlineto{\pgfqpoint{3.735741in}{2.219139in}}%
\pgfpathlineto{\pgfqpoint{3.739999in}{2.219139in}}%
\pgfpathlineto{\pgfqpoint{3.739999in}{2.214881in}}%
\pgfpathmoveto{\pgfqpoint{3.735741in}{2.219139in}}%
\pgfpathlineto{\pgfqpoint{3.735741in}{2.219139in}}%
\pgfpathlineto{\pgfqpoint{3.735741in}{2.223396in}}%
\pgfpathlineto{\pgfqpoint{3.739999in}{2.223396in}}%
\pgfpathlineto{\pgfqpoint{3.739999in}{2.219139in}}%
\pgfpathmoveto{\pgfqpoint{3.735741in}{2.223396in}}%
\pgfpathlineto{\pgfqpoint{3.735741in}{2.223396in}}%
\pgfpathlineto{\pgfqpoint{3.735741in}{2.227654in}}%
\pgfpathlineto{\pgfqpoint{3.739999in}{2.227654in}}%
\pgfpathlineto{\pgfqpoint{3.739999in}{2.223396in}}%
\pgfpathmoveto{\pgfqpoint{3.735741in}{2.227654in}}%
\pgfpathlineto{\pgfqpoint{3.735741in}{2.227654in}}%
\pgfpathlineto{\pgfqpoint{3.735741in}{2.231912in}}%
\pgfpathlineto{\pgfqpoint{3.739999in}{2.231912in}}%
\pgfpathlineto{\pgfqpoint{3.739999in}{2.227654in}}%
\pgfpathmoveto{\pgfqpoint{3.735741in}{2.231912in}}%
\pgfpathlineto{\pgfqpoint{3.735741in}{2.231912in}}%
\pgfpathlineto{\pgfqpoint{3.735741in}{2.236170in}}%
\pgfpathlineto{\pgfqpoint{3.739999in}{2.236170in}}%
\pgfpathlineto{\pgfqpoint{3.739999in}{2.231912in}}%
\pgfpathmoveto{\pgfqpoint{3.735741in}{2.236170in}}%
\pgfpathlineto{\pgfqpoint{3.735741in}{2.236170in}}%
\pgfpathlineto{\pgfqpoint{3.735741in}{2.240427in}}%
\pgfpathlineto{\pgfqpoint{3.739999in}{2.240427in}}%
\pgfpathlineto{\pgfqpoint{3.739999in}{2.236170in}}%
\pgfpathmoveto{\pgfqpoint{3.739999in}{2.210623in}}%
\pgfpathlineto{\pgfqpoint{3.739999in}{2.210623in}}%
\pgfpathlineto{\pgfqpoint{3.739999in}{2.214881in}}%
\pgfpathlineto{\pgfqpoint{3.744257in}{2.214881in}}%
\pgfpathlineto{\pgfqpoint{3.744257in}{2.210623in}}%
\pgfpathmoveto{\pgfqpoint{3.739999in}{2.214881in}}%
\pgfpathlineto{\pgfqpoint{3.739999in}{2.214881in}}%
\pgfpathlineto{\pgfqpoint{3.739999in}{2.219139in}}%
\pgfpathlineto{\pgfqpoint{3.744257in}{2.219139in}}%
\pgfpathlineto{\pgfqpoint{3.744257in}{2.214881in}}%
\pgfpathmoveto{\pgfqpoint{3.739999in}{2.219139in}}%
\pgfpathlineto{\pgfqpoint{3.739999in}{2.219139in}}%
\pgfpathlineto{\pgfqpoint{3.739999in}{2.223396in}}%
\pgfpathlineto{\pgfqpoint{3.744257in}{2.223396in}}%
\pgfpathlineto{\pgfqpoint{3.744257in}{2.219139in}}%
\pgfpathmoveto{\pgfqpoint{3.739999in}{2.223396in}}%
\pgfpathlineto{\pgfqpoint{3.739999in}{2.223396in}}%
\pgfpathlineto{\pgfqpoint{3.739999in}{2.227654in}}%
\pgfpathlineto{\pgfqpoint{3.744257in}{2.227654in}}%
\pgfpathlineto{\pgfqpoint{3.744257in}{2.223396in}}%
\pgfpathmoveto{\pgfqpoint{3.744257in}{2.223396in}}%
\pgfpathlineto{\pgfqpoint{3.744257in}{2.223396in}}%
\pgfpathlineto{\pgfqpoint{3.744257in}{2.227654in}}%
\pgfpathlineto{\pgfqpoint{3.748515in}{2.227654in}}%
\pgfpathlineto{\pgfqpoint{3.748515in}{2.223396in}}%
\pgfpathmoveto{\pgfqpoint{3.739999in}{2.227654in}}%
\pgfpathlineto{\pgfqpoint{3.739999in}{2.227654in}}%
\pgfpathlineto{\pgfqpoint{3.739999in}{2.231912in}}%
\pgfpathlineto{\pgfqpoint{3.744257in}{2.231912in}}%
\pgfpathlineto{\pgfqpoint{3.744257in}{2.227654in}}%
\pgfpathmoveto{\pgfqpoint{3.739999in}{2.231912in}}%
\pgfpathlineto{\pgfqpoint{3.739999in}{2.231912in}}%
\pgfpathlineto{\pgfqpoint{3.739999in}{2.236170in}}%
\pgfpathlineto{\pgfqpoint{3.744257in}{2.236170in}}%
\pgfpathlineto{\pgfqpoint{3.744257in}{2.231912in}}%
\pgfpathmoveto{\pgfqpoint{3.744257in}{2.227654in}}%
\pgfpathlineto{\pgfqpoint{3.744257in}{2.227654in}}%
\pgfpathlineto{\pgfqpoint{3.744257in}{2.231912in}}%
\pgfpathlineto{\pgfqpoint{3.748515in}{2.231912in}}%
\pgfpathlineto{\pgfqpoint{3.748515in}{2.227654in}}%
\pgfpathmoveto{\pgfqpoint{3.744257in}{2.231912in}}%
\pgfpathlineto{\pgfqpoint{3.744257in}{2.231912in}}%
\pgfpathlineto{\pgfqpoint{3.744257in}{2.236170in}}%
\pgfpathlineto{\pgfqpoint{3.748515in}{2.236170in}}%
\pgfpathlineto{\pgfqpoint{3.748515in}{2.231912in}}%
\pgfpathmoveto{\pgfqpoint{3.739999in}{2.236170in}}%
\pgfpathlineto{\pgfqpoint{3.739999in}{2.236170in}}%
\pgfpathlineto{\pgfqpoint{3.739999in}{2.240427in}}%
\pgfpathlineto{\pgfqpoint{3.744257in}{2.240427in}}%
\pgfpathlineto{\pgfqpoint{3.744257in}{2.236170in}}%
\pgfpathmoveto{\pgfqpoint{3.739999in}{2.240427in}}%
\pgfpathlineto{\pgfqpoint{3.739999in}{2.240427in}}%
\pgfpathlineto{\pgfqpoint{3.739999in}{2.244685in}}%
\pgfpathlineto{\pgfqpoint{3.744257in}{2.244685in}}%
\pgfpathlineto{\pgfqpoint{3.744257in}{2.240427in}}%
\pgfpathmoveto{\pgfqpoint{3.744257in}{2.236170in}}%
\pgfpathlineto{\pgfqpoint{3.744257in}{2.236170in}}%
\pgfpathlineto{\pgfqpoint{3.744257in}{2.240427in}}%
\pgfpathlineto{\pgfqpoint{3.748515in}{2.240427in}}%
\pgfpathlineto{\pgfqpoint{3.748515in}{2.236170in}}%
\pgfpathmoveto{\pgfqpoint{3.744257in}{2.240427in}}%
\pgfpathlineto{\pgfqpoint{3.744257in}{2.240427in}}%
\pgfpathlineto{\pgfqpoint{3.744257in}{2.244685in}}%
\pgfpathlineto{\pgfqpoint{3.748515in}{2.244685in}}%
\pgfpathlineto{\pgfqpoint{3.748515in}{2.240427in}}%
\pgfpathmoveto{\pgfqpoint{3.739999in}{2.244685in}}%
\pgfpathlineto{\pgfqpoint{3.739999in}{2.244685in}}%
\pgfpathlineto{\pgfqpoint{3.739999in}{2.248943in}}%
\pgfpathlineto{\pgfqpoint{3.744257in}{2.248943in}}%
\pgfpathlineto{\pgfqpoint{3.744257in}{2.244685in}}%
\pgfpathmoveto{\pgfqpoint{3.739999in}{2.248943in}}%
\pgfpathlineto{\pgfqpoint{3.739999in}{2.248943in}}%
\pgfpathlineto{\pgfqpoint{3.739999in}{2.253201in}}%
\pgfpathlineto{\pgfqpoint{3.744257in}{2.253201in}}%
\pgfpathlineto{\pgfqpoint{3.744257in}{2.248943in}}%
\pgfpathmoveto{\pgfqpoint{3.744257in}{2.244685in}}%
\pgfpathlineto{\pgfqpoint{3.744257in}{2.244685in}}%
\pgfpathlineto{\pgfqpoint{3.744257in}{2.248943in}}%
\pgfpathlineto{\pgfqpoint{3.748515in}{2.248943in}}%
\pgfpathlineto{\pgfqpoint{3.748515in}{2.244685in}}%
\pgfpathmoveto{\pgfqpoint{3.744257in}{2.248943in}}%
\pgfpathlineto{\pgfqpoint{3.744257in}{2.248943in}}%
\pgfpathlineto{\pgfqpoint{3.744257in}{2.253201in}}%
\pgfpathlineto{\pgfqpoint{3.748515in}{2.253201in}}%
\pgfpathlineto{\pgfqpoint{3.748515in}{2.248943in}}%
\pgfpathmoveto{\pgfqpoint{3.739999in}{2.253201in}}%
\pgfpathlineto{\pgfqpoint{3.739999in}{2.253201in}}%
\pgfpathlineto{\pgfqpoint{3.739999in}{2.257458in}}%
\pgfpathlineto{\pgfqpoint{3.744257in}{2.257458in}}%
\pgfpathlineto{\pgfqpoint{3.744257in}{2.253201in}}%
\pgfpathmoveto{\pgfqpoint{3.739999in}{2.257458in}}%
\pgfpathlineto{\pgfqpoint{3.739999in}{2.257458in}}%
\pgfpathlineto{\pgfqpoint{3.739999in}{2.261716in}}%
\pgfpathlineto{\pgfqpoint{3.744257in}{2.261716in}}%
\pgfpathlineto{\pgfqpoint{3.744257in}{2.257458in}}%
\pgfpathmoveto{\pgfqpoint{3.744257in}{2.253201in}}%
\pgfpathlineto{\pgfqpoint{3.744257in}{2.253201in}}%
\pgfpathlineto{\pgfqpoint{3.744257in}{2.257458in}}%
\pgfpathlineto{\pgfqpoint{3.748515in}{2.257458in}}%
\pgfpathlineto{\pgfqpoint{3.748515in}{2.253201in}}%
\pgfpathmoveto{\pgfqpoint{3.744257in}{2.257458in}}%
\pgfpathlineto{\pgfqpoint{3.744257in}{2.257458in}}%
\pgfpathlineto{\pgfqpoint{3.744257in}{2.261716in}}%
\pgfpathlineto{\pgfqpoint{3.748515in}{2.261716in}}%
\pgfpathlineto{\pgfqpoint{3.748515in}{2.257458in}}%
\pgfpathmoveto{\pgfqpoint{3.748515in}{2.248943in}}%
\pgfpathlineto{\pgfqpoint{3.748515in}{2.248943in}}%
\pgfpathlineto{\pgfqpoint{3.748515in}{2.253201in}}%
\pgfpathlineto{\pgfqpoint{3.752773in}{2.253201in}}%
\pgfpathlineto{\pgfqpoint{3.752773in}{2.248943in}}%
\pgfpathmoveto{\pgfqpoint{3.748515in}{2.253201in}}%
\pgfpathlineto{\pgfqpoint{3.748515in}{2.253201in}}%
\pgfpathlineto{\pgfqpoint{3.748515in}{2.257458in}}%
\pgfpathlineto{\pgfqpoint{3.752773in}{2.257458in}}%
\pgfpathlineto{\pgfqpoint{3.752773in}{2.253201in}}%
\pgfpathmoveto{\pgfqpoint{3.748515in}{2.257458in}}%
\pgfpathlineto{\pgfqpoint{3.748515in}{2.257458in}}%
\pgfpathlineto{\pgfqpoint{3.748515in}{2.261716in}}%
\pgfpathlineto{\pgfqpoint{3.752773in}{2.261716in}}%
\pgfpathlineto{\pgfqpoint{3.752773in}{2.257458in}}%
\pgfpathmoveto{\pgfqpoint{3.744257in}{2.261716in}}%
\pgfpathlineto{\pgfqpoint{3.744257in}{2.261716in}}%
\pgfpathlineto{\pgfqpoint{3.744257in}{2.265974in}}%
\pgfpathlineto{\pgfqpoint{3.748515in}{2.265974in}}%
\pgfpathlineto{\pgfqpoint{3.748515in}{2.261716in}}%
\pgfpathmoveto{\pgfqpoint{3.744257in}{2.265974in}}%
\pgfpathlineto{\pgfqpoint{3.744257in}{2.265974in}}%
\pgfpathlineto{\pgfqpoint{3.744257in}{2.270232in}}%
\pgfpathlineto{\pgfqpoint{3.748515in}{2.270232in}}%
\pgfpathlineto{\pgfqpoint{3.748515in}{2.265974in}}%
\pgfpathmoveto{\pgfqpoint{3.744257in}{2.270232in}}%
\pgfpathlineto{\pgfqpoint{3.744257in}{2.270232in}}%
\pgfpathlineto{\pgfqpoint{3.744257in}{2.274489in}}%
\pgfpathlineto{\pgfqpoint{3.748515in}{2.274489in}}%
\pgfpathlineto{\pgfqpoint{3.748515in}{2.270232in}}%
\pgfpathmoveto{\pgfqpoint{3.744257in}{2.274489in}}%
\pgfpathlineto{\pgfqpoint{3.744257in}{2.274489in}}%
\pgfpathlineto{\pgfqpoint{3.744257in}{2.278747in}}%
\pgfpathlineto{\pgfqpoint{3.748515in}{2.278747in}}%
\pgfpathlineto{\pgfqpoint{3.748515in}{2.274489in}}%
\pgfpathmoveto{\pgfqpoint{3.748515in}{2.261716in}}%
\pgfpathlineto{\pgfqpoint{3.748515in}{2.261716in}}%
\pgfpathlineto{\pgfqpoint{3.748515in}{2.265974in}}%
\pgfpathlineto{\pgfqpoint{3.752773in}{2.265974in}}%
\pgfpathlineto{\pgfqpoint{3.752773in}{2.261716in}}%
\pgfpathmoveto{\pgfqpoint{3.748515in}{2.265974in}}%
\pgfpathlineto{\pgfqpoint{3.748515in}{2.265974in}}%
\pgfpathlineto{\pgfqpoint{3.748515in}{2.270232in}}%
\pgfpathlineto{\pgfqpoint{3.752773in}{2.270232in}}%
\pgfpathlineto{\pgfqpoint{3.752773in}{2.265974in}}%
\pgfpathmoveto{\pgfqpoint{3.748515in}{2.270232in}}%
\pgfpathlineto{\pgfqpoint{3.748515in}{2.270232in}}%
\pgfpathlineto{\pgfqpoint{3.748515in}{2.274489in}}%
\pgfpathlineto{\pgfqpoint{3.752773in}{2.274489in}}%
\pgfpathlineto{\pgfqpoint{3.752773in}{2.270232in}}%
\pgfpathmoveto{\pgfqpoint{3.748515in}{2.274489in}}%
\pgfpathlineto{\pgfqpoint{3.748515in}{2.274489in}}%
\pgfpathlineto{\pgfqpoint{3.748515in}{2.278747in}}%
\pgfpathlineto{\pgfqpoint{3.752773in}{2.278747in}}%
\pgfpathlineto{\pgfqpoint{3.752773in}{2.274489in}}%
\pgfpathmoveto{\pgfqpoint{3.752773in}{2.270232in}}%
\pgfpathlineto{\pgfqpoint{3.752773in}{2.270232in}}%
\pgfpathlineto{\pgfqpoint{3.752773in}{2.274489in}}%
\pgfpathlineto{\pgfqpoint{3.757031in}{2.274489in}}%
\pgfpathlineto{\pgfqpoint{3.757031in}{2.270232in}}%
\pgfpathmoveto{\pgfqpoint{3.752773in}{2.274489in}}%
\pgfpathlineto{\pgfqpoint{3.752773in}{2.274489in}}%
\pgfpathlineto{\pgfqpoint{3.752773in}{2.278747in}}%
\pgfpathlineto{\pgfqpoint{3.757031in}{2.278747in}}%
\pgfpathlineto{\pgfqpoint{3.757031in}{2.274489in}}%
\pgfpathmoveto{\pgfqpoint{3.744257in}{2.278747in}}%
\pgfpathlineto{\pgfqpoint{3.744257in}{2.278747in}}%
\pgfpathlineto{\pgfqpoint{3.744257in}{2.283005in}}%
\pgfpathlineto{\pgfqpoint{3.748515in}{2.283005in}}%
\pgfpathlineto{\pgfqpoint{3.748515in}{2.278747in}}%
\pgfpathmoveto{\pgfqpoint{3.744257in}{2.283005in}}%
\pgfpathlineto{\pgfqpoint{3.744257in}{2.283005in}}%
\pgfpathlineto{\pgfqpoint{3.744257in}{2.287263in}}%
\pgfpathlineto{\pgfqpoint{3.748515in}{2.287263in}}%
\pgfpathlineto{\pgfqpoint{3.748515in}{2.283005in}}%
\pgfpathmoveto{\pgfqpoint{3.748515in}{2.278747in}}%
\pgfpathlineto{\pgfqpoint{3.748515in}{2.278747in}}%
\pgfpathlineto{\pgfqpoint{3.748515in}{2.283005in}}%
\pgfpathlineto{\pgfqpoint{3.752773in}{2.283005in}}%
\pgfpathlineto{\pgfqpoint{3.752773in}{2.278747in}}%
\pgfpathmoveto{\pgfqpoint{3.748515in}{2.283005in}}%
\pgfpathlineto{\pgfqpoint{3.748515in}{2.283005in}}%
\pgfpathlineto{\pgfqpoint{3.748515in}{2.287263in}}%
\pgfpathlineto{\pgfqpoint{3.752773in}{2.287263in}}%
\pgfpathlineto{\pgfqpoint{3.752773in}{2.283005in}}%
\pgfpathmoveto{\pgfqpoint{3.752773in}{2.278747in}}%
\pgfpathlineto{\pgfqpoint{3.752773in}{2.278747in}}%
\pgfpathlineto{\pgfqpoint{3.752773in}{2.283005in}}%
\pgfpathlineto{\pgfqpoint{3.757031in}{2.283005in}}%
\pgfpathlineto{\pgfqpoint{3.757031in}{2.278747in}}%
\pgfpathmoveto{\pgfqpoint{3.752773in}{2.283005in}}%
\pgfpathlineto{\pgfqpoint{3.752773in}{2.283005in}}%
\pgfpathlineto{\pgfqpoint{3.752773in}{2.287263in}}%
\pgfpathlineto{\pgfqpoint{3.757031in}{2.287263in}}%
\pgfpathlineto{\pgfqpoint{3.757031in}{2.283005in}}%
\pgfpathmoveto{\pgfqpoint{3.748515in}{2.287263in}}%
\pgfpathlineto{\pgfqpoint{3.748515in}{2.287263in}}%
\pgfpathlineto{\pgfqpoint{3.748515in}{2.291521in}}%
\pgfpathlineto{\pgfqpoint{3.752773in}{2.291521in}}%
\pgfpathlineto{\pgfqpoint{3.752773in}{2.287263in}}%
\pgfpathmoveto{\pgfqpoint{3.748515in}{2.291521in}}%
\pgfpathlineto{\pgfqpoint{3.748515in}{2.291521in}}%
\pgfpathlineto{\pgfqpoint{3.748515in}{2.295779in}}%
\pgfpathlineto{\pgfqpoint{3.752773in}{2.295779in}}%
\pgfpathlineto{\pgfqpoint{3.752773in}{2.291521in}}%
\pgfpathmoveto{\pgfqpoint{3.752773in}{2.287263in}}%
\pgfpathlineto{\pgfqpoint{3.752773in}{2.287263in}}%
\pgfpathlineto{\pgfqpoint{3.752773in}{2.291521in}}%
\pgfpathlineto{\pgfqpoint{3.757031in}{2.291521in}}%
\pgfpathlineto{\pgfqpoint{3.757031in}{2.287263in}}%
\pgfpathmoveto{\pgfqpoint{3.752773in}{2.291521in}}%
\pgfpathlineto{\pgfqpoint{3.752773in}{2.291521in}}%
\pgfpathlineto{\pgfqpoint{3.752773in}{2.295779in}}%
\pgfpathlineto{\pgfqpoint{3.757031in}{2.295779in}}%
\pgfpathlineto{\pgfqpoint{3.757031in}{2.291521in}}%
\pgfpathmoveto{\pgfqpoint{3.748515in}{2.295779in}}%
\pgfpathlineto{\pgfqpoint{3.748515in}{2.295779in}}%
\pgfpathlineto{\pgfqpoint{3.748515in}{2.300037in}}%
\pgfpathlineto{\pgfqpoint{3.752773in}{2.300037in}}%
\pgfpathlineto{\pgfqpoint{3.752773in}{2.295779in}}%
\pgfpathmoveto{\pgfqpoint{3.748515in}{2.300037in}}%
\pgfpathlineto{\pgfqpoint{3.748515in}{2.300037in}}%
\pgfpathlineto{\pgfqpoint{3.748515in}{2.304295in}}%
\pgfpathlineto{\pgfqpoint{3.752773in}{2.304295in}}%
\pgfpathlineto{\pgfqpoint{3.752773in}{2.300037in}}%
\pgfpathmoveto{\pgfqpoint{3.752773in}{2.295779in}}%
\pgfpathlineto{\pgfqpoint{3.752773in}{2.295779in}}%
\pgfpathlineto{\pgfqpoint{3.752773in}{2.300037in}}%
\pgfpathlineto{\pgfqpoint{3.757031in}{2.300037in}}%
\pgfpathlineto{\pgfqpoint{3.757031in}{2.295779in}}%
\pgfpathmoveto{\pgfqpoint{3.752773in}{2.300037in}}%
\pgfpathlineto{\pgfqpoint{3.752773in}{2.300037in}}%
\pgfpathlineto{\pgfqpoint{3.752773in}{2.304295in}}%
\pgfpathlineto{\pgfqpoint{3.757031in}{2.304295in}}%
\pgfpathlineto{\pgfqpoint{3.757031in}{2.300037in}}%
\pgfpathmoveto{\pgfqpoint{3.748515in}{2.304295in}}%
\pgfpathlineto{\pgfqpoint{3.748515in}{2.304295in}}%
\pgfpathlineto{\pgfqpoint{3.748515in}{2.308552in}}%
\pgfpathlineto{\pgfqpoint{3.752773in}{2.308552in}}%
\pgfpathlineto{\pgfqpoint{3.752773in}{2.304295in}}%
\pgfpathmoveto{\pgfqpoint{3.752773in}{2.304295in}}%
\pgfpathlineto{\pgfqpoint{3.752773in}{2.304295in}}%
\pgfpathlineto{\pgfqpoint{3.752773in}{2.308552in}}%
\pgfpathlineto{\pgfqpoint{3.757031in}{2.308552in}}%
\pgfpathlineto{\pgfqpoint{3.757031in}{2.304295in}}%
\pgfpathmoveto{\pgfqpoint{3.752773in}{2.308552in}}%
\pgfpathlineto{\pgfqpoint{3.752773in}{2.308552in}}%
\pgfpathlineto{\pgfqpoint{3.752773in}{2.312810in}}%
\pgfpathlineto{\pgfqpoint{3.757031in}{2.312810in}}%
\pgfpathlineto{\pgfqpoint{3.757031in}{2.308552in}}%
\pgfpathmoveto{\pgfqpoint{3.757031in}{2.295779in}}%
\pgfpathlineto{\pgfqpoint{3.757031in}{2.295779in}}%
\pgfpathlineto{\pgfqpoint{3.757031in}{2.300037in}}%
\pgfpathlineto{\pgfqpoint{3.761289in}{2.300037in}}%
\pgfpathlineto{\pgfqpoint{3.761289in}{2.295779in}}%
\pgfpathmoveto{\pgfqpoint{3.757031in}{2.300037in}}%
\pgfpathlineto{\pgfqpoint{3.757031in}{2.300037in}}%
\pgfpathlineto{\pgfqpoint{3.757031in}{2.304295in}}%
\pgfpathlineto{\pgfqpoint{3.761289in}{2.304295in}}%
\pgfpathlineto{\pgfqpoint{3.761289in}{2.300037in}}%
\pgfpathmoveto{\pgfqpoint{3.757031in}{2.304295in}}%
\pgfpathlineto{\pgfqpoint{3.757031in}{2.304295in}}%
\pgfpathlineto{\pgfqpoint{3.757031in}{2.308552in}}%
\pgfpathlineto{\pgfqpoint{3.761289in}{2.308552in}}%
\pgfpathlineto{\pgfqpoint{3.761289in}{2.304295in}}%
\pgfpathmoveto{\pgfqpoint{3.757031in}{2.308552in}}%
\pgfpathlineto{\pgfqpoint{3.757031in}{2.308552in}}%
\pgfpathlineto{\pgfqpoint{3.757031in}{2.312810in}}%
\pgfpathlineto{\pgfqpoint{3.761289in}{2.312810in}}%
\pgfpathlineto{\pgfqpoint{3.761289in}{2.308552in}}%
\pgfpathmoveto{\pgfqpoint{3.752773in}{2.312810in}}%
\pgfpathlineto{\pgfqpoint{3.752773in}{2.312810in}}%
\pgfpathlineto{\pgfqpoint{3.752773in}{2.317068in}}%
\pgfpathlineto{\pgfqpoint{3.757031in}{2.317068in}}%
\pgfpathlineto{\pgfqpoint{3.757031in}{2.312810in}}%
\pgfpathmoveto{\pgfqpoint{3.752773in}{2.317068in}}%
\pgfpathlineto{\pgfqpoint{3.752773in}{2.317068in}}%
\pgfpathlineto{\pgfqpoint{3.752773in}{2.321326in}}%
\pgfpathlineto{\pgfqpoint{3.757031in}{2.321326in}}%
\pgfpathlineto{\pgfqpoint{3.757031in}{2.317068in}}%
\pgfpathmoveto{\pgfqpoint{3.752773in}{2.321326in}}%
\pgfpathlineto{\pgfqpoint{3.752773in}{2.321326in}}%
\pgfpathlineto{\pgfqpoint{3.752773in}{2.325584in}}%
\pgfpathlineto{\pgfqpoint{3.757031in}{2.325584in}}%
\pgfpathlineto{\pgfqpoint{3.757031in}{2.321326in}}%
\pgfpathmoveto{\pgfqpoint{3.752773in}{2.325584in}}%
\pgfpathlineto{\pgfqpoint{3.752773in}{2.325584in}}%
\pgfpathlineto{\pgfqpoint{3.752773in}{2.329842in}}%
\pgfpathlineto{\pgfqpoint{3.757031in}{2.329842in}}%
\pgfpathlineto{\pgfqpoint{3.757031in}{2.325584in}}%
\pgfpathmoveto{\pgfqpoint{3.752773in}{2.329842in}}%
\pgfpathlineto{\pgfqpoint{3.752773in}{2.329842in}}%
\pgfpathlineto{\pgfqpoint{3.752773in}{2.334100in}}%
\pgfpathlineto{\pgfqpoint{3.757031in}{2.334100in}}%
\pgfpathlineto{\pgfqpoint{3.757031in}{2.329842in}}%
\pgfpathmoveto{\pgfqpoint{3.757031in}{2.312810in}}%
\pgfpathlineto{\pgfqpoint{3.757031in}{2.312810in}}%
\pgfpathlineto{\pgfqpoint{3.757031in}{2.317068in}}%
\pgfpathlineto{\pgfqpoint{3.761289in}{2.317068in}}%
\pgfpathlineto{\pgfqpoint{3.761289in}{2.312810in}}%
\pgfpathmoveto{\pgfqpoint{3.757031in}{2.317068in}}%
\pgfpathlineto{\pgfqpoint{3.757031in}{2.317068in}}%
\pgfpathlineto{\pgfqpoint{3.757031in}{2.321326in}}%
\pgfpathlineto{\pgfqpoint{3.761289in}{2.321326in}}%
\pgfpathlineto{\pgfqpoint{3.761289in}{2.317068in}}%
\pgfpathmoveto{\pgfqpoint{3.757031in}{2.321326in}}%
\pgfpathlineto{\pgfqpoint{3.757031in}{2.321326in}}%
\pgfpathlineto{\pgfqpoint{3.757031in}{2.325584in}}%
\pgfpathlineto{\pgfqpoint{3.761289in}{2.325584in}}%
\pgfpathlineto{\pgfqpoint{3.761289in}{2.321326in}}%
\pgfpathmoveto{\pgfqpoint{3.757031in}{2.325584in}}%
\pgfpathlineto{\pgfqpoint{3.757031in}{2.325584in}}%
\pgfpathlineto{\pgfqpoint{3.757031in}{2.329842in}}%
\pgfpathlineto{\pgfqpoint{3.761289in}{2.329842in}}%
\pgfpathlineto{\pgfqpoint{3.761289in}{2.325584in}}%
\pgfpathmoveto{\pgfqpoint{3.761289in}{2.321326in}}%
\pgfpathlineto{\pgfqpoint{3.761289in}{2.321326in}}%
\pgfpathlineto{\pgfqpoint{3.761289in}{2.325584in}}%
\pgfpathlineto{\pgfqpoint{3.765547in}{2.325584in}}%
\pgfpathlineto{\pgfqpoint{3.765547in}{2.321326in}}%
\pgfpathmoveto{\pgfqpoint{3.761289in}{2.325584in}}%
\pgfpathlineto{\pgfqpoint{3.761289in}{2.325584in}}%
\pgfpathlineto{\pgfqpoint{3.761289in}{2.329842in}}%
\pgfpathlineto{\pgfqpoint{3.765547in}{2.329842in}}%
\pgfpathlineto{\pgfqpoint{3.765547in}{2.325584in}}%
\pgfpathmoveto{\pgfqpoint{3.757031in}{2.329842in}}%
\pgfpathlineto{\pgfqpoint{3.757031in}{2.329842in}}%
\pgfpathlineto{\pgfqpoint{3.757031in}{2.334100in}}%
\pgfpathlineto{\pgfqpoint{3.761289in}{2.334100in}}%
\pgfpathlineto{\pgfqpoint{3.761289in}{2.329842in}}%
\pgfpathmoveto{\pgfqpoint{3.757031in}{2.334100in}}%
\pgfpathlineto{\pgfqpoint{3.757031in}{2.334100in}}%
\pgfpathlineto{\pgfqpoint{3.757031in}{2.338358in}}%
\pgfpathlineto{\pgfqpoint{3.761289in}{2.338358in}}%
\pgfpathlineto{\pgfqpoint{3.761289in}{2.334100in}}%
\pgfpathmoveto{\pgfqpoint{3.761289in}{2.329842in}}%
\pgfpathlineto{\pgfqpoint{3.761289in}{2.329842in}}%
\pgfpathlineto{\pgfqpoint{3.761289in}{2.334100in}}%
\pgfpathlineto{\pgfqpoint{3.765547in}{2.334100in}}%
\pgfpathlineto{\pgfqpoint{3.765547in}{2.329842in}}%
\pgfpathmoveto{\pgfqpoint{3.761289in}{2.334100in}}%
\pgfpathlineto{\pgfqpoint{3.761289in}{2.334100in}}%
\pgfpathlineto{\pgfqpoint{3.761289in}{2.338358in}}%
\pgfpathlineto{\pgfqpoint{3.765547in}{2.338358in}}%
\pgfpathlineto{\pgfqpoint{3.765547in}{2.334100in}}%
\pgfpathmoveto{\pgfqpoint{3.757031in}{2.338358in}}%
\pgfpathlineto{\pgfqpoint{3.757031in}{2.338358in}}%
\pgfpathlineto{\pgfqpoint{3.757031in}{2.342616in}}%
\pgfpathlineto{\pgfqpoint{3.761289in}{2.342616in}}%
\pgfpathlineto{\pgfqpoint{3.761289in}{2.338358in}}%
\pgfpathmoveto{\pgfqpoint{3.757031in}{2.342616in}}%
\pgfpathlineto{\pgfqpoint{3.757031in}{2.342616in}}%
\pgfpathlineto{\pgfqpoint{3.757031in}{2.346874in}}%
\pgfpathlineto{\pgfqpoint{3.761289in}{2.346874in}}%
\pgfpathlineto{\pgfqpoint{3.761289in}{2.342616in}}%
\pgfpathmoveto{\pgfqpoint{3.761289in}{2.338358in}}%
\pgfpathlineto{\pgfqpoint{3.761289in}{2.338358in}}%
\pgfpathlineto{\pgfqpoint{3.761289in}{2.342616in}}%
\pgfpathlineto{\pgfqpoint{3.765547in}{2.342616in}}%
\pgfpathlineto{\pgfqpoint{3.765547in}{2.338358in}}%
\pgfpathmoveto{\pgfqpoint{3.761289in}{2.342616in}}%
\pgfpathlineto{\pgfqpoint{3.761289in}{2.342616in}}%
\pgfpathlineto{\pgfqpoint{3.761289in}{2.346874in}}%
\pgfpathlineto{\pgfqpoint{3.765547in}{2.346874in}}%
\pgfpathlineto{\pgfqpoint{3.765547in}{2.342616in}}%
\pgfpathmoveto{\pgfqpoint{3.757031in}{2.346874in}}%
\pgfpathlineto{\pgfqpoint{3.757031in}{2.346874in}}%
\pgfpathlineto{\pgfqpoint{3.757031in}{2.351132in}}%
\pgfpathlineto{\pgfqpoint{3.761289in}{2.351132in}}%
\pgfpathlineto{\pgfqpoint{3.761289in}{2.346874in}}%
\pgfpathmoveto{\pgfqpoint{3.757031in}{2.351132in}}%
\pgfpathlineto{\pgfqpoint{3.757031in}{2.351132in}}%
\pgfpathlineto{\pgfqpoint{3.757031in}{2.355389in}}%
\pgfpathlineto{\pgfqpoint{3.761289in}{2.355389in}}%
\pgfpathlineto{\pgfqpoint{3.761289in}{2.351132in}}%
\pgfpathmoveto{\pgfqpoint{3.761289in}{2.346874in}}%
\pgfpathlineto{\pgfqpoint{3.761289in}{2.346874in}}%
\pgfpathlineto{\pgfqpoint{3.761289in}{2.351132in}}%
\pgfpathlineto{\pgfqpoint{3.765547in}{2.351132in}}%
\pgfpathlineto{\pgfqpoint{3.765547in}{2.346874in}}%
\pgfpathmoveto{\pgfqpoint{3.761289in}{2.351132in}}%
\pgfpathlineto{\pgfqpoint{3.761289in}{2.351132in}}%
\pgfpathlineto{\pgfqpoint{3.761289in}{2.355389in}}%
\pgfpathlineto{\pgfqpoint{3.765547in}{2.355389in}}%
\pgfpathlineto{\pgfqpoint{3.765547in}{2.351132in}}%
\pgfpathmoveto{\pgfqpoint{3.757031in}{2.355389in}}%
\pgfpathlineto{\pgfqpoint{3.757031in}{2.355389in}}%
\pgfpathlineto{\pgfqpoint{3.757031in}{2.359647in}}%
\pgfpathlineto{\pgfqpoint{3.761289in}{2.359647in}}%
\pgfpathlineto{\pgfqpoint{3.761289in}{2.355389in}}%
\pgfpathmoveto{\pgfqpoint{3.761289in}{2.355389in}}%
\pgfpathlineto{\pgfqpoint{3.761289in}{2.355389in}}%
\pgfpathlineto{\pgfqpoint{3.761289in}{2.359647in}}%
\pgfpathlineto{\pgfqpoint{3.765547in}{2.359647in}}%
\pgfpathlineto{\pgfqpoint{3.765547in}{2.355389in}}%
\pgfpathmoveto{\pgfqpoint{3.761289in}{2.359647in}}%
\pgfpathlineto{\pgfqpoint{3.761289in}{2.359647in}}%
\pgfpathlineto{\pgfqpoint{3.761289in}{2.363905in}}%
\pgfpathlineto{\pgfqpoint{3.765547in}{2.363905in}}%
\pgfpathlineto{\pgfqpoint{3.765547in}{2.359647in}}%
\pgfpathmoveto{\pgfqpoint{3.765547in}{2.346874in}}%
\pgfpathlineto{\pgfqpoint{3.765547in}{2.346874in}}%
\pgfpathlineto{\pgfqpoint{3.765547in}{2.351132in}}%
\pgfpathlineto{\pgfqpoint{3.769805in}{2.351132in}}%
\pgfpathlineto{\pgfqpoint{3.769805in}{2.346874in}}%
\pgfpathmoveto{\pgfqpoint{3.765547in}{2.351132in}}%
\pgfpathlineto{\pgfqpoint{3.765547in}{2.351132in}}%
\pgfpathlineto{\pgfqpoint{3.765547in}{2.355389in}}%
\pgfpathlineto{\pgfqpoint{3.769805in}{2.355389in}}%
\pgfpathlineto{\pgfqpoint{3.769805in}{2.351132in}}%
\pgfpathmoveto{\pgfqpoint{3.765547in}{2.355389in}}%
\pgfpathlineto{\pgfqpoint{3.765547in}{2.355389in}}%
\pgfpathlineto{\pgfqpoint{3.765547in}{2.359647in}}%
\pgfpathlineto{\pgfqpoint{3.769805in}{2.359647in}}%
\pgfpathlineto{\pgfqpoint{3.769805in}{2.355389in}}%
\pgfpathmoveto{\pgfqpoint{3.765547in}{2.359647in}}%
\pgfpathlineto{\pgfqpoint{3.765547in}{2.359647in}}%
\pgfpathlineto{\pgfqpoint{3.765547in}{2.363905in}}%
\pgfpathlineto{\pgfqpoint{3.769805in}{2.363905in}}%
\pgfpathlineto{\pgfqpoint{3.769805in}{2.359647in}}%
\pgfpathmoveto{\pgfqpoint{3.761289in}{2.363905in}}%
\pgfpathlineto{\pgfqpoint{3.761289in}{2.363905in}}%
\pgfpathlineto{\pgfqpoint{3.761289in}{2.368163in}}%
\pgfpathlineto{\pgfqpoint{3.765547in}{2.368163in}}%
\pgfpathlineto{\pgfqpoint{3.765547in}{2.363905in}}%
\pgfpathmoveto{\pgfqpoint{3.761289in}{2.368163in}}%
\pgfpathlineto{\pgfqpoint{3.761289in}{2.368163in}}%
\pgfpathlineto{\pgfqpoint{3.761289in}{2.372421in}}%
\pgfpathlineto{\pgfqpoint{3.765547in}{2.372421in}}%
\pgfpathlineto{\pgfqpoint{3.765547in}{2.368163in}}%
\pgfpathmoveto{\pgfqpoint{3.761289in}{2.372421in}}%
\pgfpathlineto{\pgfqpoint{3.761289in}{2.372421in}}%
\pgfpathlineto{\pgfqpoint{3.761289in}{2.376679in}}%
\pgfpathlineto{\pgfqpoint{3.765547in}{2.376679in}}%
\pgfpathlineto{\pgfqpoint{3.765547in}{2.372421in}}%
\pgfpathmoveto{\pgfqpoint{3.761289in}{2.376679in}}%
\pgfpathlineto{\pgfqpoint{3.761289in}{2.376679in}}%
\pgfpathlineto{\pgfqpoint{3.761289in}{2.380937in}}%
\pgfpathlineto{\pgfqpoint{3.765547in}{2.380937in}}%
\pgfpathlineto{\pgfqpoint{3.765547in}{2.376679in}}%
\pgfpathmoveto{\pgfqpoint{3.765547in}{2.363905in}}%
\pgfpathlineto{\pgfqpoint{3.765547in}{2.363905in}}%
\pgfpathlineto{\pgfqpoint{3.765547in}{2.368163in}}%
\pgfpathlineto{\pgfqpoint{3.769805in}{2.368163in}}%
\pgfpathlineto{\pgfqpoint{3.769805in}{2.363905in}}%
\pgfpathmoveto{\pgfqpoint{3.765547in}{2.368163in}}%
\pgfpathlineto{\pgfqpoint{3.765547in}{2.368163in}}%
\pgfpathlineto{\pgfqpoint{3.765547in}{2.372421in}}%
\pgfpathlineto{\pgfqpoint{3.769805in}{2.372421in}}%
\pgfpathlineto{\pgfqpoint{3.769805in}{2.368163in}}%
\pgfpathmoveto{\pgfqpoint{3.765547in}{2.372421in}}%
\pgfpathlineto{\pgfqpoint{3.765547in}{2.372421in}}%
\pgfpathlineto{\pgfqpoint{3.765547in}{2.376679in}}%
\pgfpathlineto{\pgfqpoint{3.769805in}{2.376679in}}%
\pgfpathlineto{\pgfqpoint{3.769805in}{2.372421in}}%
\pgfpathmoveto{\pgfqpoint{3.765547in}{2.376679in}}%
\pgfpathlineto{\pgfqpoint{3.765547in}{2.376679in}}%
\pgfpathlineto{\pgfqpoint{3.765547in}{2.380937in}}%
\pgfpathlineto{\pgfqpoint{3.769805in}{2.380937in}}%
\pgfpathlineto{\pgfqpoint{3.769805in}{2.376679in}}%
\pgfpathmoveto{\pgfqpoint{3.769805in}{2.372421in}}%
\pgfpathlineto{\pgfqpoint{3.769805in}{2.372421in}}%
\pgfpathlineto{\pgfqpoint{3.769805in}{2.376679in}}%
\pgfpathlineto{\pgfqpoint{3.774063in}{2.376679in}}%
\pgfpathlineto{\pgfqpoint{3.774063in}{2.372421in}}%
\pgfpathmoveto{\pgfqpoint{3.769805in}{2.376679in}}%
\pgfpathlineto{\pgfqpoint{3.769805in}{2.376679in}}%
\pgfpathlineto{\pgfqpoint{3.769805in}{2.380937in}}%
\pgfpathlineto{\pgfqpoint{3.774063in}{2.380937in}}%
\pgfpathlineto{\pgfqpoint{3.774063in}{2.376679in}}%
\pgfpathmoveto{\pgfqpoint{3.761289in}{2.380937in}}%
\pgfpathlineto{\pgfqpoint{3.761289in}{2.380937in}}%
\pgfpathlineto{\pgfqpoint{3.761289in}{2.385195in}}%
\pgfpathlineto{\pgfqpoint{3.765547in}{2.385195in}}%
\pgfpathlineto{\pgfqpoint{3.765547in}{2.380937in}}%
\pgfpathmoveto{\pgfqpoint{3.765547in}{2.380937in}}%
\pgfpathlineto{\pgfqpoint{3.765547in}{2.380937in}}%
\pgfpathlineto{\pgfqpoint{3.765547in}{2.385195in}}%
\pgfpathlineto{\pgfqpoint{3.769805in}{2.385195in}}%
\pgfpathlineto{\pgfqpoint{3.769805in}{2.380937in}}%
\pgfpathmoveto{\pgfqpoint{3.765547in}{2.385195in}}%
\pgfpathlineto{\pgfqpoint{3.765547in}{2.385195in}}%
\pgfpathlineto{\pgfqpoint{3.765547in}{2.389453in}}%
\pgfpathlineto{\pgfqpoint{3.769805in}{2.389453in}}%
\pgfpathlineto{\pgfqpoint{3.769805in}{2.385195in}}%
\pgfpathmoveto{\pgfqpoint{3.769805in}{2.380937in}}%
\pgfpathlineto{\pgfqpoint{3.769805in}{2.380937in}}%
\pgfpathlineto{\pgfqpoint{3.769805in}{2.385195in}}%
\pgfpathlineto{\pgfqpoint{3.774063in}{2.385195in}}%
\pgfpathlineto{\pgfqpoint{3.774063in}{2.380937in}}%
\pgfpathmoveto{\pgfqpoint{3.769805in}{2.385195in}}%
\pgfpathlineto{\pgfqpoint{3.769805in}{2.385195in}}%
\pgfpathlineto{\pgfqpoint{3.769805in}{2.389453in}}%
\pgfpathlineto{\pgfqpoint{3.774063in}{2.389453in}}%
\pgfpathlineto{\pgfqpoint{3.774063in}{2.385195in}}%
\pgfpathmoveto{\pgfqpoint{3.765547in}{2.389453in}}%
\pgfpathlineto{\pgfqpoint{3.765547in}{2.389453in}}%
\pgfpathlineto{\pgfqpoint{3.765547in}{2.393711in}}%
\pgfpathlineto{\pgfqpoint{3.769805in}{2.393711in}}%
\pgfpathlineto{\pgfqpoint{3.769805in}{2.389453in}}%
\pgfpathmoveto{\pgfqpoint{3.765547in}{2.393711in}}%
\pgfpathlineto{\pgfqpoint{3.765547in}{2.393711in}}%
\pgfpathlineto{\pgfqpoint{3.765547in}{2.397969in}}%
\pgfpathlineto{\pgfqpoint{3.769805in}{2.397969in}}%
\pgfpathlineto{\pgfqpoint{3.769805in}{2.393711in}}%
\pgfpathmoveto{\pgfqpoint{3.769805in}{2.389453in}}%
\pgfpathlineto{\pgfqpoint{3.769805in}{2.389453in}}%
\pgfpathlineto{\pgfqpoint{3.769805in}{2.393711in}}%
\pgfpathlineto{\pgfqpoint{3.774063in}{2.393711in}}%
\pgfpathlineto{\pgfqpoint{3.774063in}{2.389453in}}%
\pgfpathmoveto{\pgfqpoint{3.769805in}{2.393711in}}%
\pgfpathlineto{\pgfqpoint{3.769805in}{2.393711in}}%
\pgfpathlineto{\pgfqpoint{3.769805in}{2.397969in}}%
\pgfpathlineto{\pgfqpoint{3.774063in}{2.397969in}}%
\pgfpathlineto{\pgfqpoint{3.774063in}{2.393711in}}%
\pgfpathmoveto{\pgfqpoint{3.765547in}{2.397969in}}%
\pgfpathlineto{\pgfqpoint{3.765547in}{2.397969in}}%
\pgfpathlineto{\pgfqpoint{3.765547in}{2.402226in}}%
\pgfpathlineto{\pgfqpoint{3.769805in}{2.402226in}}%
\pgfpathlineto{\pgfqpoint{3.769805in}{2.397969in}}%
\pgfpathmoveto{\pgfqpoint{3.765547in}{2.402226in}}%
\pgfpathlineto{\pgfqpoint{3.765547in}{2.402226in}}%
\pgfpathlineto{\pgfqpoint{3.765547in}{2.406484in}}%
\pgfpathlineto{\pgfqpoint{3.769805in}{2.406484in}}%
\pgfpathlineto{\pgfqpoint{3.769805in}{2.402226in}}%
\pgfpathmoveto{\pgfqpoint{3.769805in}{2.397969in}}%
\pgfpathlineto{\pgfqpoint{3.769805in}{2.397969in}}%
\pgfpathlineto{\pgfqpoint{3.769805in}{2.402226in}}%
\pgfpathlineto{\pgfqpoint{3.774063in}{2.402226in}}%
\pgfpathlineto{\pgfqpoint{3.774063in}{2.397969in}}%
\pgfpathmoveto{\pgfqpoint{3.769805in}{2.402226in}}%
\pgfpathlineto{\pgfqpoint{3.769805in}{2.402226in}}%
\pgfpathlineto{\pgfqpoint{3.769805in}{2.406484in}}%
\pgfpathlineto{\pgfqpoint{3.774063in}{2.406484in}}%
\pgfpathlineto{\pgfqpoint{3.774063in}{2.402226in}}%
\pgfpathmoveto{\pgfqpoint{3.765547in}{2.406484in}}%
\pgfpathlineto{\pgfqpoint{3.765547in}{2.406484in}}%
\pgfpathlineto{\pgfqpoint{3.765547in}{2.410742in}}%
\pgfpathlineto{\pgfqpoint{3.769805in}{2.410742in}}%
\pgfpathlineto{\pgfqpoint{3.769805in}{2.406484in}}%
\pgfpathmoveto{\pgfqpoint{3.769805in}{2.406484in}}%
\pgfpathlineto{\pgfqpoint{3.769805in}{2.406484in}}%
\pgfpathlineto{\pgfqpoint{3.769805in}{2.410742in}}%
\pgfpathlineto{\pgfqpoint{3.774063in}{2.410742in}}%
\pgfpathlineto{\pgfqpoint{3.774063in}{2.406484in}}%
\pgfpathmoveto{\pgfqpoint{3.769805in}{2.410742in}}%
\pgfpathlineto{\pgfqpoint{3.769805in}{2.410742in}}%
\pgfpathlineto{\pgfqpoint{3.769805in}{2.415000in}}%
\pgfpathlineto{\pgfqpoint{3.774063in}{2.415000in}}%
\pgfpathlineto{\pgfqpoint{3.774063in}{2.410742in}}%
\pgfpathmoveto{\pgfqpoint{3.769805in}{2.415000in}}%
\pgfpathlineto{\pgfqpoint{3.769805in}{2.415000in}}%
\pgfpathlineto{\pgfqpoint{3.769805in}{2.419258in}}%
\pgfpathlineto{\pgfqpoint{3.774063in}{2.419258in}}%
\pgfpathlineto{\pgfqpoint{3.774063in}{2.415000in}}%
\pgfpathmoveto{\pgfqpoint{3.769805in}{2.419258in}}%
\pgfpathlineto{\pgfqpoint{3.769805in}{2.419258in}}%
\pgfpathlineto{\pgfqpoint{3.769805in}{2.423516in}}%
\pgfpathlineto{\pgfqpoint{3.774063in}{2.423516in}}%
\pgfpathlineto{\pgfqpoint{3.774063in}{2.419258in}}%
\pgfpathmoveto{\pgfqpoint{3.769805in}{2.423516in}}%
\pgfpathlineto{\pgfqpoint{3.769805in}{2.423516in}}%
\pgfpathlineto{\pgfqpoint{3.769805in}{2.427773in}}%
\pgfpathlineto{\pgfqpoint{3.774063in}{2.427773in}}%
\pgfpathlineto{\pgfqpoint{3.774063in}{2.423516in}}%
\pgfpathmoveto{\pgfqpoint{3.769805in}{2.427773in}}%
\pgfpathlineto{\pgfqpoint{3.769805in}{2.427773in}}%
\pgfpathlineto{\pgfqpoint{3.769805in}{2.432031in}}%
\pgfpathlineto{\pgfqpoint{3.774063in}{2.432031in}}%
\pgfpathlineto{\pgfqpoint{3.774063in}{2.427773in}}%
\pgfpathmoveto{\pgfqpoint{3.769805in}{2.432031in}}%
\pgfpathlineto{\pgfqpoint{3.769805in}{2.432031in}}%
\pgfpathlineto{\pgfqpoint{3.769805in}{2.436289in}}%
\pgfpathlineto{\pgfqpoint{3.774063in}{2.436289in}}%
\pgfpathlineto{\pgfqpoint{3.774063in}{2.432031in}}%
\pgfpathmoveto{\pgfqpoint{3.774063in}{2.397969in}}%
\pgfpathlineto{\pgfqpoint{3.774063in}{2.397969in}}%
\pgfpathlineto{\pgfqpoint{3.774063in}{2.402226in}}%
\pgfpathlineto{\pgfqpoint{3.778320in}{2.402226in}}%
\pgfpathlineto{\pgfqpoint{3.778320in}{2.397969in}}%
\pgfpathmoveto{\pgfqpoint{3.774063in}{2.402226in}}%
\pgfpathlineto{\pgfqpoint{3.774063in}{2.402226in}}%
\pgfpathlineto{\pgfqpoint{3.774063in}{2.406484in}}%
\pgfpathlineto{\pgfqpoint{3.778320in}{2.406484in}}%
\pgfpathlineto{\pgfqpoint{3.778320in}{2.402226in}}%
\pgfpathmoveto{\pgfqpoint{3.774063in}{2.406484in}}%
\pgfpathlineto{\pgfqpoint{3.774063in}{2.406484in}}%
\pgfpathlineto{\pgfqpoint{3.774063in}{2.410742in}}%
\pgfpathlineto{\pgfqpoint{3.778320in}{2.410742in}}%
\pgfpathlineto{\pgfqpoint{3.778320in}{2.406484in}}%
\pgfpathmoveto{\pgfqpoint{3.774063in}{2.410742in}}%
\pgfpathlineto{\pgfqpoint{3.774063in}{2.410742in}}%
\pgfpathlineto{\pgfqpoint{3.774063in}{2.415000in}}%
\pgfpathlineto{\pgfqpoint{3.778320in}{2.415000in}}%
\pgfpathlineto{\pgfqpoint{3.778320in}{2.410742in}}%
\pgfpathmoveto{\pgfqpoint{3.774063in}{2.415000in}}%
\pgfpathlineto{\pgfqpoint{3.774063in}{2.415000in}}%
\pgfpathlineto{\pgfqpoint{3.774063in}{2.419258in}}%
\pgfpathlineto{\pgfqpoint{3.778320in}{2.419258in}}%
\pgfpathlineto{\pgfqpoint{3.778320in}{2.415000in}}%
\pgfpathmoveto{\pgfqpoint{3.774063in}{2.419258in}}%
\pgfpathlineto{\pgfqpoint{3.774063in}{2.419258in}}%
\pgfpathlineto{\pgfqpoint{3.774063in}{2.423516in}}%
\pgfpathlineto{\pgfqpoint{3.778320in}{2.423516in}}%
\pgfpathlineto{\pgfqpoint{3.778320in}{2.419258in}}%
\pgfpathmoveto{\pgfqpoint{3.774063in}{2.423516in}}%
\pgfpathlineto{\pgfqpoint{3.774063in}{2.423516in}}%
\pgfpathlineto{\pgfqpoint{3.774063in}{2.427773in}}%
\pgfpathlineto{\pgfqpoint{3.778320in}{2.427773in}}%
\pgfpathlineto{\pgfqpoint{3.778320in}{2.423516in}}%
\pgfpathmoveto{\pgfqpoint{3.774063in}{2.427773in}}%
\pgfpathlineto{\pgfqpoint{3.774063in}{2.427773in}}%
\pgfpathlineto{\pgfqpoint{3.774063in}{2.432031in}}%
\pgfpathlineto{\pgfqpoint{3.778320in}{2.432031in}}%
\pgfpathlineto{\pgfqpoint{3.778320in}{2.427773in}}%
\pgfpathmoveto{\pgfqpoint{3.778320in}{2.423516in}}%
\pgfpathlineto{\pgfqpoint{3.778320in}{2.423516in}}%
\pgfpathlineto{\pgfqpoint{3.778320in}{2.427773in}}%
\pgfpathlineto{\pgfqpoint{3.782578in}{2.427773in}}%
\pgfpathlineto{\pgfqpoint{3.782578in}{2.423516in}}%
\pgfpathmoveto{\pgfqpoint{3.778320in}{2.427773in}}%
\pgfpathlineto{\pgfqpoint{3.778320in}{2.427773in}}%
\pgfpathlineto{\pgfqpoint{3.778320in}{2.432031in}}%
\pgfpathlineto{\pgfqpoint{3.782578in}{2.432031in}}%
\pgfpathlineto{\pgfqpoint{3.782578in}{2.427773in}}%
\pgfpathmoveto{\pgfqpoint{3.774063in}{2.432031in}}%
\pgfpathlineto{\pgfqpoint{3.774063in}{2.432031in}}%
\pgfpathlineto{\pgfqpoint{3.774063in}{2.436289in}}%
\pgfpathlineto{\pgfqpoint{3.778320in}{2.436289in}}%
\pgfpathlineto{\pgfqpoint{3.778320in}{2.432031in}}%
\pgfpathmoveto{\pgfqpoint{3.774063in}{2.436289in}}%
\pgfpathlineto{\pgfqpoint{3.774063in}{2.436289in}}%
\pgfpathlineto{\pgfqpoint{3.774063in}{2.440547in}}%
\pgfpathlineto{\pgfqpoint{3.778320in}{2.440547in}}%
\pgfpathlineto{\pgfqpoint{3.778320in}{2.436289in}}%
\pgfpathmoveto{\pgfqpoint{3.778320in}{2.432031in}}%
\pgfpathlineto{\pgfqpoint{3.778320in}{2.432031in}}%
\pgfpathlineto{\pgfqpoint{3.778320in}{2.436289in}}%
\pgfpathlineto{\pgfqpoint{3.782578in}{2.436289in}}%
\pgfpathlineto{\pgfqpoint{3.782578in}{2.432031in}}%
\pgfpathmoveto{\pgfqpoint{3.778320in}{2.436289in}}%
\pgfpathlineto{\pgfqpoint{3.778320in}{2.436289in}}%
\pgfpathlineto{\pgfqpoint{3.778320in}{2.440547in}}%
\pgfpathlineto{\pgfqpoint{3.782578in}{2.440547in}}%
\pgfpathlineto{\pgfqpoint{3.782578in}{2.436289in}}%
\pgfpathmoveto{\pgfqpoint{3.774063in}{2.440547in}}%
\pgfpathlineto{\pgfqpoint{3.774063in}{2.440547in}}%
\pgfpathlineto{\pgfqpoint{3.774063in}{2.444804in}}%
\pgfpathlineto{\pgfqpoint{3.778320in}{2.444804in}}%
\pgfpathlineto{\pgfqpoint{3.778320in}{2.440547in}}%
\pgfpathmoveto{\pgfqpoint{3.774063in}{2.444804in}}%
\pgfpathlineto{\pgfqpoint{3.774063in}{2.444804in}}%
\pgfpathlineto{\pgfqpoint{3.774063in}{2.449062in}}%
\pgfpathlineto{\pgfqpoint{3.778320in}{2.449062in}}%
\pgfpathlineto{\pgfqpoint{3.778320in}{2.444804in}}%
\pgfpathmoveto{\pgfqpoint{3.778320in}{2.440547in}}%
\pgfpathlineto{\pgfqpoint{3.778320in}{2.440547in}}%
\pgfpathlineto{\pgfqpoint{3.778320in}{2.444804in}}%
\pgfpathlineto{\pgfqpoint{3.782578in}{2.444804in}}%
\pgfpathlineto{\pgfqpoint{3.782578in}{2.440547in}}%
\pgfpathmoveto{\pgfqpoint{3.778320in}{2.444804in}}%
\pgfpathlineto{\pgfqpoint{3.778320in}{2.444804in}}%
\pgfpathlineto{\pgfqpoint{3.778320in}{2.449062in}}%
\pgfpathlineto{\pgfqpoint{3.782578in}{2.449062in}}%
\pgfpathlineto{\pgfqpoint{3.782578in}{2.444804in}}%
\pgfpathmoveto{\pgfqpoint{3.774063in}{2.449062in}}%
\pgfpathlineto{\pgfqpoint{3.774063in}{2.449062in}}%
\pgfpathlineto{\pgfqpoint{3.774063in}{2.453320in}}%
\pgfpathlineto{\pgfqpoint{3.778320in}{2.453320in}}%
\pgfpathlineto{\pgfqpoint{3.778320in}{2.449062in}}%
\pgfpathmoveto{\pgfqpoint{3.774063in}{2.453320in}}%
\pgfpathlineto{\pgfqpoint{3.774063in}{2.453320in}}%
\pgfpathlineto{\pgfqpoint{3.774063in}{2.457577in}}%
\pgfpathlineto{\pgfqpoint{3.778320in}{2.457577in}}%
\pgfpathlineto{\pgfqpoint{3.778320in}{2.453320in}}%
\pgfpathmoveto{\pgfqpoint{3.778320in}{2.449062in}}%
\pgfpathlineto{\pgfqpoint{3.778320in}{2.449062in}}%
\pgfpathlineto{\pgfqpoint{3.778320in}{2.453320in}}%
\pgfpathlineto{\pgfqpoint{3.782578in}{2.453320in}}%
\pgfpathlineto{\pgfqpoint{3.782578in}{2.449062in}}%
\pgfpathmoveto{\pgfqpoint{3.778320in}{2.453320in}}%
\pgfpathlineto{\pgfqpoint{3.778320in}{2.453320in}}%
\pgfpathlineto{\pgfqpoint{3.778320in}{2.457577in}}%
\pgfpathlineto{\pgfqpoint{3.782578in}{2.457577in}}%
\pgfpathlineto{\pgfqpoint{3.782578in}{2.453320in}}%
\pgfpathmoveto{\pgfqpoint{3.774063in}{2.457577in}}%
\pgfpathlineto{\pgfqpoint{3.774063in}{2.457577in}}%
\pgfpathlineto{\pgfqpoint{3.774063in}{2.461835in}}%
\pgfpathlineto{\pgfqpoint{3.778320in}{2.461835in}}%
\pgfpathlineto{\pgfqpoint{3.778320in}{2.457577in}}%
\pgfpathmoveto{\pgfqpoint{3.778320in}{2.457577in}}%
\pgfpathlineto{\pgfqpoint{3.778320in}{2.457577in}}%
\pgfpathlineto{\pgfqpoint{3.778320in}{2.461835in}}%
\pgfpathlineto{\pgfqpoint{3.782578in}{2.461835in}}%
\pgfpathlineto{\pgfqpoint{3.782578in}{2.457577in}}%
\pgfpathmoveto{\pgfqpoint{3.778320in}{2.461835in}}%
\pgfpathlineto{\pgfqpoint{3.778320in}{2.461835in}}%
\pgfpathlineto{\pgfqpoint{3.778320in}{2.466093in}}%
\pgfpathlineto{\pgfqpoint{3.782578in}{2.466093in}}%
\pgfpathlineto{\pgfqpoint{3.782578in}{2.461835in}}%
\pgfpathmoveto{\pgfqpoint{3.782578in}{2.449062in}}%
\pgfpathlineto{\pgfqpoint{3.782578in}{2.449062in}}%
\pgfpathlineto{\pgfqpoint{3.782578in}{2.453320in}}%
\pgfpathlineto{\pgfqpoint{3.786836in}{2.453320in}}%
\pgfpathlineto{\pgfqpoint{3.786836in}{2.449062in}}%
\pgfpathmoveto{\pgfqpoint{3.782578in}{2.453320in}}%
\pgfpathlineto{\pgfqpoint{3.782578in}{2.453320in}}%
\pgfpathlineto{\pgfqpoint{3.782578in}{2.457577in}}%
\pgfpathlineto{\pgfqpoint{3.786836in}{2.457577in}}%
\pgfpathlineto{\pgfqpoint{3.786836in}{2.453320in}}%
\pgfpathmoveto{\pgfqpoint{3.782578in}{2.457577in}}%
\pgfpathlineto{\pgfqpoint{3.782578in}{2.457577in}}%
\pgfpathlineto{\pgfqpoint{3.782578in}{2.461835in}}%
\pgfpathlineto{\pgfqpoint{3.786836in}{2.461835in}}%
\pgfpathlineto{\pgfqpoint{3.786836in}{2.457577in}}%
\pgfpathmoveto{\pgfqpoint{3.782578in}{2.461835in}}%
\pgfpathlineto{\pgfqpoint{3.782578in}{2.461835in}}%
\pgfpathlineto{\pgfqpoint{3.782578in}{2.466093in}}%
\pgfpathlineto{\pgfqpoint{3.786836in}{2.466093in}}%
\pgfpathlineto{\pgfqpoint{3.786836in}{2.461835in}}%
\pgfpathmoveto{\pgfqpoint{3.778320in}{2.466093in}}%
\pgfpathlineto{\pgfqpoint{3.778320in}{2.466093in}}%
\pgfpathlineto{\pgfqpoint{3.778320in}{2.470351in}}%
\pgfpathlineto{\pgfqpoint{3.782578in}{2.470351in}}%
\pgfpathlineto{\pgfqpoint{3.782578in}{2.466093in}}%
\pgfpathmoveto{\pgfqpoint{3.778320in}{2.470351in}}%
\pgfpathlineto{\pgfqpoint{3.778320in}{2.470351in}}%
\pgfpathlineto{\pgfqpoint{3.778320in}{2.474608in}}%
\pgfpathlineto{\pgfqpoint{3.782578in}{2.474608in}}%
\pgfpathlineto{\pgfqpoint{3.782578in}{2.470351in}}%
\pgfpathmoveto{\pgfqpoint{3.778320in}{2.474608in}}%
\pgfpathlineto{\pgfqpoint{3.778320in}{2.474608in}}%
\pgfpathlineto{\pgfqpoint{3.778320in}{2.478866in}}%
\pgfpathlineto{\pgfqpoint{3.782578in}{2.478866in}}%
\pgfpathlineto{\pgfqpoint{3.782578in}{2.474608in}}%
\pgfpathmoveto{\pgfqpoint{3.778320in}{2.478866in}}%
\pgfpathlineto{\pgfqpoint{3.778320in}{2.478866in}}%
\pgfpathlineto{\pgfqpoint{3.778320in}{2.483124in}}%
\pgfpathlineto{\pgfqpoint{3.782578in}{2.483124in}}%
\pgfpathlineto{\pgfqpoint{3.782578in}{2.478866in}}%
\pgfpathmoveto{\pgfqpoint{3.782578in}{2.466093in}}%
\pgfpathlineto{\pgfqpoint{3.782578in}{2.466093in}}%
\pgfpathlineto{\pgfqpoint{3.782578in}{2.470351in}}%
\pgfpathlineto{\pgfqpoint{3.786836in}{2.470351in}}%
\pgfpathlineto{\pgfqpoint{3.786836in}{2.466093in}}%
\pgfpathmoveto{\pgfqpoint{3.782578in}{2.470351in}}%
\pgfpathlineto{\pgfqpoint{3.782578in}{2.470351in}}%
\pgfpathlineto{\pgfqpoint{3.782578in}{2.474608in}}%
\pgfpathlineto{\pgfqpoint{3.786836in}{2.474608in}}%
\pgfpathlineto{\pgfqpoint{3.786836in}{2.470351in}}%
\pgfpathmoveto{\pgfqpoint{3.782578in}{2.474608in}}%
\pgfpathlineto{\pgfqpoint{3.782578in}{2.474608in}}%
\pgfpathlineto{\pgfqpoint{3.782578in}{2.478866in}}%
\pgfpathlineto{\pgfqpoint{3.786836in}{2.478866in}}%
\pgfpathlineto{\pgfqpoint{3.786836in}{2.474608in}}%
\pgfpathmoveto{\pgfqpoint{3.782578in}{2.478866in}}%
\pgfpathlineto{\pgfqpoint{3.782578in}{2.478866in}}%
\pgfpathlineto{\pgfqpoint{3.782578in}{2.483124in}}%
\pgfpathlineto{\pgfqpoint{3.786836in}{2.483124in}}%
\pgfpathlineto{\pgfqpoint{3.786836in}{2.478866in}}%
\pgfpathmoveto{\pgfqpoint{3.786836in}{2.474608in}}%
\pgfpathlineto{\pgfqpoint{3.786836in}{2.474608in}}%
\pgfpathlineto{\pgfqpoint{3.786836in}{2.478866in}}%
\pgfpathlineto{\pgfqpoint{3.791094in}{2.478866in}}%
\pgfpathlineto{\pgfqpoint{3.791094in}{2.474608in}}%
\pgfpathmoveto{\pgfqpoint{3.786836in}{2.478866in}}%
\pgfpathlineto{\pgfqpoint{3.786836in}{2.478866in}}%
\pgfpathlineto{\pgfqpoint{3.786836in}{2.483124in}}%
\pgfpathlineto{\pgfqpoint{3.791094in}{2.483124in}}%
\pgfpathlineto{\pgfqpoint{3.791094in}{2.478866in}}%
\pgfpathmoveto{\pgfqpoint{3.778320in}{2.483124in}}%
\pgfpathlineto{\pgfqpoint{3.778320in}{2.483124in}}%
\pgfpathlineto{\pgfqpoint{3.778320in}{2.487381in}}%
\pgfpathlineto{\pgfqpoint{3.782578in}{2.487381in}}%
\pgfpathlineto{\pgfqpoint{3.782578in}{2.483124in}}%
\pgfpathmoveto{\pgfqpoint{3.782578in}{2.483124in}}%
\pgfpathlineto{\pgfqpoint{3.782578in}{2.483124in}}%
\pgfpathlineto{\pgfqpoint{3.782578in}{2.487381in}}%
\pgfpathlineto{\pgfqpoint{3.786836in}{2.487381in}}%
\pgfpathlineto{\pgfqpoint{3.786836in}{2.483124in}}%
\pgfpathmoveto{\pgfqpoint{3.782578in}{2.487381in}}%
\pgfpathlineto{\pgfqpoint{3.782578in}{2.487381in}}%
\pgfpathlineto{\pgfqpoint{3.782578in}{2.491639in}}%
\pgfpathlineto{\pgfqpoint{3.786836in}{2.491639in}}%
\pgfpathlineto{\pgfqpoint{3.786836in}{2.487381in}}%
\pgfpathmoveto{\pgfqpoint{3.786836in}{2.483124in}}%
\pgfpathlineto{\pgfqpoint{3.786836in}{2.483124in}}%
\pgfpathlineto{\pgfqpoint{3.786836in}{2.487381in}}%
\pgfpathlineto{\pgfqpoint{3.791094in}{2.487381in}}%
\pgfpathlineto{\pgfqpoint{3.791094in}{2.483124in}}%
\pgfpathmoveto{\pgfqpoint{3.786836in}{2.487381in}}%
\pgfpathlineto{\pgfqpoint{3.786836in}{2.487381in}}%
\pgfpathlineto{\pgfqpoint{3.786836in}{2.491639in}}%
\pgfpathlineto{\pgfqpoint{3.791094in}{2.491639in}}%
\pgfpathlineto{\pgfqpoint{3.791094in}{2.487381in}}%
\pgfpathmoveto{\pgfqpoint{3.782578in}{2.491639in}}%
\pgfpathlineto{\pgfqpoint{3.782578in}{2.491639in}}%
\pgfpathlineto{\pgfqpoint{3.782578in}{2.495897in}}%
\pgfpathlineto{\pgfqpoint{3.786836in}{2.495897in}}%
\pgfpathlineto{\pgfqpoint{3.786836in}{2.491639in}}%
\pgfpathmoveto{\pgfqpoint{3.782578in}{2.495897in}}%
\pgfpathlineto{\pgfqpoint{3.782578in}{2.495897in}}%
\pgfpathlineto{\pgfqpoint{3.782578in}{2.500155in}}%
\pgfpathlineto{\pgfqpoint{3.786836in}{2.500155in}}%
\pgfpathlineto{\pgfqpoint{3.786836in}{2.495897in}}%
\pgfpathmoveto{\pgfqpoint{3.786836in}{2.491639in}}%
\pgfpathlineto{\pgfqpoint{3.786836in}{2.491639in}}%
\pgfpathlineto{\pgfqpoint{3.786836in}{2.495897in}}%
\pgfpathlineto{\pgfqpoint{3.791094in}{2.495897in}}%
\pgfpathlineto{\pgfqpoint{3.791094in}{2.491639in}}%
\pgfpathmoveto{\pgfqpoint{3.786836in}{2.495897in}}%
\pgfpathlineto{\pgfqpoint{3.786836in}{2.495897in}}%
\pgfpathlineto{\pgfqpoint{3.786836in}{2.500155in}}%
\pgfpathlineto{\pgfqpoint{3.791094in}{2.500155in}}%
\pgfpathlineto{\pgfqpoint{3.791094in}{2.495897in}}%
\pgfpathmoveto{\pgfqpoint{3.782578in}{2.500155in}}%
\pgfpathlineto{\pgfqpoint{3.782578in}{2.500155in}}%
\pgfpathlineto{\pgfqpoint{3.782578in}{2.504412in}}%
\pgfpathlineto{\pgfqpoint{3.786836in}{2.504412in}}%
\pgfpathlineto{\pgfqpoint{3.786836in}{2.500155in}}%
\pgfpathmoveto{\pgfqpoint{3.782578in}{2.504412in}}%
\pgfpathlineto{\pgfqpoint{3.782578in}{2.504412in}}%
\pgfpathlineto{\pgfqpoint{3.782578in}{2.508670in}}%
\pgfpathlineto{\pgfqpoint{3.786836in}{2.508670in}}%
\pgfpathlineto{\pgfqpoint{3.786836in}{2.504412in}}%
\pgfpathmoveto{\pgfqpoint{3.786836in}{2.500155in}}%
\pgfpathlineto{\pgfqpoint{3.786836in}{2.500155in}}%
\pgfpathlineto{\pgfqpoint{3.786836in}{2.504412in}}%
\pgfpathlineto{\pgfqpoint{3.791094in}{2.504412in}}%
\pgfpathlineto{\pgfqpoint{3.791094in}{2.500155in}}%
\pgfpathmoveto{\pgfqpoint{3.786836in}{2.504412in}}%
\pgfpathlineto{\pgfqpoint{3.786836in}{2.504412in}}%
\pgfpathlineto{\pgfqpoint{3.786836in}{2.508670in}}%
\pgfpathlineto{\pgfqpoint{3.791094in}{2.508670in}}%
\pgfpathlineto{\pgfqpoint{3.791094in}{2.504412in}}%
\pgfpathmoveto{\pgfqpoint{3.782578in}{2.508670in}}%
\pgfpathlineto{\pgfqpoint{3.782578in}{2.508670in}}%
\pgfpathlineto{\pgfqpoint{3.782578in}{2.512928in}}%
\pgfpathlineto{\pgfqpoint{3.786836in}{2.512928in}}%
\pgfpathlineto{\pgfqpoint{3.786836in}{2.508670in}}%
\pgfpathmoveto{\pgfqpoint{3.786836in}{2.508670in}}%
\pgfpathlineto{\pgfqpoint{3.786836in}{2.508670in}}%
\pgfpathlineto{\pgfqpoint{3.786836in}{2.512928in}}%
\pgfpathlineto{\pgfqpoint{3.791094in}{2.512928in}}%
\pgfpathlineto{\pgfqpoint{3.791094in}{2.508670in}}%
\pgfpathmoveto{\pgfqpoint{3.786836in}{2.512928in}}%
\pgfpathlineto{\pgfqpoint{3.786836in}{2.512928in}}%
\pgfpathlineto{\pgfqpoint{3.786836in}{2.517185in}}%
\pgfpathlineto{\pgfqpoint{3.791094in}{2.517185in}}%
\pgfpathlineto{\pgfqpoint{3.791094in}{2.512928in}}%
\pgfpathmoveto{\pgfqpoint{3.791094in}{2.500155in}}%
\pgfpathlineto{\pgfqpoint{3.791094in}{2.500155in}}%
\pgfpathlineto{\pgfqpoint{3.791094in}{2.504412in}}%
\pgfpathlineto{\pgfqpoint{3.795352in}{2.504412in}}%
\pgfpathlineto{\pgfqpoint{3.795352in}{2.500155in}}%
\pgfpathmoveto{\pgfqpoint{3.791094in}{2.504412in}}%
\pgfpathlineto{\pgfqpoint{3.791094in}{2.504412in}}%
\pgfpathlineto{\pgfqpoint{3.791094in}{2.508670in}}%
\pgfpathlineto{\pgfqpoint{3.795352in}{2.508670in}}%
\pgfpathlineto{\pgfqpoint{3.795352in}{2.504412in}}%
\pgfpathmoveto{\pgfqpoint{3.791094in}{2.508670in}}%
\pgfpathlineto{\pgfqpoint{3.791094in}{2.508670in}}%
\pgfpathlineto{\pgfqpoint{3.791094in}{2.512928in}}%
\pgfpathlineto{\pgfqpoint{3.795352in}{2.512928in}}%
\pgfpathlineto{\pgfqpoint{3.795352in}{2.508670in}}%
\pgfpathmoveto{\pgfqpoint{3.791094in}{2.512928in}}%
\pgfpathlineto{\pgfqpoint{3.791094in}{2.512928in}}%
\pgfpathlineto{\pgfqpoint{3.791094in}{2.517185in}}%
\pgfpathlineto{\pgfqpoint{3.795352in}{2.517185in}}%
\pgfpathlineto{\pgfqpoint{3.795352in}{2.512928in}}%
\pgfpathmoveto{\pgfqpoint{3.786836in}{2.517185in}}%
\pgfpathlineto{\pgfqpoint{3.786836in}{2.517185in}}%
\pgfpathlineto{\pgfqpoint{3.786836in}{2.521443in}}%
\pgfpathlineto{\pgfqpoint{3.791094in}{2.521443in}}%
\pgfpathlineto{\pgfqpoint{3.791094in}{2.517185in}}%
\pgfpathmoveto{\pgfqpoint{3.786836in}{2.521443in}}%
\pgfpathlineto{\pgfqpoint{3.786836in}{2.521443in}}%
\pgfpathlineto{\pgfqpoint{3.786836in}{2.525701in}}%
\pgfpathlineto{\pgfqpoint{3.791094in}{2.525701in}}%
\pgfpathlineto{\pgfqpoint{3.791094in}{2.521443in}}%
\pgfpathmoveto{\pgfqpoint{3.786836in}{2.525701in}}%
\pgfpathlineto{\pgfqpoint{3.786836in}{2.525701in}}%
\pgfpathlineto{\pgfqpoint{3.786836in}{2.529959in}}%
\pgfpathlineto{\pgfqpoint{3.791094in}{2.529959in}}%
\pgfpathlineto{\pgfqpoint{3.791094in}{2.525701in}}%
\pgfpathmoveto{\pgfqpoint{3.786836in}{2.529959in}}%
\pgfpathlineto{\pgfqpoint{3.786836in}{2.529959in}}%
\pgfpathlineto{\pgfqpoint{3.786836in}{2.534216in}}%
\pgfpathlineto{\pgfqpoint{3.791094in}{2.534216in}}%
\pgfpathlineto{\pgfqpoint{3.791094in}{2.529959in}}%
\pgfpathmoveto{\pgfqpoint{3.786836in}{2.534216in}}%
\pgfpathlineto{\pgfqpoint{3.786836in}{2.534216in}}%
\pgfpathlineto{\pgfqpoint{3.786836in}{2.538474in}}%
\pgfpathlineto{\pgfqpoint{3.791094in}{2.538474in}}%
\pgfpathlineto{\pgfqpoint{3.791094in}{2.534216in}}%
\pgfpathmoveto{\pgfqpoint{3.791094in}{2.517185in}}%
\pgfpathlineto{\pgfqpoint{3.791094in}{2.517185in}}%
\pgfpathlineto{\pgfqpoint{3.791094in}{2.521443in}}%
\pgfpathlineto{\pgfqpoint{3.795352in}{2.521443in}}%
\pgfpathlineto{\pgfqpoint{3.795352in}{2.517185in}}%
\pgfpathmoveto{\pgfqpoint{3.791094in}{2.521443in}}%
\pgfpathlineto{\pgfqpoint{3.791094in}{2.521443in}}%
\pgfpathlineto{\pgfqpoint{3.791094in}{2.525701in}}%
\pgfpathlineto{\pgfqpoint{3.795352in}{2.525701in}}%
\pgfpathlineto{\pgfqpoint{3.795352in}{2.521443in}}%
\pgfpathmoveto{\pgfqpoint{3.791094in}{2.525701in}}%
\pgfpathlineto{\pgfqpoint{3.791094in}{2.525701in}}%
\pgfpathlineto{\pgfqpoint{3.791094in}{2.529959in}}%
\pgfpathlineto{\pgfqpoint{3.795352in}{2.529959in}}%
\pgfpathlineto{\pgfqpoint{3.795352in}{2.525701in}}%
\pgfpathmoveto{\pgfqpoint{3.791094in}{2.529959in}}%
\pgfpathlineto{\pgfqpoint{3.791094in}{2.529959in}}%
\pgfpathlineto{\pgfqpoint{3.791094in}{2.534216in}}%
\pgfpathlineto{\pgfqpoint{3.795352in}{2.534216in}}%
\pgfpathlineto{\pgfqpoint{3.795352in}{2.529959in}}%
\pgfpathmoveto{\pgfqpoint{3.795352in}{2.525701in}}%
\pgfpathlineto{\pgfqpoint{3.795352in}{2.525701in}}%
\pgfpathlineto{\pgfqpoint{3.795352in}{2.529959in}}%
\pgfpathlineto{\pgfqpoint{3.799609in}{2.529959in}}%
\pgfpathlineto{\pgfqpoint{3.799609in}{2.525701in}}%
\pgfpathmoveto{\pgfqpoint{3.795352in}{2.529959in}}%
\pgfpathlineto{\pgfqpoint{3.795352in}{2.529959in}}%
\pgfpathlineto{\pgfqpoint{3.795352in}{2.534216in}}%
\pgfpathlineto{\pgfqpoint{3.799609in}{2.534216in}}%
\pgfpathlineto{\pgfqpoint{3.799609in}{2.529959in}}%
\pgfpathmoveto{\pgfqpoint{3.791094in}{2.534216in}}%
\pgfpathlineto{\pgfqpoint{3.791094in}{2.534216in}}%
\pgfpathlineto{\pgfqpoint{3.791094in}{2.538474in}}%
\pgfpathlineto{\pgfqpoint{3.795352in}{2.538474in}}%
\pgfpathlineto{\pgfqpoint{3.795352in}{2.534216in}}%
\pgfpathmoveto{\pgfqpoint{3.791094in}{2.538474in}}%
\pgfpathlineto{\pgfqpoint{3.791094in}{2.538474in}}%
\pgfpathlineto{\pgfqpoint{3.791094in}{2.542732in}}%
\pgfpathlineto{\pgfqpoint{3.795352in}{2.542732in}}%
\pgfpathlineto{\pgfqpoint{3.795352in}{2.538474in}}%
\pgfpathmoveto{\pgfqpoint{3.795352in}{2.534216in}}%
\pgfpathlineto{\pgfqpoint{3.795352in}{2.534216in}}%
\pgfpathlineto{\pgfqpoint{3.795352in}{2.538474in}}%
\pgfpathlineto{\pgfqpoint{3.799609in}{2.538474in}}%
\pgfpathlineto{\pgfqpoint{3.799609in}{2.534216in}}%
\pgfpathmoveto{\pgfqpoint{3.795352in}{2.538474in}}%
\pgfpathlineto{\pgfqpoint{3.795352in}{2.538474in}}%
\pgfpathlineto{\pgfqpoint{3.795352in}{2.542732in}}%
\pgfpathlineto{\pgfqpoint{3.799609in}{2.542732in}}%
\pgfpathlineto{\pgfqpoint{3.799609in}{2.538474in}}%
\pgfpathmoveto{\pgfqpoint{3.791094in}{2.542732in}}%
\pgfpathlineto{\pgfqpoint{3.791094in}{2.542732in}}%
\pgfpathlineto{\pgfqpoint{3.791094in}{2.546989in}}%
\pgfpathlineto{\pgfqpoint{3.795352in}{2.546989in}}%
\pgfpathlineto{\pgfqpoint{3.795352in}{2.542732in}}%
\pgfpathmoveto{\pgfqpoint{3.791094in}{2.546989in}}%
\pgfpathlineto{\pgfqpoint{3.791094in}{2.546989in}}%
\pgfpathlineto{\pgfqpoint{3.791094in}{2.551247in}}%
\pgfpathlineto{\pgfqpoint{3.795352in}{2.551247in}}%
\pgfpathlineto{\pgfqpoint{3.795352in}{2.546989in}}%
\pgfpathmoveto{\pgfqpoint{3.795352in}{2.542732in}}%
\pgfpathlineto{\pgfqpoint{3.795352in}{2.542732in}}%
\pgfpathlineto{\pgfqpoint{3.795352in}{2.546989in}}%
\pgfpathlineto{\pgfqpoint{3.799609in}{2.546989in}}%
\pgfpathlineto{\pgfqpoint{3.799609in}{2.542732in}}%
\pgfpathmoveto{\pgfqpoint{3.795352in}{2.546989in}}%
\pgfpathlineto{\pgfqpoint{3.795352in}{2.546989in}}%
\pgfpathlineto{\pgfqpoint{3.795352in}{2.551247in}}%
\pgfpathlineto{\pgfqpoint{3.799609in}{2.551247in}}%
\pgfpathlineto{\pgfqpoint{3.799609in}{2.546989in}}%
\pgfpathmoveto{\pgfqpoint{3.791094in}{2.551247in}}%
\pgfpathlineto{\pgfqpoint{3.791094in}{2.551247in}}%
\pgfpathlineto{\pgfqpoint{3.791094in}{2.555505in}}%
\pgfpathlineto{\pgfqpoint{3.795352in}{2.555505in}}%
\pgfpathlineto{\pgfqpoint{3.795352in}{2.551247in}}%
\pgfpathmoveto{\pgfqpoint{3.791094in}{2.555505in}}%
\pgfpathlineto{\pgfqpoint{3.791094in}{2.555505in}}%
\pgfpathlineto{\pgfqpoint{3.791094in}{2.559763in}}%
\pgfpathlineto{\pgfqpoint{3.795352in}{2.559763in}}%
\pgfpathlineto{\pgfqpoint{3.795352in}{2.555505in}}%
\pgfpathmoveto{\pgfqpoint{3.795352in}{2.551247in}}%
\pgfpathlineto{\pgfqpoint{3.795352in}{2.551247in}}%
\pgfpathlineto{\pgfqpoint{3.795352in}{2.555505in}}%
\pgfpathlineto{\pgfqpoint{3.799609in}{2.555505in}}%
\pgfpathlineto{\pgfqpoint{3.799609in}{2.551247in}}%
\pgfpathmoveto{\pgfqpoint{3.795352in}{2.555505in}}%
\pgfpathlineto{\pgfqpoint{3.795352in}{2.555505in}}%
\pgfpathlineto{\pgfqpoint{3.795352in}{2.559763in}}%
\pgfpathlineto{\pgfqpoint{3.799609in}{2.559763in}}%
\pgfpathlineto{\pgfqpoint{3.799609in}{2.555505in}}%
\pgfpathmoveto{\pgfqpoint{3.791094in}{2.559763in}}%
\pgfpathlineto{\pgfqpoint{3.791094in}{2.559763in}}%
\pgfpathlineto{\pgfqpoint{3.791094in}{2.564021in}}%
\pgfpathlineto{\pgfqpoint{3.795352in}{2.564021in}}%
\pgfpathlineto{\pgfqpoint{3.795352in}{2.559763in}}%
\pgfpathmoveto{\pgfqpoint{3.795352in}{2.559763in}}%
\pgfpathlineto{\pgfqpoint{3.795352in}{2.559763in}}%
\pgfpathlineto{\pgfqpoint{3.795352in}{2.564021in}}%
\pgfpathlineto{\pgfqpoint{3.799609in}{2.564021in}}%
\pgfpathlineto{\pgfqpoint{3.799609in}{2.559763in}}%
\pgfpathmoveto{\pgfqpoint{3.795352in}{2.564021in}}%
\pgfpathlineto{\pgfqpoint{3.795352in}{2.564021in}}%
\pgfpathlineto{\pgfqpoint{3.795352in}{2.568279in}}%
\pgfpathlineto{\pgfqpoint{3.799609in}{2.568279in}}%
\pgfpathlineto{\pgfqpoint{3.799609in}{2.564021in}}%
\pgfpathmoveto{\pgfqpoint{3.799609in}{2.555505in}}%
\pgfpathlineto{\pgfqpoint{3.799609in}{2.555505in}}%
\pgfpathlineto{\pgfqpoint{3.799609in}{2.559763in}}%
\pgfpathlineto{\pgfqpoint{3.803867in}{2.559763in}}%
\pgfpathlineto{\pgfqpoint{3.803867in}{2.555505in}}%
\pgfpathmoveto{\pgfqpoint{3.799609in}{2.559763in}}%
\pgfpathlineto{\pgfqpoint{3.799609in}{2.559763in}}%
\pgfpathlineto{\pgfqpoint{3.799609in}{2.564021in}}%
\pgfpathlineto{\pgfqpoint{3.803867in}{2.564021in}}%
\pgfpathlineto{\pgfqpoint{3.803867in}{2.559763in}}%
\pgfpathmoveto{\pgfqpoint{3.799609in}{2.564021in}}%
\pgfpathlineto{\pgfqpoint{3.799609in}{2.564021in}}%
\pgfpathlineto{\pgfqpoint{3.799609in}{2.568279in}}%
\pgfpathlineto{\pgfqpoint{3.803867in}{2.568279in}}%
\pgfpathlineto{\pgfqpoint{3.803867in}{2.564021in}}%
\pgfpathmoveto{\pgfqpoint{3.795352in}{2.568279in}}%
\pgfpathlineto{\pgfqpoint{3.795352in}{2.568279in}}%
\pgfpathlineto{\pgfqpoint{3.795352in}{2.572536in}}%
\pgfpathlineto{\pgfqpoint{3.799609in}{2.572536in}}%
\pgfpathlineto{\pgfqpoint{3.799609in}{2.568279in}}%
\pgfpathmoveto{\pgfqpoint{3.795352in}{2.572536in}}%
\pgfpathlineto{\pgfqpoint{3.795352in}{2.572536in}}%
\pgfpathlineto{\pgfqpoint{3.795352in}{2.576794in}}%
\pgfpathlineto{\pgfqpoint{3.799609in}{2.576794in}}%
\pgfpathlineto{\pgfqpoint{3.799609in}{2.572536in}}%
\pgfpathmoveto{\pgfqpoint{3.795352in}{2.576794in}}%
\pgfpathlineto{\pgfqpoint{3.795352in}{2.576794in}}%
\pgfpathlineto{\pgfqpoint{3.795352in}{2.581052in}}%
\pgfpathlineto{\pgfqpoint{3.799609in}{2.581052in}}%
\pgfpathlineto{\pgfqpoint{3.799609in}{2.576794in}}%
\pgfpathmoveto{\pgfqpoint{3.795352in}{2.581052in}}%
\pgfpathlineto{\pgfqpoint{3.795352in}{2.581052in}}%
\pgfpathlineto{\pgfqpoint{3.795352in}{2.585310in}}%
\pgfpathlineto{\pgfqpoint{3.799609in}{2.585310in}}%
\pgfpathlineto{\pgfqpoint{3.799609in}{2.581052in}}%
\pgfpathmoveto{\pgfqpoint{3.799609in}{2.568279in}}%
\pgfpathlineto{\pgfqpoint{3.799609in}{2.568279in}}%
\pgfpathlineto{\pgfqpoint{3.799609in}{2.572536in}}%
\pgfpathlineto{\pgfqpoint{3.803867in}{2.572536in}}%
\pgfpathlineto{\pgfqpoint{3.803867in}{2.568279in}}%
\pgfpathmoveto{\pgfqpoint{3.799609in}{2.572536in}}%
\pgfpathlineto{\pgfqpoint{3.799609in}{2.572536in}}%
\pgfpathlineto{\pgfqpoint{3.799609in}{2.576794in}}%
\pgfpathlineto{\pgfqpoint{3.803867in}{2.576794in}}%
\pgfpathlineto{\pgfqpoint{3.803867in}{2.572536in}}%
\pgfpathmoveto{\pgfqpoint{3.799609in}{2.576794in}}%
\pgfpathlineto{\pgfqpoint{3.799609in}{2.576794in}}%
\pgfpathlineto{\pgfqpoint{3.799609in}{2.581052in}}%
\pgfpathlineto{\pgfqpoint{3.803867in}{2.581052in}}%
\pgfpathlineto{\pgfqpoint{3.803867in}{2.576794in}}%
\pgfpathmoveto{\pgfqpoint{3.799609in}{2.581052in}}%
\pgfpathlineto{\pgfqpoint{3.799609in}{2.581052in}}%
\pgfpathlineto{\pgfqpoint{3.799609in}{2.585310in}}%
\pgfpathlineto{\pgfqpoint{3.803867in}{2.585310in}}%
\pgfpathlineto{\pgfqpoint{3.803867in}{2.581052in}}%
\pgfpathmoveto{\pgfqpoint{3.803867in}{2.581052in}}%
\pgfpathlineto{\pgfqpoint{3.803867in}{2.581052in}}%
\pgfpathlineto{\pgfqpoint{3.803867in}{2.585310in}}%
\pgfpathlineto{\pgfqpoint{3.808125in}{2.585310in}}%
\pgfpathlineto{\pgfqpoint{3.808125in}{2.581052in}}%
\pgfpathmoveto{\pgfqpoint{3.795352in}{2.585310in}}%
\pgfpathlineto{\pgfqpoint{3.795352in}{2.585310in}}%
\pgfpathlineto{\pgfqpoint{3.795352in}{2.589568in}}%
\pgfpathlineto{\pgfqpoint{3.799609in}{2.589568in}}%
\pgfpathlineto{\pgfqpoint{3.799609in}{2.585310in}}%
\pgfpathmoveto{\pgfqpoint{3.795352in}{2.589568in}}%
\pgfpathlineto{\pgfqpoint{3.795352in}{2.589568in}}%
\pgfpathlineto{\pgfqpoint{3.795352in}{2.593826in}}%
\pgfpathlineto{\pgfqpoint{3.799609in}{2.593826in}}%
\pgfpathlineto{\pgfqpoint{3.799609in}{2.589568in}}%
\pgfpathmoveto{\pgfqpoint{3.799609in}{2.585310in}}%
\pgfpathlineto{\pgfqpoint{3.799609in}{2.585310in}}%
\pgfpathlineto{\pgfqpoint{3.799609in}{2.589568in}}%
\pgfpathlineto{\pgfqpoint{3.803867in}{2.589568in}}%
\pgfpathlineto{\pgfqpoint{3.803867in}{2.585310in}}%
\pgfpathmoveto{\pgfqpoint{3.799609in}{2.589568in}}%
\pgfpathlineto{\pgfqpoint{3.799609in}{2.589568in}}%
\pgfpathlineto{\pgfqpoint{3.799609in}{2.593826in}}%
\pgfpathlineto{\pgfqpoint{3.803867in}{2.593826in}}%
\pgfpathlineto{\pgfqpoint{3.803867in}{2.589568in}}%
\pgfpathmoveto{\pgfqpoint{3.803867in}{2.585310in}}%
\pgfpathlineto{\pgfqpoint{3.803867in}{2.585310in}}%
\pgfpathlineto{\pgfqpoint{3.803867in}{2.589568in}}%
\pgfpathlineto{\pgfqpoint{3.808125in}{2.589568in}}%
\pgfpathlineto{\pgfqpoint{3.808125in}{2.585310in}}%
\pgfpathmoveto{\pgfqpoint{3.803867in}{2.589568in}}%
\pgfpathlineto{\pgfqpoint{3.803867in}{2.589568in}}%
\pgfpathlineto{\pgfqpoint{3.803867in}{2.593826in}}%
\pgfpathlineto{\pgfqpoint{3.808125in}{2.593826in}}%
\pgfpathlineto{\pgfqpoint{3.808125in}{2.589568in}}%
\pgfpathmoveto{\pgfqpoint{3.799609in}{2.593826in}}%
\pgfpathlineto{\pgfqpoint{3.799609in}{2.593826in}}%
\pgfpathlineto{\pgfqpoint{3.799609in}{2.598084in}}%
\pgfpathlineto{\pgfqpoint{3.803867in}{2.598084in}}%
\pgfpathlineto{\pgfqpoint{3.803867in}{2.593826in}}%
\pgfpathmoveto{\pgfqpoint{3.799609in}{2.598084in}}%
\pgfpathlineto{\pgfqpoint{3.799609in}{2.598084in}}%
\pgfpathlineto{\pgfqpoint{3.799609in}{2.602341in}}%
\pgfpathlineto{\pgfqpoint{3.803867in}{2.602341in}}%
\pgfpathlineto{\pgfqpoint{3.803867in}{2.598084in}}%
\pgfpathmoveto{\pgfqpoint{3.803867in}{2.593826in}}%
\pgfpathlineto{\pgfqpoint{3.803867in}{2.593826in}}%
\pgfpathlineto{\pgfqpoint{3.803867in}{2.598084in}}%
\pgfpathlineto{\pgfqpoint{3.808125in}{2.598084in}}%
\pgfpathlineto{\pgfqpoint{3.808125in}{2.593826in}}%
\pgfpathmoveto{\pgfqpoint{3.803867in}{2.598084in}}%
\pgfpathlineto{\pgfqpoint{3.803867in}{2.598084in}}%
\pgfpathlineto{\pgfqpoint{3.803867in}{2.602341in}}%
\pgfpathlineto{\pgfqpoint{3.808125in}{2.602341in}}%
\pgfpathlineto{\pgfqpoint{3.808125in}{2.598084in}}%
\pgfpathmoveto{\pgfqpoint{3.799609in}{2.602341in}}%
\pgfpathlineto{\pgfqpoint{3.799609in}{2.602341in}}%
\pgfpathlineto{\pgfqpoint{3.799609in}{2.606599in}}%
\pgfpathlineto{\pgfqpoint{3.803867in}{2.606599in}}%
\pgfpathlineto{\pgfqpoint{3.803867in}{2.602341in}}%
\pgfpathmoveto{\pgfqpoint{3.799609in}{2.606599in}}%
\pgfpathlineto{\pgfqpoint{3.799609in}{2.606599in}}%
\pgfpathlineto{\pgfqpoint{3.799609in}{2.610857in}}%
\pgfpathlineto{\pgfqpoint{3.803867in}{2.610857in}}%
\pgfpathlineto{\pgfqpoint{3.803867in}{2.606599in}}%
\pgfpathmoveto{\pgfqpoint{3.803867in}{2.602341in}}%
\pgfpathlineto{\pgfqpoint{3.803867in}{2.602341in}}%
\pgfpathlineto{\pgfqpoint{3.803867in}{2.606599in}}%
\pgfpathlineto{\pgfqpoint{3.808125in}{2.606599in}}%
\pgfpathlineto{\pgfqpoint{3.808125in}{2.602341in}}%
\pgfpathmoveto{\pgfqpoint{3.803867in}{2.606599in}}%
\pgfpathlineto{\pgfqpoint{3.803867in}{2.606599in}}%
\pgfpathlineto{\pgfqpoint{3.803867in}{2.610857in}}%
\pgfpathlineto{\pgfqpoint{3.808125in}{2.610857in}}%
\pgfpathlineto{\pgfqpoint{3.808125in}{2.606599in}}%
\pgfpathmoveto{\pgfqpoint{3.799609in}{2.610857in}}%
\pgfpathlineto{\pgfqpoint{3.799609in}{2.610857in}}%
\pgfpathlineto{\pgfqpoint{3.799609in}{2.615115in}}%
\pgfpathlineto{\pgfqpoint{3.803867in}{2.615115in}}%
\pgfpathlineto{\pgfqpoint{3.803867in}{2.610857in}}%
\pgfpathmoveto{\pgfqpoint{3.799609in}{2.615115in}}%
\pgfpathlineto{\pgfqpoint{3.799609in}{2.615115in}}%
\pgfpathlineto{\pgfqpoint{3.799609in}{2.619373in}}%
\pgfpathlineto{\pgfqpoint{3.803867in}{2.619373in}}%
\pgfpathlineto{\pgfqpoint{3.803867in}{2.615115in}}%
\pgfpathmoveto{\pgfqpoint{3.803867in}{2.610857in}}%
\pgfpathlineto{\pgfqpoint{3.803867in}{2.610857in}}%
\pgfpathlineto{\pgfqpoint{3.803867in}{2.615115in}}%
\pgfpathlineto{\pgfqpoint{3.808125in}{2.615115in}}%
\pgfpathlineto{\pgfqpoint{3.808125in}{2.610857in}}%
\pgfpathmoveto{\pgfqpoint{3.803867in}{2.615115in}}%
\pgfpathlineto{\pgfqpoint{3.803867in}{2.615115in}}%
\pgfpathlineto{\pgfqpoint{3.803867in}{2.619373in}}%
\pgfpathlineto{\pgfqpoint{3.808125in}{2.619373in}}%
\pgfpathlineto{\pgfqpoint{3.808125in}{2.615115in}}%
\pgfpathmoveto{\pgfqpoint{3.808125in}{2.610857in}}%
\pgfpathlineto{\pgfqpoint{3.808125in}{2.610857in}}%
\pgfpathlineto{\pgfqpoint{3.808125in}{2.615115in}}%
\pgfpathlineto{\pgfqpoint{3.812383in}{2.615115in}}%
\pgfpathlineto{\pgfqpoint{3.812383in}{2.610857in}}%
\pgfpathmoveto{\pgfqpoint{3.808125in}{2.615115in}}%
\pgfpathlineto{\pgfqpoint{3.808125in}{2.615115in}}%
\pgfpathlineto{\pgfqpoint{3.808125in}{2.619373in}}%
\pgfpathlineto{\pgfqpoint{3.812383in}{2.619373in}}%
\pgfpathlineto{\pgfqpoint{3.812383in}{2.615115in}}%
\pgfpathmoveto{\pgfqpoint{3.803867in}{2.619373in}}%
\pgfpathlineto{\pgfqpoint{3.803867in}{2.619373in}}%
\pgfpathlineto{\pgfqpoint{3.803867in}{2.623631in}}%
\pgfpathlineto{\pgfqpoint{3.808125in}{2.623631in}}%
\pgfpathlineto{\pgfqpoint{3.808125in}{2.619373in}}%
\pgfpathmoveto{\pgfqpoint{3.803867in}{2.623631in}}%
\pgfpathlineto{\pgfqpoint{3.803867in}{2.623631in}}%
\pgfpathlineto{\pgfqpoint{3.803867in}{2.627889in}}%
\pgfpathlineto{\pgfqpoint{3.808125in}{2.627889in}}%
\pgfpathlineto{\pgfqpoint{3.808125in}{2.623631in}}%
\pgfpathmoveto{\pgfqpoint{3.803867in}{2.627889in}}%
\pgfpathlineto{\pgfqpoint{3.803867in}{2.627889in}}%
\pgfpathlineto{\pgfqpoint{3.803867in}{2.632147in}}%
\pgfpathlineto{\pgfqpoint{3.808125in}{2.632147in}}%
\pgfpathlineto{\pgfqpoint{3.808125in}{2.627889in}}%
\pgfpathmoveto{\pgfqpoint{3.803867in}{2.632147in}}%
\pgfpathlineto{\pgfqpoint{3.803867in}{2.632147in}}%
\pgfpathlineto{\pgfqpoint{3.803867in}{2.636404in}}%
\pgfpathlineto{\pgfqpoint{3.808125in}{2.636404in}}%
\pgfpathlineto{\pgfqpoint{3.808125in}{2.632147in}}%
\pgfpathmoveto{\pgfqpoint{3.803867in}{2.636404in}}%
\pgfpathlineto{\pgfqpoint{3.803867in}{2.636404in}}%
\pgfpathlineto{\pgfqpoint{3.803867in}{2.640662in}}%
\pgfpathlineto{\pgfqpoint{3.808125in}{2.640662in}}%
\pgfpathlineto{\pgfqpoint{3.808125in}{2.636404in}}%
\pgfpathmoveto{\pgfqpoint{3.803867in}{2.640662in}}%
\pgfpathlineto{\pgfqpoint{3.803867in}{2.640662in}}%
\pgfpathlineto{\pgfqpoint{3.803867in}{2.644920in}}%
\pgfpathlineto{\pgfqpoint{3.808125in}{2.644920in}}%
\pgfpathlineto{\pgfqpoint{3.808125in}{2.640662in}}%
\pgfpathmoveto{\pgfqpoint{3.803867in}{2.644920in}}%
\pgfpathlineto{\pgfqpoint{3.803867in}{2.644920in}}%
\pgfpathlineto{\pgfqpoint{3.803867in}{2.649178in}}%
\pgfpathlineto{\pgfqpoint{3.808125in}{2.649178in}}%
\pgfpathlineto{\pgfqpoint{3.808125in}{2.644920in}}%
\pgfpathmoveto{\pgfqpoint{3.808125in}{2.619373in}}%
\pgfpathlineto{\pgfqpoint{3.808125in}{2.619373in}}%
\pgfpathlineto{\pgfqpoint{3.808125in}{2.623631in}}%
\pgfpathlineto{\pgfqpoint{3.812383in}{2.623631in}}%
\pgfpathlineto{\pgfqpoint{3.812383in}{2.619373in}}%
\pgfpathmoveto{\pgfqpoint{3.808125in}{2.623631in}}%
\pgfpathlineto{\pgfqpoint{3.808125in}{2.623631in}}%
\pgfpathlineto{\pgfqpoint{3.808125in}{2.627889in}}%
\pgfpathlineto{\pgfqpoint{3.812383in}{2.627889in}}%
\pgfpathlineto{\pgfqpoint{3.812383in}{2.623631in}}%
\pgfpathmoveto{\pgfqpoint{3.808125in}{2.627889in}}%
\pgfpathlineto{\pgfqpoint{3.808125in}{2.627889in}}%
\pgfpathlineto{\pgfqpoint{3.808125in}{2.632147in}}%
\pgfpathlineto{\pgfqpoint{3.812383in}{2.632147in}}%
\pgfpathlineto{\pgfqpoint{3.812383in}{2.627889in}}%
\pgfpathmoveto{\pgfqpoint{3.808125in}{2.632147in}}%
\pgfpathlineto{\pgfqpoint{3.808125in}{2.632147in}}%
\pgfpathlineto{\pgfqpoint{3.808125in}{2.636404in}}%
\pgfpathlineto{\pgfqpoint{3.812383in}{2.636404in}}%
\pgfpathlineto{\pgfqpoint{3.812383in}{2.632147in}}%
\pgfpathmoveto{\pgfqpoint{3.808125in}{2.636404in}}%
\pgfpathlineto{\pgfqpoint{3.808125in}{2.636404in}}%
\pgfpathlineto{\pgfqpoint{3.808125in}{2.640662in}}%
\pgfpathlineto{\pgfqpoint{3.812383in}{2.640662in}}%
\pgfpathlineto{\pgfqpoint{3.812383in}{2.636404in}}%
\pgfpathmoveto{\pgfqpoint{3.808125in}{2.640662in}}%
\pgfpathlineto{\pgfqpoint{3.808125in}{2.640662in}}%
\pgfpathlineto{\pgfqpoint{3.808125in}{2.644920in}}%
\pgfpathlineto{\pgfqpoint{3.812383in}{2.644920in}}%
\pgfpathlineto{\pgfqpoint{3.812383in}{2.640662in}}%
\pgfpathmoveto{\pgfqpoint{3.812383in}{2.640662in}}%
\pgfpathlineto{\pgfqpoint{3.812383in}{2.640662in}}%
\pgfpathlineto{\pgfqpoint{3.812383in}{2.644920in}}%
\pgfpathlineto{\pgfqpoint{3.816641in}{2.644920in}}%
\pgfpathlineto{\pgfqpoint{3.816641in}{2.640662in}}%
\pgfpathmoveto{\pgfqpoint{3.808125in}{2.644920in}}%
\pgfpathlineto{\pgfqpoint{3.808125in}{2.644920in}}%
\pgfpathlineto{\pgfqpoint{3.808125in}{2.649178in}}%
\pgfpathlineto{\pgfqpoint{3.812383in}{2.649178in}}%
\pgfpathlineto{\pgfqpoint{3.812383in}{2.644920in}}%
\pgfpathmoveto{\pgfqpoint{3.808125in}{2.649178in}}%
\pgfpathlineto{\pgfqpoint{3.808125in}{2.649178in}}%
\pgfpathlineto{\pgfqpoint{3.808125in}{2.653436in}}%
\pgfpathlineto{\pgfqpoint{3.812383in}{2.653436in}}%
\pgfpathlineto{\pgfqpoint{3.812383in}{2.649178in}}%
\pgfpathmoveto{\pgfqpoint{3.812383in}{2.644920in}}%
\pgfpathlineto{\pgfqpoint{3.812383in}{2.644920in}}%
\pgfpathlineto{\pgfqpoint{3.812383in}{2.649178in}}%
\pgfpathlineto{\pgfqpoint{3.816641in}{2.649178in}}%
\pgfpathlineto{\pgfqpoint{3.816641in}{2.644920in}}%
\pgfpathmoveto{\pgfqpoint{3.812383in}{2.649178in}}%
\pgfpathlineto{\pgfqpoint{3.812383in}{2.649178in}}%
\pgfpathlineto{\pgfqpoint{3.812383in}{2.653436in}}%
\pgfpathlineto{\pgfqpoint{3.816641in}{2.653436in}}%
\pgfpathlineto{\pgfqpoint{3.816641in}{2.649178in}}%
\pgfpathmoveto{\pgfqpoint{3.808125in}{2.653436in}}%
\pgfpathlineto{\pgfqpoint{3.808125in}{2.653436in}}%
\pgfpathlineto{\pgfqpoint{3.808125in}{2.657694in}}%
\pgfpathlineto{\pgfqpoint{3.812383in}{2.657694in}}%
\pgfpathlineto{\pgfqpoint{3.812383in}{2.653436in}}%
\pgfpathmoveto{\pgfqpoint{3.808125in}{2.657694in}}%
\pgfpathlineto{\pgfqpoint{3.808125in}{2.657694in}}%
\pgfpathlineto{\pgfqpoint{3.808125in}{2.661952in}}%
\pgfpathlineto{\pgfqpoint{3.812383in}{2.661952in}}%
\pgfpathlineto{\pgfqpoint{3.812383in}{2.657694in}}%
\pgfpathmoveto{\pgfqpoint{3.812383in}{2.653436in}}%
\pgfpathlineto{\pgfqpoint{3.812383in}{2.653436in}}%
\pgfpathlineto{\pgfqpoint{3.812383in}{2.657694in}}%
\pgfpathlineto{\pgfqpoint{3.816641in}{2.657694in}}%
\pgfpathlineto{\pgfqpoint{3.816641in}{2.653436in}}%
\pgfpathmoveto{\pgfqpoint{3.812383in}{2.657694in}}%
\pgfpathlineto{\pgfqpoint{3.812383in}{2.657694in}}%
\pgfpathlineto{\pgfqpoint{3.812383in}{2.661952in}}%
\pgfpathlineto{\pgfqpoint{3.816641in}{2.661952in}}%
\pgfpathlineto{\pgfqpoint{3.816641in}{2.657694in}}%
\pgfpathmoveto{\pgfqpoint{3.808125in}{2.661952in}}%
\pgfpathlineto{\pgfqpoint{3.808125in}{2.661952in}}%
\pgfpathlineto{\pgfqpoint{3.808125in}{2.666209in}}%
\pgfpathlineto{\pgfqpoint{3.812383in}{2.666209in}}%
\pgfpathlineto{\pgfqpoint{3.812383in}{2.661952in}}%
\pgfpathmoveto{\pgfqpoint{3.808125in}{2.666209in}}%
\pgfpathlineto{\pgfqpoint{3.808125in}{2.666209in}}%
\pgfpathlineto{\pgfqpoint{3.808125in}{2.670467in}}%
\pgfpathlineto{\pgfqpoint{3.812383in}{2.670467in}}%
\pgfpathlineto{\pgfqpoint{3.812383in}{2.666209in}}%
\pgfpathmoveto{\pgfqpoint{3.812383in}{2.661952in}}%
\pgfpathlineto{\pgfqpoint{3.812383in}{2.661952in}}%
\pgfpathlineto{\pgfqpoint{3.812383in}{2.666209in}}%
\pgfpathlineto{\pgfqpoint{3.816641in}{2.666209in}}%
\pgfpathlineto{\pgfqpoint{3.816641in}{2.661952in}}%
\pgfpathmoveto{\pgfqpoint{3.812383in}{2.666209in}}%
\pgfpathlineto{\pgfqpoint{3.812383in}{2.666209in}}%
\pgfpathlineto{\pgfqpoint{3.812383in}{2.670467in}}%
\pgfpathlineto{\pgfqpoint{3.816641in}{2.670467in}}%
\pgfpathlineto{\pgfqpoint{3.816641in}{2.666209in}}%
\pgfpathmoveto{\pgfqpoint{3.816641in}{2.666209in}}%
\pgfpathlineto{\pgfqpoint{3.816641in}{2.666209in}}%
\pgfpathlineto{\pgfqpoint{3.816641in}{2.670467in}}%
\pgfpathlineto{\pgfqpoint{3.820898in}{2.670467in}}%
\pgfpathlineto{\pgfqpoint{3.820898in}{2.666209in}}%
\pgfpathmoveto{\pgfqpoint{3.808125in}{2.670467in}}%
\pgfpathlineto{\pgfqpoint{3.808125in}{2.670467in}}%
\pgfpathlineto{\pgfqpoint{3.808125in}{2.674725in}}%
\pgfpathlineto{\pgfqpoint{3.812383in}{2.674725in}}%
\pgfpathlineto{\pgfqpoint{3.812383in}{2.670467in}}%
\pgfpathmoveto{\pgfqpoint{3.808125in}{2.674725in}}%
\pgfpathlineto{\pgfqpoint{3.808125in}{2.674725in}}%
\pgfpathlineto{\pgfqpoint{3.808125in}{2.678983in}}%
\pgfpathlineto{\pgfqpoint{3.812383in}{2.678983in}}%
\pgfpathlineto{\pgfqpoint{3.812383in}{2.674725in}}%
\pgfpathmoveto{\pgfqpoint{3.812383in}{2.670467in}}%
\pgfpathlineto{\pgfqpoint{3.812383in}{2.670467in}}%
\pgfpathlineto{\pgfqpoint{3.812383in}{2.674725in}}%
\pgfpathlineto{\pgfqpoint{3.816641in}{2.674725in}}%
\pgfpathlineto{\pgfqpoint{3.816641in}{2.670467in}}%
\pgfpathmoveto{\pgfqpoint{3.812383in}{2.674725in}}%
\pgfpathlineto{\pgfqpoint{3.812383in}{2.674725in}}%
\pgfpathlineto{\pgfqpoint{3.812383in}{2.678983in}}%
\pgfpathlineto{\pgfqpoint{3.816641in}{2.678983in}}%
\pgfpathlineto{\pgfqpoint{3.816641in}{2.674725in}}%
\pgfpathmoveto{\pgfqpoint{3.812383in}{2.678983in}}%
\pgfpathlineto{\pgfqpoint{3.812383in}{2.678983in}}%
\pgfpathlineto{\pgfqpoint{3.812383in}{2.683241in}}%
\pgfpathlineto{\pgfqpoint{3.816641in}{2.683241in}}%
\pgfpathlineto{\pgfqpoint{3.816641in}{2.678983in}}%
\pgfpathmoveto{\pgfqpoint{3.812383in}{2.683241in}}%
\pgfpathlineto{\pgfqpoint{3.812383in}{2.683241in}}%
\pgfpathlineto{\pgfqpoint{3.812383in}{2.687499in}}%
\pgfpathlineto{\pgfqpoint{3.816641in}{2.687499in}}%
\pgfpathlineto{\pgfqpoint{3.816641in}{2.683241in}}%
\pgfpathmoveto{\pgfqpoint{3.816641in}{2.670467in}}%
\pgfpathlineto{\pgfqpoint{3.816641in}{2.670467in}}%
\pgfpathlineto{\pgfqpoint{3.816641in}{2.674725in}}%
\pgfpathlineto{\pgfqpoint{3.820898in}{2.674725in}}%
\pgfpathlineto{\pgfqpoint{3.820898in}{2.670467in}}%
\pgfpathmoveto{\pgfqpoint{3.816641in}{2.674725in}}%
\pgfpathlineto{\pgfqpoint{3.816641in}{2.674725in}}%
\pgfpathlineto{\pgfqpoint{3.816641in}{2.678983in}}%
\pgfpathlineto{\pgfqpoint{3.820898in}{2.678983in}}%
\pgfpathlineto{\pgfqpoint{3.820898in}{2.674725in}}%
\pgfpathmoveto{\pgfqpoint{3.816641in}{2.678983in}}%
\pgfpathlineto{\pgfqpoint{3.816641in}{2.678983in}}%
\pgfpathlineto{\pgfqpoint{3.816641in}{2.683241in}}%
\pgfpathlineto{\pgfqpoint{3.820898in}{2.683241in}}%
\pgfpathlineto{\pgfqpoint{3.820898in}{2.678983in}}%
\pgfpathmoveto{\pgfqpoint{3.816641in}{2.683241in}}%
\pgfpathlineto{\pgfqpoint{3.816641in}{2.683241in}}%
\pgfpathlineto{\pgfqpoint{3.816641in}{2.687499in}}%
\pgfpathlineto{\pgfqpoint{3.820898in}{2.687499in}}%
\pgfpathlineto{\pgfqpoint{3.820898in}{2.683241in}}%
\pgfpathmoveto{\pgfqpoint{3.812383in}{2.687499in}}%
\pgfpathlineto{\pgfqpoint{3.812383in}{2.687499in}}%
\pgfpathlineto{\pgfqpoint{3.812383in}{2.691757in}}%
\pgfpathlineto{\pgfqpoint{3.816641in}{2.691757in}}%
\pgfpathlineto{\pgfqpoint{3.816641in}{2.687499in}}%
\pgfpathmoveto{\pgfqpoint{3.812383in}{2.691757in}}%
\pgfpathlineto{\pgfqpoint{3.812383in}{2.691757in}}%
\pgfpathlineto{\pgfqpoint{3.812383in}{2.696015in}}%
\pgfpathlineto{\pgfqpoint{3.816641in}{2.696015in}}%
\pgfpathlineto{\pgfqpoint{3.816641in}{2.691757in}}%
\pgfpathmoveto{\pgfqpoint{3.812383in}{2.696015in}}%
\pgfpathlineto{\pgfqpoint{3.812383in}{2.696015in}}%
\pgfpathlineto{\pgfqpoint{3.812383in}{2.700273in}}%
\pgfpathlineto{\pgfqpoint{3.816641in}{2.700273in}}%
\pgfpathlineto{\pgfqpoint{3.816641in}{2.696015in}}%
\pgfpathmoveto{\pgfqpoint{3.812383in}{2.700273in}}%
\pgfpathlineto{\pgfqpoint{3.812383in}{2.700273in}}%
\pgfpathlineto{\pgfqpoint{3.812383in}{2.704531in}}%
\pgfpathlineto{\pgfqpoint{3.816641in}{2.704531in}}%
\pgfpathlineto{\pgfqpoint{3.816641in}{2.700273in}}%
\pgfpathmoveto{\pgfqpoint{3.816641in}{2.687499in}}%
\pgfpathlineto{\pgfqpoint{3.816641in}{2.687499in}}%
\pgfpathlineto{\pgfqpoint{3.816641in}{2.691757in}}%
\pgfpathlineto{\pgfqpoint{3.820898in}{2.691757in}}%
\pgfpathlineto{\pgfqpoint{3.820898in}{2.687499in}}%
\pgfpathmoveto{\pgfqpoint{3.816641in}{2.691757in}}%
\pgfpathlineto{\pgfqpoint{3.816641in}{2.691757in}}%
\pgfpathlineto{\pgfqpoint{3.816641in}{2.696015in}}%
\pgfpathlineto{\pgfqpoint{3.820898in}{2.696015in}}%
\pgfpathlineto{\pgfqpoint{3.820898in}{2.691757in}}%
\pgfpathmoveto{\pgfqpoint{3.816641in}{2.696015in}}%
\pgfpathlineto{\pgfqpoint{3.816641in}{2.696015in}}%
\pgfpathlineto{\pgfqpoint{3.816641in}{2.700273in}}%
\pgfpathlineto{\pgfqpoint{3.820898in}{2.700273in}}%
\pgfpathlineto{\pgfqpoint{3.820898in}{2.696015in}}%
\pgfpathmoveto{\pgfqpoint{3.816641in}{2.700273in}}%
\pgfpathlineto{\pgfqpoint{3.816641in}{2.700273in}}%
\pgfpathlineto{\pgfqpoint{3.816641in}{2.704531in}}%
\pgfpathlineto{\pgfqpoint{3.820898in}{2.704531in}}%
\pgfpathlineto{\pgfqpoint{3.820898in}{2.700273in}}%
\pgfpathmoveto{\pgfqpoint{3.820898in}{2.696015in}}%
\pgfpathlineto{\pgfqpoint{3.820898in}{2.696015in}}%
\pgfpathlineto{\pgfqpoint{3.820898in}{2.700273in}}%
\pgfpathlineto{\pgfqpoint{3.825156in}{2.700273in}}%
\pgfpathlineto{\pgfqpoint{3.825156in}{2.696015in}}%
\pgfpathmoveto{\pgfqpoint{3.820898in}{2.700273in}}%
\pgfpathlineto{\pgfqpoint{3.820898in}{2.700273in}}%
\pgfpathlineto{\pgfqpoint{3.820898in}{2.704531in}}%
\pgfpathlineto{\pgfqpoint{3.825156in}{2.704531in}}%
\pgfpathlineto{\pgfqpoint{3.825156in}{2.700273in}}%
\pgfpathmoveto{\pgfqpoint{3.816641in}{2.704531in}}%
\pgfpathlineto{\pgfqpoint{3.816641in}{2.704531in}}%
\pgfpathlineto{\pgfqpoint{3.816641in}{2.708789in}}%
\pgfpathlineto{\pgfqpoint{3.820898in}{2.708789in}}%
\pgfpathlineto{\pgfqpoint{3.820898in}{2.704531in}}%
\pgfpathmoveto{\pgfqpoint{3.816641in}{2.708789in}}%
\pgfpathlineto{\pgfqpoint{3.816641in}{2.708789in}}%
\pgfpathlineto{\pgfqpoint{3.816641in}{2.713046in}}%
\pgfpathlineto{\pgfqpoint{3.820898in}{2.713046in}}%
\pgfpathlineto{\pgfqpoint{3.820898in}{2.708789in}}%
\pgfpathmoveto{\pgfqpoint{3.820898in}{2.704531in}}%
\pgfpathlineto{\pgfqpoint{3.820898in}{2.704531in}}%
\pgfpathlineto{\pgfqpoint{3.820898in}{2.708789in}}%
\pgfpathlineto{\pgfqpoint{3.825156in}{2.708789in}}%
\pgfpathlineto{\pgfqpoint{3.825156in}{2.704531in}}%
\pgfpathmoveto{\pgfqpoint{3.820898in}{2.708789in}}%
\pgfpathlineto{\pgfqpoint{3.820898in}{2.708789in}}%
\pgfpathlineto{\pgfqpoint{3.820898in}{2.713046in}}%
\pgfpathlineto{\pgfqpoint{3.825156in}{2.713046in}}%
\pgfpathlineto{\pgfqpoint{3.825156in}{2.708789in}}%
\pgfpathmoveto{\pgfqpoint{3.816641in}{2.713046in}}%
\pgfpathlineto{\pgfqpoint{3.816641in}{2.713046in}}%
\pgfpathlineto{\pgfqpoint{3.816641in}{2.717304in}}%
\pgfpathlineto{\pgfqpoint{3.820898in}{2.717304in}}%
\pgfpathlineto{\pgfqpoint{3.820898in}{2.713046in}}%
\pgfpathmoveto{\pgfqpoint{3.816641in}{2.717304in}}%
\pgfpathlineto{\pgfqpoint{3.816641in}{2.717304in}}%
\pgfpathlineto{\pgfqpoint{3.816641in}{2.721562in}}%
\pgfpathlineto{\pgfqpoint{3.820898in}{2.721562in}}%
\pgfpathlineto{\pgfqpoint{3.820898in}{2.717304in}}%
\pgfpathmoveto{\pgfqpoint{3.820898in}{2.713046in}}%
\pgfpathlineto{\pgfqpoint{3.820898in}{2.713046in}}%
\pgfpathlineto{\pgfqpoint{3.820898in}{2.717304in}}%
\pgfpathlineto{\pgfqpoint{3.825156in}{2.717304in}}%
\pgfpathlineto{\pgfqpoint{3.825156in}{2.713046in}}%
\pgfpathmoveto{\pgfqpoint{3.820898in}{2.717304in}}%
\pgfpathlineto{\pgfqpoint{3.820898in}{2.717304in}}%
\pgfpathlineto{\pgfqpoint{3.820898in}{2.721562in}}%
\pgfpathlineto{\pgfqpoint{3.825156in}{2.721562in}}%
\pgfpathlineto{\pgfqpoint{3.825156in}{2.717304in}}%
\pgfpathmoveto{\pgfqpoint{3.816641in}{2.721562in}}%
\pgfpathlineto{\pgfqpoint{3.816641in}{2.721562in}}%
\pgfpathlineto{\pgfqpoint{3.816641in}{2.725820in}}%
\pgfpathlineto{\pgfqpoint{3.820898in}{2.725820in}}%
\pgfpathlineto{\pgfqpoint{3.820898in}{2.721562in}}%
\pgfpathmoveto{\pgfqpoint{3.816641in}{2.725820in}}%
\pgfpathlineto{\pgfqpoint{3.816641in}{2.725820in}}%
\pgfpathlineto{\pgfqpoint{3.816641in}{2.730078in}}%
\pgfpathlineto{\pgfqpoint{3.820898in}{2.730078in}}%
\pgfpathlineto{\pgfqpoint{3.820898in}{2.725820in}}%
\pgfpathmoveto{\pgfqpoint{3.820898in}{2.721562in}}%
\pgfpathlineto{\pgfqpoint{3.820898in}{2.721562in}}%
\pgfpathlineto{\pgfqpoint{3.820898in}{2.725820in}}%
\pgfpathlineto{\pgfqpoint{3.825156in}{2.725820in}}%
\pgfpathlineto{\pgfqpoint{3.825156in}{2.721562in}}%
\pgfpathmoveto{\pgfqpoint{3.820898in}{2.725820in}}%
\pgfpathlineto{\pgfqpoint{3.820898in}{2.725820in}}%
\pgfpathlineto{\pgfqpoint{3.820898in}{2.730078in}}%
\pgfpathlineto{\pgfqpoint{3.825156in}{2.730078in}}%
\pgfpathlineto{\pgfqpoint{3.825156in}{2.725820in}}%
\pgfpathmoveto{\pgfqpoint{3.816641in}{2.730078in}}%
\pgfpathlineto{\pgfqpoint{3.816641in}{2.730078in}}%
\pgfpathlineto{\pgfqpoint{3.816641in}{2.734336in}}%
\pgfpathlineto{\pgfqpoint{3.820898in}{2.734336in}}%
\pgfpathlineto{\pgfqpoint{3.820898in}{2.730078in}}%
\pgfpathmoveto{\pgfqpoint{3.820898in}{2.730078in}}%
\pgfpathlineto{\pgfqpoint{3.820898in}{2.730078in}}%
\pgfpathlineto{\pgfqpoint{3.820898in}{2.734336in}}%
\pgfpathlineto{\pgfqpoint{3.825156in}{2.734336in}}%
\pgfpathlineto{\pgfqpoint{3.825156in}{2.730078in}}%
\pgfpathmoveto{\pgfqpoint{3.820898in}{2.734336in}}%
\pgfpathlineto{\pgfqpoint{3.820898in}{2.734336in}}%
\pgfpathlineto{\pgfqpoint{3.820898in}{2.738594in}}%
\pgfpathlineto{\pgfqpoint{3.825156in}{2.738594in}}%
\pgfpathlineto{\pgfqpoint{3.825156in}{2.734336in}}%
\pgfpathmoveto{\pgfqpoint{3.820898in}{2.738594in}}%
\pgfpathlineto{\pgfqpoint{3.820898in}{2.738594in}}%
\pgfpathlineto{\pgfqpoint{3.820898in}{2.742852in}}%
\pgfpathlineto{\pgfqpoint{3.825156in}{2.742852in}}%
\pgfpathlineto{\pgfqpoint{3.825156in}{2.738594in}}%
\pgfpathmoveto{\pgfqpoint{3.820898in}{2.742852in}}%
\pgfpathlineto{\pgfqpoint{3.820898in}{2.742852in}}%
\pgfpathlineto{\pgfqpoint{3.820898in}{2.747110in}}%
\pgfpathlineto{\pgfqpoint{3.825156in}{2.747110in}}%
\pgfpathlineto{\pgfqpoint{3.825156in}{2.742852in}}%
\pgfpathmoveto{\pgfqpoint{3.820898in}{2.747110in}}%
\pgfpathlineto{\pgfqpoint{3.820898in}{2.747110in}}%
\pgfpathlineto{\pgfqpoint{3.820898in}{2.751368in}}%
\pgfpathlineto{\pgfqpoint{3.825156in}{2.751368in}}%
\pgfpathlineto{\pgfqpoint{3.825156in}{2.747110in}}%
\pgfpathmoveto{\pgfqpoint{3.820898in}{2.751368in}}%
\pgfpathlineto{\pgfqpoint{3.820898in}{2.751368in}}%
\pgfpathlineto{\pgfqpoint{3.820898in}{2.755626in}}%
\pgfpathlineto{\pgfqpoint{3.825156in}{2.755626in}}%
\pgfpathlineto{\pgfqpoint{3.825156in}{2.751368in}}%
\pgfpathmoveto{\pgfqpoint{3.825156in}{2.725820in}}%
\pgfpathlineto{\pgfqpoint{3.825156in}{2.725820in}}%
\pgfpathlineto{\pgfqpoint{3.825156in}{2.730078in}}%
\pgfpathlineto{\pgfqpoint{3.829414in}{2.730078in}}%
\pgfpathlineto{\pgfqpoint{3.829414in}{2.725820in}}%
\pgfpathmoveto{\pgfqpoint{3.825156in}{2.730078in}}%
\pgfpathlineto{\pgfqpoint{3.825156in}{2.730078in}}%
\pgfpathlineto{\pgfqpoint{3.825156in}{2.734336in}}%
\pgfpathlineto{\pgfqpoint{3.829414in}{2.734336in}}%
\pgfpathlineto{\pgfqpoint{3.829414in}{2.730078in}}%
\pgfpathmoveto{\pgfqpoint{3.825156in}{2.734336in}}%
\pgfpathlineto{\pgfqpoint{3.825156in}{2.734336in}}%
\pgfpathlineto{\pgfqpoint{3.825156in}{2.738594in}}%
\pgfpathlineto{\pgfqpoint{3.829414in}{2.738594in}}%
\pgfpathlineto{\pgfqpoint{3.829414in}{2.734336in}}%
\pgfpathmoveto{\pgfqpoint{3.825156in}{2.738594in}}%
\pgfpathlineto{\pgfqpoint{3.825156in}{2.738594in}}%
\pgfpathlineto{\pgfqpoint{3.825156in}{2.742852in}}%
\pgfpathlineto{\pgfqpoint{3.829414in}{2.742852in}}%
\pgfpathlineto{\pgfqpoint{3.829414in}{2.738594in}}%
\pgfpathmoveto{\pgfqpoint{3.825156in}{2.742852in}}%
\pgfpathlineto{\pgfqpoint{3.825156in}{2.742852in}}%
\pgfpathlineto{\pgfqpoint{3.825156in}{2.747110in}}%
\pgfpathlineto{\pgfqpoint{3.829414in}{2.747110in}}%
\pgfpathlineto{\pgfqpoint{3.829414in}{2.742852in}}%
\pgfpathmoveto{\pgfqpoint{3.825156in}{2.747110in}}%
\pgfpathlineto{\pgfqpoint{3.825156in}{2.747110in}}%
\pgfpathlineto{\pgfqpoint{3.825156in}{2.751368in}}%
\pgfpathlineto{\pgfqpoint{3.829414in}{2.751368in}}%
\pgfpathlineto{\pgfqpoint{3.829414in}{2.747110in}}%
\pgfpathmoveto{\pgfqpoint{3.825156in}{2.751368in}}%
\pgfpathlineto{\pgfqpoint{3.825156in}{2.751368in}}%
\pgfpathlineto{\pgfqpoint{3.825156in}{2.755626in}}%
\pgfpathlineto{\pgfqpoint{3.829414in}{2.755626in}}%
\pgfpathlineto{\pgfqpoint{3.829414in}{2.751368in}}%
\pgfpathmoveto{\pgfqpoint{3.820898in}{2.755626in}}%
\pgfpathlineto{\pgfqpoint{3.820898in}{2.755626in}}%
\pgfpathlineto{\pgfqpoint{3.820898in}{2.759884in}}%
\pgfpathlineto{\pgfqpoint{3.825156in}{2.759884in}}%
\pgfpathlineto{\pgfqpoint{3.825156in}{2.755626in}}%
\pgfpathmoveto{\pgfqpoint{3.820898in}{2.759884in}}%
\pgfpathlineto{\pgfqpoint{3.820898in}{2.759884in}}%
\pgfpathlineto{\pgfqpoint{3.820898in}{2.764142in}}%
\pgfpathlineto{\pgfqpoint{3.825156in}{2.764142in}}%
\pgfpathlineto{\pgfqpoint{3.825156in}{2.759884in}}%
\pgfpathmoveto{\pgfqpoint{3.825156in}{2.755626in}}%
\pgfpathlineto{\pgfqpoint{3.825156in}{2.755626in}}%
\pgfpathlineto{\pgfqpoint{3.825156in}{2.759884in}}%
\pgfpathlineto{\pgfqpoint{3.829414in}{2.759884in}}%
\pgfpathlineto{\pgfqpoint{3.829414in}{2.755626in}}%
\pgfpathmoveto{\pgfqpoint{3.825156in}{2.759884in}}%
\pgfpathlineto{\pgfqpoint{3.825156in}{2.759884in}}%
\pgfpathlineto{\pgfqpoint{3.825156in}{2.764142in}}%
\pgfpathlineto{\pgfqpoint{3.829414in}{2.764142in}}%
\pgfpathlineto{\pgfqpoint{3.829414in}{2.759884in}}%
\pgfpathmoveto{\pgfqpoint{3.829414in}{2.755626in}}%
\pgfpathlineto{\pgfqpoint{3.829414in}{2.755626in}}%
\pgfpathlineto{\pgfqpoint{3.829414in}{2.759884in}}%
\pgfpathlineto{\pgfqpoint{3.833672in}{2.759884in}}%
\pgfpathlineto{\pgfqpoint{3.833672in}{2.755626in}}%
\pgfpathmoveto{\pgfqpoint{3.829414in}{2.759884in}}%
\pgfpathlineto{\pgfqpoint{3.829414in}{2.759884in}}%
\pgfpathlineto{\pgfqpoint{3.829414in}{2.764142in}}%
\pgfpathlineto{\pgfqpoint{3.833672in}{2.764142in}}%
\pgfpathlineto{\pgfqpoint{3.833672in}{2.759884in}}%
\pgfpathmoveto{\pgfqpoint{3.825156in}{2.764142in}}%
\pgfpathlineto{\pgfqpoint{3.825156in}{2.764142in}}%
\pgfpathlineto{\pgfqpoint{3.825156in}{2.768400in}}%
\pgfpathlineto{\pgfqpoint{3.829414in}{2.768400in}}%
\pgfpathlineto{\pgfqpoint{3.829414in}{2.764142in}}%
\pgfpathmoveto{\pgfqpoint{3.825156in}{2.768400in}}%
\pgfpathlineto{\pgfqpoint{3.825156in}{2.768400in}}%
\pgfpathlineto{\pgfqpoint{3.825156in}{2.772658in}}%
\pgfpathlineto{\pgfqpoint{3.829414in}{2.772658in}}%
\pgfpathlineto{\pgfqpoint{3.829414in}{2.768400in}}%
\pgfpathmoveto{\pgfqpoint{3.829414in}{2.764142in}}%
\pgfpathlineto{\pgfqpoint{3.829414in}{2.764142in}}%
\pgfpathlineto{\pgfqpoint{3.829414in}{2.768400in}}%
\pgfpathlineto{\pgfqpoint{3.833672in}{2.768400in}}%
\pgfpathlineto{\pgfqpoint{3.833672in}{2.764142in}}%
\pgfpathmoveto{\pgfqpoint{3.829414in}{2.768400in}}%
\pgfpathlineto{\pgfqpoint{3.829414in}{2.768400in}}%
\pgfpathlineto{\pgfqpoint{3.829414in}{2.772658in}}%
\pgfpathlineto{\pgfqpoint{3.833672in}{2.772658in}}%
\pgfpathlineto{\pgfqpoint{3.833672in}{2.768400in}}%
\pgfpathmoveto{\pgfqpoint{3.825156in}{2.772658in}}%
\pgfpathlineto{\pgfqpoint{3.825156in}{2.772658in}}%
\pgfpathlineto{\pgfqpoint{3.825156in}{2.776916in}}%
\pgfpathlineto{\pgfqpoint{3.829414in}{2.776916in}}%
\pgfpathlineto{\pgfqpoint{3.829414in}{2.772658in}}%
\pgfpathmoveto{\pgfqpoint{3.825156in}{2.776916in}}%
\pgfpathlineto{\pgfqpoint{3.825156in}{2.776916in}}%
\pgfpathlineto{\pgfqpoint{3.825156in}{2.781174in}}%
\pgfpathlineto{\pgfqpoint{3.829414in}{2.781174in}}%
\pgfpathlineto{\pgfqpoint{3.829414in}{2.776916in}}%
\pgfpathmoveto{\pgfqpoint{3.829414in}{2.772658in}}%
\pgfpathlineto{\pgfqpoint{3.829414in}{2.772658in}}%
\pgfpathlineto{\pgfqpoint{3.829414in}{2.776916in}}%
\pgfpathlineto{\pgfqpoint{3.833672in}{2.776916in}}%
\pgfpathlineto{\pgfqpoint{3.833672in}{2.772658in}}%
\pgfpathmoveto{\pgfqpoint{3.829414in}{2.776916in}}%
\pgfpathlineto{\pgfqpoint{3.829414in}{2.776916in}}%
\pgfpathlineto{\pgfqpoint{3.829414in}{2.781174in}}%
\pgfpathlineto{\pgfqpoint{3.833672in}{2.781174in}}%
\pgfpathlineto{\pgfqpoint{3.833672in}{2.776916in}}%
\pgfpathmoveto{\pgfqpoint{3.825156in}{2.781174in}}%
\pgfpathlineto{\pgfqpoint{3.825156in}{2.781174in}}%
\pgfpathlineto{\pgfqpoint{3.825156in}{2.785432in}}%
\pgfpathlineto{\pgfqpoint{3.829414in}{2.785432in}}%
\pgfpathlineto{\pgfqpoint{3.829414in}{2.781174in}}%
\pgfpathmoveto{\pgfqpoint{3.825156in}{2.785432in}}%
\pgfpathlineto{\pgfqpoint{3.825156in}{2.785432in}}%
\pgfpathlineto{\pgfqpoint{3.825156in}{2.789690in}}%
\pgfpathlineto{\pgfqpoint{3.829414in}{2.789690in}}%
\pgfpathlineto{\pgfqpoint{3.829414in}{2.785432in}}%
\pgfpathmoveto{\pgfqpoint{3.829414in}{2.781174in}}%
\pgfpathlineto{\pgfqpoint{3.829414in}{2.781174in}}%
\pgfpathlineto{\pgfqpoint{3.829414in}{2.785432in}}%
\pgfpathlineto{\pgfqpoint{3.833672in}{2.785432in}}%
\pgfpathlineto{\pgfqpoint{3.833672in}{2.781174in}}%
\pgfpathmoveto{\pgfqpoint{3.829414in}{2.785432in}}%
\pgfpathlineto{\pgfqpoint{3.829414in}{2.785432in}}%
\pgfpathlineto{\pgfqpoint{3.829414in}{2.789690in}}%
\pgfpathlineto{\pgfqpoint{3.833672in}{2.789690in}}%
\pgfpathlineto{\pgfqpoint{3.833672in}{2.785432in}}%
\pgfpathmoveto{\pgfqpoint{3.833672in}{2.785432in}}%
\pgfpathlineto{\pgfqpoint{3.833672in}{2.785432in}}%
\pgfpathlineto{\pgfqpoint{3.833672in}{2.789690in}}%
\pgfpathlineto{\pgfqpoint{3.837930in}{2.789690in}}%
\pgfpathlineto{\pgfqpoint{3.837930in}{2.785432in}}%
\pgfpathmoveto{\pgfqpoint{3.825156in}{2.789690in}}%
\pgfpathlineto{\pgfqpoint{3.825156in}{2.789690in}}%
\pgfpathlineto{\pgfqpoint{3.825156in}{2.793948in}}%
\pgfpathlineto{\pgfqpoint{3.829414in}{2.793948in}}%
\pgfpathlineto{\pgfqpoint{3.829414in}{2.789690in}}%
\pgfpathmoveto{\pgfqpoint{3.829414in}{2.789690in}}%
\pgfpathlineto{\pgfqpoint{3.829414in}{2.789690in}}%
\pgfpathlineto{\pgfqpoint{3.829414in}{2.793948in}}%
\pgfpathlineto{\pgfqpoint{3.833672in}{2.793948in}}%
\pgfpathlineto{\pgfqpoint{3.833672in}{2.789690in}}%
\pgfpathmoveto{\pgfqpoint{3.829414in}{2.793948in}}%
\pgfpathlineto{\pgfqpoint{3.829414in}{2.793948in}}%
\pgfpathlineto{\pgfqpoint{3.829414in}{2.798206in}}%
\pgfpathlineto{\pgfqpoint{3.833672in}{2.798206in}}%
\pgfpathlineto{\pgfqpoint{3.833672in}{2.793948in}}%
\pgfpathmoveto{\pgfqpoint{3.829414in}{2.798206in}}%
\pgfpathlineto{\pgfqpoint{3.829414in}{2.798206in}}%
\pgfpathlineto{\pgfqpoint{3.829414in}{2.802464in}}%
\pgfpathlineto{\pgfqpoint{3.833672in}{2.802464in}}%
\pgfpathlineto{\pgfqpoint{3.833672in}{2.798206in}}%
\pgfpathmoveto{\pgfqpoint{3.829414in}{2.802464in}}%
\pgfpathlineto{\pgfqpoint{3.829414in}{2.802464in}}%
\pgfpathlineto{\pgfqpoint{3.829414in}{2.806722in}}%
\pgfpathlineto{\pgfqpoint{3.833672in}{2.806722in}}%
\pgfpathlineto{\pgfqpoint{3.833672in}{2.802464in}}%
\pgfpathmoveto{\pgfqpoint{3.833672in}{2.789690in}}%
\pgfpathlineto{\pgfqpoint{3.833672in}{2.789690in}}%
\pgfpathlineto{\pgfqpoint{3.833672in}{2.793948in}}%
\pgfpathlineto{\pgfqpoint{3.837930in}{2.793948in}}%
\pgfpathlineto{\pgfqpoint{3.837930in}{2.789690in}}%
\pgfpathmoveto{\pgfqpoint{3.833672in}{2.793948in}}%
\pgfpathlineto{\pgfqpoint{3.833672in}{2.793948in}}%
\pgfpathlineto{\pgfqpoint{3.833672in}{2.798206in}}%
\pgfpathlineto{\pgfqpoint{3.837930in}{2.798206in}}%
\pgfpathlineto{\pgfqpoint{3.837930in}{2.793948in}}%
\pgfpathmoveto{\pgfqpoint{3.833672in}{2.798206in}}%
\pgfpathlineto{\pgfqpoint{3.833672in}{2.798206in}}%
\pgfpathlineto{\pgfqpoint{3.833672in}{2.802464in}}%
\pgfpathlineto{\pgfqpoint{3.837930in}{2.802464in}}%
\pgfpathlineto{\pgfqpoint{3.837930in}{2.798206in}}%
\pgfpathmoveto{\pgfqpoint{3.833672in}{2.802464in}}%
\pgfpathlineto{\pgfqpoint{3.833672in}{2.802464in}}%
\pgfpathlineto{\pgfqpoint{3.833672in}{2.806722in}}%
\pgfpathlineto{\pgfqpoint{3.837930in}{2.806722in}}%
\pgfpathlineto{\pgfqpoint{3.837930in}{2.802464in}}%
\pgfpathmoveto{\pgfqpoint{3.829414in}{2.806722in}}%
\pgfpathlineto{\pgfqpoint{3.829414in}{2.806722in}}%
\pgfpathlineto{\pgfqpoint{3.829414in}{2.810980in}}%
\pgfpathlineto{\pgfqpoint{3.833672in}{2.810980in}}%
\pgfpathlineto{\pgfqpoint{3.833672in}{2.806722in}}%
\pgfpathmoveto{\pgfqpoint{3.829414in}{2.810980in}}%
\pgfpathlineto{\pgfqpoint{3.829414in}{2.810980in}}%
\pgfpathlineto{\pgfqpoint{3.829414in}{2.815238in}}%
\pgfpathlineto{\pgfqpoint{3.833672in}{2.815238in}}%
\pgfpathlineto{\pgfqpoint{3.833672in}{2.810980in}}%
\pgfpathmoveto{\pgfqpoint{3.829414in}{2.815238in}}%
\pgfpathlineto{\pgfqpoint{3.829414in}{2.815238in}}%
\pgfpathlineto{\pgfqpoint{3.829414in}{2.819496in}}%
\pgfpathlineto{\pgfqpoint{3.833672in}{2.819496in}}%
\pgfpathlineto{\pgfqpoint{3.833672in}{2.815238in}}%
\pgfpathmoveto{\pgfqpoint{3.829414in}{2.819496in}}%
\pgfpathlineto{\pgfqpoint{3.829414in}{2.819496in}}%
\pgfpathlineto{\pgfqpoint{3.829414in}{2.823754in}}%
\pgfpathlineto{\pgfqpoint{3.833672in}{2.823754in}}%
\pgfpathlineto{\pgfqpoint{3.833672in}{2.819496in}}%
\pgfpathmoveto{\pgfqpoint{3.833672in}{2.806722in}}%
\pgfpathlineto{\pgfqpoint{3.833672in}{2.806722in}}%
\pgfpathlineto{\pgfqpoint{3.833672in}{2.810980in}}%
\pgfpathlineto{\pgfqpoint{3.837930in}{2.810980in}}%
\pgfpathlineto{\pgfqpoint{3.837930in}{2.806722in}}%
\pgfpathmoveto{\pgfqpoint{3.833672in}{2.810980in}}%
\pgfpathlineto{\pgfqpoint{3.833672in}{2.810980in}}%
\pgfpathlineto{\pgfqpoint{3.833672in}{2.815238in}}%
\pgfpathlineto{\pgfqpoint{3.837930in}{2.815238in}}%
\pgfpathlineto{\pgfqpoint{3.837930in}{2.810980in}}%
\pgfpathmoveto{\pgfqpoint{3.833672in}{2.815238in}}%
\pgfpathlineto{\pgfqpoint{3.833672in}{2.815238in}}%
\pgfpathlineto{\pgfqpoint{3.833672in}{2.819496in}}%
\pgfpathlineto{\pgfqpoint{3.837930in}{2.819496in}}%
\pgfpathlineto{\pgfqpoint{3.837930in}{2.815238in}}%
\pgfpathmoveto{\pgfqpoint{3.833672in}{2.819496in}}%
\pgfpathlineto{\pgfqpoint{3.833672in}{2.819496in}}%
\pgfpathlineto{\pgfqpoint{3.833672in}{2.823754in}}%
\pgfpathlineto{\pgfqpoint{3.837930in}{2.823754in}}%
\pgfpathlineto{\pgfqpoint{3.837930in}{2.819496in}}%
\pgfpathmoveto{\pgfqpoint{3.837930in}{2.815238in}}%
\pgfpathlineto{\pgfqpoint{3.837930in}{2.815238in}}%
\pgfpathlineto{\pgfqpoint{3.837930in}{2.819496in}}%
\pgfpathlineto{\pgfqpoint{3.842187in}{2.819496in}}%
\pgfpathlineto{\pgfqpoint{3.842187in}{2.815238in}}%
\pgfpathmoveto{\pgfqpoint{3.837930in}{2.819496in}}%
\pgfpathlineto{\pgfqpoint{3.837930in}{2.819496in}}%
\pgfpathlineto{\pgfqpoint{3.837930in}{2.823754in}}%
\pgfpathlineto{\pgfqpoint{3.842187in}{2.823754in}}%
\pgfpathlineto{\pgfqpoint{3.842187in}{2.819496in}}%
\pgfpathmoveto{\pgfqpoint{3.833672in}{2.823754in}}%
\pgfpathlineto{\pgfqpoint{3.833672in}{2.823754in}}%
\pgfpathlineto{\pgfqpoint{3.833672in}{2.828011in}}%
\pgfpathlineto{\pgfqpoint{3.837930in}{2.828011in}}%
\pgfpathlineto{\pgfqpoint{3.837930in}{2.823754in}}%
\pgfpathmoveto{\pgfqpoint{3.833672in}{2.828011in}}%
\pgfpathlineto{\pgfqpoint{3.833672in}{2.828011in}}%
\pgfpathlineto{\pgfqpoint{3.833672in}{2.832269in}}%
\pgfpathlineto{\pgfqpoint{3.837930in}{2.832269in}}%
\pgfpathlineto{\pgfqpoint{3.837930in}{2.828011in}}%
\pgfpathmoveto{\pgfqpoint{3.837930in}{2.823754in}}%
\pgfpathlineto{\pgfqpoint{3.837930in}{2.823754in}}%
\pgfpathlineto{\pgfqpoint{3.837930in}{2.828011in}}%
\pgfpathlineto{\pgfqpoint{3.842187in}{2.828011in}}%
\pgfpathlineto{\pgfqpoint{3.842187in}{2.823754in}}%
\pgfpathmoveto{\pgfqpoint{3.837930in}{2.828011in}}%
\pgfpathlineto{\pgfqpoint{3.837930in}{2.828011in}}%
\pgfpathlineto{\pgfqpoint{3.837930in}{2.832269in}}%
\pgfpathlineto{\pgfqpoint{3.842187in}{2.832269in}}%
\pgfpathlineto{\pgfqpoint{3.842187in}{2.828011in}}%
\pgfpathmoveto{\pgfqpoint{3.833672in}{2.832269in}}%
\pgfpathlineto{\pgfqpoint{3.833672in}{2.832269in}}%
\pgfpathlineto{\pgfqpoint{3.833672in}{2.836526in}}%
\pgfpathlineto{\pgfqpoint{3.837930in}{2.836526in}}%
\pgfpathlineto{\pgfqpoint{3.837930in}{2.832269in}}%
\pgfpathmoveto{\pgfqpoint{3.833672in}{2.836526in}}%
\pgfpathlineto{\pgfqpoint{3.833672in}{2.836526in}}%
\pgfpathlineto{\pgfqpoint{3.833672in}{2.840784in}}%
\pgfpathlineto{\pgfqpoint{3.837930in}{2.840784in}}%
\pgfpathlineto{\pgfqpoint{3.837930in}{2.836526in}}%
\pgfpathmoveto{\pgfqpoint{3.837930in}{2.832269in}}%
\pgfpathlineto{\pgfqpoint{3.837930in}{2.832269in}}%
\pgfpathlineto{\pgfqpoint{3.837930in}{2.836526in}}%
\pgfpathlineto{\pgfqpoint{3.842187in}{2.836526in}}%
\pgfpathlineto{\pgfqpoint{3.842187in}{2.832269in}}%
\pgfpathmoveto{\pgfqpoint{3.837930in}{2.836526in}}%
\pgfpathlineto{\pgfqpoint{3.837930in}{2.836526in}}%
\pgfpathlineto{\pgfqpoint{3.837930in}{2.840784in}}%
\pgfpathlineto{\pgfqpoint{3.842187in}{2.840784in}}%
\pgfpathlineto{\pgfqpoint{3.842187in}{2.836526in}}%
\pgfpathmoveto{\pgfqpoint{3.833672in}{2.840784in}}%
\pgfpathlineto{\pgfqpoint{3.833672in}{2.840784in}}%
\pgfpathlineto{\pgfqpoint{3.833672in}{2.845042in}}%
\pgfpathlineto{\pgfqpoint{3.837930in}{2.845042in}}%
\pgfpathlineto{\pgfqpoint{3.837930in}{2.840784in}}%
\pgfpathmoveto{\pgfqpoint{3.833672in}{2.845042in}}%
\pgfpathlineto{\pgfqpoint{3.833672in}{2.845042in}}%
\pgfpathlineto{\pgfqpoint{3.833672in}{2.849299in}}%
\pgfpathlineto{\pgfqpoint{3.837930in}{2.849299in}}%
\pgfpathlineto{\pgfqpoint{3.837930in}{2.845042in}}%
\pgfpathmoveto{\pgfqpoint{3.837930in}{2.840784in}}%
\pgfpathlineto{\pgfqpoint{3.837930in}{2.840784in}}%
\pgfpathlineto{\pgfqpoint{3.837930in}{2.845042in}}%
\pgfpathlineto{\pgfqpoint{3.842187in}{2.845042in}}%
\pgfpathlineto{\pgfqpoint{3.842187in}{2.840784in}}%
\pgfpathmoveto{\pgfqpoint{3.837930in}{2.845042in}}%
\pgfpathlineto{\pgfqpoint{3.837930in}{2.845042in}}%
\pgfpathlineto{\pgfqpoint{3.837930in}{2.849299in}}%
\pgfpathlineto{\pgfqpoint{3.842187in}{2.849299in}}%
\pgfpathlineto{\pgfqpoint{3.842187in}{2.845042in}}%
\pgfpathmoveto{\pgfqpoint{3.833672in}{2.849299in}}%
\pgfpathlineto{\pgfqpoint{3.833672in}{2.849299in}}%
\pgfpathlineto{\pgfqpoint{3.833672in}{2.853557in}}%
\pgfpathlineto{\pgfqpoint{3.837930in}{2.853557in}}%
\pgfpathlineto{\pgfqpoint{3.837930in}{2.849299in}}%
\pgfpathmoveto{\pgfqpoint{3.837930in}{2.849299in}}%
\pgfpathlineto{\pgfqpoint{3.837930in}{2.849299in}}%
\pgfpathlineto{\pgfqpoint{3.837930in}{2.853557in}}%
\pgfpathlineto{\pgfqpoint{3.842187in}{2.853557in}}%
\pgfpathlineto{\pgfqpoint{3.842187in}{2.849299in}}%
\pgfpathmoveto{\pgfqpoint{3.837930in}{2.853557in}}%
\pgfpathlineto{\pgfqpoint{3.837930in}{2.853557in}}%
\pgfpathlineto{\pgfqpoint{3.837930in}{2.857814in}}%
\pgfpathlineto{\pgfqpoint{3.842187in}{2.857814in}}%
\pgfpathlineto{\pgfqpoint{3.842187in}{2.853557in}}%
\pgfpathmoveto{\pgfqpoint{3.837930in}{2.857814in}}%
\pgfpathlineto{\pgfqpoint{3.837930in}{2.857814in}}%
\pgfpathlineto{\pgfqpoint{3.837930in}{2.862072in}}%
\pgfpathlineto{\pgfqpoint{3.842187in}{2.862072in}}%
\pgfpathlineto{\pgfqpoint{3.842187in}{2.857814in}}%
\pgfpathmoveto{\pgfqpoint{3.837930in}{2.862072in}}%
\pgfpathlineto{\pgfqpoint{3.837930in}{2.862072in}}%
\pgfpathlineto{\pgfqpoint{3.837930in}{2.866329in}}%
\pgfpathlineto{\pgfqpoint{3.842187in}{2.866329in}}%
\pgfpathlineto{\pgfqpoint{3.842187in}{2.862072in}}%
\pgfpathmoveto{\pgfqpoint{3.837930in}{2.866329in}}%
\pgfpathlineto{\pgfqpoint{3.837930in}{2.866329in}}%
\pgfpathlineto{\pgfqpoint{3.837930in}{2.870587in}}%
\pgfpathlineto{\pgfqpoint{3.842187in}{2.870587in}}%
\pgfpathlineto{\pgfqpoint{3.842187in}{2.866329in}}%
\pgfpathmoveto{\pgfqpoint{3.837930in}{2.870587in}}%
\pgfpathlineto{\pgfqpoint{3.837930in}{2.870587in}}%
\pgfpathlineto{\pgfqpoint{3.837930in}{2.874845in}}%
\pgfpathlineto{\pgfqpoint{3.842187in}{2.874845in}}%
\pgfpathlineto{\pgfqpoint{3.842187in}{2.870587in}}%
\pgfpathmoveto{\pgfqpoint{3.837930in}{2.874845in}}%
\pgfpathlineto{\pgfqpoint{3.837930in}{2.874845in}}%
\pgfpathlineto{\pgfqpoint{3.837930in}{2.879102in}}%
\pgfpathlineto{\pgfqpoint{3.842187in}{2.879102in}}%
\pgfpathlineto{\pgfqpoint{3.842187in}{2.874845in}}%
\pgfpathmoveto{\pgfqpoint{3.837930in}{2.879102in}}%
\pgfpathlineto{\pgfqpoint{3.837930in}{2.879102in}}%
\pgfpathlineto{\pgfqpoint{3.837930in}{2.883360in}}%
\pgfpathlineto{\pgfqpoint{3.842187in}{2.883360in}}%
\pgfpathlineto{\pgfqpoint{3.842187in}{2.879102in}}%
\pgfpathmoveto{\pgfqpoint{3.842187in}{2.845042in}}%
\pgfpathlineto{\pgfqpoint{3.842187in}{2.845042in}}%
\pgfpathlineto{\pgfqpoint{3.842187in}{2.849299in}}%
\pgfpathlineto{\pgfqpoint{3.846445in}{2.849299in}}%
\pgfpathlineto{\pgfqpoint{3.846445in}{2.845042in}}%
\pgfpathmoveto{\pgfqpoint{3.842187in}{2.849299in}}%
\pgfpathlineto{\pgfqpoint{3.842187in}{2.849299in}}%
\pgfpathlineto{\pgfqpoint{3.842187in}{2.853557in}}%
\pgfpathlineto{\pgfqpoint{3.846445in}{2.853557in}}%
\pgfpathlineto{\pgfqpoint{3.846445in}{2.849299in}}%
\pgfpathmoveto{\pgfqpoint{3.842187in}{2.853557in}}%
\pgfpathlineto{\pgfqpoint{3.842187in}{2.853557in}}%
\pgfpathlineto{\pgfqpoint{3.842187in}{2.857814in}}%
\pgfpathlineto{\pgfqpoint{3.846445in}{2.857814in}}%
\pgfpathlineto{\pgfqpoint{3.846445in}{2.853557in}}%
\pgfpathmoveto{\pgfqpoint{3.842187in}{2.857814in}}%
\pgfpathlineto{\pgfqpoint{3.842187in}{2.857814in}}%
\pgfpathlineto{\pgfqpoint{3.842187in}{2.862072in}}%
\pgfpathlineto{\pgfqpoint{3.846445in}{2.862072in}}%
\pgfpathlineto{\pgfqpoint{3.846445in}{2.857814in}}%
\pgfpathmoveto{\pgfqpoint{3.842187in}{2.862072in}}%
\pgfpathlineto{\pgfqpoint{3.842187in}{2.862072in}}%
\pgfpathlineto{\pgfqpoint{3.842187in}{2.866329in}}%
\pgfpathlineto{\pgfqpoint{3.846445in}{2.866329in}}%
\pgfpathlineto{\pgfqpoint{3.846445in}{2.862072in}}%
\pgfpathmoveto{\pgfqpoint{3.842187in}{2.866329in}}%
\pgfpathlineto{\pgfqpoint{3.842187in}{2.866329in}}%
\pgfpathlineto{\pgfqpoint{3.842187in}{2.870587in}}%
\pgfpathlineto{\pgfqpoint{3.846445in}{2.870587in}}%
\pgfpathlineto{\pgfqpoint{3.846445in}{2.866329in}}%
\pgfpathmoveto{\pgfqpoint{3.842187in}{2.870587in}}%
\pgfpathlineto{\pgfqpoint{3.842187in}{2.870587in}}%
\pgfpathlineto{\pgfqpoint{3.842187in}{2.874845in}}%
\pgfpathlineto{\pgfqpoint{3.846445in}{2.874845in}}%
\pgfpathlineto{\pgfqpoint{3.846445in}{2.870587in}}%
\pgfpathmoveto{\pgfqpoint{3.842187in}{2.874845in}}%
\pgfpathlineto{\pgfqpoint{3.842187in}{2.874845in}}%
\pgfpathlineto{\pgfqpoint{3.842187in}{2.879102in}}%
\pgfpathlineto{\pgfqpoint{3.846445in}{2.879102in}}%
\pgfpathlineto{\pgfqpoint{3.846445in}{2.874845in}}%
\pgfpathmoveto{\pgfqpoint{3.842187in}{2.879102in}}%
\pgfpathlineto{\pgfqpoint{3.842187in}{2.879102in}}%
\pgfpathlineto{\pgfqpoint{3.842187in}{2.883360in}}%
\pgfpathlineto{\pgfqpoint{3.846445in}{2.883360in}}%
\pgfpathlineto{\pgfqpoint{3.846445in}{2.879102in}}%
\pgfpathmoveto{\pgfqpoint{3.846445in}{2.879102in}}%
\pgfpathlineto{\pgfqpoint{3.846445in}{2.879102in}}%
\pgfpathlineto{\pgfqpoint{3.846445in}{2.883360in}}%
\pgfpathlineto{\pgfqpoint{3.850703in}{2.883360in}}%
\pgfpathlineto{\pgfqpoint{3.850703in}{2.879102in}}%
\pgfpathmoveto{\pgfqpoint{3.842187in}{2.883360in}}%
\pgfpathlineto{\pgfqpoint{3.842187in}{2.883360in}}%
\pgfpathlineto{\pgfqpoint{3.842187in}{2.887617in}}%
\pgfpathlineto{\pgfqpoint{3.846445in}{2.887617in}}%
\pgfpathlineto{\pgfqpoint{3.846445in}{2.883360in}}%
\pgfpathmoveto{\pgfqpoint{3.842187in}{2.887617in}}%
\pgfpathlineto{\pgfqpoint{3.842187in}{2.887617in}}%
\pgfpathlineto{\pgfqpoint{3.842187in}{2.891875in}}%
\pgfpathlineto{\pgfqpoint{3.846445in}{2.891875in}}%
\pgfpathlineto{\pgfqpoint{3.846445in}{2.887617in}}%
\pgfpathmoveto{\pgfqpoint{3.846445in}{2.883360in}}%
\pgfpathlineto{\pgfqpoint{3.846445in}{2.883360in}}%
\pgfpathlineto{\pgfqpoint{3.846445in}{2.887617in}}%
\pgfpathlineto{\pgfqpoint{3.850703in}{2.887617in}}%
\pgfpathlineto{\pgfqpoint{3.850703in}{2.883360in}}%
\pgfpathmoveto{\pgfqpoint{3.846445in}{2.887617in}}%
\pgfpathlineto{\pgfqpoint{3.846445in}{2.887617in}}%
\pgfpathlineto{\pgfqpoint{3.846445in}{2.891875in}}%
\pgfpathlineto{\pgfqpoint{3.850703in}{2.891875in}}%
\pgfpathlineto{\pgfqpoint{3.850703in}{2.887617in}}%
\pgfpathmoveto{\pgfqpoint{3.842187in}{2.891875in}}%
\pgfpathlineto{\pgfqpoint{3.842187in}{2.891875in}}%
\pgfpathlineto{\pgfqpoint{3.842187in}{2.896133in}}%
\pgfpathlineto{\pgfqpoint{3.846445in}{2.896133in}}%
\pgfpathlineto{\pgfqpoint{3.846445in}{2.891875in}}%
\pgfpathmoveto{\pgfqpoint{3.842187in}{2.896133in}}%
\pgfpathlineto{\pgfqpoint{3.842187in}{2.896133in}}%
\pgfpathlineto{\pgfqpoint{3.842187in}{2.900390in}}%
\pgfpathlineto{\pgfqpoint{3.846445in}{2.900390in}}%
\pgfpathlineto{\pgfqpoint{3.846445in}{2.896133in}}%
\pgfpathmoveto{\pgfqpoint{3.846445in}{2.891875in}}%
\pgfpathlineto{\pgfqpoint{3.846445in}{2.891875in}}%
\pgfpathlineto{\pgfqpoint{3.846445in}{2.896133in}}%
\pgfpathlineto{\pgfqpoint{3.850703in}{2.896133in}}%
\pgfpathlineto{\pgfqpoint{3.850703in}{2.891875in}}%
\pgfpathmoveto{\pgfqpoint{3.846445in}{2.896133in}}%
\pgfpathlineto{\pgfqpoint{3.846445in}{2.896133in}}%
\pgfpathlineto{\pgfqpoint{3.846445in}{2.900390in}}%
\pgfpathlineto{\pgfqpoint{3.850703in}{2.900390in}}%
\pgfpathlineto{\pgfqpoint{3.850703in}{2.896133in}}%
\pgfpathmoveto{\pgfqpoint{3.842187in}{2.900390in}}%
\pgfpathlineto{\pgfqpoint{3.842187in}{2.900390in}}%
\pgfpathlineto{\pgfqpoint{3.842187in}{2.904648in}}%
\pgfpathlineto{\pgfqpoint{3.846445in}{2.904648in}}%
\pgfpathlineto{\pgfqpoint{3.846445in}{2.900390in}}%
\pgfpathmoveto{\pgfqpoint{3.842187in}{2.904648in}}%
\pgfpathlineto{\pgfqpoint{3.842187in}{2.904648in}}%
\pgfpathlineto{\pgfqpoint{3.842187in}{2.908905in}}%
\pgfpathlineto{\pgfqpoint{3.846445in}{2.908905in}}%
\pgfpathlineto{\pgfqpoint{3.846445in}{2.904648in}}%
\pgfpathmoveto{\pgfqpoint{3.846445in}{2.900390in}}%
\pgfpathlineto{\pgfqpoint{3.846445in}{2.900390in}}%
\pgfpathlineto{\pgfqpoint{3.846445in}{2.904648in}}%
\pgfpathlineto{\pgfqpoint{3.850703in}{2.904648in}}%
\pgfpathlineto{\pgfqpoint{3.850703in}{2.900390in}}%
\pgfpathmoveto{\pgfqpoint{3.846445in}{2.904648in}}%
\pgfpathlineto{\pgfqpoint{3.846445in}{2.904648in}}%
\pgfpathlineto{\pgfqpoint{3.846445in}{2.908905in}}%
\pgfpathlineto{\pgfqpoint{3.850703in}{2.908905in}}%
\pgfpathlineto{\pgfqpoint{3.850703in}{2.904648in}}%
\pgfpathmoveto{\pgfqpoint{3.842187in}{2.908905in}}%
\pgfpathlineto{\pgfqpoint{3.842187in}{2.908905in}}%
\pgfpathlineto{\pgfqpoint{3.842187in}{2.913163in}}%
\pgfpathlineto{\pgfqpoint{3.846445in}{2.913163in}}%
\pgfpathlineto{\pgfqpoint{3.846445in}{2.908905in}}%
\pgfpathmoveto{\pgfqpoint{3.842187in}{2.913163in}}%
\pgfpathlineto{\pgfqpoint{3.842187in}{2.913163in}}%
\pgfpathlineto{\pgfqpoint{3.842187in}{2.917421in}}%
\pgfpathlineto{\pgfqpoint{3.846445in}{2.917421in}}%
\pgfpathlineto{\pgfqpoint{3.846445in}{2.913163in}}%
\pgfpathmoveto{\pgfqpoint{3.846445in}{2.908905in}}%
\pgfpathlineto{\pgfqpoint{3.846445in}{2.908905in}}%
\pgfpathlineto{\pgfqpoint{3.846445in}{2.913163in}}%
\pgfpathlineto{\pgfqpoint{3.850703in}{2.913163in}}%
\pgfpathlineto{\pgfqpoint{3.850703in}{2.908905in}}%
\pgfpathmoveto{\pgfqpoint{3.846445in}{2.913163in}}%
\pgfpathlineto{\pgfqpoint{3.846445in}{2.913163in}}%
\pgfpathlineto{\pgfqpoint{3.846445in}{2.917421in}}%
\pgfpathlineto{\pgfqpoint{3.850703in}{2.917421in}}%
\pgfpathlineto{\pgfqpoint{3.850703in}{2.913163in}}%
\pgfpathmoveto{\pgfqpoint{3.846445in}{2.917421in}}%
\pgfpathlineto{\pgfqpoint{3.846445in}{2.917421in}}%
\pgfpathlineto{\pgfqpoint{3.846445in}{2.921678in}}%
\pgfpathlineto{\pgfqpoint{3.850703in}{2.921678in}}%
\pgfpathlineto{\pgfqpoint{3.850703in}{2.917421in}}%
\pgfpathmoveto{\pgfqpoint{3.846445in}{2.921678in}}%
\pgfpathlineto{\pgfqpoint{3.846445in}{2.921678in}}%
\pgfpathlineto{\pgfqpoint{3.846445in}{2.925936in}}%
\pgfpathlineto{\pgfqpoint{3.850703in}{2.925936in}}%
\pgfpathlineto{\pgfqpoint{3.850703in}{2.921678in}}%
\pgfpathmoveto{\pgfqpoint{3.850703in}{2.908905in}}%
\pgfpathlineto{\pgfqpoint{3.850703in}{2.908905in}}%
\pgfpathlineto{\pgfqpoint{3.850703in}{2.913163in}}%
\pgfpathlineto{\pgfqpoint{3.854961in}{2.913163in}}%
\pgfpathlineto{\pgfqpoint{3.854961in}{2.908905in}}%
\pgfpathmoveto{\pgfqpoint{3.850703in}{2.913163in}}%
\pgfpathlineto{\pgfqpoint{3.850703in}{2.913163in}}%
\pgfpathlineto{\pgfqpoint{3.850703in}{2.917421in}}%
\pgfpathlineto{\pgfqpoint{3.854961in}{2.917421in}}%
\pgfpathlineto{\pgfqpoint{3.854961in}{2.913163in}}%
\pgfpathmoveto{\pgfqpoint{3.850703in}{2.917421in}}%
\pgfpathlineto{\pgfqpoint{3.850703in}{2.917421in}}%
\pgfpathlineto{\pgfqpoint{3.850703in}{2.921678in}}%
\pgfpathlineto{\pgfqpoint{3.854961in}{2.921678in}}%
\pgfpathlineto{\pgfqpoint{3.854961in}{2.917421in}}%
\pgfpathmoveto{\pgfqpoint{3.850703in}{2.921678in}}%
\pgfpathlineto{\pgfqpoint{3.850703in}{2.921678in}}%
\pgfpathlineto{\pgfqpoint{3.850703in}{2.925936in}}%
\pgfpathlineto{\pgfqpoint{3.854961in}{2.925936in}}%
\pgfpathlineto{\pgfqpoint{3.854961in}{2.921678in}}%
\pgfpathmoveto{\pgfqpoint{3.846445in}{2.925936in}}%
\pgfpathlineto{\pgfqpoint{3.846445in}{2.925936in}}%
\pgfpathlineto{\pgfqpoint{3.846445in}{2.930193in}}%
\pgfpathlineto{\pgfqpoint{3.850703in}{2.930193in}}%
\pgfpathlineto{\pgfqpoint{3.850703in}{2.925936in}}%
\pgfpathmoveto{\pgfqpoint{3.846445in}{2.930193in}}%
\pgfpathlineto{\pgfqpoint{3.846445in}{2.930193in}}%
\pgfpathlineto{\pgfqpoint{3.846445in}{2.934451in}}%
\pgfpathlineto{\pgfqpoint{3.850703in}{2.934451in}}%
\pgfpathlineto{\pgfqpoint{3.850703in}{2.930193in}}%
\pgfpathmoveto{\pgfqpoint{3.846445in}{2.934451in}}%
\pgfpathlineto{\pgfqpoint{3.846445in}{2.934451in}}%
\pgfpathlineto{\pgfqpoint{3.846445in}{2.938708in}}%
\pgfpathlineto{\pgfqpoint{3.850703in}{2.938708in}}%
\pgfpathlineto{\pgfqpoint{3.850703in}{2.934451in}}%
\pgfpathmoveto{\pgfqpoint{3.846445in}{2.938708in}}%
\pgfpathlineto{\pgfqpoint{3.846445in}{2.938708in}}%
\pgfpathlineto{\pgfqpoint{3.846445in}{2.942966in}}%
\pgfpathlineto{\pgfqpoint{3.850703in}{2.942966in}}%
\pgfpathlineto{\pgfqpoint{3.850703in}{2.938708in}}%
\pgfpathmoveto{\pgfqpoint{3.850703in}{2.925936in}}%
\pgfpathlineto{\pgfqpoint{3.850703in}{2.925936in}}%
\pgfpathlineto{\pgfqpoint{3.850703in}{2.930193in}}%
\pgfpathlineto{\pgfqpoint{3.854961in}{2.930193in}}%
\pgfpathlineto{\pgfqpoint{3.854961in}{2.925936in}}%
\pgfpathmoveto{\pgfqpoint{3.850703in}{2.930193in}}%
\pgfpathlineto{\pgfqpoint{3.850703in}{2.930193in}}%
\pgfpathlineto{\pgfqpoint{3.850703in}{2.934451in}}%
\pgfpathlineto{\pgfqpoint{3.854961in}{2.934451in}}%
\pgfpathlineto{\pgfqpoint{3.854961in}{2.930193in}}%
\pgfpathmoveto{\pgfqpoint{3.850703in}{2.934451in}}%
\pgfpathlineto{\pgfqpoint{3.850703in}{2.934451in}}%
\pgfpathlineto{\pgfqpoint{3.850703in}{2.938708in}}%
\pgfpathlineto{\pgfqpoint{3.854961in}{2.938708in}}%
\pgfpathlineto{\pgfqpoint{3.854961in}{2.934451in}}%
\pgfpathmoveto{\pgfqpoint{3.850703in}{2.938708in}}%
\pgfpathlineto{\pgfqpoint{3.850703in}{2.938708in}}%
\pgfpathlineto{\pgfqpoint{3.850703in}{2.942966in}}%
\pgfpathlineto{\pgfqpoint{3.854961in}{2.942966in}}%
\pgfpathlineto{\pgfqpoint{3.854961in}{2.938708in}}%
\pgfpathmoveto{\pgfqpoint{3.846445in}{2.942966in}}%
\pgfpathlineto{\pgfqpoint{3.846445in}{2.942966in}}%
\pgfpathlineto{\pgfqpoint{3.846445in}{2.947224in}}%
\pgfpathlineto{\pgfqpoint{3.850703in}{2.947224in}}%
\pgfpathlineto{\pgfqpoint{3.850703in}{2.942966in}}%
\pgfpathmoveto{\pgfqpoint{3.850703in}{2.942966in}}%
\pgfpathlineto{\pgfqpoint{3.850703in}{2.942966in}}%
\pgfpathlineto{\pgfqpoint{3.850703in}{2.947224in}}%
\pgfpathlineto{\pgfqpoint{3.854961in}{2.947224in}}%
\pgfpathlineto{\pgfqpoint{3.854961in}{2.942966in}}%
\pgfpathmoveto{\pgfqpoint{3.850703in}{2.947224in}}%
\pgfpathlineto{\pgfqpoint{3.850703in}{2.947224in}}%
\pgfpathlineto{\pgfqpoint{3.850703in}{2.951481in}}%
\pgfpathlineto{\pgfqpoint{3.854961in}{2.951481in}}%
\pgfpathlineto{\pgfqpoint{3.854961in}{2.947224in}}%
\pgfpathmoveto{\pgfqpoint{3.854961in}{2.942966in}}%
\pgfpathlineto{\pgfqpoint{3.854961in}{2.942966in}}%
\pgfpathlineto{\pgfqpoint{3.854961in}{2.947224in}}%
\pgfpathlineto{\pgfqpoint{3.859219in}{2.947224in}}%
\pgfpathlineto{\pgfqpoint{3.859219in}{2.942966in}}%
\pgfpathmoveto{\pgfqpoint{3.854961in}{2.947224in}}%
\pgfpathlineto{\pgfqpoint{3.854961in}{2.947224in}}%
\pgfpathlineto{\pgfqpoint{3.854961in}{2.951481in}}%
\pgfpathlineto{\pgfqpoint{3.859219in}{2.951481in}}%
\pgfpathlineto{\pgfqpoint{3.859219in}{2.947224in}}%
\pgfpathmoveto{\pgfqpoint{3.850703in}{2.951481in}}%
\pgfpathlineto{\pgfqpoint{3.850703in}{2.951481in}}%
\pgfpathlineto{\pgfqpoint{3.850703in}{2.955739in}}%
\pgfpathlineto{\pgfqpoint{3.854961in}{2.955739in}}%
\pgfpathlineto{\pgfqpoint{3.854961in}{2.951481in}}%
\pgfpathmoveto{\pgfqpoint{3.850703in}{2.955739in}}%
\pgfpathlineto{\pgfqpoint{3.850703in}{2.955739in}}%
\pgfpathlineto{\pgfqpoint{3.850703in}{2.959996in}}%
\pgfpathlineto{\pgfqpoint{3.854961in}{2.959996in}}%
\pgfpathlineto{\pgfqpoint{3.854961in}{2.955739in}}%
\pgfpathmoveto{\pgfqpoint{3.854961in}{2.951481in}}%
\pgfpathlineto{\pgfqpoint{3.854961in}{2.951481in}}%
\pgfpathlineto{\pgfqpoint{3.854961in}{2.955739in}}%
\pgfpathlineto{\pgfqpoint{3.859219in}{2.955739in}}%
\pgfpathlineto{\pgfqpoint{3.859219in}{2.951481in}}%
\pgfpathmoveto{\pgfqpoint{3.854961in}{2.955739in}}%
\pgfpathlineto{\pgfqpoint{3.854961in}{2.955739in}}%
\pgfpathlineto{\pgfqpoint{3.854961in}{2.959996in}}%
\pgfpathlineto{\pgfqpoint{3.859219in}{2.959996in}}%
\pgfpathlineto{\pgfqpoint{3.859219in}{2.955739in}}%
\pgfpathmoveto{\pgfqpoint{3.850703in}{2.959996in}}%
\pgfpathlineto{\pgfqpoint{3.850703in}{2.959996in}}%
\pgfpathlineto{\pgfqpoint{3.850703in}{2.964254in}}%
\pgfpathlineto{\pgfqpoint{3.854961in}{2.964254in}}%
\pgfpathlineto{\pgfqpoint{3.854961in}{2.959996in}}%
\pgfpathmoveto{\pgfqpoint{3.850703in}{2.964254in}}%
\pgfpathlineto{\pgfqpoint{3.850703in}{2.964254in}}%
\pgfpathlineto{\pgfqpoint{3.850703in}{2.968512in}}%
\pgfpathlineto{\pgfqpoint{3.854961in}{2.968512in}}%
\pgfpathlineto{\pgfqpoint{3.854961in}{2.964254in}}%
\pgfpathmoveto{\pgfqpoint{3.854961in}{2.959996in}}%
\pgfpathlineto{\pgfqpoint{3.854961in}{2.959996in}}%
\pgfpathlineto{\pgfqpoint{3.854961in}{2.964254in}}%
\pgfpathlineto{\pgfqpoint{3.859219in}{2.964254in}}%
\pgfpathlineto{\pgfqpoint{3.859219in}{2.959996in}}%
\pgfpathmoveto{\pgfqpoint{3.854961in}{2.964254in}}%
\pgfpathlineto{\pgfqpoint{3.854961in}{2.964254in}}%
\pgfpathlineto{\pgfqpoint{3.854961in}{2.968512in}}%
\pgfpathlineto{\pgfqpoint{3.859219in}{2.968512in}}%
\pgfpathlineto{\pgfqpoint{3.859219in}{2.964254in}}%
\pgfpathmoveto{\pgfqpoint{3.850703in}{2.968512in}}%
\pgfpathlineto{\pgfqpoint{3.850703in}{2.968512in}}%
\pgfpathlineto{\pgfqpoint{3.850703in}{2.972770in}}%
\pgfpathlineto{\pgfqpoint{3.854961in}{2.972770in}}%
\pgfpathlineto{\pgfqpoint{3.854961in}{2.968512in}}%
\pgfpathmoveto{\pgfqpoint{3.850703in}{2.972770in}}%
\pgfpathlineto{\pgfqpoint{3.850703in}{2.972770in}}%
\pgfpathlineto{\pgfqpoint{3.850703in}{2.977028in}}%
\pgfpathlineto{\pgfqpoint{3.854961in}{2.977028in}}%
\pgfpathlineto{\pgfqpoint{3.854961in}{2.972770in}}%
\pgfpathmoveto{\pgfqpoint{3.854961in}{2.968512in}}%
\pgfpathlineto{\pgfqpoint{3.854961in}{2.968512in}}%
\pgfpathlineto{\pgfqpoint{3.854961in}{2.972770in}}%
\pgfpathlineto{\pgfqpoint{3.859219in}{2.972770in}}%
\pgfpathlineto{\pgfqpoint{3.859219in}{2.968512in}}%
\pgfpathmoveto{\pgfqpoint{3.854961in}{2.972770in}}%
\pgfpathlineto{\pgfqpoint{3.854961in}{2.972770in}}%
\pgfpathlineto{\pgfqpoint{3.854961in}{2.977028in}}%
\pgfpathlineto{\pgfqpoint{3.859219in}{2.977028in}}%
\pgfpathlineto{\pgfqpoint{3.859219in}{2.972770in}}%
\pgfpathmoveto{\pgfqpoint{3.850703in}{2.977028in}}%
\pgfpathlineto{\pgfqpoint{3.850703in}{2.977028in}}%
\pgfpathlineto{\pgfqpoint{3.850703in}{2.981286in}}%
\pgfpathlineto{\pgfqpoint{3.854961in}{2.981286in}}%
\pgfpathlineto{\pgfqpoint{3.854961in}{2.977028in}}%
\pgfpathmoveto{\pgfqpoint{3.854961in}{2.977028in}}%
\pgfpathlineto{\pgfqpoint{3.854961in}{2.977028in}}%
\pgfpathlineto{\pgfqpoint{3.854961in}{2.981286in}}%
\pgfpathlineto{\pgfqpoint{3.859219in}{2.981286in}}%
\pgfpathlineto{\pgfqpoint{3.859219in}{2.977028in}}%
\pgfpathmoveto{\pgfqpoint{3.854961in}{2.981286in}}%
\pgfpathlineto{\pgfqpoint{3.854961in}{2.981286in}}%
\pgfpathlineto{\pgfqpoint{3.854961in}{2.985543in}}%
\pgfpathlineto{\pgfqpoint{3.859219in}{2.985543in}}%
\pgfpathlineto{\pgfqpoint{3.859219in}{2.981286in}}%
\pgfpathmoveto{\pgfqpoint{3.854961in}{2.985543in}}%
\pgfpathlineto{\pgfqpoint{3.854961in}{2.985543in}}%
\pgfpathlineto{\pgfqpoint{3.854961in}{2.989801in}}%
\pgfpathlineto{\pgfqpoint{3.859219in}{2.989801in}}%
\pgfpathlineto{\pgfqpoint{3.859219in}{2.985543in}}%
\pgfpathmoveto{\pgfqpoint{3.854961in}{2.989801in}}%
\pgfpathlineto{\pgfqpoint{3.854961in}{2.989801in}}%
\pgfpathlineto{\pgfqpoint{3.854961in}{2.994059in}}%
\pgfpathlineto{\pgfqpoint{3.859219in}{2.994059in}}%
\pgfpathlineto{\pgfqpoint{3.859219in}{2.989801in}}%
\pgfpathmoveto{\pgfqpoint{3.859219in}{2.972770in}}%
\pgfpathlineto{\pgfqpoint{3.859219in}{2.972770in}}%
\pgfpathlineto{\pgfqpoint{3.859219in}{2.977028in}}%
\pgfpathlineto{\pgfqpoint{3.863477in}{2.977028in}}%
\pgfpathlineto{\pgfqpoint{3.863477in}{2.972770in}}%
\pgfpathmoveto{\pgfqpoint{3.859219in}{2.977028in}}%
\pgfpathlineto{\pgfqpoint{3.859219in}{2.977028in}}%
\pgfpathlineto{\pgfqpoint{3.859219in}{2.981286in}}%
\pgfpathlineto{\pgfqpoint{3.863477in}{2.981286in}}%
\pgfpathlineto{\pgfqpoint{3.863477in}{2.977028in}}%
\pgfpathmoveto{\pgfqpoint{3.859219in}{2.981286in}}%
\pgfpathlineto{\pgfqpoint{3.859219in}{2.981286in}}%
\pgfpathlineto{\pgfqpoint{3.859219in}{2.985543in}}%
\pgfpathlineto{\pgfqpoint{3.863477in}{2.985543in}}%
\pgfpathlineto{\pgfqpoint{3.863477in}{2.981286in}}%
\pgfpathmoveto{\pgfqpoint{3.859219in}{2.985543in}}%
\pgfpathlineto{\pgfqpoint{3.859219in}{2.985543in}}%
\pgfpathlineto{\pgfqpoint{3.859219in}{2.989801in}}%
\pgfpathlineto{\pgfqpoint{3.863477in}{2.989801in}}%
\pgfpathlineto{\pgfqpoint{3.863477in}{2.985543in}}%
\pgfpathmoveto{\pgfqpoint{3.859219in}{2.989801in}}%
\pgfpathlineto{\pgfqpoint{3.859219in}{2.989801in}}%
\pgfpathlineto{\pgfqpoint{3.859219in}{2.994059in}}%
\pgfpathlineto{\pgfqpoint{3.863477in}{2.994059in}}%
\pgfpathlineto{\pgfqpoint{3.863477in}{2.989801in}}%
\pgfpathmoveto{\pgfqpoint{3.854961in}{2.994059in}}%
\pgfpathlineto{\pgfqpoint{3.854961in}{2.994059in}}%
\pgfpathlineto{\pgfqpoint{3.854961in}{2.998317in}}%
\pgfpathlineto{\pgfqpoint{3.859219in}{2.998317in}}%
\pgfpathlineto{\pgfqpoint{3.859219in}{2.994059in}}%
\pgfpathmoveto{\pgfqpoint{3.854961in}{2.998317in}}%
\pgfpathlineto{\pgfqpoint{3.854961in}{2.998317in}}%
\pgfpathlineto{\pgfqpoint{3.854961in}{3.002575in}}%
\pgfpathlineto{\pgfqpoint{3.859219in}{3.002575in}}%
\pgfpathlineto{\pgfqpoint{3.859219in}{2.998317in}}%
\pgfpathmoveto{\pgfqpoint{3.854961in}{3.002575in}}%
\pgfpathlineto{\pgfqpoint{3.854961in}{3.002575in}}%
\pgfpathlineto{\pgfqpoint{3.854961in}{3.006833in}}%
\pgfpathlineto{\pgfqpoint{3.859219in}{3.006833in}}%
\pgfpathlineto{\pgfqpoint{3.859219in}{3.002575in}}%
\pgfpathmoveto{\pgfqpoint{3.854961in}{3.006833in}}%
\pgfpathlineto{\pgfqpoint{3.854961in}{3.006833in}}%
\pgfpathlineto{\pgfqpoint{3.854961in}{3.011090in}}%
\pgfpathlineto{\pgfqpoint{3.859219in}{3.011090in}}%
\pgfpathlineto{\pgfqpoint{3.859219in}{3.006833in}}%
\pgfpathmoveto{\pgfqpoint{3.859219in}{2.994059in}}%
\pgfpathlineto{\pgfqpoint{3.859219in}{2.994059in}}%
\pgfpathlineto{\pgfqpoint{3.859219in}{2.998317in}}%
\pgfpathlineto{\pgfqpoint{3.863477in}{2.998317in}}%
\pgfpathlineto{\pgfqpoint{3.863477in}{2.994059in}}%
\pgfpathmoveto{\pgfqpoint{3.859219in}{2.998317in}}%
\pgfpathlineto{\pgfqpoint{3.859219in}{2.998317in}}%
\pgfpathlineto{\pgfqpoint{3.859219in}{3.002575in}}%
\pgfpathlineto{\pgfqpoint{3.863477in}{3.002575in}}%
\pgfpathlineto{\pgfqpoint{3.863477in}{2.998317in}}%
\pgfpathmoveto{\pgfqpoint{3.859219in}{3.002575in}}%
\pgfpathlineto{\pgfqpoint{3.859219in}{3.002575in}}%
\pgfpathlineto{\pgfqpoint{3.859219in}{3.006833in}}%
\pgfpathlineto{\pgfqpoint{3.863477in}{3.006833in}}%
\pgfpathlineto{\pgfqpoint{3.863477in}{3.002575in}}%
\pgfpathmoveto{\pgfqpoint{3.859219in}{3.006833in}}%
\pgfpathlineto{\pgfqpoint{3.859219in}{3.006833in}}%
\pgfpathlineto{\pgfqpoint{3.859219in}{3.011090in}}%
\pgfpathlineto{\pgfqpoint{3.863477in}{3.011090in}}%
\pgfpathlineto{\pgfqpoint{3.863477in}{3.006833in}}%
\pgfpathmoveto{\pgfqpoint{3.863477in}{3.006833in}}%
\pgfpathlineto{\pgfqpoint{3.863477in}{3.006833in}}%
\pgfpathlineto{\pgfqpoint{3.863477in}{3.011090in}}%
\pgfpathlineto{\pgfqpoint{3.867734in}{3.011090in}}%
\pgfpathlineto{\pgfqpoint{3.867734in}{3.006833in}}%
\pgfpathmoveto{\pgfqpoint{3.859219in}{3.011090in}}%
\pgfpathlineto{\pgfqpoint{3.859219in}{3.011090in}}%
\pgfpathlineto{\pgfqpoint{3.859219in}{3.015348in}}%
\pgfpathlineto{\pgfqpoint{3.863477in}{3.015348in}}%
\pgfpathlineto{\pgfqpoint{3.863477in}{3.011090in}}%
\pgfpathmoveto{\pgfqpoint{3.859219in}{3.015348in}}%
\pgfpathlineto{\pgfqpoint{3.859219in}{3.015348in}}%
\pgfpathlineto{\pgfqpoint{3.859219in}{3.019606in}}%
\pgfpathlineto{\pgfqpoint{3.863477in}{3.019606in}}%
\pgfpathlineto{\pgfqpoint{3.863477in}{3.015348in}}%
\pgfpathmoveto{\pgfqpoint{3.863477in}{3.011090in}}%
\pgfpathlineto{\pgfqpoint{3.863477in}{3.011090in}}%
\pgfpathlineto{\pgfqpoint{3.863477in}{3.015348in}}%
\pgfpathlineto{\pgfqpoint{3.867734in}{3.015348in}}%
\pgfpathlineto{\pgfqpoint{3.867734in}{3.011090in}}%
\pgfpathmoveto{\pgfqpoint{3.863477in}{3.015348in}}%
\pgfpathlineto{\pgfqpoint{3.863477in}{3.015348in}}%
\pgfpathlineto{\pgfqpoint{3.863477in}{3.019606in}}%
\pgfpathlineto{\pgfqpoint{3.867734in}{3.019606in}}%
\pgfpathlineto{\pgfqpoint{3.867734in}{3.015348in}}%
\pgfpathmoveto{\pgfqpoint{3.859219in}{3.019606in}}%
\pgfpathlineto{\pgfqpoint{3.859219in}{3.019606in}}%
\pgfpathlineto{\pgfqpoint{3.859219in}{3.023864in}}%
\pgfpathlineto{\pgfqpoint{3.863477in}{3.023864in}}%
\pgfpathlineto{\pgfqpoint{3.863477in}{3.019606in}}%
\pgfpathmoveto{\pgfqpoint{3.859219in}{3.023864in}}%
\pgfpathlineto{\pgfqpoint{3.859219in}{3.023864in}}%
\pgfpathlineto{\pgfqpoint{3.859219in}{3.028122in}}%
\pgfpathlineto{\pgfqpoint{3.863477in}{3.028122in}}%
\pgfpathlineto{\pgfqpoint{3.863477in}{3.023864in}}%
\pgfpathmoveto{\pgfqpoint{3.863477in}{3.019606in}}%
\pgfpathlineto{\pgfqpoint{3.863477in}{3.019606in}}%
\pgfpathlineto{\pgfqpoint{3.863477in}{3.023864in}}%
\pgfpathlineto{\pgfqpoint{3.867734in}{3.023864in}}%
\pgfpathlineto{\pgfqpoint{3.867734in}{3.019606in}}%
\pgfpathmoveto{\pgfqpoint{3.863477in}{3.023864in}}%
\pgfpathlineto{\pgfqpoint{3.863477in}{3.023864in}}%
\pgfpathlineto{\pgfqpoint{3.863477in}{3.028122in}}%
\pgfpathlineto{\pgfqpoint{3.867734in}{3.028122in}}%
\pgfpathlineto{\pgfqpoint{3.867734in}{3.023864in}}%
\pgfpathmoveto{\pgfqpoint{3.859219in}{3.028122in}}%
\pgfpathlineto{\pgfqpoint{3.859219in}{3.028122in}}%
\pgfpathlineto{\pgfqpoint{3.859219in}{3.032380in}}%
\pgfpathlineto{\pgfqpoint{3.863477in}{3.032380in}}%
\pgfpathlineto{\pgfqpoint{3.863477in}{3.028122in}}%
\pgfpathmoveto{\pgfqpoint{3.859219in}{3.032380in}}%
\pgfpathlineto{\pgfqpoint{3.859219in}{3.032380in}}%
\pgfpathlineto{\pgfqpoint{3.859219in}{3.036637in}}%
\pgfpathlineto{\pgfqpoint{3.863477in}{3.036637in}}%
\pgfpathlineto{\pgfqpoint{3.863477in}{3.032380in}}%
\pgfpathmoveto{\pgfqpoint{3.863477in}{3.028122in}}%
\pgfpathlineto{\pgfqpoint{3.863477in}{3.028122in}}%
\pgfpathlineto{\pgfqpoint{3.863477in}{3.032380in}}%
\pgfpathlineto{\pgfqpoint{3.867734in}{3.032380in}}%
\pgfpathlineto{\pgfqpoint{3.867734in}{3.028122in}}%
\pgfpathmoveto{\pgfqpoint{3.863477in}{3.032380in}}%
\pgfpathlineto{\pgfqpoint{3.863477in}{3.032380in}}%
\pgfpathlineto{\pgfqpoint{3.863477in}{3.036637in}}%
\pgfpathlineto{\pgfqpoint{3.867734in}{3.036637in}}%
\pgfpathlineto{\pgfqpoint{3.867734in}{3.032380in}}%
\pgfpathmoveto{\pgfqpoint{3.859219in}{3.036637in}}%
\pgfpathlineto{\pgfqpoint{3.859219in}{3.036637in}}%
\pgfpathlineto{\pgfqpoint{3.859219in}{3.040895in}}%
\pgfpathlineto{\pgfqpoint{3.863477in}{3.040895in}}%
\pgfpathlineto{\pgfqpoint{3.863477in}{3.036637in}}%
\pgfpathmoveto{\pgfqpoint{3.859219in}{3.040895in}}%
\pgfpathlineto{\pgfqpoint{3.859219in}{3.040895in}}%
\pgfpathlineto{\pgfqpoint{3.859219in}{3.045153in}}%
\pgfpathlineto{\pgfqpoint{3.863477in}{3.045153in}}%
\pgfpathlineto{\pgfqpoint{3.863477in}{3.040895in}}%
\pgfpathmoveto{\pgfqpoint{3.863477in}{3.036637in}}%
\pgfpathlineto{\pgfqpoint{3.863477in}{3.036637in}}%
\pgfpathlineto{\pgfqpoint{3.863477in}{3.040895in}}%
\pgfpathlineto{\pgfqpoint{3.867734in}{3.040895in}}%
\pgfpathlineto{\pgfqpoint{3.867734in}{3.036637in}}%
\pgfpathmoveto{\pgfqpoint{3.863477in}{3.040895in}}%
\pgfpathlineto{\pgfqpoint{3.863477in}{3.040895in}}%
\pgfpathlineto{\pgfqpoint{3.863477in}{3.045153in}}%
\pgfpathlineto{\pgfqpoint{3.867734in}{3.045153in}}%
\pgfpathlineto{\pgfqpoint{3.867734in}{3.040895in}}%
\pgfpathmoveto{\pgfqpoint{3.867734in}{3.036637in}}%
\pgfpathlineto{\pgfqpoint{3.867734in}{3.036637in}}%
\pgfpathlineto{\pgfqpoint{3.867734in}{3.040895in}}%
\pgfpathlineto{\pgfqpoint{3.871992in}{3.040895in}}%
\pgfpathlineto{\pgfqpoint{3.871992in}{3.036637in}}%
\pgfpathmoveto{\pgfqpoint{3.867734in}{3.040895in}}%
\pgfpathlineto{\pgfqpoint{3.867734in}{3.040895in}}%
\pgfpathlineto{\pgfqpoint{3.867734in}{3.045153in}}%
\pgfpathlineto{\pgfqpoint{3.871992in}{3.045153in}}%
\pgfpathlineto{\pgfqpoint{3.871992in}{3.040895in}}%
\pgfpathmoveto{\pgfqpoint{3.863477in}{3.045153in}}%
\pgfpathlineto{\pgfqpoint{3.863477in}{3.045153in}}%
\pgfpathlineto{\pgfqpoint{3.863477in}{3.049411in}}%
\pgfpathlineto{\pgfqpoint{3.867734in}{3.049411in}}%
\pgfpathlineto{\pgfqpoint{3.867734in}{3.045153in}}%
\pgfpathmoveto{\pgfqpoint{3.863477in}{3.049411in}}%
\pgfpathlineto{\pgfqpoint{3.863477in}{3.049411in}}%
\pgfpathlineto{\pgfqpoint{3.863477in}{3.053669in}}%
\pgfpathlineto{\pgfqpoint{3.867734in}{3.053669in}}%
\pgfpathlineto{\pgfqpoint{3.867734in}{3.049411in}}%
\pgfpathmoveto{\pgfqpoint{3.863477in}{3.053669in}}%
\pgfpathlineto{\pgfqpoint{3.863477in}{3.053669in}}%
\pgfpathlineto{\pgfqpoint{3.863477in}{3.057927in}}%
\pgfpathlineto{\pgfqpoint{3.867734in}{3.057927in}}%
\pgfpathlineto{\pgfqpoint{3.867734in}{3.053669in}}%
\pgfpathmoveto{\pgfqpoint{3.863477in}{3.057927in}}%
\pgfpathlineto{\pgfqpoint{3.863477in}{3.057927in}}%
\pgfpathlineto{\pgfqpoint{3.863477in}{3.062184in}}%
\pgfpathlineto{\pgfqpoint{3.867734in}{3.062184in}}%
\pgfpathlineto{\pgfqpoint{3.867734in}{3.057927in}}%
\pgfpathmoveto{\pgfqpoint{3.867734in}{3.045153in}}%
\pgfpathlineto{\pgfqpoint{3.867734in}{3.045153in}}%
\pgfpathlineto{\pgfqpoint{3.867734in}{3.049411in}}%
\pgfpathlineto{\pgfqpoint{3.871992in}{3.049411in}}%
\pgfpathlineto{\pgfqpoint{3.871992in}{3.045153in}}%
\pgfpathmoveto{\pgfqpoint{3.867734in}{3.049411in}}%
\pgfpathlineto{\pgfqpoint{3.867734in}{3.049411in}}%
\pgfpathlineto{\pgfqpoint{3.867734in}{3.053669in}}%
\pgfpathlineto{\pgfqpoint{3.871992in}{3.053669in}}%
\pgfpathlineto{\pgfqpoint{3.871992in}{3.049411in}}%
\pgfpathmoveto{\pgfqpoint{3.867734in}{3.053669in}}%
\pgfpathlineto{\pgfqpoint{3.867734in}{3.053669in}}%
\pgfpathlineto{\pgfqpoint{3.867734in}{3.057927in}}%
\pgfpathlineto{\pgfqpoint{3.871992in}{3.057927in}}%
\pgfpathlineto{\pgfqpoint{3.871992in}{3.053669in}}%
\pgfpathmoveto{\pgfqpoint{3.867734in}{3.057927in}}%
\pgfpathlineto{\pgfqpoint{3.867734in}{3.057927in}}%
\pgfpathlineto{\pgfqpoint{3.867734in}{3.062184in}}%
\pgfpathlineto{\pgfqpoint{3.871992in}{3.062184in}}%
\pgfpathlineto{\pgfqpoint{3.871992in}{3.057927in}}%
\pgfpathmoveto{\pgfqpoint{3.863477in}{3.062184in}}%
\pgfpathlineto{\pgfqpoint{3.863477in}{3.062184in}}%
\pgfpathlineto{\pgfqpoint{3.863477in}{3.066442in}}%
\pgfpathlineto{\pgfqpoint{3.867734in}{3.066442in}}%
\pgfpathlineto{\pgfqpoint{3.867734in}{3.062184in}}%
\pgfpathmoveto{\pgfqpoint{3.863477in}{3.066442in}}%
\pgfpathlineto{\pgfqpoint{3.863477in}{3.066442in}}%
\pgfpathlineto{\pgfqpoint{3.863477in}{3.070700in}}%
\pgfpathlineto{\pgfqpoint{3.867734in}{3.070700in}}%
\pgfpathlineto{\pgfqpoint{3.867734in}{3.066442in}}%
\pgfpathmoveto{\pgfqpoint{3.863477in}{3.070700in}}%
\pgfpathlineto{\pgfqpoint{3.863477in}{3.070700in}}%
\pgfpathlineto{\pgfqpoint{3.863477in}{3.074958in}}%
\pgfpathlineto{\pgfqpoint{3.867734in}{3.074958in}}%
\pgfpathlineto{\pgfqpoint{3.867734in}{3.070700in}}%
\pgfpathmoveto{\pgfqpoint{3.867734in}{3.062184in}}%
\pgfpathlineto{\pgfqpoint{3.867734in}{3.062184in}}%
\pgfpathlineto{\pgfqpoint{3.867734in}{3.066442in}}%
\pgfpathlineto{\pgfqpoint{3.871992in}{3.066442in}}%
\pgfpathlineto{\pgfqpoint{3.871992in}{3.062184in}}%
\pgfpathmoveto{\pgfqpoint{3.867734in}{3.066442in}}%
\pgfpathlineto{\pgfqpoint{3.867734in}{3.066442in}}%
\pgfpathlineto{\pgfqpoint{3.867734in}{3.070700in}}%
\pgfpathlineto{\pgfqpoint{3.871992in}{3.070700in}}%
\pgfpathlineto{\pgfqpoint{3.871992in}{3.066442in}}%
\pgfpathmoveto{\pgfqpoint{3.867734in}{3.070700in}}%
\pgfpathlineto{\pgfqpoint{3.867734in}{3.070700in}}%
\pgfpathlineto{\pgfqpoint{3.867734in}{3.074958in}}%
\pgfpathlineto{\pgfqpoint{3.871992in}{3.074958in}}%
\pgfpathlineto{\pgfqpoint{3.871992in}{3.070700in}}%
\pgfpathmoveto{\pgfqpoint{3.867734in}{3.074958in}}%
\pgfpathlineto{\pgfqpoint{3.867734in}{3.074958in}}%
\pgfpathlineto{\pgfqpoint{3.867734in}{3.079216in}}%
\pgfpathlineto{\pgfqpoint{3.871992in}{3.079216in}}%
\pgfpathlineto{\pgfqpoint{3.871992in}{3.074958in}}%
\pgfpathmoveto{\pgfqpoint{3.871992in}{3.070700in}}%
\pgfpathlineto{\pgfqpoint{3.871992in}{3.070700in}}%
\pgfpathlineto{\pgfqpoint{3.871992in}{3.074958in}}%
\pgfpathlineto{\pgfqpoint{3.876250in}{3.074958in}}%
\pgfpathlineto{\pgfqpoint{3.876250in}{3.070700in}}%
\pgfpathmoveto{\pgfqpoint{3.871992in}{3.074958in}}%
\pgfpathlineto{\pgfqpoint{3.871992in}{3.074958in}}%
\pgfpathlineto{\pgfqpoint{3.871992in}{3.079216in}}%
\pgfpathlineto{\pgfqpoint{3.876250in}{3.079216in}}%
\pgfpathlineto{\pgfqpoint{3.876250in}{3.074958in}}%
\pgfpathmoveto{\pgfqpoint{3.867734in}{3.079216in}}%
\pgfpathlineto{\pgfqpoint{3.867734in}{3.079216in}}%
\pgfpathlineto{\pgfqpoint{3.867734in}{3.083474in}}%
\pgfpathlineto{\pgfqpoint{3.871992in}{3.083474in}}%
\pgfpathlineto{\pgfqpoint{3.871992in}{3.079216in}}%
\pgfpathmoveto{\pgfqpoint{3.867734in}{3.083474in}}%
\pgfpathlineto{\pgfqpoint{3.867734in}{3.083474in}}%
\pgfpathlineto{\pgfqpoint{3.867734in}{3.087731in}}%
\pgfpathlineto{\pgfqpoint{3.871992in}{3.087731in}}%
\pgfpathlineto{\pgfqpoint{3.871992in}{3.083474in}}%
\pgfpathmoveto{\pgfqpoint{3.871992in}{3.079216in}}%
\pgfpathlineto{\pgfqpoint{3.871992in}{3.079216in}}%
\pgfpathlineto{\pgfqpoint{3.871992in}{3.083474in}}%
\pgfpathlineto{\pgfqpoint{3.876250in}{3.083474in}}%
\pgfpathlineto{\pgfqpoint{3.876250in}{3.079216in}}%
\pgfpathmoveto{\pgfqpoint{3.871992in}{3.083474in}}%
\pgfpathlineto{\pgfqpoint{3.871992in}{3.083474in}}%
\pgfpathlineto{\pgfqpoint{3.871992in}{3.087731in}}%
\pgfpathlineto{\pgfqpoint{3.876250in}{3.087731in}}%
\pgfpathlineto{\pgfqpoint{3.876250in}{3.083474in}}%
\pgfpathmoveto{\pgfqpoint{3.867734in}{3.087731in}}%
\pgfpathlineto{\pgfqpoint{3.867734in}{3.087731in}}%
\pgfpathlineto{\pgfqpoint{3.867734in}{3.091989in}}%
\pgfpathlineto{\pgfqpoint{3.871992in}{3.091989in}}%
\pgfpathlineto{\pgfqpoint{3.871992in}{3.087731in}}%
\pgfpathmoveto{\pgfqpoint{3.867734in}{3.091989in}}%
\pgfpathlineto{\pgfqpoint{3.867734in}{3.091989in}}%
\pgfpathlineto{\pgfqpoint{3.867734in}{3.096247in}}%
\pgfpathlineto{\pgfqpoint{3.871992in}{3.096247in}}%
\pgfpathlineto{\pgfqpoint{3.871992in}{3.091989in}}%
\pgfpathmoveto{\pgfqpoint{3.871992in}{3.087731in}}%
\pgfpathlineto{\pgfqpoint{3.871992in}{3.087731in}}%
\pgfpathlineto{\pgfqpoint{3.871992in}{3.091989in}}%
\pgfpathlineto{\pgfqpoint{3.876250in}{3.091989in}}%
\pgfpathlineto{\pgfqpoint{3.876250in}{3.087731in}}%
\pgfpathmoveto{\pgfqpoint{3.871992in}{3.091989in}}%
\pgfpathlineto{\pgfqpoint{3.871992in}{3.091989in}}%
\pgfpathlineto{\pgfqpoint{3.871992in}{3.096247in}}%
\pgfpathlineto{\pgfqpoint{3.876250in}{3.096247in}}%
\pgfpathlineto{\pgfqpoint{3.876250in}{3.091989in}}%
\pgfpathmoveto{\pgfqpoint{3.867734in}{3.096247in}}%
\pgfpathlineto{\pgfqpoint{3.867734in}{3.096247in}}%
\pgfpathlineto{\pgfqpoint{3.867734in}{3.100505in}}%
\pgfpathlineto{\pgfqpoint{3.871992in}{3.100505in}}%
\pgfpathlineto{\pgfqpoint{3.871992in}{3.096247in}}%
\pgfpathmoveto{\pgfqpoint{3.867734in}{3.100505in}}%
\pgfpathlineto{\pgfqpoint{3.867734in}{3.100505in}}%
\pgfpathlineto{\pgfqpoint{3.867734in}{3.104763in}}%
\pgfpathlineto{\pgfqpoint{3.871992in}{3.104763in}}%
\pgfpathlineto{\pgfqpoint{3.871992in}{3.100505in}}%
\pgfpathmoveto{\pgfqpoint{3.871992in}{3.096247in}}%
\pgfpathlineto{\pgfqpoint{3.871992in}{3.096247in}}%
\pgfpathlineto{\pgfqpoint{3.871992in}{3.100505in}}%
\pgfpathlineto{\pgfqpoint{3.876250in}{3.100505in}}%
\pgfpathlineto{\pgfqpoint{3.876250in}{3.096247in}}%
\pgfpathmoveto{\pgfqpoint{3.871992in}{3.100505in}}%
\pgfpathlineto{\pgfqpoint{3.871992in}{3.100505in}}%
\pgfpathlineto{\pgfqpoint{3.871992in}{3.104763in}}%
\pgfpathlineto{\pgfqpoint{3.876250in}{3.104763in}}%
\pgfpathlineto{\pgfqpoint{3.876250in}{3.100505in}}%
\pgfpathmoveto{\pgfqpoint{3.867734in}{3.104763in}}%
\pgfpathlineto{\pgfqpoint{3.867734in}{3.104763in}}%
\pgfpathlineto{\pgfqpoint{3.867734in}{3.109021in}}%
\pgfpathlineto{\pgfqpoint{3.871992in}{3.109021in}}%
\pgfpathlineto{\pgfqpoint{3.871992in}{3.104763in}}%
\pgfpathmoveto{\pgfqpoint{3.871992in}{3.104763in}}%
\pgfpathlineto{\pgfqpoint{3.871992in}{3.104763in}}%
\pgfpathlineto{\pgfqpoint{3.871992in}{3.109021in}}%
\pgfpathlineto{\pgfqpoint{3.876250in}{3.109021in}}%
\pgfpathlineto{\pgfqpoint{3.876250in}{3.104763in}}%
\pgfpathmoveto{\pgfqpoint{3.871992in}{3.109021in}}%
\pgfpathlineto{\pgfqpoint{3.871992in}{3.109021in}}%
\pgfpathlineto{\pgfqpoint{3.871992in}{3.113278in}}%
\pgfpathlineto{\pgfqpoint{3.876250in}{3.113278in}}%
\pgfpathlineto{\pgfqpoint{3.876250in}{3.109021in}}%
\pgfpathmoveto{\pgfqpoint{3.871992in}{3.113278in}}%
\pgfpathlineto{\pgfqpoint{3.871992in}{3.113278in}}%
\pgfpathlineto{\pgfqpoint{3.871992in}{3.117536in}}%
\pgfpathlineto{\pgfqpoint{3.876250in}{3.117536in}}%
\pgfpathlineto{\pgfqpoint{3.876250in}{3.113278in}}%
\pgfpathmoveto{\pgfqpoint{3.871992in}{3.117536in}}%
\pgfpathlineto{\pgfqpoint{3.871992in}{3.117536in}}%
\pgfpathlineto{\pgfqpoint{3.871992in}{3.121794in}}%
\pgfpathlineto{\pgfqpoint{3.876250in}{3.121794in}}%
\pgfpathlineto{\pgfqpoint{3.876250in}{3.117536in}}%
\pgfpathmoveto{\pgfqpoint{3.871992in}{3.121794in}}%
\pgfpathlineto{\pgfqpoint{3.871992in}{3.121794in}}%
\pgfpathlineto{\pgfqpoint{3.871992in}{3.126052in}}%
\pgfpathlineto{\pgfqpoint{3.876250in}{3.126052in}}%
\pgfpathlineto{\pgfqpoint{3.876250in}{3.121794in}}%
\pgfpathmoveto{\pgfqpoint{3.871992in}{3.126052in}}%
\pgfpathlineto{\pgfqpoint{3.871992in}{3.126052in}}%
\pgfpathlineto{\pgfqpoint{3.871992in}{3.130310in}}%
\pgfpathlineto{\pgfqpoint{3.876250in}{3.130310in}}%
\pgfpathlineto{\pgfqpoint{3.876250in}{3.126052in}}%
\pgfpathmoveto{\pgfqpoint{3.871992in}{3.130310in}}%
\pgfpathlineto{\pgfqpoint{3.871992in}{3.130310in}}%
\pgfpathlineto{\pgfqpoint{3.871992in}{3.134567in}}%
\pgfpathlineto{\pgfqpoint{3.876250in}{3.134567in}}%
\pgfpathlineto{\pgfqpoint{3.876250in}{3.130310in}}%
\pgfpathmoveto{\pgfqpoint{3.871992in}{3.134567in}}%
\pgfpathlineto{\pgfqpoint{3.871992in}{3.134567in}}%
\pgfpathlineto{\pgfqpoint{3.871992in}{3.138825in}}%
\pgfpathlineto{\pgfqpoint{3.876250in}{3.138825in}}%
\pgfpathlineto{\pgfqpoint{3.876250in}{3.134567in}}%
\pgfpathmoveto{\pgfqpoint{3.871992in}{3.138825in}}%
\pgfpathlineto{\pgfqpoint{3.871992in}{3.138825in}}%
\pgfpathlineto{\pgfqpoint{3.871992in}{3.143083in}}%
\pgfpathlineto{\pgfqpoint{3.876250in}{3.143083in}}%
\pgfpathlineto{\pgfqpoint{3.876250in}{3.138825in}}%
\pgfpathmoveto{\pgfqpoint{3.876250in}{3.104763in}}%
\pgfpathlineto{\pgfqpoint{3.876250in}{3.104763in}}%
\pgfpathlineto{\pgfqpoint{3.876250in}{3.109021in}}%
\pgfpathlineto{\pgfqpoint{3.880508in}{3.109021in}}%
\pgfpathlineto{\pgfqpoint{3.880508in}{3.104763in}}%
\pgfpathmoveto{\pgfqpoint{3.876250in}{3.109021in}}%
\pgfpathlineto{\pgfqpoint{3.876250in}{3.109021in}}%
\pgfpathlineto{\pgfqpoint{3.876250in}{3.113278in}}%
\pgfpathlineto{\pgfqpoint{3.880508in}{3.113278in}}%
\pgfpathlineto{\pgfqpoint{3.880508in}{3.109021in}}%
\pgfpathmoveto{\pgfqpoint{3.876250in}{3.113278in}}%
\pgfpathlineto{\pgfqpoint{3.876250in}{3.113278in}}%
\pgfpathlineto{\pgfqpoint{3.876250in}{3.117536in}}%
\pgfpathlineto{\pgfqpoint{3.880508in}{3.117536in}}%
\pgfpathlineto{\pgfqpoint{3.880508in}{3.113278in}}%
\pgfpathmoveto{\pgfqpoint{3.876250in}{3.117536in}}%
\pgfpathlineto{\pgfqpoint{3.876250in}{3.117536in}}%
\pgfpathlineto{\pgfqpoint{3.876250in}{3.121794in}}%
\pgfpathlineto{\pgfqpoint{3.880508in}{3.121794in}}%
\pgfpathlineto{\pgfqpoint{3.880508in}{3.117536in}}%
\pgfpathmoveto{\pgfqpoint{3.876250in}{3.121794in}}%
\pgfpathlineto{\pgfqpoint{3.876250in}{3.121794in}}%
\pgfpathlineto{\pgfqpoint{3.876250in}{3.126052in}}%
\pgfpathlineto{\pgfqpoint{3.880508in}{3.126052in}}%
\pgfpathlineto{\pgfqpoint{3.880508in}{3.121794in}}%
\pgfpathmoveto{\pgfqpoint{3.876250in}{3.126052in}}%
\pgfpathlineto{\pgfqpoint{3.876250in}{3.126052in}}%
\pgfpathlineto{\pgfqpoint{3.876250in}{3.130310in}}%
\pgfpathlineto{\pgfqpoint{3.880508in}{3.130310in}}%
\pgfpathlineto{\pgfqpoint{3.880508in}{3.126052in}}%
\pgfpathmoveto{\pgfqpoint{3.876250in}{3.130310in}}%
\pgfpathlineto{\pgfqpoint{3.876250in}{3.130310in}}%
\pgfpathlineto{\pgfqpoint{3.876250in}{3.134567in}}%
\pgfpathlineto{\pgfqpoint{3.880508in}{3.134567in}}%
\pgfpathlineto{\pgfqpoint{3.880508in}{3.130310in}}%
\pgfpathmoveto{\pgfqpoint{3.876250in}{3.134567in}}%
\pgfpathlineto{\pgfqpoint{3.876250in}{3.134567in}}%
\pgfpathlineto{\pgfqpoint{3.876250in}{3.138825in}}%
\pgfpathlineto{\pgfqpoint{3.880508in}{3.138825in}}%
\pgfpathlineto{\pgfqpoint{3.880508in}{3.134567in}}%
\pgfpathmoveto{\pgfqpoint{3.876250in}{3.138825in}}%
\pgfpathlineto{\pgfqpoint{3.876250in}{3.138825in}}%
\pgfpathlineto{\pgfqpoint{3.876250in}{3.143083in}}%
\pgfpathlineto{\pgfqpoint{3.880508in}{3.143083in}}%
\pgfpathlineto{\pgfqpoint{3.880508in}{3.138825in}}%
\pgfpathmoveto{\pgfqpoint{3.876250in}{3.143083in}}%
\pgfpathlineto{\pgfqpoint{3.876250in}{3.143083in}}%
\pgfpathlineto{\pgfqpoint{3.876250in}{3.147341in}}%
\pgfpathlineto{\pgfqpoint{3.880508in}{3.147341in}}%
\pgfpathlineto{\pgfqpoint{3.880508in}{3.143083in}}%
\pgfpathmoveto{\pgfqpoint{3.880508in}{3.138825in}}%
\pgfpathlineto{\pgfqpoint{3.880508in}{3.138825in}}%
\pgfpathlineto{\pgfqpoint{3.880508in}{3.143083in}}%
\pgfpathlineto{\pgfqpoint{3.884766in}{3.143083in}}%
\pgfpathlineto{\pgfqpoint{3.884766in}{3.138825in}}%
\pgfpathmoveto{\pgfqpoint{3.880508in}{3.143083in}}%
\pgfpathlineto{\pgfqpoint{3.880508in}{3.143083in}}%
\pgfpathlineto{\pgfqpoint{3.880508in}{3.147341in}}%
\pgfpathlineto{\pgfqpoint{3.884766in}{3.147341in}}%
\pgfpathlineto{\pgfqpoint{3.884766in}{3.143083in}}%
\pgfpathmoveto{\pgfqpoint{3.876250in}{3.147341in}}%
\pgfpathlineto{\pgfqpoint{3.876250in}{3.147341in}}%
\pgfpathlineto{\pgfqpoint{3.876250in}{3.151599in}}%
\pgfpathlineto{\pgfqpoint{3.880508in}{3.151599in}}%
\pgfpathlineto{\pgfqpoint{3.880508in}{3.147341in}}%
\pgfpathmoveto{\pgfqpoint{3.876250in}{3.151599in}}%
\pgfpathlineto{\pgfqpoint{3.876250in}{3.151599in}}%
\pgfpathlineto{\pgfqpoint{3.876250in}{3.155856in}}%
\pgfpathlineto{\pgfqpoint{3.880508in}{3.155856in}}%
\pgfpathlineto{\pgfqpoint{3.880508in}{3.151599in}}%
\pgfpathmoveto{\pgfqpoint{3.880508in}{3.147341in}}%
\pgfpathlineto{\pgfqpoint{3.880508in}{3.147341in}}%
\pgfpathlineto{\pgfqpoint{3.880508in}{3.151599in}}%
\pgfpathlineto{\pgfqpoint{3.884766in}{3.151599in}}%
\pgfpathlineto{\pgfqpoint{3.884766in}{3.147341in}}%
\pgfpathmoveto{\pgfqpoint{3.880508in}{3.151599in}}%
\pgfpathlineto{\pgfqpoint{3.880508in}{3.151599in}}%
\pgfpathlineto{\pgfqpoint{3.880508in}{3.155856in}}%
\pgfpathlineto{\pgfqpoint{3.884766in}{3.155856in}}%
\pgfpathlineto{\pgfqpoint{3.884766in}{3.151599in}}%
\pgfpathmoveto{\pgfqpoint{3.876250in}{3.155856in}}%
\pgfpathlineto{\pgfqpoint{3.876250in}{3.155856in}}%
\pgfpathlineto{\pgfqpoint{3.876250in}{3.160114in}}%
\pgfpathlineto{\pgfqpoint{3.880508in}{3.160114in}}%
\pgfpathlineto{\pgfqpoint{3.880508in}{3.155856in}}%
\pgfpathmoveto{\pgfqpoint{3.876250in}{3.160114in}}%
\pgfpathlineto{\pgfqpoint{3.876250in}{3.160114in}}%
\pgfpathlineto{\pgfqpoint{3.876250in}{3.164372in}}%
\pgfpathlineto{\pgfqpoint{3.880508in}{3.164372in}}%
\pgfpathlineto{\pgfqpoint{3.880508in}{3.160114in}}%
\pgfpathmoveto{\pgfqpoint{3.880508in}{3.155856in}}%
\pgfpathlineto{\pgfqpoint{3.880508in}{3.155856in}}%
\pgfpathlineto{\pgfqpoint{3.880508in}{3.160114in}}%
\pgfpathlineto{\pgfqpoint{3.884766in}{3.160114in}}%
\pgfpathlineto{\pgfqpoint{3.884766in}{3.155856in}}%
\pgfpathmoveto{\pgfqpoint{3.880508in}{3.160114in}}%
\pgfpathlineto{\pgfqpoint{3.880508in}{3.160114in}}%
\pgfpathlineto{\pgfqpoint{3.880508in}{3.164372in}}%
\pgfpathlineto{\pgfqpoint{3.884766in}{3.164372in}}%
\pgfpathlineto{\pgfqpoint{3.884766in}{3.160114in}}%
\pgfpathmoveto{\pgfqpoint{3.876250in}{3.164372in}}%
\pgfpathlineto{\pgfqpoint{3.876250in}{3.164372in}}%
\pgfpathlineto{\pgfqpoint{3.876250in}{3.168630in}}%
\pgfpathlineto{\pgfqpoint{3.880508in}{3.168630in}}%
\pgfpathlineto{\pgfqpoint{3.880508in}{3.164372in}}%
\pgfpathmoveto{\pgfqpoint{3.876250in}{3.168630in}}%
\pgfpathlineto{\pgfqpoint{3.876250in}{3.168630in}}%
\pgfpathlineto{\pgfqpoint{3.876250in}{3.172888in}}%
\pgfpathlineto{\pgfqpoint{3.880508in}{3.172888in}}%
\pgfpathlineto{\pgfqpoint{3.880508in}{3.168630in}}%
\pgfpathmoveto{\pgfqpoint{3.880508in}{3.164372in}}%
\pgfpathlineto{\pgfqpoint{3.880508in}{3.164372in}}%
\pgfpathlineto{\pgfqpoint{3.880508in}{3.168630in}}%
\pgfpathlineto{\pgfqpoint{3.884766in}{3.168630in}}%
\pgfpathlineto{\pgfqpoint{3.884766in}{3.164372in}}%
\pgfpathmoveto{\pgfqpoint{3.880508in}{3.168630in}}%
\pgfpathlineto{\pgfqpoint{3.880508in}{3.168630in}}%
\pgfpathlineto{\pgfqpoint{3.880508in}{3.172888in}}%
\pgfpathlineto{\pgfqpoint{3.884766in}{3.172888in}}%
\pgfpathlineto{\pgfqpoint{3.884766in}{3.168630in}}%
\pgfpathmoveto{\pgfqpoint{3.876250in}{3.172888in}}%
\pgfpathlineto{\pgfqpoint{3.876250in}{3.172888in}}%
\pgfpathlineto{\pgfqpoint{3.876250in}{3.177146in}}%
\pgfpathlineto{\pgfqpoint{3.880508in}{3.177146in}}%
\pgfpathlineto{\pgfqpoint{3.880508in}{3.172888in}}%
\pgfpathmoveto{\pgfqpoint{3.880508in}{3.172888in}}%
\pgfpathlineto{\pgfqpoint{3.880508in}{3.172888in}}%
\pgfpathlineto{\pgfqpoint{3.880508in}{3.177146in}}%
\pgfpathlineto{\pgfqpoint{3.884766in}{3.177146in}}%
\pgfpathlineto{\pgfqpoint{3.884766in}{3.172888in}}%
\pgfpathmoveto{\pgfqpoint{3.880508in}{3.177146in}}%
\pgfpathlineto{\pgfqpoint{3.880508in}{3.177146in}}%
\pgfpathlineto{\pgfqpoint{3.880508in}{3.181403in}}%
\pgfpathlineto{\pgfqpoint{3.884766in}{3.181403in}}%
\pgfpathlineto{\pgfqpoint{3.884766in}{3.177146in}}%
\pgfpathmoveto{\pgfqpoint{3.884766in}{3.172888in}}%
\pgfpathlineto{\pgfqpoint{3.884766in}{3.172888in}}%
\pgfpathlineto{\pgfqpoint{3.884766in}{3.177146in}}%
\pgfpathlineto{\pgfqpoint{3.889023in}{3.177146in}}%
\pgfpathlineto{\pgfqpoint{3.889023in}{3.172888in}}%
\pgfpathmoveto{\pgfqpoint{3.884766in}{3.177146in}}%
\pgfpathlineto{\pgfqpoint{3.884766in}{3.177146in}}%
\pgfpathlineto{\pgfqpoint{3.884766in}{3.181403in}}%
\pgfpathlineto{\pgfqpoint{3.889023in}{3.181403in}}%
\pgfpathlineto{\pgfqpoint{3.889023in}{3.177146in}}%
\pgfpathmoveto{\pgfqpoint{3.880508in}{3.181403in}}%
\pgfpathlineto{\pgfqpoint{3.880508in}{3.181403in}}%
\pgfpathlineto{\pgfqpoint{3.880508in}{3.185661in}}%
\pgfpathlineto{\pgfqpoint{3.884766in}{3.185661in}}%
\pgfpathlineto{\pgfqpoint{3.884766in}{3.181403in}}%
\pgfpathmoveto{\pgfqpoint{3.880508in}{3.185661in}}%
\pgfpathlineto{\pgfqpoint{3.880508in}{3.185661in}}%
\pgfpathlineto{\pgfqpoint{3.880508in}{3.189919in}}%
\pgfpathlineto{\pgfqpoint{3.884766in}{3.189919in}}%
\pgfpathlineto{\pgfqpoint{3.884766in}{3.185661in}}%
\pgfpathmoveto{\pgfqpoint{3.880508in}{3.189919in}}%
\pgfpathlineto{\pgfqpoint{3.880508in}{3.189919in}}%
\pgfpathlineto{\pgfqpoint{3.880508in}{3.194177in}}%
\pgfpathlineto{\pgfqpoint{3.884766in}{3.194177in}}%
\pgfpathlineto{\pgfqpoint{3.884766in}{3.189919in}}%
\pgfpathmoveto{\pgfqpoint{3.880508in}{3.194177in}}%
\pgfpathlineto{\pgfqpoint{3.880508in}{3.194177in}}%
\pgfpathlineto{\pgfqpoint{3.880508in}{3.198435in}}%
\pgfpathlineto{\pgfqpoint{3.884766in}{3.198435in}}%
\pgfpathlineto{\pgfqpoint{3.884766in}{3.194177in}}%
\pgfpathmoveto{\pgfqpoint{3.884766in}{3.181403in}}%
\pgfpathlineto{\pgfqpoint{3.884766in}{3.181403in}}%
\pgfpathlineto{\pgfqpoint{3.884766in}{3.185661in}}%
\pgfpathlineto{\pgfqpoint{3.889023in}{3.185661in}}%
\pgfpathlineto{\pgfqpoint{3.889023in}{3.181403in}}%
\pgfpathmoveto{\pgfqpoint{3.884766in}{3.185661in}}%
\pgfpathlineto{\pgfqpoint{3.884766in}{3.185661in}}%
\pgfpathlineto{\pgfqpoint{3.884766in}{3.189919in}}%
\pgfpathlineto{\pgfqpoint{3.889023in}{3.189919in}}%
\pgfpathlineto{\pgfqpoint{3.889023in}{3.185661in}}%
\pgfpathmoveto{\pgfqpoint{3.884766in}{3.189919in}}%
\pgfpathlineto{\pgfqpoint{3.884766in}{3.189919in}}%
\pgfpathlineto{\pgfqpoint{3.884766in}{3.194177in}}%
\pgfpathlineto{\pgfqpoint{3.889023in}{3.194177in}}%
\pgfpathlineto{\pgfqpoint{3.889023in}{3.189919in}}%
\pgfpathmoveto{\pgfqpoint{3.884766in}{3.194177in}}%
\pgfpathlineto{\pgfqpoint{3.884766in}{3.194177in}}%
\pgfpathlineto{\pgfqpoint{3.884766in}{3.198435in}}%
\pgfpathlineto{\pgfqpoint{3.889023in}{3.198435in}}%
\pgfpathlineto{\pgfqpoint{3.889023in}{3.194177in}}%
\pgfpathmoveto{\pgfqpoint{3.880508in}{3.198435in}}%
\pgfpathlineto{\pgfqpoint{3.880508in}{3.198435in}}%
\pgfpathlineto{\pgfqpoint{3.880508in}{3.202692in}}%
\pgfpathlineto{\pgfqpoint{3.884766in}{3.202692in}}%
\pgfpathlineto{\pgfqpoint{3.884766in}{3.198435in}}%
\pgfpathmoveto{\pgfqpoint{3.880508in}{3.202692in}}%
\pgfpathlineto{\pgfqpoint{3.880508in}{3.202692in}}%
\pgfpathlineto{\pgfqpoint{3.880508in}{3.206950in}}%
\pgfpathlineto{\pgfqpoint{3.884766in}{3.206950in}}%
\pgfpathlineto{\pgfqpoint{3.884766in}{3.202692in}}%
\pgfpathmoveto{\pgfqpoint{3.880508in}{3.206950in}}%
\pgfpathlineto{\pgfqpoint{3.880508in}{3.206950in}}%
\pgfpathlineto{\pgfqpoint{3.880508in}{3.211208in}}%
\pgfpathlineto{\pgfqpoint{3.884766in}{3.211208in}}%
\pgfpathlineto{\pgfqpoint{3.884766in}{3.206950in}}%
\pgfpathmoveto{\pgfqpoint{3.884766in}{3.198435in}}%
\pgfpathlineto{\pgfqpoint{3.884766in}{3.198435in}}%
\pgfpathlineto{\pgfqpoint{3.884766in}{3.202692in}}%
\pgfpathlineto{\pgfqpoint{3.889023in}{3.202692in}}%
\pgfpathlineto{\pgfqpoint{3.889023in}{3.198435in}}%
\pgfpathmoveto{\pgfqpoint{3.884766in}{3.202692in}}%
\pgfpathlineto{\pgfqpoint{3.884766in}{3.202692in}}%
\pgfpathlineto{\pgfqpoint{3.884766in}{3.206950in}}%
\pgfpathlineto{\pgfqpoint{3.889023in}{3.206950in}}%
\pgfpathlineto{\pgfqpoint{3.889023in}{3.202692in}}%
\pgfpathmoveto{\pgfqpoint{3.884766in}{3.206950in}}%
\pgfpathlineto{\pgfqpoint{3.884766in}{3.206950in}}%
\pgfpathlineto{\pgfqpoint{3.884766in}{3.211208in}}%
\pgfpathlineto{\pgfqpoint{3.889023in}{3.211208in}}%
\pgfpathlineto{\pgfqpoint{3.889023in}{3.206950in}}%
\pgfpathmoveto{\pgfqpoint{3.884766in}{3.211208in}}%
\pgfpathlineto{\pgfqpoint{3.884766in}{3.211208in}}%
\pgfpathlineto{\pgfqpoint{3.884766in}{3.215466in}}%
\pgfpathlineto{\pgfqpoint{3.889023in}{3.215466in}}%
\pgfpathlineto{\pgfqpoint{3.889023in}{3.211208in}}%
\pgfpathmoveto{\pgfqpoint{3.889023in}{3.206950in}}%
\pgfpathlineto{\pgfqpoint{3.889023in}{3.206950in}}%
\pgfpathlineto{\pgfqpoint{3.889023in}{3.211208in}}%
\pgfpathlineto{\pgfqpoint{3.893281in}{3.211208in}}%
\pgfpathlineto{\pgfqpoint{3.893281in}{3.206950in}}%
\pgfpathmoveto{\pgfqpoint{3.889023in}{3.211208in}}%
\pgfpathlineto{\pgfqpoint{3.889023in}{3.211208in}}%
\pgfpathlineto{\pgfqpoint{3.889023in}{3.215466in}}%
\pgfpathlineto{\pgfqpoint{3.893281in}{3.215466in}}%
\pgfpathlineto{\pgfqpoint{3.893281in}{3.211208in}}%
\pgfpathmoveto{\pgfqpoint{3.884766in}{3.215466in}}%
\pgfpathlineto{\pgfqpoint{3.884766in}{3.215466in}}%
\pgfpathlineto{\pgfqpoint{3.884766in}{3.219724in}}%
\pgfpathlineto{\pgfqpoint{3.889023in}{3.219724in}}%
\pgfpathlineto{\pgfqpoint{3.889023in}{3.215466in}}%
\pgfpathmoveto{\pgfqpoint{3.884766in}{3.219724in}}%
\pgfpathlineto{\pgfqpoint{3.884766in}{3.219724in}}%
\pgfpathlineto{\pgfqpoint{3.884766in}{3.223981in}}%
\pgfpathlineto{\pgfqpoint{3.889023in}{3.223981in}}%
\pgfpathlineto{\pgfqpoint{3.889023in}{3.219724in}}%
\pgfpathmoveto{\pgfqpoint{3.889023in}{3.215466in}}%
\pgfpathlineto{\pgfqpoint{3.889023in}{3.215466in}}%
\pgfpathlineto{\pgfqpoint{3.889023in}{3.219724in}}%
\pgfpathlineto{\pgfqpoint{3.893281in}{3.219724in}}%
\pgfpathlineto{\pgfqpoint{3.893281in}{3.215466in}}%
\pgfpathmoveto{\pgfqpoint{3.889023in}{3.219724in}}%
\pgfpathlineto{\pgfqpoint{3.889023in}{3.219724in}}%
\pgfpathlineto{\pgfqpoint{3.889023in}{3.223981in}}%
\pgfpathlineto{\pgfqpoint{3.893281in}{3.223981in}}%
\pgfpathlineto{\pgfqpoint{3.893281in}{3.219724in}}%
\pgfpathmoveto{\pgfqpoint{3.884766in}{3.223981in}}%
\pgfpathlineto{\pgfqpoint{3.884766in}{3.223981in}}%
\pgfpathlineto{\pgfqpoint{3.884766in}{3.228239in}}%
\pgfpathlineto{\pgfqpoint{3.889023in}{3.228239in}}%
\pgfpathlineto{\pgfqpoint{3.889023in}{3.223981in}}%
\pgfpathmoveto{\pgfqpoint{3.884766in}{3.228239in}}%
\pgfpathlineto{\pgfqpoint{3.884766in}{3.228239in}}%
\pgfpathlineto{\pgfqpoint{3.884766in}{3.232497in}}%
\pgfpathlineto{\pgfqpoint{3.889023in}{3.232497in}}%
\pgfpathlineto{\pgfqpoint{3.889023in}{3.228239in}}%
\pgfpathmoveto{\pgfqpoint{3.889023in}{3.223981in}}%
\pgfpathlineto{\pgfqpoint{3.889023in}{3.223981in}}%
\pgfpathlineto{\pgfqpoint{3.889023in}{3.228239in}}%
\pgfpathlineto{\pgfqpoint{3.893281in}{3.228239in}}%
\pgfpathlineto{\pgfqpoint{3.893281in}{3.223981in}}%
\pgfpathmoveto{\pgfqpoint{3.889023in}{3.228239in}}%
\pgfpathlineto{\pgfqpoint{3.889023in}{3.228239in}}%
\pgfpathlineto{\pgfqpoint{3.889023in}{3.232497in}}%
\pgfpathlineto{\pgfqpoint{3.893281in}{3.232497in}}%
\pgfpathlineto{\pgfqpoint{3.893281in}{3.228239in}}%
\pgfpathmoveto{\pgfqpoint{3.884766in}{3.232497in}}%
\pgfpathlineto{\pgfqpoint{3.884766in}{3.232497in}}%
\pgfpathlineto{\pgfqpoint{3.884766in}{3.236755in}}%
\pgfpathlineto{\pgfqpoint{3.889023in}{3.236755in}}%
\pgfpathlineto{\pgfqpoint{3.889023in}{3.232497in}}%
\pgfpathmoveto{\pgfqpoint{3.884766in}{3.236755in}}%
\pgfpathlineto{\pgfqpoint{3.884766in}{3.236755in}}%
\pgfpathlineto{\pgfqpoint{3.884766in}{3.241013in}}%
\pgfpathlineto{\pgfqpoint{3.889023in}{3.241013in}}%
\pgfpathlineto{\pgfqpoint{3.889023in}{3.236755in}}%
\pgfpathmoveto{\pgfqpoint{3.889023in}{3.232497in}}%
\pgfpathlineto{\pgfqpoint{3.889023in}{3.232497in}}%
\pgfpathlineto{\pgfqpoint{3.889023in}{3.236755in}}%
\pgfpathlineto{\pgfqpoint{3.893281in}{3.236755in}}%
\pgfpathlineto{\pgfqpoint{3.893281in}{3.232497in}}%
\pgfpathmoveto{\pgfqpoint{3.889023in}{3.236755in}}%
\pgfpathlineto{\pgfqpoint{3.889023in}{3.236755in}}%
\pgfpathlineto{\pgfqpoint{3.889023in}{3.241013in}}%
\pgfpathlineto{\pgfqpoint{3.893281in}{3.241013in}}%
\pgfpathlineto{\pgfqpoint{3.893281in}{3.236755in}}%
\pgfpathmoveto{\pgfqpoint{3.884766in}{3.241013in}}%
\pgfpathlineto{\pgfqpoint{3.884766in}{3.241013in}}%
\pgfpathlineto{\pgfqpoint{3.884766in}{3.245270in}}%
\pgfpathlineto{\pgfqpoint{3.889023in}{3.245270in}}%
\pgfpathlineto{\pgfqpoint{3.889023in}{3.241013in}}%
\pgfpathmoveto{\pgfqpoint{3.889023in}{3.241013in}}%
\pgfpathlineto{\pgfqpoint{3.889023in}{3.241013in}}%
\pgfpathlineto{\pgfqpoint{3.889023in}{3.245270in}}%
\pgfpathlineto{\pgfqpoint{3.893281in}{3.245270in}}%
\pgfpathlineto{\pgfqpoint{3.893281in}{3.241013in}}%
\pgfpathmoveto{\pgfqpoint{3.889023in}{3.245270in}}%
\pgfpathlineto{\pgfqpoint{3.889023in}{3.245270in}}%
\pgfpathlineto{\pgfqpoint{3.889023in}{3.249528in}}%
\pgfpathlineto{\pgfqpoint{3.893281in}{3.249528in}}%
\pgfpathlineto{\pgfqpoint{3.893281in}{3.245270in}}%
\pgfpathmoveto{\pgfqpoint{3.889023in}{3.249528in}}%
\pgfpathlineto{\pgfqpoint{3.889023in}{3.249528in}}%
\pgfpathlineto{\pgfqpoint{3.889023in}{3.253786in}}%
\pgfpathlineto{\pgfqpoint{3.893281in}{3.253786in}}%
\pgfpathlineto{\pgfqpoint{3.893281in}{3.249528in}}%
\pgfpathmoveto{\pgfqpoint{3.889023in}{3.253786in}}%
\pgfpathlineto{\pgfqpoint{3.889023in}{3.253786in}}%
\pgfpathlineto{\pgfqpoint{3.889023in}{3.258044in}}%
\pgfpathlineto{\pgfqpoint{3.893281in}{3.258044in}}%
\pgfpathlineto{\pgfqpoint{3.893281in}{3.253786in}}%
\pgfpathmoveto{\pgfqpoint{3.889023in}{3.258044in}}%
\pgfpathlineto{\pgfqpoint{3.889023in}{3.258044in}}%
\pgfpathlineto{\pgfqpoint{3.889023in}{3.262302in}}%
\pgfpathlineto{\pgfqpoint{3.893281in}{3.262302in}}%
\pgfpathlineto{\pgfqpoint{3.893281in}{3.258044in}}%
\pgfpathmoveto{\pgfqpoint{3.889023in}{3.262302in}}%
\pgfpathlineto{\pgfqpoint{3.889023in}{3.262302in}}%
\pgfpathlineto{\pgfqpoint{3.889023in}{3.266559in}}%
\pgfpathlineto{\pgfqpoint{3.893281in}{3.266559in}}%
\pgfpathlineto{\pgfqpoint{3.893281in}{3.262302in}}%
\pgfpathmoveto{\pgfqpoint{3.893281in}{3.241013in}}%
\pgfpathlineto{\pgfqpoint{3.893281in}{3.241013in}}%
\pgfpathlineto{\pgfqpoint{3.893281in}{3.245270in}}%
\pgfpathlineto{\pgfqpoint{3.897539in}{3.245270in}}%
\pgfpathlineto{\pgfqpoint{3.897539in}{3.241013in}}%
\pgfpathmoveto{\pgfqpoint{3.893281in}{3.245270in}}%
\pgfpathlineto{\pgfqpoint{3.893281in}{3.245270in}}%
\pgfpathlineto{\pgfqpoint{3.893281in}{3.249528in}}%
\pgfpathlineto{\pgfqpoint{3.897539in}{3.249528in}}%
\pgfpathlineto{\pgfqpoint{3.897539in}{3.245270in}}%
\pgfpathmoveto{\pgfqpoint{3.893281in}{3.249528in}}%
\pgfpathlineto{\pgfqpoint{3.893281in}{3.249528in}}%
\pgfpathlineto{\pgfqpoint{3.893281in}{3.253786in}}%
\pgfpathlineto{\pgfqpoint{3.897539in}{3.253786in}}%
\pgfpathlineto{\pgfqpoint{3.897539in}{3.249528in}}%
\pgfpathmoveto{\pgfqpoint{3.893281in}{3.253786in}}%
\pgfpathlineto{\pgfqpoint{3.893281in}{3.253786in}}%
\pgfpathlineto{\pgfqpoint{3.893281in}{3.258044in}}%
\pgfpathlineto{\pgfqpoint{3.897539in}{3.258044in}}%
\pgfpathlineto{\pgfqpoint{3.897539in}{3.253786in}}%
\pgfpathmoveto{\pgfqpoint{3.893281in}{3.258044in}}%
\pgfpathlineto{\pgfqpoint{3.893281in}{3.258044in}}%
\pgfpathlineto{\pgfqpoint{3.893281in}{3.262302in}}%
\pgfpathlineto{\pgfqpoint{3.897539in}{3.262302in}}%
\pgfpathlineto{\pgfqpoint{3.897539in}{3.258044in}}%
\pgfpathmoveto{\pgfqpoint{3.893281in}{3.262302in}}%
\pgfpathlineto{\pgfqpoint{3.893281in}{3.262302in}}%
\pgfpathlineto{\pgfqpoint{3.893281in}{3.266559in}}%
\pgfpathlineto{\pgfqpoint{3.897539in}{3.266559in}}%
\pgfpathlineto{\pgfqpoint{3.897539in}{3.262302in}}%
\pgfpathmoveto{\pgfqpoint{3.889023in}{3.266559in}}%
\pgfpathlineto{\pgfqpoint{3.889023in}{3.266559in}}%
\pgfpathlineto{\pgfqpoint{3.889023in}{3.270817in}}%
\pgfpathlineto{\pgfqpoint{3.893281in}{3.270817in}}%
\pgfpathlineto{\pgfqpoint{3.893281in}{3.266559in}}%
\pgfpathmoveto{\pgfqpoint{3.889023in}{3.270817in}}%
\pgfpathlineto{\pgfqpoint{3.889023in}{3.270817in}}%
\pgfpathlineto{\pgfqpoint{3.889023in}{3.275075in}}%
\pgfpathlineto{\pgfqpoint{3.893281in}{3.275075in}}%
\pgfpathlineto{\pgfqpoint{3.893281in}{3.270817in}}%
\pgfpathmoveto{\pgfqpoint{3.889023in}{3.275075in}}%
\pgfpathlineto{\pgfqpoint{3.889023in}{3.275075in}}%
\pgfpathlineto{\pgfqpoint{3.889023in}{3.279333in}}%
\pgfpathlineto{\pgfqpoint{3.893281in}{3.279333in}}%
\pgfpathlineto{\pgfqpoint{3.893281in}{3.275075in}}%
\pgfpathmoveto{\pgfqpoint{3.893281in}{3.266559in}}%
\pgfpathlineto{\pgfqpoint{3.893281in}{3.266559in}}%
\pgfpathlineto{\pgfqpoint{3.893281in}{3.270817in}}%
\pgfpathlineto{\pgfqpoint{3.897539in}{3.270817in}}%
\pgfpathlineto{\pgfqpoint{3.897539in}{3.266559in}}%
\pgfpathmoveto{\pgfqpoint{3.893281in}{3.270817in}}%
\pgfpathlineto{\pgfqpoint{3.893281in}{3.270817in}}%
\pgfpathlineto{\pgfqpoint{3.893281in}{3.275075in}}%
\pgfpathlineto{\pgfqpoint{3.897539in}{3.275075in}}%
\pgfpathlineto{\pgfqpoint{3.897539in}{3.270817in}}%
\pgfpathmoveto{\pgfqpoint{3.893281in}{3.275075in}}%
\pgfpathlineto{\pgfqpoint{3.893281in}{3.275075in}}%
\pgfpathlineto{\pgfqpoint{3.893281in}{3.279333in}}%
\pgfpathlineto{\pgfqpoint{3.897539in}{3.279333in}}%
\pgfpathlineto{\pgfqpoint{3.897539in}{3.275075in}}%
\pgfpathmoveto{\pgfqpoint{3.893281in}{3.279333in}}%
\pgfpathlineto{\pgfqpoint{3.893281in}{3.279333in}}%
\pgfpathlineto{\pgfqpoint{3.893281in}{3.283591in}}%
\pgfpathlineto{\pgfqpoint{3.897539in}{3.283591in}}%
\pgfpathlineto{\pgfqpoint{3.897539in}{3.279333in}}%
\pgfpathmoveto{\pgfqpoint{3.897539in}{3.275075in}}%
\pgfpathlineto{\pgfqpoint{3.897539in}{3.275075in}}%
\pgfpathlineto{\pgfqpoint{3.897539in}{3.279333in}}%
\pgfpathlineto{\pgfqpoint{3.901797in}{3.279333in}}%
\pgfpathlineto{\pgfqpoint{3.901797in}{3.275075in}}%
\pgfpathmoveto{\pgfqpoint{3.897539in}{3.279333in}}%
\pgfpathlineto{\pgfqpoint{3.897539in}{3.279333in}}%
\pgfpathlineto{\pgfqpoint{3.897539in}{3.283591in}}%
\pgfpathlineto{\pgfqpoint{3.901797in}{3.283591in}}%
\pgfpathlineto{\pgfqpoint{3.901797in}{3.279333in}}%
\pgfpathmoveto{\pgfqpoint{3.893281in}{3.283591in}}%
\pgfpathlineto{\pgfqpoint{3.893281in}{3.283591in}}%
\pgfpathlineto{\pgfqpoint{3.893281in}{3.287848in}}%
\pgfpathlineto{\pgfqpoint{3.897539in}{3.287848in}}%
\pgfpathlineto{\pgfqpoint{3.897539in}{3.283591in}}%
\pgfpathmoveto{\pgfqpoint{3.893281in}{3.287848in}}%
\pgfpathlineto{\pgfqpoint{3.893281in}{3.287848in}}%
\pgfpathlineto{\pgfqpoint{3.893281in}{3.292106in}}%
\pgfpathlineto{\pgfqpoint{3.897539in}{3.292106in}}%
\pgfpathlineto{\pgfqpoint{3.897539in}{3.287848in}}%
\pgfpathmoveto{\pgfqpoint{3.897539in}{3.283591in}}%
\pgfpathlineto{\pgfqpoint{3.897539in}{3.283591in}}%
\pgfpathlineto{\pgfqpoint{3.897539in}{3.287848in}}%
\pgfpathlineto{\pgfqpoint{3.901797in}{3.287848in}}%
\pgfpathlineto{\pgfqpoint{3.901797in}{3.283591in}}%
\pgfpathmoveto{\pgfqpoint{3.897539in}{3.287848in}}%
\pgfpathlineto{\pgfqpoint{3.897539in}{3.287848in}}%
\pgfpathlineto{\pgfqpoint{3.897539in}{3.292106in}}%
\pgfpathlineto{\pgfqpoint{3.901797in}{3.292106in}}%
\pgfpathlineto{\pgfqpoint{3.901797in}{3.287848in}}%
\pgfpathmoveto{\pgfqpoint{3.893281in}{3.292106in}}%
\pgfpathlineto{\pgfqpoint{3.893281in}{3.292106in}}%
\pgfpathlineto{\pgfqpoint{3.893281in}{3.296364in}}%
\pgfpathlineto{\pgfqpoint{3.897539in}{3.296364in}}%
\pgfpathlineto{\pgfqpoint{3.897539in}{3.292106in}}%
\pgfpathmoveto{\pgfqpoint{3.893281in}{3.296364in}}%
\pgfpathlineto{\pgfqpoint{3.893281in}{3.296364in}}%
\pgfpathlineto{\pgfqpoint{3.893281in}{3.300622in}}%
\pgfpathlineto{\pgfqpoint{3.897539in}{3.300622in}}%
\pgfpathlineto{\pgfqpoint{3.897539in}{3.296364in}}%
\pgfpathmoveto{\pgfqpoint{3.897539in}{3.292106in}}%
\pgfpathlineto{\pgfqpoint{3.897539in}{3.292106in}}%
\pgfpathlineto{\pgfqpoint{3.897539in}{3.296364in}}%
\pgfpathlineto{\pgfqpoint{3.901797in}{3.296364in}}%
\pgfpathlineto{\pgfqpoint{3.901797in}{3.292106in}}%
\pgfpathmoveto{\pgfqpoint{3.897539in}{3.296364in}}%
\pgfpathlineto{\pgfqpoint{3.897539in}{3.296364in}}%
\pgfpathlineto{\pgfqpoint{3.897539in}{3.300622in}}%
\pgfpathlineto{\pgfqpoint{3.901797in}{3.300622in}}%
\pgfpathlineto{\pgfqpoint{3.901797in}{3.296364in}}%
\pgfpathmoveto{\pgfqpoint{3.893281in}{3.300622in}}%
\pgfpathlineto{\pgfqpoint{3.893281in}{3.300622in}}%
\pgfpathlineto{\pgfqpoint{3.893281in}{3.304880in}}%
\pgfpathlineto{\pgfqpoint{3.897539in}{3.304880in}}%
\pgfpathlineto{\pgfqpoint{3.897539in}{3.300622in}}%
\pgfpathmoveto{\pgfqpoint{3.893281in}{3.304880in}}%
\pgfpathlineto{\pgfqpoint{3.893281in}{3.304880in}}%
\pgfpathlineto{\pgfqpoint{3.893281in}{3.309137in}}%
\pgfpathlineto{\pgfqpoint{3.897539in}{3.309137in}}%
\pgfpathlineto{\pgfqpoint{3.897539in}{3.304880in}}%
\pgfpathmoveto{\pgfqpoint{3.897539in}{3.300622in}}%
\pgfpathlineto{\pgfqpoint{3.897539in}{3.300622in}}%
\pgfpathlineto{\pgfqpoint{3.897539in}{3.304880in}}%
\pgfpathlineto{\pgfqpoint{3.901797in}{3.304880in}}%
\pgfpathlineto{\pgfqpoint{3.901797in}{3.300622in}}%
\pgfpathmoveto{\pgfqpoint{3.897539in}{3.304880in}}%
\pgfpathlineto{\pgfqpoint{3.897539in}{3.304880in}}%
\pgfpathlineto{\pgfqpoint{3.897539in}{3.309137in}}%
\pgfpathlineto{\pgfqpoint{3.901797in}{3.309137in}}%
\pgfpathlineto{\pgfqpoint{3.901797in}{3.304880in}}%
\pgfpathmoveto{\pgfqpoint{3.893281in}{3.309137in}}%
\pgfpathlineto{\pgfqpoint{3.893281in}{3.309137in}}%
\pgfpathlineto{\pgfqpoint{3.893281in}{3.313395in}}%
\pgfpathlineto{\pgfqpoint{3.897539in}{3.313395in}}%
\pgfpathlineto{\pgfqpoint{3.897539in}{3.309137in}}%
\pgfpathmoveto{\pgfqpoint{3.897539in}{3.309137in}}%
\pgfpathlineto{\pgfqpoint{3.897539in}{3.309137in}}%
\pgfpathlineto{\pgfqpoint{3.897539in}{3.313395in}}%
\pgfpathlineto{\pgfqpoint{3.901797in}{3.313395in}}%
\pgfpathlineto{\pgfqpoint{3.901797in}{3.309137in}}%
\pgfpathmoveto{\pgfqpoint{3.897539in}{3.313395in}}%
\pgfpathlineto{\pgfqpoint{3.897539in}{3.313395in}}%
\pgfpathlineto{\pgfqpoint{3.897539in}{3.317653in}}%
\pgfpathlineto{\pgfqpoint{3.901797in}{3.317653in}}%
\pgfpathlineto{\pgfqpoint{3.901797in}{3.313395in}}%
\pgfpathmoveto{\pgfqpoint{3.901797in}{3.313395in}}%
\pgfpathlineto{\pgfqpoint{3.901797in}{3.313395in}}%
\pgfpathlineto{\pgfqpoint{3.901797in}{3.317653in}}%
\pgfpathlineto{\pgfqpoint{3.906055in}{3.317653in}}%
\pgfpathlineto{\pgfqpoint{3.906055in}{3.313395in}}%
\pgfpathmoveto{\pgfqpoint{3.897539in}{3.317653in}}%
\pgfpathlineto{\pgfqpoint{3.897539in}{3.317653in}}%
\pgfpathlineto{\pgfqpoint{3.897539in}{3.321911in}}%
\pgfpathlineto{\pgfqpoint{3.901797in}{3.321911in}}%
\pgfpathlineto{\pgfqpoint{3.901797in}{3.317653in}}%
\pgfpathmoveto{\pgfqpoint{3.897539in}{3.321911in}}%
\pgfpathlineto{\pgfqpoint{3.897539in}{3.321911in}}%
\pgfpathlineto{\pgfqpoint{3.897539in}{3.326168in}}%
\pgfpathlineto{\pgfqpoint{3.901797in}{3.326168in}}%
\pgfpathlineto{\pgfqpoint{3.901797in}{3.321911in}}%
\pgfpathmoveto{\pgfqpoint{3.897539in}{3.326168in}}%
\pgfpathlineto{\pgfqpoint{3.897539in}{3.326168in}}%
\pgfpathlineto{\pgfqpoint{3.897539in}{3.330426in}}%
\pgfpathlineto{\pgfqpoint{3.901797in}{3.330426in}}%
\pgfpathlineto{\pgfqpoint{3.901797in}{3.326168in}}%
\pgfpathmoveto{\pgfqpoint{3.897539in}{3.330426in}}%
\pgfpathlineto{\pgfqpoint{3.897539in}{3.330426in}}%
\pgfpathlineto{\pgfqpoint{3.897539in}{3.334684in}}%
\pgfpathlineto{\pgfqpoint{3.901797in}{3.334684in}}%
\pgfpathlineto{\pgfqpoint{3.901797in}{3.330426in}}%
\pgfpathmoveto{\pgfqpoint{3.901797in}{3.317653in}}%
\pgfpathlineto{\pgfqpoint{3.901797in}{3.317653in}}%
\pgfpathlineto{\pgfqpoint{3.901797in}{3.321911in}}%
\pgfpathlineto{\pgfqpoint{3.906055in}{3.321911in}}%
\pgfpathlineto{\pgfqpoint{3.906055in}{3.317653in}}%
\pgfpathmoveto{\pgfqpoint{3.901797in}{3.321911in}}%
\pgfpathlineto{\pgfqpoint{3.901797in}{3.321911in}}%
\pgfpathlineto{\pgfqpoint{3.901797in}{3.326168in}}%
\pgfpathlineto{\pgfqpoint{3.906055in}{3.326168in}}%
\pgfpathlineto{\pgfqpoint{3.906055in}{3.321911in}}%
\pgfpathmoveto{\pgfqpoint{3.901797in}{3.326168in}}%
\pgfpathlineto{\pgfqpoint{3.901797in}{3.326168in}}%
\pgfpathlineto{\pgfqpoint{3.901797in}{3.330426in}}%
\pgfpathlineto{\pgfqpoint{3.906055in}{3.330426in}}%
\pgfpathlineto{\pgfqpoint{3.906055in}{3.326168in}}%
\pgfpathmoveto{\pgfqpoint{3.901797in}{3.330426in}}%
\pgfpathlineto{\pgfqpoint{3.901797in}{3.330426in}}%
\pgfpathlineto{\pgfqpoint{3.901797in}{3.334684in}}%
\pgfpathlineto{\pgfqpoint{3.906055in}{3.334684in}}%
\pgfpathlineto{\pgfqpoint{3.906055in}{3.330426in}}%
\pgfpathmoveto{\pgfqpoint{3.897539in}{3.334684in}}%
\pgfpathlineto{\pgfqpoint{3.897539in}{3.334684in}}%
\pgfpathlineto{\pgfqpoint{3.897539in}{3.338942in}}%
\pgfpathlineto{\pgfqpoint{3.901797in}{3.338942in}}%
\pgfpathlineto{\pgfqpoint{3.901797in}{3.334684in}}%
\pgfpathmoveto{\pgfqpoint{3.897539in}{3.338942in}}%
\pgfpathlineto{\pgfqpoint{3.897539in}{3.338942in}}%
\pgfpathlineto{\pgfqpoint{3.897539in}{3.343200in}}%
\pgfpathlineto{\pgfqpoint{3.901797in}{3.343200in}}%
\pgfpathlineto{\pgfqpoint{3.901797in}{3.338942in}}%
\pgfpathmoveto{\pgfqpoint{3.897539in}{3.343200in}}%
\pgfpathlineto{\pgfqpoint{3.897539in}{3.343200in}}%
\pgfpathlineto{\pgfqpoint{3.897539in}{3.347457in}}%
\pgfpathlineto{\pgfqpoint{3.901797in}{3.347457in}}%
\pgfpathlineto{\pgfqpoint{3.901797in}{3.343200in}}%
\pgfpathmoveto{\pgfqpoint{3.897539in}{3.347457in}}%
\pgfpathlineto{\pgfqpoint{3.897539in}{3.347457in}}%
\pgfpathlineto{\pgfqpoint{3.897539in}{3.351715in}}%
\pgfpathlineto{\pgfqpoint{3.901797in}{3.351715in}}%
\pgfpathlineto{\pgfqpoint{3.901797in}{3.347457in}}%
\pgfpathmoveto{\pgfqpoint{3.901797in}{3.334684in}}%
\pgfpathlineto{\pgfqpoint{3.901797in}{3.334684in}}%
\pgfpathlineto{\pgfqpoint{3.901797in}{3.338942in}}%
\pgfpathlineto{\pgfqpoint{3.906055in}{3.338942in}}%
\pgfpathlineto{\pgfqpoint{3.906055in}{3.334684in}}%
\pgfpathmoveto{\pgfqpoint{3.901797in}{3.338942in}}%
\pgfpathlineto{\pgfqpoint{3.901797in}{3.338942in}}%
\pgfpathlineto{\pgfqpoint{3.901797in}{3.343200in}}%
\pgfpathlineto{\pgfqpoint{3.906055in}{3.343200in}}%
\pgfpathlineto{\pgfqpoint{3.906055in}{3.338942in}}%
\pgfpathmoveto{\pgfqpoint{3.901797in}{3.343200in}}%
\pgfpathlineto{\pgfqpoint{3.901797in}{3.343200in}}%
\pgfpathlineto{\pgfqpoint{3.901797in}{3.347457in}}%
\pgfpathlineto{\pgfqpoint{3.906055in}{3.347457in}}%
\pgfpathlineto{\pgfqpoint{3.906055in}{3.343200in}}%
\pgfpathmoveto{\pgfqpoint{3.901797in}{3.347457in}}%
\pgfpathlineto{\pgfqpoint{3.901797in}{3.347457in}}%
\pgfpathlineto{\pgfqpoint{3.901797in}{3.351715in}}%
\pgfpathlineto{\pgfqpoint{3.906055in}{3.351715in}}%
\pgfpathlineto{\pgfqpoint{3.906055in}{3.347457in}}%
\pgfpathmoveto{\pgfqpoint{3.906055in}{3.347457in}}%
\pgfpathlineto{\pgfqpoint{3.906055in}{3.347457in}}%
\pgfpathlineto{\pgfqpoint{3.906055in}{3.351715in}}%
\pgfpathlineto{\pgfqpoint{3.910312in}{3.351715in}}%
\pgfpathlineto{\pgfqpoint{3.910312in}{3.347457in}}%
\pgfpathmoveto{\pgfqpoint{3.901797in}{3.351715in}}%
\pgfpathlineto{\pgfqpoint{3.901797in}{3.351715in}}%
\pgfpathlineto{\pgfqpoint{3.901797in}{3.355973in}}%
\pgfpathlineto{\pgfqpoint{3.906055in}{3.355973in}}%
\pgfpathlineto{\pgfqpoint{3.906055in}{3.351715in}}%
\pgfpathmoveto{\pgfqpoint{3.901797in}{3.355973in}}%
\pgfpathlineto{\pgfqpoint{3.901797in}{3.355973in}}%
\pgfpathlineto{\pgfqpoint{3.901797in}{3.360231in}}%
\pgfpathlineto{\pgfqpoint{3.906055in}{3.360231in}}%
\pgfpathlineto{\pgfqpoint{3.906055in}{3.355973in}}%
\pgfpathmoveto{\pgfqpoint{3.906055in}{3.351715in}}%
\pgfpathlineto{\pgfqpoint{3.906055in}{3.351715in}}%
\pgfpathlineto{\pgfqpoint{3.906055in}{3.355973in}}%
\pgfpathlineto{\pgfqpoint{3.910312in}{3.355973in}}%
\pgfpathlineto{\pgfqpoint{3.910312in}{3.351715in}}%
\pgfpathmoveto{\pgfqpoint{3.906055in}{3.355973in}}%
\pgfpathlineto{\pgfqpoint{3.906055in}{3.355973in}}%
\pgfpathlineto{\pgfqpoint{3.906055in}{3.360231in}}%
\pgfpathlineto{\pgfqpoint{3.910312in}{3.360231in}}%
\pgfpathlineto{\pgfqpoint{3.910312in}{3.355973in}}%
\pgfpathmoveto{\pgfqpoint{3.901797in}{3.360231in}}%
\pgfpathlineto{\pgfqpoint{3.901797in}{3.360231in}}%
\pgfpathlineto{\pgfqpoint{3.901797in}{3.364489in}}%
\pgfpathlineto{\pgfqpoint{3.906055in}{3.364489in}}%
\pgfpathlineto{\pgfqpoint{3.906055in}{3.360231in}}%
\pgfpathmoveto{\pgfqpoint{3.901797in}{3.364489in}}%
\pgfpathlineto{\pgfqpoint{3.901797in}{3.364489in}}%
\pgfpathlineto{\pgfqpoint{3.901797in}{3.368746in}}%
\pgfpathlineto{\pgfqpoint{3.906055in}{3.368746in}}%
\pgfpathlineto{\pgfqpoint{3.906055in}{3.364489in}}%
\pgfpathmoveto{\pgfqpoint{3.906055in}{3.360231in}}%
\pgfpathlineto{\pgfqpoint{3.906055in}{3.360231in}}%
\pgfpathlineto{\pgfqpoint{3.906055in}{3.364489in}}%
\pgfpathlineto{\pgfqpoint{3.910312in}{3.364489in}}%
\pgfpathlineto{\pgfqpoint{3.910312in}{3.360231in}}%
\pgfpathmoveto{\pgfqpoint{3.906055in}{3.364489in}}%
\pgfpathlineto{\pgfqpoint{3.906055in}{3.364489in}}%
\pgfpathlineto{\pgfqpoint{3.906055in}{3.368746in}}%
\pgfpathlineto{\pgfqpoint{3.910312in}{3.368746in}}%
\pgfpathlineto{\pgfqpoint{3.910312in}{3.364489in}}%
\pgfpathmoveto{\pgfqpoint{3.901797in}{3.368746in}}%
\pgfpathlineto{\pgfqpoint{3.901797in}{3.368746in}}%
\pgfpathlineto{\pgfqpoint{3.901797in}{3.373004in}}%
\pgfpathlineto{\pgfqpoint{3.906055in}{3.373004in}}%
\pgfpathlineto{\pgfqpoint{3.906055in}{3.368746in}}%
\pgfpathmoveto{\pgfqpoint{3.901797in}{3.373004in}}%
\pgfpathlineto{\pgfqpoint{3.901797in}{3.373004in}}%
\pgfpathlineto{\pgfqpoint{3.901797in}{3.377262in}}%
\pgfpathlineto{\pgfqpoint{3.906055in}{3.377262in}}%
\pgfpathlineto{\pgfqpoint{3.906055in}{3.373004in}}%
\pgfpathmoveto{\pgfqpoint{3.906055in}{3.368746in}}%
\pgfpathlineto{\pgfqpoint{3.906055in}{3.368746in}}%
\pgfpathlineto{\pgfqpoint{3.906055in}{3.373004in}}%
\pgfpathlineto{\pgfqpoint{3.910312in}{3.373004in}}%
\pgfpathlineto{\pgfqpoint{3.910312in}{3.368746in}}%
\pgfpathmoveto{\pgfqpoint{3.906055in}{3.373004in}}%
\pgfpathlineto{\pgfqpoint{3.906055in}{3.373004in}}%
\pgfpathlineto{\pgfqpoint{3.906055in}{3.377262in}}%
\pgfpathlineto{\pgfqpoint{3.910312in}{3.377262in}}%
\pgfpathlineto{\pgfqpoint{3.910312in}{3.373004in}}%
\pgfpathmoveto{\pgfqpoint{3.901797in}{3.377262in}}%
\pgfpathlineto{\pgfqpoint{3.901797in}{3.377262in}}%
\pgfpathlineto{\pgfqpoint{3.901797in}{3.381520in}}%
\pgfpathlineto{\pgfqpoint{3.906055in}{3.381520in}}%
\pgfpathlineto{\pgfqpoint{3.906055in}{3.377262in}}%
\pgfpathmoveto{\pgfqpoint{3.901797in}{3.381520in}}%
\pgfpathlineto{\pgfqpoint{3.901797in}{3.381520in}}%
\pgfpathlineto{\pgfqpoint{3.901797in}{3.385778in}}%
\pgfpathlineto{\pgfqpoint{3.906055in}{3.385778in}}%
\pgfpathlineto{\pgfqpoint{3.906055in}{3.381520in}}%
\pgfpathmoveto{\pgfqpoint{3.906055in}{3.377262in}}%
\pgfpathlineto{\pgfqpoint{3.906055in}{3.377262in}}%
\pgfpathlineto{\pgfqpoint{3.906055in}{3.381520in}}%
\pgfpathlineto{\pgfqpoint{3.910312in}{3.381520in}}%
\pgfpathlineto{\pgfqpoint{3.910312in}{3.377262in}}%
\pgfpathmoveto{\pgfqpoint{3.906055in}{3.381520in}}%
\pgfpathlineto{\pgfqpoint{3.906055in}{3.381520in}}%
\pgfpathlineto{\pgfqpoint{3.906055in}{3.385778in}}%
\pgfpathlineto{\pgfqpoint{3.910312in}{3.385778in}}%
\pgfpathlineto{\pgfqpoint{3.910312in}{3.381520in}}%
\pgfpathmoveto{\pgfqpoint{3.906055in}{3.385778in}}%
\pgfpathlineto{\pgfqpoint{3.906055in}{3.385778in}}%
\pgfpathlineto{\pgfqpoint{3.906055in}{3.390036in}}%
\pgfpathlineto{\pgfqpoint{3.910312in}{3.390036in}}%
\pgfpathlineto{\pgfqpoint{3.910312in}{3.385778in}}%
\pgfpathmoveto{\pgfqpoint{3.906055in}{3.390036in}}%
\pgfpathlineto{\pgfqpoint{3.906055in}{3.390036in}}%
\pgfpathlineto{\pgfqpoint{3.906055in}{3.394294in}}%
\pgfpathlineto{\pgfqpoint{3.910312in}{3.394294in}}%
\pgfpathlineto{\pgfqpoint{3.910312in}{3.390036in}}%
\pgfpathmoveto{\pgfqpoint{3.906055in}{3.394294in}}%
\pgfpathlineto{\pgfqpoint{3.906055in}{3.394294in}}%
\pgfpathlineto{\pgfqpoint{3.906055in}{3.398552in}}%
\pgfpathlineto{\pgfqpoint{3.910312in}{3.398552in}}%
\pgfpathlineto{\pgfqpoint{3.910312in}{3.394294in}}%
\pgfpathmoveto{\pgfqpoint{3.906055in}{3.398552in}}%
\pgfpathlineto{\pgfqpoint{3.906055in}{3.398552in}}%
\pgfpathlineto{\pgfqpoint{3.906055in}{3.402810in}}%
\pgfpathlineto{\pgfqpoint{3.910312in}{3.402810in}}%
\pgfpathlineto{\pgfqpoint{3.910312in}{3.398552in}}%
\pgfpathmoveto{\pgfqpoint{3.906055in}{3.402810in}}%
\pgfpathlineto{\pgfqpoint{3.906055in}{3.402810in}}%
\pgfpathlineto{\pgfqpoint{3.906055in}{3.407068in}}%
\pgfpathlineto{\pgfqpoint{3.910312in}{3.407068in}}%
\pgfpathlineto{\pgfqpoint{3.910312in}{3.402810in}}%
\pgfpathmoveto{\pgfqpoint{3.906055in}{3.407068in}}%
\pgfpathlineto{\pgfqpoint{3.906055in}{3.407068in}}%
\pgfpathlineto{\pgfqpoint{3.906055in}{3.411326in}}%
\pgfpathlineto{\pgfqpoint{3.910312in}{3.411326in}}%
\pgfpathlineto{\pgfqpoint{3.910312in}{3.407068in}}%
\pgfpathmoveto{\pgfqpoint{3.906055in}{3.411326in}}%
\pgfpathlineto{\pgfqpoint{3.906055in}{3.411326in}}%
\pgfpathlineto{\pgfqpoint{3.906055in}{3.415584in}}%
\pgfpathlineto{\pgfqpoint{3.910312in}{3.415584in}}%
\pgfpathlineto{\pgfqpoint{3.910312in}{3.411326in}}%
\pgfpathmoveto{\pgfqpoint{3.906055in}{3.415584in}}%
\pgfpathlineto{\pgfqpoint{3.906055in}{3.415584in}}%
\pgfpathlineto{\pgfqpoint{3.906055in}{3.419842in}}%
\pgfpathlineto{\pgfqpoint{3.910312in}{3.419842in}}%
\pgfpathlineto{\pgfqpoint{3.910312in}{3.415584in}}%
\pgfpathmoveto{\pgfqpoint{3.906055in}{3.419842in}}%
\pgfpathlineto{\pgfqpoint{3.906055in}{3.419842in}}%
\pgfpathlineto{\pgfqpoint{3.906055in}{3.424100in}}%
\pgfpathlineto{\pgfqpoint{3.910312in}{3.424100in}}%
\pgfpathlineto{\pgfqpoint{3.910312in}{3.419842in}}%
\pgfpathmoveto{\pgfqpoint{3.910312in}{3.385778in}}%
\pgfpathlineto{\pgfqpoint{3.910312in}{3.385778in}}%
\pgfpathlineto{\pgfqpoint{3.910312in}{3.390036in}}%
\pgfpathlineto{\pgfqpoint{3.914570in}{3.390036in}}%
\pgfpathlineto{\pgfqpoint{3.914570in}{3.385778in}}%
\pgfpathmoveto{\pgfqpoint{3.910312in}{3.390036in}}%
\pgfpathlineto{\pgfqpoint{3.910312in}{3.390036in}}%
\pgfpathlineto{\pgfqpoint{3.910312in}{3.394294in}}%
\pgfpathlineto{\pgfqpoint{3.914570in}{3.394294in}}%
\pgfpathlineto{\pgfqpoint{3.914570in}{3.390036in}}%
\pgfpathmoveto{\pgfqpoint{3.910312in}{3.394294in}}%
\pgfpathlineto{\pgfqpoint{3.910312in}{3.394294in}}%
\pgfpathlineto{\pgfqpoint{3.910312in}{3.398552in}}%
\pgfpathlineto{\pgfqpoint{3.914570in}{3.398552in}}%
\pgfpathlineto{\pgfqpoint{3.914570in}{3.394294in}}%
\pgfpathmoveto{\pgfqpoint{3.910312in}{3.398552in}}%
\pgfpathlineto{\pgfqpoint{3.910312in}{3.398552in}}%
\pgfpathlineto{\pgfqpoint{3.910312in}{3.402810in}}%
\pgfpathlineto{\pgfqpoint{3.914570in}{3.402810in}}%
\pgfpathlineto{\pgfqpoint{3.914570in}{3.398552in}}%
\pgfpathmoveto{\pgfqpoint{3.910312in}{3.402810in}}%
\pgfpathlineto{\pgfqpoint{3.910312in}{3.402810in}}%
\pgfpathlineto{\pgfqpoint{3.910312in}{3.407068in}}%
\pgfpathlineto{\pgfqpoint{3.914570in}{3.407068in}}%
\pgfpathlineto{\pgfqpoint{3.914570in}{3.402810in}}%
\pgfpathmoveto{\pgfqpoint{3.910312in}{3.407068in}}%
\pgfpathlineto{\pgfqpoint{3.910312in}{3.407068in}}%
\pgfpathlineto{\pgfqpoint{3.910312in}{3.411326in}}%
\pgfpathlineto{\pgfqpoint{3.914570in}{3.411326in}}%
\pgfpathlineto{\pgfqpoint{3.914570in}{3.407068in}}%
\pgfpathmoveto{\pgfqpoint{3.910312in}{3.411326in}}%
\pgfpathlineto{\pgfqpoint{3.910312in}{3.411326in}}%
\pgfpathlineto{\pgfqpoint{3.910312in}{3.415584in}}%
\pgfpathlineto{\pgfqpoint{3.914570in}{3.415584in}}%
\pgfpathlineto{\pgfqpoint{3.914570in}{3.411326in}}%
\pgfpathmoveto{\pgfqpoint{3.910312in}{3.415584in}}%
\pgfpathlineto{\pgfqpoint{3.910312in}{3.415584in}}%
\pgfpathlineto{\pgfqpoint{3.910312in}{3.419842in}}%
\pgfpathlineto{\pgfqpoint{3.914570in}{3.419842in}}%
\pgfpathlineto{\pgfqpoint{3.914570in}{3.415584in}}%
\pgfpathmoveto{\pgfqpoint{3.910312in}{3.419842in}}%
\pgfpathlineto{\pgfqpoint{3.910312in}{3.419842in}}%
\pgfpathlineto{\pgfqpoint{3.910312in}{3.424100in}}%
\pgfpathlineto{\pgfqpoint{3.914570in}{3.424100in}}%
\pgfpathlineto{\pgfqpoint{3.914570in}{3.419842in}}%
\pgfpathmoveto{\pgfqpoint{3.910312in}{3.424100in}}%
\pgfpathlineto{\pgfqpoint{3.910312in}{3.424100in}}%
\pgfpathlineto{\pgfqpoint{3.910312in}{3.428358in}}%
\pgfpathlineto{\pgfqpoint{3.914570in}{3.428358in}}%
\pgfpathlineto{\pgfqpoint{3.914570in}{3.424100in}}%
\pgfpathmoveto{\pgfqpoint{3.914570in}{3.419842in}}%
\pgfpathlineto{\pgfqpoint{3.914570in}{3.419842in}}%
\pgfpathlineto{\pgfqpoint{3.914570in}{3.424100in}}%
\pgfpathlineto{\pgfqpoint{3.918828in}{3.424100in}}%
\pgfpathlineto{\pgfqpoint{3.918828in}{3.419842in}}%
\pgfpathmoveto{\pgfqpoint{3.914570in}{3.424100in}}%
\pgfpathlineto{\pgfqpoint{3.914570in}{3.424100in}}%
\pgfpathlineto{\pgfqpoint{3.914570in}{3.428358in}}%
\pgfpathlineto{\pgfqpoint{3.918828in}{3.428358in}}%
\pgfpathlineto{\pgfqpoint{3.918828in}{3.424100in}}%
\pgfpathmoveto{\pgfqpoint{3.910312in}{3.428358in}}%
\pgfpathlineto{\pgfqpoint{3.910312in}{3.428358in}}%
\pgfpathlineto{\pgfqpoint{3.910312in}{3.432616in}}%
\pgfpathlineto{\pgfqpoint{3.914570in}{3.432616in}}%
\pgfpathlineto{\pgfqpoint{3.914570in}{3.428358in}}%
\pgfpathmoveto{\pgfqpoint{3.910312in}{3.432616in}}%
\pgfpathlineto{\pgfqpoint{3.910312in}{3.432616in}}%
\pgfpathlineto{\pgfqpoint{3.910312in}{3.436874in}}%
\pgfpathlineto{\pgfqpoint{3.914570in}{3.436874in}}%
\pgfpathlineto{\pgfqpoint{3.914570in}{3.432616in}}%
\pgfpathmoveto{\pgfqpoint{3.914570in}{3.428358in}}%
\pgfpathlineto{\pgfqpoint{3.914570in}{3.428358in}}%
\pgfpathlineto{\pgfqpoint{3.914570in}{3.432616in}}%
\pgfpathlineto{\pgfqpoint{3.918828in}{3.432616in}}%
\pgfpathlineto{\pgfqpoint{3.918828in}{3.428358in}}%
\pgfpathmoveto{\pgfqpoint{3.914570in}{3.432616in}}%
\pgfpathlineto{\pgfqpoint{3.914570in}{3.432616in}}%
\pgfpathlineto{\pgfqpoint{3.914570in}{3.436874in}}%
\pgfpathlineto{\pgfqpoint{3.918828in}{3.436874in}}%
\pgfpathlineto{\pgfqpoint{3.918828in}{3.432616in}}%
\pgfpathmoveto{\pgfqpoint{3.910312in}{3.436874in}}%
\pgfpathlineto{\pgfqpoint{3.910312in}{3.436874in}}%
\pgfpathlineto{\pgfqpoint{3.910312in}{3.441132in}}%
\pgfpathlineto{\pgfqpoint{3.914570in}{3.441132in}}%
\pgfpathlineto{\pgfqpoint{3.914570in}{3.436874in}}%
\pgfpathmoveto{\pgfqpoint{3.910312in}{3.441132in}}%
\pgfpathlineto{\pgfqpoint{3.910312in}{3.441132in}}%
\pgfpathlineto{\pgfqpoint{3.910312in}{3.445389in}}%
\pgfpathlineto{\pgfqpoint{3.914570in}{3.445389in}}%
\pgfpathlineto{\pgfqpoint{3.914570in}{3.441132in}}%
\pgfpathmoveto{\pgfqpoint{3.914570in}{3.436874in}}%
\pgfpathlineto{\pgfqpoint{3.914570in}{3.436874in}}%
\pgfpathlineto{\pgfqpoint{3.914570in}{3.441132in}}%
\pgfpathlineto{\pgfqpoint{3.918828in}{3.441132in}}%
\pgfpathlineto{\pgfqpoint{3.918828in}{3.436874in}}%
\pgfpathmoveto{\pgfqpoint{3.914570in}{3.441132in}}%
\pgfpathlineto{\pgfqpoint{3.914570in}{3.441132in}}%
\pgfpathlineto{\pgfqpoint{3.914570in}{3.445389in}}%
\pgfpathlineto{\pgfqpoint{3.918828in}{3.445389in}}%
\pgfpathlineto{\pgfqpoint{3.918828in}{3.441132in}}%
\pgfpathmoveto{\pgfqpoint{3.910312in}{3.445389in}}%
\pgfpathlineto{\pgfqpoint{3.910312in}{3.445389in}}%
\pgfpathlineto{\pgfqpoint{3.910312in}{3.449647in}}%
\pgfpathlineto{\pgfqpoint{3.914570in}{3.449647in}}%
\pgfpathlineto{\pgfqpoint{3.914570in}{3.445389in}}%
\pgfpathmoveto{\pgfqpoint{3.910312in}{3.449647in}}%
\pgfpathlineto{\pgfqpoint{3.910312in}{3.449647in}}%
\pgfpathlineto{\pgfqpoint{3.910312in}{3.453905in}}%
\pgfpathlineto{\pgfqpoint{3.914570in}{3.453905in}}%
\pgfpathlineto{\pgfqpoint{3.914570in}{3.449647in}}%
\pgfpathmoveto{\pgfqpoint{3.914570in}{3.445389in}}%
\pgfpathlineto{\pgfqpoint{3.914570in}{3.445389in}}%
\pgfpathlineto{\pgfqpoint{3.914570in}{3.449647in}}%
\pgfpathlineto{\pgfqpoint{3.918828in}{3.449647in}}%
\pgfpathlineto{\pgfqpoint{3.918828in}{3.445389in}}%
\pgfpathmoveto{\pgfqpoint{3.914570in}{3.449647in}}%
\pgfpathlineto{\pgfqpoint{3.914570in}{3.449647in}}%
\pgfpathlineto{\pgfqpoint{3.914570in}{3.453905in}}%
\pgfpathlineto{\pgfqpoint{3.918828in}{3.453905in}}%
\pgfpathlineto{\pgfqpoint{3.918828in}{3.449647in}}%
\pgfpathmoveto{\pgfqpoint{3.910312in}{3.453905in}}%
\pgfpathlineto{\pgfqpoint{3.910312in}{3.453905in}}%
\pgfpathlineto{\pgfqpoint{3.910312in}{3.458163in}}%
\pgfpathlineto{\pgfqpoint{3.914570in}{3.458163in}}%
\pgfpathlineto{\pgfqpoint{3.914570in}{3.453905in}}%
\pgfpathmoveto{\pgfqpoint{3.914570in}{3.453905in}}%
\pgfpathlineto{\pgfqpoint{3.914570in}{3.453905in}}%
\pgfpathlineto{\pgfqpoint{3.914570in}{3.458163in}}%
\pgfpathlineto{\pgfqpoint{3.918828in}{3.458163in}}%
\pgfpathlineto{\pgfqpoint{3.918828in}{3.453905in}}%
\pgfpathmoveto{\pgfqpoint{3.914570in}{3.458163in}}%
\pgfpathlineto{\pgfqpoint{3.914570in}{3.458163in}}%
\pgfpathlineto{\pgfqpoint{3.914570in}{3.462421in}}%
\pgfpathlineto{\pgfqpoint{3.918828in}{3.462421in}}%
\pgfpathlineto{\pgfqpoint{3.918828in}{3.458163in}}%
\pgfpathmoveto{\pgfqpoint{3.914570in}{3.462421in}}%
\pgfpathlineto{\pgfqpoint{3.914570in}{3.462421in}}%
\pgfpathlineto{\pgfqpoint{3.914570in}{3.466679in}}%
\pgfpathlineto{\pgfqpoint{3.918828in}{3.466679in}}%
\pgfpathlineto{\pgfqpoint{3.918828in}{3.462421in}}%
\pgfpathmoveto{\pgfqpoint{3.914570in}{3.466679in}}%
\pgfpathlineto{\pgfqpoint{3.914570in}{3.466679in}}%
\pgfpathlineto{\pgfqpoint{3.914570in}{3.470937in}}%
\pgfpathlineto{\pgfqpoint{3.918828in}{3.470937in}}%
\pgfpathlineto{\pgfqpoint{3.918828in}{3.466679in}}%
\pgfpathmoveto{\pgfqpoint{3.918828in}{3.458163in}}%
\pgfpathlineto{\pgfqpoint{3.918828in}{3.458163in}}%
\pgfpathlineto{\pgfqpoint{3.918828in}{3.462421in}}%
\pgfpathlineto{\pgfqpoint{3.923086in}{3.462421in}}%
\pgfpathlineto{\pgfqpoint{3.923086in}{3.458163in}}%
\pgfpathmoveto{\pgfqpoint{3.918828in}{3.462421in}}%
\pgfpathlineto{\pgfqpoint{3.918828in}{3.462421in}}%
\pgfpathlineto{\pgfqpoint{3.918828in}{3.466679in}}%
\pgfpathlineto{\pgfqpoint{3.923086in}{3.466679in}}%
\pgfpathlineto{\pgfqpoint{3.923086in}{3.462421in}}%
\pgfpathmoveto{\pgfqpoint{3.918828in}{3.466679in}}%
\pgfpathlineto{\pgfqpoint{3.918828in}{3.466679in}}%
\pgfpathlineto{\pgfqpoint{3.918828in}{3.470937in}}%
\pgfpathlineto{\pgfqpoint{3.923086in}{3.470937in}}%
\pgfpathlineto{\pgfqpoint{3.923086in}{3.466679in}}%
\pgfpathmoveto{\pgfqpoint{3.914570in}{3.470937in}}%
\pgfpathlineto{\pgfqpoint{3.914570in}{3.470937in}}%
\pgfpathlineto{\pgfqpoint{3.914570in}{3.475195in}}%
\pgfpathlineto{\pgfqpoint{3.918828in}{3.475195in}}%
\pgfpathlineto{\pgfqpoint{3.918828in}{3.470937in}}%
\pgfpathmoveto{\pgfqpoint{3.914570in}{3.475195in}}%
\pgfpathlineto{\pgfqpoint{3.914570in}{3.475195in}}%
\pgfpathlineto{\pgfqpoint{3.914570in}{3.479453in}}%
\pgfpathlineto{\pgfqpoint{3.918828in}{3.479453in}}%
\pgfpathlineto{\pgfqpoint{3.918828in}{3.475195in}}%
\pgfpathmoveto{\pgfqpoint{3.914570in}{3.479453in}}%
\pgfpathlineto{\pgfqpoint{3.914570in}{3.479453in}}%
\pgfpathlineto{\pgfqpoint{3.914570in}{3.483711in}}%
\pgfpathlineto{\pgfqpoint{3.918828in}{3.483711in}}%
\pgfpathlineto{\pgfqpoint{3.918828in}{3.479453in}}%
\pgfpathmoveto{\pgfqpoint{3.914570in}{3.483711in}}%
\pgfpathlineto{\pgfqpoint{3.914570in}{3.483711in}}%
\pgfpathlineto{\pgfqpoint{3.914570in}{3.487969in}}%
\pgfpathlineto{\pgfqpoint{3.918828in}{3.487969in}}%
\pgfpathlineto{\pgfqpoint{3.918828in}{3.483711in}}%
\pgfpathmoveto{\pgfqpoint{3.918828in}{3.470937in}}%
\pgfpathlineto{\pgfqpoint{3.918828in}{3.470937in}}%
\pgfpathlineto{\pgfqpoint{3.918828in}{3.475195in}}%
\pgfpathlineto{\pgfqpoint{3.923086in}{3.475195in}}%
\pgfpathlineto{\pgfqpoint{3.923086in}{3.470937in}}%
\pgfpathmoveto{\pgfqpoint{3.918828in}{3.475195in}}%
\pgfpathlineto{\pgfqpoint{3.918828in}{3.475195in}}%
\pgfpathlineto{\pgfqpoint{3.918828in}{3.479453in}}%
\pgfpathlineto{\pgfqpoint{3.923086in}{3.479453in}}%
\pgfpathlineto{\pgfqpoint{3.923086in}{3.475195in}}%
\pgfpathmoveto{\pgfqpoint{3.918828in}{3.479453in}}%
\pgfpathlineto{\pgfqpoint{3.918828in}{3.479453in}}%
\pgfpathlineto{\pgfqpoint{3.918828in}{3.483711in}}%
\pgfpathlineto{\pgfqpoint{3.923086in}{3.483711in}}%
\pgfpathlineto{\pgfqpoint{3.923086in}{3.479453in}}%
\pgfpathmoveto{\pgfqpoint{3.918828in}{3.483711in}}%
\pgfpathlineto{\pgfqpoint{3.918828in}{3.483711in}}%
\pgfpathlineto{\pgfqpoint{3.918828in}{3.487969in}}%
\pgfpathlineto{\pgfqpoint{3.923086in}{3.487969in}}%
\pgfpathlineto{\pgfqpoint{3.923086in}{3.483711in}}%
\pgfpathmoveto{\pgfqpoint{3.914570in}{3.487969in}}%
\pgfpathlineto{\pgfqpoint{3.914570in}{3.487969in}}%
\pgfpathlineto{\pgfqpoint{3.914570in}{3.492227in}}%
\pgfpathlineto{\pgfqpoint{3.918828in}{3.492227in}}%
\pgfpathlineto{\pgfqpoint{3.918828in}{3.487969in}}%
\pgfpathmoveto{\pgfqpoint{3.914570in}{3.492227in}}%
\pgfpathlineto{\pgfqpoint{3.914570in}{3.492227in}}%
\pgfpathlineto{\pgfqpoint{3.914570in}{3.496485in}}%
\pgfpathlineto{\pgfqpoint{3.918828in}{3.496485in}}%
\pgfpathlineto{\pgfqpoint{3.918828in}{3.492227in}}%
\pgfpathmoveto{\pgfqpoint{3.918828in}{3.487969in}}%
\pgfpathlineto{\pgfqpoint{3.918828in}{3.487969in}}%
\pgfpathlineto{\pgfqpoint{3.918828in}{3.492227in}}%
\pgfpathlineto{\pgfqpoint{3.923086in}{3.492227in}}%
\pgfpathlineto{\pgfqpoint{3.923086in}{3.487969in}}%
\pgfpathmoveto{\pgfqpoint{3.918828in}{3.492227in}}%
\pgfpathlineto{\pgfqpoint{3.918828in}{3.492227in}}%
\pgfpathlineto{\pgfqpoint{3.918828in}{3.496485in}}%
\pgfpathlineto{\pgfqpoint{3.923086in}{3.496485in}}%
\pgfpathlineto{\pgfqpoint{3.923086in}{3.492227in}}%
\pgfpathmoveto{\pgfqpoint{3.923086in}{3.492227in}}%
\pgfpathlineto{\pgfqpoint{3.923086in}{3.492227in}}%
\pgfpathlineto{\pgfqpoint{3.923086in}{3.496485in}}%
\pgfpathlineto{\pgfqpoint{3.927344in}{3.496485in}}%
\pgfpathlineto{\pgfqpoint{3.927344in}{3.492227in}}%
\pgfpathmoveto{\pgfqpoint{3.918828in}{3.496485in}}%
\pgfpathlineto{\pgfqpoint{3.918828in}{3.496485in}}%
\pgfpathlineto{\pgfqpoint{3.918828in}{3.500743in}}%
\pgfpathlineto{\pgfqpoint{3.923086in}{3.500743in}}%
\pgfpathlineto{\pgfqpoint{3.923086in}{3.496485in}}%
\pgfpathmoveto{\pgfqpoint{3.918828in}{3.500743in}}%
\pgfpathlineto{\pgfqpoint{3.918828in}{3.500743in}}%
\pgfpathlineto{\pgfqpoint{3.918828in}{3.505001in}}%
\pgfpathlineto{\pgfqpoint{3.923086in}{3.505001in}}%
\pgfpathlineto{\pgfqpoint{3.923086in}{3.500743in}}%
\pgfpathmoveto{\pgfqpoint{3.923086in}{3.496485in}}%
\pgfpathlineto{\pgfqpoint{3.923086in}{3.496485in}}%
\pgfpathlineto{\pgfqpoint{3.923086in}{3.500743in}}%
\pgfpathlineto{\pgfqpoint{3.927344in}{3.500743in}}%
\pgfpathlineto{\pgfqpoint{3.927344in}{3.496485in}}%
\pgfpathmoveto{\pgfqpoint{3.923086in}{3.500743in}}%
\pgfpathlineto{\pgfqpoint{3.923086in}{3.500743in}}%
\pgfpathlineto{\pgfqpoint{3.923086in}{3.505001in}}%
\pgfpathlineto{\pgfqpoint{3.927344in}{3.505001in}}%
\pgfpathlineto{\pgfqpoint{3.927344in}{3.500743in}}%
\pgfpathmoveto{\pgfqpoint{3.918828in}{3.505001in}}%
\pgfpathlineto{\pgfqpoint{3.918828in}{3.505001in}}%
\pgfpathlineto{\pgfqpoint{3.918828in}{3.509259in}}%
\pgfpathlineto{\pgfqpoint{3.923086in}{3.509259in}}%
\pgfpathlineto{\pgfqpoint{3.923086in}{3.505001in}}%
\pgfpathmoveto{\pgfqpoint{3.918828in}{3.509259in}}%
\pgfpathlineto{\pgfqpoint{3.918828in}{3.509259in}}%
\pgfpathlineto{\pgfqpoint{3.918828in}{3.513517in}}%
\pgfpathlineto{\pgfqpoint{3.923086in}{3.513517in}}%
\pgfpathlineto{\pgfqpoint{3.923086in}{3.509259in}}%
\pgfpathmoveto{\pgfqpoint{3.923086in}{3.505001in}}%
\pgfpathlineto{\pgfqpoint{3.923086in}{3.505001in}}%
\pgfpathlineto{\pgfqpoint{3.923086in}{3.509259in}}%
\pgfpathlineto{\pgfqpoint{3.927344in}{3.509259in}}%
\pgfpathlineto{\pgfqpoint{3.927344in}{3.505001in}}%
\pgfpathmoveto{\pgfqpoint{3.923086in}{3.509259in}}%
\pgfpathlineto{\pgfqpoint{3.923086in}{3.509259in}}%
\pgfpathlineto{\pgfqpoint{3.923086in}{3.513517in}}%
\pgfpathlineto{\pgfqpoint{3.927344in}{3.513517in}}%
\pgfpathlineto{\pgfqpoint{3.927344in}{3.509259in}}%
\pgfpathmoveto{\pgfqpoint{3.918828in}{3.513517in}}%
\pgfpathlineto{\pgfqpoint{3.918828in}{3.513517in}}%
\pgfpathlineto{\pgfqpoint{3.918828in}{3.517774in}}%
\pgfpathlineto{\pgfqpoint{3.923086in}{3.517774in}}%
\pgfpathlineto{\pgfqpoint{3.923086in}{3.513517in}}%
\pgfpathmoveto{\pgfqpoint{3.918828in}{3.517774in}}%
\pgfpathlineto{\pgfqpoint{3.918828in}{3.517774in}}%
\pgfpathlineto{\pgfqpoint{3.918828in}{3.522032in}}%
\pgfpathlineto{\pgfqpoint{3.923086in}{3.522032in}}%
\pgfpathlineto{\pgfqpoint{3.923086in}{3.517774in}}%
\pgfpathmoveto{\pgfqpoint{3.923086in}{3.513517in}}%
\pgfpathlineto{\pgfqpoint{3.923086in}{3.513517in}}%
\pgfpathlineto{\pgfqpoint{3.923086in}{3.517774in}}%
\pgfpathlineto{\pgfqpoint{3.927344in}{3.517774in}}%
\pgfpathlineto{\pgfqpoint{3.927344in}{3.513517in}}%
\pgfpathmoveto{\pgfqpoint{3.923086in}{3.517774in}}%
\pgfpathlineto{\pgfqpoint{3.923086in}{3.517774in}}%
\pgfpathlineto{\pgfqpoint{3.923086in}{3.522032in}}%
\pgfpathlineto{\pgfqpoint{3.927344in}{3.522032in}}%
\pgfpathlineto{\pgfqpoint{3.927344in}{3.517774in}}%
\pgfpathmoveto{\pgfqpoint{3.918828in}{3.522032in}}%
\pgfpathlineto{\pgfqpoint{3.918828in}{3.522032in}}%
\pgfpathlineto{\pgfqpoint{3.918828in}{3.526290in}}%
\pgfpathlineto{\pgfqpoint{3.923086in}{3.526290in}}%
\pgfpathlineto{\pgfqpoint{3.923086in}{3.522032in}}%
\pgfpathmoveto{\pgfqpoint{3.918828in}{3.526290in}}%
\pgfpathlineto{\pgfqpoint{3.918828in}{3.526290in}}%
\pgfpathlineto{\pgfqpoint{3.918828in}{3.530548in}}%
\pgfpathlineto{\pgfqpoint{3.923086in}{3.530548in}}%
\pgfpathlineto{\pgfqpoint{3.923086in}{3.526290in}}%
\pgfpathmoveto{\pgfqpoint{3.923086in}{3.522032in}}%
\pgfpathlineto{\pgfqpoint{3.923086in}{3.522032in}}%
\pgfpathlineto{\pgfqpoint{3.923086in}{3.526290in}}%
\pgfpathlineto{\pgfqpoint{3.927344in}{3.526290in}}%
\pgfpathlineto{\pgfqpoint{3.927344in}{3.522032in}}%
\pgfpathmoveto{\pgfqpoint{3.923086in}{3.526290in}}%
\pgfpathlineto{\pgfqpoint{3.923086in}{3.526290in}}%
\pgfpathlineto{\pgfqpoint{3.923086in}{3.530548in}}%
\pgfpathlineto{\pgfqpoint{3.927344in}{3.530548in}}%
\pgfpathlineto{\pgfqpoint{3.927344in}{3.526290in}}%
\pgfpathmoveto{\pgfqpoint{3.923086in}{3.530548in}}%
\pgfpathlineto{\pgfqpoint{3.923086in}{3.530548in}}%
\pgfpathlineto{\pgfqpoint{3.923086in}{3.534806in}}%
\pgfpathlineto{\pgfqpoint{3.927344in}{3.534806in}}%
\pgfpathlineto{\pgfqpoint{3.927344in}{3.530548in}}%
\pgfpathmoveto{\pgfqpoint{3.923086in}{3.534806in}}%
\pgfpathlineto{\pgfqpoint{3.923086in}{3.534806in}}%
\pgfpathlineto{\pgfqpoint{3.923086in}{3.539064in}}%
\pgfpathlineto{\pgfqpoint{3.927344in}{3.539064in}}%
\pgfpathlineto{\pgfqpoint{3.927344in}{3.534806in}}%
\pgfpathmoveto{\pgfqpoint{3.927344in}{3.530548in}}%
\pgfpathlineto{\pgfqpoint{3.927344in}{3.530548in}}%
\pgfpathlineto{\pgfqpoint{3.927344in}{3.534806in}}%
\pgfpathlineto{\pgfqpoint{3.931602in}{3.534806in}}%
\pgfpathlineto{\pgfqpoint{3.931602in}{3.530548in}}%
\pgfpathmoveto{\pgfqpoint{3.927344in}{3.534806in}}%
\pgfpathlineto{\pgfqpoint{3.927344in}{3.534806in}}%
\pgfpathlineto{\pgfqpoint{3.927344in}{3.539064in}}%
\pgfpathlineto{\pgfqpoint{3.931602in}{3.539064in}}%
\pgfpathlineto{\pgfqpoint{3.931602in}{3.534806in}}%
\pgfpathmoveto{\pgfqpoint{3.923086in}{3.539064in}}%
\pgfpathlineto{\pgfqpoint{3.923086in}{3.539064in}}%
\pgfpathlineto{\pgfqpoint{3.923086in}{3.543322in}}%
\pgfpathlineto{\pgfqpoint{3.927344in}{3.543322in}}%
\pgfpathlineto{\pgfqpoint{3.927344in}{3.539064in}}%
\pgfpathmoveto{\pgfqpoint{3.923086in}{3.543322in}}%
\pgfpathlineto{\pgfqpoint{3.923086in}{3.543322in}}%
\pgfpathlineto{\pgfqpoint{3.923086in}{3.547580in}}%
\pgfpathlineto{\pgfqpoint{3.927344in}{3.547580in}}%
\pgfpathlineto{\pgfqpoint{3.927344in}{3.543322in}}%
\pgfpathmoveto{\pgfqpoint{3.923086in}{3.547580in}}%
\pgfpathlineto{\pgfqpoint{3.923086in}{3.547580in}}%
\pgfpathlineto{\pgfqpoint{3.923086in}{3.551838in}}%
\pgfpathlineto{\pgfqpoint{3.927344in}{3.551838in}}%
\pgfpathlineto{\pgfqpoint{3.927344in}{3.547580in}}%
\pgfpathmoveto{\pgfqpoint{3.923086in}{3.551838in}}%
\pgfpathlineto{\pgfqpoint{3.923086in}{3.551838in}}%
\pgfpathlineto{\pgfqpoint{3.923086in}{3.556095in}}%
\pgfpathlineto{\pgfqpoint{3.927344in}{3.556095in}}%
\pgfpathlineto{\pgfqpoint{3.927344in}{3.551838in}}%
\pgfpathmoveto{\pgfqpoint{3.923086in}{3.556095in}}%
\pgfpathlineto{\pgfqpoint{3.923086in}{3.556095in}}%
\pgfpathlineto{\pgfqpoint{3.923086in}{3.560353in}}%
\pgfpathlineto{\pgfqpoint{3.927344in}{3.560353in}}%
\pgfpathlineto{\pgfqpoint{3.927344in}{3.556095in}}%
\pgfpathmoveto{\pgfqpoint{3.923086in}{3.560353in}}%
\pgfpathlineto{\pgfqpoint{3.923086in}{3.560353in}}%
\pgfpathlineto{\pgfqpoint{3.923086in}{3.564611in}}%
\pgfpathlineto{\pgfqpoint{3.927344in}{3.564611in}}%
\pgfpathlineto{\pgfqpoint{3.927344in}{3.560353in}}%
\pgfpathmoveto{\pgfqpoint{3.923086in}{3.564611in}}%
\pgfpathlineto{\pgfqpoint{3.923086in}{3.564611in}}%
\pgfpathlineto{\pgfqpoint{3.923086in}{3.568869in}}%
\pgfpathlineto{\pgfqpoint{3.927344in}{3.568869in}}%
\pgfpathlineto{\pgfqpoint{3.927344in}{3.564611in}}%
\pgfpathmoveto{\pgfqpoint{3.927344in}{3.539064in}}%
\pgfpathlineto{\pgfqpoint{3.927344in}{3.539064in}}%
\pgfpathlineto{\pgfqpoint{3.927344in}{3.543322in}}%
\pgfpathlineto{\pgfqpoint{3.931602in}{3.543322in}}%
\pgfpathlineto{\pgfqpoint{3.931602in}{3.539064in}}%
\pgfpathmoveto{\pgfqpoint{3.927344in}{3.543322in}}%
\pgfpathlineto{\pgfqpoint{3.927344in}{3.543322in}}%
\pgfpathlineto{\pgfqpoint{3.927344in}{3.547580in}}%
\pgfpathlineto{\pgfqpoint{3.931602in}{3.547580in}}%
\pgfpathlineto{\pgfqpoint{3.931602in}{3.543322in}}%
\pgfpathmoveto{\pgfqpoint{3.927344in}{3.547580in}}%
\pgfpathlineto{\pgfqpoint{3.927344in}{3.547580in}}%
\pgfpathlineto{\pgfqpoint{3.927344in}{3.551838in}}%
\pgfpathlineto{\pgfqpoint{3.931602in}{3.551838in}}%
\pgfpathlineto{\pgfqpoint{3.931602in}{3.547580in}}%
\pgfpathmoveto{\pgfqpoint{3.927344in}{3.551838in}}%
\pgfpathlineto{\pgfqpoint{3.927344in}{3.551838in}}%
\pgfpathlineto{\pgfqpoint{3.927344in}{3.556095in}}%
\pgfpathlineto{\pgfqpoint{3.931602in}{3.556095in}}%
\pgfpathlineto{\pgfqpoint{3.931602in}{3.551838in}}%
\pgfpathmoveto{\pgfqpoint{3.927344in}{3.556095in}}%
\pgfpathlineto{\pgfqpoint{3.927344in}{3.556095in}}%
\pgfpathlineto{\pgfqpoint{3.927344in}{3.560353in}}%
\pgfpathlineto{\pgfqpoint{3.931602in}{3.560353in}}%
\pgfpathlineto{\pgfqpoint{3.931602in}{3.556095in}}%
\pgfpathmoveto{\pgfqpoint{3.927344in}{3.560353in}}%
\pgfpathlineto{\pgfqpoint{3.927344in}{3.560353in}}%
\pgfpathlineto{\pgfqpoint{3.927344in}{3.564611in}}%
\pgfpathlineto{\pgfqpoint{3.931602in}{3.564611in}}%
\pgfpathlineto{\pgfqpoint{3.931602in}{3.560353in}}%
\pgfpathmoveto{\pgfqpoint{3.927344in}{3.564611in}}%
\pgfpathlineto{\pgfqpoint{3.927344in}{3.564611in}}%
\pgfpathlineto{\pgfqpoint{3.927344in}{3.568869in}}%
\pgfpathlineto{\pgfqpoint{3.931602in}{3.568869in}}%
\pgfpathlineto{\pgfqpoint{3.931602in}{3.564611in}}%
\pgfpathmoveto{\pgfqpoint{3.927344in}{3.568869in}}%
\pgfpathlineto{\pgfqpoint{3.927344in}{3.568869in}}%
\pgfpathlineto{\pgfqpoint{3.927344in}{3.573127in}}%
\pgfpathlineto{\pgfqpoint{3.931602in}{3.573127in}}%
\pgfpathlineto{\pgfqpoint{3.931602in}{3.568869in}}%
\pgfpathmoveto{\pgfqpoint{3.931602in}{3.568869in}}%
\pgfpathlineto{\pgfqpoint{3.931602in}{3.568869in}}%
\pgfpathlineto{\pgfqpoint{3.931602in}{3.573127in}}%
\pgfpathlineto{\pgfqpoint{3.935860in}{3.573127in}}%
\pgfpathlineto{\pgfqpoint{3.935860in}{3.568869in}}%
\pgfpathmoveto{\pgfqpoint{3.927344in}{3.573127in}}%
\pgfpathlineto{\pgfqpoint{3.927344in}{3.573127in}}%
\pgfpathlineto{\pgfqpoint{3.927344in}{3.577385in}}%
\pgfpathlineto{\pgfqpoint{3.931602in}{3.577385in}}%
\pgfpathlineto{\pgfqpoint{3.931602in}{3.573127in}}%
\pgfpathmoveto{\pgfqpoint{3.927344in}{3.577385in}}%
\pgfpathlineto{\pgfqpoint{3.927344in}{3.577385in}}%
\pgfpathlineto{\pgfqpoint{3.927344in}{3.581643in}}%
\pgfpathlineto{\pgfqpoint{3.931602in}{3.581643in}}%
\pgfpathlineto{\pgfqpoint{3.931602in}{3.577385in}}%
\pgfpathmoveto{\pgfqpoint{3.931602in}{3.573127in}}%
\pgfpathlineto{\pgfqpoint{3.931602in}{3.573127in}}%
\pgfpathlineto{\pgfqpoint{3.931602in}{3.577385in}}%
\pgfpathlineto{\pgfqpoint{3.935860in}{3.577385in}}%
\pgfpathlineto{\pgfqpoint{3.935860in}{3.573127in}}%
\pgfpathmoveto{\pgfqpoint{3.931602in}{3.577385in}}%
\pgfpathlineto{\pgfqpoint{3.931602in}{3.577385in}}%
\pgfpathlineto{\pgfqpoint{3.931602in}{3.581643in}}%
\pgfpathlineto{\pgfqpoint{3.935860in}{3.581643in}}%
\pgfpathlineto{\pgfqpoint{3.935860in}{3.577385in}}%
\pgfpathmoveto{\pgfqpoint{3.927344in}{3.581643in}}%
\pgfpathlineto{\pgfqpoint{3.927344in}{3.581643in}}%
\pgfpathlineto{\pgfqpoint{3.927344in}{3.585901in}}%
\pgfpathlineto{\pgfqpoint{3.931602in}{3.585901in}}%
\pgfpathlineto{\pgfqpoint{3.931602in}{3.581643in}}%
\pgfpathmoveto{\pgfqpoint{3.927344in}{3.585901in}}%
\pgfpathlineto{\pgfqpoint{3.927344in}{3.585901in}}%
\pgfpathlineto{\pgfqpoint{3.927344in}{3.590159in}}%
\pgfpathlineto{\pgfqpoint{3.931602in}{3.590159in}}%
\pgfpathlineto{\pgfqpoint{3.931602in}{3.585901in}}%
\pgfpathmoveto{\pgfqpoint{3.931602in}{3.581643in}}%
\pgfpathlineto{\pgfqpoint{3.931602in}{3.581643in}}%
\pgfpathlineto{\pgfqpoint{3.931602in}{3.585901in}}%
\pgfpathlineto{\pgfqpoint{3.935860in}{3.585901in}}%
\pgfpathlineto{\pgfqpoint{3.935860in}{3.581643in}}%
\pgfpathmoveto{\pgfqpoint{3.931602in}{3.585901in}}%
\pgfpathlineto{\pgfqpoint{3.931602in}{3.585901in}}%
\pgfpathlineto{\pgfqpoint{3.931602in}{3.590159in}}%
\pgfpathlineto{\pgfqpoint{3.935860in}{3.590159in}}%
\pgfpathlineto{\pgfqpoint{3.935860in}{3.585901in}}%
\pgfpathmoveto{\pgfqpoint{3.927344in}{3.590159in}}%
\pgfpathlineto{\pgfqpoint{3.927344in}{3.590159in}}%
\pgfpathlineto{\pgfqpoint{3.927344in}{3.594417in}}%
\pgfpathlineto{\pgfqpoint{3.931602in}{3.594417in}}%
\pgfpathlineto{\pgfqpoint{3.931602in}{3.590159in}}%
\pgfpathmoveto{\pgfqpoint{3.927344in}{3.594417in}}%
\pgfpathlineto{\pgfqpoint{3.927344in}{3.594417in}}%
\pgfpathlineto{\pgfqpoint{3.927344in}{3.598674in}}%
\pgfpathlineto{\pgfqpoint{3.931602in}{3.598674in}}%
\pgfpathlineto{\pgfqpoint{3.931602in}{3.594417in}}%
\pgfpathmoveto{\pgfqpoint{3.931602in}{3.590159in}}%
\pgfpathlineto{\pgfqpoint{3.931602in}{3.590159in}}%
\pgfpathlineto{\pgfqpoint{3.931602in}{3.594417in}}%
\pgfpathlineto{\pgfqpoint{3.935860in}{3.594417in}}%
\pgfpathlineto{\pgfqpoint{3.935860in}{3.590159in}}%
\pgfpathmoveto{\pgfqpoint{3.931602in}{3.594417in}}%
\pgfpathlineto{\pgfqpoint{3.931602in}{3.594417in}}%
\pgfpathlineto{\pgfqpoint{3.931602in}{3.598674in}}%
\pgfpathlineto{\pgfqpoint{3.935860in}{3.598674in}}%
\pgfpathlineto{\pgfqpoint{3.935860in}{3.594417in}}%
\pgfpathmoveto{\pgfqpoint{3.927344in}{3.598674in}}%
\pgfpathlineto{\pgfqpoint{3.927344in}{3.598674in}}%
\pgfpathlineto{\pgfqpoint{3.927344in}{3.602932in}}%
\pgfpathlineto{\pgfqpoint{3.931602in}{3.602932in}}%
\pgfpathlineto{\pgfqpoint{3.931602in}{3.598674in}}%
\pgfpathmoveto{\pgfqpoint{3.927344in}{3.602932in}}%
\pgfpathlineto{\pgfqpoint{3.927344in}{3.602932in}}%
\pgfpathlineto{\pgfqpoint{3.927344in}{3.607190in}}%
\pgfpathlineto{\pgfqpoint{3.931602in}{3.607190in}}%
\pgfpathlineto{\pgfqpoint{3.931602in}{3.602932in}}%
\pgfpathmoveto{\pgfqpoint{3.931602in}{3.598674in}}%
\pgfpathlineto{\pgfqpoint{3.931602in}{3.598674in}}%
\pgfpathlineto{\pgfqpoint{3.931602in}{3.602932in}}%
\pgfpathlineto{\pgfqpoint{3.935860in}{3.602932in}}%
\pgfpathlineto{\pgfqpoint{3.935860in}{3.598674in}}%
\pgfpathmoveto{\pgfqpoint{3.931602in}{3.602932in}}%
\pgfpathlineto{\pgfqpoint{3.931602in}{3.602932in}}%
\pgfpathlineto{\pgfqpoint{3.931602in}{3.607190in}}%
\pgfpathlineto{\pgfqpoint{3.935860in}{3.607190in}}%
\pgfpathlineto{\pgfqpoint{3.935860in}{3.602932in}}%
\pgfpathmoveto{\pgfqpoint{3.931602in}{3.607190in}}%
\pgfpathlineto{\pgfqpoint{3.931602in}{3.607190in}}%
\pgfpathlineto{\pgfqpoint{3.931602in}{3.611448in}}%
\pgfpathlineto{\pgfqpoint{3.935860in}{3.611448in}}%
\pgfpathlineto{\pgfqpoint{3.935860in}{3.607190in}}%
\pgfpathmoveto{\pgfqpoint{3.931602in}{3.611448in}}%
\pgfpathlineto{\pgfqpoint{3.931602in}{3.611448in}}%
\pgfpathlineto{\pgfqpoint{3.931602in}{3.615706in}}%
\pgfpathlineto{\pgfqpoint{3.935860in}{3.615706in}}%
\pgfpathlineto{\pgfqpoint{3.935860in}{3.611448in}}%
\pgfpathmoveto{\pgfqpoint{3.931602in}{3.615706in}}%
\pgfpathlineto{\pgfqpoint{3.931602in}{3.615706in}}%
\pgfpathlineto{\pgfqpoint{3.931602in}{3.619964in}}%
\pgfpathlineto{\pgfqpoint{3.935860in}{3.619964in}}%
\pgfpathlineto{\pgfqpoint{3.935860in}{3.615706in}}%
\pgfpathmoveto{\pgfqpoint{3.931602in}{3.619964in}}%
\pgfpathlineto{\pgfqpoint{3.931602in}{3.619964in}}%
\pgfpathlineto{\pgfqpoint{3.931602in}{3.624222in}}%
\pgfpathlineto{\pgfqpoint{3.935860in}{3.624222in}}%
\pgfpathlineto{\pgfqpoint{3.935860in}{3.619964in}}%
\pgfpathmoveto{\pgfqpoint{3.935860in}{3.607190in}}%
\pgfpathlineto{\pgfqpoint{3.935860in}{3.607190in}}%
\pgfpathlineto{\pgfqpoint{3.935860in}{3.611448in}}%
\pgfpathlineto{\pgfqpoint{3.940118in}{3.611448in}}%
\pgfpathlineto{\pgfqpoint{3.940118in}{3.607190in}}%
\pgfpathmoveto{\pgfqpoint{3.935860in}{3.611448in}}%
\pgfpathlineto{\pgfqpoint{3.935860in}{3.611448in}}%
\pgfpathlineto{\pgfqpoint{3.935860in}{3.615706in}}%
\pgfpathlineto{\pgfqpoint{3.940118in}{3.615706in}}%
\pgfpathlineto{\pgfqpoint{3.940118in}{3.611448in}}%
\pgfpathmoveto{\pgfqpoint{3.935860in}{3.615706in}}%
\pgfpathlineto{\pgfqpoint{3.935860in}{3.615706in}}%
\pgfpathlineto{\pgfqpoint{3.935860in}{3.619964in}}%
\pgfpathlineto{\pgfqpoint{3.940118in}{3.619964in}}%
\pgfpathlineto{\pgfqpoint{3.940118in}{3.615706in}}%
\pgfpathmoveto{\pgfqpoint{3.935860in}{3.619964in}}%
\pgfpathlineto{\pgfqpoint{3.935860in}{3.619964in}}%
\pgfpathlineto{\pgfqpoint{3.935860in}{3.624222in}}%
\pgfpathlineto{\pgfqpoint{3.940118in}{3.624222in}}%
\pgfpathlineto{\pgfqpoint{3.940118in}{3.619964in}}%
\pgfpathmoveto{\pgfqpoint{3.931602in}{3.624222in}}%
\pgfpathlineto{\pgfqpoint{3.931602in}{3.624222in}}%
\pgfpathlineto{\pgfqpoint{3.931602in}{3.628480in}}%
\pgfpathlineto{\pgfqpoint{3.935860in}{3.628480in}}%
\pgfpathlineto{\pgfqpoint{3.935860in}{3.624222in}}%
\pgfpathmoveto{\pgfqpoint{3.931602in}{3.628480in}}%
\pgfpathlineto{\pgfqpoint{3.931602in}{3.628480in}}%
\pgfpathlineto{\pgfqpoint{3.931602in}{3.632738in}}%
\pgfpathlineto{\pgfqpoint{3.935860in}{3.632738in}}%
\pgfpathlineto{\pgfqpoint{3.935860in}{3.628480in}}%
\pgfpathmoveto{\pgfqpoint{3.931602in}{3.632738in}}%
\pgfpathlineto{\pgfqpoint{3.931602in}{3.632738in}}%
\pgfpathlineto{\pgfqpoint{3.931602in}{3.636995in}}%
\pgfpathlineto{\pgfqpoint{3.935860in}{3.636995in}}%
\pgfpathlineto{\pgfqpoint{3.935860in}{3.632738in}}%
\pgfpathmoveto{\pgfqpoint{3.931602in}{3.636995in}}%
\pgfpathlineto{\pgfqpoint{3.931602in}{3.636995in}}%
\pgfpathlineto{\pgfqpoint{3.931602in}{3.641253in}}%
\pgfpathlineto{\pgfqpoint{3.935860in}{3.641253in}}%
\pgfpathlineto{\pgfqpoint{3.935860in}{3.636995in}}%
\pgfpathmoveto{\pgfqpoint{3.935860in}{3.624222in}}%
\pgfpathlineto{\pgfqpoint{3.935860in}{3.624222in}}%
\pgfpathlineto{\pgfqpoint{3.935860in}{3.628480in}}%
\pgfpathlineto{\pgfqpoint{3.940118in}{3.628480in}}%
\pgfpathlineto{\pgfqpoint{3.940118in}{3.624222in}}%
\pgfpathmoveto{\pgfqpoint{3.935860in}{3.628480in}}%
\pgfpathlineto{\pgfqpoint{3.935860in}{3.628480in}}%
\pgfpathlineto{\pgfqpoint{3.935860in}{3.632738in}}%
\pgfpathlineto{\pgfqpoint{3.940118in}{3.632738in}}%
\pgfpathlineto{\pgfqpoint{3.940118in}{3.628480in}}%
\pgfpathmoveto{\pgfqpoint{3.935860in}{3.632738in}}%
\pgfpathlineto{\pgfqpoint{3.935860in}{3.632738in}}%
\pgfpathlineto{\pgfqpoint{3.935860in}{3.636995in}}%
\pgfpathlineto{\pgfqpoint{3.940118in}{3.636995in}}%
\pgfpathlineto{\pgfqpoint{3.940118in}{3.632738in}}%
\pgfpathmoveto{\pgfqpoint{3.935860in}{3.636995in}}%
\pgfpathlineto{\pgfqpoint{3.935860in}{3.636995in}}%
\pgfpathlineto{\pgfqpoint{3.935860in}{3.641253in}}%
\pgfpathlineto{\pgfqpoint{3.940118in}{3.641253in}}%
\pgfpathlineto{\pgfqpoint{3.940118in}{3.636995in}}%
\pgfpathmoveto{\pgfqpoint{3.931602in}{3.641253in}}%
\pgfpathlineto{\pgfqpoint{3.931602in}{3.641253in}}%
\pgfpathlineto{\pgfqpoint{3.931602in}{3.645511in}}%
\pgfpathlineto{\pgfqpoint{3.935860in}{3.645511in}}%
\pgfpathlineto{\pgfqpoint{3.935860in}{3.641253in}}%
\pgfpathmoveto{\pgfqpoint{3.935860in}{3.641253in}}%
\pgfpathlineto{\pgfqpoint{3.935860in}{3.641253in}}%
\pgfpathlineto{\pgfqpoint{3.935860in}{3.645511in}}%
\pgfpathlineto{\pgfqpoint{3.940118in}{3.645511in}}%
\pgfpathlineto{\pgfqpoint{3.940118in}{3.641253in}}%
\pgfpathmoveto{\pgfqpoint{3.935860in}{3.645511in}}%
\pgfpathlineto{\pgfqpoint{3.935860in}{3.645511in}}%
\pgfpathlineto{\pgfqpoint{3.935860in}{3.649769in}}%
\pgfpathlineto{\pgfqpoint{3.940118in}{3.649769in}}%
\pgfpathlineto{\pgfqpoint{3.940118in}{3.645511in}}%
\pgfpathmoveto{\pgfqpoint{3.940118in}{3.645511in}}%
\pgfpathlineto{\pgfqpoint{3.940118in}{3.645511in}}%
\pgfpathlineto{\pgfqpoint{3.940118in}{3.649769in}}%
\pgfpathlineto{\pgfqpoint{3.944376in}{3.649769in}}%
\pgfpathlineto{\pgfqpoint{3.944376in}{3.645511in}}%
\pgfpathmoveto{\pgfqpoint{3.935860in}{3.649769in}}%
\pgfpathlineto{\pgfqpoint{3.935860in}{3.649769in}}%
\pgfpathlineto{\pgfqpoint{3.935860in}{3.654027in}}%
\pgfpathlineto{\pgfqpoint{3.940118in}{3.654027in}}%
\pgfpathlineto{\pgfqpoint{3.940118in}{3.649769in}}%
\pgfpathmoveto{\pgfqpoint{3.935860in}{3.654027in}}%
\pgfpathlineto{\pgfqpoint{3.935860in}{3.654027in}}%
\pgfpathlineto{\pgfqpoint{3.935860in}{3.658284in}}%
\pgfpathlineto{\pgfqpoint{3.940118in}{3.658284in}}%
\pgfpathlineto{\pgfqpoint{3.940118in}{3.654027in}}%
\pgfpathmoveto{\pgfqpoint{3.940118in}{3.649769in}}%
\pgfpathlineto{\pgfqpoint{3.940118in}{3.649769in}}%
\pgfpathlineto{\pgfqpoint{3.940118in}{3.654027in}}%
\pgfpathlineto{\pgfqpoint{3.944376in}{3.654027in}}%
\pgfpathlineto{\pgfqpoint{3.944376in}{3.649769in}}%
\pgfpathmoveto{\pgfqpoint{3.940118in}{3.654027in}}%
\pgfpathlineto{\pgfqpoint{3.940118in}{3.654027in}}%
\pgfpathlineto{\pgfqpoint{3.940118in}{3.658284in}}%
\pgfpathlineto{\pgfqpoint{3.944376in}{3.658284in}}%
\pgfpathlineto{\pgfqpoint{3.944376in}{3.654027in}}%
\pgfpathmoveto{\pgfqpoint{3.935860in}{3.658284in}}%
\pgfpathlineto{\pgfqpoint{3.935860in}{3.658284in}}%
\pgfpathlineto{\pgfqpoint{3.935860in}{3.662542in}}%
\pgfpathlineto{\pgfqpoint{3.940118in}{3.662542in}}%
\pgfpathlineto{\pgfqpoint{3.940118in}{3.658284in}}%
\pgfpathmoveto{\pgfqpoint{3.935860in}{3.662542in}}%
\pgfpathlineto{\pgfqpoint{3.935860in}{3.662542in}}%
\pgfpathlineto{\pgfqpoint{3.935860in}{3.666800in}}%
\pgfpathlineto{\pgfqpoint{3.940118in}{3.666800in}}%
\pgfpathlineto{\pgfqpoint{3.940118in}{3.662542in}}%
\pgfpathmoveto{\pgfqpoint{3.940118in}{3.658284in}}%
\pgfpathlineto{\pgfqpoint{3.940118in}{3.658284in}}%
\pgfpathlineto{\pgfqpoint{3.940118in}{3.662542in}}%
\pgfpathlineto{\pgfqpoint{3.944376in}{3.662542in}}%
\pgfpathlineto{\pgfqpoint{3.944376in}{3.658284in}}%
\pgfpathmoveto{\pgfqpoint{3.940118in}{3.662542in}}%
\pgfpathlineto{\pgfqpoint{3.940118in}{3.662542in}}%
\pgfpathlineto{\pgfqpoint{3.940118in}{3.666800in}}%
\pgfpathlineto{\pgfqpoint{3.944376in}{3.666800in}}%
\pgfpathlineto{\pgfqpoint{3.944376in}{3.662542in}}%
\pgfpathmoveto{\pgfqpoint{3.935860in}{3.666800in}}%
\pgfpathlineto{\pgfqpoint{3.935860in}{3.666800in}}%
\pgfpathlineto{\pgfqpoint{3.935860in}{3.671058in}}%
\pgfpathlineto{\pgfqpoint{3.940118in}{3.671058in}}%
\pgfpathlineto{\pgfqpoint{3.940118in}{3.666800in}}%
\pgfpathmoveto{\pgfqpoint{3.935860in}{3.671058in}}%
\pgfpathlineto{\pgfqpoint{3.935860in}{3.671058in}}%
\pgfpathlineto{\pgfqpoint{3.935860in}{3.675316in}}%
\pgfpathlineto{\pgfqpoint{3.940118in}{3.675316in}}%
\pgfpathlineto{\pgfqpoint{3.940118in}{3.671058in}}%
\pgfpathmoveto{\pgfqpoint{3.940118in}{3.666800in}}%
\pgfpathlineto{\pgfqpoint{3.940118in}{3.666800in}}%
\pgfpathlineto{\pgfqpoint{3.940118in}{3.671058in}}%
\pgfpathlineto{\pgfqpoint{3.944376in}{3.671058in}}%
\pgfpathlineto{\pgfqpoint{3.944376in}{3.666800in}}%
\pgfpathmoveto{\pgfqpoint{3.940118in}{3.671058in}}%
\pgfpathlineto{\pgfqpoint{3.940118in}{3.671058in}}%
\pgfpathlineto{\pgfqpoint{3.940118in}{3.675316in}}%
\pgfpathlineto{\pgfqpoint{3.944376in}{3.675316in}}%
\pgfpathlineto{\pgfqpoint{3.944376in}{3.671058in}}%
\pgfpathmoveto{\pgfqpoint{3.935860in}{3.675316in}}%
\pgfpathlineto{\pgfqpoint{3.935860in}{3.675316in}}%
\pgfpathlineto{\pgfqpoint{3.935860in}{3.679573in}}%
\pgfpathlineto{\pgfqpoint{3.940118in}{3.679573in}}%
\pgfpathlineto{\pgfqpoint{3.940118in}{3.675316in}}%
\pgfpathmoveto{\pgfqpoint{3.935860in}{3.679573in}}%
\pgfpathlineto{\pgfqpoint{3.935860in}{3.679573in}}%
\pgfpathlineto{\pgfqpoint{3.935860in}{3.683831in}}%
\pgfpathlineto{\pgfqpoint{3.940118in}{3.683831in}}%
\pgfpathlineto{\pgfqpoint{3.940118in}{3.679573in}}%
\pgfpathmoveto{\pgfqpoint{3.940118in}{3.675316in}}%
\pgfpathlineto{\pgfqpoint{3.940118in}{3.675316in}}%
\pgfpathlineto{\pgfqpoint{3.940118in}{3.679573in}}%
\pgfpathlineto{\pgfqpoint{3.944376in}{3.679573in}}%
\pgfpathlineto{\pgfqpoint{3.944376in}{3.675316in}}%
\pgfpathmoveto{\pgfqpoint{3.940118in}{3.679573in}}%
\pgfpathlineto{\pgfqpoint{3.940118in}{3.679573in}}%
\pgfpathlineto{\pgfqpoint{3.940118in}{3.683831in}}%
\pgfpathlineto{\pgfqpoint{3.944376in}{3.683831in}}%
\pgfpathlineto{\pgfqpoint{3.944376in}{3.679573in}}%
\pgfpathmoveto{\pgfqpoint{3.940118in}{3.683831in}}%
\pgfpathlineto{\pgfqpoint{3.940118in}{3.683831in}}%
\pgfpathlineto{\pgfqpoint{3.940118in}{3.688089in}}%
\pgfpathlineto{\pgfqpoint{3.944376in}{3.688089in}}%
\pgfpathlineto{\pgfqpoint{3.944376in}{3.683831in}}%
\pgfpathmoveto{\pgfqpoint{3.940118in}{3.688089in}}%
\pgfpathlineto{\pgfqpoint{3.940118in}{3.688089in}}%
\pgfpathlineto{\pgfqpoint{3.940118in}{3.692347in}}%
\pgfpathlineto{\pgfqpoint{3.944376in}{3.692347in}}%
\pgfpathlineto{\pgfqpoint{3.944376in}{3.688089in}}%
\pgfpathmoveto{\pgfqpoint{3.940118in}{3.692347in}}%
\pgfpathlineto{\pgfqpoint{3.940118in}{3.692347in}}%
\pgfpathlineto{\pgfqpoint{3.940118in}{3.696604in}}%
\pgfpathlineto{\pgfqpoint{3.944376in}{3.696604in}}%
\pgfpathlineto{\pgfqpoint{3.944376in}{3.692347in}}%
\pgfpathmoveto{\pgfqpoint{3.940118in}{3.696604in}}%
\pgfpathlineto{\pgfqpoint{3.940118in}{3.696604in}}%
\pgfpathlineto{\pgfqpoint{3.940118in}{3.700862in}}%
\pgfpathlineto{\pgfqpoint{3.944376in}{3.700862in}}%
\pgfpathlineto{\pgfqpoint{3.944376in}{3.696604in}}%
\pgfpathmoveto{\pgfqpoint{3.940118in}{3.700862in}}%
\pgfpathlineto{\pgfqpoint{3.940118in}{3.700862in}}%
\pgfpathlineto{\pgfqpoint{3.940118in}{3.705120in}}%
\pgfpathlineto{\pgfqpoint{3.944376in}{3.705120in}}%
\pgfpathlineto{\pgfqpoint{3.944376in}{3.700862in}}%
\pgfpathmoveto{\pgfqpoint{3.940118in}{3.705120in}}%
\pgfpathlineto{\pgfqpoint{3.940118in}{3.705120in}}%
\pgfpathlineto{\pgfqpoint{3.940118in}{3.709378in}}%
\pgfpathlineto{\pgfqpoint{3.944376in}{3.709378in}}%
\pgfpathlineto{\pgfqpoint{3.944376in}{3.705120in}}%
\pgfpathmoveto{\pgfqpoint{3.944376in}{3.683831in}}%
\pgfpathlineto{\pgfqpoint{3.944376in}{3.683831in}}%
\pgfpathlineto{\pgfqpoint{3.944376in}{3.688089in}}%
\pgfpathlineto{\pgfqpoint{3.948634in}{3.688089in}}%
\pgfpathlineto{\pgfqpoint{3.948634in}{3.683831in}}%
\pgfpathmoveto{\pgfqpoint{3.944376in}{3.688089in}}%
\pgfpathlineto{\pgfqpoint{3.944376in}{3.688089in}}%
\pgfpathlineto{\pgfqpoint{3.944376in}{3.692347in}}%
\pgfpathlineto{\pgfqpoint{3.948634in}{3.692347in}}%
\pgfpathlineto{\pgfqpoint{3.948634in}{3.688089in}}%
\pgfpathmoveto{\pgfqpoint{3.944376in}{3.692347in}}%
\pgfpathlineto{\pgfqpoint{3.944376in}{3.692347in}}%
\pgfpathlineto{\pgfqpoint{3.944376in}{3.696604in}}%
\pgfpathlineto{\pgfqpoint{3.948634in}{3.696604in}}%
\pgfpathlineto{\pgfqpoint{3.948634in}{3.692347in}}%
\pgfpathmoveto{\pgfqpoint{3.944376in}{3.696604in}}%
\pgfpathlineto{\pgfqpoint{3.944376in}{3.696604in}}%
\pgfpathlineto{\pgfqpoint{3.944376in}{3.700862in}}%
\pgfpathlineto{\pgfqpoint{3.948634in}{3.700862in}}%
\pgfpathlineto{\pgfqpoint{3.948634in}{3.696604in}}%
\pgfpathmoveto{\pgfqpoint{3.944376in}{3.700862in}}%
\pgfpathlineto{\pgfqpoint{3.944376in}{3.700862in}}%
\pgfpathlineto{\pgfqpoint{3.944376in}{3.705120in}}%
\pgfpathlineto{\pgfqpoint{3.948634in}{3.705120in}}%
\pgfpathlineto{\pgfqpoint{3.948634in}{3.700862in}}%
\pgfpathmoveto{\pgfqpoint{3.944376in}{3.705120in}}%
\pgfpathlineto{\pgfqpoint{3.944376in}{3.705120in}}%
\pgfpathlineto{\pgfqpoint{3.944376in}{3.709378in}}%
\pgfpathlineto{\pgfqpoint{3.948634in}{3.709378in}}%
\pgfpathlineto{\pgfqpoint{3.948634in}{3.705120in}}%
\pgfpathmoveto{\pgfqpoint{3.940118in}{3.709378in}}%
\pgfpathlineto{\pgfqpoint{3.940118in}{3.709378in}}%
\pgfpathlineto{\pgfqpoint{3.940118in}{3.713636in}}%
\pgfpathlineto{\pgfqpoint{3.944376in}{3.713636in}}%
\pgfpathlineto{\pgfqpoint{3.944376in}{3.709378in}}%
\pgfpathmoveto{\pgfqpoint{3.940118in}{3.713636in}}%
\pgfpathlineto{\pgfqpoint{3.940118in}{3.713636in}}%
\pgfpathlineto{\pgfqpoint{3.940118in}{3.717893in}}%
\pgfpathlineto{\pgfqpoint{3.944376in}{3.717893in}}%
\pgfpathlineto{\pgfqpoint{3.944376in}{3.713636in}}%
\pgfpathmoveto{\pgfqpoint{3.940118in}{3.717893in}}%
\pgfpathlineto{\pgfqpoint{3.940118in}{3.717893in}}%
\pgfpathlineto{\pgfqpoint{3.940118in}{3.722151in}}%
\pgfpathlineto{\pgfqpoint{3.944376in}{3.722151in}}%
\pgfpathlineto{\pgfqpoint{3.944376in}{3.717893in}}%
\pgfpathmoveto{\pgfqpoint{3.944376in}{3.709378in}}%
\pgfpathlineto{\pgfqpoint{3.944376in}{3.709378in}}%
\pgfpathlineto{\pgfqpoint{3.944376in}{3.713636in}}%
\pgfpathlineto{\pgfqpoint{3.948634in}{3.713636in}}%
\pgfpathlineto{\pgfqpoint{3.948634in}{3.709378in}}%
\pgfpathmoveto{\pgfqpoint{3.944376in}{3.713636in}}%
\pgfpathlineto{\pgfqpoint{3.944376in}{3.713636in}}%
\pgfpathlineto{\pgfqpoint{3.944376in}{3.717893in}}%
\pgfpathlineto{\pgfqpoint{3.948634in}{3.717893in}}%
\pgfpathlineto{\pgfqpoint{3.948634in}{3.713636in}}%
\pgfpathmoveto{\pgfqpoint{3.944376in}{3.717893in}}%
\pgfpathlineto{\pgfqpoint{3.944376in}{3.717893in}}%
\pgfpathlineto{\pgfqpoint{3.944376in}{3.722151in}}%
\pgfpathlineto{\pgfqpoint{3.948634in}{3.722151in}}%
\pgfpathlineto{\pgfqpoint{3.948634in}{3.717893in}}%
\pgfpathmoveto{\pgfqpoint{3.944376in}{3.722151in}}%
\pgfpathlineto{\pgfqpoint{3.944376in}{3.722151in}}%
\pgfpathlineto{\pgfqpoint{3.944376in}{3.726409in}}%
\pgfpathlineto{\pgfqpoint{3.948634in}{3.726409in}}%
\pgfpathlineto{\pgfqpoint{3.948634in}{3.722151in}}%
\pgfpathmoveto{\pgfqpoint{3.948634in}{3.722151in}}%
\pgfpathlineto{\pgfqpoint{3.948634in}{3.722151in}}%
\pgfpathlineto{\pgfqpoint{3.948634in}{3.726409in}}%
\pgfpathlineto{\pgfqpoint{3.952892in}{3.726409in}}%
\pgfpathlineto{\pgfqpoint{3.952892in}{3.722151in}}%
\pgfpathmoveto{\pgfqpoint{3.944376in}{3.726409in}}%
\pgfpathlineto{\pgfqpoint{3.944376in}{3.726409in}}%
\pgfpathlineto{\pgfqpoint{3.944376in}{3.730667in}}%
\pgfpathlineto{\pgfqpoint{3.948634in}{3.730667in}}%
\pgfpathlineto{\pgfqpoint{3.948634in}{3.726409in}}%
\pgfpathmoveto{\pgfqpoint{3.944376in}{3.730667in}}%
\pgfpathlineto{\pgfqpoint{3.944376in}{3.730667in}}%
\pgfpathlineto{\pgfqpoint{3.944376in}{3.734924in}}%
\pgfpathlineto{\pgfqpoint{3.948634in}{3.734924in}}%
\pgfpathlineto{\pgfqpoint{3.948634in}{3.730667in}}%
\pgfpathmoveto{\pgfqpoint{3.948634in}{3.726409in}}%
\pgfpathlineto{\pgfqpoint{3.948634in}{3.726409in}}%
\pgfpathlineto{\pgfqpoint{3.948634in}{3.730667in}}%
\pgfpathlineto{\pgfqpoint{3.952892in}{3.730667in}}%
\pgfpathlineto{\pgfqpoint{3.952892in}{3.726409in}}%
\pgfpathmoveto{\pgfqpoint{3.948634in}{3.730667in}}%
\pgfpathlineto{\pgfqpoint{3.948634in}{3.730667in}}%
\pgfpathlineto{\pgfqpoint{3.948634in}{3.734924in}}%
\pgfpathlineto{\pgfqpoint{3.952892in}{3.734924in}}%
\pgfpathlineto{\pgfqpoint{3.952892in}{3.730667in}}%
\pgfpathmoveto{\pgfqpoint{3.944376in}{3.734924in}}%
\pgfpathlineto{\pgfqpoint{3.944376in}{3.734924in}}%
\pgfpathlineto{\pgfqpoint{3.944376in}{3.739182in}}%
\pgfpathlineto{\pgfqpoint{3.948634in}{3.739182in}}%
\pgfpathlineto{\pgfqpoint{3.948634in}{3.734924in}}%
\pgfpathmoveto{\pgfqpoint{3.944376in}{3.739182in}}%
\pgfpathlineto{\pgfqpoint{3.944376in}{3.739182in}}%
\pgfpathlineto{\pgfqpoint{3.944376in}{3.743440in}}%
\pgfpathlineto{\pgfqpoint{3.948634in}{3.743440in}}%
\pgfpathlineto{\pgfqpoint{3.948634in}{3.739182in}}%
\pgfpathmoveto{\pgfqpoint{3.948634in}{3.734924in}}%
\pgfpathlineto{\pgfqpoint{3.948634in}{3.734924in}}%
\pgfpathlineto{\pgfqpoint{3.948634in}{3.739182in}}%
\pgfpathlineto{\pgfqpoint{3.952892in}{3.739182in}}%
\pgfpathlineto{\pgfqpoint{3.952892in}{3.734924in}}%
\pgfpathmoveto{\pgfqpoint{3.948634in}{3.739182in}}%
\pgfpathlineto{\pgfqpoint{3.948634in}{3.739182in}}%
\pgfpathlineto{\pgfqpoint{3.948634in}{3.743440in}}%
\pgfpathlineto{\pgfqpoint{3.952892in}{3.743440in}}%
\pgfpathlineto{\pgfqpoint{3.952892in}{3.739182in}}%
\pgfpathmoveto{\pgfqpoint{3.944376in}{3.743440in}}%
\pgfpathlineto{\pgfqpoint{3.944376in}{3.743440in}}%
\pgfpathlineto{\pgfqpoint{3.944376in}{3.747698in}}%
\pgfpathlineto{\pgfqpoint{3.948634in}{3.747698in}}%
\pgfpathlineto{\pgfqpoint{3.948634in}{3.743440in}}%
\pgfpathmoveto{\pgfqpoint{3.944376in}{3.747698in}}%
\pgfpathlineto{\pgfqpoint{3.944376in}{3.747698in}}%
\pgfpathlineto{\pgfqpoint{3.944376in}{3.751955in}}%
\pgfpathlineto{\pgfqpoint{3.948634in}{3.751955in}}%
\pgfpathlineto{\pgfqpoint{3.948634in}{3.747698in}}%
\pgfpathmoveto{\pgfqpoint{3.948634in}{3.743440in}}%
\pgfpathlineto{\pgfqpoint{3.948634in}{3.743440in}}%
\pgfpathlineto{\pgfqpoint{3.948634in}{3.747698in}}%
\pgfpathlineto{\pgfqpoint{3.952892in}{3.747698in}}%
\pgfpathlineto{\pgfqpoint{3.952892in}{3.743440in}}%
\pgfpathmoveto{\pgfqpoint{3.948634in}{3.747698in}}%
\pgfpathlineto{\pgfqpoint{3.948634in}{3.747698in}}%
\pgfpathlineto{\pgfqpoint{3.948634in}{3.751955in}}%
\pgfpathlineto{\pgfqpoint{3.952892in}{3.751955in}}%
\pgfpathlineto{\pgfqpoint{3.952892in}{3.747698in}}%
\pgfpathmoveto{\pgfqpoint{3.944376in}{3.751955in}}%
\pgfpathlineto{\pgfqpoint{3.944376in}{3.751955in}}%
\pgfpathlineto{\pgfqpoint{3.944376in}{3.756213in}}%
\pgfpathlineto{\pgfqpoint{3.948634in}{3.756213in}}%
\pgfpathlineto{\pgfqpoint{3.948634in}{3.751955in}}%
\pgfpathmoveto{\pgfqpoint{3.944376in}{3.756213in}}%
\pgfpathlineto{\pgfqpoint{3.944376in}{3.756213in}}%
\pgfpathlineto{\pgfqpoint{3.944376in}{3.760471in}}%
\pgfpathlineto{\pgfqpoint{3.948634in}{3.760471in}}%
\pgfpathlineto{\pgfqpoint{3.948634in}{3.756213in}}%
\pgfpathmoveto{\pgfqpoint{3.948634in}{3.751955in}}%
\pgfpathlineto{\pgfqpoint{3.948634in}{3.751955in}}%
\pgfpathlineto{\pgfqpoint{3.948634in}{3.756213in}}%
\pgfpathlineto{\pgfqpoint{3.952892in}{3.756213in}}%
\pgfpathlineto{\pgfqpoint{3.952892in}{3.751955in}}%
\pgfpathmoveto{\pgfqpoint{3.948634in}{3.756213in}}%
\pgfpathlineto{\pgfqpoint{3.948634in}{3.756213in}}%
\pgfpathlineto{\pgfqpoint{3.948634in}{3.760471in}}%
\pgfpathlineto{\pgfqpoint{3.952892in}{3.760471in}}%
\pgfpathlineto{\pgfqpoint{3.952892in}{3.756213in}}%
\pgfpathmoveto{\pgfqpoint{3.948634in}{3.760471in}}%
\pgfpathlineto{\pgfqpoint{3.948634in}{3.760471in}}%
\pgfpathlineto{\pgfqpoint{3.948634in}{3.764729in}}%
\pgfpathlineto{\pgfqpoint{3.952892in}{3.764729in}}%
\pgfpathlineto{\pgfqpoint{3.952892in}{3.760471in}}%
\pgfpathmoveto{\pgfqpoint{3.948634in}{3.764729in}}%
\pgfpathlineto{\pgfqpoint{3.948634in}{3.764729in}}%
\pgfpathlineto{\pgfqpoint{3.948634in}{3.768987in}}%
\pgfpathlineto{\pgfqpoint{3.952892in}{3.768987in}}%
\pgfpathlineto{\pgfqpoint{3.952892in}{3.764729in}}%
\pgfpathmoveto{\pgfqpoint{3.948634in}{3.768987in}}%
\pgfpathlineto{\pgfqpoint{3.948634in}{3.768987in}}%
\pgfpathlineto{\pgfqpoint{3.948634in}{3.773244in}}%
\pgfpathlineto{\pgfqpoint{3.952892in}{3.773244in}}%
\pgfpathlineto{\pgfqpoint{3.952892in}{3.768987in}}%
\pgfpathmoveto{\pgfqpoint{3.948634in}{3.773244in}}%
\pgfpathlineto{\pgfqpoint{3.948634in}{3.773244in}}%
\pgfpathlineto{\pgfqpoint{3.948634in}{3.777502in}}%
\pgfpathlineto{\pgfqpoint{3.952892in}{3.777502in}}%
\pgfpathlineto{\pgfqpoint{3.952892in}{3.773244in}}%
\pgfpathmoveto{\pgfqpoint{3.952892in}{3.760471in}}%
\pgfpathlineto{\pgfqpoint{3.952892in}{3.760471in}}%
\pgfpathlineto{\pgfqpoint{3.952892in}{3.764729in}}%
\pgfpathlineto{\pgfqpoint{3.957149in}{3.764729in}}%
\pgfpathlineto{\pgfqpoint{3.957149in}{3.760471in}}%
\pgfpathmoveto{\pgfqpoint{3.952892in}{3.764729in}}%
\pgfpathlineto{\pgfqpoint{3.952892in}{3.764729in}}%
\pgfpathlineto{\pgfqpoint{3.952892in}{3.768987in}}%
\pgfpathlineto{\pgfqpoint{3.957149in}{3.768987in}}%
\pgfpathlineto{\pgfqpoint{3.957149in}{3.764729in}}%
\pgfpathmoveto{\pgfqpoint{3.952892in}{3.768987in}}%
\pgfpathlineto{\pgfqpoint{3.952892in}{3.768987in}}%
\pgfpathlineto{\pgfqpoint{3.952892in}{3.773244in}}%
\pgfpathlineto{\pgfqpoint{3.957149in}{3.773244in}}%
\pgfpathlineto{\pgfqpoint{3.957149in}{3.768987in}}%
\pgfpathmoveto{\pgfqpoint{3.952892in}{3.773244in}}%
\pgfpathlineto{\pgfqpoint{3.952892in}{3.773244in}}%
\pgfpathlineto{\pgfqpoint{3.952892in}{3.777502in}}%
\pgfpathlineto{\pgfqpoint{3.957149in}{3.777502in}}%
\pgfpathlineto{\pgfqpoint{3.957149in}{3.773244in}}%
\pgfpathmoveto{\pgfqpoint{3.948634in}{3.777502in}}%
\pgfpathlineto{\pgfqpoint{3.948634in}{3.777502in}}%
\pgfpathlineto{\pgfqpoint{3.948634in}{3.781760in}}%
\pgfpathlineto{\pgfqpoint{3.952892in}{3.781760in}}%
\pgfpathlineto{\pgfqpoint{3.952892in}{3.777502in}}%
\pgfpathmoveto{\pgfqpoint{3.948634in}{3.781760in}}%
\pgfpathlineto{\pgfqpoint{3.948634in}{3.781760in}}%
\pgfpathlineto{\pgfqpoint{3.948634in}{3.786018in}}%
\pgfpathlineto{\pgfqpoint{3.952892in}{3.786018in}}%
\pgfpathlineto{\pgfqpoint{3.952892in}{3.781760in}}%
\pgfpathmoveto{\pgfqpoint{3.948634in}{3.786018in}}%
\pgfpathlineto{\pgfqpoint{3.948634in}{3.786018in}}%
\pgfpathlineto{\pgfqpoint{3.948634in}{3.790275in}}%
\pgfpathlineto{\pgfqpoint{3.952892in}{3.790275in}}%
\pgfpathlineto{\pgfqpoint{3.952892in}{3.786018in}}%
\pgfpathmoveto{\pgfqpoint{3.948634in}{3.790275in}}%
\pgfpathlineto{\pgfqpoint{3.948634in}{3.790275in}}%
\pgfpathlineto{\pgfqpoint{3.948634in}{3.794533in}}%
\pgfpathlineto{\pgfqpoint{3.952892in}{3.794533in}}%
\pgfpathlineto{\pgfqpoint{3.952892in}{3.790275in}}%
\pgfpathmoveto{\pgfqpoint{3.952892in}{3.777502in}}%
\pgfpathlineto{\pgfqpoint{3.952892in}{3.777502in}}%
\pgfpathlineto{\pgfqpoint{3.952892in}{3.781760in}}%
\pgfpathlineto{\pgfqpoint{3.957149in}{3.781760in}}%
\pgfpathlineto{\pgfqpoint{3.957149in}{3.777502in}}%
\pgfpathmoveto{\pgfqpoint{3.952892in}{3.781760in}}%
\pgfpathlineto{\pgfqpoint{3.952892in}{3.781760in}}%
\pgfpathlineto{\pgfqpoint{3.952892in}{3.786018in}}%
\pgfpathlineto{\pgfqpoint{3.957149in}{3.786018in}}%
\pgfpathlineto{\pgfqpoint{3.957149in}{3.781760in}}%
\pgfpathmoveto{\pgfqpoint{3.952892in}{3.786018in}}%
\pgfpathlineto{\pgfqpoint{3.952892in}{3.786018in}}%
\pgfpathlineto{\pgfqpoint{3.952892in}{3.790275in}}%
\pgfpathlineto{\pgfqpoint{3.957149in}{3.790275in}}%
\pgfpathlineto{\pgfqpoint{3.957149in}{3.786018in}}%
\pgfpathmoveto{\pgfqpoint{3.952892in}{3.790275in}}%
\pgfpathlineto{\pgfqpoint{3.952892in}{3.790275in}}%
\pgfpathlineto{\pgfqpoint{3.952892in}{3.794533in}}%
\pgfpathlineto{\pgfqpoint{3.957149in}{3.794533in}}%
\pgfpathlineto{\pgfqpoint{3.957149in}{3.790275in}}%
\pgfpathmoveto{\pgfqpoint{3.948634in}{3.794533in}}%
\pgfpathlineto{\pgfqpoint{3.948634in}{3.794533in}}%
\pgfpathlineto{\pgfqpoint{3.948634in}{3.798791in}}%
\pgfpathlineto{\pgfqpoint{3.952892in}{3.798791in}}%
\pgfpathlineto{\pgfqpoint{3.952892in}{3.794533in}}%
\pgfpathmoveto{\pgfqpoint{3.952892in}{3.794533in}}%
\pgfpathlineto{\pgfqpoint{3.952892in}{3.794533in}}%
\pgfpathlineto{\pgfqpoint{3.952892in}{3.798791in}}%
\pgfpathlineto{\pgfqpoint{3.957149in}{3.798791in}}%
\pgfpathlineto{\pgfqpoint{3.957149in}{3.794533in}}%
\pgfpathmoveto{\pgfqpoint{3.952892in}{3.798791in}}%
\pgfpathlineto{\pgfqpoint{3.952892in}{3.798791in}}%
\pgfpathlineto{\pgfqpoint{3.952892in}{3.803049in}}%
\pgfpathlineto{\pgfqpoint{3.957149in}{3.803049in}}%
\pgfpathlineto{\pgfqpoint{3.957149in}{3.798791in}}%
\pgfpathmoveto{\pgfqpoint{3.952892in}{3.803049in}}%
\pgfpathlineto{\pgfqpoint{3.952892in}{3.803049in}}%
\pgfpathlineto{\pgfqpoint{3.952892in}{3.807306in}}%
\pgfpathlineto{\pgfqpoint{3.957149in}{3.807306in}}%
\pgfpathlineto{\pgfqpoint{3.957149in}{3.803049in}}%
\pgfpathmoveto{\pgfqpoint{3.952892in}{3.807306in}}%
\pgfpathlineto{\pgfqpoint{3.952892in}{3.807306in}}%
\pgfpathlineto{\pgfqpoint{3.952892in}{3.811564in}}%
\pgfpathlineto{\pgfqpoint{3.957149in}{3.811564in}}%
\pgfpathlineto{\pgfqpoint{3.957149in}{3.807306in}}%
\pgfpathmoveto{\pgfqpoint{3.957149in}{3.803049in}}%
\pgfpathlineto{\pgfqpoint{3.957149in}{3.803049in}}%
\pgfpathlineto{\pgfqpoint{3.957149in}{3.807306in}}%
\pgfpathlineto{\pgfqpoint{3.961407in}{3.807306in}}%
\pgfpathlineto{\pgfqpoint{3.961407in}{3.803049in}}%
\pgfpathmoveto{\pgfqpoint{3.957149in}{3.807306in}}%
\pgfpathlineto{\pgfqpoint{3.957149in}{3.807306in}}%
\pgfpathlineto{\pgfqpoint{3.957149in}{3.811564in}}%
\pgfpathlineto{\pgfqpoint{3.961407in}{3.811564in}}%
\pgfpathlineto{\pgfqpoint{3.961407in}{3.807306in}}%
\pgfpathmoveto{\pgfqpoint{3.952892in}{3.811564in}}%
\pgfpathlineto{\pgfqpoint{3.952892in}{3.811564in}}%
\pgfpathlineto{\pgfqpoint{3.952892in}{3.815822in}}%
\pgfpathlineto{\pgfqpoint{3.957149in}{3.815822in}}%
\pgfpathlineto{\pgfqpoint{3.957149in}{3.811564in}}%
\pgfpathmoveto{\pgfqpoint{3.952892in}{3.815822in}}%
\pgfpathlineto{\pgfqpoint{3.952892in}{3.815822in}}%
\pgfpathlineto{\pgfqpoint{3.952892in}{3.820080in}}%
\pgfpathlineto{\pgfqpoint{3.957149in}{3.820080in}}%
\pgfpathlineto{\pgfqpoint{3.957149in}{3.815822in}}%
\pgfpathmoveto{\pgfqpoint{3.957149in}{3.811564in}}%
\pgfpathlineto{\pgfqpoint{3.957149in}{3.811564in}}%
\pgfpathlineto{\pgfqpoint{3.957149in}{3.815822in}}%
\pgfpathlineto{\pgfqpoint{3.961407in}{3.815822in}}%
\pgfpathlineto{\pgfqpoint{3.961407in}{3.811564in}}%
\pgfpathmoveto{\pgfqpoint{3.957149in}{3.815822in}}%
\pgfpathlineto{\pgfqpoint{3.957149in}{3.815822in}}%
\pgfpathlineto{\pgfqpoint{3.957149in}{3.820080in}}%
\pgfpathlineto{\pgfqpoint{3.961407in}{3.820080in}}%
\pgfpathlineto{\pgfqpoint{3.961407in}{3.815822in}}%
\pgfpathmoveto{\pgfqpoint{3.952892in}{3.820080in}}%
\pgfpathlineto{\pgfqpoint{3.952892in}{3.820080in}}%
\pgfpathlineto{\pgfqpoint{3.952892in}{3.824337in}}%
\pgfpathlineto{\pgfqpoint{3.957149in}{3.824337in}}%
\pgfpathlineto{\pgfqpoint{3.957149in}{3.820080in}}%
\pgfpathmoveto{\pgfqpoint{3.952892in}{3.824337in}}%
\pgfpathlineto{\pgfqpoint{3.952892in}{3.824337in}}%
\pgfpathlineto{\pgfqpoint{3.952892in}{3.828595in}}%
\pgfpathlineto{\pgfqpoint{3.957149in}{3.828595in}}%
\pgfpathlineto{\pgfqpoint{3.957149in}{3.824337in}}%
\pgfpathmoveto{\pgfqpoint{3.957149in}{3.820080in}}%
\pgfpathlineto{\pgfqpoint{3.957149in}{3.820080in}}%
\pgfpathlineto{\pgfqpoint{3.957149in}{3.824337in}}%
\pgfpathlineto{\pgfqpoint{3.961407in}{3.824337in}}%
\pgfpathlineto{\pgfqpoint{3.961407in}{3.820080in}}%
\pgfpathmoveto{\pgfqpoint{3.957149in}{3.824337in}}%
\pgfpathlineto{\pgfqpoint{3.957149in}{3.824337in}}%
\pgfpathlineto{\pgfqpoint{3.957149in}{3.828595in}}%
\pgfpathlineto{\pgfqpoint{3.961407in}{3.828595in}}%
\pgfpathlineto{\pgfqpoint{3.961407in}{3.824337in}}%
\pgfpathmoveto{\pgfqpoint{3.952892in}{3.828595in}}%
\pgfpathlineto{\pgfqpoint{3.952892in}{3.828595in}}%
\pgfpathlineto{\pgfqpoint{3.952892in}{3.832853in}}%
\pgfpathlineto{\pgfqpoint{3.957149in}{3.832853in}}%
\pgfpathlineto{\pgfqpoint{3.957149in}{3.828595in}}%
\pgfpathmoveto{\pgfqpoint{3.952892in}{3.832853in}}%
\pgfpathlineto{\pgfqpoint{3.952892in}{3.832853in}}%
\pgfpathlineto{\pgfqpoint{3.952892in}{3.837111in}}%
\pgfpathlineto{\pgfqpoint{3.957149in}{3.837111in}}%
\pgfpathlineto{\pgfqpoint{3.957149in}{3.832853in}}%
\pgfpathmoveto{\pgfqpoint{3.957149in}{3.828595in}}%
\pgfpathlineto{\pgfqpoint{3.957149in}{3.828595in}}%
\pgfpathlineto{\pgfqpoint{3.957149in}{3.832853in}}%
\pgfpathlineto{\pgfqpoint{3.961407in}{3.832853in}}%
\pgfpathlineto{\pgfqpoint{3.961407in}{3.828595in}}%
\pgfpathmoveto{\pgfqpoint{3.957149in}{3.832853in}}%
\pgfpathlineto{\pgfqpoint{3.957149in}{3.832853in}}%
\pgfpathlineto{\pgfqpoint{3.957149in}{3.837111in}}%
\pgfpathlineto{\pgfqpoint{3.961407in}{3.837111in}}%
\pgfpathlineto{\pgfqpoint{3.961407in}{3.832853in}}%
\pgfpathmoveto{\pgfqpoint{3.952892in}{3.837111in}}%
\pgfpathlineto{\pgfqpoint{3.952892in}{3.837111in}}%
\pgfpathlineto{\pgfqpoint{3.952892in}{3.841368in}}%
\pgfpathlineto{\pgfqpoint{3.957149in}{3.841368in}}%
\pgfpathlineto{\pgfqpoint{3.957149in}{3.837111in}}%
\pgfpathmoveto{\pgfqpoint{3.957149in}{3.837111in}}%
\pgfpathlineto{\pgfqpoint{3.957149in}{3.837111in}}%
\pgfpathlineto{\pgfqpoint{3.957149in}{3.841368in}}%
\pgfpathlineto{\pgfqpoint{3.961407in}{3.841368in}}%
\pgfpathlineto{\pgfqpoint{3.961407in}{3.837111in}}%
\pgfpathmoveto{\pgfqpoint{3.957149in}{3.841368in}}%
\pgfpathlineto{\pgfqpoint{3.957149in}{3.841368in}}%
\pgfpathlineto{\pgfqpoint{3.957149in}{3.845626in}}%
\pgfpathlineto{\pgfqpoint{3.961407in}{3.845626in}}%
\pgfpathlineto{\pgfqpoint{3.961407in}{3.841368in}}%
\pgfpathmoveto{\pgfqpoint{3.961407in}{3.841368in}}%
\pgfpathlineto{\pgfqpoint{3.961407in}{3.841368in}}%
\pgfpathlineto{\pgfqpoint{3.961407in}{3.845626in}}%
\pgfpathlineto{\pgfqpoint{3.965665in}{3.845626in}}%
\pgfpathlineto{\pgfqpoint{3.965665in}{3.841368in}}%
\pgfpathmoveto{\pgfqpoint{3.957149in}{3.845626in}}%
\pgfpathlineto{\pgfqpoint{3.957149in}{3.845626in}}%
\pgfpathlineto{\pgfqpoint{3.957149in}{3.849884in}}%
\pgfpathlineto{\pgfqpoint{3.961407in}{3.849884in}}%
\pgfpathlineto{\pgfqpoint{3.961407in}{3.845626in}}%
\pgfpathmoveto{\pgfqpoint{3.957149in}{3.849884in}}%
\pgfpathlineto{\pgfqpoint{3.957149in}{3.849884in}}%
\pgfpathlineto{\pgfqpoint{3.957149in}{3.854142in}}%
\pgfpathlineto{\pgfqpoint{3.961407in}{3.854142in}}%
\pgfpathlineto{\pgfqpoint{3.961407in}{3.849884in}}%
\pgfpathmoveto{\pgfqpoint{3.957149in}{3.854142in}}%
\pgfpathlineto{\pgfqpoint{3.957149in}{3.854142in}}%
\pgfpathlineto{\pgfqpoint{3.957149in}{3.858399in}}%
\pgfpathlineto{\pgfqpoint{3.961407in}{3.858399in}}%
\pgfpathlineto{\pgfqpoint{3.961407in}{3.854142in}}%
\pgfpathmoveto{\pgfqpoint{3.957149in}{3.858399in}}%
\pgfpathlineto{\pgfqpoint{3.957149in}{3.858399in}}%
\pgfpathlineto{\pgfqpoint{3.957149in}{3.862657in}}%
\pgfpathlineto{\pgfqpoint{3.961407in}{3.862657in}}%
\pgfpathlineto{\pgfqpoint{3.961407in}{3.858399in}}%
\pgfpathmoveto{\pgfqpoint{3.957149in}{3.862657in}}%
\pgfpathlineto{\pgfqpoint{3.957149in}{3.862657in}}%
\pgfpathlineto{\pgfqpoint{3.957149in}{3.866915in}}%
\pgfpathlineto{\pgfqpoint{3.961407in}{3.866915in}}%
\pgfpathlineto{\pgfqpoint{3.961407in}{3.862657in}}%
\pgfpathmoveto{\pgfqpoint{3.957149in}{3.866915in}}%
\pgfpathlineto{\pgfqpoint{3.957149in}{3.866915in}}%
\pgfpathlineto{\pgfqpoint{3.957149in}{3.871173in}}%
\pgfpathlineto{\pgfqpoint{3.961407in}{3.871173in}}%
\pgfpathlineto{\pgfqpoint{3.961407in}{3.866915in}}%
\pgfpathmoveto{\pgfqpoint{3.957149in}{3.871173in}}%
\pgfpathlineto{\pgfqpoint{3.957149in}{3.871173in}}%
\pgfpathlineto{\pgfqpoint{3.957149in}{3.875430in}}%
\pgfpathlineto{\pgfqpoint{3.961407in}{3.875430in}}%
\pgfpathlineto{\pgfqpoint{3.961407in}{3.871173in}}%
\pgfpathmoveto{\pgfqpoint{3.957149in}{3.875430in}}%
\pgfpathlineto{\pgfqpoint{3.957149in}{3.875430in}}%
\pgfpathlineto{\pgfqpoint{3.957149in}{3.879688in}}%
\pgfpathlineto{\pgfqpoint{3.961407in}{3.879688in}}%
\pgfpathlineto{\pgfqpoint{3.961407in}{3.875430in}}%
\pgfpathmoveto{\pgfqpoint{3.961407in}{3.845626in}}%
\pgfpathlineto{\pgfqpoint{3.961407in}{3.845626in}}%
\pgfpathlineto{\pgfqpoint{3.961407in}{3.849884in}}%
\pgfpathlineto{\pgfqpoint{3.965665in}{3.849884in}}%
\pgfpathlineto{\pgfqpoint{3.965665in}{3.845626in}}%
\pgfpathmoveto{\pgfqpoint{3.961407in}{3.849884in}}%
\pgfpathlineto{\pgfqpoint{3.961407in}{3.849884in}}%
\pgfpathlineto{\pgfqpoint{3.961407in}{3.854142in}}%
\pgfpathlineto{\pgfqpoint{3.965665in}{3.854142in}}%
\pgfpathlineto{\pgfqpoint{3.965665in}{3.849884in}}%
\pgfpathmoveto{\pgfqpoint{3.961407in}{3.854142in}}%
\pgfpathlineto{\pgfqpoint{3.961407in}{3.854142in}}%
\pgfpathlineto{\pgfqpoint{3.961407in}{3.858399in}}%
\pgfpathlineto{\pgfqpoint{3.965665in}{3.858399in}}%
\pgfpathlineto{\pgfqpoint{3.965665in}{3.854142in}}%
\pgfpathmoveto{\pgfqpoint{3.961407in}{3.858399in}}%
\pgfpathlineto{\pgfqpoint{3.961407in}{3.858399in}}%
\pgfpathlineto{\pgfqpoint{3.961407in}{3.862657in}}%
\pgfpathlineto{\pgfqpoint{3.965665in}{3.862657in}}%
\pgfpathlineto{\pgfqpoint{3.965665in}{3.858399in}}%
\pgfpathmoveto{\pgfqpoint{3.961407in}{3.862657in}}%
\pgfpathlineto{\pgfqpoint{3.961407in}{3.862657in}}%
\pgfpathlineto{\pgfqpoint{3.961407in}{3.866915in}}%
\pgfpathlineto{\pgfqpoint{3.965665in}{3.866915in}}%
\pgfpathlineto{\pgfqpoint{3.965665in}{3.862657in}}%
\pgfpathmoveto{\pgfqpoint{3.961407in}{3.866915in}}%
\pgfpathlineto{\pgfqpoint{3.961407in}{3.866915in}}%
\pgfpathlineto{\pgfqpoint{3.961407in}{3.871173in}}%
\pgfpathlineto{\pgfqpoint{3.965665in}{3.871173in}}%
\pgfpathlineto{\pgfqpoint{3.965665in}{3.866915in}}%
\pgfpathmoveto{\pgfqpoint{3.961407in}{3.871173in}}%
\pgfpathlineto{\pgfqpoint{3.961407in}{3.871173in}}%
\pgfpathlineto{\pgfqpoint{3.961407in}{3.875430in}}%
\pgfpathlineto{\pgfqpoint{3.965665in}{3.875430in}}%
\pgfpathlineto{\pgfqpoint{3.965665in}{3.871173in}}%
\pgfpathmoveto{\pgfqpoint{3.961407in}{3.875430in}}%
\pgfpathlineto{\pgfqpoint{3.961407in}{3.875430in}}%
\pgfpathlineto{\pgfqpoint{3.961407in}{3.879688in}}%
\pgfpathlineto{\pgfqpoint{3.965665in}{3.879688in}}%
\pgfpathlineto{\pgfqpoint{3.965665in}{3.875430in}}%
\pgfpathmoveto{\pgfqpoint{3.961407in}{3.879688in}}%
\pgfpathlineto{\pgfqpoint{3.961407in}{3.879688in}}%
\pgfpathlineto{\pgfqpoint{3.961407in}{3.883946in}}%
\pgfpathlineto{\pgfqpoint{3.965665in}{3.883946in}}%
\pgfpathlineto{\pgfqpoint{3.965665in}{3.879688in}}%
\pgfpathmoveto{\pgfqpoint{3.961407in}{3.883946in}}%
\pgfpathlineto{\pgfqpoint{3.961407in}{3.883946in}}%
\pgfpathlineto{\pgfqpoint{3.961407in}{3.888204in}}%
\pgfpathlineto{\pgfqpoint{3.965665in}{3.888204in}}%
\pgfpathlineto{\pgfqpoint{3.965665in}{3.883946in}}%
\pgfpathmoveto{\pgfqpoint{3.965665in}{3.883946in}}%
\pgfpathlineto{\pgfqpoint{3.965665in}{3.883946in}}%
\pgfpathlineto{\pgfqpoint{3.965665in}{3.888204in}}%
\pgfpathlineto{\pgfqpoint{3.969923in}{3.888204in}}%
\pgfpathlineto{\pgfqpoint{3.969923in}{3.883946in}}%
\pgfpathmoveto{\pgfqpoint{3.961407in}{3.888204in}}%
\pgfpathlineto{\pgfqpoint{3.961407in}{3.888204in}}%
\pgfpathlineto{\pgfqpoint{3.961407in}{3.892461in}}%
\pgfpathlineto{\pgfqpoint{3.965665in}{3.892461in}}%
\pgfpathlineto{\pgfqpoint{3.965665in}{3.888204in}}%
\pgfpathmoveto{\pgfqpoint{3.961407in}{3.892461in}}%
\pgfpathlineto{\pgfqpoint{3.961407in}{3.892461in}}%
\pgfpathlineto{\pgfqpoint{3.961407in}{3.896719in}}%
\pgfpathlineto{\pgfqpoint{3.965665in}{3.896719in}}%
\pgfpathlineto{\pgfqpoint{3.965665in}{3.892461in}}%
\pgfpathmoveto{\pgfqpoint{3.965665in}{3.888204in}}%
\pgfpathlineto{\pgfqpoint{3.965665in}{3.888204in}}%
\pgfpathlineto{\pgfqpoint{3.965665in}{3.892461in}}%
\pgfpathlineto{\pgfqpoint{3.969923in}{3.892461in}}%
\pgfpathlineto{\pgfqpoint{3.969923in}{3.888204in}}%
\pgfpathmoveto{\pgfqpoint{3.965665in}{3.892461in}}%
\pgfpathlineto{\pgfqpoint{3.965665in}{3.892461in}}%
\pgfpathlineto{\pgfqpoint{3.965665in}{3.896719in}}%
\pgfpathlineto{\pgfqpoint{3.969923in}{3.896719in}}%
\pgfpathlineto{\pgfqpoint{3.969923in}{3.892461in}}%
\pgfpathmoveto{\pgfqpoint{3.961407in}{3.896719in}}%
\pgfpathlineto{\pgfqpoint{3.961407in}{3.896719in}}%
\pgfpathlineto{\pgfqpoint{3.961407in}{3.900977in}}%
\pgfpathlineto{\pgfqpoint{3.965665in}{3.900977in}}%
\pgfpathlineto{\pgfqpoint{3.965665in}{3.896719in}}%
\pgfpathmoveto{\pgfqpoint{3.961407in}{3.900977in}}%
\pgfpathlineto{\pgfqpoint{3.961407in}{3.900977in}}%
\pgfpathlineto{\pgfqpoint{3.961407in}{3.905235in}}%
\pgfpathlineto{\pgfqpoint{3.965665in}{3.905235in}}%
\pgfpathlineto{\pgfqpoint{3.965665in}{3.900977in}}%
\pgfpathmoveto{\pgfqpoint{3.965665in}{3.896719in}}%
\pgfpathlineto{\pgfqpoint{3.965665in}{3.896719in}}%
\pgfpathlineto{\pgfqpoint{3.965665in}{3.900977in}}%
\pgfpathlineto{\pgfqpoint{3.969923in}{3.900977in}}%
\pgfpathlineto{\pgfqpoint{3.969923in}{3.896719in}}%
\pgfpathmoveto{\pgfqpoint{3.965665in}{3.900977in}}%
\pgfpathlineto{\pgfqpoint{3.965665in}{3.900977in}}%
\pgfpathlineto{\pgfqpoint{3.965665in}{3.905235in}}%
\pgfpathlineto{\pgfqpoint{3.969923in}{3.905235in}}%
\pgfpathlineto{\pgfqpoint{3.969923in}{3.900977in}}%
\pgfpathmoveto{\pgfqpoint{3.961407in}{3.905235in}}%
\pgfpathlineto{\pgfqpoint{3.961407in}{3.905235in}}%
\pgfpathlineto{\pgfqpoint{3.961407in}{3.909492in}}%
\pgfpathlineto{\pgfqpoint{3.965665in}{3.909492in}}%
\pgfpathlineto{\pgfqpoint{3.965665in}{3.905235in}}%
\pgfpathmoveto{\pgfqpoint{3.961407in}{3.909492in}}%
\pgfpathlineto{\pgfqpoint{3.961407in}{3.909492in}}%
\pgfpathlineto{\pgfqpoint{3.961407in}{3.913750in}}%
\pgfpathlineto{\pgfqpoint{3.965665in}{3.913750in}}%
\pgfpathlineto{\pgfqpoint{3.965665in}{3.909492in}}%
\pgfpathmoveto{\pgfqpoint{3.965665in}{3.905235in}}%
\pgfpathlineto{\pgfqpoint{3.965665in}{3.905235in}}%
\pgfpathlineto{\pgfqpoint{3.965665in}{3.909492in}}%
\pgfpathlineto{\pgfqpoint{3.969923in}{3.909492in}}%
\pgfpathlineto{\pgfqpoint{3.969923in}{3.905235in}}%
\pgfpathmoveto{\pgfqpoint{3.965665in}{3.909492in}}%
\pgfpathlineto{\pgfqpoint{3.965665in}{3.909492in}}%
\pgfpathlineto{\pgfqpoint{3.965665in}{3.913750in}}%
\pgfpathlineto{\pgfqpoint{3.969923in}{3.913750in}}%
\pgfpathlineto{\pgfqpoint{3.969923in}{3.909492in}}%
\pgfpathmoveto{\pgfqpoint{3.961407in}{3.913750in}}%
\pgfpathlineto{\pgfqpoint{3.961407in}{3.913750in}}%
\pgfpathlineto{\pgfqpoint{3.961407in}{3.918008in}}%
\pgfpathlineto{\pgfqpoint{3.965665in}{3.918008in}}%
\pgfpathlineto{\pgfqpoint{3.965665in}{3.913750in}}%
\pgfpathmoveto{\pgfqpoint{3.961407in}{3.918008in}}%
\pgfpathlineto{\pgfqpoint{3.961407in}{3.918008in}}%
\pgfpathlineto{\pgfqpoint{3.961407in}{3.922266in}}%
\pgfpathlineto{\pgfqpoint{3.965665in}{3.922266in}}%
\pgfpathlineto{\pgfqpoint{3.965665in}{3.918008in}}%
\pgfpathmoveto{\pgfqpoint{3.965665in}{3.913750in}}%
\pgfpathlineto{\pgfqpoint{3.965665in}{3.913750in}}%
\pgfpathlineto{\pgfqpoint{3.965665in}{3.918008in}}%
\pgfpathlineto{\pgfqpoint{3.969923in}{3.918008in}}%
\pgfpathlineto{\pgfqpoint{3.969923in}{3.913750in}}%
\pgfpathmoveto{\pgfqpoint{3.965665in}{3.918008in}}%
\pgfpathlineto{\pgfqpoint{3.965665in}{3.918008in}}%
\pgfpathlineto{\pgfqpoint{3.965665in}{3.922266in}}%
\pgfpathlineto{\pgfqpoint{3.969923in}{3.922266in}}%
\pgfpathlineto{\pgfqpoint{3.969923in}{3.918008in}}%
\pgfpathmoveto{\pgfqpoint{3.965665in}{3.922266in}}%
\pgfpathlineto{\pgfqpoint{3.965665in}{3.922266in}}%
\pgfpathlineto{\pgfqpoint{3.965665in}{3.926523in}}%
\pgfpathlineto{\pgfqpoint{3.969923in}{3.926523in}}%
\pgfpathlineto{\pgfqpoint{3.969923in}{3.922266in}}%
\pgfpathmoveto{\pgfqpoint{3.965665in}{3.926523in}}%
\pgfpathlineto{\pgfqpoint{3.965665in}{3.926523in}}%
\pgfpathlineto{\pgfqpoint{3.965665in}{3.930781in}}%
\pgfpathlineto{\pgfqpoint{3.969923in}{3.930781in}}%
\pgfpathlineto{\pgfqpoint{3.969923in}{3.926523in}}%
\pgfpathmoveto{\pgfqpoint{3.969923in}{3.922266in}}%
\pgfpathlineto{\pgfqpoint{3.969923in}{3.922266in}}%
\pgfpathlineto{\pgfqpoint{3.969923in}{3.926523in}}%
\pgfpathlineto{\pgfqpoint{3.974181in}{3.926523in}}%
\pgfpathlineto{\pgfqpoint{3.974181in}{3.922266in}}%
\pgfpathmoveto{\pgfqpoint{3.969923in}{3.926523in}}%
\pgfpathlineto{\pgfqpoint{3.969923in}{3.926523in}}%
\pgfpathlineto{\pgfqpoint{3.969923in}{3.930781in}}%
\pgfpathlineto{\pgfqpoint{3.974181in}{3.930781in}}%
\pgfpathlineto{\pgfqpoint{3.974181in}{3.926523in}}%
\pgfpathmoveto{\pgfqpoint{3.965665in}{3.930781in}}%
\pgfpathlineto{\pgfqpoint{3.965665in}{3.930781in}}%
\pgfpathlineto{\pgfqpoint{3.965665in}{3.935039in}}%
\pgfpathlineto{\pgfqpoint{3.969923in}{3.935039in}}%
\pgfpathlineto{\pgfqpoint{3.969923in}{3.930781in}}%
\pgfpathmoveto{\pgfqpoint{3.965665in}{3.935039in}}%
\pgfpathlineto{\pgfqpoint{3.965665in}{3.935039in}}%
\pgfpathlineto{\pgfqpoint{3.965665in}{3.939296in}}%
\pgfpathlineto{\pgfqpoint{3.969923in}{3.939296in}}%
\pgfpathlineto{\pgfqpoint{3.969923in}{3.935039in}}%
\pgfpathmoveto{\pgfqpoint{3.965665in}{3.939296in}}%
\pgfpathlineto{\pgfqpoint{3.965665in}{3.939296in}}%
\pgfpathlineto{\pgfqpoint{3.965665in}{3.943554in}}%
\pgfpathlineto{\pgfqpoint{3.969923in}{3.943554in}}%
\pgfpathlineto{\pgfqpoint{3.969923in}{3.939296in}}%
\pgfpathmoveto{\pgfqpoint{3.965665in}{3.943554in}}%
\pgfpathlineto{\pgfqpoint{3.965665in}{3.943554in}}%
\pgfpathlineto{\pgfqpoint{3.965665in}{3.947812in}}%
\pgfpathlineto{\pgfqpoint{3.969923in}{3.947812in}}%
\pgfpathlineto{\pgfqpoint{3.969923in}{3.943554in}}%
\pgfpathmoveto{\pgfqpoint{3.969923in}{3.930781in}}%
\pgfpathlineto{\pgfqpoint{3.969923in}{3.930781in}}%
\pgfpathlineto{\pgfqpoint{3.969923in}{3.935039in}}%
\pgfpathlineto{\pgfqpoint{3.974181in}{3.935039in}}%
\pgfpathlineto{\pgfqpoint{3.974181in}{3.930781in}}%
\pgfpathmoveto{\pgfqpoint{3.969923in}{3.935039in}}%
\pgfpathlineto{\pgfqpoint{3.969923in}{3.935039in}}%
\pgfpathlineto{\pgfqpoint{3.969923in}{3.939296in}}%
\pgfpathlineto{\pgfqpoint{3.974181in}{3.939296in}}%
\pgfpathlineto{\pgfqpoint{3.974181in}{3.935039in}}%
\pgfpathmoveto{\pgfqpoint{3.969923in}{3.939296in}}%
\pgfpathlineto{\pgfqpoint{3.969923in}{3.939296in}}%
\pgfpathlineto{\pgfqpoint{3.969923in}{3.943554in}}%
\pgfpathlineto{\pgfqpoint{3.974181in}{3.943554in}}%
\pgfpathlineto{\pgfqpoint{3.974181in}{3.939296in}}%
\pgfpathmoveto{\pgfqpoint{3.969923in}{3.943554in}}%
\pgfpathlineto{\pgfqpoint{3.969923in}{3.943554in}}%
\pgfpathlineto{\pgfqpoint{3.969923in}{3.947812in}}%
\pgfpathlineto{\pgfqpoint{3.974181in}{3.947812in}}%
\pgfpathlineto{\pgfqpoint{3.974181in}{3.943554in}}%
\pgfpathmoveto{\pgfqpoint{3.965665in}{3.947812in}}%
\pgfpathlineto{\pgfqpoint{3.965665in}{3.947812in}}%
\pgfpathlineto{\pgfqpoint{3.965665in}{3.952069in}}%
\pgfpathlineto{\pgfqpoint{3.969923in}{3.952069in}}%
\pgfpathlineto{\pgfqpoint{3.969923in}{3.947812in}}%
\pgfpathmoveto{\pgfqpoint{3.965665in}{3.952069in}}%
\pgfpathlineto{\pgfqpoint{3.965665in}{3.952069in}}%
\pgfpathlineto{\pgfqpoint{3.965665in}{3.956327in}}%
\pgfpathlineto{\pgfqpoint{3.969923in}{3.956327in}}%
\pgfpathlineto{\pgfqpoint{3.969923in}{3.952069in}}%
\pgfpathmoveto{\pgfqpoint{3.965665in}{3.956327in}}%
\pgfpathlineto{\pgfqpoint{3.965665in}{3.956327in}}%
\pgfpathlineto{\pgfqpoint{3.965665in}{3.960585in}}%
\pgfpathlineto{\pgfqpoint{3.969923in}{3.960585in}}%
\pgfpathlineto{\pgfqpoint{3.969923in}{3.956327in}}%
\pgfpathmoveto{\pgfqpoint{3.969923in}{3.947812in}}%
\pgfpathlineto{\pgfqpoint{3.969923in}{3.947812in}}%
\pgfpathlineto{\pgfqpoint{3.969923in}{3.952069in}}%
\pgfpathlineto{\pgfqpoint{3.974181in}{3.952069in}}%
\pgfpathlineto{\pgfqpoint{3.974181in}{3.947812in}}%
\pgfpathmoveto{\pgfqpoint{3.969923in}{3.952069in}}%
\pgfpathlineto{\pgfqpoint{3.969923in}{3.952069in}}%
\pgfpathlineto{\pgfqpoint{3.969923in}{3.956327in}}%
\pgfpathlineto{\pgfqpoint{3.974181in}{3.956327in}}%
\pgfpathlineto{\pgfqpoint{3.974181in}{3.952069in}}%
\pgfpathmoveto{\pgfqpoint{3.969923in}{3.956327in}}%
\pgfpathlineto{\pgfqpoint{3.969923in}{3.956327in}}%
\pgfpathlineto{\pgfqpoint{3.969923in}{3.960585in}}%
\pgfpathlineto{\pgfqpoint{3.974181in}{3.960585in}}%
\pgfpathlineto{\pgfqpoint{3.974181in}{3.956327in}}%
\pgfpathmoveto{\pgfqpoint{3.969923in}{3.960585in}}%
\pgfpathlineto{\pgfqpoint{3.969923in}{3.960585in}}%
\pgfpathlineto{\pgfqpoint{3.969923in}{3.964843in}}%
\pgfpathlineto{\pgfqpoint{3.974181in}{3.964843in}}%
\pgfpathlineto{\pgfqpoint{3.974181in}{3.960585in}}%
\pgfpathmoveto{\pgfqpoint{3.969923in}{3.964843in}}%
\pgfpathlineto{\pgfqpoint{3.969923in}{3.964843in}}%
\pgfpathlineto{\pgfqpoint{3.969923in}{3.969100in}}%
\pgfpathlineto{\pgfqpoint{3.974181in}{3.969100in}}%
\pgfpathlineto{\pgfqpoint{3.974181in}{3.964843in}}%
\pgfpathmoveto{\pgfqpoint{3.969923in}{3.969100in}}%
\pgfpathlineto{\pgfqpoint{3.969923in}{3.969100in}}%
\pgfpathlineto{\pgfqpoint{3.969923in}{3.973358in}}%
\pgfpathlineto{\pgfqpoint{3.974181in}{3.973358in}}%
\pgfpathlineto{\pgfqpoint{3.974181in}{3.969100in}}%
\pgfpathmoveto{\pgfqpoint{3.974181in}{3.964843in}}%
\pgfpathlineto{\pgfqpoint{3.974181in}{3.964843in}}%
\pgfpathlineto{\pgfqpoint{3.974181in}{3.969100in}}%
\pgfpathlineto{\pgfqpoint{3.978439in}{3.969100in}}%
\pgfpathlineto{\pgfqpoint{3.978439in}{3.964843in}}%
\pgfpathmoveto{\pgfqpoint{3.974181in}{3.969100in}}%
\pgfpathlineto{\pgfqpoint{3.974181in}{3.969100in}}%
\pgfpathlineto{\pgfqpoint{3.974181in}{3.973358in}}%
\pgfpathlineto{\pgfqpoint{3.978439in}{3.973358in}}%
\pgfpathlineto{\pgfqpoint{3.978439in}{3.969100in}}%
\pgfpathmoveto{\pgfqpoint{3.969923in}{3.973358in}}%
\pgfpathlineto{\pgfqpoint{3.969923in}{3.973358in}}%
\pgfpathlineto{\pgfqpoint{3.969923in}{3.977616in}}%
\pgfpathlineto{\pgfqpoint{3.974181in}{3.977616in}}%
\pgfpathlineto{\pgfqpoint{3.974181in}{3.973358in}}%
\pgfpathmoveto{\pgfqpoint{3.969923in}{3.977616in}}%
\pgfpathlineto{\pgfqpoint{3.969923in}{3.977616in}}%
\pgfpathlineto{\pgfqpoint{3.969923in}{3.981873in}}%
\pgfpathlineto{\pgfqpoint{3.974181in}{3.981873in}}%
\pgfpathlineto{\pgfqpoint{3.974181in}{3.977616in}}%
\pgfpathmoveto{\pgfqpoint{3.974181in}{3.973358in}}%
\pgfpathlineto{\pgfqpoint{3.974181in}{3.973358in}}%
\pgfpathlineto{\pgfqpoint{3.974181in}{3.977616in}}%
\pgfpathlineto{\pgfqpoint{3.978439in}{3.977616in}}%
\pgfpathlineto{\pgfqpoint{3.978439in}{3.973358in}}%
\pgfpathmoveto{\pgfqpoint{3.974181in}{3.977616in}}%
\pgfpathlineto{\pgfqpoint{3.974181in}{3.977616in}}%
\pgfpathlineto{\pgfqpoint{3.974181in}{3.981873in}}%
\pgfpathlineto{\pgfqpoint{3.978439in}{3.981873in}}%
\pgfpathlineto{\pgfqpoint{3.978439in}{3.977616in}}%
\pgfpathmoveto{\pgfqpoint{3.969923in}{3.981873in}}%
\pgfpathlineto{\pgfqpoint{3.969923in}{3.981873in}}%
\pgfpathlineto{\pgfqpoint{3.969923in}{3.986131in}}%
\pgfpathlineto{\pgfqpoint{3.974181in}{3.986131in}}%
\pgfpathlineto{\pgfqpoint{3.974181in}{3.981873in}}%
\pgfpathmoveto{\pgfqpoint{3.969923in}{3.986131in}}%
\pgfpathlineto{\pgfqpoint{3.969923in}{3.986131in}}%
\pgfpathlineto{\pgfqpoint{3.969923in}{3.990389in}}%
\pgfpathlineto{\pgfqpoint{3.974181in}{3.990389in}}%
\pgfpathlineto{\pgfqpoint{3.974181in}{3.986131in}}%
\pgfpathmoveto{\pgfqpoint{3.974181in}{3.981873in}}%
\pgfpathlineto{\pgfqpoint{3.974181in}{3.981873in}}%
\pgfpathlineto{\pgfqpoint{3.974181in}{3.986131in}}%
\pgfpathlineto{\pgfqpoint{3.978439in}{3.986131in}}%
\pgfpathlineto{\pgfqpoint{3.978439in}{3.981873in}}%
\pgfpathmoveto{\pgfqpoint{3.974181in}{3.986131in}}%
\pgfpathlineto{\pgfqpoint{3.974181in}{3.986131in}}%
\pgfpathlineto{\pgfqpoint{3.974181in}{3.990389in}}%
\pgfpathlineto{\pgfqpoint{3.978439in}{3.990389in}}%
\pgfpathlineto{\pgfqpoint{3.978439in}{3.986131in}}%
\pgfpathmoveto{\pgfqpoint{3.969923in}{3.990389in}}%
\pgfpathlineto{\pgfqpoint{3.969923in}{3.990389in}}%
\pgfpathlineto{\pgfqpoint{3.969923in}{3.994647in}}%
\pgfpathlineto{\pgfqpoint{3.974181in}{3.994647in}}%
\pgfpathlineto{\pgfqpoint{3.974181in}{3.990389in}}%
\pgfpathmoveto{\pgfqpoint{3.969923in}{3.994647in}}%
\pgfpathlineto{\pgfqpoint{3.969923in}{3.994647in}}%
\pgfpathlineto{\pgfqpoint{3.969923in}{3.998904in}}%
\pgfpathlineto{\pgfqpoint{3.974181in}{3.998904in}}%
\pgfpathlineto{\pgfqpoint{3.974181in}{3.994647in}}%
\pgfpathmoveto{\pgfqpoint{3.974181in}{3.990389in}}%
\pgfpathlineto{\pgfqpoint{3.974181in}{3.990389in}}%
\pgfpathlineto{\pgfqpoint{3.974181in}{3.994647in}}%
\pgfpathlineto{\pgfqpoint{3.978439in}{3.994647in}}%
\pgfpathlineto{\pgfqpoint{3.978439in}{3.990389in}}%
\pgfpathmoveto{\pgfqpoint{3.974181in}{3.994647in}}%
\pgfpathlineto{\pgfqpoint{3.974181in}{3.994647in}}%
\pgfpathlineto{\pgfqpoint{3.974181in}{3.998904in}}%
\pgfpathlineto{\pgfqpoint{3.978439in}{3.998904in}}%
\pgfpathlineto{\pgfqpoint{3.978439in}{3.994647in}}%
\pgfpathmoveto{\pgfqpoint{3.969923in}{3.998904in}}%
\pgfpathlineto{\pgfqpoint{3.969923in}{3.998904in}}%
\pgfpathlineto{\pgfqpoint{3.969923in}{4.003162in}}%
\pgfpathlineto{\pgfqpoint{3.974181in}{4.003162in}}%
\pgfpathlineto{\pgfqpoint{3.974181in}{3.998904in}}%
\pgfpathmoveto{\pgfqpoint{3.974181in}{3.998904in}}%
\pgfpathlineto{\pgfqpoint{3.974181in}{3.998904in}}%
\pgfpathlineto{\pgfqpoint{3.974181in}{4.003162in}}%
\pgfpathlineto{\pgfqpoint{3.978439in}{4.003162in}}%
\pgfpathlineto{\pgfqpoint{3.978439in}{3.998904in}}%
\pgfpathmoveto{\pgfqpoint{3.974181in}{4.003162in}}%
\pgfpathlineto{\pgfqpoint{3.974181in}{4.003162in}}%
\pgfpathlineto{\pgfqpoint{3.974181in}{4.007420in}}%
\pgfpathlineto{\pgfqpoint{3.978439in}{4.007420in}}%
\pgfpathlineto{\pgfqpoint{3.978439in}{4.003162in}}%
\pgfpathmoveto{\pgfqpoint{3.974181in}{4.007420in}}%
\pgfpathlineto{\pgfqpoint{3.974181in}{4.007420in}}%
\pgfpathlineto{\pgfqpoint{3.974181in}{4.011677in}}%
\pgfpathlineto{\pgfqpoint{3.978439in}{4.011677in}}%
\pgfpathlineto{\pgfqpoint{3.978439in}{4.007420in}}%
\pgfpathmoveto{\pgfqpoint{3.974181in}{4.011677in}}%
\pgfpathlineto{\pgfqpoint{3.974181in}{4.011677in}}%
\pgfpathlineto{\pgfqpoint{3.974181in}{4.015935in}}%
\pgfpathlineto{\pgfqpoint{3.978439in}{4.015935in}}%
\pgfpathlineto{\pgfqpoint{3.978439in}{4.011677in}}%
\pgfpathmoveto{\pgfqpoint{3.974181in}{4.015935in}}%
\pgfpathlineto{\pgfqpoint{3.974181in}{4.015935in}}%
\pgfpathlineto{\pgfqpoint{3.974181in}{4.020193in}}%
\pgfpathlineto{\pgfqpoint{3.978439in}{4.020193in}}%
\pgfpathlineto{\pgfqpoint{3.978439in}{4.015935in}}%
\pgfpathmoveto{\pgfqpoint{3.974181in}{4.020193in}}%
\pgfpathlineto{\pgfqpoint{3.974181in}{4.020193in}}%
\pgfpathlineto{\pgfqpoint{3.974181in}{4.024451in}}%
\pgfpathlineto{\pgfqpoint{3.978439in}{4.024451in}}%
\pgfpathlineto{\pgfqpoint{3.978439in}{4.020193in}}%
\pgfpathmoveto{\pgfqpoint{3.974181in}{4.024451in}}%
\pgfpathlineto{\pgfqpoint{3.974181in}{4.024451in}}%
\pgfpathlineto{\pgfqpoint{3.974181in}{4.028708in}}%
\pgfpathlineto{\pgfqpoint{3.978439in}{4.028708in}}%
\pgfpathlineto{\pgfqpoint{3.978439in}{4.024451in}}%
\pgfpathmoveto{\pgfqpoint{3.974181in}{4.028708in}}%
\pgfpathlineto{\pgfqpoint{3.974181in}{4.028708in}}%
\pgfpathlineto{\pgfqpoint{3.974181in}{4.032966in}}%
\pgfpathlineto{\pgfqpoint{3.978439in}{4.032966in}}%
\pgfpathlineto{\pgfqpoint{3.978439in}{4.028708in}}%
\pgfpathmoveto{\pgfqpoint{3.974181in}{4.032966in}}%
\pgfpathlineto{\pgfqpoint{3.974181in}{4.032966in}}%
\pgfpathlineto{\pgfqpoint{3.974181in}{4.037224in}}%
\pgfpathlineto{\pgfqpoint{3.978439in}{4.037224in}}%
\pgfpathlineto{\pgfqpoint{3.978439in}{4.032966in}}%
\pgfpathmoveto{\pgfqpoint{3.974181in}{4.037224in}}%
\pgfpathlineto{\pgfqpoint{3.974181in}{4.037224in}}%
\pgfpathlineto{\pgfqpoint{3.974181in}{4.041481in}}%
\pgfpathlineto{\pgfqpoint{3.978439in}{4.041481in}}%
\pgfpathlineto{\pgfqpoint{3.978439in}{4.037224in}}%
\pgfpathmoveto{\pgfqpoint{3.974181in}{4.041481in}}%
\pgfpathlineto{\pgfqpoint{3.974181in}{4.041481in}}%
\pgfpathlineto{\pgfqpoint{3.974181in}{4.045739in}}%
\pgfpathlineto{\pgfqpoint{3.978439in}{4.045739in}}%
\pgfpathlineto{\pgfqpoint{3.978439in}{4.041481in}}%
\pgfpathmoveto{\pgfqpoint{3.978439in}{4.007420in}}%
\pgfpathlineto{\pgfqpoint{3.978439in}{4.007420in}}%
\pgfpathlineto{\pgfqpoint{3.978439in}{4.011677in}}%
\pgfpathlineto{\pgfqpoint{3.982697in}{4.011677in}}%
\pgfpathlineto{\pgfqpoint{3.982697in}{4.007420in}}%
\pgfpathmoveto{\pgfqpoint{3.978439in}{4.011677in}}%
\pgfpathlineto{\pgfqpoint{3.978439in}{4.011677in}}%
\pgfpathlineto{\pgfqpoint{3.978439in}{4.015935in}}%
\pgfpathlineto{\pgfqpoint{3.982697in}{4.015935in}}%
\pgfpathlineto{\pgfqpoint{3.982697in}{4.011677in}}%
\pgfpathmoveto{\pgfqpoint{3.978439in}{4.015935in}}%
\pgfpathlineto{\pgfqpoint{3.978439in}{4.015935in}}%
\pgfpathlineto{\pgfqpoint{3.978439in}{4.020193in}}%
\pgfpathlineto{\pgfqpoint{3.982697in}{4.020193in}}%
\pgfpathlineto{\pgfqpoint{3.982697in}{4.015935in}}%
\pgfpathmoveto{\pgfqpoint{3.978439in}{4.020193in}}%
\pgfpathlineto{\pgfqpoint{3.978439in}{4.020193in}}%
\pgfpathlineto{\pgfqpoint{3.978439in}{4.024451in}}%
\pgfpathlineto{\pgfqpoint{3.982697in}{4.024451in}}%
\pgfpathlineto{\pgfqpoint{3.982697in}{4.020193in}}%
\pgfpathmoveto{\pgfqpoint{3.978439in}{4.024451in}}%
\pgfpathlineto{\pgfqpoint{3.978439in}{4.024451in}}%
\pgfpathlineto{\pgfqpoint{3.978439in}{4.028708in}}%
\pgfpathlineto{\pgfqpoint{3.982697in}{4.028708in}}%
\pgfpathlineto{\pgfqpoint{3.982697in}{4.024451in}}%
\pgfpathmoveto{\pgfqpoint{3.978439in}{4.028708in}}%
\pgfpathlineto{\pgfqpoint{3.978439in}{4.028708in}}%
\pgfpathlineto{\pgfqpoint{3.978439in}{4.032966in}}%
\pgfpathlineto{\pgfqpoint{3.982697in}{4.032966in}}%
\pgfpathlineto{\pgfqpoint{3.982697in}{4.028708in}}%
\pgfpathmoveto{\pgfqpoint{3.978439in}{4.032966in}}%
\pgfpathlineto{\pgfqpoint{3.978439in}{4.032966in}}%
\pgfpathlineto{\pgfqpoint{3.978439in}{4.037224in}}%
\pgfpathlineto{\pgfqpoint{3.982697in}{4.037224in}}%
\pgfpathlineto{\pgfqpoint{3.982697in}{4.032966in}}%
\pgfpathmoveto{\pgfqpoint{3.978439in}{4.037224in}}%
\pgfpathlineto{\pgfqpoint{3.978439in}{4.037224in}}%
\pgfpathlineto{\pgfqpoint{3.978439in}{4.041481in}}%
\pgfpathlineto{\pgfqpoint{3.982697in}{4.041481in}}%
\pgfpathlineto{\pgfqpoint{3.982697in}{4.037224in}}%
\pgfpathmoveto{\pgfqpoint{3.978439in}{4.041481in}}%
\pgfpathlineto{\pgfqpoint{3.978439in}{4.041481in}}%
\pgfpathlineto{\pgfqpoint{3.978439in}{4.045739in}}%
\pgfpathlineto{\pgfqpoint{3.982697in}{4.045739in}}%
\pgfpathlineto{\pgfqpoint{3.982697in}{4.041481in}}%
\pgfpathmoveto{\pgfqpoint{3.978439in}{4.045739in}}%
\pgfpathlineto{\pgfqpoint{3.978439in}{4.045739in}}%
\pgfpathlineto{\pgfqpoint{3.978439in}{4.049997in}}%
\pgfpathlineto{\pgfqpoint{3.982697in}{4.049997in}}%
\pgfpathlineto{\pgfqpoint{3.982697in}{4.045739in}}%
\pgfpathmoveto{\pgfqpoint{3.982697in}{4.045739in}}%
\pgfpathlineto{\pgfqpoint{3.982697in}{4.045739in}}%
\pgfpathlineto{\pgfqpoint{3.982697in}{4.049997in}}%
\pgfpathlineto{\pgfqpoint{3.986955in}{4.049997in}}%
\pgfpathlineto{\pgfqpoint{3.986955in}{4.045739in}}%
\pgfpathmoveto{\pgfqpoint{3.978439in}{4.049997in}}%
\pgfpathlineto{\pgfqpoint{3.978439in}{4.049997in}}%
\pgfpathlineto{\pgfqpoint{3.978439in}{4.054255in}}%
\pgfpathlineto{\pgfqpoint{3.982697in}{4.054255in}}%
\pgfpathlineto{\pgfqpoint{3.982697in}{4.049997in}}%
\pgfpathmoveto{\pgfqpoint{3.978439in}{4.054255in}}%
\pgfpathlineto{\pgfqpoint{3.978439in}{4.054255in}}%
\pgfpathlineto{\pgfqpoint{3.978439in}{4.058512in}}%
\pgfpathlineto{\pgfqpoint{3.982697in}{4.058512in}}%
\pgfpathlineto{\pgfqpoint{3.982697in}{4.054255in}}%
\pgfpathmoveto{\pgfqpoint{3.982697in}{4.049997in}}%
\pgfpathlineto{\pgfqpoint{3.982697in}{4.049997in}}%
\pgfpathlineto{\pgfqpoint{3.982697in}{4.054255in}}%
\pgfpathlineto{\pgfqpoint{3.986955in}{4.054255in}}%
\pgfpathlineto{\pgfqpoint{3.986955in}{4.049997in}}%
\pgfpathmoveto{\pgfqpoint{3.982697in}{4.054255in}}%
\pgfpathlineto{\pgfqpoint{3.982697in}{4.054255in}}%
\pgfpathlineto{\pgfqpoint{3.982697in}{4.058512in}}%
\pgfpathlineto{\pgfqpoint{3.986955in}{4.058512in}}%
\pgfpathlineto{\pgfqpoint{3.986955in}{4.054255in}}%
\pgfpathmoveto{\pgfqpoint{3.978439in}{4.058512in}}%
\pgfpathlineto{\pgfqpoint{3.978439in}{4.058512in}}%
\pgfpathlineto{\pgfqpoint{3.978439in}{4.062770in}}%
\pgfpathlineto{\pgfqpoint{3.982697in}{4.062770in}}%
\pgfpathlineto{\pgfqpoint{3.982697in}{4.058512in}}%
\pgfpathmoveto{\pgfqpoint{3.978439in}{4.062770in}}%
\pgfpathlineto{\pgfqpoint{3.978439in}{4.062770in}}%
\pgfpathlineto{\pgfqpoint{3.978439in}{4.067028in}}%
\pgfpathlineto{\pgfqpoint{3.982697in}{4.067028in}}%
\pgfpathlineto{\pgfqpoint{3.982697in}{4.062770in}}%
\pgfpathmoveto{\pgfqpoint{3.982697in}{4.058512in}}%
\pgfpathlineto{\pgfqpoint{3.982697in}{4.058512in}}%
\pgfpathlineto{\pgfqpoint{3.982697in}{4.062770in}}%
\pgfpathlineto{\pgfqpoint{3.986955in}{4.062770in}}%
\pgfpathlineto{\pgfqpoint{3.986955in}{4.058512in}}%
\pgfpathmoveto{\pgfqpoint{3.982697in}{4.062770in}}%
\pgfpathlineto{\pgfqpoint{3.982697in}{4.062770in}}%
\pgfpathlineto{\pgfqpoint{3.982697in}{4.067028in}}%
\pgfpathlineto{\pgfqpoint{3.986955in}{4.067028in}}%
\pgfpathlineto{\pgfqpoint{3.986955in}{4.062770in}}%
\pgfpathmoveto{\pgfqpoint{3.978439in}{4.067028in}}%
\pgfpathlineto{\pgfqpoint{3.978439in}{4.067028in}}%
\pgfpathlineto{\pgfqpoint{3.978439in}{4.071286in}}%
\pgfpathlineto{\pgfqpoint{3.982697in}{4.071286in}}%
\pgfpathlineto{\pgfqpoint{3.982697in}{4.067028in}}%
\pgfpathmoveto{\pgfqpoint{3.978439in}{4.071286in}}%
\pgfpathlineto{\pgfqpoint{3.978439in}{4.071286in}}%
\pgfpathlineto{\pgfqpoint{3.978439in}{4.075544in}}%
\pgfpathlineto{\pgfqpoint{3.982697in}{4.075544in}}%
\pgfpathlineto{\pgfqpoint{3.982697in}{4.071286in}}%
\pgfpathmoveto{\pgfqpoint{3.982697in}{4.067028in}}%
\pgfpathlineto{\pgfqpoint{3.982697in}{4.067028in}}%
\pgfpathlineto{\pgfqpoint{3.982697in}{4.071286in}}%
\pgfpathlineto{\pgfqpoint{3.986955in}{4.071286in}}%
\pgfpathlineto{\pgfqpoint{3.986955in}{4.067028in}}%
\pgfpathmoveto{\pgfqpoint{3.982697in}{4.071286in}}%
\pgfpathlineto{\pgfqpoint{3.982697in}{4.071286in}}%
\pgfpathlineto{\pgfqpoint{3.982697in}{4.075544in}}%
\pgfpathlineto{\pgfqpoint{3.986955in}{4.075544in}}%
\pgfpathlineto{\pgfqpoint{3.986955in}{4.071286in}}%
\pgfpathmoveto{\pgfqpoint{3.978439in}{4.075544in}}%
\pgfpathlineto{\pgfqpoint{3.978439in}{4.075544in}}%
\pgfpathlineto{\pgfqpoint{3.978439in}{4.079801in}}%
\pgfpathlineto{\pgfqpoint{3.982697in}{4.079801in}}%
\pgfpathlineto{\pgfqpoint{3.982697in}{4.075544in}}%
\pgfpathmoveto{\pgfqpoint{3.978439in}{4.079801in}}%
\pgfpathlineto{\pgfqpoint{3.978439in}{4.079801in}}%
\pgfpathlineto{\pgfqpoint{3.978439in}{4.084059in}}%
\pgfpathlineto{\pgfqpoint{3.982697in}{4.084059in}}%
\pgfpathlineto{\pgfqpoint{3.982697in}{4.079801in}}%
\pgfpathmoveto{\pgfqpoint{3.982697in}{4.075544in}}%
\pgfpathlineto{\pgfqpoint{3.982697in}{4.075544in}}%
\pgfpathlineto{\pgfqpoint{3.982697in}{4.079801in}}%
\pgfpathlineto{\pgfqpoint{3.986955in}{4.079801in}}%
\pgfpathlineto{\pgfqpoint{3.986955in}{4.075544in}}%
\pgfpathmoveto{\pgfqpoint{3.982697in}{4.079801in}}%
\pgfpathlineto{\pgfqpoint{3.982697in}{4.079801in}}%
\pgfpathlineto{\pgfqpoint{3.982697in}{4.084059in}}%
\pgfpathlineto{\pgfqpoint{3.986955in}{4.084059in}}%
\pgfpathlineto{\pgfqpoint{3.986955in}{4.079801in}}%
\pgfpathmoveto{\pgfqpoint{3.982697in}{4.084059in}}%
\pgfpathlineto{\pgfqpoint{3.982697in}{4.084059in}}%
\pgfpathlineto{\pgfqpoint{3.982697in}{4.088317in}}%
\pgfpathlineto{\pgfqpoint{3.986955in}{4.088317in}}%
\pgfpathlineto{\pgfqpoint{3.986955in}{4.084059in}}%
\pgfpathmoveto{\pgfqpoint{3.982697in}{4.088317in}}%
\pgfpathlineto{\pgfqpoint{3.982697in}{4.088317in}}%
\pgfpathlineto{\pgfqpoint{3.982697in}{4.092575in}}%
\pgfpathlineto{\pgfqpoint{3.986955in}{4.092575in}}%
\pgfpathlineto{\pgfqpoint{3.986955in}{4.088317in}}%
\pgfpathmoveto{\pgfqpoint{3.982697in}{4.092575in}}%
\pgfpathlineto{\pgfqpoint{3.982697in}{4.092575in}}%
\pgfpathlineto{\pgfqpoint{3.982697in}{4.096833in}}%
\pgfpathlineto{\pgfqpoint{3.986955in}{4.096833in}}%
\pgfpathlineto{\pgfqpoint{3.986955in}{4.092575in}}%
\pgfpathmoveto{\pgfqpoint{3.982697in}{4.096833in}}%
\pgfpathlineto{\pgfqpoint{3.982697in}{4.096833in}}%
\pgfpathlineto{\pgfqpoint{3.982697in}{4.101090in}}%
\pgfpathlineto{\pgfqpoint{3.986955in}{4.101090in}}%
\pgfpathlineto{\pgfqpoint{3.986955in}{4.096833in}}%
\pgfpathmoveto{\pgfqpoint{3.986955in}{4.088317in}}%
\pgfpathlineto{\pgfqpoint{3.986955in}{4.088317in}}%
\pgfpathlineto{\pgfqpoint{3.986955in}{4.092575in}}%
\pgfpathlineto{\pgfqpoint{3.991213in}{4.092575in}}%
\pgfpathlineto{\pgfqpoint{3.991213in}{4.088317in}}%
\pgfpathmoveto{\pgfqpoint{3.986955in}{4.092575in}}%
\pgfpathlineto{\pgfqpoint{3.986955in}{4.092575in}}%
\pgfpathlineto{\pgfqpoint{3.986955in}{4.096833in}}%
\pgfpathlineto{\pgfqpoint{3.991213in}{4.096833in}}%
\pgfpathlineto{\pgfqpoint{3.991213in}{4.092575in}}%
\pgfpathmoveto{\pgfqpoint{3.986955in}{4.096833in}}%
\pgfpathlineto{\pgfqpoint{3.986955in}{4.096833in}}%
\pgfpathlineto{\pgfqpoint{3.986955in}{4.101090in}}%
\pgfpathlineto{\pgfqpoint{3.991213in}{4.101090in}}%
\pgfpathlineto{\pgfqpoint{3.991213in}{4.096833in}}%
\pgfpathmoveto{\pgfqpoint{3.982697in}{4.101090in}}%
\pgfpathlineto{\pgfqpoint{3.982697in}{4.101090in}}%
\pgfpathlineto{\pgfqpoint{3.982697in}{4.105348in}}%
\pgfpathlineto{\pgfqpoint{3.986955in}{4.105348in}}%
\pgfpathlineto{\pgfqpoint{3.986955in}{4.101090in}}%
\pgfpathmoveto{\pgfqpoint{3.982697in}{4.105348in}}%
\pgfpathlineto{\pgfqpoint{3.982697in}{4.105348in}}%
\pgfpathlineto{\pgfqpoint{3.982697in}{4.109606in}}%
\pgfpathlineto{\pgfqpoint{3.986955in}{4.109606in}}%
\pgfpathlineto{\pgfqpoint{3.986955in}{4.105348in}}%
\pgfpathmoveto{\pgfqpoint{3.982697in}{4.109606in}}%
\pgfpathlineto{\pgfqpoint{3.982697in}{4.109606in}}%
\pgfpathlineto{\pgfqpoint{3.982697in}{4.113864in}}%
\pgfpathlineto{\pgfqpoint{3.986955in}{4.113864in}}%
\pgfpathlineto{\pgfqpoint{3.986955in}{4.109606in}}%
\pgfpathmoveto{\pgfqpoint{3.982697in}{4.113864in}}%
\pgfpathlineto{\pgfqpoint{3.982697in}{4.113864in}}%
\pgfpathlineto{\pgfqpoint{3.982697in}{4.118122in}}%
\pgfpathlineto{\pgfqpoint{3.986955in}{4.118122in}}%
\pgfpathlineto{\pgfqpoint{3.986955in}{4.113864in}}%
\pgfpathmoveto{\pgfqpoint{3.986955in}{4.101090in}}%
\pgfpathlineto{\pgfqpoint{3.986955in}{4.101090in}}%
\pgfpathlineto{\pgfqpoint{3.986955in}{4.105348in}}%
\pgfpathlineto{\pgfqpoint{3.991213in}{4.105348in}}%
\pgfpathlineto{\pgfqpoint{3.991213in}{4.101090in}}%
\pgfpathmoveto{\pgfqpoint{3.986955in}{4.105348in}}%
\pgfpathlineto{\pgfqpoint{3.986955in}{4.105348in}}%
\pgfpathlineto{\pgfqpoint{3.986955in}{4.109606in}}%
\pgfpathlineto{\pgfqpoint{3.991213in}{4.109606in}}%
\pgfpathlineto{\pgfqpoint{3.991213in}{4.105348in}}%
\pgfpathmoveto{\pgfqpoint{3.986955in}{4.109606in}}%
\pgfpathlineto{\pgfqpoint{3.986955in}{4.109606in}}%
\pgfpathlineto{\pgfqpoint{3.986955in}{4.113864in}}%
\pgfpathlineto{\pgfqpoint{3.991213in}{4.113864in}}%
\pgfpathlineto{\pgfqpoint{3.991213in}{4.109606in}}%
\pgfpathmoveto{\pgfqpoint{3.986955in}{4.113864in}}%
\pgfpathlineto{\pgfqpoint{3.986955in}{4.113864in}}%
\pgfpathlineto{\pgfqpoint{3.986955in}{4.118122in}}%
\pgfpathlineto{\pgfqpoint{3.991213in}{4.118122in}}%
\pgfpathlineto{\pgfqpoint{3.991213in}{4.113864in}}%
\pgfpathmoveto{\pgfqpoint{3.982697in}{4.118122in}}%
\pgfpathlineto{\pgfqpoint{3.982697in}{4.118122in}}%
\pgfpathlineto{\pgfqpoint{3.982697in}{4.122380in}}%
\pgfpathlineto{\pgfqpoint{3.986955in}{4.122380in}}%
\pgfpathlineto{\pgfqpoint{3.986955in}{4.118122in}}%
\pgfpathmoveto{\pgfqpoint{3.982697in}{4.122380in}}%
\pgfpathlineto{\pgfqpoint{3.982697in}{4.122380in}}%
\pgfpathlineto{\pgfqpoint{3.982697in}{4.126637in}}%
\pgfpathlineto{\pgfqpoint{3.986955in}{4.126637in}}%
\pgfpathlineto{\pgfqpoint{3.986955in}{4.122380in}}%
\pgfpathmoveto{\pgfqpoint{3.986955in}{4.118122in}}%
\pgfpathlineto{\pgfqpoint{3.986955in}{4.118122in}}%
\pgfpathlineto{\pgfqpoint{3.986955in}{4.122380in}}%
\pgfpathlineto{\pgfqpoint{3.991213in}{4.122380in}}%
\pgfpathlineto{\pgfqpoint{3.991213in}{4.118122in}}%
\pgfpathmoveto{\pgfqpoint{3.986955in}{4.122380in}}%
\pgfpathlineto{\pgfqpoint{3.986955in}{4.122380in}}%
\pgfpathlineto{\pgfqpoint{3.986955in}{4.126637in}}%
\pgfpathlineto{\pgfqpoint{3.991213in}{4.126637in}}%
\pgfpathlineto{\pgfqpoint{3.991213in}{4.122380in}}%
\pgfpathmoveto{\pgfqpoint{3.986955in}{4.126637in}}%
\pgfpathlineto{\pgfqpoint{3.986955in}{4.126637in}}%
\pgfpathlineto{\pgfqpoint{3.986955in}{4.130895in}}%
\pgfpathlineto{\pgfqpoint{3.991213in}{4.130895in}}%
\pgfpathlineto{\pgfqpoint{3.991213in}{4.126637in}}%
\pgfpathmoveto{\pgfqpoint{3.986955in}{4.130895in}}%
\pgfpathlineto{\pgfqpoint{3.986955in}{4.130895in}}%
\pgfpathlineto{\pgfqpoint{3.986955in}{4.135153in}}%
\pgfpathlineto{\pgfqpoint{3.991213in}{4.135153in}}%
\pgfpathlineto{\pgfqpoint{3.991213in}{4.130895in}}%
\pgfpathmoveto{\pgfqpoint{3.991213in}{4.130895in}}%
\pgfpathlineto{\pgfqpoint{3.991213in}{4.130895in}}%
\pgfpathlineto{\pgfqpoint{3.991213in}{4.135153in}}%
\pgfpathlineto{\pgfqpoint{3.995471in}{4.135153in}}%
\pgfpathlineto{\pgfqpoint{3.995471in}{4.130895in}}%
\pgfpathmoveto{\pgfqpoint{3.986955in}{4.135153in}}%
\pgfpathlineto{\pgfqpoint{3.986955in}{4.135153in}}%
\pgfpathlineto{\pgfqpoint{3.986955in}{4.139411in}}%
\pgfpathlineto{\pgfqpoint{3.991213in}{4.139411in}}%
\pgfpathlineto{\pgfqpoint{3.991213in}{4.135153in}}%
\pgfpathmoveto{\pgfqpoint{3.986955in}{4.139411in}}%
\pgfpathlineto{\pgfqpoint{3.986955in}{4.139411in}}%
\pgfpathlineto{\pgfqpoint{3.986955in}{4.143669in}}%
\pgfpathlineto{\pgfqpoint{3.991213in}{4.143669in}}%
\pgfpathlineto{\pgfqpoint{3.991213in}{4.139411in}}%
\pgfpathmoveto{\pgfqpoint{3.991213in}{4.135153in}}%
\pgfpathlineto{\pgfqpoint{3.991213in}{4.135153in}}%
\pgfpathlineto{\pgfqpoint{3.991213in}{4.139411in}}%
\pgfpathlineto{\pgfqpoint{3.995471in}{4.139411in}}%
\pgfpathlineto{\pgfqpoint{3.995471in}{4.135153in}}%
\pgfpathmoveto{\pgfqpoint{3.991213in}{4.139411in}}%
\pgfpathlineto{\pgfqpoint{3.991213in}{4.139411in}}%
\pgfpathlineto{\pgfqpoint{3.991213in}{4.143669in}}%
\pgfpathlineto{\pgfqpoint{3.995471in}{4.143669in}}%
\pgfpathlineto{\pgfqpoint{3.995471in}{4.139411in}}%
\pgfpathmoveto{\pgfqpoint{3.986955in}{4.143669in}}%
\pgfpathlineto{\pgfqpoint{3.986955in}{4.143669in}}%
\pgfpathlineto{\pgfqpoint{3.986955in}{4.147926in}}%
\pgfpathlineto{\pgfqpoint{3.991213in}{4.147926in}}%
\pgfpathlineto{\pgfqpoint{3.991213in}{4.143669in}}%
\pgfpathmoveto{\pgfqpoint{3.986955in}{4.147926in}}%
\pgfpathlineto{\pgfqpoint{3.986955in}{4.147926in}}%
\pgfpathlineto{\pgfqpoint{3.986955in}{4.152184in}}%
\pgfpathlineto{\pgfqpoint{3.991213in}{4.152184in}}%
\pgfpathlineto{\pgfqpoint{3.991213in}{4.147926in}}%
\pgfpathmoveto{\pgfqpoint{3.991213in}{4.143669in}}%
\pgfpathlineto{\pgfqpoint{3.991213in}{4.143669in}}%
\pgfpathlineto{\pgfqpoint{3.991213in}{4.147926in}}%
\pgfpathlineto{\pgfqpoint{3.995471in}{4.147926in}}%
\pgfpathlineto{\pgfqpoint{3.995471in}{4.143669in}}%
\pgfpathmoveto{\pgfqpoint{3.991213in}{4.147926in}}%
\pgfpathlineto{\pgfqpoint{3.991213in}{4.147926in}}%
\pgfpathlineto{\pgfqpoint{3.991213in}{4.152184in}}%
\pgfpathlineto{\pgfqpoint{3.995471in}{4.152184in}}%
\pgfpathlineto{\pgfqpoint{3.995471in}{4.147926in}}%
\pgfpathmoveto{\pgfqpoint{3.986955in}{4.152184in}}%
\pgfpathlineto{\pgfqpoint{3.986955in}{4.152184in}}%
\pgfpathlineto{\pgfqpoint{3.986955in}{4.156442in}}%
\pgfpathlineto{\pgfqpoint{3.991213in}{4.156442in}}%
\pgfpathlineto{\pgfqpoint{3.991213in}{4.152184in}}%
\pgfpathmoveto{\pgfqpoint{3.986955in}{4.156442in}}%
\pgfpathlineto{\pgfqpoint{3.986955in}{4.156442in}}%
\pgfpathlineto{\pgfqpoint{3.986955in}{4.160700in}}%
\pgfpathlineto{\pgfqpoint{3.991213in}{4.160700in}}%
\pgfpathlineto{\pgfqpoint{3.991213in}{4.156442in}}%
\pgfpathmoveto{\pgfqpoint{3.991213in}{4.152184in}}%
\pgfpathlineto{\pgfqpoint{3.991213in}{4.152184in}}%
\pgfpathlineto{\pgfqpoint{3.991213in}{4.156442in}}%
\pgfpathlineto{\pgfqpoint{3.995471in}{4.156442in}}%
\pgfpathlineto{\pgfqpoint{3.995471in}{4.152184in}}%
\pgfpathmoveto{\pgfqpoint{3.991213in}{4.156442in}}%
\pgfpathlineto{\pgfqpoint{3.991213in}{4.156442in}}%
\pgfpathlineto{\pgfqpoint{3.991213in}{4.160700in}}%
\pgfpathlineto{\pgfqpoint{3.995471in}{4.160700in}}%
\pgfpathlineto{\pgfqpoint{3.995471in}{4.156442in}}%
\pgfpathmoveto{\pgfqpoint{3.986955in}{4.160700in}}%
\pgfpathlineto{\pgfqpoint{3.986955in}{4.160700in}}%
\pgfpathlineto{\pgfqpoint{3.986955in}{4.164958in}}%
\pgfpathlineto{\pgfqpoint{3.991213in}{4.164958in}}%
\pgfpathlineto{\pgfqpoint{3.991213in}{4.160700in}}%
\pgfpathmoveto{\pgfqpoint{3.986955in}{4.164958in}}%
\pgfpathlineto{\pgfqpoint{3.986955in}{4.164958in}}%
\pgfpathlineto{\pgfqpoint{3.986955in}{4.169215in}}%
\pgfpathlineto{\pgfqpoint{3.991213in}{4.169215in}}%
\pgfpathlineto{\pgfqpoint{3.991213in}{4.164958in}}%
\pgfpathmoveto{\pgfqpoint{3.991213in}{4.160700in}}%
\pgfpathlineto{\pgfqpoint{3.991213in}{4.160700in}}%
\pgfpathlineto{\pgfqpoint{3.991213in}{4.164958in}}%
\pgfpathlineto{\pgfqpoint{3.995471in}{4.164958in}}%
\pgfpathlineto{\pgfqpoint{3.995471in}{4.160700in}}%
\pgfpathmoveto{\pgfqpoint{3.991213in}{4.164958in}}%
\pgfpathlineto{\pgfqpoint{3.991213in}{4.164958in}}%
\pgfpathlineto{\pgfqpoint{3.991213in}{4.169215in}}%
\pgfpathlineto{\pgfqpoint{3.995471in}{4.169215in}}%
\pgfpathlineto{\pgfqpoint{3.995471in}{4.164958in}}%
\pgfpathmoveto{\pgfqpoint{3.991213in}{4.169215in}}%
\pgfpathlineto{\pgfqpoint{3.991213in}{4.169215in}}%
\pgfpathlineto{\pgfqpoint{3.991213in}{4.173473in}}%
\pgfpathlineto{\pgfqpoint{3.995471in}{4.173473in}}%
\pgfpathlineto{\pgfqpoint{3.995471in}{4.169215in}}%
\pgfpathmoveto{\pgfqpoint{3.991213in}{4.173473in}}%
\pgfpathlineto{\pgfqpoint{3.991213in}{4.173473in}}%
\pgfpathlineto{\pgfqpoint{3.991213in}{4.177731in}}%
\pgfpathlineto{\pgfqpoint{3.995471in}{4.177731in}}%
\pgfpathlineto{\pgfqpoint{3.995471in}{4.173473in}}%
\pgfpathmoveto{\pgfqpoint{3.991213in}{4.177731in}}%
\pgfpathlineto{\pgfqpoint{3.991213in}{4.177731in}}%
\pgfpathlineto{\pgfqpoint{3.991213in}{4.181989in}}%
\pgfpathlineto{\pgfqpoint{3.995471in}{4.181989in}}%
\pgfpathlineto{\pgfqpoint{3.995471in}{4.177731in}}%
\pgfpathmoveto{\pgfqpoint{3.991213in}{4.181989in}}%
\pgfpathlineto{\pgfqpoint{3.991213in}{4.181989in}}%
\pgfpathlineto{\pgfqpoint{3.991213in}{4.186247in}}%
\pgfpathlineto{\pgfqpoint{3.995471in}{4.186247in}}%
\pgfpathlineto{\pgfqpoint{3.995471in}{4.181989in}}%
\pgfpathmoveto{\pgfqpoint{3.995471in}{4.173473in}}%
\pgfpathlineto{\pgfqpoint{3.995471in}{4.173473in}}%
\pgfpathlineto{\pgfqpoint{3.995471in}{4.177731in}}%
\pgfpathlineto{\pgfqpoint{3.999729in}{4.177731in}}%
\pgfpathlineto{\pgfqpoint{3.999729in}{4.173473in}}%
\pgfpathmoveto{\pgfqpoint{3.995471in}{4.177731in}}%
\pgfpathlineto{\pgfqpoint{3.995471in}{4.177731in}}%
\pgfpathlineto{\pgfqpoint{3.995471in}{4.181989in}}%
\pgfpathlineto{\pgfqpoint{3.999729in}{4.181989in}}%
\pgfpathlineto{\pgfqpoint{3.999729in}{4.177731in}}%
\pgfpathmoveto{\pgfqpoint{3.995471in}{4.181989in}}%
\pgfpathlineto{\pgfqpoint{3.995471in}{4.181989in}}%
\pgfpathlineto{\pgfqpoint{3.995471in}{4.186247in}}%
\pgfpathlineto{\pgfqpoint{3.999729in}{4.186247in}}%
\pgfpathlineto{\pgfqpoint{3.999729in}{4.181989in}}%
\pgfpathmoveto{\pgfqpoint{3.991213in}{4.186247in}}%
\pgfpathlineto{\pgfqpoint{3.991213in}{4.186247in}}%
\pgfpathlineto{\pgfqpoint{3.991213in}{4.190504in}}%
\pgfpathlineto{\pgfqpoint{3.995471in}{4.190504in}}%
\pgfpathlineto{\pgfqpoint{3.995471in}{4.186247in}}%
\pgfpathmoveto{\pgfqpoint{3.991213in}{4.190504in}}%
\pgfpathlineto{\pgfqpoint{3.991213in}{4.190504in}}%
\pgfpathlineto{\pgfqpoint{3.991213in}{4.194762in}}%
\pgfpathlineto{\pgfqpoint{3.995471in}{4.194762in}}%
\pgfpathlineto{\pgfqpoint{3.995471in}{4.190504in}}%
\pgfpathmoveto{\pgfqpoint{3.991213in}{4.194762in}}%
\pgfpathlineto{\pgfqpoint{3.991213in}{4.194762in}}%
\pgfpathlineto{\pgfqpoint{3.991213in}{4.199020in}}%
\pgfpathlineto{\pgfqpoint{3.995471in}{4.199020in}}%
\pgfpathlineto{\pgfqpoint{3.995471in}{4.194762in}}%
\pgfpathmoveto{\pgfqpoint{3.991213in}{4.199020in}}%
\pgfpathlineto{\pgfqpoint{3.991213in}{4.199020in}}%
\pgfpathlineto{\pgfqpoint{3.991213in}{4.203278in}}%
\pgfpathlineto{\pgfqpoint{3.995471in}{4.203278in}}%
\pgfpathlineto{\pgfqpoint{3.995471in}{4.199020in}}%
\pgfpathmoveto{\pgfqpoint{3.991213in}{4.203278in}}%
\pgfpathlineto{\pgfqpoint{3.991213in}{4.203278in}}%
\pgfpathlineto{\pgfqpoint{3.991213in}{4.207536in}}%
\pgfpathlineto{\pgfqpoint{3.995471in}{4.207536in}}%
\pgfpathlineto{\pgfqpoint{3.995471in}{4.203278in}}%
\pgfpathmoveto{\pgfqpoint{3.991213in}{4.207536in}}%
\pgfpathlineto{\pgfqpoint{3.991213in}{4.207536in}}%
\pgfpathlineto{\pgfqpoint{3.991213in}{4.211794in}}%
\pgfpathlineto{\pgfqpoint{3.995471in}{4.211794in}}%
\pgfpathlineto{\pgfqpoint{3.995471in}{4.207536in}}%
\pgfpathmoveto{\pgfqpoint{3.995471in}{4.186247in}}%
\pgfpathlineto{\pgfqpoint{3.995471in}{4.186247in}}%
\pgfpathlineto{\pgfqpoint{3.995471in}{4.190504in}}%
\pgfpathlineto{\pgfqpoint{3.999729in}{4.190504in}}%
\pgfpathlineto{\pgfqpoint{3.999729in}{4.186247in}}%
\pgfpathmoveto{\pgfqpoint{3.995471in}{4.190504in}}%
\pgfpathlineto{\pgfqpoint{3.995471in}{4.190504in}}%
\pgfpathlineto{\pgfqpoint{3.995471in}{4.194762in}}%
\pgfpathlineto{\pgfqpoint{3.999729in}{4.194762in}}%
\pgfpathlineto{\pgfqpoint{3.999729in}{4.190504in}}%
\pgfpathmoveto{\pgfqpoint{3.995471in}{4.194762in}}%
\pgfpathlineto{\pgfqpoint{3.995471in}{4.194762in}}%
\pgfpathlineto{\pgfqpoint{3.995471in}{4.199020in}}%
\pgfpathlineto{\pgfqpoint{3.999729in}{4.199020in}}%
\pgfpathlineto{\pgfqpoint{3.999729in}{4.194762in}}%
\pgfpathmoveto{\pgfqpoint{3.995471in}{4.199020in}}%
\pgfpathlineto{\pgfqpoint{3.995471in}{4.199020in}}%
\pgfpathlineto{\pgfqpoint{3.995471in}{4.203278in}}%
\pgfpathlineto{\pgfqpoint{3.999729in}{4.203278in}}%
\pgfpathlineto{\pgfqpoint{3.999729in}{4.199020in}}%
\pgfpathmoveto{\pgfqpoint{3.995471in}{4.203278in}}%
\pgfpathlineto{\pgfqpoint{3.995471in}{4.203278in}}%
\pgfpathlineto{\pgfqpoint{3.995471in}{4.207536in}}%
\pgfpathlineto{\pgfqpoint{3.999729in}{4.207536in}}%
\pgfpathlineto{\pgfqpoint{3.999729in}{4.203278in}}%
\pgfpathmoveto{\pgfqpoint{3.995471in}{4.207536in}}%
\pgfpathlineto{\pgfqpoint{3.995471in}{4.207536in}}%
\pgfpathlineto{\pgfqpoint{3.995471in}{4.211794in}}%
\pgfpathlineto{\pgfqpoint{3.999729in}{4.211794in}}%
\pgfpathlineto{\pgfqpoint{3.999729in}{4.207536in}}%
\pgfpathmoveto{\pgfqpoint{3.995471in}{4.211794in}}%
\pgfpathlineto{\pgfqpoint{3.995471in}{4.211794in}}%
\pgfpathlineto{\pgfqpoint{3.995471in}{4.216051in}}%
\pgfpathlineto{\pgfqpoint{3.999729in}{4.216051in}}%
\pgfpathlineto{\pgfqpoint{3.999729in}{4.211794in}}%
\pgfpathmoveto{\pgfqpoint{3.995471in}{4.216051in}}%
\pgfpathlineto{\pgfqpoint{3.995471in}{4.216051in}}%
\pgfpathlineto{\pgfqpoint{3.995471in}{4.220309in}}%
\pgfpathlineto{\pgfqpoint{3.999729in}{4.220309in}}%
\pgfpathlineto{\pgfqpoint{3.999729in}{4.216051in}}%
\pgfpathmoveto{\pgfqpoint{3.999729in}{4.216051in}}%
\pgfpathlineto{\pgfqpoint{3.999729in}{4.216051in}}%
\pgfpathlineto{\pgfqpoint{3.999729in}{4.220309in}}%
\pgfpathlineto{\pgfqpoint{4.003987in}{4.220309in}}%
\pgfpathlineto{\pgfqpoint{4.003987in}{4.216051in}}%
\pgfpathmoveto{\pgfqpoint{3.995471in}{4.220309in}}%
\pgfpathlineto{\pgfqpoint{3.995471in}{4.220309in}}%
\pgfpathlineto{\pgfqpoint{3.995471in}{4.224567in}}%
\pgfpathlineto{\pgfqpoint{3.999729in}{4.224567in}}%
\pgfpathlineto{\pgfqpoint{3.999729in}{4.220309in}}%
\pgfpathmoveto{\pgfqpoint{3.995471in}{4.224567in}}%
\pgfpathlineto{\pgfqpoint{3.995471in}{4.224567in}}%
\pgfpathlineto{\pgfqpoint{3.995471in}{4.228825in}}%
\pgfpathlineto{\pgfqpoint{3.999729in}{4.228825in}}%
\pgfpathlineto{\pgfqpoint{3.999729in}{4.224567in}}%
\pgfpathmoveto{\pgfqpoint{3.999729in}{4.220309in}}%
\pgfpathlineto{\pgfqpoint{3.999729in}{4.220309in}}%
\pgfpathlineto{\pgfqpoint{3.999729in}{4.224567in}}%
\pgfpathlineto{\pgfqpoint{4.003987in}{4.224567in}}%
\pgfpathlineto{\pgfqpoint{4.003987in}{4.220309in}}%
\pgfpathmoveto{\pgfqpoint{3.999729in}{4.224567in}}%
\pgfpathlineto{\pgfqpoint{3.999729in}{4.224567in}}%
\pgfpathlineto{\pgfqpoint{3.999729in}{4.228825in}}%
\pgfpathlineto{\pgfqpoint{4.003987in}{4.228825in}}%
\pgfpathlineto{\pgfqpoint{4.003987in}{4.224567in}}%
\pgfpathmoveto{\pgfqpoint{3.995471in}{4.228825in}}%
\pgfpathlineto{\pgfqpoint{3.995471in}{4.228825in}}%
\pgfpathlineto{\pgfqpoint{3.995471in}{4.233083in}}%
\pgfpathlineto{\pgfqpoint{3.999729in}{4.233083in}}%
\pgfpathlineto{\pgfqpoint{3.999729in}{4.228825in}}%
\pgfpathmoveto{\pgfqpoint{3.995471in}{4.233083in}}%
\pgfpathlineto{\pgfqpoint{3.995471in}{4.233083in}}%
\pgfpathlineto{\pgfqpoint{3.995471in}{4.237341in}}%
\pgfpathlineto{\pgfqpoint{3.999729in}{4.237341in}}%
\pgfpathlineto{\pgfqpoint{3.999729in}{4.233083in}}%
\pgfpathmoveto{\pgfqpoint{3.999729in}{4.228825in}}%
\pgfpathlineto{\pgfqpoint{3.999729in}{4.228825in}}%
\pgfpathlineto{\pgfqpoint{3.999729in}{4.233083in}}%
\pgfpathlineto{\pgfqpoint{4.003987in}{4.233083in}}%
\pgfpathlineto{\pgfqpoint{4.003987in}{4.228825in}}%
\pgfpathmoveto{\pgfqpoint{3.999729in}{4.233083in}}%
\pgfpathlineto{\pgfqpoint{3.999729in}{4.233083in}}%
\pgfpathlineto{\pgfqpoint{3.999729in}{4.237341in}}%
\pgfpathlineto{\pgfqpoint{4.003987in}{4.237341in}}%
\pgfpathlineto{\pgfqpoint{4.003987in}{4.233083in}}%
\pgfpathmoveto{\pgfqpoint{3.995471in}{4.237341in}}%
\pgfpathlineto{\pgfqpoint{3.995471in}{4.237341in}}%
\pgfpathlineto{\pgfqpoint{3.995471in}{4.241598in}}%
\pgfpathlineto{\pgfqpoint{3.999729in}{4.241598in}}%
\pgfpathlineto{\pgfqpoint{3.999729in}{4.237341in}}%
\pgfpathmoveto{\pgfqpoint{3.995471in}{4.241598in}}%
\pgfpathlineto{\pgfqpoint{3.995471in}{4.241598in}}%
\pgfpathlineto{\pgfqpoint{3.995471in}{4.245856in}}%
\pgfpathlineto{\pgfqpoint{3.999729in}{4.245856in}}%
\pgfpathlineto{\pgfqpoint{3.999729in}{4.241598in}}%
\pgfpathmoveto{\pgfqpoint{3.999729in}{4.237341in}}%
\pgfpathlineto{\pgfqpoint{3.999729in}{4.237341in}}%
\pgfpathlineto{\pgfqpoint{3.999729in}{4.241598in}}%
\pgfpathlineto{\pgfqpoint{4.003987in}{4.241598in}}%
\pgfpathlineto{\pgfqpoint{4.003987in}{4.237341in}}%
\pgfpathmoveto{\pgfqpoint{3.999729in}{4.241598in}}%
\pgfpathlineto{\pgfqpoint{3.999729in}{4.241598in}}%
\pgfpathlineto{\pgfqpoint{3.999729in}{4.245856in}}%
\pgfpathlineto{\pgfqpoint{4.003987in}{4.245856in}}%
\pgfpathlineto{\pgfqpoint{4.003987in}{4.241598in}}%
\pgfpathmoveto{\pgfqpoint{3.995471in}{4.245856in}}%
\pgfpathlineto{\pgfqpoint{3.995471in}{4.245856in}}%
\pgfpathlineto{\pgfqpoint{3.995471in}{4.250114in}}%
\pgfpathlineto{\pgfqpoint{3.999729in}{4.250114in}}%
\pgfpathlineto{\pgfqpoint{3.999729in}{4.245856in}}%
\pgfpathmoveto{\pgfqpoint{3.995471in}{4.250114in}}%
\pgfpathlineto{\pgfqpoint{3.995471in}{4.250114in}}%
\pgfpathlineto{\pgfqpoint{3.995471in}{4.254372in}}%
\pgfpathlineto{\pgfqpoint{3.999729in}{4.254372in}}%
\pgfpathlineto{\pgfqpoint{3.999729in}{4.250114in}}%
\pgfpathmoveto{\pgfqpoint{3.999729in}{4.245856in}}%
\pgfpathlineto{\pgfqpoint{3.999729in}{4.245856in}}%
\pgfpathlineto{\pgfqpoint{3.999729in}{4.250114in}}%
\pgfpathlineto{\pgfqpoint{4.003987in}{4.250114in}}%
\pgfpathlineto{\pgfqpoint{4.003987in}{4.245856in}}%
\pgfpathmoveto{\pgfqpoint{3.999729in}{4.250114in}}%
\pgfpathlineto{\pgfqpoint{3.999729in}{4.250114in}}%
\pgfpathlineto{\pgfqpoint{3.999729in}{4.254372in}}%
\pgfpathlineto{\pgfqpoint{4.003987in}{4.254372in}}%
\pgfpathlineto{\pgfqpoint{4.003987in}{4.250114in}}%
\pgfpathmoveto{\pgfqpoint{3.999729in}{4.254372in}}%
\pgfpathlineto{\pgfqpoint{3.999729in}{4.254372in}}%
\pgfpathlineto{\pgfqpoint{3.999729in}{4.258630in}}%
\pgfpathlineto{\pgfqpoint{4.003987in}{4.258630in}}%
\pgfpathlineto{\pgfqpoint{4.003987in}{4.254372in}}%
\pgfpathmoveto{\pgfqpoint{3.999729in}{4.258630in}}%
\pgfpathlineto{\pgfqpoint{3.999729in}{4.258630in}}%
\pgfpathlineto{\pgfqpoint{3.999729in}{4.262888in}}%
\pgfpathlineto{\pgfqpoint{4.003987in}{4.262888in}}%
\pgfpathlineto{\pgfqpoint{4.003987in}{4.258630in}}%
\pgfpathmoveto{\pgfqpoint{3.999729in}{4.262888in}}%
\pgfpathlineto{\pgfqpoint{3.999729in}{4.262888in}}%
\pgfpathlineto{\pgfqpoint{3.999729in}{4.267145in}}%
\pgfpathlineto{\pgfqpoint{4.003987in}{4.267145in}}%
\pgfpathlineto{\pgfqpoint{4.003987in}{4.262888in}}%
\pgfpathmoveto{\pgfqpoint{3.999729in}{4.267145in}}%
\pgfpathlineto{\pgfqpoint{3.999729in}{4.267145in}}%
\pgfpathlineto{\pgfqpoint{3.999729in}{4.271403in}}%
\pgfpathlineto{\pgfqpoint{4.003987in}{4.271403in}}%
\pgfpathlineto{\pgfqpoint{4.003987in}{4.267145in}}%
\pgfpathmoveto{\pgfqpoint{4.003987in}{4.258630in}}%
\pgfpathlineto{\pgfqpoint{4.003987in}{4.258630in}}%
\pgfpathlineto{\pgfqpoint{4.003987in}{4.262888in}}%
\pgfpathlineto{\pgfqpoint{4.008244in}{4.262888in}}%
\pgfpathlineto{\pgfqpoint{4.008244in}{4.258630in}}%
\pgfpathmoveto{\pgfqpoint{4.003987in}{4.262888in}}%
\pgfpathlineto{\pgfqpoint{4.003987in}{4.262888in}}%
\pgfpathlineto{\pgfqpoint{4.003987in}{4.267145in}}%
\pgfpathlineto{\pgfqpoint{4.008244in}{4.267145in}}%
\pgfpathlineto{\pgfqpoint{4.008244in}{4.262888in}}%
\pgfpathmoveto{\pgfqpoint{4.003987in}{4.267145in}}%
\pgfpathlineto{\pgfqpoint{4.003987in}{4.267145in}}%
\pgfpathlineto{\pgfqpoint{4.003987in}{4.271403in}}%
\pgfpathlineto{\pgfqpoint{4.008244in}{4.271403in}}%
\pgfpathlineto{\pgfqpoint{4.008244in}{4.267145in}}%
\pgfpathmoveto{\pgfqpoint{3.999729in}{4.271403in}}%
\pgfpathlineto{\pgfqpoint{3.999729in}{4.271403in}}%
\pgfpathlineto{\pgfqpoint{3.999729in}{4.275661in}}%
\pgfpathlineto{\pgfqpoint{4.003987in}{4.275661in}}%
\pgfpathlineto{\pgfqpoint{4.003987in}{4.271403in}}%
\pgfpathmoveto{\pgfqpoint{3.999729in}{4.275661in}}%
\pgfpathlineto{\pgfqpoint{3.999729in}{4.275661in}}%
\pgfpathlineto{\pgfqpoint{3.999729in}{4.279919in}}%
\pgfpathlineto{\pgfqpoint{4.003987in}{4.279919in}}%
\pgfpathlineto{\pgfqpoint{4.003987in}{4.275661in}}%
\pgfpathmoveto{\pgfqpoint{3.999729in}{4.279919in}}%
\pgfpathlineto{\pgfqpoint{3.999729in}{4.279919in}}%
\pgfpathlineto{\pgfqpoint{3.999729in}{4.284177in}}%
\pgfpathlineto{\pgfqpoint{4.003987in}{4.284177in}}%
\pgfpathlineto{\pgfqpoint{4.003987in}{4.279919in}}%
\pgfpathmoveto{\pgfqpoint{3.999729in}{4.284177in}}%
\pgfpathlineto{\pgfqpoint{3.999729in}{4.284177in}}%
\pgfpathlineto{\pgfqpoint{3.999729in}{4.288435in}}%
\pgfpathlineto{\pgfqpoint{4.003987in}{4.288435in}}%
\pgfpathlineto{\pgfqpoint{4.003987in}{4.284177in}}%
\pgfpathmoveto{\pgfqpoint{4.003987in}{4.271403in}}%
\pgfpathlineto{\pgfqpoint{4.003987in}{4.271403in}}%
\pgfpathlineto{\pgfqpoint{4.003987in}{4.275661in}}%
\pgfpathlineto{\pgfqpoint{4.008244in}{4.275661in}}%
\pgfpathlineto{\pgfqpoint{4.008244in}{4.271403in}}%
\pgfpathmoveto{\pgfqpoint{4.003987in}{4.275661in}}%
\pgfpathlineto{\pgfqpoint{4.003987in}{4.275661in}}%
\pgfpathlineto{\pgfqpoint{4.003987in}{4.279919in}}%
\pgfpathlineto{\pgfqpoint{4.008244in}{4.279919in}}%
\pgfpathlineto{\pgfqpoint{4.008244in}{4.275661in}}%
\pgfpathmoveto{\pgfqpoint{4.003987in}{4.279919in}}%
\pgfpathlineto{\pgfqpoint{4.003987in}{4.279919in}}%
\pgfpathlineto{\pgfqpoint{4.003987in}{4.284177in}}%
\pgfpathlineto{\pgfqpoint{4.008244in}{4.284177in}}%
\pgfpathlineto{\pgfqpoint{4.008244in}{4.279919in}}%
\pgfpathmoveto{\pgfqpoint{4.003987in}{4.284177in}}%
\pgfpathlineto{\pgfqpoint{4.003987in}{4.284177in}}%
\pgfpathlineto{\pgfqpoint{4.003987in}{4.288435in}}%
\pgfpathlineto{\pgfqpoint{4.008244in}{4.288435in}}%
\pgfpathlineto{\pgfqpoint{4.008244in}{4.284177in}}%
\pgfpathmoveto{\pgfqpoint{3.999729in}{4.288435in}}%
\pgfpathlineto{\pgfqpoint{3.999729in}{4.288435in}}%
\pgfpathlineto{\pgfqpoint{3.999729in}{4.292692in}}%
\pgfpathlineto{\pgfqpoint{4.003987in}{4.292692in}}%
\pgfpathlineto{\pgfqpoint{4.003987in}{4.288435in}}%
\pgfpathmoveto{\pgfqpoint{3.999729in}{4.292692in}}%
\pgfpathlineto{\pgfqpoint{3.999729in}{4.292692in}}%
\pgfpathlineto{\pgfqpoint{3.999729in}{4.296950in}}%
\pgfpathlineto{\pgfqpoint{4.003987in}{4.296950in}}%
\pgfpathlineto{\pgfqpoint{4.003987in}{4.292692in}}%
\pgfpathmoveto{\pgfqpoint{4.003987in}{4.288435in}}%
\pgfpathlineto{\pgfqpoint{4.003987in}{4.288435in}}%
\pgfpathlineto{\pgfqpoint{4.003987in}{4.292692in}}%
\pgfpathlineto{\pgfqpoint{4.008244in}{4.292692in}}%
\pgfpathlineto{\pgfqpoint{4.008244in}{4.288435in}}%
\pgfpathmoveto{\pgfqpoint{4.003987in}{4.292692in}}%
\pgfpathlineto{\pgfqpoint{4.003987in}{4.292692in}}%
\pgfpathlineto{\pgfqpoint{4.003987in}{4.296950in}}%
\pgfpathlineto{\pgfqpoint{4.008244in}{4.296950in}}%
\pgfpathlineto{\pgfqpoint{4.008244in}{4.292692in}}%
\pgfpathmoveto{\pgfqpoint{4.003987in}{4.296950in}}%
\pgfpathlineto{\pgfqpoint{4.003987in}{4.296950in}}%
\pgfpathlineto{\pgfqpoint{4.003987in}{4.301208in}}%
\pgfpathlineto{\pgfqpoint{4.008244in}{4.301208in}}%
\pgfpathlineto{\pgfqpoint{4.008244in}{4.296950in}}%
\pgfpathmoveto{\pgfqpoint{4.003987in}{4.301208in}}%
\pgfpathlineto{\pgfqpoint{4.003987in}{4.301208in}}%
\pgfpathlineto{\pgfqpoint{4.003987in}{4.305466in}}%
\pgfpathlineto{\pgfqpoint{4.008244in}{4.305466in}}%
\pgfpathlineto{\pgfqpoint{4.008244in}{4.301208in}}%
\pgfpathmoveto{\pgfqpoint{4.003987in}{4.305466in}}%
\pgfpathlineto{\pgfqpoint{4.003987in}{4.305466in}}%
\pgfpathlineto{\pgfqpoint{4.003987in}{4.309724in}}%
\pgfpathlineto{\pgfqpoint{4.008244in}{4.309724in}}%
\pgfpathlineto{\pgfqpoint{4.008244in}{4.305466in}}%
\pgfpathmoveto{\pgfqpoint{4.003987in}{4.309724in}}%
\pgfpathlineto{\pgfqpoint{4.003987in}{4.309724in}}%
\pgfpathlineto{\pgfqpoint{4.003987in}{4.313982in}}%
\pgfpathlineto{\pgfqpoint{4.008244in}{4.313982in}}%
\pgfpathlineto{\pgfqpoint{4.008244in}{4.309724in}}%
\pgfpathmoveto{\pgfqpoint{4.008244in}{4.305466in}}%
\pgfpathlineto{\pgfqpoint{4.008244in}{4.305466in}}%
\pgfpathlineto{\pgfqpoint{4.008244in}{4.309724in}}%
\pgfpathlineto{\pgfqpoint{4.012502in}{4.309724in}}%
\pgfpathlineto{\pgfqpoint{4.012502in}{4.305466in}}%
\pgfpathmoveto{\pgfqpoint{4.008244in}{4.309724in}}%
\pgfpathlineto{\pgfqpoint{4.008244in}{4.309724in}}%
\pgfpathlineto{\pgfqpoint{4.008244in}{4.313982in}}%
\pgfpathlineto{\pgfqpoint{4.012502in}{4.313982in}}%
\pgfpathlineto{\pgfqpoint{4.012502in}{4.309724in}}%
\pgfpathmoveto{\pgfqpoint{4.003987in}{4.313982in}}%
\pgfpathlineto{\pgfqpoint{4.003987in}{4.313982in}}%
\pgfpathlineto{\pgfqpoint{4.003987in}{4.318239in}}%
\pgfpathlineto{\pgfqpoint{4.008244in}{4.318239in}}%
\pgfpathlineto{\pgfqpoint{4.008244in}{4.313982in}}%
\pgfpathmoveto{\pgfqpoint{4.003987in}{4.318239in}}%
\pgfpathlineto{\pgfqpoint{4.003987in}{4.318239in}}%
\pgfpathlineto{\pgfqpoint{4.003987in}{4.322497in}}%
\pgfpathlineto{\pgfqpoint{4.008244in}{4.322497in}}%
\pgfpathlineto{\pgfqpoint{4.008244in}{4.318239in}}%
\pgfpathmoveto{\pgfqpoint{4.008244in}{4.313982in}}%
\pgfpathlineto{\pgfqpoint{4.008244in}{4.313982in}}%
\pgfpathlineto{\pgfqpoint{4.008244in}{4.318239in}}%
\pgfpathlineto{\pgfqpoint{4.012502in}{4.318239in}}%
\pgfpathlineto{\pgfqpoint{4.012502in}{4.313982in}}%
\pgfpathmoveto{\pgfqpoint{4.008244in}{4.318239in}}%
\pgfpathlineto{\pgfqpoint{4.008244in}{4.318239in}}%
\pgfpathlineto{\pgfqpoint{4.008244in}{4.322497in}}%
\pgfpathlineto{\pgfqpoint{4.012502in}{4.322497in}}%
\pgfpathlineto{\pgfqpoint{4.012502in}{4.318239in}}%
\pgfpathmoveto{\pgfqpoint{4.003987in}{4.322497in}}%
\pgfpathlineto{\pgfqpoint{4.003987in}{4.322497in}}%
\pgfpathlineto{\pgfqpoint{4.003987in}{4.326755in}}%
\pgfpathlineto{\pgfqpoint{4.008244in}{4.326755in}}%
\pgfpathlineto{\pgfqpoint{4.008244in}{4.322497in}}%
\pgfpathmoveto{\pgfqpoint{4.003987in}{4.326755in}}%
\pgfpathlineto{\pgfqpoint{4.003987in}{4.326755in}}%
\pgfpathlineto{\pgfqpoint{4.003987in}{4.331013in}}%
\pgfpathlineto{\pgfqpoint{4.008244in}{4.331013in}}%
\pgfpathlineto{\pgfqpoint{4.008244in}{4.326755in}}%
\pgfpathmoveto{\pgfqpoint{4.008244in}{4.322497in}}%
\pgfpathlineto{\pgfqpoint{4.008244in}{4.322497in}}%
\pgfpathlineto{\pgfqpoint{4.008244in}{4.326755in}}%
\pgfpathlineto{\pgfqpoint{4.012502in}{4.326755in}}%
\pgfpathlineto{\pgfqpoint{4.012502in}{4.322497in}}%
\pgfpathmoveto{\pgfqpoint{4.008244in}{4.326755in}}%
\pgfpathlineto{\pgfqpoint{4.008244in}{4.326755in}}%
\pgfpathlineto{\pgfqpoint{4.008244in}{4.331013in}}%
\pgfpathlineto{\pgfqpoint{4.012502in}{4.331013in}}%
\pgfpathlineto{\pgfqpoint{4.012502in}{4.326755in}}%
\pgfpathmoveto{\pgfqpoint{4.003987in}{4.331013in}}%
\pgfpathlineto{\pgfqpoint{4.003987in}{4.331013in}}%
\pgfpathlineto{\pgfqpoint{4.003987in}{4.335271in}}%
\pgfpathlineto{\pgfqpoint{4.008244in}{4.335271in}}%
\pgfpathlineto{\pgfqpoint{4.008244in}{4.331013in}}%
\pgfpathmoveto{\pgfqpoint{4.003987in}{4.335271in}}%
\pgfpathlineto{\pgfqpoint{4.003987in}{4.335271in}}%
\pgfpathlineto{\pgfqpoint{4.003987in}{4.339529in}}%
\pgfpathlineto{\pgfqpoint{4.008244in}{4.339529in}}%
\pgfpathlineto{\pgfqpoint{4.008244in}{4.335271in}}%
\pgfpathmoveto{\pgfqpoint{4.008244in}{4.331013in}}%
\pgfpathlineto{\pgfqpoint{4.008244in}{4.331013in}}%
\pgfpathlineto{\pgfqpoint{4.008244in}{4.335271in}}%
\pgfpathlineto{\pgfqpoint{4.012502in}{4.335271in}}%
\pgfpathlineto{\pgfqpoint{4.012502in}{4.331013in}}%
\pgfpathmoveto{\pgfqpoint{4.008244in}{4.335271in}}%
\pgfpathlineto{\pgfqpoint{4.008244in}{4.335271in}}%
\pgfpathlineto{\pgfqpoint{4.008244in}{4.339529in}}%
\pgfpathlineto{\pgfqpoint{4.012502in}{4.339529in}}%
\pgfpathlineto{\pgfqpoint{4.012502in}{4.335271in}}%
\pgfpathmoveto{\pgfqpoint{4.003987in}{4.339529in}}%
\pgfpathlineto{\pgfqpoint{4.003987in}{4.339529in}}%
\pgfpathlineto{\pgfqpoint{4.003987in}{4.343787in}}%
\pgfpathlineto{\pgfqpoint{4.008244in}{4.343787in}}%
\pgfpathlineto{\pgfqpoint{4.008244in}{4.339529in}}%
\pgfpathmoveto{\pgfqpoint{4.008244in}{4.339529in}}%
\pgfpathlineto{\pgfqpoint{4.008244in}{4.339529in}}%
\pgfpathlineto{\pgfqpoint{4.008244in}{4.343787in}}%
\pgfpathlineto{\pgfqpoint{4.012502in}{4.343787in}}%
\pgfpathlineto{\pgfqpoint{4.012502in}{4.339529in}}%
\pgfpathmoveto{\pgfqpoint{4.008244in}{4.343787in}}%
\pgfpathlineto{\pgfqpoint{4.008244in}{4.343787in}}%
\pgfpathlineto{\pgfqpoint{4.008244in}{4.348045in}}%
\pgfpathlineto{\pgfqpoint{4.012502in}{4.348045in}}%
\pgfpathlineto{\pgfqpoint{4.012502in}{4.343787in}}%
\pgfpathmoveto{\pgfqpoint{4.008244in}{4.348045in}}%
\pgfpathlineto{\pgfqpoint{4.008244in}{4.348045in}}%
\pgfpathlineto{\pgfqpoint{4.008244in}{4.352303in}}%
\pgfpathlineto{\pgfqpoint{4.012502in}{4.352303in}}%
\pgfpathlineto{\pgfqpoint{4.012502in}{4.348045in}}%
\pgfpathmoveto{\pgfqpoint{4.008244in}{4.352303in}}%
\pgfpathlineto{\pgfqpoint{4.008244in}{4.352303in}}%
\pgfpathlineto{\pgfqpoint{4.008244in}{4.356561in}}%
\pgfpathlineto{\pgfqpoint{4.012502in}{4.356561in}}%
\pgfpathlineto{\pgfqpoint{4.012502in}{4.352303in}}%
\pgfpathmoveto{\pgfqpoint{4.008244in}{4.356561in}}%
\pgfpathlineto{\pgfqpoint{4.008244in}{4.356561in}}%
\pgfpathlineto{\pgfqpoint{4.008244in}{4.360819in}}%
\pgfpathlineto{\pgfqpoint{4.012502in}{4.360819in}}%
\pgfpathlineto{\pgfqpoint{4.012502in}{4.356561in}}%
\pgfpathmoveto{\pgfqpoint{4.008244in}{4.360819in}}%
\pgfpathlineto{\pgfqpoint{4.008244in}{4.360819in}}%
\pgfpathlineto{\pgfqpoint{4.008244in}{4.365077in}}%
\pgfpathlineto{\pgfqpoint{4.012502in}{4.365077in}}%
\pgfpathlineto{\pgfqpoint{4.012502in}{4.360819in}}%
\pgfpathmoveto{\pgfqpoint{4.008244in}{4.365077in}}%
\pgfpathlineto{\pgfqpoint{4.008244in}{4.365077in}}%
\pgfpathlineto{\pgfqpoint{4.008244in}{4.369335in}}%
\pgfpathlineto{\pgfqpoint{4.012502in}{4.369335in}}%
\pgfpathlineto{\pgfqpoint{4.012502in}{4.365077in}}%
\pgfpathmoveto{\pgfqpoint{4.008244in}{4.369335in}}%
\pgfpathlineto{\pgfqpoint{4.008244in}{4.369335in}}%
\pgfpathlineto{\pgfqpoint{4.008244in}{4.373593in}}%
\pgfpathlineto{\pgfqpoint{4.012502in}{4.373593in}}%
\pgfpathlineto{\pgfqpoint{4.012502in}{4.369335in}}%
\pgfpathmoveto{\pgfqpoint{4.008244in}{4.373593in}}%
\pgfpathlineto{\pgfqpoint{4.008244in}{4.373593in}}%
\pgfpathlineto{\pgfqpoint{4.008244in}{4.377851in}}%
\pgfpathlineto{\pgfqpoint{4.012502in}{4.377851in}}%
\pgfpathlineto{\pgfqpoint{4.012502in}{4.373593in}}%
\pgfpathmoveto{\pgfqpoint{4.008244in}{4.377851in}}%
\pgfpathlineto{\pgfqpoint{4.008244in}{4.377851in}}%
\pgfpathlineto{\pgfqpoint{4.008244in}{4.382109in}}%
\pgfpathlineto{\pgfqpoint{4.012502in}{4.382109in}}%
\pgfpathlineto{\pgfqpoint{4.012502in}{4.377851in}}%
\pgfpathmoveto{\pgfqpoint{4.008244in}{4.382109in}}%
\pgfpathlineto{\pgfqpoint{4.008244in}{4.382109in}}%
\pgfpathlineto{\pgfqpoint{4.008244in}{4.386367in}}%
\pgfpathlineto{\pgfqpoint{4.012502in}{4.386367in}}%
\pgfpathlineto{\pgfqpoint{4.012502in}{4.382109in}}%
\pgfpathmoveto{\pgfqpoint{4.012502in}{4.348045in}}%
\pgfpathlineto{\pgfqpoint{4.012502in}{4.348045in}}%
\pgfpathlineto{\pgfqpoint{4.012502in}{4.352303in}}%
\pgfpathlineto{\pgfqpoint{4.016760in}{4.352303in}}%
\pgfpathlineto{\pgfqpoint{4.016760in}{4.348045in}}%
\pgfpathmoveto{\pgfqpoint{4.012502in}{4.352303in}}%
\pgfpathlineto{\pgfqpoint{4.012502in}{4.352303in}}%
\pgfpathlineto{\pgfqpoint{4.012502in}{4.356561in}}%
\pgfpathlineto{\pgfqpoint{4.016760in}{4.356561in}}%
\pgfpathlineto{\pgfqpoint{4.016760in}{4.352303in}}%
\pgfpathmoveto{\pgfqpoint{4.012502in}{4.356561in}}%
\pgfpathlineto{\pgfqpoint{4.012502in}{4.356561in}}%
\pgfpathlineto{\pgfqpoint{4.012502in}{4.360819in}}%
\pgfpathlineto{\pgfqpoint{4.016760in}{4.360819in}}%
\pgfpathlineto{\pgfqpoint{4.016760in}{4.356561in}}%
\pgfpathmoveto{\pgfqpoint{4.012502in}{4.360819in}}%
\pgfpathlineto{\pgfqpoint{4.012502in}{4.360819in}}%
\pgfpathlineto{\pgfqpoint{4.012502in}{4.365077in}}%
\pgfpathlineto{\pgfqpoint{4.016760in}{4.365077in}}%
\pgfpathlineto{\pgfqpoint{4.016760in}{4.360819in}}%
\pgfpathmoveto{\pgfqpoint{4.012502in}{4.365077in}}%
\pgfpathlineto{\pgfqpoint{4.012502in}{4.365077in}}%
\pgfpathlineto{\pgfqpoint{4.012502in}{4.369335in}}%
\pgfpathlineto{\pgfqpoint{4.016760in}{4.369335in}}%
\pgfpathlineto{\pgfqpoint{4.016760in}{4.365077in}}%
\pgfpathmoveto{\pgfqpoint{4.012502in}{4.369335in}}%
\pgfpathlineto{\pgfqpoint{4.012502in}{4.369335in}}%
\pgfpathlineto{\pgfqpoint{4.012502in}{4.373593in}}%
\pgfpathlineto{\pgfqpoint{4.016760in}{4.373593in}}%
\pgfpathlineto{\pgfqpoint{4.016760in}{4.369335in}}%
\pgfpathmoveto{\pgfqpoint{4.012502in}{4.373593in}}%
\pgfpathlineto{\pgfqpoint{4.012502in}{4.373593in}}%
\pgfpathlineto{\pgfqpoint{4.012502in}{4.377851in}}%
\pgfpathlineto{\pgfqpoint{4.016760in}{4.377851in}}%
\pgfpathlineto{\pgfqpoint{4.016760in}{4.373593in}}%
\pgfpathmoveto{\pgfqpoint{4.012502in}{4.377851in}}%
\pgfpathlineto{\pgfqpoint{4.012502in}{4.377851in}}%
\pgfpathlineto{\pgfqpoint{4.012502in}{4.382109in}}%
\pgfpathlineto{\pgfqpoint{4.016760in}{4.382109in}}%
\pgfpathlineto{\pgfqpoint{4.016760in}{4.377851in}}%
\pgfpathmoveto{\pgfqpoint{4.012502in}{4.382109in}}%
\pgfpathlineto{\pgfqpoint{4.012502in}{4.382109in}}%
\pgfpathlineto{\pgfqpoint{4.012502in}{4.386367in}}%
\pgfpathlineto{\pgfqpoint{4.016760in}{4.386367in}}%
\pgfpathlineto{\pgfqpoint{4.016760in}{4.382109in}}%
\pgfpathmoveto{\pgfqpoint{4.012502in}{4.386367in}}%
\pgfpathlineto{\pgfqpoint{4.012502in}{4.386367in}}%
\pgfpathlineto{\pgfqpoint{4.012502in}{4.390625in}}%
\pgfpathlineto{\pgfqpoint{4.016760in}{4.390625in}}%
\pgfpathlineto{\pgfqpoint{4.016760in}{4.386367in}}%
\pgfpathmoveto{\pgfqpoint{4.012502in}{4.390625in}}%
\pgfpathlineto{\pgfqpoint{4.012502in}{4.390625in}}%
\pgfpathlineto{\pgfqpoint{4.012502in}{4.394883in}}%
\pgfpathlineto{\pgfqpoint{4.016760in}{4.394883in}}%
\pgfpathlineto{\pgfqpoint{4.016760in}{4.390625in}}%
\pgfpathmoveto{\pgfqpoint{4.012502in}{4.394883in}}%
\pgfpathlineto{\pgfqpoint{4.012502in}{4.394883in}}%
\pgfpathlineto{\pgfqpoint{4.012502in}{4.399141in}}%
\pgfpathlineto{\pgfqpoint{4.016760in}{4.399141in}}%
\pgfpathlineto{\pgfqpoint{4.016760in}{4.394883in}}%
\pgfpathmoveto{\pgfqpoint{4.016760in}{4.390625in}}%
\pgfpathlineto{\pgfqpoint{4.016760in}{4.390625in}}%
\pgfpathlineto{\pgfqpoint{4.016760in}{4.394883in}}%
\pgfpathlineto{\pgfqpoint{4.021018in}{4.394883in}}%
\pgfpathlineto{\pgfqpoint{4.021018in}{4.390625in}}%
\pgfpathmoveto{\pgfqpoint{4.016760in}{4.394883in}}%
\pgfpathlineto{\pgfqpoint{4.016760in}{4.394883in}}%
\pgfpathlineto{\pgfqpoint{4.016760in}{4.399141in}}%
\pgfpathlineto{\pgfqpoint{4.021018in}{4.399141in}}%
\pgfpathlineto{\pgfqpoint{4.021018in}{4.394883in}}%
\pgfpathmoveto{\pgfqpoint{4.012502in}{4.399141in}}%
\pgfpathlineto{\pgfqpoint{4.012502in}{4.399141in}}%
\pgfpathlineto{\pgfqpoint{4.012502in}{4.403399in}}%
\pgfpathlineto{\pgfqpoint{4.016760in}{4.403399in}}%
\pgfpathlineto{\pgfqpoint{4.016760in}{4.399141in}}%
\pgfpathmoveto{\pgfqpoint{4.012502in}{4.403399in}}%
\pgfpathlineto{\pgfqpoint{4.012502in}{4.403399in}}%
\pgfpathlineto{\pgfqpoint{4.012502in}{4.407657in}}%
\pgfpathlineto{\pgfqpoint{4.016760in}{4.407657in}}%
\pgfpathlineto{\pgfqpoint{4.016760in}{4.403399in}}%
\pgfpathmoveto{\pgfqpoint{4.016760in}{4.399141in}}%
\pgfpathlineto{\pgfqpoint{4.016760in}{4.399141in}}%
\pgfpathlineto{\pgfqpoint{4.016760in}{4.403399in}}%
\pgfpathlineto{\pgfqpoint{4.021018in}{4.403399in}}%
\pgfpathlineto{\pgfqpoint{4.021018in}{4.399141in}}%
\pgfpathmoveto{\pgfqpoint{4.016760in}{4.403399in}}%
\pgfpathlineto{\pgfqpoint{4.016760in}{4.403399in}}%
\pgfpathlineto{\pgfqpoint{4.016760in}{4.407657in}}%
\pgfpathlineto{\pgfqpoint{4.021018in}{4.407657in}}%
\pgfpathlineto{\pgfqpoint{4.021018in}{4.403399in}}%
\pgfpathmoveto{\pgfqpoint{4.012502in}{4.407657in}}%
\pgfpathlineto{\pgfqpoint{4.012502in}{4.407657in}}%
\pgfpathlineto{\pgfqpoint{4.012502in}{4.411915in}}%
\pgfpathlineto{\pgfqpoint{4.016760in}{4.411915in}}%
\pgfpathlineto{\pgfqpoint{4.016760in}{4.407657in}}%
\pgfpathmoveto{\pgfqpoint{4.012502in}{4.411915in}}%
\pgfpathlineto{\pgfqpoint{4.012502in}{4.411915in}}%
\pgfpathlineto{\pgfqpoint{4.012502in}{4.416173in}}%
\pgfpathlineto{\pgfqpoint{4.016760in}{4.416173in}}%
\pgfpathlineto{\pgfqpoint{4.016760in}{4.411915in}}%
\pgfpathmoveto{\pgfqpoint{4.016760in}{4.407657in}}%
\pgfpathlineto{\pgfqpoint{4.016760in}{4.407657in}}%
\pgfpathlineto{\pgfqpoint{4.016760in}{4.411915in}}%
\pgfpathlineto{\pgfqpoint{4.021018in}{4.411915in}}%
\pgfpathlineto{\pgfqpoint{4.021018in}{4.407657in}}%
\pgfpathmoveto{\pgfqpoint{4.016760in}{4.411915in}}%
\pgfpathlineto{\pgfqpoint{4.016760in}{4.411915in}}%
\pgfpathlineto{\pgfqpoint{4.016760in}{4.416173in}}%
\pgfpathlineto{\pgfqpoint{4.021018in}{4.416173in}}%
\pgfpathlineto{\pgfqpoint{4.021018in}{4.411915in}}%
\pgfpathmoveto{\pgfqpoint{4.012502in}{4.416173in}}%
\pgfpathlineto{\pgfqpoint{4.012502in}{4.416173in}}%
\pgfpathlineto{\pgfqpoint{4.012502in}{4.420431in}}%
\pgfpathlineto{\pgfqpoint{4.016760in}{4.420431in}}%
\pgfpathlineto{\pgfqpoint{4.016760in}{4.416173in}}%
\pgfpathmoveto{\pgfqpoint{4.012502in}{4.420431in}}%
\pgfpathlineto{\pgfqpoint{4.012502in}{4.420431in}}%
\pgfpathlineto{\pgfqpoint{4.012502in}{4.424689in}}%
\pgfpathlineto{\pgfqpoint{4.016760in}{4.424689in}}%
\pgfpathlineto{\pgfqpoint{4.016760in}{4.420431in}}%
\pgfpathmoveto{\pgfqpoint{4.016760in}{4.416173in}}%
\pgfpathlineto{\pgfqpoint{4.016760in}{4.416173in}}%
\pgfpathlineto{\pgfqpoint{4.016760in}{4.420431in}}%
\pgfpathlineto{\pgfqpoint{4.021018in}{4.420431in}}%
\pgfpathlineto{\pgfqpoint{4.021018in}{4.416173in}}%
\pgfpathmoveto{\pgfqpoint{4.016760in}{4.420431in}}%
\pgfpathlineto{\pgfqpoint{4.016760in}{4.420431in}}%
\pgfpathlineto{\pgfqpoint{4.016760in}{4.424689in}}%
\pgfpathlineto{\pgfqpoint{4.021018in}{4.424689in}}%
\pgfpathlineto{\pgfqpoint{4.021018in}{4.420431in}}%
\pgfpathmoveto{\pgfqpoint{4.012502in}{4.424689in}}%
\pgfpathlineto{\pgfqpoint{4.012502in}{4.424689in}}%
\pgfpathlineto{\pgfqpoint{4.012502in}{4.428947in}}%
\pgfpathlineto{\pgfqpoint{4.016760in}{4.428947in}}%
\pgfpathlineto{\pgfqpoint{4.016760in}{4.424689in}}%
\pgfpathmoveto{\pgfqpoint{4.016760in}{4.424689in}}%
\pgfpathlineto{\pgfqpoint{4.016760in}{4.424689in}}%
\pgfpathlineto{\pgfqpoint{4.016760in}{4.428947in}}%
\pgfpathlineto{\pgfqpoint{4.021018in}{4.428947in}}%
\pgfpathlineto{\pgfqpoint{4.021018in}{4.424689in}}%
\pgfpathmoveto{\pgfqpoint{4.016760in}{4.428947in}}%
\pgfpathlineto{\pgfqpoint{4.016760in}{4.428947in}}%
\pgfpathlineto{\pgfqpoint{4.016760in}{4.433205in}}%
\pgfpathlineto{\pgfqpoint{4.021018in}{4.433205in}}%
\pgfpathlineto{\pgfqpoint{4.021018in}{4.428947in}}%
\pgfpathmoveto{\pgfqpoint{4.016760in}{4.433205in}}%
\pgfpathlineto{\pgfqpoint{4.016760in}{4.433205in}}%
\pgfpathlineto{\pgfqpoint{4.016760in}{4.437462in}}%
\pgfpathlineto{\pgfqpoint{4.021018in}{4.437462in}}%
\pgfpathlineto{\pgfqpoint{4.021018in}{4.433205in}}%
\pgfpathmoveto{\pgfqpoint{4.016760in}{4.437462in}}%
\pgfpathlineto{\pgfqpoint{4.016760in}{4.437462in}}%
\pgfpathlineto{\pgfqpoint{4.016760in}{4.441720in}}%
\pgfpathlineto{\pgfqpoint{4.021018in}{4.441720in}}%
\pgfpathlineto{\pgfqpoint{4.021018in}{4.437462in}}%
\pgfpathmoveto{\pgfqpoint{4.021018in}{4.437462in}}%
\pgfpathlineto{\pgfqpoint{4.021018in}{4.437462in}}%
\pgfpathlineto{\pgfqpoint{4.021018in}{4.441720in}}%
\pgfpathlineto{\pgfqpoint{4.025276in}{4.441720in}}%
\pgfpathlineto{\pgfqpoint{4.025276in}{4.437462in}}%
\pgfpathmoveto{\pgfqpoint{4.016760in}{4.441720in}}%
\pgfpathlineto{\pgfqpoint{4.016760in}{4.441720in}}%
\pgfpathlineto{\pgfqpoint{4.016760in}{4.445978in}}%
\pgfpathlineto{\pgfqpoint{4.021018in}{4.445978in}}%
\pgfpathlineto{\pgfqpoint{4.021018in}{4.441720in}}%
\pgfpathmoveto{\pgfqpoint{4.016760in}{4.445978in}}%
\pgfpathlineto{\pgfqpoint{4.016760in}{4.445978in}}%
\pgfpathlineto{\pgfqpoint{4.016760in}{4.450236in}}%
\pgfpathlineto{\pgfqpoint{4.021018in}{4.450236in}}%
\pgfpathlineto{\pgfqpoint{4.021018in}{4.445978in}}%
\pgfpathmoveto{\pgfqpoint{4.016760in}{4.450236in}}%
\pgfpathlineto{\pgfqpoint{4.016760in}{4.450236in}}%
\pgfpathlineto{\pgfqpoint{4.016760in}{4.454494in}}%
\pgfpathlineto{\pgfqpoint{4.021018in}{4.454494in}}%
\pgfpathlineto{\pgfqpoint{4.021018in}{4.450236in}}%
\pgfpathmoveto{\pgfqpoint{4.016760in}{4.454494in}}%
\pgfpathlineto{\pgfqpoint{4.016760in}{4.454494in}}%
\pgfpathlineto{\pgfqpoint{4.016760in}{4.458752in}}%
\pgfpathlineto{\pgfqpoint{4.021018in}{4.458752in}}%
\pgfpathlineto{\pgfqpoint{4.021018in}{4.454494in}}%
\pgfpathmoveto{\pgfqpoint{4.021018in}{4.441720in}}%
\pgfpathlineto{\pgfqpoint{4.021018in}{4.441720in}}%
\pgfpathlineto{\pgfqpoint{4.021018in}{4.445978in}}%
\pgfpathlineto{\pgfqpoint{4.025276in}{4.445978in}}%
\pgfpathlineto{\pgfqpoint{4.025276in}{4.441720in}}%
\pgfpathmoveto{\pgfqpoint{4.021018in}{4.445978in}}%
\pgfpathlineto{\pgfqpoint{4.021018in}{4.445978in}}%
\pgfpathlineto{\pgfqpoint{4.021018in}{4.450236in}}%
\pgfpathlineto{\pgfqpoint{4.025276in}{4.450236in}}%
\pgfpathlineto{\pgfqpoint{4.025276in}{4.445978in}}%
\pgfpathmoveto{\pgfqpoint{4.021018in}{4.450236in}}%
\pgfpathlineto{\pgfqpoint{4.021018in}{4.450236in}}%
\pgfpathlineto{\pgfqpoint{4.021018in}{4.454494in}}%
\pgfpathlineto{\pgfqpoint{4.025276in}{4.454494in}}%
\pgfpathlineto{\pgfqpoint{4.025276in}{4.450236in}}%
\pgfpathmoveto{\pgfqpoint{4.021018in}{4.454494in}}%
\pgfpathlineto{\pgfqpoint{4.021018in}{4.454494in}}%
\pgfpathlineto{\pgfqpoint{4.021018in}{4.458752in}}%
\pgfpathlineto{\pgfqpoint{4.025276in}{4.458752in}}%
\pgfpathlineto{\pgfqpoint{4.025276in}{4.454494in}}%
\pgfpathmoveto{\pgfqpoint{4.016760in}{4.458752in}}%
\pgfpathlineto{\pgfqpoint{4.016760in}{4.458752in}}%
\pgfpathlineto{\pgfqpoint{4.016760in}{4.463010in}}%
\pgfpathlineto{\pgfqpoint{4.021018in}{4.463010in}}%
\pgfpathlineto{\pgfqpoint{4.021018in}{4.458752in}}%
\pgfpathmoveto{\pgfqpoint{4.016760in}{4.463010in}}%
\pgfpathlineto{\pgfqpoint{4.016760in}{4.463010in}}%
\pgfpathlineto{\pgfqpoint{4.016760in}{4.467268in}}%
\pgfpathlineto{\pgfqpoint{4.021018in}{4.467268in}}%
\pgfpathlineto{\pgfqpoint{4.021018in}{4.463010in}}%
\pgfpathmoveto{\pgfqpoint{4.016760in}{4.467268in}}%
\pgfpathlineto{\pgfqpoint{4.016760in}{4.467268in}}%
\pgfpathlineto{\pgfqpoint{4.016760in}{4.471525in}}%
\pgfpathlineto{\pgfqpoint{4.021018in}{4.471525in}}%
\pgfpathlineto{\pgfqpoint{4.021018in}{4.467268in}}%
\pgfpathmoveto{\pgfqpoint{4.016760in}{4.471525in}}%
\pgfpathlineto{\pgfqpoint{4.016760in}{4.471525in}}%
\pgfpathlineto{\pgfqpoint{4.016760in}{4.475783in}}%
\pgfpathlineto{\pgfqpoint{4.021018in}{4.475783in}}%
\pgfpathlineto{\pgfqpoint{4.021018in}{4.471525in}}%
\pgfpathmoveto{\pgfqpoint{4.021018in}{4.458752in}}%
\pgfpathlineto{\pgfqpoint{4.021018in}{4.458752in}}%
\pgfpathlineto{\pgfqpoint{4.021018in}{4.463010in}}%
\pgfpathlineto{\pgfqpoint{4.025276in}{4.463010in}}%
\pgfpathlineto{\pgfqpoint{4.025276in}{4.458752in}}%
\pgfpathmoveto{\pgfqpoint{4.021018in}{4.463010in}}%
\pgfpathlineto{\pgfqpoint{4.021018in}{4.463010in}}%
\pgfpathlineto{\pgfqpoint{4.021018in}{4.467268in}}%
\pgfpathlineto{\pgfqpoint{4.025276in}{4.467268in}}%
\pgfpathlineto{\pgfqpoint{4.025276in}{4.463010in}}%
\pgfpathmoveto{\pgfqpoint{4.021018in}{4.467268in}}%
\pgfpathlineto{\pgfqpoint{4.021018in}{4.467268in}}%
\pgfpathlineto{\pgfqpoint{4.021018in}{4.471525in}}%
\pgfpathlineto{\pgfqpoint{4.025276in}{4.471525in}}%
\pgfpathlineto{\pgfqpoint{4.025276in}{4.467268in}}%
\pgfpathmoveto{\pgfqpoint{4.021018in}{4.471525in}}%
\pgfpathlineto{\pgfqpoint{4.021018in}{4.471525in}}%
\pgfpathlineto{\pgfqpoint{4.021018in}{4.475783in}}%
\pgfpathlineto{\pgfqpoint{4.025276in}{4.475783in}}%
\pgfpathlineto{\pgfqpoint{4.025276in}{4.471525in}}%
\pgfpathmoveto{\pgfqpoint{4.021018in}{4.475783in}}%
\pgfpathlineto{\pgfqpoint{4.021018in}{4.475783in}}%
\pgfpathlineto{\pgfqpoint{4.021018in}{4.480041in}}%
\pgfpathlineto{\pgfqpoint{4.025276in}{4.480041in}}%
\pgfpathlineto{\pgfqpoint{4.025276in}{4.475783in}}%
\pgfpathmoveto{\pgfqpoint{4.021018in}{4.480041in}}%
\pgfpathlineto{\pgfqpoint{4.021018in}{4.480041in}}%
\pgfpathlineto{\pgfqpoint{4.021018in}{4.484298in}}%
\pgfpathlineto{\pgfqpoint{4.025276in}{4.484298in}}%
\pgfpathlineto{\pgfqpoint{4.025276in}{4.480041in}}%
\pgfpathmoveto{\pgfqpoint{4.025276in}{4.480041in}}%
\pgfpathlineto{\pgfqpoint{4.025276in}{4.480041in}}%
\pgfpathlineto{\pgfqpoint{4.025276in}{4.484298in}}%
\pgfpathlineto{\pgfqpoint{4.029534in}{4.484298in}}%
\pgfpathlineto{\pgfqpoint{4.029534in}{4.480041in}}%
\pgfpathmoveto{\pgfqpoint{4.021018in}{4.484298in}}%
\pgfpathlineto{\pgfqpoint{4.021018in}{4.484298in}}%
\pgfpathlineto{\pgfqpoint{4.021018in}{4.488556in}}%
\pgfpathlineto{\pgfqpoint{4.025276in}{4.488556in}}%
\pgfpathlineto{\pgfqpoint{4.025276in}{4.484298in}}%
\pgfpathmoveto{\pgfqpoint{4.021018in}{4.488556in}}%
\pgfpathlineto{\pgfqpoint{4.021018in}{4.488556in}}%
\pgfpathlineto{\pgfqpoint{4.021018in}{4.492814in}}%
\pgfpathlineto{\pgfqpoint{4.025276in}{4.492814in}}%
\pgfpathlineto{\pgfqpoint{4.025276in}{4.488556in}}%
\pgfpathmoveto{\pgfqpoint{4.025276in}{4.484298in}}%
\pgfpathlineto{\pgfqpoint{4.025276in}{4.484298in}}%
\pgfpathlineto{\pgfqpoint{4.025276in}{4.488556in}}%
\pgfpathlineto{\pgfqpoint{4.029534in}{4.488556in}}%
\pgfpathlineto{\pgfqpoint{4.029534in}{4.484298in}}%
\pgfpathmoveto{\pgfqpoint{4.025276in}{4.488556in}}%
\pgfpathlineto{\pgfqpoint{4.025276in}{4.488556in}}%
\pgfpathlineto{\pgfqpoint{4.025276in}{4.492814in}}%
\pgfpathlineto{\pgfqpoint{4.029534in}{4.492814in}}%
\pgfpathlineto{\pgfqpoint{4.029534in}{4.488556in}}%
\pgfpathmoveto{\pgfqpoint{4.021018in}{4.492814in}}%
\pgfpathlineto{\pgfqpoint{4.021018in}{4.492814in}}%
\pgfpathlineto{\pgfqpoint{4.021018in}{4.497071in}}%
\pgfpathlineto{\pgfqpoint{4.025276in}{4.497071in}}%
\pgfpathlineto{\pgfqpoint{4.025276in}{4.492814in}}%
\pgfpathmoveto{\pgfqpoint{4.021018in}{4.497071in}}%
\pgfpathlineto{\pgfqpoint{4.021018in}{4.497071in}}%
\pgfpathlineto{\pgfqpoint{4.021018in}{4.501329in}}%
\pgfpathlineto{\pgfqpoint{4.025276in}{4.501329in}}%
\pgfpathlineto{\pgfqpoint{4.025276in}{4.497071in}}%
\pgfpathmoveto{\pgfqpoint{4.025276in}{4.492814in}}%
\pgfpathlineto{\pgfqpoint{4.025276in}{4.492814in}}%
\pgfpathlineto{\pgfqpoint{4.025276in}{4.497071in}}%
\pgfpathlineto{\pgfqpoint{4.029534in}{4.497071in}}%
\pgfpathlineto{\pgfqpoint{4.029534in}{4.492814in}}%
\pgfpathmoveto{\pgfqpoint{4.025276in}{4.497071in}}%
\pgfpathlineto{\pgfqpoint{4.025276in}{4.497071in}}%
\pgfpathlineto{\pgfqpoint{4.025276in}{4.501329in}}%
\pgfpathlineto{\pgfqpoint{4.029534in}{4.501329in}}%
\pgfpathlineto{\pgfqpoint{4.029534in}{4.497071in}}%
\pgfpathmoveto{\pgfqpoint{4.021018in}{4.501329in}}%
\pgfpathlineto{\pgfqpoint{4.021018in}{4.501329in}}%
\pgfpathlineto{\pgfqpoint{4.021018in}{4.505587in}}%
\pgfpathlineto{\pgfqpoint{4.025276in}{4.505587in}}%
\pgfpathlineto{\pgfqpoint{4.025276in}{4.501329in}}%
\pgfpathmoveto{\pgfqpoint{4.021018in}{4.505587in}}%
\pgfpathlineto{\pgfqpoint{4.021018in}{4.505587in}}%
\pgfpathlineto{\pgfqpoint{4.021018in}{4.509845in}}%
\pgfpathlineto{\pgfqpoint{4.025276in}{4.509845in}}%
\pgfpathlineto{\pgfqpoint{4.025276in}{4.505587in}}%
\pgfpathmoveto{\pgfqpoint{4.025276in}{4.501329in}}%
\pgfpathlineto{\pgfqpoint{4.025276in}{4.501329in}}%
\pgfpathlineto{\pgfqpoint{4.025276in}{4.505587in}}%
\pgfpathlineto{\pgfqpoint{4.029534in}{4.505587in}}%
\pgfpathlineto{\pgfqpoint{4.029534in}{4.501329in}}%
\pgfpathmoveto{\pgfqpoint{4.025276in}{4.505587in}}%
\pgfpathlineto{\pgfqpoint{4.025276in}{4.505587in}}%
\pgfpathlineto{\pgfqpoint{4.025276in}{4.509845in}}%
\pgfpathlineto{\pgfqpoint{4.029534in}{4.509845in}}%
\pgfpathlineto{\pgfqpoint{4.029534in}{4.505587in}}%
\pgfpathmoveto{\pgfqpoint{4.021018in}{4.509845in}}%
\pgfpathlineto{\pgfqpoint{4.021018in}{4.509845in}}%
\pgfpathlineto{\pgfqpoint{4.021018in}{4.514102in}}%
\pgfpathlineto{\pgfqpoint{4.025276in}{4.514102in}}%
\pgfpathlineto{\pgfqpoint{4.025276in}{4.509845in}}%
\pgfpathmoveto{\pgfqpoint{4.021018in}{4.514102in}}%
\pgfpathlineto{\pgfqpoint{4.021018in}{4.514102in}}%
\pgfpathlineto{\pgfqpoint{4.021018in}{4.518360in}}%
\pgfpathlineto{\pgfqpoint{4.025276in}{4.518360in}}%
\pgfpathlineto{\pgfqpoint{4.025276in}{4.514102in}}%
\pgfpathmoveto{\pgfqpoint{4.025276in}{4.509845in}}%
\pgfpathlineto{\pgfqpoint{4.025276in}{4.509845in}}%
\pgfpathlineto{\pgfqpoint{4.025276in}{4.514102in}}%
\pgfpathlineto{\pgfqpoint{4.029534in}{4.514102in}}%
\pgfpathlineto{\pgfqpoint{4.029534in}{4.509845in}}%
\pgfpathmoveto{\pgfqpoint{4.025276in}{4.514102in}}%
\pgfpathlineto{\pgfqpoint{4.025276in}{4.514102in}}%
\pgfpathlineto{\pgfqpoint{4.025276in}{4.518360in}}%
\pgfpathlineto{\pgfqpoint{4.029534in}{4.518360in}}%
\pgfpathlineto{\pgfqpoint{4.029534in}{4.514102in}}%
\pgfpathmoveto{\pgfqpoint{4.025276in}{4.518360in}}%
\pgfpathlineto{\pgfqpoint{4.025276in}{4.518360in}}%
\pgfpathlineto{\pgfqpoint{4.025276in}{4.522618in}}%
\pgfpathlineto{\pgfqpoint{4.029534in}{4.522618in}}%
\pgfpathlineto{\pgfqpoint{4.029534in}{4.518360in}}%
\pgfpathmoveto{\pgfqpoint{4.025276in}{4.522618in}}%
\pgfpathlineto{\pgfqpoint{4.025276in}{4.522618in}}%
\pgfpathlineto{\pgfqpoint{4.025276in}{4.526875in}}%
\pgfpathlineto{\pgfqpoint{4.029534in}{4.526875in}}%
\pgfpathlineto{\pgfqpoint{4.029534in}{4.522618in}}%
\pgfpathmoveto{\pgfqpoint{4.025276in}{4.526875in}}%
\pgfpathlineto{\pgfqpoint{4.025276in}{4.526875in}}%
\pgfpathlineto{\pgfqpoint{4.025276in}{4.531133in}}%
\pgfpathlineto{\pgfqpoint{4.029534in}{4.531133in}}%
\pgfpathlineto{\pgfqpoint{4.029534in}{4.526875in}}%
\pgfpathmoveto{\pgfqpoint{4.025276in}{4.531133in}}%
\pgfpathlineto{\pgfqpoint{4.025276in}{4.531133in}}%
\pgfpathlineto{\pgfqpoint{4.025276in}{4.535391in}}%
\pgfpathlineto{\pgfqpoint{4.029534in}{4.535391in}}%
\pgfpathlineto{\pgfqpoint{4.029534in}{4.531133in}}%
\pgfpathmoveto{\pgfqpoint{4.025276in}{4.535391in}}%
\pgfpathlineto{\pgfqpoint{4.025276in}{4.535391in}}%
\pgfpathlineto{\pgfqpoint{4.025276in}{4.539648in}}%
\pgfpathlineto{\pgfqpoint{4.029534in}{4.539648in}}%
\pgfpathlineto{\pgfqpoint{4.029534in}{4.535391in}}%
\pgfpathmoveto{\pgfqpoint{4.025276in}{4.539648in}}%
\pgfpathlineto{\pgfqpoint{4.025276in}{4.539648in}}%
\pgfpathlineto{\pgfqpoint{4.025276in}{4.543906in}}%
\pgfpathlineto{\pgfqpoint{4.029534in}{4.543906in}}%
\pgfpathlineto{\pgfqpoint{4.029534in}{4.539648in}}%
\pgfpathmoveto{\pgfqpoint{4.025276in}{4.543906in}}%
\pgfpathlineto{\pgfqpoint{4.025276in}{4.543906in}}%
\pgfpathlineto{\pgfqpoint{4.025276in}{4.548164in}}%
\pgfpathlineto{\pgfqpoint{4.029534in}{4.548164in}}%
\pgfpathlineto{\pgfqpoint{4.029534in}{4.543906in}}%
\pgfpathmoveto{\pgfqpoint{4.025276in}{4.548164in}}%
\pgfpathlineto{\pgfqpoint{4.025276in}{4.548164in}}%
\pgfpathlineto{\pgfqpoint{4.025276in}{4.552421in}}%
\pgfpathlineto{\pgfqpoint{4.029534in}{4.552421in}}%
\pgfpathlineto{\pgfqpoint{4.029534in}{4.548164in}}%
\pgfpathmoveto{\pgfqpoint{4.025276in}{4.552421in}}%
\pgfpathlineto{\pgfqpoint{4.025276in}{4.552421in}}%
\pgfpathlineto{\pgfqpoint{4.025276in}{4.556679in}}%
\pgfpathlineto{\pgfqpoint{4.029534in}{4.556679in}}%
\pgfpathlineto{\pgfqpoint{4.029534in}{4.552421in}}%
\pgfpathmoveto{\pgfqpoint{4.025276in}{4.556679in}}%
\pgfpathlineto{\pgfqpoint{4.025276in}{4.556679in}}%
\pgfpathlineto{\pgfqpoint{4.025276in}{4.560937in}}%
\pgfpathlineto{\pgfqpoint{4.029534in}{4.560937in}}%
\pgfpathlineto{\pgfqpoint{4.029534in}{4.556679in}}%
\pgfpathmoveto{\pgfqpoint{4.029534in}{4.526875in}}%
\pgfpathlineto{\pgfqpoint{4.029534in}{4.526875in}}%
\pgfpathlineto{\pgfqpoint{4.029534in}{4.531133in}}%
\pgfpathlineto{\pgfqpoint{4.033792in}{4.531133in}}%
\pgfpathlineto{\pgfqpoint{4.033792in}{4.526875in}}%
\pgfpathmoveto{\pgfqpoint{4.029534in}{4.531133in}}%
\pgfpathlineto{\pgfqpoint{4.029534in}{4.531133in}}%
\pgfpathlineto{\pgfqpoint{4.029534in}{4.535391in}}%
\pgfpathlineto{\pgfqpoint{4.033792in}{4.535391in}}%
\pgfpathlineto{\pgfqpoint{4.033792in}{4.531133in}}%
\pgfpathmoveto{\pgfqpoint{4.029534in}{4.535391in}}%
\pgfpathlineto{\pgfqpoint{4.029534in}{4.535391in}}%
\pgfpathlineto{\pgfqpoint{4.029534in}{4.539648in}}%
\pgfpathlineto{\pgfqpoint{4.033792in}{4.539648in}}%
\pgfpathlineto{\pgfqpoint{4.033792in}{4.535391in}}%
\pgfpathmoveto{\pgfqpoint{4.029534in}{4.539648in}}%
\pgfpathlineto{\pgfqpoint{4.029534in}{4.539648in}}%
\pgfpathlineto{\pgfqpoint{4.029534in}{4.543906in}}%
\pgfpathlineto{\pgfqpoint{4.033792in}{4.543906in}}%
\pgfpathlineto{\pgfqpoint{4.033792in}{4.539648in}}%
\pgfpathmoveto{\pgfqpoint{4.029534in}{4.543906in}}%
\pgfpathlineto{\pgfqpoint{4.029534in}{4.543906in}}%
\pgfpathlineto{\pgfqpoint{4.029534in}{4.548164in}}%
\pgfpathlineto{\pgfqpoint{4.033792in}{4.548164in}}%
\pgfpathlineto{\pgfqpoint{4.033792in}{4.543906in}}%
\pgfpathmoveto{\pgfqpoint{4.029534in}{4.548164in}}%
\pgfpathlineto{\pgfqpoint{4.029534in}{4.548164in}}%
\pgfpathlineto{\pgfqpoint{4.029534in}{4.552421in}}%
\pgfpathlineto{\pgfqpoint{4.033792in}{4.552421in}}%
\pgfpathlineto{\pgfqpoint{4.033792in}{4.548164in}}%
\pgfpathmoveto{\pgfqpoint{4.029534in}{4.552421in}}%
\pgfpathlineto{\pgfqpoint{4.029534in}{4.552421in}}%
\pgfpathlineto{\pgfqpoint{4.029534in}{4.556679in}}%
\pgfpathlineto{\pgfqpoint{4.033792in}{4.556679in}}%
\pgfpathlineto{\pgfqpoint{4.033792in}{4.552421in}}%
\pgfpathmoveto{\pgfqpoint{4.029534in}{4.556679in}}%
\pgfpathlineto{\pgfqpoint{4.029534in}{4.556679in}}%
\pgfpathlineto{\pgfqpoint{4.029534in}{4.560937in}}%
\pgfpathlineto{\pgfqpoint{4.033792in}{4.560937in}}%
\pgfpathlineto{\pgfqpoint{4.033792in}{4.556679in}}%
\pgfpathmoveto{\pgfqpoint{4.025276in}{4.560937in}}%
\pgfpathlineto{\pgfqpoint{4.025276in}{4.560937in}}%
\pgfpathlineto{\pgfqpoint{4.025276in}{4.565194in}}%
\pgfpathlineto{\pgfqpoint{4.029534in}{4.565194in}}%
\pgfpathlineto{\pgfqpoint{4.029534in}{4.560937in}}%
\pgfpathmoveto{\pgfqpoint{4.029534in}{4.560937in}}%
\pgfpathlineto{\pgfqpoint{4.029534in}{4.560937in}}%
\pgfpathlineto{\pgfqpoint{4.029534in}{4.565194in}}%
\pgfpathlineto{\pgfqpoint{4.033792in}{4.565194in}}%
\pgfpathlineto{\pgfqpoint{4.033792in}{4.560937in}}%
\pgfpathmoveto{\pgfqpoint{4.029534in}{4.565194in}}%
\pgfpathlineto{\pgfqpoint{4.029534in}{4.565194in}}%
\pgfpathlineto{\pgfqpoint{4.029534in}{4.569452in}}%
\pgfpathlineto{\pgfqpoint{4.033792in}{4.569452in}}%
\pgfpathlineto{\pgfqpoint{4.033792in}{4.565194in}}%
\pgfpathmoveto{\pgfqpoint{4.029534in}{4.569452in}}%
\pgfpathlineto{\pgfqpoint{4.029534in}{4.569452in}}%
\pgfpathlineto{\pgfqpoint{4.029534in}{4.573710in}}%
\pgfpathlineto{\pgfqpoint{4.033792in}{4.573710in}}%
\pgfpathlineto{\pgfqpoint{4.033792in}{4.569452in}}%
\pgfpathmoveto{\pgfqpoint{4.029534in}{4.573710in}}%
\pgfpathlineto{\pgfqpoint{4.029534in}{4.573710in}}%
\pgfpathlineto{\pgfqpoint{4.029534in}{4.577967in}}%
\pgfpathlineto{\pgfqpoint{4.033792in}{4.577967in}}%
\pgfpathlineto{\pgfqpoint{4.033792in}{4.573710in}}%
\pgfpathmoveto{\pgfqpoint{4.033792in}{4.573710in}}%
\pgfpathlineto{\pgfqpoint{4.033792in}{4.573710in}}%
\pgfpathlineto{\pgfqpoint{4.033792in}{4.577967in}}%
\pgfpathlineto{\pgfqpoint{4.038050in}{4.577967in}}%
\pgfpathlineto{\pgfqpoint{4.038050in}{4.573710in}}%
\pgfpathmoveto{\pgfqpoint{4.029534in}{4.577967in}}%
\pgfpathlineto{\pgfqpoint{4.029534in}{4.577967in}}%
\pgfpathlineto{\pgfqpoint{4.029534in}{4.582225in}}%
\pgfpathlineto{\pgfqpoint{4.033792in}{4.582225in}}%
\pgfpathlineto{\pgfqpoint{4.033792in}{4.577967in}}%
\pgfpathmoveto{\pgfqpoint{4.029534in}{4.582225in}}%
\pgfpathlineto{\pgfqpoint{4.029534in}{4.582225in}}%
\pgfpathlineto{\pgfqpoint{4.029534in}{4.586483in}}%
\pgfpathlineto{\pgfqpoint{4.033792in}{4.586483in}}%
\pgfpathlineto{\pgfqpoint{4.033792in}{4.582225in}}%
\pgfpathmoveto{\pgfqpoint{4.033792in}{4.577967in}}%
\pgfpathlineto{\pgfqpoint{4.033792in}{4.577967in}}%
\pgfpathlineto{\pgfqpoint{4.033792in}{4.582225in}}%
\pgfpathlineto{\pgfqpoint{4.038050in}{4.582225in}}%
\pgfpathlineto{\pgfqpoint{4.038050in}{4.577967in}}%
\pgfpathmoveto{\pgfqpoint{4.033792in}{4.582225in}}%
\pgfpathlineto{\pgfqpoint{4.033792in}{4.582225in}}%
\pgfpathlineto{\pgfqpoint{4.033792in}{4.586483in}}%
\pgfpathlineto{\pgfqpoint{4.038050in}{4.586483in}}%
\pgfpathlineto{\pgfqpoint{4.038050in}{4.582225in}}%
\pgfpathmoveto{\pgfqpoint{4.029534in}{4.586483in}}%
\pgfpathlineto{\pgfqpoint{4.029534in}{4.586483in}}%
\pgfpathlineto{\pgfqpoint{4.029534in}{4.590741in}}%
\pgfpathlineto{\pgfqpoint{4.033792in}{4.590741in}}%
\pgfpathlineto{\pgfqpoint{4.033792in}{4.586483in}}%
\pgfpathmoveto{\pgfqpoint{4.029534in}{4.590741in}}%
\pgfpathlineto{\pgfqpoint{4.029534in}{4.590741in}}%
\pgfpathlineto{\pgfqpoint{4.029534in}{4.594998in}}%
\pgfpathlineto{\pgfqpoint{4.033792in}{4.594998in}}%
\pgfpathlineto{\pgfqpoint{4.033792in}{4.590741in}}%
\pgfpathmoveto{\pgfqpoint{4.033792in}{4.586483in}}%
\pgfpathlineto{\pgfqpoint{4.033792in}{4.586483in}}%
\pgfpathlineto{\pgfqpoint{4.033792in}{4.590741in}}%
\pgfpathlineto{\pgfqpoint{4.038050in}{4.590741in}}%
\pgfpathlineto{\pgfqpoint{4.038050in}{4.586483in}}%
\pgfpathmoveto{\pgfqpoint{4.033792in}{4.590741in}}%
\pgfpathlineto{\pgfqpoint{4.033792in}{4.590741in}}%
\pgfpathlineto{\pgfqpoint{4.033792in}{4.594998in}}%
\pgfpathlineto{\pgfqpoint{4.038050in}{4.594998in}}%
\pgfpathlineto{\pgfqpoint{4.038050in}{4.590741in}}%
\pgfpathclose%
\pgfusepath{fill}%
\end{pgfscope}%
\begin{pgfscope}%
\pgfpathrectangle{\pgfqpoint{1.049063in}{0.235000in}}{\pgfqpoint{4.360000in}{4.360000in}}%
\pgfusepath{clip}%
\pgfsetrectcap%
\pgfsetroundjoin%
\pgfsetlinewidth{0.803000pt}%
\definecolor{currentstroke}{rgb}{0.690196,0.690196,0.690196}%
\pgfsetstrokecolor{currentstroke}%
\pgfsetdash{}{0pt}%
\pgfpathmoveto{\pgfqpoint{1.049063in}{0.235000in}}%
\pgfpathlineto{\pgfqpoint{1.049063in}{4.595000in}}%
\pgfusepath{stroke}%
\end{pgfscope}%
\begin{pgfscope}%
\pgfsetbuttcap%
\pgfsetroundjoin%
\definecolor{currentfill}{rgb}{0.000000,0.000000,0.000000}%
\pgfsetfillcolor{currentfill}%
\pgfsetlinewidth{0.803000pt}%
\definecolor{currentstroke}{rgb}{0.000000,0.000000,0.000000}%
\pgfsetstrokecolor{currentstroke}%
\pgfsetdash{}{0pt}%
\pgfsys@defobject{currentmarker}{\pgfqpoint{0.000000in}{-0.048611in}}{\pgfqpoint{0.000000in}{0.000000in}}{%
\pgfpathmoveto{\pgfqpoint{0.000000in}{0.000000in}}%
\pgfpathlineto{\pgfqpoint{0.000000in}{-0.048611in}}%
\pgfusepath{stroke,fill}%
}%
\begin{pgfscope}%
\pgfsys@transformshift{1.049063in}{2.415000in}%
\pgfsys@useobject{currentmarker}{}%
\end{pgfscope}%
\end{pgfscope}%
\begin{pgfscope}%
\definecolor{textcolor}{rgb}{0.000000,0.000000,0.000000}%
\pgfsetstrokecolor{textcolor}%
\pgfsetfillcolor{textcolor}%
\pgftext[x=1.049063in,y=2.317778in,,top]{\color{textcolor}\sffamily\fontsize{10.000000}{12.000000}\selectfont −20}%
\end{pgfscope}%
\begin{pgfscope}%
\pgfpathrectangle{\pgfqpoint{1.049063in}{0.235000in}}{\pgfqpoint{4.360000in}{4.360000in}}%
\pgfusepath{clip}%
\pgfsetrectcap%
\pgfsetroundjoin%
\pgfsetlinewidth{0.803000pt}%
\definecolor{currentstroke}{rgb}{0.690196,0.690196,0.690196}%
\pgfsetstrokecolor{currentstroke}%
\pgfsetdash{}{0pt}%
\pgfpathmoveto{\pgfqpoint{1.594062in}{0.235000in}}%
\pgfpathlineto{\pgfqpoint{1.594062in}{4.595000in}}%
\pgfusepath{stroke}%
\end{pgfscope}%
\begin{pgfscope}%
\pgfsetbuttcap%
\pgfsetroundjoin%
\definecolor{currentfill}{rgb}{0.000000,0.000000,0.000000}%
\pgfsetfillcolor{currentfill}%
\pgfsetlinewidth{0.803000pt}%
\definecolor{currentstroke}{rgb}{0.000000,0.000000,0.000000}%
\pgfsetstrokecolor{currentstroke}%
\pgfsetdash{}{0pt}%
\pgfsys@defobject{currentmarker}{\pgfqpoint{0.000000in}{-0.048611in}}{\pgfqpoint{0.000000in}{0.000000in}}{%
\pgfpathmoveto{\pgfqpoint{0.000000in}{0.000000in}}%
\pgfpathlineto{\pgfqpoint{0.000000in}{-0.048611in}}%
\pgfusepath{stroke,fill}%
}%
\begin{pgfscope}%
\pgfsys@transformshift{1.594062in}{2.415000in}%
\pgfsys@useobject{currentmarker}{}%
\end{pgfscope}%
\end{pgfscope}%
\begin{pgfscope}%
\definecolor{textcolor}{rgb}{0.000000,0.000000,0.000000}%
\pgfsetstrokecolor{textcolor}%
\pgfsetfillcolor{textcolor}%
\pgftext[x=1.594062in,y=2.317778in,,top]{\color{textcolor}\sffamily\fontsize{10.000000}{12.000000}\selectfont −15}%
\end{pgfscope}%
\begin{pgfscope}%
\pgfpathrectangle{\pgfqpoint{1.049063in}{0.235000in}}{\pgfqpoint{4.360000in}{4.360000in}}%
\pgfusepath{clip}%
\pgfsetrectcap%
\pgfsetroundjoin%
\pgfsetlinewidth{0.803000pt}%
\definecolor{currentstroke}{rgb}{0.690196,0.690196,0.690196}%
\pgfsetstrokecolor{currentstroke}%
\pgfsetdash{}{0pt}%
\pgfpathmoveto{\pgfqpoint{2.139063in}{0.235000in}}%
\pgfpathlineto{\pgfqpoint{2.139063in}{4.595000in}}%
\pgfusepath{stroke}%
\end{pgfscope}%
\begin{pgfscope}%
\pgfsetbuttcap%
\pgfsetroundjoin%
\definecolor{currentfill}{rgb}{0.000000,0.000000,0.000000}%
\pgfsetfillcolor{currentfill}%
\pgfsetlinewidth{0.803000pt}%
\definecolor{currentstroke}{rgb}{0.000000,0.000000,0.000000}%
\pgfsetstrokecolor{currentstroke}%
\pgfsetdash{}{0pt}%
\pgfsys@defobject{currentmarker}{\pgfqpoint{0.000000in}{-0.048611in}}{\pgfqpoint{0.000000in}{0.000000in}}{%
\pgfpathmoveto{\pgfqpoint{0.000000in}{0.000000in}}%
\pgfpathlineto{\pgfqpoint{0.000000in}{-0.048611in}}%
\pgfusepath{stroke,fill}%
}%
\begin{pgfscope}%
\pgfsys@transformshift{2.139063in}{2.415000in}%
\pgfsys@useobject{currentmarker}{}%
\end{pgfscope}%
\end{pgfscope}%
\begin{pgfscope}%
\definecolor{textcolor}{rgb}{0.000000,0.000000,0.000000}%
\pgfsetstrokecolor{textcolor}%
\pgfsetfillcolor{textcolor}%
\pgftext[x=2.139063in,y=2.317778in,,top]{\color{textcolor}\sffamily\fontsize{10.000000}{12.000000}\selectfont −10}%
\end{pgfscope}%
\begin{pgfscope}%
\pgfpathrectangle{\pgfqpoint{1.049063in}{0.235000in}}{\pgfqpoint{4.360000in}{4.360000in}}%
\pgfusepath{clip}%
\pgfsetrectcap%
\pgfsetroundjoin%
\pgfsetlinewidth{0.803000pt}%
\definecolor{currentstroke}{rgb}{0.690196,0.690196,0.690196}%
\pgfsetstrokecolor{currentstroke}%
\pgfsetdash{}{0pt}%
\pgfpathmoveto{\pgfqpoint{2.684063in}{0.235000in}}%
\pgfpathlineto{\pgfqpoint{2.684063in}{4.595000in}}%
\pgfusepath{stroke}%
\end{pgfscope}%
\begin{pgfscope}%
\pgfsetbuttcap%
\pgfsetroundjoin%
\definecolor{currentfill}{rgb}{0.000000,0.000000,0.000000}%
\pgfsetfillcolor{currentfill}%
\pgfsetlinewidth{0.803000pt}%
\definecolor{currentstroke}{rgb}{0.000000,0.000000,0.000000}%
\pgfsetstrokecolor{currentstroke}%
\pgfsetdash{}{0pt}%
\pgfsys@defobject{currentmarker}{\pgfqpoint{0.000000in}{-0.048611in}}{\pgfqpoint{0.000000in}{0.000000in}}{%
\pgfpathmoveto{\pgfqpoint{0.000000in}{0.000000in}}%
\pgfpathlineto{\pgfqpoint{0.000000in}{-0.048611in}}%
\pgfusepath{stroke,fill}%
}%
\begin{pgfscope}%
\pgfsys@transformshift{2.684063in}{2.415000in}%
\pgfsys@useobject{currentmarker}{}%
\end{pgfscope}%
\end{pgfscope}%
\begin{pgfscope}%
\definecolor{textcolor}{rgb}{0.000000,0.000000,0.000000}%
\pgfsetstrokecolor{textcolor}%
\pgfsetfillcolor{textcolor}%
\pgftext[x=2.684063in,y=2.317778in,,top]{\color{textcolor}\sffamily\fontsize{10.000000}{12.000000}\selectfont −5}%
\end{pgfscope}%
\begin{pgfscope}%
\pgfpathrectangle{\pgfqpoint{1.049063in}{0.235000in}}{\pgfqpoint{4.360000in}{4.360000in}}%
\pgfusepath{clip}%
\pgfsetrectcap%
\pgfsetroundjoin%
\pgfsetlinewidth{0.803000pt}%
\definecolor{currentstroke}{rgb}{0.690196,0.690196,0.690196}%
\pgfsetstrokecolor{currentstroke}%
\pgfsetdash{}{0pt}%
\pgfpathmoveto{\pgfqpoint{3.229062in}{0.235000in}}%
\pgfpathlineto{\pgfqpoint{3.229062in}{4.595000in}}%
\pgfusepath{stroke}%
\end{pgfscope}%
\begin{pgfscope}%
\pgfsetbuttcap%
\pgfsetroundjoin%
\definecolor{currentfill}{rgb}{0.000000,0.000000,0.000000}%
\pgfsetfillcolor{currentfill}%
\pgfsetlinewidth{0.803000pt}%
\definecolor{currentstroke}{rgb}{0.000000,0.000000,0.000000}%
\pgfsetstrokecolor{currentstroke}%
\pgfsetdash{}{0pt}%
\pgfsys@defobject{currentmarker}{\pgfqpoint{0.000000in}{-0.048611in}}{\pgfqpoint{0.000000in}{0.000000in}}{%
\pgfpathmoveto{\pgfqpoint{0.000000in}{0.000000in}}%
\pgfpathlineto{\pgfqpoint{0.000000in}{-0.048611in}}%
\pgfusepath{stroke,fill}%
}%
\begin{pgfscope}%
\pgfsys@transformshift{3.229062in}{2.415000in}%
\pgfsys@useobject{currentmarker}{}%
\end{pgfscope}%
\end{pgfscope}%
\begin{pgfscope}%
\definecolor{textcolor}{rgb}{0.000000,0.000000,0.000000}%
\pgfsetstrokecolor{textcolor}%
\pgfsetfillcolor{textcolor}%
\pgftext[x=3.229062in,y=2.317778in,,top]{\color{textcolor}\sffamily\fontsize{10.000000}{12.000000}\selectfont 0}%
\end{pgfscope}%
\begin{pgfscope}%
\pgfpathrectangle{\pgfqpoint{1.049063in}{0.235000in}}{\pgfqpoint{4.360000in}{4.360000in}}%
\pgfusepath{clip}%
\pgfsetrectcap%
\pgfsetroundjoin%
\pgfsetlinewidth{0.803000pt}%
\definecolor{currentstroke}{rgb}{0.690196,0.690196,0.690196}%
\pgfsetstrokecolor{currentstroke}%
\pgfsetdash{}{0pt}%
\pgfpathmoveto{\pgfqpoint{3.774062in}{0.235000in}}%
\pgfpathlineto{\pgfqpoint{3.774062in}{4.595000in}}%
\pgfusepath{stroke}%
\end{pgfscope}%
\begin{pgfscope}%
\pgfsetbuttcap%
\pgfsetroundjoin%
\definecolor{currentfill}{rgb}{0.000000,0.000000,0.000000}%
\pgfsetfillcolor{currentfill}%
\pgfsetlinewidth{0.803000pt}%
\definecolor{currentstroke}{rgb}{0.000000,0.000000,0.000000}%
\pgfsetstrokecolor{currentstroke}%
\pgfsetdash{}{0pt}%
\pgfsys@defobject{currentmarker}{\pgfqpoint{0.000000in}{-0.048611in}}{\pgfqpoint{0.000000in}{0.000000in}}{%
\pgfpathmoveto{\pgfqpoint{0.000000in}{0.000000in}}%
\pgfpathlineto{\pgfqpoint{0.000000in}{-0.048611in}}%
\pgfusepath{stroke,fill}%
}%
\begin{pgfscope}%
\pgfsys@transformshift{3.774062in}{2.415000in}%
\pgfsys@useobject{currentmarker}{}%
\end{pgfscope}%
\end{pgfscope}%
\begin{pgfscope}%
\definecolor{textcolor}{rgb}{0.000000,0.000000,0.000000}%
\pgfsetstrokecolor{textcolor}%
\pgfsetfillcolor{textcolor}%
\pgftext[x=3.774062in,y=2.317778in,,top]{\color{textcolor}\sffamily\fontsize{10.000000}{12.000000}\selectfont 5}%
\end{pgfscope}%
\begin{pgfscope}%
\pgfpathrectangle{\pgfqpoint{1.049063in}{0.235000in}}{\pgfqpoint{4.360000in}{4.360000in}}%
\pgfusepath{clip}%
\pgfsetrectcap%
\pgfsetroundjoin%
\pgfsetlinewidth{0.803000pt}%
\definecolor{currentstroke}{rgb}{0.690196,0.690196,0.690196}%
\pgfsetstrokecolor{currentstroke}%
\pgfsetdash{}{0pt}%
\pgfpathmoveto{\pgfqpoint{4.319063in}{0.235000in}}%
\pgfpathlineto{\pgfqpoint{4.319063in}{4.595000in}}%
\pgfusepath{stroke}%
\end{pgfscope}%
\begin{pgfscope}%
\pgfsetbuttcap%
\pgfsetroundjoin%
\definecolor{currentfill}{rgb}{0.000000,0.000000,0.000000}%
\pgfsetfillcolor{currentfill}%
\pgfsetlinewidth{0.803000pt}%
\definecolor{currentstroke}{rgb}{0.000000,0.000000,0.000000}%
\pgfsetstrokecolor{currentstroke}%
\pgfsetdash{}{0pt}%
\pgfsys@defobject{currentmarker}{\pgfqpoint{0.000000in}{-0.048611in}}{\pgfqpoint{0.000000in}{0.000000in}}{%
\pgfpathmoveto{\pgfqpoint{0.000000in}{0.000000in}}%
\pgfpathlineto{\pgfqpoint{0.000000in}{-0.048611in}}%
\pgfusepath{stroke,fill}%
}%
\begin{pgfscope}%
\pgfsys@transformshift{4.319063in}{2.415000in}%
\pgfsys@useobject{currentmarker}{}%
\end{pgfscope}%
\end{pgfscope}%
\begin{pgfscope}%
\definecolor{textcolor}{rgb}{0.000000,0.000000,0.000000}%
\pgfsetstrokecolor{textcolor}%
\pgfsetfillcolor{textcolor}%
\pgftext[x=4.319063in,y=2.317778in,,top]{\color{textcolor}\sffamily\fontsize{10.000000}{12.000000}\selectfont 10}%
\end{pgfscope}%
\begin{pgfscope}%
\pgfpathrectangle{\pgfqpoint{1.049063in}{0.235000in}}{\pgfqpoint{4.360000in}{4.360000in}}%
\pgfusepath{clip}%
\pgfsetrectcap%
\pgfsetroundjoin%
\pgfsetlinewidth{0.803000pt}%
\definecolor{currentstroke}{rgb}{0.690196,0.690196,0.690196}%
\pgfsetstrokecolor{currentstroke}%
\pgfsetdash{}{0pt}%
\pgfpathmoveto{\pgfqpoint{4.864063in}{0.235000in}}%
\pgfpathlineto{\pgfqpoint{4.864063in}{4.595000in}}%
\pgfusepath{stroke}%
\end{pgfscope}%
\begin{pgfscope}%
\pgfsetbuttcap%
\pgfsetroundjoin%
\definecolor{currentfill}{rgb}{0.000000,0.000000,0.000000}%
\pgfsetfillcolor{currentfill}%
\pgfsetlinewidth{0.803000pt}%
\definecolor{currentstroke}{rgb}{0.000000,0.000000,0.000000}%
\pgfsetstrokecolor{currentstroke}%
\pgfsetdash{}{0pt}%
\pgfsys@defobject{currentmarker}{\pgfqpoint{0.000000in}{-0.048611in}}{\pgfqpoint{0.000000in}{0.000000in}}{%
\pgfpathmoveto{\pgfqpoint{0.000000in}{0.000000in}}%
\pgfpathlineto{\pgfqpoint{0.000000in}{-0.048611in}}%
\pgfusepath{stroke,fill}%
}%
\begin{pgfscope}%
\pgfsys@transformshift{4.864063in}{2.415000in}%
\pgfsys@useobject{currentmarker}{}%
\end{pgfscope}%
\end{pgfscope}%
\begin{pgfscope}%
\definecolor{textcolor}{rgb}{0.000000,0.000000,0.000000}%
\pgfsetstrokecolor{textcolor}%
\pgfsetfillcolor{textcolor}%
\pgftext[x=4.864063in,y=2.317778in,,top]{\color{textcolor}\sffamily\fontsize{10.000000}{12.000000}\selectfont 15}%
\end{pgfscope}%
\begin{pgfscope}%
\pgfpathrectangle{\pgfqpoint{1.049063in}{0.235000in}}{\pgfqpoint{4.360000in}{4.360000in}}%
\pgfusepath{clip}%
\pgfsetrectcap%
\pgfsetroundjoin%
\pgfsetlinewidth{0.803000pt}%
\definecolor{currentstroke}{rgb}{0.690196,0.690196,0.690196}%
\pgfsetstrokecolor{currentstroke}%
\pgfsetdash{}{0pt}%
\pgfpathmoveto{\pgfqpoint{5.409063in}{0.235000in}}%
\pgfpathlineto{\pgfqpoint{5.409063in}{4.595000in}}%
\pgfusepath{stroke}%
\end{pgfscope}%
\begin{pgfscope}%
\pgfsetbuttcap%
\pgfsetroundjoin%
\definecolor{currentfill}{rgb}{0.000000,0.000000,0.000000}%
\pgfsetfillcolor{currentfill}%
\pgfsetlinewidth{0.803000pt}%
\definecolor{currentstroke}{rgb}{0.000000,0.000000,0.000000}%
\pgfsetstrokecolor{currentstroke}%
\pgfsetdash{}{0pt}%
\pgfsys@defobject{currentmarker}{\pgfqpoint{0.000000in}{-0.048611in}}{\pgfqpoint{0.000000in}{0.000000in}}{%
\pgfpathmoveto{\pgfqpoint{0.000000in}{0.000000in}}%
\pgfpathlineto{\pgfqpoint{0.000000in}{-0.048611in}}%
\pgfusepath{stroke,fill}%
}%
\begin{pgfscope}%
\pgfsys@transformshift{5.409063in}{2.415000in}%
\pgfsys@useobject{currentmarker}{}%
\end{pgfscope}%
\end{pgfscope}%
\begin{pgfscope}%
\definecolor{textcolor}{rgb}{0.000000,0.000000,0.000000}%
\pgfsetstrokecolor{textcolor}%
\pgfsetfillcolor{textcolor}%
\pgftext[x=5.409063in,y=2.317778in,,top]{\color{textcolor}\sffamily\fontsize{10.000000}{12.000000}\selectfont 20}%
\end{pgfscope}%
\begin{pgfscope}%
\definecolor{textcolor}{rgb}{0.000000,0.000000,0.000000}%
\pgfsetstrokecolor{textcolor}%
\pgfsetfillcolor{textcolor}%
\pgftext[x=5.409063in,y=2.127809in,,top]{\color{textcolor}\sffamily\fontsize{10.000000}{12.000000}\selectfont x}%
\end{pgfscope}%
\begin{pgfscope}%
\pgfpathrectangle{\pgfqpoint{1.049063in}{0.235000in}}{\pgfqpoint{4.360000in}{4.360000in}}%
\pgfusepath{clip}%
\pgfsetrectcap%
\pgfsetroundjoin%
\pgfsetlinewidth{0.803000pt}%
\definecolor{currentstroke}{rgb}{0.690196,0.690196,0.690196}%
\pgfsetstrokecolor{currentstroke}%
\pgfsetdash{}{0pt}%
\pgfpathmoveto{\pgfqpoint{1.049063in}{0.235000in}}%
\pgfpathlineto{\pgfqpoint{5.409063in}{0.235000in}}%
\pgfusepath{stroke}%
\end{pgfscope}%
\begin{pgfscope}%
\pgfsetbuttcap%
\pgfsetroundjoin%
\definecolor{currentfill}{rgb}{0.000000,0.000000,0.000000}%
\pgfsetfillcolor{currentfill}%
\pgfsetlinewidth{0.803000pt}%
\definecolor{currentstroke}{rgb}{0.000000,0.000000,0.000000}%
\pgfsetstrokecolor{currentstroke}%
\pgfsetdash{}{0pt}%
\pgfsys@defobject{currentmarker}{\pgfqpoint{-0.048611in}{0.000000in}}{\pgfqpoint{-0.000000in}{0.000000in}}{%
\pgfpathmoveto{\pgfqpoint{-0.000000in}{0.000000in}}%
\pgfpathlineto{\pgfqpoint{-0.048611in}{0.000000in}}%
\pgfusepath{stroke,fill}%
}%
\begin{pgfscope}%
\pgfsys@transformshift{3.229062in}{0.235000in}%
\pgfsys@useobject{currentmarker}{}%
\end{pgfscope}%
\end{pgfscope}%
\begin{pgfscope}%
\definecolor{textcolor}{rgb}{0.000000,0.000000,0.000000}%
\pgfsetstrokecolor{textcolor}%
\pgfsetfillcolor{textcolor}%
\pgftext[x=2.838736in, y=0.182238in, left, base]{\color{textcolor}\sffamily\fontsize{10.000000}{12.000000}\selectfont −20}%
\end{pgfscope}%
\begin{pgfscope}%
\pgfpathrectangle{\pgfqpoint{1.049063in}{0.235000in}}{\pgfqpoint{4.360000in}{4.360000in}}%
\pgfusepath{clip}%
\pgfsetrectcap%
\pgfsetroundjoin%
\pgfsetlinewidth{0.803000pt}%
\definecolor{currentstroke}{rgb}{0.690196,0.690196,0.690196}%
\pgfsetstrokecolor{currentstroke}%
\pgfsetdash{}{0pt}%
\pgfpathmoveto{\pgfqpoint{1.049063in}{0.780000in}}%
\pgfpathlineto{\pgfqpoint{5.409063in}{0.780000in}}%
\pgfusepath{stroke}%
\end{pgfscope}%
\begin{pgfscope}%
\pgfsetbuttcap%
\pgfsetroundjoin%
\definecolor{currentfill}{rgb}{0.000000,0.000000,0.000000}%
\pgfsetfillcolor{currentfill}%
\pgfsetlinewidth{0.803000pt}%
\definecolor{currentstroke}{rgb}{0.000000,0.000000,0.000000}%
\pgfsetstrokecolor{currentstroke}%
\pgfsetdash{}{0pt}%
\pgfsys@defobject{currentmarker}{\pgfqpoint{-0.048611in}{0.000000in}}{\pgfqpoint{-0.000000in}{0.000000in}}{%
\pgfpathmoveto{\pgfqpoint{-0.000000in}{0.000000in}}%
\pgfpathlineto{\pgfqpoint{-0.048611in}{0.000000in}}%
\pgfusepath{stroke,fill}%
}%
\begin{pgfscope}%
\pgfsys@transformshift{3.229062in}{0.780000in}%
\pgfsys@useobject{currentmarker}{}%
\end{pgfscope}%
\end{pgfscope}%
\begin{pgfscope}%
\definecolor{textcolor}{rgb}{0.000000,0.000000,0.000000}%
\pgfsetstrokecolor{textcolor}%
\pgfsetfillcolor{textcolor}%
\pgftext[x=2.838736in, y=0.727238in, left, base]{\color{textcolor}\sffamily\fontsize{10.000000}{12.000000}\selectfont −15}%
\end{pgfscope}%
\begin{pgfscope}%
\pgfpathrectangle{\pgfqpoint{1.049063in}{0.235000in}}{\pgfqpoint{4.360000in}{4.360000in}}%
\pgfusepath{clip}%
\pgfsetrectcap%
\pgfsetroundjoin%
\pgfsetlinewidth{0.803000pt}%
\definecolor{currentstroke}{rgb}{0.690196,0.690196,0.690196}%
\pgfsetstrokecolor{currentstroke}%
\pgfsetdash{}{0pt}%
\pgfpathmoveto{\pgfqpoint{1.049063in}{1.325000in}}%
\pgfpathlineto{\pgfqpoint{5.409063in}{1.325000in}}%
\pgfusepath{stroke}%
\end{pgfscope}%
\begin{pgfscope}%
\pgfsetbuttcap%
\pgfsetroundjoin%
\definecolor{currentfill}{rgb}{0.000000,0.000000,0.000000}%
\pgfsetfillcolor{currentfill}%
\pgfsetlinewidth{0.803000pt}%
\definecolor{currentstroke}{rgb}{0.000000,0.000000,0.000000}%
\pgfsetstrokecolor{currentstroke}%
\pgfsetdash{}{0pt}%
\pgfsys@defobject{currentmarker}{\pgfqpoint{-0.048611in}{0.000000in}}{\pgfqpoint{-0.000000in}{0.000000in}}{%
\pgfpathmoveto{\pgfqpoint{-0.000000in}{0.000000in}}%
\pgfpathlineto{\pgfqpoint{-0.048611in}{0.000000in}}%
\pgfusepath{stroke,fill}%
}%
\begin{pgfscope}%
\pgfsys@transformshift{3.229062in}{1.325000in}%
\pgfsys@useobject{currentmarker}{}%
\end{pgfscope}%
\end{pgfscope}%
\begin{pgfscope}%
\definecolor{textcolor}{rgb}{0.000000,0.000000,0.000000}%
\pgfsetstrokecolor{textcolor}%
\pgfsetfillcolor{textcolor}%
\pgftext[x=2.838736in, y=1.272238in, left, base]{\color{textcolor}\sffamily\fontsize{10.000000}{12.000000}\selectfont −10}%
\end{pgfscope}%
\begin{pgfscope}%
\pgfpathrectangle{\pgfqpoint{1.049063in}{0.235000in}}{\pgfqpoint{4.360000in}{4.360000in}}%
\pgfusepath{clip}%
\pgfsetrectcap%
\pgfsetroundjoin%
\pgfsetlinewidth{0.803000pt}%
\definecolor{currentstroke}{rgb}{0.690196,0.690196,0.690196}%
\pgfsetstrokecolor{currentstroke}%
\pgfsetdash{}{0pt}%
\pgfpathmoveto{\pgfqpoint{1.049063in}{1.870000in}}%
\pgfpathlineto{\pgfqpoint{5.409063in}{1.870000in}}%
\pgfusepath{stroke}%
\end{pgfscope}%
\begin{pgfscope}%
\pgfsetbuttcap%
\pgfsetroundjoin%
\definecolor{currentfill}{rgb}{0.000000,0.000000,0.000000}%
\pgfsetfillcolor{currentfill}%
\pgfsetlinewidth{0.803000pt}%
\definecolor{currentstroke}{rgb}{0.000000,0.000000,0.000000}%
\pgfsetstrokecolor{currentstroke}%
\pgfsetdash{}{0pt}%
\pgfsys@defobject{currentmarker}{\pgfqpoint{-0.048611in}{0.000000in}}{\pgfqpoint{-0.000000in}{0.000000in}}{%
\pgfpathmoveto{\pgfqpoint{-0.000000in}{0.000000in}}%
\pgfpathlineto{\pgfqpoint{-0.048611in}{0.000000in}}%
\pgfusepath{stroke,fill}%
}%
\begin{pgfscope}%
\pgfsys@transformshift{3.229062in}{1.870000in}%
\pgfsys@useobject{currentmarker}{}%
\end{pgfscope}%
\end{pgfscope}%
\begin{pgfscope}%
\definecolor{textcolor}{rgb}{0.000000,0.000000,0.000000}%
\pgfsetstrokecolor{textcolor}%
\pgfsetfillcolor{textcolor}%
\pgftext[x=2.927101in, y=1.817238in, left, base]{\color{textcolor}\sffamily\fontsize{10.000000}{12.000000}\selectfont −5}%
\end{pgfscope}%
\begin{pgfscope}%
\pgfpathrectangle{\pgfqpoint{1.049063in}{0.235000in}}{\pgfqpoint{4.360000in}{4.360000in}}%
\pgfusepath{clip}%
\pgfsetrectcap%
\pgfsetroundjoin%
\pgfsetlinewidth{0.803000pt}%
\definecolor{currentstroke}{rgb}{0.690196,0.690196,0.690196}%
\pgfsetstrokecolor{currentstroke}%
\pgfsetdash{}{0pt}%
\pgfpathmoveto{\pgfqpoint{1.049063in}{2.415000in}}%
\pgfpathlineto{\pgfqpoint{5.409063in}{2.415000in}}%
\pgfusepath{stroke}%
\end{pgfscope}%
\begin{pgfscope}%
\pgfsetbuttcap%
\pgfsetroundjoin%
\definecolor{currentfill}{rgb}{0.000000,0.000000,0.000000}%
\pgfsetfillcolor{currentfill}%
\pgfsetlinewidth{0.803000pt}%
\definecolor{currentstroke}{rgb}{0.000000,0.000000,0.000000}%
\pgfsetstrokecolor{currentstroke}%
\pgfsetdash{}{0pt}%
\pgfsys@defobject{currentmarker}{\pgfqpoint{-0.048611in}{0.000000in}}{\pgfqpoint{-0.000000in}{0.000000in}}{%
\pgfpathmoveto{\pgfqpoint{-0.000000in}{0.000000in}}%
\pgfpathlineto{\pgfqpoint{-0.048611in}{0.000000in}}%
\pgfusepath{stroke,fill}%
}%
\begin{pgfscope}%
\pgfsys@transformshift{3.229062in}{2.415000in}%
\pgfsys@useobject{currentmarker}{}%
\end{pgfscope}%
\end{pgfscope}%
\begin{pgfscope}%
\definecolor{textcolor}{rgb}{0.000000,0.000000,0.000000}%
\pgfsetstrokecolor{textcolor}%
\pgfsetfillcolor{textcolor}%
\pgftext[x=3.043475in, y=2.362238in, left, base]{\color{textcolor}\sffamily\fontsize{10.000000}{12.000000}\selectfont 0}%
\end{pgfscope}%
\begin{pgfscope}%
\pgfpathrectangle{\pgfqpoint{1.049063in}{0.235000in}}{\pgfqpoint{4.360000in}{4.360000in}}%
\pgfusepath{clip}%
\pgfsetrectcap%
\pgfsetroundjoin%
\pgfsetlinewidth{0.803000pt}%
\definecolor{currentstroke}{rgb}{0.690196,0.690196,0.690196}%
\pgfsetstrokecolor{currentstroke}%
\pgfsetdash{}{0pt}%
\pgfpathmoveto{\pgfqpoint{1.049063in}{2.960000in}}%
\pgfpathlineto{\pgfqpoint{5.409063in}{2.960000in}}%
\pgfusepath{stroke}%
\end{pgfscope}%
\begin{pgfscope}%
\pgfsetbuttcap%
\pgfsetroundjoin%
\definecolor{currentfill}{rgb}{0.000000,0.000000,0.000000}%
\pgfsetfillcolor{currentfill}%
\pgfsetlinewidth{0.803000pt}%
\definecolor{currentstroke}{rgb}{0.000000,0.000000,0.000000}%
\pgfsetstrokecolor{currentstroke}%
\pgfsetdash{}{0pt}%
\pgfsys@defobject{currentmarker}{\pgfqpoint{-0.048611in}{0.000000in}}{\pgfqpoint{-0.000000in}{0.000000in}}{%
\pgfpathmoveto{\pgfqpoint{-0.000000in}{0.000000in}}%
\pgfpathlineto{\pgfqpoint{-0.048611in}{0.000000in}}%
\pgfusepath{stroke,fill}%
}%
\begin{pgfscope}%
\pgfsys@transformshift{3.229062in}{2.960000in}%
\pgfsys@useobject{currentmarker}{}%
\end{pgfscope}%
\end{pgfscope}%
\begin{pgfscope}%
\definecolor{textcolor}{rgb}{0.000000,0.000000,0.000000}%
\pgfsetstrokecolor{textcolor}%
\pgfsetfillcolor{textcolor}%
\pgftext[x=3.043475in, y=2.907238in, left, base]{\color{textcolor}\sffamily\fontsize{10.000000}{12.000000}\selectfont 5}%
\end{pgfscope}%
\begin{pgfscope}%
\pgfpathrectangle{\pgfqpoint{1.049063in}{0.235000in}}{\pgfqpoint{4.360000in}{4.360000in}}%
\pgfusepath{clip}%
\pgfsetrectcap%
\pgfsetroundjoin%
\pgfsetlinewidth{0.803000pt}%
\definecolor{currentstroke}{rgb}{0.690196,0.690196,0.690196}%
\pgfsetstrokecolor{currentstroke}%
\pgfsetdash{}{0pt}%
\pgfpathmoveto{\pgfqpoint{1.049063in}{3.505000in}}%
\pgfpathlineto{\pgfqpoint{5.409063in}{3.505000in}}%
\pgfusepath{stroke}%
\end{pgfscope}%
\begin{pgfscope}%
\pgfsetbuttcap%
\pgfsetroundjoin%
\definecolor{currentfill}{rgb}{0.000000,0.000000,0.000000}%
\pgfsetfillcolor{currentfill}%
\pgfsetlinewidth{0.803000pt}%
\definecolor{currentstroke}{rgb}{0.000000,0.000000,0.000000}%
\pgfsetstrokecolor{currentstroke}%
\pgfsetdash{}{0pt}%
\pgfsys@defobject{currentmarker}{\pgfqpoint{-0.048611in}{0.000000in}}{\pgfqpoint{-0.000000in}{0.000000in}}{%
\pgfpathmoveto{\pgfqpoint{-0.000000in}{0.000000in}}%
\pgfpathlineto{\pgfqpoint{-0.048611in}{0.000000in}}%
\pgfusepath{stroke,fill}%
}%
\begin{pgfscope}%
\pgfsys@transformshift{3.229062in}{3.505000in}%
\pgfsys@useobject{currentmarker}{}%
\end{pgfscope}%
\end{pgfscope}%
\begin{pgfscope}%
\definecolor{textcolor}{rgb}{0.000000,0.000000,0.000000}%
\pgfsetstrokecolor{textcolor}%
\pgfsetfillcolor{textcolor}%
\pgftext[x=2.955110in, y=3.452238in, left, base]{\color{textcolor}\sffamily\fontsize{10.000000}{12.000000}\selectfont 10}%
\end{pgfscope}%
\begin{pgfscope}%
\pgfpathrectangle{\pgfqpoint{1.049063in}{0.235000in}}{\pgfqpoint{4.360000in}{4.360000in}}%
\pgfusepath{clip}%
\pgfsetrectcap%
\pgfsetroundjoin%
\pgfsetlinewidth{0.803000pt}%
\definecolor{currentstroke}{rgb}{0.690196,0.690196,0.690196}%
\pgfsetstrokecolor{currentstroke}%
\pgfsetdash{}{0pt}%
\pgfpathmoveto{\pgfqpoint{1.049063in}{4.050000in}}%
\pgfpathlineto{\pgfqpoint{5.409063in}{4.050000in}}%
\pgfusepath{stroke}%
\end{pgfscope}%
\begin{pgfscope}%
\pgfsetbuttcap%
\pgfsetroundjoin%
\definecolor{currentfill}{rgb}{0.000000,0.000000,0.000000}%
\pgfsetfillcolor{currentfill}%
\pgfsetlinewidth{0.803000pt}%
\definecolor{currentstroke}{rgb}{0.000000,0.000000,0.000000}%
\pgfsetstrokecolor{currentstroke}%
\pgfsetdash{}{0pt}%
\pgfsys@defobject{currentmarker}{\pgfqpoint{-0.048611in}{0.000000in}}{\pgfqpoint{-0.000000in}{0.000000in}}{%
\pgfpathmoveto{\pgfqpoint{-0.000000in}{0.000000in}}%
\pgfpathlineto{\pgfqpoint{-0.048611in}{0.000000in}}%
\pgfusepath{stroke,fill}%
}%
\begin{pgfscope}%
\pgfsys@transformshift{3.229062in}{4.050000in}%
\pgfsys@useobject{currentmarker}{}%
\end{pgfscope}%
\end{pgfscope}%
\begin{pgfscope}%
\definecolor{textcolor}{rgb}{0.000000,0.000000,0.000000}%
\pgfsetstrokecolor{textcolor}%
\pgfsetfillcolor{textcolor}%
\pgftext[x=2.955110in, y=3.997238in, left, base]{\color{textcolor}\sffamily\fontsize{10.000000}{12.000000}\selectfont 15}%
\end{pgfscope}%
\begin{pgfscope}%
\pgfpathrectangle{\pgfqpoint{1.049063in}{0.235000in}}{\pgfqpoint{4.360000in}{4.360000in}}%
\pgfusepath{clip}%
\pgfsetrectcap%
\pgfsetroundjoin%
\pgfsetlinewidth{0.803000pt}%
\definecolor{currentstroke}{rgb}{0.690196,0.690196,0.690196}%
\pgfsetstrokecolor{currentstroke}%
\pgfsetdash{}{0pt}%
\pgfpathmoveto{\pgfqpoint{1.049063in}{4.595000in}}%
\pgfpathlineto{\pgfqpoint{5.409063in}{4.595000in}}%
\pgfusepath{stroke}%
\end{pgfscope}%
\begin{pgfscope}%
\pgfsetbuttcap%
\pgfsetroundjoin%
\definecolor{currentfill}{rgb}{0.000000,0.000000,0.000000}%
\pgfsetfillcolor{currentfill}%
\pgfsetlinewidth{0.803000pt}%
\definecolor{currentstroke}{rgb}{0.000000,0.000000,0.000000}%
\pgfsetstrokecolor{currentstroke}%
\pgfsetdash{}{0pt}%
\pgfsys@defobject{currentmarker}{\pgfqpoint{-0.048611in}{0.000000in}}{\pgfqpoint{-0.000000in}{0.000000in}}{%
\pgfpathmoveto{\pgfqpoint{-0.000000in}{0.000000in}}%
\pgfpathlineto{\pgfqpoint{-0.048611in}{0.000000in}}%
\pgfusepath{stroke,fill}%
}%
\begin{pgfscope}%
\pgfsys@transformshift{3.229062in}{4.595000in}%
\pgfsys@useobject{currentmarker}{}%
\end{pgfscope}%
\end{pgfscope}%
\begin{pgfscope}%
\definecolor{textcolor}{rgb}{0.000000,0.000000,0.000000}%
\pgfsetstrokecolor{textcolor}%
\pgfsetfillcolor{textcolor}%
\pgftext[x=2.955110in, y=4.542238in, left, base]{\color{textcolor}\sffamily\fontsize{10.000000}{12.000000}\selectfont 20}%
\end{pgfscope}%
\begin{pgfscope}%
\definecolor{textcolor}{rgb}{0.000000,0.000000,0.000000}%
\pgfsetstrokecolor{textcolor}%
\pgfsetfillcolor{textcolor}%
\pgftext[x=2.783180in,y=4.595000in,,bottom,rotate=90.000000]{\color{textcolor}\sffamily\fontsize{10.000000}{12.000000}\selectfont y}%
\end{pgfscope}%
\begin{pgfscope}%
\pgfsetrectcap%
\pgfsetmiterjoin%
\pgfsetlinewidth{0.803000pt}%
\definecolor{currentstroke}{rgb}{0.000000,0.000000,0.000000}%
\pgfsetstrokecolor{currentstroke}%
\pgfsetdash{}{0pt}%
\pgfpathmoveto{\pgfqpoint{3.229062in}{0.235000in}}%
\pgfpathlineto{\pgfqpoint{3.229062in}{4.595000in}}%
\pgfusepath{stroke}%
\end{pgfscope}%
\begin{pgfscope}%
\pgfsetrectcap%
\pgfsetmiterjoin%
\pgfsetlinewidth{0.000000pt}%
\definecolor{currentstroke}{rgb}{0.000000,0.000000,0.000000}%
\pgfsetstrokecolor{currentstroke}%
\pgfsetstrokeopacity{0.000000}%
\pgfsetdash{}{0pt}%
\pgfpathmoveto{\pgfqpoint{5.409063in}{0.235000in}}%
\pgfpathlineto{\pgfqpoint{5.409063in}{4.595000in}}%
\pgfusepath{}%
\end{pgfscope}%
\begin{pgfscope}%
\pgfsetrectcap%
\pgfsetmiterjoin%
\pgfsetlinewidth{0.803000pt}%
\definecolor{currentstroke}{rgb}{0.000000,0.000000,0.000000}%
\pgfsetstrokecolor{currentstroke}%
\pgfsetdash{}{0pt}%
\pgfpathmoveto{\pgfqpoint{1.049063in}{2.415000in}}%
\pgfpathlineto{\pgfqpoint{5.409063in}{2.415000in}}%
\pgfusepath{stroke}%
\end{pgfscope}%
\begin{pgfscope}%
\pgfsetrectcap%
\pgfsetmiterjoin%
\pgfsetlinewidth{0.000000pt}%
\definecolor{currentstroke}{rgb}{0.000000,0.000000,0.000000}%
\pgfsetstrokecolor{currentstroke}%
\pgfsetstrokeopacity{0.000000}%
\pgfsetdash{}{0pt}%
\pgfpathmoveto{\pgfqpoint{1.049063in}{4.595000in}}%
\pgfpathlineto{\pgfqpoint{5.409063in}{4.595000in}}%
\pgfusepath{}%
\end{pgfscope}%
\end{pgfpicture}%
\makeatother%
\endgroup%
}\end{solution} \part[1] $f(x)=-x^2-5x+6$\begin{solution} \scalebox{.6}{%% Creator: Matplotlib, PGF backend
%%
%% To include the figure in your LaTeX document, write
%%   \input{<filename>.pgf}
%%
%% Make sure the required packages are loaded in your preamble
%%   \usepackage{pgf}
%%
%% and, on pdftex
%%   \usepackage[utf8]{inputenc}\DeclareUnicodeCharacter{2212}{-}
%%
%% or, on luatex and xetex
%%   \usepackage{unicode-math}
%%
%% Figures using additional raster images can only be included by \input if
%% they are in the same directory as the main LaTeX file. For loading figures
%% from other directories you can use the `import` package
%%   \usepackage{import}
%%
%% and then include the figures with
%%   \import{<path to file>}{<filename>.pgf}
%%
%% Matplotlib used the following preamble
%%   \usepackage{fontspec}
%%   \setmainfont{DejaVuSerif.ttf}[Path=/home/hp/Mis_aplicaciones/anaconda3/lib/python3.6/site-packages/matplotlib/mpl-data/fonts/ttf/]
%%   \setsansfont{DejaVuSans.ttf}[Path=/home/hp/Mis_aplicaciones/anaconda3/lib/python3.6/site-packages/matplotlib/mpl-data/fonts/ttf/]
%%   \setmonofont{DejaVuSansMono.ttf}[Path=/home/hp/Mis_aplicaciones/anaconda3/lib/python3.6/site-packages/matplotlib/mpl-data/fonts/ttf/]
%%
\begingroup%
\makeatletter%
\begin{pgfpicture}%
\pgfpathrectangle{\pgfpointorigin}{\pgfqpoint{6.400000in}{4.800000in}}%
\pgfusepath{use as bounding box, clip}%
\begin{pgfscope}%
\pgfsetbuttcap%
\pgfsetmiterjoin%
\definecolor{currentfill}{rgb}{1.000000,1.000000,1.000000}%
\pgfsetfillcolor{currentfill}%
\pgfsetlinewidth{0.000000pt}%
\definecolor{currentstroke}{rgb}{1.000000,1.000000,1.000000}%
\pgfsetstrokecolor{currentstroke}%
\pgfsetdash{}{0pt}%
\pgfpathmoveto{\pgfqpoint{0.000000in}{0.000000in}}%
\pgfpathlineto{\pgfqpoint{6.400000in}{0.000000in}}%
\pgfpathlineto{\pgfqpoint{6.400000in}{4.800000in}}%
\pgfpathlineto{\pgfqpoint{0.000000in}{4.800000in}}%
\pgfpathclose%
\pgfusepath{fill}%
\end{pgfscope}%
\begin{pgfscope}%
\pgfsetbuttcap%
\pgfsetmiterjoin%
\definecolor{currentfill}{rgb}{1.000000,1.000000,1.000000}%
\pgfsetfillcolor{currentfill}%
\pgfsetlinewidth{0.000000pt}%
\definecolor{currentstroke}{rgb}{0.000000,0.000000,0.000000}%
\pgfsetstrokecolor{currentstroke}%
\pgfsetstrokeopacity{0.000000}%
\pgfsetdash{}{0pt}%
\pgfpathmoveto{\pgfqpoint{1.049063in}{0.235000in}}%
\pgfpathlineto{\pgfqpoint{5.409063in}{0.235000in}}%
\pgfpathlineto{\pgfqpoint{5.409063in}{4.595000in}}%
\pgfpathlineto{\pgfqpoint{1.049063in}{4.595000in}}%
\pgfpathclose%
\pgfusepath{fill}%
\end{pgfscope}%
\begin{pgfscope}%
\pgfpathrectangle{\pgfqpoint{1.049063in}{0.235000in}}{\pgfqpoint{4.360000in}{4.360000in}}%
\pgfusepath{clip}%
\pgfsetbuttcap%
\pgfsetmiterjoin%
\definecolor{currentfill}{rgb}{0.000000,0.000000,1.000000}%
\pgfsetfillcolor{currentfill}%
\pgfsetlinewidth{0.000000pt}%
\definecolor{currentstroke}{rgb}{0.000000,0.000000,0.000000}%
\pgfsetstrokecolor{currentstroke}%
\pgfsetstrokeopacity{0.000000}%
\pgfsetdash{}{0pt}%
\pgfpathmoveto{\pgfqpoint{2.334924in}{0.235002in}}%
\pgfpathlineto{\pgfqpoint{2.334924in}{0.239260in}}%
\pgfpathlineto{\pgfqpoint{2.339182in}{0.239260in}}%
\pgfpathlineto{\pgfqpoint{2.339182in}{0.235002in}}%
\pgfpathmoveto{\pgfqpoint{2.334924in}{0.239260in}}%
\pgfpathlineto{\pgfqpoint{2.334924in}{0.239260in}}%
\pgfpathlineto{\pgfqpoint{2.334924in}{0.243518in}}%
\pgfpathlineto{\pgfqpoint{2.339182in}{0.243518in}}%
\pgfpathlineto{\pgfqpoint{2.339182in}{0.239260in}}%
\pgfpathmoveto{\pgfqpoint{2.339182in}{0.235002in}}%
\pgfpathlineto{\pgfqpoint{2.339182in}{0.235002in}}%
\pgfpathlineto{\pgfqpoint{2.339182in}{0.239260in}}%
\pgfpathlineto{\pgfqpoint{2.343440in}{0.239260in}}%
\pgfpathlineto{\pgfqpoint{2.343440in}{0.235002in}}%
\pgfpathmoveto{\pgfqpoint{2.339182in}{0.239260in}}%
\pgfpathlineto{\pgfqpoint{2.339182in}{0.239260in}}%
\pgfpathlineto{\pgfqpoint{2.339182in}{0.243518in}}%
\pgfpathlineto{\pgfqpoint{2.343440in}{0.243518in}}%
\pgfpathlineto{\pgfqpoint{2.343440in}{0.239260in}}%
\pgfpathmoveto{\pgfqpoint{2.334924in}{0.243518in}}%
\pgfpathlineto{\pgfqpoint{2.334924in}{0.243518in}}%
\pgfpathlineto{\pgfqpoint{2.334924in}{0.247775in}}%
\pgfpathlineto{\pgfqpoint{2.339182in}{0.247775in}}%
\pgfpathlineto{\pgfqpoint{2.339182in}{0.243518in}}%
\pgfpathmoveto{\pgfqpoint{2.334924in}{0.247775in}}%
\pgfpathlineto{\pgfqpoint{2.334924in}{0.247775in}}%
\pgfpathlineto{\pgfqpoint{2.334924in}{0.252033in}}%
\pgfpathlineto{\pgfqpoint{2.339182in}{0.252033in}}%
\pgfpathlineto{\pgfqpoint{2.339182in}{0.247775in}}%
\pgfpathmoveto{\pgfqpoint{2.339182in}{0.243518in}}%
\pgfpathlineto{\pgfqpoint{2.339182in}{0.243518in}}%
\pgfpathlineto{\pgfqpoint{2.339182in}{0.247775in}}%
\pgfpathlineto{\pgfqpoint{2.343440in}{0.247775in}}%
\pgfpathlineto{\pgfqpoint{2.343440in}{0.243518in}}%
\pgfpathmoveto{\pgfqpoint{2.339182in}{0.247775in}}%
\pgfpathlineto{\pgfqpoint{2.339182in}{0.247775in}}%
\pgfpathlineto{\pgfqpoint{2.339182in}{0.252033in}}%
\pgfpathlineto{\pgfqpoint{2.343440in}{0.252033in}}%
\pgfpathlineto{\pgfqpoint{2.343440in}{0.247775in}}%
\pgfpathmoveto{\pgfqpoint{2.334924in}{0.252033in}}%
\pgfpathlineto{\pgfqpoint{2.334924in}{0.252033in}}%
\pgfpathlineto{\pgfqpoint{2.334924in}{0.256290in}}%
\pgfpathlineto{\pgfqpoint{2.339182in}{0.256290in}}%
\pgfpathlineto{\pgfqpoint{2.339182in}{0.252033in}}%
\pgfpathmoveto{\pgfqpoint{2.334924in}{0.256290in}}%
\pgfpathlineto{\pgfqpoint{2.334924in}{0.256290in}}%
\pgfpathlineto{\pgfqpoint{2.334924in}{0.260548in}}%
\pgfpathlineto{\pgfqpoint{2.339182in}{0.260548in}}%
\pgfpathlineto{\pgfqpoint{2.339182in}{0.256290in}}%
\pgfpathmoveto{\pgfqpoint{2.339182in}{0.252033in}}%
\pgfpathlineto{\pgfqpoint{2.339182in}{0.252033in}}%
\pgfpathlineto{\pgfqpoint{2.339182in}{0.256290in}}%
\pgfpathlineto{\pgfqpoint{2.343440in}{0.256290in}}%
\pgfpathlineto{\pgfqpoint{2.343440in}{0.252033in}}%
\pgfpathmoveto{\pgfqpoint{2.339182in}{0.256290in}}%
\pgfpathlineto{\pgfqpoint{2.339182in}{0.256290in}}%
\pgfpathlineto{\pgfqpoint{2.339182in}{0.260548in}}%
\pgfpathlineto{\pgfqpoint{2.343440in}{0.260548in}}%
\pgfpathlineto{\pgfqpoint{2.343440in}{0.256290in}}%
\pgfpathmoveto{\pgfqpoint{2.334924in}{0.260548in}}%
\pgfpathlineto{\pgfqpoint{2.334924in}{0.260548in}}%
\pgfpathlineto{\pgfqpoint{2.334924in}{0.264806in}}%
\pgfpathlineto{\pgfqpoint{2.339182in}{0.264806in}}%
\pgfpathlineto{\pgfqpoint{2.339182in}{0.260548in}}%
\pgfpathmoveto{\pgfqpoint{2.334924in}{0.264806in}}%
\pgfpathlineto{\pgfqpoint{2.334924in}{0.264806in}}%
\pgfpathlineto{\pgfqpoint{2.334924in}{0.269063in}}%
\pgfpathlineto{\pgfqpoint{2.339182in}{0.269063in}}%
\pgfpathlineto{\pgfqpoint{2.339182in}{0.264806in}}%
\pgfpathmoveto{\pgfqpoint{2.339182in}{0.260548in}}%
\pgfpathlineto{\pgfqpoint{2.339182in}{0.260548in}}%
\pgfpathlineto{\pgfqpoint{2.339182in}{0.264806in}}%
\pgfpathlineto{\pgfqpoint{2.343440in}{0.264806in}}%
\pgfpathlineto{\pgfqpoint{2.343440in}{0.260548in}}%
\pgfpathmoveto{\pgfqpoint{2.339182in}{0.264806in}}%
\pgfpathlineto{\pgfqpoint{2.339182in}{0.264806in}}%
\pgfpathlineto{\pgfqpoint{2.339182in}{0.269063in}}%
\pgfpathlineto{\pgfqpoint{2.343440in}{0.269063in}}%
\pgfpathlineto{\pgfqpoint{2.343440in}{0.264806in}}%
\pgfpathmoveto{\pgfqpoint{2.334924in}{0.269063in}}%
\pgfpathlineto{\pgfqpoint{2.334924in}{0.269063in}}%
\pgfpathlineto{\pgfqpoint{2.334924in}{0.273321in}}%
\pgfpathlineto{\pgfqpoint{2.339182in}{0.273321in}}%
\pgfpathlineto{\pgfqpoint{2.339182in}{0.269063in}}%
\pgfpathmoveto{\pgfqpoint{2.334924in}{0.273321in}}%
\pgfpathlineto{\pgfqpoint{2.334924in}{0.273321in}}%
\pgfpathlineto{\pgfqpoint{2.334924in}{0.277579in}}%
\pgfpathlineto{\pgfqpoint{2.339182in}{0.277579in}}%
\pgfpathlineto{\pgfqpoint{2.339182in}{0.273321in}}%
\pgfpathmoveto{\pgfqpoint{2.339182in}{0.269063in}}%
\pgfpathlineto{\pgfqpoint{2.339182in}{0.269063in}}%
\pgfpathlineto{\pgfqpoint{2.339182in}{0.273321in}}%
\pgfpathlineto{\pgfqpoint{2.343440in}{0.273321in}}%
\pgfpathlineto{\pgfqpoint{2.343440in}{0.269063in}}%
\pgfpathmoveto{\pgfqpoint{2.339182in}{0.273321in}}%
\pgfpathlineto{\pgfqpoint{2.339182in}{0.273321in}}%
\pgfpathlineto{\pgfqpoint{2.339182in}{0.277579in}}%
\pgfpathlineto{\pgfqpoint{2.343440in}{0.277579in}}%
\pgfpathlineto{\pgfqpoint{2.343440in}{0.273321in}}%
\pgfpathmoveto{\pgfqpoint{2.339182in}{0.277579in}}%
\pgfpathlineto{\pgfqpoint{2.339182in}{0.277579in}}%
\pgfpathlineto{\pgfqpoint{2.339182in}{0.281836in}}%
\pgfpathlineto{\pgfqpoint{2.343440in}{0.281836in}}%
\pgfpathlineto{\pgfqpoint{2.343440in}{0.277579in}}%
\pgfpathmoveto{\pgfqpoint{2.339182in}{0.281836in}}%
\pgfpathlineto{\pgfqpoint{2.339182in}{0.281836in}}%
\pgfpathlineto{\pgfqpoint{2.339182in}{0.286094in}}%
\pgfpathlineto{\pgfqpoint{2.343440in}{0.286094in}}%
\pgfpathlineto{\pgfqpoint{2.343440in}{0.281836in}}%
\pgfpathmoveto{\pgfqpoint{2.339182in}{0.286094in}}%
\pgfpathlineto{\pgfqpoint{2.339182in}{0.286094in}}%
\pgfpathlineto{\pgfqpoint{2.339182in}{0.290351in}}%
\pgfpathlineto{\pgfqpoint{2.343440in}{0.290351in}}%
\pgfpathlineto{\pgfqpoint{2.343440in}{0.286094in}}%
\pgfpathmoveto{\pgfqpoint{2.339182in}{0.290351in}}%
\pgfpathlineto{\pgfqpoint{2.339182in}{0.290351in}}%
\pgfpathlineto{\pgfqpoint{2.339182in}{0.294609in}}%
\pgfpathlineto{\pgfqpoint{2.343440in}{0.294609in}}%
\pgfpathlineto{\pgfqpoint{2.343440in}{0.290351in}}%
\pgfpathmoveto{\pgfqpoint{2.339182in}{0.294609in}}%
\pgfpathlineto{\pgfqpoint{2.339182in}{0.294609in}}%
\pgfpathlineto{\pgfqpoint{2.339182in}{0.298867in}}%
\pgfpathlineto{\pgfqpoint{2.343440in}{0.298867in}}%
\pgfpathlineto{\pgfqpoint{2.343440in}{0.294609in}}%
\pgfpathmoveto{\pgfqpoint{2.339182in}{0.298867in}}%
\pgfpathlineto{\pgfqpoint{2.339182in}{0.298867in}}%
\pgfpathlineto{\pgfqpoint{2.339182in}{0.303124in}}%
\pgfpathlineto{\pgfqpoint{2.343440in}{0.303124in}}%
\pgfpathlineto{\pgfqpoint{2.343440in}{0.298867in}}%
\pgfpathmoveto{\pgfqpoint{2.339182in}{0.303124in}}%
\pgfpathlineto{\pgfqpoint{2.339182in}{0.303124in}}%
\pgfpathlineto{\pgfqpoint{2.339182in}{0.307382in}}%
\pgfpathlineto{\pgfqpoint{2.343440in}{0.307382in}}%
\pgfpathlineto{\pgfqpoint{2.343440in}{0.303124in}}%
\pgfpathmoveto{\pgfqpoint{2.339182in}{0.307382in}}%
\pgfpathlineto{\pgfqpoint{2.339182in}{0.307382in}}%
\pgfpathlineto{\pgfqpoint{2.339182in}{0.311639in}}%
\pgfpathlineto{\pgfqpoint{2.343440in}{0.311639in}}%
\pgfpathlineto{\pgfqpoint{2.343440in}{0.307382in}}%
\pgfpathmoveto{\pgfqpoint{2.339182in}{0.311639in}}%
\pgfpathlineto{\pgfqpoint{2.339182in}{0.311639in}}%
\pgfpathlineto{\pgfqpoint{2.339182in}{0.315897in}}%
\pgfpathlineto{\pgfqpoint{2.343440in}{0.315897in}}%
\pgfpathlineto{\pgfqpoint{2.343440in}{0.311639in}}%
\pgfpathmoveto{\pgfqpoint{2.339182in}{0.315897in}}%
\pgfpathlineto{\pgfqpoint{2.339182in}{0.315897in}}%
\pgfpathlineto{\pgfqpoint{2.339182in}{0.320155in}}%
\pgfpathlineto{\pgfqpoint{2.343440in}{0.320155in}}%
\pgfpathlineto{\pgfqpoint{2.343440in}{0.315897in}}%
\pgfpathmoveto{\pgfqpoint{2.339182in}{0.320155in}}%
\pgfpathlineto{\pgfqpoint{2.339182in}{0.320155in}}%
\pgfpathlineto{\pgfqpoint{2.339182in}{0.324412in}}%
\pgfpathlineto{\pgfqpoint{2.343440in}{0.324412in}}%
\pgfpathlineto{\pgfqpoint{2.343440in}{0.320155in}}%
\pgfpathmoveto{\pgfqpoint{2.343440in}{0.277579in}}%
\pgfpathlineto{\pgfqpoint{2.343440in}{0.277579in}}%
\pgfpathlineto{\pgfqpoint{2.343440in}{0.281836in}}%
\pgfpathlineto{\pgfqpoint{2.347697in}{0.281836in}}%
\pgfpathlineto{\pgfqpoint{2.347697in}{0.277579in}}%
\pgfpathmoveto{\pgfqpoint{2.343440in}{0.281836in}}%
\pgfpathlineto{\pgfqpoint{2.343440in}{0.281836in}}%
\pgfpathlineto{\pgfqpoint{2.343440in}{0.286094in}}%
\pgfpathlineto{\pgfqpoint{2.347697in}{0.286094in}}%
\pgfpathlineto{\pgfqpoint{2.347697in}{0.281836in}}%
\pgfpathmoveto{\pgfqpoint{2.343440in}{0.286094in}}%
\pgfpathlineto{\pgfqpoint{2.343440in}{0.286094in}}%
\pgfpathlineto{\pgfqpoint{2.343440in}{0.290351in}}%
\pgfpathlineto{\pgfqpoint{2.347697in}{0.290351in}}%
\pgfpathlineto{\pgfqpoint{2.347697in}{0.286094in}}%
\pgfpathmoveto{\pgfqpoint{2.343440in}{0.290351in}}%
\pgfpathlineto{\pgfqpoint{2.343440in}{0.290351in}}%
\pgfpathlineto{\pgfqpoint{2.343440in}{0.294609in}}%
\pgfpathlineto{\pgfqpoint{2.347697in}{0.294609in}}%
\pgfpathlineto{\pgfqpoint{2.347697in}{0.290351in}}%
\pgfpathmoveto{\pgfqpoint{2.343440in}{0.294609in}}%
\pgfpathlineto{\pgfqpoint{2.343440in}{0.294609in}}%
\pgfpathlineto{\pgfqpoint{2.343440in}{0.298867in}}%
\pgfpathlineto{\pgfqpoint{2.347697in}{0.298867in}}%
\pgfpathlineto{\pgfqpoint{2.347697in}{0.294609in}}%
\pgfpathmoveto{\pgfqpoint{2.343440in}{0.298867in}}%
\pgfpathlineto{\pgfqpoint{2.343440in}{0.298867in}}%
\pgfpathlineto{\pgfqpoint{2.343440in}{0.303124in}}%
\pgfpathlineto{\pgfqpoint{2.347697in}{0.303124in}}%
\pgfpathlineto{\pgfqpoint{2.347697in}{0.298867in}}%
\pgfpathmoveto{\pgfqpoint{2.343440in}{0.303124in}}%
\pgfpathlineto{\pgfqpoint{2.343440in}{0.303124in}}%
\pgfpathlineto{\pgfqpoint{2.343440in}{0.307382in}}%
\pgfpathlineto{\pgfqpoint{2.347697in}{0.307382in}}%
\pgfpathlineto{\pgfqpoint{2.347697in}{0.303124in}}%
\pgfpathmoveto{\pgfqpoint{2.343440in}{0.307382in}}%
\pgfpathlineto{\pgfqpoint{2.343440in}{0.307382in}}%
\pgfpathlineto{\pgfqpoint{2.343440in}{0.311639in}}%
\pgfpathlineto{\pgfqpoint{2.347697in}{0.311639in}}%
\pgfpathlineto{\pgfqpoint{2.347697in}{0.307382in}}%
\pgfpathmoveto{\pgfqpoint{2.343440in}{0.311639in}}%
\pgfpathlineto{\pgfqpoint{2.343440in}{0.311639in}}%
\pgfpathlineto{\pgfqpoint{2.343440in}{0.315897in}}%
\pgfpathlineto{\pgfqpoint{2.347697in}{0.315897in}}%
\pgfpathlineto{\pgfqpoint{2.347697in}{0.311639in}}%
\pgfpathmoveto{\pgfqpoint{2.343440in}{0.315897in}}%
\pgfpathlineto{\pgfqpoint{2.343440in}{0.315897in}}%
\pgfpathlineto{\pgfqpoint{2.343440in}{0.320155in}}%
\pgfpathlineto{\pgfqpoint{2.347697in}{0.320155in}}%
\pgfpathlineto{\pgfqpoint{2.347697in}{0.315897in}}%
\pgfpathmoveto{\pgfqpoint{2.343440in}{0.320155in}}%
\pgfpathlineto{\pgfqpoint{2.343440in}{0.320155in}}%
\pgfpathlineto{\pgfqpoint{2.343440in}{0.324412in}}%
\pgfpathlineto{\pgfqpoint{2.347697in}{0.324412in}}%
\pgfpathlineto{\pgfqpoint{2.347697in}{0.320155in}}%
\pgfpathmoveto{\pgfqpoint{2.343440in}{0.324412in}}%
\pgfpathlineto{\pgfqpoint{2.343440in}{0.324412in}}%
\pgfpathlineto{\pgfqpoint{2.343440in}{0.328670in}}%
\pgfpathlineto{\pgfqpoint{2.347697in}{0.328670in}}%
\pgfpathlineto{\pgfqpoint{2.347697in}{0.324412in}}%
\pgfpathmoveto{\pgfqpoint{2.347697in}{0.324412in}}%
\pgfpathlineto{\pgfqpoint{2.347697in}{0.324412in}}%
\pgfpathlineto{\pgfqpoint{2.347697in}{0.328670in}}%
\pgfpathlineto{\pgfqpoint{2.351955in}{0.328670in}}%
\pgfpathlineto{\pgfqpoint{2.351955in}{0.324412in}}%
\pgfpathmoveto{\pgfqpoint{2.343440in}{0.328670in}}%
\pgfpathlineto{\pgfqpoint{2.343440in}{0.328670in}}%
\pgfpathlineto{\pgfqpoint{2.343440in}{0.332928in}}%
\pgfpathlineto{\pgfqpoint{2.347697in}{0.332928in}}%
\pgfpathlineto{\pgfqpoint{2.347697in}{0.328670in}}%
\pgfpathmoveto{\pgfqpoint{2.343440in}{0.332928in}}%
\pgfpathlineto{\pgfqpoint{2.343440in}{0.332928in}}%
\pgfpathlineto{\pgfqpoint{2.343440in}{0.337185in}}%
\pgfpathlineto{\pgfqpoint{2.347697in}{0.337185in}}%
\pgfpathlineto{\pgfqpoint{2.347697in}{0.332928in}}%
\pgfpathmoveto{\pgfqpoint{2.347697in}{0.328670in}}%
\pgfpathlineto{\pgfqpoint{2.347697in}{0.328670in}}%
\pgfpathlineto{\pgfqpoint{2.347697in}{0.332928in}}%
\pgfpathlineto{\pgfqpoint{2.351955in}{0.332928in}}%
\pgfpathlineto{\pgfqpoint{2.351955in}{0.328670in}}%
\pgfpathmoveto{\pgfqpoint{2.347697in}{0.332928in}}%
\pgfpathlineto{\pgfqpoint{2.347697in}{0.332928in}}%
\pgfpathlineto{\pgfqpoint{2.347697in}{0.337185in}}%
\pgfpathlineto{\pgfqpoint{2.351955in}{0.337185in}}%
\pgfpathlineto{\pgfqpoint{2.351955in}{0.332928in}}%
\pgfpathmoveto{\pgfqpoint{2.343440in}{0.337185in}}%
\pgfpathlineto{\pgfqpoint{2.343440in}{0.337185in}}%
\pgfpathlineto{\pgfqpoint{2.343440in}{0.341443in}}%
\pgfpathlineto{\pgfqpoint{2.347697in}{0.341443in}}%
\pgfpathlineto{\pgfqpoint{2.347697in}{0.337185in}}%
\pgfpathmoveto{\pgfqpoint{2.343440in}{0.341443in}}%
\pgfpathlineto{\pgfqpoint{2.343440in}{0.341443in}}%
\pgfpathlineto{\pgfqpoint{2.343440in}{0.345700in}}%
\pgfpathlineto{\pgfqpoint{2.347697in}{0.345700in}}%
\pgfpathlineto{\pgfqpoint{2.347697in}{0.341443in}}%
\pgfpathmoveto{\pgfqpoint{2.347697in}{0.337185in}}%
\pgfpathlineto{\pgfqpoint{2.347697in}{0.337185in}}%
\pgfpathlineto{\pgfqpoint{2.347697in}{0.341443in}}%
\pgfpathlineto{\pgfqpoint{2.351955in}{0.341443in}}%
\pgfpathlineto{\pgfqpoint{2.351955in}{0.337185in}}%
\pgfpathmoveto{\pgfqpoint{2.347697in}{0.341443in}}%
\pgfpathlineto{\pgfqpoint{2.347697in}{0.341443in}}%
\pgfpathlineto{\pgfqpoint{2.347697in}{0.345700in}}%
\pgfpathlineto{\pgfqpoint{2.351955in}{0.345700in}}%
\pgfpathlineto{\pgfqpoint{2.351955in}{0.341443in}}%
\pgfpathmoveto{\pgfqpoint{2.343440in}{0.345700in}}%
\pgfpathlineto{\pgfqpoint{2.343440in}{0.345700in}}%
\pgfpathlineto{\pgfqpoint{2.343440in}{0.349958in}}%
\pgfpathlineto{\pgfqpoint{2.347697in}{0.349958in}}%
\pgfpathlineto{\pgfqpoint{2.347697in}{0.345700in}}%
\pgfpathmoveto{\pgfqpoint{2.343440in}{0.349958in}}%
\pgfpathlineto{\pgfqpoint{2.343440in}{0.349958in}}%
\pgfpathlineto{\pgfqpoint{2.343440in}{0.354216in}}%
\pgfpathlineto{\pgfqpoint{2.347697in}{0.354216in}}%
\pgfpathlineto{\pgfqpoint{2.347697in}{0.349958in}}%
\pgfpathmoveto{\pgfqpoint{2.347697in}{0.345700in}}%
\pgfpathlineto{\pgfqpoint{2.347697in}{0.345700in}}%
\pgfpathlineto{\pgfqpoint{2.347697in}{0.349958in}}%
\pgfpathlineto{\pgfqpoint{2.351955in}{0.349958in}}%
\pgfpathlineto{\pgfqpoint{2.351955in}{0.345700in}}%
\pgfpathmoveto{\pgfqpoint{2.347697in}{0.349958in}}%
\pgfpathlineto{\pgfqpoint{2.347697in}{0.349958in}}%
\pgfpathlineto{\pgfqpoint{2.347697in}{0.354216in}}%
\pgfpathlineto{\pgfqpoint{2.351955in}{0.354216in}}%
\pgfpathlineto{\pgfqpoint{2.351955in}{0.349958in}}%
\pgfpathmoveto{\pgfqpoint{2.343440in}{0.354216in}}%
\pgfpathlineto{\pgfqpoint{2.343440in}{0.354216in}}%
\pgfpathlineto{\pgfqpoint{2.343440in}{0.358473in}}%
\pgfpathlineto{\pgfqpoint{2.347697in}{0.358473in}}%
\pgfpathlineto{\pgfqpoint{2.347697in}{0.354216in}}%
\pgfpathmoveto{\pgfqpoint{2.343440in}{0.358473in}}%
\pgfpathlineto{\pgfqpoint{2.343440in}{0.358473in}}%
\pgfpathlineto{\pgfqpoint{2.343440in}{0.362731in}}%
\pgfpathlineto{\pgfqpoint{2.347697in}{0.362731in}}%
\pgfpathlineto{\pgfqpoint{2.347697in}{0.358473in}}%
\pgfpathmoveto{\pgfqpoint{2.347697in}{0.354216in}}%
\pgfpathlineto{\pgfqpoint{2.347697in}{0.354216in}}%
\pgfpathlineto{\pgfqpoint{2.347697in}{0.358473in}}%
\pgfpathlineto{\pgfqpoint{2.351955in}{0.358473in}}%
\pgfpathlineto{\pgfqpoint{2.351955in}{0.354216in}}%
\pgfpathmoveto{\pgfqpoint{2.347697in}{0.358473in}}%
\pgfpathlineto{\pgfqpoint{2.347697in}{0.358473in}}%
\pgfpathlineto{\pgfqpoint{2.347697in}{0.362731in}}%
\pgfpathlineto{\pgfqpoint{2.351955in}{0.362731in}}%
\pgfpathlineto{\pgfqpoint{2.351955in}{0.358473in}}%
\pgfpathmoveto{\pgfqpoint{2.343440in}{0.362731in}}%
\pgfpathlineto{\pgfqpoint{2.343440in}{0.362731in}}%
\pgfpathlineto{\pgfqpoint{2.343440in}{0.366988in}}%
\pgfpathlineto{\pgfqpoint{2.347697in}{0.366988in}}%
\pgfpathlineto{\pgfqpoint{2.347697in}{0.362731in}}%
\pgfpathmoveto{\pgfqpoint{2.343440in}{0.366988in}}%
\pgfpathlineto{\pgfqpoint{2.343440in}{0.366988in}}%
\pgfpathlineto{\pgfqpoint{2.343440in}{0.371246in}}%
\pgfpathlineto{\pgfqpoint{2.347697in}{0.371246in}}%
\pgfpathlineto{\pgfqpoint{2.347697in}{0.366988in}}%
\pgfpathmoveto{\pgfqpoint{2.347697in}{0.362731in}}%
\pgfpathlineto{\pgfqpoint{2.347697in}{0.362731in}}%
\pgfpathlineto{\pgfqpoint{2.347697in}{0.366988in}}%
\pgfpathlineto{\pgfqpoint{2.351955in}{0.366988in}}%
\pgfpathlineto{\pgfqpoint{2.351955in}{0.362731in}}%
\pgfpathmoveto{\pgfqpoint{2.347697in}{0.366988in}}%
\pgfpathlineto{\pgfqpoint{2.347697in}{0.366988in}}%
\pgfpathlineto{\pgfqpoint{2.347697in}{0.371246in}}%
\pgfpathlineto{\pgfqpoint{2.351955in}{0.371246in}}%
\pgfpathlineto{\pgfqpoint{2.351955in}{0.366988in}}%
\pgfpathmoveto{\pgfqpoint{2.347697in}{0.371246in}}%
\pgfpathlineto{\pgfqpoint{2.347697in}{0.371246in}}%
\pgfpathlineto{\pgfqpoint{2.347697in}{0.375504in}}%
\pgfpathlineto{\pgfqpoint{2.351955in}{0.375504in}}%
\pgfpathlineto{\pgfqpoint{2.351955in}{0.371246in}}%
\pgfpathmoveto{\pgfqpoint{2.347697in}{0.375504in}}%
\pgfpathlineto{\pgfqpoint{2.347697in}{0.375504in}}%
\pgfpathlineto{\pgfqpoint{2.347697in}{0.379762in}}%
\pgfpathlineto{\pgfqpoint{2.351955in}{0.379762in}}%
\pgfpathlineto{\pgfqpoint{2.351955in}{0.375504in}}%
\pgfpathmoveto{\pgfqpoint{2.347697in}{0.379762in}}%
\pgfpathlineto{\pgfqpoint{2.347697in}{0.379762in}}%
\pgfpathlineto{\pgfqpoint{2.347697in}{0.384020in}}%
\pgfpathlineto{\pgfqpoint{2.351955in}{0.384020in}}%
\pgfpathlineto{\pgfqpoint{2.351955in}{0.379762in}}%
\pgfpathmoveto{\pgfqpoint{2.347697in}{0.384020in}}%
\pgfpathlineto{\pgfqpoint{2.347697in}{0.384020in}}%
\pgfpathlineto{\pgfqpoint{2.347697in}{0.388278in}}%
\pgfpathlineto{\pgfqpoint{2.351955in}{0.388278in}}%
\pgfpathlineto{\pgfqpoint{2.351955in}{0.384020in}}%
\pgfpathmoveto{\pgfqpoint{2.351955in}{0.371246in}}%
\pgfpathlineto{\pgfqpoint{2.351955in}{0.371246in}}%
\pgfpathlineto{\pgfqpoint{2.351955in}{0.375504in}}%
\pgfpathlineto{\pgfqpoint{2.356213in}{0.375504in}}%
\pgfpathlineto{\pgfqpoint{2.356213in}{0.371246in}}%
\pgfpathmoveto{\pgfqpoint{2.351955in}{0.375504in}}%
\pgfpathlineto{\pgfqpoint{2.351955in}{0.375504in}}%
\pgfpathlineto{\pgfqpoint{2.351955in}{0.379762in}}%
\pgfpathlineto{\pgfqpoint{2.356213in}{0.379762in}}%
\pgfpathlineto{\pgfqpoint{2.356213in}{0.375504in}}%
\pgfpathmoveto{\pgfqpoint{2.351955in}{0.379762in}}%
\pgfpathlineto{\pgfqpoint{2.351955in}{0.379762in}}%
\pgfpathlineto{\pgfqpoint{2.351955in}{0.384020in}}%
\pgfpathlineto{\pgfqpoint{2.356213in}{0.384020in}}%
\pgfpathlineto{\pgfqpoint{2.356213in}{0.379762in}}%
\pgfpathmoveto{\pgfqpoint{2.351955in}{0.384020in}}%
\pgfpathlineto{\pgfqpoint{2.351955in}{0.384020in}}%
\pgfpathlineto{\pgfqpoint{2.351955in}{0.388278in}}%
\pgfpathlineto{\pgfqpoint{2.356213in}{0.388278in}}%
\pgfpathlineto{\pgfqpoint{2.356213in}{0.384020in}}%
\pgfpathmoveto{\pgfqpoint{2.347697in}{0.388278in}}%
\pgfpathlineto{\pgfqpoint{2.347697in}{0.388278in}}%
\pgfpathlineto{\pgfqpoint{2.347697in}{0.392536in}}%
\pgfpathlineto{\pgfqpoint{2.351955in}{0.392536in}}%
\pgfpathlineto{\pgfqpoint{2.351955in}{0.388278in}}%
\pgfpathmoveto{\pgfqpoint{2.347697in}{0.392536in}}%
\pgfpathlineto{\pgfqpoint{2.347697in}{0.392536in}}%
\pgfpathlineto{\pgfqpoint{2.347697in}{0.396794in}}%
\pgfpathlineto{\pgfqpoint{2.351955in}{0.396794in}}%
\pgfpathlineto{\pgfqpoint{2.351955in}{0.392536in}}%
\pgfpathmoveto{\pgfqpoint{2.347697in}{0.396794in}}%
\pgfpathlineto{\pgfqpoint{2.347697in}{0.396794in}}%
\pgfpathlineto{\pgfqpoint{2.347697in}{0.401052in}}%
\pgfpathlineto{\pgfqpoint{2.351955in}{0.401052in}}%
\pgfpathlineto{\pgfqpoint{2.351955in}{0.396794in}}%
\pgfpathmoveto{\pgfqpoint{2.347697in}{0.401052in}}%
\pgfpathlineto{\pgfqpoint{2.347697in}{0.401052in}}%
\pgfpathlineto{\pgfqpoint{2.347697in}{0.405310in}}%
\pgfpathlineto{\pgfqpoint{2.351955in}{0.405310in}}%
\pgfpathlineto{\pgfqpoint{2.351955in}{0.401052in}}%
\pgfpathmoveto{\pgfqpoint{2.351955in}{0.388278in}}%
\pgfpathlineto{\pgfqpoint{2.351955in}{0.388278in}}%
\pgfpathlineto{\pgfqpoint{2.351955in}{0.392536in}}%
\pgfpathlineto{\pgfqpoint{2.356213in}{0.392536in}}%
\pgfpathlineto{\pgfqpoint{2.356213in}{0.388278in}}%
\pgfpathmoveto{\pgfqpoint{2.351955in}{0.392536in}}%
\pgfpathlineto{\pgfqpoint{2.351955in}{0.392536in}}%
\pgfpathlineto{\pgfqpoint{2.351955in}{0.396794in}}%
\pgfpathlineto{\pgfqpoint{2.356213in}{0.396794in}}%
\pgfpathlineto{\pgfqpoint{2.356213in}{0.392536in}}%
\pgfpathmoveto{\pgfqpoint{2.351955in}{0.396794in}}%
\pgfpathlineto{\pgfqpoint{2.351955in}{0.396794in}}%
\pgfpathlineto{\pgfqpoint{2.351955in}{0.401052in}}%
\pgfpathlineto{\pgfqpoint{2.356213in}{0.401052in}}%
\pgfpathlineto{\pgfqpoint{2.356213in}{0.396794in}}%
\pgfpathmoveto{\pgfqpoint{2.351955in}{0.401052in}}%
\pgfpathlineto{\pgfqpoint{2.351955in}{0.401052in}}%
\pgfpathlineto{\pgfqpoint{2.351955in}{0.405310in}}%
\pgfpathlineto{\pgfqpoint{2.356213in}{0.405310in}}%
\pgfpathlineto{\pgfqpoint{2.356213in}{0.401052in}}%
\pgfpathmoveto{\pgfqpoint{2.347697in}{0.405310in}}%
\pgfpathlineto{\pgfqpoint{2.347697in}{0.405310in}}%
\pgfpathlineto{\pgfqpoint{2.347697in}{0.409568in}}%
\pgfpathlineto{\pgfqpoint{2.351955in}{0.409568in}}%
\pgfpathlineto{\pgfqpoint{2.351955in}{0.405310in}}%
\pgfpathmoveto{\pgfqpoint{2.347697in}{0.409568in}}%
\pgfpathlineto{\pgfqpoint{2.347697in}{0.409568in}}%
\pgfpathlineto{\pgfqpoint{2.347697in}{0.413826in}}%
\pgfpathlineto{\pgfqpoint{2.351955in}{0.413826in}}%
\pgfpathlineto{\pgfqpoint{2.351955in}{0.409568in}}%
\pgfpathmoveto{\pgfqpoint{2.347697in}{0.413826in}}%
\pgfpathlineto{\pgfqpoint{2.347697in}{0.413826in}}%
\pgfpathlineto{\pgfqpoint{2.347697in}{0.418084in}}%
\pgfpathlineto{\pgfqpoint{2.351955in}{0.418084in}}%
\pgfpathlineto{\pgfqpoint{2.351955in}{0.413826in}}%
\pgfpathmoveto{\pgfqpoint{2.351955in}{0.405310in}}%
\pgfpathlineto{\pgfqpoint{2.351955in}{0.405310in}}%
\pgfpathlineto{\pgfqpoint{2.351955in}{0.409568in}}%
\pgfpathlineto{\pgfqpoint{2.356213in}{0.409568in}}%
\pgfpathlineto{\pgfqpoint{2.356213in}{0.405310in}}%
\pgfpathmoveto{\pgfqpoint{2.351955in}{0.409568in}}%
\pgfpathlineto{\pgfqpoint{2.351955in}{0.409568in}}%
\pgfpathlineto{\pgfqpoint{2.351955in}{0.413826in}}%
\pgfpathlineto{\pgfqpoint{2.356213in}{0.413826in}}%
\pgfpathlineto{\pgfqpoint{2.356213in}{0.409568in}}%
\pgfpathmoveto{\pgfqpoint{2.351955in}{0.413826in}}%
\pgfpathlineto{\pgfqpoint{2.351955in}{0.413826in}}%
\pgfpathlineto{\pgfqpoint{2.351955in}{0.418084in}}%
\pgfpathlineto{\pgfqpoint{2.356213in}{0.418084in}}%
\pgfpathlineto{\pgfqpoint{2.356213in}{0.413826in}}%
\pgfpathmoveto{\pgfqpoint{2.351955in}{0.418084in}}%
\pgfpathlineto{\pgfqpoint{2.351955in}{0.418084in}}%
\pgfpathlineto{\pgfqpoint{2.351955in}{0.422342in}}%
\pgfpathlineto{\pgfqpoint{2.356213in}{0.422342in}}%
\pgfpathlineto{\pgfqpoint{2.356213in}{0.418084in}}%
\pgfpathmoveto{\pgfqpoint{2.351955in}{0.422342in}}%
\pgfpathlineto{\pgfqpoint{2.351955in}{0.422342in}}%
\pgfpathlineto{\pgfqpoint{2.351955in}{0.426600in}}%
\pgfpathlineto{\pgfqpoint{2.356213in}{0.426600in}}%
\pgfpathlineto{\pgfqpoint{2.356213in}{0.422342in}}%
\pgfpathmoveto{\pgfqpoint{2.351955in}{0.426600in}}%
\pgfpathlineto{\pgfqpoint{2.351955in}{0.426600in}}%
\pgfpathlineto{\pgfqpoint{2.351955in}{0.430857in}}%
\pgfpathlineto{\pgfqpoint{2.356213in}{0.430857in}}%
\pgfpathlineto{\pgfqpoint{2.356213in}{0.426600in}}%
\pgfpathmoveto{\pgfqpoint{2.356213in}{0.422342in}}%
\pgfpathlineto{\pgfqpoint{2.356213in}{0.422342in}}%
\pgfpathlineto{\pgfqpoint{2.356213in}{0.426600in}}%
\pgfpathlineto{\pgfqpoint{2.360471in}{0.426600in}}%
\pgfpathlineto{\pgfqpoint{2.360471in}{0.422342in}}%
\pgfpathmoveto{\pgfqpoint{2.356213in}{0.426600in}}%
\pgfpathlineto{\pgfqpoint{2.356213in}{0.426600in}}%
\pgfpathlineto{\pgfqpoint{2.356213in}{0.430857in}}%
\pgfpathlineto{\pgfqpoint{2.360471in}{0.430857in}}%
\pgfpathlineto{\pgfqpoint{2.360471in}{0.426600in}}%
\pgfpathmoveto{\pgfqpoint{2.351955in}{0.430857in}}%
\pgfpathlineto{\pgfqpoint{2.351955in}{0.430857in}}%
\pgfpathlineto{\pgfqpoint{2.351955in}{0.435115in}}%
\pgfpathlineto{\pgfqpoint{2.356213in}{0.435115in}}%
\pgfpathlineto{\pgfqpoint{2.356213in}{0.430857in}}%
\pgfpathmoveto{\pgfqpoint{2.351955in}{0.435115in}}%
\pgfpathlineto{\pgfqpoint{2.351955in}{0.435115in}}%
\pgfpathlineto{\pgfqpoint{2.351955in}{0.439373in}}%
\pgfpathlineto{\pgfqpoint{2.356213in}{0.439373in}}%
\pgfpathlineto{\pgfqpoint{2.356213in}{0.435115in}}%
\pgfpathmoveto{\pgfqpoint{2.356213in}{0.430857in}}%
\pgfpathlineto{\pgfqpoint{2.356213in}{0.430857in}}%
\pgfpathlineto{\pgfqpoint{2.356213in}{0.435115in}}%
\pgfpathlineto{\pgfqpoint{2.360471in}{0.435115in}}%
\pgfpathlineto{\pgfqpoint{2.360471in}{0.430857in}}%
\pgfpathmoveto{\pgfqpoint{2.356213in}{0.435115in}}%
\pgfpathlineto{\pgfqpoint{2.356213in}{0.435115in}}%
\pgfpathlineto{\pgfqpoint{2.356213in}{0.439373in}}%
\pgfpathlineto{\pgfqpoint{2.360471in}{0.439373in}}%
\pgfpathlineto{\pgfqpoint{2.360471in}{0.435115in}}%
\pgfpathmoveto{\pgfqpoint{2.351955in}{0.439373in}}%
\pgfpathlineto{\pgfqpoint{2.351955in}{0.439373in}}%
\pgfpathlineto{\pgfqpoint{2.351955in}{0.443631in}}%
\pgfpathlineto{\pgfqpoint{2.356213in}{0.443631in}}%
\pgfpathlineto{\pgfqpoint{2.356213in}{0.439373in}}%
\pgfpathmoveto{\pgfqpoint{2.351955in}{0.443631in}}%
\pgfpathlineto{\pgfqpoint{2.351955in}{0.443631in}}%
\pgfpathlineto{\pgfqpoint{2.351955in}{0.447889in}}%
\pgfpathlineto{\pgfqpoint{2.356213in}{0.447889in}}%
\pgfpathlineto{\pgfqpoint{2.356213in}{0.443631in}}%
\pgfpathmoveto{\pgfqpoint{2.356213in}{0.439373in}}%
\pgfpathlineto{\pgfqpoint{2.356213in}{0.439373in}}%
\pgfpathlineto{\pgfqpoint{2.356213in}{0.443631in}}%
\pgfpathlineto{\pgfqpoint{2.360471in}{0.443631in}}%
\pgfpathlineto{\pgfqpoint{2.360471in}{0.439373in}}%
\pgfpathmoveto{\pgfqpoint{2.356213in}{0.443631in}}%
\pgfpathlineto{\pgfqpoint{2.356213in}{0.443631in}}%
\pgfpathlineto{\pgfqpoint{2.356213in}{0.447889in}}%
\pgfpathlineto{\pgfqpoint{2.360471in}{0.447889in}}%
\pgfpathlineto{\pgfqpoint{2.360471in}{0.443631in}}%
\pgfpathmoveto{\pgfqpoint{2.351955in}{0.447889in}}%
\pgfpathlineto{\pgfqpoint{2.351955in}{0.447889in}}%
\pgfpathlineto{\pgfqpoint{2.351955in}{0.452147in}}%
\pgfpathlineto{\pgfqpoint{2.356213in}{0.452147in}}%
\pgfpathlineto{\pgfqpoint{2.356213in}{0.447889in}}%
\pgfpathmoveto{\pgfqpoint{2.351955in}{0.452147in}}%
\pgfpathlineto{\pgfqpoint{2.351955in}{0.452147in}}%
\pgfpathlineto{\pgfqpoint{2.351955in}{0.456405in}}%
\pgfpathlineto{\pgfqpoint{2.356213in}{0.456405in}}%
\pgfpathlineto{\pgfqpoint{2.356213in}{0.452147in}}%
\pgfpathmoveto{\pgfqpoint{2.356213in}{0.447889in}}%
\pgfpathlineto{\pgfqpoint{2.356213in}{0.447889in}}%
\pgfpathlineto{\pgfqpoint{2.356213in}{0.452147in}}%
\pgfpathlineto{\pgfqpoint{2.360471in}{0.452147in}}%
\pgfpathlineto{\pgfqpoint{2.360471in}{0.447889in}}%
\pgfpathmoveto{\pgfqpoint{2.356213in}{0.452147in}}%
\pgfpathlineto{\pgfqpoint{2.356213in}{0.452147in}}%
\pgfpathlineto{\pgfqpoint{2.356213in}{0.456405in}}%
\pgfpathlineto{\pgfqpoint{2.360471in}{0.456405in}}%
\pgfpathlineto{\pgfqpoint{2.360471in}{0.452147in}}%
\pgfpathmoveto{\pgfqpoint{2.351955in}{0.456405in}}%
\pgfpathlineto{\pgfqpoint{2.351955in}{0.456405in}}%
\pgfpathlineto{\pgfqpoint{2.351955in}{0.460663in}}%
\pgfpathlineto{\pgfqpoint{2.356213in}{0.460663in}}%
\pgfpathlineto{\pgfqpoint{2.356213in}{0.456405in}}%
\pgfpathmoveto{\pgfqpoint{2.351955in}{0.460663in}}%
\pgfpathlineto{\pgfqpoint{2.351955in}{0.460663in}}%
\pgfpathlineto{\pgfqpoint{2.351955in}{0.464921in}}%
\pgfpathlineto{\pgfqpoint{2.356213in}{0.464921in}}%
\pgfpathlineto{\pgfqpoint{2.356213in}{0.460663in}}%
\pgfpathmoveto{\pgfqpoint{2.356213in}{0.456405in}}%
\pgfpathlineto{\pgfqpoint{2.356213in}{0.456405in}}%
\pgfpathlineto{\pgfqpoint{2.356213in}{0.460663in}}%
\pgfpathlineto{\pgfqpoint{2.360471in}{0.460663in}}%
\pgfpathlineto{\pgfqpoint{2.360471in}{0.456405in}}%
\pgfpathmoveto{\pgfqpoint{2.356213in}{0.460663in}}%
\pgfpathlineto{\pgfqpoint{2.356213in}{0.460663in}}%
\pgfpathlineto{\pgfqpoint{2.356213in}{0.464921in}}%
\pgfpathlineto{\pgfqpoint{2.360471in}{0.464921in}}%
\pgfpathlineto{\pgfqpoint{2.360471in}{0.460663in}}%
\pgfpathmoveto{\pgfqpoint{2.351955in}{0.464921in}}%
\pgfpathlineto{\pgfqpoint{2.351955in}{0.464921in}}%
\pgfpathlineto{\pgfqpoint{2.351955in}{0.469179in}}%
\pgfpathlineto{\pgfqpoint{2.356213in}{0.469179in}}%
\pgfpathlineto{\pgfqpoint{2.356213in}{0.464921in}}%
\pgfpathmoveto{\pgfqpoint{2.356213in}{0.464921in}}%
\pgfpathlineto{\pgfqpoint{2.356213in}{0.464921in}}%
\pgfpathlineto{\pgfqpoint{2.356213in}{0.469179in}}%
\pgfpathlineto{\pgfqpoint{2.360471in}{0.469179in}}%
\pgfpathlineto{\pgfqpoint{2.360471in}{0.464921in}}%
\pgfpathmoveto{\pgfqpoint{2.356213in}{0.469179in}}%
\pgfpathlineto{\pgfqpoint{2.356213in}{0.469179in}}%
\pgfpathlineto{\pgfqpoint{2.356213in}{0.473437in}}%
\pgfpathlineto{\pgfqpoint{2.360471in}{0.473437in}}%
\pgfpathlineto{\pgfqpoint{2.360471in}{0.469179in}}%
\pgfpathmoveto{\pgfqpoint{2.360471in}{0.464921in}}%
\pgfpathlineto{\pgfqpoint{2.360471in}{0.464921in}}%
\pgfpathlineto{\pgfqpoint{2.360471in}{0.469179in}}%
\pgfpathlineto{\pgfqpoint{2.364728in}{0.469179in}}%
\pgfpathlineto{\pgfqpoint{2.364728in}{0.464921in}}%
\pgfpathmoveto{\pgfqpoint{2.360471in}{0.469179in}}%
\pgfpathlineto{\pgfqpoint{2.360471in}{0.469179in}}%
\pgfpathlineto{\pgfqpoint{2.360471in}{0.473437in}}%
\pgfpathlineto{\pgfqpoint{2.364728in}{0.473437in}}%
\pgfpathlineto{\pgfqpoint{2.364728in}{0.469179in}}%
\pgfpathmoveto{\pgfqpoint{2.356213in}{0.473437in}}%
\pgfpathlineto{\pgfqpoint{2.356213in}{0.473437in}}%
\pgfpathlineto{\pgfqpoint{2.356213in}{0.477695in}}%
\pgfpathlineto{\pgfqpoint{2.360471in}{0.477695in}}%
\pgfpathlineto{\pgfqpoint{2.360471in}{0.473437in}}%
\pgfpathmoveto{\pgfqpoint{2.356213in}{0.477695in}}%
\pgfpathlineto{\pgfqpoint{2.356213in}{0.477695in}}%
\pgfpathlineto{\pgfqpoint{2.356213in}{0.481953in}}%
\pgfpathlineto{\pgfqpoint{2.360471in}{0.481953in}}%
\pgfpathlineto{\pgfqpoint{2.360471in}{0.477695in}}%
\pgfpathmoveto{\pgfqpoint{2.356213in}{0.481953in}}%
\pgfpathlineto{\pgfqpoint{2.356213in}{0.481953in}}%
\pgfpathlineto{\pgfqpoint{2.356213in}{0.486211in}}%
\pgfpathlineto{\pgfqpoint{2.360471in}{0.486211in}}%
\pgfpathlineto{\pgfqpoint{2.360471in}{0.481953in}}%
\pgfpathmoveto{\pgfqpoint{2.356213in}{0.486211in}}%
\pgfpathlineto{\pgfqpoint{2.356213in}{0.486211in}}%
\pgfpathlineto{\pgfqpoint{2.356213in}{0.490469in}}%
\pgfpathlineto{\pgfqpoint{2.360471in}{0.490469in}}%
\pgfpathlineto{\pgfqpoint{2.360471in}{0.486211in}}%
\pgfpathmoveto{\pgfqpoint{2.356213in}{0.490469in}}%
\pgfpathlineto{\pgfqpoint{2.356213in}{0.490469in}}%
\pgfpathlineto{\pgfqpoint{2.356213in}{0.494727in}}%
\pgfpathlineto{\pgfqpoint{2.360471in}{0.494727in}}%
\pgfpathlineto{\pgfqpoint{2.360471in}{0.490469in}}%
\pgfpathmoveto{\pgfqpoint{2.356213in}{0.494727in}}%
\pgfpathlineto{\pgfqpoint{2.356213in}{0.494727in}}%
\pgfpathlineto{\pgfqpoint{2.356213in}{0.498985in}}%
\pgfpathlineto{\pgfqpoint{2.360471in}{0.498985in}}%
\pgfpathlineto{\pgfqpoint{2.360471in}{0.494727in}}%
\pgfpathmoveto{\pgfqpoint{2.356213in}{0.498985in}}%
\pgfpathlineto{\pgfqpoint{2.356213in}{0.498985in}}%
\pgfpathlineto{\pgfqpoint{2.356213in}{0.503243in}}%
\pgfpathlineto{\pgfqpoint{2.360471in}{0.503243in}}%
\pgfpathlineto{\pgfqpoint{2.360471in}{0.498985in}}%
\pgfpathmoveto{\pgfqpoint{2.356213in}{0.503243in}}%
\pgfpathlineto{\pgfqpoint{2.356213in}{0.503243in}}%
\pgfpathlineto{\pgfqpoint{2.356213in}{0.507501in}}%
\pgfpathlineto{\pgfqpoint{2.360471in}{0.507501in}}%
\pgfpathlineto{\pgfqpoint{2.360471in}{0.503243in}}%
\pgfpathmoveto{\pgfqpoint{2.360471in}{0.473437in}}%
\pgfpathlineto{\pgfqpoint{2.360471in}{0.473437in}}%
\pgfpathlineto{\pgfqpoint{2.360471in}{0.477695in}}%
\pgfpathlineto{\pgfqpoint{2.364728in}{0.477695in}}%
\pgfpathlineto{\pgfqpoint{2.364728in}{0.473437in}}%
\pgfpathmoveto{\pgfqpoint{2.360471in}{0.477695in}}%
\pgfpathlineto{\pgfqpoint{2.360471in}{0.477695in}}%
\pgfpathlineto{\pgfqpoint{2.360471in}{0.481953in}}%
\pgfpathlineto{\pgfqpoint{2.364728in}{0.481953in}}%
\pgfpathlineto{\pgfqpoint{2.364728in}{0.477695in}}%
\pgfpathmoveto{\pgfqpoint{2.360471in}{0.481953in}}%
\pgfpathlineto{\pgfqpoint{2.360471in}{0.481953in}}%
\pgfpathlineto{\pgfqpoint{2.360471in}{0.486211in}}%
\pgfpathlineto{\pgfqpoint{2.364728in}{0.486211in}}%
\pgfpathlineto{\pgfqpoint{2.364728in}{0.481953in}}%
\pgfpathmoveto{\pgfqpoint{2.360471in}{0.486211in}}%
\pgfpathlineto{\pgfqpoint{2.360471in}{0.486211in}}%
\pgfpathlineto{\pgfqpoint{2.360471in}{0.490469in}}%
\pgfpathlineto{\pgfqpoint{2.364728in}{0.490469in}}%
\pgfpathlineto{\pgfqpoint{2.364728in}{0.486211in}}%
\pgfpathmoveto{\pgfqpoint{2.360471in}{0.490469in}}%
\pgfpathlineto{\pgfqpoint{2.360471in}{0.490469in}}%
\pgfpathlineto{\pgfqpoint{2.360471in}{0.494727in}}%
\pgfpathlineto{\pgfqpoint{2.364728in}{0.494727in}}%
\pgfpathlineto{\pgfqpoint{2.364728in}{0.490469in}}%
\pgfpathmoveto{\pgfqpoint{2.360471in}{0.494727in}}%
\pgfpathlineto{\pgfqpoint{2.360471in}{0.494727in}}%
\pgfpathlineto{\pgfqpoint{2.360471in}{0.498985in}}%
\pgfpathlineto{\pgfqpoint{2.364728in}{0.498985in}}%
\pgfpathlineto{\pgfqpoint{2.364728in}{0.494727in}}%
\pgfpathmoveto{\pgfqpoint{2.360471in}{0.498985in}}%
\pgfpathlineto{\pgfqpoint{2.360471in}{0.498985in}}%
\pgfpathlineto{\pgfqpoint{2.360471in}{0.503243in}}%
\pgfpathlineto{\pgfqpoint{2.364728in}{0.503243in}}%
\pgfpathlineto{\pgfqpoint{2.364728in}{0.498985in}}%
\pgfpathmoveto{\pgfqpoint{2.360471in}{0.503243in}}%
\pgfpathlineto{\pgfqpoint{2.360471in}{0.503243in}}%
\pgfpathlineto{\pgfqpoint{2.360471in}{0.507501in}}%
\pgfpathlineto{\pgfqpoint{2.364728in}{0.507501in}}%
\pgfpathlineto{\pgfqpoint{2.364728in}{0.503243in}}%
\pgfpathmoveto{\pgfqpoint{2.356213in}{0.507501in}}%
\pgfpathlineto{\pgfqpoint{2.356213in}{0.507501in}}%
\pgfpathlineto{\pgfqpoint{2.356213in}{0.511759in}}%
\pgfpathlineto{\pgfqpoint{2.360471in}{0.511759in}}%
\pgfpathlineto{\pgfqpoint{2.360471in}{0.507501in}}%
\pgfpathmoveto{\pgfqpoint{2.360471in}{0.507501in}}%
\pgfpathlineto{\pgfqpoint{2.360471in}{0.507501in}}%
\pgfpathlineto{\pgfqpoint{2.360471in}{0.511759in}}%
\pgfpathlineto{\pgfqpoint{2.364728in}{0.511759in}}%
\pgfpathlineto{\pgfqpoint{2.364728in}{0.507501in}}%
\pgfpathmoveto{\pgfqpoint{2.360471in}{0.511759in}}%
\pgfpathlineto{\pgfqpoint{2.360471in}{0.511759in}}%
\pgfpathlineto{\pgfqpoint{2.360471in}{0.516016in}}%
\pgfpathlineto{\pgfqpoint{2.364728in}{0.516016in}}%
\pgfpathlineto{\pgfqpoint{2.364728in}{0.511759in}}%
\pgfpathmoveto{\pgfqpoint{2.364728in}{0.511759in}}%
\pgfpathlineto{\pgfqpoint{2.364728in}{0.511759in}}%
\pgfpathlineto{\pgfqpoint{2.364728in}{0.516016in}}%
\pgfpathlineto{\pgfqpoint{2.368986in}{0.516016in}}%
\pgfpathlineto{\pgfqpoint{2.368986in}{0.511759in}}%
\pgfpathmoveto{\pgfqpoint{2.360471in}{0.516016in}}%
\pgfpathlineto{\pgfqpoint{2.360471in}{0.516016in}}%
\pgfpathlineto{\pgfqpoint{2.360471in}{0.520274in}}%
\pgfpathlineto{\pgfqpoint{2.364728in}{0.520274in}}%
\pgfpathlineto{\pgfqpoint{2.364728in}{0.516016in}}%
\pgfpathmoveto{\pgfqpoint{2.360471in}{0.520274in}}%
\pgfpathlineto{\pgfqpoint{2.360471in}{0.520274in}}%
\pgfpathlineto{\pgfqpoint{2.360471in}{0.524532in}}%
\pgfpathlineto{\pgfqpoint{2.364728in}{0.524532in}}%
\pgfpathlineto{\pgfqpoint{2.364728in}{0.520274in}}%
\pgfpathmoveto{\pgfqpoint{2.364728in}{0.516016in}}%
\pgfpathlineto{\pgfqpoint{2.364728in}{0.516016in}}%
\pgfpathlineto{\pgfqpoint{2.364728in}{0.520274in}}%
\pgfpathlineto{\pgfqpoint{2.368986in}{0.520274in}}%
\pgfpathlineto{\pgfqpoint{2.368986in}{0.516016in}}%
\pgfpathmoveto{\pgfqpoint{2.364728in}{0.520274in}}%
\pgfpathlineto{\pgfqpoint{2.364728in}{0.520274in}}%
\pgfpathlineto{\pgfqpoint{2.364728in}{0.524532in}}%
\pgfpathlineto{\pgfqpoint{2.368986in}{0.524532in}}%
\pgfpathlineto{\pgfqpoint{2.368986in}{0.520274in}}%
\pgfpathmoveto{\pgfqpoint{2.360471in}{0.524532in}}%
\pgfpathlineto{\pgfqpoint{2.360471in}{0.524532in}}%
\pgfpathlineto{\pgfqpoint{2.360471in}{0.528790in}}%
\pgfpathlineto{\pgfqpoint{2.364728in}{0.528790in}}%
\pgfpathlineto{\pgfqpoint{2.364728in}{0.524532in}}%
\pgfpathmoveto{\pgfqpoint{2.360471in}{0.528790in}}%
\pgfpathlineto{\pgfqpoint{2.360471in}{0.528790in}}%
\pgfpathlineto{\pgfqpoint{2.360471in}{0.533048in}}%
\pgfpathlineto{\pgfqpoint{2.364728in}{0.533048in}}%
\pgfpathlineto{\pgfqpoint{2.364728in}{0.528790in}}%
\pgfpathmoveto{\pgfqpoint{2.364728in}{0.524532in}}%
\pgfpathlineto{\pgfqpoint{2.364728in}{0.524532in}}%
\pgfpathlineto{\pgfqpoint{2.364728in}{0.528790in}}%
\pgfpathlineto{\pgfqpoint{2.368986in}{0.528790in}}%
\pgfpathlineto{\pgfqpoint{2.368986in}{0.524532in}}%
\pgfpathmoveto{\pgfqpoint{2.364728in}{0.528790in}}%
\pgfpathlineto{\pgfqpoint{2.364728in}{0.528790in}}%
\pgfpathlineto{\pgfqpoint{2.364728in}{0.533048in}}%
\pgfpathlineto{\pgfqpoint{2.368986in}{0.533048in}}%
\pgfpathlineto{\pgfqpoint{2.368986in}{0.528790in}}%
\pgfpathmoveto{\pgfqpoint{2.360471in}{0.533048in}}%
\pgfpathlineto{\pgfqpoint{2.360471in}{0.533048in}}%
\pgfpathlineto{\pgfqpoint{2.360471in}{0.537306in}}%
\pgfpathlineto{\pgfqpoint{2.364728in}{0.537306in}}%
\pgfpathlineto{\pgfqpoint{2.364728in}{0.533048in}}%
\pgfpathmoveto{\pgfqpoint{2.360471in}{0.537306in}}%
\pgfpathlineto{\pgfqpoint{2.360471in}{0.537306in}}%
\pgfpathlineto{\pgfqpoint{2.360471in}{0.541564in}}%
\pgfpathlineto{\pgfqpoint{2.364728in}{0.541564in}}%
\pgfpathlineto{\pgfqpoint{2.364728in}{0.537306in}}%
\pgfpathmoveto{\pgfqpoint{2.364728in}{0.533048in}}%
\pgfpathlineto{\pgfqpoint{2.364728in}{0.533048in}}%
\pgfpathlineto{\pgfqpoint{2.364728in}{0.537306in}}%
\pgfpathlineto{\pgfqpoint{2.368986in}{0.537306in}}%
\pgfpathlineto{\pgfqpoint{2.368986in}{0.533048in}}%
\pgfpathmoveto{\pgfqpoint{2.364728in}{0.537306in}}%
\pgfpathlineto{\pgfqpoint{2.364728in}{0.537306in}}%
\pgfpathlineto{\pgfqpoint{2.364728in}{0.541564in}}%
\pgfpathlineto{\pgfqpoint{2.368986in}{0.541564in}}%
\pgfpathlineto{\pgfqpoint{2.368986in}{0.537306in}}%
\pgfpathmoveto{\pgfqpoint{2.360471in}{0.541564in}}%
\pgfpathlineto{\pgfqpoint{2.360471in}{0.541564in}}%
\pgfpathlineto{\pgfqpoint{2.360471in}{0.545822in}}%
\pgfpathlineto{\pgfqpoint{2.364728in}{0.545822in}}%
\pgfpathlineto{\pgfqpoint{2.364728in}{0.541564in}}%
\pgfpathmoveto{\pgfqpoint{2.360471in}{0.545822in}}%
\pgfpathlineto{\pgfqpoint{2.360471in}{0.545822in}}%
\pgfpathlineto{\pgfqpoint{2.360471in}{0.550080in}}%
\pgfpathlineto{\pgfqpoint{2.364728in}{0.550080in}}%
\pgfpathlineto{\pgfqpoint{2.364728in}{0.545822in}}%
\pgfpathmoveto{\pgfqpoint{2.364728in}{0.541564in}}%
\pgfpathlineto{\pgfqpoint{2.364728in}{0.541564in}}%
\pgfpathlineto{\pgfqpoint{2.364728in}{0.545822in}}%
\pgfpathlineto{\pgfqpoint{2.368986in}{0.545822in}}%
\pgfpathlineto{\pgfqpoint{2.368986in}{0.541564in}}%
\pgfpathmoveto{\pgfqpoint{2.364728in}{0.545822in}}%
\pgfpathlineto{\pgfqpoint{2.364728in}{0.545822in}}%
\pgfpathlineto{\pgfqpoint{2.364728in}{0.550080in}}%
\pgfpathlineto{\pgfqpoint{2.368986in}{0.550080in}}%
\pgfpathlineto{\pgfqpoint{2.368986in}{0.545822in}}%
\pgfpathmoveto{\pgfqpoint{2.360471in}{0.550080in}}%
\pgfpathlineto{\pgfqpoint{2.360471in}{0.550080in}}%
\pgfpathlineto{\pgfqpoint{2.360471in}{0.554337in}}%
\pgfpathlineto{\pgfqpoint{2.364728in}{0.554337in}}%
\pgfpathlineto{\pgfqpoint{2.364728in}{0.550080in}}%
\pgfpathmoveto{\pgfqpoint{2.360471in}{0.554337in}}%
\pgfpathlineto{\pgfqpoint{2.360471in}{0.554337in}}%
\pgfpathlineto{\pgfqpoint{2.360471in}{0.558595in}}%
\pgfpathlineto{\pgfqpoint{2.364728in}{0.558595in}}%
\pgfpathlineto{\pgfqpoint{2.364728in}{0.554337in}}%
\pgfpathmoveto{\pgfqpoint{2.364728in}{0.550080in}}%
\pgfpathlineto{\pgfqpoint{2.364728in}{0.550080in}}%
\pgfpathlineto{\pgfqpoint{2.364728in}{0.554337in}}%
\pgfpathlineto{\pgfqpoint{2.368986in}{0.554337in}}%
\pgfpathlineto{\pgfqpoint{2.368986in}{0.550080in}}%
\pgfpathmoveto{\pgfqpoint{2.364728in}{0.554337in}}%
\pgfpathlineto{\pgfqpoint{2.364728in}{0.554337in}}%
\pgfpathlineto{\pgfqpoint{2.364728in}{0.558595in}}%
\pgfpathlineto{\pgfqpoint{2.368986in}{0.558595in}}%
\pgfpathlineto{\pgfqpoint{2.368986in}{0.554337in}}%
\pgfpathmoveto{\pgfqpoint{2.364728in}{0.558595in}}%
\pgfpathlineto{\pgfqpoint{2.364728in}{0.558595in}}%
\pgfpathlineto{\pgfqpoint{2.364728in}{0.562853in}}%
\pgfpathlineto{\pgfqpoint{2.368986in}{0.562853in}}%
\pgfpathlineto{\pgfqpoint{2.368986in}{0.558595in}}%
\pgfpathmoveto{\pgfqpoint{2.364728in}{0.562853in}}%
\pgfpathlineto{\pgfqpoint{2.364728in}{0.562853in}}%
\pgfpathlineto{\pgfqpoint{2.364728in}{0.567111in}}%
\pgfpathlineto{\pgfqpoint{2.368986in}{0.567111in}}%
\pgfpathlineto{\pgfqpoint{2.368986in}{0.562853in}}%
\pgfpathmoveto{\pgfqpoint{2.364728in}{0.567111in}}%
\pgfpathlineto{\pgfqpoint{2.364728in}{0.567111in}}%
\pgfpathlineto{\pgfqpoint{2.364728in}{0.571369in}}%
\pgfpathlineto{\pgfqpoint{2.368986in}{0.571369in}}%
\pgfpathlineto{\pgfqpoint{2.368986in}{0.567111in}}%
\pgfpathmoveto{\pgfqpoint{2.364728in}{0.571369in}}%
\pgfpathlineto{\pgfqpoint{2.364728in}{0.571369in}}%
\pgfpathlineto{\pgfqpoint{2.364728in}{0.575627in}}%
\pgfpathlineto{\pgfqpoint{2.368986in}{0.575627in}}%
\pgfpathlineto{\pgfqpoint{2.368986in}{0.571369in}}%
\pgfpathmoveto{\pgfqpoint{2.368986in}{0.558595in}}%
\pgfpathlineto{\pgfqpoint{2.368986in}{0.558595in}}%
\pgfpathlineto{\pgfqpoint{2.368986in}{0.562853in}}%
\pgfpathlineto{\pgfqpoint{2.373244in}{0.562853in}}%
\pgfpathlineto{\pgfqpoint{2.373244in}{0.558595in}}%
\pgfpathmoveto{\pgfqpoint{2.368986in}{0.562853in}}%
\pgfpathlineto{\pgfqpoint{2.368986in}{0.562853in}}%
\pgfpathlineto{\pgfqpoint{2.368986in}{0.567111in}}%
\pgfpathlineto{\pgfqpoint{2.373244in}{0.567111in}}%
\pgfpathlineto{\pgfqpoint{2.373244in}{0.562853in}}%
\pgfpathmoveto{\pgfqpoint{2.368986in}{0.567111in}}%
\pgfpathlineto{\pgfqpoint{2.368986in}{0.567111in}}%
\pgfpathlineto{\pgfqpoint{2.368986in}{0.571369in}}%
\pgfpathlineto{\pgfqpoint{2.373244in}{0.571369in}}%
\pgfpathlineto{\pgfqpoint{2.373244in}{0.567111in}}%
\pgfpathmoveto{\pgfqpoint{2.368986in}{0.571369in}}%
\pgfpathlineto{\pgfqpoint{2.368986in}{0.571369in}}%
\pgfpathlineto{\pgfqpoint{2.368986in}{0.575627in}}%
\pgfpathlineto{\pgfqpoint{2.373244in}{0.575627in}}%
\pgfpathlineto{\pgfqpoint{2.373244in}{0.571369in}}%
\pgfpathmoveto{\pgfqpoint{2.364728in}{0.575627in}}%
\pgfpathlineto{\pgfqpoint{2.364728in}{0.575627in}}%
\pgfpathlineto{\pgfqpoint{2.364728in}{0.579885in}}%
\pgfpathlineto{\pgfqpoint{2.368986in}{0.579885in}}%
\pgfpathlineto{\pgfqpoint{2.368986in}{0.575627in}}%
\pgfpathmoveto{\pgfqpoint{2.364728in}{0.579885in}}%
\pgfpathlineto{\pgfqpoint{2.364728in}{0.579885in}}%
\pgfpathlineto{\pgfqpoint{2.364728in}{0.584143in}}%
\pgfpathlineto{\pgfqpoint{2.368986in}{0.584143in}}%
\pgfpathlineto{\pgfqpoint{2.368986in}{0.579885in}}%
\pgfpathmoveto{\pgfqpoint{2.364728in}{0.584143in}}%
\pgfpathlineto{\pgfqpoint{2.364728in}{0.584143in}}%
\pgfpathlineto{\pgfqpoint{2.364728in}{0.588401in}}%
\pgfpathlineto{\pgfqpoint{2.368986in}{0.588401in}}%
\pgfpathlineto{\pgfqpoint{2.368986in}{0.584143in}}%
\pgfpathmoveto{\pgfqpoint{2.364728in}{0.588401in}}%
\pgfpathlineto{\pgfqpoint{2.364728in}{0.588401in}}%
\pgfpathlineto{\pgfqpoint{2.364728in}{0.592658in}}%
\pgfpathlineto{\pgfqpoint{2.368986in}{0.592658in}}%
\pgfpathlineto{\pgfqpoint{2.368986in}{0.588401in}}%
\pgfpathmoveto{\pgfqpoint{2.368986in}{0.575627in}}%
\pgfpathlineto{\pgfqpoint{2.368986in}{0.575627in}}%
\pgfpathlineto{\pgfqpoint{2.368986in}{0.579885in}}%
\pgfpathlineto{\pgfqpoint{2.373244in}{0.579885in}}%
\pgfpathlineto{\pgfqpoint{2.373244in}{0.575627in}}%
\pgfpathmoveto{\pgfqpoint{2.368986in}{0.579885in}}%
\pgfpathlineto{\pgfqpoint{2.368986in}{0.579885in}}%
\pgfpathlineto{\pgfqpoint{2.368986in}{0.584143in}}%
\pgfpathlineto{\pgfqpoint{2.373244in}{0.584143in}}%
\pgfpathlineto{\pgfqpoint{2.373244in}{0.579885in}}%
\pgfpathmoveto{\pgfqpoint{2.368986in}{0.584143in}}%
\pgfpathlineto{\pgfqpoint{2.368986in}{0.584143in}}%
\pgfpathlineto{\pgfqpoint{2.368986in}{0.588401in}}%
\pgfpathlineto{\pgfqpoint{2.373244in}{0.588401in}}%
\pgfpathlineto{\pgfqpoint{2.373244in}{0.584143in}}%
\pgfpathmoveto{\pgfqpoint{2.368986in}{0.588401in}}%
\pgfpathlineto{\pgfqpoint{2.368986in}{0.588401in}}%
\pgfpathlineto{\pgfqpoint{2.368986in}{0.592658in}}%
\pgfpathlineto{\pgfqpoint{2.373244in}{0.592658in}}%
\pgfpathlineto{\pgfqpoint{2.373244in}{0.588401in}}%
\pgfpathmoveto{\pgfqpoint{2.364728in}{0.592658in}}%
\pgfpathlineto{\pgfqpoint{2.364728in}{0.592658in}}%
\pgfpathlineto{\pgfqpoint{2.364728in}{0.596916in}}%
\pgfpathlineto{\pgfqpoint{2.368986in}{0.596916in}}%
\pgfpathlineto{\pgfqpoint{2.368986in}{0.592658in}}%
\pgfpathmoveto{\pgfqpoint{2.364728in}{0.596916in}}%
\pgfpathlineto{\pgfqpoint{2.364728in}{0.596916in}}%
\pgfpathlineto{\pgfqpoint{2.364728in}{0.601174in}}%
\pgfpathlineto{\pgfqpoint{2.368986in}{0.601174in}}%
\pgfpathlineto{\pgfqpoint{2.368986in}{0.596916in}}%
\pgfpathmoveto{\pgfqpoint{2.364728in}{0.601174in}}%
\pgfpathlineto{\pgfqpoint{2.364728in}{0.601174in}}%
\pgfpathlineto{\pgfqpoint{2.364728in}{0.605432in}}%
\pgfpathlineto{\pgfqpoint{2.368986in}{0.605432in}}%
\pgfpathlineto{\pgfqpoint{2.368986in}{0.601174in}}%
\pgfpathmoveto{\pgfqpoint{2.368986in}{0.592658in}}%
\pgfpathlineto{\pgfqpoint{2.368986in}{0.592658in}}%
\pgfpathlineto{\pgfqpoint{2.368986in}{0.596916in}}%
\pgfpathlineto{\pgfqpoint{2.373244in}{0.596916in}}%
\pgfpathlineto{\pgfqpoint{2.373244in}{0.592658in}}%
\pgfpathmoveto{\pgfqpoint{2.368986in}{0.596916in}}%
\pgfpathlineto{\pgfqpoint{2.368986in}{0.596916in}}%
\pgfpathlineto{\pgfqpoint{2.368986in}{0.601174in}}%
\pgfpathlineto{\pgfqpoint{2.373244in}{0.601174in}}%
\pgfpathlineto{\pgfqpoint{2.373244in}{0.596916in}}%
\pgfpathmoveto{\pgfqpoint{2.368986in}{0.601174in}}%
\pgfpathlineto{\pgfqpoint{2.368986in}{0.601174in}}%
\pgfpathlineto{\pgfqpoint{2.368986in}{0.605432in}}%
\pgfpathlineto{\pgfqpoint{2.373244in}{0.605432in}}%
\pgfpathlineto{\pgfqpoint{2.373244in}{0.601174in}}%
\pgfpathmoveto{\pgfqpoint{2.368986in}{0.605432in}}%
\pgfpathlineto{\pgfqpoint{2.368986in}{0.605432in}}%
\pgfpathlineto{\pgfqpoint{2.368986in}{0.609690in}}%
\pgfpathlineto{\pgfqpoint{2.373244in}{0.609690in}}%
\pgfpathlineto{\pgfqpoint{2.373244in}{0.605432in}}%
\pgfpathmoveto{\pgfqpoint{2.373244in}{0.605432in}}%
\pgfpathlineto{\pgfqpoint{2.373244in}{0.605432in}}%
\pgfpathlineto{\pgfqpoint{2.373244in}{0.609690in}}%
\pgfpathlineto{\pgfqpoint{2.377501in}{0.609690in}}%
\pgfpathlineto{\pgfqpoint{2.377501in}{0.605432in}}%
\pgfpathmoveto{\pgfqpoint{2.368986in}{0.609690in}}%
\pgfpathlineto{\pgfqpoint{2.368986in}{0.609690in}}%
\pgfpathlineto{\pgfqpoint{2.368986in}{0.613948in}}%
\pgfpathlineto{\pgfqpoint{2.373244in}{0.613948in}}%
\pgfpathlineto{\pgfqpoint{2.373244in}{0.609690in}}%
\pgfpathmoveto{\pgfqpoint{2.368986in}{0.613948in}}%
\pgfpathlineto{\pgfqpoint{2.368986in}{0.613948in}}%
\pgfpathlineto{\pgfqpoint{2.368986in}{0.618206in}}%
\pgfpathlineto{\pgfqpoint{2.373244in}{0.618206in}}%
\pgfpathlineto{\pgfqpoint{2.373244in}{0.613948in}}%
\pgfpathmoveto{\pgfqpoint{2.373244in}{0.609690in}}%
\pgfpathlineto{\pgfqpoint{2.373244in}{0.609690in}}%
\pgfpathlineto{\pgfqpoint{2.373244in}{0.613948in}}%
\pgfpathlineto{\pgfqpoint{2.377501in}{0.613948in}}%
\pgfpathlineto{\pgfqpoint{2.377501in}{0.609690in}}%
\pgfpathmoveto{\pgfqpoint{2.373244in}{0.613948in}}%
\pgfpathlineto{\pgfqpoint{2.373244in}{0.613948in}}%
\pgfpathlineto{\pgfqpoint{2.373244in}{0.618206in}}%
\pgfpathlineto{\pgfqpoint{2.377501in}{0.618206in}}%
\pgfpathlineto{\pgfqpoint{2.377501in}{0.613948in}}%
\pgfpathmoveto{\pgfqpoint{2.368986in}{0.618206in}}%
\pgfpathlineto{\pgfqpoint{2.368986in}{0.618206in}}%
\pgfpathlineto{\pgfqpoint{2.368986in}{0.622464in}}%
\pgfpathlineto{\pgfqpoint{2.373244in}{0.622464in}}%
\pgfpathlineto{\pgfqpoint{2.373244in}{0.618206in}}%
\pgfpathmoveto{\pgfqpoint{2.368986in}{0.622464in}}%
\pgfpathlineto{\pgfqpoint{2.368986in}{0.622464in}}%
\pgfpathlineto{\pgfqpoint{2.368986in}{0.626721in}}%
\pgfpathlineto{\pgfqpoint{2.373244in}{0.626721in}}%
\pgfpathlineto{\pgfqpoint{2.373244in}{0.622464in}}%
\pgfpathmoveto{\pgfqpoint{2.373244in}{0.618206in}}%
\pgfpathlineto{\pgfqpoint{2.373244in}{0.618206in}}%
\pgfpathlineto{\pgfqpoint{2.373244in}{0.622464in}}%
\pgfpathlineto{\pgfqpoint{2.377501in}{0.622464in}}%
\pgfpathlineto{\pgfqpoint{2.377501in}{0.618206in}}%
\pgfpathmoveto{\pgfqpoint{2.373244in}{0.622464in}}%
\pgfpathlineto{\pgfqpoint{2.373244in}{0.622464in}}%
\pgfpathlineto{\pgfqpoint{2.373244in}{0.626721in}}%
\pgfpathlineto{\pgfqpoint{2.377501in}{0.626721in}}%
\pgfpathlineto{\pgfqpoint{2.377501in}{0.622464in}}%
\pgfpathmoveto{\pgfqpoint{2.368986in}{0.626721in}}%
\pgfpathlineto{\pgfqpoint{2.368986in}{0.626721in}}%
\pgfpathlineto{\pgfqpoint{2.368986in}{0.630979in}}%
\pgfpathlineto{\pgfqpoint{2.373244in}{0.630979in}}%
\pgfpathlineto{\pgfqpoint{2.373244in}{0.626721in}}%
\pgfpathmoveto{\pgfqpoint{2.368986in}{0.630979in}}%
\pgfpathlineto{\pgfqpoint{2.368986in}{0.630979in}}%
\pgfpathlineto{\pgfqpoint{2.368986in}{0.635237in}}%
\pgfpathlineto{\pgfqpoint{2.373244in}{0.635237in}}%
\pgfpathlineto{\pgfqpoint{2.373244in}{0.630979in}}%
\pgfpathmoveto{\pgfqpoint{2.373244in}{0.626721in}}%
\pgfpathlineto{\pgfqpoint{2.373244in}{0.626721in}}%
\pgfpathlineto{\pgfqpoint{2.373244in}{0.630979in}}%
\pgfpathlineto{\pgfqpoint{2.377501in}{0.630979in}}%
\pgfpathlineto{\pgfqpoint{2.377501in}{0.626721in}}%
\pgfpathmoveto{\pgfqpoint{2.373244in}{0.630979in}}%
\pgfpathlineto{\pgfqpoint{2.373244in}{0.630979in}}%
\pgfpathlineto{\pgfqpoint{2.373244in}{0.635237in}}%
\pgfpathlineto{\pgfqpoint{2.377501in}{0.635237in}}%
\pgfpathlineto{\pgfqpoint{2.377501in}{0.630979in}}%
\pgfpathmoveto{\pgfqpoint{2.368986in}{0.635237in}}%
\pgfpathlineto{\pgfqpoint{2.368986in}{0.635237in}}%
\pgfpathlineto{\pgfqpoint{2.368986in}{0.639495in}}%
\pgfpathlineto{\pgfqpoint{2.373244in}{0.639495in}}%
\pgfpathlineto{\pgfqpoint{2.373244in}{0.635237in}}%
\pgfpathmoveto{\pgfqpoint{2.368986in}{0.639495in}}%
\pgfpathlineto{\pgfqpoint{2.368986in}{0.639495in}}%
\pgfpathlineto{\pgfqpoint{2.368986in}{0.643753in}}%
\pgfpathlineto{\pgfqpoint{2.373244in}{0.643753in}}%
\pgfpathlineto{\pgfqpoint{2.373244in}{0.639495in}}%
\pgfpathmoveto{\pgfqpoint{2.373244in}{0.635237in}}%
\pgfpathlineto{\pgfqpoint{2.373244in}{0.635237in}}%
\pgfpathlineto{\pgfqpoint{2.373244in}{0.639495in}}%
\pgfpathlineto{\pgfqpoint{2.377501in}{0.639495in}}%
\pgfpathlineto{\pgfqpoint{2.377501in}{0.635237in}}%
\pgfpathmoveto{\pgfqpoint{2.373244in}{0.639495in}}%
\pgfpathlineto{\pgfqpoint{2.373244in}{0.639495in}}%
\pgfpathlineto{\pgfqpoint{2.373244in}{0.643753in}}%
\pgfpathlineto{\pgfqpoint{2.377501in}{0.643753in}}%
\pgfpathlineto{\pgfqpoint{2.377501in}{0.639495in}}%
\pgfpathmoveto{\pgfqpoint{2.368986in}{0.643753in}}%
\pgfpathlineto{\pgfqpoint{2.368986in}{0.643753in}}%
\pgfpathlineto{\pgfqpoint{2.368986in}{0.648011in}}%
\pgfpathlineto{\pgfqpoint{2.373244in}{0.648011in}}%
\pgfpathlineto{\pgfqpoint{2.373244in}{0.643753in}}%
\pgfpathmoveto{\pgfqpoint{2.368986in}{0.648011in}}%
\pgfpathlineto{\pgfqpoint{2.368986in}{0.648011in}}%
\pgfpathlineto{\pgfqpoint{2.368986in}{0.652269in}}%
\pgfpathlineto{\pgfqpoint{2.373244in}{0.652269in}}%
\pgfpathlineto{\pgfqpoint{2.373244in}{0.648011in}}%
\pgfpathmoveto{\pgfqpoint{2.373244in}{0.643753in}}%
\pgfpathlineto{\pgfqpoint{2.373244in}{0.643753in}}%
\pgfpathlineto{\pgfqpoint{2.373244in}{0.648011in}}%
\pgfpathlineto{\pgfqpoint{2.377501in}{0.648011in}}%
\pgfpathlineto{\pgfqpoint{2.377501in}{0.643753in}}%
\pgfpathmoveto{\pgfqpoint{2.373244in}{0.648011in}}%
\pgfpathlineto{\pgfqpoint{2.373244in}{0.648011in}}%
\pgfpathlineto{\pgfqpoint{2.373244in}{0.652269in}}%
\pgfpathlineto{\pgfqpoint{2.377501in}{0.652269in}}%
\pgfpathlineto{\pgfqpoint{2.377501in}{0.648011in}}%
\pgfpathmoveto{\pgfqpoint{2.373244in}{0.652269in}}%
\pgfpathlineto{\pgfqpoint{2.373244in}{0.652269in}}%
\pgfpathlineto{\pgfqpoint{2.373244in}{0.656526in}}%
\pgfpathlineto{\pgfqpoint{2.377501in}{0.656526in}}%
\pgfpathlineto{\pgfqpoint{2.377501in}{0.652269in}}%
\pgfpathmoveto{\pgfqpoint{2.373244in}{0.656526in}}%
\pgfpathlineto{\pgfqpoint{2.373244in}{0.656526in}}%
\pgfpathlineto{\pgfqpoint{2.373244in}{0.660784in}}%
\pgfpathlineto{\pgfqpoint{2.377501in}{0.660784in}}%
\pgfpathlineto{\pgfqpoint{2.377501in}{0.656526in}}%
\pgfpathmoveto{\pgfqpoint{2.373244in}{0.660784in}}%
\pgfpathlineto{\pgfqpoint{2.373244in}{0.660784in}}%
\pgfpathlineto{\pgfqpoint{2.373244in}{0.665042in}}%
\pgfpathlineto{\pgfqpoint{2.377501in}{0.665042in}}%
\pgfpathlineto{\pgfqpoint{2.377501in}{0.660784in}}%
\pgfpathmoveto{\pgfqpoint{2.373244in}{0.665042in}}%
\pgfpathlineto{\pgfqpoint{2.373244in}{0.665042in}}%
\pgfpathlineto{\pgfqpoint{2.373244in}{0.669300in}}%
\pgfpathlineto{\pgfqpoint{2.377501in}{0.669300in}}%
\pgfpathlineto{\pgfqpoint{2.377501in}{0.665042in}}%
\pgfpathmoveto{\pgfqpoint{2.373244in}{0.669300in}}%
\pgfpathlineto{\pgfqpoint{2.373244in}{0.669300in}}%
\pgfpathlineto{\pgfqpoint{2.373244in}{0.673558in}}%
\pgfpathlineto{\pgfqpoint{2.377501in}{0.673558in}}%
\pgfpathlineto{\pgfqpoint{2.377501in}{0.669300in}}%
\pgfpathmoveto{\pgfqpoint{2.373244in}{0.673558in}}%
\pgfpathlineto{\pgfqpoint{2.373244in}{0.673558in}}%
\pgfpathlineto{\pgfqpoint{2.373244in}{0.677815in}}%
\pgfpathlineto{\pgfqpoint{2.377501in}{0.677815in}}%
\pgfpathlineto{\pgfqpoint{2.377501in}{0.673558in}}%
\pgfpathmoveto{\pgfqpoint{2.373244in}{0.677815in}}%
\pgfpathlineto{\pgfqpoint{2.373244in}{0.677815in}}%
\pgfpathlineto{\pgfqpoint{2.373244in}{0.682073in}}%
\pgfpathlineto{\pgfqpoint{2.377501in}{0.682073in}}%
\pgfpathlineto{\pgfqpoint{2.377501in}{0.677815in}}%
\pgfpathmoveto{\pgfqpoint{2.373244in}{0.682073in}}%
\pgfpathlineto{\pgfqpoint{2.373244in}{0.682073in}}%
\pgfpathlineto{\pgfqpoint{2.373244in}{0.686331in}}%
\pgfpathlineto{\pgfqpoint{2.377501in}{0.686331in}}%
\pgfpathlineto{\pgfqpoint{2.377501in}{0.682073in}}%
\pgfpathmoveto{\pgfqpoint{2.373244in}{0.686331in}}%
\pgfpathlineto{\pgfqpoint{2.373244in}{0.686331in}}%
\pgfpathlineto{\pgfqpoint{2.373244in}{0.690589in}}%
\pgfpathlineto{\pgfqpoint{2.377501in}{0.690589in}}%
\pgfpathlineto{\pgfqpoint{2.377501in}{0.686331in}}%
\pgfpathmoveto{\pgfqpoint{2.373244in}{0.690589in}}%
\pgfpathlineto{\pgfqpoint{2.373244in}{0.690589in}}%
\pgfpathlineto{\pgfqpoint{2.373244in}{0.694847in}}%
\pgfpathlineto{\pgfqpoint{2.377501in}{0.694847in}}%
\pgfpathlineto{\pgfqpoint{2.377501in}{0.690589in}}%
\pgfpathmoveto{\pgfqpoint{2.373244in}{0.694847in}}%
\pgfpathlineto{\pgfqpoint{2.373244in}{0.694847in}}%
\pgfpathlineto{\pgfqpoint{2.373244in}{0.699104in}}%
\pgfpathlineto{\pgfqpoint{2.377501in}{0.699104in}}%
\pgfpathlineto{\pgfqpoint{2.377501in}{0.694847in}}%
\pgfpathmoveto{\pgfqpoint{2.377501in}{0.652269in}}%
\pgfpathlineto{\pgfqpoint{2.377501in}{0.652269in}}%
\pgfpathlineto{\pgfqpoint{2.377501in}{0.656526in}}%
\pgfpathlineto{\pgfqpoint{2.381759in}{0.656526in}}%
\pgfpathlineto{\pgfqpoint{2.381759in}{0.652269in}}%
\pgfpathmoveto{\pgfqpoint{2.377501in}{0.656526in}}%
\pgfpathlineto{\pgfqpoint{2.377501in}{0.656526in}}%
\pgfpathlineto{\pgfqpoint{2.377501in}{0.660784in}}%
\pgfpathlineto{\pgfqpoint{2.381759in}{0.660784in}}%
\pgfpathlineto{\pgfqpoint{2.381759in}{0.656526in}}%
\pgfpathmoveto{\pgfqpoint{2.377501in}{0.660784in}}%
\pgfpathlineto{\pgfqpoint{2.377501in}{0.660784in}}%
\pgfpathlineto{\pgfqpoint{2.377501in}{0.665042in}}%
\pgfpathlineto{\pgfqpoint{2.381759in}{0.665042in}}%
\pgfpathlineto{\pgfqpoint{2.381759in}{0.660784in}}%
\pgfpathmoveto{\pgfqpoint{2.377501in}{0.665042in}}%
\pgfpathlineto{\pgfqpoint{2.377501in}{0.665042in}}%
\pgfpathlineto{\pgfqpoint{2.377501in}{0.669300in}}%
\pgfpathlineto{\pgfqpoint{2.381759in}{0.669300in}}%
\pgfpathlineto{\pgfqpoint{2.381759in}{0.665042in}}%
\pgfpathmoveto{\pgfqpoint{2.377501in}{0.669300in}}%
\pgfpathlineto{\pgfqpoint{2.377501in}{0.669300in}}%
\pgfpathlineto{\pgfqpoint{2.377501in}{0.673558in}}%
\pgfpathlineto{\pgfqpoint{2.381759in}{0.673558in}}%
\pgfpathlineto{\pgfqpoint{2.381759in}{0.669300in}}%
\pgfpathmoveto{\pgfqpoint{2.377501in}{0.673558in}}%
\pgfpathlineto{\pgfqpoint{2.377501in}{0.673558in}}%
\pgfpathlineto{\pgfqpoint{2.377501in}{0.677815in}}%
\pgfpathlineto{\pgfqpoint{2.381759in}{0.677815in}}%
\pgfpathlineto{\pgfqpoint{2.381759in}{0.673558in}}%
\pgfpathmoveto{\pgfqpoint{2.377501in}{0.677815in}}%
\pgfpathlineto{\pgfqpoint{2.377501in}{0.677815in}}%
\pgfpathlineto{\pgfqpoint{2.377501in}{0.682073in}}%
\pgfpathlineto{\pgfqpoint{2.381759in}{0.682073in}}%
\pgfpathlineto{\pgfqpoint{2.381759in}{0.677815in}}%
\pgfpathmoveto{\pgfqpoint{2.377501in}{0.682073in}}%
\pgfpathlineto{\pgfqpoint{2.377501in}{0.682073in}}%
\pgfpathlineto{\pgfqpoint{2.377501in}{0.686331in}}%
\pgfpathlineto{\pgfqpoint{2.381759in}{0.686331in}}%
\pgfpathlineto{\pgfqpoint{2.381759in}{0.682073in}}%
\pgfpathmoveto{\pgfqpoint{2.377501in}{0.686331in}}%
\pgfpathlineto{\pgfqpoint{2.377501in}{0.686331in}}%
\pgfpathlineto{\pgfqpoint{2.377501in}{0.690589in}}%
\pgfpathlineto{\pgfqpoint{2.381759in}{0.690589in}}%
\pgfpathlineto{\pgfqpoint{2.381759in}{0.686331in}}%
\pgfpathmoveto{\pgfqpoint{2.377501in}{0.690589in}}%
\pgfpathlineto{\pgfqpoint{2.377501in}{0.690589in}}%
\pgfpathlineto{\pgfqpoint{2.377501in}{0.694847in}}%
\pgfpathlineto{\pgfqpoint{2.381759in}{0.694847in}}%
\pgfpathlineto{\pgfqpoint{2.381759in}{0.690589in}}%
\pgfpathmoveto{\pgfqpoint{2.377501in}{0.694847in}}%
\pgfpathlineto{\pgfqpoint{2.377501in}{0.694847in}}%
\pgfpathlineto{\pgfqpoint{2.377501in}{0.699104in}}%
\pgfpathlineto{\pgfqpoint{2.381759in}{0.699104in}}%
\pgfpathlineto{\pgfqpoint{2.381759in}{0.694847in}}%
\pgfpathmoveto{\pgfqpoint{2.377501in}{0.699104in}}%
\pgfpathlineto{\pgfqpoint{2.377501in}{0.699104in}}%
\pgfpathlineto{\pgfqpoint{2.377501in}{0.703362in}}%
\pgfpathlineto{\pgfqpoint{2.381759in}{0.703362in}}%
\pgfpathlineto{\pgfqpoint{2.381759in}{0.699104in}}%
\pgfpathmoveto{\pgfqpoint{2.381759in}{0.694847in}}%
\pgfpathlineto{\pgfqpoint{2.381759in}{0.694847in}}%
\pgfpathlineto{\pgfqpoint{2.381759in}{0.699104in}}%
\pgfpathlineto{\pgfqpoint{2.386017in}{0.699104in}}%
\pgfpathlineto{\pgfqpoint{2.386017in}{0.694847in}}%
\pgfpathmoveto{\pgfqpoint{2.381759in}{0.699104in}}%
\pgfpathlineto{\pgfqpoint{2.381759in}{0.699104in}}%
\pgfpathlineto{\pgfqpoint{2.381759in}{0.703362in}}%
\pgfpathlineto{\pgfqpoint{2.386017in}{0.703362in}}%
\pgfpathlineto{\pgfqpoint{2.386017in}{0.699104in}}%
\pgfpathmoveto{\pgfqpoint{2.377501in}{0.703362in}}%
\pgfpathlineto{\pgfqpoint{2.377501in}{0.703362in}}%
\pgfpathlineto{\pgfqpoint{2.377501in}{0.707620in}}%
\pgfpathlineto{\pgfqpoint{2.381759in}{0.707620in}}%
\pgfpathlineto{\pgfqpoint{2.381759in}{0.703362in}}%
\pgfpathmoveto{\pgfqpoint{2.377501in}{0.707620in}}%
\pgfpathlineto{\pgfqpoint{2.377501in}{0.707620in}}%
\pgfpathlineto{\pgfqpoint{2.377501in}{0.711878in}}%
\pgfpathlineto{\pgfqpoint{2.381759in}{0.711878in}}%
\pgfpathlineto{\pgfqpoint{2.381759in}{0.707620in}}%
\pgfpathmoveto{\pgfqpoint{2.381759in}{0.703362in}}%
\pgfpathlineto{\pgfqpoint{2.381759in}{0.703362in}}%
\pgfpathlineto{\pgfqpoint{2.381759in}{0.707620in}}%
\pgfpathlineto{\pgfqpoint{2.386017in}{0.707620in}}%
\pgfpathlineto{\pgfqpoint{2.386017in}{0.703362in}}%
\pgfpathmoveto{\pgfqpoint{2.381759in}{0.707620in}}%
\pgfpathlineto{\pgfqpoint{2.381759in}{0.707620in}}%
\pgfpathlineto{\pgfqpoint{2.381759in}{0.711878in}}%
\pgfpathlineto{\pgfqpoint{2.386017in}{0.711878in}}%
\pgfpathlineto{\pgfqpoint{2.386017in}{0.707620in}}%
\pgfpathmoveto{\pgfqpoint{2.377501in}{0.711878in}}%
\pgfpathlineto{\pgfqpoint{2.377501in}{0.711878in}}%
\pgfpathlineto{\pgfqpoint{2.377501in}{0.716136in}}%
\pgfpathlineto{\pgfqpoint{2.381759in}{0.716136in}}%
\pgfpathlineto{\pgfqpoint{2.381759in}{0.711878in}}%
\pgfpathmoveto{\pgfqpoint{2.377501in}{0.716136in}}%
\pgfpathlineto{\pgfqpoint{2.377501in}{0.716136in}}%
\pgfpathlineto{\pgfqpoint{2.377501in}{0.720393in}}%
\pgfpathlineto{\pgfqpoint{2.381759in}{0.720393in}}%
\pgfpathlineto{\pgfqpoint{2.381759in}{0.716136in}}%
\pgfpathmoveto{\pgfqpoint{2.381759in}{0.711878in}}%
\pgfpathlineto{\pgfqpoint{2.381759in}{0.711878in}}%
\pgfpathlineto{\pgfqpoint{2.381759in}{0.716136in}}%
\pgfpathlineto{\pgfqpoint{2.386017in}{0.716136in}}%
\pgfpathlineto{\pgfqpoint{2.386017in}{0.711878in}}%
\pgfpathmoveto{\pgfqpoint{2.381759in}{0.716136in}}%
\pgfpathlineto{\pgfqpoint{2.381759in}{0.716136in}}%
\pgfpathlineto{\pgfqpoint{2.381759in}{0.720393in}}%
\pgfpathlineto{\pgfqpoint{2.386017in}{0.720393in}}%
\pgfpathlineto{\pgfqpoint{2.386017in}{0.716136in}}%
\pgfpathmoveto{\pgfqpoint{2.377501in}{0.720393in}}%
\pgfpathlineto{\pgfqpoint{2.377501in}{0.720393in}}%
\pgfpathlineto{\pgfqpoint{2.377501in}{0.724651in}}%
\pgfpathlineto{\pgfqpoint{2.381759in}{0.724651in}}%
\pgfpathlineto{\pgfqpoint{2.381759in}{0.720393in}}%
\pgfpathmoveto{\pgfqpoint{2.377501in}{0.724651in}}%
\pgfpathlineto{\pgfqpoint{2.377501in}{0.724651in}}%
\pgfpathlineto{\pgfqpoint{2.377501in}{0.728909in}}%
\pgfpathlineto{\pgfqpoint{2.381759in}{0.728909in}}%
\pgfpathlineto{\pgfqpoint{2.381759in}{0.724651in}}%
\pgfpathmoveto{\pgfqpoint{2.381759in}{0.720393in}}%
\pgfpathlineto{\pgfqpoint{2.381759in}{0.720393in}}%
\pgfpathlineto{\pgfqpoint{2.381759in}{0.724651in}}%
\pgfpathlineto{\pgfqpoint{2.386017in}{0.724651in}}%
\pgfpathlineto{\pgfqpoint{2.386017in}{0.720393in}}%
\pgfpathmoveto{\pgfqpoint{2.381759in}{0.724651in}}%
\pgfpathlineto{\pgfqpoint{2.381759in}{0.724651in}}%
\pgfpathlineto{\pgfqpoint{2.381759in}{0.728909in}}%
\pgfpathlineto{\pgfqpoint{2.386017in}{0.728909in}}%
\pgfpathlineto{\pgfqpoint{2.386017in}{0.724651in}}%
\pgfpathmoveto{\pgfqpoint{2.377501in}{0.728909in}}%
\pgfpathlineto{\pgfqpoint{2.377501in}{0.728909in}}%
\pgfpathlineto{\pgfqpoint{2.377501in}{0.733167in}}%
\pgfpathlineto{\pgfqpoint{2.381759in}{0.733167in}}%
\pgfpathlineto{\pgfqpoint{2.381759in}{0.728909in}}%
\pgfpathmoveto{\pgfqpoint{2.377501in}{0.733167in}}%
\pgfpathlineto{\pgfqpoint{2.377501in}{0.733167in}}%
\pgfpathlineto{\pgfqpoint{2.377501in}{0.737425in}}%
\pgfpathlineto{\pgfqpoint{2.381759in}{0.737425in}}%
\pgfpathlineto{\pgfqpoint{2.381759in}{0.733167in}}%
\pgfpathmoveto{\pgfqpoint{2.381759in}{0.728909in}}%
\pgfpathlineto{\pgfqpoint{2.381759in}{0.728909in}}%
\pgfpathlineto{\pgfqpoint{2.381759in}{0.733167in}}%
\pgfpathlineto{\pgfqpoint{2.386017in}{0.733167in}}%
\pgfpathlineto{\pgfqpoint{2.386017in}{0.728909in}}%
\pgfpathmoveto{\pgfqpoint{2.381759in}{0.733167in}}%
\pgfpathlineto{\pgfqpoint{2.381759in}{0.733167in}}%
\pgfpathlineto{\pgfqpoint{2.381759in}{0.737425in}}%
\pgfpathlineto{\pgfqpoint{2.386017in}{0.737425in}}%
\pgfpathlineto{\pgfqpoint{2.386017in}{0.733167in}}%
\pgfpathmoveto{\pgfqpoint{2.377501in}{0.737425in}}%
\pgfpathlineto{\pgfqpoint{2.377501in}{0.737425in}}%
\pgfpathlineto{\pgfqpoint{2.377501in}{0.741682in}}%
\pgfpathlineto{\pgfqpoint{2.381759in}{0.741682in}}%
\pgfpathlineto{\pgfqpoint{2.381759in}{0.737425in}}%
\pgfpathmoveto{\pgfqpoint{2.381759in}{0.737425in}}%
\pgfpathlineto{\pgfqpoint{2.381759in}{0.737425in}}%
\pgfpathlineto{\pgfqpoint{2.381759in}{0.741682in}}%
\pgfpathlineto{\pgfqpoint{2.386017in}{0.741682in}}%
\pgfpathlineto{\pgfqpoint{2.386017in}{0.737425in}}%
\pgfpathmoveto{\pgfqpoint{2.381759in}{0.741682in}}%
\pgfpathlineto{\pgfqpoint{2.381759in}{0.741682in}}%
\pgfpathlineto{\pgfqpoint{2.381759in}{0.745940in}}%
\pgfpathlineto{\pgfqpoint{2.386017in}{0.745940in}}%
\pgfpathlineto{\pgfqpoint{2.386017in}{0.741682in}}%
\pgfpathmoveto{\pgfqpoint{2.386017in}{0.741682in}}%
\pgfpathlineto{\pgfqpoint{2.386017in}{0.741682in}}%
\pgfpathlineto{\pgfqpoint{2.386017in}{0.745940in}}%
\pgfpathlineto{\pgfqpoint{2.390275in}{0.745940in}}%
\pgfpathlineto{\pgfqpoint{2.390275in}{0.741682in}}%
\pgfpathmoveto{\pgfqpoint{2.381759in}{0.745940in}}%
\pgfpathlineto{\pgfqpoint{2.381759in}{0.745940in}}%
\pgfpathlineto{\pgfqpoint{2.381759in}{0.750198in}}%
\pgfpathlineto{\pgfqpoint{2.386017in}{0.750198in}}%
\pgfpathlineto{\pgfqpoint{2.386017in}{0.745940in}}%
\pgfpathmoveto{\pgfqpoint{2.381759in}{0.750198in}}%
\pgfpathlineto{\pgfqpoint{2.381759in}{0.750198in}}%
\pgfpathlineto{\pgfqpoint{2.381759in}{0.754456in}}%
\pgfpathlineto{\pgfqpoint{2.386017in}{0.754456in}}%
\pgfpathlineto{\pgfqpoint{2.386017in}{0.750198in}}%
\pgfpathmoveto{\pgfqpoint{2.381759in}{0.754456in}}%
\pgfpathlineto{\pgfqpoint{2.381759in}{0.754456in}}%
\pgfpathlineto{\pgfqpoint{2.381759in}{0.758714in}}%
\pgfpathlineto{\pgfqpoint{2.386017in}{0.758714in}}%
\pgfpathlineto{\pgfqpoint{2.386017in}{0.754456in}}%
\pgfpathmoveto{\pgfqpoint{2.381759in}{0.758714in}}%
\pgfpathlineto{\pgfqpoint{2.381759in}{0.758714in}}%
\pgfpathlineto{\pgfqpoint{2.381759in}{0.762971in}}%
\pgfpathlineto{\pgfqpoint{2.386017in}{0.762971in}}%
\pgfpathlineto{\pgfqpoint{2.386017in}{0.758714in}}%
\pgfpathmoveto{\pgfqpoint{2.386017in}{0.745940in}}%
\pgfpathlineto{\pgfqpoint{2.386017in}{0.745940in}}%
\pgfpathlineto{\pgfqpoint{2.386017in}{0.750198in}}%
\pgfpathlineto{\pgfqpoint{2.390275in}{0.750198in}}%
\pgfpathlineto{\pgfqpoint{2.390275in}{0.745940in}}%
\pgfpathmoveto{\pgfqpoint{2.386017in}{0.750198in}}%
\pgfpathlineto{\pgfqpoint{2.386017in}{0.750198in}}%
\pgfpathlineto{\pgfqpoint{2.386017in}{0.754456in}}%
\pgfpathlineto{\pgfqpoint{2.390275in}{0.754456in}}%
\pgfpathlineto{\pgfqpoint{2.390275in}{0.750198in}}%
\pgfpathmoveto{\pgfqpoint{2.386017in}{0.754456in}}%
\pgfpathlineto{\pgfqpoint{2.386017in}{0.754456in}}%
\pgfpathlineto{\pgfqpoint{2.386017in}{0.758714in}}%
\pgfpathlineto{\pgfqpoint{2.390275in}{0.758714in}}%
\pgfpathlineto{\pgfqpoint{2.390275in}{0.754456in}}%
\pgfpathmoveto{\pgfqpoint{2.386017in}{0.758714in}}%
\pgfpathlineto{\pgfqpoint{2.386017in}{0.758714in}}%
\pgfpathlineto{\pgfqpoint{2.386017in}{0.762971in}}%
\pgfpathlineto{\pgfqpoint{2.390275in}{0.762971in}}%
\pgfpathlineto{\pgfqpoint{2.390275in}{0.758714in}}%
\pgfpathmoveto{\pgfqpoint{2.381759in}{0.762971in}}%
\pgfpathlineto{\pgfqpoint{2.381759in}{0.762971in}}%
\pgfpathlineto{\pgfqpoint{2.381759in}{0.767229in}}%
\pgfpathlineto{\pgfqpoint{2.386017in}{0.767229in}}%
\pgfpathlineto{\pgfqpoint{2.386017in}{0.762971in}}%
\pgfpathmoveto{\pgfqpoint{2.381759in}{0.767229in}}%
\pgfpathlineto{\pgfqpoint{2.381759in}{0.767229in}}%
\pgfpathlineto{\pgfqpoint{2.381759in}{0.771487in}}%
\pgfpathlineto{\pgfqpoint{2.386017in}{0.771487in}}%
\pgfpathlineto{\pgfqpoint{2.386017in}{0.767229in}}%
\pgfpathmoveto{\pgfqpoint{2.381759in}{0.771487in}}%
\pgfpathlineto{\pgfqpoint{2.381759in}{0.771487in}}%
\pgfpathlineto{\pgfqpoint{2.381759in}{0.775745in}}%
\pgfpathlineto{\pgfqpoint{2.386017in}{0.775745in}}%
\pgfpathlineto{\pgfqpoint{2.386017in}{0.771487in}}%
\pgfpathmoveto{\pgfqpoint{2.381759in}{0.775745in}}%
\pgfpathlineto{\pgfqpoint{2.381759in}{0.775745in}}%
\pgfpathlineto{\pgfqpoint{2.381759in}{0.780002in}}%
\pgfpathlineto{\pgfqpoint{2.386017in}{0.780002in}}%
\pgfpathlineto{\pgfqpoint{2.386017in}{0.775745in}}%
\pgfpathmoveto{\pgfqpoint{2.386017in}{0.762971in}}%
\pgfpathlineto{\pgfqpoint{2.386017in}{0.762971in}}%
\pgfpathlineto{\pgfqpoint{2.386017in}{0.767229in}}%
\pgfpathlineto{\pgfqpoint{2.390275in}{0.767229in}}%
\pgfpathlineto{\pgfqpoint{2.390275in}{0.762971in}}%
\pgfpathmoveto{\pgfqpoint{2.386017in}{0.767229in}}%
\pgfpathlineto{\pgfqpoint{2.386017in}{0.767229in}}%
\pgfpathlineto{\pgfqpoint{2.386017in}{0.771487in}}%
\pgfpathlineto{\pgfqpoint{2.390275in}{0.771487in}}%
\pgfpathlineto{\pgfqpoint{2.390275in}{0.767229in}}%
\pgfpathmoveto{\pgfqpoint{2.386017in}{0.771487in}}%
\pgfpathlineto{\pgfqpoint{2.386017in}{0.771487in}}%
\pgfpathlineto{\pgfqpoint{2.386017in}{0.775745in}}%
\pgfpathlineto{\pgfqpoint{2.390275in}{0.775745in}}%
\pgfpathlineto{\pgfqpoint{2.390275in}{0.771487in}}%
\pgfpathmoveto{\pgfqpoint{2.386017in}{0.775745in}}%
\pgfpathlineto{\pgfqpoint{2.386017in}{0.775745in}}%
\pgfpathlineto{\pgfqpoint{2.386017in}{0.780002in}}%
\pgfpathlineto{\pgfqpoint{2.390275in}{0.780002in}}%
\pgfpathlineto{\pgfqpoint{2.390275in}{0.775745in}}%
\pgfpathmoveto{\pgfqpoint{2.381759in}{0.780002in}}%
\pgfpathlineto{\pgfqpoint{2.381759in}{0.780002in}}%
\pgfpathlineto{\pgfqpoint{2.381759in}{0.784260in}}%
\pgfpathlineto{\pgfqpoint{2.386017in}{0.784260in}}%
\pgfpathlineto{\pgfqpoint{2.386017in}{0.780002in}}%
\pgfpathmoveto{\pgfqpoint{2.381759in}{0.784260in}}%
\pgfpathlineto{\pgfqpoint{2.381759in}{0.784260in}}%
\pgfpathlineto{\pgfqpoint{2.381759in}{0.788518in}}%
\pgfpathlineto{\pgfqpoint{2.386017in}{0.788518in}}%
\pgfpathlineto{\pgfqpoint{2.386017in}{0.784260in}}%
\pgfpathmoveto{\pgfqpoint{2.386017in}{0.780002in}}%
\pgfpathlineto{\pgfqpoint{2.386017in}{0.780002in}}%
\pgfpathlineto{\pgfqpoint{2.386017in}{0.784260in}}%
\pgfpathlineto{\pgfqpoint{2.390275in}{0.784260in}}%
\pgfpathlineto{\pgfqpoint{2.390275in}{0.780002in}}%
\pgfpathmoveto{\pgfqpoint{2.386017in}{0.784260in}}%
\pgfpathlineto{\pgfqpoint{2.386017in}{0.784260in}}%
\pgfpathlineto{\pgfqpoint{2.386017in}{0.788518in}}%
\pgfpathlineto{\pgfqpoint{2.390275in}{0.788518in}}%
\pgfpathlineto{\pgfqpoint{2.390275in}{0.784260in}}%
\pgfpathmoveto{\pgfqpoint{2.390275in}{0.784260in}}%
\pgfpathlineto{\pgfqpoint{2.390275in}{0.784260in}}%
\pgfpathlineto{\pgfqpoint{2.390275in}{0.788518in}}%
\pgfpathlineto{\pgfqpoint{2.394532in}{0.788518in}}%
\pgfpathlineto{\pgfqpoint{2.394532in}{0.784260in}}%
\pgfpathmoveto{\pgfqpoint{2.386017in}{0.788518in}}%
\pgfpathlineto{\pgfqpoint{2.386017in}{0.788518in}}%
\pgfpathlineto{\pgfqpoint{2.386017in}{0.792775in}}%
\pgfpathlineto{\pgfqpoint{2.390275in}{0.792775in}}%
\pgfpathlineto{\pgfqpoint{2.390275in}{0.788518in}}%
\pgfpathmoveto{\pgfqpoint{2.386017in}{0.792775in}}%
\pgfpathlineto{\pgfqpoint{2.386017in}{0.792775in}}%
\pgfpathlineto{\pgfqpoint{2.386017in}{0.797033in}}%
\pgfpathlineto{\pgfqpoint{2.390275in}{0.797033in}}%
\pgfpathlineto{\pgfqpoint{2.390275in}{0.792775in}}%
\pgfpathmoveto{\pgfqpoint{2.390275in}{0.788518in}}%
\pgfpathlineto{\pgfqpoint{2.390275in}{0.788518in}}%
\pgfpathlineto{\pgfqpoint{2.390275in}{0.792775in}}%
\pgfpathlineto{\pgfqpoint{2.394532in}{0.792775in}}%
\pgfpathlineto{\pgfqpoint{2.394532in}{0.788518in}}%
\pgfpathmoveto{\pgfqpoint{2.390275in}{0.792775in}}%
\pgfpathlineto{\pgfqpoint{2.390275in}{0.792775in}}%
\pgfpathlineto{\pgfqpoint{2.390275in}{0.797033in}}%
\pgfpathlineto{\pgfqpoint{2.394532in}{0.797033in}}%
\pgfpathlineto{\pgfqpoint{2.394532in}{0.792775in}}%
\pgfpathmoveto{\pgfqpoint{2.386017in}{0.797033in}}%
\pgfpathlineto{\pgfqpoint{2.386017in}{0.797033in}}%
\pgfpathlineto{\pgfqpoint{2.386017in}{0.801291in}}%
\pgfpathlineto{\pgfqpoint{2.390275in}{0.801291in}}%
\pgfpathlineto{\pgfqpoint{2.390275in}{0.797033in}}%
\pgfpathmoveto{\pgfqpoint{2.386017in}{0.801291in}}%
\pgfpathlineto{\pgfqpoint{2.386017in}{0.801291in}}%
\pgfpathlineto{\pgfqpoint{2.386017in}{0.805548in}}%
\pgfpathlineto{\pgfqpoint{2.390275in}{0.805548in}}%
\pgfpathlineto{\pgfqpoint{2.390275in}{0.801291in}}%
\pgfpathmoveto{\pgfqpoint{2.390275in}{0.797033in}}%
\pgfpathlineto{\pgfqpoint{2.390275in}{0.797033in}}%
\pgfpathlineto{\pgfqpoint{2.390275in}{0.801291in}}%
\pgfpathlineto{\pgfqpoint{2.394532in}{0.801291in}}%
\pgfpathlineto{\pgfqpoint{2.394532in}{0.797033in}}%
\pgfpathmoveto{\pgfqpoint{2.390275in}{0.801291in}}%
\pgfpathlineto{\pgfqpoint{2.390275in}{0.801291in}}%
\pgfpathlineto{\pgfqpoint{2.390275in}{0.805548in}}%
\pgfpathlineto{\pgfqpoint{2.394532in}{0.805548in}}%
\pgfpathlineto{\pgfqpoint{2.394532in}{0.801291in}}%
\pgfpathmoveto{\pgfqpoint{2.386017in}{0.805548in}}%
\pgfpathlineto{\pgfqpoint{2.386017in}{0.805548in}}%
\pgfpathlineto{\pgfqpoint{2.386017in}{0.809806in}}%
\pgfpathlineto{\pgfqpoint{2.390275in}{0.809806in}}%
\pgfpathlineto{\pgfqpoint{2.390275in}{0.805548in}}%
\pgfpathmoveto{\pgfqpoint{2.386017in}{0.809806in}}%
\pgfpathlineto{\pgfqpoint{2.386017in}{0.809806in}}%
\pgfpathlineto{\pgfqpoint{2.386017in}{0.814063in}}%
\pgfpathlineto{\pgfqpoint{2.390275in}{0.814063in}}%
\pgfpathlineto{\pgfqpoint{2.390275in}{0.809806in}}%
\pgfpathmoveto{\pgfqpoint{2.390275in}{0.805548in}}%
\pgfpathlineto{\pgfqpoint{2.390275in}{0.805548in}}%
\pgfpathlineto{\pgfqpoint{2.390275in}{0.809806in}}%
\pgfpathlineto{\pgfqpoint{2.394532in}{0.809806in}}%
\pgfpathlineto{\pgfqpoint{2.394532in}{0.805548in}}%
\pgfpathmoveto{\pgfqpoint{2.390275in}{0.809806in}}%
\pgfpathlineto{\pgfqpoint{2.390275in}{0.809806in}}%
\pgfpathlineto{\pgfqpoint{2.390275in}{0.814063in}}%
\pgfpathlineto{\pgfqpoint{2.394532in}{0.814063in}}%
\pgfpathlineto{\pgfqpoint{2.394532in}{0.809806in}}%
\pgfpathmoveto{\pgfqpoint{2.386017in}{0.814063in}}%
\pgfpathlineto{\pgfqpoint{2.386017in}{0.814063in}}%
\pgfpathlineto{\pgfqpoint{2.386017in}{0.818321in}}%
\pgfpathlineto{\pgfqpoint{2.390275in}{0.818321in}}%
\pgfpathlineto{\pgfqpoint{2.390275in}{0.814063in}}%
\pgfpathmoveto{\pgfqpoint{2.386017in}{0.818321in}}%
\pgfpathlineto{\pgfqpoint{2.386017in}{0.818321in}}%
\pgfpathlineto{\pgfqpoint{2.386017in}{0.822579in}}%
\pgfpathlineto{\pgfqpoint{2.390275in}{0.822579in}}%
\pgfpathlineto{\pgfqpoint{2.390275in}{0.818321in}}%
\pgfpathmoveto{\pgfqpoint{2.390275in}{0.814063in}}%
\pgfpathlineto{\pgfqpoint{2.390275in}{0.814063in}}%
\pgfpathlineto{\pgfqpoint{2.390275in}{0.818321in}}%
\pgfpathlineto{\pgfqpoint{2.394532in}{0.818321in}}%
\pgfpathlineto{\pgfqpoint{2.394532in}{0.814063in}}%
\pgfpathmoveto{\pgfqpoint{2.390275in}{0.818321in}}%
\pgfpathlineto{\pgfqpoint{2.390275in}{0.818321in}}%
\pgfpathlineto{\pgfqpoint{2.390275in}{0.822579in}}%
\pgfpathlineto{\pgfqpoint{2.394532in}{0.822579in}}%
\pgfpathlineto{\pgfqpoint{2.394532in}{0.818321in}}%
\pgfpathmoveto{\pgfqpoint{2.386017in}{0.822579in}}%
\pgfpathlineto{\pgfqpoint{2.386017in}{0.822579in}}%
\pgfpathlineto{\pgfqpoint{2.386017in}{0.826836in}}%
\pgfpathlineto{\pgfqpoint{2.390275in}{0.826836in}}%
\pgfpathlineto{\pgfqpoint{2.390275in}{0.822579in}}%
\pgfpathmoveto{\pgfqpoint{2.386017in}{0.826836in}}%
\pgfpathlineto{\pgfqpoint{2.386017in}{0.826836in}}%
\pgfpathlineto{\pgfqpoint{2.386017in}{0.831094in}}%
\pgfpathlineto{\pgfqpoint{2.390275in}{0.831094in}}%
\pgfpathlineto{\pgfqpoint{2.390275in}{0.826836in}}%
\pgfpathmoveto{\pgfqpoint{2.390275in}{0.822579in}}%
\pgfpathlineto{\pgfqpoint{2.390275in}{0.822579in}}%
\pgfpathlineto{\pgfqpoint{2.390275in}{0.826836in}}%
\pgfpathlineto{\pgfqpoint{2.394532in}{0.826836in}}%
\pgfpathlineto{\pgfqpoint{2.394532in}{0.822579in}}%
\pgfpathmoveto{\pgfqpoint{2.390275in}{0.826836in}}%
\pgfpathlineto{\pgfqpoint{2.390275in}{0.826836in}}%
\pgfpathlineto{\pgfqpoint{2.390275in}{0.831094in}}%
\pgfpathlineto{\pgfqpoint{2.394532in}{0.831094in}}%
\pgfpathlineto{\pgfqpoint{2.394532in}{0.826836in}}%
\pgfpathmoveto{\pgfqpoint{2.390275in}{0.831094in}}%
\pgfpathlineto{\pgfqpoint{2.390275in}{0.831094in}}%
\pgfpathlineto{\pgfqpoint{2.390275in}{0.835351in}}%
\pgfpathlineto{\pgfqpoint{2.394532in}{0.835351in}}%
\pgfpathlineto{\pgfqpoint{2.394532in}{0.831094in}}%
\pgfpathmoveto{\pgfqpoint{2.390275in}{0.835351in}}%
\pgfpathlineto{\pgfqpoint{2.390275in}{0.835351in}}%
\pgfpathlineto{\pgfqpoint{2.390275in}{0.839609in}}%
\pgfpathlineto{\pgfqpoint{2.394532in}{0.839609in}}%
\pgfpathlineto{\pgfqpoint{2.394532in}{0.835351in}}%
\pgfpathmoveto{\pgfqpoint{2.390275in}{0.839609in}}%
\pgfpathlineto{\pgfqpoint{2.390275in}{0.839609in}}%
\pgfpathlineto{\pgfqpoint{2.390275in}{0.843867in}}%
\pgfpathlineto{\pgfqpoint{2.394532in}{0.843867in}}%
\pgfpathlineto{\pgfqpoint{2.394532in}{0.839609in}}%
\pgfpathmoveto{\pgfqpoint{2.390275in}{0.843867in}}%
\pgfpathlineto{\pgfqpoint{2.390275in}{0.843867in}}%
\pgfpathlineto{\pgfqpoint{2.390275in}{0.848124in}}%
\pgfpathlineto{\pgfqpoint{2.394532in}{0.848124in}}%
\pgfpathlineto{\pgfqpoint{2.394532in}{0.843867in}}%
\pgfpathmoveto{\pgfqpoint{2.394532in}{0.826836in}}%
\pgfpathlineto{\pgfqpoint{2.394532in}{0.826836in}}%
\pgfpathlineto{\pgfqpoint{2.394532in}{0.831094in}}%
\pgfpathlineto{\pgfqpoint{2.398790in}{0.831094in}}%
\pgfpathlineto{\pgfqpoint{2.398790in}{0.826836in}}%
\pgfpathmoveto{\pgfqpoint{2.394532in}{0.831094in}}%
\pgfpathlineto{\pgfqpoint{2.394532in}{0.831094in}}%
\pgfpathlineto{\pgfqpoint{2.394532in}{0.835351in}}%
\pgfpathlineto{\pgfqpoint{2.398790in}{0.835351in}}%
\pgfpathlineto{\pgfqpoint{2.398790in}{0.831094in}}%
\pgfpathmoveto{\pgfqpoint{2.394532in}{0.835351in}}%
\pgfpathlineto{\pgfqpoint{2.394532in}{0.835351in}}%
\pgfpathlineto{\pgfqpoint{2.394532in}{0.839609in}}%
\pgfpathlineto{\pgfqpoint{2.398790in}{0.839609in}}%
\pgfpathlineto{\pgfqpoint{2.398790in}{0.835351in}}%
\pgfpathmoveto{\pgfqpoint{2.394532in}{0.839609in}}%
\pgfpathlineto{\pgfqpoint{2.394532in}{0.839609in}}%
\pgfpathlineto{\pgfqpoint{2.394532in}{0.843867in}}%
\pgfpathlineto{\pgfqpoint{2.398790in}{0.843867in}}%
\pgfpathlineto{\pgfqpoint{2.398790in}{0.839609in}}%
\pgfpathmoveto{\pgfqpoint{2.394532in}{0.843867in}}%
\pgfpathlineto{\pgfqpoint{2.394532in}{0.843867in}}%
\pgfpathlineto{\pgfqpoint{2.394532in}{0.848124in}}%
\pgfpathlineto{\pgfqpoint{2.398790in}{0.848124in}}%
\pgfpathlineto{\pgfqpoint{2.398790in}{0.843867in}}%
\pgfpathmoveto{\pgfqpoint{2.390275in}{0.848124in}}%
\pgfpathlineto{\pgfqpoint{2.390275in}{0.848124in}}%
\pgfpathlineto{\pgfqpoint{2.390275in}{0.852382in}}%
\pgfpathlineto{\pgfqpoint{2.394532in}{0.852382in}}%
\pgfpathlineto{\pgfqpoint{2.394532in}{0.848124in}}%
\pgfpathmoveto{\pgfqpoint{2.390275in}{0.852382in}}%
\pgfpathlineto{\pgfqpoint{2.390275in}{0.852382in}}%
\pgfpathlineto{\pgfqpoint{2.390275in}{0.856640in}}%
\pgfpathlineto{\pgfqpoint{2.394532in}{0.856640in}}%
\pgfpathlineto{\pgfqpoint{2.394532in}{0.852382in}}%
\pgfpathmoveto{\pgfqpoint{2.390275in}{0.856640in}}%
\pgfpathlineto{\pgfqpoint{2.390275in}{0.856640in}}%
\pgfpathlineto{\pgfqpoint{2.390275in}{0.860897in}}%
\pgfpathlineto{\pgfqpoint{2.394532in}{0.860897in}}%
\pgfpathlineto{\pgfqpoint{2.394532in}{0.856640in}}%
\pgfpathmoveto{\pgfqpoint{2.390275in}{0.860897in}}%
\pgfpathlineto{\pgfqpoint{2.390275in}{0.860897in}}%
\pgfpathlineto{\pgfqpoint{2.390275in}{0.865155in}}%
\pgfpathlineto{\pgfqpoint{2.394532in}{0.865155in}}%
\pgfpathlineto{\pgfqpoint{2.394532in}{0.860897in}}%
\pgfpathmoveto{\pgfqpoint{2.390275in}{0.865155in}}%
\pgfpathlineto{\pgfqpoint{2.390275in}{0.865155in}}%
\pgfpathlineto{\pgfqpoint{2.390275in}{0.869412in}}%
\pgfpathlineto{\pgfqpoint{2.394532in}{0.869412in}}%
\pgfpathlineto{\pgfqpoint{2.394532in}{0.865155in}}%
\pgfpathmoveto{\pgfqpoint{2.390275in}{0.869412in}}%
\pgfpathlineto{\pgfqpoint{2.390275in}{0.869412in}}%
\pgfpathlineto{\pgfqpoint{2.390275in}{0.873670in}}%
\pgfpathlineto{\pgfqpoint{2.394532in}{0.873670in}}%
\pgfpathlineto{\pgfqpoint{2.394532in}{0.869412in}}%
\pgfpathmoveto{\pgfqpoint{2.394532in}{0.848124in}}%
\pgfpathlineto{\pgfqpoint{2.394532in}{0.848124in}}%
\pgfpathlineto{\pgfqpoint{2.394532in}{0.852382in}}%
\pgfpathlineto{\pgfqpoint{2.398790in}{0.852382in}}%
\pgfpathlineto{\pgfqpoint{2.398790in}{0.848124in}}%
\pgfpathmoveto{\pgfqpoint{2.394532in}{0.852382in}}%
\pgfpathlineto{\pgfqpoint{2.394532in}{0.852382in}}%
\pgfpathlineto{\pgfqpoint{2.394532in}{0.856640in}}%
\pgfpathlineto{\pgfqpoint{2.398790in}{0.856640in}}%
\pgfpathlineto{\pgfqpoint{2.398790in}{0.852382in}}%
\pgfpathmoveto{\pgfqpoint{2.394532in}{0.856640in}}%
\pgfpathlineto{\pgfqpoint{2.394532in}{0.856640in}}%
\pgfpathlineto{\pgfqpoint{2.394532in}{0.860897in}}%
\pgfpathlineto{\pgfqpoint{2.398790in}{0.860897in}}%
\pgfpathlineto{\pgfqpoint{2.398790in}{0.856640in}}%
\pgfpathmoveto{\pgfqpoint{2.394532in}{0.860897in}}%
\pgfpathlineto{\pgfqpoint{2.394532in}{0.860897in}}%
\pgfpathlineto{\pgfqpoint{2.394532in}{0.865155in}}%
\pgfpathlineto{\pgfqpoint{2.398790in}{0.865155in}}%
\pgfpathlineto{\pgfqpoint{2.398790in}{0.860897in}}%
\pgfpathmoveto{\pgfqpoint{2.394532in}{0.865155in}}%
\pgfpathlineto{\pgfqpoint{2.394532in}{0.865155in}}%
\pgfpathlineto{\pgfqpoint{2.394532in}{0.869412in}}%
\pgfpathlineto{\pgfqpoint{2.398790in}{0.869412in}}%
\pgfpathlineto{\pgfqpoint{2.398790in}{0.865155in}}%
\pgfpathmoveto{\pgfqpoint{2.394532in}{0.869412in}}%
\pgfpathlineto{\pgfqpoint{2.394532in}{0.869412in}}%
\pgfpathlineto{\pgfqpoint{2.394532in}{0.873670in}}%
\pgfpathlineto{\pgfqpoint{2.398790in}{0.873670in}}%
\pgfpathlineto{\pgfqpoint{2.398790in}{0.869412in}}%
\pgfpathmoveto{\pgfqpoint{2.394532in}{0.873670in}}%
\pgfpathlineto{\pgfqpoint{2.394532in}{0.873670in}}%
\pgfpathlineto{\pgfqpoint{2.394532in}{0.877928in}}%
\pgfpathlineto{\pgfqpoint{2.398790in}{0.877928in}}%
\pgfpathlineto{\pgfqpoint{2.398790in}{0.873670in}}%
\pgfpathmoveto{\pgfqpoint{2.394532in}{0.877928in}}%
\pgfpathlineto{\pgfqpoint{2.394532in}{0.877928in}}%
\pgfpathlineto{\pgfqpoint{2.394532in}{0.882185in}}%
\pgfpathlineto{\pgfqpoint{2.398790in}{0.882185in}}%
\pgfpathlineto{\pgfqpoint{2.398790in}{0.877928in}}%
\pgfpathmoveto{\pgfqpoint{2.398790in}{0.873670in}}%
\pgfpathlineto{\pgfqpoint{2.398790in}{0.873670in}}%
\pgfpathlineto{\pgfqpoint{2.398790in}{0.877928in}}%
\pgfpathlineto{\pgfqpoint{2.403048in}{0.877928in}}%
\pgfpathlineto{\pgfqpoint{2.403048in}{0.873670in}}%
\pgfpathmoveto{\pgfqpoint{2.398790in}{0.877928in}}%
\pgfpathlineto{\pgfqpoint{2.398790in}{0.877928in}}%
\pgfpathlineto{\pgfqpoint{2.398790in}{0.882185in}}%
\pgfpathlineto{\pgfqpoint{2.403048in}{0.882185in}}%
\pgfpathlineto{\pgfqpoint{2.403048in}{0.877928in}}%
\pgfpathmoveto{\pgfqpoint{2.394532in}{0.882185in}}%
\pgfpathlineto{\pgfqpoint{2.394532in}{0.882185in}}%
\pgfpathlineto{\pgfqpoint{2.394532in}{0.886443in}}%
\pgfpathlineto{\pgfqpoint{2.398790in}{0.886443in}}%
\pgfpathlineto{\pgfqpoint{2.398790in}{0.882185in}}%
\pgfpathmoveto{\pgfqpoint{2.394532in}{0.886443in}}%
\pgfpathlineto{\pgfqpoint{2.394532in}{0.886443in}}%
\pgfpathlineto{\pgfqpoint{2.394532in}{0.890700in}}%
\pgfpathlineto{\pgfqpoint{2.398790in}{0.890700in}}%
\pgfpathlineto{\pgfqpoint{2.398790in}{0.886443in}}%
\pgfpathmoveto{\pgfqpoint{2.398790in}{0.882185in}}%
\pgfpathlineto{\pgfqpoint{2.398790in}{0.882185in}}%
\pgfpathlineto{\pgfqpoint{2.398790in}{0.886443in}}%
\pgfpathlineto{\pgfqpoint{2.403048in}{0.886443in}}%
\pgfpathlineto{\pgfqpoint{2.403048in}{0.882185in}}%
\pgfpathmoveto{\pgfqpoint{2.398790in}{0.886443in}}%
\pgfpathlineto{\pgfqpoint{2.398790in}{0.886443in}}%
\pgfpathlineto{\pgfqpoint{2.398790in}{0.890700in}}%
\pgfpathlineto{\pgfqpoint{2.403048in}{0.890700in}}%
\pgfpathlineto{\pgfqpoint{2.403048in}{0.886443in}}%
\pgfpathmoveto{\pgfqpoint{2.394532in}{0.890700in}}%
\pgfpathlineto{\pgfqpoint{2.394532in}{0.890700in}}%
\pgfpathlineto{\pgfqpoint{2.394532in}{0.894958in}}%
\pgfpathlineto{\pgfqpoint{2.398790in}{0.894958in}}%
\pgfpathlineto{\pgfqpoint{2.398790in}{0.890700in}}%
\pgfpathmoveto{\pgfqpoint{2.394532in}{0.894958in}}%
\pgfpathlineto{\pgfqpoint{2.394532in}{0.894958in}}%
\pgfpathlineto{\pgfqpoint{2.394532in}{0.899216in}}%
\pgfpathlineto{\pgfqpoint{2.398790in}{0.899216in}}%
\pgfpathlineto{\pgfqpoint{2.398790in}{0.894958in}}%
\pgfpathmoveto{\pgfqpoint{2.398790in}{0.890700in}}%
\pgfpathlineto{\pgfqpoint{2.398790in}{0.890700in}}%
\pgfpathlineto{\pgfqpoint{2.398790in}{0.894958in}}%
\pgfpathlineto{\pgfqpoint{2.403048in}{0.894958in}}%
\pgfpathlineto{\pgfqpoint{2.403048in}{0.890700in}}%
\pgfpathmoveto{\pgfqpoint{2.398790in}{0.894958in}}%
\pgfpathlineto{\pgfqpoint{2.398790in}{0.894958in}}%
\pgfpathlineto{\pgfqpoint{2.398790in}{0.899216in}}%
\pgfpathlineto{\pgfqpoint{2.403048in}{0.899216in}}%
\pgfpathlineto{\pgfqpoint{2.403048in}{0.894958in}}%
\pgfpathmoveto{\pgfqpoint{2.394532in}{0.899216in}}%
\pgfpathlineto{\pgfqpoint{2.394532in}{0.899216in}}%
\pgfpathlineto{\pgfqpoint{2.394532in}{0.903473in}}%
\pgfpathlineto{\pgfqpoint{2.398790in}{0.903473in}}%
\pgfpathlineto{\pgfqpoint{2.398790in}{0.899216in}}%
\pgfpathmoveto{\pgfqpoint{2.394532in}{0.903473in}}%
\pgfpathlineto{\pgfqpoint{2.394532in}{0.903473in}}%
\pgfpathlineto{\pgfqpoint{2.394532in}{0.907731in}}%
\pgfpathlineto{\pgfqpoint{2.398790in}{0.907731in}}%
\pgfpathlineto{\pgfqpoint{2.398790in}{0.903473in}}%
\pgfpathmoveto{\pgfqpoint{2.398790in}{0.899216in}}%
\pgfpathlineto{\pgfqpoint{2.398790in}{0.899216in}}%
\pgfpathlineto{\pgfqpoint{2.398790in}{0.903473in}}%
\pgfpathlineto{\pgfqpoint{2.403048in}{0.903473in}}%
\pgfpathlineto{\pgfqpoint{2.403048in}{0.899216in}}%
\pgfpathmoveto{\pgfqpoint{2.398790in}{0.903473in}}%
\pgfpathlineto{\pgfqpoint{2.398790in}{0.903473in}}%
\pgfpathlineto{\pgfqpoint{2.398790in}{0.907731in}}%
\pgfpathlineto{\pgfqpoint{2.403048in}{0.907731in}}%
\pgfpathlineto{\pgfqpoint{2.403048in}{0.903473in}}%
\pgfpathmoveto{\pgfqpoint{2.394532in}{0.907731in}}%
\pgfpathlineto{\pgfqpoint{2.394532in}{0.907731in}}%
\pgfpathlineto{\pgfqpoint{2.394532in}{0.911989in}}%
\pgfpathlineto{\pgfqpoint{2.398790in}{0.911989in}}%
\pgfpathlineto{\pgfqpoint{2.398790in}{0.907731in}}%
\pgfpathmoveto{\pgfqpoint{2.394532in}{0.911989in}}%
\pgfpathlineto{\pgfqpoint{2.394532in}{0.911989in}}%
\pgfpathlineto{\pgfqpoint{2.394532in}{0.916246in}}%
\pgfpathlineto{\pgfqpoint{2.398790in}{0.916246in}}%
\pgfpathlineto{\pgfqpoint{2.398790in}{0.911989in}}%
\pgfpathmoveto{\pgfqpoint{2.398790in}{0.907731in}}%
\pgfpathlineto{\pgfqpoint{2.398790in}{0.907731in}}%
\pgfpathlineto{\pgfqpoint{2.398790in}{0.911989in}}%
\pgfpathlineto{\pgfqpoint{2.403048in}{0.911989in}}%
\pgfpathlineto{\pgfqpoint{2.403048in}{0.907731in}}%
\pgfpathmoveto{\pgfqpoint{2.398790in}{0.911989in}}%
\pgfpathlineto{\pgfqpoint{2.398790in}{0.911989in}}%
\pgfpathlineto{\pgfqpoint{2.398790in}{0.916246in}}%
\pgfpathlineto{\pgfqpoint{2.403048in}{0.916246in}}%
\pgfpathlineto{\pgfqpoint{2.403048in}{0.911989in}}%
\pgfpathmoveto{\pgfqpoint{2.394532in}{0.916246in}}%
\pgfpathlineto{\pgfqpoint{2.394532in}{0.916246in}}%
\pgfpathlineto{\pgfqpoint{2.394532in}{0.920504in}}%
\pgfpathlineto{\pgfqpoint{2.398790in}{0.920504in}}%
\pgfpathlineto{\pgfqpoint{2.398790in}{0.916246in}}%
\pgfpathmoveto{\pgfqpoint{2.398790in}{0.916246in}}%
\pgfpathlineto{\pgfqpoint{2.398790in}{0.916246in}}%
\pgfpathlineto{\pgfqpoint{2.398790in}{0.920504in}}%
\pgfpathlineto{\pgfqpoint{2.403048in}{0.920504in}}%
\pgfpathlineto{\pgfqpoint{2.403048in}{0.916246in}}%
\pgfpathmoveto{\pgfqpoint{2.398790in}{0.920504in}}%
\pgfpathlineto{\pgfqpoint{2.398790in}{0.920504in}}%
\pgfpathlineto{\pgfqpoint{2.398790in}{0.924762in}}%
\pgfpathlineto{\pgfqpoint{2.403048in}{0.924762in}}%
\pgfpathlineto{\pgfqpoint{2.403048in}{0.920504in}}%
\pgfpathmoveto{\pgfqpoint{2.398790in}{0.924762in}}%
\pgfpathlineto{\pgfqpoint{2.398790in}{0.924762in}}%
\pgfpathlineto{\pgfqpoint{2.398790in}{0.929020in}}%
\pgfpathlineto{\pgfqpoint{2.403048in}{0.929020in}}%
\pgfpathlineto{\pgfqpoint{2.403048in}{0.924762in}}%
\pgfpathmoveto{\pgfqpoint{2.398790in}{0.929020in}}%
\pgfpathlineto{\pgfqpoint{2.398790in}{0.929020in}}%
\pgfpathlineto{\pgfqpoint{2.398790in}{0.933278in}}%
\pgfpathlineto{\pgfqpoint{2.403048in}{0.933278in}}%
\pgfpathlineto{\pgfqpoint{2.403048in}{0.929020in}}%
\pgfpathmoveto{\pgfqpoint{2.403048in}{0.916246in}}%
\pgfpathlineto{\pgfqpoint{2.403048in}{0.916246in}}%
\pgfpathlineto{\pgfqpoint{2.403048in}{0.920504in}}%
\pgfpathlineto{\pgfqpoint{2.407306in}{0.920504in}}%
\pgfpathlineto{\pgfqpoint{2.407306in}{0.916246in}}%
\pgfpathmoveto{\pgfqpoint{2.403048in}{0.920504in}}%
\pgfpathlineto{\pgfqpoint{2.403048in}{0.920504in}}%
\pgfpathlineto{\pgfqpoint{2.403048in}{0.924762in}}%
\pgfpathlineto{\pgfqpoint{2.407306in}{0.924762in}}%
\pgfpathlineto{\pgfqpoint{2.407306in}{0.920504in}}%
\pgfpathmoveto{\pgfqpoint{2.403048in}{0.924762in}}%
\pgfpathlineto{\pgfqpoint{2.403048in}{0.924762in}}%
\pgfpathlineto{\pgfqpoint{2.403048in}{0.929020in}}%
\pgfpathlineto{\pgfqpoint{2.407306in}{0.929020in}}%
\pgfpathlineto{\pgfqpoint{2.407306in}{0.924762in}}%
\pgfpathmoveto{\pgfqpoint{2.403048in}{0.929020in}}%
\pgfpathlineto{\pgfqpoint{2.403048in}{0.929020in}}%
\pgfpathlineto{\pgfqpoint{2.403048in}{0.933278in}}%
\pgfpathlineto{\pgfqpoint{2.407306in}{0.933278in}}%
\pgfpathlineto{\pgfqpoint{2.407306in}{0.929020in}}%
\pgfpathmoveto{\pgfqpoint{2.398790in}{0.933278in}}%
\pgfpathlineto{\pgfqpoint{2.398790in}{0.933278in}}%
\pgfpathlineto{\pgfqpoint{2.398790in}{0.937535in}}%
\pgfpathlineto{\pgfqpoint{2.403048in}{0.937535in}}%
\pgfpathlineto{\pgfqpoint{2.403048in}{0.933278in}}%
\pgfpathmoveto{\pgfqpoint{2.398790in}{0.937535in}}%
\pgfpathlineto{\pgfqpoint{2.398790in}{0.937535in}}%
\pgfpathlineto{\pgfqpoint{2.398790in}{0.941793in}}%
\pgfpathlineto{\pgfqpoint{2.403048in}{0.941793in}}%
\pgfpathlineto{\pgfqpoint{2.403048in}{0.937535in}}%
\pgfpathmoveto{\pgfqpoint{2.398790in}{0.941793in}}%
\pgfpathlineto{\pgfqpoint{2.398790in}{0.941793in}}%
\pgfpathlineto{\pgfqpoint{2.398790in}{0.946051in}}%
\pgfpathlineto{\pgfqpoint{2.403048in}{0.946051in}}%
\pgfpathlineto{\pgfqpoint{2.403048in}{0.941793in}}%
\pgfpathmoveto{\pgfqpoint{2.398790in}{0.946051in}}%
\pgfpathlineto{\pgfqpoint{2.398790in}{0.946051in}}%
\pgfpathlineto{\pgfqpoint{2.398790in}{0.950309in}}%
\pgfpathlineto{\pgfqpoint{2.403048in}{0.950309in}}%
\pgfpathlineto{\pgfqpoint{2.403048in}{0.946051in}}%
\pgfpathmoveto{\pgfqpoint{2.403048in}{0.933278in}}%
\pgfpathlineto{\pgfqpoint{2.403048in}{0.933278in}}%
\pgfpathlineto{\pgfqpoint{2.403048in}{0.937535in}}%
\pgfpathlineto{\pgfqpoint{2.407306in}{0.937535in}}%
\pgfpathlineto{\pgfqpoint{2.407306in}{0.933278in}}%
\pgfpathmoveto{\pgfqpoint{2.403048in}{0.937535in}}%
\pgfpathlineto{\pgfqpoint{2.403048in}{0.937535in}}%
\pgfpathlineto{\pgfqpoint{2.403048in}{0.941793in}}%
\pgfpathlineto{\pgfqpoint{2.407306in}{0.941793in}}%
\pgfpathlineto{\pgfqpoint{2.407306in}{0.937535in}}%
\pgfpathmoveto{\pgfqpoint{2.403048in}{0.941793in}}%
\pgfpathlineto{\pgfqpoint{2.403048in}{0.941793in}}%
\pgfpathlineto{\pgfqpoint{2.403048in}{0.946051in}}%
\pgfpathlineto{\pgfqpoint{2.407306in}{0.946051in}}%
\pgfpathlineto{\pgfqpoint{2.407306in}{0.941793in}}%
\pgfpathmoveto{\pgfqpoint{2.403048in}{0.946051in}}%
\pgfpathlineto{\pgfqpoint{2.403048in}{0.946051in}}%
\pgfpathlineto{\pgfqpoint{2.403048in}{0.950309in}}%
\pgfpathlineto{\pgfqpoint{2.407306in}{0.950309in}}%
\pgfpathlineto{\pgfqpoint{2.407306in}{0.946051in}}%
\pgfpathmoveto{\pgfqpoint{2.398790in}{0.950309in}}%
\pgfpathlineto{\pgfqpoint{2.398790in}{0.950309in}}%
\pgfpathlineto{\pgfqpoint{2.398790in}{0.954567in}}%
\pgfpathlineto{\pgfqpoint{2.403048in}{0.954567in}}%
\pgfpathlineto{\pgfqpoint{2.403048in}{0.950309in}}%
\pgfpathmoveto{\pgfqpoint{2.398790in}{0.954567in}}%
\pgfpathlineto{\pgfqpoint{2.398790in}{0.954567in}}%
\pgfpathlineto{\pgfqpoint{2.398790in}{0.958825in}}%
\pgfpathlineto{\pgfqpoint{2.403048in}{0.958825in}}%
\pgfpathlineto{\pgfqpoint{2.403048in}{0.954567in}}%
\pgfpathmoveto{\pgfqpoint{2.398790in}{0.958825in}}%
\pgfpathlineto{\pgfqpoint{2.398790in}{0.958825in}}%
\pgfpathlineto{\pgfqpoint{2.398790in}{0.963082in}}%
\pgfpathlineto{\pgfqpoint{2.403048in}{0.963082in}}%
\pgfpathlineto{\pgfqpoint{2.403048in}{0.958825in}}%
\pgfpathmoveto{\pgfqpoint{2.403048in}{0.950309in}}%
\pgfpathlineto{\pgfqpoint{2.403048in}{0.950309in}}%
\pgfpathlineto{\pgfqpoint{2.403048in}{0.954567in}}%
\pgfpathlineto{\pgfqpoint{2.407306in}{0.954567in}}%
\pgfpathlineto{\pgfqpoint{2.407306in}{0.950309in}}%
\pgfpathmoveto{\pgfqpoint{2.403048in}{0.954567in}}%
\pgfpathlineto{\pgfqpoint{2.403048in}{0.954567in}}%
\pgfpathlineto{\pgfqpoint{2.403048in}{0.958825in}}%
\pgfpathlineto{\pgfqpoint{2.407306in}{0.958825in}}%
\pgfpathlineto{\pgfqpoint{2.407306in}{0.954567in}}%
\pgfpathmoveto{\pgfqpoint{2.403048in}{0.958825in}}%
\pgfpathlineto{\pgfqpoint{2.403048in}{0.958825in}}%
\pgfpathlineto{\pgfqpoint{2.403048in}{0.963082in}}%
\pgfpathlineto{\pgfqpoint{2.407306in}{0.963082in}}%
\pgfpathlineto{\pgfqpoint{2.407306in}{0.958825in}}%
\pgfpathmoveto{\pgfqpoint{2.403048in}{0.963082in}}%
\pgfpathlineto{\pgfqpoint{2.403048in}{0.963082in}}%
\pgfpathlineto{\pgfqpoint{2.403048in}{0.967340in}}%
\pgfpathlineto{\pgfqpoint{2.407306in}{0.967340in}}%
\pgfpathlineto{\pgfqpoint{2.407306in}{0.963082in}}%
\pgfpathmoveto{\pgfqpoint{2.407306in}{0.958825in}}%
\pgfpathlineto{\pgfqpoint{2.407306in}{0.958825in}}%
\pgfpathlineto{\pgfqpoint{2.407306in}{0.963082in}}%
\pgfpathlineto{\pgfqpoint{2.411563in}{0.963082in}}%
\pgfpathlineto{\pgfqpoint{2.411563in}{0.958825in}}%
\pgfpathmoveto{\pgfqpoint{2.407306in}{0.963082in}}%
\pgfpathlineto{\pgfqpoint{2.407306in}{0.963082in}}%
\pgfpathlineto{\pgfqpoint{2.407306in}{0.967340in}}%
\pgfpathlineto{\pgfqpoint{2.411563in}{0.967340in}}%
\pgfpathlineto{\pgfqpoint{2.411563in}{0.963082in}}%
\pgfpathmoveto{\pgfqpoint{2.403048in}{0.967340in}}%
\pgfpathlineto{\pgfqpoint{2.403048in}{0.967340in}}%
\pgfpathlineto{\pgfqpoint{2.403048in}{0.971598in}}%
\pgfpathlineto{\pgfqpoint{2.407306in}{0.971598in}}%
\pgfpathlineto{\pgfqpoint{2.407306in}{0.967340in}}%
\pgfpathmoveto{\pgfqpoint{2.403048in}{0.971598in}}%
\pgfpathlineto{\pgfqpoint{2.403048in}{0.971598in}}%
\pgfpathlineto{\pgfqpoint{2.403048in}{0.975856in}}%
\pgfpathlineto{\pgfqpoint{2.407306in}{0.975856in}}%
\pgfpathlineto{\pgfqpoint{2.407306in}{0.971598in}}%
\pgfpathmoveto{\pgfqpoint{2.407306in}{0.967340in}}%
\pgfpathlineto{\pgfqpoint{2.407306in}{0.967340in}}%
\pgfpathlineto{\pgfqpoint{2.407306in}{0.971598in}}%
\pgfpathlineto{\pgfqpoint{2.411563in}{0.971598in}}%
\pgfpathlineto{\pgfqpoint{2.411563in}{0.967340in}}%
\pgfpathmoveto{\pgfqpoint{2.407306in}{0.971598in}}%
\pgfpathlineto{\pgfqpoint{2.407306in}{0.971598in}}%
\pgfpathlineto{\pgfqpoint{2.407306in}{0.975856in}}%
\pgfpathlineto{\pgfqpoint{2.411563in}{0.975856in}}%
\pgfpathlineto{\pgfqpoint{2.411563in}{0.971598in}}%
\pgfpathmoveto{\pgfqpoint{2.403048in}{0.975856in}}%
\pgfpathlineto{\pgfqpoint{2.403048in}{0.975856in}}%
\pgfpathlineto{\pgfqpoint{2.403048in}{0.980114in}}%
\pgfpathlineto{\pgfqpoint{2.407306in}{0.980114in}}%
\pgfpathlineto{\pgfqpoint{2.407306in}{0.975856in}}%
\pgfpathmoveto{\pgfqpoint{2.403048in}{0.980114in}}%
\pgfpathlineto{\pgfqpoint{2.403048in}{0.980114in}}%
\pgfpathlineto{\pgfqpoint{2.403048in}{0.984372in}}%
\pgfpathlineto{\pgfqpoint{2.407306in}{0.984372in}}%
\pgfpathlineto{\pgfqpoint{2.407306in}{0.980114in}}%
\pgfpathmoveto{\pgfqpoint{2.407306in}{0.975856in}}%
\pgfpathlineto{\pgfqpoint{2.407306in}{0.975856in}}%
\pgfpathlineto{\pgfqpoint{2.407306in}{0.980114in}}%
\pgfpathlineto{\pgfqpoint{2.411563in}{0.980114in}}%
\pgfpathlineto{\pgfqpoint{2.411563in}{0.975856in}}%
\pgfpathmoveto{\pgfqpoint{2.407306in}{0.980114in}}%
\pgfpathlineto{\pgfqpoint{2.407306in}{0.980114in}}%
\pgfpathlineto{\pgfqpoint{2.407306in}{0.984372in}}%
\pgfpathlineto{\pgfqpoint{2.411563in}{0.984372in}}%
\pgfpathlineto{\pgfqpoint{2.411563in}{0.980114in}}%
\pgfpathmoveto{\pgfqpoint{2.403048in}{0.984372in}}%
\pgfpathlineto{\pgfqpoint{2.403048in}{0.984372in}}%
\pgfpathlineto{\pgfqpoint{2.403048in}{0.988629in}}%
\pgfpathlineto{\pgfqpoint{2.407306in}{0.988629in}}%
\pgfpathlineto{\pgfqpoint{2.407306in}{0.984372in}}%
\pgfpathmoveto{\pgfqpoint{2.403048in}{0.988629in}}%
\pgfpathlineto{\pgfqpoint{2.403048in}{0.988629in}}%
\pgfpathlineto{\pgfqpoint{2.403048in}{0.992887in}}%
\pgfpathlineto{\pgfqpoint{2.407306in}{0.992887in}}%
\pgfpathlineto{\pgfqpoint{2.407306in}{0.988629in}}%
\pgfpathmoveto{\pgfqpoint{2.407306in}{0.984372in}}%
\pgfpathlineto{\pgfqpoint{2.407306in}{0.984372in}}%
\pgfpathlineto{\pgfqpoint{2.407306in}{0.988629in}}%
\pgfpathlineto{\pgfqpoint{2.411563in}{0.988629in}}%
\pgfpathlineto{\pgfqpoint{2.411563in}{0.984372in}}%
\pgfpathmoveto{\pgfqpoint{2.407306in}{0.988629in}}%
\pgfpathlineto{\pgfqpoint{2.407306in}{0.988629in}}%
\pgfpathlineto{\pgfqpoint{2.407306in}{0.992887in}}%
\pgfpathlineto{\pgfqpoint{2.411563in}{0.992887in}}%
\pgfpathlineto{\pgfqpoint{2.411563in}{0.988629in}}%
\pgfpathmoveto{\pgfqpoint{2.403048in}{0.992887in}}%
\pgfpathlineto{\pgfqpoint{2.403048in}{0.992887in}}%
\pgfpathlineto{\pgfqpoint{2.403048in}{0.997145in}}%
\pgfpathlineto{\pgfqpoint{2.407306in}{0.997145in}}%
\pgfpathlineto{\pgfqpoint{2.407306in}{0.992887in}}%
\pgfpathmoveto{\pgfqpoint{2.403048in}{0.997145in}}%
\pgfpathlineto{\pgfqpoint{2.403048in}{0.997145in}}%
\pgfpathlineto{\pgfqpoint{2.403048in}{1.001403in}}%
\pgfpathlineto{\pgfqpoint{2.407306in}{1.001403in}}%
\pgfpathlineto{\pgfqpoint{2.407306in}{0.997145in}}%
\pgfpathmoveto{\pgfqpoint{2.407306in}{0.992887in}}%
\pgfpathlineto{\pgfqpoint{2.407306in}{0.992887in}}%
\pgfpathlineto{\pgfqpoint{2.407306in}{0.997145in}}%
\pgfpathlineto{\pgfqpoint{2.411563in}{0.997145in}}%
\pgfpathlineto{\pgfqpoint{2.411563in}{0.992887in}}%
\pgfpathmoveto{\pgfqpoint{2.407306in}{0.997145in}}%
\pgfpathlineto{\pgfqpoint{2.407306in}{0.997145in}}%
\pgfpathlineto{\pgfqpoint{2.407306in}{1.001403in}}%
\pgfpathlineto{\pgfqpoint{2.411563in}{1.001403in}}%
\pgfpathlineto{\pgfqpoint{2.411563in}{0.997145in}}%
\pgfpathmoveto{\pgfqpoint{2.403048in}{1.001403in}}%
\pgfpathlineto{\pgfqpoint{2.403048in}{1.001403in}}%
\pgfpathlineto{\pgfqpoint{2.403048in}{1.005661in}}%
\pgfpathlineto{\pgfqpoint{2.407306in}{1.005661in}}%
\pgfpathlineto{\pgfqpoint{2.407306in}{1.001403in}}%
\pgfpathmoveto{\pgfqpoint{2.407306in}{1.001403in}}%
\pgfpathlineto{\pgfqpoint{2.407306in}{1.001403in}}%
\pgfpathlineto{\pgfqpoint{2.407306in}{1.005661in}}%
\pgfpathlineto{\pgfqpoint{2.411563in}{1.005661in}}%
\pgfpathlineto{\pgfqpoint{2.411563in}{1.001403in}}%
\pgfpathmoveto{\pgfqpoint{2.407306in}{1.005661in}}%
\pgfpathlineto{\pgfqpoint{2.407306in}{1.005661in}}%
\pgfpathlineto{\pgfqpoint{2.407306in}{1.009919in}}%
\pgfpathlineto{\pgfqpoint{2.411563in}{1.009919in}}%
\pgfpathlineto{\pgfqpoint{2.411563in}{1.005661in}}%
\pgfpathmoveto{\pgfqpoint{2.407306in}{1.009919in}}%
\pgfpathlineto{\pgfqpoint{2.407306in}{1.009919in}}%
\pgfpathlineto{\pgfqpoint{2.407306in}{1.014176in}}%
\pgfpathlineto{\pgfqpoint{2.411563in}{1.014176in}}%
\pgfpathlineto{\pgfqpoint{2.411563in}{1.009919in}}%
\pgfpathmoveto{\pgfqpoint{2.407306in}{1.014176in}}%
\pgfpathlineto{\pgfqpoint{2.407306in}{1.014176in}}%
\pgfpathlineto{\pgfqpoint{2.407306in}{1.018434in}}%
\pgfpathlineto{\pgfqpoint{2.411563in}{1.018434in}}%
\pgfpathlineto{\pgfqpoint{2.411563in}{1.014176in}}%
\pgfpathmoveto{\pgfqpoint{2.407306in}{1.018434in}}%
\pgfpathlineto{\pgfqpoint{2.407306in}{1.018434in}}%
\pgfpathlineto{\pgfqpoint{2.407306in}{1.022692in}}%
\pgfpathlineto{\pgfqpoint{2.411563in}{1.022692in}}%
\pgfpathlineto{\pgfqpoint{2.411563in}{1.018434in}}%
\pgfpathmoveto{\pgfqpoint{2.407306in}{1.022692in}}%
\pgfpathlineto{\pgfqpoint{2.407306in}{1.022692in}}%
\pgfpathlineto{\pgfqpoint{2.407306in}{1.026950in}}%
\pgfpathlineto{\pgfqpoint{2.411563in}{1.026950in}}%
\pgfpathlineto{\pgfqpoint{2.411563in}{1.022692in}}%
\pgfpathmoveto{\pgfqpoint{2.407306in}{1.026950in}}%
\pgfpathlineto{\pgfqpoint{2.407306in}{1.026950in}}%
\pgfpathlineto{\pgfqpoint{2.407306in}{1.031208in}}%
\pgfpathlineto{\pgfqpoint{2.411563in}{1.031208in}}%
\pgfpathlineto{\pgfqpoint{2.411563in}{1.026950in}}%
\pgfpathmoveto{\pgfqpoint{2.407306in}{1.031208in}}%
\pgfpathlineto{\pgfqpoint{2.407306in}{1.031208in}}%
\pgfpathlineto{\pgfqpoint{2.407306in}{1.035466in}}%
\pgfpathlineto{\pgfqpoint{2.411563in}{1.035466in}}%
\pgfpathlineto{\pgfqpoint{2.411563in}{1.031208in}}%
\pgfpathmoveto{\pgfqpoint{2.407306in}{1.035466in}}%
\pgfpathlineto{\pgfqpoint{2.407306in}{1.035466in}}%
\pgfpathlineto{\pgfqpoint{2.407306in}{1.039723in}}%
\pgfpathlineto{\pgfqpoint{2.411563in}{1.039723in}}%
\pgfpathlineto{\pgfqpoint{2.411563in}{1.035466in}}%
\pgfpathmoveto{\pgfqpoint{2.407306in}{1.039723in}}%
\pgfpathlineto{\pgfqpoint{2.407306in}{1.039723in}}%
\pgfpathlineto{\pgfqpoint{2.407306in}{1.043981in}}%
\pgfpathlineto{\pgfqpoint{2.411563in}{1.043981in}}%
\pgfpathlineto{\pgfqpoint{2.411563in}{1.039723in}}%
\pgfpathmoveto{\pgfqpoint{2.407306in}{1.043981in}}%
\pgfpathlineto{\pgfqpoint{2.407306in}{1.043981in}}%
\pgfpathlineto{\pgfqpoint{2.407306in}{1.048239in}}%
\pgfpathlineto{\pgfqpoint{2.411563in}{1.048239in}}%
\pgfpathlineto{\pgfqpoint{2.411563in}{1.043981in}}%
\pgfpathmoveto{\pgfqpoint{2.411563in}{1.001403in}}%
\pgfpathlineto{\pgfqpoint{2.411563in}{1.001403in}}%
\pgfpathlineto{\pgfqpoint{2.411563in}{1.005661in}}%
\pgfpathlineto{\pgfqpoint{2.415821in}{1.005661in}}%
\pgfpathlineto{\pgfqpoint{2.415821in}{1.001403in}}%
\pgfpathmoveto{\pgfqpoint{2.411563in}{1.005661in}}%
\pgfpathlineto{\pgfqpoint{2.411563in}{1.005661in}}%
\pgfpathlineto{\pgfqpoint{2.411563in}{1.009919in}}%
\pgfpathlineto{\pgfqpoint{2.415821in}{1.009919in}}%
\pgfpathlineto{\pgfqpoint{2.415821in}{1.005661in}}%
\pgfpathmoveto{\pgfqpoint{2.411563in}{1.009919in}}%
\pgfpathlineto{\pgfqpoint{2.411563in}{1.009919in}}%
\pgfpathlineto{\pgfqpoint{2.411563in}{1.014176in}}%
\pgfpathlineto{\pgfqpoint{2.415821in}{1.014176in}}%
\pgfpathlineto{\pgfqpoint{2.415821in}{1.009919in}}%
\pgfpathmoveto{\pgfqpoint{2.411563in}{1.014176in}}%
\pgfpathlineto{\pgfqpoint{2.411563in}{1.014176in}}%
\pgfpathlineto{\pgfqpoint{2.411563in}{1.018434in}}%
\pgfpathlineto{\pgfqpoint{2.415821in}{1.018434in}}%
\pgfpathlineto{\pgfqpoint{2.415821in}{1.014176in}}%
\pgfpathmoveto{\pgfqpoint{2.411563in}{1.018434in}}%
\pgfpathlineto{\pgfqpoint{2.411563in}{1.018434in}}%
\pgfpathlineto{\pgfqpoint{2.411563in}{1.022692in}}%
\pgfpathlineto{\pgfqpoint{2.415821in}{1.022692in}}%
\pgfpathlineto{\pgfqpoint{2.415821in}{1.018434in}}%
\pgfpathmoveto{\pgfqpoint{2.411563in}{1.022692in}}%
\pgfpathlineto{\pgfqpoint{2.411563in}{1.022692in}}%
\pgfpathlineto{\pgfqpoint{2.411563in}{1.026950in}}%
\pgfpathlineto{\pgfqpoint{2.415821in}{1.026950in}}%
\pgfpathlineto{\pgfqpoint{2.415821in}{1.022692in}}%
\pgfpathmoveto{\pgfqpoint{2.411563in}{1.026950in}}%
\pgfpathlineto{\pgfqpoint{2.411563in}{1.026950in}}%
\pgfpathlineto{\pgfqpoint{2.411563in}{1.031208in}}%
\pgfpathlineto{\pgfqpoint{2.415821in}{1.031208in}}%
\pgfpathlineto{\pgfqpoint{2.415821in}{1.026950in}}%
\pgfpathmoveto{\pgfqpoint{2.411563in}{1.031208in}}%
\pgfpathlineto{\pgfqpoint{2.411563in}{1.031208in}}%
\pgfpathlineto{\pgfqpoint{2.411563in}{1.035466in}}%
\pgfpathlineto{\pgfqpoint{2.415821in}{1.035466in}}%
\pgfpathlineto{\pgfqpoint{2.415821in}{1.031208in}}%
\pgfpathmoveto{\pgfqpoint{2.411563in}{1.035466in}}%
\pgfpathlineto{\pgfqpoint{2.411563in}{1.035466in}}%
\pgfpathlineto{\pgfqpoint{2.411563in}{1.039723in}}%
\pgfpathlineto{\pgfqpoint{2.415821in}{1.039723in}}%
\pgfpathlineto{\pgfqpoint{2.415821in}{1.035466in}}%
\pgfpathmoveto{\pgfqpoint{2.411563in}{1.039723in}}%
\pgfpathlineto{\pgfqpoint{2.411563in}{1.039723in}}%
\pgfpathlineto{\pgfqpoint{2.411563in}{1.043981in}}%
\pgfpathlineto{\pgfqpoint{2.415821in}{1.043981in}}%
\pgfpathlineto{\pgfqpoint{2.415821in}{1.039723in}}%
\pgfpathmoveto{\pgfqpoint{2.411563in}{1.043981in}}%
\pgfpathlineto{\pgfqpoint{2.411563in}{1.043981in}}%
\pgfpathlineto{\pgfqpoint{2.411563in}{1.048239in}}%
\pgfpathlineto{\pgfqpoint{2.415821in}{1.048239in}}%
\pgfpathlineto{\pgfqpoint{2.415821in}{1.043981in}}%
\pgfpathmoveto{\pgfqpoint{2.411563in}{1.048239in}}%
\pgfpathlineto{\pgfqpoint{2.411563in}{1.048239in}}%
\pgfpathlineto{\pgfqpoint{2.411563in}{1.052497in}}%
\pgfpathlineto{\pgfqpoint{2.415821in}{1.052497in}}%
\pgfpathlineto{\pgfqpoint{2.415821in}{1.048239in}}%
\pgfpathmoveto{\pgfqpoint{2.415821in}{1.043981in}}%
\pgfpathlineto{\pgfqpoint{2.415821in}{1.043981in}}%
\pgfpathlineto{\pgfqpoint{2.415821in}{1.048239in}}%
\pgfpathlineto{\pgfqpoint{2.420079in}{1.048239in}}%
\pgfpathlineto{\pgfqpoint{2.420079in}{1.043981in}}%
\pgfpathmoveto{\pgfqpoint{2.415821in}{1.048239in}}%
\pgfpathlineto{\pgfqpoint{2.415821in}{1.048239in}}%
\pgfpathlineto{\pgfqpoint{2.415821in}{1.052497in}}%
\pgfpathlineto{\pgfqpoint{2.420079in}{1.052497in}}%
\pgfpathlineto{\pgfqpoint{2.420079in}{1.048239in}}%
\pgfpathmoveto{\pgfqpoint{2.411563in}{1.052497in}}%
\pgfpathlineto{\pgfqpoint{2.411563in}{1.052497in}}%
\pgfpathlineto{\pgfqpoint{2.411563in}{1.056755in}}%
\pgfpathlineto{\pgfqpoint{2.415821in}{1.056755in}}%
\pgfpathlineto{\pgfqpoint{2.415821in}{1.052497in}}%
\pgfpathmoveto{\pgfqpoint{2.411563in}{1.056755in}}%
\pgfpathlineto{\pgfqpoint{2.411563in}{1.056755in}}%
\pgfpathlineto{\pgfqpoint{2.411563in}{1.061013in}}%
\pgfpathlineto{\pgfqpoint{2.415821in}{1.061013in}}%
\pgfpathlineto{\pgfqpoint{2.415821in}{1.056755in}}%
\pgfpathmoveto{\pgfqpoint{2.415821in}{1.052497in}}%
\pgfpathlineto{\pgfqpoint{2.415821in}{1.052497in}}%
\pgfpathlineto{\pgfqpoint{2.415821in}{1.056755in}}%
\pgfpathlineto{\pgfqpoint{2.420079in}{1.056755in}}%
\pgfpathlineto{\pgfqpoint{2.420079in}{1.052497in}}%
\pgfpathmoveto{\pgfqpoint{2.415821in}{1.056755in}}%
\pgfpathlineto{\pgfqpoint{2.415821in}{1.056755in}}%
\pgfpathlineto{\pgfqpoint{2.415821in}{1.061013in}}%
\pgfpathlineto{\pgfqpoint{2.420079in}{1.061013in}}%
\pgfpathlineto{\pgfqpoint{2.420079in}{1.056755in}}%
\pgfpathmoveto{\pgfqpoint{2.411563in}{1.061013in}}%
\pgfpathlineto{\pgfqpoint{2.411563in}{1.061013in}}%
\pgfpathlineto{\pgfqpoint{2.411563in}{1.065271in}}%
\pgfpathlineto{\pgfqpoint{2.415821in}{1.065271in}}%
\pgfpathlineto{\pgfqpoint{2.415821in}{1.061013in}}%
\pgfpathmoveto{\pgfqpoint{2.411563in}{1.065271in}}%
\pgfpathlineto{\pgfqpoint{2.411563in}{1.065271in}}%
\pgfpathlineto{\pgfqpoint{2.411563in}{1.069529in}}%
\pgfpathlineto{\pgfqpoint{2.415821in}{1.069529in}}%
\pgfpathlineto{\pgfqpoint{2.415821in}{1.065271in}}%
\pgfpathmoveto{\pgfqpoint{2.415821in}{1.061013in}}%
\pgfpathlineto{\pgfqpoint{2.415821in}{1.061013in}}%
\pgfpathlineto{\pgfqpoint{2.415821in}{1.065271in}}%
\pgfpathlineto{\pgfqpoint{2.420079in}{1.065271in}}%
\pgfpathlineto{\pgfqpoint{2.420079in}{1.061013in}}%
\pgfpathmoveto{\pgfqpoint{2.415821in}{1.065271in}}%
\pgfpathlineto{\pgfqpoint{2.415821in}{1.065271in}}%
\pgfpathlineto{\pgfqpoint{2.415821in}{1.069529in}}%
\pgfpathlineto{\pgfqpoint{2.420079in}{1.069529in}}%
\pgfpathlineto{\pgfqpoint{2.420079in}{1.065271in}}%
\pgfpathmoveto{\pgfqpoint{2.411563in}{1.069529in}}%
\pgfpathlineto{\pgfqpoint{2.411563in}{1.069529in}}%
\pgfpathlineto{\pgfqpoint{2.411563in}{1.073787in}}%
\pgfpathlineto{\pgfqpoint{2.415821in}{1.073787in}}%
\pgfpathlineto{\pgfqpoint{2.415821in}{1.069529in}}%
\pgfpathmoveto{\pgfqpoint{2.411563in}{1.073787in}}%
\pgfpathlineto{\pgfqpoint{2.411563in}{1.073787in}}%
\pgfpathlineto{\pgfqpoint{2.411563in}{1.078045in}}%
\pgfpathlineto{\pgfqpoint{2.415821in}{1.078045in}}%
\pgfpathlineto{\pgfqpoint{2.415821in}{1.073787in}}%
\pgfpathmoveto{\pgfqpoint{2.415821in}{1.069529in}}%
\pgfpathlineto{\pgfqpoint{2.415821in}{1.069529in}}%
\pgfpathlineto{\pgfqpoint{2.415821in}{1.073787in}}%
\pgfpathlineto{\pgfqpoint{2.420079in}{1.073787in}}%
\pgfpathlineto{\pgfqpoint{2.420079in}{1.069529in}}%
\pgfpathmoveto{\pgfqpoint{2.415821in}{1.073787in}}%
\pgfpathlineto{\pgfqpoint{2.415821in}{1.073787in}}%
\pgfpathlineto{\pgfqpoint{2.415821in}{1.078045in}}%
\pgfpathlineto{\pgfqpoint{2.420079in}{1.078045in}}%
\pgfpathlineto{\pgfqpoint{2.420079in}{1.073787in}}%
\pgfpathmoveto{\pgfqpoint{2.411563in}{1.078045in}}%
\pgfpathlineto{\pgfqpoint{2.411563in}{1.078045in}}%
\pgfpathlineto{\pgfqpoint{2.411563in}{1.082303in}}%
\pgfpathlineto{\pgfqpoint{2.415821in}{1.082303in}}%
\pgfpathlineto{\pgfqpoint{2.415821in}{1.078045in}}%
\pgfpathmoveto{\pgfqpoint{2.411563in}{1.082303in}}%
\pgfpathlineto{\pgfqpoint{2.411563in}{1.082303in}}%
\pgfpathlineto{\pgfqpoint{2.411563in}{1.086561in}}%
\pgfpathlineto{\pgfqpoint{2.415821in}{1.086561in}}%
\pgfpathlineto{\pgfqpoint{2.415821in}{1.082303in}}%
\pgfpathmoveto{\pgfqpoint{2.415821in}{1.078045in}}%
\pgfpathlineto{\pgfqpoint{2.415821in}{1.078045in}}%
\pgfpathlineto{\pgfqpoint{2.415821in}{1.082303in}}%
\pgfpathlineto{\pgfqpoint{2.420079in}{1.082303in}}%
\pgfpathlineto{\pgfqpoint{2.420079in}{1.078045in}}%
\pgfpathmoveto{\pgfqpoint{2.415821in}{1.082303in}}%
\pgfpathlineto{\pgfqpoint{2.415821in}{1.082303in}}%
\pgfpathlineto{\pgfqpoint{2.415821in}{1.086561in}}%
\pgfpathlineto{\pgfqpoint{2.420079in}{1.086561in}}%
\pgfpathlineto{\pgfqpoint{2.420079in}{1.082303in}}%
\pgfpathmoveto{\pgfqpoint{2.411563in}{1.086561in}}%
\pgfpathlineto{\pgfqpoint{2.411563in}{1.086561in}}%
\pgfpathlineto{\pgfqpoint{2.411563in}{1.090819in}}%
\pgfpathlineto{\pgfqpoint{2.415821in}{1.090819in}}%
\pgfpathlineto{\pgfqpoint{2.415821in}{1.086561in}}%
\pgfpathmoveto{\pgfqpoint{2.415821in}{1.086561in}}%
\pgfpathlineto{\pgfqpoint{2.415821in}{1.086561in}}%
\pgfpathlineto{\pgfqpoint{2.415821in}{1.090819in}}%
\pgfpathlineto{\pgfqpoint{2.420079in}{1.090819in}}%
\pgfpathlineto{\pgfqpoint{2.420079in}{1.086561in}}%
\pgfpathmoveto{\pgfqpoint{2.415821in}{1.090819in}}%
\pgfpathlineto{\pgfqpoint{2.415821in}{1.090819in}}%
\pgfpathlineto{\pgfqpoint{2.415821in}{1.095077in}}%
\pgfpathlineto{\pgfqpoint{2.420079in}{1.095077in}}%
\pgfpathlineto{\pgfqpoint{2.420079in}{1.090819in}}%
\pgfpathmoveto{\pgfqpoint{2.415821in}{1.095077in}}%
\pgfpathlineto{\pgfqpoint{2.415821in}{1.095077in}}%
\pgfpathlineto{\pgfqpoint{2.415821in}{1.099335in}}%
\pgfpathlineto{\pgfqpoint{2.420079in}{1.099335in}}%
\pgfpathlineto{\pgfqpoint{2.420079in}{1.095077in}}%
\pgfpathmoveto{\pgfqpoint{2.415821in}{1.099335in}}%
\pgfpathlineto{\pgfqpoint{2.415821in}{1.099335in}}%
\pgfpathlineto{\pgfqpoint{2.415821in}{1.103593in}}%
\pgfpathlineto{\pgfqpoint{2.420079in}{1.103593in}}%
\pgfpathlineto{\pgfqpoint{2.420079in}{1.099335in}}%
\pgfpathmoveto{\pgfqpoint{2.420079in}{1.086561in}}%
\pgfpathlineto{\pgfqpoint{2.420079in}{1.086561in}}%
\pgfpathlineto{\pgfqpoint{2.420079in}{1.090819in}}%
\pgfpathlineto{\pgfqpoint{2.424337in}{1.090819in}}%
\pgfpathlineto{\pgfqpoint{2.424337in}{1.086561in}}%
\pgfpathmoveto{\pgfqpoint{2.420079in}{1.090819in}}%
\pgfpathlineto{\pgfqpoint{2.420079in}{1.090819in}}%
\pgfpathlineto{\pgfqpoint{2.420079in}{1.095077in}}%
\pgfpathlineto{\pgfqpoint{2.424337in}{1.095077in}}%
\pgfpathlineto{\pgfqpoint{2.424337in}{1.090819in}}%
\pgfpathmoveto{\pgfqpoint{2.420079in}{1.095077in}}%
\pgfpathlineto{\pgfqpoint{2.420079in}{1.095077in}}%
\pgfpathlineto{\pgfqpoint{2.420079in}{1.099335in}}%
\pgfpathlineto{\pgfqpoint{2.424337in}{1.099335in}}%
\pgfpathlineto{\pgfqpoint{2.424337in}{1.095077in}}%
\pgfpathmoveto{\pgfqpoint{2.420079in}{1.099335in}}%
\pgfpathlineto{\pgfqpoint{2.420079in}{1.099335in}}%
\pgfpathlineto{\pgfqpoint{2.420079in}{1.103593in}}%
\pgfpathlineto{\pgfqpoint{2.424337in}{1.103593in}}%
\pgfpathlineto{\pgfqpoint{2.424337in}{1.099335in}}%
\pgfpathmoveto{\pgfqpoint{2.415821in}{1.103593in}}%
\pgfpathlineto{\pgfqpoint{2.415821in}{1.103593in}}%
\pgfpathlineto{\pgfqpoint{2.415821in}{1.107851in}}%
\pgfpathlineto{\pgfqpoint{2.420079in}{1.107851in}}%
\pgfpathlineto{\pgfqpoint{2.420079in}{1.103593in}}%
\pgfpathmoveto{\pgfqpoint{2.415821in}{1.107851in}}%
\pgfpathlineto{\pgfqpoint{2.415821in}{1.107851in}}%
\pgfpathlineto{\pgfqpoint{2.415821in}{1.112109in}}%
\pgfpathlineto{\pgfqpoint{2.420079in}{1.112109in}}%
\pgfpathlineto{\pgfqpoint{2.420079in}{1.107851in}}%
\pgfpathmoveto{\pgfqpoint{2.415821in}{1.112109in}}%
\pgfpathlineto{\pgfqpoint{2.415821in}{1.112109in}}%
\pgfpathlineto{\pgfqpoint{2.415821in}{1.116367in}}%
\pgfpathlineto{\pgfqpoint{2.420079in}{1.116367in}}%
\pgfpathlineto{\pgfqpoint{2.420079in}{1.112109in}}%
\pgfpathmoveto{\pgfqpoint{2.415821in}{1.116367in}}%
\pgfpathlineto{\pgfqpoint{2.415821in}{1.116367in}}%
\pgfpathlineto{\pgfqpoint{2.415821in}{1.120625in}}%
\pgfpathlineto{\pgfqpoint{2.420079in}{1.120625in}}%
\pgfpathlineto{\pgfqpoint{2.420079in}{1.116367in}}%
\pgfpathmoveto{\pgfqpoint{2.420079in}{1.103593in}}%
\pgfpathlineto{\pgfqpoint{2.420079in}{1.103593in}}%
\pgfpathlineto{\pgfqpoint{2.420079in}{1.107851in}}%
\pgfpathlineto{\pgfqpoint{2.424337in}{1.107851in}}%
\pgfpathlineto{\pgfqpoint{2.424337in}{1.103593in}}%
\pgfpathmoveto{\pgfqpoint{2.420079in}{1.107851in}}%
\pgfpathlineto{\pgfqpoint{2.420079in}{1.107851in}}%
\pgfpathlineto{\pgfqpoint{2.420079in}{1.112109in}}%
\pgfpathlineto{\pgfqpoint{2.424337in}{1.112109in}}%
\pgfpathlineto{\pgfqpoint{2.424337in}{1.107851in}}%
\pgfpathmoveto{\pgfqpoint{2.420079in}{1.112109in}}%
\pgfpathlineto{\pgfqpoint{2.420079in}{1.112109in}}%
\pgfpathlineto{\pgfqpoint{2.420079in}{1.116367in}}%
\pgfpathlineto{\pgfqpoint{2.424337in}{1.116367in}}%
\pgfpathlineto{\pgfqpoint{2.424337in}{1.112109in}}%
\pgfpathmoveto{\pgfqpoint{2.420079in}{1.116367in}}%
\pgfpathlineto{\pgfqpoint{2.420079in}{1.116367in}}%
\pgfpathlineto{\pgfqpoint{2.420079in}{1.120625in}}%
\pgfpathlineto{\pgfqpoint{2.424337in}{1.120625in}}%
\pgfpathlineto{\pgfqpoint{2.424337in}{1.116367in}}%
\pgfpathmoveto{\pgfqpoint{2.415821in}{1.120625in}}%
\pgfpathlineto{\pgfqpoint{2.415821in}{1.120625in}}%
\pgfpathlineto{\pgfqpoint{2.415821in}{1.124884in}}%
\pgfpathlineto{\pgfqpoint{2.420079in}{1.124884in}}%
\pgfpathlineto{\pgfqpoint{2.420079in}{1.120625in}}%
\pgfpathmoveto{\pgfqpoint{2.415821in}{1.124884in}}%
\pgfpathlineto{\pgfqpoint{2.415821in}{1.124884in}}%
\pgfpathlineto{\pgfqpoint{2.415821in}{1.129142in}}%
\pgfpathlineto{\pgfqpoint{2.420079in}{1.129142in}}%
\pgfpathlineto{\pgfqpoint{2.420079in}{1.124884in}}%
\pgfpathmoveto{\pgfqpoint{2.415821in}{1.129142in}}%
\pgfpathlineto{\pgfqpoint{2.415821in}{1.129142in}}%
\pgfpathlineto{\pgfqpoint{2.415821in}{1.133400in}}%
\pgfpathlineto{\pgfqpoint{2.420079in}{1.133400in}}%
\pgfpathlineto{\pgfqpoint{2.420079in}{1.129142in}}%
\pgfpathmoveto{\pgfqpoint{2.420079in}{1.120625in}}%
\pgfpathlineto{\pgfqpoint{2.420079in}{1.120625in}}%
\pgfpathlineto{\pgfqpoint{2.420079in}{1.124884in}}%
\pgfpathlineto{\pgfqpoint{2.424337in}{1.124884in}}%
\pgfpathlineto{\pgfqpoint{2.424337in}{1.120625in}}%
\pgfpathmoveto{\pgfqpoint{2.420079in}{1.124884in}}%
\pgfpathlineto{\pgfqpoint{2.420079in}{1.124884in}}%
\pgfpathlineto{\pgfqpoint{2.420079in}{1.129142in}}%
\pgfpathlineto{\pgfqpoint{2.424337in}{1.129142in}}%
\pgfpathlineto{\pgfqpoint{2.424337in}{1.124884in}}%
\pgfpathmoveto{\pgfqpoint{2.420079in}{1.129142in}}%
\pgfpathlineto{\pgfqpoint{2.420079in}{1.129142in}}%
\pgfpathlineto{\pgfqpoint{2.420079in}{1.133400in}}%
\pgfpathlineto{\pgfqpoint{2.424337in}{1.133400in}}%
\pgfpathlineto{\pgfqpoint{2.424337in}{1.129142in}}%
\pgfpathmoveto{\pgfqpoint{2.420079in}{1.133400in}}%
\pgfpathlineto{\pgfqpoint{2.420079in}{1.133400in}}%
\pgfpathlineto{\pgfqpoint{2.420079in}{1.137658in}}%
\pgfpathlineto{\pgfqpoint{2.424337in}{1.137658in}}%
\pgfpathlineto{\pgfqpoint{2.424337in}{1.133400in}}%
\pgfpathmoveto{\pgfqpoint{2.424337in}{1.129142in}}%
\pgfpathlineto{\pgfqpoint{2.424337in}{1.129142in}}%
\pgfpathlineto{\pgfqpoint{2.424337in}{1.133400in}}%
\pgfpathlineto{\pgfqpoint{2.428594in}{1.133400in}}%
\pgfpathlineto{\pgfqpoint{2.428594in}{1.129142in}}%
\pgfpathmoveto{\pgfqpoint{2.424337in}{1.133400in}}%
\pgfpathlineto{\pgfqpoint{2.424337in}{1.133400in}}%
\pgfpathlineto{\pgfqpoint{2.424337in}{1.137658in}}%
\pgfpathlineto{\pgfqpoint{2.428594in}{1.137658in}}%
\pgfpathlineto{\pgfqpoint{2.428594in}{1.133400in}}%
\pgfpathmoveto{\pgfqpoint{2.420079in}{1.137658in}}%
\pgfpathlineto{\pgfqpoint{2.420079in}{1.137658in}}%
\pgfpathlineto{\pgfqpoint{2.420079in}{1.141916in}}%
\pgfpathlineto{\pgfqpoint{2.424337in}{1.141916in}}%
\pgfpathlineto{\pgfqpoint{2.424337in}{1.137658in}}%
\pgfpathmoveto{\pgfqpoint{2.420079in}{1.141916in}}%
\pgfpathlineto{\pgfqpoint{2.420079in}{1.141916in}}%
\pgfpathlineto{\pgfqpoint{2.420079in}{1.146174in}}%
\pgfpathlineto{\pgfqpoint{2.424337in}{1.146174in}}%
\pgfpathlineto{\pgfqpoint{2.424337in}{1.141916in}}%
\pgfpathmoveto{\pgfqpoint{2.424337in}{1.137658in}}%
\pgfpathlineto{\pgfqpoint{2.424337in}{1.137658in}}%
\pgfpathlineto{\pgfqpoint{2.424337in}{1.141916in}}%
\pgfpathlineto{\pgfqpoint{2.428594in}{1.141916in}}%
\pgfpathlineto{\pgfqpoint{2.428594in}{1.137658in}}%
\pgfpathmoveto{\pgfqpoint{2.424337in}{1.141916in}}%
\pgfpathlineto{\pgfqpoint{2.424337in}{1.141916in}}%
\pgfpathlineto{\pgfqpoint{2.424337in}{1.146174in}}%
\pgfpathlineto{\pgfqpoint{2.428594in}{1.146174in}}%
\pgfpathlineto{\pgfqpoint{2.428594in}{1.141916in}}%
\pgfpathmoveto{\pgfqpoint{2.420079in}{1.146174in}}%
\pgfpathlineto{\pgfqpoint{2.420079in}{1.146174in}}%
\pgfpathlineto{\pgfqpoint{2.420079in}{1.150432in}}%
\pgfpathlineto{\pgfqpoint{2.424337in}{1.150432in}}%
\pgfpathlineto{\pgfqpoint{2.424337in}{1.146174in}}%
\pgfpathmoveto{\pgfqpoint{2.420079in}{1.150432in}}%
\pgfpathlineto{\pgfqpoint{2.420079in}{1.150432in}}%
\pgfpathlineto{\pgfqpoint{2.420079in}{1.154690in}}%
\pgfpathlineto{\pgfqpoint{2.424337in}{1.154690in}}%
\pgfpathlineto{\pgfqpoint{2.424337in}{1.150432in}}%
\pgfpathmoveto{\pgfqpoint{2.424337in}{1.146174in}}%
\pgfpathlineto{\pgfqpoint{2.424337in}{1.146174in}}%
\pgfpathlineto{\pgfqpoint{2.424337in}{1.150432in}}%
\pgfpathlineto{\pgfqpoint{2.428594in}{1.150432in}}%
\pgfpathlineto{\pgfqpoint{2.428594in}{1.146174in}}%
\pgfpathmoveto{\pgfqpoint{2.424337in}{1.150432in}}%
\pgfpathlineto{\pgfqpoint{2.424337in}{1.150432in}}%
\pgfpathlineto{\pgfqpoint{2.424337in}{1.154690in}}%
\pgfpathlineto{\pgfqpoint{2.428594in}{1.154690in}}%
\pgfpathlineto{\pgfqpoint{2.428594in}{1.150432in}}%
\pgfpathmoveto{\pgfqpoint{2.420079in}{1.154690in}}%
\pgfpathlineto{\pgfqpoint{2.420079in}{1.154690in}}%
\pgfpathlineto{\pgfqpoint{2.420079in}{1.158948in}}%
\pgfpathlineto{\pgfqpoint{2.424337in}{1.158948in}}%
\pgfpathlineto{\pgfqpoint{2.424337in}{1.154690in}}%
\pgfpathmoveto{\pgfqpoint{2.420079in}{1.158948in}}%
\pgfpathlineto{\pgfqpoint{2.420079in}{1.158948in}}%
\pgfpathlineto{\pgfqpoint{2.420079in}{1.163206in}}%
\pgfpathlineto{\pgfqpoint{2.424337in}{1.163206in}}%
\pgfpathlineto{\pgfqpoint{2.424337in}{1.158948in}}%
\pgfpathmoveto{\pgfqpoint{2.424337in}{1.154690in}}%
\pgfpathlineto{\pgfqpoint{2.424337in}{1.154690in}}%
\pgfpathlineto{\pgfqpoint{2.424337in}{1.158948in}}%
\pgfpathlineto{\pgfqpoint{2.428594in}{1.158948in}}%
\pgfpathlineto{\pgfqpoint{2.428594in}{1.154690in}}%
\pgfpathmoveto{\pgfqpoint{2.424337in}{1.158948in}}%
\pgfpathlineto{\pgfqpoint{2.424337in}{1.158948in}}%
\pgfpathlineto{\pgfqpoint{2.424337in}{1.163206in}}%
\pgfpathlineto{\pgfqpoint{2.428594in}{1.163206in}}%
\pgfpathlineto{\pgfqpoint{2.428594in}{1.158948in}}%
\pgfpathmoveto{\pgfqpoint{2.420079in}{1.163206in}}%
\pgfpathlineto{\pgfqpoint{2.420079in}{1.163206in}}%
\pgfpathlineto{\pgfqpoint{2.420079in}{1.167464in}}%
\pgfpathlineto{\pgfqpoint{2.424337in}{1.167464in}}%
\pgfpathlineto{\pgfqpoint{2.424337in}{1.163206in}}%
\pgfpathmoveto{\pgfqpoint{2.420079in}{1.167464in}}%
\pgfpathlineto{\pgfqpoint{2.420079in}{1.167464in}}%
\pgfpathlineto{\pgfqpoint{2.420079in}{1.171722in}}%
\pgfpathlineto{\pgfqpoint{2.424337in}{1.171722in}}%
\pgfpathlineto{\pgfqpoint{2.424337in}{1.167464in}}%
\pgfpathmoveto{\pgfqpoint{2.424337in}{1.163206in}}%
\pgfpathlineto{\pgfqpoint{2.424337in}{1.163206in}}%
\pgfpathlineto{\pgfqpoint{2.424337in}{1.167464in}}%
\pgfpathlineto{\pgfqpoint{2.428594in}{1.167464in}}%
\pgfpathlineto{\pgfqpoint{2.428594in}{1.163206in}}%
\pgfpathmoveto{\pgfqpoint{2.424337in}{1.167464in}}%
\pgfpathlineto{\pgfqpoint{2.424337in}{1.167464in}}%
\pgfpathlineto{\pgfqpoint{2.424337in}{1.171722in}}%
\pgfpathlineto{\pgfqpoint{2.428594in}{1.171722in}}%
\pgfpathlineto{\pgfqpoint{2.428594in}{1.167464in}}%
\pgfpathmoveto{\pgfqpoint{2.420079in}{1.171722in}}%
\pgfpathlineto{\pgfqpoint{2.420079in}{1.171722in}}%
\pgfpathlineto{\pgfqpoint{2.420079in}{1.175980in}}%
\pgfpathlineto{\pgfqpoint{2.424337in}{1.175980in}}%
\pgfpathlineto{\pgfqpoint{2.424337in}{1.171722in}}%
\pgfpathmoveto{\pgfqpoint{2.424337in}{1.171722in}}%
\pgfpathlineto{\pgfqpoint{2.424337in}{1.171722in}}%
\pgfpathlineto{\pgfqpoint{2.424337in}{1.175980in}}%
\pgfpathlineto{\pgfqpoint{2.428594in}{1.175980in}}%
\pgfpathlineto{\pgfqpoint{2.428594in}{1.171722in}}%
\pgfpathmoveto{\pgfqpoint{2.424337in}{1.175980in}}%
\pgfpathlineto{\pgfqpoint{2.424337in}{1.175980in}}%
\pgfpathlineto{\pgfqpoint{2.424337in}{1.180238in}}%
\pgfpathlineto{\pgfqpoint{2.428594in}{1.180238in}}%
\pgfpathlineto{\pgfqpoint{2.428594in}{1.175980in}}%
\pgfpathmoveto{\pgfqpoint{2.424337in}{1.180238in}}%
\pgfpathlineto{\pgfqpoint{2.424337in}{1.180238in}}%
\pgfpathlineto{\pgfqpoint{2.424337in}{1.184496in}}%
\pgfpathlineto{\pgfqpoint{2.428594in}{1.184496in}}%
\pgfpathlineto{\pgfqpoint{2.428594in}{1.180238in}}%
\pgfpathmoveto{\pgfqpoint{2.424337in}{1.184496in}}%
\pgfpathlineto{\pgfqpoint{2.424337in}{1.184496in}}%
\pgfpathlineto{\pgfqpoint{2.424337in}{1.188754in}}%
\pgfpathlineto{\pgfqpoint{2.428594in}{1.188754in}}%
\pgfpathlineto{\pgfqpoint{2.428594in}{1.184496in}}%
\pgfpathmoveto{\pgfqpoint{2.428594in}{1.167464in}}%
\pgfpathlineto{\pgfqpoint{2.428594in}{1.167464in}}%
\pgfpathlineto{\pgfqpoint{2.428594in}{1.171722in}}%
\pgfpathlineto{\pgfqpoint{2.432852in}{1.171722in}}%
\pgfpathlineto{\pgfqpoint{2.432852in}{1.167464in}}%
\pgfpathmoveto{\pgfqpoint{2.428594in}{1.171722in}}%
\pgfpathlineto{\pgfqpoint{2.428594in}{1.171722in}}%
\pgfpathlineto{\pgfqpoint{2.428594in}{1.175980in}}%
\pgfpathlineto{\pgfqpoint{2.432852in}{1.175980in}}%
\pgfpathlineto{\pgfqpoint{2.432852in}{1.171722in}}%
\pgfpathmoveto{\pgfqpoint{2.428594in}{1.175980in}}%
\pgfpathlineto{\pgfqpoint{2.428594in}{1.175980in}}%
\pgfpathlineto{\pgfqpoint{2.428594in}{1.180238in}}%
\pgfpathlineto{\pgfqpoint{2.432852in}{1.180238in}}%
\pgfpathlineto{\pgfqpoint{2.432852in}{1.175980in}}%
\pgfpathmoveto{\pgfqpoint{2.428594in}{1.180238in}}%
\pgfpathlineto{\pgfqpoint{2.428594in}{1.180238in}}%
\pgfpathlineto{\pgfqpoint{2.428594in}{1.184496in}}%
\pgfpathlineto{\pgfqpoint{2.432852in}{1.184496in}}%
\pgfpathlineto{\pgfqpoint{2.432852in}{1.180238in}}%
\pgfpathmoveto{\pgfqpoint{2.428594in}{1.184496in}}%
\pgfpathlineto{\pgfqpoint{2.428594in}{1.184496in}}%
\pgfpathlineto{\pgfqpoint{2.428594in}{1.188754in}}%
\pgfpathlineto{\pgfqpoint{2.432852in}{1.188754in}}%
\pgfpathlineto{\pgfqpoint{2.432852in}{1.184496in}}%
\pgfpathmoveto{\pgfqpoint{2.424337in}{1.188754in}}%
\pgfpathlineto{\pgfqpoint{2.424337in}{1.188754in}}%
\pgfpathlineto{\pgfqpoint{2.424337in}{1.193012in}}%
\pgfpathlineto{\pgfqpoint{2.428594in}{1.193012in}}%
\pgfpathlineto{\pgfqpoint{2.428594in}{1.188754in}}%
\pgfpathmoveto{\pgfqpoint{2.424337in}{1.193012in}}%
\pgfpathlineto{\pgfqpoint{2.424337in}{1.193012in}}%
\pgfpathlineto{\pgfqpoint{2.424337in}{1.197269in}}%
\pgfpathlineto{\pgfqpoint{2.428594in}{1.197269in}}%
\pgfpathlineto{\pgfqpoint{2.428594in}{1.193012in}}%
\pgfpathmoveto{\pgfqpoint{2.424337in}{1.197269in}}%
\pgfpathlineto{\pgfqpoint{2.424337in}{1.197269in}}%
\pgfpathlineto{\pgfqpoint{2.424337in}{1.201527in}}%
\pgfpathlineto{\pgfqpoint{2.428594in}{1.201527in}}%
\pgfpathlineto{\pgfqpoint{2.428594in}{1.197269in}}%
\pgfpathmoveto{\pgfqpoint{2.424337in}{1.201527in}}%
\pgfpathlineto{\pgfqpoint{2.424337in}{1.201527in}}%
\pgfpathlineto{\pgfqpoint{2.424337in}{1.205785in}}%
\pgfpathlineto{\pgfqpoint{2.428594in}{1.205785in}}%
\pgfpathlineto{\pgfqpoint{2.428594in}{1.201527in}}%
\pgfpathmoveto{\pgfqpoint{2.424337in}{1.205785in}}%
\pgfpathlineto{\pgfqpoint{2.424337in}{1.205785in}}%
\pgfpathlineto{\pgfqpoint{2.424337in}{1.210042in}}%
\pgfpathlineto{\pgfqpoint{2.428594in}{1.210042in}}%
\pgfpathlineto{\pgfqpoint{2.428594in}{1.205785in}}%
\pgfpathmoveto{\pgfqpoint{2.424337in}{1.210042in}}%
\pgfpathlineto{\pgfqpoint{2.424337in}{1.210042in}}%
\pgfpathlineto{\pgfqpoint{2.424337in}{1.214300in}}%
\pgfpathlineto{\pgfqpoint{2.428594in}{1.214300in}}%
\pgfpathlineto{\pgfqpoint{2.428594in}{1.210042in}}%
\pgfpathmoveto{\pgfqpoint{2.428594in}{1.188754in}}%
\pgfpathlineto{\pgfqpoint{2.428594in}{1.188754in}}%
\pgfpathlineto{\pgfqpoint{2.428594in}{1.193012in}}%
\pgfpathlineto{\pgfqpoint{2.432852in}{1.193012in}}%
\pgfpathlineto{\pgfqpoint{2.432852in}{1.188754in}}%
\pgfpathmoveto{\pgfqpoint{2.428594in}{1.193012in}}%
\pgfpathlineto{\pgfqpoint{2.428594in}{1.193012in}}%
\pgfpathlineto{\pgfqpoint{2.428594in}{1.197269in}}%
\pgfpathlineto{\pgfqpoint{2.432852in}{1.197269in}}%
\pgfpathlineto{\pgfqpoint{2.432852in}{1.193012in}}%
\pgfpathmoveto{\pgfqpoint{2.428594in}{1.197269in}}%
\pgfpathlineto{\pgfqpoint{2.428594in}{1.197269in}}%
\pgfpathlineto{\pgfqpoint{2.428594in}{1.201527in}}%
\pgfpathlineto{\pgfqpoint{2.432852in}{1.201527in}}%
\pgfpathlineto{\pgfqpoint{2.432852in}{1.197269in}}%
\pgfpathmoveto{\pgfqpoint{2.428594in}{1.201527in}}%
\pgfpathlineto{\pgfqpoint{2.428594in}{1.201527in}}%
\pgfpathlineto{\pgfqpoint{2.428594in}{1.205785in}}%
\pgfpathlineto{\pgfqpoint{2.432852in}{1.205785in}}%
\pgfpathlineto{\pgfqpoint{2.432852in}{1.201527in}}%
\pgfpathmoveto{\pgfqpoint{2.428594in}{1.205785in}}%
\pgfpathlineto{\pgfqpoint{2.428594in}{1.205785in}}%
\pgfpathlineto{\pgfqpoint{2.428594in}{1.210042in}}%
\pgfpathlineto{\pgfqpoint{2.432852in}{1.210042in}}%
\pgfpathlineto{\pgfqpoint{2.432852in}{1.205785in}}%
\pgfpathmoveto{\pgfqpoint{2.428594in}{1.210042in}}%
\pgfpathlineto{\pgfqpoint{2.428594in}{1.210042in}}%
\pgfpathlineto{\pgfqpoint{2.428594in}{1.214300in}}%
\pgfpathlineto{\pgfqpoint{2.432852in}{1.214300in}}%
\pgfpathlineto{\pgfqpoint{2.432852in}{1.210042in}}%
\pgfpathmoveto{\pgfqpoint{2.432852in}{1.210042in}}%
\pgfpathlineto{\pgfqpoint{2.432852in}{1.210042in}}%
\pgfpathlineto{\pgfqpoint{2.432852in}{1.214300in}}%
\pgfpathlineto{\pgfqpoint{2.437110in}{1.214300in}}%
\pgfpathlineto{\pgfqpoint{2.437110in}{1.210042in}}%
\pgfpathmoveto{\pgfqpoint{2.428594in}{1.214300in}}%
\pgfpathlineto{\pgfqpoint{2.428594in}{1.214300in}}%
\pgfpathlineto{\pgfqpoint{2.428594in}{1.218558in}}%
\pgfpathlineto{\pgfqpoint{2.432852in}{1.218558in}}%
\pgfpathlineto{\pgfqpoint{2.432852in}{1.214300in}}%
\pgfpathmoveto{\pgfqpoint{2.428594in}{1.218558in}}%
\pgfpathlineto{\pgfqpoint{2.428594in}{1.218558in}}%
\pgfpathlineto{\pgfqpoint{2.428594in}{1.222815in}}%
\pgfpathlineto{\pgfqpoint{2.432852in}{1.222815in}}%
\pgfpathlineto{\pgfqpoint{2.432852in}{1.218558in}}%
\pgfpathmoveto{\pgfqpoint{2.432852in}{1.214300in}}%
\pgfpathlineto{\pgfqpoint{2.432852in}{1.214300in}}%
\pgfpathlineto{\pgfqpoint{2.432852in}{1.218558in}}%
\pgfpathlineto{\pgfqpoint{2.437110in}{1.218558in}}%
\pgfpathlineto{\pgfqpoint{2.437110in}{1.214300in}}%
\pgfpathmoveto{\pgfqpoint{2.432852in}{1.218558in}}%
\pgfpathlineto{\pgfqpoint{2.432852in}{1.218558in}}%
\pgfpathlineto{\pgfqpoint{2.432852in}{1.222815in}}%
\pgfpathlineto{\pgfqpoint{2.437110in}{1.222815in}}%
\pgfpathlineto{\pgfqpoint{2.437110in}{1.218558in}}%
\pgfpathmoveto{\pgfqpoint{2.428594in}{1.222815in}}%
\pgfpathlineto{\pgfqpoint{2.428594in}{1.222815in}}%
\pgfpathlineto{\pgfqpoint{2.428594in}{1.227073in}}%
\pgfpathlineto{\pgfqpoint{2.432852in}{1.227073in}}%
\pgfpathlineto{\pgfqpoint{2.432852in}{1.222815in}}%
\pgfpathmoveto{\pgfqpoint{2.428594in}{1.227073in}}%
\pgfpathlineto{\pgfqpoint{2.428594in}{1.227073in}}%
\pgfpathlineto{\pgfqpoint{2.428594in}{1.231330in}}%
\pgfpathlineto{\pgfqpoint{2.432852in}{1.231330in}}%
\pgfpathlineto{\pgfqpoint{2.432852in}{1.227073in}}%
\pgfpathmoveto{\pgfqpoint{2.432852in}{1.222815in}}%
\pgfpathlineto{\pgfqpoint{2.432852in}{1.222815in}}%
\pgfpathlineto{\pgfqpoint{2.432852in}{1.227073in}}%
\pgfpathlineto{\pgfqpoint{2.437110in}{1.227073in}}%
\pgfpathlineto{\pgfqpoint{2.437110in}{1.222815in}}%
\pgfpathmoveto{\pgfqpoint{2.432852in}{1.227073in}}%
\pgfpathlineto{\pgfqpoint{2.432852in}{1.227073in}}%
\pgfpathlineto{\pgfqpoint{2.432852in}{1.231330in}}%
\pgfpathlineto{\pgfqpoint{2.437110in}{1.231330in}}%
\pgfpathlineto{\pgfqpoint{2.437110in}{1.227073in}}%
\pgfpathmoveto{\pgfqpoint{2.428594in}{1.231330in}}%
\pgfpathlineto{\pgfqpoint{2.428594in}{1.231330in}}%
\pgfpathlineto{\pgfqpoint{2.428594in}{1.235588in}}%
\pgfpathlineto{\pgfqpoint{2.432852in}{1.235588in}}%
\pgfpathlineto{\pgfqpoint{2.432852in}{1.231330in}}%
\pgfpathmoveto{\pgfqpoint{2.428594in}{1.235588in}}%
\pgfpathlineto{\pgfqpoint{2.428594in}{1.235588in}}%
\pgfpathlineto{\pgfqpoint{2.428594in}{1.239846in}}%
\pgfpathlineto{\pgfqpoint{2.432852in}{1.239846in}}%
\pgfpathlineto{\pgfqpoint{2.432852in}{1.235588in}}%
\pgfpathmoveto{\pgfqpoint{2.432852in}{1.231330in}}%
\pgfpathlineto{\pgfqpoint{2.432852in}{1.231330in}}%
\pgfpathlineto{\pgfqpoint{2.432852in}{1.235588in}}%
\pgfpathlineto{\pgfqpoint{2.437110in}{1.235588in}}%
\pgfpathlineto{\pgfqpoint{2.437110in}{1.231330in}}%
\pgfpathmoveto{\pgfqpoint{2.432852in}{1.235588in}}%
\pgfpathlineto{\pgfqpoint{2.432852in}{1.235588in}}%
\pgfpathlineto{\pgfqpoint{2.432852in}{1.239846in}}%
\pgfpathlineto{\pgfqpoint{2.437110in}{1.239846in}}%
\pgfpathlineto{\pgfqpoint{2.437110in}{1.235588in}}%
\pgfpathmoveto{\pgfqpoint{2.428594in}{1.239846in}}%
\pgfpathlineto{\pgfqpoint{2.428594in}{1.239846in}}%
\pgfpathlineto{\pgfqpoint{2.428594in}{1.244103in}}%
\pgfpathlineto{\pgfqpoint{2.432852in}{1.244103in}}%
\pgfpathlineto{\pgfqpoint{2.432852in}{1.239846in}}%
\pgfpathmoveto{\pgfqpoint{2.428594in}{1.244103in}}%
\pgfpathlineto{\pgfqpoint{2.428594in}{1.244103in}}%
\pgfpathlineto{\pgfqpoint{2.428594in}{1.248361in}}%
\pgfpathlineto{\pgfqpoint{2.432852in}{1.248361in}}%
\pgfpathlineto{\pgfqpoint{2.432852in}{1.244103in}}%
\pgfpathmoveto{\pgfqpoint{2.432852in}{1.239846in}}%
\pgfpathlineto{\pgfqpoint{2.432852in}{1.239846in}}%
\pgfpathlineto{\pgfqpoint{2.432852in}{1.244103in}}%
\pgfpathlineto{\pgfqpoint{2.437110in}{1.244103in}}%
\pgfpathlineto{\pgfqpoint{2.437110in}{1.239846in}}%
\pgfpathmoveto{\pgfqpoint{2.432852in}{1.244103in}}%
\pgfpathlineto{\pgfqpoint{2.432852in}{1.244103in}}%
\pgfpathlineto{\pgfqpoint{2.432852in}{1.248361in}}%
\pgfpathlineto{\pgfqpoint{2.437110in}{1.248361in}}%
\pgfpathlineto{\pgfqpoint{2.437110in}{1.244103in}}%
\pgfpathmoveto{\pgfqpoint{2.428594in}{1.248361in}}%
\pgfpathlineto{\pgfqpoint{2.428594in}{1.248361in}}%
\pgfpathlineto{\pgfqpoint{2.428594in}{1.252619in}}%
\pgfpathlineto{\pgfqpoint{2.432852in}{1.252619in}}%
\pgfpathlineto{\pgfqpoint{2.432852in}{1.248361in}}%
\pgfpathmoveto{\pgfqpoint{2.428594in}{1.252619in}}%
\pgfpathlineto{\pgfqpoint{2.428594in}{1.252619in}}%
\pgfpathlineto{\pgfqpoint{2.428594in}{1.256876in}}%
\pgfpathlineto{\pgfqpoint{2.432852in}{1.256876in}}%
\pgfpathlineto{\pgfqpoint{2.432852in}{1.252619in}}%
\pgfpathmoveto{\pgfqpoint{2.432852in}{1.248361in}}%
\pgfpathlineto{\pgfqpoint{2.432852in}{1.248361in}}%
\pgfpathlineto{\pgfqpoint{2.432852in}{1.252619in}}%
\pgfpathlineto{\pgfqpoint{2.437110in}{1.252619in}}%
\pgfpathlineto{\pgfqpoint{2.437110in}{1.248361in}}%
\pgfpathmoveto{\pgfqpoint{2.432852in}{1.252619in}}%
\pgfpathlineto{\pgfqpoint{2.432852in}{1.252619in}}%
\pgfpathlineto{\pgfqpoint{2.432852in}{1.256876in}}%
\pgfpathlineto{\pgfqpoint{2.437110in}{1.256876in}}%
\pgfpathlineto{\pgfqpoint{2.437110in}{1.252619in}}%
\pgfpathmoveto{\pgfqpoint{2.437110in}{1.252619in}}%
\pgfpathlineto{\pgfqpoint{2.437110in}{1.252619in}}%
\pgfpathlineto{\pgfqpoint{2.437110in}{1.256876in}}%
\pgfpathlineto{\pgfqpoint{2.441368in}{1.256876in}}%
\pgfpathlineto{\pgfqpoint{2.441368in}{1.252619in}}%
\pgfpathmoveto{\pgfqpoint{2.432852in}{1.256876in}}%
\pgfpathlineto{\pgfqpoint{2.432852in}{1.256876in}}%
\pgfpathlineto{\pgfqpoint{2.432852in}{1.261134in}}%
\pgfpathlineto{\pgfqpoint{2.437110in}{1.261134in}}%
\pgfpathlineto{\pgfqpoint{2.437110in}{1.256876in}}%
\pgfpathmoveto{\pgfqpoint{2.432852in}{1.261134in}}%
\pgfpathlineto{\pgfqpoint{2.432852in}{1.261134in}}%
\pgfpathlineto{\pgfqpoint{2.432852in}{1.265392in}}%
\pgfpathlineto{\pgfqpoint{2.437110in}{1.265392in}}%
\pgfpathlineto{\pgfqpoint{2.437110in}{1.261134in}}%
\pgfpathmoveto{\pgfqpoint{2.432852in}{1.265392in}}%
\pgfpathlineto{\pgfqpoint{2.432852in}{1.265392in}}%
\pgfpathlineto{\pgfqpoint{2.432852in}{1.269649in}}%
\pgfpathlineto{\pgfqpoint{2.437110in}{1.269649in}}%
\pgfpathlineto{\pgfqpoint{2.437110in}{1.265392in}}%
\pgfpathmoveto{\pgfqpoint{2.432852in}{1.269649in}}%
\pgfpathlineto{\pgfqpoint{2.432852in}{1.269649in}}%
\pgfpathlineto{\pgfqpoint{2.432852in}{1.273907in}}%
\pgfpathlineto{\pgfqpoint{2.437110in}{1.273907in}}%
\pgfpathlineto{\pgfqpoint{2.437110in}{1.269649in}}%
\pgfpathmoveto{\pgfqpoint{2.437110in}{1.256876in}}%
\pgfpathlineto{\pgfqpoint{2.437110in}{1.256876in}}%
\pgfpathlineto{\pgfqpoint{2.437110in}{1.261134in}}%
\pgfpathlineto{\pgfqpoint{2.441368in}{1.261134in}}%
\pgfpathlineto{\pgfqpoint{2.441368in}{1.256876in}}%
\pgfpathmoveto{\pgfqpoint{2.437110in}{1.261134in}}%
\pgfpathlineto{\pgfqpoint{2.437110in}{1.261134in}}%
\pgfpathlineto{\pgfqpoint{2.437110in}{1.265392in}}%
\pgfpathlineto{\pgfqpoint{2.441368in}{1.265392in}}%
\pgfpathlineto{\pgfqpoint{2.441368in}{1.261134in}}%
\pgfpathmoveto{\pgfqpoint{2.437110in}{1.265392in}}%
\pgfpathlineto{\pgfqpoint{2.437110in}{1.265392in}}%
\pgfpathlineto{\pgfqpoint{2.437110in}{1.269649in}}%
\pgfpathlineto{\pgfqpoint{2.441368in}{1.269649in}}%
\pgfpathlineto{\pgfqpoint{2.441368in}{1.265392in}}%
\pgfpathmoveto{\pgfqpoint{2.437110in}{1.269649in}}%
\pgfpathlineto{\pgfqpoint{2.437110in}{1.269649in}}%
\pgfpathlineto{\pgfqpoint{2.437110in}{1.273907in}}%
\pgfpathlineto{\pgfqpoint{2.441368in}{1.273907in}}%
\pgfpathlineto{\pgfqpoint{2.441368in}{1.269649in}}%
\pgfpathmoveto{\pgfqpoint{2.432852in}{1.273907in}}%
\pgfpathlineto{\pgfqpoint{2.432852in}{1.273907in}}%
\pgfpathlineto{\pgfqpoint{2.432852in}{1.278165in}}%
\pgfpathlineto{\pgfqpoint{2.437110in}{1.278165in}}%
\pgfpathlineto{\pgfqpoint{2.437110in}{1.273907in}}%
\pgfpathmoveto{\pgfqpoint{2.432852in}{1.278165in}}%
\pgfpathlineto{\pgfqpoint{2.432852in}{1.278165in}}%
\pgfpathlineto{\pgfqpoint{2.432852in}{1.282422in}}%
\pgfpathlineto{\pgfqpoint{2.437110in}{1.282422in}}%
\pgfpathlineto{\pgfqpoint{2.437110in}{1.278165in}}%
\pgfpathmoveto{\pgfqpoint{2.432852in}{1.282422in}}%
\pgfpathlineto{\pgfqpoint{2.432852in}{1.282422in}}%
\pgfpathlineto{\pgfqpoint{2.432852in}{1.286680in}}%
\pgfpathlineto{\pgfqpoint{2.437110in}{1.286680in}}%
\pgfpathlineto{\pgfqpoint{2.437110in}{1.282422in}}%
\pgfpathmoveto{\pgfqpoint{2.432852in}{1.286680in}}%
\pgfpathlineto{\pgfqpoint{2.432852in}{1.286680in}}%
\pgfpathlineto{\pgfqpoint{2.432852in}{1.290938in}}%
\pgfpathlineto{\pgfqpoint{2.437110in}{1.290938in}}%
\pgfpathlineto{\pgfqpoint{2.437110in}{1.286680in}}%
\pgfpathmoveto{\pgfqpoint{2.437110in}{1.273907in}}%
\pgfpathlineto{\pgfqpoint{2.437110in}{1.273907in}}%
\pgfpathlineto{\pgfqpoint{2.437110in}{1.278165in}}%
\pgfpathlineto{\pgfqpoint{2.441368in}{1.278165in}}%
\pgfpathlineto{\pgfqpoint{2.441368in}{1.273907in}}%
\pgfpathmoveto{\pgfqpoint{2.437110in}{1.278165in}}%
\pgfpathlineto{\pgfqpoint{2.437110in}{1.278165in}}%
\pgfpathlineto{\pgfqpoint{2.437110in}{1.282422in}}%
\pgfpathlineto{\pgfqpoint{2.441368in}{1.282422in}}%
\pgfpathlineto{\pgfqpoint{2.441368in}{1.278165in}}%
\pgfpathmoveto{\pgfqpoint{2.437110in}{1.282422in}}%
\pgfpathlineto{\pgfqpoint{2.437110in}{1.282422in}}%
\pgfpathlineto{\pgfqpoint{2.437110in}{1.286680in}}%
\pgfpathlineto{\pgfqpoint{2.441368in}{1.286680in}}%
\pgfpathlineto{\pgfqpoint{2.441368in}{1.282422in}}%
\pgfpathmoveto{\pgfqpoint{2.437110in}{1.286680in}}%
\pgfpathlineto{\pgfqpoint{2.437110in}{1.286680in}}%
\pgfpathlineto{\pgfqpoint{2.437110in}{1.290938in}}%
\pgfpathlineto{\pgfqpoint{2.441368in}{1.290938in}}%
\pgfpathlineto{\pgfqpoint{2.441368in}{1.286680in}}%
\pgfpathmoveto{\pgfqpoint{2.432852in}{1.290938in}}%
\pgfpathlineto{\pgfqpoint{2.432852in}{1.290938in}}%
\pgfpathlineto{\pgfqpoint{2.432852in}{1.295195in}}%
\pgfpathlineto{\pgfqpoint{2.437110in}{1.295195in}}%
\pgfpathlineto{\pgfqpoint{2.437110in}{1.290938in}}%
\pgfpathmoveto{\pgfqpoint{2.432852in}{1.295195in}}%
\pgfpathlineto{\pgfqpoint{2.432852in}{1.295195in}}%
\pgfpathlineto{\pgfqpoint{2.432852in}{1.299453in}}%
\pgfpathlineto{\pgfqpoint{2.437110in}{1.299453in}}%
\pgfpathlineto{\pgfqpoint{2.437110in}{1.295195in}}%
\pgfpathmoveto{\pgfqpoint{2.437110in}{1.290938in}}%
\pgfpathlineto{\pgfqpoint{2.437110in}{1.290938in}}%
\pgfpathlineto{\pgfqpoint{2.437110in}{1.295195in}}%
\pgfpathlineto{\pgfqpoint{2.441368in}{1.295195in}}%
\pgfpathlineto{\pgfqpoint{2.441368in}{1.290938in}}%
\pgfpathmoveto{\pgfqpoint{2.437110in}{1.295195in}}%
\pgfpathlineto{\pgfqpoint{2.437110in}{1.295195in}}%
\pgfpathlineto{\pgfqpoint{2.437110in}{1.299453in}}%
\pgfpathlineto{\pgfqpoint{2.441368in}{1.299453in}}%
\pgfpathlineto{\pgfqpoint{2.441368in}{1.295195in}}%
\pgfpathmoveto{\pgfqpoint{2.441368in}{1.290938in}}%
\pgfpathlineto{\pgfqpoint{2.441368in}{1.290938in}}%
\pgfpathlineto{\pgfqpoint{2.441368in}{1.295195in}}%
\pgfpathlineto{\pgfqpoint{2.445625in}{1.295195in}}%
\pgfpathlineto{\pgfqpoint{2.445625in}{1.290938in}}%
\pgfpathmoveto{\pgfqpoint{2.441368in}{1.295195in}}%
\pgfpathlineto{\pgfqpoint{2.441368in}{1.295195in}}%
\pgfpathlineto{\pgfqpoint{2.441368in}{1.299453in}}%
\pgfpathlineto{\pgfqpoint{2.445625in}{1.299453in}}%
\pgfpathlineto{\pgfqpoint{2.445625in}{1.295195in}}%
\pgfpathmoveto{\pgfqpoint{2.437110in}{1.299453in}}%
\pgfpathlineto{\pgfqpoint{2.437110in}{1.299453in}}%
\pgfpathlineto{\pgfqpoint{2.437110in}{1.303711in}}%
\pgfpathlineto{\pgfqpoint{2.441368in}{1.303711in}}%
\pgfpathlineto{\pgfqpoint{2.441368in}{1.299453in}}%
\pgfpathmoveto{\pgfqpoint{2.437110in}{1.303711in}}%
\pgfpathlineto{\pgfqpoint{2.437110in}{1.303711in}}%
\pgfpathlineto{\pgfqpoint{2.437110in}{1.307968in}}%
\pgfpathlineto{\pgfqpoint{2.441368in}{1.307968in}}%
\pgfpathlineto{\pgfqpoint{2.441368in}{1.303711in}}%
\pgfpathmoveto{\pgfqpoint{2.441368in}{1.299453in}}%
\pgfpathlineto{\pgfqpoint{2.441368in}{1.299453in}}%
\pgfpathlineto{\pgfqpoint{2.441368in}{1.303711in}}%
\pgfpathlineto{\pgfqpoint{2.445625in}{1.303711in}}%
\pgfpathlineto{\pgfqpoint{2.445625in}{1.299453in}}%
\pgfpathmoveto{\pgfqpoint{2.441368in}{1.303711in}}%
\pgfpathlineto{\pgfqpoint{2.441368in}{1.303711in}}%
\pgfpathlineto{\pgfqpoint{2.441368in}{1.307968in}}%
\pgfpathlineto{\pgfqpoint{2.445625in}{1.307968in}}%
\pgfpathlineto{\pgfqpoint{2.445625in}{1.303711in}}%
\pgfpathmoveto{\pgfqpoint{2.437110in}{1.307968in}}%
\pgfpathlineto{\pgfqpoint{2.437110in}{1.307968in}}%
\pgfpathlineto{\pgfqpoint{2.437110in}{1.312226in}}%
\pgfpathlineto{\pgfqpoint{2.441368in}{1.312226in}}%
\pgfpathlineto{\pgfqpoint{2.441368in}{1.307968in}}%
\pgfpathmoveto{\pgfqpoint{2.437110in}{1.312226in}}%
\pgfpathlineto{\pgfqpoint{2.437110in}{1.312226in}}%
\pgfpathlineto{\pgfqpoint{2.437110in}{1.316484in}}%
\pgfpathlineto{\pgfqpoint{2.441368in}{1.316484in}}%
\pgfpathlineto{\pgfqpoint{2.441368in}{1.312226in}}%
\pgfpathmoveto{\pgfqpoint{2.441368in}{1.307968in}}%
\pgfpathlineto{\pgfqpoint{2.441368in}{1.307968in}}%
\pgfpathlineto{\pgfqpoint{2.441368in}{1.312226in}}%
\pgfpathlineto{\pgfqpoint{2.445625in}{1.312226in}}%
\pgfpathlineto{\pgfqpoint{2.445625in}{1.307968in}}%
\pgfpathmoveto{\pgfqpoint{2.441368in}{1.312226in}}%
\pgfpathlineto{\pgfqpoint{2.441368in}{1.312226in}}%
\pgfpathlineto{\pgfqpoint{2.441368in}{1.316484in}}%
\pgfpathlineto{\pgfqpoint{2.445625in}{1.316484in}}%
\pgfpathlineto{\pgfqpoint{2.445625in}{1.312226in}}%
\pgfpathmoveto{\pgfqpoint{2.437110in}{1.316484in}}%
\pgfpathlineto{\pgfqpoint{2.437110in}{1.316484in}}%
\pgfpathlineto{\pgfqpoint{2.437110in}{1.320741in}}%
\pgfpathlineto{\pgfqpoint{2.441368in}{1.320741in}}%
\pgfpathlineto{\pgfqpoint{2.441368in}{1.316484in}}%
\pgfpathmoveto{\pgfqpoint{2.437110in}{1.320741in}}%
\pgfpathlineto{\pgfqpoint{2.437110in}{1.320741in}}%
\pgfpathlineto{\pgfqpoint{2.437110in}{1.324999in}}%
\pgfpathlineto{\pgfqpoint{2.441368in}{1.324999in}}%
\pgfpathlineto{\pgfqpoint{2.441368in}{1.320741in}}%
\pgfpathmoveto{\pgfqpoint{2.441368in}{1.316484in}}%
\pgfpathlineto{\pgfqpoint{2.441368in}{1.316484in}}%
\pgfpathlineto{\pgfqpoint{2.441368in}{1.320741in}}%
\pgfpathlineto{\pgfqpoint{2.445625in}{1.320741in}}%
\pgfpathlineto{\pgfqpoint{2.445625in}{1.316484in}}%
\pgfpathmoveto{\pgfqpoint{2.441368in}{1.320741in}}%
\pgfpathlineto{\pgfqpoint{2.441368in}{1.320741in}}%
\pgfpathlineto{\pgfqpoint{2.441368in}{1.324999in}}%
\pgfpathlineto{\pgfqpoint{2.445625in}{1.324999in}}%
\pgfpathlineto{\pgfqpoint{2.445625in}{1.320741in}}%
\pgfpathmoveto{\pgfqpoint{2.437110in}{1.324999in}}%
\pgfpathlineto{\pgfqpoint{2.437110in}{1.324999in}}%
\pgfpathlineto{\pgfqpoint{2.437110in}{1.329257in}}%
\pgfpathlineto{\pgfqpoint{2.441368in}{1.329257in}}%
\pgfpathlineto{\pgfqpoint{2.441368in}{1.324999in}}%
\pgfpathmoveto{\pgfqpoint{2.437110in}{1.329257in}}%
\pgfpathlineto{\pgfqpoint{2.437110in}{1.329257in}}%
\pgfpathlineto{\pgfqpoint{2.437110in}{1.333515in}}%
\pgfpathlineto{\pgfqpoint{2.441368in}{1.333515in}}%
\pgfpathlineto{\pgfqpoint{2.441368in}{1.329257in}}%
\pgfpathmoveto{\pgfqpoint{2.441368in}{1.324999in}}%
\pgfpathlineto{\pgfqpoint{2.441368in}{1.324999in}}%
\pgfpathlineto{\pgfqpoint{2.441368in}{1.329257in}}%
\pgfpathlineto{\pgfqpoint{2.445625in}{1.329257in}}%
\pgfpathlineto{\pgfqpoint{2.445625in}{1.324999in}}%
\pgfpathmoveto{\pgfqpoint{2.441368in}{1.329257in}}%
\pgfpathlineto{\pgfqpoint{2.441368in}{1.329257in}}%
\pgfpathlineto{\pgfqpoint{2.441368in}{1.333515in}}%
\pgfpathlineto{\pgfqpoint{2.445625in}{1.333515in}}%
\pgfpathlineto{\pgfqpoint{2.445625in}{1.329257in}}%
\pgfpathmoveto{\pgfqpoint{2.437110in}{1.333515in}}%
\pgfpathlineto{\pgfqpoint{2.437110in}{1.333515in}}%
\pgfpathlineto{\pgfqpoint{2.437110in}{1.337772in}}%
\pgfpathlineto{\pgfqpoint{2.441368in}{1.337772in}}%
\pgfpathlineto{\pgfqpoint{2.441368in}{1.333515in}}%
\pgfpathmoveto{\pgfqpoint{2.441368in}{1.333515in}}%
\pgfpathlineto{\pgfqpoint{2.441368in}{1.333515in}}%
\pgfpathlineto{\pgfqpoint{2.441368in}{1.337772in}}%
\pgfpathlineto{\pgfqpoint{2.445625in}{1.337772in}}%
\pgfpathlineto{\pgfqpoint{2.445625in}{1.333515in}}%
\pgfpathmoveto{\pgfqpoint{2.441368in}{1.337772in}}%
\pgfpathlineto{\pgfqpoint{2.441368in}{1.337772in}}%
\pgfpathlineto{\pgfqpoint{2.441368in}{1.342030in}}%
\pgfpathlineto{\pgfqpoint{2.445625in}{1.342030in}}%
\pgfpathlineto{\pgfqpoint{2.445625in}{1.337772in}}%
\pgfpathmoveto{\pgfqpoint{2.441368in}{1.342030in}}%
\pgfpathlineto{\pgfqpoint{2.441368in}{1.342030in}}%
\pgfpathlineto{\pgfqpoint{2.441368in}{1.346288in}}%
\pgfpathlineto{\pgfqpoint{2.445625in}{1.346288in}}%
\pgfpathlineto{\pgfqpoint{2.445625in}{1.342030in}}%
\pgfpathmoveto{\pgfqpoint{2.441368in}{1.346288in}}%
\pgfpathlineto{\pgfqpoint{2.441368in}{1.346288in}}%
\pgfpathlineto{\pgfqpoint{2.441368in}{1.350546in}}%
\pgfpathlineto{\pgfqpoint{2.445625in}{1.350546in}}%
\pgfpathlineto{\pgfqpoint{2.445625in}{1.346288in}}%
\pgfpathmoveto{\pgfqpoint{2.441368in}{1.350546in}}%
\pgfpathlineto{\pgfqpoint{2.441368in}{1.350546in}}%
\pgfpathlineto{\pgfqpoint{2.441368in}{1.354804in}}%
\pgfpathlineto{\pgfqpoint{2.445625in}{1.354804in}}%
\pgfpathlineto{\pgfqpoint{2.445625in}{1.350546in}}%
\pgfpathmoveto{\pgfqpoint{2.441368in}{1.354804in}}%
\pgfpathlineto{\pgfqpoint{2.441368in}{1.354804in}}%
\pgfpathlineto{\pgfqpoint{2.441368in}{1.359062in}}%
\pgfpathlineto{\pgfqpoint{2.445625in}{1.359062in}}%
\pgfpathlineto{\pgfqpoint{2.445625in}{1.354804in}}%
\pgfpathmoveto{\pgfqpoint{2.441368in}{1.359062in}}%
\pgfpathlineto{\pgfqpoint{2.441368in}{1.359062in}}%
\pgfpathlineto{\pgfqpoint{2.441368in}{1.363320in}}%
\pgfpathlineto{\pgfqpoint{2.445625in}{1.363320in}}%
\pgfpathlineto{\pgfqpoint{2.445625in}{1.359062in}}%
\pgfpathmoveto{\pgfqpoint{2.441368in}{1.363320in}}%
\pgfpathlineto{\pgfqpoint{2.441368in}{1.363320in}}%
\pgfpathlineto{\pgfqpoint{2.441368in}{1.367577in}}%
\pgfpathlineto{\pgfqpoint{2.445625in}{1.367577in}}%
\pgfpathlineto{\pgfqpoint{2.445625in}{1.363320in}}%
\pgfpathmoveto{\pgfqpoint{2.441368in}{1.367577in}}%
\pgfpathlineto{\pgfqpoint{2.441368in}{1.367577in}}%
\pgfpathlineto{\pgfqpoint{2.441368in}{1.371835in}}%
\pgfpathlineto{\pgfqpoint{2.445625in}{1.371835in}}%
\pgfpathlineto{\pgfqpoint{2.445625in}{1.367577in}}%
\pgfpathmoveto{\pgfqpoint{2.441368in}{1.371835in}}%
\pgfpathlineto{\pgfqpoint{2.441368in}{1.371835in}}%
\pgfpathlineto{\pgfqpoint{2.441368in}{1.376093in}}%
\pgfpathlineto{\pgfqpoint{2.445625in}{1.376093in}}%
\pgfpathlineto{\pgfqpoint{2.445625in}{1.371835in}}%
\pgfpathmoveto{\pgfqpoint{2.441368in}{1.376093in}}%
\pgfpathlineto{\pgfqpoint{2.441368in}{1.376093in}}%
\pgfpathlineto{\pgfqpoint{2.441368in}{1.380351in}}%
\pgfpathlineto{\pgfqpoint{2.445625in}{1.380351in}}%
\pgfpathlineto{\pgfqpoint{2.445625in}{1.376093in}}%
\pgfpathmoveto{\pgfqpoint{2.445625in}{1.333515in}}%
\pgfpathlineto{\pgfqpoint{2.445625in}{1.333515in}}%
\pgfpathlineto{\pgfqpoint{2.445625in}{1.337772in}}%
\pgfpathlineto{\pgfqpoint{2.449883in}{1.337772in}}%
\pgfpathlineto{\pgfqpoint{2.449883in}{1.333515in}}%
\pgfpathmoveto{\pgfqpoint{2.445625in}{1.337772in}}%
\pgfpathlineto{\pgfqpoint{2.445625in}{1.337772in}}%
\pgfpathlineto{\pgfqpoint{2.445625in}{1.342030in}}%
\pgfpathlineto{\pgfqpoint{2.449883in}{1.342030in}}%
\pgfpathlineto{\pgfqpoint{2.449883in}{1.337772in}}%
\pgfpathmoveto{\pgfqpoint{2.445625in}{1.342030in}}%
\pgfpathlineto{\pgfqpoint{2.445625in}{1.342030in}}%
\pgfpathlineto{\pgfqpoint{2.445625in}{1.346288in}}%
\pgfpathlineto{\pgfqpoint{2.449883in}{1.346288in}}%
\pgfpathlineto{\pgfqpoint{2.449883in}{1.342030in}}%
\pgfpathmoveto{\pgfqpoint{2.445625in}{1.346288in}}%
\pgfpathlineto{\pgfqpoint{2.445625in}{1.346288in}}%
\pgfpathlineto{\pgfqpoint{2.445625in}{1.350546in}}%
\pgfpathlineto{\pgfqpoint{2.449883in}{1.350546in}}%
\pgfpathlineto{\pgfqpoint{2.449883in}{1.346288in}}%
\pgfpathmoveto{\pgfqpoint{2.445625in}{1.350546in}}%
\pgfpathlineto{\pgfqpoint{2.445625in}{1.350546in}}%
\pgfpathlineto{\pgfqpoint{2.445625in}{1.354804in}}%
\pgfpathlineto{\pgfqpoint{2.449883in}{1.354804in}}%
\pgfpathlineto{\pgfqpoint{2.449883in}{1.350546in}}%
\pgfpathmoveto{\pgfqpoint{2.445625in}{1.354804in}}%
\pgfpathlineto{\pgfqpoint{2.445625in}{1.354804in}}%
\pgfpathlineto{\pgfqpoint{2.445625in}{1.359062in}}%
\pgfpathlineto{\pgfqpoint{2.449883in}{1.359062in}}%
\pgfpathlineto{\pgfqpoint{2.449883in}{1.354804in}}%
\pgfpathmoveto{\pgfqpoint{2.445625in}{1.359062in}}%
\pgfpathlineto{\pgfqpoint{2.445625in}{1.359062in}}%
\pgfpathlineto{\pgfqpoint{2.445625in}{1.363320in}}%
\pgfpathlineto{\pgfqpoint{2.449883in}{1.363320in}}%
\pgfpathlineto{\pgfqpoint{2.449883in}{1.359062in}}%
\pgfpathmoveto{\pgfqpoint{2.445625in}{1.363320in}}%
\pgfpathlineto{\pgfqpoint{2.445625in}{1.363320in}}%
\pgfpathlineto{\pgfqpoint{2.445625in}{1.367577in}}%
\pgfpathlineto{\pgfqpoint{2.449883in}{1.367577in}}%
\pgfpathlineto{\pgfqpoint{2.449883in}{1.363320in}}%
\pgfpathmoveto{\pgfqpoint{2.445625in}{1.367577in}}%
\pgfpathlineto{\pgfqpoint{2.445625in}{1.367577in}}%
\pgfpathlineto{\pgfqpoint{2.445625in}{1.371835in}}%
\pgfpathlineto{\pgfqpoint{2.449883in}{1.371835in}}%
\pgfpathlineto{\pgfqpoint{2.449883in}{1.367577in}}%
\pgfpathmoveto{\pgfqpoint{2.445625in}{1.371835in}}%
\pgfpathlineto{\pgfqpoint{2.445625in}{1.371835in}}%
\pgfpathlineto{\pgfqpoint{2.445625in}{1.376093in}}%
\pgfpathlineto{\pgfqpoint{2.449883in}{1.376093in}}%
\pgfpathlineto{\pgfqpoint{2.449883in}{1.371835in}}%
\pgfpathmoveto{\pgfqpoint{2.449883in}{1.371835in}}%
\pgfpathlineto{\pgfqpoint{2.449883in}{1.371835in}}%
\pgfpathlineto{\pgfqpoint{2.449883in}{1.376093in}}%
\pgfpathlineto{\pgfqpoint{2.454141in}{1.376093in}}%
\pgfpathlineto{\pgfqpoint{2.454141in}{1.371835in}}%
\pgfpathmoveto{\pgfqpoint{2.445625in}{1.376093in}}%
\pgfpathlineto{\pgfqpoint{2.445625in}{1.376093in}}%
\pgfpathlineto{\pgfqpoint{2.445625in}{1.380351in}}%
\pgfpathlineto{\pgfqpoint{2.449883in}{1.380351in}}%
\pgfpathlineto{\pgfqpoint{2.449883in}{1.376093in}}%
\pgfpathmoveto{\pgfqpoint{2.445625in}{1.380351in}}%
\pgfpathlineto{\pgfqpoint{2.445625in}{1.380351in}}%
\pgfpathlineto{\pgfqpoint{2.445625in}{1.384609in}}%
\pgfpathlineto{\pgfqpoint{2.449883in}{1.384609in}}%
\pgfpathlineto{\pgfqpoint{2.449883in}{1.380351in}}%
\pgfpathmoveto{\pgfqpoint{2.449883in}{1.376093in}}%
\pgfpathlineto{\pgfqpoint{2.449883in}{1.376093in}}%
\pgfpathlineto{\pgfqpoint{2.449883in}{1.380351in}}%
\pgfpathlineto{\pgfqpoint{2.454141in}{1.380351in}}%
\pgfpathlineto{\pgfqpoint{2.454141in}{1.376093in}}%
\pgfpathmoveto{\pgfqpoint{2.449883in}{1.380351in}}%
\pgfpathlineto{\pgfqpoint{2.449883in}{1.380351in}}%
\pgfpathlineto{\pgfqpoint{2.449883in}{1.384609in}}%
\pgfpathlineto{\pgfqpoint{2.454141in}{1.384609in}}%
\pgfpathlineto{\pgfqpoint{2.454141in}{1.380351in}}%
\pgfpathmoveto{\pgfqpoint{2.445625in}{1.384609in}}%
\pgfpathlineto{\pgfqpoint{2.445625in}{1.384609in}}%
\pgfpathlineto{\pgfqpoint{2.445625in}{1.388867in}}%
\pgfpathlineto{\pgfqpoint{2.449883in}{1.388867in}}%
\pgfpathlineto{\pgfqpoint{2.449883in}{1.384609in}}%
\pgfpathmoveto{\pgfqpoint{2.445625in}{1.388867in}}%
\pgfpathlineto{\pgfqpoint{2.445625in}{1.388867in}}%
\pgfpathlineto{\pgfqpoint{2.445625in}{1.393124in}}%
\pgfpathlineto{\pgfqpoint{2.449883in}{1.393124in}}%
\pgfpathlineto{\pgfqpoint{2.449883in}{1.388867in}}%
\pgfpathmoveto{\pgfqpoint{2.449883in}{1.384609in}}%
\pgfpathlineto{\pgfqpoint{2.449883in}{1.384609in}}%
\pgfpathlineto{\pgfqpoint{2.449883in}{1.388867in}}%
\pgfpathlineto{\pgfqpoint{2.454141in}{1.388867in}}%
\pgfpathlineto{\pgfqpoint{2.454141in}{1.384609in}}%
\pgfpathmoveto{\pgfqpoint{2.449883in}{1.388867in}}%
\pgfpathlineto{\pgfqpoint{2.449883in}{1.388867in}}%
\pgfpathlineto{\pgfqpoint{2.449883in}{1.393124in}}%
\pgfpathlineto{\pgfqpoint{2.454141in}{1.393124in}}%
\pgfpathlineto{\pgfqpoint{2.454141in}{1.388867in}}%
\pgfpathmoveto{\pgfqpoint{2.445625in}{1.393124in}}%
\pgfpathlineto{\pgfqpoint{2.445625in}{1.393124in}}%
\pgfpathlineto{\pgfqpoint{2.445625in}{1.397382in}}%
\pgfpathlineto{\pgfqpoint{2.449883in}{1.397382in}}%
\pgfpathlineto{\pgfqpoint{2.449883in}{1.393124in}}%
\pgfpathmoveto{\pgfqpoint{2.445625in}{1.397382in}}%
\pgfpathlineto{\pgfqpoint{2.445625in}{1.397382in}}%
\pgfpathlineto{\pgfqpoint{2.445625in}{1.401640in}}%
\pgfpathlineto{\pgfqpoint{2.449883in}{1.401640in}}%
\pgfpathlineto{\pgfqpoint{2.449883in}{1.397382in}}%
\pgfpathmoveto{\pgfqpoint{2.449883in}{1.393124in}}%
\pgfpathlineto{\pgfqpoint{2.449883in}{1.393124in}}%
\pgfpathlineto{\pgfqpoint{2.449883in}{1.397382in}}%
\pgfpathlineto{\pgfqpoint{2.454141in}{1.397382in}}%
\pgfpathlineto{\pgfqpoint{2.454141in}{1.393124in}}%
\pgfpathmoveto{\pgfqpoint{2.449883in}{1.397382in}}%
\pgfpathlineto{\pgfqpoint{2.449883in}{1.397382in}}%
\pgfpathlineto{\pgfqpoint{2.449883in}{1.401640in}}%
\pgfpathlineto{\pgfqpoint{2.454141in}{1.401640in}}%
\pgfpathlineto{\pgfqpoint{2.454141in}{1.397382in}}%
\pgfpathmoveto{\pgfqpoint{2.445625in}{1.401640in}}%
\pgfpathlineto{\pgfqpoint{2.445625in}{1.401640in}}%
\pgfpathlineto{\pgfqpoint{2.445625in}{1.405898in}}%
\pgfpathlineto{\pgfqpoint{2.449883in}{1.405898in}}%
\pgfpathlineto{\pgfqpoint{2.449883in}{1.401640in}}%
\pgfpathmoveto{\pgfqpoint{2.445625in}{1.405898in}}%
\pgfpathlineto{\pgfqpoint{2.445625in}{1.405898in}}%
\pgfpathlineto{\pgfqpoint{2.445625in}{1.410156in}}%
\pgfpathlineto{\pgfqpoint{2.449883in}{1.410156in}}%
\pgfpathlineto{\pgfqpoint{2.449883in}{1.405898in}}%
\pgfpathmoveto{\pgfqpoint{2.449883in}{1.401640in}}%
\pgfpathlineto{\pgfqpoint{2.449883in}{1.401640in}}%
\pgfpathlineto{\pgfqpoint{2.449883in}{1.405898in}}%
\pgfpathlineto{\pgfqpoint{2.454141in}{1.405898in}}%
\pgfpathlineto{\pgfqpoint{2.454141in}{1.401640in}}%
\pgfpathmoveto{\pgfqpoint{2.449883in}{1.405898in}}%
\pgfpathlineto{\pgfqpoint{2.449883in}{1.405898in}}%
\pgfpathlineto{\pgfqpoint{2.449883in}{1.410156in}}%
\pgfpathlineto{\pgfqpoint{2.454141in}{1.410156in}}%
\pgfpathlineto{\pgfqpoint{2.454141in}{1.405898in}}%
\pgfpathmoveto{\pgfqpoint{2.445625in}{1.410156in}}%
\pgfpathlineto{\pgfqpoint{2.445625in}{1.410156in}}%
\pgfpathlineto{\pgfqpoint{2.445625in}{1.414414in}}%
\pgfpathlineto{\pgfqpoint{2.449883in}{1.414414in}}%
\pgfpathlineto{\pgfqpoint{2.449883in}{1.410156in}}%
\pgfpathmoveto{\pgfqpoint{2.445625in}{1.414414in}}%
\pgfpathlineto{\pgfqpoint{2.445625in}{1.414414in}}%
\pgfpathlineto{\pgfqpoint{2.445625in}{1.418672in}}%
\pgfpathlineto{\pgfqpoint{2.449883in}{1.418672in}}%
\pgfpathlineto{\pgfqpoint{2.449883in}{1.414414in}}%
\pgfpathmoveto{\pgfqpoint{2.449883in}{1.410156in}}%
\pgfpathlineto{\pgfqpoint{2.449883in}{1.410156in}}%
\pgfpathlineto{\pgfqpoint{2.449883in}{1.414414in}}%
\pgfpathlineto{\pgfqpoint{2.454141in}{1.414414in}}%
\pgfpathlineto{\pgfqpoint{2.454141in}{1.410156in}}%
\pgfpathmoveto{\pgfqpoint{2.449883in}{1.414414in}}%
\pgfpathlineto{\pgfqpoint{2.449883in}{1.414414in}}%
\pgfpathlineto{\pgfqpoint{2.449883in}{1.418672in}}%
\pgfpathlineto{\pgfqpoint{2.454141in}{1.418672in}}%
\pgfpathlineto{\pgfqpoint{2.454141in}{1.414414in}}%
\pgfpathmoveto{\pgfqpoint{2.449883in}{1.418672in}}%
\pgfpathlineto{\pgfqpoint{2.449883in}{1.418672in}}%
\pgfpathlineto{\pgfqpoint{2.449883in}{1.422929in}}%
\pgfpathlineto{\pgfqpoint{2.454141in}{1.422929in}}%
\pgfpathlineto{\pgfqpoint{2.454141in}{1.418672in}}%
\pgfpathmoveto{\pgfqpoint{2.449883in}{1.422929in}}%
\pgfpathlineto{\pgfqpoint{2.449883in}{1.422929in}}%
\pgfpathlineto{\pgfqpoint{2.449883in}{1.427187in}}%
\pgfpathlineto{\pgfqpoint{2.454141in}{1.427187in}}%
\pgfpathlineto{\pgfqpoint{2.454141in}{1.422929in}}%
\pgfpathmoveto{\pgfqpoint{2.454141in}{1.410156in}}%
\pgfpathlineto{\pgfqpoint{2.454141in}{1.410156in}}%
\pgfpathlineto{\pgfqpoint{2.454141in}{1.414414in}}%
\pgfpathlineto{\pgfqpoint{2.458399in}{1.414414in}}%
\pgfpathlineto{\pgfqpoint{2.458399in}{1.410156in}}%
\pgfpathmoveto{\pgfqpoint{2.454141in}{1.414414in}}%
\pgfpathlineto{\pgfqpoint{2.454141in}{1.414414in}}%
\pgfpathlineto{\pgfqpoint{2.454141in}{1.418672in}}%
\pgfpathlineto{\pgfqpoint{2.458399in}{1.418672in}}%
\pgfpathlineto{\pgfqpoint{2.458399in}{1.414414in}}%
\pgfpathmoveto{\pgfqpoint{2.454141in}{1.418672in}}%
\pgfpathlineto{\pgfqpoint{2.454141in}{1.418672in}}%
\pgfpathlineto{\pgfqpoint{2.454141in}{1.422929in}}%
\pgfpathlineto{\pgfqpoint{2.458399in}{1.422929in}}%
\pgfpathlineto{\pgfqpoint{2.458399in}{1.418672in}}%
\pgfpathmoveto{\pgfqpoint{2.454141in}{1.422929in}}%
\pgfpathlineto{\pgfqpoint{2.454141in}{1.422929in}}%
\pgfpathlineto{\pgfqpoint{2.454141in}{1.427187in}}%
\pgfpathlineto{\pgfqpoint{2.458399in}{1.427187in}}%
\pgfpathlineto{\pgfqpoint{2.458399in}{1.422929in}}%
\pgfpathmoveto{\pgfqpoint{2.449883in}{1.427187in}}%
\pgfpathlineto{\pgfqpoint{2.449883in}{1.427187in}}%
\pgfpathlineto{\pgfqpoint{2.449883in}{1.431445in}}%
\pgfpathlineto{\pgfqpoint{2.454141in}{1.431445in}}%
\pgfpathlineto{\pgfqpoint{2.454141in}{1.427187in}}%
\pgfpathmoveto{\pgfqpoint{2.449883in}{1.431445in}}%
\pgfpathlineto{\pgfqpoint{2.449883in}{1.431445in}}%
\pgfpathlineto{\pgfqpoint{2.449883in}{1.435703in}}%
\pgfpathlineto{\pgfqpoint{2.454141in}{1.435703in}}%
\pgfpathlineto{\pgfqpoint{2.454141in}{1.431445in}}%
\pgfpathmoveto{\pgfqpoint{2.449883in}{1.435703in}}%
\pgfpathlineto{\pgfqpoint{2.449883in}{1.435703in}}%
\pgfpathlineto{\pgfqpoint{2.449883in}{1.439961in}}%
\pgfpathlineto{\pgfqpoint{2.454141in}{1.439961in}}%
\pgfpathlineto{\pgfqpoint{2.454141in}{1.435703in}}%
\pgfpathmoveto{\pgfqpoint{2.449883in}{1.439961in}}%
\pgfpathlineto{\pgfqpoint{2.449883in}{1.439961in}}%
\pgfpathlineto{\pgfqpoint{2.449883in}{1.444219in}}%
\pgfpathlineto{\pgfqpoint{2.454141in}{1.444219in}}%
\pgfpathlineto{\pgfqpoint{2.454141in}{1.439961in}}%
\pgfpathmoveto{\pgfqpoint{2.454141in}{1.427187in}}%
\pgfpathlineto{\pgfqpoint{2.454141in}{1.427187in}}%
\pgfpathlineto{\pgfqpoint{2.454141in}{1.431445in}}%
\pgfpathlineto{\pgfqpoint{2.458399in}{1.431445in}}%
\pgfpathlineto{\pgfqpoint{2.458399in}{1.427187in}}%
\pgfpathmoveto{\pgfqpoint{2.454141in}{1.431445in}}%
\pgfpathlineto{\pgfqpoint{2.454141in}{1.431445in}}%
\pgfpathlineto{\pgfqpoint{2.454141in}{1.435703in}}%
\pgfpathlineto{\pgfqpoint{2.458399in}{1.435703in}}%
\pgfpathlineto{\pgfqpoint{2.458399in}{1.431445in}}%
\pgfpathmoveto{\pgfqpoint{2.454141in}{1.435703in}}%
\pgfpathlineto{\pgfqpoint{2.454141in}{1.435703in}}%
\pgfpathlineto{\pgfqpoint{2.454141in}{1.439961in}}%
\pgfpathlineto{\pgfqpoint{2.458399in}{1.439961in}}%
\pgfpathlineto{\pgfqpoint{2.458399in}{1.435703in}}%
\pgfpathmoveto{\pgfqpoint{2.454141in}{1.439961in}}%
\pgfpathlineto{\pgfqpoint{2.454141in}{1.439961in}}%
\pgfpathlineto{\pgfqpoint{2.454141in}{1.444219in}}%
\pgfpathlineto{\pgfqpoint{2.458399in}{1.444219in}}%
\pgfpathlineto{\pgfqpoint{2.458399in}{1.439961in}}%
\pgfpathmoveto{\pgfqpoint{2.449883in}{1.444219in}}%
\pgfpathlineto{\pgfqpoint{2.449883in}{1.444219in}}%
\pgfpathlineto{\pgfqpoint{2.449883in}{1.448477in}}%
\pgfpathlineto{\pgfqpoint{2.454141in}{1.448477in}}%
\pgfpathlineto{\pgfqpoint{2.454141in}{1.444219in}}%
\pgfpathmoveto{\pgfqpoint{2.449883in}{1.448477in}}%
\pgfpathlineto{\pgfqpoint{2.449883in}{1.448477in}}%
\pgfpathlineto{\pgfqpoint{2.449883in}{1.452734in}}%
\pgfpathlineto{\pgfqpoint{2.454141in}{1.452734in}}%
\pgfpathlineto{\pgfqpoint{2.454141in}{1.448477in}}%
\pgfpathmoveto{\pgfqpoint{2.449883in}{1.452734in}}%
\pgfpathlineto{\pgfqpoint{2.449883in}{1.452734in}}%
\pgfpathlineto{\pgfqpoint{2.449883in}{1.456992in}}%
\pgfpathlineto{\pgfqpoint{2.454141in}{1.456992in}}%
\pgfpathlineto{\pgfqpoint{2.454141in}{1.452734in}}%
\pgfpathmoveto{\pgfqpoint{2.454141in}{1.444219in}}%
\pgfpathlineto{\pgfqpoint{2.454141in}{1.444219in}}%
\pgfpathlineto{\pgfqpoint{2.454141in}{1.448477in}}%
\pgfpathlineto{\pgfqpoint{2.458399in}{1.448477in}}%
\pgfpathlineto{\pgfqpoint{2.458399in}{1.444219in}}%
\pgfpathmoveto{\pgfqpoint{2.454141in}{1.448477in}}%
\pgfpathlineto{\pgfqpoint{2.454141in}{1.448477in}}%
\pgfpathlineto{\pgfqpoint{2.454141in}{1.452734in}}%
\pgfpathlineto{\pgfqpoint{2.458399in}{1.452734in}}%
\pgfpathlineto{\pgfqpoint{2.458399in}{1.448477in}}%
\pgfpathmoveto{\pgfqpoint{2.458399in}{1.448477in}}%
\pgfpathlineto{\pgfqpoint{2.458399in}{1.448477in}}%
\pgfpathlineto{\pgfqpoint{2.458399in}{1.452734in}}%
\pgfpathlineto{\pgfqpoint{2.462656in}{1.452734in}}%
\pgfpathlineto{\pgfqpoint{2.462656in}{1.448477in}}%
\pgfpathmoveto{\pgfqpoint{2.454141in}{1.452734in}}%
\pgfpathlineto{\pgfqpoint{2.454141in}{1.452734in}}%
\pgfpathlineto{\pgfqpoint{2.454141in}{1.456992in}}%
\pgfpathlineto{\pgfqpoint{2.458399in}{1.456992in}}%
\pgfpathlineto{\pgfqpoint{2.458399in}{1.452734in}}%
\pgfpathmoveto{\pgfqpoint{2.454141in}{1.456992in}}%
\pgfpathlineto{\pgfqpoint{2.454141in}{1.456992in}}%
\pgfpathlineto{\pgfqpoint{2.454141in}{1.461250in}}%
\pgfpathlineto{\pgfqpoint{2.458399in}{1.461250in}}%
\pgfpathlineto{\pgfqpoint{2.458399in}{1.456992in}}%
\pgfpathmoveto{\pgfqpoint{2.458399in}{1.452734in}}%
\pgfpathlineto{\pgfqpoint{2.458399in}{1.452734in}}%
\pgfpathlineto{\pgfqpoint{2.458399in}{1.456992in}}%
\pgfpathlineto{\pgfqpoint{2.462656in}{1.456992in}}%
\pgfpathlineto{\pgfqpoint{2.462656in}{1.452734in}}%
\pgfpathmoveto{\pgfqpoint{2.458399in}{1.456992in}}%
\pgfpathlineto{\pgfqpoint{2.458399in}{1.456992in}}%
\pgfpathlineto{\pgfqpoint{2.458399in}{1.461250in}}%
\pgfpathlineto{\pgfqpoint{2.462656in}{1.461250in}}%
\pgfpathlineto{\pgfqpoint{2.462656in}{1.456992in}}%
\pgfpathmoveto{\pgfqpoint{2.454141in}{1.461250in}}%
\pgfpathlineto{\pgfqpoint{2.454141in}{1.461250in}}%
\pgfpathlineto{\pgfqpoint{2.454141in}{1.465508in}}%
\pgfpathlineto{\pgfqpoint{2.458399in}{1.465508in}}%
\pgfpathlineto{\pgfqpoint{2.458399in}{1.461250in}}%
\pgfpathmoveto{\pgfqpoint{2.454141in}{1.465508in}}%
\pgfpathlineto{\pgfqpoint{2.454141in}{1.465508in}}%
\pgfpathlineto{\pgfqpoint{2.454141in}{1.469766in}}%
\pgfpathlineto{\pgfqpoint{2.458399in}{1.469766in}}%
\pgfpathlineto{\pgfqpoint{2.458399in}{1.465508in}}%
\pgfpathmoveto{\pgfqpoint{2.458399in}{1.461250in}}%
\pgfpathlineto{\pgfqpoint{2.458399in}{1.461250in}}%
\pgfpathlineto{\pgfqpoint{2.458399in}{1.465508in}}%
\pgfpathlineto{\pgfqpoint{2.462656in}{1.465508in}}%
\pgfpathlineto{\pgfqpoint{2.462656in}{1.461250in}}%
\pgfpathmoveto{\pgfqpoint{2.458399in}{1.465508in}}%
\pgfpathlineto{\pgfqpoint{2.458399in}{1.465508in}}%
\pgfpathlineto{\pgfqpoint{2.458399in}{1.469766in}}%
\pgfpathlineto{\pgfqpoint{2.462656in}{1.469766in}}%
\pgfpathlineto{\pgfqpoint{2.462656in}{1.465508in}}%
\pgfpathmoveto{\pgfqpoint{2.454141in}{1.469766in}}%
\pgfpathlineto{\pgfqpoint{2.454141in}{1.469766in}}%
\pgfpathlineto{\pgfqpoint{2.454141in}{1.474023in}}%
\pgfpathlineto{\pgfqpoint{2.458399in}{1.474023in}}%
\pgfpathlineto{\pgfqpoint{2.458399in}{1.469766in}}%
\pgfpathmoveto{\pgfqpoint{2.454141in}{1.474023in}}%
\pgfpathlineto{\pgfqpoint{2.454141in}{1.474023in}}%
\pgfpathlineto{\pgfqpoint{2.454141in}{1.478281in}}%
\pgfpathlineto{\pgfqpoint{2.458399in}{1.478281in}}%
\pgfpathlineto{\pgfqpoint{2.458399in}{1.474023in}}%
\pgfpathmoveto{\pgfqpoint{2.458399in}{1.469766in}}%
\pgfpathlineto{\pgfqpoint{2.458399in}{1.469766in}}%
\pgfpathlineto{\pgfqpoint{2.458399in}{1.474023in}}%
\pgfpathlineto{\pgfqpoint{2.462656in}{1.474023in}}%
\pgfpathlineto{\pgfqpoint{2.462656in}{1.469766in}}%
\pgfpathmoveto{\pgfqpoint{2.458399in}{1.474023in}}%
\pgfpathlineto{\pgfqpoint{2.458399in}{1.474023in}}%
\pgfpathlineto{\pgfqpoint{2.458399in}{1.478281in}}%
\pgfpathlineto{\pgfqpoint{2.462656in}{1.478281in}}%
\pgfpathlineto{\pgfqpoint{2.462656in}{1.474023in}}%
\pgfpathmoveto{\pgfqpoint{2.454141in}{1.478281in}}%
\pgfpathlineto{\pgfqpoint{2.454141in}{1.478281in}}%
\pgfpathlineto{\pgfqpoint{2.454141in}{1.482539in}}%
\pgfpathlineto{\pgfqpoint{2.458399in}{1.482539in}}%
\pgfpathlineto{\pgfqpoint{2.458399in}{1.478281in}}%
\pgfpathmoveto{\pgfqpoint{2.454141in}{1.482539in}}%
\pgfpathlineto{\pgfqpoint{2.454141in}{1.482539in}}%
\pgfpathlineto{\pgfqpoint{2.454141in}{1.486797in}}%
\pgfpathlineto{\pgfqpoint{2.458399in}{1.486797in}}%
\pgfpathlineto{\pgfqpoint{2.458399in}{1.482539in}}%
\pgfpathmoveto{\pgfqpoint{2.458399in}{1.478281in}}%
\pgfpathlineto{\pgfqpoint{2.458399in}{1.478281in}}%
\pgfpathlineto{\pgfqpoint{2.458399in}{1.482539in}}%
\pgfpathlineto{\pgfqpoint{2.462656in}{1.482539in}}%
\pgfpathlineto{\pgfqpoint{2.462656in}{1.478281in}}%
\pgfpathmoveto{\pgfqpoint{2.458399in}{1.482539in}}%
\pgfpathlineto{\pgfqpoint{2.458399in}{1.482539in}}%
\pgfpathlineto{\pgfqpoint{2.458399in}{1.486797in}}%
\pgfpathlineto{\pgfqpoint{2.462656in}{1.486797in}}%
\pgfpathlineto{\pgfqpoint{2.462656in}{1.482539in}}%
\pgfpathmoveto{\pgfqpoint{2.454141in}{1.486797in}}%
\pgfpathlineto{\pgfqpoint{2.454141in}{1.486797in}}%
\pgfpathlineto{\pgfqpoint{2.454141in}{1.491055in}}%
\pgfpathlineto{\pgfqpoint{2.458399in}{1.491055in}}%
\pgfpathlineto{\pgfqpoint{2.458399in}{1.486797in}}%
\pgfpathmoveto{\pgfqpoint{2.454141in}{1.491055in}}%
\pgfpathlineto{\pgfqpoint{2.454141in}{1.491055in}}%
\pgfpathlineto{\pgfqpoint{2.454141in}{1.495312in}}%
\pgfpathlineto{\pgfqpoint{2.458399in}{1.495312in}}%
\pgfpathlineto{\pgfqpoint{2.458399in}{1.491055in}}%
\pgfpathmoveto{\pgfqpoint{2.458399in}{1.486797in}}%
\pgfpathlineto{\pgfqpoint{2.458399in}{1.486797in}}%
\pgfpathlineto{\pgfqpoint{2.458399in}{1.491055in}}%
\pgfpathlineto{\pgfqpoint{2.462656in}{1.491055in}}%
\pgfpathlineto{\pgfqpoint{2.462656in}{1.486797in}}%
\pgfpathmoveto{\pgfqpoint{2.458399in}{1.491055in}}%
\pgfpathlineto{\pgfqpoint{2.458399in}{1.491055in}}%
\pgfpathlineto{\pgfqpoint{2.458399in}{1.495312in}}%
\pgfpathlineto{\pgfqpoint{2.462656in}{1.495312in}}%
\pgfpathlineto{\pgfqpoint{2.462656in}{1.491055in}}%
\pgfpathmoveto{\pgfqpoint{2.462656in}{1.486797in}}%
\pgfpathlineto{\pgfqpoint{2.462656in}{1.486797in}}%
\pgfpathlineto{\pgfqpoint{2.462656in}{1.491055in}}%
\pgfpathlineto{\pgfqpoint{2.466914in}{1.491055in}}%
\pgfpathlineto{\pgfqpoint{2.466914in}{1.486797in}}%
\pgfpathmoveto{\pgfqpoint{2.462656in}{1.491055in}}%
\pgfpathlineto{\pgfqpoint{2.462656in}{1.491055in}}%
\pgfpathlineto{\pgfqpoint{2.462656in}{1.495312in}}%
\pgfpathlineto{\pgfqpoint{2.466914in}{1.495312in}}%
\pgfpathlineto{\pgfqpoint{2.466914in}{1.491055in}}%
\pgfpathmoveto{\pgfqpoint{2.458399in}{1.495312in}}%
\pgfpathlineto{\pgfqpoint{2.458399in}{1.495312in}}%
\pgfpathlineto{\pgfqpoint{2.458399in}{1.499570in}}%
\pgfpathlineto{\pgfqpoint{2.462656in}{1.499570in}}%
\pgfpathlineto{\pgfqpoint{2.462656in}{1.495312in}}%
\pgfpathmoveto{\pgfqpoint{2.458399in}{1.499570in}}%
\pgfpathlineto{\pgfqpoint{2.458399in}{1.499570in}}%
\pgfpathlineto{\pgfqpoint{2.458399in}{1.503828in}}%
\pgfpathlineto{\pgfqpoint{2.462656in}{1.503828in}}%
\pgfpathlineto{\pgfqpoint{2.462656in}{1.499570in}}%
\pgfpathmoveto{\pgfqpoint{2.458399in}{1.503828in}}%
\pgfpathlineto{\pgfqpoint{2.458399in}{1.503828in}}%
\pgfpathlineto{\pgfqpoint{2.458399in}{1.508086in}}%
\pgfpathlineto{\pgfqpoint{2.462656in}{1.508086in}}%
\pgfpathlineto{\pgfqpoint{2.462656in}{1.503828in}}%
\pgfpathmoveto{\pgfqpoint{2.458399in}{1.508086in}}%
\pgfpathlineto{\pgfqpoint{2.458399in}{1.508086in}}%
\pgfpathlineto{\pgfqpoint{2.458399in}{1.512343in}}%
\pgfpathlineto{\pgfqpoint{2.462656in}{1.512343in}}%
\pgfpathlineto{\pgfqpoint{2.462656in}{1.508086in}}%
\pgfpathmoveto{\pgfqpoint{2.458399in}{1.512343in}}%
\pgfpathlineto{\pgfqpoint{2.458399in}{1.512343in}}%
\pgfpathlineto{\pgfqpoint{2.458399in}{1.516601in}}%
\pgfpathlineto{\pgfqpoint{2.462656in}{1.516601in}}%
\pgfpathlineto{\pgfqpoint{2.462656in}{1.512343in}}%
\pgfpathmoveto{\pgfqpoint{2.458399in}{1.516601in}}%
\pgfpathlineto{\pgfqpoint{2.458399in}{1.516601in}}%
\pgfpathlineto{\pgfqpoint{2.458399in}{1.520859in}}%
\pgfpathlineto{\pgfqpoint{2.462656in}{1.520859in}}%
\pgfpathlineto{\pgfqpoint{2.462656in}{1.516601in}}%
\pgfpathmoveto{\pgfqpoint{2.458399in}{1.520859in}}%
\pgfpathlineto{\pgfqpoint{2.458399in}{1.520859in}}%
\pgfpathlineto{\pgfqpoint{2.458399in}{1.525117in}}%
\pgfpathlineto{\pgfqpoint{2.462656in}{1.525117in}}%
\pgfpathlineto{\pgfqpoint{2.462656in}{1.520859in}}%
\pgfpathmoveto{\pgfqpoint{2.458399in}{1.525117in}}%
\pgfpathlineto{\pgfqpoint{2.458399in}{1.525117in}}%
\pgfpathlineto{\pgfqpoint{2.458399in}{1.529375in}}%
\pgfpathlineto{\pgfqpoint{2.462656in}{1.529375in}}%
\pgfpathlineto{\pgfqpoint{2.462656in}{1.525117in}}%
\pgfpathmoveto{\pgfqpoint{2.462656in}{1.495312in}}%
\pgfpathlineto{\pgfqpoint{2.462656in}{1.495312in}}%
\pgfpathlineto{\pgfqpoint{2.462656in}{1.499570in}}%
\pgfpathlineto{\pgfqpoint{2.466914in}{1.499570in}}%
\pgfpathlineto{\pgfqpoint{2.466914in}{1.495312in}}%
\pgfpathmoveto{\pgfqpoint{2.462656in}{1.499570in}}%
\pgfpathlineto{\pgfqpoint{2.462656in}{1.499570in}}%
\pgfpathlineto{\pgfqpoint{2.462656in}{1.503828in}}%
\pgfpathlineto{\pgfqpoint{2.466914in}{1.503828in}}%
\pgfpathlineto{\pgfqpoint{2.466914in}{1.499570in}}%
\pgfpathmoveto{\pgfqpoint{2.462656in}{1.503828in}}%
\pgfpathlineto{\pgfqpoint{2.462656in}{1.503828in}}%
\pgfpathlineto{\pgfqpoint{2.462656in}{1.508086in}}%
\pgfpathlineto{\pgfqpoint{2.466914in}{1.508086in}}%
\pgfpathlineto{\pgfqpoint{2.466914in}{1.503828in}}%
\pgfpathmoveto{\pgfqpoint{2.462656in}{1.508086in}}%
\pgfpathlineto{\pgfqpoint{2.462656in}{1.508086in}}%
\pgfpathlineto{\pgfqpoint{2.462656in}{1.512343in}}%
\pgfpathlineto{\pgfqpoint{2.466914in}{1.512343in}}%
\pgfpathlineto{\pgfqpoint{2.466914in}{1.508086in}}%
\pgfpathmoveto{\pgfqpoint{2.462656in}{1.512343in}}%
\pgfpathlineto{\pgfqpoint{2.462656in}{1.512343in}}%
\pgfpathlineto{\pgfqpoint{2.462656in}{1.516601in}}%
\pgfpathlineto{\pgfqpoint{2.466914in}{1.516601in}}%
\pgfpathlineto{\pgfqpoint{2.466914in}{1.512343in}}%
\pgfpathmoveto{\pgfqpoint{2.462656in}{1.516601in}}%
\pgfpathlineto{\pgfqpoint{2.462656in}{1.516601in}}%
\pgfpathlineto{\pgfqpoint{2.462656in}{1.520859in}}%
\pgfpathlineto{\pgfqpoint{2.466914in}{1.520859in}}%
\pgfpathlineto{\pgfqpoint{2.466914in}{1.516601in}}%
\pgfpathmoveto{\pgfqpoint{2.462656in}{1.520859in}}%
\pgfpathlineto{\pgfqpoint{2.462656in}{1.520859in}}%
\pgfpathlineto{\pgfqpoint{2.462656in}{1.525117in}}%
\pgfpathlineto{\pgfqpoint{2.466914in}{1.525117in}}%
\pgfpathlineto{\pgfqpoint{2.466914in}{1.520859in}}%
\pgfpathmoveto{\pgfqpoint{2.462656in}{1.525117in}}%
\pgfpathlineto{\pgfqpoint{2.462656in}{1.525117in}}%
\pgfpathlineto{\pgfqpoint{2.462656in}{1.529375in}}%
\pgfpathlineto{\pgfqpoint{2.466914in}{1.529375in}}%
\pgfpathlineto{\pgfqpoint{2.466914in}{1.525117in}}%
\pgfpathmoveto{\pgfqpoint{2.466914in}{1.525117in}}%
\pgfpathlineto{\pgfqpoint{2.466914in}{1.525117in}}%
\pgfpathlineto{\pgfqpoint{2.466914in}{1.529375in}}%
\pgfpathlineto{\pgfqpoint{2.471172in}{1.529375in}}%
\pgfpathlineto{\pgfqpoint{2.471172in}{1.525117in}}%
\pgfpathmoveto{\pgfqpoint{2.458399in}{1.529375in}}%
\pgfpathlineto{\pgfqpoint{2.458399in}{1.529375in}}%
\pgfpathlineto{\pgfqpoint{2.458399in}{1.533632in}}%
\pgfpathlineto{\pgfqpoint{2.462656in}{1.533632in}}%
\pgfpathlineto{\pgfqpoint{2.462656in}{1.529375in}}%
\pgfpathmoveto{\pgfqpoint{2.462656in}{1.529375in}}%
\pgfpathlineto{\pgfqpoint{2.462656in}{1.529375in}}%
\pgfpathlineto{\pgfqpoint{2.462656in}{1.533632in}}%
\pgfpathlineto{\pgfqpoint{2.466914in}{1.533632in}}%
\pgfpathlineto{\pgfqpoint{2.466914in}{1.529375in}}%
\pgfpathmoveto{\pgfqpoint{2.462656in}{1.533632in}}%
\pgfpathlineto{\pgfqpoint{2.462656in}{1.533632in}}%
\pgfpathlineto{\pgfqpoint{2.462656in}{1.537890in}}%
\pgfpathlineto{\pgfqpoint{2.466914in}{1.537890in}}%
\pgfpathlineto{\pgfqpoint{2.466914in}{1.533632in}}%
\pgfpathmoveto{\pgfqpoint{2.466914in}{1.529375in}}%
\pgfpathlineto{\pgfqpoint{2.466914in}{1.529375in}}%
\pgfpathlineto{\pgfqpoint{2.466914in}{1.533632in}}%
\pgfpathlineto{\pgfqpoint{2.471172in}{1.533632in}}%
\pgfpathlineto{\pgfqpoint{2.471172in}{1.529375in}}%
\pgfpathmoveto{\pgfqpoint{2.466914in}{1.533632in}}%
\pgfpathlineto{\pgfqpoint{2.466914in}{1.533632in}}%
\pgfpathlineto{\pgfqpoint{2.466914in}{1.537890in}}%
\pgfpathlineto{\pgfqpoint{2.471172in}{1.537890in}}%
\pgfpathlineto{\pgfqpoint{2.471172in}{1.533632in}}%
\pgfpathmoveto{\pgfqpoint{2.462656in}{1.537890in}}%
\pgfpathlineto{\pgfqpoint{2.462656in}{1.537890in}}%
\pgfpathlineto{\pgfqpoint{2.462656in}{1.542148in}}%
\pgfpathlineto{\pgfqpoint{2.466914in}{1.542148in}}%
\pgfpathlineto{\pgfqpoint{2.466914in}{1.537890in}}%
\pgfpathmoveto{\pgfqpoint{2.462656in}{1.542148in}}%
\pgfpathlineto{\pgfqpoint{2.462656in}{1.542148in}}%
\pgfpathlineto{\pgfqpoint{2.462656in}{1.546406in}}%
\pgfpathlineto{\pgfqpoint{2.466914in}{1.546406in}}%
\pgfpathlineto{\pgfqpoint{2.466914in}{1.542148in}}%
\pgfpathmoveto{\pgfqpoint{2.466914in}{1.537890in}}%
\pgfpathlineto{\pgfqpoint{2.466914in}{1.537890in}}%
\pgfpathlineto{\pgfqpoint{2.466914in}{1.542148in}}%
\pgfpathlineto{\pgfqpoint{2.471172in}{1.542148in}}%
\pgfpathlineto{\pgfqpoint{2.471172in}{1.537890in}}%
\pgfpathmoveto{\pgfqpoint{2.466914in}{1.542148in}}%
\pgfpathlineto{\pgfqpoint{2.466914in}{1.542148in}}%
\pgfpathlineto{\pgfqpoint{2.466914in}{1.546406in}}%
\pgfpathlineto{\pgfqpoint{2.471172in}{1.546406in}}%
\pgfpathlineto{\pgfqpoint{2.471172in}{1.542148in}}%
\pgfpathmoveto{\pgfqpoint{2.462656in}{1.546406in}}%
\pgfpathlineto{\pgfqpoint{2.462656in}{1.546406in}}%
\pgfpathlineto{\pgfqpoint{2.462656in}{1.550663in}}%
\pgfpathlineto{\pgfqpoint{2.466914in}{1.550663in}}%
\pgfpathlineto{\pgfqpoint{2.466914in}{1.546406in}}%
\pgfpathmoveto{\pgfqpoint{2.462656in}{1.550663in}}%
\pgfpathlineto{\pgfqpoint{2.462656in}{1.550663in}}%
\pgfpathlineto{\pgfqpoint{2.462656in}{1.554921in}}%
\pgfpathlineto{\pgfqpoint{2.466914in}{1.554921in}}%
\pgfpathlineto{\pgfqpoint{2.466914in}{1.550663in}}%
\pgfpathmoveto{\pgfqpoint{2.466914in}{1.546406in}}%
\pgfpathlineto{\pgfqpoint{2.466914in}{1.546406in}}%
\pgfpathlineto{\pgfqpoint{2.466914in}{1.550663in}}%
\pgfpathlineto{\pgfqpoint{2.471172in}{1.550663in}}%
\pgfpathlineto{\pgfqpoint{2.471172in}{1.546406in}}%
\pgfpathmoveto{\pgfqpoint{2.466914in}{1.550663in}}%
\pgfpathlineto{\pgfqpoint{2.466914in}{1.550663in}}%
\pgfpathlineto{\pgfqpoint{2.466914in}{1.554921in}}%
\pgfpathlineto{\pgfqpoint{2.471172in}{1.554921in}}%
\pgfpathlineto{\pgfqpoint{2.471172in}{1.550663in}}%
\pgfpathmoveto{\pgfqpoint{2.462656in}{1.554921in}}%
\pgfpathlineto{\pgfqpoint{2.462656in}{1.554921in}}%
\pgfpathlineto{\pgfqpoint{2.462656in}{1.559179in}}%
\pgfpathlineto{\pgfqpoint{2.466914in}{1.559179in}}%
\pgfpathlineto{\pgfqpoint{2.466914in}{1.554921in}}%
\pgfpathmoveto{\pgfqpoint{2.462656in}{1.559179in}}%
\pgfpathlineto{\pgfqpoint{2.462656in}{1.559179in}}%
\pgfpathlineto{\pgfqpoint{2.462656in}{1.563437in}}%
\pgfpathlineto{\pgfqpoint{2.466914in}{1.563437in}}%
\pgfpathlineto{\pgfqpoint{2.466914in}{1.559179in}}%
\pgfpathmoveto{\pgfqpoint{2.466914in}{1.554921in}}%
\pgfpathlineto{\pgfqpoint{2.466914in}{1.554921in}}%
\pgfpathlineto{\pgfqpoint{2.466914in}{1.559179in}}%
\pgfpathlineto{\pgfqpoint{2.471172in}{1.559179in}}%
\pgfpathlineto{\pgfqpoint{2.471172in}{1.554921in}}%
\pgfpathmoveto{\pgfqpoint{2.466914in}{1.559179in}}%
\pgfpathlineto{\pgfqpoint{2.466914in}{1.559179in}}%
\pgfpathlineto{\pgfqpoint{2.466914in}{1.563437in}}%
\pgfpathlineto{\pgfqpoint{2.471172in}{1.563437in}}%
\pgfpathlineto{\pgfqpoint{2.471172in}{1.559179in}}%
\pgfpathmoveto{\pgfqpoint{2.462656in}{1.563437in}}%
\pgfpathlineto{\pgfqpoint{2.462656in}{1.563437in}}%
\pgfpathlineto{\pgfqpoint{2.462656in}{1.567695in}}%
\pgfpathlineto{\pgfqpoint{2.466914in}{1.567695in}}%
\pgfpathlineto{\pgfqpoint{2.466914in}{1.563437in}}%
\pgfpathmoveto{\pgfqpoint{2.462656in}{1.567695in}}%
\pgfpathlineto{\pgfqpoint{2.462656in}{1.567695in}}%
\pgfpathlineto{\pgfqpoint{2.462656in}{1.571952in}}%
\pgfpathlineto{\pgfqpoint{2.466914in}{1.571952in}}%
\pgfpathlineto{\pgfqpoint{2.466914in}{1.567695in}}%
\pgfpathmoveto{\pgfqpoint{2.466914in}{1.563437in}}%
\pgfpathlineto{\pgfqpoint{2.466914in}{1.563437in}}%
\pgfpathlineto{\pgfqpoint{2.466914in}{1.567695in}}%
\pgfpathlineto{\pgfqpoint{2.471172in}{1.567695in}}%
\pgfpathlineto{\pgfqpoint{2.471172in}{1.563437in}}%
\pgfpathmoveto{\pgfqpoint{2.466914in}{1.567695in}}%
\pgfpathlineto{\pgfqpoint{2.466914in}{1.567695in}}%
\pgfpathlineto{\pgfqpoint{2.466914in}{1.571952in}}%
\pgfpathlineto{\pgfqpoint{2.471172in}{1.571952in}}%
\pgfpathlineto{\pgfqpoint{2.471172in}{1.567695in}}%
\pgfpathmoveto{\pgfqpoint{2.466914in}{1.571952in}}%
\pgfpathlineto{\pgfqpoint{2.466914in}{1.571952in}}%
\pgfpathlineto{\pgfqpoint{2.466914in}{1.576210in}}%
\pgfpathlineto{\pgfqpoint{2.471172in}{1.576210in}}%
\pgfpathlineto{\pgfqpoint{2.471172in}{1.571952in}}%
\pgfpathmoveto{\pgfqpoint{2.466914in}{1.576210in}}%
\pgfpathlineto{\pgfqpoint{2.466914in}{1.576210in}}%
\pgfpathlineto{\pgfqpoint{2.466914in}{1.580468in}}%
\pgfpathlineto{\pgfqpoint{2.471172in}{1.580468in}}%
\pgfpathlineto{\pgfqpoint{2.471172in}{1.576210in}}%
\pgfpathmoveto{\pgfqpoint{2.471172in}{1.563437in}}%
\pgfpathlineto{\pgfqpoint{2.471172in}{1.563437in}}%
\pgfpathlineto{\pgfqpoint{2.471172in}{1.567695in}}%
\pgfpathlineto{\pgfqpoint{2.475430in}{1.567695in}}%
\pgfpathlineto{\pgfqpoint{2.475430in}{1.563437in}}%
\pgfpathmoveto{\pgfqpoint{2.471172in}{1.567695in}}%
\pgfpathlineto{\pgfqpoint{2.471172in}{1.567695in}}%
\pgfpathlineto{\pgfqpoint{2.471172in}{1.571952in}}%
\pgfpathlineto{\pgfqpoint{2.475430in}{1.571952in}}%
\pgfpathlineto{\pgfqpoint{2.475430in}{1.567695in}}%
\pgfpathmoveto{\pgfqpoint{2.471172in}{1.571952in}}%
\pgfpathlineto{\pgfqpoint{2.471172in}{1.571952in}}%
\pgfpathlineto{\pgfqpoint{2.471172in}{1.576210in}}%
\pgfpathlineto{\pgfqpoint{2.475430in}{1.576210in}}%
\pgfpathlineto{\pgfqpoint{2.475430in}{1.571952in}}%
\pgfpathmoveto{\pgfqpoint{2.471172in}{1.576210in}}%
\pgfpathlineto{\pgfqpoint{2.471172in}{1.576210in}}%
\pgfpathlineto{\pgfqpoint{2.471172in}{1.580468in}}%
\pgfpathlineto{\pgfqpoint{2.475430in}{1.580468in}}%
\pgfpathlineto{\pgfqpoint{2.475430in}{1.576210in}}%
\pgfpathmoveto{\pgfqpoint{2.466914in}{1.580468in}}%
\pgfpathlineto{\pgfqpoint{2.466914in}{1.580468in}}%
\pgfpathlineto{\pgfqpoint{2.466914in}{1.584726in}}%
\pgfpathlineto{\pgfqpoint{2.471172in}{1.584726in}}%
\pgfpathlineto{\pgfqpoint{2.471172in}{1.580468in}}%
\pgfpathmoveto{\pgfqpoint{2.466914in}{1.584726in}}%
\pgfpathlineto{\pgfqpoint{2.466914in}{1.584726in}}%
\pgfpathlineto{\pgfqpoint{2.466914in}{1.588983in}}%
\pgfpathlineto{\pgfqpoint{2.471172in}{1.588983in}}%
\pgfpathlineto{\pgfqpoint{2.471172in}{1.584726in}}%
\pgfpathmoveto{\pgfqpoint{2.466914in}{1.588983in}}%
\pgfpathlineto{\pgfqpoint{2.466914in}{1.588983in}}%
\pgfpathlineto{\pgfqpoint{2.466914in}{1.593241in}}%
\pgfpathlineto{\pgfqpoint{2.471172in}{1.593241in}}%
\pgfpathlineto{\pgfqpoint{2.471172in}{1.588983in}}%
\pgfpathmoveto{\pgfqpoint{2.466914in}{1.593241in}}%
\pgfpathlineto{\pgfqpoint{2.466914in}{1.593241in}}%
\pgfpathlineto{\pgfqpoint{2.466914in}{1.597499in}}%
\pgfpathlineto{\pgfqpoint{2.471172in}{1.597499in}}%
\pgfpathlineto{\pgfqpoint{2.471172in}{1.593241in}}%
\pgfpathmoveto{\pgfqpoint{2.471172in}{1.580468in}}%
\pgfpathlineto{\pgfqpoint{2.471172in}{1.580468in}}%
\pgfpathlineto{\pgfqpoint{2.471172in}{1.584726in}}%
\pgfpathlineto{\pgfqpoint{2.475430in}{1.584726in}}%
\pgfpathlineto{\pgfqpoint{2.475430in}{1.580468in}}%
\pgfpathmoveto{\pgfqpoint{2.471172in}{1.584726in}}%
\pgfpathlineto{\pgfqpoint{2.471172in}{1.584726in}}%
\pgfpathlineto{\pgfqpoint{2.471172in}{1.588983in}}%
\pgfpathlineto{\pgfqpoint{2.475430in}{1.588983in}}%
\pgfpathlineto{\pgfqpoint{2.475430in}{1.584726in}}%
\pgfpathmoveto{\pgfqpoint{2.471172in}{1.588983in}}%
\pgfpathlineto{\pgfqpoint{2.471172in}{1.588983in}}%
\pgfpathlineto{\pgfqpoint{2.471172in}{1.593241in}}%
\pgfpathlineto{\pgfqpoint{2.475430in}{1.593241in}}%
\pgfpathlineto{\pgfqpoint{2.475430in}{1.588983in}}%
\pgfpathmoveto{\pgfqpoint{2.471172in}{1.593241in}}%
\pgfpathlineto{\pgfqpoint{2.471172in}{1.593241in}}%
\pgfpathlineto{\pgfqpoint{2.471172in}{1.597499in}}%
\pgfpathlineto{\pgfqpoint{2.475430in}{1.597499in}}%
\pgfpathlineto{\pgfqpoint{2.475430in}{1.593241in}}%
\pgfpathmoveto{\pgfqpoint{2.466914in}{1.597499in}}%
\pgfpathlineto{\pgfqpoint{2.466914in}{1.597499in}}%
\pgfpathlineto{\pgfqpoint{2.466914in}{1.601757in}}%
\pgfpathlineto{\pgfqpoint{2.471172in}{1.601757in}}%
\pgfpathlineto{\pgfqpoint{2.471172in}{1.597499in}}%
\pgfpathmoveto{\pgfqpoint{2.466914in}{1.601757in}}%
\pgfpathlineto{\pgfqpoint{2.466914in}{1.601757in}}%
\pgfpathlineto{\pgfqpoint{2.466914in}{1.606015in}}%
\pgfpathlineto{\pgfqpoint{2.471172in}{1.606015in}}%
\pgfpathlineto{\pgfqpoint{2.471172in}{1.601757in}}%
\pgfpathmoveto{\pgfqpoint{2.466914in}{1.606015in}}%
\pgfpathlineto{\pgfqpoint{2.466914in}{1.606015in}}%
\pgfpathlineto{\pgfqpoint{2.466914in}{1.610273in}}%
\pgfpathlineto{\pgfqpoint{2.471172in}{1.610273in}}%
\pgfpathlineto{\pgfqpoint{2.471172in}{1.606015in}}%
\pgfpathmoveto{\pgfqpoint{2.471172in}{1.597499in}}%
\pgfpathlineto{\pgfqpoint{2.471172in}{1.597499in}}%
\pgfpathlineto{\pgfqpoint{2.471172in}{1.601757in}}%
\pgfpathlineto{\pgfqpoint{2.475430in}{1.601757in}}%
\pgfpathlineto{\pgfqpoint{2.475430in}{1.597499in}}%
\pgfpathmoveto{\pgfqpoint{2.471172in}{1.601757in}}%
\pgfpathlineto{\pgfqpoint{2.471172in}{1.601757in}}%
\pgfpathlineto{\pgfqpoint{2.471172in}{1.606015in}}%
\pgfpathlineto{\pgfqpoint{2.475430in}{1.606015in}}%
\pgfpathlineto{\pgfqpoint{2.475430in}{1.601757in}}%
\pgfpathmoveto{\pgfqpoint{2.475430in}{1.601757in}}%
\pgfpathlineto{\pgfqpoint{2.475430in}{1.601757in}}%
\pgfpathlineto{\pgfqpoint{2.475430in}{1.606015in}}%
\pgfpathlineto{\pgfqpoint{2.479687in}{1.606015in}}%
\pgfpathlineto{\pgfqpoint{2.479687in}{1.601757in}}%
\pgfpathmoveto{\pgfqpoint{2.471172in}{1.606015in}}%
\pgfpathlineto{\pgfqpoint{2.471172in}{1.606015in}}%
\pgfpathlineto{\pgfqpoint{2.471172in}{1.610273in}}%
\pgfpathlineto{\pgfqpoint{2.475430in}{1.610273in}}%
\pgfpathlineto{\pgfqpoint{2.475430in}{1.606015in}}%
\pgfpathmoveto{\pgfqpoint{2.471172in}{1.610273in}}%
\pgfpathlineto{\pgfqpoint{2.471172in}{1.610273in}}%
\pgfpathlineto{\pgfqpoint{2.471172in}{1.614531in}}%
\pgfpathlineto{\pgfqpoint{2.475430in}{1.614531in}}%
\pgfpathlineto{\pgfqpoint{2.475430in}{1.610273in}}%
\pgfpathmoveto{\pgfqpoint{2.475430in}{1.606015in}}%
\pgfpathlineto{\pgfqpoint{2.475430in}{1.606015in}}%
\pgfpathlineto{\pgfqpoint{2.475430in}{1.610273in}}%
\pgfpathlineto{\pgfqpoint{2.479687in}{1.610273in}}%
\pgfpathlineto{\pgfqpoint{2.479687in}{1.606015in}}%
\pgfpathmoveto{\pgfqpoint{2.475430in}{1.610273in}}%
\pgfpathlineto{\pgfqpoint{2.475430in}{1.610273in}}%
\pgfpathlineto{\pgfqpoint{2.475430in}{1.614531in}}%
\pgfpathlineto{\pgfqpoint{2.479687in}{1.614531in}}%
\pgfpathlineto{\pgfqpoint{2.479687in}{1.610273in}}%
\pgfpathmoveto{\pgfqpoint{2.471172in}{1.614531in}}%
\pgfpathlineto{\pgfqpoint{2.471172in}{1.614531in}}%
\pgfpathlineto{\pgfqpoint{2.471172in}{1.618788in}}%
\pgfpathlineto{\pgfqpoint{2.475430in}{1.618788in}}%
\pgfpathlineto{\pgfqpoint{2.475430in}{1.614531in}}%
\pgfpathmoveto{\pgfqpoint{2.471172in}{1.618788in}}%
\pgfpathlineto{\pgfqpoint{2.471172in}{1.618788in}}%
\pgfpathlineto{\pgfqpoint{2.471172in}{1.623046in}}%
\pgfpathlineto{\pgfqpoint{2.475430in}{1.623046in}}%
\pgfpathlineto{\pgfqpoint{2.475430in}{1.618788in}}%
\pgfpathmoveto{\pgfqpoint{2.475430in}{1.614531in}}%
\pgfpathlineto{\pgfqpoint{2.475430in}{1.614531in}}%
\pgfpathlineto{\pgfqpoint{2.475430in}{1.618788in}}%
\pgfpathlineto{\pgfqpoint{2.479687in}{1.618788in}}%
\pgfpathlineto{\pgfqpoint{2.479687in}{1.614531in}}%
\pgfpathmoveto{\pgfqpoint{2.475430in}{1.618788in}}%
\pgfpathlineto{\pgfqpoint{2.475430in}{1.618788in}}%
\pgfpathlineto{\pgfqpoint{2.475430in}{1.623046in}}%
\pgfpathlineto{\pgfqpoint{2.479687in}{1.623046in}}%
\pgfpathlineto{\pgfqpoint{2.479687in}{1.618788in}}%
\pgfpathmoveto{\pgfqpoint{2.471172in}{1.623046in}}%
\pgfpathlineto{\pgfqpoint{2.471172in}{1.623046in}}%
\pgfpathlineto{\pgfqpoint{2.471172in}{1.627304in}}%
\pgfpathlineto{\pgfqpoint{2.475430in}{1.627304in}}%
\pgfpathlineto{\pgfqpoint{2.475430in}{1.623046in}}%
\pgfpathmoveto{\pgfqpoint{2.471172in}{1.627304in}}%
\pgfpathlineto{\pgfqpoint{2.471172in}{1.627304in}}%
\pgfpathlineto{\pgfqpoint{2.471172in}{1.631562in}}%
\pgfpathlineto{\pgfqpoint{2.475430in}{1.631562in}}%
\pgfpathlineto{\pgfqpoint{2.475430in}{1.627304in}}%
\pgfpathmoveto{\pgfqpoint{2.475430in}{1.623046in}}%
\pgfpathlineto{\pgfqpoint{2.475430in}{1.623046in}}%
\pgfpathlineto{\pgfqpoint{2.475430in}{1.627304in}}%
\pgfpathlineto{\pgfqpoint{2.479687in}{1.627304in}}%
\pgfpathlineto{\pgfqpoint{2.479687in}{1.623046in}}%
\pgfpathmoveto{\pgfqpoint{2.475430in}{1.627304in}}%
\pgfpathlineto{\pgfqpoint{2.475430in}{1.627304in}}%
\pgfpathlineto{\pgfqpoint{2.475430in}{1.631562in}}%
\pgfpathlineto{\pgfqpoint{2.479687in}{1.631562in}}%
\pgfpathlineto{\pgfqpoint{2.479687in}{1.627304in}}%
\pgfpathmoveto{\pgfqpoint{2.471172in}{1.631562in}}%
\pgfpathlineto{\pgfqpoint{2.471172in}{1.631562in}}%
\pgfpathlineto{\pgfqpoint{2.471172in}{1.635820in}}%
\pgfpathlineto{\pgfqpoint{2.475430in}{1.635820in}}%
\pgfpathlineto{\pgfqpoint{2.475430in}{1.631562in}}%
\pgfpathmoveto{\pgfqpoint{2.471172in}{1.635820in}}%
\pgfpathlineto{\pgfqpoint{2.471172in}{1.635820in}}%
\pgfpathlineto{\pgfqpoint{2.471172in}{1.640078in}}%
\pgfpathlineto{\pgfqpoint{2.475430in}{1.640078in}}%
\pgfpathlineto{\pgfqpoint{2.475430in}{1.635820in}}%
\pgfpathmoveto{\pgfqpoint{2.475430in}{1.631562in}}%
\pgfpathlineto{\pgfqpoint{2.475430in}{1.631562in}}%
\pgfpathlineto{\pgfqpoint{2.475430in}{1.635820in}}%
\pgfpathlineto{\pgfqpoint{2.479687in}{1.635820in}}%
\pgfpathlineto{\pgfqpoint{2.479687in}{1.631562in}}%
\pgfpathmoveto{\pgfqpoint{2.475430in}{1.635820in}}%
\pgfpathlineto{\pgfqpoint{2.475430in}{1.635820in}}%
\pgfpathlineto{\pgfqpoint{2.475430in}{1.640078in}}%
\pgfpathlineto{\pgfqpoint{2.479687in}{1.640078in}}%
\pgfpathlineto{\pgfqpoint{2.479687in}{1.635820in}}%
\pgfpathmoveto{\pgfqpoint{2.471172in}{1.640078in}}%
\pgfpathlineto{\pgfqpoint{2.471172in}{1.640078in}}%
\pgfpathlineto{\pgfqpoint{2.471172in}{1.644336in}}%
\pgfpathlineto{\pgfqpoint{2.475430in}{1.644336in}}%
\pgfpathlineto{\pgfqpoint{2.475430in}{1.640078in}}%
\pgfpathmoveto{\pgfqpoint{2.471172in}{1.644336in}}%
\pgfpathlineto{\pgfqpoint{2.471172in}{1.644336in}}%
\pgfpathlineto{\pgfqpoint{2.471172in}{1.648594in}}%
\pgfpathlineto{\pgfqpoint{2.475430in}{1.648594in}}%
\pgfpathlineto{\pgfqpoint{2.475430in}{1.644336in}}%
\pgfpathmoveto{\pgfqpoint{2.475430in}{1.640078in}}%
\pgfpathlineto{\pgfqpoint{2.475430in}{1.640078in}}%
\pgfpathlineto{\pgfqpoint{2.475430in}{1.644336in}}%
\pgfpathlineto{\pgfqpoint{2.479687in}{1.644336in}}%
\pgfpathlineto{\pgfqpoint{2.479687in}{1.640078in}}%
\pgfpathmoveto{\pgfqpoint{2.475430in}{1.644336in}}%
\pgfpathlineto{\pgfqpoint{2.475430in}{1.644336in}}%
\pgfpathlineto{\pgfqpoint{2.475430in}{1.648594in}}%
\pgfpathlineto{\pgfqpoint{2.479687in}{1.648594in}}%
\pgfpathlineto{\pgfqpoint{2.479687in}{1.644336in}}%
\pgfpathmoveto{\pgfqpoint{2.475430in}{1.648594in}}%
\pgfpathlineto{\pgfqpoint{2.475430in}{1.648594in}}%
\pgfpathlineto{\pgfqpoint{2.475430in}{1.652852in}}%
\pgfpathlineto{\pgfqpoint{2.479687in}{1.652852in}}%
\pgfpathlineto{\pgfqpoint{2.479687in}{1.648594in}}%
\pgfpathmoveto{\pgfqpoint{2.475430in}{1.652852in}}%
\pgfpathlineto{\pgfqpoint{2.475430in}{1.652852in}}%
\pgfpathlineto{\pgfqpoint{2.475430in}{1.657109in}}%
\pgfpathlineto{\pgfqpoint{2.479687in}{1.657109in}}%
\pgfpathlineto{\pgfqpoint{2.479687in}{1.652852in}}%
\pgfpathmoveto{\pgfqpoint{2.475430in}{1.657109in}}%
\pgfpathlineto{\pgfqpoint{2.475430in}{1.657109in}}%
\pgfpathlineto{\pgfqpoint{2.475430in}{1.661367in}}%
\pgfpathlineto{\pgfqpoint{2.479687in}{1.661367in}}%
\pgfpathlineto{\pgfqpoint{2.479687in}{1.657109in}}%
\pgfpathmoveto{\pgfqpoint{2.475430in}{1.661367in}}%
\pgfpathlineto{\pgfqpoint{2.475430in}{1.661367in}}%
\pgfpathlineto{\pgfqpoint{2.475430in}{1.665625in}}%
\pgfpathlineto{\pgfqpoint{2.479687in}{1.665625in}}%
\pgfpathlineto{\pgfqpoint{2.479687in}{1.661367in}}%
\pgfpathmoveto{\pgfqpoint{2.475430in}{1.665625in}}%
\pgfpathlineto{\pgfqpoint{2.475430in}{1.665625in}}%
\pgfpathlineto{\pgfqpoint{2.475430in}{1.669883in}}%
\pgfpathlineto{\pgfqpoint{2.479687in}{1.669883in}}%
\pgfpathlineto{\pgfqpoint{2.479687in}{1.665625in}}%
\pgfpathmoveto{\pgfqpoint{2.475430in}{1.669883in}}%
\pgfpathlineto{\pgfqpoint{2.475430in}{1.669883in}}%
\pgfpathlineto{\pgfqpoint{2.475430in}{1.674141in}}%
\pgfpathlineto{\pgfqpoint{2.479687in}{1.674141in}}%
\pgfpathlineto{\pgfqpoint{2.479687in}{1.669883in}}%
\pgfpathmoveto{\pgfqpoint{2.475430in}{1.674141in}}%
\pgfpathlineto{\pgfqpoint{2.475430in}{1.674141in}}%
\pgfpathlineto{\pgfqpoint{2.475430in}{1.678399in}}%
\pgfpathlineto{\pgfqpoint{2.479687in}{1.678399in}}%
\pgfpathlineto{\pgfqpoint{2.479687in}{1.674141in}}%
\pgfpathmoveto{\pgfqpoint{2.475430in}{1.678399in}}%
\pgfpathlineto{\pgfqpoint{2.475430in}{1.678399in}}%
\pgfpathlineto{\pgfqpoint{2.475430in}{1.682657in}}%
\pgfpathlineto{\pgfqpoint{2.479687in}{1.682657in}}%
\pgfpathlineto{\pgfqpoint{2.479687in}{1.678399in}}%
\pgfpathmoveto{\pgfqpoint{2.475430in}{1.682657in}}%
\pgfpathlineto{\pgfqpoint{2.475430in}{1.682657in}}%
\pgfpathlineto{\pgfqpoint{2.475430in}{1.686915in}}%
\pgfpathlineto{\pgfqpoint{2.479687in}{1.686915in}}%
\pgfpathlineto{\pgfqpoint{2.479687in}{1.682657in}}%
\pgfpathmoveto{\pgfqpoint{2.479687in}{1.640078in}}%
\pgfpathlineto{\pgfqpoint{2.479687in}{1.640078in}}%
\pgfpathlineto{\pgfqpoint{2.479687in}{1.644336in}}%
\pgfpathlineto{\pgfqpoint{2.483945in}{1.644336in}}%
\pgfpathlineto{\pgfqpoint{2.483945in}{1.640078in}}%
\pgfpathmoveto{\pgfqpoint{2.479687in}{1.644336in}}%
\pgfpathlineto{\pgfqpoint{2.479687in}{1.644336in}}%
\pgfpathlineto{\pgfqpoint{2.479687in}{1.648594in}}%
\pgfpathlineto{\pgfqpoint{2.483945in}{1.648594in}}%
\pgfpathlineto{\pgfqpoint{2.483945in}{1.644336in}}%
\pgfpathmoveto{\pgfqpoint{2.479687in}{1.648594in}}%
\pgfpathlineto{\pgfqpoint{2.479687in}{1.648594in}}%
\pgfpathlineto{\pgfqpoint{2.479687in}{1.652852in}}%
\pgfpathlineto{\pgfqpoint{2.483945in}{1.652852in}}%
\pgfpathlineto{\pgfqpoint{2.483945in}{1.648594in}}%
\pgfpathmoveto{\pgfqpoint{2.479687in}{1.652852in}}%
\pgfpathlineto{\pgfqpoint{2.479687in}{1.652852in}}%
\pgfpathlineto{\pgfqpoint{2.479687in}{1.657109in}}%
\pgfpathlineto{\pgfqpoint{2.483945in}{1.657109in}}%
\pgfpathlineto{\pgfqpoint{2.483945in}{1.652852in}}%
\pgfpathmoveto{\pgfqpoint{2.479687in}{1.657109in}}%
\pgfpathlineto{\pgfqpoint{2.479687in}{1.657109in}}%
\pgfpathlineto{\pgfqpoint{2.479687in}{1.661367in}}%
\pgfpathlineto{\pgfqpoint{2.483945in}{1.661367in}}%
\pgfpathlineto{\pgfqpoint{2.483945in}{1.657109in}}%
\pgfpathmoveto{\pgfqpoint{2.479687in}{1.661367in}}%
\pgfpathlineto{\pgfqpoint{2.479687in}{1.661367in}}%
\pgfpathlineto{\pgfqpoint{2.479687in}{1.665625in}}%
\pgfpathlineto{\pgfqpoint{2.483945in}{1.665625in}}%
\pgfpathlineto{\pgfqpoint{2.483945in}{1.661367in}}%
\pgfpathmoveto{\pgfqpoint{2.479687in}{1.665625in}}%
\pgfpathlineto{\pgfqpoint{2.479687in}{1.665625in}}%
\pgfpathlineto{\pgfqpoint{2.479687in}{1.669883in}}%
\pgfpathlineto{\pgfqpoint{2.483945in}{1.669883in}}%
\pgfpathlineto{\pgfqpoint{2.483945in}{1.665625in}}%
\pgfpathmoveto{\pgfqpoint{2.479687in}{1.669883in}}%
\pgfpathlineto{\pgfqpoint{2.479687in}{1.669883in}}%
\pgfpathlineto{\pgfqpoint{2.479687in}{1.674141in}}%
\pgfpathlineto{\pgfqpoint{2.483945in}{1.674141in}}%
\pgfpathlineto{\pgfqpoint{2.483945in}{1.669883in}}%
\pgfpathmoveto{\pgfqpoint{2.479687in}{1.674141in}}%
\pgfpathlineto{\pgfqpoint{2.479687in}{1.674141in}}%
\pgfpathlineto{\pgfqpoint{2.479687in}{1.678399in}}%
\pgfpathlineto{\pgfqpoint{2.483945in}{1.678399in}}%
\pgfpathlineto{\pgfqpoint{2.483945in}{1.674141in}}%
\pgfpathmoveto{\pgfqpoint{2.479687in}{1.678399in}}%
\pgfpathlineto{\pgfqpoint{2.479687in}{1.678399in}}%
\pgfpathlineto{\pgfqpoint{2.479687in}{1.682657in}}%
\pgfpathlineto{\pgfqpoint{2.483945in}{1.682657in}}%
\pgfpathlineto{\pgfqpoint{2.483945in}{1.678399in}}%
\pgfpathmoveto{\pgfqpoint{2.483945in}{1.678399in}}%
\pgfpathlineto{\pgfqpoint{2.483945in}{1.678399in}}%
\pgfpathlineto{\pgfqpoint{2.483945in}{1.682657in}}%
\pgfpathlineto{\pgfqpoint{2.488203in}{1.682657in}}%
\pgfpathlineto{\pgfqpoint{2.488203in}{1.678399in}}%
\pgfpathmoveto{\pgfqpoint{2.479687in}{1.682657in}}%
\pgfpathlineto{\pgfqpoint{2.479687in}{1.682657in}}%
\pgfpathlineto{\pgfqpoint{2.479687in}{1.686915in}}%
\pgfpathlineto{\pgfqpoint{2.483945in}{1.686915in}}%
\pgfpathlineto{\pgfqpoint{2.483945in}{1.682657in}}%
\pgfpathmoveto{\pgfqpoint{2.479687in}{1.686915in}}%
\pgfpathlineto{\pgfqpoint{2.479687in}{1.686915in}}%
\pgfpathlineto{\pgfqpoint{2.479687in}{1.691173in}}%
\pgfpathlineto{\pgfqpoint{2.483945in}{1.691173in}}%
\pgfpathlineto{\pgfqpoint{2.483945in}{1.686915in}}%
\pgfpathmoveto{\pgfqpoint{2.483945in}{1.682657in}}%
\pgfpathlineto{\pgfqpoint{2.483945in}{1.682657in}}%
\pgfpathlineto{\pgfqpoint{2.483945in}{1.686915in}}%
\pgfpathlineto{\pgfqpoint{2.488203in}{1.686915in}}%
\pgfpathlineto{\pgfqpoint{2.488203in}{1.682657in}}%
\pgfpathmoveto{\pgfqpoint{2.483945in}{1.686915in}}%
\pgfpathlineto{\pgfqpoint{2.483945in}{1.686915in}}%
\pgfpathlineto{\pgfqpoint{2.483945in}{1.691173in}}%
\pgfpathlineto{\pgfqpoint{2.488203in}{1.691173in}}%
\pgfpathlineto{\pgfqpoint{2.488203in}{1.686915in}}%
\pgfpathmoveto{\pgfqpoint{2.479687in}{1.691173in}}%
\pgfpathlineto{\pgfqpoint{2.479687in}{1.691173in}}%
\pgfpathlineto{\pgfqpoint{2.479687in}{1.695430in}}%
\pgfpathlineto{\pgfqpoint{2.483945in}{1.695430in}}%
\pgfpathlineto{\pgfqpoint{2.483945in}{1.691173in}}%
\pgfpathmoveto{\pgfqpoint{2.479687in}{1.695430in}}%
\pgfpathlineto{\pgfqpoint{2.479687in}{1.695430in}}%
\pgfpathlineto{\pgfqpoint{2.479687in}{1.699688in}}%
\pgfpathlineto{\pgfqpoint{2.483945in}{1.699688in}}%
\pgfpathlineto{\pgfqpoint{2.483945in}{1.695430in}}%
\pgfpathmoveto{\pgfqpoint{2.483945in}{1.691173in}}%
\pgfpathlineto{\pgfqpoint{2.483945in}{1.691173in}}%
\pgfpathlineto{\pgfqpoint{2.483945in}{1.695430in}}%
\pgfpathlineto{\pgfqpoint{2.488203in}{1.695430in}}%
\pgfpathlineto{\pgfqpoint{2.488203in}{1.691173in}}%
\pgfpathmoveto{\pgfqpoint{2.483945in}{1.695430in}}%
\pgfpathlineto{\pgfqpoint{2.483945in}{1.695430in}}%
\pgfpathlineto{\pgfqpoint{2.483945in}{1.699688in}}%
\pgfpathlineto{\pgfqpoint{2.488203in}{1.699688in}}%
\pgfpathlineto{\pgfqpoint{2.488203in}{1.695430in}}%
\pgfpathmoveto{\pgfqpoint{2.479687in}{1.699688in}}%
\pgfpathlineto{\pgfqpoint{2.479687in}{1.699688in}}%
\pgfpathlineto{\pgfqpoint{2.479687in}{1.703946in}}%
\pgfpathlineto{\pgfqpoint{2.483945in}{1.703946in}}%
\pgfpathlineto{\pgfqpoint{2.483945in}{1.699688in}}%
\pgfpathmoveto{\pgfqpoint{2.479687in}{1.703946in}}%
\pgfpathlineto{\pgfqpoint{2.479687in}{1.703946in}}%
\pgfpathlineto{\pgfqpoint{2.479687in}{1.708204in}}%
\pgfpathlineto{\pgfqpoint{2.483945in}{1.708204in}}%
\pgfpathlineto{\pgfqpoint{2.483945in}{1.703946in}}%
\pgfpathmoveto{\pgfqpoint{2.483945in}{1.699688in}}%
\pgfpathlineto{\pgfqpoint{2.483945in}{1.699688in}}%
\pgfpathlineto{\pgfqpoint{2.483945in}{1.703946in}}%
\pgfpathlineto{\pgfqpoint{2.488203in}{1.703946in}}%
\pgfpathlineto{\pgfqpoint{2.488203in}{1.699688in}}%
\pgfpathmoveto{\pgfqpoint{2.483945in}{1.703946in}}%
\pgfpathlineto{\pgfqpoint{2.483945in}{1.703946in}}%
\pgfpathlineto{\pgfqpoint{2.483945in}{1.708204in}}%
\pgfpathlineto{\pgfqpoint{2.488203in}{1.708204in}}%
\pgfpathlineto{\pgfqpoint{2.488203in}{1.703946in}}%
\pgfpathmoveto{\pgfqpoint{2.479687in}{1.708204in}}%
\pgfpathlineto{\pgfqpoint{2.479687in}{1.708204in}}%
\pgfpathlineto{\pgfqpoint{2.479687in}{1.712462in}}%
\pgfpathlineto{\pgfqpoint{2.483945in}{1.712462in}}%
\pgfpathlineto{\pgfqpoint{2.483945in}{1.708204in}}%
\pgfpathmoveto{\pgfqpoint{2.479687in}{1.712462in}}%
\pgfpathlineto{\pgfqpoint{2.479687in}{1.712462in}}%
\pgfpathlineto{\pgfqpoint{2.479687in}{1.716720in}}%
\pgfpathlineto{\pgfqpoint{2.483945in}{1.716720in}}%
\pgfpathlineto{\pgfqpoint{2.483945in}{1.712462in}}%
\pgfpathmoveto{\pgfqpoint{2.483945in}{1.708204in}}%
\pgfpathlineto{\pgfqpoint{2.483945in}{1.708204in}}%
\pgfpathlineto{\pgfqpoint{2.483945in}{1.712462in}}%
\pgfpathlineto{\pgfqpoint{2.488203in}{1.712462in}}%
\pgfpathlineto{\pgfqpoint{2.488203in}{1.708204in}}%
\pgfpathmoveto{\pgfqpoint{2.483945in}{1.712462in}}%
\pgfpathlineto{\pgfqpoint{2.483945in}{1.712462in}}%
\pgfpathlineto{\pgfqpoint{2.483945in}{1.716720in}}%
\pgfpathlineto{\pgfqpoint{2.488203in}{1.716720in}}%
\pgfpathlineto{\pgfqpoint{2.488203in}{1.712462in}}%
\pgfpathmoveto{\pgfqpoint{2.488203in}{1.712462in}}%
\pgfpathlineto{\pgfqpoint{2.488203in}{1.712462in}}%
\pgfpathlineto{\pgfqpoint{2.488203in}{1.716720in}}%
\pgfpathlineto{\pgfqpoint{2.492461in}{1.716720in}}%
\pgfpathlineto{\pgfqpoint{2.492461in}{1.712462in}}%
\pgfpathmoveto{\pgfqpoint{2.479687in}{1.716720in}}%
\pgfpathlineto{\pgfqpoint{2.479687in}{1.716720in}}%
\pgfpathlineto{\pgfqpoint{2.479687in}{1.720978in}}%
\pgfpathlineto{\pgfqpoint{2.483945in}{1.720978in}}%
\pgfpathlineto{\pgfqpoint{2.483945in}{1.716720in}}%
\pgfpathmoveto{\pgfqpoint{2.479687in}{1.720978in}}%
\pgfpathlineto{\pgfqpoint{2.479687in}{1.720978in}}%
\pgfpathlineto{\pgfqpoint{2.479687in}{1.725236in}}%
\pgfpathlineto{\pgfqpoint{2.483945in}{1.725236in}}%
\pgfpathlineto{\pgfqpoint{2.483945in}{1.720978in}}%
\pgfpathmoveto{\pgfqpoint{2.483945in}{1.716720in}}%
\pgfpathlineto{\pgfqpoint{2.483945in}{1.716720in}}%
\pgfpathlineto{\pgfqpoint{2.483945in}{1.720978in}}%
\pgfpathlineto{\pgfqpoint{2.488203in}{1.720978in}}%
\pgfpathlineto{\pgfqpoint{2.488203in}{1.716720in}}%
\pgfpathmoveto{\pgfqpoint{2.483945in}{1.720978in}}%
\pgfpathlineto{\pgfqpoint{2.483945in}{1.720978in}}%
\pgfpathlineto{\pgfqpoint{2.483945in}{1.725236in}}%
\pgfpathlineto{\pgfqpoint{2.488203in}{1.725236in}}%
\pgfpathlineto{\pgfqpoint{2.488203in}{1.720978in}}%
\pgfpathmoveto{\pgfqpoint{2.483945in}{1.725236in}}%
\pgfpathlineto{\pgfqpoint{2.483945in}{1.725236in}}%
\pgfpathlineto{\pgfqpoint{2.483945in}{1.729494in}}%
\pgfpathlineto{\pgfqpoint{2.488203in}{1.729494in}}%
\pgfpathlineto{\pgfqpoint{2.488203in}{1.725236in}}%
\pgfpathmoveto{\pgfqpoint{2.483945in}{1.729494in}}%
\pgfpathlineto{\pgfqpoint{2.483945in}{1.729494in}}%
\pgfpathlineto{\pgfqpoint{2.483945in}{1.733751in}}%
\pgfpathlineto{\pgfqpoint{2.488203in}{1.733751in}}%
\pgfpathlineto{\pgfqpoint{2.488203in}{1.729494in}}%
\pgfpathmoveto{\pgfqpoint{2.488203in}{1.716720in}}%
\pgfpathlineto{\pgfqpoint{2.488203in}{1.716720in}}%
\pgfpathlineto{\pgfqpoint{2.488203in}{1.720978in}}%
\pgfpathlineto{\pgfqpoint{2.492461in}{1.720978in}}%
\pgfpathlineto{\pgfqpoint{2.492461in}{1.716720in}}%
\pgfpathmoveto{\pgfqpoint{2.488203in}{1.720978in}}%
\pgfpathlineto{\pgfqpoint{2.488203in}{1.720978in}}%
\pgfpathlineto{\pgfqpoint{2.488203in}{1.725236in}}%
\pgfpathlineto{\pgfqpoint{2.492461in}{1.725236in}}%
\pgfpathlineto{\pgfqpoint{2.492461in}{1.720978in}}%
\pgfpathmoveto{\pgfqpoint{2.488203in}{1.725236in}}%
\pgfpathlineto{\pgfqpoint{2.488203in}{1.725236in}}%
\pgfpathlineto{\pgfqpoint{2.488203in}{1.729494in}}%
\pgfpathlineto{\pgfqpoint{2.492461in}{1.729494in}}%
\pgfpathlineto{\pgfqpoint{2.492461in}{1.725236in}}%
\pgfpathmoveto{\pgfqpoint{2.488203in}{1.729494in}}%
\pgfpathlineto{\pgfqpoint{2.488203in}{1.729494in}}%
\pgfpathlineto{\pgfqpoint{2.488203in}{1.733751in}}%
\pgfpathlineto{\pgfqpoint{2.492461in}{1.733751in}}%
\pgfpathlineto{\pgfqpoint{2.492461in}{1.729494in}}%
\pgfpathmoveto{\pgfqpoint{2.483945in}{1.733751in}}%
\pgfpathlineto{\pgfqpoint{2.483945in}{1.733751in}}%
\pgfpathlineto{\pgfqpoint{2.483945in}{1.738009in}}%
\pgfpathlineto{\pgfqpoint{2.488203in}{1.738009in}}%
\pgfpathlineto{\pgfqpoint{2.488203in}{1.733751in}}%
\pgfpathmoveto{\pgfqpoint{2.483945in}{1.738009in}}%
\pgfpathlineto{\pgfqpoint{2.483945in}{1.738009in}}%
\pgfpathlineto{\pgfqpoint{2.483945in}{1.742267in}}%
\pgfpathlineto{\pgfqpoint{2.488203in}{1.742267in}}%
\pgfpathlineto{\pgfqpoint{2.488203in}{1.738009in}}%
\pgfpathmoveto{\pgfqpoint{2.483945in}{1.742267in}}%
\pgfpathlineto{\pgfqpoint{2.483945in}{1.742267in}}%
\pgfpathlineto{\pgfqpoint{2.483945in}{1.746525in}}%
\pgfpathlineto{\pgfqpoint{2.488203in}{1.746525in}}%
\pgfpathlineto{\pgfqpoint{2.488203in}{1.742267in}}%
\pgfpathmoveto{\pgfqpoint{2.483945in}{1.746525in}}%
\pgfpathlineto{\pgfqpoint{2.483945in}{1.746525in}}%
\pgfpathlineto{\pgfqpoint{2.483945in}{1.750783in}}%
\pgfpathlineto{\pgfqpoint{2.488203in}{1.750783in}}%
\pgfpathlineto{\pgfqpoint{2.488203in}{1.746525in}}%
\pgfpathmoveto{\pgfqpoint{2.488203in}{1.733751in}}%
\pgfpathlineto{\pgfqpoint{2.488203in}{1.733751in}}%
\pgfpathlineto{\pgfqpoint{2.488203in}{1.738009in}}%
\pgfpathlineto{\pgfqpoint{2.492461in}{1.738009in}}%
\pgfpathlineto{\pgfqpoint{2.492461in}{1.733751in}}%
\pgfpathmoveto{\pgfqpoint{2.488203in}{1.738009in}}%
\pgfpathlineto{\pgfqpoint{2.488203in}{1.738009in}}%
\pgfpathlineto{\pgfqpoint{2.488203in}{1.742267in}}%
\pgfpathlineto{\pgfqpoint{2.492461in}{1.742267in}}%
\pgfpathlineto{\pgfqpoint{2.492461in}{1.738009in}}%
\pgfpathmoveto{\pgfqpoint{2.488203in}{1.742267in}}%
\pgfpathlineto{\pgfqpoint{2.488203in}{1.742267in}}%
\pgfpathlineto{\pgfqpoint{2.488203in}{1.746525in}}%
\pgfpathlineto{\pgfqpoint{2.492461in}{1.746525in}}%
\pgfpathlineto{\pgfqpoint{2.492461in}{1.742267in}}%
\pgfpathmoveto{\pgfqpoint{2.488203in}{1.746525in}}%
\pgfpathlineto{\pgfqpoint{2.488203in}{1.746525in}}%
\pgfpathlineto{\pgfqpoint{2.488203in}{1.750783in}}%
\pgfpathlineto{\pgfqpoint{2.492461in}{1.750783in}}%
\pgfpathlineto{\pgfqpoint{2.492461in}{1.746525in}}%
\pgfpathmoveto{\pgfqpoint{2.483945in}{1.750783in}}%
\pgfpathlineto{\pgfqpoint{2.483945in}{1.750783in}}%
\pgfpathlineto{\pgfqpoint{2.483945in}{1.755040in}}%
\pgfpathlineto{\pgfqpoint{2.488203in}{1.755040in}}%
\pgfpathlineto{\pgfqpoint{2.488203in}{1.750783in}}%
\pgfpathmoveto{\pgfqpoint{2.483945in}{1.755040in}}%
\pgfpathlineto{\pgfqpoint{2.483945in}{1.755040in}}%
\pgfpathlineto{\pgfqpoint{2.483945in}{1.759298in}}%
\pgfpathlineto{\pgfqpoint{2.488203in}{1.759298in}}%
\pgfpathlineto{\pgfqpoint{2.488203in}{1.755040in}}%
\pgfpathmoveto{\pgfqpoint{2.488203in}{1.750783in}}%
\pgfpathlineto{\pgfqpoint{2.488203in}{1.750783in}}%
\pgfpathlineto{\pgfqpoint{2.488203in}{1.755040in}}%
\pgfpathlineto{\pgfqpoint{2.492461in}{1.755040in}}%
\pgfpathlineto{\pgfqpoint{2.492461in}{1.750783in}}%
\pgfpathmoveto{\pgfqpoint{2.488203in}{1.755040in}}%
\pgfpathlineto{\pgfqpoint{2.488203in}{1.755040in}}%
\pgfpathlineto{\pgfqpoint{2.488203in}{1.759298in}}%
\pgfpathlineto{\pgfqpoint{2.492461in}{1.759298in}}%
\pgfpathlineto{\pgfqpoint{2.492461in}{1.755040in}}%
\pgfpathmoveto{\pgfqpoint{2.492461in}{1.750783in}}%
\pgfpathlineto{\pgfqpoint{2.492461in}{1.750783in}}%
\pgfpathlineto{\pgfqpoint{2.492461in}{1.755040in}}%
\pgfpathlineto{\pgfqpoint{2.496718in}{1.755040in}}%
\pgfpathlineto{\pgfqpoint{2.496718in}{1.750783in}}%
\pgfpathmoveto{\pgfqpoint{2.492461in}{1.755040in}}%
\pgfpathlineto{\pgfqpoint{2.492461in}{1.755040in}}%
\pgfpathlineto{\pgfqpoint{2.492461in}{1.759298in}}%
\pgfpathlineto{\pgfqpoint{2.496718in}{1.759298in}}%
\pgfpathlineto{\pgfqpoint{2.496718in}{1.755040in}}%
\pgfpathmoveto{\pgfqpoint{2.488203in}{1.759298in}}%
\pgfpathlineto{\pgfqpoint{2.488203in}{1.759298in}}%
\pgfpathlineto{\pgfqpoint{2.488203in}{1.763556in}}%
\pgfpathlineto{\pgfqpoint{2.492461in}{1.763556in}}%
\pgfpathlineto{\pgfqpoint{2.492461in}{1.759298in}}%
\pgfpathmoveto{\pgfqpoint{2.488203in}{1.763556in}}%
\pgfpathlineto{\pgfqpoint{2.488203in}{1.763556in}}%
\pgfpathlineto{\pgfqpoint{2.488203in}{1.767814in}}%
\pgfpathlineto{\pgfqpoint{2.492461in}{1.767814in}}%
\pgfpathlineto{\pgfqpoint{2.492461in}{1.763556in}}%
\pgfpathmoveto{\pgfqpoint{2.492461in}{1.759298in}}%
\pgfpathlineto{\pgfqpoint{2.492461in}{1.759298in}}%
\pgfpathlineto{\pgfqpoint{2.492461in}{1.763556in}}%
\pgfpathlineto{\pgfqpoint{2.496718in}{1.763556in}}%
\pgfpathlineto{\pgfqpoint{2.496718in}{1.759298in}}%
\pgfpathmoveto{\pgfqpoint{2.492461in}{1.763556in}}%
\pgfpathlineto{\pgfqpoint{2.492461in}{1.763556in}}%
\pgfpathlineto{\pgfqpoint{2.492461in}{1.767814in}}%
\pgfpathlineto{\pgfqpoint{2.496718in}{1.767814in}}%
\pgfpathlineto{\pgfqpoint{2.496718in}{1.763556in}}%
\pgfpathmoveto{\pgfqpoint{2.488203in}{1.767814in}}%
\pgfpathlineto{\pgfqpoint{2.488203in}{1.767814in}}%
\pgfpathlineto{\pgfqpoint{2.488203in}{1.772071in}}%
\pgfpathlineto{\pgfqpoint{2.492461in}{1.772071in}}%
\pgfpathlineto{\pgfqpoint{2.492461in}{1.767814in}}%
\pgfpathmoveto{\pgfqpoint{2.488203in}{1.772071in}}%
\pgfpathlineto{\pgfqpoint{2.488203in}{1.772071in}}%
\pgfpathlineto{\pgfqpoint{2.488203in}{1.776329in}}%
\pgfpathlineto{\pgfqpoint{2.492461in}{1.776329in}}%
\pgfpathlineto{\pgfqpoint{2.492461in}{1.772071in}}%
\pgfpathmoveto{\pgfqpoint{2.492461in}{1.767814in}}%
\pgfpathlineto{\pgfqpoint{2.492461in}{1.767814in}}%
\pgfpathlineto{\pgfqpoint{2.492461in}{1.772071in}}%
\pgfpathlineto{\pgfqpoint{2.496718in}{1.772071in}}%
\pgfpathlineto{\pgfqpoint{2.496718in}{1.767814in}}%
\pgfpathmoveto{\pgfqpoint{2.492461in}{1.772071in}}%
\pgfpathlineto{\pgfqpoint{2.492461in}{1.772071in}}%
\pgfpathlineto{\pgfqpoint{2.492461in}{1.776329in}}%
\pgfpathlineto{\pgfqpoint{2.496718in}{1.776329in}}%
\pgfpathlineto{\pgfqpoint{2.496718in}{1.772071in}}%
\pgfpathmoveto{\pgfqpoint{2.488203in}{1.776329in}}%
\pgfpathlineto{\pgfqpoint{2.488203in}{1.776329in}}%
\pgfpathlineto{\pgfqpoint{2.488203in}{1.780587in}}%
\pgfpathlineto{\pgfqpoint{2.492461in}{1.780587in}}%
\pgfpathlineto{\pgfqpoint{2.492461in}{1.776329in}}%
\pgfpathmoveto{\pgfqpoint{2.488203in}{1.780587in}}%
\pgfpathlineto{\pgfqpoint{2.488203in}{1.780587in}}%
\pgfpathlineto{\pgfqpoint{2.488203in}{1.784845in}}%
\pgfpathlineto{\pgfqpoint{2.492461in}{1.784845in}}%
\pgfpathlineto{\pgfqpoint{2.492461in}{1.780587in}}%
\pgfpathmoveto{\pgfqpoint{2.492461in}{1.776329in}}%
\pgfpathlineto{\pgfqpoint{2.492461in}{1.776329in}}%
\pgfpathlineto{\pgfqpoint{2.492461in}{1.780587in}}%
\pgfpathlineto{\pgfqpoint{2.496718in}{1.780587in}}%
\pgfpathlineto{\pgfqpoint{2.496718in}{1.776329in}}%
\pgfpathmoveto{\pgfqpoint{2.492461in}{1.780587in}}%
\pgfpathlineto{\pgfqpoint{2.492461in}{1.780587in}}%
\pgfpathlineto{\pgfqpoint{2.492461in}{1.784845in}}%
\pgfpathlineto{\pgfqpoint{2.496718in}{1.784845in}}%
\pgfpathlineto{\pgfqpoint{2.496718in}{1.780587in}}%
\pgfpathmoveto{\pgfqpoint{2.488203in}{1.784845in}}%
\pgfpathlineto{\pgfqpoint{2.488203in}{1.784845in}}%
\pgfpathlineto{\pgfqpoint{2.488203in}{1.789102in}}%
\pgfpathlineto{\pgfqpoint{2.492461in}{1.789102in}}%
\pgfpathlineto{\pgfqpoint{2.492461in}{1.784845in}}%
\pgfpathmoveto{\pgfqpoint{2.488203in}{1.789102in}}%
\pgfpathlineto{\pgfqpoint{2.488203in}{1.789102in}}%
\pgfpathlineto{\pgfqpoint{2.488203in}{1.793360in}}%
\pgfpathlineto{\pgfqpoint{2.492461in}{1.793360in}}%
\pgfpathlineto{\pgfqpoint{2.492461in}{1.789102in}}%
\pgfpathmoveto{\pgfqpoint{2.492461in}{1.784845in}}%
\pgfpathlineto{\pgfqpoint{2.492461in}{1.784845in}}%
\pgfpathlineto{\pgfqpoint{2.492461in}{1.789102in}}%
\pgfpathlineto{\pgfqpoint{2.496718in}{1.789102in}}%
\pgfpathlineto{\pgfqpoint{2.496718in}{1.784845in}}%
\pgfpathmoveto{\pgfqpoint{2.492461in}{1.789102in}}%
\pgfpathlineto{\pgfqpoint{2.492461in}{1.789102in}}%
\pgfpathlineto{\pgfqpoint{2.492461in}{1.793360in}}%
\pgfpathlineto{\pgfqpoint{2.496718in}{1.793360in}}%
\pgfpathlineto{\pgfqpoint{2.496718in}{1.789102in}}%
\pgfpathmoveto{\pgfqpoint{2.488203in}{1.793360in}}%
\pgfpathlineto{\pgfqpoint{2.488203in}{1.793360in}}%
\pgfpathlineto{\pgfqpoint{2.488203in}{1.797618in}}%
\pgfpathlineto{\pgfqpoint{2.492461in}{1.797618in}}%
\pgfpathlineto{\pgfqpoint{2.492461in}{1.793360in}}%
\pgfpathmoveto{\pgfqpoint{2.492461in}{1.793360in}}%
\pgfpathlineto{\pgfqpoint{2.492461in}{1.793360in}}%
\pgfpathlineto{\pgfqpoint{2.492461in}{1.797618in}}%
\pgfpathlineto{\pgfqpoint{2.496718in}{1.797618in}}%
\pgfpathlineto{\pgfqpoint{2.496718in}{1.793360in}}%
\pgfpathmoveto{\pgfqpoint{2.492461in}{1.797618in}}%
\pgfpathlineto{\pgfqpoint{2.492461in}{1.797618in}}%
\pgfpathlineto{\pgfqpoint{2.492461in}{1.801876in}}%
\pgfpathlineto{\pgfqpoint{2.496718in}{1.801876in}}%
\pgfpathlineto{\pgfqpoint{2.496718in}{1.797618in}}%
\pgfpathmoveto{\pgfqpoint{2.496718in}{1.784845in}}%
\pgfpathlineto{\pgfqpoint{2.496718in}{1.784845in}}%
\pgfpathlineto{\pgfqpoint{2.496718in}{1.789102in}}%
\pgfpathlineto{\pgfqpoint{2.500976in}{1.789102in}}%
\pgfpathlineto{\pgfqpoint{2.500976in}{1.784845in}}%
\pgfpathmoveto{\pgfqpoint{2.496718in}{1.789102in}}%
\pgfpathlineto{\pgfqpoint{2.496718in}{1.789102in}}%
\pgfpathlineto{\pgfqpoint{2.496718in}{1.793360in}}%
\pgfpathlineto{\pgfqpoint{2.500976in}{1.793360in}}%
\pgfpathlineto{\pgfqpoint{2.500976in}{1.789102in}}%
\pgfpathmoveto{\pgfqpoint{2.496718in}{1.793360in}}%
\pgfpathlineto{\pgfqpoint{2.496718in}{1.793360in}}%
\pgfpathlineto{\pgfqpoint{2.496718in}{1.797618in}}%
\pgfpathlineto{\pgfqpoint{2.500976in}{1.797618in}}%
\pgfpathlineto{\pgfqpoint{2.500976in}{1.793360in}}%
\pgfpathmoveto{\pgfqpoint{2.496718in}{1.797618in}}%
\pgfpathlineto{\pgfqpoint{2.496718in}{1.797618in}}%
\pgfpathlineto{\pgfqpoint{2.496718in}{1.801876in}}%
\pgfpathlineto{\pgfqpoint{2.500976in}{1.801876in}}%
\pgfpathlineto{\pgfqpoint{2.500976in}{1.797618in}}%
\pgfpathmoveto{\pgfqpoint{2.492461in}{1.801876in}}%
\pgfpathlineto{\pgfqpoint{2.492461in}{1.801876in}}%
\pgfpathlineto{\pgfqpoint{2.492461in}{1.806134in}}%
\pgfpathlineto{\pgfqpoint{2.496718in}{1.806134in}}%
\pgfpathlineto{\pgfqpoint{2.496718in}{1.801876in}}%
\pgfpathmoveto{\pgfqpoint{2.492461in}{1.806134in}}%
\pgfpathlineto{\pgfqpoint{2.492461in}{1.806134in}}%
\pgfpathlineto{\pgfqpoint{2.492461in}{1.810391in}}%
\pgfpathlineto{\pgfqpoint{2.496718in}{1.810391in}}%
\pgfpathlineto{\pgfqpoint{2.496718in}{1.806134in}}%
\pgfpathmoveto{\pgfqpoint{2.492461in}{1.810391in}}%
\pgfpathlineto{\pgfqpoint{2.492461in}{1.810391in}}%
\pgfpathlineto{\pgfqpoint{2.492461in}{1.814649in}}%
\pgfpathlineto{\pgfqpoint{2.496718in}{1.814649in}}%
\pgfpathlineto{\pgfqpoint{2.496718in}{1.810391in}}%
\pgfpathmoveto{\pgfqpoint{2.492461in}{1.814649in}}%
\pgfpathlineto{\pgfqpoint{2.492461in}{1.814649in}}%
\pgfpathlineto{\pgfqpoint{2.492461in}{1.818907in}}%
\pgfpathlineto{\pgfqpoint{2.496718in}{1.818907in}}%
\pgfpathlineto{\pgfqpoint{2.496718in}{1.814649in}}%
\pgfpathmoveto{\pgfqpoint{2.492461in}{1.818907in}}%
\pgfpathlineto{\pgfqpoint{2.492461in}{1.818907in}}%
\pgfpathlineto{\pgfqpoint{2.492461in}{1.823165in}}%
\pgfpathlineto{\pgfqpoint{2.496718in}{1.823165in}}%
\pgfpathlineto{\pgfqpoint{2.496718in}{1.818907in}}%
\pgfpathmoveto{\pgfqpoint{2.492461in}{1.823165in}}%
\pgfpathlineto{\pgfqpoint{2.492461in}{1.823165in}}%
\pgfpathlineto{\pgfqpoint{2.492461in}{1.827422in}}%
\pgfpathlineto{\pgfqpoint{2.496718in}{1.827422in}}%
\pgfpathlineto{\pgfqpoint{2.496718in}{1.823165in}}%
\pgfpathmoveto{\pgfqpoint{2.492461in}{1.827422in}}%
\pgfpathlineto{\pgfqpoint{2.492461in}{1.827422in}}%
\pgfpathlineto{\pgfqpoint{2.492461in}{1.831680in}}%
\pgfpathlineto{\pgfqpoint{2.496718in}{1.831680in}}%
\pgfpathlineto{\pgfqpoint{2.496718in}{1.827422in}}%
\pgfpathmoveto{\pgfqpoint{2.496718in}{1.801876in}}%
\pgfpathlineto{\pgfqpoint{2.496718in}{1.801876in}}%
\pgfpathlineto{\pgfqpoint{2.496718in}{1.806134in}}%
\pgfpathlineto{\pgfqpoint{2.500976in}{1.806134in}}%
\pgfpathlineto{\pgfqpoint{2.500976in}{1.801876in}}%
\pgfpathmoveto{\pgfqpoint{2.496718in}{1.806134in}}%
\pgfpathlineto{\pgfqpoint{2.496718in}{1.806134in}}%
\pgfpathlineto{\pgfqpoint{2.496718in}{1.810391in}}%
\pgfpathlineto{\pgfqpoint{2.500976in}{1.810391in}}%
\pgfpathlineto{\pgfqpoint{2.500976in}{1.806134in}}%
\pgfpathmoveto{\pgfqpoint{2.496718in}{1.810391in}}%
\pgfpathlineto{\pgfqpoint{2.496718in}{1.810391in}}%
\pgfpathlineto{\pgfqpoint{2.496718in}{1.814649in}}%
\pgfpathlineto{\pgfqpoint{2.500976in}{1.814649in}}%
\pgfpathlineto{\pgfqpoint{2.500976in}{1.810391in}}%
\pgfpathmoveto{\pgfqpoint{2.496718in}{1.814649in}}%
\pgfpathlineto{\pgfqpoint{2.496718in}{1.814649in}}%
\pgfpathlineto{\pgfqpoint{2.496718in}{1.818907in}}%
\pgfpathlineto{\pgfqpoint{2.500976in}{1.818907in}}%
\pgfpathlineto{\pgfqpoint{2.500976in}{1.814649in}}%
\pgfpathmoveto{\pgfqpoint{2.496718in}{1.818907in}}%
\pgfpathlineto{\pgfqpoint{2.496718in}{1.818907in}}%
\pgfpathlineto{\pgfqpoint{2.496718in}{1.823165in}}%
\pgfpathlineto{\pgfqpoint{2.500976in}{1.823165in}}%
\pgfpathlineto{\pgfqpoint{2.500976in}{1.818907in}}%
\pgfpathmoveto{\pgfqpoint{2.496718in}{1.823165in}}%
\pgfpathlineto{\pgfqpoint{2.496718in}{1.823165in}}%
\pgfpathlineto{\pgfqpoint{2.496718in}{1.827422in}}%
\pgfpathlineto{\pgfqpoint{2.500976in}{1.827422in}}%
\pgfpathlineto{\pgfqpoint{2.500976in}{1.823165in}}%
\pgfpathmoveto{\pgfqpoint{2.500976in}{1.823165in}}%
\pgfpathlineto{\pgfqpoint{2.500976in}{1.823165in}}%
\pgfpathlineto{\pgfqpoint{2.500976in}{1.827422in}}%
\pgfpathlineto{\pgfqpoint{2.505234in}{1.827422in}}%
\pgfpathlineto{\pgfqpoint{2.505234in}{1.823165in}}%
\pgfpathmoveto{\pgfqpoint{2.496718in}{1.827422in}}%
\pgfpathlineto{\pgfqpoint{2.496718in}{1.827422in}}%
\pgfpathlineto{\pgfqpoint{2.496718in}{1.831680in}}%
\pgfpathlineto{\pgfqpoint{2.500976in}{1.831680in}}%
\pgfpathlineto{\pgfqpoint{2.500976in}{1.827422in}}%
\pgfpathmoveto{\pgfqpoint{2.496718in}{1.831680in}}%
\pgfpathlineto{\pgfqpoint{2.496718in}{1.831680in}}%
\pgfpathlineto{\pgfqpoint{2.496718in}{1.835938in}}%
\pgfpathlineto{\pgfqpoint{2.500976in}{1.835938in}}%
\pgfpathlineto{\pgfqpoint{2.500976in}{1.831680in}}%
\pgfpathmoveto{\pgfqpoint{2.500976in}{1.827422in}}%
\pgfpathlineto{\pgfqpoint{2.500976in}{1.827422in}}%
\pgfpathlineto{\pgfqpoint{2.500976in}{1.831680in}}%
\pgfpathlineto{\pgfqpoint{2.505234in}{1.831680in}}%
\pgfpathlineto{\pgfqpoint{2.505234in}{1.827422in}}%
\pgfpathmoveto{\pgfqpoint{2.500976in}{1.831680in}}%
\pgfpathlineto{\pgfqpoint{2.500976in}{1.831680in}}%
\pgfpathlineto{\pgfqpoint{2.500976in}{1.835938in}}%
\pgfpathlineto{\pgfqpoint{2.505234in}{1.835938in}}%
\pgfpathlineto{\pgfqpoint{2.505234in}{1.831680in}}%
\pgfpathmoveto{\pgfqpoint{2.496718in}{1.835938in}}%
\pgfpathlineto{\pgfqpoint{2.496718in}{1.835938in}}%
\pgfpathlineto{\pgfqpoint{2.496718in}{1.840196in}}%
\pgfpathlineto{\pgfqpoint{2.500976in}{1.840196in}}%
\pgfpathlineto{\pgfqpoint{2.500976in}{1.835938in}}%
\pgfpathmoveto{\pgfqpoint{2.496718in}{1.840196in}}%
\pgfpathlineto{\pgfqpoint{2.496718in}{1.840196in}}%
\pgfpathlineto{\pgfqpoint{2.496718in}{1.844454in}}%
\pgfpathlineto{\pgfqpoint{2.500976in}{1.844454in}}%
\pgfpathlineto{\pgfqpoint{2.500976in}{1.840196in}}%
\pgfpathmoveto{\pgfqpoint{2.500976in}{1.835938in}}%
\pgfpathlineto{\pgfqpoint{2.500976in}{1.835938in}}%
\pgfpathlineto{\pgfqpoint{2.500976in}{1.840196in}}%
\pgfpathlineto{\pgfqpoint{2.505234in}{1.840196in}}%
\pgfpathlineto{\pgfqpoint{2.505234in}{1.835938in}}%
\pgfpathmoveto{\pgfqpoint{2.500976in}{1.840196in}}%
\pgfpathlineto{\pgfqpoint{2.500976in}{1.840196in}}%
\pgfpathlineto{\pgfqpoint{2.500976in}{1.844454in}}%
\pgfpathlineto{\pgfqpoint{2.505234in}{1.844454in}}%
\pgfpathlineto{\pgfqpoint{2.505234in}{1.840196in}}%
\pgfpathmoveto{\pgfqpoint{2.496718in}{1.844454in}}%
\pgfpathlineto{\pgfqpoint{2.496718in}{1.844454in}}%
\pgfpathlineto{\pgfqpoint{2.496718in}{1.848711in}}%
\pgfpathlineto{\pgfqpoint{2.500976in}{1.848711in}}%
\pgfpathlineto{\pgfqpoint{2.500976in}{1.844454in}}%
\pgfpathmoveto{\pgfqpoint{2.496718in}{1.848711in}}%
\pgfpathlineto{\pgfqpoint{2.496718in}{1.848711in}}%
\pgfpathlineto{\pgfqpoint{2.496718in}{1.852969in}}%
\pgfpathlineto{\pgfqpoint{2.500976in}{1.852969in}}%
\pgfpathlineto{\pgfqpoint{2.500976in}{1.848711in}}%
\pgfpathmoveto{\pgfqpoint{2.500976in}{1.844454in}}%
\pgfpathlineto{\pgfqpoint{2.500976in}{1.844454in}}%
\pgfpathlineto{\pgfqpoint{2.500976in}{1.848711in}}%
\pgfpathlineto{\pgfqpoint{2.505234in}{1.848711in}}%
\pgfpathlineto{\pgfqpoint{2.505234in}{1.844454in}}%
\pgfpathmoveto{\pgfqpoint{2.500976in}{1.848711in}}%
\pgfpathlineto{\pgfqpoint{2.500976in}{1.848711in}}%
\pgfpathlineto{\pgfqpoint{2.500976in}{1.852969in}}%
\pgfpathlineto{\pgfqpoint{2.505234in}{1.852969in}}%
\pgfpathlineto{\pgfqpoint{2.505234in}{1.848711in}}%
\pgfpathmoveto{\pgfqpoint{2.496718in}{1.852969in}}%
\pgfpathlineto{\pgfqpoint{2.496718in}{1.852969in}}%
\pgfpathlineto{\pgfqpoint{2.496718in}{1.857227in}}%
\pgfpathlineto{\pgfqpoint{2.500976in}{1.857227in}}%
\pgfpathlineto{\pgfqpoint{2.500976in}{1.852969in}}%
\pgfpathmoveto{\pgfqpoint{2.496718in}{1.857227in}}%
\pgfpathlineto{\pgfqpoint{2.496718in}{1.857227in}}%
\pgfpathlineto{\pgfqpoint{2.496718in}{1.861485in}}%
\pgfpathlineto{\pgfqpoint{2.500976in}{1.861485in}}%
\pgfpathlineto{\pgfqpoint{2.500976in}{1.857227in}}%
\pgfpathmoveto{\pgfqpoint{2.500976in}{1.852969in}}%
\pgfpathlineto{\pgfqpoint{2.500976in}{1.852969in}}%
\pgfpathlineto{\pgfqpoint{2.500976in}{1.857227in}}%
\pgfpathlineto{\pgfqpoint{2.505234in}{1.857227in}}%
\pgfpathlineto{\pgfqpoint{2.505234in}{1.852969in}}%
\pgfpathmoveto{\pgfqpoint{2.500976in}{1.857227in}}%
\pgfpathlineto{\pgfqpoint{2.500976in}{1.857227in}}%
\pgfpathlineto{\pgfqpoint{2.500976in}{1.861485in}}%
\pgfpathlineto{\pgfqpoint{2.505234in}{1.861485in}}%
\pgfpathlineto{\pgfqpoint{2.505234in}{1.857227in}}%
\pgfpathmoveto{\pgfqpoint{2.496718in}{1.861485in}}%
\pgfpathlineto{\pgfqpoint{2.496718in}{1.861485in}}%
\pgfpathlineto{\pgfqpoint{2.496718in}{1.865742in}}%
\pgfpathlineto{\pgfqpoint{2.500976in}{1.865742in}}%
\pgfpathlineto{\pgfqpoint{2.500976in}{1.861485in}}%
\pgfpathmoveto{\pgfqpoint{2.496718in}{1.865742in}}%
\pgfpathlineto{\pgfqpoint{2.496718in}{1.865742in}}%
\pgfpathlineto{\pgfqpoint{2.496718in}{1.870000in}}%
\pgfpathlineto{\pgfqpoint{2.500976in}{1.870000in}}%
\pgfpathlineto{\pgfqpoint{2.500976in}{1.865742in}}%
\pgfpathmoveto{\pgfqpoint{2.500976in}{1.861485in}}%
\pgfpathlineto{\pgfqpoint{2.500976in}{1.861485in}}%
\pgfpathlineto{\pgfqpoint{2.500976in}{1.865742in}}%
\pgfpathlineto{\pgfqpoint{2.505234in}{1.865742in}}%
\pgfpathlineto{\pgfqpoint{2.505234in}{1.861485in}}%
\pgfpathmoveto{\pgfqpoint{2.500976in}{1.865742in}}%
\pgfpathlineto{\pgfqpoint{2.500976in}{1.865742in}}%
\pgfpathlineto{\pgfqpoint{2.500976in}{1.870000in}}%
\pgfpathlineto{\pgfqpoint{2.505234in}{1.870000in}}%
\pgfpathlineto{\pgfqpoint{2.505234in}{1.865742in}}%
\pgfpathmoveto{\pgfqpoint{2.505234in}{1.857227in}}%
\pgfpathlineto{\pgfqpoint{2.505234in}{1.857227in}}%
\pgfpathlineto{\pgfqpoint{2.505234in}{1.861485in}}%
\pgfpathlineto{\pgfqpoint{2.509492in}{1.861485in}}%
\pgfpathlineto{\pgfqpoint{2.509492in}{1.857227in}}%
\pgfpathmoveto{\pgfqpoint{2.505234in}{1.861485in}}%
\pgfpathlineto{\pgfqpoint{2.505234in}{1.861485in}}%
\pgfpathlineto{\pgfqpoint{2.505234in}{1.865742in}}%
\pgfpathlineto{\pgfqpoint{2.509492in}{1.865742in}}%
\pgfpathlineto{\pgfqpoint{2.509492in}{1.861485in}}%
\pgfpathmoveto{\pgfqpoint{2.505234in}{1.865742in}}%
\pgfpathlineto{\pgfqpoint{2.505234in}{1.865742in}}%
\pgfpathlineto{\pgfqpoint{2.505234in}{1.870000in}}%
\pgfpathlineto{\pgfqpoint{2.509492in}{1.870000in}}%
\pgfpathlineto{\pgfqpoint{2.509492in}{1.865742in}}%
\pgfpathmoveto{\pgfqpoint{2.500976in}{1.870000in}}%
\pgfpathlineto{\pgfqpoint{2.500976in}{1.870000in}}%
\pgfpathlineto{\pgfqpoint{2.500976in}{1.874258in}}%
\pgfpathlineto{\pgfqpoint{2.505234in}{1.874258in}}%
\pgfpathlineto{\pgfqpoint{2.505234in}{1.870000in}}%
\pgfpathmoveto{\pgfqpoint{2.500976in}{1.874258in}}%
\pgfpathlineto{\pgfqpoint{2.500976in}{1.874258in}}%
\pgfpathlineto{\pgfqpoint{2.500976in}{1.878516in}}%
\pgfpathlineto{\pgfqpoint{2.505234in}{1.878516in}}%
\pgfpathlineto{\pgfqpoint{2.505234in}{1.874258in}}%
\pgfpathmoveto{\pgfqpoint{2.500976in}{1.878516in}}%
\pgfpathlineto{\pgfqpoint{2.500976in}{1.878516in}}%
\pgfpathlineto{\pgfqpoint{2.500976in}{1.882774in}}%
\pgfpathlineto{\pgfqpoint{2.505234in}{1.882774in}}%
\pgfpathlineto{\pgfqpoint{2.505234in}{1.878516in}}%
\pgfpathmoveto{\pgfqpoint{2.500976in}{1.882774in}}%
\pgfpathlineto{\pgfqpoint{2.500976in}{1.882774in}}%
\pgfpathlineto{\pgfqpoint{2.500976in}{1.887032in}}%
\pgfpathlineto{\pgfqpoint{2.505234in}{1.887032in}}%
\pgfpathlineto{\pgfqpoint{2.505234in}{1.882774in}}%
\pgfpathmoveto{\pgfqpoint{2.505234in}{1.870000in}}%
\pgfpathlineto{\pgfqpoint{2.505234in}{1.870000in}}%
\pgfpathlineto{\pgfqpoint{2.505234in}{1.874258in}}%
\pgfpathlineto{\pgfqpoint{2.509492in}{1.874258in}}%
\pgfpathlineto{\pgfqpoint{2.509492in}{1.870000in}}%
\pgfpathmoveto{\pgfqpoint{2.505234in}{1.874258in}}%
\pgfpathlineto{\pgfqpoint{2.505234in}{1.874258in}}%
\pgfpathlineto{\pgfqpoint{2.505234in}{1.878516in}}%
\pgfpathlineto{\pgfqpoint{2.509492in}{1.878516in}}%
\pgfpathlineto{\pgfqpoint{2.509492in}{1.874258in}}%
\pgfpathmoveto{\pgfqpoint{2.505234in}{1.878516in}}%
\pgfpathlineto{\pgfqpoint{2.505234in}{1.878516in}}%
\pgfpathlineto{\pgfqpoint{2.505234in}{1.882774in}}%
\pgfpathlineto{\pgfqpoint{2.509492in}{1.882774in}}%
\pgfpathlineto{\pgfqpoint{2.509492in}{1.878516in}}%
\pgfpathmoveto{\pgfqpoint{2.505234in}{1.882774in}}%
\pgfpathlineto{\pgfqpoint{2.505234in}{1.882774in}}%
\pgfpathlineto{\pgfqpoint{2.505234in}{1.887032in}}%
\pgfpathlineto{\pgfqpoint{2.509492in}{1.887032in}}%
\pgfpathlineto{\pgfqpoint{2.509492in}{1.882774in}}%
\pgfpathmoveto{\pgfqpoint{2.500976in}{1.887032in}}%
\pgfpathlineto{\pgfqpoint{2.500976in}{1.887032in}}%
\pgfpathlineto{\pgfqpoint{2.500976in}{1.891289in}}%
\pgfpathlineto{\pgfqpoint{2.505234in}{1.891289in}}%
\pgfpathlineto{\pgfqpoint{2.505234in}{1.887032in}}%
\pgfpathmoveto{\pgfqpoint{2.500976in}{1.891289in}}%
\pgfpathlineto{\pgfqpoint{2.500976in}{1.891289in}}%
\pgfpathlineto{\pgfqpoint{2.500976in}{1.895547in}}%
\pgfpathlineto{\pgfqpoint{2.505234in}{1.895547in}}%
\pgfpathlineto{\pgfqpoint{2.505234in}{1.891289in}}%
\pgfpathmoveto{\pgfqpoint{2.500976in}{1.895547in}}%
\pgfpathlineto{\pgfqpoint{2.500976in}{1.895547in}}%
\pgfpathlineto{\pgfqpoint{2.500976in}{1.899805in}}%
\pgfpathlineto{\pgfqpoint{2.505234in}{1.899805in}}%
\pgfpathlineto{\pgfqpoint{2.505234in}{1.895547in}}%
\pgfpathmoveto{\pgfqpoint{2.500976in}{1.899805in}}%
\pgfpathlineto{\pgfqpoint{2.500976in}{1.899805in}}%
\pgfpathlineto{\pgfqpoint{2.500976in}{1.904063in}}%
\pgfpathlineto{\pgfqpoint{2.505234in}{1.904063in}}%
\pgfpathlineto{\pgfqpoint{2.505234in}{1.899805in}}%
\pgfpathmoveto{\pgfqpoint{2.505234in}{1.887032in}}%
\pgfpathlineto{\pgfqpoint{2.505234in}{1.887032in}}%
\pgfpathlineto{\pgfqpoint{2.505234in}{1.891289in}}%
\pgfpathlineto{\pgfqpoint{2.509492in}{1.891289in}}%
\pgfpathlineto{\pgfqpoint{2.509492in}{1.887032in}}%
\pgfpathmoveto{\pgfqpoint{2.505234in}{1.891289in}}%
\pgfpathlineto{\pgfqpoint{2.505234in}{1.891289in}}%
\pgfpathlineto{\pgfqpoint{2.505234in}{1.895547in}}%
\pgfpathlineto{\pgfqpoint{2.509492in}{1.895547in}}%
\pgfpathlineto{\pgfqpoint{2.509492in}{1.891289in}}%
\pgfpathmoveto{\pgfqpoint{2.509492in}{1.891289in}}%
\pgfpathlineto{\pgfqpoint{2.509492in}{1.891289in}}%
\pgfpathlineto{\pgfqpoint{2.509492in}{1.895547in}}%
\pgfpathlineto{\pgfqpoint{2.513749in}{1.895547in}}%
\pgfpathlineto{\pgfqpoint{2.513749in}{1.891289in}}%
\pgfpathmoveto{\pgfqpoint{2.505234in}{1.895547in}}%
\pgfpathlineto{\pgfqpoint{2.505234in}{1.895547in}}%
\pgfpathlineto{\pgfqpoint{2.505234in}{1.899805in}}%
\pgfpathlineto{\pgfqpoint{2.509492in}{1.899805in}}%
\pgfpathlineto{\pgfqpoint{2.509492in}{1.895547in}}%
\pgfpathmoveto{\pgfqpoint{2.505234in}{1.899805in}}%
\pgfpathlineto{\pgfqpoint{2.505234in}{1.899805in}}%
\pgfpathlineto{\pgfqpoint{2.505234in}{1.904063in}}%
\pgfpathlineto{\pgfqpoint{2.509492in}{1.904063in}}%
\pgfpathlineto{\pgfqpoint{2.509492in}{1.899805in}}%
\pgfpathmoveto{\pgfqpoint{2.509492in}{1.895547in}}%
\pgfpathlineto{\pgfqpoint{2.509492in}{1.895547in}}%
\pgfpathlineto{\pgfqpoint{2.509492in}{1.899805in}}%
\pgfpathlineto{\pgfqpoint{2.513749in}{1.899805in}}%
\pgfpathlineto{\pgfqpoint{2.513749in}{1.895547in}}%
\pgfpathmoveto{\pgfqpoint{2.509492in}{1.899805in}}%
\pgfpathlineto{\pgfqpoint{2.509492in}{1.899805in}}%
\pgfpathlineto{\pgfqpoint{2.509492in}{1.904063in}}%
\pgfpathlineto{\pgfqpoint{2.513749in}{1.904063in}}%
\pgfpathlineto{\pgfqpoint{2.513749in}{1.899805in}}%
\pgfpathmoveto{\pgfqpoint{2.505234in}{1.904063in}}%
\pgfpathlineto{\pgfqpoint{2.505234in}{1.904063in}}%
\pgfpathlineto{\pgfqpoint{2.505234in}{1.908321in}}%
\pgfpathlineto{\pgfqpoint{2.509492in}{1.908321in}}%
\pgfpathlineto{\pgfqpoint{2.509492in}{1.904063in}}%
\pgfpathmoveto{\pgfqpoint{2.505234in}{1.908321in}}%
\pgfpathlineto{\pgfqpoint{2.505234in}{1.908321in}}%
\pgfpathlineto{\pgfqpoint{2.505234in}{1.912579in}}%
\pgfpathlineto{\pgfqpoint{2.509492in}{1.912579in}}%
\pgfpathlineto{\pgfqpoint{2.509492in}{1.908321in}}%
\pgfpathmoveto{\pgfqpoint{2.509492in}{1.904063in}}%
\pgfpathlineto{\pgfqpoint{2.509492in}{1.904063in}}%
\pgfpathlineto{\pgfqpoint{2.509492in}{1.908321in}}%
\pgfpathlineto{\pgfqpoint{2.513749in}{1.908321in}}%
\pgfpathlineto{\pgfqpoint{2.513749in}{1.904063in}}%
\pgfpathmoveto{\pgfqpoint{2.509492in}{1.908321in}}%
\pgfpathlineto{\pgfqpoint{2.509492in}{1.908321in}}%
\pgfpathlineto{\pgfqpoint{2.509492in}{1.912579in}}%
\pgfpathlineto{\pgfqpoint{2.513749in}{1.912579in}}%
\pgfpathlineto{\pgfqpoint{2.513749in}{1.908321in}}%
\pgfpathmoveto{\pgfqpoint{2.505234in}{1.912579in}}%
\pgfpathlineto{\pgfqpoint{2.505234in}{1.912579in}}%
\pgfpathlineto{\pgfqpoint{2.505234in}{1.916836in}}%
\pgfpathlineto{\pgfqpoint{2.509492in}{1.916836in}}%
\pgfpathlineto{\pgfqpoint{2.509492in}{1.912579in}}%
\pgfpathmoveto{\pgfqpoint{2.505234in}{1.916836in}}%
\pgfpathlineto{\pgfqpoint{2.505234in}{1.916836in}}%
\pgfpathlineto{\pgfqpoint{2.505234in}{1.921094in}}%
\pgfpathlineto{\pgfqpoint{2.509492in}{1.921094in}}%
\pgfpathlineto{\pgfqpoint{2.509492in}{1.916836in}}%
\pgfpathmoveto{\pgfqpoint{2.509492in}{1.912579in}}%
\pgfpathlineto{\pgfqpoint{2.509492in}{1.912579in}}%
\pgfpathlineto{\pgfqpoint{2.509492in}{1.916836in}}%
\pgfpathlineto{\pgfqpoint{2.513749in}{1.916836in}}%
\pgfpathlineto{\pgfqpoint{2.513749in}{1.912579in}}%
\pgfpathmoveto{\pgfqpoint{2.509492in}{1.916836in}}%
\pgfpathlineto{\pgfqpoint{2.509492in}{1.916836in}}%
\pgfpathlineto{\pgfqpoint{2.509492in}{1.921094in}}%
\pgfpathlineto{\pgfqpoint{2.513749in}{1.921094in}}%
\pgfpathlineto{\pgfqpoint{2.513749in}{1.916836in}}%
\pgfpathmoveto{\pgfqpoint{2.505234in}{1.921094in}}%
\pgfpathlineto{\pgfqpoint{2.505234in}{1.921094in}}%
\pgfpathlineto{\pgfqpoint{2.505234in}{1.925352in}}%
\pgfpathlineto{\pgfqpoint{2.509492in}{1.925352in}}%
\pgfpathlineto{\pgfqpoint{2.509492in}{1.921094in}}%
\pgfpathmoveto{\pgfqpoint{2.505234in}{1.925352in}}%
\pgfpathlineto{\pgfqpoint{2.505234in}{1.925352in}}%
\pgfpathlineto{\pgfqpoint{2.505234in}{1.929610in}}%
\pgfpathlineto{\pgfqpoint{2.509492in}{1.929610in}}%
\pgfpathlineto{\pgfqpoint{2.509492in}{1.925352in}}%
\pgfpathmoveto{\pgfqpoint{2.509492in}{1.921094in}}%
\pgfpathlineto{\pgfqpoint{2.509492in}{1.921094in}}%
\pgfpathlineto{\pgfqpoint{2.509492in}{1.925352in}}%
\pgfpathlineto{\pgfqpoint{2.513749in}{1.925352in}}%
\pgfpathlineto{\pgfqpoint{2.513749in}{1.921094in}}%
\pgfpathmoveto{\pgfqpoint{2.509492in}{1.925352in}}%
\pgfpathlineto{\pgfqpoint{2.509492in}{1.925352in}}%
\pgfpathlineto{\pgfqpoint{2.509492in}{1.929610in}}%
\pgfpathlineto{\pgfqpoint{2.513749in}{1.929610in}}%
\pgfpathlineto{\pgfqpoint{2.513749in}{1.925352in}}%
\pgfpathmoveto{\pgfqpoint{2.505234in}{1.929610in}}%
\pgfpathlineto{\pgfqpoint{2.505234in}{1.929610in}}%
\pgfpathlineto{\pgfqpoint{2.505234in}{1.933868in}}%
\pgfpathlineto{\pgfqpoint{2.509492in}{1.933868in}}%
\pgfpathlineto{\pgfqpoint{2.509492in}{1.929610in}}%
\pgfpathmoveto{\pgfqpoint{2.505234in}{1.933868in}}%
\pgfpathlineto{\pgfqpoint{2.505234in}{1.933868in}}%
\pgfpathlineto{\pgfqpoint{2.505234in}{1.938126in}}%
\pgfpathlineto{\pgfqpoint{2.509492in}{1.938126in}}%
\pgfpathlineto{\pgfqpoint{2.509492in}{1.933868in}}%
\pgfpathmoveto{\pgfqpoint{2.509492in}{1.929610in}}%
\pgfpathlineto{\pgfqpoint{2.509492in}{1.929610in}}%
\pgfpathlineto{\pgfqpoint{2.509492in}{1.933868in}}%
\pgfpathlineto{\pgfqpoint{2.513749in}{1.933868in}}%
\pgfpathlineto{\pgfqpoint{2.513749in}{1.929610in}}%
\pgfpathmoveto{\pgfqpoint{2.509492in}{1.933868in}}%
\pgfpathlineto{\pgfqpoint{2.509492in}{1.933868in}}%
\pgfpathlineto{\pgfqpoint{2.509492in}{1.938126in}}%
\pgfpathlineto{\pgfqpoint{2.513749in}{1.938126in}}%
\pgfpathlineto{\pgfqpoint{2.513749in}{1.933868in}}%
\pgfpathmoveto{\pgfqpoint{2.513749in}{1.929610in}}%
\pgfpathlineto{\pgfqpoint{2.513749in}{1.929610in}}%
\pgfpathlineto{\pgfqpoint{2.513749in}{1.933868in}}%
\pgfpathlineto{\pgfqpoint{2.518007in}{1.933868in}}%
\pgfpathlineto{\pgfqpoint{2.518007in}{1.929610in}}%
\pgfpathmoveto{\pgfqpoint{2.513749in}{1.933868in}}%
\pgfpathlineto{\pgfqpoint{2.513749in}{1.933868in}}%
\pgfpathlineto{\pgfqpoint{2.513749in}{1.938126in}}%
\pgfpathlineto{\pgfqpoint{2.518007in}{1.938126in}}%
\pgfpathlineto{\pgfqpoint{2.518007in}{1.933868in}}%
\pgfpathmoveto{\pgfqpoint{2.509492in}{1.938126in}}%
\pgfpathlineto{\pgfqpoint{2.509492in}{1.938126in}}%
\pgfpathlineto{\pgfqpoint{2.509492in}{1.942384in}}%
\pgfpathlineto{\pgfqpoint{2.513749in}{1.942384in}}%
\pgfpathlineto{\pgfqpoint{2.513749in}{1.938126in}}%
\pgfpathmoveto{\pgfqpoint{2.509492in}{1.942384in}}%
\pgfpathlineto{\pgfqpoint{2.509492in}{1.942384in}}%
\pgfpathlineto{\pgfqpoint{2.509492in}{1.946641in}}%
\pgfpathlineto{\pgfqpoint{2.513749in}{1.946641in}}%
\pgfpathlineto{\pgfqpoint{2.513749in}{1.942384in}}%
\pgfpathmoveto{\pgfqpoint{2.509492in}{1.946641in}}%
\pgfpathlineto{\pgfqpoint{2.509492in}{1.946641in}}%
\pgfpathlineto{\pgfqpoint{2.509492in}{1.950899in}}%
\pgfpathlineto{\pgfqpoint{2.513749in}{1.950899in}}%
\pgfpathlineto{\pgfqpoint{2.513749in}{1.946641in}}%
\pgfpathmoveto{\pgfqpoint{2.509492in}{1.950899in}}%
\pgfpathlineto{\pgfqpoint{2.509492in}{1.950899in}}%
\pgfpathlineto{\pgfqpoint{2.509492in}{1.955157in}}%
\pgfpathlineto{\pgfqpoint{2.513749in}{1.955157in}}%
\pgfpathlineto{\pgfqpoint{2.513749in}{1.950899in}}%
\pgfpathmoveto{\pgfqpoint{2.509492in}{1.955157in}}%
\pgfpathlineto{\pgfqpoint{2.509492in}{1.955157in}}%
\pgfpathlineto{\pgfqpoint{2.509492in}{1.959415in}}%
\pgfpathlineto{\pgfqpoint{2.513749in}{1.959415in}}%
\pgfpathlineto{\pgfqpoint{2.513749in}{1.955157in}}%
\pgfpathmoveto{\pgfqpoint{2.509492in}{1.959415in}}%
\pgfpathlineto{\pgfqpoint{2.509492in}{1.959415in}}%
\pgfpathlineto{\pgfqpoint{2.509492in}{1.963673in}}%
\pgfpathlineto{\pgfqpoint{2.513749in}{1.963673in}}%
\pgfpathlineto{\pgfqpoint{2.513749in}{1.959415in}}%
\pgfpathmoveto{\pgfqpoint{2.509492in}{1.963673in}}%
\pgfpathlineto{\pgfqpoint{2.509492in}{1.963673in}}%
\pgfpathlineto{\pgfqpoint{2.509492in}{1.967931in}}%
\pgfpathlineto{\pgfqpoint{2.513749in}{1.967931in}}%
\pgfpathlineto{\pgfqpoint{2.513749in}{1.963673in}}%
\pgfpathmoveto{\pgfqpoint{2.509492in}{1.967931in}}%
\pgfpathlineto{\pgfqpoint{2.509492in}{1.967931in}}%
\pgfpathlineto{\pgfqpoint{2.509492in}{1.972188in}}%
\pgfpathlineto{\pgfqpoint{2.513749in}{1.972188in}}%
\pgfpathlineto{\pgfqpoint{2.513749in}{1.967931in}}%
\pgfpathmoveto{\pgfqpoint{2.509492in}{1.972188in}}%
\pgfpathlineto{\pgfqpoint{2.509492in}{1.972188in}}%
\pgfpathlineto{\pgfqpoint{2.509492in}{1.976446in}}%
\pgfpathlineto{\pgfqpoint{2.513749in}{1.976446in}}%
\pgfpathlineto{\pgfqpoint{2.513749in}{1.972188in}}%
\pgfpathmoveto{\pgfqpoint{2.513749in}{1.938126in}}%
\pgfpathlineto{\pgfqpoint{2.513749in}{1.938126in}}%
\pgfpathlineto{\pgfqpoint{2.513749in}{1.942384in}}%
\pgfpathlineto{\pgfqpoint{2.518007in}{1.942384in}}%
\pgfpathlineto{\pgfqpoint{2.518007in}{1.938126in}}%
\pgfpathmoveto{\pgfqpoint{2.513749in}{1.942384in}}%
\pgfpathlineto{\pgfqpoint{2.513749in}{1.942384in}}%
\pgfpathlineto{\pgfqpoint{2.513749in}{1.946641in}}%
\pgfpathlineto{\pgfqpoint{2.518007in}{1.946641in}}%
\pgfpathlineto{\pgfqpoint{2.518007in}{1.942384in}}%
\pgfpathmoveto{\pgfqpoint{2.513749in}{1.946641in}}%
\pgfpathlineto{\pgfqpoint{2.513749in}{1.946641in}}%
\pgfpathlineto{\pgfqpoint{2.513749in}{1.950899in}}%
\pgfpathlineto{\pgfqpoint{2.518007in}{1.950899in}}%
\pgfpathlineto{\pgfqpoint{2.518007in}{1.946641in}}%
\pgfpathmoveto{\pgfqpoint{2.513749in}{1.950899in}}%
\pgfpathlineto{\pgfqpoint{2.513749in}{1.950899in}}%
\pgfpathlineto{\pgfqpoint{2.513749in}{1.955157in}}%
\pgfpathlineto{\pgfqpoint{2.518007in}{1.955157in}}%
\pgfpathlineto{\pgfqpoint{2.518007in}{1.950899in}}%
\pgfpathmoveto{\pgfqpoint{2.513749in}{1.955157in}}%
\pgfpathlineto{\pgfqpoint{2.513749in}{1.955157in}}%
\pgfpathlineto{\pgfqpoint{2.513749in}{1.959415in}}%
\pgfpathlineto{\pgfqpoint{2.518007in}{1.959415in}}%
\pgfpathlineto{\pgfqpoint{2.518007in}{1.955157in}}%
\pgfpathmoveto{\pgfqpoint{2.513749in}{1.959415in}}%
\pgfpathlineto{\pgfqpoint{2.513749in}{1.959415in}}%
\pgfpathlineto{\pgfqpoint{2.513749in}{1.963673in}}%
\pgfpathlineto{\pgfqpoint{2.518007in}{1.963673in}}%
\pgfpathlineto{\pgfqpoint{2.518007in}{1.959415in}}%
\pgfpathmoveto{\pgfqpoint{2.513749in}{1.963673in}}%
\pgfpathlineto{\pgfqpoint{2.513749in}{1.963673in}}%
\pgfpathlineto{\pgfqpoint{2.513749in}{1.967931in}}%
\pgfpathlineto{\pgfqpoint{2.518007in}{1.967931in}}%
\pgfpathlineto{\pgfqpoint{2.518007in}{1.963673in}}%
\pgfpathmoveto{\pgfqpoint{2.513749in}{1.967931in}}%
\pgfpathlineto{\pgfqpoint{2.513749in}{1.967931in}}%
\pgfpathlineto{\pgfqpoint{2.513749in}{1.972188in}}%
\pgfpathlineto{\pgfqpoint{2.518007in}{1.972188in}}%
\pgfpathlineto{\pgfqpoint{2.518007in}{1.967931in}}%
\pgfpathmoveto{\pgfqpoint{2.518007in}{1.963673in}}%
\pgfpathlineto{\pgfqpoint{2.518007in}{1.963673in}}%
\pgfpathlineto{\pgfqpoint{2.518007in}{1.967931in}}%
\pgfpathlineto{\pgfqpoint{2.522265in}{1.967931in}}%
\pgfpathlineto{\pgfqpoint{2.522265in}{1.963673in}}%
\pgfpathmoveto{\pgfqpoint{2.518007in}{1.967931in}}%
\pgfpathlineto{\pgfqpoint{2.518007in}{1.967931in}}%
\pgfpathlineto{\pgfqpoint{2.518007in}{1.972188in}}%
\pgfpathlineto{\pgfqpoint{2.522265in}{1.972188in}}%
\pgfpathlineto{\pgfqpoint{2.522265in}{1.967931in}}%
\pgfpathmoveto{\pgfqpoint{2.513749in}{1.972188in}}%
\pgfpathlineto{\pgfqpoint{2.513749in}{1.972188in}}%
\pgfpathlineto{\pgfqpoint{2.513749in}{1.976446in}}%
\pgfpathlineto{\pgfqpoint{2.518007in}{1.976446in}}%
\pgfpathlineto{\pgfqpoint{2.518007in}{1.972188in}}%
\pgfpathmoveto{\pgfqpoint{2.513749in}{1.976446in}}%
\pgfpathlineto{\pgfqpoint{2.513749in}{1.976446in}}%
\pgfpathlineto{\pgfqpoint{2.513749in}{1.980704in}}%
\pgfpathlineto{\pgfqpoint{2.518007in}{1.980704in}}%
\pgfpathlineto{\pgfqpoint{2.518007in}{1.976446in}}%
\pgfpathmoveto{\pgfqpoint{2.518007in}{1.972188in}}%
\pgfpathlineto{\pgfqpoint{2.518007in}{1.972188in}}%
\pgfpathlineto{\pgfqpoint{2.518007in}{1.976446in}}%
\pgfpathlineto{\pgfqpoint{2.522265in}{1.976446in}}%
\pgfpathlineto{\pgfqpoint{2.522265in}{1.972188in}}%
\pgfpathmoveto{\pgfqpoint{2.518007in}{1.976446in}}%
\pgfpathlineto{\pgfqpoint{2.518007in}{1.976446in}}%
\pgfpathlineto{\pgfqpoint{2.518007in}{1.980704in}}%
\pgfpathlineto{\pgfqpoint{2.522265in}{1.980704in}}%
\pgfpathlineto{\pgfqpoint{2.522265in}{1.976446in}}%
\pgfpathmoveto{\pgfqpoint{2.513749in}{1.980704in}}%
\pgfpathlineto{\pgfqpoint{2.513749in}{1.980704in}}%
\pgfpathlineto{\pgfqpoint{2.513749in}{1.984962in}}%
\pgfpathlineto{\pgfqpoint{2.518007in}{1.984962in}}%
\pgfpathlineto{\pgfqpoint{2.518007in}{1.980704in}}%
\pgfpathmoveto{\pgfqpoint{2.513749in}{1.984962in}}%
\pgfpathlineto{\pgfqpoint{2.513749in}{1.984962in}}%
\pgfpathlineto{\pgfqpoint{2.513749in}{1.989220in}}%
\pgfpathlineto{\pgfqpoint{2.518007in}{1.989220in}}%
\pgfpathlineto{\pgfqpoint{2.518007in}{1.984962in}}%
\pgfpathmoveto{\pgfqpoint{2.518007in}{1.980704in}}%
\pgfpathlineto{\pgfqpoint{2.518007in}{1.980704in}}%
\pgfpathlineto{\pgfqpoint{2.518007in}{1.984962in}}%
\pgfpathlineto{\pgfqpoint{2.522265in}{1.984962in}}%
\pgfpathlineto{\pgfqpoint{2.522265in}{1.980704in}}%
\pgfpathmoveto{\pgfqpoint{2.518007in}{1.984962in}}%
\pgfpathlineto{\pgfqpoint{2.518007in}{1.984962in}}%
\pgfpathlineto{\pgfqpoint{2.518007in}{1.989220in}}%
\pgfpathlineto{\pgfqpoint{2.522265in}{1.989220in}}%
\pgfpathlineto{\pgfqpoint{2.522265in}{1.984962in}}%
\pgfpathmoveto{\pgfqpoint{2.513749in}{1.989220in}}%
\pgfpathlineto{\pgfqpoint{2.513749in}{1.989220in}}%
\pgfpathlineto{\pgfqpoint{2.513749in}{1.993478in}}%
\pgfpathlineto{\pgfqpoint{2.518007in}{1.993478in}}%
\pgfpathlineto{\pgfqpoint{2.518007in}{1.989220in}}%
\pgfpathmoveto{\pgfqpoint{2.513749in}{1.993478in}}%
\pgfpathlineto{\pgfqpoint{2.513749in}{1.993478in}}%
\pgfpathlineto{\pgfqpoint{2.513749in}{1.997735in}}%
\pgfpathlineto{\pgfqpoint{2.518007in}{1.997735in}}%
\pgfpathlineto{\pgfqpoint{2.518007in}{1.993478in}}%
\pgfpathmoveto{\pgfqpoint{2.518007in}{1.989220in}}%
\pgfpathlineto{\pgfqpoint{2.518007in}{1.989220in}}%
\pgfpathlineto{\pgfqpoint{2.518007in}{1.993478in}}%
\pgfpathlineto{\pgfqpoint{2.522265in}{1.993478in}}%
\pgfpathlineto{\pgfqpoint{2.522265in}{1.989220in}}%
\pgfpathmoveto{\pgfqpoint{2.518007in}{1.993478in}}%
\pgfpathlineto{\pgfqpoint{2.518007in}{1.993478in}}%
\pgfpathlineto{\pgfqpoint{2.518007in}{1.997735in}}%
\pgfpathlineto{\pgfqpoint{2.522265in}{1.997735in}}%
\pgfpathlineto{\pgfqpoint{2.522265in}{1.993478in}}%
\pgfpathmoveto{\pgfqpoint{2.513749in}{1.997735in}}%
\pgfpathlineto{\pgfqpoint{2.513749in}{1.997735in}}%
\pgfpathlineto{\pgfqpoint{2.513749in}{2.001993in}}%
\pgfpathlineto{\pgfqpoint{2.518007in}{2.001993in}}%
\pgfpathlineto{\pgfqpoint{2.518007in}{1.997735in}}%
\pgfpathmoveto{\pgfqpoint{2.513749in}{2.001993in}}%
\pgfpathlineto{\pgfqpoint{2.513749in}{2.001993in}}%
\pgfpathlineto{\pgfqpoint{2.513749in}{2.006251in}}%
\pgfpathlineto{\pgfqpoint{2.518007in}{2.006251in}}%
\pgfpathlineto{\pgfqpoint{2.518007in}{2.001993in}}%
\pgfpathmoveto{\pgfqpoint{2.518007in}{1.997735in}}%
\pgfpathlineto{\pgfqpoint{2.518007in}{1.997735in}}%
\pgfpathlineto{\pgfqpoint{2.518007in}{2.001993in}}%
\pgfpathlineto{\pgfqpoint{2.522265in}{2.001993in}}%
\pgfpathlineto{\pgfqpoint{2.522265in}{1.997735in}}%
\pgfpathmoveto{\pgfqpoint{2.518007in}{2.001993in}}%
\pgfpathlineto{\pgfqpoint{2.518007in}{2.001993in}}%
\pgfpathlineto{\pgfqpoint{2.518007in}{2.006251in}}%
\pgfpathlineto{\pgfqpoint{2.522265in}{2.006251in}}%
\pgfpathlineto{\pgfqpoint{2.522265in}{2.001993in}}%
\pgfpathmoveto{\pgfqpoint{2.522265in}{1.997735in}}%
\pgfpathlineto{\pgfqpoint{2.522265in}{1.997735in}}%
\pgfpathlineto{\pgfqpoint{2.522265in}{2.001993in}}%
\pgfpathlineto{\pgfqpoint{2.526523in}{2.001993in}}%
\pgfpathlineto{\pgfqpoint{2.526523in}{1.997735in}}%
\pgfpathmoveto{\pgfqpoint{2.522265in}{2.001993in}}%
\pgfpathlineto{\pgfqpoint{2.522265in}{2.001993in}}%
\pgfpathlineto{\pgfqpoint{2.522265in}{2.006251in}}%
\pgfpathlineto{\pgfqpoint{2.526523in}{2.006251in}}%
\pgfpathlineto{\pgfqpoint{2.526523in}{2.001993in}}%
\pgfpathmoveto{\pgfqpoint{2.513749in}{2.006251in}}%
\pgfpathlineto{\pgfqpoint{2.513749in}{2.006251in}}%
\pgfpathlineto{\pgfqpoint{2.513749in}{2.010509in}}%
\pgfpathlineto{\pgfqpoint{2.518007in}{2.010509in}}%
\pgfpathlineto{\pgfqpoint{2.518007in}{2.006251in}}%
\pgfpathmoveto{\pgfqpoint{2.518007in}{2.006251in}}%
\pgfpathlineto{\pgfqpoint{2.518007in}{2.006251in}}%
\pgfpathlineto{\pgfqpoint{2.518007in}{2.010509in}}%
\pgfpathlineto{\pgfqpoint{2.522265in}{2.010509in}}%
\pgfpathlineto{\pgfqpoint{2.522265in}{2.006251in}}%
\pgfpathmoveto{\pgfqpoint{2.518007in}{2.010509in}}%
\pgfpathlineto{\pgfqpoint{2.518007in}{2.010509in}}%
\pgfpathlineto{\pgfqpoint{2.518007in}{2.014767in}}%
\pgfpathlineto{\pgfqpoint{2.522265in}{2.014767in}}%
\pgfpathlineto{\pgfqpoint{2.522265in}{2.010509in}}%
\pgfpathmoveto{\pgfqpoint{2.518007in}{2.014767in}}%
\pgfpathlineto{\pgfqpoint{2.518007in}{2.014767in}}%
\pgfpathlineto{\pgfqpoint{2.518007in}{2.019024in}}%
\pgfpathlineto{\pgfqpoint{2.522265in}{2.019024in}}%
\pgfpathlineto{\pgfqpoint{2.522265in}{2.014767in}}%
\pgfpathmoveto{\pgfqpoint{2.518007in}{2.019024in}}%
\pgfpathlineto{\pgfqpoint{2.518007in}{2.019024in}}%
\pgfpathlineto{\pgfqpoint{2.518007in}{2.023282in}}%
\pgfpathlineto{\pgfqpoint{2.522265in}{2.023282in}}%
\pgfpathlineto{\pgfqpoint{2.522265in}{2.019024in}}%
\pgfpathmoveto{\pgfqpoint{2.522265in}{2.006251in}}%
\pgfpathlineto{\pgfqpoint{2.522265in}{2.006251in}}%
\pgfpathlineto{\pgfqpoint{2.522265in}{2.010509in}}%
\pgfpathlineto{\pgfqpoint{2.526523in}{2.010509in}}%
\pgfpathlineto{\pgfqpoint{2.526523in}{2.006251in}}%
\pgfpathmoveto{\pgfqpoint{2.522265in}{2.010509in}}%
\pgfpathlineto{\pgfqpoint{2.522265in}{2.010509in}}%
\pgfpathlineto{\pgfqpoint{2.522265in}{2.014767in}}%
\pgfpathlineto{\pgfqpoint{2.526523in}{2.014767in}}%
\pgfpathlineto{\pgfqpoint{2.526523in}{2.010509in}}%
\pgfpathmoveto{\pgfqpoint{2.522265in}{2.014767in}}%
\pgfpathlineto{\pgfqpoint{2.522265in}{2.014767in}}%
\pgfpathlineto{\pgfqpoint{2.522265in}{2.019024in}}%
\pgfpathlineto{\pgfqpoint{2.526523in}{2.019024in}}%
\pgfpathlineto{\pgfqpoint{2.526523in}{2.014767in}}%
\pgfpathmoveto{\pgfqpoint{2.522265in}{2.019024in}}%
\pgfpathlineto{\pgfqpoint{2.522265in}{2.019024in}}%
\pgfpathlineto{\pgfqpoint{2.522265in}{2.023282in}}%
\pgfpathlineto{\pgfqpoint{2.526523in}{2.023282in}}%
\pgfpathlineto{\pgfqpoint{2.526523in}{2.019024in}}%
\pgfpathmoveto{\pgfqpoint{2.518007in}{2.023282in}}%
\pgfpathlineto{\pgfqpoint{2.518007in}{2.023282in}}%
\pgfpathlineto{\pgfqpoint{2.518007in}{2.027540in}}%
\pgfpathlineto{\pgfqpoint{2.522265in}{2.027540in}}%
\pgfpathlineto{\pgfqpoint{2.522265in}{2.023282in}}%
\pgfpathmoveto{\pgfqpoint{2.518007in}{2.027540in}}%
\pgfpathlineto{\pgfqpoint{2.518007in}{2.027540in}}%
\pgfpathlineto{\pgfqpoint{2.518007in}{2.031798in}}%
\pgfpathlineto{\pgfqpoint{2.522265in}{2.031798in}}%
\pgfpathlineto{\pgfqpoint{2.522265in}{2.027540in}}%
\pgfpathmoveto{\pgfqpoint{2.518007in}{2.031798in}}%
\pgfpathlineto{\pgfqpoint{2.518007in}{2.031798in}}%
\pgfpathlineto{\pgfqpoint{2.518007in}{2.036055in}}%
\pgfpathlineto{\pgfqpoint{2.522265in}{2.036055in}}%
\pgfpathlineto{\pgfqpoint{2.522265in}{2.031798in}}%
\pgfpathmoveto{\pgfqpoint{2.518007in}{2.036055in}}%
\pgfpathlineto{\pgfqpoint{2.518007in}{2.036055in}}%
\pgfpathlineto{\pgfqpoint{2.518007in}{2.040313in}}%
\pgfpathlineto{\pgfqpoint{2.522265in}{2.040313in}}%
\pgfpathlineto{\pgfqpoint{2.522265in}{2.036055in}}%
\pgfpathmoveto{\pgfqpoint{2.522265in}{2.023282in}}%
\pgfpathlineto{\pgfqpoint{2.522265in}{2.023282in}}%
\pgfpathlineto{\pgfqpoint{2.522265in}{2.027540in}}%
\pgfpathlineto{\pgfqpoint{2.526523in}{2.027540in}}%
\pgfpathlineto{\pgfqpoint{2.526523in}{2.023282in}}%
\pgfpathmoveto{\pgfqpoint{2.522265in}{2.027540in}}%
\pgfpathlineto{\pgfqpoint{2.522265in}{2.027540in}}%
\pgfpathlineto{\pgfqpoint{2.522265in}{2.031798in}}%
\pgfpathlineto{\pgfqpoint{2.526523in}{2.031798in}}%
\pgfpathlineto{\pgfqpoint{2.526523in}{2.027540in}}%
\pgfpathmoveto{\pgfqpoint{2.522265in}{2.031798in}}%
\pgfpathlineto{\pgfqpoint{2.522265in}{2.031798in}}%
\pgfpathlineto{\pgfqpoint{2.522265in}{2.036055in}}%
\pgfpathlineto{\pgfqpoint{2.526523in}{2.036055in}}%
\pgfpathlineto{\pgfqpoint{2.526523in}{2.031798in}}%
\pgfpathmoveto{\pgfqpoint{2.522265in}{2.036055in}}%
\pgfpathlineto{\pgfqpoint{2.522265in}{2.036055in}}%
\pgfpathlineto{\pgfqpoint{2.522265in}{2.040313in}}%
\pgfpathlineto{\pgfqpoint{2.526523in}{2.040313in}}%
\pgfpathlineto{\pgfqpoint{2.526523in}{2.036055in}}%
\pgfpathmoveto{\pgfqpoint{2.526523in}{2.031798in}}%
\pgfpathlineto{\pgfqpoint{2.526523in}{2.031798in}}%
\pgfpathlineto{\pgfqpoint{2.526523in}{2.036055in}}%
\pgfpathlineto{\pgfqpoint{2.530780in}{2.036055in}}%
\pgfpathlineto{\pgfqpoint{2.530780in}{2.031798in}}%
\pgfpathmoveto{\pgfqpoint{2.526523in}{2.036055in}}%
\pgfpathlineto{\pgfqpoint{2.526523in}{2.036055in}}%
\pgfpathlineto{\pgfqpoint{2.526523in}{2.040313in}}%
\pgfpathlineto{\pgfqpoint{2.530780in}{2.040313in}}%
\pgfpathlineto{\pgfqpoint{2.530780in}{2.036055in}}%
\pgfpathmoveto{\pgfqpoint{2.518007in}{2.040313in}}%
\pgfpathlineto{\pgfqpoint{2.518007in}{2.040313in}}%
\pgfpathlineto{\pgfqpoint{2.518007in}{2.044571in}}%
\pgfpathlineto{\pgfqpoint{2.522265in}{2.044571in}}%
\pgfpathlineto{\pgfqpoint{2.522265in}{2.040313in}}%
\pgfpathmoveto{\pgfqpoint{2.522265in}{2.040313in}}%
\pgfpathlineto{\pgfqpoint{2.522265in}{2.040313in}}%
\pgfpathlineto{\pgfqpoint{2.522265in}{2.044571in}}%
\pgfpathlineto{\pgfqpoint{2.526523in}{2.044571in}}%
\pgfpathlineto{\pgfqpoint{2.526523in}{2.040313in}}%
\pgfpathmoveto{\pgfqpoint{2.522265in}{2.044571in}}%
\pgfpathlineto{\pgfqpoint{2.522265in}{2.044571in}}%
\pgfpathlineto{\pgfqpoint{2.522265in}{2.048829in}}%
\pgfpathlineto{\pgfqpoint{2.526523in}{2.048829in}}%
\pgfpathlineto{\pgfqpoint{2.526523in}{2.044571in}}%
\pgfpathmoveto{\pgfqpoint{2.526523in}{2.040313in}}%
\pgfpathlineto{\pgfqpoint{2.526523in}{2.040313in}}%
\pgfpathlineto{\pgfqpoint{2.526523in}{2.044571in}}%
\pgfpathlineto{\pgfqpoint{2.530780in}{2.044571in}}%
\pgfpathlineto{\pgfqpoint{2.530780in}{2.040313in}}%
\pgfpathmoveto{\pgfqpoint{2.526523in}{2.044571in}}%
\pgfpathlineto{\pgfqpoint{2.526523in}{2.044571in}}%
\pgfpathlineto{\pgfqpoint{2.526523in}{2.048829in}}%
\pgfpathlineto{\pgfqpoint{2.530780in}{2.048829in}}%
\pgfpathlineto{\pgfqpoint{2.530780in}{2.044571in}}%
\pgfpathmoveto{\pgfqpoint{2.522265in}{2.048829in}}%
\pgfpathlineto{\pgfqpoint{2.522265in}{2.048829in}}%
\pgfpathlineto{\pgfqpoint{2.522265in}{2.053086in}}%
\pgfpathlineto{\pgfqpoint{2.526523in}{2.053086in}}%
\pgfpathlineto{\pgfqpoint{2.526523in}{2.048829in}}%
\pgfpathmoveto{\pgfqpoint{2.522265in}{2.053086in}}%
\pgfpathlineto{\pgfqpoint{2.522265in}{2.053086in}}%
\pgfpathlineto{\pgfqpoint{2.522265in}{2.057344in}}%
\pgfpathlineto{\pgfqpoint{2.526523in}{2.057344in}}%
\pgfpathlineto{\pgfqpoint{2.526523in}{2.053086in}}%
\pgfpathmoveto{\pgfqpoint{2.526523in}{2.048829in}}%
\pgfpathlineto{\pgfqpoint{2.526523in}{2.048829in}}%
\pgfpathlineto{\pgfqpoint{2.526523in}{2.053086in}}%
\pgfpathlineto{\pgfqpoint{2.530780in}{2.053086in}}%
\pgfpathlineto{\pgfqpoint{2.530780in}{2.048829in}}%
\pgfpathmoveto{\pgfqpoint{2.526523in}{2.053086in}}%
\pgfpathlineto{\pgfqpoint{2.526523in}{2.053086in}}%
\pgfpathlineto{\pgfqpoint{2.526523in}{2.057344in}}%
\pgfpathlineto{\pgfqpoint{2.530780in}{2.057344in}}%
\pgfpathlineto{\pgfqpoint{2.530780in}{2.053086in}}%
\pgfpathmoveto{\pgfqpoint{2.522265in}{2.057344in}}%
\pgfpathlineto{\pgfqpoint{2.522265in}{2.057344in}}%
\pgfpathlineto{\pgfqpoint{2.522265in}{2.061602in}}%
\pgfpathlineto{\pgfqpoint{2.526523in}{2.061602in}}%
\pgfpathlineto{\pgfqpoint{2.526523in}{2.057344in}}%
\pgfpathmoveto{\pgfqpoint{2.522265in}{2.061602in}}%
\pgfpathlineto{\pgfqpoint{2.522265in}{2.061602in}}%
\pgfpathlineto{\pgfqpoint{2.522265in}{2.065860in}}%
\pgfpathlineto{\pgfqpoint{2.526523in}{2.065860in}}%
\pgfpathlineto{\pgfqpoint{2.526523in}{2.061602in}}%
\pgfpathmoveto{\pgfqpoint{2.526523in}{2.057344in}}%
\pgfpathlineto{\pgfqpoint{2.526523in}{2.057344in}}%
\pgfpathlineto{\pgfqpoint{2.526523in}{2.061602in}}%
\pgfpathlineto{\pgfqpoint{2.530780in}{2.061602in}}%
\pgfpathlineto{\pgfqpoint{2.530780in}{2.057344in}}%
\pgfpathmoveto{\pgfqpoint{2.526523in}{2.061602in}}%
\pgfpathlineto{\pgfqpoint{2.526523in}{2.061602in}}%
\pgfpathlineto{\pgfqpoint{2.526523in}{2.065860in}}%
\pgfpathlineto{\pgfqpoint{2.530780in}{2.065860in}}%
\pgfpathlineto{\pgfqpoint{2.530780in}{2.061602in}}%
\pgfpathmoveto{\pgfqpoint{2.522265in}{2.065860in}}%
\pgfpathlineto{\pgfqpoint{2.522265in}{2.065860in}}%
\pgfpathlineto{\pgfqpoint{2.522265in}{2.070117in}}%
\pgfpathlineto{\pgfqpoint{2.526523in}{2.070117in}}%
\pgfpathlineto{\pgfqpoint{2.526523in}{2.065860in}}%
\pgfpathmoveto{\pgfqpoint{2.522265in}{2.070117in}}%
\pgfpathlineto{\pgfqpoint{2.522265in}{2.070117in}}%
\pgfpathlineto{\pgfqpoint{2.522265in}{2.074375in}}%
\pgfpathlineto{\pgfqpoint{2.526523in}{2.074375in}}%
\pgfpathlineto{\pgfqpoint{2.526523in}{2.070117in}}%
\pgfpathmoveto{\pgfqpoint{2.526523in}{2.065860in}}%
\pgfpathlineto{\pgfqpoint{2.526523in}{2.065860in}}%
\pgfpathlineto{\pgfqpoint{2.526523in}{2.070117in}}%
\pgfpathlineto{\pgfqpoint{2.530780in}{2.070117in}}%
\pgfpathlineto{\pgfqpoint{2.530780in}{2.065860in}}%
\pgfpathmoveto{\pgfqpoint{2.526523in}{2.070117in}}%
\pgfpathlineto{\pgfqpoint{2.526523in}{2.070117in}}%
\pgfpathlineto{\pgfqpoint{2.526523in}{2.074375in}}%
\pgfpathlineto{\pgfqpoint{2.530780in}{2.074375in}}%
\pgfpathlineto{\pgfqpoint{2.530780in}{2.070117in}}%
\pgfpathmoveto{\pgfqpoint{2.530780in}{2.061602in}}%
\pgfpathlineto{\pgfqpoint{2.530780in}{2.061602in}}%
\pgfpathlineto{\pgfqpoint{2.530780in}{2.065860in}}%
\pgfpathlineto{\pgfqpoint{2.535038in}{2.065860in}}%
\pgfpathlineto{\pgfqpoint{2.535038in}{2.061602in}}%
\pgfpathmoveto{\pgfqpoint{2.530780in}{2.065860in}}%
\pgfpathlineto{\pgfqpoint{2.530780in}{2.065860in}}%
\pgfpathlineto{\pgfqpoint{2.530780in}{2.070117in}}%
\pgfpathlineto{\pgfqpoint{2.535038in}{2.070117in}}%
\pgfpathlineto{\pgfqpoint{2.535038in}{2.065860in}}%
\pgfpathmoveto{\pgfqpoint{2.530780in}{2.070117in}}%
\pgfpathlineto{\pgfqpoint{2.530780in}{2.070117in}}%
\pgfpathlineto{\pgfqpoint{2.530780in}{2.074375in}}%
\pgfpathlineto{\pgfqpoint{2.535038in}{2.074375in}}%
\pgfpathlineto{\pgfqpoint{2.535038in}{2.070117in}}%
\pgfpathmoveto{\pgfqpoint{2.522265in}{2.074375in}}%
\pgfpathlineto{\pgfqpoint{2.522265in}{2.074375in}}%
\pgfpathlineto{\pgfqpoint{2.522265in}{2.078633in}}%
\pgfpathlineto{\pgfqpoint{2.526523in}{2.078633in}}%
\pgfpathlineto{\pgfqpoint{2.526523in}{2.074375in}}%
\pgfpathmoveto{\pgfqpoint{2.526523in}{2.074375in}}%
\pgfpathlineto{\pgfqpoint{2.526523in}{2.074375in}}%
\pgfpathlineto{\pgfqpoint{2.526523in}{2.078633in}}%
\pgfpathlineto{\pgfqpoint{2.530780in}{2.078633in}}%
\pgfpathlineto{\pgfqpoint{2.530780in}{2.074375in}}%
\pgfpathmoveto{\pgfqpoint{2.526523in}{2.078633in}}%
\pgfpathlineto{\pgfqpoint{2.526523in}{2.078633in}}%
\pgfpathlineto{\pgfqpoint{2.526523in}{2.082891in}}%
\pgfpathlineto{\pgfqpoint{2.530780in}{2.082891in}}%
\pgfpathlineto{\pgfqpoint{2.530780in}{2.078633in}}%
\pgfpathmoveto{\pgfqpoint{2.526523in}{2.082891in}}%
\pgfpathlineto{\pgfqpoint{2.526523in}{2.082891in}}%
\pgfpathlineto{\pgfqpoint{2.526523in}{2.087148in}}%
\pgfpathlineto{\pgfqpoint{2.530780in}{2.087148in}}%
\pgfpathlineto{\pgfqpoint{2.530780in}{2.082891in}}%
\pgfpathmoveto{\pgfqpoint{2.526523in}{2.087148in}}%
\pgfpathlineto{\pgfqpoint{2.526523in}{2.087148in}}%
\pgfpathlineto{\pgfqpoint{2.526523in}{2.091406in}}%
\pgfpathlineto{\pgfqpoint{2.530780in}{2.091406in}}%
\pgfpathlineto{\pgfqpoint{2.530780in}{2.087148in}}%
\pgfpathmoveto{\pgfqpoint{2.526523in}{2.091406in}}%
\pgfpathlineto{\pgfqpoint{2.526523in}{2.091406in}}%
\pgfpathlineto{\pgfqpoint{2.526523in}{2.095664in}}%
\pgfpathlineto{\pgfqpoint{2.530780in}{2.095664in}}%
\pgfpathlineto{\pgfqpoint{2.530780in}{2.091406in}}%
\pgfpathmoveto{\pgfqpoint{2.526523in}{2.095664in}}%
\pgfpathlineto{\pgfqpoint{2.526523in}{2.095664in}}%
\pgfpathlineto{\pgfqpoint{2.526523in}{2.099922in}}%
\pgfpathlineto{\pgfqpoint{2.530780in}{2.099922in}}%
\pgfpathlineto{\pgfqpoint{2.530780in}{2.095664in}}%
\pgfpathmoveto{\pgfqpoint{2.526523in}{2.099922in}}%
\pgfpathlineto{\pgfqpoint{2.526523in}{2.099922in}}%
\pgfpathlineto{\pgfqpoint{2.526523in}{2.104179in}}%
\pgfpathlineto{\pgfqpoint{2.530780in}{2.104179in}}%
\pgfpathlineto{\pgfqpoint{2.530780in}{2.099922in}}%
\pgfpathmoveto{\pgfqpoint{2.526523in}{2.104179in}}%
\pgfpathlineto{\pgfqpoint{2.526523in}{2.104179in}}%
\pgfpathlineto{\pgfqpoint{2.526523in}{2.108437in}}%
\pgfpathlineto{\pgfqpoint{2.530780in}{2.108437in}}%
\pgfpathlineto{\pgfqpoint{2.530780in}{2.104179in}}%
\pgfpathmoveto{\pgfqpoint{2.530780in}{2.074375in}}%
\pgfpathlineto{\pgfqpoint{2.530780in}{2.074375in}}%
\pgfpathlineto{\pgfqpoint{2.530780in}{2.078633in}}%
\pgfpathlineto{\pgfqpoint{2.535038in}{2.078633in}}%
\pgfpathlineto{\pgfqpoint{2.535038in}{2.074375in}}%
\pgfpathmoveto{\pgfqpoint{2.530780in}{2.078633in}}%
\pgfpathlineto{\pgfqpoint{2.530780in}{2.078633in}}%
\pgfpathlineto{\pgfqpoint{2.530780in}{2.082891in}}%
\pgfpathlineto{\pgfqpoint{2.535038in}{2.082891in}}%
\pgfpathlineto{\pgfqpoint{2.535038in}{2.078633in}}%
\pgfpathmoveto{\pgfqpoint{2.530780in}{2.082891in}}%
\pgfpathlineto{\pgfqpoint{2.530780in}{2.082891in}}%
\pgfpathlineto{\pgfqpoint{2.530780in}{2.087148in}}%
\pgfpathlineto{\pgfqpoint{2.535038in}{2.087148in}}%
\pgfpathlineto{\pgfqpoint{2.535038in}{2.082891in}}%
\pgfpathmoveto{\pgfqpoint{2.530780in}{2.087148in}}%
\pgfpathlineto{\pgfqpoint{2.530780in}{2.087148in}}%
\pgfpathlineto{\pgfqpoint{2.530780in}{2.091406in}}%
\pgfpathlineto{\pgfqpoint{2.535038in}{2.091406in}}%
\pgfpathlineto{\pgfqpoint{2.535038in}{2.087148in}}%
\pgfpathmoveto{\pgfqpoint{2.530780in}{2.091406in}}%
\pgfpathlineto{\pgfqpoint{2.530780in}{2.091406in}}%
\pgfpathlineto{\pgfqpoint{2.530780in}{2.095664in}}%
\pgfpathlineto{\pgfqpoint{2.535038in}{2.095664in}}%
\pgfpathlineto{\pgfqpoint{2.535038in}{2.091406in}}%
\pgfpathmoveto{\pgfqpoint{2.530780in}{2.095664in}}%
\pgfpathlineto{\pgfqpoint{2.530780in}{2.095664in}}%
\pgfpathlineto{\pgfqpoint{2.530780in}{2.099922in}}%
\pgfpathlineto{\pgfqpoint{2.535038in}{2.099922in}}%
\pgfpathlineto{\pgfqpoint{2.535038in}{2.095664in}}%
\pgfpathmoveto{\pgfqpoint{2.535038in}{2.095664in}}%
\pgfpathlineto{\pgfqpoint{2.535038in}{2.095664in}}%
\pgfpathlineto{\pgfqpoint{2.535038in}{2.099922in}}%
\pgfpathlineto{\pgfqpoint{2.539296in}{2.099922in}}%
\pgfpathlineto{\pgfqpoint{2.539296in}{2.095664in}}%
\pgfpathmoveto{\pgfqpoint{2.530780in}{2.099922in}}%
\pgfpathlineto{\pgfqpoint{2.530780in}{2.099922in}}%
\pgfpathlineto{\pgfqpoint{2.530780in}{2.104179in}}%
\pgfpathlineto{\pgfqpoint{2.535038in}{2.104179in}}%
\pgfpathlineto{\pgfqpoint{2.535038in}{2.099922in}}%
\pgfpathmoveto{\pgfqpoint{2.530780in}{2.104179in}}%
\pgfpathlineto{\pgfqpoint{2.530780in}{2.104179in}}%
\pgfpathlineto{\pgfqpoint{2.530780in}{2.108437in}}%
\pgfpathlineto{\pgfqpoint{2.535038in}{2.108437in}}%
\pgfpathlineto{\pgfqpoint{2.535038in}{2.104179in}}%
\pgfpathmoveto{\pgfqpoint{2.535038in}{2.099922in}}%
\pgfpathlineto{\pgfqpoint{2.535038in}{2.099922in}}%
\pgfpathlineto{\pgfqpoint{2.535038in}{2.104179in}}%
\pgfpathlineto{\pgfqpoint{2.539296in}{2.104179in}}%
\pgfpathlineto{\pgfqpoint{2.539296in}{2.099922in}}%
\pgfpathmoveto{\pgfqpoint{2.535038in}{2.104179in}}%
\pgfpathlineto{\pgfqpoint{2.535038in}{2.104179in}}%
\pgfpathlineto{\pgfqpoint{2.535038in}{2.108437in}}%
\pgfpathlineto{\pgfqpoint{2.539296in}{2.108437in}}%
\pgfpathlineto{\pgfqpoint{2.539296in}{2.104179in}}%
\pgfpathmoveto{\pgfqpoint{2.530780in}{2.108437in}}%
\pgfpathlineto{\pgfqpoint{2.530780in}{2.108437in}}%
\pgfpathlineto{\pgfqpoint{2.530780in}{2.112695in}}%
\pgfpathlineto{\pgfqpoint{2.535038in}{2.112695in}}%
\pgfpathlineto{\pgfqpoint{2.535038in}{2.108437in}}%
\pgfpathmoveto{\pgfqpoint{2.530780in}{2.112695in}}%
\pgfpathlineto{\pgfqpoint{2.530780in}{2.112695in}}%
\pgfpathlineto{\pgfqpoint{2.530780in}{2.116953in}}%
\pgfpathlineto{\pgfqpoint{2.535038in}{2.116953in}}%
\pgfpathlineto{\pgfqpoint{2.535038in}{2.112695in}}%
\pgfpathmoveto{\pgfqpoint{2.535038in}{2.108437in}}%
\pgfpathlineto{\pgfqpoint{2.535038in}{2.108437in}}%
\pgfpathlineto{\pgfqpoint{2.535038in}{2.112695in}}%
\pgfpathlineto{\pgfqpoint{2.539296in}{2.112695in}}%
\pgfpathlineto{\pgfqpoint{2.539296in}{2.108437in}}%
\pgfpathmoveto{\pgfqpoint{2.535038in}{2.112695in}}%
\pgfpathlineto{\pgfqpoint{2.535038in}{2.112695in}}%
\pgfpathlineto{\pgfqpoint{2.535038in}{2.116953in}}%
\pgfpathlineto{\pgfqpoint{2.539296in}{2.116953in}}%
\pgfpathlineto{\pgfqpoint{2.539296in}{2.112695in}}%
\pgfpathmoveto{\pgfqpoint{2.530780in}{2.116953in}}%
\pgfpathlineto{\pgfqpoint{2.530780in}{2.116953in}}%
\pgfpathlineto{\pgfqpoint{2.530780in}{2.121210in}}%
\pgfpathlineto{\pgfqpoint{2.535038in}{2.121210in}}%
\pgfpathlineto{\pgfqpoint{2.535038in}{2.116953in}}%
\pgfpathmoveto{\pgfqpoint{2.530780in}{2.121210in}}%
\pgfpathlineto{\pgfqpoint{2.530780in}{2.121210in}}%
\pgfpathlineto{\pgfqpoint{2.530780in}{2.125468in}}%
\pgfpathlineto{\pgfqpoint{2.535038in}{2.125468in}}%
\pgfpathlineto{\pgfqpoint{2.535038in}{2.121210in}}%
\pgfpathmoveto{\pgfqpoint{2.535038in}{2.116953in}}%
\pgfpathlineto{\pgfqpoint{2.535038in}{2.116953in}}%
\pgfpathlineto{\pgfqpoint{2.535038in}{2.121210in}}%
\pgfpathlineto{\pgfqpoint{2.539296in}{2.121210in}}%
\pgfpathlineto{\pgfqpoint{2.539296in}{2.116953in}}%
\pgfpathmoveto{\pgfqpoint{2.535038in}{2.121210in}}%
\pgfpathlineto{\pgfqpoint{2.535038in}{2.121210in}}%
\pgfpathlineto{\pgfqpoint{2.535038in}{2.125468in}}%
\pgfpathlineto{\pgfqpoint{2.539296in}{2.125468in}}%
\pgfpathlineto{\pgfqpoint{2.539296in}{2.121210in}}%
\pgfpathmoveto{\pgfqpoint{2.530780in}{2.125468in}}%
\pgfpathlineto{\pgfqpoint{2.530780in}{2.125468in}}%
\pgfpathlineto{\pgfqpoint{2.530780in}{2.129726in}}%
\pgfpathlineto{\pgfqpoint{2.535038in}{2.129726in}}%
\pgfpathlineto{\pgfqpoint{2.535038in}{2.125468in}}%
\pgfpathmoveto{\pgfqpoint{2.530780in}{2.129726in}}%
\pgfpathlineto{\pgfqpoint{2.530780in}{2.129726in}}%
\pgfpathlineto{\pgfqpoint{2.530780in}{2.133984in}}%
\pgfpathlineto{\pgfqpoint{2.535038in}{2.133984in}}%
\pgfpathlineto{\pgfqpoint{2.535038in}{2.129726in}}%
\pgfpathmoveto{\pgfqpoint{2.535038in}{2.125468in}}%
\pgfpathlineto{\pgfqpoint{2.535038in}{2.125468in}}%
\pgfpathlineto{\pgfqpoint{2.535038in}{2.129726in}}%
\pgfpathlineto{\pgfqpoint{2.539296in}{2.129726in}}%
\pgfpathlineto{\pgfqpoint{2.539296in}{2.125468in}}%
\pgfpathmoveto{\pgfqpoint{2.535038in}{2.129726in}}%
\pgfpathlineto{\pgfqpoint{2.535038in}{2.129726in}}%
\pgfpathlineto{\pgfqpoint{2.535038in}{2.133984in}}%
\pgfpathlineto{\pgfqpoint{2.539296in}{2.133984in}}%
\pgfpathlineto{\pgfqpoint{2.539296in}{2.129726in}}%
\pgfpathmoveto{\pgfqpoint{2.530780in}{2.133984in}}%
\pgfpathlineto{\pgfqpoint{2.530780in}{2.133984in}}%
\pgfpathlineto{\pgfqpoint{2.530780in}{2.138241in}}%
\pgfpathlineto{\pgfqpoint{2.535038in}{2.138241in}}%
\pgfpathlineto{\pgfqpoint{2.535038in}{2.133984in}}%
\pgfpathmoveto{\pgfqpoint{2.530780in}{2.138241in}}%
\pgfpathlineto{\pgfqpoint{2.530780in}{2.138241in}}%
\pgfpathlineto{\pgfqpoint{2.530780in}{2.142499in}}%
\pgfpathlineto{\pgfqpoint{2.535038in}{2.142499in}}%
\pgfpathlineto{\pgfqpoint{2.535038in}{2.138241in}}%
\pgfpathmoveto{\pgfqpoint{2.535038in}{2.133984in}}%
\pgfpathlineto{\pgfqpoint{2.535038in}{2.133984in}}%
\pgfpathlineto{\pgfqpoint{2.535038in}{2.138241in}}%
\pgfpathlineto{\pgfqpoint{2.539296in}{2.138241in}}%
\pgfpathlineto{\pgfqpoint{2.539296in}{2.133984in}}%
\pgfpathmoveto{\pgfqpoint{2.535038in}{2.138241in}}%
\pgfpathlineto{\pgfqpoint{2.535038in}{2.138241in}}%
\pgfpathlineto{\pgfqpoint{2.535038in}{2.142499in}}%
\pgfpathlineto{\pgfqpoint{2.539296in}{2.142499in}}%
\pgfpathlineto{\pgfqpoint{2.539296in}{2.138241in}}%
\pgfpathmoveto{\pgfqpoint{2.539296in}{2.129726in}}%
\pgfpathlineto{\pgfqpoint{2.539296in}{2.129726in}}%
\pgfpathlineto{\pgfqpoint{2.539296in}{2.133984in}}%
\pgfpathlineto{\pgfqpoint{2.543554in}{2.133984in}}%
\pgfpathlineto{\pgfqpoint{2.543554in}{2.129726in}}%
\pgfpathmoveto{\pgfqpoint{2.539296in}{2.133984in}}%
\pgfpathlineto{\pgfqpoint{2.539296in}{2.133984in}}%
\pgfpathlineto{\pgfqpoint{2.539296in}{2.138241in}}%
\pgfpathlineto{\pgfqpoint{2.543554in}{2.138241in}}%
\pgfpathlineto{\pgfqpoint{2.543554in}{2.133984in}}%
\pgfpathmoveto{\pgfqpoint{2.539296in}{2.138241in}}%
\pgfpathlineto{\pgfqpoint{2.539296in}{2.138241in}}%
\pgfpathlineto{\pgfqpoint{2.539296in}{2.142499in}}%
\pgfpathlineto{\pgfqpoint{2.543554in}{2.142499in}}%
\pgfpathlineto{\pgfqpoint{2.543554in}{2.138241in}}%
\pgfpathmoveto{\pgfqpoint{2.535038in}{2.142499in}}%
\pgfpathlineto{\pgfqpoint{2.535038in}{2.142499in}}%
\pgfpathlineto{\pgfqpoint{2.535038in}{2.146757in}}%
\pgfpathlineto{\pgfqpoint{2.539296in}{2.146757in}}%
\pgfpathlineto{\pgfqpoint{2.539296in}{2.142499in}}%
\pgfpathmoveto{\pgfqpoint{2.535038in}{2.146757in}}%
\pgfpathlineto{\pgfqpoint{2.535038in}{2.146757in}}%
\pgfpathlineto{\pgfqpoint{2.535038in}{2.151015in}}%
\pgfpathlineto{\pgfqpoint{2.539296in}{2.151015in}}%
\pgfpathlineto{\pgfqpoint{2.539296in}{2.146757in}}%
\pgfpathmoveto{\pgfqpoint{2.535038in}{2.151015in}}%
\pgfpathlineto{\pgfqpoint{2.535038in}{2.151015in}}%
\pgfpathlineto{\pgfqpoint{2.535038in}{2.155273in}}%
\pgfpathlineto{\pgfqpoint{2.539296in}{2.155273in}}%
\pgfpathlineto{\pgfqpoint{2.539296in}{2.151015in}}%
\pgfpathmoveto{\pgfqpoint{2.535038in}{2.155273in}}%
\pgfpathlineto{\pgfqpoint{2.535038in}{2.155273in}}%
\pgfpathlineto{\pgfqpoint{2.535038in}{2.159531in}}%
\pgfpathlineto{\pgfqpoint{2.539296in}{2.159531in}}%
\pgfpathlineto{\pgfqpoint{2.539296in}{2.155273in}}%
\pgfpathmoveto{\pgfqpoint{2.539296in}{2.142499in}}%
\pgfpathlineto{\pgfqpoint{2.539296in}{2.142499in}}%
\pgfpathlineto{\pgfqpoint{2.539296in}{2.146757in}}%
\pgfpathlineto{\pgfqpoint{2.543554in}{2.146757in}}%
\pgfpathlineto{\pgfqpoint{2.543554in}{2.142499in}}%
\pgfpathmoveto{\pgfqpoint{2.539296in}{2.146757in}}%
\pgfpathlineto{\pgfqpoint{2.539296in}{2.146757in}}%
\pgfpathlineto{\pgfqpoint{2.539296in}{2.151015in}}%
\pgfpathlineto{\pgfqpoint{2.543554in}{2.151015in}}%
\pgfpathlineto{\pgfqpoint{2.543554in}{2.146757in}}%
\pgfpathmoveto{\pgfqpoint{2.539296in}{2.151015in}}%
\pgfpathlineto{\pgfqpoint{2.539296in}{2.151015in}}%
\pgfpathlineto{\pgfqpoint{2.539296in}{2.155273in}}%
\pgfpathlineto{\pgfqpoint{2.543554in}{2.155273in}}%
\pgfpathlineto{\pgfqpoint{2.543554in}{2.151015in}}%
\pgfpathmoveto{\pgfqpoint{2.539296in}{2.155273in}}%
\pgfpathlineto{\pgfqpoint{2.539296in}{2.155273in}}%
\pgfpathlineto{\pgfqpoint{2.539296in}{2.159531in}}%
\pgfpathlineto{\pgfqpoint{2.543554in}{2.159531in}}%
\pgfpathlineto{\pgfqpoint{2.543554in}{2.155273in}}%
\pgfpathmoveto{\pgfqpoint{2.535038in}{2.159531in}}%
\pgfpathlineto{\pgfqpoint{2.535038in}{2.159531in}}%
\pgfpathlineto{\pgfqpoint{2.535038in}{2.163789in}}%
\pgfpathlineto{\pgfqpoint{2.539296in}{2.163789in}}%
\pgfpathlineto{\pgfqpoint{2.539296in}{2.159531in}}%
\pgfpathmoveto{\pgfqpoint{2.535038in}{2.163789in}}%
\pgfpathlineto{\pgfqpoint{2.535038in}{2.163789in}}%
\pgfpathlineto{\pgfqpoint{2.535038in}{2.168047in}}%
\pgfpathlineto{\pgfqpoint{2.539296in}{2.168047in}}%
\pgfpathlineto{\pgfqpoint{2.539296in}{2.163789in}}%
\pgfpathmoveto{\pgfqpoint{2.535038in}{2.168047in}}%
\pgfpathlineto{\pgfqpoint{2.535038in}{2.168047in}}%
\pgfpathlineto{\pgfqpoint{2.535038in}{2.172305in}}%
\pgfpathlineto{\pgfqpoint{2.539296in}{2.172305in}}%
\pgfpathlineto{\pgfqpoint{2.539296in}{2.168047in}}%
\pgfpathmoveto{\pgfqpoint{2.535038in}{2.172305in}}%
\pgfpathlineto{\pgfqpoint{2.535038in}{2.172305in}}%
\pgfpathlineto{\pgfqpoint{2.535038in}{2.176563in}}%
\pgfpathlineto{\pgfqpoint{2.539296in}{2.176563in}}%
\pgfpathlineto{\pgfqpoint{2.539296in}{2.172305in}}%
\pgfpathmoveto{\pgfqpoint{2.539296in}{2.159531in}}%
\pgfpathlineto{\pgfqpoint{2.539296in}{2.159531in}}%
\pgfpathlineto{\pgfqpoint{2.539296in}{2.163789in}}%
\pgfpathlineto{\pgfqpoint{2.543554in}{2.163789in}}%
\pgfpathlineto{\pgfqpoint{2.543554in}{2.159531in}}%
\pgfpathmoveto{\pgfqpoint{2.539296in}{2.163789in}}%
\pgfpathlineto{\pgfqpoint{2.539296in}{2.163789in}}%
\pgfpathlineto{\pgfqpoint{2.539296in}{2.168047in}}%
\pgfpathlineto{\pgfqpoint{2.543554in}{2.168047in}}%
\pgfpathlineto{\pgfqpoint{2.543554in}{2.163789in}}%
\pgfpathmoveto{\pgfqpoint{2.543554in}{2.163789in}}%
\pgfpathlineto{\pgfqpoint{2.543554in}{2.163789in}}%
\pgfpathlineto{\pgfqpoint{2.543554in}{2.168047in}}%
\pgfpathlineto{\pgfqpoint{2.547811in}{2.168047in}}%
\pgfpathlineto{\pgfqpoint{2.547811in}{2.163789in}}%
\pgfpathmoveto{\pgfqpoint{2.539296in}{2.168047in}}%
\pgfpathlineto{\pgfqpoint{2.539296in}{2.168047in}}%
\pgfpathlineto{\pgfqpoint{2.539296in}{2.172305in}}%
\pgfpathlineto{\pgfqpoint{2.543554in}{2.172305in}}%
\pgfpathlineto{\pgfqpoint{2.543554in}{2.168047in}}%
\pgfpathmoveto{\pgfqpoint{2.539296in}{2.172305in}}%
\pgfpathlineto{\pgfqpoint{2.539296in}{2.172305in}}%
\pgfpathlineto{\pgfqpoint{2.539296in}{2.176563in}}%
\pgfpathlineto{\pgfqpoint{2.543554in}{2.176563in}}%
\pgfpathlineto{\pgfqpoint{2.543554in}{2.172305in}}%
\pgfpathmoveto{\pgfqpoint{2.543554in}{2.168047in}}%
\pgfpathlineto{\pgfqpoint{2.543554in}{2.168047in}}%
\pgfpathlineto{\pgfqpoint{2.543554in}{2.172305in}}%
\pgfpathlineto{\pgfqpoint{2.547811in}{2.172305in}}%
\pgfpathlineto{\pgfqpoint{2.547811in}{2.168047in}}%
\pgfpathmoveto{\pgfqpoint{2.543554in}{2.172305in}}%
\pgfpathlineto{\pgfqpoint{2.543554in}{2.172305in}}%
\pgfpathlineto{\pgfqpoint{2.543554in}{2.176563in}}%
\pgfpathlineto{\pgfqpoint{2.547811in}{2.176563in}}%
\pgfpathlineto{\pgfqpoint{2.547811in}{2.172305in}}%
\pgfpathmoveto{\pgfqpoint{2.539296in}{2.176563in}}%
\pgfpathlineto{\pgfqpoint{2.539296in}{2.176563in}}%
\pgfpathlineto{\pgfqpoint{2.539296in}{2.180821in}}%
\pgfpathlineto{\pgfqpoint{2.543554in}{2.180821in}}%
\pgfpathlineto{\pgfqpoint{2.543554in}{2.176563in}}%
\pgfpathmoveto{\pgfqpoint{2.539296in}{2.180821in}}%
\pgfpathlineto{\pgfqpoint{2.539296in}{2.180821in}}%
\pgfpathlineto{\pgfqpoint{2.539296in}{2.185079in}}%
\pgfpathlineto{\pgfqpoint{2.543554in}{2.185079in}}%
\pgfpathlineto{\pgfqpoint{2.543554in}{2.180821in}}%
\pgfpathmoveto{\pgfqpoint{2.543554in}{2.176563in}}%
\pgfpathlineto{\pgfqpoint{2.543554in}{2.176563in}}%
\pgfpathlineto{\pgfqpoint{2.543554in}{2.180821in}}%
\pgfpathlineto{\pgfqpoint{2.547811in}{2.180821in}}%
\pgfpathlineto{\pgfqpoint{2.547811in}{2.176563in}}%
\pgfpathmoveto{\pgfqpoint{2.543554in}{2.180821in}}%
\pgfpathlineto{\pgfqpoint{2.543554in}{2.180821in}}%
\pgfpathlineto{\pgfqpoint{2.543554in}{2.185079in}}%
\pgfpathlineto{\pgfqpoint{2.547811in}{2.185079in}}%
\pgfpathlineto{\pgfqpoint{2.547811in}{2.180821in}}%
\pgfpathmoveto{\pgfqpoint{2.539296in}{2.185079in}}%
\pgfpathlineto{\pgfqpoint{2.539296in}{2.185079in}}%
\pgfpathlineto{\pgfqpoint{2.539296in}{2.189337in}}%
\pgfpathlineto{\pgfqpoint{2.543554in}{2.189337in}}%
\pgfpathlineto{\pgfqpoint{2.543554in}{2.185079in}}%
\pgfpathmoveto{\pgfqpoint{2.539296in}{2.189337in}}%
\pgfpathlineto{\pgfqpoint{2.539296in}{2.189337in}}%
\pgfpathlineto{\pgfqpoint{2.539296in}{2.193595in}}%
\pgfpathlineto{\pgfqpoint{2.543554in}{2.193595in}}%
\pgfpathlineto{\pgfqpoint{2.543554in}{2.189337in}}%
\pgfpathmoveto{\pgfqpoint{2.543554in}{2.185079in}}%
\pgfpathlineto{\pgfqpoint{2.543554in}{2.185079in}}%
\pgfpathlineto{\pgfqpoint{2.543554in}{2.189337in}}%
\pgfpathlineto{\pgfqpoint{2.547811in}{2.189337in}}%
\pgfpathlineto{\pgfqpoint{2.547811in}{2.185079in}}%
\pgfpathmoveto{\pgfqpoint{2.543554in}{2.189337in}}%
\pgfpathlineto{\pgfqpoint{2.543554in}{2.189337in}}%
\pgfpathlineto{\pgfqpoint{2.543554in}{2.193595in}}%
\pgfpathlineto{\pgfqpoint{2.547811in}{2.193595in}}%
\pgfpathlineto{\pgfqpoint{2.547811in}{2.189337in}}%
\pgfpathmoveto{\pgfqpoint{2.539296in}{2.193595in}}%
\pgfpathlineto{\pgfqpoint{2.539296in}{2.193595in}}%
\pgfpathlineto{\pgfqpoint{2.539296in}{2.197853in}}%
\pgfpathlineto{\pgfqpoint{2.543554in}{2.197853in}}%
\pgfpathlineto{\pgfqpoint{2.543554in}{2.193595in}}%
\pgfpathmoveto{\pgfqpoint{2.539296in}{2.197853in}}%
\pgfpathlineto{\pgfqpoint{2.539296in}{2.197853in}}%
\pgfpathlineto{\pgfqpoint{2.539296in}{2.202111in}}%
\pgfpathlineto{\pgfqpoint{2.543554in}{2.202111in}}%
\pgfpathlineto{\pgfqpoint{2.543554in}{2.197853in}}%
\pgfpathmoveto{\pgfqpoint{2.543554in}{2.193595in}}%
\pgfpathlineto{\pgfqpoint{2.543554in}{2.193595in}}%
\pgfpathlineto{\pgfqpoint{2.543554in}{2.197853in}}%
\pgfpathlineto{\pgfqpoint{2.547811in}{2.197853in}}%
\pgfpathlineto{\pgfqpoint{2.547811in}{2.193595in}}%
\pgfpathmoveto{\pgfqpoint{2.543554in}{2.197853in}}%
\pgfpathlineto{\pgfqpoint{2.543554in}{2.197853in}}%
\pgfpathlineto{\pgfqpoint{2.543554in}{2.202111in}}%
\pgfpathlineto{\pgfqpoint{2.547811in}{2.202111in}}%
\pgfpathlineto{\pgfqpoint{2.547811in}{2.197853in}}%
\pgfpathmoveto{\pgfqpoint{2.539296in}{2.202111in}}%
\pgfpathlineto{\pgfqpoint{2.539296in}{2.202111in}}%
\pgfpathlineto{\pgfqpoint{2.539296in}{2.206369in}}%
\pgfpathlineto{\pgfqpoint{2.543554in}{2.206369in}}%
\pgfpathlineto{\pgfqpoint{2.543554in}{2.202111in}}%
\pgfpathmoveto{\pgfqpoint{2.539296in}{2.206369in}}%
\pgfpathlineto{\pgfqpoint{2.539296in}{2.206369in}}%
\pgfpathlineto{\pgfqpoint{2.539296in}{2.210626in}}%
\pgfpathlineto{\pgfqpoint{2.543554in}{2.210626in}}%
\pgfpathlineto{\pgfqpoint{2.543554in}{2.206369in}}%
\pgfpathmoveto{\pgfqpoint{2.543554in}{2.202111in}}%
\pgfpathlineto{\pgfqpoint{2.543554in}{2.202111in}}%
\pgfpathlineto{\pgfqpoint{2.543554in}{2.206369in}}%
\pgfpathlineto{\pgfqpoint{2.547811in}{2.206369in}}%
\pgfpathlineto{\pgfqpoint{2.547811in}{2.202111in}}%
\pgfpathmoveto{\pgfqpoint{2.543554in}{2.206369in}}%
\pgfpathlineto{\pgfqpoint{2.543554in}{2.206369in}}%
\pgfpathlineto{\pgfqpoint{2.543554in}{2.210626in}}%
\pgfpathlineto{\pgfqpoint{2.547811in}{2.210626in}}%
\pgfpathlineto{\pgfqpoint{2.547811in}{2.206369in}}%
\pgfpathmoveto{\pgfqpoint{2.543554in}{2.210626in}}%
\pgfpathlineto{\pgfqpoint{2.543554in}{2.210626in}}%
\pgfpathlineto{\pgfqpoint{2.543554in}{2.214884in}}%
\pgfpathlineto{\pgfqpoint{2.547811in}{2.214884in}}%
\pgfpathlineto{\pgfqpoint{2.547811in}{2.210626in}}%
\pgfpathmoveto{\pgfqpoint{2.543554in}{2.214884in}}%
\pgfpathlineto{\pgfqpoint{2.543554in}{2.214884in}}%
\pgfpathlineto{\pgfqpoint{2.543554in}{2.219142in}}%
\pgfpathlineto{\pgfqpoint{2.547811in}{2.219142in}}%
\pgfpathlineto{\pgfqpoint{2.547811in}{2.214884in}}%
\pgfpathmoveto{\pgfqpoint{2.543554in}{2.219142in}}%
\pgfpathlineto{\pgfqpoint{2.543554in}{2.219142in}}%
\pgfpathlineto{\pgfqpoint{2.543554in}{2.223400in}}%
\pgfpathlineto{\pgfqpoint{2.547811in}{2.223400in}}%
\pgfpathlineto{\pgfqpoint{2.547811in}{2.219142in}}%
\pgfpathmoveto{\pgfqpoint{2.543554in}{2.223400in}}%
\pgfpathlineto{\pgfqpoint{2.543554in}{2.223400in}}%
\pgfpathlineto{\pgfqpoint{2.543554in}{2.227658in}}%
\pgfpathlineto{\pgfqpoint{2.547811in}{2.227658in}}%
\pgfpathlineto{\pgfqpoint{2.547811in}{2.223400in}}%
\pgfpathmoveto{\pgfqpoint{2.543554in}{2.227658in}}%
\pgfpathlineto{\pgfqpoint{2.543554in}{2.227658in}}%
\pgfpathlineto{\pgfqpoint{2.543554in}{2.231916in}}%
\pgfpathlineto{\pgfqpoint{2.547811in}{2.231916in}}%
\pgfpathlineto{\pgfqpoint{2.547811in}{2.227658in}}%
\pgfpathmoveto{\pgfqpoint{2.543554in}{2.231916in}}%
\pgfpathlineto{\pgfqpoint{2.543554in}{2.231916in}}%
\pgfpathlineto{\pgfqpoint{2.543554in}{2.236174in}}%
\pgfpathlineto{\pgfqpoint{2.547811in}{2.236174in}}%
\pgfpathlineto{\pgfqpoint{2.547811in}{2.231916in}}%
\pgfpathmoveto{\pgfqpoint{2.543554in}{2.236174in}}%
\pgfpathlineto{\pgfqpoint{2.543554in}{2.236174in}}%
\pgfpathlineto{\pgfqpoint{2.543554in}{2.240432in}}%
\pgfpathlineto{\pgfqpoint{2.547811in}{2.240432in}}%
\pgfpathlineto{\pgfqpoint{2.547811in}{2.236174in}}%
\pgfpathmoveto{\pgfqpoint{2.547811in}{2.193595in}}%
\pgfpathlineto{\pgfqpoint{2.547811in}{2.193595in}}%
\pgfpathlineto{\pgfqpoint{2.547811in}{2.197853in}}%
\pgfpathlineto{\pgfqpoint{2.552069in}{2.197853in}}%
\pgfpathlineto{\pgfqpoint{2.552069in}{2.193595in}}%
\pgfpathmoveto{\pgfqpoint{2.547811in}{2.197853in}}%
\pgfpathlineto{\pgfqpoint{2.547811in}{2.197853in}}%
\pgfpathlineto{\pgfqpoint{2.547811in}{2.202111in}}%
\pgfpathlineto{\pgfqpoint{2.552069in}{2.202111in}}%
\pgfpathlineto{\pgfqpoint{2.552069in}{2.197853in}}%
\pgfpathmoveto{\pgfqpoint{2.547811in}{2.202111in}}%
\pgfpathlineto{\pgfqpoint{2.547811in}{2.202111in}}%
\pgfpathlineto{\pgfqpoint{2.547811in}{2.206369in}}%
\pgfpathlineto{\pgfqpoint{2.552069in}{2.206369in}}%
\pgfpathlineto{\pgfqpoint{2.552069in}{2.202111in}}%
\pgfpathmoveto{\pgfqpoint{2.547811in}{2.206369in}}%
\pgfpathlineto{\pgfqpoint{2.547811in}{2.206369in}}%
\pgfpathlineto{\pgfqpoint{2.547811in}{2.210626in}}%
\pgfpathlineto{\pgfqpoint{2.552069in}{2.210626in}}%
\pgfpathlineto{\pgfqpoint{2.552069in}{2.206369in}}%
\pgfpathmoveto{\pgfqpoint{2.547811in}{2.210626in}}%
\pgfpathlineto{\pgfqpoint{2.547811in}{2.210626in}}%
\pgfpathlineto{\pgfqpoint{2.547811in}{2.214884in}}%
\pgfpathlineto{\pgfqpoint{2.552069in}{2.214884in}}%
\pgfpathlineto{\pgfqpoint{2.552069in}{2.210626in}}%
\pgfpathmoveto{\pgfqpoint{2.547811in}{2.214884in}}%
\pgfpathlineto{\pgfqpoint{2.547811in}{2.214884in}}%
\pgfpathlineto{\pgfqpoint{2.547811in}{2.219142in}}%
\pgfpathlineto{\pgfqpoint{2.552069in}{2.219142in}}%
\pgfpathlineto{\pgfqpoint{2.552069in}{2.214884in}}%
\pgfpathmoveto{\pgfqpoint{2.547811in}{2.219142in}}%
\pgfpathlineto{\pgfqpoint{2.547811in}{2.219142in}}%
\pgfpathlineto{\pgfqpoint{2.547811in}{2.223400in}}%
\pgfpathlineto{\pgfqpoint{2.552069in}{2.223400in}}%
\pgfpathlineto{\pgfqpoint{2.552069in}{2.219142in}}%
\pgfpathmoveto{\pgfqpoint{2.547811in}{2.223400in}}%
\pgfpathlineto{\pgfqpoint{2.547811in}{2.223400in}}%
\pgfpathlineto{\pgfqpoint{2.547811in}{2.227658in}}%
\pgfpathlineto{\pgfqpoint{2.552069in}{2.227658in}}%
\pgfpathlineto{\pgfqpoint{2.552069in}{2.223400in}}%
\pgfpathmoveto{\pgfqpoint{2.547811in}{2.227658in}}%
\pgfpathlineto{\pgfqpoint{2.547811in}{2.227658in}}%
\pgfpathlineto{\pgfqpoint{2.547811in}{2.231916in}}%
\pgfpathlineto{\pgfqpoint{2.552069in}{2.231916in}}%
\pgfpathlineto{\pgfqpoint{2.552069in}{2.227658in}}%
\pgfpathmoveto{\pgfqpoint{2.547811in}{2.231916in}}%
\pgfpathlineto{\pgfqpoint{2.547811in}{2.231916in}}%
\pgfpathlineto{\pgfqpoint{2.547811in}{2.236174in}}%
\pgfpathlineto{\pgfqpoint{2.552069in}{2.236174in}}%
\pgfpathlineto{\pgfqpoint{2.552069in}{2.231916in}}%
\pgfpathmoveto{\pgfqpoint{2.552069in}{2.227658in}}%
\pgfpathlineto{\pgfqpoint{2.552069in}{2.227658in}}%
\pgfpathlineto{\pgfqpoint{2.552069in}{2.231916in}}%
\pgfpathlineto{\pgfqpoint{2.556327in}{2.231916in}}%
\pgfpathlineto{\pgfqpoint{2.556327in}{2.227658in}}%
\pgfpathmoveto{\pgfqpoint{2.552069in}{2.231916in}}%
\pgfpathlineto{\pgfqpoint{2.552069in}{2.231916in}}%
\pgfpathlineto{\pgfqpoint{2.552069in}{2.236174in}}%
\pgfpathlineto{\pgfqpoint{2.556327in}{2.236174in}}%
\pgfpathlineto{\pgfqpoint{2.556327in}{2.231916in}}%
\pgfpathmoveto{\pgfqpoint{2.547811in}{2.236174in}}%
\pgfpathlineto{\pgfqpoint{2.547811in}{2.236174in}}%
\pgfpathlineto{\pgfqpoint{2.547811in}{2.240432in}}%
\pgfpathlineto{\pgfqpoint{2.552069in}{2.240432in}}%
\pgfpathlineto{\pgfqpoint{2.552069in}{2.236174in}}%
\pgfpathmoveto{\pgfqpoint{2.547811in}{2.240432in}}%
\pgfpathlineto{\pgfqpoint{2.547811in}{2.240432in}}%
\pgfpathlineto{\pgfqpoint{2.547811in}{2.244690in}}%
\pgfpathlineto{\pgfqpoint{2.552069in}{2.244690in}}%
\pgfpathlineto{\pgfqpoint{2.552069in}{2.240432in}}%
\pgfpathmoveto{\pgfqpoint{2.552069in}{2.236174in}}%
\pgfpathlineto{\pgfqpoint{2.552069in}{2.236174in}}%
\pgfpathlineto{\pgfqpoint{2.552069in}{2.240432in}}%
\pgfpathlineto{\pgfqpoint{2.556327in}{2.240432in}}%
\pgfpathlineto{\pgfqpoint{2.556327in}{2.236174in}}%
\pgfpathmoveto{\pgfqpoint{2.552069in}{2.240432in}}%
\pgfpathlineto{\pgfqpoint{2.552069in}{2.240432in}}%
\pgfpathlineto{\pgfqpoint{2.552069in}{2.244690in}}%
\pgfpathlineto{\pgfqpoint{2.556327in}{2.244690in}}%
\pgfpathlineto{\pgfqpoint{2.556327in}{2.240432in}}%
\pgfpathmoveto{\pgfqpoint{2.547811in}{2.244690in}}%
\pgfpathlineto{\pgfqpoint{2.547811in}{2.244690in}}%
\pgfpathlineto{\pgfqpoint{2.547811in}{2.248948in}}%
\pgfpathlineto{\pgfqpoint{2.552069in}{2.248948in}}%
\pgfpathlineto{\pgfqpoint{2.552069in}{2.244690in}}%
\pgfpathmoveto{\pgfqpoint{2.547811in}{2.248948in}}%
\pgfpathlineto{\pgfqpoint{2.547811in}{2.248948in}}%
\pgfpathlineto{\pgfqpoint{2.547811in}{2.253206in}}%
\pgfpathlineto{\pgfqpoint{2.552069in}{2.253206in}}%
\pgfpathlineto{\pgfqpoint{2.552069in}{2.248948in}}%
\pgfpathmoveto{\pgfqpoint{2.552069in}{2.244690in}}%
\pgfpathlineto{\pgfqpoint{2.552069in}{2.244690in}}%
\pgfpathlineto{\pgfqpoint{2.552069in}{2.248948in}}%
\pgfpathlineto{\pgfqpoint{2.556327in}{2.248948in}}%
\pgfpathlineto{\pgfqpoint{2.556327in}{2.244690in}}%
\pgfpathmoveto{\pgfqpoint{2.552069in}{2.248948in}}%
\pgfpathlineto{\pgfqpoint{2.552069in}{2.248948in}}%
\pgfpathlineto{\pgfqpoint{2.552069in}{2.253206in}}%
\pgfpathlineto{\pgfqpoint{2.556327in}{2.253206in}}%
\pgfpathlineto{\pgfqpoint{2.556327in}{2.248948in}}%
\pgfpathmoveto{\pgfqpoint{2.547811in}{2.253206in}}%
\pgfpathlineto{\pgfqpoint{2.547811in}{2.253206in}}%
\pgfpathlineto{\pgfqpoint{2.547811in}{2.257464in}}%
\pgfpathlineto{\pgfqpoint{2.552069in}{2.257464in}}%
\pgfpathlineto{\pgfqpoint{2.552069in}{2.253206in}}%
\pgfpathmoveto{\pgfqpoint{2.547811in}{2.257464in}}%
\pgfpathlineto{\pgfqpoint{2.547811in}{2.257464in}}%
\pgfpathlineto{\pgfqpoint{2.547811in}{2.261722in}}%
\pgfpathlineto{\pgfqpoint{2.552069in}{2.261722in}}%
\pgfpathlineto{\pgfqpoint{2.552069in}{2.257464in}}%
\pgfpathmoveto{\pgfqpoint{2.552069in}{2.253206in}}%
\pgfpathlineto{\pgfqpoint{2.552069in}{2.253206in}}%
\pgfpathlineto{\pgfqpoint{2.552069in}{2.257464in}}%
\pgfpathlineto{\pgfqpoint{2.556327in}{2.257464in}}%
\pgfpathlineto{\pgfqpoint{2.556327in}{2.253206in}}%
\pgfpathmoveto{\pgfqpoint{2.552069in}{2.257464in}}%
\pgfpathlineto{\pgfqpoint{2.552069in}{2.257464in}}%
\pgfpathlineto{\pgfqpoint{2.552069in}{2.261722in}}%
\pgfpathlineto{\pgfqpoint{2.556327in}{2.261722in}}%
\pgfpathlineto{\pgfqpoint{2.556327in}{2.257464in}}%
\pgfpathmoveto{\pgfqpoint{2.556327in}{2.257464in}}%
\pgfpathlineto{\pgfqpoint{2.556327in}{2.257464in}}%
\pgfpathlineto{\pgfqpoint{2.556327in}{2.261722in}}%
\pgfpathlineto{\pgfqpoint{2.560585in}{2.261722in}}%
\pgfpathlineto{\pgfqpoint{2.560585in}{2.257464in}}%
\pgfpathmoveto{\pgfqpoint{2.547811in}{2.261722in}}%
\pgfpathlineto{\pgfqpoint{2.547811in}{2.261722in}}%
\pgfpathlineto{\pgfqpoint{2.547811in}{2.265980in}}%
\pgfpathlineto{\pgfqpoint{2.552069in}{2.265980in}}%
\pgfpathlineto{\pgfqpoint{2.552069in}{2.261722in}}%
\pgfpathmoveto{\pgfqpoint{2.547811in}{2.265980in}}%
\pgfpathlineto{\pgfqpoint{2.547811in}{2.265980in}}%
\pgfpathlineto{\pgfqpoint{2.547811in}{2.270238in}}%
\pgfpathlineto{\pgfqpoint{2.552069in}{2.270238in}}%
\pgfpathlineto{\pgfqpoint{2.552069in}{2.265980in}}%
\pgfpathmoveto{\pgfqpoint{2.552069in}{2.261722in}}%
\pgfpathlineto{\pgfqpoint{2.552069in}{2.261722in}}%
\pgfpathlineto{\pgfqpoint{2.552069in}{2.265980in}}%
\pgfpathlineto{\pgfqpoint{2.556327in}{2.265980in}}%
\pgfpathlineto{\pgfqpoint{2.556327in}{2.261722in}}%
\pgfpathmoveto{\pgfqpoint{2.552069in}{2.265980in}}%
\pgfpathlineto{\pgfqpoint{2.552069in}{2.265980in}}%
\pgfpathlineto{\pgfqpoint{2.552069in}{2.270238in}}%
\pgfpathlineto{\pgfqpoint{2.556327in}{2.270238in}}%
\pgfpathlineto{\pgfqpoint{2.556327in}{2.265980in}}%
\pgfpathmoveto{\pgfqpoint{2.547811in}{2.270238in}}%
\pgfpathlineto{\pgfqpoint{2.547811in}{2.270238in}}%
\pgfpathlineto{\pgfqpoint{2.547811in}{2.274496in}}%
\pgfpathlineto{\pgfqpoint{2.552069in}{2.274496in}}%
\pgfpathlineto{\pgfqpoint{2.552069in}{2.270238in}}%
\pgfpathmoveto{\pgfqpoint{2.552069in}{2.270238in}}%
\pgfpathlineto{\pgfqpoint{2.552069in}{2.270238in}}%
\pgfpathlineto{\pgfqpoint{2.552069in}{2.274496in}}%
\pgfpathlineto{\pgfqpoint{2.556327in}{2.274496in}}%
\pgfpathlineto{\pgfqpoint{2.556327in}{2.270238in}}%
\pgfpathmoveto{\pgfqpoint{2.552069in}{2.274496in}}%
\pgfpathlineto{\pgfqpoint{2.552069in}{2.274496in}}%
\pgfpathlineto{\pgfqpoint{2.552069in}{2.278754in}}%
\pgfpathlineto{\pgfqpoint{2.556327in}{2.278754in}}%
\pgfpathlineto{\pgfqpoint{2.556327in}{2.274496in}}%
\pgfpathmoveto{\pgfqpoint{2.556327in}{2.261722in}}%
\pgfpathlineto{\pgfqpoint{2.556327in}{2.261722in}}%
\pgfpathlineto{\pgfqpoint{2.556327in}{2.265980in}}%
\pgfpathlineto{\pgfqpoint{2.560585in}{2.265980in}}%
\pgfpathlineto{\pgfqpoint{2.560585in}{2.261722in}}%
\pgfpathmoveto{\pgfqpoint{2.556327in}{2.265980in}}%
\pgfpathlineto{\pgfqpoint{2.556327in}{2.265980in}}%
\pgfpathlineto{\pgfqpoint{2.556327in}{2.270238in}}%
\pgfpathlineto{\pgfqpoint{2.560585in}{2.270238in}}%
\pgfpathlineto{\pgfqpoint{2.560585in}{2.265980in}}%
\pgfpathmoveto{\pgfqpoint{2.556327in}{2.270238in}}%
\pgfpathlineto{\pgfqpoint{2.556327in}{2.270238in}}%
\pgfpathlineto{\pgfqpoint{2.556327in}{2.274496in}}%
\pgfpathlineto{\pgfqpoint{2.560585in}{2.274496in}}%
\pgfpathlineto{\pgfqpoint{2.560585in}{2.270238in}}%
\pgfpathmoveto{\pgfqpoint{2.556327in}{2.274496in}}%
\pgfpathlineto{\pgfqpoint{2.556327in}{2.274496in}}%
\pgfpathlineto{\pgfqpoint{2.556327in}{2.278754in}}%
\pgfpathlineto{\pgfqpoint{2.560585in}{2.278754in}}%
\pgfpathlineto{\pgfqpoint{2.560585in}{2.274496in}}%
\pgfpathmoveto{\pgfqpoint{2.552069in}{2.278754in}}%
\pgfpathlineto{\pgfqpoint{2.552069in}{2.278754in}}%
\pgfpathlineto{\pgfqpoint{2.552069in}{2.283011in}}%
\pgfpathlineto{\pgfqpoint{2.556327in}{2.283011in}}%
\pgfpathlineto{\pgfqpoint{2.556327in}{2.278754in}}%
\pgfpathmoveto{\pgfqpoint{2.552069in}{2.283011in}}%
\pgfpathlineto{\pgfqpoint{2.552069in}{2.283011in}}%
\pgfpathlineto{\pgfqpoint{2.552069in}{2.287269in}}%
\pgfpathlineto{\pgfqpoint{2.556327in}{2.287269in}}%
\pgfpathlineto{\pgfqpoint{2.556327in}{2.283011in}}%
\pgfpathmoveto{\pgfqpoint{2.552069in}{2.287269in}}%
\pgfpathlineto{\pgfqpoint{2.552069in}{2.287269in}}%
\pgfpathlineto{\pgfqpoint{2.552069in}{2.291526in}}%
\pgfpathlineto{\pgfqpoint{2.556327in}{2.291526in}}%
\pgfpathlineto{\pgfqpoint{2.556327in}{2.287269in}}%
\pgfpathmoveto{\pgfqpoint{2.552069in}{2.291526in}}%
\pgfpathlineto{\pgfqpoint{2.552069in}{2.291526in}}%
\pgfpathlineto{\pgfqpoint{2.552069in}{2.295784in}}%
\pgfpathlineto{\pgfqpoint{2.556327in}{2.295784in}}%
\pgfpathlineto{\pgfqpoint{2.556327in}{2.291526in}}%
\pgfpathmoveto{\pgfqpoint{2.556327in}{2.278754in}}%
\pgfpathlineto{\pgfqpoint{2.556327in}{2.278754in}}%
\pgfpathlineto{\pgfqpoint{2.556327in}{2.283011in}}%
\pgfpathlineto{\pgfqpoint{2.560585in}{2.283011in}}%
\pgfpathlineto{\pgfqpoint{2.560585in}{2.278754in}}%
\pgfpathmoveto{\pgfqpoint{2.556327in}{2.283011in}}%
\pgfpathlineto{\pgfqpoint{2.556327in}{2.283011in}}%
\pgfpathlineto{\pgfqpoint{2.556327in}{2.287269in}}%
\pgfpathlineto{\pgfqpoint{2.560585in}{2.287269in}}%
\pgfpathlineto{\pgfqpoint{2.560585in}{2.283011in}}%
\pgfpathmoveto{\pgfqpoint{2.556327in}{2.287269in}}%
\pgfpathlineto{\pgfqpoint{2.556327in}{2.287269in}}%
\pgfpathlineto{\pgfqpoint{2.556327in}{2.291526in}}%
\pgfpathlineto{\pgfqpoint{2.560585in}{2.291526in}}%
\pgfpathlineto{\pgfqpoint{2.560585in}{2.287269in}}%
\pgfpathmoveto{\pgfqpoint{2.556327in}{2.291526in}}%
\pgfpathlineto{\pgfqpoint{2.556327in}{2.291526in}}%
\pgfpathlineto{\pgfqpoint{2.556327in}{2.295784in}}%
\pgfpathlineto{\pgfqpoint{2.560585in}{2.295784in}}%
\pgfpathlineto{\pgfqpoint{2.560585in}{2.291526in}}%
\pgfpathmoveto{\pgfqpoint{2.560585in}{2.287269in}}%
\pgfpathlineto{\pgfqpoint{2.560585in}{2.287269in}}%
\pgfpathlineto{\pgfqpoint{2.560585in}{2.291526in}}%
\pgfpathlineto{\pgfqpoint{2.564842in}{2.291526in}}%
\pgfpathlineto{\pgfqpoint{2.564842in}{2.287269in}}%
\pgfpathmoveto{\pgfqpoint{2.560585in}{2.291526in}}%
\pgfpathlineto{\pgfqpoint{2.560585in}{2.291526in}}%
\pgfpathlineto{\pgfqpoint{2.560585in}{2.295784in}}%
\pgfpathlineto{\pgfqpoint{2.564842in}{2.295784in}}%
\pgfpathlineto{\pgfqpoint{2.564842in}{2.291526in}}%
\pgfpathmoveto{\pgfqpoint{2.552069in}{2.295784in}}%
\pgfpathlineto{\pgfqpoint{2.552069in}{2.295784in}}%
\pgfpathlineto{\pgfqpoint{2.552069in}{2.300042in}}%
\pgfpathlineto{\pgfqpoint{2.556327in}{2.300042in}}%
\pgfpathlineto{\pgfqpoint{2.556327in}{2.295784in}}%
\pgfpathmoveto{\pgfqpoint{2.552069in}{2.300042in}}%
\pgfpathlineto{\pgfqpoint{2.552069in}{2.300042in}}%
\pgfpathlineto{\pgfqpoint{2.552069in}{2.304299in}}%
\pgfpathlineto{\pgfqpoint{2.556327in}{2.304299in}}%
\pgfpathlineto{\pgfqpoint{2.556327in}{2.300042in}}%
\pgfpathmoveto{\pgfqpoint{2.556327in}{2.295784in}}%
\pgfpathlineto{\pgfqpoint{2.556327in}{2.295784in}}%
\pgfpathlineto{\pgfqpoint{2.556327in}{2.300042in}}%
\pgfpathlineto{\pgfqpoint{2.560585in}{2.300042in}}%
\pgfpathlineto{\pgfqpoint{2.560585in}{2.295784in}}%
\pgfpathmoveto{\pgfqpoint{2.556327in}{2.300042in}}%
\pgfpathlineto{\pgfqpoint{2.556327in}{2.300042in}}%
\pgfpathlineto{\pgfqpoint{2.556327in}{2.304299in}}%
\pgfpathlineto{\pgfqpoint{2.560585in}{2.304299in}}%
\pgfpathlineto{\pgfqpoint{2.560585in}{2.300042in}}%
\pgfpathmoveto{\pgfqpoint{2.560585in}{2.295784in}}%
\pgfpathlineto{\pgfqpoint{2.560585in}{2.295784in}}%
\pgfpathlineto{\pgfqpoint{2.560585in}{2.300042in}}%
\pgfpathlineto{\pgfqpoint{2.564842in}{2.300042in}}%
\pgfpathlineto{\pgfqpoint{2.564842in}{2.295784in}}%
\pgfpathmoveto{\pgfqpoint{2.560585in}{2.300042in}}%
\pgfpathlineto{\pgfqpoint{2.560585in}{2.300042in}}%
\pgfpathlineto{\pgfqpoint{2.560585in}{2.304299in}}%
\pgfpathlineto{\pgfqpoint{2.564842in}{2.304299in}}%
\pgfpathlineto{\pgfqpoint{2.564842in}{2.300042in}}%
\pgfpathmoveto{\pgfqpoint{2.556327in}{2.304299in}}%
\pgfpathlineto{\pgfqpoint{2.556327in}{2.304299in}}%
\pgfpathlineto{\pgfqpoint{2.556327in}{2.308557in}}%
\pgfpathlineto{\pgfqpoint{2.560585in}{2.308557in}}%
\pgfpathlineto{\pgfqpoint{2.560585in}{2.304299in}}%
\pgfpathmoveto{\pgfqpoint{2.556327in}{2.308557in}}%
\pgfpathlineto{\pgfqpoint{2.556327in}{2.308557in}}%
\pgfpathlineto{\pgfqpoint{2.556327in}{2.312814in}}%
\pgfpathlineto{\pgfqpoint{2.560585in}{2.312814in}}%
\pgfpathlineto{\pgfqpoint{2.560585in}{2.308557in}}%
\pgfpathmoveto{\pgfqpoint{2.560585in}{2.304299in}}%
\pgfpathlineto{\pgfqpoint{2.560585in}{2.304299in}}%
\pgfpathlineto{\pgfqpoint{2.560585in}{2.308557in}}%
\pgfpathlineto{\pgfqpoint{2.564842in}{2.308557in}}%
\pgfpathlineto{\pgfqpoint{2.564842in}{2.304299in}}%
\pgfpathmoveto{\pgfqpoint{2.560585in}{2.308557in}}%
\pgfpathlineto{\pgfqpoint{2.560585in}{2.308557in}}%
\pgfpathlineto{\pgfqpoint{2.560585in}{2.312814in}}%
\pgfpathlineto{\pgfqpoint{2.564842in}{2.312814in}}%
\pgfpathlineto{\pgfqpoint{2.564842in}{2.308557in}}%
\pgfpathmoveto{\pgfqpoint{2.556327in}{2.312814in}}%
\pgfpathlineto{\pgfqpoint{2.556327in}{2.312814in}}%
\pgfpathlineto{\pgfqpoint{2.556327in}{2.317072in}}%
\pgfpathlineto{\pgfqpoint{2.560585in}{2.317072in}}%
\pgfpathlineto{\pgfqpoint{2.560585in}{2.312814in}}%
\pgfpathmoveto{\pgfqpoint{2.556327in}{2.317072in}}%
\pgfpathlineto{\pgfqpoint{2.556327in}{2.317072in}}%
\pgfpathlineto{\pgfqpoint{2.556327in}{2.321329in}}%
\pgfpathlineto{\pgfqpoint{2.560585in}{2.321329in}}%
\pgfpathlineto{\pgfqpoint{2.560585in}{2.317072in}}%
\pgfpathmoveto{\pgfqpoint{2.560585in}{2.312814in}}%
\pgfpathlineto{\pgfqpoint{2.560585in}{2.312814in}}%
\pgfpathlineto{\pgfqpoint{2.560585in}{2.317072in}}%
\pgfpathlineto{\pgfqpoint{2.564842in}{2.317072in}}%
\pgfpathlineto{\pgfqpoint{2.564842in}{2.312814in}}%
\pgfpathmoveto{\pgfqpoint{2.560585in}{2.317072in}}%
\pgfpathlineto{\pgfqpoint{2.560585in}{2.317072in}}%
\pgfpathlineto{\pgfqpoint{2.560585in}{2.321329in}}%
\pgfpathlineto{\pgfqpoint{2.564842in}{2.321329in}}%
\pgfpathlineto{\pgfqpoint{2.564842in}{2.317072in}}%
\pgfpathmoveto{\pgfqpoint{2.556327in}{2.321329in}}%
\pgfpathlineto{\pgfqpoint{2.556327in}{2.321329in}}%
\pgfpathlineto{\pgfqpoint{2.556327in}{2.325587in}}%
\pgfpathlineto{\pgfqpoint{2.560585in}{2.325587in}}%
\pgfpathlineto{\pgfqpoint{2.560585in}{2.321329in}}%
\pgfpathmoveto{\pgfqpoint{2.556327in}{2.325587in}}%
\pgfpathlineto{\pgfqpoint{2.556327in}{2.325587in}}%
\pgfpathlineto{\pgfqpoint{2.556327in}{2.329845in}}%
\pgfpathlineto{\pgfqpoint{2.560585in}{2.329845in}}%
\pgfpathlineto{\pgfqpoint{2.560585in}{2.325587in}}%
\pgfpathmoveto{\pgfqpoint{2.560585in}{2.321329in}}%
\pgfpathlineto{\pgfqpoint{2.560585in}{2.321329in}}%
\pgfpathlineto{\pgfqpoint{2.560585in}{2.325587in}}%
\pgfpathlineto{\pgfqpoint{2.564842in}{2.325587in}}%
\pgfpathlineto{\pgfqpoint{2.564842in}{2.321329in}}%
\pgfpathmoveto{\pgfqpoint{2.560585in}{2.325587in}}%
\pgfpathlineto{\pgfqpoint{2.560585in}{2.325587in}}%
\pgfpathlineto{\pgfqpoint{2.560585in}{2.329845in}}%
\pgfpathlineto{\pgfqpoint{2.564842in}{2.329845in}}%
\pgfpathlineto{\pgfqpoint{2.564842in}{2.325587in}}%
\pgfpathmoveto{\pgfqpoint{2.556327in}{2.329845in}}%
\pgfpathlineto{\pgfqpoint{2.556327in}{2.329845in}}%
\pgfpathlineto{\pgfqpoint{2.556327in}{2.334102in}}%
\pgfpathlineto{\pgfqpoint{2.560585in}{2.334102in}}%
\pgfpathlineto{\pgfqpoint{2.560585in}{2.329845in}}%
\pgfpathmoveto{\pgfqpoint{2.560585in}{2.329845in}}%
\pgfpathlineto{\pgfqpoint{2.560585in}{2.329845in}}%
\pgfpathlineto{\pgfqpoint{2.560585in}{2.334102in}}%
\pgfpathlineto{\pgfqpoint{2.564842in}{2.334102in}}%
\pgfpathlineto{\pgfqpoint{2.564842in}{2.329845in}}%
\pgfpathmoveto{\pgfqpoint{2.560585in}{2.334102in}}%
\pgfpathlineto{\pgfqpoint{2.560585in}{2.334102in}}%
\pgfpathlineto{\pgfqpoint{2.560585in}{2.338360in}}%
\pgfpathlineto{\pgfqpoint{2.564842in}{2.338360in}}%
\pgfpathlineto{\pgfqpoint{2.564842in}{2.334102in}}%
\pgfpathmoveto{\pgfqpoint{2.560585in}{2.338360in}}%
\pgfpathlineto{\pgfqpoint{2.560585in}{2.338360in}}%
\pgfpathlineto{\pgfqpoint{2.560585in}{2.342617in}}%
\pgfpathlineto{\pgfqpoint{2.564842in}{2.342617in}}%
\pgfpathlineto{\pgfqpoint{2.564842in}{2.338360in}}%
\pgfpathmoveto{\pgfqpoint{2.560585in}{2.342617in}}%
\pgfpathlineto{\pgfqpoint{2.560585in}{2.342617in}}%
\pgfpathlineto{\pgfqpoint{2.560585in}{2.346875in}}%
\pgfpathlineto{\pgfqpoint{2.564842in}{2.346875in}}%
\pgfpathlineto{\pgfqpoint{2.564842in}{2.342617in}}%
\pgfpathmoveto{\pgfqpoint{2.564842in}{2.317072in}}%
\pgfpathlineto{\pgfqpoint{2.564842in}{2.317072in}}%
\pgfpathlineto{\pgfqpoint{2.564842in}{2.321329in}}%
\pgfpathlineto{\pgfqpoint{2.569100in}{2.321329in}}%
\pgfpathlineto{\pgfqpoint{2.569100in}{2.317072in}}%
\pgfpathmoveto{\pgfqpoint{2.564842in}{2.321329in}}%
\pgfpathlineto{\pgfqpoint{2.564842in}{2.321329in}}%
\pgfpathlineto{\pgfqpoint{2.564842in}{2.325587in}}%
\pgfpathlineto{\pgfqpoint{2.569100in}{2.325587in}}%
\pgfpathlineto{\pgfqpoint{2.569100in}{2.321329in}}%
\pgfpathmoveto{\pgfqpoint{2.564842in}{2.325587in}}%
\pgfpathlineto{\pgfqpoint{2.564842in}{2.325587in}}%
\pgfpathlineto{\pgfqpoint{2.564842in}{2.329845in}}%
\pgfpathlineto{\pgfqpoint{2.569100in}{2.329845in}}%
\pgfpathlineto{\pgfqpoint{2.569100in}{2.325587in}}%
\pgfpathmoveto{\pgfqpoint{2.564842in}{2.329845in}}%
\pgfpathlineto{\pgfqpoint{2.564842in}{2.329845in}}%
\pgfpathlineto{\pgfqpoint{2.564842in}{2.334102in}}%
\pgfpathlineto{\pgfqpoint{2.569100in}{2.334102in}}%
\pgfpathlineto{\pgfqpoint{2.569100in}{2.329845in}}%
\pgfpathmoveto{\pgfqpoint{2.564842in}{2.334102in}}%
\pgfpathlineto{\pgfqpoint{2.564842in}{2.334102in}}%
\pgfpathlineto{\pgfqpoint{2.564842in}{2.338360in}}%
\pgfpathlineto{\pgfqpoint{2.569100in}{2.338360in}}%
\pgfpathlineto{\pgfqpoint{2.569100in}{2.334102in}}%
\pgfpathmoveto{\pgfqpoint{2.564842in}{2.338360in}}%
\pgfpathlineto{\pgfqpoint{2.564842in}{2.338360in}}%
\pgfpathlineto{\pgfqpoint{2.564842in}{2.342617in}}%
\pgfpathlineto{\pgfqpoint{2.569100in}{2.342617in}}%
\pgfpathlineto{\pgfqpoint{2.569100in}{2.338360in}}%
\pgfpathmoveto{\pgfqpoint{2.564842in}{2.342617in}}%
\pgfpathlineto{\pgfqpoint{2.564842in}{2.342617in}}%
\pgfpathlineto{\pgfqpoint{2.564842in}{2.346875in}}%
\pgfpathlineto{\pgfqpoint{2.569100in}{2.346875in}}%
\pgfpathlineto{\pgfqpoint{2.569100in}{2.342617in}}%
\pgfpathmoveto{\pgfqpoint{2.560585in}{2.346875in}}%
\pgfpathlineto{\pgfqpoint{2.560585in}{2.346875in}}%
\pgfpathlineto{\pgfqpoint{2.560585in}{2.351132in}}%
\pgfpathlineto{\pgfqpoint{2.564842in}{2.351132in}}%
\pgfpathlineto{\pgfqpoint{2.564842in}{2.346875in}}%
\pgfpathmoveto{\pgfqpoint{2.560585in}{2.351132in}}%
\pgfpathlineto{\pgfqpoint{2.560585in}{2.351132in}}%
\pgfpathlineto{\pgfqpoint{2.560585in}{2.355390in}}%
\pgfpathlineto{\pgfqpoint{2.564842in}{2.355390in}}%
\pgfpathlineto{\pgfqpoint{2.564842in}{2.351132in}}%
\pgfpathmoveto{\pgfqpoint{2.560585in}{2.355390in}}%
\pgfpathlineto{\pgfqpoint{2.560585in}{2.355390in}}%
\pgfpathlineto{\pgfqpoint{2.560585in}{2.359648in}}%
\pgfpathlineto{\pgfqpoint{2.564842in}{2.359648in}}%
\pgfpathlineto{\pgfqpoint{2.564842in}{2.355390in}}%
\pgfpathmoveto{\pgfqpoint{2.560585in}{2.359648in}}%
\pgfpathlineto{\pgfqpoint{2.560585in}{2.359648in}}%
\pgfpathlineto{\pgfqpoint{2.560585in}{2.363905in}}%
\pgfpathlineto{\pgfqpoint{2.564842in}{2.363905in}}%
\pgfpathlineto{\pgfqpoint{2.564842in}{2.359648in}}%
\pgfpathmoveto{\pgfqpoint{2.564842in}{2.346875in}}%
\pgfpathlineto{\pgfqpoint{2.564842in}{2.346875in}}%
\pgfpathlineto{\pgfqpoint{2.564842in}{2.351132in}}%
\pgfpathlineto{\pgfqpoint{2.569100in}{2.351132in}}%
\pgfpathlineto{\pgfqpoint{2.569100in}{2.346875in}}%
\pgfpathmoveto{\pgfqpoint{2.564842in}{2.351132in}}%
\pgfpathlineto{\pgfqpoint{2.564842in}{2.351132in}}%
\pgfpathlineto{\pgfqpoint{2.564842in}{2.355390in}}%
\pgfpathlineto{\pgfqpoint{2.569100in}{2.355390in}}%
\pgfpathlineto{\pgfqpoint{2.569100in}{2.351132in}}%
\pgfpathmoveto{\pgfqpoint{2.569100in}{2.351132in}}%
\pgfpathlineto{\pgfqpoint{2.569100in}{2.351132in}}%
\pgfpathlineto{\pgfqpoint{2.569100in}{2.355390in}}%
\pgfpathlineto{\pgfqpoint{2.573358in}{2.355390in}}%
\pgfpathlineto{\pgfqpoint{2.573358in}{2.351132in}}%
\pgfpathmoveto{\pgfqpoint{2.564842in}{2.355390in}}%
\pgfpathlineto{\pgfqpoint{2.564842in}{2.355390in}}%
\pgfpathlineto{\pgfqpoint{2.564842in}{2.359648in}}%
\pgfpathlineto{\pgfqpoint{2.569100in}{2.359648in}}%
\pgfpathlineto{\pgfqpoint{2.569100in}{2.355390in}}%
\pgfpathmoveto{\pgfqpoint{2.564842in}{2.359648in}}%
\pgfpathlineto{\pgfqpoint{2.564842in}{2.359648in}}%
\pgfpathlineto{\pgfqpoint{2.564842in}{2.363905in}}%
\pgfpathlineto{\pgfqpoint{2.569100in}{2.363905in}}%
\pgfpathlineto{\pgfqpoint{2.569100in}{2.359648in}}%
\pgfpathmoveto{\pgfqpoint{2.569100in}{2.355390in}}%
\pgfpathlineto{\pgfqpoint{2.569100in}{2.355390in}}%
\pgfpathlineto{\pgfqpoint{2.569100in}{2.359648in}}%
\pgfpathlineto{\pgfqpoint{2.573358in}{2.359648in}}%
\pgfpathlineto{\pgfqpoint{2.573358in}{2.355390in}}%
\pgfpathmoveto{\pgfqpoint{2.569100in}{2.359648in}}%
\pgfpathlineto{\pgfqpoint{2.569100in}{2.359648in}}%
\pgfpathlineto{\pgfqpoint{2.569100in}{2.363905in}}%
\pgfpathlineto{\pgfqpoint{2.573358in}{2.363905in}}%
\pgfpathlineto{\pgfqpoint{2.573358in}{2.359648in}}%
\pgfpathmoveto{\pgfqpoint{2.564842in}{2.363905in}}%
\pgfpathlineto{\pgfqpoint{2.564842in}{2.363905in}}%
\pgfpathlineto{\pgfqpoint{2.564842in}{2.368163in}}%
\pgfpathlineto{\pgfqpoint{2.569100in}{2.368163in}}%
\pgfpathlineto{\pgfqpoint{2.569100in}{2.363905in}}%
\pgfpathmoveto{\pgfqpoint{2.564842in}{2.368163in}}%
\pgfpathlineto{\pgfqpoint{2.564842in}{2.368163in}}%
\pgfpathlineto{\pgfqpoint{2.564842in}{2.372420in}}%
\pgfpathlineto{\pgfqpoint{2.569100in}{2.372420in}}%
\pgfpathlineto{\pgfqpoint{2.569100in}{2.368163in}}%
\pgfpathmoveto{\pgfqpoint{2.569100in}{2.363905in}}%
\pgfpathlineto{\pgfqpoint{2.569100in}{2.363905in}}%
\pgfpathlineto{\pgfqpoint{2.569100in}{2.368163in}}%
\pgfpathlineto{\pgfqpoint{2.573358in}{2.368163in}}%
\pgfpathlineto{\pgfqpoint{2.573358in}{2.363905in}}%
\pgfpathmoveto{\pgfqpoint{2.569100in}{2.368163in}}%
\pgfpathlineto{\pgfqpoint{2.569100in}{2.368163in}}%
\pgfpathlineto{\pgfqpoint{2.569100in}{2.372420in}}%
\pgfpathlineto{\pgfqpoint{2.573358in}{2.372420in}}%
\pgfpathlineto{\pgfqpoint{2.573358in}{2.368163in}}%
\pgfpathmoveto{\pgfqpoint{2.564842in}{2.372420in}}%
\pgfpathlineto{\pgfqpoint{2.564842in}{2.372420in}}%
\pgfpathlineto{\pgfqpoint{2.564842in}{2.376678in}}%
\pgfpathlineto{\pgfqpoint{2.569100in}{2.376678in}}%
\pgfpathlineto{\pgfqpoint{2.569100in}{2.372420in}}%
\pgfpathmoveto{\pgfqpoint{2.564842in}{2.376678in}}%
\pgfpathlineto{\pgfqpoint{2.564842in}{2.376678in}}%
\pgfpathlineto{\pgfqpoint{2.564842in}{2.380936in}}%
\pgfpathlineto{\pgfqpoint{2.569100in}{2.380936in}}%
\pgfpathlineto{\pgfqpoint{2.569100in}{2.376678in}}%
\pgfpathmoveto{\pgfqpoint{2.569100in}{2.372420in}}%
\pgfpathlineto{\pgfqpoint{2.569100in}{2.372420in}}%
\pgfpathlineto{\pgfqpoint{2.569100in}{2.376678in}}%
\pgfpathlineto{\pgfqpoint{2.573358in}{2.376678in}}%
\pgfpathlineto{\pgfqpoint{2.573358in}{2.372420in}}%
\pgfpathmoveto{\pgfqpoint{2.569100in}{2.376678in}}%
\pgfpathlineto{\pgfqpoint{2.569100in}{2.376678in}}%
\pgfpathlineto{\pgfqpoint{2.569100in}{2.380936in}}%
\pgfpathlineto{\pgfqpoint{2.573358in}{2.380936in}}%
\pgfpathlineto{\pgfqpoint{2.573358in}{2.376678in}}%
\pgfpathmoveto{\pgfqpoint{2.564842in}{2.380936in}}%
\pgfpathlineto{\pgfqpoint{2.564842in}{2.380936in}}%
\pgfpathlineto{\pgfqpoint{2.564842in}{2.385193in}}%
\pgfpathlineto{\pgfqpoint{2.569100in}{2.385193in}}%
\pgfpathlineto{\pgfqpoint{2.569100in}{2.380936in}}%
\pgfpathmoveto{\pgfqpoint{2.564842in}{2.385193in}}%
\pgfpathlineto{\pgfqpoint{2.564842in}{2.385193in}}%
\pgfpathlineto{\pgfqpoint{2.564842in}{2.389451in}}%
\pgfpathlineto{\pgfqpoint{2.569100in}{2.389451in}}%
\pgfpathlineto{\pgfqpoint{2.569100in}{2.385193in}}%
\pgfpathmoveto{\pgfqpoint{2.569100in}{2.380936in}}%
\pgfpathlineto{\pgfqpoint{2.569100in}{2.380936in}}%
\pgfpathlineto{\pgfqpoint{2.569100in}{2.385193in}}%
\pgfpathlineto{\pgfqpoint{2.573358in}{2.385193in}}%
\pgfpathlineto{\pgfqpoint{2.573358in}{2.380936in}}%
\pgfpathmoveto{\pgfqpoint{2.569100in}{2.385193in}}%
\pgfpathlineto{\pgfqpoint{2.569100in}{2.385193in}}%
\pgfpathlineto{\pgfqpoint{2.569100in}{2.389451in}}%
\pgfpathlineto{\pgfqpoint{2.573358in}{2.389451in}}%
\pgfpathlineto{\pgfqpoint{2.573358in}{2.385193in}}%
\pgfpathmoveto{\pgfqpoint{2.564842in}{2.389451in}}%
\pgfpathlineto{\pgfqpoint{2.564842in}{2.389451in}}%
\pgfpathlineto{\pgfqpoint{2.564842in}{2.393708in}}%
\pgfpathlineto{\pgfqpoint{2.569100in}{2.393708in}}%
\pgfpathlineto{\pgfqpoint{2.569100in}{2.389451in}}%
\pgfpathmoveto{\pgfqpoint{2.564842in}{2.393708in}}%
\pgfpathlineto{\pgfqpoint{2.564842in}{2.393708in}}%
\pgfpathlineto{\pgfqpoint{2.564842in}{2.397966in}}%
\pgfpathlineto{\pgfqpoint{2.569100in}{2.397966in}}%
\pgfpathlineto{\pgfqpoint{2.569100in}{2.393708in}}%
\pgfpathmoveto{\pgfqpoint{2.569100in}{2.389451in}}%
\pgfpathlineto{\pgfqpoint{2.569100in}{2.389451in}}%
\pgfpathlineto{\pgfqpoint{2.569100in}{2.393708in}}%
\pgfpathlineto{\pgfqpoint{2.573358in}{2.393708in}}%
\pgfpathlineto{\pgfqpoint{2.573358in}{2.389451in}}%
\pgfpathmoveto{\pgfqpoint{2.569100in}{2.393708in}}%
\pgfpathlineto{\pgfqpoint{2.569100in}{2.393708in}}%
\pgfpathlineto{\pgfqpoint{2.569100in}{2.397966in}}%
\pgfpathlineto{\pgfqpoint{2.573358in}{2.397966in}}%
\pgfpathlineto{\pgfqpoint{2.573358in}{2.393708in}}%
\pgfpathmoveto{\pgfqpoint{2.573358in}{2.380936in}}%
\pgfpathlineto{\pgfqpoint{2.573358in}{2.380936in}}%
\pgfpathlineto{\pgfqpoint{2.573358in}{2.385193in}}%
\pgfpathlineto{\pgfqpoint{2.577615in}{2.385193in}}%
\pgfpathlineto{\pgfqpoint{2.577615in}{2.380936in}}%
\pgfpathmoveto{\pgfqpoint{2.573358in}{2.385193in}}%
\pgfpathlineto{\pgfqpoint{2.573358in}{2.385193in}}%
\pgfpathlineto{\pgfqpoint{2.573358in}{2.389451in}}%
\pgfpathlineto{\pgfqpoint{2.577615in}{2.389451in}}%
\pgfpathlineto{\pgfqpoint{2.577615in}{2.385193in}}%
\pgfpathmoveto{\pgfqpoint{2.573358in}{2.389451in}}%
\pgfpathlineto{\pgfqpoint{2.573358in}{2.389451in}}%
\pgfpathlineto{\pgfqpoint{2.573358in}{2.393708in}}%
\pgfpathlineto{\pgfqpoint{2.577615in}{2.393708in}}%
\pgfpathlineto{\pgfqpoint{2.577615in}{2.389451in}}%
\pgfpathmoveto{\pgfqpoint{2.573358in}{2.393708in}}%
\pgfpathlineto{\pgfqpoint{2.573358in}{2.393708in}}%
\pgfpathlineto{\pgfqpoint{2.573358in}{2.397966in}}%
\pgfpathlineto{\pgfqpoint{2.577615in}{2.397966in}}%
\pgfpathlineto{\pgfqpoint{2.577615in}{2.393708in}}%
\pgfpathmoveto{\pgfqpoint{2.569100in}{2.397966in}}%
\pgfpathlineto{\pgfqpoint{2.569100in}{2.397966in}}%
\pgfpathlineto{\pgfqpoint{2.569100in}{2.402223in}}%
\pgfpathlineto{\pgfqpoint{2.573358in}{2.402223in}}%
\pgfpathlineto{\pgfqpoint{2.573358in}{2.397966in}}%
\pgfpathmoveto{\pgfqpoint{2.569100in}{2.402223in}}%
\pgfpathlineto{\pgfqpoint{2.569100in}{2.402223in}}%
\pgfpathlineto{\pgfqpoint{2.569100in}{2.406481in}}%
\pgfpathlineto{\pgfqpoint{2.573358in}{2.406481in}}%
\pgfpathlineto{\pgfqpoint{2.573358in}{2.402223in}}%
\pgfpathmoveto{\pgfqpoint{2.569100in}{2.406481in}}%
\pgfpathlineto{\pgfqpoint{2.569100in}{2.406481in}}%
\pgfpathlineto{\pgfqpoint{2.569100in}{2.410739in}}%
\pgfpathlineto{\pgfqpoint{2.573358in}{2.410739in}}%
\pgfpathlineto{\pgfqpoint{2.573358in}{2.406481in}}%
\pgfpathmoveto{\pgfqpoint{2.569100in}{2.410739in}}%
\pgfpathlineto{\pgfqpoint{2.569100in}{2.410739in}}%
\pgfpathlineto{\pgfqpoint{2.569100in}{2.414996in}}%
\pgfpathlineto{\pgfqpoint{2.573358in}{2.414996in}}%
\pgfpathlineto{\pgfqpoint{2.573358in}{2.410739in}}%
\pgfpathmoveto{\pgfqpoint{2.573358in}{2.397966in}}%
\pgfpathlineto{\pgfqpoint{2.573358in}{2.397966in}}%
\pgfpathlineto{\pgfqpoint{2.573358in}{2.402223in}}%
\pgfpathlineto{\pgfqpoint{2.577615in}{2.402223in}}%
\pgfpathlineto{\pgfqpoint{2.577615in}{2.397966in}}%
\pgfpathmoveto{\pgfqpoint{2.573358in}{2.402223in}}%
\pgfpathlineto{\pgfqpoint{2.573358in}{2.402223in}}%
\pgfpathlineto{\pgfqpoint{2.573358in}{2.406481in}}%
\pgfpathlineto{\pgfqpoint{2.577615in}{2.406481in}}%
\pgfpathlineto{\pgfqpoint{2.577615in}{2.402223in}}%
\pgfpathmoveto{\pgfqpoint{2.573358in}{2.406481in}}%
\pgfpathlineto{\pgfqpoint{2.573358in}{2.406481in}}%
\pgfpathlineto{\pgfqpoint{2.573358in}{2.410739in}}%
\pgfpathlineto{\pgfqpoint{2.577615in}{2.410739in}}%
\pgfpathlineto{\pgfqpoint{2.577615in}{2.406481in}}%
\pgfpathmoveto{\pgfqpoint{2.573358in}{2.410739in}}%
\pgfpathlineto{\pgfqpoint{2.573358in}{2.410739in}}%
\pgfpathlineto{\pgfqpoint{2.573358in}{2.414996in}}%
\pgfpathlineto{\pgfqpoint{2.577615in}{2.414996in}}%
\pgfpathlineto{\pgfqpoint{2.577615in}{2.410739in}}%
\pgfpathmoveto{\pgfqpoint{2.577615in}{2.410739in}}%
\pgfpathlineto{\pgfqpoint{2.577615in}{2.410739in}}%
\pgfpathlineto{\pgfqpoint{2.577615in}{2.414996in}}%
\pgfpathlineto{\pgfqpoint{2.581873in}{2.414996in}}%
\pgfpathlineto{\pgfqpoint{2.581873in}{2.410739in}}%
\pgfpathmoveto{\pgfqpoint{2.569100in}{2.414996in}}%
\pgfpathlineto{\pgfqpoint{2.569100in}{2.414996in}}%
\pgfpathlineto{\pgfqpoint{2.569100in}{2.419254in}}%
\pgfpathlineto{\pgfqpoint{2.573358in}{2.419254in}}%
\pgfpathlineto{\pgfqpoint{2.573358in}{2.414996in}}%
\pgfpathmoveto{\pgfqpoint{2.569100in}{2.419254in}}%
\pgfpathlineto{\pgfqpoint{2.569100in}{2.419254in}}%
\pgfpathlineto{\pgfqpoint{2.569100in}{2.423512in}}%
\pgfpathlineto{\pgfqpoint{2.573358in}{2.423512in}}%
\pgfpathlineto{\pgfqpoint{2.573358in}{2.419254in}}%
\pgfpathmoveto{\pgfqpoint{2.569100in}{2.423512in}}%
\pgfpathlineto{\pgfqpoint{2.569100in}{2.423512in}}%
\pgfpathlineto{\pgfqpoint{2.569100in}{2.427770in}}%
\pgfpathlineto{\pgfqpoint{2.573358in}{2.427770in}}%
\pgfpathlineto{\pgfqpoint{2.573358in}{2.423512in}}%
\pgfpathmoveto{\pgfqpoint{2.573358in}{2.414996in}}%
\pgfpathlineto{\pgfqpoint{2.573358in}{2.414996in}}%
\pgfpathlineto{\pgfqpoint{2.573358in}{2.419254in}}%
\pgfpathlineto{\pgfqpoint{2.577615in}{2.419254in}}%
\pgfpathlineto{\pgfqpoint{2.577615in}{2.414996in}}%
\pgfpathmoveto{\pgfqpoint{2.573358in}{2.419254in}}%
\pgfpathlineto{\pgfqpoint{2.573358in}{2.419254in}}%
\pgfpathlineto{\pgfqpoint{2.573358in}{2.423512in}}%
\pgfpathlineto{\pgfqpoint{2.577615in}{2.423512in}}%
\pgfpathlineto{\pgfqpoint{2.577615in}{2.419254in}}%
\pgfpathmoveto{\pgfqpoint{2.577615in}{2.414996in}}%
\pgfpathlineto{\pgfqpoint{2.577615in}{2.414996in}}%
\pgfpathlineto{\pgfqpoint{2.577615in}{2.419254in}}%
\pgfpathlineto{\pgfqpoint{2.581873in}{2.419254in}}%
\pgfpathlineto{\pgfqpoint{2.581873in}{2.414996in}}%
\pgfpathmoveto{\pgfqpoint{2.577615in}{2.419254in}}%
\pgfpathlineto{\pgfqpoint{2.577615in}{2.419254in}}%
\pgfpathlineto{\pgfqpoint{2.577615in}{2.423512in}}%
\pgfpathlineto{\pgfqpoint{2.581873in}{2.423512in}}%
\pgfpathlineto{\pgfqpoint{2.581873in}{2.419254in}}%
\pgfpathmoveto{\pgfqpoint{2.573358in}{2.423512in}}%
\pgfpathlineto{\pgfqpoint{2.573358in}{2.423512in}}%
\pgfpathlineto{\pgfqpoint{2.573358in}{2.427770in}}%
\pgfpathlineto{\pgfqpoint{2.577615in}{2.427770in}}%
\pgfpathlineto{\pgfqpoint{2.577615in}{2.423512in}}%
\pgfpathmoveto{\pgfqpoint{2.573358in}{2.427770in}}%
\pgfpathlineto{\pgfqpoint{2.573358in}{2.427770in}}%
\pgfpathlineto{\pgfqpoint{2.573358in}{2.432028in}}%
\pgfpathlineto{\pgfqpoint{2.577615in}{2.432028in}}%
\pgfpathlineto{\pgfqpoint{2.577615in}{2.427770in}}%
\pgfpathmoveto{\pgfqpoint{2.577615in}{2.423512in}}%
\pgfpathlineto{\pgfqpoint{2.577615in}{2.423512in}}%
\pgfpathlineto{\pgfqpoint{2.577615in}{2.427770in}}%
\pgfpathlineto{\pgfqpoint{2.581873in}{2.427770in}}%
\pgfpathlineto{\pgfqpoint{2.581873in}{2.423512in}}%
\pgfpathmoveto{\pgfqpoint{2.577615in}{2.427770in}}%
\pgfpathlineto{\pgfqpoint{2.577615in}{2.427770in}}%
\pgfpathlineto{\pgfqpoint{2.577615in}{2.432028in}}%
\pgfpathlineto{\pgfqpoint{2.581873in}{2.432028in}}%
\pgfpathlineto{\pgfqpoint{2.581873in}{2.427770in}}%
\pgfpathmoveto{\pgfqpoint{2.573358in}{2.432028in}}%
\pgfpathlineto{\pgfqpoint{2.573358in}{2.432028in}}%
\pgfpathlineto{\pgfqpoint{2.573358in}{2.436286in}}%
\pgfpathlineto{\pgfqpoint{2.577615in}{2.436286in}}%
\pgfpathlineto{\pgfqpoint{2.577615in}{2.432028in}}%
\pgfpathmoveto{\pgfqpoint{2.573358in}{2.436286in}}%
\pgfpathlineto{\pgfqpoint{2.573358in}{2.436286in}}%
\pgfpathlineto{\pgfqpoint{2.573358in}{2.440544in}}%
\pgfpathlineto{\pgfqpoint{2.577615in}{2.440544in}}%
\pgfpathlineto{\pgfqpoint{2.577615in}{2.436286in}}%
\pgfpathmoveto{\pgfqpoint{2.577615in}{2.432028in}}%
\pgfpathlineto{\pgfqpoint{2.577615in}{2.432028in}}%
\pgfpathlineto{\pgfqpoint{2.577615in}{2.436286in}}%
\pgfpathlineto{\pgfqpoint{2.581873in}{2.436286in}}%
\pgfpathlineto{\pgfqpoint{2.581873in}{2.432028in}}%
\pgfpathmoveto{\pgfqpoint{2.577615in}{2.436286in}}%
\pgfpathlineto{\pgfqpoint{2.577615in}{2.436286in}}%
\pgfpathlineto{\pgfqpoint{2.577615in}{2.440544in}}%
\pgfpathlineto{\pgfqpoint{2.581873in}{2.440544in}}%
\pgfpathlineto{\pgfqpoint{2.581873in}{2.436286in}}%
\pgfpathmoveto{\pgfqpoint{2.573358in}{2.440544in}}%
\pgfpathlineto{\pgfqpoint{2.573358in}{2.440544in}}%
\pgfpathlineto{\pgfqpoint{2.573358in}{2.444802in}}%
\pgfpathlineto{\pgfqpoint{2.577615in}{2.444802in}}%
\pgfpathlineto{\pgfqpoint{2.577615in}{2.440544in}}%
\pgfpathmoveto{\pgfqpoint{2.573358in}{2.444802in}}%
\pgfpathlineto{\pgfqpoint{2.573358in}{2.444802in}}%
\pgfpathlineto{\pgfqpoint{2.573358in}{2.449060in}}%
\pgfpathlineto{\pgfqpoint{2.577615in}{2.449060in}}%
\pgfpathlineto{\pgfqpoint{2.577615in}{2.444802in}}%
\pgfpathmoveto{\pgfqpoint{2.577615in}{2.440544in}}%
\pgfpathlineto{\pgfqpoint{2.577615in}{2.440544in}}%
\pgfpathlineto{\pgfqpoint{2.577615in}{2.444802in}}%
\pgfpathlineto{\pgfqpoint{2.581873in}{2.444802in}}%
\pgfpathlineto{\pgfqpoint{2.581873in}{2.440544in}}%
\pgfpathmoveto{\pgfqpoint{2.577615in}{2.444802in}}%
\pgfpathlineto{\pgfqpoint{2.577615in}{2.444802in}}%
\pgfpathlineto{\pgfqpoint{2.577615in}{2.449060in}}%
\pgfpathlineto{\pgfqpoint{2.581873in}{2.449060in}}%
\pgfpathlineto{\pgfqpoint{2.581873in}{2.444802in}}%
\pgfpathmoveto{\pgfqpoint{2.573358in}{2.449060in}}%
\pgfpathlineto{\pgfqpoint{2.573358in}{2.449060in}}%
\pgfpathlineto{\pgfqpoint{2.573358in}{2.453318in}}%
\pgfpathlineto{\pgfqpoint{2.577615in}{2.453318in}}%
\pgfpathlineto{\pgfqpoint{2.577615in}{2.449060in}}%
\pgfpathmoveto{\pgfqpoint{2.573358in}{2.453318in}}%
\pgfpathlineto{\pgfqpoint{2.573358in}{2.453318in}}%
\pgfpathlineto{\pgfqpoint{2.573358in}{2.457576in}}%
\pgfpathlineto{\pgfqpoint{2.577615in}{2.457576in}}%
\pgfpathlineto{\pgfqpoint{2.577615in}{2.453318in}}%
\pgfpathmoveto{\pgfqpoint{2.577615in}{2.449060in}}%
\pgfpathlineto{\pgfqpoint{2.577615in}{2.449060in}}%
\pgfpathlineto{\pgfqpoint{2.577615in}{2.453318in}}%
\pgfpathlineto{\pgfqpoint{2.581873in}{2.453318in}}%
\pgfpathlineto{\pgfqpoint{2.581873in}{2.449060in}}%
\pgfpathmoveto{\pgfqpoint{2.577615in}{2.453318in}}%
\pgfpathlineto{\pgfqpoint{2.577615in}{2.453318in}}%
\pgfpathlineto{\pgfqpoint{2.577615in}{2.457576in}}%
\pgfpathlineto{\pgfqpoint{2.581873in}{2.457576in}}%
\pgfpathlineto{\pgfqpoint{2.581873in}{2.453318in}}%
\pgfpathmoveto{\pgfqpoint{2.577615in}{2.457576in}}%
\pgfpathlineto{\pgfqpoint{2.577615in}{2.457576in}}%
\pgfpathlineto{\pgfqpoint{2.577615in}{2.461834in}}%
\pgfpathlineto{\pgfqpoint{2.581873in}{2.461834in}}%
\pgfpathlineto{\pgfqpoint{2.581873in}{2.457576in}}%
\pgfpathmoveto{\pgfqpoint{2.577615in}{2.461834in}}%
\pgfpathlineto{\pgfqpoint{2.577615in}{2.461834in}}%
\pgfpathlineto{\pgfqpoint{2.577615in}{2.466092in}}%
\pgfpathlineto{\pgfqpoint{2.581873in}{2.466092in}}%
\pgfpathlineto{\pgfqpoint{2.581873in}{2.461834in}}%
\pgfpathmoveto{\pgfqpoint{2.577615in}{2.466092in}}%
\pgfpathlineto{\pgfqpoint{2.577615in}{2.466092in}}%
\pgfpathlineto{\pgfqpoint{2.577615in}{2.470350in}}%
\pgfpathlineto{\pgfqpoint{2.581873in}{2.470350in}}%
\pgfpathlineto{\pgfqpoint{2.581873in}{2.466092in}}%
\pgfpathmoveto{\pgfqpoint{2.577615in}{2.470350in}}%
\pgfpathlineto{\pgfqpoint{2.577615in}{2.470350in}}%
\pgfpathlineto{\pgfqpoint{2.577615in}{2.474608in}}%
\pgfpathlineto{\pgfqpoint{2.581873in}{2.474608in}}%
\pgfpathlineto{\pgfqpoint{2.581873in}{2.470350in}}%
\pgfpathmoveto{\pgfqpoint{2.577615in}{2.474608in}}%
\pgfpathlineto{\pgfqpoint{2.577615in}{2.474608in}}%
\pgfpathlineto{\pgfqpoint{2.577615in}{2.478866in}}%
\pgfpathlineto{\pgfqpoint{2.581873in}{2.478866in}}%
\pgfpathlineto{\pgfqpoint{2.581873in}{2.474608in}}%
\pgfpathmoveto{\pgfqpoint{2.577615in}{2.478866in}}%
\pgfpathlineto{\pgfqpoint{2.577615in}{2.478866in}}%
\pgfpathlineto{\pgfqpoint{2.577615in}{2.483124in}}%
\pgfpathlineto{\pgfqpoint{2.581873in}{2.483124in}}%
\pgfpathlineto{\pgfqpoint{2.581873in}{2.478866in}}%
\pgfpathmoveto{\pgfqpoint{2.581873in}{2.440544in}}%
\pgfpathlineto{\pgfqpoint{2.581873in}{2.440544in}}%
\pgfpathlineto{\pgfqpoint{2.581873in}{2.444802in}}%
\pgfpathlineto{\pgfqpoint{2.586131in}{2.444802in}}%
\pgfpathlineto{\pgfqpoint{2.586131in}{2.440544in}}%
\pgfpathmoveto{\pgfqpoint{2.581873in}{2.444802in}}%
\pgfpathlineto{\pgfqpoint{2.581873in}{2.444802in}}%
\pgfpathlineto{\pgfqpoint{2.581873in}{2.449060in}}%
\pgfpathlineto{\pgfqpoint{2.586131in}{2.449060in}}%
\pgfpathlineto{\pgfqpoint{2.586131in}{2.444802in}}%
\pgfpathmoveto{\pgfqpoint{2.581873in}{2.449060in}}%
\pgfpathlineto{\pgfqpoint{2.581873in}{2.449060in}}%
\pgfpathlineto{\pgfqpoint{2.581873in}{2.453318in}}%
\pgfpathlineto{\pgfqpoint{2.586131in}{2.453318in}}%
\pgfpathlineto{\pgfqpoint{2.586131in}{2.449060in}}%
\pgfpathmoveto{\pgfqpoint{2.581873in}{2.453318in}}%
\pgfpathlineto{\pgfqpoint{2.581873in}{2.453318in}}%
\pgfpathlineto{\pgfqpoint{2.581873in}{2.457576in}}%
\pgfpathlineto{\pgfqpoint{2.586131in}{2.457576in}}%
\pgfpathlineto{\pgfqpoint{2.586131in}{2.453318in}}%
\pgfpathmoveto{\pgfqpoint{2.581873in}{2.457576in}}%
\pgfpathlineto{\pgfqpoint{2.581873in}{2.457576in}}%
\pgfpathlineto{\pgfqpoint{2.581873in}{2.461834in}}%
\pgfpathlineto{\pgfqpoint{2.586131in}{2.461834in}}%
\pgfpathlineto{\pgfqpoint{2.586131in}{2.457576in}}%
\pgfpathmoveto{\pgfqpoint{2.581873in}{2.461834in}}%
\pgfpathlineto{\pgfqpoint{2.581873in}{2.461834in}}%
\pgfpathlineto{\pgfqpoint{2.581873in}{2.466092in}}%
\pgfpathlineto{\pgfqpoint{2.586131in}{2.466092in}}%
\pgfpathlineto{\pgfqpoint{2.586131in}{2.461834in}}%
\pgfpathmoveto{\pgfqpoint{2.581873in}{2.466092in}}%
\pgfpathlineto{\pgfqpoint{2.581873in}{2.466092in}}%
\pgfpathlineto{\pgfqpoint{2.581873in}{2.470350in}}%
\pgfpathlineto{\pgfqpoint{2.586131in}{2.470350in}}%
\pgfpathlineto{\pgfqpoint{2.586131in}{2.466092in}}%
\pgfpathmoveto{\pgfqpoint{2.581873in}{2.470350in}}%
\pgfpathlineto{\pgfqpoint{2.581873in}{2.470350in}}%
\pgfpathlineto{\pgfqpoint{2.581873in}{2.474608in}}%
\pgfpathlineto{\pgfqpoint{2.586131in}{2.474608in}}%
\pgfpathlineto{\pgfqpoint{2.586131in}{2.470350in}}%
\pgfpathmoveto{\pgfqpoint{2.586131in}{2.466092in}}%
\pgfpathlineto{\pgfqpoint{2.586131in}{2.466092in}}%
\pgfpathlineto{\pgfqpoint{2.586131in}{2.470350in}}%
\pgfpathlineto{\pgfqpoint{2.590389in}{2.470350in}}%
\pgfpathlineto{\pgfqpoint{2.590389in}{2.466092in}}%
\pgfpathmoveto{\pgfqpoint{2.586131in}{2.470350in}}%
\pgfpathlineto{\pgfqpoint{2.586131in}{2.470350in}}%
\pgfpathlineto{\pgfqpoint{2.586131in}{2.474608in}}%
\pgfpathlineto{\pgfqpoint{2.590389in}{2.474608in}}%
\pgfpathlineto{\pgfqpoint{2.590389in}{2.470350in}}%
\pgfpathmoveto{\pgfqpoint{2.581873in}{2.474608in}}%
\pgfpathlineto{\pgfqpoint{2.581873in}{2.474608in}}%
\pgfpathlineto{\pgfqpoint{2.581873in}{2.478866in}}%
\pgfpathlineto{\pgfqpoint{2.586131in}{2.478866in}}%
\pgfpathlineto{\pgfqpoint{2.586131in}{2.474608in}}%
\pgfpathmoveto{\pgfqpoint{2.581873in}{2.478866in}}%
\pgfpathlineto{\pgfqpoint{2.581873in}{2.478866in}}%
\pgfpathlineto{\pgfqpoint{2.581873in}{2.483124in}}%
\pgfpathlineto{\pgfqpoint{2.586131in}{2.483124in}}%
\pgfpathlineto{\pgfqpoint{2.586131in}{2.478866in}}%
\pgfpathmoveto{\pgfqpoint{2.586131in}{2.474608in}}%
\pgfpathlineto{\pgfqpoint{2.586131in}{2.474608in}}%
\pgfpathlineto{\pgfqpoint{2.586131in}{2.478866in}}%
\pgfpathlineto{\pgfqpoint{2.590389in}{2.478866in}}%
\pgfpathlineto{\pgfqpoint{2.590389in}{2.474608in}}%
\pgfpathmoveto{\pgfqpoint{2.586131in}{2.478866in}}%
\pgfpathlineto{\pgfqpoint{2.586131in}{2.478866in}}%
\pgfpathlineto{\pgfqpoint{2.586131in}{2.483124in}}%
\pgfpathlineto{\pgfqpoint{2.590389in}{2.483124in}}%
\pgfpathlineto{\pgfqpoint{2.590389in}{2.478866in}}%
\pgfpathmoveto{\pgfqpoint{2.577615in}{2.483124in}}%
\pgfpathlineto{\pgfqpoint{2.577615in}{2.483124in}}%
\pgfpathlineto{\pgfqpoint{2.577615in}{2.487382in}}%
\pgfpathlineto{\pgfqpoint{2.581873in}{2.487382in}}%
\pgfpathlineto{\pgfqpoint{2.581873in}{2.483124in}}%
\pgfpathmoveto{\pgfqpoint{2.581873in}{2.483124in}}%
\pgfpathlineto{\pgfqpoint{2.581873in}{2.483124in}}%
\pgfpathlineto{\pgfqpoint{2.581873in}{2.487382in}}%
\pgfpathlineto{\pgfqpoint{2.586131in}{2.487382in}}%
\pgfpathlineto{\pgfqpoint{2.586131in}{2.483124in}}%
\pgfpathmoveto{\pgfqpoint{2.581873in}{2.487382in}}%
\pgfpathlineto{\pgfqpoint{2.581873in}{2.487382in}}%
\pgfpathlineto{\pgfqpoint{2.581873in}{2.491640in}}%
\pgfpathlineto{\pgfqpoint{2.586131in}{2.491640in}}%
\pgfpathlineto{\pgfqpoint{2.586131in}{2.487382in}}%
\pgfpathmoveto{\pgfqpoint{2.586131in}{2.483124in}}%
\pgfpathlineto{\pgfqpoint{2.586131in}{2.483124in}}%
\pgfpathlineto{\pgfqpoint{2.586131in}{2.487382in}}%
\pgfpathlineto{\pgfqpoint{2.590389in}{2.487382in}}%
\pgfpathlineto{\pgfqpoint{2.590389in}{2.483124in}}%
\pgfpathmoveto{\pgfqpoint{2.586131in}{2.487382in}}%
\pgfpathlineto{\pgfqpoint{2.586131in}{2.487382in}}%
\pgfpathlineto{\pgfqpoint{2.586131in}{2.491640in}}%
\pgfpathlineto{\pgfqpoint{2.590389in}{2.491640in}}%
\pgfpathlineto{\pgfqpoint{2.590389in}{2.487382in}}%
\pgfpathmoveto{\pgfqpoint{2.581873in}{2.491640in}}%
\pgfpathlineto{\pgfqpoint{2.581873in}{2.491640in}}%
\pgfpathlineto{\pgfqpoint{2.581873in}{2.495898in}}%
\pgfpathlineto{\pgfqpoint{2.586131in}{2.495898in}}%
\pgfpathlineto{\pgfqpoint{2.586131in}{2.491640in}}%
\pgfpathmoveto{\pgfqpoint{2.581873in}{2.495898in}}%
\pgfpathlineto{\pgfqpoint{2.581873in}{2.495898in}}%
\pgfpathlineto{\pgfqpoint{2.581873in}{2.500156in}}%
\pgfpathlineto{\pgfqpoint{2.586131in}{2.500156in}}%
\pgfpathlineto{\pgfqpoint{2.586131in}{2.495898in}}%
\pgfpathmoveto{\pgfqpoint{2.586131in}{2.491640in}}%
\pgfpathlineto{\pgfqpoint{2.586131in}{2.491640in}}%
\pgfpathlineto{\pgfqpoint{2.586131in}{2.495898in}}%
\pgfpathlineto{\pgfqpoint{2.590389in}{2.495898in}}%
\pgfpathlineto{\pgfqpoint{2.590389in}{2.491640in}}%
\pgfpathmoveto{\pgfqpoint{2.586131in}{2.495898in}}%
\pgfpathlineto{\pgfqpoint{2.586131in}{2.495898in}}%
\pgfpathlineto{\pgfqpoint{2.586131in}{2.500156in}}%
\pgfpathlineto{\pgfqpoint{2.590389in}{2.500156in}}%
\pgfpathlineto{\pgfqpoint{2.590389in}{2.495898in}}%
\pgfpathmoveto{\pgfqpoint{2.590389in}{2.495898in}}%
\pgfpathlineto{\pgfqpoint{2.590389in}{2.495898in}}%
\pgfpathlineto{\pgfqpoint{2.590389in}{2.500156in}}%
\pgfpathlineto{\pgfqpoint{2.594646in}{2.500156in}}%
\pgfpathlineto{\pgfqpoint{2.594646in}{2.495898in}}%
\pgfpathmoveto{\pgfqpoint{2.581873in}{2.500156in}}%
\pgfpathlineto{\pgfqpoint{2.581873in}{2.500156in}}%
\pgfpathlineto{\pgfqpoint{2.581873in}{2.504414in}}%
\pgfpathlineto{\pgfqpoint{2.586131in}{2.504414in}}%
\pgfpathlineto{\pgfqpoint{2.586131in}{2.500156in}}%
\pgfpathmoveto{\pgfqpoint{2.581873in}{2.504414in}}%
\pgfpathlineto{\pgfqpoint{2.581873in}{2.504414in}}%
\pgfpathlineto{\pgfqpoint{2.581873in}{2.508672in}}%
\pgfpathlineto{\pgfqpoint{2.586131in}{2.508672in}}%
\pgfpathlineto{\pgfqpoint{2.586131in}{2.504414in}}%
\pgfpathmoveto{\pgfqpoint{2.586131in}{2.500156in}}%
\pgfpathlineto{\pgfqpoint{2.586131in}{2.500156in}}%
\pgfpathlineto{\pgfqpoint{2.586131in}{2.504414in}}%
\pgfpathlineto{\pgfqpoint{2.590389in}{2.504414in}}%
\pgfpathlineto{\pgfqpoint{2.590389in}{2.500156in}}%
\pgfpathmoveto{\pgfqpoint{2.586131in}{2.504414in}}%
\pgfpathlineto{\pgfqpoint{2.586131in}{2.504414in}}%
\pgfpathlineto{\pgfqpoint{2.586131in}{2.508672in}}%
\pgfpathlineto{\pgfqpoint{2.590389in}{2.508672in}}%
\pgfpathlineto{\pgfqpoint{2.590389in}{2.504414in}}%
\pgfpathmoveto{\pgfqpoint{2.581873in}{2.508672in}}%
\pgfpathlineto{\pgfqpoint{2.581873in}{2.508672in}}%
\pgfpathlineto{\pgfqpoint{2.581873in}{2.512930in}}%
\pgfpathlineto{\pgfqpoint{2.586131in}{2.512930in}}%
\pgfpathlineto{\pgfqpoint{2.586131in}{2.508672in}}%
\pgfpathmoveto{\pgfqpoint{2.586131in}{2.508672in}}%
\pgfpathlineto{\pgfqpoint{2.586131in}{2.508672in}}%
\pgfpathlineto{\pgfqpoint{2.586131in}{2.512930in}}%
\pgfpathlineto{\pgfqpoint{2.590389in}{2.512930in}}%
\pgfpathlineto{\pgfqpoint{2.590389in}{2.508672in}}%
\pgfpathmoveto{\pgfqpoint{2.586131in}{2.512930in}}%
\pgfpathlineto{\pgfqpoint{2.586131in}{2.512930in}}%
\pgfpathlineto{\pgfqpoint{2.586131in}{2.517188in}}%
\pgfpathlineto{\pgfqpoint{2.590389in}{2.517188in}}%
\pgfpathlineto{\pgfqpoint{2.590389in}{2.512930in}}%
\pgfpathmoveto{\pgfqpoint{2.590389in}{2.500156in}}%
\pgfpathlineto{\pgfqpoint{2.590389in}{2.500156in}}%
\pgfpathlineto{\pgfqpoint{2.590389in}{2.504414in}}%
\pgfpathlineto{\pgfqpoint{2.594646in}{2.504414in}}%
\pgfpathlineto{\pgfqpoint{2.594646in}{2.500156in}}%
\pgfpathmoveto{\pgfqpoint{2.590389in}{2.504414in}}%
\pgfpathlineto{\pgfqpoint{2.590389in}{2.504414in}}%
\pgfpathlineto{\pgfqpoint{2.590389in}{2.508672in}}%
\pgfpathlineto{\pgfqpoint{2.594646in}{2.508672in}}%
\pgfpathlineto{\pgfqpoint{2.594646in}{2.504414in}}%
\pgfpathmoveto{\pgfqpoint{2.590389in}{2.508672in}}%
\pgfpathlineto{\pgfqpoint{2.590389in}{2.508672in}}%
\pgfpathlineto{\pgfqpoint{2.590389in}{2.512930in}}%
\pgfpathlineto{\pgfqpoint{2.594646in}{2.512930in}}%
\pgfpathlineto{\pgfqpoint{2.594646in}{2.508672in}}%
\pgfpathmoveto{\pgfqpoint{2.590389in}{2.512930in}}%
\pgfpathlineto{\pgfqpoint{2.590389in}{2.512930in}}%
\pgfpathlineto{\pgfqpoint{2.590389in}{2.517188in}}%
\pgfpathlineto{\pgfqpoint{2.594646in}{2.517188in}}%
\pgfpathlineto{\pgfqpoint{2.594646in}{2.512930in}}%
\pgfpathmoveto{\pgfqpoint{2.586131in}{2.517188in}}%
\pgfpathlineto{\pgfqpoint{2.586131in}{2.517188in}}%
\pgfpathlineto{\pgfqpoint{2.586131in}{2.521446in}}%
\pgfpathlineto{\pgfqpoint{2.590389in}{2.521446in}}%
\pgfpathlineto{\pgfqpoint{2.590389in}{2.517188in}}%
\pgfpathmoveto{\pgfqpoint{2.586131in}{2.521446in}}%
\pgfpathlineto{\pgfqpoint{2.586131in}{2.521446in}}%
\pgfpathlineto{\pgfqpoint{2.586131in}{2.525704in}}%
\pgfpathlineto{\pgfqpoint{2.590389in}{2.525704in}}%
\pgfpathlineto{\pgfqpoint{2.590389in}{2.521446in}}%
\pgfpathmoveto{\pgfqpoint{2.586131in}{2.525704in}}%
\pgfpathlineto{\pgfqpoint{2.586131in}{2.525704in}}%
\pgfpathlineto{\pgfqpoint{2.586131in}{2.529962in}}%
\pgfpathlineto{\pgfqpoint{2.590389in}{2.529962in}}%
\pgfpathlineto{\pgfqpoint{2.590389in}{2.525704in}}%
\pgfpathmoveto{\pgfqpoint{2.586131in}{2.529962in}}%
\pgfpathlineto{\pgfqpoint{2.586131in}{2.529962in}}%
\pgfpathlineto{\pgfqpoint{2.586131in}{2.534220in}}%
\pgfpathlineto{\pgfqpoint{2.590389in}{2.534220in}}%
\pgfpathlineto{\pgfqpoint{2.590389in}{2.529962in}}%
\pgfpathmoveto{\pgfqpoint{2.590389in}{2.517188in}}%
\pgfpathlineto{\pgfqpoint{2.590389in}{2.517188in}}%
\pgfpathlineto{\pgfqpoint{2.590389in}{2.521446in}}%
\pgfpathlineto{\pgfqpoint{2.594646in}{2.521446in}}%
\pgfpathlineto{\pgfqpoint{2.594646in}{2.517188in}}%
\pgfpathmoveto{\pgfqpoint{2.590389in}{2.521446in}}%
\pgfpathlineto{\pgfqpoint{2.590389in}{2.521446in}}%
\pgfpathlineto{\pgfqpoint{2.590389in}{2.525704in}}%
\pgfpathlineto{\pgfqpoint{2.594646in}{2.525704in}}%
\pgfpathlineto{\pgfqpoint{2.594646in}{2.521446in}}%
\pgfpathmoveto{\pgfqpoint{2.590389in}{2.525704in}}%
\pgfpathlineto{\pgfqpoint{2.590389in}{2.525704in}}%
\pgfpathlineto{\pgfqpoint{2.590389in}{2.529962in}}%
\pgfpathlineto{\pgfqpoint{2.594646in}{2.529962in}}%
\pgfpathlineto{\pgfqpoint{2.594646in}{2.525704in}}%
\pgfpathmoveto{\pgfqpoint{2.590389in}{2.529962in}}%
\pgfpathlineto{\pgfqpoint{2.590389in}{2.529962in}}%
\pgfpathlineto{\pgfqpoint{2.590389in}{2.534220in}}%
\pgfpathlineto{\pgfqpoint{2.594646in}{2.534220in}}%
\pgfpathlineto{\pgfqpoint{2.594646in}{2.529962in}}%
\pgfpathmoveto{\pgfqpoint{2.594646in}{2.525704in}}%
\pgfpathlineto{\pgfqpoint{2.594646in}{2.525704in}}%
\pgfpathlineto{\pgfqpoint{2.594646in}{2.529962in}}%
\pgfpathlineto{\pgfqpoint{2.598904in}{2.529962in}}%
\pgfpathlineto{\pgfqpoint{2.598904in}{2.525704in}}%
\pgfpathmoveto{\pgfqpoint{2.594646in}{2.529962in}}%
\pgfpathlineto{\pgfqpoint{2.594646in}{2.529962in}}%
\pgfpathlineto{\pgfqpoint{2.594646in}{2.534220in}}%
\pgfpathlineto{\pgfqpoint{2.598904in}{2.534220in}}%
\pgfpathlineto{\pgfqpoint{2.598904in}{2.529962in}}%
\pgfpathmoveto{\pgfqpoint{2.586131in}{2.534220in}}%
\pgfpathlineto{\pgfqpoint{2.586131in}{2.534220in}}%
\pgfpathlineto{\pgfqpoint{2.586131in}{2.538478in}}%
\pgfpathlineto{\pgfqpoint{2.590389in}{2.538478in}}%
\pgfpathlineto{\pgfqpoint{2.590389in}{2.534220in}}%
\pgfpathmoveto{\pgfqpoint{2.586131in}{2.538478in}}%
\pgfpathlineto{\pgfqpoint{2.586131in}{2.538478in}}%
\pgfpathlineto{\pgfqpoint{2.586131in}{2.542736in}}%
\pgfpathlineto{\pgfqpoint{2.590389in}{2.542736in}}%
\pgfpathlineto{\pgfqpoint{2.590389in}{2.538478in}}%
\pgfpathmoveto{\pgfqpoint{2.590389in}{2.534220in}}%
\pgfpathlineto{\pgfqpoint{2.590389in}{2.534220in}}%
\pgfpathlineto{\pgfqpoint{2.590389in}{2.538478in}}%
\pgfpathlineto{\pgfqpoint{2.594646in}{2.538478in}}%
\pgfpathlineto{\pgfqpoint{2.594646in}{2.534220in}}%
\pgfpathmoveto{\pgfqpoint{2.590389in}{2.538478in}}%
\pgfpathlineto{\pgfqpoint{2.590389in}{2.538478in}}%
\pgfpathlineto{\pgfqpoint{2.590389in}{2.542736in}}%
\pgfpathlineto{\pgfqpoint{2.594646in}{2.542736in}}%
\pgfpathlineto{\pgfqpoint{2.594646in}{2.538478in}}%
\pgfpathmoveto{\pgfqpoint{2.594646in}{2.534220in}}%
\pgfpathlineto{\pgfqpoint{2.594646in}{2.534220in}}%
\pgfpathlineto{\pgfqpoint{2.594646in}{2.538478in}}%
\pgfpathlineto{\pgfqpoint{2.598904in}{2.538478in}}%
\pgfpathlineto{\pgfqpoint{2.598904in}{2.534220in}}%
\pgfpathmoveto{\pgfqpoint{2.594646in}{2.538478in}}%
\pgfpathlineto{\pgfqpoint{2.594646in}{2.538478in}}%
\pgfpathlineto{\pgfqpoint{2.594646in}{2.542736in}}%
\pgfpathlineto{\pgfqpoint{2.598904in}{2.542736in}}%
\pgfpathlineto{\pgfqpoint{2.598904in}{2.538478in}}%
\pgfpathmoveto{\pgfqpoint{2.590389in}{2.542736in}}%
\pgfpathlineto{\pgfqpoint{2.590389in}{2.542736in}}%
\pgfpathlineto{\pgfqpoint{2.590389in}{2.546994in}}%
\pgfpathlineto{\pgfqpoint{2.594646in}{2.546994in}}%
\pgfpathlineto{\pgfqpoint{2.594646in}{2.542736in}}%
\pgfpathmoveto{\pgfqpoint{2.590389in}{2.546994in}}%
\pgfpathlineto{\pgfqpoint{2.590389in}{2.546994in}}%
\pgfpathlineto{\pgfqpoint{2.590389in}{2.551252in}}%
\pgfpathlineto{\pgfqpoint{2.594646in}{2.551252in}}%
\pgfpathlineto{\pgfqpoint{2.594646in}{2.546994in}}%
\pgfpathmoveto{\pgfqpoint{2.594646in}{2.542736in}}%
\pgfpathlineto{\pgfqpoint{2.594646in}{2.542736in}}%
\pgfpathlineto{\pgfqpoint{2.594646in}{2.546994in}}%
\pgfpathlineto{\pgfqpoint{2.598904in}{2.546994in}}%
\pgfpathlineto{\pgfqpoint{2.598904in}{2.542736in}}%
\pgfpathmoveto{\pgfqpoint{2.594646in}{2.546994in}}%
\pgfpathlineto{\pgfqpoint{2.594646in}{2.546994in}}%
\pgfpathlineto{\pgfqpoint{2.594646in}{2.551252in}}%
\pgfpathlineto{\pgfqpoint{2.598904in}{2.551252in}}%
\pgfpathlineto{\pgfqpoint{2.598904in}{2.546994in}}%
\pgfpathmoveto{\pgfqpoint{2.590389in}{2.551252in}}%
\pgfpathlineto{\pgfqpoint{2.590389in}{2.551252in}}%
\pgfpathlineto{\pgfqpoint{2.590389in}{2.555510in}}%
\pgfpathlineto{\pgfqpoint{2.594646in}{2.555510in}}%
\pgfpathlineto{\pgfqpoint{2.594646in}{2.551252in}}%
\pgfpathmoveto{\pgfqpoint{2.590389in}{2.555510in}}%
\pgfpathlineto{\pgfqpoint{2.590389in}{2.555510in}}%
\pgfpathlineto{\pgfqpoint{2.590389in}{2.559768in}}%
\pgfpathlineto{\pgfqpoint{2.594646in}{2.559768in}}%
\pgfpathlineto{\pgfqpoint{2.594646in}{2.555510in}}%
\pgfpathmoveto{\pgfqpoint{2.594646in}{2.551252in}}%
\pgfpathlineto{\pgfqpoint{2.594646in}{2.551252in}}%
\pgfpathlineto{\pgfqpoint{2.594646in}{2.555510in}}%
\pgfpathlineto{\pgfqpoint{2.598904in}{2.555510in}}%
\pgfpathlineto{\pgfqpoint{2.598904in}{2.551252in}}%
\pgfpathmoveto{\pgfqpoint{2.594646in}{2.555510in}}%
\pgfpathlineto{\pgfqpoint{2.594646in}{2.555510in}}%
\pgfpathlineto{\pgfqpoint{2.594646in}{2.559768in}}%
\pgfpathlineto{\pgfqpoint{2.598904in}{2.559768in}}%
\pgfpathlineto{\pgfqpoint{2.598904in}{2.555510in}}%
\pgfpathmoveto{\pgfqpoint{2.590389in}{2.559768in}}%
\pgfpathlineto{\pgfqpoint{2.590389in}{2.559768in}}%
\pgfpathlineto{\pgfqpoint{2.590389in}{2.564026in}}%
\pgfpathlineto{\pgfqpoint{2.594646in}{2.564026in}}%
\pgfpathlineto{\pgfqpoint{2.594646in}{2.559768in}}%
\pgfpathmoveto{\pgfqpoint{2.590389in}{2.564026in}}%
\pgfpathlineto{\pgfqpoint{2.590389in}{2.564026in}}%
\pgfpathlineto{\pgfqpoint{2.590389in}{2.568283in}}%
\pgfpathlineto{\pgfqpoint{2.594646in}{2.568283in}}%
\pgfpathlineto{\pgfqpoint{2.594646in}{2.564026in}}%
\pgfpathmoveto{\pgfqpoint{2.594646in}{2.559768in}}%
\pgfpathlineto{\pgfqpoint{2.594646in}{2.559768in}}%
\pgfpathlineto{\pgfqpoint{2.594646in}{2.564026in}}%
\pgfpathlineto{\pgfqpoint{2.598904in}{2.564026in}}%
\pgfpathlineto{\pgfqpoint{2.598904in}{2.559768in}}%
\pgfpathmoveto{\pgfqpoint{2.594646in}{2.564026in}}%
\pgfpathlineto{\pgfqpoint{2.594646in}{2.564026in}}%
\pgfpathlineto{\pgfqpoint{2.594646in}{2.568283in}}%
\pgfpathlineto{\pgfqpoint{2.598904in}{2.568283in}}%
\pgfpathlineto{\pgfqpoint{2.598904in}{2.564026in}}%
\pgfpathmoveto{\pgfqpoint{2.590389in}{2.568283in}}%
\pgfpathlineto{\pgfqpoint{2.590389in}{2.568283in}}%
\pgfpathlineto{\pgfqpoint{2.590389in}{2.572541in}}%
\pgfpathlineto{\pgfqpoint{2.594646in}{2.572541in}}%
\pgfpathlineto{\pgfqpoint{2.594646in}{2.568283in}}%
\pgfpathmoveto{\pgfqpoint{2.594646in}{2.568283in}}%
\pgfpathlineto{\pgfqpoint{2.594646in}{2.568283in}}%
\pgfpathlineto{\pgfqpoint{2.594646in}{2.572541in}}%
\pgfpathlineto{\pgfqpoint{2.598904in}{2.572541in}}%
\pgfpathlineto{\pgfqpoint{2.598904in}{2.568283in}}%
\pgfpathmoveto{\pgfqpoint{2.594646in}{2.572541in}}%
\pgfpathlineto{\pgfqpoint{2.594646in}{2.572541in}}%
\pgfpathlineto{\pgfqpoint{2.594646in}{2.576799in}}%
\pgfpathlineto{\pgfqpoint{2.598904in}{2.576799in}}%
\pgfpathlineto{\pgfqpoint{2.598904in}{2.572541in}}%
\pgfpathmoveto{\pgfqpoint{2.594646in}{2.576799in}}%
\pgfpathlineto{\pgfqpoint{2.594646in}{2.576799in}}%
\pgfpathlineto{\pgfqpoint{2.594646in}{2.581057in}}%
\pgfpathlineto{\pgfqpoint{2.598904in}{2.581057in}}%
\pgfpathlineto{\pgfqpoint{2.598904in}{2.576799in}}%
\pgfpathmoveto{\pgfqpoint{2.594646in}{2.581057in}}%
\pgfpathlineto{\pgfqpoint{2.594646in}{2.581057in}}%
\pgfpathlineto{\pgfqpoint{2.594646in}{2.585314in}}%
\pgfpathlineto{\pgfqpoint{2.598904in}{2.585314in}}%
\pgfpathlineto{\pgfqpoint{2.598904in}{2.581057in}}%
\pgfpathmoveto{\pgfqpoint{2.598904in}{2.551252in}}%
\pgfpathlineto{\pgfqpoint{2.598904in}{2.551252in}}%
\pgfpathlineto{\pgfqpoint{2.598904in}{2.555510in}}%
\pgfpathlineto{\pgfqpoint{2.603162in}{2.555510in}}%
\pgfpathlineto{\pgfqpoint{2.603162in}{2.551252in}}%
\pgfpathmoveto{\pgfqpoint{2.598904in}{2.555510in}}%
\pgfpathlineto{\pgfqpoint{2.598904in}{2.555510in}}%
\pgfpathlineto{\pgfqpoint{2.598904in}{2.559768in}}%
\pgfpathlineto{\pgfqpoint{2.603162in}{2.559768in}}%
\pgfpathlineto{\pgfqpoint{2.603162in}{2.555510in}}%
\pgfpathmoveto{\pgfqpoint{2.598904in}{2.559768in}}%
\pgfpathlineto{\pgfqpoint{2.598904in}{2.559768in}}%
\pgfpathlineto{\pgfqpoint{2.598904in}{2.564026in}}%
\pgfpathlineto{\pgfqpoint{2.603162in}{2.564026in}}%
\pgfpathlineto{\pgfqpoint{2.603162in}{2.559768in}}%
\pgfpathmoveto{\pgfqpoint{2.598904in}{2.564026in}}%
\pgfpathlineto{\pgfqpoint{2.598904in}{2.564026in}}%
\pgfpathlineto{\pgfqpoint{2.598904in}{2.568283in}}%
\pgfpathlineto{\pgfqpoint{2.603162in}{2.568283in}}%
\pgfpathlineto{\pgfqpoint{2.603162in}{2.564026in}}%
\pgfpathmoveto{\pgfqpoint{2.598904in}{2.568283in}}%
\pgfpathlineto{\pgfqpoint{2.598904in}{2.568283in}}%
\pgfpathlineto{\pgfqpoint{2.598904in}{2.572541in}}%
\pgfpathlineto{\pgfqpoint{2.603162in}{2.572541in}}%
\pgfpathlineto{\pgfqpoint{2.603162in}{2.568283in}}%
\pgfpathmoveto{\pgfqpoint{2.598904in}{2.572541in}}%
\pgfpathlineto{\pgfqpoint{2.598904in}{2.572541in}}%
\pgfpathlineto{\pgfqpoint{2.598904in}{2.576799in}}%
\pgfpathlineto{\pgfqpoint{2.603162in}{2.576799in}}%
\pgfpathlineto{\pgfqpoint{2.603162in}{2.572541in}}%
\pgfpathmoveto{\pgfqpoint{2.598904in}{2.576799in}}%
\pgfpathlineto{\pgfqpoint{2.598904in}{2.576799in}}%
\pgfpathlineto{\pgfqpoint{2.598904in}{2.581057in}}%
\pgfpathlineto{\pgfqpoint{2.603162in}{2.581057in}}%
\pgfpathlineto{\pgfqpoint{2.603162in}{2.576799in}}%
\pgfpathmoveto{\pgfqpoint{2.598904in}{2.581057in}}%
\pgfpathlineto{\pgfqpoint{2.598904in}{2.581057in}}%
\pgfpathlineto{\pgfqpoint{2.598904in}{2.585314in}}%
\pgfpathlineto{\pgfqpoint{2.603162in}{2.585314in}}%
\pgfpathlineto{\pgfqpoint{2.603162in}{2.581057in}}%
\pgfpathmoveto{\pgfqpoint{2.603162in}{2.581057in}}%
\pgfpathlineto{\pgfqpoint{2.603162in}{2.581057in}}%
\pgfpathlineto{\pgfqpoint{2.603162in}{2.585314in}}%
\pgfpathlineto{\pgfqpoint{2.607420in}{2.585314in}}%
\pgfpathlineto{\pgfqpoint{2.607420in}{2.581057in}}%
\pgfpathmoveto{\pgfqpoint{2.594646in}{2.585314in}}%
\pgfpathlineto{\pgfqpoint{2.594646in}{2.585314in}}%
\pgfpathlineto{\pgfqpoint{2.594646in}{2.589572in}}%
\pgfpathlineto{\pgfqpoint{2.598904in}{2.589572in}}%
\pgfpathlineto{\pgfqpoint{2.598904in}{2.585314in}}%
\pgfpathmoveto{\pgfqpoint{2.594646in}{2.589572in}}%
\pgfpathlineto{\pgfqpoint{2.594646in}{2.589572in}}%
\pgfpathlineto{\pgfqpoint{2.594646in}{2.593830in}}%
\pgfpathlineto{\pgfqpoint{2.598904in}{2.593830in}}%
\pgfpathlineto{\pgfqpoint{2.598904in}{2.589572in}}%
\pgfpathmoveto{\pgfqpoint{2.594646in}{2.593830in}}%
\pgfpathlineto{\pgfqpoint{2.594646in}{2.593830in}}%
\pgfpathlineto{\pgfqpoint{2.594646in}{2.598088in}}%
\pgfpathlineto{\pgfqpoint{2.598904in}{2.598088in}}%
\pgfpathlineto{\pgfqpoint{2.598904in}{2.593830in}}%
\pgfpathmoveto{\pgfqpoint{2.598904in}{2.585314in}}%
\pgfpathlineto{\pgfqpoint{2.598904in}{2.585314in}}%
\pgfpathlineto{\pgfqpoint{2.598904in}{2.589572in}}%
\pgfpathlineto{\pgfqpoint{2.603162in}{2.589572in}}%
\pgfpathlineto{\pgfqpoint{2.603162in}{2.585314in}}%
\pgfpathmoveto{\pgfqpoint{2.598904in}{2.589572in}}%
\pgfpathlineto{\pgfqpoint{2.598904in}{2.589572in}}%
\pgfpathlineto{\pgfqpoint{2.598904in}{2.593830in}}%
\pgfpathlineto{\pgfqpoint{2.603162in}{2.593830in}}%
\pgfpathlineto{\pgfqpoint{2.603162in}{2.589572in}}%
\pgfpathmoveto{\pgfqpoint{2.603162in}{2.585314in}}%
\pgfpathlineto{\pgfqpoint{2.603162in}{2.585314in}}%
\pgfpathlineto{\pgfqpoint{2.603162in}{2.589572in}}%
\pgfpathlineto{\pgfqpoint{2.607420in}{2.589572in}}%
\pgfpathlineto{\pgfqpoint{2.607420in}{2.585314in}}%
\pgfpathmoveto{\pgfqpoint{2.603162in}{2.589572in}}%
\pgfpathlineto{\pgfqpoint{2.603162in}{2.589572in}}%
\pgfpathlineto{\pgfqpoint{2.603162in}{2.593830in}}%
\pgfpathlineto{\pgfqpoint{2.607420in}{2.593830in}}%
\pgfpathlineto{\pgfqpoint{2.607420in}{2.589572in}}%
\pgfpathmoveto{\pgfqpoint{2.598904in}{2.593830in}}%
\pgfpathlineto{\pgfqpoint{2.598904in}{2.593830in}}%
\pgfpathlineto{\pgfqpoint{2.598904in}{2.598088in}}%
\pgfpathlineto{\pgfqpoint{2.603162in}{2.598088in}}%
\pgfpathlineto{\pgfqpoint{2.603162in}{2.593830in}}%
\pgfpathmoveto{\pgfqpoint{2.598904in}{2.598088in}}%
\pgfpathlineto{\pgfqpoint{2.598904in}{2.598088in}}%
\pgfpathlineto{\pgfqpoint{2.598904in}{2.602346in}}%
\pgfpathlineto{\pgfqpoint{2.603162in}{2.602346in}}%
\pgfpathlineto{\pgfqpoint{2.603162in}{2.598088in}}%
\pgfpathmoveto{\pgfqpoint{2.603162in}{2.593830in}}%
\pgfpathlineto{\pgfqpoint{2.603162in}{2.593830in}}%
\pgfpathlineto{\pgfqpoint{2.603162in}{2.598088in}}%
\pgfpathlineto{\pgfqpoint{2.607420in}{2.598088in}}%
\pgfpathlineto{\pgfqpoint{2.607420in}{2.593830in}}%
\pgfpathmoveto{\pgfqpoint{2.603162in}{2.598088in}}%
\pgfpathlineto{\pgfqpoint{2.603162in}{2.598088in}}%
\pgfpathlineto{\pgfqpoint{2.603162in}{2.602346in}}%
\pgfpathlineto{\pgfqpoint{2.607420in}{2.602346in}}%
\pgfpathlineto{\pgfqpoint{2.607420in}{2.598088in}}%
\pgfpathmoveto{\pgfqpoint{2.598904in}{2.602346in}}%
\pgfpathlineto{\pgfqpoint{2.598904in}{2.602346in}}%
\pgfpathlineto{\pgfqpoint{2.598904in}{2.606603in}}%
\pgfpathlineto{\pgfqpoint{2.603162in}{2.606603in}}%
\pgfpathlineto{\pgfqpoint{2.603162in}{2.602346in}}%
\pgfpathmoveto{\pgfqpoint{2.598904in}{2.606603in}}%
\pgfpathlineto{\pgfqpoint{2.598904in}{2.606603in}}%
\pgfpathlineto{\pgfqpoint{2.598904in}{2.610861in}}%
\pgfpathlineto{\pgfqpoint{2.603162in}{2.610861in}}%
\pgfpathlineto{\pgfqpoint{2.603162in}{2.606603in}}%
\pgfpathmoveto{\pgfqpoint{2.603162in}{2.602346in}}%
\pgfpathlineto{\pgfqpoint{2.603162in}{2.602346in}}%
\pgfpathlineto{\pgfqpoint{2.603162in}{2.606603in}}%
\pgfpathlineto{\pgfqpoint{2.607420in}{2.606603in}}%
\pgfpathlineto{\pgfqpoint{2.607420in}{2.602346in}}%
\pgfpathmoveto{\pgfqpoint{2.603162in}{2.606603in}}%
\pgfpathlineto{\pgfqpoint{2.603162in}{2.606603in}}%
\pgfpathlineto{\pgfqpoint{2.603162in}{2.610861in}}%
\pgfpathlineto{\pgfqpoint{2.607420in}{2.610861in}}%
\pgfpathlineto{\pgfqpoint{2.607420in}{2.606603in}}%
\pgfpathmoveto{\pgfqpoint{2.598904in}{2.610861in}}%
\pgfpathlineto{\pgfqpoint{2.598904in}{2.610861in}}%
\pgfpathlineto{\pgfqpoint{2.598904in}{2.615119in}}%
\pgfpathlineto{\pgfqpoint{2.603162in}{2.615119in}}%
\pgfpathlineto{\pgfqpoint{2.603162in}{2.610861in}}%
\pgfpathmoveto{\pgfqpoint{2.598904in}{2.615119in}}%
\pgfpathlineto{\pgfqpoint{2.598904in}{2.615119in}}%
\pgfpathlineto{\pgfqpoint{2.598904in}{2.619377in}}%
\pgfpathlineto{\pgfqpoint{2.603162in}{2.619377in}}%
\pgfpathlineto{\pgfqpoint{2.603162in}{2.615119in}}%
\pgfpathmoveto{\pgfqpoint{2.603162in}{2.610861in}}%
\pgfpathlineto{\pgfqpoint{2.603162in}{2.610861in}}%
\pgfpathlineto{\pgfqpoint{2.603162in}{2.615119in}}%
\pgfpathlineto{\pgfqpoint{2.607420in}{2.615119in}}%
\pgfpathlineto{\pgfqpoint{2.607420in}{2.610861in}}%
\pgfpathmoveto{\pgfqpoint{2.603162in}{2.615119in}}%
\pgfpathlineto{\pgfqpoint{2.603162in}{2.615119in}}%
\pgfpathlineto{\pgfqpoint{2.603162in}{2.619377in}}%
\pgfpathlineto{\pgfqpoint{2.607420in}{2.619377in}}%
\pgfpathlineto{\pgfqpoint{2.607420in}{2.615119in}}%
\pgfpathmoveto{\pgfqpoint{2.607420in}{2.606603in}}%
\pgfpathlineto{\pgfqpoint{2.607420in}{2.606603in}}%
\pgfpathlineto{\pgfqpoint{2.607420in}{2.610861in}}%
\pgfpathlineto{\pgfqpoint{2.611677in}{2.610861in}}%
\pgfpathlineto{\pgfqpoint{2.611677in}{2.606603in}}%
\pgfpathmoveto{\pgfqpoint{2.607420in}{2.610861in}}%
\pgfpathlineto{\pgfqpoint{2.607420in}{2.610861in}}%
\pgfpathlineto{\pgfqpoint{2.607420in}{2.615119in}}%
\pgfpathlineto{\pgfqpoint{2.611677in}{2.615119in}}%
\pgfpathlineto{\pgfqpoint{2.611677in}{2.610861in}}%
\pgfpathmoveto{\pgfqpoint{2.607420in}{2.615119in}}%
\pgfpathlineto{\pgfqpoint{2.607420in}{2.615119in}}%
\pgfpathlineto{\pgfqpoint{2.607420in}{2.619377in}}%
\pgfpathlineto{\pgfqpoint{2.611677in}{2.619377in}}%
\pgfpathlineto{\pgfqpoint{2.611677in}{2.615119in}}%
\pgfpathmoveto{\pgfqpoint{2.598904in}{2.619377in}}%
\pgfpathlineto{\pgfqpoint{2.598904in}{2.619377in}}%
\pgfpathlineto{\pgfqpoint{2.598904in}{2.623635in}}%
\pgfpathlineto{\pgfqpoint{2.603162in}{2.623635in}}%
\pgfpathlineto{\pgfqpoint{2.603162in}{2.619377in}}%
\pgfpathmoveto{\pgfqpoint{2.598904in}{2.623635in}}%
\pgfpathlineto{\pgfqpoint{2.598904in}{2.623635in}}%
\pgfpathlineto{\pgfqpoint{2.598904in}{2.627892in}}%
\pgfpathlineto{\pgfqpoint{2.603162in}{2.627892in}}%
\pgfpathlineto{\pgfqpoint{2.603162in}{2.623635in}}%
\pgfpathmoveto{\pgfqpoint{2.603162in}{2.619377in}}%
\pgfpathlineto{\pgfqpoint{2.603162in}{2.619377in}}%
\pgfpathlineto{\pgfqpoint{2.603162in}{2.623635in}}%
\pgfpathlineto{\pgfqpoint{2.607420in}{2.623635in}}%
\pgfpathlineto{\pgfqpoint{2.607420in}{2.619377in}}%
\pgfpathmoveto{\pgfqpoint{2.603162in}{2.623635in}}%
\pgfpathlineto{\pgfqpoint{2.603162in}{2.623635in}}%
\pgfpathlineto{\pgfqpoint{2.603162in}{2.627892in}}%
\pgfpathlineto{\pgfqpoint{2.607420in}{2.627892in}}%
\pgfpathlineto{\pgfqpoint{2.607420in}{2.623635in}}%
\pgfpathmoveto{\pgfqpoint{2.603162in}{2.627892in}}%
\pgfpathlineto{\pgfqpoint{2.603162in}{2.627892in}}%
\pgfpathlineto{\pgfqpoint{2.603162in}{2.632150in}}%
\pgfpathlineto{\pgfqpoint{2.607420in}{2.632150in}}%
\pgfpathlineto{\pgfqpoint{2.607420in}{2.627892in}}%
\pgfpathmoveto{\pgfqpoint{2.603162in}{2.632150in}}%
\pgfpathlineto{\pgfqpoint{2.603162in}{2.632150in}}%
\pgfpathlineto{\pgfqpoint{2.603162in}{2.636408in}}%
\pgfpathlineto{\pgfqpoint{2.607420in}{2.636408in}}%
\pgfpathlineto{\pgfqpoint{2.607420in}{2.632150in}}%
\pgfpathmoveto{\pgfqpoint{2.607420in}{2.619377in}}%
\pgfpathlineto{\pgfqpoint{2.607420in}{2.619377in}}%
\pgfpathlineto{\pgfqpoint{2.607420in}{2.623635in}}%
\pgfpathlineto{\pgfqpoint{2.611677in}{2.623635in}}%
\pgfpathlineto{\pgfqpoint{2.611677in}{2.619377in}}%
\pgfpathmoveto{\pgfqpoint{2.607420in}{2.623635in}}%
\pgfpathlineto{\pgfqpoint{2.607420in}{2.623635in}}%
\pgfpathlineto{\pgfqpoint{2.607420in}{2.627892in}}%
\pgfpathlineto{\pgfqpoint{2.611677in}{2.627892in}}%
\pgfpathlineto{\pgfqpoint{2.611677in}{2.623635in}}%
\pgfpathmoveto{\pgfqpoint{2.607420in}{2.627892in}}%
\pgfpathlineto{\pgfqpoint{2.607420in}{2.627892in}}%
\pgfpathlineto{\pgfqpoint{2.607420in}{2.632150in}}%
\pgfpathlineto{\pgfqpoint{2.611677in}{2.632150in}}%
\pgfpathlineto{\pgfqpoint{2.611677in}{2.627892in}}%
\pgfpathmoveto{\pgfqpoint{2.607420in}{2.632150in}}%
\pgfpathlineto{\pgfqpoint{2.607420in}{2.632150in}}%
\pgfpathlineto{\pgfqpoint{2.607420in}{2.636408in}}%
\pgfpathlineto{\pgfqpoint{2.611677in}{2.636408in}}%
\pgfpathlineto{\pgfqpoint{2.611677in}{2.632150in}}%
\pgfpathmoveto{\pgfqpoint{2.603162in}{2.636408in}}%
\pgfpathlineto{\pgfqpoint{2.603162in}{2.636408in}}%
\pgfpathlineto{\pgfqpoint{2.603162in}{2.640666in}}%
\pgfpathlineto{\pgfqpoint{2.607420in}{2.640666in}}%
\pgfpathlineto{\pgfqpoint{2.607420in}{2.636408in}}%
\pgfpathmoveto{\pgfqpoint{2.603162in}{2.640666in}}%
\pgfpathlineto{\pgfqpoint{2.603162in}{2.640666in}}%
\pgfpathlineto{\pgfqpoint{2.603162in}{2.644924in}}%
\pgfpathlineto{\pgfqpoint{2.607420in}{2.644924in}}%
\pgfpathlineto{\pgfqpoint{2.607420in}{2.640666in}}%
\pgfpathmoveto{\pgfqpoint{2.603162in}{2.644924in}}%
\pgfpathlineto{\pgfqpoint{2.603162in}{2.644924in}}%
\pgfpathlineto{\pgfqpoint{2.603162in}{2.649181in}}%
\pgfpathlineto{\pgfqpoint{2.607420in}{2.649181in}}%
\pgfpathlineto{\pgfqpoint{2.607420in}{2.644924in}}%
\pgfpathmoveto{\pgfqpoint{2.603162in}{2.649181in}}%
\pgfpathlineto{\pgfqpoint{2.603162in}{2.649181in}}%
\pgfpathlineto{\pgfqpoint{2.603162in}{2.653439in}}%
\pgfpathlineto{\pgfqpoint{2.607420in}{2.653439in}}%
\pgfpathlineto{\pgfqpoint{2.607420in}{2.649181in}}%
\pgfpathmoveto{\pgfqpoint{2.607420in}{2.636408in}}%
\pgfpathlineto{\pgfqpoint{2.607420in}{2.636408in}}%
\pgfpathlineto{\pgfqpoint{2.607420in}{2.640666in}}%
\pgfpathlineto{\pgfqpoint{2.611677in}{2.640666in}}%
\pgfpathlineto{\pgfqpoint{2.611677in}{2.636408in}}%
\pgfpathmoveto{\pgfqpoint{2.607420in}{2.640666in}}%
\pgfpathlineto{\pgfqpoint{2.607420in}{2.640666in}}%
\pgfpathlineto{\pgfqpoint{2.607420in}{2.644924in}}%
\pgfpathlineto{\pgfqpoint{2.611677in}{2.644924in}}%
\pgfpathlineto{\pgfqpoint{2.611677in}{2.640666in}}%
\pgfpathmoveto{\pgfqpoint{2.611677in}{2.636408in}}%
\pgfpathlineto{\pgfqpoint{2.611677in}{2.636408in}}%
\pgfpathlineto{\pgfqpoint{2.611677in}{2.640666in}}%
\pgfpathlineto{\pgfqpoint{2.615935in}{2.640666in}}%
\pgfpathlineto{\pgfqpoint{2.615935in}{2.636408in}}%
\pgfpathmoveto{\pgfqpoint{2.611677in}{2.640666in}}%
\pgfpathlineto{\pgfqpoint{2.611677in}{2.640666in}}%
\pgfpathlineto{\pgfqpoint{2.611677in}{2.644924in}}%
\pgfpathlineto{\pgfqpoint{2.615935in}{2.644924in}}%
\pgfpathlineto{\pgfqpoint{2.615935in}{2.640666in}}%
\pgfpathmoveto{\pgfqpoint{2.607420in}{2.644924in}}%
\pgfpathlineto{\pgfqpoint{2.607420in}{2.644924in}}%
\pgfpathlineto{\pgfqpoint{2.607420in}{2.649181in}}%
\pgfpathlineto{\pgfqpoint{2.611677in}{2.649181in}}%
\pgfpathlineto{\pgfqpoint{2.611677in}{2.644924in}}%
\pgfpathmoveto{\pgfqpoint{2.607420in}{2.649181in}}%
\pgfpathlineto{\pgfqpoint{2.607420in}{2.649181in}}%
\pgfpathlineto{\pgfqpoint{2.607420in}{2.653439in}}%
\pgfpathlineto{\pgfqpoint{2.611677in}{2.653439in}}%
\pgfpathlineto{\pgfqpoint{2.611677in}{2.649181in}}%
\pgfpathmoveto{\pgfqpoint{2.611677in}{2.644924in}}%
\pgfpathlineto{\pgfqpoint{2.611677in}{2.644924in}}%
\pgfpathlineto{\pgfqpoint{2.611677in}{2.649181in}}%
\pgfpathlineto{\pgfqpoint{2.615935in}{2.649181in}}%
\pgfpathlineto{\pgfqpoint{2.615935in}{2.644924in}}%
\pgfpathmoveto{\pgfqpoint{2.611677in}{2.649181in}}%
\pgfpathlineto{\pgfqpoint{2.611677in}{2.649181in}}%
\pgfpathlineto{\pgfqpoint{2.611677in}{2.653439in}}%
\pgfpathlineto{\pgfqpoint{2.615935in}{2.653439in}}%
\pgfpathlineto{\pgfqpoint{2.615935in}{2.649181in}}%
\pgfpathmoveto{\pgfqpoint{2.607420in}{2.653439in}}%
\pgfpathlineto{\pgfqpoint{2.607420in}{2.653439in}}%
\pgfpathlineto{\pgfqpoint{2.607420in}{2.657697in}}%
\pgfpathlineto{\pgfqpoint{2.611677in}{2.657697in}}%
\pgfpathlineto{\pgfqpoint{2.611677in}{2.653439in}}%
\pgfpathmoveto{\pgfqpoint{2.607420in}{2.657697in}}%
\pgfpathlineto{\pgfqpoint{2.607420in}{2.657697in}}%
\pgfpathlineto{\pgfqpoint{2.607420in}{2.661955in}}%
\pgfpathlineto{\pgfqpoint{2.611677in}{2.661955in}}%
\pgfpathlineto{\pgfqpoint{2.611677in}{2.657697in}}%
\pgfpathmoveto{\pgfqpoint{2.611677in}{2.653439in}}%
\pgfpathlineto{\pgfqpoint{2.611677in}{2.653439in}}%
\pgfpathlineto{\pgfqpoint{2.611677in}{2.657697in}}%
\pgfpathlineto{\pgfqpoint{2.615935in}{2.657697in}}%
\pgfpathlineto{\pgfqpoint{2.615935in}{2.653439in}}%
\pgfpathmoveto{\pgfqpoint{2.611677in}{2.657697in}}%
\pgfpathlineto{\pgfqpoint{2.611677in}{2.657697in}}%
\pgfpathlineto{\pgfqpoint{2.611677in}{2.661955in}}%
\pgfpathlineto{\pgfqpoint{2.615935in}{2.661955in}}%
\pgfpathlineto{\pgfqpoint{2.615935in}{2.657697in}}%
\pgfpathmoveto{\pgfqpoint{2.607420in}{2.661955in}}%
\pgfpathlineto{\pgfqpoint{2.607420in}{2.661955in}}%
\pgfpathlineto{\pgfqpoint{2.607420in}{2.666213in}}%
\pgfpathlineto{\pgfqpoint{2.611677in}{2.666213in}}%
\pgfpathlineto{\pgfqpoint{2.611677in}{2.661955in}}%
\pgfpathmoveto{\pgfqpoint{2.607420in}{2.666213in}}%
\pgfpathlineto{\pgfqpoint{2.607420in}{2.666213in}}%
\pgfpathlineto{\pgfqpoint{2.607420in}{2.670470in}}%
\pgfpathlineto{\pgfqpoint{2.611677in}{2.670470in}}%
\pgfpathlineto{\pgfqpoint{2.611677in}{2.666213in}}%
\pgfpathmoveto{\pgfqpoint{2.611677in}{2.661955in}}%
\pgfpathlineto{\pgfqpoint{2.611677in}{2.661955in}}%
\pgfpathlineto{\pgfqpoint{2.611677in}{2.666213in}}%
\pgfpathlineto{\pgfqpoint{2.615935in}{2.666213in}}%
\pgfpathlineto{\pgfqpoint{2.615935in}{2.661955in}}%
\pgfpathmoveto{\pgfqpoint{2.611677in}{2.666213in}}%
\pgfpathlineto{\pgfqpoint{2.611677in}{2.666213in}}%
\pgfpathlineto{\pgfqpoint{2.611677in}{2.670470in}}%
\pgfpathlineto{\pgfqpoint{2.615935in}{2.670470in}}%
\pgfpathlineto{\pgfqpoint{2.615935in}{2.666213in}}%
\pgfpathmoveto{\pgfqpoint{2.607420in}{2.670470in}}%
\pgfpathlineto{\pgfqpoint{2.607420in}{2.670470in}}%
\pgfpathlineto{\pgfqpoint{2.607420in}{2.674728in}}%
\pgfpathlineto{\pgfqpoint{2.611677in}{2.674728in}}%
\pgfpathlineto{\pgfqpoint{2.611677in}{2.670470in}}%
\pgfpathmoveto{\pgfqpoint{2.607420in}{2.674728in}}%
\pgfpathlineto{\pgfqpoint{2.607420in}{2.674728in}}%
\pgfpathlineto{\pgfqpoint{2.607420in}{2.678986in}}%
\pgfpathlineto{\pgfqpoint{2.611677in}{2.678986in}}%
\pgfpathlineto{\pgfqpoint{2.611677in}{2.674728in}}%
\pgfpathmoveto{\pgfqpoint{2.611677in}{2.670470in}}%
\pgfpathlineto{\pgfqpoint{2.611677in}{2.670470in}}%
\pgfpathlineto{\pgfqpoint{2.611677in}{2.674728in}}%
\pgfpathlineto{\pgfqpoint{2.615935in}{2.674728in}}%
\pgfpathlineto{\pgfqpoint{2.615935in}{2.670470in}}%
\pgfpathmoveto{\pgfqpoint{2.611677in}{2.674728in}}%
\pgfpathlineto{\pgfqpoint{2.611677in}{2.674728in}}%
\pgfpathlineto{\pgfqpoint{2.611677in}{2.678986in}}%
\pgfpathlineto{\pgfqpoint{2.615935in}{2.678986in}}%
\pgfpathlineto{\pgfqpoint{2.615935in}{2.674728in}}%
\pgfpathmoveto{\pgfqpoint{2.607420in}{2.678986in}}%
\pgfpathlineto{\pgfqpoint{2.607420in}{2.678986in}}%
\pgfpathlineto{\pgfqpoint{2.607420in}{2.683244in}}%
\pgfpathlineto{\pgfqpoint{2.611677in}{2.683244in}}%
\pgfpathlineto{\pgfqpoint{2.611677in}{2.678986in}}%
\pgfpathmoveto{\pgfqpoint{2.611677in}{2.678986in}}%
\pgfpathlineto{\pgfqpoint{2.611677in}{2.678986in}}%
\pgfpathlineto{\pgfqpoint{2.611677in}{2.683244in}}%
\pgfpathlineto{\pgfqpoint{2.615935in}{2.683244in}}%
\pgfpathlineto{\pgfqpoint{2.615935in}{2.678986in}}%
\pgfpathmoveto{\pgfqpoint{2.611677in}{2.683244in}}%
\pgfpathlineto{\pgfqpoint{2.611677in}{2.683244in}}%
\pgfpathlineto{\pgfqpoint{2.611677in}{2.687502in}}%
\pgfpathlineto{\pgfqpoint{2.615935in}{2.687502in}}%
\pgfpathlineto{\pgfqpoint{2.615935in}{2.683244in}}%
\pgfpathmoveto{\pgfqpoint{2.615935in}{2.661955in}}%
\pgfpathlineto{\pgfqpoint{2.615935in}{2.661955in}}%
\pgfpathlineto{\pgfqpoint{2.615935in}{2.666213in}}%
\pgfpathlineto{\pgfqpoint{2.620193in}{2.666213in}}%
\pgfpathlineto{\pgfqpoint{2.620193in}{2.661955in}}%
\pgfpathmoveto{\pgfqpoint{2.615935in}{2.666213in}}%
\pgfpathlineto{\pgfqpoint{2.615935in}{2.666213in}}%
\pgfpathlineto{\pgfqpoint{2.615935in}{2.670470in}}%
\pgfpathlineto{\pgfqpoint{2.620193in}{2.670470in}}%
\pgfpathlineto{\pgfqpoint{2.620193in}{2.666213in}}%
\pgfpathmoveto{\pgfqpoint{2.615935in}{2.670470in}}%
\pgfpathlineto{\pgfqpoint{2.615935in}{2.670470in}}%
\pgfpathlineto{\pgfqpoint{2.615935in}{2.674728in}}%
\pgfpathlineto{\pgfqpoint{2.620193in}{2.674728in}}%
\pgfpathlineto{\pgfqpoint{2.620193in}{2.670470in}}%
\pgfpathmoveto{\pgfqpoint{2.615935in}{2.674728in}}%
\pgfpathlineto{\pgfqpoint{2.615935in}{2.674728in}}%
\pgfpathlineto{\pgfqpoint{2.615935in}{2.678986in}}%
\pgfpathlineto{\pgfqpoint{2.620193in}{2.678986in}}%
\pgfpathlineto{\pgfqpoint{2.620193in}{2.674728in}}%
\pgfpathmoveto{\pgfqpoint{2.615935in}{2.678986in}}%
\pgfpathlineto{\pgfqpoint{2.615935in}{2.678986in}}%
\pgfpathlineto{\pgfqpoint{2.615935in}{2.683244in}}%
\pgfpathlineto{\pgfqpoint{2.620193in}{2.683244in}}%
\pgfpathlineto{\pgfqpoint{2.620193in}{2.678986in}}%
\pgfpathmoveto{\pgfqpoint{2.615935in}{2.683244in}}%
\pgfpathlineto{\pgfqpoint{2.615935in}{2.683244in}}%
\pgfpathlineto{\pgfqpoint{2.615935in}{2.687502in}}%
\pgfpathlineto{\pgfqpoint{2.620193in}{2.687502in}}%
\pgfpathlineto{\pgfqpoint{2.620193in}{2.683244in}}%
\pgfpathmoveto{\pgfqpoint{2.611677in}{2.687502in}}%
\pgfpathlineto{\pgfqpoint{2.611677in}{2.687502in}}%
\pgfpathlineto{\pgfqpoint{2.611677in}{2.691759in}}%
\pgfpathlineto{\pgfqpoint{2.615935in}{2.691759in}}%
\pgfpathlineto{\pgfqpoint{2.615935in}{2.687502in}}%
\pgfpathmoveto{\pgfqpoint{2.611677in}{2.691759in}}%
\pgfpathlineto{\pgfqpoint{2.611677in}{2.691759in}}%
\pgfpathlineto{\pgfqpoint{2.611677in}{2.696017in}}%
\pgfpathlineto{\pgfqpoint{2.615935in}{2.696017in}}%
\pgfpathlineto{\pgfqpoint{2.615935in}{2.691759in}}%
\pgfpathmoveto{\pgfqpoint{2.611677in}{2.696017in}}%
\pgfpathlineto{\pgfqpoint{2.611677in}{2.696017in}}%
\pgfpathlineto{\pgfqpoint{2.611677in}{2.700275in}}%
\pgfpathlineto{\pgfqpoint{2.615935in}{2.700275in}}%
\pgfpathlineto{\pgfqpoint{2.615935in}{2.696017in}}%
\pgfpathmoveto{\pgfqpoint{2.611677in}{2.700275in}}%
\pgfpathlineto{\pgfqpoint{2.611677in}{2.700275in}}%
\pgfpathlineto{\pgfqpoint{2.611677in}{2.704533in}}%
\pgfpathlineto{\pgfqpoint{2.615935in}{2.704533in}}%
\pgfpathlineto{\pgfqpoint{2.615935in}{2.700275in}}%
\pgfpathmoveto{\pgfqpoint{2.611677in}{2.704533in}}%
\pgfpathlineto{\pgfqpoint{2.611677in}{2.704533in}}%
\pgfpathlineto{\pgfqpoint{2.611677in}{2.708791in}}%
\pgfpathlineto{\pgfqpoint{2.615935in}{2.708791in}}%
\pgfpathlineto{\pgfqpoint{2.615935in}{2.704533in}}%
\pgfpathmoveto{\pgfqpoint{2.615935in}{2.687502in}}%
\pgfpathlineto{\pgfqpoint{2.615935in}{2.687502in}}%
\pgfpathlineto{\pgfqpoint{2.615935in}{2.691759in}}%
\pgfpathlineto{\pgfqpoint{2.620193in}{2.691759in}}%
\pgfpathlineto{\pgfqpoint{2.620193in}{2.687502in}}%
\pgfpathmoveto{\pgfqpoint{2.615935in}{2.691759in}}%
\pgfpathlineto{\pgfqpoint{2.615935in}{2.691759in}}%
\pgfpathlineto{\pgfqpoint{2.615935in}{2.696017in}}%
\pgfpathlineto{\pgfqpoint{2.620193in}{2.696017in}}%
\pgfpathlineto{\pgfqpoint{2.620193in}{2.691759in}}%
\pgfpathmoveto{\pgfqpoint{2.620193in}{2.687502in}}%
\pgfpathlineto{\pgfqpoint{2.620193in}{2.687502in}}%
\pgfpathlineto{\pgfqpoint{2.620193in}{2.691759in}}%
\pgfpathlineto{\pgfqpoint{2.624451in}{2.691759in}}%
\pgfpathlineto{\pgfqpoint{2.624451in}{2.687502in}}%
\pgfpathmoveto{\pgfqpoint{2.620193in}{2.691759in}}%
\pgfpathlineto{\pgfqpoint{2.620193in}{2.691759in}}%
\pgfpathlineto{\pgfqpoint{2.620193in}{2.696017in}}%
\pgfpathlineto{\pgfqpoint{2.624451in}{2.696017in}}%
\pgfpathlineto{\pgfqpoint{2.624451in}{2.691759in}}%
\pgfpathmoveto{\pgfqpoint{2.615935in}{2.696017in}}%
\pgfpathlineto{\pgfqpoint{2.615935in}{2.696017in}}%
\pgfpathlineto{\pgfqpoint{2.615935in}{2.700275in}}%
\pgfpathlineto{\pgfqpoint{2.620193in}{2.700275in}}%
\pgfpathlineto{\pgfqpoint{2.620193in}{2.696017in}}%
\pgfpathmoveto{\pgfqpoint{2.615935in}{2.700275in}}%
\pgfpathlineto{\pgfqpoint{2.615935in}{2.700275in}}%
\pgfpathlineto{\pgfqpoint{2.615935in}{2.704533in}}%
\pgfpathlineto{\pgfqpoint{2.620193in}{2.704533in}}%
\pgfpathlineto{\pgfqpoint{2.620193in}{2.700275in}}%
\pgfpathmoveto{\pgfqpoint{2.620193in}{2.696017in}}%
\pgfpathlineto{\pgfqpoint{2.620193in}{2.696017in}}%
\pgfpathlineto{\pgfqpoint{2.620193in}{2.700275in}}%
\pgfpathlineto{\pgfqpoint{2.624451in}{2.700275in}}%
\pgfpathlineto{\pgfqpoint{2.624451in}{2.696017in}}%
\pgfpathmoveto{\pgfqpoint{2.620193in}{2.700275in}}%
\pgfpathlineto{\pgfqpoint{2.620193in}{2.700275in}}%
\pgfpathlineto{\pgfqpoint{2.620193in}{2.704533in}}%
\pgfpathlineto{\pgfqpoint{2.624451in}{2.704533in}}%
\pgfpathlineto{\pgfqpoint{2.624451in}{2.700275in}}%
\pgfpathmoveto{\pgfqpoint{2.615935in}{2.704533in}}%
\pgfpathlineto{\pgfqpoint{2.615935in}{2.704533in}}%
\pgfpathlineto{\pgfqpoint{2.615935in}{2.708791in}}%
\pgfpathlineto{\pgfqpoint{2.620193in}{2.708791in}}%
\pgfpathlineto{\pgfqpoint{2.620193in}{2.704533in}}%
\pgfpathmoveto{\pgfqpoint{2.615935in}{2.708791in}}%
\pgfpathlineto{\pgfqpoint{2.615935in}{2.708791in}}%
\pgfpathlineto{\pgfqpoint{2.615935in}{2.713048in}}%
\pgfpathlineto{\pgfqpoint{2.620193in}{2.713048in}}%
\pgfpathlineto{\pgfqpoint{2.620193in}{2.708791in}}%
\pgfpathmoveto{\pgfqpoint{2.620193in}{2.704533in}}%
\pgfpathlineto{\pgfqpoint{2.620193in}{2.704533in}}%
\pgfpathlineto{\pgfqpoint{2.620193in}{2.708791in}}%
\pgfpathlineto{\pgfqpoint{2.624451in}{2.708791in}}%
\pgfpathlineto{\pgfqpoint{2.624451in}{2.704533in}}%
\pgfpathmoveto{\pgfqpoint{2.620193in}{2.708791in}}%
\pgfpathlineto{\pgfqpoint{2.620193in}{2.708791in}}%
\pgfpathlineto{\pgfqpoint{2.620193in}{2.713048in}}%
\pgfpathlineto{\pgfqpoint{2.624451in}{2.713048in}}%
\pgfpathlineto{\pgfqpoint{2.624451in}{2.708791in}}%
\pgfpathmoveto{\pgfqpoint{2.615935in}{2.713048in}}%
\pgfpathlineto{\pgfqpoint{2.615935in}{2.713048in}}%
\pgfpathlineto{\pgfqpoint{2.615935in}{2.717306in}}%
\pgfpathlineto{\pgfqpoint{2.620193in}{2.717306in}}%
\pgfpathlineto{\pgfqpoint{2.620193in}{2.713048in}}%
\pgfpathmoveto{\pgfqpoint{2.615935in}{2.717306in}}%
\pgfpathlineto{\pgfqpoint{2.615935in}{2.717306in}}%
\pgfpathlineto{\pgfqpoint{2.615935in}{2.721564in}}%
\pgfpathlineto{\pgfqpoint{2.620193in}{2.721564in}}%
\pgfpathlineto{\pgfqpoint{2.620193in}{2.717306in}}%
\pgfpathmoveto{\pgfqpoint{2.620193in}{2.713048in}}%
\pgfpathlineto{\pgfqpoint{2.620193in}{2.713048in}}%
\pgfpathlineto{\pgfqpoint{2.620193in}{2.717306in}}%
\pgfpathlineto{\pgfqpoint{2.624451in}{2.717306in}}%
\pgfpathlineto{\pgfqpoint{2.624451in}{2.713048in}}%
\pgfpathmoveto{\pgfqpoint{2.620193in}{2.717306in}}%
\pgfpathlineto{\pgfqpoint{2.620193in}{2.717306in}}%
\pgfpathlineto{\pgfqpoint{2.620193in}{2.721564in}}%
\pgfpathlineto{\pgfqpoint{2.624451in}{2.721564in}}%
\pgfpathlineto{\pgfqpoint{2.624451in}{2.717306in}}%
\pgfpathmoveto{\pgfqpoint{2.624451in}{2.713048in}}%
\pgfpathlineto{\pgfqpoint{2.624451in}{2.713048in}}%
\pgfpathlineto{\pgfqpoint{2.624451in}{2.717306in}}%
\pgfpathlineto{\pgfqpoint{2.628708in}{2.717306in}}%
\pgfpathlineto{\pgfqpoint{2.628708in}{2.713048in}}%
\pgfpathmoveto{\pgfqpoint{2.624451in}{2.717306in}}%
\pgfpathlineto{\pgfqpoint{2.624451in}{2.717306in}}%
\pgfpathlineto{\pgfqpoint{2.624451in}{2.721564in}}%
\pgfpathlineto{\pgfqpoint{2.628708in}{2.721564in}}%
\pgfpathlineto{\pgfqpoint{2.628708in}{2.717306in}}%
\pgfpathmoveto{\pgfqpoint{2.615935in}{2.721564in}}%
\pgfpathlineto{\pgfqpoint{2.615935in}{2.721564in}}%
\pgfpathlineto{\pgfqpoint{2.615935in}{2.725822in}}%
\pgfpathlineto{\pgfqpoint{2.620193in}{2.725822in}}%
\pgfpathlineto{\pgfqpoint{2.620193in}{2.721564in}}%
\pgfpathmoveto{\pgfqpoint{2.615935in}{2.725822in}}%
\pgfpathlineto{\pgfqpoint{2.615935in}{2.725822in}}%
\pgfpathlineto{\pgfqpoint{2.615935in}{2.730080in}}%
\pgfpathlineto{\pgfqpoint{2.620193in}{2.730080in}}%
\pgfpathlineto{\pgfqpoint{2.620193in}{2.725822in}}%
\pgfpathmoveto{\pgfqpoint{2.620193in}{2.721564in}}%
\pgfpathlineto{\pgfqpoint{2.620193in}{2.721564in}}%
\pgfpathlineto{\pgfqpoint{2.620193in}{2.725822in}}%
\pgfpathlineto{\pgfqpoint{2.624451in}{2.725822in}}%
\pgfpathlineto{\pgfqpoint{2.624451in}{2.721564in}}%
\pgfpathmoveto{\pgfqpoint{2.620193in}{2.725822in}}%
\pgfpathlineto{\pgfqpoint{2.620193in}{2.725822in}}%
\pgfpathlineto{\pgfqpoint{2.620193in}{2.730080in}}%
\pgfpathlineto{\pgfqpoint{2.624451in}{2.730080in}}%
\pgfpathlineto{\pgfqpoint{2.624451in}{2.725822in}}%
\pgfpathmoveto{\pgfqpoint{2.615935in}{2.730080in}}%
\pgfpathlineto{\pgfqpoint{2.615935in}{2.730080in}}%
\pgfpathlineto{\pgfqpoint{2.615935in}{2.734337in}}%
\pgfpathlineto{\pgfqpoint{2.620193in}{2.734337in}}%
\pgfpathlineto{\pgfqpoint{2.620193in}{2.730080in}}%
\pgfpathmoveto{\pgfqpoint{2.620193in}{2.730080in}}%
\pgfpathlineto{\pgfqpoint{2.620193in}{2.730080in}}%
\pgfpathlineto{\pgfqpoint{2.620193in}{2.734337in}}%
\pgfpathlineto{\pgfqpoint{2.624451in}{2.734337in}}%
\pgfpathlineto{\pgfqpoint{2.624451in}{2.730080in}}%
\pgfpathmoveto{\pgfqpoint{2.620193in}{2.734337in}}%
\pgfpathlineto{\pgfqpoint{2.620193in}{2.734337in}}%
\pgfpathlineto{\pgfqpoint{2.620193in}{2.738595in}}%
\pgfpathlineto{\pgfqpoint{2.624451in}{2.738595in}}%
\pgfpathlineto{\pgfqpoint{2.624451in}{2.734337in}}%
\pgfpathmoveto{\pgfqpoint{2.624451in}{2.721564in}}%
\pgfpathlineto{\pgfqpoint{2.624451in}{2.721564in}}%
\pgfpathlineto{\pgfqpoint{2.624451in}{2.725822in}}%
\pgfpathlineto{\pgfqpoint{2.628708in}{2.725822in}}%
\pgfpathlineto{\pgfqpoint{2.628708in}{2.721564in}}%
\pgfpathmoveto{\pgfqpoint{2.624451in}{2.725822in}}%
\pgfpathlineto{\pgfqpoint{2.624451in}{2.725822in}}%
\pgfpathlineto{\pgfqpoint{2.624451in}{2.730080in}}%
\pgfpathlineto{\pgfqpoint{2.628708in}{2.730080in}}%
\pgfpathlineto{\pgfqpoint{2.628708in}{2.725822in}}%
\pgfpathmoveto{\pgfqpoint{2.624451in}{2.730080in}}%
\pgfpathlineto{\pgfqpoint{2.624451in}{2.730080in}}%
\pgfpathlineto{\pgfqpoint{2.624451in}{2.734337in}}%
\pgfpathlineto{\pgfqpoint{2.628708in}{2.734337in}}%
\pgfpathlineto{\pgfqpoint{2.628708in}{2.730080in}}%
\pgfpathmoveto{\pgfqpoint{2.624451in}{2.734337in}}%
\pgfpathlineto{\pgfqpoint{2.624451in}{2.734337in}}%
\pgfpathlineto{\pgfqpoint{2.624451in}{2.738595in}}%
\pgfpathlineto{\pgfqpoint{2.628708in}{2.738595in}}%
\pgfpathlineto{\pgfqpoint{2.628708in}{2.734337in}}%
\pgfpathmoveto{\pgfqpoint{2.620193in}{2.738595in}}%
\pgfpathlineto{\pgfqpoint{2.620193in}{2.738595in}}%
\pgfpathlineto{\pgfqpoint{2.620193in}{2.742853in}}%
\pgfpathlineto{\pgfqpoint{2.624451in}{2.742853in}}%
\pgfpathlineto{\pgfqpoint{2.624451in}{2.738595in}}%
\pgfpathmoveto{\pgfqpoint{2.620193in}{2.742853in}}%
\pgfpathlineto{\pgfqpoint{2.620193in}{2.742853in}}%
\pgfpathlineto{\pgfqpoint{2.620193in}{2.747111in}}%
\pgfpathlineto{\pgfqpoint{2.624451in}{2.747111in}}%
\pgfpathlineto{\pgfqpoint{2.624451in}{2.742853in}}%
\pgfpathmoveto{\pgfqpoint{2.620193in}{2.747111in}}%
\pgfpathlineto{\pgfqpoint{2.620193in}{2.747111in}}%
\pgfpathlineto{\pgfqpoint{2.620193in}{2.751368in}}%
\pgfpathlineto{\pgfqpoint{2.624451in}{2.751368in}}%
\pgfpathlineto{\pgfqpoint{2.624451in}{2.747111in}}%
\pgfpathmoveto{\pgfqpoint{2.620193in}{2.751368in}}%
\pgfpathlineto{\pgfqpoint{2.620193in}{2.751368in}}%
\pgfpathlineto{\pgfqpoint{2.620193in}{2.755626in}}%
\pgfpathlineto{\pgfqpoint{2.624451in}{2.755626in}}%
\pgfpathlineto{\pgfqpoint{2.624451in}{2.751368in}}%
\pgfpathmoveto{\pgfqpoint{2.624451in}{2.738595in}}%
\pgfpathlineto{\pgfqpoint{2.624451in}{2.738595in}}%
\pgfpathlineto{\pgfqpoint{2.624451in}{2.742853in}}%
\pgfpathlineto{\pgfqpoint{2.628708in}{2.742853in}}%
\pgfpathlineto{\pgfqpoint{2.628708in}{2.738595in}}%
\pgfpathmoveto{\pgfqpoint{2.624451in}{2.742853in}}%
\pgfpathlineto{\pgfqpoint{2.624451in}{2.742853in}}%
\pgfpathlineto{\pgfqpoint{2.624451in}{2.747111in}}%
\pgfpathlineto{\pgfqpoint{2.628708in}{2.747111in}}%
\pgfpathlineto{\pgfqpoint{2.628708in}{2.742853in}}%
\pgfpathmoveto{\pgfqpoint{2.628708in}{2.738595in}}%
\pgfpathlineto{\pgfqpoint{2.628708in}{2.738595in}}%
\pgfpathlineto{\pgfqpoint{2.628708in}{2.742853in}}%
\pgfpathlineto{\pgfqpoint{2.632966in}{2.742853in}}%
\pgfpathlineto{\pgfqpoint{2.632966in}{2.738595in}}%
\pgfpathmoveto{\pgfqpoint{2.628708in}{2.742853in}}%
\pgfpathlineto{\pgfqpoint{2.628708in}{2.742853in}}%
\pgfpathlineto{\pgfqpoint{2.628708in}{2.747111in}}%
\pgfpathlineto{\pgfqpoint{2.632966in}{2.747111in}}%
\pgfpathlineto{\pgfqpoint{2.632966in}{2.742853in}}%
\pgfpathmoveto{\pgfqpoint{2.624451in}{2.747111in}}%
\pgfpathlineto{\pgfqpoint{2.624451in}{2.747111in}}%
\pgfpathlineto{\pgfqpoint{2.624451in}{2.751368in}}%
\pgfpathlineto{\pgfqpoint{2.628708in}{2.751368in}}%
\pgfpathlineto{\pgfqpoint{2.628708in}{2.747111in}}%
\pgfpathmoveto{\pgfqpoint{2.624451in}{2.751368in}}%
\pgfpathlineto{\pgfqpoint{2.624451in}{2.751368in}}%
\pgfpathlineto{\pgfqpoint{2.624451in}{2.755626in}}%
\pgfpathlineto{\pgfqpoint{2.628708in}{2.755626in}}%
\pgfpathlineto{\pgfqpoint{2.628708in}{2.751368in}}%
\pgfpathmoveto{\pgfqpoint{2.628708in}{2.747111in}}%
\pgfpathlineto{\pgfqpoint{2.628708in}{2.747111in}}%
\pgfpathlineto{\pgfqpoint{2.628708in}{2.751368in}}%
\pgfpathlineto{\pgfqpoint{2.632966in}{2.751368in}}%
\pgfpathlineto{\pgfqpoint{2.632966in}{2.747111in}}%
\pgfpathmoveto{\pgfqpoint{2.628708in}{2.751368in}}%
\pgfpathlineto{\pgfqpoint{2.628708in}{2.751368in}}%
\pgfpathlineto{\pgfqpoint{2.628708in}{2.755626in}}%
\pgfpathlineto{\pgfqpoint{2.632966in}{2.755626in}}%
\pgfpathlineto{\pgfqpoint{2.632966in}{2.751368in}}%
\pgfpathmoveto{\pgfqpoint{2.620193in}{2.755626in}}%
\pgfpathlineto{\pgfqpoint{2.620193in}{2.755626in}}%
\pgfpathlineto{\pgfqpoint{2.620193in}{2.759884in}}%
\pgfpathlineto{\pgfqpoint{2.624451in}{2.759884in}}%
\pgfpathlineto{\pgfqpoint{2.624451in}{2.755626in}}%
\pgfpathmoveto{\pgfqpoint{2.624451in}{2.755626in}}%
\pgfpathlineto{\pgfqpoint{2.624451in}{2.755626in}}%
\pgfpathlineto{\pgfqpoint{2.624451in}{2.759884in}}%
\pgfpathlineto{\pgfqpoint{2.628708in}{2.759884in}}%
\pgfpathlineto{\pgfqpoint{2.628708in}{2.755626in}}%
\pgfpathmoveto{\pgfqpoint{2.624451in}{2.759884in}}%
\pgfpathlineto{\pgfqpoint{2.624451in}{2.759884in}}%
\pgfpathlineto{\pgfqpoint{2.624451in}{2.764142in}}%
\pgfpathlineto{\pgfqpoint{2.628708in}{2.764142in}}%
\pgfpathlineto{\pgfqpoint{2.628708in}{2.759884in}}%
\pgfpathmoveto{\pgfqpoint{2.628708in}{2.755626in}}%
\pgfpathlineto{\pgfqpoint{2.628708in}{2.755626in}}%
\pgfpathlineto{\pgfqpoint{2.628708in}{2.759884in}}%
\pgfpathlineto{\pgfqpoint{2.632966in}{2.759884in}}%
\pgfpathlineto{\pgfqpoint{2.632966in}{2.755626in}}%
\pgfpathmoveto{\pgfqpoint{2.628708in}{2.759884in}}%
\pgfpathlineto{\pgfqpoint{2.628708in}{2.759884in}}%
\pgfpathlineto{\pgfqpoint{2.628708in}{2.764142in}}%
\pgfpathlineto{\pgfqpoint{2.632966in}{2.764142in}}%
\pgfpathlineto{\pgfqpoint{2.632966in}{2.759884in}}%
\pgfpathmoveto{\pgfqpoint{2.624451in}{2.764142in}}%
\pgfpathlineto{\pgfqpoint{2.624451in}{2.764142in}}%
\pgfpathlineto{\pgfqpoint{2.624451in}{2.768400in}}%
\pgfpathlineto{\pgfqpoint{2.628708in}{2.768400in}}%
\pgfpathlineto{\pgfqpoint{2.628708in}{2.764142in}}%
\pgfpathmoveto{\pgfqpoint{2.624451in}{2.768400in}}%
\pgfpathlineto{\pgfqpoint{2.624451in}{2.768400in}}%
\pgfpathlineto{\pgfqpoint{2.624451in}{2.772657in}}%
\pgfpathlineto{\pgfqpoint{2.628708in}{2.772657in}}%
\pgfpathlineto{\pgfqpoint{2.628708in}{2.768400in}}%
\pgfpathmoveto{\pgfqpoint{2.628708in}{2.764142in}}%
\pgfpathlineto{\pgfqpoint{2.628708in}{2.764142in}}%
\pgfpathlineto{\pgfqpoint{2.628708in}{2.768400in}}%
\pgfpathlineto{\pgfqpoint{2.632966in}{2.768400in}}%
\pgfpathlineto{\pgfqpoint{2.632966in}{2.764142in}}%
\pgfpathmoveto{\pgfqpoint{2.628708in}{2.768400in}}%
\pgfpathlineto{\pgfqpoint{2.628708in}{2.768400in}}%
\pgfpathlineto{\pgfqpoint{2.628708in}{2.772657in}}%
\pgfpathlineto{\pgfqpoint{2.632966in}{2.772657in}}%
\pgfpathlineto{\pgfqpoint{2.632966in}{2.768400in}}%
\pgfpathmoveto{\pgfqpoint{2.624451in}{2.772657in}}%
\pgfpathlineto{\pgfqpoint{2.624451in}{2.772657in}}%
\pgfpathlineto{\pgfqpoint{2.624451in}{2.776915in}}%
\pgfpathlineto{\pgfqpoint{2.628708in}{2.776915in}}%
\pgfpathlineto{\pgfqpoint{2.628708in}{2.772657in}}%
\pgfpathmoveto{\pgfqpoint{2.624451in}{2.776915in}}%
\pgfpathlineto{\pgfqpoint{2.624451in}{2.776915in}}%
\pgfpathlineto{\pgfqpoint{2.624451in}{2.781173in}}%
\pgfpathlineto{\pgfqpoint{2.628708in}{2.781173in}}%
\pgfpathlineto{\pgfqpoint{2.628708in}{2.776915in}}%
\pgfpathmoveto{\pgfqpoint{2.628708in}{2.772657in}}%
\pgfpathlineto{\pgfqpoint{2.628708in}{2.772657in}}%
\pgfpathlineto{\pgfqpoint{2.628708in}{2.776915in}}%
\pgfpathlineto{\pgfqpoint{2.632966in}{2.776915in}}%
\pgfpathlineto{\pgfqpoint{2.632966in}{2.772657in}}%
\pgfpathmoveto{\pgfqpoint{2.628708in}{2.776915in}}%
\pgfpathlineto{\pgfqpoint{2.628708in}{2.776915in}}%
\pgfpathlineto{\pgfqpoint{2.628708in}{2.781173in}}%
\pgfpathlineto{\pgfqpoint{2.632966in}{2.781173in}}%
\pgfpathlineto{\pgfqpoint{2.632966in}{2.776915in}}%
\pgfpathmoveto{\pgfqpoint{2.624451in}{2.781173in}}%
\pgfpathlineto{\pgfqpoint{2.624451in}{2.781173in}}%
\pgfpathlineto{\pgfqpoint{2.624451in}{2.785431in}}%
\pgfpathlineto{\pgfqpoint{2.628708in}{2.785431in}}%
\pgfpathlineto{\pgfqpoint{2.628708in}{2.781173in}}%
\pgfpathmoveto{\pgfqpoint{2.628708in}{2.781173in}}%
\pgfpathlineto{\pgfqpoint{2.628708in}{2.781173in}}%
\pgfpathlineto{\pgfqpoint{2.628708in}{2.785431in}}%
\pgfpathlineto{\pgfqpoint{2.632966in}{2.785431in}}%
\pgfpathlineto{\pgfqpoint{2.632966in}{2.781173in}}%
\pgfpathmoveto{\pgfqpoint{2.628708in}{2.785431in}}%
\pgfpathlineto{\pgfqpoint{2.628708in}{2.785431in}}%
\pgfpathlineto{\pgfqpoint{2.628708in}{2.789689in}}%
\pgfpathlineto{\pgfqpoint{2.632966in}{2.789689in}}%
\pgfpathlineto{\pgfqpoint{2.632966in}{2.785431in}}%
\pgfpathmoveto{\pgfqpoint{2.632966in}{2.764142in}}%
\pgfpathlineto{\pgfqpoint{2.632966in}{2.764142in}}%
\pgfpathlineto{\pgfqpoint{2.632966in}{2.768400in}}%
\pgfpathlineto{\pgfqpoint{2.637224in}{2.768400in}}%
\pgfpathlineto{\pgfqpoint{2.637224in}{2.764142in}}%
\pgfpathmoveto{\pgfqpoint{2.632966in}{2.768400in}}%
\pgfpathlineto{\pgfqpoint{2.632966in}{2.768400in}}%
\pgfpathlineto{\pgfqpoint{2.632966in}{2.772657in}}%
\pgfpathlineto{\pgfqpoint{2.637224in}{2.772657in}}%
\pgfpathlineto{\pgfqpoint{2.637224in}{2.768400in}}%
\pgfpathmoveto{\pgfqpoint{2.632966in}{2.772657in}}%
\pgfpathlineto{\pgfqpoint{2.632966in}{2.772657in}}%
\pgfpathlineto{\pgfqpoint{2.632966in}{2.776915in}}%
\pgfpathlineto{\pgfqpoint{2.637224in}{2.776915in}}%
\pgfpathlineto{\pgfqpoint{2.637224in}{2.772657in}}%
\pgfpathmoveto{\pgfqpoint{2.632966in}{2.776915in}}%
\pgfpathlineto{\pgfqpoint{2.632966in}{2.776915in}}%
\pgfpathlineto{\pgfqpoint{2.632966in}{2.781173in}}%
\pgfpathlineto{\pgfqpoint{2.637224in}{2.781173in}}%
\pgfpathlineto{\pgfqpoint{2.637224in}{2.776915in}}%
\pgfpathmoveto{\pgfqpoint{2.632966in}{2.781173in}}%
\pgfpathlineto{\pgfqpoint{2.632966in}{2.781173in}}%
\pgfpathlineto{\pgfqpoint{2.632966in}{2.785431in}}%
\pgfpathlineto{\pgfqpoint{2.637224in}{2.785431in}}%
\pgfpathlineto{\pgfqpoint{2.637224in}{2.781173in}}%
\pgfpathmoveto{\pgfqpoint{2.632966in}{2.785431in}}%
\pgfpathlineto{\pgfqpoint{2.632966in}{2.785431in}}%
\pgfpathlineto{\pgfqpoint{2.632966in}{2.789689in}}%
\pgfpathlineto{\pgfqpoint{2.637224in}{2.789689in}}%
\pgfpathlineto{\pgfqpoint{2.637224in}{2.785431in}}%
\pgfpathmoveto{\pgfqpoint{2.628708in}{2.789689in}}%
\pgfpathlineto{\pgfqpoint{2.628708in}{2.789689in}}%
\pgfpathlineto{\pgfqpoint{2.628708in}{2.793946in}}%
\pgfpathlineto{\pgfqpoint{2.632966in}{2.793946in}}%
\pgfpathlineto{\pgfqpoint{2.632966in}{2.789689in}}%
\pgfpathmoveto{\pgfqpoint{2.628708in}{2.793946in}}%
\pgfpathlineto{\pgfqpoint{2.628708in}{2.793946in}}%
\pgfpathlineto{\pgfqpoint{2.628708in}{2.798204in}}%
\pgfpathlineto{\pgfqpoint{2.632966in}{2.798204in}}%
\pgfpathlineto{\pgfqpoint{2.632966in}{2.793946in}}%
\pgfpathmoveto{\pgfqpoint{2.628708in}{2.798204in}}%
\pgfpathlineto{\pgfqpoint{2.628708in}{2.798204in}}%
\pgfpathlineto{\pgfqpoint{2.628708in}{2.802462in}}%
\pgfpathlineto{\pgfqpoint{2.632966in}{2.802462in}}%
\pgfpathlineto{\pgfqpoint{2.632966in}{2.798204in}}%
\pgfpathmoveto{\pgfqpoint{2.628708in}{2.802462in}}%
\pgfpathlineto{\pgfqpoint{2.628708in}{2.802462in}}%
\pgfpathlineto{\pgfqpoint{2.628708in}{2.806720in}}%
\pgfpathlineto{\pgfqpoint{2.632966in}{2.806720in}}%
\pgfpathlineto{\pgfqpoint{2.632966in}{2.802462in}}%
\pgfpathmoveto{\pgfqpoint{2.628708in}{2.806720in}}%
\pgfpathlineto{\pgfqpoint{2.628708in}{2.806720in}}%
\pgfpathlineto{\pgfqpoint{2.628708in}{2.810978in}}%
\pgfpathlineto{\pgfqpoint{2.632966in}{2.810978in}}%
\pgfpathlineto{\pgfqpoint{2.632966in}{2.806720in}}%
\pgfpathmoveto{\pgfqpoint{2.632966in}{2.789689in}}%
\pgfpathlineto{\pgfqpoint{2.632966in}{2.789689in}}%
\pgfpathlineto{\pgfqpoint{2.632966in}{2.793946in}}%
\pgfpathlineto{\pgfqpoint{2.637224in}{2.793946in}}%
\pgfpathlineto{\pgfqpoint{2.637224in}{2.789689in}}%
\pgfpathmoveto{\pgfqpoint{2.632966in}{2.793946in}}%
\pgfpathlineto{\pgfqpoint{2.632966in}{2.793946in}}%
\pgfpathlineto{\pgfqpoint{2.632966in}{2.798204in}}%
\pgfpathlineto{\pgfqpoint{2.637224in}{2.798204in}}%
\pgfpathlineto{\pgfqpoint{2.637224in}{2.793946in}}%
\pgfpathmoveto{\pgfqpoint{2.637224in}{2.789689in}}%
\pgfpathlineto{\pgfqpoint{2.637224in}{2.789689in}}%
\pgfpathlineto{\pgfqpoint{2.637224in}{2.793946in}}%
\pgfpathlineto{\pgfqpoint{2.641482in}{2.793946in}}%
\pgfpathlineto{\pgfqpoint{2.641482in}{2.789689in}}%
\pgfpathmoveto{\pgfqpoint{2.637224in}{2.793946in}}%
\pgfpathlineto{\pgfqpoint{2.637224in}{2.793946in}}%
\pgfpathlineto{\pgfqpoint{2.637224in}{2.798204in}}%
\pgfpathlineto{\pgfqpoint{2.641482in}{2.798204in}}%
\pgfpathlineto{\pgfqpoint{2.641482in}{2.793946in}}%
\pgfpathmoveto{\pgfqpoint{2.632966in}{2.798204in}}%
\pgfpathlineto{\pgfqpoint{2.632966in}{2.798204in}}%
\pgfpathlineto{\pgfqpoint{2.632966in}{2.802462in}}%
\pgfpathlineto{\pgfqpoint{2.637224in}{2.802462in}}%
\pgfpathlineto{\pgfqpoint{2.637224in}{2.798204in}}%
\pgfpathmoveto{\pgfqpoint{2.632966in}{2.802462in}}%
\pgfpathlineto{\pgfqpoint{2.632966in}{2.802462in}}%
\pgfpathlineto{\pgfqpoint{2.632966in}{2.806720in}}%
\pgfpathlineto{\pgfqpoint{2.637224in}{2.806720in}}%
\pgfpathlineto{\pgfqpoint{2.637224in}{2.802462in}}%
\pgfpathmoveto{\pgfqpoint{2.637224in}{2.798204in}}%
\pgfpathlineto{\pgfqpoint{2.637224in}{2.798204in}}%
\pgfpathlineto{\pgfqpoint{2.637224in}{2.802462in}}%
\pgfpathlineto{\pgfqpoint{2.641482in}{2.802462in}}%
\pgfpathlineto{\pgfqpoint{2.641482in}{2.798204in}}%
\pgfpathmoveto{\pgfqpoint{2.637224in}{2.802462in}}%
\pgfpathlineto{\pgfqpoint{2.637224in}{2.802462in}}%
\pgfpathlineto{\pgfqpoint{2.637224in}{2.806720in}}%
\pgfpathlineto{\pgfqpoint{2.641482in}{2.806720in}}%
\pgfpathlineto{\pgfqpoint{2.641482in}{2.802462in}}%
\pgfpathmoveto{\pgfqpoint{2.632966in}{2.806720in}}%
\pgfpathlineto{\pgfqpoint{2.632966in}{2.806720in}}%
\pgfpathlineto{\pgfqpoint{2.632966in}{2.810978in}}%
\pgfpathlineto{\pgfqpoint{2.637224in}{2.810978in}}%
\pgfpathlineto{\pgfqpoint{2.637224in}{2.806720in}}%
\pgfpathmoveto{\pgfqpoint{2.632966in}{2.810978in}}%
\pgfpathlineto{\pgfqpoint{2.632966in}{2.810978in}}%
\pgfpathlineto{\pgfqpoint{2.632966in}{2.815235in}}%
\pgfpathlineto{\pgfqpoint{2.637224in}{2.815235in}}%
\pgfpathlineto{\pgfqpoint{2.637224in}{2.810978in}}%
\pgfpathmoveto{\pgfqpoint{2.637224in}{2.806720in}}%
\pgfpathlineto{\pgfqpoint{2.637224in}{2.806720in}}%
\pgfpathlineto{\pgfqpoint{2.637224in}{2.810978in}}%
\pgfpathlineto{\pgfqpoint{2.641482in}{2.810978in}}%
\pgfpathlineto{\pgfqpoint{2.641482in}{2.806720in}}%
\pgfpathmoveto{\pgfqpoint{2.637224in}{2.810978in}}%
\pgfpathlineto{\pgfqpoint{2.637224in}{2.810978in}}%
\pgfpathlineto{\pgfqpoint{2.637224in}{2.815235in}}%
\pgfpathlineto{\pgfqpoint{2.641482in}{2.815235in}}%
\pgfpathlineto{\pgfqpoint{2.641482in}{2.810978in}}%
\pgfpathmoveto{\pgfqpoint{2.632966in}{2.815235in}}%
\pgfpathlineto{\pgfqpoint{2.632966in}{2.815235in}}%
\pgfpathlineto{\pgfqpoint{2.632966in}{2.819493in}}%
\pgfpathlineto{\pgfqpoint{2.637224in}{2.819493in}}%
\pgfpathlineto{\pgfqpoint{2.637224in}{2.815235in}}%
\pgfpathmoveto{\pgfqpoint{2.632966in}{2.819493in}}%
\pgfpathlineto{\pgfqpoint{2.632966in}{2.819493in}}%
\pgfpathlineto{\pgfqpoint{2.632966in}{2.823751in}}%
\pgfpathlineto{\pgfqpoint{2.637224in}{2.823751in}}%
\pgfpathlineto{\pgfqpoint{2.637224in}{2.819493in}}%
\pgfpathmoveto{\pgfqpoint{2.637224in}{2.815235in}}%
\pgfpathlineto{\pgfqpoint{2.637224in}{2.815235in}}%
\pgfpathlineto{\pgfqpoint{2.637224in}{2.819493in}}%
\pgfpathlineto{\pgfqpoint{2.641482in}{2.819493in}}%
\pgfpathlineto{\pgfqpoint{2.641482in}{2.815235in}}%
\pgfpathmoveto{\pgfqpoint{2.637224in}{2.819493in}}%
\pgfpathlineto{\pgfqpoint{2.637224in}{2.819493in}}%
\pgfpathlineto{\pgfqpoint{2.637224in}{2.823751in}}%
\pgfpathlineto{\pgfqpoint{2.641482in}{2.823751in}}%
\pgfpathlineto{\pgfqpoint{2.641482in}{2.819493in}}%
\pgfpathmoveto{\pgfqpoint{2.641482in}{2.815235in}}%
\pgfpathlineto{\pgfqpoint{2.641482in}{2.815235in}}%
\pgfpathlineto{\pgfqpoint{2.641482in}{2.819493in}}%
\pgfpathlineto{\pgfqpoint{2.645739in}{2.819493in}}%
\pgfpathlineto{\pgfqpoint{2.645739in}{2.815235in}}%
\pgfpathmoveto{\pgfqpoint{2.641482in}{2.819493in}}%
\pgfpathlineto{\pgfqpoint{2.641482in}{2.819493in}}%
\pgfpathlineto{\pgfqpoint{2.641482in}{2.823751in}}%
\pgfpathlineto{\pgfqpoint{2.645739in}{2.823751in}}%
\pgfpathlineto{\pgfqpoint{2.645739in}{2.819493in}}%
\pgfpathmoveto{\pgfqpoint{2.632966in}{2.823751in}}%
\pgfpathlineto{\pgfqpoint{2.632966in}{2.823751in}}%
\pgfpathlineto{\pgfqpoint{2.632966in}{2.828009in}}%
\pgfpathlineto{\pgfqpoint{2.637224in}{2.828009in}}%
\pgfpathlineto{\pgfqpoint{2.637224in}{2.823751in}}%
\pgfpathmoveto{\pgfqpoint{2.632966in}{2.828009in}}%
\pgfpathlineto{\pgfqpoint{2.632966in}{2.828009in}}%
\pgfpathlineto{\pgfqpoint{2.632966in}{2.832266in}}%
\pgfpathlineto{\pgfqpoint{2.637224in}{2.832266in}}%
\pgfpathlineto{\pgfqpoint{2.637224in}{2.828009in}}%
\pgfpathmoveto{\pgfqpoint{2.637224in}{2.823751in}}%
\pgfpathlineto{\pgfqpoint{2.637224in}{2.823751in}}%
\pgfpathlineto{\pgfqpoint{2.637224in}{2.828009in}}%
\pgfpathlineto{\pgfqpoint{2.641482in}{2.828009in}}%
\pgfpathlineto{\pgfqpoint{2.641482in}{2.823751in}}%
\pgfpathmoveto{\pgfqpoint{2.637224in}{2.828009in}}%
\pgfpathlineto{\pgfqpoint{2.637224in}{2.828009in}}%
\pgfpathlineto{\pgfqpoint{2.637224in}{2.832266in}}%
\pgfpathlineto{\pgfqpoint{2.641482in}{2.832266in}}%
\pgfpathlineto{\pgfqpoint{2.641482in}{2.828009in}}%
\pgfpathmoveto{\pgfqpoint{2.632966in}{2.832266in}}%
\pgfpathlineto{\pgfqpoint{2.632966in}{2.832266in}}%
\pgfpathlineto{\pgfqpoint{2.632966in}{2.836524in}}%
\pgfpathlineto{\pgfqpoint{2.637224in}{2.836524in}}%
\pgfpathlineto{\pgfqpoint{2.637224in}{2.832266in}}%
\pgfpathmoveto{\pgfqpoint{2.637224in}{2.832266in}}%
\pgfpathlineto{\pgfqpoint{2.637224in}{2.832266in}}%
\pgfpathlineto{\pgfqpoint{2.637224in}{2.836524in}}%
\pgfpathlineto{\pgfqpoint{2.641482in}{2.836524in}}%
\pgfpathlineto{\pgfqpoint{2.641482in}{2.832266in}}%
\pgfpathmoveto{\pgfqpoint{2.637224in}{2.836524in}}%
\pgfpathlineto{\pgfqpoint{2.637224in}{2.836524in}}%
\pgfpathlineto{\pgfqpoint{2.637224in}{2.840782in}}%
\pgfpathlineto{\pgfqpoint{2.641482in}{2.840782in}}%
\pgfpathlineto{\pgfqpoint{2.641482in}{2.836524in}}%
\pgfpathmoveto{\pgfqpoint{2.641482in}{2.823751in}}%
\pgfpathlineto{\pgfqpoint{2.641482in}{2.823751in}}%
\pgfpathlineto{\pgfqpoint{2.641482in}{2.828009in}}%
\pgfpathlineto{\pgfqpoint{2.645739in}{2.828009in}}%
\pgfpathlineto{\pgfqpoint{2.645739in}{2.823751in}}%
\pgfpathmoveto{\pgfqpoint{2.641482in}{2.828009in}}%
\pgfpathlineto{\pgfqpoint{2.641482in}{2.828009in}}%
\pgfpathlineto{\pgfqpoint{2.641482in}{2.832266in}}%
\pgfpathlineto{\pgfqpoint{2.645739in}{2.832266in}}%
\pgfpathlineto{\pgfqpoint{2.645739in}{2.828009in}}%
\pgfpathmoveto{\pgfqpoint{2.641482in}{2.832266in}}%
\pgfpathlineto{\pgfqpoint{2.641482in}{2.832266in}}%
\pgfpathlineto{\pgfqpoint{2.641482in}{2.836524in}}%
\pgfpathlineto{\pgfqpoint{2.645739in}{2.836524in}}%
\pgfpathlineto{\pgfqpoint{2.645739in}{2.832266in}}%
\pgfpathmoveto{\pgfqpoint{2.641482in}{2.836524in}}%
\pgfpathlineto{\pgfqpoint{2.641482in}{2.836524in}}%
\pgfpathlineto{\pgfqpoint{2.641482in}{2.840782in}}%
\pgfpathlineto{\pgfqpoint{2.645739in}{2.840782in}}%
\pgfpathlineto{\pgfqpoint{2.645739in}{2.836524in}}%
\pgfpathmoveto{\pgfqpoint{2.637224in}{2.840782in}}%
\pgfpathlineto{\pgfqpoint{2.637224in}{2.840782in}}%
\pgfpathlineto{\pgfqpoint{2.637224in}{2.845039in}}%
\pgfpathlineto{\pgfqpoint{2.641482in}{2.845039in}}%
\pgfpathlineto{\pgfqpoint{2.641482in}{2.840782in}}%
\pgfpathmoveto{\pgfqpoint{2.637224in}{2.845039in}}%
\pgfpathlineto{\pgfqpoint{2.637224in}{2.845039in}}%
\pgfpathlineto{\pgfqpoint{2.637224in}{2.849297in}}%
\pgfpathlineto{\pgfqpoint{2.641482in}{2.849297in}}%
\pgfpathlineto{\pgfqpoint{2.641482in}{2.845039in}}%
\pgfpathmoveto{\pgfqpoint{2.637224in}{2.849297in}}%
\pgfpathlineto{\pgfqpoint{2.637224in}{2.849297in}}%
\pgfpathlineto{\pgfqpoint{2.637224in}{2.853555in}}%
\pgfpathlineto{\pgfqpoint{2.641482in}{2.853555in}}%
\pgfpathlineto{\pgfqpoint{2.641482in}{2.849297in}}%
\pgfpathmoveto{\pgfqpoint{2.637224in}{2.853555in}}%
\pgfpathlineto{\pgfqpoint{2.637224in}{2.853555in}}%
\pgfpathlineto{\pgfqpoint{2.637224in}{2.857813in}}%
\pgfpathlineto{\pgfqpoint{2.641482in}{2.857813in}}%
\pgfpathlineto{\pgfqpoint{2.641482in}{2.853555in}}%
\pgfpathmoveto{\pgfqpoint{2.641482in}{2.840782in}}%
\pgfpathlineto{\pgfqpoint{2.641482in}{2.840782in}}%
\pgfpathlineto{\pgfqpoint{2.641482in}{2.845039in}}%
\pgfpathlineto{\pgfqpoint{2.645739in}{2.845039in}}%
\pgfpathlineto{\pgfqpoint{2.645739in}{2.840782in}}%
\pgfpathmoveto{\pgfqpoint{2.641482in}{2.845039in}}%
\pgfpathlineto{\pgfqpoint{2.641482in}{2.845039in}}%
\pgfpathlineto{\pgfqpoint{2.641482in}{2.849297in}}%
\pgfpathlineto{\pgfqpoint{2.645739in}{2.849297in}}%
\pgfpathlineto{\pgfqpoint{2.645739in}{2.845039in}}%
\pgfpathmoveto{\pgfqpoint{2.645739in}{2.840782in}}%
\pgfpathlineto{\pgfqpoint{2.645739in}{2.840782in}}%
\pgfpathlineto{\pgfqpoint{2.645739in}{2.845039in}}%
\pgfpathlineto{\pgfqpoint{2.649997in}{2.845039in}}%
\pgfpathlineto{\pgfqpoint{2.649997in}{2.840782in}}%
\pgfpathmoveto{\pgfqpoint{2.645739in}{2.845039in}}%
\pgfpathlineto{\pgfqpoint{2.645739in}{2.845039in}}%
\pgfpathlineto{\pgfqpoint{2.645739in}{2.849297in}}%
\pgfpathlineto{\pgfqpoint{2.649997in}{2.849297in}}%
\pgfpathlineto{\pgfqpoint{2.649997in}{2.845039in}}%
\pgfpathmoveto{\pgfqpoint{2.641482in}{2.849297in}}%
\pgfpathlineto{\pgfqpoint{2.641482in}{2.849297in}}%
\pgfpathlineto{\pgfqpoint{2.641482in}{2.853555in}}%
\pgfpathlineto{\pgfqpoint{2.645739in}{2.853555in}}%
\pgfpathlineto{\pgfqpoint{2.645739in}{2.849297in}}%
\pgfpathmoveto{\pgfqpoint{2.641482in}{2.853555in}}%
\pgfpathlineto{\pgfqpoint{2.641482in}{2.853555in}}%
\pgfpathlineto{\pgfqpoint{2.641482in}{2.857813in}}%
\pgfpathlineto{\pgfqpoint{2.645739in}{2.857813in}}%
\pgfpathlineto{\pgfqpoint{2.645739in}{2.853555in}}%
\pgfpathmoveto{\pgfqpoint{2.645739in}{2.849297in}}%
\pgfpathlineto{\pgfqpoint{2.645739in}{2.849297in}}%
\pgfpathlineto{\pgfqpoint{2.645739in}{2.853555in}}%
\pgfpathlineto{\pgfqpoint{2.649997in}{2.853555in}}%
\pgfpathlineto{\pgfqpoint{2.649997in}{2.849297in}}%
\pgfpathmoveto{\pgfqpoint{2.645739in}{2.853555in}}%
\pgfpathlineto{\pgfqpoint{2.645739in}{2.853555in}}%
\pgfpathlineto{\pgfqpoint{2.645739in}{2.857813in}}%
\pgfpathlineto{\pgfqpoint{2.649997in}{2.857813in}}%
\pgfpathlineto{\pgfqpoint{2.649997in}{2.853555in}}%
\pgfpathmoveto{\pgfqpoint{2.637224in}{2.857813in}}%
\pgfpathlineto{\pgfqpoint{2.637224in}{2.857813in}}%
\pgfpathlineto{\pgfqpoint{2.637224in}{2.862070in}}%
\pgfpathlineto{\pgfqpoint{2.641482in}{2.862070in}}%
\pgfpathlineto{\pgfqpoint{2.641482in}{2.857813in}}%
\pgfpathmoveto{\pgfqpoint{2.641482in}{2.857813in}}%
\pgfpathlineto{\pgfqpoint{2.641482in}{2.857813in}}%
\pgfpathlineto{\pgfqpoint{2.641482in}{2.862070in}}%
\pgfpathlineto{\pgfqpoint{2.645739in}{2.862070in}}%
\pgfpathlineto{\pgfqpoint{2.645739in}{2.857813in}}%
\pgfpathmoveto{\pgfqpoint{2.641482in}{2.862070in}}%
\pgfpathlineto{\pgfqpoint{2.641482in}{2.862070in}}%
\pgfpathlineto{\pgfqpoint{2.641482in}{2.866328in}}%
\pgfpathlineto{\pgfqpoint{2.645739in}{2.866328in}}%
\pgfpathlineto{\pgfqpoint{2.645739in}{2.862070in}}%
\pgfpathmoveto{\pgfqpoint{2.645739in}{2.857813in}}%
\pgfpathlineto{\pgfqpoint{2.645739in}{2.857813in}}%
\pgfpathlineto{\pgfqpoint{2.645739in}{2.862070in}}%
\pgfpathlineto{\pgfqpoint{2.649997in}{2.862070in}}%
\pgfpathlineto{\pgfqpoint{2.649997in}{2.857813in}}%
\pgfpathmoveto{\pgfqpoint{2.645739in}{2.862070in}}%
\pgfpathlineto{\pgfqpoint{2.645739in}{2.862070in}}%
\pgfpathlineto{\pgfqpoint{2.645739in}{2.866328in}}%
\pgfpathlineto{\pgfqpoint{2.649997in}{2.866328in}}%
\pgfpathlineto{\pgfqpoint{2.649997in}{2.862070in}}%
\pgfpathmoveto{\pgfqpoint{2.641482in}{2.866328in}}%
\pgfpathlineto{\pgfqpoint{2.641482in}{2.866328in}}%
\pgfpathlineto{\pgfqpoint{2.641482in}{2.870586in}}%
\pgfpathlineto{\pgfqpoint{2.645739in}{2.870586in}}%
\pgfpathlineto{\pgfqpoint{2.645739in}{2.866328in}}%
\pgfpathmoveto{\pgfqpoint{2.641482in}{2.870586in}}%
\pgfpathlineto{\pgfqpoint{2.641482in}{2.870586in}}%
\pgfpathlineto{\pgfqpoint{2.641482in}{2.874844in}}%
\pgfpathlineto{\pgfqpoint{2.645739in}{2.874844in}}%
\pgfpathlineto{\pgfqpoint{2.645739in}{2.870586in}}%
\pgfpathmoveto{\pgfqpoint{2.645739in}{2.866328in}}%
\pgfpathlineto{\pgfqpoint{2.645739in}{2.866328in}}%
\pgfpathlineto{\pgfqpoint{2.645739in}{2.870586in}}%
\pgfpathlineto{\pgfqpoint{2.649997in}{2.870586in}}%
\pgfpathlineto{\pgfqpoint{2.649997in}{2.866328in}}%
\pgfpathmoveto{\pgfqpoint{2.645739in}{2.870586in}}%
\pgfpathlineto{\pgfqpoint{2.645739in}{2.870586in}}%
\pgfpathlineto{\pgfqpoint{2.645739in}{2.874844in}}%
\pgfpathlineto{\pgfqpoint{2.649997in}{2.874844in}}%
\pgfpathlineto{\pgfqpoint{2.649997in}{2.870586in}}%
\pgfpathmoveto{\pgfqpoint{2.641482in}{2.874844in}}%
\pgfpathlineto{\pgfqpoint{2.641482in}{2.874844in}}%
\pgfpathlineto{\pgfqpoint{2.641482in}{2.879101in}}%
\pgfpathlineto{\pgfqpoint{2.645739in}{2.879101in}}%
\pgfpathlineto{\pgfqpoint{2.645739in}{2.874844in}}%
\pgfpathmoveto{\pgfqpoint{2.641482in}{2.879101in}}%
\pgfpathlineto{\pgfqpoint{2.641482in}{2.879101in}}%
\pgfpathlineto{\pgfqpoint{2.641482in}{2.883359in}}%
\pgfpathlineto{\pgfqpoint{2.645739in}{2.883359in}}%
\pgfpathlineto{\pgfqpoint{2.645739in}{2.879101in}}%
\pgfpathmoveto{\pgfqpoint{2.645739in}{2.874844in}}%
\pgfpathlineto{\pgfqpoint{2.645739in}{2.874844in}}%
\pgfpathlineto{\pgfqpoint{2.645739in}{2.879101in}}%
\pgfpathlineto{\pgfqpoint{2.649997in}{2.879101in}}%
\pgfpathlineto{\pgfqpoint{2.649997in}{2.874844in}}%
\pgfpathmoveto{\pgfqpoint{2.645739in}{2.879101in}}%
\pgfpathlineto{\pgfqpoint{2.645739in}{2.879101in}}%
\pgfpathlineto{\pgfqpoint{2.645739in}{2.883359in}}%
\pgfpathlineto{\pgfqpoint{2.649997in}{2.883359in}}%
\pgfpathlineto{\pgfqpoint{2.649997in}{2.879101in}}%
\pgfpathmoveto{\pgfqpoint{2.641482in}{2.883359in}}%
\pgfpathlineto{\pgfqpoint{2.641482in}{2.883359in}}%
\pgfpathlineto{\pgfqpoint{2.641482in}{2.887617in}}%
\pgfpathlineto{\pgfqpoint{2.645739in}{2.887617in}}%
\pgfpathlineto{\pgfqpoint{2.645739in}{2.883359in}}%
\pgfpathmoveto{\pgfqpoint{2.645739in}{2.883359in}}%
\pgfpathlineto{\pgfqpoint{2.645739in}{2.883359in}}%
\pgfpathlineto{\pgfqpoint{2.645739in}{2.887617in}}%
\pgfpathlineto{\pgfqpoint{2.649997in}{2.887617in}}%
\pgfpathlineto{\pgfqpoint{2.649997in}{2.883359in}}%
\pgfpathmoveto{\pgfqpoint{2.645739in}{2.887617in}}%
\pgfpathlineto{\pgfqpoint{2.645739in}{2.887617in}}%
\pgfpathlineto{\pgfqpoint{2.645739in}{2.891874in}}%
\pgfpathlineto{\pgfqpoint{2.649997in}{2.891874in}}%
\pgfpathlineto{\pgfqpoint{2.649997in}{2.887617in}}%
\pgfpathmoveto{\pgfqpoint{2.649997in}{2.866328in}}%
\pgfpathlineto{\pgfqpoint{2.649997in}{2.866328in}}%
\pgfpathlineto{\pgfqpoint{2.649997in}{2.870586in}}%
\pgfpathlineto{\pgfqpoint{2.654255in}{2.870586in}}%
\pgfpathlineto{\pgfqpoint{2.654255in}{2.866328in}}%
\pgfpathmoveto{\pgfqpoint{2.649997in}{2.870586in}}%
\pgfpathlineto{\pgfqpoint{2.649997in}{2.870586in}}%
\pgfpathlineto{\pgfqpoint{2.649997in}{2.874844in}}%
\pgfpathlineto{\pgfqpoint{2.654255in}{2.874844in}}%
\pgfpathlineto{\pgfqpoint{2.654255in}{2.870586in}}%
\pgfpathmoveto{\pgfqpoint{2.649997in}{2.874844in}}%
\pgfpathlineto{\pgfqpoint{2.649997in}{2.874844in}}%
\pgfpathlineto{\pgfqpoint{2.649997in}{2.879101in}}%
\pgfpathlineto{\pgfqpoint{2.654255in}{2.879101in}}%
\pgfpathlineto{\pgfqpoint{2.654255in}{2.874844in}}%
\pgfpathmoveto{\pgfqpoint{2.649997in}{2.879101in}}%
\pgfpathlineto{\pgfqpoint{2.649997in}{2.879101in}}%
\pgfpathlineto{\pgfqpoint{2.649997in}{2.883359in}}%
\pgfpathlineto{\pgfqpoint{2.654255in}{2.883359in}}%
\pgfpathlineto{\pgfqpoint{2.654255in}{2.879101in}}%
\pgfpathmoveto{\pgfqpoint{2.649997in}{2.883359in}}%
\pgfpathlineto{\pgfqpoint{2.649997in}{2.883359in}}%
\pgfpathlineto{\pgfqpoint{2.649997in}{2.887617in}}%
\pgfpathlineto{\pgfqpoint{2.654255in}{2.887617in}}%
\pgfpathlineto{\pgfqpoint{2.654255in}{2.883359in}}%
\pgfpathmoveto{\pgfqpoint{2.649997in}{2.887617in}}%
\pgfpathlineto{\pgfqpoint{2.649997in}{2.887617in}}%
\pgfpathlineto{\pgfqpoint{2.649997in}{2.891874in}}%
\pgfpathlineto{\pgfqpoint{2.654255in}{2.891874in}}%
\pgfpathlineto{\pgfqpoint{2.654255in}{2.887617in}}%
\pgfpathmoveto{\pgfqpoint{2.654255in}{2.887617in}}%
\pgfpathlineto{\pgfqpoint{2.654255in}{2.887617in}}%
\pgfpathlineto{\pgfqpoint{2.654255in}{2.891874in}}%
\pgfpathlineto{\pgfqpoint{2.658512in}{2.891874in}}%
\pgfpathlineto{\pgfqpoint{2.658512in}{2.887617in}}%
\pgfpathmoveto{\pgfqpoint{2.645739in}{2.891874in}}%
\pgfpathlineto{\pgfqpoint{2.645739in}{2.891874in}}%
\pgfpathlineto{\pgfqpoint{2.645739in}{2.896132in}}%
\pgfpathlineto{\pgfqpoint{2.649997in}{2.896132in}}%
\pgfpathlineto{\pgfqpoint{2.649997in}{2.891874in}}%
\pgfpathmoveto{\pgfqpoint{2.645739in}{2.896132in}}%
\pgfpathlineto{\pgfqpoint{2.645739in}{2.896132in}}%
\pgfpathlineto{\pgfqpoint{2.645739in}{2.900390in}}%
\pgfpathlineto{\pgfqpoint{2.649997in}{2.900390in}}%
\pgfpathlineto{\pgfqpoint{2.649997in}{2.896132in}}%
\pgfpathmoveto{\pgfqpoint{2.645739in}{2.900390in}}%
\pgfpathlineto{\pgfqpoint{2.645739in}{2.900390in}}%
\pgfpathlineto{\pgfqpoint{2.645739in}{2.904648in}}%
\pgfpathlineto{\pgfqpoint{2.649997in}{2.904648in}}%
\pgfpathlineto{\pgfqpoint{2.649997in}{2.900390in}}%
\pgfpathmoveto{\pgfqpoint{2.645739in}{2.904648in}}%
\pgfpathlineto{\pgfqpoint{2.645739in}{2.904648in}}%
\pgfpathlineto{\pgfqpoint{2.645739in}{2.908905in}}%
\pgfpathlineto{\pgfqpoint{2.649997in}{2.908905in}}%
\pgfpathlineto{\pgfqpoint{2.649997in}{2.904648in}}%
\pgfpathmoveto{\pgfqpoint{2.645739in}{2.908905in}}%
\pgfpathlineto{\pgfqpoint{2.645739in}{2.908905in}}%
\pgfpathlineto{\pgfqpoint{2.645739in}{2.913163in}}%
\pgfpathlineto{\pgfqpoint{2.649997in}{2.913163in}}%
\pgfpathlineto{\pgfqpoint{2.649997in}{2.908905in}}%
\pgfpathmoveto{\pgfqpoint{2.649997in}{2.891874in}}%
\pgfpathlineto{\pgfqpoint{2.649997in}{2.891874in}}%
\pgfpathlineto{\pgfqpoint{2.649997in}{2.896132in}}%
\pgfpathlineto{\pgfqpoint{2.654255in}{2.896132in}}%
\pgfpathlineto{\pgfqpoint{2.654255in}{2.891874in}}%
\pgfpathmoveto{\pgfqpoint{2.649997in}{2.896132in}}%
\pgfpathlineto{\pgfqpoint{2.649997in}{2.896132in}}%
\pgfpathlineto{\pgfqpoint{2.649997in}{2.900390in}}%
\pgfpathlineto{\pgfqpoint{2.654255in}{2.900390in}}%
\pgfpathlineto{\pgfqpoint{2.654255in}{2.896132in}}%
\pgfpathmoveto{\pgfqpoint{2.654255in}{2.891874in}}%
\pgfpathlineto{\pgfqpoint{2.654255in}{2.891874in}}%
\pgfpathlineto{\pgfqpoint{2.654255in}{2.896132in}}%
\pgfpathlineto{\pgfqpoint{2.658512in}{2.896132in}}%
\pgfpathlineto{\pgfqpoint{2.658512in}{2.891874in}}%
\pgfpathmoveto{\pgfqpoint{2.654255in}{2.896132in}}%
\pgfpathlineto{\pgfqpoint{2.654255in}{2.896132in}}%
\pgfpathlineto{\pgfqpoint{2.654255in}{2.900390in}}%
\pgfpathlineto{\pgfqpoint{2.658512in}{2.900390in}}%
\pgfpathlineto{\pgfqpoint{2.658512in}{2.896132in}}%
\pgfpathmoveto{\pgfqpoint{2.649997in}{2.900390in}}%
\pgfpathlineto{\pgfqpoint{2.649997in}{2.900390in}}%
\pgfpathlineto{\pgfqpoint{2.649997in}{2.904648in}}%
\pgfpathlineto{\pgfqpoint{2.654255in}{2.904648in}}%
\pgfpathlineto{\pgfqpoint{2.654255in}{2.900390in}}%
\pgfpathmoveto{\pgfqpoint{2.649997in}{2.904648in}}%
\pgfpathlineto{\pgfqpoint{2.649997in}{2.904648in}}%
\pgfpathlineto{\pgfqpoint{2.649997in}{2.908905in}}%
\pgfpathlineto{\pgfqpoint{2.654255in}{2.908905in}}%
\pgfpathlineto{\pgfqpoint{2.654255in}{2.904648in}}%
\pgfpathmoveto{\pgfqpoint{2.654255in}{2.900390in}}%
\pgfpathlineto{\pgfqpoint{2.654255in}{2.900390in}}%
\pgfpathlineto{\pgfqpoint{2.654255in}{2.904648in}}%
\pgfpathlineto{\pgfqpoint{2.658512in}{2.904648in}}%
\pgfpathlineto{\pgfqpoint{2.658512in}{2.900390in}}%
\pgfpathmoveto{\pgfqpoint{2.654255in}{2.904648in}}%
\pgfpathlineto{\pgfqpoint{2.654255in}{2.904648in}}%
\pgfpathlineto{\pgfqpoint{2.654255in}{2.908905in}}%
\pgfpathlineto{\pgfqpoint{2.658512in}{2.908905in}}%
\pgfpathlineto{\pgfqpoint{2.658512in}{2.904648in}}%
\pgfpathmoveto{\pgfqpoint{2.649997in}{2.908905in}}%
\pgfpathlineto{\pgfqpoint{2.649997in}{2.908905in}}%
\pgfpathlineto{\pgfqpoint{2.649997in}{2.913163in}}%
\pgfpathlineto{\pgfqpoint{2.654255in}{2.913163in}}%
\pgfpathlineto{\pgfqpoint{2.654255in}{2.908905in}}%
\pgfpathmoveto{\pgfqpoint{2.649997in}{2.913163in}}%
\pgfpathlineto{\pgfqpoint{2.649997in}{2.913163in}}%
\pgfpathlineto{\pgfqpoint{2.649997in}{2.917421in}}%
\pgfpathlineto{\pgfqpoint{2.654255in}{2.917421in}}%
\pgfpathlineto{\pgfqpoint{2.654255in}{2.913163in}}%
\pgfpathmoveto{\pgfqpoint{2.654255in}{2.908905in}}%
\pgfpathlineto{\pgfqpoint{2.654255in}{2.908905in}}%
\pgfpathlineto{\pgfqpoint{2.654255in}{2.913163in}}%
\pgfpathlineto{\pgfqpoint{2.658512in}{2.913163in}}%
\pgfpathlineto{\pgfqpoint{2.658512in}{2.908905in}}%
\pgfpathmoveto{\pgfqpoint{2.654255in}{2.913163in}}%
\pgfpathlineto{\pgfqpoint{2.654255in}{2.913163in}}%
\pgfpathlineto{\pgfqpoint{2.654255in}{2.917421in}}%
\pgfpathlineto{\pgfqpoint{2.658512in}{2.917421in}}%
\pgfpathlineto{\pgfqpoint{2.658512in}{2.913163in}}%
\pgfpathmoveto{\pgfqpoint{2.649997in}{2.917421in}}%
\pgfpathlineto{\pgfqpoint{2.649997in}{2.917421in}}%
\pgfpathlineto{\pgfqpoint{2.649997in}{2.921678in}}%
\pgfpathlineto{\pgfqpoint{2.654255in}{2.921678in}}%
\pgfpathlineto{\pgfqpoint{2.654255in}{2.917421in}}%
\pgfpathmoveto{\pgfqpoint{2.649997in}{2.921678in}}%
\pgfpathlineto{\pgfqpoint{2.649997in}{2.921678in}}%
\pgfpathlineto{\pgfqpoint{2.649997in}{2.925936in}}%
\pgfpathlineto{\pgfqpoint{2.654255in}{2.925936in}}%
\pgfpathlineto{\pgfqpoint{2.654255in}{2.921678in}}%
\pgfpathmoveto{\pgfqpoint{2.654255in}{2.917421in}}%
\pgfpathlineto{\pgfqpoint{2.654255in}{2.917421in}}%
\pgfpathlineto{\pgfqpoint{2.654255in}{2.921678in}}%
\pgfpathlineto{\pgfqpoint{2.658512in}{2.921678in}}%
\pgfpathlineto{\pgfqpoint{2.658512in}{2.917421in}}%
\pgfpathmoveto{\pgfqpoint{2.654255in}{2.921678in}}%
\pgfpathlineto{\pgfqpoint{2.654255in}{2.921678in}}%
\pgfpathlineto{\pgfqpoint{2.654255in}{2.925936in}}%
\pgfpathlineto{\pgfqpoint{2.658512in}{2.925936in}}%
\pgfpathlineto{\pgfqpoint{2.658512in}{2.921678in}}%
\pgfpathmoveto{\pgfqpoint{2.658512in}{2.913163in}}%
\pgfpathlineto{\pgfqpoint{2.658512in}{2.913163in}}%
\pgfpathlineto{\pgfqpoint{2.658512in}{2.917421in}}%
\pgfpathlineto{\pgfqpoint{2.662770in}{2.917421in}}%
\pgfpathlineto{\pgfqpoint{2.662770in}{2.913163in}}%
\pgfpathmoveto{\pgfqpoint{2.658512in}{2.917421in}}%
\pgfpathlineto{\pgfqpoint{2.658512in}{2.917421in}}%
\pgfpathlineto{\pgfqpoint{2.658512in}{2.921678in}}%
\pgfpathlineto{\pgfqpoint{2.662770in}{2.921678in}}%
\pgfpathlineto{\pgfqpoint{2.662770in}{2.917421in}}%
\pgfpathmoveto{\pgfqpoint{2.658512in}{2.921678in}}%
\pgfpathlineto{\pgfqpoint{2.658512in}{2.921678in}}%
\pgfpathlineto{\pgfqpoint{2.658512in}{2.925936in}}%
\pgfpathlineto{\pgfqpoint{2.662770in}{2.925936in}}%
\pgfpathlineto{\pgfqpoint{2.662770in}{2.921678in}}%
\pgfpathmoveto{\pgfqpoint{2.649997in}{2.925936in}}%
\pgfpathlineto{\pgfqpoint{2.649997in}{2.925936in}}%
\pgfpathlineto{\pgfqpoint{2.649997in}{2.930194in}}%
\pgfpathlineto{\pgfqpoint{2.654255in}{2.930194in}}%
\pgfpathlineto{\pgfqpoint{2.654255in}{2.925936in}}%
\pgfpathmoveto{\pgfqpoint{2.649997in}{2.930194in}}%
\pgfpathlineto{\pgfqpoint{2.649997in}{2.930194in}}%
\pgfpathlineto{\pgfqpoint{2.649997in}{2.934452in}}%
\pgfpathlineto{\pgfqpoint{2.654255in}{2.934452in}}%
\pgfpathlineto{\pgfqpoint{2.654255in}{2.930194in}}%
\pgfpathmoveto{\pgfqpoint{2.654255in}{2.925936in}}%
\pgfpathlineto{\pgfqpoint{2.654255in}{2.925936in}}%
\pgfpathlineto{\pgfqpoint{2.654255in}{2.930194in}}%
\pgfpathlineto{\pgfqpoint{2.658512in}{2.930194in}}%
\pgfpathlineto{\pgfqpoint{2.658512in}{2.925936in}}%
\pgfpathmoveto{\pgfqpoint{2.654255in}{2.930194in}}%
\pgfpathlineto{\pgfqpoint{2.654255in}{2.930194in}}%
\pgfpathlineto{\pgfqpoint{2.654255in}{2.934452in}}%
\pgfpathlineto{\pgfqpoint{2.658512in}{2.934452in}}%
\pgfpathlineto{\pgfqpoint{2.658512in}{2.930194in}}%
\pgfpathmoveto{\pgfqpoint{2.654255in}{2.934452in}}%
\pgfpathlineto{\pgfqpoint{2.654255in}{2.934452in}}%
\pgfpathlineto{\pgfqpoint{2.654255in}{2.938709in}}%
\pgfpathlineto{\pgfqpoint{2.658512in}{2.938709in}}%
\pgfpathlineto{\pgfqpoint{2.658512in}{2.934452in}}%
\pgfpathmoveto{\pgfqpoint{2.654255in}{2.938709in}}%
\pgfpathlineto{\pgfqpoint{2.654255in}{2.938709in}}%
\pgfpathlineto{\pgfqpoint{2.654255in}{2.942967in}}%
\pgfpathlineto{\pgfqpoint{2.658512in}{2.942967in}}%
\pgfpathlineto{\pgfqpoint{2.658512in}{2.938709in}}%
\pgfpathmoveto{\pgfqpoint{2.658512in}{2.925936in}}%
\pgfpathlineto{\pgfqpoint{2.658512in}{2.925936in}}%
\pgfpathlineto{\pgfqpoint{2.658512in}{2.930194in}}%
\pgfpathlineto{\pgfqpoint{2.662770in}{2.930194in}}%
\pgfpathlineto{\pgfqpoint{2.662770in}{2.925936in}}%
\pgfpathmoveto{\pgfqpoint{2.658512in}{2.930194in}}%
\pgfpathlineto{\pgfqpoint{2.658512in}{2.930194in}}%
\pgfpathlineto{\pgfqpoint{2.658512in}{2.934452in}}%
\pgfpathlineto{\pgfqpoint{2.662770in}{2.934452in}}%
\pgfpathlineto{\pgfqpoint{2.662770in}{2.930194in}}%
\pgfpathmoveto{\pgfqpoint{2.658512in}{2.934452in}}%
\pgfpathlineto{\pgfqpoint{2.658512in}{2.934452in}}%
\pgfpathlineto{\pgfqpoint{2.658512in}{2.938709in}}%
\pgfpathlineto{\pgfqpoint{2.662770in}{2.938709in}}%
\pgfpathlineto{\pgfqpoint{2.662770in}{2.934452in}}%
\pgfpathmoveto{\pgfqpoint{2.658512in}{2.938709in}}%
\pgfpathlineto{\pgfqpoint{2.658512in}{2.938709in}}%
\pgfpathlineto{\pgfqpoint{2.658512in}{2.942967in}}%
\pgfpathlineto{\pgfqpoint{2.662770in}{2.942967in}}%
\pgfpathlineto{\pgfqpoint{2.662770in}{2.938709in}}%
\pgfpathmoveto{\pgfqpoint{2.662770in}{2.934452in}}%
\pgfpathlineto{\pgfqpoint{2.662770in}{2.934452in}}%
\pgfpathlineto{\pgfqpoint{2.662770in}{2.938709in}}%
\pgfpathlineto{\pgfqpoint{2.667028in}{2.938709in}}%
\pgfpathlineto{\pgfqpoint{2.667028in}{2.934452in}}%
\pgfpathmoveto{\pgfqpoint{2.662770in}{2.938709in}}%
\pgfpathlineto{\pgfqpoint{2.662770in}{2.938709in}}%
\pgfpathlineto{\pgfqpoint{2.662770in}{2.942967in}}%
\pgfpathlineto{\pgfqpoint{2.667028in}{2.942967in}}%
\pgfpathlineto{\pgfqpoint{2.667028in}{2.938709in}}%
\pgfpathmoveto{\pgfqpoint{2.654255in}{2.942967in}}%
\pgfpathlineto{\pgfqpoint{2.654255in}{2.942967in}}%
\pgfpathlineto{\pgfqpoint{2.654255in}{2.947225in}}%
\pgfpathlineto{\pgfqpoint{2.658512in}{2.947225in}}%
\pgfpathlineto{\pgfqpoint{2.658512in}{2.942967in}}%
\pgfpathmoveto{\pgfqpoint{2.654255in}{2.947225in}}%
\pgfpathlineto{\pgfqpoint{2.654255in}{2.947225in}}%
\pgfpathlineto{\pgfqpoint{2.654255in}{2.951483in}}%
\pgfpathlineto{\pgfqpoint{2.658512in}{2.951483in}}%
\pgfpathlineto{\pgfqpoint{2.658512in}{2.947225in}}%
\pgfpathmoveto{\pgfqpoint{2.654255in}{2.951483in}}%
\pgfpathlineto{\pgfqpoint{2.654255in}{2.951483in}}%
\pgfpathlineto{\pgfqpoint{2.654255in}{2.955740in}}%
\pgfpathlineto{\pgfqpoint{2.658512in}{2.955740in}}%
\pgfpathlineto{\pgfqpoint{2.658512in}{2.951483in}}%
\pgfpathmoveto{\pgfqpoint{2.654255in}{2.955740in}}%
\pgfpathlineto{\pgfqpoint{2.654255in}{2.955740in}}%
\pgfpathlineto{\pgfqpoint{2.654255in}{2.959998in}}%
\pgfpathlineto{\pgfqpoint{2.658512in}{2.959998in}}%
\pgfpathlineto{\pgfqpoint{2.658512in}{2.955740in}}%
\pgfpathmoveto{\pgfqpoint{2.658512in}{2.942967in}}%
\pgfpathlineto{\pgfqpoint{2.658512in}{2.942967in}}%
\pgfpathlineto{\pgfqpoint{2.658512in}{2.947225in}}%
\pgfpathlineto{\pgfqpoint{2.662770in}{2.947225in}}%
\pgfpathlineto{\pgfqpoint{2.662770in}{2.942967in}}%
\pgfpathmoveto{\pgfqpoint{2.658512in}{2.947225in}}%
\pgfpathlineto{\pgfqpoint{2.658512in}{2.947225in}}%
\pgfpathlineto{\pgfqpoint{2.658512in}{2.951483in}}%
\pgfpathlineto{\pgfqpoint{2.662770in}{2.951483in}}%
\pgfpathlineto{\pgfqpoint{2.662770in}{2.947225in}}%
\pgfpathmoveto{\pgfqpoint{2.662770in}{2.942967in}}%
\pgfpathlineto{\pgfqpoint{2.662770in}{2.942967in}}%
\pgfpathlineto{\pgfqpoint{2.662770in}{2.947225in}}%
\pgfpathlineto{\pgfqpoint{2.667028in}{2.947225in}}%
\pgfpathlineto{\pgfqpoint{2.667028in}{2.942967in}}%
\pgfpathmoveto{\pgfqpoint{2.662770in}{2.947225in}}%
\pgfpathlineto{\pgfqpoint{2.662770in}{2.947225in}}%
\pgfpathlineto{\pgfqpoint{2.662770in}{2.951483in}}%
\pgfpathlineto{\pgfqpoint{2.667028in}{2.951483in}}%
\pgfpathlineto{\pgfqpoint{2.667028in}{2.947225in}}%
\pgfpathmoveto{\pgfqpoint{2.658512in}{2.951483in}}%
\pgfpathlineto{\pgfqpoint{2.658512in}{2.951483in}}%
\pgfpathlineto{\pgfqpoint{2.658512in}{2.955740in}}%
\pgfpathlineto{\pgfqpoint{2.662770in}{2.955740in}}%
\pgfpathlineto{\pgfqpoint{2.662770in}{2.951483in}}%
\pgfpathmoveto{\pgfqpoint{2.658512in}{2.955740in}}%
\pgfpathlineto{\pgfqpoint{2.658512in}{2.955740in}}%
\pgfpathlineto{\pgfqpoint{2.658512in}{2.959998in}}%
\pgfpathlineto{\pgfqpoint{2.662770in}{2.959998in}}%
\pgfpathlineto{\pgfqpoint{2.662770in}{2.955740in}}%
\pgfpathmoveto{\pgfqpoint{2.662770in}{2.951483in}}%
\pgfpathlineto{\pgfqpoint{2.662770in}{2.951483in}}%
\pgfpathlineto{\pgfqpoint{2.662770in}{2.955740in}}%
\pgfpathlineto{\pgfqpoint{2.667028in}{2.955740in}}%
\pgfpathlineto{\pgfqpoint{2.667028in}{2.951483in}}%
\pgfpathmoveto{\pgfqpoint{2.662770in}{2.955740in}}%
\pgfpathlineto{\pgfqpoint{2.662770in}{2.955740in}}%
\pgfpathlineto{\pgfqpoint{2.662770in}{2.959998in}}%
\pgfpathlineto{\pgfqpoint{2.667028in}{2.959998in}}%
\pgfpathlineto{\pgfqpoint{2.667028in}{2.955740in}}%
\pgfpathmoveto{\pgfqpoint{2.667028in}{2.955740in}}%
\pgfpathlineto{\pgfqpoint{2.667028in}{2.955740in}}%
\pgfpathlineto{\pgfqpoint{2.667028in}{2.959998in}}%
\pgfpathlineto{\pgfqpoint{2.671286in}{2.959998in}}%
\pgfpathlineto{\pgfqpoint{2.671286in}{2.955740in}}%
\pgfpathmoveto{\pgfqpoint{2.658512in}{2.959998in}}%
\pgfpathlineto{\pgfqpoint{2.658512in}{2.959998in}}%
\pgfpathlineto{\pgfqpoint{2.658512in}{2.964256in}}%
\pgfpathlineto{\pgfqpoint{2.662770in}{2.964256in}}%
\pgfpathlineto{\pgfqpoint{2.662770in}{2.959998in}}%
\pgfpathmoveto{\pgfqpoint{2.658512in}{2.964256in}}%
\pgfpathlineto{\pgfqpoint{2.658512in}{2.964256in}}%
\pgfpathlineto{\pgfqpoint{2.658512in}{2.968514in}}%
\pgfpathlineto{\pgfqpoint{2.662770in}{2.968514in}}%
\pgfpathlineto{\pgfqpoint{2.662770in}{2.964256in}}%
\pgfpathmoveto{\pgfqpoint{2.662770in}{2.959998in}}%
\pgfpathlineto{\pgfqpoint{2.662770in}{2.959998in}}%
\pgfpathlineto{\pgfqpoint{2.662770in}{2.964256in}}%
\pgfpathlineto{\pgfqpoint{2.667028in}{2.964256in}}%
\pgfpathlineto{\pgfqpoint{2.667028in}{2.959998in}}%
\pgfpathmoveto{\pgfqpoint{2.662770in}{2.964256in}}%
\pgfpathlineto{\pgfqpoint{2.662770in}{2.964256in}}%
\pgfpathlineto{\pgfqpoint{2.662770in}{2.968514in}}%
\pgfpathlineto{\pgfqpoint{2.667028in}{2.968514in}}%
\pgfpathlineto{\pgfqpoint{2.667028in}{2.964256in}}%
\pgfpathmoveto{\pgfqpoint{2.658512in}{2.968514in}}%
\pgfpathlineto{\pgfqpoint{2.658512in}{2.968514in}}%
\pgfpathlineto{\pgfqpoint{2.658512in}{2.972772in}}%
\pgfpathlineto{\pgfqpoint{2.662770in}{2.972772in}}%
\pgfpathlineto{\pgfqpoint{2.662770in}{2.968514in}}%
\pgfpathmoveto{\pgfqpoint{2.658512in}{2.972772in}}%
\pgfpathlineto{\pgfqpoint{2.658512in}{2.972772in}}%
\pgfpathlineto{\pgfqpoint{2.658512in}{2.977030in}}%
\pgfpathlineto{\pgfqpoint{2.662770in}{2.977030in}}%
\pgfpathlineto{\pgfqpoint{2.662770in}{2.972772in}}%
\pgfpathmoveto{\pgfqpoint{2.662770in}{2.968514in}}%
\pgfpathlineto{\pgfqpoint{2.662770in}{2.968514in}}%
\pgfpathlineto{\pgfqpoint{2.662770in}{2.972772in}}%
\pgfpathlineto{\pgfqpoint{2.667028in}{2.972772in}}%
\pgfpathlineto{\pgfqpoint{2.667028in}{2.968514in}}%
\pgfpathmoveto{\pgfqpoint{2.662770in}{2.972772in}}%
\pgfpathlineto{\pgfqpoint{2.662770in}{2.972772in}}%
\pgfpathlineto{\pgfqpoint{2.662770in}{2.977030in}}%
\pgfpathlineto{\pgfqpoint{2.667028in}{2.977030in}}%
\pgfpathlineto{\pgfqpoint{2.667028in}{2.972772in}}%
\pgfpathmoveto{\pgfqpoint{2.658512in}{2.977030in}}%
\pgfpathlineto{\pgfqpoint{2.658512in}{2.977030in}}%
\pgfpathlineto{\pgfqpoint{2.658512in}{2.981287in}}%
\pgfpathlineto{\pgfqpoint{2.662770in}{2.981287in}}%
\pgfpathlineto{\pgfqpoint{2.662770in}{2.977030in}}%
\pgfpathmoveto{\pgfqpoint{2.662770in}{2.977030in}}%
\pgfpathlineto{\pgfqpoint{2.662770in}{2.977030in}}%
\pgfpathlineto{\pgfqpoint{2.662770in}{2.981287in}}%
\pgfpathlineto{\pgfqpoint{2.667028in}{2.981287in}}%
\pgfpathlineto{\pgfqpoint{2.667028in}{2.977030in}}%
\pgfpathmoveto{\pgfqpoint{2.662770in}{2.981287in}}%
\pgfpathlineto{\pgfqpoint{2.662770in}{2.981287in}}%
\pgfpathlineto{\pgfqpoint{2.662770in}{2.985545in}}%
\pgfpathlineto{\pgfqpoint{2.667028in}{2.985545in}}%
\pgfpathlineto{\pgfqpoint{2.667028in}{2.981287in}}%
\pgfpathmoveto{\pgfqpoint{2.662770in}{2.985545in}}%
\pgfpathlineto{\pgfqpoint{2.662770in}{2.985545in}}%
\pgfpathlineto{\pgfqpoint{2.662770in}{2.989803in}}%
\pgfpathlineto{\pgfqpoint{2.667028in}{2.989803in}}%
\pgfpathlineto{\pgfqpoint{2.667028in}{2.985545in}}%
\pgfpathmoveto{\pgfqpoint{2.662770in}{2.989803in}}%
\pgfpathlineto{\pgfqpoint{2.662770in}{2.989803in}}%
\pgfpathlineto{\pgfqpoint{2.662770in}{2.994061in}}%
\pgfpathlineto{\pgfqpoint{2.667028in}{2.994061in}}%
\pgfpathlineto{\pgfqpoint{2.667028in}{2.989803in}}%
\pgfpathmoveto{\pgfqpoint{2.667028in}{2.959998in}}%
\pgfpathlineto{\pgfqpoint{2.667028in}{2.959998in}}%
\pgfpathlineto{\pgfqpoint{2.667028in}{2.964256in}}%
\pgfpathlineto{\pgfqpoint{2.671286in}{2.964256in}}%
\pgfpathlineto{\pgfqpoint{2.671286in}{2.959998in}}%
\pgfpathmoveto{\pgfqpoint{2.667028in}{2.964256in}}%
\pgfpathlineto{\pgfqpoint{2.667028in}{2.964256in}}%
\pgfpathlineto{\pgfqpoint{2.667028in}{2.968514in}}%
\pgfpathlineto{\pgfqpoint{2.671286in}{2.968514in}}%
\pgfpathlineto{\pgfqpoint{2.671286in}{2.964256in}}%
\pgfpathmoveto{\pgfqpoint{2.667028in}{2.968514in}}%
\pgfpathlineto{\pgfqpoint{2.667028in}{2.968514in}}%
\pgfpathlineto{\pgfqpoint{2.667028in}{2.972772in}}%
\pgfpathlineto{\pgfqpoint{2.671286in}{2.972772in}}%
\pgfpathlineto{\pgfqpoint{2.671286in}{2.968514in}}%
\pgfpathmoveto{\pgfqpoint{2.667028in}{2.972772in}}%
\pgfpathlineto{\pgfqpoint{2.667028in}{2.972772in}}%
\pgfpathlineto{\pgfqpoint{2.667028in}{2.977030in}}%
\pgfpathlineto{\pgfqpoint{2.671286in}{2.977030in}}%
\pgfpathlineto{\pgfqpoint{2.671286in}{2.972772in}}%
\pgfpathmoveto{\pgfqpoint{2.667028in}{2.977030in}}%
\pgfpathlineto{\pgfqpoint{2.667028in}{2.977030in}}%
\pgfpathlineto{\pgfqpoint{2.667028in}{2.981287in}}%
\pgfpathlineto{\pgfqpoint{2.671286in}{2.981287in}}%
\pgfpathlineto{\pgfqpoint{2.671286in}{2.977030in}}%
\pgfpathmoveto{\pgfqpoint{2.667028in}{2.981287in}}%
\pgfpathlineto{\pgfqpoint{2.667028in}{2.981287in}}%
\pgfpathlineto{\pgfqpoint{2.667028in}{2.985545in}}%
\pgfpathlineto{\pgfqpoint{2.671286in}{2.985545in}}%
\pgfpathlineto{\pgfqpoint{2.671286in}{2.981287in}}%
\pgfpathmoveto{\pgfqpoint{2.671286in}{2.981287in}}%
\pgfpathlineto{\pgfqpoint{2.671286in}{2.981287in}}%
\pgfpathlineto{\pgfqpoint{2.671286in}{2.985545in}}%
\pgfpathlineto{\pgfqpoint{2.675543in}{2.985545in}}%
\pgfpathlineto{\pgfqpoint{2.675543in}{2.981287in}}%
\pgfpathmoveto{\pgfqpoint{2.667028in}{2.985545in}}%
\pgfpathlineto{\pgfqpoint{2.667028in}{2.985545in}}%
\pgfpathlineto{\pgfqpoint{2.667028in}{2.989803in}}%
\pgfpathlineto{\pgfqpoint{2.671286in}{2.989803in}}%
\pgfpathlineto{\pgfqpoint{2.671286in}{2.985545in}}%
\pgfpathmoveto{\pgfqpoint{2.667028in}{2.989803in}}%
\pgfpathlineto{\pgfqpoint{2.667028in}{2.989803in}}%
\pgfpathlineto{\pgfqpoint{2.667028in}{2.994061in}}%
\pgfpathlineto{\pgfqpoint{2.671286in}{2.994061in}}%
\pgfpathlineto{\pgfqpoint{2.671286in}{2.989803in}}%
\pgfpathmoveto{\pgfqpoint{2.671286in}{2.985545in}}%
\pgfpathlineto{\pgfqpoint{2.671286in}{2.985545in}}%
\pgfpathlineto{\pgfqpoint{2.671286in}{2.989803in}}%
\pgfpathlineto{\pgfqpoint{2.675543in}{2.989803in}}%
\pgfpathlineto{\pgfqpoint{2.675543in}{2.985545in}}%
\pgfpathmoveto{\pgfqpoint{2.671286in}{2.989803in}}%
\pgfpathlineto{\pgfqpoint{2.671286in}{2.989803in}}%
\pgfpathlineto{\pgfqpoint{2.671286in}{2.994061in}}%
\pgfpathlineto{\pgfqpoint{2.675543in}{2.994061in}}%
\pgfpathlineto{\pgfqpoint{2.675543in}{2.989803in}}%
\pgfpathmoveto{\pgfqpoint{2.662770in}{2.994061in}}%
\pgfpathlineto{\pgfqpoint{2.662770in}{2.994061in}}%
\pgfpathlineto{\pgfqpoint{2.662770in}{2.998319in}}%
\pgfpathlineto{\pgfqpoint{2.667028in}{2.998319in}}%
\pgfpathlineto{\pgfqpoint{2.667028in}{2.994061in}}%
\pgfpathmoveto{\pgfqpoint{2.662770in}{2.998319in}}%
\pgfpathlineto{\pgfqpoint{2.662770in}{2.998319in}}%
\pgfpathlineto{\pgfqpoint{2.662770in}{3.002577in}}%
\pgfpathlineto{\pgfqpoint{2.667028in}{3.002577in}}%
\pgfpathlineto{\pgfqpoint{2.667028in}{2.998319in}}%
\pgfpathmoveto{\pgfqpoint{2.667028in}{2.994061in}}%
\pgfpathlineto{\pgfqpoint{2.667028in}{2.994061in}}%
\pgfpathlineto{\pgfqpoint{2.667028in}{2.998319in}}%
\pgfpathlineto{\pgfqpoint{2.671286in}{2.998319in}}%
\pgfpathlineto{\pgfqpoint{2.671286in}{2.994061in}}%
\pgfpathmoveto{\pgfqpoint{2.667028in}{2.998319in}}%
\pgfpathlineto{\pgfqpoint{2.667028in}{2.998319in}}%
\pgfpathlineto{\pgfqpoint{2.667028in}{3.002577in}}%
\pgfpathlineto{\pgfqpoint{2.671286in}{3.002577in}}%
\pgfpathlineto{\pgfqpoint{2.671286in}{2.998319in}}%
\pgfpathmoveto{\pgfqpoint{2.671286in}{2.994061in}}%
\pgfpathlineto{\pgfqpoint{2.671286in}{2.994061in}}%
\pgfpathlineto{\pgfqpoint{2.671286in}{2.998319in}}%
\pgfpathlineto{\pgfqpoint{2.675543in}{2.998319in}}%
\pgfpathlineto{\pgfqpoint{2.675543in}{2.994061in}}%
\pgfpathmoveto{\pgfqpoint{2.671286in}{2.998319in}}%
\pgfpathlineto{\pgfqpoint{2.671286in}{2.998319in}}%
\pgfpathlineto{\pgfqpoint{2.671286in}{3.002577in}}%
\pgfpathlineto{\pgfqpoint{2.675543in}{3.002577in}}%
\pgfpathlineto{\pgfqpoint{2.675543in}{2.998319in}}%
\pgfpathmoveto{\pgfqpoint{2.667028in}{3.002577in}}%
\pgfpathlineto{\pgfqpoint{2.667028in}{3.002577in}}%
\pgfpathlineto{\pgfqpoint{2.667028in}{3.006835in}}%
\pgfpathlineto{\pgfqpoint{2.671286in}{3.006835in}}%
\pgfpathlineto{\pgfqpoint{2.671286in}{3.002577in}}%
\pgfpathmoveto{\pgfqpoint{2.667028in}{3.006835in}}%
\pgfpathlineto{\pgfqpoint{2.667028in}{3.006835in}}%
\pgfpathlineto{\pgfqpoint{2.667028in}{3.011093in}}%
\pgfpathlineto{\pgfqpoint{2.671286in}{3.011093in}}%
\pgfpathlineto{\pgfqpoint{2.671286in}{3.006835in}}%
\pgfpathmoveto{\pgfqpoint{2.671286in}{3.002577in}}%
\pgfpathlineto{\pgfqpoint{2.671286in}{3.002577in}}%
\pgfpathlineto{\pgfqpoint{2.671286in}{3.006835in}}%
\pgfpathlineto{\pgfqpoint{2.675543in}{3.006835in}}%
\pgfpathlineto{\pgfqpoint{2.675543in}{3.002577in}}%
\pgfpathmoveto{\pgfqpoint{2.671286in}{3.006835in}}%
\pgfpathlineto{\pgfqpoint{2.671286in}{3.006835in}}%
\pgfpathlineto{\pgfqpoint{2.671286in}{3.011093in}}%
\pgfpathlineto{\pgfqpoint{2.675543in}{3.011093in}}%
\pgfpathlineto{\pgfqpoint{2.675543in}{3.006835in}}%
\pgfpathmoveto{\pgfqpoint{2.675543in}{3.002577in}}%
\pgfpathlineto{\pgfqpoint{2.675543in}{3.002577in}}%
\pgfpathlineto{\pgfqpoint{2.675543in}{3.006835in}}%
\pgfpathlineto{\pgfqpoint{2.679801in}{3.006835in}}%
\pgfpathlineto{\pgfqpoint{2.679801in}{3.002577in}}%
\pgfpathmoveto{\pgfqpoint{2.675543in}{3.006835in}}%
\pgfpathlineto{\pgfqpoint{2.675543in}{3.006835in}}%
\pgfpathlineto{\pgfqpoint{2.675543in}{3.011093in}}%
\pgfpathlineto{\pgfqpoint{2.679801in}{3.011093in}}%
\pgfpathlineto{\pgfqpoint{2.679801in}{3.006835in}}%
\pgfpathmoveto{\pgfqpoint{2.667028in}{3.011093in}}%
\pgfpathlineto{\pgfqpoint{2.667028in}{3.011093in}}%
\pgfpathlineto{\pgfqpoint{2.667028in}{3.015351in}}%
\pgfpathlineto{\pgfqpoint{2.671286in}{3.015351in}}%
\pgfpathlineto{\pgfqpoint{2.671286in}{3.011093in}}%
\pgfpathmoveto{\pgfqpoint{2.667028in}{3.015351in}}%
\pgfpathlineto{\pgfqpoint{2.667028in}{3.015351in}}%
\pgfpathlineto{\pgfqpoint{2.667028in}{3.019609in}}%
\pgfpathlineto{\pgfqpoint{2.671286in}{3.019609in}}%
\pgfpathlineto{\pgfqpoint{2.671286in}{3.015351in}}%
\pgfpathmoveto{\pgfqpoint{2.671286in}{3.011093in}}%
\pgfpathlineto{\pgfqpoint{2.671286in}{3.011093in}}%
\pgfpathlineto{\pgfqpoint{2.671286in}{3.015351in}}%
\pgfpathlineto{\pgfqpoint{2.675543in}{3.015351in}}%
\pgfpathlineto{\pgfqpoint{2.675543in}{3.011093in}}%
\pgfpathmoveto{\pgfqpoint{2.671286in}{3.015351in}}%
\pgfpathlineto{\pgfqpoint{2.671286in}{3.015351in}}%
\pgfpathlineto{\pgfqpoint{2.671286in}{3.019609in}}%
\pgfpathlineto{\pgfqpoint{2.675543in}{3.019609in}}%
\pgfpathlineto{\pgfqpoint{2.675543in}{3.015351in}}%
\pgfpathmoveto{\pgfqpoint{2.667028in}{3.019609in}}%
\pgfpathlineto{\pgfqpoint{2.667028in}{3.019609in}}%
\pgfpathlineto{\pgfqpoint{2.667028in}{3.023867in}}%
\pgfpathlineto{\pgfqpoint{2.671286in}{3.023867in}}%
\pgfpathlineto{\pgfqpoint{2.671286in}{3.019609in}}%
\pgfpathmoveto{\pgfqpoint{2.667028in}{3.023867in}}%
\pgfpathlineto{\pgfqpoint{2.667028in}{3.023867in}}%
\pgfpathlineto{\pgfqpoint{2.667028in}{3.028124in}}%
\pgfpathlineto{\pgfqpoint{2.671286in}{3.028124in}}%
\pgfpathlineto{\pgfqpoint{2.671286in}{3.023867in}}%
\pgfpathmoveto{\pgfqpoint{2.671286in}{3.019609in}}%
\pgfpathlineto{\pgfqpoint{2.671286in}{3.019609in}}%
\pgfpathlineto{\pgfqpoint{2.671286in}{3.023867in}}%
\pgfpathlineto{\pgfqpoint{2.675543in}{3.023867in}}%
\pgfpathlineto{\pgfqpoint{2.675543in}{3.019609in}}%
\pgfpathmoveto{\pgfqpoint{2.671286in}{3.023867in}}%
\pgfpathlineto{\pgfqpoint{2.671286in}{3.023867in}}%
\pgfpathlineto{\pgfqpoint{2.671286in}{3.028124in}}%
\pgfpathlineto{\pgfqpoint{2.675543in}{3.028124in}}%
\pgfpathlineto{\pgfqpoint{2.675543in}{3.023867in}}%
\pgfpathmoveto{\pgfqpoint{2.675543in}{3.011093in}}%
\pgfpathlineto{\pgfqpoint{2.675543in}{3.011093in}}%
\pgfpathlineto{\pgfqpoint{2.675543in}{3.015351in}}%
\pgfpathlineto{\pgfqpoint{2.679801in}{3.015351in}}%
\pgfpathlineto{\pgfqpoint{2.679801in}{3.011093in}}%
\pgfpathmoveto{\pgfqpoint{2.675543in}{3.015351in}}%
\pgfpathlineto{\pgfqpoint{2.675543in}{3.015351in}}%
\pgfpathlineto{\pgfqpoint{2.675543in}{3.019609in}}%
\pgfpathlineto{\pgfqpoint{2.679801in}{3.019609in}}%
\pgfpathlineto{\pgfqpoint{2.679801in}{3.015351in}}%
\pgfpathmoveto{\pgfqpoint{2.675543in}{3.019609in}}%
\pgfpathlineto{\pgfqpoint{2.675543in}{3.019609in}}%
\pgfpathlineto{\pgfqpoint{2.675543in}{3.023867in}}%
\pgfpathlineto{\pgfqpoint{2.679801in}{3.023867in}}%
\pgfpathlineto{\pgfqpoint{2.679801in}{3.019609in}}%
\pgfpathmoveto{\pgfqpoint{2.675543in}{3.023867in}}%
\pgfpathlineto{\pgfqpoint{2.675543in}{3.023867in}}%
\pgfpathlineto{\pgfqpoint{2.675543in}{3.028124in}}%
\pgfpathlineto{\pgfqpoint{2.679801in}{3.028124in}}%
\pgfpathlineto{\pgfqpoint{2.679801in}{3.023867in}}%
\pgfpathmoveto{\pgfqpoint{2.679801in}{3.023867in}}%
\pgfpathlineto{\pgfqpoint{2.679801in}{3.023867in}}%
\pgfpathlineto{\pgfqpoint{2.679801in}{3.028124in}}%
\pgfpathlineto{\pgfqpoint{2.684059in}{3.028124in}}%
\pgfpathlineto{\pgfqpoint{2.684059in}{3.023867in}}%
\pgfpathmoveto{\pgfqpoint{2.671286in}{3.028124in}}%
\pgfpathlineto{\pgfqpoint{2.671286in}{3.028124in}}%
\pgfpathlineto{\pgfqpoint{2.671286in}{3.032382in}}%
\pgfpathlineto{\pgfqpoint{2.675543in}{3.032382in}}%
\pgfpathlineto{\pgfqpoint{2.675543in}{3.028124in}}%
\pgfpathmoveto{\pgfqpoint{2.671286in}{3.032382in}}%
\pgfpathlineto{\pgfqpoint{2.671286in}{3.032382in}}%
\pgfpathlineto{\pgfqpoint{2.671286in}{3.036640in}}%
\pgfpathlineto{\pgfqpoint{2.675543in}{3.036640in}}%
\pgfpathlineto{\pgfqpoint{2.675543in}{3.032382in}}%
\pgfpathmoveto{\pgfqpoint{2.671286in}{3.036640in}}%
\pgfpathlineto{\pgfqpoint{2.671286in}{3.036640in}}%
\pgfpathlineto{\pgfqpoint{2.671286in}{3.040898in}}%
\pgfpathlineto{\pgfqpoint{2.675543in}{3.040898in}}%
\pgfpathlineto{\pgfqpoint{2.675543in}{3.036640in}}%
\pgfpathmoveto{\pgfqpoint{2.671286in}{3.040898in}}%
\pgfpathlineto{\pgfqpoint{2.671286in}{3.040898in}}%
\pgfpathlineto{\pgfqpoint{2.671286in}{3.045156in}}%
\pgfpathlineto{\pgfqpoint{2.675543in}{3.045156in}}%
\pgfpathlineto{\pgfqpoint{2.675543in}{3.040898in}}%
\pgfpathmoveto{\pgfqpoint{2.675543in}{3.028124in}}%
\pgfpathlineto{\pgfqpoint{2.675543in}{3.028124in}}%
\pgfpathlineto{\pgfqpoint{2.675543in}{3.032382in}}%
\pgfpathlineto{\pgfqpoint{2.679801in}{3.032382in}}%
\pgfpathlineto{\pgfqpoint{2.679801in}{3.028124in}}%
\pgfpathmoveto{\pgfqpoint{2.675543in}{3.032382in}}%
\pgfpathlineto{\pgfqpoint{2.675543in}{3.032382in}}%
\pgfpathlineto{\pgfqpoint{2.675543in}{3.036640in}}%
\pgfpathlineto{\pgfqpoint{2.679801in}{3.036640in}}%
\pgfpathlineto{\pgfqpoint{2.679801in}{3.032382in}}%
\pgfpathmoveto{\pgfqpoint{2.679801in}{3.028124in}}%
\pgfpathlineto{\pgfqpoint{2.679801in}{3.028124in}}%
\pgfpathlineto{\pgfqpoint{2.679801in}{3.032382in}}%
\pgfpathlineto{\pgfqpoint{2.684059in}{3.032382in}}%
\pgfpathlineto{\pgfqpoint{2.684059in}{3.028124in}}%
\pgfpathmoveto{\pgfqpoint{2.679801in}{3.032382in}}%
\pgfpathlineto{\pgfqpoint{2.679801in}{3.032382in}}%
\pgfpathlineto{\pgfqpoint{2.679801in}{3.036640in}}%
\pgfpathlineto{\pgfqpoint{2.684059in}{3.036640in}}%
\pgfpathlineto{\pgfqpoint{2.684059in}{3.032382in}}%
\pgfpathmoveto{\pgfqpoint{2.675543in}{3.036640in}}%
\pgfpathlineto{\pgfqpoint{2.675543in}{3.036640in}}%
\pgfpathlineto{\pgfqpoint{2.675543in}{3.040898in}}%
\pgfpathlineto{\pgfqpoint{2.679801in}{3.040898in}}%
\pgfpathlineto{\pgfqpoint{2.679801in}{3.036640in}}%
\pgfpathmoveto{\pgfqpoint{2.675543in}{3.040898in}}%
\pgfpathlineto{\pgfqpoint{2.675543in}{3.040898in}}%
\pgfpathlineto{\pgfqpoint{2.675543in}{3.045156in}}%
\pgfpathlineto{\pgfqpoint{2.679801in}{3.045156in}}%
\pgfpathlineto{\pgfqpoint{2.679801in}{3.040898in}}%
\pgfpathmoveto{\pgfqpoint{2.679801in}{3.036640in}}%
\pgfpathlineto{\pgfqpoint{2.679801in}{3.036640in}}%
\pgfpathlineto{\pgfqpoint{2.679801in}{3.040898in}}%
\pgfpathlineto{\pgfqpoint{2.684059in}{3.040898in}}%
\pgfpathlineto{\pgfqpoint{2.684059in}{3.036640in}}%
\pgfpathmoveto{\pgfqpoint{2.679801in}{3.040898in}}%
\pgfpathlineto{\pgfqpoint{2.679801in}{3.040898in}}%
\pgfpathlineto{\pgfqpoint{2.679801in}{3.045156in}}%
\pgfpathlineto{\pgfqpoint{2.684059in}{3.045156in}}%
\pgfpathlineto{\pgfqpoint{2.684059in}{3.040898in}}%
\pgfpathmoveto{\pgfqpoint{2.671286in}{3.045156in}}%
\pgfpathlineto{\pgfqpoint{2.671286in}{3.045156in}}%
\pgfpathlineto{\pgfqpoint{2.671286in}{3.049414in}}%
\pgfpathlineto{\pgfqpoint{2.675543in}{3.049414in}}%
\pgfpathlineto{\pgfqpoint{2.675543in}{3.045156in}}%
\pgfpathmoveto{\pgfqpoint{2.675543in}{3.045156in}}%
\pgfpathlineto{\pgfqpoint{2.675543in}{3.045156in}}%
\pgfpathlineto{\pgfqpoint{2.675543in}{3.049414in}}%
\pgfpathlineto{\pgfqpoint{2.679801in}{3.049414in}}%
\pgfpathlineto{\pgfqpoint{2.679801in}{3.045156in}}%
\pgfpathmoveto{\pgfqpoint{2.675543in}{3.049414in}}%
\pgfpathlineto{\pgfqpoint{2.675543in}{3.049414in}}%
\pgfpathlineto{\pgfqpoint{2.675543in}{3.053672in}}%
\pgfpathlineto{\pgfqpoint{2.679801in}{3.053672in}}%
\pgfpathlineto{\pgfqpoint{2.679801in}{3.049414in}}%
\pgfpathmoveto{\pgfqpoint{2.679801in}{3.045156in}}%
\pgfpathlineto{\pgfqpoint{2.679801in}{3.045156in}}%
\pgfpathlineto{\pgfqpoint{2.679801in}{3.049414in}}%
\pgfpathlineto{\pgfqpoint{2.684059in}{3.049414in}}%
\pgfpathlineto{\pgfqpoint{2.684059in}{3.045156in}}%
\pgfpathmoveto{\pgfqpoint{2.679801in}{3.049414in}}%
\pgfpathlineto{\pgfqpoint{2.679801in}{3.049414in}}%
\pgfpathlineto{\pgfqpoint{2.679801in}{3.053672in}}%
\pgfpathlineto{\pgfqpoint{2.684059in}{3.053672in}}%
\pgfpathlineto{\pgfqpoint{2.684059in}{3.049414in}}%
\pgfpathmoveto{\pgfqpoint{2.675543in}{3.053672in}}%
\pgfpathlineto{\pgfqpoint{2.675543in}{3.053672in}}%
\pgfpathlineto{\pgfqpoint{2.675543in}{3.057930in}}%
\pgfpathlineto{\pgfqpoint{2.679801in}{3.057930in}}%
\pgfpathlineto{\pgfqpoint{2.679801in}{3.053672in}}%
\pgfpathmoveto{\pgfqpoint{2.675543in}{3.057930in}}%
\pgfpathlineto{\pgfqpoint{2.675543in}{3.057930in}}%
\pgfpathlineto{\pgfqpoint{2.675543in}{3.062188in}}%
\pgfpathlineto{\pgfqpoint{2.679801in}{3.062188in}}%
\pgfpathlineto{\pgfqpoint{2.679801in}{3.057930in}}%
\pgfpathmoveto{\pgfqpoint{2.679801in}{3.053672in}}%
\pgfpathlineto{\pgfqpoint{2.679801in}{3.053672in}}%
\pgfpathlineto{\pgfqpoint{2.679801in}{3.057930in}}%
\pgfpathlineto{\pgfqpoint{2.684059in}{3.057930in}}%
\pgfpathlineto{\pgfqpoint{2.684059in}{3.053672in}}%
\pgfpathmoveto{\pgfqpoint{2.679801in}{3.057930in}}%
\pgfpathlineto{\pgfqpoint{2.679801in}{3.057930in}}%
\pgfpathlineto{\pgfqpoint{2.679801in}{3.062188in}}%
\pgfpathlineto{\pgfqpoint{2.684059in}{3.062188in}}%
\pgfpathlineto{\pgfqpoint{2.684059in}{3.057930in}}%
\pgfpathmoveto{\pgfqpoint{2.675543in}{3.062188in}}%
\pgfpathlineto{\pgfqpoint{2.675543in}{3.062188in}}%
\pgfpathlineto{\pgfqpoint{2.675543in}{3.066446in}}%
\pgfpathlineto{\pgfqpoint{2.679801in}{3.066446in}}%
\pgfpathlineto{\pgfqpoint{2.679801in}{3.062188in}}%
\pgfpathmoveto{\pgfqpoint{2.675543in}{3.066446in}}%
\pgfpathlineto{\pgfqpoint{2.675543in}{3.066446in}}%
\pgfpathlineto{\pgfqpoint{2.675543in}{3.070703in}}%
\pgfpathlineto{\pgfqpoint{2.679801in}{3.070703in}}%
\pgfpathlineto{\pgfqpoint{2.679801in}{3.066446in}}%
\pgfpathmoveto{\pgfqpoint{2.679801in}{3.062188in}}%
\pgfpathlineto{\pgfqpoint{2.679801in}{3.062188in}}%
\pgfpathlineto{\pgfqpoint{2.679801in}{3.066446in}}%
\pgfpathlineto{\pgfqpoint{2.684059in}{3.066446in}}%
\pgfpathlineto{\pgfqpoint{2.684059in}{3.062188in}}%
\pgfpathmoveto{\pgfqpoint{2.679801in}{3.066446in}}%
\pgfpathlineto{\pgfqpoint{2.679801in}{3.066446in}}%
\pgfpathlineto{\pgfqpoint{2.679801in}{3.070703in}}%
\pgfpathlineto{\pgfqpoint{2.684059in}{3.070703in}}%
\pgfpathlineto{\pgfqpoint{2.684059in}{3.066446in}}%
\pgfpathmoveto{\pgfqpoint{2.679801in}{3.070703in}}%
\pgfpathlineto{\pgfqpoint{2.679801in}{3.070703in}}%
\pgfpathlineto{\pgfqpoint{2.679801in}{3.074961in}}%
\pgfpathlineto{\pgfqpoint{2.684059in}{3.074961in}}%
\pgfpathlineto{\pgfqpoint{2.684059in}{3.070703in}}%
\pgfpathmoveto{\pgfqpoint{2.679801in}{3.074961in}}%
\pgfpathlineto{\pgfqpoint{2.679801in}{3.074961in}}%
\pgfpathlineto{\pgfqpoint{2.679801in}{3.079219in}}%
\pgfpathlineto{\pgfqpoint{2.684059in}{3.079219in}}%
\pgfpathlineto{\pgfqpoint{2.684059in}{3.074961in}}%
\pgfpathmoveto{\pgfqpoint{2.679801in}{3.079219in}}%
\pgfpathlineto{\pgfqpoint{2.679801in}{3.079219in}}%
\pgfpathlineto{\pgfqpoint{2.679801in}{3.083477in}}%
\pgfpathlineto{\pgfqpoint{2.684059in}{3.083477in}}%
\pgfpathlineto{\pgfqpoint{2.684059in}{3.079219in}}%
\pgfpathmoveto{\pgfqpoint{2.679801in}{3.083477in}}%
\pgfpathlineto{\pgfqpoint{2.679801in}{3.083477in}}%
\pgfpathlineto{\pgfqpoint{2.679801in}{3.087735in}}%
\pgfpathlineto{\pgfqpoint{2.684059in}{3.087735in}}%
\pgfpathlineto{\pgfqpoint{2.684059in}{3.083477in}}%
\pgfpathmoveto{\pgfqpoint{2.679801in}{3.087735in}}%
\pgfpathlineto{\pgfqpoint{2.679801in}{3.087735in}}%
\pgfpathlineto{\pgfqpoint{2.679801in}{3.091993in}}%
\pgfpathlineto{\pgfqpoint{2.684059in}{3.091993in}}%
\pgfpathlineto{\pgfqpoint{2.684059in}{3.087735in}}%
\pgfpathmoveto{\pgfqpoint{2.684059in}{3.045156in}}%
\pgfpathlineto{\pgfqpoint{2.684059in}{3.045156in}}%
\pgfpathlineto{\pgfqpoint{2.684059in}{3.049414in}}%
\pgfpathlineto{\pgfqpoint{2.688317in}{3.049414in}}%
\pgfpathlineto{\pgfqpoint{2.688317in}{3.045156in}}%
\pgfpathmoveto{\pgfqpoint{2.684059in}{3.049414in}}%
\pgfpathlineto{\pgfqpoint{2.684059in}{3.049414in}}%
\pgfpathlineto{\pgfqpoint{2.684059in}{3.053672in}}%
\pgfpathlineto{\pgfqpoint{2.688317in}{3.053672in}}%
\pgfpathlineto{\pgfqpoint{2.688317in}{3.049414in}}%
\pgfpathmoveto{\pgfqpoint{2.684059in}{3.053672in}}%
\pgfpathlineto{\pgfqpoint{2.684059in}{3.053672in}}%
\pgfpathlineto{\pgfqpoint{2.684059in}{3.057930in}}%
\pgfpathlineto{\pgfqpoint{2.688317in}{3.057930in}}%
\pgfpathlineto{\pgfqpoint{2.688317in}{3.053672in}}%
\pgfpathmoveto{\pgfqpoint{2.684059in}{3.057930in}}%
\pgfpathlineto{\pgfqpoint{2.684059in}{3.057930in}}%
\pgfpathlineto{\pgfqpoint{2.684059in}{3.062188in}}%
\pgfpathlineto{\pgfqpoint{2.688317in}{3.062188in}}%
\pgfpathlineto{\pgfqpoint{2.688317in}{3.057930in}}%
\pgfpathmoveto{\pgfqpoint{2.684059in}{3.062188in}}%
\pgfpathlineto{\pgfqpoint{2.684059in}{3.062188in}}%
\pgfpathlineto{\pgfqpoint{2.684059in}{3.066446in}}%
\pgfpathlineto{\pgfqpoint{2.688317in}{3.066446in}}%
\pgfpathlineto{\pgfqpoint{2.688317in}{3.062188in}}%
\pgfpathmoveto{\pgfqpoint{2.684059in}{3.066446in}}%
\pgfpathlineto{\pgfqpoint{2.684059in}{3.066446in}}%
\pgfpathlineto{\pgfqpoint{2.684059in}{3.070703in}}%
\pgfpathlineto{\pgfqpoint{2.688317in}{3.070703in}}%
\pgfpathlineto{\pgfqpoint{2.688317in}{3.066446in}}%
\pgfpathmoveto{\pgfqpoint{2.688317in}{3.066446in}}%
\pgfpathlineto{\pgfqpoint{2.688317in}{3.066446in}}%
\pgfpathlineto{\pgfqpoint{2.688317in}{3.070703in}}%
\pgfpathlineto{\pgfqpoint{2.692575in}{3.070703in}}%
\pgfpathlineto{\pgfqpoint{2.692575in}{3.066446in}}%
\pgfpathmoveto{\pgfqpoint{2.684059in}{3.070703in}}%
\pgfpathlineto{\pgfqpoint{2.684059in}{3.070703in}}%
\pgfpathlineto{\pgfqpoint{2.684059in}{3.074961in}}%
\pgfpathlineto{\pgfqpoint{2.688317in}{3.074961in}}%
\pgfpathlineto{\pgfqpoint{2.688317in}{3.070703in}}%
\pgfpathmoveto{\pgfqpoint{2.684059in}{3.074961in}}%
\pgfpathlineto{\pgfqpoint{2.684059in}{3.074961in}}%
\pgfpathlineto{\pgfqpoint{2.684059in}{3.079219in}}%
\pgfpathlineto{\pgfqpoint{2.688317in}{3.079219in}}%
\pgfpathlineto{\pgfqpoint{2.688317in}{3.074961in}}%
\pgfpathmoveto{\pgfqpoint{2.688317in}{3.070703in}}%
\pgfpathlineto{\pgfqpoint{2.688317in}{3.070703in}}%
\pgfpathlineto{\pgfqpoint{2.688317in}{3.074961in}}%
\pgfpathlineto{\pgfqpoint{2.692575in}{3.074961in}}%
\pgfpathlineto{\pgfqpoint{2.692575in}{3.070703in}}%
\pgfpathmoveto{\pgfqpoint{2.688317in}{3.074961in}}%
\pgfpathlineto{\pgfqpoint{2.688317in}{3.074961in}}%
\pgfpathlineto{\pgfqpoint{2.688317in}{3.079219in}}%
\pgfpathlineto{\pgfqpoint{2.692575in}{3.079219in}}%
\pgfpathlineto{\pgfqpoint{2.692575in}{3.074961in}}%
\pgfpathmoveto{\pgfqpoint{2.684059in}{3.079219in}}%
\pgfpathlineto{\pgfqpoint{2.684059in}{3.079219in}}%
\pgfpathlineto{\pgfqpoint{2.684059in}{3.083477in}}%
\pgfpathlineto{\pgfqpoint{2.688317in}{3.083477in}}%
\pgfpathlineto{\pgfqpoint{2.688317in}{3.079219in}}%
\pgfpathmoveto{\pgfqpoint{2.684059in}{3.083477in}}%
\pgfpathlineto{\pgfqpoint{2.684059in}{3.083477in}}%
\pgfpathlineto{\pgfqpoint{2.684059in}{3.087735in}}%
\pgfpathlineto{\pgfqpoint{2.688317in}{3.087735in}}%
\pgfpathlineto{\pgfqpoint{2.688317in}{3.083477in}}%
\pgfpathmoveto{\pgfqpoint{2.688317in}{3.079219in}}%
\pgfpathlineto{\pgfqpoint{2.688317in}{3.079219in}}%
\pgfpathlineto{\pgfqpoint{2.688317in}{3.083477in}}%
\pgfpathlineto{\pgfqpoint{2.692575in}{3.083477in}}%
\pgfpathlineto{\pgfqpoint{2.692575in}{3.079219in}}%
\pgfpathmoveto{\pgfqpoint{2.688317in}{3.083477in}}%
\pgfpathlineto{\pgfqpoint{2.688317in}{3.083477in}}%
\pgfpathlineto{\pgfqpoint{2.688317in}{3.087735in}}%
\pgfpathlineto{\pgfqpoint{2.692575in}{3.087735in}}%
\pgfpathlineto{\pgfqpoint{2.692575in}{3.083477in}}%
\pgfpathmoveto{\pgfqpoint{2.684059in}{3.087735in}}%
\pgfpathlineto{\pgfqpoint{2.684059in}{3.087735in}}%
\pgfpathlineto{\pgfqpoint{2.684059in}{3.091993in}}%
\pgfpathlineto{\pgfqpoint{2.688317in}{3.091993in}}%
\pgfpathlineto{\pgfqpoint{2.688317in}{3.087735in}}%
\pgfpathmoveto{\pgfqpoint{2.684059in}{3.091993in}}%
\pgfpathlineto{\pgfqpoint{2.684059in}{3.091993in}}%
\pgfpathlineto{\pgfqpoint{2.684059in}{3.096251in}}%
\pgfpathlineto{\pgfqpoint{2.688317in}{3.096251in}}%
\pgfpathlineto{\pgfqpoint{2.688317in}{3.091993in}}%
\pgfpathmoveto{\pgfqpoint{2.688317in}{3.087735in}}%
\pgfpathlineto{\pgfqpoint{2.688317in}{3.087735in}}%
\pgfpathlineto{\pgfqpoint{2.688317in}{3.091993in}}%
\pgfpathlineto{\pgfqpoint{2.692575in}{3.091993in}}%
\pgfpathlineto{\pgfqpoint{2.692575in}{3.087735in}}%
\pgfpathmoveto{\pgfqpoint{2.688317in}{3.091993in}}%
\pgfpathlineto{\pgfqpoint{2.688317in}{3.091993in}}%
\pgfpathlineto{\pgfqpoint{2.688317in}{3.096251in}}%
\pgfpathlineto{\pgfqpoint{2.692575in}{3.096251in}}%
\pgfpathlineto{\pgfqpoint{2.692575in}{3.091993in}}%
\pgfpathmoveto{\pgfqpoint{2.692575in}{3.087735in}}%
\pgfpathlineto{\pgfqpoint{2.692575in}{3.087735in}}%
\pgfpathlineto{\pgfqpoint{2.692575in}{3.091993in}}%
\pgfpathlineto{\pgfqpoint{2.696833in}{3.091993in}}%
\pgfpathlineto{\pgfqpoint{2.696833in}{3.087735in}}%
\pgfpathmoveto{\pgfqpoint{2.692575in}{3.091993in}}%
\pgfpathlineto{\pgfqpoint{2.692575in}{3.091993in}}%
\pgfpathlineto{\pgfqpoint{2.692575in}{3.096251in}}%
\pgfpathlineto{\pgfqpoint{2.696833in}{3.096251in}}%
\pgfpathlineto{\pgfqpoint{2.696833in}{3.091993in}}%
\pgfpathmoveto{\pgfqpoint{2.684059in}{3.096251in}}%
\pgfpathlineto{\pgfqpoint{2.684059in}{3.096251in}}%
\pgfpathlineto{\pgfqpoint{2.684059in}{3.100509in}}%
\pgfpathlineto{\pgfqpoint{2.688317in}{3.100509in}}%
\pgfpathlineto{\pgfqpoint{2.688317in}{3.096251in}}%
\pgfpathmoveto{\pgfqpoint{2.684059in}{3.100509in}}%
\pgfpathlineto{\pgfqpoint{2.684059in}{3.100509in}}%
\pgfpathlineto{\pgfqpoint{2.684059in}{3.104766in}}%
\pgfpathlineto{\pgfqpoint{2.688317in}{3.104766in}}%
\pgfpathlineto{\pgfqpoint{2.688317in}{3.100509in}}%
\pgfpathmoveto{\pgfqpoint{2.688317in}{3.096251in}}%
\pgfpathlineto{\pgfqpoint{2.688317in}{3.096251in}}%
\pgfpathlineto{\pgfqpoint{2.688317in}{3.100509in}}%
\pgfpathlineto{\pgfqpoint{2.692575in}{3.100509in}}%
\pgfpathlineto{\pgfqpoint{2.692575in}{3.096251in}}%
\pgfpathmoveto{\pgfqpoint{2.688317in}{3.100509in}}%
\pgfpathlineto{\pgfqpoint{2.688317in}{3.100509in}}%
\pgfpathlineto{\pgfqpoint{2.688317in}{3.104766in}}%
\pgfpathlineto{\pgfqpoint{2.692575in}{3.104766in}}%
\pgfpathlineto{\pgfqpoint{2.692575in}{3.100509in}}%
\pgfpathmoveto{\pgfqpoint{2.684059in}{3.104766in}}%
\pgfpathlineto{\pgfqpoint{2.684059in}{3.104766in}}%
\pgfpathlineto{\pgfqpoint{2.684059in}{3.109024in}}%
\pgfpathlineto{\pgfqpoint{2.688317in}{3.109024in}}%
\pgfpathlineto{\pgfqpoint{2.688317in}{3.104766in}}%
\pgfpathmoveto{\pgfqpoint{2.684059in}{3.109024in}}%
\pgfpathlineto{\pgfqpoint{2.684059in}{3.109024in}}%
\pgfpathlineto{\pgfqpoint{2.684059in}{3.113282in}}%
\pgfpathlineto{\pgfqpoint{2.688317in}{3.113282in}}%
\pgfpathlineto{\pgfqpoint{2.688317in}{3.109024in}}%
\pgfpathmoveto{\pgfqpoint{2.688317in}{3.104766in}}%
\pgfpathlineto{\pgfqpoint{2.688317in}{3.104766in}}%
\pgfpathlineto{\pgfqpoint{2.688317in}{3.109024in}}%
\pgfpathlineto{\pgfqpoint{2.692575in}{3.109024in}}%
\pgfpathlineto{\pgfqpoint{2.692575in}{3.104766in}}%
\pgfpathmoveto{\pgfqpoint{2.688317in}{3.109024in}}%
\pgfpathlineto{\pgfqpoint{2.688317in}{3.109024in}}%
\pgfpathlineto{\pgfqpoint{2.688317in}{3.113282in}}%
\pgfpathlineto{\pgfqpoint{2.692575in}{3.113282in}}%
\pgfpathlineto{\pgfqpoint{2.692575in}{3.109024in}}%
\pgfpathmoveto{\pgfqpoint{2.692575in}{3.096251in}}%
\pgfpathlineto{\pgfqpoint{2.692575in}{3.096251in}}%
\pgfpathlineto{\pgfqpoint{2.692575in}{3.100509in}}%
\pgfpathlineto{\pgfqpoint{2.696833in}{3.100509in}}%
\pgfpathlineto{\pgfqpoint{2.696833in}{3.096251in}}%
\pgfpathmoveto{\pgfqpoint{2.692575in}{3.100509in}}%
\pgfpathlineto{\pgfqpoint{2.692575in}{3.100509in}}%
\pgfpathlineto{\pgfqpoint{2.692575in}{3.104766in}}%
\pgfpathlineto{\pgfqpoint{2.696833in}{3.104766in}}%
\pgfpathlineto{\pgfqpoint{2.696833in}{3.100509in}}%
\pgfpathmoveto{\pgfqpoint{2.692575in}{3.104766in}}%
\pgfpathlineto{\pgfqpoint{2.692575in}{3.104766in}}%
\pgfpathlineto{\pgfqpoint{2.692575in}{3.109024in}}%
\pgfpathlineto{\pgfqpoint{2.696833in}{3.109024in}}%
\pgfpathlineto{\pgfqpoint{2.696833in}{3.104766in}}%
\pgfpathmoveto{\pgfqpoint{2.692575in}{3.109024in}}%
\pgfpathlineto{\pgfqpoint{2.692575in}{3.109024in}}%
\pgfpathlineto{\pgfqpoint{2.692575in}{3.113282in}}%
\pgfpathlineto{\pgfqpoint{2.696833in}{3.113282in}}%
\pgfpathlineto{\pgfqpoint{2.696833in}{3.109024in}}%
\pgfpathmoveto{\pgfqpoint{2.696833in}{3.109024in}}%
\pgfpathlineto{\pgfqpoint{2.696833in}{3.109024in}}%
\pgfpathlineto{\pgfqpoint{2.696833in}{3.113282in}}%
\pgfpathlineto{\pgfqpoint{2.701090in}{3.113282in}}%
\pgfpathlineto{\pgfqpoint{2.701090in}{3.109024in}}%
\pgfpathmoveto{\pgfqpoint{2.688317in}{3.113282in}}%
\pgfpathlineto{\pgfqpoint{2.688317in}{3.113282in}}%
\pgfpathlineto{\pgfqpoint{2.688317in}{3.117539in}}%
\pgfpathlineto{\pgfqpoint{2.692575in}{3.117539in}}%
\pgfpathlineto{\pgfqpoint{2.692575in}{3.113282in}}%
\pgfpathmoveto{\pgfqpoint{2.688317in}{3.117539in}}%
\pgfpathlineto{\pgfqpoint{2.688317in}{3.117539in}}%
\pgfpathlineto{\pgfqpoint{2.688317in}{3.121797in}}%
\pgfpathlineto{\pgfqpoint{2.692575in}{3.121797in}}%
\pgfpathlineto{\pgfqpoint{2.692575in}{3.117539in}}%
\pgfpathmoveto{\pgfqpoint{2.688317in}{3.121797in}}%
\pgfpathlineto{\pgfqpoint{2.688317in}{3.121797in}}%
\pgfpathlineto{\pgfqpoint{2.688317in}{3.126055in}}%
\pgfpathlineto{\pgfqpoint{2.692575in}{3.126055in}}%
\pgfpathlineto{\pgfqpoint{2.692575in}{3.121797in}}%
\pgfpathmoveto{\pgfqpoint{2.688317in}{3.126055in}}%
\pgfpathlineto{\pgfqpoint{2.688317in}{3.126055in}}%
\pgfpathlineto{\pgfqpoint{2.688317in}{3.130312in}}%
\pgfpathlineto{\pgfqpoint{2.692575in}{3.130312in}}%
\pgfpathlineto{\pgfqpoint{2.692575in}{3.126055in}}%
\pgfpathmoveto{\pgfqpoint{2.692575in}{3.113282in}}%
\pgfpathlineto{\pgfqpoint{2.692575in}{3.113282in}}%
\pgfpathlineto{\pgfqpoint{2.692575in}{3.117539in}}%
\pgfpathlineto{\pgfqpoint{2.696833in}{3.117539in}}%
\pgfpathlineto{\pgfqpoint{2.696833in}{3.113282in}}%
\pgfpathmoveto{\pgfqpoint{2.692575in}{3.117539in}}%
\pgfpathlineto{\pgfqpoint{2.692575in}{3.117539in}}%
\pgfpathlineto{\pgfqpoint{2.692575in}{3.121797in}}%
\pgfpathlineto{\pgfqpoint{2.696833in}{3.121797in}}%
\pgfpathlineto{\pgfqpoint{2.696833in}{3.117539in}}%
\pgfpathmoveto{\pgfqpoint{2.696833in}{3.113282in}}%
\pgfpathlineto{\pgfqpoint{2.696833in}{3.113282in}}%
\pgfpathlineto{\pgfqpoint{2.696833in}{3.117539in}}%
\pgfpathlineto{\pgfqpoint{2.701090in}{3.117539in}}%
\pgfpathlineto{\pgfqpoint{2.701090in}{3.113282in}}%
\pgfpathmoveto{\pgfqpoint{2.696833in}{3.117539in}}%
\pgfpathlineto{\pgfqpoint{2.696833in}{3.117539in}}%
\pgfpathlineto{\pgfqpoint{2.696833in}{3.121797in}}%
\pgfpathlineto{\pgfqpoint{2.701090in}{3.121797in}}%
\pgfpathlineto{\pgfqpoint{2.701090in}{3.117539in}}%
\pgfpathmoveto{\pgfqpoint{2.692575in}{3.121797in}}%
\pgfpathlineto{\pgfqpoint{2.692575in}{3.121797in}}%
\pgfpathlineto{\pgfqpoint{2.692575in}{3.126055in}}%
\pgfpathlineto{\pgfqpoint{2.696833in}{3.126055in}}%
\pgfpathlineto{\pgfqpoint{2.696833in}{3.121797in}}%
\pgfpathmoveto{\pgfqpoint{2.692575in}{3.126055in}}%
\pgfpathlineto{\pgfqpoint{2.692575in}{3.126055in}}%
\pgfpathlineto{\pgfqpoint{2.692575in}{3.130312in}}%
\pgfpathlineto{\pgfqpoint{2.696833in}{3.130312in}}%
\pgfpathlineto{\pgfqpoint{2.696833in}{3.126055in}}%
\pgfpathmoveto{\pgfqpoint{2.696833in}{3.121797in}}%
\pgfpathlineto{\pgfqpoint{2.696833in}{3.121797in}}%
\pgfpathlineto{\pgfqpoint{2.696833in}{3.126055in}}%
\pgfpathlineto{\pgfqpoint{2.701090in}{3.126055in}}%
\pgfpathlineto{\pgfqpoint{2.701090in}{3.121797in}}%
\pgfpathmoveto{\pgfqpoint{2.696833in}{3.126055in}}%
\pgfpathlineto{\pgfqpoint{2.696833in}{3.126055in}}%
\pgfpathlineto{\pgfqpoint{2.696833in}{3.130312in}}%
\pgfpathlineto{\pgfqpoint{2.701090in}{3.130312in}}%
\pgfpathlineto{\pgfqpoint{2.701090in}{3.126055in}}%
\pgfpathmoveto{\pgfqpoint{2.701090in}{3.126055in}}%
\pgfpathlineto{\pgfqpoint{2.701090in}{3.126055in}}%
\pgfpathlineto{\pgfqpoint{2.701090in}{3.130312in}}%
\pgfpathlineto{\pgfqpoint{2.705348in}{3.130312in}}%
\pgfpathlineto{\pgfqpoint{2.705348in}{3.126055in}}%
\pgfpathmoveto{\pgfqpoint{2.688317in}{3.130312in}}%
\pgfpathlineto{\pgfqpoint{2.688317in}{3.130312in}}%
\pgfpathlineto{\pgfqpoint{2.688317in}{3.134570in}}%
\pgfpathlineto{\pgfqpoint{2.692575in}{3.134570in}}%
\pgfpathlineto{\pgfqpoint{2.692575in}{3.130312in}}%
\pgfpathmoveto{\pgfqpoint{2.692575in}{3.130312in}}%
\pgfpathlineto{\pgfqpoint{2.692575in}{3.130312in}}%
\pgfpathlineto{\pgfqpoint{2.692575in}{3.134570in}}%
\pgfpathlineto{\pgfqpoint{2.696833in}{3.134570in}}%
\pgfpathlineto{\pgfqpoint{2.696833in}{3.130312in}}%
\pgfpathmoveto{\pgfqpoint{2.692575in}{3.134570in}}%
\pgfpathlineto{\pgfqpoint{2.692575in}{3.134570in}}%
\pgfpathlineto{\pgfqpoint{2.692575in}{3.138828in}}%
\pgfpathlineto{\pgfqpoint{2.696833in}{3.138828in}}%
\pgfpathlineto{\pgfqpoint{2.696833in}{3.134570in}}%
\pgfpathmoveto{\pgfqpoint{2.696833in}{3.130312in}}%
\pgfpathlineto{\pgfqpoint{2.696833in}{3.130312in}}%
\pgfpathlineto{\pgfqpoint{2.696833in}{3.134570in}}%
\pgfpathlineto{\pgfqpoint{2.701090in}{3.134570in}}%
\pgfpathlineto{\pgfqpoint{2.701090in}{3.130312in}}%
\pgfpathmoveto{\pgfqpoint{2.696833in}{3.134570in}}%
\pgfpathlineto{\pgfqpoint{2.696833in}{3.134570in}}%
\pgfpathlineto{\pgfqpoint{2.696833in}{3.138828in}}%
\pgfpathlineto{\pgfqpoint{2.701090in}{3.138828in}}%
\pgfpathlineto{\pgfqpoint{2.701090in}{3.134570in}}%
\pgfpathmoveto{\pgfqpoint{2.692575in}{3.138828in}}%
\pgfpathlineto{\pgfqpoint{2.692575in}{3.138828in}}%
\pgfpathlineto{\pgfqpoint{2.692575in}{3.143085in}}%
\pgfpathlineto{\pgfqpoint{2.696833in}{3.143085in}}%
\pgfpathlineto{\pgfqpoint{2.696833in}{3.138828in}}%
\pgfpathmoveto{\pgfqpoint{2.692575in}{3.143085in}}%
\pgfpathlineto{\pgfqpoint{2.692575in}{3.143085in}}%
\pgfpathlineto{\pgfqpoint{2.692575in}{3.147343in}}%
\pgfpathlineto{\pgfqpoint{2.696833in}{3.147343in}}%
\pgfpathlineto{\pgfqpoint{2.696833in}{3.143085in}}%
\pgfpathmoveto{\pgfqpoint{2.696833in}{3.138828in}}%
\pgfpathlineto{\pgfqpoint{2.696833in}{3.138828in}}%
\pgfpathlineto{\pgfqpoint{2.696833in}{3.143085in}}%
\pgfpathlineto{\pgfqpoint{2.701090in}{3.143085in}}%
\pgfpathlineto{\pgfqpoint{2.701090in}{3.138828in}}%
\pgfpathmoveto{\pgfqpoint{2.696833in}{3.143085in}}%
\pgfpathlineto{\pgfqpoint{2.696833in}{3.143085in}}%
\pgfpathlineto{\pgfqpoint{2.696833in}{3.147343in}}%
\pgfpathlineto{\pgfqpoint{2.701090in}{3.147343in}}%
\pgfpathlineto{\pgfqpoint{2.701090in}{3.143085in}}%
\pgfpathmoveto{\pgfqpoint{2.692575in}{3.147343in}}%
\pgfpathlineto{\pgfqpoint{2.692575in}{3.147343in}}%
\pgfpathlineto{\pgfqpoint{2.692575in}{3.151601in}}%
\pgfpathlineto{\pgfqpoint{2.696833in}{3.151601in}}%
\pgfpathlineto{\pgfqpoint{2.696833in}{3.147343in}}%
\pgfpathmoveto{\pgfqpoint{2.692575in}{3.151601in}}%
\pgfpathlineto{\pgfqpoint{2.692575in}{3.151601in}}%
\pgfpathlineto{\pgfqpoint{2.692575in}{3.155858in}}%
\pgfpathlineto{\pgfqpoint{2.696833in}{3.155858in}}%
\pgfpathlineto{\pgfqpoint{2.696833in}{3.151601in}}%
\pgfpathmoveto{\pgfqpoint{2.696833in}{3.147343in}}%
\pgfpathlineto{\pgfqpoint{2.696833in}{3.147343in}}%
\pgfpathlineto{\pgfqpoint{2.696833in}{3.151601in}}%
\pgfpathlineto{\pgfqpoint{2.701090in}{3.151601in}}%
\pgfpathlineto{\pgfqpoint{2.701090in}{3.147343in}}%
\pgfpathmoveto{\pgfqpoint{2.696833in}{3.151601in}}%
\pgfpathlineto{\pgfqpoint{2.696833in}{3.151601in}}%
\pgfpathlineto{\pgfqpoint{2.696833in}{3.155858in}}%
\pgfpathlineto{\pgfqpoint{2.701090in}{3.155858in}}%
\pgfpathlineto{\pgfqpoint{2.701090in}{3.151601in}}%
\pgfpathmoveto{\pgfqpoint{2.696833in}{3.155858in}}%
\pgfpathlineto{\pgfqpoint{2.696833in}{3.155858in}}%
\pgfpathlineto{\pgfqpoint{2.696833in}{3.160116in}}%
\pgfpathlineto{\pgfqpoint{2.701090in}{3.160116in}}%
\pgfpathlineto{\pgfqpoint{2.701090in}{3.155858in}}%
\pgfpathmoveto{\pgfqpoint{2.696833in}{3.160116in}}%
\pgfpathlineto{\pgfqpoint{2.696833in}{3.160116in}}%
\pgfpathlineto{\pgfqpoint{2.696833in}{3.164374in}}%
\pgfpathlineto{\pgfqpoint{2.701090in}{3.164374in}}%
\pgfpathlineto{\pgfqpoint{2.701090in}{3.160116in}}%
\pgfpathmoveto{\pgfqpoint{2.701090in}{3.130312in}}%
\pgfpathlineto{\pgfqpoint{2.701090in}{3.130312in}}%
\pgfpathlineto{\pgfqpoint{2.701090in}{3.134570in}}%
\pgfpathlineto{\pgfqpoint{2.705348in}{3.134570in}}%
\pgfpathlineto{\pgfqpoint{2.705348in}{3.130312in}}%
\pgfpathmoveto{\pgfqpoint{2.701090in}{3.134570in}}%
\pgfpathlineto{\pgfqpoint{2.701090in}{3.134570in}}%
\pgfpathlineto{\pgfqpoint{2.701090in}{3.138828in}}%
\pgfpathlineto{\pgfqpoint{2.705348in}{3.138828in}}%
\pgfpathlineto{\pgfqpoint{2.705348in}{3.134570in}}%
\pgfpathmoveto{\pgfqpoint{2.701090in}{3.138828in}}%
\pgfpathlineto{\pgfqpoint{2.701090in}{3.138828in}}%
\pgfpathlineto{\pgfqpoint{2.701090in}{3.143085in}}%
\pgfpathlineto{\pgfqpoint{2.705348in}{3.143085in}}%
\pgfpathlineto{\pgfqpoint{2.705348in}{3.138828in}}%
\pgfpathmoveto{\pgfqpoint{2.701090in}{3.143085in}}%
\pgfpathlineto{\pgfqpoint{2.701090in}{3.143085in}}%
\pgfpathlineto{\pgfqpoint{2.701090in}{3.147343in}}%
\pgfpathlineto{\pgfqpoint{2.705348in}{3.147343in}}%
\pgfpathlineto{\pgfqpoint{2.705348in}{3.143085in}}%
\pgfpathmoveto{\pgfqpoint{2.701090in}{3.147343in}}%
\pgfpathlineto{\pgfqpoint{2.701090in}{3.147343in}}%
\pgfpathlineto{\pgfqpoint{2.701090in}{3.151601in}}%
\pgfpathlineto{\pgfqpoint{2.705348in}{3.151601in}}%
\pgfpathlineto{\pgfqpoint{2.705348in}{3.147343in}}%
\pgfpathmoveto{\pgfqpoint{2.701090in}{3.151601in}}%
\pgfpathlineto{\pgfqpoint{2.701090in}{3.151601in}}%
\pgfpathlineto{\pgfqpoint{2.701090in}{3.155858in}}%
\pgfpathlineto{\pgfqpoint{2.705348in}{3.155858in}}%
\pgfpathlineto{\pgfqpoint{2.705348in}{3.151601in}}%
\pgfpathmoveto{\pgfqpoint{2.705348in}{3.147343in}}%
\pgfpathlineto{\pgfqpoint{2.705348in}{3.147343in}}%
\pgfpathlineto{\pgfqpoint{2.705348in}{3.151601in}}%
\pgfpathlineto{\pgfqpoint{2.709606in}{3.151601in}}%
\pgfpathlineto{\pgfqpoint{2.709606in}{3.147343in}}%
\pgfpathmoveto{\pgfqpoint{2.705348in}{3.151601in}}%
\pgfpathlineto{\pgfqpoint{2.705348in}{3.151601in}}%
\pgfpathlineto{\pgfqpoint{2.705348in}{3.155858in}}%
\pgfpathlineto{\pgfqpoint{2.709606in}{3.155858in}}%
\pgfpathlineto{\pgfqpoint{2.709606in}{3.151601in}}%
\pgfpathmoveto{\pgfqpoint{2.701090in}{3.155858in}}%
\pgfpathlineto{\pgfqpoint{2.701090in}{3.155858in}}%
\pgfpathlineto{\pgfqpoint{2.701090in}{3.160116in}}%
\pgfpathlineto{\pgfqpoint{2.705348in}{3.160116in}}%
\pgfpathlineto{\pgfqpoint{2.705348in}{3.155858in}}%
\pgfpathmoveto{\pgfqpoint{2.701090in}{3.160116in}}%
\pgfpathlineto{\pgfqpoint{2.701090in}{3.160116in}}%
\pgfpathlineto{\pgfqpoint{2.701090in}{3.164374in}}%
\pgfpathlineto{\pgfqpoint{2.705348in}{3.164374in}}%
\pgfpathlineto{\pgfqpoint{2.705348in}{3.160116in}}%
\pgfpathmoveto{\pgfqpoint{2.705348in}{3.155858in}}%
\pgfpathlineto{\pgfqpoint{2.705348in}{3.155858in}}%
\pgfpathlineto{\pgfqpoint{2.705348in}{3.160116in}}%
\pgfpathlineto{\pgfqpoint{2.709606in}{3.160116in}}%
\pgfpathlineto{\pgfqpoint{2.709606in}{3.155858in}}%
\pgfpathmoveto{\pgfqpoint{2.705348in}{3.160116in}}%
\pgfpathlineto{\pgfqpoint{2.705348in}{3.160116in}}%
\pgfpathlineto{\pgfqpoint{2.705348in}{3.164374in}}%
\pgfpathlineto{\pgfqpoint{2.709606in}{3.164374in}}%
\pgfpathlineto{\pgfqpoint{2.709606in}{3.160116in}}%
\pgfpathmoveto{\pgfqpoint{2.696833in}{3.164374in}}%
\pgfpathlineto{\pgfqpoint{2.696833in}{3.164374in}}%
\pgfpathlineto{\pgfqpoint{2.696833in}{3.168631in}}%
\pgfpathlineto{\pgfqpoint{2.701090in}{3.168631in}}%
\pgfpathlineto{\pgfqpoint{2.701090in}{3.164374in}}%
\pgfpathmoveto{\pgfqpoint{2.696833in}{3.168631in}}%
\pgfpathlineto{\pgfqpoint{2.696833in}{3.168631in}}%
\pgfpathlineto{\pgfqpoint{2.696833in}{3.172889in}}%
\pgfpathlineto{\pgfqpoint{2.701090in}{3.172889in}}%
\pgfpathlineto{\pgfqpoint{2.701090in}{3.168631in}}%
\pgfpathmoveto{\pgfqpoint{2.701090in}{3.164374in}}%
\pgfpathlineto{\pgfqpoint{2.701090in}{3.164374in}}%
\pgfpathlineto{\pgfqpoint{2.701090in}{3.168631in}}%
\pgfpathlineto{\pgfqpoint{2.705348in}{3.168631in}}%
\pgfpathlineto{\pgfqpoint{2.705348in}{3.164374in}}%
\pgfpathmoveto{\pgfqpoint{2.701090in}{3.168631in}}%
\pgfpathlineto{\pgfqpoint{2.701090in}{3.168631in}}%
\pgfpathlineto{\pgfqpoint{2.701090in}{3.172889in}}%
\pgfpathlineto{\pgfqpoint{2.705348in}{3.172889in}}%
\pgfpathlineto{\pgfqpoint{2.705348in}{3.168631in}}%
\pgfpathmoveto{\pgfqpoint{2.705348in}{3.164374in}}%
\pgfpathlineto{\pgfqpoint{2.705348in}{3.164374in}}%
\pgfpathlineto{\pgfqpoint{2.705348in}{3.168631in}}%
\pgfpathlineto{\pgfqpoint{2.709606in}{3.168631in}}%
\pgfpathlineto{\pgfqpoint{2.709606in}{3.164374in}}%
\pgfpathmoveto{\pgfqpoint{2.705348in}{3.168631in}}%
\pgfpathlineto{\pgfqpoint{2.705348in}{3.168631in}}%
\pgfpathlineto{\pgfqpoint{2.705348in}{3.172889in}}%
\pgfpathlineto{\pgfqpoint{2.709606in}{3.172889in}}%
\pgfpathlineto{\pgfqpoint{2.709606in}{3.168631in}}%
\pgfpathmoveto{\pgfqpoint{2.701090in}{3.172889in}}%
\pgfpathlineto{\pgfqpoint{2.701090in}{3.172889in}}%
\pgfpathlineto{\pgfqpoint{2.701090in}{3.177147in}}%
\pgfpathlineto{\pgfqpoint{2.705348in}{3.177147in}}%
\pgfpathlineto{\pgfqpoint{2.705348in}{3.172889in}}%
\pgfpathmoveto{\pgfqpoint{2.701090in}{3.177147in}}%
\pgfpathlineto{\pgfqpoint{2.701090in}{3.177147in}}%
\pgfpathlineto{\pgfqpoint{2.701090in}{3.181404in}}%
\pgfpathlineto{\pgfqpoint{2.705348in}{3.181404in}}%
\pgfpathlineto{\pgfqpoint{2.705348in}{3.177147in}}%
\pgfpathmoveto{\pgfqpoint{2.705348in}{3.172889in}}%
\pgfpathlineto{\pgfqpoint{2.705348in}{3.172889in}}%
\pgfpathlineto{\pgfqpoint{2.705348in}{3.177147in}}%
\pgfpathlineto{\pgfqpoint{2.709606in}{3.177147in}}%
\pgfpathlineto{\pgfqpoint{2.709606in}{3.172889in}}%
\pgfpathmoveto{\pgfqpoint{2.705348in}{3.177147in}}%
\pgfpathlineto{\pgfqpoint{2.705348in}{3.177147in}}%
\pgfpathlineto{\pgfqpoint{2.705348in}{3.181404in}}%
\pgfpathlineto{\pgfqpoint{2.709606in}{3.181404in}}%
\pgfpathlineto{\pgfqpoint{2.709606in}{3.177147in}}%
\pgfpathmoveto{\pgfqpoint{2.709606in}{3.168631in}}%
\pgfpathlineto{\pgfqpoint{2.709606in}{3.168631in}}%
\pgfpathlineto{\pgfqpoint{2.709606in}{3.172889in}}%
\pgfpathlineto{\pgfqpoint{2.713864in}{3.172889in}}%
\pgfpathlineto{\pgfqpoint{2.713864in}{3.168631in}}%
\pgfpathmoveto{\pgfqpoint{2.709606in}{3.172889in}}%
\pgfpathlineto{\pgfqpoint{2.709606in}{3.172889in}}%
\pgfpathlineto{\pgfqpoint{2.709606in}{3.177147in}}%
\pgfpathlineto{\pgfqpoint{2.713864in}{3.177147in}}%
\pgfpathlineto{\pgfqpoint{2.713864in}{3.172889in}}%
\pgfpathmoveto{\pgfqpoint{2.709606in}{3.177147in}}%
\pgfpathlineto{\pgfqpoint{2.709606in}{3.177147in}}%
\pgfpathlineto{\pgfqpoint{2.709606in}{3.181404in}}%
\pgfpathlineto{\pgfqpoint{2.713864in}{3.181404in}}%
\pgfpathlineto{\pgfqpoint{2.713864in}{3.177147in}}%
\pgfpathmoveto{\pgfqpoint{2.701090in}{3.181404in}}%
\pgfpathlineto{\pgfqpoint{2.701090in}{3.181404in}}%
\pgfpathlineto{\pgfqpoint{2.701090in}{3.185662in}}%
\pgfpathlineto{\pgfqpoint{2.705348in}{3.185662in}}%
\pgfpathlineto{\pgfqpoint{2.705348in}{3.181404in}}%
\pgfpathmoveto{\pgfqpoint{2.701090in}{3.185662in}}%
\pgfpathlineto{\pgfqpoint{2.701090in}{3.185662in}}%
\pgfpathlineto{\pgfqpoint{2.701090in}{3.189920in}}%
\pgfpathlineto{\pgfqpoint{2.705348in}{3.189920in}}%
\pgfpathlineto{\pgfqpoint{2.705348in}{3.185662in}}%
\pgfpathmoveto{\pgfqpoint{2.705348in}{3.181404in}}%
\pgfpathlineto{\pgfqpoint{2.705348in}{3.181404in}}%
\pgfpathlineto{\pgfqpoint{2.705348in}{3.185662in}}%
\pgfpathlineto{\pgfqpoint{2.709606in}{3.185662in}}%
\pgfpathlineto{\pgfqpoint{2.709606in}{3.181404in}}%
\pgfpathmoveto{\pgfqpoint{2.705348in}{3.185662in}}%
\pgfpathlineto{\pgfqpoint{2.705348in}{3.185662in}}%
\pgfpathlineto{\pgfqpoint{2.705348in}{3.189920in}}%
\pgfpathlineto{\pgfqpoint{2.709606in}{3.189920in}}%
\pgfpathlineto{\pgfqpoint{2.709606in}{3.185662in}}%
\pgfpathmoveto{\pgfqpoint{2.701090in}{3.189920in}}%
\pgfpathlineto{\pgfqpoint{2.701090in}{3.189920in}}%
\pgfpathlineto{\pgfqpoint{2.701090in}{3.194177in}}%
\pgfpathlineto{\pgfqpoint{2.705348in}{3.194177in}}%
\pgfpathlineto{\pgfqpoint{2.705348in}{3.189920in}}%
\pgfpathmoveto{\pgfqpoint{2.705348in}{3.189920in}}%
\pgfpathlineto{\pgfqpoint{2.705348in}{3.189920in}}%
\pgfpathlineto{\pgfqpoint{2.705348in}{3.194177in}}%
\pgfpathlineto{\pgfqpoint{2.709606in}{3.194177in}}%
\pgfpathlineto{\pgfqpoint{2.709606in}{3.189920in}}%
\pgfpathmoveto{\pgfqpoint{2.705348in}{3.194177in}}%
\pgfpathlineto{\pgfqpoint{2.705348in}{3.194177in}}%
\pgfpathlineto{\pgfqpoint{2.705348in}{3.198435in}}%
\pgfpathlineto{\pgfqpoint{2.709606in}{3.198435in}}%
\pgfpathlineto{\pgfqpoint{2.709606in}{3.194177in}}%
\pgfpathmoveto{\pgfqpoint{2.709606in}{3.181404in}}%
\pgfpathlineto{\pgfqpoint{2.709606in}{3.181404in}}%
\pgfpathlineto{\pgfqpoint{2.709606in}{3.185662in}}%
\pgfpathlineto{\pgfqpoint{2.713864in}{3.185662in}}%
\pgfpathlineto{\pgfqpoint{2.713864in}{3.181404in}}%
\pgfpathmoveto{\pgfqpoint{2.709606in}{3.185662in}}%
\pgfpathlineto{\pgfqpoint{2.709606in}{3.185662in}}%
\pgfpathlineto{\pgfqpoint{2.709606in}{3.189920in}}%
\pgfpathlineto{\pgfqpoint{2.713864in}{3.189920in}}%
\pgfpathlineto{\pgfqpoint{2.713864in}{3.185662in}}%
\pgfpathmoveto{\pgfqpoint{2.713864in}{3.185662in}}%
\pgfpathlineto{\pgfqpoint{2.713864in}{3.185662in}}%
\pgfpathlineto{\pgfqpoint{2.713864in}{3.189920in}}%
\pgfpathlineto{\pgfqpoint{2.718122in}{3.189920in}}%
\pgfpathlineto{\pgfqpoint{2.718122in}{3.185662in}}%
\pgfpathmoveto{\pgfqpoint{2.709606in}{3.189920in}}%
\pgfpathlineto{\pgfqpoint{2.709606in}{3.189920in}}%
\pgfpathlineto{\pgfqpoint{2.709606in}{3.194177in}}%
\pgfpathlineto{\pgfqpoint{2.713864in}{3.194177in}}%
\pgfpathlineto{\pgfqpoint{2.713864in}{3.189920in}}%
\pgfpathmoveto{\pgfqpoint{2.709606in}{3.194177in}}%
\pgfpathlineto{\pgfqpoint{2.709606in}{3.194177in}}%
\pgfpathlineto{\pgfqpoint{2.709606in}{3.198435in}}%
\pgfpathlineto{\pgfqpoint{2.713864in}{3.198435in}}%
\pgfpathlineto{\pgfqpoint{2.713864in}{3.194177in}}%
\pgfpathmoveto{\pgfqpoint{2.713864in}{3.189920in}}%
\pgfpathlineto{\pgfqpoint{2.713864in}{3.189920in}}%
\pgfpathlineto{\pgfqpoint{2.713864in}{3.194177in}}%
\pgfpathlineto{\pgfqpoint{2.718122in}{3.194177in}}%
\pgfpathlineto{\pgfqpoint{2.718122in}{3.189920in}}%
\pgfpathmoveto{\pgfqpoint{2.713864in}{3.194177in}}%
\pgfpathlineto{\pgfqpoint{2.713864in}{3.194177in}}%
\pgfpathlineto{\pgfqpoint{2.713864in}{3.198435in}}%
\pgfpathlineto{\pgfqpoint{2.718122in}{3.198435in}}%
\pgfpathlineto{\pgfqpoint{2.718122in}{3.194177in}}%
\pgfpathmoveto{\pgfqpoint{2.705348in}{3.198435in}}%
\pgfpathlineto{\pgfqpoint{2.705348in}{3.198435in}}%
\pgfpathlineto{\pgfqpoint{2.705348in}{3.202693in}}%
\pgfpathlineto{\pgfqpoint{2.709606in}{3.202693in}}%
\pgfpathlineto{\pgfqpoint{2.709606in}{3.198435in}}%
\pgfpathmoveto{\pgfqpoint{2.705348in}{3.202693in}}%
\pgfpathlineto{\pgfqpoint{2.705348in}{3.202693in}}%
\pgfpathlineto{\pgfqpoint{2.705348in}{3.206950in}}%
\pgfpathlineto{\pgfqpoint{2.709606in}{3.206950in}}%
\pgfpathlineto{\pgfqpoint{2.709606in}{3.202693in}}%
\pgfpathmoveto{\pgfqpoint{2.705348in}{3.206950in}}%
\pgfpathlineto{\pgfqpoint{2.705348in}{3.206950in}}%
\pgfpathlineto{\pgfqpoint{2.705348in}{3.211208in}}%
\pgfpathlineto{\pgfqpoint{2.709606in}{3.211208in}}%
\pgfpathlineto{\pgfqpoint{2.709606in}{3.206950in}}%
\pgfpathmoveto{\pgfqpoint{2.705348in}{3.211208in}}%
\pgfpathlineto{\pgfqpoint{2.705348in}{3.211208in}}%
\pgfpathlineto{\pgfqpoint{2.705348in}{3.215466in}}%
\pgfpathlineto{\pgfqpoint{2.709606in}{3.215466in}}%
\pgfpathlineto{\pgfqpoint{2.709606in}{3.211208in}}%
\pgfpathmoveto{\pgfqpoint{2.709606in}{3.198435in}}%
\pgfpathlineto{\pgfqpoint{2.709606in}{3.198435in}}%
\pgfpathlineto{\pgfqpoint{2.709606in}{3.202693in}}%
\pgfpathlineto{\pgfqpoint{2.713864in}{3.202693in}}%
\pgfpathlineto{\pgfqpoint{2.713864in}{3.198435in}}%
\pgfpathmoveto{\pgfqpoint{2.709606in}{3.202693in}}%
\pgfpathlineto{\pgfqpoint{2.709606in}{3.202693in}}%
\pgfpathlineto{\pgfqpoint{2.709606in}{3.206950in}}%
\pgfpathlineto{\pgfqpoint{2.713864in}{3.206950in}}%
\pgfpathlineto{\pgfqpoint{2.713864in}{3.202693in}}%
\pgfpathmoveto{\pgfqpoint{2.713864in}{3.198435in}}%
\pgfpathlineto{\pgfqpoint{2.713864in}{3.198435in}}%
\pgfpathlineto{\pgfqpoint{2.713864in}{3.202693in}}%
\pgfpathlineto{\pgfqpoint{2.718122in}{3.202693in}}%
\pgfpathlineto{\pgfqpoint{2.718122in}{3.198435in}}%
\pgfpathmoveto{\pgfqpoint{2.713864in}{3.202693in}}%
\pgfpathlineto{\pgfqpoint{2.713864in}{3.202693in}}%
\pgfpathlineto{\pgfqpoint{2.713864in}{3.206950in}}%
\pgfpathlineto{\pgfqpoint{2.718122in}{3.206950in}}%
\pgfpathlineto{\pgfqpoint{2.718122in}{3.202693in}}%
\pgfpathmoveto{\pgfqpoint{2.709606in}{3.206950in}}%
\pgfpathlineto{\pgfqpoint{2.709606in}{3.206950in}}%
\pgfpathlineto{\pgfqpoint{2.709606in}{3.211208in}}%
\pgfpathlineto{\pgfqpoint{2.713864in}{3.211208in}}%
\pgfpathlineto{\pgfqpoint{2.713864in}{3.206950in}}%
\pgfpathmoveto{\pgfqpoint{2.709606in}{3.211208in}}%
\pgfpathlineto{\pgfqpoint{2.709606in}{3.211208in}}%
\pgfpathlineto{\pgfqpoint{2.709606in}{3.215466in}}%
\pgfpathlineto{\pgfqpoint{2.713864in}{3.215466in}}%
\pgfpathlineto{\pgfqpoint{2.713864in}{3.211208in}}%
\pgfpathmoveto{\pgfqpoint{2.713864in}{3.206950in}}%
\pgfpathlineto{\pgfqpoint{2.713864in}{3.206950in}}%
\pgfpathlineto{\pgfqpoint{2.713864in}{3.211208in}}%
\pgfpathlineto{\pgfqpoint{2.718122in}{3.211208in}}%
\pgfpathlineto{\pgfqpoint{2.718122in}{3.206950in}}%
\pgfpathmoveto{\pgfqpoint{2.713864in}{3.211208in}}%
\pgfpathlineto{\pgfqpoint{2.713864in}{3.211208in}}%
\pgfpathlineto{\pgfqpoint{2.713864in}{3.215466in}}%
\pgfpathlineto{\pgfqpoint{2.718122in}{3.215466in}}%
\pgfpathlineto{\pgfqpoint{2.718122in}{3.211208in}}%
\pgfpathmoveto{\pgfqpoint{2.709606in}{3.215466in}}%
\pgfpathlineto{\pgfqpoint{2.709606in}{3.215466in}}%
\pgfpathlineto{\pgfqpoint{2.709606in}{3.219723in}}%
\pgfpathlineto{\pgfqpoint{2.713864in}{3.219723in}}%
\pgfpathlineto{\pgfqpoint{2.713864in}{3.215466in}}%
\pgfpathmoveto{\pgfqpoint{2.709606in}{3.219723in}}%
\pgfpathlineto{\pgfqpoint{2.709606in}{3.219723in}}%
\pgfpathlineto{\pgfqpoint{2.709606in}{3.223981in}}%
\pgfpathlineto{\pgfqpoint{2.713864in}{3.223981in}}%
\pgfpathlineto{\pgfqpoint{2.713864in}{3.219723in}}%
\pgfpathmoveto{\pgfqpoint{2.713864in}{3.215466in}}%
\pgfpathlineto{\pgfqpoint{2.713864in}{3.215466in}}%
\pgfpathlineto{\pgfqpoint{2.713864in}{3.219723in}}%
\pgfpathlineto{\pgfqpoint{2.718122in}{3.219723in}}%
\pgfpathlineto{\pgfqpoint{2.718122in}{3.215466in}}%
\pgfpathmoveto{\pgfqpoint{2.713864in}{3.219723in}}%
\pgfpathlineto{\pgfqpoint{2.713864in}{3.219723in}}%
\pgfpathlineto{\pgfqpoint{2.713864in}{3.223981in}}%
\pgfpathlineto{\pgfqpoint{2.718122in}{3.223981in}}%
\pgfpathlineto{\pgfqpoint{2.718122in}{3.219723in}}%
\pgfpathmoveto{\pgfqpoint{2.709606in}{3.223981in}}%
\pgfpathlineto{\pgfqpoint{2.709606in}{3.223981in}}%
\pgfpathlineto{\pgfqpoint{2.709606in}{3.228239in}}%
\pgfpathlineto{\pgfqpoint{2.713864in}{3.228239in}}%
\pgfpathlineto{\pgfqpoint{2.713864in}{3.223981in}}%
\pgfpathmoveto{\pgfqpoint{2.709606in}{3.228239in}}%
\pgfpathlineto{\pgfqpoint{2.709606in}{3.228239in}}%
\pgfpathlineto{\pgfqpoint{2.709606in}{3.232497in}}%
\pgfpathlineto{\pgfqpoint{2.713864in}{3.232497in}}%
\pgfpathlineto{\pgfqpoint{2.713864in}{3.228239in}}%
\pgfpathmoveto{\pgfqpoint{2.713864in}{3.223981in}}%
\pgfpathlineto{\pgfqpoint{2.713864in}{3.223981in}}%
\pgfpathlineto{\pgfqpoint{2.713864in}{3.228239in}}%
\pgfpathlineto{\pgfqpoint{2.718122in}{3.228239in}}%
\pgfpathlineto{\pgfqpoint{2.718122in}{3.223981in}}%
\pgfpathmoveto{\pgfqpoint{2.713864in}{3.228239in}}%
\pgfpathlineto{\pgfqpoint{2.713864in}{3.228239in}}%
\pgfpathlineto{\pgfqpoint{2.713864in}{3.232497in}}%
\pgfpathlineto{\pgfqpoint{2.718122in}{3.232497in}}%
\pgfpathlineto{\pgfqpoint{2.718122in}{3.228239in}}%
\pgfpathmoveto{\pgfqpoint{2.718122in}{3.206950in}}%
\pgfpathlineto{\pgfqpoint{2.718122in}{3.206950in}}%
\pgfpathlineto{\pgfqpoint{2.718122in}{3.211208in}}%
\pgfpathlineto{\pgfqpoint{2.722380in}{3.211208in}}%
\pgfpathlineto{\pgfqpoint{2.722380in}{3.206950in}}%
\pgfpathmoveto{\pgfqpoint{2.718122in}{3.211208in}}%
\pgfpathlineto{\pgfqpoint{2.718122in}{3.211208in}}%
\pgfpathlineto{\pgfqpoint{2.718122in}{3.215466in}}%
\pgfpathlineto{\pgfqpoint{2.722380in}{3.215466in}}%
\pgfpathlineto{\pgfqpoint{2.722380in}{3.211208in}}%
\pgfpathmoveto{\pgfqpoint{2.718122in}{3.215466in}}%
\pgfpathlineto{\pgfqpoint{2.718122in}{3.215466in}}%
\pgfpathlineto{\pgfqpoint{2.718122in}{3.219723in}}%
\pgfpathlineto{\pgfqpoint{2.722380in}{3.219723in}}%
\pgfpathlineto{\pgfqpoint{2.722380in}{3.215466in}}%
\pgfpathmoveto{\pgfqpoint{2.718122in}{3.219723in}}%
\pgfpathlineto{\pgfqpoint{2.718122in}{3.219723in}}%
\pgfpathlineto{\pgfqpoint{2.718122in}{3.223981in}}%
\pgfpathlineto{\pgfqpoint{2.722380in}{3.223981in}}%
\pgfpathlineto{\pgfqpoint{2.722380in}{3.219723in}}%
\pgfpathmoveto{\pgfqpoint{2.718122in}{3.223981in}}%
\pgfpathlineto{\pgfqpoint{2.718122in}{3.223981in}}%
\pgfpathlineto{\pgfqpoint{2.718122in}{3.228239in}}%
\pgfpathlineto{\pgfqpoint{2.722380in}{3.228239in}}%
\pgfpathlineto{\pgfqpoint{2.722380in}{3.223981in}}%
\pgfpathmoveto{\pgfqpoint{2.718122in}{3.228239in}}%
\pgfpathlineto{\pgfqpoint{2.718122in}{3.228239in}}%
\pgfpathlineto{\pgfqpoint{2.718122in}{3.232497in}}%
\pgfpathlineto{\pgfqpoint{2.722380in}{3.232497in}}%
\pgfpathlineto{\pgfqpoint{2.722380in}{3.228239in}}%
\pgfpathmoveto{\pgfqpoint{2.722380in}{3.223981in}}%
\pgfpathlineto{\pgfqpoint{2.722380in}{3.223981in}}%
\pgfpathlineto{\pgfqpoint{2.722380in}{3.228239in}}%
\pgfpathlineto{\pgfqpoint{2.726638in}{3.228239in}}%
\pgfpathlineto{\pgfqpoint{2.726638in}{3.223981in}}%
\pgfpathmoveto{\pgfqpoint{2.722380in}{3.228239in}}%
\pgfpathlineto{\pgfqpoint{2.722380in}{3.228239in}}%
\pgfpathlineto{\pgfqpoint{2.722380in}{3.232497in}}%
\pgfpathlineto{\pgfqpoint{2.726638in}{3.232497in}}%
\pgfpathlineto{\pgfqpoint{2.726638in}{3.228239in}}%
\pgfpathmoveto{\pgfqpoint{2.713864in}{3.232497in}}%
\pgfpathlineto{\pgfqpoint{2.713864in}{3.232497in}}%
\pgfpathlineto{\pgfqpoint{2.713864in}{3.236754in}}%
\pgfpathlineto{\pgfqpoint{2.718122in}{3.236754in}}%
\pgfpathlineto{\pgfqpoint{2.718122in}{3.232497in}}%
\pgfpathmoveto{\pgfqpoint{2.713864in}{3.236754in}}%
\pgfpathlineto{\pgfqpoint{2.713864in}{3.236754in}}%
\pgfpathlineto{\pgfqpoint{2.713864in}{3.241012in}}%
\pgfpathlineto{\pgfqpoint{2.718122in}{3.241012in}}%
\pgfpathlineto{\pgfqpoint{2.718122in}{3.236754in}}%
\pgfpathmoveto{\pgfqpoint{2.713864in}{3.241012in}}%
\pgfpathlineto{\pgfqpoint{2.713864in}{3.241012in}}%
\pgfpathlineto{\pgfqpoint{2.713864in}{3.245270in}}%
\pgfpathlineto{\pgfqpoint{2.718122in}{3.245270in}}%
\pgfpathlineto{\pgfqpoint{2.718122in}{3.241012in}}%
\pgfpathmoveto{\pgfqpoint{2.713864in}{3.245270in}}%
\pgfpathlineto{\pgfqpoint{2.713864in}{3.245270in}}%
\pgfpathlineto{\pgfqpoint{2.713864in}{3.249528in}}%
\pgfpathlineto{\pgfqpoint{2.718122in}{3.249528in}}%
\pgfpathlineto{\pgfqpoint{2.718122in}{3.245270in}}%
\pgfpathmoveto{\pgfqpoint{2.713864in}{3.249528in}}%
\pgfpathlineto{\pgfqpoint{2.713864in}{3.249528in}}%
\pgfpathlineto{\pgfqpoint{2.713864in}{3.253786in}}%
\pgfpathlineto{\pgfqpoint{2.718122in}{3.253786in}}%
\pgfpathlineto{\pgfqpoint{2.718122in}{3.249528in}}%
\pgfpathmoveto{\pgfqpoint{2.718122in}{3.232497in}}%
\pgfpathlineto{\pgfqpoint{2.718122in}{3.232497in}}%
\pgfpathlineto{\pgfqpoint{2.718122in}{3.236754in}}%
\pgfpathlineto{\pgfqpoint{2.722380in}{3.236754in}}%
\pgfpathlineto{\pgfqpoint{2.722380in}{3.232497in}}%
\pgfpathmoveto{\pgfqpoint{2.718122in}{3.236754in}}%
\pgfpathlineto{\pgfqpoint{2.718122in}{3.236754in}}%
\pgfpathlineto{\pgfqpoint{2.718122in}{3.241012in}}%
\pgfpathlineto{\pgfqpoint{2.722380in}{3.241012in}}%
\pgfpathlineto{\pgfqpoint{2.722380in}{3.236754in}}%
\pgfpathmoveto{\pgfqpoint{2.722380in}{3.232497in}}%
\pgfpathlineto{\pgfqpoint{2.722380in}{3.232497in}}%
\pgfpathlineto{\pgfqpoint{2.722380in}{3.236754in}}%
\pgfpathlineto{\pgfqpoint{2.726638in}{3.236754in}}%
\pgfpathlineto{\pgfqpoint{2.726638in}{3.232497in}}%
\pgfpathmoveto{\pgfqpoint{2.722380in}{3.236754in}}%
\pgfpathlineto{\pgfqpoint{2.722380in}{3.236754in}}%
\pgfpathlineto{\pgfqpoint{2.722380in}{3.241012in}}%
\pgfpathlineto{\pgfqpoint{2.726638in}{3.241012in}}%
\pgfpathlineto{\pgfqpoint{2.726638in}{3.236754in}}%
\pgfpathmoveto{\pgfqpoint{2.718122in}{3.241012in}}%
\pgfpathlineto{\pgfqpoint{2.718122in}{3.241012in}}%
\pgfpathlineto{\pgfqpoint{2.718122in}{3.245270in}}%
\pgfpathlineto{\pgfqpoint{2.722380in}{3.245270in}}%
\pgfpathlineto{\pgfqpoint{2.722380in}{3.241012in}}%
\pgfpathmoveto{\pgfqpoint{2.718122in}{3.245270in}}%
\pgfpathlineto{\pgfqpoint{2.718122in}{3.245270in}}%
\pgfpathlineto{\pgfqpoint{2.718122in}{3.249528in}}%
\pgfpathlineto{\pgfqpoint{2.722380in}{3.249528in}}%
\pgfpathlineto{\pgfqpoint{2.722380in}{3.245270in}}%
\pgfpathmoveto{\pgfqpoint{2.722380in}{3.241012in}}%
\pgfpathlineto{\pgfqpoint{2.722380in}{3.241012in}}%
\pgfpathlineto{\pgfqpoint{2.722380in}{3.245270in}}%
\pgfpathlineto{\pgfqpoint{2.726638in}{3.245270in}}%
\pgfpathlineto{\pgfqpoint{2.726638in}{3.241012in}}%
\pgfpathmoveto{\pgfqpoint{2.722380in}{3.245270in}}%
\pgfpathlineto{\pgfqpoint{2.722380in}{3.245270in}}%
\pgfpathlineto{\pgfqpoint{2.722380in}{3.249528in}}%
\pgfpathlineto{\pgfqpoint{2.726638in}{3.249528in}}%
\pgfpathlineto{\pgfqpoint{2.726638in}{3.245270in}}%
\pgfpathmoveto{\pgfqpoint{2.726638in}{3.241012in}}%
\pgfpathlineto{\pgfqpoint{2.726638in}{3.241012in}}%
\pgfpathlineto{\pgfqpoint{2.726638in}{3.245270in}}%
\pgfpathlineto{\pgfqpoint{2.730896in}{3.245270in}}%
\pgfpathlineto{\pgfqpoint{2.730896in}{3.241012in}}%
\pgfpathmoveto{\pgfqpoint{2.726638in}{3.245270in}}%
\pgfpathlineto{\pgfqpoint{2.726638in}{3.245270in}}%
\pgfpathlineto{\pgfqpoint{2.726638in}{3.249528in}}%
\pgfpathlineto{\pgfqpoint{2.730896in}{3.249528in}}%
\pgfpathlineto{\pgfqpoint{2.730896in}{3.245270in}}%
\pgfpathmoveto{\pgfqpoint{2.718122in}{3.249528in}}%
\pgfpathlineto{\pgfqpoint{2.718122in}{3.249528in}}%
\pgfpathlineto{\pgfqpoint{2.718122in}{3.253786in}}%
\pgfpathlineto{\pgfqpoint{2.722380in}{3.253786in}}%
\pgfpathlineto{\pgfqpoint{2.722380in}{3.249528in}}%
\pgfpathmoveto{\pgfqpoint{2.718122in}{3.253786in}}%
\pgfpathlineto{\pgfqpoint{2.718122in}{3.253786in}}%
\pgfpathlineto{\pgfqpoint{2.718122in}{3.258044in}}%
\pgfpathlineto{\pgfqpoint{2.722380in}{3.258044in}}%
\pgfpathlineto{\pgfqpoint{2.722380in}{3.253786in}}%
\pgfpathmoveto{\pgfqpoint{2.722380in}{3.249528in}}%
\pgfpathlineto{\pgfqpoint{2.722380in}{3.249528in}}%
\pgfpathlineto{\pgfqpoint{2.722380in}{3.253786in}}%
\pgfpathlineto{\pgfqpoint{2.726638in}{3.253786in}}%
\pgfpathlineto{\pgfqpoint{2.726638in}{3.249528in}}%
\pgfpathmoveto{\pgfqpoint{2.722380in}{3.253786in}}%
\pgfpathlineto{\pgfqpoint{2.722380in}{3.253786in}}%
\pgfpathlineto{\pgfqpoint{2.722380in}{3.258044in}}%
\pgfpathlineto{\pgfqpoint{2.726638in}{3.258044in}}%
\pgfpathlineto{\pgfqpoint{2.726638in}{3.253786in}}%
\pgfpathmoveto{\pgfqpoint{2.718122in}{3.258044in}}%
\pgfpathlineto{\pgfqpoint{2.718122in}{3.258044in}}%
\pgfpathlineto{\pgfqpoint{2.718122in}{3.262302in}}%
\pgfpathlineto{\pgfqpoint{2.722380in}{3.262302in}}%
\pgfpathlineto{\pgfqpoint{2.722380in}{3.258044in}}%
\pgfpathmoveto{\pgfqpoint{2.718122in}{3.262302in}}%
\pgfpathlineto{\pgfqpoint{2.718122in}{3.262302in}}%
\pgfpathlineto{\pgfqpoint{2.718122in}{3.266560in}}%
\pgfpathlineto{\pgfqpoint{2.722380in}{3.266560in}}%
\pgfpathlineto{\pgfqpoint{2.722380in}{3.262302in}}%
\pgfpathmoveto{\pgfqpoint{2.722380in}{3.258044in}}%
\pgfpathlineto{\pgfqpoint{2.722380in}{3.258044in}}%
\pgfpathlineto{\pgfqpoint{2.722380in}{3.262302in}}%
\pgfpathlineto{\pgfqpoint{2.726638in}{3.262302in}}%
\pgfpathlineto{\pgfqpoint{2.726638in}{3.258044in}}%
\pgfpathmoveto{\pgfqpoint{2.722380in}{3.262302in}}%
\pgfpathlineto{\pgfqpoint{2.722380in}{3.262302in}}%
\pgfpathlineto{\pgfqpoint{2.722380in}{3.266560in}}%
\pgfpathlineto{\pgfqpoint{2.726638in}{3.266560in}}%
\pgfpathlineto{\pgfqpoint{2.726638in}{3.262302in}}%
\pgfpathmoveto{\pgfqpoint{2.726638in}{3.249528in}}%
\pgfpathlineto{\pgfqpoint{2.726638in}{3.249528in}}%
\pgfpathlineto{\pgfqpoint{2.726638in}{3.253786in}}%
\pgfpathlineto{\pgfqpoint{2.730896in}{3.253786in}}%
\pgfpathlineto{\pgfqpoint{2.730896in}{3.249528in}}%
\pgfpathmoveto{\pgfqpoint{2.726638in}{3.253786in}}%
\pgfpathlineto{\pgfqpoint{2.726638in}{3.253786in}}%
\pgfpathlineto{\pgfqpoint{2.726638in}{3.258044in}}%
\pgfpathlineto{\pgfqpoint{2.730896in}{3.258044in}}%
\pgfpathlineto{\pgfqpoint{2.730896in}{3.253786in}}%
\pgfpathmoveto{\pgfqpoint{2.726638in}{3.258044in}}%
\pgfpathlineto{\pgfqpoint{2.726638in}{3.258044in}}%
\pgfpathlineto{\pgfqpoint{2.726638in}{3.262302in}}%
\pgfpathlineto{\pgfqpoint{2.730896in}{3.262302in}}%
\pgfpathlineto{\pgfqpoint{2.730896in}{3.258044in}}%
\pgfpathmoveto{\pgfqpoint{2.726638in}{3.262302in}}%
\pgfpathlineto{\pgfqpoint{2.726638in}{3.262302in}}%
\pgfpathlineto{\pgfqpoint{2.726638in}{3.266560in}}%
\pgfpathlineto{\pgfqpoint{2.730896in}{3.266560in}}%
\pgfpathlineto{\pgfqpoint{2.730896in}{3.262302in}}%
\pgfpathmoveto{\pgfqpoint{2.730896in}{3.258044in}}%
\pgfpathlineto{\pgfqpoint{2.730896in}{3.258044in}}%
\pgfpathlineto{\pgfqpoint{2.730896in}{3.262302in}}%
\pgfpathlineto{\pgfqpoint{2.735153in}{3.262302in}}%
\pgfpathlineto{\pgfqpoint{2.735153in}{3.258044in}}%
\pgfpathmoveto{\pgfqpoint{2.730896in}{3.262302in}}%
\pgfpathlineto{\pgfqpoint{2.730896in}{3.262302in}}%
\pgfpathlineto{\pgfqpoint{2.730896in}{3.266560in}}%
\pgfpathlineto{\pgfqpoint{2.735153in}{3.266560in}}%
\pgfpathlineto{\pgfqpoint{2.735153in}{3.262302in}}%
\pgfpathmoveto{\pgfqpoint{2.718122in}{3.266560in}}%
\pgfpathlineto{\pgfqpoint{2.718122in}{3.266560in}}%
\pgfpathlineto{\pgfqpoint{2.718122in}{3.270818in}}%
\pgfpathlineto{\pgfqpoint{2.722380in}{3.270818in}}%
\pgfpathlineto{\pgfqpoint{2.722380in}{3.266560in}}%
\pgfpathmoveto{\pgfqpoint{2.722380in}{3.266560in}}%
\pgfpathlineto{\pgfqpoint{2.722380in}{3.266560in}}%
\pgfpathlineto{\pgfqpoint{2.722380in}{3.270818in}}%
\pgfpathlineto{\pgfqpoint{2.726638in}{3.270818in}}%
\pgfpathlineto{\pgfqpoint{2.726638in}{3.266560in}}%
\pgfpathmoveto{\pgfqpoint{2.722380in}{3.270818in}}%
\pgfpathlineto{\pgfqpoint{2.722380in}{3.270818in}}%
\pgfpathlineto{\pgfqpoint{2.722380in}{3.275076in}}%
\pgfpathlineto{\pgfqpoint{2.726638in}{3.275076in}}%
\pgfpathlineto{\pgfqpoint{2.726638in}{3.270818in}}%
\pgfpathmoveto{\pgfqpoint{2.722380in}{3.275076in}}%
\pgfpathlineto{\pgfqpoint{2.722380in}{3.275076in}}%
\pgfpathlineto{\pgfqpoint{2.722380in}{3.279334in}}%
\pgfpathlineto{\pgfqpoint{2.726638in}{3.279334in}}%
\pgfpathlineto{\pgfqpoint{2.726638in}{3.275076in}}%
\pgfpathmoveto{\pgfqpoint{2.722380in}{3.279334in}}%
\pgfpathlineto{\pgfqpoint{2.722380in}{3.279334in}}%
\pgfpathlineto{\pgfqpoint{2.722380in}{3.283592in}}%
\pgfpathlineto{\pgfqpoint{2.726638in}{3.283592in}}%
\pgfpathlineto{\pgfqpoint{2.726638in}{3.279334in}}%
\pgfpathmoveto{\pgfqpoint{2.726638in}{3.266560in}}%
\pgfpathlineto{\pgfqpoint{2.726638in}{3.266560in}}%
\pgfpathlineto{\pgfqpoint{2.726638in}{3.270818in}}%
\pgfpathlineto{\pgfqpoint{2.730896in}{3.270818in}}%
\pgfpathlineto{\pgfqpoint{2.730896in}{3.266560in}}%
\pgfpathmoveto{\pgfqpoint{2.726638in}{3.270818in}}%
\pgfpathlineto{\pgfqpoint{2.726638in}{3.270818in}}%
\pgfpathlineto{\pgfqpoint{2.726638in}{3.275076in}}%
\pgfpathlineto{\pgfqpoint{2.730896in}{3.275076in}}%
\pgfpathlineto{\pgfqpoint{2.730896in}{3.270818in}}%
\pgfpathmoveto{\pgfqpoint{2.730896in}{3.266560in}}%
\pgfpathlineto{\pgfqpoint{2.730896in}{3.266560in}}%
\pgfpathlineto{\pgfqpoint{2.730896in}{3.270818in}}%
\pgfpathlineto{\pgfqpoint{2.735153in}{3.270818in}}%
\pgfpathlineto{\pgfqpoint{2.735153in}{3.266560in}}%
\pgfpathmoveto{\pgfqpoint{2.730896in}{3.270818in}}%
\pgfpathlineto{\pgfqpoint{2.730896in}{3.270818in}}%
\pgfpathlineto{\pgfqpoint{2.730896in}{3.275076in}}%
\pgfpathlineto{\pgfqpoint{2.735153in}{3.275076in}}%
\pgfpathlineto{\pgfqpoint{2.735153in}{3.270818in}}%
\pgfpathmoveto{\pgfqpoint{2.726638in}{3.275076in}}%
\pgfpathlineto{\pgfqpoint{2.726638in}{3.275076in}}%
\pgfpathlineto{\pgfqpoint{2.726638in}{3.279334in}}%
\pgfpathlineto{\pgfqpoint{2.730896in}{3.279334in}}%
\pgfpathlineto{\pgfqpoint{2.730896in}{3.275076in}}%
\pgfpathmoveto{\pgfqpoint{2.726638in}{3.279334in}}%
\pgfpathlineto{\pgfqpoint{2.726638in}{3.279334in}}%
\pgfpathlineto{\pgfqpoint{2.726638in}{3.283592in}}%
\pgfpathlineto{\pgfqpoint{2.730896in}{3.283592in}}%
\pgfpathlineto{\pgfqpoint{2.730896in}{3.279334in}}%
\pgfpathmoveto{\pgfqpoint{2.730896in}{3.275076in}}%
\pgfpathlineto{\pgfqpoint{2.730896in}{3.275076in}}%
\pgfpathlineto{\pgfqpoint{2.730896in}{3.279334in}}%
\pgfpathlineto{\pgfqpoint{2.735153in}{3.279334in}}%
\pgfpathlineto{\pgfqpoint{2.735153in}{3.275076in}}%
\pgfpathmoveto{\pgfqpoint{2.730896in}{3.279334in}}%
\pgfpathlineto{\pgfqpoint{2.730896in}{3.279334in}}%
\pgfpathlineto{\pgfqpoint{2.730896in}{3.283592in}}%
\pgfpathlineto{\pgfqpoint{2.735153in}{3.283592in}}%
\pgfpathlineto{\pgfqpoint{2.735153in}{3.279334in}}%
\pgfpathmoveto{\pgfqpoint{2.722380in}{3.283592in}}%
\pgfpathlineto{\pgfqpoint{2.722380in}{3.283592in}}%
\pgfpathlineto{\pgfqpoint{2.722380in}{3.287850in}}%
\pgfpathlineto{\pgfqpoint{2.726638in}{3.287850in}}%
\pgfpathlineto{\pgfqpoint{2.726638in}{3.283592in}}%
\pgfpathmoveto{\pgfqpoint{2.726638in}{3.283592in}}%
\pgfpathlineto{\pgfqpoint{2.726638in}{3.283592in}}%
\pgfpathlineto{\pgfqpoint{2.726638in}{3.287850in}}%
\pgfpathlineto{\pgfqpoint{2.730896in}{3.287850in}}%
\pgfpathlineto{\pgfqpoint{2.730896in}{3.283592in}}%
\pgfpathmoveto{\pgfqpoint{2.726638in}{3.287850in}}%
\pgfpathlineto{\pgfqpoint{2.726638in}{3.287850in}}%
\pgfpathlineto{\pgfqpoint{2.726638in}{3.292108in}}%
\pgfpathlineto{\pgfqpoint{2.730896in}{3.292108in}}%
\pgfpathlineto{\pgfqpoint{2.730896in}{3.287850in}}%
\pgfpathmoveto{\pgfqpoint{2.730896in}{3.283592in}}%
\pgfpathlineto{\pgfqpoint{2.730896in}{3.283592in}}%
\pgfpathlineto{\pgfqpoint{2.730896in}{3.287850in}}%
\pgfpathlineto{\pgfqpoint{2.735153in}{3.287850in}}%
\pgfpathlineto{\pgfqpoint{2.735153in}{3.283592in}}%
\pgfpathmoveto{\pgfqpoint{2.730896in}{3.287850in}}%
\pgfpathlineto{\pgfqpoint{2.730896in}{3.287850in}}%
\pgfpathlineto{\pgfqpoint{2.730896in}{3.292108in}}%
\pgfpathlineto{\pgfqpoint{2.735153in}{3.292108in}}%
\pgfpathlineto{\pgfqpoint{2.735153in}{3.287850in}}%
\pgfpathmoveto{\pgfqpoint{2.726638in}{3.292108in}}%
\pgfpathlineto{\pgfqpoint{2.726638in}{3.292108in}}%
\pgfpathlineto{\pgfqpoint{2.726638in}{3.296366in}}%
\pgfpathlineto{\pgfqpoint{2.730896in}{3.296366in}}%
\pgfpathlineto{\pgfqpoint{2.730896in}{3.292108in}}%
\pgfpathmoveto{\pgfqpoint{2.726638in}{3.296366in}}%
\pgfpathlineto{\pgfqpoint{2.726638in}{3.296366in}}%
\pgfpathlineto{\pgfqpoint{2.726638in}{3.300624in}}%
\pgfpathlineto{\pgfqpoint{2.730896in}{3.300624in}}%
\pgfpathlineto{\pgfqpoint{2.730896in}{3.296366in}}%
\pgfpathmoveto{\pgfqpoint{2.730896in}{3.292108in}}%
\pgfpathlineto{\pgfqpoint{2.730896in}{3.292108in}}%
\pgfpathlineto{\pgfqpoint{2.730896in}{3.296366in}}%
\pgfpathlineto{\pgfqpoint{2.735153in}{3.296366in}}%
\pgfpathlineto{\pgfqpoint{2.735153in}{3.292108in}}%
\pgfpathmoveto{\pgfqpoint{2.730896in}{3.296366in}}%
\pgfpathlineto{\pgfqpoint{2.730896in}{3.296366in}}%
\pgfpathlineto{\pgfqpoint{2.730896in}{3.300624in}}%
\pgfpathlineto{\pgfqpoint{2.735153in}{3.300624in}}%
\pgfpathlineto{\pgfqpoint{2.735153in}{3.296366in}}%
\pgfpathmoveto{\pgfqpoint{2.735153in}{3.275076in}}%
\pgfpathlineto{\pgfqpoint{2.735153in}{3.275076in}}%
\pgfpathlineto{\pgfqpoint{2.735153in}{3.279334in}}%
\pgfpathlineto{\pgfqpoint{2.739411in}{3.279334in}}%
\pgfpathlineto{\pgfqpoint{2.739411in}{3.275076in}}%
\pgfpathmoveto{\pgfqpoint{2.735153in}{3.279334in}}%
\pgfpathlineto{\pgfqpoint{2.735153in}{3.279334in}}%
\pgfpathlineto{\pgfqpoint{2.735153in}{3.283592in}}%
\pgfpathlineto{\pgfqpoint{2.739411in}{3.283592in}}%
\pgfpathlineto{\pgfqpoint{2.739411in}{3.279334in}}%
\pgfpathmoveto{\pgfqpoint{2.735153in}{3.283592in}}%
\pgfpathlineto{\pgfqpoint{2.735153in}{3.283592in}}%
\pgfpathlineto{\pgfqpoint{2.735153in}{3.287850in}}%
\pgfpathlineto{\pgfqpoint{2.739411in}{3.287850in}}%
\pgfpathlineto{\pgfqpoint{2.739411in}{3.283592in}}%
\pgfpathmoveto{\pgfqpoint{2.735153in}{3.287850in}}%
\pgfpathlineto{\pgfqpoint{2.735153in}{3.287850in}}%
\pgfpathlineto{\pgfqpoint{2.735153in}{3.292108in}}%
\pgfpathlineto{\pgfqpoint{2.739411in}{3.292108in}}%
\pgfpathlineto{\pgfqpoint{2.739411in}{3.287850in}}%
\pgfpathmoveto{\pgfqpoint{2.735153in}{3.292108in}}%
\pgfpathlineto{\pgfqpoint{2.735153in}{3.292108in}}%
\pgfpathlineto{\pgfqpoint{2.735153in}{3.296366in}}%
\pgfpathlineto{\pgfqpoint{2.739411in}{3.296366in}}%
\pgfpathlineto{\pgfqpoint{2.739411in}{3.292108in}}%
\pgfpathmoveto{\pgfqpoint{2.735153in}{3.296366in}}%
\pgfpathlineto{\pgfqpoint{2.735153in}{3.296366in}}%
\pgfpathlineto{\pgfqpoint{2.735153in}{3.300624in}}%
\pgfpathlineto{\pgfqpoint{2.739411in}{3.300624in}}%
\pgfpathlineto{\pgfqpoint{2.739411in}{3.296366in}}%
\pgfpathmoveto{\pgfqpoint{2.739411in}{3.292108in}}%
\pgfpathlineto{\pgfqpoint{2.739411in}{3.292108in}}%
\pgfpathlineto{\pgfqpoint{2.739411in}{3.296366in}}%
\pgfpathlineto{\pgfqpoint{2.743669in}{3.296366in}}%
\pgfpathlineto{\pgfqpoint{2.743669in}{3.292108in}}%
\pgfpathmoveto{\pgfqpoint{2.739411in}{3.296366in}}%
\pgfpathlineto{\pgfqpoint{2.739411in}{3.296366in}}%
\pgfpathlineto{\pgfqpoint{2.739411in}{3.300624in}}%
\pgfpathlineto{\pgfqpoint{2.743669in}{3.300624in}}%
\pgfpathlineto{\pgfqpoint{2.743669in}{3.296366in}}%
\pgfpathmoveto{\pgfqpoint{2.726638in}{3.300624in}}%
\pgfpathlineto{\pgfqpoint{2.726638in}{3.300624in}}%
\pgfpathlineto{\pgfqpoint{2.726638in}{3.304882in}}%
\pgfpathlineto{\pgfqpoint{2.730896in}{3.304882in}}%
\pgfpathlineto{\pgfqpoint{2.730896in}{3.300624in}}%
\pgfpathmoveto{\pgfqpoint{2.730896in}{3.300624in}}%
\pgfpathlineto{\pgfqpoint{2.730896in}{3.300624in}}%
\pgfpathlineto{\pgfqpoint{2.730896in}{3.304882in}}%
\pgfpathlineto{\pgfqpoint{2.735153in}{3.304882in}}%
\pgfpathlineto{\pgfqpoint{2.735153in}{3.300624in}}%
\pgfpathmoveto{\pgfqpoint{2.730896in}{3.304882in}}%
\pgfpathlineto{\pgfqpoint{2.730896in}{3.304882in}}%
\pgfpathlineto{\pgfqpoint{2.730896in}{3.309140in}}%
\pgfpathlineto{\pgfqpoint{2.735153in}{3.309140in}}%
\pgfpathlineto{\pgfqpoint{2.735153in}{3.304882in}}%
\pgfpathmoveto{\pgfqpoint{2.730896in}{3.309140in}}%
\pgfpathlineto{\pgfqpoint{2.730896in}{3.309140in}}%
\pgfpathlineto{\pgfqpoint{2.730896in}{3.313398in}}%
\pgfpathlineto{\pgfqpoint{2.735153in}{3.313398in}}%
\pgfpathlineto{\pgfqpoint{2.735153in}{3.309140in}}%
\pgfpathmoveto{\pgfqpoint{2.730896in}{3.313398in}}%
\pgfpathlineto{\pgfqpoint{2.730896in}{3.313398in}}%
\pgfpathlineto{\pgfqpoint{2.730896in}{3.317656in}}%
\pgfpathlineto{\pgfqpoint{2.735153in}{3.317656in}}%
\pgfpathlineto{\pgfqpoint{2.735153in}{3.313398in}}%
\pgfpathmoveto{\pgfqpoint{2.730896in}{3.317656in}}%
\pgfpathlineto{\pgfqpoint{2.730896in}{3.317656in}}%
\pgfpathlineto{\pgfqpoint{2.730896in}{3.321914in}}%
\pgfpathlineto{\pgfqpoint{2.735153in}{3.321914in}}%
\pgfpathlineto{\pgfqpoint{2.735153in}{3.317656in}}%
\pgfpathmoveto{\pgfqpoint{2.735153in}{3.300624in}}%
\pgfpathlineto{\pgfqpoint{2.735153in}{3.300624in}}%
\pgfpathlineto{\pgfqpoint{2.735153in}{3.304882in}}%
\pgfpathlineto{\pgfqpoint{2.739411in}{3.304882in}}%
\pgfpathlineto{\pgfqpoint{2.739411in}{3.300624in}}%
\pgfpathmoveto{\pgfqpoint{2.735153in}{3.304882in}}%
\pgfpathlineto{\pgfqpoint{2.735153in}{3.304882in}}%
\pgfpathlineto{\pgfqpoint{2.735153in}{3.309140in}}%
\pgfpathlineto{\pgfqpoint{2.739411in}{3.309140in}}%
\pgfpathlineto{\pgfqpoint{2.739411in}{3.304882in}}%
\pgfpathmoveto{\pgfqpoint{2.739411in}{3.300624in}}%
\pgfpathlineto{\pgfqpoint{2.739411in}{3.300624in}}%
\pgfpathlineto{\pgfqpoint{2.739411in}{3.304882in}}%
\pgfpathlineto{\pgfqpoint{2.743669in}{3.304882in}}%
\pgfpathlineto{\pgfqpoint{2.743669in}{3.300624in}}%
\pgfpathmoveto{\pgfqpoint{2.739411in}{3.304882in}}%
\pgfpathlineto{\pgfqpoint{2.739411in}{3.304882in}}%
\pgfpathlineto{\pgfqpoint{2.739411in}{3.309140in}}%
\pgfpathlineto{\pgfqpoint{2.743669in}{3.309140in}}%
\pgfpathlineto{\pgfqpoint{2.743669in}{3.304882in}}%
\pgfpathmoveto{\pgfqpoint{2.735153in}{3.309140in}}%
\pgfpathlineto{\pgfqpoint{2.735153in}{3.309140in}}%
\pgfpathlineto{\pgfqpoint{2.735153in}{3.313398in}}%
\pgfpathlineto{\pgfqpoint{2.739411in}{3.313398in}}%
\pgfpathlineto{\pgfqpoint{2.739411in}{3.309140in}}%
\pgfpathmoveto{\pgfqpoint{2.735153in}{3.313398in}}%
\pgfpathlineto{\pgfqpoint{2.735153in}{3.313398in}}%
\pgfpathlineto{\pgfqpoint{2.735153in}{3.317656in}}%
\pgfpathlineto{\pgfqpoint{2.739411in}{3.317656in}}%
\pgfpathlineto{\pgfqpoint{2.739411in}{3.313398in}}%
\pgfpathmoveto{\pgfqpoint{2.739411in}{3.309140in}}%
\pgfpathlineto{\pgfqpoint{2.739411in}{3.309140in}}%
\pgfpathlineto{\pgfqpoint{2.739411in}{3.313398in}}%
\pgfpathlineto{\pgfqpoint{2.743669in}{3.313398in}}%
\pgfpathlineto{\pgfqpoint{2.743669in}{3.309140in}}%
\pgfpathmoveto{\pgfqpoint{2.739411in}{3.313398in}}%
\pgfpathlineto{\pgfqpoint{2.739411in}{3.313398in}}%
\pgfpathlineto{\pgfqpoint{2.739411in}{3.317656in}}%
\pgfpathlineto{\pgfqpoint{2.743669in}{3.317656in}}%
\pgfpathlineto{\pgfqpoint{2.743669in}{3.313398in}}%
\pgfpathmoveto{\pgfqpoint{2.743669in}{3.309140in}}%
\pgfpathlineto{\pgfqpoint{2.743669in}{3.309140in}}%
\pgfpathlineto{\pgfqpoint{2.743669in}{3.313398in}}%
\pgfpathlineto{\pgfqpoint{2.747927in}{3.313398in}}%
\pgfpathlineto{\pgfqpoint{2.747927in}{3.309140in}}%
\pgfpathmoveto{\pgfqpoint{2.743669in}{3.313398in}}%
\pgfpathlineto{\pgfqpoint{2.743669in}{3.313398in}}%
\pgfpathlineto{\pgfqpoint{2.743669in}{3.317656in}}%
\pgfpathlineto{\pgfqpoint{2.747927in}{3.317656in}}%
\pgfpathlineto{\pgfqpoint{2.747927in}{3.313398in}}%
\pgfpathmoveto{\pgfqpoint{2.735153in}{3.317656in}}%
\pgfpathlineto{\pgfqpoint{2.735153in}{3.317656in}}%
\pgfpathlineto{\pgfqpoint{2.735153in}{3.321914in}}%
\pgfpathlineto{\pgfqpoint{2.739411in}{3.321914in}}%
\pgfpathlineto{\pgfqpoint{2.739411in}{3.317656in}}%
\pgfpathmoveto{\pgfqpoint{2.735153in}{3.321914in}}%
\pgfpathlineto{\pgfqpoint{2.735153in}{3.321914in}}%
\pgfpathlineto{\pgfqpoint{2.735153in}{3.326172in}}%
\pgfpathlineto{\pgfqpoint{2.739411in}{3.326172in}}%
\pgfpathlineto{\pgfqpoint{2.739411in}{3.321914in}}%
\pgfpathmoveto{\pgfqpoint{2.739411in}{3.317656in}}%
\pgfpathlineto{\pgfqpoint{2.739411in}{3.317656in}}%
\pgfpathlineto{\pgfqpoint{2.739411in}{3.321914in}}%
\pgfpathlineto{\pgfqpoint{2.743669in}{3.321914in}}%
\pgfpathlineto{\pgfqpoint{2.743669in}{3.317656in}}%
\pgfpathmoveto{\pgfqpoint{2.739411in}{3.321914in}}%
\pgfpathlineto{\pgfqpoint{2.739411in}{3.321914in}}%
\pgfpathlineto{\pgfqpoint{2.739411in}{3.326172in}}%
\pgfpathlineto{\pgfqpoint{2.743669in}{3.326172in}}%
\pgfpathlineto{\pgfqpoint{2.743669in}{3.321914in}}%
\pgfpathmoveto{\pgfqpoint{2.735153in}{3.326172in}}%
\pgfpathlineto{\pgfqpoint{2.735153in}{3.326172in}}%
\pgfpathlineto{\pgfqpoint{2.735153in}{3.330430in}}%
\pgfpathlineto{\pgfqpoint{2.739411in}{3.330430in}}%
\pgfpathlineto{\pgfqpoint{2.739411in}{3.326172in}}%
\pgfpathmoveto{\pgfqpoint{2.735153in}{3.330430in}}%
\pgfpathlineto{\pgfqpoint{2.735153in}{3.330430in}}%
\pgfpathlineto{\pgfqpoint{2.735153in}{3.334688in}}%
\pgfpathlineto{\pgfqpoint{2.739411in}{3.334688in}}%
\pgfpathlineto{\pgfqpoint{2.739411in}{3.330430in}}%
\pgfpathmoveto{\pgfqpoint{2.739411in}{3.326172in}}%
\pgfpathlineto{\pgfqpoint{2.739411in}{3.326172in}}%
\pgfpathlineto{\pgfqpoint{2.739411in}{3.330430in}}%
\pgfpathlineto{\pgfqpoint{2.743669in}{3.330430in}}%
\pgfpathlineto{\pgfqpoint{2.743669in}{3.326172in}}%
\pgfpathmoveto{\pgfqpoint{2.739411in}{3.330430in}}%
\pgfpathlineto{\pgfqpoint{2.739411in}{3.330430in}}%
\pgfpathlineto{\pgfqpoint{2.739411in}{3.334688in}}%
\pgfpathlineto{\pgfqpoint{2.743669in}{3.334688in}}%
\pgfpathlineto{\pgfqpoint{2.743669in}{3.330430in}}%
\pgfpathmoveto{\pgfqpoint{2.743669in}{3.317656in}}%
\pgfpathlineto{\pgfqpoint{2.743669in}{3.317656in}}%
\pgfpathlineto{\pgfqpoint{2.743669in}{3.321914in}}%
\pgfpathlineto{\pgfqpoint{2.747927in}{3.321914in}}%
\pgfpathlineto{\pgfqpoint{2.747927in}{3.317656in}}%
\pgfpathmoveto{\pgfqpoint{2.743669in}{3.321914in}}%
\pgfpathlineto{\pgfqpoint{2.743669in}{3.321914in}}%
\pgfpathlineto{\pgfqpoint{2.743669in}{3.326172in}}%
\pgfpathlineto{\pgfqpoint{2.747927in}{3.326172in}}%
\pgfpathlineto{\pgfqpoint{2.747927in}{3.321914in}}%
\pgfpathmoveto{\pgfqpoint{2.743669in}{3.326172in}}%
\pgfpathlineto{\pgfqpoint{2.743669in}{3.326172in}}%
\pgfpathlineto{\pgfqpoint{2.743669in}{3.330430in}}%
\pgfpathlineto{\pgfqpoint{2.747927in}{3.330430in}}%
\pgfpathlineto{\pgfqpoint{2.747927in}{3.326172in}}%
\pgfpathmoveto{\pgfqpoint{2.743669in}{3.330430in}}%
\pgfpathlineto{\pgfqpoint{2.743669in}{3.330430in}}%
\pgfpathlineto{\pgfqpoint{2.743669in}{3.334688in}}%
\pgfpathlineto{\pgfqpoint{2.747927in}{3.334688in}}%
\pgfpathlineto{\pgfqpoint{2.747927in}{3.330430in}}%
\pgfpathmoveto{\pgfqpoint{2.747927in}{3.326172in}}%
\pgfpathlineto{\pgfqpoint{2.747927in}{3.326172in}}%
\pgfpathlineto{\pgfqpoint{2.747927in}{3.330430in}}%
\pgfpathlineto{\pgfqpoint{2.752185in}{3.330430in}}%
\pgfpathlineto{\pgfqpoint{2.752185in}{3.326172in}}%
\pgfpathmoveto{\pgfqpoint{2.747927in}{3.330430in}}%
\pgfpathlineto{\pgfqpoint{2.747927in}{3.330430in}}%
\pgfpathlineto{\pgfqpoint{2.747927in}{3.334688in}}%
\pgfpathlineto{\pgfqpoint{2.752185in}{3.334688in}}%
\pgfpathlineto{\pgfqpoint{2.752185in}{3.330430in}}%
\pgfpathmoveto{\pgfqpoint{2.735153in}{3.334688in}}%
\pgfpathlineto{\pgfqpoint{2.735153in}{3.334688in}}%
\pgfpathlineto{\pgfqpoint{2.735153in}{3.338946in}}%
\pgfpathlineto{\pgfqpoint{2.739411in}{3.338946in}}%
\pgfpathlineto{\pgfqpoint{2.739411in}{3.334688in}}%
\pgfpathmoveto{\pgfqpoint{2.739411in}{3.334688in}}%
\pgfpathlineto{\pgfqpoint{2.739411in}{3.334688in}}%
\pgfpathlineto{\pgfqpoint{2.739411in}{3.338946in}}%
\pgfpathlineto{\pgfqpoint{2.743669in}{3.338946in}}%
\pgfpathlineto{\pgfqpoint{2.743669in}{3.334688in}}%
\pgfpathmoveto{\pgfqpoint{2.739411in}{3.338946in}}%
\pgfpathlineto{\pgfqpoint{2.739411in}{3.338946in}}%
\pgfpathlineto{\pgfqpoint{2.739411in}{3.343204in}}%
\pgfpathlineto{\pgfqpoint{2.743669in}{3.343204in}}%
\pgfpathlineto{\pgfqpoint{2.743669in}{3.338946in}}%
\pgfpathmoveto{\pgfqpoint{2.739411in}{3.343204in}}%
\pgfpathlineto{\pgfqpoint{2.739411in}{3.343204in}}%
\pgfpathlineto{\pgfqpoint{2.739411in}{3.347462in}}%
\pgfpathlineto{\pgfqpoint{2.743669in}{3.347462in}}%
\pgfpathlineto{\pgfqpoint{2.743669in}{3.343204in}}%
\pgfpathmoveto{\pgfqpoint{2.739411in}{3.347462in}}%
\pgfpathlineto{\pgfqpoint{2.739411in}{3.347462in}}%
\pgfpathlineto{\pgfqpoint{2.739411in}{3.351720in}}%
\pgfpathlineto{\pgfqpoint{2.743669in}{3.351720in}}%
\pgfpathlineto{\pgfqpoint{2.743669in}{3.347462in}}%
\pgfpathmoveto{\pgfqpoint{2.743669in}{3.334688in}}%
\pgfpathlineto{\pgfqpoint{2.743669in}{3.334688in}}%
\pgfpathlineto{\pgfqpoint{2.743669in}{3.338946in}}%
\pgfpathlineto{\pgfqpoint{2.747927in}{3.338946in}}%
\pgfpathlineto{\pgfqpoint{2.747927in}{3.334688in}}%
\pgfpathmoveto{\pgfqpoint{2.743669in}{3.338946in}}%
\pgfpathlineto{\pgfqpoint{2.743669in}{3.338946in}}%
\pgfpathlineto{\pgfqpoint{2.743669in}{3.343204in}}%
\pgfpathlineto{\pgfqpoint{2.747927in}{3.343204in}}%
\pgfpathlineto{\pgfqpoint{2.747927in}{3.338946in}}%
\pgfpathmoveto{\pgfqpoint{2.747927in}{3.334688in}}%
\pgfpathlineto{\pgfqpoint{2.747927in}{3.334688in}}%
\pgfpathlineto{\pgfqpoint{2.747927in}{3.338946in}}%
\pgfpathlineto{\pgfqpoint{2.752185in}{3.338946in}}%
\pgfpathlineto{\pgfqpoint{2.752185in}{3.334688in}}%
\pgfpathmoveto{\pgfqpoint{2.747927in}{3.338946in}}%
\pgfpathlineto{\pgfqpoint{2.747927in}{3.338946in}}%
\pgfpathlineto{\pgfqpoint{2.747927in}{3.343204in}}%
\pgfpathlineto{\pgfqpoint{2.752185in}{3.343204in}}%
\pgfpathlineto{\pgfqpoint{2.752185in}{3.338946in}}%
\pgfpathmoveto{\pgfqpoint{2.743669in}{3.343204in}}%
\pgfpathlineto{\pgfqpoint{2.743669in}{3.343204in}}%
\pgfpathlineto{\pgfqpoint{2.743669in}{3.347462in}}%
\pgfpathlineto{\pgfqpoint{2.747927in}{3.347462in}}%
\pgfpathlineto{\pgfqpoint{2.747927in}{3.343204in}}%
\pgfpathmoveto{\pgfqpoint{2.743669in}{3.347462in}}%
\pgfpathlineto{\pgfqpoint{2.743669in}{3.347462in}}%
\pgfpathlineto{\pgfqpoint{2.743669in}{3.351720in}}%
\pgfpathlineto{\pgfqpoint{2.747927in}{3.351720in}}%
\pgfpathlineto{\pgfqpoint{2.747927in}{3.347462in}}%
\pgfpathmoveto{\pgfqpoint{2.747927in}{3.343204in}}%
\pgfpathlineto{\pgfqpoint{2.747927in}{3.343204in}}%
\pgfpathlineto{\pgfqpoint{2.747927in}{3.347462in}}%
\pgfpathlineto{\pgfqpoint{2.752185in}{3.347462in}}%
\pgfpathlineto{\pgfqpoint{2.752185in}{3.343204in}}%
\pgfpathmoveto{\pgfqpoint{2.747927in}{3.347462in}}%
\pgfpathlineto{\pgfqpoint{2.747927in}{3.347462in}}%
\pgfpathlineto{\pgfqpoint{2.747927in}{3.351720in}}%
\pgfpathlineto{\pgfqpoint{2.752185in}{3.351720in}}%
\pgfpathlineto{\pgfqpoint{2.752185in}{3.347462in}}%
\pgfpathmoveto{\pgfqpoint{2.739411in}{3.351720in}}%
\pgfpathlineto{\pgfqpoint{2.739411in}{3.351720in}}%
\pgfpathlineto{\pgfqpoint{2.739411in}{3.355978in}}%
\pgfpathlineto{\pgfqpoint{2.743669in}{3.355978in}}%
\pgfpathlineto{\pgfqpoint{2.743669in}{3.351720in}}%
\pgfpathmoveto{\pgfqpoint{2.743669in}{3.351720in}}%
\pgfpathlineto{\pgfqpoint{2.743669in}{3.351720in}}%
\pgfpathlineto{\pgfqpoint{2.743669in}{3.355978in}}%
\pgfpathlineto{\pgfqpoint{2.747927in}{3.355978in}}%
\pgfpathlineto{\pgfqpoint{2.747927in}{3.351720in}}%
\pgfpathmoveto{\pgfqpoint{2.743669in}{3.355978in}}%
\pgfpathlineto{\pgfqpoint{2.743669in}{3.355978in}}%
\pgfpathlineto{\pgfqpoint{2.743669in}{3.360236in}}%
\pgfpathlineto{\pgfqpoint{2.747927in}{3.360236in}}%
\pgfpathlineto{\pgfqpoint{2.747927in}{3.355978in}}%
\pgfpathmoveto{\pgfqpoint{2.747927in}{3.351720in}}%
\pgfpathlineto{\pgfqpoint{2.747927in}{3.351720in}}%
\pgfpathlineto{\pgfqpoint{2.747927in}{3.355978in}}%
\pgfpathlineto{\pgfqpoint{2.752185in}{3.355978in}}%
\pgfpathlineto{\pgfqpoint{2.752185in}{3.351720in}}%
\pgfpathmoveto{\pgfqpoint{2.747927in}{3.355978in}}%
\pgfpathlineto{\pgfqpoint{2.747927in}{3.355978in}}%
\pgfpathlineto{\pgfqpoint{2.747927in}{3.360236in}}%
\pgfpathlineto{\pgfqpoint{2.752185in}{3.360236in}}%
\pgfpathlineto{\pgfqpoint{2.752185in}{3.355978in}}%
\pgfpathmoveto{\pgfqpoint{2.743669in}{3.360236in}}%
\pgfpathlineto{\pgfqpoint{2.743669in}{3.360236in}}%
\pgfpathlineto{\pgfqpoint{2.743669in}{3.364493in}}%
\pgfpathlineto{\pgfqpoint{2.747927in}{3.364493in}}%
\pgfpathlineto{\pgfqpoint{2.747927in}{3.360236in}}%
\pgfpathmoveto{\pgfqpoint{2.743669in}{3.364493in}}%
\pgfpathlineto{\pgfqpoint{2.743669in}{3.364493in}}%
\pgfpathlineto{\pgfqpoint{2.743669in}{3.368751in}}%
\pgfpathlineto{\pgfqpoint{2.747927in}{3.368751in}}%
\pgfpathlineto{\pgfqpoint{2.747927in}{3.364493in}}%
\pgfpathmoveto{\pgfqpoint{2.747927in}{3.360236in}}%
\pgfpathlineto{\pgfqpoint{2.747927in}{3.360236in}}%
\pgfpathlineto{\pgfqpoint{2.747927in}{3.364493in}}%
\pgfpathlineto{\pgfqpoint{2.752185in}{3.364493in}}%
\pgfpathlineto{\pgfqpoint{2.752185in}{3.360236in}}%
\pgfpathmoveto{\pgfqpoint{2.747927in}{3.364493in}}%
\pgfpathlineto{\pgfqpoint{2.747927in}{3.364493in}}%
\pgfpathlineto{\pgfqpoint{2.747927in}{3.368751in}}%
\pgfpathlineto{\pgfqpoint{2.752185in}{3.368751in}}%
\pgfpathlineto{\pgfqpoint{2.752185in}{3.364493in}}%
\pgfpathmoveto{\pgfqpoint{2.752185in}{3.343204in}}%
\pgfpathlineto{\pgfqpoint{2.752185in}{3.343204in}}%
\pgfpathlineto{\pgfqpoint{2.752185in}{3.347462in}}%
\pgfpathlineto{\pgfqpoint{2.756443in}{3.347462in}}%
\pgfpathlineto{\pgfqpoint{2.756443in}{3.343204in}}%
\pgfpathmoveto{\pgfqpoint{2.752185in}{3.347462in}}%
\pgfpathlineto{\pgfqpoint{2.752185in}{3.347462in}}%
\pgfpathlineto{\pgfqpoint{2.752185in}{3.351720in}}%
\pgfpathlineto{\pgfqpoint{2.756443in}{3.351720in}}%
\pgfpathlineto{\pgfqpoint{2.756443in}{3.347462in}}%
\pgfpathmoveto{\pgfqpoint{2.752185in}{3.351720in}}%
\pgfpathlineto{\pgfqpoint{2.752185in}{3.351720in}}%
\pgfpathlineto{\pgfqpoint{2.752185in}{3.355978in}}%
\pgfpathlineto{\pgfqpoint{2.756443in}{3.355978in}}%
\pgfpathlineto{\pgfqpoint{2.756443in}{3.351720in}}%
\pgfpathmoveto{\pgfqpoint{2.752185in}{3.355978in}}%
\pgfpathlineto{\pgfqpoint{2.752185in}{3.355978in}}%
\pgfpathlineto{\pgfqpoint{2.752185in}{3.360236in}}%
\pgfpathlineto{\pgfqpoint{2.756443in}{3.360236in}}%
\pgfpathlineto{\pgfqpoint{2.756443in}{3.355978in}}%
\pgfpathmoveto{\pgfqpoint{2.752185in}{3.360236in}}%
\pgfpathlineto{\pgfqpoint{2.752185in}{3.360236in}}%
\pgfpathlineto{\pgfqpoint{2.752185in}{3.364493in}}%
\pgfpathlineto{\pgfqpoint{2.756443in}{3.364493in}}%
\pgfpathlineto{\pgfqpoint{2.756443in}{3.360236in}}%
\pgfpathmoveto{\pgfqpoint{2.752185in}{3.364493in}}%
\pgfpathlineto{\pgfqpoint{2.752185in}{3.364493in}}%
\pgfpathlineto{\pgfqpoint{2.752185in}{3.368751in}}%
\pgfpathlineto{\pgfqpoint{2.756443in}{3.368751in}}%
\pgfpathlineto{\pgfqpoint{2.756443in}{3.364493in}}%
\pgfpathmoveto{\pgfqpoint{2.756443in}{3.360236in}}%
\pgfpathlineto{\pgfqpoint{2.756443in}{3.360236in}}%
\pgfpathlineto{\pgfqpoint{2.756443in}{3.364493in}}%
\pgfpathlineto{\pgfqpoint{2.760701in}{3.364493in}}%
\pgfpathlineto{\pgfqpoint{2.760701in}{3.360236in}}%
\pgfpathmoveto{\pgfqpoint{2.756443in}{3.364493in}}%
\pgfpathlineto{\pgfqpoint{2.756443in}{3.364493in}}%
\pgfpathlineto{\pgfqpoint{2.756443in}{3.368751in}}%
\pgfpathlineto{\pgfqpoint{2.760701in}{3.368751in}}%
\pgfpathlineto{\pgfqpoint{2.760701in}{3.364493in}}%
\pgfpathmoveto{\pgfqpoint{2.743669in}{3.368751in}}%
\pgfpathlineto{\pgfqpoint{2.743669in}{3.368751in}}%
\pgfpathlineto{\pgfqpoint{2.743669in}{3.373009in}}%
\pgfpathlineto{\pgfqpoint{2.747927in}{3.373009in}}%
\pgfpathlineto{\pgfqpoint{2.747927in}{3.368751in}}%
\pgfpathmoveto{\pgfqpoint{2.747927in}{3.368751in}}%
\pgfpathlineto{\pgfqpoint{2.747927in}{3.368751in}}%
\pgfpathlineto{\pgfqpoint{2.747927in}{3.373009in}}%
\pgfpathlineto{\pgfqpoint{2.752185in}{3.373009in}}%
\pgfpathlineto{\pgfqpoint{2.752185in}{3.368751in}}%
\pgfpathmoveto{\pgfqpoint{2.747927in}{3.373009in}}%
\pgfpathlineto{\pgfqpoint{2.747927in}{3.373009in}}%
\pgfpathlineto{\pgfqpoint{2.747927in}{3.377267in}}%
\pgfpathlineto{\pgfqpoint{2.752185in}{3.377267in}}%
\pgfpathlineto{\pgfqpoint{2.752185in}{3.373009in}}%
\pgfpathmoveto{\pgfqpoint{2.747927in}{3.377267in}}%
\pgfpathlineto{\pgfqpoint{2.747927in}{3.377267in}}%
\pgfpathlineto{\pgfqpoint{2.747927in}{3.381525in}}%
\pgfpathlineto{\pgfqpoint{2.752185in}{3.381525in}}%
\pgfpathlineto{\pgfqpoint{2.752185in}{3.377267in}}%
\pgfpathmoveto{\pgfqpoint{2.747927in}{3.381525in}}%
\pgfpathlineto{\pgfqpoint{2.747927in}{3.381525in}}%
\pgfpathlineto{\pgfqpoint{2.747927in}{3.385783in}}%
\pgfpathlineto{\pgfqpoint{2.752185in}{3.385783in}}%
\pgfpathlineto{\pgfqpoint{2.752185in}{3.381525in}}%
\pgfpathmoveto{\pgfqpoint{2.747927in}{3.385783in}}%
\pgfpathlineto{\pgfqpoint{2.747927in}{3.385783in}}%
\pgfpathlineto{\pgfqpoint{2.747927in}{3.390041in}}%
\pgfpathlineto{\pgfqpoint{2.752185in}{3.390041in}}%
\pgfpathlineto{\pgfqpoint{2.752185in}{3.385783in}}%
\pgfpathmoveto{\pgfqpoint{2.752185in}{3.368751in}}%
\pgfpathlineto{\pgfqpoint{2.752185in}{3.368751in}}%
\pgfpathlineto{\pgfqpoint{2.752185in}{3.373009in}}%
\pgfpathlineto{\pgfqpoint{2.756443in}{3.373009in}}%
\pgfpathlineto{\pgfqpoint{2.756443in}{3.368751in}}%
\pgfpathmoveto{\pgfqpoint{2.752185in}{3.373009in}}%
\pgfpathlineto{\pgfqpoint{2.752185in}{3.373009in}}%
\pgfpathlineto{\pgfqpoint{2.752185in}{3.377267in}}%
\pgfpathlineto{\pgfqpoint{2.756443in}{3.377267in}}%
\pgfpathlineto{\pgfqpoint{2.756443in}{3.373009in}}%
\pgfpathmoveto{\pgfqpoint{2.756443in}{3.368751in}}%
\pgfpathlineto{\pgfqpoint{2.756443in}{3.368751in}}%
\pgfpathlineto{\pgfqpoint{2.756443in}{3.373009in}}%
\pgfpathlineto{\pgfqpoint{2.760701in}{3.373009in}}%
\pgfpathlineto{\pgfqpoint{2.760701in}{3.368751in}}%
\pgfpathmoveto{\pgfqpoint{2.756443in}{3.373009in}}%
\pgfpathlineto{\pgfqpoint{2.756443in}{3.373009in}}%
\pgfpathlineto{\pgfqpoint{2.756443in}{3.377267in}}%
\pgfpathlineto{\pgfqpoint{2.760701in}{3.377267in}}%
\pgfpathlineto{\pgfqpoint{2.760701in}{3.373009in}}%
\pgfpathmoveto{\pgfqpoint{2.752185in}{3.377267in}}%
\pgfpathlineto{\pgfqpoint{2.752185in}{3.377267in}}%
\pgfpathlineto{\pgfqpoint{2.752185in}{3.381525in}}%
\pgfpathlineto{\pgfqpoint{2.756443in}{3.381525in}}%
\pgfpathlineto{\pgfqpoint{2.756443in}{3.377267in}}%
\pgfpathmoveto{\pgfqpoint{2.752185in}{3.381525in}}%
\pgfpathlineto{\pgfqpoint{2.752185in}{3.381525in}}%
\pgfpathlineto{\pgfqpoint{2.752185in}{3.385783in}}%
\pgfpathlineto{\pgfqpoint{2.756443in}{3.385783in}}%
\pgfpathlineto{\pgfqpoint{2.756443in}{3.381525in}}%
\pgfpathmoveto{\pgfqpoint{2.756443in}{3.377267in}}%
\pgfpathlineto{\pgfqpoint{2.756443in}{3.377267in}}%
\pgfpathlineto{\pgfqpoint{2.756443in}{3.381525in}}%
\pgfpathlineto{\pgfqpoint{2.760701in}{3.381525in}}%
\pgfpathlineto{\pgfqpoint{2.760701in}{3.377267in}}%
\pgfpathmoveto{\pgfqpoint{2.756443in}{3.381525in}}%
\pgfpathlineto{\pgfqpoint{2.756443in}{3.381525in}}%
\pgfpathlineto{\pgfqpoint{2.756443in}{3.385783in}}%
\pgfpathlineto{\pgfqpoint{2.760701in}{3.385783in}}%
\pgfpathlineto{\pgfqpoint{2.760701in}{3.381525in}}%
\pgfpathmoveto{\pgfqpoint{2.760701in}{3.373009in}}%
\pgfpathlineto{\pgfqpoint{2.760701in}{3.373009in}}%
\pgfpathlineto{\pgfqpoint{2.760701in}{3.377267in}}%
\pgfpathlineto{\pgfqpoint{2.764959in}{3.377267in}}%
\pgfpathlineto{\pgfqpoint{2.764959in}{3.373009in}}%
\pgfpathmoveto{\pgfqpoint{2.760701in}{3.377267in}}%
\pgfpathlineto{\pgfqpoint{2.760701in}{3.377267in}}%
\pgfpathlineto{\pgfqpoint{2.760701in}{3.381525in}}%
\pgfpathlineto{\pgfqpoint{2.764959in}{3.381525in}}%
\pgfpathlineto{\pgfqpoint{2.764959in}{3.377267in}}%
\pgfpathmoveto{\pgfqpoint{2.760701in}{3.381525in}}%
\pgfpathlineto{\pgfqpoint{2.760701in}{3.381525in}}%
\pgfpathlineto{\pgfqpoint{2.760701in}{3.385783in}}%
\pgfpathlineto{\pgfqpoint{2.764959in}{3.385783in}}%
\pgfpathlineto{\pgfqpoint{2.764959in}{3.381525in}}%
\pgfpathmoveto{\pgfqpoint{2.752185in}{3.385783in}}%
\pgfpathlineto{\pgfqpoint{2.752185in}{3.385783in}}%
\pgfpathlineto{\pgfqpoint{2.752185in}{3.390041in}}%
\pgfpathlineto{\pgfqpoint{2.756443in}{3.390041in}}%
\pgfpathlineto{\pgfqpoint{2.756443in}{3.385783in}}%
\pgfpathmoveto{\pgfqpoint{2.752185in}{3.390041in}}%
\pgfpathlineto{\pgfqpoint{2.752185in}{3.390041in}}%
\pgfpathlineto{\pgfqpoint{2.752185in}{3.394299in}}%
\pgfpathlineto{\pgfqpoint{2.756443in}{3.394299in}}%
\pgfpathlineto{\pgfqpoint{2.756443in}{3.390041in}}%
\pgfpathmoveto{\pgfqpoint{2.756443in}{3.385783in}}%
\pgfpathlineto{\pgfqpoint{2.756443in}{3.385783in}}%
\pgfpathlineto{\pgfqpoint{2.756443in}{3.390041in}}%
\pgfpathlineto{\pgfqpoint{2.760701in}{3.390041in}}%
\pgfpathlineto{\pgfqpoint{2.760701in}{3.385783in}}%
\pgfpathmoveto{\pgfqpoint{2.756443in}{3.390041in}}%
\pgfpathlineto{\pgfqpoint{2.756443in}{3.390041in}}%
\pgfpathlineto{\pgfqpoint{2.756443in}{3.394299in}}%
\pgfpathlineto{\pgfqpoint{2.760701in}{3.394299in}}%
\pgfpathlineto{\pgfqpoint{2.760701in}{3.390041in}}%
\pgfpathmoveto{\pgfqpoint{2.752185in}{3.394299in}}%
\pgfpathlineto{\pgfqpoint{2.752185in}{3.394299in}}%
\pgfpathlineto{\pgfqpoint{2.752185in}{3.398557in}}%
\pgfpathlineto{\pgfqpoint{2.756443in}{3.398557in}}%
\pgfpathlineto{\pgfqpoint{2.756443in}{3.394299in}}%
\pgfpathmoveto{\pgfqpoint{2.752185in}{3.398557in}}%
\pgfpathlineto{\pgfqpoint{2.752185in}{3.398557in}}%
\pgfpathlineto{\pgfqpoint{2.752185in}{3.402814in}}%
\pgfpathlineto{\pgfqpoint{2.756443in}{3.402814in}}%
\pgfpathlineto{\pgfqpoint{2.756443in}{3.398557in}}%
\pgfpathmoveto{\pgfqpoint{2.756443in}{3.394299in}}%
\pgfpathlineto{\pgfqpoint{2.756443in}{3.394299in}}%
\pgfpathlineto{\pgfqpoint{2.756443in}{3.398557in}}%
\pgfpathlineto{\pgfqpoint{2.760701in}{3.398557in}}%
\pgfpathlineto{\pgfqpoint{2.760701in}{3.394299in}}%
\pgfpathmoveto{\pgfqpoint{2.756443in}{3.398557in}}%
\pgfpathlineto{\pgfqpoint{2.756443in}{3.398557in}}%
\pgfpathlineto{\pgfqpoint{2.756443in}{3.402814in}}%
\pgfpathlineto{\pgfqpoint{2.760701in}{3.402814in}}%
\pgfpathlineto{\pgfqpoint{2.760701in}{3.398557in}}%
\pgfpathmoveto{\pgfqpoint{2.760701in}{3.385783in}}%
\pgfpathlineto{\pgfqpoint{2.760701in}{3.385783in}}%
\pgfpathlineto{\pgfqpoint{2.760701in}{3.390041in}}%
\pgfpathlineto{\pgfqpoint{2.764959in}{3.390041in}}%
\pgfpathlineto{\pgfqpoint{2.764959in}{3.385783in}}%
\pgfpathmoveto{\pgfqpoint{2.760701in}{3.390041in}}%
\pgfpathlineto{\pgfqpoint{2.760701in}{3.390041in}}%
\pgfpathlineto{\pgfqpoint{2.760701in}{3.394299in}}%
\pgfpathlineto{\pgfqpoint{2.764959in}{3.394299in}}%
\pgfpathlineto{\pgfqpoint{2.764959in}{3.390041in}}%
\pgfpathmoveto{\pgfqpoint{2.764959in}{3.390041in}}%
\pgfpathlineto{\pgfqpoint{2.764959in}{3.390041in}}%
\pgfpathlineto{\pgfqpoint{2.764959in}{3.394299in}}%
\pgfpathlineto{\pgfqpoint{2.769217in}{3.394299in}}%
\pgfpathlineto{\pgfqpoint{2.769217in}{3.390041in}}%
\pgfpathmoveto{\pgfqpoint{2.760701in}{3.394299in}}%
\pgfpathlineto{\pgfqpoint{2.760701in}{3.394299in}}%
\pgfpathlineto{\pgfqpoint{2.760701in}{3.398557in}}%
\pgfpathlineto{\pgfqpoint{2.764959in}{3.398557in}}%
\pgfpathlineto{\pgfqpoint{2.764959in}{3.394299in}}%
\pgfpathmoveto{\pgfqpoint{2.760701in}{3.398557in}}%
\pgfpathlineto{\pgfqpoint{2.760701in}{3.398557in}}%
\pgfpathlineto{\pgfqpoint{2.760701in}{3.402814in}}%
\pgfpathlineto{\pgfqpoint{2.764959in}{3.402814in}}%
\pgfpathlineto{\pgfqpoint{2.764959in}{3.398557in}}%
\pgfpathmoveto{\pgfqpoint{2.764959in}{3.394299in}}%
\pgfpathlineto{\pgfqpoint{2.764959in}{3.394299in}}%
\pgfpathlineto{\pgfqpoint{2.764959in}{3.398557in}}%
\pgfpathlineto{\pgfqpoint{2.769217in}{3.398557in}}%
\pgfpathlineto{\pgfqpoint{2.769217in}{3.394299in}}%
\pgfpathmoveto{\pgfqpoint{2.764959in}{3.398557in}}%
\pgfpathlineto{\pgfqpoint{2.764959in}{3.398557in}}%
\pgfpathlineto{\pgfqpoint{2.764959in}{3.402814in}}%
\pgfpathlineto{\pgfqpoint{2.769217in}{3.402814in}}%
\pgfpathlineto{\pgfqpoint{2.769217in}{3.398557in}}%
\pgfpathmoveto{\pgfqpoint{2.752185in}{3.402814in}}%
\pgfpathlineto{\pgfqpoint{2.752185in}{3.402814in}}%
\pgfpathlineto{\pgfqpoint{2.752185in}{3.407072in}}%
\pgfpathlineto{\pgfqpoint{2.756443in}{3.407072in}}%
\pgfpathlineto{\pgfqpoint{2.756443in}{3.402814in}}%
\pgfpathmoveto{\pgfqpoint{2.756443in}{3.402814in}}%
\pgfpathlineto{\pgfqpoint{2.756443in}{3.402814in}}%
\pgfpathlineto{\pgfqpoint{2.756443in}{3.407072in}}%
\pgfpathlineto{\pgfqpoint{2.760701in}{3.407072in}}%
\pgfpathlineto{\pgfqpoint{2.760701in}{3.402814in}}%
\pgfpathmoveto{\pgfqpoint{2.756443in}{3.407072in}}%
\pgfpathlineto{\pgfqpoint{2.756443in}{3.407072in}}%
\pgfpathlineto{\pgfqpoint{2.756443in}{3.411330in}}%
\pgfpathlineto{\pgfqpoint{2.760701in}{3.411330in}}%
\pgfpathlineto{\pgfqpoint{2.760701in}{3.407072in}}%
\pgfpathmoveto{\pgfqpoint{2.756443in}{3.411330in}}%
\pgfpathlineto{\pgfqpoint{2.756443in}{3.411330in}}%
\pgfpathlineto{\pgfqpoint{2.756443in}{3.415588in}}%
\pgfpathlineto{\pgfqpoint{2.760701in}{3.415588in}}%
\pgfpathlineto{\pgfqpoint{2.760701in}{3.411330in}}%
\pgfpathmoveto{\pgfqpoint{2.756443in}{3.415588in}}%
\pgfpathlineto{\pgfqpoint{2.756443in}{3.415588in}}%
\pgfpathlineto{\pgfqpoint{2.756443in}{3.419846in}}%
\pgfpathlineto{\pgfqpoint{2.760701in}{3.419846in}}%
\pgfpathlineto{\pgfqpoint{2.760701in}{3.415588in}}%
\pgfpathmoveto{\pgfqpoint{2.760701in}{3.402814in}}%
\pgfpathlineto{\pgfqpoint{2.760701in}{3.402814in}}%
\pgfpathlineto{\pgfqpoint{2.760701in}{3.407072in}}%
\pgfpathlineto{\pgfqpoint{2.764959in}{3.407072in}}%
\pgfpathlineto{\pgfqpoint{2.764959in}{3.402814in}}%
\pgfpathmoveto{\pgfqpoint{2.760701in}{3.407072in}}%
\pgfpathlineto{\pgfqpoint{2.760701in}{3.407072in}}%
\pgfpathlineto{\pgfqpoint{2.760701in}{3.411330in}}%
\pgfpathlineto{\pgfqpoint{2.764959in}{3.411330in}}%
\pgfpathlineto{\pgfqpoint{2.764959in}{3.407072in}}%
\pgfpathmoveto{\pgfqpoint{2.764959in}{3.402814in}}%
\pgfpathlineto{\pgfqpoint{2.764959in}{3.402814in}}%
\pgfpathlineto{\pgfqpoint{2.764959in}{3.407072in}}%
\pgfpathlineto{\pgfqpoint{2.769217in}{3.407072in}}%
\pgfpathlineto{\pgfqpoint{2.769217in}{3.402814in}}%
\pgfpathmoveto{\pgfqpoint{2.764959in}{3.407072in}}%
\pgfpathlineto{\pgfqpoint{2.764959in}{3.407072in}}%
\pgfpathlineto{\pgfqpoint{2.764959in}{3.411330in}}%
\pgfpathlineto{\pgfqpoint{2.769217in}{3.411330in}}%
\pgfpathlineto{\pgfqpoint{2.769217in}{3.407072in}}%
\pgfpathmoveto{\pgfqpoint{2.760701in}{3.411330in}}%
\pgfpathlineto{\pgfqpoint{2.760701in}{3.411330in}}%
\pgfpathlineto{\pgfqpoint{2.760701in}{3.415588in}}%
\pgfpathlineto{\pgfqpoint{2.764959in}{3.415588in}}%
\pgfpathlineto{\pgfqpoint{2.764959in}{3.411330in}}%
\pgfpathmoveto{\pgfqpoint{2.760701in}{3.415588in}}%
\pgfpathlineto{\pgfqpoint{2.760701in}{3.415588in}}%
\pgfpathlineto{\pgfqpoint{2.760701in}{3.419846in}}%
\pgfpathlineto{\pgfqpoint{2.764959in}{3.419846in}}%
\pgfpathlineto{\pgfqpoint{2.764959in}{3.415588in}}%
\pgfpathmoveto{\pgfqpoint{2.764959in}{3.411330in}}%
\pgfpathlineto{\pgfqpoint{2.764959in}{3.411330in}}%
\pgfpathlineto{\pgfqpoint{2.764959in}{3.415588in}}%
\pgfpathlineto{\pgfqpoint{2.769217in}{3.415588in}}%
\pgfpathlineto{\pgfqpoint{2.769217in}{3.411330in}}%
\pgfpathmoveto{\pgfqpoint{2.764959in}{3.415588in}}%
\pgfpathlineto{\pgfqpoint{2.764959in}{3.415588in}}%
\pgfpathlineto{\pgfqpoint{2.764959in}{3.419846in}}%
\pgfpathlineto{\pgfqpoint{2.769217in}{3.419846in}}%
\pgfpathlineto{\pgfqpoint{2.769217in}{3.415588in}}%
\pgfpathmoveto{\pgfqpoint{2.760701in}{3.419846in}}%
\pgfpathlineto{\pgfqpoint{2.760701in}{3.419846in}}%
\pgfpathlineto{\pgfqpoint{2.760701in}{3.424104in}}%
\pgfpathlineto{\pgfqpoint{2.764959in}{3.424104in}}%
\pgfpathlineto{\pgfqpoint{2.764959in}{3.419846in}}%
\pgfpathmoveto{\pgfqpoint{2.760701in}{3.424104in}}%
\pgfpathlineto{\pgfqpoint{2.760701in}{3.424104in}}%
\pgfpathlineto{\pgfqpoint{2.760701in}{3.428362in}}%
\pgfpathlineto{\pgfqpoint{2.764959in}{3.428362in}}%
\pgfpathlineto{\pgfqpoint{2.764959in}{3.424104in}}%
\pgfpathmoveto{\pgfqpoint{2.764959in}{3.419846in}}%
\pgfpathlineto{\pgfqpoint{2.764959in}{3.419846in}}%
\pgfpathlineto{\pgfqpoint{2.764959in}{3.424104in}}%
\pgfpathlineto{\pgfqpoint{2.769217in}{3.424104in}}%
\pgfpathlineto{\pgfqpoint{2.769217in}{3.419846in}}%
\pgfpathmoveto{\pgfqpoint{2.764959in}{3.424104in}}%
\pgfpathlineto{\pgfqpoint{2.764959in}{3.424104in}}%
\pgfpathlineto{\pgfqpoint{2.764959in}{3.428362in}}%
\pgfpathlineto{\pgfqpoint{2.769217in}{3.428362in}}%
\pgfpathlineto{\pgfqpoint{2.769217in}{3.424104in}}%
\pgfpathmoveto{\pgfqpoint{2.760701in}{3.428362in}}%
\pgfpathlineto{\pgfqpoint{2.760701in}{3.428362in}}%
\pgfpathlineto{\pgfqpoint{2.760701in}{3.432619in}}%
\pgfpathlineto{\pgfqpoint{2.764959in}{3.432619in}}%
\pgfpathlineto{\pgfqpoint{2.764959in}{3.428362in}}%
\pgfpathmoveto{\pgfqpoint{2.760701in}{3.432619in}}%
\pgfpathlineto{\pgfqpoint{2.760701in}{3.432619in}}%
\pgfpathlineto{\pgfqpoint{2.760701in}{3.436877in}}%
\pgfpathlineto{\pgfqpoint{2.764959in}{3.436877in}}%
\pgfpathlineto{\pgfqpoint{2.764959in}{3.432619in}}%
\pgfpathmoveto{\pgfqpoint{2.764959in}{3.428362in}}%
\pgfpathlineto{\pgfqpoint{2.764959in}{3.428362in}}%
\pgfpathlineto{\pgfqpoint{2.764959in}{3.432619in}}%
\pgfpathlineto{\pgfqpoint{2.769217in}{3.432619in}}%
\pgfpathlineto{\pgfqpoint{2.769217in}{3.428362in}}%
\pgfpathmoveto{\pgfqpoint{2.764959in}{3.432619in}}%
\pgfpathlineto{\pgfqpoint{2.764959in}{3.432619in}}%
\pgfpathlineto{\pgfqpoint{2.764959in}{3.436877in}}%
\pgfpathlineto{\pgfqpoint{2.769217in}{3.436877in}}%
\pgfpathlineto{\pgfqpoint{2.769217in}{3.432619in}}%
\pgfpathmoveto{\pgfqpoint{2.769217in}{3.402814in}}%
\pgfpathlineto{\pgfqpoint{2.769217in}{3.402814in}}%
\pgfpathlineto{\pgfqpoint{2.769217in}{3.407072in}}%
\pgfpathlineto{\pgfqpoint{2.773474in}{3.407072in}}%
\pgfpathlineto{\pgfqpoint{2.773474in}{3.402814in}}%
\pgfpathmoveto{\pgfqpoint{2.769217in}{3.407072in}}%
\pgfpathlineto{\pgfqpoint{2.769217in}{3.407072in}}%
\pgfpathlineto{\pgfqpoint{2.769217in}{3.411330in}}%
\pgfpathlineto{\pgfqpoint{2.773474in}{3.411330in}}%
\pgfpathlineto{\pgfqpoint{2.773474in}{3.407072in}}%
\pgfpathmoveto{\pgfqpoint{2.769217in}{3.411330in}}%
\pgfpathlineto{\pgfqpoint{2.769217in}{3.411330in}}%
\pgfpathlineto{\pgfqpoint{2.769217in}{3.415588in}}%
\pgfpathlineto{\pgfqpoint{2.773474in}{3.415588in}}%
\pgfpathlineto{\pgfqpoint{2.773474in}{3.411330in}}%
\pgfpathmoveto{\pgfqpoint{2.769217in}{3.415588in}}%
\pgfpathlineto{\pgfqpoint{2.769217in}{3.415588in}}%
\pgfpathlineto{\pgfqpoint{2.769217in}{3.419846in}}%
\pgfpathlineto{\pgfqpoint{2.773474in}{3.419846in}}%
\pgfpathlineto{\pgfqpoint{2.773474in}{3.415588in}}%
\pgfpathmoveto{\pgfqpoint{2.769217in}{3.419846in}}%
\pgfpathlineto{\pgfqpoint{2.769217in}{3.419846in}}%
\pgfpathlineto{\pgfqpoint{2.769217in}{3.424104in}}%
\pgfpathlineto{\pgfqpoint{2.773474in}{3.424104in}}%
\pgfpathlineto{\pgfqpoint{2.773474in}{3.419846in}}%
\pgfpathmoveto{\pgfqpoint{2.769217in}{3.424104in}}%
\pgfpathlineto{\pgfqpoint{2.769217in}{3.424104in}}%
\pgfpathlineto{\pgfqpoint{2.769217in}{3.428362in}}%
\pgfpathlineto{\pgfqpoint{2.773474in}{3.428362in}}%
\pgfpathlineto{\pgfqpoint{2.773474in}{3.424104in}}%
\pgfpathmoveto{\pgfqpoint{2.773474in}{3.419846in}}%
\pgfpathlineto{\pgfqpoint{2.773474in}{3.419846in}}%
\pgfpathlineto{\pgfqpoint{2.773474in}{3.424104in}}%
\pgfpathlineto{\pgfqpoint{2.777732in}{3.424104in}}%
\pgfpathlineto{\pgfqpoint{2.777732in}{3.419846in}}%
\pgfpathmoveto{\pgfqpoint{2.773474in}{3.424104in}}%
\pgfpathlineto{\pgfqpoint{2.773474in}{3.424104in}}%
\pgfpathlineto{\pgfqpoint{2.773474in}{3.428362in}}%
\pgfpathlineto{\pgfqpoint{2.777732in}{3.428362in}}%
\pgfpathlineto{\pgfqpoint{2.777732in}{3.424104in}}%
\pgfpathmoveto{\pgfqpoint{2.769217in}{3.428362in}}%
\pgfpathlineto{\pgfqpoint{2.769217in}{3.428362in}}%
\pgfpathlineto{\pgfqpoint{2.769217in}{3.432619in}}%
\pgfpathlineto{\pgfqpoint{2.773474in}{3.432619in}}%
\pgfpathlineto{\pgfqpoint{2.773474in}{3.428362in}}%
\pgfpathmoveto{\pgfqpoint{2.769217in}{3.432619in}}%
\pgfpathlineto{\pgfqpoint{2.769217in}{3.432619in}}%
\pgfpathlineto{\pgfqpoint{2.769217in}{3.436877in}}%
\pgfpathlineto{\pgfqpoint{2.773474in}{3.436877in}}%
\pgfpathlineto{\pgfqpoint{2.773474in}{3.432619in}}%
\pgfpathmoveto{\pgfqpoint{2.773474in}{3.428362in}}%
\pgfpathlineto{\pgfqpoint{2.773474in}{3.428362in}}%
\pgfpathlineto{\pgfqpoint{2.773474in}{3.432619in}}%
\pgfpathlineto{\pgfqpoint{2.777732in}{3.432619in}}%
\pgfpathlineto{\pgfqpoint{2.777732in}{3.428362in}}%
\pgfpathmoveto{\pgfqpoint{2.773474in}{3.432619in}}%
\pgfpathlineto{\pgfqpoint{2.773474in}{3.432619in}}%
\pgfpathlineto{\pgfqpoint{2.773474in}{3.436877in}}%
\pgfpathlineto{\pgfqpoint{2.777732in}{3.436877in}}%
\pgfpathlineto{\pgfqpoint{2.777732in}{3.432619in}}%
\pgfpathmoveto{\pgfqpoint{2.777732in}{3.432619in}}%
\pgfpathlineto{\pgfqpoint{2.777732in}{3.432619in}}%
\pgfpathlineto{\pgfqpoint{2.777732in}{3.436877in}}%
\pgfpathlineto{\pgfqpoint{2.781990in}{3.436877in}}%
\pgfpathlineto{\pgfqpoint{2.781990in}{3.432619in}}%
\pgfpathmoveto{\pgfqpoint{2.764959in}{3.436877in}}%
\pgfpathlineto{\pgfqpoint{2.764959in}{3.436877in}}%
\pgfpathlineto{\pgfqpoint{2.764959in}{3.441135in}}%
\pgfpathlineto{\pgfqpoint{2.769217in}{3.441135in}}%
\pgfpathlineto{\pgfqpoint{2.769217in}{3.436877in}}%
\pgfpathmoveto{\pgfqpoint{2.764959in}{3.441135in}}%
\pgfpathlineto{\pgfqpoint{2.764959in}{3.441135in}}%
\pgfpathlineto{\pgfqpoint{2.764959in}{3.445393in}}%
\pgfpathlineto{\pgfqpoint{2.769217in}{3.445393in}}%
\pgfpathlineto{\pgfqpoint{2.769217in}{3.441135in}}%
\pgfpathmoveto{\pgfqpoint{2.764959in}{3.445393in}}%
\pgfpathlineto{\pgfqpoint{2.764959in}{3.445393in}}%
\pgfpathlineto{\pgfqpoint{2.764959in}{3.449651in}}%
\pgfpathlineto{\pgfqpoint{2.769217in}{3.449651in}}%
\pgfpathlineto{\pgfqpoint{2.769217in}{3.445393in}}%
\pgfpathmoveto{\pgfqpoint{2.769217in}{3.436877in}}%
\pgfpathlineto{\pgfqpoint{2.769217in}{3.436877in}}%
\pgfpathlineto{\pgfqpoint{2.769217in}{3.441135in}}%
\pgfpathlineto{\pgfqpoint{2.773474in}{3.441135in}}%
\pgfpathlineto{\pgfqpoint{2.773474in}{3.436877in}}%
\pgfpathmoveto{\pgfqpoint{2.769217in}{3.441135in}}%
\pgfpathlineto{\pgfqpoint{2.769217in}{3.441135in}}%
\pgfpathlineto{\pgfqpoint{2.769217in}{3.445393in}}%
\pgfpathlineto{\pgfqpoint{2.773474in}{3.445393in}}%
\pgfpathlineto{\pgfqpoint{2.773474in}{3.441135in}}%
\pgfpathmoveto{\pgfqpoint{2.773474in}{3.436877in}}%
\pgfpathlineto{\pgfqpoint{2.773474in}{3.436877in}}%
\pgfpathlineto{\pgfqpoint{2.773474in}{3.441135in}}%
\pgfpathlineto{\pgfqpoint{2.777732in}{3.441135in}}%
\pgfpathlineto{\pgfqpoint{2.777732in}{3.436877in}}%
\pgfpathmoveto{\pgfqpoint{2.773474in}{3.441135in}}%
\pgfpathlineto{\pgfqpoint{2.773474in}{3.441135in}}%
\pgfpathlineto{\pgfqpoint{2.773474in}{3.445393in}}%
\pgfpathlineto{\pgfqpoint{2.777732in}{3.445393in}}%
\pgfpathlineto{\pgfqpoint{2.777732in}{3.441135in}}%
\pgfpathmoveto{\pgfqpoint{2.769217in}{3.445393in}}%
\pgfpathlineto{\pgfqpoint{2.769217in}{3.445393in}}%
\pgfpathlineto{\pgfqpoint{2.769217in}{3.449651in}}%
\pgfpathlineto{\pgfqpoint{2.773474in}{3.449651in}}%
\pgfpathlineto{\pgfqpoint{2.773474in}{3.445393in}}%
\pgfpathmoveto{\pgfqpoint{2.769217in}{3.449651in}}%
\pgfpathlineto{\pgfqpoint{2.769217in}{3.449651in}}%
\pgfpathlineto{\pgfqpoint{2.769217in}{3.453909in}}%
\pgfpathlineto{\pgfqpoint{2.773474in}{3.453909in}}%
\pgfpathlineto{\pgfqpoint{2.773474in}{3.449651in}}%
\pgfpathmoveto{\pgfqpoint{2.773474in}{3.445393in}}%
\pgfpathlineto{\pgfqpoint{2.773474in}{3.445393in}}%
\pgfpathlineto{\pgfqpoint{2.773474in}{3.449651in}}%
\pgfpathlineto{\pgfqpoint{2.777732in}{3.449651in}}%
\pgfpathlineto{\pgfqpoint{2.777732in}{3.445393in}}%
\pgfpathmoveto{\pgfqpoint{2.773474in}{3.449651in}}%
\pgfpathlineto{\pgfqpoint{2.773474in}{3.449651in}}%
\pgfpathlineto{\pgfqpoint{2.773474in}{3.453909in}}%
\pgfpathlineto{\pgfqpoint{2.777732in}{3.453909in}}%
\pgfpathlineto{\pgfqpoint{2.777732in}{3.449651in}}%
\pgfpathmoveto{\pgfqpoint{2.777732in}{3.436877in}}%
\pgfpathlineto{\pgfqpoint{2.777732in}{3.436877in}}%
\pgfpathlineto{\pgfqpoint{2.777732in}{3.441135in}}%
\pgfpathlineto{\pgfqpoint{2.781990in}{3.441135in}}%
\pgfpathlineto{\pgfqpoint{2.781990in}{3.436877in}}%
\pgfpathmoveto{\pgfqpoint{2.777732in}{3.441135in}}%
\pgfpathlineto{\pgfqpoint{2.777732in}{3.441135in}}%
\pgfpathlineto{\pgfqpoint{2.777732in}{3.445393in}}%
\pgfpathlineto{\pgfqpoint{2.781990in}{3.445393in}}%
\pgfpathlineto{\pgfqpoint{2.781990in}{3.441135in}}%
\pgfpathmoveto{\pgfqpoint{2.777732in}{3.445393in}}%
\pgfpathlineto{\pgfqpoint{2.777732in}{3.445393in}}%
\pgfpathlineto{\pgfqpoint{2.777732in}{3.449651in}}%
\pgfpathlineto{\pgfqpoint{2.781990in}{3.449651in}}%
\pgfpathlineto{\pgfqpoint{2.781990in}{3.445393in}}%
\pgfpathmoveto{\pgfqpoint{2.777732in}{3.449651in}}%
\pgfpathlineto{\pgfqpoint{2.777732in}{3.449651in}}%
\pgfpathlineto{\pgfqpoint{2.777732in}{3.453909in}}%
\pgfpathlineto{\pgfqpoint{2.781990in}{3.453909in}}%
\pgfpathlineto{\pgfqpoint{2.781990in}{3.449651in}}%
\pgfpathmoveto{\pgfqpoint{2.781990in}{3.445393in}}%
\pgfpathlineto{\pgfqpoint{2.781990in}{3.445393in}}%
\pgfpathlineto{\pgfqpoint{2.781990in}{3.449651in}}%
\pgfpathlineto{\pgfqpoint{2.786248in}{3.449651in}}%
\pgfpathlineto{\pgfqpoint{2.786248in}{3.445393in}}%
\pgfpathmoveto{\pgfqpoint{2.781990in}{3.449651in}}%
\pgfpathlineto{\pgfqpoint{2.781990in}{3.449651in}}%
\pgfpathlineto{\pgfqpoint{2.781990in}{3.453909in}}%
\pgfpathlineto{\pgfqpoint{2.786248in}{3.453909in}}%
\pgfpathlineto{\pgfqpoint{2.786248in}{3.449651in}}%
\pgfpathmoveto{\pgfqpoint{2.769217in}{3.453909in}}%
\pgfpathlineto{\pgfqpoint{2.769217in}{3.453909in}}%
\pgfpathlineto{\pgfqpoint{2.769217in}{3.458167in}}%
\pgfpathlineto{\pgfqpoint{2.773474in}{3.458167in}}%
\pgfpathlineto{\pgfqpoint{2.773474in}{3.453909in}}%
\pgfpathmoveto{\pgfqpoint{2.769217in}{3.458167in}}%
\pgfpathlineto{\pgfqpoint{2.769217in}{3.458167in}}%
\pgfpathlineto{\pgfqpoint{2.769217in}{3.462424in}}%
\pgfpathlineto{\pgfqpoint{2.773474in}{3.462424in}}%
\pgfpathlineto{\pgfqpoint{2.773474in}{3.458167in}}%
\pgfpathmoveto{\pgfqpoint{2.773474in}{3.453909in}}%
\pgfpathlineto{\pgfqpoint{2.773474in}{3.453909in}}%
\pgfpathlineto{\pgfqpoint{2.773474in}{3.458167in}}%
\pgfpathlineto{\pgfqpoint{2.777732in}{3.458167in}}%
\pgfpathlineto{\pgfqpoint{2.777732in}{3.453909in}}%
\pgfpathmoveto{\pgfqpoint{2.773474in}{3.458167in}}%
\pgfpathlineto{\pgfqpoint{2.773474in}{3.458167in}}%
\pgfpathlineto{\pgfqpoint{2.773474in}{3.462424in}}%
\pgfpathlineto{\pgfqpoint{2.777732in}{3.462424in}}%
\pgfpathlineto{\pgfqpoint{2.777732in}{3.458167in}}%
\pgfpathmoveto{\pgfqpoint{2.769217in}{3.462424in}}%
\pgfpathlineto{\pgfqpoint{2.769217in}{3.462424in}}%
\pgfpathlineto{\pgfqpoint{2.769217in}{3.466682in}}%
\pgfpathlineto{\pgfqpoint{2.773474in}{3.466682in}}%
\pgfpathlineto{\pgfqpoint{2.773474in}{3.462424in}}%
\pgfpathmoveto{\pgfqpoint{2.773474in}{3.462424in}}%
\pgfpathlineto{\pgfqpoint{2.773474in}{3.462424in}}%
\pgfpathlineto{\pgfqpoint{2.773474in}{3.466682in}}%
\pgfpathlineto{\pgfqpoint{2.777732in}{3.466682in}}%
\pgfpathlineto{\pgfqpoint{2.777732in}{3.462424in}}%
\pgfpathmoveto{\pgfqpoint{2.773474in}{3.466682in}}%
\pgfpathlineto{\pgfqpoint{2.773474in}{3.466682in}}%
\pgfpathlineto{\pgfqpoint{2.773474in}{3.470940in}}%
\pgfpathlineto{\pgfqpoint{2.777732in}{3.470940in}}%
\pgfpathlineto{\pgfqpoint{2.777732in}{3.466682in}}%
\pgfpathmoveto{\pgfqpoint{2.777732in}{3.453909in}}%
\pgfpathlineto{\pgfqpoint{2.777732in}{3.453909in}}%
\pgfpathlineto{\pgfqpoint{2.777732in}{3.458167in}}%
\pgfpathlineto{\pgfqpoint{2.781990in}{3.458167in}}%
\pgfpathlineto{\pgfqpoint{2.781990in}{3.453909in}}%
\pgfpathmoveto{\pgfqpoint{2.777732in}{3.458167in}}%
\pgfpathlineto{\pgfqpoint{2.777732in}{3.458167in}}%
\pgfpathlineto{\pgfqpoint{2.777732in}{3.462424in}}%
\pgfpathlineto{\pgfqpoint{2.781990in}{3.462424in}}%
\pgfpathlineto{\pgfqpoint{2.781990in}{3.458167in}}%
\pgfpathmoveto{\pgfqpoint{2.781990in}{3.453909in}}%
\pgfpathlineto{\pgfqpoint{2.781990in}{3.453909in}}%
\pgfpathlineto{\pgfqpoint{2.781990in}{3.458167in}}%
\pgfpathlineto{\pgfqpoint{2.786248in}{3.458167in}}%
\pgfpathlineto{\pgfqpoint{2.786248in}{3.453909in}}%
\pgfpathmoveto{\pgfqpoint{2.781990in}{3.458167in}}%
\pgfpathlineto{\pgfqpoint{2.781990in}{3.458167in}}%
\pgfpathlineto{\pgfqpoint{2.781990in}{3.462424in}}%
\pgfpathlineto{\pgfqpoint{2.786248in}{3.462424in}}%
\pgfpathlineto{\pgfqpoint{2.786248in}{3.458167in}}%
\pgfpathmoveto{\pgfqpoint{2.777732in}{3.462424in}}%
\pgfpathlineto{\pgfqpoint{2.777732in}{3.462424in}}%
\pgfpathlineto{\pgfqpoint{2.777732in}{3.466682in}}%
\pgfpathlineto{\pgfqpoint{2.781990in}{3.466682in}}%
\pgfpathlineto{\pgfqpoint{2.781990in}{3.462424in}}%
\pgfpathmoveto{\pgfqpoint{2.777732in}{3.466682in}}%
\pgfpathlineto{\pgfqpoint{2.777732in}{3.466682in}}%
\pgfpathlineto{\pgfqpoint{2.777732in}{3.470940in}}%
\pgfpathlineto{\pgfqpoint{2.781990in}{3.470940in}}%
\pgfpathlineto{\pgfqpoint{2.781990in}{3.466682in}}%
\pgfpathmoveto{\pgfqpoint{2.781990in}{3.462424in}}%
\pgfpathlineto{\pgfqpoint{2.781990in}{3.462424in}}%
\pgfpathlineto{\pgfqpoint{2.781990in}{3.466682in}}%
\pgfpathlineto{\pgfqpoint{2.786248in}{3.466682in}}%
\pgfpathlineto{\pgfqpoint{2.786248in}{3.462424in}}%
\pgfpathmoveto{\pgfqpoint{2.781990in}{3.466682in}}%
\pgfpathlineto{\pgfqpoint{2.781990in}{3.466682in}}%
\pgfpathlineto{\pgfqpoint{2.781990in}{3.470940in}}%
\pgfpathlineto{\pgfqpoint{2.786248in}{3.470940in}}%
\pgfpathlineto{\pgfqpoint{2.786248in}{3.466682in}}%
\pgfpathmoveto{\pgfqpoint{2.773474in}{3.470940in}}%
\pgfpathlineto{\pgfqpoint{2.773474in}{3.470940in}}%
\pgfpathlineto{\pgfqpoint{2.773474in}{3.475198in}}%
\pgfpathlineto{\pgfqpoint{2.777732in}{3.475198in}}%
\pgfpathlineto{\pgfqpoint{2.777732in}{3.470940in}}%
\pgfpathmoveto{\pgfqpoint{2.773474in}{3.475198in}}%
\pgfpathlineto{\pgfqpoint{2.773474in}{3.475198in}}%
\pgfpathlineto{\pgfqpoint{2.773474in}{3.479456in}}%
\pgfpathlineto{\pgfqpoint{2.777732in}{3.479456in}}%
\pgfpathlineto{\pgfqpoint{2.777732in}{3.475198in}}%
\pgfpathmoveto{\pgfqpoint{2.777732in}{3.470940in}}%
\pgfpathlineto{\pgfqpoint{2.777732in}{3.470940in}}%
\pgfpathlineto{\pgfqpoint{2.777732in}{3.475198in}}%
\pgfpathlineto{\pgfqpoint{2.781990in}{3.475198in}}%
\pgfpathlineto{\pgfqpoint{2.781990in}{3.470940in}}%
\pgfpathmoveto{\pgfqpoint{2.777732in}{3.475198in}}%
\pgfpathlineto{\pgfqpoint{2.777732in}{3.475198in}}%
\pgfpathlineto{\pgfqpoint{2.777732in}{3.479456in}}%
\pgfpathlineto{\pgfqpoint{2.781990in}{3.479456in}}%
\pgfpathlineto{\pgfqpoint{2.781990in}{3.475198in}}%
\pgfpathmoveto{\pgfqpoint{2.781990in}{3.470940in}}%
\pgfpathlineto{\pgfqpoint{2.781990in}{3.470940in}}%
\pgfpathlineto{\pgfqpoint{2.781990in}{3.475198in}}%
\pgfpathlineto{\pgfqpoint{2.786248in}{3.475198in}}%
\pgfpathlineto{\pgfqpoint{2.786248in}{3.470940in}}%
\pgfpathmoveto{\pgfqpoint{2.781990in}{3.475198in}}%
\pgfpathlineto{\pgfqpoint{2.781990in}{3.475198in}}%
\pgfpathlineto{\pgfqpoint{2.781990in}{3.479456in}}%
\pgfpathlineto{\pgfqpoint{2.786248in}{3.479456in}}%
\pgfpathlineto{\pgfqpoint{2.786248in}{3.475198in}}%
\pgfpathmoveto{\pgfqpoint{2.777732in}{3.479456in}}%
\pgfpathlineto{\pgfqpoint{2.777732in}{3.479456in}}%
\pgfpathlineto{\pgfqpoint{2.777732in}{3.483714in}}%
\pgfpathlineto{\pgfqpoint{2.781990in}{3.483714in}}%
\pgfpathlineto{\pgfqpoint{2.781990in}{3.479456in}}%
\pgfpathmoveto{\pgfqpoint{2.777732in}{3.483714in}}%
\pgfpathlineto{\pgfqpoint{2.777732in}{3.483714in}}%
\pgfpathlineto{\pgfqpoint{2.777732in}{3.487972in}}%
\pgfpathlineto{\pgfqpoint{2.781990in}{3.487972in}}%
\pgfpathlineto{\pgfqpoint{2.781990in}{3.483714in}}%
\pgfpathmoveto{\pgfqpoint{2.781990in}{3.479456in}}%
\pgfpathlineto{\pgfqpoint{2.781990in}{3.479456in}}%
\pgfpathlineto{\pgfqpoint{2.781990in}{3.483714in}}%
\pgfpathlineto{\pgfqpoint{2.786248in}{3.483714in}}%
\pgfpathlineto{\pgfqpoint{2.786248in}{3.479456in}}%
\pgfpathmoveto{\pgfqpoint{2.781990in}{3.483714in}}%
\pgfpathlineto{\pgfqpoint{2.781990in}{3.483714in}}%
\pgfpathlineto{\pgfqpoint{2.781990in}{3.487972in}}%
\pgfpathlineto{\pgfqpoint{2.786248in}{3.487972in}}%
\pgfpathlineto{\pgfqpoint{2.786248in}{3.483714in}}%
\pgfpathmoveto{\pgfqpoint{2.777732in}{3.487972in}}%
\pgfpathlineto{\pgfqpoint{2.777732in}{3.487972in}}%
\pgfpathlineto{\pgfqpoint{2.777732in}{3.492230in}}%
\pgfpathlineto{\pgfqpoint{2.781990in}{3.492230in}}%
\pgfpathlineto{\pgfqpoint{2.781990in}{3.487972in}}%
\pgfpathmoveto{\pgfqpoint{2.781990in}{3.487972in}}%
\pgfpathlineto{\pgfqpoint{2.781990in}{3.487972in}}%
\pgfpathlineto{\pgfqpoint{2.781990in}{3.492230in}}%
\pgfpathlineto{\pgfqpoint{2.786248in}{3.492230in}}%
\pgfpathlineto{\pgfqpoint{2.786248in}{3.487972in}}%
\pgfpathmoveto{\pgfqpoint{2.781990in}{3.492230in}}%
\pgfpathlineto{\pgfqpoint{2.781990in}{3.492230in}}%
\pgfpathlineto{\pgfqpoint{2.781990in}{3.496487in}}%
\pgfpathlineto{\pgfqpoint{2.786248in}{3.496487in}}%
\pgfpathlineto{\pgfqpoint{2.786248in}{3.492230in}}%
\pgfpathmoveto{\pgfqpoint{2.781990in}{3.496487in}}%
\pgfpathlineto{\pgfqpoint{2.781990in}{3.496487in}}%
\pgfpathlineto{\pgfqpoint{2.781990in}{3.500745in}}%
\pgfpathlineto{\pgfqpoint{2.786248in}{3.500745in}}%
\pgfpathlineto{\pgfqpoint{2.786248in}{3.496487in}}%
\pgfpathmoveto{\pgfqpoint{2.781990in}{3.500745in}}%
\pgfpathlineto{\pgfqpoint{2.781990in}{3.500745in}}%
\pgfpathlineto{\pgfqpoint{2.781990in}{3.505003in}}%
\pgfpathlineto{\pgfqpoint{2.786248in}{3.505003in}}%
\pgfpathlineto{\pgfqpoint{2.786248in}{3.500745in}}%
\pgfpathmoveto{\pgfqpoint{2.786248in}{3.462424in}}%
\pgfpathlineto{\pgfqpoint{2.786248in}{3.462424in}}%
\pgfpathlineto{\pgfqpoint{2.786248in}{3.466682in}}%
\pgfpathlineto{\pgfqpoint{2.790506in}{3.466682in}}%
\pgfpathlineto{\pgfqpoint{2.790506in}{3.462424in}}%
\pgfpathmoveto{\pgfqpoint{2.786248in}{3.466682in}}%
\pgfpathlineto{\pgfqpoint{2.786248in}{3.466682in}}%
\pgfpathlineto{\pgfqpoint{2.786248in}{3.470940in}}%
\pgfpathlineto{\pgfqpoint{2.790506in}{3.470940in}}%
\pgfpathlineto{\pgfqpoint{2.790506in}{3.466682in}}%
\pgfpathmoveto{\pgfqpoint{2.786248in}{3.470940in}}%
\pgfpathlineto{\pgfqpoint{2.786248in}{3.470940in}}%
\pgfpathlineto{\pgfqpoint{2.786248in}{3.475198in}}%
\pgfpathlineto{\pgfqpoint{2.790506in}{3.475198in}}%
\pgfpathlineto{\pgfqpoint{2.790506in}{3.470940in}}%
\pgfpathmoveto{\pgfqpoint{2.786248in}{3.475198in}}%
\pgfpathlineto{\pgfqpoint{2.786248in}{3.475198in}}%
\pgfpathlineto{\pgfqpoint{2.786248in}{3.479456in}}%
\pgfpathlineto{\pgfqpoint{2.790506in}{3.479456in}}%
\pgfpathlineto{\pgfqpoint{2.790506in}{3.475198in}}%
\pgfpathmoveto{\pgfqpoint{2.790506in}{3.475198in}}%
\pgfpathlineto{\pgfqpoint{2.790506in}{3.475198in}}%
\pgfpathlineto{\pgfqpoint{2.790506in}{3.479456in}}%
\pgfpathlineto{\pgfqpoint{2.794764in}{3.479456in}}%
\pgfpathlineto{\pgfqpoint{2.794764in}{3.475198in}}%
\pgfpathmoveto{\pgfqpoint{2.786248in}{3.479456in}}%
\pgfpathlineto{\pgfqpoint{2.786248in}{3.479456in}}%
\pgfpathlineto{\pgfqpoint{2.786248in}{3.483714in}}%
\pgfpathlineto{\pgfqpoint{2.790506in}{3.483714in}}%
\pgfpathlineto{\pgfqpoint{2.790506in}{3.479456in}}%
\pgfpathmoveto{\pgfqpoint{2.786248in}{3.483714in}}%
\pgfpathlineto{\pgfqpoint{2.786248in}{3.483714in}}%
\pgfpathlineto{\pgfqpoint{2.786248in}{3.487972in}}%
\pgfpathlineto{\pgfqpoint{2.790506in}{3.487972in}}%
\pgfpathlineto{\pgfqpoint{2.790506in}{3.483714in}}%
\pgfpathmoveto{\pgfqpoint{2.790506in}{3.479456in}}%
\pgfpathlineto{\pgfqpoint{2.790506in}{3.479456in}}%
\pgfpathlineto{\pgfqpoint{2.790506in}{3.483714in}}%
\pgfpathlineto{\pgfqpoint{2.794764in}{3.483714in}}%
\pgfpathlineto{\pgfqpoint{2.794764in}{3.479456in}}%
\pgfpathmoveto{\pgfqpoint{2.790506in}{3.483714in}}%
\pgfpathlineto{\pgfqpoint{2.790506in}{3.483714in}}%
\pgfpathlineto{\pgfqpoint{2.790506in}{3.487972in}}%
\pgfpathlineto{\pgfqpoint{2.794764in}{3.487972in}}%
\pgfpathlineto{\pgfqpoint{2.794764in}{3.483714in}}%
\pgfpathmoveto{\pgfqpoint{2.786248in}{3.487972in}}%
\pgfpathlineto{\pgfqpoint{2.786248in}{3.487972in}}%
\pgfpathlineto{\pgfqpoint{2.786248in}{3.492230in}}%
\pgfpathlineto{\pgfqpoint{2.790506in}{3.492230in}}%
\pgfpathlineto{\pgfqpoint{2.790506in}{3.487972in}}%
\pgfpathmoveto{\pgfqpoint{2.786248in}{3.492230in}}%
\pgfpathlineto{\pgfqpoint{2.786248in}{3.492230in}}%
\pgfpathlineto{\pgfqpoint{2.786248in}{3.496487in}}%
\pgfpathlineto{\pgfqpoint{2.790506in}{3.496487in}}%
\pgfpathlineto{\pgfqpoint{2.790506in}{3.492230in}}%
\pgfpathmoveto{\pgfqpoint{2.790506in}{3.487972in}}%
\pgfpathlineto{\pgfqpoint{2.790506in}{3.487972in}}%
\pgfpathlineto{\pgfqpoint{2.790506in}{3.492230in}}%
\pgfpathlineto{\pgfqpoint{2.794764in}{3.492230in}}%
\pgfpathlineto{\pgfqpoint{2.794764in}{3.487972in}}%
\pgfpathmoveto{\pgfqpoint{2.790506in}{3.492230in}}%
\pgfpathlineto{\pgfqpoint{2.790506in}{3.492230in}}%
\pgfpathlineto{\pgfqpoint{2.790506in}{3.496487in}}%
\pgfpathlineto{\pgfqpoint{2.794764in}{3.496487in}}%
\pgfpathlineto{\pgfqpoint{2.794764in}{3.492230in}}%
\pgfpathmoveto{\pgfqpoint{2.786248in}{3.496487in}}%
\pgfpathlineto{\pgfqpoint{2.786248in}{3.496487in}}%
\pgfpathlineto{\pgfqpoint{2.786248in}{3.500745in}}%
\pgfpathlineto{\pgfqpoint{2.790506in}{3.500745in}}%
\pgfpathlineto{\pgfqpoint{2.790506in}{3.496487in}}%
\pgfpathmoveto{\pgfqpoint{2.786248in}{3.500745in}}%
\pgfpathlineto{\pgfqpoint{2.786248in}{3.500745in}}%
\pgfpathlineto{\pgfqpoint{2.786248in}{3.505003in}}%
\pgfpathlineto{\pgfqpoint{2.790506in}{3.505003in}}%
\pgfpathlineto{\pgfqpoint{2.790506in}{3.500745in}}%
\pgfpathmoveto{\pgfqpoint{2.790506in}{3.496487in}}%
\pgfpathlineto{\pgfqpoint{2.790506in}{3.496487in}}%
\pgfpathlineto{\pgfqpoint{2.790506in}{3.500745in}}%
\pgfpathlineto{\pgfqpoint{2.794764in}{3.500745in}}%
\pgfpathlineto{\pgfqpoint{2.794764in}{3.496487in}}%
\pgfpathmoveto{\pgfqpoint{2.790506in}{3.500745in}}%
\pgfpathlineto{\pgfqpoint{2.790506in}{3.500745in}}%
\pgfpathlineto{\pgfqpoint{2.790506in}{3.505003in}}%
\pgfpathlineto{\pgfqpoint{2.794764in}{3.505003in}}%
\pgfpathlineto{\pgfqpoint{2.794764in}{3.500745in}}%
\pgfpathmoveto{\pgfqpoint{2.794764in}{3.487972in}}%
\pgfpathlineto{\pgfqpoint{2.794764in}{3.487972in}}%
\pgfpathlineto{\pgfqpoint{2.794764in}{3.492230in}}%
\pgfpathlineto{\pgfqpoint{2.799022in}{3.492230in}}%
\pgfpathlineto{\pgfqpoint{2.799022in}{3.487972in}}%
\pgfpathmoveto{\pgfqpoint{2.794764in}{3.492230in}}%
\pgfpathlineto{\pgfqpoint{2.794764in}{3.492230in}}%
\pgfpathlineto{\pgfqpoint{2.794764in}{3.496487in}}%
\pgfpathlineto{\pgfqpoint{2.799022in}{3.496487in}}%
\pgfpathlineto{\pgfqpoint{2.799022in}{3.492230in}}%
\pgfpathmoveto{\pgfqpoint{2.794764in}{3.496487in}}%
\pgfpathlineto{\pgfqpoint{2.794764in}{3.496487in}}%
\pgfpathlineto{\pgfqpoint{2.794764in}{3.500745in}}%
\pgfpathlineto{\pgfqpoint{2.799022in}{3.500745in}}%
\pgfpathlineto{\pgfqpoint{2.799022in}{3.496487in}}%
\pgfpathmoveto{\pgfqpoint{2.794764in}{3.500745in}}%
\pgfpathlineto{\pgfqpoint{2.794764in}{3.500745in}}%
\pgfpathlineto{\pgfqpoint{2.794764in}{3.505003in}}%
\pgfpathlineto{\pgfqpoint{2.799022in}{3.505003in}}%
\pgfpathlineto{\pgfqpoint{2.799022in}{3.500745in}}%
\pgfpathmoveto{\pgfqpoint{2.799022in}{3.500745in}}%
\pgfpathlineto{\pgfqpoint{2.799022in}{3.500745in}}%
\pgfpathlineto{\pgfqpoint{2.799022in}{3.505003in}}%
\pgfpathlineto{\pgfqpoint{2.803280in}{3.505003in}}%
\pgfpathlineto{\pgfqpoint{2.803280in}{3.500745in}}%
\pgfpathmoveto{\pgfqpoint{2.781990in}{3.505003in}}%
\pgfpathlineto{\pgfqpoint{2.781990in}{3.505003in}}%
\pgfpathlineto{\pgfqpoint{2.781990in}{3.509261in}}%
\pgfpathlineto{\pgfqpoint{2.786248in}{3.509261in}}%
\pgfpathlineto{\pgfqpoint{2.786248in}{3.505003in}}%
\pgfpathmoveto{\pgfqpoint{2.786248in}{3.505003in}}%
\pgfpathlineto{\pgfqpoint{2.786248in}{3.505003in}}%
\pgfpathlineto{\pgfqpoint{2.786248in}{3.509261in}}%
\pgfpathlineto{\pgfqpoint{2.790506in}{3.509261in}}%
\pgfpathlineto{\pgfqpoint{2.790506in}{3.505003in}}%
\pgfpathmoveto{\pgfqpoint{2.786248in}{3.509261in}}%
\pgfpathlineto{\pgfqpoint{2.786248in}{3.509261in}}%
\pgfpathlineto{\pgfqpoint{2.786248in}{3.513519in}}%
\pgfpathlineto{\pgfqpoint{2.790506in}{3.513519in}}%
\pgfpathlineto{\pgfqpoint{2.790506in}{3.509261in}}%
\pgfpathmoveto{\pgfqpoint{2.790506in}{3.505003in}}%
\pgfpathlineto{\pgfqpoint{2.790506in}{3.505003in}}%
\pgfpathlineto{\pgfqpoint{2.790506in}{3.509261in}}%
\pgfpathlineto{\pgfqpoint{2.794764in}{3.509261in}}%
\pgfpathlineto{\pgfqpoint{2.794764in}{3.505003in}}%
\pgfpathmoveto{\pgfqpoint{2.790506in}{3.509261in}}%
\pgfpathlineto{\pgfqpoint{2.790506in}{3.509261in}}%
\pgfpathlineto{\pgfqpoint{2.790506in}{3.513519in}}%
\pgfpathlineto{\pgfqpoint{2.794764in}{3.513519in}}%
\pgfpathlineto{\pgfqpoint{2.794764in}{3.509261in}}%
\pgfpathmoveto{\pgfqpoint{2.786248in}{3.513519in}}%
\pgfpathlineto{\pgfqpoint{2.786248in}{3.513519in}}%
\pgfpathlineto{\pgfqpoint{2.786248in}{3.517776in}}%
\pgfpathlineto{\pgfqpoint{2.790506in}{3.517776in}}%
\pgfpathlineto{\pgfqpoint{2.790506in}{3.513519in}}%
\pgfpathmoveto{\pgfqpoint{2.786248in}{3.517776in}}%
\pgfpathlineto{\pgfqpoint{2.786248in}{3.517776in}}%
\pgfpathlineto{\pgfqpoint{2.786248in}{3.522034in}}%
\pgfpathlineto{\pgfqpoint{2.790506in}{3.522034in}}%
\pgfpathlineto{\pgfqpoint{2.790506in}{3.517776in}}%
\pgfpathmoveto{\pgfqpoint{2.790506in}{3.513519in}}%
\pgfpathlineto{\pgfqpoint{2.790506in}{3.513519in}}%
\pgfpathlineto{\pgfqpoint{2.790506in}{3.517776in}}%
\pgfpathlineto{\pgfqpoint{2.794764in}{3.517776in}}%
\pgfpathlineto{\pgfqpoint{2.794764in}{3.513519in}}%
\pgfpathmoveto{\pgfqpoint{2.790506in}{3.517776in}}%
\pgfpathlineto{\pgfqpoint{2.790506in}{3.517776in}}%
\pgfpathlineto{\pgfqpoint{2.790506in}{3.522034in}}%
\pgfpathlineto{\pgfqpoint{2.794764in}{3.522034in}}%
\pgfpathlineto{\pgfqpoint{2.794764in}{3.517776in}}%
\pgfpathmoveto{\pgfqpoint{2.794764in}{3.505003in}}%
\pgfpathlineto{\pgfqpoint{2.794764in}{3.505003in}}%
\pgfpathlineto{\pgfqpoint{2.794764in}{3.509261in}}%
\pgfpathlineto{\pgfqpoint{2.799022in}{3.509261in}}%
\pgfpathlineto{\pgfqpoint{2.799022in}{3.505003in}}%
\pgfpathmoveto{\pgfqpoint{2.794764in}{3.509261in}}%
\pgfpathlineto{\pgfqpoint{2.794764in}{3.509261in}}%
\pgfpathlineto{\pgfqpoint{2.794764in}{3.513519in}}%
\pgfpathlineto{\pgfqpoint{2.799022in}{3.513519in}}%
\pgfpathlineto{\pgfqpoint{2.799022in}{3.509261in}}%
\pgfpathmoveto{\pgfqpoint{2.799022in}{3.505003in}}%
\pgfpathlineto{\pgfqpoint{2.799022in}{3.505003in}}%
\pgfpathlineto{\pgfqpoint{2.799022in}{3.509261in}}%
\pgfpathlineto{\pgfqpoint{2.803280in}{3.509261in}}%
\pgfpathlineto{\pgfqpoint{2.803280in}{3.505003in}}%
\pgfpathmoveto{\pgfqpoint{2.799022in}{3.509261in}}%
\pgfpathlineto{\pgfqpoint{2.799022in}{3.509261in}}%
\pgfpathlineto{\pgfqpoint{2.799022in}{3.513519in}}%
\pgfpathlineto{\pgfqpoint{2.803280in}{3.513519in}}%
\pgfpathlineto{\pgfqpoint{2.803280in}{3.509261in}}%
\pgfpathmoveto{\pgfqpoint{2.794764in}{3.513519in}}%
\pgfpathlineto{\pgfqpoint{2.794764in}{3.513519in}}%
\pgfpathlineto{\pgfqpoint{2.794764in}{3.517776in}}%
\pgfpathlineto{\pgfqpoint{2.799022in}{3.517776in}}%
\pgfpathlineto{\pgfqpoint{2.799022in}{3.513519in}}%
\pgfpathmoveto{\pgfqpoint{2.794764in}{3.517776in}}%
\pgfpathlineto{\pgfqpoint{2.794764in}{3.517776in}}%
\pgfpathlineto{\pgfqpoint{2.794764in}{3.522034in}}%
\pgfpathlineto{\pgfqpoint{2.799022in}{3.522034in}}%
\pgfpathlineto{\pgfqpoint{2.799022in}{3.517776in}}%
\pgfpathmoveto{\pgfqpoint{2.799022in}{3.513519in}}%
\pgfpathlineto{\pgfqpoint{2.799022in}{3.513519in}}%
\pgfpathlineto{\pgfqpoint{2.799022in}{3.517776in}}%
\pgfpathlineto{\pgfqpoint{2.803280in}{3.517776in}}%
\pgfpathlineto{\pgfqpoint{2.803280in}{3.513519in}}%
\pgfpathmoveto{\pgfqpoint{2.799022in}{3.517776in}}%
\pgfpathlineto{\pgfqpoint{2.799022in}{3.517776in}}%
\pgfpathlineto{\pgfqpoint{2.799022in}{3.522034in}}%
\pgfpathlineto{\pgfqpoint{2.803280in}{3.522034in}}%
\pgfpathlineto{\pgfqpoint{2.803280in}{3.517776in}}%
\pgfpathmoveto{\pgfqpoint{2.790506in}{3.522034in}}%
\pgfpathlineto{\pgfqpoint{2.790506in}{3.522034in}}%
\pgfpathlineto{\pgfqpoint{2.790506in}{3.526292in}}%
\pgfpathlineto{\pgfqpoint{2.794764in}{3.526292in}}%
\pgfpathlineto{\pgfqpoint{2.794764in}{3.522034in}}%
\pgfpathmoveto{\pgfqpoint{2.790506in}{3.526292in}}%
\pgfpathlineto{\pgfqpoint{2.790506in}{3.526292in}}%
\pgfpathlineto{\pgfqpoint{2.790506in}{3.530550in}}%
\pgfpathlineto{\pgfqpoint{2.794764in}{3.530550in}}%
\pgfpathlineto{\pgfqpoint{2.794764in}{3.526292in}}%
\pgfpathmoveto{\pgfqpoint{2.790506in}{3.530550in}}%
\pgfpathlineto{\pgfqpoint{2.790506in}{3.530550in}}%
\pgfpathlineto{\pgfqpoint{2.790506in}{3.534807in}}%
\pgfpathlineto{\pgfqpoint{2.794764in}{3.534807in}}%
\pgfpathlineto{\pgfqpoint{2.794764in}{3.530550in}}%
\pgfpathmoveto{\pgfqpoint{2.794764in}{3.522034in}}%
\pgfpathlineto{\pgfqpoint{2.794764in}{3.522034in}}%
\pgfpathlineto{\pgfqpoint{2.794764in}{3.526292in}}%
\pgfpathlineto{\pgfqpoint{2.799022in}{3.526292in}}%
\pgfpathlineto{\pgfqpoint{2.799022in}{3.522034in}}%
\pgfpathmoveto{\pgfqpoint{2.794764in}{3.526292in}}%
\pgfpathlineto{\pgfqpoint{2.794764in}{3.526292in}}%
\pgfpathlineto{\pgfqpoint{2.794764in}{3.530550in}}%
\pgfpathlineto{\pgfqpoint{2.799022in}{3.530550in}}%
\pgfpathlineto{\pgfqpoint{2.799022in}{3.526292in}}%
\pgfpathmoveto{\pgfqpoint{2.799022in}{3.522034in}}%
\pgfpathlineto{\pgfqpoint{2.799022in}{3.522034in}}%
\pgfpathlineto{\pgfqpoint{2.799022in}{3.526292in}}%
\pgfpathlineto{\pgfqpoint{2.803280in}{3.526292in}}%
\pgfpathlineto{\pgfqpoint{2.803280in}{3.522034in}}%
\pgfpathmoveto{\pgfqpoint{2.799022in}{3.526292in}}%
\pgfpathlineto{\pgfqpoint{2.799022in}{3.526292in}}%
\pgfpathlineto{\pgfqpoint{2.799022in}{3.530550in}}%
\pgfpathlineto{\pgfqpoint{2.803280in}{3.530550in}}%
\pgfpathlineto{\pgfqpoint{2.803280in}{3.526292in}}%
\pgfpathmoveto{\pgfqpoint{2.794764in}{3.530550in}}%
\pgfpathlineto{\pgfqpoint{2.794764in}{3.530550in}}%
\pgfpathlineto{\pgfqpoint{2.794764in}{3.534807in}}%
\pgfpathlineto{\pgfqpoint{2.799022in}{3.534807in}}%
\pgfpathlineto{\pgfqpoint{2.799022in}{3.530550in}}%
\pgfpathmoveto{\pgfqpoint{2.794764in}{3.534807in}}%
\pgfpathlineto{\pgfqpoint{2.794764in}{3.534807in}}%
\pgfpathlineto{\pgfqpoint{2.794764in}{3.539065in}}%
\pgfpathlineto{\pgfqpoint{2.799022in}{3.539065in}}%
\pgfpathlineto{\pgfqpoint{2.799022in}{3.534807in}}%
\pgfpathmoveto{\pgfqpoint{2.799022in}{3.530550in}}%
\pgfpathlineto{\pgfqpoint{2.799022in}{3.530550in}}%
\pgfpathlineto{\pgfqpoint{2.799022in}{3.534807in}}%
\pgfpathlineto{\pgfqpoint{2.803280in}{3.534807in}}%
\pgfpathlineto{\pgfqpoint{2.803280in}{3.530550in}}%
\pgfpathmoveto{\pgfqpoint{2.799022in}{3.534807in}}%
\pgfpathlineto{\pgfqpoint{2.799022in}{3.534807in}}%
\pgfpathlineto{\pgfqpoint{2.799022in}{3.539065in}}%
\pgfpathlineto{\pgfqpoint{2.803280in}{3.539065in}}%
\pgfpathlineto{\pgfqpoint{2.803280in}{3.534807in}}%
\pgfpathmoveto{\pgfqpoint{2.803280in}{3.509261in}}%
\pgfpathlineto{\pgfqpoint{2.803280in}{3.509261in}}%
\pgfpathlineto{\pgfqpoint{2.803280in}{3.513519in}}%
\pgfpathlineto{\pgfqpoint{2.807537in}{3.513519in}}%
\pgfpathlineto{\pgfqpoint{2.807537in}{3.509261in}}%
\pgfpathmoveto{\pgfqpoint{2.803280in}{3.513519in}}%
\pgfpathlineto{\pgfqpoint{2.803280in}{3.513519in}}%
\pgfpathlineto{\pgfqpoint{2.803280in}{3.517776in}}%
\pgfpathlineto{\pgfqpoint{2.807537in}{3.517776in}}%
\pgfpathlineto{\pgfqpoint{2.807537in}{3.513519in}}%
\pgfpathmoveto{\pgfqpoint{2.803280in}{3.517776in}}%
\pgfpathlineto{\pgfqpoint{2.803280in}{3.517776in}}%
\pgfpathlineto{\pgfqpoint{2.803280in}{3.522034in}}%
\pgfpathlineto{\pgfqpoint{2.807537in}{3.522034in}}%
\pgfpathlineto{\pgfqpoint{2.807537in}{3.517776in}}%
\pgfpathmoveto{\pgfqpoint{2.803280in}{3.522034in}}%
\pgfpathlineto{\pgfqpoint{2.803280in}{3.522034in}}%
\pgfpathlineto{\pgfqpoint{2.803280in}{3.526292in}}%
\pgfpathlineto{\pgfqpoint{2.807537in}{3.526292in}}%
\pgfpathlineto{\pgfqpoint{2.807537in}{3.522034in}}%
\pgfpathmoveto{\pgfqpoint{2.803280in}{3.526292in}}%
\pgfpathlineto{\pgfqpoint{2.803280in}{3.526292in}}%
\pgfpathlineto{\pgfqpoint{2.803280in}{3.530550in}}%
\pgfpathlineto{\pgfqpoint{2.807537in}{3.530550in}}%
\pgfpathlineto{\pgfqpoint{2.807537in}{3.526292in}}%
\pgfpathmoveto{\pgfqpoint{2.807537in}{3.522034in}}%
\pgfpathlineto{\pgfqpoint{2.807537in}{3.522034in}}%
\pgfpathlineto{\pgfqpoint{2.807537in}{3.526292in}}%
\pgfpathlineto{\pgfqpoint{2.811795in}{3.526292in}}%
\pgfpathlineto{\pgfqpoint{2.811795in}{3.522034in}}%
\pgfpathmoveto{\pgfqpoint{2.807537in}{3.526292in}}%
\pgfpathlineto{\pgfqpoint{2.807537in}{3.526292in}}%
\pgfpathlineto{\pgfqpoint{2.807537in}{3.530550in}}%
\pgfpathlineto{\pgfqpoint{2.811795in}{3.530550in}}%
\pgfpathlineto{\pgfqpoint{2.811795in}{3.526292in}}%
\pgfpathmoveto{\pgfqpoint{2.803280in}{3.530550in}}%
\pgfpathlineto{\pgfqpoint{2.803280in}{3.530550in}}%
\pgfpathlineto{\pgfqpoint{2.803280in}{3.534807in}}%
\pgfpathlineto{\pgfqpoint{2.807537in}{3.534807in}}%
\pgfpathlineto{\pgfqpoint{2.807537in}{3.530550in}}%
\pgfpathmoveto{\pgfqpoint{2.803280in}{3.534807in}}%
\pgfpathlineto{\pgfqpoint{2.803280in}{3.534807in}}%
\pgfpathlineto{\pgfqpoint{2.803280in}{3.539065in}}%
\pgfpathlineto{\pgfqpoint{2.807537in}{3.539065in}}%
\pgfpathlineto{\pgfqpoint{2.807537in}{3.534807in}}%
\pgfpathmoveto{\pgfqpoint{2.807537in}{3.530550in}}%
\pgfpathlineto{\pgfqpoint{2.807537in}{3.530550in}}%
\pgfpathlineto{\pgfqpoint{2.807537in}{3.534807in}}%
\pgfpathlineto{\pgfqpoint{2.811795in}{3.534807in}}%
\pgfpathlineto{\pgfqpoint{2.811795in}{3.530550in}}%
\pgfpathmoveto{\pgfqpoint{2.807537in}{3.534807in}}%
\pgfpathlineto{\pgfqpoint{2.807537in}{3.534807in}}%
\pgfpathlineto{\pgfqpoint{2.807537in}{3.539065in}}%
\pgfpathlineto{\pgfqpoint{2.811795in}{3.539065in}}%
\pgfpathlineto{\pgfqpoint{2.811795in}{3.534807in}}%
\pgfpathmoveto{\pgfqpoint{2.811795in}{3.534807in}}%
\pgfpathlineto{\pgfqpoint{2.811795in}{3.534807in}}%
\pgfpathlineto{\pgfqpoint{2.811795in}{3.539065in}}%
\pgfpathlineto{\pgfqpoint{2.816053in}{3.539065in}}%
\pgfpathlineto{\pgfqpoint{2.816053in}{3.534807in}}%
\pgfpathmoveto{\pgfqpoint{2.794764in}{3.539065in}}%
\pgfpathlineto{\pgfqpoint{2.794764in}{3.539065in}}%
\pgfpathlineto{\pgfqpoint{2.794764in}{3.543323in}}%
\pgfpathlineto{\pgfqpoint{2.799022in}{3.543323in}}%
\pgfpathlineto{\pgfqpoint{2.799022in}{3.539065in}}%
\pgfpathmoveto{\pgfqpoint{2.794764in}{3.543323in}}%
\pgfpathlineto{\pgfqpoint{2.794764in}{3.543323in}}%
\pgfpathlineto{\pgfqpoint{2.794764in}{3.547580in}}%
\pgfpathlineto{\pgfqpoint{2.799022in}{3.547580in}}%
\pgfpathlineto{\pgfqpoint{2.799022in}{3.543323in}}%
\pgfpathmoveto{\pgfqpoint{2.799022in}{3.539065in}}%
\pgfpathlineto{\pgfqpoint{2.799022in}{3.539065in}}%
\pgfpathlineto{\pgfqpoint{2.799022in}{3.543323in}}%
\pgfpathlineto{\pgfqpoint{2.803280in}{3.543323in}}%
\pgfpathlineto{\pgfqpoint{2.803280in}{3.539065in}}%
\pgfpathmoveto{\pgfqpoint{2.799022in}{3.543323in}}%
\pgfpathlineto{\pgfqpoint{2.799022in}{3.543323in}}%
\pgfpathlineto{\pgfqpoint{2.799022in}{3.547580in}}%
\pgfpathlineto{\pgfqpoint{2.803280in}{3.547580in}}%
\pgfpathlineto{\pgfqpoint{2.803280in}{3.543323in}}%
\pgfpathmoveto{\pgfqpoint{2.799022in}{3.547580in}}%
\pgfpathlineto{\pgfqpoint{2.799022in}{3.547580in}}%
\pgfpathlineto{\pgfqpoint{2.799022in}{3.551838in}}%
\pgfpathlineto{\pgfqpoint{2.803280in}{3.551838in}}%
\pgfpathlineto{\pgfqpoint{2.803280in}{3.547580in}}%
\pgfpathmoveto{\pgfqpoint{2.799022in}{3.551838in}}%
\pgfpathlineto{\pgfqpoint{2.799022in}{3.551838in}}%
\pgfpathlineto{\pgfqpoint{2.799022in}{3.556096in}}%
\pgfpathlineto{\pgfqpoint{2.803280in}{3.556096in}}%
\pgfpathlineto{\pgfqpoint{2.803280in}{3.551838in}}%
\pgfpathmoveto{\pgfqpoint{2.803280in}{3.539065in}}%
\pgfpathlineto{\pgfqpoint{2.803280in}{3.539065in}}%
\pgfpathlineto{\pgfqpoint{2.803280in}{3.543323in}}%
\pgfpathlineto{\pgfqpoint{2.807537in}{3.543323in}}%
\pgfpathlineto{\pgfqpoint{2.807537in}{3.539065in}}%
\pgfpathmoveto{\pgfqpoint{2.803280in}{3.543323in}}%
\pgfpathlineto{\pgfqpoint{2.803280in}{3.543323in}}%
\pgfpathlineto{\pgfqpoint{2.803280in}{3.547580in}}%
\pgfpathlineto{\pgfqpoint{2.807537in}{3.547580in}}%
\pgfpathlineto{\pgfqpoint{2.807537in}{3.543323in}}%
\pgfpathmoveto{\pgfqpoint{2.807537in}{3.539065in}}%
\pgfpathlineto{\pgfqpoint{2.807537in}{3.539065in}}%
\pgfpathlineto{\pgfqpoint{2.807537in}{3.543323in}}%
\pgfpathlineto{\pgfqpoint{2.811795in}{3.543323in}}%
\pgfpathlineto{\pgfqpoint{2.811795in}{3.539065in}}%
\pgfpathmoveto{\pgfqpoint{2.807537in}{3.543323in}}%
\pgfpathlineto{\pgfqpoint{2.807537in}{3.543323in}}%
\pgfpathlineto{\pgfqpoint{2.807537in}{3.547580in}}%
\pgfpathlineto{\pgfqpoint{2.811795in}{3.547580in}}%
\pgfpathlineto{\pgfqpoint{2.811795in}{3.543323in}}%
\pgfpathmoveto{\pgfqpoint{2.803280in}{3.547580in}}%
\pgfpathlineto{\pgfqpoint{2.803280in}{3.547580in}}%
\pgfpathlineto{\pgfqpoint{2.803280in}{3.551838in}}%
\pgfpathlineto{\pgfqpoint{2.807537in}{3.551838in}}%
\pgfpathlineto{\pgfqpoint{2.807537in}{3.547580in}}%
\pgfpathmoveto{\pgfqpoint{2.803280in}{3.551838in}}%
\pgfpathlineto{\pgfqpoint{2.803280in}{3.551838in}}%
\pgfpathlineto{\pgfqpoint{2.803280in}{3.556096in}}%
\pgfpathlineto{\pgfqpoint{2.807537in}{3.556096in}}%
\pgfpathlineto{\pgfqpoint{2.807537in}{3.551838in}}%
\pgfpathmoveto{\pgfqpoint{2.807537in}{3.547580in}}%
\pgfpathlineto{\pgfqpoint{2.807537in}{3.547580in}}%
\pgfpathlineto{\pgfqpoint{2.807537in}{3.551838in}}%
\pgfpathlineto{\pgfqpoint{2.811795in}{3.551838in}}%
\pgfpathlineto{\pgfqpoint{2.811795in}{3.547580in}}%
\pgfpathmoveto{\pgfqpoint{2.807537in}{3.551838in}}%
\pgfpathlineto{\pgfqpoint{2.807537in}{3.551838in}}%
\pgfpathlineto{\pgfqpoint{2.807537in}{3.556096in}}%
\pgfpathlineto{\pgfqpoint{2.811795in}{3.556096in}}%
\pgfpathlineto{\pgfqpoint{2.811795in}{3.551838in}}%
\pgfpathmoveto{\pgfqpoint{2.811795in}{3.539065in}}%
\pgfpathlineto{\pgfqpoint{2.811795in}{3.539065in}}%
\pgfpathlineto{\pgfqpoint{2.811795in}{3.543323in}}%
\pgfpathlineto{\pgfqpoint{2.816053in}{3.543323in}}%
\pgfpathlineto{\pgfqpoint{2.816053in}{3.539065in}}%
\pgfpathmoveto{\pgfqpoint{2.811795in}{3.543323in}}%
\pgfpathlineto{\pgfqpoint{2.811795in}{3.543323in}}%
\pgfpathlineto{\pgfqpoint{2.811795in}{3.547580in}}%
\pgfpathlineto{\pgfqpoint{2.816053in}{3.547580in}}%
\pgfpathlineto{\pgfqpoint{2.816053in}{3.543323in}}%
\pgfpathmoveto{\pgfqpoint{2.811795in}{3.547580in}}%
\pgfpathlineto{\pgfqpoint{2.811795in}{3.547580in}}%
\pgfpathlineto{\pgfqpoint{2.811795in}{3.551838in}}%
\pgfpathlineto{\pgfqpoint{2.816053in}{3.551838in}}%
\pgfpathlineto{\pgfqpoint{2.816053in}{3.547580in}}%
\pgfpathmoveto{\pgfqpoint{2.811795in}{3.551838in}}%
\pgfpathlineto{\pgfqpoint{2.811795in}{3.551838in}}%
\pgfpathlineto{\pgfqpoint{2.811795in}{3.556096in}}%
\pgfpathlineto{\pgfqpoint{2.816053in}{3.556096in}}%
\pgfpathlineto{\pgfqpoint{2.816053in}{3.551838in}}%
\pgfpathmoveto{\pgfqpoint{2.816053in}{3.547580in}}%
\pgfpathlineto{\pgfqpoint{2.816053in}{3.547580in}}%
\pgfpathlineto{\pgfqpoint{2.816053in}{3.551838in}}%
\pgfpathlineto{\pgfqpoint{2.820311in}{3.551838in}}%
\pgfpathlineto{\pgfqpoint{2.820311in}{3.547580in}}%
\pgfpathmoveto{\pgfqpoint{2.816053in}{3.551838in}}%
\pgfpathlineto{\pgfqpoint{2.816053in}{3.551838in}}%
\pgfpathlineto{\pgfqpoint{2.816053in}{3.556096in}}%
\pgfpathlineto{\pgfqpoint{2.820311in}{3.556096in}}%
\pgfpathlineto{\pgfqpoint{2.820311in}{3.551838in}}%
\pgfpathmoveto{\pgfqpoint{2.803280in}{3.556096in}}%
\pgfpathlineto{\pgfqpoint{2.803280in}{3.556096in}}%
\pgfpathlineto{\pgfqpoint{2.803280in}{3.560354in}}%
\pgfpathlineto{\pgfqpoint{2.807537in}{3.560354in}}%
\pgfpathlineto{\pgfqpoint{2.807537in}{3.556096in}}%
\pgfpathmoveto{\pgfqpoint{2.803280in}{3.560354in}}%
\pgfpathlineto{\pgfqpoint{2.803280in}{3.560354in}}%
\pgfpathlineto{\pgfqpoint{2.803280in}{3.564611in}}%
\pgfpathlineto{\pgfqpoint{2.807537in}{3.564611in}}%
\pgfpathlineto{\pgfqpoint{2.807537in}{3.560354in}}%
\pgfpathmoveto{\pgfqpoint{2.807537in}{3.556096in}}%
\pgfpathlineto{\pgfqpoint{2.807537in}{3.556096in}}%
\pgfpathlineto{\pgfqpoint{2.807537in}{3.560354in}}%
\pgfpathlineto{\pgfqpoint{2.811795in}{3.560354in}}%
\pgfpathlineto{\pgfqpoint{2.811795in}{3.556096in}}%
\pgfpathmoveto{\pgfqpoint{2.807537in}{3.560354in}}%
\pgfpathlineto{\pgfqpoint{2.807537in}{3.560354in}}%
\pgfpathlineto{\pgfqpoint{2.807537in}{3.564611in}}%
\pgfpathlineto{\pgfqpoint{2.811795in}{3.564611in}}%
\pgfpathlineto{\pgfqpoint{2.811795in}{3.560354in}}%
\pgfpathmoveto{\pgfqpoint{2.803280in}{3.564611in}}%
\pgfpathlineto{\pgfqpoint{2.803280in}{3.564611in}}%
\pgfpathlineto{\pgfqpoint{2.803280in}{3.568869in}}%
\pgfpathlineto{\pgfqpoint{2.807537in}{3.568869in}}%
\pgfpathlineto{\pgfqpoint{2.807537in}{3.564611in}}%
\pgfpathmoveto{\pgfqpoint{2.807537in}{3.564611in}}%
\pgfpathlineto{\pgfqpoint{2.807537in}{3.564611in}}%
\pgfpathlineto{\pgfqpoint{2.807537in}{3.568869in}}%
\pgfpathlineto{\pgfqpoint{2.811795in}{3.568869in}}%
\pgfpathlineto{\pgfqpoint{2.811795in}{3.564611in}}%
\pgfpathmoveto{\pgfqpoint{2.807537in}{3.568869in}}%
\pgfpathlineto{\pgfqpoint{2.807537in}{3.568869in}}%
\pgfpathlineto{\pgfqpoint{2.807537in}{3.573127in}}%
\pgfpathlineto{\pgfqpoint{2.811795in}{3.573127in}}%
\pgfpathlineto{\pgfqpoint{2.811795in}{3.568869in}}%
\pgfpathmoveto{\pgfqpoint{2.811795in}{3.556096in}}%
\pgfpathlineto{\pgfqpoint{2.811795in}{3.556096in}}%
\pgfpathlineto{\pgfqpoint{2.811795in}{3.560354in}}%
\pgfpathlineto{\pgfqpoint{2.816053in}{3.560354in}}%
\pgfpathlineto{\pgfqpoint{2.816053in}{3.556096in}}%
\pgfpathmoveto{\pgfqpoint{2.811795in}{3.560354in}}%
\pgfpathlineto{\pgfqpoint{2.811795in}{3.560354in}}%
\pgfpathlineto{\pgfqpoint{2.811795in}{3.564611in}}%
\pgfpathlineto{\pgfqpoint{2.816053in}{3.564611in}}%
\pgfpathlineto{\pgfqpoint{2.816053in}{3.560354in}}%
\pgfpathmoveto{\pgfqpoint{2.816053in}{3.556096in}}%
\pgfpathlineto{\pgfqpoint{2.816053in}{3.556096in}}%
\pgfpathlineto{\pgfqpoint{2.816053in}{3.560354in}}%
\pgfpathlineto{\pgfqpoint{2.820311in}{3.560354in}}%
\pgfpathlineto{\pgfqpoint{2.820311in}{3.556096in}}%
\pgfpathmoveto{\pgfqpoint{2.816053in}{3.560354in}}%
\pgfpathlineto{\pgfqpoint{2.816053in}{3.560354in}}%
\pgfpathlineto{\pgfqpoint{2.816053in}{3.564611in}}%
\pgfpathlineto{\pgfqpoint{2.820311in}{3.564611in}}%
\pgfpathlineto{\pgfqpoint{2.820311in}{3.560354in}}%
\pgfpathmoveto{\pgfqpoint{2.811795in}{3.564611in}}%
\pgfpathlineto{\pgfqpoint{2.811795in}{3.564611in}}%
\pgfpathlineto{\pgfqpoint{2.811795in}{3.568869in}}%
\pgfpathlineto{\pgfqpoint{2.816053in}{3.568869in}}%
\pgfpathlineto{\pgfqpoint{2.816053in}{3.564611in}}%
\pgfpathmoveto{\pgfqpoint{2.811795in}{3.568869in}}%
\pgfpathlineto{\pgfqpoint{2.811795in}{3.568869in}}%
\pgfpathlineto{\pgfqpoint{2.811795in}{3.573127in}}%
\pgfpathlineto{\pgfqpoint{2.816053in}{3.573127in}}%
\pgfpathlineto{\pgfqpoint{2.816053in}{3.568869in}}%
\pgfpathmoveto{\pgfqpoint{2.816053in}{3.564611in}}%
\pgfpathlineto{\pgfqpoint{2.816053in}{3.564611in}}%
\pgfpathlineto{\pgfqpoint{2.816053in}{3.568869in}}%
\pgfpathlineto{\pgfqpoint{2.820311in}{3.568869in}}%
\pgfpathlineto{\pgfqpoint{2.820311in}{3.564611in}}%
\pgfpathmoveto{\pgfqpoint{2.816053in}{3.568869in}}%
\pgfpathlineto{\pgfqpoint{2.816053in}{3.568869in}}%
\pgfpathlineto{\pgfqpoint{2.816053in}{3.573127in}}%
\pgfpathlineto{\pgfqpoint{2.820311in}{3.573127in}}%
\pgfpathlineto{\pgfqpoint{2.820311in}{3.568869in}}%
\pgfpathmoveto{\pgfqpoint{2.807537in}{3.573127in}}%
\pgfpathlineto{\pgfqpoint{2.807537in}{3.573127in}}%
\pgfpathlineto{\pgfqpoint{2.807537in}{3.577385in}}%
\pgfpathlineto{\pgfqpoint{2.811795in}{3.577385in}}%
\pgfpathlineto{\pgfqpoint{2.811795in}{3.573127in}}%
\pgfpathmoveto{\pgfqpoint{2.807537in}{3.577385in}}%
\pgfpathlineto{\pgfqpoint{2.807537in}{3.577385in}}%
\pgfpathlineto{\pgfqpoint{2.807537in}{3.581642in}}%
\pgfpathlineto{\pgfqpoint{2.811795in}{3.581642in}}%
\pgfpathlineto{\pgfqpoint{2.811795in}{3.577385in}}%
\pgfpathmoveto{\pgfqpoint{2.811795in}{3.573127in}}%
\pgfpathlineto{\pgfqpoint{2.811795in}{3.573127in}}%
\pgfpathlineto{\pgfqpoint{2.811795in}{3.577385in}}%
\pgfpathlineto{\pgfqpoint{2.816053in}{3.577385in}}%
\pgfpathlineto{\pgfqpoint{2.816053in}{3.573127in}}%
\pgfpathmoveto{\pgfqpoint{2.811795in}{3.577385in}}%
\pgfpathlineto{\pgfqpoint{2.811795in}{3.577385in}}%
\pgfpathlineto{\pgfqpoint{2.811795in}{3.581642in}}%
\pgfpathlineto{\pgfqpoint{2.816053in}{3.581642in}}%
\pgfpathlineto{\pgfqpoint{2.816053in}{3.577385in}}%
\pgfpathmoveto{\pgfqpoint{2.816053in}{3.573127in}}%
\pgfpathlineto{\pgfqpoint{2.816053in}{3.573127in}}%
\pgfpathlineto{\pgfqpoint{2.816053in}{3.577385in}}%
\pgfpathlineto{\pgfqpoint{2.820311in}{3.577385in}}%
\pgfpathlineto{\pgfqpoint{2.820311in}{3.573127in}}%
\pgfpathmoveto{\pgfqpoint{2.816053in}{3.577385in}}%
\pgfpathlineto{\pgfqpoint{2.816053in}{3.577385in}}%
\pgfpathlineto{\pgfqpoint{2.816053in}{3.581642in}}%
\pgfpathlineto{\pgfqpoint{2.820311in}{3.581642in}}%
\pgfpathlineto{\pgfqpoint{2.820311in}{3.577385in}}%
\pgfpathmoveto{\pgfqpoint{2.811795in}{3.581642in}}%
\pgfpathlineto{\pgfqpoint{2.811795in}{3.581642in}}%
\pgfpathlineto{\pgfqpoint{2.811795in}{3.585900in}}%
\pgfpathlineto{\pgfqpoint{2.816053in}{3.585900in}}%
\pgfpathlineto{\pgfqpoint{2.816053in}{3.581642in}}%
\pgfpathmoveto{\pgfqpoint{2.811795in}{3.585900in}}%
\pgfpathlineto{\pgfqpoint{2.811795in}{3.585900in}}%
\pgfpathlineto{\pgfqpoint{2.811795in}{3.590158in}}%
\pgfpathlineto{\pgfqpoint{2.816053in}{3.590158in}}%
\pgfpathlineto{\pgfqpoint{2.816053in}{3.585900in}}%
\pgfpathmoveto{\pgfqpoint{2.816053in}{3.581642in}}%
\pgfpathlineto{\pgfqpoint{2.816053in}{3.581642in}}%
\pgfpathlineto{\pgfqpoint{2.816053in}{3.585900in}}%
\pgfpathlineto{\pgfqpoint{2.820311in}{3.585900in}}%
\pgfpathlineto{\pgfqpoint{2.820311in}{3.581642in}}%
\pgfpathmoveto{\pgfqpoint{2.816053in}{3.585900in}}%
\pgfpathlineto{\pgfqpoint{2.816053in}{3.585900in}}%
\pgfpathlineto{\pgfqpoint{2.816053in}{3.590158in}}%
\pgfpathlineto{\pgfqpoint{2.820311in}{3.590158in}}%
\pgfpathlineto{\pgfqpoint{2.820311in}{3.585900in}}%
\pgfpathmoveto{\pgfqpoint{2.811795in}{3.590158in}}%
\pgfpathlineto{\pgfqpoint{2.811795in}{3.590158in}}%
\pgfpathlineto{\pgfqpoint{2.811795in}{3.594415in}}%
\pgfpathlineto{\pgfqpoint{2.816053in}{3.594415in}}%
\pgfpathlineto{\pgfqpoint{2.816053in}{3.590158in}}%
\pgfpathmoveto{\pgfqpoint{2.816053in}{3.590158in}}%
\pgfpathlineto{\pgfqpoint{2.816053in}{3.590158in}}%
\pgfpathlineto{\pgfqpoint{2.816053in}{3.594415in}}%
\pgfpathlineto{\pgfqpoint{2.820311in}{3.594415in}}%
\pgfpathlineto{\pgfqpoint{2.820311in}{3.590158in}}%
\pgfpathmoveto{\pgfqpoint{2.816053in}{3.594415in}}%
\pgfpathlineto{\pgfqpoint{2.816053in}{3.594415in}}%
\pgfpathlineto{\pgfqpoint{2.816053in}{3.598673in}}%
\pgfpathlineto{\pgfqpoint{2.820311in}{3.598673in}}%
\pgfpathlineto{\pgfqpoint{2.820311in}{3.594415in}}%
\pgfpathmoveto{\pgfqpoint{2.816053in}{3.598673in}}%
\pgfpathlineto{\pgfqpoint{2.816053in}{3.598673in}}%
\pgfpathlineto{\pgfqpoint{2.816053in}{3.602931in}}%
\pgfpathlineto{\pgfqpoint{2.820311in}{3.602931in}}%
\pgfpathlineto{\pgfqpoint{2.820311in}{3.598673in}}%
\pgfpathmoveto{\pgfqpoint{2.820311in}{3.556096in}}%
\pgfpathlineto{\pgfqpoint{2.820311in}{3.556096in}}%
\pgfpathlineto{\pgfqpoint{2.820311in}{3.560354in}}%
\pgfpathlineto{\pgfqpoint{2.824569in}{3.560354in}}%
\pgfpathlineto{\pgfqpoint{2.824569in}{3.556096in}}%
\pgfpathmoveto{\pgfqpoint{2.820311in}{3.560354in}}%
\pgfpathlineto{\pgfqpoint{2.820311in}{3.560354in}}%
\pgfpathlineto{\pgfqpoint{2.820311in}{3.564611in}}%
\pgfpathlineto{\pgfqpoint{2.824569in}{3.564611in}}%
\pgfpathlineto{\pgfqpoint{2.824569in}{3.560354in}}%
\pgfpathmoveto{\pgfqpoint{2.820311in}{3.564611in}}%
\pgfpathlineto{\pgfqpoint{2.820311in}{3.564611in}}%
\pgfpathlineto{\pgfqpoint{2.820311in}{3.568869in}}%
\pgfpathlineto{\pgfqpoint{2.824569in}{3.568869in}}%
\pgfpathlineto{\pgfqpoint{2.824569in}{3.564611in}}%
\pgfpathmoveto{\pgfqpoint{2.820311in}{3.568869in}}%
\pgfpathlineto{\pgfqpoint{2.820311in}{3.568869in}}%
\pgfpathlineto{\pgfqpoint{2.820311in}{3.573127in}}%
\pgfpathlineto{\pgfqpoint{2.824569in}{3.573127in}}%
\pgfpathlineto{\pgfqpoint{2.824569in}{3.568869in}}%
\pgfpathmoveto{\pgfqpoint{2.824569in}{3.568869in}}%
\pgfpathlineto{\pgfqpoint{2.824569in}{3.568869in}}%
\pgfpathlineto{\pgfqpoint{2.824569in}{3.573127in}}%
\pgfpathlineto{\pgfqpoint{2.828827in}{3.573127in}}%
\pgfpathlineto{\pgfqpoint{2.828827in}{3.568869in}}%
\pgfpathmoveto{\pgfqpoint{2.820311in}{3.573127in}}%
\pgfpathlineto{\pgfqpoint{2.820311in}{3.573127in}}%
\pgfpathlineto{\pgfqpoint{2.820311in}{3.577385in}}%
\pgfpathlineto{\pgfqpoint{2.824569in}{3.577385in}}%
\pgfpathlineto{\pgfqpoint{2.824569in}{3.573127in}}%
\pgfpathmoveto{\pgfqpoint{2.820311in}{3.577385in}}%
\pgfpathlineto{\pgfqpoint{2.820311in}{3.577385in}}%
\pgfpathlineto{\pgfqpoint{2.820311in}{3.581642in}}%
\pgfpathlineto{\pgfqpoint{2.824569in}{3.581642in}}%
\pgfpathlineto{\pgfqpoint{2.824569in}{3.577385in}}%
\pgfpathmoveto{\pgfqpoint{2.824569in}{3.573127in}}%
\pgfpathlineto{\pgfqpoint{2.824569in}{3.573127in}}%
\pgfpathlineto{\pgfqpoint{2.824569in}{3.577385in}}%
\pgfpathlineto{\pgfqpoint{2.828827in}{3.577385in}}%
\pgfpathlineto{\pgfqpoint{2.828827in}{3.573127in}}%
\pgfpathmoveto{\pgfqpoint{2.824569in}{3.577385in}}%
\pgfpathlineto{\pgfqpoint{2.824569in}{3.577385in}}%
\pgfpathlineto{\pgfqpoint{2.824569in}{3.581642in}}%
\pgfpathlineto{\pgfqpoint{2.828827in}{3.581642in}}%
\pgfpathlineto{\pgfqpoint{2.828827in}{3.577385in}}%
\pgfpathmoveto{\pgfqpoint{2.820311in}{3.581642in}}%
\pgfpathlineto{\pgfqpoint{2.820311in}{3.581642in}}%
\pgfpathlineto{\pgfqpoint{2.820311in}{3.585900in}}%
\pgfpathlineto{\pgfqpoint{2.824569in}{3.585900in}}%
\pgfpathlineto{\pgfqpoint{2.824569in}{3.581642in}}%
\pgfpathmoveto{\pgfqpoint{2.820311in}{3.585900in}}%
\pgfpathlineto{\pgfqpoint{2.820311in}{3.585900in}}%
\pgfpathlineto{\pgfqpoint{2.820311in}{3.590158in}}%
\pgfpathlineto{\pgfqpoint{2.824569in}{3.590158in}}%
\pgfpathlineto{\pgfqpoint{2.824569in}{3.585900in}}%
\pgfpathmoveto{\pgfqpoint{2.824569in}{3.581642in}}%
\pgfpathlineto{\pgfqpoint{2.824569in}{3.581642in}}%
\pgfpathlineto{\pgfqpoint{2.824569in}{3.585900in}}%
\pgfpathlineto{\pgfqpoint{2.828827in}{3.585900in}}%
\pgfpathlineto{\pgfqpoint{2.828827in}{3.581642in}}%
\pgfpathmoveto{\pgfqpoint{2.824569in}{3.585900in}}%
\pgfpathlineto{\pgfqpoint{2.824569in}{3.585900in}}%
\pgfpathlineto{\pgfqpoint{2.824569in}{3.590158in}}%
\pgfpathlineto{\pgfqpoint{2.828827in}{3.590158in}}%
\pgfpathlineto{\pgfqpoint{2.828827in}{3.585900in}}%
\pgfpathmoveto{\pgfqpoint{2.828827in}{3.577385in}}%
\pgfpathlineto{\pgfqpoint{2.828827in}{3.577385in}}%
\pgfpathlineto{\pgfqpoint{2.828827in}{3.581642in}}%
\pgfpathlineto{\pgfqpoint{2.833085in}{3.581642in}}%
\pgfpathlineto{\pgfqpoint{2.833085in}{3.577385in}}%
\pgfpathmoveto{\pgfqpoint{2.828827in}{3.581642in}}%
\pgfpathlineto{\pgfqpoint{2.828827in}{3.581642in}}%
\pgfpathlineto{\pgfqpoint{2.828827in}{3.585900in}}%
\pgfpathlineto{\pgfqpoint{2.833085in}{3.585900in}}%
\pgfpathlineto{\pgfqpoint{2.833085in}{3.581642in}}%
\pgfpathmoveto{\pgfqpoint{2.828827in}{3.585900in}}%
\pgfpathlineto{\pgfqpoint{2.828827in}{3.585900in}}%
\pgfpathlineto{\pgfqpoint{2.828827in}{3.590158in}}%
\pgfpathlineto{\pgfqpoint{2.833085in}{3.590158in}}%
\pgfpathlineto{\pgfqpoint{2.833085in}{3.585900in}}%
\pgfpathmoveto{\pgfqpoint{2.833085in}{3.585900in}}%
\pgfpathlineto{\pgfqpoint{2.833085in}{3.585900in}}%
\pgfpathlineto{\pgfqpoint{2.833085in}{3.590158in}}%
\pgfpathlineto{\pgfqpoint{2.837342in}{3.590158in}}%
\pgfpathlineto{\pgfqpoint{2.837342in}{3.585900in}}%
\pgfpathmoveto{\pgfqpoint{2.820311in}{3.590158in}}%
\pgfpathlineto{\pgfqpoint{2.820311in}{3.590158in}}%
\pgfpathlineto{\pgfqpoint{2.820311in}{3.594415in}}%
\pgfpathlineto{\pgfqpoint{2.824569in}{3.594415in}}%
\pgfpathlineto{\pgfqpoint{2.824569in}{3.590158in}}%
\pgfpathmoveto{\pgfqpoint{2.820311in}{3.594415in}}%
\pgfpathlineto{\pgfqpoint{2.820311in}{3.594415in}}%
\pgfpathlineto{\pgfqpoint{2.820311in}{3.598673in}}%
\pgfpathlineto{\pgfqpoint{2.824569in}{3.598673in}}%
\pgfpathlineto{\pgfqpoint{2.824569in}{3.594415in}}%
\pgfpathmoveto{\pgfqpoint{2.824569in}{3.590158in}}%
\pgfpathlineto{\pgfqpoint{2.824569in}{3.590158in}}%
\pgfpathlineto{\pgfqpoint{2.824569in}{3.594415in}}%
\pgfpathlineto{\pgfqpoint{2.828827in}{3.594415in}}%
\pgfpathlineto{\pgfqpoint{2.828827in}{3.590158in}}%
\pgfpathmoveto{\pgfqpoint{2.824569in}{3.594415in}}%
\pgfpathlineto{\pgfqpoint{2.824569in}{3.594415in}}%
\pgfpathlineto{\pgfqpoint{2.824569in}{3.598673in}}%
\pgfpathlineto{\pgfqpoint{2.828827in}{3.598673in}}%
\pgfpathlineto{\pgfqpoint{2.828827in}{3.594415in}}%
\pgfpathmoveto{\pgfqpoint{2.820311in}{3.598673in}}%
\pgfpathlineto{\pgfqpoint{2.820311in}{3.598673in}}%
\pgfpathlineto{\pgfqpoint{2.820311in}{3.602931in}}%
\pgfpathlineto{\pgfqpoint{2.824569in}{3.602931in}}%
\pgfpathlineto{\pgfqpoint{2.824569in}{3.598673in}}%
\pgfpathmoveto{\pgfqpoint{2.820311in}{3.602931in}}%
\pgfpathlineto{\pgfqpoint{2.820311in}{3.602931in}}%
\pgfpathlineto{\pgfqpoint{2.820311in}{3.607189in}}%
\pgfpathlineto{\pgfqpoint{2.824569in}{3.607189in}}%
\pgfpathlineto{\pgfqpoint{2.824569in}{3.602931in}}%
\pgfpathmoveto{\pgfqpoint{2.824569in}{3.598673in}}%
\pgfpathlineto{\pgfqpoint{2.824569in}{3.598673in}}%
\pgfpathlineto{\pgfqpoint{2.824569in}{3.602931in}}%
\pgfpathlineto{\pgfqpoint{2.828827in}{3.602931in}}%
\pgfpathlineto{\pgfqpoint{2.828827in}{3.598673in}}%
\pgfpathmoveto{\pgfqpoint{2.824569in}{3.602931in}}%
\pgfpathlineto{\pgfqpoint{2.824569in}{3.602931in}}%
\pgfpathlineto{\pgfqpoint{2.824569in}{3.607189in}}%
\pgfpathlineto{\pgfqpoint{2.828827in}{3.607189in}}%
\pgfpathlineto{\pgfqpoint{2.828827in}{3.602931in}}%
\pgfpathmoveto{\pgfqpoint{2.828827in}{3.590158in}}%
\pgfpathlineto{\pgfqpoint{2.828827in}{3.590158in}}%
\pgfpathlineto{\pgfqpoint{2.828827in}{3.594415in}}%
\pgfpathlineto{\pgfqpoint{2.833085in}{3.594415in}}%
\pgfpathlineto{\pgfqpoint{2.833085in}{3.590158in}}%
\pgfpathmoveto{\pgfqpoint{2.828827in}{3.594415in}}%
\pgfpathlineto{\pgfqpoint{2.828827in}{3.594415in}}%
\pgfpathlineto{\pgfqpoint{2.828827in}{3.598673in}}%
\pgfpathlineto{\pgfqpoint{2.833085in}{3.598673in}}%
\pgfpathlineto{\pgfqpoint{2.833085in}{3.594415in}}%
\pgfpathmoveto{\pgfqpoint{2.833085in}{3.590158in}}%
\pgfpathlineto{\pgfqpoint{2.833085in}{3.590158in}}%
\pgfpathlineto{\pgfqpoint{2.833085in}{3.594415in}}%
\pgfpathlineto{\pgfqpoint{2.837342in}{3.594415in}}%
\pgfpathlineto{\pgfqpoint{2.837342in}{3.590158in}}%
\pgfpathmoveto{\pgfqpoint{2.833085in}{3.594415in}}%
\pgfpathlineto{\pgfqpoint{2.833085in}{3.594415in}}%
\pgfpathlineto{\pgfqpoint{2.833085in}{3.598673in}}%
\pgfpathlineto{\pgfqpoint{2.837342in}{3.598673in}}%
\pgfpathlineto{\pgfqpoint{2.837342in}{3.594415in}}%
\pgfpathmoveto{\pgfqpoint{2.828827in}{3.598673in}}%
\pgfpathlineto{\pgfqpoint{2.828827in}{3.598673in}}%
\pgfpathlineto{\pgfqpoint{2.828827in}{3.602931in}}%
\pgfpathlineto{\pgfqpoint{2.833085in}{3.602931in}}%
\pgfpathlineto{\pgfqpoint{2.833085in}{3.598673in}}%
\pgfpathmoveto{\pgfqpoint{2.828827in}{3.602931in}}%
\pgfpathlineto{\pgfqpoint{2.828827in}{3.602931in}}%
\pgfpathlineto{\pgfqpoint{2.828827in}{3.607189in}}%
\pgfpathlineto{\pgfqpoint{2.833085in}{3.607189in}}%
\pgfpathlineto{\pgfqpoint{2.833085in}{3.602931in}}%
\pgfpathmoveto{\pgfqpoint{2.833085in}{3.598673in}}%
\pgfpathlineto{\pgfqpoint{2.833085in}{3.598673in}}%
\pgfpathlineto{\pgfqpoint{2.833085in}{3.602931in}}%
\pgfpathlineto{\pgfqpoint{2.837342in}{3.602931in}}%
\pgfpathlineto{\pgfqpoint{2.837342in}{3.598673in}}%
\pgfpathmoveto{\pgfqpoint{2.833085in}{3.602931in}}%
\pgfpathlineto{\pgfqpoint{2.833085in}{3.602931in}}%
\pgfpathlineto{\pgfqpoint{2.833085in}{3.607189in}}%
\pgfpathlineto{\pgfqpoint{2.837342in}{3.607189in}}%
\pgfpathlineto{\pgfqpoint{2.837342in}{3.602931in}}%
\pgfpathmoveto{\pgfqpoint{2.837342in}{3.594415in}}%
\pgfpathlineto{\pgfqpoint{2.837342in}{3.594415in}}%
\pgfpathlineto{\pgfqpoint{2.837342in}{3.598673in}}%
\pgfpathlineto{\pgfqpoint{2.841600in}{3.598673in}}%
\pgfpathlineto{\pgfqpoint{2.841600in}{3.594415in}}%
\pgfpathmoveto{\pgfqpoint{2.837342in}{3.598673in}}%
\pgfpathlineto{\pgfqpoint{2.837342in}{3.598673in}}%
\pgfpathlineto{\pgfqpoint{2.837342in}{3.602931in}}%
\pgfpathlineto{\pgfqpoint{2.841600in}{3.602931in}}%
\pgfpathlineto{\pgfqpoint{2.841600in}{3.598673in}}%
\pgfpathmoveto{\pgfqpoint{2.837342in}{3.602931in}}%
\pgfpathlineto{\pgfqpoint{2.837342in}{3.602931in}}%
\pgfpathlineto{\pgfqpoint{2.837342in}{3.607189in}}%
\pgfpathlineto{\pgfqpoint{2.841600in}{3.607189in}}%
\pgfpathlineto{\pgfqpoint{2.841600in}{3.602931in}}%
\pgfpathmoveto{\pgfqpoint{2.820311in}{3.607189in}}%
\pgfpathlineto{\pgfqpoint{2.820311in}{3.607189in}}%
\pgfpathlineto{\pgfqpoint{2.820311in}{3.611446in}}%
\pgfpathlineto{\pgfqpoint{2.824569in}{3.611446in}}%
\pgfpathlineto{\pgfqpoint{2.824569in}{3.607189in}}%
\pgfpathmoveto{\pgfqpoint{2.820311in}{3.611446in}}%
\pgfpathlineto{\pgfqpoint{2.820311in}{3.611446in}}%
\pgfpathlineto{\pgfqpoint{2.820311in}{3.615704in}}%
\pgfpathlineto{\pgfqpoint{2.824569in}{3.615704in}}%
\pgfpathlineto{\pgfqpoint{2.824569in}{3.611446in}}%
\pgfpathmoveto{\pgfqpoint{2.824569in}{3.607189in}}%
\pgfpathlineto{\pgfqpoint{2.824569in}{3.607189in}}%
\pgfpathlineto{\pgfqpoint{2.824569in}{3.611446in}}%
\pgfpathlineto{\pgfqpoint{2.828827in}{3.611446in}}%
\pgfpathlineto{\pgfqpoint{2.828827in}{3.607189in}}%
\pgfpathmoveto{\pgfqpoint{2.824569in}{3.611446in}}%
\pgfpathlineto{\pgfqpoint{2.824569in}{3.611446in}}%
\pgfpathlineto{\pgfqpoint{2.824569in}{3.615704in}}%
\pgfpathlineto{\pgfqpoint{2.828827in}{3.615704in}}%
\pgfpathlineto{\pgfqpoint{2.828827in}{3.611446in}}%
\pgfpathmoveto{\pgfqpoint{2.824569in}{3.615704in}}%
\pgfpathlineto{\pgfqpoint{2.824569in}{3.615704in}}%
\pgfpathlineto{\pgfqpoint{2.824569in}{3.619962in}}%
\pgfpathlineto{\pgfqpoint{2.828827in}{3.619962in}}%
\pgfpathlineto{\pgfqpoint{2.828827in}{3.615704in}}%
\pgfpathmoveto{\pgfqpoint{2.824569in}{3.619962in}}%
\pgfpathlineto{\pgfqpoint{2.824569in}{3.619962in}}%
\pgfpathlineto{\pgfqpoint{2.824569in}{3.624220in}}%
\pgfpathlineto{\pgfqpoint{2.828827in}{3.624220in}}%
\pgfpathlineto{\pgfqpoint{2.828827in}{3.619962in}}%
\pgfpathmoveto{\pgfqpoint{2.828827in}{3.607189in}}%
\pgfpathlineto{\pgfqpoint{2.828827in}{3.607189in}}%
\pgfpathlineto{\pgfqpoint{2.828827in}{3.611446in}}%
\pgfpathlineto{\pgfqpoint{2.833085in}{3.611446in}}%
\pgfpathlineto{\pgfqpoint{2.833085in}{3.607189in}}%
\pgfpathmoveto{\pgfqpoint{2.828827in}{3.611446in}}%
\pgfpathlineto{\pgfqpoint{2.828827in}{3.611446in}}%
\pgfpathlineto{\pgfqpoint{2.828827in}{3.615704in}}%
\pgfpathlineto{\pgfqpoint{2.833085in}{3.615704in}}%
\pgfpathlineto{\pgfqpoint{2.833085in}{3.611446in}}%
\pgfpathmoveto{\pgfqpoint{2.833085in}{3.607189in}}%
\pgfpathlineto{\pgfqpoint{2.833085in}{3.607189in}}%
\pgfpathlineto{\pgfqpoint{2.833085in}{3.611446in}}%
\pgfpathlineto{\pgfqpoint{2.837342in}{3.611446in}}%
\pgfpathlineto{\pgfqpoint{2.837342in}{3.607189in}}%
\pgfpathmoveto{\pgfqpoint{2.833085in}{3.611446in}}%
\pgfpathlineto{\pgfqpoint{2.833085in}{3.611446in}}%
\pgfpathlineto{\pgfqpoint{2.833085in}{3.615704in}}%
\pgfpathlineto{\pgfqpoint{2.837342in}{3.615704in}}%
\pgfpathlineto{\pgfqpoint{2.837342in}{3.611446in}}%
\pgfpathmoveto{\pgfqpoint{2.828827in}{3.615704in}}%
\pgfpathlineto{\pgfqpoint{2.828827in}{3.615704in}}%
\pgfpathlineto{\pgfqpoint{2.828827in}{3.619962in}}%
\pgfpathlineto{\pgfqpoint{2.833085in}{3.619962in}}%
\pgfpathlineto{\pgfqpoint{2.833085in}{3.615704in}}%
\pgfpathmoveto{\pgfqpoint{2.828827in}{3.619962in}}%
\pgfpathlineto{\pgfqpoint{2.828827in}{3.619962in}}%
\pgfpathlineto{\pgfqpoint{2.828827in}{3.624220in}}%
\pgfpathlineto{\pgfqpoint{2.833085in}{3.624220in}}%
\pgfpathlineto{\pgfqpoint{2.833085in}{3.619962in}}%
\pgfpathmoveto{\pgfqpoint{2.833085in}{3.615704in}}%
\pgfpathlineto{\pgfqpoint{2.833085in}{3.615704in}}%
\pgfpathlineto{\pgfqpoint{2.833085in}{3.619962in}}%
\pgfpathlineto{\pgfqpoint{2.837342in}{3.619962in}}%
\pgfpathlineto{\pgfqpoint{2.837342in}{3.615704in}}%
\pgfpathmoveto{\pgfqpoint{2.833085in}{3.619962in}}%
\pgfpathlineto{\pgfqpoint{2.833085in}{3.619962in}}%
\pgfpathlineto{\pgfqpoint{2.833085in}{3.624220in}}%
\pgfpathlineto{\pgfqpoint{2.837342in}{3.624220in}}%
\pgfpathlineto{\pgfqpoint{2.837342in}{3.619962in}}%
\pgfpathmoveto{\pgfqpoint{2.828827in}{3.624220in}}%
\pgfpathlineto{\pgfqpoint{2.828827in}{3.624220in}}%
\pgfpathlineto{\pgfqpoint{2.828827in}{3.628477in}}%
\pgfpathlineto{\pgfqpoint{2.833085in}{3.628477in}}%
\pgfpathlineto{\pgfqpoint{2.833085in}{3.624220in}}%
\pgfpathmoveto{\pgfqpoint{2.828827in}{3.628477in}}%
\pgfpathlineto{\pgfqpoint{2.828827in}{3.628477in}}%
\pgfpathlineto{\pgfqpoint{2.828827in}{3.632735in}}%
\pgfpathlineto{\pgfqpoint{2.833085in}{3.632735in}}%
\pgfpathlineto{\pgfqpoint{2.833085in}{3.628477in}}%
\pgfpathmoveto{\pgfqpoint{2.833085in}{3.624220in}}%
\pgfpathlineto{\pgfqpoint{2.833085in}{3.624220in}}%
\pgfpathlineto{\pgfqpoint{2.833085in}{3.628477in}}%
\pgfpathlineto{\pgfqpoint{2.837342in}{3.628477in}}%
\pgfpathlineto{\pgfqpoint{2.837342in}{3.624220in}}%
\pgfpathmoveto{\pgfqpoint{2.833085in}{3.628477in}}%
\pgfpathlineto{\pgfqpoint{2.833085in}{3.628477in}}%
\pgfpathlineto{\pgfqpoint{2.833085in}{3.632735in}}%
\pgfpathlineto{\pgfqpoint{2.837342in}{3.632735in}}%
\pgfpathlineto{\pgfqpoint{2.837342in}{3.628477in}}%
\pgfpathmoveto{\pgfqpoint{2.833085in}{3.632735in}}%
\pgfpathlineto{\pgfqpoint{2.833085in}{3.632735in}}%
\pgfpathlineto{\pgfqpoint{2.833085in}{3.636993in}}%
\pgfpathlineto{\pgfqpoint{2.837342in}{3.636993in}}%
\pgfpathlineto{\pgfqpoint{2.837342in}{3.632735in}}%
\pgfpathmoveto{\pgfqpoint{2.833085in}{3.636993in}}%
\pgfpathlineto{\pgfqpoint{2.833085in}{3.636993in}}%
\pgfpathlineto{\pgfqpoint{2.833085in}{3.641250in}}%
\pgfpathlineto{\pgfqpoint{2.837342in}{3.641250in}}%
\pgfpathlineto{\pgfqpoint{2.837342in}{3.636993in}}%
\pgfpathmoveto{\pgfqpoint{2.837342in}{3.607189in}}%
\pgfpathlineto{\pgfqpoint{2.837342in}{3.607189in}}%
\pgfpathlineto{\pgfqpoint{2.837342in}{3.611446in}}%
\pgfpathlineto{\pgfqpoint{2.841600in}{3.611446in}}%
\pgfpathlineto{\pgfqpoint{2.841600in}{3.607189in}}%
\pgfpathmoveto{\pgfqpoint{2.837342in}{3.611446in}}%
\pgfpathlineto{\pgfqpoint{2.837342in}{3.611446in}}%
\pgfpathlineto{\pgfqpoint{2.837342in}{3.615704in}}%
\pgfpathlineto{\pgfqpoint{2.841600in}{3.615704in}}%
\pgfpathlineto{\pgfqpoint{2.841600in}{3.611446in}}%
\pgfpathmoveto{\pgfqpoint{2.841600in}{3.607189in}}%
\pgfpathlineto{\pgfqpoint{2.841600in}{3.607189in}}%
\pgfpathlineto{\pgfqpoint{2.841600in}{3.611446in}}%
\pgfpathlineto{\pgfqpoint{2.845858in}{3.611446in}}%
\pgfpathlineto{\pgfqpoint{2.845858in}{3.607189in}}%
\pgfpathmoveto{\pgfqpoint{2.841600in}{3.611446in}}%
\pgfpathlineto{\pgfqpoint{2.841600in}{3.611446in}}%
\pgfpathlineto{\pgfqpoint{2.841600in}{3.615704in}}%
\pgfpathlineto{\pgfqpoint{2.845858in}{3.615704in}}%
\pgfpathlineto{\pgfqpoint{2.845858in}{3.611446in}}%
\pgfpathmoveto{\pgfqpoint{2.837342in}{3.615704in}}%
\pgfpathlineto{\pgfqpoint{2.837342in}{3.615704in}}%
\pgfpathlineto{\pgfqpoint{2.837342in}{3.619962in}}%
\pgfpathlineto{\pgfqpoint{2.841600in}{3.619962in}}%
\pgfpathlineto{\pgfqpoint{2.841600in}{3.615704in}}%
\pgfpathmoveto{\pgfqpoint{2.837342in}{3.619962in}}%
\pgfpathlineto{\pgfqpoint{2.837342in}{3.619962in}}%
\pgfpathlineto{\pgfqpoint{2.837342in}{3.624220in}}%
\pgfpathlineto{\pgfqpoint{2.841600in}{3.624220in}}%
\pgfpathlineto{\pgfqpoint{2.841600in}{3.619962in}}%
\pgfpathmoveto{\pgfqpoint{2.841600in}{3.615704in}}%
\pgfpathlineto{\pgfqpoint{2.841600in}{3.615704in}}%
\pgfpathlineto{\pgfqpoint{2.841600in}{3.619962in}}%
\pgfpathlineto{\pgfqpoint{2.845858in}{3.619962in}}%
\pgfpathlineto{\pgfqpoint{2.845858in}{3.615704in}}%
\pgfpathmoveto{\pgfqpoint{2.841600in}{3.619962in}}%
\pgfpathlineto{\pgfqpoint{2.841600in}{3.619962in}}%
\pgfpathlineto{\pgfqpoint{2.841600in}{3.624220in}}%
\pgfpathlineto{\pgfqpoint{2.845858in}{3.624220in}}%
\pgfpathlineto{\pgfqpoint{2.845858in}{3.619962in}}%
\pgfpathmoveto{\pgfqpoint{2.845858in}{3.615704in}}%
\pgfpathlineto{\pgfqpoint{2.845858in}{3.615704in}}%
\pgfpathlineto{\pgfqpoint{2.845858in}{3.619962in}}%
\pgfpathlineto{\pgfqpoint{2.850116in}{3.619962in}}%
\pgfpathlineto{\pgfqpoint{2.850116in}{3.615704in}}%
\pgfpathmoveto{\pgfqpoint{2.845858in}{3.619962in}}%
\pgfpathlineto{\pgfqpoint{2.845858in}{3.619962in}}%
\pgfpathlineto{\pgfqpoint{2.845858in}{3.624220in}}%
\pgfpathlineto{\pgfqpoint{2.850116in}{3.624220in}}%
\pgfpathlineto{\pgfqpoint{2.850116in}{3.619962in}}%
\pgfpathmoveto{\pgfqpoint{2.837342in}{3.624220in}}%
\pgfpathlineto{\pgfqpoint{2.837342in}{3.624220in}}%
\pgfpathlineto{\pgfqpoint{2.837342in}{3.628477in}}%
\pgfpathlineto{\pgfqpoint{2.841600in}{3.628477in}}%
\pgfpathlineto{\pgfqpoint{2.841600in}{3.624220in}}%
\pgfpathmoveto{\pgfqpoint{2.837342in}{3.628477in}}%
\pgfpathlineto{\pgfqpoint{2.837342in}{3.628477in}}%
\pgfpathlineto{\pgfqpoint{2.837342in}{3.632735in}}%
\pgfpathlineto{\pgfqpoint{2.841600in}{3.632735in}}%
\pgfpathlineto{\pgfqpoint{2.841600in}{3.628477in}}%
\pgfpathmoveto{\pgfqpoint{2.841600in}{3.624220in}}%
\pgfpathlineto{\pgfqpoint{2.841600in}{3.624220in}}%
\pgfpathlineto{\pgfqpoint{2.841600in}{3.628477in}}%
\pgfpathlineto{\pgfqpoint{2.845858in}{3.628477in}}%
\pgfpathlineto{\pgfqpoint{2.845858in}{3.624220in}}%
\pgfpathmoveto{\pgfqpoint{2.841600in}{3.628477in}}%
\pgfpathlineto{\pgfqpoint{2.841600in}{3.628477in}}%
\pgfpathlineto{\pgfqpoint{2.841600in}{3.632735in}}%
\pgfpathlineto{\pgfqpoint{2.845858in}{3.632735in}}%
\pgfpathlineto{\pgfqpoint{2.845858in}{3.628477in}}%
\pgfpathmoveto{\pgfqpoint{2.837342in}{3.632735in}}%
\pgfpathlineto{\pgfqpoint{2.837342in}{3.632735in}}%
\pgfpathlineto{\pgfqpoint{2.837342in}{3.636993in}}%
\pgfpathlineto{\pgfqpoint{2.841600in}{3.636993in}}%
\pgfpathlineto{\pgfqpoint{2.841600in}{3.632735in}}%
\pgfpathmoveto{\pgfqpoint{2.837342in}{3.636993in}}%
\pgfpathlineto{\pgfqpoint{2.837342in}{3.636993in}}%
\pgfpathlineto{\pgfqpoint{2.837342in}{3.641250in}}%
\pgfpathlineto{\pgfqpoint{2.841600in}{3.641250in}}%
\pgfpathlineto{\pgfqpoint{2.841600in}{3.636993in}}%
\pgfpathmoveto{\pgfqpoint{2.841600in}{3.632735in}}%
\pgfpathlineto{\pgfqpoint{2.841600in}{3.632735in}}%
\pgfpathlineto{\pgfqpoint{2.841600in}{3.636993in}}%
\pgfpathlineto{\pgfqpoint{2.845858in}{3.636993in}}%
\pgfpathlineto{\pgfqpoint{2.845858in}{3.632735in}}%
\pgfpathmoveto{\pgfqpoint{2.841600in}{3.636993in}}%
\pgfpathlineto{\pgfqpoint{2.841600in}{3.636993in}}%
\pgfpathlineto{\pgfqpoint{2.841600in}{3.641250in}}%
\pgfpathlineto{\pgfqpoint{2.845858in}{3.641250in}}%
\pgfpathlineto{\pgfqpoint{2.845858in}{3.636993in}}%
\pgfpathmoveto{\pgfqpoint{2.845858in}{3.624220in}}%
\pgfpathlineto{\pgfqpoint{2.845858in}{3.624220in}}%
\pgfpathlineto{\pgfqpoint{2.845858in}{3.628477in}}%
\pgfpathlineto{\pgfqpoint{2.850116in}{3.628477in}}%
\pgfpathlineto{\pgfqpoint{2.850116in}{3.624220in}}%
\pgfpathmoveto{\pgfqpoint{2.845858in}{3.628477in}}%
\pgfpathlineto{\pgfqpoint{2.845858in}{3.628477in}}%
\pgfpathlineto{\pgfqpoint{2.845858in}{3.632735in}}%
\pgfpathlineto{\pgfqpoint{2.850116in}{3.632735in}}%
\pgfpathlineto{\pgfqpoint{2.850116in}{3.628477in}}%
\pgfpathmoveto{\pgfqpoint{2.850116in}{3.624220in}}%
\pgfpathlineto{\pgfqpoint{2.850116in}{3.624220in}}%
\pgfpathlineto{\pgfqpoint{2.850116in}{3.628477in}}%
\pgfpathlineto{\pgfqpoint{2.854374in}{3.628477in}}%
\pgfpathlineto{\pgfqpoint{2.854374in}{3.624220in}}%
\pgfpathmoveto{\pgfqpoint{2.850116in}{3.628477in}}%
\pgfpathlineto{\pgfqpoint{2.850116in}{3.628477in}}%
\pgfpathlineto{\pgfqpoint{2.850116in}{3.632735in}}%
\pgfpathlineto{\pgfqpoint{2.854374in}{3.632735in}}%
\pgfpathlineto{\pgfqpoint{2.854374in}{3.628477in}}%
\pgfpathmoveto{\pgfqpoint{2.845858in}{3.632735in}}%
\pgfpathlineto{\pgfqpoint{2.845858in}{3.632735in}}%
\pgfpathlineto{\pgfqpoint{2.845858in}{3.636993in}}%
\pgfpathlineto{\pgfqpoint{2.850116in}{3.636993in}}%
\pgfpathlineto{\pgfqpoint{2.850116in}{3.632735in}}%
\pgfpathmoveto{\pgfqpoint{2.845858in}{3.636993in}}%
\pgfpathlineto{\pgfqpoint{2.845858in}{3.636993in}}%
\pgfpathlineto{\pgfqpoint{2.845858in}{3.641250in}}%
\pgfpathlineto{\pgfqpoint{2.850116in}{3.641250in}}%
\pgfpathlineto{\pgfqpoint{2.850116in}{3.636993in}}%
\pgfpathmoveto{\pgfqpoint{2.850116in}{3.632735in}}%
\pgfpathlineto{\pgfqpoint{2.850116in}{3.632735in}}%
\pgfpathlineto{\pgfqpoint{2.850116in}{3.636993in}}%
\pgfpathlineto{\pgfqpoint{2.854374in}{3.636993in}}%
\pgfpathlineto{\pgfqpoint{2.854374in}{3.632735in}}%
\pgfpathmoveto{\pgfqpoint{2.850116in}{3.636993in}}%
\pgfpathlineto{\pgfqpoint{2.850116in}{3.636993in}}%
\pgfpathlineto{\pgfqpoint{2.850116in}{3.641250in}}%
\pgfpathlineto{\pgfqpoint{2.854374in}{3.641250in}}%
\pgfpathlineto{\pgfqpoint{2.854374in}{3.636993in}}%
\pgfpathmoveto{\pgfqpoint{2.854374in}{3.632735in}}%
\pgfpathlineto{\pgfqpoint{2.854374in}{3.632735in}}%
\pgfpathlineto{\pgfqpoint{2.854374in}{3.636993in}}%
\pgfpathlineto{\pgfqpoint{2.858631in}{3.636993in}}%
\pgfpathlineto{\pgfqpoint{2.858631in}{3.632735in}}%
\pgfpathmoveto{\pgfqpoint{2.854374in}{3.636993in}}%
\pgfpathlineto{\pgfqpoint{2.854374in}{3.636993in}}%
\pgfpathlineto{\pgfqpoint{2.854374in}{3.641250in}}%
\pgfpathlineto{\pgfqpoint{2.858631in}{3.641250in}}%
\pgfpathlineto{\pgfqpoint{2.858631in}{3.636993in}}%
\pgfpathmoveto{\pgfqpoint{2.858631in}{3.636993in}}%
\pgfpathlineto{\pgfqpoint{2.858631in}{3.636993in}}%
\pgfpathlineto{\pgfqpoint{2.858631in}{3.641250in}}%
\pgfpathlineto{\pgfqpoint{2.862889in}{3.641250in}}%
\pgfpathlineto{\pgfqpoint{2.862889in}{3.636993in}}%
\pgfpathmoveto{\pgfqpoint{2.837342in}{3.641250in}}%
\pgfpathlineto{\pgfqpoint{2.837342in}{3.641250in}}%
\pgfpathlineto{\pgfqpoint{2.837342in}{3.645508in}}%
\pgfpathlineto{\pgfqpoint{2.841600in}{3.645508in}}%
\pgfpathlineto{\pgfqpoint{2.841600in}{3.641250in}}%
\pgfpathmoveto{\pgfqpoint{2.837342in}{3.645508in}}%
\pgfpathlineto{\pgfqpoint{2.837342in}{3.645508in}}%
\pgfpathlineto{\pgfqpoint{2.837342in}{3.649766in}}%
\pgfpathlineto{\pgfqpoint{2.841600in}{3.649766in}}%
\pgfpathlineto{\pgfqpoint{2.841600in}{3.645508in}}%
\pgfpathmoveto{\pgfqpoint{2.841600in}{3.641250in}}%
\pgfpathlineto{\pgfqpoint{2.841600in}{3.641250in}}%
\pgfpathlineto{\pgfqpoint{2.841600in}{3.645508in}}%
\pgfpathlineto{\pgfqpoint{2.845858in}{3.645508in}}%
\pgfpathlineto{\pgfqpoint{2.845858in}{3.641250in}}%
\pgfpathmoveto{\pgfqpoint{2.841600in}{3.645508in}}%
\pgfpathlineto{\pgfqpoint{2.841600in}{3.645508in}}%
\pgfpathlineto{\pgfqpoint{2.841600in}{3.649766in}}%
\pgfpathlineto{\pgfqpoint{2.845858in}{3.649766in}}%
\pgfpathlineto{\pgfqpoint{2.845858in}{3.645508in}}%
\pgfpathmoveto{\pgfqpoint{2.837342in}{3.649766in}}%
\pgfpathlineto{\pgfqpoint{2.837342in}{3.649766in}}%
\pgfpathlineto{\pgfqpoint{2.837342in}{3.654024in}}%
\pgfpathlineto{\pgfqpoint{2.841600in}{3.654024in}}%
\pgfpathlineto{\pgfqpoint{2.841600in}{3.649766in}}%
\pgfpathmoveto{\pgfqpoint{2.841600in}{3.649766in}}%
\pgfpathlineto{\pgfqpoint{2.841600in}{3.649766in}}%
\pgfpathlineto{\pgfqpoint{2.841600in}{3.654024in}}%
\pgfpathlineto{\pgfqpoint{2.845858in}{3.654024in}}%
\pgfpathlineto{\pgfqpoint{2.845858in}{3.649766in}}%
\pgfpathmoveto{\pgfqpoint{2.841600in}{3.654024in}}%
\pgfpathlineto{\pgfqpoint{2.841600in}{3.654024in}}%
\pgfpathlineto{\pgfqpoint{2.841600in}{3.658282in}}%
\pgfpathlineto{\pgfqpoint{2.845858in}{3.658282in}}%
\pgfpathlineto{\pgfqpoint{2.845858in}{3.654024in}}%
\pgfpathmoveto{\pgfqpoint{2.845858in}{3.641250in}}%
\pgfpathlineto{\pgfqpoint{2.845858in}{3.641250in}}%
\pgfpathlineto{\pgfqpoint{2.845858in}{3.645508in}}%
\pgfpathlineto{\pgfqpoint{2.850116in}{3.645508in}}%
\pgfpathlineto{\pgfqpoint{2.850116in}{3.641250in}}%
\pgfpathmoveto{\pgfqpoint{2.845858in}{3.645508in}}%
\pgfpathlineto{\pgfqpoint{2.845858in}{3.645508in}}%
\pgfpathlineto{\pgfqpoint{2.845858in}{3.649766in}}%
\pgfpathlineto{\pgfqpoint{2.850116in}{3.649766in}}%
\pgfpathlineto{\pgfqpoint{2.850116in}{3.645508in}}%
\pgfpathmoveto{\pgfqpoint{2.850116in}{3.641250in}}%
\pgfpathlineto{\pgfqpoint{2.850116in}{3.641250in}}%
\pgfpathlineto{\pgfqpoint{2.850116in}{3.645508in}}%
\pgfpathlineto{\pgfqpoint{2.854374in}{3.645508in}}%
\pgfpathlineto{\pgfqpoint{2.854374in}{3.641250in}}%
\pgfpathmoveto{\pgfqpoint{2.850116in}{3.645508in}}%
\pgfpathlineto{\pgfqpoint{2.850116in}{3.645508in}}%
\pgfpathlineto{\pgfqpoint{2.850116in}{3.649766in}}%
\pgfpathlineto{\pgfqpoint{2.854374in}{3.649766in}}%
\pgfpathlineto{\pgfqpoint{2.854374in}{3.645508in}}%
\pgfpathmoveto{\pgfqpoint{2.845858in}{3.649766in}}%
\pgfpathlineto{\pgfqpoint{2.845858in}{3.649766in}}%
\pgfpathlineto{\pgfqpoint{2.845858in}{3.654024in}}%
\pgfpathlineto{\pgfqpoint{2.850116in}{3.654024in}}%
\pgfpathlineto{\pgfqpoint{2.850116in}{3.649766in}}%
\pgfpathmoveto{\pgfqpoint{2.845858in}{3.654024in}}%
\pgfpathlineto{\pgfqpoint{2.845858in}{3.654024in}}%
\pgfpathlineto{\pgfqpoint{2.845858in}{3.658282in}}%
\pgfpathlineto{\pgfqpoint{2.850116in}{3.658282in}}%
\pgfpathlineto{\pgfqpoint{2.850116in}{3.654024in}}%
\pgfpathmoveto{\pgfqpoint{2.850116in}{3.649766in}}%
\pgfpathlineto{\pgfqpoint{2.850116in}{3.649766in}}%
\pgfpathlineto{\pgfqpoint{2.850116in}{3.654024in}}%
\pgfpathlineto{\pgfqpoint{2.854374in}{3.654024in}}%
\pgfpathlineto{\pgfqpoint{2.854374in}{3.649766in}}%
\pgfpathmoveto{\pgfqpoint{2.850116in}{3.654024in}}%
\pgfpathlineto{\pgfqpoint{2.850116in}{3.654024in}}%
\pgfpathlineto{\pgfqpoint{2.850116in}{3.658282in}}%
\pgfpathlineto{\pgfqpoint{2.854374in}{3.658282in}}%
\pgfpathlineto{\pgfqpoint{2.854374in}{3.654024in}}%
\pgfpathmoveto{\pgfqpoint{2.841600in}{3.658282in}}%
\pgfpathlineto{\pgfqpoint{2.841600in}{3.658282in}}%
\pgfpathlineto{\pgfqpoint{2.841600in}{3.662540in}}%
\pgfpathlineto{\pgfqpoint{2.845858in}{3.662540in}}%
\pgfpathlineto{\pgfqpoint{2.845858in}{3.658282in}}%
\pgfpathmoveto{\pgfqpoint{2.845858in}{3.658282in}}%
\pgfpathlineto{\pgfqpoint{2.845858in}{3.658282in}}%
\pgfpathlineto{\pgfqpoint{2.845858in}{3.662540in}}%
\pgfpathlineto{\pgfqpoint{2.850116in}{3.662540in}}%
\pgfpathlineto{\pgfqpoint{2.850116in}{3.658282in}}%
\pgfpathmoveto{\pgfqpoint{2.845858in}{3.662540in}}%
\pgfpathlineto{\pgfqpoint{2.845858in}{3.662540in}}%
\pgfpathlineto{\pgfqpoint{2.845858in}{3.666797in}}%
\pgfpathlineto{\pgfqpoint{2.850116in}{3.666797in}}%
\pgfpathlineto{\pgfqpoint{2.850116in}{3.662540in}}%
\pgfpathmoveto{\pgfqpoint{2.850116in}{3.658282in}}%
\pgfpathlineto{\pgfqpoint{2.850116in}{3.658282in}}%
\pgfpathlineto{\pgfqpoint{2.850116in}{3.662540in}}%
\pgfpathlineto{\pgfqpoint{2.854374in}{3.662540in}}%
\pgfpathlineto{\pgfqpoint{2.854374in}{3.658282in}}%
\pgfpathmoveto{\pgfqpoint{2.850116in}{3.662540in}}%
\pgfpathlineto{\pgfqpoint{2.850116in}{3.662540in}}%
\pgfpathlineto{\pgfqpoint{2.850116in}{3.666797in}}%
\pgfpathlineto{\pgfqpoint{2.854374in}{3.666797in}}%
\pgfpathlineto{\pgfqpoint{2.854374in}{3.662540in}}%
\pgfpathmoveto{\pgfqpoint{2.845858in}{3.666797in}}%
\pgfpathlineto{\pgfqpoint{2.845858in}{3.666797in}}%
\pgfpathlineto{\pgfqpoint{2.845858in}{3.671055in}}%
\pgfpathlineto{\pgfqpoint{2.850116in}{3.671055in}}%
\pgfpathlineto{\pgfqpoint{2.850116in}{3.666797in}}%
\pgfpathmoveto{\pgfqpoint{2.850116in}{3.666797in}}%
\pgfpathlineto{\pgfqpoint{2.850116in}{3.666797in}}%
\pgfpathlineto{\pgfqpoint{2.850116in}{3.671055in}}%
\pgfpathlineto{\pgfqpoint{2.854374in}{3.671055in}}%
\pgfpathlineto{\pgfqpoint{2.854374in}{3.666797in}}%
\pgfpathmoveto{\pgfqpoint{2.850116in}{3.671055in}}%
\pgfpathlineto{\pgfqpoint{2.850116in}{3.671055in}}%
\pgfpathlineto{\pgfqpoint{2.850116in}{3.675313in}}%
\pgfpathlineto{\pgfqpoint{2.854374in}{3.675313in}}%
\pgfpathlineto{\pgfqpoint{2.854374in}{3.671055in}}%
\pgfpathmoveto{\pgfqpoint{2.850116in}{3.675313in}}%
\pgfpathlineto{\pgfqpoint{2.850116in}{3.675313in}}%
\pgfpathlineto{\pgfqpoint{2.850116in}{3.679571in}}%
\pgfpathlineto{\pgfqpoint{2.854374in}{3.679571in}}%
\pgfpathlineto{\pgfqpoint{2.854374in}{3.675313in}}%
\pgfpathmoveto{\pgfqpoint{2.854374in}{3.641250in}}%
\pgfpathlineto{\pgfqpoint{2.854374in}{3.641250in}}%
\pgfpathlineto{\pgfqpoint{2.854374in}{3.645508in}}%
\pgfpathlineto{\pgfqpoint{2.858631in}{3.645508in}}%
\pgfpathlineto{\pgfqpoint{2.858631in}{3.641250in}}%
\pgfpathmoveto{\pgfqpoint{2.854374in}{3.645508in}}%
\pgfpathlineto{\pgfqpoint{2.854374in}{3.645508in}}%
\pgfpathlineto{\pgfqpoint{2.854374in}{3.649766in}}%
\pgfpathlineto{\pgfqpoint{2.858631in}{3.649766in}}%
\pgfpathlineto{\pgfqpoint{2.858631in}{3.645508in}}%
\pgfpathmoveto{\pgfqpoint{2.858631in}{3.641250in}}%
\pgfpathlineto{\pgfqpoint{2.858631in}{3.641250in}}%
\pgfpathlineto{\pgfqpoint{2.858631in}{3.645508in}}%
\pgfpathlineto{\pgfqpoint{2.862889in}{3.645508in}}%
\pgfpathlineto{\pgfqpoint{2.862889in}{3.641250in}}%
\pgfpathmoveto{\pgfqpoint{2.858631in}{3.645508in}}%
\pgfpathlineto{\pgfqpoint{2.858631in}{3.645508in}}%
\pgfpathlineto{\pgfqpoint{2.858631in}{3.649766in}}%
\pgfpathlineto{\pgfqpoint{2.862889in}{3.649766in}}%
\pgfpathlineto{\pgfqpoint{2.862889in}{3.645508in}}%
\pgfpathmoveto{\pgfqpoint{2.854374in}{3.649766in}}%
\pgfpathlineto{\pgfqpoint{2.854374in}{3.649766in}}%
\pgfpathlineto{\pgfqpoint{2.854374in}{3.654024in}}%
\pgfpathlineto{\pgfqpoint{2.858631in}{3.654024in}}%
\pgfpathlineto{\pgfqpoint{2.858631in}{3.649766in}}%
\pgfpathmoveto{\pgfqpoint{2.854374in}{3.654024in}}%
\pgfpathlineto{\pgfqpoint{2.854374in}{3.654024in}}%
\pgfpathlineto{\pgfqpoint{2.854374in}{3.658282in}}%
\pgfpathlineto{\pgfqpoint{2.858631in}{3.658282in}}%
\pgfpathlineto{\pgfqpoint{2.858631in}{3.654024in}}%
\pgfpathmoveto{\pgfqpoint{2.858631in}{3.649766in}}%
\pgfpathlineto{\pgfqpoint{2.858631in}{3.649766in}}%
\pgfpathlineto{\pgfqpoint{2.858631in}{3.654024in}}%
\pgfpathlineto{\pgfqpoint{2.862889in}{3.654024in}}%
\pgfpathlineto{\pgfqpoint{2.862889in}{3.649766in}}%
\pgfpathmoveto{\pgfqpoint{2.858631in}{3.654024in}}%
\pgfpathlineto{\pgfqpoint{2.858631in}{3.654024in}}%
\pgfpathlineto{\pgfqpoint{2.858631in}{3.658282in}}%
\pgfpathlineto{\pgfqpoint{2.862889in}{3.658282in}}%
\pgfpathlineto{\pgfqpoint{2.862889in}{3.654024in}}%
\pgfpathmoveto{\pgfqpoint{2.862889in}{3.645508in}}%
\pgfpathlineto{\pgfqpoint{2.862889in}{3.645508in}}%
\pgfpathlineto{\pgfqpoint{2.862889in}{3.649766in}}%
\pgfpathlineto{\pgfqpoint{2.867147in}{3.649766in}}%
\pgfpathlineto{\pgfqpoint{2.867147in}{3.645508in}}%
\pgfpathmoveto{\pgfqpoint{2.862889in}{3.649766in}}%
\pgfpathlineto{\pgfqpoint{2.862889in}{3.649766in}}%
\pgfpathlineto{\pgfqpoint{2.862889in}{3.654024in}}%
\pgfpathlineto{\pgfqpoint{2.867147in}{3.654024in}}%
\pgfpathlineto{\pgfqpoint{2.867147in}{3.649766in}}%
\pgfpathmoveto{\pgfqpoint{2.862889in}{3.654024in}}%
\pgfpathlineto{\pgfqpoint{2.862889in}{3.654024in}}%
\pgfpathlineto{\pgfqpoint{2.862889in}{3.658282in}}%
\pgfpathlineto{\pgfqpoint{2.867147in}{3.658282in}}%
\pgfpathlineto{\pgfqpoint{2.867147in}{3.654024in}}%
\pgfpathmoveto{\pgfqpoint{2.867147in}{3.654024in}}%
\pgfpathlineto{\pgfqpoint{2.867147in}{3.654024in}}%
\pgfpathlineto{\pgfqpoint{2.867147in}{3.658282in}}%
\pgfpathlineto{\pgfqpoint{2.871405in}{3.658282in}}%
\pgfpathlineto{\pgfqpoint{2.871405in}{3.654024in}}%
\pgfpathmoveto{\pgfqpoint{2.854374in}{3.658282in}}%
\pgfpathlineto{\pgfqpoint{2.854374in}{3.658282in}}%
\pgfpathlineto{\pgfqpoint{2.854374in}{3.662540in}}%
\pgfpathlineto{\pgfqpoint{2.858631in}{3.662540in}}%
\pgfpathlineto{\pgfqpoint{2.858631in}{3.658282in}}%
\pgfpathmoveto{\pgfqpoint{2.854374in}{3.662540in}}%
\pgfpathlineto{\pgfqpoint{2.854374in}{3.662540in}}%
\pgfpathlineto{\pgfqpoint{2.854374in}{3.666797in}}%
\pgfpathlineto{\pgfqpoint{2.858631in}{3.666797in}}%
\pgfpathlineto{\pgfqpoint{2.858631in}{3.662540in}}%
\pgfpathmoveto{\pgfqpoint{2.858631in}{3.658282in}}%
\pgfpathlineto{\pgfqpoint{2.858631in}{3.658282in}}%
\pgfpathlineto{\pgfqpoint{2.858631in}{3.662540in}}%
\pgfpathlineto{\pgfqpoint{2.862889in}{3.662540in}}%
\pgfpathlineto{\pgfqpoint{2.862889in}{3.658282in}}%
\pgfpathmoveto{\pgfqpoint{2.858631in}{3.662540in}}%
\pgfpathlineto{\pgfqpoint{2.858631in}{3.662540in}}%
\pgfpathlineto{\pgfqpoint{2.858631in}{3.666797in}}%
\pgfpathlineto{\pgfqpoint{2.862889in}{3.666797in}}%
\pgfpathlineto{\pgfqpoint{2.862889in}{3.662540in}}%
\pgfpathmoveto{\pgfqpoint{2.854374in}{3.666797in}}%
\pgfpathlineto{\pgfqpoint{2.854374in}{3.666797in}}%
\pgfpathlineto{\pgfqpoint{2.854374in}{3.671055in}}%
\pgfpathlineto{\pgfqpoint{2.858631in}{3.671055in}}%
\pgfpathlineto{\pgfqpoint{2.858631in}{3.666797in}}%
\pgfpathmoveto{\pgfqpoint{2.854374in}{3.671055in}}%
\pgfpathlineto{\pgfqpoint{2.854374in}{3.671055in}}%
\pgfpathlineto{\pgfqpoint{2.854374in}{3.675313in}}%
\pgfpathlineto{\pgfqpoint{2.858631in}{3.675313in}}%
\pgfpathlineto{\pgfqpoint{2.858631in}{3.671055in}}%
\pgfpathmoveto{\pgfqpoint{2.858631in}{3.666797in}}%
\pgfpathlineto{\pgfqpoint{2.858631in}{3.666797in}}%
\pgfpathlineto{\pgfqpoint{2.858631in}{3.671055in}}%
\pgfpathlineto{\pgfqpoint{2.862889in}{3.671055in}}%
\pgfpathlineto{\pgfqpoint{2.862889in}{3.666797in}}%
\pgfpathmoveto{\pgfqpoint{2.858631in}{3.671055in}}%
\pgfpathlineto{\pgfqpoint{2.858631in}{3.671055in}}%
\pgfpathlineto{\pgfqpoint{2.858631in}{3.675313in}}%
\pgfpathlineto{\pgfqpoint{2.862889in}{3.675313in}}%
\pgfpathlineto{\pgfqpoint{2.862889in}{3.671055in}}%
\pgfpathmoveto{\pgfqpoint{2.862889in}{3.658282in}}%
\pgfpathlineto{\pgfqpoint{2.862889in}{3.658282in}}%
\pgfpathlineto{\pgfqpoint{2.862889in}{3.662540in}}%
\pgfpathlineto{\pgfqpoint{2.867147in}{3.662540in}}%
\pgfpathlineto{\pgfqpoint{2.867147in}{3.658282in}}%
\pgfpathmoveto{\pgfqpoint{2.862889in}{3.662540in}}%
\pgfpathlineto{\pgfqpoint{2.862889in}{3.662540in}}%
\pgfpathlineto{\pgfqpoint{2.862889in}{3.666797in}}%
\pgfpathlineto{\pgfqpoint{2.867147in}{3.666797in}}%
\pgfpathlineto{\pgfqpoint{2.867147in}{3.662540in}}%
\pgfpathmoveto{\pgfqpoint{2.867147in}{3.658282in}}%
\pgfpathlineto{\pgfqpoint{2.867147in}{3.658282in}}%
\pgfpathlineto{\pgfqpoint{2.867147in}{3.662540in}}%
\pgfpathlineto{\pgfqpoint{2.871405in}{3.662540in}}%
\pgfpathlineto{\pgfqpoint{2.871405in}{3.658282in}}%
\pgfpathmoveto{\pgfqpoint{2.867147in}{3.662540in}}%
\pgfpathlineto{\pgfqpoint{2.867147in}{3.662540in}}%
\pgfpathlineto{\pgfqpoint{2.867147in}{3.666797in}}%
\pgfpathlineto{\pgfqpoint{2.871405in}{3.666797in}}%
\pgfpathlineto{\pgfqpoint{2.871405in}{3.662540in}}%
\pgfpathmoveto{\pgfqpoint{2.862889in}{3.666797in}}%
\pgfpathlineto{\pgfqpoint{2.862889in}{3.666797in}}%
\pgfpathlineto{\pgfqpoint{2.862889in}{3.671055in}}%
\pgfpathlineto{\pgfqpoint{2.867147in}{3.671055in}}%
\pgfpathlineto{\pgfqpoint{2.867147in}{3.666797in}}%
\pgfpathmoveto{\pgfqpoint{2.862889in}{3.671055in}}%
\pgfpathlineto{\pgfqpoint{2.862889in}{3.671055in}}%
\pgfpathlineto{\pgfqpoint{2.862889in}{3.675313in}}%
\pgfpathlineto{\pgfqpoint{2.867147in}{3.675313in}}%
\pgfpathlineto{\pgfqpoint{2.867147in}{3.671055in}}%
\pgfpathmoveto{\pgfqpoint{2.867147in}{3.666797in}}%
\pgfpathlineto{\pgfqpoint{2.867147in}{3.666797in}}%
\pgfpathlineto{\pgfqpoint{2.867147in}{3.671055in}}%
\pgfpathlineto{\pgfqpoint{2.871405in}{3.671055in}}%
\pgfpathlineto{\pgfqpoint{2.871405in}{3.666797in}}%
\pgfpathmoveto{\pgfqpoint{2.867147in}{3.671055in}}%
\pgfpathlineto{\pgfqpoint{2.867147in}{3.671055in}}%
\pgfpathlineto{\pgfqpoint{2.867147in}{3.675313in}}%
\pgfpathlineto{\pgfqpoint{2.871405in}{3.675313in}}%
\pgfpathlineto{\pgfqpoint{2.871405in}{3.671055in}}%
\pgfpathmoveto{\pgfqpoint{2.871405in}{3.658282in}}%
\pgfpathlineto{\pgfqpoint{2.871405in}{3.658282in}}%
\pgfpathlineto{\pgfqpoint{2.871405in}{3.662540in}}%
\pgfpathlineto{\pgfqpoint{2.875663in}{3.662540in}}%
\pgfpathlineto{\pgfqpoint{2.875663in}{3.658282in}}%
\pgfpathmoveto{\pgfqpoint{2.871405in}{3.662540in}}%
\pgfpathlineto{\pgfqpoint{2.871405in}{3.662540in}}%
\pgfpathlineto{\pgfqpoint{2.871405in}{3.666797in}}%
\pgfpathlineto{\pgfqpoint{2.875663in}{3.666797in}}%
\pgfpathlineto{\pgfqpoint{2.875663in}{3.662540in}}%
\pgfpathmoveto{\pgfqpoint{2.871405in}{3.666797in}}%
\pgfpathlineto{\pgfqpoint{2.871405in}{3.666797in}}%
\pgfpathlineto{\pgfqpoint{2.871405in}{3.671055in}}%
\pgfpathlineto{\pgfqpoint{2.875663in}{3.671055in}}%
\pgfpathlineto{\pgfqpoint{2.875663in}{3.666797in}}%
\pgfpathmoveto{\pgfqpoint{2.871405in}{3.671055in}}%
\pgfpathlineto{\pgfqpoint{2.871405in}{3.671055in}}%
\pgfpathlineto{\pgfqpoint{2.871405in}{3.675313in}}%
\pgfpathlineto{\pgfqpoint{2.875663in}{3.675313in}}%
\pgfpathlineto{\pgfqpoint{2.875663in}{3.671055in}}%
\pgfpathmoveto{\pgfqpoint{2.875663in}{3.666797in}}%
\pgfpathlineto{\pgfqpoint{2.875663in}{3.666797in}}%
\pgfpathlineto{\pgfqpoint{2.875663in}{3.671055in}}%
\pgfpathlineto{\pgfqpoint{2.879921in}{3.671055in}}%
\pgfpathlineto{\pgfqpoint{2.879921in}{3.666797in}}%
\pgfpathmoveto{\pgfqpoint{2.875663in}{3.671055in}}%
\pgfpathlineto{\pgfqpoint{2.875663in}{3.671055in}}%
\pgfpathlineto{\pgfqpoint{2.875663in}{3.675313in}}%
\pgfpathlineto{\pgfqpoint{2.879921in}{3.675313in}}%
\pgfpathlineto{\pgfqpoint{2.879921in}{3.671055in}}%
\pgfpathmoveto{\pgfqpoint{2.879921in}{3.671055in}}%
\pgfpathlineto{\pgfqpoint{2.879921in}{3.671055in}}%
\pgfpathlineto{\pgfqpoint{2.879921in}{3.675313in}}%
\pgfpathlineto{\pgfqpoint{2.884178in}{3.675313in}}%
\pgfpathlineto{\pgfqpoint{2.884178in}{3.671055in}}%
\pgfpathmoveto{\pgfqpoint{2.854374in}{3.675313in}}%
\pgfpathlineto{\pgfqpoint{2.854374in}{3.675313in}}%
\pgfpathlineto{\pgfqpoint{2.854374in}{3.679571in}}%
\pgfpathlineto{\pgfqpoint{2.858631in}{3.679571in}}%
\pgfpathlineto{\pgfqpoint{2.858631in}{3.675313in}}%
\pgfpathmoveto{\pgfqpoint{2.854374in}{3.679571in}}%
\pgfpathlineto{\pgfqpoint{2.854374in}{3.679571in}}%
\pgfpathlineto{\pgfqpoint{2.854374in}{3.683829in}}%
\pgfpathlineto{\pgfqpoint{2.858631in}{3.683829in}}%
\pgfpathlineto{\pgfqpoint{2.858631in}{3.679571in}}%
\pgfpathmoveto{\pgfqpoint{2.858631in}{3.675313in}}%
\pgfpathlineto{\pgfqpoint{2.858631in}{3.675313in}}%
\pgfpathlineto{\pgfqpoint{2.858631in}{3.679571in}}%
\pgfpathlineto{\pgfqpoint{2.862889in}{3.679571in}}%
\pgfpathlineto{\pgfqpoint{2.862889in}{3.675313in}}%
\pgfpathmoveto{\pgfqpoint{2.858631in}{3.679571in}}%
\pgfpathlineto{\pgfqpoint{2.858631in}{3.679571in}}%
\pgfpathlineto{\pgfqpoint{2.858631in}{3.683829in}}%
\pgfpathlineto{\pgfqpoint{2.862889in}{3.683829in}}%
\pgfpathlineto{\pgfqpoint{2.862889in}{3.679571in}}%
\pgfpathmoveto{\pgfqpoint{2.858631in}{3.683829in}}%
\pgfpathlineto{\pgfqpoint{2.858631in}{3.683829in}}%
\pgfpathlineto{\pgfqpoint{2.858631in}{3.688087in}}%
\pgfpathlineto{\pgfqpoint{2.862889in}{3.688087in}}%
\pgfpathlineto{\pgfqpoint{2.862889in}{3.683829in}}%
\pgfpathmoveto{\pgfqpoint{2.858631in}{3.688087in}}%
\pgfpathlineto{\pgfqpoint{2.858631in}{3.688087in}}%
\pgfpathlineto{\pgfqpoint{2.858631in}{3.692345in}}%
\pgfpathlineto{\pgfqpoint{2.862889in}{3.692345in}}%
\pgfpathlineto{\pgfqpoint{2.862889in}{3.688087in}}%
\pgfpathmoveto{\pgfqpoint{2.862889in}{3.675313in}}%
\pgfpathlineto{\pgfqpoint{2.862889in}{3.675313in}}%
\pgfpathlineto{\pgfqpoint{2.862889in}{3.679571in}}%
\pgfpathlineto{\pgfqpoint{2.867147in}{3.679571in}}%
\pgfpathlineto{\pgfqpoint{2.867147in}{3.675313in}}%
\pgfpathmoveto{\pgfqpoint{2.862889in}{3.679571in}}%
\pgfpathlineto{\pgfqpoint{2.862889in}{3.679571in}}%
\pgfpathlineto{\pgfqpoint{2.862889in}{3.683829in}}%
\pgfpathlineto{\pgfqpoint{2.867147in}{3.683829in}}%
\pgfpathlineto{\pgfqpoint{2.867147in}{3.679571in}}%
\pgfpathmoveto{\pgfqpoint{2.867147in}{3.675313in}}%
\pgfpathlineto{\pgfqpoint{2.867147in}{3.675313in}}%
\pgfpathlineto{\pgfqpoint{2.867147in}{3.679571in}}%
\pgfpathlineto{\pgfqpoint{2.871405in}{3.679571in}}%
\pgfpathlineto{\pgfqpoint{2.871405in}{3.675313in}}%
\pgfpathmoveto{\pgfqpoint{2.867147in}{3.679571in}}%
\pgfpathlineto{\pgfqpoint{2.867147in}{3.679571in}}%
\pgfpathlineto{\pgfqpoint{2.867147in}{3.683829in}}%
\pgfpathlineto{\pgfqpoint{2.871405in}{3.683829in}}%
\pgfpathlineto{\pgfqpoint{2.871405in}{3.679571in}}%
\pgfpathmoveto{\pgfqpoint{2.862889in}{3.683829in}}%
\pgfpathlineto{\pgfqpoint{2.862889in}{3.683829in}}%
\pgfpathlineto{\pgfqpoint{2.862889in}{3.688087in}}%
\pgfpathlineto{\pgfqpoint{2.867147in}{3.688087in}}%
\pgfpathlineto{\pgfqpoint{2.867147in}{3.683829in}}%
\pgfpathmoveto{\pgfqpoint{2.862889in}{3.688087in}}%
\pgfpathlineto{\pgfqpoint{2.862889in}{3.688087in}}%
\pgfpathlineto{\pgfqpoint{2.862889in}{3.692345in}}%
\pgfpathlineto{\pgfqpoint{2.867147in}{3.692345in}}%
\pgfpathlineto{\pgfqpoint{2.867147in}{3.688087in}}%
\pgfpathmoveto{\pgfqpoint{2.867147in}{3.683829in}}%
\pgfpathlineto{\pgfqpoint{2.867147in}{3.683829in}}%
\pgfpathlineto{\pgfqpoint{2.867147in}{3.688087in}}%
\pgfpathlineto{\pgfqpoint{2.871405in}{3.688087in}}%
\pgfpathlineto{\pgfqpoint{2.871405in}{3.683829in}}%
\pgfpathmoveto{\pgfqpoint{2.867147in}{3.688087in}}%
\pgfpathlineto{\pgfqpoint{2.867147in}{3.688087in}}%
\pgfpathlineto{\pgfqpoint{2.867147in}{3.692345in}}%
\pgfpathlineto{\pgfqpoint{2.871405in}{3.692345in}}%
\pgfpathlineto{\pgfqpoint{2.871405in}{3.688087in}}%
\pgfpathmoveto{\pgfqpoint{2.862889in}{3.692345in}}%
\pgfpathlineto{\pgfqpoint{2.862889in}{3.692345in}}%
\pgfpathlineto{\pgfqpoint{2.862889in}{3.696602in}}%
\pgfpathlineto{\pgfqpoint{2.867147in}{3.696602in}}%
\pgfpathlineto{\pgfqpoint{2.867147in}{3.692345in}}%
\pgfpathmoveto{\pgfqpoint{2.862889in}{3.696602in}}%
\pgfpathlineto{\pgfqpoint{2.862889in}{3.696602in}}%
\pgfpathlineto{\pgfqpoint{2.862889in}{3.700860in}}%
\pgfpathlineto{\pgfqpoint{2.867147in}{3.700860in}}%
\pgfpathlineto{\pgfqpoint{2.867147in}{3.696602in}}%
\pgfpathmoveto{\pgfqpoint{2.867147in}{3.692345in}}%
\pgfpathlineto{\pgfqpoint{2.867147in}{3.692345in}}%
\pgfpathlineto{\pgfqpoint{2.867147in}{3.696602in}}%
\pgfpathlineto{\pgfqpoint{2.871405in}{3.696602in}}%
\pgfpathlineto{\pgfqpoint{2.871405in}{3.692345in}}%
\pgfpathmoveto{\pgfqpoint{2.867147in}{3.696602in}}%
\pgfpathlineto{\pgfqpoint{2.867147in}{3.696602in}}%
\pgfpathlineto{\pgfqpoint{2.867147in}{3.700860in}}%
\pgfpathlineto{\pgfqpoint{2.871405in}{3.700860in}}%
\pgfpathlineto{\pgfqpoint{2.871405in}{3.696602in}}%
\pgfpathmoveto{\pgfqpoint{2.867147in}{3.700860in}}%
\pgfpathlineto{\pgfqpoint{2.867147in}{3.700860in}}%
\pgfpathlineto{\pgfqpoint{2.867147in}{3.705118in}}%
\pgfpathlineto{\pgfqpoint{2.871405in}{3.705118in}}%
\pgfpathlineto{\pgfqpoint{2.871405in}{3.700860in}}%
\pgfpathmoveto{\pgfqpoint{2.871405in}{3.675313in}}%
\pgfpathlineto{\pgfqpoint{2.871405in}{3.675313in}}%
\pgfpathlineto{\pgfqpoint{2.871405in}{3.679571in}}%
\pgfpathlineto{\pgfqpoint{2.875663in}{3.679571in}}%
\pgfpathlineto{\pgfqpoint{2.875663in}{3.675313in}}%
\pgfpathmoveto{\pgfqpoint{2.871405in}{3.679571in}}%
\pgfpathlineto{\pgfqpoint{2.871405in}{3.679571in}}%
\pgfpathlineto{\pgfqpoint{2.871405in}{3.683829in}}%
\pgfpathlineto{\pgfqpoint{2.875663in}{3.683829in}}%
\pgfpathlineto{\pgfqpoint{2.875663in}{3.679571in}}%
\pgfpathmoveto{\pgfqpoint{2.875663in}{3.675313in}}%
\pgfpathlineto{\pgfqpoint{2.875663in}{3.675313in}}%
\pgfpathlineto{\pgfqpoint{2.875663in}{3.679571in}}%
\pgfpathlineto{\pgfqpoint{2.879921in}{3.679571in}}%
\pgfpathlineto{\pgfqpoint{2.879921in}{3.675313in}}%
\pgfpathmoveto{\pgfqpoint{2.875663in}{3.679571in}}%
\pgfpathlineto{\pgfqpoint{2.875663in}{3.679571in}}%
\pgfpathlineto{\pgfqpoint{2.875663in}{3.683829in}}%
\pgfpathlineto{\pgfqpoint{2.879921in}{3.683829in}}%
\pgfpathlineto{\pgfqpoint{2.879921in}{3.679571in}}%
\pgfpathmoveto{\pgfqpoint{2.871405in}{3.683829in}}%
\pgfpathlineto{\pgfqpoint{2.871405in}{3.683829in}}%
\pgfpathlineto{\pgfqpoint{2.871405in}{3.688087in}}%
\pgfpathlineto{\pgfqpoint{2.875663in}{3.688087in}}%
\pgfpathlineto{\pgfqpoint{2.875663in}{3.683829in}}%
\pgfpathmoveto{\pgfqpoint{2.871405in}{3.688087in}}%
\pgfpathlineto{\pgfqpoint{2.871405in}{3.688087in}}%
\pgfpathlineto{\pgfqpoint{2.871405in}{3.692345in}}%
\pgfpathlineto{\pgfqpoint{2.875663in}{3.692345in}}%
\pgfpathlineto{\pgfqpoint{2.875663in}{3.688087in}}%
\pgfpathmoveto{\pgfqpoint{2.875663in}{3.683829in}}%
\pgfpathlineto{\pgfqpoint{2.875663in}{3.683829in}}%
\pgfpathlineto{\pgfqpoint{2.875663in}{3.688087in}}%
\pgfpathlineto{\pgfqpoint{2.879921in}{3.688087in}}%
\pgfpathlineto{\pgfqpoint{2.879921in}{3.683829in}}%
\pgfpathmoveto{\pgfqpoint{2.875663in}{3.688087in}}%
\pgfpathlineto{\pgfqpoint{2.875663in}{3.688087in}}%
\pgfpathlineto{\pgfqpoint{2.875663in}{3.692345in}}%
\pgfpathlineto{\pgfqpoint{2.879921in}{3.692345in}}%
\pgfpathlineto{\pgfqpoint{2.879921in}{3.688087in}}%
\pgfpathmoveto{\pgfqpoint{2.879921in}{3.675313in}}%
\pgfpathlineto{\pgfqpoint{2.879921in}{3.675313in}}%
\pgfpathlineto{\pgfqpoint{2.879921in}{3.679571in}}%
\pgfpathlineto{\pgfqpoint{2.884178in}{3.679571in}}%
\pgfpathlineto{\pgfqpoint{2.884178in}{3.675313in}}%
\pgfpathmoveto{\pgfqpoint{2.879921in}{3.679571in}}%
\pgfpathlineto{\pgfqpoint{2.879921in}{3.679571in}}%
\pgfpathlineto{\pgfqpoint{2.879921in}{3.683829in}}%
\pgfpathlineto{\pgfqpoint{2.884178in}{3.683829in}}%
\pgfpathlineto{\pgfqpoint{2.884178in}{3.679571in}}%
\pgfpathmoveto{\pgfqpoint{2.884178in}{3.679571in}}%
\pgfpathlineto{\pgfqpoint{2.884178in}{3.679571in}}%
\pgfpathlineto{\pgfqpoint{2.884178in}{3.683829in}}%
\pgfpathlineto{\pgfqpoint{2.888436in}{3.683829in}}%
\pgfpathlineto{\pgfqpoint{2.888436in}{3.679571in}}%
\pgfpathmoveto{\pgfqpoint{2.879921in}{3.683829in}}%
\pgfpathlineto{\pgfqpoint{2.879921in}{3.683829in}}%
\pgfpathlineto{\pgfqpoint{2.879921in}{3.688087in}}%
\pgfpathlineto{\pgfqpoint{2.884178in}{3.688087in}}%
\pgfpathlineto{\pgfqpoint{2.884178in}{3.683829in}}%
\pgfpathmoveto{\pgfqpoint{2.879921in}{3.688087in}}%
\pgfpathlineto{\pgfqpoint{2.879921in}{3.688087in}}%
\pgfpathlineto{\pgfqpoint{2.879921in}{3.692345in}}%
\pgfpathlineto{\pgfqpoint{2.884178in}{3.692345in}}%
\pgfpathlineto{\pgfqpoint{2.884178in}{3.688087in}}%
\pgfpathmoveto{\pgfqpoint{2.884178in}{3.683829in}}%
\pgfpathlineto{\pgfqpoint{2.884178in}{3.683829in}}%
\pgfpathlineto{\pgfqpoint{2.884178in}{3.688087in}}%
\pgfpathlineto{\pgfqpoint{2.888436in}{3.688087in}}%
\pgfpathlineto{\pgfqpoint{2.888436in}{3.683829in}}%
\pgfpathmoveto{\pgfqpoint{2.884178in}{3.688087in}}%
\pgfpathlineto{\pgfqpoint{2.884178in}{3.688087in}}%
\pgfpathlineto{\pgfqpoint{2.884178in}{3.692345in}}%
\pgfpathlineto{\pgfqpoint{2.888436in}{3.692345in}}%
\pgfpathlineto{\pgfqpoint{2.888436in}{3.688087in}}%
\pgfpathmoveto{\pgfqpoint{2.871405in}{3.692345in}}%
\pgfpathlineto{\pgfqpoint{2.871405in}{3.692345in}}%
\pgfpathlineto{\pgfqpoint{2.871405in}{3.696602in}}%
\pgfpathlineto{\pgfqpoint{2.875663in}{3.696602in}}%
\pgfpathlineto{\pgfqpoint{2.875663in}{3.692345in}}%
\pgfpathmoveto{\pgfqpoint{2.871405in}{3.696602in}}%
\pgfpathlineto{\pgfqpoint{2.871405in}{3.696602in}}%
\pgfpathlineto{\pgfqpoint{2.871405in}{3.700860in}}%
\pgfpathlineto{\pgfqpoint{2.875663in}{3.700860in}}%
\pgfpathlineto{\pgfqpoint{2.875663in}{3.696602in}}%
\pgfpathmoveto{\pgfqpoint{2.875663in}{3.692345in}}%
\pgfpathlineto{\pgfqpoint{2.875663in}{3.692345in}}%
\pgfpathlineto{\pgfqpoint{2.875663in}{3.696602in}}%
\pgfpathlineto{\pgfqpoint{2.879921in}{3.696602in}}%
\pgfpathlineto{\pgfqpoint{2.879921in}{3.692345in}}%
\pgfpathmoveto{\pgfqpoint{2.875663in}{3.696602in}}%
\pgfpathlineto{\pgfqpoint{2.875663in}{3.696602in}}%
\pgfpathlineto{\pgfqpoint{2.875663in}{3.700860in}}%
\pgfpathlineto{\pgfqpoint{2.879921in}{3.700860in}}%
\pgfpathlineto{\pgfqpoint{2.879921in}{3.696602in}}%
\pgfpathmoveto{\pgfqpoint{2.871405in}{3.700860in}}%
\pgfpathlineto{\pgfqpoint{2.871405in}{3.700860in}}%
\pgfpathlineto{\pgfqpoint{2.871405in}{3.705118in}}%
\pgfpathlineto{\pgfqpoint{2.875663in}{3.705118in}}%
\pgfpathlineto{\pgfqpoint{2.875663in}{3.700860in}}%
\pgfpathmoveto{\pgfqpoint{2.871405in}{3.705118in}}%
\pgfpathlineto{\pgfqpoint{2.871405in}{3.705118in}}%
\pgfpathlineto{\pgfqpoint{2.871405in}{3.709376in}}%
\pgfpathlineto{\pgfqpoint{2.875663in}{3.709376in}}%
\pgfpathlineto{\pgfqpoint{2.875663in}{3.705118in}}%
\pgfpathmoveto{\pgfqpoint{2.875663in}{3.700860in}}%
\pgfpathlineto{\pgfqpoint{2.875663in}{3.700860in}}%
\pgfpathlineto{\pgfqpoint{2.875663in}{3.705118in}}%
\pgfpathlineto{\pgfqpoint{2.879921in}{3.705118in}}%
\pgfpathlineto{\pgfqpoint{2.879921in}{3.700860in}}%
\pgfpathmoveto{\pgfqpoint{2.875663in}{3.705118in}}%
\pgfpathlineto{\pgfqpoint{2.875663in}{3.705118in}}%
\pgfpathlineto{\pgfqpoint{2.875663in}{3.709376in}}%
\pgfpathlineto{\pgfqpoint{2.879921in}{3.709376in}}%
\pgfpathlineto{\pgfqpoint{2.879921in}{3.705118in}}%
\pgfpathmoveto{\pgfqpoint{2.879921in}{3.692345in}}%
\pgfpathlineto{\pgfqpoint{2.879921in}{3.692345in}}%
\pgfpathlineto{\pgfqpoint{2.879921in}{3.696602in}}%
\pgfpathlineto{\pgfqpoint{2.884178in}{3.696602in}}%
\pgfpathlineto{\pgfqpoint{2.884178in}{3.692345in}}%
\pgfpathmoveto{\pgfqpoint{2.879921in}{3.696602in}}%
\pgfpathlineto{\pgfqpoint{2.879921in}{3.696602in}}%
\pgfpathlineto{\pgfqpoint{2.879921in}{3.700860in}}%
\pgfpathlineto{\pgfqpoint{2.884178in}{3.700860in}}%
\pgfpathlineto{\pgfqpoint{2.884178in}{3.696602in}}%
\pgfpathmoveto{\pgfqpoint{2.884178in}{3.692345in}}%
\pgfpathlineto{\pgfqpoint{2.884178in}{3.692345in}}%
\pgfpathlineto{\pgfqpoint{2.884178in}{3.696602in}}%
\pgfpathlineto{\pgfqpoint{2.888436in}{3.696602in}}%
\pgfpathlineto{\pgfqpoint{2.888436in}{3.692345in}}%
\pgfpathmoveto{\pgfqpoint{2.884178in}{3.696602in}}%
\pgfpathlineto{\pgfqpoint{2.884178in}{3.696602in}}%
\pgfpathlineto{\pgfqpoint{2.884178in}{3.700860in}}%
\pgfpathlineto{\pgfqpoint{2.888436in}{3.700860in}}%
\pgfpathlineto{\pgfqpoint{2.888436in}{3.696602in}}%
\pgfpathmoveto{\pgfqpoint{2.879921in}{3.700860in}}%
\pgfpathlineto{\pgfqpoint{2.879921in}{3.700860in}}%
\pgfpathlineto{\pgfqpoint{2.879921in}{3.705118in}}%
\pgfpathlineto{\pgfqpoint{2.884178in}{3.705118in}}%
\pgfpathlineto{\pgfqpoint{2.884178in}{3.700860in}}%
\pgfpathmoveto{\pgfqpoint{2.879921in}{3.705118in}}%
\pgfpathlineto{\pgfqpoint{2.879921in}{3.705118in}}%
\pgfpathlineto{\pgfqpoint{2.879921in}{3.709376in}}%
\pgfpathlineto{\pgfqpoint{2.884178in}{3.709376in}}%
\pgfpathlineto{\pgfqpoint{2.884178in}{3.705118in}}%
\pgfpathmoveto{\pgfqpoint{2.884178in}{3.700860in}}%
\pgfpathlineto{\pgfqpoint{2.884178in}{3.700860in}}%
\pgfpathlineto{\pgfqpoint{2.884178in}{3.705118in}}%
\pgfpathlineto{\pgfqpoint{2.888436in}{3.705118in}}%
\pgfpathlineto{\pgfqpoint{2.888436in}{3.700860in}}%
\pgfpathmoveto{\pgfqpoint{2.884178in}{3.705118in}}%
\pgfpathlineto{\pgfqpoint{2.884178in}{3.705118in}}%
\pgfpathlineto{\pgfqpoint{2.884178in}{3.709376in}}%
\pgfpathlineto{\pgfqpoint{2.888436in}{3.709376in}}%
\pgfpathlineto{\pgfqpoint{2.888436in}{3.705118in}}%
\pgfpathmoveto{\pgfqpoint{2.871405in}{3.709376in}}%
\pgfpathlineto{\pgfqpoint{2.871405in}{3.709376in}}%
\pgfpathlineto{\pgfqpoint{2.871405in}{3.713634in}}%
\pgfpathlineto{\pgfqpoint{2.875663in}{3.713634in}}%
\pgfpathlineto{\pgfqpoint{2.875663in}{3.709376in}}%
\pgfpathmoveto{\pgfqpoint{2.875663in}{3.709376in}}%
\pgfpathlineto{\pgfqpoint{2.875663in}{3.709376in}}%
\pgfpathlineto{\pgfqpoint{2.875663in}{3.713634in}}%
\pgfpathlineto{\pgfqpoint{2.879921in}{3.713634in}}%
\pgfpathlineto{\pgfqpoint{2.879921in}{3.709376in}}%
\pgfpathmoveto{\pgfqpoint{2.875663in}{3.713634in}}%
\pgfpathlineto{\pgfqpoint{2.875663in}{3.713634in}}%
\pgfpathlineto{\pgfqpoint{2.875663in}{3.717892in}}%
\pgfpathlineto{\pgfqpoint{2.879921in}{3.717892in}}%
\pgfpathlineto{\pgfqpoint{2.879921in}{3.713634in}}%
\pgfpathmoveto{\pgfqpoint{2.879921in}{3.709376in}}%
\pgfpathlineto{\pgfqpoint{2.879921in}{3.709376in}}%
\pgfpathlineto{\pgfqpoint{2.879921in}{3.713634in}}%
\pgfpathlineto{\pgfqpoint{2.884178in}{3.713634in}}%
\pgfpathlineto{\pgfqpoint{2.884178in}{3.709376in}}%
\pgfpathmoveto{\pgfqpoint{2.879921in}{3.713634in}}%
\pgfpathlineto{\pgfqpoint{2.879921in}{3.713634in}}%
\pgfpathlineto{\pgfqpoint{2.879921in}{3.717892in}}%
\pgfpathlineto{\pgfqpoint{2.884178in}{3.717892in}}%
\pgfpathlineto{\pgfqpoint{2.884178in}{3.713634in}}%
\pgfpathmoveto{\pgfqpoint{2.884178in}{3.709376in}}%
\pgfpathlineto{\pgfqpoint{2.884178in}{3.709376in}}%
\pgfpathlineto{\pgfqpoint{2.884178in}{3.713634in}}%
\pgfpathlineto{\pgfqpoint{2.888436in}{3.713634in}}%
\pgfpathlineto{\pgfqpoint{2.888436in}{3.709376in}}%
\pgfpathmoveto{\pgfqpoint{2.884178in}{3.713634in}}%
\pgfpathlineto{\pgfqpoint{2.884178in}{3.713634in}}%
\pgfpathlineto{\pgfqpoint{2.884178in}{3.717892in}}%
\pgfpathlineto{\pgfqpoint{2.888436in}{3.717892in}}%
\pgfpathlineto{\pgfqpoint{2.888436in}{3.713634in}}%
\pgfpathmoveto{\pgfqpoint{2.879921in}{3.717892in}}%
\pgfpathlineto{\pgfqpoint{2.879921in}{3.717892in}}%
\pgfpathlineto{\pgfqpoint{2.879921in}{3.722149in}}%
\pgfpathlineto{\pgfqpoint{2.884178in}{3.722149in}}%
\pgfpathlineto{\pgfqpoint{2.884178in}{3.717892in}}%
\pgfpathmoveto{\pgfqpoint{2.879921in}{3.722149in}}%
\pgfpathlineto{\pgfqpoint{2.879921in}{3.722149in}}%
\pgfpathlineto{\pgfqpoint{2.879921in}{3.726407in}}%
\pgfpathlineto{\pgfqpoint{2.884178in}{3.726407in}}%
\pgfpathlineto{\pgfqpoint{2.884178in}{3.722149in}}%
\pgfpathmoveto{\pgfqpoint{2.884178in}{3.717892in}}%
\pgfpathlineto{\pgfqpoint{2.884178in}{3.717892in}}%
\pgfpathlineto{\pgfqpoint{2.884178in}{3.722149in}}%
\pgfpathlineto{\pgfqpoint{2.888436in}{3.722149in}}%
\pgfpathlineto{\pgfqpoint{2.888436in}{3.717892in}}%
\pgfpathmoveto{\pgfqpoint{2.884178in}{3.722149in}}%
\pgfpathlineto{\pgfqpoint{2.884178in}{3.722149in}}%
\pgfpathlineto{\pgfqpoint{2.884178in}{3.726407in}}%
\pgfpathlineto{\pgfqpoint{2.888436in}{3.726407in}}%
\pgfpathlineto{\pgfqpoint{2.888436in}{3.722149in}}%
\pgfpathmoveto{\pgfqpoint{2.884178in}{3.726407in}}%
\pgfpathlineto{\pgfqpoint{2.884178in}{3.726407in}}%
\pgfpathlineto{\pgfqpoint{2.884178in}{3.730665in}}%
\pgfpathlineto{\pgfqpoint{2.888436in}{3.730665in}}%
\pgfpathlineto{\pgfqpoint{2.888436in}{3.726407in}}%
\pgfpathmoveto{\pgfqpoint{2.888436in}{3.683829in}}%
\pgfpathlineto{\pgfqpoint{2.888436in}{3.683829in}}%
\pgfpathlineto{\pgfqpoint{2.888436in}{3.688087in}}%
\pgfpathlineto{\pgfqpoint{2.892694in}{3.688087in}}%
\pgfpathlineto{\pgfqpoint{2.892694in}{3.683829in}}%
\pgfpathmoveto{\pgfqpoint{2.888436in}{3.688087in}}%
\pgfpathlineto{\pgfqpoint{2.888436in}{3.688087in}}%
\pgfpathlineto{\pgfqpoint{2.888436in}{3.692345in}}%
\pgfpathlineto{\pgfqpoint{2.892694in}{3.692345in}}%
\pgfpathlineto{\pgfqpoint{2.892694in}{3.688087in}}%
\pgfpathmoveto{\pgfqpoint{2.892694in}{3.688087in}}%
\pgfpathlineto{\pgfqpoint{2.892694in}{3.688087in}}%
\pgfpathlineto{\pgfqpoint{2.892694in}{3.692345in}}%
\pgfpathlineto{\pgfqpoint{2.896952in}{3.692345in}}%
\pgfpathlineto{\pgfqpoint{2.896952in}{3.688087in}}%
\pgfpathmoveto{\pgfqpoint{2.888436in}{3.692345in}}%
\pgfpathlineto{\pgfqpoint{2.888436in}{3.692345in}}%
\pgfpathlineto{\pgfqpoint{2.888436in}{3.696602in}}%
\pgfpathlineto{\pgfqpoint{2.892694in}{3.696602in}}%
\pgfpathlineto{\pgfqpoint{2.892694in}{3.692345in}}%
\pgfpathmoveto{\pgfqpoint{2.888436in}{3.696602in}}%
\pgfpathlineto{\pgfqpoint{2.888436in}{3.696602in}}%
\pgfpathlineto{\pgfqpoint{2.888436in}{3.700860in}}%
\pgfpathlineto{\pgfqpoint{2.892694in}{3.700860in}}%
\pgfpathlineto{\pgfqpoint{2.892694in}{3.696602in}}%
\pgfpathmoveto{\pgfqpoint{2.892694in}{3.692345in}}%
\pgfpathlineto{\pgfqpoint{2.892694in}{3.692345in}}%
\pgfpathlineto{\pgfqpoint{2.892694in}{3.696602in}}%
\pgfpathlineto{\pgfqpoint{2.896952in}{3.696602in}}%
\pgfpathlineto{\pgfqpoint{2.896952in}{3.692345in}}%
\pgfpathmoveto{\pgfqpoint{2.892694in}{3.696602in}}%
\pgfpathlineto{\pgfqpoint{2.892694in}{3.696602in}}%
\pgfpathlineto{\pgfqpoint{2.892694in}{3.700860in}}%
\pgfpathlineto{\pgfqpoint{2.896952in}{3.700860in}}%
\pgfpathlineto{\pgfqpoint{2.896952in}{3.696602in}}%
\pgfpathmoveto{\pgfqpoint{2.888436in}{3.700860in}}%
\pgfpathlineto{\pgfqpoint{2.888436in}{3.700860in}}%
\pgfpathlineto{\pgfqpoint{2.888436in}{3.705118in}}%
\pgfpathlineto{\pgfqpoint{2.892694in}{3.705118in}}%
\pgfpathlineto{\pgfqpoint{2.892694in}{3.700860in}}%
\pgfpathmoveto{\pgfqpoint{2.888436in}{3.705118in}}%
\pgfpathlineto{\pgfqpoint{2.888436in}{3.705118in}}%
\pgfpathlineto{\pgfqpoint{2.888436in}{3.709376in}}%
\pgfpathlineto{\pgfqpoint{2.892694in}{3.709376in}}%
\pgfpathlineto{\pgfqpoint{2.892694in}{3.705118in}}%
\pgfpathmoveto{\pgfqpoint{2.892694in}{3.700860in}}%
\pgfpathlineto{\pgfqpoint{2.892694in}{3.700860in}}%
\pgfpathlineto{\pgfqpoint{2.892694in}{3.705118in}}%
\pgfpathlineto{\pgfqpoint{2.896952in}{3.705118in}}%
\pgfpathlineto{\pgfqpoint{2.896952in}{3.700860in}}%
\pgfpathmoveto{\pgfqpoint{2.892694in}{3.705118in}}%
\pgfpathlineto{\pgfqpoint{2.892694in}{3.705118in}}%
\pgfpathlineto{\pgfqpoint{2.892694in}{3.709376in}}%
\pgfpathlineto{\pgfqpoint{2.896952in}{3.709376in}}%
\pgfpathlineto{\pgfqpoint{2.896952in}{3.705118in}}%
\pgfpathmoveto{\pgfqpoint{2.896952in}{3.692345in}}%
\pgfpathlineto{\pgfqpoint{2.896952in}{3.692345in}}%
\pgfpathlineto{\pgfqpoint{2.896952in}{3.696602in}}%
\pgfpathlineto{\pgfqpoint{2.901210in}{3.696602in}}%
\pgfpathlineto{\pgfqpoint{2.901210in}{3.692345in}}%
\pgfpathmoveto{\pgfqpoint{2.896952in}{3.696602in}}%
\pgfpathlineto{\pgfqpoint{2.896952in}{3.696602in}}%
\pgfpathlineto{\pgfqpoint{2.896952in}{3.700860in}}%
\pgfpathlineto{\pgfqpoint{2.901210in}{3.700860in}}%
\pgfpathlineto{\pgfqpoint{2.901210in}{3.696602in}}%
\pgfpathmoveto{\pgfqpoint{2.901210in}{3.696602in}}%
\pgfpathlineto{\pgfqpoint{2.901210in}{3.696602in}}%
\pgfpathlineto{\pgfqpoint{2.901210in}{3.700860in}}%
\pgfpathlineto{\pgfqpoint{2.905467in}{3.700860in}}%
\pgfpathlineto{\pgfqpoint{2.905467in}{3.696602in}}%
\pgfpathmoveto{\pgfqpoint{2.896952in}{3.700860in}}%
\pgfpathlineto{\pgfqpoint{2.896952in}{3.700860in}}%
\pgfpathlineto{\pgfqpoint{2.896952in}{3.705118in}}%
\pgfpathlineto{\pgfqpoint{2.901210in}{3.705118in}}%
\pgfpathlineto{\pgfqpoint{2.901210in}{3.700860in}}%
\pgfpathmoveto{\pgfqpoint{2.896952in}{3.705118in}}%
\pgfpathlineto{\pgfqpoint{2.896952in}{3.705118in}}%
\pgfpathlineto{\pgfqpoint{2.896952in}{3.709376in}}%
\pgfpathlineto{\pgfqpoint{2.901210in}{3.709376in}}%
\pgfpathlineto{\pgfqpoint{2.901210in}{3.705118in}}%
\pgfpathmoveto{\pgfqpoint{2.901210in}{3.700860in}}%
\pgfpathlineto{\pgfqpoint{2.901210in}{3.700860in}}%
\pgfpathlineto{\pgfqpoint{2.901210in}{3.705118in}}%
\pgfpathlineto{\pgfqpoint{2.905467in}{3.705118in}}%
\pgfpathlineto{\pgfqpoint{2.905467in}{3.700860in}}%
\pgfpathmoveto{\pgfqpoint{2.901210in}{3.705118in}}%
\pgfpathlineto{\pgfqpoint{2.901210in}{3.705118in}}%
\pgfpathlineto{\pgfqpoint{2.901210in}{3.709376in}}%
\pgfpathlineto{\pgfqpoint{2.905467in}{3.709376in}}%
\pgfpathlineto{\pgfqpoint{2.905467in}{3.705118in}}%
\pgfpathmoveto{\pgfqpoint{2.905467in}{3.700860in}}%
\pgfpathlineto{\pgfqpoint{2.905467in}{3.700860in}}%
\pgfpathlineto{\pgfqpoint{2.905467in}{3.705118in}}%
\pgfpathlineto{\pgfqpoint{2.909725in}{3.705118in}}%
\pgfpathlineto{\pgfqpoint{2.909725in}{3.700860in}}%
\pgfpathmoveto{\pgfqpoint{2.905467in}{3.705118in}}%
\pgfpathlineto{\pgfqpoint{2.905467in}{3.705118in}}%
\pgfpathlineto{\pgfqpoint{2.905467in}{3.709376in}}%
\pgfpathlineto{\pgfqpoint{2.909725in}{3.709376in}}%
\pgfpathlineto{\pgfqpoint{2.909725in}{3.705118in}}%
\pgfpathmoveto{\pgfqpoint{2.909725in}{3.705118in}}%
\pgfpathlineto{\pgfqpoint{2.909725in}{3.705118in}}%
\pgfpathlineto{\pgfqpoint{2.909725in}{3.709376in}}%
\pgfpathlineto{\pgfqpoint{2.913983in}{3.709376in}}%
\pgfpathlineto{\pgfqpoint{2.913983in}{3.705118in}}%
\pgfpathmoveto{\pgfqpoint{2.888436in}{3.709376in}}%
\pgfpathlineto{\pgfqpoint{2.888436in}{3.709376in}}%
\pgfpathlineto{\pgfqpoint{2.888436in}{3.713634in}}%
\pgfpathlineto{\pgfqpoint{2.892694in}{3.713634in}}%
\pgfpathlineto{\pgfqpoint{2.892694in}{3.709376in}}%
\pgfpathmoveto{\pgfqpoint{2.888436in}{3.713634in}}%
\pgfpathlineto{\pgfqpoint{2.888436in}{3.713634in}}%
\pgfpathlineto{\pgfqpoint{2.888436in}{3.717892in}}%
\pgfpathlineto{\pgfqpoint{2.892694in}{3.717892in}}%
\pgfpathlineto{\pgfqpoint{2.892694in}{3.713634in}}%
\pgfpathmoveto{\pgfqpoint{2.892694in}{3.709376in}}%
\pgfpathlineto{\pgfqpoint{2.892694in}{3.709376in}}%
\pgfpathlineto{\pgfqpoint{2.892694in}{3.713634in}}%
\pgfpathlineto{\pgfqpoint{2.896952in}{3.713634in}}%
\pgfpathlineto{\pgfqpoint{2.896952in}{3.709376in}}%
\pgfpathmoveto{\pgfqpoint{2.892694in}{3.713634in}}%
\pgfpathlineto{\pgfqpoint{2.892694in}{3.713634in}}%
\pgfpathlineto{\pgfqpoint{2.892694in}{3.717892in}}%
\pgfpathlineto{\pgfqpoint{2.896952in}{3.717892in}}%
\pgfpathlineto{\pgfqpoint{2.896952in}{3.713634in}}%
\pgfpathmoveto{\pgfqpoint{2.888436in}{3.717892in}}%
\pgfpathlineto{\pgfqpoint{2.888436in}{3.717892in}}%
\pgfpathlineto{\pgfqpoint{2.888436in}{3.722149in}}%
\pgfpathlineto{\pgfqpoint{2.892694in}{3.722149in}}%
\pgfpathlineto{\pgfqpoint{2.892694in}{3.717892in}}%
\pgfpathmoveto{\pgfqpoint{2.888436in}{3.722149in}}%
\pgfpathlineto{\pgfqpoint{2.888436in}{3.722149in}}%
\pgfpathlineto{\pgfqpoint{2.888436in}{3.726407in}}%
\pgfpathlineto{\pgfqpoint{2.892694in}{3.726407in}}%
\pgfpathlineto{\pgfqpoint{2.892694in}{3.722149in}}%
\pgfpathmoveto{\pgfqpoint{2.892694in}{3.717892in}}%
\pgfpathlineto{\pgfqpoint{2.892694in}{3.717892in}}%
\pgfpathlineto{\pgfqpoint{2.892694in}{3.722149in}}%
\pgfpathlineto{\pgfqpoint{2.896952in}{3.722149in}}%
\pgfpathlineto{\pgfqpoint{2.896952in}{3.717892in}}%
\pgfpathmoveto{\pgfqpoint{2.892694in}{3.722149in}}%
\pgfpathlineto{\pgfqpoint{2.892694in}{3.722149in}}%
\pgfpathlineto{\pgfqpoint{2.892694in}{3.726407in}}%
\pgfpathlineto{\pgfqpoint{2.896952in}{3.726407in}}%
\pgfpathlineto{\pgfqpoint{2.896952in}{3.722149in}}%
\pgfpathmoveto{\pgfqpoint{2.896952in}{3.709376in}}%
\pgfpathlineto{\pgfqpoint{2.896952in}{3.709376in}}%
\pgfpathlineto{\pgfqpoint{2.896952in}{3.713634in}}%
\pgfpathlineto{\pgfqpoint{2.901210in}{3.713634in}}%
\pgfpathlineto{\pgfqpoint{2.901210in}{3.709376in}}%
\pgfpathmoveto{\pgfqpoint{2.896952in}{3.713634in}}%
\pgfpathlineto{\pgfqpoint{2.896952in}{3.713634in}}%
\pgfpathlineto{\pgfqpoint{2.896952in}{3.717892in}}%
\pgfpathlineto{\pgfqpoint{2.901210in}{3.717892in}}%
\pgfpathlineto{\pgfqpoint{2.901210in}{3.713634in}}%
\pgfpathmoveto{\pgfqpoint{2.901210in}{3.709376in}}%
\pgfpathlineto{\pgfqpoint{2.901210in}{3.709376in}}%
\pgfpathlineto{\pgfqpoint{2.901210in}{3.713634in}}%
\pgfpathlineto{\pgfqpoint{2.905467in}{3.713634in}}%
\pgfpathlineto{\pgfqpoint{2.905467in}{3.709376in}}%
\pgfpathmoveto{\pgfqpoint{2.901210in}{3.713634in}}%
\pgfpathlineto{\pgfqpoint{2.901210in}{3.713634in}}%
\pgfpathlineto{\pgfqpoint{2.901210in}{3.717892in}}%
\pgfpathlineto{\pgfqpoint{2.905467in}{3.717892in}}%
\pgfpathlineto{\pgfqpoint{2.905467in}{3.713634in}}%
\pgfpathmoveto{\pgfqpoint{2.896952in}{3.717892in}}%
\pgfpathlineto{\pgfqpoint{2.896952in}{3.717892in}}%
\pgfpathlineto{\pgfqpoint{2.896952in}{3.722149in}}%
\pgfpathlineto{\pgfqpoint{2.901210in}{3.722149in}}%
\pgfpathlineto{\pgfqpoint{2.901210in}{3.717892in}}%
\pgfpathmoveto{\pgfqpoint{2.896952in}{3.722149in}}%
\pgfpathlineto{\pgfqpoint{2.896952in}{3.722149in}}%
\pgfpathlineto{\pgfqpoint{2.896952in}{3.726407in}}%
\pgfpathlineto{\pgfqpoint{2.901210in}{3.726407in}}%
\pgfpathlineto{\pgfqpoint{2.901210in}{3.722149in}}%
\pgfpathmoveto{\pgfqpoint{2.901210in}{3.717892in}}%
\pgfpathlineto{\pgfqpoint{2.901210in}{3.717892in}}%
\pgfpathlineto{\pgfqpoint{2.901210in}{3.722149in}}%
\pgfpathlineto{\pgfqpoint{2.905467in}{3.722149in}}%
\pgfpathlineto{\pgfqpoint{2.905467in}{3.717892in}}%
\pgfpathmoveto{\pgfqpoint{2.901210in}{3.722149in}}%
\pgfpathlineto{\pgfqpoint{2.901210in}{3.722149in}}%
\pgfpathlineto{\pgfqpoint{2.901210in}{3.726407in}}%
\pgfpathlineto{\pgfqpoint{2.905467in}{3.726407in}}%
\pgfpathlineto{\pgfqpoint{2.905467in}{3.722149in}}%
\pgfpathmoveto{\pgfqpoint{2.888436in}{3.726407in}}%
\pgfpathlineto{\pgfqpoint{2.888436in}{3.726407in}}%
\pgfpathlineto{\pgfqpoint{2.888436in}{3.730665in}}%
\pgfpathlineto{\pgfqpoint{2.892694in}{3.730665in}}%
\pgfpathlineto{\pgfqpoint{2.892694in}{3.726407in}}%
\pgfpathmoveto{\pgfqpoint{2.888436in}{3.730665in}}%
\pgfpathlineto{\pgfqpoint{2.888436in}{3.730665in}}%
\pgfpathlineto{\pgfqpoint{2.888436in}{3.734923in}}%
\pgfpathlineto{\pgfqpoint{2.892694in}{3.734923in}}%
\pgfpathlineto{\pgfqpoint{2.892694in}{3.730665in}}%
\pgfpathmoveto{\pgfqpoint{2.892694in}{3.726407in}}%
\pgfpathlineto{\pgfqpoint{2.892694in}{3.726407in}}%
\pgfpathlineto{\pgfqpoint{2.892694in}{3.730665in}}%
\pgfpathlineto{\pgfqpoint{2.896952in}{3.730665in}}%
\pgfpathlineto{\pgfqpoint{2.896952in}{3.726407in}}%
\pgfpathmoveto{\pgfqpoint{2.892694in}{3.730665in}}%
\pgfpathlineto{\pgfqpoint{2.892694in}{3.730665in}}%
\pgfpathlineto{\pgfqpoint{2.892694in}{3.734923in}}%
\pgfpathlineto{\pgfqpoint{2.896952in}{3.734923in}}%
\pgfpathlineto{\pgfqpoint{2.896952in}{3.730665in}}%
\pgfpathmoveto{\pgfqpoint{2.892694in}{3.734923in}}%
\pgfpathlineto{\pgfqpoint{2.892694in}{3.734923in}}%
\pgfpathlineto{\pgfqpoint{2.892694in}{3.739181in}}%
\pgfpathlineto{\pgfqpoint{2.896952in}{3.739181in}}%
\pgfpathlineto{\pgfqpoint{2.896952in}{3.734923in}}%
\pgfpathmoveto{\pgfqpoint{2.896952in}{3.726407in}}%
\pgfpathlineto{\pgfqpoint{2.896952in}{3.726407in}}%
\pgfpathlineto{\pgfqpoint{2.896952in}{3.730665in}}%
\pgfpathlineto{\pgfqpoint{2.901210in}{3.730665in}}%
\pgfpathlineto{\pgfqpoint{2.901210in}{3.726407in}}%
\pgfpathmoveto{\pgfqpoint{2.896952in}{3.730665in}}%
\pgfpathlineto{\pgfqpoint{2.896952in}{3.730665in}}%
\pgfpathlineto{\pgfqpoint{2.896952in}{3.734923in}}%
\pgfpathlineto{\pgfqpoint{2.901210in}{3.734923in}}%
\pgfpathlineto{\pgfqpoint{2.901210in}{3.730665in}}%
\pgfpathmoveto{\pgfqpoint{2.901210in}{3.726407in}}%
\pgfpathlineto{\pgfqpoint{2.901210in}{3.726407in}}%
\pgfpathlineto{\pgfqpoint{2.901210in}{3.730665in}}%
\pgfpathlineto{\pgfqpoint{2.905467in}{3.730665in}}%
\pgfpathlineto{\pgfqpoint{2.905467in}{3.726407in}}%
\pgfpathmoveto{\pgfqpoint{2.901210in}{3.730665in}}%
\pgfpathlineto{\pgfqpoint{2.901210in}{3.730665in}}%
\pgfpathlineto{\pgfqpoint{2.901210in}{3.734923in}}%
\pgfpathlineto{\pgfqpoint{2.905467in}{3.734923in}}%
\pgfpathlineto{\pgfqpoint{2.905467in}{3.730665in}}%
\pgfpathmoveto{\pgfqpoint{2.896952in}{3.734923in}}%
\pgfpathlineto{\pgfqpoint{2.896952in}{3.734923in}}%
\pgfpathlineto{\pgfqpoint{2.896952in}{3.739181in}}%
\pgfpathlineto{\pgfqpoint{2.901210in}{3.739181in}}%
\pgfpathlineto{\pgfqpoint{2.901210in}{3.734923in}}%
\pgfpathmoveto{\pgfqpoint{2.896952in}{3.739181in}}%
\pgfpathlineto{\pgfqpoint{2.896952in}{3.739181in}}%
\pgfpathlineto{\pgfqpoint{2.896952in}{3.743439in}}%
\pgfpathlineto{\pgfqpoint{2.901210in}{3.743439in}}%
\pgfpathlineto{\pgfqpoint{2.901210in}{3.739181in}}%
\pgfpathmoveto{\pgfqpoint{2.901210in}{3.734923in}}%
\pgfpathlineto{\pgfqpoint{2.901210in}{3.734923in}}%
\pgfpathlineto{\pgfqpoint{2.901210in}{3.739181in}}%
\pgfpathlineto{\pgfqpoint{2.905467in}{3.739181in}}%
\pgfpathlineto{\pgfqpoint{2.905467in}{3.734923in}}%
\pgfpathmoveto{\pgfqpoint{2.901210in}{3.739181in}}%
\pgfpathlineto{\pgfqpoint{2.901210in}{3.739181in}}%
\pgfpathlineto{\pgfqpoint{2.901210in}{3.743439in}}%
\pgfpathlineto{\pgfqpoint{2.905467in}{3.743439in}}%
\pgfpathlineto{\pgfqpoint{2.905467in}{3.739181in}}%
\pgfpathmoveto{\pgfqpoint{2.905467in}{3.709376in}}%
\pgfpathlineto{\pgfqpoint{2.905467in}{3.709376in}}%
\pgfpathlineto{\pgfqpoint{2.905467in}{3.713634in}}%
\pgfpathlineto{\pgfqpoint{2.909725in}{3.713634in}}%
\pgfpathlineto{\pgfqpoint{2.909725in}{3.709376in}}%
\pgfpathmoveto{\pgfqpoint{2.905467in}{3.713634in}}%
\pgfpathlineto{\pgfqpoint{2.905467in}{3.713634in}}%
\pgfpathlineto{\pgfqpoint{2.905467in}{3.717892in}}%
\pgfpathlineto{\pgfqpoint{2.909725in}{3.717892in}}%
\pgfpathlineto{\pgfqpoint{2.909725in}{3.713634in}}%
\pgfpathmoveto{\pgfqpoint{2.909725in}{3.709376in}}%
\pgfpathlineto{\pgfqpoint{2.909725in}{3.709376in}}%
\pgfpathlineto{\pgfqpoint{2.909725in}{3.713634in}}%
\pgfpathlineto{\pgfqpoint{2.913983in}{3.713634in}}%
\pgfpathlineto{\pgfqpoint{2.913983in}{3.709376in}}%
\pgfpathmoveto{\pgfqpoint{2.909725in}{3.713634in}}%
\pgfpathlineto{\pgfqpoint{2.909725in}{3.713634in}}%
\pgfpathlineto{\pgfqpoint{2.909725in}{3.717892in}}%
\pgfpathlineto{\pgfqpoint{2.913983in}{3.717892in}}%
\pgfpathlineto{\pgfqpoint{2.913983in}{3.713634in}}%
\pgfpathmoveto{\pgfqpoint{2.905467in}{3.717892in}}%
\pgfpathlineto{\pgfqpoint{2.905467in}{3.717892in}}%
\pgfpathlineto{\pgfqpoint{2.905467in}{3.722149in}}%
\pgfpathlineto{\pgfqpoint{2.909725in}{3.722149in}}%
\pgfpathlineto{\pgfqpoint{2.909725in}{3.717892in}}%
\pgfpathmoveto{\pgfqpoint{2.905467in}{3.722149in}}%
\pgfpathlineto{\pgfqpoint{2.905467in}{3.722149in}}%
\pgfpathlineto{\pgfqpoint{2.905467in}{3.726407in}}%
\pgfpathlineto{\pgfqpoint{2.909725in}{3.726407in}}%
\pgfpathlineto{\pgfqpoint{2.909725in}{3.722149in}}%
\pgfpathmoveto{\pgfqpoint{2.909725in}{3.717892in}}%
\pgfpathlineto{\pgfqpoint{2.909725in}{3.717892in}}%
\pgfpathlineto{\pgfqpoint{2.909725in}{3.722149in}}%
\pgfpathlineto{\pgfqpoint{2.913983in}{3.722149in}}%
\pgfpathlineto{\pgfqpoint{2.913983in}{3.717892in}}%
\pgfpathmoveto{\pgfqpoint{2.909725in}{3.722149in}}%
\pgfpathlineto{\pgfqpoint{2.909725in}{3.722149in}}%
\pgfpathlineto{\pgfqpoint{2.909725in}{3.726407in}}%
\pgfpathlineto{\pgfqpoint{2.913983in}{3.726407in}}%
\pgfpathlineto{\pgfqpoint{2.913983in}{3.722149in}}%
\pgfpathmoveto{\pgfqpoint{2.913983in}{3.709376in}}%
\pgfpathlineto{\pgfqpoint{2.913983in}{3.709376in}}%
\pgfpathlineto{\pgfqpoint{2.913983in}{3.713634in}}%
\pgfpathlineto{\pgfqpoint{2.918241in}{3.713634in}}%
\pgfpathlineto{\pgfqpoint{2.918241in}{3.709376in}}%
\pgfpathmoveto{\pgfqpoint{2.913983in}{3.713634in}}%
\pgfpathlineto{\pgfqpoint{2.913983in}{3.713634in}}%
\pgfpathlineto{\pgfqpoint{2.913983in}{3.717892in}}%
\pgfpathlineto{\pgfqpoint{2.918241in}{3.717892in}}%
\pgfpathlineto{\pgfqpoint{2.918241in}{3.713634in}}%
\pgfpathmoveto{\pgfqpoint{2.918241in}{3.713634in}}%
\pgfpathlineto{\pgfqpoint{2.918241in}{3.713634in}}%
\pgfpathlineto{\pgfqpoint{2.918241in}{3.717892in}}%
\pgfpathlineto{\pgfqpoint{2.922499in}{3.717892in}}%
\pgfpathlineto{\pgfqpoint{2.922499in}{3.713634in}}%
\pgfpathmoveto{\pgfqpoint{2.913983in}{3.717892in}}%
\pgfpathlineto{\pgfqpoint{2.913983in}{3.717892in}}%
\pgfpathlineto{\pgfqpoint{2.913983in}{3.722149in}}%
\pgfpathlineto{\pgfqpoint{2.918241in}{3.722149in}}%
\pgfpathlineto{\pgfqpoint{2.918241in}{3.717892in}}%
\pgfpathmoveto{\pgfqpoint{2.913983in}{3.722149in}}%
\pgfpathlineto{\pgfqpoint{2.913983in}{3.722149in}}%
\pgfpathlineto{\pgfqpoint{2.913983in}{3.726407in}}%
\pgfpathlineto{\pgfqpoint{2.918241in}{3.726407in}}%
\pgfpathlineto{\pgfqpoint{2.918241in}{3.722149in}}%
\pgfpathmoveto{\pgfqpoint{2.918241in}{3.717892in}}%
\pgfpathlineto{\pgfqpoint{2.918241in}{3.717892in}}%
\pgfpathlineto{\pgfqpoint{2.918241in}{3.722149in}}%
\pgfpathlineto{\pgfqpoint{2.922499in}{3.722149in}}%
\pgfpathlineto{\pgfqpoint{2.922499in}{3.717892in}}%
\pgfpathmoveto{\pgfqpoint{2.918241in}{3.722149in}}%
\pgfpathlineto{\pgfqpoint{2.918241in}{3.722149in}}%
\pgfpathlineto{\pgfqpoint{2.918241in}{3.726407in}}%
\pgfpathlineto{\pgfqpoint{2.922499in}{3.726407in}}%
\pgfpathlineto{\pgfqpoint{2.922499in}{3.722149in}}%
\pgfpathmoveto{\pgfqpoint{2.905467in}{3.726407in}}%
\pgfpathlineto{\pgfqpoint{2.905467in}{3.726407in}}%
\pgfpathlineto{\pgfqpoint{2.905467in}{3.730665in}}%
\pgfpathlineto{\pgfqpoint{2.909725in}{3.730665in}}%
\pgfpathlineto{\pgfqpoint{2.909725in}{3.726407in}}%
\pgfpathmoveto{\pgfqpoint{2.905467in}{3.730665in}}%
\pgfpathlineto{\pgfqpoint{2.905467in}{3.730665in}}%
\pgfpathlineto{\pgfqpoint{2.905467in}{3.734923in}}%
\pgfpathlineto{\pgfqpoint{2.909725in}{3.734923in}}%
\pgfpathlineto{\pgfqpoint{2.909725in}{3.730665in}}%
\pgfpathmoveto{\pgfqpoint{2.909725in}{3.726407in}}%
\pgfpathlineto{\pgfqpoint{2.909725in}{3.726407in}}%
\pgfpathlineto{\pgfqpoint{2.909725in}{3.730665in}}%
\pgfpathlineto{\pgfqpoint{2.913983in}{3.730665in}}%
\pgfpathlineto{\pgfqpoint{2.913983in}{3.726407in}}%
\pgfpathmoveto{\pgfqpoint{2.909725in}{3.730665in}}%
\pgfpathlineto{\pgfqpoint{2.909725in}{3.730665in}}%
\pgfpathlineto{\pgfqpoint{2.909725in}{3.734923in}}%
\pgfpathlineto{\pgfqpoint{2.913983in}{3.734923in}}%
\pgfpathlineto{\pgfqpoint{2.913983in}{3.730665in}}%
\pgfpathmoveto{\pgfqpoint{2.905467in}{3.734923in}}%
\pgfpathlineto{\pgfqpoint{2.905467in}{3.734923in}}%
\pgfpathlineto{\pgfqpoint{2.905467in}{3.739181in}}%
\pgfpathlineto{\pgfqpoint{2.909725in}{3.739181in}}%
\pgfpathlineto{\pgfqpoint{2.909725in}{3.734923in}}%
\pgfpathmoveto{\pgfqpoint{2.905467in}{3.739181in}}%
\pgfpathlineto{\pgfqpoint{2.905467in}{3.739181in}}%
\pgfpathlineto{\pgfqpoint{2.905467in}{3.743439in}}%
\pgfpathlineto{\pgfqpoint{2.909725in}{3.743439in}}%
\pgfpathlineto{\pgfqpoint{2.909725in}{3.739181in}}%
\pgfpathmoveto{\pgfqpoint{2.909725in}{3.734923in}}%
\pgfpathlineto{\pgfqpoint{2.909725in}{3.734923in}}%
\pgfpathlineto{\pgfqpoint{2.909725in}{3.739181in}}%
\pgfpathlineto{\pgfqpoint{2.913983in}{3.739181in}}%
\pgfpathlineto{\pgfqpoint{2.913983in}{3.734923in}}%
\pgfpathmoveto{\pgfqpoint{2.909725in}{3.739181in}}%
\pgfpathlineto{\pgfqpoint{2.909725in}{3.739181in}}%
\pgfpathlineto{\pgfqpoint{2.909725in}{3.743439in}}%
\pgfpathlineto{\pgfqpoint{2.913983in}{3.743439in}}%
\pgfpathlineto{\pgfqpoint{2.913983in}{3.739181in}}%
\pgfpathmoveto{\pgfqpoint{2.913983in}{3.726407in}}%
\pgfpathlineto{\pgfqpoint{2.913983in}{3.726407in}}%
\pgfpathlineto{\pgfqpoint{2.913983in}{3.730665in}}%
\pgfpathlineto{\pgfqpoint{2.918241in}{3.730665in}}%
\pgfpathlineto{\pgfqpoint{2.918241in}{3.726407in}}%
\pgfpathmoveto{\pgfqpoint{2.913983in}{3.730665in}}%
\pgfpathlineto{\pgfqpoint{2.913983in}{3.730665in}}%
\pgfpathlineto{\pgfqpoint{2.913983in}{3.734923in}}%
\pgfpathlineto{\pgfqpoint{2.918241in}{3.734923in}}%
\pgfpathlineto{\pgfqpoint{2.918241in}{3.730665in}}%
\pgfpathmoveto{\pgfqpoint{2.918241in}{3.726407in}}%
\pgfpathlineto{\pgfqpoint{2.918241in}{3.726407in}}%
\pgfpathlineto{\pgfqpoint{2.918241in}{3.730665in}}%
\pgfpathlineto{\pgfqpoint{2.922499in}{3.730665in}}%
\pgfpathlineto{\pgfqpoint{2.922499in}{3.726407in}}%
\pgfpathmoveto{\pgfqpoint{2.918241in}{3.730665in}}%
\pgfpathlineto{\pgfqpoint{2.918241in}{3.730665in}}%
\pgfpathlineto{\pgfqpoint{2.918241in}{3.734923in}}%
\pgfpathlineto{\pgfqpoint{2.922499in}{3.734923in}}%
\pgfpathlineto{\pgfqpoint{2.922499in}{3.730665in}}%
\pgfpathmoveto{\pgfqpoint{2.913983in}{3.734923in}}%
\pgfpathlineto{\pgfqpoint{2.913983in}{3.734923in}}%
\pgfpathlineto{\pgfqpoint{2.913983in}{3.739181in}}%
\pgfpathlineto{\pgfqpoint{2.918241in}{3.739181in}}%
\pgfpathlineto{\pgfqpoint{2.918241in}{3.734923in}}%
\pgfpathmoveto{\pgfqpoint{2.913983in}{3.739181in}}%
\pgfpathlineto{\pgfqpoint{2.913983in}{3.739181in}}%
\pgfpathlineto{\pgfqpoint{2.913983in}{3.743439in}}%
\pgfpathlineto{\pgfqpoint{2.918241in}{3.743439in}}%
\pgfpathlineto{\pgfqpoint{2.918241in}{3.739181in}}%
\pgfpathmoveto{\pgfqpoint{2.918241in}{3.734923in}}%
\pgfpathlineto{\pgfqpoint{2.918241in}{3.734923in}}%
\pgfpathlineto{\pgfqpoint{2.918241in}{3.739181in}}%
\pgfpathlineto{\pgfqpoint{2.922499in}{3.739181in}}%
\pgfpathlineto{\pgfqpoint{2.922499in}{3.734923in}}%
\pgfpathmoveto{\pgfqpoint{2.918241in}{3.739181in}}%
\pgfpathlineto{\pgfqpoint{2.918241in}{3.739181in}}%
\pgfpathlineto{\pgfqpoint{2.918241in}{3.743439in}}%
\pgfpathlineto{\pgfqpoint{2.922499in}{3.743439in}}%
\pgfpathlineto{\pgfqpoint{2.922499in}{3.739181in}}%
\pgfpathmoveto{\pgfqpoint{2.901210in}{3.743439in}}%
\pgfpathlineto{\pgfqpoint{2.901210in}{3.743439in}}%
\pgfpathlineto{\pgfqpoint{2.901210in}{3.747696in}}%
\pgfpathlineto{\pgfqpoint{2.905467in}{3.747696in}}%
\pgfpathlineto{\pgfqpoint{2.905467in}{3.743439in}}%
\pgfpathmoveto{\pgfqpoint{2.905467in}{3.743439in}}%
\pgfpathlineto{\pgfqpoint{2.905467in}{3.743439in}}%
\pgfpathlineto{\pgfqpoint{2.905467in}{3.747696in}}%
\pgfpathlineto{\pgfqpoint{2.909725in}{3.747696in}}%
\pgfpathlineto{\pgfqpoint{2.909725in}{3.743439in}}%
\pgfpathmoveto{\pgfqpoint{2.905467in}{3.747696in}}%
\pgfpathlineto{\pgfqpoint{2.905467in}{3.747696in}}%
\pgfpathlineto{\pgfqpoint{2.905467in}{3.751954in}}%
\pgfpathlineto{\pgfqpoint{2.909725in}{3.751954in}}%
\pgfpathlineto{\pgfqpoint{2.909725in}{3.747696in}}%
\pgfpathmoveto{\pgfqpoint{2.909725in}{3.743439in}}%
\pgfpathlineto{\pgfqpoint{2.909725in}{3.743439in}}%
\pgfpathlineto{\pgfqpoint{2.909725in}{3.747696in}}%
\pgfpathlineto{\pgfqpoint{2.913983in}{3.747696in}}%
\pgfpathlineto{\pgfqpoint{2.913983in}{3.743439in}}%
\pgfpathmoveto{\pgfqpoint{2.909725in}{3.747696in}}%
\pgfpathlineto{\pgfqpoint{2.909725in}{3.747696in}}%
\pgfpathlineto{\pgfqpoint{2.909725in}{3.751954in}}%
\pgfpathlineto{\pgfqpoint{2.913983in}{3.751954in}}%
\pgfpathlineto{\pgfqpoint{2.913983in}{3.747696in}}%
\pgfpathmoveto{\pgfqpoint{2.909725in}{3.751954in}}%
\pgfpathlineto{\pgfqpoint{2.909725in}{3.751954in}}%
\pgfpathlineto{\pgfqpoint{2.909725in}{3.756212in}}%
\pgfpathlineto{\pgfqpoint{2.913983in}{3.756212in}}%
\pgfpathlineto{\pgfqpoint{2.913983in}{3.751954in}}%
\pgfpathmoveto{\pgfqpoint{2.913983in}{3.743439in}}%
\pgfpathlineto{\pgfqpoint{2.913983in}{3.743439in}}%
\pgfpathlineto{\pgfqpoint{2.913983in}{3.747696in}}%
\pgfpathlineto{\pgfqpoint{2.918241in}{3.747696in}}%
\pgfpathlineto{\pgfqpoint{2.918241in}{3.743439in}}%
\pgfpathmoveto{\pgfqpoint{2.913983in}{3.747696in}}%
\pgfpathlineto{\pgfqpoint{2.913983in}{3.747696in}}%
\pgfpathlineto{\pgfqpoint{2.913983in}{3.751954in}}%
\pgfpathlineto{\pgfqpoint{2.918241in}{3.751954in}}%
\pgfpathlineto{\pgfqpoint{2.918241in}{3.747696in}}%
\pgfpathmoveto{\pgfqpoint{2.918241in}{3.743439in}}%
\pgfpathlineto{\pgfqpoint{2.918241in}{3.743439in}}%
\pgfpathlineto{\pgfqpoint{2.918241in}{3.747696in}}%
\pgfpathlineto{\pgfqpoint{2.922499in}{3.747696in}}%
\pgfpathlineto{\pgfqpoint{2.922499in}{3.743439in}}%
\pgfpathmoveto{\pgfqpoint{2.918241in}{3.747696in}}%
\pgfpathlineto{\pgfqpoint{2.918241in}{3.747696in}}%
\pgfpathlineto{\pgfqpoint{2.918241in}{3.751954in}}%
\pgfpathlineto{\pgfqpoint{2.922499in}{3.751954in}}%
\pgfpathlineto{\pgfqpoint{2.922499in}{3.747696in}}%
\pgfpathmoveto{\pgfqpoint{2.913983in}{3.751954in}}%
\pgfpathlineto{\pgfqpoint{2.913983in}{3.751954in}}%
\pgfpathlineto{\pgfqpoint{2.913983in}{3.756212in}}%
\pgfpathlineto{\pgfqpoint{2.918241in}{3.756212in}}%
\pgfpathlineto{\pgfqpoint{2.918241in}{3.751954in}}%
\pgfpathmoveto{\pgfqpoint{2.913983in}{3.756212in}}%
\pgfpathlineto{\pgfqpoint{2.913983in}{3.756212in}}%
\pgfpathlineto{\pgfqpoint{2.913983in}{3.760470in}}%
\pgfpathlineto{\pgfqpoint{2.918241in}{3.760470in}}%
\pgfpathlineto{\pgfqpoint{2.918241in}{3.756212in}}%
\pgfpathmoveto{\pgfqpoint{2.918241in}{3.751954in}}%
\pgfpathlineto{\pgfqpoint{2.918241in}{3.751954in}}%
\pgfpathlineto{\pgfqpoint{2.918241in}{3.756212in}}%
\pgfpathlineto{\pgfqpoint{2.922499in}{3.756212in}}%
\pgfpathlineto{\pgfqpoint{2.922499in}{3.751954in}}%
\pgfpathmoveto{\pgfqpoint{2.918241in}{3.756212in}}%
\pgfpathlineto{\pgfqpoint{2.918241in}{3.756212in}}%
\pgfpathlineto{\pgfqpoint{2.918241in}{3.760470in}}%
\pgfpathlineto{\pgfqpoint{2.922499in}{3.760470in}}%
\pgfpathlineto{\pgfqpoint{2.922499in}{3.756212in}}%
\pgfpathmoveto{\pgfqpoint{2.918241in}{3.760470in}}%
\pgfpathlineto{\pgfqpoint{2.918241in}{3.760470in}}%
\pgfpathlineto{\pgfqpoint{2.918241in}{3.764728in}}%
\pgfpathlineto{\pgfqpoint{2.922499in}{3.764728in}}%
\pgfpathlineto{\pgfqpoint{2.922499in}{3.760470in}}%
\pgfpathmoveto{\pgfqpoint{2.922499in}{3.717892in}}%
\pgfpathlineto{\pgfqpoint{2.922499in}{3.717892in}}%
\pgfpathlineto{\pgfqpoint{2.922499in}{3.722149in}}%
\pgfpathlineto{\pgfqpoint{2.926757in}{3.722149in}}%
\pgfpathlineto{\pgfqpoint{2.926757in}{3.717892in}}%
\pgfpathmoveto{\pgfqpoint{2.922499in}{3.722149in}}%
\pgfpathlineto{\pgfqpoint{2.922499in}{3.722149in}}%
\pgfpathlineto{\pgfqpoint{2.922499in}{3.726407in}}%
\pgfpathlineto{\pgfqpoint{2.926757in}{3.726407in}}%
\pgfpathlineto{\pgfqpoint{2.926757in}{3.722149in}}%
\pgfpathmoveto{\pgfqpoint{2.926757in}{3.717892in}}%
\pgfpathlineto{\pgfqpoint{2.926757in}{3.717892in}}%
\pgfpathlineto{\pgfqpoint{2.926757in}{3.722149in}}%
\pgfpathlineto{\pgfqpoint{2.931014in}{3.722149in}}%
\pgfpathlineto{\pgfqpoint{2.931014in}{3.717892in}}%
\pgfpathmoveto{\pgfqpoint{2.926757in}{3.722149in}}%
\pgfpathlineto{\pgfqpoint{2.926757in}{3.722149in}}%
\pgfpathlineto{\pgfqpoint{2.926757in}{3.726407in}}%
\pgfpathlineto{\pgfqpoint{2.931014in}{3.726407in}}%
\pgfpathlineto{\pgfqpoint{2.931014in}{3.722149in}}%
\pgfpathmoveto{\pgfqpoint{2.931014in}{3.722149in}}%
\pgfpathlineto{\pgfqpoint{2.931014in}{3.722149in}}%
\pgfpathlineto{\pgfqpoint{2.931014in}{3.726407in}}%
\pgfpathlineto{\pgfqpoint{2.935272in}{3.726407in}}%
\pgfpathlineto{\pgfqpoint{2.935272in}{3.722149in}}%
\pgfpathmoveto{\pgfqpoint{2.935272in}{3.722149in}}%
\pgfpathlineto{\pgfqpoint{2.935272in}{3.722149in}}%
\pgfpathlineto{\pgfqpoint{2.935272in}{3.726407in}}%
\pgfpathlineto{\pgfqpoint{2.939530in}{3.726407in}}%
\pgfpathlineto{\pgfqpoint{2.939530in}{3.722149in}}%
\pgfpathmoveto{\pgfqpoint{2.922499in}{3.726407in}}%
\pgfpathlineto{\pgfqpoint{2.922499in}{3.726407in}}%
\pgfpathlineto{\pgfqpoint{2.922499in}{3.730665in}}%
\pgfpathlineto{\pgfqpoint{2.926757in}{3.730665in}}%
\pgfpathlineto{\pgfqpoint{2.926757in}{3.726407in}}%
\pgfpathmoveto{\pgfqpoint{2.922499in}{3.730665in}}%
\pgfpathlineto{\pgfqpoint{2.922499in}{3.730665in}}%
\pgfpathlineto{\pgfqpoint{2.922499in}{3.734923in}}%
\pgfpathlineto{\pgfqpoint{2.926757in}{3.734923in}}%
\pgfpathlineto{\pgfqpoint{2.926757in}{3.730665in}}%
\pgfpathmoveto{\pgfqpoint{2.926757in}{3.726407in}}%
\pgfpathlineto{\pgfqpoint{2.926757in}{3.726407in}}%
\pgfpathlineto{\pgfqpoint{2.926757in}{3.730665in}}%
\pgfpathlineto{\pgfqpoint{2.931014in}{3.730665in}}%
\pgfpathlineto{\pgfqpoint{2.931014in}{3.726407in}}%
\pgfpathmoveto{\pgfqpoint{2.926757in}{3.730665in}}%
\pgfpathlineto{\pgfqpoint{2.926757in}{3.730665in}}%
\pgfpathlineto{\pgfqpoint{2.926757in}{3.734923in}}%
\pgfpathlineto{\pgfqpoint{2.931014in}{3.734923in}}%
\pgfpathlineto{\pgfqpoint{2.931014in}{3.730665in}}%
\pgfpathmoveto{\pgfqpoint{2.922499in}{3.734923in}}%
\pgfpathlineto{\pgfqpoint{2.922499in}{3.734923in}}%
\pgfpathlineto{\pgfqpoint{2.922499in}{3.739181in}}%
\pgfpathlineto{\pgfqpoint{2.926757in}{3.739181in}}%
\pgfpathlineto{\pgfqpoint{2.926757in}{3.734923in}}%
\pgfpathmoveto{\pgfqpoint{2.922499in}{3.739181in}}%
\pgfpathlineto{\pgfqpoint{2.922499in}{3.739181in}}%
\pgfpathlineto{\pgfqpoint{2.922499in}{3.743439in}}%
\pgfpathlineto{\pgfqpoint{2.926757in}{3.743439in}}%
\pgfpathlineto{\pgfqpoint{2.926757in}{3.739181in}}%
\pgfpathmoveto{\pgfqpoint{2.926757in}{3.734923in}}%
\pgfpathlineto{\pgfqpoint{2.926757in}{3.734923in}}%
\pgfpathlineto{\pgfqpoint{2.926757in}{3.739181in}}%
\pgfpathlineto{\pgfqpoint{2.931014in}{3.739181in}}%
\pgfpathlineto{\pgfqpoint{2.931014in}{3.734923in}}%
\pgfpathmoveto{\pgfqpoint{2.926757in}{3.739181in}}%
\pgfpathlineto{\pgfqpoint{2.926757in}{3.739181in}}%
\pgfpathlineto{\pgfqpoint{2.926757in}{3.743439in}}%
\pgfpathlineto{\pgfqpoint{2.931014in}{3.743439in}}%
\pgfpathlineto{\pgfqpoint{2.931014in}{3.739181in}}%
\pgfpathmoveto{\pgfqpoint{2.931014in}{3.726407in}}%
\pgfpathlineto{\pgfqpoint{2.931014in}{3.726407in}}%
\pgfpathlineto{\pgfqpoint{2.931014in}{3.730665in}}%
\pgfpathlineto{\pgfqpoint{2.935272in}{3.730665in}}%
\pgfpathlineto{\pgfqpoint{2.935272in}{3.726407in}}%
\pgfpathmoveto{\pgfqpoint{2.931014in}{3.730665in}}%
\pgfpathlineto{\pgfqpoint{2.931014in}{3.730665in}}%
\pgfpathlineto{\pgfqpoint{2.931014in}{3.734923in}}%
\pgfpathlineto{\pgfqpoint{2.935272in}{3.734923in}}%
\pgfpathlineto{\pgfqpoint{2.935272in}{3.730665in}}%
\pgfpathmoveto{\pgfqpoint{2.935272in}{3.726407in}}%
\pgfpathlineto{\pgfqpoint{2.935272in}{3.726407in}}%
\pgfpathlineto{\pgfqpoint{2.935272in}{3.730665in}}%
\pgfpathlineto{\pgfqpoint{2.939530in}{3.730665in}}%
\pgfpathlineto{\pgfqpoint{2.939530in}{3.726407in}}%
\pgfpathmoveto{\pgfqpoint{2.935272in}{3.730665in}}%
\pgfpathlineto{\pgfqpoint{2.935272in}{3.730665in}}%
\pgfpathlineto{\pgfqpoint{2.935272in}{3.734923in}}%
\pgfpathlineto{\pgfqpoint{2.939530in}{3.734923in}}%
\pgfpathlineto{\pgfqpoint{2.939530in}{3.730665in}}%
\pgfpathmoveto{\pgfqpoint{2.931014in}{3.734923in}}%
\pgfpathlineto{\pgfqpoint{2.931014in}{3.734923in}}%
\pgfpathlineto{\pgfqpoint{2.931014in}{3.739181in}}%
\pgfpathlineto{\pgfqpoint{2.935272in}{3.739181in}}%
\pgfpathlineto{\pgfqpoint{2.935272in}{3.734923in}}%
\pgfpathmoveto{\pgfqpoint{2.931014in}{3.739181in}}%
\pgfpathlineto{\pgfqpoint{2.931014in}{3.739181in}}%
\pgfpathlineto{\pgfqpoint{2.931014in}{3.743439in}}%
\pgfpathlineto{\pgfqpoint{2.935272in}{3.743439in}}%
\pgfpathlineto{\pgfqpoint{2.935272in}{3.739181in}}%
\pgfpathmoveto{\pgfqpoint{2.935272in}{3.734923in}}%
\pgfpathlineto{\pgfqpoint{2.935272in}{3.734923in}}%
\pgfpathlineto{\pgfqpoint{2.935272in}{3.739181in}}%
\pgfpathlineto{\pgfqpoint{2.939530in}{3.739181in}}%
\pgfpathlineto{\pgfqpoint{2.939530in}{3.734923in}}%
\pgfpathmoveto{\pgfqpoint{2.935272in}{3.739181in}}%
\pgfpathlineto{\pgfqpoint{2.935272in}{3.739181in}}%
\pgfpathlineto{\pgfqpoint{2.935272in}{3.743439in}}%
\pgfpathlineto{\pgfqpoint{2.939530in}{3.743439in}}%
\pgfpathlineto{\pgfqpoint{2.939530in}{3.739181in}}%
\pgfpathmoveto{\pgfqpoint{2.939530in}{3.722149in}}%
\pgfpathlineto{\pgfqpoint{2.939530in}{3.722149in}}%
\pgfpathlineto{\pgfqpoint{2.939530in}{3.726407in}}%
\pgfpathlineto{\pgfqpoint{2.943788in}{3.726407in}}%
\pgfpathlineto{\pgfqpoint{2.943788in}{3.722149in}}%
\pgfpathmoveto{\pgfqpoint{2.939530in}{3.726407in}}%
\pgfpathlineto{\pgfqpoint{2.939530in}{3.726407in}}%
\pgfpathlineto{\pgfqpoint{2.939530in}{3.730665in}}%
\pgfpathlineto{\pgfqpoint{2.943788in}{3.730665in}}%
\pgfpathlineto{\pgfqpoint{2.943788in}{3.726407in}}%
\pgfpathmoveto{\pgfqpoint{2.939530in}{3.730665in}}%
\pgfpathlineto{\pgfqpoint{2.939530in}{3.730665in}}%
\pgfpathlineto{\pgfqpoint{2.939530in}{3.734923in}}%
\pgfpathlineto{\pgfqpoint{2.943788in}{3.734923in}}%
\pgfpathlineto{\pgfqpoint{2.943788in}{3.730665in}}%
\pgfpathmoveto{\pgfqpoint{2.943788in}{3.726407in}}%
\pgfpathlineto{\pgfqpoint{2.943788in}{3.726407in}}%
\pgfpathlineto{\pgfqpoint{2.943788in}{3.730665in}}%
\pgfpathlineto{\pgfqpoint{2.948046in}{3.730665in}}%
\pgfpathlineto{\pgfqpoint{2.948046in}{3.726407in}}%
\pgfpathmoveto{\pgfqpoint{2.943788in}{3.730665in}}%
\pgfpathlineto{\pgfqpoint{2.943788in}{3.730665in}}%
\pgfpathlineto{\pgfqpoint{2.943788in}{3.734923in}}%
\pgfpathlineto{\pgfqpoint{2.948046in}{3.734923in}}%
\pgfpathlineto{\pgfqpoint{2.948046in}{3.730665in}}%
\pgfpathmoveto{\pgfqpoint{2.939530in}{3.734923in}}%
\pgfpathlineto{\pgfqpoint{2.939530in}{3.734923in}}%
\pgfpathlineto{\pgfqpoint{2.939530in}{3.739181in}}%
\pgfpathlineto{\pgfqpoint{2.943788in}{3.739181in}}%
\pgfpathlineto{\pgfqpoint{2.943788in}{3.734923in}}%
\pgfpathmoveto{\pgfqpoint{2.939530in}{3.739181in}}%
\pgfpathlineto{\pgfqpoint{2.939530in}{3.739181in}}%
\pgfpathlineto{\pgfqpoint{2.939530in}{3.743439in}}%
\pgfpathlineto{\pgfqpoint{2.943788in}{3.743439in}}%
\pgfpathlineto{\pgfqpoint{2.943788in}{3.739181in}}%
\pgfpathmoveto{\pgfqpoint{2.943788in}{3.734923in}}%
\pgfpathlineto{\pgfqpoint{2.943788in}{3.734923in}}%
\pgfpathlineto{\pgfqpoint{2.943788in}{3.739181in}}%
\pgfpathlineto{\pgfqpoint{2.948046in}{3.739181in}}%
\pgfpathlineto{\pgfqpoint{2.948046in}{3.734923in}}%
\pgfpathmoveto{\pgfqpoint{2.943788in}{3.739181in}}%
\pgfpathlineto{\pgfqpoint{2.943788in}{3.739181in}}%
\pgfpathlineto{\pgfqpoint{2.943788in}{3.743439in}}%
\pgfpathlineto{\pgfqpoint{2.948046in}{3.743439in}}%
\pgfpathlineto{\pgfqpoint{2.948046in}{3.739181in}}%
\pgfpathmoveto{\pgfqpoint{2.948046in}{3.726407in}}%
\pgfpathlineto{\pgfqpoint{2.948046in}{3.726407in}}%
\pgfpathlineto{\pgfqpoint{2.948046in}{3.730665in}}%
\pgfpathlineto{\pgfqpoint{2.952303in}{3.730665in}}%
\pgfpathlineto{\pgfqpoint{2.952303in}{3.726407in}}%
\pgfpathmoveto{\pgfqpoint{2.948046in}{3.730665in}}%
\pgfpathlineto{\pgfqpoint{2.948046in}{3.730665in}}%
\pgfpathlineto{\pgfqpoint{2.948046in}{3.734923in}}%
\pgfpathlineto{\pgfqpoint{2.952303in}{3.734923in}}%
\pgfpathlineto{\pgfqpoint{2.952303in}{3.730665in}}%
\pgfpathmoveto{\pgfqpoint{2.952303in}{3.726407in}}%
\pgfpathlineto{\pgfqpoint{2.952303in}{3.726407in}}%
\pgfpathlineto{\pgfqpoint{2.952303in}{3.730665in}}%
\pgfpathlineto{\pgfqpoint{2.956561in}{3.730665in}}%
\pgfpathlineto{\pgfqpoint{2.956561in}{3.726407in}}%
\pgfpathmoveto{\pgfqpoint{2.952303in}{3.730665in}}%
\pgfpathlineto{\pgfqpoint{2.952303in}{3.730665in}}%
\pgfpathlineto{\pgfqpoint{2.952303in}{3.734923in}}%
\pgfpathlineto{\pgfqpoint{2.956561in}{3.734923in}}%
\pgfpathlineto{\pgfqpoint{2.956561in}{3.730665in}}%
\pgfpathmoveto{\pgfqpoint{2.948046in}{3.734923in}}%
\pgfpathlineto{\pgfqpoint{2.948046in}{3.734923in}}%
\pgfpathlineto{\pgfqpoint{2.948046in}{3.739181in}}%
\pgfpathlineto{\pgfqpoint{2.952303in}{3.739181in}}%
\pgfpathlineto{\pgfqpoint{2.952303in}{3.734923in}}%
\pgfpathmoveto{\pgfqpoint{2.948046in}{3.739181in}}%
\pgfpathlineto{\pgfqpoint{2.948046in}{3.739181in}}%
\pgfpathlineto{\pgfqpoint{2.948046in}{3.743439in}}%
\pgfpathlineto{\pgfqpoint{2.952303in}{3.743439in}}%
\pgfpathlineto{\pgfqpoint{2.952303in}{3.739181in}}%
\pgfpathmoveto{\pgfqpoint{2.952303in}{3.734923in}}%
\pgfpathlineto{\pgfqpoint{2.952303in}{3.734923in}}%
\pgfpathlineto{\pgfqpoint{2.952303in}{3.739181in}}%
\pgfpathlineto{\pgfqpoint{2.956561in}{3.739181in}}%
\pgfpathlineto{\pgfqpoint{2.956561in}{3.734923in}}%
\pgfpathmoveto{\pgfqpoint{2.952303in}{3.739181in}}%
\pgfpathlineto{\pgfqpoint{2.952303in}{3.739181in}}%
\pgfpathlineto{\pgfqpoint{2.952303in}{3.743439in}}%
\pgfpathlineto{\pgfqpoint{2.956561in}{3.743439in}}%
\pgfpathlineto{\pgfqpoint{2.956561in}{3.739181in}}%
\pgfpathmoveto{\pgfqpoint{2.922499in}{3.743439in}}%
\pgfpathlineto{\pgfqpoint{2.922499in}{3.743439in}}%
\pgfpathlineto{\pgfqpoint{2.922499in}{3.747696in}}%
\pgfpathlineto{\pgfqpoint{2.926757in}{3.747696in}}%
\pgfpathlineto{\pgfqpoint{2.926757in}{3.743439in}}%
\pgfpathmoveto{\pgfqpoint{2.922499in}{3.747696in}}%
\pgfpathlineto{\pgfqpoint{2.922499in}{3.747696in}}%
\pgfpathlineto{\pgfqpoint{2.922499in}{3.751954in}}%
\pgfpathlineto{\pgfqpoint{2.926757in}{3.751954in}}%
\pgfpathlineto{\pgfqpoint{2.926757in}{3.747696in}}%
\pgfpathmoveto{\pgfqpoint{2.926757in}{3.743439in}}%
\pgfpathlineto{\pgfqpoint{2.926757in}{3.743439in}}%
\pgfpathlineto{\pgfqpoint{2.926757in}{3.747696in}}%
\pgfpathlineto{\pgfqpoint{2.931014in}{3.747696in}}%
\pgfpathlineto{\pgfqpoint{2.931014in}{3.743439in}}%
\pgfpathmoveto{\pgfqpoint{2.926757in}{3.747696in}}%
\pgfpathlineto{\pgfqpoint{2.926757in}{3.747696in}}%
\pgfpathlineto{\pgfqpoint{2.926757in}{3.751954in}}%
\pgfpathlineto{\pgfqpoint{2.931014in}{3.751954in}}%
\pgfpathlineto{\pgfqpoint{2.931014in}{3.747696in}}%
\pgfpathmoveto{\pgfqpoint{2.922499in}{3.751954in}}%
\pgfpathlineto{\pgfqpoint{2.922499in}{3.751954in}}%
\pgfpathlineto{\pgfqpoint{2.922499in}{3.756212in}}%
\pgfpathlineto{\pgfqpoint{2.926757in}{3.756212in}}%
\pgfpathlineto{\pgfqpoint{2.926757in}{3.751954in}}%
\pgfpathmoveto{\pgfqpoint{2.922499in}{3.756212in}}%
\pgfpathlineto{\pgfqpoint{2.922499in}{3.756212in}}%
\pgfpathlineto{\pgfqpoint{2.922499in}{3.760470in}}%
\pgfpathlineto{\pgfqpoint{2.926757in}{3.760470in}}%
\pgfpathlineto{\pgfqpoint{2.926757in}{3.756212in}}%
\pgfpathmoveto{\pgfqpoint{2.926757in}{3.751954in}}%
\pgfpathlineto{\pgfqpoint{2.926757in}{3.751954in}}%
\pgfpathlineto{\pgfqpoint{2.926757in}{3.756212in}}%
\pgfpathlineto{\pgfqpoint{2.931014in}{3.756212in}}%
\pgfpathlineto{\pgfqpoint{2.931014in}{3.751954in}}%
\pgfpathmoveto{\pgfqpoint{2.926757in}{3.756212in}}%
\pgfpathlineto{\pgfqpoint{2.926757in}{3.756212in}}%
\pgfpathlineto{\pgfqpoint{2.926757in}{3.760470in}}%
\pgfpathlineto{\pgfqpoint{2.931014in}{3.760470in}}%
\pgfpathlineto{\pgfqpoint{2.931014in}{3.756212in}}%
\pgfpathmoveto{\pgfqpoint{2.931014in}{3.743439in}}%
\pgfpathlineto{\pgfqpoint{2.931014in}{3.743439in}}%
\pgfpathlineto{\pgfqpoint{2.931014in}{3.747696in}}%
\pgfpathlineto{\pgfqpoint{2.935272in}{3.747696in}}%
\pgfpathlineto{\pgfqpoint{2.935272in}{3.743439in}}%
\pgfpathmoveto{\pgfqpoint{2.931014in}{3.747696in}}%
\pgfpathlineto{\pgfqpoint{2.931014in}{3.747696in}}%
\pgfpathlineto{\pgfqpoint{2.931014in}{3.751954in}}%
\pgfpathlineto{\pgfqpoint{2.935272in}{3.751954in}}%
\pgfpathlineto{\pgfqpoint{2.935272in}{3.747696in}}%
\pgfpathmoveto{\pgfqpoint{2.935272in}{3.743439in}}%
\pgfpathlineto{\pgfqpoint{2.935272in}{3.743439in}}%
\pgfpathlineto{\pgfqpoint{2.935272in}{3.747696in}}%
\pgfpathlineto{\pgfqpoint{2.939530in}{3.747696in}}%
\pgfpathlineto{\pgfqpoint{2.939530in}{3.743439in}}%
\pgfpathmoveto{\pgfqpoint{2.935272in}{3.747696in}}%
\pgfpathlineto{\pgfqpoint{2.935272in}{3.747696in}}%
\pgfpathlineto{\pgfqpoint{2.935272in}{3.751954in}}%
\pgfpathlineto{\pgfqpoint{2.939530in}{3.751954in}}%
\pgfpathlineto{\pgfqpoint{2.939530in}{3.747696in}}%
\pgfpathmoveto{\pgfqpoint{2.931014in}{3.751954in}}%
\pgfpathlineto{\pgfqpoint{2.931014in}{3.751954in}}%
\pgfpathlineto{\pgfqpoint{2.931014in}{3.756212in}}%
\pgfpathlineto{\pgfqpoint{2.935272in}{3.756212in}}%
\pgfpathlineto{\pgfqpoint{2.935272in}{3.751954in}}%
\pgfpathmoveto{\pgfqpoint{2.931014in}{3.756212in}}%
\pgfpathlineto{\pgfqpoint{2.931014in}{3.756212in}}%
\pgfpathlineto{\pgfqpoint{2.931014in}{3.760470in}}%
\pgfpathlineto{\pgfqpoint{2.935272in}{3.760470in}}%
\pgfpathlineto{\pgfqpoint{2.935272in}{3.756212in}}%
\pgfpathmoveto{\pgfqpoint{2.935272in}{3.751954in}}%
\pgfpathlineto{\pgfqpoint{2.935272in}{3.751954in}}%
\pgfpathlineto{\pgfqpoint{2.935272in}{3.756212in}}%
\pgfpathlineto{\pgfqpoint{2.939530in}{3.756212in}}%
\pgfpathlineto{\pgfqpoint{2.939530in}{3.751954in}}%
\pgfpathmoveto{\pgfqpoint{2.935272in}{3.756212in}}%
\pgfpathlineto{\pgfqpoint{2.935272in}{3.756212in}}%
\pgfpathlineto{\pgfqpoint{2.935272in}{3.760470in}}%
\pgfpathlineto{\pgfqpoint{2.939530in}{3.760470in}}%
\pgfpathlineto{\pgfqpoint{2.939530in}{3.756212in}}%
\pgfpathmoveto{\pgfqpoint{2.922499in}{3.760470in}}%
\pgfpathlineto{\pgfqpoint{2.922499in}{3.760470in}}%
\pgfpathlineto{\pgfqpoint{2.922499in}{3.764728in}}%
\pgfpathlineto{\pgfqpoint{2.926757in}{3.764728in}}%
\pgfpathlineto{\pgfqpoint{2.926757in}{3.760470in}}%
\pgfpathmoveto{\pgfqpoint{2.926757in}{3.760470in}}%
\pgfpathlineto{\pgfqpoint{2.926757in}{3.760470in}}%
\pgfpathlineto{\pgfqpoint{2.926757in}{3.764728in}}%
\pgfpathlineto{\pgfqpoint{2.931014in}{3.764728in}}%
\pgfpathlineto{\pgfqpoint{2.931014in}{3.760470in}}%
\pgfpathmoveto{\pgfqpoint{2.926757in}{3.764728in}}%
\pgfpathlineto{\pgfqpoint{2.926757in}{3.764728in}}%
\pgfpathlineto{\pgfqpoint{2.926757in}{3.768986in}}%
\pgfpathlineto{\pgfqpoint{2.931014in}{3.768986in}}%
\pgfpathlineto{\pgfqpoint{2.931014in}{3.764728in}}%
\pgfpathmoveto{\pgfqpoint{2.931014in}{3.760470in}}%
\pgfpathlineto{\pgfqpoint{2.931014in}{3.760470in}}%
\pgfpathlineto{\pgfqpoint{2.931014in}{3.764728in}}%
\pgfpathlineto{\pgfqpoint{2.935272in}{3.764728in}}%
\pgfpathlineto{\pgfqpoint{2.935272in}{3.760470in}}%
\pgfpathmoveto{\pgfqpoint{2.931014in}{3.764728in}}%
\pgfpathlineto{\pgfqpoint{2.931014in}{3.764728in}}%
\pgfpathlineto{\pgfqpoint{2.931014in}{3.768986in}}%
\pgfpathlineto{\pgfqpoint{2.935272in}{3.768986in}}%
\pgfpathlineto{\pgfqpoint{2.935272in}{3.764728in}}%
\pgfpathmoveto{\pgfqpoint{2.935272in}{3.760470in}}%
\pgfpathlineto{\pgfqpoint{2.935272in}{3.760470in}}%
\pgfpathlineto{\pgfqpoint{2.935272in}{3.764728in}}%
\pgfpathlineto{\pgfqpoint{2.939530in}{3.764728in}}%
\pgfpathlineto{\pgfqpoint{2.939530in}{3.760470in}}%
\pgfpathmoveto{\pgfqpoint{2.935272in}{3.764728in}}%
\pgfpathlineto{\pgfqpoint{2.935272in}{3.764728in}}%
\pgfpathlineto{\pgfqpoint{2.935272in}{3.768986in}}%
\pgfpathlineto{\pgfqpoint{2.939530in}{3.768986in}}%
\pgfpathlineto{\pgfqpoint{2.939530in}{3.764728in}}%
\pgfpathmoveto{\pgfqpoint{2.939530in}{3.743439in}}%
\pgfpathlineto{\pgfqpoint{2.939530in}{3.743439in}}%
\pgfpathlineto{\pgfqpoint{2.939530in}{3.747696in}}%
\pgfpathlineto{\pgfqpoint{2.943788in}{3.747696in}}%
\pgfpathlineto{\pgfqpoint{2.943788in}{3.743439in}}%
\pgfpathmoveto{\pgfqpoint{2.939530in}{3.747696in}}%
\pgfpathlineto{\pgfqpoint{2.939530in}{3.747696in}}%
\pgfpathlineto{\pgfqpoint{2.939530in}{3.751954in}}%
\pgfpathlineto{\pgfqpoint{2.943788in}{3.751954in}}%
\pgfpathlineto{\pgfqpoint{2.943788in}{3.747696in}}%
\pgfpathmoveto{\pgfqpoint{2.943788in}{3.743439in}}%
\pgfpathlineto{\pgfqpoint{2.943788in}{3.743439in}}%
\pgfpathlineto{\pgfqpoint{2.943788in}{3.747696in}}%
\pgfpathlineto{\pgfqpoint{2.948046in}{3.747696in}}%
\pgfpathlineto{\pgfqpoint{2.948046in}{3.743439in}}%
\pgfpathmoveto{\pgfqpoint{2.943788in}{3.747696in}}%
\pgfpathlineto{\pgfqpoint{2.943788in}{3.747696in}}%
\pgfpathlineto{\pgfqpoint{2.943788in}{3.751954in}}%
\pgfpathlineto{\pgfqpoint{2.948046in}{3.751954in}}%
\pgfpathlineto{\pgfqpoint{2.948046in}{3.747696in}}%
\pgfpathmoveto{\pgfqpoint{2.939530in}{3.751954in}}%
\pgfpathlineto{\pgfqpoint{2.939530in}{3.751954in}}%
\pgfpathlineto{\pgfqpoint{2.939530in}{3.756212in}}%
\pgfpathlineto{\pgfqpoint{2.943788in}{3.756212in}}%
\pgfpathlineto{\pgfqpoint{2.943788in}{3.751954in}}%
\pgfpathmoveto{\pgfqpoint{2.939530in}{3.756212in}}%
\pgfpathlineto{\pgfqpoint{2.939530in}{3.756212in}}%
\pgfpathlineto{\pgfqpoint{2.939530in}{3.760470in}}%
\pgfpathlineto{\pgfqpoint{2.943788in}{3.760470in}}%
\pgfpathlineto{\pgfqpoint{2.943788in}{3.756212in}}%
\pgfpathmoveto{\pgfqpoint{2.943788in}{3.751954in}}%
\pgfpathlineto{\pgfqpoint{2.943788in}{3.751954in}}%
\pgfpathlineto{\pgfqpoint{2.943788in}{3.756212in}}%
\pgfpathlineto{\pgfqpoint{2.948046in}{3.756212in}}%
\pgfpathlineto{\pgfqpoint{2.948046in}{3.751954in}}%
\pgfpathmoveto{\pgfqpoint{2.943788in}{3.756212in}}%
\pgfpathlineto{\pgfqpoint{2.943788in}{3.756212in}}%
\pgfpathlineto{\pgfqpoint{2.943788in}{3.760470in}}%
\pgfpathlineto{\pgfqpoint{2.948046in}{3.760470in}}%
\pgfpathlineto{\pgfqpoint{2.948046in}{3.756212in}}%
\pgfpathmoveto{\pgfqpoint{2.948046in}{3.743439in}}%
\pgfpathlineto{\pgfqpoint{2.948046in}{3.743439in}}%
\pgfpathlineto{\pgfqpoint{2.948046in}{3.747696in}}%
\pgfpathlineto{\pgfqpoint{2.952303in}{3.747696in}}%
\pgfpathlineto{\pgfqpoint{2.952303in}{3.743439in}}%
\pgfpathmoveto{\pgfqpoint{2.948046in}{3.747696in}}%
\pgfpathlineto{\pgfqpoint{2.948046in}{3.747696in}}%
\pgfpathlineto{\pgfqpoint{2.948046in}{3.751954in}}%
\pgfpathlineto{\pgfqpoint{2.952303in}{3.751954in}}%
\pgfpathlineto{\pgfqpoint{2.952303in}{3.747696in}}%
\pgfpathmoveto{\pgfqpoint{2.952303in}{3.743439in}}%
\pgfpathlineto{\pgfqpoint{2.952303in}{3.743439in}}%
\pgfpathlineto{\pgfqpoint{2.952303in}{3.747696in}}%
\pgfpathlineto{\pgfqpoint{2.956561in}{3.747696in}}%
\pgfpathlineto{\pgfqpoint{2.956561in}{3.743439in}}%
\pgfpathmoveto{\pgfqpoint{2.952303in}{3.747696in}}%
\pgfpathlineto{\pgfqpoint{2.952303in}{3.747696in}}%
\pgfpathlineto{\pgfqpoint{2.952303in}{3.751954in}}%
\pgfpathlineto{\pgfqpoint{2.956561in}{3.751954in}}%
\pgfpathlineto{\pgfqpoint{2.956561in}{3.747696in}}%
\pgfpathmoveto{\pgfqpoint{2.948046in}{3.751954in}}%
\pgfpathlineto{\pgfqpoint{2.948046in}{3.751954in}}%
\pgfpathlineto{\pgfqpoint{2.948046in}{3.756212in}}%
\pgfpathlineto{\pgfqpoint{2.952303in}{3.756212in}}%
\pgfpathlineto{\pgfqpoint{2.952303in}{3.751954in}}%
\pgfpathmoveto{\pgfqpoint{2.948046in}{3.756212in}}%
\pgfpathlineto{\pgfqpoint{2.948046in}{3.756212in}}%
\pgfpathlineto{\pgfqpoint{2.948046in}{3.760470in}}%
\pgfpathlineto{\pgfqpoint{2.952303in}{3.760470in}}%
\pgfpathlineto{\pgfqpoint{2.952303in}{3.756212in}}%
\pgfpathmoveto{\pgfqpoint{2.952303in}{3.751954in}}%
\pgfpathlineto{\pgfqpoint{2.952303in}{3.751954in}}%
\pgfpathlineto{\pgfqpoint{2.952303in}{3.756212in}}%
\pgfpathlineto{\pgfqpoint{2.956561in}{3.756212in}}%
\pgfpathlineto{\pgfqpoint{2.956561in}{3.751954in}}%
\pgfpathmoveto{\pgfqpoint{2.952303in}{3.756212in}}%
\pgfpathlineto{\pgfqpoint{2.952303in}{3.756212in}}%
\pgfpathlineto{\pgfqpoint{2.952303in}{3.760470in}}%
\pgfpathlineto{\pgfqpoint{2.956561in}{3.760470in}}%
\pgfpathlineto{\pgfqpoint{2.956561in}{3.756212in}}%
\pgfpathmoveto{\pgfqpoint{2.939530in}{3.760470in}}%
\pgfpathlineto{\pgfqpoint{2.939530in}{3.760470in}}%
\pgfpathlineto{\pgfqpoint{2.939530in}{3.764728in}}%
\pgfpathlineto{\pgfqpoint{2.943788in}{3.764728in}}%
\pgfpathlineto{\pgfqpoint{2.943788in}{3.760470in}}%
\pgfpathmoveto{\pgfqpoint{2.939530in}{3.764728in}}%
\pgfpathlineto{\pgfqpoint{2.939530in}{3.764728in}}%
\pgfpathlineto{\pgfqpoint{2.939530in}{3.768986in}}%
\pgfpathlineto{\pgfqpoint{2.943788in}{3.768986in}}%
\pgfpathlineto{\pgfqpoint{2.943788in}{3.764728in}}%
\pgfpathmoveto{\pgfqpoint{2.943788in}{3.760470in}}%
\pgfpathlineto{\pgfqpoint{2.943788in}{3.760470in}}%
\pgfpathlineto{\pgfqpoint{2.943788in}{3.764728in}}%
\pgfpathlineto{\pgfqpoint{2.948046in}{3.764728in}}%
\pgfpathlineto{\pgfqpoint{2.948046in}{3.760470in}}%
\pgfpathmoveto{\pgfqpoint{2.943788in}{3.764728in}}%
\pgfpathlineto{\pgfqpoint{2.943788in}{3.764728in}}%
\pgfpathlineto{\pgfqpoint{2.943788in}{3.768986in}}%
\pgfpathlineto{\pgfqpoint{2.948046in}{3.768986in}}%
\pgfpathlineto{\pgfqpoint{2.948046in}{3.764728in}}%
\pgfpathmoveto{\pgfqpoint{2.939530in}{3.768986in}}%
\pgfpathlineto{\pgfqpoint{2.939530in}{3.768986in}}%
\pgfpathlineto{\pgfqpoint{2.939530in}{3.773243in}}%
\pgfpathlineto{\pgfqpoint{2.943788in}{3.773243in}}%
\pgfpathlineto{\pgfqpoint{2.943788in}{3.768986in}}%
\pgfpathmoveto{\pgfqpoint{2.943788in}{3.768986in}}%
\pgfpathlineto{\pgfqpoint{2.943788in}{3.768986in}}%
\pgfpathlineto{\pgfqpoint{2.943788in}{3.773243in}}%
\pgfpathlineto{\pgfqpoint{2.948046in}{3.773243in}}%
\pgfpathlineto{\pgfqpoint{2.948046in}{3.768986in}}%
\pgfpathmoveto{\pgfqpoint{2.948046in}{3.760470in}}%
\pgfpathlineto{\pgfqpoint{2.948046in}{3.760470in}}%
\pgfpathlineto{\pgfqpoint{2.948046in}{3.764728in}}%
\pgfpathlineto{\pgfqpoint{2.952303in}{3.764728in}}%
\pgfpathlineto{\pgfqpoint{2.952303in}{3.760470in}}%
\pgfpathmoveto{\pgfqpoint{2.948046in}{3.764728in}}%
\pgfpathlineto{\pgfqpoint{2.948046in}{3.764728in}}%
\pgfpathlineto{\pgfqpoint{2.948046in}{3.768986in}}%
\pgfpathlineto{\pgfqpoint{2.952303in}{3.768986in}}%
\pgfpathlineto{\pgfqpoint{2.952303in}{3.764728in}}%
\pgfpathmoveto{\pgfqpoint{2.952303in}{3.760470in}}%
\pgfpathlineto{\pgfqpoint{2.952303in}{3.760470in}}%
\pgfpathlineto{\pgfqpoint{2.952303in}{3.764728in}}%
\pgfpathlineto{\pgfqpoint{2.956561in}{3.764728in}}%
\pgfpathlineto{\pgfqpoint{2.956561in}{3.760470in}}%
\pgfpathmoveto{\pgfqpoint{2.952303in}{3.764728in}}%
\pgfpathlineto{\pgfqpoint{2.952303in}{3.764728in}}%
\pgfpathlineto{\pgfqpoint{2.952303in}{3.768986in}}%
\pgfpathlineto{\pgfqpoint{2.956561in}{3.768986in}}%
\pgfpathlineto{\pgfqpoint{2.956561in}{3.764728in}}%
\pgfpathmoveto{\pgfqpoint{2.948046in}{3.768986in}}%
\pgfpathlineto{\pgfqpoint{2.948046in}{3.768986in}}%
\pgfpathlineto{\pgfqpoint{2.948046in}{3.773243in}}%
\pgfpathlineto{\pgfqpoint{2.952303in}{3.773243in}}%
\pgfpathlineto{\pgfqpoint{2.952303in}{3.768986in}}%
\pgfpathmoveto{\pgfqpoint{2.952303in}{3.768986in}}%
\pgfpathlineto{\pgfqpoint{2.952303in}{3.768986in}}%
\pgfpathlineto{\pgfqpoint{2.952303in}{3.773243in}}%
\pgfpathlineto{\pgfqpoint{2.956561in}{3.773243in}}%
\pgfpathlineto{\pgfqpoint{2.956561in}{3.768986in}}%
\pgfpathmoveto{\pgfqpoint{3.088556in}{3.568869in}}%
\pgfpathlineto{\pgfqpoint{3.088556in}{3.568869in}}%
\pgfpathlineto{\pgfqpoint{3.088556in}{3.573127in}}%
\pgfpathlineto{\pgfqpoint{3.092814in}{3.573127in}}%
\pgfpathlineto{\pgfqpoint{3.092814in}{3.568869in}}%
\pgfpathmoveto{\pgfqpoint{3.054493in}{3.636993in}}%
\pgfpathlineto{\pgfqpoint{3.054493in}{3.636993in}}%
\pgfpathlineto{\pgfqpoint{3.054493in}{3.641250in}}%
\pgfpathlineto{\pgfqpoint{3.058751in}{3.641250in}}%
\pgfpathlineto{\pgfqpoint{3.058751in}{3.636993in}}%
\pgfpathmoveto{\pgfqpoint{3.080040in}{3.585900in}}%
\pgfpathlineto{\pgfqpoint{3.080040in}{3.585900in}}%
\pgfpathlineto{\pgfqpoint{3.080040in}{3.590158in}}%
\pgfpathlineto{\pgfqpoint{3.084298in}{3.590158in}}%
\pgfpathlineto{\pgfqpoint{3.084298in}{3.585900in}}%
\pgfpathmoveto{\pgfqpoint{3.084298in}{3.577385in}}%
\pgfpathlineto{\pgfqpoint{3.084298in}{3.577385in}}%
\pgfpathlineto{\pgfqpoint{3.084298in}{3.581642in}}%
\pgfpathlineto{\pgfqpoint{3.088556in}{3.581642in}}%
\pgfpathlineto{\pgfqpoint{3.088556in}{3.577385in}}%
\pgfpathmoveto{\pgfqpoint{3.088556in}{3.573127in}}%
\pgfpathlineto{\pgfqpoint{3.088556in}{3.573127in}}%
\pgfpathlineto{\pgfqpoint{3.088556in}{3.577385in}}%
\pgfpathlineto{\pgfqpoint{3.092814in}{3.577385in}}%
\pgfpathlineto{\pgfqpoint{3.092814in}{3.573127in}}%
\pgfpathmoveto{\pgfqpoint{3.088556in}{3.577385in}}%
\pgfpathlineto{\pgfqpoint{3.088556in}{3.577385in}}%
\pgfpathlineto{\pgfqpoint{3.088556in}{3.581642in}}%
\pgfpathlineto{\pgfqpoint{3.092814in}{3.581642in}}%
\pgfpathlineto{\pgfqpoint{3.092814in}{3.577385in}}%
\pgfpathmoveto{\pgfqpoint{3.084298in}{3.581642in}}%
\pgfpathlineto{\pgfqpoint{3.084298in}{3.581642in}}%
\pgfpathlineto{\pgfqpoint{3.084298in}{3.585900in}}%
\pgfpathlineto{\pgfqpoint{3.088556in}{3.585900in}}%
\pgfpathlineto{\pgfqpoint{3.088556in}{3.581642in}}%
\pgfpathmoveto{\pgfqpoint{3.084298in}{3.585900in}}%
\pgfpathlineto{\pgfqpoint{3.084298in}{3.585900in}}%
\pgfpathlineto{\pgfqpoint{3.084298in}{3.590158in}}%
\pgfpathlineto{\pgfqpoint{3.088556in}{3.590158in}}%
\pgfpathlineto{\pgfqpoint{3.088556in}{3.585900in}}%
\pgfpathmoveto{\pgfqpoint{3.088556in}{3.581642in}}%
\pgfpathlineto{\pgfqpoint{3.088556in}{3.581642in}}%
\pgfpathlineto{\pgfqpoint{3.088556in}{3.585900in}}%
\pgfpathlineto{\pgfqpoint{3.092814in}{3.585900in}}%
\pgfpathlineto{\pgfqpoint{3.092814in}{3.581642in}}%
\pgfpathmoveto{\pgfqpoint{3.088556in}{3.585900in}}%
\pgfpathlineto{\pgfqpoint{3.088556in}{3.585900in}}%
\pgfpathlineto{\pgfqpoint{3.088556in}{3.590158in}}%
\pgfpathlineto{\pgfqpoint{3.092814in}{3.590158in}}%
\pgfpathlineto{\pgfqpoint{3.092814in}{3.585900in}}%
\pgfpathmoveto{\pgfqpoint{3.075782in}{3.594415in}}%
\pgfpathlineto{\pgfqpoint{3.075782in}{3.594415in}}%
\pgfpathlineto{\pgfqpoint{3.075782in}{3.598673in}}%
\pgfpathlineto{\pgfqpoint{3.080040in}{3.598673in}}%
\pgfpathlineto{\pgfqpoint{3.080040in}{3.594415in}}%
\pgfpathmoveto{\pgfqpoint{3.080040in}{3.590158in}}%
\pgfpathlineto{\pgfqpoint{3.080040in}{3.590158in}}%
\pgfpathlineto{\pgfqpoint{3.080040in}{3.594415in}}%
\pgfpathlineto{\pgfqpoint{3.084298in}{3.594415in}}%
\pgfpathlineto{\pgfqpoint{3.084298in}{3.590158in}}%
\pgfpathmoveto{\pgfqpoint{3.080040in}{3.594415in}}%
\pgfpathlineto{\pgfqpoint{3.080040in}{3.594415in}}%
\pgfpathlineto{\pgfqpoint{3.080040in}{3.598673in}}%
\pgfpathlineto{\pgfqpoint{3.084298in}{3.598673in}}%
\pgfpathlineto{\pgfqpoint{3.084298in}{3.594415in}}%
\pgfpathmoveto{\pgfqpoint{3.075782in}{3.598673in}}%
\pgfpathlineto{\pgfqpoint{3.075782in}{3.598673in}}%
\pgfpathlineto{\pgfqpoint{3.075782in}{3.602931in}}%
\pgfpathlineto{\pgfqpoint{3.080040in}{3.602931in}}%
\pgfpathlineto{\pgfqpoint{3.080040in}{3.598673in}}%
\pgfpathmoveto{\pgfqpoint{3.075782in}{3.602931in}}%
\pgfpathlineto{\pgfqpoint{3.075782in}{3.602931in}}%
\pgfpathlineto{\pgfqpoint{3.075782in}{3.607189in}}%
\pgfpathlineto{\pgfqpoint{3.080040in}{3.607189in}}%
\pgfpathlineto{\pgfqpoint{3.080040in}{3.602931in}}%
\pgfpathmoveto{\pgfqpoint{3.080040in}{3.598673in}}%
\pgfpathlineto{\pgfqpoint{3.080040in}{3.598673in}}%
\pgfpathlineto{\pgfqpoint{3.080040in}{3.602931in}}%
\pgfpathlineto{\pgfqpoint{3.084298in}{3.602931in}}%
\pgfpathlineto{\pgfqpoint{3.084298in}{3.598673in}}%
\pgfpathmoveto{\pgfqpoint{3.080040in}{3.602931in}}%
\pgfpathlineto{\pgfqpoint{3.080040in}{3.602931in}}%
\pgfpathlineto{\pgfqpoint{3.080040in}{3.607189in}}%
\pgfpathlineto{\pgfqpoint{3.084298in}{3.607189in}}%
\pgfpathlineto{\pgfqpoint{3.084298in}{3.602931in}}%
\pgfpathmoveto{\pgfqpoint{3.084298in}{3.590158in}}%
\pgfpathlineto{\pgfqpoint{3.084298in}{3.590158in}}%
\pgfpathlineto{\pgfqpoint{3.084298in}{3.594415in}}%
\pgfpathlineto{\pgfqpoint{3.088556in}{3.594415in}}%
\pgfpathlineto{\pgfqpoint{3.088556in}{3.590158in}}%
\pgfpathmoveto{\pgfqpoint{3.084298in}{3.594415in}}%
\pgfpathlineto{\pgfqpoint{3.084298in}{3.594415in}}%
\pgfpathlineto{\pgfqpoint{3.084298in}{3.598673in}}%
\pgfpathlineto{\pgfqpoint{3.088556in}{3.598673in}}%
\pgfpathlineto{\pgfqpoint{3.088556in}{3.594415in}}%
\pgfpathmoveto{\pgfqpoint{3.088556in}{3.590158in}}%
\pgfpathlineto{\pgfqpoint{3.088556in}{3.590158in}}%
\pgfpathlineto{\pgfqpoint{3.088556in}{3.594415in}}%
\pgfpathlineto{\pgfqpoint{3.092814in}{3.594415in}}%
\pgfpathlineto{\pgfqpoint{3.092814in}{3.590158in}}%
\pgfpathmoveto{\pgfqpoint{3.088556in}{3.594415in}}%
\pgfpathlineto{\pgfqpoint{3.088556in}{3.594415in}}%
\pgfpathlineto{\pgfqpoint{3.088556in}{3.598673in}}%
\pgfpathlineto{\pgfqpoint{3.092814in}{3.598673in}}%
\pgfpathlineto{\pgfqpoint{3.092814in}{3.594415in}}%
\pgfpathmoveto{\pgfqpoint{3.084298in}{3.598673in}}%
\pgfpathlineto{\pgfqpoint{3.084298in}{3.598673in}}%
\pgfpathlineto{\pgfqpoint{3.084298in}{3.602931in}}%
\pgfpathlineto{\pgfqpoint{3.088556in}{3.602931in}}%
\pgfpathlineto{\pgfqpoint{3.088556in}{3.598673in}}%
\pgfpathmoveto{\pgfqpoint{3.084298in}{3.602931in}}%
\pgfpathlineto{\pgfqpoint{3.084298in}{3.602931in}}%
\pgfpathlineto{\pgfqpoint{3.084298in}{3.607189in}}%
\pgfpathlineto{\pgfqpoint{3.088556in}{3.607189in}}%
\pgfpathlineto{\pgfqpoint{3.088556in}{3.602931in}}%
\pgfpathmoveto{\pgfqpoint{3.088556in}{3.598673in}}%
\pgfpathlineto{\pgfqpoint{3.088556in}{3.598673in}}%
\pgfpathlineto{\pgfqpoint{3.088556in}{3.602931in}}%
\pgfpathlineto{\pgfqpoint{3.092814in}{3.602931in}}%
\pgfpathlineto{\pgfqpoint{3.092814in}{3.598673in}}%
\pgfpathmoveto{\pgfqpoint{3.071524in}{3.607189in}}%
\pgfpathlineto{\pgfqpoint{3.071524in}{3.607189in}}%
\pgfpathlineto{\pgfqpoint{3.071524in}{3.611446in}}%
\pgfpathlineto{\pgfqpoint{3.075782in}{3.611446in}}%
\pgfpathlineto{\pgfqpoint{3.075782in}{3.607189in}}%
\pgfpathmoveto{\pgfqpoint{3.071524in}{3.611446in}}%
\pgfpathlineto{\pgfqpoint{3.071524in}{3.611446in}}%
\pgfpathlineto{\pgfqpoint{3.071524in}{3.615704in}}%
\pgfpathlineto{\pgfqpoint{3.075782in}{3.615704in}}%
\pgfpathlineto{\pgfqpoint{3.075782in}{3.611446in}}%
\pgfpathmoveto{\pgfqpoint{3.067267in}{3.615704in}}%
\pgfpathlineto{\pgfqpoint{3.067267in}{3.615704in}}%
\pgfpathlineto{\pgfqpoint{3.067267in}{3.619962in}}%
\pgfpathlineto{\pgfqpoint{3.071524in}{3.619962in}}%
\pgfpathlineto{\pgfqpoint{3.071524in}{3.615704in}}%
\pgfpathmoveto{\pgfqpoint{3.067267in}{3.619962in}}%
\pgfpathlineto{\pgfqpoint{3.067267in}{3.619962in}}%
\pgfpathlineto{\pgfqpoint{3.067267in}{3.624220in}}%
\pgfpathlineto{\pgfqpoint{3.071524in}{3.624220in}}%
\pgfpathlineto{\pgfqpoint{3.071524in}{3.619962in}}%
\pgfpathmoveto{\pgfqpoint{3.071524in}{3.615704in}}%
\pgfpathlineto{\pgfqpoint{3.071524in}{3.615704in}}%
\pgfpathlineto{\pgfqpoint{3.071524in}{3.619962in}}%
\pgfpathlineto{\pgfqpoint{3.075782in}{3.619962in}}%
\pgfpathlineto{\pgfqpoint{3.075782in}{3.615704in}}%
\pgfpathmoveto{\pgfqpoint{3.071524in}{3.619962in}}%
\pgfpathlineto{\pgfqpoint{3.071524in}{3.619962in}}%
\pgfpathlineto{\pgfqpoint{3.071524in}{3.624220in}}%
\pgfpathlineto{\pgfqpoint{3.075782in}{3.624220in}}%
\pgfpathlineto{\pgfqpoint{3.075782in}{3.619962in}}%
\pgfpathmoveto{\pgfqpoint{3.063009in}{3.624220in}}%
\pgfpathlineto{\pgfqpoint{3.063009in}{3.624220in}}%
\pgfpathlineto{\pgfqpoint{3.063009in}{3.628477in}}%
\pgfpathlineto{\pgfqpoint{3.067267in}{3.628477in}}%
\pgfpathlineto{\pgfqpoint{3.067267in}{3.624220in}}%
\pgfpathmoveto{\pgfqpoint{3.063009in}{3.628477in}}%
\pgfpathlineto{\pgfqpoint{3.063009in}{3.628477in}}%
\pgfpathlineto{\pgfqpoint{3.063009in}{3.632735in}}%
\pgfpathlineto{\pgfqpoint{3.067267in}{3.632735in}}%
\pgfpathlineto{\pgfqpoint{3.067267in}{3.628477in}}%
\pgfpathmoveto{\pgfqpoint{3.058751in}{3.632735in}}%
\pgfpathlineto{\pgfqpoint{3.058751in}{3.632735in}}%
\pgfpathlineto{\pgfqpoint{3.058751in}{3.636993in}}%
\pgfpathlineto{\pgfqpoint{3.063009in}{3.636993in}}%
\pgfpathlineto{\pgfqpoint{3.063009in}{3.632735in}}%
\pgfpathmoveto{\pgfqpoint{3.058751in}{3.636993in}}%
\pgfpathlineto{\pgfqpoint{3.058751in}{3.636993in}}%
\pgfpathlineto{\pgfqpoint{3.058751in}{3.641250in}}%
\pgfpathlineto{\pgfqpoint{3.063009in}{3.641250in}}%
\pgfpathlineto{\pgfqpoint{3.063009in}{3.636993in}}%
\pgfpathmoveto{\pgfqpoint{3.063009in}{3.632735in}}%
\pgfpathlineto{\pgfqpoint{3.063009in}{3.632735in}}%
\pgfpathlineto{\pgfqpoint{3.063009in}{3.636993in}}%
\pgfpathlineto{\pgfqpoint{3.067267in}{3.636993in}}%
\pgfpathlineto{\pgfqpoint{3.067267in}{3.632735in}}%
\pgfpathmoveto{\pgfqpoint{3.063009in}{3.636993in}}%
\pgfpathlineto{\pgfqpoint{3.063009in}{3.636993in}}%
\pgfpathlineto{\pgfqpoint{3.063009in}{3.641250in}}%
\pgfpathlineto{\pgfqpoint{3.067267in}{3.641250in}}%
\pgfpathlineto{\pgfqpoint{3.067267in}{3.636993in}}%
\pgfpathmoveto{\pgfqpoint{3.067267in}{3.624220in}}%
\pgfpathlineto{\pgfqpoint{3.067267in}{3.624220in}}%
\pgfpathlineto{\pgfqpoint{3.067267in}{3.628477in}}%
\pgfpathlineto{\pgfqpoint{3.071524in}{3.628477in}}%
\pgfpathlineto{\pgfqpoint{3.071524in}{3.624220in}}%
\pgfpathmoveto{\pgfqpoint{3.067267in}{3.628477in}}%
\pgfpathlineto{\pgfqpoint{3.067267in}{3.628477in}}%
\pgfpathlineto{\pgfqpoint{3.067267in}{3.632735in}}%
\pgfpathlineto{\pgfqpoint{3.071524in}{3.632735in}}%
\pgfpathlineto{\pgfqpoint{3.071524in}{3.628477in}}%
\pgfpathmoveto{\pgfqpoint{3.071524in}{3.624220in}}%
\pgfpathlineto{\pgfqpoint{3.071524in}{3.624220in}}%
\pgfpathlineto{\pgfqpoint{3.071524in}{3.628477in}}%
\pgfpathlineto{\pgfqpoint{3.075782in}{3.628477in}}%
\pgfpathlineto{\pgfqpoint{3.075782in}{3.624220in}}%
\pgfpathmoveto{\pgfqpoint{3.071524in}{3.628477in}}%
\pgfpathlineto{\pgfqpoint{3.071524in}{3.628477in}}%
\pgfpathlineto{\pgfqpoint{3.071524in}{3.632735in}}%
\pgfpathlineto{\pgfqpoint{3.075782in}{3.632735in}}%
\pgfpathlineto{\pgfqpoint{3.075782in}{3.628477in}}%
\pgfpathmoveto{\pgfqpoint{3.067267in}{3.632735in}}%
\pgfpathlineto{\pgfqpoint{3.067267in}{3.632735in}}%
\pgfpathlineto{\pgfqpoint{3.067267in}{3.636993in}}%
\pgfpathlineto{\pgfqpoint{3.071524in}{3.636993in}}%
\pgfpathlineto{\pgfqpoint{3.071524in}{3.632735in}}%
\pgfpathmoveto{\pgfqpoint{3.067267in}{3.636993in}}%
\pgfpathlineto{\pgfqpoint{3.067267in}{3.636993in}}%
\pgfpathlineto{\pgfqpoint{3.067267in}{3.641250in}}%
\pgfpathlineto{\pgfqpoint{3.071524in}{3.641250in}}%
\pgfpathlineto{\pgfqpoint{3.071524in}{3.636993in}}%
\pgfpathmoveto{\pgfqpoint{3.071524in}{3.632735in}}%
\pgfpathlineto{\pgfqpoint{3.071524in}{3.632735in}}%
\pgfpathlineto{\pgfqpoint{3.071524in}{3.636993in}}%
\pgfpathlineto{\pgfqpoint{3.075782in}{3.636993in}}%
\pgfpathlineto{\pgfqpoint{3.075782in}{3.632735in}}%
\pgfpathmoveto{\pgfqpoint{3.071524in}{3.636993in}}%
\pgfpathlineto{\pgfqpoint{3.071524in}{3.636993in}}%
\pgfpathlineto{\pgfqpoint{3.071524in}{3.641250in}}%
\pgfpathlineto{\pgfqpoint{3.075782in}{3.641250in}}%
\pgfpathlineto{\pgfqpoint{3.075782in}{3.636993in}}%
\pgfpathmoveto{\pgfqpoint{3.075782in}{3.607189in}}%
\pgfpathlineto{\pgfqpoint{3.075782in}{3.607189in}}%
\pgfpathlineto{\pgfqpoint{3.075782in}{3.611446in}}%
\pgfpathlineto{\pgfqpoint{3.080040in}{3.611446in}}%
\pgfpathlineto{\pgfqpoint{3.080040in}{3.607189in}}%
\pgfpathmoveto{\pgfqpoint{3.075782in}{3.611446in}}%
\pgfpathlineto{\pgfqpoint{3.075782in}{3.611446in}}%
\pgfpathlineto{\pgfqpoint{3.075782in}{3.615704in}}%
\pgfpathlineto{\pgfqpoint{3.080040in}{3.615704in}}%
\pgfpathlineto{\pgfqpoint{3.080040in}{3.611446in}}%
\pgfpathmoveto{\pgfqpoint{3.080040in}{3.607189in}}%
\pgfpathlineto{\pgfqpoint{3.080040in}{3.607189in}}%
\pgfpathlineto{\pgfqpoint{3.080040in}{3.611446in}}%
\pgfpathlineto{\pgfqpoint{3.084298in}{3.611446in}}%
\pgfpathlineto{\pgfqpoint{3.084298in}{3.607189in}}%
\pgfpathmoveto{\pgfqpoint{3.080040in}{3.611446in}}%
\pgfpathlineto{\pgfqpoint{3.080040in}{3.611446in}}%
\pgfpathlineto{\pgfqpoint{3.080040in}{3.615704in}}%
\pgfpathlineto{\pgfqpoint{3.084298in}{3.615704in}}%
\pgfpathlineto{\pgfqpoint{3.084298in}{3.611446in}}%
\pgfpathmoveto{\pgfqpoint{3.075782in}{3.615704in}}%
\pgfpathlineto{\pgfqpoint{3.075782in}{3.615704in}}%
\pgfpathlineto{\pgfqpoint{3.075782in}{3.619962in}}%
\pgfpathlineto{\pgfqpoint{3.080040in}{3.619962in}}%
\pgfpathlineto{\pgfqpoint{3.080040in}{3.615704in}}%
\pgfpathmoveto{\pgfqpoint{3.075782in}{3.619962in}}%
\pgfpathlineto{\pgfqpoint{3.075782in}{3.619962in}}%
\pgfpathlineto{\pgfqpoint{3.075782in}{3.624220in}}%
\pgfpathlineto{\pgfqpoint{3.080040in}{3.624220in}}%
\pgfpathlineto{\pgfqpoint{3.080040in}{3.619962in}}%
\pgfpathmoveto{\pgfqpoint{3.080040in}{3.615704in}}%
\pgfpathlineto{\pgfqpoint{3.080040in}{3.615704in}}%
\pgfpathlineto{\pgfqpoint{3.080040in}{3.619962in}}%
\pgfpathlineto{\pgfqpoint{3.084298in}{3.619962in}}%
\pgfpathlineto{\pgfqpoint{3.084298in}{3.615704in}}%
\pgfpathmoveto{\pgfqpoint{3.080040in}{3.619962in}}%
\pgfpathlineto{\pgfqpoint{3.080040in}{3.619962in}}%
\pgfpathlineto{\pgfqpoint{3.080040in}{3.624220in}}%
\pgfpathlineto{\pgfqpoint{3.084298in}{3.624220in}}%
\pgfpathlineto{\pgfqpoint{3.084298in}{3.619962in}}%
\pgfpathmoveto{\pgfqpoint{3.084298in}{3.607189in}}%
\pgfpathlineto{\pgfqpoint{3.084298in}{3.607189in}}%
\pgfpathlineto{\pgfqpoint{3.084298in}{3.611446in}}%
\pgfpathlineto{\pgfqpoint{3.088556in}{3.611446in}}%
\pgfpathlineto{\pgfqpoint{3.088556in}{3.607189in}}%
\pgfpathmoveto{\pgfqpoint{3.084298in}{3.611446in}}%
\pgfpathlineto{\pgfqpoint{3.084298in}{3.611446in}}%
\pgfpathlineto{\pgfqpoint{3.084298in}{3.615704in}}%
\pgfpathlineto{\pgfqpoint{3.088556in}{3.615704in}}%
\pgfpathlineto{\pgfqpoint{3.088556in}{3.611446in}}%
\pgfpathmoveto{\pgfqpoint{3.075782in}{3.624220in}}%
\pgfpathlineto{\pgfqpoint{3.075782in}{3.624220in}}%
\pgfpathlineto{\pgfqpoint{3.075782in}{3.628477in}}%
\pgfpathlineto{\pgfqpoint{3.080040in}{3.628477in}}%
\pgfpathlineto{\pgfqpoint{3.080040in}{3.624220in}}%
\pgfpathmoveto{\pgfqpoint{3.075782in}{3.628477in}}%
\pgfpathlineto{\pgfqpoint{3.075782in}{3.628477in}}%
\pgfpathlineto{\pgfqpoint{3.075782in}{3.632735in}}%
\pgfpathlineto{\pgfqpoint{3.080040in}{3.632735in}}%
\pgfpathlineto{\pgfqpoint{3.080040in}{3.628477in}}%
\pgfpathmoveto{\pgfqpoint{3.003398in}{3.705118in}}%
\pgfpathlineto{\pgfqpoint{3.003398in}{3.705118in}}%
\pgfpathlineto{\pgfqpoint{3.003398in}{3.709376in}}%
\pgfpathlineto{\pgfqpoint{3.007656in}{3.709376in}}%
\pgfpathlineto{\pgfqpoint{3.007656in}{3.705118in}}%
\pgfpathmoveto{\pgfqpoint{3.020430in}{3.688087in}}%
\pgfpathlineto{\pgfqpoint{3.020430in}{3.688087in}}%
\pgfpathlineto{\pgfqpoint{3.020430in}{3.692345in}}%
\pgfpathlineto{\pgfqpoint{3.024688in}{3.692345in}}%
\pgfpathlineto{\pgfqpoint{3.024688in}{3.688087in}}%
\pgfpathmoveto{\pgfqpoint{3.011914in}{3.696602in}}%
\pgfpathlineto{\pgfqpoint{3.011914in}{3.696602in}}%
\pgfpathlineto{\pgfqpoint{3.011914in}{3.700860in}}%
\pgfpathlineto{\pgfqpoint{3.016172in}{3.700860in}}%
\pgfpathlineto{\pgfqpoint{3.016172in}{3.696602in}}%
\pgfpathmoveto{\pgfqpoint{3.007656in}{3.700860in}}%
\pgfpathlineto{\pgfqpoint{3.007656in}{3.700860in}}%
\pgfpathlineto{\pgfqpoint{3.007656in}{3.705118in}}%
\pgfpathlineto{\pgfqpoint{3.011914in}{3.705118in}}%
\pgfpathlineto{\pgfqpoint{3.011914in}{3.700860in}}%
\pgfpathmoveto{\pgfqpoint{3.007656in}{3.705118in}}%
\pgfpathlineto{\pgfqpoint{3.007656in}{3.705118in}}%
\pgfpathlineto{\pgfqpoint{3.007656in}{3.709376in}}%
\pgfpathlineto{\pgfqpoint{3.011914in}{3.709376in}}%
\pgfpathlineto{\pgfqpoint{3.011914in}{3.705118in}}%
\pgfpathmoveto{\pgfqpoint{3.011914in}{3.700860in}}%
\pgfpathlineto{\pgfqpoint{3.011914in}{3.700860in}}%
\pgfpathlineto{\pgfqpoint{3.011914in}{3.705118in}}%
\pgfpathlineto{\pgfqpoint{3.016172in}{3.705118in}}%
\pgfpathlineto{\pgfqpoint{3.016172in}{3.700860in}}%
\pgfpathmoveto{\pgfqpoint{3.011914in}{3.705118in}}%
\pgfpathlineto{\pgfqpoint{3.011914in}{3.705118in}}%
\pgfpathlineto{\pgfqpoint{3.011914in}{3.709376in}}%
\pgfpathlineto{\pgfqpoint{3.016172in}{3.709376in}}%
\pgfpathlineto{\pgfqpoint{3.016172in}{3.705118in}}%
\pgfpathmoveto{\pgfqpoint{3.016172in}{3.692345in}}%
\pgfpathlineto{\pgfqpoint{3.016172in}{3.692345in}}%
\pgfpathlineto{\pgfqpoint{3.016172in}{3.696602in}}%
\pgfpathlineto{\pgfqpoint{3.020430in}{3.696602in}}%
\pgfpathlineto{\pgfqpoint{3.020430in}{3.692345in}}%
\pgfpathmoveto{\pgfqpoint{3.016172in}{3.696602in}}%
\pgfpathlineto{\pgfqpoint{3.016172in}{3.696602in}}%
\pgfpathlineto{\pgfqpoint{3.016172in}{3.700860in}}%
\pgfpathlineto{\pgfqpoint{3.020430in}{3.700860in}}%
\pgfpathlineto{\pgfqpoint{3.020430in}{3.696602in}}%
\pgfpathmoveto{\pgfqpoint{3.020430in}{3.692345in}}%
\pgfpathlineto{\pgfqpoint{3.020430in}{3.692345in}}%
\pgfpathlineto{\pgfqpoint{3.020430in}{3.696602in}}%
\pgfpathlineto{\pgfqpoint{3.024688in}{3.696602in}}%
\pgfpathlineto{\pgfqpoint{3.024688in}{3.692345in}}%
\pgfpathmoveto{\pgfqpoint{3.020430in}{3.696602in}}%
\pgfpathlineto{\pgfqpoint{3.020430in}{3.696602in}}%
\pgfpathlineto{\pgfqpoint{3.020430in}{3.700860in}}%
\pgfpathlineto{\pgfqpoint{3.024688in}{3.700860in}}%
\pgfpathlineto{\pgfqpoint{3.024688in}{3.696602in}}%
\pgfpathmoveto{\pgfqpoint{3.016172in}{3.700860in}}%
\pgfpathlineto{\pgfqpoint{3.016172in}{3.700860in}}%
\pgfpathlineto{\pgfqpoint{3.016172in}{3.705118in}}%
\pgfpathlineto{\pgfqpoint{3.020430in}{3.705118in}}%
\pgfpathlineto{\pgfqpoint{3.020430in}{3.700860in}}%
\pgfpathmoveto{\pgfqpoint{3.016172in}{3.705118in}}%
\pgfpathlineto{\pgfqpoint{3.016172in}{3.705118in}}%
\pgfpathlineto{\pgfqpoint{3.016172in}{3.709376in}}%
\pgfpathlineto{\pgfqpoint{3.020430in}{3.709376in}}%
\pgfpathlineto{\pgfqpoint{3.020430in}{3.705118in}}%
\pgfpathmoveto{\pgfqpoint{3.020430in}{3.700860in}}%
\pgfpathlineto{\pgfqpoint{3.020430in}{3.700860in}}%
\pgfpathlineto{\pgfqpoint{3.020430in}{3.705118in}}%
\pgfpathlineto{\pgfqpoint{3.024688in}{3.705118in}}%
\pgfpathlineto{\pgfqpoint{3.024688in}{3.700860in}}%
\pgfpathmoveto{\pgfqpoint{3.020430in}{3.705118in}}%
\pgfpathlineto{\pgfqpoint{3.020430in}{3.705118in}}%
\pgfpathlineto{\pgfqpoint{3.020430in}{3.709376in}}%
\pgfpathlineto{\pgfqpoint{3.024688in}{3.709376in}}%
\pgfpathlineto{\pgfqpoint{3.024688in}{3.705118in}}%
\pgfpathmoveto{\pgfqpoint{2.956561in}{3.726407in}}%
\pgfpathlineto{\pgfqpoint{2.956561in}{3.726407in}}%
\pgfpathlineto{\pgfqpoint{2.956561in}{3.730665in}}%
\pgfpathlineto{\pgfqpoint{2.960819in}{3.730665in}}%
\pgfpathlineto{\pgfqpoint{2.960819in}{3.726407in}}%
\pgfpathmoveto{\pgfqpoint{2.956561in}{3.730665in}}%
\pgfpathlineto{\pgfqpoint{2.956561in}{3.730665in}}%
\pgfpathlineto{\pgfqpoint{2.956561in}{3.734923in}}%
\pgfpathlineto{\pgfqpoint{2.960819in}{3.734923in}}%
\pgfpathlineto{\pgfqpoint{2.960819in}{3.730665in}}%
\pgfpathmoveto{\pgfqpoint{2.960819in}{3.726407in}}%
\pgfpathlineto{\pgfqpoint{2.960819in}{3.726407in}}%
\pgfpathlineto{\pgfqpoint{2.960819in}{3.730665in}}%
\pgfpathlineto{\pgfqpoint{2.965077in}{3.730665in}}%
\pgfpathlineto{\pgfqpoint{2.965077in}{3.726407in}}%
\pgfpathmoveto{\pgfqpoint{2.960819in}{3.730665in}}%
\pgfpathlineto{\pgfqpoint{2.960819in}{3.730665in}}%
\pgfpathlineto{\pgfqpoint{2.960819in}{3.734923in}}%
\pgfpathlineto{\pgfqpoint{2.965077in}{3.734923in}}%
\pgfpathlineto{\pgfqpoint{2.965077in}{3.730665in}}%
\pgfpathmoveto{\pgfqpoint{2.956561in}{3.734923in}}%
\pgfpathlineto{\pgfqpoint{2.956561in}{3.734923in}}%
\pgfpathlineto{\pgfqpoint{2.956561in}{3.739181in}}%
\pgfpathlineto{\pgfqpoint{2.960819in}{3.739181in}}%
\pgfpathlineto{\pgfqpoint{2.960819in}{3.734923in}}%
\pgfpathmoveto{\pgfqpoint{2.956561in}{3.739181in}}%
\pgfpathlineto{\pgfqpoint{2.956561in}{3.739181in}}%
\pgfpathlineto{\pgfqpoint{2.956561in}{3.743439in}}%
\pgfpathlineto{\pgfqpoint{2.960819in}{3.743439in}}%
\pgfpathlineto{\pgfqpoint{2.960819in}{3.739181in}}%
\pgfpathmoveto{\pgfqpoint{2.960819in}{3.734923in}}%
\pgfpathlineto{\pgfqpoint{2.960819in}{3.734923in}}%
\pgfpathlineto{\pgfqpoint{2.960819in}{3.739181in}}%
\pgfpathlineto{\pgfqpoint{2.965077in}{3.739181in}}%
\pgfpathlineto{\pgfqpoint{2.965077in}{3.734923in}}%
\pgfpathmoveto{\pgfqpoint{2.960819in}{3.739181in}}%
\pgfpathlineto{\pgfqpoint{2.960819in}{3.739181in}}%
\pgfpathlineto{\pgfqpoint{2.960819in}{3.743439in}}%
\pgfpathlineto{\pgfqpoint{2.965077in}{3.743439in}}%
\pgfpathlineto{\pgfqpoint{2.965077in}{3.739181in}}%
\pgfpathmoveto{\pgfqpoint{2.965077in}{3.726407in}}%
\pgfpathlineto{\pgfqpoint{2.965077in}{3.726407in}}%
\pgfpathlineto{\pgfqpoint{2.965077in}{3.730665in}}%
\pgfpathlineto{\pgfqpoint{2.969335in}{3.730665in}}%
\pgfpathlineto{\pgfqpoint{2.969335in}{3.726407in}}%
\pgfpathmoveto{\pgfqpoint{2.965077in}{3.730665in}}%
\pgfpathlineto{\pgfqpoint{2.965077in}{3.730665in}}%
\pgfpathlineto{\pgfqpoint{2.965077in}{3.734923in}}%
\pgfpathlineto{\pgfqpoint{2.969335in}{3.734923in}}%
\pgfpathlineto{\pgfqpoint{2.969335in}{3.730665in}}%
\pgfpathmoveto{\pgfqpoint{2.969335in}{3.726407in}}%
\pgfpathlineto{\pgfqpoint{2.969335in}{3.726407in}}%
\pgfpathlineto{\pgfqpoint{2.969335in}{3.730665in}}%
\pgfpathlineto{\pgfqpoint{2.973593in}{3.730665in}}%
\pgfpathlineto{\pgfqpoint{2.973593in}{3.726407in}}%
\pgfpathmoveto{\pgfqpoint{2.969335in}{3.730665in}}%
\pgfpathlineto{\pgfqpoint{2.969335in}{3.730665in}}%
\pgfpathlineto{\pgfqpoint{2.969335in}{3.734923in}}%
\pgfpathlineto{\pgfqpoint{2.973593in}{3.734923in}}%
\pgfpathlineto{\pgfqpoint{2.973593in}{3.730665in}}%
\pgfpathmoveto{\pgfqpoint{2.965077in}{3.734923in}}%
\pgfpathlineto{\pgfqpoint{2.965077in}{3.734923in}}%
\pgfpathlineto{\pgfqpoint{2.965077in}{3.739181in}}%
\pgfpathlineto{\pgfqpoint{2.969335in}{3.739181in}}%
\pgfpathlineto{\pgfqpoint{2.969335in}{3.734923in}}%
\pgfpathmoveto{\pgfqpoint{2.965077in}{3.739181in}}%
\pgfpathlineto{\pgfqpoint{2.965077in}{3.739181in}}%
\pgfpathlineto{\pgfqpoint{2.965077in}{3.743439in}}%
\pgfpathlineto{\pgfqpoint{2.969335in}{3.743439in}}%
\pgfpathlineto{\pgfqpoint{2.969335in}{3.739181in}}%
\pgfpathmoveto{\pgfqpoint{2.969335in}{3.734923in}}%
\pgfpathlineto{\pgfqpoint{2.969335in}{3.734923in}}%
\pgfpathlineto{\pgfqpoint{2.969335in}{3.739181in}}%
\pgfpathlineto{\pgfqpoint{2.973593in}{3.739181in}}%
\pgfpathlineto{\pgfqpoint{2.973593in}{3.734923in}}%
\pgfpathmoveto{\pgfqpoint{2.969335in}{3.739181in}}%
\pgfpathlineto{\pgfqpoint{2.969335in}{3.739181in}}%
\pgfpathlineto{\pgfqpoint{2.969335in}{3.743439in}}%
\pgfpathlineto{\pgfqpoint{2.973593in}{3.743439in}}%
\pgfpathlineto{\pgfqpoint{2.973593in}{3.739181in}}%
\pgfpathmoveto{\pgfqpoint{2.973593in}{3.722149in}}%
\pgfpathlineto{\pgfqpoint{2.973593in}{3.722149in}}%
\pgfpathlineto{\pgfqpoint{2.973593in}{3.726407in}}%
\pgfpathlineto{\pgfqpoint{2.977851in}{3.726407in}}%
\pgfpathlineto{\pgfqpoint{2.977851in}{3.722149in}}%
\pgfpathmoveto{\pgfqpoint{2.977851in}{3.722149in}}%
\pgfpathlineto{\pgfqpoint{2.977851in}{3.722149in}}%
\pgfpathlineto{\pgfqpoint{2.977851in}{3.726407in}}%
\pgfpathlineto{\pgfqpoint{2.982109in}{3.726407in}}%
\pgfpathlineto{\pgfqpoint{2.982109in}{3.722149in}}%
\pgfpathmoveto{\pgfqpoint{2.982109in}{3.722149in}}%
\pgfpathlineto{\pgfqpoint{2.982109in}{3.722149in}}%
\pgfpathlineto{\pgfqpoint{2.982109in}{3.726407in}}%
\pgfpathlineto{\pgfqpoint{2.986367in}{3.726407in}}%
\pgfpathlineto{\pgfqpoint{2.986367in}{3.722149in}}%
\pgfpathmoveto{\pgfqpoint{2.986367in}{3.717892in}}%
\pgfpathlineto{\pgfqpoint{2.986367in}{3.717892in}}%
\pgfpathlineto{\pgfqpoint{2.986367in}{3.722149in}}%
\pgfpathlineto{\pgfqpoint{2.990624in}{3.722149in}}%
\pgfpathlineto{\pgfqpoint{2.990624in}{3.717892in}}%
\pgfpathmoveto{\pgfqpoint{2.986367in}{3.722149in}}%
\pgfpathlineto{\pgfqpoint{2.986367in}{3.722149in}}%
\pgfpathlineto{\pgfqpoint{2.986367in}{3.726407in}}%
\pgfpathlineto{\pgfqpoint{2.990624in}{3.726407in}}%
\pgfpathlineto{\pgfqpoint{2.990624in}{3.722149in}}%
\pgfpathmoveto{\pgfqpoint{2.973593in}{3.726407in}}%
\pgfpathlineto{\pgfqpoint{2.973593in}{3.726407in}}%
\pgfpathlineto{\pgfqpoint{2.973593in}{3.730665in}}%
\pgfpathlineto{\pgfqpoint{2.977851in}{3.730665in}}%
\pgfpathlineto{\pgfqpoint{2.977851in}{3.726407in}}%
\pgfpathmoveto{\pgfqpoint{2.973593in}{3.730665in}}%
\pgfpathlineto{\pgfqpoint{2.973593in}{3.730665in}}%
\pgfpathlineto{\pgfqpoint{2.973593in}{3.734923in}}%
\pgfpathlineto{\pgfqpoint{2.977851in}{3.734923in}}%
\pgfpathlineto{\pgfqpoint{2.977851in}{3.730665in}}%
\pgfpathmoveto{\pgfqpoint{2.977851in}{3.726407in}}%
\pgfpathlineto{\pgfqpoint{2.977851in}{3.726407in}}%
\pgfpathlineto{\pgfqpoint{2.977851in}{3.730665in}}%
\pgfpathlineto{\pgfqpoint{2.982109in}{3.730665in}}%
\pgfpathlineto{\pgfqpoint{2.982109in}{3.726407in}}%
\pgfpathmoveto{\pgfqpoint{2.977851in}{3.730665in}}%
\pgfpathlineto{\pgfqpoint{2.977851in}{3.730665in}}%
\pgfpathlineto{\pgfqpoint{2.977851in}{3.734923in}}%
\pgfpathlineto{\pgfqpoint{2.982109in}{3.734923in}}%
\pgfpathlineto{\pgfqpoint{2.982109in}{3.730665in}}%
\pgfpathmoveto{\pgfqpoint{2.973593in}{3.734923in}}%
\pgfpathlineto{\pgfqpoint{2.973593in}{3.734923in}}%
\pgfpathlineto{\pgfqpoint{2.973593in}{3.739181in}}%
\pgfpathlineto{\pgfqpoint{2.977851in}{3.739181in}}%
\pgfpathlineto{\pgfqpoint{2.977851in}{3.734923in}}%
\pgfpathmoveto{\pgfqpoint{2.973593in}{3.739181in}}%
\pgfpathlineto{\pgfqpoint{2.973593in}{3.739181in}}%
\pgfpathlineto{\pgfqpoint{2.973593in}{3.743439in}}%
\pgfpathlineto{\pgfqpoint{2.977851in}{3.743439in}}%
\pgfpathlineto{\pgfqpoint{2.977851in}{3.739181in}}%
\pgfpathmoveto{\pgfqpoint{2.977851in}{3.734923in}}%
\pgfpathlineto{\pgfqpoint{2.977851in}{3.734923in}}%
\pgfpathlineto{\pgfqpoint{2.977851in}{3.739181in}}%
\pgfpathlineto{\pgfqpoint{2.982109in}{3.739181in}}%
\pgfpathlineto{\pgfqpoint{2.982109in}{3.734923in}}%
\pgfpathmoveto{\pgfqpoint{2.977851in}{3.739181in}}%
\pgfpathlineto{\pgfqpoint{2.977851in}{3.739181in}}%
\pgfpathlineto{\pgfqpoint{2.977851in}{3.743439in}}%
\pgfpathlineto{\pgfqpoint{2.982109in}{3.743439in}}%
\pgfpathlineto{\pgfqpoint{2.982109in}{3.739181in}}%
\pgfpathmoveto{\pgfqpoint{2.982109in}{3.726407in}}%
\pgfpathlineto{\pgfqpoint{2.982109in}{3.726407in}}%
\pgfpathlineto{\pgfqpoint{2.982109in}{3.730665in}}%
\pgfpathlineto{\pgfqpoint{2.986367in}{3.730665in}}%
\pgfpathlineto{\pgfqpoint{2.986367in}{3.726407in}}%
\pgfpathmoveto{\pgfqpoint{2.982109in}{3.730665in}}%
\pgfpathlineto{\pgfqpoint{2.982109in}{3.730665in}}%
\pgfpathlineto{\pgfqpoint{2.982109in}{3.734923in}}%
\pgfpathlineto{\pgfqpoint{2.986367in}{3.734923in}}%
\pgfpathlineto{\pgfqpoint{2.986367in}{3.730665in}}%
\pgfpathmoveto{\pgfqpoint{2.986367in}{3.726407in}}%
\pgfpathlineto{\pgfqpoint{2.986367in}{3.726407in}}%
\pgfpathlineto{\pgfqpoint{2.986367in}{3.730665in}}%
\pgfpathlineto{\pgfqpoint{2.990624in}{3.730665in}}%
\pgfpathlineto{\pgfqpoint{2.990624in}{3.726407in}}%
\pgfpathmoveto{\pgfqpoint{2.986367in}{3.730665in}}%
\pgfpathlineto{\pgfqpoint{2.986367in}{3.730665in}}%
\pgfpathlineto{\pgfqpoint{2.986367in}{3.734923in}}%
\pgfpathlineto{\pgfqpoint{2.990624in}{3.734923in}}%
\pgfpathlineto{\pgfqpoint{2.990624in}{3.730665in}}%
\pgfpathmoveto{\pgfqpoint{2.982109in}{3.734923in}}%
\pgfpathlineto{\pgfqpoint{2.982109in}{3.734923in}}%
\pgfpathlineto{\pgfqpoint{2.982109in}{3.739181in}}%
\pgfpathlineto{\pgfqpoint{2.986367in}{3.739181in}}%
\pgfpathlineto{\pgfqpoint{2.986367in}{3.734923in}}%
\pgfpathmoveto{\pgfqpoint{2.982109in}{3.739181in}}%
\pgfpathlineto{\pgfqpoint{2.982109in}{3.739181in}}%
\pgfpathlineto{\pgfqpoint{2.982109in}{3.743439in}}%
\pgfpathlineto{\pgfqpoint{2.986367in}{3.743439in}}%
\pgfpathlineto{\pgfqpoint{2.986367in}{3.739181in}}%
\pgfpathmoveto{\pgfqpoint{2.986367in}{3.734923in}}%
\pgfpathlineto{\pgfqpoint{2.986367in}{3.734923in}}%
\pgfpathlineto{\pgfqpoint{2.986367in}{3.739181in}}%
\pgfpathlineto{\pgfqpoint{2.990624in}{3.739181in}}%
\pgfpathlineto{\pgfqpoint{2.990624in}{3.734923in}}%
\pgfpathmoveto{\pgfqpoint{2.986367in}{3.739181in}}%
\pgfpathlineto{\pgfqpoint{2.986367in}{3.739181in}}%
\pgfpathlineto{\pgfqpoint{2.986367in}{3.743439in}}%
\pgfpathlineto{\pgfqpoint{2.990624in}{3.743439in}}%
\pgfpathlineto{\pgfqpoint{2.990624in}{3.739181in}}%
\pgfpathmoveto{\pgfqpoint{2.956561in}{3.743439in}}%
\pgfpathlineto{\pgfqpoint{2.956561in}{3.743439in}}%
\pgfpathlineto{\pgfqpoint{2.956561in}{3.747696in}}%
\pgfpathlineto{\pgfqpoint{2.960819in}{3.747696in}}%
\pgfpathlineto{\pgfqpoint{2.960819in}{3.743439in}}%
\pgfpathmoveto{\pgfqpoint{2.956561in}{3.747696in}}%
\pgfpathlineto{\pgfqpoint{2.956561in}{3.747696in}}%
\pgfpathlineto{\pgfqpoint{2.956561in}{3.751954in}}%
\pgfpathlineto{\pgfqpoint{2.960819in}{3.751954in}}%
\pgfpathlineto{\pgfqpoint{2.960819in}{3.747696in}}%
\pgfpathmoveto{\pgfqpoint{2.960819in}{3.743439in}}%
\pgfpathlineto{\pgfqpoint{2.960819in}{3.743439in}}%
\pgfpathlineto{\pgfqpoint{2.960819in}{3.747696in}}%
\pgfpathlineto{\pgfqpoint{2.965077in}{3.747696in}}%
\pgfpathlineto{\pgfqpoint{2.965077in}{3.743439in}}%
\pgfpathmoveto{\pgfqpoint{2.960819in}{3.747696in}}%
\pgfpathlineto{\pgfqpoint{2.960819in}{3.747696in}}%
\pgfpathlineto{\pgfqpoint{2.960819in}{3.751954in}}%
\pgfpathlineto{\pgfqpoint{2.965077in}{3.751954in}}%
\pgfpathlineto{\pgfqpoint{2.965077in}{3.747696in}}%
\pgfpathmoveto{\pgfqpoint{2.956561in}{3.751954in}}%
\pgfpathlineto{\pgfqpoint{2.956561in}{3.751954in}}%
\pgfpathlineto{\pgfqpoint{2.956561in}{3.756212in}}%
\pgfpathlineto{\pgfqpoint{2.960819in}{3.756212in}}%
\pgfpathlineto{\pgfqpoint{2.960819in}{3.751954in}}%
\pgfpathmoveto{\pgfqpoint{2.956561in}{3.756212in}}%
\pgfpathlineto{\pgfqpoint{2.956561in}{3.756212in}}%
\pgfpathlineto{\pgfqpoint{2.956561in}{3.760470in}}%
\pgfpathlineto{\pgfqpoint{2.960819in}{3.760470in}}%
\pgfpathlineto{\pgfqpoint{2.960819in}{3.756212in}}%
\pgfpathmoveto{\pgfqpoint{2.960819in}{3.751954in}}%
\pgfpathlineto{\pgfqpoint{2.960819in}{3.751954in}}%
\pgfpathlineto{\pgfqpoint{2.960819in}{3.756212in}}%
\pgfpathlineto{\pgfqpoint{2.965077in}{3.756212in}}%
\pgfpathlineto{\pgfqpoint{2.965077in}{3.751954in}}%
\pgfpathmoveto{\pgfqpoint{2.960819in}{3.756212in}}%
\pgfpathlineto{\pgfqpoint{2.960819in}{3.756212in}}%
\pgfpathlineto{\pgfqpoint{2.960819in}{3.760470in}}%
\pgfpathlineto{\pgfqpoint{2.965077in}{3.760470in}}%
\pgfpathlineto{\pgfqpoint{2.965077in}{3.756212in}}%
\pgfpathmoveto{\pgfqpoint{2.965077in}{3.743439in}}%
\pgfpathlineto{\pgfqpoint{2.965077in}{3.743439in}}%
\pgfpathlineto{\pgfqpoint{2.965077in}{3.747696in}}%
\pgfpathlineto{\pgfqpoint{2.969335in}{3.747696in}}%
\pgfpathlineto{\pgfqpoint{2.969335in}{3.743439in}}%
\pgfpathmoveto{\pgfqpoint{2.965077in}{3.747696in}}%
\pgfpathlineto{\pgfqpoint{2.965077in}{3.747696in}}%
\pgfpathlineto{\pgfqpoint{2.965077in}{3.751954in}}%
\pgfpathlineto{\pgfqpoint{2.969335in}{3.751954in}}%
\pgfpathlineto{\pgfqpoint{2.969335in}{3.747696in}}%
\pgfpathmoveto{\pgfqpoint{2.969335in}{3.743439in}}%
\pgfpathlineto{\pgfqpoint{2.969335in}{3.743439in}}%
\pgfpathlineto{\pgfqpoint{2.969335in}{3.747696in}}%
\pgfpathlineto{\pgfqpoint{2.973593in}{3.747696in}}%
\pgfpathlineto{\pgfqpoint{2.973593in}{3.743439in}}%
\pgfpathmoveto{\pgfqpoint{2.969335in}{3.747696in}}%
\pgfpathlineto{\pgfqpoint{2.969335in}{3.747696in}}%
\pgfpathlineto{\pgfqpoint{2.969335in}{3.751954in}}%
\pgfpathlineto{\pgfqpoint{2.973593in}{3.751954in}}%
\pgfpathlineto{\pgfqpoint{2.973593in}{3.747696in}}%
\pgfpathmoveto{\pgfqpoint{2.965077in}{3.751954in}}%
\pgfpathlineto{\pgfqpoint{2.965077in}{3.751954in}}%
\pgfpathlineto{\pgfqpoint{2.965077in}{3.756212in}}%
\pgfpathlineto{\pgfqpoint{2.969335in}{3.756212in}}%
\pgfpathlineto{\pgfqpoint{2.969335in}{3.751954in}}%
\pgfpathmoveto{\pgfqpoint{2.965077in}{3.756212in}}%
\pgfpathlineto{\pgfqpoint{2.965077in}{3.756212in}}%
\pgfpathlineto{\pgfqpoint{2.965077in}{3.760470in}}%
\pgfpathlineto{\pgfqpoint{2.969335in}{3.760470in}}%
\pgfpathlineto{\pgfqpoint{2.969335in}{3.756212in}}%
\pgfpathmoveto{\pgfqpoint{2.969335in}{3.751954in}}%
\pgfpathlineto{\pgfqpoint{2.969335in}{3.751954in}}%
\pgfpathlineto{\pgfqpoint{2.969335in}{3.756212in}}%
\pgfpathlineto{\pgfqpoint{2.973593in}{3.756212in}}%
\pgfpathlineto{\pgfqpoint{2.973593in}{3.751954in}}%
\pgfpathmoveto{\pgfqpoint{2.969335in}{3.756212in}}%
\pgfpathlineto{\pgfqpoint{2.969335in}{3.756212in}}%
\pgfpathlineto{\pgfqpoint{2.969335in}{3.760470in}}%
\pgfpathlineto{\pgfqpoint{2.973593in}{3.760470in}}%
\pgfpathlineto{\pgfqpoint{2.973593in}{3.756212in}}%
\pgfpathmoveto{\pgfqpoint{2.956561in}{3.760470in}}%
\pgfpathlineto{\pgfqpoint{2.956561in}{3.760470in}}%
\pgfpathlineto{\pgfqpoint{2.956561in}{3.764728in}}%
\pgfpathlineto{\pgfqpoint{2.960819in}{3.764728in}}%
\pgfpathlineto{\pgfqpoint{2.960819in}{3.760470in}}%
\pgfpathmoveto{\pgfqpoint{2.956561in}{3.764728in}}%
\pgfpathlineto{\pgfqpoint{2.956561in}{3.764728in}}%
\pgfpathlineto{\pgfqpoint{2.956561in}{3.768986in}}%
\pgfpathlineto{\pgfqpoint{2.960819in}{3.768986in}}%
\pgfpathlineto{\pgfqpoint{2.960819in}{3.764728in}}%
\pgfpathmoveto{\pgfqpoint{2.960819in}{3.760470in}}%
\pgfpathlineto{\pgfqpoint{2.960819in}{3.760470in}}%
\pgfpathlineto{\pgfqpoint{2.960819in}{3.764728in}}%
\pgfpathlineto{\pgfqpoint{2.965077in}{3.764728in}}%
\pgfpathlineto{\pgfqpoint{2.965077in}{3.760470in}}%
\pgfpathmoveto{\pgfqpoint{2.960819in}{3.764728in}}%
\pgfpathlineto{\pgfqpoint{2.960819in}{3.764728in}}%
\pgfpathlineto{\pgfqpoint{2.960819in}{3.768986in}}%
\pgfpathlineto{\pgfqpoint{2.965077in}{3.768986in}}%
\pgfpathlineto{\pgfqpoint{2.965077in}{3.764728in}}%
\pgfpathmoveto{\pgfqpoint{2.956561in}{3.768986in}}%
\pgfpathlineto{\pgfqpoint{2.956561in}{3.768986in}}%
\pgfpathlineto{\pgfqpoint{2.956561in}{3.773243in}}%
\pgfpathlineto{\pgfqpoint{2.960819in}{3.773243in}}%
\pgfpathlineto{\pgfqpoint{2.960819in}{3.768986in}}%
\pgfpathmoveto{\pgfqpoint{2.960819in}{3.768986in}}%
\pgfpathlineto{\pgfqpoint{2.960819in}{3.768986in}}%
\pgfpathlineto{\pgfqpoint{2.960819in}{3.773243in}}%
\pgfpathlineto{\pgfqpoint{2.965077in}{3.773243in}}%
\pgfpathlineto{\pgfqpoint{2.965077in}{3.768986in}}%
\pgfpathmoveto{\pgfqpoint{2.965077in}{3.760470in}}%
\pgfpathlineto{\pgfqpoint{2.965077in}{3.760470in}}%
\pgfpathlineto{\pgfqpoint{2.965077in}{3.764728in}}%
\pgfpathlineto{\pgfqpoint{2.969335in}{3.764728in}}%
\pgfpathlineto{\pgfqpoint{2.969335in}{3.760470in}}%
\pgfpathmoveto{\pgfqpoint{2.965077in}{3.764728in}}%
\pgfpathlineto{\pgfqpoint{2.965077in}{3.764728in}}%
\pgfpathlineto{\pgfqpoint{2.965077in}{3.768986in}}%
\pgfpathlineto{\pgfqpoint{2.969335in}{3.768986in}}%
\pgfpathlineto{\pgfqpoint{2.969335in}{3.764728in}}%
\pgfpathmoveto{\pgfqpoint{2.969335in}{3.760470in}}%
\pgfpathlineto{\pgfqpoint{2.969335in}{3.760470in}}%
\pgfpathlineto{\pgfqpoint{2.969335in}{3.764728in}}%
\pgfpathlineto{\pgfqpoint{2.973593in}{3.764728in}}%
\pgfpathlineto{\pgfqpoint{2.973593in}{3.760470in}}%
\pgfpathmoveto{\pgfqpoint{2.969335in}{3.764728in}}%
\pgfpathlineto{\pgfqpoint{2.969335in}{3.764728in}}%
\pgfpathlineto{\pgfqpoint{2.969335in}{3.768986in}}%
\pgfpathlineto{\pgfqpoint{2.973593in}{3.768986in}}%
\pgfpathlineto{\pgfqpoint{2.973593in}{3.764728in}}%
\pgfpathmoveto{\pgfqpoint{2.965077in}{3.768986in}}%
\pgfpathlineto{\pgfqpoint{2.965077in}{3.768986in}}%
\pgfpathlineto{\pgfqpoint{2.965077in}{3.773243in}}%
\pgfpathlineto{\pgfqpoint{2.969335in}{3.773243in}}%
\pgfpathlineto{\pgfqpoint{2.969335in}{3.768986in}}%
\pgfpathmoveto{\pgfqpoint{2.973593in}{3.743439in}}%
\pgfpathlineto{\pgfqpoint{2.973593in}{3.743439in}}%
\pgfpathlineto{\pgfqpoint{2.973593in}{3.747696in}}%
\pgfpathlineto{\pgfqpoint{2.977851in}{3.747696in}}%
\pgfpathlineto{\pgfqpoint{2.977851in}{3.743439in}}%
\pgfpathmoveto{\pgfqpoint{2.973593in}{3.747696in}}%
\pgfpathlineto{\pgfqpoint{2.973593in}{3.747696in}}%
\pgfpathlineto{\pgfqpoint{2.973593in}{3.751954in}}%
\pgfpathlineto{\pgfqpoint{2.977851in}{3.751954in}}%
\pgfpathlineto{\pgfqpoint{2.977851in}{3.747696in}}%
\pgfpathmoveto{\pgfqpoint{2.977851in}{3.743439in}}%
\pgfpathlineto{\pgfqpoint{2.977851in}{3.743439in}}%
\pgfpathlineto{\pgfqpoint{2.977851in}{3.747696in}}%
\pgfpathlineto{\pgfqpoint{2.982109in}{3.747696in}}%
\pgfpathlineto{\pgfqpoint{2.982109in}{3.743439in}}%
\pgfpathmoveto{\pgfqpoint{2.977851in}{3.747696in}}%
\pgfpathlineto{\pgfqpoint{2.977851in}{3.747696in}}%
\pgfpathlineto{\pgfqpoint{2.977851in}{3.751954in}}%
\pgfpathlineto{\pgfqpoint{2.982109in}{3.751954in}}%
\pgfpathlineto{\pgfqpoint{2.982109in}{3.747696in}}%
\pgfpathmoveto{\pgfqpoint{2.973593in}{3.751954in}}%
\pgfpathlineto{\pgfqpoint{2.973593in}{3.751954in}}%
\pgfpathlineto{\pgfqpoint{2.973593in}{3.756212in}}%
\pgfpathlineto{\pgfqpoint{2.977851in}{3.756212in}}%
\pgfpathlineto{\pgfqpoint{2.977851in}{3.751954in}}%
\pgfpathmoveto{\pgfqpoint{2.973593in}{3.756212in}}%
\pgfpathlineto{\pgfqpoint{2.973593in}{3.756212in}}%
\pgfpathlineto{\pgfqpoint{2.973593in}{3.760470in}}%
\pgfpathlineto{\pgfqpoint{2.977851in}{3.760470in}}%
\pgfpathlineto{\pgfqpoint{2.977851in}{3.756212in}}%
\pgfpathmoveto{\pgfqpoint{2.977851in}{3.751954in}}%
\pgfpathlineto{\pgfqpoint{2.977851in}{3.751954in}}%
\pgfpathlineto{\pgfqpoint{2.977851in}{3.756212in}}%
\pgfpathlineto{\pgfqpoint{2.982109in}{3.756212in}}%
\pgfpathlineto{\pgfqpoint{2.982109in}{3.751954in}}%
\pgfpathmoveto{\pgfqpoint{2.977851in}{3.756212in}}%
\pgfpathlineto{\pgfqpoint{2.977851in}{3.756212in}}%
\pgfpathlineto{\pgfqpoint{2.977851in}{3.760470in}}%
\pgfpathlineto{\pgfqpoint{2.982109in}{3.760470in}}%
\pgfpathlineto{\pgfqpoint{2.982109in}{3.756212in}}%
\pgfpathmoveto{\pgfqpoint{2.982109in}{3.743439in}}%
\pgfpathlineto{\pgfqpoint{2.982109in}{3.743439in}}%
\pgfpathlineto{\pgfqpoint{2.982109in}{3.747696in}}%
\pgfpathlineto{\pgfqpoint{2.986367in}{3.747696in}}%
\pgfpathlineto{\pgfqpoint{2.986367in}{3.743439in}}%
\pgfpathmoveto{\pgfqpoint{2.982109in}{3.747696in}}%
\pgfpathlineto{\pgfqpoint{2.982109in}{3.747696in}}%
\pgfpathlineto{\pgfqpoint{2.982109in}{3.751954in}}%
\pgfpathlineto{\pgfqpoint{2.986367in}{3.751954in}}%
\pgfpathlineto{\pgfqpoint{2.986367in}{3.747696in}}%
\pgfpathmoveto{\pgfqpoint{2.986367in}{3.743439in}}%
\pgfpathlineto{\pgfqpoint{2.986367in}{3.743439in}}%
\pgfpathlineto{\pgfqpoint{2.986367in}{3.747696in}}%
\pgfpathlineto{\pgfqpoint{2.990624in}{3.747696in}}%
\pgfpathlineto{\pgfqpoint{2.990624in}{3.743439in}}%
\pgfpathmoveto{\pgfqpoint{2.986367in}{3.747696in}}%
\pgfpathlineto{\pgfqpoint{2.986367in}{3.747696in}}%
\pgfpathlineto{\pgfqpoint{2.986367in}{3.751954in}}%
\pgfpathlineto{\pgfqpoint{2.990624in}{3.751954in}}%
\pgfpathlineto{\pgfqpoint{2.990624in}{3.747696in}}%
\pgfpathmoveto{\pgfqpoint{2.982109in}{3.751954in}}%
\pgfpathlineto{\pgfqpoint{2.982109in}{3.751954in}}%
\pgfpathlineto{\pgfqpoint{2.982109in}{3.756212in}}%
\pgfpathlineto{\pgfqpoint{2.986367in}{3.756212in}}%
\pgfpathlineto{\pgfqpoint{2.986367in}{3.751954in}}%
\pgfpathmoveto{\pgfqpoint{2.982109in}{3.756212in}}%
\pgfpathlineto{\pgfqpoint{2.982109in}{3.756212in}}%
\pgfpathlineto{\pgfqpoint{2.982109in}{3.760470in}}%
\pgfpathlineto{\pgfqpoint{2.986367in}{3.760470in}}%
\pgfpathlineto{\pgfqpoint{2.986367in}{3.756212in}}%
\pgfpathmoveto{\pgfqpoint{2.986367in}{3.751954in}}%
\pgfpathlineto{\pgfqpoint{2.986367in}{3.751954in}}%
\pgfpathlineto{\pgfqpoint{2.986367in}{3.756212in}}%
\pgfpathlineto{\pgfqpoint{2.990624in}{3.756212in}}%
\pgfpathlineto{\pgfqpoint{2.990624in}{3.751954in}}%
\pgfpathmoveto{\pgfqpoint{2.986367in}{3.756212in}}%
\pgfpathlineto{\pgfqpoint{2.986367in}{3.756212in}}%
\pgfpathlineto{\pgfqpoint{2.986367in}{3.760470in}}%
\pgfpathlineto{\pgfqpoint{2.990624in}{3.760470in}}%
\pgfpathlineto{\pgfqpoint{2.990624in}{3.756212in}}%
\pgfpathmoveto{\pgfqpoint{2.973593in}{3.760470in}}%
\pgfpathlineto{\pgfqpoint{2.973593in}{3.760470in}}%
\pgfpathlineto{\pgfqpoint{2.973593in}{3.764728in}}%
\pgfpathlineto{\pgfqpoint{2.977851in}{3.764728in}}%
\pgfpathlineto{\pgfqpoint{2.977851in}{3.760470in}}%
\pgfpathmoveto{\pgfqpoint{2.973593in}{3.764728in}}%
\pgfpathlineto{\pgfqpoint{2.973593in}{3.764728in}}%
\pgfpathlineto{\pgfqpoint{2.973593in}{3.768986in}}%
\pgfpathlineto{\pgfqpoint{2.977851in}{3.768986in}}%
\pgfpathlineto{\pgfqpoint{2.977851in}{3.764728in}}%
\pgfpathmoveto{\pgfqpoint{2.977851in}{3.760470in}}%
\pgfpathlineto{\pgfqpoint{2.977851in}{3.760470in}}%
\pgfpathlineto{\pgfqpoint{2.977851in}{3.764728in}}%
\pgfpathlineto{\pgfqpoint{2.982109in}{3.764728in}}%
\pgfpathlineto{\pgfqpoint{2.982109in}{3.760470in}}%
\pgfpathmoveto{\pgfqpoint{2.977851in}{3.764728in}}%
\pgfpathlineto{\pgfqpoint{2.977851in}{3.764728in}}%
\pgfpathlineto{\pgfqpoint{2.977851in}{3.768986in}}%
\pgfpathlineto{\pgfqpoint{2.982109in}{3.768986in}}%
\pgfpathlineto{\pgfqpoint{2.982109in}{3.764728in}}%
\pgfpathmoveto{\pgfqpoint{2.982109in}{3.760470in}}%
\pgfpathlineto{\pgfqpoint{2.982109in}{3.760470in}}%
\pgfpathlineto{\pgfqpoint{2.982109in}{3.764728in}}%
\pgfpathlineto{\pgfqpoint{2.986367in}{3.764728in}}%
\pgfpathlineto{\pgfqpoint{2.986367in}{3.760470in}}%
\pgfpathmoveto{\pgfqpoint{2.986367in}{3.760470in}}%
\pgfpathlineto{\pgfqpoint{2.986367in}{3.760470in}}%
\pgfpathlineto{\pgfqpoint{2.986367in}{3.764728in}}%
\pgfpathlineto{\pgfqpoint{2.990624in}{3.764728in}}%
\pgfpathlineto{\pgfqpoint{2.990624in}{3.760470in}}%
\pgfpathmoveto{\pgfqpoint{2.994882in}{3.713634in}}%
\pgfpathlineto{\pgfqpoint{2.994882in}{3.713634in}}%
\pgfpathlineto{\pgfqpoint{2.994882in}{3.717892in}}%
\pgfpathlineto{\pgfqpoint{2.999140in}{3.717892in}}%
\pgfpathlineto{\pgfqpoint{2.999140in}{3.713634in}}%
\pgfpathmoveto{\pgfqpoint{2.990624in}{3.717892in}}%
\pgfpathlineto{\pgfqpoint{2.990624in}{3.717892in}}%
\pgfpathlineto{\pgfqpoint{2.990624in}{3.722149in}}%
\pgfpathlineto{\pgfqpoint{2.994882in}{3.722149in}}%
\pgfpathlineto{\pgfqpoint{2.994882in}{3.717892in}}%
\pgfpathmoveto{\pgfqpoint{2.990624in}{3.722149in}}%
\pgfpathlineto{\pgfqpoint{2.990624in}{3.722149in}}%
\pgfpathlineto{\pgfqpoint{2.990624in}{3.726407in}}%
\pgfpathlineto{\pgfqpoint{2.994882in}{3.726407in}}%
\pgfpathlineto{\pgfqpoint{2.994882in}{3.722149in}}%
\pgfpathmoveto{\pgfqpoint{2.994882in}{3.717892in}}%
\pgfpathlineto{\pgfqpoint{2.994882in}{3.717892in}}%
\pgfpathlineto{\pgfqpoint{2.994882in}{3.722149in}}%
\pgfpathlineto{\pgfqpoint{2.999140in}{3.722149in}}%
\pgfpathlineto{\pgfqpoint{2.999140in}{3.717892in}}%
\pgfpathmoveto{\pgfqpoint{2.994882in}{3.722149in}}%
\pgfpathlineto{\pgfqpoint{2.994882in}{3.722149in}}%
\pgfpathlineto{\pgfqpoint{2.994882in}{3.726407in}}%
\pgfpathlineto{\pgfqpoint{2.999140in}{3.726407in}}%
\pgfpathlineto{\pgfqpoint{2.999140in}{3.722149in}}%
\pgfpathmoveto{\pgfqpoint{2.999140in}{3.709376in}}%
\pgfpathlineto{\pgfqpoint{2.999140in}{3.709376in}}%
\pgfpathlineto{\pgfqpoint{2.999140in}{3.713634in}}%
\pgfpathlineto{\pgfqpoint{3.003398in}{3.713634in}}%
\pgfpathlineto{\pgfqpoint{3.003398in}{3.709376in}}%
\pgfpathmoveto{\pgfqpoint{2.999140in}{3.713634in}}%
\pgfpathlineto{\pgfqpoint{2.999140in}{3.713634in}}%
\pgfpathlineto{\pgfqpoint{2.999140in}{3.717892in}}%
\pgfpathlineto{\pgfqpoint{3.003398in}{3.717892in}}%
\pgfpathlineto{\pgfqpoint{3.003398in}{3.713634in}}%
\pgfpathmoveto{\pgfqpoint{3.003398in}{3.709376in}}%
\pgfpathlineto{\pgfqpoint{3.003398in}{3.709376in}}%
\pgfpathlineto{\pgfqpoint{3.003398in}{3.713634in}}%
\pgfpathlineto{\pgfqpoint{3.007656in}{3.713634in}}%
\pgfpathlineto{\pgfqpoint{3.007656in}{3.709376in}}%
\pgfpathmoveto{\pgfqpoint{3.003398in}{3.713634in}}%
\pgfpathlineto{\pgfqpoint{3.003398in}{3.713634in}}%
\pgfpathlineto{\pgfqpoint{3.003398in}{3.717892in}}%
\pgfpathlineto{\pgfqpoint{3.007656in}{3.717892in}}%
\pgfpathlineto{\pgfqpoint{3.007656in}{3.713634in}}%
\pgfpathmoveto{\pgfqpoint{2.999140in}{3.717892in}}%
\pgfpathlineto{\pgfqpoint{2.999140in}{3.717892in}}%
\pgfpathlineto{\pgfqpoint{2.999140in}{3.722149in}}%
\pgfpathlineto{\pgfqpoint{3.003398in}{3.722149in}}%
\pgfpathlineto{\pgfqpoint{3.003398in}{3.717892in}}%
\pgfpathmoveto{\pgfqpoint{2.999140in}{3.722149in}}%
\pgfpathlineto{\pgfqpoint{2.999140in}{3.722149in}}%
\pgfpathlineto{\pgfqpoint{2.999140in}{3.726407in}}%
\pgfpathlineto{\pgfqpoint{3.003398in}{3.726407in}}%
\pgfpathlineto{\pgfqpoint{3.003398in}{3.722149in}}%
\pgfpathmoveto{\pgfqpoint{3.003398in}{3.717892in}}%
\pgfpathlineto{\pgfqpoint{3.003398in}{3.717892in}}%
\pgfpathlineto{\pgfqpoint{3.003398in}{3.722149in}}%
\pgfpathlineto{\pgfqpoint{3.007656in}{3.722149in}}%
\pgfpathlineto{\pgfqpoint{3.007656in}{3.717892in}}%
\pgfpathmoveto{\pgfqpoint{3.003398in}{3.722149in}}%
\pgfpathlineto{\pgfqpoint{3.003398in}{3.722149in}}%
\pgfpathlineto{\pgfqpoint{3.003398in}{3.726407in}}%
\pgfpathlineto{\pgfqpoint{3.007656in}{3.726407in}}%
\pgfpathlineto{\pgfqpoint{3.007656in}{3.722149in}}%
\pgfpathmoveto{\pgfqpoint{2.990624in}{3.726407in}}%
\pgfpathlineto{\pgfqpoint{2.990624in}{3.726407in}}%
\pgfpathlineto{\pgfqpoint{2.990624in}{3.730665in}}%
\pgfpathlineto{\pgfqpoint{2.994882in}{3.730665in}}%
\pgfpathlineto{\pgfqpoint{2.994882in}{3.726407in}}%
\pgfpathmoveto{\pgfqpoint{2.990624in}{3.730665in}}%
\pgfpathlineto{\pgfqpoint{2.990624in}{3.730665in}}%
\pgfpathlineto{\pgfqpoint{2.990624in}{3.734923in}}%
\pgfpathlineto{\pgfqpoint{2.994882in}{3.734923in}}%
\pgfpathlineto{\pgfqpoint{2.994882in}{3.730665in}}%
\pgfpathmoveto{\pgfqpoint{2.994882in}{3.726407in}}%
\pgfpathlineto{\pgfqpoint{2.994882in}{3.726407in}}%
\pgfpathlineto{\pgfqpoint{2.994882in}{3.730665in}}%
\pgfpathlineto{\pgfqpoint{2.999140in}{3.730665in}}%
\pgfpathlineto{\pgfqpoint{2.999140in}{3.726407in}}%
\pgfpathmoveto{\pgfqpoint{2.994882in}{3.730665in}}%
\pgfpathlineto{\pgfqpoint{2.994882in}{3.730665in}}%
\pgfpathlineto{\pgfqpoint{2.994882in}{3.734923in}}%
\pgfpathlineto{\pgfqpoint{2.999140in}{3.734923in}}%
\pgfpathlineto{\pgfqpoint{2.999140in}{3.730665in}}%
\pgfpathmoveto{\pgfqpoint{2.990624in}{3.734923in}}%
\pgfpathlineto{\pgfqpoint{2.990624in}{3.734923in}}%
\pgfpathlineto{\pgfqpoint{2.990624in}{3.739181in}}%
\pgfpathlineto{\pgfqpoint{2.994882in}{3.739181in}}%
\pgfpathlineto{\pgfqpoint{2.994882in}{3.734923in}}%
\pgfpathmoveto{\pgfqpoint{2.990624in}{3.739181in}}%
\pgfpathlineto{\pgfqpoint{2.990624in}{3.739181in}}%
\pgfpathlineto{\pgfqpoint{2.990624in}{3.743439in}}%
\pgfpathlineto{\pgfqpoint{2.994882in}{3.743439in}}%
\pgfpathlineto{\pgfqpoint{2.994882in}{3.739181in}}%
\pgfpathmoveto{\pgfqpoint{2.994882in}{3.734923in}}%
\pgfpathlineto{\pgfqpoint{2.994882in}{3.734923in}}%
\pgfpathlineto{\pgfqpoint{2.994882in}{3.739181in}}%
\pgfpathlineto{\pgfqpoint{2.999140in}{3.739181in}}%
\pgfpathlineto{\pgfqpoint{2.999140in}{3.734923in}}%
\pgfpathmoveto{\pgfqpoint{2.994882in}{3.739181in}}%
\pgfpathlineto{\pgfqpoint{2.994882in}{3.739181in}}%
\pgfpathlineto{\pgfqpoint{2.994882in}{3.743439in}}%
\pgfpathlineto{\pgfqpoint{2.999140in}{3.743439in}}%
\pgfpathlineto{\pgfqpoint{2.999140in}{3.739181in}}%
\pgfpathmoveto{\pgfqpoint{2.999140in}{3.726407in}}%
\pgfpathlineto{\pgfqpoint{2.999140in}{3.726407in}}%
\pgfpathlineto{\pgfqpoint{2.999140in}{3.730665in}}%
\pgfpathlineto{\pgfqpoint{3.003398in}{3.730665in}}%
\pgfpathlineto{\pgfqpoint{3.003398in}{3.726407in}}%
\pgfpathmoveto{\pgfqpoint{2.999140in}{3.730665in}}%
\pgfpathlineto{\pgfqpoint{2.999140in}{3.730665in}}%
\pgfpathlineto{\pgfqpoint{2.999140in}{3.734923in}}%
\pgfpathlineto{\pgfqpoint{3.003398in}{3.734923in}}%
\pgfpathlineto{\pgfqpoint{3.003398in}{3.730665in}}%
\pgfpathmoveto{\pgfqpoint{3.003398in}{3.726407in}}%
\pgfpathlineto{\pgfqpoint{3.003398in}{3.726407in}}%
\pgfpathlineto{\pgfqpoint{3.003398in}{3.730665in}}%
\pgfpathlineto{\pgfqpoint{3.007656in}{3.730665in}}%
\pgfpathlineto{\pgfqpoint{3.007656in}{3.726407in}}%
\pgfpathmoveto{\pgfqpoint{3.003398in}{3.730665in}}%
\pgfpathlineto{\pgfqpoint{3.003398in}{3.730665in}}%
\pgfpathlineto{\pgfqpoint{3.003398in}{3.734923in}}%
\pgfpathlineto{\pgfqpoint{3.007656in}{3.734923in}}%
\pgfpathlineto{\pgfqpoint{3.007656in}{3.730665in}}%
\pgfpathmoveto{\pgfqpoint{2.999140in}{3.734923in}}%
\pgfpathlineto{\pgfqpoint{2.999140in}{3.734923in}}%
\pgfpathlineto{\pgfqpoint{2.999140in}{3.739181in}}%
\pgfpathlineto{\pgfqpoint{3.003398in}{3.739181in}}%
\pgfpathlineto{\pgfqpoint{3.003398in}{3.734923in}}%
\pgfpathmoveto{\pgfqpoint{2.999140in}{3.739181in}}%
\pgfpathlineto{\pgfqpoint{2.999140in}{3.739181in}}%
\pgfpathlineto{\pgfqpoint{2.999140in}{3.743439in}}%
\pgfpathlineto{\pgfqpoint{3.003398in}{3.743439in}}%
\pgfpathlineto{\pgfqpoint{3.003398in}{3.739181in}}%
\pgfpathmoveto{\pgfqpoint{3.003398in}{3.734923in}}%
\pgfpathlineto{\pgfqpoint{3.003398in}{3.734923in}}%
\pgfpathlineto{\pgfqpoint{3.003398in}{3.739181in}}%
\pgfpathlineto{\pgfqpoint{3.007656in}{3.739181in}}%
\pgfpathlineto{\pgfqpoint{3.007656in}{3.734923in}}%
\pgfpathmoveto{\pgfqpoint{3.003398in}{3.739181in}}%
\pgfpathlineto{\pgfqpoint{3.003398in}{3.739181in}}%
\pgfpathlineto{\pgfqpoint{3.003398in}{3.743439in}}%
\pgfpathlineto{\pgfqpoint{3.007656in}{3.743439in}}%
\pgfpathlineto{\pgfqpoint{3.007656in}{3.739181in}}%
\pgfpathmoveto{\pgfqpoint{3.007656in}{3.709376in}}%
\pgfpathlineto{\pgfqpoint{3.007656in}{3.709376in}}%
\pgfpathlineto{\pgfqpoint{3.007656in}{3.713634in}}%
\pgfpathlineto{\pgfqpoint{3.011914in}{3.713634in}}%
\pgfpathlineto{\pgfqpoint{3.011914in}{3.709376in}}%
\pgfpathmoveto{\pgfqpoint{3.007656in}{3.713634in}}%
\pgfpathlineto{\pgfqpoint{3.007656in}{3.713634in}}%
\pgfpathlineto{\pgfqpoint{3.007656in}{3.717892in}}%
\pgfpathlineto{\pgfqpoint{3.011914in}{3.717892in}}%
\pgfpathlineto{\pgfqpoint{3.011914in}{3.713634in}}%
\pgfpathmoveto{\pgfqpoint{3.011914in}{3.709376in}}%
\pgfpathlineto{\pgfqpoint{3.011914in}{3.709376in}}%
\pgfpathlineto{\pgfqpoint{3.011914in}{3.713634in}}%
\pgfpathlineto{\pgfqpoint{3.016172in}{3.713634in}}%
\pgfpathlineto{\pgfqpoint{3.016172in}{3.709376in}}%
\pgfpathmoveto{\pgfqpoint{3.011914in}{3.713634in}}%
\pgfpathlineto{\pgfqpoint{3.011914in}{3.713634in}}%
\pgfpathlineto{\pgfqpoint{3.011914in}{3.717892in}}%
\pgfpathlineto{\pgfqpoint{3.016172in}{3.717892in}}%
\pgfpathlineto{\pgfqpoint{3.016172in}{3.713634in}}%
\pgfpathmoveto{\pgfqpoint{3.007656in}{3.717892in}}%
\pgfpathlineto{\pgfqpoint{3.007656in}{3.717892in}}%
\pgfpathlineto{\pgfqpoint{3.007656in}{3.722149in}}%
\pgfpathlineto{\pgfqpoint{3.011914in}{3.722149in}}%
\pgfpathlineto{\pgfqpoint{3.011914in}{3.717892in}}%
\pgfpathmoveto{\pgfqpoint{3.007656in}{3.722149in}}%
\pgfpathlineto{\pgfqpoint{3.007656in}{3.722149in}}%
\pgfpathlineto{\pgfqpoint{3.007656in}{3.726407in}}%
\pgfpathlineto{\pgfqpoint{3.011914in}{3.726407in}}%
\pgfpathlineto{\pgfqpoint{3.011914in}{3.722149in}}%
\pgfpathmoveto{\pgfqpoint{3.011914in}{3.717892in}}%
\pgfpathlineto{\pgfqpoint{3.011914in}{3.717892in}}%
\pgfpathlineto{\pgfqpoint{3.011914in}{3.722149in}}%
\pgfpathlineto{\pgfqpoint{3.016172in}{3.722149in}}%
\pgfpathlineto{\pgfqpoint{3.016172in}{3.717892in}}%
\pgfpathmoveto{\pgfqpoint{3.011914in}{3.722149in}}%
\pgfpathlineto{\pgfqpoint{3.011914in}{3.722149in}}%
\pgfpathlineto{\pgfqpoint{3.011914in}{3.726407in}}%
\pgfpathlineto{\pgfqpoint{3.016172in}{3.726407in}}%
\pgfpathlineto{\pgfqpoint{3.016172in}{3.722149in}}%
\pgfpathmoveto{\pgfqpoint{3.016172in}{3.709376in}}%
\pgfpathlineto{\pgfqpoint{3.016172in}{3.709376in}}%
\pgfpathlineto{\pgfqpoint{3.016172in}{3.713634in}}%
\pgfpathlineto{\pgfqpoint{3.020430in}{3.713634in}}%
\pgfpathlineto{\pgfqpoint{3.020430in}{3.709376in}}%
\pgfpathmoveto{\pgfqpoint{3.016172in}{3.713634in}}%
\pgfpathlineto{\pgfqpoint{3.016172in}{3.713634in}}%
\pgfpathlineto{\pgfqpoint{3.016172in}{3.717892in}}%
\pgfpathlineto{\pgfqpoint{3.020430in}{3.717892in}}%
\pgfpathlineto{\pgfqpoint{3.020430in}{3.713634in}}%
\pgfpathmoveto{\pgfqpoint{3.020430in}{3.709376in}}%
\pgfpathlineto{\pgfqpoint{3.020430in}{3.709376in}}%
\pgfpathlineto{\pgfqpoint{3.020430in}{3.713634in}}%
\pgfpathlineto{\pgfqpoint{3.024688in}{3.713634in}}%
\pgfpathlineto{\pgfqpoint{3.024688in}{3.709376in}}%
\pgfpathmoveto{\pgfqpoint{3.020430in}{3.713634in}}%
\pgfpathlineto{\pgfqpoint{3.020430in}{3.713634in}}%
\pgfpathlineto{\pgfqpoint{3.020430in}{3.717892in}}%
\pgfpathlineto{\pgfqpoint{3.024688in}{3.717892in}}%
\pgfpathlineto{\pgfqpoint{3.024688in}{3.713634in}}%
\pgfpathmoveto{\pgfqpoint{3.016172in}{3.717892in}}%
\pgfpathlineto{\pgfqpoint{3.016172in}{3.717892in}}%
\pgfpathlineto{\pgfqpoint{3.016172in}{3.722149in}}%
\pgfpathlineto{\pgfqpoint{3.020430in}{3.722149in}}%
\pgfpathlineto{\pgfqpoint{3.020430in}{3.717892in}}%
\pgfpathmoveto{\pgfqpoint{3.016172in}{3.722149in}}%
\pgfpathlineto{\pgfqpoint{3.016172in}{3.722149in}}%
\pgfpathlineto{\pgfqpoint{3.016172in}{3.726407in}}%
\pgfpathlineto{\pgfqpoint{3.020430in}{3.726407in}}%
\pgfpathlineto{\pgfqpoint{3.020430in}{3.722149in}}%
\pgfpathmoveto{\pgfqpoint{3.020430in}{3.717892in}}%
\pgfpathlineto{\pgfqpoint{3.020430in}{3.717892in}}%
\pgfpathlineto{\pgfqpoint{3.020430in}{3.722149in}}%
\pgfpathlineto{\pgfqpoint{3.024688in}{3.722149in}}%
\pgfpathlineto{\pgfqpoint{3.024688in}{3.717892in}}%
\pgfpathmoveto{\pgfqpoint{3.020430in}{3.722149in}}%
\pgfpathlineto{\pgfqpoint{3.020430in}{3.722149in}}%
\pgfpathlineto{\pgfqpoint{3.020430in}{3.726407in}}%
\pgfpathlineto{\pgfqpoint{3.024688in}{3.726407in}}%
\pgfpathlineto{\pgfqpoint{3.024688in}{3.722149in}}%
\pgfpathmoveto{\pgfqpoint{3.007656in}{3.726407in}}%
\pgfpathlineto{\pgfqpoint{3.007656in}{3.726407in}}%
\pgfpathlineto{\pgfqpoint{3.007656in}{3.730665in}}%
\pgfpathlineto{\pgfqpoint{3.011914in}{3.730665in}}%
\pgfpathlineto{\pgfqpoint{3.011914in}{3.726407in}}%
\pgfpathmoveto{\pgfqpoint{3.007656in}{3.730665in}}%
\pgfpathlineto{\pgfqpoint{3.007656in}{3.730665in}}%
\pgfpathlineto{\pgfqpoint{3.007656in}{3.734923in}}%
\pgfpathlineto{\pgfqpoint{3.011914in}{3.734923in}}%
\pgfpathlineto{\pgfqpoint{3.011914in}{3.730665in}}%
\pgfpathmoveto{\pgfqpoint{3.011914in}{3.726407in}}%
\pgfpathlineto{\pgfqpoint{3.011914in}{3.726407in}}%
\pgfpathlineto{\pgfqpoint{3.011914in}{3.730665in}}%
\pgfpathlineto{\pgfqpoint{3.016172in}{3.730665in}}%
\pgfpathlineto{\pgfqpoint{3.016172in}{3.726407in}}%
\pgfpathmoveto{\pgfqpoint{3.011914in}{3.730665in}}%
\pgfpathlineto{\pgfqpoint{3.011914in}{3.730665in}}%
\pgfpathlineto{\pgfqpoint{3.011914in}{3.734923in}}%
\pgfpathlineto{\pgfqpoint{3.016172in}{3.734923in}}%
\pgfpathlineto{\pgfqpoint{3.016172in}{3.730665in}}%
\pgfpathmoveto{\pgfqpoint{3.007656in}{3.734923in}}%
\pgfpathlineto{\pgfqpoint{3.007656in}{3.734923in}}%
\pgfpathlineto{\pgfqpoint{3.007656in}{3.739181in}}%
\pgfpathlineto{\pgfqpoint{3.011914in}{3.739181in}}%
\pgfpathlineto{\pgfqpoint{3.011914in}{3.734923in}}%
\pgfpathmoveto{\pgfqpoint{3.007656in}{3.739181in}}%
\pgfpathlineto{\pgfqpoint{3.007656in}{3.739181in}}%
\pgfpathlineto{\pgfqpoint{3.007656in}{3.743439in}}%
\pgfpathlineto{\pgfqpoint{3.011914in}{3.743439in}}%
\pgfpathlineto{\pgfqpoint{3.011914in}{3.739181in}}%
\pgfpathmoveto{\pgfqpoint{3.011914in}{3.734923in}}%
\pgfpathlineto{\pgfqpoint{3.011914in}{3.734923in}}%
\pgfpathlineto{\pgfqpoint{3.011914in}{3.739181in}}%
\pgfpathlineto{\pgfqpoint{3.016172in}{3.739181in}}%
\pgfpathlineto{\pgfqpoint{3.016172in}{3.734923in}}%
\pgfpathmoveto{\pgfqpoint{3.016172in}{3.726407in}}%
\pgfpathlineto{\pgfqpoint{3.016172in}{3.726407in}}%
\pgfpathlineto{\pgfqpoint{3.016172in}{3.730665in}}%
\pgfpathlineto{\pgfqpoint{3.020430in}{3.730665in}}%
\pgfpathlineto{\pgfqpoint{3.020430in}{3.726407in}}%
\pgfpathmoveto{\pgfqpoint{3.016172in}{3.730665in}}%
\pgfpathlineto{\pgfqpoint{3.016172in}{3.730665in}}%
\pgfpathlineto{\pgfqpoint{3.016172in}{3.734923in}}%
\pgfpathlineto{\pgfqpoint{3.020430in}{3.734923in}}%
\pgfpathlineto{\pgfqpoint{3.020430in}{3.730665in}}%
\pgfpathmoveto{\pgfqpoint{3.020430in}{3.726407in}}%
\pgfpathlineto{\pgfqpoint{3.020430in}{3.726407in}}%
\pgfpathlineto{\pgfqpoint{3.020430in}{3.730665in}}%
\pgfpathlineto{\pgfqpoint{3.024688in}{3.730665in}}%
\pgfpathlineto{\pgfqpoint{3.024688in}{3.726407in}}%
\pgfpathmoveto{\pgfqpoint{2.990624in}{3.743439in}}%
\pgfpathlineto{\pgfqpoint{2.990624in}{3.743439in}}%
\pgfpathlineto{\pgfqpoint{2.990624in}{3.747696in}}%
\pgfpathlineto{\pgfqpoint{2.994882in}{3.747696in}}%
\pgfpathlineto{\pgfqpoint{2.994882in}{3.743439in}}%
\pgfpathmoveto{\pgfqpoint{2.990624in}{3.747696in}}%
\pgfpathlineto{\pgfqpoint{2.990624in}{3.747696in}}%
\pgfpathlineto{\pgfqpoint{2.990624in}{3.751954in}}%
\pgfpathlineto{\pgfqpoint{2.994882in}{3.751954in}}%
\pgfpathlineto{\pgfqpoint{2.994882in}{3.747696in}}%
\pgfpathmoveto{\pgfqpoint{2.994882in}{3.743439in}}%
\pgfpathlineto{\pgfqpoint{2.994882in}{3.743439in}}%
\pgfpathlineto{\pgfqpoint{2.994882in}{3.747696in}}%
\pgfpathlineto{\pgfqpoint{2.999140in}{3.747696in}}%
\pgfpathlineto{\pgfqpoint{2.999140in}{3.743439in}}%
\pgfpathmoveto{\pgfqpoint{2.994882in}{3.747696in}}%
\pgfpathlineto{\pgfqpoint{2.994882in}{3.747696in}}%
\pgfpathlineto{\pgfqpoint{2.994882in}{3.751954in}}%
\pgfpathlineto{\pgfqpoint{2.999140in}{3.751954in}}%
\pgfpathlineto{\pgfqpoint{2.999140in}{3.747696in}}%
\pgfpathmoveto{\pgfqpoint{2.990624in}{3.751954in}}%
\pgfpathlineto{\pgfqpoint{2.990624in}{3.751954in}}%
\pgfpathlineto{\pgfqpoint{2.990624in}{3.756212in}}%
\pgfpathlineto{\pgfqpoint{2.994882in}{3.756212in}}%
\pgfpathlineto{\pgfqpoint{2.994882in}{3.751954in}}%
\pgfpathmoveto{\pgfqpoint{2.990624in}{3.756212in}}%
\pgfpathlineto{\pgfqpoint{2.990624in}{3.756212in}}%
\pgfpathlineto{\pgfqpoint{2.990624in}{3.760470in}}%
\pgfpathlineto{\pgfqpoint{2.994882in}{3.760470in}}%
\pgfpathlineto{\pgfqpoint{2.994882in}{3.756212in}}%
\pgfpathmoveto{\pgfqpoint{2.994882in}{3.751954in}}%
\pgfpathlineto{\pgfqpoint{2.994882in}{3.751954in}}%
\pgfpathlineto{\pgfqpoint{2.994882in}{3.756212in}}%
\pgfpathlineto{\pgfqpoint{2.999140in}{3.756212in}}%
\pgfpathlineto{\pgfqpoint{2.999140in}{3.751954in}}%
\pgfpathmoveto{\pgfqpoint{2.999140in}{3.743439in}}%
\pgfpathlineto{\pgfqpoint{2.999140in}{3.743439in}}%
\pgfpathlineto{\pgfqpoint{2.999140in}{3.747696in}}%
\pgfpathlineto{\pgfqpoint{3.003398in}{3.747696in}}%
\pgfpathlineto{\pgfqpoint{3.003398in}{3.743439in}}%
\pgfpathmoveto{\pgfqpoint{2.999140in}{3.747696in}}%
\pgfpathlineto{\pgfqpoint{2.999140in}{3.747696in}}%
\pgfpathlineto{\pgfqpoint{2.999140in}{3.751954in}}%
\pgfpathlineto{\pgfqpoint{3.003398in}{3.751954in}}%
\pgfpathlineto{\pgfqpoint{3.003398in}{3.747696in}}%
\pgfpathmoveto{\pgfqpoint{3.003398in}{3.743439in}}%
\pgfpathlineto{\pgfqpoint{3.003398in}{3.743439in}}%
\pgfpathlineto{\pgfqpoint{3.003398in}{3.747696in}}%
\pgfpathlineto{\pgfqpoint{3.007656in}{3.747696in}}%
\pgfpathlineto{\pgfqpoint{3.007656in}{3.743439in}}%
\pgfpathmoveto{\pgfqpoint{3.033203in}{3.671055in}}%
\pgfpathlineto{\pgfqpoint{3.033203in}{3.671055in}}%
\pgfpathlineto{\pgfqpoint{3.033203in}{3.675313in}}%
\pgfpathlineto{\pgfqpoint{3.037461in}{3.675313in}}%
\pgfpathlineto{\pgfqpoint{3.037461in}{3.671055in}}%
\pgfpathmoveto{\pgfqpoint{3.037461in}{3.666797in}}%
\pgfpathlineto{\pgfqpoint{3.037461in}{3.666797in}}%
\pgfpathlineto{\pgfqpoint{3.037461in}{3.671055in}}%
\pgfpathlineto{\pgfqpoint{3.041719in}{3.671055in}}%
\pgfpathlineto{\pgfqpoint{3.041719in}{3.666797in}}%
\pgfpathmoveto{\pgfqpoint{3.037461in}{3.671055in}}%
\pgfpathlineto{\pgfqpoint{3.037461in}{3.671055in}}%
\pgfpathlineto{\pgfqpoint{3.037461in}{3.675313in}}%
\pgfpathlineto{\pgfqpoint{3.041719in}{3.675313in}}%
\pgfpathlineto{\pgfqpoint{3.041719in}{3.671055in}}%
\pgfpathmoveto{\pgfqpoint{3.045977in}{3.654024in}}%
\pgfpathlineto{\pgfqpoint{3.045977in}{3.654024in}}%
\pgfpathlineto{\pgfqpoint{3.045977in}{3.658282in}}%
\pgfpathlineto{\pgfqpoint{3.050235in}{3.658282in}}%
\pgfpathlineto{\pgfqpoint{3.050235in}{3.654024in}}%
\pgfpathmoveto{\pgfqpoint{3.050235in}{3.645508in}}%
\pgfpathlineto{\pgfqpoint{3.050235in}{3.645508in}}%
\pgfpathlineto{\pgfqpoint{3.050235in}{3.649766in}}%
\pgfpathlineto{\pgfqpoint{3.054493in}{3.649766in}}%
\pgfpathlineto{\pgfqpoint{3.054493in}{3.645508in}}%
\pgfpathmoveto{\pgfqpoint{3.054493in}{3.641250in}}%
\pgfpathlineto{\pgfqpoint{3.054493in}{3.641250in}}%
\pgfpathlineto{\pgfqpoint{3.054493in}{3.645508in}}%
\pgfpathlineto{\pgfqpoint{3.058751in}{3.645508in}}%
\pgfpathlineto{\pgfqpoint{3.058751in}{3.641250in}}%
\pgfpathmoveto{\pgfqpoint{3.054493in}{3.645508in}}%
\pgfpathlineto{\pgfqpoint{3.054493in}{3.645508in}}%
\pgfpathlineto{\pgfqpoint{3.054493in}{3.649766in}}%
\pgfpathlineto{\pgfqpoint{3.058751in}{3.649766in}}%
\pgfpathlineto{\pgfqpoint{3.058751in}{3.645508in}}%
\pgfpathmoveto{\pgfqpoint{3.050235in}{3.649766in}}%
\pgfpathlineto{\pgfqpoint{3.050235in}{3.649766in}}%
\pgfpathlineto{\pgfqpoint{3.050235in}{3.654024in}}%
\pgfpathlineto{\pgfqpoint{3.054493in}{3.654024in}}%
\pgfpathlineto{\pgfqpoint{3.054493in}{3.649766in}}%
\pgfpathmoveto{\pgfqpoint{3.050235in}{3.654024in}}%
\pgfpathlineto{\pgfqpoint{3.050235in}{3.654024in}}%
\pgfpathlineto{\pgfqpoint{3.050235in}{3.658282in}}%
\pgfpathlineto{\pgfqpoint{3.054493in}{3.658282in}}%
\pgfpathlineto{\pgfqpoint{3.054493in}{3.654024in}}%
\pgfpathmoveto{\pgfqpoint{3.054493in}{3.649766in}}%
\pgfpathlineto{\pgfqpoint{3.054493in}{3.649766in}}%
\pgfpathlineto{\pgfqpoint{3.054493in}{3.654024in}}%
\pgfpathlineto{\pgfqpoint{3.058751in}{3.654024in}}%
\pgfpathlineto{\pgfqpoint{3.058751in}{3.649766in}}%
\pgfpathmoveto{\pgfqpoint{3.054493in}{3.654024in}}%
\pgfpathlineto{\pgfqpoint{3.054493in}{3.654024in}}%
\pgfpathlineto{\pgfqpoint{3.054493in}{3.658282in}}%
\pgfpathlineto{\pgfqpoint{3.058751in}{3.658282in}}%
\pgfpathlineto{\pgfqpoint{3.058751in}{3.654024in}}%
\pgfpathmoveto{\pgfqpoint{3.041719in}{3.658282in}}%
\pgfpathlineto{\pgfqpoint{3.041719in}{3.658282in}}%
\pgfpathlineto{\pgfqpoint{3.041719in}{3.662540in}}%
\pgfpathlineto{\pgfqpoint{3.045977in}{3.662540in}}%
\pgfpathlineto{\pgfqpoint{3.045977in}{3.658282in}}%
\pgfpathmoveto{\pgfqpoint{3.041719in}{3.662540in}}%
\pgfpathlineto{\pgfqpoint{3.041719in}{3.662540in}}%
\pgfpathlineto{\pgfqpoint{3.041719in}{3.666797in}}%
\pgfpathlineto{\pgfqpoint{3.045977in}{3.666797in}}%
\pgfpathlineto{\pgfqpoint{3.045977in}{3.662540in}}%
\pgfpathmoveto{\pgfqpoint{3.045977in}{3.658282in}}%
\pgfpathlineto{\pgfqpoint{3.045977in}{3.658282in}}%
\pgfpathlineto{\pgfqpoint{3.045977in}{3.662540in}}%
\pgfpathlineto{\pgfqpoint{3.050235in}{3.662540in}}%
\pgfpathlineto{\pgfqpoint{3.050235in}{3.658282in}}%
\pgfpathmoveto{\pgfqpoint{3.045977in}{3.662540in}}%
\pgfpathlineto{\pgfqpoint{3.045977in}{3.662540in}}%
\pgfpathlineto{\pgfqpoint{3.045977in}{3.666797in}}%
\pgfpathlineto{\pgfqpoint{3.050235in}{3.666797in}}%
\pgfpathlineto{\pgfqpoint{3.050235in}{3.662540in}}%
\pgfpathmoveto{\pgfqpoint{3.041719in}{3.666797in}}%
\pgfpathlineto{\pgfqpoint{3.041719in}{3.666797in}}%
\pgfpathlineto{\pgfqpoint{3.041719in}{3.671055in}}%
\pgfpathlineto{\pgfqpoint{3.045977in}{3.671055in}}%
\pgfpathlineto{\pgfqpoint{3.045977in}{3.666797in}}%
\pgfpathmoveto{\pgfqpoint{3.041719in}{3.671055in}}%
\pgfpathlineto{\pgfqpoint{3.041719in}{3.671055in}}%
\pgfpathlineto{\pgfqpoint{3.041719in}{3.675313in}}%
\pgfpathlineto{\pgfqpoint{3.045977in}{3.675313in}}%
\pgfpathlineto{\pgfqpoint{3.045977in}{3.671055in}}%
\pgfpathmoveto{\pgfqpoint{3.045977in}{3.666797in}}%
\pgfpathlineto{\pgfqpoint{3.045977in}{3.666797in}}%
\pgfpathlineto{\pgfqpoint{3.045977in}{3.671055in}}%
\pgfpathlineto{\pgfqpoint{3.050235in}{3.671055in}}%
\pgfpathlineto{\pgfqpoint{3.050235in}{3.666797in}}%
\pgfpathmoveto{\pgfqpoint{3.045977in}{3.671055in}}%
\pgfpathlineto{\pgfqpoint{3.045977in}{3.671055in}}%
\pgfpathlineto{\pgfqpoint{3.045977in}{3.675313in}}%
\pgfpathlineto{\pgfqpoint{3.050235in}{3.675313in}}%
\pgfpathlineto{\pgfqpoint{3.050235in}{3.671055in}}%
\pgfpathmoveto{\pgfqpoint{3.050235in}{3.658282in}}%
\pgfpathlineto{\pgfqpoint{3.050235in}{3.658282in}}%
\pgfpathlineto{\pgfqpoint{3.050235in}{3.662540in}}%
\pgfpathlineto{\pgfqpoint{3.054493in}{3.662540in}}%
\pgfpathlineto{\pgfqpoint{3.054493in}{3.658282in}}%
\pgfpathmoveto{\pgfqpoint{3.050235in}{3.662540in}}%
\pgfpathlineto{\pgfqpoint{3.050235in}{3.662540in}}%
\pgfpathlineto{\pgfqpoint{3.050235in}{3.666797in}}%
\pgfpathlineto{\pgfqpoint{3.054493in}{3.666797in}}%
\pgfpathlineto{\pgfqpoint{3.054493in}{3.662540in}}%
\pgfpathmoveto{\pgfqpoint{3.054493in}{3.658282in}}%
\pgfpathlineto{\pgfqpoint{3.054493in}{3.658282in}}%
\pgfpathlineto{\pgfqpoint{3.054493in}{3.662540in}}%
\pgfpathlineto{\pgfqpoint{3.058751in}{3.662540in}}%
\pgfpathlineto{\pgfqpoint{3.058751in}{3.658282in}}%
\pgfpathmoveto{\pgfqpoint{3.054493in}{3.662540in}}%
\pgfpathlineto{\pgfqpoint{3.054493in}{3.662540in}}%
\pgfpathlineto{\pgfqpoint{3.054493in}{3.666797in}}%
\pgfpathlineto{\pgfqpoint{3.058751in}{3.666797in}}%
\pgfpathlineto{\pgfqpoint{3.058751in}{3.662540in}}%
\pgfpathmoveto{\pgfqpoint{3.050235in}{3.666797in}}%
\pgfpathlineto{\pgfqpoint{3.050235in}{3.666797in}}%
\pgfpathlineto{\pgfqpoint{3.050235in}{3.671055in}}%
\pgfpathlineto{\pgfqpoint{3.054493in}{3.671055in}}%
\pgfpathlineto{\pgfqpoint{3.054493in}{3.666797in}}%
\pgfpathmoveto{\pgfqpoint{3.050235in}{3.671055in}}%
\pgfpathlineto{\pgfqpoint{3.050235in}{3.671055in}}%
\pgfpathlineto{\pgfqpoint{3.050235in}{3.675313in}}%
\pgfpathlineto{\pgfqpoint{3.054493in}{3.675313in}}%
\pgfpathlineto{\pgfqpoint{3.054493in}{3.671055in}}%
\pgfpathmoveto{\pgfqpoint{3.054493in}{3.666797in}}%
\pgfpathlineto{\pgfqpoint{3.054493in}{3.666797in}}%
\pgfpathlineto{\pgfqpoint{3.054493in}{3.671055in}}%
\pgfpathlineto{\pgfqpoint{3.058751in}{3.671055in}}%
\pgfpathlineto{\pgfqpoint{3.058751in}{3.666797in}}%
\pgfpathmoveto{\pgfqpoint{3.054493in}{3.671055in}}%
\pgfpathlineto{\pgfqpoint{3.054493in}{3.671055in}}%
\pgfpathlineto{\pgfqpoint{3.054493in}{3.675313in}}%
\pgfpathlineto{\pgfqpoint{3.058751in}{3.675313in}}%
\pgfpathlineto{\pgfqpoint{3.058751in}{3.671055in}}%
\pgfpathmoveto{\pgfqpoint{3.028946in}{3.679571in}}%
\pgfpathlineto{\pgfqpoint{3.028946in}{3.679571in}}%
\pgfpathlineto{\pgfqpoint{3.028946in}{3.683829in}}%
\pgfpathlineto{\pgfqpoint{3.033203in}{3.683829in}}%
\pgfpathlineto{\pgfqpoint{3.033203in}{3.679571in}}%
\pgfpathmoveto{\pgfqpoint{3.024688in}{3.683829in}}%
\pgfpathlineto{\pgfqpoint{3.024688in}{3.683829in}}%
\pgfpathlineto{\pgfqpoint{3.024688in}{3.688087in}}%
\pgfpathlineto{\pgfqpoint{3.028946in}{3.688087in}}%
\pgfpathlineto{\pgfqpoint{3.028946in}{3.683829in}}%
\pgfpathmoveto{\pgfqpoint{3.024688in}{3.688087in}}%
\pgfpathlineto{\pgfqpoint{3.024688in}{3.688087in}}%
\pgfpathlineto{\pgfqpoint{3.024688in}{3.692345in}}%
\pgfpathlineto{\pgfqpoint{3.028946in}{3.692345in}}%
\pgfpathlineto{\pgfqpoint{3.028946in}{3.688087in}}%
\pgfpathmoveto{\pgfqpoint{3.028946in}{3.683829in}}%
\pgfpathlineto{\pgfqpoint{3.028946in}{3.683829in}}%
\pgfpathlineto{\pgfqpoint{3.028946in}{3.688087in}}%
\pgfpathlineto{\pgfqpoint{3.033203in}{3.688087in}}%
\pgfpathlineto{\pgfqpoint{3.033203in}{3.683829in}}%
\pgfpathmoveto{\pgfqpoint{3.028946in}{3.688087in}}%
\pgfpathlineto{\pgfqpoint{3.028946in}{3.688087in}}%
\pgfpathlineto{\pgfqpoint{3.028946in}{3.692345in}}%
\pgfpathlineto{\pgfqpoint{3.033203in}{3.692345in}}%
\pgfpathlineto{\pgfqpoint{3.033203in}{3.688087in}}%
\pgfpathmoveto{\pgfqpoint{3.033203in}{3.675313in}}%
\pgfpathlineto{\pgfqpoint{3.033203in}{3.675313in}}%
\pgfpathlineto{\pgfqpoint{3.033203in}{3.679571in}}%
\pgfpathlineto{\pgfqpoint{3.037461in}{3.679571in}}%
\pgfpathlineto{\pgfqpoint{3.037461in}{3.675313in}}%
\pgfpathmoveto{\pgfqpoint{3.033203in}{3.679571in}}%
\pgfpathlineto{\pgfqpoint{3.033203in}{3.679571in}}%
\pgfpathlineto{\pgfqpoint{3.033203in}{3.683829in}}%
\pgfpathlineto{\pgfqpoint{3.037461in}{3.683829in}}%
\pgfpathlineto{\pgfqpoint{3.037461in}{3.679571in}}%
\pgfpathmoveto{\pgfqpoint{3.037461in}{3.675313in}}%
\pgfpathlineto{\pgfqpoint{3.037461in}{3.675313in}}%
\pgfpathlineto{\pgfqpoint{3.037461in}{3.679571in}}%
\pgfpathlineto{\pgfqpoint{3.041719in}{3.679571in}}%
\pgfpathlineto{\pgfqpoint{3.041719in}{3.675313in}}%
\pgfpathmoveto{\pgfqpoint{3.037461in}{3.679571in}}%
\pgfpathlineto{\pgfqpoint{3.037461in}{3.679571in}}%
\pgfpathlineto{\pgfqpoint{3.037461in}{3.683829in}}%
\pgfpathlineto{\pgfqpoint{3.041719in}{3.683829in}}%
\pgfpathlineto{\pgfqpoint{3.041719in}{3.679571in}}%
\pgfpathmoveto{\pgfqpoint{3.033203in}{3.683829in}}%
\pgfpathlineto{\pgfqpoint{3.033203in}{3.683829in}}%
\pgfpathlineto{\pgfqpoint{3.033203in}{3.688087in}}%
\pgfpathlineto{\pgfqpoint{3.037461in}{3.688087in}}%
\pgfpathlineto{\pgfqpoint{3.037461in}{3.683829in}}%
\pgfpathmoveto{\pgfqpoint{3.033203in}{3.688087in}}%
\pgfpathlineto{\pgfqpoint{3.033203in}{3.688087in}}%
\pgfpathlineto{\pgfqpoint{3.033203in}{3.692345in}}%
\pgfpathlineto{\pgfqpoint{3.037461in}{3.692345in}}%
\pgfpathlineto{\pgfqpoint{3.037461in}{3.688087in}}%
\pgfpathmoveto{\pgfqpoint{3.037461in}{3.683829in}}%
\pgfpathlineto{\pgfqpoint{3.037461in}{3.683829in}}%
\pgfpathlineto{\pgfqpoint{3.037461in}{3.688087in}}%
\pgfpathlineto{\pgfqpoint{3.041719in}{3.688087in}}%
\pgfpathlineto{\pgfqpoint{3.041719in}{3.683829in}}%
\pgfpathmoveto{\pgfqpoint{3.037461in}{3.688087in}}%
\pgfpathlineto{\pgfqpoint{3.037461in}{3.688087in}}%
\pgfpathlineto{\pgfqpoint{3.037461in}{3.692345in}}%
\pgfpathlineto{\pgfqpoint{3.041719in}{3.692345in}}%
\pgfpathlineto{\pgfqpoint{3.041719in}{3.688087in}}%
\pgfpathmoveto{\pgfqpoint{3.024688in}{3.692345in}}%
\pgfpathlineto{\pgfqpoint{3.024688in}{3.692345in}}%
\pgfpathlineto{\pgfqpoint{3.024688in}{3.696602in}}%
\pgfpathlineto{\pgfqpoint{3.028946in}{3.696602in}}%
\pgfpathlineto{\pgfqpoint{3.028946in}{3.692345in}}%
\pgfpathmoveto{\pgfqpoint{3.024688in}{3.696602in}}%
\pgfpathlineto{\pgfqpoint{3.024688in}{3.696602in}}%
\pgfpathlineto{\pgfqpoint{3.024688in}{3.700860in}}%
\pgfpathlineto{\pgfqpoint{3.028946in}{3.700860in}}%
\pgfpathlineto{\pgfqpoint{3.028946in}{3.696602in}}%
\pgfpathmoveto{\pgfqpoint{3.028946in}{3.692345in}}%
\pgfpathlineto{\pgfqpoint{3.028946in}{3.692345in}}%
\pgfpathlineto{\pgfqpoint{3.028946in}{3.696602in}}%
\pgfpathlineto{\pgfqpoint{3.033203in}{3.696602in}}%
\pgfpathlineto{\pgfqpoint{3.033203in}{3.692345in}}%
\pgfpathmoveto{\pgfqpoint{3.028946in}{3.696602in}}%
\pgfpathlineto{\pgfqpoint{3.028946in}{3.696602in}}%
\pgfpathlineto{\pgfqpoint{3.028946in}{3.700860in}}%
\pgfpathlineto{\pgfqpoint{3.033203in}{3.700860in}}%
\pgfpathlineto{\pgfqpoint{3.033203in}{3.696602in}}%
\pgfpathmoveto{\pgfqpoint{3.024688in}{3.700860in}}%
\pgfpathlineto{\pgfqpoint{3.024688in}{3.700860in}}%
\pgfpathlineto{\pgfqpoint{3.024688in}{3.705118in}}%
\pgfpathlineto{\pgfqpoint{3.028946in}{3.705118in}}%
\pgfpathlineto{\pgfqpoint{3.028946in}{3.700860in}}%
\pgfpathmoveto{\pgfqpoint{3.024688in}{3.705118in}}%
\pgfpathlineto{\pgfqpoint{3.024688in}{3.705118in}}%
\pgfpathlineto{\pgfqpoint{3.024688in}{3.709376in}}%
\pgfpathlineto{\pgfqpoint{3.028946in}{3.709376in}}%
\pgfpathlineto{\pgfqpoint{3.028946in}{3.705118in}}%
\pgfpathmoveto{\pgfqpoint{3.028946in}{3.700860in}}%
\pgfpathlineto{\pgfqpoint{3.028946in}{3.700860in}}%
\pgfpathlineto{\pgfqpoint{3.028946in}{3.705118in}}%
\pgfpathlineto{\pgfqpoint{3.033203in}{3.705118in}}%
\pgfpathlineto{\pgfqpoint{3.033203in}{3.700860in}}%
\pgfpathmoveto{\pgfqpoint{3.028946in}{3.705118in}}%
\pgfpathlineto{\pgfqpoint{3.028946in}{3.705118in}}%
\pgfpathlineto{\pgfqpoint{3.028946in}{3.709376in}}%
\pgfpathlineto{\pgfqpoint{3.033203in}{3.709376in}}%
\pgfpathlineto{\pgfqpoint{3.033203in}{3.705118in}}%
\pgfpathmoveto{\pgfqpoint{3.033203in}{3.692345in}}%
\pgfpathlineto{\pgfqpoint{3.033203in}{3.692345in}}%
\pgfpathlineto{\pgfqpoint{3.033203in}{3.696602in}}%
\pgfpathlineto{\pgfqpoint{3.037461in}{3.696602in}}%
\pgfpathlineto{\pgfqpoint{3.037461in}{3.692345in}}%
\pgfpathmoveto{\pgfqpoint{3.033203in}{3.696602in}}%
\pgfpathlineto{\pgfqpoint{3.033203in}{3.696602in}}%
\pgfpathlineto{\pgfqpoint{3.033203in}{3.700860in}}%
\pgfpathlineto{\pgfqpoint{3.037461in}{3.700860in}}%
\pgfpathlineto{\pgfqpoint{3.037461in}{3.696602in}}%
\pgfpathmoveto{\pgfqpoint{3.037461in}{3.692345in}}%
\pgfpathlineto{\pgfqpoint{3.037461in}{3.692345in}}%
\pgfpathlineto{\pgfqpoint{3.037461in}{3.696602in}}%
\pgfpathlineto{\pgfqpoint{3.041719in}{3.696602in}}%
\pgfpathlineto{\pgfqpoint{3.041719in}{3.692345in}}%
\pgfpathmoveto{\pgfqpoint{3.037461in}{3.696602in}}%
\pgfpathlineto{\pgfqpoint{3.037461in}{3.696602in}}%
\pgfpathlineto{\pgfqpoint{3.037461in}{3.700860in}}%
\pgfpathlineto{\pgfqpoint{3.041719in}{3.700860in}}%
\pgfpathlineto{\pgfqpoint{3.041719in}{3.696602in}}%
\pgfpathmoveto{\pgfqpoint{3.033203in}{3.700860in}}%
\pgfpathlineto{\pgfqpoint{3.033203in}{3.700860in}}%
\pgfpathlineto{\pgfqpoint{3.033203in}{3.705118in}}%
\pgfpathlineto{\pgfqpoint{3.037461in}{3.705118in}}%
\pgfpathlineto{\pgfqpoint{3.037461in}{3.700860in}}%
\pgfpathmoveto{\pgfqpoint{3.033203in}{3.705118in}}%
\pgfpathlineto{\pgfqpoint{3.033203in}{3.705118in}}%
\pgfpathlineto{\pgfqpoint{3.033203in}{3.709376in}}%
\pgfpathlineto{\pgfqpoint{3.037461in}{3.709376in}}%
\pgfpathlineto{\pgfqpoint{3.037461in}{3.705118in}}%
\pgfpathmoveto{\pgfqpoint{3.037461in}{3.700860in}}%
\pgfpathlineto{\pgfqpoint{3.037461in}{3.700860in}}%
\pgfpathlineto{\pgfqpoint{3.037461in}{3.705118in}}%
\pgfpathlineto{\pgfqpoint{3.041719in}{3.705118in}}%
\pgfpathlineto{\pgfqpoint{3.041719in}{3.700860in}}%
\pgfpathmoveto{\pgfqpoint{3.041719in}{3.675313in}}%
\pgfpathlineto{\pgfqpoint{3.041719in}{3.675313in}}%
\pgfpathlineto{\pgfqpoint{3.041719in}{3.679571in}}%
\pgfpathlineto{\pgfqpoint{3.045977in}{3.679571in}}%
\pgfpathlineto{\pgfqpoint{3.045977in}{3.675313in}}%
\pgfpathmoveto{\pgfqpoint{3.041719in}{3.679571in}}%
\pgfpathlineto{\pgfqpoint{3.041719in}{3.679571in}}%
\pgfpathlineto{\pgfqpoint{3.041719in}{3.683829in}}%
\pgfpathlineto{\pgfqpoint{3.045977in}{3.683829in}}%
\pgfpathlineto{\pgfqpoint{3.045977in}{3.679571in}}%
\pgfpathmoveto{\pgfqpoint{3.045977in}{3.675313in}}%
\pgfpathlineto{\pgfqpoint{3.045977in}{3.675313in}}%
\pgfpathlineto{\pgfqpoint{3.045977in}{3.679571in}}%
\pgfpathlineto{\pgfqpoint{3.050235in}{3.679571in}}%
\pgfpathlineto{\pgfqpoint{3.050235in}{3.675313in}}%
\pgfpathmoveto{\pgfqpoint{3.045977in}{3.679571in}}%
\pgfpathlineto{\pgfqpoint{3.045977in}{3.679571in}}%
\pgfpathlineto{\pgfqpoint{3.045977in}{3.683829in}}%
\pgfpathlineto{\pgfqpoint{3.050235in}{3.683829in}}%
\pgfpathlineto{\pgfqpoint{3.050235in}{3.679571in}}%
\pgfpathmoveto{\pgfqpoint{3.041719in}{3.683829in}}%
\pgfpathlineto{\pgfqpoint{3.041719in}{3.683829in}}%
\pgfpathlineto{\pgfqpoint{3.041719in}{3.688087in}}%
\pgfpathlineto{\pgfqpoint{3.045977in}{3.688087in}}%
\pgfpathlineto{\pgfqpoint{3.045977in}{3.683829in}}%
\pgfpathmoveto{\pgfqpoint{3.041719in}{3.688087in}}%
\pgfpathlineto{\pgfqpoint{3.041719in}{3.688087in}}%
\pgfpathlineto{\pgfqpoint{3.041719in}{3.692345in}}%
\pgfpathlineto{\pgfqpoint{3.045977in}{3.692345in}}%
\pgfpathlineto{\pgfqpoint{3.045977in}{3.688087in}}%
\pgfpathmoveto{\pgfqpoint{3.045977in}{3.683829in}}%
\pgfpathlineto{\pgfqpoint{3.045977in}{3.683829in}}%
\pgfpathlineto{\pgfqpoint{3.045977in}{3.688087in}}%
\pgfpathlineto{\pgfqpoint{3.050235in}{3.688087in}}%
\pgfpathlineto{\pgfqpoint{3.050235in}{3.683829in}}%
\pgfpathmoveto{\pgfqpoint{3.045977in}{3.688087in}}%
\pgfpathlineto{\pgfqpoint{3.045977in}{3.688087in}}%
\pgfpathlineto{\pgfqpoint{3.045977in}{3.692345in}}%
\pgfpathlineto{\pgfqpoint{3.050235in}{3.692345in}}%
\pgfpathlineto{\pgfqpoint{3.050235in}{3.688087in}}%
\pgfpathmoveto{\pgfqpoint{3.050235in}{3.675313in}}%
\pgfpathlineto{\pgfqpoint{3.050235in}{3.675313in}}%
\pgfpathlineto{\pgfqpoint{3.050235in}{3.679571in}}%
\pgfpathlineto{\pgfqpoint{3.054493in}{3.679571in}}%
\pgfpathlineto{\pgfqpoint{3.054493in}{3.675313in}}%
\pgfpathmoveto{\pgfqpoint{3.050235in}{3.679571in}}%
\pgfpathlineto{\pgfqpoint{3.050235in}{3.679571in}}%
\pgfpathlineto{\pgfqpoint{3.050235in}{3.683829in}}%
\pgfpathlineto{\pgfqpoint{3.054493in}{3.683829in}}%
\pgfpathlineto{\pgfqpoint{3.054493in}{3.679571in}}%
\pgfpathmoveto{\pgfqpoint{3.054493in}{3.675313in}}%
\pgfpathlineto{\pgfqpoint{3.054493in}{3.675313in}}%
\pgfpathlineto{\pgfqpoint{3.054493in}{3.679571in}}%
\pgfpathlineto{\pgfqpoint{3.058751in}{3.679571in}}%
\pgfpathlineto{\pgfqpoint{3.058751in}{3.675313in}}%
\pgfpathmoveto{\pgfqpoint{3.041719in}{3.692345in}}%
\pgfpathlineto{\pgfqpoint{3.041719in}{3.692345in}}%
\pgfpathlineto{\pgfqpoint{3.041719in}{3.696602in}}%
\pgfpathlineto{\pgfqpoint{3.045977in}{3.696602in}}%
\pgfpathlineto{\pgfqpoint{3.045977in}{3.692345in}}%
\pgfpathmoveto{\pgfqpoint{3.041719in}{3.696602in}}%
\pgfpathlineto{\pgfqpoint{3.041719in}{3.696602in}}%
\pgfpathlineto{\pgfqpoint{3.041719in}{3.700860in}}%
\pgfpathlineto{\pgfqpoint{3.045977in}{3.700860in}}%
\pgfpathlineto{\pgfqpoint{3.045977in}{3.696602in}}%
\pgfpathmoveto{\pgfqpoint{3.058751in}{3.641250in}}%
\pgfpathlineto{\pgfqpoint{3.058751in}{3.641250in}}%
\pgfpathlineto{\pgfqpoint{3.058751in}{3.645508in}}%
\pgfpathlineto{\pgfqpoint{3.063009in}{3.645508in}}%
\pgfpathlineto{\pgfqpoint{3.063009in}{3.641250in}}%
\pgfpathmoveto{\pgfqpoint{3.058751in}{3.645508in}}%
\pgfpathlineto{\pgfqpoint{3.058751in}{3.645508in}}%
\pgfpathlineto{\pgfqpoint{3.058751in}{3.649766in}}%
\pgfpathlineto{\pgfqpoint{3.063009in}{3.649766in}}%
\pgfpathlineto{\pgfqpoint{3.063009in}{3.645508in}}%
\pgfpathmoveto{\pgfqpoint{3.063009in}{3.641250in}}%
\pgfpathlineto{\pgfqpoint{3.063009in}{3.641250in}}%
\pgfpathlineto{\pgfqpoint{3.063009in}{3.645508in}}%
\pgfpathlineto{\pgfqpoint{3.067267in}{3.645508in}}%
\pgfpathlineto{\pgfqpoint{3.067267in}{3.641250in}}%
\pgfpathmoveto{\pgfqpoint{3.063009in}{3.645508in}}%
\pgfpathlineto{\pgfqpoint{3.063009in}{3.645508in}}%
\pgfpathlineto{\pgfqpoint{3.063009in}{3.649766in}}%
\pgfpathlineto{\pgfqpoint{3.067267in}{3.649766in}}%
\pgfpathlineto{\pgfqpoint{3.067267in}{3.645508in}}%
\pgfpathmoveto{\pgfqpoint{3.058751in}{3.649766in}}%
\pgfpathlineto{\pgfqpoint{3.058751in}{3.649766in}}%
\pgfpathlineto{\pgfqpoint{3.058751in}{3.654024in}}%
\pgfpathlineto{\pgfqpoint{3.063009in}{3.654024in}}%
\pgfpathlineto{\pgfqpoint{3.063009in}{3.649766in}}%
\pgfpathmoveto{\pgfqpoint{3.058751in}{3.654024in}}%
\pgfpathlineto{\pgfqpoint{3.058751in}{3.654024in}}%
\pgfpathlineto{\pgfqpoint{3.058751in}{3.658282in}}%
\pgfpathlineto{\pgfqpoint{3.063009in}{3.658282in}}%
\pgfpathlineto{\pgfqpoint{3.063009in}{3.654024in}}%
\pgfpathmoveto{\pgfqpoint{3.063009in}{3.649766in}}%
\pgfpathlineto{\pgfqpoint{3.063009in}{3.649766in}}%
\pgfpathlineto{\pgfqpoint{3.063009in}{3.654024in}}%
\pgfpathlineto{\pgfqpoint{3.067267in}{3.654024in}}%
\pgfpathlineto{\pgfqpoint{3.067267in}{3.649766in}}%
\pgfpathmoveto{\pgfqpoint{3.063009in}{3.654024in}}%
\pgfpathlineto{\pgfqpoint{3.063009in}{3.654024in}}%
\pgfpathlineto{\pgfqpoint{3.063009in}{3.658282in}}%
\pgfpathlineto{\pgfqpoint{3.067267in}{3.658282in}}%
\pgfpathlineto{\pgfqpoint{3.067267in}{3.654024in}}%
\pgfpathmoveto{\pgfqpoint{3.067267in}{3.641250in}}%
\pgfpathlineto{\pgfqpoint{3.067267in}{3.641250in}}%
\pgfpathlineto{\pgfqpoint{3.067267in}{3.645508in}}%
\pgfpathlineto{\pgfqpoint{3.071524in}{3.645508in}}%
\pgfpathlineto{\pgfqpoint{3.071524in}{3.641250in}}%
\pgfpathmoveto{\pgfqpoint{3.067267in}{3.645508in}}%
\pgfpathlineto{\pgfqpoint{3.067267in}{3.645508in}}%
\pgfpathlineto{\pgfqpoint{3.067267in}{3.649766in}}%
\pgfpathlineto{\pgfqpoint{3.071524in}{3.649766in}}%
\pgfpathlineto{\pgfqpoint{3.071524in}{3.645508in}}%
\pgfpathmoveto{\pgfqpoint{3.067267in}{3.649766in}}%
\pgfpathlineto{\pgfqpoint{3.067267in}{3.649766in}}%
\pgfpathlineto{\pgfqpoint{3.067267in}{3.654024in}}%
\pgfpathlineto{\pgfqpoint{3.071524in}{3.654024in}}%
\pgfpathlineto{\pgfqpoint{3.071524in}{3.649766in}}%
\pgfpathmoveto{\pgfqpoint{3.058751in}{3.658282in}}%
\pgfpathlineto{\pgfqpoint{3.058751in}{3.658282in}}%
\pgfpathlineto{\pgfqpoint{3.058751in}{3.662540in}}%
\pgfpathlineto{\pgfqpoint{3.063009in}{3.662540in}}%
\pgfpathlineto{\pgfqpoint{3.063009in}{3.658282in}}%
\pgfpathmoveto{\pgfqpoint{3.058751in}{3.662540in}}%
\pgfpathlineto{\pgfqpoint{3.058751in}{3.662540in}}%
\pgfpathlineto{\pgfqpoint{3.058751in}{3.666797in}}%
\pgfpathlineto{\pgfqpoint{3.063009in}{3.666797in}}%
\pgfpathlineto{\pgfqpoint{3.063009in}{3.662540in}}%
\pgfpathmoveto{\pgfqpoint{3.063009in}{3.658282in}}%
\pgfpathlineto{\pgfqpoint{3.063009in}{3.658282in}}%
\pgfpathlineto{\pgfqpoint{3.063009in}{3.662540in}}%
\pgfpathlineto{\pgfqpoint{3.067267in}{3.662540in}}%
\pgfpathlineto{\pgfqpoint{3.067267in}{3.658282in}}%
\pgfpathmoveto{\pgfqpoint{3.058751in}{3.666797in}}%
\pgfpathlineto{\pgfqpoint{3.058751in}{3.666797in}}%
\pgfpathlineto{\pgfqpoint{3.058751in}{3.671055in}}%
\pgfpathlineto{\pgfqpoint{3.063009in}{3.671055in}}%
\pgfpathlineto{\pgfqpoint{3.063009in}{3.666797in}}%
\pgfpathmoveto{\pgfqpoint{3.024688in}{3.709376in}}%
\pgfpathlineto{\pgfqpoint{3.024688in}{3.709376in}}%
\pgfpathlineto{\pgfqpoint{3.024688in}{3.713634in}}%
\pgfpathlineto{\pgfqpoint{3.028946in}{3.713634in}}%
\pgfpathlineto{\pgfqpoint{3.028946in}{3.709376in}}%
\pgfpathmoveto{\pgfqpoint{3.024688in}{3.713634in}}%
\pgfpathlineto{\pgfqpoint{3.024688in}{3.713634in}}%
\pgfpathlineto{\pgfqpoint{3.024688in}{3.717892in}}%
\pgfpathlineto{\pgfqpoint{3.028946in}{3.717892in}}%
\pgfpathlineto{\pgfqpoint{3.028946in}{3.713634in}}%
\pgfpathmoveto{\pgfqpoint{3.028946in}{3.709376in}}%
\pgfpathlineto{\pgfqpoint{3.028946in}{3.709376in}}%
\pgfpathlineto{\pgfqpoint{3.028946in}{3.713634in}}%
\pgfpathlineto{\pgfqpoint{3.033203in}{3.713634in}}%
\pgfpathlineto{\pgfqpoint{3.033203in}{3.709376in}}%
\pgfpathmoveto{\pgfqpoint{3.028946in}{3.713634in}}%
\pgfpathlineto{\pgfqpoint{3.028946in}{3.713634in}}%
\pgfpathlineto{\pgfqpoint{3.028946in}{3.717892in}}%
\pgfpathlineto{\pgfqpoint{3.033203in}{3.717892in}}%
\pgfpathlineto{\pgfqpoint{3.033203in}{3.713634in}}%
\pgfpathmoveto{\pgfqpoint{3.024688in}{3.717892in}}%
\pgfpathlineto{\pgfqpoint{3.024688in}{3.717892in}}%
\pgfpathlineto{\pgfqpoint{3.024688in}{3.722149in}}%
\pgfpathlineto{\pgfqpoint{3.028946in}{3.722149in}}%
\pgfpathlineto{\pgfqpoint{3.028946in}{3.717892in}}%
\pgfpathmoveto{\pgfqpoint{3.024688in}{3.722149in}}%
\pgfpathlineto{\pgfqpoint{3.024688in}{3.722149in}}%
\pgfpathlineto{\pgfqpoint{3.024688in}{3.726407in}}%
\pgfpathlineto{\pgfqpoint{3.028946in}{3.726407in}}%
\pgfpathlineto{\pgfqpoint{3.028946in}{3.722149in}}%
\pgfpathmoveto{\pgfqpoint{3.033203in}{3.709376in}}%
\pgfpathlineto{\pgfqpoint{3.033203in}{3.709376in}}%
\pgfpathlineto{\pgfqpoint{3.033203in}{3.713634in}}%
\pgfpathlineto{\pgfqpoint{3.037461in}{3.713634in}}%
\pgfpathlineto{\pgfqpoint{3.037461in}{3.709376in}}%
\pgfpathmoveto{\pgfqpoint{3.224804in}{3.066446in}}%
\pgfpathlineto{\pgfqpoint{3.224804in}{3.066446in}}%
\pgfpathlineto{\pgfqpoint{3.224804in}{3.070703in}}%
\pgfpathlineto{\pgfqpoint{3.229061in}{3.070703in}}%
\pgfpathlineto{\pgfqpoint{3.229061in}{3.066446in}}%
\pgfpathmoveto{\pgfqpoint{3.224804in}{3.070703in}}%
\pgfpathlineto{\pgfqpoint{3.224804in}{3.070703in}}%
\pgfpathlineto{\pgfqpoint{3.224804in}{3.074961in}}%
\pgfpathlineto{\pgfqpoint{3.229061in}{3.074961in}}%
\pgfpathlineto{\pgfqpoint{3.229061in}{3.070703in}}%
\pgfpathmoveto{\pgfqpoint{3.224804in}{3.074961in}}%
\pgfpathlineto{\pgfqpoint{3.224804in}{3.074961in}}%
\pgfpathlineto{\pgfqpoint{3.224804in}{3.079219in}}%
\pgfpathlineto{\pgfqpoint{3.229061in}{3.079219in}}%
\pgfpathlineto{\pgfqpoint{3.229061in}{3.074961in}}%
\pgfpathmoveto{\pgfqpoint{3.224804in}{3.079219in}}%
\pgfpathlineto{\pgfqpoint{3.224804in}{3.079219in}}%
\pgfpathlineto{\pgfqpoint{3.224804in}{3.083477in}}%
\pgfpathlineto{\pgfqpoint{3.229061in}{3.083477in}}%
\pgfpathlineto{\pgfqpoint{3.229061in}{3.079219in}}%
\pgfpathmoveto{\pgfqpoint{3.224804in}{3.083477in}}%
\pgfpathlineto{\pgfqpoint{3.224804in}{3.083477in}}%
\pgfpathlineto{\pgfqpoint{3.224804in}{3.087735in}}%
\pgfpathlineto{\pgfqpoint{3.229061in}{3.087735in}}%
\pgfpathlineto{\pgfqpoint{3.229061in}{3.083477in}}%
\pgfpathmoveto{\pgfqpoint{3.220546in}{3.087735in}}%
\pgfpathlineto{\pgfqpoint{3.220546in}{3.087735in}}%
\pgfpathlineto{\pgfqpoint{3.220546in}{3.091993in}}%
\pgfpathlineto{\pgfqpoint{3.224804in}{3.091993in}}%
\pgfpathlineto{\pgfqpoint{3.224804in}{3.087735in}}%
\pgfpathmoveto{\pgfqpoint{3.220546in}{3.091993in}}%
\pgfpathlineto{\pgfqpoint{3.220546in}{3.091993in}}%
\pgfpathlineto{\pgfqpoint{3.220546in}{3.096251in}}%
\pgfpathlineto{\pgfqpoint{3.224804in}{3.096251in}}%
\pgfpathlineto{\pgfqpoint{3.224804in}{3.091993in}}%
\pgfpathmoveto{\pgfqpoint{3.224804in}{3.087735in}}%
\pgfpathlineto{\pgfqpoint{3.224804in}{3.087735in}}%
\pgfpathlineto{\pgfqpoint{3.224804in}{3.091993in}}%
\pgfpathlineto{\pgfqpoint{3.229061in}{3.091993in}}%
\pgfpathlineto{\pgfqpoint{3.229061in}{3.087735in}}%
\pgfpathmoveto{\pgfqpoint{3.216288in}{3.109024in}}%
\pgfpathlineto{\pgfqpoint{3.216288in}{3.109024in}}%
\pgfpathlineto{\pgfqpoint{3.216288in}{3.113282in}}%
\pgfpathlineto{\pgfqpoint{3.220546in}{3.113282in}}%
\pgfpathlineto{\pgfqpoint{3.220546in}{3.109024in}}%
\pgfpathmoveto{\pgfqpoint{3.220546in}{3.096251in}}%
\pgfpathlineto{\pgfqpoint{3.220546in}{3.096251in}}%
\pgfpathlineto{\pgfqpoint{3.220546in}{3.100509in}}%
\pgfpathlineto{\pgfqpoint{3.224804in}{3.100509in}}%
\pgfpathlineto{\pgfqpoint{3.224804in}{3.096251in}}%
\pgfpathmoveto{\pgfqpoint{3.220546in}{3.100509in}}%
\pgfpathlineto{\pgfqpoint{3.220546in}{3.100509in}}%
\pgfpathlineto{\pgfqpoint{3.220546in}{3.104766in}}%
\pgfpathlineto{\pgfqpoint{3.224804in}{3.104766in}}%
\pgfpathlineto{\pgfqpoint{3.224804in}{3.100509in}}%
\pgfpathmoveto{\pgfqpoint{3.220546in}{3.104766in}}%
\pgfpathlineto{\pgfqpoint{3.220546in}{3.104766in}}%
\pgfpathlineto{\pgfqpoint{3.220546in}{3.109024in}}%
\pgfpathlineto{\pgfqpoint{3.224804in}{3.109024in}}%
\pgfpathlineto{\pgfqpoint{3.224804in}{3.104766in}}%
\pgfpathmoveto{\pgfqpoint{3.220546in}{3.109024in}}%
\pgfpathlineto{\pgfqpoint{3.220546in}{3.109024in}}%
\pgfpathlineto{\pgfqpoint{3.220546in}{3.113282in}}%
\pgfpathlineto{\pgfqpoint{3.224804in}{3.113282in}}%
\pgfpathlineto{\pgfqpoint{3.224804in}{3.109024in}}%
\pgfpathmoveto{\pgfqpoint{3.216288in}{3.113282in}}%
\pgfpathlineto{\pgfqpoint{3.216288in}{3.113282in}}%
\pgfpathlineto{\pgfqpoint{3.216288in}{3.117539in}}%
\pgfpathlineto{\pgfqpoint{3.220546in}{3.117539in}}%
\pgfpathlineto{\pgfqpoint{3.220546in}{3.113282in}}%
\pgfpathmoveto{\pgfqpoint{3.216288in}{3.117539in}}%
\pgfpathlineto{\pgfqpoint{3.216288in}{3.117539in}}%
\pgfpathlineto{\pgfqpoint{3.216288in}{3.121797in}}%
\pgfpathlineto{\pgfqpoint{3.220546in}{3.121797in}}%
\pgfpathlineto{\pgfqpoint{3.220546in}{3.117539in}}%
\pgfpathmoveto{\pgfqpoint{3.212030in}{3.126055in}}%
\pgfpathlineto{\pgfqpoint{3.212030in}{3.126055in}}%
\pgfpathlineto{\pgfqpoint{3.212030in}{3.130312in}}%
\pgfpathlineto{\pgfqpoint{3.216288in}{3.130312in}}%
\pgfpathlineto{\pgfqpoint{3.216288in}{3.126055in}}%
\pgfpathmoveto{\pgfqpoint{3.216288in}{3.121797in}}%
\pgfpathlineto{\pgfqpoint{3.216288in}{3.121797in}}%
\pgfpathlineto{\pgfqpoint{3.216288in}{3.126055in}}%
\pgfpathlineto{\pgfqpoint{3.220546in}{3.126055in}}%
\pgfpathlineto{\pgfqpoint{3.220546in}{3.121797in}}%
\pgfpathmoveto{\pgfqpoint{3.216288in}{3.126055in}}%
\pgfpathlineto{\pgfqpoint{3.216288in}{3.126055in}}%
\pgfpathlineto{\pgfqpoint{3.216288in}{3.130312in}}%
\pgfpathlineto{\pgfqpoint{3.220546in}{3.130312in}}%
\pgfpathlineto{\pgfqpoint{3.220546in}{3.126055in}}%
\pgfpathmoveto{\pgfqpoint{3.207773in}{3.147343in}}%
\pgfpathlineto{\pgfqpoint{3.207773in}{3.147343in}}%
\pgfpathlineto{\pgfqpoint{3.207773in}{3.151601in}}%
\pgfpathlineto{\pgfqpoint{3.212030in}{3.151601in}}%
\pgfpathlineto{\pgfqpoint{3.212030in}{3.147343in}}%
\pgfpathmoveto{\pgfqpoint{3.207773in}{3.151601in}}%
\pgfpathlineto{\pgfqpoint{3.207773in}{3.151601in}}%
\pgfpathlineto{\pgfqpoint{3.207773in}{3.155858in}}%
\pgfpathlineto{\pgfqpoint{3.212030in}{3.155858in}}%
\pgfpathlineto{\pgfqpoint{3.212030in}{3.151601in}}%
\pgfpathmoveto{\pgfqpoint{3.207773in}{3.155858in}}%
\pgfpathlineto{\pgfqpoint{3.207773in}{3.155858in}}%
\pgfpathlineto{\pgfqpoint{3.207773in}{3.160116in}}%
\pgfpathlineto{\pgfqpoint{3.212030in}{3.160116in}}%
\pgfpathlineto{\pgfqpoint{3.212030in}{3.155858in}}%
\pgfpathmoveto{\pgfqpoint{3.207773in}{3.160116in}}%
\pgfpathlineto{\pgfqpoint{3.207773in}{3.160116in}}%
\pgfpathlineto{\pgfqpoint{3.207773in}{3.164374in}}%
\pgfpathlineto{\pgfqpoint{3.212030in}{3.164374in}}%
\pgfpathlineto{\pgfqpoint{3.212030in}{3.160116in}}%
\pgfpathmoveto{\pgfqpoint{3.212030in}{3.130312in}}%
\pgfpathlineto{\pgfqpoint{3.212030in}{3.130312in}}%
\pgfpathlineto{\pgfqpoint{3.212030in}{3.134570in}}%
\pgfpathlineto{\pgfqpoint{3.216288in}{3.134570in}}%
\pgfpathlineto{\pgfqpoint{3.216288in}{3.130312in}}%
\pgfpathmoveto{\pgfqpoint{3.212030in}{3.134570in}}%
\pgfpathlineto{\pgfqpoint{3.212030in}{3.134570in}}%
\pgfpathlineto{\pgfqpoint{3.212030in}{3.138828in}}%
\pgfpathlineto{\pgfqpoint{3.216288in}{3.138828in}}%
\pgfpathlineto{\pgfqpoint{3.216288in}{3.134570in}}%
\pgfpathmoveto{\pgfqpoint{3.216288in}{3.130312in}}%
\pgfpathlineto{\pgfqpoint{3.216288in}{3.130312in}}%
\pgfpathlineto{\pgfqpoint{3.216288in}{3.134570in}}%
\pgfpathlineto{\pgfqpoint{3.220546in}{3.134570in}}%
\pgfpathlineto{\pgfqpoint{3.220546in}{3.130312in}}%
\pgfpathmoveto{\pgfqpoint{3.212030in}{3.138828in}}%
\pgfpathlineto{\pgfqpoint{3.212030in}{3.138828in}}%
\pgfpathlineto{\pgfqpoint{3.212030in}{3.143085in}}%
\pgfpathlineto{\pgfqpoint{3.216288in}{3.143085in}}%
\pgfpathlineto{\pgfqpoint{3.216288in}{3.138828in}}%
\pgfpathmoveto{\pgfqpoint{3.212030in}{3.143085in}}%
\pgfpathlineto{\pgfqpoint{3.212030in}{3.143085in}}%
\pgfpathlineto{\pgfqpoint{3.212030in}{3.147343in}}%
\pgfpathlineto{\pgfqpoint{3.216288in}{3.147343in}}%
\pgfpathlineto{\pgfqpoint{3.216288in}{3.143085in}}%
\pgfpathmoveto{\pgfqpoint{3.212030in}{3.147343in}}%
\pgfpathlineto{\pgfqpoint{3.212030in}{3.147343in}}%
\pgfpathlineto{\pgfqpoint{3.212030in}{3.151601in}}%
\pgfpathlineto{\pgfqpoint{3.216288in}{3.151601in}}%
\pgfpathlineto{\pgfqpoint{3.216288in}{3.147343in}}%
\pgfpathmoveto{\pgfqpoint{3.212030in}{3.151601in}}%
\pgfpathlineto{\pgfqpoint{3.212030in}{3.151601in}}%
\pgfpathlineto{\pgfqpoint{3.212030in}{3.155858in}}%
\pgfpathlineto{\pgfqpoint{3.216288in}{3.155858in}}%
\pgfpathlineto{\pgfqpoint{3.216288in}{3.151601in}}%
\pgfpathmoveto{\pgfqpoint{3.190742in}{3.223981in}}%
\pgfpathlineto{\pgfqpoint{3.190742in}{3.223981in}}%
\pgfpathlineto{\pgfqpoint{3.190742in}{3.228239in}}%
\pgfpathlineto{\pgfqpoint{3.194999in}{3.228239in}}%
\pgfpathlineto{\pgfqpoint{3.194999in}{3.223981in}}%
\pgfpathmoveto{\pgfqpoint{3.190742in}{3.228239in}}%
\pgfpathlineto{\pgfqpoint{3.190742in}{3.228239in}}%
\pgfpathlineto{\pgfqpoint{3.190742in}{3.232497in}}%
\pgfpathlineto{\pgfqpoint{3.194999in}{3.232497in}}%
\pgfpathlineto{\pgfqpoint{3.194999in}{3.228239in}}%
\pgfpathmoveto{\pgfqpoint{3.203515in}{3.168631in}}%
\pgfpathlineto{\pgfqpoint{3.203515in}{3.168631in}}%
\pgfpathlineto{\pgfqpoint{3.203515in}{3.172889in}}%
\pgfpathlineto{\pgfqpoint{3.207773in}{3.172889in}}%
\pgfpathlineto{\pgfqpoint{3.207773in}{3.168631in}}%
\pgfpathmoveto{\pgfqpoint{3.207773in}{3.164374in}}%
\pgfpathlineto{\pgfqpoint{3.207773in}{3.164374in}}%
\pgfpathlineto{\pgfqpoint{3.207773in}{3.168631in}}%
\pgfpathlineto{\pgfqpoint{3.212030in}{3.168631in}}%
\pgfpathlineto{\pgfqpoint{3.212030in}{3.164374in}}%
\pgfpathmoveto{\pgfqpoint{3.207773in}{3.168631in}}%
\pgfpathlineto{\pgfqpoint{3.207773in}{3.168631in}}%
\pgfpathlineto{\pgfqpoint{3.207773in}{3.172889in}}%
\pgfpathlineto{\pgfqpoint{3.212030in}{3.172889in}}%
\pgfpathlineto{\pgfqpoint{3.212030in}{3.168631in}}%
\pgfpathmoveto{\pgfqpoint{3.203515in}{3.172889in}}%
\pgfpathlineto{\pgfqpoint{3.203515in}{3.172889in}}%
\pgfpathlineto{\pgfqpoint{3.203515in}{3.177147in}}%
\pgfpathlineto{\pgfqpoint{3.207773in}{3.177147in}}%
\pgfpathlineto{\pgfqpoint{3.207773in}{3.172889in}}%
\pgfpathmoveto{\pgfqpoint{3.203515in}{3.177147in}}%
\pgfpathlineto{\pgfqpoint{3.203515in}{3.177147in}}%
\pgfpathlineto{\pgfqpoint{3.203515in}{3.181404in}}%
\pgfpathlineto{\pgfqpoint{3.207773in}{3.181404in}}%
\pgfpathlineto{\pgfqpoint{3.207773in}{3.177147in}}%
\pgfpathmoveto{\pgfqpoint{3.199257in}{3.185662in}}%
\pgfpathlineto{\pgfqpoint{3.199257in}{3.185662in}}%
\pgfpathlineto{\pgfqpoint{3.199257in}{3.189920in}}%
\pgfpathlineto{\pgfqpoint{3.203515in}{3.189920in}}%
\pgfpathlineto{\pgfqpoint{3.203515in}{3.185662in}}%
\pgfpathmoveto{\pgfqpoint{3.199257in}{3.189920in}}%
\pgfpathlineto{\pgfqpoint{3.199257in}{3.189920in}}%
\pgfpathlineto{\pgfqpoint{3.199257in}{3.194177in}}%
\pgfpathlineto{\pgfqpoint{3.203515in}{3.194177in}}%
\pgfpathlineto{\pgfqpoint{3.203515in}{3.189920in}}%
\pgfpathmoveto{\pgfqpoint{3.199257in}{3.194177in}}%
\pgfpathlineto{\pgfqpoint{3.199257in}{3.194177in}}%
\pgfpathlineto{\pgfqpoint{3.199257in}{3.198435in}}%
\pgfpathlineto{\pgfqpoint{3.203515in}{3.198435in}}%
\pgfpathlineto{\pgfqpoint{3.203515in}{3.194177in}}%
\pgfpathmoveto{\pgfqpoint{3.203515in}{3.181404in}}%
\pgfpathlineto{\pgfqpoint{3.203515in}{3.181404in}}%
\pgfpathlineto{\pgfqpoint{3.203515in}{3.185662in}}%
\pgfpathlineto{\pgfqpoint{3.207773in}{3.185662in}}%
\pgfpathlineto{\pgfqpoint{3.207773in}{3.181404in}}%
\pgfpathmoveto{\pgfqpoint{3.203515in}{3.185662in}}%
\pgfpathlineto{\pgfqpoint{3.203515in}{3.185662in}}%
\pgfpathlineto{\pgfqpoint{3.203515in}{3.189920in}}%
\pgfpathlineto{\pgfqpoint{3.207773in}{3.189920in}}%
\pgfpathlineto{\pgfqpoint{3.207773in}{3.185662in}}%
\pgfpathmoveto{\pgfqpoint{3.203515in}{3.189920in}}%
\pgfpathlineto{\pgfqpoint{3.203515in}{3.189920in}}%
\pgfpathlineto{\pgfqpoint{3.203515in}{3.194177in}}%
\pgfpathlineto{\pgfqpoint{3.207773in}{3.194177in}}%
\pgfpathlineto{\pgfqpoint{3.207773in}{3.189920in}}%
\pgfpathmoveto{\pgfqpoint{3.199257in}{3.198435in}}%
\pgfpathlineto{\pgfqpoint{3.199257in}{3.198435in}}%
\pgfpathlineto{\pgfqpoint{3.199257in}{3.202693in}}%
\pgfpathlineto{\pgfqpoint{3.203515in}{3.202693in}}%
\pgfpathlineto{\pgfqpoint{3.203515in}{3.198435in}}%
\pgfpathmoveto{\pgfqpoint{3.199257in}{3.202693in}}%
\pgfpathlineto{\pgfqpoint{3.199257in}{3.202693in}}%
\pgfpathlineto{\pgfqpoint{3.199257in}{3.206950in}}%
\pgfpathlineto{\pgfqpoint{3.203515in}{3.206950in}}%
\pgfpathlineto{\pgfqpoint{3.203515in}{3.202693in}}%
\pgfpathmoveto{\pgfqpoint{3.194999in}{3.206950in}}%
\pgfpathlineto{\pgfqpoint{3.194999in}{3.206950in}}%
\pgfpathlineto{\pgfqpoint{3.194999in}{3.211208in}}%
\pgfpathlineto{\pgfqpoint{3.199257in}{3.211208in}}%
\pgfpathlineto{\pgfqpoint{3.199257in}{3.206950in}}%
\pgfpathmoveto{\pgfqpoint{3.194999in}{3.211208in}}%
\pgfpathlineto{\pgfqpoint{3.194999in}{3.211208in}}%
\pgfpathlineto{\pgfqpoint{3.194999in}{3.215466in}}%
\pgfpathlineto{\pgfqpoint{3.199257in}{3.215466in}}%
\pgfpathlineto{\pgfqpoint{3.199257in}{3.211208in}}%
\pgfpathmoveto{\pgfqpoint{3.199257in}{3.206950in}}%
\pgfpathlineto{\pgfqpoint{3.199257in}{3.206950in}}%
\pgfpathlineto{\pgfqpoint{3.199257in}{3.211208in}}%
\pgfpathlineto{\pgfqpoint{3.203515in}{3.211208in}}%
\pgfpathlineto{\pgfqpoint{3.203515in}{3.206950in}}%
\pgfpathmoveto{\pgfqpoint{3.199257in}{3.211208in}}%
\pgfpathlineto{\pgfqpoint{3.199257in}{3.211208in}}%
\pgfpathlineto{\pgfqpoint{3.199257in}{3.215466in}}%
\pgfpathlineto{\pgfqpoint{3.203515in}{3.215466in}}%
\pgfpathlineto{\pgfqpoint{3.203515in}{3.211208in}}%
\pgfpathmoveto{\pgfqpoint{3.194999in}{3.215466in}}%
\pgfpathlineto{\pgfqpoint{3.194999in}{3.215466in}}%
\pgfpathlineto{\pgfqpoint{3.194999in}{3.219723in}}%
\pgfpathlineto{\pgfqpoint{3.199257in}{3.219723in}}%
\pgfpathlineto{\pgfqpoint{3.199257in}{3.215466in}}%
\pgfpathmoveto{\pgfqpoint{3.194999in}{3.219723in}}%
\pgfpathlineto{\pgfqpoint{3.194999in}{3.219723in}}%
\pgfpathlineto{\pgfqpoint{3.194999in}{3.223981in}}%
\pgfpathlineto{\pgfqpoint{3.199257in}{3.223981in}}%
\pgfpathlineto{\pgfqpoint{3.199257in}{3.219723in}}%
\pgfpathmoveto{\pgfqpoint{3.194999in}{3.223981in}}%
\pgfpathlineto{\pgfqpoint{3.194999in}{3.223981in}}%
\pgfpathlineto{\pgfqpoint{3.194999in}{3.228239in}}%
\pgfpathlineto{\pgfqpoint{3.199257in}{3.228239in}}%
\pgfpathlineto{\pgfqpoint{3.199257in}{3.223981in}}%
\pgfpathmoveto{\pgfqpoint{3.194999in}{3.228239in}}%
\pgfpathlineto{\pgfqpoint{3.194999in}{3.228239in}}%
\pgfpathlineto{\pgfqpoint{3.194999in}{3.232497in}}%
\pgfpathlineto{\pgfqpoint{3.199257in}{3.232497in}}%
\pgfpathlineto{\pgfqpoint{3.199257in}{3.228239in}}%
\pgfpathmoveto{\pgfqpoint{3.156680in}{3.360236in}}%
\pgfpathlineto{\pgfqpoint{3.156680in}{3.360236in}}%
\pgfpathlineto{\pgfqpoint{3.156680in}{3.364493in}}%
\pgfpathlineto{\pgfqpoint{3.160938in}{3.364493in}}%
\pgfpathlineto{\pgfqpoint{3.160938in}{3.360236in}}%
\pgfpathmoveto{\pgfqpoint{3.156680in}{3.364493in}}%
\pgfpathlineto{\pgfqpoint{3.156680in}{3.364493in}}%
\pgfpathlineto{\pgfqpoint{3.156680in}{3.368751in}}%
\pgfpathlineto{\pgfqpoint{3.160938in}{3.368751in}}%
\pgfpathlineto{\pgfqpoint{3.160938in}{3.364493in}}%
\pgfpathmoveto{\pgfqpoint{3.190742in}{3.232497in}}%
\pgfpathlineto{\pgfqpoint{3.190742in}{3.232497in}}%
\pgfpathlineto{\pgfqpoint{3.190742in}{3.236754in}}%
\pgfpathlineto{\pgfqpoint{3.194999in}{3.236754in}}%
\pgfpathlineto{\pgfqpoint{3.194999in}{3.232497in}}%
\pgfpathmoveto{\pgfqpoint{3.190742in}{3.236754in}}%
\pgfpathlineto{\pgfqpoint{3.190742in}{3.236754in}}%
\pgfpathlineto{\pgfqpoint{3.190742in}{3.241012in}}%
\pgfpathlineto{\pgfqpoint{3.194999in}{3.241012in}}%
\pgfpathlineto{\pgfqpoint{3.194999in}{3.236754in}}%
\pgfpathmoveto{\pgfqpoint{3.186484in}{3.241012in}}%
\pgfpathlineto{\pgfqpoint{3.186484in}{3.241012in}}%
\pgfpathlineto{\pgfqpoint{3.186484in}{3.245270in}}%
\pgfpathlineto{\pgfqpoint{3.190742in}{3.245270in}}%
\pgfpathlineto{\pgfqpoint{3.190742in}{3.241012in}}%
\pgfpathmoveto{\pgfqpoint{3.186484in}{3.245270in}}%
\pgfpathlineto{\pgfqpoint{3.186484in}{3.245270in}}%
\pgfpathlineto{\pgfqpoint{3.186484in}{3.249528in}}%
\pgfpathlineto{\pgfqpoint{3.190742in}{3.249528in}}%
\pgfpathlineto{\pgfqpoint{3.190742in}{3.245270in}}%
\pgfpathmoveto{\pgfqpoint{3.190742in}{3.241012in}}%
\pgfpathlineto{\pgfqpoint{3.190742in}{3.241012in}}%
\pgfpathlineto{\pgfqpoint{3.190742in}{3.245270in}}%
\pgfpathlineto{\pgfqpoint{3.194999in}{3.245270in}}%
\pgfpathlineto{\pgfqpoint{3.194999in}{3.241012in}}%
\pgfpathmoveto{\pgfqpoint{3.190742in}{3.245270in}}%
\pgfpathlineto{\pgfqpoint{3.190742in}{3.245270in}}%
\pgfpathlineto{\pgfqpoint{3.190742in}{3.249528in}}%
\pgfpathlineto{\pgfqpoint{3.194999in}{3.249528in}}%
\pgfpathlineto{\pgfqpoint{3.194999in}{3.245270in}}%
\pgfpathmoveto{\pgfqpoint{3.182226in}{3.258044in}}%
\pgfpathlineto{\pgfqpoint{3.182226in}{3.258044in}}%
\pgfpathlineto{\pgfqpoint{3.182226in}{3.262302in}}%
\pgfpathlineto{\pgfqpoint{3.186484in}{3.262302in}}%
\pgfpathlineto{\pgfqpoint{3.186484in}{3.258044in}}%
\pgfpathmoveto{\pgfqpoint{3.182226in}{3.262302in}}%
\pgfpathlineto{\pgfqpoint{3.182226in}{3.262302in}}%
\pgfpathlineto{\pgfqpoint{3.182226in}{3.266560in}}%
\pgfpathlineto{\pgfqpoint{3.186484in}{3.266560in}}%
\pgfpathlineto{\pgfqpoint{3.186484in}{3.262302in}}%
\pgfpathmoveto{\pgfqpoint{3.186484in}{3.249528in}}%
\pgfpathlineto{\pgfqpoint{3.186484in}{3.249528in}}%
\pgfpathlineto{\pgfqpoint{3.186484in}{3.253786in}}%
\pgfpathlineto{\pgfqpoint{3.190742in}{3.253786in}}%
\pgfpathlineto{\pgfqpoint{3.190742in}{3.249528in}}%
\pgfpathmoveto{\pgfqpoint{3.186484in}{3.253786in}}%
\pgfpathlineto{\pgfqpoint{3.186484in}{3.253786in}}%
\pgfpathlineto{\pgfqpoint{3.186484in}{3.258044in}}%
\pgfpathlineto{\pgfqpoint{3.190742in}{3.258044in}}%
\pgfpathlineto{\pgfqpoint{3.190742in}{3.253786in}}%
\pgfpathmoveto{\pgfqpoint{3.190742in}{3.249528in}}%
\pgfpathlineto{\pgfqpoint{3.190742in}{3.249528in}}%
\pgfpathlineto{\pgfqpoint{3.190742in}{3.253786in}}%
\pgfpathlineto{\pgfqpoint{3.194999in}{3.253786in}}%
\pgfpathlineto{\pgfqpoint{3.194999in}{3.249528in}}%
\pgfpathmoveto{\pgfqpoint{3.186484in}{3.258044in}}%
\pgfpathlineto{\pgfqpoint{3.186484in}{3.258044in}}%
\pgfpathlineto{\pgfqpoint{3.186484in}{3.262302in}}%
\pgfpathlineto{\pgfqpoint{3.190742in}{3.262302in}}%
\pgfpathlineto{\pgfqpoint{3.190742in}{3.258044in}}%
\pgfpathmoveto{\pgfqpoint{3.186484in}{3.262302in}}%
\pgfpathlineto{\pgfqpoint{3.186484in}{3.262302in}}%
\pgfpathlineto{\pgfqpoint{3.186484in}{3.266560in}}%
\pgfpathlineto{\pgfqpoint{3.190742in}{3.266560in}}%
\pgfpathlineto{\pgfqpoint{3.190742in}{3.262302in}}%
\pgfpathmoveto{\pgfqpoint{3.173711in}{3.292108in}}%
\pgfpathlineto{\pgfqpoint{3.173711in}{3.292108in}}%
\pgfpathlineto{\pgfqpoint{3.173711in}{3.296366in}}%
\pgfpathlineto{\pgfqpoint{3.177969in}{3.296366in}}%
\pgfpathlineto{\pgfqpoint{3.177969in}{3.292108in}}%
\pgfpathmoveto{\pgfqpoint{3.173711in}{3.296366in}}%
\pgfpathlineto{\pgfqpoint{3.173711in}{3.296366in}}%
\pgfpathlineto{\pgfqpoint{3.173711in}{3.300624in}}%
\pgfpathlineto{\pgfqpoint{3.177969in}{3.300624in}}%
\pgfpathlineto{\pgfqpoint{3.177969in}{3.296366in}}%
\pgfpathmoveto{\pgfqpoint{3.182226in}{3.266560in}}%
\pgfpathlineto{\pgfqpoint{3.182226in}{3.266560in}}%
\pgfpathlineto{\pgfqpoint{3.182226in}{3.270818in}}%
\pgfpathlineto{\pgfqpoint{3.186484in}{3.270818in}}%
\pgfpathlineto{\pgfqpoint{3.186484in}{3.266560in}}%
\pgfpathmoveto{\pgfqpoint{3.182226in}{3.270818in}}%
\pgfpathlineto{\pgfqpoint{3.182226in}{3.270818in}}%
\pgfpathlineto{\pgfqpoint{3.182226in}{3.275076in}}%
\pgfpathlineto{\pgfqpoint{3.186484in}{3.275076in}}%
\pgfpathlineto{\pgfqpoint{3.186484in}{3.270818in}}%
\pgfpathmoveto{\pgfqpoint{3.177969in}{3.275076in}}%
\pgfpathlineto{\pgfqpoint{3.177969in}{3.275076in}}%
\pgfpathlineto{\pgfqpoint{3.177969in}{3.279334in}}%
\pgfpathlineto{\pgfqpoint{3.182226in}{3.279334in}}%
\pgfpathlineto{\pgfqpoint{3.182226in}{3.275076in}}%
\pgfpathmoveto{\pgfqpoint{3.177969in}{3.279334in}}%
\pgfpathlineto{\pgfqpoint{3.177969in}{3.279334in}}%
\pgfpathlineto{\pgfqpoint{3.177969in}{3.283592in}}%
\pgfpathlineto{\pgfqpoint{3.182226in}{3.283592in}}%
\pgfpathlineto{\pgfqpoint{3.182226in}{3.279334in}}%
\pgfpathmoveto{\pgfqpoint{3.182226in}{3.275076in}}%
\pgfpathlineto{\pgfqpoint{3.182226in}{3.275076in}}%
\pgfpathlineto{\pgfqpoint{3.182226in}{3.279334in}}%
\pgfpathlineto{\pgfqpoint{3.186484in}{3.279334in}}%
\pgfpathlineto{\pgfqpoint{3.186484in}{3.275076in}}%
\pgfpathmoveto{\pgfqpoint{3.182226in}{3.279334in}}%
\pgfpathlineto{\pgfqpoint{3.182226in}{3.279334in}}%
\pgfpathlineto{\pgfqpoint{3.182226in}{3.283592in}}%
\pgfpathlineto{\pgfqpoint{3.186484in}{3.283592in}}%
\pgfpathlineto{\pgfqpoint{3.186484in}{3.279334in}}%
\pgfpathmoveto{\pgfqpoint{3.186484in}{3.266560in}}%
\pgfpathlineto{\pgfqpoint{3.186484in}{3.266560in}}%
\pgfpathlineto{\pgfqpoint{3.186484in}{3.270818in}}%
\pgfpathlineto{\pgfqpoint{3.190742in}{3.270818in}}%
\pgfpathlineto{\pgfqpoint{3.190742in}{3.266560in}}%
\pgfpathmoveto{\pgfqpoint{3.177969in}{3.283592in}}%
\pgfpathlineto{\pgfqpoint{3.177969in}{3.283592in}}%
\pgfpathlineto{\pgfqpoint{3.177969in}{3.287850in}}%
\pgfpathlineto{\pgfqpoint{3.182226in}{3.287850in}}%
\pgfpathlineto{\pgfqpoint{3.182226in}{3.283592in}}%
\pgfpathmoveto{\pgfqpoint{3.177969in}{3.287850in}}%
\pgfpathlineto{\pgfqpoint{3.177969in}{3.287850in}}%
\pgfpathlineto{\pgfqpoint{3.177969in}{3.292108in}}%
\pgfpathlineto{\pgfqpoint{3.182226in}{3.292108in}}%
\pgfpathlineto{\pgfqpoint{3.182226in}{3.287850in}}%
\pgfpathmoveto{\pgfqpoint{3.182226in}{3.283592in}}%
\pgfpathlineto{\pgfqpoint{3.182226in}{3.283592in}}%
\pgfpathlineto{\pgfqpoint{3.182226in}{3.287850in}}%
\pgfpathlineto{\pgfqpoint{3.186484in}{3.287850in}}%
\pgfpathlineto{\pgfqpoint{3.186484in}{3.283592in}}%
\pgfpathmoveto{\pgfqpoint{3.177969in}{3.292108in}}%
\pgfpathlineto{\pgfqpoint{3.177969in}{3.292108in}}%
\pgfpathlineto{\pgfqpoint{3.177969in}{3.296366in}}%
\pgfpathlineto{\pgfqpoint{3.182226in}{3.296366in}}%
\pgfpathlineto{\pgfqpoint{3.182226in}{3.292108in}}%
\pgfpathmoveto{\pgfqpoint{3.177969in}{3.296366in}}%
\pgfpathlineto{\pgfqpoint{3.177969in}{3.296366in}}%
\pgfpathlineto{\pgfqpoint{3.177969in}{3.300624in}}%
\pgfpathlineto{\pgfqpoint{3.182226in}{3.300624in}}%
\pgfpathlineto{\pgfqpoint{3.182226in}{3.296366in}}%
\pgfpathmoveto{\pgfqpoint{3.173711in}{3.300624in}}%
\pgfpathlineto{\pgfqpoint{3.173711in}{3.300624in}}%
\pgfpathlineto{\pgfqpoint{3.173711in}{3.304882in}}%
\pgfpathlineto{\pgfqpoint{3.177969in}{3.304882in}}%
\pgfpathlineto{\pgfqpoint{3.177969in}{3.300624in}}%
\pgfpathmoveto{\pgfqpoint{3.173711in}{3.304882in}}%
\pgfpathlineto{\pgfqpoint{3.173711in}{3.304882in}}%
\pgfpathlineto{\pgfqpoint{3.173711in}{3.309140in}}%
\pgfpathlineto{\pgfqpoint{3.177969in}{3.309140in}}%
\pgfpathlineto{\pgfqpoint{3.177969in}{3.304882in}}%
\pgfpathmoveto{\pgfqpoint{3.169453in}{3.309140in}}%
\pgfpathlineto{\pgfqpoint{3.169453in}{3.309140in}}%
\pgfpathlineto{\pgfqpoint{3.169453in}{3.313398in}}%
\pgfpathlineto{\pgfqpoint{3.173711in}{3.313398in}}%
\pgfpathlineto{\pgfqpoint{3.173711in}{3.309140in}}%
\pgfpathmoveto{\pgfqpoint{3.169453in}{3.313398in}}%
\pgfpathlineto{\pgfqpoint{3.169453in}{3.313398in}}%
\pgfpathlineto{\pgfqpoint{3.169453in}{3.317656in}}%
\pgfpathlineto{\pgfqpoint{3.173711in}{3.317656in}}%
\pgfpathlineto{\pgfqpoint{3.173711in}{3.313398in}}%
\pgfpathmoveto{\pgfqpoint{3.173711in}{3.309140in}}%
\pgfpathlineto{\pgfqpoint{3.173711in}{3.309140in}}%
\pgfpathlineto{\pgfqpoint{3.173711in}{3.313398in}}%
\pgfpathlineto{\pgfqpoint{3.177969in}{3.313398in}}%
\pgfpathlineto{\pgfqpoint{3.177969in}{3.309140in}}%
\pgfpathmoveto{\pgfqpoint{3.173711in}{3.313398in}}%
\pgfpathlineto{\pgfqpoint{3.173711in}{3.313398in}}%
\pgfpathlineto{\pgfqpoint{3.173711in}{3.317656in}}%
\pgfpathlineto{\pgfqpoint{3.177969in}{3.317656in}}%
\pgfpathlineto{\pgfqpoint{3.177969in}{3.313398in}}%
\pgfpathmoveto{\pgfqpoint{3.165195in}{3.326172in}}%
\pgfpathlineto{\pgfqpoint{3.165195in}{3.326172in}}%
\pgfpathlineto{\pgfqpoint{3.165195in}{3.330430in}}%
\pgfpathlineto{\pgfqpoint{3.169453in}{3.330430in}}%
\pgfpathlineto{\pgfqpoint{3.169453in}{3.326172in}}%
\pgfpathmoveto{\pgfqpoint{3.165195in}{3.330430in}}%
\pgfpathlineto{\pgfqpoint{3.165195in}{3.330430in}}%
\pgfpathlineto{\pgfqpoint{3.165195in}{3.334688in}}%
\pgfpathlineto{\pgfqpoint{3.169453in}{3.334688in}}%
\pgfpathlineto{\pgfqpoint{3.169453in}{3.330430in}}%
\pgfpathmoveto{\pgfqpoint{3.169453in}{3.317656in}}%
\pgfpathlineto{\pgfqpoint{3.169453in}{3.317656in}}%
\pgfpathlineto{\pgfqpoint{3.169453in}{3.321914in}}%
\pgfpathlineto{\pgfqpoint{3.173711in}{3.321914in}}%
\pgfpathlineto{\pgfqpoint{3.173711in}{3.317656in}}%
\pgfpathmoveto{\pgfqpoint{3.169453in}{3.321914in}}%
\pgfpathlineto{\pgfqpoint{3.169453in}{3.321914in}}%
\pgfpathlineto{\pgfqpoint{3.169453in}{3.326172in}}%
\pgfpathlineto{\pgfqpoint{3.173711in}{3.326172in}}%
\pgfpathlineto{\pgfqpoint{3.173711in}{3.321914in}}%
\pgfpathmoveto{\pgfqpoint{3.173711in}{3.317656in}}%
\pgfpathlineto{\pgfqpoint{3.173711in}{3.317656in}}%
\pgfpathlineto{\pgfqpoint{3.173711in}{3.321914in}}%
\pgfpathlineto{\pgfqpoint{3.177969in}{3.321914in}}%
\pgfpathlineto{\pgfqpoint{3.177969in}{3.317656in}}%
\pgfpathmoveto{\pgfqpoint{3.169453in}{3.326172in}}%
\pgfpathlineto{\pgfqpoint{3.169453in}{3.326172in}}%
\pgfpathlineto{\pgfqpoint{3.169453in}{3.330430in}}%
\pgfpathlineto{\pgfqpoint{3.173711in}{3.330430in}}%
\pgfpathlineto{\pgfqpoint{3.173711in}{3.326172in}}%
\pgfpathmoveto{\pgfqpoint{3.169453in}{3.330430in}}%
\pgfpathlineto{\pgfqpoint{3.169453in}{3.330430in}}%
\pgfpathlineto{\pgfqpoint{3.169453in}{3.334688in}}%
\pgfpathlineto{\pgfqpoint{3.173711in}{3.334688in}}%
\pgfpathlineto{\pgfqpoint{3.173711in}{3.330430in}}%
\pgfpathmoveto{\pgfqpoint{3.177969in}{3.300624in}}%
\pgfpathlineto{\pgfqpoint{3.177969in}{3.300624in}}%
\pgfpathlineto{\pgfqpoint{3.177969in}{3.304882in}}%
\pgfpathlineto{\pgfqpoint{3.182226in}{3.304882in}}%
\pgfpathlineto{\pgfqpoint{3.182226in}{3.300624in}}%
\pgfpathmoveto{\pgfqpoint{3.165195in}{3.334688in}}%
\pgfpathlineto{\pgfqpoint{3.165195in}{3.334688in}}%
\pgfpathlineto{\pgfqpoint{3.165195in}{3.338946in}}%
\pgfpathlineto{\pgfqpoint{3.169453in}{3.338946in}}%
\pgfpathlineto{\pgfqpoint{3.169453in}{3.334688in}}%
\pgfpathmoveto{\pgfqpoint{3.165195in}{3.338946in}}%
\pgfpathlineto{\pgfqpoint{3.165195in}{3.338946in}}%
\pgfpathlineto{\pgfqpoint{3.165195in}{3.343204in}}%
\pgfpathlineto{\pgfqpoint{3.169453in}{3.343204in}}%
\pgfpathlineto{\pgfqpoint{3.169453in}{3.338946in}}%
\pgfpathmoveto{\pgfqpoint{3.160938in}{3.343204in}}%
\pgfpathlineto{\pgfqpoint{3.160938in}{3.343204in}}%
\pgfpathlineto{\pgfqpoint{3.160938in}{3.347462in}}%
\pgfpathlineto{\pgfqpoint{3.165195in}{3.347462in}}%
\pgfpathlineto{\pgfqpoint{3.165195in}{3.343204in}}%
\pgfpathmoveto{\pgfqpoint{3.160938in}{3.347462in}}%
\pgfpathlineto{\pgfqpoint{3.160938in}{3.347462in}}%
\pgfpathlineto{\pgfqpoint{3.160938in}{3.351720in}}%
\pgfpathlineto{\pgfqpoint{3.165195in}{3.351720in}}%
\pgfpathlineto{\pgfqpoint{3.165195in}{3.347462in}}%
\pgfpathmoveto{\pgfqpoint{3.165195in}{3.343204in}}%
\pgfpathlineto{\pgfqpoint{3.165195in}{3.343204in}}%
\pgfpathlineto{\pgfqpoint{3.165195in}{3.347462in}}%
\pgfpathlineto{\pgfqpoint{3.169453in}{3.347462in}}%
\pgfpathlineto{\pgfqpoint{3.169453in}{3.343204in}}%
\pgfpathmoveto{\pgfqpoint{3.165195in}{3.347462in}}%
\pgfpathlineto{\pgfqpoint{3.165195in}{3.347462in}}%
\pgfpathlineto{\pgfqpoint{3.165195in}{3.351720in}}%
\pgfpathlineto{\pgfqpoint{3.169453in}{3.351720in}}%
\pgfpathlineto{\pgfqpoint{3.169453in}{3.347462in}}%
\pgfpathmoveto{\pgfqpoint{3.169453in}{3.334688in}}%
\pgfpathlineto{\pgfqpoint{3.169453in}{3.334688in}}%
\pgfpathlineto{\pgfqpoint{3.169453in}{3.338946in}}%
\pgfpathlineto{\pgfqpoint{3.173711in}{3.338946in}}%
\pgfpathlineto{\pgfqpoint{3.173711in}{3.334688in}}%
\pgfpathmoveto{\pgfqpoint{3.160938in}{3.351720in}}%
\pgfpathlineto{\pgfqpoint{3.160938in}{3.351720in}}%
\pgfpathlineto{\pgfqpoint{3.160938in}{3.355978in}}%
\pgfpathlineto{\pgfqpoint{3.165195in}{3.355978in}}%
\pgfpathlineto{\pgfqpoint{3.165195in}{3.351720in}}%
\pgfpathmoveto{\pgfqpoint{3.160938in}{3.355978in}}%
\pgfpathlineto{\pgfqpoint{3.160938in}{3.355978in}}%
\pgfpathlineto{\pgfqpoint{3.160938in}{3.360236in}}%
\pgfpathlineto{\pgfqpoint{3.165195in}{3.360236in}}%
\pgfpathlineto{\pgfqpoint{3.165195in}{3.355978in}}%
\pgfpathmoveto{\pgfqpoint{3.165195in}{3.351720in}}%
\pgfpathlineto{\pgfqpoint{3.165195in}{3.351720in}}%
\pgfpathlineto{\pgfqpoint{3.165195in}{3.355978in}}%
\pgfpathlineto{\pgfqpoint{3.169453in}{3.355978in}}%
\pgfpathlineto{\pgfqpoint{3.169453in}{3.351720in}}%
\pgfpathmoveto{\pgfqpoint{3.160938in}{3.360236in}}%
\pgfpathlineto{\pgfqpoint{3.160938in}{3.360236in}}%
\pgfpathlineto{\pgfqpoint{3.160938in}{3.364493in}}%
\pgfpathlineto{\pgfqpoint{3.165195in}{3.364493in}}%
\pgfpathlineto{\pgfqpoint{3.165195in}{3.360236in}}%
\pgfpathmoveto{\pgfqpoint{3.160938in}{3.364493in}}%
\pgfpathlineto{\pgfqpoint{3.160938in}{3.364493in}}%
\pgfpathlineto{\pgfqpoint{3.160938in}{3.368751in}}%
\pgfpathlineto{\pgfqpoint{3.165195in}{3.368751in}}%
\pgfpathlineto{\pgfqpoint{3.165195in}{3.364493in}}%
\pgfpathmoveto{\pgfqpoint{3.152422in}{3.373009in}}%
\pgfpathlineto{\pgfqpoint{3.152422in}{3.373009in}}%
\pgfpathlineto{\pgfqpoint{3.152422in}{3.377267in}}%
\pgfpathlineto{\pgfqpoint{3.156680in}{3.377267in}}%
\pgfpathlineto{\pgfqpoint{3.156680in}{3.373009in}}%
\pgfpathmoveto{\pgfqpoint{3.156680in}{3.368751in}}%
\pgfpathlineto{\pgfqpoint{3.156680in}{3.368751in}}%
\pgfpathlineto{\pgfqpoint{3.156680in}{3.373009in}}%
\pgfpathlineto{\pgfqpoint{3.160938in}{3.373009in}}%
\pgfpathlineto{\pgfqpoint{3.160938in}{3.368751in}}%
\pgfpathmoveto{\pgfqpoint{3.156680in}{3.373009in}}%
\pgfpathlineto{\pgfqpoint{3.156680in}{3.373009in}}%
\pgfpathlineto{\pgfqpoint{3.156680in}{3.377267in}}%
\pgfpathlineto{\pgfqpoint{3.160938in}{3.377267in}}%
\pgfpathlineto{\pgfqpoint{3.160938in}{3.373009in}}%
\pgfpathmoveto{\pgfqpoint{3.152422in}{3.377267in}}%
\pgfpathlineto{\pgfqpoint{3.152422in}{3.377267in}}%
\pgfpathlineto{\pgfqpoint{3.152422in}{3.381525in}}%
\pgfpathlineto{\pgfqpoint{3.156680in}{3.381525in}}%
\pgfpathlineto{\pgfqpoint{3.156680in}{3.377267in}}%
\pgfpathmoveto{\pgfqpoint{3.152422in}{3.381525in}}%
\pgfpathlineto{\pgfqpoint{3.152422in}{3.381525in}}%
\pgfpathlineto{\pgfqpoint{3.152422in}{3.385783in}}%
\pgfpathlineto{\pgfqpoint{3.156680in}{3.385783in}}%
\pgfpathlineto{\pgfqpoint{3.156680in}{3.381525in}}%
\pgfpathmoveto{\pgfqpoint{3.156680in}{3.377267in}}%
\pgfpathlineto{\pgfqpoint{3.156680in}{3.377267in}}%
\pgfpathlineto{\pgfqpoint{3.156680in}{3.381525in}}%
\pgfpathlineto{\pgfqpoint{3.160938in}{3.381525in}}%
\pgfpathlineto{\pgfqpoint{3.160938in}{3.377267in}}%
\pgfpathmoveto{\pgfqpoint{3.156680in}{3.381525in}}%
\pgfpathlineto{\pgfqpoint{3.156680in}{3.381525in}}%
\pgfpathlineto{\pgfqpoint{3.156680in}{3.385783in}}%
\pgfpathlineto{\pgfqpoint{3.160938in}{3.385783in}}%
\pgfpathlineto{\pgfqpoint{3.160938in}{3.381525in}}%
\pgfpathmoveto{\pgfqpoint{3.148164in}{3.390041in}}%
\pgfpathlineto{\pgfqpoint{3.148164in}{3.390041in}}%
\pgfpathlineto{\pgfqpoint{3.148164in}{3.394299in}}%
\pgfpathlineto{\pgfqpoint{3.152422in}{3.394299in}}%
\pgfpathlineto{\pgfqpoint{3.152422in}{3.390041in}}%
\pgfpathmoveto{\pgfqpoint{3.148164in}{3.394299in}}%
\pgfpathlineto{\pgfqpoint{3.148164in}{3.394299in}}%
\pgfpathlineto{\pgfqpoint{3.148164in}{3.398557in}}%
\pgfpathlineto{\pgfqpoint{3.152422in}{3.398557in}}%
\pgfpathlineto{\pgfqpoint{3.152422in}{3.394299in}}%
\pgfpathmoveto{\pgfqpoint{3.148164in}{3.398557in}}%
\pgfpathlineto{\pgfqpoint{3.148164in}{3.398557in}}%
\pgfpathlineto{\pgfqpoint{3.148164in}{3.402814in}}%
\pgfpathlineto{\pgfqpoint{3.152422in}{3.402814in}}%
\pgfpathlineto{\pgfqpoint{3.152422in}{3.398557in}}%
\pgfpathmoveto{\pgfqpoint{3.152422in}{3.385783in}}%
\pgfpathlineto{\pgfqpoint{3.152422in}{3.385783in}}%
\pgfpathlineto{\pgfqpoint{3.152422in}{3.390041in}}%
\pgfpathlineto{\pgfqpoint{3.156680in}{3.390041in}}%
\pgfpathlineto{\pgfqpoint{3.156680in}{3.385783in}}%
\pgfpathmoveto{\pgfqpoint{3.152422in}{3.390041in}}%
\pgfpathlineto{\pgfqpoint{3.152422in}{3.390041in}}%
\pgfpathlineto{\pgfqpoint{3.152422in}{3.394299in}}%
\pgfpathlineto{\pgfqpoint{3.156680in}{3.394299in}}%
\pgfpathlineto{\pgfqpoint{3.156680in}{3.390041in}}%
\pgfpathmoveto{\pgfqpoint{3.156680in}{3.385783in}}%
\pgfpathlineto{\pgfqpoint{3.156680in}{3.385783in}}%
\pgfpathlineto{\pgfqpoint{3.156680in}{3.390041in}}%
\pgfpathlineto{\pgfqpoint{3.160938in}{3.390041in}}%
\pgfpathlineto{\pgfqpoint{3.160938in}{3.385783in}}%
\pgfpathmoveto{\pgfqpoint{3.152422in}{3.394299in}}%
\pgfpathlineto{\pgfqpoint{3.152422in}{3.394299in}}%
\pgfpathlineto{\pgfqpoint{3.152422in}{3.398557in}}%
\pgfpathlineto{\pgfqpoint{3.156680in}{3.398557in}}%
\pgfpathlineto{\pgfqpoint{3.156680in}{3.394299in}}%
\pgfpathmoveto{\pgfqpoint{3.152422in}{3.398557in}}%
\pgfpathlineto{\pgfqpoint{3.152422in}{3.398557in}}%
\pgfpathlineto{\pgfqpoint{3.152422in}{3.402814in}}%
\pgfpathlineto{\pgfqpoint{3.156680in}{3.402814in}}%
\pgfpathlineto{\pgfqpoint{3.156680in}{3.398557in}}%
\pgfpathmoveto{\pgfqpoint{3.139649in}{3.419846in}}%
\pgfpathlineto{\pgfqpoint{3.139649in}{3.419846in}}%
\pgfpathlineto{\pgfqpoint{3.139649in}{3.424104in}}%
\pgfpathlineto{\pgfqpoint{3.143907in}{3.424104in}}%
\pgfpathlineto{\pgfqpoint{3.143907in}{3.419846in}}%
\pgfpathmoveto{\pgfqpoint{3.139649in}{3.424104in}}%
\pgfpathlineto{\pgfqpoint{3.139649in}{3.424104in}}%
\pgfpathlineto{\pgfqpoint{3.139649in}{3.428362in}}%
\pgfpathlineto{\pgfqpoint{3.143907in}{3.428362in}}%
\pgfpathlineto{\pgfqpoint{3.143907in}{3.424104in}}%
\pgfpathmoveto{\pgfqpoint{3.135391in}{3.432619in}}%
\pgfpathlineto{\pgfqpoint{3.135391in}{3.432619in}}%
\pgfpathlineto{\pgfqpoint{3.135391in}{3.436877in}}%
\pgfpathlineto{\pgfqpoint{3.139649in}{3.436877in}}%
\pgfpathlineto{\pgfqpoint{3.139649in}{3.432619in}}%
\pgfpathmoveto{\pgfqpoint{3.139649in}{3.428362in}}%
\pgfpathlineto{\pgfqpoint{3.139649in}{3.428362in}}%
\pgfpathlineto{\pgfqpoint{3.139649in}{3.432619in}}%
\pgfpathlineto{\pgfqpoint{3.143907in}{3.432619in}}%
\pgfpathlineto{\pgfqpoint{3.143907in}{3.428362in}}%
\pgfpathmoveto{\pgfqpoint{3.139649in}{3.432619in}}%
\pgfpathlineto{\pgfqpoint{3.139649in}{3.432619in}}%
\pgfpathlineto{\pgfqpoint{3.139649in}{3.436877in}}%
\pgfpathlineto{\pgfqpoint{3.143907in}{3.436877in}}%
\pgfpathlineto{\pgfqpoint{3.143907in}{3.432619in}}%
\pgfpathmoveto{\pgfqpoint{3.143907in}{3.402814in}}%
\pgfpathlineto{\pgfqpoint{3.143907in}{3.402814in}}%
\pgfpathlineto{\pgfqpoint{3.143907in}{3.407072in}}%
\pgfpathlineto{\pgfqpoint{3.148164in}{3.407072in}}%
\pgfpathlineto{\pgfqpoint{3.148164in}{3.402814in}}%
\pgfpathmoveto{\pgfqpoint{3.143907in}{3.407072in}}%
\pgfpathlineto{\pgfqpoint{3.143907in}{3.407072in}}%
\pgfpathlineto{\pgfqpoint{3.143907in}{3.411330in}}%
\pgfpathlineto{\pgfqpoint{3.148164in}{3.411330in}}%
\pgfpathlineto{\pgfqpoint{3.148164in}{3.407072in}}%
\pgfpathmoveto{\pgfqpoint{3.148164in}{3.402814in}}%
\pgfpathlineto{\pgfqpoint{3.148164in}{3.402814in}}%
\pgfpathlineto{\pgfqpoint{3.148164in}{3.407072in}}%
\pgfpathlineto{\pgfqpoint{3.152422in}{3.407072in}}%
\pgfpathlineto{\pgfqpoint{3.152422in}{3.402814in}}%
\pgfpathmoveto{\pgfqpoint{3.148164in}{3.407072in}}%
\pgfpathlineto{\pgfqpoint{3.148164in}{3.407072in}}%
\pgfpathlineto{\pgfqpoint{3.148164in}{3.411330in}}%
\pgfpathlineto{\pgfqpoint{3.152422in}{3.411330in}}%
\pgfpathlineto{\pgfqpoint{3.152422in}{3.407072in}}%
\pgfpathmoveto{\pgfqpoint{3.143907in}{3.411330in}}%
\pgfpathlineto{\pgfqpoint{3.143907in}{3.411330in}}%
\pgfpathlineto{\pgfqpoint{3.143907in}{3.415588in}}%
\pgfpathlineto{\pgfqpoint{3.148164in}{3.415588in}}%
\pgfpathlineto{\pgfqpoint{3.148164in}{3.411330in}}%
\pgfpathmoveto{\pgfqpoint{3.143907in}{3.415588in}}%
\pgfpathlineto{\pgfqpoint{3.143907in}{3.415588in}}%
\pgfpathlineto{\pgfqpoint{3.143907in}{3.419846in}}%
\pgfpathlineto{\pgfqpoint{3.148164in}{3.419846in}}%
\pgfpathlineto{\pgfqpoint{3.148164in}{3.415588in}}%
\pgfpathmoveto{\pgfqpoint{3.148164in}{3.411330in}}%
\pgfpathlineto{\pgfqpoint{3.148164in}{3.411330in}}%
\pgfpathlineto{\pgfqpoint{3.148164in}{3.415588in}}%
\pgfpathlineto{\pgfqpoint{3.152422in}{3.415588in}}%
\pgfpathlineto{\pgfqpoint{3.152422in}{3.411330in}}%
\pgfpathmoveto{\pgfqpoint{3.148164in}{3.415588in}}%
\pgfpathlineto{\pgfqpoint{3.148164in}{3.415588in}}%
\pgfpathlineto{\pgfqpoint{3.148164in}{3.419846in}}%
\pgfpathlineto{\pgfqpoint{3.152422in}{3.419846in}}%
\pgfpathlineto{\pgfqpoint{3.152422in}{3.415588in}}%
\pgfpathmoveto{\pgfqpoint{3.152422in}{3.402814in}}%
\pgfpathlineto{\pgfqpoint{3.152422in}{3.402814in}}%
\pgfpathlineto{\pgfqpoint{3.152422in}{3.407072in}}%
\pgfpathlineto{\pgfqpoint{3.156680in}{3.407072in}}%
\pgfpathlineto{\pgfqpoint{3.156680in}{3.402814in}}%
\pgfpathmoveto{\pgfqpoint{3.143907in}{3.419846in}}%
\pgfpathlineto{\pgfqpoint{3.143907in}{3.419846in}}%
\pgfpathlineto{\pgfqpoint{3.143907in}{3.424104in}}%
\pgfpathlineto{\pgfqpoint{3.148164in}{3.424104in}}%
\pgfpathlineto{\pgfqpoint{3.148164in}{3.419846in}}%
\pgfpathmoveto{\pgfqpoint{3.143907in}{3.424104in}}%
\pgfpathlineto{\pgfqpoint{3.143907in}{3.424104in}}%
\pgfpathlineto{\pgfqpoint{3.143907in}{3.428362in}}%
\pgfpathlineto{\pgfqpoint{3.148164in}{3.428362in}}%
\pgfpathlineto{\pgfqpoint{3.148164in}{3.424104in}}%
\pgfpathmoveto{\pgfqpoint{3.143907in}{3.428362in}}%
\pgfpathlineto{\pgfqpoint{3.143907in}{3.428362in}}%
\pgfpathlineto{\pgfqpoint{3.143907in}{3.432619in}}%
\pgfpathlineto{\pgfqpoint{3.148164in}{3.432619in}}%
\pgfpathlineto{\pgfqpoint{3.148164in}{3.428362in}}%
\pgfpathmoveto{\pgfqpoint{3.143907in}{3.432619in}}%
\pgfpathlineto{\pgfqpoint{3.143907in}{3.432619in}}%
\pgfpathlineto{\pgfqpoint{3.143907in}{3.436877in}}%
\pgfpathlineto{\pgfqpoint{3.148164in}{3.436877in}}%
\pgfpathlineto{\pgfqpoint{3.148164in}{3.432619in}}%
\pgfpathmoveto{\pgfqpoint{3.122618in}{3.475198in}}%
\pgfpathlineto{\pgfqpoint{3.122618in}{3.475198in}}%
\pgfpathlineto{\pgfqpoint{3.122618in}{3.479456in}}%
\pgfpathlineto{\pgfqpoint{3.126876in}{3.479456in}}%
\pgfpathlineto{\pgfqpoint{3.126876in}{3.475198in}}%
\pgfpathmoveto{\pgfqpoint{3.122618in}{3.479456in}}%
\pgfpathlineto{\pgfqpoint{3.122618in}{3.479456in}}%
\pgfpathlineto{\pgfqpoint{3.122618in}{3.483714in}}%
\pgfpathlineto{\pgfqpoint{3.126876in}{3.483714in}}%
\pgfpathlineto{\pgfqpoint{3.126876in}{3.479456in}}%
\pgfpathmoveto{\pgfqpoint{3.122618in}{3.483714in}}%
\pgfpathlineto{\pgfqpoint{3.122618in}{3.483714in}}%
\pgfpathlineto{\pgfqpoint{3.122618in}{3.487972in}}%
\pgfpathlineto{\pgfqpoint{3.126876in}{3.487972in}}%
\pgfpathlineto{\pgfqpoint{3.126876in}{3.483714in}}%
\pgfpathmoveto{\pgfqpoint{3.114103in}{3.500745in}}%
\pgfpathlineto{\pgfqpoint{3.114103in}{3.500745in}}%
\pgfpathlineto{\pgfqpoint{3.114103in}{3.505003in}}%
\pgfpathlineto{\pgfqpoint{3.118360in}{3.505003in}}%
\pgfpathlineto{\pgfqpoint{3.118360in}{3.500745in}}%
\pgfpathmoveto{\pgfqpoint{3.118360in}{3.487972in}}%
\pgfpathlineto{\pgfqpoint{3.118360in}{3.487972in}}%
\pgfpathlineto{\pgfqpoint{3.118360in}{3.492230in}}%
\pgfpathlineto{\pgfqpoint{3.122618in}{3.492230in}}%
\pgfpathlineto{\pgfqpoint{3.122618in}{3.487972in}}%
\pgfpathmoveto{\pgfqpoint{3.118360in}{3.492230in}}%
\pgfpathlineto{\pgfqpoint{3.118360in}{3.492230in}}%
\pgfpathlineto{\pgfqpoint{3.118360in}{3.496487in}}%
\pgfpathlineto{\pgfqpoint{3.122618in}{3.496487in}}%
\pgfpathlineto{\pgfqpoint{3.122618in}{3.492230in}}%
\pgfpathmoveto{\pgfqpoint{3.122618in}{3.487972in}}%
\pgfpathlineto{\pgfqpoint{3.122618in}{3.487972in}}%
\pgfpathlineto{\pgfqpoint{3.122618in}{3.492230in}}%
\pgfpathlineto{\pgfqpoint{3.126876in}{3.492230in}}%
\pgfpathlineto{\pgfqpoint{3.126876in}{3.487972in}}%
\pgfpathmoveto{\pgfqpoint{3.122618in}{3.492230in}}%
\pgfpathlineto{\pgfqpoint{3.122618in}{3.492230in}}%
\pgfpathlineto{\pgfqpoint{3.122618in}{3.496487in}}%
\pgfpathlineto{\pgfqpoint{3.126876in}{3.496487in}}%
\pgfpathlineto{\pgfqpoint{3.126876in}{3.492230in}}%
\pgfpathmoveto{\pgfqpoint{3.118360in}{3.496487in}}%
\pgfpathlineto{\pgfqpoint{3.118360in}{3.496487in}}%
\pgfpathlineto{\pgfqpoint{3.118360in}{3.500745in}}%
\pgfpathlineto{\pgfqpoint{3.122618in}{3.500745in}}%
\pgfpathlineto{\pgfqpoint{3.122618in}{3.496487in}}%
\pgfpathmoveto{\pgfqpoint{3.118360in}{3.500745in}}%
\pgfpathlineto{\pgfqpoint{3.118360in}{3.500745in}}%
\pgfpathlineto{\pgfqpoint{3.118360in}{3.505003in}}%
\pgfpathlineto{\pgfqpoint{3.122618in}{3.505003in}}%
\pgfpathlineto{\pgfqpoint{3.122618in}{3.500745in}}%
\pgfpathmoveto{\pgfqpoint{3.122618in}{3.496487in}}%
\pgfpathlineto{\pgfqpoint{3.122618in}{3.496487in}}%
\pgfpathlineto{\pgfqpoint{3.122618in}{3.500745in}}%
\pgfpathlineto{\pgfqpoint{3.126876in}{3.500745in}}%
\pgfpathlineto{\pgfqpoint{3.126876in}{3.496487in}}%
\pgfpathmoveto{\pgfqpoint{3.122618in}{3.500745in}}%
\pgfpathlineto{\pgfqpoint{3.122618in}{3.500745in}}%
\pgfpathlineto{\pgfqpoint{3.122618in}{3.505003in}}%
\pgfpathlineto{\pgfqpoint{3.126876in}{3.505003in}}%
\pgfpathlineto{\pgfqpoint{3.126876in}{3.500745in}}%
\pgfpathmoveto{\pgfqpoint{3.131134in}{3.445393in}}%
\pgfpathlineto{\pgfqpoint{3.131134in}{3.445393in}}%
\pgfpathlineto{\pgfqpoint{3.131134in}{3.449651in}}%
\pgfpathlineto{\pgfqpoint{3.135391in}{3.449651in}}%
\pgfpathlineto{\pgfqpoint{3.135391in}{3.445393in}}%
\pgfpathmoveto{\pgfqpoint{3.131134in}{3.449651in}}%
\pgfpathlineto{\pgfqpoint{3.131134in}{3.449651in}}%
\pgfpathlineto{\pgfqpoint{3.131134in}{3.453909in}}%
\pgfpathlineto{\pgfqpoint{3.135391in}{3.453909in}}%
\pgfpathlineto{\pgfqpoint{3.135391in}{3.449651in}}%
\pgfpathmoveto{\pgfqpoint{3.135391in}{3.436877in}}%
\pgfpathlineto{\pgfqpoint{3.135391in}{3.436877in}}%
\pgfpathlineto{\pgfqpoint{3.135391in}{3.441135in}}%
\pgfpathlineto{\pgfqpoint{3.139649in}{3.441135in}}%
\pgfpathlineto{\pgfqpoint{3.139649in}{3.436877in}}%
\pgfpathmoveto{\pgfqpoint{3.135391in}{3.441135in}}%
\pgfpathlineto{\pgfqpoint{3.135391in}{3.441135in}}%
\pgfpathlineto{\pgfqpoint{3.135391in}{3.445393in}}%
\pgfpathlineto{\pgfqpoint{3.139649in}{3.445393in}}%
\pgfpathlineto{\pgfqpoint{3.139649in}{3.441135in}}%
\pgfpathmoveto{\pgfqpoint{3.139649in}{3.436877in}}%
\pgfpathlineto{\pgfqpoint{3.139649in}{3.436877in}}%
\pgfpathlineto{\pgfqpoint{3.139649in}{3.441135in}}%
\pgfpathlineto{\pgfqpoint{3.143907in}{3.441135in}}%
\pgfpathlineto{\pgfqpoint{3.143907in}{3.436877in}}%
\pgfpathmoveto{\pgfqpoint{3.139649in}{3.441135in}}%
\pgfpathlineto{\pgfqpoint{3.139649in}{3.441135in}}%
\pgfpathlineto{\pgfqpoint{3.139649in}{3.445393in}}%
\pgfpathlineto{\pgfqpoint{3.143907in}{3.445393in}}%
\pgfpathlineto{\pgfqpoint{3.143907in}{3.441135in}}%
\pgfpathmoveto{\pgfqpoint{3.135391in}{3.445393in}}%
\pgfpathlineto{\pgfqpoint{3.135391in}{3.445393in}}%
\pgfpathlineto{\pgfqpoint{3.135391in}{3.449651in}}%
\pgfpathlineto{\pgfqpoint{3.139649in}{3.449651in}}%
\pgfpathlineto{\pgfqpoint{3.139649in}{3.445393in}}%
\pgfpathmoveto{\pgfqpoint{3.135391in}{3.449651in}}%
\pgfpathlineto{\pgfqpoint{3.135391in}{3.449651in}}%
\pgfpathlineto{\pgfqpoint{3.135391in}{3.453909in}}%
\pgfpathlineto{\pgfqpoint{3.139649in}{3.453909in}}%
\pgfpathlineto{\pgfqpoint{3.139649in}{3.449651in}}%
\pgfpathmoveto{\pgfqpoint{3.139649in}{3.445393in}}%
\pgfpathlineto{\pgfqpoint{3.139649in}{3.445393in}}%
\pgfpathlineto{\pgfqpoint{3.139649in}{3.449651in}}%
\pgfpathlineto{\pgfqpoint{3.143907in}{3.449651in}}%
\pgfpathlineto{\pgfqpoint{3.143907in}{3.445393in}}%
\pgfpathmoveto{\pgfqpoint{3.131134in}{3.453909in}}%
\pgfpathlineto{\pgfqpoint{3.131134in}{3.453909in}}%
\pgfpathlineto{\pgfqpoint{3.131134in}{3.458167in}}%
\pgfpathlineto{\pgfqpoint{3.135391in}{3.458167in}}%
\pgfpathlineto{\pgfqpoint{3.135391in}{3.453909in}}%
\pgfpathmoveto{\pgfqpoint{3.131134in}{3.458167in}}%
\pgfpathlineto{\pgfqpoint{3.131134in}{3.458167in}}%
\pgfpathlineto{\pgfqpoint{3.131134in}{3.462424in}}%
\pgfpathlineto{\pgfqpoint{3.135391in}{3.462424in}}%
\pgfpathlineto{\pgfqpoint{3.135391in}{3.458167in}}%
\pgfpathmoveto{\pgfqpoint{3.126876in}{3.462424in}}%
\pgfpathlineto{\pgfqpoint{3.126876in}{3.462424in}}%
\pgfpathlineto{\pgfqpoint{3.126876in}{3.466682in}}%
\pgfpathlineto{\pgfqpoint{3.131134in}{3.466682in}}%
\pgfpathlineto{\pgfqpoint{3.131134in}{3.462424in}}%
\pgfpathmoveto{\pgfqpoint{3.126876in}{3.466682in}}%
\pgfpathlineto{\pgfqpoint{3.126876in}{3.466682in}}%
\pgfpathlineto{\pgfqpoint{3.126876in}{3.470940in}}%
\pgfpathlineto{\pgfqpoint{3.131134in}{3.470940in}}%
\pgfpathlineto{\pgfqpoint{3.131134in}{3.466682in}}%
\pgfpathmoveto{\pgfqpoint{3.131134in}{3.462424in}}%
\pgfpathlineto{\pgfqpoint{3.131134in}{3.462424in}}%
\pgfpathlineto{\pgfqpoint{3.131134in}{3.466682in}}%
\pgfpathlineto{\pgfqpoint{3.135391in}{3.466682in}}%
\pgfpathlineto{\pgfqpoint{3.135391in}{3.462424in}}%
\pgfpathmoveto{\pgfqpoint{3.131134in}{3.466682in}}%
\pgfpathlineto{\pgfqpoint{3.131134in}{3.466682in}}%
\pgfpathlineto{\pgfqpoint{3.131134in}{3.470940in}}%
\pgfpathlineto{\pgfqpoint{3.135391in}{3.470940in}}%
\pgfpathlineto{\pgfqpoint{3.135391in}{3.466682in}}%
\pgfpathmoveto{\pgfqpoint{3.135391in}{3.453909in}}%
\pgfpathlineto{\pgfqpoint{3.135391in}{3.453909in}}%
\pgfpathlineto{\pgfqpoint{3.135391in}{3.458167in}}%
\pgfpathlineto{\pgfqpoint{3.139649in}{3.458167in}}%
\pgfpathlineto{\pgfqpoint{3.139649in}{3.453909in}}%
\pgfpathmoveto{\pgfqpoint{3.135391in}{3.458167in}}%
\pgfpathlineto{\pgfqpoint{3.135391in}{3.458167in}}%
\pgfpathlineto{\pgfqpoint{3.135391in}{3.462424in}}%
\pgfpathlineto{\pgfqpoint{3.139649in}{3.462424in}}%
\pgfpathlineto{\pgfqpoint{3.139649in}{3.458167in}}%
\pgfpathmoveto{\pgfqpoint{3.135391in}{3.462424in}}%
\pgfpathlineto{\pgfqpoint{3.135391in}{3.462424in}}%
\pgfpathlineto{\pgfqpoint{3.135391in}{3.466682in}}%
\pgfpathlineto{\pgfqpoint{3.139649in}{3.466682in}}%
\pgfpathlineto{\pgfqpoint{3.139649in}{3.462424in}}%
\pgfpathmoveto{\pgfqpoint{3.126876in}{3.470940in}}%
\pgfpathlineto{\pgfqpoint{3.126876in}{3.470940in}}%
\pgfpathlineto{\pgfqpoint{3.126876in}{3.475198in}}%
\pgfpathlineto{\pgfqpoint{3.131134in}{3.475198in}}%
\pgfpathlineto{\pgfqpoint{3.131134in}{3.470940in}}%
\pgfpathmoveto{\pgfqpoint{3.126876in}{3.475198in}}%
\pgfpathlineto{\pgfqpoint{3.126876in}{3.475198in}}%
\pgfpathlineto{\pgfqpoint{3.126876in}{3.479456in}}%
\pgfpathlineto{\pgfqpoint{3.131134in}{3.479456in}}%
\pgfpathlineto{\pgfqpoint{3.131134in}{3.475198in}}%
\pgfpathmoveto{\pgfqpoint{3.131134in}{3.470940in}}%
\pgfpathlineto{\pgfqpoint{3.131134in}{3.470940in}}%
\pgfpathlineto{\pgfqpoint{3.131134in}{3.475198in}}%
\pgfpathlineto{\pgfqpoint{3.135391in}{3.475198in}}%
\pgfpathlineto{\pgfqpoint{3.135391in}{3.470940in}}%
\pgfpathmoveto{\pgfqpoint{3.131134in}{3.475198in}}%
\pgfpathlineto{\pgfqpoint{3.131134in}{3.475198in}}%
\pgfpathlineto{\pgfqpoint{3.131134in}{3.479456in}}%
\pgfpathlineto{\pgfqpoint{3.135391in}{3.479456in}}%
\pgfpathlineto{\pgfqpoint{3.135391in}{3.475198in}}%
\pgfpathmoveto{\pgfqpoint{3.126876in}{3.479456in}}%
\pgfpathlineto{\pgfqpoint{3.126876in}{3.479456in}}%
\pgfpathlineto{\pgfqpoint{3.126876in}{3.483714in}}%
\pgfpathlineto{\pgfqpoint{3.131134in}{3.483714in}}%
\pgfpathlineto{\pgfqpoint{3.131134in}{3.479456in}}%
\pgfpathmoveto{\pgfqpoint{3.126876in}{3.483714in}}%
\pgfpathlineto{\pgfqpoint{3.126876in}{3.483714in}}%
\pgfpathlineto{\pgfqpoint{3.126876in}{3.487972in}}%
\pgfpathlineto{\pgfqpoint{3.131134in}{3.487972in}}%
\pgfpathlineto{\pgfqpoint{3.131134in}{3.483714in}}%
\pgfpathmoveto{\pgfqpoint{3.126876in}{3.487972in}}%
\pgfpathlineto{\pgfqpoint{3.126876in}{3.487972in}}%
\pgfpathlineto{\pgfqpoint{3.126876in}{3.492230in}}%
\pgfpathlineto{\pgfqpoint{3.131134in}{3.492230in}}%
\pgfpathlineto{\pgfqpoint{3.131134in}{3.487972in}}%
\pgfpathmoveto{\pgfqpoint{3.160938in}{3.368751in}}%
\pgfpathlineto{\pgfqpoint{3.160938in}{3.368751in}}%
\pgfpathlineto{\pgfqpoint{3.160938in}{3.373009in}}%
\pgfpathlineto{\pgfqpoint{3.165195in}{3.373009in}}%
\pgfpathlineto{\pgfqpoint{3.165195in}{3.368751in}}%
\pgfpathmoveto{\pgfqpoint{3.105587in}{3.522034in}}%
\pgfpathlineto{\pgfqpoint{3.105587in}{3.522034in}}%
\pgfpathlineto{\pgfqpoint{3.105587in}{3.526292in}}%
\pgfpathlineto{\pgfqpoint{3.109845in}{3.526292in}}%
\pgfpathlineto{\pgfqpoint{3.109845in}{3.522034in}}%
\pgfpathmoveto{\pgfqpoint{3.105587in}{3.526292in}}%
\pgfpathlineto{\pgfqpoint{3.105587in}{3.526292in}}%
\pgfpathlineto{\pgfqpoint{3.105587in}{3.530550in}}%
\pgfpathlineto{\pgfqpoint{3.109845in}{3.530550in}}%
\pgfpathlineto{\pgfqpoint{3.109845in}{3.526292in}}%
\pgfpathmoveto{\pgfqpoint{3.101329in}{3.534807in}}%
\pgfpathlineto{\pgfqpoint{3.101329in}{3.534807in}}%
\pgfpathlineto{\pgfqpoint{3.101329in}{3.539065in}}%
\pgfpathlineto{\pgfqpoint{3.105587in}{3.539065in}}%
\pgfpathlineto{\pgfqpoint{3.105587in}{3.534807in}}%
\pgfpathmoveto{\pgfqpoint{3.105587in}{3.530550in}}%
\pgfpathlineto{\pgfqpoint{3.105587in}{3.530550in}}%
\pgfpathlineto{\pgfqpoint{3.105587in}{3.534807in}}%
\pgfpathlineto{\pgfqpoint{3.109845in}{3.534807in}}%
\pgfpathlineto{\pgfqpoint{3.109845in}{3.530550in}}%
\pgfpathmoveto{\pgfqpoint{3.105587in}{3.534807in}}%
\pgfpathlineto{\pgfqpoint{3.105587in}{3.534807in}}%
\pgfpathlineto{\pgfqpoint{3.105587in}{3.539065in}}%
\pgfpathlineto{\pgfqpoint{3.109845in}{3.539065in}}%
\pgfpathlineto{\pgfqpoint{3.109845in}{3.534807in}}%
\pgfpathmoveto{\pgfqpoint{3.109845in}{3.509261in}}%
\pgfpathlineto{\pgfqpoint{3.109845in}{3.509261in}}%
\pgfpathlineto{\pgfqpoint{3.109845in}{3.513519in}}%
\pgfpathlineto{\pgfqpoint{3.114103in}{3.513519in}}%
\pgfpathlineto{\pgfqpoint{3.114103in}{3.509261in}}%
\pgfpathmoveto{\pgfqpoint{3.114103in}{3.505003in}}%
\pgfpathlineto{\pgfqpoint{3.114103in}{3.505003in}}%
\pgfpathlineto{\pgfqpoint{3.114103in}{3.509261in}}%
\pgfpathlineto{\pgfqpoint{3.118360in}{3.509261in}}%
\pgfpathlineto{\pgfqpoint{3.118360in}{3.505003in}}%
\pgfpathmoveto{\pgfqpoint{3.114103in}{3.509261in}}%
\pgfpathlineto{\pgfqpoint{3.114103in}{3.509261in}}%
\pgfpathlineto{\pgfqpoint{3.114103in}{3.513519in}}%
\pgfpathlineto{\pgfqpoint{3.118360in}{3.513519in}}%
\pgfpathlineto{\pgfqpoint{3.118360in}{3.509261in}}%
\pgfpathmoveto{\pgfqpoint{3.109845in}{3.513519in}}%
\pgfpathlineto{\pgfqpoint{3.109845in}{3.513519in}}%
\pgfpathlineto{\pgfqpoint{3.109845in}{3.517776in}}%
\pgfpathlineto{\pgfqpoint{3.114103in}{3.517776in}}%
\pgfpathlineto{\pgfqpoint{3.114103in}{3.513519in}}%
\pgfpathmoveto{\pgfqpoint{3.109845in}{3.517776in}}%
\pgfpathlineto{\pgfqpoint{3.109845in}{3.517776in}}%
\pgfpathlineto{\pgfqpoint{3.109845in}{3.522034in}}%
\pgfpathlineto{\pgfqpoint{3.114103in}{3.522034in}}%
\pgfpathlineto{\pgfqpoint{3.114103in}{3.517776in}}%
\pgfpathmoveto{\pgfqpoint{3.114103in}{3.513519in}}%
\pgfpathlineto{\pgfqpoint{3.114103in}{3.513519in}}%
\pgfpathlineto{\pgfqpoint{3.114103in}{3.517776in}}%
\pgfpathlineto{\pgfqpoint{3.118360in}{3.517776in}}%
\pgfpathlineto{\pgfqpoint{3.118360in}{3.513519in}}%
\pgfpathmoveto{\pgfqpoint{3.114103in}{3.517776in}}%
\pgfpathlineto{\pgfqpoint{3.114103in}{3.517776in}}%
\pgfpathlineto{\pgfqpoint{3.114103in}{3.522034in}}%
\pgfpathlineto{\pgfqpoint{3.118360in}{3.522034in}}%
\pgfpathlineto{\pgfqpoint{3.118360in}{3.517776in}}%
\pgfpathmoveto{\pgfqpoint{3.118360in}{3.505003in}}%
\pgfpathlineto{\pgfqpoint{3.118360in}{3.505003in}}%
\pgfpathlineto{\pgfqpoint{3.118360in}{3.509261in}}%
\pgfpathlineto{\pgfqpoint{3.122618in}{3.509261in}}%
\pgfpathlineto{\pgfqpoint{3.122618in}{3.505003in}}%
\pgfpathmoveto{\pgfqpoint{3.118360in}{3.509261in}}%
\pgfpathlineto{\pgfqpoint{3.118360in}{3.509261in}}%
\pgfpathlineto{\pgfqpoint{3.118360in}{3.513519in}}%
\pgfpathlineto{\pgfqpoint{3.122618in}{3.513519in}}%
\pgfpathlineto{\pgfqpoint{3.122618in}{3.509261in}}%
\pgfpathmoveto{\pgfqpoint{3.122618in}{3.505003in}}%
\pgfpathlineto{\pgfqpoint{3.122618in}{3.505003in}}%
\pgfpathlineto{\pgfqpoint{3.122618in}{3.509261in}}%
\pgfpathlineto{\pgfqpoint{3.126876in}{3.509261in}}%
\pgfpathlineto{\pgfqpoint{3.126876in}{3.505003in}}%
\pgfpathmoveto{\pgfqpoint{3.118360in}{3.513519in}}%
\pgfpathlineto{\pgfqpoint{3.118360in}{3.513519in}}%
\pgfpathlineto{\pgfqpoint{3.118360in}{3.517776in}}%
\pgfpathlineto{\pgfqpoint{3.122618in}{3.517776in}}%
\pgfpathlineto{\pgfqpoint{3.122618in}{3.513519in}}%
\pgfpathmoveto{\pgfqpoint{3.118360in}{3.517776in}}%
\pgfpathlineto{\pgfqpoint{3.118360in}{3.517776in}}%
\pgfpathlineto{\pgfqpoint{3.118360in}{3.522034in}}%
\pgfpathlineto{\pgfqpoint{3.122618in}{3.522034in}}%
\pgfpathlineto{\pgfqpoint{3.122618in}{3.517776in}}%
\pgfpathmoveto{\pgfqpoint{3.109845in}{3.522034in}}%
\pgfpathlineto{\pgfqpoint{3.109845in}{3.522034in}}%
\pgfpathlineto{\pgfqpoint{3.109845in}{3.526292in}}%
\pgfpathlineto{\pgfqpoint{3.114103in}{3.526292in}}%
\pgfpathlineto{\pgfqpoint{3.114103in}{3.522034in}}%
\pgfpathmoveto{\pgfqpoint{3.109845in}{3.526292in}}%
\pgfpathlineto{\pgfqpoint{3.109845in}{3.526292in}}%
\pgfpathlineto{\pgfqpoint{3.109845in}{3.530550in}}%
\pgfpathlineto{\pgfqpoint{3.114103in}{3.530550in}}%
\pgfpathlineto{\pgfqpoint{3.114103in}{3.526292in}}%
\pgfpathmoveto{\pgfqpoint{3.114103in}{3.522034in}}%
\pgfpathlineto{\pgfqpoint{3.114103in}{3.522034in}}%
\pgfpathlineto{\pgfqpoint{3.114103in}{3.526292in}}%
\pgfpathlineto{\pgfqpoint{3.118360in}{3.526292in}}%
\pgfpathlineto{\pgfqpoint{3.118360in}{3.522034in}}%
\pgfpathmoveto{\pgfqpoint{3.114103in}{3.526292in}}%
\pgfpathlineto{\pgfqpoint{3.114103in}{3.526292in}}%
\pgfpathlineto{\pgfqpoint{3.114103in}{3.530550in}}%
\pgfpathlineto{\pgfqpoint{3.118360in}{3.530550in}}%
\pgfpathlineto{\pgfqpoint{3.118360in}{3.526292in}}%
\pgfpathmoveto{\pgfqpoint{3.109845in}{3.530550in}}%
\pgfpathlineto{\pgfqpoint{3.109845in}{3.530550in}}%
\pgfpathlineto{\pgfqpoint{3.109845in}{3.534807in}}%
\pgfpathlineto{\pgfqpoint{3.114103in}{3.534807in}}%
\pgfpathlineto{\pgfqpoint{3.114103in}{3.530550in}}%
\pgfpathmoveto{\pgfqpoint{3.109845in}{3.534807in}}%
\pgfpathlineto{\pgfqpoint{3.109845in}{3.534807in}}%
\pgfpathlineto{\pgfqpoint{3.109845in}{3.539065in}}%
\pgfpathlineto{\pgfqpoint{3.114103in}{3.539065in}}%
\pgfpathlineto{\pgfqpoint{3.114103in}{3.534807in}}%
\pgfpathmoveto{\pgfqpoint{3.114103in}{3.530550in}}%
\pgfpathlineto{\pgfqpoint{3.114103in}{3.530550in}}%
\pgfpathlineto{\pgfqpoint{3.114103in}{3.534807in}}%
\pgfpathlineto{\pgfqpoint{3.118360in}{3.534807in}}%
\pgfpathlineto{\pgfqpoint{3.118360in}{3.530550in}}%
\pgfpathmoveto{\pgfqpoint{3.097072in}{3.547580in}}%
\pgfpathlineto{\pgfqpoint{3.097072in}{3.547580in}}%
\pgfpathlineto{\pgfqpoint{3.097072in}{3.551838in}}%
\pgfpathlineto{\pgfqpoint{3.101329in}{3.551838in}}%
\pgfpathlineto{\pgfqpoint{3.101329in}{3.547580in}}%
\pgfpathmoveto{\pgfqpoint{3.097072in}{3.551838in}}%
\pgfpathlineto{\pgfqpoint{3.097072in}{3.551838in}}%
\pgfpathlineto{\pgfqpoint{3.097072in}{3.556096in}}%
\pgfpathlineto{\pgfqpoint{3.101329in}{3.556096in}}%
\pgfpathlineto{\pgfqpoint{3.101329in}{3.551838in}}%
\pgfpathmoveto{\pgfqpoint{3.101329in}{3.539065in}}%
\pgfpathlineto{\pgfqpoint{3.101329in}{3.539065in}}%
\pgfpathlineto{\pgfqpoint{3.101329in}{3.543323in}}%
\pgfpathlineto{\pgfqpoint{3.105587in}{3.543323in}}%
\pgfpathlineto{\pgfqpoint{3.105587in}{3.539065in}}%
\pgfpathmoveto{\pgfqpoint{3.101329in}{3.543323in}}%
\pgfpathlineto{\pgfqpoint{3.101329in}{3.543323in}}%
\pgfpathlineto{\pgfqpoint{3.101329in}{3.547580in}}%
\pgfpathlineto{\pgfqpoint{3.105587in}{3.547580in}}%
\pgfpathlineto{\pgfqpoint{3.105587in}{3.543323in}}%
\pgfpathmoveto{\pgfqpoint{3.105587in}{3.539065in}}%
\pgfpathlineto{\pgfqpoint{3.105587in}{3.539065in}}%
\pgfpathlineto{\pgfqpoint{3.105587in}{3.543323in}}%
\pgfpathlineto{\pgfqpoint{3.109845in}{3.543323in}}%
\pgfpathlineto{\pgfqpoint{3.109845in}{3.539065in}}%
\pgfpathmoveto{\pgfqpoint{3.105587in}{3.543323in}}%
\pgfpathlineto{\pgfqpoint{3.105587in}{3.543323in}}%
\pgfpathlineto{\pgfqpoint{3.105587in}{3.547580in}}%
\pgfpathlineto{\pgfqpoint{3.109845in}{3.547580in}}%
\pgfpathlineto{\pgfqpoint{3.109845in}{3.543323in}}%
\pgfpathmoveto{\pgfqpoint{3.101329in}{3.547580in}}%
\pgfpathlineto{\pgfqpoint{3.101329in}{3.547580in}}%
\pgfpathlineto{\pgfqpoint{3.101329in}{3.551838in}}%
\pgfpathlineto{\pgfqpoint{3.105587in}{3.551838in}}%
\pgfpathlineto{\pgfqpoint{3.105587in}{3.547580in}}%
\pgfpathmoveto{\pgfqpoint{3.101329in}{3.551838in}}%
\pgfpathlineto{\pgfqpoint{3.101329in}{3.551838in}}%
\pgfpathlineto{\pgfqpoint{3.101329in}{3.556096in}}%
\pgfpathlineto{\pgfqpoint{3.105587in}{3.556096in}}%
\pgfpathlineto{\pgfqpoint{3.105587in}{3.551838in}}%
\pgfpathmoveto{\pgfqpoint{3.105587in}{3.547580in}}%
\pgfpathlineto{\pgfqpoint{3.105587in}{3.547580in}}%
\pgfpathlineto{\pgfqpoint{3.105587in}{3.551838in}}%
\pgfpathlineto{\pgfqpoint{3.109845in}{3.551838in}}%
\pgfpathlineto{\pgfqpoint{3.109845in}{3.547580in}}%
\pgfpathmoveto{\pgfqpoint{3.105587in}{3.551838in}}%
\pgfpathlineto{\pgfqpoint{3.105587in}{3.551838in}}%
\pgfpathlineto{\pgfqpoint{3.105587in}{3.556096in}}%
\pgfpathlineto{\pgfqpoint{3.109845in}{3.556096in}}%
\pgfpathlineto{\pgfqpoint{3.109845in}{3.551838in}}%
\pgfpathmoveto{\pgfqpoint{3.092814in}{3.556096in}}%
\pgfpathlineto{\pgfqpoint{3.092814in}{3.556096in}}%
\pgfpathlineto{\pgfqpoint{3.092814in}{3.560354in}}%
\pgfpathlineto{\pgfqpoint{3.097072in}{3.560354in}}%
\pgfpathlineto{\pgfqpoint{3.097072in}{3.556096in}}%
\pgfpathmoveto{\pgfqpoint{3.092814in}{3.560354in}}%
\pgfpathlineto{\pgfqpoint{3.092814in}{3.560354in}}%
\pgfpathlineto{\pgfqpoint{3.092814in}{3.564611in}}%
\pgfpathlineto{\pgfqpoint{3.097072in}{3.564611in}}%
\pgfpathlineto{\pgfqpoint{3.097072in}{3.560354in}}%
\pgfpathmoveto{\pgfqpoint{3.097072in}{3.556096in}}%
\pgfpathlineto{\pgfqpoint{3.097072in}{3.556096in}}%
\pgfpathlineto{\pgfqpoint{3.097072in}{3.560354in}}%
\pgfpathlineto{\pgfqpoint{3.101329in}{3.560354in}}%
\pgfpathlineto{\pgfqpoint{3.101329in}{3.556096in}}%
\pgfpathmoveto{\pgfqpoint{3.097072in}{3.560354in}}%
\pgfpathlineto{\pgfqpoint{3.097072in}{3.560354in}}%
\pgfpathlineto{\pgfqpoint{3.097072in}{3.564611in}}%
\pgfpathlineto{\pgfqpoint{3.101329in}{3.564611in}}%
\pgfpathlineto{\pgfqpoint{3.101329in}{3.560354in}}%
\pgfpathmoveto{\pgfqpoint{3.092814in}{3.564611in}}%
\pgfpathlineto{\pgfqpoint{3.092814in}{3.564611in}}%
\pgfpathlineto{\pgfqpoint{3.092814in}{3.568869in}}%
\pgfpathlineto{\pgfqpoint{3.097072in}{3.568869in}}%
\pgfpathlineto{\pgfqpoint{3.097072in}{3.564611in}}%
\pgfpathmoveto{\pgfqpoint{3.092814in}{3.568869in}}%
\pgfpathlineto{\pgfqpoint{3.092814in}{3.568869in}}%
\pgfpathlineto{\pgfqpoint{3.092814in}{3.573127in}}%
\pgfpathlineto{\pgfqpoint{3.097072in}{3.573127in}}%
\pgfpathlineto{\pgfqpoint{3.097072in}{3.568869in}}%
\pgfpathmoveto{\pgfqpoint{3.097072in}{3.564611in}}%
\pgfpathlineto{\pgfqpoint{3.097072in}{3.564611in}}%
\pgfpathlineto{\pgfqpoint{3.097072in}{3.568869in}}%
\pgfpathlineto{\pgfqpoint{3.101329in}{3.568869in}}%
\pgfpathlineto{\pgfqpoint{3.101329in}{3.564611in}}%
\pgfpathmoveto{\pgfqpoint{3.097072in}{3.568869in}}%
\pgfpathlineto{\pgfqpoint{3.097072in}{3.568869in}}%
\pgfpathlineto{\pgfqpoint{3.097072in}{3.573127in}}%
\pgfpathlineto{\pgfqpoint{3.101329in}{3.573127in}}%
\pgfpathlineto{\pgfqpoint{3.101329in}{3.568869in}}%
\pgfpathmoveto{\pgfqpoint{3.101329in}{3.556096in}}%
\pgfpathlineto{\pgfqpoint{3.101329in}{3.556096in}}%
\pgfpathlineto{\pgfqpoint{3.101329in}{3.560354in}}%
\pgfpathlineto{\pgfqpoint{3.105587in}{3.560354in}}%
\pgfpathlineto{\pgfqpoint{3.105587in}{3.556096in}}%
\pgfpathmoveto{\pgfqpoint{3.101329in}{3.560354in}}%
\pgfpathlineto{\pgfqpoint{3.101329in}{3.560354in}}%
\pgfpathlineto{\pgfqpoint{3.101329in}{3.564611in}}%
\pgfpathlineto{\pgfqpoint{3.105587in}{3.564611in}}%
\pgfpathlineto{\pgfqpoint{3.105587in}{3.560354in}}%
\pgfpathmoveto{\pgfqpoint{3.101329in}{3.564611in}}%
\pgfpathlineto{\pgfqpoint{3.101329in}{3.564611in}}%
\pgfpathlineto{\pgfqpoint{3.101329in}{3.568869in}}%
\pgfpathlineto{\pgfqpoint{3.105587in}{3.568869in}}%
\pgfpathlineto{\pgfqpoint{3.105587in}{3.564611in}}%
\pgfpathmoveto{\pgfqpoint{3.109845in}{3.539065in}}%
\pgfpathlineto{\pgfqpoint{3.109845in}{3.539065in}}%
\pgfpathlineto{\pgfqpoint{3.109845in}{3.543323in}}%
\pgfpathlineto{\pgfqpoint{3.114103in}{3.543323in}}%
\pgfpathlineto{\pgfqpoint{3.114103in}{3.539065in}}%
\pgfpathmoveto{\pgfqpoint{3.109845in}{3.543323in}}%
\pgfpathlineto{\pgfqpoint{3.109845in}{3.543323in}}%
\pgfpathlineto{\pgfqpoint{3.109845in}{3.547580in}}%
\pgfpathlineto{\pgfqpoint{3.114103in}{3.547580in}}%
\pgfpathlineto{\pgfqpoint{3.114103in}{3.543323in}}%
\pgfpathmoveto{\pgfqpoint{3.092814in}{3.573127in}}%
\pgfpathlineto{\pgfqpoint{3.092814in}{3.573127in}}%
\pgfpathlineto{\pgfqpoint{3.092814in}{3.577385in}}%
\pgfpathlineto{\pgfqpoint{3.097072in}{3.577385in}}%
\pgfpathlineto{\pgfqpoint{3.097072in}{3.573127in}}%
\pgfpathmoveto{\pgfqpoint{3.092814in}{3.577385in}}%
\pgfpathlineto{\pgfqpoint{3.092814in}{3.577385in}}%
\pgfpathlineto{\pgfqpoint{3.092814in}{3.581642in}}%
\pgfpathlineto{\pgfqpoint{3.097072in}{3.581642in}}%
\pgfpathlineto{\pgfqpoint{3.097072in}{3.577385in}}%
\pgfpathmoveto{\pgfqpoint{3.097072in}{3.573127in}}%
\pgfpathlineto{\pgfqpoint{3.097072in}{3.573127in}}%
\pgfpathlineto{\pgfqpoint{3.097072in}{3.577385in}}%
\pgfpathlineto{\pgfqpoint{3.101329in}{3.577385in}}%
\pgfpathlineto{\pgfqpoint{3.101329in}{3.573127in}}%
\pgfpathmoveto{\pgfqpoint{3.097072in}{3.577385in}}%
\pgfpathlineto{\pgfqpoint{3.097072in}{3.577385in}}%
\pgfpathlineto{\pgfqpoint{3.097072in}{3.581642in}}%
\pgfpathlineto{\pgfqpoint{3.101329in}{3.581642in}}%
\pgfpathlineto{\pgfqpoint{3.101329in}{3.577385in}}%
\pgfpathmoveto{\pgfqpoint{3.092814in}{3.581642in}}%
\pgfpathlineto{\pgfqpoint{3.092814in}{3.581642in}}%
\pgfpathlineto{\pgfqpoint{3.092814in}{3.585900in}}%
\pgfpathlineto{\pgfqpoint{3.097072in}{3.585900in}}%
\pgfpathlineto{\pgfqpoint{3.097072in}{3.581642in}}%
\pgfpathmoveto{\pgfqpoint{3.092814in}{3.585900in}}%
\pgfpathlineto{\pgfqpoint{3.092814in}{3.585900in}}%
\pgfpathlineto{\pgfqpoint{3.092814in}{3.590158in}}%
\pgfpathlineto{\pgfqpoint{3.097072in}{3.590158in}}%
\pgfpathlineto{\pgfqpoint{3.097072in}{3.585900in}}%
\pgfpathmoveto{\pgfqpoint{3.092814in}{3.590158in}}%
\pgfpathlineto{\pgfqpoint{3.092814in}{3.590158in}}%
\pgfpathlineto{\pgfqpoint{3.092814in}{3.594415in}}%
\pgfpathlineto{\pgfqpoint{3.097072in}{3.594415in}}%
\pgfpathlineto{\pgfqpoint{3.097072in}{3.590158in}}%
\pgfpathmoveto{\pgfqpoint{3.361051in}{2.214884in}}%
\pgfpathlineto{\pgfqpoint{3.361051in}{2.214884in}}%
\pgfpathlineto{\pgfqpoint{3.361051in}{2.219142in}}%
\pgfpathlineto{\pgfqpoint{3.365309in}{2.219142in}}%
\pgfpathlineto{\pgfqpoint{3.365309in}{2.214884in}}%
\pgfpathmoveto{\pgfqpoint{3.361051in}{2.219142in}}%
\pgfpathlineto{\pgfqpoint{3.361051in}{2.219142in}}%
\pgfpathlineto{\pgfqpoint{3.361051in}{2.223400in}}%
\pgfpathlineto{\pgfqpoint{3.365309in}{2.223400in}}%
\pgfpathlineto{\pgfqpoint{3.365309in}{2.219142in}}%
\pgfpathmoveto{\pgfqpoint{3.361051in}{2.223400in}}%
\pgfpathlineto{\pgfqpoint{3.361051in}{2.223400in}}%
\pgfpathlineto{\pgfqpoint{3.361051in}{2.227658in}}%
\pgfpathlineto{\pgfqpoint{3.365309in}{2.227658in}}%
\pgfpathlineto{\pgfqpoint{3.365309in}{2.223400in}}%
\pgfpathmoveto{\pgfqpoint{3.361051in}{2.227658in}}%
\pgfpathlineto{\pgfqpoint{3.361051in}{2.227658in}}%
\pgfpathlineto{\pgfqpoint{3.361051in}{2.231916in}}%
\pgfpathlineto{\pgfqpoint{3.365309in}{2.231916in}}%
\pgfpathlineto{\pgfqpoint{3.365309in}{2.227658in}}%
\pgfpathmoveto{\pgfqpoint{3.361051in}{2.231916in}}%
\pgfpathlineto{\pgfqpoint{3.361051in}{2.231916in}}%
\pgfpathlineto{\pgfqpoint{3.361051in}{2.236174in}}%
\pgfpathlineto{\pgfqpoint{3.365309in}{2.236174in}}%
\pgfpathlineto{\pgfqpoint{3.365309in}{2.231916in}}%
\pgfpathmoveto{\pgfqpoint{3.361051in}{2.236174in}}%
\pgfpathlineto{\pgfqpoint{3.361051in}{2.236174in}}%
\pgfpathlineto{\pgfqpoint{3.361051in}{2.240432in}}%
\pgfpathlineto{\pgfqpoint{3.365309in}{2.240432in}}%
\pgfpathlineto{\pgfqpoint{3.365309in}{2.236174in}}%
\pgfpathmoveto{\pgfqpoint{3.361051in}{2.240432in}}%
\pgfpathlineto{\pgfqpoint{3.361051in}{2.240432in}}%
\pgfpathlineto{\pgfqpoint{3.361051in}{2.244690in}}%
\pgfpathlineto{\pgfqpoint{3.365309in}{2.244690in}}%
\pgfpathlineto{\pgfqpoint{3.365309in}{2.240432in}}%
\pgfpathmoveto{\pgfqpoint{3.356793in}{2.248948in}}%
\pgfpathlineto{\pgfqpoint{3.356793in}{2.248948in}}%
\pgfpathlineto{\pgfqpoint{3.356793in}{2.253206in}}%
\pgfpathlineto{\pgfqpoint{3.361051in}{2.253206in}}%
\pgfpathlineto{\pgfqpoint{3.361051in}{2.248948in}}%
\pgfpathmoveto{\pgfqpoint{3.361051in}{2.244690in}}%
\pgfpathlineto{\pgfqpoint{3.361051in}{2.244690in}}%
\pgfpathlineto{\pgfqpoint{3.361051in}{2.248948in}}%
\pgfpathlineto{\pgfqpoint{3.365309in}{2.248948in}}%
\pgfpathlineto{\pgfqpoint{3.365309in}{2.244690in}}%
\pgfpathmoveto{\pgfqpoint{3.361051in}{2.248948in}}%
\pgfpathlineto{\pgfqpoint{3.361051in}{2.248948in}}%
\pgfpathlineto{\pgfqpoint{3.361051in}{2.253206in}}%
\pgfpathlineto{\pgfqpoint{3.365309in}{2.253206in}}%
\pgfpathlineto{\pgfqpoint{3.365309in}{2.248948in}}%
\pgfpathmoveto{\pgfqpoint{3.356793in}{2.253206in}}%
\pgfpathlineto{\pgfqpoint{3.356793in}{2.253206in}}%
\pgfpathlineto{\pgfqpoint{3.356793in}{2.257464in}}%
\pgfpathlineto{\pgfqpoint{3.361051in}{2.257464in}}%
\pgfpathlineto{\pgfqpoint{3.361051in}{2.253206in}}%
\pgfpathmoveto{\pgfqpoint{3.356793in}{2.257464in}}%
\pgfpathlineto{\pgfqpoint{3.356793in}{2.257464in}}%
\pgfpathlineto{\pgfqpoint{3.356793in}{2.261722in}}%
\pgfpathlineto{\pgfqpoint{3.361051in}{2.261722in}}%
\pgfpathlineto{\pgfqpoint{3.361051in}{2.257464in}}%
\pgfpathmoveto{\pgfqpoint{3.356793in}{2.261722in}}%
\pgfpathlineto{\pgfqpoint{3.356793in}{2.261722in}}%
\pgfpathlineto{\pgfqpoint{3.356793in}{2.265980in}}%
\pgfpathlineto{\pgfqpoint{3.361051in}{2.265980in}}%
\pgfpathlineto{\pgfqpoint{3.361051in}{2.261722in}}%
\pgfpathmoveto{\pgfqpoint{3.356793in}{2.265980in}}%
\pgfpathlineto{\pgfqpoint{3.356793in}{2.265980in}}%
\pgfpathlineto{\pgfqpoint{3.356793in}{2.270238in}}%
\pgfpathlineto{\pgfqpoint{3.361051in}{2.270238in}}%
\pgfpathlineto{\pgfqpoint{3.361051in}{2.265980in}}%
\pgfpathmoveto{\pgfqpoint{3.356793in}{2.270238in}}%
\pgfpathlineto{\pgfqpoint{3.356793in}{2.270238in}}%
\pgfpathlineto{\pgfqpoint{3.356793in}{2.274496in}}%
\pgfpathlineto{\pgfqpoint{3.361051in}{2.274496in}}%
\pgfpathlineto{\pgfqpoint{3.361051in}{2.270238in}}%
\pgfpathmoveto{\pgfqpoint{3.356793in}{2.274496in}}%
\pgfpathlineto{\pgfqpoint{3.356793in}{2.274496in}}%
\pgfpathlineto{\pgfqpoint{3.356793in}{2.278754in}}%
\pgfpathlineto{\pgfqpoint{3.361051in}{2.278754in}}%
\pgfpathlineto{\pgfqpoint{3.361051in}{2.274496in}}%
\pgfpathmoveto{\pgfqpoint{3.352535in}{2.278754in}}%
\pgfpathlineto{\pgfqpoint{3.352535in}{2.278754in}}%
\pgfpathlineto{\pgfqpoint{3.352535in}{2.283011in}}%
\pgfpathlineto{\pgfqpoint{3.356793in}{2.283011in}}%
\pgfpathlineto{\pgfqpoint{3.356793in}{2.278754in}}%
\pgfpathmoveto{\pgfqpoint{3.352535in}{2.283011in}}%
\pgfpathlineto{\pgfqpoint{3.352535in}{2.283011in}}%
\pgfpathlineto{\pgfqpoint{3.352535in}{2.287269in}}%
\pgfpathlineto{\pgfqpoint{3.356793in}{2.287269in}}%
\pgfpathlineto{\pgfqpoint{3.356793in}{2.283011in}}%
\pgfpathmoveto{\pgfqpoint{3.352535in}{2.287269in}}%
\pgfpathlineto{\pgfqpoint{3.352535in}{2.287269in}}%
\pgfpathlineto{\pgfqpoint{3.352535in}{2.291526in}}%
\pgfpathlineto{\pgfqpoint{3.356793in}{2.291526in}}%
\pgfpathlineto{\pgfqpoint{3.356793in}{2.287269in}}%
\pgfpathmoveto{\pgfqpoint{3.352535in}{2.291526in}}%
\pgfpathlineto{\pgfqpoint{3.352535in}{2.291526in}}%
\pgfpathlineto{\pgfqpoint{3.352535in}{2.295784in}}%
\pgfpathlineto{\pgfqpoint{3.356793in}{2.295784in}}%
\pgfpathlineto{\pgfqpoint{3.356793in}{2.291526in}}%
\pgfpathmoveto{\pgfqpoint{3.356793in}{2.278754in}}%
\pgfpathlineto{\pgfqpoint{3.356793in}{2.278754in}}%
\pgfpathlineto{\pgfqpoint{3.356793in}{2.283011in}}%
\pgfpathlineto{\pgfqpoint{3.361051in}{2.283011in}}%
\pgfpathlineto{\pgfqpoint{3.361051in}{2.278754in}}%
\pgfpathmoveto{\pgfqpoint{3.352535in}{2.295784in}}%
\pgfpathlineto{\pgfqpoint{3.352535in}{2.295784in}}%
\pgfpathlineto{\pgfqpoint{3.352535in}{2.300042in}}%
\pgfpathlineto{\pgfqpoint{3.356793in}{2.300042in}}%
\pgfpathlineto{\pgfqpoint{3.356793in}{2.295784in}}%
\pgfpathmoveto{\pgfqpoint{3.352535in}{2.300042in}}%
\pgfpathlineto{\pgfqpoint{3.352535in}{2.300042in}}%
\pgfpathlineto{\pgfqpoint{3.352535in}{2.304299in}}%
\pgfpathlineto{\pgfqpoint{3.356793in}{2.304299in}}%
\pgfpathlineto{\pgfqpoint{3.356793in}{2.300042in}}%
\pgfpathmoveto{\pgfqpoint{3.348278in}{2.308557in}}%
\pgfpathlineto{\pgfqpoint{3.348278in}{2.308557in}}%
\pgfpathlineto{\pgfqpoint{3.348278in}{2.312814in}}%
\pgfpathlineto{\pgfqpoint{3.352535in}{2.312814in}}%
\pgfpathlineto{\pgfqpoint{3.352535in}{2.308557in}}%
\pgfpathmoveto{\pgfqpoint{3.352535in}{2.304299in}}%
\pgfpathlineto{\pgfqpoint{3.352535in}{2.304299in}}%
\pgfpathlineto{\pgfqpoint{3.352535in}{2.308557in}}%
\pgfpathlineto{\pgfqpoint{3.356793in}{2.308557in}}%
\pgfpathlineto{\pgfqpoint{3.356793in}{2.304299in}}%
\pgfpathmoveto{\pgfqpoint{3.352535in}{2.308557in}}%
\pgfpathlineto{\pgfqpoint{3.352535in}{2.308557in}}%
\pgfpathlineto{\pgfqpoint{3.352535in}{2.312814in}}%
\pgfpathlineto{\pgfqpoint{3.356793in}{2.312814in}}%
\pgfpathlineto{\pgfqpoint{3.356793in}{2.308557in}}%
\pgfpathmoveto{\pgfqpoint{3.344020in}{2.338360in}}%
\pgfpathlineto{\pgfqpoint{3.344020in}{2.338360in}}%
\pgfpathlineto{\pgfqpoint{3.344020in}{2.342617in}}%
\pgfpathlineto{\pgfqpoint{3.348278in}{2.342617in}}%
\pgfpathlineto{\pgfqpoint{3.348278in}{2.338360in}}%
\pgfpathmoveto{\pgfqpoint{3.344020in}{2.342617in}}%
\pgfpathlineto{\pgfqpoint{3.344020in}{2.342617in}}%
\pgfpathlineto{\pgfqpoint{3.344020in}{2.346875in}}%
\pgfpathlineto{\pgfqpoint{3.348278in}{2.346875in}}%
\pgfpathlineto{\pgfqpoint{3.348278in}{2.342617in}}%
\pgfpathmoveto{\pgfqpoint{3.348278in}{2.312814in}}%
\pgfpathlineto{\pgfqpoint{3.348278in}{2.312814in}}%
\pgfpathlineto{\pgfqpoint{3.348278in}{2.317072in}}%
\pgfpathlineto{\pgfqpoint{3.352535in}{2.317072in}}%
\pgfpathlineto{\pgfqpoint{3.352535in}{2.312814in}}%
\pgfpathmoveto{\pgfqpoint{3.348278in}{2.317072in}}%
\pgfpathlineto{\pgfqpoint{3.348278in}{2.317072in}}%
\pgfpathlineto{\pgfqpoint{3.348278in}{2.321329in}}%
\pgfpathlineto{\pgfqpoint{3.352535in}{2.321329in}}%
\pgfpathlineto{\pgfqpoint{3.352535in}{2.317072in}}%
\pgfpathmoveto{\pgfqpoint{3.348278in}{2.321329in}}%
\pgfpathlineto{\pgfqpoint{3.348278in}{2.321329in}}%
\pgfpathlineto{\pgfqpoint{3.348278in}{2.325587in}}%
\pgfpathlineto{\pgfqpoint{3.352535in}{2.325587in}}%
\pgfpathlineto{\pgfqpoint{3.352535in}{2.321329in}}%
\pgfpathmoveto{\pgfqpoint{3.348278in}{2.325587in}}%
\pgfpathlineto{\pgfqpoint{3.348278in}{2.325587in}}%
\pgfpathlineto{\pgfqpoint{3.348278in}{2.329845in}}%
\pgfpathlineto{\pgfqpoint{3.352535in}{2.329845in}}%
\pgfpathlineto{\pgfqpoint{3.352535in}{2.325587in}}%
\pgfpathmoveto{\pgfqpoint{3.348278in}{2.329845in}}%
\pgfpathlineto{\pgfqpoint{3.348278in}{2.329845in}}%
\pgfpathlineto{\pgfqpoint{3.348278in}{2.334102in}}%
\pgfpathlineto{\pgfqpoint{3.352535in}{2.334102in}}%
\pgfpathlineto{\pgfqpoint{3.352535in}{2.329845in}}%
\pgfpathmoveto{\pgfqpoint{3.348278in}{2.334102in}}%
\pgfpathlineto{\pgfqpoint{3.348278in}{2.334102in}}%
\pgfpathlineto{\pgfqpoint{3.348278in}{2.338360in}}%
\pgfpathlineto{\pgfqpoint{3.352535in}{2.338360in}}%
\pgfpathlineto{\pgfqpoint{3.352535in}{2.334102in}}%
\pgfpathmoveto{\pgfqpoint{3.348278in}{2.338360in}}%
\pgfpathlineto{\pgfqpoint{3.348278in}{2.338360in}}%
\pgfpathlineto{\pgfqpoint{3.348278in}{2.342617in}}%
\pgfpathlineto{\pgfqpoint{3.352535in}{2.342617in}}%
\pgfpathlineto{\pgfqpoint{3.352535in}{2.338360in}}%
\pgfpathmoveto{\pgfqpoint{3.344020in}{2.346875in}}%
\pgfpathlineto{\pgfqpoint{3.344020in}{2.346875in}}%
\pgfpathlineto{\pgfqpoint{3.344020in}{2.351132in}}%
\pgfpathlineto{\pgfqpoint{3.348278in}{2.351132in}}%
\pgfpathlineto{\pgfqpoint{3.348278in}{2.346875in}}%
\pgfpathmoveto{\pgfqpoint{3.344020in}{2.351132in}}%
\pgfpathlineto{\pgfqpoint{3.344020in}{2.351132in}}%
\pgfpathlineto{\pgfqpoint{3.344020in}{2.355390in}}%
\pgfpathlineto{\pgfqpoint{3.348278in}{2.355390in}}%
\pgfpathlineto{\pgfqpoint{3.348278in}{2.351132in}}%
\pgfpathmoveto{\pgfqpoint{3.344020in}{2.355390in}}%
\pgfpathlineto{\pgfqpoint{3.344020in}{2.355390in}}%
\pgfpathlineto{\pgfqpoint{3.344020in}{2.359648in}}%
\pgfpathlineto{\pgfqpoint{3.348278in}{2.359648in}}%
\pgfpathlineto{\pgfqpoint{3.348278in}{2.355390in}}%
\pgfpathmoveto{\pgfqpoint{3.344020in}{2.359648in}}%
\pgfpathlineto{\pgfqpoint{3.344020in}{2.359648in}}%
\pgfpathlineto{\pgfqpoint{3.344020in}{2.363905in}}%
\pgfpathlineto{\pgfqpoint{3.348278in}{2.363905in}}%
\pgfpathlineto{\pgfqpoint{3.348278in}{2.359648in}}%
\pgfpathmoveto{\pgfqpoint{3.344020in}{2.363905in}}%
\pgfpathlineto{\pgfqpoint{3.344020in}{2.363905in}}%
\pgfpathlineto{\pgfqpoint{3.344020in}{2.368163in}}%
\pgfpathlineto{\pgfqpoint{3.348278in}{2.368163in}}%
\pgfpathlineto{\pgfqpoint{3.348278in}{2.363905in}}%
\pgfpathmoveto{\pgfqpoint{3.344020in}{2.368163in}}%
\pgfpathlineto{\pgfqpoint{3.344020in}{2.368163in}}%
\pgfpathlineto{\pgfqpoint{3.344020in}{2.372420in}}%
\pgfpathlineto{\pgfqpoint{3.348278in}{2.372420in}}%
\pgfpathlineto{\pgfqpoint{3.348278in}{2.368163in}}%
\pgfpathmoveto{\pgfqpoint{3.339762in}{2.372420in}}%
\pgfpathlineto{\pgfqpoint{3.339762in}{2.372420in}}%
\pgfpathlineto{\pgfqpoint{3.339762in}{2.376678in}}%
\pgfpathlineto{\pgfqpoint{3.344020in}{2.376678in}}%
\pgfpathlineto{\pgfqpoint{3.344020in}{2.372420in}}%
\pgfpathmoveto{\pgfqpoint{3.339762in}{2.376678in}}%
\pgfpathlineto{\pgfqpoint{3.339762in}{2.376678in}}%
\pgfpathlineto{\pgfqpoint{3.339762in}{2.380936in}}%
\pgfpathlineto{\pgfqpoint{3.344020in}{2.380936in}}%
\pgfpathlineto{\pgfqpoint{3.344020in}{2.376678in}}%
\pgfpathmoveto{\pgfqpoint{3.344020in}{2.372420in}}%
\pgfpathlineto{\pgfqpoint{3.344020in}{2.372420in}}%
\pgfpathlineto{\pgfqpoint{3.344020in}{2.376678in}}%
\pgfpathlineto{\pgfqpoint{3.348278in}{2.376678in}}%
\pgfpathlineto{\pgfqpoint{3.348278in}{2.372420in}}%
\pgfpathmoveto{\pgfqpoint{3.339762in}{2.380936in}}%
\pgfpathlineto{\pgfqpoint{3.339762in}{2.380936in}}%
\pgfpathlineto{\pgfqpoint{3.339762in}{2.385193in}}%
\pgfpathlineto{\pgfqpoint{3.344020in}{2.385193in}}%
\pgfpathlineto{\pgfqpoint{3.344020in}{2.380936in}}%
\pgfpathmoveto{\pgfqpoint{3.339762in}{2.385193in}}%
\pgfpathlineto{\pgfqpoint{3.339762in}{2.385193in}}%
\pgfpathlineto{\pgfqpoint{3.339762in}{2.389451in}}%
\pgfpathlineto{\pgfqpoint{3.344020in}{2.389451in}}%
\pgfpathlineto{\pgfqpoint{3.344020in}{2.385193in}}%
\pgfpathmoveto{\pgfqpoint{3.339762in}{2.389451in}}%
\pgfpathlineto{\pgfqpoint{3.339762in}{2.389451in}}%
\pgfpathlineto{\pgfqpoint{3.339762in}{2.393708in}}%
\pgfpathlineto{\pgfqpoint{3.344020in}{2.393708in}}%
\pgfpathlineto{\pgfqpoint{3.344020in}{2.389451in}}%
\pgfpathmoveto{\pgfqpoint{3.339762in}{2.393708in}}%
\pgfpathlineto{\pgfqpoint{3.339762in}{2.393708in}}%
\pgfpathlineto{\pgfqpoint{3.339762in}{2.397966in}}%
\pgfpathlineto{\pgfqpoint{3.344020in}{2.397966in}}%
\pgfpathlineto{\pgfqpoint{3.344020in}{2.393708in}}%
\pgfpathmoveto{\pgfqpoint{3.335505in}{2.402223in}}%
\pgfpathlineto{\pgfqpoint{3.335505in}{2.402223in}}%
\pgfpathlineto{\pgfqpoint{3.335505in}{2.406481in}}%
\pgfpathlineto{\pgfqpoint{3.339762in}{2.406481in}}%
\pgfpathlineto{\pgfqpoint{3.339762in}{2.402223in}}%
\pgfpathmoveto{\pgfqpoint{3.335505in}{2.406481in}}%
\pgfpathlineto{\pgfqpoint{3.335505in}{2.406481in}}%
\pgfpathlineto{\pgfqpoint{3.335505in}{2.410739in}}%
\pgfpathlineto{\pgfqpoint{3.339762in}{2.410739in}}%
\pgfpathlineto{\pgfqpoint{3.339762in}{2.406481in}}%
\pgfpathmoveto{\pgfqpoint{3.335505in}{2.410739in}}%
\pgfpathlineto{\pgfqpoint{3.335505in}{2.410739in}}%
\pgfpathlineto{\pgfqpoint{3.335505in}{2.414996in}}%
\pgfpathlineto{\pgfqpoint{3.339762in}{2.414996in}}%
\pgfpathlineto{\pgfqpoint{3.339762in}{2.410739in}}%
\pgfpathmoveto{\pgfqpoint{3.339762in}{2.397966in}}%
\pgfpathlineto{\pgfqpoint{3.339762in}{2.397966in}}%
\pgfpathlineto{\pgfqpoint{3.339762in}{2.402223in}}%
\pgfpathlineto{\pgfqpoint{3.344020in}{2.402223in}}%
\pgfpathlineto{\pgfqpoint{3.344020in}{2.397966in}}%
\pgfpathmoveto{\pgfqpoint{3.339762in}{2.402223in}}%
\pgfpathlineto{\pgfqpoint{3.339762in}{2.402223in}}%
\pgfpathlineto{\pgfqpoint{3.339762in}{2.406481in}}%
\pgfpathlineto{\pgfqpoint{3.344020in}{2.406481in}}%
\pgfpathlineto{\pgfqpoint{3.344020in}{2.402223in}}%
\pgfpathmoveto{\pgfqpoint{3.326989in}{2.461834in}}%
\pgfpathlineto{\pgfqpoint{3.326989in}{2.461834in}}%
\pgfpathlineto{\pgfqpoint{3.326989in}{2.466092in}}%
\pgfpathlineto{\pgfqpoint{3.331247in}{2.466092in}}%
\pgfpathlineto{\pgfqpoint{3.331247in}{2.461834in}}%
\pgfpathmoveto{\pgfqpoint{3.326989in}{2.466092in}}%
\pgfpathlineto{\pgfqpoint{3.326989in}{2.466092in}}%
\pgfpathlineto{\pgfqpoint{3.326989in}{2.470350in}}%
\pgfpathlineto{\pgfqpoint{3.331247in}{2.470350in}}%
\pgfpathlineto{\pgfqpoint{3.331247in}{2.466092in}}%
\pgfpathmoveto{\pgfqpoint{3.326989in}{2.470350in}}%
\pgfpathlineto{\pgfqpoint{3.326989in}{2.470350in}}%
\pgfpathlineto{\pgfqpoint{3.326989in}{2.474608in}}%
\pgfpathlineto{\pgfqpoint{3.331247in}{2.474608in}}%
\pgfpathlineto{\pgfqpoint{3.331247in}{2.470350in}}%
\pgfpathmoveto{\pgfqpoint{3.326989in}{2.474608in}}%
\pgfpathlineto{\pgfqpoint{3.326989in}{2.474608in}}%
\pgfpathlineto{\pgfqpoint{3.326989in}{2.478866in}}%
\pgfpathlineto{\pgfqpoint{3.331247in}{2.478866in}}%
\pgfpathlineto{\pgfqpoint{3.331247in}{2.474608in}}%
\pgfpathmoveto{\pgfqpoint{3.326989in}{2.478866in}}%
\pgfpathlineto{\pgfqpoint{3.326989in}{2.478866in}}%
\pgfpathlineto{\pgfqpoint{3.326989in}{2.483124in}}%
\pgfpathlineto{\pgfqpoint{3.331247in}{2.483124in}}%
\pgfpathlineto{\pgfqpoint{3.331247in}{2.478866in}}%
\pgfpathmoveto{\pgfqpoint{3.335505in}{2.414996in}}%
\pgfpathlineto{\pgfqpoint{3.335505in}{2.414996in}}%
\pgfpathlineto{\pgfqpoint{3.335505in}{2.419254in}}%
\pgfpathlineto{\pgfqpoint{3.339762in}{2.419254in}}%
\pgfpathlineto{\pgfqpoint{3.339762in}{2.414996in}}%
\pgfpathmoveto{\pgfqpoint{3.335505in}{2.419254in}}%
\pgfpathlineto{\pgfqpoint{3.335505in}{2.419254in}}%
\pgfpathlineto{\pgfqpoint{3.335505in}{2.423512in}}%
\pgfpathlineto{\pgfqpoint{3.339762in}{2.423512in}}%
\pgfpathlineto{\pgfqpoint{3.339762in}{2.419254in}}%
\pgfpathmoveto{\pgfqpoint{3.335505in}{2.423512in}}%
\pgfpathlineto{\pgfqpoint{3.335505in}{2.423512in}}%
\pgfpathlineto{\pgfqpoint{3.335505in}{2.427770in}}%
\pgfpathlineto{\pgfqpoint{3.339762in}{2.427770in}}%
\pgfpathlineto{\pgfqpoint{3.339762in}{2.423512in}}%
\pgfpathmoveto{\pgfqpoint{3.335505in}{2.427770in}}%
\pgfpathlineto{\pgfqpoint{3.335505in}{2.427770in}}%
\pgfpathlineto{\pgfqpoint{3.335505in}{2.432028in}}%
\pgfpathlineto{\pgfqpoint{3.339762in}{2.432028in}}%
\pgfpathlineto{\pgfqpoint{3.339762in}{2.427770in}}%
\pgfpathmoveto{\pgfqpoint{3.331247in}{2.432028in}}%
\pgfpathlineto{\pgfqpoint{3.331247in}{2.432028in}}%
\pgfpathlineto{\pgfqpoint{3.331247in}{2.436286in}}%
\pgfpathlineto{\pgfqpoint{3.335505in}{2.436286in}}%
\pgfpathlineto{\pgfqpoint{3.335505in}{2.432028in}}%
\pgfpathmoveto{\pgfqpoint{3.331247in}{2.436286in}}%
\pgfpathlineto{\pgfqpoint{3.331247in}{2.436286in}}%
\pgfpathlineto{\pgfqpoint{3.331247in}{2.440544in}}%
\pgfpathlineto{\pgfqpoint{3.335505in}{2.440544in}}%
\pgfpathlineto{\pgfqpoint{3.335505in}{2.436286in}}%
\pgfpathmoveto{\pgfqpoint{3.335505in}{2.432028in}}%
\pgfpathlineto{\pgfqpoint{3.335505in}{2.432028in}}%
\pgfpathlineto{\pgfqpoint{3.335505in}{2.436286in}}%
\pgfpathlineto{\pgfqpoint{3.339762in}{2.436286in}}%
\pgfpathlineto{\pgfqpoint{3.339762in}{2.432028in}}%
\pgfpathmoveto{\pgfqpoint{3.331247in}{2.440544in}}%
\pgfpathlineto{\pgfqpoint{3.331247in}{2.440544in}}%
\pgfpathlineto{\pgfqpoint{3.331247in}{2.444802in}}%
\pgfpathlineto{\pgfqpoint{3.335505in}{2.444802in}}%
\pgfpathlineto{\pgfqpoint{3.335505in}{2.440544in}}%
\pgfpathmoveto{\pgfqpoint{3.331247in}{2.444802in}}%
\pgfpathlineto{\pgfqpoint{3.331247in}{2.444802in}}%
\pgfpathlineto{\pgfqpoint{3.331247in}{2.449060in}}%
\pgfpathlineto{\pgfqpoint{3.335505in}{2.449060in}}%
\pgfpathlineto{\pgfqpoint{3.335505in}{2.444802in}}%
\pgfpathmoveto{\pgfqpoint{3.331247in}{2.449060in}}%
\pgfpathlineto{\pgfqpoint{3.331247in}{2.449060in}}%
\pgfpathlineto{\pgfqpoint{3.331247in}{2.453318in}}%
\pgfpathlineto{\pgfqpoint{3.335505in}{2.453318in}}%
\pgfpathlineto{\pgfqpoint{3.335505in}{2.449060in}}%
\pgfpathmoveto{\pgfqpoint{3.331247in}{2.453318in}}%
\pgfpathlineto{\pgfqpoint{3.331247in}{2.453318in}}%
\pgfpathlineto{\pgfqpoint{3.331247in}{2.457576in}}%
\pgfpathlineto{\pgfqpoint{3.335505in}{2.457576in}}%
\pgfpathlineto{\pgfqpoint{3.335505in}{2.453318in}}%
\pgfpathmoveto{\pgfqpoint{3.331247in}{2.457576in}}%
\pgfpathlineto{\pgfqpoint{3.331247in}{2.457576in}}%
\pgfpathlineto{\pgfqpoint{3.331247in}{2.461834in}}%
\pgfpathlineto{\pgfqpoint{3.335505in}{2.461834in}}%
\pgfpathlineto{\pgfqpoint{3.335505in}{2.457576in}}%
\pgfpathmoveto{\pgfqpoint{3.331247in}{2.461834in}}%
\pgfpathlineto{\pgfqpoint{3.331247in}{2.461834in}}%
\pgfpathlineto{\pgfqpoint{3.331247in}{2.466092in}}%
\pgfpathlineto{\pgfqpoint{3.335505in}{2.466092in}}%
\pgfpathlineto{\pgfqpoint{3.335505in}{2.461834in}}%
\pgfpathmoveto{\pgfqpoint{3.322731in}{2.487382in}}%
\pgfpathlineto{\pgfqpoint{3.322731in}{2.487382in}}%
\pgfpathlineto{\pgfqpoint{3.322731in}{2.491640in}}%
\pgfpathlineto{\pgfqpoint{3.326989in}{2.491640in}}%
\pgfpathlineto{\pgfqpoint{3.326989in}{2.487382in}}%
\pgfpathmoveto{\pgfqpoint{3.326989in}{2.483124in}}%
\pgfpathlineto{\pgfqpoint{3.326989in}{2.483124in}}%
\pgfpathlineto{\pgfqpoint{3.326989in}{2.487382in}}%
\pgfpathlineto{\pgfqpoint{3.331247in}{2.487382in}}%
\pgfpathlineto{\pgfqpoint{3.331247in}{2.483124in}}%
\pgfpathmoveto{\pgfqpoint{3.326989in}{2.487382in}}%
\pgfpathlineto{\pgfqpoint{3.326989in}{2.487382in}}%
\pgfpathlineto{\pgfqpoint{3.326989in}{2.491640in}}%
\pgfpathlineto{\pgfqpoint{3.331247in}{2.491640in}}%
\pgfpathlineto{\pgfqpoint{3.331247in}{2.487382in}}%
\pgfpathmoveto{\pgfqpoint{3.322731in}{2.491640in}}%
\pgfpathlineto{\pgfqpoint{3.322731in}{2.491640in}}%
\pgfpathlineto{\pgfqpoint{3.322731in}{2.495898in}}%
\pgfpathlineto{\pgfqpoint{3.326989in}{2.495898in}}%
\pgfpathlineto{\pgfqpoint{3.326989in}{2.491640in}}%
\pgfpathmoveto{\pgfqpoint{3.322731in}{2.495898in}}%
\pgfpathlineto{\pgfqpoint{3.322731in}{2.495898in}}%
\pgfpathlineto{\pgfqpoint{3.322731in}{2.500156in}}%
\pgfpathlineto{\pgfqpoint{3.326989in}{2.500156in}}%
\pgfpathlineto{\pgfqpoint{3.326989in}{2.495898in}}%
\pgfpathmoveto{\pgfqpoint{3.322731in}{2.500156in}}%
\pgfpathlineto{\pgfqpoint{3.322731in}{2.500156in}}%
\pgfpathlineto{\pgfqpoint{3.322731in}{2.504414in}}%
\pgfpathlineto{\pgfqpoint{3.326989in}{2.504414in}}%
\pgfpathlineto{\pgfqpoint{3.326989in}{2.500156in}}%
\pgfpathmoveto{\pgfqpoint{3.322731in}{2.504414in}}%
\pgfpathlineto{\pgfqpoint{3.322731in}{2.504414in}}%
\pgfpathlineto{\pgfqpoint{3.322731in}{2.508672in}}%
\pgfpathlineto{\pgfqpoint{3.326989in}{2.508672in}}%
\pgfpathlineto{\pgfqpoint{3.326989in}{2.504414in}}%
\pgfpathmoveto{\pgfqpoint{3.322731in}{2.508672in}}%
\pgfpathlineto{\pgfqpoint{3.322731in}{2.508672in}}%
\pgfpathlineto{\pgfqpoint{3.322731in}{2.512930in}}%
\pgfpathlineto{\pgfqpoint{3.326989in}{2.512930in}}%
\pgfpathlineto{\pgfqpoint{3.326989in}{2.508672in}}%
\pgfpathmoveto{\pgfqpoint{3.322731in}{2.512930in}}%
\pgfpathlineto{\pgfqpoint{3.322731in}{2.512930in}}%
\pgfpathlineto{\pgfqpoint{3.322731in}{2.517188in}}%
\pgfpathlineto{\pgfqpoint{3.326989in}{2.517188in}}%
\pgfpathlineto{\pgfqpoint{3.326989in}{2.512930in}}%
\pgfpathmoveto{\pgfqpoint{3.318474in}{2.517188in}}%
\pgfpathlineto{\pgfqpoint{3.318474in}{2.517188in}}%
\pgfpathlineto{\pgfqpoint{3.318474in}{2.521446in}}%
\pgfpathlineto{\pgfqpoint{3.322731in}{2.521446in}}%
\pgfpathlineto{\pgfqpoint{3.322731in}{2.517188in}}%
\pgfpathmoveto{\pgfqpoint{3.318474in}{2.521446in}}%
\pgfpathlineto{\pgfqpoint{3.318474in}{2.521446in}}%
\pgfpathlineto{\pgfqpoint{3.318474in}{2.525704in}}%
\pgfpathlineto{\pgfqpoint{3.322731in}{2.525704in}}%
\pgfpathlineto{\pgfqpoint{3.322731in}{2.521446in}}%
\pgfpathmoveto{\pgfqpoint{3.318474in}{2.525704in}}%
\pgfpathlineto{\pgfqpoint{3.318474in}{2.525704in}}%
\pgfpathlineto{\pgfqpoint{3.318474in}{2.529962in}}%
\pgfpathlineto{\pgfqpoint{3.322731in}{2.529962in}}%
\pgfpathlineto{\pgfqpoint{3.322731in}{2.525704in}}%
\pgfpathmoveto{\pgfqpoint{3.318474in}{2.529962in}}%
\pgfpathlineto{\pgfqpoint{3.318474in}{2.529962in}}%
\pgfpathlineto{\pgfqpoint{3.318474in}{2.534220in}}%
\pgfpathlineto{\pgfqpoint{3.322731in}{2.534220in}}%
\pgfpathlineto{\pgfqpoint{3.322731in}{2.529962in}}%
\pgfpathmoveto{\pgfqpoint{3.322731in}{2.517188in}}%
\pgfpathlineto{\pgfqpoint{3.322731in}{2.517188in}}%
\pgfpathlineto{\pgfqpoint{3.322731in}{2.521446in}}%
\pgfpathlineto{\pgfqpoint{3.326989in}{2.521446in}}%
\pgfpathlineto{\pgfqpoint{3.326989in}{2.517188in}}%
\pgfpathmoveto{\pgfqpoint{3.318474in}{2.534220in}}%
\pgfpathlineto{\pgfqpoint{3.318474in}{2.534220in}}%
\pgfpathlineto{\pgfqpoint{3.318474in}{2.538478in}}%
\pgfpathlineto{\pgfqpoint{3.322731in}{2.538478in}}%
\pgfpathlineto{\pgfqpoint{3.322731in}{2.534220in}}%
\pgfpathmoveto{\pgfqpoint{3.318474in}{2.538478in}}%
\pgfpathlineto{\pgfqpoint{3.318474in}{2.538478in}}%
\pgfpathlineto{\pgfqpoint{3.318474in}{2.542736in}}%
\pgfpathlineto{\pgfqpoint{3.322731in}{2.542736in}}%
\pgfpathlineto{\pgfqpoint{3.322731in}{2.538478in}}%
\pgfpathmoveto{\pgfqpoint{3.314216in}{2.546994in}}%
\pgfpathlineto{\pgfqpoint{3.314216in}{2.546994in}}%
\pgfpathlineto{\pgfqpoint{3.314216in}{2.551252in}}%
\pgfpathlineto{\pgfqpoint{3.318474in}{2.551252in}}%
\pgfpathlineto{\pgfqpoint{3.318474in}{2.546994in}}%
\pgfpathmoveto{\pgfqpoint{3.318474in}{2.542736in}}%
\pgfpathlineto{\pgfqpoint{3.318474in}{2.542736in}}%
\pgfpathlineto{\pgfqpoint{3.318474in}{2.546994in}}%
\pgfpathlineto{\pgfqpoint{3.322731in}{2.546994in}}%
\pgfpathlineto{\pgfqpoint{3.322731in}{2.542736in}}%
\pgfpathmoveto{\pgfqpoint{3.318474in}{2.546994in}}%
\pgfpathlineto{\pgfqpoint{3.318474in}{2.546994in}}%
\pgfpathlineto{\pgfqpoint{3.318474in}{2.551252in}}%
\pgfpathlineto{\pgfqpoint{3.322731in}{2.551252in}}%
\pgfpathlineto{\pgfqpoint{3.322731in}{2.546994in}}%
\pgfpathmoveto{\pgfqpoint{3.292927in}{2.683244in}}%
\pgfpathlineto{\pgfqpoint{3.292927in}{2.683244in}}%
\pgfpathlineto{\pgfqpoint{3.292927in}{2.687502in}}%
\pgfpathlineto{\pgfqpoint{3.297185in}{2.687502in}}%
\pgfpathlineto{\pgfqpoint{3.297185in}{2.683244in}}%
\pgfpathmoveto{\pgfqpoint{3.309958in}{2.572541in}}%
\pgfpathlineto{\pgfqpoint{3.309958in}{2.572541in}}%
\pgfpathlineto{\pgfqpoint{3.309958in}{2.576799in}}%
\pgfpathlineto{\pgfqpoint{3.314216in}{2.576799in}}%
\pgfpathlineto{\pgfqpoint{3.314216in}{2.572541in}}%
\pgfpathmoveto{\pgfqpoint{3.309958in}{2.576799in}}%
\pgfpathlineto{\pgfqpoint{3.309958in}{2.576799in}}%
\pgfpathlineto{\pgfqpoint{3.309958in}{2.581057in}}%
\pgfpathlineto{\pgfqpoint{3.314216in}{2.581057in}}%
\pgfpathlineto{\pgfqpoint{3.314216in}{2.576799in}}%
\pgfpathmoveto{\pgfqpoint{3.309958in}{2.581057in}}%
\pgfpathlineto{\pgfqpoint{3.309958in}{2.581057in}}%
\pgfpathlineto{\pgfqpoint{3.309958in}{2.585314in}}%
\pgfpathlineto{\pgfqpoint{3.314216in}{2.585314in}}%
\pgfpathlineto{\pgfqpoint{3.314216in}{2.581057in}}%
\pgfpathmoveto{\pgfqpoint{3.314216in}{2.551252in}}%
\pgfpathlineto{\pgfqpoint{3.314216in}{2.551252in}}%
\pgfpathlineto{\pgfqpoint{3.314216in}{2.555510in}}%
\pgfpathlineto{\pgfqpoint{3.318474in}{2.555510in}}%
\pgfpathlineto{\pgfqpoint{3.318474in}{2.551252in}}%
\pgfpathmoveto{\pgfqpoint{3.314216in}{2.555510in}}%
\pgfpathlineto{\pgfqpoint{3.314216in}{2.555510in}}%
\pgfpathlineto{\pgfqpoint{3.314216in}{2.559768in}}%
\pgfpathlineto{\pgfqpoint{3.318474in}{2.559768in}}%
\pgfpathlineto{\pgfqpoint{3.318474in}{2.555510in}}%
\pgfpathmoveto{\pgfqpoint{3.314216in}{2.559768in}}%
\pgfpathlineto{\pgfqpoint{3.314216in}{2.559768in}}%
\pgfpathlineto{\pgfqpoint{3.314216in}{2.564026in}}%
\pgfpathlineto{\pgfqpoint{3.318474in}{2.564026in}}%
\pgfpathlineto{\pgfqpoint{3.318474in}{2.559768in}}%
\pgfpathmoveto{\pgfqpoint{3.314216in}{2.564026in}}%
\pgfpathlineto{\pgfqpoint{3.314216in}{2.564026in}}%
\pgfpathlineto{\pgfqpoint{3.314216in}{2.568283in}}%
\pgfpathlineto{\pgfqpoint{3.318474in}{2.568283in}}%
\pgfpathlineto{\pgfqpoint{3.318474in}{2.564026in}}%
\pgfpathmoveto{\pgfqpoint{3.314216in}{2.568283in}}%
\pgfpathlineto{\pgfqpoint{3.314216in}{2.568283in}}%
\pgfpathlineto{\pgfqpoint{3.314216in}{2.572541in}}%
\pgfpathlineto{\pgfqpoint{3.318474in}{2.572541in}}%
\pgfpathlineto{\pgfqpoint{3.318474in}{2.568283in}}%
\pgfpathmoveto{\pgfqpoint{3.314216in}{2.572541in}}%
\pgfpathlineto{\pgfqpoint{3.314216in}{2.572541in}}%
\pgfpathlineto{\pgfqpoint{3.314216in}{2.576799in}}%
\pgfpathlineto{\pgfqpoint{3.318474in}{2.576799in}}%
\pgfpathlineto{\pgfqpoint{3.318474in}{2.572541in}}%
\pgfpathmoveto{\pgfqpoint{3.309958in}{2.585314in}}%
\pgfpathlineto{\pgfqpoint{3.309958in}{2.585314in}}%
\pgfpathlineto{\pgfqpoint{3.309958in}{2.589572in}}%
\pgfpathlineto{\pgfqpoint{3.314216in}{2.589572in}}%
\pgfpathlineto{\pgfqpoint{3.314216in}{2.585314in}}%
\pgfpathmoveto{\pgfqpoint{3.309958in}{2.589572in}}%
\pgfpathlineto{\pgfqpoint{3.309958in}{2.589572in}}%
\pgfpathlineto{\pgfqpoint{3.309958in}{2.593830in}}%
\pgfpathlineto{\pgfqpoint{3.314216in}{2.593830in}}%
\pgfpathlineto{\pgfqpoint{3.314216in}{2.589572in}}%
\pgfpathmoveto{\pgfqpoint{3.309958in}{2.593830in}}%
\pgfpathlineto{\pgfqpoint{3.309958in}{2.593830in}}%
\pgfpathlineto{\pgfqpoint{3.309958in}{2.598088in}}%
\pgfpathlineto{\pgfqpoint{3.314216in}{2.598088in}}%
\pgfpathlineto{\pgfqpoint{3.314216in}{2.593830in}}%
\pgfpathmoveto{\pgfqpoint{3.309958in}{2.598088in}}%
\pgfpathlineto{\pgfqpoint{3.309958in}{2.598088in}}%
\pgfpathlineto{\pgfqpoint{3.309958in}{2.602346in}}%
\pgfpathlineto{\pgfqpoint{3.314216in}{2.602346in}}%
\pgfpathlineto{\pgfqpoint{3.314216in}{2.598088in}}%
\pgfpathmoveto{\pgfqpoint{3.305700in}{2.602346in}}%
\pgfpathlineto{\pgfqpoint{3.305700in}{2.602346in}}%
\pgfpathlineto{\pgfqpoint{3.305700in}{2.606603in}}%
\pgfpathlineto{\pgfqpoint{3.309958in}{2.606603in}}%
\pgfpathlineto{\pgfqpoint{3.309958in}{2.602346in}}%
\pgfpathmoveto{\pgfqpoint{3.305700in}{2.606603in}}%
\pgfpathlineto{\pgfqpoint{3.305700in}{2.606603in}}%
\pgfpathlineto{\pgfqpoint{3.305700in}{2.610861in}}%
\pgfpathlineto{\pgfqpoint{3.309958in}{2.610861in}}%
\pgfpathlineto{\pgfqpoint{3.309958in}{2.606603in}}%
\pgfpathmoveto{\pgfqpoint{3.309958in}{2.602346in}}%
\pgfpathlineto{\pgfqpoint{3.309958in}{2.602346in}}%
\pgfpathlineto{\pgfqpoint{3.309958in}{2.606603in}}%
\pgfpathlineto{\pgfqpoint{3.314216in}{2.606603in}}%
\pgfpathlineto{\pgfqpoint{3.314216in}{2.602346in}}%
\pgfpathmoveto{\pgfqpoint{3.305700in}{2.610861in}}%
\pgfpathlineto{\pgfqpoint{3.305700in}{2.610861in}}%
\pgfpathlineto{\pgfqpoint{3.305700in}{2.615119in}}%
\pgfpathlineto{\pgfqpoint{3.309958in}{2.615119in}}%
\pgfpathlineto{\pgfqpoint{3.309958in}{2.610861in}}%
\pgfpathmoveto{\pgfqpoint{3.305700in}{2.615119in}}%
\pgfpathlineto{\pgfqpoint{3.305700in}{2.615119in}}%
\pgfpathlineto{\pgfqpoint{3.305700in}{2.619377in}}%
\pgfpathlineto{\pgfqpoint{3.309958in}{2.619377in}}%
\pgfpathlineto{\pgfqpoint{3.309958in}{2.615119in}}%
\pgfpathmoveto{\pgfqpoint{3.301443in}{2.627892in}}%
\pgfpathlineto{\pgfqpoint{3.301443in}{2.627892in}}%
\pgfpathlineto{\pgfqpoint{3.301443in}{2.632150in}}%
\pgfpathlineto{\pgfqpoint{3.305700in}{2.632150in}}%
\pgfpathlineto{\pgfqpoint{3.305700in}{2.627892in}}%
\pgfpathmoveto{\pgfqpoint{3.301443in}{2.632150in}}%
\pgfpathlineto{\pgfqpoint{3.301443in}{2.632150in}}%
\pgfpathlineto{\pgfqpoint{3.301443in}{2.636408in}}%
\pgfpathlineto{\pgfqpoint{3.305700in}{2.636408in}}%
\pgfpathlineto{\pgfqpoint{3.305700in}{2.632150in}}%
\pgfpathmoveto{\pgfqpoint{3.305700in}{2.619377in}}%
\pgfpathlineto{\pgfqpoint{3.305700in}{2.619377in}}%
\pgfpathlineto{\pgfqpoint{3.305700in}{2.623635in}}%
\pgfpathlineto{\pgfqpoint{3.309958in}{2.623635in}}%
\pgfpathlineto{\pgfqpoint{3.309958in}{2.619377in}}%
\pgfpathmoveto{\pgfqpoint{3.305700in}{2.623635in}}%
\pgfpathlineto{\pgfqpoint{3.305700in}{2.623635in}}%
\pgfpathlineto{\pgfqpoint{3.305700in}{2.627892in}}%
\pgfpathlineto{\pgfqpoint{3.309958in}{2.627892in}}%
\pgfpathlineto{\pgfqpoint{3.309958in}{2.623635in}}%
\pgfpathmoveto{\pgfqpoint{3.305700in}{2.627892in}}%
\pgfpathlineto{\pgfqpoint{3.305700in}{2.627892in}}%
\pgfpathlineto{\pgfqpoint{3.305700in}{2.632150in}}%
\pgfpathlineto{\pgfqpoint{3.309958in}{2.632150in}}%
\pgfpathlineto{\pgfqpoint{3.309958in}{2.627892in}}%
\pgfpathmoveto{\pgfqpoint{3.301443in}{2.636408in}}%
\pgfpathlineto{\pgfqpoint{3.301443in}{2.636408in}}%
\pgfpathlineto{\pgfqpoint{3.301443in}{2.640666in}}%
\pgfpathlineto{\pgfqpoint{3.305700in}{2.640666in}}%
\pgfpathlineto{\pgfqpoint{3.305700in}{2.636408in}}%
\pgfpathmoveto{\pgfqpoint{3.301443in}{2.640666in}}%
\pgfpathlineto{\pgfqpoint{3.301443in}{2.640666in}}%
\pgfpathlineto{\pgfqpoint{3.301443in}{2.644924in}}%
\pgfpathlineto{\pgfqpoint{3.305700in}{2.644924in}}%
\pgfpathlineto{\pgfqpoint{3.305700in}{2.640666in}}%
\pgfpathmoveto{\pgfqpoint{3.301443in}{2.644924in}}%
\pgfpathlineto{\pgfqpoint{3.301443in}{2.644924in}}%
\pgfpathlineto{\pgfqpoint{3.301443in}{2.649181in}}%
\pgfpathlineto{\pgfqpoint{3.305700in}{2.649181in}}%
\pgfpathlineto{\pgfqpoint{3.305700in}{2.644924in}}%
\pgfpathmoveto{\pgfqpoint{3.301443in}{2.649181in}}%
\pgfpathlineto{\pgfqpoint{3.301443in}{2.649181in}}%
\pgfpathlineto{\pgfqpoint{3.301443in}{2.653439in}}%
\pgfpathlineto{\pgfqpoint{3.305700in}{2.653439in}}%
\pgfpathlineto{\pgfqpoint{3.305700in}{2.649181in}}%
\pgfpathmoveto{\pgfqpoint{3.297185in}{2.657697in}}%
\pgfpathlineto{\pgfqpoint{3.297185in}{2.657697in}}%
\pgfpathlineto{\pgfqpoint{3.297185in}{2.661955in}}%
\pgfpathlineto{\pgfqpoint{3.301443in}{2.661955in}}%
\pgfpathlineto{\pgfqpoint{3.301443in}{2.657697in}}%
\pgfpathmoveto{\pgfqpoint{3.301443in}{2.653439in}}%
\pgfpathlineto{\pgfqpoint{3.301443in}{2.653439in}}%
\pgfpathlineto{\pgfqpoint{3.301443in}{2.657697in}}%
\pgfpathlineto{\pgfqpoint{3.305700in}{2.657697in}}%
\pgfpathlineto{\pgfqpoint{3.305700in}{2.653439in}}%
\pgfpathmoveto{\pgfqpoint{3.301443in}{2.657697in}}%
\pgfpathlineto{\pgfqpoint{3.301443in}{2.657697in}}%
\pgfpathlineto{\pgfqpoint{3.301443in}{2.661955in}}%
\pgfpathlineto{\pgfqpoint{3.305700in}{2.661955in}}%
\pgfpathlineto{\pgfqpoint{3.305700in}{2.657697in}}%
\pgfpathmoveto{\pgfqpoint{3.297185in}{2.661955in}}%
\pgfpathlineto{\pgfqpoint{3.297185in}{2.661955in}}%
\pgfpathlineto{\pgfqpoint{3.297185in}{2.666213in}}%
\pgfpathlineto{\pgfqpoint{3.301443in}{2.666213in}}%
\pgfpathlineto{\pgfqpoint{3.301443in}{2.661955in}}%
\pgfpathmoveto{\pgfqpoint{3.297185in}{2.666213in}}%
\pgfpathlineto{\pgfqpoint{3.297185in}{2.666213in}}%
\pgfpathlineto{\pgfqpoint{3.297185in}{2.670470in}}%
\pgfpathlineto{\pgfqpoint{3.301443in}{2.670470in}}%
\pgfpathlineto{\pgfqpoint{3.301443in}{2.666213in}}%
\pgfpathmoveto{\pgfqpoint{3.297185in}{2.670470in}}%
\pgfpathlineto{\pgfqpoint{3.297185in}{2.670470in}}%
\pgfpathlineto{\pgfqpoint{3.297185in}{2.674728in}}%
\pgfpathlineto{\pgfqpoint{3.301443in}{2.674728in}}%
\pgfpathlineto{\pgfqpoint{3.301443in}{2.670470in}}%
\pgfpathmoveto{\pgfqpoint{3.297185in}{2.674728in}}%
\pgfpathlineto{\pgfqpoint{3.297185in}{2.674728in}}%
\pgfpathlineto{\pgfqpoint{3.297185in}{2.678986in}}%
\pgfpathlineto{\pgfqpoint{3.301443in}{2.678986in}}%
\pgfpathlineto{\pgfqpoint{3.301443in}{2.674728in}}%
\pgfpathmoveto{\pgfqpoint{3.297185in}{2.678986in}}%
\pgfpathlineto{\pgfqpoint{3.297185in}{2.678986in}}%
\pgfpathlineto{\pgfqpoint{3.297185in}{2.683244in}}%
\pgfpathlineto{\pgfqpoint{3.301443in}{2.683244in}}%
\pgfpathlineto{\pgfqpoint{3.301443in}{2.678986in}}%
\pgfpathmoveto{\pgfqpoint{3.297185in}{2.683244in}}%
\pgfpathlineto{\pgfqpoint{3.297185in}{2.683244in}}%
\pgfpathlineto{\pgfqpoint{3.297185in}{2.687502in}}%
\pgfpathlineto{\pgfqpoint{3.301443in}{2.687502in}}%
\pgfpathlineto{\pgfqpoint{3.301443in}{2.683244in}}%
\pgfpathmoveto{\pgfqpoint{3.292927in}{2.687502in}}%
\pgfpathlineto{\pgfqpoint{3.292927in}{2.687502in}}%
\pgfpathlineto{\pgfqpoint{3.292927in}{2.691759in}}%
\pgfpathlineto{\pgfqpoint{3.297185in}{2.691759in}}%
\pgfpathlineto{\pgfqpoint{3.297185in}{2.687502in}}%
\pgfpathmoveto{\pgfqpoint{3.292927in}{2.691759in}}%
\pgfpathlineto{\pgfqpoint{3.292927in}{2.691759in}}%
\pgfpathlineto{\pgfqpoint{3.292927in}{2.696017in}}%
\pgfpathlineto{\pgfqpoint{3.297185in}{2.696017in}}%
\pgfpathlineto{\pgfqpoint{3.297185in}{2.691759in}}%
\pgfpathmoveto{\pgfqpoint{3.292927in}{2.696017in}}%
\pgfpathlineto{\pgfqpoint{3.292927in}{2.696017in}}%
\pgfpathlineto{\pgfqpoint{3.292927in}{2.700275in}}%
\pgfpathlineto{\pgfqpoint{3.297185in}{2.700275in}}%
\pgfpathlineto{\pgfqpoint{3.297185in}{2.696017in}}%
\pgfpathmoveto{\pgfqpoint{3.292927in}{2.700275in}}%
\pgfpathlineto{\pgfqpoint{3.292927in}{2.700275in}}%
\pgfpathlineto{\pgfqpoint{3.292927in}{2.704533in}}%
\pgfpathlineto{\pgfqpoint{3.297185in}{2.704533in}}%
\pgfpathlineto{\pgfqpoint{3.297185in}{2.700275in}}%
\pgfpathmoveto{\pgfqpoint{3.288670in}{2.708791in}}%
\pgfpathlineto{\pgfqpoint{3.288670in}{2.708791in}}%
\pgfpathlineto{\pgfqpoint{3.288670in}{2.713048in}}%
\pgfpathlineto{\pgfqpoint{3.292927in}{2.713048in}}%
\pgfpathlineto{\pgfqpoint{3.292927in}{2.708791in}}%
\pgfpathmoveto{\pgfqpoint{3.292927in}{2.704533in}}%
\pgfpathlineto{\pgfqpoint{3.292927in}{2.704533in}}%
\pgfpathlineto{\pgfqpoint{3.292927in}{2.708791in}}%
\pgfpathlineto{\pgfqpoint{3.297185in}{2.708791in}}%
\pgfpathlineto{\pgfqpoint{3.297185in}{2.704533in}}%
\pgfpathmoveto{\pgfqpoint{3.292927in}{2.708791in}}%
\pgfpathlineto{\pgfqpoint{3.292927in}{2.708791in}}%
\pgfpathlineto{\pgfqpoint{3.292927in}{2.713048in}}%
\pgfpathlineto{\pgfqpoint{3.297185in}{2.713048in}}%
\pgfpathlineto{\pgfqpoint{3.297185in}{2.708791in}}%
\pgfpathmoveto{\pgfqpoint{3.288670in}{2.713048in}}%
\pgfpathlineto{\pgfqpoint{3.288670in}{2.713048in}}%
\pgfpathlineto{\pgfqpoint{3.288670in}{2.717306in}}%
\pgfpathlineto{\pgfqpoint{3.292927in}{2.717306in}}%
\pgfpathlineto{\pgfqpoint{3.292927in}{2.713048in}}%
\pgfpathmoveto{\pgfqpoint{3.288670in}{2.717306in}}%
\pgfpathlineto{\pgfqpoint{3.288670in}{2.717306in}}%
\pgfpathlineto{\pgfqpoint{3.288670in}{2.721564in}}%
\pgfpathlineto{\pgfqpoint{3.292927in}{2.721564in}}%
\pgfpathlineto{\pgfqpoint{3.292927in}{2.717306in}}%
\pgfpathmoveto{\pgfqpoint{3.284412in}{2.734337in}}%
\pgfpathlineto{\pgfqpoint{3.284412in}{2.734337in}}%
\pgfpathlineto{\pgfqpoint{3.284412in}{2.738595in}}%
\pgfpathlineto{\pgfqpoint{3.288670in}{2.738595in}}%
\pgfpathlineto{\pgfqpoint{3.288670in}{2.734337in}}%
\pgfpathmoveto{\pgfqpoint{3.288670in}{2.721564in}}%
\pgfpathlineto{\pgfqpoint{3.288670in}{2.721564in}}%
\pgfpathlineto{\pgfqpoint{3.288670in}{2.725822in}}%
\pgfpathlineto{\pgfqpoint{3.292927in}{2.725822in}}%
\pgfpathlineto{\pgfqpoint{3.292927in}{2.721564in}}%
\pgfpathmoveto{\pgfqpoint{3.288670in}{2.725822in}}%
\pgfpathlineto{\pgfqpoint{3.288670in}{2.725822in}}%
\pgfpathlineto{\pgfqpoint{3.288670in}{2.730080in}}%
\pgfpathlineto{\pgfqpoint{3.292927in}{2.730080in}}%
\pgfpathlineto{\pgfqpoint{3.292927in}{2.725822in}}%
\pgfpathmoveto{\pgfqpoint{3.288670in}{2.730080in}}%
\pgfpathlineto{\pgfqpoint{3.288670in}{2.730080in}}%
\pgfpathlineto{\pgfqpoint{3.288670in}{2.734337in}}%
\pgfpathlineto{\pgfqpoint{3.292927in}{2.734337in}}%
\pgfpathlineto{\pgfqpoint{3.292927in}{2.730080in}}%
\pgfpathmoveto{\pgfqpoint{3.288670in}{2.734337in}}%
\pgfpathlineto{\pgfqpoint{3.288670in}{2.734337in}}%
\pgfpathlineto{\pgfqpoint{3.288670in}{2.738595in}}%
\pgfpathlineto{\pgfqpoint{3.292927in}{2.738595in}}%
\pgfpathlineto{\pgfqpoint{3.292927in}{2.734337in}}%
\pgfpathmoveto{\pgfqpoint{3.284412in}{2.738595in}}%
\pgfpathlineto{\pgfqpoint{3.284412in}{2.738595in}}%
\pgfpathlineto{\pgfqpoint{3.284412in}{2.742853in}}%
\pgfpathlineto{\pgfqpoint{3.288670in}{2.742853in}}%
\pgfpathlineto{\pgfqpoint{3.288670in}{2.738595in}}%
\pgfpathmoveto{\pgfqpoint{3.284412in}{2.742853in}}%
\pgfpathlineto{\pgfqpoint{3.284412in}{2.742853in}}%
\pgfpathlineto{\pgfqpoint{3.284412in}{2.747111in}}%
\pgfpathlineto{\pgfqpoint{3.288670in}{2.747111in}}%
\pgfpathlineto{\pgfqpoint{3.288670in}{2.742853in}}%
\pgfpathmoveto{\pgfqpoint{3.284412in}{2.747111in}}%
\pgfpathlineto{\pgfqpoint{3.284412in}{2.747111in}}%
\pgfpathlineto{\pgfqpoint{3.284412in}{2.751368in}}%
\pgfpathlineto{\pgfqpoint{3.288670in}{2.751368in}}%
\pgfpathlineto{\pgfqpoint{3.288670in}{2.747111in}}%
\pgfpathmoveto{\pgfqpoint{3.284412in}{2.751368in}}%
\pgfpathlineto{\pgfqpoint{3.284412in}{2.751368in}}%
\pgfpathlineto{\pgfqpoint{3.284412in}{2.755626in}}%
\pgfpathlineto{\pgfqpoint{3.288670in}{2.755626in}}%
\pgfpathlineto{\pgfqpoint{3.288670in}{2.751368in}}%
\pgfpathmoveto{\pgfqpoint{3.275896in}{2.785431in}}%
\pgfpathlineto{\pgfqpoint{3.275896in}{2.785431in}}%
\pgfpathlineto{\pgfqpoint{3.275896in}{2.789689in}}%
\pgfpathlineto{\pgfqpoint{3.280154in}{2.789689in}}%
\pgfpathlineto{\pgfqpoint{3.280154in}{2.785431in}}%
\pgfpathmoveto{\pgfqpoint{3.284412in}{2.755626in}}%
\pgfpathlineto{\pgfqpoint{3.284412in}{2.755626in}}%
\pgfpathlineto{\pgfqpoint{3.284412in}{2.759884in}}%
\pgfpathlineto{\pgfqpoint{3.288670in}{2.759884in}}%
\pgfpathlineto{\pgfqpoint{3.288670in}{2.755626in}}%
\pgfpathmoveto{\pgfqpoint{3.284412in}{2.759884in}}%
\pgfpathlineto{\pgfqpoint{3.284412in}{2.759884in}}%
\pgfpathlineto{\pgfqpoint{3.284412in}{2.764142in}}%
\pgfpathlineto{\pgfqpoint{3.288670in}{2.764142in}}%
\pgfpathlineto{\pgfqpoint{3.288670in}{2.759884in}}%
\pgfpathmoveto{\pgfqpoint{3.280154in}{2.764142in}}%
\pgfpathlineto{\pgfqpoint{3.280154in}{2.764142in}}%
\pgfpathlineto{\pgfqpoint{3.280154in}{2.768400in}}%
\pgfpathlineto{\pgfqpoint{3.284412in}{2.768400in}}%
\pgfpathlineto{\pgfqpoint{3.284412in}{2.764142in}}%
\pgfpathmoveto{\pgfqpoint{3.280154in}{2.768400in}}%
\pgfpathlineto{\pgfqpoint{3.280154in}{2.768400in}}%
\pgfpathlineto{\pgfqpoint{3.280154in}{2.772657in}}%
\pgfpathlineto{\pgfqpoint{3.284412in}{2.772657in}}%
\pgfpathlineto{\pgfqpoint{3.284412in}{2.768400in}}%
\pgfpathmoveto{\pgfqpoint{3.284412in}{2.764142in}}%
\pgfpathlineto{\pgfqpoint{3.284412in}{2.764142in}}%
\pgfpathlineto{\pgfqpoint{3.284412in}{2.768400in}}%
\pgfpathlineto{\pgfqpoint{3.288670in}{2.768400in}}%
\pgfpathlineto{\pgfqpoint{3.288670in}{2.764142in}}%
\pgfpathmoveto{\pgfqpoint{3.280154in}{2.772657in}}%
\pgfpathlineto{\pgfqpoint{3.280154in}{2.772657in}}%
\pgfpathlineto{\pgfqpoint{3.280154in}{2.776915in}}%
\pgfpathlineto{\pgfqpoint{3.284412in}{2.776915in}}%
\pgfpathlineto{\pgfqpoint{3.284412in}{2.772657in}}%
\pgfpathmoveto{\pgfqpoint{3.280154in}{2.776915in}}%
\pgfpathlineto{\pgfqpoint{3.280154in}{2.776915in}}%
\pgfpathlineto{\pgfqpoint{3.280154in}{2.781173in}}%
\pgfpathlineto{\pgfqpoint{3.284412in}{2.781173in}}%
\pgfpathlineto{\pgfqpoint{3.284412in}{2.776915in}}%
\pgfpathmoveto{\pgfqpoint{3.280154in}{2.781173in}}%
\pgfpathlineto{\pgfqpoint{3.280154in}{2.781173in}}%
\pgfpathlineto{\pgfqpoint{3.280154in}{2.785431in}}%
\pgfpathlineto{\pgfqpoint{3.284412in}{2.785431in}}%
\pgfpathlineto{\pgfqpoint{3.284412in}{2.781173in}}%
\pgfpathmoveto{\pgfqpoint{3.280154in}{2.785431in}}%
\pgfpathlineto{\pgfqpoint{3.280154in}{2.785431in}}%
\pgfpathlineto{\pgfqpoint{3.280154in}{2.789689in}}%
\pgfpathlineto{\pgfqpoint{3.284412in}{2.789689in}}%
\pgfpathlineto{\pgfqpoint{3.284412in}{2.785431in}}%
\pgfpathmoveto{\pgfqpoint{3.275896in}{2.789689in}}%
\pgfpathlineto{\pgfqpoint{3.275896in}{2.789689in}}%
\pgfpathlineto{\pgfqpoint{3.275896in}{2.793946in}}%
\pgfpathlineto{\pgfqpoint{3.280154in}{2.793946in}}%
\pgfpathlineto{\pgfqpoint{3.280154in}{2.789689in}}%
\pgfpathmoveto{\pgfqpoint{3.275896in}{2.793946in}}%
\pgfpathlineto{\pgfqpoint{3.275896in}{2.793946in}}%
\pgfpathlineto{\pgfqpoint{3.275896in}{2.798204in}}%
\pgfpathlineto{\pgfqpoint{3.280154in}{2.798204in}}%
\pgfpathlineto{\pgfqpoint{3.280154in}{2.793946in}}%
\pgfpathmoveto{\pgfqpoint{3.275896in}{2.798204in}}%
\pgfpathlineto{\pgfqpoint{3.275896in}{2.798204in}}%
\pgfpathlineto{\pgfqpoint{3.275896in}{2.802462in}}%
\pgfpathlineto{\pgfqpoint{3.280154in}{2.802462in}}%
\pgfpathlineto{\pgfqpoint{3.280154in}{2.798204in}}%
\pgfpathmoveto{\pgfqpoint{3.275896in}{2.802462in}}%
\pgfpathlineto{\pgfqpoint{3.275896in}{2.802462in}}%
\pgfpathlineto{\pgfqpoint{3.275896in}{2.806720in}}%
\pgfpathlineto{\pgfqpoint{3.280154in}{2.806720in}}%
\pgfpathlineto{\pgfqpoint{3.280154in}{2.802462in}}%
\pgfpathmoveto{\pgfqpoint{3.271639in}{2.810978in}}%
\pgfpathlineto{\pgfqpoint{3.271639in}{2.810978in}}%
\pgfpathlineto{\pgfqpoint{3.271639in}{2.815235in}}%
\pgfpathlineto{\pgfqpoint{3.275896in}{2.815235in}}%
\pgfpathlineto{\pgfqpoint{3.275896in}{2.810978in}}%
\pgfpathmoveto{\pgfqpoint{3.275896in}{2.806720in}}%
\pgfpathlineto{\pgfqpoint{3.275896in}{2.806720in}}%
\pgfpathlineto{\pgfqpoint{3.275896in}{2.810978in}}%
\pgfpathlineto{\pgfqpoint{3.280154in}{2.810978in}}%
\pgfpathlineto{\pgfqpoint{3.280154in}{2.806720in}}%
\pgfpathmoveto{\pgfqpoint{3.275896in}{2.810978in}}%
\pgfpathlineto{\pgfqpoint{3.275896in}{2.810978in}}%
\pgfpathlineto{\pgfqpoint{3.275896in}{2.815235in}}%
\pgfpathlineto{\pgfqpoint{3.280154in}{2.815235in}}%
\pgfpathlineto{\pgfqpoint{3.280154in}{2.810978in}}%
\pgfpathmoveto{\pgfqpoint{3.271639in}{2.815235in}}%
\pgfpathlineto{\pgfqpoint{3.271639in}{2.815235in}}%
\pgfpathlineto{\pgfqpoint{3.271639in}{2.819493in}}%
\pgfpathlineto{\pgfqpoint{3.275896in}{2.819493in}}%
\pgfpathlineto{\pgfqpoint{3.275896in}{2.815235in}}%
\pgfpathmoveto{\pgfqpoint{3.271639in}{2.819493in}}%
\pgfpathlineto{\pgfqpoint{3.271639in}{2.819493in}}%
\pgfpathlineto{\pgfqpoint{3.271639in}{2.823751in}}%
\pgfpathlineto{\pgfqpoint{3.275896in}{2.823751in}}%
\pgfpathlineto{\pgfqpoint{3.275896in}{2.819493in}}%
\pgfpathmoveto{\pgfqpoint{3.258865in}{2.887617in}}%
\pgfpathlineto{\pgfqpoint{3.258865in}{2.887617in}}%
\pgfpathlineto{\pgfqpoint{3.258865in}{2.891874in}}%
\pgfpathlineto{\pgfqpoint{3.263123in}{2.891874in}}%
\pgfpathlineto{\pgfqpoint{3.263123in}{2.887617in}}%
\pgfpathmoveto{\pgfqpoint{3.267381in}{2.836524in}}%
\pgfpathlineto{\pgfqpoint{3.267381in}{2.836524in}}%
\pgfpathlineto{\pgfqpoint{3.267381in}{2.840782in}}%
\pgfpathlineto{\pgfqpoint{3.271639in}{2.840782in}}%
\pgfpathlineto{\pgfqpoint{3.271639in}{2.836524in}}%
\pgfpathmoveto{\pgfqpoint{3.271639in}{2.823751in}}%
\pgfpathlineto{\pgfqpoint{3.271639in}{2.823751in}}%
\pgfpathlineto{\pgfqpoint{3.271639in}{2.828009in}}%
\pgfpathlineto{\pgfqpoint{3.275896in}{2.828009in}}%
\pgfpathlineto{\pgfqpoint{3.275896in}{2.823751in}}%
\pgfpathmoveto{\pgfqpoint{3.271639in}{2.828009in}}%
\pgfpathlineto{\pgfqpoint{3.271639in}{2.828009in}}%
\pgfpathlineto{\pgfqpoint{3.271639in}{2.832266in}}%
\pgfpathlineto{\pgfqpoint{3.275896in}{2.832266in}}%
\pgfpathlineto{\pgfqpoint{3.275896in}{2.828009in}}%
\pgfpathmoveto{\pgfqpoint{3.271639in}{2.832266in}}%
\pgfpathlineto{\pgfqpoint{3.271639in}{2.832266in}}%
\pgfpathlineto{\pgfqpoint{3.271639in}{2.836524in}}%
\pgfpathlineto{\pgfqpoint{3.275896in}{2.836524in}}%
\pgfpathlineto{\pgfqpoint{3.275896in}{2.832266in}}%
\pgfpathmoveto{\pgfqpoint{3.271639in}{2.836524in}}%
\pgfpathlineto{\pgfqpoint{3.271639in}{2.836524in}}%
\pgfpathlineto{\pgfqpoint{3.271639in}{2.840782in}}%
\pgfpathlineto{\pgfqpoint{3.275896in}{2.840782in}}%
\pgfpathlineto{\pgfqpoint{3.275896in}{2.836524in}}%
\pgfpathmoveto{\pgfqpoint{3.267381in}{2.840782in}}%
\pgfpathlineto{\pgfqpoint{3.267381in}{2.840782in}}%
\pgfpathlineto{\pgfqpoint{3.267381in}{2.845039in}}%
\pgfpathlineto{\pgfqpoint{3.271639in}{2.845039in}}%
\pgfpathlineto{\pgfqpoint{3.271639in}{2.840782in}}%
\pgfpathmoveto{\pgfqpoint{3.267381in}{2.845039in}}%
\pgfpathlineto{\pgfqpoint{3.267381in}{2.845039in}}%
\pgfpathlineto{\pgfqpoint{3.267381in}{2.849297in}}%
\pgfpathlineto{\pgfqpoint{3.271639in}{2.849297in}}%
\pgfpathlineto{\pgfqpoint{3.271639in}{2.845039in}}%
\pgfpathmoveto{\pgfqpoint{3.267381in}{2.849297in}}%
\pgfpathlineto{\pgfqpoint{3.267381in}{2.849297in}}%
\pgfpathlineto{\pgfqpoint{3.267381in}{2.853555in}}%
\pgfpathlineto{\pgfqpoint{3.271639in}{2.853555in}}%
\pgfpathlineto{\pgfqpoint{3.271639in}{2.849297in}}%
\pgfpathmoveto{\pgfqpoint{3.267381in}{2.853555in}}%
\pgfpathlineto{\pgfqpoint{3.267381in}{2.853555in}}%
\pgfpathlineto{\pgfqpoint{3.267381in}{2.857813in}}%
\pgfpathlineto{\pgfqpoint{3.271639in}{2.857813in}}%
\pgfpathlineto{\pgfqpoint{3.271639in}{2.853555in}}%
\pgfpathmoveto{\pgfqpoint{3.263123in}{2.862070in}}%
\pgfpathlineto{\pgfqpoint{3.263123in}{2.862070in}}%
\pgfpathlineto{\pgfqpoint{3.263123in}{2.866328in}}%
\pgfpathlineto{\pgfqpoint{3.267381in}{2.866328in}}%
\pgfpathlineto{\pgfqpoint{3.267381in}{2.862070in}}%
\pgfpathmoveto{\pgfqpoint{3.267381in}{2.857813in}}%
\pgfpathlineto{\pgfqpoint{3.267381in}{2.857813in}}%
\pgfpathlineto{\pgfqpoint{3.267381in}{2.862070in}}%
\pgfpathlineto{\pgfqpoint{3.271639in}{2.862070in}}%
\pgfpathlineto{\pgfqpoint{3.271639in}{2.857813in}}%
\pgfpathmoveto{\pgfqpoint{3.267381in}{2.862070in}}%
\pgfpathlineto{\pgfqpoint{3.267381in}{2.862070in}}%
\pgfpathlineto{\pgfqpoint{3.267381in}{2.866328in}}%
\pgfpathlineto{\pgfqpoint{3.271639in}{2.866328in}}%
\pgfpathlineto{\pgfqpoint{3.271639in}{2.862070in}}%
\pgfpathmoveto{\pgfqpoint{3.263123in}{2.866328in}}%
\pgfpathlineto{\pgfqpoint{3.263123in}{2.866328in}}%
\pgfpathlineto{\pgfqpoint{3.263123in}{2.870586in}}%
\pgfpathlineto{\pgfqpoint{3.267381in}{2.870586in}}%
\pgfpathlineto{\pgfqpoint{3.267381in}{2.866328in}}%
\pgfpathmoveto{\pgfqpoint{3.263123in}{2.870586in}}%
\pgfpathlineto{\pgfqpoint{3.263123in}{2.870586in}}%
\pgfpathlineto{\pgfqpoint{3.263123in}{2.874844in}}%
\pgfpathlineto{\pgfqpoint{3.267381in}{2.874844in}}%
\pgfpathlineto{\pgfqpoint{3.267381in}{2.870586in}}%
\pgfpathmoveto{\pgfqpoint{3.263123in}{2.874844in}}%
\pgfpathlineto{\pgfqpoint{3.263123in}{2.874844in}}%
\pgfpathlineto{\pgfqpoint{3.263123in}{2.879101in}}%
\pgfpathlineto{\pgfqpoint{3.267381in}{2.879101in}}%
\pgfpathlineto{\pgfqpoint{3.267381in}{2.874844in}}%
\pgfpathmoveto{\pgfqpoint{3.263123in}{2.879101in}}%
\pgfpathlineto{\pgfqpoint{3.263123in}{2.879101in}}%
\pgfpathlineto{\pgfqpoint{3.263123in}{2.883359in}}%
\pgfpathlineto{\pgfqpoint{3.267381in}{2.883359in}}%
\pgfpathlineto{\pgfqpoint{3.267381in}{2.879101in}}%
\pgfpathmoveto{\pgfqpoint{3.263123in}{2.883359in}}%
\pgfpathlineto{\pgfqpoint{3.263123in}{2.883359in}}%
\pgfpathlineto{\pgfqpoint{3.263123in}{2.887617in}}%
\pgfpathlineto{\pgfqpoint{3.267381in}{2.887617in}}%
\pgfpathlineto{\pgfqpoint{3.267381in}{2.883359in}}%
\pgfpathmoveto{\pgfqpoint{3.263123in}{2.887617in}}%
\pgfpathlineto{\pgfqpoint{3.263123in}{2.887617in}}%
\pgfpathlineto{\pgfqpoint{3.263123in}{2.891874in}}%
\pgfpathlineto{\pgfqpoint{3.267381in}{2.891874in}}%
\pgfpathlineto{\pgfqpoint{3.267381in}{2.887617in}}%
\pgfpathmoveto{\pgfqpoint{3.258865in}{2.891874in}}%
\pgfpathlineto{\pgfqpoint{3.258865in}{2.891874in}}%
\pgfpathlineto{\pgfqpoint{3.258865in}{2.896132in}}%
\pgfpathlineto{\pgfqpoint{3.263123in}{2.896132in}}%
\pgfpathlineto{\pgfqpoint{3.263123in}{2.891874in}}%
\pgfpathmoveto{\pgfqpoint{3.258865in}{2.896132in}}%
\pgfpathlineto{\pgfqpoint{3.258865in}{2.896132in}}%
\pgfpathlineto{\pgfqpoint{3.258865in}{2.900390in}}%
\pgfpathlineto{\pgfqpoint{3.263123in}{2.900390in}}%
\pgfpathlineto{\pgfqpoint{3.263123in}{2.896132in}}%
\pgfpathmoveto{\pgfqpoint{3.258865in}{2.900390in}}%
\pgfpathlineto{\pgfqpoint{3.258865in}{2.900390in}}%
\pgfpathlineto{\pgfqpoint{3.258865in}{2.904648in}}%
\pgfpathlineto{\pgfqpoint{3.263123in}{2.904648in}}%
\pgfpathlineto{\pgfqpoint{3.263123in}{2.900390in}}%
\pgfpathmoveto{\pgfqpoint{3.258865in}{2.904648in}}%
\pgfpathlineto{\pgfqpoint{3.258865in}{2.904648in}}%
\pgfpathlineto{\pgfqpoint{3.258865in}{2.908905in}}%
\pgfpathlineto{\pgfqpoint{3.263123in}{2.908905in}}%
\pgfpathlineto{\pgfqpoint{3.263123in}{2.904648in}}%
\pgfpathmoveto{\pgfqpoint{3.254608in}{2.908905in}}%
\pgfpathlineto{\pgfqpoint{3.254608in}{2.908905in}}%
\pgfpathlineto{\pgfqpoint{3.254608in}{2.913163in}}%
\pgfpathlineto{\pgfqpoint{3.258865in}{2.913163in}}%
\pgfpathlineto{\pgfqpoint{3.258865in}{2.908905in}}%
\pgfpathmoveto{\pgfqpoint{3.254608in}{2.913163in}}%
\pgfpathlineto{\pgfqpoint{3.254608in}{2.913163in}}%
\pgfpathlineto{\pgfqpoint{3.254608in}{2.917421in}}%
\pgfpathlineto{\pgfqpoint{3.258865in}{2.917421in}}%
\pgfpathlineto{\pgfqpoint{3.258865in}{2.913163in}}%
\pgfpathmoveto{\pgfqpoint{3.258865in}{2.908905in}}%
\pgfpathlineto{\pgfqpoint{3.258865in}{2.908905in}}%
\pgfpathlineto{\pgfqpoint{3.258865in}{2.913163in}}%
\pgfpathlineto{\pgfqpoint{3.263123in}{2.913163in}}%
\pgfpathlineto{\pgfqpoint{3.263123in}{2.908905in}}%
\pgfpathmoveto{\pgfqpoint{3.254608in}{2.917421in}}%
\pgfpathlineto{\pgfqpoint{3.254608in}{2.917421in}}%
\pgfpathlineto{\pgfqpoint{3.254608in}{2.921678in}}%
\pgfpathlineto{\pgfqpoint{3.258865in}{2.921678in}}%
\pgfpathlineto{\pgfqpoint{3.258865in}{2.917421in}}%
\pgfpathmoveto{\pgfqpoint{3.254608in}{2.921678in}}%
\pgfpathlineto{\pgfqpoint{3.254608in}{2.921678in}}%
\pgfpathlineto{\pgfqpoint{3.254608in}{2.925936in}}%
\pgfpathlineto{\pgfqpoint{3.258865in}{2.925936in}}%
\pgfpathlineto{\pgfqpoint{3.258865in}{2.921678in}}%
\pgfpathmoveto{\pgfqpoint{3.250350in}{2.934452in}}%
\pgfpathlineto{\pgfqpoint{3.250350in}{2.934452in}}%
\pgfpathlineto{\pgfqpoint{3.250350in}{2.938709in}}%
\pgfpathlineto{\pgfqpoint{3.254608in}{2.938709in}}%
\pgfpathlineto{\pgfqpoint{3.254608in}{2.934452in}}%
\pgfpathmoveto{\pgfqpoint{3.250350in}{2.938709in}}%
\pgfpathlineto{\pgfqpoint{3.250350in}{2.938709in}}%
\pgfpathlineto{\pgfqpoint{3.250350in}{2.942967in}}%
\pgfpathlineto{\pgfqpoint{3.254608in}{2.942967in}}%
\pgfpathlineto{\pgfqpoint{3.254608in}{2.938709in}}%
\pgfpathmoveto{\pgfqpoint{3.254608in}{2.925936in}}%
\pgfpathlineto{\pgfqpoint{3.254608in}{2.925936in}}%
\pgfpathlineto{\pgfqpoint{3.254608in}{2.930194in}}%
\pgfpathlineto{\pgfqpoint{3.258865in}{2.930194in}}%
\pgfpathlineto{\pgfqpoint{3.258865in}{2.925936in}}%
\pgfpathmoveto{\pgfqpoint{3.254608in}{2.930194in}}%
\pgfpathlineto{\pgfqpoint{3.254608in}{2.930194in}}%
\pgfpathlineto{\pgfqpoint{3.254608in}{2.934452in}}%
\pgfpathlineto{\pgfqpoint{3.258865in}{2.934452in}}%
\pgfpathlineto{\pgfqpoint{3.258865in}{2.930194in}}%
\pgfpathmoveto{\pgfqpoint{3.254608in}{2.934452in}}%
\pgfpathlineto{\pgfqpoint{3.254608in}{2.934452in}}%
\pgfpathlineto{\pgfqpoint{3.254608in}{2.938709in}}%
\pgfpathlineto{\pgfqpoint{3.258865in}{2.938709in}}%
\pgfpathlineto{\pgfqpoint{3.258865in}{2.934452in}}%
\pgfpathmoveto{\pgfqpoint{3.250350in}{2.942967in}}%
\pgfpathlineto{\pgfqpoint{3.250350in}{2.942967in}}%
\pgfpathlineto{\pgfqpoint{3.250350in}{2.947225in}}%
\pgfpathlineto{\pgfqpoint{3.254608in}{2.947225in}}%
\pgfpathlineto{\pgfqpoint{3.254608in}{2.942967in}}%
\pgfpathmoveto{\pgfqpoint{3.250350in}{2.947225in}}%
\pgfpathlineto{\pgfqpoint{3.250350in}{2.947225in}}%
\pgfpathlineto{\pgfqpoint{3.250350in}{2.951483in}}%
\pgfpathlineto{\pgfqpoint{3.254608in}{2.951483in}}%
\pgfpathlineto{\pgfqpoint{3.254608in}{2.947225in}}%
\pgfpathmoveto{\pgfqpoint{3.246092in}{2.955740in}}%
\pgfpathlineto{\pgfqpoint{3.246092in}{2.955740in}}%
\pgfpathlineto{\pgfqpoint{3.246092in}{2.959998in}}%
\pgfpathlineto{\pgfqpoint{3.250350in}{2.959998in}}%
\pgfpathlineto{\pgfqpoint{3.250350in}{2.955740in}}%
\pgfpathmoveto{\pgfqpoint{3.250350in}{2.951483in}}%
\pgfpathlineto{\pgfqpoint{3.250350in}{2.951483in}}%
\pgfpathlineto{\pgfqpoint{3.250350in}{2.955740in}}%
\pgfpathlineto{\pgfqpoint{3.254608in}{2.955740in}}%
\pgfpathlineto{\pgfqpoint{3.254608in}{2.951483in}}%
\pgfpathmoveto{\pgfqpoint{3.250350in}{2.955740in}}%
\pgfpathlineto{\pgfqpoint{3.250350in}{2.955740in}}%
\pgfpathlineto{\pgfqpoint{3.250350in}{2.959998in}}%
\pgfpathlineto{\pgfqpoint{3.254608in}{2.959998in}}%
\pgfpathlineto{\pgfqpoint{3.254608in}{2.955740in}}%
\pgfpathmoveto{\pgfqpoint{3.241835in}{2.977030in}}%
\pgfpathlineto{\pgfqpoint{3.241835in}{2.977030in}}%
\pgfpathlineto{\pgfqpoint{3.241835in}{2.981287in}}%
\pgfpathlineto{\pgfqpoint{3.246092in}{2.981287in}}%
\pgfpathlineto{\pgfqpoint{3.246092in}{2.977030in}}%
\pgfpathmoveto{\pgfqpoint{3.241835in}{2.981287in}}%
\pgfpathlineto{\pgfqpoint{3.241835in}{2.981287in}}%
\pgfpathlineto{\pgfqpoint{3.241835in}{2.985545in}}%
\pgfpathlineto{\pgfqpoint{3.246092in}{2.985545in}}%
\pgfpathlineto{\pgfqpoint{3.246092in}{2.981287in}}%
\pgfpathmoveto{\pgfqpoint{3.241835in}{2.985545in}}%
\pgfpathlineto{\pgfqpoint{3.241835in}{2.985545in}}%
\pgfpathlineto{\pgfqpoint{3.241835in}{2.989803in}}%
\pgfpathlineto{\pgfqpoint{3.246092in}{2.989803in}}%
\pgfpathlineto{\pgfqpoint{3.246092in}{2.985545in}}%
\pgfpathmoveto{\pgfqpoint{3.241835in}{2.989803in}}%
\pgfpathlineto{\pgfqpoint{3.241835in}{2.989803in}}%
\pgfpathlineto{\pgfqpoint{3.241835in}{2.994061in}}%
\pgfpathlineto{\pgfqpoint{3.246092in}{2.994061in}}%
\pgfpathlineto{\pgfqpoint{3.246092in}{2.989803in}}%
\pgfpathmoveto{\pgfqpoint{3.246092in}{2.959998in}}%
\pgfpathlineto{\pgfqpoint{3.246092in}{2.959998in}}%
\pgfpathlineto{\pgfqpoint{3.246092in}{2.964256in}}%
\pgfpathlineto{\pgfqpoint{3.250350in}{2.964256in}}%
\pgfpathlineto{\pgfqpoint{3.250350in}{2.959998in}}%
\pgfpathmoveto{\pgfqpoint{3.246092in}{2.964256in}}%
\pgfpathlineto{\pgfqpoint{3.246092in}{2.964256in}}%
\pgfpathlineto{\pgfqpoint{3.246092in}{2.968514in}}%
\pgfpathlineto{\pgfqpoint{3.250350in}{2.968514in}}%
\pgfpathlineto{\pgfqpoint{3.250350in}{2.964256in}}%
\pgfpathmoveto{\pgfqpoint{3.246092in}{2.968514in}}%
\pgfpathlineto{\pgfqpoint{3.246092in}{2.968514in}}%
\pgfpathlineto{\pgfqpoint{3.246092in}{2.972772in}}%
\pgfpathlineto{\pgfqpoint{3.250350in}{2.972772in}}%
\pgfpathlineto{\pgfqpoint{3.250350in}{2.968514in}}%
\pgfpathmoveto{\pgfqpoint{3.246092in}{2.972772in}}%
\pgfpathlineto{\pgfqpoint{3.246092in}{2.972772in}}%
\pgfpathlineto{\pgfqpoint{3.246092in}{2.977030in}}%
\pgfpathlineto{\pgfqpoint{3.250350in}{2.977030in}}%
\pgfpathlineto{\pgfqpoint{3.250350in}{2.972772in}}%
\pgfpathmoveto{\pgfqpoint{3.246092in}{2.977030in}}%
\pgfpathlineto{\pgfqpoint{3.246092in}{2.977030in}}%
\pgfpathlineto{\pgfqpoint{3.246092in}{2.981287in}}%
\pgfpathlineto{\pgfqpoint{3.250350in}{2.981287in}}%
\pgfpathlineto{\pgfqpoint{3.250350in}{2.977030in}}%
\pgfpathmoveto{\pgfqpoint{3.241835in}{2.994061in}}%
\pgfpathlineto{\pgfqpoint{3.241835in}{2.994061in}}%
\pgfpathlineto{\pgfqpoint{3.241835in}{2.998319in}}%
\pgfpathlineto{\pgfqpoint{3.246092in}{2.998319in}}%
\pgfpathlineto{\pgfqpoint{3.246092in}{2.994061in}}%
\pgfpathmoveto{\pgfqpoint{3.241835in}{2.998319in}}%
\pgfpathlineto{\pgfqpoint{3.241835in}{2.998319in}}%
\pgfpathlineto{\pgfqpoint{3.241835in}{3.002577in}}%
\pgfpathlineto{\pgfqpoint{3.246092in}{3.002577in}}%
\pgfpathlineto{\pgfqpoint{3.246092in}{2.998319in}}%
\pgfpathmoveto{\pgfqpoint{3.237577in}{3.002577in}}%
\pgfpathlineto{\pgfqpoint{3.237577in}{3.002577in}}%
\pgfpathlineto{\pgfqpoint{3.237577in}{3.006835in}}%
\pgfpathlineto{\pgfqpoint{3.241835in}{3.006835in}}%
\pgfpathlineto{\pgfqpoint{3.241835in}{3.002577in}}%
\pgfpathmoveto{\pgfqpoint{3.237577in}{3.006835in}}%
\pgfpathlineto{\pgfqpoint{3.237577in}{3.006835in}}%
\pgfpathlineto{\pgfqpoint{3.237577in}{3.011093in}}%
\pgfpathlineto{\pgfqpoint{3.241835in}{3.011093in}}%
\pgfpathlineto{\pgfqpoint{3.241835in}{3.006835in}}%
\pgfpathmoveto{\pgfqpoint{3.241835in}{3.002577in}}%
\pgfpathlineto{\pgfqpoint{3.241835in}{3.002577in}}%
\pgfpathlineto{\pgfqpoint{3.241835in}{3.006835in}}%
\pgfpathlineto{\pgfqpoint{3.246092in}{3.006835in}}%
\pgfpathlineto{\pgfqpoint{3.246092in}{3.002577in}}%
\pgfpathmoveto{\pgfqpoint{3.233319in}{3.023867in}}%
\pgfpathlineto{\pgfqpoint{3.233319in}{3.023867in}}%
\pgfpathlineto{\pgfqpoint{3.233319in}{3.028124in}}%
\pgfpathlineto{\pgfqpoint{3.237577in}{3.028124in}}%
\pgfpathlineto{\pgfqpoint{3.237577in}{3.023867in}}%
\pgfpathmoveto{\pgfqpoint{3.237577in}{3.011093in}}%
\pgfpathlineto{\pgfqpoint{3.237577in}{3.011093in}}%
\pgfpathlineto{\pgfqpoint{3.237577in}{3.015351in}}%
\pgfpathlineto{\pgfqpoint{3.241835in}{3.015351in}}%
\pgfpathlineto{\pgfqpoint{3.241835in}{3.011093in}}%
\pgfpathmoveto{\pgfqpoint{3.237577in}{3.015351in}}%
\pgfpathlineto{\pgfqpoint{3.237577in}{3.015351in}}%
\pgfpathlineto{\pgfqpoint{3.237577in}{3.019609in}}%
\pgfpathlineto{\pgfqpoint{3.241835in}{3.019609in}}%
\pgfpathlineto{\pgfqpoint{3.241835in}{3.015351in}}%
\pgfpathmoveto{\pgfqpoint{3.237577in}{3.019609in}}%
\pgfpathlineto{\pgfqpoint{3.237577in}{3.019609in}}%
\pgfpathlineto{\pgfqpoint{3.237577in}{3.023867in}}%
\pgfpathlineto{\pgfqpoint{3.241835in}{3.023867in}}%
\pgfpathlineto{\pgfqpoint{3.241835in}{3.019609in}}%
\pgfpathmoveto{\pgfqpoint{3.237577in}{3.023867in}}%
\pgfpathlineto{\pgfqpoint{3.237577in}{3.023867in}}%
\pgfpathlineto{\pgfqpoint{3.237577in}{3.028124in}}%
\pgfpathlineto{\pgfqpoint{3.241835in}{3.028124in}}%
\pgfpathlineto{\pgfqpoint{3.241835in}{3.023867in}}%
\pgfpathmoveto{\pgfqpoint{3.233319in}{3.028124in}}%
\pgfpathlineto{\pgfqpoint{3.233319in}{3.028124in}}%
\pgfpathlineto{\pgfqpoint{3.233319in}{3.032382in}}%
\pgfpathlineto{\pgfqpoint{3.237577in}{3.032382in}}%
\pgfpathlineto{\pgfqpoint{3.237577in}{3.028124in}}%
\pgfpathmoveto{\pgfqpoint{3.233319in}{3.032382in}}%
\pgfpathlineto{\pgfqpoint{3.233319in}{3.032382in}}%
\pgfpathlineto{\pgfqpoint{3.233319in}{3.036640in}}%
\pgfpathlineto{\pgfqpoint{3.237577in}{3.036640in}}%
\pgfpathlineto{\pgfqpoint{3.237577in}{3.032382in}}%
\pgfpathmoveto{\pgfqpoint{3.233319in}{3.036640in}}%
\pgfpathlineto{\pgfqpoint{3.233319in}{3.036640in}}%
\pgfpathlineto{\pgfqpoint{3.233319in}{3.040898in}}%
\pgfpathlineto{\pgfqpoint{3.237577in}{3.040898in}}%
\pgfpathlineto{\pgfqpoint{3.237577in}{3.036640in}}%
\pgfpathmoveto{\pgfqpoint{3.233319in}{3.040898in}}%
\pgfpathlineto{\pgfqpoint{3.233319in}{3.040898in}}%
\pgfpathlineto{\pgfqpoint{3.233319in}{3.045156in}}%
\pgfpathlineto{\pgfqpoint{3.237577in}{3.045156in}}%
\pgfpathlineto{\pgfqpoint{3.237577in}{3.040898in}}%
\pgfpathmoveto{\pgfqpoint{3.229061in}{3.045156in}}%
\pgfpathlineto{\pgfqpoint{3.229061in}{3.045156in}}%
\pgfpathlineto{\pgfqpoint{3.229061in}{3.049414in}}%
\pgfpathlineto{\pgfqpoint{3.233319in}{3.049414in}}%
\pgfpathlineto{\pgfqpoint{3.233319in}{3.045156in}}%
\pgfpathmoveto{\pgfqpoint{3.229061in}{3.049414in}}%
\pgfpathlineto{\pgfqpoint{3.229061in}{3.049414in}}%
\pgfpathlineto{\pgfqpoint{3.229061in}{3.053672in}}%
\pgfpathlineto{\pgfqpoint{3.233319in}{3.053672in}}%
\pgfpathlineto{\pgfqpoint{3.233319in}{3.049414in}}%
\pgfpathmoveto{\pgfqpoint{3.233319in}{3.045156in}}%
\pgfpathlineto{\pgfqpoint{3.233319in}{3.045156in}}%
\pgfpathlineto{\pgfqpoint{3.233319in}{3.049414in}}%
\pgfpathlineto{\pgfqpoint{3.237577in}{3.049414in}}%
\pgfpathlineto{\pgfqpoint{3.237577in}{3.045156in}}%
\pgfpathmoveto{\pgfqpoint{3.229061in}{3.053672in}}%
\pgfpathlineto{\pgfqpoint{3.229061in}{3.053672in}}%
\pgfpathlineto{\pgfqpoint{3.229061in}{3.057930in}}%
\pgfpathlineto{\pgfqpoint{3.233319in}{3.057930in}}%
\pgfpathlineto{\pgfqpoint{3.233319in}{3.053672in}}%
\pgfpathmoveto{\pgfqpoint{3.229061in}{3.057930in}}%
\pgfpathlineto{\pgfqpoint{3.229061in}{3.057930in}}%
\pgfpathlineto{\pgfqpoint{3.229061in}{3.062188in}}%
\pgfpathlineto{\pgfqpoint{3.233319in}{3.062188in}}%
\pgfpathlineto{\pgfqpoint{3.233319in}{3.057930in}}%
\pgfpathmoveto{\pgfqpoint{3.229061in}{3.062188in}}%
\pgfpathlineto{\pgfqpoint{3.229061in}{3.062188in}}%
\pgfpathlineto{\pgfqpoint{3.229061in}{3.066446in}}%
\pgfpathlineto{\pgfqpoint{3.233319in}{3.066446in}}%
\pgfpathlineto{\pgfqpoint{3.233319in}{3.062188in}}%
\pgfpathmoveto{\pgfqpoint{3.229061in}{3.066446in}}%
\pgfpathlineto{\pgfqpoint{3.229061in}{3.066446in}}%
\pgfpathlineto{\pgfqpoint{3.229061in}{3.070703in}}%
\pgfpathlineto{\pgfqpoint{3.233319in}{3.070703in}}%
\pgfpathlineto{\pgfqpoint{3.233319in}{3.066446in}}%
\pgfpathmoveto{\pgfqpoint{3.497307in}{1.022692in}}%
\pgfpathlineto{\pgfqpoint{3.497307in}{1.022692in}}%
\pgfpathlineto{\pgfqpoint{3.497307in}{1.026950in}}%
\pgfpathlineto{\pgfqpoint{3.501565in}{1.026950in}}%
\pgfpathlineto{\pgfqpoint{3.501565in}{1.022692in}}%
\pgfpathmoveto{\pgfqpoint{3.497307in}{1.026950in}}%
\pgfpathlineto{\pgfqpoint{3.497307in}{1.026950in}}%
\pgfpathlineto{\pgfqpoint{3.497307in}{1.031208in}}%
\pgfpathlineto{\pgfqpoint{3.501565in}{1.031208in}}%
\pgfpathlineto{\pgfqpoint{3.501565in}{1.026950in}}%
\pgfpathmoveto{\pgfqpoint{3.497307in}{1.031208in}}%
\pgfpathlineto{\pgfqpoint{3.497307in}{1.031208in}}%
\pgfpathlineto{\pgfqpoint{3.497307in}{1.035466in}}%
\pgfpathlineto{\pgfqpoint{3.501565in}{1.035466in}}%
\pgfpathlineto{\pgfqpoint{3.501565in}{1.031208in}}%
\pgfpathmoveto{\pgfqpoint{3.497307in}{1.035466in}}%
\pgfpathlineto{\pgfqpoint{3.497307in}{1.035466in}}%
\pgfpathlineto{\pgfqpoint{3.497307in}{1.039723in}}%
\pgfpathlineto{\pgfqpoint{3.501565in}{1.039723in}}%
\pgfpathlineto{\pgfqpoint{3.501565in}{1.035466in}}%
\pgfpathmoveto{\pgfqpoint{3.497307in}{1.039723in}}%
\pgfpathlineto{\pgfqpoint{3.497307in}{1.039723in}}%
\pgfpathlineto{\pgfqpoint{3.497307in}{1.043981in}}%
\pgfpathlineto{\pgfqpoint{3.501565in}{1.043981in}}%
\pgfpathlineto{\pgfqpoint{3.501565in}{1.039723in}}%
\pgfpathmoveto{\pgfqpoint{3.497307in}{1.043981in}}%
\pgfpathlineto{\pgfqpoint{3.497307in}{1.043981in}}%
\pgfpathlineto{\pgfqpoint{3.497307in}{1.048239in}}%
\pgfpathlineto{\pgfqpoint{3.501565in}{1.048239in}}%
\pgfpathlineto{\pgfqpoint{3.501565in}{1.043981in}}%
\pgfpathmoveto{\pgfqpoint{3.497307in}{1.048239in}}%
\pgfpathlineto{\pgfqpoint{3.497307in}{1.048239in}}%
\pgfpathlineto{\pgfqpoint{3.497307in}{1.052497in}}%
\pgfpathlineto{\pgfqpoint{3.501565in}{1.052497in}}%
\pgfpathlineto{\pgfqpoint{3.501565in}{1.048239in}}%
\pgfpathmoveto{\pgfqpoint{3.497307in}{1.052497in}}%
\pgfpathlineto{\pgfqpoint{3.497307in}{1.052497in}}%
\pgfpathlineto{\pgfqpoint{3.497307in}{1.056755in}}%
\pgfpathlineto{\pgfqpoint{3.501565in}{1.056755in}}%
\pgfpathlineto{\pgfqpoint{3.501565in}{1.052497in}}%
\pgfpathmoveto{\pgfqpoint{3.497307in}{1.056755in}}%
\pgfpathlineto{\pgfqpoint{3.497307in}{1.056755in}}%
\pgfpathlineto{\pgfqpoint{3.497307in}{1.061013in}}%
\pgfpathlineto{\pgfqpoint{3.501565in}{1.061013in}}%
\pgfpathlineto{\pgfqpoint{3.501565in}{1.056755in}}%
\pgfpathmoveto{\pgfqpoint{3.493049in}{1.065271in}}%
\pgfpathlineto{\pgfqpoint{3.493049in}{1.065271in}}%
\pgfpathlineto{\pgfqpoint{3.493049in}{1.069529in}}%
\pgfpathlineto{\pgfqpoint{3.497307in}{1.069529in}}%
\pgfpathlineto{\pgfqpoint{3.497307in}{1.065271in}}%
\pgfpathmoveto{\pgfqpoint{3.497307in}{1.061013in}}%
\pgfpathlineto{\pgfqpoint{3.497307in}{1.061013in}}%
\pgfpathlineto{\pgfqpoint{3.497307in}{1.065271in}}%
\pgfpathlineto{\pgfqpoint{3.501565in}{1.065271in}}%
\pgfpathlineto{\pgfqpoint{3.501565in}{1.061013in}}%
\pgfpathmoveto{\pgfqpoint{3.497307in}{1.065271in}}%
\pgfpathlineto{\pgfqpoint{3.497307in}{1.065271in}}%
\pgfpathlineto{\pgfqpoint{3.497307in}{1.069529in}}%
\pgfpathlineto{\pgfqpoint{3.501565in}{1.069529in}}%
\pgfpathlineto{\pgfqpoint{3.501565in}{1.065271in}}%
\pgfpathmoveto{\pgfqpoint{3.493049in}{1.069529in}}%
\pgfpathlineto{\pgfqpoint{3.493049in}{1.069529in}}%
\pgfpathlineto{\pgfqpoint{3.493049in}{1.073787in}}%
\pgfpathlineto{\pgfqpoint{3.497307in}{1.073787in}}%
\pgfpathlineto{\pgfqpoint{3.497307in}{1.069529in}}%
\pgfpathmoveto{\pgfqpoint{3.493049in}{1.073787in}}%
\pgfpathlineto{\pgfqpoint{3.493049in}{1.073787in}}%
\pgfpathlineto{\pgfqpoint{3.493049in}{1.078045in}}%
\pgfpathlineto{\pgfqpoint{3.497307in}{1.078045in}}%
\pgfpathlineto{\pgfqpoint{3.497307in}{1.073787in}}%
\pgfpathmoveto{\pgfqpoint{3.493049in}{1.078045in}}%
\pgfpathlineto{\pgfqpoint{3.493049in}{1.078045in}}%
\pgfpathlineto{\pgfqpoint{3.493049in}{1.082303in}}%
\pgfpathlineto{\pgfqpoint{3.497307in}{1.082303in}}%
\pgfpathlineto{\pgfqpoint{3.497307in}{1.078045in}}%
\pgfpathmoveto{\pgfqpoint{3.493049in}{1.082303in}}%
\pgfpathlineto{\pgfqpoint{3.493049in}{1.082303in}}%
\pgfpathlineto{\pgfqpoint{3.493049in}{1.086561in}}%
\pgfpathlineto{\pgfqpoint{3.497307in}{1.086561in}}%
\pgfpathlineto{\pgfqpoint{3.497307in}{1.082303in}}%
\pgfpathmoveto{\pgfqpoint{3.493049in}{1.086561in}}%
\pgfpathlineto{\pgfqpoint{3.493049in}{1.086561in}}%
\pgfpathlineto{\pgfqpoint{3.493049in}{1.090819in}}%
\pgfpathlineto{\pgfqpoint{3.497307in}{1.090819in}}%
\pgfpathlineto{\pgfqpoint{3.497307in}{1.086561in}}%
\pgfpathmoveto{\pgfqpoint{3.493049in}{1.090819in}}%
\pgfpathlineto{\pgfqpoint{3.493049in}{1.090819in}}%
\pgfpathlineto{\pgfqpoint{3.493049in}{1.095077in}}%
\pgfpathlineto{\pgfqpoint{3.497307in}{1.095077in}}%
\pgfpathlineto{\pgfqpoint{3.497307in}{1.090819in}}%
\pgfpathmoveto{\pgfqpoint{3.493049in}{1.095077in}}%
\pgfpathlineto{\pgfqpoint{3.493049in}{1.095077in}}%
\pgfpathlineto{\pgfqpoint{3.493049in}{1.099335in}}%
\pgfpathlineto{\pgfqpoint{3.497307in}{1.099335in}}%
\pgfpathlineto{\pgfqpoint{3.497307in}{1.095077in}}%
\pgfpathmoveto{\pgfqpoint{3.493049in}{1.099335in}}%
\pgfpathlineto{\pgfqpoint{3.493049in}{1.099335in}}%
\pgfpathlineto{\pgfqpoint{3.493049in}{1.103593in}}%
\pgfpathlineto{\pgfqpoint{3.497307in}{1.103593in}}%
\pgfpathlineto{\pgfqpoint{3.497307in}{1.099335in}}%
\pgfpathmoveto{\pgfqpoint{3.488791in}{1.107851in}}%
\pgfpathlineto{\pgfqpoint{3.488791in}{1.107851in}}%
\pgfpathlineto{\pgfqpoint{3.488791in}{1.112109in}}%
\pgfpathlineto{\pgfqpoint{3.493049in}{1.112109in}}%
\pgfpathlineto{\pgfqpoint{3.493049in}{1.107851in}}%
\pgfpathmoveto{\pgfqpoint{3.488791in}{1.112109in}}%
\pgfpathlineto{\pgfqpoint{3.488791in}{1.112109in}}%
\pgfpathlineto{\pgfqpoint{3.488791in}{1.116367in}}%
\pgfpathlineto{\pgfqpoint{3.493049in}{1.116367in}}%
\pgfpathlineto{\pgfqpoint{3.493049in}{1.112109in}}%
\pgfpathmoveto{\pgfqpoint{3.488791in}{1.116367in}}%
\pgfpathlineto{\pgfqpoint{3.488791in}{1.116367in}}%
\pgfpathlineto{\pgfqpoint{3.488791in}{1.120625in}}%
\pgfpathlineto{\pgfqpoint{3.493049in}{1.120625in}}%
\pgfpathlineto{\pgfqpoint{3.493049in}{1.116367in}}%
\pgfpathmoveto{\pgfqpoint{3.493049in}{1.103593in}}%
\pgfpathlineto{\pgfqpoint{3.493049in}{1.103593in}}%
\pgfpathlineto{\pgfqpoint{3.493049in}{1.107851in}}%
\pgfpathlineto{\pgfqpoint{3.497307in}{1.107851in}}%
\pgfpathlineto{\pgfqpoint{3.497307in}{1.103593in}}%
\pgfpathmoveto{\pgfqpoint{3.493049in}{1.107851in}}%
\pgfpathlineto{\pgfqpoint{3.493049in}{1.107851in}}%
\pgfpathlineto{\pgfqpoint{3.493049in}{1.112109in}}%
\pgfpathlineto{\pgfqpoint{3.497307in}{1.112109in}}%
\pgfpathlineto{\pgfqpoint{3.497307in}{1.107851in}}%
\pgfpathmoveto{\pgfqpoint{3.488791in}{1.120625in}}%
\pgfpathlineto{\pgfqpoint{3.488791in}{1.120625in}}%
\pgfpathlineto{\pgfqpoint{3.488791in}{1.124884in}}%
\pgfpathlineto{\pgfqpoint{3.493049in}{1.124884in}}%
\pgfpathlineto{\pgfqpoint{3.493049in}{1.120625in}}%
\pgfpathmoveto{\pgfqpoint{3.488791in}{1.124884in}}%
\pgfpathlineto{\pgfqpoint{3.488791in}{1.124884in}}%
\pgfpathlineto{\pgfqpoint{3.488791in}{1.129142in}}%
\pgfpathlineto{\pgfqpoint{3.493049in}{1.129142in}}%
\pgfpathlineto{\pgfqpoint{3.493049in}{1.124884in}}%
\pgfpathmoveto{\pgfqpoint{3.488791in}{1.129142in}}%
\pgfpathlineto{\pgfqpoint{3.488791in}{1.129142in}}%
\pgfpathlineto{\pgfqpoint{3.488791in}{1.133400in}}%
\pgfpathlineto{\pgfqpoint{3.493049in}{1.133400in}}%
\pgfpathlineto{\pgfqpoint{3.493049in}{1.129142in}}%
\pgfpathmoveto{\pgfqpoint{3.488791in}{1.133400in}}%
\pgfpathlineto{\pgfqpoint{3.488791in}{1.133400in}}%
\pgfpathlineto{\pgfqpoint{3.488791in}{1.137658in}}%
\pgfpathlineto{\pgfqpoint{3.493049in}{1.137658in}}%
\pgfpathlineto{\pgfqpoint{3.493049in}{1.133400in}}%
\pgfpathmoveto{\pgfqpoint{3.488791in}{1.137658in}}%
\pgfpathlineto{\pgfqpoint{3.488791in}{1.137658in}}%
\pgfpathlineto{\pgfqpoint{3.488791in}{1.141916in}}%
\pgfpathlineto{\pgfqpoint{3.493049in}{1.141916in}}%
\pgfpathlineto{\pgfqpoint{3.493049in}{1.137658in}}%
\pgfpathmoveto{\pgfqpoint{3.488791in}{1.141916in}}%
\pgfpathlineto{\pgfqpoint{3.488791in}{1.141916in}}%
\pgfpathlineto{\pgfqpoint{3.488791in}{1.146174in}}%
\pgfpathlineto{\pgfqpoint{3.493049in}{1.146174in}}%
\pgfpathlineto{\pgfqpoint{3.493049in}{1.141916in}}%
\pgfpathmoveto{\pgfqpoint{3.484533in}{1.150432in}}%
\pgfpathlineto{\pgfqpoint{3.484533in}{1.150432in}}%
\pgfpathlineto{\pgfqpoint{3.484533in}{1.154690in}}%
\pgfpathlineto{\pgfqpoint{3.488791in}{1.154690in}}%
\pgfpathlineto{\pgfqpoint{3.488791in}{1.150432in}}%
\pgfpathmoveto{\pgfqpoint{3.488791in}{1.146174in}}%
\pgfpathlineto{\pgfqpoint{3.488791in}{1.146174in}}%
\pgfpathlineto{\pgfqpoint{3.488791in}{1.150432in}}%
\pgfpathlineto{\pgfqpoint{3.493049in}{1.150432in}}%
\pgfpathlineto{\pgfqpoint{3.493049in}{1.146174in}}%
\pgfpathmoveto{\pgfqpoint{3.488791in}{1.150432in}}%
\pgfpathlineto{\pgfqpoint{3.488791in}{1.150432in}}%
\pgfpathlineto{\pgfqpoint{3.488791in}{1.154690in}}%
\pgfpathlineto{\pgfqpoint{3.493049in}{1.154690in}}%
\pgfpathlineto{\pgfqpoint{3.493049in}{1.150432in}}%
\pgfpathmoveto{\pgfqpoint{3.484533in}{1.154690in}}%
\pgfpathlineto{\pgfqpoint{3.484533in}{1.154690in}}%
\pgfpathlineto{\pgfqpoint{3.484533in}{1.158948in}}%
\pgfpathlineto{\pgfqpoint{3.488791in}{1.158948in}}%
\pgfpathlineto{\pgfqpoint{3.488791in}{1.154690in}}%
\pgfpathmoveto{\pgfqpoint{3.484533in}{1.158948in}}%
\pgfpathlineto{\pgfqpoint{3.484533in}{1.158948in}}%
\pgfpathlineto{\pgfqpoint{3.484533in}{1.163206in}}%
\pgfpathlineto{\pgfqpoint{3.488791in}{1.163206in}}%
\pgfpathlineto{\pgfqpoint{3.488791in}{1.158948in}}%
\pgfpathmoveto{\pgfqpoint{3.484533in}{1.163206in}}%
\pgfpathlineto{\pgfqpoint{3.484533in}{1.163206in}}%
\pgfpathlineto{\pgfqpoint{3.484533in}{1.167464in}}%
\pgfpathlineto{\pgfqpoint{3.488791in}{1.167464in}}%
\pgfpathlineto{\pgfqpoint{3.488791in}{1.163206in}}%
\pgfpathmoveto{\pgfqpoint{3.484533in}{1.167464in}}%
\pgfpathlineto{\pgfqpoint{3.484533in}{1.167464in}}%
\pgfpathlineto{\pgfqpoint{3.484533in}{1.171722in}}%
\pgfpathlineto{\pgfqpoint{3.488791in}{1.171722in}}%
\pgfpathlineto{\pgfqpoint{3.488791in}{1.167464in}}%
\pgfpathmoveto{\pgfqpoint{3.484533in}{1.171722in}}%
\pgfpathlineto{\pgfqpoint{3.484533in}{1.171722in}}%
\pgfpathlineto{\pgfqpoint{3.484533in}{1.175980in}}%
\pgfpathlineto{\pgfqpoint{3.488791in}{1.175980in}}%
\pgfpathlineto{\pgfqpoint{3.488791in}{1.171722in}}%
\pgfpathmoveto{\pgfqpoint{3.484533in}{1.175980in}}%
\pgfpathlineto{\pgfqpoint{3.484533in}{1.175980in}}%
\pgfpathlineto{\pgfqpoint{3.484533in}{1.180238in}}%
\pgfpathlineto{\pgfqpoint{3.488791in}{1.180238in}}%
\pgfpathlineto{\pgfqpoint{3.488791in}{1.175980in}}%
\pgfpathmoveto{\pgfqpoint{3.484533in}{1.180238in}}%
\pgfpathlineto{\pgfqpoint{3.484533in}{1.180238in}}%
\pgfpathlineto{\pgfqpoint{3.484533in}{1.184496in}}%
\pgfpathlineto{\pgfqpoint{3.488791in}{1.184496in}}%
\pgfpathlineto{\pgfqpoint{3.488791in}{1.180238in}}%
\pgfpathmoveto{\pgfqpoint{3.484533in}{1.184496in}}%
\pgfpathlineto{\pgfqpoint{3.484533in}{1.184496in}}%
\pgfpathlineto{\pgfqpoint{3.484533in}{1.188754in}}%
\pgfpathlineto{\pgfqpoint{3.488791in}{1.188754in}}%
\pgfpathlineto{\pgfqpoint{3.488791in}{1.184496in}}%
\pgfpathmoveto{\pgfqpoint{3.480275in}{1.188754in}}%
\pgfpathlineto{\pgfqpoint{3.480275in}{1.188754in}}%
\pgfpathlineto{\pgfqpoint{3.480275in}{1.193012in}}%
\pgfpathlineto{\pgfqpoint{3.484533in}{1.193012in}}%
\pgfpathlineto{\pgfqpoint{3.484533in}{1.188754in}}%
\pgfpathmoveto{\pgfqpoint{3.480275in}{1.193012in}}%
\pgfpathlineto{\pgfqpoint{3.480275in}{1.193012in}}%
\pgfpathlineto{\pgfqpoint{3.480275in}{1.197269in}}%
\pgfpathlineto{\pgfqpoint{3.484533in}{1.197269in}}%
\pgfpathlineto{\pgfqpoint{3.484533in}{1.193012in}}%
\pgfpathmoveto{\pgfqpoint{3.480275in}{1.197269in}}%
\pgfpathlineto{\pgfqpoint{3.480275in}{1.197269in}}%
\pgfpathlineto{\pgfqpoint{3.480275in}{1.201527in}}%
\pgfpathlineto{\pgfqpoint{3.484533in}{1.201527in}}%
\pgfpathlineto{\pgfqpoint{3.484533in}{1.197269in}}%
\pgfpathmoveto{\pgfqpoint{3.480275in}{1.201527in}}%
\pgfpathlineto{\pgfqpoint{3.480275in}{1.201527in}}%
\pgfpathlineto{\pgfqpoint{3.480275in}{1.205785in}}%
\pgfpathlineto{\pgfqpoint{3.484533in}{1.205785in}}%
\pgfpathlineto{\pgfqpoint{3.484533in}{1.201527in}}%
\pgfpathmoveto{\pgfqpoint{3.480275in}{1.205785in}}%
\pgfpathlineto{\pgfqpoint{3.480275in}{1.205785in}}%
\pgfpathlineto{\pgfqpoint{3.480275in}{1.210042in}}%
\pgfpathlineto{\pgfqpoint{3.484533in}{1.210042in}}%
\pgfpathlineto{\pgfqpoint{3.484533in}{1.205785in}}%
\pgfpathmoveto{\pgfqpoint{3.480275in}{1.210042in}}%
\pgfpathlineto{\pgfqpoint{3.480275in}{1.210042in}}%
\pgfpathlineto{\pgfqpoint{3.480275in}{1.214300in}}%
\pgfpathlineto{\pgfqpoint{3.484533in}{1.214300in}}%
\pgfpathlineto{\pgfqpoint{3.484533in}{1.210042in}}%
\pgfpathmoveto{\pgfqpoint{3.480275in}{1.214300in}}%
\pgfpathlineto{\pgfqpoint{3.480275in}{1.214300in}}%
\pgfpathlineto{\pgfqpoint{3.480275in}{1.218558in}}%
\pgfpathlineto{\pgfqpoint{3.484533in}{1.218558in}}%
\pgfpathlineto{\pgfqpoint{3.484533in}{1.214300in}}%
\pgfpathmoveto{\pgfqpoint{3.480275in}{1.218558in}}%
\pgfpathlineto{\pgfqpoint{3.480275in}{1.218558in}}%
\pgfpathlineto{\pgfqpoint{3.480275in}{1.222815in}}%
\pgfpathlineto{\pgfqpoint{3.484533in}{1.222815in}}%
\pgfpathlineto{\pgfqpoint{3.484533in}{1.218558in}}%
\pgfpathmoveto{\pgfqpoint{3.484533in}{1.188754in}}%
\pgfpathlineto{\pgfqpoint{3.484533in}{1.188754in}}%
\pgfpathlineto{\pgfqpoint{3.484533in}{1.193012in}}%
\pgfpathlineto{\pgfqpoint{3.488791in}{1.193012in}}%
\pgfpathlineto{\pgfqpoint{3.488791in}{1.188754in}}%
\pgfpathmoveto{\pgfqpoint{3.480275in}{1.222815in}}%
\pgfpathlineto{\pgfqpoint{3.480275in}{1.222815in}}%
\pgfpathlineto{\pgfqpoint{3.480275in}{1.227073in}}%
\pgfpathlineto{\pgfqpoint{3.484533in}{1.227073in}}%
\pgfpathlineto{\pgfqpoint{3.484533in}{1.222815in}}%
\pgfpathmoveto{\pgfqpoint{3.480275in}{1.227073in}}%
\pgfpathlineto{\pgfqpoint{3.480275in}{1.227073in}}%
\pgfpathlineto{\pgfqpoint{3.480275in}{1.231330in}}%
\pgfpathlineto{\pgfqpoint{3.484533in}{1.231330in}}%
\pgfpathlineto{\pgfqpoint{3.484533in}{1.227073in}}%
\pgfpathmoveto{\pgfqpoint{3.476017in}{1.231330in}}%
\pgfpathlineto{\pgfqpoint{3.476017in}{1.231330in}}%
\pgfpathlineto{\pgfqpoint{3.476017in}{1.235588in}}%
\pgfpathlineto{\pgfqpoint{3.480275in}{1.235588in}}%
\pgfpathlineto{\pgfqpoint{3.480275in}{1.231330in}}%
\pgfpathmoveto{\pgfqpoint{3.476017in}{1.235588in}}%
\pgfpathlineto{\pgfqpoint{3.476017in}{1.235588in}}%
\pgfpathlineto{\pgfqpoint{3.476017in}{1.239846in}}%
\pgfpathlineto{\pgfqpoint{3.480275in}{1.239846in}}%
\pgfpathlineto{\pgfqpoint{3.480275in}{1.235588in}}%
\pgfpathmoveto{\pgfqpoint{3.480275in}{1.231330in}}%
\pgfpathlineto{\pgfqpoint{3.480275in}{1.231330in}}%
\pgfpathlineto{\pgfqpoint{3.480275in}{1.235588in}}%
\pgfpathlineto{\pgfqpoint{3.484533in}{1.235588in}}%
\pgfpathlineto{\pgfqpoint{3.484533in}{1.231330in}}%
\pgfpathmoveto{\pgfqpoint{3.476017in}{1.239846in}}%
\pgfpathlineto{\pgfqpoint{3.476017in}{1.239846in}}%
\pgfpathlineto{\pgfqpoint{3.476017in}{1.244103in}}%
\pgfpathlineto{\pgfqpoint{3.480275in}{1.244103in}}%
\pgfpathlineto{\pgfqpoint{3.480275in}{1.239846in}}%
\pgfpathmoveto{\pgfqpoint{3.476017in}{1.244103in}}%
\pgfpathlineto{\pgfqpoint{3.476017in}{1.244103in}}%
\pgfpathlineto{\pgfqpoint{3.476017in}{1.248361in}}%
\pgfpathlineto{\pgfqpoint{3.480275in}{1.248361in}}%
\pgfpathlineto{\pgfqpoint{3.480275in}{1.244103in}}%
\pgfpathmoveto{\pgfqpoint{3.476017in}{1.248361in}}%
\pgfpathlineto{\pgfqpoint{3.476017in}{1.248361in}}%
\pgfpathlineto{\pgfqpoint{3.476017in}{1.252619in}}%
\pgfpathlineto{\pgfqpoint{3.480275in}{1.252619in}}%
\pgfpathlineto{\pgfqpoint{3.480275in}{1.248361in}}%
\pgfpathmoveto{\pgfqpoint{3.476017in}{1.252619in}}%
\pgfpathlineto{\pgfqpoint{3.476017in}{1.252619in}}%
\pgfpathlineto{\pgfqpoint{3.476017in}{1.256876in}}%
\pgfpathlineto{\pgfqpoint{3.480275in}{1.256876in}}%
\pgfpathlineto{\pgfqpoint{3.480275in}{1.252619in}}%
\pgfpathmoveto{\pgfqpoint{3.476017in}{1.256876in}}%
\pgfpathlineto{\pgfqpoint{3.476017in}{1.256876in}}%
\pgfpathlineto{\pgfqpoint{3.476017in}{1.261134in}}%
\pgfpathlineto{\pgfqpoint{3.480275in}{1.261134in}}%
\pgfpathlineto{\pgfqpoint{3.480275in}{1.256876in}}%
\pgfpathmoveto{\pgfqpoint{3.476017in}{1.261134in}}%
\pgfpathlineto{\pgfqpoint{3.476017in}{1.261134in}}%
\pgfpathlineto{\pgfqpoint{3.476017in}{1.265392in}}%
\pgfpathlineto{\pgfqpoint{3.480275in}{1.265392in}}%
\pgfpathlineto{\pgfqpoint{3.480275in}{1.261134in}}%
\pgfpathmoveto{\pgfqpoint{3.476017in}{1.265392in}}%
\pgfpathlineto{\pgfqpoint{3.476017in}{1.265392in}}%
\pgfpathlineto{\pgfqpoint{3.476017in}{1.269649in}}%
\pgfpathlineto{\pgfqpoint{3.480275in}{1.269649in}}%
\pgfpathlineto{\pgfqpoint{3.480275in}{1.265392in}}%
\pgfpathmoveto{\pgfqpoint{3.476017in}{1.269649in}}%
\pgfpathlineto{\pgfqpoint{3.476017in}{1.269649in}}%
\pgfpathlineto{\pgfqpoint{3.476017in}{1.273907in}}%
\pgfpathlineto{\pgfqpoint{3.480275in}{1.273907in}}%
\pgfpathlineto{\pgfqpoint{3.480275in}{1.269649in}}%
\pgfpathmoveto{\pgfqpoint{3.471759in}{1.273907in}}%
\pgfpathlineto{\pgfqpoint{3.471759in}{1.273907in}}%
\pgfpathlineto{\pgfqpoint{3.471759in}{1.278165in}}%
\pgfpathlineto{\pgfqpoint{3.476017in}{1.278165in}}%
\pgfpathlineto{\pgfqpoint{3.476017in}{1.273907in}}%
\pgfpathmoveto{\pgfqpoint{3.471759in}{1.278165in}}%
\pgfpathlineto{\pgfqpoint{3.471759in}{1.278165in}}%
\pgfpathlineto{\pgfqpoint{3.471759in}{1.282422in}}%
\pgfpathlineto{\pgfqpoint{3.476017in}{1.282422in}}%
\pgfpathlineto{\pgfqpoint{3.476017in}{1.278165in}}%
\pgfpathmoveto{\pgfqpoint{3.471759in}{1.282422in}}%
\pgfpathlineto{\pgfqpoint{3.471759in}{1.282422in}}%
\pgfpathlineto{\pgfqpoint{3.471759in}{1.286680in}}%
\pgfpathlineto{\pgfqpoint{3.476017in}{1.286680in}}%
\pgfpathlineto{\pgfqpoint{3.476017in}{1.282422in}}%
\pgfpathmoveto{\pgfqpoint{3.471759in}{1.286680in}}%
\pgfpathlineto{\pgfqpoint{3.471759in}{1.286680in}}%
\pgfpathlineto{\pgfqpoint{3.471759in}{1.290938in}}%
\pgfpathlineto{\pgfqpoint{3.476017in}{1.290938in}}%
\pgfpathlineto{\pgfqpoint{3.476017in}{1.286680in}}%
\pgfpathmoveto{\pgfqpoint{3.476017in}{1.273907in}}%
\pgfpathlineto{\pgfqpoint{3.476017in}{1.273907in}}%
\pgfpathlineto{\pgfqpoint{3.476017in}{1.278165in}}%
\pgfpathlineto{\pgfqpoint{3.480275in}{1.278165in}}%
\pgfpathlineto{\pgfqpoint{3.480275in}{1.273907in}}%
\pgfpathmoveto{\pgfqpoint{3.471759in}{1.290938in}}%
\pgfpathlineto{\pgfqpoint{3.471759in}{1.290938in}}%
\pgfpathlineto{\pgfqpoint{3.471759in}{1.295195in}}%
\pgfpathlineto{\pgfqpoint{3.476017in}{1.295195in}}%
\pgfpathlineto{\pgfqpoint{3.476017in}{1.290938in}}%
\pgfpathmoveto{\pgfqpoint{3.471759in}{1.295195in}}%
\pgfpathlineto{\pgfqpoint{3.471759in}{1.295195in}}%
\pgfpathlineto{\pgfqpoint{3.471759in}{1.299453in}}%
\pgfpathlineto{\pgfqpoint{3.476017in}{1.299453in}}%
\pgfpathlineto{\pgfqpoint{3.476017in}{1.295195in}}%
\pgfpathmoveto{\pgfqpoint{3.471759in}{1.299453in}}%
\pgfpathlineto{\pgfqpoint{3.471759in}{1.299453in}}%
\pgfpathlineto{\pgfqpoint{3.471759in}{1.303711in}}%
\pgfpathlineto{\pgfqpoint{3.476017in}{1.303711in}}%
\pgfpathlineto{\pgfqpoint{3.476017in}{1.299453in}}%
\pgfpathmoveto{\pgfqpoint{3.471759in}{1.303711in}}%
\pgfpathlineto{\pgfqpoint{3.471759in}{1.303711in}}%
\pgfpathlineto{\pgfqpoint{3.471759in}{1.307968in}}%
\pgfpathlineto{\pgfqpoint{3.476017in}{1.307968in}}%
\pgfpathlineto{\pgfqpoint{3.476017in}{1.303711in}}%
\pgfpathmoveto{\pgfqpoint{3.467501in}{1.312226in}}%
\pgfpathlineto{\pgfqpoint{3.467501in}{1.312226in}}%
\pgfpathlineto{\pgfqpoint{3.467501in}{1.316484in}}%
\pgfpathlineto{\pgfqpoint{3.471759in}{1.316484in}}%
\pgfpathlineto{\pgfqpoint{3.471759in}{1.312226in}}%
\pgfpathmoveto{\pgfqpoint{3.471759in}{1.307968in}}%
\pgfpathlineto{\pgfqpoint{3.471759in}{1.307968in}}%
\pgfpathlineto{\pgfqpoint{3.471759in}{1.312226in}}%
\pgfpathlineto{\pgfqpoint{3.476017in}{1.312226in}}%
\pgfpathlineto{\pgfqpoint{3.476017in}{1.307968in}}%
\pgfpathmoveto{\pgfqpoint{3.471759in}{1.312226in}}%
\pgfpathlineto{\pgfqpoint{3.471759in}{1.312226in}}%
\pgfpathlineto{\pgfqpoint{3.471759in}{1.316484in}}%
\pgfpathlineto{\pgfqpoint{3.476017in}{1.316484in}}%
\pgfpathlineto{\pgfqpoint{3.476017in}{1.312226in}}%
\pgfpathmoveto{\pgfqpoint{3.467501in}{1.316484in}}%
\pgfpathlineto{\pgfqpoint{3.467501in}{1.316484in}}%
\pgfpathlineto{\pgfqpoint{3.467501in}{1.320741in}}%
\pgfpathlineto{\pgfqpoint{3.471759in}{1.320741in}}%
\pgfpathlineto{\pgfqpoint{3.471759in}{1.316484in}}%
\pgfpathmoveto{\pgfqpoint{3.467501in}{1.320741in}}%
\pgfpathlineto{\pgfqpoint{3.467501in}{1.320741in}}%
\pgfpathlineto{\pgfqpoint{3.467501in}{1.324999in}}%
\pgfpathlineto{\pgfqpoint{3.471759in}{1.324999in}}%
\pgfpathlineto{\pgfqpoint{3.471759in}{1.320741in}}%
\pgfpathmoveto{\pgfqpoint{3.463243in}{1.354804in}}%
\pgfpathlineto{\pgfqpoint{3.463243in}{1.354804in}}%
\pgfpathlineto{\pgfqpoint{3.463243in}{1.359062in}}%
\pgfpathlineto{\pgfqpoint{3.467501in}{1.359062in}}%
\pgfpathlineto{\pgfqpoint{3.467501in}{1.354804in}}%
\pgfpathmoveto{\pgfqpoint{3.463243in}{1.359062in}}%
\pgfpathlineto{\pgfqpoint{3.463243in}{1.359062in}}%
\pgfpathlineto{\pgfqpoint{3.463243in}{1.363320in}}%
\pgfpathlineto{\pgfqpoint{3.467501in}{1.363320in}}%
\pgfpathlineto{\pgfqpoint{3.467501in}{1.359062in}}%
\pgfpathmoveto{\pgfqpoint{3.463243in}{1.363320in}}%
\pgfpathlineto{\pgfqpoint{3.463243in}{1.363320in}}%
\pgfpathlineto{\pgfqpoint{3.463243in}{1.367577in}}%
\pgfpathlineto{\pgfqpoint{3.467501in}{1.367577in}}%
\pgfpathlineto{\pgfqpoint{3.467501in}{1.363320in}}%
\pgfpathmoveto{\pgfqpoint{3.463243in}{1.367577in}}%
\pgfpathlineto{\pgfqpoint{3.463243in}{1.367577in}}%
\pgfpathlineto{\pgfqpoint{3.463243in}{1.371835in}}%
\pgfpathlineto{\pgfqpoint{3.467501in}{1.371835in}}%
\pgfpathlineto{\pgfqpoint{3.467501in}{1.367577in}}%
\pgfpathmoveto{\pgfqpoint{3.463243in}{1.371835in}}%
\pgfpathlineto{\pgfqpoint{3.463243in}{1.371835in}}%
\pgfpathlineto{\pgfqpoint{3.463243in}{1.376093in}}%
\pgfpathlineto{\pgfqpoint{3.467501in}{1.376093in}}%
\pgfpathlineto{\pgfqpoint{3.467501in}{1.371835in}}%
\pgfpathmoveto{\pgfqpoint{3.463243in}{1.376093in}}%
\pgfpathlineto{\pgfqpoint{3.463243in}{1.376093in}}%
\pgfpathlineto{\pgfqpoint{3.463243in}{1.380351in}}%
\pgfpathlineto{\pgfqpoint{3.467501in}{1.380351in}}%
\pgfpathlineto{\pgfqpoint{3.467501in}{1.376093in}}%
\pgfpathmoveto{\pgfqpoint{3.463243in}{1.380351in}}%
\pgfpathlineto{\pgfqpoint{3.463243in}{1.380351in}}%
\pgfpathlineto{\pgfqpoint{3.463243in}{1.384609in}}%
\pgfpathlineto{\pgfqpoint{3.467501in}{1.384609in}}%
\pgfpathlineto{\pgfqpoint{3.467501in}{1.380351in}}%
\pgfpathmoveto{\pgfqpoint{3.463243in}{1.384609in}}%
\pgfpathlineto{\pgfqpoint{3.463243in}{1.384609in}}%
\pgfpathlineto{\pgfqpoint{3.463243in}{1.388867in}}%
\pgfpathlineto{\pgfqpoint{3.467501in}{1.388867in}}%
\pgfpathlineto{\pgfqpoint{3.467501in}{1.384609in}}%
\pgfpathmoveto{\pgfqpoint{3.463243in}{1.388867in}}%
\pgfpathlineto{\pgfqpoint{3.463243in}{1.388867in}}%
\pgfpathlineto{\pgfqpoint{3.463243in}{1.393124in}}%
\pgfpathlineto{\pgfqpoint{3.467501in}{1.393124in}}%
\pgfpathlineto{\pgfqpoint{3.467501in}{1.388867in}}%
\pgfpathmoveto{\pgfqpoint{3.467501in}{1.324999in}}%
\pgfpathlineto{\pgfqpoint{3.467501in}{1.324999in}}%
\pgfpathlineto{\pgfqpoint{3.467501in}{1.329257in}}%
\pgfpathlineto{\pgfqpoint{3.471759in}{1.329257in}}%
\pgfpathlineto{\pgfqpoint{3.471759in}{1.324999in}}%
\pgfpathmoveto{\pgfqpoint{3.467501in}{1.329257in}}%
\pgfpathlineto{\pgfqpoint{3.467501in}{1.329257in}}%
\pgfpathlineto{\pgfqpoint{3.467501in}{1.333515in}}%
\pgfpathlineto{\pgfqpoint{3.471759in}{1.333515in}}%
\pgfpathlineto{\pgfqpoint{3.471759in}{1.329257in}}%
\pgfpathmoveto{\pgfqpoint{3.467501in}{1.333515in}}%
\pgfpathlineto{\pgfqpoint{3.467501in}{1.333515in}}%
\pgfpathlineto{\pgfqpoint{3.467501in}{1.337772in}}%
\pgfpathlineto{\pgfqpoint{3.471759in}{1.337772in}}%
\pgfpathlineto{\pgfqpoint{3.471759in}{1.333515in}}%
\pgfpathmoveto{\pgfqpoint{3.467501in}{1.337772in}}%
\pgfpathlineto{\pgfqpoint{3.467501in}{1.337772in}}%
\pgfpathlineto{\pgfqpoint{3.467501in}{1.342030in}}%
\pgfpathlineto{\pgfqpoint{3.471759in}{1.342030in}}%
\pgfpathlineto{\pgfqpoint{3.471759in}{1.337772in}}%
\pgfpathmoveto{\pgfqpoint{3.467501in}{1.342030in}}%
\pgfpathlineto{\pgfqpoint{3.467501in}{1.342030in}}%
\pgfpathlineto{\pgfqpoint{3.467501in}{1.346288in}}%
\pgfpathlineto{\pgfqpoint{3.471759in}{1.346288in}}%
\pgfpathlineto{\pgfqpoint{3.471759in}{1.342030in}}%
\pgfpathmoveto{\pgfqpoint{3.467501in}{1.346288in}}%
\pgfpathlineto{\pgfqpoint{3.467501in}{1.346288in}}%
\pgfpathlineto{\pgfqpoint{3.467501in}{1.350546in}}%
\pgfpathlineto{\pgfqpoint{3.471759in}{1.350546in}}%
\pgfpathlineto{\pgfqpoint{3.471759in}{1.346288in}}%
\pgfpathmoveto{\pgfqpoint{3.467501in}{1.350546in}}%
\pgfpathlineto{\pgfqpoint{3.467501in}{1.350546in}}%
\pgfpathlineto{\pgfqpoint{3.467501in}{1.354804in}}%
\pgfpathlineto{\pgfqpoint{3.471759in}{1.354804in}}%
\pgfpathlineto{\pgfqpoint{3.471759in}{1.350546in}}%
\pgfpathmoveto{\pgfqpoint{3.467501in}{1.354804in}}%
\pgfpathlineto{\pgfqpoint{3.467501in}{1.354804in}}%
\pgfpathlineto{\pgfqpoint{3.467501in}{1.359062in}}%
\pgfpathlineto{\pgfqpoint{3.471759in}{1.359062in}}%
\pgfpathlineto{\pgfqpoint{3.471759in}{1.354804in}}%
\pgfpathmoveto{\pgfqpoint{3.458985in}{1.393124in}}%
\pgfpathlineto{\pgfqpoint{3.458985in}{1.393124in}}%
\pgfpathlineto{\pgfqpoint{3.458985in}{1.397382in}}%
\pgfpathlineto{\pgfqpoint{3.463243in}{1.397382in}}%
\pgfpathlineto{\pgfqpoint{3.463243in}{1.393124in}}%
\pgfpathmoveto{\pgfqpoint{3.458985in}{1.397382in}}%
\pgfpathlineto{\pgfqpoint{3.458985in}{1.397382in}}%
\pgfpathlineto{\pgfqpoint{3.458985in}{1.401640in}}%
\pgfpathlineto{\pgfqpoint{3.463243in}{1.401640in}}%
\pgfpathlineto{\pgfqpoint{3.463243in}{1.397382in}}%
\pgfpathmoveto{\pgfqpoint{3.463243in}{1.393124in}}%
\pgfpathlineto{\pgfqpoint{3.463243in}{1.393124in}}%
\pgfpathlineto{\pgfqpoint{3.463243in}{1.397382in}}%
\pgfpathlineto{\pgfqpoint{3.467501in}{1.397382in}}%
\pgfpathlineto{\pgfqpoint{3.467501in}{1.393124in}}%
\pgfpathmoveto{\pgfqpoint{3.458985in}{1.401640in}}%
\pgfpathlineto{\pgfqpoint{3.458985in}{1.401640in}}%
\pgfpathlineto{\pgfqpoint{3.458985in}{1.405898in}}%
\pgfpathlineto{\pgfqpoint{3.463243in}{1.405898in}}%
\pgfpathlineto{\pgfqpoint{3.463243in}{1.401640in}}%
\pgfpathmoveto{\pgfqpoint{3.458985in}{1.405898in}}%
\pgfpathlineto{\pgfqpoint{3.458985in}{1.405898in}}%
\pgfpathlineto{\pgfqpoint{3.458985in}{1.410156in}}%
\pgfpathlineto{\pgfqpoint{3.463243in}{1.410156in}}%
\pgfpathlineto{\pgfqpoint{3.463243in}{1.405898in}}%
\pgfpathmoveto{\pgfqpoint{3.458985in}{1.410156in}}%
\pgfpathlineto{\pgfqpoint{3.458985in}{1.410156in}}%
\pgfpathlineto{\pgfqpoint{3.458985in}{1.414414in}}%
\pgfpathlineto{\pgfqpoint{3.463243in}{1.414414in}}%
\pgfpathlineto{\pgfqpoint{3.463243in}{1.410156in}}%
\pgfpathmoveto{\pgfqpoint{3.458985in}{1.414414in}}%
\pgfpathlineto{\pgfqpoint{3.458985in}{1.414414in}}%
\pgfpathlineto{\pgfqpoint{3.458985in}{1.418672in}}%
\pgfpathlineto{\pgfqpoint{3.463243in}{1.418672in}}%
\pgfpathlineto{\pgfqpoint{3.463243in}{1.414414in}}%
\pgfpathmoveto{\pgfqpoint{3.458985in}{1.418672in}}%
\pgfpathlineto{\pgfqpoint{3.458985in}{1.418672in}}%
\pgfpathlineto{\pgfqpoint{3.458985in}{1.422929in}}%
\pgfpathlineto{\pgfqpoint{3.463243in}{1.422929in}}%
\pgfpathlineto{\pgfqpoint{3.463243in}{1.418672in}}%
\pgfpathmoveto{\pgfqpoint{3.458985in}{1.422929in}}%
\pgfpathlineto{\pgfqpoint{3.458985in}{1.422929in}}%
\pgfpathlineto{\pgfqpoint{3.458985in}{1.427187in}}%
\pgfpathlineto{\pgfqpoint{3.463243in}{1.427187in}}%
\pgfpathlineto{\pgfqpoint{3.463243in}{1.422929in}}%
\pgfpathmoveto{\pgfqpoint{3.454727in}{1.431445in}}%
\pgfpathlineto{\pgfqpoint{3.454727in}{1.431445in}}%
\pgfpathlineto{\pgfqpoint{3.454727in}{1.435703in}}%
\pgfpathlineto{\pgfqpoint{3.458985in}{1.435703in}}%
\pgfpathlineto{\pgfqpoint{3.458985in}{1.431445in}}%
\pgfpathmoveto{\pgfqpoint{3.454727in}{1.435703in}}%
\pgfpathlineto{\pgfqpoint{3.454727in}{1.435703in}}%
\pgfpathlineto{\pgfqpoint{3.454727in}{1.439961in}}%
\pgfpathlineto{\pgfqpoint{3.458985in}{1.439961in}}%
\pgfpathlineto{\pgfqpoint{3.458985in}{1.435703in}}%
\pgfpathmoveto{\pgfqpoint{3.454727in}{1.439961in}}%
\pgfpathlineto{\pgfqpoint{3.454727in}{1.439961in}}%
\pgfpathlineto{\pgfqpoint{3.454727in}{1.444219in}}%
\pgfpathlineto{\pgfqpoint{3.458985in}{1.444219in}}%
\pgfpathlineto{\pgfqpoint{3.458985in}{1.439961in}}%
\pgfpathmoveto{\pgfqpoint{3.458985in}{1.427187in}}%
\pgfpathlineto{\pgfqpoint{3.458985in}{1.427187in}}%
\pgfpathlineto{\pgfqpoint{3.458985in}{1.431445in}}%
\pgfpathlineto{\pgfqpoint{3.463243in}{1.431445in}}%
\pgfpathlineto{\pgfqpoint{3.463243in}{1.427187in}}%
\pgfpathmoveto{\pgfqpoint{3.458985in}{1.431445in}}%
\pgfpathlineto{\pgfqpoint{3.458985in}{1.431445in}}%
\pgfpathlineto{\pgfqpoint{3.458985in}{1.435703in}}%
\pgfpathlineto{\pgfqpoint{3.463243in}{1.435703in}}%
\pgfpathlineto{\pgfqpoint{3.463243in}{1.431445in}}%
\pgfpathmoveto{\pgfqpoint{3.454727in}{1.444219in}}%
\pgfpathlineto{\pgfqpoint{3.454727in}{1.444219in}}%
\pgfpathlineto{\pgfqpoint{3.454727in}{1.448477in}}%
\pgfpathlineto{\pgfqpoint{3.458985in}{1.448477in}}%
\pgfpathlineto{\pgfqpoint{3.458985in}{1.444219in}}%
\pgfpathmoveto{\pgfqpoint{3.454727in}{1.448477in}}%
\pgfpathlineto{\pgfqpoint{3.454727in}{1.448477in}}%
\pgfpathlineto{\pgfqpoint{3.454727in}{1.452734in}}%
\pgfpathlineto{\pgfqpoint{3.458985in}{1.452734in}}%
\pgfpathlineto{\pgfqpoint{3.458985in}{1.448477in}}%
\pgfpathmoveto{\pgfqpoint{3.454727in}{1.452734in}}%
\pgfpathlineto{\pgfqpoint{3.454727in}{1.452734in}}%
\pgfpathlineto{\pgfqpoint{3.454727in}{1.456992in}}%
\pgfpathlineto{\pgfqpoint{3.458985in}{1.456992in}}%
\pgfpathlineto{\pgfqpoint{3.458985in}{1.452734in}}%
\pgfpathmoveto{\pgfqpoint{3.454727in}{1.456992in}}%
\pgfpathlineto{\pgfqpoint{3.454727in}{1.456992in}}%
\pgfpathlineto{\pgfqpoint{3.454727in}{1.461250in}}%
\pgfpathlineto{\pgfqpoint{3.458985in}{1.461250in}}%
\pgfpathlineto{\pgfqpoint{3.458985in}{1.456992in}}%
\pgfpathmoveto{\pgfqpoint{3.454727in}{1.461250in}}%
\pgfpathlineto{\pgfqpoint{3.454727in}{1.461250in}}%
\pgfpathlineto{\pgfqpoint{3.454727in}{1.465508in}}%
\pgfpathlineto{\pgfqpoint{3.458985in}{1.465508in}}%
\pgfpathlineto{\pgfqpoint{3.458985in}{1.461250in}}%
\pgfpathmoveto{\pgfqpoint{3.454727in}{1.465508in}}%
\pgfpathlineto{\pgfqpoint{3.454727in}{1.465508in}}%
\pgfpathlineto{\pgfqpoint{3.454727in}{1.469766in}}%
\pgfpathlineto{\pgfqpoint{3.458985in}{1.469766in}}%
\pgfpathlineto{\pgfqpoint{3.458985in}{1.465508in}}%
\pgfpathmoveto{\pgfqpoint{3.450469in}{1.469766in}}%
\pgfpathlineto{\pgfqpoint{3.450469in}{1.469766in}}%
\pgfpathlineto{\pgfqpoint{3.450469in}{1.474023in}}%
\pgfpathlineto{\pgfqpoint{3.454727in}{1.474023in}}%
\pgfpathlineto{\pgfqpoint{3.454727in}{1.469766in}}%
\pgfpathmoveto{\pgfqpoint{3.450469in}{1.474023in}}%
\pgfpathlineto{\pgfqpoint{3.450469in}{1.474023in}}%
\pgfpathlineto{\pgfqpoint{3.450469in}{1.478281in}}%
\pgfpathlineto{\pgfqpoint{3.454727in}{1.478281in}}%
\pgfpathlineto{\pgfqpoint{3.454727in}{1.474023in}}%
\pgfpathmoveto{\pgfqpoint{3.454727in}{1.469766in}}%
\pgfpathlineto{\pgfqpoint{3.454727in}{1.469766in}}%
\pgfpathlineto{\pgfqpoint{3.454727in}{1.474023in}}%
\pgfpathlineto{\pgfqpoint{3.458985in}{1.474023in}}%
\pgfpathlineto{\pgfqpoint{3.458985in}{1.469766in}}%
\pgfpathmoveto{\pgfqpoint{3.450469in}{1.478281in}}%
\pgfpathlineto{\pgfqpoint{3.450469in}{1.478281in}}%
\pgfpathlineto{\pgfqpoint{3.450469in}{1.482539in}}%
\pgfpathlineto{\pgfqpoint{3.454727in}{1.482539in}}%
\pgfpathlineto{\pgfqpoint{3.454727in}{1.478281in}}%
\pgfpathmoveto{\pgfqpoint{3.450469in}{1.482539in}}%
\pgfpathlineto{\pgfqpoint{3.450469in}{1.482539in}}%
\pgfpathlineto{\pgfqpoint{3.450469in}{1.486797in}}%
\pgfpathlineto{\pgfqpoint{3.454727in}{1.486797in}}%
\pgfpathlineto{\pgfqpoint{3.454727in}{1.482539in}}%
\pgfpathmoveto{\pgfqpoint{3.450469in}{1.486797in}}%
\pgfpathlineto{\pgfqpoint{3.450469in}{1.486797in}}%
\pgfpathlineto{\pgfqpoint{3.450469in}{1.491055in}}%
\pgfpathlineto{\pgfqpoint{3.454727in}{1.491055in}}%
\pgfpathlineto{\pgfqpoint{3.454727in}{1.486797in}}%
\pgfpathmoveto{\pgfqpoint{3.450469in}{1.491055in}}%
\pgfpathlineto{\pgfqpoint{3.450469in}{1.491055in}}%
\pgfpathlineto{\pgfqpoint{3.450469in}{1.495312in}}%
\pgfpathlineto{\pgfqpoint{3.454727in}{1.495312in}}%
\pgfpathlineto{\pgfqpoint{3.454727in}{1.491055in}}%
\pgfpathmoveto{\pgfqpoint{3.446211in}{1.508086in}}%
\pgfpathlineto{\pgfqpoint{3.446211in}{1.508086in}}%
\pgfpathlineto{\pgfqpoint{3.446211in}{1.512343in}}%
\pgfpathlineto{\pgfqpoint{3.450469in}{1.512343in}}%
\pgfpathlineto{\pgfqpoint{3.450469in}{1.508086in}}%
\pgfpathmoveto{\pgfqpoint{3.446211in}{1.512343in}}%
\pgfpathlineto{\pgfqpoint{3.446211in}{1.512343in}}%
\pgfpathlineto{\pgfqpoint{3.446211in}{1.516601in}}%
\pgfpathlineto{\pgfqpoint{3.450469in}{1.516601in}}%
\pgfpathlineto{\pgfqpoint{3.450469in}{1.512343in}}%
\pgfpathmoveto{\pgfqpoint{3.446211in}{1.516601in}}%
\pgfpathlineto{\pgfqpoint{3.446211in}{1.516601in}}%
\pgfpathlineto{\pgfqpoint{3.446211in}{1.520859in}}%
\pgfpathlineto{\pgfqpoint{3.450469in}{1.520859in}}%
\pgfpathlineto{\pgfqpoint{3.450469in}{1.516601in}}%
\pgfpathmoveto{\pgfqpoint{3.446211in}{1.520859in}}%
\pgfpathlineto{\pgfqpoint{3.446211in}{1.520859in}}%
\pgfpathlineto{\pgfqpoint{3.446211in}{1.525117in}}%
\pgfpathlineto{\pgfqpoint{3.450469in}{1.525117in}}%
\pgfpathlineto{\pgfqpoint{3.450469in}{1.520859in}}%
\pgfpathmoveto{\pgfqpoint{3.446211in}{1.525117in}}%
\pgfpathlineto{\pgfqpoint{3.446211in}{1.525117in}}%
\pgfpathlineto{\pgfqpoint{3.446211in}{1.529375in}}%
\pgfpathlineto{\pgfqpoint{3.450469in}{1.529375in}}%
\pgfpathlineto{\pgfqpoint{3.450469in}{1.525117in}}%
\pgfpathmoveto{\pgfqpoint{3.450469in}{1.495312in}}%
\pgfpathlineto{\pgfqpoint{3.450469in}{1.495312in}}%
\pgfpathlineto{\pgfqpoint{3.450469in}{1.499570in}}%
\pgfpathlineto{\pgfqpoint{3.454727in}{1.499570in}}%
\pgfpathlineto{\pgfqpoint{3.454727in}{1.495312in}}%
\pgfpathmoveto{\pgfqpoint{3.450469in}{1.499570in}}%
\pgfpathlineto{\pgfqpoint{3.450469in}{1.499570in}}%
\pgfpathlineto{\pgfqpoint{3.450469in}{1.503828in}}%
\pgfpathlineto{\pgfqpoint{3.454727in}{1.503828in}}%
\pgfpathlineto{\pgfqpoint{3.454727in}{1.499570in}}%
\pgfpathmoveto{\pgfqpoint{3.450469in}{1.503828in}}%
\pgfpathlineto{\pgfqpoint{3.450469in}{1.503828in}}%
\pgfpathlineto{\pgfqpoint{3.450469in}{1.508086in}}%
\pgfpathlineto{\pgfqpoint{3.454727in}{1.508086in}}%
\pgfpathlineto{\pgfqpoint{3.454727in}{1.503828in}}%
\pgfpathmoveto{\pgfqpoint{3.450469in}{1.508086in}}%
\pgfpathlineto{\pgfqpoint{3.450469in}{1.508086in}}%
\pgfpathlineto{\pgfqpoint{3.450469in}{1.512343in}}%
\pgfpathlineto{\pgfqpoint{3.454727in}{1.512343in}}%
\pgfpathlineto{\pgfqpoint{3.454727in}{1.508086in}}%
\pgfpathmoveto{\pgfqpoint{3.446211in}{1.529375in}}%
\pgfpathlineto{\pgfqpoint{3.446211in}{1.529375in}}%
\pgfpathlineto{\pgfqpoint{3.446211in}{1.533632in}}%
\pgfpathlineto{\pgfqpoint{3.450469in}{1.533632in}}%
\pgfpathlineto{\pgfqpoint{3.450469in}{1.529375in}}%
\pgfpathmoveto{\pgfqpoint{3.446211in}{1.533632in}}%
\pgfpathlineto{\pgfqpoint{3.446211in}{1.533632in}}%
\pgfpathlineto{\pgfqpoint{3.446211in}{1.537890in}}%
\pgfpathlineto{\pgfqpoint{3.450469in}{1.537890in}}%
\pgfpathlineto{\pgfqpoint{3.450469in}{1.533632in}}%
\pgfpathmoveto{\pgfqpoint{3.446211in}{1.537890in}}%
\pgfpathlineto{\pgfqpoint{3.446211in}{1.537890in}}%
\pgfpathlineto{\pgfqpoint{3.446211in}{1.542148in}}%
\pgfpathlineto{\pgfqpoint{3.450469in}{1.542148in}}%
\pgfpathlineto{\pgfqpoint{3.450469in}{1.537890in}}%
\pgfpathmoveto{\pgfqpoint{3.446211in}{1.542148in}}%
\pgfpathlineto{\pgfqpoint{3.446211in}{1.542148in}}%
\pgfpathlineto{\pgfqpoint{3.446211in}{1.546406in}}%
\pgfpathlineto{\pgfqpoint{3.450469in}{1.546406in}}%
\pgfpathlineto{\pgfqpoint{3.450469in}{1.542148in}}%
\pgfpathmoveto{\pgfqpoint{3.441953in}{1.546406in}}%
\pgfpathlineto{\pgfqpoint{3.441953in}{1.546406in}}%
\pgfpathlineto{\pgfqpoint{3.441953in}{1.550663in}}%
\pgfpathlineto{\pgfqpoint{3.446211in}{1.550663in}}%
\pgfpathlineto{\pgfqpoint{3.446211in}{1.546406in}}%
\pgfpathmoveto{\pgfqpoint{3.441953in}{1.550663in}}%
\pgfpathlineto{\pgfqpoint{3.441953in}{1.550663in}}%
\pgfpathlineto{\pgfqpoint{3.441953in}{1.554921in}}%
\pgfpathlineto{\pgfqpoint{3.446211in}{1.554921in}}%
\pgfpathlineto{\pgfqpoint{3.446211in}{1.550663in}}%
\pgfpathmoveto{\pgfqpoint{3.446211in}{1.546406in}}%
\pgfpathlineto{\pgfqpoint{3.446211in}{1.546406in}}%
\pgfpathlineto{\pgfqpoint{3.446211in}{1.550663in}}%
\pgfpathlineto{\pgfqpoint{3.450469in}{1.550663in}}%
\pgfpathlineto{\pgfqpoint{3.450469in}{1.546406in}}%
\pgfpathmoveto{\pgfqpoint{3.441953in}{1.554921in}}%
\pgfpathlineto{\pgfqpoint{3.441953in}{1.554921in}}%
\pgfpathlineto{\pgfqpoint{3.441953in}{1.559179in}}%
\pgfpathlineto{\pgfqpoint{3.446211in}{1.559179in}}%
\pgfpathlineto{\pgfqpoint{3.446211in}{1.554921in}}%
\pgfpathmoveto{\pgfqpoint{3.441953in}{1.559179in}}%
\pgfpathlineto{\pgfqpoint{3.441953in}{1.559179in}}%
\pgfpathlineto{\pgfqpoint{3.441953in}{1.563437in}}%
\pgfpathlineto{\pgfqpoint{3.446211in}{1.563437in}}%
\pgfpathlineto{\pgfqpoint{3.446211in}{1.559179in}}%
\pgfpathmoveto{\pgfqpoint{3.441953in}{1.563437in}}%
\pgfpathlineto{\pgfqpoint{3.441953in}{1.563437in}}%
\pgfpathlineto{\pgfqpoint{3.441953in}{1.567695in}}%
\pgfpathlineto{\pgfqpoint{3.446211in}{1.567695in}}%
\pgfpathlineto{\pgfqpoint{3.446211in}{1.563437in}}%
\pgfpathmoveto{\pgfqpoint{3.441953in}{1.567695in}}%
\pgfpathlineto{\pgfqpoint{3.441953in}{1.567695in}}%
\pgfpathlineto{\pgfqpoint{3.441953in}{1.571952in}}%
\pgfpathlineto{\pgfqpoint{3.446211in}{1.571952in}}%
\pgfpathlineto{\pgfqpoint{3.446211in}{1.567695in}}%
\pgfpathmoveto{\pgfqpoint{3.441953in}{1.571952in}}%
\pgfpathlineto{\pgfqpoint{3.441953in}{1.571952in}}%
\pgfpathlineto{\pgfqpoint{3.441953in}{1.576210in}}%
\pgfpathlineto{\pgfqpoint{3.446211in}{1.576210in}}%
\pgfpathlineto{\pgfqpoint{3.446211in}{1.571952in}}%
\pgfpathmoveto{\pgfqpoint{3.441953in}{1.576210in}}%
\pgfpathlineto{\pgfqpoint{3.441953in}{1.576210in}}%
\pgfpathlineto{\pgfqpoint{3.441953in}{1.580468in}}%
\pgfpathlineto{\pgfqpoint{3.446211in}{1.580468in}}%
\pgfpathlineto{\pgfqpoint{3.446211in}{1.576210in}}%
\pgfpathmoveto{\pgfqpoint{3.437695in}{1.584726in}}%
\pgfpathlineto{\pgfqpoint{3.437695in}{1.584726in}}%
\pgfpathlineto{\pgfqpoint{3.437695in}{1.588983in}}%
\pgfpathlineto{\pgfqpoint{3.441953in}{1.588983in}}%
\pgfpathlineto{\pgfqpoint{3.441953in}{1.584726in}}%
\pgfpathmoveto{\pgfqpoint{3.437695in}{1.588983in}}%
\pgfpathlineto{\pgfqpoint{3.437695in}{1.588983in}}%
\pgfpathlineto{\pgfqpoint{3.437695in}{1.593241in}}%
\pgfpathlineto{\pgfqpoint{3.441953in}{1.593241in}}%
\pgfpathlineto{\pgfqpoint{3.441953in}{1.588983in}}%
\pgfpathmoveto{\pgfqpoint{3.437695in}{1.593241in}}%
\pgfpathlineto{\pgfqpoint{3.437695in}{1.593241in}}%
\pgfpathlineto{\pgfqpoint{3.437695in}{1.597499in}}%
\pgfpathlineto{\pgfqpoint{3.441953in}{1.597499in}}%
\pgfpathlineto{\pgfqpoint{3.441953in}{1.593241in}}%
\pgfpathmoveto{\pgfqpoint{3.441953in}{1.580468in}}%
\pgfpathlineto{\pgfqpoint{3.441953in}{1.580468in}}%
\pgfpathlineto{\pgfqpoint{3.441953in}{1.584726in}}%
\pgfpathlineto{\pgfqpoint{3.446211in}{1.584726in}}%
\pgfpathlineto{\pgfqpoint{3.446211in}{1.580468in}}%
\pgfpathmoveto{\pgfqpoint{3.441953in}{1.584726in}}%
\pgfpathlineto{\pgfqpoint{3.441953in}{1.584726in}}%
\pgfpathlineto{\pgfqpoint{3.441953in}{1.588983in}}%
\pgfpathlineto{\pgfqpoint{3.446211in}{1.588983in}}%
\pgfpathlineto{\pgfqpoint{3.446211in}{1.584726in}}%
\pgfpathmoveto{\pgfqpoint{3.429179in}{1.661367in}}%
\pgfpathlineto{\pgfqpoint{3.429179in}{1.661367in}}%
\pgfpathlineto{\pgfqpoint{3.429179in}{1.665625in}}%
\pgfpathlineto{\pgfqpoint{3.433437in}{1.665625in}}%
\pgfpathlineto{\pgfqpoint{3.433437in}{1.661367in}}%
\pgfpathmoveto{\pgfqpoint{3.429179in}{1.665625in}}%
\pgfpathlineto{\pgfqpoint{3.429179in}{1.665625in}}%
\pgfpathlineto{\pgfqpoint{3.429179in}{1.669883in}}%
\pgfpathlineto{\pgfqpoint{3.433437in}{1.669883in}}%
\pgfpathlineto{\pgfqpoint{3.433437in}{1.665625in}}%
\pgfpathmoveto{\pgfqpoint{3.429179in}{1.669883in}}%
\pgfpathlineto{\pgfqpoint{3.429179in}{1.669883in}}%
\pgfpathlineto{\pgfqpoint{3.429179in}{1.674141in}}%
\pgfpathlineto{\pgfqpoint{3.433437in}{1.674141in}}%
\pgfpathlineto{\pgfqpoint{3.433437in}{1.669883in}}%
\pgfpathmoveto{\pgfqpoint{3.429179in}{1.674141in}}%
\pgfpathlineto{\pgfqpoint{3.429179in}{1.674141in}}%
\pgfpathlineto{\pgfqpoint{3.429179in}{1.678399in}}%
\pgfpathlineto{\pgfqpoint{3.433437in}{1.678399in}}%
\pgfpathlineto{\pgfqpoint{3.433437in}{1.674141in}}%
\pgfpathmoveto{\pgfqpoint{3.429179in}{1.678399in}}%
\pgfpathlineto{\pgfqpoint{3.429179in}{1.678399in}}%
\pgfpathlineto{\pgfqpoint{3.429179in}{1.682657in}}%
\pgfpathlineto{\pgfqpoint{3.433437in}{1.682657in}}%
\pgfpathlineto{\pgfqpoint{3.433437in}{1.678399in}}%
\pgfpathmoveto{\pgfqpoint{3.429179in}{1.682657in}}%
\pgfpathlineto{\pgfqpoint{3.429179in}{1.682657in}}%
\pgfpathlineto{\pgfqpoint{3.429179in}{1.686915in}}%
\pgfpathlineto{\pgfqpoint{3.433437in}{1.686915in}}%
\pgfpathlineto{\pgfqpoint{3.433437in}{1.682657in}}%
\pgfpathmoveto{\pgfqpoint{3.429179in}{1.686915in}}%
\pgfpathlineto{\pgfqpoint{3.429179in}{1.686915in}}%
\pgfpathlineto{\pgfqpoint{3.429179in}{1.691173in}}%
\pgfpathlineto{\pgfqpoint{3.433437in}{1.691173in}}%
\pgfpathlineto{\pgfqpoint{3.433437in}{1.686915in}}%
\pgfpathmoveto{\pgfqpoint{3.429179in}{1.691173in}}%
\pgfpathlineto{\pgfqpoint{3.429179in}{1.691173in}}%
\pgfpathlineto{\pgfqpoint{3.429179in}{1.695430in}}%
\pgfpathlineto{\pgfqpoint{3.433437in}{1.695430in}}%
\pgfpathlineto{\pgfqpoint{3.433437in}{1.691173in}}%
\pgfpathmoveto{\pgfqpoint{3.429179in}{1.695430in}}%
\pgfpathlineto{\pgfqpoint{3.429179in}{1.695430in}}%
\pgfpathlineto{\pgfqpoint{3.429179in}{1.699688in}}%
\pgfpathlineto{\pgfqpoint{3.433437in}{1.699688in}}%
\pgfpathlineto{\pgfqpoint{3.433437in}{1.695430in}}%
\pgfpathmoveto{\pgfqpoint{3.424921in}{1.699688in}}%
\pgfpathlineto{\pgfqpoint{3.424921in}{1.699688in}}%
\pgfpathlineto{\pgfqpoint{3.424921in}{1.703946in}}%
\pgfpathlineto{\pgfqpoint{3.429179in}{1.703946in}}%
\pgfpathlineto{\pgfqpoint{3.429179in}{1.699688in}}%
\pgfpathmoveto{\pgfqpoint{3.424921in}{1.703946in}}%
\pgfpathlineto{\pgfqpoint{3.424921in}{1.703946in}}%
\pgfpathlineto{\pgfqpoint{3.424921in}{1.708204in}}%
\pgfpathlineto{\pgfqpoint{3.429179in}{1.708204in}}%
\pgfpathlineto{\pgfqpoint{3.429179in}{1.703946in}}%
\pgfpathmoveto{\pgfqpoint{3.429179in}{1.699688in}}%
\pgfpathlineto{\pgfqpoint{3.429179in}{1.699688in}}%
\pgfpathlineto{\pgfqpoint{3.429179in}{1.703946in}}%
\pgfpathlineto{\pgfqpoint{3.433437in}{1.703946in}}%
\pgfpathlineto{\pgfqpoint{3.433437in}{1.699688in}}%
\pgfpathmoveto{\pgfqpoint{3.424921in}{1.708204in}}%
\pgfpathlineto{\pgfqpoint{3.424921in}{1.708204in}}%
\pgfpathlineto{\pgfqpoint{3.424921in}{1.712462in}}%
\pgfpathlineto{\pgfqpoint{3.429179in}{1.712462in}}%
\pgfpathlineto{\pgfqpoint{3.429179in}{1.708204in}}%
\pgfpathmoveto{\pgfqpoint{3.424921in}{1.712462in}}%
\pgfpathlineto{\pgfqpoint{3.424921in}{1.712462in}}%
\pgfpathlineto{\pgfqpoint{3.424921in}{1.716720in}}%
\pgfpathlineto{\pgfqpoint{3.429179in}{1.716720in}}%
\pgfpathlineto{\pgfqpoint{3.429179in}{1.712462in}}%
\pgfpathmoveto{\pgfqpoint{3.424921in}{1.716720in}}%
\pgfpathlineto{\pgfqpoint{3.424921in}{1.716720in}}%
\pgfpathlineto{\pgfqpoint{3.424921in}{1.720978in}}%
\pgfpathlineto{\pgfqpoint{3.429179in}{1.720978in}}%
\pgfpathlineto{\pgfqpoint{3.429179in}{1.716720in}}%
\pgfpathmoveto{\pgfqpoint{3.424921in}{1.720978in}}%
\pgfpathlineto{\pgfqpoint{3.424921in}{1.720978in}}%
\pgfpathlineto{\pgfqpoint{3.424921in}{1.725236in}}%
\pgfpathlineto{\pgfqpoint{3.429179in}{1.725236in}}%
\pgfpathlineto{\pgfqpoint{3.429179in}{1.720978in}}%
\pgfpathmoveto{\pgfqpoint{3.424921in}{1.725236in}}%
\pgfpathlineto{\pgfqpoint{3.424921in}{1.725236in}}%
\pgfpathlineto{\pgfqpoint{3.424921in}{1.729494in}}%
\pgfpathlineto{\pgfqpoint{3.429179in}{1.729494in}}%
\pgfpathlineto{\pgfqpoint{3.429179in}{1.725236in}}%
\pgfpathmoveto{\pgfqpoint{3.424921in}{1.729494in}}%
\pgfpathlineto{\pgfqpoint{3.424921in}{1.729494in}}%
\pgfpathlineto{\pgfqpoint{3.424921in}{1.733751in}}%
\pgfpathlineto{\pgfqpoint{3.429179in}{1.733751in}}%
\pgfpathlineto{\pgfqpoint{3.429179in}{1.729494in}}%
\pgfpathmoveto{\pgfqpoint{3.437695in}{1.597499in}}%
\pgfpathlineto{\pgfqpoint{3.437695in}{1.597499in}}%
\pgfpathlineto{\pgfqpoint{3.437695in}{1.601757in}}%
\pgfpathlineto{\pgfqpoint{3.441953in}{1.601757in}}%
\pgfpathlineto{\pgfqpoint{3.441953in}{1.597499in}}%
\pgfpathmoveto{\pgfqpoint{3.437695in}{1.601757in}}%
\pgfpathlineto{\pgfqpoint{3.437695in}{1.601757in}}%
\pgfpathlineto{\pgfqpoint{3.437695in}{1.606015in}}%
\pgfpathlineto{\pgfqpoint{3.441953in}{1.606015in}}%
\pgfpathlineto{\pgfqpoint{3.441953in}{1.601757in}}%
\pgfpathmoveto{\pgfqpoint{3.437695in}{1.606015in}}%
\pgfpathlineto{\pgfqpoint{3.437695in}{1.606015in}}%
\pgfpathlineto{\pgfqpoint{3.437695in}{1.610273in}}%
\pgfpathlineto{\pgfqpoint{3.441953in}{1.610273in}}%
\pgfpathlineto{\pgfqpoint{3.441953in}{1.606015in}}%
\pgfpathmoveto{\pgfqpoint{3.437695in}{1.610273in}}%
\pgfpathlineto{\pgfqpoint{3.437695in}{1.610273in}}%
\pgfpathlineto{\pgfqpoint{3.437695in}{1.614531in}}%
\pgfpathlineto{\pgfqpoint{3.441953in}{1.614531in}}%
\pgfpathlineto{\pgfqpoint{3.441953in}{1.610273in}}%
\pgfpathmoveto{\pgfqpoint{3.437695in}{1.614531in}}%
\pgfpathlineto{\pgfqpoint{3.437695in}{1.614531in}}%
\pgfpathlineto{\pgfqpoint{3.437695in}{1.618788in}}%
\pgfpathlineto{\pgfqpoint{3.441953in}{1.618788in}}%
\pgfpathlineto{\pgfqpoint{3.441953in}{1.614531in}}%
\pgfpathmoveto{\pgfqpoint{3.437695in}{1.618788in}}%
\pgfpathlineto{\pgfqpoint{3.437695in}{1.618788in}}%
\pgfpathlineto{\pgfqpoint{3.437695in}{1.623046in}}%
\pgfpathlineto{\pgfqpoint{3.441953in}{1.623046in}}%
\pgfpathlineto{\pgfqpoint{3.441953in}{1.618788in}}%
\pgfpathmoveto{\pgfqpoint{3.433437in}{1.623046in}}%
\pgfpathlineto{\pgfqpoint{3.433437in}{1.623046in}}%
\pgfpathlineto{\pgfqpoint{3.433437in}{1.627304in}}%
\pgfpathlineto{\pgfqpoint{3.437695in}{1.627304in}}%
\pgfpathlineto{\pgfqpoint{3.437695in}{1.623046in}}%
\pgfpathmoveto{\pgfqpoint{3.433437in}{1.627304in}}%
\pgfpathlineto{\pgfqpoint{3.433437in}{1.627304in}}%
\pgfpathlineto{\pgfqpoint{3.433437in}{1.631562in}}%
\pgfpathlineto{\pgfqpoint{3.437695in}{1.631562in}}%
\pgfpathlineto{\pgfqpoint{3.437695in}{1.627304in}}%
\pgfpathmoveto{\pgfqpoint{3.437695in}{1.623046in}}%
\pgfpathlineto{\pgfqpoint{3.437695in}{1.623046in}}%
\pgfpathlineto{\pgfqpoint{3.437695in}{1.627304in}}%
\pgfpathlineto{\pgfqpoint{3.441953in}{1.627304in}}%
\pgfpathlineto{\pgfqpoint{3.441953in}{1.623046in}}%
\pgfpathmoveto{\pgfqpoint{3.433437in}{1.631562in}}%
\pgfpathlineto{\pgfqpoint{3.433437in}{1.631562in}}%
\pgfpathlineto{\pgfqpoint{3.433437in}{1.635820in}}%
\pgfpathlineto{\pgfqpoint{3.437695in}{1.635820in}}%
\pgfpathlineto{\pgfqpoint{3.437695in}{1.631562in}}%
\pgfpathmoveto{\pgfqpoint{3.433437in}{1.635820in}}%
\pgfpathlineto{\pgfqpoint{3.433437in}{1.635820in}}%
\pgfpathlineto{\pgfqpoint{3.433437in}{1.640078in}}%
\pgfpathlineto{\pgfqpoint{3.437695in}{1.640078in}}%
\pgfpathlineto{\pgfqpoint{3.437695in}{1.635820in}}%
\pgfpathmoveto{\pgfqpoint{3.433437in}{1.640078in}}%
\pgfpathlineto{\pgfqpoint{3.433437in}{1.640078in}}%
\pgfpathlineto{\pgfqpoint{3.433437in}{1.644336in}}%
\pgfpathlineto{\pgfqpoint{3.437695in}{1.644336in}}%
\pgfpathlineto{\pgfqpoint{3.437695in}{1.640078in}}%
\pgfpathmoveto{\pgfqpoint{3.433437in}{1.644336in}}%
\pgfpathlineto{\pgfqpoint{3.433437in}{1.644336in}}%
\pgfpathlineto{\pgfqpoint{3.433437in}{1.648594in}}%
\pgfpathlineto{\pgfqpoint{3.437695in}{1.648594in}}%
\pgfpathlineto{\pgfqpoint{3.437695in}{1.644336in}}%
\pgfpathmoveto{\pgfqpoint{3.433437in}{1.648594in}}%
\pgfpathlineto{\pgfqpoint{3.433437in}{1.648594in}}%
\pgfpathlineto{\pgfqpoint{3.433437in}{1.652852in}}%
\pgfpathlineto{\pgfqpoint{3.437695in}{1.652852in}}%
\pgfpathlineto{\pgfqpoint{3.437695in}{1.648594in}}%
\pgfpathmoveto{\pgfqpoint{3.433437in}{1.652852in}}%
\pgfpathlineto{\pgfqpoint{3.433437in}{1.652852in}}%
\pgfpathlineto{\pgfqpoint{3.433437in}{1.657109in}}%
\pgfpathlineto{\pgfqpoint{3.437695in}{1.657109in}}%
\pgfpathlineto{\pgfqpoint{3.437695in}{1.652852in}}%
\pgfpathmoveto{\pgfqpoint{3.433437in}{1.657109in}}%
\pgfpathlineto{\pgfqpoint{3.433437in}{1.657109in}}%
\pgfpathlineto{\pgfqpoint{3.433437in}{1.661367in}}%
\pgfpathlineto{\pgfqpoint{3.437695in}{1.661367in}}%
\pgfpathlineto{\pgfqpoint{3.437695in}{1.657109in}}%
\pgfpathmoveto{\pgfqpoint{3.433437in}{1.661367in}}%
\pgfpathlineto{\pgfqpoint{3.433437in}{1.661367in}}%
\pgfpathlineto{\pgfqpoint{3.433437in}{1.665625in}}%
\pgfpathlineto{\pgfqpoint{3.437695in}{1.665625in}}%
\pgfpathlineto{\pgfqpoint{3.437695in}{1.661367in}}%
\pgfpathmoveto{\pgfqpoint{3.420663in}{1.733751in}}%
\pgfpathlineto{\pgfqpoint{3.420663in}{1.733751in}}%
\pgfpathlineto{\pgfqpoint{3.420663in}{1.738009in}}%
\pgfpathlineto{\pgfqpoint{3.424921in}{1.738009in}}%
\pgfpathlineto{\pgfqpoint{3.424921in}{1.733751in}}%
\pgfpathmoveto{\pgfqpoint{3.420663in}{1.738009in}}%
\pgfpathlineto{\pgfqpoint{3.420663in}{1.738009in}}%
\pgfpathlineto{\pgfqpoint{3.420663in}{1.742267in}}%
\pgfpathlineto{\pgfqpoint{3.424921in}{1.742267in}}%
\pgfpathlineto{\pgfqpoint{3.424921in}{1.738009in}}%
\pgfpathmoveto{\pgfqpoint{3.420663in}{1.742267in}}%
\pgfpathlineto{\pgfqpoint{3.420663in}{1.742267in}}%
\pgfpathlineto{\pgfqpoint{3.420663in}{1.746525in}}%
\pgfpathlineto{\pgfqpoint{3.424921in}{1.746525in}}%
\pgfpathlineto{\pgfqpoint{3.424921in}{1.742267in}}%
\pgfpathmoveto{\pgfqpoint{3.420663in}{1.746525in}}%
\pgfpathlineto{\pgfqpoint{3.420663in}{1.746525in}}%
\pgfpathlineto{\pgfqpoint{3.420663in}{1.750783in}}%
\pgfpathlineto{\pgfqpoint{3.424921in}{1.750783in}}%
\pgfpathlineto{\pgfqpoint{3.424921in}{1.746525in}}%
\pgfpathmoveto{\pgfqpoint{3.424921in}{1.733751in}}%
\pgfpathlineto{\pgfqpoint{3.424921in}{1.733751in}}%
\pgfpathlineto{\pgfqpoint{3.424921in}{1.738009in}}%
\pgfpathlineto{\pgfqpoint{3.429179in}{1.738009in}}%
\pgfpathlineto{\pgfqpoint{3.429179in}{1.733751in}}%
\pgfpathmoveto{\pgfqpoint{3.420663in}{1.750783in}}%
\pgfpathlineto{\pgfqpoint{3.420663in}{1.750783in}}%
\pgfpathlineto{\pgfqpoint{3.420663in}{1.755040in}}%
\pgfpathlineto{\pgfqpoint{3.424921in}{1.755040in}}%
\pgfpathlineto{\pgfqpoint{3.424921in}{1.750783in}}%
\pgfpathmoveto{\pgfqpoint{3.420663in}{1.755040in}}%
\pgfpathlineto{\pgfqpoint{3.420663in}{1.755040in}}%
\pgfpathlineto{\pgfqpoint{3.420663in}{1.759298in}}%
\pgfpathlineto{\pgfqpoint{3.424921in}{1.759298in}}%
\pgfpathlineto{\pgfqpoint{3.424921in}{1.755040in}}%
\pgfpathmoveto{\pgfqpoint{3.420663in}{1.759298in}}%
\pgfpathlineto{\pgfqpoint{3.420663in}{1.759298in}}%
\pgfpathlineto{\pgfqpoint{3.420663in}{1.763556in}}%
\pgfpathlineto{\pgfqpoint{3.424921in}{1.763556in}}%
\pgfpathlineto{\pgfqpoint{3.424921in}{1.759298in}}%
\pgfpathmoveto{\pgfqpoint{3.420663in}{1.763556in}}%
\pgfpathlineto{\pgfqpoint{3.420663in}{1.763556in}}%
\pgfpathlineto{\pgfqpoint{3.420663in}{1.767814in}}%
\pgfpathlineto{\pgfqpoint{3.424921in}{1.767814in}}%
\pgfpathlineto{\pgfqpoint{3.424921in}{1.763556in}}%
\pgfpathmoveto{\pgfqpoint{3.416405in}{1.772071in}}%
\pgfpathlineto{\pgfqpoint{3.416405in}{1.772071in}}%
\pgfpathlineto{\pgfqpoint{3.416405in}{1.776329in}}%
\pgfpathlineto{\pgfqpoint{3.420663in}{1.776329in}}%
\pgfpathlineto{\pgfqpoint{3.420663in}{1.772071in}}%
\pgfpathmoveto{\pgfqpoint{3.420663in}{1.767814in}}%
\pgfpathlineto{\pgfqpoint{3.420663in}{1.767814in}}%
\pgfpathlineto{\pgfqpoint{3.420663in}{1.772071in}}%
\pgfpathlineto{\pgfqpoint{3.424921in}{1.772071in}}%
\pgfpathlineto{\pgfqpoint{3.424921in}{1.767814in}}%
\pgfpathmoveto{\pgfqpoint{3.420663in}{1.772071in}}%
\pgfpathlineto{\pgfqpoint{3.420663in}{1.772071in}}%
\pgfpathlineto{\pgfqpoint{3.420663in}{1.776329in}}%
\pgfpathlineto{\pgfqpoint{3.424921in}{1.776329in}}%
\pgfpathlineto{\pgfqpoint{3.424921in}{1.772071in}}%
\pgfpathmoveto{\pgfqpoint{3.416405in}{1.776329in}}%
\pgfpathlineto{\pgfqpoint{3.416405in}{1.776329in}}%
\pgfpathlineto{\pgfqpoint{3.416405in}{1.780587in}}%
\pgfpathlineto{\pgfqpoint{3.420663in}{1.780587in}}%
\pgfpathlineto{\pgfqpoint{3.420663in}{1.776329in}}%
\pgfpathmoveto{\pgfqpoint{3.416405in}{1.780587in}}%
\pgfpathlineto{\pgfqpoint{3.416405in}{1.780587in}}%
\pgfpathlineto{\pgfqpoint{3.416405in}{1.784845in}}%
\pgfpathlineto{\pgfqpoint{3.420663in}{1.784845in}}%
\pgfpathlineto{\pgfqpoint{3.420663in}{1.780587in}}%
\pgfpathmoveto{\pgfqpoint{3.416405in}{1.784845in}}%
\pgfpathlineto{\pgfqpoint{3.416405in}{1.784845in}}%
\pgfpathlineto{\pgfqpoint{3.416405in}{1.789102in}}%
\pgfpathlineto{\pgfqpoint{3.420663in}{1.789102in}}%
\pgfpathlineto{\pgfqpoint{3.420663in}{1.784845in}}%
\pgfpathmoveto{\pgfqpoint{3.416405in}{1.789102in}}%
\pgfpathlineto{\pgfqpoint{3.416405in}{1.789102in}}%
\pgfpathlineto{\pgfqpoint{3.416405in}{1.793360in}}%
\pgfpathlineto{\pgfqpoint{3.420663in}{1.793360in}}%
\pgfpathlineto{\pgfqpoint{3.420663in}{1.789102in}}%
\pgfpathmoveto{\pgfqpoint{3.416405in}{1.793360in}}%
\pgfpathlineto{\pgfqpoint{3.416405in}{1.793360in}}%
\pgfpathlineto{\pgfqpoint{3.416405in}{1.797618in}}%
\pgfpathlineto{\pgfqpoint{3.420663in}{1.797618in}}%
\pgfpathlineto{\pgfqpoint{3.420663in}{1.793360in}}%
\pgfpathmoveto{\pgfqpoint{3.416405in}{1.797618in}}%
\pgfpathlineto{\pgfqpoint{3.416405in}{1.797618in}}%
\pgfpathlineto{\pgfqpoint{3.416405in}{1.801876in}}%
\pgfpathlineto{\pgfqpoint{3.420663in}{1.801876in}}%
\pgfpathlineto{\pgfqpoint{3.420663in}{1.797618in}}%
\pgfpathmoveto{\pgfqpoint{3.412147in}{1.806134in}}%
\pgfpathlineto{\pgfqpoint{3.412147in}{1.806134in}}%
\pgfpathlineto{\pgfqpoint{3.412147in}{1.810391in}}%
\pgfpathlineto{\pgfqpoint{3.416405in}{1.810391in}}%
\pgfpathlineto{\pgfqpoint{3.416405in}{1.806134in}}%
\pgfpathmoveto{\pgfqpoint{3.412147in}{1.810391in}}%
\pgfpathlineto{\pgfqpoint{3.412147in}{1.810391in}}%
\pgfpathlineto{\pgfqpoint{3.412147in}{1.814649in}}%
\pgfpathlineto{\pgfqpoint{3.416405in}{1.814649in}}%
\pgfpathlineto{\pgfqpoint{3.416405in}{1.810391in}}%
\pgfpathmoveto{\pgfqpoint{3.412147in}{1.814649in}}%
\pgfpathlineto{\pgfqpoint{3.412147in}{1.814649in}}%
\pgfpathlineto{\pgfqpoint{3.412147in}{1.818907in}}%
\pgfpathlineto{\pgfqpoint{3.416405in}{1.818907in}}%
\pgfpathlineto{\pgfqpoint{3.416405in}{1.814649in}}%
\pgfpathmoveto{\pgfqpoint{3.412147in}{1.818907in}}%
\pgfpathlineto{\pgfqpoint{3.412147in}{1.818907in}}%
\pgfpathlineto{\pgfqpoint{3.412147in}{1.823165in}}%
\pgfpathlineto{\pgfqpoint{3.416405in}{1.823165in}}%
\pgfpathlineto{\pgfqpoint{3.416405in}{1.818907in}}%
\pgfpathmoveto{\pgfqpoint{3.412147in}{1.823165in}}%
\pgfpathlineto{\pgfqpoint{3.412147in}{1.823165in}}%
\pgfpathlineto{\pgfqpoint{3.412147in}{1.827422in}}%
\pgfpathlineto{\pgfqpoint{3.416405in}{1.827422in}}%
\pgfpathlineto{\pgfqpoint{3.416405in}{1.823165in}}%
\pgfpathmoveto{\pgfqpoint{3.412147in}{1.827422in}}%
\pgfpathlineto{\pgfqpoint{3.412147in}{1.827422in}}%
\pgfpathlineto{\pgfqpoint{3.412147in}{1.831680in}}%
\pgfpathlineto{\pgfqpoint{3.416405in}{1.831680in}}%
\pgfpathlineto{\pgfqpoint{3.416405in}{1.827422in}}%
\pgfpathmoveto{\pgfqpoint{3.412147in}{1.831680in}}%
\pgfpathlineto{\pgfqpoint{3.412147in}{1.831680in}}%
\pgfpathlineto{\pgfqpoint{3.412147in}{1.835938in}}%
\pgfpathlineto{\pgfqpoint{3.416405in}{1.835938in}}%
\pgfpathlineto{\pgfqpoint{3.416405in}{1.831680in}}%
\pgfpathmoveto{\pgfqpoint{3.416405in}{1.801876in}}%
\pgfpathlineto{\pgfqpoint{3.416405in}{1.801876in}}%
\pgfpathlineto{\pgfqpoint{3.416405in}{1.806134in}}%
\pgfpathlineto{\pgfqpoint{3.420663in}{1.806134in}}%
\pgfpathlineto{\pgfqpoint{3.420663in}{1.801876in}}%
\pgfpathmoveto{\pgfqpoint{3.416405in}{1.806134in}}%
\pgfpathlineto{\pgfqpoint{3.416405in}{1.806134in}}%
\pgfpathlineto{\pgfqpoint{3.416405in}{1.810391in}}%
\pgfpathlineto{\pgfqpoint{3.420663in}{1.810391in}}%
\pgfpathlineto{\pgfqpoint{3.420663in}{1.806134in}}%
\pgfpathmoveto{\pgfqpoint{3.412147in}{1.835938in}}%
\pgfpathlineto{\pgfqpoint{3.412147in}{1.835938in}}%
\pgfpathlineto{\pgfqpoint{3.412147in}{1.840196in}}%
\pgfpathlineto{\pgfqpoint{3.416405in}{1.840196in}}%
\pgfpathlineto{\pgfqpoint{3.416405in}{1.835938in}}%
\pgfpathmoveto{\pgfqpoint{3.412147in}{1.840196in}}%
\pgfpathlineto{\pgfqpoint{3.412147in}{1.840196in}}%
\pgfpathlineto{\pgfqpoint{3.412147in}{1.844454in}}%
\pgfpathlineto{\pgfqpoint{3.416405in}{1.844454in}}%
\pgfpathlineto{\pgfqpoint{3.416405in}{1.840196in}}%
\pgfpathmoveto{\pgfqpoint{3.407889in}{1.844454in}}%
\pgfpathlineto{\pgfqpoint{3.407889in}{1.844454in}}%
\pgfpathlineto{\pgfqpoint{3.407889in}{1.848711in}}%
\pgfpathlineto{\pgfqpoint{3.412147in}{1.848711in}}%
\pgfpathlineto{\pgfqpoint{3.412147in}{1.844454in}}%
\pgfpathmoveto{\pgfqpoint{3.407889in}{1.848711in}}%
\pgfpathlineto{\pgfqpoint{3.407889in}{1.848711in}}%
\pgfpathlineto{\pgfqpoint{3.407889in}{1.852969in}}%
\pgfpathlineto{\pgfqpoint{3.412147in}{1.852969in}}%
\pgfpathlineto{\pgfqpoint{3.412147in}{1.848711in}}%
\pgfpathmoveto{\pgfqpoint{3.412147in}{1.844454in}}%
\pgfpathlineto{\pgfqpoint{3.412147in}{1.844454in}}%
\pgfpathlineto{\pgfqpoint{3.412147in}{1.848711in}}%
\pgfpathlineto{\pgfqpoint{3.416405in}{1.848711in}}%
\pgfpathlineto{\pgfqpoint{3.416405in}{1.844454in}}%
\pgfpathmoveto{\pgfqpoint{3.407889in}{1.852969in}}%
\pgfpathlineto{\pgfqpoint{3.407889in}{1.852969in}}%
\pgfpathlineto{\pgfqpoint{3.407889in}{1.857227in}}%
\pgfpathlineto{\pgfqpoint{3.412147in}{1.857227in}}%
\pgfpathlineto{\pgfqpoint{3.412147in}{1.852969in}}%
\pgfpathmoveto{\pgfqpoint{3.407889in}{1.857227in}}%
\pgfpathlineto{\pgfqpoint{3.407889in}{1.857227in}}%
\pgfpathlineto{\pgfqpoint{3.407889in}{1.861485in}}%
\pgfpathlineto{\pgfqpoint{3.412147in}{1.861485in}}%
\pgfpathlineto{\pgfqpoint{3.412147in}{1.857227in}}%
\pgfpathmoveto{\pgfqpoint{3.407889in}{1.861485in}}%
\pgfpathlineto{\pgfqpoint{3.407889in}{1.861485in}}%
\pgfpathlineto{\pgfqpoint{3.407889in}{1.865742in}}%
\pgfpathlineto{\pgfqpoint{3.412147in}{1.865742in}}%
\pgfpathlineto{\pgfqpoint{3.412147in}{1.861485in}}%
\pgfpathmoveto{\pgfqpoint{3.407889in}{1.865742in}}%
\pgfpathlineto{\pgfqpoint{3.407889in}{1.865742in}}%
\pgfpathlineto{\pgfqpoint{3.407889in}{1.870000in}}%
\pgfpathlineto{\pgfqpoint{3.412147in}{1.870000in}}%
\pgfpathlineto{\pgfqpoint{3.412147in}{1.865742in}}%
\pgfpathmoveto{\pgfqpoint{3.403631in}{1.878516in}}%
\pgfpathlineto{\pgfqpoint{3.403631in}{1.878516in}}%
\pgfpathlineto{\pgfqpoint{3.403631in}{1.882774in}}%
\pgfpathlineto{\pgfqpoint{3.407889in}{1.882774in}}%
\pgfpathlineto{\pgfqpoint{3.407889in}{1.878516in}}%
\pgfpathmoveto{\pgfqpoint{3.403631in}{1.882774in}}%
\pgfpathlineto{\pgfqpoint{3.403631in}{1.882774in}}%
\pgfpathlineto{\pgfqpoint{3.403631in}{1.887032in}}%
\pgfpathlineto{\pgfqpoint{3.407889in}{1.887032in}}%
\pgfpathlineto{\pgfqpoint{3.407889in}{1.882774in}}%
\pgfpathmoveto{\pgfqpoint{3.407889in}{1.870000in}}%
\pgfpathlineto{\pgfqpoint{3.407889in}{1.870000in}}%
\pgfpathlineto{\pgfqpoint{3.407889in}{1.874258in}}%
\pgfpathlineto{\pgfqpoint{3.412147in}{1.874258in}}%
\pgfpathlineto{\pgfqpoint{3.412147in}{1.870000in}}%
\pgfpathmoveto{\pgfqpoint{3.407889in}{1.874258in}}%
\pgfpathlineto{\pgfqpoint{3.407889in}{1.874258in}}%
\pgfpathlineto{\pgfqpoint{3.407889in}{1.878516in}}%
\pgfpathlineto{\pgfqpoint{3.412147in}{1.878516in}}%
\pgfpathlineto{\pgfqpoint{3.412147in}{1.874258in}}%
\pgfpathmoveto{\pgfqpoint{3.407889in}{1.878516in}}%
\pgfpathlineto{\pgfqpoint{3.407889in}{1.878516in}}%
\pgfpathlineto{\pgfqpoint{3.407889in}{1.882774in}}%
\pgfpathlineto{\pgfqpoint{3.412147in}{1.882774in}}%
\pgfpathlineto{\pgfqpoint{3.412147in}{1.878516in}}%
\pgfpathmoveto{\pgfqpoint{3.403631in}{1.887032in}}%
\pgfpathlineto{\pgfqpoint{3.403631in}{1.887032in}}%
\pgfpathlineto{\pgfqpoint{3.403631in}{1.891289in}}%
\pgfpathlineto{\pgfqpoint{3.407889in}{1.891289in}}%
\pgfpathlineto{\pgfqpoint{3.407889in}{1.887032in}}%
\pgfpathmoveto{\pgfqpoint{3.403631in}{1.891289in}}%
\pgfpathlineto{\pgfqpoint{3.403631in}{1.891289in}}%
\pgfpathlineto{\pgfqpoint{3.403631in}{1.895547in}}%
\pgfpathlineto{\pgfqpoint{3.407889in}{1.895547in}}%
\pgfpathlineto{\pgfqpoint{3.407889in}{1.891289in}}%
\pgfpathmoveto{\pgfqpoint{3.403631in}{1.895547in}}%
\pgfpathlineto{\pgfqpoint{3.403631in}{1.895547in}}%
\pgfpathlineto{\pgfqpoint{3.403631in}{1.899805in}}%
\pgfpathlineto{\pgfqpoint{3.407889in}{1.899805in}}%
\pgfpathlineto{\pgfqpoint{3.407889in}{1.895547in}}%
\pgfpathmoveto{\pgfqpoint{3.403631in}{1.899805in}}%
\pgfpathlineto{\pgfqpoint{3.403631in}{1.899805in}}%
\pgfpathlineto{\pgfqpoint{3.403631in}{1.904063in}}%
\pgfpathlineto{\pgfqpoint{3.407889in}{1.904063in}}%
\pgfpathlineto{\pgfqpoint{3.407889in}{1.899805in}}%
\pgfpathmoveto{\pgfqpoint{3.403631in}{1.904063in}}%
\pgfpathlineto{\pgfqpoint{3.403631in}{1.904063in}}%
\pgfpathlineto{\pgfqpoint{3.403631in}{1.908321in}}%
\pgfpathlineto{\pgfqpoint{3.407889in}{1.908321in}}%
\pgfpathlineto{\pgfqpoint{3.407889in}{1.904063in}}%
\pgfpathmoveto{\pgfqpoint{3.403631in}{1.908321in}}%
\pgfpathlineto{\pgfqpoint{3.403631in}{1.908321in}}%
\pgfpathlineto{\pgfqpoint{3.403631in}{1.912579in}}%
\pgfpathlineto{\pgfqpoint{3.407889in}{1.912579in}}%
\pgfpathlineto{\pgfqpoint{3.407889in}{1.908321in}}%
\pgfpathmoveto{\pgfqpoint{3.399373in}{1.912579in}}%
\pgfpathlineto{\pgfqpoint{3.399373in}{1.912579in}}%
\pgfpathlineto{\pgfqpoint{3.399373in}{1.916836in}}%
\pgfpathlineto{\pgfqpoint{3.403631in}{1.916836in}}%
\pgfpathlineto{\pgfqpoint{3.403631in}{1.912579in}}%
\pgfpathmoveto{\pgfqpoint{3.399373in}{1.916836in}}%
\pgfpathlineto{\pgfqpoint{3.399373in}{1.916836in}}%
\pgfpathlineto{\pgfqpoint{3.399373in}{1.921094in}}%
\pgfpathlineto{\pgfqpoint{3.403631in}{1.921094in}}%
\pgfpathlineto{\pgfqpoint{3.403631in}{1.916836in}}%
\pgfpathmoveto{\pgfqpoint{3.403631in}{1.912579in}}%
\pgfpathlineto{\pgfqpoint{3.403631in}{1.912579in}}%
\pgfpathlineto{\pgfqpoint{3.403631in}{1.916836in}}%
\pgfpathlineto{\pgfqpoint{3.407889in}{1.916836in}}%
\pgfpathlineto{\pgfqpoint{3.407889in}{1.912579in}}%
\pgfpathmoveto{\pgfqpoint{3.399373in}{1.921094in}}%
\pgfpathlineto{\pgfqpoint{3.399373in}{1.921094in}}%
\pgfpathlineto{\pgfqpoint{3.399373in}{1.925352in}}%
\pgfpathlineto{\pgfqpoint{3.403631in}{1.925352in}}%
\pgfpathlineto{\pgfqpoint{3.403631in}{1.921094in}}%
\pgfpathmoveto{\pgfqpoint{3.399373in}{1.925352in}}%
\pgfpathlineto{\pgfqpoint{3.399373in}{1.925352in}}%
\pgfpathlineto{\pgfqpoint{3.399373in}{1.929610in}}%
\pgfpathlineto{\pgfqpoint{3.403631in}{1.929610in}}%
\pgfpathlineto{\pgfqpoint{3.403631in}{1.925352in}}%
\pgfpathmoveto{\pgfqpoint{3.399373in}{1.929610in}}%
\pgfpathlineto{\pgfqpoint{3.399373in}{1.929610in}}%
\pgfpathlineto{\pgfqpoint{3.399373in}{1.933868in}}%
\pgfpathlineto{\pgfqpoint{3.403631in}{1.933868in}}%
\pgfpathlineto{\pgfqpoint{3.403631in}{1.929610in}}%
\pgfpathmoveto{\pgfqpoint{3.399373in}{1.933868in}}%
\pgfpathlineto{\pgfqpoint{3.399373in}{1.933868in}}%
\pgfpathlineto{\pgfqpoint{3.399373in}{1.938126in}}%
\pgfpathlineto{\pgfqpoint{3.403631in}{1.938126in}}%
\pgfpathlineto{\pgfqpoint{3.403631in}{1.933868in}}%
\pgfpathmoveto{\pgfqpoint{3.395115in}{1.950899in}}%
\pgfpathlineto{\pgfqpoint{3.395115in}{1.950899in}}%
\pgfpathlineto{\pgfqpoint{3.395115in}{1.955157in}}%
\pgfpathlineto{\pgfqpoint{3.399373in}{1.955157in}}%
\pgfpathlineto{\pgfqpoint{3.399373in}{1.950899in}}%
\pgfpathmoveto{\pgfqpoint{3.395115in}{1.955157in}}%
\pgfpathlineto{\pgfqpoint{3.395115in}{1.955157in}}%
\pgfpathlineto{\pgfqpoint{3.395115in}{1.959415in}}%
\pgfpathlineto{\pgfqpoint{3.399373in}{1.959415in}}%
\pgfpathlineto{\pgfqpoint{3.399373in}{1.955157in}}%
\pgfpathmoveto{\pgfqpoint{3.395115in}{1.959415in}}%
\pgfpathlineto{\pgfqpoint{3.395115in}{1.959415in}}%
\pgfpathlineto{\pgfqpoint{3.395115in}{1.963673in}}%
\pgfpathlineto{\pgfqpoint{3.399373in}{1.963673in}}%
\pgfpathlineto{\pgfqpoint{3.399373in}{1.959415in}}%
\pgfpathmoveto{\pgfqpoint{3.395115in}{1.963673in}}%
\pgfpathlineto{\pgfqpoint{3.395115in}{1.963673in}}%
\pgfpathlineto{\pgfqpoint{3.395115in}{1.967931in}}%
\pgfpathlineto{\pgfqpoint{3.399373in}{1.967931in}}%
\pgfpathlineto{\pgfqpoint{3.399373in}{1.963673in}}%
\pgfpathmoveto{\pgfqpoint{3.395115in}{1.967931in}}%
\pgfpathlineto{\pgfqpoint{3.395115in}{1.967931in}}%
\pgfpathlineto{\pgfqpoint{3.395115in}{1.972188in}}%
\pgfpathlineto{\pgfqpoint{3.399373in}{1.972188in}}%
\pgfpathlineto{\pgfqpoint{3.399373in}{1.967931in}}%
\pgfpathmoveto{\pgfqpoint{3.395115in}{1.972188in}}%
\pgfpathlineto{\pgfqpoint{3.395115in}{1.972188in}}%
\pgfpathlineto{\pgfqpoint{3.395115in}{1.976446in}}%
\pgfpathlineto{\pgfqpoint{3.399373in}{1.976446in}}%
\pgfpathlineto{\pgfqpoint{3.399373in}{1.972188in}}%
\pgfpathmoveto{\pgfqpoint{3.395115in}{1.976446in}}%
\pgfpathlineto{\pgfqpoint{3.395115in}{1.976446in}}%
\pgfpathlineto{\pgfqpoint{3.395115in}{1.980704in}}%
\pgfpathlineto{\pgfqpoint{3.399373in}{1.980704in}}%
\pgfpathlineto{\pgfqpoint{3.399373in}{1.976446in}}%
\pgfpathmoveto{\pgfqpoint{3.390857in}{1.984962in}}%
\pgfpathlineto{\pgfqpoint{3.390857in}{1.984962in}}%
\pgfpathlineto{\pgfqpoint{3.390857in}{1.989220in}}%
\pgfpathlineto{\pgfqpoint{3.395115in}{1.989220in}}%
\pgfpathlineto{\pgfqpoint{3.395115in}{1.984962in}}%
\pgfpathmoveto{\pgfqpoint{3.395115in}{1.980704in}}%
\pgfpathlineto{\pgfqpoint{3.395115in}{1.980704in}}%
\pgfpathlineto{\pgfqpoint{3.395115in}{1.984962in}}%
\pgfpathlineto{\pgfqpoint{3.399373in}{1.984962in}}%
\pgfpathlineto{\pgfqpoint{3.399373in}{1.980704in}}%
\pgfpathmoveto{\pgfqpoint{3.395115in}{1.984962in}}%
\pgfpathlineto{\pgfqpoint{3.395115in}{1.984962in}}%
\pgfpathlineto{\pgfqpoint{3.395115in}{1.989220in}}%
\pgfpathlineto{\pgfqpoint{3.399373in}{1.989220in}}%
\pgfpathlineto{\pgfqpoint{3.399373in}{1.984962in}}%
\pgfpathmoveto{\pgfqpoint{3.390857in}{1.989220in}}%
\pgfpathlineto{\pgfqpoint{3.390857in}{1.989220in}}%
\pgfpathlineto{\pgfqpoint{3.390857in}{1.993478in}}%
\pgfpathlineto{\pgfqpoint{3.395115in}{1.993478in}}%
\pgfpathlineto{\pgfqpoint{3.395115in}{1.989220in}}%
\pgfpathmoveto{\pgfqpoint{3.390857in}{1.993478in}}%
\pgfpathlineto{\pgfqpoint{3.390857in}{1.993478in}}%
\pgfpathlineto{\pgfqpoint{3.390857in}{1.997735in}}%
\pgfpathlineto{\pgfqpoint{3.395115in}{1.997735in}}%
\pgfpathlineto{\pgfqpoint{3.395115in}{1.993478in}}%
\pgfpathmoveto{\pgfqpoint{3.390857in}{1.997735in}}%
\pgfpathlineto{\pgfqpoint{3.390857in}{1.997735in}}%
\pgfpathlineto{\pgfqpoint{3.390857in}{2.001993in}}%
\pgfpathlineto{\pgfqpoint{3.395115in}{2.001993in}}%
\pgfpathlineto{\pgfqpoint{3.395115in}{1.997735in}}%
\pgfpathmoveto{\pgfqpoint{3.390857in}{2.001993in}}%
\pgfpathlineto{\pgfqpoint{3.390857in}{2.001993in}}%
\pgfpathlineto{\pgfqpoint{3.390857in}{2.006251in}}%
\pgfpathlineto{\pgfqpoint{3.395115in}{2.006251in}}%
\pgfpathlineto{\pgfqpoint{3.395115in}{2.001993in}}%
\pgfpathmoveto{\pgfqpoint{3.399373in}{1.938126in}}%
\pgfpathlineto{\pgfqpoint{3.399373in}{1.938126in}}%
\pgfpathlineto{\pgfqpoint{3.399373in}{1.942384in}}%
\pgfpathlineto{\pgfqpoint{3.403631in}{1.942384in}}%
\pgfpathlineto{\pgfqpoint{3.403631in}{1.938126in}}%
\pgfpathmoveto{\pgfqpoint{3.399373in}{1.942384in}}%
\pgfpathlineto{\pgfqpoint{3.399373in}{1.942384in}}%
\pgfpathlineto{\pgfqpoint{3.399373in}{1.946641in}}%
\pgfpathlineto{\pgfqpoint{3.403631in}{1.946641in}}%
\pgfpathlineto{\pgfqpoint{3.403631in}{1.942384in}}%
\pgfpathmoveto{\pgfqpoint{3.399373in}{1.946641in}}%
\pgfpathlineto{\pgfqpoint{3.399373in}{1.946641in}}%
\pgfpathlineto{\pgfqpoint{3.399373in}{1.950899in}}%
\pgfpathlineto{\pgfqpoint{3.403631in}{1.950899in}}%
\pgfpathlineto{\pgfqpoint{3.403631in}{1.946641in}}%
\pgfpathmoveto{\pgfqpoint{3.399373in}{1.950899in}}%
\pgfpathlineto{\pgfqpoint{3.399373in}{1.950899in}}%
\pgfpathlineto{\pgfqpoint{3.399373in}{1.955157in}}%
\pgfpathlineto{\pgfqpoint{3.403631in}{1.955157in}}%
\pgfpathlineto{\pgfqpoint{3.403631in}{1.950899in}}%
\pgfpathmoveto{\pgfqpoint{3.386599in}{2.019024in}}%
\pgfpathlineto{\pgfqpoint{3.386599in}{2.019024in}}%
\pgfpathlineto{\pgfqpoint{3.386599in}{2.023282in}}%
\pgfpathlineto{\pgfqpoint{3.390857in}{2.023282in}}%
\pgfpathlineto{\pgfqpoint{3.390857in}{2.019024in}}%
\pgfpathmoveto{\pgfqpoint{3.390857in}{2.006251in}}%
\pgfpathlineto{\pgfqpoint{3.390857in}{2.006251in}}%
\pgfpathlineto{\pgfqpoint{3.390857in}{2.010509in}}%
\pgfpathlineto{\pgfqpoint{3.395115in}{2.010509in}}%
\pgfpathlineto{\pgfqpoint{3.395115in}{2.006251in}}%
\pgfpathmoveto{\pgfqpoint{3.390857in}{2.010509in}}%
\pgfpathlineto{\pgfqpoint{3.390857in}{2.010509in}}%
\pgfpathlineto{\pgfqpoint{3.390857in}{2.014767in}}%
\pgfpathlineto{\pgfqpoint{3.395115in}{2.014767in}}%
\pgfpathlineto{\pgfqpoint{3.395115in}{2.010509in}}%
\pgfpathmoveto{\pgfqpoint{3.390857in}{2.014767in}}%
\pgfpathlineto{\pgfqpoint{3.390857in}{2.014767in}}%
\pgfpathlineto{\pgfqpoint{3.390857in}{2.019024in}}%
\pgfpathlineto{\pgfqpoint{3.395115in}{2.019024in}}%
\pgfpathlineto{\pgfqpoint{3.395115in}{2.014767in}}%
\pgfpathmoveto{\pgfqpoint{3.390857in}{2.019024in}}%
\pgfpathlineto{\pgfqpoint{3.390857in}{2.019024in}}%
\pgfpathlineto{\pgfqpoint{3.390857in}{2.023282in}}%
\pgfpathlineto{\pgfqpoint{3.395115in}{2.023282in}}%
\pgfpathlineto{\pgfqpoint{3.395115in}{2.019024in}}%
\pgfpathmoveto{\pgfqpoint{3.386599in}{2.023282in}}%
\pgfpathlineto{\pgfqpoint{3.386599in}{2.023282in}}%
\pgfpathlineto{\pgfqpoint{3.386599in}{2.027540in}}%
\pgfpathlineto{\pgfqpoint{3.390857in}{2.027540in}}%
\pgfpathlineto{\pgfqpoint{3.390857in}{2.023282in}}%
\pgfpathmoveto{\pgfqpoint{3.386599in}{2.027540in}}%
\pgfpathlineto{\pgfqpoint{3.386599in}{2.027540in}}%
\pgfpathlineto{\pgfqpoint{3.386599in}{2.031798in}}%
\pgfpathlineto{\pgfqpoint{3.390857in}{2.031798in}}%
\pgfpathlineto{\pgfqpoint{3.390857in}{2.027540in}}%
\pgfpathmoveto{\pgfqpoint{3.386599in}{2.031798in}}%
\pgfpathlineto{\pgfqpoint{3.386599in}{2.031798in}}%
\pgfpathlineto{\pgfqpoint{3.386599in}{2.036055in}}%
\pgfpathlineto{\pgfqpoint{3.390857in}{2.036055in}}%
\pgfpathlineto{\pgfqpoint{3.390857in}{2.031798in}}%
\pgfpathmoveto{\pgfqpoint{3.386599in}{2.036055in}}%
\pgfpathlineto{\pgfqpoint{3.386599in}{2.036055in}}%
\pgfpathlineto{\pgfqpoint{3.386599in}{2.040313in}}%
\pgfpathlineto{\pgfqpoint{3.390857in}{2.040313in}}%
\pgfpathlineto{\pgfqpoint{3.390857in}{2.036055in}}%
\pgfpathmoveto{\pgfqpoint{3.386599in}{2.040313in}}%
\pgfpathlineto{\pgfqpoint{3.386599in}{2.040313in}}%
\pgfpathlineto{\pgfqpoint{3.386599in}{2.044571in}}%
\pgfpathlineto{\pgfqpoint{3.390857in}{2.044571in}}%
\pgfpathlineto{\pgfqpoint{3.390857in}{2.040313in}}%
\pgfpathmoveto{\pgfqpoint{3.386599in}{2.044571in}}%
\pgfpathlineto{\pgfqpoint{3.386599in}{2.044571in}}%
\pgfpathlineto{\pgfqpoint{3.386599in}{2.048829in}}%
\pgfpathlineto{\pgfqpoint{3.390857in}{2.048829in}}%
\pgfpathlineto{\pgfqpoint{3.390857in}{2.044571in}}%
\pgfpathmoveto{\pgfqpoint{3.382341in}{2.053086in}}%
\pgfpathlineto{\pgfqpoint{3.382341in}{2.053086in}}%
\pgfpathlineto{\pgfqpoint{3.382341in}{2.057344in}}%
\pgfpathlineto{\pgfqpoint{3.386599in}{2.057344in}}%
\pgfpathlineto{\pgfqpoint{3.386599in}{2.053086in}}%
\pgfpathmoveto{\pgfqpoint{3.386599in}{2.048829in}}%
\pgfpathlineto{\pgfqpoint{3.386599in}{2.048829in}}%
\pgfpathlineto{\pgfqpoint{3.386599in}{2.053086in}}%
\pgfpathlineto{\pgfqpoint{3.390857in}{2.053086in}}%
\pgfpathlineto{\pgfqpoint{3.390857in}{2.048829in}}%
\pgfpathmoveto{\pgfqpoint{3.386599in}{2.053086in}}%
\pgfpathlineto{\pgfqpoint{3.386599in}{2.053086in}}%
\pgfpathlineto{\pgfqpoint{3.386599in}{2.057344in}}%
\pgfpathlineto{\pgfqpoint{3.390857in}{2.057344in}}%
\pgfpathlineto{\pgfqpoint{3.390857in}{2.053086in}}%
\pgfpathmoveto{\pgfqpoint{3.382341in}{2.057344in}}%
\pgfpathlineto{\pgfqpoint{3.382341in}{2.057344in}}%
\pgfpathlineto{\pgfqpoint{3.382341in}{2.061602in}}%
\pgfpathlineto{\pgfqpoint{3.386599in}{2.061602in}}%
\pgfpathlineto{\pgfqpoint{3.386599in}{2.057344in}}%
\pgfpathmoveto{\pgfqpoint{3.382341in}{2.061602in}}%
\pgfpathlineto{\pgfqpoint{3.382341in}{2.061602in}}%
\pgfpathlineto{\pgfqpoint{3.382341in}{2.065860in}}%
\pgfpathlineto{\pgfqpoint{3.386599in}{2.065860in}}%
\pgfpathlineto{\pgfqpoint{3.386599in}{2.061602in}}%
\pgfpathmoveto{\pgfqpoint{3.382341in}{2.065860in}}%
\pgfpathlineto{\pgfqpoint{3.382341in}{2.065860in}}%
\pgfpathlineto{\pgfqpoint{3.382341in}{2.070117in}}%
\pgfpathlineto{\pgfqpoint{3.386599in}{2.070117in}}%
\pgfpathlineto{\pgfqpoint{3.386599in}{2.065860in}}%
\pgfpathmoveto{\pgfqpoint{3.382341in}{2.070117in}}%
\pgfpathlineto{\pgfqpoint{3.382341in}{2.070117in}}%
\pgfpathlineto{\pgfqpoint{3.382341in}{2.074375in}}%
\pgfpathlineto{\pgfqpoint{3.386599in}{2.074375in}}%
\pgfpathlineto{\pgfqpoint{3.386599in}{2.070117in}}%
\pgfpathmoveto{\pgfqpoint{3.378083in}{2.082891in}}%
\pgfpathlineto{\pgfqpoint{3.378083in}{2.082891in}}%
\pgfpathlineto{\pgfqpoint{3.378083in}{2.087148in}}%
\pgfpathlineto{\pgfqpoint{3.382341in}{2.087148in}}%
\pgfpathlineto{\pgfqpoint{3.382341in}{2.082891in}}%
\pgfpathmoveto{\pgfqpoint{3.378083in}{2.087148in}}%
\pgfpathlineto{\pgfqpoint{3.378083in}{2.087148in}}%
\pgfpathlineto{\pgfqpoint{3.378083in}{2.091406in}}%
\pgfpathlineto{\pgfqpoint{3.382341in}{2.091406in}}%
\pgfpathlineto{\pgfqpoint{3.382341in}{2.087148in}}%
\pgfpathmoveto{\pgfqpoint{3.378083in}{2.091406in}}%
\pgfpathlineto{\pgfqpoint{3.378083in}{2.091406in}}%
\pgfpathlineto{\pgfqpoint{3.378083in}{2.095664in}}%
\pgfpathlineto{\pgfqpoint{3.382341in}{2.095664in}}%
\pgfpathlineto{\pgfqpoint{3.382341in}{2.091406in}}%
\pgfpathmoveto{\pgfqpoint{3.378083in}{2.095664in}}%
\pgfpathlineto{\pgfqpoint{3.378083in}{2.095664in}}%
\pgfpathlineto{\pgfqpoint{3.378083in}{2.099922in}}%
\pgfpathlineto{\pgfqpoint{3.382341in}{2.099922in}}%
\pgfpathlineto{\pgfqpoint{3.382341in}{2.095664in}}%
\pgfpathmoveto{\pgfqpoint{3.378083in}{2.099922in}}%
\pgfpathlineto{\pgfqpoint{3.378083in}{2.099922in}}%
\pgfpathlineto{\pgfqpoint{3.378083in}{2.104179in}}%
\pgfpathlineto{\pgfqpoint{3.382341in}{2.104179in}}%
\pgfpathlineto{\pgfqpoint{3.382341in}{2.099922in}}%
\pgfpathmoveto{\pgfqpoint{3.378083in}{2.104179in}}%
\pgfpathlineto{\pgfqpoint{3.378083in}{2.104179in}}%
\pgfpathlineto{\pgfqpoint{3.378083in}{2.108437in}}%
\pgfpathlineto{\pgfqpoint{3.382341in}{2.108437in}}%
\pgfpathlineto{\pgfqpoint{3.382341in}{2.104179in}}%
\pgfpathmoveto{\pgfqpoint{3.382341in}{2.074375in}}%
\pgfpathlineto{\pgfqpoint{3.382341in}{2.074375in}}%
\pgfpathlineto{\pgfqpoint{3.382341in}{2.078633in}}%
\pgfpathlineto{\pgfqpoint{3.386599in}{2.078633in}}%
\pgfpathlineto{\pgfqpoint{3.386599in}{2.074375in}}%
\pgfpathmoveto{\pgfqpoint{3.382341in}{2.078633in}}%
\pgfpathlineto{\pgfqpoint{3.382341in}{2.078633in}}%
\pgfpathlineto{\pgfqpoint{3.382341in}{2.082891in}}%
\pgfpathlineto{\pgfqpoint{3.386599in}{2.082891in}}%
\pgfpathlineto{\pgfqpoint{3.386599in}{2.078633in}}%
\pgfpathmoveto{\pgfqpoint{3.382341in}{2.082891in}}%
\pgfpathlineto{\pgfqpoint{3.382341in}{2.082891in}}%
\pgfpathlineto{\pgfqpoint{3.382341in}{2.087148in}}%
\pgfpathlineto{\pgfqpoint{3.386599in}{2.087148in}}%
\pgfpathlineto{\pgfqpoint{3.386599in}{2.082891in}}%
\pgfpathmoveto{\pgfqpoint{3.378083in}{2.108437in}}%
\pgfpathlineto{\pgfqpoint{3.378083in}{2.108437in}}%
\pgfpathlineto{\pgfqpoint{3.378083in}{2.112695in}}%
\pgfpathlineto{\pgfqpoint{3.382341in}{2.112695in}}%
\pgfpathlineto{\pgfqpoint{3.382341in}{2.108437in}}%
\pgfpathmoveto{\pgfqpoint{3.378083in}{2.112695in}}%
\pgfpathlineto{\pgfqpoint{3.378083in}{2.112695in}}%
\pgfpathlineto{\pgfqpoint{3.378083in}{2.116953in}}%
\pgfpathlineto{\pgfqpoint{3.382341in}{2.116953in}}%
\pgfpathlineto{\pgfqpoint{3.382341in}{2.112695in}}%
\pgfpathmoveto{\pgfqpoint{3.373825in}{2.116953in}}%
\pgfpathlineto{\pgfqpoint{3.373825in}{2.116953in}}%
\pgfpathlineto{\pgfqpoint{3.373825in}{2.121210in}}%
\pgfpathlineto{\pgfqpoint{3.378083in}{2.121210in}}%
\pgfpathlineto{\pgfqpoint{3.378083in}{2.116953in}}%
\pgfpathmoveto{\pgfqpoint{3.373825in}{2.121210in}}%
\pgfpathlineto{\pgfqpoint{3.373825in}{2.121210in}}%
\pgfpathlineto{\pgfqpoint{3.373825in}{2.125468in}}%
\pgfpathlineto{\pgfqpoint{3.378083in}{2.125468in}}%
\pgfpathlineto{\pgfqpoint{3.378083in}{2.121210in}}%
\pgfpathmoveto{\pgfqpoint{3.378083in}{2.116953in}}%
\pgfpathlineto{\pgfqpoint{3.378083in}{2.116953in}}%
\pgfpathlineto{\pgfqpoint{3.378083in}{2.121210in}}%
\pgfpathlineto{\pgfqpoint{3.382341in}{2.121210in}}%
\pgfpathlineto{\pgfqpoint{3.382341in}{2.116953in}}%
\pgfpathmoveto{\pgfqpoint{3.373825in}{2.125468in}}%
\pgfpathlineto{\pgfqpoint{3.373825in}{2.125468in}}%
\pgfpathlineto{\pgfqpoint{3.373825in}{2.129726in}}%
\pgfpathlineto{\pgfqpoint{3.378083in}{2.129726in}}%
\pgfpathlineto{\pgfqpoint{3.378083in}{2.125468in}}%
\pgfpathmoveto{\pgfqpoint{3.373825in}{2.129726in}}%
\pgfpathlineto{\pgfqpoint{3.373825in}{2.129726in}}%
\pgfpathlineto{\pgfqpoint{3.373825in}{2.133984in}}%
\pgfpathlineto{\pgfqpoint{3.378083in}{2.133984in}}%
\pgfpathlineto{\pgfqpoint{3.378083in}{2.129726in}}%
\pgfpathmoveto{\pgfqpoint{3.373825in}{2.133984in}}%
\pgfpathlineto{\pgfqpoint{3.373825in}{2.133984in}}%
\pgfpathlineto{\pgfqpoint{3.373825in}{2.138241in}}%
\pgfpathlineto{\pgfqpoint{3.378083in}{2.138241in}}%
\pgfpathlineto{\pgfqpoint{3.378083in}{2.133984in}}%
\pgfpathmoveto{\pgfqpoint{3.373825in}{2.138241in}}%
\pgfpathlineto{\pgfqpoint{3.373825in}{2.138241in}}%
\pgfpathlineto{\pgfqpoint{3.373825in}{2.142499in}}%
\pgfpathlineto{\pgfqpoint{3.378083in}{2.142499in}}%
\pgfpathlineto{\pgfqpoint{3.378083in}{2.138241in}}%
\pgfpathmoveto{\pgfqpoint{3.369567in}{2.151015in}}%
\pgfpathlineto{\pgfqpoint{3.369567in}{2.151015in}}%
\pgfpathlineto{\pgfqpoint{3.369567in}{2.155273in}}%
\pgfpathlineto{\pgfqpoint{3.373825in}{2.155273in}}%
\pgfpathlineto{\pgfqpoint{3.373825in}{2.151015in}}%
\pgfpathmoveto{\pgfqpoint{3.369567in}{2.155273in}}%
\pgfpathlineto{\pgfqpoint{3.369567in}{2.155273in}}%
\pgfpathlineto{\pgfqpoint{3.369567in}{2.159531in}}%
\pgfpathlineto{\pgfqpoint{3.373825in}{2.159531in}}%
\pgfpathlineto{\pgfqpoint{3.373825in}{2.155273in}}%
\pgfpathmoveto{\pgfqpoint{3.373825in}{2.142499in}}%
\pgfpathlineto{\pgfqpoint{3.373825in}{2.142499in}}%
\pgfpathlineto{\pgfqpoint{3.373825in}{2.146757in}}%
\pgfpathlineto{\pgfqpoint{3.378083in}{2.146757in}}%
\pgfpathlineto{\pgfqpoint{3.378083in}{2.142499in}}%
\pgfpathmoveto{\pgfqpoint{3.373825in}{2.146757in}}%
\pgfpathlineto{\pgfqpoint{3.373825in}{2.146757in}}%
\pgfpathlineto{\pgfqpoint{3.373825in}{2.151015in}}%
\pgfpathlineto{\pgfqpoint{3.378083in}{2.151015in}}%
\pgfpathlineto{\pgfqpoint{3.378083in}{2.146757in}}%
\pgfpathmoveto{\pgfqpoint{3.373825in}{2.151015in}}%
\pgfpathlineto{\pgfqpoint{3.373825in}{2.151015in}}%
\pgfpathlineto{\pgfqpoint{3.373825in}{2.155273in}}%
\pgfpathlineto{\pgfqpoint{3.378083in}{2.155273in}}%
\pgfpathlineto{\pgfqpoint{3.378083in}{2.151015in}}%
\pgfpathmoveto{\pgfqpoint{3.369567in}{2.159531in}}%
\pgfpathlineto{\pgfqpoint{3.369567in}{2.159531in}}%
\pgfpathlineto{\pgfqpoint{3.369567in}{2.163789in}}%
\pgfpathlineto{\pgfqpoint{3.373825in}{2.163789in}}%
\pgfpathlineto{\pgfqpoint{3.373825in}{2.159531in}}%
\pgfpathmoveto{\pgfqpoint{3.369567in}{2.163789in}}%
\pgfpathlineto{\pgfqpoint{3.369567in}{2.163789in}}%
\pgfpathlineto{\pgfqpoint{3.369567in}{2.168047in}}%
\pgfpathlineto{\pgfqpoint{3.373825in}{2.168047in}}%
\pgfpathlineto{\pgfqpoint{3.373825in}{2.163789in}}%
\pgfpathmoveto{\pgfqpoint{3.369567in}{2.168047in}}%
\pgfpathlineto{\pgfqpoint{3.369567in}{2.168047in}}%
\pgfpathlineto{\pgfqpoint{3.369567in}{2.172305in}}%
\pgfpathlineto{\pgfqpoint{3.373825in}{2.172305in}}%
\pgfpathlineto{\pgfqpoint{3.373825in}{2.168047in}}%
\pgfpathmoveto{\pgfqpoint{3.369567in}{2.172305in}}%
\pgfpathlineto{\pgfqpoint{3.369567in}{2.172305in}}%
\pgfpathlineto{\pgfqpoint{3.369567in}{2.176563in}}%
\pgfpathlineto{\pgfqpoint{3.373825in}{2.176563in}}%
\pgfpathlineto{\pgfqpoint{3.373825in}{2.172305in}}%
\pgfpathmoveto{\pgfqpoint{3.369567in}{2.176563in}}%
\pgfpathlineto{\pgfqpoint{3.369567in}{2.176563in}}%
\pgfpathlineto{\pgfqpoint{3.369567in}{2.180821in}}%
\pgfpathlineto{\pgfqpoint{3.373825in}{2.180821in}}%
\pgfpathlineto{\pgfqpoint{3.373825in}{2.176563in}}%
\pgfpathmoveto{\pgfqpoint{3.369567in}{2.180821in}}%
\pgfpathlineto{\pgfqpoint{3.369567in}{2.180821in}}%
\pgfpathlineto{\pgfqpoint{3.369567in}{2.185079in}}%
\pgfpathlineto{\pgfqpoint{3.373825in}{2.185079in}}%
\pgfpathlineto{\pgfqpoint{3.373825in}{2.180821in}}%
\pgfpathmoveto{\pgfqpoint{3.365309in}{2.185079in}}%
\pgfpathlineto{\pgfqpoint{3.365309in}{2.185079in}}%
\pgfpathlineto{\pgfqpoint{3.365309in}{2.189337in}}%
\pgfpathlineto{\pgfqpoint{3.369567in}{2.189337in}}%
\pgfpathlineto{\pgfqpoint{3.369567in}{2.185079in}}%
\pgfpathmoveto{\pgfqpoint{3.365309in}{2.189337in}}%
\pgfpathlineto{\pgfqpoint{3.365309in}{2.189337in}}%
\pgfpathlineto{\pgfqpoint{3.365309in}{2.193595in}}%
\pgfpathlineto{\pgfqpoint{3.369567in}{2.193595in}}%
\pgfpathlineto{\pgfqpoint{3.369567in}{2.189337in}}%
\pgfpathmoveto{\pgfqpoint{3.369567in}{2.185079in}}%
\pgfpathlineto{\pgfqpoint{3.369567in}{2.185079in}}%
\pgfpathlineto{\pgfqpoint{3.369567in}{2.189337in}}%
\pgfpathlineto{\pgfqpoint{3.373825in}{2.189337in}}%
\pgfpathlineto{\pgfqpoint{3.373825in}{2.185079in}}%
\pgfpathmoveto{\pgfqpoint{3.365309in}{2.193595in}}%
\pgfpathlineto{\pgfqpoint{3.365309in}{2.193595in}}%
\pgfpathlineto{\pgfqpoint{3.365309in}{2.197853in}}%
\pgfpathlineto{\pgfqpoint{3.369567in}{2.197853in}}%
\pgfpathlineto{\pgfqpoint{3.369567in}{2.193595in}}%
\pgfpathmoveto{\pgfqpoint{3.365309in}{2.197853in}}%
\pgfpathlineto{\pgfqpoint{3.365309in}{2.197853in}}%
\pgfpathlineto{\pgfqpoint{3.365309in}{2.202111in}}%
\pgfpathlineto{\pgfqpoint{3.369567in}{2.202111in}}%
\pgfpathlineto{\pgfqpoint{3.369567in}{2.197853in}}%
\pgfpathmoveto{\pgfqpoint{3.365309in}{2.202111in}}%
\pgfpathlineto{\pgfqpoint{3.365309in}{2.202111in}}%
\pgfpathlineto{\pgfqpoint{3.365309in}{2.206369in}}%
\pgfpathlineto{\pgfqpoint{3.369567in}{2.206369in}}%
\pgfpathlineto{\pgfqpoint{3.369567in}{2.202111in}}%
\pgfpathmoveto{\pgfqpoint{3.365309in}{2.206369in}}%
\pgfpathlineto{\pgfqpoint{3.365309in}{2.206369in}}%
\pgfpathlineto{\pgfqpoint{3.365309in}{2.210626in}}%
\pgfpathlineto{\pgfqpoint{3.369567in}{2.210626in}}%
\pgfpathlineto{\pgfqpoint{3.369567in}{2.206369in}}%
\pgfpathmoveto{\pgfqpoint{3.365309in}{2.210626in}}%
\pgfpathlineto{\pgfqpoint{3.365309in}{2.210626in}}%
\pgfpathlineto{\pgfqpoint{3.365309in}{2.214884in}}%
\pgfpathlineto{\pgfqpoint{3.369567in}{2.214884in}}%
\pgfpathlineto{\pgfqpoint{3.369567in}{2.210626in}}%
\pgfpathmoveto{\pgfqpoint{3.365309in}{2.214884in}}%
\pgfpathlineto{\pgfqpoint{3.365309in}{2.214884in}}%
\pgfpathlineto{\pgfqpoint{3.365309in}{2.219142in}}%
\pgfpathlineto{\pgfqpoint{3.369567in}{2.219142in}}%
\pgfpathlineto{\pgfqpoint{3.369567in}{2.214884in}}%
\pgfpathmoveto{\pgfqpoint{3.565433in}{0.298867in}}%
\pgfpathlineto{\pgfqpoint{3.565433in}{0.298867in}}%
\pgfpathlineto{\pgfqpoint{3.565433in}{0.303124in}}%
\pgfpathlineto{\pgfqpoint{3.569691in}{0.303124in}}%
\pgfpathlineto{\pgfqpoint{3.569691in}{0.298867in}}%
\pgfpathmoveto{\pgfqpoint{3.565433in}{0.303124in}}%
\pgfpathlineto{\pgfqpoint{3.565433in}{0.303124in}}%
\pgfpathlineto{\pgfqpoint{3.565433in}{0.307382in}}%
\pgfpathlineto{\pgfqpoint{3.569691in}{0.307382in}}%
\pgfpathlineto{\pgfqpoint{3.569691in}{0.303124in}}%
\pgfpathmoveto{\pgfqpoint{3.565433in}{0.307382in}}%
\pgfpathlineto{\pgfqpoint{3.565433in}{0.307382in}}%
\pgfpathlineto{\pgfqpoint{3.565433in}{0.311639in}}%
\pgfpathlineto{\pgfqpoint{3.569691in}{0.311639in}}%
\pgfpathlineto{\pgfqpoint{3.569691in}{0.307382in}}%
\pgfpathmoveto{\pgfqpoint{3.565433in}{0.311639in}}%
\pgfpathlineto{\pgfqpoint{3.565433in}{0.311639in}}%
\pgfpathlineto{\pgfqpoint{3.565433in}{0.315897in}}%
\pgfpathlineto{\pgfqpoint{3.569691in}{0.315897in}}%
\pgfpathlineto{\pgfqpoint{3.569691in}{0.311639in}}%
\pgfpathmoveto{\pgfqpoint{3.565433in}{0.315897in}}%
\pgfpathlineto{\pgfqpoint{3.565433in}{0.315897in}}%
\pgfpathlineto{\pgfqpoint{3.565433in}{0.320155in}}%
\pgfpathlineto{\pgfqpoint{3.569691in}{0.320155in}}%
\pgfpathlineto{\pgfqpoint{3.569691in}{0.315897in}}%
\pgfpathmoveto{\pgfqpoint{3.565433in}{0.320155in}}%
\pgfpathlineto{\pgfqpoint{3.565433in}{0.320155in}}%
\pgfpathlineto{\pgfqpoint{3.565433in}{0.324412in}}%
\pgfpathlineto{\pgfqpoint{3.569691in}{0.324412in}}%
\pgfpathlineto{\pgfqpoint{3.569691in}{0.320155in}}%
\pgfpathmoveto{\pgfqpoint{3.565433in}{0.324412in}}%
\pgfpathlineto{\pgfqpoint{3.565433in}{0.324412in}}%
\pgfpathlineto{\pgfqpoint{3.565433in}{0.328670in}}%
\pgfpathlineto{\pgfqpoint{3.569691in}{0.328670in}}%
\pgfpathlineto{\pgfqpoint{3.569691in}{0.324412in}}%
\pgfpathmoveto{\pgfqpoint{3.565433in}{0.328670in}}%
\pgfpathlineto{\pgfqpoint{3.565433in}{0.328670in}}%
\pgfpathlineto{\pgfqpoint{3.565433in}{0.332928in}}%
\pgfpathlineto{\pgfqpoint{3.569691in}{0.332928in}}%
\pgfpathlineto{\pgfqpoint{3.569691in}{0.328670in}}%
\pgfpathmoveto{\pgfqpoint{3.565433in}{0.332928in}}%
\pgfpathlineto{\pgfqpoint{3.565433in}{0.332928in}}%
\pgfpathlineto{\pgfqpoint{3.565433in}{0.337185in}}%
\pgfpathlineto{\pgfqpoint{3.569691in}{0.337185in}}%
\pgfpathlineto{\pgfqpoint{3.569691in}{0.332928in}}%
\pgfpathmoveto{\pgfqpoint{3.565433in}{0.337185in}}%
\pgfpathlineto{\pgfqpoint{3.565433in}{0.337185in}}%
\pgfpathlineto{\pgfqpoint{3.565433in}{0.341443in}}%
\pgfpathlineto{\pgfqpoint{3.569691in}{0.341443in}}%
\pgfpathlineto{\pgfqpoint{3.569691in}{0.337185in}}%
\pgfpathmoveto{\pgfqpoint{3.565433in}{0.341443in}}%
\pgfpathlineto{\pgfqpoint{3.565433in}{0.341443in}}%
\pgfpathlineto{\pgfqpoint{3.565433in}{0.345700in}}%
\pgfpathlineto{\pgfqpoint{3.569691in}{0.345700in}}%
\pgfpathlineto{\pgfqpoint{3.569691in}{0.341443in}}%
\pgfpathmoveto{\pgfqpoint{3.561175in}{0.345700in}}%
\pgfpathlineto{\pgfqpoint{3.561175in}{0.345700in}}%
\pgfpathlineto{\pgfqpoint{3.561175in}{0.349958in}}%
\pgfpathlineto{\pgfqpoint{3.565433in}{0.349958in}}%
\pgfpathlineto{\pgfqpoint{3.565433in}{0.345700in}}%
\pgfpathmoveto{\pgfqpoint{3.561175in}{0.349958in}}%
\pgfpathlineto{\pgfqpoint{3.561175in}{0.349958in}}%
\pgfpathlineto{\pgfqpoint{3.561175in}{0.354216in}}%
\pgfpathlineto{\pgfqpoint{3.565433in}{0.354216in}}%
\pgfpathlineto{\pgfqpoint{3.565433in}{0.349958in}}%
\pgfpathmoveto{\pgfqpoint{3.565433in}{0.345700in}}%
\pgfpathlineto{\pgfqpoint{3.565433in}{0.345700in}}%
\pgfpathlineto{\pgfqpoint{3.565433in}{0.349958in}}%
\pgfpathlineto{\pgfqpoint{3.569691in}{0.349958in}}%
\pgfpathlineto{\pgfqpoint{3.569691in}{0.345700in}}%
\pgfpathmoveto{\pgfqpoint{3.561175in}{0.354216in}}%
\pgfpathlineto{\pgfqpoint{3.561175in}{0.354216in}}%
\pgfpathlineto{\pgfqpoint{3.561175in}{0.358473in}}%
\pgfpathlineto{\pgfqpoint{3.565433in}{0.358473in}}%
\pgfpathlineto{\pgfqpoint{3.565433in}{0.354216in}}%
\pgfpathmoveto{\pgfqpoint{3.561175in}{0.358473in}}%
\pgfpathlineto{\pgfqpoint{3.561175in}{0.358473in}}%
\pgfpathlineto{\pgfqpoint{3.561175in}{0.362731in}}%
\pgfpathlineto{\pgfqpoint{3.565433in}{0.362731in}}%
\pgfpathlineto{\pgfqpoint{3.565433in}{0.358473in}}%
\pgfpathmoveto{\pgfqpoint{3.561175in}{0.362731in}}%
\pgfpathlineto{\pgfqpoint{3.561175in}{0.362731in}}%
\pgfpathlineto{\pgfqpoint{3.561175in}{0.366988in}}%
\pgfpathlineto{\pgfqpoint{3.565433in}{0.366988in}}%
\pgfpathlineto{\pgfqpoint{3.565433in}{0.362731in}}%
\pgfpathmoveto{\pgfqpoint{3.561175in}{0.366988in}}%
\pgfpathlineto{\pgfqpoint{3.561175in}{0.366988in}}%
\pgfpathlineto{\pgfqpoint{3.561175in}{0.371246in}}%
\pgfpathlineto{\pgfqpoint{3.565433in}{0.371246in}}%
\pgfpathlineto{\pgfqpoint{3.565433in}{0.366988in}}%
\pgfpathmoveto{\pgfqpoint{3.573949in}{0.235002in}}%
\pgfpathlineto{\pgfqpoint{3.573949in}{0.235002in}}%
\pgfpathlineto{\pgfqpoint{3.573949in}{0.239260in}}%
\pgfpathlineto{\pgfqpoint{3.578206in}{0.239260in}}%
\pgfpathlineto{\pgfqpoint{3.578206in}{0.235002in}}%
\pgfpathmoveto{\pgfqpoint{3.573949in}{0.239260in}}%
\pgfpathlineto{\pgfqpoint{3.573949in}{0.239260in}}%
\pgfpathlineto{\pgfqpoint{3.573949in}{0.243518in}}%
\pgfpathlineto{\pgfqpoint{3.578206in}{0.243518in}}%
\pgfpathlineto{\pgfqpoint{3.578206in}{0.239260in}}%
\pgfpathmoveto{\pgfqpoint{3.573949in}{0.243518in}}%
\pgfpathlineto{\pgfqpoint{3.573949in}{0.243518in}}%
\pgfpathlineto{\pgfqpoint{3.573949in}{0.247775in}}%
\pgfpathlineto{\pgfqpoint{3.578206in}{0.247775in}}%
\pgfpathlineto{\pgfqpoint{3.578206in}{0.243518in}}%
\pgfpathmoveto{\pgfqpoint{3.573949in}{0.247775in}}%
\pgfpathlineto{\pgfqpoint{3.573949in}{0.247775in}}%
\pgfpathlineto{\pgfqpoint{3.573949in}{0.252033in}}%
\pgfpathlineto{\pgfqpoint{3.578206in}{0.252033in}}%
\pgfpathlineto{\pgfqpoint{3.578206in}{0.247775in}}%
\pgfpathmoveto{\pgfqpoint{3.569691in}{0.252033in}}%
\pgfpathlineto{\pgfqpoint{3.569691in}{0.252033in}}%
\pgfpathlineto{\pgfqpoint{3.569691in}{0.256290in}}%
\pgfpathlineto{\pgfqpoint{3.573949in}{0.256290in}}%
\pgfpathlineto{\pgfqpoint{3.573949in}{0.252033in}}%
\pgfpathmoveto{\pgfqpoint{3.569691in}{0.256290in}}%
\pgfpathlineto{\pgfqpoint{3.569691in}{0.256290in}}%
\pgfpathlineto{\pgfqpoint{3.569691in}{0.260548in}}%
\pgfpathlineto{\pgfqpoint{3.573949in}{0.260548in}}%
\pgfpathlineto{\pgfqpoint{3.573949in}{0.256290in}}%
\pgfpathmoveto{\pgfqpoint{3.573949in}{0.252033in}}%
\pgfpathlineto{\pgfqpoint{3.573949in}{0.252033in}}%
\pgfpathlineto{\pgfqpoint{3.573949in}{0.256290in}}%
\pgfpathlineto{\pgfqpoint{3.578206in}{0.256290in}}%
\pgfpathlineto{\pgfqpoint{3.578206in}{0.252033in}}%
\pgfpathmoveto{\pgfqpoint{3.569691in}{0.260548in}}%
\pgfpathlineto{\pgfqpoint{3.569691in}{0.260548in}}%
\pgfpathlineto{\pgfqpoint{3.569691in}{0.264806in}}%
\pgfpathlineto{\pgfqpoint{3.573949in}{0.264806in}}%
\pgfpathlineto{\pgfqpoint{3.573949in}{0.260548in}}%
\pgfpathmoveto{\pgfqpoint{3.569691in}{0.264806in}}%
\pgfpathlineto{\pgfqpoint{3.569691in}{0.264806in}}%
\pgfpathlineto{\pgfqpoint{3.569691in}{0.269063in}}%
\pgfpathlineto{\pgfqpoint{3.573949in}{0.269063in}}%
\pgfpathlineto{\pgfqpoint{3.573949in}{0.264806in}}%
\pgfpathmoveto{\pgfqpoint{3.569691in}{0.269063in}}%
\pgfpathlineto{\pgfqpoint{3.569691in}{0.269063in}}%
\pgfpathlineto{\pgfqpoint{3.569691in}{0.273321in}}%
\pgfpathlineto{\pgfqpoint{3.573949in}{0.273321in}}%
\pgfpathlineto{\pgfqpoint{3.573949in}{0.269063in}}%
\pgfpathmoveto{\pgfqpoint{3.569691in}{0.273321in}}%
\pgfpathlineto{\pgfqpoint{3.569691in}{0.273321in}}%
\pgfpathlineto{\pgfqpoint{3.569691in}{0.277579in}}%
\pgfpathlineto{\pgfqpoint{3.573949in}{0.277579in}}%
\pgfpathlineto{\pgfqpoint{3.573949in}{0.273321in}}%
\pgfpathmoveto{\pgfqpoint{3.569691in}{0.277579in}}%
\pgfpathlineto{\pgfqpoint{3.569691in}{0.277579in}}%
\pgfpathlineto{\pgfqpoint{3.569691in}{0.281836in}}%
\pgfpathlineto{\pgfqpoint{3.573949in}{0.281836in}}%
\pgfpathlineto{\pgfqpoint{3.573949in}{0.277579in}}%
\pgfpathmoveto{\pgfqpoint{3.569691in}{0.281836in}}%
\pgfpathlineto{\pgfqpoint{3.569691in}{0.281836in}}%
\pgfpathlineto{\pgfqpoint{3.569691in}{0.286094in}}%
\pgfpathlineto{\pgfqpoint{3.573949in}{0.286094in}}%
\pgfpathlineto{\pgfqpoint{3.573949in}{0.281836in}}%
\pgfpathmoveto{\pgfqpoint{3.569691in}{0.286094in}}%
\pgfpathlineto{\pgfqpoint{3.569691in}{0.286094in}}%
\pgfpathlineto{\pgfqpoint{3.569691in}{0.290351in}}%
\pgfpathlineto{\pgfqpoint{3.573949in}{0.290351in}}%
\pgfpathlineto{\pgfqpoint{3.573949in}{0.286094in}}%
\pgfpathmoveto{\pgfqpoint{3.569691in}{0.290351in}}%
\pgfpathlineto{\pgfqpoint{3.569691in}{0.290351in}}%
\pgfpathlineto{\pgfqpoint{3.569691in}{0.294609in}}%
\pgfpathlineto{\pgfqpoint{3.573949in}{0.294609in}}%
\pgfpathlineto{\pgfqpoint{3.573949in}{0.290351in}}%
\pgfpathmoveto{\pgfqpoint{3.569691in}{0.294609in}}%
\pgfpathlineto{\pgfqpoint{3.569691in}{0.294609in}}%
\pgfpathlineto{\pgfqpoint{3.569691in}{0.298867in}}%
\pgfpathlineto{\pgfqpoint{3.573949in}{0.298867in}}%
\pgfpathlineto{\pgfqpoint{3.573949in}{0.294609in}}%
\pgfpathmoveto{\pgfqpoint{3.569691in}{0.298867in}}%
\pgfpathlineto{\pgfqpoint{3.569691in}{0.298867in}}%
\pgfpathlineto{\pgfqpoint{3.569691in}{0.303124in}}%
\pgfpathlineto{\pgfqpoint{3.573949in}{0.303124in}}%
\pgfpathlineto{\pgfqpoint{3.573949in}{0.298867in}}%
\pgfpathmoveto{\pgfqpoint{3.561175in}{0.371246in}}%
\pgfpathlineto{\pgfqpoint{3.561175in}{0.371246in}}%
\pgfpathlineto{\pgfqpoint{3.561175in}{0.375504in}}%
\pgfpathlineto{\pgfqpoint{3.565433in}{0.375504in}}%
\pgfpathlineto{\pgfqpoint{3.565433in}{0.371246in}}%
\pgfpathmoveto{\pgfqpoint{3.561175in}{0.375504in}}%
\pgfpathlineto{\pgfqpoint{3.561175in}{0.375504in}}%
\pgfpathlineto{\pgfqpoint{3.561175in}{0.379762in}}%
\pgfpathlineto{\pgfqpoint{3.565433in}{0.379762in}}%
\pgfpathlineto{\pgfqpoint{3.565433in}{0.375504in}}%
\pgfpathmoveto{\pgfqpoint{3.561175in}{0.379762in}}%
\pgfpathlineto{\pgfqpoint{3.561175in}{0.379762in}}%
\pgfpathlineto{\pgfqpoint{3.561175in}{0.384020in}}%
\pgfpathlineto{\pgfqpoint{3.565433in}{0.384020in}}%
\pgfpathlineto{\pgfqpoint{3.565433in}{0.379762in}}%
\pgfpathmoveto{\pgfqpoint{3.561175in}{0.384020in}}%
\pgfpathlineto{\pgfqpoint{3.561175in}{0.384020in}}%
\pgfpathlineto{\pgfqpoint{3.561175in}{0.388278in}}%
\pgfpathlineto{\pgfqpoint{3.565433in}{0.388278in}}%
\pgfpathlineto{\pgfqpoint{3.565433in}{0.384020in}}%
\pgfpathmoveto{\pgfqpoint{3.556917in}{0.392536in}}%
\pgfpathlineto{\pgfqpoint{3.556917in}{0.392536in}}%
\pgfpathlineto{\pgfqpoint{3.556917in}{0.396794in}}%
\pgfpathlineto{\pgfqpoint{3.561175in}{0.396794in}}%
\pgfpathlineto{\pgfqpoint{3.561175in}{0.392536in}}%
\pgfpathmoveto{\pgfqpoint{3.556917in}{0.396794in}}%
\pgfpathlineto{\pgfqpoint{3.556917in}{0.396794in}}%
\pgfpathlineto{\pgfqpoint{3.556917in}{0.401052in}}%
\pgfpathlineto{\pgfqpoint{3.561175in}{0.401052in}}%
\pgfpathlineto{\pgfqpoint{3.561175in}{0.396794in}}%
\pgfpathmoveto{\pgfqpoint{3.556917in}{0.401052in}}%
\pgfpathlineto{\pgfqpoint{3.556917in}{0.401052in}}%
\pgfpathlineto{\pgfqpoint{3.556917in}{0.405310in}}%
\pgfpathlineto{\pgfqpoint{3.561175in}{0.405310in}}%
\pgfpathlineto{\pgfqpoint{3.561175in}{0.401052in}}%
\pgfpathmoveto{\pgfqpoint{3.561175in}{0.388278in}}%
\pgfpathlineto{\pgfqpoint{3.561175in}{0.388278in}}%
\pgfpathlineto{\pgfqpoint{3.561175in}{0.392536in}}%
\pgfpathlineto{\pgfqpoint{3.565433in}{0.392536in}}%
\pgfpathlineto{\pgfqpoint{3.565433in}{0.388278in}}%
\pgfpathmoveto{\pgfqpoint{3.561175in}{0.392536in}}%
\pgfpathlineto{\pgfqpoint{3.561175in}{0.392536in}}%
\pgfpathlineto{\pgfqpoint{3.561175in}{0.396794in}}%
\pgfpathlineto{\pgfqpoint{3.565433in}{0.396794in}}%
\pgfpathlineto{\pgfqpoint{3.565433in}{0.392536in}}%
\pgfpathmoveto{\pgfqpoint{3.556917in}{0.405310in}}%
\pgfpathlineto{\pgfqpoint{3.556917in}{0.405310in}}%
\pgfpathlineto{\pgfqpoint{3.556917in}{0.409568in}}%
\pgfpathlineto{\pgfqpoint{3.561175in}{0.409568in}}%
\pgfpathlineto{\pgfqpoint{3.561175in}{0.405310in}}%
\pgfpathmoveto{\pgfqpoint{3.556917in}{0.409568in}}%
\pgfpathlineto{\pgfqpoint{3.556917in}{0.409568in}}%
\pgfpathlineto{\pgfqpoint{3.556917in}{0.413826in}}%
\pgfpathlineto{\pgfqpoint{3.561175in}{0.413826in}}%
\pgfpathlineto{\pgfqpoint{3.561175in}{0.409568in}}%
\pgfpathmoveto{\pgfqpoint{3.556917in}{0.413826in}}%
\pgfpathlineto{\pgfqpoint{3.556917in}{0.413826in}}%
\pgfpathlineto{\pgfqpoint{3.556917in}{0.418084in}}%
\pgfpathlineto{\pgfqpoint{3.561175in}{0.418084in}}%
\pgfpathlineto{\pgfqpoint{3.561175in}{0.413826in}}%
\pgfpathmoveto{\pgfqpoint{3.556917in}{0.418084in}}%
\pgfpathlineto{\pgfqpoint{3.556917in}{0.418084in}}%
\pgfpathlineto{\pgfqpoint{3.556917in}{0.422342in}}%
\pgfpathlineto{\pgfqpoint{3.561175in}{0.422342in}}%
\pgfpathlineto{\pgfqpoint{3.561175in}{0.418084in}}%
\pgfpathmoveto{\pgfqpoint{3.556917in}{0.422342in}}%
\pgfpathlineto{\pgfqpoint{3.556917in}{0.422342in}}%
\pgfpathlineto{\pgfqpoint{3.556917in}{0.426600in}}%
\pgfpathlineto{\pgfqpoint{3.561175in}{0.426600in}}%
\pgfpathlineto{\pgfqpoint{3.561175in}{0.422342in}}%
\pgfpathmoveto{\pgfqpoint{3.556917in}{0.426600in}}%
\pgfpathlineto{\pgfqpoint{3.556917in}{0.426600in}}%
\pgfpathlineto{\pgfqpoint{3.556917in}{0.430857in}}%
\pgfpathlineto{\pgfqpoint{3.561175in}{0.430857in}}%
\pgfpathlineto{\pgfqpoint{3.561175in}{0.426600in}}%
\pgfpathmoveto{\pgfqpoint{3.556917in}{0.430857in}}%
\pgfpathlineto{\pgfqpoint{3.556917in}{0.430857in}}%
\pgfpathlineto{\pgfqpoint{3.556917in}{0.435115in}}%
\pgfpathlineto{\pgfqpoint{3.561175in}{0.435115in}}%
\pgfpathlineto{\pgfqpoint{3.561175in}{0.430857in}}%
\pgfpathmoveto{\pgfqpoint{3.556917in}{0.435115in}}%
\pgfpathlineto{\pgfqpoint{3.556917in}{0.435115in}}%
\pgfpathlineto{\pgfqpoint{3.556917in}{0.439373in}}%
\pgfpathlineto{\pgfqpoint{3.561175in}{0.439373in}}%
\pgfpathlineto{\pgfqpoint{3.561175in}{0.435115in}}%
\pgfpathmoveto{\pgfqpoint{3.552659in}{0.439373in}}%
\pgfpathlineto{\pgfqpoint{3.552659in}{0.439373in}}%
\pgfpathlineto{\pgfqpoint{3.552659in}{0.443631in}}%
\pgfpathlineto{\pgfqpoint{3.556917in}{0.443631in}}%
\pgfpathlineto{\pgfqpoint{3.556917in}{0.439373in}}%
\pgfpathmoveto{\pgfqpoint{3.552659in}{0.443631in}}%
\pgfpathlineto{\pgfqpoint{3.552659in}{0.443631in}}%
\pgfpathlineto{\pgfqpoint{3.552659in}{0.447889in}}%
\pgfpathlineto{\pgfqpoint{3.556917in}{0.447889in}}%
\pgfpathlineto{\pgfqpoint{3.556917in}{0.443631in}}%
\pgfpathmoveto{\pgfqpoint{3.556917in}{0.439373in}}%
\pgfpathlineto{\pgfqpoint{3.556917in}{0.439373in}}%
\pgfpathlineto{\pgfqpoint{3.556917in}{0.443631in}}%
\pgfpathlineto{\pgfqpoint{3.561175in}{0.443631in}}%
\pgfpathlineto{\pgfqpoint{3.561175in}{0.439373in}}%
\pgfpathmoveto{\pgfqpoint{3.552659in}{0.447889in}}%
\pgfpathlineto{\pgfqpoint{3.552659in}{0.447889in}}%
\pgfpathlineto{\pgfqpoint{3.552659in}{0.452147in}}%
\pgfpathlineto{\pgfqpoint{3.556917in}{0.452147in}}%
\pgfpathlineto{\pgfqpoint{3.556917in}{0.447889in}}%
\pgfpathmoveto{\pgfqpoint{3.552659in}{0.452147in}}%
\pgfpathlineto{\pgfqpoint{3.552659in}{0.452147in}}%
\pgfpathlineto{\pgfqpoint{3.552659in}{0.456405in}}%
\pgfpathlineto{\pgfqpoint{3.556917in}{0.456405in}}%
\pgfpathlineto{\pgfqpoint{3.556917in}{0.452147in}}%
\pgfpathmoveto{\pgfqpoint{3.552659in}{0.456405in}}%
\pgfpathlineto{\pgfqpoint{3.552659in}{0.456405in}}%
\pgfpathlineto{\pgfqpoint{3.552659in}{0.460663in}}%
\pgfpathlineto{\pgfqpoint{3.556917in}{0.460663in}}%
\pgfpathlineto{\pgfqpoint{3.556917in}{0.456405in}}%
\pgfpathmoveto{\pgfqpoint{3.552659in}{0.460663in}}%
\pgfpathlineto{\pgfqpoint{3.552659in}{0.460663in}}%
\pgfpathlineto{\pgfqpoint{3.552659in}{0.464921in}}%
\pgfpathlineto{\pgfqpoint{3.556917in}{0.464921in}}%
\pgfpathlineto{\pgfqpoint{3.556917in}{0.460663in}}%
\pgfpathmoveto{\pgfqpoint{3.552659in}{0.464921in}}%
\pgfpathlineto{\pgfqpoint{3.552659in}{0.464921in}}%
\pgfpathlineto{\pgfqpoint{3.552659in}{0.469179in}}%
\pgfpathlineto{\pgfqpoint{3.556917in}{0.469179in}}%
\pgfpathlineto{\pgfqpoint{3.556917in}{0.464921in}}%
\pgfpathmoveto{\pgfqpoint{3.552659in}{0.469179in}}%
\pgfpathlineto{\pgfqpoint{3.552659in}{0.469179in}}%
\pgfpathlineto{\pgfqpoint{3.552659in}{0.473437in}}%
\pgfpathlineto{\pgfqpoint{3.556917in}{0.473437in}}%
\pgfpathlineto{\pgfqpoint{3.556917in}{0.469179in}}%
\pgfpathmoveto{\pgfqpoint{3.548401in}{0.486211in}}%
\pgfpathlineto{\pgfqpoint{3.548401in}{0.486211in}}%
\pgfpathlineto{\pgfqpoint{3.548401in}{0.490469in}}%
\pgfpathlineto{\pgfqpoint{3.552659in}{0.490469in}}%
\pgfpathlineto{\pgfqpoint{3.552659in}{0.486211in}}%
\pgfpathmoveto{\pgfqpoint{3.548401in}{0.490469in}}%
\pgfpathlineto{\pgfqpoint{3.548401in}{0.490469in}}%
\pgfpathlineto{\pgfqpoint{3.548401in}{0.494727in}}%
\pgfpathlineto{\pgfqpoint{3.552659in}{0.494727in}}%
\pgfpathlineto{\pgfqpoint{3.552659in}{0.490469in}}%
\pgfpathmoveto{\pgfqpoint{3.548401in}{0.494727in}}%
\pgfpathlineto{\pgfqpoint{3.548401in}{0.494727in}}%
\pgfpathlineto{\pgfqpoint{3.548401in}{0.498985in}}%
\pgfpathlineto{\pgfqpoint{3.552659in}{0.498985in}}%
\pgfpathlineto{\pgfqpoint{3.552659in}{0.494727in}}%
\pgfpathmoveto{\pgfqpoint{3.548401in}{0.498985in}}%
\pgfpathlineto{\pgfqpoint{3.548401in}{0.498985in}}%
\pgfpathlineto{\pgfqpoint{3.548401in}{0.503243in}}%
\pgfpathlineto{\pgfqpoint{3.552659in}{0.503243in}}%
\pgfpathlineto{\pgfqpoint{3.552659in}{0.498985in}}%
\pgfpathmoveto{\pgfqpoint{3.548401in}{0.503243in}}%
\pgfpathlineto{\pgfqpoint{3.548401in}{0.503243in}}%
\pgfpathlineto{\pgfqpoint{3.548401in}{0.507501in}}%
\pgfpathlineto{\pgfqpoint{3.552659in}{0.507501in}}%
\pgfpathlineto{\pgfqpoint{3.552659in}{0.503243in}}%
\pgfpathmoveto{\pgfqpoint{3.552659in}{0.473437in}}%
\pgfpathlineto{\pgfqpoint{3.552659in}{0.473437in}}%
\pgfpathlineto{\pgfqpoint{3.552659in}{0.477695in}}%
\pgfpathlineto{\pgfqpoint{3.556917in}{0.477695in}}%
\pgfpathlineto{\pgfqpoint{3.556917in}{0.473437in}}%
\pgfpathmoveto{\pgfqpoint{3.552659in}{0.477695in}}%
\pgfpathlineto{\pgfqpoint{3.552659in}{0.477695in}}%
\pgfpathlineto{\pgfqpoint{3.552659in}{0.481953in}}%
\pgfpathlineto{\pgfqpoint{3.556917in}{0.481953in}}%
\pgfpathlineto{\pgfqpoint{3.556917in}{0.477695in}}%
\pgfpathmoveto{\pgfqpoint{3.552659in}{0.481953in}}%
\pgfpathlineto{\pgfqpoint{3.552659in}{0.481953in}}%
\pgfpathlineto{\pgfqpoint{3.552659in}{0.486211in}}%
\pgfpathlineto{\pgfqpoint{3.556917in}{0.486211in}}%
\pgfpathlineto{\pgfqpoint{3.556917in}{0.481953in}}%
\pgfpathmoveto{\pgfqpoint{3.552659in}{0.486211in}}%
\pgfpathlineto{\pgfqpoint{3.552659in}{0.486211in}}%
\pgfpathlineto{\pgfqpoint{3.552659in}{0.490469in}}%
\pgfpathlineto{\pgfqpoint{3.556917in}{0.490469in}}%
\pgfpathlineto{\pgfqpoint{3.556917in}{0.486211in}}%
\pgfpathmoveto{\pgfqpoint{3.548401in}{0.507501in}}%
\pgfpathlineto{\pgfqpoint{3.548401in}{0.507501in}}%
\pgfpathlineto{\pgfqpoint{3.548401in}{0.511759in}}%
\pgfpathlineto{\pgfqpoint{3.552659in}{0.511759in}}%
\pgfpathlineto{\pgfqpoint{3.552659in}{0.507501in}}%
\pgfpathmoveto{\pgfqpoint{3.548401in}{0.511759in}}%
\pgfpathlineto{\pgfqpoint{3.548401in}{0.511759in}}%
\pgfpathlineto{\pgfqpoint{3.548401in}{0.516016in}}%
\pgfpathlineto{\pgfqpoint{3.552659in}{0.516016in}}%
\pgfpathlineto{\pgfqpoint{3.552659in}{0.511759in}}%
\pgfpathmoveto{\pgfqpoint{3.548401in}{0.516016in}}%
\pgfpathlineto{\pgfqpoint{3.548401in}{0.516016in}}%
\pgfpathlineto{\pgfqpoint{3.548401in}{0.520274in}}%
\pgfpathlineto{\pgfqpoint{3.552659in}{0.520274in}}%
\pgfpathlineto{\pgfqpoint{3.552659in}{0.516016in}}%
\pgfpathmoveto{\pgfqpoint{3.548401in}{0.520274in}}%
\pgfpathlineto{\pgfqpoint{3.548401in}{0.520274in}}%
\pgfpathlineto{\pgfqpoint{3.548401in}{0.524532in}}%
\pgfpathlineto{\pgfqpoint{3.552659in}{0.524532in}}%
\pgfpathlineto{\pgfqpoint{3.552659in}{0.520274in}}%
\pgfpathmoveto{\pgfqpoint{3.548401in}{0.524532in}}%
\pgfpathlineto{\pgfqpoint{3.548401in}{0.524532in}}%
\pgfpathlineto{\pgfqpoint{3.548401in}{0.528790in}}%
\pgfpathlineto{\pgfqpoint{3.552659in}{0.528790in}}%
\pgfpathlineto{\pgfqpoint{3.552659in}{0.524532in}}%
\pgfpathmoveto{\pgfqpoint{3.548401in}{0.528790in}}%
\pgfpathlineto{\pgfqpoint{3.548401in}{0.528790in}}%
\pgfpathlineto{\pgfqpoint{3.548401in}{0.533048in}}%
\pgfpathlineto{\pgfqpoint{3.552659in}{0.533048in}}%
\pgfpathlineto{\pgfqpoint{3.552659in}{0.528790in}}%
\pgfpathmoveto{\pgfqpoint{3.544143in}{0.533048in}}%
\pgfpathlineto{\pgfqpoint{3.544143in}{0.533048in}}%
\pgfpathlineto{\pgfqpoint{3.544143in}{0.537306in}}%
\pgfpathlineto{\pgfqpoint{3.548401in}{0.537306in}}%
\pgfpathlineto{\pgfqpoint{3.548401in}{0.533048in}}%
\pgfpathmoveto{\pgfqpoint{3.544143in}{0.537306in}}%
\pgfpathlineto{\pgfqpoint{3.544143in}{0.537306in}}%
\pgfpathlineto{\pgfqpoint{3.544143in}{0.541564in}}%
\pgfpathlineto{\pgfqpoint{3.548401in}{0.541564in}}%
\pgfpathlineto{\pgfqpoint{3.548401in}{0.537306in}}%
\pgfpathmoveto{\pgfqpoint{3.548401in}{0.533048in}}%
\pgfpathlineto{\pgfqpoint{3.548401in}{0.533048in}}%
\pgfpathlineto{\pgfqpoint{3.548401in}{0.537306in}}%
\pgfpathlineto{\pgfqpoint{3.552659in}{0.537306in}}%
\pgfpathlineto{\pgfqpoint{3.552659in}{0.533048in}}%
\pgfpathmoveto{\pgfqpoint{3.544143in}{0.541564in}}%
\pgfpathlineto{\pgfqpoint{3.544143in}{0.541564in}}%
\pgfpathlineto{\pgfqpoint{3.544143in}{0.545822in}}%
\pgfpathlineto{\pgfqpoint{3.548401in}{0.545822in}}%
\pgfpathlineto{\pgfqpoint{3.548401in}{0.541564in}}%
\pgfpathmoveto{\pgfqpoint{3.544143in}{0.545822in}}%
\pgfpathlineto{\pgfqpoint{3.544143in}{0.545822in}}%
\pgfpathlineto{\pgfqpoint{3.544143in}{0.550080in}}%
\pgfpathlineto{\pgfqpoint{3.548401in}{0.550080in}}%
\pgfpathlineto{\pgfqpoint{3.548401in}{0.545822in}}%
\pgfpathmoveto{\pgfqpoint{3.544143in}{0.550080in}}%
\pgfpathlineto{\pgfqpoint{3.544143in}{0.550080in}}%
\pgfpathlineto{\pgfqpoint{3.544143in}{0.554337in}}%
\pgfpathlineto{\pgfqpoint{3.548401in}{0.554337in}}%
\pgfpathlineto{\pgfqpoint{3.548401in}{0.550080in}}%
\pgfpathmoveto{\pgfqpoint{3.544143in}{0.554337in}}%
\pgfpathlineto{\pgfqpoint{3.544143in}{0.554337in}}%
\pgfpathlineto{\pgfqpoint{3.544143in}{0.558595in}}%
\pgfpathlineto{\pgfqpoint{3.548401in}{0.558595in}}%
\pgfpathlineto{\pgfqpoint{3.548401in}{0.554337in}}%
\pgfpathmoveto{\pgfqpoint{3.544143in}{0.558595in}}%
\pgfpathlineto{\pgfqpoint{3.544143in}{0.558595in}}%
\pgfpathlineto{\pgfqpoint{3.544143in}{0.562853in}}%
\pgfpathlineto{\pgfqpoint{3.548401in}{0.562853in}}%
\pgfpathlineto{\pgfqpoint{3.548401in}{0.558595in}}%
\pgfpathmoveto{\pgfqpoint{3.544143in}{0.562853in}}%
\pgfpathlineto{\pgfqpoint{3.544143in}{0.562853in}}%
\pgfpathlineto{\pgfqpoint{3.544143in}{0.567111in}}%
\pgfpathlineto{\pgfqpoint{3.548401in}{0.567111in}}%
\pgfpathlineto{\pgfqpoint{3.548401in}{0.562853in}}%
\pgfpathmoveto{\pgfqpoint{3.544143in}{0.567111in}}%
\pgfpathlineto{\pgfqpoint{3.544143in}{0.567111in}}%
\pgfpathlineto{\pgfqpoint{3.544143in}{0.571369in}}%
\pgfpathlineto{\pgfqpoint{3.548401in}{0.571369in}}%
\pgfpathlineto{\pgfqpoint{3.548401in}{0.567111in}}%
\pgfpathmoveto{\pgfqpoint{3.544143in}{0.571369in}}%
\pgfpathlineto{\pgfqpoint{3.544143in}{0.571369in}}%
\pgfpathlineto{\pgfqpoint{3.544143in}{0.575627in}}%
\pgfpathlineto{\pgfqpoint{3.548401in}{0.575627in}}%
\pgfpathlineto{\pgfqpoint{3.548401in}{0.571369in}}%
\pgfpathmoveto{\pgfqpoint{3.539886in}{0.579885in}}%
\pgfpathlineto{\pgfqpoint{3.539886in}{0.579885in}}%
\pgfpathlineto{\pgfqpoint{3.539886in}{0.584143in}}%
\pgfpathlineto{\pgfqpoint{3.544143in}{0.584143in}}%
\pgfpathlineto{\pgfqpoint{3.544143in}{0.579885in}}%
\pgfpathmoveto{\pgfqpoint{3.539886in}{0.584143in}}%
\pgfpathlineto{\pgfqpoint{3.539886in}{0.584143in}}%
\pgfpathlineto{\pgfqpoint{3.539886in}{0.588401in}}%
\pgfpathlineto{\pgfqpoint{3.544143in}{0.588401in}}%
\pgfpathlineto{\pgfqpoint{3.544143in}{0.584143in}}%
\pgfpathmoveto{\pgfqpoint{3.539886in}{0.588401in}}%
\pgfpathlineto{\pgfqpoint{3.539886in}{0.588401in}}%
\pgfpathlineto{\pgfqpoint{3.539886in}{0.592658in}}%
\pgfpathlineto{\pgfqpoint{3.544143in}{0.592658in}}%
\pgfpathlineto{\pgfqpoint{3.544143in}{0.588401in}}%
\pgfpathmoveto{\pgfqpoint{3.544143in}{0.575627in}}%
\pgfpathlineto{\pgfqpoint{3.544143in}{0.575627in}}%
\pgfpathlineto{\pgfqpoint{3.544143in}{0.579885in}}%
\pgfpathlineto{\pgfqpoint{3.548401in}{0.579885in}}%
\pgfpathlineto{\pgfqpoint{3.548401in}{0.575627in}}%
\pgfpathmoveto{\pgfqpoint{3.544143in}{0.579885in}}%
\pgfpathlineto{\pgfqpoint{3.544143in}{0.579885in}}%
\pgfpathlineto{\pgfqpoint{3.544143in}{0.584143in}}%
\pgfpathlineto{\pgfqpoint{3.548401in}{0.584143in}}%
\pgfpathlineto{\pgfqpoint{3.548401in}{0.579885in}}%
\pgfpathmoveto{\pgfqpoint{3.539886in}{0.592658in}}%
\pgfpathlineto{\pgfqpoint{3.539886in}{0.592658in}}%
\pgfpathlineto{\pgfqpoint{3.539886in}{0.596916in}}%
\pgfpathlineto{\pgfqpoint{3.544143in}{0.596916in}}%
\pgfpathlineto{\pgfqpoint{3.544143in}{0.592658in}}%
\pgfpathmoveto{\pgfqpoint{3.539886in}{0.596916in}}%
\pgfpathlineto{\pgfqpoint{3.539886in}{0.596916in}}%
\pgfpathlineto{\pgfqpoint{3.539886in}{0.601174in}}%
\pgfpathlineto{\pgfqpoint{3.544143in}{0.601174in}}%
\pgfpathlineto{\pgfqpoint{3.544143in}{0.596916in}}%
\pgfpathmoveto{\pgfqpoint{3.539886in}{0.601174in}}%
\pgfpathlineto{\pgfqpoint{3.539886in}{0.601174in}}%
\pgfpathlineto{\pgfqpoint{3.539886in}{0.605432in}}%
\pgfpathlineto{\pgfqpoint{3.544143in}{0.605432in}}%
\pgfpathlineto{\pgfqpoint{3.544143in}{0.601174in}}%
\pgfpathmoveto{\pgfqpoint{3.539886in}{0.605432in}}%
\pgfpathlineto{\pgfqpoint{3.539886in}{0.605432in}}%
\pgfpathlineto{\pgfqpoint{3.539886in}{0.609690in}}%
\pgfpathlineto{\pgfqpoint{3.544143in}{0.609690in}}%
\pgfpathlineto{\pgfqpoint{3.544143in}{0.605432in}}%
\pgfpathmoveto{\pgfqpoint{3.539886in}{0.609690in}}%
\pgfpathlineto{\pgfqpoint{3.539886in}{0.609690in}}%
\pgfpathlineto{\pgfqpoint{3.539886in}{0.613948in}}%
\pgfpathlineto{\pgfqpoint{3.544143in}{0.613948in}}%
\pgfpathlineto{\pgfqpoint{3.544143in}{0.609690in}}%
\pgfpathmoveto{\pgfqpoint{3.539886in}{0.613948in}}%
\pgfpathlineto{\pgfqpoint{3.539886in}{0.613948in}}%
\pgfpathlineto{\pgfqpoint{3.539886in}{0.618206in}}%
\pgfpathlineto{\pgfqpoint{3.544143in}{0.618206in}}%
\pgfpathlineto{\pgfqpoint{3.544143in}{0.613948in}}%
\pgfpathmoveto{\pgfqpoint{3.539886in}{0.618206in}}%
\pgfpathlineto{\pgfqpoint{3.539886in}{0.618206in}}%
\pgfpathlineto{\pgfqpoint{3.539886in}{0.622464in}}%
\pgfpathlineto{\pgfqpoint{3.544143in}{0.622464in}}%
\pgfpathlineto{\pgfqpoint{3.544143in}{0.618206in}}%
\pgfpathmoveto{\pgfqpoint{3.539886in}{0.622464in}}%
\pgfpathlineto{\pgfqpoint{3.539886in}{0.622464in}}%
\pgfpathlineto{\pgfqpoint{3.539886in}{0.626721in}}%
\pgfpathlineto{\pgfqpoint{3.544143in}{0.626721in}}%
\pgfpathlineto{\pgfqpoint{3.544143in}{0.622464in}}%
\pgfpathmoveto{\pgfqpoint{3.535628in}{0.626721in}}%
\pgfpathlineto{\pgfqpoint{3.535628in}{0.626721in}}%
\pgfpathlineto{\pgfqpoint{3.535628in}{0.630979in}}%
\pgfpathlineto{\pgfqpoint{3.539886in}{0.630979in}}%
\pgfpathlineto{\pgfqpoint{3.539886in}{0.626721in}}%
\pgfpathmoveto{\pgfqpoint{3.535628in}{0.630979in}}%
\pgfpathlineto{\pgfqpoint{3.535628in}{0.630979in}}%
\pgfpathlineto{\pgfqpoint{3.535628in}{0.635237in}}%
\pgfpathlineto{\pgfqpoint{3.539886in}{0.635237in}}%
\pgfpathlineto{\pgfqpoint{3.539886in}{0.630979in}}%
\pgfpathmoveto{\pgfqpoint{3.539886in}{0.626721in}}%
\pgfpathlineto{\pgfqpoint{3.539886in}{0.626721in}}%
\pgfpathlineto{\pgfqpoint{3.539886in}{0.630979in}}%
\pgfpathlineto{\pgfqpoint{3.544143in}{0.630979in}}%
\pgfpathlineto{\pgfqpoint{3.544143in}{0.626721in}}%
\pgfpathmoveto{\pgfqpoint{3.535628in}{0.635237in}}%
\pgfpathlineto{\pgfqpoint{3.535628in}{0.635237in}}%
\pgfpathlineto{\pgfqpoint{3.535628in}{0.639495in}}%
\pgfpathlineto{\pgfqpoint{3.539886in}{0.639495in}}%
\pgfpathlineto{\pgfqpoint{3.539886in}{0.635237in}}%
\pgfpathmoveto{\pgfqpoint{3.535628in}{0.639495in}}%
\pgfpathlineto{\pgfqpoint{3.535628in}{0.639495in}}%
\pgfpathlineto{\pgfqpoint{3.535628in}{0.643753in}}%
\pgfpathlineto{\pgfqpoint{3.539886in}{0.643753in}}%
\pgfpathlineto{\pgfqpoint{3.539886in}{0.639495in}}%
\pgfpathmoveto{\pgfqpoint{3.531370in}{0.673558in}}%
\pgfpathlineto{\pgfqpoint{3.531370in}{0.673558in}}%
\pgfpathlineto{\pgfqpoint{3.531370in}{0.677815in}}%
\pgfpathlineto{\pgfqpoint{3.535628in}{0.677815in}}%
\pgfpathlineto{\pgfqpoint{3.535628in}{0.673558in}}%
\pgfpathmoveto{\pgfqpoint{3.531370in}{0.677815in}}%
\pgfpathlineto{\pgfqpoint{3.531370in}{0.677815in}}%
\pgfpathlineto{\pgfqpoint{3.531370in}{0.682073in}}%
\pgfpathlineto{\pgfqpoint{3.535628in}{0.682073in}}%
\pgfpathlineto{\pgfqpoint{3.535628in}{0.677815in}}%
\pgfpathmoveto{\pgfqpoint{3.531370in}{0.682073in}}%
\pgfpathlineto{\pgfqpoint{3.531370in}{0.682073in}}%
\pgfpathlineto{\pgfqpoint{3.531370in}{0.686331in}}%
\pgfpathlineto{\pgfqpoint{3.535628in}{0.686331in}}%
\pgfpathlineto{\pgfqpoint{3.535628in}{0.682073in}}%
\pgfpathmoveto{\pgfqpoint{3.531370in}{0.686331in}}%
\pgfpathlineto{\pgfqpoint{3.531370in}{0.686331in}}%
\pgfpathlineto{\pgfqpoint{3.531370in}{0.690589in}}%
\pgfpathlineto{\pgfqpoint{3.535628in}{0.690589in}}%
\pgfpathlineto{\pgfqpoint{3.535628in}{0.686331in}}%
\pgfpathmoveto{\pgfqpoint{3.531370in}{0.690589in}}%
\pgfpathlineto{\pgfqpoint{3.531370in}{0.690589in}}%
\pgfpathlineto{\pgfqpoint{3.531370in}{0.694847in}}%
\pgfpathlineto{\pgfqpoint{3.535628in}{0.694847in}}%
\pgfpathlineto{\pgfqpoint{3.535628in}{0.690589in}}%
\pgfpathmoveto{\pgfqpoint{3.531370in}{0.694847in}}%
\pgfpathlineto{\pgfqpoint{3.531370in}{0.694847in}}%
\pgfpathlineto{\pgfqpoint{3.531370in}{0.699104in}}%
\pgfpathlineto{\pgfqpoint{3.535628in}{0.699104in}}%
\pgfpathlineto{\pgfqpoint{3.535628in}{0.694847in}}%
\pgfpathmoveto{\pgfqpoint{3.531370in}{0.699104in}}%
\pgfpathlineto{\pgfqpoint{3.531370in}{0.699104in}}%
\pgfpathlineto{\pgfqpoint{3.531370in}{0.703362in}}%
\pgfpathlineto{\pgfqpoint{3.535628in}{0.703362in}}%
\pgfpathlineto{\pgfqpoint{3.535628in}{0.699104in}}%
\pgfpathmoveto{\pgfqpoint{3.531370in}{0.703362in}}%
\pgfpathlineto{\pgfqpoint{3.531370in}{0.703362in}}%
\pgfpathlineto{\pgfqpoint{3.531370in}{0.707620in}}%
\pgfpathlineto{\pgfqpoint{3.535628in}{0.707620in}}%
\pgfpathlineto{\pgfqpoint{3.535628in}{0.703362in}}%
\pgfpathmoveto{\pgfqpoint{3.531370in}{0.707620in}}%
\pgfpathlineto{\pgfqpoint{3.531370in}{0.707620in}}%
\pgfpathlineto{\pgfqpoint{3.531370in}{0.711878in}}%
\pgfpathlineto{\pgfqpoint{3.535628in}{0.711878in}}%
\pgfpathlineto{\pgfqpoint{3.535628in}{0.707620in}}%
\pgfpathmoveto{\pgfqpoint{3.535628in}{0.643753in}}%
\pgfpathlineto{\pgfqpoint{3.535628in}{0.643753in}}%
\pgfpathlineto{\pgfqpoint{3.535628in}{0.648011in}}%
\pgfpathlineto{\pgfqpoint{3.539886in}{0.648011in}}%
\pgfpathlineto{\pgfqpoint{3.539886in}{0.643753in}}%
\pgfpathmoveto{\pgfqpoint{3.535628in}{0.648011in}}%
\pgfpathlineto{\pgfqpoint{3.535628in}{0.648011in}}%
\pgfpathlineto{\pgfqpoint{3.535628in}{0.652269in}}%
\pgfpathlineto{\pgfqpoint{3.539886in}{0.652269in}}%
\pgfpathlineto{\pgfqpoint{3.539886in}{0.648011in}}%
\pgfpathmoveto{\pgfqpoint{3.535628in}{0.652269in}}%
\pgfpathlineto{\pgfqpoint{3.535628in}{0.652269in}}%
\pgfpathlineto{\pgfqpoint{3.535628in}{0.656526in}}%
\pgfpathlineto{\pgfqpoint{3.539886in}{0.656526in}}%
\pgfpathlineto{\pgfqpoint{3.539886in}{0.652269in}}%
\pgfpathmoveto{\pgfqpoint{3.535628in}{0.656526in}}%
\pgfpathlineto{\pgfqpoint{3.535628in}{0.656526in}}%
\pgfpathlineto{\pgfqpoint{3.535628in}{0.660784in}}%
\pgfpathlineto{\pgfqpoint{3.539886in}{0.660784in}}%
\pgfpathlineto{\pgfqpoint{3.539886in}{0.656526in}}%
\pgfpathmoveto{\pgfqpoint{3.535628in}{0.660784in}}%
\pgfpathlineto{\pgfqpoint{3.535628in}{0.660784in}}%
\pgfpathlineto{\pgfqpoint{3.535628in}{0.665042in}}%
\pgfpathlineto{\pgfqpoint{3.539886in}{0.665042in}}%
\pgfpathlineto{\pgfqpoint{3.539886in}{0.660784in}}%
\pgfpathmoveto{\pgfqpoint{3.535628in}{0.665042in}}%
\pgfpathlineto{\pgfqpoint{3.535628in}{0.665042in}}%
\pgfpathlineto{\pgfqpoint{3.535628in}{0.669300in}}%
\pgfpathlineto{\pgfqpoint{3.539886in}{0.669300in}}%
\pgfpathlineto{\pgfqpoint{3.539886in}{0.665042in}}%
\pgfpathmoveto{\pgfqpoint{3.535628in}{0.669300in}}%
\pgfpathlineto{\pgfqpoint{3.535628in}{0.669300in}}%
\pgfpathlineto{\pgfqpoint{3.535628in}{0.673558in}}%
\pgfpathlineto{\pgfqpoint{3.539886in}{0.673558in}}%
\pgfpathlineto{\pgfqpoint{3.539886in}{0.669300in}}%
\pgfpathmoveto{\pgfqpoint{3.535628in}{0.673558in}}%
\pgfpathlineto{\pgfqpoint{3.535628in}{0.673558in}}%
\pgfpathlineto{\pgfqpoint{3.535628in}{0.677815in}}%
\pgfpathlineto{\pgfqpoint{3.539886in}{0.677815in}}%
\pgfpathlineto{\pgfqpoint{3.539886in}{0.673558in}}%
\pgfpathmoveto{\pgfqpoint{3.527112in}{0.716136in}}%
\pgfpathlineto{\pgfqpoint{3.527112in}{0.716136in}}%
\pgfpathlineto{\pgfqpoint{3.527112in}{0.720393in}}%
\pgfpathlineto{\pgfqpoint{3.531370in}{0.720393in}}%
\pgfpathlineto{\pgfqpoint{3.531370in}{0.716136in}}%
\pgfpathmoveto{\pgfqpoint{3.531370in}{0.711878in}}%
\pgfpathlineto{\pgfqpoint{3.531370in}{0.711878in}}%
\pgfpathlineto{\pgfqpoint{3.531370in}{0.716136in}}%
\pgfpathlineto{\pgfqpoint{3.535628in}{0.716136in}}%
\pgfpathlineto{\pgfqpoint{3.535628in}{0.711878in}}%
\pgfpathmoveto{\pgfqpoint{3.531370in}{0.716136in}}%
\pgfpathlineto{\pgfqpoint{3.531370in}{0.716136in}}%
\pgfpathlineto{\pgfqpoint{3.531370in}{0.720393in}}%
\pgfpathlineto{\pgfqpoint{3.535628in}{0.720393in}}%
\pgfpathlineto{\pgfqpoint{3.535628in}{0.716136in}}%
\pgfpathmoveto{\pgfqpoint{3.527112in}{0.720393in}}%
\pgfpathlineto{\pgfqpoint{3.527112in}{0.720393in}}%
\pgfpathlineto{\pgfqpoint{3.527112in}{0.724651in}}%
\pgfpathlineto{\pgfqpoint{3.531370in}{0.724651in}}%
\pgfpathlineto{\pgfqpoint{3.531370in}{0.720393in}}%
\pgfpathmoveto{\pgfqpoint{3.527112in}{0.724651in}}%
\pgfpathlineto{\pgfqpoint{3.527112in}{0.724651in}}%
\pgfpathlineto{\pgfqpoint{3.527112in}{0.728909in}}%
\pgfpathlineto{\pgfqpoint{3.531370in}{0.728909in}}%
\pgfpathlineto{\pgfqpoint{3.531370in}{0.724651in}}%
\pgfpathmoveto{\pgfqpoint{3.527112in}{0.728909in}}%
\pgfpathlineto{\pgfqpoint{3.527112in}{0.728909in}}%
\pgfpathlineto{\pgfqpoint{3.527112in}{0.733167in}}%
\pgfpathlineto{\pgfqpoint{3.531370in}{0.733167in}}%
\pgfpathlineto{\pgfqpoint{3.531370in}{0.728909in}}%
\pgfpathmoveto{\pgfqpoint{3.527112in}{0.733167in}}%
\pgfpathlineto{\pgfqpoint{3.527112in}{0.733167in}}%
\pgfpathlineto{\pgfqpoint{3.527112in}{0.737425in}}%
\pgfpathlineto{\pgfqpoint{3.531370in}{0.737425in}}%
\pgfpathlineto{\pgfqpoint{3.531370in}{0.733167in}}%
\pgfpathmoveto{\pgfqpoint{3.527112in}{0.737425in}}%
\pgfpathlineto{\pgfqpoint{3.527112in}{0.737425in}}%
\pgfpathlineto{\pgfqpoint{3.527112in}{0.741682in}}%
\pgfpathlineto{\pgfqpoint{3.531370in}{0.741682in}}%
\pgfpathlineto{\pgfqpoint{3.531370in}{0.737425in}}%
\pgfpathmoveto{\pgfqpoint{3.527112in}{0.741682in}}%
\pgfpathlineto{\pgfqpoint{3.527112in}{0.741682in}}%
\pgfpathlineto{\pgfqpoint{3.527112in}{0.745940in}}%
\pgfpathlineto{\pgfqpoint{3.531370in}{0.745940in}}%
\pgfpathlineto{\pgfqpoint{3.531370in}{0.741682in}}%
\pgfpathmoveto{\pgfqpoint{3.527112in}{0.745940in}}%
\pgfpathlineto{\pgfqpoint{3.527112in}{0.745940in}}%
\pgfpathlineto{\pgfqpoint{3.527112in}{0.750198in}}%
\pgfpathlineto{\pgfqpoint{3.531370in}{0.750198in}}%
\pgfpathlineto{\pgfqpoint{3.531370in}{0.745940in}}%
\pgfpathmoveto{\pgfqpoint{3.527112in}{0.750198in}}%
\pgfpathlineto{\pgfqpoint{3.527112in}{0.750198in}}%
\pgfpathlineto{\pgfqpoint{3.527112in}{0.754456in}}%
\pgfpathlineto{\pgfqpoint{3.531370in}{0.754456in}}%
\pgfpathlineto{\pgfqpoint{3.531370in}{0.750198in}}%
\pgfpathmoveto{\pgfqpoint{3.527112in}{0.754456in}}%
\pgfpathlineto{\pgfqpoint{3.527112in}{0.754456in}}%
\pgfpathlineto{\pgfqpoint{3.527112in}{0.758714in}}%
\pgfpathlineto{\pgfqpoint{3.531370in}{0.758714in}}%
\pgfpathlineto{\pgfqpoint{3.531370in}{0.754456in}}%
\pgfpathmoveto{\pgfqpoint{3.527112in}{0.758714in}}%
\pgfpathlineto{\pgfqpoint{3.527112in}{0.758714in}}%
\pgfpathlineto{\pgfqpoint{3.527112in}{0.762971in}}%
\pgfpathlineto{\pgfqpoint{3.531370in}{0.762971in}}%
\pgfpathlineto{\pgfqpoint{3.531370in}{0.758714in}}%
\pgfpathmoveto{\pgfqpoint{3.522854in}{0.762971in}}%
\pgfpathlineto{\pgfqpoint{3.522854in}{0.762971in}}%
\pgfpathlineto{\pgfqpoint{3.522854in}{0.767229in}}%
\pgfpathlineto{\pgfqpoint{3.527112in}{0.767229in}}%
\pgfpathlineto{\pgfqpoint{3.527112in}{0.762971in}}%
\pgfpathmoveto{\pgfqpoint{3.522854in}{0.767229in}}%
\pgfpathlineto{\pgfqpoint{3.522854in}{0.767229in}}%
\pgfpathlineto{\pgfqpoint{3.522854in}{0.771487in}}%
\pgfpathlineto{\pgfqpoint{3.527112in}{0.771487in}}%
\pgfpathlineto{\pgfqpoint{3.527112in}{0.767229in}}%
\pgfpathmoveto{\pgfqpoint{3.522854in}{0.771487in}}%
\pgfpathlineto{\pgfqpoint{3.522854in}{0.771487in}}%
\pgfpathlineto{\pgfqpoint{3.522854in}{0.775745in}}%
\pgfpathlineto{\pgfqpoint{3.527112in}{0.775745in}}%
\pgfpathlineto{\pgfqpoint{3.527112in}{0.771487in}}%
\pgfpathmoveto{\pgfqpoint{3.522854in}{0.775745in}}%
\pgfpathlineto{\pgfqpoint{3.522854in}{0.775745in}}%
\pgfpathlineto{\pgfqpoint{3.522854in}{0.780002in}}%
\pgfpathlineto{\pgfqpoint{3.527112in}{0.780002in}}%
\pgfpathlineto{\pgfqpoint{3.527112in}{0.775745in}}%
\pgfpathmoveto{\pgfqpoint{3.527112in}{0.762971in}}%
\pgfpathlineto{\pgfqpoint{3.527112in}{0.762971in}}%
\pgfpathlineto{\pgfqpoint{3.527112in}{0.767229in}}%
\pgfpathlineto{\pgfqpoint{3.531370in}{0.767229in}}%
\pgfpathlineto{\pgfqpoint{3.531370in}{0.762971in}}%
\pgfpathmoveto{\pgfqpoint{3.522854in}{0.780002in}}%
\pgfpathlineto{\pgfqpoint{3.522854in}{0.780002in}}%
\pgfpathlineto{\pgfqpoint{3.522854in}{0.784260in}}%
\pgfpathlineto{\pgfqpoint{3.527112in}{0.784260in}}%
\pgfpathlineto{\pgfqpoint{3.527112in}{0.780002in}}%
\pgfpathmoveto{\pgfqpoint{3.522854in}{0.784260in}}%
\pgfpathlineto{\pgfqpoint{3.522854in}{0.784260in}}%
\pgfpathlineto{\pgfqpoint{3.522854in}{0.788518in}}%
\pgfpathlineto{\pgfqpoint{3.527112in}{0.788518in}}%
\pgfpathlineto{\pgfqpoint{3.527112in}{0.784260in}}%
\pgfpathmoveto{\pgfqpoint{3.522854in}{0.788518in}}%
\pgfpathlineto{\pgfqpoint{3.522854in}{0.788518in}}%
\pgfpathlineto{\pgfqpoint{3.522854in}{0.792775in}}%
\pgfpathlineto{\pgfqpoint{3.527112in}{0.792775in}}%
\pgfpathlineto{\pgfqpoint{3.527112in}{0.788518in}}%
\pgfpathmoveto{\pgfqpoint{3.522854in}{0.792775in}}%
\pgfpathlineto{\pgfqpoint{3.522854in}{0.792775in}}%
\pgfpathlineto{\pgfqpoint{3.522854in}{0.797033in}}%
\pgfpathlineto{\pgfqpoint{3.527112in}{0.797033in}}%
\pgfpathlineto{\pgfqpoint{3.527112in}{0.792775in}}%
\pgfpathmoveto{\pgfqpoint{3.522854in}{0.797033in}}%
\pgfpathlineto{\pgfqpoint{3.522854in}{0.797033in}}%
\pgfpathlineto{\pgfqpoint{3.522854in}{0.801291in}}%
\pgfpathlineto{\pgfqpoint{3.527112in}{0.801291in}}%
\pgfpathlineto{\pgfqpoint{3.527112in}{0.797033in}}%
\pgfpathmoveto{\pgfqpoint{3.522854in}{0.801291in}}%
\pgfpathlineto{\pgfqpoint{3.522854in}{0.801291in}}%
\pgfpathlineto{\pgfqpoint{3.522854in}{0.805548in}}%
\pgfpathlineto{\pgfqpoint{3.527112in}{0.805548in}}%
\pgfpathlineto{\pgfqpoint{3.527112in}{0.801291in}}%
\pgfpathmoveto{\pgfqpoint{3.518596in}{0.805548in}}%
\pgfpathlineto{\pgfqpoint{3.518596in}{0.805548in}}%
\pgfpathlineto{\pgfqpoint{3.518596in}{0.809806in}}%
\pgfpathlineto{\pgfqpoint{3.522854in}{0.809806in}}%
\pgfpathlineto{\pgfqpoint{3.522854in}{0.805548in}}%
\pgfpathmoveto{\pgfqpoint{3.518596in}{0.809806in}}%
\pgfpathlineto{\pgfqpoint{3.518596in}{0.809806in}}%
\pgfpathlineto{\pgfqpoint{3.518596in}{0.814063in}}%
\pgfpathlineto{\pgfqpoint{3.522854in}{0.814063in}}%
\pgfpathlineto{\pgfqpoint{3.522854in}{0.809806in}}%
\pgfpathmoveto{\pgfqpoint{3.522854in}{0.805548in}}%
\pgfpathlineto{\pgfqpoint{3.522854in}{0.805548in}}%
\pgfpathlineto{\pgfqpoint{3.522854in}{0.809806in}}%
\pgfpathlineto{\pgfqpoint{3.527112in}{0.809806in}}%
\pgfpathlineto{\pgfqpoint{3.527112in}{0.805548in}}%
\pgfpathmoveto{\pgfqpoint{3.518596in}{0.814063in}}%
\pgfpathlineto{\pgfqpoint{3.518596in}{0.814063in}}%
\pgfpathlineto{\pgfqpoint{3.518596in}{0.818321in}}%
\pgfpathlineto{\pgfqpoint{3.522854in}{0.818321in}}%
\pgfpathlineto{\pgfqpoint{3.522854in}{0.814063in}}%
\pgfpathmoveto{\pgfqpoint{3.518596in}{0.818321in}}%
\pgfpathlineto{\pgfqpoint{3.518596in}{0.818321in}}%
\pgfpathlineto{\pgfqpoint{3.518596in}{0.822579in}}%
\pgfpathlineto{\pgfqpoint{3.522854in}{0.822579in}}%
\pgfpathlineto{\pgfqpoint{3.522854in}{0.818321in}}%
\pgfpathmoveto{\pgfqpoint{3.518596in}{0.822579in}}%
\pgfpathlineto{\pgfqpoint{3.518596in}{0.822579in}}%
\pgfpathlineto{\pgfqpoint{3.518596in}{0.826836in}}%
\pgfpathlineto{\pgfqpoint{3.522854in}{0.826836in}}%
\pgfpathlineto{\pgfqpoint{3.522854in}{0.822579in}}%
\pgfpathmoveto{\pgfqpoint{3.518596in}{0.826836in}}%
\pgfpathlineto{\pgfqpoint{3.518596in}{0.826836in}}%
\pgfpathlineto{\pgfqpoint{3.518596in}{0.831094in}}%
\pgfpathlineto{\pgfqpoint{3.522854in}{0.831094in}}%
\pgfpathlineto{\pgfqpoint{3.522854in}{0.826836in}}%
\pgfpathmoveto{\pgfqpoint{3.518596in}{0.831094in}}%
\pgfpathlineto{\pgfqpoint{3.518596in}{0.831094in}}%
\pgfpathlineto{\pgfqpoint{3.518596in}{0.835351in}}%
\pgfpathlineto{\pgfqpoint{3.522854in}{0.835351in}}%
\pgfpathlineto{\pgfqpoint{3.522854in}{0.831094in}}%
\pgfpathmoveto{\pgfqpoint{3.518596in}{0.835351in}}%
\pgfpathlineto{\pgfqpoint{3.518596in}{0.835351in}}%
\pgfpathlineto{\pgfqpoint{3.518596in}{0.839609in}}%
\pgfpathlineto{\pgfqpoint{3.522854in}{0.839609in}}%
\pgfpathlineto{\pgfqpoint{3.522854in}{0.835351in}}%
\pgfpathmoveto{\pgfqpoint{3.518596in}{0.839609in}}%
\pgfpathlineto{\pgfqpoint{3.518596in}{0.839609in}}%
\pgfpathlineto{\pgfqpoint{3.518596in}{0.843867in}}%
\pgfpathlineto{\pgfqpoint{3.522854in}{0.843867in}}%
\pgfpathlineto{\pgfqpoint{3.522854in}{0.839609in}}%
\pgfpathmoveto{\pgfqpoint{3.518596in}{0.843867in}}%
\pgfpathlineto{\pgfqpoint{3.518596in}{0.843867in}}%
\pgfpathlineto{\pgfqpoint{3.518596in}{0.848124in}}%
\pgfpathlineto{\pgfqpoint{3.522854in}{0.848124in}}%
\pgfpathlineto{\pgfqpoint{3.522854in}{0.843867in}}%
\pgfpathmoveto{\pgfqpoint{3.514338in}{0.848124in}}%
\pgfpathlineto{\pgfqpoint{3.514338in}{0.848124in}}%
\pgfpathlineto{\pgfqpoint{3.514338in}{0.852382in}}%
\pgfpathlineto{\pgfqpoint{3.518596in}{0.852382in}}%
\pgfpathlineto{\pgfqpoint{3.518596in}{0.848124in}}%
\pgfpathmoveto{\pgfqpoint{3.514338in}{0.852382in}}%
\pgfpathlineto{\pgfqpoint{3.514338in}{0.852382in}}%
\pgfpathlineto{\pgfqpoint{3.514338in}{0.856640in}}%
\pgfpathlineto{\pgfqpoint{3.518596in}{0.856640in}}%
\pgfpathlineto{\pgfqpoint{3.518596in}{0.852382in}}%
\pgfpathmoveto{\pgfqpoint{3.514338in}{0.856640in}}%
\pgfpathlineto{\pgfqpoint{3.514338in}{0.856640in}}%
\pgfpathlineto{\pgfqpoint{3.514338in}{0.860897in}}%
\pgfpathlineto{\pgfqpoint{3.518596in}{0.860897in}}%
\pgfpathlineto{\pgfqpoint{3.518596in}{0.856640in}}%
\pgfpathmoveto{\pgfqpoint{3.514338in}{0.860897in}}%
\pgfpathlineto{\pgfqpoint{3.514338in}{0.860897in}}%
\pgfpathlineto{\pgfqpoint{3.514338in}{0.865155in}}%
\pgfpathlineto{\pgfqpoint{3.518596in}{0.865155in}}%
\pgfpathlineto{\pgfqpoint{3.518596in}{0.860897in}}%
\pgfpathmoveto{\pgfqpoint{3.514338in}{0.865155in}}%
\pgfpathlineto{\pgfqpoint{3.514338in}{0.865155in}}%
\pgfpathlineto{\pgfqpoint{3.514338in}{0.869412in}}%
\pgfpathlineto{\pgfqpoint{3.518596in}{0.869412in}}%
\pgfpathlineto{\pgfqpoint{3.518596in}{0.865155in}}%
\pgfpathmoveto{\pgfqpoint{3.514338in}{0.869412in}}%
\pgfpathlineto{\pgfqpoint{3.514338in}{0.869412in}}%
\pgfpathlineto{\pgfqpoint{3.514338in}{0.873670in}}%
\pgfpathlineto{\pgfqpoint{3.518596in}{0.873670in}}%
\pgfpathlineto{\pgfqpoint{3.518596in}{0.869412in}}%
\pgfpathmoveto{\pgfqpoint{3.514338in}{0.873670in}}%
\pgfpathlineto{\pgfqpoint{3.514338in}{0.873670in}}%
\pgfpathlineto{\pgfqpoint{3.514338in}{0.877928in}}%
\pgfpathlineto{\pgfqpoint{3.518596in}{0.877928in}}%
\pgfpathlineto{\pgfqpoint{3.518596in}{0.873670in}}%
\pgfpathmoveto{\pgfqpoint{3.514338in}{0.877928in}}%
\pgfpathlineto{\pgfqpoint{3.514338in}{0.877928in}}%
\pgfpathlineto{\pgfqpoint{3.514338in}{0.882185in}}%
\pgfpathlineto{\pgfqpoint{3.518596in}{0.882185in}}%
\pgfpathlineto{\pgfqpoint{3.518596in}{0.877928in}}%
\pgfpathmoveto{\pgfqpoint{3.518596in}{0.848124in}}%
\pgfpathlineto{\pgfqpoint{3.518596in}{0.848124in}}%
\pgfpathlineto{\pgfqpoint{3.518596in}{0.852382in}}%
\pgfpathlineto{\pgfqpoint{3.522854in}{0.852382in}}%
\pgfpathlineto{\pgfqpoint{3.522854in}{0.848124in}}%
\pgfpathmoveto{\pgfqpoint{3.514338in}{0.882185in}}%
\pgfpathlineto{\pgfqpoint{3.514338in}{0.882185in}}%
\pgfpathlineto{\pgfqpoint{3.514338in}{0.886443in}}%
\pgfpathlineto{\pgfqpoint{3.518596in}{0.886443in}}%
\pgfpathlineto{\pgfqpoint{3.518596in}{0.882185in}}%
\pgfpathmoveto{\pgfqpoint{3.514338in}{0.886443in}}%
\pgfpathlineto{\pgfqpoint{3.514338in}{0.886443in}}%
\pgfpathlineto{\pgfqpoint{3.514338in}{0.890700in}}%
\pgfpathlineto{\pgfqpoint{3.518596in}{0.890700in}}%
\pgfpathlineto{\pgfqpoint{3.518596in}{0.886443in}}%
\pgfpathmoveto{\pgfqpoint{3.510081in}{0.894958in}}%
\pgfpathlineto{\pgfqpoint{3.510081in}{0.894958in}}%
\pgfpathlineto{\pgfqpoint{3.510081in}{0.899216in}}%
\pgfpathlineto{\pgfqpoint{3.514338in}{0.899216in}}%
\pgfpathlineto{\pgfqpoint{3.514338in}{0.894958in}}%
\pgfpathmoveto{\pgfqpoint{3.514338in}{0.890700in}}%
\pgfpathlineto{\pgfqpoint{3.514338in}{0.890700in}}%
\pgfpathlineto{\pgfqpoint{3.514338in}{0.894958in}}%
\pgfpathlineto{\pgfqpoint{3.518596in}{0.894958in}}%
\pgfpathlineto{\pgfqpoint{3.518596in}{0.890700in}}%
\pgfpathmoveto{\pgfqpoint{3.514338in}{0.894958in}}%
\pgfpathlineto{\pgfqpoint{3.514338in}{0.894958in}}%
\pgfpathlineto{\pgfqpoint{3.514338in}{0.899216in}}%
\pgfpathlineto{\pgfqpoint{3.518596in}{0.899216in}}%
\pgfpathlineto{\pgfqpoint{3.518596in}{0.894958in}}%
\pgfpathmoveto{\pgfqpoint{3.510081in}{0.899216in}}%
\pgfpathlineto{\pgfqpoint{3.510081in}{0.899216in}}%
\pgfpathlineto{\pgfqpoint{3.510081in}{0.903473in}}%
\pgfpathlineto{\pgfqpoint{3.514338in}{0.903473in}}%
\pgfpathlineto{\pgfqpoint{3.514338in}{0.899216in}}%
\pgfpathmoveto{\pgfqpoint{3.510081in}{0.903473in}}%
\pgfpathlineto{\pgfqpoint{3.510081in}{0.903473in}}%
\pgfpathlineto{\pgfqpoint{3.510081in}{0.907731in}}%
\pgfpathlineto{\pgfqpoint{3.514338in}{0.907731in}}%
\pgfpathlineto{\pgfqpoint{3.514338in}{0.903473in}}%
\pgfpathmoveto{\pgfqpoint{3.510081in}{0.907731in}}%
\pgfpathlineto{\pgfqpoint{3.510081in}{0.907731in}}%
\pgfpathlineto{\pgfqpoint{3.510081in}{0.911989in}}%
\pgfpathlineto{\pgfqpoint{3.514338in}{0.911989in}}%
\pgfpathlineto{\pgfqpoint{3.514338in}{0.907731in}}%
\pgfpathmoveto{\pgfqpoint{3.510081in}{0.911989in}}%
\pgfpathlineto{\pgfqpoint{3.510081in}{0.911989in}}%
\pgfpathlineto{\pgfqpoint{3.510081in}{0.916246in}}%
\pgfpathlineto{\pgfqpoint{3.514338in}{0.916246in}}%
\pgfpathlineto{\pgfqpoint{3.514338in}{0.911989in}}%
\pgfpathmoveto{\pgfqpoint{3.510081in}{0.916246in}}%
\pgfpathlineto{\pgfqpoint{3.510081in}{0.916246in}}%
\pgfpathlineto{\pgfqpoint{3.510081in}{0.920504in}}%
\pgfpathlineto{\pgfqpoint{3.514338in}{0.920504in}}%
\pgfpathlineto{\pgfqpoint{3.514338in}{0.916246in}}%
\pgfpathmoveto{\pgfqpoint{3.510081in}{0.920504in}}%
\pgfpathlineto{\pgfqpoint{3.510081in}{0.920504in}}%
\pgfpathlineto{\pgfqpoint{3.510081in}{0.924762in}}%
\pgfpathlineto{\pgfqpoint{3.514338in}{0.924762in}}%
\pgfpathlineto{\pgfqpoint{3.514338in}{0.920504in}}%
\pgfpathmoveto{\pgfqpoint{3.510081in}{0.924762in}}%
\pgfpathlineto{\pgfqpoint{3.510081in}{0.924762in}}%
\pgfpathlineto{\pgfqpoint{3.510081in}{0.929020in}}%
\pgfpathlineto{\pgfqpoint{3.514338in}{0.929020in}}%
\pgfpathlineto{\pgfqpoint{3.514338in}{0.924762in}}%
\pgfpathmoveto{\pgfqpoint{3.510081in}{0.929020in}}%
\pgfpathlineto{\pgfqpoint{3.510081in}{0.929020in}}%
\pgfpathlineto{\pgfqpoint{3.510081in}{0.933278in}}%
\pgfpathlineto{\pgfqpoint{3.514338in}{0.933278in}}%
\pgfpathlineto{\pgfqpoint{3.514338in}{0.929020in}}%
\pgfpathmoveto{\pgfqpoint{3.505823in}{0.937535in}}%
\pgfpathlineto{\pgfqpoint{3.505823in}{0.937535in}}%
\pgfpathlineto{\pgfqpoint{3.505823in}{0.941793in}}%
\pgfpathlineto{\pgfqpoint{3.510081in}{0.941793in}}%
\pgfpathlineto{\pgfqpoint{3.510081in}{0.937535in}}%
\pgfpathmoveto{\pgfqpoint{3.505823in}{0.941793in}}%
\pgfpathlineto{\pgfqpoint{3.505823in}{0.941793in}}%
\pgfpathlineto{\pgfqpoint{3.505823in}{0.946051in}}%
\pgfpathlineto{\pgfqpoint{3.510081in}{0.946051in}}%
\pgfpathlineto{\pgfqpoint{3.510081in}{0.941793in}}%
\pgfpathmoveto{\pgfqpoint{3.505823in}{0.946051in}}%
\pgfpathlineto{\pgfqpoint{3.505823in}{0.946051in}}%
\pgfpathlineto{\pgfqpoint{3.505823in}{0.950309in}}%
\pgfpathlineto{\pgfqpoint{3.510081in}{0.950309in}}%
\pgfpathlineto{\pgfqpoint{3.510081in}{0.946051in}}%
\pgfpathmoveto{\pgfqpoint{3.510081in}{0.933278in}}%
\pgfpathlineto{\pgfqpoint{3.510081in}{0.933278in}}%
\pgfpathlineto{\pgfqpoint{3.510081in}{0.937535in}}%
\pgfpathlineto{\pgfqpoint{3.514338in}{0.937535in}}%
\pgfpathlineto{\pgfqpoint{3.514338in}{0.933278in}}%
\pgfpathmoveto{\pgfqpoint{3.510081in}{0.937535in}}%
\pgfpathlineto{\pgfqpoint{3.510081in}{0.937535in}}%
\pgfpathlineto{\pgfqpoint{3.510081in}{0.941793in}}%
\pgfpathlineto{\pgfqpoint{3.514338in}{0.941793in}}%
\pgfpathlineto{\pgfqpoint{3.514338in}{0.937535in}}%
\pgfpathmoveto{\pgfqpoint{3.505823in}{0.950309in}}%
\pgfpathlineto{\pgfqpoint{3.505823in}{0.950309in}}%
\pgfpathlineto{\pgfqpoint{3.505823in}{0.954567in}}%
\pgfpathlineto{\pgfqpoint{3.510081in}{0.954567in}}%
\pgfpathlineto{\pgfqpoint{3.510081in}{0.950309in}}%
\pgfpathmoveto{\pgfqpoint{3.505823in}{0.954567in}}%
\pgfpathlineto{\pgfqpoint{3.505823in}{0.954567in}}%
\pgfpathlineto{\pgfqpoint{3.505823in}{0.958825in}}%
\pgfpathlineto{\pgfqpoint{3.510081in}{0.958825in}}%
\pgfpathlineto{\pgfqpoint{3.510081in}{0.954567in}}%
\pgfpathmoveto{\pgfqpoint{3.505823in}{0.958825in}}%
\pgfpathlineto{\pgfqpoint{3.505823in}{0.958825in}}%
\pgfpathlineto{\pgfqpoint{3.505823in}{0.963082in}}%
\pgfpathlineto{\pgfqpoint{3.510081in}{0.963082in}}%
\pgfpathlineto{\pgfqpoint{3.510081in}{0.958825in}}%
\pgfpathmoveto{\pgfqpoint{3.505823in}{0.963082in}}%
\pgfpathlineto{\pgfqpoint{3.505823in}{0.963082in}}%
\pgfpathlineto{\pgfqpoint{3.505823in}{0.967340in}}%
\pgfpathlineto{\pgfqpoint{3.510081in}{0.967340in}}%
\pgfpathlineto{\pgfqpoint{3.510081in}{0.963082in}}%
\pgfpathmoveto{\pgfqpoint{3.505823in}{0.967340in}}%
\pgfpathlineto{\pgfqpoint{3.505823in}{0.967340in}}%
\pgfpathlineto{\pgfqpoint{3.505823in}{0.971598in}}%
\pgfpathlineto{\pgfqpoint{3.510081in}{0.971598in}}%
\pgfpathlineto{\pgfqpoint{3.510081in}{0.967340in}}%
\pgfpathmoveto{\pgfqpoint{3.505823in}{0.971598in}}%
\pgfpathlineto{\pgfqpoint{3.505823in}{0.971598in}}%
\pgfpathlineto{\pgfqpoint{3.505823in}{0.975856in}}%
\pgfpathlineto{\pgfqpoint{3.510081in}{0.975856in}}%
\pgfpathlineto{\pgfqpoint{3.510081in}{0.971598in}}%
\pgfpathmoveto{\pgfqpoint{3.501565in}{0.980114in}}%
\pgfpathlineto{\pgfqpoint{3.501565in}{0.980114in}}%
\pgfpathlineto{\pgfqpoint{3.501565in}{0.984372in}}%
\pgfpathlineto{\pgfqpoint{3.505823in}{0.984372in}}%
\pgfpathlineto{\pgfqpoint{3.505823in}{0.980114in}}%
\pgfpathmoveto{\pgfqpoint{3.505823in}{0.975856in}}%
\pgfpathlineto{\pgfqpoint{3.505823in}{0.975856in}}%
\pgfpathlineto{\pgfqpoint{3.505823in}{0.980114in}}%
\pgfpathlineto{\pgfqpoint{3.510081in}{0.980114in}}%
\pgfpathlineto{\pgfqpoint{3.510081in}{0.975856in}}%
\pgfpathmoveto{\pgfqpoint{3.505823in}{0.980114in}}%
\pgfpathlineto{\pgfqpoint{3.505823in}{0.980114in}}%
\pgfpathlineto{\pgfqpoint{3.505823in}{0.984372in}}%
\pgfpathlineto{\pgfqpoint{3.510081in}{0.984372in}}%
\pgfpathlineto{\pgfqpoint{3.510081in}{0.980114in}}%
\pgfpathmoveto{\pgfqpoint{3.501565in}{0.984372in}}%
\pgfpathlineto{\pgfqpoint{3.501565in}{0.984372in}}%
\pgfpathlineto{\pgfqpoint{3.501565in}{0.988629in}}%
\pgfpathlineto{\pgfqpoint{3.505823in}{0.988629in}}%
\pgfpathlineto{\pgfqpoint{3.505823in}{0.984372in}}%
\pgfpathmoveto{\pgfqpoint{3.501565in}{0.988629in}}%
\pgfpathlineto{\pgfqpoint{3.501565in}{0.988629in}}%
\pgfpathlineto{\pgfqpoint{3.501565in}{0.992887in}}%
\pgfpathlineto{\pgfqpoint{3.505823in}{0.992887in}}%
\pgfpathlineto{\pgfqpoint{3.505823in}{0.988629in}}%
\pgfpathmoveto{\pgfqpoint{3.501565in}{0.992887in}}%
\pgfpathlineto{\pgfqpoint{3.501565in}{0.992887in}}%
\pgfpathlineto{\pgfqpoint{3.501565in}{0.997145in}}%
\pgfpathlineto{\pgfqpoint{3.505823in}{0.997145in}}%
\pgfpathlineto{\pgfqpoint{3.505823in}{0.992887in}}%
\pgfpathmoveto{\pgfqpoint{3.501565in}{0.997145in}}%
\pgfpathlineto{\pgfqpoint{3.501565in}{0.997145in}}%
\pgfpathlineto{\pgfqpoint{3.501565in}{1.001403in}}%
\pgfpathlineto{\pgfqpoint{3.505823in}{1.001403in}}%
\pgfpathlineto{\pgfqpoint{3.505823in}{0.997145in}}%
\pgfpathmoveto{\pgfqpoint{3.501565in}{1.001403in}}%
\pgfpathlineto{\pgfqpoint{3.501565in}{1.001403in}}%
\pgfpathlineto{\pgfqpoint{3.501565in}{1.005661in}}%
\pgfpathlineto{\pgfqpoint{3.505823in}{1.005661in}}%
\pgfpathlineto{\pgfqpoint{3.505823in}{1.001403in}}%
\pgfpathmoveto{\pgfqpoint{3.501565in}{1.005661in}}%
\pgfpathlineto{\pgfqpoint{3.501565in}{1.005661in}}%
\pgfpathlineto{\pgfqpoint{3.501565in}{1.009919in}}%
\pgfpathlineto{\pgfqpoint{3.505823in}{1.009919in}}%
\pgfpathlineto{\pgfqpoint{3.505823in}{1.005661in}}%
\pgfpathmoveto{\pgfqpoint{3.501565in}{1.009919in}}%
\pgfpathlineto{\pgfqpoint{3.501565in}{1.009919in}}%
\pgfpathlineto{\pgfqpoint{3.501565in}{1.014176in}}%
\pgfpathlineto{\pgfqpoint{3.505823in}{1.014176in}}%
\pgfpathlineto{\pgfqpoint{3.505823in}{1.009919in}}%
\pgfpathmoveto{\pgfqpoint{3.501565in}{1.014176in}}%
\pgfpathlineto{\pgfqpoint{3.501565in}{1.014176in}}%
\pgfpathlineto{\pgfqpoint{3.501565in}{1.018434in}}%
\pgfpathlineto{\pgfqpoint{3.505823in}{1.018434in}}%
\pgfpathlineto{\pgfqpoint{3.505823in}{1.014176in}}%
\pgfpathmoveto{\pgfqpoint{3.501565in}{1.018434in}}%
\pgfpathlineto{\pgfqpoint{3.501565in}{1.018434in}}%
\pgfpathlineto{\pgfqpoint{3.501565in}{1.022692in}}%
\pgfpathlineto{\pgfqpoint{3.505823in}{1.022692in}}%
\pgfpathlineto{\pgfqpoint{3.505823in}{1.018434in}}%
\pgfpathmoveto{\pgfqpoint{3.501565in}{1.022692in}}%
\pgfpathlineto{\pgfqpoint{3.501565in}{1.022692in}}%
\pgfpathlineto{\pgfqpoint{3.501565in}{1.026950in}}%
\pgfpathlineto{\pgfqpoint{3.505823in}{1.026950in}}%
\pgfpathlineto{\pgfqpoint{3.505823in}{1.022692in}}%
\pgfpathclose%
\pgfusepath{fill}%
\end{pgfscope}%
\begin{pgfscope}%
\pgfpathrectangle{\pgfqpoint{1.049063in}{0.235000in}}{\pgfqpoint{4.360000in}{4.360000in}}%
\pgfusepath{clip}%
\pgfsetrectcap%
\pgfsetroundjoin%
\pgfsetlinewidth{0.803000pt}%
\definecolor{currentstroke}{rgb}{0.690196,0.690196,0.690196}%
\pgfsetstrokecolor{currentstroke}%
\pgfsetdash{}{0pt}%
\pgfpathmoveto{\pgfqpoint{1.049063in}{0.235000in}}%
\pgfpathlineto{\pgfqpoint{1.049063in}{4.595000in}}%
\pgfusepath{stroke}%
\end{pgfscope}%
\begin{pgfscope}%
\pgfsetbuttcap%
\pgfsetroundjoin%
\definecolor{currentfill}{rgb}{0.000000,0.000000,0.000000}%
\pgfsetfillcolor{currentfill}%
\pgfsetlinewidth{0.803000pt}%
\definecolor{currentstroke}{rgb}{0.000000,0.000000,0.000000}%
\pgfsetstrokecolor{currentstroke}%
\pgfsetdash{}{0pt}%
\pgfsys@defobject{currentmarker}{\pgfqpoint{0.000000in}{-0.048611in}}{\pgfqpoint{0.000000in}{0.000000in}}{%
\pgfpathmoveto{\pgfqpoint{0.000000in}{0.000000in}}%
\pgfpathlineto{\pgfqpoint{0.000000in}{-0.048611in}}%
\pgfusepath{stroke,fill}%
}%
\begin{pgfscope}%
\pgfsys@transformshift{1.049063in}{2.415000in}%
\pgfsys@useobject{currentmarker}{}%
\end{pgfscope}%
\end{pgfscope}%
\begin{pgfscope}%
\definecolor{textcolor}{rgb}{0.000000,0.000000,0.000000}%
\pgfsetstrokecolor{textcolor}%
\pgfsetfillcolor{textcolor}%
\pgftext[x=1.049063in,y=2.317778in,,top]{\color{textcolor}\sffamily\fontsize{10.000000}{12.000000}\selectfont −20}%
\end{pgfscope}%
\begin{pgfscope}%
\pgfpathrectangle{\pgfqpoint{1.049063in}{0.235000in}}{\pgfqpoint{4.360000in}{4.360000in}}%
\pgfusepath{clip}%
\pgfsetrectcap%
\pgfsetroundjoin%
\pgfsetlinewidth{0.803000pt}%
\definecolor{currentstroke}{rgb}{0.690196,0.690196,0.690196}%
\pgfsetstrokecolor{currentstroke}%
\pgfsetdash{}{0pt}%
\pgfpathmoveto{\pgfqpoint{1.594062in}{0.235000in}}%
\pgfpathlineto{\pgfqpoint{1.594062in}{4.595000in}}%
\pgfusepath{stroke}%
\end{pgfscope}%
\begin{pgfscope}%
\pgfsetbuttcap%
\pgfsetroundjoin%
\definecolor{currentfill}{rgb}{0.000000,0.000000,0.000000}%
\pgfsetfillcolor{currentfill}%
\pgfsetlinewidth{0.803000pt}%
\definecolor{currentstroke}{rgb}{0.000000,0.000000,0.000000}%
\pgfsetstrokecolor{currentstroke}%
\pgfsetdash{}{0pt}%
\pgfsys@defobject{currentmarker}{\pgfqpoint{0.000000in}{-0.048611in}}{\pgfqpoint{0.000000in}{0.000000in}}{%
\pgfpathmoveto{\pgfqpoint{0.000000in}{0.000000in}}%
\pgfpathlineto{\pgfqpoint{0.000000in}{-0.048611in}}%
\pgfusepath{stroke,fill}%
}%
\begin{pgfscope}%
\pgfsys@transformshift{1.594062in}{2.415000in}%
\pgfsys@useobject{currentmarker}{}%
\end{pgfscope}%
\end{pgfscope}%
\begin{pgfscope}%
\definecolor{textcolor}{rgb}{0.000000,0.000000,0.000000}%
\pgfsetstrokecolor{textcolor}%
\pgfsetfillcolor{textcolor}%
\pgftext[x=1.594062in,y=2.317778in,,top]{\color{textcolor}\sffamily\fontsize{10.000000}{12.000000}\selectfont −15}%
\end{pgfscope}%
\begin{pgfscope}%
\pgfpathrectangle{\pgfqpoint{1.049063in}{0.235000in}}{\pgfqpoint{4.360000in}{4.360000in}}%
\pgfusepath{clip}%
\pgfsetrectcap%
\pgfsetroundjoin%
\pgfsetlinewidth{0.803000pt}%
\definecolor{currentstroke}{rgb}{0.690196,0.690196,0.690196}%
\pgfsetstrokecolor{currentstroke}%
\pgfsetdash{}{0pt}%
\pgfpathmoveto{\pgfqpoint{2.139063in}{0.235000in}}%
\pgfpathlineto{\pgfqpoint{2.139063in}{4.595000in}}%
\pgfusepath{stroke}%
\end{pgfscope}%
\begin{pgfscope}%
\pgfsetbuttcap%
\pgfsetroundjoin%
\definecolor{currentfill}{rgb}{0.000000,0.000000,0.000000}%
\pgfsetfillcolor{currentfill}%
\pgfsetlinewidth{0.803000pt}%
\definecolor{currentstroke}{rgb}{0.000000,0.000000,0.000000}%
\pgfsetstrokecolor{currentstroke}%
\pgfsetdash{}{0pt}%
\pgfsys@defobject{currentmarker}{\pgfqpoint{0.000000in}{-0.048611in}}{\pgfqpoint{0.000000in}{0.000000in}}{%
\pgfpathmoveto{\pgfqpoint{0.000000in}{0.000000in}}%
\pgfpathlineto{\pgfqpoint{0.000000in}{-0.048611in}}%
\pgfusepath{stroke,fill}%
}%
\begin{pgfscope}%
\pgfsys@transformshift{2.139063in}{2.415000in}%
\pgfsys@useobject{currentmarker}{}%
\end{pgfscope}%
\end{pgfscope}%
\begin{pgfscope}%
\definecolor{textcolor}{rgb}{0.000000,0.000000,0.000000}%
\pgfsetstrokecolor{textcolor}%
\pgfsetfillcolor{textcolor}%
\pgftext[x=2.139063in,y=2.317778in,,top]{\color{textcolor}\sffamily\fontsize{10.000000}{12.000000}\selectfont −10}%
\end{pgfscope}%
\begin{pgfscope}%
\pgfpathrectangle{\pgfqpoint{1.049063in}{0.235000in}}{\pgfqpoint{4.360000in}{4.360000in}}%
\pgfusepath{clip}%
\pgfsetrectcap%
\pgfsetroundjoin%
\pgfsetlinewidth{0.803000pt}%
\definecolor{currentstroke}{rgb}{0.690196,0.690196,0.690196}%
\pgfsetstrokecolor{currentstroke}%
\pgfsetdash{}{0pt}%
\pgfpathmoveto{\pgfqpoint{2.684063in}{0.235000in}}%
\pgfpathlineto{\pgfqpoint{2.684063in}{4.595000in}}%
\pgfusepath{stroke}%
\end{pgfscope}%
\begin{pgfscope}%
\pgfsetbuttcap%
\pgfsetroundjoin%
\definecolor{currentfill}{rgb}{0.000000,0.000000,0.000000}%
\pgfsetfillcolor{currentfill}%
\pgfsetlinewidth{0.803000pt}%
\definecolor{currentstroke}{rgb}{0.000000,0.000000,0.000000}%
\pgfsetstrokecolor{currentstroke}%
\pgfsetdash{}{0pt}%
\pgfsys@defobject{currentmarker}{\pgfqpoint{0.000000in}{-0.048611in}}{\pgfqpoint{0.000000in}{0.000000in}}{%
\pgfpathmoveto{\pgfqpoint{0.000000in}{0.000000in}}%
\pgfpathlineto{\pgfqpoint{0.000000in}{-0.048611in}}%
\pgfusepath{stroke,fill}%
}%
\begin{pgfscope}%
\pgfsys@transformshift{2.684063in}{2.415000in}%
\pgfsys@useobject{currentmarker}{}%
\end{pgfscope}%
\end{pgfscope}%
\begin{pgfscope}%
\definecolor{textcolor}{rgb}{0.000000,0.000000,0.000000}%
\pgfsetstrokecolor{textcolor}%
\pgfsetfillcolor{textcolor}%
\pgftext[x=2.684063in,y=2.317778in,,top]{\color{textcolor}\sffamily\fontsize{10.000000}{12.000000}\selectfont −5}%
\end{pgfscope}%
\begin{pgfscope}%
\pgfpathrectangle{\pgfqpoint{1.049063in}{0.235000in}}{\pgfqpoint{4.360000in}{4.360000in}}%
\pgfusepath{clip}%
\pgfsetrectcap%
\pgfsetroundjoin%
\pgfsetlinewidth{0.803000pt}%
\definecolor{currentstroke}{rgb}{0.690196,0.690196,0.690196}%
\pgfsetstrokecolor{currentstroke}%
\pgfsetdash{}{0pt}%
\pgfpathmoveto{\pgfqpoint{3.229062in}{0.235000in}}%
\pgfpathlineto{\pgfqpoint{3.229062in}{4.595000in}}%
\pgfusepath{stroke}%
\end{pgfscope}%
\begin{pgfscope}%
\pgfsetbuttcap%
\pgfsetroundjoin%
\definecolor{currentfill}{rgb}{0.000000,0.000000,0.000000}%
\pgfsetfillcolor{currentfill}%
\pgfsetlinewidth{0.803000pt}%
\definecolor{currentstroke}{rgb}{0.000000,0.000000,0.000000}%
\pgfsetstrokecolor{currentstroke}%
\pgfsetdash{}{0pt}%
\pgfsys@defobject{currentmarker}{\pgfqpoint{0.000000in}{-0.048611in}}{\pgfqpoint{0.000000in}{0.000000in}}{%
\pgfpathmoveto{\pgfqpoint{0.000000in}{0.000000in}}%
\pgfpathlineto{\pgfqpoint{0.000000in}{-0.048611in}}%
\pgfusepath{stroke,fill}%
}%
\begin{pgfscope}%
\pgfsys@transformshift{3.229062in}{2.415000in}%
\pgfsys@useobject{currentmarker}{}%
\end{pgfscope}%
\end{pgfscope}%
\begin{pgfscope}%
\definecolor{textcolor}{rgb}{0.000000,0.000000,0.000000}%
\pgfsetstrokecolor{textcolor}%
\pgfsetfillcolor{textcolor}%
\pgftext[x=3.229062in,y=2.317778in,,top]{\color{textcolor}\sffamily\fontsize{10.000000}{12.000000}\selectfont 0}%
\end{pgfscope}%
\begin{pgfscope}%
\pgfpathrectangle{\pgfqpoint{1.049063in}{0.235000in}}{\pgfqpoint{4.360000in}{4.360000in}}%
\pgfusepath{clip}%
\pgfsetrectcap%
\pgfsetroundjoin%
\pgfsetlinewidth{0.803000pt}%
\definecolor{currentstroke}{rgb}{0.690196,0.690196,0.690196}%
\pgfsetstrokecolor{currentstroke}%
\pgfsetdash{}{0pt}%
\pgfpathmoveto{\pgfqpoint{3.774062in}{0.235000in}}%
\pgfpathlineto{\pgfqpoint{3.774062in}{4.595000in}}%
\pgfusepath{stroke}%
\end{pgfscope}%
\begin{pgfscope}%
\pgfsetbuttcap%
\pgfsetroundjoin%
\definecolor{currentfill}{rgb}{0.000000,0.000000,0.000000}%
\pgfsetfillcolor{currentfill}%
\pgfsetlinewidth{0.803000pt}%
\definecolor{currentstroke}{rgb}{0.000000,0.000000,0.000000}%
\pgfsetstrokecolor{currentstroke}%
\pgfsetdash{}{0pt}%
\pgfsys@defobject{currentmarker}{\pgfqpoint{0.000000in}{-0.048611in}}{\pgfqpoint{0.000000in}{0.000000in}}{%
\pgfpathmoveto{\pgfqpoint{0.000000in}{0.000000in}}%
\pgfpathlineto{\pgfqpoint{0.000000in}{-0.048611in}}%
\pgfusepath{stroke,fill}%
}%
\begin{pgfscope}%
\pgfsys@transformshift{3.774062in}{2.415000in}%
\pgfsys@useobject{currentmarker}{}%
\end{pgfscope}%
\end{pgfscope}%
\begin{pgfscope}%
\definecolor{textcolor}{rgb}{0.000000,0.000000,0.000000}%
\pgfsetstrokecolor{textcolor}%
\pgfsetfillcolor{textcolor}%
\pgftext[x=3.774062in,y=2.317778in,,top]{\color{textcolor}\sffamily\fontsize{10.000000}{12.000000}\selectfont 5}%
\end{pgfscope}%
\begin{pgfscope}%
\pgfpathrectangle{\pgfqpoint{1.049063in}{0.235000in}}{\pgfqpoint{4.360000in}{4.360000in}}%
\pgfusepath{clip}%
\pgfsetrectcap%
\pgfsetroundjoin%
\pgfsetlinewidth{0.803000pt}%
\definecolor{currentstroke}{rgb}{0.690196,0.690196,0.690196}%
\pgfsetstrokecolor{currentstroke}%
\pgfsetdash{}{0pt}%
\pgfpathmoveto{\pgfqpoint{4.319063in}{0.235000in}}%
\pgfpathlineto{\pgfqpoint{4.319063in}{4.595000in}}%
\pgfusepath{stroke}%
\end{pgfscope}%
\begin{pgfscope}%
\pgfsetbuttcap%
\pgfsetroundjoin%
\definecolor{currentfill}{rgb}{0.000000,0.000000,0.000000}%
\pgfsetfillcolor{currentfill}%
\pgfsetlinewidth{0.803000pt}%
\definecolor{currentstroke}{rgb}{0.000000,0.000000,0.000000}%
\pgfsetstrokecolor{currentstroke}%
\pgfsetdash{}{0pt}%
\pgfsys@defobject{currentmarker}{\pgfqpoint{0.000000in}{-0.048611in}}{\pgfqpoint{0.000000in}{0.000000in}}{%
\pgfpathmoveto{\pgfqpoint{0.000000in}{0.000000in}}%
\pgfpathlineto{\pgfqpoint{0.000000in}{-0.048611in}}%
\pgfusepath{stroke,fill}%
}%
\begin{pgfscope}%
\pgfsys@transformshift{4.319063in}{2.415000in}%
\pgfsys@useobject{currentmarker}{}%
\end{pgfscope}%
\end{pgfscope}%
\begin{pgfscope}%
\definecolor{textcolor}{rgb}{0.000000,0.000000,0.000000}%
\pgfsetstrokecolor{textcolor}%
\pgfsetfillcolor{textcolor}%
\pgftext[x=4.319063in,y=2.317778in,,top]{\color{textcolor}\sffamily\fontsize{10.000000}{12.000000}\selectfont 10}%
\end{pgfscope}%
\begin{pgfscope}%
\pgfpathrectangle{\pgfqpoint{1.049063in}{0.235000in}}{\pgfqpoint{4.360000in}{4.360000in}}%
\pgfusepath{clip}%
\pgfsetrectcap%
\pgfsetroundjoin%
\pgfsetlinewidth{0.803000pt}%
\definecolor{currentstroke}{rgb}{0.690196,0.690196,0.690196}%
\pgfsetstrokecolor{currentstroke}%
\pgfsetdash{}{0pt}%
\pgfpathmoveto{\pgfqpoint{4.864063in}{0.235000in}}%
\pgfpathlineto{\pgfqpoint{4.864063in}{4.595000in}}%
\pgfusepath{stroke}%
\end{pgfscope}%
\begin{pgfscope}%
\pgfsetbuttcap%
\pgfsetroundjoin%
\definecolor{currentfill}{rgb}{0.000000,0.000000,0.000000}%
\pgfsetfillcolor{currentfill}%
\pgfsetlinewidth{0.803000pt}%
\definecolor{currentstroke}{rgb}{0.000000,0.000000,0.000000}%
\pgfsetstrokecolor{currentstroke}%
\pgfsetdash{}{0pt}%
\pgfsys@defobject{currentmarker}{\pgfqpoint{0.000000in}{-0.048611in}}{\pgfqpoint{0.000000in}{0.000000in}}{%
\pgfpathmoveto{\pgfqpoint{0.000000in}{0.000000in}}%
\pgfpathlineto{\pgfqpoint{0.000000in}{-0.048611in}}%
\pgfusepath{stroke,fill}%
}%
\begin{pgfscope}%
\pgfsys@transformshift{4.864063in}{2.415000in}%
\pgfsys@useobject{currentmarker}{}%
\end{pgfscope}%
\end{pgfscope}%
\begin{pgfscope}%
\definecolor{textcolor}{rgb}{0.000000,0.000000,0.000000}%
\pgfsetstrokecolor{textcolor}%
\pgfsetfillcolor{textcolor}%
\pgftext[x=4.864063in,y=2.317778in,,top]{\color{textcolor}\sffamily\fontsize{10.000000}{12.000000}\selectfont 15}%
\end{pgfscope}%
\begin{pgfscope}%
\pgfpathrectangle{\pgfqpoint{1.049063in}{0.235000in}}{\pgfqpoint{4.360000in}{4.360000in}}%
\pgfusepath{clip}%
\pgfsetrectcap%
\pgfsetroundjoin%
\pgfsetlinewidth{0.803000pt}%
\definecolor{currentstroke}{rgb}{0.690196,0.690196,0.690196}%
\pgfsetstrokecolor{currentstroke}%
\pgfsetdash{}{0pt}%
\pgfpathmoveto{\pgfqpoint{5.409063in}{0.235000in}}%
\pgfpathlineto{\pgfqpoint{5.409063in}{4.595000in}}%
\pgfusepath{stroke}%
\end{pgfscope}%
\begin{pgfscope}%
\pgfsetbuttcap%
\pgfsetroundjoin%
\definecolor{currentfill}{rgb}{0.000000,0.000000,0.000000}%
\pgfsetfillcolor{currentfill}%
\pgfsetlinewidth{0.803000pt}%
\definecolor{currentstroke}{rgb}{0.000000,0.000000,0.000000}%
\pgfsetstrokecolor{currentstroke}%
\pgfsetdash{}{0pt}%
\pgfsys@defobject{currentmarker}{\pgfqpoint{0.000000in}{-0.048611in}}{\pgfqpoint{0.000000in}{0.000000in}}{%
\pgfpathmoveto{\pgfqpoint{0.000000in}{0.000000in}}%
\pgfpathlineto{\pgfqpoint{0.000000in}{-0.048611in}}%
\pgfusepath{stroke,fill}%
}%
\begin{pgfscope}%
\pgfsys@transformshift{5.409063in}{2.415000in}%
\pgfsys@useobject{currentmarker}{}%
\end{pgfscope}%
\end{pgfscope}%
\begin{pgfscope}%
\definecolor{textcolor}{rgb}{0.000000,0.000000,0.000000}%
\pgfsetstrokecolor{textcolor}%
\pgfsetfillcolor{textcolor}%
\pgftext[x=5.409063in,y=2.317778in,,top]{\color{textcolor}\sffamily\fontsize{10.000000}{12.000000}\selectfont 20}%
\end{pgfscope}%
\begin{pgfscope}%
\definecolor{textcolor}{rgb}{0.000000,0.000000,0.000000}%
\pgfsetstrokecolor{textcolor}%
\pgfsetfillcolor{textcolor}%
\pgftext[x=5.409063in,y=2.127809in,,top]{\color{textcolor}\sffamily\fontsize{10.000000}{12.000000}\selectfont x}%
\end{pgfscope}%
\begin{pgfscope}%
\pgfpathrectangle{\pgfqpoint{1.049063in}{0.235000in}}{\pgfqpoint{4.360000in}{4.360000in}}%
\pgfusepath{clip}%
\pgfsetrectcap%
\pgfsetroundjoin%
\pgfsetlinewidth{0.803000pt}%
\definecolor{currentstroke}{rgb}{0.690196,0.690196,0.690196}%
\pgfsetstrokecolor{currentstroke}%
\pgfsetdash{}{0pt}%
\pgfpathmoveto{\pgfqpoint{1.049063in}{0.235000in}}%
\pgfpathlineto{\pgfqpoint{5.409063in}{0.235000in}}%
\pgfusepath{stroke}%
\end{pgfscope}%
\begin{pgfscope}%
\pgfsetbuttcap%
\pgfsetroundjoin%
\definecolor{currentfill}{rgb}{0.000000,0.000000,0.000000}%
\pgfsetfillcolor{currentfill}%
\pgfsetlinewidth{0.803000pt}%
\definecolor{currentstroke}{rgb}{0.000000,0.000000,0.000000}%
\pgfsetstrokecolor{currentstroke}%
\pgfsetdash{}{0pt}%
\pgfsys@defobject{currentmarker}{\pgfqpoint{-0.048611in}{0.000000in}}{\pgfqpoint{-0.000000in}{0.000000in}}{%
\pgfpathmoveto{\pgfqpoint{-0.000000in}{0.000000in}}%
\pgfpathlineto{\pgfqpoint{-0.048611in}{0.000000in}}%
\pgfusepath{stroke,fill}%
}%
\begin{pgfscope}%
\pgfsys@transformshift{3.229062in}{0.235000in}%
\pgfsys@useobject{currentmarker}{}%
\end{pgfscope}%
\end{pgfscope}%
\begin{pgfscope}%
\definecolor{textcolor}{rgb}{0.000000,0.000000,0.000000}%
\pgfsetstrokecolor{textcolor}%
\pgfsetfillcolor{textcolor}%
\pgftext[x=2.838736in, y=0.182238in, left, base]{\color{textcolor}\sffamily\fontsize{10.000000}{12.000000}\selectfont −20}%
\end{pgfscope}%
\begin{pgfscope}%
\pgfpathrectangle{\pgfqpoint{1.049063in}{0.235000in}}{\pgfqpoint{4.360000in}{4.360000in}}%
\pgfusepath{clip}%
\pgfsetrectcap%
\pgfsetroundjoin%
\pgfsetlinewidth{0.803000pt}%
\definecolor{currentstroke}{rgb}{0.690196,0.690196,0.690196}%
\pgfsetstrokecolor{currentstroke}%
\pgfsetdash{}{0pt}%
\pgfpathmoveto{\pgfqpoint{1.049063in}{0.780000in}}%
\pgfpathlineto{\pgfqpoint{5.409063in}{0.780000in}}%
\pgfusepath{stroke}%
\end{pgfscope}%
\begin{pgfscope}%
\pgfsetbuttcap%
\pgfsetroundjoin%
\definecolor{currentfill}{rgb}{0.000000,0.000000,0.000000}%
\pgfsetfillcolor{currentfill}%
\pgfsetlinewidth{0.803000pt}%
\definecolor{currentstroke}{rgb}{0.000000,0.000000,0.000000}%
\pgfsetstrokecolor{currentstroke}%
\pgfsetdash{}{0pt}%
\pgfsys@defobject{currentmarker}{\pgfqpoint{-0.048611in}{0.000000in}}{\pgfqpoint{-0.000000in}{0.000000in}}{%
\pgfpathmoveto{\pgfqpoint{-0.000000in}{0.000000in}}%
\pgfpathlineto{\pgfqpoint{-0.048611in}{0.000000in}}%
\pgfusepath{stroke,fill}%
}%
\begin{pgfscope}%
\pgfsys@transformshift{3.229062in}{0.780000in}%
\pgfsys@useobject{currentmarker}{}%
\end{pgfscope}%
\end{pgfscope}%
\begin{pgfscope}%
\definecolor{textcolor}{rgb}{0.000000,0.000000,0.000000}%
\pgfsetstrokecolor{textcolor}%
\pgfsetfillcolor{textcolor}%
\pgftext[x=2.838736in, y=0.727238in, left, base]{\color{textcolor}\sffamily\fontsize{10.000000}{12.000000}\selectfont −15}%
\end{pgfscope}%
\begin{pgfscope}%
\pgfpathrectangle{\pgfqpoint{1.049063in}{0.235000in}}{\pgfqpoint{4.360000in}{4.360000in}}%
\pgfusepath{clip}%
\pgfsetrectcap%
\pgfsetroundjoin%
\pgfsetlinewidth{0.803000pt}%
\definecolor{currentstroke}{rgb}{0.690196,0.690196,0.690196}%
\pgfsetstrokecolor{currentstroke}%
\pgfsetdash{}{0pt}%
\pgfpathmoveto{\pgfqpoint{1.049063in}{1.325000in}}%
\pgfpathlineto{\pgfqpoint{5.409063in}{1.325000in}}%
\pgfusepath{stroke}%
\end{pgfscope}%
\begin{pgfscope}%
\pgfsetbuttcap%
\pgfsetroundjoin%
\definecolor{currentfill}{rgb}{0.000000,0.000000,0.000000}%
\pgfsetfillcolor{currentfill}%
\pgfsetlinewidth{0.803000pt}%
\definecolor{currentstroke}{rgb}{0.000000,0.000000,0.000000}%
\pgfsetstrokecolor{currentstroke}%
\pgfsetdash{}{0pt}%
\pgfsys@defobject{currentmarker}{\pgfqpoint{-0.048611in}{0.000000in}}{\pgfqpoint{-0.000000in}{0.000000in}}{%
\pgfpathmoveto{\pgfqpoint{-0.000000in}{0.000000in}}%
\pgfpathlineto{\pgfqpoint{-0.048611in}{0.000000in}}%
\pgfusepath{stroke,fill}%
}%
\begin{pgfscope}%
\pgfsys@transformshift{3.229062in}{1.325000in}%
\pgfsys@useobject{currentmarker}{}%
\end{pgfscope}%
\end{pgfscope}%
\begin{pgfscope}%
\definecolor{textcolor}{rgb}{0.000000,0.000000,0.000000}%
\pgfsetstrokecolor{textcolor}%
\pgfsetfillcolor{textcolor}%
\pgftext[x=2.838736in, y=1.272238in, left, base]{\color{textcolor}\sffamily\fontsize{10.000000}{12.000000}\selectfont −10}%
\end{pgfscope}%
\begin{pgfscope}%
\pgfpathrectangle{\pgfqpoint{1.049063in}{0.235000in}}{\pgfqpoint{4.360000in}{4.360000in}}%
\pgfusepath{clip}%
\pgfsetrectcap%
\pgfsetroundjoin%
\pgfsetlinewidth{0.803000pt}%
\definecolor{currentstroke}{rgb}{0.690196,0.690196,0.690196}%
\pgfsetstrokecolor{currentstroke}%
\pgfsetdash{}{0pt}%
\pgfpathmoveto{\pgfqpoint{1.049063in}{1.870000in}}%
\pgfpathlineto{\pgfqpoint{5.409063in}{1.870000in}}%
\pgfusepath{stroke}%
\end{pgfscope}%
\begin{pgfscope}%
\pgfsetbuttcap%
\pgfsetroundjoin%
\definecolor{currentfill}{rgb}{0.000000,0.000000,0.000000}%
\pgfsetfillcolor{currentfill}%
\pgfsetlinewidth{0.803000pt}%
\definecolor{currentstroke}{rgb}{0.000000,0.000000,0.000000}%
\pgfsetstrokecolor{currentstroke}%
\pgfsetdash{}{0pt}%
\pgfsys@defobject{currentmarker}{\pgfqpoint{-0.048611in}{0.000000in}}{\pgfqpoint{-0.000000in}{0.000000in}}{%
\pgfpathmoveto{\pgfqpoint{-0.000000in}{0.000000in}}%
\pgfpathlineto{\pgfqpoint{-0.048611in}{0.000000in}}%
\pgfusepath{stroke,fill}%
}%
\begin{pgfscope}%
\pgfsys@transformshift{3.229062in}{1.870000in}%
\pgfsys@useobject{currentmarker}{}%
\end{pgfscope}%
\end{pgfscope}%
\begin{pgfscope}%
\definecolor{textcolor}{rgb}{0.000000,0.000000,0.000000}%
\pgfsetstrokecolor{textcolor}%
\pgfsetfillcolor{textcolor}%
\pgftext[x=2.927101in, y=1.817238in, left, base]{\color{textcolor}\sffamily\fontsize{10.000000}{12.000000}\selectfont −5}%
\end{pgfscope}%
\begin{pgfscope}%
\pgfpathrectangle{\pgfqpoint{1.049063in}{0.235000in}}{\pgfqpoint{4.360000in}{4.360000in}}%
\pgfusepath{clip}%
\pgfsetrectcap%
\pgfsetroundjoin%
\pgfsetlinewidth{0.803000pt}%
\definecolor{currentstroke}{rgb}{0.690196,0.690196,0.690196}%
\pgfsetstrokecolor{currentstroke}%
\pgfsetdash{}{0pt}%
\pgfpathmoveto{\pgfqpoint{1.049063in}{2.415000in}}%
\pgfpathlineto{\pgfqpoint{5.409063in}{2.415000in}}%
\pgfusepath{stroke}%
\end{pgfscope}%
\begin{pgfscope}%
\pgfsetbuttcap%
\pgfsetroundjoin%
\definecolor{currentfill}{rgb}{0.000000,0.000000,0.000000}%
\pgfsetfillcolor{currentfill}%
\pgfsetlinewidth{0.803000pt}%
\definecolor{currentstroke}{rgb}{0.000000,0.000000,0.000000}%
\pgfsetstrokecolor{currentstroke}%
\pgfsetdash{}{0pt}%
\pgfsys@defobject{currentmarker}{\pgfqpoint{-0.048611in}{0.000000in}}{\pgfqpoint{-0.000000in}{0.000000in}}{%
\pgfpathmoveto{\pgfqpoint{-0.000000in}{0.000000in}}%
\pgfpathlineto{\pgfqpoint{-0.048611in}{0.000000in}}%
\pgfusepath{stroke,fill}%
}%
\begin{pgfscope}%
\pgfsys@transformshift{3.229062in}{2.415000in}%
\pgfsys@useobject{currentmarker}{}%
\end{pgfscope}%
\end{pgfscope}%
\begin{pgfscope}%
\definecolor{textcolor}{rgb}{0.000000,0.000000,0.000000}%
\pgfsetstrokecolor{textcolor}%
\pgfsetfillcolor{textcolor}%
\pgftext[x=3.043475in, y=2.362238in, left, base]{\color{textcolor}\sffamily\fontsize{10.000000}{12.000000}\selectfont 0}%
\end{pgfscope}%
\begin{pgfscope}%
\pgfpathrectangle{\pgfqpoint{1.049063in}{0.235000in}}{\pgfqpoint{4.360000in}{4.360000in}}%
\pgfusepath{clip}%
\pgfsetrectcap%
\pgfsetroundjoin%
\pgfsetlinewidth{0.803000pt}%
\definecolor{currentstroke}{rgb}{0.690196,0.690196,0.690196}%
\pgfsetstrokecolor{currentstroke}%
\pgfsetdash{}{0pt}%
\pgfpathmoveto{\pgfqpoint{1.049063in}{2.960000in}}%
\pgfpathlineto{\pgfqpoint{5.409063in}{2.960000in}}%
\pgfusepath{stroke}%
\end{pgfscope}%
\begin{pgfscope}%
\pgfsetbuttcap%
\pgfsetroundjoin%
\definecolor{currentfill}{rgb}{0.000000,0.000000,0.000000}%
\pgfsetfillcolor{currentfill}%
\pgfsetlinewidth{0.803000pt}%
\definecolor{currentstroke}{rgb}{0.000000,0.000000,0.000000}%
\pgfsetstrokecolor{currentstroke}%
\pgfsetdash{}{0pt}%
\pgfsys@defobject{currentmarker}{\pgfqpoint{-0.048611in}{0.000000in}}{\pgfqpoint{-0.000000in}{0.000000in}}{%
\pgfpathmoveto{\pgfqpoint{-0.000000in}{0.000000in}}%
\pgfpathlineto{\pgfqpoint{-0.048611in}{0.000000in}}%
\pgfusepath{stroke,fill}%
}%
\begin{pgfscope}%
\pgfsys@transformshift{3.229062in}{2.960000in}%
\pgfsys@useobject{currentmarker}{}%
\end{pgfscope}%
\end{pgfscope}%
\begin{pgfscope}%
\definecolor{textcolor}{rgb}{0.000000,0.000000,0.000000}%
\pgfsetstrokecolor{textcolor}%
\pgfsetfillcolor{textcolor}%
\pgftext[x=3.043475in, y=2.907238in, left, base]{\color{textcolor}\sffamily\fontsize{10.000000}{12.000000}\selectfont 5}%
\end{pgfscope}%
\begin{pgfscope}%
\pgfpathrectangle{\pgfqpoint{1.049063in}{0.235000in}}{\pgfqpoint{4.360000in}{4.360000in}}%
\pgfusepath{clip}%
\pgfsetrectcap%
\pgfsetroundjoin%
\pgfsetlinewidth{0.803000pt}%
\definecolor{currentstroke}{rgb}{0.690196,0.690196,0.690196}%
\pgfsetstrokecolor{currentstroke}%
\pgfsetdash{}{0pt}%
\pgfpathmoveto{\pgfqpoint{1.049063in}{3.505000in}}%
\pgfpathlineto{\pgfqpoint{5.409063in}{3.505000in}}%
\pgfusepath{stroke}%
\end{pgfscope}%
\begin{pgfscope}%
\pgfsetbuttcap%
\pgfsetroundjoin%
\definecolor{currentfill}{rgb}{0.000000,0.000000,0.000000}%
\pgfsetfillcolor{currentfill}%
\pgfsetlinewidth{0.803000pt}%
\definecolor{currentstroke}{rgb}{0.000000,0.000000,0.000000}%
\pgfsetstrokecolor{currentstroke}%
\pgfsetdash{}{0pt}%
\pgfsys@defobject{currentmarker}{\pgfqpoint{-0.048611in}{0.000000in}}{\pgfqpoint{-0.000000in}{0.000000in}}{%
\pgfpathmoveto{\pgfqpoint{-0.000000in}{0.000000in}}%
\pgfpathlineto{\pgfqpoint{-0.048611in}{0.000000in}}%
\pgfusepath{stroke,fill}%
}%
\begin{pgfscope}%
\pgfsys@transformshift{3.229062in}{3.505000in}%
\pgfsys@useobject{currentmarker}{}%
\end{pgfscope}%
\end{pgfscope}%
\begin{pgfscope}%
\definecolor{textcolor}{rgb}{0.000000,0.000000,0.000000}%
\pgfsetstrokecolor{textcolor}%
\pgfsetfillcolor{textcolor}%
\pgftext[x=2.955110in, y=3.452238in, left, base]{\color{textcolor}\sffamily\fontsize{10.000000}{12.000000}\selectfont 10}%
\end{pgfscope}%
\begin{pgfscope}%
\pgfpathrectangle{\pgfqpoint{1.049063in}{0.235000in}}{\pgfqpoint{4.360000in}{4.360000in}}%
\pgfusepath{clip}%
\pgfsetrectcap%
\pgfsetroundjoin%
\pgfsetlinewidth{0.803000pt}%
\definecolor{currentstroke}{rgb}{0.690196,0.690196,0.690196}%
\pgfsetstrokecolor{currentstroke}%
\pgfsetdash{}{0pt}%
\pgfpathmoveto{\pgfqpoint{1.049063in}{4.050000in}}%
\pgfpathlineto{\pgfqpoint{5.409063in}{4.050000in}}%
\pgfusepath{stroke}%
\end{pgfscope}%
\begin{pgfscope}%
\pgfsetbuttcap%
\pgfsetroundjoin%
\definecolor{currentfill}{rgb}{0.000000,0.000000,0.000000}%
\pgfsetfillcolor{currentfill}%
\pgfsetlinewidth{0.803000pt}%
\definecolor{currentstroke}{rgb}{0.000000,0.000000,0.000000}%
\pgfsetstrokecolor{currentstroke}%
\pgfsetdash{}{0pt}%
\pgfsys@defobject{currentmarker}{\pgfqpoint{-0.048611in}{0.000000in}}{\pgfqpoint{-0.000000in}{0.000000in}}{%
\pgfpathmoveto{\pgfqpoint{-0.000000in}{0.000000in}}%
\pgfpathlineto{\pgfqpoint{-0.048611in}{0.000000in}}%
\pgfusepath{stroke,fill}%
}%
\begin{pgfscope}%
\pgfsys@transformshift{3.229062in}{4.050000in}%
\pgfsys@useobject{currentmarker}{}%
\end{pgfscope}%
\end{pgfscope}%
\begin{pgfscope}%
\definecolor{textcolor}{rgb}{0.000000,0.000000,0.000000}%
\pgfsetstrokecolor{textcolor}%
\pgfsetfillcolor{textcolor}%
\pgftext[x=2.955110in, y=3.997238in, left, base]{\color{textcolor}\sffamily\fontsize{10.000000}{12.000000}\selectfont 15}%
\end{pgfscope}%
\begin{pgfscope}%
\pgfpathrectangle{\pgfqpoint{1.049063in}{0.235000in}}{\pgfqpoint{4.360000in}{4.360000in}}%
\pgfusepath{clip}%
\pgfsetrectcap%
\pgfsetroundjoin%
\pgfsetlinewidth{0.803000pt}%
\definecolor{currentstroke}{rgb}{0.690196,0.690196,0.690196}%
\pgfsetstrokecolor{currentstroke}%
\pgfsetdash{}{0pt}%
\pgfpathmoveto{\pgfqpoint{1.049063in}{4.595000in}}%
\pgfpathlineto{\pgfqpoint{5.409063in}{4.595000in}}%
\pgfusepath{stroke}%
\end{pgfscope}%
\begin{pgfscope}%
\pgfsetbuttcap%
\pgfsetroundjoin%
\definecolor{currentfill}{rgb}{0.000000,0.000000,0.000000}%
\pgfsetfillcolor{currentfill}%
\pgfsetlinewidth{0.803000pt}%
\definecolor{currentstroke}{rgb}{0.000000,0.000000,0.000000}%
\pgfsetstrokecolor{currentstroke}%
\pgfsetdash{}{0pt}%
\pgfsys@defobject{currentmarker}{\pgfqpoint{-0.048611in}{0.000000in}}{\pgfqpoint{-0.000000in}{0.000000in}}{%
\pgfpathmoveto{\pgfqpoint{-0.000000in}{0.000000in}}%
\pgfpathlineto{\pgfqpoint{-0.048611in}{0.000000in}}%
\pgfusepath{stroke,fill}%
}%
\begin{pgfscope}%
\pgfsys@transformshift{3.229062in}{4.595000in}%
\pgfsys@useobject{currentmarker}{}%
\end{pgfscope}%
\end{pgfscope}%
\begin{pgfscope}%
\definecolor{textcolor}{rgb}{0.000000,0.000000,0.000000}%
\pgfsetstrokecolor{textcolor}%
\pgfsetfillcolor{textcolor}%
\pgftext[x=2.955110in, y=4.542238in, left, base]{\color{textcolor}\sffamily\fontsize{10.000000}{12.000000}\selectfont 20}%
\end{pgfscope}%
\begin{pgfscope}%
\definecolor{textcolor}{rgb}{0.000000,0.000000,0.000000}%
\pgfsetstrokecolor{textcolor}%
\pgfsetfillcolor{textcolor}%
\pgftext[x=2.783180in,y=4.595000in,,bottom,rotate=90.000000]{\color{textcolor}\sffamily\fontsize{10.000000}{12.000000}\selectfont y}%
\end{pgfscope}%
\begin{pgfscope}%
\pgfsetrectcap%
\pgfsetmiterjoin%
\pgfsetlinewidth{0.803000pt}%
\definecolor{currentstroke}{rgb}{0.000000,0.000000,0.000000}%
\pgfsetstrokecolor{currentstroke}%
\pgfsetdash{}{0pt}%
\pgfpathmoveto{\pgfqpoint{3.229062in}{0.235000in}}%
\pgfpathlineto{\pgfqpoint{3.229062in}{4.595000in}}%
\pgfusepath{stroke}%
\end{pgfscope}%
\begin{pgfscope}%
\pgfsetrectcap%
\pgfsetmiterjoin%
\pgfsetlinewidth{0.000000pt}%
\definecolor{currentstroke}{rgb}{0.000000,0.000000,0.000000}%
\pgfsetstrokecolor{currentstroke}%
\pgfsetstrokeopacity{0.000000}%
\pgfsetdash{}{0pt}%
\pgfpathmoveto{\pgfqpoint{5.409063in}{0.235000in}}%
\pgfpathlineto{\pgfqpoint{5.409063in}{4.595000in}}%
\pgfusepath{}%
\end{pgfscope}%
\begin{pgfscope}%
\pgfsetrectcap%
\pgfsetmiterjoin%
\pgfsetlinewidth{0.803000pt}%
\definecolor{currentstroke}{rgb}{0.000000,0.000000,0.000000}%
\pgfsetstrokecolor{currentstroke}%
\pgfsetdash{}{0pt}%
\pgfpathmoveto{\pgfqpoint{1.049063in}{2.415000in}}%
\pgfpathlineto{\pgfqpoint{5.409063in}{2.415000in}}%
\pgfusepath{stroke}%
\end{pgfscope}%
\begin{pgfscope}%
\pgfsetrectcap%
\pgfsetmiterjoin%
\pgfsetlinewidth{0.000000pt}%
\definecolor{currentstroke}{rgb}{0.000000,0.000000,0.000000}%
\pgfsetstrokecolor{currentstroke}%
\pgfsetstrokeopacity{0.000000}%
\pgfsetdash{}{0pt}%
\pgfpathmoveto{\pgfqpoint{1.049063in}{4.595000in}}%
\pgfpathlineto{\pgfqpoint{5.409063in}{4.595000in}}%
\pgfusepath{}%
\end{pgfscope}%
\end{pgfpicture}%
\makeatother%
\endgroup%
}\end{solution} \part[1] $f(x)=-x^2+4x+5$\begin{solution} \scalebox{.6}{%% Creator: Matplotlib, PGF backend
%%
%% To include the figure in your LaTeX document, write
%%   \input{<filename>.pgf}
%%
%% Make sure the required packages are loaded in your preamble
%%   \usepackage{pgf}
%%
%% and, on pdftex
%%   \usepackage[utf8]{inputenc}\DeclareUnicodeCharacter{2212}{-}
%%
%% or, on luatex and xetex
%%   \usepackage{unicode-math}
%%
%% Figures using additional raster images can only be included by \input if
%% they are in the same directory as the main LaTeX file. For loading figures
%% from other directories you can use the `import` package
%%   \usepackage{import}
%%
%% and then include the figures with
%%   \import{<path to file>}{<filename>.pgf}
%%
%% Matplotlib used the following preamble
%%   \usepackage{fontspec}
%%   \setmainfont{DejaVuSerif.ttf}[Path=/home/hp/Mis_aplicaciones/anaconda3/lib/python3.6/site-packages/matplotlib/mpl-data/fonts/ttf/]
%%   \setsansfont{DejaVuSans.ttf}[Path=/home/hp/Mis_aplicaciones/anaconda3/lib/python3.6/site-packages/matplotlib/mpl-data/fonts/ttf/]
%%   \setmonofont{DejaVuSansMono.ttf}[Path=/home/hp/Mis_aplicaciones/anaconda3/lib/python3.6/site-packages/matplotlib/mpl-data/fonts/ttf/]
%%
\begingroup%
\makeatletter%
\begin{pgfpicture}%
\pgfpathrectangle{\pgfpointorigin}{\pgfqpoint{6.400000in}{4.800000in}}%
\pgfusepath{use as bounding box, clip}%
\begin{pgfscope}%
\pgfsetbuttcap%
\pgfsetmiterjoin%
\definecolor{currentfill}{rgb}{1.000000,1.000000,1.000000}%
\pgfsetfillcolor{currentfill}%
\pgfsetlinewidth{0.000000pt}%
\definecolor{currentstroke}{rgb}{1.000000,1.000000,1.000000}%
\pgfsetstrokecolor{currentstroke}%
\pgfsetdash{}{0pt}%
\pgfpathmoveto{\pgfqpoint{0.000000in}{0.000000in}}%
\pgfpathlineto{\pgfqpoint{6.400000in}{0.000000in}}%
\pgfpathlineto{\pgfqpoint{6.400000in}{4.800000in}}%
\pgfpathlineto{\pgfqpoint{0.000000in}{4.800000in}}%
\pgfpathclose%
\pgfusepath{fill}%
\end{pgfscope}%
\begin{pgfscope}%
\pgfsetbuttcap%
\pgfsetmiterjoin%
\definecolor{currentfill}{rgb}{1.000000,1.000000,1.000000}%
\pgfsetfillcolor{currentfill}%
\pgfsetlinewidth{0.000000pt}%
\definecolor{currentstroke}{rgb}{0.000000,0.000000,0.000000}%
\pgfsetstrokecolor{currentstroke}%
\pgfsetstrokeopacity{0.000000}%
\pgfsetdash{}{0pt}%
\pgfpathmoveto{\pgfqpoint{1.049063in}{0.235000in}}%
\pgfpathlineto{\pgfqpoint{5.409063in}{0.235000in}}%
\pgfpathlineto{\pgfqpoint{5.409063in}{4.595000in}}%
\pgfpathlineto{\pgfqpoint{1.049063in}{4.595000in}}%
\pgfpathclose%
\pgfusepath{fill}%
\end{pgfscope}%
\begin{pgfscope}%
\pgfpathrectangle{\pgfqpoint{1.049063in}{0.235000in}}{\pgfqpoint{4.360000in}{4.360000in}}%
\pgfusepath{clip}%
\pgfsetbuttcap%
\pgfsetmiterjoin%
\definecolor{currentfill}{rgb}{0.000000,0.000000,1.000000}%
\pgfsetfillcolor{currentfill}%
\pgfsetlinewidth{0.000000pt}%
\definecolor{currentstroke}{rgb}{0.000000,0.000000,0.000000}%
\pgfsetstrokecolor{currentstroke}%
\pgfsetstrokeopacity{0.000000}%
\pgfsetdash{}{0pt}%
\pgfpathmoveto{\pgfqpoint{2.858633in}{0.234999in}}%
\pgfpathlineto{\pgfqpoint{2.858633in}{0.239257in}}%
\pgfpathlineto{\pgfqpoint{2.862891in}{0.239257in}}%
\pgfpathlineto{\pgfqpoint{2.862891in}{0.234999in}}%
\pgfpathmoveto{\pgfqpoint{2.858633in}{0.239257in}}%
\pgfpathlineto{\pgfqpoint{2.858633in}{0.239257in}}%
\pgfpathlineto{\pgfqpoint{2.858633in}{0.243515in}}%
\pgfpathlineto{\pgfqpoint{2.862891in}{0.243515in}}%
\pgfpathlineto{\pgfqpoint{2.862891in}{0.239257in}}%
\pgfpathmoveto{\pgfqpoint{2.858633in}{0.243515in}}%
\pgfpathlineto{\pgfqpoint{2.858633in}{0.243515in}}%
\pgfpathlineto{\pgfqpoint{2.858633in}{0.247772in}}%
\pgfpathlineto{\pgfqpoint{2.862891in}{0.247772in}}%
\pgfpathlineto{\pgfqpoint{2.862891in}{0.243515in}}%
\pgfpathmoveto{\pgfqpoint{2.858633in}{0.247772in}}%
\pgfpathlineto{\pgfqpoint{2.858633in}{0.247772in}}%
\pgfpathlineto{\pgfqpoint{2.858633in}{0.252030in}}%
\pgfpathlineto{\pgfqpoint{2.862891in}{0.252030in}}%
\pgfpathlineto{\pgfqpoint{2.862891in}{0.247772in}}%
\pgfpathmoveto{\pgfqpoint{2.858633in}{0.252030in}}%
\pgfpathlineto{\pgfqpoint{2.858633in}{0.252030in}}%
\pgfpathlineto{\pgfqpoint{2.858633in}{0.256288in}}%
\pgfpathlineto{\pgfqpoint{2.862891in}{0.256288in}}%
\pgfpathlineto{\pgfqpoint{2.862891in}{0.252030in}}%
\pgfpathmoveto{\pgfqpoint{2.858633in}{0.256288in}}%
\pgfpathlineto{\pgfqpoint{2.858633in}{0.256288in}}%
\pgfpathlineto{\pgfqpoint{2.858633in}{0.260546in}}%
\pgfpathlineto{\pgfqpoint{2.862891in}{0.260546in}}%
\pgfpathlineto{\pgfqpoint{2.862891in}{0.256288in}}%
\pgfpathmoveto{\pgfqpoint{2.858633in}{0.260546in}}%
\pgfpathlineto{\pgfqpoint{2.858633in}{0.260546in}}%
\pgfpathlineto{\pgfqpoint{2.858633in}{0.264804in}}%
\pgfpathlineto{\pgfqpoint{2.862891in}{0.264804in}}%
\pgfpathlineto{\pgfqpoint{2.862891in}{0.260546in}}%
\pgfpathmoveto{\pgfqpoint{2.858633in}{0.264804in}}%
\pgfpathlineto{\pgfqpoint{2.858633in}{0.264804in}}%
\pgfpathlineto{\pgfqpoint{2.858633in}{0.269062in}}%
\pgfpathlineto{\pgfqpoint{2.862891in}{0.269062in}}%
\pgfpathlineto{\pgfqpoint{2.862891in}{0.264804in}}%
\pgfpathmoveto{\pgfqpoint{2.862891in}{0.264804in}}%
\pgfpathlineto{\pgfqpoint{2.862891in}{0.264804in}}%
\pgfpathlineto{\pgfqpoint{2.862891in}{0.269062in}}%
\pgfpathlineto{\pgfqpoint{2.867149in}{0.269062in}}%
\pgfpathlineto{\pgfqpoint{2.867149in}{0.264804in}}%
\pgfpathmoveto{\pgfqpoint{2.862891in}{0.269062in}}%
\pgfpathlineto{\pgfqpoint{2.862891in}{0.269062in}}%
\pgfpathlineto{\pgfqpoint{2.862891in}{0.273320in}}%
\pgfpathlineto{\pgfqpoint{2.867149in}{0.273320in}}%
\pgfpathlineto{\pgfqpoint{2.867149in}{0.269062in}}%
\pgfpathmoveto{\pgfqpoint{2.862891in}{0.273320in}}%
\pgfpathlineto{\pgfqpoint{2.862891in}{0.273320in}}%
\pgfpathlineto{\pgfqpoint{2.862891in}{0.277578in}}%
\pgfpathlineto{\pgfqpoint{2.867149in}{0.277578in}}%
\pgfpathlineto{\pgfqpoint{2.867149in}{0.273320in}}%
\pgfpathmoveto{\pgfqpoint{2.862891in}{0.277578in}}%
\pgfpathlineto{\pgfqpoint{2.862891in}{0.277578in}}%
\pgfpathlineto{\pgfqpoint{2.862891in}{0.281835in}}%
\pgfpathlineto{\pgfqpoint{2.867149in}{0.281835in}}%
\pgfpathlineto{\pgfqpoint{2.867149in}{0.277578in}}%
\pgfpathmoveto{\pgfqpoint{2.862891in}{0.281835in}}%
\pgfpathlineto{\pgfqpoint{2.862891in}{0.281835in}}%
\pgfpathlineto{\pgfqpoint{2.862891in}{0.286093in}}%
\pgfpathlineto{\pgfqpoint{2.867149in}{0.286093in}}%
\pgfpathlineto{\pgfqpoint{2.867149in}{0.281835in}}%
\pgfpathmoveto{\pgfqpoint{2.862891in}{0.286093in}}%
\pgfpathlineto{\pgfqpoint{2.862891in}{0.286093in}}%
\pgfpathlineto{\pgfqpoint{2.862891in}{0.290351in}}%
\pgfpathlineto{\pgfqpoint{2.867149in}{0.290351in}}%
\pgfpathlineto{\pgfqpoint{2.867149in}{0.286093in}}%
\pgfpathmoveto{\pgfqpoint{2.862891in}{0.290351in}}%
\pgfpathlineto{\pgfqpoint{2.862891in}{0.290351in}}%
\pgfpathlineto{\pgfqpoint{2.862891in}{0.294609in}}%
\pgfpathlineto{\pgfqpoint{2.867149in}{0.294609in}}%
\pgfpathlineto{\pgfqpoint{2.867149in}{0.290351in}}%
\pgfpathmoveto{\pgfqpoint{2.862891in}{0.294609in}}%
\pgfpathlineto{\pgfqpoint{2.862891in}{0.294609in}}%
\pgfpathlineto{\pgfqpoint{2.862891in}{0.298867in}}%
\pgfpathlineto{\pgfqpoint{2.867149in}{0.298867in}}%
\pgfpathlineto{\pgfqpoint{2.867149in}{0.294609in}}%
\pgfpathmoveto{\pgfqpoint{2.862891in}{0.298867in}}%
\pgfpathlineto{\pgfqpoint{2.862891in}{0.298867in}}%
\pgfpathlineto{\pgfqpoint{2.862891in}{0.303125in}}%
\pgfpathlineto{\pgfqpoint{2.867149in}{0.303125in}}%
\pgfpathlineto{\pgfqpoint{2.867149in}{0.298867in}}%
\pgfpathmoveto{\pgfqpoint{2.862891in}{0.303125in}}%
\pgfpathlineto{\pgfqpoint{2.862891in}{0.303125in}}%
\pgfpathlineto{\pgfqpoint{2.862891in}{0.307383in}}%
\pgfpathlineto{\pgfqpoint{2.867149in}{0.307383in}}%
\pgfpathlineto{\pgfqpoint{2.867149in}{0.303125in}}%
\pgfpathmoveto{\pgfqpoint{2.862891in}{0.307383in}}%
\pgfpathlineto{\pgfqpoint{2.862891in}{0.307383in}}%
\pgfpathlineto{\pgfqpoint{2.862891in}{0.311641in}}%
\pgfpathlineto{\pgfqpoint{2.867149in}{0.311641in}}%
\pgfpathlineto{\pgfqpoint{2.867149in}{0.307383in}}%
\pgfpathmoveto{\pgfqpoint{2.867149in}{0.307383in}}%
\pgfpathlineto{\pgfqpoint{2.867149in}{0.307383in}}%
\pgfpathlineto{\pgfqpoint{2.867149in}{0.311641in}}%
\pgfpathlineto{\pgfqpoint{2.871407in}{0.311641in}}%
\pgfpathlineto{\pgfqpoint{2.871407in}{0.307383in}}%
\pgfpathmoveto{\pgfqpoint{2.867149in}{0.311641in}}%
\pgfpathlineto{\pgfqpoint{2.867149in}{0.311641in}}%
\pgfpathlineto{\pgfqpoint{2.867149in}{0.315898in}}%
\pgfpathlineto{\pgfqpoint{2.871407in}{0.315898in}}%
\pgfpathlineto{\pgfqpoint{2.871407in}{0.311641in}}%
\pgfpathmoveto{\pgfqpoint{2.867149in}{0.315898in}}%
\pgfpathlineto{\pgfqpoint{2.867149in}{0.315898in}}%
\pgfpathlineto{\pgfqpoint{2.867149in}{0.320156in}}%
\pgfpathlineto{\pgfqpoint{2.871407in}{0.320156in}}%
\pgfpathlineto{\pgfqpoint{2.871407in}{0.315898in}}%
\pgfpathmoveto{\pgfqpoint{2.867149in}{0.320156in}}%
\pgfpathlineto{\pgfqpoint{2.867149in}{0.320156in}}%
\pgfpathlineto{\pgfqpoint{2.867149in}{0.324414in}}%
\pgfpathlineto{\pgfqpoint{2.871407in}{0.324414in}}%
\pgfpathlineto{\pgfqpoint{2.871407in}{0.320156in}}%
\pgfpathmoveto{\pgfqpoint{2.867149in}{0.324414in}}%
\pgfpathlineto{\pgfqpoint{2.867149in}{0.324414in}}%
\pgfpathlineto{\pgfqpoint{2.867149in}{0.328672in}}%
\pgfpathlineto{\pgfqpoint{2.871407in}{0.328672in}}%
\pgfpathlineto{\pgfqpoint{2.871407in}{0.324414in}}%
\pgfpathmoveto{\pgfqpoint{2.867149in}{0.328672in}}%
\pgfpathlineto{\pgfqpoint{2.867149in}{0.328672in}}%
\pgfpathlineto{\pgfqpoint{2.867149in}{0.332930in}}%
\pgfpathlineto{\pgfqpoint{2.871407in}{0.332930in}}%
\pgfpathlineto{\pgfqpoint{2.871407in}{0.328672in}}%
\pgfpathmoveto{\pgfqpoint{2.867149in}{0.332930in}}%
\pgfpathlineto{\pgfqpoint{2.867149in}{0.332930in}}%
\pgfpathlineto{\pgfqpoint{2.867149in}{0.337188in}}%
\pgfpathlineto{\pgfqpoint{2.871407in}{0.337188in}}%
\pgfpathlineto{\pgfqpoint{2.871407in}{0.332930in}}%
\pgfpathmoveto{\pgfqpoint{2.867149in}{0.337188in}}%
\pgfpathlineto{\pgfqpoint{2.867149in}{0.337188in}}%
\pgfpathlineto{\pgfqpoint{2.867149in}{0.341446in}}%
\pgfpathlineto{\pgfqpoint{2.871407in}{0.341446in}}%
\pgfpathlineto{\pgfqpoint{2.871407in}{0.337188in}}%
\pgfpathmoveto{\pgfqpoint{2.867149in}{0.341446in}}%
\pgfpathlineto{\pgfqpoint{2.867149in}{0.341446in}}%
\pgfpathlineto{\pgfqpoint{2.867149in}{0.345704in}}%
\pgfpathlineto{\pgfqpoint{2.871407in}{0.345704in}}%
\pgfpathlineto{\pgfqpoint{2.871407in}{0.341446in}}%
\pgfpathmoveto{\pgfqpoint{2.867149in}{0.345704in}}%
\pgfpathlineto{\pgfqpoint{2.867149in}{0.345704in}}%
\pgfpathlineto{\pgfqpoint{2.867149in}{0.349961in}}%
\pgfpathlineto{\pgfqpoint{2.871407in}{0.349961in}}%
\pgfpathlineto{\pgfqpoint{2.871407in}{0.345704in}}%
\pgfpathmoveto{\pgfqpoint{2.867149in}{0.349961in}}%
\pgfpathlineto{\pgfqpoint{2.867149in}{0.349961in}}%
\pgfpathlineto{\pgfqpoint{2.867149in}{0.354219in}}%
\pgfpathlineto{\pgfqpoint{2.871407in}{0.354219in}}%
\pgfpathlineto{\pgfqpoint{2.871407in}{0.349961in}}%
\pgfpathmoveto{\pgfqpoint{2.867149in}{0.354219in}}%
\pgfpathlineto{\pgfqpoint{2.867149in}{0.354219in}}%
\pgfpathlineto{\pgfqpoint{2.867149in}{0.358477in}}%
\pgfpathlineto{\pgfqpoint{2.871407in}{0.358477in}}%
\pgfpathlineto{\pgfqpoint{2.871407in}{0.354219in}}%
\pgfpathmoveto{\pgfqpoint{2.871407in}{0.354219in}}%
\pgfpathlineto{\pgfqpoint{2.871407in}{0.354219in}}%
\pgfpathlineto{\pgfqpoint{2.871407in}{0.358477in}}%
\pgfpathlineto{\pgfqpoint{2.875665in}{0.358477in}}%
\pgfpathlineto{\pgfqpoint{2.875665in}{0.354219in}}%
\pgfpathmoveto{\pgfqpoint{2.871407in}{0.358477in}}%
\pgfpathlineto{\pgfqpoint{2.871407in}{0.358477in}}%
\pgfpathlineto{\pgfqpoint{2.871407in}{0.362735in}}%
\pgfpathlineto{\pgfqpoint{2.875665in}{0.362735in}}%
\pgfpathlineto{\pgfqpoint{2.875665in}{0.358477in}}%
\pgfpathmoveto{\pgfqpoint{2.871407in}{0.362735in}}%
\pgfpathlineto{\pgfqpoint{2.871407in}{0.362735in}}%
\pgfpathlineto{\pgfqpoint{2.871407in}{0.366993in}}%
\pgfpathlineto{\pgfqpoint{2.875665in}{0.366993in}}%
\pgfpathlineto{\pgfqpoint{2.875665in}{0.362735in}}%
\pgfpathmoveto{\pgfqpoint{2.871407in}{0.366993in}}%
\pgfpathlineto{\pgfqpoint{2.871407in}{0.366993in}}%
\pgfpathlineto{\pgfqpoint{2.871407in}{0.371251in}}%
\pgfpathlineto{\pgfqpoint{2.875665in}{0.371251in}}%
\pgfpathlineto{\pgfqpoint{2.875665in}{0.366993in}}%
\pgfpathmoveto{\pgfqpoint{2.871407in}{0.371251in}}%
\pgfpathlineto{\pgfqpoint{2.871407in}{0.371251in}}%
\pgfpathlineto{\pgfqpoint{2.871407in}{0.375509in}}%
\pgfpathlineto{\pgfqpoint{2.875665in}{0.375509in}}%
\pgfpathlineto{\pgfqpoint{2.875665in}{0.371251in}}%
\pgfpathmoveto{\pgfqpoint{2.871407in}{0.375509in}}%
\pgfpathlineto{\pgfqpoint{2.871407in}{0.375509in}}%
\pgfpathlineto{\pgfqpoint{2.871407in}{0.379767in}}%
\pgfpathlineto{\pgfqpoint{2.875665in}{0.379767in}}%
\pgfpathlineto{\pgfqpoint{2.875665in}{0.375509in}}%
\pgfpathmoveto{\pgfqpoint{2.871407in}{0.379767in}}%
\pgfpathlineto{\pgfqpoint{2.871407in}{0.379767in}}%
\pgfpathlineto{\pgfqpoint{2.871407in}{0.384024in}}%
\pgfpathlineto{\pgfqpoint{2.875665in}{0.384024in}}%
\pgfpathlineto{\pgfqpoint{2.875665in}{0.379767in}}%
\pgfpathmoveto{\pgfqpoint{2.871407in}{0.384024in}}%
\pgfpathlineto{\pgfqpoint{2.871407in}{0.384024in}}%
\pgfpathlineto{\pgfqpoint{2.871407in}{0.388282in}}%
\pgfpathlineto{\pgfqpoint{2.875665in}{0.388282in}}%
\pgfpathlineto{\pgfqpoint{2.875665in}{0.384024in}}%
\pgfpathmoveto{\pgfqpoint{2.871407in}{0.388282in}}%
\pgfpathlineto{\pgfqpoint{2.871407in}{0.388282in}}%
\pgfpathlineto{\pgfqpoint{2.871407in}{0.392540in}}%
\pgfpathlineto{\pgfqpoint{2.875665in}{0.392540in}}%
\pgfpathlineto{\pgfqpoint{2.875665in}{0.388282in}}%
\pgfpathmoveto{\pgfqpoint{2.871407in}{0.392540in}}%
\pgfpathlineto{\pgfqpoint{2.871407in}{0.392540in}}%
\pgfpathlineto{\pgfqpoint{2.871407in}{0.396798in}}%
\pgfpathlineto{\pgfqpoint{2.875665in}{0.396798in}}%
\pgfpathlineto{\pgfqpoint{2.875665in}{0.392540in}}%
\pgfpathmoveto{\pgfqpoint{2.871407in}{0.396798in}}%
\pgfpathlineto{\pgfqpoint{2.871407in}{0.396798in}}%
\pgfpathlineto{\pgfqpoint{2.871407in}{0.401056in}}%
\pgfpathlineto{\pgfqpoint{2.875665in}{0.401056in}}%
\pgfpathlineto{\pgfqpoint{2.875665in}{0.396798in}}%
\pgfpathmoveto{\pgfqpoint{2.875665in}{0.396798in}}%
\pgfpathlineto{\pgfqpoint{2.875665in}{0.396798in}}%
\pgfpathlineto{\pgfqpoint{2.875665in}{0.401056in}}%
\pgfpathlineto{\pgfqpoint{2.879923in}{0.401056in}}%
\pgfpathlineto{\pgfqpoint{2.879923in}{0.396798in}}%
\pgfpathmoveto{\pgfqpoint{2.875665in}{0.401056in}}%
\pgfpathlineto{\pgfqpoint{2.875665in}{0.401056in}}%
\pgfpathlineto{\pgfqpoint{2.875665in}{0.405313in}}%
\pgfpathlineto{\pgfqpoint{2.879923in}{0.405313in}}%
\pgfpathlineto{\pgfqpoint{2.879923in}{0.401056in}}%
\pgfpathmoveto{\pgfqpoint{2.875665in}{0.405313in}}%
\pgfpathlineto{\pgfqpoint{2.875665in}{0.405313in}}%
\pgfpathlineto{\pgfqpoint{2.875665in}{0.409571in}}%
\pgfpathlineto{\pgfqpoint{2.879923in}{0.409571in}}%
\pgfpathlineto{\pgfqpoint{2.879923in}{0.405313in}}%
\pgfpathmoveto{\pgfqpoint{2.875665in}{0.409571in}}%
\pgfpathlineto{\pgfqpoint{2.875665in}{0.409571in}}%
\pgfpathlineto{\pgfqpoint{2.875665in}{0.413829in}}%
\pgfpathlineto{\pgfqpoint{2.879923in}{0.413829in}}%
\pgfpathlineto{\pgfqpoint{2.879923in}{0.409571in}}%
\pgfpathmoveto{\pgfqpoint{2.875665in}{0.413829in}}%
\pgfpathlineto{\pgfqpoint{2.875665in}{0.413829in}}%
\pgfpathlineto{\pgfqpoint{2.875665in}{0.418087in}}%
\pgfpathlineto{\pgfqpoint{2.879923in}{0.418087in}}%
\pgfpathlineto{\pgfqpoint{2.879923in}{0.413829in}}%
\pgfpathmoveto{\pgfqpoint{2.875665in}{0.418087in}}%
\pgfpathlineto{\pgfqpoint{2.875665in}{0.418087in}}%
\pgfpathlineto{\pgfqpoint{2.875665in}{0.422345in}}%
\pgfpathlineto{\pgfqpoint{2.879923in}{0.422345in}}%
\pgfpathlineto{\pgfqpoint{2.879923in}{0.418087in}}%
\pgfpathmoveto{\pgfqpoint{2.875665in}{0.422345in}}%
\pgfpathlineto{\pgfqpoint{2.875665in}{0.422345in}}%
\pgfpathlineto{\pgfqpoint{2.875665in}{0.426603in}}%
\pgfpathlineto{\pgfqpoint{2.879923in}{0.426603in}}%
\pgfpathlineto{\pgfqpoint{2.879923in}{0.422345in}}%
\pgfpathmoveto{\pgfqpoint{2.875665in}{0.426603in}}%
\pgfpathlineto{\pgfqpoint{2.875665in}{0.426603in}}%
\pgfpathlineto{\pgfqpoint{2.875665in}{0.430860in}}%
\pgfpathlineto{\pgfqpoint{2.879923in}{0.430860in}}%
\pgfpathlineto{\pgfqpoint{2.879923in}{0.426603in}}%
\pgfpathmoveto{\pgfqpoint{2.875665in}{0.430860in}}%
\pgfpathlineto{\pgfqpoint{2.875665in}{0.430860in}}%
\pgfpathlineto{\pgfqpoint{2.875665in}{0.435118in}}%
\pgfpathlineto{\pgfqpoint{2.879923in}{0.435118in}}%
\pgfpathlineto{\pgfqpoint{2.879923in}{0.430860in}}%
\pgfpathmoveto{\pgfqpoint{2.875665in}{0.435118in}}%
\pgfpathlineto{\pgfqpoint{2.875665in}{0.435118in}}%
\pgfpathlineto{\pgfqpoint{2.875665in}{0.439376in}}%
\pgfpathlineto{\pgfqpoint{2.879923in}{0.439376in}}%
\pgfpathlineto{\pgfqpoint{2.879923in}{0.435118in}}%
\pgfpathmoveto{\pgfqpoint{2.875665in}{0.439376in}}%
\pgfpathlineto{\pgfqpoint{2.875665in}{0.439376in}}%
\pgfpathlineto{\pgfqpoint{2.875665in}{0.443634in}}%
\pgfpathlineto{\pgfqpoint{2.879923in}{0.443634in}}%
\pgfpathlineto{\pgfqpoint{2.879923in}{0.439376in}}%
\pgfpathmoveto{\pgfqpoint{2.875665in}{0.443634in}}%
\pgfpathlineto{\pgfqpoint{2.875665in}{0.443634in}}%
\pgfpathlineto{\pgfqpoint{2.875665in}{0.447892in}}%
\pgfpathlineto{\pgfqpoint{2.879923in}{0.447892in}}%
\pgfpathlineto{\pgfqpoint{2.879923in}{0.443634in}}%
\pgfpathmoveto{\pgfqpoint{2.879923in}{0.443634in}}%
\pgfpathlineto{\pgfqpoint{2.879923in}{0.443634in}}%
\pgfpathlineto{\pgfqpoint{2.879923in}{0.447892in}}%
\pgfpathlineto{\pgfqpoint{2.884181in}{0.447892in}}%
\pgfpathlineto{\pgfqpoint{2.884181in}{0.443634in}}%
\pgfpathmoveto{\pgfqpoint{2.879923in}{0.447892in}}%
\pgfpathlineto{\pgfqpoint{2.879923in}{0.447892in}}%
\pgfpathlineto{\pgfqpoint{2.879923in}{0.452150in}}%
\pgfpathlineto{\pgfqpoint{2.884181in}{0.452150in}}%
\pgfpathlineto{\pgfqpoint{2.884181in}{0.447892in}}%
\pgfpathmoveto{\pgfqpoint{2.879923in}{0.452150in}}%
\pgfpathlineto{\pgfqpoint{2.879923in}{0.452150in}}%
\pgfpathlineto{\pgfqpoint{2.879923in}{0.456407in}}%
\pgfpathlineto{\pgfqpoint{2.884181in}{0.456407in}}%
\pgfpathlineto{\pgfqpoint{2.884181in}{0.452150in}}%
\pgfpathmoveto{\pgfqpoint{2.879923in}{0.456407in}}%
\pgfpathlineto{\pgfqpoint{2.879923in}{0.456407in}}%
\pgfpathlineto{\pgfqpoint{2.879923in}{0.460665in}}%
\pgfpathlineto{\pgfqpoint{2.884181in}{0.460665in}}%
\pgfpathlineto{\pgfqpoint{2.884181in}{0.456407in}}%
\pgfpathmoveto{\pgfqpoint{2.879923in}{0.460665in}}%
\pgfpathlineto{\pgfqpoint{2.879923in}{0.460665in}}%
\pgfpathlineto{\pgfqpoint{2.879923in}{0.464923in}}%
\pgfpathlineto{\pgfqpoint{2.884181in}{0.464923in}}%
\pgfpathlineto{\pgfqpoint{2.884181in}{0.460665in}}%
\pgfpathmoveto{\pgfqpoint{2.879923in}{0.464923in}}%
\pgfpathlineto{\pgfqpoint{2.879923in}{0.464923in}}%
\pgfpathlineto{\pgfqpoint{2.879923in}{0.469181in}}%
\pgfpathlineto{\pgfqpoint{2.884181in}{0.469181in}}%
\pgfpathlineto{\pgfqpoint{2.884181in}{0.464923in}}%
\pgfpathmoveto{\pgfqpoint{2.879923in}{0.469181in}}%
\pgfpathlineto{\pgfqpoint{2.879923in}{0.469181in}}%
\pgfpathlineto{\pgfqpoint{2.879923in}{0.473439in}}%
\pgfpathlineto{\pgfqpoint{2.884181in}{0.473439in}}%
\pgfpathlineto{\pgfqpoint{2.884181in}{0.469181in}}%
\pgfpathmoveto{\pgfqpoint{2.879923in}{0.473439in}}%
\pgfpathlineto{\pgfqpoint{2.879923in}{0.473439in}}%
\pgfpathlineto{\pgfqpoint{2.879923in}{0.477696in}}%
\pgfpathlineto{\pgfqpoint{2.884181in}{0.477696in}}%
\pgfpathlineto{\pgfqpoint{2.884181in}{0.473439in}}%
\pgfpathmoveto{\pgfqpoint{2.879923in}{0.477696in}}%
\pgfpathlineto{\pgfqpoint{2.879923in}{0.477696in}}%
\pgfpathlineto{\pgfqpoint{2.879923in}{0.481954in}}%
\pgfpathlineto{\pgfqpoint{2.884181in}{0.481954in}}%
\pgfpathlineto{\pgfqpoint{2.884181in}{0.477696in}}%
\pgfpathmoveto{\pgfqpoint{2.879923in}{0.481954in}}%
\pgfpathlineto{\pgfqpoint{2.879923in}{0.481954in}}%
\pgfpathlineto{\pgfqpoint{2.879923in}{0.486212in}}%
\pgfpathlineto{\pgfqpoint{2.884181in}{0.486212in}}%
\pgfpathlineto{\pgfqpoint{2.884181in}{0.481954in}}%
\pgfpathmoveto{\pgfqpoint{2.879923in}{0.486212in}}%
\pgfpathlineto{\pgfqpoint{2.879923in}{0.486212in}}%
\pgfpathlineto{\pgfqpoint{2.879923in}{0.490470in}}%
\pgfpathlineto{\pgfqpoint{2.884181in}{0.490470in}}%
\pgfpathlineto{\pgfqpoint{2.884181in}{0.486212in}}%
\pgfpathmoveto{\pgfqpoint{2.884181in}{0.486212in}}%
\pgfpathlineto{\pgfqpoint{2.884181in}{0.486212in}}%
\pgfpathlineto{\pgfqpoint{2.884181in}{0.490470in}}%
\pgfpathlineto{\pgfqpoint{2.888439in}{0.490470in}}%
\pgfpathlineto{\pgfqpoint{2.888439in}{0.486212in}}%
\pgfpathmoveto{\pgfqpoint{2.884181in}{0.490470in}}%
\pgfpathlineto{\pgfqpoint{2.884181in}{0.490470in}}%
\pgfpathlineto{\pgfqpoint{2.884181in}{0.494728in}}%
\pgfpathlineto{\pgfqpoint{2.888439in}{0.494728in}}%
\pgfpathlineto{\pgfqpoint{2.888439in}{0.490470in}}%
\pgfpathmoveto{\pgfqpoint{2.884181in}{0.494728in}}%
\pgfpathlineto{\pgfqpoint{2.884181in}{0.494728in}}%
\pgfpathlineto{\pgfqpoint{2.884181in}{0.498986in}}%
\pgfpathlineto{\pgfqpoint{2.888439in}{0.498986in}}%
\pgfpathlineto{\pgfqpoint{2.888439in}{0.494728in}}%
\pgfpathmoveto{\pgfqpoint{2.884181in}{0.498986in}}%
\pgfpathlineto{\pgfqpoint{2.884181in}{0.498986in}}%
\pgfpathlineto{\pgfqpoint{2.884181in}{0.503243in}}%
\pgfpathlineto{\pgfqpoint{2.888439in}{0.503243in}}%
\pgfpathlineto{\pgfqpoint{2.888439in}{0.498986in}}%
\pgfpathmoveto{\pgfqpoint{2.884181in}{0.503243in}}%
\pgfpathlineto{\pgfqpoint{2.884181in}{0.503243in}}%
\pgfpathlineto{\pgfqpoint{2.884181in}{0.507501in}}%
\pgfpathlineto{\pgfqpoint{2.888439in}{0.507501in}}%
\pgfpathlineto{\pgfqpoint{2.888439in}{0.503243in}}%
\pgfpathmoveto{\pgfqpoint{2.884181in}{0.507501in}}%
\pgfpathlineto{\pgfqpoint{2.884181in}{0.507501in}}%
\pgfpathlineto{\pgfqpoint{2.884181in}{0.511759in}}%
\pgfpathlineto{\pgfqpoint{2.888439in}{0.511759in}}%
\pgfpathlineto{\pgfqpoint{2.888439in}{0.507501in}}%
\pgfpathmoveto{\pgfqpoint{2.884181in}{0.511759in}}%
\pgfpathlineto{\pgfqpoint{2.884181in}{0.511759in}}%
\pgfpathlineto{\pgfqpoint{2.884181in}{0.516017in}}%
\pgfpathlineto{\pgfqpoint{2.888439in}{0.516017in}}%
\pgfpathlineto{\pgfqpoint{2.888439in}{0.511759in}}%
\pgfpathmoveto{\pgfqpoint{2.884181in}{0.516017in}}%
\pgfpathlineto{\pgfqpoint{2.884181in}{0.516017in}}%
\pgfpathlineto{\pgfqpoint{2.884181in}{0.520274in}}%
\pgfpathlineto{\pgfqpoint{2.888439in}{0.520274in}}%
\pgfpathlineto{\pgfqpoint{2.888439in}{0.516017in}}%
\pgfpathmoveto{\pgfqpoint{2.884181in}{0.520274in}}%
\pgfpathlineto{\pgfqpoint{2.884181in}{0.520274in}}%
\pgfpathlineto{\pgfqpoint{2.884181in}{0.524532in}}%
\pgfpathlineto{\pgfqpoint{2.888439in}{0.524532in}}%
\pgfpathlineto{\pgfqpoint{2.888439in}{0.520274in}}%
\pgfpathmoveto{\pgfqpoint{2.884181in}{0.524532in}}%
\pgfpathlineto{\pgfqpoint{2.884181in}{0.524532in}}%
\pgfpathlineto{\pgfqpoint{2.884181in}{0.528790in}}%
\pgfpathlineto{\pgfqpoint{2.888439in}{0.528790in}}%
\pgfpathlineto{\pgfqpoint{2.888439in}{0.524532in}}%
\pgfpathmoveto{\pgfqpoint{2.884181in}{0.528790in}}%
\pgfpathlineto{\pgfqpoint{2.884181in}{0.528790in}}%
\pgfpathlineto{\pgfqpoint{2.884181in}{0.533047in}}%
\pgfpathlineto{\pgfqpoint{2.888439in}{0.533047in}}%
\pgfpathlineto{\pgfqpoint{2.888439in}{0.528790in}}%
\pgfpathmoveto{\pgfqpoint{2.884181in}{0.533047in}}%
\pgfpathlineto{\pgfqpoint{2.884181in}{0.533047in}}%
\pgfpathlineto{\pgfqpoint{2.884181in}{0.537305in}}%
\pgfpathlineto{\pgfqpoint{2.888439in}{0.537305in}}%
\pgfpathlineto{\pgfqpoint{2.888439in}{0.533047in}}%
\pgfpathmoveto{\pgfqpoint{2.888439in}{0.533047in}}%
\pgfpathlineto{\pgfqpoint{2.888439in}{0.533047in}}%
\pgfpathlineto{\pgfqpoint{2.888439in}{0.537305in}}%
\pgfpathlineto{\pgfqpoint{2.892697in}{0.537305in}}%
\pgfpathlineto{\pgfqpoint{2.892697in}{0.533047in}}%
\pgfpathmoveto{\pgfqpoint{2.888439in}{0.537305in}}%
\pgfpathlineto{\pgfqpoint{2.888439in}{0.537305in}}%
\pgfpathlineto{\pgfqpoint{2.888439in}{0.541563in}}%
\pgfpathlineto{\pgfqpoint{2.892697in}{0.541563in}}%
\pgfpathlineto{\pgfqpoint{2.892697in}{0.537305in}}%
\pgfpathmoveto{\pgfqpoint{2.888439in}{0.541563in}}%
\pgfpathlineto{\pgfqpoint{2.888439in}{0.541563in}}%
\pgfpathlineto{\pgfqpoint{2.888439in}{0.545820in}}%
\pgfpathlineto{\pgfqpoint{2.892697in}{0.545820in}}%
\pgfpathlineto{\pgfqpoint{2.892697in}{0.541563in}}%
\pgfpathmoveto{\pgfqpoint{2.888439in}{0.545820in}}%
\pgfpathlineto{\pgfqpoint{2.888439in}{0.545820in}}%
\pgfpathlineto{\pgfqpoint{2.888439in}{0.550078in}}%
\pgfpathlineto{\pgfqpoint{2.892697in}{0.550078in}}%
\pgfpathlineto{\pgfqpoint{2.892697in}{0.545820in}}%
\pgfpathmoveto{\pgfqpoint{2.888439in}{0.550078in}}%
\pgfpathlineto{\pgfqpoint{2.888439in}{0.550078in}}%
\pgfpathlineto{\pgfqpoint{2.888439in}{0.554336in}}%
\pgfpathlineto{\pgfqpoint{2.892697in}{0.554336in}}%
\pgfpathlineto{\pgfqpoint{2.892697in}{0.550078in}}%
\pgfpathmoveto{\pgfqpoint{2.888439in}{0.554336in}}%
\pgfpathlineto{\pgfqpoint{2.888439in}{0.554336in}}%
\pgfpathlineto{\pgfqpoint{2.888439in}{0.558593in}}%
\pgfpathlineto{\pgfqpoint{2.892697in}{0.558593in}}%
\pgfpathlineto{\pgfqpoint{2.892697in}{0.554336in}}%
\pgfpathmoveto{\pgfqpoint{2.888439in}{0.558593in}}%
\pgfpathlineto{\pgfqpoint{2.888439in}{0.558593in}}%
\pgfpathlineto{\pgfqpoint{2.888439in}{0.562851in}}%
\pgfpathlineto{\pgfqpoint{2.892697in}{0.562851in}}%
\pgfpathlineto{\pgfqpoint{2.892697in}{0.558593in}}%
\pgfpathmoveto{\pgfqpoint{2.888439in}{0.562851in}}%
\pgfpathlineto{\pgfqpoint{2.888439in}{0.562851in}}%
\pgfpathlineto{\pgfqpoint{2.888439in}{0.567109in}}%
\pgfpathlineto{\pgfqpoint{2.892697in}{0.567109in}}%
\pgfpathlineto{\pgfqpoint{2.892697in}{0.562851in}}%
\pgfpathmoveto{\pgfqpoint{2.888439in}{0.567109in}}%
\pgfpathlineto{\pgfqpoint{2.888439in}{0.567109in}}%
\pgfpathlineto{\pgfqpoint{2.888439in}{0.571366in}}%
\pgfpathlineto{\pgfqpoint{2.892697in}{0.571366in}}%
\pgfpathlineto{\pgfqpoint{2.892697in}{0.567109in}}%
\pgfpathmoveto{\pgfqpoint{2.888439in}{0.571366in}}%
\pgfpathlineto{\pgfqpoint{2.888439in}{0.571366in}}%
\pgfpathlineto{\pgfqpoint{2.888439in}{0.575624in}}%
\pgfpathlineto{\pgfqpoint{2.892697in}{0.575624in}}%
\pgfpathlineto{\pgfqpoint{2.892697in}{0.571366in}}%
\pgfpathmoveto{\pgfqpoint{2.888439in}{0.575624in}}%
\pgfpathlineto{\pgfqpoint{2.888439in}{0.575624in}}%
\pgfpathlineto{\pgfqpoint{2.888439in}{0.579882in}}%
\pgfpathlineto{\pgfqpoint{2.892697in}{0.579882in}}%
\pgfpathlineto{\pgfqpoint{2.892697in}{0.575624in}}%
\pgfpathmoveto{\pgfqpoint{2.892697in}{0.575624in}}%
\pgfpathlineto{\pgfqpoint{2.892697in}{0.575624in}}%
\pgfpathlineto{\pgfqpoint{2.892697in}{0.579882in}}%
\pgfpathlineto{\pgfqpoint{2.896955in}{0.579882in}}%
\pgfpathlineto{\pgfqpoint{2.896955in}{0.575624in}}%
\pgfpathmoveto{\pgfqpoint{2.892697in}{0.579882in}}%
\pgfpathlineto{\pgfqpoint{2.892697in}{0.579882in}}%
\pgfpathlineto{\pgfqpoint{2.892697in}{0.584139in}}%
\pgfpathlineto{\pgfqpoint{2.896955in}{0.584139in}}%
\pgfpathlineto{\pgfqpoint{2.896955in}{0.579882in}}%
\pgfpathmoveto{\pgfqpoint{2.892697in}{0.584139in}}%
\pgfpathlineto{\pgfqpoint{2.892697in}{0.584139in}}%
\pgfpathlineto{\pgfqpoint{2.892697in}{0.588397in}}%
\pgfpathlineto{\pgfqpoint{2.896955in}{0.588397in}}%
\pgfpathlineto{\pgfqpoint{2.896955in}{0.584139in}}%
\pgfpathmoveto{\pgfqpoint{2.892697in}{0.588397in}}%
\pgfpathlineto{\pgfqpoint{2.892697in}{0.588397in}}%
\pgfpathlineto{\pgfqpoint{2.892697in}{0.592655in}}%
\pgfpathlineto{\pgfqpoint{2.896955in}{0.592655in}}%
\pgfpathlineto{\pgfqpoint{2.896955in}{0.588397in}}%
\pgfpathmoveto{\pgfqpoint{2.892697in}{0.592655in}}%
\pgfpathlineto{\pgfqpoint{2.892697in}{0.592655in}}%
\pgfpathlineto{\pgfqpoint{2.892697in}{0.596912in}}%
\pgfpathlineto{\pgfqpoint{2.896955in}{0.596912in}}%
\pgfpathlineto{\pgfqpoint{2.896955in}{0.592655in}}%
\pgfpathmoveto{\pgfqpoint{2.892697in}{0.596912in}}%
\pgfpathlineto{\pgfqpoint{2.892697in}{0.596912in}}%
\pgfpathlineto{\pgfqpoint{2.892697in}{0.601170in}}%
\pgfpathlineto{\pgfqpoint{2.896955in}{0.601170in}}%
\pgfpathlineto{\pgfqpoint{2.896955in}{0.596912in}}%
\pgfpathmoveto{\pgfqpoint{2.892697in}{0.601170in}}%
\pgfpathlineto{\pgfqpoint{2.892697in}{0.601170in}}%
\pgfpathlineto{\pgfqpoint{2.892697in}{0.605428in}}%
\pgfpathlineto{\pgfqpoint{2.896955in}{0.605428in}}%
\pgfpathlineto{\pgfqpoint{2.896955in}{0.601170in}}%
\pgfpathmoveto{\pgfqpoint{2.892697in}{0.605428in}}%
\pgfpathlineto{\pgfqpoint{2.892697in}{0.605428in}}%
\pgfpathlineto{\pgfqpoint{2.892697in}{0.609685in}}%
\pgfpathlineto{\pgfqpoint{2.896955in}{0.609685in}}%
\pgfpathlineto{\pgfqpoint{2.896955in}{0.605428in}}%
\pgfpathmoveto{\pgfqpoint{2.892697in}{0.609685in}}%
\pgfpathlineto{\pgfqpoint{2.892697in}{0.609685in}}%
\pgfpathlineto{\pgfqpoint{2.892697in}{0.613943in}}%
\pgfpathlineto{\pgfqpoint{2.896955in}{0.613943in}}%
\pgfpathlineto{\pgfqpoint{2.896955in}{0.609685in}}%
\pgfpathmoveto{\pgfqpoint{2.892697in}{0.613943in}}%
\pgfpathlineto{\pgfqpoint{2.892697in}{0.613943in}}%
\pgfpathlineto{\pgfqpoint{2.892697in}{0.618201in}}%
\pgfpathlineto{\pgfqpoint{2.896955in}{0.618201in}}%
\pgfpathlineto{\pgfqpoint{2.896955in}{0.613943in}}%
\pgfpathmoveto{\pgfqpoint{2.892697in}{0.618201in}}%
\pgfpathlineto{\pgfqpoint{2.892697in}{0.618201in}}%
\pgfpathlineto{\pgfqpoint{2.892697in}{0.622458in}}%
\pgfpathlineto{\pgfqpoint{2.896955in}{0.622458in}}%
\pgfpathlineto{\pgfqpoint{2.896955in}{0.618201in}}%
\pgfpathmoveto{\pgfqpoint{2.896955in}{0.618201in}}%
\pgfpathlineto{\pgfqpoint{2.896955in}{0.618201in}}%
\pgfpathlineto{\pgfqpoint{2.896955in}{0.622458in}}%
\pgfpathlineto{\pgfqpoint{2.901213in}{0.622458in}}%
\pgfpathlineto{\pgfqpoint{2.901213in}{0.618201in}}%
\pgfpathmoveto{\pgfqpoint{2.896955in}{0.622458in}}%
\pgfpathlineto{\pgfqpoint{2.896955in}{0.622458in}}%
\pgfpathlineto{\pgfqpoint{2.896955in}{0.626716in}}%
\pgfpathlineto{\pgfqpoint{2.901213in}{0.626716in}}%
\pgfpathlineto{\pgfqpoint{2.901213in}{0.622458in}}%
\pgfpathmoveto{\pgfqpoint{2.896955in}{0.626716in}}%
\pgfpathlineto{\pgfqpoint{2.896955in}{0.626716in}}%
\pgfpathlineto{\pgfqpoint{2.896955in}{0.630974in}}%
\pgfpathlineto{\pgfqpoint{2.901213in}{0.630974in}}%
\pgfpathlineto{\pgfqpoint{2.901213in}{0.626716in}}%
\pgfpathmoveto{\pgfqpoint{2.896955in}{0.630974in}}%
\pgfpathlineto{\pgfqpoint{2.896955in}{0.630974in}}%
\pgfpathlineto{\pgfqpoint{2.896955in}{0.635231in}}%
\pgfpathlineto{\pgfqpoint{2.901213in}{0.635231in}}%
\pgfpathlineto{\pgfqpoint{2.901213in}{0.630974in}}%
\pgfpathmoveto{\pgfqpoint{2.896955in}{0.635231in}}%
\pgfpathlineto{\pgfqpoint{2.896955in}{0.635231in}}%
\pgfpathlineto{\pgfqpoint{2.896955in}{0.639489in}}%
\pgfpathlineto{\pgfqpoint{2.901213in}{0.639489in}}%
\pgfpathlineto{\pgfqpoint{2.901213in}{0.635231in}}%
\pgfpathmoveto{\pgfqpoint{2.896955in}{0.639489in}}%
\pgfpathlineto{\pgfqpoint{2.896955in}{0.639489in}}%
\pgfpathlineto{\pgfqpoint{2.896955in}{0.643747in}}%
\pgfpathlineto{\pgfqpoint{2.901213in}{0.643747in}}%
\pgfpathlineto{\pgfqpoint{2.901213in}{0.639489in}}%
\pgfpathmoveto{\pgfqpoint{2.896955in}{0.643747in}}%
\pgfpathlineto{\pgfqpoint{2.896955in}{0.643747in}}%
\pgfpathlineto{\pgfqpoint{2.896955in}{0.648005in}}%
\pgfpathlineto{\pgfqpoint{2.901213in}{0.648005in}}%
\pgfpathlineto{\pgfqpoint{2.901213in}{0.643747in}}%
\pgfpathmoveto{\pgfqpoint{2.896955in}{0.648005in}}%
\pgfpathlineto{\pgfqpoint{2.896955in}{0.648005in}}%
\pgfpathlineto{\pgfqpoint{2.896955in}{0.652263in}}%
\pgfpathlineto{\pgfqpoint{2.901213in}{0.652263in}}%
\pgfpathlineto{\pgfqpoint{2.901213in}{0.648005in}}%
\pgfpathmoveto{\pgfqpoint{2.896955in}{0.652263in}}%
\pgfpathlineto{\pgfqpoint{2.896955in}{0.652263in}}%
\pgfpathlineto{\pgfqpoint{2.896955in}{0.656521in}}%
\pgfpathlineto{\pgfqpoint{2.901213in}{0.656521in}}%
\pgfpathlineto{\pgfqpoint{2.901213in}{0.652263in}}%
\pgfpathmoveto{\pgfqpoint{2.896955in}{0.656521in}}%
\pgfpathlineto{\pgfqpoint{2.896955in}{0.656521in}}%
\pgfpathlineto{\pgfqpoint{2.896955in}{0.660779in}}%
\pgfpathlineto{\pgfqpoint{2.901213in}{0.660779in}}%
\pgfpathlineto{\pgfqpoint{2.901213in}{0.656521in}}%
\pgfpathmoveto{\pgfqpoint{2.896955in}{0.660779in}}%
\pgfpathlineto{\pgfqpoint{2.896955in}{0.660779in}}%
\pgfpathlineto{\pgfqpoint{2.896955in}{0.665037in}}%
\pgfpathlineto{\pgfqpoint{2.901213in}{0.665037in}}%
\pgfpathlineto{\pgfqpoint{2.901213in}{0.660779in}}%
\pgfpathmoveto{\pgfqpoint{2.901213in}{0.660779in}}%
\pgfpathlineto{\pgfqpoint{2.901213in}{0.660779in}}%
\pgfpathlineto{\pgfqpoint{2.901213in}{0.665037in}}%
\pgfpathlineto{\pgfqpoint{2.905470in}{0.665037in}}%
\pgfpathlineto{\pgfqpoint{2.905470in}{0.660779in}}%
\pgfpathmoveto{\pgfqpoint{2.901213in}{0.665037in}}%
\pgfpathlineto{\pgfqpoint{2.901213in}{0.665037in}}%
\pgfpathlineto{\pgfqpoint{2.901213in}{0.669295in}}%
\pgfpathlineto{\pgfqpoint{2.905470in}{0.669295in}}%
\pgfpathlineto{\pgfqpoint{2.905470in}{0.665037in}}%
\pgfpathmoveto{\pgfqpoint{2.901213in}{0.669295in}}%
\pgfpathlineto{\pgfqpoint{2.901213in}{0.669295in}}%
\pgfpathlineto{\pgfqpoint{2.901213in}{0.673553in}}%
\pgfpathlineto{\pgfqpoint{2.905470in}{0.673553in}}%
\pgfpathlineto{\pgfqpoint{2.905470in}{0.669295in}}%
\pgfpathmoveto{\pgfqpoint{2.901213in}{0.673553in}}%
\pgfpathlineto{\pgfqpoint{2.901213in}{0.673553in}}%
\pgfpathlineto{\pgfqpoint{2.901213in}{0.677811in}}%
\pgfpathlineto{\pgfqpoint{2.905470in}{0.677811in}}%
\pgfpathlineto{\pgfqpoint{2.905470in}{0.673553in}}%
\pgfpathmoveto{\pgfqpoint{2.901213in}{0.677811in}}%
\pgfpathlineto{\pgfqpoint{2.901213in}{0.677811in}}%
\pgfpathlineto{\pgfqpoint{2.901213in}{0.682069in}}%
\pgfpathlineto{\pgfqpoint{2.905470in}{0.682069in}}%
\pgfpathlineto{\pgfqpoint{2.905470in}{0.677811in}}%
\pgfpathmoveto{\pgfqpoint{2.901213in}{0.682069in}}%
\pgfpathlineto{\pgfqpoint{2.901213in}{0.682069in}}%
\pgfpathlineto{\pgfqpoint{2.901213in}{0.686327in}}%
\pgfpathlineto{\pgfqpoint{2.905470in}{0.686327in}}%
\pgfpathlineto{\pgfqpoint{2.905470in}{0.682069in}}%
\pgfpathmoveto{\pgfqpoint{2.901213in}{0.686327in}}%
\pgfpathlineto{\pgfqpoint{2.901213in}{0.686327in}}%
\pgfpathlineto{\pgfqpoint{2.901213in}{0.690585in}}%
\pgfpathlineto{\pgfqpoint{2.905470in}{0.690585in}}%
\pgfpathlineto{\pgfqpoint{2.905470in}{0.686327in}}%
\pgfpathmoveto{\pgfqpoint{2.901213in}{0.690585in}}%
\pgfpathlineto{\pgfqpoint{2.901213in}{0.690585in}}%
\pgfpathlineto{\pgfqpoint{2.901213in}{0.694843in}}%
\pgfpathlineto{\pgfqpoint{2.905470in}{0.694843in}}%
\pgfpathlineto{\pgfqpoint{2.905470in}{0.690585in}}%
\pgfpathmoveto{\pgfqpoint{2.901213in}{0.694843in}}%
\pgfpathlineto{\pgfqpoint{2.901213in}{0.694843in}}%
\pgfpathlineto{\pgfqpoint{2.901213in}{0.699101in}}%
\pgfpathlineto{\pgfqpoint{2.905470in}{0.699101in}}%
\pgfpathlineto{\pgfqpoint{2.905470in}{0.694843in}}%
\pgfpathmoveto{\pgfqpoint{2.901213in}{0.699101in}}%
\pgfpathlineto{\pgfqpoint{2.901213in}{0.699101in}}%
\pgfpathlineto{\pgfqpoint{2.901213in}{0.703359in}}%
\pgfpathlineto{\pgfqpoint{2.905470in}{0.703359in}}%
\pgfpathlineto{\pgfqpoint{2.905470in}{0.699101in}}%
\pgfpathmoveto{\pgfqpoint{2.901213in}{0.703359in}}%
\pgfpathlineto{\pgfqpoint{2.901213in}{0.703359in}}%
\pgfpathlineto{\pgfqpoint{2.901213in}{0.707617in}}%
\pgfpathlineto{\pgfqpoint{2.905470in}{0.707617in}}%
\pgfpathlineto{\pgfqpoint{2.905470in}{0.703359in}}%
\pgfpathmoveto{\pgfqpoint{2.905470in}{0.703359in}}%
\pgfpathlineto{\pgfqpoint{2.905470in}{0.703359in}}%
\pgfpathlineto{\pgfqpoint{2.905470in}{0.707617in}}%
\pgfpathlineto{\pgfqpoint{2.909728in}{0.707617in}}%
\pgfpathlineto{\pgfqpoint{2.909728in}{0.703359in}}%
\pgfpathmoveto{\pgfqpoint{2.905470in}{0.707617in}}%
\pgfpathlineto{\pgfqpoint{2.905470in}{0.707617in}}%
\pgfpathlineto{\pgfqpoint{2.905470in}{0.711875in}}%
\pgfpathlineto{\pgfqpoint{2.909728in}{0.711875in}}%
\pgfpathlineto{\pgfqpoint{2.909728in}{0.707617in}}%
\pgfpathmoveto{\pgfqpoint{2.905470in}{0.711875in}}%
\pgfpathlineto{\pgfqpoint{2.905470in}{0.711875in}}%
\pgfpathlineto{\pgfqpoint{2.905470in}{0.716133in}}%
\pgfpathlineto{\pgfqpoint{2.909728in}{0.716133in}}%
\pgfpathlineto{\pgfqpoint{2.909728in}{0.711875in}}%
\pgfpathmoveto{\pgfqpoint{2.905470in}{0.716133in}}%
\pgfpathlineto{\pgfqpoint{2.905470in}{0.716133in}}%
\pgfpathlineto{\pgfqpoint{2.905470in}{0.720391in}}%
\pgfpathlineto{\pgfqpoint{2.909728in}{0.720391in}}%
\pgfpathlineto{\pgfqpoint{2.909728in}{0.716133in}}%
\pgfpathmoveto{\pgfqpoint{2.905470in}{0.720391in}}%
\pgfpathlineto{\pgfqpoint{2.905470in}{0.720391in}}%
\pgfpathlineto{\pgfqpoint{2.905470in}{0.724649in}}%
\pgfpathlineto{\pgfqpoint{2.909728in}{0.724649in}}%
\pgfpathlineto{\pgfqpoint{2.909728in}{0.720391in}}%
\pgfpathmoveto{\pgfqpoint{2.905470in}{0.724649in}}%
\pgfpathlineto{\pgfqpoint{2.905470in}{0.724649in}}%
\pgfpathlineto{\pgfqpoint{2.905470in}{0.728907in}}%
\pgfpathlineto{\pgfqpoint{2.909728in}{0.728907in}}%
\pgfpathlineto{\pgfqpoint{2.909728in}{0.724649in}}%
\pgfpathmoveto{\pgfqpoint{2.905470in}{0.728907in}}%
\pgfpathlineto{\pgfqpoint{2.905470in}{0.728907in}}%
\pgfpathlineto{\pgfqpoint{2.905470in}{0.733165in}}%
\pgfpathlineto{\pgfqpoint{2.909728in}{0.733165in}}%
\pgfpathlineto{\pgfqpoint{2.909728in}{0.728907in}}%
\pgfpathmoveto{\pgfqpoint{2.905470in}{0.733165in}}%
\pgfpathlineto{\pgfqpoint{2.905470in}{0.733165in}}%
\pgfpathlineto{\pgfqpoint{2.905470in}{0.737424in}}%
\pgfpathlineto{\pgfqpoint{2.909728in}{0.737424in}}%
\pgfpathlineto{\pgfqpoint{2.909728in}{0.733165in}}%
\pgfpathmoveto{\pgfqpoint{2.905470in}{0.737424in}}%
\pgfpathlineto{\pgfqpoint{2.905470in}{0.737424in}}%
\pgfpathlineto{\pgfqpoint{2.905470in}{0.741682in}}%
\pgfpathlineto{\pgfqpoint{2.909728in}{0.741682in}}%
\pgfpathlineto{\pgfqpoint{2.909728in}{0.737424in}}%
\pgfpathmoveto{\pgfqpoint{2.905470in}{0.741682in}}%
\pgfpathlineto{\pgfqpoint{2.905470in}{0.741682in}}%
\pgfpathlineto{\pgfqpoint{2.905470in}{0.745940in}}%
\pgfpathlineto{\pgfqpoint{2.909728in}{0.745940in}}%
\pgfpathlineto{\pgfqpoint{2.909728in}{0.741682in}}%
\pgfpathmoveto{\pgfqpoint{2.905470in}{0.745940in}}%
\pgfpathlineto{\pgfqpoint{2.905470in}{0.745940in}}%
\pgfpathlineto{\pgfqpoint{2.905470in}{0.750198in}}%
\pgfpathlineto{\pgfqpoint{2.909728in}{0.750198in}}%
\pgfpathlineto{\pgfqpoint{2.909728in}{0.745940in}}%
\pgfpathmoveto{\pgfqpoint{2.909728in}{0.745940in}}%
\pgfpathlineto{\pgfqpoint{2.909728in}{0.745940in}}%
\pgfpathlineto{\pgfqpoint{2.909728in}{0.750198in}}%
\pgfpathlineto{\pgfqpoint{2.913986in}{0.750198in}}%
\pgfpathlineto{\pgfqpoint{2.913986in}{0.745940in}}%
\pgfpathmoveto{\pgfqpoint{2.909728in}{0.750198in}}%
\pgfpathlineto{\pgfqpoint{2.909728in}{0.750198in}}%
\pgfpathlineto{\pgfqpoint{2.909728in}{0.754456in}}%
\pgfpathlineto{\pgfqpoint{2.913986in}{0.754456in}}%
\pgfpathlineto{\pgfqpoint{2.913986in}{0.750198in}}%
\pgfpathmoveto{\pgfqpoint{2.909728in}{0.754456in}}%
\pgfpathlineto{\pgfqpoint{2.909728in}{0.754456in}}%
\pgfpathlineto{\pgfqpoint{2.909728in}{0.758714in}}%
\pgfpathlineto{\pgfqpoint{2.913986in}{0.758714in}}%
\pgfpathlineto{\pgfqpoint{2.913986in}{0.754456in}}%
\pgfpathmoveto{\pgfqpoint{2.909728in}{0.758714in}}%
\pgfpathlineto{\pgfqpoint{2.909728in}{0.758714in}}%
\pgfpathlineto{\pgfqpoint{2.909728in}{0.762972in}}%
\pgfpathlineto{\pgfqpoint{2.913986in}{0.762972in}}%
\pgfpathlineto{\pgfqpoint{2.913986in}{0.758714in}}%
\pgfpathmoveto{\pgfqpoint{2.909728in}{0.762972in}}%
\pgfpathlineto{\pgfqpoint{2.909728in}{0.762972in}}%
\pgfpathlineto{\pgfqpoint{2.909728in}{0.767230in}}%
\pgfpathlineto{\pgfqpoint{2.913986in}{0.767230in}}%
\pgfpathlineto{\pgfqpoint{2.913986in}{0.762972in}}%
\pgfpathmoveto{\pgfqpoint{2.909728in}{0.767230in}}%
\pgfpathlineto{\pgfqpoint{2.909728in}{0.767230in}}%
\pgfpathlineto{\pgfqpoint{2.909728in}{0.771488in}}%
\pgfpathlineto{\pgfqpoint{2.913986in}{0.771488in}}%
\pgfpathlineto{\pgfqpoint{2.913986in}{0.767230in}}%
\pgfpathmoveto{\pgfqpoint{2.909728in}{0.771488in}}%
\pgfpathlineto{\pgfqpoint{2.909728in}{0.771488in}}%
\pgfpathlineto{\pgfqpoint{2.909728in}{0.775746in}}%
\pgfpathlineto{\pgfqpoint{2.913986in}{0.775746in}}%
\pgfpathlineto{\pgfqpoint{2.913986in}{0.771488in}}%
\pgfpathmoveto{\pgfqpoint{2.909728in}{0.775746in}}%
\pgfpathlineto{\pgfqpoint{2.909728in}{0.775746in}}%
\pgfpathlineto{\pgfqpoint{2.909728in}{0.780004in}}%
\pgfpathlineto{\pgfqpoint{2.913986in}{0.780004in}}%
\pgfpathlineto{\pgfqpoint{2.913986in}{0.775746in}}%
\pgfpathmoveto{\pgfqpoint{2.909728in}{0.780004in}}%
\pgfpathlineto{\pgfqpoint{2.909728in}{0.780004in}}%
\pgfpathlineto{\pgfqpoint{2.909728in}{0.784262in}}%
\pgfpathlineto{\pgfqpoint{2.913986in}{0.784262in}}%
\pgfpathlineto{\pgfqpoint{2.913986in}{0.780004in}}%
\pgfpathmoveto{\pgfqpoint{2.909728in}{0.784262in}}%
\pgfpathlineto{\pgfqpoint{2.909728in}{0.784262in}}%
\pgfpathlineto{\pgfqpoint{2.909728in}{0.788519in}}%
\pgfpathlineto{\pgfqpoint{2.913986in}{0.788519in}}%
\pgfpathlineto{\pgfqpoint{2.913986in}{0.784262in}}%
\pgfpathmoveto{\pgfqpoint{2.909728in}{0.788519in}}%
\pgfpathlineto{\pgfqpoint{2.909728in}{0.788519in}}%
\pgfpathlineto{\pgfqpoint{2.909728in}{0.792777in}}%
\pgfpathlineto{\pgfqpoint{2.913986in}{0.792777in}}%
\pgfpathlineto{\pgfqpoint{2.913986in}{0.788519in}}%
\pgfpathmoveto{\pgfqpoint{2.913986in}{0.788519in}}%
\pgfpathlineto{\pgfqpoint{2.913986in}{0.788519in}}%
\pgfpathlineto{\pgfqpoint{2.913986in}{0.792777in}}%
\pgfpathlineto{\pgfqpoint{2.918244in}{0.792777in}}%
\pgfpathlineto{\pgfqpoint{2.918244in}{0.788519in}}%
\pgfpathmoveto{\pgfqpoint{2.913986in}{0.792777in}}%
\pgfpathlineto{\pgfqpoint{2.913986in}{0.792777in}}%
\pgfpathlineto{\pgfqpoint{2.913986in}{0.797034in}}%
\pgfpathlineto{\pgfqpoint{2.918244in}{0.797034in}}%
\pgfpathlineto{\pgfqpoint{2.918244in}{0.792777in}}%
\pgfpathmoveto{\pgfqpoint{2.913986in}{0.797034in}}%
\pgfpathlineto{\pgfqpoint{2.913986in}{0.797034in}}%
\pgfpathlineto{\pgfqpoint{2.913986in}{0.801292in}}%
\pgfpathlineto{\pgfqpoint{2.918244in}{0.801292in}}%
\pgfpathlineto{\pgfqpoint{2.918244in}{0.797034in}}%
\pgfpathmoveto{\pgfqpoint{2.913986in}{0.801292in}}%
\pgfpathlineto{\pgfqpoint{2.913986in}{0.801292in}}%
\pgfpathlineto{\pgfqpoint{2.913986in}{0.805550in}}%
\pgfpathlineto{\pgfqpoint{2.918244in}{0.805550in}}%
\pgfpathlineto{\pgfqpoint{2.918244in}{0.801292in}}%
\pgfpathmoveto{\pgfqpoint{2.913986in}{0.805550in}}%
\pgfpathlineto{\pgfqpoint{2.913986in}{0.805550in}}%
\pgfpathlineto{\pgfqpoint{2.913986in}{0.809807in}}%
\pgfpathlineto{\pgfqpoint{2.918244in}{0.809807in}}%
\pgfpathlineto{\pgfqpoint{2.918244in}{0.805550in}}%
\pgfpathmoveto{\pgfqpoint{2.913986in}{0.809807in}}%
\pgfpathlineto{\pgfqpoint{2.913986in}{0.809807in}}%
\pgfpathlineto{\pgfqpoint{2.913986in}{0.814065in}}%
\pgfpathlineto{\pgfqpoint{2.918244in}{0.814065in}}%
\pgfpathlineto{\pgfqpoint{2.918244in}{0.809807in}}%
\pgfpathmoveto{\pgfqpoint{2.913986in}{0.814065in}}%
\pgfpathlineto{\pgfqpoint{2.913986in}{0.814065in}}%
\pgfpathlineto{\pgfqpoint{2.913986in}{0.818322in}}%
\pgfpathlineto{\pgfqpoint{2.918244in}{0.818322in}}%
\pgfpathlineto{\pgfqpoint{2.918244in}{0.814065in}}%
\pgfpathmoveto{\pgfqpoint{2.913986in}{0.818322in}}%
\pgfpathlineto{\pgfqpoint{2.913986in}{0.818322in}}%
\pgfpathlineto{\pgfqpoint{2.913986in}{0.822580in}}%
\pgfpathlineto{\pgfqpoint{2.918244in}{0.822580in}}%
\pgfpathlineto{\pgfqpoint{2.918244in}{0.818322in}}%
\pgfpathmoveto{\pgfqpoint{2.913986in}{0.822580in}}%
\pgfpathlineto{\pgfqpoint{2.913986in}{0.822580in}}%
\pgfpathlineto{\pgfqpoint{2.913986in}{0.826838in}}%
\pgfpathlineto{\pgfqpoint{2.918244in}{0.826838in}}%
\pgfpathlineto{\pgfqpoint{2.918244in}{0.822580in}}%
\pgfpathmoveto{\pgfqpoint{2.913986in}{0.826838in}}%
\pgfpathlineto{\pgfqpoint{2.913986in}{0.826838in}}%
\pgfpathlineto{\pgfqpoint{2.913986in}{0.831095in}}%
\pgfpathlineto{\pgfqpoint{2.918244in}{0.831095in}}%
\pgfpathlineto{\pgfqpoint{2.918244in}{0.826838in}}%
\pgfpathmoveto{\pgfqpoint{2.918244in}{0.826838in}}%
\pgfpathlineto{\pgfqpoint{2.918244in}{0.826838in}}%
\pgfpathlineto{\pgfqpoint{2.918244in}{0.831095in}}%
\pgfpathlineto{\pgfqpoint{2.922502in}{0.831095in}}%
\pgfpathlineto{\pgfqpoint{2.922502in}{0.826838in}}%
\pgfpathmoveto{\pgfqpoint{2.918244in}{0.831095in}}%
\pgfpathlineto{\pgfqpoint{2.918244in}{0.831095in}}%
\pgfpathlineto{\pgfqpoint{2.918244in}{0.835353in}}%
\pgfpathlineto{\pgfqpoint{2.922502in}{0.835353in}}%
\pgfpathlineto{\pgfqpoint{2.922502in}{0.831095in}}%
\pgfpathmoveto{\pgfqpoint{2.918244in}{0.835353in}}%
\pgfpathlineto{\pgfqpoint{2.918244in}{0.835353in}}%
\pgfpathlineto{\pgfqpoint{2.918244in}{0.839610in}}%
\pgfpathlineto{\pgfqpoint{2.922502in}{0.839610in}}%
\pgfpathlineto{\pgfqpoint{2.922502in}{0.835353in}}%
\pgfpathmoveto{\pgfqpoint{2.918244in}{0.839610in}}%
\pgfpathlineto{\pgfqpoint{2.918244in}{0.839610in}}%
\pgfpathlineto{\pgfqpoint{2.918244in}{0.843868in}}%
\pgfpathlineto{\pgfqpoint{2.922502in}{0.843868in}}%
\pgfpathlineto{\pgfqpoint{2.922502in}{0.839610in}}%
\pgfpathmoveto{\pgfqpoint{2.918244in}{0.843868in}}%
\pgfpathlineto{\pgfqpoint{2.918244in}{0.843868in}}%
\pgfpathlineto{\pgfqpoint{2.918244in}{0.848126in}}%
\pgfpathlineto{\pgfqpoint{2.922502in}{0.848126in}}%
\pgfpathlineto{\pgfqpoint{2.922502in}{0.843868in}}%
\pgfpathmoveto{\pgfqpoint{2.918244in}{0.848126in}}%
\pgfpathlineto{\pgfqpoint{2.918244in}{0.848126in}}%
\pgfpathlineto{\pgfqpoint{2.918244in}{0.852383in}}%
\pgfpathlineto{\pgfqpoint{2.922502in}{0.852383in}}%
\pgfpathlineto{\pgfqpoint{2.922502in}{0.848126in}}%
\pgfpathmoveto{\pgfqpoint{2.918244in}{0.852383in}}%
\pgfpathlineto{\pgfqpoint{2.918244in}{0.852383in}}%
\pgfpathlineto{\pgfqpoint{2.918244in}{0.856641in}}%
\pgfpathlineto{\pgfqpoint{2.922502in}{0.856641in}}%
\pgfpathlineto{\pgfqpoint{2.922502in}{0.852383in}}%
\pgfpathmoveto{\pgfqpoint{2.918244in}{0.856641in}}%
\pgfpathlineto{\pgfqpoint{2.918244in}{0.856641in}}%
\pgfpathlineto{\pgfqpoint{2.918244in}{0.860898in}}%
\pgfpathlineto{\pgfqpoint{2.922502in}{0.860898in}}%
\pgfpathlineto{\pgfqpoint{2.922502in}{0.856641in}}%
\pgfpathmoveto{\pgfqpoint{2.918244in}{0.860898in}}%
\pgfpathlineto{\pgfqpoint{2.918244in}{0.860898in}}%
\pgfpathlineto{\pgfqpoint{2.918244in}{0.865156in}}%
\pgfpathlineto{\pgfqpoint{2.922502in}{0.865156in}}%
\pgfpathlineto{\pgfqpoint{2.922502in}{0.860898in}}%
\pgfpathmoveto{\pgfqpoint{2.918244in}{0.865156in}}%
\pgfpathlineto{\pgfqpoint{2.918244in}{0.865156in}}%
\pgfpathlineto{\pgfqpoint{2.918244in}{0.869414in}}%
\pgfpathlineto{\pgfqpoint{2.922502in}{0.869414in}}%
\pgfpathlineto{\pgfqpoint{2.922502in}{0.865156in}}%
\pgfpathmoveto{\pgfqpoint{2.918244in}{0.869414in}}%
\pgfpathlineto{\pgfqpoint{2.918244in}{0.869414in}}%
\pgfpathlineto{\pgfqpoint{2.918244in}{0.873671in}}%
\pgfpathlineto{\pgfqpoint{2.922502in}{0.873671in}}%
\pgfpathlineto{\pgfqpoint{2.922502in}{0.869414in}}%
\pgfpathmoveto{\pgfqpoint{2.922502in}{0.869414in}}%
\pgfpathlineto{\pgfqpoint{2.922502in}{0.869414in}}%
\pgfpathlineto{\pgfqpoint{2.922502in}{0.873671in}}%
\pgfpathlineto{\pgfqpoint{2.926760in}{0.873671in}}%
\pgfpathlineto{\pgfqpoint{2.926760in}{0.869414in}}%
\pgfpathmoveto{\pgfqpoint{2.922502in}{0.873671in}}%
\pgfpathlineto{\pgfqpoint{2.922502in}{0.873671in}}%
\pgfpathlineto{\pgfqpoint{2.922502in}{0.877929in}}%
\pgfpathlineto{\pgfqpoint{2.926760in}{0.877929in}}%
\pgfpathlineto{\pgfqpoint{2.926760in}{0.873671in}}%
\pgfpathmoveto{\pgfqpoint{2.922502in}{0.877929in}}%
\pgfpathlineto{\pgfqpoint{2.922502in}{0.877929in}}%
\pgfpathlineto{\pgfqpoint{2.922502in}{0.882186in}}%
\pgfpathlineto{\pgfqpoint{2.926760in}{0.882186in}}%
\pgfpathlineto{\pgfqpoint{2.926760in}{0.877929in}}%
\pgfpathmoveto{\pgfqpoint{2.922502in}{0.882186in}}%
\pgfpathlineto{\pgfqpoint{2.922502in}{0.882186in}}%
\pgfpathlineto{\pgfqpoint{2.922502in}{0.886444in}}%
\pgfpathlineto{\pgfqpoint{2.926760in}{0.886444in}}%
\pgfpathlineto{\pgfqpoint{2.926760in}{0.882186in}}%
\pgfpathmoveto{\pgfqpoint{2.922502in}{0.886444in}}%
\pgfpathlineto{\pgfqpoint{2.922502in}{0.886444in}}%
\pgfpathlineto{\pgfqpoint{2.922502in}{0.890702in}}%
\pgfpathlineto{\pgfqpoint{2.926760in}{0.890702in}}%
\pgfpathlineto{\pgfqpoint{2.926760in}{0.886444in}}%
\pgfpathmoveto{\pgfqpoint{2.922502in}{0.890702in}}%
\pgfpathlineto{\pgfqpoint{2.922502in}{0.890702in}}%
\pgfpathlineto{\pgfqpoint{2.922502in}{0.894959in}}%
\pgfpathlineto{\pgfqpoint{2.926760in}{0.894959in}}%
\pgfpathlineto{\pgfqpoint{2.926760in}{0.890702in}}%
\pgfpathmoveto{\pgfqpoint{2.922502in}{0.894959in}}%
\pgfpathlineto{\pgfqpoint{2.922502in}{0.894959in}}%
\pgfpathlineto{\pgfqpoint{2.922502in}{0.899217in}}%
\pgfpathlineto{\pgfqpoint{2.926760in}{0.899217in}}%
\pgfpathlineto{\pgfqpoint{2.926760in}{0.894959in}}%
\pgfpathmoveto{\pgfqpoint{2.922502in}{0.899217in}}%
\pgfpathlineto{\pgfqpoint{2.922502in}{0.899217in}}%
\pgfpathlineto{\pgfqpoint{2.922502in}{0.903474in}}%
\pgfpathlineto{\pgfqpoint{2.926760in}{0.903474in}}%
\pgfpathlineto{\pgfqpoint{2.926760in}{0.899217in}}%
\pgfpathmoveto{\pgfqpoint{2.922502in}{0.903474in}}%
\pgfpathlineto{\pgfqpoint{2.922502in}{0.903474in}}%
\pgfpathlineto{\pgfqpoint{2.922502in}{0.907732in}}%
\pgfpathlineto{\pgfqpoint{2.926760in}{0.907732in}}%
\pgfpathlineto{\pgfqpoint{2.926760in}{0.903474in}}%
\pgfpathmoveto{\pgfqpoint{2.922502in}{0.907732in}}%
\pgfpathlineto{\pgfqpoint{2.922502in}{0.907732in}}%
\pgfpathlineto{\pgfqpoint{2.922502in}{0.911990in}}%
\pgfpathlineto{\pgfqpoint{2.926760in}{0.911990in}}%
\pgfpathlineto{\pgfqpoint{2.926760in}{0.907732in}}%
\pgfpathmoveto{\pgfqpoint{2.922502in}{0.911990in}}%
\pgfpathlineto{\pgfqpoint{2.922502in}{0.911990in}}%
\pgfpathlineto{\pgfqpoint{2.922502in}{0.916247in}}%
\pgfpathlineto{\pgfqpoint{2.926760in}{0.916247in}}%
\pgfpathlineto{\pgfqpoint{2.926760in}{0.911990in}}%
\pgfpathmoveto{\pgfqpoint{2.926760in}{0.911990in}}%
\pgfpathlineto{\pgfqpoint{2.926760in}{0.911990in}}%
\pgfpathlineto{\pgfqpoint{2.926760in}{0.916247in}}%
\pgfpathlineto{\pgfqpoint{2.931018in}{0.916247in}}%
\pgfpathlineto{\pgfqpoint{2.931018in}{0.911990in}}%
\pgfpathmoveto{\pgfqpoint{2.926760in}{0.916247in}}%
\pgfpathlineto{\pgfqpoint{2.926760in}{0.916247in}}%
\pgfpathlineto{\pgfqpoint{2.926760in}{0.920505in}}%
\pgfpathlineto{\pgfqpoint{2.931018in}{0.920505in}}%
\pgfpathlineto{\pgfqpoint{2.931018in}{0.916247in}}%
\pgfpathmoveto{\pgfqpoint{2.926760in}{0.920505in}}%
\pgfpathlineto{\pgfqpoint{2.926760in}{0.920505in}}%
\pgfpathlineto{\pgfqpoint{2.926760in}{0.924763in}}%
\pgfpathlineto{\pgfqpoint{2.931018in}{0.924763in}}%
\pgfpathlineto{\pgfqpoint{2.931018in}{0.920505in}}%
\pgfpathmoveto{\pgfqpoint{2.926760in}{0.924763in}}%
\pgfpathlineto{\pgfqpoint{2.926760in}{0.924763in}}%
\pgfpathlineto{\pgfqpoint{2.926760in}{0.929021in}}%
\pgfpathlineto{\pgfqpoint{2.931018in}{0.929021in}}%
\pgfpathlineto{\pgfqpoint{2.931018in}{0.924763in}}%
\pgfpathmoveto{\pgfqpoint{2.926760in}{0.929021in}}%
\pgfpathlineto{\pgfqpoint{2.926760in}{0.929021in}}%
\pgfpathlineto{\pgfqpoint{2.926760in}{0.933279in}}%
\pgfpathlineto{\pgfqpoint{2.931018in}{0.933279in}}%
\pgfpathlineto{\pgfqpoint{2.931018in}{0.929021in}}%
\pgfpathmoveto{\pgfqpoint{2.926760in}{0.933279in}}%
\pgfpathlineto{\pgfqpoint{2.926760in}{0.933279in}}%
\pgfpathlineto{\pgfqpoint{2.926760in}{0.937537in}}%
\pgfpathlineto{\pgfqpoint{2.931018in}{0.937537in}}%
\pgfpathlineto{\pgfqpoint{2.931018in}{0.933279in}}%
\pgfpathmoveto{\pgfqpoint{2.926760in}{0.937537in}}%
\pgfpathlineto{\pgfqpoint{2.926760in}{0.937537in}}%
\pgfpathlineto{\pgfqpoint{2.926760in}{0.941795in}}%
\pgfpathlineto{\pgfqpoint{2.931018in}{0.941795in}}%
\pgfpathlineto{\pgfqpoint{2.931018in}{0.937537in}}%
\pgfpathmoveto{\pgfqpoint{2.926760in}{0.941795in}}%
\pgfpathlineto{\pgfqpoint{2.926760in}{0.941795in}}%
\pgfpathlineto{\pgfqpoint{2.926760in}{0.946053in}}%
\pgfpathlineto{\pgfqpoint{2.931018in}{0.946053in}}%
\pgfpathlineto{\pgfqpoint{2.931018in}{0.941795in}}%
\pgfpathmoveto{\pgfqpoint{2.926760in}{0.946053in}}%
\pgfpathlineto{\pgfqpoint{2.926760in}{0.946053in}}%
\pgfpathlineto{\pgfqpoint{2.926760in}{0.950311in}}%
\pgfpathlineto{\pgfqpoint{2.931018in}{0.950311in}}%
\pgfpathlineto{\pgfqpoint{2.931018in}{0.946053in}}%
\pgfpathmoveto{\pgfqpoint{2.926760in}{0.950311in}}%
\pgfpathlineto{\pgfqpoint{2.926760in}{0.950311in}}%
\pgfpathlineto{\pgfqpoint{2.926760in}{0.954569in}}%
\pgfpathlineto{\pgfqpoint{2.931018in}{0.954569in}}%
\pgfpathlineto{\pgfqpoint{2.931018in}{0.950311in}}%
\pgfpathmoveto{\pgfqpoint{2.931018in}{0.950311in}}%
\pgfpathlineto{\pgfqpoint{2.931018in}{0.950311in}}%
\pgfpathlineto{\pgfqpoint{2.931018in}{0.954569in}}%
\pgfpathlineto{\pgfqpoint{2.935276in}{0.954569in}}%
\pgfpathlineto{\pgfqpoint{2.935276in}{0.950311in}}%
\pgfpathmoveto{\pgfqpoint{2.931018in}{0.954569in}}%
\pgfpathlineto{\pgfqpoint{2.931018in}{0.954569in}}%
\pgfpathlineto{\pgfqpoint{2.931018in}{0.958827in}}%
\pgfpathlineto{\pgfqpoint{2.935276in}{0.958827in}}%
\pgfpathlineto{\pgfqpoint{2.935276in}{0.954569in}}%
\pgfpathmoveto{\pgfqpoint{2.931018in}{0.958827in}}%
\pgfpathlineto{\pgfqpoint{2.931018in}{0.958827in}}%
\pgfpathlineto{\pgfqpoint{2.931018in}{0.963085in}}%
\pgfpathlineto{\pgfqpoint{2.935276in}{0.963085in}}%
\pgfpathlineto{\pgfqpoint{2.935276in}{0.958827in}}%
\pgfpathmoveto{\pgfqpoint{2.931018in}{0.963085in}}%
\pgfpathlineto{\pgfqpoint{2.931018in}{0.963085in}}%
\pgfpathlineto{\pgfqpoint{2.931018in}{0.967343in}}%
\pgfpathlineto{\pgfqpoint{2.935276in}{0.967343in}}%
\pgfpathlineto{\pgfqpoint{2.935276in}{0.963085in}}%
\pgfpathmoveto{\pgfqpoint{2.931018in}{0.967343in}}%
\pgfpathlineto{\pgfqpoint{2.931018in}{0.967343in}}%
\pgfpathlineto{\pgfqpoint{2.931018in}{0.971601in}}%
\pgfpathlineto{\pgfqpoint{2.935276in}{0.971601in}}%
\pgfpathlineto{\pgfqpoint{2.935276in}{0.967343in}}%
\pgfpathmoveto{\pgfqpoint{2.931018in}{0.971601in}}%
\pgfpathlineto{\pgfqpoint{2.931018in}{0.971601in}}%
\pgfpathlineto{\pgfqpoint{2.931018in}{0.975859in}}%
\pgfpathlineto{\pgfqpoint{2.935276in}{0.975859in}}%
\pgfpathlineto{\pgfqpoint{2.935276in}{0.971601in}}%
\pgfpathmoveto{\pgfqpoint{2.931018in}{0.975859in}}%
\pgfpathlineto{\pgfqpoint{2.931018in}{0.975859in}}%
\pgfpathlineto{\pgfqpoint{2.931018in}{0.980117in}}%
\pgfpathlineto{\pgfqpoint{2.935276in}{0.980117in}}%
\pgfpathlineto{\pgfqpoint{2.935276in}{0.975859in}}%
\pgfpathmoveto{\pgfqpoint{2.931018in}{0.980117in}}%
\pgfpathlineto{\pgfqpoint{2.931018in}{0.980117in}}%
\pgfpathlineto{\pgfqpoint{2.931018in}{0.984375in}}%
\pgfpathlineto{\pgfqpoint{2.935276in}{0.984375in}}%
\pgfpathlineto{\pgfqpoint{2.935276in}{0.980117in}}%
\pgfpathmoveto{\pgfqpoint{2.931018in}{0.984375in}}%
\pgfpathlineto{\pgfqpoint{2.931018in}{0.984375in}}%
\pgfpathlineto{\pgfqpoint{2.931018in}{0.988633in}}%
\pgfpathlineto{\pgfqpoint{2.935276in}{0.988633in}}%
\pgfpathlineto{\pgfqpoint{2.935276in}{0.984375in}}%
\pgfpathmoveto{\pgfqpoint{2.931018in}{0.988633in}}%
\pgfpathlineto{\pgfqpoint{2.931018in}{0.988633in}}%
\pgfpathlineto{\pgfqpoint{2.931018in}{0.992891in}}%
\pgfpathlineto{\pgfqpoint{2.935276in}{0.992891in}}%
\pgfpathlineto{\pgfqpoint{2.935276in}{0.988633in}}%
\pgfpathmoveto{\pgfqpoint{2.931018in}{0.992891in}}%
\pgfpathlineto{\pgfqpoint{2.931018in}{0.992891in}}%
\pgfpathlineto{\pgfqpoint{2.931018in}{0.997148in}}%
\pgfpathlineto{\pgfqpoint{2.935276in}{0.997148in}}%
\pgfpathlineto{\pgfqpoint{2.935276in}{0.992891in}}%
\pgfpathmoveto{\pgfqpoint{2.935276in}{0.992891in}}%
\pgfpathlineto{\pgfqpoint{2.935276in}{0.992891in}}%
\pgfpathlineto{\pgfqpoint{2.935276in}{0.997148in}}%
\pgfpathlineto{\pgfqpoint{2.939534in}{0.997148in}}%
\pgfpathlineto{\pgfqpoint{2.939534in}{0.992891in}}%
\pgfpathmoveto{\pgfqpoint{2.935276in}{0.997148in}}%
\pgfpathlineto{\pgfqpoint{2.935276in}{0.997148in}}%
\pgfpathlineto{\pgfqpoint{2.935276in}{1.001406in}}%
\pgfpathlineto{\pgfqpoint{2.939534in}{1.001406in}}%
\pgfpathlineto{\pgfqpoint{2.939534in}{0.997148in}}%
\pgfpathmoveto{\pgfqpoint{2.935276in}{1.001406in}}%
\pgfpathlineto{\pgfqpoint{2.935276in}{1.001406in}}%
\pgfpathlineto{\pgfqpoint{2.935276in}{1.005664in}}%
\pgfpathlineto{\pgfqpoint{2.939534in}{1.005664in}}%
\pgfpathlineto{\pgfqpoint{2.939534in}{1.001406in}}%
\pgfpathmoveto{\pgfqpoint{2.935276in}{1.005664in}}%
\pgfpathlineto{\pgfqpoint{2.935276in}{1.005664in}}%
\pgfpathlineto{\pgfqpoint{2.935276in}{1.009922in}}%
\pgfpathlineto{\pgfqpoint{2.939534in}{1.009922in}}%
\pgfpathlineto{\pgfqpoint{2.939534in}{1.005664in}}%
\pgfpathmoveto{\pgfqpoint{2.935276in}{1.009922in}}%
\pgfpathlineto{\pgfqpoint{2.935276in}{1.009922in}}%
\pgfpathlineto{\pgfqpoint{2.935276in}{1.014180in}}%
\pgfpathlineto{\pgfqpoint{2.939534in}{1.014180in}}%
\pgfpathlineto{\pgfqpoint{2.939534in}{1.009922in}}%
\pgfpathmoveto{\pgfqpoint{2.935276in}{1.014180in}}%
\pgfpathlineto{\pgfqpoint{2.935276in}{1.014180in}}%
\pgfpathlineto{\pgfqpoint{2.935276in}{1.018438in}}%
\pgfpathlineto{\pgfqpoint{2.939534in}{1.018438in}}%
\pgfpathlineto{\pgfqpoint{2.939534in}{1.014180in}}%
\pgfpathmoveto{\pgfqpoint{2.935276in}{1.018438in}}%
\pgfpathlineto{\pgfqpoint{2.935276in}{1.018438in}}%
\pgfpathlineto{\pgfqpoint{2.935276in}{1.022696in}}%
\pgfpathlineto{\pgfqpoint{2.939534in}{1.022696in}}%
\pgfpathlineto{\pgfqpoint{2.939534in}{1.018438in}}%
\pgfpathmoveto{\pgfqpoint{2.935276in}{1.022696in}}%
\pgfpathlineto{\pgfqpoint{2.935276in}{1.022696in}}%
\pgfpathlineto{\pgfqpoint{2.935276in}{1.026954in}}%
\pgfpathlineto{\pgfqpoint{2.939534in}{1.026954in}}%
\pgfpathlineto{\pgfqpoint{2.939534in}{1.022696in}}%
\pgfpathmoveto{\pgfqpoint{2.935276in}{1.026954in}}%
\pgfpathlineto{\pgfqpoint{2.935276in}{1.026954in}}%
\pgfpathlineto{\pgfqpoint{2.935276in}{1.031212in}}%
\pgfpathlineto{\pgfqpoint{2.939534in}{1.031212in}}%
\pgfpathlineto{\pgfqpoint{2.939534in}{1.026954in}}%
\pgfpathmoveto{\pgfqpoint{2.935276in}{1.031212in}}%
\pgfpathlineto{\pgfqpoint{2.935276in}{1.031212in}}%
\pgfpathlineto{\pgfqpoint{2.935276in}{1.035470in}}%
\pgfpathlineto{\pgfqpoint{2.939534in}{1.035470in}}%
\pgfpathlineto{\pgfqpoint{2.939534in}{1.031212in}}%
\pgfpathmoveto{\pgfqpoint{2.939534in}{1.031212in}}%
\pgfpathlineto{\pgfqpoint{2.939534in}{1.031212in}}%
\pgfpathlineto{\pgfqpoint{2.939534in}{1.035470in}}%
\pgfpathlineto{\pgfqpoint{2.943792in}{1.035470in}}%
\pgfpathlineto{\pgfqpoint{2.943792in}{1.031212in}}%
\pgfpathmoveto{\pgfqpoint{2.939534in}{1.035470in}}%
\pgfpathlineto{\pgfqpoint{2.939534in}{1.035470in}}%
\pgfpathlineto{\pgfqpoint{2.939534in}{1.039728in}}%
\pgfpathlineto{\pgfqpoint{2.943792in}{1.039728in}}%
\pgfpathlineto{\pgfqpoint{2.943792in}{1.035470in}}%
\pgfpathmoveto{\pgfqpoint{2.939534in}{1.039728in}}%
\pgfpathlineto{\pgfqpoint{2.939534in}{1.039728in}}%
\pgfpathlineto{\pgfqpoint{2.939534in}{1.043986in}}%
\pgfpathlineto{\pgfqpoint{2.943792in}{1.043986in}}%
\pgfpathlineto{\pgfqpoint{2.943792in}{1.039728in}}%
\pgfpathmoveto{\pgfqpoint{2.939534in}{1.043986in}}%
\pgfpathlineto{\pgfqpoint{2.939534in}{1.043986in}}%
\pgfpathlineto{\pgfqpoint{2.939534in}{1.048244in}}%
\pgfpathlineto{\pgfqpoint{2.943792in}{1.048244in}}%
\pgfpathlineto{\pgfqpoint{2.943792in}{1.043986in}}%
\pgfpathmoveto{\pgfqpoint{2.939534in}{1.048244in}}%
\pgfpathlineto{\pgfqpoint{2.939534in}{1.048244in}}%
\pgfpathlineto{\pgfqpoint{2.939534in}{1.052502in}}%
\pgfpathlineto{\pgfqpoint{2.943792in}{1.052502in}}%
\pgfpathlineto{\pgfqpoint{2.943792in}{1.048244in}}%
\pgfpathmoveto{\pgfqpoint{2.939534in}{1.052502in}}%
\pgfpathlineto{\pgfqpoint{2.939534in}{1.052502in}}%
\pgfpathlineto{\pgfqpoint{2.939534in}{1.056760in}}%
\pgfpathlineto{\pgfqpoint{2.943792in}{1.056760in}}%
\pgfpathlineto{\pgfqpoint{2.943792in}{1.052502in}}%
\pgfpathmoveto{\pgfqpoint{2.939534in}{1.056760in}}%
\pgfpathlineto{\pgfqpoint{2.939534in}{1.056760in}}%
\pgfpathlineto{\pgfqpoint{2.939534in}{1.061018in}}%
\pgfpathlineto{\pgfqpoint{2.943792in}{1.061018in}}%
\pgfpathlineto{\pgfqpoint{2.943792in}{1.056760in}}%
\pgfpathmoveto{\pgfqpoint{2.939534in}{1.061018in}}%
\pgfpathlineto{\pgfqpoint{2.939534in}{1.061018in}}%
\pgfpathlineto{\pgfqpoint{2.939534in}{1.065276in}}%
\pgfpathlineto{\pgfqpoint{2.943792in}{1.065276in}}%
\pgfpathlineto{\pgfqpoint{2.943792in}{1.061018in}}%
\pgfpathmoveto{\pgfqpoint{2.939534in}{1.065276in}}%
\pgfpathlineto{\pgfqpoint{2.939534in}{1.065276in}}%
\pgfpathlineto{\pgfqpoint{2.939534in}{1.069533in}}%
\pgfpathlineto{\pgfqpoint{2.943792in}{1.069533in}}%
\pgfpathlineto{\pgfqpoint{2.943792in}{1.065276in}}%
\pgfpathmoveto{\pgfqpoint{2.939534in}{1.069533in}}%
\pgfpathlineto{\pgfqpoint{2.939534in}{1.069533in}}%
\pgfpathlineto{\pgfqpoint{2.939534in}{1.073791in}}%
\pgfpathlineto{\pgfqpoint{2.943792in}{1.073791in}}%
\pgfpathlineto{\pgfqpoint{2.943792in}{1.069533in}}%
\pgfpathmoveto{\pgfqpoint{2.943792in}{1.069533in}}%
\pgfpathlineto{\pgfqpoint{2.943792in}{1.069533in}}%
\pgfpathlineto{\pgfqpoint{2.943792in}{1.073791in}}%
\pgfpathlineto{\pgfqpoint{2.948050in}{1.073791in}}%
\pgfpathlineto{\pgfqpoint{2.948050in}{1.069533in}}%
\pgfpathmoveto{\pgfqpoint{2.943792in}{1.073791in}}%
\pgfpathlineto{\pgfqpoint{2.943792in}{1.073791in}}%
\pgfpathlineto{\pgfqpoint{2.943792in}{1.078049in}}%
\pgfpathlineto{\pgfqpoint{2.948050in}{1.078049in}}%
\pgfpathlineto{\pgfqpoint{2.948050in}{1.073791in}}%
\pgfpathmoveto{\pgfqpoint{2.943792in}{1.078049in}}%
\pgfpathlineto{\pgfqpoint{2.943792in}{1.078049in}}%
\pgfpathlineto{\pgfqpoint{2.943792in}{1.082307in}}%
\pgfpathlineto{\pgfqpoint{2.948050in}{1.082307in}}%
\pgfpathlineto{\pgfqpoint{2.948050in}{1.078049in}}%
\pgfpathmoveto{\pgfqpoint{2.943792in}{1.082307in}}%
\pgfpathlineto{\pgfqpoint{2.943792in}{1.082307in}}%
\pgfpathlineto{\pgfqpoint{2.943792in}{1.086565in}}%
\pgfpathlineto{\pgfqpoint{2.948050in}{1.086565in}}%
\pgfpathlineto{\pgfqpoint{2.948050in}{1.082307in}}%
\pgfpathmoveto{\pgfqpoint{2.943792in}{1.086565in}}%
\pgfpathlineto{\pgfqpoint{2.943792in}{1.086565in}}%
\pgfpathlineto{\pgfqpoint{2.943792in}{1.090823in}}%
\pgfpathlineto{\pgfqpoint{2.948050in}{1.090823in}}%
\pgfpathlineto{\pgfqpoint{2.948050in}{1.086565in}}%
\pgfpathmoveto{\pgfqpoint{2.943792in}{1.090823in}}%
\pgfpathlineto{\pgfqpoint{2.943792in}{1.090823in}}%
\pgfpathlineto{\pgfqpoint{2.943792in}{1.095081in}}%
\pgfpathlineto{\pgfqpoint{2.948050in}{1.095081in}}%
\pgfpathlineto{\pgfqpoint{2.948050in}{1.090823in}}%
\pgfpathmoveto{\pgfqpoint{2.943792in}{1.095081in}}%
\pgfpathlineto{\pgfqpoint{2.943792in}{1.095081in}}%
\pgfpathlineto{\pgfqpoint{2.943792in}{1.099339in}}%
\pgfpathlineto{\pgfqpoint{2.948050in}{1.099339in}}%
\pgfpathlineto{\pgfqpoint{2.948050in}{1.095081in}}%
\pgfpathmoveto{\pgfqpoint{2.943792in}{1.099339in}}%
\pgfpathlineto{\pgfqpoint{2.943792in}{1.099339in}}%
\pgfpathlineto{\pgfqpoint{2.943792in}{1.103597in}}%
\pgfpathlineto{\pgfqpoint{2.948050in}{1.103597in}}%
\pgfpathlineto{\pgfqpoint{2.948050in}{1.099339in}}%
\pgfpathmoveto{\pgfqpoint{2.943792in}{1.103597in}}%
\pgfpathlineto{\pgfqpoint{2.943792in}{1.103597in}}%
\pgfpathlineto{\pgfqpoint{2.943792in}{1.107854in}}%
\pgfpathlineto{\pgfqpoint{2.948050in}{1.107854in}}%
\pgfpathlineto{\pgfqpoint{2.948050in}{1.103597in}}%
\pgfpathmoveto{\pgfqpoint{2.943792in}{1.107854in}}%
\pgfpathlineto{\pgfqpoint{2.943792in}{1.107854in}}%
\pgfpathlineto{\pgfqpoint{2.943792in}{1.112112in}}%
\pgfpathlineto{\pgfqpoint{2.948050in}{1.112112in}}%
\pgfpathlineto{\pgfqpoint{2.948050in}{1.107854in}}%
\pgfpathmoveto{\pgfqpoint{2.948050in}{1.107854in}}%
\pgfpathlineto{\pgfqpoint{2.948050in}{1.107854in}}%
\pgfpathlineto{\pgfqpoint{2.948050in}{1.112112in}}%
\pgfpathlineto{\pgfqpoint{2.952308in}{1.112112in}}%
\pgfpathlineto{\pgfqpoint{2.952308in}{1.107854in}}%
\pgfpathmoveto{\pgfqpoint{2.948050in}{1.112112in}}%
\pgfpathlineto{\pgfqpoint{2.948050in}{1.112112in}}%
\pgfpathlineto{\pgfqpoint{2.948050in}{1.116370in}}%
\pgfpathlineto{\pgfqpoint{2.952308in}{1.116370in}}%
\pgfpathlineto{\pgfqpoint{2.952308in}{1.112112in}}%
\pgfpathmoveto{\pgfqpoint{2.948050in}{1.116370in}}%
\pgfpathlineto{\pgfqpoint{2.948050in}{1.116370in}}%
\pgfpathlineto{\pgfqpoint{2.948050in}{1.120628in}}%
\pgfpathlineto{\pgfqpoint{2.952308in}{1.120628in}}%
\pgfpathlineto{\pgfqpoint{2.952308in}{1.116370in}}%
\pgfpathmoveto{\pgfqpoint{2.948050in}{1.120628in}}%
\pgfpathlineto{\pgfqpoint{2.948050in}{1.120628in}}%
\pgfpathlineto{\pgfqpoint{2.948050in}{1.124886in}}%
\pgfpathlineto{\pgfqpoint{2.952308in}{1.124886in}}%
\pgfpathlineto{\pgfqpoint{2.952308in}{1.120628in}}%
\pgfpathmoveto{\pgfqpoint{2.948050in}{1.124886in}}%
\pgfpathlineto{\pgfqpoint{2.948050in}{1.124886in}}%
\pgfpathlineto{\pgfqpoint{2.948050in}{1.129144in}}%
\pgfpathlineto{\pgfqpoint{2.952308in}{1.129144in}}%
\pgfpathlineto{\pgfqpoint{2.952308in}{1.124886in}}%
\pgfpathmoveto{\pgfqpoint{2.948050in}{1.129144in}}%
\pgfpathlineto{\pgfqpoint{2.948050in}{1.129144in}}%
\pgfpathlineto{\pgfqpoint{2.948050in}{1.133402in}}%
\pgfpathlineto{\pgfqpoint{2.952308in}{1.133402in}}%
\pgfpathlineto{\pgfqpoint{2.952308in}{1.129144in}}%
\pgfpathmoveto{\pgfqpoint{2.948050in}{1.133402in}}%
\pgfpathlineto{\pgfqpoint{2.948050in}{1.133402in}}%
\pgfpathlineto{\pgfqpoint{2.948050in}{1.137660in}}%
\pgfpathlineto{\pgfqpoint{2.952308in}{1.137660in}}%
\pgfpathlineto{\pgfqpoint{2.952308in}{1.133402in}}%
\pgfpathmoveto{\pgfqpoint{2.948050in}{1.137660in}}%
\pgfpathlineto{\pgfqpoint{2.948050in}{1.137660in}}%
\pgfpathlineto{\pgfqpoint{2.948050in}{1.141917in}}%
\pgfpathlineto{\pgfqpoint{2.952308in}{1.141917in}}%
\pgfpathlineto{\pgfqpoint{2.952308in}{1.137660in}}%
\pgfpathmoveto{\pgfqpoint{2.948050in}{1.141917in}}%
\pgfpathlineto{\pgfqpoint{2.948050in}{1.141917in}}%
\pgfpathlineto{\pgfqpoint{2.948050in}{1.146175in}}%
\pgfpathlineto{\pgfqpoint{2.952308in}{1.146175in}}%
\pgfpathlineto{\pgfqpoint{2.952308in}{1.141917in}}%
\pgfpathmoveto{\pgfqpoint{2.948050in}{1.146175in}}%
\pgfpathlineto{\pgfqpoint{2.948050in}{1.146175in}}%
\pgfpathlineto{\pgfqpoint{2.948050in}{1.150433in}}%
\pgfpathlineto{\pgfqpoint{2.952308in}{1.150433in}}%
\pgfpathlineto{\pgfqpoint{2.952308in}{1.146175in}}%
\pgfpathmoveto{\pgfqpoint{2.952308in}{1.146175in}}%
\pgfpathlineto{\pgfqpoint{2.952308in}{1.146175in}}%
\pgfpathlineto{\pgfqpoint{2.952308in}{1.150433in}}%
\pgfpathlineto{\pgfqpoint{2.956566in}{1.150433in}}%
\pgfpathlineto{\pgfqpoint{2.956566in}{1.146175in}}%
\pgfpathmoveto{\pgfqpoint{2.952308in}{1.150433in}}%
\pgfpathlineto{\pgfqpoint{2.952308in}{1.150433in}}%
\pgfpathlineto{\pgfqpoint{2.952308in}{1.154691in}}%
\pgfpathlineto{\pgfqpoint{2.956566in}{1.154691in}}%
\pgfpathlineto{\pgfqpoint{2.956566in}{1.150433in}}%
\pgfpathmoveto{\pgfqpoint{2.952308in}{1.154691in}}%
\pgfpathlineto{\pgfqpoint{2.952308in}{1.154691in}}%
\pgfpathlineto{\pgfqpoint{2.952308in}{1.158949in}}%
\pgfpathlineto{\pgfqpoint{2.956566in}{1.158949in}}%
\pgfpathlineto{\pgfqpoint{2.956566in}{1.154691in}}%
\pgfpathmoveto{\pgfqpoint{2.952308in}{1.158949in}}%
\pgfpathlineto{\pgfqpoint{2.952308in}{1.158949in}}%
\pgfpathlineto{\pgfqpoint{2.952308in}{1.163207in}}%
\pgfpathlineto{\pgfqpoint{2.956566in}{1.163207in}}%
\pgfpathlineto{\pgfqpoint{2.956566in}{1.158949in}}%
\pgfpathmoveto{\pgfqpoint{2.952308in}{1.163207in}}%
\pgfpathlineto{\pgfqpoint{2.952308in}{1.163207in}}%
\pgfpathlineto{\pgfqpoint{2.952308in}{1.167465in}}%
\pgfpathlineto{\pgfqpoint{2.956566in}{1.167465in}}%
\pgfpathlineto{\pgfqpoint{2.956566in}{1.163207in}}%
\pgfpathmoveto{\pgfqpoint{2.952308in}{1.167465in}}%
\pgfpathlineto{\pgfqpoint{2.952308in}{1.167465in}}%
\pgfpathlineto{\pgfqpoint{2.952308in}{1.171723in}}%
\pgfpathlineto{\pgfqpoint{2.956566in}{1.171723in}}%
\pgfpathlineto{\pgfqpoint{2.956566in}{1.167465in}}%
\pgfpathmoveto{\pgfqpoint{2.952308in}{1.171723in}}%
\pgfpathlineto{\pgfqpoint{2.952308in}{1.171723in}}%
\pgfpathlineto{\pgfqpoint{2.952308in}{1.175981in}}%
\pgfpathlineto{\pgfqpoint{2.956566in}{1.175981in}}%
\pgfpathlineto{\pgfqpoint{2.956566in}{1.171723in}}%
\pgfpathmoveto{\pgfqpoint{2.952308in}{1.175981in}}%
\pgfpathlineto{\pgfqpoint{2.952308in}{1.175981in}}%
\pgfpathlineto{\pgfqpoint{2.952308in}{1.180238in}}%
\pgfpathlineto{\pgfqpoint{2.956566in}{1.180238in}}%
\pgfpathlineto{\pgfqpoint{2.956566in}{1.175981in}}%
\pgfpathmoveto{\pgfqpoint{2.952308in}{1.180238in}}%
\pgfpathlineto{\pgfqpoint{2.952308in}{1.180238in}}%
\pgfpathlineto{\pgfqpoint{2.952308in}{1.184496in}}%
\pgfpathlineto{\pgfqpoint{2.956566in}{1.184496in}}%
\pgfpathlineto{\pgfqpoint{2.956566in}{1.180238in}}%
\pgfpathmoveto{\pgfqpoint{2.952308in}{1.184496in}}%
\pgfpathlineto{\pgfqpoint{2.952308in}{1.184496in}}%
\pgfpathlineto{\pgfqpoint{2.952308in}{1.188754in}}%
\pgfpathlineto{\pgfqpoint{2.956566in}{1.188754in}}%
\pgfpathlineto{\pgfqpoint{2.956566in}{1.184496in}}%
\pgfpathmoveto{\pgfqpoint{2.952308in}{1.188754in}}%
\pgfpathlineto{\pgfqpoint{2.952308in}{1.188754in}}%
\pgfpathlineto{\pgfqpoint{2.952308in}{1.193012in}}%
\pgfpathlineto{\pgfqpoint{2.956566in}{1.193012in}}%
\pgfpathlineto{\pgfqpoint{2.956566in}{1.188754in}}%
\pgfpathmoveto{\pgfqpoint{2.956566in}{1.188754in}}%
\pgfpathlineto{\pgfqpoint{2.956566in}{1.188754in}}%
\pgfpathlineto{\pgfqpoint{2.956566in}{1.193012in}}%
\pgfpathlineto{\pgfqpoint{2.960824in}{1.193012in}}%
\pgfpathlineto{\pgfqpoint{2.960824in}{1.188754in}}%
\pgfpathmoveto{\pgfqpoint{2.956566in}{1.193012in}}%
\pgfpathlineto{\pgfqpoint{2.956566in}{1.193012in}}%
\pgfpathlineto{\pgfqpoint{2.956566in}{1.197269in}}%
\pgfpathlineto{\pgfqpoint{2.960824in}{1.197269in}}%
\pgfpathlineto{\pgfqpoint{2.960824in}{1.193012in}}%
\pgfpathmoveto{\pgfqpoint{2.956566in}{1.197269in}}%
\pgfpathlineto{\pgfqpoint{2.956566in}{1.197269in}}%
\pgfpathlineto{\pgfqpoint{2.956566in}{1.201527in}}%
\pgfpathlineto{\pgfqpoint{2.960824in}{1.201527in}}%
\pgfpathlineto{\pgfqpoint{2.960824in}{1.197269in}}%
\pgfpathmoveto{\pgfqpoint{2.956566in}{1.201527in}}%
\pgfpathlineto{\pgfqpoint{2.956566in}{1.201527in}}%
\pgfpathlineto{\pgfqpoint{2.956566in}{1.205784in}}%
\pgfpathlineto{\pgfqpoint{2.960824in}{1.205784in}}%
\pgfpathlineto{\pgfqpoint{2.960824in}{1.201527in}}%
\pgfpathmoveto{\pgfqpoint{2.956566in}{1.205784in}}%
\pgfpathlineto{\pgfqpoint{2.956566in}{1.205784in}}%
\pgfpathlineto{\pgfqpoint{2.956566in}{1.210042in}}%
\pgfpathlineto{\pgfqpoint{2.960824in}{1.210042in}}%
\pgfpathlineto{\pgfqpoint{2.960824in}{1.205784in}}%
\pgfpathmoveto{\pgfqpoint{2.956566in}{1.210042in}}%
\pgfpathlineto{\pgfqpoint{2.956566in}{1.210042in}}%
\pgfpathlineto{\pgfqpoint{2.956566in}{1.214300in}}%
\pgfpathlineto{\pgfqpoint{2.960824in}{1.214300in}}%
\pgfpathlineto{\pgfqpoint{2.960824in}{1.210042in}}%
\pgfpathmoveto{\pgfqpoint{2.956566in}{1.214300in}}%
\pgfpathlineto{\pgfqpoint{2.956566in}{1.214300in}}%
\pgfpathlineto{\pgfqpoint{2.956566in}{1.218557in}}%
\pgfpathlineto{\pgfqpoint{2.960824in}{1.218557in}}%
\pgfpathlineto{\pgfqpoint{2.960824in}{1.214300in}}%
\pgfpathmoveto{\pgfqpoint{2.956566in}{1.218557in}}%
\pgfpathlineto{\pgfqpoint{2.956566in}{1.218557in}}%
\pgfpathlineto{\pgfqpoint{2.956566in}{1.222815in}}%
\pgfpathlineto{\pgfqpoint{2.960824in}{1.222815in}}%
\pgfpathlineto{\pgfqpoint{2.960824in}{1.218557in}}%
\pgfpathmoveto{\pgfqpoint{2.956566in}{1.222815in}}%
\pgfpathlineto{\pgfqpoint{2.956566in}{1.222815in}}%
\pgfpathlineto{\pgfqpoint{2.956566in}{1.227072in}}%
\pgfpathlineto{\pgfqpoint{2.960824in}{1.227072in}}%
\pgfpathlineto{\pgfqpoint{2.960824in}{1.222815in}}%
\pgfpathmoveto{\pgfqpoint{2.960824in}{1.222815in}}%
\pgfpathlineto{\pgfqpoint{2.960824in}{1.222815in}}%
\pgfpathlineto{\pgfqpoint{2.960824in}{1.227072in}}%
\pgfpathlineto{\pgfqpoint{2.965081in}{1.227072in}}%
\pgfpathlineto{\pgfqpoint{2.965081in}{1.222815in}}%
\pgfpathmoveto{\pgfqpoint{2.960824in}{1.227072in}}%
\pgfpathlineto{\pgfqpoint{2.960824in}{1.227072in}}%
\pgfpathlineto{\pgfqpoint{2.960824in}{1.231330in}}%
\pgfpathlineto{\pgfqpoint{2.965081in}{1.231330in}}%
\pgfpathlineto{\pgfqpoint{2.965081in}{1.227072in}}%
\pgfpathmoveto{\pgfqpoint{2.960824in}{1.231330in}}%
\pgfpathlineto{\pgfqpoint{2.960824in}{1.231330in}}%
\pgfpathlineto{\pgfqpoint{2.960824in}{1.235587in}}%
\pgfpathlineto{\pgfqpoint{2.965081in}{1.235587in}}%
\pgfpathlineto{\pgfqpoint{2.965081in}{1.231330in}}%
\pgfpathmoveto{\pgfqpoint{2.960824in}{1.235587in}}%
\pgfpathlineto{\pgfqpoint{2.960824in}{1.235587in}}%
\pgfpathlineto{\pgfqpoint{2.960824in}{1.239845in}}%
\pgfpathlineto{\pgfqpoint{2.965081in}{1.239845in}}%
\pgfpathlineto{\pgfqpoint{2.965081in}{1.235587in}}%
\pgfpathmoveto{\pgfqpoint{2.960824in}{1.239845in}}%
\pgfpathlineto{\pgfqpoint{2.960824in}{1.239845in}}%
\pgfpathlineto{\pgfqpoint{2.960824in}{1.244103in}}%
\pgfpathlineto{\pgfqpoint{2.965081in}{1.244103in}}%
\pgfpathlineto{\pgfqpoint{2.965081in}{1.239845in}}%
\pgfpathmoveto{\pgfqpoint{2.960824in}{1.244103in}}%
\pgfpathlineto{\pgfqpoint{2.960824in}{1.244103in}}%
\pgfpathlineto{\pgfqpoint{2.960824in}{1.248360in}}%
\pgfpathlineto{\pgfqpoint{2.965081in}{1.248360in}}%
\pgfpathlineto{\pgfqpoint{2.965081in}{1.244103in}}%
\pgfpathmoveto{\pgfqpoint{2.960824in}{1.248360in}}%
\pgfpathlineto{\pgfqpoint{2.960824in}{1.248360in}}%
\pgfpathlineto{\pgfqpoint{2.960824in}{1.252618in}}%
\pgfpathlineto{\pgfqpoint{2.965081in}{1.252618in}}%
\pgfpathlineto{\pgfqpoint{2.965081in}{1.248360in}}%
\pgfpathmoveto{\pgfqpoint{2.960824in}{1.252618in}}%
\pgfpathlineto{\pgfqpoint{2.960824in}{1.252618in}}%
\pgfpathlineto{\pgfqpoint{2.960824in}{1.256875in}}%
\pgfpathlineto{\pgfqpoint{2.965081in}{1.256875in}}%
\pgfpathlineto{\pgfqpoint{2.965081in}{1.252618in}}%
\pgfpathmoveto{\pgfqpoint{2.960824in}{1.256875in}}%
\pgfpathlineto{\pgfqpoint{2.960824in}{1.256875in}}%
\pgfpathlineto{\pgfqpoint{2.960824in}{1.261133in}}%
\pgfpathlineto{\pgfqpoint{2.965081in}{1.261133in}}%
\pgfpathlineto{\pgfqpoint{2.965081in}{1.256875in}}%
\pgfpathmoveto{\pgfqpoint{2.960824in}{1.261133in}}%
\pgfpathlineto{\pgfqpoint{2.960824in}{1.261133in}}%
\pgfpathlineto{\pgfqpoint{2.960824in}{1.265390in}}%
\pgfpathlineto{\pgfqpoint{2.965081in}{1.265390in}}%
\pgfpathlineto{\pgfqpoint{2.965081in}{1.261133in}}%
\pgfpathmoveto{\pgfqpoint{2.965081in}{1.261133in}}%
\pgfpathlineto{\pgfqpoint{2.965081in}{1.261133in}}%
\pgfpathlineto{\pgfqpoint{2.965081in}{1.265390in}}%
\pgfpathlineto{\pgfqpoint{2.969339in}{1.265390in}}%
\pgfpathlineto{\pgfqpoint{2.969339in}{1.261133in}}%
\pgfpathmoveto{\pgfqpoint{2.965081in}{1.265390in}}%
\pgfpathlineto{\pgfqpoint{2.965081in}{1.265390in}}%
\pgfpathlineto{\pgfqpoint{2.965081in}{1.269648in}}%
\pgfpathlineto{\pgfqpoint{2.969339in}{1.269648in}}%
\pgfpathlineto{\pgfqpoint{2.969339in}{1.265390in}}%
\pgfpathmoveto{\pgfqpoint{2.965081in}{1.269648in}}%
\pgfpathlineto{\pgfqpoint{2.965081in}{1.269648in}}%
\pgfpathlineto{\pgfqpoint{2.965081in}{1.273906in}}%
\pgfpathlineto{\pgfqpoint{2.969339in}{1.273906in}}%
\pgfpathlineto{\pgfqpoint{2.969339in}{1.269648in}}%
\pgfpathmoveto{\pgfqpoint{2.965081in}{1.273906in}}%
\pgfpathlineto{\pgfqpoint{2.965081in}{1.273906in}}%
\pgfpathlineto{\pgfqpoint{2.965081in}{1.278163in}}%
\pgfpathlineto{\pgfqpoint{2.969339in}{1.278163in}}%
\pgfpathlineto{\pgfqpoint{2.969339in}{1.273906in}}%
\pgfpathmoveto{\pgfqpoint{2.965081in}{1.278163in}}%
\pgfpathlineto{\pgfqpoint{2.965081in}{1.278163in}}%
\pgfpathlineto{\pgfqpoint{2.965081in}{1.282421in}}%
\pgfpathlineto{\pgfqpoint{2.969339in}{1.282421in}}%
\pgfpathlineto{\pgfqpoint{2.969339in}{1.278163in}}%
\pgfpathmoveto{\pgfqpoint{2.965081in}{1.282421in}}%
\pgfpathlineto{\pgfqpoint{2.965081in}{1.282421in}}%
\pgfpathlineto{\pgfqpoint{2.965081in}{1.286678in}}%
\pgfpathlineto{\pgfqpoint{2.969339in}{1.286678in}}%
\pgfpathlineto{\pgfqpoint{2.969339in}{1.282421in}}%
\pgfpathmoveto{\pgfqpoint{2.965081in}{1.286678in}}%
\pgfpathlineto{\pgfqpoint{2.965081in}{1.286678in}}%
\pgfpathlineto{\pgfqpoint{2.965081in}{1.290936in}}%
\pgfpathlineto{\pgfqpoint{2.969339in}{1.290936in}}%
\pgfpathlineto{\pgfqpoint{2.969339in}{1.286678in}}%
\pgfpathmoveto{\pgfqpoint{2.965081in}{1.290936in}}%
\pgfpathlineto{\pgfqpoint{2.965081in}{1.290936in}}%
\pgfpathlineto{\pgfqpoint{2.965081in}{1.295193in}}%
\pgfpathlineto{\pgfqpoint{2.969339in}{1.295193in}}%
\pgfpathlineto{\pgfqpoint{2.969339in}{1.290936in}}%
\pgfpathmoveto{\pgfqpoint{2.965081in}{1.295193in}}%
\pgfpathlineto{\pgfqpoint{2.965081in}{1.295193in}}%
\pgfpathlineto{\pgfqpoint{2.965081in}{1.299451in}}%
\pgfpathlineto{\pgfqpoint{2.969339in}{1.299451in}}%
\pgfpathlineto{\pgfqpoint{2.969339in}{1.295193in}}%
\pgfpathmoveto{\pgfqpoint{2.965081in}{1.299451in}}%
\pgfpathlineto{\pgfqpoint{2.965081in}{1.299451in}}%
\pgfpathlineto{\pgfqpoint{2.965081in}{1.303709in}}%
\pgfpathlineto{\pgfqpoint{2.969339in}{1.303709in}}%
\pgfpathlineto{\pgfqpoint{2.969339in}{1.299451in}}%
\pgfpathmoveto{\pgfqpoint{2.969339in}{1.299451in}}%
\pgfpathlineto{\pgfqpoint{2.969339in}{1.299451in}}%
\pgfpathlineto{\pgfqpoint{2.969339in}{1.303709in}}%
\pgfpathlineto{\pgfqpoint{2.973597in}{1.303709in}}%
\pgfpathlineto{\pgfqpoint{2.973597in}{1.299451in}}%
\pgfpathmoveto{\pgfqpoint{2.969339in}{1.303709in}}%
\pgfpathlineto{\pgfqpoint{2.969339in}{1.303709in}}%
\pgfpathlineto{\pgfqpoint{2.969339in}{1.307966in}}%
\pgfpathlineto{\pgfqpoint{2.973597in}{1.307966in}}%
\pgfpathlineto{\pgfqpoint{2.973597in}{1.303709in}}%
\pgfpathmoveto{\pgfqpoint{2.969339in}{1.307966in}}%
\pgfpathlineto{\pgfqpoint{2.969339in}{1.307966in}}%
\pgfpathlineto{\pgfqpoint{2.969339in}{1.312224in}}%
\pgfpathlineto{\pgfqpoint{2.973597in}{1.312224in}}%
\pgfpathlineto{\pgfqpoint{2.973597in}{1.307966in}}%
\pgfpathmoveto{\pgfqpoint{2.969339in}{1.312224in}}%
\pgfpathlineto{\pgfqpoint{2.969339in}{1.312224in}}%
\pgfpathlineto{\pgfqpoint{2.969339in}{1.316481in}}%
\pgfpathlineto{\pgfqpoint{2.973597in}{1.316481in}}%
\pgfpathlineto{\pgfqpoint{2.973597in}{1.312224in}}%
\pgfpathmoveto{\pgfqpoint{2.969339in}{1.316481in}}%
\pgfpathlineto{\pgfqpoint{2.969339in}{1.316481in}}%
\pgfpathlineto{\pgfqpoint{2.969339in}{1.320739in}}%
\pgfpathlineto{\pgfqpoint{2.973597in}{1.320739in}}%
\pgfpathlineto{\pgfqpoint{2.973597in}{1.316481in}}%
\pgfpathmoveto{\pgfqpoint{2.969339in}{1.320739in}}%
\pgfpathlineto{\pgfqpoint{2.969339in}{1.320739in}}%
\pgfpathlineto{\pgfqpoint{2.969339in}{1.324996in}}%
\pgfpathlineto{\pgfqpoint{2.973597in}{1.324996in}}%
\pgfpathlineto{\pgfqpoint{2.973597in}{1.320739in}}%
\pgfpathmoveto{\pgfqpoint{2.969339in}{1.324996in}}%
\pgfpathlineto{\pgfqpoint{2.969339in}{1.324996in}}%
\pgfpathlineto{\pgfqpoint{2.969339in}{1.329254in}}%
\pgfpathlineto{\pgfqpoint{2.973597in}{1.329254in}}%
\pgfpathlineto{\pgfqpoint{2.973597in}{1.324996in}}%
\pgfpathmoveto{\pgfqpoint{2.969339in}{1.329254in}}%
\pgfpathlineto{\pgfqpoint{2.969339in}{1.329254in}}%
\pgfpathlineto{\pgfqpoint{2.969339in}{1.333512in}}%
\pgfpathlineto{\pgfqpoint{2.973597in}{1.333512in}}%
\pgfpathlineto{\pgfqpoint{2.973597in}{1.329254in}}%
\pgfpathmoveto{\pgfqpoint{2.969339in}{1.333512in}}%
\pgfpathlineto{\pgfqpoint{2.969339in}{1.333512in}}%
\pgfpathlineto{\pgfqpoint{2.969339in}{1.337770in}}%
\pgfpathlineto{\pgfqpoint{2.973597in}{1.337770in}}%
\pgfpathlineto{\pgfqpoint{2.973597in}{1.333512in}}%
\pgfpathmoveto{\pgfqpoint{2.969339in}{1.337770in}}%
\pgfpathlineto{\pgfqpoint{2.969339in}{1.337770in}}%
\pgfpathlineto{\pgfqpoint{2.969339in}{1.342028in}}%
\pgfpathlineto{\pgfqpoint{2.973597in}{1.342028in}}%
\pgfpathlineto{\pgfqpoint{2.973597in}{1.337770in}}%
\pgfpathmoveto{\pgfqpoint{2.973597in}{1.337770in}}%
\pgfpathlineto{\pgfqpoint{2.973597in}{1.337770in}}%
\pgfpathlineto{\pgfqpoint{2.973597in}{1.342028in}}%
\pgfpathlineto{\pgfqpoint{2.977855in}{1.342028in}}%
\pgfpathlineto{\pgfqpoint{2.977855in}{1.337770in}}%
\pgfpathmoveto{\pgfqpoint{2.973597in}{1.342028in}}%
\pgfpathlineto{\pgfqpoint{2.973597in}{1.342028in}}%
\pgfpathlineto{\pgfqpoint{2.973597in}{1.346286in}}%
\pgfpathlineto{\pgfqpoint{2.977855in}{1.346286in}}%
\pgfpathlineto{\pgfqpoint{2.977855in}{1.342028in}}%
\pgfpathmoveto{\pgfqpoint{2.973597in}{1.346286in}}%
\pgfpathlineto{\pgfqpoint{2.973597in}{1.346286in}}%
\pgfpathlineto{\pgfqpoint{2.973597in}{1.350544in}}%
\pgfpathlineto{\pgfqpoint{2.977855in}{1.350544in}}%
\pgfpathlineto{\pgfqpoint{2.977855in}{1.346286in}}%
\pgfpathmoveto{\pgfqpoint{2.973597in}{1.350544in}}%
\pgfpathlineto{\pgfqpoint{2.973597in}{1.350544in}}%
\pgfpathlineto{\pgfqpoint{2.973597in}{1.354802in}}%
\pgfpathlineto{\pgfqpoint{2.977855in}{1.354802in}}%
\pgfpathlineto{\pgfqpoint{2.977855in}{1.350544in}}%
\pgfpathmoveto{\pgfqpoint{2.973597in}{1.354802in}}%
\pgfpathlineto{\pgfqpoint{2.973597in}{1.354802in}}%
\pgfpathlineto{\pgfqpoint{2.973597in}{1.359059in}}%
\pgfpathlineto{\pgfqpoint{2.977855in}{1.359059in}}%
\pgfpathlineto{\pgfqpoint{2.977855in}{1.354802in}}%
\pgfpathmoveto{\pgfqpoint{2.973597in}{1.359059in}}%
\pgfpathlineto{\pgfqpoint{2.973597in}{1.359059in}}%
\pgfpathlineto{\pgfqpoint{2.973597in}{1.363317in}}%
\pgfpathlineto{\pgfqpoint{2.977855in}{1.363317in}}%
\pgfpathlineto{\pgfqpoint{2.977855in}{1.359059in}}%
\pgfpathmoveto{\pgfqpoint{2.973597in}{1.363317in}}%
\pgfpathlineto{\pgfqpoint{2.973597in}{1.363317in}}%
\pgfpathlineto{\pgfqpoint{2.973597in}{1.367575in}}%
\pgfpathlineto{\pgfqpoint{2.977855in}{1.367575in}}%
\pgfpathlineto{\pgfqpoint{2.977855in}{1.363317in}}%
\pgfpathmoveto{\pgfqpoint{2.973597in}{1.367575in}}%
\pgfpathlineto{\pgfqpoint{2.973597in}{1.367575in}}%
\pgfpathlineto{\pgfqpoint{2.973597in}{1.371833in}}%
\pgfpathlineto{\pgfqpoint{2.977855in}{1.371833in}}%
\pgfpathlineto{\pgfqpoint{2.977855in}{1.367575in}}%
\pgfpathmoveto{\pgfqpoint{2.973597in}{1.371833in}}%
\pgfpathlineto{\pgfqpoint{2.973597in}{1.371833in}}%
\pgfpathlineto{\pgfqpoint{2.973597in}{1.376091in}}%
\pgfpathlineto{\pgfqpoint{2.977855in}{1.376091in}}%
\pgfpathlineto{\pgfqpoint{2.977855in}{1.371833in}}%
\pgfpathmoveto{\pgfqpoint{2.973597in}{1.376091in}}%
\pgfpathlineto{\pgfqpoint{2.973597in}{1.376091in}}%
\pgfpathlineto{\pgfqpoint{2.973597in}{1.380349in}}%
\pgfpathlineto{\pgfqpoint{2.977855in}{1.380349in}}%
\pgfpathlineto{\pgfqpoint{2.977855in}{1.376091in}}%
\pgfpathmoveto{\pgfqpoint{2.977855in}{1.376091in}}%
\pgfpathlineto{\pgfqpoint{2.977855in}{1.376091in}}%
\pgfpathlineto{\pgfqpoint{2.977855in}{1.380349in}}%
\pgfpathlineto{\pgfqpoint{2.982113in}{1.380349in}}%
\pgfpathlineto{\pgfqpoint{2.982113in}{1.376091in}}%
\pgfpathmoveto{\pgfqpoint{2.977855in}{1.380349in}}%
\pgfpathlineto{\pgfqpoint{2.977855in}{1.380349in}}%
\pgfpathlineto{\pgfqpoint{2.977855in}{1.384607in}}%
\pgfpathlineto{\pgfqpoint{2.982113in}{1.384607in}}%
\pgfpathlineto{\pgfqpoint{2.982113in}{1.380349in}}%
\pgfpathmoveto{\pgfqpoint{2.977855in}{1.384607in}}%
\pgfpathlineto{\pgfqpoint{2.977855in}{1.384607in}}%
\pgfpathlineto{\pgfqpoint{2.977855in}{1.388865in}}%
\pgfpathlineto{\pgfqpoint{2.982113in}{1.388865in}}%
\pgfpathlineto{\pgfqpoint{2.982113in}{1.384607in}}%
\pgfpathmoveto{\pgfqpoint{2.977855in}{1.388865in}}%
\pgfpathlineto{\pgfqpoint{2.977855in}{1.388865in}}%
\pgfpathlineto{\pgfqpoint{2.977855in}{1.393123in}}%
\pgfpathlineto{\pgfqpoint{2.982113in}{1.393123in}}%
\pgfpathlineto{\pgfqpoint{2.982113in}{1.388865in}}%
\pgfpathmoveto{\pgfqpoint{2.977855in}{1.393123in}}%
\pgfpathlineto{\pgfqpoint{2.977855in}{1.393123in}}%
\pgfpathlineto{\pgfqpoint{2.977855in}{1.397380in}}%
\pgfpathlineto{\pgfqpoint{2.982113in}{1.397380in}}%
\pgfpathlineto{\pgfqpoint{2.982113in}{1.393123in}}%
\pgfpathmoveto{\pgfqpoint{2.977855in}{1.397380in}}%
\pgfpathlineto{\pgfqpoint{2.977855in}{1.397380in}}%
\pgfpathlineto{\pgfqpoint{2.977855in}{1.401638in}}%
\pgfpathlineto{\pgfqpoint{2.982113in}{1.401638in}}%
\pgfpathlineto{\pgfqpoint{2.982113in}{1.397380in}}%
\pgfpathmoveto{\pgfqpoint{2.977855in}{1.401638in}}%
\pgfpathlineto{\pgfqpoint{2.977855in}{1.401638in}}%
\pgfpathlineto{\pgfqpoint{2.977855in}{1.405896in}}%
\pgfpathlineto{\pgfqpoint{2.982113in}{1.405896in}}%
\pgfpathlineto{\pgfqpoint{2.982113in}{1.401638in}}%
\pgfpathmoveto{\pgfqpoint{2.977855in}{1.405896in}}%
\pgfpathlineto{\pgfqpoint{2.977855in}{1.405896in}}%
\pgfpathlineto{\pgfqpoint{2.977855in}{1.410154in}}%
\pgfpathlineto{\pgfqpoint{2.982113in}{1.410154in}}%
\pgfpathlineto{\pgfqpoint{2.982113in}{1.405896in}}%
\pgfpathmoveto{\pgfqpoint{2.977855in}{1.410154in}}%
\pgfpathlineto{\pgfqpoint{2.977855in}{1.410154in}}%
\pgfpathlineto{\pgfqpoint{2.977855in}{1.414412in}}%
\pgfpathlineto{\pgfqpoint{2.982113in}{1.414412in}}%
\pgfpathlineto{\pgfqpoint{2.982113in}{1.410154in}}%
\pgfpathmoveto{\pgfqpoint{2.982113in}{1.410154in}}%
\pgfpathlineto{\pgfqpoint{2.982113in}{1.410154in}}%
\pgfpathlineto{\pgfqpoint{2.982113in}{1.414412in}}%
\pgfpathlineto{\pgfqpoint{2.986370in}{1.414412in}}%
\pgfpathlineto{\pgfqpoint{2.986370in}{1.410154in}}%
\pgfpathmoveto{\pgfqpoint{2.982113in}{1.414412in}}%
\pgfpathlineto{\pgfqpoint{2.982113in}{1.414412in}}%
\pgfpathlineto{\pgfqpoint{2.982113in}{1.418670in}}%
\pgfpathlineto{\pgfqpoint{2.986370in}{1.418670in}}%
\pgfpathlineto{\pgfqpoint{2.986370in}{1.414412in}}%
\pgfpathmoveto{\pgfqpoint{2.982113in}{1.418670in}}%
\pgfpathlineto{\pgfqpoint{2.982113in}{1.418670in}}%
\pgfpathlineto{\pgfqpoint{2.982113in}{1.422928in}}%
\pgfpathlineto{\pgfqpoint{2.986370in}{1.422928in}}%
\pgfpathlineto{\pgfqpoint{2.986370in}{1.418670in}}%
\pgfpathmoveto{\pgfqpoint{2.982113in}{1.422928in}}%
\pgfpathlineto{\pgfqpoint{2.982113in}{1.422928in}}%
\pgfpathlineto{\pgfqpoint{2.982113in}{1.427186in}}%
\pgfpathlineto{\pgfqpoint{2.986370in}{1.427186in}}%
\pgfpathlineto{\pgfqpoint{2.986370in}{1.422928in}}%
\pgfpathmoveto{\pgfqpoint{2.982113in}{1.427186in}}%
\pgfpathlineto{\pgfqpoint{2.982113in}{1.427186in}}%
\pgfpathlineto{\pgfqpoint{2.982113in}{1.431443in}}%
\pgfpathlineto{\pgfqpoint{2.986370in}{1.431443in}}%
\pgfpathlineto{\pgfqpoint{2.986370in}{1.427186in}}%
\pgfpathmoveto{\pgfqpoint{2.982113in}{1.431443in}}%
\pgfpathlineto{\pgfqpoint{2.982113in}{1.431443in}}%
\pgfpathlineto{\pgfqpoint{2.982113in}{1.435701in}}%
\pgfpathlineto{\pgfqpoint{2.986370in}{1.435701in}}%
\pgfpathlineto{\pgfqpoint{2.986370in}{1.431443in}}%
\pgfpathmoveto{\pgfqpoint{2.982113in}{1.435701in}}%
\pgfpathlineto{\pgfqpoint{2.982113in}{1.435701in}}%
\pgfpathlineto{\pgfqpoint{2.982113in}{1.439959in}}%
\pgfpathlineto{\pgfqpoint{2.986370in}{1.439959in}}%
\pgfpathlineto{\pgfqpoint{2.986370in}{1.435701in}}%
\pgfpathmoveto{\pgfqpoint{2.982113in}{1.439959in}}%
\pgfpathlineto{\pgfqpoint{2.982113in}{1.439959in}}%
\pgfpathlineto{\pgfqpoint{2.982113in}{1.444217in}}%
\pgfpathlineto{\pgfqpoint{2.986370in}{1.444217in}}%
\pgfpathlineto{\pgfqpoint{2.986370in}{1.439959in}}%
\pgfpathmoveto{\pgfqpoint{2.982113in}{1.444217in}}%
\pgfpathlineto{\pgfqpoint{2.982113in}{1.444217in}}%
\pgfpathlineto{\pgfqpoint{2.982113in}{1.448475in}}%
\pgfpathlineto{\pgfqpoint{2.986370in}{1.448475in}}%
\pgfpathlineto{\pgfqpoint{2.986370in}{1.444217in}}%
\pgfpathmoveto{\pgfqpoint{2.982113in}{1.448475in}}%
\pgfpathlineto{\pgfqpoint{2.982113in}{1.448475in}}%
\pgfpathlineto{\pgfqpoint{2.982113in}{1.452733in}}%
\pgfpathlineto{\pgfqpoint{2.986370in}{1.452733in}}%
\pgfpathlineto{\pgfqpoint{2.986370in}{1.448475in}}%
\pgfpathmoveto{\pgfqpoint{2.986370in}{1.448475in}}%
\pgfpathlineto{\pgfqpoint{2.986370in}{1.448475in}}%
\pgfpathlineto{\pgfqpoint{2.986370in}{1.452733in}}%
\pgfpathlineto{\pgfqpoint{2.990628in}{1.452733in}}%
\pgfpathlineto{\pgfqpoint{2.990628in}{1.448475in}}%
\pgfpathmoveto{\pgfqpoint{2.986370in}{1.452733in}}%
\pgfpathlineto{\pgfqpoint{2.986370in}{1.452733in}}%
\pgfpathlineto{\pgfqpoint{2.986370in}{1.456991in}}%
\pgfpathlineto{\pgfqpoint{2.990628in}{1.456991in}}%
\pgfpathlineto{\pgfqpoint{2.990628in}{1.452733in}}%
\pgfpathmoveto{\pgfqpoint{2.986370in}{1.456991in}}%
\pgfpathlineto{\pgfqpoint{2.986370in}{1.456991in}}%
\pgfpathlineto{\pgfqpoint{2.986370in}{1.461249in}}%
\pgfpathlineto{\pgfqpoint{2.990628in}{1.461249in}}%
\pgfpathlineto{\pgfqpoint{2.990628in}{1.456991in}}%
\pgfpathmoveto{\pgfqpoint{2.986370in}{1.461249in}}%
\pgfpathlineto{\pgfqpoint{2.986370in}{1.461249in}}%
\pgfpathlineto{\pgfqpoint{2.986370in}{1.465506in}}%
\pgfpathlineto{\pgfqpoint{2.990628in}{1.465506in}}%
\pgfpathlineto{\pgfqpoint{2.990628in}{1.461249in}}%
\pgfpathmoveto{\pgfqpoint{2.986370in}{1.465506in}}%
\pgfpathlineto{\pgfqpoint{2.986370in}{1.465506in}}%
\pgfpathlineto{\pgfqpoint{2.986370in}{1.469764in}}%
\pgfpathlineto{\pgfqpoint{2.990628in}{1.469764in}}%
\pgfpathlineto{\pgfqpoint{2.990628in}{1.465506in}}%
\pgfpathmoveto{\pgfqpoint{2.986370in}{1.469764in}}%
\pgfpathlineto{\pgfqpoint{2.986370in}{1.469764in}}%
\pgfpathlineto{\pgfqpoint{2.986370in}{1.474022in}}%
\pgfpathlineto{\pgfqpoint{2.990628in}{1.474022in}}%
\pgfpathlineto{\pgfqpoint{2.990628in}{1.469764in}}%
\pgfpathmoveto{\pgfqpoint{2.986370in}{1.474022in}}%
\pgfpathlineto{\pgfqpoint{2.986370in}{1.474022in}}%
\pgfpathlineto{\pgfqpoint{2.986370in}{1.478280in}}%
\pgfpathlineto{\pgfqpoint{2.990628in}{1.478280in}}%
\pgfpathlineto{\pgfqpoint{2.990628in}{1.474022in}}%
\pgfpathmoveto{\pgfqpoint{2.986370in}{1.478280in}}%
\pgfpathlineto{\pgfqpoint{2.986370in}{1.478280in}}%
\pgfpathlineto{\pgfqpoint{2.986370in}{1.482538in}}%
\pgfpathlineto{\pgfqpoint{2.990628in}{1.482538in}}%
\pgfpathlineto{\pgfqpoint{2.990628in}{1.478280in}}%
\pgfpathmoveto{\pgfqpoint{2.986370in}{1.482538in}}%
\pgfpathlineto{\pgfqpoint{2.986370in}{1.482538in}}%
\pgfpathlineto{\pgfqpoint{2.986370in}{1.486796in}}%
\pgfpathlineto{\pgfqpoint{2.990628in}{1.486796in}}%
\pgfpathlineto{\pgfqpoint{2.990628in}{1.482538in}}%
\pgfpathmoveto{\pgfqpoint{2.990628in}{1.482538in}}%
\pgfpathlineto{\pgfqpoint{2.990628in}{1.482538in}}%
\pgfpathlineto{\pgfqpoint{2.990628in}{1.486796in}}%
\pgfpathlineto{\pgfqpoint{2.994886in}{1.486796in}}%
\pgfpathlineto{\pgfqpoint{2.994886in}{1.482538in}}%
\pgfpathmoveto{\pgfqpoint{2.990628in}{1.486796in}}%
\pgfpathlineto{\pgfqpoint{2.990628in}{1.486796in}}%
\pgfpathlineto{\pgfqpoint{2.990628in}{1.491054in}}%
\pgfpathlineto{\pgfqpoint{2.994886in}{1.491054in}}%
\pgfpathlineto{\pgfqpoint{2.994886in}{1.486796in}}%
\pgfpathmoveto{\pgfqpoint{2.990628in}{1.491054in}}%
\pgfpathlineto{\pgfqpoint{2.990628in}{1.491054in}}%
\pgfpathlineto{\pgfqpoint{2.990628in}{1.495311in}}%
\pgfpathlineto{\pgfqpoint{2.994886in}{1.495311in}}%
\pgfpathlineto{\pgfqpoint{2.994886in}{1.491054in}}%
\pgfpathmoveto{\pgfqpoint{2.990628in}{1.495311in}}%
\pgfpathlineto{\pgfqpoint{2.990628in}{1.495311in}}%
\pgfpathlineto{\pgfqpoint{2.990628in}{1.499569in}}%
\pgfpathlineto{\pgfqpoint{2.994886in}{1.499569in}}%
\pgfpathlineto{\pgfqpoint{2.994886in}{1.495311in}}%
\pgfpathmoveto{\pgfqpoint{2.990628in}{1.499569in}}%
\pgfpathlineto{\pgfqpoint{2.990628in}{1.499569in}}%
\pgfpathlineto{\pgfqpoint{2.990628in}{1.503827in}}%
\pgfpathlineto{\pgfqpoint{2.994886in}{1.503827in}}%
\pgfpathlineto{\pgfqpoint{2.994886in}{1.499569in}}%
\pgfpathmoveto{\pgfqpoint{2.990628in}{1.503827in}}%
\pgfpathlineto{\pgfqpoint{2.990628in}{1.503827in}}%
\pgfpathlineto{\pgfqpoint{2.990628in}{1.508085in}}%
\pgfpathlineto{\pgfqpoint{2.994886in}{1.508085in}}%
\pgfpathlineto{\pgfqpoint{2.994886in}{1.503827in}}%
\pgfpathmoveto{\pgfqpoint{2.990628in}{1.508085in}}%
\pgfpathlineto{\pgfqpoint{2.990628in}{1.508085in}}%
\pgfpathlineto{\pgfqpoint{2.990628in}{1.512343in}}%
\pgfpathlineto{\pgfqpoint{2.994886in}{1.512343in}}%
\pgfpathlineto{\pgfqpoint{2.994886in}{1.508085in}}%
\pgfpathmoveto{\pgfqpoint{2.990628in}{1.512343in}}%
\pgfpathlineto{\pgfqpoint{2.990628in}{1.512343in}}%
\pgfpathlineto{\pgfqpoint{2.990628in}{1.516601in}}%
\pgfpathlineto{\pgfqpoint{2.994886in}{1.516601in}}%
\pgfpathlineto{\pgfqpoint{2.994886in}{1.512343in}}%
\pgfpathmoveto{\pgfqpoint{2.990628in}{1.516601in}}%
\pgfpathlineto{\pgfqpoint{2.990628in}{1.516601in}}%
\pgfpathlineto{\pgfqpoint{2.990628in}{1.520858in}}%
\pgfpathlineto{\pgfqpoint{2.994886in}{1.520858in}}%
\pgfpathlineto{\pgfqpoint{2.994886in}{1.516601in}}%
\pgfpathmoveto{\pgfqpoint{2.994886in}{1.516601in}}%
\pgfpathlineto{\pgfqpoint{2.994886in}{1.516601in}}%
\pgfpathlineto{\pgfqpoint{2.994886in}{1.520858in}}%
\pgfpathlineto{\pgfqpoint{2.999144in}{1.520858in}}%
\pgfpathlineto{\pgfqpoint{2.999144in}{1.516601in}}%
\pgfpathmoveto{\pgfqpoint{2.994886in}{1.520858in}}%
\pgfpathlineto{\pgfqpoint{2.994886in}{1.520858in}}%
\pgfpathlineto{\pgfqpoint{2.994886in}{1.525116in}}%
\pgfpathlineto{\pgfqpoint{2.999144in}{1.525116in}}%
\pgfpathlineto{\pgfqpoint{2.999144in}{1.520858in}}%
\pgfpathmoveto{\pgfqpoint{2.994886in}{1.525116in}}%
\pgfpathlineto{\pgfqpoint{2.994886in}{1.525116in}}%
\pgfpathlineto{\pgfqpoint{2.994886in}{1.529374in}}%
\pgfpathlineto{\pgfqpoint{2.999144in}{1.529374in}}%
\pgfpathlineto{\pgfqpoint{2.999144in}{1.525116in}}%
\pgfpathmoveto{\pgfqpoint{2.994886in}{1.529374in}}%
\pgfpathlineto{\pgfqpoint{2.994886in}{1.529374in}}%
\pgfpathlineto{\pgfqpoint{2.994886in}{1.533632in}}%
\pgfpathlineto{\pgfqpoint{2.999144in}{1.533632in}}%
\pgfpathlineto{\pgfqpoint{2.999144in}{1.529374in}}%
\pgfpathmoveto{\pgfqpoint{2.994886in}{1.533632in}}%
\pgfpathlineto{\pgfqpoint{2.994886in}{1.533632in}}%
\pgfpathlineto{\pgfqpoint{2.994886in}{1.537890in}}%
\pgfpathlineto{\pgfqpoint{2.999144in}{1.537890in}}%
\pgfpathlineto{\pgfqpoint{2.999144in}{1.533632in}}%
\pgfpathmoveto{\pgfqpoint{2.994886in}{1.537890in}}%
\pgfpathlineto{\pgfqpoint{2.994886in}{1.537890in}}%
\pgfpathlineto{\pgfqpoint{2.994886in}{1.542148in}}%
\pgfpathlineto{\pgfqpoint{2.999144in}{1.542148in}}%
\pgfpathlineto{\pgfqpoint{2.999144in}{1.537890in}}%
\pgfpathmoveto{\pgfqpoint{2.994886in}{1.542148in}}%
\pgfpathlineto{\pgfqpoint{2.994886in}{1.542148in}}%
\pgfpathlineto{\pgfqpoint{2.994886in}{1.546405in}}%
\pgfpathlineto{\pgfqpoint{2.999144in}{1.546405in}}%
\pgfpathlineto{\pgfqpoint{2.999144in}{1.542148in}}%
\pgfpathmoveto{\pgfqpoint{2.994886in}{1.546405in}}%
\pgfpathlineto{\pgfqpoint{2.994886in}{1.546405in}}%
\pgfpathlineto{\pgfqpoint{2.994886in}{1.550663in}}%
\pgfpathlineto{\pgfqpoint{2.999144in}{1.550663in}}%
\pgfpathlineto{\pgfqpoint{2.999144in}{1.546405in}}%
\pgfpathmoveto{\pgfqpoint{2.994886in}{1.550663in}}%
\pgfpathlineto{\pgfqpoint{2.994886in}{1.550663in}}%
\pgfpathlineto{\pgfqpoint{2.994886in}{1.554921in}}%
\pgfpathlineto{\pgfqpoint{2.999144in}{1.554921in}}%
\pgfpathlineto{\pgfqpoint{2.999144in}{1.550663in}}%
\pgfpathmoveto{\pgfqpoint{2.994886in}{1.554921in}}%
\pgfpathlineto{\pgfqpoint{2.994886in}{1.554921in}}%
\pgfpathlineto{\pgfqpoint{2.994886in}{1.559179in}}%
\pgfpathlineto{\pgfqpoint{2.999144in}{1.559179in}}%
\pgfpathlineto{\pgfqpoint{2.999144in}{1.554921in}}%
\pgfpathmoveto{\pgfqpoint{2.999144in}{1.554921in}}%
\pgfpathlineto{\pgfqpoint{2.999144in}{1.554921in}}%
\pgfpathlineto{\pgfqpoint{2.999144in}{1.559179in}}%
\pgfpathlineto{\pgfqpoint{3.003402in}{1.559179in}}%
\pgfpathlineto{\pgfqpoint{3.003402in}{1.554921in}}%
\pgfpathmoveto{\pgfqpoint{2.999144in}{1.559179in}}%
\pgfpathlineto{\pgfqpoint{2.999144in}{1.559179in}}%
\pgfpathlineto{\pgfqpoint{2.999144in}{1.563437in}}%
\pgfpathlineto{\pgfqpoint{3.003402in}{1.563437in}}%
\pgfpathlineto{\pgfqpoint{3.003402in}{1.559179in}}%
\pgfpathmoveto{\pgfqpoint{2.999144in}{1.563437in}}%
\pgfpathlineto{\pgfqpoint{2.999144in}{1.563437in}}%
\pgfpathlineto{\pgfqpoint{2.999144in}{1.567695in}}%
\pgfpathlineto{\pgfqpoint{3.003402in}{1.567695in}}%
\pgfpathlineto{\pgfqpoint{3.003402in}{1.563437in}}%
\pgfpathmoveto{\pgfqpoint{2.999144in}{1.567695in}}%
\pgfpathlineto{\pgfqpoint{2.999144in}{1.567695in}}%
\pgfpathlineto{\pgfqpoint{2.999144in}{1.571953in}}%
\pgfpathlineto{\pgfqpoint{3.003402in}{1.571953in}}%
\pgfpathlineto{\pgfqpoint{3.003402in}{1.567695in}}%
\pgfpathmoveto{\pgfqpoint{2.999144in}{1.571953in}}%
\pgfpathlineto{\pgfqpoint{2.999144in}{1.571953in}}%
\pgfpathlineto{\pgfqpoint{2.999144in}{1.576210in}}%
\pgfpathlineto{\pgfqpoint{3.003402in}{1.576210in}}%
\pgfpathlineto{\pgfqpoint{3.003402in}{1.571953in}}%
\pgfpathmoveto{\pgfqpoint{2.999144in}{1.576210in}}%
\pgfpathlineto{\pgfqpoint{2.999144in}{1.576210in}}%
\pgfpathlineto{\pgfqpoint{2.999144in}{1.580468in}}%
\pgfpathlineto{\pgfqpoint{3.003402in}{1.580468in}}%
\pgfpathlineto{\pgfqpoint{3.003402in}{1.576210in}}%
\pgfpathmoveto{\pgfqpoint{2.999144in}{1.580468in}}%
\pgfpathlineto{\pgfqpoint{2.999144in}{1.580468in}}%
\pgfpathlineto{\pgfqpoint{2.999144in}{1.584726in}}%
\pgfpathlineto{\pgfqpoint{3.003402in}{1.584726in}}%
\pgfpathlineto{\pgfqpoint{3.003402in}{1.580468in}}%
\pgfpathmoveto{\pgfqpoint{2.999144in}{1.584726in}}%
\pgfpathlineto{\pgfqpoint{2.999144in}{1.584726in}}%
\pgfpathlineto{\pgfqpoint{2.999144in}{1.588984in}}%
\pgfpathlineto{\pgfqpoint{3.003402in}{1.588984in}}%
\pgfpathlineto{\pgfqpoint{3.003402in}{1.584726in}}%
\pgfpathmoveto{\pgfqpoint{2.999144in}{1.588984in}}%
\pgfpathlineto{\pgfqpoint{2.999144in}{1.588984in}}%
\pgfpathlineto{\pgfqpoint{2.999144in}{1.593242in}}%
\pgfpathlineto{\pgfqpoint{3.003402in}{1.593242in}}%
\pgfpathlineto{\pgfqpoint{3.003402in}{1.588984in}}%
\pgfpathmoveto{\pgfqpoint{3.003402in}{1.588984in}}%
\pgfpathlineto{\pgfqpoint{3.003402in}{1.588984in}}%
\pgfpathlineto{\pgfqpoint{3.003402in}{1.593242in}}%
\pgfpathlineto{\pgfqpoint{3.007660in}{1.593242in}}%
\pgfpathlineto{\pgfqpoint{3.007660in}{1.588984in}}%
\pgfpathmoveto{\pgfqpoint{3.003402in}{1.593242in}}%
\pgfpathlineto{\pgfqpoint{3.003402in}{1.593242in}}%
\pgfpathlineto{\pgfqpoint{3.003402in}{1.597500in}}%
\pgfpathlineto{\pgfqpoint{3.007660in}{1.597500in}}%
\pgfpathlineto{\pgfqpoint{3.007660in}{1.593242in}}%
\pgfpathmoveto{\pgfqpoint{3.003402in}{1.597500in}}%
\pgfpathlineto{\pgfqpoint{3.003402in}{1.597500in}}%
\pgfpathlineto{\pgfqpoint{3.003402in}{1.601758in}}%
\pgfpathlineto{\pgfqpoint{3.007660in}{1.601758in}}%
\pgfpathlineto{\pgfqpoint{3.007660in}{1.597500in}}%
\pgfpathmoveto{\pgfqpoint{3.003402in}{1.601758in}}%
\pgfpathlineto{\pgfqpoint{3.003402in}{1.601758in}}%
\pgfpathlineto{\pgfqpoint{3.003402in}{1.606015in}}%
\pgfpathlineto{\pgfqpoint{3.007660in}{1.606015in}}%
\pgfpathlineto{\pgfqpoint{3.007660in}{1.601758in}}%
\pgfpathmoveto{\pgfqpoint{3.003402in}{1.606015in}}%
\pgfpathlineto{\pgfqpoint{3.003402in}{1.606015in}}%
\pgfpathlineto{\pgfqpoint{3.003402in}{1.610273in}}%
\pgfpathlineto{\pgfqpoint{3.007660in}{1.610273in}}%
\pgfpathlineto{\pgfqpoint{3.007660in}{1.606015in}}%
\pgfpathmoveto{\pgfqpoint{3.003402in}{1.610273in}}%
\pgfpathlineto{\pgfqpoint{3.003402in}{1.610273in}}%
\pgfpathlineto{\pgfqpoint{3.003402in}{1.614531in}}%
\pgfpathlineto{\pgfqpoint{3.007660in}{1.614531in}}%
\pgfpathlineto{\pgfqpoint{3.007660in}{1.610273in}}%
\pgfpathmoveto{\pgfqpoint{3.003402in}{1.614531in}}%
\pgfpathlineto{\pgfqpoint{3.003402in}{1.614531in}}%
\pgfpathlineto{\pgfqpoint{3.003402in}{1.618789in}}%
\pgfpathlineto{\pgfqpoint{3.007660in}{1.618789in}}%
\pgfpathlineto{\pgfqpoint{3.007660in}{1.614531in}}%
\pgfpathmoveto{\pgfqpoint{3.003402in}{1.618789in}}%
\pgfpathlineto{\pgfqpoint{3.003402in}{1.618789in}}%
\pgfpathlineto{\pgfqpoint{3.003402in}{1.623047in}}%
\pgfpathlineto{\pgfqpoint{3.007660in}{1.623047in}}%
\pgfpathlineto{\pgfqpoint{3.007660in}{1.618789in}}%
\pgfpathmoveto{\pgfqpoint{3.003402in}{1.623047in}}%
\pgfpathlineto{\pgfqpoint{3.003402in}{1.623047in}}%
\pgfpathlineto{\pgfqpoint{3.003402in}{1.627305in}}%
\pgfpathlineto{\pgfqpoint{3.007660in}{1.627305in}}%
\pgfpathlineto{\pgfqpoint{3.007660in}{1.623047in}}%
\pgfpathmoveto{\pgfqpoint{3.007660in}{1.623047in}}%
\pgfpathlineto{\pgfqpoint{3.007660in}{1.623047in}}%
\pgfpathlineto{\pgfqpoint{3.007660in}{1.627305in}}%
\pgfpathlineto{\pgfqpoint{3.011917in}{1.627305in}}%
\pgfpathlineto{\pgfqpoint{3.011917in}{1.623047in}}%
\pgfpathmoveto{\pgfqpoint{3.007660in}{1.627305in}}%
\pgfpathlineto{\pgfqpoint{3.007660in}{1.627305in}}%
\pgfpathlineto{\pgfqpoint{3.007660in}{1.631563in}}%
\pgfpathlineto{\pgfqpoint{3.011917in}{1.631563in}}%
\pgfpathlineto{\pgfqpoint{3.011917in}{1.627305in}}%
\pgfpathmoveto{\pgfqpoint{3.007660in}{1.631563in}}%
\pgfpathlineto{\pgfqpoint{3.007660in}{1.631563in}}%
\pgfpathlineto{\pgfqpoint{3.007660in}{1.635821in}}%
\pgfpathlineto{\pgfqpoint{3.011917in}{1.635821in}}%
\pgfpathlineto{\pgfqpoint{3.011917in}{1.631563in}}%
\pgfpathmoveto{\pgfqpoint{3.007660in}{1.635821in}}%
\pgfpathlineto{\pgfqpoint{3.007660in}{1.635821in}}%
\pgfpathlineto{\pgfqpoint{3.007660in}{1.640079in}}%
\pgfpathlineto{\pgfqpoint{3.011917in}{1.640079in}}%
\pgfpathlineto{\pgfqpoint{3.011917in}{1.635821in}}%
\pgfpathmoveto{\pgfqpoint{3.007660in}{1.640079in}}%
\pgfpathlineto{\pgfqpoint{3.007660in}{1.640079in}}%
\pgfpathlineto{\pgfqpoint{3.007660in}{1.644337in}}%
\pgfpathlineto{\pgfqpoint{3.011917in}{1.644337in}}%
\pgfpathlineto{\pgfqpoint{3.011917in}{1.640079in}}%
\pgfpathmoveto{\pgfqpoint{3.007660in}{1.644337in}}%
\pgfpathlineto{\pgfqpoint{3.007660in}{1.644337in}}%
\pgfpathlineto{\pgfqpoint{3.007660in}{1.648595in}}%
\pgfpathlineto{\pgfqpoint{3.011917in}{1.648595in}}%
\pgfpathlineto{\pgfqpoint{3.011917in}{1.644337in}}%
\pgfpathmoveto{\pgfqpoint{3.007660in}{1.648595in}}%
\pgfpathlineto{\pgfqpoint{3.007660in}{1.648595in}}%
\pgfpathlineto{\pgfqpoint{3.007660in}{1.652853in}}%
\pgfpathlineto{\pgfqpoint{3.011917in}{1.652853in}}%
\pgfpathlineto{\pgfqpoint{3.011917in}{1.648595in}}%
\pgfpathmoveto{\pgfqpoint{3.007660in}{1.652853in}}%
\pgfpathlineto{\pgfqpoint{3.007660in}{1.652853in}}%
\pgfpathlineto{\pgfqpoint{3.007660in}{1.657111in}}%
\pgfpathlineto{\pgfqpoint{3.011917in}{1.657111in}}%
\pgfpathlineto{\pgfqpoint{3.011917in}{1.652853in}}%
\pgfpathmoveto{\pgfqpoint{3.007660in}{1.657111in}}%
\pgfpathlineto{\pgfqpoint{3.007660in}{1.657111in}}%
\pgfpathlineto{\pgfqpoint{3.007660in}{1.661369in}}%
\pgfpathlineto{\pgfqpoint{3.011917in}{1.661369in}}%
\pgfpathlineto{\pgfqpoint{3.011917in}{1.657111in}}%
\pgfpathmoveto{\pgfqpoint{3.011917in}{1.657111in}}%
\pgfpathlineto{\pgfqpoint{3.011917in}{1.657111in}}%
\pgfpathlineto{\pgfqpoint{3.011917in}{1.661369in}}%
\pgfpathlineto{\pgfqpoint{3.016175in}{1.661369in}}%
\pgfpathlineto{\pgfqpoint{3.016175in}{1.657111in}}%
\pgfpathmoveto{\pgfqpoint{3.011917in}{1.661369in}}%
\pgfpathlineto{\pgfqpoint{3.011917in}{1.661369in}}%
\pgfpathlineto{\pgfqpoint{3.011917in}{1.665627in}}%
\pgfpathlineto{\pgfqpoint{3.016175in}{1.665627in}}%
\pgfpathlineto{\pgfqpoint{3.016175in}{1.661369in}}%
\pgfpathmoveto{\pgfqpoint{3.011917in}{1.665627in}}%
\pgfpathlineto{\pgfqpoint{3.011917in}{1.665627in}}%
\pgfpathlineto{\pgfqpoint{3.011917in}{1.669884in}}%
\pgfpathlineto{\pgfqpoint{3.016175in}{1.669884in}}%
\pgfpathlineto{\pgfqpoint{3.016175in}{1.665627in}}%
\pgfpathmoveto{\pgfqpoint{3.011917in}{1.669884in}}%
\pgfpathlineto{\pgfqpoint{3.011917in}{1.669884in}}%
\pgfpathlineto{\pgfqpoint{3.011917in}{1.674142in}}%
\pgfpathlineto{\pgfqpoint{3.016175in}{1.674142in}}%
\pgfpathlineto{\pgfqpoint{3.016175in}{1.669884in}}%
\pgfpathmoveto{\pgfqpoint{3.011917in}{1.674142in}}%
\pgfpathlineto{\pgfqpoint{3.011917in}{1.674142in}}%
\pgfpathlineto{\pgfqpoint{3.011917in}{1.678400in}}%
\pgfpathlineto{\pgfqpoint{3.016175in}{1.678400in}}%
\pgfpathlineto{\pgfqpoint{3.016175in}{1.674142in}}%
\pgfpathmoveto{\pgfqpoint{3.011917in}{1.678400in}}%
\pgfpathlineto{\pgfqpoint{3.011917in}{1.678400in}}%
\pgfpathlineto{\pgfqpoint{3.011917in}{1.682658in}}%
\pgfpathlineto{\pgfqpoint{3.016175in}{1.682658in}}%
\pgfpathlineto{\pgfqpoint{3.016175in}{1.678400in}}%
\pgfpathmoveto{\pgfqpoint{3.011917in}{1.682658in}}%
\pgfpathlineto{\pgfqpoint{3.011917in}{1.682658in}}%
\pgfpathlineto{\pgfqpoint{3.011917in}{1.686916in}}%
\pgfpathlineto{\pgfqpoint{3.016175in}{1.686916in}}%
\pgfpathlineto{\pgfqpoint{3.016175in}{1.682658in}}%
\pgfpathmoveto{\pgfqpoint{3.011917in}{1.686916in}}%
\pgfpathlineto{\pgfqpoint{3.011917in}{1.686916in}}%
\pgfpathlineto{\pgfqpoint{3.011917in}{1.691174in}}%
\pgfpathlineto{\pgfqpoint{3.016175in}{1.691174in}}%
\pgfpathlineto{\pgfqpoint{3.016175in}{1.686916in}}%
\pgfpathmoveto{\pgfqpoint{3.011917in}{1.691174in}}%
\pgfpathlineto{\pgfqpoint{3.011917in}{1.691174in}}%
\pgfpathlineto{\pgfqpoint{3.011917in}{1.695432in}}%
\pgfpathlineto{\pgfqpoint{3.016175in}{1.695432in}}%
\pgfpathlineto{\pgfqpoint{3.016175in}{1.691174in}}%
\pgfpathmoveto{\pgfqpoint{3.016175in}{1.691174in}}%
\pgfpathlineto{\pgfqpoint{3.016175in}{1.691174in}}%
\pgfpathlineto{\pgfqpoint{3.016175in}{1.695432in}}%
\pgfpathlineto{\pgfqpoint{3.020433in}{1.695432in}}%
\pgfpathlineto{\pgfqpoint{3.020433in}{1.691174in}}%
\pgfpathmoveto{\pgfqpoint{3.016175in}{1.695432in}}%
\pgfpathlineto{\pgfqpoint{3.016175in}{1.695432in}}%
\pgfpathlineto{\pgfqpoint{3.016175in}{1.699690in}}%
\pgfpathlineto{\pgfqpoint{3.020433in}{1.699690in}}%
\pgfpathlineto{\pgfqpoint{3.020433in}{1.695432in}}%
\pgfpathmoveto{\pgfqpoint{3.016175in}{1.699690in}}%
\pgfpathlineto{\pgfqpoint{3.016175in}{1.699690in}}%
\pgfpathlineto{\pgfqpoint{3.016175in}{1.703948in}}%
\pgfpathlineto{\pgfqpoint{3.020433in}{1.703948in}}%
\pgfpathlineto{\pgfqpoint{3.020433in}{1.699690in}}%
\pgfpathmoveto{\pgfqpoint{3.016175in}{1.703948in}}%
\pgfpathlineto{\pgfqpoint{3.016175in}{1.703948in}}%
\pgfpathlineto{\pgfqpoint{3.016175in}{1.708206in}}%
\pgfpathlineto{\pgfqpoint{3.020433in}{1.708206in}}%
\pgfpathlineto{\pgfqpoint{3.020433in}{1.703948in}}%
\pgfpathmoveto{\pgfqpoint{3.016175in}{1.708206in}}%
\pgfpathlineto{\pgfqpoint{3.016175in}{1.708206in}}%
\pgfpathlineto{\pgfqpoint{3.016175in}{1.712464in}}%
\pgfpathlineto{\pgfqpoint{3.020433in}{1.712464in}}%
\pgfpathlineto{\pgfqpoint{3.020433in}{1.708206in}}%
\pgfpathmoveto{\pgfqpoint{3.016175in}{1.712464in}}%
\pgfpathlineto{\pgfqpoint{3.016175in}{1.712464in}}%
\pgfpathlineto{\pgfqpoint{3.016175in}{1.716722in}}%
\pgfpathlineto{\pgfqpoint{3.020433in}{1.716722in}}%
\pgfpathlineto{\pgfqpoint{3.020433in}{1.712464in}}%
\pgfpathmoveto{\pgfqpoint{3.016175in}{1.716722in}}%
\pgfpathlineto{\pgfqpoint{3.016175in}{1.716722in}}%
\pgfpathlineto{\pgfqpoint{3.016175in}{1.720980in}}%
\pgfpathlineto{\pgfqpoint{3.020433in}{1.720980in}}%
\pgfpathlineto{\pgfqpoint{3.020433in}{1.716722in}}%
\pgfpathmoveto{\pgfqpoint{3.016175in}{1.720980in}}%
\pgfpathlineto{\pgfqpoint{3.016175in}{1.720980in}}%
\pgfpathlineto{\pgfqpoint{3.016175in}{1.725238in}}%
\pgfpathlineto{\pgfqpoint{3.020433in}{1.725238in}}%
\pgfpathlineto{\pgfqpoint{3.020433in}{1.720980in}}%
\pgfpathmoveto{\pgfqpoint{3.016175in}{1.725238in}}%
\pgfpathlineto{\pgfqpoint{3.016175in}{1.725238in}}%
\pgfpathlineto{\pgfqpoint{3.016175in}{1.729496in}}%
\pgfpathlineto{\pgfqpoint{3.020433in}{1.729496in}}%
\pgfpathlineto{\pgfqpoint{3.020433in}{1.725238in}}%
\pgfpathmoveto{\pgfqpoint{3.020433in}{1.725238in}}%
\pgfpathlineto{\pgfqpoint{3.020433in}{1.725238in}}%
\pgfpathlineto{\pgfqpoint{3.020433in}{1.729496in}}%
\pgfpathlineto{\pgfqpoint{3.024691in}{1.729496in}}%
\pgfpathlineto{\pgfqpoint{3.024691in}{1.725238in}}%
\pgfpathmoveto{\pgfqpoint{3.020433in}{1.729496in}}%
\pgfpathlineto{\pgfqpoint{3.020433in}{1.729496in}}%
\pgfpathlineto{\pgfqpoint{3.020433in}{1.733753in}}%
\pgfpathlineto{\pgfqpoint{3.024691in}{1.733753in}}%
\pgfpathlineto{\pgfqpoint{3.024691in}{1.729496in}}%
\pgfpathmoveto{\pgfqpoint{3.020433in}{1.733753in}}%
\pgfpathlineto{\pgfqpoint{3.020433in}{1.733753in}}%
\pgfpathlineto{\pgfqpoint{3.020433in}{1.738011in}}%
\pgfpathlineto{\pgfqpoint{3.024691in}{1.738011in}}%
\pgfpathlineto{\pgfqpoint{3.024691in}{1.733753in}}%
\pgfpathmoveto{\pgfqpoint{3.020433in}{1.738011in}}%
\pgfpathlineto{\pgfqpoint{3.020433in}{1.738011in}}%
\pgfpathlineto{\pgfqpoint{3.020433in}{1.742269in}}%
\pgfpathlineto{\pgfqpoint{3.024691in}{1.742269in}}%
\pgfpathlineto{\pgfqpoint{3.024691in}{1.738011in}}%
\pgfpathmoveto{\pgfqpoint{3.020433in}{1.742269in}}%
\pgfpathlineto{\pgfqpoint{3.020433in}{1.742269in}}%
\pgfpathlineto{\pgfqpoint{3.020433in}{1.746526in}}%
\pgfpathlineto{\pgfqpoint{3.024691in}{1.746526in}}%
\pgfpathlineto{\pgfqpoint{3.024691in}{1.742269in}}%
\pgfpathmoveto{\pgfqpoint{3.020433in}{1.746526in}}%
\pgfpathlineto{\pgfqpoint{3.020433in}{1.746526in}}%
\pgfpathlineto{\pgfqpoint{3.020433in}{1.750784in}}%
\pgfpathlineto{\pgfqpoint{3.024691in}{1.750784in}}%
\pgfpathlineto{\pgfqpoint{3.024691in}{1.746526in}}%
\pgfpathmoveto{\pgfqpoint{3.020433in}{1.750784in}}%
\pgfpathlineto{\pgfqpoint{3.020433in}{1.750784in}}%
\pgfpathlineto{\pgfqpoint{3.020433in}{1.755041in}}%
\pgfpathlineto{\pgfqpoint{3.024691in}{1.755041in}}%
\pgfpathlineto{\pgfqpoint{3.024691in}{1.750784in}}%
\pgfpathmoveto{\pgfqpoint{3.020433in}{1.755041in}}%
\pgfpathlineto{\pgfqpoint{3.020433in}{1.755041in}}%
\pgfpathlineto{\pgfqpoint{3.020433in}{1.759299in}}%
\pgfpathlineto{\pgfqpoint{3.024691in}{1.759299in}}%
\pgfpathlineto{\pgfqpoint{3.024691in}{1.755041in}}%
\pgfpathmoveto{\pgfqpoint{3.020433in}{1.759299in}}%
\pgfpathlineto{\pgfqpoint{3.020433in}{1.759299in}}%
\pgfpathlineto{\pgfqpoint{3.020433in}{1.763557in}}%
\pgfpathlineto{\pgfqpoint{3.024691in}{1.763557in}}%
\pgfpathlineto{\pgfqpoint{3.024691in}{1.759299in}}%
\pgfpathmoveto{\pgfqpoint{3.024691in}{1.759299in}}%
\pgfpathlineto{\pgfqpoint{3.024691in}{1.759299in}}%
\pgfpathlineto{\pgfqpoint{3.024691in}{1.763557in}}%
\pgfpathlineto{\pgfqpoint{3.028949in}{1.763557in}}%
\pgfpathlineto{\pgfqpoint{3.028949in}{1.759299in}}%
\pgfpathmoveto{\pgfqpoint{3.024691in}{1.763557in}}%
\pgfpathlineto{\pgfqpoint{3.024691in}{1.763557in}}%
\pgfpathlineto{\pgfqpoint{3.024691in}{1.767814in}}%
\pgfpathlineto{\pgfqpoint{3.028949in}{1.767814in}}%
\pgfpathlineto{\pgfqpoint{3.028949in}{1.763557in}}%
\pgfpathmoveto{\pgfqpoint{3.024691in}{1.767814in}}%
\pgfpathlineto{\pgfqpoint{3.024691in}{1.767814in}}%
\pgfpathlineto{\pgfqpoint{3.024691in}{1.772072in}}%
\pgfpathlineto{\pgfqpoint{3.028949in}{1.772072in}}%
\pgfpathlineto{\pgfqpoint{3.028949in}{1.767814in}}%
\pgfpathmoveto{\pgfqpoint{3.024691in}{1.772072in}}%
\pgfpathlineto{\pgfqpoint{3.024691in}{1.772072in}}%
\pgfpathlineto{\pgfqpoint{3.024691in}{1.776329in}}%
\pgfpathlineto{\pgfqpoint{3.028949in}{1.776329in}}%
\pgfpathlineto{\pgfqpoint{3.028949in}{1.772072in}}%
\pgfpathmoveto{\pgfqpoint{3.024691in}{1.776329in}}%
\pgfpathlineto{\pgfqpoint{3.024691in}{1.776329in}}%
\pgfpathlineto{\pgfqpoint{3.024691in}{1.780587in}}%
\pgfpathlineto{\pgfqpoint{3.028949in}{1.780587in}}%
\pgfpathlineto{\pgfqpoint{3.028949in}{1.776329in}}%
\pgfpathmoveto{\pgfqpoint{3.024691in}{1.780587in}}%
\pgfpathlineto{\pgfqpoint{3.024691in}{1.780587in}}%
\pgfpathlineto{\pgfqpoint{3.024691in}{1.784844in}}%
\pgfpathlineto{\pgfqpoint{3.028949in}{1.784844in}}%
\pgfpathlineto{\pgfqpoint{3.028949in}{1.780587in}}%
\pgfpathmoveto{\pgfqpoint{3.024691in}{1.784844in}}%
\pgfpathlineto{\pgfqpoint{3.024691in}{1.784844in}}%
\pgfpathlineto{\pgfqpoint{3.024691in}{1.789102in}}%
\pgfpathlineto{\pgfqpoint{3.028949in}{1.789102in}}%
\pgfpathlineto{\pgfqpoint{3.028949in}{1.784844in}}%
\pgfpathmoveto{\pgfqpoint{3.024691in}{1.789102in}}%
\pgfpathlineto{\pgfqpoint{3.024691in}{1.789102in}}%
\pgfpathlineto{\pgfqpoint{3.024691in}{1.793360in}}%
\pgfpathlineto{\pgfqpoint{3.028949in}{1.793360in}}%
\pgfpathlineto{\pgfqpoint{3.028949in}{1.789102in}}%
\pgfpathmoveto{\pgfqpoint{3.028949in}{1.789102in}}%
\pgfpathlineto{\pgfqpoint{3.028949in}{1.789102in}}%
\pgfpathlineto{\pgfqpoint{3.028949in}{1.793360in}}%
\pgfpathlineto{\pgfqpoint{3.033207in}{1.793360in}}%
\pgfpathlineto{\pgfqpoint{3.033207in}{1.789102in}}%
\pgfpathmoveto{\pgfqpoint{3.028949in}{1.793360in}}%
\pgfpathlineto{\pgfqpoint{3.028949in}{1.793360in}}%
\pgfpathlineto{\pgfqpoint{3.028949in}{1.797617in}}%
\pgfpathlineto{\pgfqpoint{3.033207in}{1.797617in}}%
\pgfpathlineto{\pgfqpoint{3.033207in}{1.793360in}}%
\pgfpathmoveto{\pgfqpoint{3.028949in}{1.797617in}}%
\pgfpathlineto{\pgfqpoint{3.028949in}{1.797617in}}%
\pgfpathlineto{\pgfqpoint{3.028949in}{1.801875in}}%
\pgfpathlineto{\pgfqpoint{3.033207in}{1.801875in}}%
\pgfpathlineto{\pgfqpoint{3.033207in}{1.797617in}}%
\pgfpathmoveto{\pgfqpoint{3.028949in}{1.801875in}}%
\pgfpathlineto{\pgfqpoint{3.028949in}{1.801875in}}%
\pgfpathlineto{\pgfqpoint{3.028949in}{1.806132in}}%
\pgfpathlineto{\pgfqpoint{3.033207in}{1.806132in}}%
\pgfpathlineto{\pgfqpoint{3.033207in}{1.801875in}}%
\pgfpathmoveto{\pgfqpoint{3.028949in}{1.806132in}}%
\pgfpathlineto{\pgfqpoint{3.028949in}{1.806132in}}%
\pgfpathlineto{\pgfqpoint{3.028949in}{1.810390in}}%
\pgfpathlineto{\pgfqpoint{3.033207in}{1.810390in}}%
\pgfpathlineto{\pgfqpoint{3.033207in}{1.806132in}}%
\pgfpathmoveto{\pgfqpoint{3.028949in}{1.810390in}}%
\pgfpathlineto{\pgfqpoint{3.028949in}{1.810390in}}%
\pgfpathlineto{\pgfqpoint{3.028949in}{1.814647in}}%
\pgfpathlineto{\pgfqpoint{3.033207in}{1.814647in}}%
\pgfpathlineto{\pgfqpoint{3.033207in}{1.810390in}}%
\pgfpathmoveto{\pgfqpoint{3.028949in}{1.814647in}}%
\pgfpathlineto{\pgfqpoint{3.028949in}{1.814647in}}%
\pgfpathlineto{\pgfqpoint{3.028949in}{1.818905in}}%
\pgfpathlineto{\pgfqpoint{3.033207in}{1.818905in}}%
\pgfpathlineto{\pgfqpoint{3.033207in}{1.814647in}}%
\pgfpathmoveto{\pgfqpoint{3.028949in}{1.818905in}}%
\pgfpathlineto{\pgfqpoint{3.028949in}{1.818905in}}%
\pgfpathlineto{\pgfqpoint{3.028949in}{1.823163in}}%
\pgfpathlineto{\pgfqpoint{3.033207in}{1.823163in}}%
\pgfpathlineto{\pgfqpoint{3.033207in}{1.818905in}}%
\pgfpathmoveto{\pgfqpoint{3.028949in}{1.823163in}}%
\pgfpathlineto{\pgfqpoint{3.028949in}{1.823163in}}%
\pgfpathlineto{\pgfqpoint{3.028949in}{1.827420in}}%
\pgfpathlineto{\pgfqpoint{3.033207in}{1.827420in}}%
\pgfpathlineto{\pgfqpoint{3.033207in}{1.823163in}}%
\pgfpathmoveto{\pgfqpoint{3.033207in}{1.823163in}}%
\pgfpathlineto{\pgfqpoint{3.033207in}{1.823163in}}%
\pgfpathlineto{\pgfqpoint{3.033207in}{1.827420in}}%
\pgfpathlineto{\pgfqpoint{3.037464in}{1.827420in}}%
\pgfpathlineto{\pgfqpoint{3.037464in}{1.823163in}}%
\pgfpathmoveto{\pgfqpoint{3.033207in}{1.827420in}}%
\pgfpathlineto{\pgfqpoint{3.033207in}{1.827420in}}%
\pgfpathlineto{\pgfqpoint{3.033207in}{1.831678in}}%
\pgfpathlineto{\pgfqpoint{3.037464in}{1.831678in}}%
\pgfpathlineto{\pgfqpoint{3.037464in}{1.827420in}}%
\pgfpathmoveto{\pgfqpoint{3.033207in}{1.831678in}}%
\pgfpathlineto{\pgfqpoint{3.033207in}{1.831678in}}%
\pgfpathlineto{\pgfqpoint{3.033207in}{1.835935in}}%
\pgfpathlineto{\pgfqpoint{3.037464in}{1.835935in}}%
\pgfpathlineto{\pgfqpoint{3.037464in}{1.831678in}}%
\pgfpathmoveto{\pgfqpoint{3.033207in}{1.835935in}}%
\pgfpathlineto{\pgfqpoint{3.033207in}{1.835935in}}%
\pgfpathlineto{\pgfqpoint{3.033207in}{1.840193in}}%
\pgfpathlineto{\pgfqpoint{3.037464in}{1.840193in}}%
\pgfpathlineto{\pgfqpoint{3.037464in}{1.835935in}}%
\pgfpathmoveto{\pgfqpoint{3.033207in}{1.840193in}}%
\pgfpathlineto{\pgfqpoint{3.033207in}{1.840193in}}%
\pgfpathlineto{\pgfqpoint{3.033207in}{1.844451in}}%
\pgfpathlineto{\pgfqpoint{3.037464in}{1.844451in}}%
\pgfpathlineto{\pgfqpoint{3.037464in}{1.840193in}}%
\pgfpathmoveto{\pgfqpoint{3.033207in}{1.844451in}}%
\pgfpathlineto{\pgfqpoint{3.033207in}{1.844451in}}%
\pgfpathlineto{\pgfqpoint{3.033207in}{1.848708in}}%
\pgfpathlineto{\pgfqpoint{3.037464in}{1.848708in}}%
\pgfpathlineto{\pgfqpoint{3.037464in}{1.844451in}}%
\pgfpathmoveto{\pgfqpoint{3.033207in}{1.848708in}}%
\pgfpathlineto{\pgfqpoint{3.033207in}{1.848708in}}%
\pgfpathlineto{\pgfqpoint{3.033207in}{1.852966in}}%
\pgfpathlineto{\pgfqpoint{3.037464in}{1.852966in}}%
\pgfpathlineto{\pgfqpoint{3.037464in}{1.848708in}}%
\pgfpathmoveto{\pgfqpoint{3.033207in}{1.852966in}}%
\pgfpathlineto{\pgfqpoint{3.033207in}{1.852966in}}%
\pgfpathlineto{\pgfqpoint{3.033207in}{1.857223in}}%
\pgfpathlineto{\pgfqpoint{3.037464in}{1.857223in}}%
\pgfpathlineto{\pgfqpoint{3.037464in}{1.852966in}}%
\pgfpathmoveto{\pgfqpoint{3.037464in}{1.852966in}}%
\pgfpathlineto{\pgfqpoint{3.037464in}{1.852966in}}%
\pgfpathlineto{\pgfqpoint{3.037464in}{1.857223in}}%
\pgfpathlineto{\pgfqpoint{3.041722in}{1.857223in}}%
\pgfpathlineto{\pgfqpoint{3.041722in}{1.852966in}}%
\pgfpathmoveto{\pgfqpoint{3.037464in}{1.857223in}}%
\pgfpathlineto{\pgfqpoint{3.037464in}{1.857223in}}%
\pgfpathlineto{\pgfqpoint{3.037464in}{1.861481in}}%
\pgfpathlineto{\pgfqpoint{3.041722in}{1.861481in}}%
\pgfpathlineto{\pgfqpoint{3.041722in}{1.857223in}}%
\pgfpathmoveto{\pgfqpoint{3.037464in}{1.861481in}}%
\pgfpathlineto{\pgfqpoint{3.037464in}{1.861481in}}%
\pgfpathlineto{\pgfqpoint{3.037464in}{1.865738in}}%
\pgfpathlineto{\pgfqpoint{3.041722in}{1.865738in}}%
\pgfpathlineto{\pgfqpoint{3.041722in}{1.861481in}}%
\pgfpathmoveto{\pgfqpoint{3.037464in}{1.865738in}}%
\pgfpathlineto{\pgfqpoint{3.037464in}{1.865738in}}%
\pgfpathlineto{\pgfqpoint{3.037464in}{1.869996in}}%
\pgfpathlineto{\pgfqpoint{3.041722in}{1.869996in}}%
\pgfpathlineto{\pgfqpoint{3.041722in}{1.865738in}}%
\pgfpathmoveto{\pgfqpoint{3.037464in}{1.869996in}}%
\pgfpathlineto{\pgfqpoint{3.037464in}{1.869996in}}%
\pgfpathlineto{\pgfqpoint{3.037464in}{1.874254in}}%
\pgfpathlineto{\pgfqpoint{3.041722in}{1.874254in}}%
\pgfpathlineto{\pgfqpoint{3.041722in}{1.869996in}}%
\pgfpathmoveto{\pgfqpoint{3.037464in}{1.874254in}}%
\pgfpathlineto{\pgfqpoint{3.037464in}{1.874254in}}%
\pgfpathlineto{\pgfqpoint{3.037464in}{1.878512in}}%
\pgfpathlineto{\pgfqpoint{3.041722in}{1.878512in}}%
\pgfpathlineto{\pgfqpoint{3.041722in}{1.874254in}}%
\pgfpathmoveto{\pgfqpoint{3.037464in}{1.878512in}}%
\pgfpathlineto{\pgfqpoint{3.037464in}{1.878512in}}%
\pgfpathlineto{\pgfqpoint{3.037464in}{1.882769in}}%
\pgfpathlineto{\pgfqpoint{3.041722in}{1.882769in}}%
\pgfpathlineto{\pgfqpoint{3.041722in}{1.878512in}}%
\pgfpathmoveto{\pgfqpoint{3.037464in}{1.882769in}}%
\pgfpathlineto{\pgfqpoint{3.037464in}{1.882769in}}%
\pgfpathlineto{\pgfqpoint{3.037464in}{1.887027in}}%
\pgfpathlineto{\pgfqpoint{3.041722in}{1.887027in}}%
\pgfpathlineto{\pgfqpoint{3.041722in}{1.882769in}}%
\pgfpathmoveto{\pgfqpoint{3.037464in}{1.887027in}}%
\pgfpathlineto{\pgfqpoint{3.037464in}{1.887027in}}%
\pgfpathlineto{\pgfqpoint{3.037464in}{1.891285in}}%
\pgfpathlineto{\pgfqpoint{3.041722in}{1.891285in}}%
\pgfpathlineto{\pgfqpoint{3.041722in}{1.887027in}}%
\pgfpathmoveto{\pgfqpoint{3.041722in}{1.887027in}}%
\pgfpathlineto{\pgfqpoint{3.041722in}{1.887027in}}%
\pgfpathlineto{\pgfqpoint{3.041722in}{1.891285in}}%
\pgfpathlineto{\pgfqpoint{3.045980in}{1.891285in}}%
\pgfpathlineto{\pgfqpoint{3.045980in}{1.887027in}}%
\pgfpathmoveto{\pgfqpoint{3.041722in}{1.891285in}}%
\pgfpathlineto{\pgfqpoint{3.041722in}{1.891285in}}%
\pgfpathlineto{\pgfqpoint{3.041722in}{1.895543in}}%
\pgfpathlineto{\pgfqpoint{3.045980in}{1.895543in}}%
\pgfpathlineto{\pgfqpoint{3.045980in}{1.891285in}}%
\pgfpathmoveto{\pgfqpoint{3.041722in}{1.895543in}}%
\pgfpathlineto{\pgfqpoint{3.041722in}{1.895543in}}%
\pgfpathlineto{\pgfqpoint{3.041722in}{1.899801in}}%
\pgfpathlineto{\pgfqpoint{3.045980in}{1.899801in}}%
\pgfpathlineto{\pgfqpoint{3.045980in}{1.895543in}}%
\pgfpathmoveto{\pgfqpoint{3.041722in}{1.899801in}}%
\pgfpathlineto{\pgfqpoint{3.041722in}{1.899801in}}%
\pgfpathlineto{\pgfqpoint{3.041722in}{1.904059in}}%
\pgfpathlineto{\pgfqpoint{3.045980in}{1.904059in}}%
\pgfpathlineto{\pgfqpoint{3.045980in}{1.899801in}}%
\pgfpathmoveto{\pgfqpoint{3.041722in}{1.904059in}}%
\pgfpathlineto{\pgfqpoint{3.041722in}{1.904059in}}%
\pgfpathlineto{\pgfqpoint{3.041722in}{1.908316in}}%
\pgfpathlineto{\pgfqpoint{3.045980in}{1.908316in}}%
\pgfpathlineto{\pgfqpoint{3.045980in}{1.904059in}}%
\pgfpathmoveto{\pgfqpoint{3.041722in}{1.908316in}}%
\pgfpathlineto{\pgfqpoint{3.041722in}{1.908316in}}%
\pgfpathlineto{\pgfqpoint{3.041722in}{1.912574in}}%
\pgfpathlineto{\pgfqpoint{3.045980in}{1.912574in}}%
\pgfpathlineto{\pgfqpoint{3.045980in}{1.908316in}}%
\pgfpathmoveto{\pgfqpoint{3.041722in}{1.912574in}}%
\pgfpathlineto{\pgfqpoint{3.041722in}{1.912574in}}%
\pgfpathlineto{\pgfqpoint{3.041722in}{1.916832in}}%
\pgfpathlineto{\pgfqpoint{3.045980in}{1.916832in}}%
\pgfpathlineto{\pgfqpoint{3.045980in}{1.912574in}}%
\pgfpathmoveto{\pgfqpoint{3.041722in}{1.916832in}}%
\pgfpathlineto{\pgfqpoint{3.041722in}{1.916832in}}%
\pgfpathlineto{\pgfqpoint{3.041722in}{1.921090in}}%
\pgfpathlineto{\pgfqpoint{3.045980in}{1.921090in}}%
\pgfpathlineto{\pgfqpoint{3.045980in}{1.916832in}}%
\pgfpathmoveto{\pgfqpoint{3.045980in}{1.916832in}}%
\pgfpathlineto{\pgfqpoint{3.045980in}{1.916832in}}%
\pgfpathlineto{\pgfqpoint{3.045980in}{1.921090in}}%
\pgfpathlineto{\pgfqpoint{3.050238in}{1.921090in}}%
\pgfpathlineto{\pgfqpoint{3.050238in}{1.916832in}}%
\pgfpathmoveto{\pgfqpoint{3.045980in}{1.921090in}}%
\pgfpathlineto{\pgfqpoint{3.045980in}{1.921090in}}%
\pgfpathlineto{\pgfqpoint{3.045980in}{1.925348in}}%
\pgfpathlineto{\pgfqpoint{3.050238in}{1.925348in}}%
\pgfpathlineto{\pgfqpoint{3.050238in}{1.921090in}}%
\pgfpathmoveto{\pgfqpoint{3.045980in}{1.925348in}}%
\pgfpathlineto{\pgfqpoint{3.045980in}{1.925348in}}%
\pgfpathlineto{\pgfqpoint{3.045980in}{1.929605in}}%
\pgfpathlineto{\pgfqpoint{3.050238in}{1.929605in}}%
\pgfpathlineto{\pgfqpoint{3.050238in}{1.925348in}}%
\pgfpathmoveto{\pgfqpoint{3.045980in}{1.929605in}}%
\pgfpathlineto{\pgfqpoint{3.045980in}{1.929605in}}%
\pgfpathlineto{\pgfqpoint{3.045980in}{1.933863in}}%
\pgfpathlineto{\pgfqpoint{3.050238in}{1.933863in}}%
\pgfpathlineto{\pgfqpoint{3.050238in}{1.929605in}}%
\pgfpathmoveto{\pgfqpoint{3.045980in}{1.933863in}}%
\pgfpathlineto{\pgfqpoint{3.045980in}{1.933863in}}%
\pgfpathlineto{\pgfqpoint{3.045980in}{1.938121in}}%
\pgfpathlineto{\pgfqpoint{3.050238in}{1.938121in}}%
\pgfpathlineto{\pgfqpoint{3.050238in}{1.933863in}}%
\pgfpathmoveto{\pgfqpoint{3.045980in}{1.938121in}}%
\pgfpathlineto{\pgfqpoint{3.045980in}{1.938121in}}%
\pgfpathlineto{\pgfqpoint{3.045980in}{1.942379in}}%
\pgfpathlineto{\pgfqpoint{3.050238in}{1.942379in}}%
\pgfpathlineto{\pgfqpoint{3.050238in}{1.938121in}}%
\pgfpathmoveto{\pgfqpoint{3.045980in}{1.942379in}}%
\pgfpathlineto{\pgfqpoint{3.045980in}{1.942379in}}%
\pgfpathlineto{\pgfqpoint{3.045980in}{1.946637in}}%
\pgfpathlineto{\pgfqpoint{3.050238in}{1.946637in}}%
\pgfpathlineto{\pgfqpoint{3.050238in}{1.942379in}}%
\pgfpathmoveto{\pgfqpoint{3.045980in}{1.946637in}}%
\pgfpathlineto{\pgfqpoint{3.045980in}{1.946637in}}%
\pgfpathlineto{\pgfqpoint{3.045980in}{1.950895in}}%
\pgfpathlineto{\pgfqpoint{3.050238in}{1.950895in}}%
\pgfpathlineto{\pgfqpoint{3.050238in}{1.946637in}}%
\pgfpathmoveto{\pgfqpoint{3.045980in}{1.950895in}}%
\pgfpathlineto{\pgfqpoint{3.045980in}{1.950895in}}%
\pgfpathlineto{\pgfqpoint{3.045980in}{1.955152in}}%
\pgfpathlineto{\pgfqpoint{3.050238in}{1.955152in}}%
\pgfpathlineto{\pgfqpoint{3.050238in}{1.950895in}}%
\pgfpathmoveto{\pgfqpoint{3.050238in}{1.950895in}}%
\pgfpathlineto{\pgfqpoint{3.050238in}{1.950895in}}%
\pgfpathlineto{\pgfqpoint{3.050238in}{1.955152in}}%
\pgfpathlineto{\pgfqpoint{3.054496in}{1.955152in}}%
\pgfpathlineto{\pgfqpoint{3.054496in}{1.950895in}}%
\pgfpathmoveto{\pgfqpoint{3.050238in}{1.955152in}}%
\pgfpathlineto{\pgfqpoint{3.050238in}{1.955152in}}%
\pgfpathlineto{\pgfqpoint{3.050238in}{1.959410in}}%
\pgfpathlineto{\pgfqpoint{3.054496in}{1.959410in}}%
\pgfpathlineto{\pgfqpoint{3.054496in}{1.955152in}}%
\pgfpathmoveto{\pgfqpoint{3.050238in}{1.959410in}}%
\pgfpathlineto{\pgfqpoint{3.050238in}{1.959410in}}%
\pgfpathlineto{\pgfqpoint{3.050238in}{1.963668in}}%
\pgfpathlineto{\pgfqpoint{3.054496in}{1.963668in}}%
\pgfpathlineto{\pgfqpoint{3.054496in}{1.959410in}}%
\pgfpathmoveto{\pgfqpoint{3.050238in}{1.963668in}}%
\pgfpathlineto{\pgfqpoint{3.050238in}{1.963668in}}%
\pgfpathlineto{\pgfqpoint{3.050238in}{1.967926in}}%
\pgfpathlineto{\pgfqpoint{3.054496in}{1.967926in}}%
\pgfpathlineto{\pgfqpoint{3.054496in}{1.963668in}}%
\pgfpathmoveto{\pgfqpoint{3.050238in}{1.967926in}}%
\pgfpathlineto{\pgfqpoint{3.050238in}{1.967926in}}%
\pgfpathlineto{\pgfqpoint{3.050238in}{1.972184in}}%
\pgfpathlineto{\pgfqpoint{3.054496in}{1.972184in}}%
\pgfpathlineto{\pgfqpoint{3.054496in}{1.967926in}}%
\pgfpathmoveto{\pgfqpoint{3.050238in}{1.972184in}}%
\pgfpathlineto{\pgfqpoint{3.050238in}{1.972184in}}%
\pgfpathlineto{\pgfqpoint{3.050238in}{1.976441in}}%
\pgfpathlineto{\pgfqpoint{3.054496in}{1.976441in}}%
\pgfpathlineto{\pgfqpoint{3.054496in}{1.972184in}}%
\pgfpathmoveto{\pgfqpoint{3.050238in}{1.976441in}}%
\pgfpathlineto{\pgfqpoint{3.050238in}{1.976441in}}%
\pgfpathlineto{\pgfqpoint{3.050238in}{1.980699in}}%
\pgfpathlineto{\pgfqpoint{3.054496in}{1.980699in}}%
\pgfpathlineto{\pgfqpoint{3.054496in}{1.976441in}}%
\pgfpathmoveto{\pgfqpoint{3.050238in}{1.980699in}}%
\pgfpathlineto{\pgfqpoint{3.050238in}{1.980699in}}%
\pgfpathlineto{\pgfqpoint{3.050238in}{1.984957in}}%
\pgfpathlineto{\pgfqpoint{3.054496in}{1.984957in}}%
\pgfpathlineto{\pgfqpoint{3.054496in}{1.980699in}}%
\pgfpathmoveto{\pgfqpoint{3.054496in}{1.980699in}}%
\pgfpathlineto{\pgfqpoint{3.054496in}{1.980699in}}%
\pgfpathlineto{\pgfqpoint{3.054496in}{1.984957in}}%
\pgfpathlineto{\pgfqpoint{3.058753in}{1.984957in}}%
\pgfpathlineto{\pgfqpoint{3.058753in}{1.980699in}}%
\pgfpathmoveto{\pgfqpoint{3.054496in}{1.984957in}}%
\pgfpathlineto{\pgfqpoint{3.054496in}{1.984957in}}%
\pgfpathlineto{\pgfqpoint{3.054496in}{1.989215in}}%
\pgfpathlineto{\pgfqpoint{3.058753in}{1.989215in}}%
\pgfpathlineto{\pgfqpoint{3.058753in}{1.984957in}}%
\pgfpathmoveto{\pgfqpoint{3.054496in}{1.989215in}}%
\pgfpathlineto{\pgfqpoint{3.054496in}{1.989215in}}%
\pgfpathlineto{\pgfqpoint{3.054496in}{1.993473in}}%
\pgfpathlineto{\pgfqpoint{3.058753in}{1.993473in}}%
\pgfpathlineto{\pgfqpoint{3.058753in}{1.989215in}}%
\pgfpathmoveto{\pgfqpoint{3.054496in}{1.993473in}}%
\pgfpathlineto{\pgfqpoint{3.054496in}{1.993473in}}%
\pgfpathlineto{\pgfqpoint{3.054496in}{1.997731in}}%
\pgfpathlineto{\pgfqpoint{3.058753in}{1.997731in}}%
\pgfpathlineto{\pgfqpoint{3.058753in}{1.993473in}}%
\pgfpathmoveto{\pgfqpoint{3.054496in}{1.997731in}}%
\pgfpathlineto{\pgfqpoint{3.054496in}{1.997731in}}%
\pgfpathlineto{\pgfqpoint{3.054496in}{2.001988in}}%
\pgfpathlineto{\pgfqpoint{3.058753in}{2.001988in}}%
\pgfpathlineto{\pgfqpoint{3.058753in}{1.997731in}}%
\pgfpathmoveto{\pgfqpoint{3.054496in}{2.001988in}}%
\pgfpathlineto{\pgfqpoint{3.054496in}{2.001988in}}%
\pgfpathlineto{\pgfqpoint{3.054496in}{2.006246in}}%
\pgfpathlineto{\pgfqpoint{3.058753in}{2.006246in}}%
\pgfpathlineto{\pgfqpoint{3.058753in}{2.001988in}}%
\pgfpathmoveto{\pgfqpoint{3.054496in}{2.006246in}}%
\pgfpathlineto{\pgfqpoint{3.054496in}{2.006246in}}%
\pgfpathlineto{\pgfqpoint{3.054496in}{2.010504in}}%
\pgfpathlineto{\pgfqpoint{3.058753in}{2.010504in}}%
\pgfpathlineto{\pgfqpoint{3.058753in}{2.006246in}}%
\pgfpathmoveto{\pgfqpoint{3.054496in}{2.010504in}}%
\pgfpathlineto{\pgfqpoint{3.054496in}{2.010504in}}%
\pgfpathlineto{\pgfqpoint{3.054496in}{2.014762in}}%
\pgfpathlineto{\pgfqpoint{3.058753in}{2.014762in}}%
\pgfpathlineto{\pgfqpoint{3.058753in}{2.010504in}}%
\pgfpathmoveto{\pgfqpoint{3.058753in}{2.010504in}}%
\pgfpathlineto{\pgfqpoint{3.058753in}{2.010504in}}%
\pgfpathlineto{\pgfqpoint{3.058753in}{2.014762in}}%
\pgfpathlineto{\pgfqpoint{3.063011in}{2.014762in}}%
\pgfpathlineto{\pgfqpoint{3.063011in}{2.010504in}}%
\pgfpathmoveto{\pgfqpoint{3.058753in}{2.014762in}}%
\pgfpathlineto{\pgfqpoint{3.058753in}{2.014762in}}%
\pgfpathlineto{\pgfqpoint{3.058753in}{2.019020in}}%
\pgfpathlineto{\pgfqpoint{3.063011in}{2.019020in}}%
\pgfpathlineto{\pgfqpoint{3.063011in}{2.014762in}}%
\pgfpathmoveto{\pgfqpoint{3.058753in}{2.019020in}}%
\pgfpathlineto{\pgfqpoint{3.058753in}{2.019020in}}%
\pgfpathlineto{\pgfqpoint{3.058753in}{2.023278in}}%
\pgfpathlineto{\pgfqpoint{3.063011in}{2.023278in}}%
\pgfpathlineto{\pgfqpoint{3.063011in}{2.019020in}}%
\pgfpathmoveto{\pgfqpoint{3.058753in}{2.023278in}}%
\pgfpathlineto{\pgfqpoint{3.058753in}{2.023278in}}%
\pgfpathlineto{\pgfqpoint{3.058753in}{2.027536in}}%
\pgfpathlineto{\pgfqpoint{3.063011in}{2.027536in}}%
\pgfpathlineto{\pgfqpoint{3.063011in}{2.023278in}}%
\pgfpathmoveto{\pgfqpoint{3.058753in}{2.027536in}}%
\pgfpathlineto{\pgfqpoint{3.058753in}{2.027536in}}%
\pgfpathlineto{\pgfqpoint{3.058753in}{2.031793in}}%
\pgfpathlineto{\pgfqpoint{3.063011in}{2.031793in}}%
\pgfpathlineto{\pgfqpoint{3.063011in}{2.027536in}}%
\pgfpathmoveto{\pgfqpoint{3.058753in}{2.031793in}}%
\pgfpathlineto{\pgfqpoint{3.058753in}{2.031793in}}%
\pgfpathlineto{\pgfqpoint{3.058753in}{2.036051in}}%
\pgfpathlineto{\pgfqpoint{3.063011in}{2.036051in}}%
\pgfpathlineto{\pgfqpoint{3.063011in}{2.031793in}}%
\pgfpathmoveto{\pgfqpoint{3.058753in}{2.036051in}}%
\pgfpathlineto{\pgfqpoint{3.058753in}{2.036051in}}%
\pgfpathlineto{\pgfqpoint{3.058753in}{2.040309in}}%
\pgfpathlineto{\pgfqpoint{3.063011in}{2.040309in}}%
\pgfpathlineto{\pgfqpoint{3.063011in}{2.036051in}}%
\pgfpathmoveto{\pgfqpoint{3.058753in}{2.040309in}}%
\pgfpathlineto{\pgfqpoint{3.058753in}{2.040309in}}%
\pgfpathlineto{\pgfqpoint{3.058753in}{2.044567in}}%
\pgfpathlineto{\pgfqpoint{3.063011in}{2.044567in}}%
\pgfpathlineto{\pgfqpoint{3.063011in}{2.040309in}}%
\pgfpathmoveto{\pgfqpoint{3.063011in}{2.040309in}}%
\pgfpathlineto{\pgfqpoint{3.063011in}{2.040309in}}%
\pgfpathlineto{\pgfqpoint{3.063011in}{2.044567in}}%
\pgfpathlineto{\pgfqpoint{3.067269in}{2.044567in}}%
\pgfpathlineto{\pgfqpoint{3.067269in}{2.040309in}}%
\pgfpathmoveto{\pgfqpoint{3.063011in}{2.044567in}}%
\pgfpathlineto{\pgfqpoint{3.063011in}{2.044567in}}%
\pgfpathlineto{\pgfqpoint{3.063011in}{2.048825in}}%
\pgfpathlineto{\pgfqpoint{3.067269in}{2.048825in}}%
\pgfpathlineto{\pgfqpoint{3.067269in}{2.044567in}}%
\pgfpathmoveto{\pgfqpoint{3.063011in}{2.048825in}}%
\pgfpathlineto{\pgfqpoint{3.063011in}{2.048825in}}%
\pgfpathlineto{\pgfqpoint{3.063011in}{2.053083in}}%
\pgfpathlineto{\pgfqpoint{3.067269in}{2.053083in}}%
\pgfpathlineto{\pgfqpoint{3.067269in}{2.048825in}}%
\pgfpathmoveto{\pgfqpoint{3.063011in}{2.053083in}}%
\pgfpathlineto{\pgfqpoint{3.063011in}{2.053083in}}%
\pgfpathlineto{\pgfqpoint{3.063011in}{2.057341in}}%
\pgfpathlineto{\pgfqpoint{3.067269in}{2.057341in}}%
\pgfpathlineto{\pgfqpoint{3.067269in}{2.053083in}}%
\pgfpathmoveto{\pgfqpoint{3.063011in}{2.057341in}}%
\pgfpathlineto{\pgfqpoint{3.063011in}{2.057341in}}%
\pgfpathlineto{\pgfqpoint{3.063011in}{2.061599in}}%
\pgfpathlineto{\pgfqpoint{3.067269in}{2.061599in}}%
\pgfpathlineto{\pgfqpoint{3.067269in}{2.057341in}}%
\pgfpathmoveto{\pgfqpoint{3.063011in}{2.061599in}}%
\pgfpathlineto{\pgfqpoint{3.063011in}{2.061599in}}%
\pgfpathlineto{\pgfqpoint{3.063011in}{2.065856in}}%
\pgfpathlineto{\pgfqpoint{3.067269in}{2.065856in}}%
\pgfpathlineto{\pgfqpoint{3.067269in}{2.061599in}}%
\pgfpathmoveto{\pgfqpoint{3.063011in}{2.065856in}}%
\pgfpathlineto{\pgfqpoint{3.063011in}{2.065856in}}%
\pgfpathlineto{\pgfqpoint{3.063011in}{2.070114in}}%
\pgfpathlineto{\pgfqpoint{3.067269in}{2.070114in}}%
\pgfpathlineto{\pgfqpoint{3.067269in}{2.065856in}}%
\pgfpathmoveto{\pgfqpoint{3.063011in}{2.070114in}}%
\pgfpathlineto{\pgfqpoint{3.063011in}{2.070114in}}%
\pgfpathlineto{\pgfqpoint{3.063011in}{2.074372in}}%
\pgfpathlineto{\pgfqpoint{3.067269in}{2.074372in}}%
\pgfpathlineto{\pgfqpoint{3.067269in}{2.070114in}}%
\pgfpathmoveto{\pgfqpoint{3.067269in}{2.070114in}}%
\pgfpathlineto{\pgfqpoint{3.067269in}{2.070114in}}%
\pgfpathlineto{\pgfqpoint{3.067269in}{2.074372in}}%
\pgfpathlineto{\pgfqpoint{3.071527in}{2.074372in}}%
\pgfpathlineto{\pgfqpoint{3.071527in}{2.070114in}}%
\pgfpathmoveto{\pgfqpoint{3.067269in}{2.074372in}}%
\pgfpathlineto{\pgfqpoint{3.067269in}{2.074372in}}%
\pgfpathlineto{\pgfqpoint{3.067269in}{2.078630in}}%
\pgfpathlineto{\pgfqpoint{3.071527in}{2.078630in}}%
\pgfpathlineto{\pgfqpoint{3.071527in}{2.074372in}}%
\pgfpathmoveto{\pgfqpoint{3.067269in}{2.078630in}}%
\pgfpathlineto{\pgfqpoint{3.067269in}{2.078630in}}%
\pgfpathlineto{\pgfqpoint{3.067269in}{2.082888in}}%
\pgfpathlineto{\pgfqpoint{3.071527in}{2.082888in}}%
\pgfpathlineto{\pgfqpoint{3.071527in}{2.078630in}}%
\pgfpathmoveto{\pgfqpoint{3.067269in}{2.082888in}}%
\pgfpathlineto{\pgfqpoint{3.067269in}{2.082888in}}%
\pgfpathlineto{\pgfqpoint{3.067269in}{2.087146in}}%
\pgfpathlineto{\pgfqpoint{3.071527in}{2.087146in}}%
\pgfpathlineto{\pgfqpoint{3.071527in}{2.082888in}}%
\pgfpathmoveto{\pgfqpoint{3.067269in}{2.087146in}}%
\pgfpathlineto{\pgfqpoint{3.067269in}{2.087146in}}%
\pgfpathlineto{\pgfqpoint{3.067269in}{2.091404in}}%
\pgfpathlineto{\pgfqpoint{3.071527in}{2.091404in}}%
\pgfpathlineto{\pgfqpoint{3.071527in}{2.087146in}}%
\pgfpathmoveto{\pgfqpoint{3.067269in}{2.091404in}}%
\pgfpathlineto{\pgfqpoint{3.067269in}{2.091404in}}%
\pgfpathlineto{\pgfqpoint{3.067269in}{2.095661in}}%
\pgfpathlineto{\pgfqpoint{3.071527in}{2.095661in}}%
\pgfpathlineto{\pgfqpoint{3.071527in}{2.091404in}}%
\pgfpathmoveto{\pgfqpoint{3.067269in}{2.095661in}}%
\pgfpathlineto{\pgfqpoint{3.067269in}{2.095661in}}%
\pgfpathlineto{\pgfqpoint{3.067269in}{2.099919in}}%
\pgfpathlineto{\pgfqpoint{3.071527in}{2.099919in}}%
\pgfpathlineto{\pgfqpoint{3.071527in}{2.095661in}}%
\pgfpathmoveto{\pgfqpoint{3.067269in}{2.099919in}}%
\pgfpathlineto{\pgfqpoint{3.067269in}{2.099919in}}%
\pgfpathlineto{\pgfqpoint{3.067269in}{2.104177in}}%
\pgfpathlineto{\pgfqpoint{3.071527in}{2.104177in}}%
\pgfpathlineto{\pgfqpoint{3.071527in}{2.099919in}}%
\pgfpathmoveto{\pgfqpoint{3.071527in}{2.099919in}}%
\pgfpathlineto{\pgfqpoint{3.071527in}{2.099919in}}%
\pgfpathlineto{\pgfqpoint{3.071527in}{2.104177in}}%
\pgfpathlineto{\pgfqpoint{3.075785in}{2.104177in}}%
\pgfpathlineto{\pgfqpoint{3.075785in}{2.099919in}}%
\pgfpathmoveto{\pgfqpoint{3.071527in}{2.104177in}}%
\pgfpathlineto{\pgfqpoint{3.071527in}{2.104177in}}%
\pgfpathlineto{\pgfqpoint{3.071527in}{2.108435in}}%
\pgfpathlineto{\pgfqpoint{3.075785in}{2.108435in}}%
\pgfpathlineto{\pgfqpoint{3.075785in}{2.104177in}}%
\pgfpathmoveto{\pgfqpoint{3.071527in}{2.108435in}}%
\pgfpathlineto{\pgfqpoint{3.071527in}{2.108435in}}%
\pgfpathlineto{\pgfqpoint{3.071527in}{2.112693in}}%
\pgfpathlineto{\pgfqpoint{3.075785in}{2.112693in}}%
\pgfpathlineto{\pgfqpoint{3.075785in}{2.108435in}}%
\pgfpathmoveto{\pgfqpoint{3.071527in}{2.112693in}}%
\pgfpathlineto{\pgfqpoint{3.071527in}{2.112693in}}%
\pgfpathlineto{\pgfqpoint{3.071527in}{2.116951in}}%
\pgfpathlineto{\pgfqpoint{3.075785in}{2.116951in}}%
\pgfpathlineto{\pgfqpoint{3.075785in}{2.112693in}}%
\pgfpathmoveto{\pgfqpoint{3.071527in}{2.116951in}}%
\pgfpathlineto{\pgfqpoint{3.071527in}{2.116951in}}%
\pgfpathlineto{\pgfqpoint{3.071527in}{2.121209in}}%
\pgfpathlineto{\pgfqpoint{3.075785in}{2.121209in}}%
\pgfpathlineto{\pgfqpoint{3.075785in}{2.116951in}}%
\pgfpathmoveto{\pgfqpoint{3.071527in}{2.121209in}}%
\pgfpathlineto{\pgfqpoint{3.071527in}{2.121209in}}%
\pgfpathlineto{\pgfqpoint{3.071527in}{2.125467in}}%
\pgfpathlineto{\pgfqpoint{3.075785in}{2.125467in}}%
\pgfpathlineto{\pgfqpoint{3.075785in}{2.121209in}}%
\pgfpathmoveto{\pgfqpoint{3.071527in}{2.125467in}}%
\pgfpathlineto{\pgfqpoint{3.071527in}{2.125467in}}%
\pgfpathlineto{\pgfqpoint{3.071527in}{2.129724in}}%
\pgfpathlineto{\pgfqpoint{3.075785in}{2.129724in}}%
\pgfpathlineto{\pgfqpoint{3.075785in}{2.125467in}}%
\pgfpathmoveto{\pgfqpoint{3.071527in}{2.129724in}}%
\pgfpathlineto{\pgfqpoint{3.071527in}{2.129724in}}%
\pgfpathlineto{\pgfqpoint{3.071527in}{2.133982in}}%
\pgfpathlineto{\pgfqpoint{3.075785in}{2.133982in}}%
\pgfpathlineto{\pgfqpoint{3.075785in}{2.129724in}}%
\pgfpathmoveto{\pgfqpoint{3.075785in}{2.129724in}}%
\pgfpathlineto{\pgfqpoint{3.075785in}{2.129724in}}%
\pgfpathlineto{\pgfqpoint{3.075785in}{2.133982in}}%
\pgfpathlineto{\pgfqpoint{3.080043in}{2.133982in}}%
\pgfpathlineto{\pgfqpoint{3.080043in}{2.129724in}}%
\pgfpathmoveto{\pgfqpoint{3.075785in}{2.133982in}}%
\pgfpathlineto{\pgfqpoint{3.075785in}{2.133982in}}%
\pgfpathlineto{\pgfqpoint{3.075785in}{2.138240in}}%
\pgfpathlineto{\pgfqpoint{3.080043in}{2.138240in}}%
\pgfpathlineto{\pgfqpoint{3.080043in}{2.133982in}}%
\pgfpathmoveto{\pgfqpoint{3.075785in}{2.138240in}}%
\pgfpathlineto{\pgfqpoint{3.075785in}{2.138240in}}%
\pgfpathlineto{\pgfqpoint{3.075785in}{2.142498in}}%
\pgfpathlineto{\pgfqpoint{3.080043in}{2.142498in}}%
\pgfpathlineto{\pgfqpoint{3.080043in}{2.138240in}}%
\pgfpathmoveto{\pgfqpoint{3.075785in}{2.142498in}}%
\pgfpathlineto{\pgfqpoint{3.075785in}{2.142498in}}%
\pgfpathlineto{\pgfqpoint{3.075785in}{2.146756in}}%
\pgfpathlineto{\pgfqpoint{3.080043in}{2.146756in}}%
\pgfpathlineto{\pgfqpoint{3.080043in}{2.142498in}}%
\pgfpathmoveto{\pgfqpoint{3.075785in}{2.146756in}}%
\pgfpathlineto{\pgfqpoint{3.075785in}{2.146756in}}%
\pgfpathlineto{\pgfqpoint{3.075785in}{2.151014in}}%
\pgfpathlineto{\pgfqpoint{3.080043in}{2.151014in}}%
\pgfpathlineto{\pgfqpoint{3.080043in}{2.146756in}}%
\pgfpathmoveto{\pgfqpoint{3.075785in}{2.151014in}}%
\pgfpathlineto{\pgfqpoint{3.075785in}{2.151014in}}%
\pgfpathlineto{\pgfqpoint{3.075785in}{2.155272in}}%
\pgfpathlineto{\pgfqpoint{3.080043in}{2.155272in}}%
\pgfpathlineto{\pgfqpoint{3.080043in}{2.151014in}}%
\pgfpathmoveto{\pgfqpoint{3.075785in}{2.155272in}}%
\pgfpathlineto{\pgfqpoint{3.075785in}{2.155272in}}%
\pgfpathlineto{\pgfqpoint{3.075785in}{2.159530in}}%
\pgfpathlineto{\pgfqpoint{3.080043in}{2.159530in}}%
\pgfpathlineto{\pgfqpoint{3.080043in}{2.155272in}}%
\pgfpathmoveto{\pgfqpoint{3.075785in}{2.159530in}}%
\pgfpathlineto{\pgfqpoint{3.075785in}{2.159530in}}%
\pgfpathlineto{\pgfqpoint{3.075785in}{2.163787in}}%
\pgfpathlineto{\pgfqpoint{3.080043in}{2.163787in}}%
\pgfpathlineto{\pgfqpoint{3.080043in}{2.159530in}}%
\pgfpathmoveto{\pgfqpoint{3.080043in}{2.159530in}}%
\pgfpathlineto{\pgfqpoint{3.080043in}{2.159530in}}%
\pgfpathlineto{\pgfqpoint{3.080043in}{2.163787in}}%
\pgfpathlineto{\pgfqpoint{3.084300in}{2.163787in}}%
\pgfpathlineto{\pgfqpoint{3.084300in}{2.159530in}}%
\pgfpathmoveto{\pgfqpoint{3.080043in}{2.163787in}}%
\pgfpathlineto{\pgfqpoint{3.080043in}{2.163787in}}%
\pgfpathlineto{\pgfqpoint{3.080043in}{2.168045in}}%
\pgfpathlineto{\pgfqpoint{3.084300in}{2.168045in}}%
\pgfpathlineto{\pgfqpoint{3.084300in}{2.163787in}}%
\pgfpathmoveto{\pgfqpoint{3.080043in}{2.168045in}}%
\pgfpathlineto{\pgfqpoint{3.080043in}{2.168045in}}%
\pgfpathlineto{\pgfqpoint{3.080043in}{2.172303in}}%
\pgfpathlineto{\pgfqpoint{3.084300in}{2.172303in}}%
\pgfpathlineto{\pgfqpoint{3.084300in}{2.168045in}}%
\pgfpathmoveto{\pgfqpoint{3.080043in}{2.172303in}}%
\pgfpathlineto{\pgfqpoint{3.080043in}{2.172303in}}%
\pgfpathlineto{\pgfqpoint{3.080043in}{2.176561in}}%
\pgfpathlineto{\pgfqpoint{3.084300in}{2.176561in}}%
\pgfpathlineto{\pgfqpoint{3.084300in}{2.172303in}}%
\pgfpathmoveto{\pgfqpoint{3.080043in}{2.176561in}}%
\pgfpathlineto{\pgfqpoint{3.080043in}{2.176561in}}%
\pgfpathlineto{\pgfqpoint{3.080043in}{2.180819in}}%
\pgfpathlineto{\pgfqpoint{3.084300in}{2.180819in}}%
\pgfpathlineto{\pgfqpoint{3.084300in}{2.176561in}}%
\pgfpathmoveto{\pgfqpoint{3.080043in}{2.180819in}}%
\pgfpathlineto{\pgfqpoint{3.080043in}{2.180819in}}%
\pgfpathlineto{\pgfqpoint{3.080043in}{2.185077in}}%
\pgfpathlineto{\pgfqpoint{3.084300in}{2.185077in}}%
\pgfpathlineto{\pgfqpoint{3.084300in}{2.180819in}}%
\pgfpathmoveto{\pgfqpoint{3.080043in}{2.185077in}}%
\pgfpathlineto{\pgfqpoint{3.080043in}{2.185077in}}%
\pgfpathlineto{\pgfqpoint{3.080043in}{2.189335in}}%
\pgfpathlineto{\pgfqpoint{3.084300in}{2.189335in}}%
\pgfpathlineto{\pgfqpoint{3.084300in}{2.185077in}}%
\pgfpathmoveto{\pgfqpoint{3.084300in}{2.185077in}}%
\pgfpathlineto{\pgfqpoint{3.084300in}{2.185077in}}%
\pgfpathlineto{\pgfqpoint{3.084300in}{2.189335in}}%
\pgfpathlineto{\pgfqpoint{3.088558in}{2.189335in}}%
\pgfpathlineto{\pgfqpoint{3.088558in}{2.185077in}}%
\pgfpathmoveto{\pgfqpoint{3.084300in}{2.189335in}}%
\pgfpathlineto{\pgfqpoint{3.084300in}{2.189335in}}%
\pgfpathlineto{\pgfqpoint{3.084300in}{2.193592in}}%
\pgfpathlineto{\pgfqpoint{3.088558in}{2.193592in}}%
\pgfpathlineto{\pgfqpoint{3.088558in}{2.189335in}}%
\pgfpathmoveto{\pgfqpoint{3.084300in}{2.193592in}}%
\pgfpathlineto{\pgfqpoint{3.084300in}{2.193592in}}%
\pgfpathlineto{\pgfqpoint{3.084300in}{2.197850in}}%
\pgfpathlineto{\pgfqpoint{3.088558in}{2.197850in}}%
\pgfpathlineto{\pgfqpoint{3.088558in}{2.193592in}}%
\pgfpathmoveto{\pgfqpoint{3.084300in}{2.197850in}}%
\pgfpathlineto{\pgfqpoint{3.084300in}{2.197850in}}%
\pgfpathlineto{\pgfqpoint{3.084300in}{2.202108in}}%
\pgfpathlineto{\pgfqpoint{3.088558in}{2.202108in}}%
\pgfpathlineto{\pgfqpoint{3.088558in}{2.197850in}}%
\pgfpathmoveto{\pgfqpoint{3.084300in}{2.202108in}}%
\pgfpathlineto{\pgfqpoint{3.084300in}{2.202108in}}%
\pgfpathlineto{\pgfqpoint{3.084300in}{2.206366in}}%
\pgfpathlineto{\pgfqpoint{3.088558in}{2.206366in}}%
\pgfpathlineto{\pgfqpoint{3.088558in}{2.202108in}}%
\pgfpathmoveto{\pgfqpoint{3.084300in}{2.206366in}}%
\pgfpathlineto{\pgfqpoint{3.084300in}{2.206366in}}%
\pgfpathlineto{\pgfqpoint{3.084300in}{2.210624in}}%
\pgfpathlineto{\pgfqpoint{3.088558in}{2.210624in}}%
\pgfpathlineto{\pgfqpoint{3.088558in}{2.206366in}}%
\pgfpathmoveto{\pgfqpoint{3.084300in}{2.210624in}}%
\pgfpathlineto{\pgfqpoint{3.084300in}{2.210624in}}%
\pgfpathlineto{\pgfqpoint{3.084300in}{2.214882in}}%
\pgfpathlineto{\pgfqpoint{3.088558in}{2.214882in}}%
\pgfpathlineto{\pgfqpoint{3.088558in}{2.210624in}}%
\pgfpathmoveto{\pgfqpoint{3.084300in}{2.214882in}}%
\pgfpathlineto{\pgfqpoint{3.084300in}{2.214882in}}%
\pgfpathlineto{\pgfqpoint{3.084300in}{2.219140in}}%
\pgfpathlineto{\pgfqpoint{3.088558in}{2.219140in}}%
\pgfpathlineto{\pgfqpoint{3.088558in}{2.214882in}}%
\pgfpathmoveto{\pgfqpoint{3.088558in}{2.214882in}}%
\pgfpathlineto{\pgfqpoint{3.088558in}{2.214882in}}%
\pgfpathlineto{\pgfqpoint{3.088558in}{2.219140in}}%
\pgfpathlineto{\pgfqpoint{3.092816in}{2.219140in}}%
\pgfpathlineto{\pgfqpoint{3.092816in}{2.214882in}}%
\pgfpathmoveto{\pgfqpoint{3.088558in}{2.219140in}}%
\pgfpathlineto{\pgfqpoint{3.088558in}{2.219140in}}%
\pgfpathlineto{\pgfqpoint{3.088558in}{2.223397in}}%
\pgfpathlineto{\pgfqpoint{3.092816in}{2.223397in}}%
\pgfpathlineto{\pgfqpoint{3.092816in}{2.219140in}}%
\pgfpathmoveto{\pgfqpoint{3.088558in}{2.223397in}}%
\pgfpathlineto{\pgfqpoint{3.088558in}{2.223397in}}%
\pgfpathlineto{\pgfqpoint{3.088558in}{2.227655in}}%
\pgfpathlineto{\pgfqpoint{3.092816in}{2.227655in}}%
\pgfpathlineto{\pgfqpoint{3.092816in}{2.223397in}}%
\pgfpathmoveto{\pgfqpoint{3.088558in}{2.227655in}}%
\pgfpathlineto{\pgfqpoint{3.088558in}{2.227655in}}%
\pgfpathlineto{\pgfqpoint{3.088558in}{2.231913in}}%
\pgfpathlineto{\pgfqpoint{3.092816in}{2.231913in}}%
\pgfpathlineto{\pgfqpoint{3.092816in}{2.227655in}}%
\pgfpathmoveto{\pgfqpoint{3.088558in}{2.231913in}}%
\pgfpathlineto{\pgfqpoint{3.088558in}{2.231913in}}%
\pgfpathlineto{\pgfqpoint{3.088558in}{2.236171in}}%
\pgfpathlineto{\pgfqpoint{3.092816in}{2.236171in}}%
\pgfpathlineto{\pgfqpoint{3.092816in}{2.231913in}}%
\pgfpathmoveto{\pgfqpoint{3.088558in}{2.236171in}}%
\pgfpathlineto{\pgfqpoint{3.088558in}{2.236171in}}%
\pgfpathlineto{\pgfqpoint{3.088558in}{2.240429in}}%
\pgfpathlineto{\pgfqpoint{3.092816in}{2.240429in}}%
\pgfpathlineto{\pgfqpoint{3.092816in}{2.236171in}}%
\pgfpathmoveto{\pgfqpoint{3.088558in}{2.240429in}}%
\pgfpathlineto{\pgfqpoint{3.088558in}{2.240429in}}%
\pgfpathlineto{\pgfqpoint{3.088558in}{2.244687in}}%
\pgfpathlineto{\pgfqpoint{3.092816in}{2.244687in}}%
\pgfpathlineto{\pgfqpoint{3.092816in}{2.240429in}}%
\pgfpathmoveto{\pgfqpoint{3.088558in}{2.244687in}}%
\pgfpathlineto{\pgfqpoint{3.088558in}{2.244687in}}%
\pgfpathlineto{\pgfqpoint{3.088558in}{2.248945in}}%
\pgfpathlineto{\pgfqpoint{3.092816in}{2.248945in}}%
\pgfpathlineto{\pgfqpoint{3.092816in}{2.244687in}}%
\pgfpathmoveto{\pgfqpoint{3.092816in}{2.244687in}}%
\pgfpathlineto{\pgfqpoint{3.092816in}{2.244687in}}%
\pgfpathlineto{\pgfqpoint{3.092816in}{2.248945in}}%
\pgfpathlineto{\pgfqpoint{3.097074in}{2.248945in}}%
\pgfpathlineto{\pgfqpoint{3.097074in}{2.244687in}}%
\pgfpathmoveto{\pgfqpoint{3.092816in}{2.248945in}}%
\pgfpathlineto{\pgfqpoint{3.092816in}{2.248945in}}%
\pgfpathlineto{\pgfqpoint{3.092816in}{2.253203in}}%
\pgfpathlineto{\pgfqpoint{3.097074in}{2.253203in}}%
\pgfpathlineto{\pgfqpoint{3.097074in}{2.248945in}}%
\pgfpathmoveto{\pgfqpoint{3.092816in}{2.253203in}}%
\pgfpathlineto{\pgfqpoint{3.092816in}{2.253203in}}%
\pgfpathlineto{\pgfqpoint{3.092816in}{2.257460in}}%
\pgfpathlineto{\pgfqpoint{3.097074in}{2.257460in}}%
\pgfpathlineto{\pgfqpoint{3.097074in}{2.253203in}}%
\pgfpathmoveto{\pgfqpoint{3.092816in}{2.257460in}}%
\pgfpathlineto{\pgfqpoint{3.092816in}{2.257460in}}%
\pgfpathlineto{\pgfqpoint{3.092816in}{2.261718in}}%
\pgfpathlineto{\pgfqpoint{3.097074in}{2.261718in}}%
\pgfpathlineto{\pgfqpoint{3.097074in}{2.257460in}}%
\pgfpathmoveto{\pgfqpoint{3.092816in}{2.261718in}}%
\pgfpathlineto{\pgfqpoint{3.092816in}{2.261718in}}%
\pgfpathlineto{\pgfqpoint{3.092816in}{2.265976in}}%
\pgfpathlineto{\pgfqpoint{3.097074in}{2.265976in}}%
\pgfpathlineto{\pgfqpoint{3.097074in}{2.261718in}}%
\pgfpathmoveto{\pgfqpoint{3.092816in}{2.265976in}}%
\pgfpathlineto{\pgfqpoint{3.092816in}{2.265976in}}%
\pgfpathlineto{\pgfqpoint{3.092816in}{2.270234in}}%
\pgfpathlineto{\pgfqpoint{3.097074in}{2.270234in}}%
\pgfpathlineto{\pgfqpoint{3.097074in}{2.265976in}}%
\pgfpathmoveto{\pgfqpoint{3.092816in}{2.270234in}}%
\pgfpathlineto{\pgfqpoint{3.092816in}{2.270234in}}%
\pgfpathlineto{\pgfqpoint{3.092816in}{2.274492in}}%
\pgfpathlineto{\pgfqpoint{3.097074in}{2.274492in}}%
\pgfpathlineto{\pgfqpoint{3.097074in}{2.270234in}}%
\pgfpathmoveto{\pgfqpoint{3.097074in}{2.270234in}}%
\pgfpathlineto{\pgfqpoint{3.097074in}{2.270234in}}%
\pgfpathlineto{\pgfqpoint{3.097074in}{2.274492in}}%
\pgfpathlineto{\pgfqpoint{3.101331in}{2.274492in}}%
\pgfpathlineto{\pgfqpoint{3.101331in}{2.270234in}}%
\pgfpathmoveto{\pgfqpoint{3.097074in}{2.274492in}}%
\pgfpathlineto{\pgfqpoint{3.097074in}{2.274492in}}%
\pgfpathlineto{\pgfqpoint{3.097074in}{2.278750in}}%
\pgfpathlineto{\pgfqpoint{3.101331in}{2.278750in}}%
\pgfpathlineto{\pgfqpoint{3.101331in}{2.274492in}}%
\pgfpathmoveto{\pgfqpoint{3.097074in}{2.278750in}}%
\pgfpathlineto{\pgfqpoint{3.097074in}{2.278750in}}%
\pgfpathlineto{\pgfqpoint{3.097074in}{2.283007in}}%
\pgfpathlineto{\pgfqpoint{3.101331in}{2.283007in}}%
\pgfpathlineto{\pgfqpoint{3.101331in}{2.278750in}}%
\pgfpathmoveto{\pgfqpoint{3.097074in}{2.283007in}}%
\pgfpathlineto{\pgfqpoint{3.097074in}{2.283007in}}%
\pgfpathlineto{\pgfqpoint{3.097074in}{2.287265in}}%
\pgfpathlineto{\pgfqpoint{3.101331in}{2.287265in}}%
\pgfpathlineto{\pgfqpoint{3.101331in}{2.283007in}}%
\pgfpathmoveto{\pgfqpoint{3.097074in}{2.287265in}}%
\pgfpathlineto{\pgfqpoint{3.097074in}{2.287265in}}%
\pgfpathlineto{\pgfqpoint{3.097074in}{2.291523in}}%
\pgfpathlineto{\pgfqpoint{3.101331in}{2.291523in}}%
\pgfpathlineto{\pgfqpoint{3.101331in}{2.287265in}}%
\pgfpathmoveto{\pgfqpoint{3.097074in}{2.291523in}}%
\pgfpathlineto{\pgfqpoint{3.097074in}{2.291523in}}%
\pgfpathlineto{\pgfqpoint{3.097074in}{2.295781in}}%
\pgfpathlineto{\pgfqpoint{3.101331in}{2.295781in}}%
\pgfpathlineto{\pgfqpoint{3.101331in}{2.291523in}}%
\pgfpathmoveto{\pgfqpoint{3.097074in}{2.295781in}}%
\pgfpathlineto{\pgfqpoint{3.097074in}{2.295781in}}%
\pgfpathlineto{\pgfqpoint{3.097074in}{2.300039in}}%
\pgfpathlineto{\pgfqpoint{3.101331in}{2.300039in}}%
\pgfpathlineto{\pgfqpoint{3.101331in}{2.295781in}}%
\pgfpathmoveto{\pgfqpoint{3.101331in}{2.295781in}}%
\pgfpathlineto{\pgfqpoint{3.101331in}{2.295781in}}%
\pgfpathlineto{\pgfqpoint{3.101331in}{2.300039in}}%
\pgfpathlineto{\pgfqpoint{3.105589in}{2.300039in}}%
\pgfpathlineto{\pgfqpoint{3.105589in}{2.295781in}}%
\pgfpathmoveto{\pgfqpoint{3.101331in}{2.300039in}}%
\pgfpathlineto{\pgfqpoint{3.101331in}{2.300039in}}%
\pgfpathlineto{\pgfqpoint{3.101331in}{2.304296in}}%
\pgfpathlineto{\pgfqpoint{3.105589in}{2.304296in}}%
\pgfpathlineto{\pgfqpoint{3.105589in}{2.300039in}}%
\pgfpathmoveto{\pgfqpoint{3.101331in}{2.304296in}}%
\pgfpathlineto{\pgfqpoint{3.101331in}{2.304296in}}%
\pgfpathlineto{\pgfqpoint{3.101331in}{2.308554in}}%
\pgfpathlineto{\pgfqpoint{3.105589in}{2.308554in}}%
\pgfpathlineto{\pgfqpoint{3.105589in}{2.304296in}}%
\pgfpathmoveto{\pgfqpoint{3.101331in}{2.308554in}}%
\pgfpathlineto{\pgfqpoint{3.101331in}{2.308554in}}%
\pgfpathlineto{\pgfqpoint{3.101331in}{2.312812in}}%
\pgfpathlineto{\pgfqpoint{3.105589in}{2.312812in}}%
\pgfpathlineto{\pgfqpoint{3.105589in}{2.308554in}}%
\pgfpathmoveto{\pgfqpoint{3.101331in}{2.312812in}}%
\pgfpathlineto{\pgfqpoint{3.101331in}{2.312812in}}%
\pgfpathlineto{\pgfqpoint{3.101331in}{2.317070in}}%
\pgfpathlineto{\pgfqpoint{3.105589in}{2.317070in}}%
\pgfpathlineto{\pgfqpoint{3.105589in}{2.312812in}}%
\pgfpathmoveto{\pgfqpoint{3.101331in}{2.317070in}}%
\pgfpathlineto{\pgfqpoint{3.101331in}{2.317070in}}%
\pgfpathlineto{\pgfqpoint{3.101331in}{2.321327in}}%
\pgfpathlineto{\pgfqpoint{3.105589in}{2.321327in}}%
\pgfpathlineto{\pgfqpoint{3.105589in}{2.317070in}}%
\pgfpathmoveto{\pgfqpoint{3.101331in}{2.321327in}}%
\pgfpathlineto{\pgfqpoint{3.101331in}{2.321327in}}%
\pgfpathlineto{\pgfqpoint{3.101331in}{2.325585in}}%
\pgfpathlineto{\pgfqpoint{3.105589in}{2.325585in}}%
\pgfpathlineto{\pgfqpoint{3.105589in}{2.321327in}}%
\pgfpathmoveto{\pgfqpoint{3.101331in}{2.325585in}}%
\pgfpathlineto{\pgfqpoint{3.101331in}{2.325585in}}%
\pgfpathlineto{\pgfqpoint{3.101331in}{2.329843in}}%
\pgfpathlineto{\pgfqpoint{3.105589in}{2.329843in}}%
\pgfpathlineto{\pgfqpoint{3.105589in}{2.325585in}}%
\pgfpathmoveto{\pgfqpoint{3.105589in}{2.325585in}}%
\pgfpathlineto{\pgfqpoint{3.105589in}{2.325585in}}%
\pgfpathlineto{\pgfqpoint{3.105589in}{2.329843in}}%
\pgfpathlineto{\pgfqpoint{3.109846in}{2.329843in}}%
\pgfpathlineto{\pgfqpoint{3.109846in}{2.325585in}}%
\pgfpathmoveto{\pgfqpoint{3.105589in}{2.329843in}}%
\pgfpathlineto{\pgfqpoint{3.105589in}{2.329843in}}%
\pgfpathlineto{\pgfqpoint{3.105589in}{2.334101in}}%
\pgfpathlineto{\pgfqpoint{3.109846in}{2.334101in}}%
\pgfpathlineto{\pgfqpoint{3.109846in}{2.329843in}}%
\pgfpathmoveto{\pgfqpoint{3.105589in}{2.334101in}}%
\pgfpathlineto{\pgfqpoint{3.105589in}{2.334101in}}%
\pgfpathlineto{\pgfqpoint{3.105589in}{2.338359in}}%
\pgfpathlineto{\pgfqpoint{3.109846in}{2.338359in}}%
\pgfpathlineto{\pgfqpoint{3.109846in}{2.334101in}}%
\pgfpathmoveto{\pgfqpoint{3.105589in}{2.338359in}}%
\pgfpathlineto{\pgfqpoint{3.105589in}{2.338359in}}%
\pgfpathlineto{\pgfqpoint{3.105589in}{2.342616in}}%
\pgfpathlineto{\pgfqpoint{3.109846in}{2.342616in}}%
\pgfpathlineto{\pgfqpoint{3.109846in}{2.338359in}}%
\pgfpathmoveto{\pgfqpoint{3.105589in}{2.342616in}}%
\pgfpathlineto{\pgfqpoint{3.105589in}{2.342616in}}%
\pgfpathlineto{\pgfqpoint{3.105589in}{2.346874in}}%
\pgfpathlineto{\pgfqpoint{3.109846in}{2.346874in}}%
\pgfpathlineto{\pgfqpoint{3.109846in}{2.342616in}}%
\pgfpathmoveto{\pgfqpoint{3.105589in}{2.346874in}}%
\pgfpathlineto{\pgfqpoint{3.105589in}{2.346874in}}%
\pgfpathlineto{\pgfqpoint{3.105589in}{2.351132in}}%
\pgfpathlineto{\pgfqpoint{3.109846in}{2.351132in}}%
\pgfpathlineto{\pgfqpoint{3.109846in}{2.346874in}}%
\pgfpathmoveto{\pgfqpoint{3.105589in}{2.351132in}}%
\pgfpathlineto{\pgfqpoint{3.105589in}{2.351132in}}%
\pgfpathlineto{\pgfqpoint{3.105589in}{2.355390in}}%
\pgfpathlineto{\pgfqpoint{3.109846in}{2.355390in}}%
\pgfpathlineto{\pgfqpoint{3.109846in}{2.351132in}}%
\pgfpathmoveto{\pgfqpoint{3.109846in}{2.351132in}}%
\pgfpathlineto{\pgfqpoint{3.109846in}{2.351132in}}%
\pgfpathlineto{\pgfqpoint{3.109846in}{2.355390in}}%
\pgfpathlineto{\pgfqpoint{3.114104in}{2.355390in}}%
\pgfpathlineto{\pgfqpoint{3.114104in}{2.351132in}}%
\pgfpathmoveto{\pgfqpoint{3.109846in}{2.355390in}}%
\pgfpathlineto{\pgfqpoint{3.109846in}{2.355390in}}%
\pgfpathlineto{\pgfqpoint{3.109846in}{2.359647in}}%
\pgfpathlineto{\pgfqpoint{3.114104in}{2.359647in}}%
\pgfpathlineto{\pgfqpoint{3.114104in}{2.355390in}}%
\pgfpathmoveto{\pgfqpoint{3.109846in}{2.359647in}}%
\pgfpathlineto{\pgfqpoint{3.109846in}{2.359647in}}%
\pgfpathlineto{\pgfqpoint{3.109846in}{2.363905in}}%
\pgfpathlineto{\pgfqpoint{3.114104in}{2.363905in}}%
\pgfpathlineto{\pgfqpoint{3.114104in}{2.359647in}}%
\pgfpathmoveto{\pgfqpoint{3.109846in}{2.363905in}}%
\pgfpathlineto{\pgfqpoint{3.109846in}{2.363905in}}%
\pgfpathlineto{\pgfqpoint{3.109846in}{2.368163in}}%
\pgfpathlineto{\pgfqpoint{3.114104in}{2.368163in}}%
\pgfpathlineto{\pgfqpoint{3.114104in}{2.363905in}}%
\pgfpathmoveto{\pgfqpoint{3.109846in}{2.368163in}}%
\pgfpathlineto{\pgfqpoint{3.109846in}{2.368163in}}%
\pgfpathlineto{\pgfqpoint{3.109846in}{2.372421in}}%
\pgfpathlineto{\pgfqpoint{3.114104in}{2.372421in}}%
\pgfpathlineto{\pgfqpoint{3.114104in}{2.368163in}}%
\pgfpathmoveto{\pgfqpoint{3.109846in}{2.372421in}}%
\pgfpathlineto{\pgfqpoint{3.109846in}{2.372421in}}%
\pgfpathlineto{\pgfqpoint{3.109846in}{2.376679in}}%
\pgfpathlineto{\pgfqpoint{3.114104in}{2.376679in}}%
\pgfpathlineto{\pgfqpoint{3.114104in}{2.372421in}}%
\pgfpathmoveto{\pgfqpoint{3.109846in}{2.376679in}}%
\pgfpathlineto{\pgfqpoint{3.109846in}{2.376679in}}%
\pgfpathlineto{\pgfqpoint{3.109846in}{2.380936in}}%
\pgfpathlineto{\pgfqpoint{3.114104in}{2.380936in}}%
\pgfpathlineto{\pgfqpoint{3.114104in}{2.376679in}}%
\pgfpathmoveto{\pgfqpoint{3.114104in}{2.376679in}}%
\pgfpathlineto{\pgfqpoint{3.114104in}{2.376679in}}%
\pgfpathlineto{\pgfqpoint{3.114104in}{2.380936in}}%
\pgfpathlineto{\pgfqpoint{3.118362in}{2.380936in}}%
\pgfpathlineto{\pgfqpoint{3.118362in}{2.376679in}}%
\pgfpathmoveto{\pgfqpoint{3.114104in}{2.380936in}}%
\pgfpathlineto{\pgfqpoint{3.114104in}{2.380936in}}%
\pgfpathlineto{\pgfqpoint{3.114104in}{2.385194in}}%
\pgfpathlineto{\pgfqpoint{3.118362in}{2.385194in}}%
\pgfpathlineto{\pgfqpoint{3.118362in}{2.380936in}}%
\pgfpathmoveto{\pgfqpoint{3.114104in}{2.385194in}}%
\pgfpathlineto{\pgfqpoint{3.114104in}{2.385194in}}%
\pgfpathlineto{\pgfqpoint{3.114104in}{2.389452in}}%
\pgfpathlineto{\pgfqpoint{3.118362in}{2.389452in}}%
\pgfpathlineto{\pgfqpoint{3.118362in}{2.385194in}}%
\pgfpathmoveto{\pgfqpoint{3.114104in}{2.389452in}}%
\pgfpathlineto{\pgfqpoint{3.114104in}{2.389452in}}%
\pgfpathlineto{\pgfqpoint{3.114104in}{2.393710in}}%
\pgfpathlineto{\pgfqpoint{3.118362in}{2.393710in}}%
\pgfpathlineto{\pgfqpoint{3.118362in}{2.389452in}}%
\pgfpathmoveto{\pgfqpoint{3.114104in}{2.393710in}}%
\pgfpathlineto{\pgfqpoint{3.114104in}{2.393710in}}%
\pgfpathlineto{\pgfqpoint{3.114104in}{2.397967in}}%
\pgfpathlineto{\pgfqpoint{3.118362in}{2.397967in}}%
\pgfpathlineto{\pgfqpoint{3.118362in}{2.393710in}}%
\pgfpathmoveto{\pgfqpoint{3.114104in}{2.397967in}}%
\pgfpathlineto{\pgfqpoint{3.114104in}{2.397967in}}%
\pgfpathlineto{\pgfqpoint{3.114104in}{2.402225in}}%
\pgfpathlineto{\pgfqpoint{3.118362in}{2.402225in}}%
\pgfpathlineto{\pgfqpoint{3.118362in}{2.397967in}}%
\pgfpathmoveto{\pgfqpoint{3.114104in}{2.402225in}}%
\pgfpathlineto{\pgfqpoint{3.114104in}{2.402225in}}%
\pgfpathlineto{\pgfqpoint{3.114104in}{2.406483in}}%
\pgfpathlineto{\pgfqpoint{3.118362in}{2.406483in}}%
\pgfpathlineto{\pgfqpoint{3.118362in}{2.402225in}}%
\pgfpathmoveto{\pgfqpoint{3.118362in}{2.402225in}}%
\pgfpathlineto{\pgfqpoint{3.118362in}{2.402225in}}%
\pgfpathlineto{\pgfqpoint{3.118362in}{2.406483in}}%
\pgfpathlineto{\pgfqpoint{3.122619in}{2.406483in}}%
\pgfpathlineto{\pgfqpoint{3.122619in}{2.402225in}}%
\pgfpathmoveto{\pgfqpoint{3.118362in}{2.406483in}}%
\pgfpathlineto{\pgfqpoint{3.118362in}{2.406483in}}%
\pgfpathlineto{\pgfqpoint{3.118362in}{2.410741in}}%
\pgfpathlineto{\pgfqpoint{3.122619in}{2.410741in}}%
\pgfpathlineto{\pgfqpoint{3.122619in}{2.406483in}}%
\pgfpathmoveto{\pgfqpoint{3.118362in}{2.410741in}}%
\pgfpathlineto{\pgfqpoint{3.118362in}{2.410741in}}%
\pgfpathlineto{\pgfqpoint{3.118362in}{2.414998in}}%
\pgfpathlineto{\pgfqpoint{3.122619in}{2.414998in}}%
\pgfpathlineto{\pgfqpoint{3.122619in}{2.410741in}}%
\pgfpathmoveto{\pgfqpoint{3.118362in}{2.414998in}}%
\pgfpathlineto{\pgfqpoint{3.118362in}{2.414998in}}%
\pgfpathlineto{\pgfqpoint{3.118362in}{2.419256in}}%
\pgfpathlineto{\pgfqpoint{3.122619in}{2.419256in}}%
\pgfpathlineto{\pgfqpoint{3.122619in}{2.414998in}}%
\pgfpathmoveto{\pgfqpoint{3.118362in}{2.419256in}}%
\pgfpathlineto{\pgfqpoint{3.118362in}{2.419256in}}%
\pgfpathlineto{\pgfqpoint{3.118362in}{2.423514in}}%
\pgfpathlineto{\pgfqpoint{3.122619in}{2.423514in}}%
\pgfpathlineto{\pgfqpoint{3.122619in}{2.419256in}}%
\pgfpathmoveto{\pgfqpoint{3.118362in}{2.423514in}}%
\pgfpathlineto{\pgfqpoint{3.118362in}{2.423514in}}%
\pgfpathlineto{\pgfqpoint{3.118362in}{2.427772in}}%
\pgfpathlineto{\pgfqpoint{3.122619in}{2.427772in}}%
\pgfpathlineto{\pgfqpoint{3.122619in}{2.423514in}}%
\pgfpathmoveto{\pgfqpoint{3.118362in}{2.427772in}}%
\pgfpathlineto{\pgfqpoint{3.118362in}{2.427772in}}%
\pgfpathlineto{\pgfqpoint{3.118362in}{2.432030in}}%
\pgfpathlineto{\pgfqpoint{3.122619in}{2.432030in}}%
\pgfpathlineto{\pgfqpoint{3.122619in}{2.427772in}}%
\pgfpathmoveto{\pgfqpoint{3.122619in}{2.427772in}}%
\pgfpathlineto{\pgfqpoint{3.122619in}{2.427772in}}%
\pgfpathlineto{\pgfqpoint{3.122619in}{2.432030in}}%
\pgfpathlineto{\pgfqpoint{3.126877in}{2.432030in}}%
\pgfpathlineto{\pgfqpoint{3.126877in}{2.427772in}}%
\pgfpathmoveto{\pgfqpoint{3.122619in}{2.432030in}}%
\pgfpathlineto{\pgfqpoint{3.122619in}{2.432030in}}%
\pgfpathlineto{\pgfqpoint{3.122619in}{2.436288in}}%
\pgfpathlineto{\pgfqpoint{3.126877in}{2.436288in}}%
\pgfpathlineto{\pgfqpoint{3.126877in}{2.432030in}}%
\pgfpathmoveto{\pgfqpoint{3.122619in}{2.436288in}}%
\pgfpathlineto{\pgfqpoint{3.122619in}{2.436288in}}%
\pgfpathlineto{\pgfqpoint{3.122619in}{2.440546in}}%
\pgfpathlineto{\pgfqpoint{3.126877in}{2.440546in}}%
\pgfpathlineto{\pgfqpoint{3.126877in}{2.436288in}}%
\pgfpathmoveto{\pgfqpoint{3.122619in}{2.440546in}}%
\pgfpathlineto{\pgfqpoint{3.122619in}{2.440546in}}%
\pgfpathlineto{\pgfqpoint{3.122619in}{2.444803in}}%
\pgfpathlineto{\pgfqpoint{3.126877in}{2.444803in}}%
\pgfpathlineto{\pgfqpoint{3.126877in}{2.440546in}}%
\pgfpathmoveto{\pgfqpoint{3.122619in}{2.444803in}}%
\pgfpathlineto{\pgfqpoint{3.122619in}{2.444803in}}%
\pgfpathlineto{\pgfqpoint{3.122619in}{2.449061in}}%
\pgfpathlineto{\pgfqpoint{3.126877in}{2.449061in}}%
\pgfpathlineto{\pgfqpoint{3.126877in}{2.444803in}}%
\pgfpathmoveto{\pgfqpoint{3.122619in}{2.449061in}}%
\pgfpathlineto{\pgfqpoint{3.122619in}{2.449061in}}%
\pgfpathlineto{\pgfqpoint{3.122619in}{2.453319in}}%
\pgfpathlineto{\pgfqpoint{3.126877in}{2.453319in}}%
\pgfpathlineto{\pgfqpoint{3.126877in}{2.449061in}}%
\pgfpathmoveto{\pgfqpoint{3.122619in}{2.453319in}}%
\pgfpathlineto{\pgfqpoint{3.122619in}{2.453319in}}%
\pgfpathlineto{\pgfqpoint{3.122619in}{2.457577in}}%
\pgfpathlineto{\pgfqpoint{3.126877in}{2.457577in}}%
\pgfpathlineto{\pgfqpoint{3.126877in}{2.453319in}}%
\pgfpathmoveto{\pgfqpoint{3.126877in}{2.453319in}}%
\pgfpathlineto{\pgfqpoint{3.126877in}{2.453319in}}%
\pgfpathlineto{\pgfqpoint{3.126877in}{2.457577in}}%
\pgfpathlineto{\pgfqpoint{3.131134in}{2.457577in}}%
\pgfpathlineto{\pgfqpoint{3.131134in}{2.453319in}}%
\pgfpathmoveto{\pgfqpoint{3.126877in}{2.457577in}}%
\pgfpathlineto{\pgfqpoint{3.126877in}{2.457577in}}%
\pgfpathlineto{\pgfqpoint{3.126877in}{2.461835in}}%
\pgfpathlineto{\pgfqpoint{3.131134in}{2.461835in}}%
\pgfpathlineto{\pgfqpoint{3.131134in}{2.457577in}}%
\pgfpathmoveto{\pgfqpoint{3.126877in}{2.461835in}}%
\pgfpathlineto{\pgfqpoint{3.126877in}{2.461835in}}%
\pgfpathlineto{\pgfqpoint{3.126877in}{2.466093in}}%
\pgfpathlineto{\pgfqpoint{3.131134in}{2.466093in}}%
\pgfpathlineto{\pgfqpoint{3.131134in}{2.461835in}}%
\pgfpathmoveto{\pgfqpoint{3.126877in}{2.466093in}}%
\pgfpathlineto{\pgfqpoint{3.126877in}{2.466093in}}%
\pgfpathlineto{\pgfqpoint{3.126877in}{2.470351in}}%
\pgfpathlineto{\pgfqpoint{3.131134in}{2.470351in}}%
\pgfpathlineto{\pgfqpoint{3.131134in}{2.466093in}}%
\pgfpathmoveto{\pgfqpoint{3.126877in}{2.470351in}}%
\pgfpathlineto{\pgfqpoint{3.126877in}{2.470351in}}%
\pgfpathlineto{\pgfqpoint{3.126877in}{2.474608in}}%
\pgfpathlineto{\pgfqpoint{3.131134in}{2.474608in}}%
\pgfpathlineto{\pgfqpoint{3.131134in}{2.470351in}}%
\pgfpathmoveto{\pgfqpoint{3.126877in}{2.474608in}}%
\pgfpathlineto{\pgfqpoint{3.126877in}{2.474608in}}%
\pgfpathlineto{\pgfqpoint{3.126877in}{2.478866in}}%
\pgfpathlineto{\pgfqpoint{3.131134in}{2.478866in}}%
\pgfpathlineto{\pgfqpoint{3.131134in}{2.474608in}}%
\pgfpathmoveto{\pgfqpoint{3.126877in}{2.478866in}}%
\pgfpathlineto{\pgfqpoint{3.126877in}{2.478866in}}%
\pgfpathlineto{\pgfqpoint{3.126877in}{2.483124in}}%
\pgfpathlineto{\pgfqpoint{3.131134in}{2.483124in}}%
\pgfpathlineto{\pgfqpoint{3.131134in}{2.478866in}}%
\pgfpathmoveto{\pgfqpoint{3.131134in}{2.478866in}}%
\pgfpathlineto{\pgfqpoint{3.131134in}{2.478866in}}%
\pgfpathlineto{\pgfqpoint{3.131134in}{2.483124in}}%
\pgfpathlineto{\pgfqpoint{3.135392in}{2.483124in}}%
\pgfpathlineto{\pgfqpoint{3.135392in}{2.478866in}}%
\pgfpathmoveto{\pgfqpoint{3.131134in}{2.483124in}}%
\pgfpathlineto{\pgfqpoint{3.131134in}{2.483124in}}%
\pgfpathlineto{\pgfqpoint{3.131134in}{2.487382in}}%
\pgfpathlineto{\pgfqpoint{3.135392in}{2.487382in}}%
\pgfpathlineto{\pgfqpoint{3.135392in}{2.483124in}}%
\pgfpathmoveto{\pgfqpoint{3.131134in}{2.487382in}}%
\pgfpathlineto{\pgfqpoint{3.131134in}{2.487382in}}%
\pgfpathlineto{\pgfqpoint{3.131134in}{2.491640in}}%
\pgfpathlineto{\pgfqpoint{3.135392in}{2.491640in}}%
\pgfpathlineto{\pgfqpoint{3.135392in}{2.487382in}}%
\pgfpathmoveto{\pgfqpoint{3.131134in}{2.491640in}}%
\pgfpathlineto{\pgfqpoint{3.131134in}{2.491640in}}%
\pgfpathlineto{\pgfqpoint{3.131134in}{2.495898in}}%
\pgfpathlineto{\pgfqpoint{3.135392in}{2.495898in}}%
\pgfpathlineto{\pgfqpoint{3.135392in}{2.491640in}}%
\pgfpathmoveto{\pgfqpoint{3.131134in}{2.495898in}}%
\pgfpathlineto{\pgfqpoint{3.131134in}{2.495898in}}%
\pgfpathlineto{\pgfqpoint{3.131134in}{2.500156in}}%
\pgfpathlineto{\pgfqpoint{3.135392in}{2.500156in}}%
\pgfpathlineto{\pgfqpoint{3.135392in}{2.495898in}}%
\pgfpathmoveto{\pgfqpoint{3.131134in}{2.500156in}}%
\pgfpathlineto{\pgfqpoint{3.131134in}{2.500156in}}%
\pgfpathlineto{\pgfqpoint{3.131134in}{2.504413in}}%
\pgfpathlineto{\pgfqpoint{3.135392in}{2.504413in}}%
\pgfpathlineto{\pgfqpoint{3.135392in}{2.500156in}}%
\pgfpathmoveto{\pgfqpoint{3.131134in}{2.504413in}}%
\pgfpathlineto{\pgfqpoint{3.131134in}{2.504413in}}%
\pgfpathlineto{\pgfqpoint{3.131134in}{2.508671in}}%
\pgfpathlineto{\pgfqpoint{3.135392in}{2.508671in}}%
\pgfpathlineto{\pgfqpoint{3.135392in}{2.504413in}}%
\pgfpathmoveto{\pgfqpoint{3.135392in}{2.504413in}}%
\pgfpathlineto{\pgfqpoint{3.135392in}{2.504413in}}%
\pgfpathlineto{\pgfqpoint{3.135392in}{2.508671in}}%
\pgfpathlineto{\pgfqpoint{3.139649in}{2.508671in}}%
\pgfpathlineto{\pgfqpoint{3.139649in}{2.504413in}}%
\pgfpathmoveto{\pgfqpoint{3.135392in}{2.508671in}}%
\pgfpathlineto{\pgfqpoint{3.135392in}{2.508671in}}%
\pgfpathlineto{\pgfqpoint{3.135392in}{2.512929in}}%
\pgfpathlineto{\pgfqpoint{3.139649in}{2.512929in}}%
\pgfpathlineto{\pgfqpoint{3.139649in}{2.508671in}}%
\pgfpathmoveto{\pgfqpoint{3.135392in}{2.512929in}}%
\pgfpathlineto{\pgfqpoint{3.135392in}{2.512929in}}%
\pgfpathlineto{\pgfqpoint{3.135392in}{2.517187in}}%
\pgfpathlineto{\pgfqpoint{3.139649in}{2.517187in}}%
\pgfpathlineto{\pgfqpoint{3.139649in}{2.512929in}}%
\pgfpathmoveto{\pgfqpoint{3.135392in}{2.517187in}}%
\pgfpathlineto{\pgfqpoint{3.135392in}{2.517187in}}%
\pgfpathlineto{\pgfqpoint{3.135392in}{2.521445in}}%
\pgfpathlineto{\pgfqpoint{3.139649in}{2.521445in}}%
\pgfpathlineto{\pgfqpoint{3.139649in}{2.517187in}}%
\pgfpathmoveto{\pgfqpoint{3.135392in}{2.521445in}}%
\pgfpathlineto{\pgfqpoint{3.135392in}{2.521445in}}%
\pgfpathlineto{\pgfqpoint{3.135392in}{2.525703in}}%
\pgfpathlineto{\pgfqpoint{3.139649in}{2.525703in}}%
\pgfpathlineto{\pgfqpoint{3.139649in}{2.521445in}}%
\pgfpathmoveto{\pgfqpoint{3.135392in}{2.525703in}}%
\pgfpathlineto{\pgfqpoint{3.135392in}{2.525703in}}%
\pgfpathlineto{\pgfqpoint{3.135392in}{2.529961in}}%
\pgfpathlineto{\pgfqpoint{3.139649in}{2.529961in}}%
\pgfpathlineto{\pgfqpoint{3.139649in}{2.525703in}}%
\pgfpathmoveto{\pgfqpoint{3.139649in}{2.525703in}}%
\pgfpathlineto{\pgfqpoint{3.139649in}{2.525703in}}%
\pgfpathlineto{\pgfqpoint{3.139649in}{2.529961in}}%
\pgfpathlineto{\pgfqpoint{3.143907in}{2.529961in}}%
\pgfpathlineto{\pgfqpoint{3.143907in}{2.525703in}}%
\pgfpathmoveto{\pgfqpoint{3.139649in}{2.529961in}}%
\pgfpathlineto{\pgfqpoint{3.139649in}{2.529961in}}%
\pgfpathlineto{\pgfqpoint{3.139649in}{2.534218in}}%
\pgfpathlineto{\pgfqpoint{3.143907in}{2.534218in}}%
\pgfpathlineto{\pgfqpoint{3.143907in}{2.529961in}}%
\pgfpathmoveto{\pgfqpoint{3.139649in}{2.534218in}}%
\pgfpathlineto{\pgfqpoint{3.139649in}{2.534218in}}%
\pgfpathlineto{\pgfqpoint{3.139649in}{2.538476in}}%
\pgfpathlineto{\pgfqpoint{3.143907in}{2.538476in}}%
\pgfpathlineto{\pgfqpoint{3.143907in}{2.534218in}}%
\pgfpathmoveto{\pgfqpoint{3.139649in}{2.538476in}}%
\pgfpathlineto{\pgfqpoint{3.139649in}{2.538476in}}%
\pgfpathlineto{\pgfqpoint{3.139649in}{2.542734in}}%
\pgfpathlineto{\pgfqpoint{3.143907in}{2.542734in}}%
\pgfpathlineto{\pgfqpoint{3.143907in}{2.538476in}}%
\pgfpathmoveto{\pgfqpoint{3.139649in}{2.542734in}}%
\pgfpathlineto{\pgfqpoint{3.139649in}{2.542734in}}%
\pgfpathlineto{\pgfqpoint{3.139649in}{2.546992in}}%
\pgfpathlineto{\pgfqpoint{3.143907in}{2.546992in}}%
\pgfpathlineto{\pgfqpoint{3.143907in}{2.542734in}}%
\pgfpathmoveto{\pgfqpoint{3.139649in}{2.546992in}}%
\pgfpathlineto{\pgfqpoint{3.139649in}{2.546992in}}%
\pgfpathlineto{\pgfqpoint{3.139649in}{2.551250in}}%
\pgfpathlineto{\pgfqpoint{3.143907in}{2.551250in}}%
\pgfpathlineto{\pgfqpoint{3.143907in}{2.546992in}}%
\pgfpathmoveto{\pgfqpoint{3.139649in}{2.551250in}}%
\pgfpathlineto{\pgfqpoint{3.139649in}{2.551250in}}%
\pgfpathlineto{\pgfqpoint{3.139649in}{2.555508in}}%
\pgfpathlineto{\pgfqpoint{3.143907in}{2.555508in}}%
\pgfpathlineto{\pgfqpoint{3.143907in}{2.551250in}}%
\pgfpathmoveto{\pgfqpoint{3.143907in}{2.551250in}}%
\pgfpathlineto{\pgfqpoint{3.143907in}{2.551250in}}%
\pgfpathlineto{\pgfqpoint{3.143907in}{2.555508in}}%
\pgfpathlineto{\pgfqpoint{3.148165in}{2.555508in}}%
\pgfpathlineto{\pgfqpoint{3.148165in}{2.551250in}}%
\pgfpathmoveto{\pgfqpoint{3.143907in}{2.555508in}}%
\pgfpathlineto{\pgfqpoint{3.143907in}{2.555508in}}%
\pgfpathlineto{\pgfqpoint{3.143907in}{2.559766in}}%
\pgfpathlineto{\pgfqpoint{3.148165in}{2.559766in}}%
\pgfpathlineto{\pgfqpoint{3.148165in}{2.555508in}}%
\pgfpathmoveto{\pgfqpoint{3.143907in}{2.559766in}}%
\pgfpathlineto{\pgfqpoint{3.143907in}{2.559766in}}%
\pgfpathlineto{\pgfqpoint{3.143907in}{2.564023in}}%
\pgfpathlineto{\pgfqpoint{3.148165in}{2.564023in}}%
\pgfpathlineto{\pgfqpoint{3.148165in}{2.559766in}}%
\pgfpathmoveto{\pgfqpoint{3.143907in}{2.564023in}}%
\pgfpathlineto{\pgfqpoint{3.143907in}{2.564023in}}%
\pgfpathlineto{\pgfqpoint{3.143907in}{2.568281in}}%
\pgfpathlineto{\pgfqpoint{3.148165in}{2.568281in}}%
\pgfpathlineto{\pgfqpoint{3.148165in}{2.564023in}}%
\pgfpathmoveto{\pgfqpoint{3.143907in}{2.568281in}}%
\pgfpathlineto{\pgfqpoint{3.143907in}{2.568281in}}%
\pgfpathlineto{\pgfqpoint{3.143907in}{2.572539in}}%
\pgfpathlineto{\pgfqpoint{3.148165in}{2.572539in}}%
\pgfpathlineto{\pgfqpoint{3.148165in}{2.568281in}}%
\pgfpathmoveto{\pgfqpoint{3.143907in}{2.572539in}}%
\pgfpathlineto{\pgfqpoint{3.143907in}{2.572539in}}%
\pgfpathlineto{\pgfqpoint{3.143907in}{2.576797in}}%
\pgfpathlineto{\pgfqpoint{3.148165in}{2.576797in}}%
\pgfpathlineto{\pgfqpoint{3.148165in}{2.572539in}}%
\pgfpathmoveto{\pgfqpoint{3.148165in}{2.572539in}}%
\pgfpathlineto{\pgfqpoint{3.148165in}{2.572539in}}%
\pgfpathlineto{\pgfqpoint{3.148165in}{2.576797in}}%
\pgfpathlineto{\pgfqpoint{3.152422in}{2.576797in}}%
\pgfpathlineto{\pgfqpoint{3.152422in}{2.572539in}}%
\pgfpathmoveto{\pgfqpoint{3.148165in}{2.576797in}}%
\pgfpathlineto{\pgfqpoint{3.148165in}{2.576797in}}%
\pgfpathlineto{\pgfqpoint{3.148165in}{2.581054in}}%
\pgfpathlineto{\pgfqpoint{3.152422in}{2.581054in}}%
\pgfpathlineto{\pgfqpoint{3.152422in}{2.576797in}}%
\pgfpathmoveto{\pgfqpoint{3.148165in}{2.581054in}}%
\pgfpathlineto{\pgfqpoint{3.148165in}{2.581054in}}%
\pgfpathlineto{\pgfqpoint{3.148165in}{2.585312in}}%
\pgfpathlineto{\pgfqpoint{3.152422in}{2.585312in}}%
\pgfpathlineto{\pgfqpoint{3.152422in}{2.581054in}}%
\pgfpathmoveto{\pgfqpoint{3.148165in}{2.585312in}}%
\pgfpathlineto{\pgfqpoint{3.148165in}{2.585312in}}%
\pgfpathlineto{\pgfqpoint{3.148165in}{2.589570in}}%
\pgfpathlineto{\pgfqpoint{3.152422in}{2.589570in}}%
\pgfpathlineto{\pgfqpoint{3.152422in}{2.585312in}}%
\pgfpathmoveto{\pgfqpoint{3.148165in}{2.589570in}}%
\pgfpathlineto{\pgfqpoint{3.148165in}{2.589570in}}%
\pgfpathlineto{\pgfqpoint{3.148165in}{2.593828in}}%
\pgfpathlineto{\pgfqpoint{3.152422in}{2.593828in}}%
\pgfpathlineto{\pgfqpoint{3.152422in}{2.589570in}}%
\pgfpathmoveto{\pgfqpoint{3.148165in}{2.593828in}}%
\pgfpathlineto{\pgfqpoint{3.148165in}{2.593828in}}%
\pgfpathlineto{\pgfqpoint{3.148165in}{2.598086in}}%
\pgfpathlineto{\pgfqpoint{3.152422in}{2.598086in}}%
\pgfpathlineto{\pgfqpoint{3.152422in}{2.593828in}}%
\pgfpathmoveto{\pgfqpoint{3.148165in}{2.598086in}}%
\pgfpathlineto{\pgfqpoint{3.148165in}{2.598086in}}%
\pgfpathlineto{\pgfqpoint{3.148165in}{2.602343in}}%
\pgfpathlineto{\pgfqpoint{3.152422in}{2.602343in}}%
\pgfpathlineto{\pgfqpoint{3.152422in}{2.598086in}}%
\pgfpathmoveto{\pgfqpoint{3.152422in}{2.598086in}}%
\pgfpathlineto{\pgfqpoint{3.152422in}{2.598086in}}%
\pgfpathlineto{\pgfqpoint{3.152422in}{2.602343in}}%
\pgfpathlineto{\pgfqpoint{3.156680in}{2.602343in}}%
\pgfpathlineto{\pgfqpoint{3.156680in}{2.598086in}}%
\pgfpathmoveto{\pgfqpoint{3.152422in}{2.602343in}}%
\pgfpathlineto{\pgfqpoint{3.152422in}{2.602343in}}%
\pgfpathlineto{\pgfqpoint{3.152422in}{2.606601in}}%
\pgfpathlineto{\pgfqpoint{3.156680in}{2.606601in}}%
\pgfpathlineto{\pgfqpoint{3.156680in}{2.602343in}}%
\pgfpathmoveto{\pgfqpoint{3.152422in}{2.606601in}}%
\pgfpathlineto{\pgfqpoint{3.152422in}{2.606601in}}%
\pgfpathlineto{\pgfqpoint{3.152422in}{2.610859in}}%
\pgfpathlineto{\pgfqpoint{3.156680in}{2.610859in}}%
\pgfpathlineto{\pgfqpoint{3.156680in}{2.606601in}}%
\pgfpathmoveto{\pgfqpoint{3.152422in}{2.610859in}}%
\pgfpathlineto{\pgfqpoint{3.152422in}{2.610859in}}%
\pgfpathlineto{\pgfqpoint{3.152422in}{2.615117in}}%
\pgfpathlineto{\pgfqpoint{3.156680in}{2.615117in}}%
\pgfpathlineto{\pgfqpoint{3.156680in}{2.610859in}}%
\pgfpathmoveto{\pgfqpoint{3.152422in}{2.615117in}}%
\pgfpathlineto{\pgfqpoint{3.152422in}{2.615117in}}%
\pgfpathlineto{\pgfqpoint{3.152422in}{2.619375in}}%
\pgfpathlineto{\pgfqpoint{3.156680in}{2.619375in}}%
\pgfpathlineto{\pgfqpoint{3.156680in}{2.615117in}}%
\pgfpathmoveto{\pgfqpoint{3.152422in}{2.619375in}}%
\pgfpathlineto{\pgfqpoint{3.152422in}{2.619375in}}%
\pgfpathlineto{\pgfqpoint{3.152422in}{2.623632in}}%
\pgfpathlineto{\pgfqpoint{3.156680in}{2.623632in}}%
\pgfpathlineto{\pgfqpoint{3.156680in}{2.619375in}}%
\pgfpathmoveto{\pgfqpoint{3.156680in}{2.619375in}}%
\pgfpathlineto{\pgfqpoint{3.156680in}{2.619375in}}%
\pgfpathlineto{\pgfqpoint{3.156680in}{2.623632in}}%
\pgfpathlineto{\pgfqpoint{3.160937in}{2.623632in}}%
\pgfpathlineto{\pgfqpoint{3.160937in}{2.619375in}}%
\pgfpathmoveto{\pgfqpoint{3.156680in}{2.623632in}}%
\pgfpathlineto{\pgfqpoint{3.156680in}{2.623632in}}%
\pgfpathlineto{\pgfqpoint{3.156680in}{2.627890in}}%
\pgfpathlineto{\pgfqpoint{3.160937in}{2.627890in}}%
\pgfpathlineto{\pgfqpoint{3.160937in}{2.623632in}}%
\pgfpathmoveto{\pgfqpoint{3.156680in}{2.627890in}}%
\pgfpathlineto{\pgfqpoint{3.156680in}{2.627890in}}%
\pgfpathlineto{\pgfqpoint{3.156680in}{2.632148in}}%
\pgfpathlineto{\pgfqpoint{3.160937in}{2.632148in}}%
\pgfpathlineto{\pgfqpoint{3.160937in}{2.627890in}}%
\pgfpathmoveto{\pgfqpoint{3.156680in}{2.632148in}}%
\pgfpathlineto{\pgfqpoint{3.156680in}{2.632148in}}%
\pgfpathlineto{\pgfqpoint{3.156680in}{2.636406in}}%
\pgfpathlineto{\pgfqpoint{3.160937in}{2.636406in}}%
\pgfpathlineto{\pgfqpoint{3.160937in}{2.632148in}}%
\pgfpathmoveto{\pgfqpoint{3.156680in}{2.636406in}}%
\pgfpathlineto{\pgfqpoint{3.156680in}{2.636406in}}%
\pgfpathlineto{\pgfqpoint{3.156680in}{2.640664in}}%
\pgfpathlineto{\pgfqpoint{3.160937in}{2.640664in}}%
\pgfpathlineto{\pgfqpoint{3.160937in}{2.636406in}}%
\pgfpathmoveto{\pgfqpoint{3.156680in}{2.640664in}}%
\pgfpathlineto{\pgfqpoint{3.156680in}{2.640664in}}%
\pgfpathlineto{\pgfqpoint{3.156680in}{2.644921in}}%
\pgfpathlineto{\pgfqpoint{3.160937in}{2.644921in}}%
\pgfpathlineto{\pgfqpoint{3.160937in}{2.640664in}}%
\pgfpathmoveto{\pgfqpoint{3.160937in}{2.640664in}}%
\pgfpathlineto{\pgfqpoint{3.160937in}{2.640664in}}%
\pgfpathlineto{\pgfqpoint{3.160937in}{2.644921in}}%
\pgfpathlineto{\pgfqpoint{3.165195in}{2.644921in}}%
\pgfpathlineto{\pgfqpoint{3.165195in}{2.640664in}}%
\pgfpathmoveto{\pgfqpoint{3.160937in}{2.644921in}}%
\pgfpathlineto{\pgfqpoint{3.160937in}{2.644921in}}%
\pgfpathlineto{\pgfqpoint{3.160937in}{2.649179in}}%
\pgfpathlineto{\pgfqpoint{3.165195in}{2.649179in}}%
\pgfpathlineto{\pgfqpoint{3.165195in}{2.644921in}}%
\pgfpathmoveto{\pgfqpoint{3.160937in}{2.649179in}}%
\pgfpathlineto{\pgfqpoint{3.160937in}{2.649179in}}%
\pgfpathlineto{\pgfqpoint{3.160937in}{2.653437in}}%
\pgfpathlineto{\pgfqpoint{3.165195in}{2.653437in}}%
\pgfpathlineto{\pgfqpoint{3.165195in}{2.649179in}}%
\pgfpathmoveto{\pgfqpoint{3.160937in}{2.653437in}}%
\pgfpathlineto{\pgfqpoint{3.160937in}{2.653437in}}%
\pgfpathlineto{\pgfqpoint{3.160937in}{2.657695in}}%
\pgfpathlineto{\pgfqpoint{3.165195in}{2.657695in}}%
\pgfpathlineto{\pgfqpoint{3.165195in}{2.653437in}}%
\pgfpathmoveto{\pgfqpoint{3.160937in}{2.657695in}}%
\pgfpathlineto{\pgfqpoint{3.160937in}{2.657695in}}%
\pgfpathlineto{\pgfqpoint{3.160937in}{2.661953in}}%
\pgfpathlineto{\pgfqpoint{3.165195in}{2.661953in}}%
\pgfpathlineto{\pgfqpoint{3.165195in}{2.657695in}}%
\pgfpathmoveto{\pgfqpoint{3.160937in}{2.661953in}}%
\pgfpathlineto{\pgfqpoint{3.160937in}{2.661953in}}%
\pgfpathlineto{\pgfqpoint{3.160937in}{2.666210in}}%
\pgfpathlineto{\pgfqpoint{3.165195in}{2.666210in}}%
\pgfpathlineto{\pgfqpoint{3.165195in}{2.661953in}}%
\pgfpathmoveto{\pgfqpoint{3.160937in}{2.666210in}}%
\pgfpathlineto{\pgfqpoint{3.160937in}{2.666210in}}%
\pgfpathlineto{\pgfqpoint{3.160937in}{2.670468in}}%
\pgfpathlineto{\pgfqpoint{3.165195in}{2.670468in}}%
\pgfpathlineto{\pgfqpoint{3.165195in}{2.666210in}}%
\pgfpathmoveto{\pgfqpoint{3.165195in}{2.666210in}}%
\pgfpathlineto{\pgfqpoint{3.165195in}{2.666210in}}%
\pgfpathlineto{\pgfqpoint{3.165195in}{2.670468in}}%
\pgfpathlineto{\pgfqpoint{3.169452in}{2.670468in}}%
\pgfpathlineto{\pgfqpoint{3.169452in}{2.666210in}}%
\pgfpathmoveto{\pgfqpoint{3.165195in}{2.670468in}}%
\pgfpathlineto{\pgfqpoint{3.165195in}{2.670468in}}%
\pgfpathlineto{\pgfqpoint{3.165195in}{2.674726in}}%
\pgfpathlineto{\pgfqpoint{3.169452in}{2.674726in}}%
\pgfpathlineto{\pgfqpoint{3.169452in}{2.670468in}}%
\pgfpathmoveto{\pgfqpoint{3.165195in}{2.674726in}}%
\pgfpathlineto{\pgfqpoint{3.165195in}{2.674726in}}%
\pgfpathlineto{\pgfqpoint{3.165195in}{2.678984in}}%
\pgfpathlineto{\pgfqpoint{3.169452in}{2.678984in}}%
\pgfpathlineto{\pgfqpoint{3.169452in}{2.674726in}}%
\pgfpathmoveto{\pgfqpoint{3.165195in}{2.678984in}}%
\pgfpathlineto{\pgfqpoint{3.165195in}{2.678984in}}%
\pgfpathlineto{\pgfqpoint{3.165195in}{2.683242in}}%
\pgfpathlineto{\pgfqpoint{3.169452in}{2.683242in}}%
\pgfpathlineto{\pgfqpoint{3.169452in}{2.678984in}}%
\pgfpathmoveto{\pgfqpoint{3.165195in}{2.683242in}}%
\pgfpathlineto{\pgfqpoint{3.165195in}{2.683242in}}%
\pgfpathlineto{\pgfqpoint{3.165195in}{2.687499in}}%
\pgfpathlineto{\pgfqpoint{3.169452in}{2.687499in}}%
\pgfpathlineto{\pgfqpoint{3.169452in}{2.683242in}}%
\pgfpathmoveto{\pgfqpoint{3.165195in}{2.687499in}}%
\pgfpathlineto{\pgfqpoint{3.165195in}{2.687499in}}%
\pgfpathlineto{\pgfqpoint{3.165195in}{2.691757in}}%
\pgfpathlineto{\pgfqpoint{3.169452in}{2.691757in}}%
\pgfpathlineto{\pgfqpoint{3.169452in}{2.687499in}}%
\pgfpathmoveto{\pgfqpoint{3.169452in}{2.687499in}}%
\pgfpathlineto{\pgfqpoint{3.169452in}{2.687499in}}%
\pgfpathlineto{\pgfqpoint{3.169452in}{2.691757in}}%
\pgfpathlineto{\pgfqpoint{3.173710in}{2.691757in}}%
\pgfpathlineto{\pgfqpoint{3.173710in}{2.687499in}}%
\pgfpathmoveto{\pgfqpoint{3.169452in}{2.691757in}}%
\pgfpathlineto{\pgfqpoint{3.169452in}{2.691757in}}%
\pgfpathlineto{\pgfqpoint{3.169452in}{2.696015in}}%
\pgfpathlineto{\pgfqpoint{3.173710in}{2.696015in}}%
\pgfpathlineto{\pgfqpoint{3.173710in}{2.691757in}}%
\pgfpathmoveto{\pgfqpoint{3.169452in}{2.696015in}}%
\pgfpathlineto{\pgfqpoint{3.169452in}{2.696015in}}%
\pgfpathlineto{\pgfqpoint{3.169452in}{2.700273in}}%
\pgfpathlineto{\pgfqpoint{3.173710in}{2.700273in}}%
\pgfpathlineto{\pgfqpoint{3.173710in}{2.696015in}}%
\pgfpathmoveto{\pgfqpoint{3.169452in}{2.700273in}}%
\pgfpathlineto{\pgfqpoint{3.169452in}{2.700273in}}%
\pgfpathlineto{\pgfqpoint{3.169452in}{2.704531in}}%
\pgfpathlineto{\pgfqpoint{3.173710in}{2.704531in}}%
\pgfpathlineto{\pgfqpoint{3.173710in}{2.700273in}}%
\pgfpathmoveto{\pgfqpoint{3.169452in}{2.704531in}}%
\pgfpathlineto{\pgfqpoint{3.169452in}{2.704531in}}%
\pgfpathlineto{\pgfqpoint{3.169452in}{2.708788in}}%
\pgfpathlineto{\pgfqpoint{3.173710in}{2.708788in}}%
\pgfpathlineto{\pgfqpoint{3.173710in}{2.704531in}}%
\pgfpathmoveto{\pgfqpoint{3.169452in}{2.708788in}}%
\pgfpathlineto{\pgfqpoint{3.169452in}{2.708788in}}%
\pgfpathlineto{\pgfqpoint{3.169452in}{2.713046in}}%
\pgfpathlineto{\pgfqpoint{3.173710in}{2.713046in}}%
\pgfpathlineto{\pgfqpoint{3.173710in}{2.708788in}}%
\pgfpathmoveto{\pgfqpoint{3.173710in}{2.708788in}}%
\pgfpathlineto{\pgfqpoint{3.173710in}{2.708788in}}%
\pgfpathlineto{\pgfqpoint{3.173710in}{2.713046in}}%
\pgfpathlineto{\pgfqpoint{3.177968in}{2.713046in}}%
\pgfpathlineto{\pgfqpoint{3.177968in}{2.708788in}}%
\pgfpathmoveto{\pgfqpoint{3.173710in}{2.713046in}}%
\pgfpathlineto{\pgfqpoint{3.173710in}{2.713046in}}%
\pgfpathlineto{\pgfqpoint{3.173710in}{2.717304in}}%
\pgfpathlineto{\pgfqpoint{3.177968in}{2.717304in}}%
\pgfpathlineto{\pgfqpoint{3.177968in}{2.713046in}}%
\pgfpathmoveto{\pgfqpoint{3.173710in}{2.717304in}}%
\pgfpathlineto{\pgfqpoint{3.173710in}{2.717304in}}%
\pgfpathlineto{\pgfqpoint{3.173710in}{2.721562in}}%
\pgfpathlineto{\pgfqpoint{3.177968in}{2.721562in}}%
\pgfpathlineto{\pgfqpoint{3.177968in}{2.717304in}}%
\pgfpathmoveto{\pgfqpoint{3.173710in}{2.721562in}}%
\pgfpathlineto{\pgfqpoint{3.173710in}{2.721562in}}%
\pgfpathlineto{\pgfqpoint{3.173710in}{2.725819in}}%
\pgfpathlineto{\pgfqpoint{3.177968in}{2.725819in}}%
\pgfpathlineto{\pgfqpoint{3.177968in}{2.721562in}}%
\pgfpathmoveto{\pgfqpoint{3.173710in}{2.725819in}}%
\pgfpathlineto{\pgfqpoint{3.173710in}{2.725819in}}%
\pgfpathlineto{\pgfqpoint{3.173710in}{2.730077in}}%
\pgfpathlineto{\pgfqpoint{3.177968in}{2.730077in}}%
\pgfpathlineto{\pgfqpoint{3.177968in}{2.725819in}}%
\pgfpathmoveto{\pgfqpoint{3.173710in}{2.730077in}}%
\pgfpathlineto{\pgfqpoint{3.173710in}{2.730077in}}%
\pgfpathlineto{\pgfqpoint{3.173710in}{2.734335in}}%
\pgfpathlineto{\pgfqpoint{3.177968in}{2.734335in}}%
\pgfpathlineto{\pgfqpoint{3.177968in}{2.730077in}}%
\pgfpathmoveto{\pgfqpoint{3.177968in}{2.730077in}}%
\pgfpathlineto{\pgfqpoint{3.177968in}{2.730077in}}%
\pgfpathlineto{\pgfqpoint{3.177968in}{2.734335in}}%
\pgfpathlineto{\pgfqpoint{3.182225in}{2.734335in}}%
\pgfpathlineto{\pgfqpoint{3.182225in}{2.730077in}}%
\pgfpathmoveto{\pgfqpoint{3.177968in}{2.734335in}}%
\pgfpathlineto{\pgfqpoint{3.177968in}{2.734335in}}%
\pgfpathlineto{\pgfqpoint{3.177968in}{2.738593in}}%
\pgfpathlineto{\pgfqpoint{3.182225in}{2.738593in}}%
\pgfpathlineto{\pgfqpoint{3.182225in}{2.734335in}}%
\pgfpathmoveto{\pgfqpoint{3.177968in}{2.738593in}}%
\pgfpathlineto{\pgfqpoint{3.177968in}{2.738593in}}%
\pgfpathlineto{\pgfqpoint{3.177968in}{2.742851in}}%
\pgfpathlineto{\pgfqpoint{3.182225in}{2.742851in}}%
\pgfpathlineto{\pgfqpoint{3.182225in}{2.738593in}}%
\pgfpathmoveto{\pgfqpoint{3.177968in}{2.742851in}}%
\pgfpathlineto{\pgfqpoint{3.177968in}{2.742851in}}%
\pgfpathlineto{\pgfqpoint{3.177968in}{2.747108in}}%
\pgfpathlineto{\pgfqpoint{3.182225in}{2.747108in}}%
\pgfpathlineto{\pgfqpoint{3.182225in}{2.742851in}}%
\pgfpathmoveto{\pgfqpoint{3.177968in}{2.747108in}}%
\pgfpathlineto{\pgfqpoint{3.177968in}{2.747108in}}%
\pgfpathlineto{\pgfqpoint{3.177968in}{2.751366in}}%
\pgfpathlineto{\pgfqpoint{3.182225in}{2.751366in}}%
\pgfpathlineto{\pgfqpoint{3.182225in}{2.747108in}}%
\pgfpathmoveto{\pgfqpoint{3.177968in}{2.751366in}}%
\pgfpathlineto{\pgfqpoint{3.177968in}{2.751366in}}%
\pgfpathlineto{\pgfqpoint{3.177968in}{2.755624in}}%
\pgfpathlineto{\pgfqpoint{3.182225in}{2.755624in}}%
\pgfpathlineto{\pgfqpoint{3.182225in}{2.751366in}}%
\pgfpathmoveto{\pgfqpoint{3.182225in}{2.751366in}}%
\pgfpathlineto{\pgfqpoint{3.182225in}{2.751366in}}%
\pgfpathlineto{\pgfqpoint{3.182225in}{2.755624in}}%
\pgfpathlineto{\pgfqpoint{3.186483in}{2.755624in}}%
\pgfpathlineto{\pgfqpoint{3.186483in}{2.751366in}}%
\pgfpathmoveto{\pgfqpoint{3.182225in}{2.755624in}}%
\pgfpathlineto{\pgfqpoint{3.182225in}{2.755624in}}%
\pgfpathlineto{\pgfqpoint{3.182225in}{2.759882in}}%
\pgfpathlineto{\pgfqpoint{3.186483in}{2.759882in}}%
\pgfpathlineto{\pgfqpoint{3.186483in}{2.755624in}}%
\pgfpathmoveto{\pgfqpoint{3.182225in}{2.759882in}}%
\pgfpathlineto{\pgfqpoint{3.182225in}{2.759882in}}%
\pgfpathlineto{\pgfqpoint{3.182225in}{2.764140in}}%
\pgfpathlineto{\pgfqpoint{3.186483in}{2.764140in}}%
\pgfpathlineto{\pgfqpoint{3.186483in}{2.759882in}}%
\pgfpathmoveto{\pgfqpoint{3.182225in}{2.764140in}}%
\pgfpathlineto{\pgfqpoint{3.182225in}{2.764140in}}%
\pgfpathlineto{\pgfqpoint{3.182225in}{2.768397in}}%
\pgfpathlineto{\pgfqpoint{3.186483in}{2.768397in}}%
\pgfpathlineto{\pgfqpoint{3.186483in}{2.764140in}}%
\pgfpathmoveto{\pgfqpoint{3.182225in}{2.768397in}}%
\pgfpathlineto{\pgfqpoint{3.182225in}{2.768397in}}%
\pgfpathlineto{\pgfqpoint{3.182225in}{2.772655in}}%
\pgfpathlineto{\pgfqpoint{3.186483in}{2.772655in}}%
\pgfpathlineto{\pgfqpoint{3.186483in}{2.768397in}}%
\pgfpathmoveto{\pgfqpoint{3.182225in}{2.772655in}}%
\pgfpathlineto{\pgfqpoint{3.182225in}{2.772655in}}%
\pgfpathlineto{\pgfqpoint{3.182225in}{2.776913in}}%
\pgfpathlineto{\pgfqpoint{3.186483in}{2.776913in}}%
\pgfpathlineto{\pgfqpoint{3.186483in}{2.772655in}}%
\pgfpathmoveto{\pgfqpoint{3.186483in}{2.772655in}}%
\pgfpathlineto{\pgfqpoint{3.186483in}{2.772655in}}%
\pgfpathlineto{\pgfqpoint{3.186483in}{2.776913in}}%
\pgfpathlineto{\pgfqpoint{3.190740in}{2.776913in}}%
\pgfpathlineto{\pgfqpoint{3.190740in}{2.772655in}}%
\pgfpathmoveto{\pgfqpoint{3.186483in}{2.776913in}}%
\pgfpathlineto{\pgfqpoint{3.186483in}{2.776913in}}%
\pgfpathlineto{\pgfqpoint{3.186483in}{2.781171in}}%
\pgfpathlineto{\pgfqpoint{3.190740in}{2.781171in}}%
\pgfpathlineto{\pgfqpoint{3.190740in}{2.776913in}}%
\pgfpathmoveto{\pgfqpoint{3.186483in}{2.781171in}}%
\pgfpathlineto{\pgfqpoint{3.186483in}{2.781171in}}%
\pgfpathlineto{\pgfqpoint{3.186483in}{2.785428in}}%
\pgfpathlineto{\pgfqpoint{3.190740in}{2.785428in}}%
\pgfpathlineto{\pgfqpoint{3.190740in}{2.781171in}}%
\pgfpathmoveto{\pgfqpoint{3.186483in}{2.785428in}}%
\pgfpathlineto{\pgfqpoint{3.186483in}{2.785428in}}%
\pgfpathlineto{\pgfqpoint{3.186483in}{2.789686in}}%
\pgfpathlineto{\pgfqpoint{3.190740in}{2.789686in}}%
\pgfpathlineto{\pgfqpoint{3.190740in}{2.785428in}}%
\pgfpathmoveto{\pgfqpoint{3.186483in}{2.789686in}}%
\pgfpathlineto{\pgfqpoint{3.186483in}{2.789686in}}%
\pgfpathlineto{\pgfqpoint{3.186483in}{2.793944in}}%
\pgfpathlineto{\pgfqpoint{3.190740in}{2.793944in}}%
\pgfpathlineto{\pgfqpoint{3.190740in}{2.789686in}}%
\pgfpathmoveto{\pgfqpoint{3.190740in}{2.789686in}}%
\pgfpathlineto{\pgfqpoint{3.190740in}{2.789686in}}%
\pgfpathlineto{\pgfqpoint{3.190740in}{2.793944in}}%
\pgfpathlineto{\pgfqpoint{3.194998in}{2.793944in}}%
\pgfpathlineto{\pgfqpoint{3.194998in}{2.789686in}}%
\pgfpathmoveto{\pgfqpoint{3.190740in}{2.793944in}}%
\pgfpathlineto{\pgfqpoint{3.190740in}{2.793944in}}%
\pgfpathlineto{\pgfqpoint{3.190740in}{2.798202in}}%
\pgfpathlineto{\pgfqpoint{3.194998in}{2.798202in}}%
\pgfpathlineto{\pgfqpoint{3.194998in}{2.793944in}}%
\pgfpathmoveto{\pgfqpoint{3.190740in}{2.798202in}}%
\pgfpathlineto{\pgfqpoint{3.190740in}{2.798202in}}%
\pgfpathlineto{\pgfqpoint{3.190740in}{2.802460in}}%
\pgfpathlineto{\pgfqpoint{3.194998in}{2.802460in}}%
\pgfpathlineto{\pgfqpoint{3.194998in}{2.798202in}}%
\pgfpathmoveto{\pgfqpoint{3.190740in}{2.802460in}}%
\pgfpathlineto{\pgfqpoint{3.190740in}{2.802460in}}%
\pgfpathlineto{\pgfqpoint{3.190740in}{2.806717in}}%
\pgfpathlineto{\pgfqpoint{3.194998in}{2.806717in}}%
\pgfpathlineto{\pgfqpoint{3.194998in}{2.802460in}}%
\pgfpathmoveto{\pgfqpoint{3.190740in}{2.806717in}}%
\pgfpathlineto{\pgfqpoint{3.190740in}{2.806717in}}%
\pgfpathlineto{\pgfqpoint{3.190740in}{2.810975in}}%
\pgfpathlineto{\pgfqpoint{3.194998in}{2.810975in}}%
\pgfpathlineto{\pgfqpoint{3.194998in}{2.806717in}}%
\pgfpathmoveto{\pgfqpoint{3.190740in}{2.810975in}}%
\pgfpathlineto{\pgfqpoint{3.190740in}{2.810975in}}%
\pgfpathlineto{\pgfqpoint{3.190740in}{2.815233in}}%
\pgfpathlineto{\pgfqpoint{3.194998in}{2.815233in}}%
\pgfpathlineto{\pgfqpoint{3.194998in}{2.810975in}}%
\pgfpathmoveto{\pgfqpoint{3.194998in}{2.810975in}}%
\pgfpathlineto{\pgfqpoint{3.194998in}{2.810975in}}%
\pgfpathlineto{\pgfqpoint{3.194998in}{2.815233in}}%
\pgfpathlineto{\pgfqpoint{3.199255in}{2.815233in}}%
\pgfpathlineto{\pgfqpoint{3.199255in}{2.810975in}}%
\pgfpathmoveto{\pgfqpoint{3.194998in}{2.815233in}}%
\pgfpathlineto{\pgfqpoint{3.194998in}{2.815233in}}%
\pgfpathlineto{\pgfqpoint{3.194998in}{2.819491in}}%
\pgfpathlineto{\pgfqpoint{3.199255in}{2.819491in}}%
\pgfpathlineto{\pgfqpoint{3.199255in}{2.815233in}}%
\pgfpathmoveto{\pgfqpoint{3.194998in}{2.819491in}}%
\pgfpathlineto{\pgfqpoint{3.194998in}{2.819491in}}%
\pgfpathlineto{\pgfqpoint{3.194998in}{2.823749in}}%
\pgfpathlineto{\pgfqpoint{3.199255in}{2.823749in}}%
\pgfpathlineto{\pgfqpoint{3.199255in}{2.819491in}}%
\pgfpathmoveto{\pgfqpoint{3.194998in}{2.823749in}}%
\pgfpathlineto{\pgfqpoint{3.194998in}{2.823749in}}%
\pgfpathlineto{\pgfqpoint{3.194998in}{2.828006in}}%
\pgfpathlineto{\pgfqpoint{3.199255in}{2.828006in}}%
\pgfpathlineto{\pgfqpoint{3.199255in}{2.823749in}}%
\pgfpathmoveto{\pgfqpoint{3.194998in}{2.828006in}}%
\pgfpathlineto{\pgfqpoint{3.194998in}{2.828006in}}%
\pgfpathlineto{\pgfqpoint{3.194998in}{2.832264in}}%
\pgfpathlineto{\pgfqpoint{3.199255in}{2.832264in}}%
\pgfpathlineto{\pgfqpoint{3.199255in}{2.828006in}}%
\pgfpathmoveto{\pgfqpoint{3.194998in}{2.832264in}}%
\pgfpathlineto{\pgfqpoint{3.194998in}{2.832264in}}%
\pgfpathlineto{\pgfqpoint{3.194998in}{2.836522in}}%
\pgfpathlineto{\pgfqpoint{3.199255in}{2.836522in}}%
\pgfpathlineto{\pgfqpoint{3.199255in}{2.832264in}}%
\pgfpathmoveto{\pgfqpoint{3.199255in}{2.832264in}}%
\pgfpathlineto{\pgfqpoint{3.199255in}{2.832264in}}%
\pgfpathlineto{\pgfqpoint{3.199255in}{2.836522in}}%
\pgfpathlineto{\pgfqpoint{3.203513in}{2.836522in}}%
\pgfpathlineto{\pgfqpoint{3.203513in}{2.832264in}}%
\pgfpathmoveto{\pgfqpoint{3.199255in}{2.836522in}}%
\pgfpathlineto{\pgfqpoint{3.199255in}{2.836522in}}%
\pgfpathlineto{\pgfqpoint{3.199255in}{2.840780in}}%
\pgfpathlineto{\pgfqpoint{3.203513in}{2.840780in}}%
\pgfpathlineto{\pgfqpoint{3.203513in}{2.836522in}}%
\pgfpathmoveto{\pgfqpoint{3.199255in}{2.840780in}}%
\pgfpathlineto{\pgfqpoint{3.199255in}{2.840780in}}%
\pgfpathlineto{\pgfqpoint{3.199255in}{2.845037in}}%
\pgfpathlineto{\pgfqpoint{3.203513in}{2.845037in}}%
\pgfpathlineto{\pgfqpoint{3.203513in}{2.840780in}}%
\pgfpathmoveto{\pgfqpoint{3.199255in}{2.845037in}}%
\pgfpathlineto{\pgfqpoint{3.199255in}{2.845037in}}%
\pgfpathlineto{\pgfqpoint{3.199255in}{2.849295in}}%
\pgfpathlineto{\pgfqpoint{3.203513in}{2.849295in}}%
\pgfpathlineto{\pgfqpoint{3.203513in}{2.845037in}}%
\pgfpathmoveto{\pgfqpoint{3.199255in}{2.849295in}}%
\pgfpathlineto{\pgfqpoint{3.199255in}{2.849295in}}%
\pgfpathlineto{\pgfqpoint{3.199255in}{2.853553in}}%
\pgfpathlineto{\pgfqpoint{3.203513in}{2.853553in}}%
\pgfpathlineto{\pgfqpoint{3.203513in}{2.849295in}}%
\pgfpathmoveto{\pgfqpoint{3.203513in}{2.849295in}}%
\pgfpathlineto{\pgfqpoint{3.203513in}{2.849295in}}%
\pgfpathlineto{\pgfqpoint{3.203513in}{2.853553in}}%
\pgfpathlineto{\pgfqpoint{3.207771in}{2.853553in}}%
\pgfpathlineto{\pgfqpoint{3.207771in}{2.849295in}}%
\pgfpathmoveto{\pgfqpoint{3.203513in}{2.853553in}}%
\pgfpathlineto{\pgfqpoint{3.203513in}{2.853553in}}%
\pgfpathlineto{\pgfqpoint{3.203513in}{2.857811in}}%
\pgfpathlineto{\pgfqpoint{3.207771in}{2.857811in}}%
\pgfpathlineto{\pgfqpoint{3.207771in}{2.853553in}}%
\pgfpathmoveto{\pgfqpoint{3.203513in}{2.857811in}}%
\pgfpathlineto{\pgfqpoint{3.203513in}{2.857811in}}%
\pgfpathlineto{\pgfqpoint{3.203513in}{2.862069in}}%
\pgfpathlineto{\pgfqpoint{3.207771in}{2.862069in}}%
\pgfpathlineto{\pgfqpoint{3.207771in}{2.857811in}}%
\pgfpathmoveto{\pgfqpoint{3.203513in}{2.862069in}}%
\pgfpathlineto{\pgfqpoint{3.203513in}{2.862069in}}%
\pgfpathlineto{\pgfqpoint{3.203513in}{2.866326in}}%
\pgfpathlineto{\pgfqpoint{3.207771in}{2.866326in}}%
\pgfpathlineto{\pgfqpoint{3.207771in}{2.862069in}}%
\pgfpathmoveto{\pgfqpoint{3.203513in}{2.866326in}}%
\pgfpathlineto{\pgfqpoint{3.203513in}{2.866326in}}%
\pgfpathlineto{\pgfqpoint{3.203513in}{2.870584in}}%
\pgfpathlineto{\pgfqpoint{3.207771in}{2.870584in}}%
\pgfpathlineto{\pgfqpoint{3.207771in}{2.866326in}}%
\pgfpathmoveto{\pgfqpoint{3.203513in}{2.870584in}}%
\pgfpathlineto{\pgfqpoint{3.203513in}{2.870584in}}%
\pgfpathlineto{\pgfqpoint{3.203513in}{2.874842in}}%
\pgfpathlineto{\pgfqpoint{3.207771in}{2.874842in}}%
\pgfpathlineto{\pgfqpoint{3.207771in}{2.870584in}}%
\pgfpathmoveto{\pgfqpoint{3.207771in}{2.870584in}}%
\pgfpathlineto{\pgfqpoint{3.207771in}{2.870584in}}%
\pgfpathlineto{\pgfqpoint{3.207771in}{2.874842in}}%
\pgfpathlineto{\pgfqpoint{3.212028in}{2.874842in}}%
\pgfpathlineto{\pgfqpoint{3.212028in}{2.870584in}}%
\pgfpathmoveto{\pgfqpoint{3.207771in}{2.874842in}}%
\pgfpathlineto{\pgfqpoint{3.207771in}{2.874842in}}%
\pgfpathlineto{\pgfqpoint{3.207771in}{2.879100in}}%
\pgfpathlineto{\pgfqpoint{3.212028in}{2.879100in}}%
\pgfpathlineto{\pgfqpoint{3.212028in}{2.874842in}}%
\pgfpathmoveto{\pgfqpoint{3.207771in}{2.879100in}}%
\pgfpathlineto{\pgfqpoint{3.207771in}{2.879100in}}%
\pgfpathlineto{\pgfqpoint{3.207771in}{2.883358in}}%
\pgfpathlineto{\pgfqpoint{3.212028in}{2.883358in}}%
\pgfpathlineto{\pgfqpoint{3.212028in}{2.879100in}}%
\pgfpathmoveto{\pgfqpoint{3.207771in}{2.883358in}}%
\pgfpathlineto{\pgfqpoint{3.207771in}{2.883358in}}%
\pgfpathlineto{\pgfqpoint{3.207771in}{2.887615in}}%
\pgfpathlineto{\pgfqpoint{3.212028in}{2.887615in}}%
\pgfpathlineto{\pgfqpoint{3.212028in}{2.883358in}}%
\pgfpathmoveto{\pgfqpoint{3.207771in}{2.887615in}}%
\pgfpathlineto{\pgfqpoint{3.207771in}{2.887615in}}%
\pgfpathlineto{\pgfqpoint{3.207771in}{2.891873in}}%
\pgfpathlineto{\pgfqpoint{3.212028in}{2.891873in}}%
\pgfpathlineto{\pgfqpoint{3.212028in}{2.887615in}}%
\pgfpathmoveto{\pgfqpoint{3.212028in}{2.887615in}}%
\pgfpathlineto{\pgfqpoint{3.212028in}{2.887615in}}%
\pgfpathlineto{\pgfqpoint{3.212028in}{2.891873in}}%
\pgfpathlineto{\pgfqpoint{3.216286in}{2.891873in}}%
\pgfpathlineto{\pgfqpoint{3.216286in}{2.887615in}}%
\pgfpathmoveto{\pgfqpoint{3.212028in}{2.891873in}}%
\pgfpathlineto{\pgfqpoint{3.212028in}{2.891873in}}%
\pgfpathlineto{\pgfqpoint{3.212028in}{2.896131in}}%
\pgfpathlineto{\pgfqpoint{3.216286in}{2.896131in}}%
\pgfpathlineto{\pgfqpoint{3.216286in}{2.891873in}}%
\pgfpathmoveto{\pgfqpoint{3.212028in}{2.896131in}}%
\pgfpathlineto{\pgfqpoint{3.212028in}{2.896131in}}%
\pgfpathlineto{\pgfqpoint{3.212028in}{2.900389in}}%
\pgfpathlineto{\pgfqpoint{3.216286in}{2.900389in}}%
\pgfpathlineto{\pgfqpoint{3.216286in}{2.896131in}}%
\pgfpathmoveto{\pgfqpoint{3.212028in}{2.900389in}}%
\pgfpathlineto{\pgfqpoint{3.212028in}{2.900389in}}%
\pgfpathlineto{\pgfqpoint{3.212028in}{2.904647in}}%
\pgfpathlineto{\pgfqpoint{3.216286in}{2.904647in}}%
\pgfpathlineto{\pgfqpoint{3.216286in}{2.900389in}}%
\pgfpathmoveto{\pgfqpoint{3.212028in}{2.904647in}}%
\pgfpathlineto{\pgfqpoint{3.212028in}{2.904647in}}%
\pgfpathlineto{\pgfqpoint{3.212028in}{2.908904in}}%
\pgfpathlineto{\pgfqpoint{3.216286in}{2.908904in}}%
\pgfpathlineto{\pgfqpoint{3.216286in}{2.904647in}}%
\pgfpathmoveto{\pgfqpoint{3.216286in}{2.904647in}}%
\pgfpathlineto{\pgfqpoint{3.216286in}{2.904647in}}%
\pgfpathlineto{\pgfqpoint{3.216286in}{2.908904in}}%
\pgfpathlineto{\pgfqpoint{3.220543in}{2.908904in}}%
\pgfpathlineto{\pgfqpoint{3.220543in}{2.904647in}}%
\pgfpathmoveto{\pgfqpoint{3.216286in}{2.908904in}}%
\pgfpathlineto{\pgfqpoint{3.216286in}{2.908904in}}%
\pgfpathlineto{\pgfqpoint{3.216286in}{2.913162in}}%
\pgfpathlineto{\pgfqpoint{3.220543in}{2.913162in}}%
\pgfpathlineto{\pgfqpoint{3.220543in}{2.908904in}}%
\pgfpathmoveto{\pgfqpoint{3.216286in}{2.913162in}}%
\pgfpathlineto{\pgfqpoint{3.216286in}{2.913162in}}%
\pgfpathlineto{\pgfqpoint{3.216286in}{2.917420in}}%
\pgfpathlineto{\pgfqpoint{3.220543in}{2.917420in}}%
\pgfpathlineto{\pgfqpoint{3.220543in}{2.913162in}}%
\pgfpathmoveto{\pgfqpoint{3.216286in}{2.917420in}}%
\pgfpathlineto{\pgfqpoint{3.216286in}{2.917420in}}%
\pgfpathlineto{\pgfqpoint{3.216286in}{2.921678in}}%
\pgfpathlineto{\pgfqpoint{3.220543in}{2.921678in}}%
\pgfpathlineto{\pgfqpoint{3.220543in}{2.917420in}}%
\pgfpathmoveto{\pgfqpoint{3.216286in}{2.921678in}}%
\pgfpathlineto{\pgfqpoint{3.216286in}{2.921678in}}%
\pgfpathlineto{\pgfqpoint{3.216286in}{2.925936in}}%
\pgfpathlineto{\pgfqpoint{3.220543in}{2.925936in}}%
\pgfpathlineto{\pgfqpoint{3.220543in}{2.921678in}}%
\pgfpathmoveto{\pgfqpoint{3.220543in}{2.921678in}}%
\pgfpathlineto{\pgfqpoint{3.220543in}{2.921678in}}%
\pgfpathlineto{\pgfqpoint{3.220543in}{2.925936in}}%
\pgfpathlineto{\pgfqpoint{3.224801in}{2.925936in}}%
\pgfpathlineto{\pgfqpoint{3.224801in}{2.921678in}}%
\pgfpathmoveto{\pgfqpoint{3.220543in}{2.925936in}}%
\pgfpathlineto{\pgfqpoint{3.220543in}{2.925936in}}%
\pgfpathlineto{\pgfqpoint{3.220543in}{2.930193in}}%
\pgfpathlineto{\pgfqpoint{3.224801in}{2.930193in}}%
\pgfpathlineto{\pgfqpoint{3.224801in}{2.925936in}}%
\pgfpathmoveto{\pgfqpoint{3.220543in}{2.930193in}}%
\pgfpathlineto{\pgfqpoint{3.220543in}{2.930193in}}%
\pgfpathlineto{\pgfqpoint{3.220543in}{2.934451in}}%
\pgfpathlineto{\pgfqpoint{3.224801in}{2.934451in}}%
\pgfpathlineto{\pgfqpoint{3.224801in}{2.930193in}}%
\pgfpathmoveto{\pgfqpoint{3.220543in}{2.934451in}}%
\pgfpathlineto{\pgfqpoint{3.220543in}{2.934451in}}%
\pgfpathlineto{\pgfqpoint{3.220543in}{2.938709in}}%
\pgfpathlineto{\pgfqpoint{3.224801in}{2.938709in}}%
\pgfpathlineto{\pgfqpoint{3.224801in}{2.934451in}}%
\pgfpathmoveto{\pgfqpoint{3.220543in}{2.938709in}}%
\pgfpathlineto{\pgfqpoint{3.220543in}{2.938709in}}%
\pgfpathlineto{\pgfqpoint{3.220543in}{2.942967in}}%
\pgfpathlineto{\pgfqpoint{3.224801in}{2.942967in}}%
\pgfpathlineto{\pgfqpoint{3.224801in}{2.938709in}}%
\pgfpathmoveto{\pgfqpoint{3.224801in}{2.938709in}}%
\pgfpathlineto{\pgfqpoint{3.224801in}{2.938709in}}%
\pgfpathlineto{\pgfqpoint{3.224801in}{2.942967in}}%
\pgfpathlineto{\pgfqpoint{3.229058in}{2.942967in}}%
\pgfpathlineto{\pgfqpoint{3.229058in}{2.938709in}}%
\pgfpathmoveto{\pgfqpoint{3.224801in}{2.942967in}}%
\pgfpathlineto{\pgfqpoint{3.224801in}{2.942967in}}%
\pgfpathlineto{\pgfqpoint{3.224801in}{2.947225in}}%
\pgfpathlineto{\pgfqpoint{3.229058in}{2.947225in}}%
\pgfpathlineto{\pgfqpoint{3.229058in}{2.942967in}}%
\pgfpathmoveto{\pgfqpoint{3.224801in}{2.947225in}}%
\pgfpathlineto{\pgfqpoint{3.224801in}{2.947225in}}%
\pgfpathlineto{\pgfqpoint{3.224801in}{2.951482in}}%
\pgfpathlineto{\pgfqpoint{3.229058in}{2.951482in}}%
\pgfpathlineto{\pgfqpoint{3.229058in}{2.947225in}}%
\pgfpathmoveto{\pgfqpoint{3.224801in}{2.951482in}}%
\pgfpathlineto{\pgfqpoint{3.224801in}{2.951482in}}%
\pgfpathlineto{\pgfqpoint{3.224801in}{2.955740in}}%
\pgfpathlineto{\pgfqpoint{3.229058in}{2.955740in}}%
\pgfpathlineto{\pgfqpoint{3.229058in}{2.951482in}}%
\pgfpathmoveto{\pgfqpoint{3.224801in}{2.955740in}}%
\pgfpathlineto{\pgfqpoint{3.224801in}{2.955740in}}%
\pgfpathlineto{\pgfqpoint{3.224801in}{2.959998in}}%
\pgfpathlineto{\pgfqpoint{3.229058in}{2.959998in}}%
\pgfpathlineto{\pgfqpoint{3.229058in}{2.955740in}}%
\pgfpathmoveto{\pgfqpoint{3.229058in}{2.955740in}}%
\pgfpathlineto{\pgfqpoint{3.229058in}{2.955740in}}%
\pgfpathlineto{\pgfqpoint{3.229058in}{2.959998in}}%
\pgfpathlineto{\pgfqpoint{3.233316in}{2.959998in}}%
\pgfpathlineto{\pgfqpoint{3.233316in}{2.955740in}}%
\pgfpathmoveto{\pgfqpoint{3.229058in}{2.959998in}}%
\pgfpathlineto{\pgfqpoint{3.229058in}{2.959998in}}%
\pgfpathlineto{\pgfqpoint{3.229058in}{2.964256in}}%
\pgfpathlineto{\pgfqpoint{3.233316in}{2.964256in}}%
\pgfpathlineto{\pgfqpoint{3.233316in}{2.959998in}}%
\pgfpathmoveto{\pgfqpoint{3.229058in}{2.964256in}}%
\pgfpathlineto{\pgfqpoint{3.229058in}{2.964256in}}%
\pgfpathlineto{\pgfqpoint{3.229058in}{2.968514in}}%
\pgfpathlineto{\pgfqpoint{3.233316in}{2.968514in}}%
\pgfpathlineto{\pgfqpoint{3.233316in}{2.964256in}}%
\pgfpathmoveto{\pgfqpoint{3.229058in}{2.968514in}}%
\pgfpathlineto{\pgfqpoint{3.229058in}{2.968514in}}%
\pgfpathlineto{\pgfqpoint{3.229058in}{2.972772in}}%
\pgfpathlineto{\pgfqpoint{3.233316in}{2.972772in}}%
\pgfpathlineto{\pgfqpoint{3.233316in}{2.968514in}}%
\pgfpathmoveto{\pgfqpoint{3.229058in}{2.972772in}}%
\pgfpathlineto{\pgfqpoint{3.229058in}{2.972772in}}%
\pgfpathlineto{\pgfqpoint{3.229058in}{2.977030in}}%
\pgfpathlineto{\pgfqpoint{3.233316in}{2.977030in}}%
\pgfpathlineto{\pgfqpoint{3.233316in}{2.972772in}}%
\pgfpathmoveto{\pgfqpoint{3.233316in}{2.972772in}}%
\pgfpathlineto{\pgfqpoint{3.233316in}{2.972772in}}%
\pgfpathlineto{\pgfqpoint{3.233316in}{2.977030in}}%
\pgfpathlineto{\pgfqpoint{3.237574in}{2.977030in}}%
\pgfpathlineto{\pgfqpoint{3.237574in}{2.972772in}}%
\pgfpathmoveto{\pgfqpoint{3.233316in}{2.977030in}}%
\pgfpathlineto{\pgfqpoint{3.233316in}{2.977030in}}%
\pgfpathlineto{\pgfqpoint{3.233316in}{2.981288in}}%
\pgfpathlineto{\pgfqpoint{3.237574in}{2.981288in}}%
\pgfpathlineto{\pgfqpoint{3.237574in}{2.977030in}}%
\pgfpathmoveto{\pgfqpoint{3.233316in}{2.981288in}}%
\pgfpathlineto{\pgfqpoint{3.233316in}{2.981288in}}%
\pgfpathlineto{\pgfqpoint{3.233316in}{2.985546in}}%
\pgfpathlineto{\pgfqpoint{3.237574in}{2.985546in}}%
\pgfpathlineto{\pgfqpoint{3.237574in}{2.981288in}}%
\pgfpathmoveto{\pgfqpoint{3.233316in}{2.985546in}}%
\pgfpathlineto{\pgfqpoint{3.233316in}{2.985546in}}%
\pgfpathlineto{\pgfqpoint{3.233316in}{2.989804in}}%
\pgfpathlineto{\pgfqpoint{3.237574in}{2.989804in}}%
\pgfpathlineto{\pgfqpoint{3.237574in}{2.985546in}}%
\pgfpathmoveto{\pgfqpoint{3.233316in}{2.989804in}}%
\pgfpathlineto{\pgfqpoint{3.233316in}{2.989804in}}%
\pgfpathlineto{\pgfqpoint{3.233316in}{2.994062in}}%
\pgfpathlineto{\pgfqpoint{3.237574in}{2.994062in}}%
\pgfpathlineto{\pgfqpoint{3.237574in}{2.989804in}}%
\pgfpathmoveto{\pgfqpoint{3.237574in}{2.989804in}}%
\pgfpathlineto{\pgfqpoint{3.237574in}{2.989804in}}%
\pgfpathlineto{\pgfqpoint{3.237574in}{2.994062in}}%
\pgfpathlineto{\pgfqpoint{3.241832in}{2.994062in}}%
\pgfpathlineto{\pgfqpoint{3.241832in}{2.989804in}}%
\pgfpathmoveto{\pgfqpoint{3.237574in}{2.994062in}}%
\pgfpathlineto{\pgfqpoint{3.237574in}{2.994062in}}%
\pgfpathlineto{\pgfqpoint{3.237574in}{2.998320in}}%
\pgfpathlineto{\pgfqpoint{3.241832in}{2.998320in}}%
\pgfpathlineto{\pgfqpoint{3.241832in}{2.994062in}}%
\pgfpathmoveto{\pgfqpoint{3.237574in}{2.998320in}}%
\pgfpathlineto{\pgfqpoint{3.237574in}{2.998320in}}%
\pgfpathlineto{\pgfqpoint{3.237574in}{3.002578in}}%
\pgfpathlineto{\pgfqpoint{3.241832in}{3.002578in}}%
\pgfpathlineto{\pgfqpoint{3.241832in}{2.998320in}}%
\pgfpathmoveto{\pgfqpoint{3.237574in}{3.002578in}}%
\pgfpathlineto{\pgfqpoint{3.237574in}{3.002578in}}%
\pgfpathlineto{\pgfqpoint{3.237574in}{3.006836in}}%
\pgfpathlineto{\pgfqpoint{3.241832in}{3.006836in}}%
\pgfpathlineto{\pgfqpoint{3.241832in}{3.002578in}}%
\pgfpathmoveto{\pgfqpoint{3.237574in}{3.006836in}}%
\pgfpathlineto{\pgfqpoint{3.237574in}{3.006836in}}%
\pgfpathlineto{\pgfqpoint{3.237574in}{3.011094in}}%
\pgfpathlineto{\pgfqpoint{3.241832in}{3.011094in}}%
\pgfpathlineto{\pgfqpoint{3.241832in}{3.006836in}}%
\pgfpathmoveto{\pgfqpoint{3.241832in}{3.006836in}}%
\pgfpathlineto{\pgfqpoint{3.241832in}{3.006836in}}%
\pgfpathlineto{\pgfqpoint{3.241832in}{3.011094in}}%
\pgfpathlineto{\pgfqpoint{3.246090in}{3.011094in}}%
\pgfpathlineto{\pgfqpoint{3.246090in}{3.006836in}}%
\pgfpathmoveto{\pgfqpoint{3.241832in}{3.011094in}}%
\pgfpathlineto{\pgfqpoint{3.241832in}{3.011094in}}%
\pgfpathlineto{\pgfqpoint{3.241832in}{3.015352in}}%
\pgfpathlineto{\pgfqpoint{3.246090in}{3.015352in}}%
\pgfpathlineto{\pgfqpoint{3.246090in}{3.011094in}}%
\pgfpathmoveto{\pgfqpoint{3.241832in}{3.015352in}}%
\pgfpathlineto{\pgfqpoint{3.241832in}{3.015352in}}%
\pgfpathlineto{\pgfqpoint{3.241832in}{3.019610in}}%
\pgfpathlineto{\pgfqpoint{3.246090in}{3.019610in}}%
\pgfpathlineto{\pgfqpoint{3.246090in}{3.015352in}}%
\pgfpathmoveto{\pgfqpoint{3.241832in}{3.019610in}}%
\pgfpathlineto{\pgfqpoint{3.241832in}{3.019610in}}%
\pgfpathlineto{\pgfqpoint{3.241832in}{3.023868in}}%
\pgfpathlineto{\pgfqpoint{3.246090in}{3.023868in}}%
\pgfpathlineto{\pgfqpoint{3.246090in}{3.019610in}}%
\pgfpathmoveto{\pgfqpoint{3.241832in}{3.023868in}}%
\pgfpathlineto{\pgfqpoint{3.241832in}{3.023868in}}%
\pgfpathlineto{\pgfqpoint{3.241832in}{3.028126in}}%
\pgfpathlineto{\pgfqpoint{3.246090in}{3.028126in}}%
\pgfpathlineto{\pgfqpoint{3.246090in}{3.023868in}}%
\pgfpathmoveto{\pgfqpoint{3.246090in}{3.023868in}}%
\pgfpathlineto{\pgfqpoint{3.246090in}{3.023868in}}%
\pgfpathlineto{\pgfqpoint{3.246090in}{3.028126in}}%
\pgfpathlineto{\pgfqpoint{3.250348in}{3.028126in}}%
\pgfpathlineto{\pgfqpoint{3.250348in}{3.023868in}}%
\pgfpathmoveto{\pgfqpoint{3.246090in}{3.028126in}}%
\pgfpathlineto{\pgfqpoint{3.246090in}{3.028126in}}%
\pgfpathlineto{\pgfqpoint{3.246090in}{3.032384in}}%
\pgfpathlineto{\pgfqpoint{3.250348in}{3.032384in}}%
\pgfpathlineto{\pgfqpoint{3.250348in}{3.028126in}}%
\pgfpathmoveto{\pgfqpoint{3.246090in}{3.032384in}}%
\pgfpathlineto{\pgfqpoint{3.246090in}{3.032384in}}%
\pgfpathlineto{\pgfqpoint{3.246090in}{3.036642in}}%
\pgfpathlineto{\pgfqpoint{3.250348in}{3.036642in}}%
\pgfpathlineto{\pgfqpoint{3.250348in}{3.032384in}}%
\pgfpathmoveto{\pgfqpoint{3.246090in}{3.036642in}}%
\pgfpathlineto{\pgfqpoint{3.246090in}{3.036642in}}%
\pgfpathlineto{\pgfqpoint{3.246090in}{3.040900in}}%
\pgfpathlineto{\pgfqpoint{3.250348in}{3.040900in}}%
\pgfpathlineto{\pgfqpoint{3.250348in}{3.036642in}}%
\pgfpathmoveto{\pgfqpoint{3.246090in}{3.040900in}}%
\pgfpathlineto{\pgfqpoint{3.246090in}{3.040900in}}%
\pgfpathlineto{\pgfqpoint{3.246090in}{3.045158in}}%
\pgfpathlineto{\pgfqpoint{3.250348in}{3.045158in}}%
\pgfpathlineto{\pgfqpoint{3.250348in}{3.040900in}}%
\pgfpathmoveto{\pgfqpoint{3.250348in}{3.036642in}}%
\pgfpathlineto{\pgfqpoint{3.250348in}{3.036642in}}%
\pgfpathlineto{\pgfqpoint{3.250348in}{3.040900in}}%
\pgfpathlineto{\pgfqpoint{3.254606in}{3.040900in}}%
\pgfpathlineto{\pgfqpoint{3.254606in}{3.036642in}}%
\pgfpathmoveto{\pgfqpoint{3.250348in}{3.040900in}}%
\pgfpathlineto{\pgfqpoint{3.250348in}{3.040900in}}%
\pgfpathlineto{\pgfqpoint{3.250348in}{3.045158in}}%
\pgfpathlineto{\pgfqpoint{3.254606in}{3.045158in}}%
\pgfpathlineto{\pgfqpoint{3.254606in}{3.040900in}}%
\pgfpathmoveto{\pgfqpoint{3.250348in}{3.045158in}}%
\pgfpathlineto{\pgfqpoint{3.250348in}{3.045158in}}%
\pgfpathlineto{\pgfqpoint{3.250348in}{3.049416in}}%
\pgfpathlineto{\pgfqpoint{3.254606in}{3.049416in}}%
\pgfpathlineto{\pgfqpoint{3.254606in}{3.045158in}}%
\pgfpathmoveto{\pgfqpoint{3.250348in}{3.049416in}}%
\pgfpathlineto{\pgfqpoint{3.250348in}{3.049416in}}%
\pgfpathlineto{\pgfqpoint{3.250348in}{3.053674in}}%
\pgfpathlineto{\pgfqpoint{3.254606in}{3.053674in}}%
\pgfpathlineto{\pgfqpoint{3.254606in}{3.049416in}}%
\pgfpathmoveto{\pgfqpoint{3.250348in}{3.053674in}}%
\pgfpathlineto{\pgfqpoint{3.250348in}{3.053674in}}%
\pgfpathlineto{\pgfqpoint{3.250348in}{3.057932in}}%
\pgfpathlineto{\pgfqpoint{3.254606in}{3.057932in}}%
\pgfpathlineto{\pgfqpoint{3.254606in}{3.053674in}}%
\pgfpathmoveto{\pgfqpoint{3.250348in}{3.057932in}}%
\pgfpathlineto{\pgfqpoint{3.250348in}{3.057932in}}%
\pgfpathlineto{\pgfqpoint{3.250348in}{3.062190in}}%
\pgfpathlineto{\pgfqpoint{3.254606in}{3.062190in}}%
\pgfpathlineto{\pgfqpoint{3.254606in}{3.057932in}}%
\pgfpathmoveto{\pgfqpoint{3.254606in}{3.053674in}}%
\pgfpathlineto{\pgfqpoint{3.254606in}{3.053674in}}%
\pgfpathlineto{\pgfqpoint{3.254606in}{3.057932in}}%
\pgfpathlineto{\pgfqpoint{3.258864in}{3.057932in}}%
\pgfpathlineto{\pgfqpoint{3.258864in}{3.053674in}}%
\pgfpathmoveto{\pgfqpoint{3.254606in}{3.057932in}}%
\pgfpathlineto{\pgfqpoint{3.254606in}{3.057932in}}%
\pgfpathlineto{\pgfqpoint{3.254606in}{3.062190in}}%
\pgfpathlineto{\pgfqpoint{3.258864in}{3.062190in}}%
\pgfpathlineto{\pgfqpoint{3.258864in}{3.057932in}}%
\pgfpathmoveto{\pgfqpoint{3.254606in}{3.062190in}}%
\pgfpathlineto{\pgfqpoint{3.254606in}{3.062190in}}%
\pgfpathlineto{\pgfqpoint{3.254606in}{3.066448in}}%
\pgfpathlineto{\pgfqpoint{3.258864in}{3.066448in}}%
\pgfpathlineto{\pgfqpoint{3.258864in}{3.062190in}}%
\pgfpathmoveto{\pgfqpoint{3.254606in}{3.066448in}}%
\pgfpathlineto{\pgfqpoint{3.254606in}{3.066448in}}%
\pgfpathlineto{\pgfqpoint{3.254606in}{3.070706in}}%
\pgfpathlineto{\pgfqpoint{3.258864in}{3.070706in}}%
\pgfpathlineto{\pgfqpoint{3.258864in}{3.066448in}}%
\pgfpathmoveto{\pgfqpoint{3.258864in}{3.066448in}}%
\pgfpathlineto{\pgfqpoint{3.258864in}{3.066448in}}%
\pgfpathlineto{\pgfqpoint{3.258864in}{3.070706in}}%
\pgfpathlineto{\pgfqpoint{3.263122in}{3.070706in}}%
\pgfpathlineto{\pgfqpoint{3.263122in}{3.066448in}}%
\pgfpathmoveto{\pgfqpoint{3.254606in}{3.070706in}}%
\pgfpathlineto{\pgfqpoint{3.254606in}{3.070706in}}%
\pgfpathlineto{\pgfqpoint{3.254606in}{3.074964in}}%
\pgfpathlineto{\pgfqpoint{3.258864in}{3.074964in}}%
\pgfpathlineto{\pgfqpoint{3.258864in}{3.070706in}}%
\pgfpathmoveto{\pgfqpoint{3.258864in}{3.070706in}}%
\pgfpathlineto{\pgfqpoint{3.258864in}{3.070706in}}%
\pgfpathlineto{\pgfqpoint{3.258864in}{3.074964in}}%
\pgfpathlineto{\pgfqpoint{3.263122in}{3.074964in}}%
\pgfpathlineto{\pgfqpoint{3.263122in}{3.070706in}}%
\pgfpathmoveto{\pgfqpoint{3.258864in}{3.074964in}}%
\pgfpathlineto{\pgfqpoint{3.258864in}{3.074964in}}%
\pgfpathlineto{\pgfqpoint{3.258864in}{3.079222in}}%
\pgfpathlineto{\pgfqpoint{3.263122in}{3.079222in}}%
\pgfpathlineto{\pgfqpoint{3.263122in}{3.074964in}}%
\pgfpathmoveto{\pgfqpoint{3.258864in}{3.079222in}}%
\pgfpathlineto{\pgfqpoint{3.258864in}{3.079222in}}%
\pgfpathlineto{\pgfqpoint{3.258864in}{3.083480in}}%
\pgfpathlineto{\pgfqpoint{3.263122in}{3.083480in}}%
\pgfpathlineto{\pgfqpoint{3.263122in}{3.079222in}}%
\pgfpathmoveto{\pgfqpoint{3.258864in}{3.083480in}}%
\pgfpathlineto{\pgfqpoint{3.258864in}{3.083480in}}%
\pgfpathlineto{\pgfqpoint{3.258864in}{3.087738in}}%
\pgfpathlineto{\pgfqpoint{3.263122in}{3.087738in}}%
\pgfpathlineto{\pgfqpoint{3.263122in}{3.083480in}}%
\pgfpathmoveto{\pgfqpoint{3.258864in}{3.087738in}}%
\pgfpathlineto{\pgfqpoint{3.258864in}{3.087738in}}%
\pgfpathlineto{\pgfqpoint{3.258864in}{3.091996in}}%
\pgfpathlineto{\pgfqpoint{3.263122in}{3.091996in}}%
\pgfpathlineto{\pgfqpoint{3.263122in}{3.087738in}}%
\pgfpathmoveto{\pgfqpoint{3.263122in}{3.079222in}}%
\pgfpathlineto{\pgfqpoint{3.263122in}{3.079222in}}%
\pgfpathlineto{\pgfqpoint{3.263122in}{3.083480in}}%
\pgfpathlineto{\pgfqpoint{3.267380in}{3.083480in}}%
\pgfpathlineto{\pgfqpoint{3.267380in}{3.079222in}}%
\pgfpathmoveto{\pgfqpoint{3.263122in}{3.083480in}}%
\pgfpathlineto{\pgfqpoint{3.263122in}{3.083480in}}%
\pgfpathlineto{\pgfqpoint{3.263122in}{3.087738in}}%
\pgfpathlineto{\pgfqpoint{3.267380in}{3.087738in}}%
\pgfpathlineto{\pgfqpoint{3.267380in}{3.083480in}}%
\pgfpathmoveto{\pgfqpoint{3.263122in}{3.087738in}}%
\pgfpathlineto{\pgfqpoint{3.263122in}{3.087738in}}%
\pgfpathlineto{\pgfqpoint{3.263122in}{3.091996in}}%
\pgfpathlineto{\pgfqpoint{3.267380in}{3.091996in}}%
\pgfpathlineto{\pgfqpoint{3.267380in}{3.087738in}}%
\pgfpathmoveto{\pgfqpoint{3.263122in}{3.091996in}}%
\pgfpathlineto{\pgfqpoint{3.263122in}{3.091996in}}%
\pgfpathlineto{\pgfqpoint{3.263122in}{3.096254in}}%
\pgfpathlineto{\pgfqpoint{3.267380in}{3.096254in}}%
\pgfpathlineto{\pgfqpoint{3.267380in}{3.091996in}}%
\pgfpathmoveto{\pgfqpoint{3.263122in}{3.096254in}}%
\pgfpathlineto{\pgfqpoint{3.263122in}{3.096254in}}%
\pgfpathlineto{\pgfqpoint{3.263122in}{3.100512in}}%
\pgfpathlineto{\pgfqpoint{3.267380in}{3.100512in}}%
\pgfpathlineto{\pgfqpoint{3.267380in}{3.096254in}}%
\pgfpathmoveto{\pgfqpoint{3.263122in}{3.100512in}}%
\pgfpathlineto{\pgfqpoint{3.263122in}{3.100512in}}%
\pgfpathlineto{\pgfqpoint{3.263122in}{3.104769in}}%
\pgfpathlineto{\pgfqpoint{3.267380in}{3.104769in}}%
\pgfpathlineto{\pgfqpoint{3.267380in}{3.100512in}}%
\pgfpathmoveto{\pgfqpoint{3.267380in}{3.096254in}}%
\pgfpathlineto{\pgfqpoint{3.267380in}{3.096254in}}%
\pgfpathlineto{\pgfqpoint{3.267380in}{3.100512in}}%
\pgfpathlineto{\pgfqpoint{3.271638in}{3.100512in}}%
\pgfpathlineto{\pgfqpoint{3.271638in}{3.096254in}}%
\pgfpathmoveto{\pgfqpoint{3.267380in}{3.100512in}}%
\pgfpathlineto{\pgfqpoint{3.267380in}{3.100512in}}%
\pgfpathlineto{\pgfqpoint{3.267380in}{3.104769in}}%
\pgfpathlineto{\pgfqpoint{3.271638in}{3.104769in}}%
\pgfpathlineto{\pgfqpoint{3.271638in}{3.100512in}}%
\pgfpathmoveto{\pgfqpoint{3.267380in}{3.104769in}}%
\pgfpathlineto{\pgfqpoint{3.267380in}{3.104769in}}%
\pgfpathlineto{\pgfqpoint{3.267380in}{3.109027in}}%
\pgfpathlineto{\pgfqpoint{3.271638in}{3.109027in}}%
\pgfpathlineto{\pgfqpoint{3.271638in}{3.104769in}}%
\pgfpathmoveto{\pgfqpoint{3.267380in}{3.109027in}}%
\pgfpathlineto{\pgfqpoint{3.267380in}{3.109027in}}%
\pgfpathlineto{\pgfqpoint{3.267380in}{3.113285in}}%
\pgfpathlineto{\pgfqpoint{3.271638in}{3.113285in}}%
\pgfpathlineto{\pgfqpoint{3.271638in}{3.109027in}}%
\pgfpathmoveto{\pgfqpoint{3.271638in}{3.109027in}}%
\pgfpathlineto{\pgfqpoint{3.271638in}{3.109027in}}%
\pgfpathlineto{\pgfqpoint{3.271638in}{3.113285in}}%
\pgfpathlineto{\pgfqpoint{3.275896in}{3.113285in}}%
\pgfpathlineto{\pgfqpoint{3.275896in}{3.109027in}}%
\pgfpathmoveto{\pgfqpoint{3.267380in}{3.113285in}}%
\pgfpathlineto{\pgfqpoint{3.267380in}{3.113285in}}%
\pgfpathlineto{\pgfqpoint{3.267380in}{3.117542in}}%
\pgfpathlineto{\pgfqpoint{3.271638in}{3.117542in}}%
\pgfpathlineto{\pgfqpoint{3.271638in}{3.113285in}}%
\pgfpathmoveto{\pgfqpoint{3.271638in}{3.113285in}}%
\pgfpathlineto{\pgfqpoint{3.271638in}{3.113285in}}%
\pgfpathlineto{\pgfqpoint{3.271638in}{3.117542in}}%
\pgfpathlineto{\pgfqpoint{3.275896in}{3.117542in}}%
\pgfpathlineto{\pgfqpoint{3.275896in}{3.113285in}}%
\pgfpathmoveto{\pgfqpoint{3.271638in}{3.117542in}}%
\pgfpathlineto{\pgfqpoint{3.271638in}{3.117542in}}%
\pgfpathlineto{\pgfqpoint{3.271638in}{3.121800in}}%
\pgfpathlineto{\pgfqpoint{3.275896in}{3.121800in}}%
\pgfpathlineto{\pgfqpoint{3.275896in}{3.117542in}}%
\pgfpathmoveto{\pgfqpoint{3.271638in}{3.121800in}}%
\pgfpathlineto{\pgfqpoint{3.271638in}{3.121800in}}%
\pgfpathlineto{\pgfqpoint{3.271638in}{3.126058in}}%
\pgfpathlineto{\pgfqpoint{3.275896in}{3.126058in}}%
\pgfpathlineto{\pgfqpoint{3.275896in}{3.121800in}}%
\pgfpathmoveto{\pgfqpoint{3.271638in}{3.126058in}}%
\pgfpathlineto{\pgfqpoint{3.271638in}{3.126058in}}%
\pgfpathlineto{\pgfqpoint{3.271638in}{3.130315in}}%
\pgfpathlineto{\pgfqpoint{3.275896in}{3.130315in}}%
\pgfpathlineto{\pgfqpoint{3.275896in}{3.126058in}}%
\pgfpathmoveto{\pgfqpoint{3.275896in}{3.121800in}}%
\pgfpathlineto{\pgfqpoint{3.275896in}{3.121800in}}%
\pgfpathlineto{\pgfqpoint{3.275896in}{3.126058in}}%
\pgfpathlineto{\pgfqpoint{3.280154in}{3.126058in}}%
\pgfpathlineto{\pgfqpoint{3.280154in}{3.121800in}}%
\pgfpathmoveto{\pgfqpoint{3.275896in}{3.126058in}}%
\pgfpathlineto{\pgfqpoint{3.275896in}{3.126058in}}%
\pgfpathlineto{\pgfqpoint{3.275896in}{3.130315in}}%
\pgfpathlineto{\pgfqpoint{3.280154in}{3.130315in}}%
\pgfpathlineto{\pgfqpoint{3.280154in}{3.126058in}}%
\pgfpathmoveto{\pgfqpoint{3.271638in}{3.130315in}}%
\pgfpathlineto{\pgfqpoint{3.271638in}{3.130315in}}%
\pgfpathlineto{\pgfqpoint{3.271638in}{3.134573in}}%
\pgfpathlineto{\pgfqpoint{3.275896in}{3.134573in}}%
\pgfpathlineto{\pgfqpoint{3.275896in}{3.130315in}}%
\pgfpathmoveto{\pgfqpoint{3.275896in}{3.130315in}}%
\pgfpathlineto{\pgfqpoint{3.275896in}{3.130315in}}%
\pgfpathlineto{\pgfqpoint{3.275896in}{3.134573in}}%
\pgfpathlineto{\pgfqpoint{3.280154in}{3.134573in}}%
\pgfpathlineto{\pgfqpoint{3.280154in}{3.130315in}}%
\pgfpathmoveto{\pgfqpoint{3.275896in}{3.134573in}}%
\pgfpathlineto{\pgfqpoint{3.275896in}{3.134573in}}%
\pgfpathlineto{\pgfqpoint{3.275896in}{3.138831in}}%
\pgfpathlineto{\pgfqpoint{3.280154in}{3.138831in}}%
\pgfpathlineto{\pgfqpoint{3.280154in}{3.134573in}}%
\pgfpathmoveto{\pgfqpoint{3.275896in}{3.138831in}}%
\pgfpathlineto{\pgfqpoint{3.275896in}{3.138831in}}%
\pgfpathlineto{\pgfqpoint{3.275896in}{3.143089in}}%
\pgfpathlineto{\pgfqpoint{3.280154in}{3.143089in}}%
\pgfpathlineto{\pgfqpoint{3.280154in}{3.138831in}}%
\pgfpathmoveto{\pgfqpoint{3.275896in}{3.143089in}}%
\pgfpathlineto{\pgfqpoint{3.275896in}{3.143089in}}%
\pgfpathlineto{\pgfqpoint{3.275896in}{3.147346in}}%
\pgfpathlineto{\pgfqpoint{3.280154in}{3.147346in}}%
\pgfpathlineto{\pgfqpoint{3.280154in}{3.143089in}}%
\pgfpathmoveto{\pgfqpoint{3.280154in}{3.134573in}}%
\pgfpathlineto{\pgfqpoint{3.280154in}{3.134573in}}%
\pgfpathlineto{\pgfqpoint{3.280154in}{3.138831in}}%
\pgfpathlineto{\pgfqpoint{3.284412in}{3.138831in}}%
\pgfpathlineto{\pgfqpoint{3.284412in}{3.134573in}}%
\pgfpathmoveto{\pgfqpoint{3.280154in}{3.138831in}}%
\pgfpathlineto{\pgfqpoint{3.280154in}{3.138831in}}%
\pgfpathlineto{\pgfqpoint{3.280154in}{3.143089in}}%
\pgfpathlineto{\pgfqpoint{3.284412in}{3.143089in}}%
\pgfpathlineto{\pgfqpoint{3.284412in}{3.138831in}}%
\pgfpathmoveto{\pgfqpoint{3.280154in}{3.143089in}}%
\pgfpathlineto{\pgfqpoint{3.280154in}{3.143089in}}%
\pgfpathlineto{\pgfqpoint{3.280154in}{3.147346in}}%
\pgfpathlineto{\pgfqpoint{3.284412in}{3.147346in}}%
\pgfpathlineto{\pgfqpoint{3.284412in}{3.143089in}}%
\pgfpathmoveto{\pgfqpoint{3.280154in}{3.147346in}}%
\pgfpathlineto{\pgfqpoint{3.280154in}{3.147346in}}%
\pgfpathlineto{\pgfqpoint{3.280154in}{3.151604in}}%
\pgfpathlineto{\pgfqpoint{3.284412in}{3.151604in}}%
\pgfpathlineto{\pgfqpoint{3.284412in}{3.147346in}}%
\pgfpathmoveto{\pgfqpoint{3.280154in}{3.151604in}}%
\pgfpathlineto{\pgfqpoint{3.280154in}{3.151604in}}%
\pgfpathlineto{\pgfqpoint{3.280154in}{3.155862in}}%
\pgfpathlineto{\pgfqpoint{3.284412in}{3.155862in}}%
\pgfpathlineto{\pgfqpoint{3.284412in}{3.151604in}}%
\pgfpathmoveto{\pgfqpoint{3.284412in}{3.147346in}}%
\pgfpathlineto{\pgfqpoint{3.284412in}{3.147346in}}%
\pgfpathlineto{\pgfqpoint{3.284412in}{3.151604in}}%
\pgfpathlineto{\pgfqpoint{3.288670in}{3.151604in}}%
\pgfpathlineto{\pgfqpoint{3.288670in}{3.147346in}}%
\pgfpathmoveto{\pgfqpoint{3.284412in}{3.151604in}}%
\pgfpathlineto{\pgfqpoint{3.284412in}{3.151604in}}%
\pgfpathlineto{\pgfqpoint{3.284412in}{3.155862in}}%
\pgfpathlineto{\pgfqpoint{3.288670in}{3.155862in}}%
\pgfpathlineto{\pgfqpoint{3.288670in}{3.151604in}}%
\pgfpathmoveto{\pgfqpoint{3.280154in}{3.155862in}}%
\pgfpathlineto{\pgfqpoint{3.280154in}{3.155862in}}%
\pgfpathlineto{\pgfqpoint{3.280154in}{3.160119in}}%
\pgfpathlineto{\pgfqpoint{3.284412in}{3.160119in}}%
\pgfpathlineto{\pgfqpoint{3.284412in}{3.155862in}}%
\pgfpathmoveto{\pgfqpoint{3.284412in}{3.155862in}}%
\pgfpathlineto{\pgfqpoint{3.284412in}{3.155862in}}%
\pgfpathlineto{\pgfqpoint{3.284412in}{3.160119in}}%
\pgfpathlineto{\pgfqpoint{3.288670in}{3.160119in}}%
\pgfpathlineto{\pgfqpoint{3.288670in}{3.155862in}}%
\pgfpathmoveto{\pgfqpoint{3.284412in}{3.160119in}}%
\pgfpathlineto{\pgfqpoint{3.284412in}{3.160119in}}%
\pgfpathlineto{\pgfqpoint{3.284412in}{3.164377in}}%
\pgfpathlineto{\pgfqpoint{3.288670in}{3.164377in}}%
\pgfpathlineto{\pgfqpoint{3.288670in}{3.160119in}}%
\pgfpathmoveto{\pgfqpoint{3.288670in}{3.160119in}}%
\pgfpathlineto{\pgfqpoint{3.288670in}{3.160119in}}%
\pgfpathlineto{\pgfqpoint{3.288670in}{3.164377in}}%
\pgfpathlineto{\pgfqpoint{3.292928in}{3.164377in}}%
\pgfpathlineto{\pgfqpoint{3.292928in}{3.160119in}}%
\pgfpathmoveto{\pgfqpoint{3.284412in}{3.164377in}}%
\pgfpathlineto{\pgfqpoint{3.284412in}{3.164377in}}%
\pgfpathlineto{\pgfqpoint{3.284412in}{3.168635in}}%
\pgfpathlineto{\pgfqpoint{3.288670in}{3.168635in}}%
\pgfpathlineto{\pgfqpoint{3.288670in}{3.164377in}}%
\pgfpathmoveto{\pgfqpoint{3.284412in}{3.168635in}}%
\pgfpathlineto{\pgfqpoint{3.284412in}{3.168635in}}%
\pgfpathlineto{\pgfqpoint{3.284412in}{3.172892in}}%
\pgfpathlineto{\pgfqpoint{3.288670in}{3.172892in}}%
\pgfpathlineto{\pgfqpoint{3.288670in}{3.168635in}}%
\pgfpathmoveto{\pgfqpoint{3.288670in}{3.164377in}}%
\pgfpathlineto{\pgfqpoint{3.288670in}{3.164377in}}%
\pgfpathlineto{\pgfqpoint{3.288670in}{3.168635in}}%
\pgfpathlineto{\pgfqpoint{3.292928in}{3.168635in}}%
\pgfpathlineto{\pgfqpoint{3.292928in}{3.164377in}}%
\pgfpathmoveto{\pgfqpoint{3.288670in}{3.168635in}}%
\pgfpathlineto{\pgfqpoint{3.288670in}{3.168635in}}%
\pgfpathlineto{\pgfqpoint{3.288670in}{3.172892in}}%
\pgfpathlineto{\pgfqpoint{3.292928in}{3.172892in}}%
\pgfpathlineto{\pgfqpoint{3.292928in}{3.168635in}}%
\pgfpathmoveto{\pgfqpoint{3.292928in}{3.168635in}}%
\pgfpathlineto{\pgfqpoint{3.292928in}{3.168635in}}%
\pgfpathlineto{\pgfqpoint{3.292928in}{3.172892in}}%
\pgfpathlineto{\pgfqpoint{3.297186in}{3.172892in}}%
\pgfpathlineto{\pgfqpoint{3.297186in}{3.168635in}}%
\pgfpathmoveto{\pgfqpoint{3.288670in}{3.172892in}}%
\pgfpathlineto{\pgfqpoint{3.288670in}{3.172892in}}%
\pgfpathlineto{\pgfqpoint{3.288670in}{3.177150in}}%
\pgfpathlineto{\pgfqpoint{3.292928in}{3.177150in}}%
\pgfpathlineto{\pgfqpoint{3.292928in}{3.172892in}}%
\pgfpathmoveto{\pgfqpoint{3.288670in}{3.177150in}}%
\pgfpathlineto{\pgfqpoint{3.288670in}{3.177150in}}%
\pgfpathlineto{\pgfqpoint{3.288670in}{3.181408in}}%
\pgfpathlineto{\pgfqpoint{3.292928in}{3.181408in}}%
\pgfpathlineto{\pgfqpoint{3.292928in}{3.177150in}}%
\pgfpathmoveto{\pgfqpoint{3.292928in}{3.172892in}}%
\pgfpathlineto{\pgfqpoint{3.292928in}{3.172892in}}%
\pgfpathlineto{\pgfqpoint{3.292928in}{3.177150in}}%
\pgfpathlineto{\pgfqpoint{3.297186in}{3.177150in}}%
\pgfpathlineto{\pgfqpoint{3.297186in}{3.172892in}}%
\pgfpathmoveto{\pgfqpoint{3.292928in}{3.177150in}}%
\pgfpathlineto{\pgfqpoint{3.292928in}{3.177150in}}%
\pgfpathlineto{\pgfqpoint{3.292928in}{3.181408in}}%
\pgfpathlineto{\pgfqpoint{3.297186in}{3.181408in}}%
\pgfpathlineto{\pgfqpoint{3.297186in}{3.177150in}}%
\pgfpathmoveto{\pgfqpoint{3.288670in}{3.181408in}}%
\pgfpathlineto{\pgfqpoint{3.288670in}{3.181408in}}%
\pgfpathlineto{\pgfqpoint{3.288670in}{3.185666in}}%
\pgfpathlineto{\pgfqpoint{3.292928in}{3.185666in}}%
\pgfpathlineto{\pgfqpoint{3.292928in}{3.181408in}}%
\pgfpathmoveto{\pgfqpoint{3.292928in}{3.181408in}}%
\pgfpathlineto{\pgfqpoint{3.292928in}{3.181408in}}%
\pgfpathlineto{\pgfqpoint{3.292928in}{3.185666in}}%
\pgfpathlineto{\pgfqpoint{3.297186in}{3.185666in}}%
\pgfpathlineto{\pgfqpoint{3.297186in}{3.181408in}}%
\pgfpathmoveto{\pgfqpoint{3.292928in}{3.185666in}}%
\pgfpathlineto{\pgfqpoint{3.292928in}{3.185666in}}%
\pgfpathlineto{\pgfqpoint{3.292928in}{3.189923in}}%
\pgfpathlineto{\pgfqpoint{3.297186in}{3.189923in}}%
\pgfpathlineto{\pgfqpoint{3.297186in}{3.185666in}}%
\pgfpathmoveto{\pgfqpoint{3.292928in}{3.189923in}}%
\pgfpathlineto{\pgfqpoint{3.292928in}{3.189923in}}%
\pgfpathlineto{\pgfqpoint{3.292928in}{3.194181in}}%
\pgfpathlineto{\pgfqpoint{3.297186in}{3.194181in}}%
\pgfpathlineto{\pgfqpoint{3.297186in}{3.189923in}}%
\pgfpathmoveto{\pgfqpoint{3.292928in}{3.194181in}}%
\pgfpathlineto{\pgfqpoint{3.292928in}{3.194181in}}%
\pgfpathlineto{\pgfqpoint{3.292928in}{3.198439in}}%
\pgfpathlineto{\pgfqpoint{3.297186in}{3.198439in}}%
\pgfpathlineto{\pgfqpoint{3.297186in}{3.194181in}}%
\pgfpathmoveto{\pgfqpoint{3.297186in}{3.181408in}}%
\pgfpathlineto{\pgfqpoint{3.297186in}{3.181408in}}%
\pgfpathlineto{\pgfqpoint{3.297186in}{3.185666in}}%
\pgfpathlineto{\pgfqpoint{3.301444in}{3.185666in}}%
\pgfpathlineto{\pgfqpoint{3.301444in}{3.181408in}}%
\pgfpathmoveto{\pgfqpoint{3.297186in}{3.185666in}}%
\pgfpathlineto{\pgfqpoint{3.297186in}{3.185666in}}%
\pgfpathlineto{\pgfqpoint{3.297186in}{3.189923in}}%
\pgfpathlineto{\pgfqpoint{3.301444in}{3.189923in}}%
\pgfpathlineto{\pgfqpoint{3.301444in}{3.185666in}}%
\pgfpathmoveto{\pgfqpoint{3.297186in}{3.189923in}}%
\pgfpathlineto{\pgfqpoint{3.297186in}{3.189923in}}%
\pgfpathlineto{\pgfqpoint{3.297186in}{3.194181in}}%
\pgfpathlineto{\pgfqpoint{3.301444in}{3.194181in}}%
\pgfpathlineto{\pgfqpoint{3.301444in}{3.189923in}}%
\pgfpathmoveto{\pgfqpoint{3.297186in}{3.194181in}}%
\pgfpathlineto{\pgfqpoint{3.297186in}{3.194181in}}%
\pgfpathlineto{\pgfqpoint{3.297186in}{3.198439in}}%
\pgfpathlineto{\pgfqpoint{3.301444in}{3.198439in}}%
\pgfpathlineto{\pgfqpoint{3.301444in}{3.194181in}}%
\pgfpathmoveto{\pgfqpoint{3.301444in}{3.194181in}}%
\pgfpathlineto{\pgfqpoint{3.301444in}{3.194181in}}%
\pgfpathlineto{\pgfqpoint{3.301444in}{3.198439in}}%
\pgfpathlineto{\pgfqpoint{3.305702in}{3.198439in}}%
\pgfpathlineto{\pgfqpoint{3.305702in}{3.194181in}}%
\pgfpathmoveto{\pgfqpoint{3.297186in}{3.198439in}}%
\pgfpathlineto{\pgfqpoint{3.297186in}{3.198439in}}%
\pgfpathlineto{\pgfqpoint{3.297186in}{3.202696in}}%
\pgfpathlineto{\pgfqpoint{3.301444in}{3.202696in}}%
\pgfpathlineto{\pgfqpoint{3.301444in}{3.198439in}}%
\pgfpathmoveto{\pgfqpoint{3.297186in}{3.202696in}}%
\pgfpathlineto{\pgfqpoint{3.297186in}{3.202696in}}%
\pgfpathlineto{\pgfqpoint{3.297186in}{3.206954in}}%
\pgfpathlineto{\pgfqpoint{3.301444in}{3.206954in}}%
\pgfpathlineto{\pgfqpoint{3.301444in}{3.202696in}}%
\pgfpathmoveto{\pgfqpoint{3.301444in}{3.198439in}}%
\pgfpathlineto{\pgfqpoint{3.301444in}{3.198439in}}%
\pgfpathlineto{\pgfqpoint{3.301444in}{3.202696in}}%
\pgfpathlineto{\pgfqpoint{3.305702in}{3.202696in}}%
\pgfpathlineto{\pgfqpoint{3.305702in}{3.198439in}}%
\pgfpathmoveto{\pgfqpoint{3.301444in}{3.202696in}}%
\pgfpathlineto{\pgfqpoint{3.301444in}{3.202696in}}%
\pgfpathlineto{\pgfqpoint{3.301444in}{3.206954in}}%
\pgfpathlineto{\pgfqpoint{3.305702in}{3.206954in}}%
\pgfpathlineto{\pgfqpoint{3.305702in}{3.202696in}}%
\pgfpathmoveto{\pgfqpoint{3.301444in}{3.206954in}}%
\pgfpathlineto{\pgfqpoint{3.301444in}{3.206954in}}%
\pgfpathlineto{\pgfqpoint{3.301444in}{3.211212in}}%
\pgfpathlineto{\pgfqpoint{3.305702in}{3.211212in}}%
\pgfpathlineto{\pgfqpoint{3.305702in}{3.206954in}}%
\pgfpathmoveto{\pgfqpoint{3.301444in}{3.211212in}}%
\pgfpathlineto{\pgfqpoint{3.301444in}{3.211212in}}%
\pgfpathlineto{\pgfqpoint{3.301444in}{3.215469in}}%
\pgfpathlineto{\pgfqpoint{3.305702in}{3.215469in}}%
\pgfpathlineto{\pgfqpoint{3.305702in}{3.211212in}}%
\pgfpathmoveto{\pgfqpoint{3.305702in}{3.202696in}}%
\pgfpathlineto{\pgfqpoint{3.305702in}{3.202696in}}%
\pgfpathlineto{\pgfqpoint{3.305702in}{3.206954in}}%
\pgfpathlineto{\pgfqpoint{3.309960in}{3.206954in}}%
\pgfpathlineto{\pgfqpoint{3.309960in}{3.202696in}}%
\pgfpathmoveto{\pgfqpoint{3.305702in}{3.206954in}}%
\pgfpathlineto{\pgfqpoint{3.305702in}{3.206954in}}%
\pgfpathlineto{\pgfqpoint{3.305702in}{3.211212in}}%
\pgfpathlineto{\pgfqpoint{3.309960in}{3.211212in}}%
\pgfpathlineto{\pgfqpoint{3.309960in}{3.206954in}}%
\pgfpathmoveto{\pgfqpoint{3.305702in}{3.211212in}}%
\pgfpathlineto{\pgfqpoint{3.305702in}{3.211212in}}%
\pgfpathlineto{\pgfqpoint{3.305702in}{3.215469in}}%
\pgfpathlineto{\pgfqpoint{3.309960in}{3.215469in}}%
\pgfpathlineto{\pgfqpoint{3.309960in}{3.211212in}}%
\pgfpathmoveto{\pgfqpoint{3.301444in}{3.215469in}}%
\pgfpathlineto{\pgfqpoint{3.301444in}{3.215469in}}%
\pgfpathlineto{\pgfqpoint{3.301444in}{3.219727in}}%
\pgfpathlineto{\pgfqpoint{3.305702in}{3.219727in}}%
\pgfpathlineto{\pgfqpoint{3.305702in}{3.215469in}}%
\pgfpathmoveto{\pgfqpoint{3.305702in}{3.215469in}}%
\pgfpathlineto{\pgfqpoint{3.305702in}{3.215469in}}%
\pgfpathlineto{\pgfqpoint{3.305702in}{3.219727in}}%
\pgfpathlineto{\pgfqpoint{3.309960in}{3.219727in}}%
\pgfpathlineto{\pgfqpoint{3.309960in}{3.215469in}}%
\pgfpathmoveto{\pgfqpoint{3.305702in}{3.219727in}}%
\pgfpathlineto{\pgfqpoint{3.305702in}{3.219727in}}%
\pgfpathlineto{\pgfqpoint{3.305702in}{3.223985in}}%
\pgfpathlineto{\pgfqpoint{3.309960in}{3.223985in}}%
\pgfpathlineto{\pgfqpoint{3.309960in}{3.219727in}}%
\pgfpathmoveto{\pgfqpoint{3.309960in}{3.215469in}}%
\pgfpathlineto{\pgfqpoint{3.309960in}{3.215469in}}%
\pgfpathlineto{\pgfqpoint{3.309960in}{3.219727in}}%
\pgfpathlineto{\pgfqpoint{3.314218in}{3.219727in}}%
\pgfpathlineto{\pgfqpoint{3.314218in}{3.215469in}}%
\pgfpathmoveto{\pgfqpoint{3.309960in}{3.219727in}}%
\pgfpathlineto{\pgfqpoint{3.309960in}{3.219727in}}%
\pgfpathlineto{\pgfqpoint{3.309960in}{3.223985in}}%
\pgfpathlineto{\pgfqpoint{3.314218in}{3.223985in}}%
\pgfpathlineto{\pgfqpoint{3.314218in}{3.219727in}}%
\pgfpathmoveto{\pgfqpoint{3.305702in}{3.223985in}}%
\pgfpathlineto{\pgfqpoint{3.305702in}{3.223985in}}%
\pgfpathlineto{\pgfqpoint{3.305702in}{3.228243in}}%
\pgfpathlineto{\pgfqpoint{3.309960in}{3.228243in}}%
\pgfpathlineto{\pgfqpoint{3.309960in}{3.223985in}}%
\pgfpathmoveto{\pgfqpoint{3.305702in}{3.228243in}}%
\pgfpathlineto{\pgfqpoint{3.305702in}{3.228243in}}%
\pgfpathlineto{\pgfqpoint{3.305702in}{3.232500in}}%
\pgfpathlineto{\pgfqpoint{3.309960in}{3.232500in}}%
\pgfpathlineto{\pgfqpoint{3.309960in}{3.228243in}}%
\pgfpathmoveto{\pgfqpoint{3.309960in}{3.223985in}}%
\pgfpathlineto{\pgfqpoint{3.309960in}{3.223985in}}%
\pgfpathlineto{\pgfqpoint{3.309960in}{3.228243in}}%
\pgfpathlineto{\pgfqpoint{3.314218in}{3.228243in}}%
\pgfpathlineto{\pgfqpoint{3.314218in}{3.223985in}}%
\pgfpathmoveto{\pgfqpoint{3.309960in}{3.228243in}}%
\pgfpathlineto{\pgfqpoint{3.309960in}{3.228243in}}%
\pgfpathlineto{\pgfqpoint{3.309960in}{3.232500in}}%
\pgfpathlineto{\pgfqpoint{3.314218in}{3.232500in}}%
\pgfpathlineto{\pgfqpoint{3.314218in}{3.228243in}}%
\pgfpathmoveto{\pgfqpoint{3.314218in}{3.223985in}}%
\pgfpathlineto{\pgfqpoint{3.314218in}{3.223985in}}%
\pgfpathlineto{\pgfqpoint{3.314218in}{3.228243in}}%
\pgfpathlineto{\pgfqpoint{3.318476in}{3.228243in}}%
\pgfpathlineto{\pgfqpoint{3.318476in}{3.223985in}}%
\pgfpathmoveto{\pgfqpoint{3.314218in}{3.228243in}}%
\pgfpathlineto{\pgfqpoint{3.314218in}{3.228243in}}%
\pgfpathlineto{\pgfqpoint{3.314218in}{3.232500in}}%
\pgfpathlineto{\pgfqpoint{3.318476in}{3.232500in}}%
\pgfpathlineto{\pgfqpoint{3.318476in}{3.228243in}}%
\pgfpathmoveto{\pgfqpoint{3.309960in}{3.232500in}}%
\pgfpathlineto{\pgfqpoint{3.309960in}{3.232500in}}%
\pgfpathlineto{\pgfqpoint{3.309960in}{3.236758in}}%
\pgfpathlineto{\pgfqpoint{3.314218in}{3.236758in}}%
\pgfpathlineto{\pgfqpoint{3.314218in}{3.232500in}}%
\pgfpathmoveto{\pgfqpoint{3.309960in}{3.236758in}}%
\pgfpathlineto{\pgfqpoint{3.309960in}{3.236758in}}%
\pgfpathlineto{\pgfqpoint{3.309960in}{3.241016in}}%
\pgfpathlineto{\pgfqpoint{3.314218in}{3.241016in}}%
\pgfpathlineto{\pgfqpoint{3.314218in}{3.236758in}}%
\pgfpathmoveto{\pgfqpoint{3.314218in}{3.232500in}}%
\pgfpathlineto{\pgfqpoint{3.314218in}{3.232500in}}%
\pgfpathlineto{\pgfqpoint{3.314218in}{3.236758in}}%
\pgfpathlineto{\pgfqpoint{3.318476in}{3.236758in}}%
\pgfpathlineto{\pgfqpoint{3.318476in}{3.232500in}}%
\pgfpathmoveto{\pgfqpoint{3.314218in}{3.236758in}}%
\pgfpathlineto{\pgfqpoint{3.314218in}{3.236758in}}%
\pgfpathlineto{\pgfqpoint{3.314218in}{3.241016in}}%
\pgfpathlineto{\pgfqpoint{3.318476in}{3.241016in}}%
\pgfpathlineto{\pgfqpoint{3.318476in}{3.236758in}}%
\pgfpathmoveto{\pgfqpoint{3.318476in}{3.236758in}}%
\pgfpathlineto{\pgfqpoint{3.318476in}{3.236758in}}%
\pgfpathlineto{\pgfqpoint{3.318476in}{3.241016in}}%
\pgfpathlineto{\pgfqpoint{3.322734in}{3.241016in}}%
\pgfpathlineto{\pgfqpoint{3.322734in}{3.236758in}}%
\pgfpathmoveto{\pgfqpoint{3.314218in}{3.241016in}}%
\pgfpathlineto{\pgfqpoint{3.314218in}{3.241016in}}%
\pgfpathlineto{\pgfqpoint{3.314218in}{3.245274in}}%
\pgfpathlineto{\pgfqpoint{3.318476in}{3.245274in}}%
\pgfpathlineto{\pgfqpoint{3.318476in}{3.241016in}}%
\pgfpathmoveto{\pgfqpoint{3.314218in}{3.245274in}}%
\pgfpathlineto{\pgfqpoint{3.314218in}{3.245274in}}%
\pgfpathlineto{\pgfqpoint{3.314218in}{3.249531in}}%
\pgfpathlineto{\pgfqpoint{3.318476in}{3.249531in}}%
\pgfpathlineto{\pgfqpoint{3.318476in}{3.245274in}}%
\pgfpathmoveto{\pgfqpoint{3.318476in}{3.241016in}}%
\pgfpathlineto{\pgfqpoint{3.318476in}{3.241016in}}%
\pgfpathlineto{\pgfqpoint{3.318476in}{3.245274in}}%
\pgfpathlineto{\pgfqpoint{3.322734in}{3.245274in}}%
\pgfpathlineto{\pgfqpoint{3.322734in}{3.241016in}}%
\pgfpathmoveto{\pgfqpoint{3.318476in}{3.245274in}}%
\pgfpathlineto{\pgfqpoint{3.318476in}{3.245274in}}%
\pgfpathlineto{\pgfqpoint{3.318476in}{3.249531in}}%
\pgfpathlineto{\pgfqpoint{3.322734in}{3.249531in}}%
\pgfpathlineto{\pgfqpoint{3.322734in}{3.245274in}}%
\pgfpathmoveto{\pgfqpoint{3.322734in}{3.245274in}}%
\pgfpathlineto{\pgfqpoint{3.322734in}{3.245274in}}%
\pgfpathlineto{\pgfqpoint{3.322734in}{3.249531in}}%
\pgfpathlineto{\pgfqpoint{3.326992in}{3.249531in}}%
\pgfpathlineto{\pgfqpoint{3.326992in}{3.245274in}}%
\pgfpathmoveto{\pgfqpoint{3.314218in}{3.249531in}}%
\pgfpathlineto{\pgfqpoint{3.314218in}{3.249531in}}%
\pgfpathlineto{\pgfqpoint{3.314218in}{3.253789in}}%
\pgfpathlineto{\pgfqpoint{3.318476in}{3.253789in}}%
\pgfpathlineto{\pgfqpoint{3.318476in}{3.249531in}}%
\pgfpathmoveto{\pgfqpoint{3.318476in}{3.249531in}}%
\pgfpathlineto{\pgfqpoint{3.318476in}{3.249531in}}%
\pgfpathlineto{\pgfqpoint{3.318476in}{3.253789in}}%
\pgfpathlineto{\pgfqpoint{3.322734in}{3.253789in}}%
\pgfpathlineto{\pgfqpoint{3.322734in}{3.249531in}}%
\pgfpathmoveto{\pgfqpoint{3.318476in}{3.253789in}}%
\pgfpathlineto{\pgfqpoint{3.318476in}{3.253789in}}%
\pgfpathlineto{\pgfqpoint{3.318476in}{3.258047in}}%
\pgfpathlineto{\pgfqpoint{3.322734in}{3.258047in}}%
\pgfpathlineto{\pgfqpoint{3.322734in}{3.253789in}}%
\pgfpathmoveto{\pgfqpoint{3.318476in}{3.258047in}}%
\pgfpathlineto{\pgfqpoint{3.318476in}{3.258047in}}%
\pgfpathlineto{\pgfqpoint{3.318476in}{3.262305in}}%
\pgfpathlineto{\pgfqpoint{3.322734in}{3.262305in}}%
\pgfpathlineto{\pgfqpoint{3.322734in}{3.258047in}}%
\pgfpathmoveto{\pgfqpoint{3.322734in}{3.249531in}}%
\pgfpathlineto{\pgfqpoint{3.322734in}{3.249531in}}%
\pgfpathlineto{\pgfqpoint{3.322734in}{3.253789in}}%
\pgfpathlineto{\pgfqpoint{3.326992in}{3.253789in}}%
\pgfpathlineto{\pgfqpoint{3.326992in}{3.249531in}}%
\pgfpathmoveto{\pgfqpoint{3.322734in}{3.253789in}}%
\pgfpathlineto{\pgfqpoint{3.322734in}{3.253789in}}%
\pgfpathlineto{\pgfqpoint{3.322734in}{3.258047in}}%
\pgfpathlineto{\pgfqpoint{3.326992in}{3.258047in}}%
\pgfpathlineto{\pgfqpoint{3.326992in}{3.253789in}}%
\pgfpathmoveto{\pgfqpoint{3.326992in}{3.253789in}}%
\pgfpathlineto{\pgfqpoint{3.326992in}{3.253789in}}%
\pgfpathlineto{\pgfqpoint{3.326992in}{3.258047in}}%
\pgfpathlineto{\pgfqpoint{3.331250in}{3.258047in}}%
\pgfpathlineto{\pgfqpoint{3.331250in}{3.253789in}}%
\pgfpathmoveto{\pgfqpoint{3.322734in}{3.258047in}}%
\pgfpathlineto{\pgfqpoint{3.322734in}{3.258047in}}%
\pgfpathlineto{\pgfqpoint{3.322734in}{3.262305in}}%
\pgfpathlineto{\pgfqpoint{3.326992in}{3.262305in}}%
\pgfpathlineto{\pgfqpoint{3.326992in}{3.258047in}}%
\pgfpathmoveto{\pgfqpoint{3.322734in}{3.262305in}}%
\pgfpathlineto{\pgfqpoint{3.322734in}{3.262305in}}%
\pgfpathlineto{\pgfqpoint{3.322734in}{3.266563in}}%
\pgfpathlineto{\pgfqpoint{3.326992in}{3.266563in}}%
\pgfpathlineto{\pgfqpoint{3.326992in}{3.262305in}}%
\pgfpathmoveto{\pgfqpoint{3.326992in}{3.258047in}}%
\pgfpathlineto{\pgfqpoint{3.326992in}{3.258047in}}%
\pgfpathlineto{\pgfqpoint{3.326992in}{3.262305in}}%
\pgfpathlineto{\pgfqpoint{3.331250in}{3.262305in}}%
\pgfpathlineto{\pgfqpoint{3.331250in}{3.258047in}}%
\pgfpathmoveto{\pgfqpoint{3.326992in}{3.262305in}}%
\pgfpathlineto{\pgfqpoint{3.326992in}{3.262305in}}%
\pgfpathlineto{\pgfqpoint{3.326992in}{3.266563in}}%
\pgfpathlineto{\pgfqpoint{3.331250in}{3.266563in}}%
\pgfpathlineto{\pgfqpoint{3.331250in}{3.262305in}}%
\pgfpathmoveto{\pgfqpoint{3.322734in}{3.266563in}}%
\pgfpathlineto{\pgfqpoint{3.322734in}{3.266563in}}%
\pgfpathlineto{\pgfqpoint{3.322734in}{3.270820in}}%
\pgfpathlineto{\pgfqpoint{3.326992in}{3.270820in}}%
\pgfpathlineto{\pgfqpoint{3.326992in}{3.266563in}}%
\pgfpathmoveto{\pgfqpoint{3.322734in}{3.270820in}}%
\pgfpathlineto{\pgfqpoint{3.322734in}{3.270820in}}%
\pgfpathlineto{\pgfqpoint{3.322734in}{3.275078in}}%
\pgfpathlineto{\pgfqpoint{3.326992in}{3.275078in}}%
\pgfpathlineto{\pgfqpoint{3.326992in}{3.270820in}}%
\pgfpathmoveto{\pgfqpoint{3.326992in}{3.266563in}}%
\pgfpathlineto{\pgfqpoint{3.326992in}{3.266563in}}%
\pgfpathlineto{\pgfqpoint{3.326992in}{3.270820in}}%
\pgfpathlineto{\pgfqpoint{3.331250in}{3.270820in}}%
\pgfpathlineto{\pgfqpoint{3.331250in}{3.266563in}}%
\pgfpathmoveto{\pgfqpoint{3.326992in}{3.270820in}}%
\pgfpathlineto{\pgfqpoint{3.326992in}{3.270820in}}%
\pgfpathlineto{\pgfqpoint{3.326992in}{3.275078in}}%
\pgfpathlineto{\pgfqpoint{3.331250in}{3.275078in}}%
\pgfpathlineto{\pgfqpoint{3.331250in}{3.270820in}}%
\pgfpathmoveto{\pgfqpoint{3.326992in}{3.275078in}}%
\pgfpathlineto{\pgfqpoint{3.326992in}{3.275078in}}%
\pgfpathlineto{\pgfqpoint{3.326992in}{3.279336in}}%
\pgfpathlineto{\pgfqpoint{3.331250in}{3.279336in}}%
\pgfpathlineto{\pgfqpoint{3.331250in}{3.275078in}}%
\pgfpathmoveto{\pgfqpoint{3.326992in}{3.279336in}}%
\pgfpathlineto{\pgfqpoint{3.326992in}{3.279336in}}%
\pgfpathlineto{\pgfqpoint{3.326992in}{3.283594in}}%
\pgfpathlineto{\pgfqpoint{3.331250in}{3.283594in}}%
\pgfpathlineto{\pgfqpoint{3.331250in}{3.279336in}}%
\pgfpathmoveto{\pgfqpoint{3.331250in}{3.262305in}}%
\pgfpathlineto{\pgfqpoint{3.331250in}{3.262305in}}%
\pgfpathlineto{\pgfqpoint{3.331250in}{3.266563in}}%
\pgfpathlineto{\pgfqpoint{3.335508in}{3.266563in}}%
\pgfpathlineto{\pgfqpoint{3.335508in}{3.262305in}}%
\pgfpathmoveto{\pgfqpoint{3.331250in}{3.266563in}}%
\pgfpathlineto{\pgfqpoint{3.331250in}{3.266563in}}%
\pgfpathlineto{\pgfqpoint{3.331250in}{3.270820in}}%
\pgfpathlineto{\pgfqpoint{3.335508in}{3.270820in}}%
\pgfpathlineto{\pgfqpoint{3.335508in}{3.266563in}}%
\pgfpathmoveto{\pgfqpoint{3.331250in}{3.270820in}}%
\pgfpathlineto{\pgfqpoint{3.331250in}{3.270820in}}%
\pgfpathlineto{\pgfqpoint{3.331250in}{3.275078in}}%
\pgfpathlineto{\pgfqpoint{3.335508in}{3.275078in}}%
\pgfpathlineto{\pgfqpoint{3.335508in}{3.270820in}}%
\pgfpathmoveto{\pgfqpoint{3.335508in}{3.270820in}}%
\pgfpathlineto{\pgfqpoint{3.335508in}{3.270820in}}%
\pgfpathlineto{\pgfqpoint{3.335508in}{3.275078in}}%
\pgfpathlineto{\pgfqpoint{3.339766in}{3.275078in}}%
\pgfpathlineto{\pgfqpoint{3.339766in}{3.270820in}}%
\pgfpathmoveto{\pgfqpoint{3.331250in}{3.275078in}}%
\pgfpathlineto{\pgfqpoint{3.331250in}{3.275078in}}%
\pgfpathlineto{\pgfqpoint{3.331250in}{3.279336in}}%
\pgfpathlineto{\pgfqpoint{3.335508in}{3.279336in}}%
\pgfpathlineto{\pgfqpoint{3.335508in}{3.275078in}}%
\pgfpathmoveto{\pgfqpoint{3.331250in}{3.279336in}}%
\pgfpathlineto{\pgfqpoint{3.331250in}{3.279336in}}%
\pgfpathlineto{\pgfqpoint{3.331250in}{3.283594in}}%
\pgfpathlineto{\pgfqpoint{3.335508in}{3.283594in}}%
\pgfpathlineto{\pgfqpoint{3.335508in}{3.279336in}}%
\pgfpathmoveto{\pgfqpoint{3.335508in}{3.275078in}}%
\pgfpathlineto{\pgfqpoint{3.335508in}{3.275078in}}%
\pgfpathlineto{\pgfqpoint{3.335508in}{3.279336in}}%
\pgfpathlineto{\pgfqpoint{3.339766in}{3.279336in}}%
\pgfpathlineto{\pgfqpoint{3.339766in}{3.275078in}}%
\pgfpathmoveto{\pgfqpoint{3.335508in}{3.279336in}}%
\pgfpathlineto{\pgfqpoint{3.335508in}{3.279336in}}%
\pgfpathlineto{\pgfqpoint{3.335508in}{3.283594in}}%
\pgfpathlineto{\pgfqpoint{3.339766in}{3.283594in}}%
\pgfpathlineto{\pgfqpoint{3.339766in}{3.279336in}}%
\pgfpathmoveto{\pgfqpoint{3.339766in}{3.279336in}}%
\pgfpathlineto{\pgfqpoint{3.339766in}{3.279336in}}%
\pgfpathlineto{\pgfqpoint{3.339766in}{3.283594in}}%
\pgfpathlineto{\pgfqpoint{3.344024in}{3.283594in}}%
\pgfpathlineto{\pgfqpoint{3.344024in}{3.279336in}}%
\pgfpathmoveto{\pgfqpoint{3.331250in}{3.283594in}}%
\pgfpathlineto{\pgfqpoint{3.331250in}{3.283594in}}%
\pgfpathlineto{\pgfqpoint{3.331250in}{3.287851in}}%
\pgfpathlineto{\pgfqpoint{3.335508in}{3.287851in}}%
\pgfpathlineto{\pgfqpoint{3.335508in}{3.283594in}}%
\pgfpathmoveto{\pgfqpoint{3.331250in}{3.287851in}}%
\pgfpathlineto{\pgfqpoint{3.331250in}{3.287851in}}%
\pgfpathlineto{\pgfqpoint{3.331250in}{3.292109in}}%
\pgfpathlineto{\pgfqpoint{3.335508in}{3.292109in}}%
\pgfpathlineto{\pgfqpoint{3.335508in}{3.287851in}}%
\pgfpathmoveto{\pgfqpoint{3.335508in}{3.283594in}}%
\pgfpathlineto{\pgfqpoint{3.335508in}{3.283594in}}%
\pgfpathlineto{\pgfqpoint{3.335508in}{3.287851in}}%
\pgfpathlineto{\pgfqpoint{3.339766in}{3.287851in}}%
\pgfpathlineto{\pgfqpoint{3.339766in}{3.283594in}}%
\pgfpathmoveto{\pgfqpoint{3.335508in}{3.287851in}}%
\pgfpathlineto{\pgfqpoint{3.335508in}{3.287851in}}%
\pgfpathlineto{\pgfqpoint{3.335508in}{3.292109in}}%
\pgfpathlineto{\pgfqpoint{3.339766in}{3.292109in}}%
\pgfpathlineto{\pgfqpoint{3.339766in}{3.287851in}}%
\pgfpathmoveto{\pgfqpoint{3.335508in}{3.292109in}}%
\pgfpathlineto{\pgfqpoint{3.335508in}{3.292109in}}%
\pgfpathlineto{\pgfqpoint{3.335508in}{3.296367in}}%
\pgfpathlineto{\pgfqpoint{3.339766in}{3.296367in}}%
\pgfpathlineto{\pgfqpoint{3.339766in}{3.292109in}}%
\pgfpathmoveto{\pgfqpoint{3.335508in}{3.296367in}}%
\pgfpathlineto{\pgfqpoint{3.335508in}{3.296367in}}%
\pgfpathlineto{\pgfqpoint{3.335508in}{3.300625in}}%
\pgfpathlineto{\pgfqpoint{3.339766in}{3.300625in}}%
\pgfpathlineto{\pgfqpoint{3.339766in}{3.296367in}}%
\pgfpathmoveto{\pgfqpoint{3.339766in}{3.283594in}}%
\pgfpathlineto{\pgfqpoint{3.339766in}{3.283594in}}%
\pgfpathlineto{\pgfqpoint{3.339766in}{3.287851in}}%
\pgfpathlineto{\pgfqpoint{3.344024in}{3.287851in}}%
\pgfpathlineto{\pgfqpoint{3.344024in}{3.283594in}}%
\pgfpathmoveto{\pgfqpoint{3.339766in}{3.287851in}}%
\pgfpathlineto{\pgfqpoint{3.339766in}{3.287851in}}%
\pgfpathlineto{\pgfqpoint{3.339766in}{3.292109in}}%
\pgfpathlineto{\pgfqpoint{3.344024in}{3.292109in}}%
\pgfpathlineto{\pgfqpoint{3.344024in}{3.287851in}}%
\pgfpathmoveto{\pgfqpoint{3.344024in}{3.287851in}}%
\pgfpathlineto{\pgfqpoint{3.344024in}{3.287851in}}%
\pgfpathlineto{\pgfqpoint{3.344024in}{3.292109in}}%
\pgfpathlineto{\pgfqpoint{3.348282in}{3.292109in}}%
\pgfpathlineto{\pgfqpoint{3.348282in}{3.287851in}}%
\pgfpathmoveto{\pgfqpoint{3.339766in}{3.292109in}}%
\pgfpathlineto{\pgfqpoint{3.339766in}{3.292109in}}%
\pgfpathlineto{\pgfqpoint{3.339766in}{3.296367in}}%
\pgfpathlineto{\pgfqpoint{3.344024in}{3.296367in}}%
\pgfpathlineto{\pgfqpoint{3.344024in}{3.292109in}}%
\pgfpathmoveto{\pgfqpoint{3.339766in}{3.296367in}}%
\pgfpathlineto{\pgfqpoint{3.339766in}{3.296367in}}%
\pgfpathlineto{\pgfqpoint{3.339766in}{3.300625in}}%
\pgfpathlineto{\pgfqpoint{3.344024in}{3.300625in}}%
\pgfpathlineto{\pgfqpoint{3.344024in}{3.296367in}}%
\pgfpathmoveto{\pgfqpoint{3.344024in}{3.292109in}}%
\pgfpathlineto{\pgfqpoint{3.344024in}{3.292109in}}%
\pgfpathlineto{\pgfqpoint{3.344024in}{3.296367in}}%
\pgfpathlineto{\pgfqpoint{3.348282in}{3.296367in}}%
\pgfpathlineto{\pgfqpoint{3.348282in}{3.292109in}}%
\pgfpathmoveto{\pgfqpoint{3.344024in}{3.296367in}}%
\pgfpathlineto{\pgfqpoint{3.344024in}{3.296367in}}%
\pgfpathlineto{\pgfqpoint{3.344024in}{3.300625in}}%
\pgfpathlineto{\pgfqpoint{3.348282in}{3.300625in}}%
\pgfpathlineto{\pgfqpoint{3.348282in}{3.296367in}}%
\pgfpathmoveto{\pgfqpoint{3.348282in}{3.296367in}}%
\pgfpathlineto{\pgfqpoint{3.348282in}{3.296367in}}%
\pgfpathlineto{\pgfqpoint{3.348282in}{3.300625in}}%
\pgfpathlineto{\pgfqpoint{3.352540in}{3.300625in}}%
\pgfpathlineto{\pgfqpoint{3.352540in}{3.296367in}}%
\pgfpathmoveto{\pgfqpoint{3.339766in}{3.300625in}}%
\pgfpathlineto{\pgfqpoint{3.339766in}{3.300625in}}%
\pgfpathlineto{\pgfqpoint{3.339766in}{3.304883in}}%
\pgfpathlineto{\pgfqpoint{3.344024in}{3.304883in}}%
\pgfpathlineto{\pgfqpoint{3.344024in}{3.300625in}}%
\pgfpathmoveto{\pgfqpoint{3.339766in}{3.304883in}}%
\pgfpathlineto{\pgfqpoint{3.339766in}{3.304883in}}%
\pgfpathlineto{\pgfqpoint{3.339766in}{3.309140in}}%
\pgfpathlineto{\pgfqpoint{3.344024in}{3.309140in}}%
\pgfpathlineto{\pgfqpoint{3.344024in}{3.304883in}}%
\pgfpathmoveto{\pgfqpoint{3.344024in}{3.300625in}}%
\pgfpathlineto{\pgfqpoint{3.344024in}{3.300625in}}%
\pgfpathlineto{\pgfqpoint{3.344024in}{3.304883in}}%
\pgfpathlineto{\pgfqpoint{3.348282in}{3.304883in}}%
\pgfpathlineto{\pgfqpoint{3.348282in}{3.300625in}}%
\pgfpathmoveto{\pgfqpoint{3.344024in}{3.304883in}}%
\pgfpathlineto{\pgfqpoint{3.344024in}{3.304883in}}%
\pgfpathlineto{\pgfqpoint{3.344024in}{3.309140in}}%
\pgfpathlineto{\pgfqpoint{3.348282in}{3.309140in}}%
\pgfpathlineto{\pgfqpoint{3.348282in}{3.304883in}}%
\pgfpathmoveto{\pgfqpoint{3.344024in}{3.309140in}}%
\pgfpathlineto{\pgfqpoint{3.344024in}{3.309140in}}%
\pgfpathlineto{\pgfqpoint{3.344024in}{3.313398in}}%
\pgfpathlineto{\pgfqpoint{3.348282in}{3.313398in}}%
\pgfpathlineto{\pgfqpoint{3.348282in}{3.309140in}}%
\pgfpathmoveto{\pgfqpoint{3.344024in}{3.313398in}}%
\pgfpathlineto{\pgfqpoint{3.344024in}{3.313398in}}%
\pgfpathlineto{\pgfqpoint{3.344024in}{3.317656in}}%
\pgfpathlineto{\pgfqpoint{3.348282in}{3.317656in}}%
\pgfpathlineto{\pgfqpoint{3.348282in}{3.313398in}}%
\pgfpathmoveto{\pgfqpoint{3.348282in}{3.300625in}}%
\pgfpathlineto{\pgfqpoint{3.348282in}{3.300625in}}%
\pgfpathlineto{\pgfqpoint{3.348282in}{3.304883in}}%
\pgfpathlineto{\pgfqpoint{3.352540in}{3.304883in}}%
\pgfpathlineto{\pgfqpoint{3.352540in}{3.300625in}}%
\pgfpathmoveto{\pgfqpoint{3.348282in}{3.304883in}}%
\pgfpathlineto{\pgfqpoint{3.348282in}{3.304883in}}%
\pgfpathlineto{\pgfqpoint{3.348282in}{3.309140in}}%
\pgfpathlineto{\pgfqpoint{3.352540in}{3.309140in}}%
\pgfpathlineto{\pgfqpoint{3.352540in}{3.304883in}}%
\pgfpathmoveto{\pgfqpoint{3.352540in}{3.300625in}}%
\pgfpathlineto{\pgfqpoint{3.352540in}{3.300625in}}%
\pgfpathlineto{\pgfqpoint{3.352540in}{3.304883in}}%
\pgfpathlineto{\pgfqpoint{3.356798in}{3.304883in}}%
\pgfpathlineto{\pgfqpoint{3.356798in}{3.300625in}}%
\pgfpathmoveto{\pgfqpoint{3.352540in}{3.304883in}}%
\pgfpathlineto{\pgfqpoint{3.352540in}{3.304883in}}%
\pgfpathlineto{\pgfqpoint{3.352540in}{3.309140in}}%
\pgfpathlineto{\pgfqpoint{3.356798in}{3.309140in}}%
\pgfpathlineto{\pgfqpoint{3.356798in}{3.304883in}}%
\pgfpathmoveto{\pgfqpoint{3.348282in}{3.309140in}}%
\pgfpathlineto{\pgfqpoint{3.348282in}{3.309140in}}%
\pgfpathlineto{\pgfqpoint{3.348282in}{3.313398in}}%
\pgfpathlineto{\pgfqpoint{3.352540in}{3.313398in}}%
\pgfpathlineto{\pgfqpoint{3.352540in}{3.309140in}}%
\pgfpathmoveto{\pgfqpoint{3.348282in}{3.313398in}}%
\pgfpathlineto{\pgfqpoint{3.348282in}{3.313398in}}%
\pgfpathlineto{\pgfqpoint{3.348282in}{3.317656in}}%
\pgfpathlineto{\pgfqpoint{3.352540in}{3.317656in}}%
\pgfpathlineto{\pgfqpoint{3.352540in}{3.313398in}}%
\pgfpathmoveto{\pgfqpoint{3.352540in}{3.309140in}}%
\pgfpathlineto{\pgfqpoint{3.352540in}{3.309140in}}%
\pgfpathlineto{\pgfqpoint{3.352540in}{3.313398in}}%
\pgfpathlineto{\pgfqpoint{3.356798in}{3.313398in}}%
\pgfpathlineto{\pgfqpoint{3.356798in}{3.309140in}}%
\pgfpathmoveto{\pgfqpoint{3.352540in}{3.313398in}}%
\pgfpathlineto{\pgfqpoint{3.352540in}{3.313398in}}%
\pgfpathlineto{\pgfqpoint{3.352540in}{3.317656in}}%
\pgfpathlineto{\pgfqpoint{3.356798in}{3.317656in}}%
\pgfpathlineto{\pgfqpoint{3.356798in}{3.313398in}}%
\pgfpathmoveto{\pgfqpoint{3.356798in}{3.309140in}}%
\pgfpathlineto{\pgfqpoint{3.356798in}{3.309140in}}%
\pgfpathlineto{\pgfqpoint{3.356798in}{3.313398in}}%
\pgfpathlineto{\pgfqpoint{3.361056in}{3.313398in}}%
\pgfpathlineto{\pgfqpoint{3.361056in}{3.309140in}}%
\pgfpathmoveto{\pgfqpoint{3.356798in}{3.313398in}}%
\pgfpathlineto{\pgfqpoint{3.356798in}{3.313398in}}%
\pgfpathlineto{\pgfqpoint{3.356798in}{3.317656in}}%
\pgfpathlineto{\pgfqpoint{3.361056in}{3.317656in}}%
\pgfpathlineto{\pgfqpoint{3.361056in}{3.313398in}}%
\pgfpathmoveto{\pgfqpoint{3.348282in}{3.317656in}}%
\pgfpathlineto{\pgfqpoint{3.348282in}{3.317656in}}%
\pgfpathlineto{\pgfqpoint{3.348282in}{3.321914in}}%
\pgfpathlineto{\pgfqpoint{3.352540in}{3.321914in}}%
\pgfpathlineto{\pgfqpoint{3.352540in}{3.317656in}}%
\pgfpathmoveto{\pgfqpoint{3.348282in}{3.321914in}}%
\pgfpathlineto{\pgfqpoint{3.348282in}{3.321914in}}%
\pgfpathlineto{\pgfqpoint{3.348282in}{3.326172in}}%
\pgfpathlineto{\pgfqpoint{3.352540in}{3.326172in}}%
\pgfpathlineto{\pgfqpoint{3.352540in}{3.321914in}}%
\pgfpathmoveto{\pgfqpoint{3.352540in}{3.317656in}}%
\pgfpathlineto{\pgfqpoint{3.352540in}{3.317656in}}%
\pgfpathlineto{\pgfqpoint{3.352540in}{3.321914in}}%
\pgfpathlineto{\pgfqpoint{3.356798in}{3.321914in}}%
\pgfpathlineto{\pgfqpoint{3.356798in}{3.317656in}}%
\pgfpathmoveto{\pgfqpoint{3.352540in}{3.321914in}}%
\pgfpathlineto{\pgfqpoint{3.352540in}{3.321914in}}%
\pgfpathlineto{\pgfqpoint{3.352540in}{3.326172in}}%
\pgfpathlineto{\pgfqpoint{3.356798in}{3.326172in}}%
\pgfpathlineto{\pgfqpoint{3.356798in}{3.321914in}}%
\pgfpathmoveto{\pgfqpoint{3.352540in}{3.326172in}}%
\pgfpathlineto{\pgfqpoint{3.352540in}{3.326172in}}%
\pgfpathlineto{\pgfqpoint{3.352540in}{3.330429in}}%
\pgfpathlineto{\pgfqpoint{3.356798in}{3.330429in}}%
\pgfpathlineto{\pgfqpoint{3.356798in}{3.326172in}}%
\pgfpathmoveto{\pgfqpoint{3.352540in}{3.330429in}}%
\pgfpathlineto{\pgfqpoint{3.352540in}{3.330429in}}%
\pgfpathlineto{\pgfqpoint{3.352540in}{3.334687in}}%
\pgfpathlineto{\pgfqpoint{3.356798in}{3.334687in}}%
\pgfpathlineto{\pgfqpoint{3.356798in}{3.330429in}}%
\pgfpathmoveto{\pgfqpoint{3.356798in}{3.317656in}}%
\pgfpathlineto{\pgfqpoint{3.356798in}{3.317656in}}%
\pgfpathlineto{\pgfqpoint{3.356798in}{3.321914in}}%
\pgfpathlineto{\pgfqpoint{3.361056in}{3.321914in}}%
\pgfpathlineto{\pgfqpoint{3.361056in}{3.317656in}}%
\pgfpathmoveto{\pgfqpoint{3.356798in}{3.321914in}}%
\pgfpathlineto{\pgfqpoint{3.356798in}{3.321914in}}%
\pgfpathlineto{\pgfqpoint{3.356798in}{3.326172in}}%
\pgfpathlineto{\pgfqpoint{3.361056in}{3.326172in}}%
\pgfpathlineto{\pgfqpoint{3.361056in}{3.321914in}}%
\pgfpathmoveto{\pgfqpoint{3.361056in}{3.317656in}}%
\pgfpathlineto{\pgfqpoint{3.361056in}{3.317656in}}%
\pgfpathlineto{\pgfqpoint{3.361056in}{3.321914in}}%
\pgfpathlineto{\pgfqpoint{3.365314in}{3.321914in}}%
\pgfpathlineto{\pgfqpoint{3.365314in}{3.317656in}}%
\pgfpathmoveto{\pgfqpoint{3.361056in}{3.321914in}}%
\pgfpathlineto{\pgfqpoint{3.361056in}{3.321914in}}%
\pgfpathlineto{\pgfqpoint{3.361056in}{3.326172in}}%
\pgfpathlineto{\pgfqpoint{3.365314in}{3.326172in}}%
\pgfpathlineto{\pgfqpoint{3.365314in}{3.321914in}}%
\pgfpathmoveto{\pgfqpoint{3.356798in}{3.326172in}}%
\pgfpathlineto{\pgfqpoint{3.356798in}{3.326172in}}%
\pgfpathlineto{\pgfqpoint{3.356798in}{3.330429in}}%
\pgfpathlineto{\pgfqpoint{3.361056in}{3.330429in}}%
\pgfpathlineto{\pgfqpoint{3.361056in}{3.326172in}}%
\pgfpathmoveto{\pgfqpoint{3.356798in}{3.330429in}}%
\pgfpathlineto{\pgfqpoint{3.356798in}{3.330429in}}%
\pgfpathlineto{\pgfqpoint{3.356798in}{3.334687in}}%
\pgfpathlineto{\pgfqpoint{3.361056in}{3.334687in}}%
\pgfpathlineto{\pgfqpoint{3.361056in}{3.330429in}}%
\pgfpathmoveto{\pgfqpoint{3.361056in}{3.326172in}}%
\pgfpathlineto{\pgfqpoint{3.361056in}{3.326172in}}%
\pgfpathlineto{\pgfqpoint{3.361056in}{3.330429in}}%
\pgfpathlineto{\pgfqpoint{3.365314in}{3.330429in}}%
\pgfpathlineto{\pgfqpoint{3.365314in}{3.326172in}}%
\pgfpathmoveto{\pgfqpoint{3.361056in}{3.330429in}}%
\pgfpathlineto{\pgfqpoint{3.361056in}{3.330429in}}%
\pgfpathlineto{\pgfqpoint{3.361056in}{3.334687in}}%
\pgfpathlineto{\pgfqpoint{3.365314in}{3.334687in}}%
\pgfpathlineto{\pgfqpoint{3.365314in}{3.330429in}}%
\pgfpathmoveto{\pgfqpoint{3.356798in}{3.334687in}}%
\pgfpathlineto{\pgfqpoint{3.356798in}{3.334687in}}%
\pgfpathlineto{\pgfqpoint{3.356798in}{3.338945in}}%
\pgfpathlineto{\pgfqpoint{3.361056in}{3.338945in}}%
\pgfpathlineto{\pgfqpoint{3.361056in}{3.334687in}}%
\pgfpathmoveto{\pgfqpoint{3.361056in}{3.334687in}}%
\pgfpathlineto{\pgfqpoint{3.361056in}{3.334687in}}%
\pgfpathlineto{\pgfqpoint{3.361056in}{3.338945in}}%
\pgfpathlineto{\pgfqpoint{3.365314in}{3.338945in}}%
\pgfpathlineto{\pgfqpoint{3.365314in}{3.334687in}}%
\pgfpathmoveto{\pgfqpoint{3.361056in}{3.338945in}}%
\pgfpathlineto{\pgfqpoint{3.361056in}{3.338945in}}%
\pgfpathlineto{\pgfqpoint{3.361056in}{3.343203in}}%
\pgfpathlineto{\pgfqpoint{3.365314in}{3.343203in}}%
\pgfpathlineto{\pgfqpoint{3.365314in}{3.338945in}}%
\pgfpathmoveto{\pgfqpoint{3.361056in}{3.343203in}}%
\pgfpathlineto{\pgfqpoint{3.361056in}{3.343203in}}%
\pgfpathlineto{\pgfqpoint{3.361056in}{3.347461in}}%
\pgfpathlineto{\pgfqpoint{3.365314in}{3.347461in}}%
\pgfpathlineto{\pgfqpoint{3.365314in}{3.343203in}}%
\pgfpathmoveto{\pgfqpoint{3.365314in}{3.321914in}}%
\pgfpathlineto{\pgfqpoint{3.365314in}{3.321914in}}%
\pgfpathlineto{\pgfqpoint{3.365314in}{3.326172in}}%
\pgfpathlineto{\pgfqpoint{3.369572in}{3.326172in}}%
\pgfpathlineto{\pgfqpoint{3.369572in}{3.321914in}}%
\pgfpathmoveto{\pgfqpoint{3.365314in}{3.326172in}}%
\pgfpathlineto{\pgfqpoint{3.365314in}{3.326172in}}%
\pgfpathlineto{\pgfqpoint{3.365314in}{3.330429in}}%
\pgfpathlineto{\pgfqpoint{3.369572in}{3.330429in}}%
\pgfpathlineto{\pgfqpoint{3.369572in}{3.326172in}}%
\pgfpathmoveto{\pgfqpoint{3.365314in}{3.330429in}}%
\pgfpathlineto{\pgfqpoint{3.365314in}{3.330429in}}%
\pgfpathlineto{\pgfqpoint{3.365314in}{3.334687in}}%
\pgfpathlineto{\pgfqpoint{3.369572in}{3.334687in}}%
\pgfpathlineto{\pgfqpoint{3.369572in}{3.330429in}}%
\pgfpathmoveto{\pgfqpoint{3.369572in}{3.326172in}}%
\pgfpathlineto{\pgfqpoint{3.369572in}{3.326172in}}%
\pgfpathlineto{\pgfqpoint{3.369572in}{3.330429in}}%
\pgfpathlineto{\pgfqpoint{3.373829in}{3.330429in}}%
\pgfpathlineto{\pgfqpoint{3.373829in}{3.326172in}}%
\pgfpathmoveto{\pgfqpoint{3.369572in}{3.330429in}}%
\pgfpathlineto{\pgfqpoint{3.369572in}{3.330429in}}%
\pgfpathlineto{\pgfqpoint{3.369572in}{3.334687in}}%
\pgfpathlineto{\pgfqpoint{3.373829in}{3.334687in}}%
\pgfpathlineto{\pgfqpoint{3.373829in}{3.330429in}}%
\pgfpathmoveto{\pgfqpoint{3.365314in}{3.334687in}}%
\pgfpathlineto{\pgfqpoint{3.365314in}{3.334687in}}%
\pgfpathlineto{\pgfqpoint{3.365314in}{3.338945in}}%
\pgfpathlineto{\pgfqpoint{3.369572in}{3.338945in}}%
\pgfpathlineto{\pgfqpoint{3.369572in}{3.334687in}}%
\pgfpathmoveto{\pgfqpoint{3.365314in}{3.338945in}}%
\pgfpathlineto{\pgfqpoint{3.365314in}{3.338945in}}%
\pgfpathlineto{\pgfqpoint{3.365314in}{3.343203in}}%
\pgfpathlineto{\pgfqpoint{3.369572in}{3.343203in}}%
\pgfpathlineto{\pgfqpoint{3.369572in}{3.338945in}}%
\pgfpathmoveto{\pgfqpoint{3.369572in}{3.334687in}}%
\pgfpathlineto{\pgfqpoint{3.369572in}{3.334687in}}%
\pgfpathlineto{\pgfqpoint{3.369572in}{3.338945in}}%
\pgfpathlineto{\pgfqpoint{3.373829in}{3.338945in}}%
\pgfpathlineto{\pgfqpoint{3.373829in}{3.334687in}}%
\pgfpathmoveto{\pgfqpoint{3.369572in}{3.338945in}}%
\pgfpathlineto{\pgfqpoint{3.369572in}{3.338945in}}%
\pgfpathlineto{\pgfqpoint{3.369572in}{3.343203in}}%
\pgfpathlineto{\pgfqpoint{3.373829in}{3.343203in}}%
\pgfpathlineto{\pgfqpoint{3.373829in}{3.338945in}}%
\pgfpathmoveto{\pgfqpoint{3.365314in}{3.343203in}}%
\pgfpathlineto{\pgfqpoint{3.365314in}{3.343203in}}%
\pgfpathlineto{\pgfqpoint{3.365314in}{3.347461in}}%
\pgfpathlineto{\pgfqpoint{3.369572in}{3.347461in}}%
\pgfpathlineto{\pgfqpoint{3.369572in}{3.343203in}}%
\pgfpathmoveto{\pgfqpoint{3.365314in}{3.347461in}}%
\pgfpathlineto{\pgfqpoint{3.365314in}{3.347461in}}%
\pgfpathlineto{\pgfqpoint{3.365314in}{3.351718in}}%
\pgfpathlineto{\pgfqpoint{3.369572in}{3.351718in}}%
\pgfpathlineto{\pgfqpoint{3.369572in}{3.347461in}}%
\pgfpathmoveto{\pgfqpoint{3.369572in}{3.343203in}}%
\pgfpathlineto{\pgfqpoint{3.369572in}{3.343203in}}%
\pgfpathlineto{\pgfqpoint{3.369572in}{3.347461in}}%
\pgfpathlineto{\pgfqpoint{3.373829in}{3.347461in}}%
\pgfpathlineto{\pgfqpoint{3.373829in}{3.343203in}}%
\pgfpathmoveto{\pgfqpoint{3.369572in}{3.347461in}}%
\pgfpathlineto{\pgfqpoint{3.369572in}{3.347461in}}%
\pgfpathlineto{\pgfqpoint{3.369572in}{3.351718in}}%
\pgfpathlineto{\pgfqpoint{3.373829in}{3.351718in}}%
\pgfpathlineto{\pgfqpoint{3.373829in}{3.347461in}}%
\pgfpathmoveto{\pgfqpoint{3.373829in}{3.334687in}}%
\pgfpathlineto{\pgfqpoint{3.373829in}{3.334687in}}%
\pgfpathlineto{\pgfqpoint{3.373829in}{3.338945in}}%
\pgfpathlineto{\pgfqpoint{3.378087in}{3.338945in}}%
\pgfpathlineto{\pgfqpoint{3.378087in}{3.334687in}}%
\pgfpathmoveto{\pgfqpoint{3.373829in}{3.338945in}}%
\pgfpathlineto{\pgfqpoint{3.373829in}{3.338945in}}%
\pgfpathlineto{\pgfqpoint{3.373829in}{3.343203in}}%
\pgfpathlineto{\pgfqpoint{3.378087in}{3.343203in}}%
\pgfpathlineto{\pgfqpoint{3.378087in}{3.338945in}}%
\pgfpathmoveto{\pgfqpoint{3.378087in}{3.338945in}}%
\pgfpathlineto{\pgfqpoint{3.378087in}{3.338945in}}%
\pgfpathlineto{\pgfqpoint{3.378087in}{3.343203in}}%
\pgfpathlineto{\pgfqpoint{3.382345in}{3.343203in}}%
\pgfpathlineto{\pgfqpoint{3.382345in}{3.338945in}}%
\pgfpathmoveto{\pgfqpoint{3.373829in}{3.343203in}}%
\pgfpathlineto{\pgfqpoint{3.373829in}{3.343203in}}%
\pgfpathlineto{\pgfqpoint{3.373829in}{3.347461in}}%
\pgfpathlineto{\pgfqpoint{3.378087in}{3.347461in}}%
\pgfpathlineto{\pgfqpoint{3.378087in}{3.343203in}}%
\pgfpathmoveto{\pgfqpoint{3.373829in}{3.347461in}}%
\pgfpathlineto{\pgfqpoint{3.373829in}{3.347461in}}%
\pgfpathlineto{\pgfqpoint{3.373829in}{3.351718in}}%
\pgfpathlineto{\pgfqpoint{3.378087in}{3.351718in}}%
\pgfpathlineto{\pgfqpoint{3.378087in}{3.347461in}}%
\pgfpathmoveto{\pgfqpoint{3.378087in}{3.343203in}}%
\pgfpathlineto{\pgfqpoint{3.378087in}{3.343203in}}%
\pgfpathlineto{\pgfqpoint{3.378087in}{3.347461in}}%
\pgfpathlineto{\pgfqpoint{3.382345in}{3.347461in}}%
\pgfpathlineto{\pgfqpoint{3.382345in}{3.343203in}}%
\pgfpathmoveto{\pgfqpoint{3.378087in}{3.347461in}}%
\pgfpathlineto{\pgfqpoint{3.378087in}{3.347461in}}%
\pgfpathlineto{\pgfqpoint{3.378087in}{3.351718in}}%
\pgfpathlineto{\pgfqpoint{3.382345in}{3.351718in}}%
\pgfpathlineto{\pgfqpoint{3.382345in}{3.347461in}}%
\pgfpathmoveto{\pgfqpoint{3.365314in}{3.351718in}}%
\pgfpathlineto{\pgfqpoint{3.365314in}{3.351718in}}%
\pgfpathlineto{\pgfqpoint{3.365314in}{3.355976in}}%
\pgfpathlineto{\pgfqpoint{3.369572in}{3.355976in}}%
\pgfpathlineto{\pgfqpoint{3.369572in}{3.351718in}}%
\pgfpathmoveto{\pgfqpoint{3.369572in}{3.351718in}}%
\pgfpathlineto{\pgfqpoint{3.369572in}{3.351718in}}%
\pgfpathlineto{\pgfqpoint{3.369572in}{3.355976in}}%
\pgfpathlineto{\pgfqpoint{3.373829in}{3.355976in}}%
\pgfpathlineto{\pgfqpoint{3.373829in}{3.351718in}}%
\pgfpathmoveto{\pgfqpoint{3.369572in}{3.355976in}}%
\pgfpathlineto{\pgfqpoint{3.369572in}{3.355976in}}%
\pgfpathlineto{\pgfqpoint{3.369572in}{3.360234in}}%
\pgfpathlineto{\pgfqpoint{3.373829in}{3.360234in}}%
\pgfpathlineto{\pgfqpoint{3.373829in}{3.355976in}}%
\pgfpathmoveto{\pgfqpoint{3.373829in}{3.351718in}}%
\pgfpathlineto{\pgfqpoint{3.373829in}{3.351718in}}%
\pgfpathlineto{\pgfqpoint{3.373829in}{3.355976in}}%
\pgfpathlineto{\pgfqpoint{3.378087in}{3.355976in}}%
\pgfpathlineto{\pgfqpoint{3.378087in}{3.351718in}}%
\pgfpathmoveto{\pgfqpoint{3.373829in}{3.355976in}}%
\pgfpathlineto{\pgfqpoint{3.373829in}{3.355976in}}%
\pgfpathlineto{\pgfqpoint{3.373829in}{3.360234in}}%
\pgfpathlineto{\pgfqpoint{3.378087in}{3.360234in}}%
\pgfpathlineto{\pgfqpoint{3.378087in}{3.355976in}}%
\pgfpathmoveto{\pgfqpoint{3.378087in}{3.351718in}}%
\pgfpathlineto{\pgfqpoint{3.378087in}{3.351718in}}%
\pgfpathlineto{\pgfqpoint{3.378087in}{3.355976in}}%
\pgfpathlineto{\pgfqpoint{3.382345in}{3.355976in}}%
\pgfpathlineto{\pgfqpoint{3.382345in}{3.351718in}}%
\pgfpathmoveto{\pgfqpoint{3.378087in}{3.355976in}}%
\pgfpathlineto{\pgfqpoint{3.378087in}{3.355976in}}%
\pgfpathlineto{\pgfqpoint{3.378087in}{3.360234in}}%
\pgfpathlineto{\pgfqpoint{3.382345in}{3.360234in}}%
\pgfpathlineto{\pgfqpoint{3.382345in}{3.355976in}}%
\pgfpathmoveto{\pgfqpoint{3.373829in}{3.360234in}}%
\pgfpathlineto{\pgfqpoint{3.373829in}{3.360234in}}%
\pgfpathlineto{\pgfqpoint{3.373829in}{3.364492in}}%
\pgfpathlineto{\pgfqpoint{3.378087in}{3.364492in}}%
\pgfpathlineto{\pgfqpoint{3.378087in}{3.360234in}}%
\pgfpathmoveto{\pgfqpoint{3.378087in}{3.360234in}}%
\pgfpathlineto{\pgfqpoint{3.378087in}{3.360234in}}%
\pgfpathlineto{\pgfqpoint{3.378087in}{3.364492in}}%
\pgfpathlineto{\pgfqpoint{3.382345in}{3.364492in}}%
\pgfpathlineto{\pgfqpoint{3.382345in}{3.360234in}}%
\pgfpathmoveto{\pgfqpoint{3.378087in}{3.364492in}}%
\pgfpathlineto{\pgfqpoint{3.378087in}{3.364492in}}%
\pgfpathlineto{\pgfqpoint{3.378087in}{3.368750in}}%
\pgfpathlineto{\pgfqpoint{3.382345in}{3.368750in}}%
\pgfpathlineto{\pgfqpoint{3.382345in}{3.364492in}}%
\pgfpathmoveto{\pgfqpoint{3.382345in}{3.343203in}}%
\pgfpathlineto{\pgfqpoint{3.382345in}{3.343203in}}%
\pgfpathlineto{\pgfqpoint{3.382345in}{3.347461in}}%
\pgfpathlineto{\pgfqpoint{3.386603in}{3.347461in}}%
\pgfpathlineto{\pgfqpoint{3.386603in}{3.343203in}}%
\pgfpathmoveto{\pgfqpoint{3.382345in}{3.347461in}}%
\pgfpathlineto{\pgfqpoint{3.382345in}{3.347461in}}%
\pgfpathlineto{\pgfqpoint{3.382345in}{3.351718in}}%
\pgfpathlineto{\pgfqpoint{3.386603in}{3.351718in}}%
\pgfpathlineto{\pgfqpoint{3.386603in}{3.347461in}}%
\pgfpathmoveto{\pgfqpoint{3.386603in}{3.347461in}}%
\pgfpathlineto{\pgfqpoint{3.386603in}{3.347461in}}%
\pgfpathlineto{\pgfqpoint{3.386603in}{3.351718in}}%
\pgfpathlineto{\pgfqpoint{3.390860in}{3.351718in}}%
\pgfpathlineto{\pgfqpoint{3.390860in}{3.347461in}}%
\pgfpathmoveto{\pgfqpoint{3.382345in}{3.351718in}}%
\pgfpathlineto{\pgfqpoint{3.382345in}{3.351718in}}%
\pgfpathlineto{\pgfqpoint{3.382345in}{3.355976in}}%
\pgfpathlineto{\pgfqpoint{3.386603in}{3.355976in}}%
\pgfpathlineto{\pgfqpoint{3.386603in}{3.351718in}}%
\pgfpathmoveto{\pgfqpoint{3.382345in}{3.355976in}}%
\pgfpathlineto{\pgfqpoint{3.382345in}{3.355976in}}%
\pgfpathlineto{\pgfqpoint{3.382345in}{3.360234in}}%
\pgfpathlineto{\pgfqpoint{3.386603in}{3.360234in}}%
\pgfpathlineto{\pgfqpoint{3.386603in}{3.355976in}}%
\pgfpathmoveto{\pgfqpoint{3.386603in}{3.351718in}}%
\pgfpathlineto{\pgfqpoint{3.386603in}{3.351718in}}%
\pgfpathlineto{\pgfqpoint{3.386603in}{3.355976in}}%
\pgfpathlineto{\pgfqpoint{3.390860in}{3.355976in}}%
\pgfpathlineto{\pgfqpoint{3.390860in}{3.351718in}}%
\pgfpathmoveto{\pgfqpoint{3.386603in}{3.355976in}}%
\pgfpathlineto{\pgfqpoint{3.386603in}{3.355976in}}%
\pgfpathlineto{\pgfqpoint{3.386603in}{3.360234in}}%
\pgfpathlineto{\pgfqpoint{3.390860in}{3.360234in}}%
\pgfpathlineto{\pgfqpoint{3.390860in}{3.355976in}}%
\pgfpathmoveto{\pgfqpoint{3.382345in}{3.360234in}}%
\pgfpathlineto{\pgfqpoint{3.382345in}{3.360234in}}%
\pgfpathlineto{\pgfqpoint{3.382345in}{3.364492in}}%
\pgfpathlineto{\pgfqpoint{3.386603in}{3.364492in}}%
\pgfpathlineto{\pgfqpoint{3.386603in}{3.360234in}}%
\pgfpathmoveto{\pgfqpoint{3.382345in}{3.364492in}}%
\pgfpathlineto{\pgfqpoint{3.382345in}{3.364492in}}%
\pgfpathlineto{\pgfqpoint{3.382345in}{3.368750in}}%
\pgfpathlineto{\pgfqpoint{3.386603in}{3.368750in}}%
\pgfpathlineto{\pgfqpoint{3.386603in}{3.364492in}}%
\pgfpathmoveto{\pgfqpoint{3.386603in}{3.360234in}}%
\pgfpathlineto{\pgfqpoint{3.386603in}{3.360234in}}%
\pgfpathlineto{\pgfqpoint{3.386603in}{3.364492in}}%
\pgfpathlineto{\pgfqpoint{3.390860in}{3.364492in}}%
\pgfpathlineto{\pgfqpoint{3.390860in}{3.360234in}}%
\pgfpathmoveto{\pgfqpoint{3.386603in}{3.364492in}}%
\pgfpathlineto{\pgfqpoint{3.386603in}{3.364492in}}%
\pgfpathlineto{\pgfqpoint{3.386603in}{3.368750in}}%
\pgfpathlineto{\pgfqpoint{3.390860in}{3.368750in}}%
\pgfpathlineto{\pgfqpoint{3.390860in}{3.364492in}}%
\pgfpathmoveto{\pgfqpoint{3.390860in}{3.351718in}}%
\pgfpathlineto{\pgfqpoint{3.390860in}{3.351718in}}%
\pgfpathlineto{\pgfqpoint{3.390860in}{3.355976in}}%
\pgfpathlineto{\pgfqpoint{3.395118in}{3.355976in}}%
\pgfpathlineto{\pgfqpoint{3.395118in}{3.351718in}}%
\pgfpathmoveto{\pgfqpoint{3.390860in}{3.355976in}}%
\pgfpathlineto{\pgfqpoint{3.390860in}{3.355976in}}%
\pgfpathlineto{\pgfqpoint{3.390860in}{3.360234in}}%
\pgfpathlineto{\pgfqpoint{3.395118in}{3.360234in}}%
\pgfpathlineto{\pgfqpoint{3.395118in}{3.355976in}}%
\pgfpathmoveto{\pgfqpoint{3.395118in}{3.355976in}}%
\pgfpathlineto{\pgfqpoint{3.395118in}{3.355976in}}%
\pgfpathlineto{\pgfqpoint{3.395118in}{3.360234in}}%
\pgfpathlineto{\pgfqpoint{3.399376in}{3.360234in}}%
\pgfpathlineto{\pgfqpoint{3.399376in}{3.355976in}}%
\pgfpathmoveto{\pgfqpoint{3.390860in}{3.360234in}}%
\pgfpathlineto{\pgfqpoint{3.390860in}{3.360234in}}%
\pgfpathlineto{\pgfqpoint{3.390860in}{3.364492in}}%
\pgfpathlineto{\pgfqpoint{3.395118in}{3.364492in}}%
\pgfpathlineto{\pgfqpoint{3.395118in}{3.360234in}}%
\pgfpathmoveto{\pgfqpoint{3.390860in}{3.364492in}}%
\pgfpathlineto{\pgfqpoint{3.390860in}{3.364492in}}%
\pgfpathlineto{\pgfqpoint{3.390860in}{3.368750in}}%
\pgfpathlineto{\pgfqpoint{3.395118in}{3.368750in}}%
\pgfpathlineto{\pgfqpoint{3.395118in}{3.364492in}}%
\pgfpathmoveto{\pgfqpoint{3.395118in}{3.360234in}}%
\pgfpathlineto{\pgfqpoint{3.395118in}{3.360234in}}%
\pgfpathlineto{\pgfqpoint{3.395118in}{3.364492in}}%
\pgfpathlineto{\pgfqpoint{3.399376in}{3.364492in}}%
\pgfpathlineto{\pgfqpoint{3.399376in}{3.360234in}}%
\pgfpathmoveto{\pgfqpoint{3.395118in}{3.364492in}}%
\pgfpathlineto{\pgfqpoint{3.395118in}{3.364492in}}%
\pgfpathlineto{\pgfqpoint{3.395118in}{3.368750in}}%
\pgfpathlineto{\pgfqpoint{3.399376in}{3.368750in}}%
\pgfpathlineto{\pgfqpoint{3.399376in}{3.364492in}}%
\pgfpathmoveto{\pgfqpoint{3.399376in}{3.360234in}}%
\pgfpathlineto{\pgfqpoint{3.399376in}{3.360234in}}%
\pgfpathlineto{\pgfqpoint{3.399376in}{3.364492in}}%
\pgfpathlineto{\pgfqpoint{3.403633in}{3.364492in}}%
\pgfpathlineto{\pgfqpoint{3.403633in}{3.360234in}}%
\pgfpathmoveto{\pgfqpoint{3.399376in}{3.364492in}}%
\pgfpathlineto{\pgfqpoint{3.399376in}{3.364492in}}%
\pgfpathlineto{\pgfqpoint{3.399376in}{3.368750in}}%
\pgfpathlineto{\pgfqpoint{3.403633in}{3.368750in}}%
\pgfpathlineto{\pgfqpoint{3.403633in}{3.364492in}}%
\pgfpathmoveto{\pgfqpoint{3.403633in}{3.364492in}}%
\pgfpathlineto{\pgfqpoint{3.403633in}{3.364492in}}%
\pgfpathlineto{\pgfqpoint{3.403633in}{3.368750in}}%
\pgfpathlineto{\pgfqpoint{3.407891in}{3.368750in}}%
\pgfpathlineto{\pgfqpoint{3.407891in}{3.364492in}}%
\pgfpathmoveto{\pgfqpoint{3.407891in}{3.364492in}}%
\pgfpathlineto{\pgfqpoint{3.407891in}{3.364492in}}%
\pgfpathlineto{\pgfqpoint{3.407891in}{3.368750in}}%
\pgfpathlineto{\pgfqpoint{3.412149in}{3.368750in}}%
\pgfpathlineto{\pgfqpoint{3.412149in}{3.364492in}}%
\pgfpathmoveto{\pgfqpoint{3.480272in}{3.364492in}}%
\pgfpathlineto{\pgfqpoint{3.480272in}{3.364492in}}%
\pgfpathlineto{\pgfqpoint{3.480272in}{3.368750in}}%
\pgfpathlineto{\pgfqpoint{3.484530in}{3.368750in}}%
\pgfpathlineto{\pgfqpoint{3.484530in}{3.364492in}}%
\pgfpathmoveto{\pgfqpoint{3.488787in}{3.355976in}}%
\pgfpathlineto{\pgfqpoint{3.488787in}{3.355976in}}%
\pgfpathlineto{\pgfqpoint{3.488787in}{3.360234in}}%
\pgfpathlineto{\pgfqpoint{3.493045in}{3.360234in}}%
\pgfpathlineto{\pgfqpoint{3.493045in}{3.355976in}}%
\pgfpathmoveto{\pgfqpoint{3.484530in}{3.360234in}}%
\pgfpathlineto{\pgfqpoint{3.484530in}{3.360234in}}%
\pgfpathlineto{\pgfqpoint{3.484530in}{3.364492in}}%
\pgfpathlineto{\pgfqpoint{3.488787in}{3.364492in}}%
\pgfpathlineto{\pgfqpoint{3.488787in}{3.360234in}}%
\pgfpathmoveto{\pgfqpoint{3.484530in}{3.364492in}}%
\pgfpathlineto{\pgfqpoint{3.484530in}{3.364492in}}%
\pgfpathlineto{\pgfqpoint{3.484530in}{3.368750in}}%
\pgfpathlineto{\pgfqpoint{3.488787in}{3.368750in}}%
\pgfpathlineto{\pgfqpoint{3.488787in}{3.364492in}}%
\pgfpathmoveto{\pgfqpoint{3.488787in}{3.360234in}}%
\pgfpathlineto{\pgfqpoint{3.488787in}{3.360234in}}%
\pgfpathlineto{\pgfqpoint{3.488787in}{3.364492in}}%
\pgfpathlineto{\pgfqpoint{3.493045in}{3.364492in}}%
\pgfpathlineto{\pgfqpoint{3.493045in}{3.360234in}}%
\pgfpathmoveto{\pgfqpoint{3.488787in}{3.364492in}}%
\pgfpathlineto{\pgfqpoint{3.488787in}{3.364492in}}%
\pgfpathlineto{\pgfqpoint{3.488787in}{3.368750in}}%
\pgfpathlineto{\pgfqpoint{3.493045in}{3.368750in}}%
\pgfpathlineto{\pgfqpoint{3.493045in}{3.364492in}}%
\pgfpathmoveto{\pgfqpoint{3.493045in}{3.351718in}}%
\pgfpathlineto{\pgfqpoint{3.493045in}{3.351718in}}%
\pgfpathlineto{\pgfqpoint{3.493045in}{3.355976in}}%
\pgfpathlineto{\pgfqpoint{3.497303in}{3.355976in}}%
\pgfpathlineto{\pgfqpoint{3.497303in}{3.351718in}}%
\pgfpathmoveto{\pgfqpoint{3.493045in}{3.355976in}}%
\pgfpathlineto{\pgfqpoint{3.493045in}{3.355976in}}%
\pgfpathlineto{\pgfqpoint{3.493045in}{3.360234in}}%
\pgfpathlineto{\pgfqpoint{3.497303in}{3.360234in}}%
\pgfpathlineto{\pgfqpoint{3.497303in}{3.355976in}}%
\pgfpathmoveto{\pgfqpoint{3.497303in}{3.351718in}}%
\pgfpathlineto{\pgfqpoint{3.497303in}{3.351718in}}%
\pgfpathlineto{\pgfqpoint{3.497303in}{3.355976in}}%
\pgfpathlineto{\pgfqpoint{3.501560in}{3.355976in}}%
\pgfpathlineto{\pgfqpoint{3.501560in}{3.351718in}}%
\pgfpathmoveto{\pgfqpoint{3.497303in}{3.355976in}}%
\pgfpathlineto{\pgfqpoint{3.497303in}{3.355976in}}%
\pgfpathlineto{\pgfqpoint{3.497303in}{3.360234in}}%
\pgfpathlineto{\pgfqpoint{3.501560in}{3.360234in}}%
\pgfpathlineto{\pgfqpoint{3.501560in}{3.355976in}}%
\pgfpathmoveto{\pgfqpoint{3.493045in}{3.360234in}}%
\pgfpathlineto{\pgfqpoint{3.493045in}{3.360234in}}%
\pgfpathlineto{\pgfqpoint{3.493045in}{3.364492in}}%
\pgfpathlineto{\pgfqpoint{3.497303in}{3.364492in}}%
\pgfpathlineto{\pgfqpoint{3.497303in}{3.360234in}}%
\pgfpathmoveto{\pgfqpoint{3.493045in}{3.364492in}}%
\pgfpathlineto{\pgfqpoint{3.493045in}{3.364492in}}%
\pgfpathlineto{\pgfqpoint{3.493045in}{3.368750in}}%
\pgfpathlineto{\pgfqpoint{3.497303in}{3.368750in}}%
\pgfpathlineto{\pgfqpoint{3.497303in}{3.364492in}}%
\pgfpathmoveto{\pgfqpoint{3.497303in}{3.360234in}}%
\pgfpathlineto{\pgfqpoint{3.497303in}{3.360234in}}%
\pgfpathlineto{\pgfqpoint{3.497303in}{3.364492in}}%
\pgfpathlineto{\pgfqpoint{3.501560in}{3.364492in}}%
\pgfpathlineto{\pgfqpoint{3.501560in}{3.360234in}}%
\pgfpathmoveto{\pgfqpoint{3.497303in}{3.364492in}}%
\pgfpathlineto{\pgfqpoint{3.497303in}{3.364492in}}%
\pgfpathlineto{\pgfqpoint{3.497303in}{3.368750in}}%
\pgfpathlineto{\pgfqpoint{3.501560in}{3.368750in}}%
\pgfpathlineto{\pgfqpoint{3.501560in}{3.364492in}}%
\pgfpathmoveto{\pgfqpoint{3.378087in}{3.368750in}}%
\pgfpathlineto{\pgfqpoint{3.378087in}{3.368750in}}%
\pgfpathlineto{\pgfqpoint{3.378087in}{3.373007in}}%
\pgfpathlineto{\pgfqpoint{3.382345in}{3.373007in}}%
\pgfpathlineto{\pgfqpoint{3.382345in}{3.368750in}}%
\pgfpathmoveto{\pgfqpoint{3.382345in}{3.368750in}}%
\pgfpathlineto{\pgfqpoint{3.382345in}{3.368750in}}%
\pgfpathlineto{\pgfqpoint{3.382345in}{3.373007in}}%
\pgfpathlineto{\pgfqpoint{3.386603in}{3.373007in}}%
\pgfpathlineto{\pgfqpoint{3.386603in}{3.368750in}}%
\pgfpathmoveto{\pgfqpoint{3.382345in}{3.373007in}}%
\pgfpathlineto{\pgfqpoint{3.382345in}{3.373007in}}%
\pgfpathlineto{\pgfqpoint{3.382345in}{3.377265in}}%
\pgfpathlineto{\pgfqpoint{3.386603in}{3.377265in}}%
\pgfpathlineto{\pgfqpoint{3.386603in}{3.373007in}}%
\pgfpathmoveto{\pgfqpoint{3.386603in}{3.368750in}}%
\pgfpathlineto{\pgfqpoint{3.386603in}{3.368750in}}%
\pgfpathlineto{\pgfqpoint{3.386603in}{3.373007in}}%
\pgfpathlineto{\pgfqpoint{3.390860in}{3.373007in}}%
\pgfpathlineto{\pgfqpoint{3.390860in}{3.368750in}}%
\pgfpathmoveto{\pgfqpoint{3.386603in}{3.373007in}}%
\pgfpathlineto{\pgfqpoint{3.386603in}{3.373007in}}%
\pgfpathlineto{\pgfqpoint{3.386603in}{3.377265in}}%
\pgfpathlineto{\pgfqpoint{3.390860in}{3.377265in}}%
\pgfpathlineto{\pgfqpoint{3.390860in}{3.373007in}}%
\pgfpathmoveto{\pgfqpoint{3.386603in}{3.377265in}}%
\pgfpathlineto{\pgfqpoint{3.386603in}{3.377265in}}%
\pgfpathlineto{\pgfqpoint{3.386603in}{3.381523in}}%
\pgfpathlineto{\pgfqpoint{3.390860in}{3.381523in}}%
\pgfpathlineto{\pgfqpoint{3.390860in}{3.377265in}}%
\pgfpathmoveto{\pgfqpoint{3.390860in}{3.368750in}}%
\pgfpathlineto{\pgfqpoint{3.390860in}{3.368750in}}%
\pgfpathlineto{\pgfqpoint{3.390860in}{3.373007in}}%
\pgfpathlineto{\pgfqpoint{3.395118in}{3.373007in}}%
\pgfpathlineto{\pgfqpoint{3.395118in}{3.368750in}}%
\pgfpathmoveto{\pgfqpoint{3.390860in}{3.373007in}}%
\pgfpathlineto{\pgfqpoint{3.390860in}{3.373007in}}%
\pgfpathlineto{\pgfqpoint{3.390860in}{3.377265in}}%
\pgfpathlineto{\pgfqpoint{3.395118in}{3.377265in}}%
\pgfpathlineto{\pgfqpoint{3.395118in}{3.373007in}}%
\pgfpathmoveto{\pgfqpoint{3.395118in}{3.368750in}}%
\pgfpathlineto{\pgfqpoint{3.395118in}{3.368750in}}%
\pgfpathlineto{\pgfqpoint{3.395118in}{3.373007in}}%
\pgfpathlineto{\pgfqpoint{3.399376in}{3.373007in}}%
\pgfpathlineto{\pgfqpoint{3.399376in}{3.368750in}}%
\pgfpathmoveto{\pgfqpoint{3.395118in}{3.373007in}}%
\pgfpathlineto{\pgfqpoint{3.395118in}{3.373007in}}%
\pgfpathlineto{\pgfqpoint{3.395118in}{3.377265in}}%
\pgfpathlineto{\pgfqpoint{3.399376in}{3.377265in}}%
\pgfpathlineto{\pgfqpoint{3.399376in}{3.373007in}}%
\pgfpathmoveto{\pgfqpoint{3.390860in}{3.377265in}}%
\pgfpathlineto{\pgfqpoint{3.390860in}{3.377265in}}%
\pgfpathlineto{\pgfqpoint{3.390860in}{3.381523in}}%
\pgfpathlineto{\pgfqpoint{3.395118in}{3.381523in}}%
\pgfpathlineto{\pgfqpoint{3.395118in}{3.377265in}}%
\pgfpathmoveto{\pgfqpoint{3.390860in}{3.381523in}}%
\pgfpathlineto{\pgfqpoint{3.390860in}{3.381523in}}%
\pgfpathlineto{\pgfqpoint{3.390860in}{3.385781in}}%
\pgfpathlineto{\pgfqpoint{3.395118in}{3.385781in}}%
\pgfpathlineto{\pgfqpoint{3.395118in}{3.381523in}}%
\pgfpathmoveto{\pgfqpoint{3.395118in}{3.377265in}}%
\pgfpathlineto{\pgfqpoint{3.395118in}{3.377265in}}%
\pgfpathlineto{\pgfqpoint{3.395118in}{3.381523in}}%
\pgfpathlineto{\pgfqpoint{3.399376in}{3.381523in}}%
\pgfpathlineto{\pgfqpoint{3.399376in}{3.377265in}}%
\pgfpathmoveto{\pgfqpoint{3.395118in}{3.381523in}}%
\pgfpathlineto{\pgfqpoint{3.395118in}{3.381523in}}%
\pgfpathlineto{\pgfqpoint{3.395118in}{3.385781in}}%
\pgfpathlineto{\pgfqpoint{3.399376in}{3.385781in}}%
\pgfpathlineto{\pgfqpoint{3.399376in}{3.381523in}}%
\pgfpathmoveto{\pgfqpoint{3.395118in}{3.385781in}}%
\pgfpathlineto{\pgfqpoint{3.395118in}{3.385781in}}%
\pgfpathlineto{\pgfqpoint{3.395118in}{3.390039in}}%
\pgfpathlineto{\pgfqpoint{3.399376in}{3.390039in}}%
\pgfpathlineto{\pgfqpoint{3.399376in}{3.385781in}}%
\pgfpathmoveto{\pgfqpoint{3.399376in}{3.368750in}}%
\pgfpathlineto{\pgfqpoint{3.399376in}{3.368750in}}%
\pgfpathlineto{\pgfqpoint{3.399376in}{3.373007in}}%
\pgfpathlineto{\pgfqpoint{3.403633in}{3.373007in}}%
\pgfpathlineto{\pgfqpoint{3.403633in}{3.368750in}}%
\pgfpathmoveto{\pgfqpoint{3.399376in}{3.373007in}}%
\pgfpathlineto{\pgfqpoint{3.399376in}{3.373007in}}%
\pgfpathlineto{\pgfqpoint{3.399376in}{3.377265in}}%
\pgfpathlineto{\pgfqpoint{3.403633in}{3.377265in}}%
\pgfpathlineto{\pgfqpoint{3.403633in}{3.373007in}}%
\pgfpathmoveto{\pgfqpoint{3.403633in}{3.368750in}}%
\pgfpathlineto{\pgfqpoint{3.403633in}{3.368750in}}%
\pgfpathlineto{\pgfqpoint{3.403633in}{3.373007in}}%
\pgfpathlineto{\pgfqpoint{3.407891in}{3.373007in}}%
\pgfpathlineto{\pgfqpoint{3.407891in}{3.368750in}}%
\pgfpathmoveto{\pgfqpoint{3.403633in}{3.373007in}}%
\pgfpathlineto{\pgfqpoint{3.403633in}{3.373007in}}%
\pgfpathlineto{\pgfqpoint{3.403633in}{3.377265in}}%
\pgfpathlineto{\pgfqpoint{3.407891in}{3.377265in}}%
\pgfpathlineto{\pgfqpoint{3.407891in}{3.373007in}}%
\pgfpathmoveto{\pgfqpoint{3.399376in}{3.377265in}}%
\pgfpathlineto{\pgfqpoint{3.399376in}{3.377265in}}%
\pgfpathlineto{\pgfqpoint{3.399376in}{3.381523in}}%
\pgfpathlineto{\pgfqpoint{3.403633in}{3.381523in}}%
\pgfpathlineto{\pgfqpoint{3.403633in}{3.377265in}}%
\pgfpathmoveto{\pgfqpoint{3.399376in}{3.381523in}}%
\pgfpathlineto{\pgfqpoint{3.399376in}{3.381523in}}%
\pgfpathlineto{\pgfqpoint{3.399376in}{3.385781in}}%
\pgfpathlineto{\pgfqpoint{3.403633in}{3.385781in}}%
\pgfpathlineto{\pgfqpoint{3.403633in}{3.381523in}}%
\pgfpathmoveto{\pgfqpoint{3.403633in}{3.377265in}}%
\pgfpathlineto{\pgfqpoint{3.403633in}{3.377265in}}%
\pgfpathlineto{\pgfqpoint{3.403633in}{3.381523in}}%
\pgfpathlineto{\pgfqpoint{3.407891in}{3.381523in}}%
\pgfpathlineto{\pgfqpoint{3.407891in}{3.377265in}}%
\pgfpathmoveto{\pgfqpoint{3.403633in}{3.381523in}}%
\pgfpathlineto{\pgfqpoint{3.403633in}{3.381523in}}%
\pgfpathlineto{\pgfqpoint{3.403633in}{3.385781in}}%
\pgfpathlineto{\pgfqpoint{3.407891in}{3.385781in}}%
\pgfpathlineto{\pgfqpoint{3.407891in}{3.381523in}}%
\pgfpathmoveto{\pgfqpoint{3.407891in}{3.368750in}}%
\pgfpathlineto{\pgfqpoint{3.407891in}{3.368750in}}%
\pgfpathlineto{\pgfqpoint{3.407891in}{3.373007in}}%
\pgfpathlineto{\pgfqpoint{3.412149in}{3.373007in}}%
\pgfpathlineto{\pgfqpoint{3.412149in}{3.368750in}}%
\pgfpathmoveto{\pgfqpoint{3.407891in}{3.373007in}}%
\pgfpathlineto{\pgfqpoint{3.407891in}{3.373007in}}%
\pgfpathlineto{\pgfqpoint{3.407891in}{3.377265in}}%
\pgfpathlineto{\pgfqpoint{3.412149in}{3.377265in}}%
\pgfpathlineto{\pgfqpoint{3.412149in}{3.373007in}}%
\pgfpathmoveto{\pgfqpoint{3.412149in}{3.368750in}}%
\pgfpathlineto{\pgfqpoint{3.412149in}{3.368750in}}%
\pgfpathlineto{\pgfqpoint{3.412149in}{3.373007in}}%
\pgfpathlineto{\pgfqpoint{3.416406in}{3.373007in}}%
\pgfpathlineto{\pgfqpoint{3.416406in}{3.368750in}}%
\pgfpathmoveto{\pgfqpoint{3.412149in}{3.373007in}}%
\pgfpathlineto{\pgfqpoint{3.412149in}{3.373007in}}%
\pgfpathlineto{\pgfqpoint{3.412149in}{3.377265in}}%
\pgfpathlineto{\pgfqpoint{3.416406in}{3.377265in}}%
\pgfpathlineto{\pgfqpoint{3.416406in}{3.373007in}}%
\pgfpathmoveto{\pgfqpoint{3.407891in}{3.377265in}}%
\pgfpathlineto{\pgfqpoint{3.407891in}{3.377265in}}%
\pgfpathlineto{\pgfqpoint{3.407891in}{3.381523in}}%
\pgfpathlineto{\pgfqpoint{3.412149in}{3.381523in}}%
\pgfpathlineto{\pgfqpoint{3.412149in}{3.377265in}}%
\pgfpathmoveto{\pgfqpoint{3.407891in}{3.381523in}}%
\pgfpathlineto{\pgfqpoint{3.407891in}{3.381523in}}%
\pgfpathlineto{\pgfqpoint{3.407891in}{3.385781in}}%
\pgfpathlineto{\pgfqpoint{3.412149in}{3.385781in}}%
\pgfpathlineto{\pgfqpoint{3.412149in}{3.381523in}}%
\pgfpathmoveto{\pgfqpoint{3.412149in}{3.377265in}}%
\pgfpathlineto{\pgfqpoint{3.412149in}{3.377265in}}%
\pgfpathlineto{\pgfqpoint{3.412149in}{3.381523in}}%
\pgfpathlineto{\pgfqpoint{3.416406in}{3.381523in}}%
\pgfpathlineto{\pgfqpoint{3.416406in}{3.377265in}}%
\pgfpathmoveto{\pgfqpoint{3.412149in}{3.381523in}}%
\pgfpathlineto{\pgfqpoint{3.412149in}{3.381523in}}%
\pgfpathlineto{\pgfqpoint{3.412149in}{3.385781in}}%
\pgfpathlineto{\pgfqpoint{3.416406in}{3.385781in}}%
\pgfpathlineto{\pgfqpoint{3.416406in}{3.381523in}}%
\pgfpathmoveto{\pgfqpoint{3.399376in}{3.385781in}}%
\pgfpathlineto{\pgfqpoint{3.399376in}{3.385781in}}%
\pgfpathlineto{\pgfqpoint{3.399376in}{3.390039in}}%
\pgfpathlineto{\pgfqpoint{3.403633in}{3.390039in}}%
\pgfpathlineto{\pgfqpoint{3.403633in}{3.385781in}}%
\pgfpathmoveto{\pgfqpoint{3.399376in}{3.390039in}}%
\pgfpathlineto{\pgfqpoint{3.399376in}{3.390039in}}%
\pgfpathlineto{\pgfqpoint{3.399376in}{3.394297in}}%
\pgfpathlineto{\pgfqpoint{3.403633in}{3.394297in}}%
\pgfpathlineto{\pgfqpoint{3.403633in}{3.390039in}}%
\pgfpathmoveto{\pgfqpoint{3.403633in}{3.385781in}}%
\pgfpathlineto{\pgfqpoint{3.403633in}{3.385781in}}%
\pgfpathlineto{\pgfqpoint{3.403633in}{3.390039in}}%
\pgfpathlineto{\pgfqpoint{3.407891in}{3.390039in}}%
\pgfpathlineto{\pgfqpoint{3.407891in}{3.385781in}}%
\pgfpathmoveto{\pgfqpoint{3.403633in}{3.390039in}}%
\pgfpathlineto{\pgfqpoint{3.403633in}{3.390039in}}%
\pgfpathlineto{\pgfqpoint{3.403633in}{3.394297in}}%
\pgfpathlineto{\pgfqpoint{3.407891in}{3.394297in}}%
\pgfpathlineto{\pgfqpoint{3.407891in}{3.390039in}}%
\pgfpathmoveto{\pgfqpoint{3.403633in}{3.394297in}}%
\pgfpathlineto{\pgfqpoint{3.403633in}{3.394297in}}%
\pgfpathlineto{\pgfqpoint{3.403633in}{3.398555in}}%
\pgfpathlineto{\pgfqpoint{3.407891in}{3.398555in}}%
\pgfpathlineto{\pgfqpoint{3.407891in}{3.394297in}}%
\pgfpathmoveto{\pgfqpoint{3.407891in}{3.385781in}}%
\pgfpathlineto{\pgfqpoint{3.407891in}{3.385781in}}%
\pgfpathlineto{\pgfqpoint{3.407891in}{3.390039in}}%
\pgfpathlineto{\pgfqpoint{3.412149in}{3.390039in}}%
\pgfpathlineto{\pgfqpoint{3.412149in}{3.385781in}}%
\pgfpathmoveto{\pgfqpoint{3.407891in}{3.390039in}}%
\pgfpathlineto{\pgfqpoint{3.407891in}{3.390039in}}%
\pgfpathlineto{\pgfqpoint{3.407891in}{3.394297in}}%
\pgfpathlineto{\pgfqpoint{3.412149in}{3.394297in}}%
\pgfpathlineto{\pgfqpoint{3.412149in}{3.390039in}}%
\pgfpathmoveto{\pgfqpoint{3.412149in}{3.385781in}}%
\pgfpathlineto{\pgfqpoint{3.412149in}{3.385781in}}%
\pgfpathlineto{\pgfqpoint{3.412149in}{3.390039in}}%
\pgfpathlineto{\pgfqpoint{3.416406in}{3.390039in}}%
\pgfpathlineto{\pgfqpoint{3.416406in}{3.385781in}}%
\pgfpathmoveto{\pgfqpoint{3.412149in}{3.390039in}}%
\pgfpathlineto{\pgfqpoint{3.412149in}{3.390039in}}%
\pgfpathlineto{\pgfqpoint{3.412149in}{3.394297in}}%
\pgfpathlineto{\pgfqpoint{3.416406in}{3.394297in}}%
\pgfpathlineto{\pgfqpoint{3.416406in}{3.390039in}}%
\pgfpathmoveto{\pgfqpoint{3.407891in}{3.394297in}}%
\pgfpathlineto{\pgfqpoint{3.407891in}{3.394297in}}%
\pgfpathlineto{\pgfqpoint{3.407891in}{3.398555in}}%
\pgfpathlineto{\pgfqpoint{3.412149in}{3.398555in}}%
\pgfpathlineto{\pgfqpoint{3.412149in}{3.394297in}}%
\pgfpathmoveto{\pgfqpoint{3.407891in}{3.398555in}}%
\pgfpathlineto{\pgfqpoint{3.407891in}{3.398555in}}%
\pgfpathlineto{\pgfqpoint{3.407891in}{3.402813in}}%
\pgfpathlineto{\pgfqpoint{3.412149in}{3.402813in}}%
\pgfpathlineto{\pgfqpoint{3.412149in}{3.398555in}}%
\pgfpathmoveto{\pgfqpoint{3.412149in}{3.394297in}}%
\pgfpathlineto{\pgfqpoint{3.412149in}{3.394297in}}%
\pgfpathlineto{\pgfqpoint{3.412149in}{3.398555in}}%
\pgfpathlineto{\pgfqpoint{3.416406in}{3.398555in}}%
\pgfpathlineto{\pgfqpoint{3.416406in}{3.394297in}}%
\pgfpathmoveto{\pgfqpoint{3.412149in}{3.398555in}}%
\pgfpathlineto{\pgfqpoint{3.412149in}{3.398555in}}%
\pgfpathlineto{\pgfqpoint{3.412149in}{3.402813in}}%
\pgfpathlineto{\pgfqpoint{3.416406in}{3.402813in}}%
\pgfpathlineto{\pgfqpoint{3.416406in}{3.398555in}}%
\pgfpathmoveto{\pgfqpoint{3.416406in}{3.368750in}}%
\pgfpathlineto{\pgfqpoint{3.416406in}{3.368750in}}%
\pgfpathlineto{\pgfqpoint{3.416406in}{3.373007in}}%
\pgfpathlineto{\pgfqpoint{3.420664in}{3.373007in}}%
\pgfpathlineto{\pgfqpoint{3.420664in}{3.368750in}}%
\pgfpathmoveto{\pgfqpoint{3.416406in}{3.373007in}}%
\pgfpathlineto{\pgfqpoint{3.416406in}{3.373007in}}%
\pgfpathlineto{\pgfqpoint{3.416406in}{3.377265in}}%
\pgfpathlineto{\pgfqpoint{3.420664in}{3.377265in}}%
\pgfpathlineto{\pgfqpoint{3.420664in}{3.373007in}}%
\pgfpathmoveto{\pgfqpoint{3.420664in}{3.373007in}}%
\pgfpathlineto{\pgfqpoint{3.420664in}{3.373007in}}%
\pgfpathlineto{\pgfqpoint{3.420664in}{3.377265in}}%
\pgfpathlineto{\pgfqpoint{3.424922in}{3.377265in}}%
\pgfpathlineto{\pgfqpoint{3.424922in}{3.373007in}}%
\pgfpathmoveto{\pgfqpoint{3.416406in}{3.377265in}}%
\pgfpathlineto{\pgfqpoint{3.416406in}{3.377265in}}%
\pgfpathlineto{\pgfqpoint{3.416406in}{3.381523in}}%
\pgfpathlineto{\pgfqpoint{3.420664in}{3.381523in}}%
\pgfpathlineto{\pgfqpoint{3.420664in}{3.377265in}}%
\pgfpathmoveto{\pgfqpoint{3.416406in}{3.381523in}}%
\pgfpathlineto{\pgfqpoint{3.416406in}{3.381523in}}%
\pgfpathlineto{\pgfqpoint{3.416406in}{3.385781in}}%
\pgfpathlineto{\pgfqpoint{3.420664in}{3.385781in}}%
\pgfpathlineto{\pgfqpoint{3.420664in}{3.381523in}}%
\pgfpathmoveto{\pgfqpoint{3.420664in}{3.377265in}}%
\pgfpathlineto{\pgfqpoint{3.420664in}{3.377265in}}%
\pgfpathlineto{\pgfqpoint{3.420664in}{3.381523in}}%
\pgfpathlineto{\pgfqpoint{3.424922in}{3.381523in}}%
\pgfpathlineto{\pgfqpoint{3.424922in}{3.377265in}}%
\pgfpathmoveto{\pgfqpoint{3.420664in}{3.381523in}}%
\pgfpathlineto{\pgfqpoint{3.420664in}{3.381523in}}%
\pgfpathlineto{\pgfqpoint{3.420664in}{3.385781in}}%
\pgfpathlineto{\pgfqpoint{3.424922in}{3.385781in}}%
\pgfpathlineto{\pgfqpoint{3.424922in}{3.381523in}}%
\pgfpathmoveto{\pgfqpoint{3.424922in}{3.373007in}}%
\pgfpathlineto{\pgfqpoint{3.424922in}{3.373007in}}%
\pgfpathlineto{\pgfqpoint{3.424922in}{3.377265in}}%
\pgfpathlineto{\pgfqpoint{3.429180in}{3.377265in}}%
\pgfpathlineto{\pgfqpoint{3.429180in}{3.373007in}}%
\pgfpathmoveto{\pgfqpoint{3.424922in}{3.377265in}}%
\pgfpathlineto{\pgfqpoint{3.424922in}{3.377265in}}%
\pgfpathlineto{\pgfqpoint{3.424922in}{3.381523in}}%
\pgfpathlineto{\pgfqpoint{3.429180in}{3.381523in}}%
\pgfpathlineto{\pgfqpoint{3.429180in}{3.377265in}}%
\pgfpathmoveto{\pgfqpoint{3.424922in}{3.381523in}}%
\pgfpathlineto{\pgfqpoint{3.424922in}{3.381523in}}%
\pgfpathlineto{\pgfqpoint{3.424922in}{3.385781in}}%
\pgfpathlineto{\pgfqpoint{3.429180in}{3.385781in}}%
\pgfpathlineto{\pgfqpoint{3.429180in}{3.381523in}}%
\pgfpathmoveto{\pgfqpoint{3.429180in}{3.377265in}}%
\pgfpathlineto{\pgfqpoint{3.429180in}{3.377265in}}%
\pgfpathlineto{\pgfqpoint{3.429180in}{3.381523in}}%
\pgfpathlineto{\pgfqpoint{3.433437in}{3.381523in}}%
\pgfpathlineto{\pgfqpoint{3.433437in}{3.377265in}}%
\pgfpathmoveto{\pgfqpoint{3.429180in}{3.381523in}}%
\pgfpathlineto{\pgfqpoint{3.429180in}{3.381523in}}%
\pgfpathlineto{\pgfqpoint{3.429180in}{3.385781in}}%
\pgfpathlineto{\pgfqpoint{3.433437in}{3.385781in}}%
\pgfpathlineto{\pgfqpoint{3.433437in}{3.381523in}}%
\pgfpathmoveto{\pgfqpoint{3.416406in}{3.385781in}}%
\pgfpathlineto{\pgfqpoint{3.416406in}{3.385781in}}%
\pgfpathlineto{\pgfqpoint{3.416406in}{3.390039in}}%
\pgfpathlineto{\pgfqpoint{3.420664in}{3.390039in}}%
\pgfpathlineto{\pgfqpoint{3.420664in}{3.385781in}}%
\pgfpathmoveto{\pgfqpoint{3.416406in}{3.390039in}}%
\pgfpathlineto{\pgfqpoint{3.416406in}{3.390039in}}%
\pgfpathlineto{\pgfqpoint{3.416406in}{3.394297in}}%
\pgfpathlineto{\pgfqpoint{3.420664in}{3.394297in}}%
\pgfpathlineto{\pgfqpoint{3.420664in}{3.390039in}}%
\pgfpathmoveto{\pgfqpoint{3.420664in}{3.385781in}}%
\pgfpathlineto{\pgfqpoint{3.420664in}{3.385781in}}%
\pgfpathlineto{\pgfqpoint{3.420664in}{3.390039in}}%
\pgfpathlineto{\pgfqpoint{3.424922in}{3.390039in}}%
\pgfpathlineto{\pgfqpoint{3.424922in}{3.385781in}}%
\pgfpathmoveto{\pgfqpoint{3.420664in}{3.390039in}}%
\pgfpathlineto{\pgfqpoint{3.420664in}{3.390039in}}%
\pgfpathlineto{\pgfqpoint{3.420664in}{3.394297in}}%
\pgfpathlineto{\pgfqpoint{3.424922in}{3.394297in}}%
\pgfpathlineto{\pgfqpoint{3.424922in}{3.390039in}}%
\pgfpathmoveto{\pgfqpoint{3.416406in}{3.394297in}}%
\pgfpathlineto{\pgfqpoint{3.416406in}{3.394297in}}%
\pgfpathlineto{\pgfqpoint{3.416406in}{3.398555in}}%
\pgfpathlineto{\pgfqpoint{3.420664in}{3.398555in}}%
\pgfpathlineto{\pgfqpoint{3.420664in}{3.394297in}}%
\pgfpathmoveto{\pgfqpoint{3.416406in}{3.398555in}}%
\pgfpathlineto{\pgfqpoint{3.416406in}{3.398555in}}%
\pgfpathlineto{\pgfqpoint{3.416406in}{3.402813in}}%
\pgfpathlineto{\pgfqpoint{3.420664in}{3.402813in}}%
\pgfpathlineto{\pgfqpoint{3.420664in}{3.398555in}}%
\pgfpathmoveto{\pgfqpoint{3.420664in}{3.394297in}}%
\pgfpathlineto{\pgfqpoint{3.420664in}{3.394297in}}%
\pgfpathlineto{\pgfqpoint{3.420664in}{3.398555in}}%
\pgfpathlineto{\pgfqpoint{3.424922in}{3.398555in}}%
\pgfpathlineto{\pgfqpoint{3.424922in}{3.394297in}}%
\pgfpathmoveto{\pgfqpoint{3.420664in}{3.398555in}}%
\pgfpathlineto{\pgfqpoint{3.420664in}{3.398555in}}%
\pgfpathlineto{\pgfqpoint{3.420664in}{3.402813in}}%
\pgfpathlineto{\pgfqpoint{3.424922in}{3.402813in}}%
\pgfpathlineto{\pgfqpoint{3.424922in}{3.398555in}}%
\pgfpathmoveto{\pgfqpoint{3.424922in}{3.385781in}}%
\pgfpathlineto{\pgfqpoint{3.424922in}{3.385781in}}%
\pgfpathlineto{\pgfqpoint{3.424922in}{3.390039in}}%
\pgfpathlineto{\pgfqpoint{3.429180in}{3.390039in}}%
\pgfpathlineto{\pgfqpoint{3.429180in}{3.385781in}}%
\pgfpathmoveto{\pgfqpoint{3.424922in}{3.390039in}}%
\pgfpathlineto{\pgfqpoint{3.424922in}{3.390039in}}%
\pgfpathlineto{\pgfqpoint{3.424922in}{3.394297in}}%
\pgfpathlineto{\pgfqpoint{3.429180in}{3.394297in}}%
\pgfpathlineto{\pgfqpoint{3.429180in}{3.390039in}}%
\pgfpathmoveto{\pgfqpoint{3.429180in}{3.385781in}}%
\pgfpathlineto{\pgfqpoint{3.429180in}{3.385781in}}%
\pgfpathlineto{\pgfqpoint{3.429180in}{3.390039in}}%
\pgfpathlineto{\pgfqpoint{3.433437in}{3.390039in}}%
\pgfpathlineto{\pgfqpoint{3.433437in}{3.385781in}}%
\pgfpathmoveto{\pgfqpoint{3.429180in}{3.390039in}}%
\pgfpathlineto{\pgfqpoint{3.429180in}{3.390039in}}%
\pgfpathlineto{\pgfqpoint{3.429180in}{3.394297in}}%
\pgfpathlineto{\pgfqpoint{3.433437in}{3.394297in}}%
\pgfpathlineto{\pgfqpoint{3.433437in}{3.390039in}}%
\pgfpathmoveto{\pgfqpoint{3.424922in}{3.394297in}}%
\pgfpathlineto{\pgfqpoint{3.424922in}{3.394297in}}%
\pgfpathlineto{\pgfqpoint{3.424922in}{3.398555in}}%
\pgfpathlineto{\pgfqpoint{3.429180in}{3.398555in}}%
\pgfpathlineto{\pgfqpoint{3.429180in}{3.394297in}}%
\pgfpathmoveto{\pgfqpoint{3.424922in}{3.398555in}}%
\pgfpathlineto{\pgfqpoint{3.424922in}{3.398555in}}%
\pgfpathlineto{\pgfqpoint{3.424922in}{3.402813in}}%
\pgfpathlineto{\pgfqpoint{3.429180in}{3.402813in}}%
\pgfpathlineto{\pgfqpoint{3.429180in}{3.398555in}}%
\pgfpathmoveto{\pgfqpoint{3.429180in}{3.394297in}}%
\pgfpathlineto{\pgfqpoint{3.429180in}{3.394297in}}%
\pgfpathlineto{\pgfqpoint{3.429180in}{3.398555in}}%
\pgfpathlineto{\pgfqpoint{3.433437in}{3.398555in}}%
\pgfpathlineto{\pgfqpoint{3.433437in}{3.394297in}}%
\pgfpathmoveto{\pgfqpoint{3.429180in}{3.398555in}}%
\pgfpathlineto{\pgfqpoint{3.429180in}{3.398555in}}%
\pgfpathlineto{\pgfqpoint{3.429180in}{3.402813in}}%
\pgfpathlineto{\pgfqpoint{3.433437in}{3.402813in}}%
\pgfpathlineto{\pgfqpoint{3.433437in}{3.398555in}}%
\pgfpathmoveto{\pgfqpoint{3.416406in}{3.402813in}}%
\pgfpathlineto{\pgfqpoint{3.416406in}{3.402813in}}%
\pgfpathlineto{\pgfqpoint{3.416406in}{3.407070in}}%
\pgfpathlineto{\pgfqpoint{3.420664in}{3.407070in}}%
\pgfpathlineto{\pgfqpoint{3.420664in}{3.402813in}}%
\pgfpathmoveto{\pgfqpoint{3.420664in}{3.402813in}}%
\pgfpathlineto{\pgfqpoint{3.420664in}{3.402813in}}%
\pgfpathlineto{\pgfqpoint{3.420664in}{3.407070in}}%
\pgfpathlineto{\pgfqpoint{3.424922in}{3.407070in}}%
\pgfpathlineto{\pgfqpoint{3.424922in}{3.402813in}}%
\pgfpathmoveto{\pgfqpoint{3.424922in}{3.402813in}}%
\pgfpathlineto{\pgfqpoint{3.424922in}{3.402813in}}%
\pgfpathlineto{\pgfqpoint{3.424922in}{3.407070in}}%
\pgfpathlineto{\pgfqpoint{3.429180in}{3.407070in}}%
\pgfpathlineto{\pgfqpoint{3.429180in}{3.402813in}}%
\pgfpathmoveto{\pgfqpoint{3.424922in}{3.407070in}}%
\pgfpathlineto{\pgfqpoint{3.424922in}{3.407070in}}%
\pgfpathlineto{\pgfqpoint{3.424922in}{3.411328in}}%
\pgfpathlineto{\pgfqpoint{3.429180in}{3.411328in}}%
\pgfpathlineto{\pgfqpoint{3.429180in}{3.407070in}}%
\pgfpathmoveto{\pgfqpoint{3.429180in}{3.402813in}}%
\pgfpathlineto{\pgfqpoint{3.429180in}{3.402813in}}%
\pgfpathlineto{\pgfqpoint{3.429180in}{3.407070in}}%
\pgfpathlineto{\pgfqpoint{3.433437in}{3.407070in}}%
\pgfpathlineto{\pgfqpoint{3.433437in}{3.402813in}}%
\pgfpathmoveto{\pgfqpoint{3.429180in}{3.407070in}}%
\pgfpathlineto{\pgfqpoint{3.429180in}{3.407070in}}%
\pgfpathlineto{\pgfqpoint{3.429180in}{3.411328in}}%
\pgfpathlineto{\pgfqpoint{3.433437in}{3.411328in}}%
\pgfpathlineto{\pgfqpoint{3.433437in}{3.407070in}}%
\pgfpathmoveto{\pgfqpoint{3.433437in}{3.377265in}}%
\pgfpathlineto{\pgfqpoint{3.433437in}{3.377265in}}%
\pgfpathlineto{\pgfqpoint{3.433437in}{3.381523in}}%
\pgfpathlineto{\pgfqpoint{3.437695in}{3.381523in}}%
\pgfpathlineto{\pgfqpoint{3.437695in}{3.377265in}}%
\pgfpathmoveto{\pgfqpoint{3.433437in}{3.381523in}}%
\pgfpathlineto{\pgfqpoint{3.433437in}{3.381523in}}%
\pgfpathlineto{\pgfqpoint{3.433437in}{3.385781in}}%
\pgfpathlineto{\pgfqpoint{3.437695in}{3.385781in}}%
\pgfpathlineto{\pgfqpoint{3.437695in}{3.381523in}}%
\pgfpathmoveto{\pgfqpoint{3.437695in}{3.377265in}}%
\pgfpathlineto{\pgfqpoint{3.437695in}{3.377265in}}%
\pgfpathlineto{\pgfqpoint{3.437695in}{3.381523in}}%
\pgfpathlineto{\pgfqpoint{3.441953in}{3.381523in}}%
\pgfpathlineto{\pgfqpoint{3.441953in}{3.377265in}}%
\pgfpathmoveto{\pgfqpoint{3.437695in}{3.381523in}}%
\pgfpathlineto{\pgfqpoint{3.437695in}{3.381523in}}%
\pgfpathlineto{\pgfqpoint{3.437695in}{3.385781in}}%
\pgfpathlineto{\pgfqpoint{3.441953in}{3.385781in}}%
\pgfpathlineto{\pgfqpoint{3.441953in}{3.381523in}}%
\pgfpathmoveto{\pgfqpoint{3.441953in}{3.377265in}}%
\pgfpathlineto{\pgfqpoint{3.441953in}{3.377265in}}%
\pgfpathlineto{\pgfqpoint{3.441953in}{3.381523in}}%
\pgfpathlineto{\pgfqpoint{3.446210in}{3.381523in}}%
\pgfpathlineto{\pgfqpoint{3.446210in}{3.377265in}}%
\pgfpathmoveto{\pgfqpoint{3.441953in}{3.381523in}}%
\pgfpathlineto{\pgfqpoint{3.441953in}{3.381523in}}%
\pgfpathlineto{\pgfqpoint{3.441953in}{3.385781in}}%
\pgfpathlineto{\pgfqpoint{3.446210in}{3.385781in}}%
\pgfpathlineto{\pgfqpoint{3.446210in}{3.381523in}}%
\pgfpathmoveto{\pgfqpoint{3.446210in}{3.377265in}}%
\pgfpathlineto{\pgfqpoint{3.446210in}{3.377265in}}%
\pgfpathlineto{\pgfqpoint{3.446210in}{3.381523in}}%
\pgfpathlineto{\pgfqpoint{3.450468in}{3.381523in}}%
\pgfpathlineto{\pgfqpoint{3.450468in}{3.377265in}}%
\pgfpathmoveto{\pgfqpoint{3.446210in}{3.381523in}}%
\pgfpathlineto{\pgfqpoint{3.446210in}{3.381523in}}%
\pgfpathlineto{\pgfqpoint{3.446210in}{3.385781in}}%
\pgfpathlineto{\pgfqpoint{3.450468in}{3.385781in}}%
\pgfpathlineto{\pgfqpoint{3.450468in}{3.381523in}}%
\pgfpathmoveto{\pgfqpoint{3.433437in}{3.385781in}}%
\pgfpathlineto{\pgfqpoint{3.433437in}{3.385781in}}%
\pgfpathlineto{\pgfqpoint{3.433437in}{3.390039in}}%
\pgfpathlineto{\pgfqpoint{3.437695in}{3.390039in}}%
\pgfpathlineto{\pgfqpoint{3.437695in}{3.385781in}}%
\pgfpathmoveto{\pgfqpoint{3.433437in}{3.390039in}}%
\pgfpathlineto{\pgfqpoint{3.433437in}{3.390039in}}%
\pgfpathlineto{\pgfqpoint{3.433437in}{3.394297in}}%
\pgfpathlineto{\pgfqpoint{3.437695in}{3.394297in}}%
\pgfpathlineto{\pgfqpoint{3.437695in}{3.390039in}}%
\pgfpathmoveto{\pgfqpoint{3.437695in}{3.385781in}}%
\pgfpathlineto{\pgfqpoint{3.437695in}{3.385781in}}%
\pgfpathlineto{\pgfqpoint{3.437695in}{3.390039in}}%
\pgfpathlineto{\pgfqpoint{3.441953in}{3.390039in}}%
\pgfpathlineto{\pgfqpoint{3.441953in}{3.385781in}}%
\pgfpathmoveto{\pgfqpoint{3.437695in}{3.390039in}}%
\pgfpathlineto{\pgfqpoint{3.437695in}{3.390039in}}%
\pgfpathlineto{\pgfqpoint{3.437695in}{3.394297in}}%
\pgfpathlineto{\pgfqpoint{3.441953in}{3.394297in}}%
\pgfpathlineto{\pgfqpoint{3.441953in}{3.390039in}}%
\pgfpathmoveto{\pgfqpoint{3.433437in}{3.394297in}}%
\pgfpathlineto{\pgfqpoint{3.433437in}{3.394297in}}%
\pgfpathlineto{\pgfqpoint{3.433437in}{3.398555in}}%
\pgfpathlineto{\pgfqpoint{3.437695in}{3.398555in}}%
\pgfpathlineto{\pgfqpoint{3.437695in}{3.394297in}}%
\pgfpathmoveto{\pgfqpoint{3.433437in}{3.398555in}}%
\pgfpathlineto{\pgfqpoint{3.433437in}{3.398555in}}%
\pgfpathlineto{\pgfqpoint{3.433437in}{3.402813in}}%
\pgfpathlineto{\pgfqpoint{3.437695in}{3.402813in}}%
\pgfpathlineto{\pgfqpoint{3.437695in}{3.398555in}}%
\pgfpathmoveto{\pgfqpoint{3.437695in}{3.394297in}}%
\pgfpathlineto{\pgfqpoint{3.437695in}{3.394297in}}%
\pgfpathlineto{\pgfqpoint{3.437695in}{3.398555in}}%
\pgfpathlineto{\pgfqpoint{3.441953in}{3.398555in}}%
\pgfpathlineto{\pgfqpoint{3.441953in}{3.394297in}}%
\pgfpathmoveto{\pgfqpoint{3.437695in}{3.398555in}}%
\pgfpathlineto{\pgfqpoint{3.437695in}{3.398555in}}%
\pgfpathlineto{\pgfqpoint{3.437695in}{3.402813in}}%
\pgfpathlineto{\pgfqpoint{3.441953in}{3.402813in}}%
\pgfpathlineto{\pgfqpoint{3.441953in}{3.398555in}}%
\pgfpathmoveto{\pgfqpoint{3.441953in}{3.385781in}}%
\pgfpathlineto{\pgfqpoint{3.441953in}{3.385781in}}%
\pgfpathlineto{\pgfqpoint{3.441953in}{3.390039in}}%
\pgfpathlineto{\pgfqpoint{3.446210in}{3.390039in}}%
\pgfpathlineto{\pgfqpoint{3.446210in}{3.385781in}}%
\pgfpathmoveto{\pgfqpoint{3.441953in}{3.390039in}}%
\pgfpathlineto{\pgfqpoint{3.441953in}{3.390039in}}%
\pgfpathlineto{\pgfqpoint{3.441953in}{3.394297in}}%
\pgfpathlineto{\pgfqpoint{3.446210in}{3.394297in}}%
\pgfpathlineto{\pgfqpoint{3.446210in}{3.390039in}}%
\pgfpathmoveto{\pgfqpoint{3.446210in}{3.385781in}}%
\pgfpathlineto{\pgfqpoint{3.446210in}{3.385781in}}%
\pgfpathlineto{\pgfqpoint{3.446210in}{3.390039in}}%
\pgfpathlineto{\pgfqpoint{3.450468in}{3.390039in}}%
\pgfpathlineto{\pgfqpoint{3.450468in}{3.385781in}}%
\pgfpathmoveto{\pgfqpoint{3.446210in}{3.390039in}}%
\pgfpathlineto{\pgfqpoint{3.446210in}{3.390039in}}%
\pgfpathlineto{\pgfqpoint{3.446210in}{3.394297in}}%
\pgfpathlineto{\pgfqpoint{3.450468in}{3.394297in}}%
\pgfpathlineto{\pgfqpoint{3.450468in}{3.390039in}}%
\pgfpathmoveto{\pgfqpoint{3.441953in}{3.394297in}}%
\pgfpathlineto{\pgfqpoint{3.441953in}{3.394297in}}%
\pgfpathlineto{\pgfqpoint{3.441953in}{3.398555in}}%
\pgfpathlineto{\pgfqpoint{3.446210in}{3.398555in}}%
\pgfpathlineto{\pgfqpoint{3.446210in}{3.394297in}}%
\pgfpathmoveto{\pgfqpoint{3.441953in}{3.398555in}}%
\pgfpathlineto{\pgfqpoint{3.441953in}{3.398555in}}%
\pgfpathlineto{\pgfqpoint{3.441953in}{3.402813in}}%
\pgfpathlineto{\pgfqpoint{3.446210in}{3.402813in}}%
\pgfpathlineto{\pgfqpoint{3.446210in}{3.398555in}}%
\pgfpathmoveto{\pgfqpoint{3.446210in}{3.394297in}}%
\pgfpathlineto{\pgfqpoint{3.446210in}{3.394297in}}%
\pgfpathlineto{\pgfqpoint{3.446210in}{3.398555in}}%
\pgfpathlineto{\pgfqpoint{3.450468in}{3.398555in}}%
\pgfpathlineto{\pgfqpoint{3.450468in}{3.394297in}}%
\pgfpathmoveto{\pgfqpoint{3.446210in}{3.398555in}}%
\pgfpathlineto{\pgfqpoint{3.446210in}{3.398555in}}%
\pgfpathlineto{\pgfqpoint{3.446210in}{3.402813in}}%
\pgfpathlineto{\pgfqpoint{3.450468in}{3.402813in}}%
\pgfpathlineto{\pgfqpoint{3.450468in}{3.398555in}}%
\pgfpathmoveto{\pgfqpoint{3.450468in}{3.377265in}}%
\pgfpathlineto{\pgfqpoint{3.450468in}{3.377265in}}%
\pgfpathlineto{\pgfqpoint{3.450468in}{3.381523in}}%
\pgfpathlineto{\pgfqpoint{3.454726in}{3.381523in}}%
\pgfpathlineto{\pgfqpoint{3.454726in}{3.377265in}}%
\pgfpathmoveto{\pgfqpoint{3.450468in}{3.381523in}}%
\pgfpathlineto{\pgfqpoint{3.450468in}{3.381523in}}%
\pgfpathlineto{\pgfqpoint{3.450468in}{3.385781in}}%
\pgfpathlineto{\pgfqpoint{3.454726in}{3.385781in}}%
\pgfpathlineto{\pgfqpoint{3.454726in}{3.381523in}}%
\pgfpathmoveto{\pgfqpoint{3.454726in}{3.377265in}}%
\pgfpathlineto{\pgfqpoint{3.454726in}{3.377265in}}%
\pgfpathlineto{\pgfqpoint{3.454726in}{3.381523in}}%
\pgfpathlineto{\pgfqpoint{3.458983in}{3.381523in}}%
\pgfpathlineto{\pgfqpoint{3.458983in}{3.377265in}}%
\pgfpathmoveto{\pgfqpoint{3.454726in}{3.381523in}}%
\pgfpathlineto{\pgfqpoint{3.454726in}{3.381523in}}%
\pgfpathlineto{\pgfqpoint{3.454726in}{3.385781in}}%
\pgfpathlineto{\pgfqpoint{3.458983in}{3.385781in}}%
\pgfpathlineto{\pgfqpoint{3.458983in}{3.381523in}}%
\pgfpathmoveto{\pgfqpoint{3.458983in}{3.373007in}}%
\pgfpathlineto{\pgfqpoint{3.458983in}{3.373007in}}%
\pgfpathlineto{\pgfqpoint{3.458983in}{3.377265in}}%
\pgfpathlineto{\pgfqpoint{3.463241in}{3.377265in}}%
\pgfpathlineto{\pgfqpoint{3.463241in}{3.373007in}}%
\pgfpathmoveto{\pgfqpoint{3.463241in}{3.373007in}}%
\pgfpathlineto{\pgfqpoint{3.463241in}{3.373007in}}%
\pgfpathlineto{\pgfqpoint{3.463241in}{3.377265in}}%
\pgfpathlineto{\pgfqpoint{3.467499in}{3.377265in}}%
\pgfpathlineto{\pgfqpoint{3.467499in}{3.373007in}}%
\pgfpathmoveto{\pgfqpoint{3.458983in}{3.377265in}}%
\pgfpathlineto{\pgfqpoint{3.458983in}{3.377265in}}%
\pgfpathlineto{\pgfqpoint{3.458983in}{3.381523in}}%
\pgfpathlineto{\pgfqpoint{3.463241in}{3.381523in}}%
\pgfpathlineto{\pgfqpoint{3.463241in}{3.377265in}}%
\pgfpathmoveto{\pgfqpoint{3.458983in}{3.381523in}}%
\pgfpathlineto{\pgfqpoint{3.458983in}{3.381523in}}%
\pgfpathlineto{\pgfqpoint{3.458983in}{3.385781in}}%
\pgfpathlineto{\pgfqpoint{3.463241in}{3.385781in}}%
\pgfpathlineto{\pgfqpoint{3.463241in}{3.381523in}}%
\pgfpathmoveto{\pgfqpoint{3.463241in}{3.377265in}}%
\pgfpathlineto{\pgfqpoint{3.463241in}{3.377265in}}%
\pgfpathlineto{\pgfqpoint{3.463241in}{3.381523in}}%
\pgfpathlineto{\pgfqpoint{3.467499in}{3.381523in}}%
\pgfpathlineto{\pgfqpoint{3.467499in}{3.377265in}}%
\pgfpathmoveto{\pgfqpoint{3.463241in}{3.381523in}}%
\pgfpathlineto{\pgfqpoint{3.463241in}{3.381523in}}%
\pgfpathlineto{\pgfqpoint{3.463241in}{3.385781in}}%
\pgfpathlineto{\pgfqpoint{3.467499in}{3.385781in}}%
\pgfpathlineto{\pgfqpoint{3.467499in}{3.381523in}}%
\pgfpathmoveto{\pgfqpoint{3.450468in}{3.385781in}}%
\pgfpathlineto{\pgfqpoint{3.450468in}{3.385781in}}%
\pgfpathlineto{\pgfqpoint{3.450468in}{3.390039in}}%
\pgfpathlineto{\pgfqpoint{3.454726in}{3.390039in}}%
\pgfpathlineto{\pgfqpoint{3.454726in}{3.385781in}}%
\pgfpathmoveto{\pgfqpoint{3.450468in}{3.390039in}}%
\pgfpathlineto{\pgfqpoint{3.450468in}{3.390039in}}%
\pgfpathlineto{\pgfqpoint{3.450468in}{3.394297in}}%
\pgfpathlineto{\pgfqpoint{3.454726in}{3.394297in}}%
\pgfpathlineto{\pgfqpoint{3.454726in}{3.390039in}}%
\pgfpathmoveto{\pgfqpoint{3.454726in}{3.385781in}}%
\pgfpathlineto{\pgfqpoint{3.454726in}{3.385781in}}%
\pgfpathlineto{\pgfqpoint{3.454726in}{3.390039in}}%
\pgfpathlineto{\pgfqpoint{3.458983in}{3.390039in}}%
\pgfpathlineto{\pgfqpoint{3.458983in}{3.385781in}}%
\pgfpathmoveto{\pgfqpoint{3.454726in}{3.390039in}}%
\pgfpathlineto{\pgfqpoint{3.454726in}{3.390039in}}%
\pgfpathlineto{\pgfqpoint{3.454726in}{3.394297in}}%
\pgfpathlineto{\pgfqpoint{3.458983in}{3.394297in}}%
\pgfpathlineto{\pgfqpoint{3.458983in}{3.390039in}}%
\pgfpathmoveto{\pgfqpoint{3.450468in}{3.394297in}}%
\pgfpathlineto{\pgfqpoint{3.450468in}{3.394297in}}%
\pgfpathlineto{\pgfqpoint{3.450468in}{3.398555in}}%
\pgfpathlineto{\pgfqpoint{3.454726in}{3.398555in}}%
\pgfpathlineto{\pgfqpoint{3.454726in}{3.394297in}}%
\pgfpathmoveto{\pgfqpoint{3.450468in}{3.398555in}}%
\pgfpathlineto{\pgfqpoint{3.450468in}{3.398555in}}%
\pgfpathlineto{\pgfqpoint{3.450468in}{3.402813in}}%
\pgfpathlineto{\pgfqpoint{3.454726in}{3.402813in}}%
\pgfpathlineto{\pgfqpoint{3.454726in}{3.398555in}}%
\pgfpathmoveto{\pgfqpoint{3.454726in}{3.394297in}}%
\pgfpathlineto{\pgfqpoint{3.454726in}{3.394297in}}%
\pgfpathlineto{\pgfqpoint{3.454726in}{3.398555in}}%
\pgfpathlineto{\pgfqpoint{3.458983in}{3.398555in}}%
\pgfpathlineto{\pgfqpoint{3.458983in}{3.394297in}}%
\pgfpathmoveto{\pgfqpoint{3.454726in}{3.398555in}}%
\pgfpathlineto{\pgfqpoint{3.454726in}{3.398555in}}%
\pgfpathlineto{\pgfqpoint{3.454726in}{3.402813in}}%
\pgfpathlineto{\pgfqpoint{3.458983in}{3.402813in}}%
\pgfpathlineto{\pgfqpoint{3.458983in}{3.398555in}}%
\pgfpathmoveto{\pgfqpoint{3.458983in}{3.385781in}}%
\pgfpathlineto{\pgfqpoint{3.458983in}{3.385781in}}%
\pgfpathlineto{\pgfqpoint{3.458983in}{3.390039in}}%
\pgfpathlineto{\pgfqpoint{3.463241in}{3.390039in}}%
\pgfpathlineto{\pgfqpoint{3.463241in}{3.385781in}}%
\pgfpathmoveto{\pgfqpoint{3.458983in}{3.390039in}}%
\pgfpathlineto{\pgfqpoint{3.458983in}{3.390039in}}%
\pgfpathlineto{\pgfqpoint{3.458983in}{3.394297in}}%
\pgfpathlineto{\pgfqpoint{3.463241in}{3.394297in}}%
\pgfpathlineto{\pgfqpoint{3.463241in}{3.390039in}}%
\pgfpathmoveto{\pgfqpoint{3.463241in}{3.385781in}}%
\pgfpathlineto{\pgfqpoint{3.463241in}{3.385781in}}%
\pgfpathlineto{\pgfqpoint{3.463241in}{3.390039in}}%
\pgfpathlineto{\pgfqpoint{3.467499in}{3.390039in}}%
\pgfpathlineto{\pgfqpoint{3.467499in}{3.385781in}}%
\pgfpathmoveto{\pgfqpoint{3.463241in}{3.390039in}}%
\pgfpathlineto{\pgfqpoint{3.463241in}{3.390039in}}%
\pgfpathlineto{\pgfqpoint{3.463241in}{3.394297in}}%
\pgfpathlineto{\pgfqpoint{3.467499in}{3.394297in}}%
\pgfpathlineto{\pgfqpoint{3.467499in}{3.390039in}}%
\pgfpathmoveto{\pgfqpoint{3.458983in}{3.394297in}}%
\pgfpathlineto{\pgfqpoint{3.458983in}{3.394297in}}%
\pgfpathlineto{\pgfqpoint{3.458983in}{3.398555in}}%
\pgfpathlineto{\pgfqpoint{3.463241in}{3.398555in}}%
\pgfpathlineto{\pgfqpoint{3.463241in}{3.394297in}}%
\pgfpathmoveto{\pgfqpoint{3.458983in}{3.398555in}}%
\pgfpathlineto{\pgfqpoint{3.458983in}{3.398555in}}%
\pgfpathlineto{\pgfqpoint{3.458983in}{3.402813in}}%
\pgfpathlineto{\pgfqpoint{3.463241in}{3.402813in}}%
\pgfpathlineto{\pgfqpoint{3.463241in}{3.398555in}}%
\pgfpathmoveto{\pgfqpoint{3.463241in}{3.394297in}}%
\pgfpathlineto{\pgfqpoint{3.463241in}{3.394297in}}%
\pgfpathlineto{\pgfqpoint{3.463241in}{3.398555in}}%
\pgfpathlineto{\pgfqpoint{3.467499in}{3.398555in}}%
\pgfpathlineto{\pgfqpoint{3.467499in}{3.394297in}}%
\pgfpathmoveto{\pgfqpoint{3.463241in}{3.398555in}}%
\pgfpathlineto{\pgfqpoint{3.463241in}{3.398555in}}%
\pgfpathlineto{\pgfqpoint{3.463241in}{3.402813in}}%
\pgfpathlineto{\pgfqpoint{3.467499in}{3.402813in}}%
\pgfpathlineto{\pgfqpoint{3.467499in}{3.398555in}}%
\pgfpathmoveto{\pgfqpoint{3.433437in}{3.402813in}}%
\pgfpathlineto{\pgfqpoint{3.433437in}{3.402813in}}%
\pgfpathlineto{\pgfqpoint{3.433437in}{3.407070in}}%
\pgfpathlineto{\pgfqpoint{3.437695in}{3.407070in}}%
\pgfpathlineto{\pgfqpoint{3.437695in}{3.402813in}}%
\pgfpathmoveto{\pgfqpoint{3.433437in}{3.407070in}}%
\pgfpathlineto{\pgfqpoint{3.433437in}{3.407070in}}%
\pgfpathlineto{\pgfqpoint{3.433437in}{3.411328in}}%
\pgfpathlineto{\pgfqpoint{3.437695in}{3.411328in}}%
\pgfpathlineto{\pgfqpoint{3.437695in}{3.407070in}}%
\pgfpathmoveto{\pgfqpoint{3.437695in}{3.402813in}}%
\pgfpathlineto{\pgfqpoint{3.437695in}{3.402813in}}%
\pgfpathlineto{\pgfqpoint{3.437695in}{3.407070in}}%
\pgfpathlineto{\pgfqpoint{3.441953in}{3.407070in}}%
\pgfpathlineto{\pgfqpoint{3.441953in}{3.402813in}}%
\pgfpathmoveto{\pgfqpoint{3.437695in}{3.407070in}}%
\pgfpathlineto{\pgfqpoint{3.437695in}{3.407070in}}%
\pgfpathlineto{\pgfqpoint{3.437695in}{3.411328in}}%
\pgfpathlineto{\pgfqpoint{3.441953in}{3.411328in}}%
\pgfpathlineto{\pgfqpoint{3.441953in}{3.407070in}}%
\pgfpathmoveto{\pgfqpoint{3.437695in}{3.411328in}}%
\pgfpathlineto{\pgfqpoint{3.437695in}{3.411328in}}%
\pgfpathlineto{\pgfqpoint{3.437695in}{3.415586in}}%
\pgfpathlineto{\pgfqpoint{3.441953in}{3.415586in}}%
\pgfpathlineto{\pgfqpoint{3.441953in}{3.411328in}}%
\pgfpathmoveto{\pgfqpoint{3.441953in}{3.402813in}}%
\pgfpathlineto{\pgfqpoint{3.441953in}{3.402813in}}%
\pgfpathlineto{\pgfqpoint{3.441953in}{3.407070in}}%
\pgfpathlineto{\pgfqpoint{3.446210in}{3.407070in}}%
\pgfpathlineto{\pgfqpoint{3.446210in}{3.402813in}}%
\pgfpathmoveto{\pgfqpoint{3.441953in}{3.407070in}}%
\pgfpathlineto{\pgfqpoint{3.441953in}{3.407070in}}%
\pgfpathlineto{\pgfqpoint{3.441953in}{3.411328in}}%
\pgfpathlineto{\pgfqpoint{3.446210in}{3.411328in}}%
\pgfpathlineto{\pgfqpoint{3.446210in}{3.407070in}}%
\pgfpathmoveto{\pgfqpoint{3.446210in}{3.402813in}}%
\pgfpathlineto{\pgfqpoint{3.446210in}{3.402813in}}%
\pgfpathlineto{\pgfqpoint{3.446210in}{3.407070in}}%
\pgfpathlineto{\pgfqpoint{3.450468in}{3.407070in}}%
\pgfpathlineto{\pgfqpoint{3.450468in}{3.402813in}}%
\pgfpathmoveto{\pgfqpoint{3.446210in}{3.407070in}}%
\pgfpathlineto{\pgfqpoint{3.446210in}{3.407070in}}%
\pgfpathlineto{\pgfqpoint{3.446210in}{3.411328in}}%
\pgfpathlineto{\pgfqpoint{3.450468in}{3.411328in}}%
\pgfpathlineto{\pgfqpoint{3.450468in}{3.407070in}}%
\pgfpathmoveto{\pgfqpoint{3.441953in}{3.411328in}}%
\pgfpathlineto{\pgfqpoint{3.441953in}{3.411328in}}%
\pgfpathlineto{\pgfqpoint{3.441953in}{3.415586in}}%
\pgfpathlineto{\pgfqpoint{3.446210in}{3.415586in}}%
\pgfpathlineto{\pgfqpoint{3.446210in}{3.411328in}}%
\pgfpathmoveto{\pgfqpoint{3.446210in}{3.411328in}}%
\pgfpathlineto{\pgfqpoint{3.446210in}{3.411328in}}%
\pgfpathlineto{\pgfqpoint{3.446210in}{3.415586in}}%
\pgfpathlineto{\pgfqpoint{3.450468in}{3.415586in}}%
\pgfpathlineto{\pgfqpoint{3.450468in}{3.411328in}}%
\pgfpathmoveto{\pgfqpoint{3.450468in}{3.402813in}}%
\pgfpathlineto{\pgfqpoint{3.450468in}{3.402813in}}%
\pgfpathlineto{\pgfqpoint{3.450468in}{3.407070in}}%
\pgfpathlineto{\pgfqpoint{3.454726in}{3.407070in}}%
\pgfpathlineto{\pgfqpoint{3.454726in}{3.402813in}}%
\pgfpathmoveto{\pgfqpoint{3.450468in}{3.407070in}}%
\pgfpathlineto{\pgfqpoint{3.450468in}{3.407070in}}%
\pgfpathlineto{\pgfqpoint{3.450468in}{3.411328in}}%
\pgfpathlineto{\pgfqpoint{3.454726in}{3.411328in}}%
\pgfpathlineto{\pgfqpoint{3.454726in}{3.407070in}}%
\pgfpathmoveto{\pgfqpoint{3.454726in}{3.402813in}}%
\pgfpathlineto{\pgfqpoint{3.454726in}{3.402813in}}%
\pgfpathlineto{\pgfqpoint{3.454726in}{3.407070in}}%
\pgfpathlineto{\pgfqpoint{3.458983in}{3.407070in}}%
\pgfpathlineto{\pgfqpoint{3.458983in}{3.402813in}}%
\pgfpathmoveto{\pgfqpoint{3.454726in}{3.407070in}}%
\pgfpathlineto{\pgfqpoint{3.454726in}{3.407070in}}%
\pgfpathlineto{\pgfqpoint{3.454726in}{3.411328in}}%
\pgfpathlineto{\pgfqpoint{3.458983in}{3.411328in}}%
\pgfpathlineto{\pgfqpoint{3.458983in}{3.407070in}}%
\pgfpathmoveto{\pgfqpoint{3.450468in}{3.411328in}}%
\pgfpathlineto{\pgfqpoint{3.450468in}{3.411328in}}%
\pgfpathlineto{\pgfqpoint{3.450468in}{3.415586in}}%
\pgfpathlineto{\pgfqpoint{3.454726in}{3.415586in}}%
\pgfpathlineto{\pgfqpoint{3.454726in}{3.411328in}}%
\pgfpathmoveto{\pgfqpoint{3.454726in}{3.411328in}}%
\pgfpathlineto{\pgfqpoint{3.454726in}{3.411328in}}%
\pgfpathlineto{\pgfqpoint{3.454726in}{3.415586in}}%
\pgfpathlineto{\pgfqpoint{3.458983in}{3.415586in}}%
\pgfpathlineto{\pgfqpoint{3.458983in}{3.411328in}}%
\pgfpathmoveto{\pgfqpoint{3.458983in}{3.402813in}}%
\pgfpathlineto{\pgfqpoint{3.458983in}{3.402813in}}%
\pgfpathlineto{\pgfqpoint{3.458983in}{3.407070in}}%
\pgfpathlineto{\pgfqpoint{3.463241in}{3.407070in}}%
\pgfpathlineto{\pgfqpoint{3.463241in}{3.402813in}}%
\pgfpathmoveto{\pgfqpoint{3.458983in}{3.407070in}}%
\pgfpathlineto{\pgfqpoint{3.458983in}{3.407070in}}%
\pgfpathlineto{\pgfqpoint{3.458983in}{3.411328in}}%
\pgfpathlineto{\pgfqpoint{3.463241in}{3.411328in}}%
\pgfpathlineto{\pgfqpoint{3.463241in}{3.407070in}}%
\pgfpathmoveto{\pgfqpoint{3.463241in}{3.402813in}}%
\pgfpathlineto{\pgfqpoint{3.463241in}{3.402813in}}%
\pgfpathlineto{\pgfqpoint{3.463241in}{3.407070in}}%
\pgfpathlineto{\pgfqpoint{3.467499in}{3.407070in}}%
\pgfpathlineto{\pgfqpoint{3.467499in}{3.402813in}}%
\pgfpathmoveto{\pgfqpoint{3.463241in}{3.407070in}}%
\pgfpathlineto{\pgfqpoint{3.463241in}{3.407070in}}%
\pgfpathlineto{\pgfqpoint{3.463241in}{3.411328in}}%
\pgfpathlineto{\pgfqpoint{3.467499in}{3.411328in}}%
\pgfpathlineto{\pgfqpoint{3.467499in}{3.407070in}}%
\pgfpathmoveto{\pgfqpoint{3.458983in}{3.411328in}}%
\pgfpathlineto{\pgfqpoint{3.458983in}{3.411328in}}%
\pgfpathlineto{\pgfqpoint{3.458983in}{3.415586in}}%
\pgfpathlineto{\pgfqpoint{3.463241in}{3.415586in}}%
\pgfpathlineto{\pgfqpoint{3.463241in}{3.411328in}}%
\pgfpathmoveto{\pgfqpoint{3.467499in}{3.373007in}}%
\pgfpathlineto{\pgfqpoint{3.467499in}{3.373007in}}%
\pgfpathlineto{\pgfqpoint{3.467499in}{3.377265in}}%
\pgfpathlineto{\pgfqpoint{3.471756in}{3.377265in}}%
\pgfpathlineto{\pgfqpoint{3.471756in}{3.373007in}}%
\pgfpathmoveto{\pgfqpoint{3.471756in}{3.368750in}}%
\pgfpathlineto{\pgfqpoint{3.471756in}{3.368750in}}%
\pgfpathlineto{\pgfqpoint{3.471756in}{3.373007in}}%
\pgfpathlineto{\pgfqpoint{3.476014in}{3.373007in}}%
\pgfpathlineto{\pgfqpoint{3.476014in}{3.368750in}}%
\pgfpathmoveto{\pgfqpoint{3.471756in}{3.373007in}}%
\pgfpathlineto{\pgfqpoint{3.471756in}{3.373007in}}%
\pgfpathlineto{\pgfqpoint{3.471756in}{3.377265in}}%
\pgfpathlineto{\pgfqpoint{3.476014in}{3.377265in}}%
\pgfpathlineto{\pgfqpoint{3.476014in}{3.373007in}}%
\pgfpathmoveto{\pgfqpoint{3.467499in}{3.377265in}}%
\pgfpathlineto{\pgfqpoint{3.467499in}{3.377265in}}%
\pgfpathlineto{\pgfqpoint{3.467499in}{3.381523in}}%
\pgfpathlineto{\pgfqpoint{3.471756in}{3.381523in}}%
\pgfpathlineto{\pgfqpoint{3.471756in}{3.377265in}}%
\pgfpathmoveto{\pgfqpoint{3.467499in}{3.381523in}}%
\pgfpathlineto{\pgfqpoint{3.467499in}{3.381523in}}%
\pgfpathlineto{\pgfqpoint{3.467499in}{3.385781in}}%
\pgfpathlineto{\pgfqpoint{3.471756in}{3.385781in}}%
\pgfpathlineto{\pgfqpoint{3.471756in}{3.381523in}}%
\pgfpathmoveto{\pgfqpoint{3.471756in}{3.377265in}}%
\pgfpathlineto{\pgfqpoint{3.471756in}{3.377265in}}%
\pgfpathlineto{\pgfqpoint{3.471756in}{3.381523in}}%
\pgfpathlineto{\pgfqpoint{3.476014in}{3.381523in}}%
\pgfpathlineto{\pgfqpoint{3.476014in}{3.377265in}}%
\pgfpathmoveto{\pgfqpoint{3.471756in}{3.381523in}}%
\pgfpathlineto{\pgfqpoint{3.471756in}{3.381523in}}%
\pgfpathlineto{\pgfqpoint{3.471756in}{3.385781in}}%
\pgfpathlineto{\pgfqpoint{3.476014in}{3.385781in}}%
\pgfpathlineto{\pgfqpoint{3.476014in}{3.381523in}}%
\pgfpathmoveto{\pgfqpoint{3.476014in}{3.368750in}}%
\pgfpathlineto{\pgfqpoint{3.476014in}{3.368750in}}%
\pgfpathlineto{\pgfqpoint{3.476014in}{3.373007in}}%
\pgfpathlineto{\pgfqpoint{3.480272in}{3.373007in}}%
\pgfpathlineto{\pgfqpoint{3.480272in}{3.368750in}}%
\pgfpathmoveto{\pgfqpoint{3.476014in}{3.373007in}}%
\pgfpathlineto{\pgfqpoint{3.476014in}{3.373007in}}%
\pgfpathlineto{\pgfqpoint{3.476014in}{3.377265in}}%
\pgfpathlineto{\pgfqpoint{3.480272in}{3.377265in}}%
\pgfpathlineto{\pgfqpoint{3.480272in}{3.373007in}}%
\pgfpathmoveto{\pgfqpoint{3.480272in}{3.368750in}}%
\pgfpathlineto{\pgfqpoint{3.480272in}{3.368750in}}%
\pgfpathlineto{\pgfqpoint{3.480272in}{3.373007in}}%
\pgfpathlineto{\pgfqpoint{3.484530in}{3.373007in}}%
\pgfpathlineto{\pgfqpoint{3.484530in}{3.368750in}}%
\pgfpathmoveto{\pgfqpoint{3.480272in}{3.373007in}}%
\pgfpathlineto{\pgfqpoint{3.480272in}{3.373007in}}%
\pgfpathlineto{\pgfqpoint{3.480272in}{3.377265in}}%
\pgfpathlineto{\pgfqpoint{3.484530in}{3.377265in}}%
\pgfpathlineto{\pgfqpoint{3.484530in}{3.373007in}}%
\pgfpathmoveto{\pgfqpoint{3.476014in}{3.377265in}}%
\pgfpathlineto{\pgfqpoint{3.476014in}{3.377265in}}%
\pgfpathlineto{\pgfqpoint{3.476014in}{3.381523in}}%
\pgfpathlineto{\pgfqpoint{3.480272in}{3.381523in}}%
\pgfpathlineto{\pgfqpoint{3.480272in}{3.377265in}}%
\pgfpathmoveto{\pgfqpoint{3.476014in}{3.381523in}}%
\pgfpathlineto{\pgfqpoint{3.476014in}{3.381523in}}%
\pgfpathlineto{\pgfqpoint{3.476014in}{3.385781in}}%
\pgfpathlineto{\pgfqpoint{3.480272in}{3.385781in}}%
\pgfpathlineto{\pgfqpoint{3.480272in}{3.381523in}}%
\pgfpathmoveto{\pgfqpoint{3.480272in}{3.377265in}}%
\pgfpathlineto{\pgfqpoint{3.480272in}{3.377265in}}%
\pgfpathlineto{\pgfqpoint{3.480272in}{3.381523in}}%
\pgfpathlineto{\pgfqpoint{3.484530in}{3.381523in}}%
\pgfpathlineto{\pgfqpoint{3.484530in}{3.377265in}}%
\pgfpathmoveto{\pgfqpoint{3.480272in}{3.381523in}}%
\pgfpathlineto{\pgfqpoint{3.480272in}{3.381523in}}%
\pgfpathlineto{\pgfqpoint{3.480272in}{3.385781in}}%
\pgfpathlineto{\pgfqpoint{3.484530in}{3.385781in}}%
\pgfpathlineto{\pgfqpoint{3.484530in}{3.381523in}}%
\pgfpathmoveto{\pgfqpoint{3.467499in}{3.385781in}}%
\pgfpathlineto{\pgfqpoint{3.467499in}{3.385781in}}%
\pgfpathlineto{\pgfqpoint{3.467499in}{3.390039in}}%
\pgfpathlineto{\pgfqpoint{3.471756in}{3.390039in}}%
\pgfpathlineto{\pgfqpoint{3.471756in}{3.385781in}}%
\pgfpathmoveto{\pgfqpoint{3.467499in}{3.390039in}}%
\pgfpathlineto{\pgfqpoint{3.467499in}{3.390039in}}%
\pgfpathlineto{\pgfqpoint{3.467499in}{3.394297in}}%
\pgfpathlineto{\pgfqpoint{3.471756in}{3.394297in}}%
\pgfpathlineto{\pgfqpoint{3.471756in}{3.390039in}}%
\pgfpathmoveto{\pgfqpoint{3.471756in}{3.385781in}}%
\pgfpathlineto{\pgfqpoint{3.471756in}{3.385781in}}%
\pgfpathlineto{\pgfqpoint{3.471756in}{3.390039in}}%
\pgfpathlineto{\pgfqpoint{3.476014in}{3.390039in}}%
\pgfpathlineto{\pgfqpoint{3.476014in}{3.385781in}}%
\pgfpathmoveto{\pgfqpoint{3.471756in}{3.390039in}}%
\pgfpathlineto{\pgfqpoint{3.471756in}{3.390039in}}%
\pgfpathlineto{\pgfqpoint{3.471756in}{3.394297in}}%
\pgfpathlineto{\pgfqpoint{3.476014in}{3.394297in}}%
\pgfpathlineto{\pgfqpoint{3.476014in}{3.390039in}}%
\pgfpathmoveto{\pgfqpoint{3.467499in}{3.394297in}}%
\pgfpathlineto{\pgfqpoint{3.467499in}{3.394297in}}%
\pgfpathlineto{\pgfqpoint{3.467499in}{3.398555in}}%
\pgfpathlineto{\pgfqpoint{3.471756in}{3.398555in}}%
\pgfpathlineto{\pgfqpoint{3.471756in}{3.394297in}}%
\pgfpathmoveto{\pgfqpoint{3.467499in}{3.398555in}}%
\pgfpathlineto{\pgfqpoint{3.467499in}{3.398555in}}%
\pgfpathlineto{\pgfqpoint{3.467499in}{3.402813in}}%
\pgfpathlineto{\pgfqpoint{3.471756in}{3.402813in}}%
\pgfpathlineto{\pgfqpoint{3.471756in}{3.398555in}}%
\pgfpathmoveto{\pgfqpoint{3.471756in}{3.394297in}}%
\pgfpathlineto{\pgfqpoint{3.471756in}{3.394297in}}%
\pgfpathlineto{\pgfqpoint{3.471756in}{3.398555in}}%
\pgfpathlineto{\pgfqpoint{3.476014in}{3.398555in}}%
\pgfpathlineto{\pgfqpoint{3.476014in}{3.394297in}}%
\pgfpathmoveto{\pgfqpoint{3.471756in}{3.398555in}}%
\pgfpathlineto{\pgfqpoint{3.471756in}{3.398555in}}%
\pgfpathlineto{\pgfqpoint{3.471756in}{3.402813in}}%
\pgfpathlineto{\pgfqpoint{3.476014in}{3.402813in}}%
\pgfpathlineto{\pgfqpoint{3.476014in}{3.398555in}}%
\pgfpathmoveto{\pgfqpoint{3.476014in}{3.385781in}}%
\pgfpathlineto{\pgfqpoint{3.476014in}{3.385781in}}%
\pgfpathlineto{\pgfqpoint{3.476014in}{3.390039in}}%
\pgfpathlineto{\pgfqpoint{3.480272in}{3.390039in}}%
\pgfpathlineto{\pgfqpoint{3.480272in}{3.385781in}}%
\pgfpathmoveto{\pgfqpoint{3.476014in}{3.390039in}}%
\pgfpathlineto{\pgfqpoint{3.476014in}{3.390039in}}%
\pgfpathlineto{\pgfqpoint{3.476014in}{3.394297in}}%
\pgfpathlineto{\pgfqpoint{3.480272in}{3.394297in}}%
\pgfpathlineto{\pgfqpoint{3.480272in}{3.390039in}}%
\pgfpathmoveto{\pgfqpoint{3.480272in}{3.385781in}}%
\pgfpathlineto{\pgfqpoint{3.480272in}{3.385781in}}%
\pgfpathlineto{\pgfqpoint{3.480272in}{3.390039in}}%
\pgfpathlineto{\pgfqpoint{3.484530in}{3.390039in}}%
\pgfpathlineto{\pgfqpoint{3.484530in}{3.385781in}}%
\pgfpathmoveto{\pgfqpoint{3.480272in}{3.390039in}}%
\pgfpathlineto{\pgfqpoint{3.480272in}{3.390039in}}%
\pgfpathlineto{\pgfqpoint{3.480272in}{3.394297in}}%
\pgfpathlineto{\pgfqpoint{3.484530in}{3.394297in}}%
\pgfpathlineto{\pgfqpoint{3.484530in}{3.390039in}}%
\pgfpathmoveto{\pgfqpoint{3.476014in}{3.394297in}}%
\pgfpathlineto{\pgfqpoint{3.476014in}{3.394297in}}%
\pgfpathlineto{\pgfqpoint{3.476014in}{3.398555in}}%
\pgfpathlineto{\pgfqpoint{3.480272in}{3.398555in}}%
\pgfpathlineto{\pgfqpoint{3.480272in}{3.394297in}}%
\pgfpathmoveto{\pgfqpoint{3.476014in}{3.398555in}}%
\pgfpathlineto{\pgfqpoint{3.476014in}{3.398555in}}%
\pgfpathlineto{\pgfqpoint{3.476014in}{3.402813in}}%
\pgfpathlineto{\pgfqpoint{3.480272in}{3.402813in}}%
\pgfpathlineto{\pgfqpoint{3.480272in}{3.398555in}}%
\pgfpathmoveto{\pgfqpoint{3.480272in}{3.394297in}}%
\pgfpathlineto{\pgfqpoint{3.480272in}{3.394297in}}%
\pgfpathlineto{\pgfqpoint{3.480272in}{3.398555in}}%
\pgfpathlineto{\pgfqpoint{3.484530in}{3.398555in}}%
\pgfpathlineto{\pgfqpoint{3.484530in}{3.394297in}}%
\pgfpathmoveto{\pgfqpoint{3.480272in}{3.398555in}}%
\pgfpathlineto{\pgfqpoint{3.480272in}{3.398555in}}%
\pgfpathlineto{\pgfqpoint{3.480272in}{3.402813in}}%
\pgfpathlineto{\pgfqpoint{3.484530in}{3.402813in}}%
\pgfpathlineto{\pgfqpoint{3.484530in}{3.398555in}}%
\pgfpathmoveto{\pgfqpoint{3.484530in}{3.368750in}}%
\pgfpathlineto{\pgfqpoint{3.484530in}{3.368750in}}%
\pgfpathlineto{\pgfqpoint{3.484530in}{3.373007in}}%
\pgfpathlineto{\pgfqpoint{3.488787in}{3.373007in}}%
\pgfpathlineto{\pgfqpoint{3.488787in}{3.368750in}}%
\pgfpathmoveto{\pgfqpoint{3.484530in}{3.373007in}}%
\pgfpathlineto{\pgfqpoint{3.484530in}{3.373007in}}%
\pgfpathlineto{\pgfqpoint{3.484530in}{3.377265in}}%
\pgfpathlineto{\pgfqpoint{3.488787in}{3.377265in}}%
\pgfpathlineto{\pgfqpoint{3.488787in}{3.373007in}}%
\pgfpathmoveto{\pgfqpoint{3.488787in}{3.368750in}}%
\pgfpathlineto{\pgfqpoint{3.488787in}{3.368750in}}%
\pgfpathlineto{\pgfqpoint{3.488787in}{3.373007in}}%
\pgfpathlineto{\pgfqpoint{3.493045in}{3.373007in}}%
\pgfpathlineto{\pgfqpoint{3.493045in}{3.368750in}}%
\pgfpathmoveto{\pgfqpoint{3.488787in}{3.373007in}}%
\pgfpathlineto{\pgfqpoint{3.488787in}{3.373007in}}%
\pgfpathlineto{\pgfqpoint{3.488787in}{3.377265in}}%
\pgfpathlineto{\pgfqpoint{3.493045in}{3.377265in}}%
\pgfpathlineto{\pgfqpoint{3.493045in}{3.373007in}}%
\pgfpathmoveto{\pgfqpoint{3.484530in}{3.377265in}}%
\pgfpathlineto{\pgfqpoint{3.484530in}{3.377265in}}%
\pgfpathlineto{\pgfqpoint{3.484530in}{3.381523in}}%
\pgfpathlineto{\pgfqpoint{3.488787in}{3.381523in}}%
\pgfpathlineto{\pgfqpoint{3.488787in}{3.377265in}}%
\pgfpathmoveto{\pgfqpoint{3.484530in}{3.381523in}}%
\pgfpathlineto{\pgfqpoint{3.484530in}{3.381523in}}%
\pgfpathlineto{\pgfqpoint{3.484530in}{3.385781in}}%
\pgfpathlineto{\pgfqpoint{3.488787in}{3.385781in}}%
\pgfpathlineto{\pgfqpoint{3.488787in}{3.381523in}}%
\pgfpathmoveto{\pgfqpoint{3.488787in}{3.377265in}}%
\pgfpathlineto{\pgfqpoint{3.488787in}{3.377265in}}%
\pgfpathlineto{\pgfqpoint{3.488787in}{3.381523in}}%
\pgfpathlineto{\pgfqpoint{3.493045in}{3.381523in}}%
\pgfpathlineto{\pgfqpoint{3.493045in}{3.377265in}}%
\pgfpathmoveto{\pgfqpoint{3.488787in}{3.381523in}}%
\pgfpathlineto{\pgfqpoint{3.488787in}{3.381523in}}%
\pgfpathlineto{\pgfqpoint{3.488787in}{3.385781in}}%
\pgfpathlineto{\pgfqpoint{3.493045in}{3.385781in}}%
\pgfpathlineto{\pgfqpoint{3.493045in}{3.381523in}}%
\pgfpathmoveto{\pgfqpoint{3.493045in}{3.368750in}}%
\pgfpathlineto{\pgfqpoint{3.493045in}{3.368750in}}%
\pgfpathlineto{\pgfqpoint{3.493045in}{3.373007in}}%
\pgfpathlineto{\pgfqpoint{3.497303in}{3.373007in}}%
\pgfpathlineto{\pgfqpoint{3.497303in}{3.368750in}}%
\pgfpathmoveto{\pgfqpoint{3.493045in}{3.373007in}}%
\pgfpathlineto{\pgfqpoint{3.493045in}{3.373007in}}%
\pgfpathlineto{\pgfqpoint{3.493045in}{3.377265in}}%
\pgfpathlineto{\pgfqpoint{3.497303in}{3.377265in}}%
\pgfpathlineto{\pgfqpoint{3.497303in}{3.373007in}}%
\pgfpathmoveto{\pgfqpoint{3.497303in}{3.368750in}}%
\pgfpathlineto{\pgfqpoint{3.497303in}{3.368750in}}%
\pgfpathlineto{\pgfqpoint{3.497303in}{3.373007in}}%
\pgfpathlineto{\pgfqpoint{3.501560in}{3.373007in}}%
\pgfpathlineto{\pgfqpoint{3.501560in}{3.368750in}}%
\pgfpathmoveto{\pgfqpoint{3.497303in}{3.373007in}}%
\pgfpathlineto{\pgfqpoint{3.497303in}{3.373007in}}%
\pgfpathlineto{\pgfqpoint{3.497303in}{3.377265in}}%
\pgfpathlineto{\pgfqpoint{3.501560in}{3.377265in}}%
\pgfpathlineto{\pgfqpoint{3.501560in}{3.373007in}}%
\pgfpathmoveto{\pgfqpoint{3.493045in}{3.377265in}}%
\pgfpathlineto{\pgfqpoint{3.493045in}{3.377265in}}%
\pgfpathlineto{\pgfqpoint{3.493045in}{3.381523in}}%
\pgfpathlineto{\pgfqpoint{3.497303in}{3.381523in}}%
\pgfpathlineto{\pgfqpoint{3.497303in}{3.377265in}}%
\pgfpathmoveto{\pgfqpoint{3.493045in}{3.381523in}}%
\pgfpathlineto{\pgfqpoint{3.493045in}{3.381523in}}%
\pgfpathlineto{\pgfqpoint{3.493045in}{3.385781in}}%
\pgfpathlineto{\pgfqpoint{3.497303in}{3.385781in}}%
\pgfpathlineto{\pgfqpoint{3.497303in}{3.381523in}}%
\pgfpathmoveto{\pgfqpoint{3.497303in}{3.377265in}}%
\pgfpathlineto{\pgfqpoint{3.497303in}{3.377265in}}%
\pgfpathlineto{\pgfqpoint{3.497303in}{3.381523in}}%
\pgfpathlineto{\pgfqpoint{3.501560in}{3.381523in}}%
\pgfpathlineto{\pgfqpoint{3.501560in}{3.377265in}}%
\pgfpathmoveto{\pgfqpoint{3.497303in}{3.381523in}}%
\pgfpathlineto{\pgfqpoint{3.497303in}{3.381523in}}%
\pgfpathlineto{\pgfqpoint{3.497303in}{3.385781in}}%
\pgfpathlineto{\pgfqpoint{3.501560in}{3.385781in}}%
\pgfpathlineto{\pgfqpoint{3.501560in}{3.381523in}}%
\pgfpathmoveto{\pgfqpoint{3.484530in}{3.385781in}}%
\pgfpathlineto{\pgfqpoint{3.484530in}{3.385781in}}%
\pgfpathlineto{\pgfqpoint{3.484530in}{3.390039in}}%
\pgfpathlineto{\pgfqpoint{3.488787in}{3.390039in}}%
\pgfpathlineto{\pgfqpoint{3.488787in}{3.385781in}}%
\pgfpathmoveto{\pgfqpoint{3.484530in}{3.390039in}}%
\pgfpathlineto{\pgfqpoint{3.484530in}{3.390039in}}%
\pgfpathlineto{\pgfqpoint{3.484530in}{3.394297in}}%
\pgfpathlineto{\pgfqpoint{3.488787in}{3.394297in}}%
\pgfpathlineto{\pgfqpoint{3.488787in}{3.390039in}}%
\pgfpathmoveto{\pgfqpoint{3.488787in}{3.385781in}}%
\pgfpathlineto{\pgfqpoint{3.488787in}{3.385781in}}%
\pgfpathlineto{\pgfqpoint{3.488787in}{3.390039in}}%
\pgfpathlineto{\pgfqpoint{3.493045in}{3.390039in}}%
\pgfpathlineto{\pgfqpoint{3.493045in}{3.385781in}}%
\pgfpathmoveto{\pgfqpoint{3.488787in}{3.390039in}}%
\pgfpathlineto{\pgfqpoint{3.488787in}{3.390039in}}%
\pgfpathlineto{\pgfqpoint{3.488787in}{3.394297in}}%
\pgfpathlineto{\pgfqpoint{3.493045in}{3.394297in}}%
\pgfpathlineto{\pgfqpoint{3.493045in}{3.390039in}}%
\pgfpathmoveto{\pgfqpoint{3.484530in}{3.394297in}}%
\pgfpathlineto{\pgfqpoint{3.484530in}{3.394297in}}%
\pgfpathlineto{\pgfqpoint{3.484530in}{3.398555in}}%
\pgfpathlineto{\pgfqpoint{3.488787in}{3.398555in}}%
\pgfpathlineto{\pgfqpoint{3.488787in}{3.394297in}}%
\pgfpathmoveto{\pgfqpoint{3.484530in}{3.398555in}}%
\pgfpathlineto{\pgfqpoint{3.484530in}{3.398555in}}%
\pgfpathlineto{\pgfqpoint{3.484530in}{3.402813in}}%
\pgfpathlineto{\pgfqpoint{3.488787in}{3.402813in}}%
\pgfpathlineto{\pgfqpoint{3.488787in}{3.398555in}}%
\pgfpathmoveto{\pgfqpoint{3.488787in}{3.394297in}}%
\pgfpathlineto{\pgfqpoint{3.488787in}{3.394297in}}%
\pgfpathlineto{\pgfqpoint{3.488787in}{3.398555in}}%
\pgfpathlineto{\pgfqpoint{3.493045in}{3.398555in}}%
\pgfpathlineto{\pgfqpoint{3.493045in}{3.394297in}}%
\pgfpathmoveto{\pgfqpoint{3.493045in}{3.385781in}}%
\pgfpathlineto{\pgfqpoint{3.493045in}{3.385781in}}%
\pgfpathlineto{\pgfqpoint{3.493045in}{3.390039in}}%
\pgfpathlineto{\pgfqpoint{3.497303in}{3.390039in}}%
\pgfpathlineto{\pgfqpoint{3.497303in}{3.385781in}}%
\pgfpathmoveto{\pgfqpoint{3.493045in}{3.390039in}}%
\pgfpathlineto{\pgfqpoint{3.493045in}{3.390039in}}%
\pgfpathlineto{\pgfqpoint{3.493045in}{3.394297in}}%
\pgfpathlineto{\pgfqpoint{3.497303in}{3.394297in}}%
\pgfpathlineto{\pgfqpoint{3.497303in}{3.390039in}}%
\pgfpathmoveto{\pgfqpoint{3.497303in}{3.385781in}}%
\pgfpathlineto{\pgfqpoint{3.497303in}{3.385781in}}%
\pgfpathlineto{\pgfqpoint{3.497303in}{3.390039in}}%
\pgfpathlineto{\pgfqpoint{3.501560in}{3.390039in}}%
\pgfpathlineto{\pgfqpoint{3.501560in}{3.385781in}}%
\pgfpathmoveto{\pgfqpoint{3.467499in}{3.402813in}}%
\pgfpathlineto{\pgfqpoint{3.467499in}{3.402813in}}%
\pgfpathlineto{\pgfqpoint{3.467499in}{3.407070in}}%
\pgfpathlineto{\pgfqpoint{3.471756in}{3.407070in}}%
\pgfpathlineto{\pgfqpoint{3.471756in}{3.402813in}}%
\pgfpathmoveto{\pgfqpoint{3.467499in}{3.407070in}}%
\pgfpathlineto{\pgfqpoint{3.467499in}{3.407070in}}%
\pgfpathlineto{\pgfqpoint{3.467499in}{3.411328in}}%
\pgfpathlineto{\pgfqpoint{3.471756in}{3.411328in}}%
\pgfpathlineto{\pgfqpoint{3.471756in}{3.407070in}}%
\pgfpathmoveto{\pgfqpoint{3.471756in}{3.402813in}}%
\pgfpathlineto{\pgfqpoint{3.471756in}{3.402813in}}%
\pgfpathlineto{\pgfqpoint{3.471756in}{3.407070in}}%
\pgfpathlineto{\pgfqpoint{3.476014in}{3.407070in}}%
\pgfpathlineto{\pgfqpoint{3.476014in}{3.402813in}}%
\pgfpathmoveto{\pgfqpoint{3.471756in}{3.407070in}}%
\pgfpathlineto{\pgfqpoint{3.471756in}{3.407070in}}%
\pgfpathlineto{\pgfqpoint{3.471756in}{3.411328in}}%
\pgfpathlineto{\pgfqpoint{3.476014in}{3.411328in}}%
\pgfpathlineto{\pgfqpoint{3.476014in}{3.407070in}}%
\pgfpathmoveto{\pgfqpoint{3.476014in}{3.402813in}}%
\pgfpathlineto{\pgfqpoint{3.476014in}{3.402813in}}%
\pgfpathlineto{\pgfqpoint{3.476014in}{3.407070in}}%
\pgfpathlineto{\pgfqpoint{3.480272in}{3.407070in}}%
\pgfpathlineto{\pgfqpoint{3.480272in}{3.402813in}}%
\pgfpathmoveto{\pgfqpoint{3.480272in}{3.402813in}}%
\pgfpathlineto{\pgfqpoint{3.480272in}{3.402813in}}%
\pgfpathlineto{\pgfqpoint{3.480272in}{3.407070in}}%
\pgfpathlineto{\pgfqpoint{3.484530in}{3.407070in}}%
\pgfpathlineto{\pgfqpoint{3.484530in}{3.402813in}}%
\pgfpathmoveto{\pgfqpoint{3.633556in}{3.040900in}}%
\pgfpathlineto{\pgfqpoint{3.633556in}{3.040900in}}%
\pgfpathlineto{\pgfqpoint{3.633556in}{3.045158in}}%
\pgfpathlineto{\pgfqpoint{3.637814in}{3.045158in}}%
\pgfpathlineto{\pgfqpoint{3.637814in}{3.040900in}}%
\pgfpathmoveto{\pgfqpoint{3.633556in}{3.045158in}}%
\pgfpathlineto{\pgfqpoint{3.633556in}{3.045158in}}%
\pgfpathlineto{\pgfqpoint{3.633556in}{3.049416in}}%
\pgfpathlineto{\pgfqpoint{3.637814in}{3.049416in}}%
\pgfpathlineto{\pgfqpoint{3.637814in}{3.045158in}}%
\pgfpathmoveto{\pgfqpoint{3.633556in}{3.049416in}}%
\pgfpathlineto{\pgfqpoint{3.633556in}{3.049416in}}%
\pgfpathlineto{\pgfqpoint{3.633556in}{3.053674in}}%
\pgfpathlineto{\pgfqpoint{3.637814in}{3.053674in}}%
\pgfpathlineto{\pgfqpoint{3.637814in}{3.049416in}}%
\pgfpathmoveto{\pgfqpoint{3.629298in}{3.057932in}}%
\pgfpathlineto{\pgfqpoint{3.629298in}{3.057932in}}%
\pgfpathlineto{\pgfqpoint{3.629298in}{3.062190in}}%
\pgfpathlineto{\pgfqpoint{3.633556in}{3.062190in}}%
\pgfpathlineto{\pgfqpoint{3.633556in}{3.057932in}}%
\pgfpathmoveto{\pgfqpoint{3.633556in}{3.053674in}}%
\pgfpathlineto{\pgfqpoint{3.633556in}{3.053674in}}%
\pgfpathlineto{\pgfqpoint{3.633556in}{3.057932in}}%
\pgfpathlineto{\pgfqpoint{3.637814in}{3.057932in}}%
\pgfpathlineto{\pgfqpoint{3.637814in}{3.053674in}}%
\pgfpathmoveto{\pgfqpoint{3.633556in}{3.057932in}}%
\pgfpathlineto{\pgfqpoint{3.633556in}{3.057932in}}%
\pgfpathlineto{\pgfqpoint{3.633556in}{3.062190in}}%
\pgfpathlineto{\pgfqpoint{3.637814in}{3.062190in}}%
\pgfpathlineto{\pgfqpoint{3.637814in}{3.057932in}}%
\pgfpathmoveto{\pgfqpoint{3.625040in}{3.070706in}}%
\pgfpathlineto{\pgfqpoint{3.625040in}{3.070706in}}%
\pgfpathlineto{\pgfqpoint{3.625040in}{3.074964in}}%
\pgfpathlineto{\pgfqpoint{3.629298in}{3.074964in}}%
\pgfpathlineto{\pgfqpoint{3.629298in}{3.070706in}}%
\pgfpathmoveto{\pgfqpoint{3.625040in}{3.074964in}}%
\pgfpathlineto{\pgfqpoint{3.625040in}{3.074964in}}%
\pgfpathlineto{\pgfqpoint{3.625040in}{3.079222in}}%
\pgfpathlineto{\pgfqpoint{3.629298in}{3.079222in}}%
\pgfpathlineto{\pgfqpoint{3.629298in}{3.074964in}}%
\pgfpathmoveto{\pgfqpoint{3.629298in}{3.062190in}}%
\pgfpathlineto{\pgfqpoint{3.629298in}{3.062190in}}%
\pgfpathlineto{\pgfqpoint{3.629298in}{3.066448in}}%
\pgfpathlineto{\pgfqpoint{3.633556in}{3.066448in}}%
\pgfpathlineto{\pgfqpoint{3.633556in}{3.062190in}}%
\pgfpathmoveto{\pgfqpoint{3.629298in}{3.066448in}}%
\pgfpathlineto{\pgfqpoint{3.629298in}{3.066448in}}%
\pgfpathlineto{\pgfqpoint{3.629298in}{3.070706in}}%
\pgfpathlineto{\pgfqpoint{3.633556in}{3.070706in}}%
\pgfpathlineto{\pgfqpoint{3.633556in}{3.066448in}}%
\pgfpathmoveto{\pgfqpoint{3.633556in}{3.062190in}}%
\pgfpathlineto{\pgfqpoint{3.633556in}{3.062190in}}%
\pgfpathlineto{\pgfqpoint{3.633556in}{3.066448in}}%
\pgfpathlineto{\pgfqpoint{3.637814in}{3.066448in}}%
\pgfpathlineto{\pgfqpoint{3.637814in}{3.062190in}}%
\pgfpathmoveto{\pgfqpoint{3.633556in}{3.066448in}}%
\pgfpathlineto{\pgfqpoint{3.633556in}{3.066448in}}%
\pgfpathlineto{\pgfqpoint{3.633556in}{3.070706in}}%
\pgfpathlineto{\pgfqpoint{3.637814in}{3.070706in}}%
\pgfpathlineto{\pgfqpoint{3.637814in}{3.066448in}}%
\pgfpathmoveto{\pgfqpoint{3.629298in}{3.070706in}}%
\pgfpathlineto{\pgfqpoint{3.629298in}{3.070706in}}%
\pgfpathlineto{\pgfqpoint{3.629298in}{3.074964in}}%
\pgfpathlineto{\pgfqpoint{3.633556in}{3.074964in}}%
\pgfpathlineto{\pgfqpoint{3.633556in}{3.070706in}}%
\pgfpathmoveto{\pgfqpoint{3.629298in}{3.074964in}}%
\pgfpathlineto{\pgfqpoint{3.629298in}{3.074964in}}%
\pgfpathlineto{\pgfqpoint{3.629298in}{3.079222in}}%
\pgfpathlineto{\pgfqpoint{3.633556in}{3.079222in}}%
\pgfpathlineto{\pgfqpoint{3.633556in}{3.074964in}}%
\pgfpathmoveto{\pgfqpoint{3.633556in}{3.070706in}}%
\pgfpathlineto{\pgfqpoint{3.633556in}{3.070706in}}%
\pgfpathlineto{\pgfqpoint{3.633556in}{3.074964in}}%
\pgfpathlineto{\pgfqpoint{3.637814in}{3.074964in}}%
\pgfpathlineto{\pgfqpoint{3.637814in}{3.070706in}}%
\pgfpathmoveto{\pgfqpoint{3.633556in}{3.074964in}}%
\pgfpathlineto{\pgfqpoint{3.633556in}{3.074964in}}%
\pgfpathlineto{\pgfqpoint{3.633556in}{3.079222in}}%
\pgfpathlineto{\pgfqpoint{3.637814in}{3.079222in}}%
\pgfpathlineto{\pgfqpoint{3.637814in}{3.074964in}}%
\pgfpathmoveto{\pgfqpoint{3.625040in}{3.079222in}}%
\pgfpathlineto{\pgfqpoint{3.625040in}{3.079222in}}%
\pgfpathlineto{\pgfqpoint{3.625040in}{3.083480in}}%
\pgfpathlineto{\pgfqpoint{3.629298in}{3.083480in}}%
\pgfpathlineto{\pgfqpoint{3.629298in}{3.079222in}}%
\pgfpathmoveto{\pgfqpoint{3.625040in}{3.083480in}}%
\pgfpathlineto{\pgfqpoint{3.625040in}{3.083480in}}%
\pgfpathlineto{\pgfqpoint{3.625040in}{3.087738in}}%
\pgfpathlineto{\pgfqpoint{3.629298in}{3.087738in}}%
\pgfpathlineto{\pgfqpoint{3.629298in}{3.083480in}}%
\pgfpathmoveto{\pgfqpoint{3.620782in}{3.087738in}}%
\pgfpathlineto{\pgfqpoint{3.620782in}{3.087738in}}%
\pgfpathlineto{\pgfqpoint{3.620782in}{3.091996in}}%
\pgfpathlineto{\pgfqpoint{3.625040in}{3.091996in}}%
\pgfpathlineto{\pgfqpoint{3.625040in}{3.087738in}}%
\pgfpathmoveto{\pgfqpoint{3.620782in}{3.091996in}}%
\pgfpathlineto{\pgfqpoint{3.620782in}{3.091996in}}%
\pgfpathlineto{\pgfqpoint{3.620782in}{3.096254in}}%
\pgfpathlineto{\pgfqpoint{3.625040in}{3.096254in}}%
\pgfpathlineto{\pgfqpoint{3.625040in}{3.091996in}}%
\pgfpathmoveto{\pgfqpoint{3.625040in}{3.087738in}}%
\pgfpathlineto{\pgfqpoint{3.625040in}{3.087738in}}%
\pgfpathlineto{\pgfqpoint{3.625040in}{3.091996in}}%
\pgfpathlineto{\pgfqpoint{3.629298in}{3.091996in}}%
\pgfpathlineto{\pgfqpoint{3.629298in}{3.087738in}}%
\pgfpathmoveto{\pgfqpoint{3.625040in}{3.091996in}}%
\pgfpathlineto{\pgfqpoint{3.625040in}{3.091996in}}%
\pgfpathlineto{\pgfqpoint{3.625040in}{3.096254in}}%
\pgfpathlineto{\pgfqpoint{3.629298in}{3.096254in}}%
\pgfpathlineto{\pgfqpoint{3.629298in}{3.091996in}}%
\pgfpathmoveto{\pgfqpoint{3.629298in}{3.079222in}}%
\pgfpathlineto{\pgfqpoint{3.629298in}{3.079222in}}%
\pgfpathlineto{\pgfqpoint{3.629298in}{3.083480in}}%
\pgfpathlineto{\pgfqpoint{3.633556in}{3.083480in}}%
\pgfpathlineto{\pgfqpoint{3.633556in}{3.079222in}}%
\pgfpathmoveto{\pgfqpoint{3.629298in}{3.083480in}}%
\pgfpathlineto{\pgfqpoint{3.629298in}{3.083480in}}%
\pgfpathlineto{\pgfqpoint{3.629298in}{3.087738in}}%
\pgfpathlineto{\pgfqpoint{3.633556in}{3.087738in}}%
\pgfpathlineto{\pgfqpoint{3.633556in}{3.083480in}}%
\pgfpathmoveto{\pgfqpoint{3.633556in}{3.079222in}}%
\pgfpathlineto{\pgfqpoint{3.633556in}{3.079222in}}%
\pgfpathlineto{\pgfqpoint{3.633556in}{3.083480in}}%
\pgfpathlineto{\pgfqpoint{3.637814in}{3.083480in}}%
\pgfpathlineto{\pgfqpoint{3.637814in}{3.079222in}}%
\pgfpathmoveto{\pgfqpoint{3.633556in}{3.083480in}}%
\pgfpathlineto{\pgfqpoint{3.633556in}{3.083480in}}%
\pgfpathlineto{\pgfqpoint{3.633556in}{3.087738in}}%
\pgfpathlineto{\pgfqpoint{3.637814in}{3.087738in}}%
\pgfpathlineto{\pgfqpoint{3.637814in}{3.083480in}}%
\pgfpathmoveto{\pgfqpoint{3.629298in}{3.087738in}}%
\pgfpathlineto{\pgfqpoint{3.629298in}{3.087738in}}%
\pgfpathlineto{\pgfqpoint{3.629298in}{3.091996in}}%
\pgfpathlineto{\pgfqpoint{3.633556in}{3.091996in}}%
\pgfpathlineto{\pgfqpoint{3.633556in}{3.087738in}}%
\pgfpathmoveto{\pgfqpoint{3.629298in}{3.091996in}}%
\pgfpathlineto{\pgfqpoint{3.629298in}{3.091996in}}%
\pgfpathlineto{\pgfqpoint{3.629298in}{3.096254in}}%
\pgfpathlineto{\pgfqpoint{3.633556in}{3.096254in}}%
\pgfpathlineto{\pgfqpoint{3.633556in}{3.091996in}}%
\pgfpathmoveto{\pgfqpoint{3.633556in}{3.087738in}}%
\pgfpathlineto{\pgfqpoint{3.633556in}{3.087738in}}%
\pgfpathlineto{\pgfqpoint{3.633556in}{3.091996in}}%
\pgfpathlineto{\pgfqpoint{3.637814in}{3.091996in}}%
\pgfpathlineto{\pgfqpoint{3.637814in}{3.087738in}}%
\pgfpathmoveto{\pgfqpoint{3.633556in}{3.091996in}}%
\pgfpathlineto{\pgfqpoint{3.633556in}{3.091996in}}%
\pgfpathlineto{\pgfqpoint{3.633556in}{3.096254in}}%
\pgfpathlineto{\pgfqpoint{3.637814in}{3.096254in}}%
\pgfpathlineto{\pgfqpoint{3.637814in}{3.091996in}}%
\pgfpathmoveto{\pgfqpoint{3.599493in}{3.151604in}}%
\pgfpathlineto{\pgfqpoint{3.599493in}{3.151604in}}%
\pgfpathlineto{\pgfqpoint{3.599493in}{3.155862in}}%
\pgfpathlineto{\pgfqpoint{3.603751in}{3.155862in}}%
\pgfpathlineto{\pgfqpoint{3.603751in}{3.151604in}}%
\pgfpathmoveto{\pgfqpoint{3.599493in}{3.155862in}}%
\pgfpathlineto{\pgfqpoint{3.599493in}{3.155862in}}%
\pgfpathlineto{\pgfqpoint{3.599493in}{3.160119in}}%
\pgfpathlineto{\pgfqpoint{3.603751in}{3.160119in}}%
\pgfpathlineto{\pgfqpoint{3.603751in}{3.155862in}}%
\pgfpathmoveto{\pgfqpoint{3.599493in}{3.160119in}}%
\pgfpathlineto{\pgfqpoint{3.599493in}{3.160119in}}%
\pgfpathlineto{\pgfqpoint{3.599493in}{3.164377in}}%
\pgfpathlineto{\pgfqpoint{3.603751in}{3.164377in}}%
\pgfpathlineto{\pgfqpoint{3.603751in}{3.160119in}}%
\pgfpathmoveto{\pgfqpoint{3.616524in}{3.100512in}}%
\pgfpathlineto{\pgfqpoint{3.616524in}{3.100512in}}%
\pgfpathlineto{\pgfqpoint{3.616524in}{3.104769in}}%
\pgfpathlineto{\pgfqpoint{3.620782in}{3.104769in}}%
\pgfpathlineto{\pgfqpoint{3.620782in}{3.100512in}}%
\pgfpathmoveto{\pgfqpoint{3.616524in}{3.104769in}}%
\pgfpathlineto{\pgfqpoint{3.616524in}{3.104769in}}%
\pgfpathlineto{\pgfqpoint{3.616524in}{3.109027in}}%
\pgfpathlineto{\pgfqpoint{3.620782in}{3.109027in}}%
\pgfpathlineto{\pgfqpoint{3.620782in}{3.104769in}}%
\pgfpathmoveto{\pgfqpoint{3.616524in}{3.109027in}}%
\pgfpathlineto{\pgfqpoint{3.616524in}{3.109027in}}%
\pgfpathlineto{\pgfqpoint{3.616524in}{3.113285in}}%
\pgfpathlineto{\pgfqpoint{3.620782in}{3.113285in}}%
\pgfpathlineto{\pgfqpoint{3.620782in}{3.109027in}}%
\pgfpathmoveto{\pgfqpoint{3.608009in}{3.126058in}}%
\pgfpathlineto{\pgfqpoint{3.608009in}{3.126058in}}%
\pgfpathlineto{\pgfqpoint{3.608009in}{3.130315in}}%
\pgfpathlineto{\pgfqpoint{3.612266in}{3.130315in}}%
\pgfpathlineto{\pgfqpoint{3.612266in}{3.126058in}}%
\pgfpathmoveto{\pgfqpoint{3.612266in}{3.113285in}}%
\pgfpathlineto{\pgfqpoint{3.612266in}{3.113285in}}%
\pgfpathlineto{\pgfqpoint{3.612266in}{3.117542in}}%
\pgfpathlineto{\pgfqpoint{3.616524in}{3.117542in}}%
\pgfpathlineto{\pgfqpoint{3.616524in}{3.113285in}}%
\pgfpathmoveto{\pgfqpoint{3.612266in}{3.117542in}}%
\pgfpathlineto{\pgfqpoint{3.612266in}{3.117542in}}%
\pgfpathlineto{\pgfqpoint{3.612266in}{3.121800in}}%
\pgfpathlineto{\pgfqpoint{3.616524in}{3.121800in}}%
\pgfpathlineto{\pgfqpoint{3.616524in}{3.117542in}}%
\pgfpathmoveto{\pgfqpoint{3.616524in}{3.113285in}}%
\pgfpathlineto{\pgfqpoint{3.616524in}{3.113285in}}%
\pgfpathlineto{\pgfqpoint{3.616524in}{3.117542in}}%
\pgfpathlineto{\pgfqpoint{3.620782in}{3.117542in}}%
\pgfpathlineto{\pgfqpoint{3.620782in}{3.113285in}}%
\pgfpathmoveto{\pgfqpoint{3.616524in}{3.117542in}}%
\pgfpathlineto{\pgfqpoint{3.616524in}{3.117542in}}%
\pgfpathlineto{\pgfqpoint{3.616524in}{3.121800in}}%
\pgfpathlineto{\pgfqpoint{3.620782in}{3.121800in}}%
\pgfpathlineto{\pgfqpoint{3.620782in}{3.117542in}}%
\pgfpathmoveto{\pgfqpoint{3.612266in}{3.121800in}}%
\pgfpathlineto{\pgfqpoint{3.612266in}{3.121800in}}%
\pgfpathlineto{\pgfqpoint{3.612266in}{3.126058in}}%
\pgfpathlineto{\pgfqpoint{3.616524in}{3.126058in}}%
\pgfpathlineto{\pgfqpoint{3.616524in}{3.121800in}}%
\pgfpathmoveto{\pgfqpoint{3.612266in}{3.126058in}}%
\pgfpathlineto{\pgfqpoint{3.612266in}{3.126058in}}%
\pgfpathlineto{\pgfqpoint{3.612266in}{3.130315in}}%
\pgfpathlineto{\pgfqpoint{3.616524in}{3.130315in}}%
\pgfpathlineto{\pgfqpoint{3.616524in}{3.126058in}}%
\pgfpathmoveto{\pgfqpoint{3.616524in}{3.121800in}}%
\pgfpathlineto{\pgfqpoint{3.616524in}{3.121800in}}%
\pgfpathlineto{\pgfqpoint{3.616524in}{3.126058in}}%
\pgfpathlineto{\pgfqpoint{3.620782in}{3.126058in}}%
\pgfpathlineto{\pgfqpoint{3.620782in}{3.121800in}}%
\pgfpathmoveto{\pgfqpoint{3.616524in}{3.126058in}}%
\pgfpathlineto{\pgfqpoint{3.616524in}{3.126058in}}%
\pgfpathlineto{\pgfqpoint{3.616524in}{3.130315in}}%
\pgfpathlineto{\pgfqpoint{3.620782in}{3.130315in}}%
\pgfpathlineto{\pgfqpoint{3.620782in}{3.126058in}}%
\pgfpathmoveto{\pgfqpoint{3.620782in}{3.096254in}}%
\pgfpathlineto{\pgfqpoint{3.620782in}{3.096254in}}%
\pgfpathlineto{\pgfqpoint{3.620782in}{3.100512in}}%
\pgfpathlineto{\pgfqpoint{3.625040in}{3.100512in}}%
\pgfpathlineto{\pgfqpoint{3.625040in}{3.096254in}}%
\pgfpathmoveto{\pgfqpoint{3.620782in}{3.100512in}}%
\pgfpathlineto{\pgfqpoint{3.620782in}{3.100512in}}%
\pgfpathlineto{\pgfqpoint{3.620782in}{3.104769in}}%
\pgfpathlineto{\pgfqpoint{3.625040in}{3.104769in}}%
\pgfpathlineto{\pgfqpoint{3.625040in}{3.100512in}}%
\pgfpathmoveto{\pgfqpoint{3.625040in}{3.096254in}}%
\pgfpathlineto{\pgfqpoint{3.625040in}{3.096254in}}%
\pgfpathlineto{\pgfqpoint{3.625040in}{3.100512in}}%
\pgfpathlineto{\pgfqpoint{3.629298in}{3.100512in}}%
\pgfpathlineto{\pgfqpoint{3.629298in}{3.096254in}}%
\pgfpathmoveto{\pgfqpoint{3.625040in}{3.100512in}}%
\pgfpathlineto{\pgfqpoint{3.625040in}{3.100512in}}%
\pgfpathlineto{\pgfqpoint{3.625040in}{3.104769in}}%
\pgfpathlineto{\pgfqpoint{3.629298in}{3.104769in}}%
\pgfpathlineto{\pgfqpoint{3.629298in}{3.100512in}}%
\pgfpathmoveto{\pgfqpoint{3.620782in}{3.104769in}}%
\pgfpathlineto{\pgfqpoint{3.620782in}{3.104769in}}%
\pgfpathlineto{\pgfqpoint{3.620782in}{3.109027in}}%
\pgfpathlineto{\pgfqpoint{3.625040in}{3.109027in}}%
\pgfpathlineto{\pgfqpoint{3.625040in}{3.104769in}}%
\pgfpathmoveto{\pgfqpoint{3.620782in}{3.109027in}}%
\pgfpathlineto{\pgfqpoint{3.620782in}{3.109027in}}%
\pgfpathlineto{\pgfqpoint{3.620782in}{3.113285in}}%
\pgfpathlineto{\pgfqpoint{3.625040in}{3.113285in}}%
\pgfpathlineto{\pgfqpoint{3.625040in}{3.109027in}}%
\pgfpathmoveto{\pgfqpoint{3.625040in}{3.104769in}}%
\pgfpathlineto{\pgfqpoint{3.625040in}{3.104769in}}%
\pgfpathlineto{\pgfqpoint{3.625040in}{3.109027in}}%
\pgfpathlineto{\pgfqpoint{3.629298in}{3.109027in}}%
\pgfpathlineto{\pgfqpoint{3.629298in}{3.104769in}}%
\pgfpathmoveto{\pgfqpoint{3.625040in}{3.109027in}}%
\pgfpathlineto{\pgfqpoint{3.625040in}{3.109027in}}%
\pgfpathlineto{\pgfqpoint{3.625040in}{3.113285in}}%
\pgfpathlineto{\pgfqpoint{3.629298in}{3.113285in}}%
\pgfpathlineto{\pgfqpoint{3.629298in}{3.109027in}}%
\pgfpathmoveto{\pgfqpoint{3.629298in}{3.096254in}}%
\pgfpathlineto{\pgfqpoint{3.629298in}{3.096254in}}%
\pgfpathlineto{\pgfqpoint{3.629298in}{3.100512in}}%
\pgfpathlineto{\pgfqpoint{3.633556in}{3.100512in}}%
\pgfpathlineto{\pgfqpoint{3.633556in}{3.096254in}}%
\pgfpathmoveto{\pgfqpoint{3.629298in}{3.100512in}}%
\pgfpathlineto{\pgfqpoint{3.629298in}{3.100512in}}%
\pgfpathlineto{\pgfqpoint{3.629298in}{3.104769in}}%
\pgfpathlineto{\pgfqpoint{3.633556in}{3.104769in}}%
\pgfpathlineto{\pgfqpoint{3.633556in}{3.100512in}}%
\pgfpathmoveto{\pgfqpoint{3.629298in}{3.104769in}}%
\pgfpathlineto{\pgfqpoint{3.629298in}{3.104769in}}%
\pgfpathlineto{\pgfqpoint{3.629298in}{3.109027in}}%
\pgfpathlineto{\pgfqpoint{3.633556in}{3.109027in}}%
\pgfpathlineto{\pgfqpoint{3.633556in}{3.104769in}}%
\pgfpathmoveto{\pgfqpoint{3.620782in}{3.113285in}}%
\pgfpathlineto{\pgfqpoint{3.620782in}{3.113285in}}%
\pgfpathlineto{\pgfqpoint{3.620782in}{3.117542in}}%
\pgfpathlineto{\pgfqpoint{3.625040in}{3.117542in}}%
\pgfpathlineto{\pgfqpoint{3.625040in}{3.113285in}}%
\pgfpathmoveto{\pgfqpoint{3.620782in}{3.117542in}}%
\pgfpathlineto{\pgfqpoint{3.620782in}{3.117542in}}%
\pgfpathlineto{\pgfqpoint{3.620782in}{3.121800in}}%
\pgfpathlineto{\pgfqpoint{3.625040in}{3.121800in}}%
\pgfpathlineto{\pgfqpoint{3.625040in}{3.117542in}}%
\pgfpathmoveto{\pgfqpoint{3.625040in}{3.113285in}}%
\pgfpathlineto{\pgfqpoint{3.625040in}{3.113285in}}%
\pgfpathlineto{\pgfqpoint{3.625040in}{3.117542in}}%
\pgfpathlineto{\pgfqpoint{3.629298in}{3.117542in}}%
\pgfpathlineto{\pgfqpoint{3.629298in}{3.113285in}}%
\pgfpathmoveto{\pgfqpoint{3.625040in}{3.117542in}}%
\pgfpathlineto{\pgfqpoint{3.625040in}{3.117542in}}%
\pgfpathlineto{\pgfqpoint{3.625040in}{3.121800in}}%
\pgfpathlineto{\pgfqpoint{3.629298in}{3.121800in}}%
\pgfpathlineto{\pgfqpoint{3.629298in}{3.117542in}}%
\pgfpathmoveto{\pgfqpoint{3.620782in}{3.121800in}}%
\pgfpathlineto{\pgfqpoint{3.620782in}{3.121800in}}%
\pgfpathlineto{\pgfqpoint{3.620782in}{3.126058in}}%
\pgfpathlineto{\pgfqpoint{3.625040in}{3.126058in}}%
\pgfpathlineto{\pgfqpoint{3.625040in}{3.121800in}}%
\pgfpathmoveto{\pgfqpoint{3.620782in}{3.126058in}}%
\pgfpathlineto{\pgfqpoint{3.620782in}{3.126058in}}%
\pgfpathlineto{\pgfqpoint{3.620782in}{3.130315in}}%
\pgfpathlineto{\pgfqpoint{3.625040in}{3.130315in}}%
\pgfpathlineto{\pgfqpoint{3.625040in}{3.126058in}}%
\pgfpathmoveto{\pgfqpoint{3.625040in}{3.121800in}}%
\pgfpathlineto{\pgfqpoint{3.625040in}{3.121800in}}%
\pgfpathlineto{\pgfqpoint{3.625040in}{3.126058in}}%
\pgfpathlineto{\pgfqpoint{3.629298in}{3.126058in}}%
\pgfpathlineto{\pgfqpoint{3.629298in}{3.121800in}}%
\pgfpathmoveto{\pgfqpoint{3.608009in}{3.130315in}}%
\pgfpathlineto{\pgfqpoint{3.608009in}{3.130315in}}%
\pgfpathlineto{\pgfqpoint{3.608009in}{3.134573in}}%
\pgfpathlineto{\pgfqpoint{3.612266in}{3.134573in}}%
\pgfpathlineto{\pgfqpoint{3.612266in}{3.130315in}}%
\pgfpathmoveto{\pgfqpoint{3.608009in}{3.134573in}}%
\pgfpathlineto{\pgfqpoint{3.608009in}{3.134573in}}%
\pgfpathlineto{\pgfqpoint{3.608009in}{3.138831in}}%
\pgfpathlineto{\pgfqpoint{3.612266in}{3.138831in}}%
\pgfpathlineto{\pgfqpoint{3.612266in}{3.134573in}}%
\pgfpathmoveto{\pgfqpoint{3.603751in}{3.138831in}}%
\pgfpathlineto{\pgfqpoint{3.603751in}{3.138831in}}%
\pgfpathlineto{\pgfqpoint{3.603751in}{3.143089in}}%
\pgfpathlineto{\pgfqpoint{3.608009in}{3.143089in}}%
\pgfpathlineto{\pgfqpoint{3.608009in}{3.138831in}}%
\pgfpathmoveto{\pgfqpoint{3.603751in}{3.143089in}}%
\pgfpathlineto{\pgfqpoint{3.603751in}{3.143089in}}%
\pgfpathlineto{\pgfqpoint{3.603751in}{3.147346in}}%
\pgfpathlineto{\pgfqpoint{3.608009in}{3.147346in}}%
\pgfpathlineto{\pgfqpoint{3.608009in}{3.143089in}}%
\pgfpathmoveto{\pgfqpoint{3.608009in}{3.138831in}}%
\pgfpathlineto{\pgfqpoint{3.608009in}{3.138831in}}%
\pgfpathlineto{\pgfqpoint{3.608009in}{3.143089in}}%
\pgfpathlineto{\pgfqpoint{3.612266in}{3.143089in}}%
\pgfpathlineto{\pgfqpoint{3.612266in}{3.138831in}}%
\pgfpathmoveto{\pgfqpoint{3.608009in}{3.143089in}}%
\pgfpathlineto{\pgfqpoint{3.608009in}{3.143089in}}%
\pgfpathlineto{\pgfqpoint{3.608009in}{3.147346in}}%
\pgfpathlineto{\pgfqpoint{3.612266in}{3.147346in}}%
\pgfpathlineto{\pgfqpoint{3.612266in}{3.143089in}}%
\pgfpathmoveto{\pgfqpoint{3.612266in}{3.130315in}}%
\pgfpathlineto{\pgfqpoint{3.612266in}{3.130315in}}%
\pgfpathlineto{\pgfqpoint{3.612266in}{3.134573in}}%
\pgfpathlineto{\pgfqpoint{3.616524in}{3.134573in}}%
\pgfpathlineto{\pgfqpoint{3.616524in}{3.130315in}}%
\pgfpathmoveto{\pgfqpoint{3.612266in}{3.134573in}}%
\pgfpathlineto{\pgfqpoint{3.612266in}{3.134573in}}%
\pgfpathlineto{\pgfqpoint{3.612266in}{3.138831in}}%
\pgfpathlineto{\pgfqpoint{3.616524in}{3.138831in}}%
\pgfpathlineto{\pgfqpoint{3.616524in}{3.134573in}}%
\pgfpathmoveto{\pgfqpoint{3.616524in}{3.130315in}}%
\pgfpathlineto{\pgfqpoint{3.616524in}{3.130315in}}%
\pgfpathlineto{\pgfqpoint{3.616524in}{3.134573in}}%
\pgfpathlineto{\pgfqpoint{3.620782in}{3.134573in}}%
\pgfpathlineto{\pgfqpoint{3.620782in}{3.130315in}}%
\pgfpathmoveto{\pgfqpoint{3.616524in}{3.134573in}}%
\pgfpathlineto{\pgfqpoint{3.616524in}{3.134573in}}%
\pgfpathlineto{\pgfqpoint{3.616524in}{3.138831in}}%
\pgfpathlineto{\pgfqpoint{3.620782in}{3.138831in}}%
\pgfpathlineto{\pgfqpoint{3.620782in}{3.134573in}}%
\pgfpathmoveto{\pgfqpoint{3.612266in}{3.138831in}}%
\pgfpathlineto{\pgfqpoint{3.612266in}{3.138831in}}%
\pgfpathlineto{\pgfqpoint{3.612266in}{3.143089in}}%
\pgfpathlineto{\pgfqpoint{3.616524in}{3.143089in}}%
\pgfpathlineto{\pgfqpoint{3.616524in}{3.138831in}}%
\pgfpathmoveto{\pgfqpoint{3.612266in}{3.143089in}}%
\pgfpathlineto{\pgfqpoint{3.612266in}{3.143089in}}%
\pgfpathlineto{\pgfqpoint{3.612266in}{3.147346in}}%
\pgfpathlineto{\pgfqpoint{3.616524in}{3.147346in}}%
\pgfpathlineto{\pgfqpoint{3.616524in}{3.143089in}}%
\pgfpathmoveto{\pgfqpoint{3.616524in}{3.138831in}}%
\pgfpathlineto{\pgfqpoint{3.616524in}{3.138831in}}%
\pgfpathlineto{\pgfqpoint{3.616524in}{3.143089in}}%
\pgfpathlineto{\pgfqpoint{3.620782in}{3.143089in}}%
\pgfpathlineto{\pgfqpoint{3.620782in}{3.138831in}}%
\pgfpathmoveto{\pgfqpoint{3.616524in}{3.143089in}}%
\pgfpathlineto{\pgfqpoint{3.616524in}{3.143089in}}%
\pgfpathlineto{\pgfqpoint{3.616524in}{3.147346in}}%
\pgfpathlineto{\pgfqpoint{3.620782in}{3.147346in}}%
\pgfpathlineto{\pgfqpoint{3.620782in}{3.143089in}}%
\pgfpathmoveto{\pgfqpoint{3.603751in}{3.147346in}}%
\pgfpathlineto{\pgfqpoint{3.603751in}{3.147346in}}%
\pgfpathlineto{\pgfqpoint{3.603751in}{3.151604in}}%
\pgfpathlineto{\pgfqpoint{3.608009in}{3.151604in}}%
\pgfpathlineto{\pgfqpoint{3.608009in}{3.147346in}}%
\pgfpathmoveto{\pgfqpoint{3.603751in}{3.151604in}}%
\pgfpathlineto{\pgfqpoint{3.603751in}{3.151604in}}%
\pgfpathlineto{\pgfqpoint{3.603751in}{3.155862in}}%
\pgfpathlineto{\pgfqpoint{3.608009in}{3.155862in}}%
\pgfpathlineto{\pgfqpoint{3.608009in}{3.151604in}}%
\pgfpathmoveto{\pgfqpoint{3.608009in}{3.147346in}}%
\pgfpathlineto{\pgfqpoint{3.608009in}{3.147346in}}%
\pgfpathlineto{\pgfqpoint{3.608009in}{3.151604in}}%
\pgfpathlineto{\pgfqpoint{3.612266in}{3.151604in}}%
\pgfpathlineto{\pgfqpoint{3.612266in}{3.147346in}}%
\pgfpathmoveto{\pgfqpoint{3.608009in}{3.151604in}}%
\pgfpathlineto{\pgfqpoint{3.608009in}{3.151604in}}%
\pgfpathlineto{\pgfqpoint{3.608009in}{3.155862in}}%
\pgfpathlineto{\pgfqpoint{3.612266in}{3.155862in}}%
\pgfpathlineto{\pgfqpoint{3.612266in}{3.151604in}}%
\pgfpathmoveto{\pgfqpoint{3.603751in}{3.155862in}}%
\pgfpathlineto{\pgfqpoint{3.603751in}{3.155862in}}%
\pgfpathlineto{\pgfqpoint{3.603751in}{3.160119in}}%
\pgfpathlineto{\pgfqpoint{3.608009in}{3.160119in}}%
\pgfpathlineto{\pgfqpoint{3.608009in}{3.155862in}}%
\pgfpathmoveto{\pgfqpoint{3.603751in}{3.160119in}}%
\pgfpathlineto{\pgfqpoint{3.603751in}{3.160119in}}%
\pgfpathlineto{\pgfqpoint{3.603751in}{3.164377in}}%
\pgfpathlineto{\pgfqpoint{3.608009in}{3.164377in}}%
\pgfpathlineto{\pgfqpoint{3.608009in}{3.160119in}}%
\pgfpathmoveto{\pgfqpoint{3.608009in}{3.155862in}}%
\pgfpathlineto{\pgfqpoint{3.608009in}{3.155862in}}%
\pgfpathlineto{\pgfqpoint{3.608009in}{3.160119in}}%
\pgfpathlineto{\pgfqpoint{3.612266in}{3.160119in}}%
\pgfpathlineto{\pgfqpoint{3.612266in}{3.155862in}}%
\pgfpathmoveto{\pgfqpoint{3.608009in}{3.160119in}}%
\pgfpathlineto{\pgfqpoint{3.608009in}{3.160119in}}%
\pgfpathlineto{\pgfqpoint{3.608009in}{3.164377in}}%
\pgfpathlineto{\pgfqpoint{3.612266in}{3.164377in}}%
\pgfpathlineto{\pgfqpoint{3.612266in}{3.160119in}}%
\pgfpathmoveto{\pgfqpoint{3.612266in}{3.147346in}}%
\pgfpathlineto{\pgfqpoint{3.612266in}{3.147346in}}%
\pgfpathlineto{\pgfqpoint{3.612266in}{3.151604in}}%
\pgfpathlineto{\pgfqpoint{3.616524in}{3.151604in}}%
\pgfpathlineto{\pgfqpoint{3.616524in}{3.147346in}}%
\pgfpathmoveto{\pgfqpoint{3.612266in}{3.151604in}}%
\pgfpathlineto{\pgfqpoint{3.612266in}{3.151604in}}%
\pgfpathlineto{\pgfqpoint{3.612266in}{3.155862in}}%
\pgfpathlineto{\pgfqpoint{3.616524in}{3.155862in}}%
\pgfpathlineto{\pgfqpoint{3.616524in}{3.151604in}}%
\pgfpathmoveto{\pgfqpoint{3.616524in}{3.147346in}}%
\pgfpathlineto{\pgfqpoint{3.616524in}{3.147346in}}%
\pgfpathlineto{\pgfqpoint{3.616524in}{3.151604in}}%
\pgfpathlineto{\pgfqpoint{3.620782in}{3.151604in}}%
\pgfpathlineto{\pgfqpoint{3.620782in}{3.147346in}}%
\pgfpathmoveto{\pgfqpoint{3.612266in}{3.155862in}}%
\pgfpathlineto{\pgfqpoint{3.612266in}{3.155862in}}%
\pgfpathlineto{\pgfqpoint{3.612266in}{3.160119in}}%
\pgfpathlineto{\pgfqpoint{3.616524in}{3.160119in}}%
\pgfpathlineto{\pgfqpoint{3.616524in}{3.155862in}}%
\pgfpathmoveto{\pgfqpoint{3.612266in}{3.160119in}}%
\pgfpathlineto{\pgfqpoint{3.612266in}{3.160119in}}%
\pgfpathlineto{\pgfqpoint{3.612266in}{3.164377in}}%
\pgfpathlineto{\pgfqpoint{3.616524in}{3.164377in}}%
\pgfpathlineto{\pgfqpoint{3.616524in}{3.160119in}}%
\pgfpathmoveto{\pgfqpoint{3.620782in}{3.130315in}}%
\pgfpathlineto{\pgfqpoint{3.620782in}{3.130315in}}%
\pgfpathlineto{\pgfqpoint{3.620782in}{3.134573in}}%
\pgfpathlineto{\pgfqpoint{3.625040in}{3.134573in}}%
\pgfpathlineto{\pgfqpoint{3.625040in}{3.130315in}}%
\pgfpathmoveto{\pgfqpoint{3.620782in}{3.134573in}}%
\pgfpathlineto{\pgfqpoint{3.620782in}{3.134573in}}%
\pgfpathlineto{\pgfqpoint{3.620782in}{3.138831in}}%
\pgfpathlineto{\pgfqpoint{3.625040in}{3.138831in}}%
\pgfpathlineto{\pgfqpoint{3.625040in}{3.134573in}}%
\pgfpathmoveto{\pgfqpoint{3.590977in}{3.177150in}}%
\pgfpathlineto{\pgfqpoint{3.590977in}{3.177150in}}%
\pgfpathlineto{\pgfqpoint{3.590977in}{3.181408in}}%
\pgfpathlineto{\pgfqpoint{3.595235in}{3.181408in}}%
\pgfpathlineto{\pgfqpoint{3.595235in}{3.177150in}}%
\pgfpathmoveto{\pgfqpoint{3.595235in}{3.164377in}}%
\pgfpathlineto{\pgfqpoint{3.595235in}{3.164377in}}%
\pgfpathlineto{\pgfqpoint{3.595235in}{3.168635in}}%
\pgfpathlineto{\pgfqpoint{3.599493in}{3.168635in}}%
\pgfpathlineto{\pgfqpoint{3.599493in}{3.164377in}}%
\pgfpathmoveto{\pgfqpoint{3.595235in}{3.168635in}}%
\pgfpathlineto{\pgfqpoint{3.595235in}{3.168635in}}%
\pgfpathlineto{\pgfqpoint{3.595235in}{3.172892in}}%
\pgfpathlineto{\pgfqpoint{3.599493in}{3.172892in}}%
\pgfpathlineto{\pgfqpoint{3.599493in}{3.168635in}}%
\pgfpathmoveto{\pgfqpoint{3.599493in}{3.164377in}}%
\pgfpathlineto{\pgfqpoint{3.599493in}{3.164377in}}%
\pgfpathlineto{\pgfqpoint{3.599493in}{3.168635in}}%
\pgfpathlineto{\pgfqpoint{3.603751in}{3.168635in}}%
\pgfpathlineto{\pgfqpoint{3.603751in}{3.164377in}}%
\pgfpathmoveto{\pgfqpoint{3.599493in}{3.168635in}}%
\pgfpathlineto{\pgfqpoint{3.599493in}{3.168635in}}%
\pgfpathlineto{\pgfqpoint{3.599493in}{3.172892in}}%
\pgfpathlineto{\pgfqpoint{3.603751in}{3.172892in}}%
\pgfpathlineto{\pgfqpoint{3.603751in}{3.168635in}}%
\pgfpathmoveto{\pgfqpoint{3.595235in}{3.172892in}}%
\pgfpathlineto{\pgfqpoint{3.595235in}{3.172892in}}%
\pgfpathlineto{\pgfqpoint{3.595235in}{3.177150in}}%
\pgfpathlineto{\pgfqpoint{3.599493in}{3.177150in}}%
\pgfpathlineto{\pgfqpoint{3.599493in}{3.172892in}}%
\pgfpathmoveto{\pgfqpoint{3.595235in}{3.177150in}}%
\pgfpathlineto{\pgfqpoint{3.595235in}{3.177150in}}%
\pgfpathlineto{\pgfqpoint{3.595235in}{3.181408in}}%
\pgfpathlineto{\pgfqpoint{3.599493in}{3.181408in}}%
\pgfpathlineto{\pgfqpoint{3.599493in}{3.177150in}}%
\pgfpathmoveto{\pgfqpoint{3.599493in}{3.172892in}}%
\pgfpathlineto{\pgfqpoint{3.599493in}{3.172892in}}%
\pgfpathlineto{\pgfqpoint{3.599493in}{3.177150in}}%
\pgfpathlineto{\pgfqpoint{3.603751in}{3.177150in}}%
\pgfpathlineto{\pgfqpoint{3.603751in}{3.172892in}}%
\pgfpathmoveto{\pgfqpoint{3.599493in}{3.177150in}}%
\pgfpathlineto{\pgfqpoint{3.599493in}{3.177150in}}%
\pgfpathlineto{\pgfqpoint{3.599493in}{3.181408in}}%
\pgfpathlineto{\pgfqpoint{3.603751in}{3.181408in}}%
\pgfpathlineto{\pgfqpoint{3.603751in}{3.177150in}}%
\pgfpathmoveto{\pgfqpoint{3.586719in}{3.185666in}}%
\pgfpathlineto{\pgfqpoint{3.586719in}{3.185666in}}%
\pgfpathlineto{\pgfqpoint{3.586719in}{3.189923in}}%
\pgfpathlineto{\pgfqpoint{3.590977in}{3.189923in}}%
\pgfpathlineto{\pgfqpoint{3.590977in}{3.185666in}}%
\pgfpathmoveto{\pgfqpoint{3.590977in}{3.181408in}}%
\pgfpathlineto{\pgfqpoint{3.590977in}{3.181408in}}%
\pgfpathlineto{\pgfqpoint{3.590977in}{3.185666in}}%
\pgfpathlineto{\pgfqpoint{3.595235in}{3.185666in}}%
\pgfpathlineto{\pgfqpoint{3.595235in}{3.181408in}}%
\pgfpathmoveto{\pgfqpoint{3.590977in}{3.185666in}}%
\pgfpathlineto{\pgfqpoint{3.590977in}{3.185666in}}%
\pgfpathlineto{\pgfqpoint{3.590977in}{3.189923in}}%
\pgfpathlineto{\pgfqpoint{3.595235in}{3.189923in}}%
\pgfpathlineto{\pgfqpoint{3.595235in}{3.185666in}}%
\pgfpathmoveto{\pgfqpoint{3.586719in}{3.189923in}}%
\pgfpathlineto{\pgfqpoint{3.586719in}{3.189923in}}%
\pgfpathlineto{\pgfqpoint{3.586719in}{3.194181in}}%
\pgfpathlineto{\pgfqpoint{3.590977in}{3.194181in}}%
\pgfpathlineto{\pgfqpoint{3.590977in}{3.189923in}}%
\pgfpathmoveto{\pgfqpoint{3.586719in}{3.194181in}}%
\pgfpathlineto{\pgfqpoint{3.586719in}{3.194181in}}%
\pgfpathlineto{\pgfqpoint{3.586719in}{3.198439in}}%
\pgfpathlineto{\pgfqpoint{3.590977in}{3.198439in}}%
\pgfpathlineto{\pgfqpoint{3.590977in}{3.194181in}}%
\pgfpathmoveto{\pgfqpoint{3.590977in}{3.189923in}}%
\pgfpathlineto{\pgfqpoint{3.590977in}{3.189923in}}%
\pgfpathlineto{\pgfqpoint{3.590977in}{3.194181in}}%
\pgfpathlineto{\pgfqpoint{3.595235in}{3.194181in}}%
\pgfpathlineto{\pgfqpoint{3.595235in}{3.189923in}}%
\pgfpathmoveto{\pgfqpoint{3.590977in}{3.194181in}}%
\pgfpathlineto{\pgfqpoint{3.590977in}{3.194181in}}%
\pgfpathlineto{\pgfqpoint{3.590977in}{3.198439in}}%
\pgfpathlineto{\pgfqpoint{3.595235in}{3.198439in}}%
\pgfpathlineto{\pgfqpoint{3.595235in}{3.194181in}}%
\pgfpathmoveto{\pgfqpoint{3.595235in}{3.181408in}}%
\pgfpathlineto{\pgfqpoint{3.595235in}{3.181408in}}%
\pgfpathlineto{\pgfqpoint{3.595235in}{3.185666in}}%
\pgfpathlineto{\pgfqpoint{3.599493in}{3.185666in}}%
\pgfpathlineto{\pgfqpoint{3.599493in}{3.181408in}}%
\pgfpathmoveto{\pgfqpoint{3.595235in}{3.185666in}}%
\pgfpathlineto{\pgfqpoint{3.595235in}{3.185666in}}%
\pgfpathlineto{\pgfqpoint{3.595235in}{3.189923in}}%
\pgfpathlineto{\pgfqpoint{3.599493in}{3.189923in}}%
\pgfpathlineto{\pgfqpoint{3.599493in}{3.185666in}}%
\pgfpathmoveto{\pgfqpoint{3.599493in}{3.181408in}}%
\pgfpathlineto{\pgfqpoint{3.599493in}{3.181408in}}%
\pgfpathlineto{\pgfqpoint{3.599493in}{3.185666in}}%
\pgfpathlineto{\pgfqpoint{3.603751in}{3.185666in}}%
\pgfpathlineto{\pgfqpoint{3.603751in}{3.181408in}}%
\pgfpathmoveto{\pgfqpoint{3.599493in}{3.185666in}}%
\pgfpathlineto{\pgfqpoint{3.599493in}{3.185666in}}%
\pgfpathlineto{\pgfqpoint{3.599493in}{3.189923in}}%
\pgfpathlineto{\pgfqpoint{3.603751in}{3.189923in}}%
\pgfpathlineto{\pgfqpoint{3.603751in}{3.185666in}}%
\pgfpathmoveto{\pgfqpoint{3.595235in}{3.189923in}}%
\pgfpathlineto{\pgfqpoint{3.595235in}{3.189923in}}%
\pgfpathlineto{\pgfqpoint{3.595235in}{3.194181in}}%
\pgfpathlineto{\pgfqpoint{3.599493in}{3.194181in}}%
\pgfpathlineto{\pgfqpoint{3.599493in}{3.189923in}}%
\pgfpathmoveto{\pgfqpoint{3.595235in}{3.194181in}}%
\pgfpathlineto{\pgfqpoint{3.595235in}{3.194181in}}%
\pgfpathlineto{\pgfqpoint{3.595235in}{3.198439in}}%
\pgfpathlineto{\pgfqpoint{3.599493in}{3.198439in}}%
\pgfpathlineto{\pgfqpoint{3.599493in}{3.194181in}}%
\pgfpathmoveto{\pgfqpoint{3.599493in}{3.189923in}}%
\pgfpathlineto{\pgfqpoint{3.599493in}{3.189923in}}%
\pgfpathlineto{\pgfqpoint{3.599493in}{3.194181in}}%
\pgfpathlineto{\pgfqpoint{3.603751in}{3.194181in}}%
\pgfpathlineto{\pgfqpoint{3.603751in}{3.189923in}}%
\pgfpathmoveto{\pgfqpoint{3.599493in}{3.194181in}}%
\pgfpathlineto{\pgfqpoint{3.599493in}{3.194181in}}%
\pgfpathlineto{\pgfqpoint{3.599493in}{3.198439in}}%
\pgfpathlineto{\pgfqpoint{3.603751in}{3.198439in}}%
\pgfpathlineto{\pgfqpoint{3.603751in}{3.194181in}}%
\pgfpathmoveto{\pgfqpoint{3.582461in}{3.198439in}}%
\pgfpathlineto{\pgfqpoint{3.582461in}{3.198439in}}%
\pgfpathlineto{\pgfqpoint{3.582461in}{3.202696in}}%
\pgfpathlineto{\pgfqpoint{3.586719in}{3.202696in}}%
\pgfpathlineto{\pgfqpoint{3.586719in}{3.198439in}}%
\pgfpathmoveto{\pgfqpoint{3.582461in}{3.202696in}}%
\pgfpathlineto{\pgfqpoint{3.582461in}{3.202696in}}%
\pgfpathlineto{\pgfqpoint{3.582461in}{3.206954in}}%
\pgfpathlineto{\pgfqpoint{3.586719in}{3.206954in}}%
\pgfpathlineto{\pgfqpoint{3.586719in}{3.202696in}}%
\pgfpathmoveto{\pgfqpoint{3.578203in}{3.206954in}}%
\pgfpathlineto{\pgfqpoint{3.578203in}{3.206954in}}%
\pgfpathlineto{\pgfqpoint{3.578203in}{3.211212in}}%
\pgfpathlineto{\pgfqpoint{3.582461in}{3.211212in}}%
\pgfpathlineto{\pgfqpoint{3.582461in}{3.206954in}}%
\pgfpathmoveto{\pgfqpoint{3.578203in}{3.211212in}}%
\pgfpathlineto{\pgfqpoint{3.578203in}{3.211212in}}%
\pgfpathlineto{\pgfqpoint{3.578203in}{3.215469in}}%
\pgfpathlineto{\pgfqpoint{3.582461in}{3.215469in}}%
\pgfpathlineto{\pgfqpoint{3.582461in}{3.211212in}}%
\pgfpathmoveto{\pgfqpoint{3.582461in}{3.206954in}}%
\pgfpathlineto{\pgfqpoint{3.582461in}{3.206954in}}%
\pgfpathlineto{\pgfqpoint{3.582461in}{3.211212in}}%
\pgfpathlineto{\pgfqpoint{3.586719in}{3.211212in}}%
\pgfpathlineto{\pgfqpoint{3.586719in}{3.206954in}}%
\pgfpathmoveto{\pgfqpoint{3.582461in}{3.211212in}}%
\pgfpathlineto{\pgfqpoint{3.582461in}{3.211212in}}%
\pgfpathlineto{\pgfqpoint{3.582461in}{3.215469in}}%
\pgfpathlineto{\pgfqpoint{3.586719in}{3.215469in}}%
\pgfpathlineto{\pgfqpoint{3.586719in}{3.211212in}}%
\pgfpathmoveto{\pgfqpoint{3.573945in}{3.219727in}}%
\pgfpathlineto{\pgfqpoint{3.573945in}{3.219727in}}%
\pgfpathlineto{\pgfqpoint{3.573945in}{3.223985in}}%
\pgfpathlineto{\pgfqpoint{3.578203in}{3.223985in}}%
\pgfpathlineto{\pgfqpoint{3.578203in}{3.219727in}}%
\pgfpathmoveto{\pgfqpoint{3.569687in}{3.228243in}}%
\pgfpathlineto{\pgfqpoint{3.569687in}{3.228243in}}%
\pgfpathlineto{\pgfqpoint{3.569687in}{3.232500in}}%
\pgfpathlineto{\pgfqpoint{3.573945in}{3.232500in}}%
\pgfpathlineto{\pgfqpoint{3.573945in}{3.228243in}}%
\pgfpathmoveto{\pgfqpoint{3.573945in}{3.223985in}}%
\pgfpathlineto{\pgfqpoint{3.573945in}{3.223985in}}%
\pgfpathlineto{\pgfqpoint{3.573945in}{3.228243in}}%
\pgfpathlineto{\pgfqpoint{3.578203in}{3.228243in}}%
\pgfpathlineto{\pgfqpoint{3.578203in}{3.223985in}}%
\pgfpathmoveto{\pgfqpoint{3.573945in}{3.228243in}}%
\pgfpathlineto{\pgfqpoint{3.573945in}{3.228243in}}%
\pgfpathlineto{\pgfqpoint{3.573945in}{3.232500in}}%
\pgfpathlineto{\pgfqpoint{3.578203in}{3.232500in}}%
\pgfpathlineto{\pgfqpoint{3.578203in}{3.228243in}}%
\pgfpathmoveto{\pgfqpoint{3.578203in}{3.215469in}}%
\pgfpathlineto{\pgfqpoint{3.578203in}{3.215469in}}%
\pgfpathlineto{\pgfqpoint{3.578203in}{3.219727in}}%
\pgfpathlineto{\pgfqpoint{3.582461in}{3.219727in}}%
\pgfpathlineto{\pgfqpoint{3.582461in}{3.215469in}}%
\pgfpathmoveto{\pgfqpoint{3.578203in}{3.219727in}}%
\pgfpathlineto{\pgfqpoint{3.578203in}{3.219727in}}%
\pgfpathlineto{\pgfqpoint{3.578203in}{3.223985in}}%
\pgfpathlineto{\pgfqpoint{3.582461in}{3.223985in}}%
\pgfpathlineto{\pgfqpoint{3.582461in}{3.219727in}}%
\pgfpathmoveto{\pgfqpoint{3.582461in}{3.215469in}}%
\pgfpathlineto{\pgfqpoint{3.582461in}{3.215469in}}%
\pgfpathlineto{\pgfqpoint{3.582461in}{3.219727in}}%
\pgfpathlineto{\pgfqpoint{3.586719in}{3.219727in}}%
\pgfpathlineto{\pgfqpoint{3.586719in}{3.215469in}}%
\pgfpathmoveto{\pgfqpoint{3.582461in}{3.219727in}}%
\pgfpathlineto{\pgfqpoint{3.582461in}{3.219727in}}%
\pgfpathlineto{\pgfqpoint{3.582461in}{3.223985in}}%
\pgfpathlineto{\pgfqpoint{3.586719in}{3.223985in}}%
\pgfpathlineto{\pgfqpoint{3.586719in}{3.219727in}}%
\pgfpathmoveto{\pgfqpoint{3.578203in}{3.223985in}}%
\pgfpathlineto{\pgfqpoint{3.578203in}{3.223985in}}%
\pgfpathlineto{\pgfqpoint{3.578203in}{3.228243in}}%
\pgfpathlineto{\pgfqpoint{3.582461in}{3.228243in}}%
\pgfpathlineto{\pgfqpoint{3.582461in}{3.223985in}}%
\pgfpathmoveto{\pgfqpoint{3.578203in}{3.228243in}}%
\pgfpathlineto{\pgfqpoint{3.578203in}{3.228243in}}%
\pgfpathlineto{\pgfqpoint{3.578203in}{3.232500in}}%
\pgfpathlineto{\pgfqpoint{3.582461in}{3.232500in}}%
\pgfpathlineto{\pgfqpoint{3.582461in}{3.228243in}}%
\pgfpathmoveto{\pgfqpoint{3.582461in}{3.223985in}}%
\pgfpathlineto{\pgfqpoint{3.582461in}{3.223985in}}%
\pgfpathlineto{\pgfqpoint{3.582461in}{3.228243in}}%
\pgfpathlineto{\pgfqpoint{3.586719in}{3.228243in}}%
\pgfpathlineto{\pgfqpoint{3.586719in}{3.223985in}}%
\pgfpathmoveto{\pgfqpoint{3.582461in}{3.228243in}}%
\pgfpathlineto{\pgfqpoint{3.582461in}{3.228243in}}%
\pgfpathlineto{\pgfqpoint{3.582461in}{3.232500in}}%
\pgfpathlineto{\pgfqpoint{3.586719in}{3.232500in}}%
\pgfpathlineto{\pgfqpoint{3.586719in}{3.228243in}}%
\pgfpathmoveto{\pgfqpoint{3.586719in}{3.198439in}}%
\pgfpathlineto{\pgfqpoint{3.586719in}{3.198439in}}%
\pgfpathlineto{\pgfqpoint{3.586719in}{3.202696in}}%
\pgfpathlineto{\pgfqpoint{3.590977in}{3.202696in}}%
\pgfpathlineto{\pgfqpoint{3.590977in}{3.198439in}}%
\pgfpathmoveto{\pgfqpoint{3.586719in}{3.202696in}}%
\pgfpathlineto{\pgfqpoint{3.586719in}{3.202696in}}%
\pgfpathlineto{\pgfqpoint{3.586719in}{3.206954in}}%
\pgfpathlineto{\pgfqpoint{3.590977in}{3.206954in}}%
\pgfpathlineto{\pgfqpoint{3.590977in}{3.202696in}}%
\pgfpathmoveto{\pgfqpoint{3.590977in}{3.198439in}}%
\pgfpathlineto{\pgfqpoint{3.590977in}{3.198439in}}%
\pgfpathlineto{\pgfqpoint{3.590977in}{3.202696in}}%
\pgfpathlineto{\pgfqpoint{3.595235in}{3.202696in}}%
\pgfpathlineto{\pgfqpoint{3.595235in}{3.198439in}}%
\pgfpathmoveto{\pgfqpoint{3.590977in}{3.202696in}}%
\pgfpathlineto{\pgfqpoint{3.590977in}{3.202696in}}%
\pgfpathlineto{\pgfqpoint{3.590977in}{3.206954in}}%
\pgfpathlineto{\pgfqpoint{3.595235in}{3.206954in}}%
\pgfpathlineto{\pgfqpoint{3.595235in}{3.202696in}}%
\pgfpathmoveto{\pgfqpoint{3.586719in}{3.206954in}}%
\pgfpathlineto{\pgfqpoint{3.586719in}{3.206954in}}%
\pgfpathlineto{\pgfqpoint{3.586719in}{3.211212in}}%
\pgfpathlineto{\pgfqpoint{3.590977in}{3.211212in}}%
\pgfpathlineto{\pgfqpoint{3.590977in}{3.206954in}}%
\pgfpathmoveto{\pgfqpoint{3.586719in}{3.211212in}}%
\pgfpathlineto{\pgfqpoint{3.586719in}{3.211212in}}%
\pgfpathlineto{\pgfqpoint{3.586719in}{3.215469in}}%
\pgfpathlineto{\pgfqpoint{3.590977in}{3.215469in}}%
\pgfpathlineto{\pgfqpoint{3.590977in}{3.211212in}}%
\pgfpathmoveto{\pgfqpoint{3.590977in}{3.206954in}}%
\pgfpathlineto{\pgfqpoint{3.590977in}{3.206954in}}%
\pgfpathlineto{\pgfqpoint{3.590977in}{3.211212in}}%
\pgfpathlineto{\pgfqpoint{3.595235in}{3.211212in}}%
\pgfpathlineto{\pgfqpoint{3.595235in}{3.206954in}}%
\pgfpathmoveto{\pgfqpoint{3.590977in}{3.211212in}}%
\pgfpathlineto{\pgfqpoint{3.590977in}{3.211212in}}%
\pgfpathlineto{\pgfqpoint{3.590977in}{3.215469in}}%
\pgfpathlineto{\pgfqpoint{3.595235in}{3.215469in}}%
\pgfpathlineto{\pgfqpoint{3.595235in}{3.211212in}}%
\pgfpathmoveto{\pgfqpoint{3.595235in}{3.198439in}}%
\pgfpathlineto{\pgfqpoint{3.595235in}{3.198439in}}%
\pgfpathlineto{\pgfqpoint{3.595235in}{3.202696in}}%
\pgfpathlineto{\pgfqpoint{3.599493in}{3.202696in}}%
\pgfpathlineto{\pgfqpoint{3.599493in}{3.198439in}}%
\pgfpathmoveto{\pgfqpoint{3.595235in}{3.202696in}}%
\pgfpathlineto{\pgfqpoint{3.595235in}{3.202696in}}%
\pgfpathlineto{\pgfqpoint{3.595235in}{3.206954in}}%
\pgfpathlineto{\pgfqpoint{3.599493in}{3.206954in}}%
\pgfpathlineto{\pgfqpoint{3.599493in}{3.202696in}}%
\pgfpathmoveto{\pgfqpoint{3.599493in}{3.198439in}}%
\pgfpathlineto{\pgfqpoint{3.599493in}{3.198439in}}%
\pgfpathlineto{\pgfqpoint{3.599493in}{3.202696in}}%
\pgfpathlineto{\pgfqpoint{3.603751in}{3.202696in}}%
\pgfpathlineto{\pgfqpoint{3.603751in}{3.198439in}}%
\pgfpathmoveto{\pgfqpoint{3.595235in}{3.206954in}}%
\pgfpathlineto{\pgfqpoint{3.595235in}{3.206954in}}%
\pgfpathlineto{\pgfqpoint{3.595235in}{3.211212in}}%
\pgfpathlineto{\pgfqpoint{3.599493in}{3.211212in}}%
\pgfpathlineto{\pgfqpoint{3.599493in}{3.206954in}}%
\pgfpathmoveto{\pgfqpoint{3.595235in}{3.211212in}}%
\pgfpathlineto{\pgfqpoint{3.595235in}{3.211212in}}%
\pgfpathlineto{\pgfqpoint{3.595235in}{3.215469in}}%
\pgfpathlineto{\pgfqpoint{3.599493in}{3.215469in}}%
\pgfpathlineto{\pgfqpoint{3.599493in}{3.211212in}}%
\pgfpathmoveto{\pgfqpoint{3.586719in}{3.215469in}}%
\pgfpathlineto{\pgfqpoint{3.586719in}{3.215469in}}%
\pgfpathlineto{\pgfqpoint{3.586719in}{3.219727in}}%
\pgfpathlineto{\pgfqpoint{3.590977in}{3.219727in}}%
\pgfpathlineto{\pgfqpoint{3.590977in}{3.215469in}}%
\pgfpathmoveto{\pgfqpoint{3.586719in}{3.219727in}}%
\pgfpathlineto{\pgfqpoint{3.586719in}{3.219727in}}%
\pgfpathlineto{\pgfqpoint{3.586719in}{3.223985in}}%
\pgfpathlineto{\pgfqpoint{3.590977in}{3.223985in}}%
\pgfpathlineto{\pgfqpoint{3.590977in}{3.219727in}}%
\pgfpathmoveto{\pgfqpoint{3.590977in}{3.215469in}}%
\pgfpathlineto{\pgfqpoint{3.590977in}{3.215469in}}%
\pgfpathlineto{\pgfqpoint{3.590977in}{3.219727in}}%
\pgfpathlineto{\pgfqpoint{3.595235in}{3.219727in}}%
\pgfpathlineto{\pgfqpoint{3.595235in}{3.215469in}}%
\pgfpathmoveto{\pgfqpoint{3.590977in}{3.219727in}}%
\pgfpathlineto{\pgfqpoint{3.590977in}{3.219727in}}%
\pgfpathlineto{\pgfqpoint{3.590977in}{3.223985in}}%
\pgfpathlineto{\pgfqpoint{3.595235in}{3.223985in}}%
\pgfpathlineto{\pgfqpoint{3.595235in}{3.219727in}}%
\pgfpathmoveto{\pgfqpoint{3.586719in}{3.223985in}}%
\pgfpathlineto{\pgfqpoint{3.586719in}{3.223985in}}%
\pgfpathlineto{\pgfqpoint{3.586719in}{3.228243in}}%
\pgfpathlineto{\pgfqpoint{3.590977in}{3.228243in}}%
\pgfpathlineto{\pgfqpoint{3.590977in}{3.223985in}}%
\pgfpathmoveto{\pgfqpoint{3.586719in}{3.228243in}}%
\pgfpathlineto{\pgfqpoint{3.586719in}{3.228243in}}%
\pgfpathlineto{\pgfqpoint{3.586719in}{3.232500in}}%
\pgfpathlineto{\pgfqpoint{3.590977in}{3.232500in}}%
\pgfpathlineto{\pgfqpoint{3.590977in}{3.228243in}}%
\pgfpathmoveto{\pgfqpoint{3.603751in}{3.164377in}}%
\pgfpathlineto{\pgfqpoint{3.603751in}{3.164377in}}%
\pgfpathlineto{\pgfqpoint{3.603751in}{3.168635in}}%
\pgfpathlineto{\pgfqpoint{3.608009in}{3.168635in}}%
\pgfpathlineto{\pgfqpoint{3.608009in}{3.164377in}}%
\pgfpathmoveto{\pgfqpoint{3.603751in}{3.168635in}}%
\pgfpathlineto{\pgfqpoint{3.603751in}{3.168635in}}%
\pgfpathlineto{\pgfqpoint{3.603751in}{3.172892in}}%
\pgfpathlineto{\pgfqpoint{3.608009in}{3.172892in}}%
\pgfpathlineto{\pgfqpoint{3.608009in}{3.168635in}}%
\pgfpathmoveto{\pgfqpoint{3.608009in}{3.164377in}}%
\pgfpathlineto{\pgfqpoint{3.608009in}{3.164377in}}%
\pgfpathlineto{\pgfqpoint{3.608009in}{3.168635in}}%
\pgfpathlineto{\pgfqpoint{3.612266in}{3.168635in}}%
\pgfpathlineto{\pgfqpoint{3.612266in}{3.164377in}}%
\pgfpathmoveto{\pgfqpoint{3.608009in}{3.168635in}}%
\pgfpathlineto{\pgfqpoint{3.608009in}{3.168635in}}%
\pgfpathlineto{\pgfqpoint{3.608009in}{3.172892in}}%
\pgfpathlineto{\pgfqpoint{3.612266in}{3.172892in}}%
\pgfpathlineto{\pgfqpoint{3.612266in}{3.168635in}}%
\pgfpathmoveto{\pgfqpoint{3.603751in}{3.172892in}}%
\pgfpathlineto{\pgfqpoint{3.603751in}{3.172892in}}%
\pgfpathlineto{\pgfqpoint{3.603751in}{3.177150in}}%
\pgfpathlineto{\pgfqpoint{3.608009in}{3.177150in}}%
\pgfpathlineto{\pgfqpoint{3.608009in}{3.172892in}}%
\pgfpathmoveto{\pgfqpoint{3.603751in}{3.177150in}}%
\pgfpathlineto{\pgfqpoint{3.603751in}{3.177150in}}%
\pgfpathlineto{\pgfqpoint{3.603751in}{3.181408in}}%
\pgfpathlineto{\pgfqpoint{3.608009in}{3.181408in}}%
\pgfpathlineto{\pgfqpoint{3.608009in}{3.177150in}}%
\pgfpathmoveto{\pgfqpoint{3.608009in}{3.172892in}}%
\pgfpathlineto{\pgfqpoint{3.608009in}{3.172892in}}%
\pgfpathlineto{\pgfqpoint{3.608009in}{3.177150in}}%
\pgfpathlineto{\pgfqpoint{3.612266in}{3.177150in}}%
\pgfpathlineto{\pgfqpoint{3.612266in}{3.172892in}}%
\pgfpathmoveto{\pgfqpoint{3.603751in}{3.181408in}}%
\pgfpathlineto{\pgfqpoint{3.603751in}{3.181408in}}%
\pgfpathlineto{\pgfqpoint{3.603751in}{3.185666in}}%
\pgfpathlineto{\pgfqpoint{3.608009in}{3.185666in}}%
\pgfpathlineto{\pgfqpoint{3.608009in}{3.181408in}}%
\pgfpathmoveto{\pgfqpoint{3.603751in}{3.185666in}}%
\pgfpathlineto{\pgfqpoint{3.603751in}{3.185666in}}%
\pgfpathlineto{\pgfqpoint{3.603751in}{3.189923in}}%
\pgfpathlineto{\pgfqpoint{3.608009in}{3.189923in}}%
\pgfpathlineto{\pgfqpoint{3.608009in}{3.185666in}}%
\pgfpathmoveto{\pgfqpoint{3.565429in}{3.241016in}}%
\pgfpathlineto{\pgfqpoint{3.565429in}{3.241016in}}%
\pgfpathlineto{\pgfqpoint{3.565429in}{3.245274in}}%
\pgfpathlineto{\pgfqpoint{3.569687in}{3.245274in}}%
\pgfpathlineto{\pgfqpoint{3.569687in}{3.241016in}}%
\pgfpathmoveto{\pgfqpoint{3.565429in}{3.245274in}}%
\pgfpathlineto{\pgfqpoint{3.565429in}{3.245274in}}%
\pgfpathlineto{\pgfqpoint{3.565429in}{3.249531in}}%
\pgfpathlineto{\pgfqpoint{3.569687in}{3.249531in}}%
\pgfpathlineto{\pgfqpoint{3.569687in}{3.245274in}}%
\pgfpathmoveto{\pgfqpoint{3.556913in}{3.258047in}}%
\pgfpathlineto{\pgfqpoint{3.556913in}{3.258047in}}%
\pgfpathlineto{\pgfqpoint{3.556913in}{3.262305in}}%
\pgfpathlineto{\pgfqpoint{3.561171in}{3.262305in}}%
\pgfpathlineto{\pgfqpoint{3.561171in}{3.258047in}}%
\pgfpathmoveto{\pgfqpoint{3.556913in}{3.262305in}}%
\pgfpathlineto{\pgfqpoint{3.556913in}{3.262305in}}%
\pgfpathlineto{\pgfqpoint{3.556913in}{3.266563in}}%
\pgfpathlineto{\pgfqpoint{3.561171in}{3.266563in}}%
\pgfpathlineto{\pgfqpoint{3.561171in}{3.262305in}}%
\pgfpathmoveto{\pgfqpoint{3.561171in}{3.249531in}}%
\pgfpathlineto{\pgfqpoint{3.561171in}{3.249531in}}%
\pgfpathlineto{\pgfqpoint{3.561171in}{3.253789in}}%
\pgfpathlineto{\pgfqpoint{3.565429in}{3.253789in}}%
\pgfpathlineto{\pgfqpoint{3.565429in}{3.249531in}}%
\pgfpathmoveto{\pgfqpoint{3.561171in}{3.253789in}}%
\pgfpathlineto{\pgfqpoint{3.561171in}{3.253789in}}%
\pgfpathlineto{\pgfqpoint{3.561171in}{3.258047in}}%
\pgfpathlineto{\pgfqpoint{3.565429in}{3.258047in}}%
\pgfpathlineto{\pgfqpoint{3.565429in}{3.253789in}}%
\pgfpathmoveto{\pgfqpoint{3.565429in}{3.249531in}}%
\pgfpathlineto{\pgfqpoint{3.565429in}{3.249531in}}%
\pgfpathlineto{\pgfqpoint{3.565429in}{3.253789in}}%
\pgfpathlineto{\pgfqpoint{3.569687in}{3.253789in}}%
\pgfpathlineto{\pgfqpoint{3.569687in}{3.249531in}}%
\pgfpathmoveto{\pgfqpoint{3.565429in}{3.253789in}}%
\pgfpathlineto{\pgfqpoint{3.565429in}{3.253789in}}%
\pgfpathlineto{\pgfqpoint{3.565429in}{3.258047in}}%
\pgfpathlineto{\pgfqpoint{3.569687in}{3.258047in}}%
\pgfpathlineto{\pgfqpoint{3.569687in}{3.253789in}}%
\pgfpathmoveto{\pgfqpoint{3.561171in}{3.258047in}}%
\pgfpathlineto{\pgfqpoint{3.561171in}{3.258047in}}%
\pgfpathlineto{\pgfqpoint{3.561171in}{3.262305in}}%
\pgfpathlineto{\pgfqpoint{3.565429in}{3.262305in}}%
\pgfpathlineto{\pgfqpoint{3.565429in}{3.258047in}}%
\pgfpathmoveto{\pgfqpoint{3.561171in}{3.262305in}}%
\pgfpathlineto{\pgfqpoint{3.561171in}{3.262305in}}%
\pgfpathlineto{\pgfqpoint{3.561171in}{3.266563in}}%
\pgfpathlineto{\pgfqpoint{3.565429in}{3.266563in}}%
\pgfpathlineto{\pgfqpoint{3.565429in}{3.262305in}}%
\pgfpathmoveto{\pgfqpoint{3.565429in}{3.258047in}}%
\pgfpathlineto{\pgfqpoint{3.565429in}{3.258047in}}%
\pgfpathlineto{\pgfqpoint{3.565429in}{3.262305in}}%
\pgfpathlineto{\pgfqpoint{3.569687in}{3.262305in}}%
\pgfpathlineto{\pgfqpoint{3.569687in}{3.258047in}}%
\pgfpathmoveto{\pgfqpoint{3.565429in}{3.262305in}}%
\pgfpathlineto{\pgfqpoint{3.565429in}{3.262305in}}%
\pgfpathlineto{\pgfqpoint{3.565429in}{3.266563in}}%
\pgfpathlineto{\pgfqpoint{3.569687in}{3.266563in}}%
\pgfpathlineto{\pgfqpoint{3.569687in}{3.262305in}}%
\pgfpathmoveto{\pgfqpoint{3.548398in}{3.275078in}}%
\pgfpathlineto{\pgfqpoint{3.548398in}{3.275078in}}%
\pgfpathlineto{\pgfqpoint{3.548398in}{3.279336in}}%
\pgfpathlineto{\pgfqpoint{3.552655in}{3.279336in}}%
\pgfpathlineto{\pgfqpoint{3.552655in}{3.275078in}}%
\pgfpathmoveto{\pgfqpoint{3.548398in}{3.279336in}}%
\pgfpathlineto{\pgfqpoint{3.548398in}{3.279336in}}%
\pgfpathlineto{\pgfqpoint{3.548398in}{3.283594in}}%
\pgfpathlineto{\pgfqpoint{3.552655in}{3.283594in}}%
\pgfpathlineto{\pgfqpoint{3.552655in}{3.279336in}}%
\pgfpathmoveto{\pgfqpoint{3.535624in}{3.296367in}}%
\pgfpathlineto{\pgfqpoint{3.535624in}{3.296367in}}%
\pgfpathlineto{\pgfqpoint{3.535624in}{3.300625in}}%
\pgfpathlineto{\pgfqpoint{3.539882in}{3.300625in}}%
\pgfpathlineto{\pgfqpoint{3.539882in}{3.296367in}}%
\pgfpathmoveto{\pgfqpoint{3.539882in}{3.292109in}}%
\pgfpathlineto{\pgfqpoint{3.539882in}{3.292109in}}%
\pgfpathlineto{\pgfqpoint{3.539882in}{3.296367in}}%
\pgfpathlineto{\pgfqpoint{3.544140in}{3.296367in}}%
\pgfpathlineto{\pgfqpoint{3.544140in}{3.292109in}}%
\pgfpathmoveto{\pgfqpoint{3.539882in}{3.296367in}}%
\pgfpathlineto{\pgfqpoint{3.539882in}{3.296367in}}%
\pgfpathlineto{\pgfqpoint{3.539882in}{3.300625in}}%
\pgfpathlineto{\pgfqpoint{3.544140in}{3.300625in}}%
\pgfpathlineto{\pgfqpoint{3.544140in}{3.296367in}}%
\pgfpathmoveto{\pgfqpoint{3.544140in}{3.283594in}}%
\pgfpathlineto{\pgfqpoint{3.544140in}{3.283594in}}%
\pgfpathlineto{\pgfqpoint{3.544140in}{3.287851in}}%
\pgfpathlineto{\pgfqpoint{3.548398in}{3.287851in}}%
\pgfpathlineto{\pgfqpoint{3.548398in}{3.283594in}}%
\pgfpathmoveto{\pgfqpoint{3.544140in}{3.287851in}}%
\pgfpathlineto{\pgfqpoint{3.544140in}{3.287851in}}%
\pgfpathlineto{\pgfqpoint{3.544140in}{3.292109in}}%
\pgfpathlineto{\pgfqpoint{3.548398in}{3.292109in}}%
\pgfpathlineto{\pgfqpoint{3.548398in}{3.287851in}}%
\pgfpathmoveto{\pgfqpoint{3.548398in}{3.283594in}}%
\pgfpathlineto{\pgfqpoint{3.548398in}{3.283594in}}%
\pgfpathlineto{\pgfqpoint{3.548398in}{3.287851in}}%
\pgfpathlineto{\pgfqpoint{3.552655in}{3.287851in}}%
\pgfpathlineto{\pgfqpoint{3.552655in}{3.283594in}}%
\pgfpathmoveto{\pgfqpoint{3.548398in}{3.287851in}}%
\pgfpathlineto{\pgfqpoint{3.548398in}{3.287851in}}%
\pgfpathlineto{\pgfqpoint{3.548398in}{3.292109in}}%
\pgfpathlineto{\pgfqpoint{3.552655in}{3.292109in}}%
\pgfpathlineto{\pgfqpoint{3.552655in}{3.287851in}}%
\pgfpathmoveto{\pgfqpoint{3.544140in}{3.292109in}}%
\pgfpathlineto{\pgfqpoint{3.544140in}{3.292109in}}%
\pgfpathlineto{\pgfqpoint{3.544140in}{3.296367in}}%
\pgfpathlineto{\pgfqpoint{3.548398in}{3.296367in}}%
\pgfpathlineto{\pgfqpoint{3.548398in}{3.292109in}}%
\pgfpathmoveto{\pgfqpoint{3.544140in}{3.296367in}}%
\pgfpathlineto{\pgfqpoint{3.544140in}{3.296367in}}%
\pgfpathlineto{\pgfqpoint{3.544140in}{3.300625in}}%
\pgfpathlineto{\pgfqpoint{3.548398in}{3.300625in}}%
\pgfpathlineto{\pgfqpoint{3.548398in}{3.296367in}}%
\pgfpathmoveto{\pgfqpoint{3.548398in}{3.292109in}}%
\pgfpathlineto{\pgfqpoint{3.548398in}{3.292109in}}%
\pgfpathlineto{\pgfqpoint{3.548398in}{3.296367in}}%
\pgfpathlineto{\pgfqpoint{3.552655in}{3.296367in}}%
\pgfpathlineto{\pgfqpoint{3.552655in}{3.292109in}}%
\pgfpathmoveto{\pgfqpoint{3.548398in}{3.296367in}}%
\pgfpathlineto{\pgfqpoint{3.548398in}{3.296367in}}%
\pgfpathlineto{\pgfqpoint{3.548398in}{3.300625in}}%
\pgfpathlineto{\pgfqpoint{3.552655in}{3.300625in}}%
\pgfpathlineto{\pgfqpoint{3.552655in}{3.296367in}}%
\pgfpathmoveto{\pgfqpoint{3.552655in}{3.266563in}}%
\pgfpathlineto{\pgfqpoint{3.552655in}{3.266563in}}%
\pgfpathlineto{\pgfqpoint{3.552655in}{3.270820in}}%
\pgfpathlineto{\pgfqpoint{3.556913in}{3.270820in}}%
\pgfpathlineto{\pgfqpoint{3.556913in}{3.266563in}}%
\pgfpathmoveto{\pgfqpoint{3.552655in}{3.270820in}}%
\pgfpathlineto{\pgfqpoint{3.552655in}{3.270820in}}%
\pgfpathlineto{\pgfqpoint{3.552655in}{3.275078in}}%
\pgfpathlineto{\pgfqpoint{3.556913in}{3.275078in}}%
\pgfpathlineto{\pgfqpoint{3.556913in}{3.270820in}}%
\pgfpathmoveto{\pgfqpoint{3.556913in}{3.266563in}}%
\pgfpathlineto{\pgfqpoint{3.556913in}{3.266563in}}%
\pgfpathlineto{\pgfqpoint{3.556913in}{3.270820in}}%
\pgfpathlineto{\pgfqpoint{3.561171in}{3.270820in}}%
\pgfpathlineto{\pgfqpoint{3.561171in}{3.266563in}}%
\pgfpathmoveto{\pgfqpoint{3.556913in}{3.270820in}}%
\pgfpathlineto{\pgfqpoint{3.556913in}{3.270820in}}%
\pgfpathlineto{\pgfqpoint{3.556913in}{3.275078in}}%
\pgfpathlineto{\pgfqpoint{3.561171in}{3.275078in}}%
\pgfpathlineto{\pgfqpoint{3.561171in}{3.270820in}}%
\pgfpathmoveto{\pgfqpoint{3.552655in}{3.275078in}}%
\pgfpathlineto{\pgfqpoint{3.552655in}{3.275078in}}%
\pgfpathlineto{\pgfqpoint{3.552655in}{3.279336in}}%
\pgfpathlineto{\pgfqpoint{3.556913in}{3.279336in}}%
\pgfpathlineto{\pgfqpoint{3.556913in}{3.275078in}}%
\pgfpathmoveto{\pgfqpoint{3.552655in}{3.279336in}}%
\pgfpathlineto{\pgfqpoint{3.552655in}{3.279336in}}%
\pgfpathlineto{\pgfqpoint{3.552655in}{3.283594in}}%
\pgfpathlineto{\pgfqpoint{3.556913in}{3.283594in}}%
\pgfpathlineto{\pgfqpoint{3.556913in}{3.279336in}}%
\pgfpathmoveto{\pgfqpoint{3.556913in}{3.275078in}}%
\pgfpathlineto{\pgfqpoint{3.556913in}{3.275078in}}%
\pgfpathlineto{\pgfqpoint{3.556913in}{3.279336in}}%
\pgfpathlineto{\pgfqpoint{3.561171in}{3.279336in}}%
\pgfpathlineto{\pgfqpoint{3.561171in}{3.275078in}}%
\pgfpathmoveto{\pgfqpoint{3.556913in}{3.279336in}}%
\pgfpathlineto{\pgfqpoint{3.556913in}{3.279336in}}%
\pgfpathlineto{\pgfqpoint{3.556913in}{3.283594in}}%
\pgfpathlineto{\pgfqpoint{3.561171in}{3.283594in}}%
\pgfpathlineto{\pgfqpoint{3.561171in}{3.279336in}}%
\pgfpathmoveto{\pgfqpoint{3.561171in}{3.266563in}}%
\pgfpathlineto{\pgfqpoint{3.561171in}{3.266563in}}%
\pgfpathlineto{\pgfqpoint{3.561171in}{3.270820in}}%
\pgfpathlineto{\pgfqpoint{3.565429in}{3.270820in}}%
\pgfpathlineto{\pgfqpoint{3.565429in}{3.266563in}}%
\pgfpathmoveto{\pgfqpoint{3.561171in}{3.270820in}}%
\pgfpathlineto{\pgfqpoint{3.561171in}{3.270820in}}%
\pgfpathlineto{\pgfqpoint{3.561171in}{3.275078in}}%
\pgfpathlineto{\pgfqpoint{3.565429in}{3.275078in}}%
\pgfpathlineto{\pgfqpoint{3.565429in}{3.270820in}}%
\pgfpathmoveto{\pgfqpoint{3.565429in}{3.266563in}}%
\pgfpathlineto{\pgfqpoint{3.565429in}{3.266563in}}%
\pgfpathlineto{\pgfqpoint{3.565429in}{3.270820in}}%
\pgfpathlineto{\pgfqpoint{3.569687in}{3.270820in}}%
\pgfpathlineto{\pgfqpoint{3.569687in}{3.266563in}}%
\pgfpathmoveto{\pgfqpoint{3.565429in}{3.270820in}}%
\pgfpathlineto{\pgfqpoint{3.565429in}{3.270820in}}%
\pgfpathlineto{\pgfqpoint{3.565429in}{3.275078in}}%
\pgfpathlineto{\pgfqpoint{3.569687in}{3.275078in}}%
\pgfpathlineto{\pgfqpoint{3.569687in}{3.270820in}}%
\pgfpathmoveto{\pgfqpoint{3.561171in}{3.275078in}}%
\pgfpathlineto{\pgfqpoint{3.561171in}{3.275078in}}%
\pgfpathlineto{\pgfqpoint{3.561171in}{3.279336in}}%
\pgfpathlineto{\pgfqpoint{3.565429in}{3.279336in}}%
\pgfpathlineto{\pgfqpoint{3.565429in}{3.275078in}}%
\pgfpathmoveto{\pgfqpoint{3.561171in}{3.279336in}}%
\pgfpathlineto{\pgfqpoint{3.561171in}{3.279336in}}%
\pgfpathlineto{\pgfqpoint{3.561171in}{3.283594in}}%
\pgfpathlineto{\pgfqpoint{3.565429in}{3.283594in}}%
\pgfpathlineto{\pgfqpoint{3.565429in}{3.279336in}}%
\pgfpathmoveto{\pgfqpoint{3.565429in}{3.275078in}}%
\pgfpathlineto{\pgfqpoint{3.565429in}{3.275078in}}%
\pgfpathlineto{\pgfqpoint{3.565429in}{3.279336in}}%
\pgfpathlineto{\pgfqpoint{3.569687in}{3.279336in}}%
\pgfpathlineto{\pgfqpoint{3.569687in}{3.275078in}}%
\pgfpathmoveto{\pgfqpoint{3.565429in}{3.279336in}}%
\pgfpathlineto{\pgfqpoint{3.565429in}{3.279336in}}%
\pgfpathlineto{\pgfqpoint{3.565429in}{3.283594in}}%
\pgfpathlineto{\pgfqpoint{3.569687in}{3.283594in}}%
\pgfpathlineto{\pgfqpoint{3.569687in}{3.279336in}}%
\pgfpathmoveto{\pgfqpoint{3.552655in}{3.283594in}}%
\pgfpathlineto{\pgfqpoint{3.552655in}{3.283594in}}%
\pgfpathlineto{\pgfqpoint{3.552655in}{3.287851in}}%
\pgfpathlineto{\pgfqpoint{3.556913in}{3.287851in}}%
\pgfpathlineto{\pgfqpoint{3.556913in}{3.283594in}}%
\pgfpathmoveto{\pgfqpoint{3.552655in}{3.287851in}}%
\pgfpathlineto{\pgfqpoint{3.552655in}{3.287851in}}%
\pgfpathlineto{\pgfqpoint{3.552655in}{3.292109in}}%
\pgfpathlineto{\pgfqpoint{3.556913in}{3.292109in}}%
\pgfpathlineto{\pgfqpoint{3.556913in}{3.287851in}}%
\pgfpathmoveto{\pgfqpoint{3.556913in}{3.283594in}}%
\pgfpathlineto{\pgfqpoint{3.556913in}{3.283594in}}%
\pgfpathlineto{\pgfqpoint{3.556913in}{3.287851in}}%
\pgfpathlineto{\pgfqpoint{3.561171in}{3.287851in}}%
\pgfpathlineto{\pgfqpoint{3.561171in}{3.283594in}}%
\pgfpathmoveto{\pgfqpoint{3.556913in}{3.287851in}}%
\pgfpathlineto{\pgfqpoint{3.556913in}{3.287851in}}%
\pgfpathlineto{\pgfqpoint{3.556913in}{3.292109in}}%
\pgfpathlineto{\pgfqpoint{3.561171in}{3.292109in}}%
\pgfpathlineto{\pgfqpoint{3.561171in}{3.287851in}}%
\pgfpathmoveto{\pgfqpoint{3.552655in}{3.292109in}}%
\pgfpathlineto{\pgfqpoint{3.552655in}{3.292109in}}%
\pgfpathlineto{\pgfqpoint{3.552655in}{3.296367in}}%
\pgfpathlineto{\pgfqpoint{3.556913in}{3.296367in}}%
\pgfpathlineto{\pgfqpoint{3.556913in}{3.292109in}}%
\pgfpathmoveto{\pgfqpoint{3.552655in}{3.296367in}}%
\pgfpathlineto{\pgfqpoint{3.552655in}{3.296367in}}%
\pgfpathlineto{\pgfqpoint{3.552655in}{3.300625in}}%
\pgfpathlineto{\pgfqpoint{3.556913in}{3.300625in}}%
\pgfpathlineto{\pgfqpoint{3.556913in}{3.296367in}}%
\pgfpathmoveto{\pgfqpoint{3.556913in}{3.292109in}}%
\pgfpathlineto{\pgfqpoint{3.556913in}{3.292109in}}%
\pgfpathlineto{\pgfqpoint{3.556913in}{3.296367in}}%
\pgfpathlineto{\pgfqpoint{3.561171in}{3.296367in}}%
\pgfpathlineto{\pgfqpoint{3.561171in}{3.292109in}}%
\pgfpathmoveto{\pgfqpoint{3.556913in}{3.296367in}}%
\pgfpathlineto{\pgfqpoint{3.556913in}{3.296367in}}%
\pgfpathlineto{\pgfqpoint{3.556913in}{3.300625in}}%
\pgfpathlineto{\pgfqpoint{3.561171in}{3.300625in}}%
\pgfpathlineto{\pgfqpoint{3.561171in}{3.296367in}}%
\pgfpathmoveto{\pgfqpoint{3.561171in}{3.283594in}}%
\pgfpathlineto{\pgfqpoint{3.561171in}{3.283594in}}%
\pgfpathlineto{\pgfqpoint{3.561171in}{3.287851in}}%
\pgfpathlineto{\pgfqpoint{3.565429in}{3.287851in}}%
\pgfpathlineto{\pgfqpoint{3.565429in}{3.283594in}}%
\pgfpathmoveto{\pgfqpoint{3.561171in}{3.287851in}}%
\pgfpathlineto{\pgfqpoint{3.561171in}{3.287851in}}%
\pgfpathlineto{\pgfqpoint{3.561171in}{3.292109in}}%
\pgfpathlineto{\pgfqpoint{3.565429in}{3.292109in}}%
\pgfpathlineto{\pgfqpoint{3.565429in}{3.287851in}}%
\pgfpathmoveto{\pgfqpoint{3.565429in}{3.283594in}}%
\pgfpathlineto{\pgfqpoint{3.565429in}{3.283594in}}%
\pgfpathlineto{\pgfqpoint{3.565429in}{3.287851in}}%
\pgfpathlineto{\pgfqpoint{3.569687in}{3.287851in}}%
\pgfpathlineto{\pgfqpoint{3.569687in}{3.283594in}}%
\pgfpathmoveto{\pgfqpoint{3.561171in}{3.292109in}}%
\pgfpathlineto{\pgfqpoint{3.561171in}{3.292109in}}%
\pgfpathlineto{\pgfqpoint{3.561171in}{3.296367in}}%
\pgfpathlineto{\pgfqpoint{3.565429in}{3.296367in}}%
\pgfpathlineto{\pgfqpoint{3.565429in}{3.292109in}}%
\pgfpathmoveto{\pgfqpoint{3.514334in}{3.330429in}}%
\pgfpathlineto{\pgfqpoint{3.514334in}{3.330429in}}%
\pgfpathlineto{\pgfqpoint{3.514334in}{3.334687in}}%
\pgfpathlineto{\pgfqpoint{3.518592in}{3.334687in}}%
\pgfpathlineto{\pgfqpoint{3.518592in}{3.330429in}}%
\pgfpathmoveto{\pgfqpoint{3.531366in}{3.304883in}}%
\pgfpathlineto{\pgfqpoint{3.531366in}{3.304883in}}%
\pgfpathlineto{\pgfqpoint{3.531366in}{3.309140in}}%
\pgfpathlineto{\pgfqpoint{3.535624in}{3.309140in}}%
\pgfpathlineto{\pgfqpoint{3.535624in}{3.304883in}}%
\pgfpathmoveto{\pgfqpoint{3.527108in}{3.313398in}}%
\pgfpathlineto{\pgfqpoint{3.527108in}{3.313398in}}%
\pgfpathlineto{\pgfqpoint{3.527108in}{3.317656in}}%
\pgfpathlineto{\pgfqpoint{3.531366in}{3.317656in}}%
\pgfpathlineto{\pgfqpoint{3.531366in}{3.313398in}}%
\pgfpathmoveto{\pgfqpoint{3.531366in}{3.309140in}}%
\pgfpathlineto{\pgfqpoint{3.531366in}{3.309140in}}%
\pgfpathlineto{\pgfqpoint{3.531366in}{3.313398in}}%
\pgfpathlineto{\pgfqpoint{3.535624in}{3.313398in}}%
\pgfpathlineto{\pgfqpoint{3.535624in}{3.309140in}}%
\pgfpathmoveto{\pgfqpoint{3.531366in}{3.313398in}}%
\pgfpathlineto{\pgfqpoint{3.531366in}{3.313398in}}%
\pgfpathlineto{\pgfqpoint{3.531366in}{3.317656in}}%
\pgfpathlineto{\pgfqpoint{3.535624in}{3.317656in}}%
\pgfpathlineto{\pgfqpoint{3.535624in}{3.313398in}}%
\pgfpathmoveto{\pgfqpoint{3.522850in}{3.317656in}}%
\pgfpathlineto{\pgfqpoint{3.522850in}{3.317656in}}%
\pgfpathlineto{\pgfqpoint{3.522850in}{3.321914in}}%
\pgfpathlineto{\pgfqpoint{3.527108in}{3.321914in}}%
\pgfpathlineto{\pgfqpoint{3.527108in}{3.317656in}}%
\pgfpathmoveto{\pgfqpoint{3.522850in}{3.321914in}}%
\pgfpathlineto{\pgfqpoint{3.522850in}{3.321914in}}%
\pgfpathlineto{\pgfqpoint{3.522850in}{3.326172in}}%
\pgfpathlineto{\pgfqpoint{3.527108in}{3.326172in}}%
\pgfpathlineto{\pgfqpoint{3.527108in}{3.321914in}}%
\pgfpathmoveto{\pgfqpoint{3.518592in}{3.326172in}}%
\pgfpathlineto{\pgfqpoint{3.518592in}{3.326172in}}%
\pgfpathlineto{\pgfqpoint{3.518592in}{3.330429in}}%
\pgfpathlineto{\pgfqpoint{3.522850in}{3.330429in}}%
\pgfpathlineto{\pgfqpoint{3.522850in}{3.326172in}}%
\pgfpathmoveto{\pgfqpoint{3.518592in}{3.330429in}}%
\pgfpathlineto{\pgfqpoint{3.518592in}{3.330429in}}%
\pgfpathlineto{\pgfqpoint{3.518592in}{3.334687in}}%
\pgfpathlineto{\pgfqpoint{3.522850in}{3.334687in}}%
\pgfpathlineto{\pgfqpoint{3.522850in}{3.330429in}}%
\pgfpathmoveto{\pgfqpoint{3.522850in}{3.326172in}}%
\pgfpathlineto{\pgfqpoint{3.522850in}{3.326172in}}%
\pgfpathlineto{\pgfqpoint{3.522850in}{3.330429in}}%
\pgfpathlineto{\pgfqpoint{3.527108in}{3.330429in}}%
\pgfpathlineto{\pgfqpoint{3.527108in}{3.326172in}}%
\pgfpathmoveto{\pgfqpoint{3.522850in}{3.330429in}}%
\pgfpathlineto{\pgfqpoint{3.522850in}{3.330429in}}%
\pgfpathlineto{\pgfqpoint{3.522850in}{3.334687in}}%
\pgfpathlineto{\pgfqpoint{3.527108in}{3.334687in}}%
\pgfpathlineto{\pgfqpoint{3.527108in}{3.330429in}}%
\pgfpathmoveto{\pgfqpoint{3.527108in}{3.317656in}}%
\pgfpathlineto{\pgfqpoint{3.527108in}{3.317656in}}%
\pgfpathlineto{\pgfqpoint{3.527108in}{3.321914in}}%
\pgfpathlineto{\pgfqpoint{3.531366in}{3.321914in}}%
\pgfpathlineto{\pgfqpoint{3.531366in}{3.317656in}}%
\pgfpathmoveto{\pgfqpoint{3.527108in}{3.321914in}}%
\pgfpathlineto{\pgfqpoint{3.527108in}{3.321914in}}%
\pgfpathlineto{\pgfqpoint{3.527108in}{3.326172in}}%
\pgfpathlineto{\pgfqpoint{3.531366in}{3.326172in}}%
\pgfpathlineto{\pgfqpoint{3.531366in}{3.321914in}}%
\pgfpathmoveto{\pgfqpoint{3.531366in}{3.317656in}}%
\pgfpathlineto{\pgfqpoint{3.531366in}{3.317656in}}%
\pgfpathlineto{\pgfqpoint{3.531366in}{3.321914in}}%
\pgfpathlineto{\pgfqpoint{3.535624in}{3.321914in}}%
\pgfpathlineto{\pgfqpoint{3.535624in}{3.317656in}}%
\pgfpathmoveto{\pgfqpoint{3.531366in}{3.321914in}}%
\pgfpathlineto{\pgfqpoint{3.531366in}{3.321914in}}%
\pgfpathlineto{\pgfqpoint{3.531366in}{3.326172in}}%
\pgfpathlineto{\pgfqpoint{3.535624in}{3.326172in}}%
\pgfpathlineto{\pgfqpoint{3.535624in}{3.321914in}}%
\pgfpathmoveto{\pgfqpoint{3.527108in}{3.326172in}}%
\pgfpathlineto{\pgfqpoint{3.527108in}{3.326172in}}%
\pgfpathlineto{\pgfqpoint{3.527108in}{3.330429in}}%
\pgfpathlineto{\pgfqpoint{3.531366in}{3.330429in}}%
\pgfpathlineto{\pgfqpoint{3.531366in}{3.326172in}}%
\pgfpathmoveto{\pgfqpoint{3.527108in}{3.330429in}}%
\pgfpathlineto{\pgfqpoint{3.527108in}{3.330429in}}%
\pgfpathlineto{\pgfqpoint{3.527108in}{3.334687in}}%
\pgfpathlineto{\pgfqpoint{3.531366in}{3.334687in}}%
\pgfpathlineto{\pgfqpoint{3.531366in}{3.330429in}}%
\pgfpathmoveto{\pgfqpoint{3.531366in}{3.326172in}}%
\pgfpathlineto{\pgfqpoint{3.531366in}{3.326172in}}%
\pgfpathlineto{\pgfqpoint{3.531366in}{3.330429in}}%
\pgfpathlineto{\pgfqpoint{3.535624in}{3.330429in}}%
\pgfpathlineto{\pgfqpoint{3.535624in}{3.326172in}}%
\pgfpathmoveto{\pgfqpoint{3.531366in}{3.330429in}}%
\pgfpathlineto{\pgfqpoint{3.531366in}{3.330429in}}%
\pgfpathlineto{\pgfqpoint{3.531366in}{3.334687in}}%
\pgfpathlineto{\pgfqpoint{3.535624in}{3.334687in}}%
\pgfpathlineto{\pgfqpoint{3.535624in}{3.330429in}}%
\pgfpathmoveto{\pgfqpoint{3.505818in}{3.338945in}}%
\pgfpathlineto{\pgfqpoint{3.505818in}{3.338945in}}%
\pgfpathlineto{\pgfqpoint{3.505818in}{3.343203in}}%
\pgfpathlineto{\pgfqpoint{3.510076in}{3.343203in}}%
\pgfpathlineto{\pgfqpoint{3.510076in}{3.338945in}}%
\pgfpathmoveto{\pgfqpoint{3.501560in}{3.343203in}}%
\pgfpathlineto{\pgfqpoint{3.501560in}{3.343203in}}%
\pgfpathlineto{\pgfqpoint{3.501560in}{3.347461in}}%
\pgfpathlineto{\pgfqpoint{3.505818in}{3.347461in}}%
\pgfpathlineto{\pgfqpoint{3.505818in}{3.343203in}}%
\pgfpathmoveto{\pgfqpoint{3.501560in}{3.347461in}}%
\pgfpathlineto{\pgfqpoint{3.501560in}{3.347461in}}%
\pgfpathlineto{\pgfqpoint{3.501560in}{3.351718in}}%
\pgfpathlineto{\pgfqpoint{3.505818in}{3.351718in}}%
\pgfpathlineto{\pgfqpoint{3.505818in}{3.347461in}}%
\pgfpathmoveto{\pgfqpoint{3.505818in}{3.343203in}}%
\pgfpathlineto{\pgfqpoint{3.505818in}{3.343203in}}%
\pgfpathlineto{\pgfqpoint{3.505818in}{3.347461in}}%
\pgfpathlineto{\pgfqpoint{3.510076in}{3.347461in}}%
\pgfpathlineto{\pgfqpoint{3.510076in}{3.343203in}}%
\pgfpathmoveto{\pgfqpoint{3.505818in}{3.347461in}}%
\pgfpathlineto{\pgfqpoint{3.505818in}{3.347461in}}%
\pgfpathlineto{\pgfqpoint{3.505818in}{3.351718in}}%
\pgfpathlineto{\pgfqpoint{3.510076in}{3.351718in}}%
\pgfpathlineto{\pgfqpoint{3.510076in}{3.347461in}}%
\pgfpathmoveto{\pgfqpoint{3.510076in}{3.334687in}}%
\pgfpathlineto{\pgfqpoint{3.510076in}{3.334687in}}%
\pgfpathlineto{\pgfqpoint{3.510076in}{3.338945in}}%
\pgfpathlineto{\pgfqpoint{3.514334in}{3.338945in}}%
\pgfpathlineto{\pgfqpoint{3.514334in}{3.334687in}}%
\pgfpathmoveto{\pgfqpoint{3.510076in}{3.338945in}}%
\pgfpathlineto{\pgfqpoint{3.510076in}{3.338945in}}%
\pgfpathlineto{\pgfqpoint{3.510076in}{3.343203in}}%
\pgfpathlineto{\pgfqpoint{3.514334in}{3.343203in}}%
\pgfpathlineto{\pgfqpoint{3.514334in}{3.338945in}}%
\pgfpathmoveto{\pgfqpoint{3.514334in}{3.334687in}}%
\pgfpathlineto{\pgfqpoint{3.514334in}{3.334687in}}%
\pgfpathlineto{\pgfqpoint{3.514334in}{3.338945in}}%
\pgfpathlineto{\pgfqpoint{3.518592in}{3.338945in}}%
\pgfpathlineto{\pgfqpoint{3.518592in}{3.334687in}}%
\pgfpathmoveto{\pgfqpoint{3.514334in}{3.338945in}}%
\pgfpathlineto{\pgfqpoint{3.514334in}{3.338945in}}%
\pgfpathlineto{\pgfqpoint{3.514334in}{3.343203in}}%
\pgfpathlineto{\pgfqpoint{3.518592in}{3.343203in}}%
\pgfpathlineto{\pgfqpoint{3.518592in}{3.338945in}}%
\pgfpathmoveto{\pgfqpoint{3.510076in}{3.343203in}}%
\pgfpathlineto{\pgfqpoint{3.510076in}{3.343203in}}%
\pgfpathlineto{\pgfqpoint{3.510076in}{3.347461in}}%
\pgfpathlineto{\pgfqpoint{3.514334in}{3.347461in}}%
\pgfpathlineto{\pgfqpoint{3.514334in}{3.343203in}}%
\pgfpathmoveto{\pgfqpoint{3.510076in}{3.347461in}}%
\pgfpathlineto{\pgfqpoint{3.510076in}{3.347461in}}%
\pgfpathlineto{\pgfqpoint{3.510076in}{3.351718in}}%
\pgfpathlineto{\pgfqpoint{3.514334in}{3.351718in}}%
\pgfpathlineto{\pgfqpoint{3.514334in}{3.347461in}}%
\pgfpathmoveto{\pgfqpoint{3.514334in}{3.343203in}}%
\pgfpathlineto{\pgfqpoint{3.514334in}{3.343203in}}%
\pgfpathlineto{\pgfqpoint{3.514334in}{3.347461in}}%
\pgfpathlineto{\pgfqpoint{3.518592in}{3.347461in}}%
\pgfpathlineto{\pgfqpoint{3.518592in}{3.343203in}}%
\pgfpathmoveto{\pgfqpoint{3.514334in}{3.347461in}}%
\pgfpathlineto{\pgfqpoint{3.514334in}{3.347461in}}%
\pgfpathlineto{\pgfqpoint{3.514334in}{3.351718in}}%
\pgfpathlineto{\pgfqpoint{3.518592in}{3.351718in}}%
\pgfpathlineto{\pgfqpoint{3.518592in}{3.347461in}}%
\pgfpathmoveto{\pgfqpoint{3.501560in}{3.351718in}}%
\pgfpathlineto{\pgfqpoint{3.501560in}{3.351718in}}%
\pgfpathlineto{\pgfqpoint{3.501560in}{3.355976in}}%
\pgfpathlineto{\pgfqpoint{3.505818in}{3.355976in}}%
\pgfpathlineto{\pgfqpoint{3.505818in}{3.351718in}}%
\pgfpathmoveto{\pgfqpoint{3.501560in}{3.355976in}}%
\pgfpathlineto{\pgfqpoint{3.501560in}{3.355976in}}%
\pgfpathlineto{\pgfqpoint{3.501560in}{3.360234in}}%
\pgfpathlineto{\pgfqpoint{3.505818in}{3.360234in}}%
\pgfpathlineto{\pgfqpoint{3.505818in}{3.355976in}}%
\pgfpathmoveto{\pgfqpoint{3.505818in}{3.351718in}}%
\pgfpathlineto{\pgfqpoint{3.505818in}{3.351718in}}%
\pgfpathlineto{\pgfqpoint{3.505818in}{3.355976in}}%
\pgfpathlineto{\pgfqpoint{3.510076in}{3.355976in}}%
\pgfpathlineto{\pgfqpoint{3.510076in}{3.351718in}}%
\pgfpathmoveto{\pgfqpoint{3.505818in}{3.355976in}}%
\pgfpathlineto{\pgfqpoint{3.505818in}{3.355976in}}%
\pgfpathlineto{\pgfqpoint{3.505818in}{3.360234in}}%
\pgfpathlineto{\pgfqpoint{3.510076in}{3.360234in}}%
\pgfpathlineto{\pgfqpoint{3.510076in}{3.355976in}}%
\pgfpathmoveto{\pgfqpoint{3.501560in}{3.360234in}}%
\pgfpathlineto{\pgfqpoint{3.501560in}{3.360234in}}%
\pgfpathlineto{\pgfqpoint{3.501560in}{3.364492in}}%
\pgfpathlineto{\pgfqpoint{3.505818in}{3.364492in}}%
\pgfpathlineto{\pgfqpoint{3.505818in}{3.360234in}}%
\pgfpathmoveto{\pgfqpoint{3.501560in}{3.364492in}}%
\pgfpathlineto{\pgfqpoint{3.501560in}{3.364492in}}%
\pgfpathlineto{\pgfqpoint{3.501560in}{3.368750in}}%
\pgfpathlineto{\pgfqpoint{3.505818in}{3.368750in}}%
\pgfpathlineto{\pgfqpoint{3.505818in}{3.364492in}}%
\pgfpathmoveto{\pgfqpoint{3.505818in}{3.360234in}}%
\pgfpathlineto{\pgfqpoint{3.505818in}{3.360234in}}%
\pgfpathlineto{\pgfqpoint{3.505818in}{3.364492in}}%
\pgfpathlineto{\pgfqpoint{3.510076in}{3.364492in}}%
\pgfpathlineto{\pgfqpoint{3.510076in}{3.360234in}}%
\pgfpathmoveto{\pgfqpoint{3.505818in}{3.364492in}}%
\pgfpathlineto{\pgfqpoint{3.505818in}{3.364492in}}%
\pgfpathlineto{\pgfqpoint{3.505818in}{3.368750in}}%
\pgfpathlineto{\pgfqpoint{3.510076in}{3.368750in}}%
\pgfpathlineto{\pgfqpoint{3.510076in}{3.364492in}}%
\pgfpathmoveto{\pgfqpoint{3.510076in}{3.351718in}}%
\pgfpathlineto{\pgfqpoint{3.510076in}{3.351718in}}%
\pgfpathlineto{\pgfqpoint{3.510076in}{3.355976in}}%
\pgfpathlineto{\pgfqpoint{3.514334in}{3.355976in}}%
\pgfpathlineto{\pgfqpoint{3.514334in}{3.351718in}}%
\pgfpathmoveto{\pgfqpoint{3.510076in}{3.355976in}}%
\pgfpathlineto{\pgfqpoint{3.510076in}{3.355976in}}%
\pgfpathlineto{\pgfqpoint{3.510076in}{3.360234in}}%
\pgfpathlineto{\pgfqpoint{3.514334in}{3.360234in}}%
\pgfpathlineto{\pgfqpoint{3.514334in}{3.355976in}}%
\pgfpathmoveto{\pgfqpoint{3.514334in}{3.351718in}}%
\pgfpathlineto{\pgfqpoint{3.514334in}{3.351718in}}%
\pgfpathlineto{\pgfqpoint{3.514334in}{3.355976in}}%
\pgfpathlineto{\pgfqpoint{3.518592in}{3.355976in}}%
\pgfpathlineto{\pgfqpoint{3.518592in}{3.351718in}}%
\pgfpathmoveto{\pgfqpoint{3.514334in}{3.355976in}}%
\pgfpathlineto{\pgfqpoint{3.514334in}{3.355976in}}%
\pgfpathlineto{\pgfqpoint{3.514334in}{3.360234in}}%
\pgfpathlineto{\pgfqpoint{3.518592in}{3.360234in}}%
\pgfpathlineto{\pgfqpoint{3.518592in}{3.355976in}}%
\pgfpathmoveto{\pgfqpoint{3.510076in}{3.360234in}}%
\pgfpathlineto{\pgfqpoint{3.510076in}{3.360234in}}%
\pgfpathlineto{\pgfqpoint{3.510076in}{3.364492in}}%
\pgfpathlineto{\pgfqpoint{3.514334in}{3.364492in}}%
\pgfpathlineto{\pgfqpoint{3.514334in}{3.360234in}}%
\pgfpathmoveto{\pgfqpoint{3.510076in}{3.364492in}}%
\pgfpathlineto{\pgfqpoint{3.510076in}{3.364492in}}%
\pgfpathlineto{\pgfqpoint{3.510076in}{3.368750in}}%
\pgfpathlineto{\pgfqpoint{3.514334in}{3.368750in}}%
\pgfpathlineto{\pgfqpoint{3.514334in}{3.364492in}}%
\pgfpathmoveto{\pgfqpoint{3.514334in}{3.360234in}}%
\pgfpathlineto{\pgfqpoint{3.514334in}{3.360234in}}%
\pgfpathlineto{\pgfqpoint{3.514334in}{3.364492in}}%
\pgfpathlineto{\pgfqpoint{3.518592in}{3.364492in}}%
\pgfpathlineto{\pgfqpoint{3.518592in}{3.360234in}}%
\pgfpathmoveto{\pgfqpoint{3.514334in}{3.364492in}}%
\pgfpathlineto{\pgfqpoint{3.514334in}{3.364492in}}%
\pgfpathlineto{\pgfqpoint{3.514334in}{3.368750in}}%
\pgfpathlineto{\pgfqpoint{3.518592in}{3.368750in}}%
\pgfpathlineto{\pgfqpoint{3.518592in}{3.364492in}}%
\pgfpathmoveto{\pgfqpoint{3.518592in}{3.334687in}}%
\pgfpathlineto{\pgfqpoint{3.518592in}{3.334687in}}%
\pgfpathlineto{\pgfqpoint{3.518592in}{3.338945in}}%
\pgfpathlineto{\pgfqpoint{3.522850in}{3.338945in}}%
\pgfpathlineto{\pgfqpoint{3.522850in}{3.334687in}}%
\pgfpathmoveto{\pgfqpoint{3.518592in}{3.338945in}}%
\pgfpathlineto{\pgfqpoint{3.518592in}{3.338945in}}%
\pgfpathlineto{\pgfqpoint{3.518592in}{3.343203in}}%
\pgfpathlineto{\pgfqpoint{3.522850in}{3.343203in}}%
\pgfpathlineto{\pgfqpoint{3.522850in}{3.338945in}}%
\pgfpathmoveto{\pgfqpoint{3.522850in}{3.334687in}}%
\pgfpathlineto{\pgfqpoint{3.522850in}{3.334687in}}%
\pgfpathlineto{\pgfqpoint{3.522850in}{3.338945in}}%
\pgfpathlineto{\pgfqpoint{3.527108in}{3.338945in}}%
\pgfpathlineto{\pgfqpoint{3.527108in}{3.334687in}}%
\pgfpathmoveto{\pgfqpoint{3.522850in}{3.338945in}}%
\pgfpathlineto{\pgfqpoint{3.522850in}{3.338945in}}%
\pgfpathlineto{\pgfqpoint{3.522850in}{3.343203in}}%
\pgfpathlineto{\pgfqpoint{3.527108in}{3.343203in}}%
\pgfpathlineto{\pgfqpoint{3.527108in}{3.338945in}}%
\pgfpathmoveto{\pgfqpoint{3.518592in}{3.343203in}}%
\pgfpathlineto{\pgfqpoint{3.518592in}{3.343203in}}%
\pgfpathlineto{\pgfqpoint{3.518592in}{3.347461in}}%
\pgfpathlineto{\pgfqpoint{3.522850in}{3.347461in}}%
\pgfpathlineto{\pgfqpoint{3.522850in}{3.343203in}}%
\pgfpathmoveto{\pgfqpoint{3.518592in}{3.347461in}}%
\pgfpathlineto{\pgfqpoint{3.518592in}{3.347461in}}%
\pgfpathlineto{\pgfqpoint{3.518592in}{3.351718in}}%
\pgfpathlineto{\pgfqpoint{3.522850in}{3.351718in}}%
\pgfpathlineto{\pgfqpoint{3.522850in}{3.347461in}}%
\pgfpathmoveto{\pgfqpoint{3.522850in}{3.343203in}}%
\pgfpathlineto{\pgfqpoint{3.522850in}{3.343203in}}%
\pgfpathlineto{\pgfqpoint{3.522850in}{3.347461in}}%
\pgfpathlineto{\pgfqpoint{3.527108in}{3.347461in}}%
\pgfpathlineto{\pgfqpoint{3.527108in}{3.343203in}}%
\pgfpathmoveto{\pgfqpoint{3.522850in}{3.347461in}}%
\pgfpathlineto{\pgfqpoint{3.522850in}{3.347461in}}%
\pgfpathlineto{\pgfqpoint{3.522850in}{3.351718in}}%
\pgfpathlineto{\pgfqpoint{3.527108in}{3.351718in}}%
\pgfpathlineto{\pgfqpoint{3.527108in}{3.347461in}}%
\pgfpathmoveto{\pgfqpoint{3.527108in}{3.334687in}}%
\pgfpathlineto{\pgfqpoint{3.527108in}{3.334687in}}%
\pgfpathlineto{\pgfqpoint{3.527108in}{3.338945in}}%
\pgfpathlineto{\pgfqpoint{3.531366in}{3.338945in}}%
\pgfpathlineto{\pgfqpoint{3.531366in}{3.334687in}}%
\pgfpathmoveto{\pgfqpoint{3.527108in}{3.338945in}}%
\pgfpathlineto{\pgfqpoint{3.527108in}{3.338945in}}%
\pgfpathlineto{\pgfqpoint{3.527108in}{3.343203in}}%
\pgfpathlineto{\pgfqpoint{3.531366in}{3.343203in}}%
\pgfpathlineto{\pgfqpoint{3.531366in}{3.338945in}}%
\pgfpathmoveto{\pgfqpoint{3.531366in}{3.334687in}}%
\pgfpathlineto{\pgfqpoint{3.531366in}{3.334687in}}%
\pgfpathlineto{\pgfqpoint{3.531366in}{3.338945in}}%
\pgfpathlineto{\pgfqpoint{3.535624in}{3.338945in}}%
\pgfpathlineto{\pgfqpoint{3.535624in}{3.334687in}}%
\pgfpathmoveto{\pgfqpoint{3.531366in}{3.338945in}}%
\pgfpathlineto{\pgfqpoint{3.531366in}{3.338945in}}%
\pgfpathlineto{\pgfqpoint{3.531366in}{3.343203in}}%
\pgfpathlineto{\pgfqpoint{3.535624in}{3.343203in}}%
\pgfpathlineto{\pgfqpoint{3.535624in}{3.338945in}}%
\pgfpathmoveto{\pgfqpoint{3.527108in}{3.343203in}}%
\pgfpathlineto{\pgfqpoint{3.527108in}{3.343203in}}%
\pgfpathlineto{\pgfqpoint{3.527108in}{3.347461in}}%
\pgfpathlineto{\pgfqpoint{3.531366in}{3.347461in}}%
\pgfpathlineto{\pgfqpoint{3.531366in}{3.343203in}}%
\pgfpathmoveto{\pgfqpoint{3.527108in}{3.347461in}}%
\pgfpathlineto{\pgfqpoint{3.527108in}{3.347461in}}%
\pgfpathlineto{\pgfqpoint{3.527108in}{3.351718in}}%
\pgfpathlineto{\pgfqpoint{3.531366in}{3.351718in}}%
\pgfpathlineto{\pgfqpoint{3.531366in}{3.347461in}}%
\pgfpathmoveto{\pgfqpoint{3.531366in}{3.343203in}}%
\pgfpathlineto{\pgfqpoint{3.531366in}{3.343203in}}%
\pgfpathlineto{\pgfqpoint{3.531366in}{3.347461in}}%
\pgfpathlineto{\pgfqpoint{3.535624in}{3.347461in}}%
\pgfpathlineto{\pgfqpoint{3.535624in}{3.343203in}}%
\pgfpathmoveto{\pgfqpoint{3.531366in}{3.347461in}}%
\pgfpathlineto{\pgfqpoint{3.531366in}{3.347461in}}%
\pgfpathlineto{\pgfqpoint{3.531366in}{3.351718in}}%
\pgfpathlineto{\pgfqpoint{3.535624in}{3.351718in}}%
\pgfpathlineto{\pgfqpoint{3.535624in}{3.347461in}}%
\pgfpathmoveto{\pgfqpoint{3.518592in}{3.351718in}}%
\pgfpathlineto{\pgfqpoint{3.518592in}{3.351718in}}%
\pgfpathlineto{\pgfqpoint{3.518592in}{3.355976in}}%
\pgfpathlineto{\pgfqpoint{3.522850in}{3.355976in}}%
\pgfpathlineto{\pgfqpoint{3.522850in}{3.351718in}}%
\pgfpathmoveto{\pgfqpoint{3.518592in}{3.355976in}}%
\pgfpathlineto{\pgfqpoint{3.518592in}{3.355976in}}%
\pgfpathlineto{\pgfqpoint{3.518592in}{3.360234in}}%
\pgfpathlineto{\pgfqpoint{3.522850in}{3.360234in}}%
\pgfpathlineto{\pgfqpoint{3.522850in}{3.355976in}}%
\pgfpathmoveto{\pgfqpoint{3.522850in}{3.351718in}}%
\pgfpathlineto{\pgfqpoint{3.522850in}{3.351718in}}%
\pgfpathlineto{\pgfqpoint{3.522850in}{3.355976in}}%
\pgfpathlineto{\pgfqpoint{3.527108in}{3.355976in}}%
\pgfpathlineto{\pgfqpoint{3.527108in}{3.351718in}}%
\pgfpathmoveto{\pgfqpoint{3.522850in}{3.355976in}}%
\pgfpathlineto{\pgfqpoint{3.522850in}{3.355976in}}%
\pgfpathlineto{\pgfqpoint{3.522850in}{3.360234in}}%
\pgfpathlineto{\pgfqpoint{3.527108in}{3.360234in}}%
\pgfpathlineto{\pgfqpoint{3.527108in}{3.355976in}}%
\pgfpathmoveto{\pgfqpoint{3.518592in}{3.360234in}}%
\pgfpathlineto{\pgfqpoint{3.518592in}{3.360234in}}%
\pgfpathlineto{\pgfqpoint{3.518592in}{3.364492in}}%
\pgfpathlineto{\pgfqpoint{3.522850in}{3.364492in}}%
\pgfpathlineto{\pgfqpoint{3.522850in}{3.360234in}}%
\pgfpathmoveto{\pgfqpoint{3.518592in}{3.364492in}}%
\pgfpathlineto{\pgfqpoint{3.518592in}{3.364492in}}%
\pgfpathlineto{\pgfqpoint{3.518592in}{3.368750in}}%
\pgfpathlineto{\pgfqpoint{3.522850in}{3.368750in}}%
\pgfpathlineto{\pgfqpoint{3.522850in}{3.364492in}}%
\pgfpathmoveto{\pgfqpoint{3.522850in}{3.360234in}}%
\pgfpathlineto{\pgfqpoint{3.522850in}{3.360234in}}%
\pgfpathlineto{\pgfqpoint{3.522850in}{3.364492in}}%
\pgfpathlineto{\pgfqpoint{3.527108in}{3.364492in}}%
\pgfpathlineto{\pgfqpoint{3.527108in}{3.360234in}}%
\pgfpathmoveto{\pgfqpoint{3.527108in}{3.351718in}}%
\pgfpathlineto{\pgfqpoint{3.527108in}{3.351718in}}%
\pgfpathlineto{\pgfqpoint{3.527108in}{3.355976in}}%
\pgfpathlineto{\pgfqpoint{3.531366in}{3.355976in}}%
\pgfpathlineto{\pgfqpoint{3.531366in}{3.351718in}}%
\pgfpathmoveto{\pgfqpoint{3.535624in}{3.300625in}}%
\pgfpathlineto{\pgfqpoint{3.535624in}{3.300625in}}%
\pgfpathlineto{\pgfqpoint{3.535624in}{3.304883in}}%
\pgfpathlineto{\pgfqpoint{3.539882in}{3.304883in}}%
\pgfpathlineto{\pgfqpoint{3.539882in}{3.300625in}}%
\pgfpathmoveto{\pgfqpoint{3.535624in}{3.304883in}}%
\pgfpathlineto{\pgfqpoint{3.535624in}{3.304883in}}%
\pgfpathlineto{\pgfqpoint{3.535624in}{3.309140in}}%
\pgfpathlineto{\pgfqpoint{3.539882in}{3.309140in}}%
\pgfpathlineto{\pgfqpoint{3.539882in}{3.304883in}}%
\pgfpathmoveto{\pgfqpoint{3.539882in}{3.300625in}}%
\pgfpathlineto{\pgfqpoint{3.539882in}{3.300625in}}%
\pgfpathlineto{\pgfqpoint{3.539882in}{3.304883in}}%
\pgfpathlineto{\pgfqpoint{3.544140in}{3.304883in}}%
\pgfpathlineto{\pgfqpoint{3.544140in}{3.300625in}}%
\pgfpathmoveto{\pgfqpoint{3.539882in}{3.304883in}}%
\pgfpathlineto{\pgfqpoint{3.539882in}{3.304883in}}%
\pgfpathlineto{\pgfqpoint{3.539882in}{3.309140in}}%
\pgfpathlineto{\pgfqpoint{3.544140in}{3.309140in}}%
\pgfpathlineto{\pgfqpoint{3.544140in}{3.304883in}}%
\pgfpathmoveto{\pgfqpoint{3.535624in}{3.309140in}}%
\pgfpathlineto{\pgfqpoint{3.535624in}{3.309140in}}%
\pgfpathlineto{\pgfqpoint{3.535624in}{3.313398in}}%
\pgfpathlineto{\pgfqpoint{3.539882in}{3.313398in}}%
\pgfpathlineto{\pgfqpoint{3.539882in}{3.309140in}}%
\pgfpathmoveto{\pgfqpoint{3.535624in}{3.313398in}}%
\pgfpathlineto{\pgfqpoint{3.535624in}{3.313398in}}%
\pgfpathlineto{\pgfqpoint{3.535624in}{3.317656in}}%
\pgfpathlineto{\pgfqpoint{3.539882in}{3.317656in}}%
\pgfpathlineto{\pgfqpoint{3.539882in}{3.313398in}}%
\pgfpathmoveto{\pgfqpoint{3.539882in}{3.309140in}}%
\pgfpathlineto{\pgfqpoint{3.539882in}{3.309140in}}%
\pgfpathlineto{\pgfqpoint{3.539882in}{3.313398in}}%
\pgfpathlineto{\pgfqpoint{3.544140in}{3.313398in}}%
\pgfpathlineto{\pgfqpoint{3.544140in}{3.309140in}}%
\pgfpathmoveto{\pgfqpoint{3.539882in}{3.313398in}}%
\pgfpathlineto{\pgfqpoint{3.539882in}{3.313398in}}%
\pgfpathlineto{\pgfqpoint{3.539882in}{3.317656in}}%
\pgfpathlineto{\pgfqpoint{3.544140in}{3.317656in}}%
\pgfpathlineto{\pgfqpoint{3.544140in}{3.313398in}}%
\pgfpathmoveto{\pgfqpoint{3.544140in}{3.300625in}}%
\pgfpathlineto{\pgfqpoint{3.544140in}{3.300625in}}%
\pgfpathlineto{\pgfqpoint{3.544140in}{3.304883in}}%
\pgfpathlineto{\pgfqpoint{3.548398in}{3.304883in}}%
\pgfpathlineto{\pgfqpoint{3.548398in}{3.300625in}}%
\pgfpathmoveto{\pgfqpoint{3.544140in}{3.304883in}}%
\pgfpathlineto{\pgfqpoint{3.544140in}{3.304883in}}%
\pgfpathlineto{\pgfqpoint{3.544140in}{3.309140in}}%
\pgfpathlineto{\pgfqpoint{3.548398in}{3.309140in}}%
\pgfpathlineto{\pgfqpoint{3.548398in}{3.304883in}}%
\pgfpathmoveto{\pgfqpoint{3.548398in}{3.300625in}}%
\pgfpathlineto{\pgfqpoint{3.548398in}{3.300625in}}%
\pgfpathlineto{\pgfqpoint{3.548398in}{3.304883in}}%
\pgfpathlineto{\pgfqpoint{3.552655in}{3.304883in}}%
\pgfpathlineto{\pgfqpoint{3.552655in}{3.300625in}}%
\pgfpathmoveto{\pgfqpoint{3.548398in}{3.304883in}}%
\pgfpathlineto{\pgfqpoint{3.548398in}{3.304883in}}%
\pgfpathlineto{\pgfqpoint{3.548398in}{3.309140in}}%
\pgfpathlineto{\pgfqpoint{3.552655in}{3.309140in}}%
\pgfpathlineto{\pgfqpoint{3.552655in}{3.304883in}}%
\pgfpathmoveto{\pgfqpoint{3.544140in}{3.309140in}}%
\pgfpathlineto{\pgfqpoint{3.544140in}{3.309140in}}%
\pgfpathlineto{\pgfqpoint{3.544140in}{3.313398in}}%
\pgfpathlineto{\pgfqpoint{3.548398in}{3.313398in}}%
\pgfpathlineto{\pgfqpoint{3.548398in}{3.309140in}}%
\pgfpathmoveto{\pgfqpoint{3.544140in}{3.313398in}}%
\pgfpathlineto{\pgfqpoint{3.544140in}{3.313398in}}%
\pgfpathlineto{\pgfqpoint{3.544140in}{3.317656in}}%
\pgfpathlineto{\pgfqpoint{3.548398in}{3.317656in}}%
\pgfpathlineto{\pgfqpoint{3.548398in}{3.313398in}}%
\pgfpathmoveto{\pgfqpoint{3.548398in}{3.309140in}}%
\pgfpathlineto{\pgfqpoint{3.548398in}{3.309140in}}%
\pgfpathlineto{\pgfqpoint{3.548398in}{3.313398in}}%
\pgfpathlineto{\pgfqpoint{3.552655in}{3.313398in}}%
\pgfpathlineto{\pgfqpoint{3.552655in}{3.309140in}}%
\pgfpathmoveto{\pgfqpoint{3.548398in}{3.313398in}}%
\pgfpathlineto{\pgfqpoint{3.548398in}{3.313398in}}%
\pgfpathlineto{\pgfqpoint{3.548398in}{3.317656in}}%
\pgfpathlineto{\pgfqpoint{3.552655in}{3.317656in}}%
\pgfpathlineto{\pgfqpoint{3.552655in}{3.313398in}}%
\pgfpathmoveto{\pgfqpoint{3.535624in}{3.317656in}}%
\pgfpathlineto{\pgfqpoint{3.535624in}{3.317656in}}%
\pgfpathlineto{\pgfqpoint{3.535624in}{3.321914in}}%
\pgfpathlineto{\pgfqpoint{3.539882in}{3.321914in}}%
\pgfpathlineto{\pgfqpoint{3.539882in}{3.317656in}}%
\pgfpathmoveto{\pgfqpoint{3.535624in}{3.321914in}}%
\pgfpathlineto{\pgfqpoint{3.535624in}{3.321914in}}%
\pgfpathlineto{\pgfqpoint{3.535624in}{3.326172in}}%
\pgfpathlineto{\pgfqpoint{3.539882in}{3.326172in}}%
\pgfpathlineto{\pgfqpoint{3.539882in}{3.321914in}}%
\pgfpathmoveto{\pgfqpoint{3.539882in}{3.317656in}}%
\pgfpathlineto{\pgfqpoint{3.539882in}{3.317656in}}%
\pgfpathlineto{\pgfqpoint{3.539882in}{3.321914in}}%
\pgfpathlineto{\pgfqpoint{3.544140in}{3.321914in}}%
\pgfpathlineto{\pgfqpoint{3.544140in}{3.317656in}}%
\pgfpathmoveto{\pgfqpoint{3.539882in}{3.321914in}}%
\pgfpathlineto{\pgfqpoint{3.539882in}{3.321914in}}%
\pgfpathlineto{\pgfqpoint{3.539882in}{3.326172in}}%
\pgfpathlineto{\pgfqpoint{3.544140in}{3.326172in}}%
\pgfpathlineto{\pgfqpoint{3.544140in}{3.321914in}}%
\pgfpathmoveto{\pgfqpoint{3.535624in}{3.326172in}}%
\pgfpathlineto{\pgfqpoint{3.535624in}{3.326172in}}%
\pgfpathlineto{\pgfqpoint{3.535624in}{3.330429in}}%
\pgfpathlineto{\pgfqpoint{3.539882in}{3.330429in}}%
\pgfpathlineto{\pgfqpoint{3.539882in}{3.326172in}}%
\pgfpathmoveto{\pgfqpoint{3.535624in}{3.330429in}}%
\pgfpathlineto{\pgfqpoint{3.535624in}{3.330429in}}%
\pgfpathlineto{\pgfqpoint{3.535624in}{3.334687in}}%
\pgfpathlineto{\pgfqpoint{3.539882in}{3.334687in}}%
\pgfpathlineto{\pgfqpoint{3.539882in}{3.330429in}}%
\pgfpathmoveto{\pgfqpoint{3.539882in}{3.326172in}}%
\pgfpathlineto{\pgfqpoint{3.539882in}{3.326172in}}%
\pgfpathlineto{\pgfqpoint{3.539882in}{3.330429in}}%
\pgfpathlineto{\pgfqpoint{3.544140in}{3.330429in}}%
\pgfpathlineto{\pgfqpoint{3.544140in}{3.326172in}}%
\pgfpathmoveto{\pgfqpoint{3.539882in}{3.330429in}}%
\pgfpathlineto{\pgfqpoint{3.539882in}{3.330429in}}%
\pgfpathlineto{\pgfqpoint{3.539882in}{3.334687in}}%
\pgfpathlineto{\pgfqpoint{3.544140in}{3.334687in}}%
\pgfpathlineto{\pgfqpoint{3.544140in}{3.330429in}}%
\pgfpathmoveto{\pgfqpoint{3.544140in}{3.317656in}}%
\pgfpathlineto{\pgfqpoint{3.544140in}{3.317656in}}%
\pgfpathlineto{\pgfqpoint{3.544140in}{3.321914in}}%
\pgfpathlineto{\pgfqpoint{3.548398in}{3.321914in}}%
\pgfpathlineto{\pgfqpoint{3.548398in}{3.317656in}}%
\pgfpathmoveto{\pgfqpoint{3.544140in}{3.321914in}}%
\pgfpathlineto{\pgfqpoint{3.544140in}{3.321914in}}%
\pgfpathlineto{\pgfqpoint{3.544140in}{3.326172in}}%
\pgfpathlineto{\pgfqpoint{3.548398in}{3.326172in}}%
\pgfpathlineto{\pgfqpoint{3.548398in}{3.321914in}}%
\pgfpathmoveto{\pgfqpoint{3.548398in}{3.317656in}}%
\pgfpathlineto{\pgfqpoint{3.548398in}{3.317656in}}%
\pgfpathlineto{\pgfqpoint{3.548398in}{3.321914in}}%
\pgfpathlineto{\pgfqpoint{3.552655in}{3.321914in}}%
\pgfpathlineto{\pgfqpoint{3.552655in}{3.317656in}}%
\pgfpathmoveto{\pgfqpoint{3.544140in}{3.326172in}}%
\pgfpathlineto{\pgfqpoint{3.544140in}{3.326172in}}%
\pgfpathlineto{\pgfqpoint{3.544140in}{3.330429in}}%
\pgfpathlineto{\pgfqpoint{3.548398in}{3.330429in}}%
\pgfpathlineto{\pgfqpoint{3.548398in}{3.326172in}}%
\pgfpathmoveto{\pgfqpoint{3.552655in}{3.300625in}}%
\pgfpathlineto{\pgfqpoint{3.552655in}{3.300625in}}%
\pgfpathlineto{\pgfqpoint{3.552655in}{3.304883in}}%
\pgfpathlineto{\pgfqpoint{3.556913in}{3.304883in}}%
\pgfpathlineto{\pgfqpoint{3.556913in}{3.300625in}}%
\pgfpathmoveto{\pgfqpoint{3.552655in}{3.304883in}}%
\pgfpathlineto{\pgfqpoint{3.552655in}{3.304883in}}%
\pgfpathlineto{\pgfqpoint{3.552655in}{3.309140in}}%
\pgfpathlineto{\pgfqpoint{3.556913in}{3.309140in}}%
\pgfpathlineto{\pgfqpoint{3.556913in}{3.304883in}}%
\pgfpathmoveto{\pgfqpoint{3.556913in}{3.300625in}}%
\pgfpathlineto{\pgfqpoint{3.556913in}{3.300625in}}%
\pgfpathlineto{\pgfqpoint{3.556913in}{3.304883in}}%
\pgfpathlineto{\pgfqpoint{3.561171in}{3.304883in}}%
\pgfpathlineto{\pgfqpoint{3.561171in}{3.300625in}}%
\pgfpathmoveto{\pgfqpoint{3.552655in}{3.309140in}}%
\pgfpathlineto{\pgfqpoint{3.552655in}{3.309140in}}%
\pgfpathlineto{\pgfqpoint{3.552655in}{3.313398in}}%
\pgfpathlineto{\pgfqpoint{3.556913in}{3.313398in}}%
\pgfpathlineto{\pgfqpoint{3.556913in}{3.309140in}}%
\pgfpathmoveto{\pgfqpoint{3.535624in}{3.334687in}}%
\pgfpathlineto{\pgfqpoint{3.535624in}{3.334687in}}%
\pgfpathlineto{\pgfqpoint{3.535624in}{3.338945in}}%
\pgfpathlineto{\pgfqpoint{3.539882in}{3.338945in}}%
\pgfpathlineto{\pgfqpoint{3.539882in}{3.334687in}}%
\pgfpathmoveto{\pgfqpoint{3.535624in}{3.338945in}}%
\pgfpathlineto{\pgfqpoint{3.535624in}{3.338945in}}%
\pgfpathlineto{\pgfqpoint{3.535624in}{3.343203in}}%
\pgfpathlineto{\pgfqpoint{3.539882in}{3.343203in}}%
\pgfpathlineto{\pgfqpoint{3.539882in}{3.338945in}}%
\pgfpathmoveto{\pgfqpoint{3.569687in}{3.232500in}}%
\pgfpathlineto{\pgfqpoint{3.569687in}{3.232500in}}%
\pgfpathlineto{\pgfqpoint{3.569687in}{3.236758in}}%
\pgfpathlineto{\pgfqpoint{3.573945in}{3.236758in}}%
\pgfpathlineto{\pgfqpoint{3.573945in}{3.232500in}}%
\pgfpathmoveto{\pgfqpoint{3.569687in}{3.236758in}}%
\pgfpathlineto{\pgfqpoint{3.569687in}{3.236758in}}%
\pgfpathlineto{\pgfqpoint{3.569687in}{3.241016in}}%
\pgfpathlineto{\pgfqpoint{3.573945in}{3.241016in}}%
\pgfpathlineto{\pgfqpoint{3.573945in}{3.236758in}}%
\pgfpathmoveto{\pgfqpoint{3.573945in}{3.232500in}}%
\pgfpathlineto{\pgfqpoint{3.573945in}{3.232500in}}%
\pgfpathlineto{\pgfqpoint{3.573945in}{3.236758in}}%
\pgfpathlineto{\pgfqpoint{3.578203in}{3.236758in}}%
\pgfpathlineto{\pgfqpoint{3.578203in}{3.232500in}}%
\pgfpathmoveto{\pgfqpoint{3.573945in}{3.236758in}}%
\pgfpathlineto{\pgfqpoint{3.573945in}{3.236758in}}%
\pgfpathlineto{\pgfqpoint{3.573945in}{3.241016in}}%
\pgfpathlineto{\pgfqpoint{3.578203in}{3.241016in}}%
\pgfpathlineto{\pgfqpoint{3.578203in}{3.236758in}}%
\pgfpathmoveto{\pgfqpoint{3.569687in}{3.241016in}}%
\pgfpathlineto{\pgfqpoint{3.569687in}{3.241016in}}%
\pgfpathlineto{\pgfqpoint{3.569687in}{3.245274in}}%
\pgfpathlineto{\pgfqpoint{3.573945in}{3.245274in}}%
\pgfpathlineto{\pgfqpoint{3.573945in}{3.241016in}}%
\pgfpathmoveto{\pgfqpoint{3.569687in}{3.245274in}}%
\pgfpathlineto{\pgfqpoint{3.569687in}{3.245274in}}%
\pgfpathlineto{\pgfqpoint{3.569687in}{3.249531in}}%
\pgfpathlineto{\pgfqpoint{3.573945in}{3.249531in}}%
\pgfpathlineto{\pgfqpoint{3.573945in}{3.245274in}}%
\pgfpathmoveto{\pgfqpoint{3.573945in}{3.241016in}}%
\pgfpathlineto{\pgfqpoint{3.573945in}{3.241016in}}%
\pgfpathlineto{\pgfqpoint{3.573945in}{3.245274in}}%
\pgfpathlineto{\pgfqpoint{3.578203in}{3.245274in}}%
\pgfpathlineto{\pgfqpoint{3.578203in}{3.241016in}}%
\pgfpathmoveto{\pgfqpoint{3.573945in}{3.245274in}}%
\pgfpathlineto{\pgfqpoint{3.573945in}{3.245274in}}%
\pgfpathlineto{\pgfqpoint{3.573945in}{3.249531in}}%
\pgfpathlineto{\pgfqpoint{3.578203in}{3.249531in}}%
\pgfpathlineto{\pgfqpoint{3.578203in}{3.245274in}}%
\pgfpathmoveto{\pgfqpoint{3.578203in}{3.232500in}}%
\pgfpathlineto{\pgfqpoint{3.578203in}{3.232500in}}%
\pgfpathlineto{\pgfqpoint{3.578203in}{3.236758in}}%
\pgfpathlineto{\pgfqpoint{3.582461in}{3.236758in}}%
\pgfpathlineto{\pgfqpoint{3.582461in}{3.232500in}}%
\pgfpathmoveto{\pgfqpoint{3.578203in}{3.236758in}}%
\pgfpathlineto{\pgfqpoint{3.578203in}{3.236758in}}%
\pgfpathlineto{\pgfqpoint{3.578203in}{3.241016in}}%
\pgfpathlineto{\pgfqpoint{3.582461in}{3.241016in}}%
\pgfpathlineto{\pgfqpoint{3.582461in}{3.236758in}}%
\pgfpathmoveto{\pgfqpoint{3.582461in}{3.232500in}}%
\pgfpathlineto{\pgfqpoint{3.582461in}{3.232500in}}%
\pgfpathlineto{\pgfqpoint{3.582461in}{3.236758in}}%
\pgfpathlineto{\pgfqpoint{3.586719in}{3.236758in}}%
\pgfpathlineto{\pgfqpoint{3.586719in}{3.232500in}}%
\pgfpathmoveto{\pgfqpoint{3.582461in}{3.236758in}}%
\pgfpathlineto{\pgfqpoint{3.582461in}{3.236758in}}%
\pgfpathlineto{\pgfqpoint{3.582461in}{3.241016in}}%
\pgfpathlineto{\pgfqpoint{3.586719in}{3.241016in}}%
\pgfpathlineto{\pgfqpoint{3.586719in}{3.236758in}}%
\pgfpathmoveto{\pgfqpoint{3.578203in}{3.241016in}}%
\pgfpathlineto{\pgfqpoint{3.578203in}{3.241016in}}%
\pgfpathlineto{\pgfqpoint{3.578203in}{3.245274in}}%
\pgfpathlineto{\pgfqpoint{3.582461in}{3.245274in}}%
\pgfpathlineto{\pgfqpoint{3.582461in}{3.241016in}}%
\pgfpathmoveto{\pgfqpoint{3.578203in}{3.245274in}}%
\pgfpathlineto{\pgfqpoint{3.578203in}{3.245274in}}%
\pgfpathlineto{\pgfqpoint{3.578203in}{3.249531in}}%
\pgfpathlineto{\pgfqpoint{3.582461in}{3.249531in}}%
\pgfpathlineto{\pgfqpoint{3.582461in}{3.245274in}}%
\pgfpathmoveto{\pgfqpoint{3.582461in}{3.241016in}}%
\pgfpathlineto{\pgfqpoint{3.582461in}{3.241016in}}%
\pgfpathlineto{\pgfqpoint{3.582461in}{3.245274in}}%
\pgfpathlineto{\pgfqpoint{3.586719in}{3.245274in}}%
\pgfpathlineto{\pgfqpoint{3.586719in}{3.241016in}}%
\pgfpathmoveto{\pgfqpoint{3.569687in}{3.249531in}}%
\pgfpathlineto{\pgfqpoint{3.569687in}{3.249531in}}%
\pgfpathlineto{\pgfqpoint{3.569687in}{3.253789in}}%
\pgfpathlineto{\pgfqpoint{3.573945in}{3.253789in}}%
\pgfpathlineto{\pgfqpoint{3.573945in}{3.249531in}}%
\pgfpathmoveto{\pgfqpoint{3.569687in}{3.253789in}}%
\pgfpathlineto{\pgfqpoint{3.569687in}{3.253789in}}%
\pgfpathlineto{\pgfqpoint{3.569687in}{3.258047in}}%
\pgfpathlineto{\pgfqpoint{3.573945in}{3.258047in}}%
\pgfpathlineto{\pgfqpoint{3.573945in}{3.253789in}}%
\pgfpathmoveto{\pgfqpoint{3.573945in}{3.249531in}}%
\pgfpathlineto{\pgfqpoint{3.573945in}{3.249531in}}%
\pgfpathlineto{\pgfqpoint{3.573945in}{3.253789in}}%
\pgfpathlineto{\pgfqpoint{3.578203in}{3.253789in}}%
\pgfpathlineto{\pgfqpoint{3.578203in}{3.249531in}}%
\pgfpathmoveto{\pgfqpoint{3.573945in}{3.253789in}}%
\pgfpathlineto{\pgfqpoint{3.573945in}{3.253789in}}%
\pgfpathlineto{\pgfqpoint{3.573945in}{3.258047in}}%
\pgfpathlineto{\pgfqpoint{3.578203in}{3.258047in}}%
\pgfpathlineto{\pgfqpoint{3.578203in}{3.253789in}}%
\pgfpathmoveto{\pgfqpoint{3.569687in}{3.258047in}}%
\pgfpathlineto{\pgfqpoint{3.569687in}{3.258047in}}%
\pgfpathlineto{\pgfqpoint{3.569687in}{3.262305in}}%
\pgfpathlineto{\pgfqpoint{3.573945in}{3.262305in}}%
\pgfpathlineto{\pgfqpoint{3.573945in}{3.258047in}}%
\pgfpathmoveto{\pgfqpoint{3.569687in}{3.262305in}}%
\pgfpathlineto{\pgfqpoint{3.569687in}{3.262305in}}%
\pgfpathlineto{\pgfqpoint{3.569687in}{3.266563in}}%
\pgfpathlineto{\pgfqpoint{3.573945in}{3.266563in}}%
\pgfpathlineto{\pgfqpoint{3.573945in}{3.262305in}}%
\pgfpathmoveto{\pgfqpoint{3.573945in}{3.258047in}}%
\pgfpathlineto{\pgfqpoint{3.573945in}{3.258047in}}%
\pgfpathlineto{\pgfqpoint{3.573945in}{3.262305in}}%
\pgfpathlineto{\pgfqpoint{3.578203in}{3.262305in}}%
\pgfpathlineto{\pgfqpoint{3.578203in}{3.258047in}}%
\pgfpathmoveto{\pgfqpoint{3.573945in}{3.262305in}}%
\pgfpathlineto{\pgfqpoint{3.573945in}{3.262305in}}%
\pgfpathlineto{\pgfqpoint{3.573945in}{3.266563in}}%
\pgfpathlineto{\pgfqpoint{3.578203in}{3.266563in}}%
\pgfpathlineto{\pgfqpoint{3.578203in}{3.262305in}}%
\pgfpathmoveto{\pgfqpoint{3.578203in}{3.249531in}}%
\pgfpathlineto{\pgfqpoint{3.578203in}{3.249531in}}%
\pgfpathlineto{\pgfqpoint{3.578203in}{3.253789in}}%
\pgfpathlineto{\pgfqpoint{3.582461in}{3.253789in}}%
\pgfpathlineto{\pgfqpoint{3.582461in}{3.249531in}}%
\pgfpathmoveto{\pgfqpoint{3.578203in}{3.253789in}}%
\pgfpathlineto{\pgfqpoint{3.578203in}{3.253789in}}%
\pgfpathlineto{\pgfqpoint{3.578203in}{3.258047in}}%
\pgfpathlineto{\pgfqpoint{3.582461in}{3.258047in}}%
\pgfpathlineto{\pgfqpoint{3.582461in}{3.253789in}}%
\pgfpathmoveto{\pgfqpoint{3.586719in}{3.232500in}}%
\pgfpathlineto{\pgfqpoint{3.586719in}{3.232500in}}%
\pgfpathlineto{\pgfqpoint{3.586719in}{3.236758in}}%
\pgfpathlineto{\pgfqpoint{3.590977in}{3.236758in}}%
\pgfpathlineto{\pgfqpoint{3.590977in}{3.232500in}}%
\pgfpathmoveto{\pgfqpoint{3.569687in}{3.266563in}}%
\pgfpathlineto{\pgfqpoint{3.569687in}{3.266563in}}%
\pgfpathlineto{\pgfqpoint{3.569687in}{3.270820in}}%
\pgfpathlineto{\pgfqpoint{3.573945in}{3.270820in}}%
\pgfpathlineto{\pgfqpoint{3.573945in}{3.266563in}}%
\pgfpathmoveto{\pgfqpoint{3.569687in}{3.270820in}}%
\pgfpathlineto{\pgfqpoint{3.569687in}{3.270820in}}%
\pgfpathlineto{\pgfqpoint{3.569687in}{3.275078in}}%
\pgfpathlineto{\pgfqpoint{3.573945in}{3.275078in}}%
\pgfpathlineto{\pgfqpoint{3.573945in}{3.270820in}}%
\pgfpathmoveto{\pgfqpoint{3.569687in}{3.275078in}}%
\pgfpathlineto{\pgfqpoint{3.569687in}{3.275078in}}%
\pgfpathlineto{\pgfqpoint{3.569687in}{3.279336in}}%
\pgfpathlineto{\pgfqpoint{3.573945in}{3.279336in}}%
\pgfpathlineto{\pgfqpoint{3.573945in}{3.275078in}}%
\pgfpathmoveto{\pgfqpoint{3.501560in}{3.368750in}}%
\pgfpathlineto{\pgfqpoint{3.501560in}{3.368750in}}%
\pgfpathlineto{\pgfqpoint{3.501560in}{3.373007in}}%
\pgfpathlineto{\pgfqpoint{3.505818in}{3.373007in}}%
\pgfpathlineto{\pgfqpoint{3.505818in}{3.368750in}}%
\pgfpathmoveto{\pgfqpoint{3.501560in}{3.373007in}}%
\pgfpathlineto{\pgfqpoint{3.501560in}{3.373007in}}%
\pgfpathlineto{\pgfqpoint{3.501560in}{3.377265in}}%
\pgfpathlineto{\pgfqpoint{3.505818in}{3.377265in}}%
\pgfpathlineto{\pgfqpoint{3.505818in}{3.373007in}}%
\pgfpathmoveto{\pgfqpoint{3.505818in}{3.368750in}}%
\pgfpathlineto{\pgfqpoint{3.505818in}{3.368750in}}%
\pgfpathlineto{\pgfqpoint{3.505818in}{3.373007in}}%
\pgfpathlineto{\pgfqpoint{3.510076in}{3.373007in}}%
\pgfpathlineto{\pgfqpoint{3.510076in}{3.368750in}}%
\pgfpathmoveto{\pgfqpoint{3.505818in}{3.373007in}}%
\pgfpathlineto{\pgfqpoint{3.505818in}{3.373007in}}%
\pgfpathlineto{\pgfqpoint{3.505818in}{3.377265in}}%
\pgfpathlineto{\pgfqpoint{3.510076in}{3.377265in}}%
\pgfpathlineto{\pgfqpoint{3.510076in}{3.373007in}}%
\pgfpathmoveto{\pgfqpoint{3.501560in}{3.377265in}}%
\pgfpathlineto{\pgfqpoint{3.501560in}{3.377265in}}%
\pgfpathlineto{\pgfqpoint{3.501560in}{3.381523in}}%
\pgfpathlineto{\pgfqpoint{3.505818in}{3.381523in}}%
\pgfpathlineto{\pgfqpoint{3.505818in}{3.377265in}}%
\pgfpathmoveto{\pgfqpoint{3.501560in}{3.381523in}}%
\pgfpathlineto{\pgfqpoint{3.501560in}{3.381523in}}%
\pgfpathlineto{\pgfqpoint{3.501560in}{3.385781in}}%
\pgfpathlineto{\pgfqpoint{3.505818in}{3.385781in}}%
\pgfpathlineto{\pgfqpoint{3.505818in}{3.381523in}}%
\pgfpathmoveto{\pgfqpoint{3.505818in}{3.377265in}}%
\pgfpathlineto{\pgfqpoint{3.505818in}{3.377265in}}%
\pgfpathlineto{\pgfqpoint{3.505818in}{3.381523in}}%
\pgfpathlineto{\pgfqpoint{3.510076in}{3.381523in}}%
\pgfpathlineto{\pgfqpoint{3.510076in}{3.377265in}}%
\pgfpathmoveto{\pgfqpoint{3.510076in}{3.368750in}}%
\pgfpathlineto{\pgfqpoint{3.510076in}{3.368750in}}%
\pgfpathlineto{\pgfqpoint{3.510076in}{3.373007in}}%
\pgfpathlineto{\pgfqpoint{3.514334in}{3.373007in}}%
\pgfpathlineto{\pgfqpoint{3.514334in}{3.368750in}}%
\pgfpathmoveto{\pgfqpoint{3.510076in}{3.373007in}}%
\pgfpathlineto{\pgfqpoint{3.510076in}{3.373007in}}%
\pgfpathlineto{\pgfqpoint{3.510076in}{3.377265in}}%
\pgfpathlineto{\pgfqpoint{3.514334in}{3.377265in}}%
\pgfpathlineto{\pgfqpoint{3.514334in}{3.373007in}}%
\pgfpathmoveto{\pgfqpoint{3.514334in}{3.368750in}}%
\pgfpathlineto{\pgfqpoint{3.514334in}{3.368750in}}%
\pgfpathlineto{\pgfqpoint{3.514334in}{3.373007in}}%
\pgfpathlineto{\pgfqpoint{3.518592in}{3.373007in}}%
\pgfpathlineto{\pgfqpoint{3.518592in}{3.368750in}}%
\pgfpathmoveto{\pgfqpoint{3.501560in}{3.385781in}}%
\pgfpathlineto{\pgfqpoint{3.501560in}{3.385781in}}%
\pgfpathlineto{\pgfqpoint{3.501560in}{3.390039in}}%
\pgfpathlineto{\pgfqpoint{3.505818in}{3.390039in}}%
\pgfpathlineto{\pgfqpoint{3.505818in}{3.385781in}}%
\pgfpathmoveto{\pgfqpoint{3.769803in}{2.397967in}}%
\pgfpathlineto{\pgfqpoint{3.769803in}{2.397967in}}%
\pgfpathlineto{\pgfqpoint{3.769803in}{2.402225in}}%
\pgfpathlineto{\pgfqpoint{3.774061in}{2.402225in}}%
\pgfpathlineto{\pgfqpoint{3.774061in}{2.397967in}}%
\pgfpathmoveto{\pgfqpoint{3.769803in}{2.402225in}}%
\pgfpathlineto{\pgfqpoint{3.769803in}{2.402225in}}%
\pgfpathlineto{\pgfqpoint{3.769803in}{2.406483in}}%
\pgfpathlineto{\pgfqpoint{3.774061in}{2.406483in}}%
\pgfpathlineto{\pgfqpoint{3.774061in}{2.402225in}}%
\pgfpathmoveto{\pgfqpoint{3.769803in}{2.406483in}}%
\pgfpathlineto{\pgfqpoint{3.769803in}{2.406483in}}%
\pgfpathlineto{\pgfqpoint{3.769803in}{2.410741in}}%
\pgfpathlineto{\pgfqpoint{3.774061in}{2.410741in}}%
\pgfpathlineto{\pgfqpoint{3.774061in}{2.406483in}}%
\pgfpathmoveto{\pgfqpoint{3.769803in}{2.410741in}}%
\pgfpathlineto{\pgfqpoint{3.769803in}{2.410741in}}%
\pgfpathlineto{\pgfqpoint{3.769803in}{2.414998in}}%
\pgfpathlineto{\pgfqpoint{3.774061in}{2.414998in}}%
\pgfpathlineto{\pgfqpoint{3.774061in}{2.410741in}}%
\pgfpathmoveto{\pgfqpoint{3.765545in}{2.419256in}}%
\pgfpathlineto{\pgfqpoint{3.765545in}{2.419256in}}%
\pgfpathlineto{\pgfqpoint{3.765545in}{2.423514in}}%
\pgfpathlineto{\pgfqpoint{3.769803in}{2.423514in}}%
\pgfpathlineto{\pgfqpoint{3.769803in}{2.419256in}}%
\pgfpathmoveto{\pgfqpoint{3.769803in}{2.414998in}}%
\pgfpathlineto{\pgfqpoint{3.769803in}{2.414998in}}%
\pgfpathlineto{\pgfqpoint{3.769803in}{2.419256in}}%
\pgfpathlineto{\pgfqpoint{3.774061in}{2.419256in}}%
\pgfpathlineto{\pgfqpoint{3.774061in}{2.414998in}}%
\pgfpathmoveto{\pgfqpoint{3.769803in}{2.419256in}}%
\pgfpathlineto{\pgfqpoint{3.769803in}{2.419256in}}%
\pgfpathlineto{\pgfqpoint{3.769803in}{2.423514in}}%
\pgfpathlineto{\pgfqpoint{3.774061in}{2.423514in}}%
\pgfpathlineto{\pgfqpoint{3.774061in}{2.419256in}}%
\pgfpathmoveto{\pgfqpoint{3.765545in}{2.423514in}}%
\pgfpathlineto{\pgfqpoint{3.765545in}{2.423514in}}%
\pgfpathlineto{\pgfqpoint{3.765545in}{2.427772in}}%
\pgfpathlineto{\pgfqpoint{3.769803in}{2.427772in}}%
\pgfpathlineto{\pgfqpoint{3.769803in}{2.423514in}}%
\pgfpathmoveto{\pgfqpoint{3.765545in}{2.427772in}}%
\pgfpathlineto{\pgfqpoint{3.765545in}{2.427772in}}%
\pgfpathlineto{\pgfqpoint{3.765545in}{2.432030in}}%
\pgfpathlineto{\pgfqpoint{3.769803in}{2.432030in}}%
\pgfpathlineto{\pgfqpoint{3.769803in}{2.427772in}}%
\pgfpathmoveto{\pgfqpoint{3.769803in}{2.423514in}}%
\pgfpathlineto{\pgfqpoint{3.769803in}{2.423514in}}%
\pgfpathlineto{\pgfqpoint{3.769803in}{2.427772in}}%
\pgfpathlineto{\pgfqpoint{3.774061in}{2.427772in}}%
\pgfpathlineto{\pgfqpoint{3.774061in}{2.423514in}}%
\pgfpathmoveto{\pgfqpoint{3.769803in}{2.427772in}}%
\pgfpathlineto{\pgfqpoint{3.769803in}{2.427772in}}%
\pgfpathlineto{\pgfqpoint{3.769803in}{2.432030in}}%
\pgfpathlineto{\pgfqpoint{3.774061in}{2.432030in}}%
\pgfpathlineto{\pgfqpoint{3.774061in}{2.427772in}}%
\pgfpathmoveto{\pgfqpoint{3.761287in}{2.444803in}}%
\pgfpathlineto{\pgfqpoint{3.761287in}{2.444803in}}%
\pgfpathlineto{\pgfqpoint{3.761287in}{2.449061in}}%
\pgfpathlineto{\pgfqpoint{3.765545in}{2.449061in}}%
\pgfpathlineto{\pgfqpoint{3.765545in}{2.444803in}}%
\pgfpathmoveto{\pgfqpoint{3.765545in}{2.432030in}}%
\pgfpathlineto{\pgfqpoint{3.765545in}{2.432030in}}%
\pgfpathlineto{\pgfqpoint{3.765545in}{2.436288in}}%
\pgfpathlineto{\pgfqpoint{3.769803in}{2.436288in}}%
\pgfpathlineto{\pgfqpoint{3.769803in}{2.432030in}}%
\pgfpathmoveto{\pgfqpoint{3.765545in}{2.436288in}}%
\pgfpathlineto{\pgfqpoint{3.765545in}{2.436288in}}%
\pgfpathlineto{\pgfqpoint{3.765545in}{2.440546in}}%
\pgfpathlineto{\pgfqpoint{3.769803in}{2.440546in}}%
\pgfpathlineto{\pgfqpoint{3.769803in}{2.436288in}}%
\pgfpathmoveto{\pgfqpoint{3.769803in}{2.432030in}}%
\pgfpathlineto{\pgfqpoint{3.769803in}{2.432030in}}%
\pgfpathlineto{\pgfqpoint{3.769803in}{2.436288in}}%
\pgfpathlineto{\pgfqpoint{3.774061in}{2.436288in}}%
\pgfpathlineto{\pgfqpoint{3.774061in}{2.432030in}}%
\pgfpathmoveto{\pgfqpoint{3.769803in}{2.436288in}}%
\pgfpathlineto{\pgfqpoint{3.769803in}{2.436288in}}%
\pgfpathlineto{\pgfqpoint{3.769803in}{2.440546in}}%
\pgfpathlineto{\pgfqpoint{3.774061in}{2.440546in}}%
\pgfpathlineto{\pgfqpoint{3.774061in}{2.436288in}}%
\pgfpathmoveto{\pgfqpoint{3.765545in}{2.440546in}}%
\pgfpathlineto{\pgfqpoint{3.765545in}{2.440546in}}%
\pgfpathlineto{\pgfqpoint{3.765545in}{2.444803in}}%
\pgfpathlineto{\pgfqpoint{3.769803in}{2.444803in}}%
\pgfpathlineto{\pgfqpoint{3.769803in}{2.440546in}}%
\pgfpathmoveto{\pgfqpoint{3.765545in}{2.444803in}}%
\pgfpathlineto{\pgfqpoint{3.765545in}{2.444803in}}%
\pgfpathlineto{\pgfqpoint{3.765545in}{2.449061in}}%
\pgfpathlineto{\pgfqpoint{3.769803in}{2.449061in}}%
\pgfpathlineto{\pgfqpoint{3.769803in}{2.444803in}}%
\pgfpathmoveto{\pgfqpoint{3.769803in}{2.440546in}}%
\pgfpathlineto{\pgfqpoint{3.769803in}{2.440546in}}%
\pgfpathlineto{\pgfqpoint{3.769803in}{2.444803in}}%
\pgfpathlineto{\pgfqpoint{3.774061in}{2.444803in}}%
\pgfpathlineto{\pgfqpoint{3.774061in}{2.440546in}}%
\pgfpathmoveto{\pgfqpoint{3.769803in}{2.444803in}}%
\pgfpathlineto{\pgfqpoint{3.769803in}{2.444803in}}%
\pgfpathlineto{\pgfqpoint{3.769803in}{2.449061in}}%
\pgfpathlineto{\pgfqpoint{3.774061in}{2.449061in}}%
\pgfpathlineto{\pgfqpoint{3.774061in}{2.444803in}}%
\pgfpathmoveto{\pgfqpoint{3.761287in}{2.449061in}}%
\pgfpathlineto{\pgfqpoint{3.761287in}{2.449061in}}%
\pgfpathlineto{\pgfqpoint{3.761287in}{2.453319in}}%
\pgfpathlineto{\pgfqpoint{3.765545in}{2.453319in}}%
\pgfpathlineto{\pgfqpoint{3.765545in}{2.449061in}}%
\pgfpathmoveto{\pgfqpoint{3.761287in}{2.453319in}}%
\pgfpathlineto{\pgfqpoint{3.761287in}{2.453319in}}%
\pgfpathlineto{\pgfqpoint{3.761287in}{2.457577in}}%
\pgfpathlineto{\pgfqpoint{3.765545in}{2.457577in}}%
\pgfpathlineto{\pgfqpoint{3.765545in}{2.453319in}}%
\pgfpathmoveto{\pgfqpoint{3.761287in}{2.457577in}}%
\pgfpathlineto{\pgfqpoint{3.761287in}{2.457577in}}%
\pgfpathlineto{\pgfqpoint{3.761287in}{2.461835in}}%
\pgfpathlineto{\pgfqpoint{3.765545in}{2.461835in}}%
\pgfpathlineto{\pgfqpoint{3.765545in}{2.457577in}}%
\pgfpathmoveto{\pgfqpoint{3.761287in}{2.461835in}}%
\pgfpathlineto{\pgfqpoint{3.761287in}{2.461835in}}%
\pgfpathlineto{\pgfqpoint{3.761287in}{2.466093in}}%
\pgfpathlineto{\pgfqpoint{3.765545in}{2.466093in}}%
\pgfpathlineto{\pgfqpoint{3.765545in}{2.461835in}}%
\pgfpathmoveto{\pgfqpoint{3.765545in}{2.449061in}}%
\pgfpathlineto{\pgfqpoint{3.765545in}{2.449061in}}%
\pgfpathlineto{\pgfqpoint{3.765545in}{2.453319in}}%
\pgfpathlineto{\pgfqpoint{3.769803in}{2.453319in}}%
\pgfpathlineto{\pgfqpoint{3.769803in}{2.449061in}}%
\pgfpathmoveto{\pgfqpoint{3.765545in}{2.453319in}}%
\pgfpathlineto{\pgfqpoint{3.765545in}{2.453319in}}%
\pgfpathlineto{\pgfqpoint{3.765545in}{2.457577in}}%
\pgfpathlineto{\pgfqpoint{3.769803in}{2.457577in}}%
\pgfpathlineto{\pgfqpoint{3.769803in}{2.453319in}}%
\pgfpathmoveto{\pgfqpoint{3.769803in}{2.449061in}}%
\pgfpathlineto{\pgfqpoint{3.769803in}{2.449061in}}%
\pgfpathlineto{\pgfqpoint{3.769803in}{2.453319in}}%
\pgfpathlineto{\pgfqpoint{3.774061in}{2.453319in}}%
\pgfpathlineto{\pgfqpoint{3.774061in}{2.449061in}}%
\pgfpathmoveto{\pgfqpoint{3.769803in}{2.453319in}}%
\pgfpathlineto{\pgfqpoint{3.769803in}{2.453319in}}%
\pgfpathlineto{\pgfqpoint{3.769803in}{2.457577in}}%
\pgfpathlineto{\pgfqpoint{3.774061in}{2.457577in}}%
\pgfpathlineto{\pgfqpoint{3.774061in}{2.453319in}}%
\pgfpathmoveto{\pgfqpoint{3.765545in}{2.457577in}}%
\pgfpathlineto{\pgfqpoint{3.765545in}{2.457577in}}%
\pgfpathlineto{\pgfqpoint{3.765545in}{2.461835in}}%
\pgfpathlineto{\pgfqpoint{3.769803in}{2.461835in}}%
\pgfpathlineto{\pgfqpoint{3.769803in}{2.457577in}}%
\pgfpathmoveto{\pgfqpoint{3.765545in}{2.461835in}}%
\pgfpathlineto{\pgfqpoint{3.765545in}{2.461835in}}%
\pgfpathlineto{\pgfqpoint{3.765545in}{2.466093in}}%
\pgfpathlineto{\pgfqpoint{3.769803in}{2.466093in}}%
\pgfpathlineto{\pgfqpoint{3.769803in}{2.461835in}}%
\pgfpathmoveto{\pgfqpoint{3.757030in}{2.470351in}}%
\pgfpathlineto{\pgfqpoint{3.757030in}{2.470351in}}%
\pgfpathlineto{\pgfqpoint{3.757030in}{2.474608in}}%
\pgfpathlineto{\pgfqpoint{3.761287in}{2.474608in}}%
\pgfpathlineto{\pgfqpoint{3.761287in}{2.470351in}}%
\pgfpathmoveto{\pgfqpoint{3.761287in}{2.466093in}}%
\pgfpathlineto{\pgfqpoint{3.761287in}{2.466093in}}%
\pgfpathlineto{\pgfqpoint{3.761287in}{2.470351in}}%
\pgfpathlineto{\pgfqpoint{3.765545in}{2.470351in}}%
\pgfpathlineto{\pgfqpoint{3.765545in}{2.466093in}}%
\pgfpathmoveto{\pgfqpoint{3.761287in}{2.470351in}}%
\pgfpathlineto{\pgfqpoint{3.761287in}{2.470351in}}%
\pgfpathlineto{\pgfqpoint{3.761287in}{2.474608in}}%
\pgfpathlineto{\pgfqpoint{3.765545in}{2.474608in}}%
\pgfpathlineto{\pgfqpoint{3.765545in}{2.470351in}}%
\pgfpathmoveto{\pgfqpoint{3.757030in}{2.474608in}}%
\pgfpathlineto{\pgfqpoint{3.757030in}{2.474608in}}%
\pgfpathlineto{\pgfqpoint{3.757030in}{2.478866in}}%
\pgfpathlineto{\pgfqpoint{3.761287in}{2.478866in}}%
\pgfpathlineto{\pgfqpoint{3.761287in}{2.474608in}}%
\pgfpathmoveto{\pgfqpoint{3.757030in}{2.478866in}}%
\pgfpathlineto{\pgfqpoint{3.757030in}{2.478866in}}%
\pgfpathlineto{\pgfqpoint{3.757030in}{2.483124in}}%
\pgfpathlineto{\pgfqpoint{3.761287in}{2.483124in}}%
\pgfpathlineto{\pgfqpoint{3.761287in}{2.478866in}}%
\pgfpathmoveto{\pgfqpoint{3.761287in}{2.474608in}}%
\pgfpathlineto{\pgfqpoint{3.761287in}{2.474608in}}%
\pgfpathlineto{\pgfqpoint{3.761287in}{2.478866in}}%
\pgfpathlineto{\pgfqpoint{3.765545in}{2.478866in}}%
\pgfpathlineto{\pgfqpoint{3.765545in}{2.474608in}}%
\pgfpathmoveto{\pgfqpoint{3.761287in}{2.478866in}}%
\pgfpathlineto{\pgfqpoint{3.761287in}{2.478866in}}%
\pgfpathlineto{\pgfqpoint{3.761287in}{2.483124in}}%
\pgfpathlineto{\pgfqpoint{3.765545in}{2.483124in}}%
\pgfpathlineto{\pgfqpoint{3.765545in}{2.478866in}}%
\pgfpathmoveto{\pgfqpoint{3.765545in}{2.466093in}}%
\pgfpathlineto{\pgfqpoint{3.765545in}{2.466093in}}%
\pgfpathlineto{\pgfqpoint{3.765545in}{2.470351in}}%
\pgfpathlineto{\pgfqpoint{3.769803in}{2.470351in}}%
\pgfpathlineto{\pgfqpoint{3.769803in}{2.466093in}}%
\pgfpathmoveto{\pgfqpoint{3.765545in}{2.470351in}}%
\pgfpathlineto{\pgfqpoint{3.765545in}{2.470351in}}%
\pgfpathlineto{\pgfqpoint{3.765545in}{2.474608in}}%
\pgfpathlineto{\pgfqpoint{3.769803in}{2.474608in}}%
\pgfpathlineto{\pgfqpoint{3.769803in}{2.470351in}}%
\pgfpathmoveto{\pgfqpoint{3.765545in}{2.474608in}}%
\pgfpathlineto{\pgfqpoint{3.765545in}{2.474608in}}%
\pgfpathlineto{\pgfqpoint{3.765545in}{2.478866in}}%
\pgfpathlineto{\pgfqpoint{3.769803in}{2.478866in}}%
\pgfpathlineto{\pgfqpoint{3.769803in}{2.474608in}}%
\pgfpathmoveto{\pgfqpoint{3.765545in}{2.478866in}}%
\pgfpathlineto{\pgfqpoint{3.765545in}{2.478866in}}%
\pgfpathlineto{\pgfqpoint{3.765545in}{2.483124in}}%
\pgfpathlineto{\pgfqpoint{3.769803in}{2.483124in}}%
\pgfpathlineto{\pgfqpoint{3.769803in}{2.478866in}}%
\pgfpathmoveto{\pgfqpoint{3.752772in}{2.495898in}}%
\pgfpathlineto{\pgfqpoint{3.752772in}{2.495898in}}%
\pgfpathlineto{\pgfqpoint{3.752772in}{2.500156in}}%
\pgfpathlineto{\pgfqpoint{3.757030in}{2.500156in}}%
\pgfpathlineto{\pgfqpoint{3.757030in}{2.495898in}}%
\pgfpathmoveto{\pgfqpoint{3.752772in}{2.500156in}}%
\pgfpathlineto{\pgfqpoint{3.752772in}{2.500156in}}%
\pgfpathlineto{\pgfqpoint{3.752772in}{2.504413in}}%
\pgfpathlineto{\pgfqpoint{3.757030in}{2.504413in}}%
\pgfpathlineto{\pgfqpoint{3.757030in}{2.500156in}}%
\pgfpathmoveto{\pgfqpoint{3.752772in}{2.504413in}}%
\pgfpathlineto{\pgfqpoint{3.752772in}{2.504413in}}%
\pgfpathlineto{\pgfqpoint{3.752772in}{2.508671in}}%
\pgfpathlineto{\pgfqpoint{3.757030in}{2.508671in}}%
\pgfpathlineto{\pgfqpoint{3.757030in}{2.504413in}}%
\pgfpathmoveto{\pgfqpoint{3.752772in}{2.508671in}}%
\pgfpathlineto{\pgfqpoint{3.752772in}{2.508671in}}%
\pgfpathlineto{\pgfqpoint{3.752772in}{2.512929in}}%
\pgfpathlineto{\pgfqpoint{3.757030in}{2.512929in}}%
\pgfpathlineto{\pgfqpoint{3.757030in}{2.508671in}}%
\pgfpathmoveto{\pgfqpoint{3.752772in}{2.512929in}}%
\pgfpathlineto{\pgfqpoint{3.752772in}{2.512929in}}%
\pgfpathlineto{\pgfqpoint{3.752772in}{2.517187in}}%
\pgfpathlineto{\pgfqpoint{3.757030in}{2.517187in}}%
\pgfpathlineto{\pgfqpoint{3.757030in}{2.512929in}}%
\pgfpathmoveto{\pgfqpoint{3.757030in}{2.483124in}}%
\pgfpathlineto{\pgfqpoint{3.757030in}{2.483124in}}%
\pgfpathlineto{\pgfqpoint{3.757030in}{2.487382in}}%
\pgfpathlineto{\pgfqpoint{3.761287in}{2.487382in}}%
\pgfpathlineto{\pgfqpoint{3.761287in}{2.483124in}}%
\pgfpathmoveto{\pgfqpoint{3.757030in}{2.487382in}}%
\pgfpathlineto{\pgfqpoint{3.757030in}{2.487382in}}%
\pgfpathlineto{\pgfqpoint{3.757030in}{2.491640in}}%
\pgfpathlineto{\pgfqpoint{3.761287in}{2.491640in}}%
\pgfpathlineto{\pgfqpoint{3.761287in}{2.487382in}}%
\pgfpathmoveto{\pgfqpoint{3.761287in}{2.483124in}}%
\pgfpathlineto{\pgfqpoint{3.761287in}{2.483124in}}%
\pgfpathlineto{\pgfqpoint{3.761287in}{2.487382in}}%
\pgfpathlineto{\pgfqpoint{3.765545in}{2.487382in}}%
\pgfpathlineto{\pgfqpoint{3.765545in}{2.483124in}}%
\pgfpathmoveto{\pgfqpoint{3.761287in}{2.487382in}}%
\pgfpathlineto{\pgfqpoint{3.761287in}{2.487382in}}%
\pgfpathlineto{\pgfqpoint{3.761287in}{2.491640in}}%
\pgfpathlineto{\pgfqpoint{3.765545in}{2.491640in}}%
\pgfpathlineto{\pgfqpoint{3.765545in}{2.487382in}}%
\pgfpathmoveto{\pgfqpoint{3.757030in}{2.491640in}}%
\pgfpathlineto{\pgfqpoint{3.757030in}{2.491640in}}%
\pgfpathlineto{\pgfqpoint{3.757030in}{2.495898in}}%
\pgfpathlineto{\pgfqpoint{3.761287in}{2.495898in}}%
\pgfpathlineto{\pgfqpoint{3.761287in}{2.491640in}}%
\pgfpathmoveto{\pgfqpoint{3.757030in}{2.495898in}}%
\pgfpathlineto{\pgfqpoint{3.757030in}{2.495898in}}%
\pgfpathlineto{\pgfqpoint{3.757030in}{2.500156in}}%
\pgfpathlineto{\pgfqpoint{3.761287in}{2.500156in}}%
\pgfpathlineto{\pgfqpoint{3.761287in}{2.495898in}}%
\pgfpathmoveto{\pgfqpoint{3.761287in}{2.491640in}}%
\pgfpathlineto{\pgfqpoint{3.761287in}{2.491640in}}%
\pgfpathlineto{\pgfqpoint{3.761287in}{2.495898in}}%
\pgfpathlineto{\pgfqpoint{3.765545in}{2.495898in}}%
\pgfpathlineto{\pgfqpoint{3.765545in}{2.491640in}}%
\pgfpathmoveto{\pgfqpoint{3.761287in}{2.495898in}}%
\pgfpathlineto{\pgfqpoint{3.761287in}{2.495898in}}%
\pgfpathlineto{\pgfqpoint{3.761287in}{2.500156in}}%
\pgfpathlineto{\pgfqpoint{3.765545in}{2.500156in}}%
\pgfpathlineto{\pgfqpoint{3.765545in}{2.495898in}}%
\pgfpathmoveto{\pgfqpoint{3.757030in}{2.500156in}}%
\pgfpathlineto{\pgfqpoint{3.757030in}{2.500156in}}%
\pgfpathlineto{\pgfqpoint{3.757030in}{2.504413in}}%
\pgfpathlineto{\pgfqpoint{3.761287in}{2.504413in}}%
\pgfpathlineto{\pgfqpoint{3.761287in}{2.500156in}}%
\pgfpathmoveto{\pgfqpoint{3.757030in}{2.504413in}}%
\pgfpathlineto{\pgfqpoint{3.757030in}{2.504413in}}%
\pgfpathlineto{\pgfqpoint{3.757030in}{2.508671in}}%
\pgfpathlineto{\pgfqpoint{3.761287in}{2.508671in}}%
\pgfpathlineto{\pgfqpoint{3.761287in}{2.504413in}}%
\pgfpathmoveto{\pgfqpoint{3.761287in}{2.500156in}}%
\pgfpathlineto{\pgfqpoint{3.761287in}{2.500156in}}%
\pgfpathlineto{\pgfqpoint{3.761287in}{2.504413in}}%
\pgfpathlineto{\pgfqpoint{3.765545in}{2.504413in}}%
\pgfpathlineto{\pgfqpoint{3.765545in}{2.500156in}}%
\pgfpathmoveto{\pgfqpoint{3.761287in}{2.504413in}}%
\pgfpathlineto{\pgfqpoint{3.761287in}{2.504413in}}%
\pgfpathlineto{\pgfqpoint{3.761287in}{2.508671in}}%
\pgfpathlineto{\pgfqpoint{3.765545in}{2.508671in}}%
\pgfpathlineto{\pgfqpoint{3.765545in}{2.504413in}}%
\pgfpathmoveto{\pgfqpoint{3.757030in}{2.508671in}}%
\pgfpathlineto{\pgfqpoint{3.757030in}{2.508671in}}%
\pgfpathlineto{\pgfqpoint{3.757030in}{2.512929in}}%
\pgfpathlineto{\pgfqpoint{3.761287in}{2.512929in}}%
\pgfpathlineto{\pgfqpoint{3.761287in}{2.508671in}}%
\pgfpathmoveto{\pgfqpoint{3.757030in}{2.512929in}}%
\pgfpathlineto{\pgfqpoint{3.757030in}{2.512929in}}%
\pgfpathlineto{\pgfqpoint{3.757030in}{2.517187in}}%
\pgfpathlineto{\pgfqpoint{3.761287in}{2.517187in}}%
\pgfpathlineto{\pgfqpoint{3.761287in}{2.512929in}}%
\pgfpathmoveto{\pgfqpoint{3.748514in}{2.521445in}}%
\pgfpathlineto{\pgfqpoint{3.748514in}{2.521445in}}%
\pgfpathlineto{\pgfqpoint{3.748514in}{2.525703in}}%
\pgfpathlineto{\pgfqpoint{3.752772in}{2.525703in}}%
\pgfpathlineto{\pgfqpoint{3.752772in}{2.521445in}}%
\pgfpathmoveto{\pgfqpoint{3.752772in}{2.517187in}}%
\pgfpathlineto{\pgfqpoint{3.752772in}{2.517187in}}%
\pgfpathlineto{\pgfqpoint{3.752772in}{2.521445in}}%
\pgfpathlineto{\pgfqpoint{3.757030in}{2.521445in}}%
\pgfpathlineto{\pgfqpoint{3.757030in}{2.517187in}}%
\pgfpathmoveto{\pgfqpoint{3.752772in}{2.521445in}}%
\pgfpathlineto{\pgfqpoint{3.752772in}{2.521445in}}%
\pgfpathlineto{\pgfqpoint{3.752772in}{2.525703in}}%
\pgfpathlineto{\pgfqpoint{3.757030in}{2.525703in}}%
\pgfpathlineto{\pgfqpoint{3.757030in}{2.521445in}}%
\pgfpathmoveto{\pgfqpoint{3.748514in}{2.525703in}}%
\pgfpathlineto{\pgfqpoint{3.748514in}{2.525703in}}%
\pgfpathlineto{\pgfqpoint{3.748514in}{2.529961in}}%
\pgfpathlineto{\pgfqpoint{3.752772in}{2.529961in}}%
\pgfpathlineto{\pgfqpoint{3.752772in}{2.525703in}}%
\pgfpathmoveto{\pgfqpoint{3.748514in}{2.529961in}}%
\pgfpathlineto{\pgfqpoint{3.748514in}{2.529961in}}%
\pgfpathlineto{\pgfqpoint{3.748514in}{2.534218in}}%
\pgfpathlineto{\pgfqpoint{3.752772in}{2.534218in}}%
\pgfpathlineto{\pgfqpoint{3.752772in}{2.529961in}}%
\pgfpathmoveto{\pgfqpoint{3.752772in}{2.525703in}}%
\pgfpathlineto{\pgfqpoint{3.752772in}{2.525703in}}%
\pgfpathlineto{\pgfqpoint{3.752772in}{2.529961in}}%
\pgfpathlineto{\pgfqpoint{3.757030in}{2.529961in}}%
\pgfpathlineto{\pgfqpoint{3.757030in}{2.525703in}}%
\pgfpathmoveto{\pgfqpoint{3.752772in}{2.529961in}}%
\pgfpathlineto{\pgfqpoint{3.752772in}{2.529961in}}%
\pgfpathlineto{\pgfqpoint{3.752772in}{2.534218in}}%
\pgfpathlineto{\pgfqpoint{3.757030in}{2.534218in}}%
\pgfpathlineto{\pgfqpoint{3.757030in}{2.529961in}}%
\pgfpathmoveto{\pgfqpoint{3.744257in}{2.542734in}}%
\pgfpathlineto{\pgfqpoint{3.744257in}{2.542734in}}%
\pgfpathlineto{\pgfqpoint{3.744257in}{2.546992in}}%
\pgfpathlineto{\pgfqpoint{3.748514in}{2.546992in}}%
\pgfpathlineto{\pgfqpoint{3.748514in}{2.542734in}}%
\pgfpathmoveto{\pgfqpoint{3.744257in}{2.546992in}}%
\pgfpathlineto{\pgfqpoint{3.744257in}{2.546992in}}%
\pgfpathlineto{\pgfqpoint{3.744257in}{2.551250in}}%
\pgfpathlineto{\pgfqpoint{3.748514in}{2.551250in}}%
\pgfpathlineto{\pgfqpoint{3.748514in}{2.546992in}}%
\pgfpathmoveto{\pgfqpoint{3.748514in}{2.534218in}}%
\pgfpathlineto{\pgfqpoint{3.748514in}{2.534218in}}%
\pgfpathlineto{\pgfqpoint{3.748514in}{2.538476in}}%
\pgfpathlineto{\pgfqpoint{3.752772in}{2.538476in}}%
\pgfpathlineto{\pgfqpoint{3.752772in}{2.534218in}}%
\pgfpathmoveto{\pgfqpoint{3.748514in}{2.538476in}}%
\pgfpathlineto{\pgfqpoint{3.748514in}{2.538476in}}%
\pgfpathlineto{\pgfqpoint{3.748514in}{2.542734in}}%
\pgfpathlineto{\pgfqpoint{3.752772in}{2.542734in}}%
\pgfpathlineto{\pgfqpoint{3.752772in}{2.538476in}}%
\pgfpathmoveto{\pgfqpoint{3.752772in}{2.534218in}}%
\pgfpathlineto{\pgfqpoint{3.752772in}{2.534218in}}%
\pgfpathlineto{\pgfqpoint{3.752772in}{2.538476in}}%
\pgfpathlineto{\pgfqpoint{3.757030in}{2.538476in}}%
\pgfpathlineto{\pgfqpoint{3.757030in}{2.534218in}}%
\pgfpathmoveto{\pgfqpoint{3.752772in}{2.538476in}}%
\pgfpathlineto{\pgfqpoint{3.752772in}{2.538476in}}%
\pgfpathlineto{\pgfqpoint{3.752772in}{2.542734in}}%
\pgfpathlineto{\pgfqpoint{3.757030in}{2.542734in}}%
\pgfpathlineto{\pgfqpoint{3.757030in}{2.538476in}}%
\pgfpathmoveto{\pgfqpoint{3.748514in}{2.542734in}}%
\pgfpathlineto{\pgfqpoint{3.748514in}{2.542734in}}%
\pgfpathlineto{\pgfqpoint{3.748514in}{2.546992in}}%
\pgfpathlineto{\pgfqpoint{3.752772in}{2.546992in}}%
\pgfpathlineto{\pgfqpoint{3.752772in}{2.542734in}}%
\pgfpathmoveto{\pgfqpoint{3.748514in}{2.546992in}}%
\pgfpathlineto{\pgfqpoint{3.748514in}{2.546992in}}%
\pgfpathlineto{\pgfqpoint{3.748514in}{2.551250in}}%
\pgfpathlineto{\pgfqpoint{3.752772in}{2.551250in}}%
\pgfpathlineto{\pgfqpoint{3.752772in}{2.546992in}}%
\pgfpathmoveto{\pgfqpoint{3.752772in}{2.542734in}}%
\pgfpathlineto{\pgfqpoint{3.752772in}{2.542734in}}%
\pgfpathlineto{\pgfqpoint{3.752772in}{2.546992in}}%
\pgfpathlineto{\pgfqpoint{3.757030in}{2.546992in}}%
\pgfpathlineto{\pgfqpoint{3.757030in}{2.542734in}}%
\pgfpathmoveto{\pgfqpoint{3.752772in}{2.546992in}}%
\pgfpathlineto{\pgfqpoint{3.752772in}{2.546992in}}%
\pgfpathlineto{\pgfqpoint{3.752772in}{2.551250in}}%
\pgfpathlineto{\pgfqpoint{3.757030in}{2.551250in}}%
\pgfpathlineto{\pgfqpoint{3.757030in}{2.546992in}}%
\pgfpathmoveto{\pgfqpoint{3.757030in}{2.517187in}}%
\pgfpathlineto{\pgfqpoint{3.757030in}{2.517187in}}%
\pgfpathlineto{\pgfqpoint{3.757030in}{2.521445in}}%
\pgfpathlineto{\pgfqpoint{3.761287in}{2.521445in}}%
\pgfpathlineto{\pgfqpoint{3.761287in}{2.517187in}}%
\pgfpathmoveto{\pgfqpoint{3.757030in}{2.521445in}}%
\pgfpathlineto{\pgfqpoint{3.757030in}{2.521445in}}%
\pgfpathlineto{\pgfqpoint{3.757030in}{2.525703in}}%
\pgfpathlineto{\pgfqpoint{3.761287in}{2.525703in}}%
\pgfpathlineto{\pgfqpoint{3.761287in}{2.521445in}}%
\pgfpathmoveto{\pgfqpoint{3.757030in}{2.525703in}}%
\pgfpathlineto{\pgfqpoint{3.757030in}{2.525703in}}%
\pgfpathlineto{\pgfqpoint{3.757030in}{2.529961in}}%
\pgfpathlineto{\pgfqpoint{3.761287in}{2.529961in}}%
\pgfpathlineto{\pgfqpoint{3.761287in}{2.525703in}}%
\pgfpathmoveto{\pgfqpoint{3.757030in}{2.529961in}}%
\pgfpathlineto{\pgfqpoint{3.757030in}{2.529961in}}%
\pgfpathlineto{\pgfqpoint{3.757030in}{2.534218in}}%
\pgfpathlineto{\pgfqpoint{3.761287in}{2.534218in}}%
\pgfpathlineto{\pgfqpoint{3.761287in}{2.529961in}}%
\pgfpathmoveto{\pgfqpoint{3.735741in}{2.589570in}}%
\pgfpathlineto{\pgfqpoint{3.735741in}{2.589570in}}%
\pgfpathlineto{\pgfqpoint{3.735741in}{2.593828in}}%
\pgfpathlineto{\pgfqpoint{3.739999in}{2.593828in}}%
\pgfpathlineto{\pgfqpoint{3.739999in}{2.589570in}}%
\pgfpathmoveto{\pgfqpoint{3.735741in}{2.593828in}}%
\pgfpathlineto{\pgfqpoint{3.735741in}{2.593828in}}%
\pgfpathlineto{\pgfqpoint{3.735741in}{2.598086in}}%
\pgfpathlineto{\pgfqpoint{3.739999in}{2.598086in}}%
\pgfpathlineto{\pgfqpoint{3.739999in}{2.593828in}}%
\pgfpathmoveto{\pgfqpoint{3.735741in}{2.598086in}}%
\pgfpathlineto{\pgfqpoint{3.735741in}{2.598086in}}%
\pgfpathlineto{\pgfqpoint{3.735741in}{2.602343in}}%
\pgfpathlineto{\pgfqpoint{3.739999in}{2.602343in}}%
\pgfpathlineto{\pgfqpoint{3.739999in}{2.598086in}}%
\pgfpathmoveto{\pgfqpoint{3.735741in}{2.602343in}}%
\pgfpathlineto{\pgfqpoint{3.735741in}{2.602343in}}%
\pgfpathlineto{\pgfqpoint{3.735741in}{2.606601in}}%
\pgfpathlineto{\pgfqpoint{3.739999in}{2.606601in}}%
\pgfpathlineto{\pgfqpoint{3.739999in}{2.602343in}}%
\pgfpathmoveto{\pgfqpoint{3.735741in}{2.606601in}}%
\pgfpathlineto{\pgfqpoint{3.735741in}{2.606601in}}%
\pgfpathlineto{\pgfqpoint{3.735741in}{2.610859in}}%
\pgfpathlineto{\pgfqpoint{3.739999in}{2.610859in}}%
\pgfpathlineto{\pgfqpoint{3.739999in}{2.606601in}}%
\pgfpathmoveto{\pgfqpoint{3.731484in}{2.610859in}}%
\pgfpathlineto{\pgfqpoint{3.731484in}{2.610859in}}%
\pgfpathlineto{\pgfqpoint{3.731484in}{2.615117in}}%
\pgfpathlineto{\pgfqpoint{3.735741in}{2.615117in}}%
\pgfpathlineto{\pgfqpoint{3.735741in}{2.610859in}}%
\pgfpathmoveto{\pgfqpoint{3.731484in}{2.615117in}}%
\pgfpathlineto{\pgfqpoint{3.731484in}{2.615117in}}%
\pgfpathlineto{\pgfqpoint{3.731484in}{2.619375in}}%
\pgfpathlineto{\pgfqpoint{3.735741in}{2.619375in}}%
\pgfpathlineto{\pgfqpoint{3.735741in}{2.615117in}}%
\pgfpathmoveto{\pgfqpoint{3.735741in}{2.610859in}}%
\pgfpathlineto{\pgfqpoint{3.735741in}{2.610859in}}%
\pgfpathlineto{\pgfqpoint{3.735741in}{2.615117in}}%
\pgfpathlineto{\pgfqpoint{3.739999in}{2.615117in}}%
\pgfpathlineto{\pgfqpoint{3.739999in}{2.610859in}}%
\pgfpathmoveto{\pgfqpoint{3.735741in}{2.615117in}}%
\pgfpathlineto{\pgfqpoint{3.735741in}{2.615117in}}%
\pgfpathlineto{\pgfqpoint{3.735741in}{2.619375in}}%
\pgfpathlineto{\pgfqpoint{3.739999in}{2.619375in}}%
\pgfpathlineto{\pgfqpoint{3.739999in}{2.615117in}}%
\pgfpathmoveto{\pgfqpoint{3.744257in}{2.551250in}}%
\pgfpathlineto{\pgfqpoint{3.744257in}{2.551250in}}%
\pgfpathlineto{\pgfqpoint{3.744257in}{2.555508in}}%
\pgfpathlineto{\pgfqpoint{3.748514in}{2.555508in}}%
\pgfpathlineto{\pgfqpoint{3.748514in}{2.551250in}}%
\pgfpathmoveto{\pgfqpoint{3.744257in}{2.555508in}}%
\pgfpathlineto{\pgfqpoint{3.744257in}{2.555508in}}%
\pgfpathlineto{\pgfqpoint{3.744257in}{2.559766in}}%
\pgfpathlineto{\pgfqpoint{3.748514in}{2.559766in}}%
\pgfpathlineto{\pgfqpoint{3.748514in}{2.555508in}}%
\pgfpathmoveto{\pgfqpoint{3.744257in}{2.559766in}}%
\pgfpathlineto{\pgfqpoint{3.744257in}{2.559766in}}%
\pgfpathlineto{\pgfqpoint{3.744257in}{2.564023in}}%
\pgfpathlineto{\pgfqpoint{3.748514in}{2.564023in}}%
\pgfpathlineto{\pgfqpoint{3.748514in}{2.559766in}}%
\pgfpathmoveto{\pgfqpoint{3.744257in}{2.564023in}}%
\pgfpathlineto{\pgfqpoint{3.744257in}{2.564023in}}%
\pgfpathlineto{\pgfqpoint{3.744257in}{2.568281in}}%
\pgfpathlineto{\pgfqpoint{3.748514in}{2.568281in}}%
\pgfpathlineto{\pgfqpoint{3.748514in}{2.564023in}}%
\pgfpathmoveto{\pgfqpoint{3.748514in}{2.551250in}}%
\pgfpathlineto{\pgfqpoint{3.748514in}{2.551250in}}%
\pgfpathlineto{\pgfqpoint{3.748514in}{2.555508in}}%
\pgfpathlineto{\pgfqpoint{3.752772in}{2.555508in}}%
\pgfpathlineto{\pgfqpoint{3.752772in}{2.551250in}}%
\pgfpathmoveto{\pgfqpoint{3.748514in}{2.555508in}}%
\pgfpathlineto{\pgfqpoint{3.748514in}{2.555508in}}%
\pgfpathlineto{\pgfqpoint{3.748514in}{2.559766in}}%
\pgfpathlineto{\pgfqpoint{3.752772in}{2.559766in}}%
\pgfpathlineto{\pgfqpoint{3.752772in}{2.555508in}}%
\pgfpathmoveto{\pgfqpoint{3.752772in}{2.551250in}}%
\pgfpathlineto{\pgfqpoint{3.752772in}{2.551250in}}%
\pgfpathlineto{\pgfqpoint{3.752772in}{2.555508in}}%
\pgfpathlineto{\pgfqpoint{3.757030in}{2.555508in}}%
\pgfpathlineto{\pgfqpoint{3.757030in}{2.551250in}}%
\pgfpathmoveto{\pgfqpoint{3.752772in}{2.555508in}}%
\pgfpathlineto{\pgfqpoint{3.752772in}{2.555508in}}%
\pgfpathlineto{\pgfqpoint{3.752772in}{2.559766in}}%
\pgfpathlineto{\pgfqpoint{3.757030in}{2.559766in}}%
\pgfpathlineto{\pgfqpoint{3.757030in}{2.555508in}}%
\pgfpathmoveto{\pgfqpoint{3.748514in}{2.559766in}}%
\pgfpathlineto{\pgfqpoint{3.748514in}{2.559766in}}%
\pgfpathlineto{\pgfqpoint{3.748514in}{2.564023in}}%
\pgfpathlineto{\pgfqpoint{3.752772in}{2.564023in}}%
\pgfpathlineto{\pgfqpoint{3.752772in}{2.559766in}}%
\pgfpathmoveto{\pgfqpoint{3.748514in}{2.564023in}}%
\pgfpathlineto{\pgfqpoint{3.748514in}{2.564023in}}%
\pgfpathlineto{\pgfqpoint{3.748514in}{2.568281in}}%
\pgfpathlineto{\pgfqpoint{3.752772in}{2.568281in}}%
\pgfpathlineto{\pgfqpoint{3.752772in}{2.564023in}}%
\pgfpathmoveto{\pgfqpoint{3.739999in}{2.568281in}}%
\pgfpathlineto{\pgfqpoint{3.739999in}{2.568281in}}%
\pgfpathlineto{\pgfqpoint{3.739999in}{2.572539in}}%
\pgfpathlineto{\pgfqpoint{3.744257in}{2.572539in}}%
\pgfpathlineto{\pgfqpoint{3.744257in}{2.568281in}}%
\pgfpathmoveto{\pgfqpoint{3.739999in}{2.572539in}}%
\pgfpathlineto{\pgfqpoint{3.739999in}{2.572539in}}%
\pgfpathlineto{\pgfqpoint{3.739999in}{2.576797in}}%
\pgfpathlineto{\pgfqpoint{3.744257in}{2.576797in}}%
\pgfpathlineto{\pgfqpoint{3.744257in}{2.572539in}}%
\pgfpathmoveto{\pgfqpoint{3.744257in}{2.568281in}}%
\pgfpathlineto{\pgfqpoint{3.744257in}{2.568281in}}%
\pgfpathlineto{\pgfqpoint{3.744257in}{2.572539in}}%
\pgfpathlineto{\pgfqpoint{3.748514in}{2.572539in}}%
\pgfpathlineto{\pgfqpoint{3.748514in}{2.568281in}}%
\pgfpathmoveto{\pgfqpoint{3.744257in}{2.572539in}}%
\pgfpathlineto{\pgfqpoint{3.744257in}{2.572539in}}%
\pgfpathlineto{\pgfqpoint{3.744257in}{2.576797in}}%
\pgfpathlineto{\pgfqpoint{3.748514in}{2.576797in}}%
\pgfpathlineto{\pgfqpoint{3.748514in}{2.572539in}}%
\pgfpathmoveto{\pgfqpoint{3.739999in}{2.576797in}}%
\pgfpathlineto{\pgfqpoint{3.739999in}{2.576797in}}%
\pgfpathlineto{\pgfqpoint{3.739999in}{2.581054in}}%
\pgfpathlineto{\pgfqpoint{3.744257in}{2.581054in}}%
\pgfpathlineto{\pgfqpoint{3.744257in}{2.576797in}}%
\pgfpathmoveto{\pgfqpoint{3.739999in}{2.581054in}}%
\pgfpathlineto{\pgfqpoint{3.739999in}{2.581054in}}%
\pgfpathlineto{\pgfqpoint{3.739999in}{2.585312in}}%
\pgfpathlineto{\pgfqpoint{3.744257in}{2.585312in}}%
\pgfpathlineto{\pgfqpoint{3.744257in}{2.581054in}}%
\pgfpathmoveto{\pgfqpoint{3.744257in}{2.576797in}}%
\pgfpathlineto{\pgfqpoint{3.744257in}{2.576797in}}%
\pgfpathlineto{\pgfqpoint{3.744257in}{2.581054in}}%
\pgfpathlineto{\pgfqpoint{3.748514in}{2.581054in}}%
\pgfpathlineto{\pgfqpoint{3.748514in}{2.576797in}}%
\pgfpathmoveto{\pgfqpoint{3.744257in}{2.581054in}}%
\pgfpathlineto{\pgfqpoint{3.744257in}{2.581054in}}%
\pgfpathlineto{\pgfqpoint{3.744257in}{2.585312in}}%
\pgfpathlineto{\pgfqpoint{3.748514in}{2.585312in}}%
\pgfpathlineto{\pgfqpoint{3.748514in}{2.581054in}}%
\pgfpathmoveto{\pgfqpoint{3.748514in}{2.568281in}}%
\pgfpathlineto{\pgfqpoint{3.748514in}{2.568281in}}%
\pgfpathlineto{\pgfqpoint{3.748514in}{2.572539in}}%
\pgfpathlineto{\pgfqpoint{3.752772in}{2.572539in}}%
\pgfpathlineto{\pgfqpoint{3.752772in}{2.568281in}}%
\pgfpathmoveto{\pgfqpoint{3.748514in}{2.572539in}}%
\pgfpathlineto{\pgfqpoint{3.748514in}{2.572539in}}%
\pgfpathlineto{\pgfqpoint{3.748514in}{2.576797in}}%
\pgfpathlineto{\pgfqpoint{3.752772in}{2.576797in}}%
\pgfpathlineto{\pgfqpoint{3.752772in}{2.572539in}}%
\pgfpathmoveto{\pgfqpoint{3.748514in}{2.576797in}}%
\pgfpathlineto{\pgfqpoint{3.748514in}{2.576797in}}%
\pgfpathlineto{\pgfqpoint{3.748514in}{2.581054in}}%
\pgfpathlineto{\pgfqpoint{3.752772in}{2.581054in}}%
\pgfpathlineto{\pgfqpoint{3.752772in}{2.576797in}}%
\pgfpathmoveto{\pgfqpoint{3.739999in}{2.585312in}}%
\pgfpathlineto{\pgfqpoint{3.739999in}{2.585312in}}%
\pgfpathlineto{\pgfqpoint{3.739999in}{2.589570in}}%
\pgfpathlineto{\pgfqpoint{3.744257in}{2.589570in}}%
\pgfpathlineto{\pgfqpoint{3.744257in}{2.585312in}}%
\pgfpathmoveto{\pgfqpoint{3.739999in}{2.589570in}}%
\pgfpathlineto{\pgfqpoint{3.739999in}{2.589570in}}%
\pgfpathlineto{\pgfqpoint{3.739999in}{2.593828in}}%
\pgfpathlineto{\pgfqpoint{3.744257in}{2.593828in}}%
\pgfpathlineto{\pgfqpoint{3.744257in}{2.589570in}}%
\pgfpathmoveto{\pgfqpoint{3.744257in}{2.585312in}}%
\pgfpathlineto{\pgfqpoint{3.744257in}{2.585312in}}%
\pgfpathlineto{\pgfqpoint{3.744257in}{2.589570in}}%
\pgfpathlineto{\pgfqpoint{3.748514in}{2.589570in}}%
\pgfpathlineto{\pgfqpoint{3.748514in}{2.585312in}}%
\pgfpathmoveto{\pgfqpoint{3.744257in}{2.589570in}}%
\pgfpathlineto{\pgfqpoint{3.744257in}{2.589570in}}%
\pgfpathlineto{\pgfqpoint{3.744257in}{2.593828in}}%
\pgfpathlineto{\pgfqpoint{3.748514in}{2.593828in}}%
\pgfpathlineto{\pgfqpoint{3.748514in}{2.589570in}}%
\pgfpathmoveto{\pgfqpoint{3.739999in}{2.593828in}}%
\pgfpathlineto{\pgfqpoint{3.739999in}{2.593828in}}%
\pgfpathlineto{\pgfqpoint{3.739999in}{2.598086in}}%
\pgfpathlineto{\pgfqpoint{3.744257in}{2.598086in}}%
\pgfpathlineto{\pgfqpoint{3.744257in}{2.593828in}}%
\pgfpathmoveto{\pgfqpoint{3.739999in}{2.598086in}}%
\pgfpathlineto{\pgfqpoint{3.739999in}{2.598086in}}%
\pgfpathlineto{\pgfqpoint{3.739999in}{2.602343in}}%
\pgfpathlineto{\pgfqpoint{3.744257in}{2.602343in}}%
\pgfpathlineto{\pgfqpoint{3.744257in}{2.598086in}}%
\pgfpathmoveto{\pgfqpoint{3.744257in}{2.593828in}}%
\pgfpathlineto{\pgfqpoint{3.744257in}{2.593828in}}%
\pgfpathlineto{\pgfqpoint{3.744257in}{2.598086in}}%
\pgfpathlineto{\pgfqpoint{3.748514in}{2.598086in}}%
\pgfpathlineto{\pgfqpoint{3.748514in}{2.593828in}}%
\pgfpathmoveto{\pgfqpoint{3.744257in}{2.598086in}}%
\pgfpathlineto{\pgfqpoint{3.744257in}{2.598086in}}%
\pgfpathlineto{\pgfqpoint{3.744257in}{2.602343in}}%
\pgfpathlineto{\pgfqpoint{3.748514in}{2.602343in}}%
\pgfpathlineto{\pgfqpoint{3.748514in}{2.598086in}}%
\pgfpathmoveto{\pgfqpoint{3.739999in}{2.602343in}}%
\pgfpathlineto{\pgfqpoint{3.739999in}{2.602343in}}%
\pgfpathlineto{\pgfqpoint{3.739999in}{2.606601in}}%
\pgfpathlineto{\pgfqpoint{3.744257in}{2.606601in}}%
\pgfpathlineto{\pgfqpoint{3.744257in}{2.602343in}}%
\pgfpathmoveto{\pgfqpoint{3.739999in}{2.606601in}}%
\pgfpathlineto{\pgfqpoint{3.739999in}{2.606601in}}%
\pgfpathlineto{\pgfqpoint{3.739999in}{2.610859in}}%
\pgfpathlineto{\pgfqpoint{3.744257in}{2.610859in}}%
\pgfpathlineto{\pgfqpoint{3.744257in}{2.606601in}}%
\pgfpathmoveto{\pgfqpoint{3.744257in}{2.602343in}}%
\pgfpathlineto{\pgfqpoint{3.744257in}{2.602343in}}%
\pgfpathlineto{\pgfqpoint{3.744257in}{2.606601in}}%
\pgfpathlineto{\pgfqpoint{3.748514in}{2.606601in}}%
\pgfpathlineto{\pgfqpoint{3.748514in}{2.602343in}}%
\pgfpathmoveto{\pgfqpoint{3.739999in}{2.610859in}}%
\pgfpathlineto{\pgfqpoint{3.739999in}{2.610859in}}%
\pgfpathlineto{\pgfqpoint{3.739999in}{2.615117in}}%
\pgfpathlineto{\pgfqpoint{3.744257in}{2.615117in}}%
\pgfpathlineto{\pgfqpoint{3.744257in}{2.610859in}}%
\pgfpathmoveto{\pgfqpoint{3.739999in}{2.615117in}}%
\pgfpathlineto{\pgfqpoint{3.739999in}{2.615117in}}%
\pgfpathlineto{\pgfqpoint{3.739999in}{2.619375in}}%
\pgfpathlineto{\pgfqpoint{3.744257in}{2.619375in}}%
\pgfpathlineto{\pgfqpoint{3.744257in}{2.615117in}}%
\pgfpathmoveto{\pgfqpoint{3.731484in}{2.619375in}}%
\pgfpathlineto{\pgfqpoint{3.731484in}{2.619375in}}%
\pgfpathlineto{\pgfqpoint{3.731484in}{2.623632in}}%
\pgfpathlineto{\pgfqpoint{3.735741in}{2.623632in}}%
\pgfpathlineto{\pgfqpoint{3.735741in}{2.619375in}}%
\pgfpathmoveto{\pgfqpoint{3.731484in}{2.623632in}}%
\pgfpathlineto{\pgfqpoint{3.731484in}{2.623632in}}%
\pgfpathlineto{\pgfqpoint{3.731484in}{2.627890in}}%
\pgfpathlineto{\pgfqpoint{3.735741in}{2.627890in}}%
\pgfpathlineto{\pgfqpoint{3.735741in}{2.623632in}}%
\pgfpathmoveto{\pgfqpoint{3.735741in}{2.619375in}}%
\pgfpathlineto{\pgfqpoint{3.735741in}{2.619375in}}%
\pgfpathlineto{\pgfqpoint{3.735741in}{2.623632in}}%
\pgfpathlineto{\pgfqpoint{3.739999in}{2.623632in}}%
\pgfpathlineto{\pgfqpoint{3.739999in}{2.619375in}}%
\pgfpathmoveto{\pgfqpoint{3.735741in}{2.623632in}}%
\pgfpathlineto{\pgfqpoint{3.735741in}{2.623632in}}%
\pgfpathlineto{\pgfqpoint{3.735741in}{2.627890in}}%
\pgfpathlineto{\pgfqpoint{3.739999in}{2.627890in}}%
\pgfpathlineto{\pgfqpoint{3.739999in}{2.623632in}}%
\pgfpathmoveto{\pgfqpoint{3.731484in}{2.627890in}}%
\pgfpathlineto{\pgfqpoint{3.731484in}{2.627890in}}%
\pgfpathlineto{\pgfqpoint{3.731484in}{2.632148in}}%
\pgfpathlineto{\pgfqpoint{3.735741in}{2.632148in}}%
\pgfpathlineto{\pgfqpoint{3.735741in}{2.627890in}}%
\pgfpathmoveto{\pgfqpoint{3.731484in}{2.632148in}}%
\pgfpathlineto{\pgfqpoint{3.731484in}{2.632148in}}%
\pgfpathlineto{\pgfqpoint{3.731484in}{2.636406in}}%
\pgfpathlineto{\pgfqpoint{3.735741in}{2.636406in}}%
\pgfpathlineto{\pgfqpoint{3.735741in}{2.632148in}}%
\pgfpathmoveto{\pgfqpoint{3.735741in}{2.627890in}}%
\pgfpathlineto{\pgfqpoint{3.735741in}{2.627890in}}%
\pgfpathlineto{\pgfqpoint{3.735741in}{2.632148in}}%
\pgfpathlineto{\pgfqpoint{3.739999in}{2.632148in}}%
\pgfpathlineto{\pgfqpoint{3.739999in}{2.627890in}}%
\pgfpathmoveto{\pgfqpoint{3.735741in}{2.632148in}}%
\pgfpathlineto{\pgfqpoint{3.735741in}{2.632148in}}%
\pgfpathlineto{\pgfqpoint{3.735741in}{2.636406in}}%
\pgfpathlineto{\pgfqpoint{3.739999in}{2.636406in}}%
\pgfpathlineto{\pgfqpoint{3.739999in}{2.632148in}}%
\pgfpathmoveto{\pgfqpoint{3.727226in}{2.636406in}}%
\pgfpathlineto{\pgfqpoint{3.727226in}{2.636406in}}%
\pgfpathlineto{\pgfqpoint{3.727226in}{2.640664in}}%
\pgfpathlineto{\pgfqpoint{3.731484in}{2.640664in}}%
\pgfpathlineto{\pgfqpoint{3.731484in}{2.636406in}}%
\pgfpathmoveto{\pgfqpoint{3.727226in}{2.640664in}}%
\pgfpathlineto{\pgfqpoint{3.727226in}{2.640664in}}%
\pgfpathlineto{\pgfqpoint{3.727226in}{2.644921in}}%
\pgfpathlineto{\pgfqpoint{3.731484in}{2.644921in}}%
\pgfpathlineto{\pgfqpoint{3.731484in}{2.640664in}}%
\pgfpathmoveto{\pgfqpoint{3.727226in}{2.644921in}}%
\pgfpathlineto{\pgfqpoint{3.727226in}{2.644921in}}%
\pgfpathlineto{\pgfqpoint{3.727226in}{2.649179in}}%
\pgfpathlineto{\pgfqpoint{3.731484in}{2.649179in}}%
\pgfpathlineto{\pgfqpoint{3.731484in}{2.644921in}}%
\pgfpathmoveto{\pgfqpoint{3.727226in}{2.649179in}}%
\pgfpathlineto{\pgfqpoint{3.727226in}{2.649179in}}%
\pgfpathlineto{\pgfqpoint{3.727226in}{2.653437in}}%
\pgfpathlineto{\pgfqpoint{3.731484in}{2.653437in}}%
\pgfpathlineto{\pgfqpoint{3.731484in}{2.649179in}}%
\pgfpathmoveto{\pgfqpoint{3.731484in}{2.636406in}}%
\pgfpathlineto{\pgfqpoint{3.731484in}{2.636406in}}%
\pgfpathlineto{\pgfqpoint{3.731484in}{2.640664in}}%
\pgfpathlineto{\pgfqpoint{3.735741in}{2.640664in}}%
\pgfpathlineto{\pgfqpoint{3.735741in}{2.636406in}}%
\pgfpathmoveto{\pgfqpoint{3.731484in}{2.640664in}}%
\pgfpathlineto{\pgfqpoint{3.731484in}{2.640664in}}%
\pgfpathlineto{\pgfqpoint{3.731484in}{2.644921in}}%
\pgfpathlineto{\pgfqpoint{3.735741in}{2.644921in}}%
\pgfpathlineto{\pgfqpoint{3.735741in}{2.640664in}}%
\pgfpathmoveto{\pgfqpoint{3.735741in}{2.636406in}}%
\pgfpathlineto{\pgfqpoint{3.735741in}{2.636406in}}%
\pgfpathlineto{\pgfqpoint{3.735741in}{2.640664in}}%
\pgfpathlineto{\pgfqpoint{3.739999in}{2.640664in}}%
\pgfpathlineto{\pgfqpoint{3.739999in}{2.636406in}}%
\pgfpathmoveto{\pgfqpoint{3.735741in}{2.640664in}}%
\pgfpathlineto{\pgfqpoint{3.735741in}{2.640664in}}%
\pgfpathlineto{\pgfqpoint{3.735741in}{2.644921in}}%
\pgfpathlineto{\pgfqpoint{3.739999in}{2.644921in}}%
\pgfpathlineto{\pgfqpoint{3.739999in}{2.640664in}}%
\pgfpathmoveto{\pgfqpoint{3.731484in}{2.644921in}}%
\pgfpathlineto{\pgfqpoint{3.731484in}{2.644921in}}%
\pgfpathlineto{\pgfqpoint{3.731484in}{2.649179in}}%
\pgfpathlineto{\pgfqpoint{3.735741in}{2.649179in}}%
\pgfpathlineto{\pgfqpoint{3.735741in}{2.644921in}}%
\pgfpathmoveto{\pgfqpoint{3.731484in}{2.649179in}}%
\pgfpathlineto{\pgfqpoint{3.731484in}{2.649179in}}%
\pgfpathlineto{\pgfqpoint{3.731484in}{2.653437in}}%
\pgfpathlineto{\pgfqpoint{3.735741in}{2.653437in}}%
\pgfpathlineto{\pgfqpoint{3.735741in}{2.649179in}}%
\pgfpathmoveto{\pgfqpoint{3.735741in}{2.644921in}}%
\pgfpathlineto{\pgfqpoint{3.735741in}{2.644921in}}%
\pgfpathlineto{\pgfqpoint{3.735741in}{2.649179in}}%
\pgfpathlineto{\pgfqpoint{3.739999in}{2.649179in}}%
\pgfpathlineto{\pgfqpoint{3.739999in}{2.644921in}}%
\pgfpathmoveto{\pgfqpoint{3.718710in}{2.678984in}}%
\pgfpathlineto{\pgfqpoint{3.718710in}{2.678984in}}%
\pgfpathlineto{\pgfqpoint{3.718710in}{2.683242in}}%
\pgfpathlineto{\pgfqpoint{3.722968in}{2.683242in}}%
\pgfpathlineto{\pgfqpoint{3.722968in}{2.678984in}}%
\pgfpathmoveto{\pgfqpoint{3.718710in}{2.683242in}}%
\pgfpathlineto{\pgfqpoint{3.718710in}{2.683242in}}%
\pgfpathlineto{\pgfqpoint{3.718710in}{2.687499in}}%
\pgfpathlineto{\pgfqpoint{3.722968in}{2.687499in}}%
\pgfpathlineto{\pgfqpoint{3.722968in}{2.683242in}}%
\pgfpathmoveto{\pgfqpoint{3.722968in}{2.657695in}}%
\pgfpathlineto{\pgfqpoint{3.722968in}{2.657695in}}%
\pgfpathlineto{\pgfqpoint{3.722968in}{2.661953in}}%
\pgfpathlineto{\pgfqpoint{3.727226in}{2.661953in}}%
\pgfpathlineto{\pgfqpoint{3.727226in}{2.657695in}}%
\pgfpathmoveto{\pgfqpoint{3.727226in}{2.653437in}}%
\pgfpathlineto{\pgfqpoint{3.727226in}{2.653437in}}%
\pgfpathlineto{\pgfqpoint{3.727226in}{2.657695in}}%
\pgfpathlineto{\pgfqpoint{3.731484in}{2.657695in}}%
\pgfpathlineto{\pgfqpoint{3.731484in}{2.653437in}}%
\pgfpathmoveto{\pgfqpoint{3.727226in}{2.657695in}}%
\pgfpathlineto{\pgfqpoint{3.727226in}{2.657695in}}%
\pgfpathlineto{\pgfqpoint{3.727226in}{2.661953in}}%
\pgfpathlineto{\pgfqpoint{3.731484in}{2.661953in}}%
\pgfpathlineto{\pgfqpoint{3.731484in}{2.657695in}}%
\pgfpathmoveto{\pgfqpoint{3.722968in}{2.661953in}}%
\pgfpathlineto{\pgfqpoint{3.722968in}{2.661953in}}%
\pgfpathlineto{\pgfqpoint{3.722968in}{2.666210in}}%
\pgfpathlineto{\pgfqpoint{3.727226in}{2.666210in}}%
\pgfpathlineto{\pgfqpoint{3.727226in}{2.661953in}}%
\pgfpathmoveto{\pgfqpoint{3.722968in}{2.666210in}}%
\pgfpathlineto{\pgfqpoint{3.722968in}{2.666210in}}%
\pgfpathlineto{\pgfqpoint{3.722968in}{2.670468in}}%
\pgfpathlineto{\pgfqpoint{3.727226in}{2.670468in}}%
\pgfpathlineto{\pgfqpoint{3.727226in}{2.666210in}}%
\pgfpathmoveto{\pgfqpoint{3.727226in}{2.661953in}}%
\pgfpathlineto{\pgfqpoint{3.727226in}{2.661953in}}%
\pgfpathlineto{\pgfqpoint{3.727226in}{2.666210in}}%
\pgfpathlineto{\pgfqpoint{3.731484in}{2.666210in}}%
\pgfpathlineto{\pgfqpoint{3.731484in}{2.661953in}}%
\pgfpathmoveto{\pgfqpoint{3.727226in}{2.666210in}}%
\pgfpathlineto{\pgfqpoint{3.727226in}{2.666210in}}%
\pgfpathlineto{\pgfqpoint{3.727226in}{2.670468in}}%
\pgfpathlineto{\pgfqpoint{3.731484in}{2.670468in}}%
\pgfpathlineto{\pgfqpoint{3.731484in}{2.666210in}}%
\pgfpathmoveto{\pgfqpoint{3.731484in}{2.653437in}}%
\pgfpathlineto{\pgfqpoint{3.731484in}{2.653437in}}%
\pgfpathlineto{\pgfqpoint{3.731484in}{2.657695in}}%
\pgfpathlineto{\pgfqpoint{3.735741in}{2.657695in}}%
\pgfpathlineto{\pgfqpoint{3.735741in}{2.653437in}}%
\pgfpathmoveto{\pgfqpoint{3.731484in}{2.657695in}}%
\pgfpathlineto{\pgfqpoint{3.731484in}{2.657695in}}%
\pgfpathlineto{\pgfqpoint{3.731484in}{2.661953in}}%
\pgfpathlineto{\pgfqpoint{3.735741in}{2.661953in}}%
\pgfpathlineto{\pgfqpoint{3.735741in}{2.657695in}}%
\pgfpathmoveto{\pgfqpoint{3.731484in}{2.661953in}}%
\pgfpathlineto{\pgfqpoint{3.731484in}{2.661953in}}%
\pgfpathlineto{\pgfqpoint{3.731484in}{2.666210in}}%
\pgfpathlineto{\pgfqpoint{3.735741in}{2.666210in}}%
\pgfpathlineto{\pgfqpoint{3.735741in}{2.661953in}}%
\pgfpathmoveto{\pgfqpoint{3.731484in}{2.666210in}}%
\pgfpathlineto{\pgfqpoint{3.731484in}{2.666210in}}%
\pgfpathlineto{\pgfqpoint{3.731484in}{2.670468in}}%
\pgfpathlineto{\pgfqpoint{3.735741in}{2.670468in}}%
\pgfpathlineto{\pgfqpoint{3.735741in}{2.666210in}}%
\pgfpathmoveto{\pgfqpoint{3.722968in}{2.670468in}}%
\pgfpathlineto{\pgfqpoint{3.722968in}{2.670468in}}%
\pgfpathlineto{\pgfqpoint{3.722968in}{2.674726in}}%
\pgfpathlineto{\pgfqpoint{3.727226in}{2.674726in}}%
\pgfpathlineto{\pgfqpoint{3.727226in}{2.670468in}}%
\pgfpathmoveto{\pgfqpoint{3.722968in}{2.674726in}}%
\pgfpathlineto{\pgfqpoint{3.722968in}{2.674726in}}%
\pgfpathlineto{\pgfqpoint{3.722968in}{2.678984in}}%
\pgfpathlineto{\pgfqpoint{3.727226in}{2.678984in}}%
\pgfpathlineto{\pgfqpoint{3.727226in}{2.674726in}}%
\pgfpathmoveto{\pgfqpoint{3.727226in}{2.670468in}}%
\pgfpathlineto{\pgfqpoint{3.727226in}{2.670468in}}%
\pgfpathlineto{\pgfqpoint{3.727226in}{2.674726in}}%
\pgfpathlineto{\pgfqpoint{3.731484in}{2.674726in}}%
\pgfpathlineto{\pgfqpoint{3.731484in}{2.670468in}}%
\pgfpathmoveto{\pgfqpoint{3.727226in}{2.674726in}}%
\pgfpathlineto{\pgfqpoint{3.727226in}{2.674726in}}%
\pgfpathlineto{\pgfqpoint{3.727226in}{2.678984in}}%
\pgfpathlineto{\pgfqpoint{3.731484in}{2.678984in}}%
\pgfpathlineto{\pgfqpoint{3.731484in}{2.674726in}}%
\pgfpathmoveto{\pgfqpoint{3.722968in}{2.678984in}}%
\pgfpathlineto{\pgfqpoint{3.722968in}{2.678984in}}%
\pgfpathlineto{\pgfqpoint{3.722968in}{2.683242in}}%
\pgfpathlineto{\pgfqpoint{3.727226in}{2.683242in}}%
\pgfpathlineto{\pgfqpoint{3.727226in}{2.678984in}}%
\pgfpathmoveto{\pgfqpoint{3.722968in}{2.683242in}}%
\pgfpathlineto{\pgfqpoint{3.722968in}{2.683242in}}%
\pgfpathlineto{\pgfqpoint{3.722968in}{2.687499in}}%
\pgfpathlineto{\pgfqpoint{3.727226in}{2.687499in}}%
\pgfpathlineto{\pgfqpoint{3.727226in}{2.683242in}}%
\pgfpathmoveto{\pgfqpoint{3.727226in}{2.678984in}}%
\pgfpathlineto{\pgfqpoint{3.727226in}{2.678984in}}%
\pgfpathlineto{\pgfqpoint{3.727226in}{2.683242in}}%
\pgfpathlineto{\pgfqpoint{3.731484in}{2.683242in}}%
\pgfpathlineto{\pgfqpoint{3.731484in}{2.678984in}}%
\pgfpathmoveto{\pgfqpoint{3.727226in}{2.683242in}}%
\pgfpathlineto{\pgfqpoint{3.727226in}{2.683242in}}%
\pgfpathlineto{\pgfqpoint{3.727226in}{2.687499in}}%
\pgfpathlineto{\pgfqpoint{3.731484in}{2.687499in}}%
\pgfpathlineto{\pgfqpoint{3.731484in}{2.683242in}}%
\pgfpathmoveto{\pgfqpoint{3.731484in}{2.670468in}}%
\pgfpathlineto{\pgfqpoint{3.731484in}{2.670468in}}%
\pgfpathlineto{\pgfqpoint{3.731484in}{2.674726in}}%
\pgfpathlineto{\pgfqpoint{3.735741in}{2.674726in}}%
\pgfpathlineto{\pgfqpoint{3.735741in}{2.670468in}}%
\pgfpathmoveto{\pgfqpoint{3.739999in}{2.619375in}}%
\pgfpathlineto{\pgfqpoint{3.739999in}{2.619375in}}%
\pgfpathlineto{\pgfqpoint{3.739999in}{2.623632in}}%
\pgfpathlineto{\pgfqpoint{3.744257in}{2.623632in}}%
\pgfpathlineto{\pgfqpoint{3.744257in}{2.619375in}}%
\pgfpathmoveto{\pgfqpoint{3.739999in}{2.623632in}}%
\pgfpathlineto{\pgfqpoint{3.739999in}{2.623632in}}%
\pgfpathlineto{\pgfqpoint{3.739999in}{2.627890in}}%
\pgfpathlineto{\pgfqpoint{3.744257in}{2.627890in}}%
\pgfpathlineto{\pgfqpoint{3.744257in}{2.623632in}}%
\pgfpathmoveto{\pgfqpoint{3.701680in}{2.764140in}}%
\pgfpathlineto{\pgfqpoint{3.701680in}{2.764140in}}%
\pgfpathlineto{\pgfqpoint{3.701680in}{2.768397in}}%
\pgfpathlineto{\pgfqpoint{3.705937in}{2.768397in}}%
\pgfpathlineto{\pgfqpoint{3.705937in}{2.764140in}}%
\pgfpathmoveto{\pgfqpoint{3.701680in}{2.768397in}}%
\pgfpathlineto{\pgfqpoint{3.701680in}{2.768397in}}%
\pgfpathlineto{\pgfqpoint{3.701680in}{2.772655in}}%
\pgfpathlineto{\pgfqpoint{3.705937in}{2.772655in}}%
\pgfpathlineto{\pgfqpoint{3.705937in}{2.768397in}}%
\pgfpathmoveto{\pgfqpoint{3.701680in}{2.772655in}}%
\pgfpathlineto{\pgfqpoint{3.701680in}{2.772655in}}%
\pgfpathlineto{\pgfqpoint{3.701680in}{2.776913in}}%
\pgfpathlineto{\pgfqpoint{3.705937in}{2.776913in}}%
\pgfpathlineto{\pgfqpoint{3.705937in}{2.772655in}}%
\pgfpathmoveto{\pgfqpoint{3.701680in}{2.776913in}}%
\pgfpathlineto{\pgfqpoint{3.701680in}{2.776913in}}%
\pgfpathlineto{\pgfqpoint{3.701680in}{2.781171in}}%
\pgfpathlineto{\pgfqpoint{3.705937in}{2.781171in}}%
\pgfpathlineto{\pgfqpoint{3.705937in}{2.776913in}}%
\pgfpathmoveto{\pgfqpoint{3.697422in}{2.781171in}}%
\pgfpathlineto{\pgfqpoint{3.697422in}{2.781171in}}%
\pgfpathlineto{\pgfqpoint{3.697422in}{2.785428in}}%
\pgfpathlineto{\pgfqpoint{3.701680in}{2.785428in}}%
\pgfpathlineto{\pgfqpoint{3.701680in}{2.781171in}}%
\pgfpathmoveto{\pgfqpoint{3.697422in}{2.785428in}}%
\pgfpathlineto{\pgfqpoint{3.697422in}{2.785428in}}%
\pgfpathlineto{\pgfqpoint{3.697422in}{2.789686in}}%
\pgfpathlineto{\pgfqpoint{3.701680in}{2.789686in}}%
\pgfpathlineto{\pgfqpoint{3.701680in}{2.785428in}}%
\pgfpathmoveto{\pgfqpoint{3.701680in}{2.781171in}}%
\pgfpathlineto{\pgfqpoint{3.701680in}{2.781171in}}%
\pgfpathlineto{\pgfqpoint{3.701680in}{2.785428in}}%
\pgfpathlineto{\pgfqpoint{3.705937in}{2.785428in}}%
\pgfpathlineto{\pgfqpoint{3.705937in}{2.781171in}}%
\pgfpathmoveto{\pgfqpoint{3.701680in}{2.785428in}}%
\pgfpathlineto{\pgfqpoint{3.701680in}{2.785428in}}%
\pgfpathlineto{\pgfqpoint{3.701680in}{2.789686in}}%
\pgfpathlineto{\pgfqpoint{3.705937in}{2.789686in}}%
\pgfpathlineto{\pgfqpoint{3.705937in}{2.785428in}}%
\pgfpathmoveto{\pgfqpoint{3.693164in}{2.802460in}}%
\pgfpathlineto{\pgfqpoint{3.693164in}{2.802460in}}%
\pgfpathlineto{\pgfqpoint{3.693164in}{2.806717in}}%
\pgfpathlineto{\pgfqpoint{3.697422in}{2.806717in}}%
\pgfpathlineto{\pgfqpoint{3.697422in}{2.802460in}}%
\pgfpathmoveto{\pgfqpoint{3.697422in}{2.789686in}}%
\pgfpathlineto{\pgfqpoint{3.697422in}{2.789686in}}%
\pgfpathlineto{\pgfqpoint{3.697422in}{2.793944in}}%
\pgfpathlineto{\pgfqpoint{3.701680in}{2.793944in}}%
\pgfpathlineto{\pgfqpoint{3.701680in}{2.789686in}}%
\pgfpathmoveto{\pgfqpoint{3.697422in}{2.793944in}}%
\pgfpathlineto{\pgfqpoint{3.697422in}{2.793944in}}%
\pgfpathlineto{\pgfqpoint{3.697422in}{2.798202in}}%
\pgfpathlineto{\pgfqpoint{3.701680in}{2.798202in}}%
\pgfpathlineto{\pgfqpoint{3.701680in}{2.793944in}}%
\pgfpathmoveto{\pgfqpoint{3.701680in}{2.789686in}}%
\pgfpathlineto{\pgfqpoint{3.701680in}{2.789686in}}%
\pgfpathlineto{\pgfqpoint{3.701680in}{2.793944in}}%
\pgfpathlineto{\pgfqpoint{3.705937in}{2.793944in}}%
\pgfpathlineto{\pgfqpoint{3.705937in}{2.789686in}}%
\pgfpathmoveto{\pgfqpoint{3.701680in}{2.793944in}}%
\pgfpathlineto{\pgfqpoint{3.701680in}{2.793944in}}%
\pgfpathlineto{\pgfqpoint{3.701680in}{2.798202in}}%
\pgfpathlineto{\pgfqpoint{3.705937in}{2.798202in}}%
\pgfpathlineto{\pgfqpoint{3.705937in}{2.793944in}}%
\pgfpathmoveto{\pgfqpoint{3.697422in}{2.798202in}}%
\pgfpathlineto{\pgfqpoint{3.697422in}{2.798202in}}%
\pgfpathlineto{\pgfqpoint{3.697422in}{2.802460in}}%
\pgfpathlineto{\pgfqpoint{3.701680in}{2.802460in}}%
\pgfpathlineto{\pgfqpoint{3.701680in}{2.798202in}}%
\pgfpathmoveto{\pgfqpoint{3.697422in}{2.802460in}}%
\pgfpathlineto{\pgfqpoint{3.697422in}{2.802460in}}%
\pgfpathlineto{\pgfqpoint{3.697422in}{2.806717in}}%
\pgfpathlineto{\pgfqpoint{3.701680in}{2.806717in}}%
\pgfpathlineto{\pgfqpoint{3.701680in}{2.802460in}}%
\pgfpathmoveto{\pgfqpoint{3.701680in}{2.798202in}}%
\pgfpathlineto{\pgfqpoint{3.701680in}{2.798202in}}%
\pgfpathlineto{\pgfqpoint{3.701680in}{2.802460in}}%
\pgfpathlineto{\pgfqpoint{3.705937in}{2.802460in}}%
\pgfpathlineto{\pgfqpoint{3.705937in}{2.798202in}}%
\pgfpathmoveto{\pgfqpoint{3.701680in}{2.802460in}}%
\pgfpathlineto{\pgfqpoint{3.701680in}{2.802460in}}%
\pgfpathlineto{\pgfqpoint{3.701680in}{2.806717in}}%
\pgfpathlineto{\pgfqpoint{3.705937in}{2.806717in}}%
\pgfpathlineto{\pgfqpoint{3.705937in}{2.802460in}}%
\pgfpathmoveto{\pgfqpoint{3.693164in}{2.806717in}}%
\pgfpathlineto{\pgfqpoint{3.693164in}{2.806717in}}%
\pgfpathlineto{\pgfqpoint{3.693164in}{2.810975in}}%
\pgfpathlineto{\pgfqpoint{3.697422in}{2.810975in}}%
\pgfpathlineto{\pgfqpoint{3.697422in}{2.806717in}}%
\pgfpathmoveto{\pgfqpoint{3.693164in}{2.810975in}}%
\pgfpathlineto{\pgfqpoint{3.693164in}{2.810975in}}%
\pgfpathlineto{\pgfqpoint{3.693164in}{2.815233in}}%
\pgfpathlineto{\pgfqpoint{3.697422in}{2.815233in}}%
\pgfpathlineto{\pgfqpoint{3.697422in}{2.810975in}}%
\pgfpathmoveto{\pgfqpoint{3.688907in}{2.819491in}}%
\pgfpathlineto{\pgfqpoint{3.688907in}{2.819491in}}%
\pgfpathlineto{\pgfqpoint{3.688907in}{2.823749in}}%
\pgfpathlineto{\pgfqpoint{3.693164in}{2.823749in}}%
\pgfpathlineto{\pgfqpoint{3.693164in}{2.819491in}}%
\pgfpathmoveto{\pgfqpoint{3.693164in}{2.815233in}}%
\pgfpathlineto{\pgfqpoint{3.693164in}{2.815233in}}%
\pgfpathlineto{\pgfqpoint{3.693164in}{2.819491in}}%
\pgfpathlineto{\pgfqpoint{3.697422in}{2.819491in}}%
\pgfpathlineto{\pgfqpoint{3.697422in}{2.815233in}}%
\pgfpathmoveto{\pgfqpoint{3.693164in}{2.819491in}}%
\pgfpathlineto{\pgfqpoint{3.693164in}{2.819491in}}%
\pgfpathlineto{\pgfqpoint{3.693164in}{2.823749in}}%
\pgfpathlineto{\pgfqpoint{3.697422in}{2.823749in}}%
\pgfpathlineto{\pgfqpoint{3.697422in}{2.819491in}}%
\pgfpathmoveto{\pgfqpoint{3.697422in}{2.806717in}}%
\pgfpathlineto{\pgfqpoint{3.697422in}{2.806717in}}%
\pgfpathlineto{\pgfqpoint{3.697422in}{2.810975in}}%
\pgfpathlineto{\pgfqpoint{3.701680in}{2.810975in}}%
\pgfpathlineto{\pgfqpoint{3.701680in}{2.806717in}}%
\pgfpathmoveto{\pgfqpoint{3.697422in}{2.810975in}}%
\pgfpathlineto{\pgfqpoint{3.697422in}{2.810975in}}%
\pgfpathlineto{\pgfqpoint{3.697422in}{2.815233in}}%
\pgfpathlineto{\pgfqpoint{3.701680in}{2.815233in}}%
\pgfpathlineto{\pgfqpoint{3.701680in}{2.810975in}}%
\pgfpathmoveto{\pgfqpoint{3.701680in}{2.806717in}}%
\pgfpathlineto{\pgfqpoint{3.701680in}{2.806717in}}%
\pgfpathlineto{\pgfqpoint{3.701680in}{2.810975in}}%
\pgfpathlineto{\pgfqpoint{3.705937in}{2.810975in}}%
\pgfpathlineto{\pgfqpoint{3.705937in}{2.806717in}}%
\pgfpathmoveto{\pgfqpoint{3.701680in}{2.810975in}}%
\pgfpathlineto{\pgfqpoint{3.701680in}{2.810975in}}%
\pgfpathlineto{\pgfqpoint{3.701680in}{2.815233in}}%
\pgfpathlineto{\pgfqpoint{3.705937in}{2.815233in}}%
\pgfpathlineto{\pgfqpoint{3.705937in}{2.810975in}}%
\pgfpathmoveto{\pgfqpoint{3.697422in}{2.815233in}}%
\pgfpathlineto{\pgfqpoint{3.697422in}{2.815233in}}%
\pgfpathlineto{\pgfqpoint{3.697422in}{2.819491in}}%
\pgfpathlineto{\pgfqpoint{3.701680in}{2.819491in}}%
\pgfpathlineto{\pgfqpoint{3.701680in}{2.815233in}}%
\pgfpathmoveto{\pgfqpoint{3.697422in}{2.819491in}}%
\pgfpathlineto{\pgfqpoint{3.697422in}{2.819491in}}%
\pgfpathlineto{\pgfqpoint{3.697422in}{2.823749in}}%
\pgfpathlineto{\pgfqpoint{3.701680in}{2.823749in}}%
\pgfpathlineto{\pgfqpoint{3.701680in}{2.819491in}}%
\pgfpathmoveto{\pgfqpoint{3.701680in}{2.815233in}}%
\pgfpathlineto{\pgfqpoint{3.701680in}{2.815233in}}%
\pgfpathlineto{\pgfqpoint{3.701680in}{2.819491in}}%
\pgfpathlineto{\pgfqpoint{3.705937in}{2.819491in}}%
\pgfpathlineto{\pgfqpoint{3.705937in}{2.815233in}}%
\pgfpathmoveto{\pgfqpoint{3.718710in}{2.687499in}}%
\pgfpathlineto{\pgfqpoint{3.718710in}{2.687499in}}%
\pgfpathlineto{\pgfqpoint{3.718710in}{2.691757in}}%
\pgfpathlineto{\pgfqpoint{3.722968in}{2.691757in}}%
\pgfpathlineto{\pgfqpoint{3.722968in}{2.687499in}}%
\pgfpathmoveto{\pgfqpoint{3.718710in}{2.691757in}}%
\pgfpathlineto{\pgfqpoint{3.718710in}{2.691757in}}%
\pgfpathlineto{\pgfqpoint{3.718710in}{2.696015in}}%
\pgfpathlineto{\pgfqpoint{3.722968in}{2.696015in}}%
\pgfpathlineto{\pgfqpoint{3.722968in}{2.691757in}}%
\pgfpathmoveto{\pgfqpoint{3.714453in}{2.700273in}}%
\pgfpathlineto{\pgfqpoint{3.714453in}{2.700273in}}%
\pgfpathlineto{\pgfqpoint{3.714453in}{2.704531in}}%
\pgfpathlineto{\pgfqpoint{3.718710in}{2.704531in}}%
\pgfpathlineto{\pgfqpoint{3.718710in}{2.700273in}}%
\pgfpathmoveto{\pgfqpoint{3.718710in}{2.696015in}}%
\pgfpathlineto{\pgfqpoint{3.718710in}{2.696015in}}%
\pgfpathlineto{\pgfqpoint{3.718710in}{2.700273in}}%
\pgfpathlineto{\pgfqpoint{3.722968in}{2.700273in}}%
\pgfpathlineto{\pgfqpoint{3.722968in}{2.696015in}}%
\pgfpathmoveto{\pgfqpoint{3.718710in}{2.700273in}}%
\pgfpathlineto{\pgfqpoint{3.718710in}{2.700273in}}%
\pgfpathlineto{\pgfqpoint{3.718710in}{2.704531in}}%
\pgfpathlineto{\pgfqpoint{3.722968in}{2.704531in}}%
\pgfpathlineto{\pgfqpoint{3.722968in}{2.700273in}}%
\pgfpathmoveto{\pgfqpoint{3.714453in}{2.704531in}}%
\pgfpathlineto{\pgfqpoint{3.714453in}{2.704531in}}%
\pgfpathlineto{\pgfqpoint{3.714453in}{2.708788in}}%
\pgfpathlineto{\pgfqpoint{3.718710in}{2.708788in}}%
\pgfpathlineto{\pgfqpoint{3.718710in}{2.704531in}}%
\pgfpathmoveto{\pgfqpoint{3.714453in}{2.708788in}}%
\pgfpathlineto{\pgfqpoint{3.714453in}{2.708788in}}%
\pgfpathlineto{\pgfqpoint{3.714453in}{2.713046in}}%
\pgfpathlineto{\pgfqpoint{3.718710in}{2.713046in}}%
\pgfpathlineto{\pgfqpoint{3.718710in}{2.708788in}}%
\pgfpathmoveto{\pgfqpoint{3.718710in}{2.704531in}}%
\pgfpathlineto{\pgfqpoint{3.718710in}{2.704531in}}%
\pgfpathlineto{\pgfqpoint{3.718710in}{2.708788in}}%
\pgfpathlineto{\pgfqpoint{3.722968in}{2.708788in}}%
\pgfpathlineto{\pgfqpoint{3.722968in}{2.704531in}}%
\pgfpathmoveto{\pgfqpoint{3.718710in}{2.708788in}}%
\pgfpathlineto{\pgfqpoint{3.718710in}{2.708788in}}%
\pgfpathlineto{\pgfqpoint{3.718710in}{2.713046in}}%
\pgfpathlineto{\pgfqpoint{3.722968in}{2.713046in}}%
\pgfpathlineto{\pgfqpoint{3.722968in}{2.708788in}}%
\pgfpathmoveto{\pgfqpoint{3.714453in}{2.713046in}}%
\pgfpathlineto{\pgfqpoint{3.714453in}{2.713046in}}%
\pgfpathlineto{\pgfqpoint{3.714453in}{2.717304in}}%
\pgfpathlineto{\pgfqpoint{3.718710in}{2.717304in}}%
\pgfpathlineto{\pgfqpoint{3.718710in}{2.713046in}}%
\pgfpathmoveto{\pgfqpoint{3.714453in}{2.717304in}}%
\pgfpathlineto{\pgfqpoint{3.714453in}{2.717304in}}%
\pgfpathlineto{\pgfqpoint{3.714453in}{2.721562in}}%
\pgfpathlineto{\pgfqpoint{3.718710in}{2.721562in}}%
\pgfpathlineto{\pgfqpoint{3.718710in}{2.717304in}}%
\pgfpathmoveto{\pgfqpoint{3.718710in}{2.713046in}}%
\pgfpathlineto{\pgfqpoint{3.718710in}{2.713046in}}%
\pgfpathlineto{\pgfqpoint{3.718710in}{2.717304in}}%
\pgfpathlineto{\pgfqpoint{3.722968in}{2.717304in}}%
\pgfpathlineto{\pgfqpoint{3.722968in}{2.713046in}}%
\pgfpathmoveto{\pgfqpoint{3.718710in}{2.717304in}}%
\pgfpathlineto{\pgfqpoint{3.718710in}{2.717304in}}%
\pgfpathlineto{\pgfqpoint{3.718710in}{2.721562in}}%
\pgfpathlineto{\pgfqpoint{3.722968in}{2.721562in}}%
\pgfpathlineto{\pgfqpoint{3.722968in}{2.717304in}}%
\pgfpathmoveto{\pgfqpoint{3.722968in}{2.687499in}}%
\pgfpathlineto{\pgfqpoint{3.722968in}{2.687499in}}%
\pgfpathlineto{\pgfqpoint{3.722968in}{2.691757in}}%
\pgfpathlineto{\pgfqpoint{3.727226in}{2.691757in}}%
\pgfpathlineto{\pgfqpoint{3.727226in}{2.687499in}}%
\pgfpathmoveto{\pgfqpoint{3.722968in}{2.691757in}}%
\pgfpathlineto{\pgfqpoint{3.722968in}{2.691757in}}%
\pgfpathlineto{\pgfqpoint{3.722968in}{2.696015in}}%
\pgfpathlineto{\pgfqpoint{3.727226in}{2.696015in}}%
\pgfpathlineto{\pgfqpoint{3.727226in}{2.691757in}}%
\pgfpathmoveto{\pgfqpoint{3.727226in}{2.687499in}}%
\pgfpathlineto{\pgfqpoint{3.727226in}{2.687499in}}%
\pgfpathlineto{\pgfqpoint{3.727226in}{2.691757in}}%
\pgfpathlineto{\pgfqpoint{3.731484in}{2.691757in}}%
\pgfpathlineto{\pgfqpoint{3.731484in}{2.687499in}}%
\pgfpathmoveto{\pgfqpoint{3.727226in}{2.691757in}}%
\pgfpathlineto{\pgfqpoint{3.727226in}{2.691757in}}%
\pgfpathlineto{\pgfqpoint{3.727226in}{2.696015in}}%
\pgfpathlineto{\pgfqpoint{3.731484in}{2.696015in}}%
\pgfpathlineto{\pgfqpoint{3.731484in}{2.691757in}}%
\pgfpathmoveto{\pgfqpoint{3.722968in}{2.696015in}}%
\pgfpathlineto{\pgfqpoint{3.722968in}{2.696015in}}%
\pgfpathlineto{\pgfqpoint{3.722968in}{2.700273in}}%
\pgfpathlineto{\pgfqpoint{3.727226in}{2.700273in}}%
\pgfpathlineto{\pgfqpoint{3.727226in}{2.696015in}}%
\pgfpathmoveto{\pgfqpoint{3.722968in}{2.700273in}}%
\pgfpathlineto{\pgfqpoint{3.722968in}{2.700273in}}%
\pgfpathlineto{\pgfqpoint{3.722968in}{2.704531in}}%
\pgfpathlineto{\pgfqpoint{3.727226in}{2.704531in}}%
\pgfpathlineto{\pgfqpoint{3.727226in}{2.700273in}}%
\pgfpathmoveto{\pgfqpoint{3.722968in}{2.704531in}}%
\pgfpathlineto{\pgfqpoint{3.722968in}{2.704531in}}%
\pgfpathlineto{\pgfqpoint{3.722968in}{2.708788in}}%
\pgfpathlineto{\pgfqpoint{3.727226in}{2.708788in}}%
\pgfpathlineto{\pgfqpoint{3.727226in}{2.704531in}}%
\pgfpathmoveto{\pgfqpoint{3.722968in}{2.708788in}}%
\pgfpathlineto{\pgfqpoint{3.722968in}{2.708788in}}%
\pgfpathlineto{\pgfqpoint{3.722968in}{2.713046in}}%
\pgfpathlineto{\pgfqpoint{3.727226in}{2.713046in}}%
\pgfpathlineto{\pgfqpoint{3.727226in}{2.708788in}}%
\pgfpathmoveto{\pgfqpoint{3.722968in}{2.713046in}}%
\pgfpathlineto{\pgfqpoint{3.722968in}{2.713046in}}%
\pgfpathlineto{\pgfqpoint{3.722968in}{2.717304in}}%
\pgfpathlineto{\pgfqpoint{3.727226in}{2.717304in}}%
\pgfpathlineto{\pgfqpoint{3.727226in}{2.713046in}}%
\pgfpathmoveto{\pgfqpoint{3.710195in}{2.721562in}}%
\pgfpathlineto{\pgfqpoint{3.710195in}{2.721562in}}%
\pgfpathlineto{\pgfqpoint{3.710195in}{2.725819in}}%
\pgfpathlineto{\pgfqpoint{3.714453in}{2.725819in}}%
\pgfpathlineto{\pgfqpoint{3.714453in}{2.721562in}}%
\pgfpathmoveto{\pgfqpoint{3.710195in}{2.725819in}}%
\pgfpathlineto{\pgfqpoint{3.710195in}{2.725819in}}%
\pgfpathlineto{\pgfqpoint{3.710195in}{2.730077in}}%
\pgfpathlineto{\pgfqpoint{3.714453in}{2.730077in}}%
\pgfpathlineto{\pgfqpoint{3.714453in}{2.725819in}}%
\pgfpathmoveto{\pgfqpoint{3.710195in}{2.730077in}}%
\pgfpathlineto{\pgfqpoint{3.710195in}{2.730077in}}%
\pgfpathlineto{\pgfqpoint{3.710195in}{2.734335in}}%
\pgfpathlineto{\pgfqpoint{3.714453in}{2.734335in}}%
\pgfpathlineto{\pgfqpoint{3.714453in}{2.730077in}}%
\pgfpathmoveto{\pgfqpoint{3.710195in}{2.734335in}}%
\pgfpathlineto{\pgfqpoint{3.710195in}{2.734335in}}%
\pgfpathlineto{\pgfqpoint{3.710195in}{2.738593in}}%
\pgfpathlineto{\pgfqpoint{3.714453in}{2.738593in}}%
\pgfpathlineto{\pgfqpoint{3.714453in}{2.734335in}}%
\pgfpathmoveto{\pgfqpoint{3.714453in}{2.721562in}}%
\pgfpathlineto{\pgfqpoint{3.714453in}{2.721562in}}%
\pgfpathlineto{\pgfqpoint{3.714453in}{2.725819in}}%
\pgfpathlineto{\pgfqpoint{3.718710in}{2.725819in}}%
\pgfpathlineto{\pgfqpoint{3.718710in}{2.721562in}}%
\pgfpathmoveto{\pgfqpoint{3.714453in}{2.725819in}}%
\pgfpathlineto{\pgfqpoint{3.714453in}{2.725819in}}%
\pgfpathlineto{\pgfqpoint{3.714453in}{2.730077in}}%
\pgfpathlineto{\pgfqpoint{3.718710in}{2.730077in}}%
\pgfpathlineto{\pgfqpoint{3.718710in}{2.725819in}}%
\pgfpathmoveto{\pgfqpoint{3.718710in}{2.721562in}}%
\pgfpathlineto{\pgfqpoint{3.718710in}{2.721562in}}%
\pgfpathlineto{\pgfqpoint{3.718710in}{2.725819in}}%
\pgfpathlineto{\pgfqpoint{3.722968in}{2.725819in}}%
\pgfpathlineto{\pgfqpoint{3.722968in}{2.721562in}}%
\pgfpathmoveto{\pgfqpoint{3.718710in}{2.725819in}}%
\pgfpathlineto{\pgfqpoint{3.718710in}{2.725819in}}%
\pgfpathlineto{\pgfqpoint{3.718710in}{2.730077in}}%
\pgfpathlineto{\pgfqpoint{3.722968in}{2.730077in}}%
\pgfpathlineto{\pgfqpoint{3.722968in}{2.725819in}}%
\pgfpathmoveto{\pgfqpoint{3.714453in}{2.730077in}}%
\pgfpathlineto{\pgfqpoint{3.714453in}{2.730077in}}%
\pgfpathlineto{\pgfqpoint{3.714453in}{2.734335in}}%
\pgfpathlineto{\pgfqpoint{3.718710in}{2.734335in}}%
\pgfpathlineto{\pgfqpoint{3.718710in}{2.730077in}}%
\pgfpathmoveto{\pgfqpoint{3.714453in}{2.734335in}}%
\pgfpathlineto{\pgfqpoint{3.714453in}{2.734335in}}%
\pgfpathlineto{\pgfqpoint{3.714453in}{2.738593in}}%
\pgfpathlineto{\pgfqpoint{3.718710in}{2.738593in}}%
\pgfpathlineto{\pgfqpoint{3.718710in}{2.734335in}}%
\pgfpathmoveto{\pgfqpoint{3.718710in}{2.730077in}}%
\pgfpathlineto{\pgfqpoint{3.718710in}{2.730077in}}%
\pgfpathlineto{\pgfqpoint{3.718710in}{2.734335in}}%
\pgfpathlineto{\pgfqpoint{3.722968in}{2.734335in}}%
\pgfpathlineto{\pgfqpoint{3.722968in}{2.730077in}}%
\pgfpathmoveto{\pgfqpoint{3.718710in}{2.734335in}}%
\pgfpathlineto{\pgfqpoint{3.718710in}{2.734335in}}%
\pgfpathlineto{\pgfqpoint{3.718710in}{2.738593in}}%
\pgfpathlineto{\pgfqpoint{3.722968in}{2.738593in}}%
\pgfpathlineto{\pgfqpoint{3.722968in}{2.734335in}}%
\pgfpathmoveto{\pgfqpoint{3.705937in}{2.742851in}}%
\pgfpathlineto{\pgfqpoint{3.705937in}{2.742851in}}%
\pgfpathlineto{\pgfqpoint{3.705937in}{2.747108in}}%
\pgfpathlineto{\pgfqpoint{3.710195in}{2.747108in}}%
\pgfpathlineto{\pgfqpoint{3.710195in}{2.742851in}}%
\pgfpathmoveto{\pgfqpoint{3.710195in}{2.738593in}}%
\pgfpathlineto{\pgfqpoint{3.710195in}{2.738593in}}%
\pgfpathlineto{\pgfqpoint{3.710195in}{2.742851in}}%
\pgfpathlineto{\pgfqpoint{3.714453in}{2.742851in}}%
\pgfpathlineto{\pgfqpoint{3.714453in}{2.738593in}}%
\pgfpathmoveto{\pgfqpoint{3.710195in}{2.742851in}}%
\pgfpathlineto{\pgfqpoint{3.710195in}{2.742851in}}%
\pgfpathlineto{\pgfqpoint{3.710195in}{2.747108in}}%
\pgfpathlineto{\pgfqpoint{3.714453in}{2.747108in}}%
\pgfpathlineto{\pgfqpoint{3.714453in}{2.742851in}}%
\pgfpathmoveto{\pgfqpoint{3.705937in}{2.747108in}}%
\pgfpathlineto{\pgfqpoint{3.705937in}{2.747108in}}%
\pgfpathlineto{\pgfqpoint{3.705937in}{2.751366in}}%
\pgfpathlineto{\pgfqpoint{3.710195in}{2.751366in}}%
\pgfpathlineto{\pgfqpoint{3.710195in}{2.747108in}}%
\pgfpathmoveto{\pgfqpoint{3.705937in}{2.751366in}}%
\pgfpathlineto{\pgfqpoint{3.705937in}{2.751366in}}%
\pgfpathlineto{\pgfqpoint{3.705937in}{2.755624in}}%
\pgfpathlineto{\pgfqpoint{3.710195in}{2.755624in}}%
\pgfpathlineto{\pgfqpoint{3.710195in}{2.751366in}}%
\pgfpathmoveto{\pgfqpoint{3.710195in}{2.747108in}}%
\pgfpathlineto{\pgfqpoint{3.710195in}{2.747108in}}%
\pgfpathlineto{\pgfqpoint{3.710195in}{2.751366in}}%
\pgfpathlineto{\pgfqpoint{3.714453in}{2.751366in}}%
\pgfpathlineto{\pgfqpoint{3.714453in}{2.747108in}}%
\pgfpathmoveto{\pgfqpoint{3.710195in}{2.751366in}}%
\pgfpathlineto{\pgfqpoint{3.710195in}{2.751366in}}%
\pgfpathlineto{\pgfqpoint{3.710195in}{2.755624in}}%
\pgfpathlineto{\pgfqpoint{3.714453in}{2.755624in}}%
\pgfpathlineto{\pgfqpoint{3.714453in}{2.751366in}}%
\pgfpathmoveto{\pgfqpoint{3.714453in}{2.738593in}}%
\pgfpathlineto{\pgfqpoint{3.714453in}{2.738593in}}%
\pgfpathlineto{\pgfqpoint{3.714453in}{2.742851in}}%
\pgfpathlineto{\pgfqpoint{3.718710in}{2.742851in}}%
\pgfpathlineto{\pgfqpoint{3.718710in}{2.738593in}}%
\pgfpathmoveto{\pgfqpoint{3.714453in}{2.742851in}}%
\pgfpathlineto{\pgfqpoint{3.714453in}{2.742851in}}%
\pgfpathlineto{\pgfqpoint{3.714453in}{2.747108in}}%
\pgfpathlineto{\pgfqpoint{3.718710in}{2.747108in}}%
\pgfpathlineto{\pgfqpoint{3.718710in}{2.742851in}}%
\pgfpathmoveto{\pgfqpoint{3.714453in}{2.747108in}}%
\pgfpathlineto{\pgfqpoint{3.714453in}{2.747108in}}%
\pgfpathlineto{\pgfqpoint{3.714453in}{2.751366in}}%
\pgfpathlineto{\pgfqpoint{3.718710in}{2.751366in}}%
\pgfpathlineto{\pgfqpoint{3.718710in}{2.747108in}}%
\pgfpathmoveto{\pgfqpoint{3.714453in}{2.751366in}}%
\pgfpathlineto{\pgfqpoint{3.714453in}{2.751366in}}%
\pgfpathlineto{\pgfqpoint{3.714453in}{2.755624in}}%
\pgfpathlineto{\pgfqpoint{3.718710in}{2.755624in}}%
\pgfpathlineto{\pgfqpoint{3.718710in}{2.751366in}}%
\pgfpathmoveto{\pgfqpoint{3.705937in}{2.755624in}}%
\pgfpathlineto{\pgfqpoint{3.705937in}{2.755624in}}%
\pgfpathlineto{\pgfqpoint{3.705937in}{2.759882in}}%
\pgfpathlineto{\pgfqpoint{3.710195in}{2.759882in}}%
\pgfpathlineto{\pgfqpoint{3.710195in}{2.755624in}}%
\pgfpathmoveto{\pgfqpoint{3.705937in}{2.759882in}}%
\pgfpathlineto{\pgfqpoint{3.705937in}{2.759882in}}%
\pgfpathlineto{\pgfqpoint{3.705937in}{2.764140in}}%
\pgfpathlineto{\pgfqpoint{3.710195in}{2.764140in}}%
\pgfpathlineto{\pgfqpoint{3.710195in}{2.759882in}}%
\pgfpathmoveto{\pgfqpoint{3.710195in}{2.755624in}}%
\pgfpathlineto{\pgfqpoint{3.710195in}{2.755624in}}%
\pgfpathlineto{\pgfqpoint{3.710195in}{2.759882in}}%
\pgfpathlineto{\pgfqpoint{3.714453in}{2.759882in}}%
\pgfpathlineto{\pgfqpoint{3.714453in}{2.755624in}}%
\pgfpathmoveto{\pgfqpoint{3.710195in}{2.759882in}}%
\pgfpathlineto{\pgfqpoint{3.710195in}{2.759882in}}%
\pgfpathlineto{\pgfqpoint{3.710195in}{2.764140in}}%
\pgfpathlineto{\pgfqpoint{3.714453in}{2.764140in}}%
\pgfpathlineto{\pgfqpoint{3.714453in}{2.759882in}}%
\pgfpathmoveto{\pgfqpoint{3.705937in}{2.764140in}}%
\pgfpathlineto{\pgfqpoint{3.705937in}{2.764140in}}%
\pgfpathlineto{\pgfqpoint{3.705937in}{2.768397in}}%
\pgfpathlineto{\pgfqpoint{3.710195in}{2.768397in}}%
\pgfpathlineto{\pgfqpoint{3.710195in}{2.764140in}}%
\pgfpathmoveto{\pgfqpoint{3.705937in}{2.768397in}}%
\pgfpathlineto{\pgfqpoint{3.705937in}{2.768397in}}%
\pgfpathlineto{\pgfqpoint{3.705937in}{2.772655in}}%
\pgfpathlineto{\pgfqpoint{3.710195in}{2.772655in}}%
\pgfpathlineto{\pgfqpoint{3.710195in}{2.768397in}}%
\pgfpathmoveto{\pgfqpoint{3.710195in}{2.764140in}}%
\pgfpathlineto{\pgfqpoint{3.710195in}{2.764140in}}%
\pgfpathlineto{\pgfqpoint{3.710195in}{2.768397in}}%
\pgfpathlineto{\pgfqpoint{3.714453in}{2.768397in}}%
\pgfpathlineto{\pgfqpoint{3.714453in}{2.764140in}}%
\pgfpathmoveto{\pgfqpoint{3.710195in}{2.768397in}}%
\pgfpathlineto{\pgfqpoint{3.710195in}{2.768397in}}%
\pgfpathlineto{\pgfqpoint{3.710195in}{2.772655in}}%
\pgfpathlineto{\pgfqpoint{3.714453in}{2.772655in}}%
\pgfpathlineto{\pgfqpoint{3.714453in}{2.768397in}}%
\pgfpathmoveto{\pgfqpoint{3.714453in}{2.755624in}}%
\pgfpathlineto{\pgfqpoint{3.714453in}{2.755624in}}%
\pgfpathlineto{\pgfqpoint{3.714453in}{2.759882in}}%
\pgfpathlineto{\pgfqpoint{3.718710in}{2.759882in}}%
\pgfpathlineto{\pgfqpoint{3.718710in}{2.755624in}}%
\pgfpathmoveto{\pgfqpoint{3.705937in}{2.772655in}}%
\pgfpathlineto{\pgfqpoint{3.705937in}{2.772655in}}%
\pgfpathlineto{\pgfqpoint{3.705937in}{2.776913in}}%
\pgfpathlineto{\pgfqpoint{3.710195in}{2.776913in}}%
\pgfpathlineto{\pgfqpoint{3.710195in}{2.772655in}}%
\pgfpathmoveto{\pgfqpoint{3.705937in}{2.776913in}}%
\pgfpathlineto{\pgfqpoint{3.705937in}{2.776913in}}%
\pgfpathlineto{\pgfqpoint{3.705937in}{2.781171in}}%
\pgfpathlineto{\pgfqpoint{3.710195in}{2.781171in}}%
\pgfpathlineto{\pgfqpoint{3.710195in}{2.776913in}}%
\pgfpathmoveto{\pgfqpoint{3.710195in}{2.772655in}}%
\pgfpathlineto{\pgfqpoint{3.710195in}{2.772655in}}%
\pgfpathlineto{\pgfqpoint{3.710195in}{2.776913in}}%
\pgfpathlineto{\pgfqpoint{3.714453in}{2.776913in}}%
\pgfpathlineto{\pgfqpoint{3.714453in}{2.772655in}}%
\pgfpathmoveto{\pgfqpoint{3.710195in}{2.776913in}}%
\pgfpathlineto{\pgfqpoint{3.710195in}{2.776913in}}%
\pgfpathlineto{\pgfqpoint{3.710195in}{2.781171in}}%
\pgfpathlineto{\pgfqpoint{3.714453in}{2.781171in}}%
\pgfpathlineto{\pgfqpoint{3.714453in}{2.776913in}}%
\pgfpathmoveto{\pgfqpoint{3.705937in}{2.781171in}}%
\pgfpathlineto{\pgfqpoint{3.705937in}{2.781171in}}%
\pgfpathlineto{\pgfqpoint{3.705937in}{2.785428in}}%
\pgfpathlineto{\pgfqpoint{3.710195in}{2.785428in}}%
\pgfpathlineto{\pgfqpoint{3.710195in}{2.781171in}}%
\pgfpathmoveto{\pgfqpoint{3.705937in}{2.785428in}}%
\pgfpathlineto{\pgfqpoint{3.705937in}{2.785428in}}%
\pgfpathlineto{\pgfqpoint{3.705937in}{2.789686in}}%
\pgfpathlineto{\pgfqpoint{3.710195in}{2.789686in}}%
\pgfpathlineto{\pgfqpoint{3.710195in}{2.785428in}}%
\pgfpathmoveto{\pgfqpoint{3.705937in}{2.789686in}}%
\pgfpathlineto{\pgfqpoint{3.705937in}{2.789686in}}%
\pgfpathlineto{\pgfqpoint{3.705937in}{2.793944in}}%
\pgfpathlineto{\pgfqpoint{3.710195in}{2.793944in}}%
\pgfpathlineto{\pgfqpoint{3.710195in}{2.789686in}}%
\pgfpathmoveto{\pgfqpoint{3.705937in}{2.793944in}}%
\pgfpathlineto{\pgfqpoint{3.705937in}{2.793944in}}%
\pgfpathlineto{\pgfqpoint{3.705937in}{2.798202in}}%
\pgfpathlineto{\pgfqpoint{3.710195in}{2.798202in}}%
\pgfpathlineto{\pgfqpoint{3.710195in}{2.793944in}}%
\pgfpathmoveto{\pgfqpoint{3.705937in}{2.798202in}}%
\pgfpathlineto{\pgfqpoint{3.705937in}{2.798202in}}%
\pgfpathlineto{\pgfqpoint{3.705937in}{2.802460in}}%
\pgfpathlineto{\pgfqpoint{3.710195in}{2.802460in}}%
\pgfpathlineto{\pgfqpoint{3.710195in}{2.798202in}}%
\pgfpathmoveto{\pgfqpoint{3.684649in}{2.840780in}}%
\pgfpathlineto{\pgfqpoint{3.684649in}{2.840780in}}%
\pgfpathlineto{\pgfqpoint{3.684649in}{2.845037in}}%
\pgfpathlineto{\pgfqpoint{3.688907in}{2.845037in}}%
\pgfpathlineto{\pgfqpoint{3.688907in}{2.840780in}}%
\pgfpathmoveto{\pgfqpoint{3.684649in}{2.845037in}}%
\pgfpathlineto{\pgfqpoint{3.684649in}{2.845037in}}%
\pgfpathlineto{\pgfqpoint{3.684649in}{2.849295in}}%
\pgfpathlineto{\pgfqpoint{3.688907in}{2.849295in}}%
\pgfpathlineto{\pgfqpoint{3.688907in}{2.845037in}}%
\pgfpathmoveto{\pgfqpoint{3.684649in}{2.849295in}}%
\pgfpathlineto{\pgfqpoint{3.684649in}{2.849295in}}%
\pgfpathlineto{\pgfqpoint{3.684649in}{2.853553in}}%
\pgfpathlineto{\pgfqpoint{3.688907in}{2.853553in}}%
\pgfpathlineto{\pgfqpoint{3.688907in}{2.849295in}}%
\pgfpathmoveto{\pgfqpoint{3.684649in}{2.853553in}}%
\pgfpathlineto{\pgfqpoint{3.684649in}{2.853553in}}%
\pgfpathlineto{\pgfqpoint{3.684649in}{2.857811in}}%
\pgfpathlineto{\pgfqpoint{3.688907in}{2.857811in}}%
\pgfpathlineto{\pgfqpoint{3.688907in}{2.853553in}}%
\pgfpathmoveto{\pgfqpoint{3.688907in}{2.823749in}}%
\pgfpathlineto{\pgfqpoint{3.688907in}{2.823749in}}%
\pgfpathlineto{\pgfqpoint{3.688907in}{2.828006in}}%
\pgfpathlineto{\pgfqpoint{3.693164in}{2.828006in}}%
\pgfpathlineto{\pgfqpoint{3.693164in}{2.823749in}}%
\pgfpathmoveto{\pgfqpoint{3.688907in}{2.828006in}}%
\pgfpathlineto{\pgfqpoint{3.688907in}{2.828006in}}%
\pgfpathlineto{\pgfqpoint{3.688907in}{2.832264in}}%
\pgfpathlineto{\pgfqpoint{3.693164in}{2.832264in}}%
\pgfpathlineto{\pgfqpoint{3.693164in}{2.828006in}}%
\pgfpathmoveto{\pgfqpoint{3.693164in}{2.823749in}}%
\pgfpathlineto{\pgfqpoint{3.693164in}{2.823749in}}%
\pgfpathlineto{\pgfqpoint{3.693164in}{2.828006in}}%
\pgfpathlineto{\pgfqpoint{3.697422in}{2.828006in}}%
\pgfpathlineto{\pgfqpoint{3.697422in}{2.823749in}}%
\pgfpathmoveto{\pgfqpoint{3.693164in}{2.828006in}}%
\pgfpathlineto{\pgfqpoint{3.693164in}{2.828006in}}%
\pgfpathlineto{\pgfqpoint{3.693164in}{2.832264in}}%
\pgfpathlineto{\pgfqpoint{3.697422in}{2.832264in}}%
\pgfpathlineto{\pgfqpoint{3.697422in}{2.828006in}}%
\pgfpathmoveto{\pgfqpoint{3.688907in}{2.832264in}}%
\pgfpathlineto{\pgfqpoint{3.688907in}{2.832264in}}%
\pgfpathlineto{\pgfqpoint{3.688907in}{2.836522in}}%
\pgfpathlineto{\pgfqpoint{3.693164in}{2.836522in}}%
\pgfpathlineto{\pgfqpoint{3.693164in}{2.832264in}}%
\pgfpathmoveto{\pgfqpoint{3.688907in}{2.836522in}}%
\pgfpathlineto{\pgfqpoint{3.688907in}{2.836522in}}%
\pgfpathlineto{\pgfqpoint{3.688907in}{2.840780in}}%
\pgfpathlineto{\pgfqpoint{3.693164in}{2.840780in}}%
\pgfpathlineto{\pgfqpoint{3.693164in}{2.836522in}}%
\pgfpathmoveto{\pgfqpoint{3.693164in}{2.832264in}}%
\pgfpathlineto{\pgfqpoint{3.693164in}{2.832264in}}%
\pgfpathlineto{\pgfqpoint{3.693164in}{2.836522in}}%
\pgfpathlineto{\pgfqpoint{3.697422in}{2.836522in}}%
\pgfpathlineto{\pgfqpoint{3.697422in}{2.832264in}}%
\pgfpathmoveto{\pgfqpoint{3.693164in}{2.836522in}}%
\pgfpathlineto{\pgfqpoint{3.693164in}{2.836522in}}%
\pgfpathlineto{\pgfqpoint{3.693164in}{2.840780in}}%
\pgfpathlineto{\pgfqpoint{3.697422in}{2.840780in}}%
\pgfpathlineto{\pgfqpoint{3.697422in}{2.836522in}}%
\pgfpathmoveto{\pgfqpoint{3.697422in}{2.823749in}}%
\pgfpathlineto{\pgfqpoint{3.697422in}{2.823749in}}%
\pgfpathlineto{\pgfqpoint{3.697422in}{2.828006in}}%
\pgfpathlineto{\pgfqpoint{3.701680in}{2.828006in}}%
\pgfpathlineto{\pgfqpoint{3.701680in}{2.823749in}}%
\pgfpathmoveto{\pgfqpoint{3.697422in}{2.828006in}}%
\pgfpathlineto{\pgfqpoint{3.697422in}{2.828006in}}%
\pgfpathlineto{\pgfqpoint{3.697422in}{2.832264in}}%
\pgfpathlineto{\pgfqpoint{3.701680in}{2.832264in}}%
\pgfpathlineto{\pgfqpoint{3.701680in}{2.828006in}}%
\pgfpathmoveto{\pgfqpoint{3.697422in}{2.832264in}}%
\pgfpathlineto{\pgfqpoint{3.697422in}{2.832264in}}%
\pgfpathlineto{\pgfqpoint{3.697422in}{2.836522in}}%
\pgfpathlineto{\pgfqpoint{3.701680in}{2.836522in}}%
\pgfpathlineto{\pgfqpoint{3.701680in}{2.832264in}}%
\pgfpathmoveto{\pgfqpoint{3.697422in}{2.836522in}}%
\pgfpathlineto{\pgfqpoint{3.697422in}{2.836522in}}%
\pgfpathlineto{\pgfqpoint{3.697422in}{2.840780in}}%
\pgfpathlineto{\pgfqpoint{3.701680in}{2.840780in}}%
\pgfpathlineto{\pgfqpoint{3.701680in}{2.836522in}}%
\pgfpathmoveto{\pgfqpoint{3.688907in}{2.840780in}}%
\pgfpathlineto{\pgfqpoint{3.688907in}{2.840780in}}%
\pgfpathlineto{\pgfqpoint{3.688907in}{2.845037in}}%
\pgfpathlineto{\pgfqpoint{3.693164in}{2.845037in}}%
\pgfpathlineto{\pgfqpoint{3.693164in}{2.840780in}}%
\pgfpathmoveto{\pgfqpoint{3.688907in}{2.845037in}}%
\pgfpathlineto{\pgfqpoint{3.688907in}{2.845037in}}%
\pgfpathlineto{\pgfqpoint{3.688907in}{2.849295in}}%
\pgfpathlineto{\pgfqpoint{3.693164in}{2.849295in}}%
\pgfpathlineto{\pgfqpoint{3.693164in}{2.845037in}}%
\pgfpathmoveto{\pgfqpoint{3.693164in}{2.840780in}}%
\pgfpathlineto{\pgfqpoint{3.693164in}{2.840780in}}%
\pgfpathlineto{\pgfqpoint{3.693164in}{2.845037in}}%
\pgfpathlineto{\pgfqpoint{3.697422in}{2.845037in}}%
\pgfpathlineto{\pgfqpoint{3.697422in}{2.840780in}}%
\pgfpathmoveto{\pgfqpoint{3.693164in}{2.845037in}}%
\pgfpathlineto{\pgfqpoint{3.693164in}{2.845037in}}%
\pgfpathlineto{\pgfqpoint{3.693164in}{2.849295in}}%
\pgfpathlineto{\pgfqpoint{3.697422in}{2.849295in}}%
\pgfpathlineto{\pgfqpoint{3.697422in}{2.845037in}}%
\pgfpathmoveto{\pgfqpoint{3.688907in}{2.849295in}}%
\pgfpathlineto{\pgfqpoint{3.688907in}{2.849295in}}%
\pgfpathlineto{\pgfqpoint{3.688907in}{2.853553in}}%
\pgfpathlineto{\pgfqpoint{3.693164in}{2.853553in}}%
\pgfpathlineto{\pgfqpoint{3.693164in}{2.849295in}}%
\pgfpathmoveto{\pgfqpoint{3.688907in}{2.853553in}}%
\pgfpathlineto{\pgfqpoint{3.688907in}{2.853553in}}%
\pgfpathlineto{\pgfqpoint{3.688907in}{2.857811in}}%
\pgfpathlineto{\pgfqpoint{3.693164in}{2.857811in}}%
\pgfpathlineto{\pgfqpoint{3.693164in}{2.853553in}}%
\pgfpathmoveto{\pgfqpoint{3.693164in}{2.849295in}}%
\pgfpathlineto{\pgfqpoint{3.693164in}{2.849295in}}%
\pgfpathlineto{\pgfqpoint{3.693164in}{2.853553in}}%
\pgfpathlineto{\pgfqpoint{3.697422in}{2.853553in}}%
\pgfpathlineto{\pgfqpoint{3.697422in}{2.849295in}}%
\pgfpathmoveto{\pgfqpoint{3.693164in}{2.853553in}}%
\pgfpathlineto{\pgfqpoint{3.693164in}{2.853553in}}%
\pgfpathlineto{\pgfqpoint{3.693164in}{2.857811in}}%
\pgfpathlineto{\pgfqpoint{3.697422in}{2.857811in}}%
\pgfpathlineto{\pgfqpoint{3.697422in}{2.853553in}}%
\pgfpathmoveto{\pgfqpoint{3.680391in}{2.857811in}}%
\pgfpathlineto{\pgfqpoint{3.680391in}{2.857811in}}%
\pgfpathlineto{\pgfqpoint{3.680391in}{2.862069in}}%
\pgfpathlineto{\pgfqpoint{3.684649in}{2.862069in}}%
\pgfpathlineto{\pgfqpoint{3.684649in}{2.857811in}}%
\pgfpathmoveto{\pgfqpoint{3.680391in}{2.862069in}}%
\pgfpathlineto{\pgfqpoint{3.680391in}{2.862069in}}%
\pgfpathlineto{\pgfqpoint{3.680391in}{2.866326in}}%
\pgfpathlineto{\pgfqpoint{3.684649in}{2.866326in}}%
\pgfpathlineto{\pgfqpoint{3.684649in}{2.862069in}}%
\pgfpathmoveto{\pgfqpoint{3.684649in}{2.857811in}}%
\pgfpathlineto{\pgfqpoint{3.684649in}{2.857811in}}%
\pgfpathlineto{\pgfqpoint{3.684649in}{2.862069in}}%
\pgfpathlineto{\pgfqpoint{3.688907in}{2.862069in}}%
\pgfpathlineto{\pgfqpoint{3.688907in}{2.857811in}}%
\pgfpathmoveto{\pgfqpoint{3.684649in}{2.862069in}}%
\pgfpathlineto{\pgfqpoint{3.684649in}{2.862069in}}%
\pgfpathlineto{\pgfqpoint{3.684649in}{2.866326in}}%
\pgfpathlineto{\pgfqpoint{3.688907in}{2.866326in}}%
\pgfpathlineto{\pgfqpoint{3.688907in}{2.862069in}}%
\pgfpathmoveto{\pgfqpoint{3.680391in}{2.866326in}}%
\pgfpathlineto{\pgfqpoint{3.680391in}{2.866326in}}%
\pgfpathlineto{\pgfqpoint{3.680391in}{2.870584in}}%
\pgfpathlineto{\pgfqpoint{3.684649in}{2.870584in}}%
\pgfpathlineto{\pgfqpoint{3.684649in}{2.866326in}}%
\pgfpathmoveto{\pgfqpoint{3.680391in}{2.870584in}}%
\pgfpathlineto{\pgfqpoint{3.680391in}{2.870584in}}%
\pgfpathlineto{\pgfqpoint{3.680391in}{2.874842in}}%
\pgfpathlineto{\pgfqpoint{3.684649in}{2.874842in}}%
\pgfpathlineto{\pgfqpoint{3.684649in}{2.870584in}}%
\pgfpathmoveto{\pgfqpoint{3.684649in}{2.866326in}}%
\pgfpathlineto{\pgfqpoint{3.684649in}{2.866326in}}%
\pgfpathlineto{\pgfqpoint{3.684649in}{2.870584in}}%
\pgfpathlineto{\pgfqpoint{3.688907in}{2.870584in}}%
\pgfpathlineto{\pgfqpoint{3.688907in}{2.866326in}}%
\pgfpathmoveto{\pgfqpoint{3.684649in}{2.870584in}}%
\pgfpathlineto{\pgfqpoint{3.684649in}{2.870584in}}%
\pgfpathlineto{\pgfqpoint{3.684649in}{2.874842in}}%
\pgfpathlineto{\pgfqpoint{3.688907in}{2.874842in}}%
\pgfpathlineto{\pgfqpoint{3.688907in}{2.870584in}}%
\pgfpathmoveto{\pgfqpoint{3.676133in}{2.879100in}}%
\pgfpathlineto{\pgfqpoint{3.676133in}{2.879100in}}%
\pgfpathlineto{\pgfqpoint{3.676133in}{2.883358in}}%
\pgfpathlineto{\pgfqpoint{3.680391in}{2.883358in}}%
\pgfpathlineto{\pgfqpoint{3.680391in}{2.879100in}}%
\pgfpathmoveto{\pgfqpoint{3.676133in}{2.883358in}}%
\pgfpathlineto{\pgfqpoint{3.676133in}{2.883358in}}%
\pgfpathlineto{\pgfqpoint{3.676133in}{2.887615in}}%
\pgfpathlineto{\pgfqpoint{3.680391in}{2.887615in}}%
\pgfpathlineto{\pgfqpoint{3.680391in}{2.883358in}}%
\pgfpathmoveto{\pgfqpoint{3.676133in}{2.887615in}}%
\pgfpathlineto{\pgfqpoint{3.676133in}{2.887615in}}%
\pgfpathlineto{\pgfqpoint{3.676133in}{2.891873in}}%
\pgfpathlineto{\pgfqpoint{3.680391in}{2.891873in}}%
\pgfpathlineto{\pgfqpoint{3.680391in}{2.887615in}}%
\pgfpathmoveto{\pgfqpoint{3.680391in}{2.874842in}}%
\pgfpathlineto{\pgfqpoint{3.680391in}{2.874842in}}%
\pgfpathlineto{\pgfqpoint{3.680391in}{2.879100in}}%
\pgfpathlineto{\pgfqpoint{3.684649in}{2.879100in}}%
\pgfpathlineto{\pgfqpoint{3.684649in}{2.874842in}}%
\pgfpathmoveto{\pgfqpoint{3.680391in}{2.879100in}}%
\pgfpathlineto{\pgfqpoint{3.680391in}{2.879100in}}%
\pgfpathlineto{\pgfqpoint{3.680391in}{2.883358in}}%
\pgfpathlineto{\pgfqpoint{3.684649in}{2.883358in}}%
\pgfpathlineto{\pgfqpoint{3.684649in}{2.879100in}}%
\pgfpathmoveto{\pgfqpoint{3.684649in}{2.874842in}}%
\pgfpathlineto{\pgfqpoint{3.684649in}{2.874842in}}%
\pgfpathlineto{\pgfqpoint{3.684649in}{2.879100in}}%
\pgfpathlineto{\pgfqpoint{3.688907in}{2.879100in}}%
\pgfpathlineto{\pgfqpoint{3.688907in}{2.874842in}}%
\pgfpathmoveto{\pgfqpoint{3.684649in}{2.879100in}}%
\pgfpathlineto{\pgfqpoint{3.684649in}{2.879100in}}%
\pgfpathlineto{\pgfqpoint{3.684649in}{2.883358in}}%
\pgfpathlineto{\pgfqpoint{3.688907in}{2.883358in}}%
\pgfpathlineto{\pgfqpoint{3.688907in}{2.879100in}}%
\pgfpathmoveto{\pgfqpoint{3.680391in}{2.883358in}}%
\pgfpathlineto{\pgfqpoint{3.680391in}{2.883358in}}%
\pgfpathlineto{\pgfqpoint{3.680391in}{2.887615in}}%
\pgfpathlineto{\pgfqpoint{3.684649in}{2.887615in}}%
\pgfpathlineto{\pgfqpoint{3.684649in}{2.883358in}}%
\pgfpathmoveto{\pgfqpoint{3.680391in}{2.887615in}}%
\pgfpathlineto{\pgfqpoint{3.680391in}{2.887615in}}%
\pgfpathlineto{\pgfqpoint{3.680391in}{2.891873in}}%
\pgfpathlineto{\pgfqpoint{3.684649in}{2.891873in}}%
\pgfpathlineto{\pgfqpoint{3.684649in}{2.887615in}}%
\pgfpathmoveto{\pgfqpoint{3.684649in}{2.883358in}}%
\pgfpathlineto{\pgfqpoint{3.684649in}{2.883358in}}%
\pgfpathlineto{\pgfqpoint{3.684649in}{2.887615in}}%
\pgfpathlineto{\pgfqpoint{3.688907in}{2.887615in}}%
\pgfpathlineto{\pgfqpoint{3.688907in}{2.883358in}}%
\pgfpathmoveto{\pgfqpoint{3.684649in}{2.887615in}}%
\pgfpathlineto{\pgfqpoint{3.684649in}{2.887615in}}%
\pgfpathlineto{\pgfqpoint{3.684649in}{2.891873in}}%
\pgfpathlineto{\pgfqpoint{3.688907in}{2.891873in}}%
\pgfpathlineto{\pgfqpoint{3.688907in}{2.887615in}}%
\pgfpathmoveto{\pgfqpoint{3.688907in}{2.857811in}}%
\pgfpathlineto{\pgfqpoint{3.688907in}{2.857811in}}%
\pgfpathlineto{\pgfqpoint{3.688907in}{2.862069in}}%
\pgfpathlineto{\pgfqpoint{3.693164in}{2.862069in}}%
\pgfpathlineto{\pgfqpoint{3.693164in}{2.857811in}}%
\pgfpathmoveto{\pgfqpoint{3.688907in}{2.862069in}}%
\pgfpathlineto{\pgfqpoint{3.688907in}{2.862069in}}%
\pgfpathlineto{\pgfqpoint{3.688907in}{2.866326in}}%
\pgfpathlineto{\pgfqpoint{3.693164in}{2.866326in}}%
\pgfpathlineto{\pgfqpoint{3.693164in}{2.862069in}}%
\pgfpathmoveto{\pgfqpoint{3.688907in}{2.866326in}}%
\pgfpathlineto{\pgfqpoint{3.688907in}{2.866326in}}%
\pgfpathlineto{\pgfqpoint{3.688907in}{2.870584in}}%
\pgfpathlineto{\pgfqpoint{3.693164in}{2.870584in}}%
\pgfpathlineto{\pgfqpoint{3.693164in}{2.866326in}}%
\pgfpathmoveto{\pgfqpoint{3.688907in}{2.870584in}}%
\pgfpathlineto{\pgfqpoint{3.688907in}{2.870584in}}%
\pgfpathlineto{\pgfqpoint{3.688907in}{2.874842in}}%
\pgfpathlineto{\pgfqpoint{3.693164in}{2.874842in}}%
\pgfpathlineto{\pgfqpoint{3.693164in}{2.870584in}}%
\pgfpathmoveto{\pgfqpoint{3.688907in}{2.874842in}}%
\pgfpathlineto{\pgfqpoint{3.688907in}{2.874842in}}%
\pgfpathlineto{\pgfqpoint{3.688907in}{2.879100in}}%
\pgfpathlineto{\pgfqpoint{3.693164in}{2.879100in}}%
\pgfpathlineto{\pgfqpoint{3.693164in}{2.874842in}}%
\pgfpathmoveto{\pgfqpoint{3.667618in}{2.913162in}}%
\pgfpathlineto{\pgfqpoint{3.667618in}{2.913162in}}%
\pgfpathlineto{\pgfqpoint{3.667618in}{2.917420in}}%
\pgfpathlineto{\pgfqpoint{3.671876in}{2.917420in}}%
\pgfpathlineto{\pgfqpoint{3.671876in}{2.913162in}}%
\pgfpathmoveto{\pgfqpoint{3.667618in}{2.917420in}}%
\pgfpathlineto{\pgfqpoint{3.667618in}{2.917420in}}%
\pgfpathlineto{\pgfqpoint{3.667618in}{2.921678in}}%
\pgfpathlineto{\pgfqpoint{3.671876in}{2.921678in}}%
\pgfpathlineto{\pgfqpoint{3.671876in}{2.917420in}}%
\pgfpathmoveto{\pgfqpoint{3.667618in}{2.921678in}}%
\pgfpathlineto{\pgfqpoint{3.667618in}{2.921678in}}%
\pgfpathlineto{\pgfqpoint{3.667618in}{2.925936in}}%
\pgfpathlineto{\pgfqpoint{3.671876in}{2.925936in}}%
\pgfpathlineto{\pgfqpoint{3.671876in}{2.921678in}}%
\pgfpathmoveto{\pgfqpoint{3.663360in}{2.930193in}}%
\pgfpathlineto{\pgfqpoint{3.663360in}{2.930193in}}%
\pgfpathlineto{\pgfqpoint{3.663360in}{2.934451in}}%
\pgfpathlineto{\pgfqpoint{3.667618in}{2.934451in}}%
\pgfpathlineto{\pgfqpoint{3.667618in}{2.930193in}}%
\pgfpathmoveto{\pgfqpoint{3.667618in}{2.925936in}}%
\pgfpathlineto{\pgfqpoint{3.667618in}{2.925936in}}%
\pgfpathlineto{\pgfqpoint{3.667618in}{2.930193in}}%
\pgfpathlineto{\pgfqpoint{3.671876in}{2.930193in}}%
\pgfpathlineto{\pgfqpoint{3.671876in}{2.925936in}}%
\pgfpathmoveto{\pgfqpoint{3.667618in}{2.930193in}}%
\pgfpathlineto{\pgfqpoint{3.667618in}{2.930193in}}%
\pgfpathlineto{\pgfqpoint{3.667618in}{2.934451in}}%
\pgfpathlineto{\pgfqpoint{3.671876in}{2.934451in}}%
\pgfpathlineto{\pgfqpoint{3.671876in}{2.930193in}}%
\pgfpathmoveto{\pgfqpoint{3.663360in}{2.934451in}}%
\pgfpathlineto{\pgfqpoint{3.663360in}{2.934451in}}%
\pgfpathlineto{\pgfqpoint{3.663360in}{2.938709in}}%
\pgfpathlineto{\pgfqpoint{3.667618in}{2.938709in}}%
\pgfpathlineto{\pgfqpoint{3.667618in}{2.934451in}}%
\pgfpathmoveto{\pgfqpoint{3.663360in}{2.938709in}}%
\pgfpathlineto{\pgfqpoint{3.663360in}{2.938709in}}%
\pgfpathlineto{\pgfqpoint{3.663360in}{2.942967in}}%
\pgfpathlineto{\pgfqpoint{3.667618in}{2.942967in}}%
\pgfpathlineto{\pgfqpoint{3.667618in}{2.938709in}}%
\pgfpathmoveto{\pgfqpoint{3.667618in}{2.934451in}}%
\pgfpathlineto{\pgfqpoint{3.667618in}{2.934451in}}%
\pgfpathlineto{\pgfqpoint{3.667618in}{2.938709in}}%
\pgfpathlineto{\pgfqpoint{3.671876in}{2.938709in}}%
\pgfpathlineto{\pgfqpoint{3.671876in}{2.934451in}}%
\pgfpathmoveto{\pgfqpoint{3.667618in}{2.938709in}}%
\pgfpathlineto{\pgfqpoint{3.667618in}{2.938709in}}%
\pgfpathlineto{\pgfqpoint{3.667618in}{2.942967in}}%
\pgfpathlineto{\pgfqpoint{3.671876in}{2.942967in}}%
\pgfpathlineto{\pgfqpoint{3.671876in}{2.938709in}}%
\pgfpathmoveto{\pgfqpoint{3.659103in}{2.947225in}}%
\pgfpathlineto{\pgfqpoint{3.659103in}{2.947225in}}%
\pgfpathlineto{\pgfqpoint{3.659103in}{2.951482in}}%
\pgfpathlineto{\pgfqpoint{3.663360in}{2.951482in}}%
\pgfpathlineto{\pgfqpoint{3.663360in}{2.947225in}}%
\pgfpathmoveto{\pgfqpoint{3.659103in}{2.951482in}}%
\pgfpathlineto{\pgfqpoint{3.659103in}{2.951482in}}%
\pgfpathlineto{\pgfqpoint{3.659103in}{2.955740in}}%
\pgfpathlineto{\pgfqpoint{3.663360in}{2.955740in}}%
\pgfpathlineto{\pgfqpoint{3.663360in}{2.951482in}}%
\pgfpathmoveto{\pgfqpoint{3.659103in}{2.955740in}}%
\pgfpathlineto{\pgfqpoint{3.659103in}{2.955740in}}%
\pgfpathlineto{\pgfqpoint{3.659103in}{2.959998in}}%
\pgfpathlineto{\pgfqpoint{3.663360in}{2.959998in}}%
\pgfpathlineto{\pgfqpoint{3.663360in}{2.955740in}}%
\pgfpathmoveto{\pgfqpoint{3.663360in}{2.942967in}}%
\pgfpathlineto{\pgfqpoint{3.663360in}{2.942967in}}%
\pgfpathlineto{\pgfqpoint{3.663360in}{2.947225in}}%
\pgfpathlineto{\pgfqpoint{3.667618in}{2.947225in}}%
\pgfpathlineto{\pgfqpoint{3.667618in}{2.942967in}}%
\pgfpathmoveto{\pgfqpoint{3.663360in}{2.947225in}}%
\pgfpathlineto{\pgfqpoint{3.663360in}{2.947225in}}%
\pgfpathlineto{\pgfqpoint{3.663360in}{2.951482in}}%
\pgfpathlineto{\pgfqpoint{3.667618in}{2.951482in}}%
\pgfpathlineto{\pgfqpoint{3.667618in}{2.947225in}}%
\pgfpathmoveto{\pgfqpoint{3.667618in}{2.942967in}}%
\pgfpathlineto{\pgfqpoint{3.667618in}{2.942967in}}%
\pgfpathlineto{\pgfqpoint{3.667618in}{2.947225in}}%
\pgfpathlineto{\pgfqpoint{3.671876in}{2.947225in}}%
\pgfpathlineto{\pgfqpoint{3.671876in}{2.942967in}}%
\pgfpathmoveto{\pgfqpoint{3.667618in}{2.947225in}}%
\pgfpathlineto{\pgfqpoint{3.667618in}{2.947225in}}%
\pgfpathlineto{\pgfqpoint{3.667618in}{2.951482in}}%
\pgfpathlineto{\pgfqpoint{3.671876in}{2.951482in}}%
\pgfpathlineto{\pgfqpoint{3.671876in}{2.947225in}}%
\pgfpathmoveto{\pgfqpoint{3.663360in}{2.951482in}}%
\pgfpathlineto{\pgfqpoint{3.663360in}{2.951482in}}%
\pgfpathlineto{\pgfqpoint{3.663360in}{2.955740in}}%
\pgfpathlineto{\pgfqpoint{3.667618in}{2.955740in}}%
\pgfpathlineto{\pgfqpoint{3.667618in}{2.951482in}}%
\pgfpathmoveto{\pgfqpoint{3.663360in}{2.955740in}}%
\pgfpathlineto{\pgfqpoint{3.663360in}{2.955740in}}%
\pgfpathlineto{\pgfqpoint{3.663360in}{2.959998in}}%
\pgfpathlineto{\pgfqpoint{3.667618in}{2.959998in}}%
\pgfpathlineto{\pgfqpoint{3.667618in}{2.955740in}}%
\pgfpathmoveto{\pgfqpoint{3.667618in}{2.951482in}}%
\pgfpathlineto{\pgfqpoint{3.667618in}{2.951482in}}%
\pgfpathlineto{\pgfqpoint{3.667618in}{2.955740in}}%
\pgfpathlineto{\pgfqpoint{3.671876in}{2.955740in}}%
\pgfpathlineto{\pgfqpoint{3.671876in}{2.951482in}}%
\pgfpathmoveto{\pgfqpoint{3.667618in}{2.955740in}}%
\pgfpathlineto{\pgfqpoint{3.667618in}{2.955740in}}%
\pgfpathlineto{\pgfqpoint{3.667618in}{2.959998in}}%
\pgfpathlineto{\pgfqpoint{3.671876in}{2.959998in}}%
\pgfpathlineto{\pgfqpoint{3.671876in}{2.955740in}}%
\pgfpathmoveto{\pgfqpoint{3.671876in}{2.896131in}}%
\pgfpathlineto{\pgfqpoint{3.671876in}{2.896131in}}%
\pgfpathlineto{\pgfqpoint{3.671876in}{2.900389in}}%
\pgfpathlineto{\pgfqpoint{3.676133in}{2.900389in}}%
\pgfpathlineto{\pgfqpoint{3.676133in}{2.896131in}}%
\pgfpathmoveto{\pgfqpoint{3.676133in}{2.891873in}}%
\pgfpathlineto{\pgfqpoint{3.676133in}{2.891873in}}%
\pgfpathlineto{\pgfqpoint{3.676133in}{2.896131in}}%
\pgfpathlineto{\pgfqpoint{3.680391in}{2.896131in}}%
\pgfpathlineto{\pgfqpoint{3.680391in}{2.891873in}}%
\pgfpathmoveto{\pgfqpoint{3.676133in}{2.896131in}}%
\pgfpathlineto{\pgfqpoint{3.676133in}{2.896131in}}%
\pgfpathlineto{\pgfqpoint{3.676133in}{2.900389in}}%
\pgfpathlineto{\pgfqpoint{3.680391in}{2.900389in}}%
\pgfpathlineto{\pgfqpoint{3.680391in}{2.896131in}}%
\pgfpathmoveto{\pgfqpoint{3.671876in}{2.900389in}}%
\pgfpathlineto{\pgfqpoint{3.671876in}{2.900389in}}%
\pgfpathlineto{\pgfqpoint{3.671876in}{2.904647in}}%
\pgfpathlineto{\pgfqpoint{3.676133in}{2.904647in}}%
\pgfpathlineto{\pgfqpoint{3.676133in}{2.900389in}}%
\pgfpathmoveto{\pgfqpoint{3.671876in}{2.904647in}}%
\pgfpathlineto{\pgfqpoint{3.671876in}{2.904647in}}%
\pgfpathlineto{\pgfqpoint{3.671876in}{2.908904in}}%
\pgfpathlineto{\pgfqpoint{3.676133in}{2.908904in}}%
\pgfpathlineto{\pgfqpoint{3.676133in}{2.904647in}}%
\pgfpathmoveto{\pgfqpoint{3.676133in}{2.900389in}}%
\pgfpathlineto{\pgfqpoint{3.676133in}{2.900389in}}%
\pgfpathlineto{\pgfqpoint{3.676133in}{2.904647in}}%
\pgfpathlineto{\pgfqpoint{3.680391in}{2.904647in}}%
\pgfpathlineto{\pgfqpoint{3.680391in}{2.900389in}}%
\pgfpathmoveto{\pgfqpoint{3.676133in}{2.904647in}}%
\pgfpathlineto{\pgfqpoint{3.676133in}{2.904647in}}%
\pgfpathlineto{\pgfqpoint{3.676133in}{2.908904in}}%
\pgfpathlineto{\pgfqpoint{3.680391in}{2.908904in}}%
\pgfpathlineto{\pgfqpoint{3.680391in}{2.904647in}}%
\pgfpathmoveto{\pgfqpoint{3.680391in}{2.891873in}}%
\pgfpathlineto{\pgfqpoint{3.680391in}{2.891873in}}%
\pgfpathlineto{\pgfqpoint{3.680391in}{2.896131in}}%
\pgfpathlineto{\pgfqpoint{3.684649in}{2.896131in}}%
\pgfpathlineto{\pgfqpoint{3.684649in}{2.891873in}}%
\pgfpathmoveto{\pgfqpoint{3.680391in}{2.896131in}}%
\pgfpathlineto{\pgfqpoint{3.680391in}{2.896131in}}%
\pgfpathlineto{\pgfqpoint{3.680391in}{2.900389in}}%
\pgfpathlineto{\pgfqpoint{3.684649in}{2.900389in}}%
\pgfpathlineto{\pgfqpoint{3.684649in}{2.896131in}}%
\pgfpathmoveto{\pgfqpoint{3.684649in}{2.891873in}}%
\pgfpathlineto{\pgfqpoint{3.684649in}{2.891873in}}%
\pgfpathlineto{\pgfqpoint{3.684649in}{2.896131in}}%
\pgfpathlineto{\pgfqpoint{3.688907in}{2.896131in}}%
\pgfpathlineto{\pgfqpoint{3.688907in}{2.891873in}}%
\pgfpathmoveto{\pgfqpoint{3.680391in}{2.900389in}}%
\pgfpathlineto{\pgfqpoint{3.680391in}{2.900389in}}%
\pgfpathlineto{\pgfqpoint{3.680391in}{2.904647in}}%
\pgfpathlineto{\pgfqpoint{3.684649in}{2.904647in}}%
\pgfpathlineto{\pgfqpoint{3.684649in}{2.900389in}}%
\pgfpathmoveto{\pgfqpoint{3.680391in}{2.904647in}}%
\pgfpathlineto{\pgfqpoint{3.680391in}{2.904647in}}%
\pgfpathlineto{\pgfqpoint{3.680391in}{2.908904in}}%
\pgfpathlineto{\pgfqpoint{3.684649in}{2.908904in}}%
\pgfpathlineto{\pgfqpoint{3.684649in}{2.904647in}}%
\pgfpathmoveto{\pgfqpoint{3.671876in}{2.908904in}}%
\pgfpathlineto{\pgfqpoint{3.671876in}{2.908904in}}%
\pgfpathlineto{\pgfqpoint{3.671876in}{2.913162in}}%
\pgfpathlineto{\pgfqpoint{3.676133in}{2.913162in}}%
\pgfpathlineto{\pgfqpoint{3.676133in}{2.908904in}}%
\pgfpathmoveto{\pgfqpoint{3.671876in}{2.913162in}}%
\pgfpathlineto{\pgfqpoint{3.671876in}{2.913162in}}%
\pgfpathlineto{\pgfqpoint{3.671876in}{2.917420in}}%
\pgfpathlineto{\pgfqpoint{3.676133in}{2.917420in}}%
\pgfpathlineto{\pgfqpoint{3.676133in}{2.913162in}}%
\pgfpathmoveto{\pgfqpoint{3.676133in}{2.908904in}}%
\pgfpathlineto{\pgfqpoint{3.676133in}{2.908904in}}%
\pgfpathlineto{\pgfqpoint{3.676133in}{2.913162in}}%
\pgfpathlineto{\pgfqpoint{3.680391in}{2.913162in}}%
\pgfpathlineto{\pgfqpoint{3.680391in}{2.908904in}}%
\pgfpathmoveto{\pgfqpoint{3.676133in}{2.913162in}}%
\pgfpathlineto{\pgfqpoint{3.676133in}{2.913162in}}%
\pgfpathlineto{\pgfqpoint{3.676133in}{2.917420in}}%
\pgfpathlineto{\pgfqpoint{3.680391in}{2.917420in}}%
\pgfpathlineto{\pgfqpoint{3.680391in}{2.913162in}}%
\pgfpathmoveto{\pgfqpoint{3.671876in}{2.917420in}}%
\pgfpathlineto{\pgfqpoint{3.671876in}{2.917420in}}%
\pgfpathlineto{\pgfqpoint{3.671876in}{2.921678in}}%
\pgfpathlineto{\pgfqpoint{3.676133in}{2.921678in}}%
\pgfpathlineto{\pgfqpoint{3.676133in}{2.917420in}}%
\pgfpathmoveto{\pgfqpoint{3.671876in}{2.921678in}}%
\pgfpathlineto{\pgfqpoint{3.671876in}{2.921678in}}%
\pgfpathlineto{\pgfqpoint{3.671876in}{2.925936in}}%
\pgfpathlineto{\pgfqpoint{3.676133in}{2.925936in}}%
\pgfpathlineto{\pgfqpoint{3.676133in}{2.921678in}}%
\pgfpathmoveto{\pgfqpoint{3.676133in}{2.917420in}}%
\pgfpathlineto{\pgfqpoint{3.676133in}{2.917420in}}%
\pgfpathlineto{\pgfqpoint{3.676133in}{2.921678in}}%
\pgfpathlineto{\pgfqpoint{3.680391in}{2.921678in}}%
\pgfpathlineto{\pgfqpoint{3.680391in}{2.917420in}}%
\pgfpathmoveto{\pgfqpoint{3.676133in}{2.921678in}}%
\pgfpathlineto{\pgfqpoint{3.676133in}{2.921678in}}%
\pgfpathlineto{\pgfqpoint{3.676133in}{2.925936in}}%
\pgfpathlineto{\pgfqpoint{3.680391in}{2.925936in}}%
\pgfpathlineto{\pgfqpoint{3.680391in}{2.921678in}}%
\pgfpathmoveto{\pgfqpoint{3.680391in}{2.908904in}}%
\pgfpathlineto{\pgfqpoint{3.680391in}{2.908904in}}%
\pgfpathlineto{\pgfqpoint{3.680391in}{2.913162in}}%
\pgfpathlineto{\pgfqpoint{3.684649in}{2.913162in}}%
\pgfpathlineto{\pgfqpoint{3.684649in}{2.908904in}}%
\pgfpathmoveto{\pgfqpoint{3.680391in}{2.913162in}}%
\pgfpathlineto{\pgfqpoint{3.680391in}{2.913162in}}%
\pgfpathlineto{\pgfqpoint{3.680391in}{2.917420in}}%
\pgfpathlineto{\pgfqpoint{3.684649in}{2.917420in}}%
\pgfpathlineto{\pgfqpoint{3.684649in}{2.913162in}}%
\pgfpathmoveto{\pgfqpoint{3.671876in}{2.925936in}}%
\pgfpathlineto{\pgfqpoint{3.671876in}{2.925936in}}%
\pgfpathlineto{\pgfqpoint{3.671876in}{2.930193in}}%
\pgfpathlineto{\pgfqpoint{3.676133in}{2.930193in}}%
\pgfpathlineto{\pgfqpoint{3.676133in}{2.925936in}}%
\pgfpathmoveto{\pgfqpoint{3.671876in}{2.930193in}}%
\pgfpathlineto{\pgfqpoint{3.671876in}{2.930193in}}%
\pgfpathlineto{\pgfqpoint{3.671876in}{2.934451in}}%
\pgfpathlineto{\pgfqpoint{3.676133in}{2.934451in}}%
\pgfpathlineto{\pgfqpoint{3.676133in}{2.930193in}}%
\pgfpathmoveto{\pgfqpoint{3.676133in}{2.925936in}}%
\pgfpathlineto{\pgfqpoint{3.676133in}{2.925936in}}%
\pgfpathlineto{\pgfqpoint{3.676133in}{2.930193in}}%
\pgfpathlineto{\pgfqpoint{3.680391in}{2.930193in}}%
\pgfpathlineto{\pgfqpoint{3.680391in}{2.925936in}}%
\pgfpathmoveto{\pgfqpoint{3.676133in}{2.930193in}}%
\pgfpathlineto{\pgfqpoint{3.676133in}{2.930193in}}%
\pgfpathlineto{\pgfqpoint{3.676133in}{2.934451in}}%
\pgfpathlineto{\pgfqpoint{3.680391in}{2.934451in}}%
\pgfpathlineto{\pgfqpoint{3.680391in}{2.930193in}}%
\pgfpathmoveto{\pgfqpoint{3.671876in}{2.934451in}}%
\pgfpathlineto{\pgfqpoint{3.671876in}{2.934451in}}%
\pgfpathlineto{\pgfqpoint{3.671876in}{2.938709in}}%
\pgfpathlineto{\pgfqpoint{3.676133in}{2.938709in}}%
\pgfpathlineto{\pgfqpoint{3.676133in}{2.934451in}}%
\pgfpathmoveto{\pgfqpoint{3.671876in}{2.938709in}}%
\pgfpathlineto{\pgfqpoint{3.671876in}{2.938709in}}%
\pgfpathlineto{\pgfqpoint{3.671876in}{2.942967in}}%
\pgfpathlineto{\pgfqpoint{3.676133in}{2.942967in}}%
\pgfpathlineto{\pgfqpoint{3.676133in}{2.938709in}}%
\pgfpathmoveto{\pgfqpoint{3.671876in}{2.942967in}}%
\pgfpathlineto{\pgfqpoint{3.671876in}{2.942967in}}%
\pgfpathlineto{\pgfqpoint{3.671876in}{2.947225in}}%
\pgfpathlineto{\pgfqpoint{3.676133in}{2.947225in}}%
\pgfpathlineto{\pgfqpoint{3.676133in}{2.942967in}}%
\pgfpathmoveto{\pgfqpoint{3.671876in}{2.947225in}}%
\pgfpathlineto{\pgfqpoint{3.671876in}{2.947225in}}%
\pgfpathlineto{\pgfqpoint{3.671876in}{2.951482in}}%
\pgfpathlineto{\pgfqpoint{3.676133in}{2.951482in}}%
\pgfpathlineto{\pgfqpoint{3.676133in}{2.947225in}}%
\pgfpathmoveto{\pgfqpoint{3.650587in}{2.981288in}}%
\pgfpathlineto{\pgfqpoint{3.650587in}{2.981288in}}%
\pgfpathlineto{\pgfqpoint{3.650587in}{2.985546in}}%
\pgfpathlineto{\pgfqpoint{3.654845in}{2.985546in}}%
\pgfpathlineto{\pgfqpoint{3.654845in}{2.981288in}}%
\pgfpathmoveto{\pgfqpoint{3.650587in}{2.985546in}}%
\pgfpathlineto{\pgfqpoint{3.650587in}{2.985546in}}%
\pgfpathlineto{\pgfqpoint{3.650587in}{2.989804in}}%
\pgfpathlineto{\pgfqpoint{3.654845in}{2.989804in}}%
\pgfpathlineto{\pgfqpoint{3.654845in}{2.985546in}}%
\pgfpathmoveto{\pgfqpoint{3.650587in}{2.989804in}}%
\pgfpathlineto{\pgfqpoint{3.650587in}{2.989804in}}%
\pgfpathlineto{\pgfqpoint{3.650587in}{2.994062in}}%
\pgfpathlineto{\pgfqpoint{3.654845in}{2.994062in}}%
\pgfpathlineto{\pgfqpoint{3.654845in}{2.989804in}}%
\pgfpathmoveto{\pgfqpoint{3.654845in}{2.964256in}}%
\pgfpathlineto{\pgfqpoint{3.654845in}{2.964256in}}%
\pgfpathlineto{\pgfqpoint{3.654845in}{2.968514in}}%
\pgfpathlineto{\pgfqpoint{3.659103in}{2.968514in}}%
\pgfpathlineto{\pgfqpoint{3.659103in}{2.964256in}}%
\pgfpathmoveto{\pgfqpoint{3.659103in}{2.959998in}}%
\pgfpathlineto{\pgfqpoint{3.659103in}{2.959998in}}%
\pgfpathlineto{\pgfqpoint{3.659103in}{2.964256in}}%
\pgfpathlineto{\pgfqpoint{3.663360in}{2.964256in}}%
\pgfpathlineto{\pgfqpoint{3.663360in}{2.959998in}}%
\pgfpathmoveto{\pgfqpoint{3.659103in}{2.964256in}}%
\pgfpathlineto{\pgfqpoint{3.659103in}{2.964256in}}%
\pgfpathlineto{\pgfqpoint{3.659103in}{2.968514in}}%
\pgfpathlineto{\pgfqpoint{3.663360in}{2.968514in}}%
\pgfpathlineto{\pgfqpoint{3.663360in}{2.964256in}}%
\pgfpathmoveto{\pgfqpoint{3.654845in}{2.968514in}}%
\pgfpathlineto{\pgfqpoint{3.654845in}{2.968514in}}%
\pgfpathlineto{\pgfqpoint{3.654845in}{2.972772in}}%
\pgfpathlineto{\pgfqpoint{3.659103in}{2.972772in}}%
\pgfpathlineto{\pgfqpoint{3.659103in}{2.968514in}}%
\pgfpathmoveto{\pgfqpoint{3.654845in}{2.972772in}}%
\pgfpathlineto{\pgfqpoint{3.654845in}{2.972772in}}%
\pgfpathlineto{\pgfqpoint{3.654845in}{2.977030in}}%
\pgfpathlineto{\pgfqpoint{3.659103in}{2.977030in}}%
\pgfpathlineto{\pgfqpoint{3.659103in}{2.972772in}}%
\pgfpathmoveto{\pgfqpoint{3.659103in}{2.968514in}}%
\pgfpathlineto{\pgfqpoint{3.659103in}{2.968514in}}%
\pgfpathlineto{\pgfqpoint{3.659103in}{2.972772in}}%
\pgfpathlineto{\pgfqpoint{3.663360in}{2.972772in}}%
\pgfpathlineto{\pgfqpoint{3.663360in}{2.968514in}}%
\pgfpathmoveto{\pgfqpoint{3.659103in}{2.972772in}}%
\pgfpathlineto{\pgfqpoint{3.659103in}{2.972772in}}%
\pgfpathlineto{\pgfqpoint{3.659103in}{2.977030in}}%
\pgfpathlineto{\pgfqpoint{3.663360in}{2.977030in}}%
\pgfpathlineto{\pgfqpoint{3.663360in}{2.972772in}}%
\pgfpathmoveto{\pgfqpoint{3.663360in}{2.959998in}}%
\pgfpathlineto{\pgfqpoint{3.663360in}{2.959998in}}%
\pgfpathlineto{\pgfqpoint{3.663360in}{2.964256in}}%
\pgfpathlineto{\pgfqpoint{3.667618in}{2.964256in}}%
\pgfpathlineto{\pgfqpoint{3.667618in}{2.959998in}}%
\pgfpathmoveto{\pgfqpoint{3.663360in}{2.964256in}}%
\pgfpathlineto{\pgfqpoint{3.663360in}{2.964256in}}%
\pgfpathlineto{\pgfqpoint{3.663360in}{2.968514in}}%
\pgfpathlineto{\pgfqpoint{3.667618in}{2.968514in}}%
\pgfpathlineto{\pgfqpoint{3.667618in}{2.964256in}}%
\pgfpathmoveto{\pgfqpoint{3.667618in}{2.959998in}}%
\pgfpathlineto{\pgfqpoint{3.667618in}{2.959998in}}%
\pgfpathlineto{\pgfqpoint{3.667618in}{2.964256in}}%
\pgfpathlineto{\pgfqpoint{3.671876in}{2.964256in}}%
\pgfpathlineto{\pgfqpoint{3.671876in}{2.959998in}}%
\pgfpathmoveto{\pgfqpoint{3.667618in}{2.964256in}}%
\pgfpathlineto{\pgfqpoint{3.667618in}{2.964256in}}%
\pgfpathlineto{\pgfqpoint{3.667618in}{2.968514in}}%
\pgfpathlineto{\pgfqpoint{3.671876in}{2.968514in}}%
\pgfpathlineto{\pgfqpoint{3.671876in}{2.964256in}}%
\pgfpathmoveto{\pgfqpoint{3.663360in}{2.968514in}}%
\pgfpathlineto{\pgfqpoint{3.663360in}{2.968514in}}%
\pgfpathlineto{\pgfqpoint{3.663360in}{2.972772in}}%
\pgfpathlineto{\pgfqpoint{3.667618in}{2.972772in}}%
\pgfpathlineto{\pgfqpoint{3.667618in}{2.968514in}}%
\pgfpathmoveto{\pgfqpoint{3.663360in}{2.972772in}}%
\pgfpathlineto{\pgfqpoint{3.663360in}{2.972772in}}%
\pgfpathlineto{\pgfqpoint{3.663360in}{2.977030in}}%
\pgfpathlineto{\pgfqpoint{3.667618in}{2.977030in}}%
\pgfpathlineto{\pgfqpoint{3.667618in}{2.972772in}}%
\pgfpathmoveto{\pgfqpoint{3.654845in}{2.977030in}}%
\pgfpathlineto{\pgfqpoint{3.654845in}{2.977030in}}%
\pgfpathlineto{\pgfqpoint{3.654845in}{2.981288in}}%
\pgfpathlineto{\pgfqpoint{3.659103in}{2.981288in}}%
\pgfpathlineto{\pgfqpoint{3.659103in}{2.977030in}}%
\pgfpathmoveto{\pgfqpoint{3.654845in}{2.981288in}}%
\pgfpathlineto{\pgfqpoint{3.654845in}{2.981288in}}%
\pgfpathlineto{\pgfqpoint{3.654845in}{2.985546in}}%
\pgfpathlineto{\pgfqpoint{3.659103in}{2.985546in}}%
\pgfpathlineto{\pgfqpoint{3.659103in}{2.981288in}}%
\pgfpathmoveto{\pgfqpoint{3.659103in}{2.977030in}}%
\pgfpathlineto{\pgfqpoint{3.659103in}{2.977030in}}%
\pgfpathlineto{\pgfqpoint{3.659103in}{2.981288in}}%
\pgfpathlineto{\pgfqpoint{3.663360in}{2.981288in}}%
\pgfpathlineto{\pgfqpoint{3.663360in}{2.977030in}}%
\pgfpathmoveto{\pgfqpoint{3.659103in}{2.981288in}}%
\pgfpathlineto{\pgfqpoint{3.659103in}{2.981288in}}%
\pgfpathlineto{\pgfqpoint{3.659103in}{2.985546in}}%
\pgfpathlineto{\pgfqpoint{3.663360in}{2.985546in}}%
\pgfpathlineto{\pgfqpoint{3.663360in}{2.981288in}}%
\pgfpathmoveto{\pgfqpoint{3.654845in}{2.985546in}}%
\pgfpathlineto{\pgfqpoint{3.654845in}{2.985546in}}%
\pgfpathlineto{\pgfqpoint{3.654845in}{2.989804in}}%
\pgfpathlineto{\pgfqpoint{3.659103in}{2.989804in}}%
\pgfpathlineto{\pgfqpoint{3.659103in}{2.985546in}}%
\pgfpathmoveto{\pgfqpoint{3.654845in}{2.989804in}}%
\pgfpathlineto{\pgfqpoint{3.654845in}{2.989804in}}%
\pgfpathlineto{\pgfqpoint{3.654845in}{2.994062in}}%
\pgfpathlineto{\pgfqpoint{3.659103in}{2.994062in}}%
\pgfpathlineto{\pgfqpoint{3.659103in}{2.989804in}}%
\pgfpathmoveto{\pgfqpoint{3.659103in}{2.985546in}}%
\pgfpathlineto{\pgfqpoint{3.659103in}{2.985546in}}%
\pgfpathlineto{\pgfqpoint{3.659103in}{2.989804in}}%
\pgfpathlineto{\pgfqpoint{3.663360in}{2.989804in}}%
\pgfpathlineto{\pgfqpoint{3.663360in}{2.985546in}}%
\pgfpathmoveto{\pgfqpoint{3.659103in}{2.989804in}}%
\pgfpathlineto{\pgfqpoint{3.659103in}{2.989804in}}%
\pgfpathlineto{\pgfqpoint{3.659103in}{2.994062in}}%
\pgfpathlineto{\pgfqpoint{3.663360in}{2.994062in}}%
\pgfpathlineto{\pgfqpoint{3.663360in}{2.989804in}}%
\pgfpathmoveto{\pgfqpoint{3.663360in}{2.977030in}}%
\pgfpathlineto{\pgfqpoint{3.663360in}{2.977030in}}%
\pgfpathlineto{\pgfqpoint{3.663360in}{2.981288in}}%
\pgfpathlineto{\pgfqpoint{3.667618in}{2.981288in}}%
\pgfpathlineto{\pgfqpoint{3.667618in}{2.977030in}}%
\pgfpathmoveto{\pgfqpoint{3.663360in}{2.981288in}}%
\pgfpathlineto{\pgfqpoint{3.663360in}{2.981288in}}%
\pgfpathlineto{\pgfqpoint{3.663360in}{2.985546in}}%
\pgfpathlineto{\pgfqpoint{3.667618in}{2.985546in}}%
\pgfpathlineto{\pgfqpoint{3.667618in}{2.981288in}}%
\pgfpathmoveto{\pgfqpoint{3.646329in}{2.998320in}}%
\pgfpathlineto{\pgfqpoint{3.646329in}{2.998320in}}%
\pgfpathlineto{\pgfqpoint{3.646329in}{3.002578in}}%
\pgfpathlineto{\pgfqpoint{3.650587in}{3.002578in}}%
\pgfpathlineto{\pgfqpoint{3.650587in}{2.998320in}}%
\pgfpathmoveto{\pgfqpoint{3.650587in}{2.994062in}}%
\pgfpathlineto{\pgfqpoint{3.650587in}{2.994062in}}%
\pgfpathlineto{\pgfqpoint{3.650587in}{2.998320in}}%
\pgfpathlineto{\pgfqpoint{3.654845in}{2.998320in}}%
\pgfpathlineto{\pgfqpoint{3.654845in}{2.994062in}}%
\pgfpathmoveto{\pgfqpoint{3.650587in}{2.998320in}}%
\pgfpathlineto{\pgfqpoint{3.650587in}{2.998320in}}%
\pgfpathlineto{\pgfqpoint{3.650587in}{3.002578in}}%
\pgfpathlineto{\pgfqpoint{3.654845in}{3.002578in}}%
\pgfpathlineto{\pgfqpoint{3.654845in}{2.998320in}}%
\pgfpathmoveto{\pgfqpoint{3.646329in}{3.002578in}}%
\pgfpathlineto{\pgfqpoint{3.646329in}{3.002578in}}%
\pgfpathlineto{\pgfqpoint{3.646329in}{3.006836in}}%
\pgfpathlineto{\pgfqpoint{3.650587in}{3.006836in}}%
\pgfpathlineto{\pgfqpoint{3.650587in}{3.002578in}}%
\pgfpathmoveto{\pgfqpoint{3.646329in}{3.006836in}}%
\pgfpathlineto{\pgfqpoint{3.646329in}{3.006836in}}%
\pgfpathlineto{\pgfqpoint{3.646329in}{3.011094in}}%
\pgfpathlineto{\pgfqpoint{3.650587in}{3.011094in}}%
\pgfpathlineto{\pgfqpoint{3.650587in}{3.006836in}}%
\pgfpathmoveto{\pgfqpoint{3.650587in}{3.002578in}}%
\pgfpathlineto{\pgfqpoint{3.650587in}{3.002578in}}%
\pgfpathlineto{\pgfqpoint{3.650587in}{3.006836in}}%
\pgfpathlineto{\pgfqpoint{3.654845in}{3.006836in}}%
\pgfpathlineto{\pgfqpoint{3.654845in}{3.002578in}}%
\pgfpathmoveto{\pgfqpoint{3.650587in}{3.006836in}}%
\pgfpathlineto{\pgfqpoint{3.650587in}{3.006836in}}%
\pgfpathlineto{\pgfqpoint{3.650587in}{3.011094in}}%
\pgfpathlineto{\pgfqpoint{3.654845in}{3.011094in}}%
\pgfpathlineto{\pgfqpoint{3.654845in}{3.006836in}}%
\pgfpathmoveto{\pgfqpoint{3.642072in}{3.011094in}}%
\pgfpathlineto{\pgfqpoint{3.642072in}{3.011094in}}%
\pgfpathlineto{\pgfqpoint{3.642072in}{3.015352in}}%
\pgfpathlineto{\pgfqpoint{3.646329in}{3.015352in}}%
\pgfpathlineto{\pgfqpoint{3.646329in}{3.011094in}}%
\pgfpathmoveto{\pgfqpoint{3.642072in}{3.015352in}}%
\pgfpathlineto{\pgfqpoint{3.642072in}{3.015352in}}%
\pgfpathlineto{\pgfqpoint{3.642072in}{3.019610in}}%
\pgfpathlineto{\pgfqpoint{3.646329in}{3.019610in}}%
\pgfpathlineto{\pgfqpoint{3.646329in}{3.015352in}}%
\pgfpathmoveto{\pgfqpoint{3.642072in}{3.019610in}}%
\pgfpathlineto{\pgfqpoint{3.642072in}{3.019610in}}%
\pgfpathlineto{\pgfqpoint{3.642072in}{3.023868in}}%
\pgfpathlineto{\pgfqpoint{3.646329in}{3.023868in}}%
\pgfpathlineto{\pgfqpoint{3.646329in}{3.019610in}}%
\pgfpathmoveto{\pgfqpoint{3.642072in}{3.023868in}}%
\pgfpathlineto{\pgfqpoint{3.642072in}{3.023868in}}%
\pgfpathlineto{\pgfqpoint{3.642072in}{3.028126in}}%
\pgfpathlineto{\pgfqpoint{3.646329in}{3.028126in}}%
\pgfpathlineto{\pgfqpoint{3.646329in}{3.023868in}}%
\pgfpathmoveto{\pgfqpoint{3.646329in}{3.011094in}}%
\pgfpathlineto{\pgfqpoint{3.646329in}{3.011094in}}%
\pgfpathlineto{\pgfqpoint{3.646329in}{3.015352in}}%
\pgfpathlineto{\pgfqpoint{3.650587in}{3.015352in}}%
\pgfpathlineto{\pgfqpoint{3.650587in}{3.011094in}}%
\pgfpathmoveto{\pgfqpoint{3.646329in}{3.015352in}}%
\pgfpathlineto{\pgfqpoint{3.646329in}{3.015352in}}%
\pgfpathlineto{\pgfqpoint{3.646329in}{3.019610in}}%
\pgfpathlineto{\pgfqpoint{3.650587in}{3.019610in}}%
\pgfpathlineto{\pgfqpoint{3.650587in}{3.015352in}}%
\pgfpathmoveto{\pgfqpoint{3.650587in}{3.011094in}}%
\pgfpathlineto{\pgfqpoint{3.650587in}{3.011094in}}%
\pgfpathlineto{\pgfqpoint{3.650587in}{3.015352in}}%
\pgfpathlineto{\pgfqpoint{3.654845in}{3.015352in}}%
\pgfpathlineto{\pgfqpoint{3.654845in}{3.011094in}}%
\pgfpathmoveto{\pgfqpoint{3.650587in}{3.015352in}}%
\pgfpathlineto{\pgfqpoint{3.650587in}{3.015352in}}%
\pgfpathlineto{\pgfqpoint{3.650587in}{3.019610in}}%
\pgfpathlineto{\pgfqpoint{3.654845in}{3.019610in}}%
\pgfpathlineto{\pgfqpoint{3.654845in}{3.015352in}}%
\pgfpathmoveto{\pgfqpoint{3.646329in}{3.019610in}}%
\pgfpathlineto{\pgfqpoint{3.646329in}{3.019610in}}%
\pgfpathlineto{\pgfqpoint{3.646329in}{3.023868in}}%
\pgfpathlineto{\pgfqpoint{3.650587in}{3.023868in}}%
\pgfpathlineto{\pgfqpoint{3.650587in}{3.019610in}}%
\pgfpathmoveto{\pgfqpoint{3.646329in}{3.023868in}}%
\pgfpathlineto{\pgfqpoint{3.646329in}{3.023868in}}%
\pgfpathlineto{\pgfqpoint{3.646329in}{3.028126in}}%
\pgfpathlineto{\pgfqpoint{3.650587in}{3.028126in}}%
\pgfpathlineto{\pgfqpoint{3.650587in}{3.023868in}}%
\pgfpathmoveto{\pgfqpoint{3.650587in}{3.019610in}}%
\pgfpathlineto{\pgfqpoint{3.650587in}{3.019610in}}%
\pgfpathlineto{\pgfqpoint{3.650587in}{3.023868in}}%
\pgfpathlineto{\pgfqpoint{3.654845in}{3.023868in}}%
\pgfpathlineto{\pgfqpoint{3.654845in}{3.019610in}}%
\pgfpathmoveto{\pgfqpoint{3.650587in}{3.023868in}}%
\pgfpathlineto{\pgfqpoint{3.650587in}{3.023868in}}%
\pgfpathlineto{\pgfqpoint{3.650587in}{3.028126in}}%
\pgfpathlineto{\pgfqpoint{3.654845in}{3.028126in}}%
\pgfpathlineto{\pgfqpoint{3.654845in}{3.023868in}}%
\pgfpathmoveto{\pgfqpoint{3.654845in}{2.994062in}}%
\pgfpathlineto{\pgfqpoint{3.654845in}{2.994062in}}%
\pgfpathlineto{\pgfqpoint{3.654845in}{2.998320in}}%
\pgfpathlineto{\pgfqpoint{3.659103in}{2.998320in}}%
\pgfpathlineto{\pgfqpoint{3.659103in}{2.994062in}}%
\pgfpathmoveto{\pgfqpoint{3.654845in}{2.998320in}}%
\pgfpathlineto{\pgfqpoint{3.654845in}{2.998320in}}%
\pgfpathlineto{\pgfqpoint{3.654845in}{3.002578in}}%
\pgfpathlineto{\pgfqpoint{3.659103in}{3.002578in}}%
\pgfpathlineto{\pgfqpoint{3.659103in}{2.998320in}}%
\pgfpathmoveto{\pgfqpoint{3.659103in}{2.994062in}}%
\pgfpathlineto{\pgfqpoint{3.659103in}{2.994062in}}%
\pgfpathlineto{\pgfqpoint{3.659103in}{2.998320in}}%
\pgfpathlineto{\pgfqpoint{3.663360in}{2.998320in}}%
\pgfpathlineto{\pgfqpoint{3.663360in}{2.994062in}}%
\pgfpathmoveto{\pgfqpoint{3.659103in}{2.998320in}}%
\pgfpathlineto{\pgfqpoint{3.659103in}{2.998320in}}%
\pgfpathlineto{\pgfqpoint{3.659103in}{3.002578in}}%
\pgfpathlineto{\pgfqpoint{3.663360in}{3.002578in}}%
\pgfpathlineto{\pgfqpoint{3.663360in}{2.998320in}}%
\pgfpathmoveto{\pgfqpoint{3.654845in}{3.002578in}}%
\pgfpathlineto{\pgfqpoint{3.654845in}{3.002578in}}%
\pgfpathlineto{\pgfqpoint{3.654845in}{3.006836in}}%
\pgfpathlineto{\pgfqpoint{3.659103in}{3.006836in}}%
\pgfpathlineto{\pgfqpoint{3.659103in}{3.002578in}}%
\pgfpathmoveto{\pgfqpoint{3.654845in}{3.006836in}}%
\pgfpathlineto{\pgfqpoint{3.654845in}{3.006836in}}%
\pgfpathlineto{\pgfqpoint{3.654845in}{3.011094in}}%
\pgfpathlineto{\pgfqpoint{3.659103in}{3.011094in}}%
\pgfpathlineto{\pgfqpoint{3.659103in}{3.006836in}}%
\pgfpathmoveto{\pgfqpoint{3.654845in}{3.011094in}}%
\pgfpathlineto{\pgfqpoint{3.654845in}{3.011094in}}%
\pgfpathlineto{\pgfqpoint{3.654845in}{3.015352in}}%
\pgfpathlineto{\pgfqpoint{3.659103in}{3.015352in}}%
\pgfpathlineto{\pgfqpoint{3.659103in}{3.011094in}}%
\pgfpathmoveto{\pgfqpoint{3.654845in}{3.015352in}}%
\pgfpathlineto{\pgfqpoint{3.654845in}{3.015352in}}%
\pgfpathlineto{\pgfqpoint{3.654845in}{3.019610in}}%
\pgfpathlineto{\pgfqpoint{3.659103in}{3.019610in}}%
\pgfpathlineto{\pgfqpoint{3.659103in}{3.015352in}}%
\pgfpathmoveto{\pgfqpoint{3.637814in}{3.028126in}}%
\pgfpathlineto{\pgfqpoint{3.637814in}{3.028126in}}%
\pgfpathlineto{\pgfqpoint{3.637814in}{3.032384in}}%
\pgfpathlineto{\pgfqpoint{3.642072in}{3.032384in}}%
\pgfpathlineto{\pgfqpoint{3.642072in}{3.028126in}}%
\pgfpathmoveto{\pgfqpoint{3.637814in}{3.032384in}}%
\pgfpathlineto{\pgfqpoint{3.637814in}{3.032384in}}%
\pgfpathlineto{\pgfqpoint{3.637814in}{3.036642in}}%
\pgfpathlineto{\pgfqpoint{3.642072in}{3.036642in}}%
\pgfpathlineto{\pgfqpoint{3.642072in}{3.032384in}}%
\pgfpathmoveto{\pgfqpoint{3.642072in}{3.028126in}}%
\pgfpathlineto{\pgfqpoint{3.642072in}{3.028126in}}%
\pgfpathlineto{\pgfqpoint{3.642072in}{3.032384in}}%
\pgfpathlineto{\pgfqpoint{3.646329in}{3.032384in}}%
\pgfpathlineto{\pgfqpoint{3.646329in}{3.028126in}}%
\pgfpathmoveto{\pgfqpoint{3.642072in}{3.032384in}}%
\pgfpathlineto{\pgfqpoint{3.642072in}{3.032384in}}%
\pgfpathlineto{\pgfqpoint{3.642072in}{3.036642in}}%
\pgfpathlineto{\pgfqpoint{3.646329in}{3.036642in}}%
\pgfpathlineto{\pgfqpoint{3.646329in}{3.032384in}}%
\pgfpathmoveto{\pgfqpoint{3.637814in}{3.036642in}}%
\pgfpathlineto{\pgfqpoint{3.637814in}{3.036642in}}%
\pgfpathlineto{\pgfqpoint{3.637814in}{3.040900in}}%
\pgfpathlineto{\pgfqpoint{3.642072in}{3.040900in}}%
\pgfpathlineto{\pgfqpoint{3.642072in}{3.036642in}}%
\pgfpathmoveto{\pgfqpoint{3.637814in}{3.040900in}}%
\pgfpathlineto{\pgfqpoint{3.637814in}{3.040900in}}%
\pgfpathlineto{\pgfqpoint{3.637814in}{3.045158in}}%
\pgfpathlineto{\pgfqpoint{3.642072in}{3.045158in}}%
\pgfpathlineto{\pgfqpoint{3.642072in}{3.040900in}}%
\pgfpathmoveto{\pgfqpoint{3.642072in}{3.036642in}}%
\pgfpathlineto{\pgfqpoint{3.642072in}{3.036642in}}%
\pgfpathlineto{\pgfqpoint{3.642072in}{3.040900in}}%
\pgfpathlineto{\pgfqpoint{3.646329in}{3.040900in}}%
\pgfpathlineto{\pgfqpoint{3.646329in}{3.036642in}}%
\pgfpathmoveto{\pgfqpoint{3.642072in}{3.040900in}}%
\pgfpathlineto{\pgfqpoint{3.642072in}{3.040900in}}%
\pgfpathlineto{\pgfqpoint{3.642072in}{3.045158in}}%
\pgfpathlineto{\pgfqpoint{3.646329in}{3.045158in}}%
\pgfpathlineto{\pgfqpoint{3.646329in}{3.040900in}}%
\pgfpathmoveto{\pgfqpoint{3.646329in}{3.028126in}}%
\pgfpathlineto{\pgfqpoint{3.646329in}{3.028126in}}%
\pgfpathlineto{\pgfqpoint{3.646329in}{3.032384in}}%
\pgfpathlineto{\pgfqpoint{3.650587in}{3.032384in}}%
\pgfpathlineto{\pgfqpoint{3.650587in}{3.028126in}}%
\pgfpathmoveto{\pgfqpoint{3.646329in}{3.032384in}}%
\pgfpathlineto{\pgfqpoint{3.646329in}{3.032384in}}%
\pgfpathlineto{\pgfqpoint{3.646329in}{3.036642in}}%
\pgfpathlineto{\pgfqpoint{3.650587in}{3.036642in}}%
\pgfpathlineto{\pgfqpoint{3.650587in}{3.032384in}}%
\pgfpathmoveto{\pgfqpoint{3.650587in}{3.028126in}}%
\pgfpathlineto{\pgfqpoint{3.650587in}{3.028126in}}%
\pgfpathlineto{\pgfqpoint{3.650587in}{3.032384in}}%
\pgfpathlineto{\pgfqpoint{3.654845in}{3.032384in}}%
\pgfpathlineto{\pgfqpoint{3.654845in}{3.028126in}}%
\pgfpathmoveto{\pgfqpoint{3.650587in}{3.032384in}}%
\pgfpathlineto{\pgfqpoint{3.650587in}{3.032384in}}%
\pgfpathlineto{\pgfqpoint{3.650587in}{3.036642in}}%
\pgfpathlineto{\pgfqpoint{3.654845in}{3.036642in}}%
\pgfpathlineto{\pgfqpoint{3.654845in}{3.032384in}}%
\pgfpathmoveto{\pgfqpoint{3.646329in}{3.036642in}}%
\pgfpathlineto{\pgfqpoint{3.646329in}{3.036642in}}%
\pgfpathlineto{\pgfqpoint{3.646329in}{3.040900in}}%
\pgfpathlineto{\pgfqpoint{3.650587in}{3.040900in}}%
\pgfpathlineto{\pgfqpoint{3.650587in}{3.036642in}}%
\pgfpathmoveto{\pgfqpoint{3.646329in}{3.040900in}}%
\pgfpathlineto{\pgfqpoint{3.646329in}{3.040900in}}%
\pgfpathlineto{\pgfqpoint{3.646329in}{3.045158in}}%
\pgfpathlineto{\pgfqpoint{3.650587in}{3.045158in}}%
\pgfpathlineto{\pgfqpoint{3.650587in}{3.040900in}}%
\pgfpathmoveto{\pgfqpoint{3.637814in}{3.045158in}}%
\pgfpathlineto{\pgfqpoint{3.637814in}{3.045158in}}%
\pgfpathlineto{\pgfqpoint{3.637814in}{3.049416in}}%
\pgfpathlineto{\pgfqpoint{3.642072in}{3.049416in}}%
\pgfpathlineto{\pgfqpoint{3.642072in}{3.045158in}}%
\pgfpathmoveto{\pgfqpoint{3.637814in}{3.049416in}}%
\pgfpathlineto{\pgfqpoint{3.637814in}{3.049416in}}%
\pgfpathlineto{\pgfqpoint{3.637814in}{3.053674in}}%
\pgfpathlineto{\pgfqpoint{3.642072in}{3.053674in}}%
\pgfpathlineto{\pgfqpoint{3.642072in}{3.049416in}}%
\pgfpathmoveto{\pgfqpoint{3.642072in}{3.045158in}}%
\pgfpathlineto{\pgfqpoint{3.642072in}{3.045158in}}%
\pgfpathlineto{\pgfqpoint{3.642072in}{3.049416in}}%
\pgfpathlineto{\pgfqpoint{3.646329in}{3.049416in}}%
\pgfpathlineto{\pgfqpoint{3.646329in}{3.045158in}}%
\pgfpathmoveto{\pgfqpoint{3.642072in}{3.049416in}}%
\pgfpathlineto{\pgfqpoint{3.642072in}{3.049416in}}%
\pgfpathlineto{\pgfqpoint{3.642072in}{3.053674in}}%
\pgfpathlineto{\pgfqpoint{3.646329in}{3.053674in}}%
\pgfpathlineto{\pgfqpoint{3.646329in}{3.049416in}}%
\pgfpathmoveto{\pgfqpoint{3.637814in}{3.053674in}}%
\pgfpathlineto{\pgfqpoint{3.637814in}{3.053674in}}%
\pgfpathlineto{\pgfqpoint{3.637814in}{3.057932in}}%
\pgfpathlineto{\pgfqpoint{3.642072in}{3.057932in}}%
\pgfpathlineto{\pgfqpoint{3.642072in}{3.053674in}}%
\pgfpathmoveto{\pgfqpoint{3.637814in}{3.057932in}}%
\pgfpathlineto{\pgfqpoint{3.637814in}{3.057932in}}%
\pgfpathlineto{\pgfqpoint{3.637814in}{3.062190in}}%
\pgfpathlineto{\pgfqpoint{3.642072in}{3.062190in}}%
\pgfpathlineto{\pgfqpoint{3.642072in}{3.057932in}}%
\pgfpathmoveto{\pgfqpoint{3.642072in}{3.053674in}}%
\pgfpathlineto{\pgfqpoint{3.642072in}{3.053674in}}%
\pgfpathlineto{\pgfqpoint{3.642072in}{3.057932in}}%
\pgfpathlineto{\pgfqpoint{3.646329in}{3.057932in}}%
\pgfpathlineto{\pgfqpoint{3.646329in}{3.053674in}}%
\pgfpathmoveto{\pgfqpoint{3.642072in}{3.057932in}}%
\pgfpathlineto{\pgfqpoint{3.642072in}{3.057932in}}%
\pgfpathlineto{\pgfqpoint{3.642072in}{3.062190in}}%
\pgfpathlineto{\pgfqpoint{3.646329in}{3.062190in}}%
\pgfpathlineto{\pgfqpoint{3.646329in}{3.057932in}}%
\pgfpathmoveto{\pgfqpoint{3.646329in}{3.045158in}}%
\pgfpathlineto{\pgfqpoint{3.646329in}{3.045158in}}%
\pgfpathlineto{\pgfqpoint{3.646329in}{3.049416in}}%
\pgfpathlineto{\pgfqpoint{3.650587in}{3.049416in}}%
\pgfpathlineto{\pgfqpoint{3.650587in}{3.045158in}}%
\pgfpathmoveto{\pgfqpoint{3.637814in}{3.062190in}}%
\pgfpathlineto{\pgfqpoint{3.637814in}{3.062190in}}%
\pgfpathlineto{\pgfqpoint{3.637814in}{3.066448in}}%
\pgfpathlineto{\pgfqpoint{3.642072in}{3.066448in}}%
\pgfpathlineto{\pgfqpoint{3.642072in}{3.062190in}}%
\pgfpathmoveto{\pgfqpoint{3.637814in}{3.066448in}}%
\pgfpathlineto{\pgfqpoint{3.637814in}{3.066448in}}%
\pgfpathlineto{\pgfqpoint{3.637814in}{3.070706in}}%
\pgfpathlineto{\pgfqpoint{3.642072in}{3.070706in}}%
\pgfpathlineto{\pgfqpoint{3.642072in}{3.066448in}}%
\pgfpathmoveto{\pgfqpoint{3.642072in}{3.062190in}}%
\pgfpathlineto{\pgfqpoint{3.642072in}{3.062190in}}%
\pgfpathlineto{\pgfqpoint{3.642072in}{3.066448in}}%
\pgfpathlineto{\pgfqpoint{3.646329in}{3.066448in}}%
\pgfpathlineto{\pgfqpoint{3.646329in}{3.062190in}}%
\pgfpathmoveto{\pgfqpoint{3.637814in}{3.070706in}}%
\pgfpathlineto{\pgfqpoint{3.637814in}{3.070706in}}%
\pgfpathlineto{\pgfqpoint{3.637814in}{3.074964in}}%
\pgfpathlineto{\pgfqpoint{3.642072in}{3.074964in}}%
\pgfpathlineto{\pgfqpoint{3.642072in}{3.070706in}}%
\pgfpathmoveto{\pgfqpoint{3.637814in}{3.074964in}}%
\pgfpathlineto{\pgfqpoint{3.637814in}{3.074964in}}%
\pgfpathlineto{\pgfqpoint{3.637814in}{3.079222in}}%
\pgfpathlineto{\pgfqpoint{3.642072in}{3.079222in}}%
\pgfpathlineto{\pgfqpoint{3.642072in}{3.074964in}}%
\pgfpathmoveto{\pgfqpoint{3.906056in}{1.405896in}}%
\pgfpathlineto{\pgfqpoint{3.906056in}{1.405896in}}%
\pgfpathlineto{\pgfqpoint{3.906056in}{1.410154in}}%
\pgfpathlineto{\pgfqpoint{3.910314in}{1.410154in}}%
\pgfpathlineto{\pgfqpoint{3.910314in}{1.405896in}}%
\pgfpathmoveto{\pgfqpoint{3.906056in}{1.410154in}}%
\pgfpathlineto{\pgfqpoint{3.906056in}{1.410154in}}%
\pgfpathlineto{\pgfqpoint{3.906056in}{1.414412in}}%
\pgfpathlineto{\pgfqpoint{3.910314in}{1.414412in}}%
\pgfpathlineto{\pgfqpoint{3.910314in}{1.410154in}}%
\pgfpathmoveto{\pgfqpoint{3.906056in}{1.414412in}}%
\pgfpathlineto{\pgfqpoint{3.906056in}{1.414412in}}%
\pgfpathlineto{\pgfqpoint{3.906056in}{1.418670in}}%
\pgfpathlineto{\pgfqpoint{3.910314in}{1.418670in}}%
\pgfpathlineto{\pgfqpoint{3.910314in}{1.414412in}}%
\pgfpathmoveto{\pgfqpoint{3.906056in}{1.418670in}}%
\pgfpathlineto{\pgfqpoint{3.906056in}{1.418670in}}%
\pgfpathlineto{\pgfqpoint{3.906056in}{1.422928in}}%
\pgfpathlineto{\pgfqpoint{3.910314in}{1.422928in}}%
\pgfpathlineto{\pgfqpoint{3.910314in}{1.418670in}}%
\pgfpathmoveto{\pgfqpoint{3.906056in}{1.422928in}}%
\pgfpathlineto{\pgfqpoint{3.906056in}{1.422928in}}%
\pgfpathlineto{\pgfqpoint{3.906056in}{1.427186in}}%
\pgfpathlineto{\pgfqpoint{3.910314in}{1.427186in}}%
\pgfpathlineto{\pgfqpoint{3.910314in}{1.422928in}}%
\pgfpathmoveto{\pgfqpoint{3.906056in}{1.427186in}}%
\pgfpathlineto{\pgfqpoint{3.906056in}{1.427186in}}%
\pgfpathlineto{\pgfqpoint{3.906056in}{1.431443in}}%
\pgfpathlineto{\pgfqpoint{3.910314in}{1.431443in}}%
\pgfpathlineto{\pgfqpoint{3.910314in}{1.427186in}}%
\pgfpathmoveto{\pgfqpoint{3.906056in}{1.431443in}}%
\pgfpathlineto{\pgfqpoint{3.906056in}{1.431443in}}%
\pgfpathlineto{\pgfqpoint{3.906056in}{1.435701in}}%
\pgfpathlineto{\pgfqpoint{3.910314in}{1.435701in}}%
\pgfpathlineto{\pgfqpoint{3.910314in}{1.431443in}}%
\pgfpathmoveto{\pgfqpoint{3.906056in}{1.435701in}}%
\pgfpathlineto{\pgfqpoint{3.906056in}{1.435701in}}%
\pgfpathlineto{\pgfqpoint{3.906056in}{1.439959in}}%
\pgfpathlineto{\pgfqpoint{3.910314in}{1.439959in}}%
\pgfpathlineto{\pgfqpoint{3.910314in}{1.435701in}}%
\pgfpathmoveto{\pgfqpoint{3.906056in}{1.439959in}}%
\pgfpathlineto{\pgfqpoint{3.906056in}{1.439959in}}%
\pgfpathlineto{\pgfqpoint{3.906056in}{1.444217in}}%
\pgfpathlineto{\pgfqpoint{3.910314in}{1.444217in}}%
\pgfpathlineto{\pgfqpoint{3.910314in}{1.439959in}}%
\pgfpathmoveto{\pgfqpoint{3.901798in}{1.444217in}}%
\pgfpathlineto{\pgfqpoint{3.901798in}{1.444217in}}%
\pgfpathlineto{\pgfqpoint{3.901798in}{1.448475in}}%
\pgfpathlineto{\pgfqpoint{3.906056in}{1.448475in}}%
\pgfpathlineto{\pgfqpoint{3.906056in}{1.444217in}}%
\pgfpathmoveto{\pgfqpoint{3.901798in}{1.448475in}}%
\pgfpathlineto{\pgfqpoint{3.901798in}{1.448475in}}%
\pgfpathlineto{\pgfqpoint{3.901798in}{1.452733in}}%
\pgfpathlineto{\pgfqpoint{3.906056in}{1.452733in}}%
\pgfpathlineto{\pgfqpoint{3.906056in}{1.448475in}}%
\pgfpathmoveto{\pgfqpoint{3.906056in}{1.444217in}}%
\pgfpathlineto{\pgfqpoint{3.906056in}{1.444217in}}%
\pgfpathlineto{\pgfqpoint{3.906056in}{1.448475in}}%
\pgfpathlineto{\pgfqpoint{3.910314in}{1.448475in}}%
\pgfpathlineto{\pgfqpoint{3.910314in}{1.444217in}}%
\pgfpathmoveto{\pgfqpoint{3.906056in}{1.448475in}}%
\pgfpathlineto{\pgfqpoint{3.906056in}{1.448475in}}%
\pgfpathlineto{\pgfqpoint{3.906056in}{1.452733in}}%
\pgfpathlineto{\pgfqpoint{3.910314in}{1.452733in}}%
\pgfpathlineto{\pgfqpoint{3.910314in}{1.448475in}}%
\pgfpathmoveto{\pgfqpoint{3.901798in}{1.452733in}}%
\pgfpathlineto{\pgfqpoint{3.901798in}{1.452733in}}%
\pgfpathlineto{\pgfqpoint{3.901798in}{1.456991in}}%
\pgfpathlineto{\pgfqpoint{3.906056in}{1.456991in}}%
\pgfpathlineto{\pgfqpoint{3.906056in}{1.452733in}}%
\pgfpathmoveto{\pgfqpoint{3.901798in}{1.456991in}}%
\pgfpathlineto{\pgfqpoint{3.901798in}{1.456991in}}%
\pgfpathlineto{\pgfqpoint{3.901798in}{1.461249in}}%
\pgfpathlineto{\pgfqpoint{3.906056in}{1.461249in}}%
\pgfpathlineto{\pgfqpoint{3.906056in}{1.456991in}}%
\pgfpathmoveto{\pgfqpoint{3.906056in}{1.452733in}}%
\pgfpathlineto{\pgfqpoint{3.906056in}{1.452733in}}%
\pgfpathlineto{\pgfqpoint{3.906056in}{1.456991in}}%
\pgfpathlineto{\pgfqpoint{3.910314in}{1.456991in}}%
\pgfpathlineto{\pgfqpoint{3.910314in}{1.452733in}}%
\pgfpathmoveto{\pgfqpoint{3.906056in}{1.456991in}}%
\pgfpathlineto{\pgfqpoint{3.906056in}{1.456991in}}%
\pgfpathlineto{\pgfqpoint{3.906056in}{1.461249in}}%
\pgfpathlineto{\pgfqpoint{3.910314in}{1.461249in}}%
\pgfpathlineto{\pgfqpoint{3.910314in}{1.456991in}}%
\pgfpathmoveto{\pgfqpoint{3.901798in}{1.461249in}}%
\pgfpathlineto{\pgfqpoint{3.901798in}{1.461249in}}%
\pgfpathlineto{\pgfqpoint{3.901798in}{1.465506in}}%
\pgfpathlineto{\pgfqpoint{3.906056in}{1.465506in}}%
\pgfpathlineto{\pgfqpoint{3.906056in}{1.461249in}}%
\pgfpathmoveto{\pgfqpoint{3.901798in}{1.465506in}}%
\pgfpathlineto{\pgfqpoint{3.901798in}{1.465506in}}%
\pgfpathlineto{\pgfqpoint{3.901798in}{1.469764in}}%
\pgfpathlineto{\pgfqpoint{3.906056in}{1.469764in}}%
\pgfpathlineto{\pgfqpoint{3.906056in}{1.465506in}}%
\pgfpathmoveto{\pgfqpoint{3.906056in}{1.461249in}}%
\pgfpathlineto{\pgfqpoint{3.906056in}{1.461249in}}%
\pgfpathlineto{\pgfqpoint{3.906056in}{1.465506in}}%
\pgfpathlineto{\pgfqpoint{3.910314in}{1.465506in}}%
\pgfpathlineto{\pgfqpoint{3.910314in}{1.461249in}}%
\pgfpathmoveto{\pgfqpoint{3.906056in}{1.465506in}}%
\pgfpathlineto{\pgfqpoint{3.906056in}{1.465506in}}%
\pgfpathlineto{\pgfqpoint{3.906056in}{1.469764in}}%
\pgfpathlineto{\pgfqpoint{3.910314in}{1.469764in}}%
\pgfpathlineto{\pgfqpoint{3.910314in}{1.465506in}}%
\pgfpathmoveto{\pgfqpoint{3.901798in}{1.469764in}}%
\pgfpathlineto{\pgfqpoint{3.901798in}{1.469764in}}%
\pgfpathlineto{\pgfqpoint{3.901798in}{1.474022in}}%
\pgfpathlineto{\pgfqpoint{3.906056in}{1.474022in}}%
\pgfpathlineto{\pgfqpoint{3.906056in}{1.469764in}}%
\pgfpathmoveto{\pgfqpoint{3.901798in}{1.474022in}}%
\pgfpathlineto{\pgfqpoint{3.901798in}{1.474022in}}%
\pgfpathlineto{\pgfqpoint{3.901798in}{1.478280in}}%
\pgfpathlineto{\pgfqpoint{3.906056in}{1.478280in}}%
\pgfpathlineto{\pgfqpoint{3.906056in}{1.474022in}}%
\pgfpathmoveto{\pgfqpoint{3.906056in}{1.469764in}}%
\pgfpathlineto{\pgfqpoint{3.906056in}{1.469764in}}%
\pgfpathlineto{\pgfqpoint{3.906056in}{1.474022in}}%
\pgfpathlineto{\pgfqpoint{3.910314in}{1.474022in}}%
\pgfpathlineto{\pgfqpoint{3.910314in}{1.469764in}}%
\pgfpathmoveto{\pgfqpoint{3.906056in}{1.474022in}}%
\pgfpathlineto{\pgfqpoint{3.906056in}{1.474022in}}%
\pgfpathlineto{\pgfqpoint{3.906056in}{1.478280in}}%
\pgfpathlineto{\pgfqpoint{3.910314in}{1.478280in}}%
\pgfpathlineto{\pgfqpoint{3.910314in}{1.474022in}}%
\pgfpathmoveto{\pgfqpoint{3.897540in}{1.478280in}}%
\pgfpathlineto{\pgfqpoint{3.897540in}{1.478280in}}%
\pgfpathlineto{\pgfqpoint{3.897540in}{1.482538in}}%
\pgfpathlineto{\pgfqpoint{3.901798in}{1.482538in}}%
\pgfpathlineto{\pgfqpoint{3.901798in}{1.478280in}}%
\pgfpathmoveto{\pgfqpoint{3.897540in}{1.482538in}}%
\pgfpathlineto{\pgfqpoint{3.897540in}{1.482538in}}%
\pgfpathlineto{\pgfqpoint{3.897540in}{1.486796in}}%
\pgfpathlineto{\pgfqpoint{3.901798in}{1.486796in}}%
\pgfpathlineto{\pgfqpoint{3.901798in}{1.482538in}}%
\pgfpathmoveto{\pgfqpoint{3.897540in}{1.486796in}}%
\pgfpathlineto{\pgfqpoint{3.897540in}{1.486796in}}%
\pgfpathlineto{\pgfqpoint{3.897540in}{1.491054in}}%
\pgfpathlineto{\pgfqpoint{3.901798in}{1.491054in}}%
\pgfpathlineto{\pgfqpoint{3.901798in}{1.486796in}}%
\pgfpathmoveto{\pgfqpoint{3.897540in}{1.491054in}}%
\pgfpathlineto{\pgfqpoint{3.897540in}{1.491054in}}%
\pgfpathlineto{\pgfqpoint{3.897540in}{1.495311in}}%
\pgfpathlineto{\pgfqpoint{3.901798in}{1.495311in}}%
\pgfpathlineto{\pgfqpoint{3.901798in}{1.491054in}}%
\pgfpathmoveto{\pgfqpoint{3.901798in}{1.478280in}}%
\pgfpathlineto{\pgfqpoint{3.901798in}{1.478280in}}%
\pgfpathlineto{\pgfqpoint{3.901798in}{1.482538in}}%
\pgfpathlineto{\pgfqpoint{3.906056in}{1.482538in}}%
\pgfpathlineto{\pgfqpoint{3.906056in}{1.478280in}}%
\pgfpathmoveto{\pgfqpoint{3.901798in}{1.482538in}}%
\pgfpathlineto{\pgfqpoint{3.901798in}{1.482538in}}%
\pgfpathlineto{\pgfqpoint{3.901798in}{1.486796in}}%
\pgfpathlineto{\pgfqpoint{3.906056in}{1.486796in}}%
\pgfpathlineto{\pgfqpoint{3.906056in}{1.482538in}}%
\pgfpathmoveto{\pgfqpoint{3.906056in}{1.478280in}}%
\pgfpathlineto{\pgfqpoint{3.906056in}{1.478280in}}%
\pgfpathlineto{\pgfqpoint{3.906056in}{1.482538in}}%
\pgfpathlineto{\pgfqpoint{3.910314in}{1.482538in}}%
\pgfpathlineto{\pgfqpoint{3.910314in}{1.478280in}}%
\pgfpathmoveto{\pgfqpoint{3.901798in}{1.486796in}}%
\pgfpathlineto{\pgfqpoint{3.901798in}{1.486796in}}%
\pgfpathlineto{\pgfqpoint{3.901798in}{1.491054in}}%
\pgfpathlineto{\pgfqpoint{3.906056in}{1.491054in}}%
\pgfpathlineto{\pgfqpoint{3.906056in}{1.486796in}}%
\pgfpathmoveto{\pgfqpoint{3.901798in}{1.491054in}}%
\pgfpathlineto{\pgfqpoint{3.901798in}{1.491054in}}%
\pgfpathlineto{\pgfqpoint{3.901798in}{1.495311in}}%
\pgfpathlineto{\pgfqpoint{3.906056in}{1.495311in}}%
\pgfpathlineto{\pgfqpoint{3.906056in}{1.491054in}}%
\pgfpathmoveto{\pgfqpoint{3.897540in}{1.495311in}}%
\pgfpathlineto{\pgfqpoint{3.897540in}{1.495311in}}%
\pgfpathlineto{\pgfqpoint{3.897540in}{1.499569in}}%
\pgfpathlineto{\pgfqpoint{3.901798in}{1.499569in}}%
\pgfpathlineto{\pgfqpoint{3.901798in}{1.495311in}}%
\pgfpathmoveto{\pgfqpoint{3.897540in}{1.499569in}}%
\pgfpathlineto{\pgfqpoint{3.897540in}{1.499569in}}%
\pgfpathlineto{\pgfqpoint{3.897540in}{1.503827in}}%
\pgfpathlineto{\pgfqpoint{3.901798in}{1.503827in}}%
\pgfpathlineto{\pgfqpoint{3.901798in}{1.499569in}}%
\pgfpathmoveto{\pgfqpoint{3.897540in}{1.503827in}}%
\pgfpathlineto{\pgfqpoint{3.897540in}{1.503827in}}%
\pgfpathlineto{\pgfqpoint{3.897540in}{1.508085in}}%
\pgfpathlineto{\pgfqpoint{3.901798in}{1.508085in}}%
\pgfpathlineto{\pgfqpoint{3.901798in}{1.503827in}}%
\pgfpathmoveto{\pgfqpoint{3.897540in}{1.508085in}}%
\pgfpathlineto{\pgfqpoint{3.897540in}{1.508085in}}%
\pgfpathlineto{\pgfqpoint{3.897540in}{1.512343in}}%
\pgfpathlineto{\pgfqpoint{3.901798in}{1.512343in}}%
\pgfpathlineto{\pgfqpoint{3.901798in}{1.508085in}}%
\pgfpathmoveto{\pgfqpoint{3.901798in}{1.495311in}}%
\pgfpathlineto{\pgfqpoint{3.901798in}{1.495311in}}%
\pgfpathlineto{\pgfqpoint{3.901798in}{1.499569in}}%
\pgfpathlineto{\pgfqpoint{3.906056in}{1.499569in}}%
\pgfpathlineto{\pgfqpoint{3.906056in}{1.495311in}}%
\pgfpathmoveto{\pgfqpoint{3.901798in}{1.499569in}}%
\pgfpathlineto{\pgfqpoint{3.901798in}{1.499569in}}%
\pgfpathlineto{\pgfqpoint{3.901798in}{1.503827in}}%
\pgfpathlineto{\pgfqpoint{3.906056in}{1.503827in}}%
\pgfpathlineto{\pgfqpoint{3.906056in}{1.499569in}}%
\pgfpathmoveto{\pgfqpoint{3.901798in}{1.503827in}}%
\pgfpathlineto{\pgfqpoint{3.901798in}{1.503827in}}%
\pgfpathlineto{\pgfqpoint{3.901798in}{1.508085in}}%
\pgfpathlineto{\pgfqpoint{3.906056in}{1.508085in}}%
\pgfpathlineto{\pgfqpoint{3.906056in}{1.503827in}}%
\pgfpathmoveto{\pgfqpoint{3.901798in}{1.508085in}}%
\pgfpathlineto{\pgfqpoint{3.901798in}{1.508085in}}%
\pgfpathlineto{\pgfqpoint{3.901798in}{1.512343in}}%
\pgfpathlineto{\pgfqpoint{3.906056in}{1.512343in}}%
\pgfpathlineto{\pgfqpoint{3.906056in}{1.508085in}}%
\pgfpathmoveto{\pgfqpoint{3.893282in}{1.516601in}}%
\pgfpathlineto{\pgfqpoint{3.893282in}{1.516601in}}%
\pgfpathlineto{\pgfqpoint{3.893282in}{1.520858in}}%
\pgfpathlineto{\pgfqpoint{3.897540in}{1.520858in}}%
\pgfpathlineto{\pgfqpoint{3.897540in}{1.516601in}}%
\pgfpathmoveto{\pgfqpoint{3.897540in}{1.512343in}}%
\pgfpathlineto{\pgfqpoint{3.897540in}{1.512343in}}%
\pgfpathlineto{\pgfqpoint{3.897540in}{1.516601in}}%
\pgfpathlineto{\pgfqpoint{3.901798in}{1.516601in}}%
\pgfpathlineto{\pgfqpoint{3.901798in}{1.512343in}}%
\pgfpathmoveto{\pgfqpoint{3.897540in}{1.516601in}}%
\pgfpathlineto{\pgfqpoint{3.897540in}{1.516601in}}%
\pgfpathlineto{\pgfqpoint{3.897540in}{1.520858in}}%
\pgfpathlineto{\pgfqpoint{3.901798in}{1.520858in}}%
\pgfpathlineto{\pgfqpoint{3.901798in}{1.516601in}}%
\pgfpathmoveto{\pgfqpoint{3.893282in}{1.520858in}}%
\pgfpathlineto{\pgfqpoint{3.893282in}{1.520858in}}%
\pgfpathlineto{\pgfqpoint{3.893282in}{1.525116in}}%
\pgfpathlineto{\pgfqpoint{3.897540in}{1.525116in}}%
\pgfpathlineto{\pgfqpoint{3.897540in}{1.520858in}}%
\pgfpathmoveto{\pgfqpoint{3.893282in}{1.525116in}}%
\pgfpathlineto{\pgfqpoint{3.893282in}{1.525116in}}%
\pgfpathlineto{\pgfqpoint{3.893282in}{1.529374in}}%
\pgfpathlineto{\pgfqpoint{3.897540in}{1.529374in}}%
\pgfpathlineto{\pgfqpoint{3.897540in}{1.525116in}}%
\pgfpathmoveto{\pgfqpoint{3.897540in}{1.520858in}}%
\pgfpathlineto{\pgfqpoint{3.897540in}{1.520858in}}%
\pgfpathlineto{\pgfqpoint{3.897540in}{1.525116in}}%
\pgfpathlineto{\pgfqpoint{3.901798in}{1.525116in}}%
\pgfpathlineto{\pgfqpoint{3.901798in}{1.520858in}}%
\pgfpathmoveto{\pgfqpoint{3.897540in}{1.525116in}}%
\pgfpathlineto{\pgfqpoint{3.897540in}{1.525116in}}%
\pgfpathlineto{\pgfqpoint{3.897540in}{1.529374in}}%
\pgfpathlineto{\pgfqpoint{3.901798in}{1.529374in}}%
\pgfpathlineto{\pgfqpoint{3.901798in}{1.525116in}}%
\pgfpathmoveto{\pgfqpoint{3.901798in}{1.512343in}}%
\pgfpathlineto{\pgfqpoint{3.901798in}{1.512343in}}%
\pgfpathlineto{\pgfqpoint{3.901798in}{1.516601in}}%
\pgfpathlineto{\pgfqpoint{3.906056in}{1.516601in}}%
\pgfpathlineto{\pgfqpoint{3.906056in}{1.512343in}}%
\pgfpathmoveto{\pgfqpoint{3.889024in}{1.550663in}}%
\pgfpathlineto{\pgfqpoint{3.889024in}{1.550663in}}%
\pgfpathlineto{\pgfqpoint{3.889024in}{1.554921in}}%
\pgfpathlineto{\pgfqpoint{3.893282in}{1.554921in}}%
\pgfpathlineto{\pgfqpoint{3.893282in}{1.550663in}}%
\pgfpathmoveto{\pgfqpoint{3.889024in}{1.554921in}}%
\pgfpathlineto{\pgfqpoint{3.889024in}{1.554921in}}%
\pgfpathlineto{\pgfqpoint{3.889024in}{1.559179in}}%
\pgfpathlineto{\pgfqpoint{3.893282in}{1.559179in}}%
\pgfpathlineto{\pgfqpoint{3.893282in}{1.554921in}}%
\pgfpathmoveto{\pgfqpoint{3.889024in}{1.559179in}}%
\pgfpathlineto{\pgfqpoint{3.889024in}{1.559179in}}%
\pgfpathlineto{\pgfqpoint{3.889024in}{1.563437in}}%
\pgfpathlineto{\pgfqpoint{3.893282in}{1.563437in}}%
\pgfpathlineto{\pgfqpoint{3.893282in}{1.559179in}}%
\pgfpathmoveto{\pgfqpoint{3.893282in}{1.529374in}}%
\pgfpathlineto{\pgfqpoint{3.893282in}{1.529374in}}%
\pgfpathlineto{\pgfqpoint{3.893282in}{1.533632in}}%
\pgfpathlineto{\pgfqpoint{3.897540in}{1.533632in}}%
\pgfpathlineto{\pgfqpoint{3.897540in}{1.529374in}}%
\pgfpathmoveto{\pgfqpoint{3.893282in}{1.533632in}}%
\pgfpathlineto{\pgfqpoint{3.893282in}{1.533632in}}%
\pgfpathlineto{\pgfqpoint{3.893282in}{1.537890in}}%
\pgfpathlineto{\pgfqpoint{3.897540in}{1.537890in}}%
\pgfpathlineto{\pgfqpoint{3.897540in}{1.533632in}}%
\pgfpathmoveto{\pgfqpoint{3.897540in}{1.529374in}}%
\pgfpathlineto{\pgfqpoint{3.897540in}{1.529374in}}%
\pgfpathlineto{\pgfqpoint{3.897540in}{1.533632in}}%
\pgfpathlineto{\pgfqpoint{3.901798in}{1.533632in}}%
\pgfpathlineto{\pgfqpoint{3.901798in}{1.529374in}}%
\pgfpathmoveto{\pgfqpoint{3.897540in}{1.533632in}}%
\pgfpathlineto{\pgfqpoint{3.897540in}{1.533632in}}%
\pgfpathlineto{\pgfqpoint{3.897540in}{1.537890in}}%
\pgfpathlineto{\pgfqpoint{3.901798in}{1.537890in}}%
\pgfpathlineto{\pgfqpoint{3.901798in}{1.533632in}}%
\pgfpathmoveto{\pgfqpoint{3.893282in}{1.537890in}}%
\pgfpathlineto{\pgfqpoint{3.893282in}{1.537890in}}%
\pgfpathlineto{\pgfqpoint{3.893282in}{1.542148in}}%
\pgfpathlineto{\pgfqpoint{3.897540in}{1.542148in}}%
\pgfpathlineto{\pgfqpoint{3.897540in}{1.537890in}}%
\pgfpathmoveto{\pgfqpoint{3.893282in}{1.542148in}}%
\pgfpathlineto{\pgfqpoint{3.893282in}{1.542148in}}%
\pgfpathlineto{\pgfqpoint{3.893282in}{1.546405in}}%
\pgfpathlineto{\pgfqpoint{3.897540in}{1.546405in}}%
\pgfpathlineto{\pgfqpoint{3.897540in}{1.542148in}}%
\pgfpathmoveto{\pgfqpoint{3.897540in}{1.537890in}}%
\pgfpathlineto{\pgfqpoint{3.897540in}{1.537890in}}%
\pgfpathlineto{\pgfqpoint{3.897540in}{1.542148in}}%
\pgfpathlineto{\pgfqpoint{3.901798in}{1.542148in}}%
\pgfpathlineto{\pgfqpoint{3.901798in}{1.537890in}}%
\pgfpathmoveto{\pgfqpoint{3.897540in}{1.542148in}}%
\pgfpathlineto{\pgfqpoint{3.897540in}{1.542148in}}%
\pgfpathlineto{\pgfqpoint{3.897540in}{1.546405in}}%
\pgfpathlineto{\pgfqpoint{3.901798in}{1.546405in}}%
\pgfpathlineto{\pgfqpoint{3.901798in}{1.542148in}}%
\pgfpathmoveto{\pgfqpoint{3.893282in}{1.546405in}}%
\pgfpathlineto{\pgfqpoint{3.893282in}{1.546405in}}%
\pgfpathlineto{\pgfqpoint{3.893282in}{1.550663in}}%
\pgfpathlineto{\pgfqpoint{3.897540in}{1.550663in}}%
\pgfpathlineto{\pgfqpoint{3.897540in}{1.546405in}}%
\pgfpathmoveto{\pgfqpoint{3.893282in}{1.550663in}}%
\pgfpathlineto{\pgfqpoint{3.893282in}{1.550663in}}%
\pgfpathlineto{\pgfqpoint{3.893282in}{1.554921in}}%
\pgfpathlineto{\pgfqpoint{3.897540in}{1.554921in}}%
\pgfpathlineto{\pgfqpoint{3.897540in}{1.550663in}}%
\pgfpathmoveto{\pgfqpoint{3.897540in}{1.546405in}}%
\pgfpathlineto{\pgfqpoint{3.897540in}{1.546405in}}%
\pgfpathlineto{\pgfqpoint{3.897540in}{1.550663in}}%
\pgfpathlineto{\pgfqpoint{3.901798in}{1.550663in}}%
\pgfpathlineto{\pgfqpoint{3.901798in}{1.546405in}}%
\pgfpathmoveto{\pgfqpoint{3.897540in}{1.550663in}}%
\pgfpathlineto{\pgfqpoint{3.897540in}{1.550663in}}%
\pgfpathlineto{\pgfqpoint{3.897540in}{1.554921in}}%
\pgfpathlineto{\pgfqpoint{3.901798in}{1.554921in}}%
\pgfpathlineto{\pgfqpoint{3.901798in}{1.550663in}}%
\pgfpathmoveto{\pgfqpoint{3.893282in}{1.554921in}}%
\pgfpathlineto{\pgfqpoint{3.893282in}{1.554921in}}%
\pgfpathlineto{\pgfqpoint{3.893282in}{1.559179in}}%
\pgfpathlineto{\pgfqpoint{3.897540in}{1.559179in}}%
\pgfpathlineto{\pgfqpoint{3.897540in}{1.554921in}}%
\pgfpathmoveto{\pgfqpoint{3.893282in}{1.559179in}}%
\pgfpathlineto{\pgfqpoint{3.893282in}{1.559179in}}%
\pgfpathlineto{\pgfqpoint{3.893282in}{1.563437in}}%
\pgfpathlineto{\pgfqpoint{3.897540in}{1.563437in}}%
\pgfpathlineto{\pgfqpoint{3.897540in}{1.559179in}}%
\pgfpathmoveto{\pgfqpoint{3.889024in}{1.563437in}}%
\pgfpathlineto{\pgfqpoint{3.889024in}{1.563437in}}%
\pgfpathlineto{\pgfqpoint{3.889024in}{1.567695in}}%
\pgfpathlineto{\pgfqpoint{3.893282in}{1.567695in}}%
\pgfpathlineto{\pgfqpoint{3.893282in}{1.563437in}}%
\pgfpathmoveto{\pgfqpoint{3.889024in}{1.567695in}}%
\pgfpathlineto{\pgfqpoint{3.889024in}{1.567695in}}%
\pgfpathlineto{\pgfqpoint{3.889024in}{1.571953in}}%
\pgfpathlineto{\pgfqpoint{3.893282in}{1.571953in}}%
\pgfpathlineto{\pgfqpoint{3.893282in}{1.567695in}}%
\pgfpathmoveto{\pgfqpoint{3.889024in}{1.571953in}}%
\pgfpathlineto{\pgfqpoint{3.889024in}{1.571953in}}%
\pgfpathlineto{\pgfqpoint{3.889024in}{1.576210in}}%
\pgfpathlineto{\pgfqpoint{3.893282in}{1.576210in}}%
\pgfpathlineto{\pgfqpoint{3.893282in}{1.571953in}}%
\pgfpathmoveto{\pgfqpoint{3.889024in}{1.576210in}}%
\pgfpathlineto{\pgfqpoint{3.889024in}{1.576210in}}%
\pgfpathlineto{\pgfqpoint{3.889024in}{1.580468in}}%
\pgfpathlineto{\pgfqpoint{3.893282in}{1.580468in}}%
\pgfpathlineto{\pgfqpoint{3.893282in}{1.576210in}}%
\pgfpathmoveto{\pgfqpoint{3.884766in}{1.584726in}}%
\pgfpathlineto{\pgfqpoint{3.884766in}{1.584726in}}%
\pgfpathlineto{\pgfqpoint{3.884766in}{1.588984in}}%
\pgfpathlineto{\pgfqpoint{3.889024in}{1.588984in}}%
\pgfpathlineto{\pgfqpoint{3.889024in}{1.584726in}}%
\pgfpathmoveto{\pgfqpoint{3.889024in}{1.580468in}}%
\pgfpathlineto{\pgfqpoint{3.889024in}{1.580468in}}%
\pgfpathlineto{\pgfqpoint{3.889024in}{1.584726in}}%
\pgfpathlineto{\pgfqpoint{3.893282in}{1.584726in}}%
\pgfpathlineto{\pgfqpoint{3.893282in}{1.580468in}}%
\pgfpathmoveto{\pgfqpoint{3.889024in}{1.584726in}}%
\pgfpathlineto{\pgfqpoint{3.889024in}{1.584726in}}%
\pgfpathlineto{\pgfqpoint{3.889024in}{1.588984in}}%
\pgfpathlineto{\pgfqpoint{3.893282in}{1.588984in}}%
\pgfpathlineto{\pgfqpoint{3.893282in}{1.584726in}}%
\pgfpathmoveto{\pgfqpoint{3.884766in}{1.588984in}}%
\pgfpathlineto{\pgfqpoint{3.884766in}{1.588984in}}%
\pgfpathlineto{\pgfqpoint{3.884766in}{1.593242in}}%
\pgfpathlineto{\pgfqpoint{3.889024in}{1.593242in}}%
\pgfpathlineto{\pgfqpoint{3.889024in}{1.588984in}}%
\pgfpathmoveto{\pgfqpoint{3.884766in}{1.593242in}}%
\pgfpathlineto{\pgfqpoint{3.884766in}{1.593242in}}%
\pgfpathlineto{\pgfqpoint{3.884766in}{1.597500in}}%
\pgfpathlineto{\pgfqpoint{3.889024in}{1.597500in}}%
\pgfpathlineto{\pgfqpoint{3.889024in}{1.593242in}}%
\pgfpathmoveto{\pgfqpoint{3.889024in}{1.588984in}}%
\pgfpathlineto{\pgfqpoint{3.889024in}{1.588984in}}%
\pgfpathlineto{\pgfqpoint{3.889024in}{1.593242in}}%
\pgfpathlineto{\pgfqpoint{3.893282in}{1.593242in}}%
\pgfpathlineto{\pgfqpoint{3.893282in}{1.588984in}}%
\pgfpathmoveto{\pgfqpoint{3.889024in}{1.593242in}}%
\pgfpathlineto{\pgfqpoint{3.889024in}{1.593242in}}%
\pgfpathlineto{\pgfqpoint{3.889024in}{1.597500in}}%
\pgfpathlineto{\pgfqpoint{3.893282in}{1.597500in}}%
\pgfpathlineto{\pgfqpoint{3.893282in}{1.593242in}}%
\pgfpathmoveto{\pgfqpoint{3.893282in}{1.563437in}}%
\pgfpathlineto{\pgfqpoint{3.893282in}{1.563437in}}%
\pgfpathlineto{\pgfqpoint{3.893282in}{1.567695in}}%
\pgfpathlineto{\pgfqpoint{3.897540in}{1.567695in}}%
\pgfpathlineto{\pgfqpoint{3.897540in}{1.563437in}}%
\pgfpathmoveto{\pgfqpoint{3.893282in}{1.567695in}}%
\pgfpathlineto{\pgfqpoint{3.893282in}{1.567695in}}%
\pgfpathlineto{\pgfqpoint{3.893282in}{1.571953in}}%
\pgfpathlineto{\pgfqpoint{3.897540in}{1.571953in}}%
\pgfpathlineto{\pgfqpoint{3.897540in}{1.567695in}}%
\pgfpathmoveto{\pgfqpoint{3.893282in}{1.571953in}}%
\pgfpathlineto{\pgfqpoint{3.893282in}{1.571953in}}%
\pgfpathlineto{\pgfqpoint{3.893282in}{1.576210in}}%
\pgfpathlineto{\pgfqpoint{3.897540in}{1.576210in}}%
\pgfpathlineto{\pgfqpoint{3.897540in}{1.571953in}}%
\pgfpathmoveto{\pgfqpoint{3.893282in}{1.576210in}}%
\pgfpathlineto{\pgfqpoint{3.893282in}{1.576210in}}%
\pgfpathlineto{\pgfqpoint{3.893282in}{1.580468in}}%
\pgfpathlineto{\pgfqpoint{3.897540in}{1.580468in}}%
\pgfpathlineto{\pgfqpoint{3.897540in}{1.576210in}}%
\pgfpathmoveto{\pgfqpoint{3.893282in}{1.580468in}}%
\pgfpathlineto{\pgfqpoint{3.893282in}{1.580468in}}%
\pgfpathlineto{\pgfqpoint{3.893282in}{1.584726in}}%
\pgfpathlineto{\pgfqpoint{3.897540in}{1.584726in}}%
\pgfpathlineto{\pgfqpoint{3.897540in}{1.580468in}}%
\pgfpathmoveto{\pgfqpoint{3.893282in}{1.584726in}}%
\pgfpathlineto{\pgfqpoint{3.893282in}{1.584726in}}%
\pgfpathlineto{\pgfqpoint{3.893282in}{1.588984in}}%
\pgfpathlineto{\pgfqpoint{3.897540in}{1.588984in}}%
\pgfpathlineto{\pgfqpoint{3.897540in}{1.584726in}}%
\pgfpathmoveto{\pgfqpoint{3.884766in}{1.597500in}}%
\pgfpathlineto{\pgfqpoint{3.884766in}{1.597500in}}%
\pgfpathlineto{\pgfqpoint{3.884766in}{1.601758in}}%
\pgfpathlineto{\pgfqpoint{3.889024in}{1.601758in}}%
\pgfpathlineto{\pgfqpoint{3.889024in}{1.597500in}}%
\pgfpathmoveto{\pgfqpoint{3.884766in}{1.601758in}}%
\pgfpathlineto{\pgfqpoint{3.884766in}{1.601758in}}%
\pgfpathlineto{\pgfqpoint{3.884766in}{1.606015in}}%
\pgfpathlineto{\pgfqpoint{3.889024in}{1.606015in}}%
\pgfpathlineto{\pgfqpoint{3.889024in}{1.601758in}}%
\pgfpathmoveto{\pgfqpoint{3.889024in}{1.597500in}}%
\pgfpathlineto{\pgfqpoint{3.889024in}{1.597500in}}%
\pgfpathlineto{\pgfqpoint{3.889024in}{1.601758in}}%
\pgfpathlineto{\pgfqpoint{3.893282in}{1.601758in}}%
\pgfpathlineto{\pgfqpoint{3.893282in}{1.597500in}}%
\pgfpathmoveto{\pgfqpoint{3.889024in}{1.601758in}}%
\pgfpathlineto{\pgfqpoint{3.889024in}{1.601758in}}%
\pgfpathlineto{\pgfqpoint{3.889024in}{1.606015in}}%
\pgfpathlineto{\pgfqpoint{3.893282in}{1.606015in}}%
\pgfpathlineto{\pgfqpoint{3.893282in}{1.601758in}}%
\pgfpathmoveto{\pgfqpoint{3.884766in}{1.606015in}}%
\pgfpathlineto{\pgfqpoint{3.884766in}{1.606015in}}%
\pgfpathlineto{\pgfqpoint{3.884766in}{1.610273in}}%
\pgfpathlineto{\pgfqpoint{3.889024in}{1.610273in}}%
\pgfpathlineto{\pgfqpoint{3.889024in}{1.606015in}}%
\pgfpathmoveto{\pgfqpoint{3.884766in}{1.610273in}}%
\pgfpathlineto{\pgfqpoint{3.884766in}{1.610273in}}%
\pgfpathlineto{\pgfqpoint{3.884766in}{1.614531in}}%
\pgfpathlineto{\pgfqpoint{3.889024in}{1.614531in}}%
\pgfpathlineto{\pgfqpoint{3.889024in}{1.610273in}}%
\pgfpathmoveto{\pgfqpoint{3.889024in}{1.606015in}}%
\pgfpathlineto{\pgfqpoint{3.889024in}{1.606015in}}%
\pgfpathlineto{\pgfqpoint{3.889024in}{1.610273in}}%
\pgfpathlineto{\pgfqpoint{3.893282in}{1.610273in}}%
\pgfpathlineto{\pgfqpoint{3.893282in}{1.606015in}}%
\pgfpathmoveto{\pgfqpoint{3.889024in}{1.610273in}}%
\pgfpathlineto{\pgfqpoint{3.889024in}{1.610273in}}%
\pgfpathlineto{\pgfqpoint{3.889024in}{1.614531in}}%
\pgfpathlineto{\pgfqpoint{3.893282in}{1.614531in}}%
\pgfpathlineto{\pgfqpoint{3.893282in}{1.610273in}}%
\pgfpathmoveto{\pgfqpoint{3.880508in}{1.618789in}}%
\pgfpathlineto{\pgfqpoint{3.880508in}{1.618789in}}%
\pgfpathlineto{\pgfqpoint{3.880508in}{1.623047in}}%
\pgfpathlineto{\pgfqpoint{3.884766in}{1.623047in}}%
\pgfpathlineto{\pgfqpoint{3.884766in}{1.618789in}}%
\pgfpathmoveto{\pgfqpoint{3.880508in}{1.623047in}}%
\pgfpathlineto{\pgfqpoint{3.880508in}{1.623047in}}%
\pgfpathlineto{\pgfqpoint{3.880508in}{1.627305in}}%
\pgfpathlineto{\pgfqpoint{3.884766in}{1.627305in}}%
\pgfpathlineto{\pgfqpoint{3.884766in}{1.623047in}}%
\pgfpathmoveto{\pgfqpoint{3.880508in}{1.627305in}}%
\pgfpathlineto{\pgfqpoint{3.880508in}{1.627305in}}%
\pgfpathlineto{\pgfqpoint{3.880508in}{1.631563in}}%
\pgfpathlineto{\pgfqpoint{3.884766in}{1.631563in}}%
\pgfpathlineto{\pgfqpoint{3.884766in}{1.627305in}}%
\pgfpathmoveto{\pgfqpoint{3.884766in}{1.614531in}}%
\pgfpathlineto{\pgfqpoint{3.884766in}{1.614531in}}%
\pgfpathlineto{\pgfqpoint{3.884766in}{1.618789in}}%
\pgfpathlineto{\pgfqpoint{3.889024in}{1.618789in}}%
\pgfpathlineto{\pgfqpoint{3.889024in}{1.614531in}}%
\pgfpathmoveto{\pgfqpoint{3.884766in}{1.618789in}}%
\pgfpathlineto{\pgfqpoint{3.884766in}{1.618789in}}%
\pgfpathlineto{\pgfqpoint{3.884766in}{1.623047in}}%
\pgfpathlineto{\pgfqpoint{3.889024in}{1.623047in}}%
\pgfpathlineto{\pgfqpoint{3.889024in}{1.618789in}}%
\pgfpathmoveto{\pgfqpoint{3.889024in}{1.614531in}}%
\pgfpathlineto{\pgfqpoint{3.889024in}{1.614531in}}%
\pgfpathlineto{\pgfqpoint{3.889024in}{1.618789in}}%
\pgfpathlineto{\pgfqpoint{3.893282in}{1.618789in}}%
\pgfpathlineto{\pgfqpoint{3.893282in}{1.614531in}}%
\pgfpathmoveto{\pgfqpoint{3.889024in}{1.618789in}}%
\pgfpathlineto{\pgfqpoint{3.889024in}{1.618789in}}%
\pgfpathlineto{\pgfqpoint{3.889024in}{1.623047in}}%
\pgfpathlineto{\pgfqpoint{3.893282in}{1.623047in}}%
\pgfpathlineto{\pgfqpoint{3.893282in}{1.618789in}}%
\pgfpathmoveto{\pgfqpoint{3.884766in}{1.623047in}}%
\pgfpathlineto{\pgfqpoint{3.884766in}{1.623047in}}%
\pgfpathlineto{\pgfqpoint{3.884766in}{1.627305in}}%
\pgfpathlineto{\pgfqpoint{3.889024in}{1.627305in}}%
\pgfpathlineto{\pgfqpoint{3.889024in}{1.623047in}}%
\pgfpathmoveto{\pgfqpoint{3.884766in}{1.627305in}}%
\pgfpathlineto{\pgfqpoint{3.884766in}{1.627305in}}%
\pgfpathlineto{\pgfqpoint{3.884766in}{1.631563in}}%
\pgfpathlineto{\pgfqpoint{3.889024in}{1.631563in}}%
\pgfpathlineto{\pgfqpoint{3.889024in}{1.627305in}}%
\pgfpathmoveto{\pgfqpoint{3.880508in}{1.631563in}}%
\pgfpathlineto{\pgfqpoint{3.880508in}{1.631563in}}%
\pgfpathlineto{\pgfqpoint{3.880508in}{1.635821in}}%
\pgfpathlineto{\pgfqpoint{3.884766in}{1.635821in}}%
\pgfpathlineto{\pgfqpoint{3.884766in}{1.631563in}}%
\pgfpathmoveto{\pgfqpoint{3.880508in}{1.635821in}}%
\pgfpathlineto{\pgfqpoint{3.880508in}{1.635821in}}%
\pgfpathlineto{\pgfqpoint{3.880508in}{1.640079in}}%
\pgfpathlineto{\pgfqpoint{3.884766in}{1.640079in}}%
\pgfpathlineto{\pgfqpoint{3.884766in}{1.635821in}}%
\pgfpathmoveto{\pgfqpoint{3.880508in}{1.640079in}}%
\pgfpathlineto{\pgfqpoint{3.880508in}{1.640079in}}%
\pgfpathlineto{\pgfqpoint{3.880508in}{1.644337in}}%
\pgfpathlineto{\pgfqpoint{3.884766in}{1.644337in}}%
\pgfpathlineto{\pgfqpoint{3.884766in}{1.640079in}}%
\pgfpathmoveto{\pgfqpoint{3.880508in}{1.644337in}}%
\pgfpathlineto{\pgfqpoint{3.880508in}{1.644337in}}%
\pgfpathlineto{\pgfqpoint{3.880508in}{1.648595in}}%
\pgfpathlineto{\pgfqpoint{3.884766in}{1.648595in}}%
\pgfpathlineto{\pgfqpoint{3.884766in}{1.644337in}}%
\pgfpathmoveto{\pgfqpoint{3.884766in}{1.631563in}}%
\pgfpathlineto{\pgfqpoint{3.884766in}{1.631563in}}%
\pgfpathlineto{\pgfqpoint{3.884766in}{1.635821in}}%
\pgfpathlineto{\pgfqpoint{3.889024in}{1.635821in}}%
\pgfpathlineto{\pgfqpoint{3.889024in}{1.631563in}}%
\pgfpathmoveto{\pgfqpoint{3.884766in}{1.635821in}}%
\pgfpathlineto{\pgfqpoint{3.884766in}{1.635821in}}%
\pgfpathlineto{\pgfqpoint{3.884766in}{1.640079in}}%
\pgfpathlineto{\pgfqpoint{3.889024in}{1.640079in}}%
\pgfpathlineto{\pgfqpoint{3.889024in}{1.635821in}}%
\pgfpathmoveto{\pgfqpoint{3.884766in}{1.640079in}}%
\pgfpathlineto{\pgfqpoint{3.884766in}{1.640079in}}%
\pgfpathlineto{\pgfqpoint{3.884766in}{1.644337in}}%
\pgfpathlineto{\pgfqpoint{3.889024in}{1.644337in}}%
\pgfpathlineto{\pgfqpoint{3.889024in}{1.640079in}}%
\pgfpathmoveto{\pgfqpoint{3.884766in}{1.644337in}}%
\pgfpathlineto{\pgfqpoint{3.884766in}{1.644337in}}%
\pgfpathlineto{\pgfqpoint{3.884766in}{1.648595in}}%
\pgfpathlineto{\pgfqpoint{3.889024in}{1.648595in}}%
\pgfpathlineto{\pgfqpoint{3.889024in}{1.644337in}}%
\pgfpathmoveto{\pgfqpoint{3.876251in}{1.652853in}}%
\pgfpathlineto{\pgfqpoint{3.876251in}{1.652853in}}%
\pgfpathlineto{\pgfqpoint{3.876251in}{1.657111in}}%
\pgfpathlineto{\pgfqpoint{3.880508in}{1.657111in}}%
\pgfpathlineto{\pgfqpoint{3.880508in}{1.652853in}}%
\pgfpathmoveto{\pgfqpoint{3.880508in}{1.648595in}}%
\pgfpathlineto{\pgfqpoint{3.880508in}{1.648595in}}%
\pgfpathlineto{\pgfqpoint{3.880508in}{1.652853in}}%
\pgfpathlineto{\pgfqpoint{3.884766in}{1.652853in}}%
\pgfpathlineto{\pgfqpoint{3.884766in}{1.648595in}}%
\pgfpathmoveto{\pgfqpoint{3.880508in}{1.652853in}}%
\pgfpathlineto{\pgfqpoint{3.880508in}{1.652853in}}%
\pgfpathlineto{\pgfqpoint{3.880508in}{1.657111in}}%
\pgfpathlineto{\pgfqpoint{3.884766in}{1.657111in}}%
\pgfpathlineto{\pgfqpoint{3.884766in}{1.652853in}}%
\pgfpathmoveto{\pgfqpoint{3.876251in}{1.657111in}}%
\pgfpathlineto{\pgfqpoint{3.876251in}{1.657111in}}%
\pgfpathlineto{\pgfqpoint{3.876251in}{1.661369in}}%
\pgfpathlineto{\pgfqpoint{3.880508in}{1.661369in}}%
\pgfpathlineto{\pgfqpoint{3.880508in}{1.657111in}}%
\pgfpathmoveto{\pgfqpoint{3.876251in}{1.661369in}}%
\pgfpathlineto{\pgfqpoint{3.876251in}{1.661369in}}%
\pgfpathlineto{\pgfqpoint{3.876251in}{1.665627in}}%
\pgfpathlineto{\pgfqpoint{3.880508in}{1.665627in}}%
\pgfpathlineto{\pgfqpoint{3.880508in}{1.661369in}}%
\pgfpathmoveto{\pgfqpoint{3.880508in}{1.657111in}}%
\pgfpathlineto{\pgfqpoint{3.880508in}{1.657111in}}%
\pgfpathlineto{\pgfqpoint{3.880508in}{1.661369in}}%
\pgfpathlineto{\pgfqpoint{3.884766in}{1.661369in}}%
\pgfpathlineto{\pgfqpoint{3.884766in}{1.657111in}}%
\pgfpathmoveto{\pgfqpoint{3.880508in}{1.661369in}}%
\pgfpathlineto{\pgfqpoint{3.880508in}{1.661369in}}%
\pgfpathlineto{\pgfqpoint{3.880508in}{1.665627in}}%
\pgfpathlineto{\pgfqpoint{3.884766in}{1.665627in}}%
\pgfpathlineto{\pgfqpoint{3.884766in}{1.661369in}}%
\pgfpathmoveto{\pgfqpoint{3.884766in}{1.648595in}}%
\pgfpathlineto{\pgfqpoint{3.884766in}{1.648595in}}%
\pgfpathlineto{\pgfqpoint{3.884766in}{1.652853in}}%
\pgfpathlineto{\pgfqpoint{3.889024in}{1.652853in}}%
\pgfpathlineto{\pgfqpoint{3.889024in}{1.648595in}}%
\pgfpathmoveto{\pgfqpoint{3.884766in}{1.652853in}}%
\pgfpathlineto{\pgfqpoint{3.884766in}{1.652853in}}%
\pgfpathlineto{\pgfqpoint{3.884766in}{1.657111in}}%
\pgfpathlineto{\pgfqpoint{3.889024in}{1.657111in}}%
\pgfpathlineto{\pgfqpoint{3.889024in}{1.652853in}}%
\pgfpathmoveto{\pgfqpoint{3.871993in}{1.686916in}}%
\pgfpathlineto{\pgfqpoint{3.871993in}{1.686916in}}%
\pgfpathlineto{\pgfqpoint{3.871993in}{1.691174in}}%
\pgfpathlineto{\pgfqpoint{3.876251in}{1.691174in}}%
\pgfpathlineto{\pgfqpoint{3.876251in}{1.686916in}}%
\pgfpathmoveto{\pgfqpoint{3.871993in}{1.691174in}}%
\pgfpathlineto{\pgfqpoint{3.871993in}{1.691174in}}%
\pgfpathlineto{\pgfqpoint{3.871993in}{1.695432in}}%
\pgfpathlineto{\pgfqpoint{3.876251in}{1.695432in}}%
\pgfpathlineto{\pgfqpoint{3.876251in}{1.691174in}}%
\pgfpathmoveto{\pgfqpoint{3.871993in}{1.695432in}}%
\pgfpathlineto{\pgfqpoint{3.871993in}{1.695432in}}%
\pgfpathlineto{\pgfqpoint{3.871993in}{1.699690in}}%
\pgfpathlineto{\pgfqpoint{3.876251in}{1.699690in}}%
\pgfpathlineto{\pgfqpoint{3.876251in}{1.695432in}}%
\pgfpathmoveto{\pgfqpoint{3.871993in}{1.699690in}}%
\pgfpathlineto{\pgfqpoint{3.871993in}{1.699690in}}%
\pgfpathlineto{\pgfqpoint{3.871993in}{1.703948in}}%
\pgfpathlineto{\pgfqpoint{3.876251in}{1.703948in}}%
\pgfpathlineto{\pgfqpoint{3.876251in}{1.699690in}}%
\pgfpathmoveto{\pgfqpoint{3.871993in}{1.703948in}}%
\pgfpathlineto{\pgfqpoint{3.871993in}{1.703948in}}%
\pgfpathlineto{\pgfqpoint{3.871993in}{1.708206in}}%
\pgfpathlineto{\pgfqpoint{3.876251in}{1.708206in}}%
\pgfpathlineto{\pgfqpoint{3.876251in}{1.703948in}}%
\pgfpathmoveto{\pgfqpoint{3.871993in}{1.708206in}}%
\pgfpathlineto{\pgfqpoint{3.871993in}{1.708206in}}%
\pgfpathlineto{\pgfqpoint{3.871993in}{1.712464in}}%
\pgfpathlineto{\pgfqpoint{3.876251in}{1.712464in}}%
\pgfpathlineto{\pgfqpoint{3.876251in}{1.708206in}}%
\pgfpathmoveto{\pgfqpoint{3.871993in}{1.712464in}}%
\pgfpathlineto{\pgfqpoint{3.871993in}{1.712464in}}%
\pgfpathlineto{\pgfqpoint{3.871993in}{1.716722in}}%
\pgfpathlineto{\pgfqpoint{3.876251in}{1.716722in}}%
\pgfpathlineto{\pgfqpoint{3.876251in}{1.712464in}}%
\pgfpathmoveto{\pgfqpoint{3.867735in}{1.720980in}}%
\pgfpathlineto{\pgfqpoint{3.867735in}{1.720980in}}%
\pgfpathlineto{\pgfqpoint{3.867735in}{1.725238in}}%
\pgfpathlineto{\pgfqpoint{3.871993in}{1.725238in}}%
\pgfpathlineto{\pgfqpoint{3.871993in}{1.720980in}}%
\pgfpathmoveto{\pgfqpoint{3.871993in}{1.716722in}}%
\pgfpathlineto{\pgfqpoint{3.871993in}{1.716722in}}%
\pgfpathlineto{\pgfqpoint{3.871993in}{1.720980in}}%
\pgfpathlineto{\pgfqpoint{3.876251in}{1.720980in}}%
\pgfpathlineto{\pgfqpoint{3.876251in}{1.716722in}}%
\pgfpathmoveto{\pgfqpoint{3.871993in}{1.720980in}}%
\pgfpathlineto{\pgfqpoint{3.871993in}{1.720980in}}%
\pgfpathlineto{\pgfqpoint{3.871993in}{1.725238in}}%
\pgfpathlineto{\pgfqpoint{3.876251in}{1.725238in}}%
\pgfpathlineto{\pgfqpoint{3.876251in}{1.720980in}}%
\pgfpathmoveto{\pgfqpoint{3.867735in}{1.725238in}}%
\pgfpathlineto{\pgfqpoint{3.867735in}{1.725238in}}%
\pgfpathlineto{\pgfqpoint{3.867735in}{1.729496in}}%
\pgfpathlineto{\pgfqpoint{3.871993in}{1.729496in}}%
\pgfpathlineto{\pgfqpoint{3.871993in}{1.725238in}}%
\pgfpathmoveto{\pgfqpoint{3.867735in}{1.729496in}}%
\pgfpathlineto{\pgfqpoint{3.867735in}{1.729496in}}%
\pgfpathlineto{\pgfqpoint{3.867735in}{1.733753in}}%
\pgfpathlineto{\pgfqpoint{3.871993in}{1.733753in}}%
\pgfpathlineto{\pgfqpoint{3.871993in}{1.729496in}}%
\pgfpathmoveto{\pgfqpoint{3.871993in}{1.725238in}}%
\pgfpathlineto{\pgfqpoint{3.871993in}{1.725238in}}%
\pgfpathlineto{\pgfqpoint{3.871993in}{1.729496in}}%
\pgfpathlineto{\pgfqpoint{3.876251in}{1.729496in}}%
\pgfpathlineto{\pgfqpoint{3.876251in}{1.725238in}}%
\pgfpathmoveto{\pgfqpoint{3.871993in}{1.729496in}}%
\pgfpathlineto{\pgfqpoint{3.871993in}{1.729496in}}%
\pgfpathlineto{\pgfqpoint{3.871993in}{1.733753in}}%
\pgfpathlineto{\pgfqpoint{3.876251in}{1.733753in}}%
\pgfpathlineto{\pgfqpoint{3.876251in}{1.729496in}}%
\pgfpathmoveto{\pgfqpoint{3.876251in}{1.665627in}}%
\pgfpathlineto{\pgfqpoint{3.876251in}{1.665627in}}%
\pgfpathlineto{\pgfqpoint{3.876251in}{1.669884in}}%
\pgfpathlineto{\pgfqpoint{3.880508in}{1.669884in}}%
\pgfpathlineto{\pgfqpoint{3.880508in}{1.665627in}}%
\pgfpathmoveto{\pgfqpoint{3.876251in}{1.669884in}}%
\pgfpathlineto{\pgfqpoint{3.876251in}{1.669884in}}%
\pgfpathlineto{\pgfqpoint{3.876251in}{1.674142in}}%
\pgfpathlineto{\pgfqpoint{3.880508in}{1.674142in}}%
\pgfpathlineto{\pgfqpoint{3.880508in}{1.669884in}}%
\pgfpathmoveto{\pgfqpoint{3.880508in}{1.665627in}}%
\pgfpathlineto{\pgfqpoint{3.880508in}{1.665627in}}%
\pgfpathlineto{\pgfqpoint{3.880508in}{1.669884in}}%
\pgfpathlineto{\pgfqpoint{3.884766in}{1.669884in}}%
\pgfpathlineto{\pgfqpoint{3.884766in}{1.665627in}}%
\pgfpathmoveto{\pgfqpoint{3.880508in}{1.669884in}}%
\pgfpathlineto{\pgfqpoint{3.880508in}{1.669884in}}%
\pgfpathlineto{\pgfqpoint{3.880508in}{1.674142in}}%
\pgfpathlineto{\pgfqpoint{3.884766in}{1.674142in}}%
\pgfpathlineto{\pgfqpoint{3.884766in}{1.669884in}}%
\pgfpathmoveto{\pgfqpoint{3.876251in}{1.674142in}}%
\pgfpathlineto{\pgfqpoint{3.876251in}{1.674142in}}%
\pgfpathlineto{\pgfqpoint{3.876251in}{1.678400in}}%
\pgfpathlineto{\pgfqpoint{3.880508in}{1.678400in}}%
\pgfpathlineto{\pgfqpoint{3.880508in}{1.674142in}}%
\pgfpathmoveto{\pgfqpoint{3.876251in}{1.678400in}}%
\pgfpathlineto{\pgfqpoint{3.876251in}{1.678400in}}%
\pgfpathlineto{\pgfqpoint{3.876251in}{1.682658in}}%
\pgfpathlineto{\pgfqpoint{3.880508in}{1.682658in}}%
\pgfpathlineto{\pgfqpoint{3.880508in}{1.678400in}}%
\pgfpathmoveto{\pgfqpoint{3.880508in}{1.674142in}}%
\pgfpathlineto{\pgfqpoint{3.880508in}{1.674142in}}%
\pgfpathlineto{\pgfqpoint{3.880508in}{1.678400in}}%
\pgfpathlineto{\pgfqpoint{3.884766in}{1.678400in}}%
\pgfpathlineto{\pgfqpoint{3.884766in}{1.674142in}}%
\pgfpathmoveto{\pgfqpoint{3.880508in}{1.678400in}}%
\pgfpathlineto{\pgfqpoint{3.880508in}{1.678400in}}%
\pgfpathlineto{\pgfqpoint{3.880508in}{1.682658in}}%
\pgfpathlineto{\pgfqpoint{3.884766in}{1.682658in}}%
\pgfpathlineto{\pgfqpoint{3.884766in}{1.678400in}}%
\pgfpathmoveto{\pgfqpoint{3.876251in}{1.682658in}}%
\pgfpathlineto{\pgfqpoint{3.876251in}{1.682658in}}%
\pgfpathlineto{\pgfqpoint{3.876251in}{1.686916in}}%
\pgfpathlineto{\pgfqpoint{3.880508in}{1.686916in}}%
\pgfpathlineto{\pgfqpoint{3.880508in}{1.682658in}}%
\pgfpathmoveto{\pgfqpoint{3.876251in}{1.686916in}}%
\pgfpathlineto{\pgfqpoint{3.876251in}{1.686916in}}%
\pgfpathlineto{\pgfqpoint{3.876251in}{1.691174in}}%
\pgfpathlineto{\pgfqpoint{3.880508in}{1.691174in}}%
\pgfpathlineto{\pgfqpoint{3.880508in}{1.686916in}}%
\pgfpathmoveto{\pgfqpoint{3.880508in}{1.682658in}}%
\pgfpathlineto{\pgfqpoint{3.880508in}{1.682658in}}%
\pgfpathlineto{\pgfqpoint{3.880508in}{1.686916in}}%
\pgfpathlineto{\pgfqpoint{3.884766in}{1.686916in}}%
\pgfpathlineto{\pgfqpoint{3.884766in}{1.682658in}}%
\pgfpathmoveto{\pgfqpoint{3.880508in}{1.686916in}}%
\pgfpathlineto{\pgfqpoint{3.880508in}{1.686916in}}%
\pgfpathlineto{\pgfqpoint{3.880508in}{1.691174in}}%
\pgfpathlineto{\pgfqpoint{3.884766in}{1.691174in}}%
\pgfpathlineto{\pgfqpoint{3.884766in}{1.686916in}}%
\pgfpathmoveto{\pgfqpoint{3.876251in}{1.691174in}}%
\pgfpathlineto{\pgfqpoint{3.876251in}{1.691174in}}%
\pgfpathlineto{\pgfqpoint{3.876251in}{1.695432in}}%
\pgfpathlineto{\pgfqpoint{3.880508in}{1.695432in}}%
\pgfpathlineto{\pgfqpoint{3.880508in}{1.691174in}}%
\pgfpathmoveto{\pgfqpoint{3.876251in}{1.695432in}}%
\pgfpathlineto{\pgfqpoint{3.876251in}{1.695432in}}%
\pgfpathlineto{\pgfqpoint{3.876251in}{1.699690in}}%
\pgfpathlineto{\pgfqpoint{3.880508in}{1.699690in}}%
\pgfpathlineto{\pgfqpoint{3.880508in}{1.695432in}}%
\pgfpathmoveto{\pgfqpoint{3.876251in}{1.699690in}}%
\pgfpathlineto{\pgfqpoint{3.876251in}{1.699690in}}%
\pgfpathlineto{\pgfqpoint{3.876251in}{1.703948in}}%
\pgfpathlineto{\pgfqpoint{3.880508in}{1.703948in}}%
\pgfpathlineto{\pgfqpoint{3.880508in}{1.699690in}}%
\pgfpathmoveto{\pgfqpoint{3.876251in}{1.703948in}}%
\pgfpathlineto{\pgfqpoint{3.876251in}{1.703948in}}%
\pgfpathlineto{\pgfqpoint{3.876251in}{1.708206in}}%
\pgfpathlineto{\pgfqpoint{3.880508in}{1.708206in}}%
\pgfpathlineto{\pgfqpoint{3.880508in}{1.703948in}}%
\pgfpathmoveto{\pgfqpoint{3.876251in}{1.708206in}}%
\pgfpathlineto{\pgfqpoint{3.876251in}{1.708206in}}%
\pgfpathlineto{\pgfqpoint{3.876251in}{1.712464in}}%
\pgfpathlineto{\pgfqpoint{3.880508in}{1.712464in}}%
\pgfpathlineto{\pgfqpoint{3.880508in}{1.708206in}}%
\pgfpathmoveto{\pgfqpoint{3.876251in}{1.712464in}}%
\pgfpathlineto{\pgfqpoint{3.876251in}{1.712464in}}%
\pgfpathlineto{\pgfqpoint{3.876251in}{1.716722in}}%
\pgfpathlineto{\pgfqpoint{3.880508in}{1.716722in}}%
\pgfpathlineto{\pgfqpoint{3.880508in}{1.712464in}}%
\pgfpathmoveto{\pgfqpoint{3.876251in}{1.716722in}}%
\pgfpathlineto{\pgfqpoint{3.876251in}{1.716722in}}%
\pgfpathlineto{\pgfqpoint{3.876251in}{1.720980in}}%
\pgfpathlineto{\pgfqpoint{3.880508in}{1.720980in}}%
\pgfpathlineto{\pgfqpoint{3.880508in}{1.716722in}}%
\pgfpathmoveto{\pgfqpoint{3.876251in}{1.720980in}}%
\pgfpathlineto{\pgfqpoint{3.876251in}{1.720980in}}%
\pgfpathlineto{\pgfqpoint{3.876251in}{1.725238in}}%
\pgfpathlineto{\pgfqpoint{3.880508in}{1.725238in}}%
\pgfpathlineto{\pgfqpoint{3.880508in}{1.720980in}}%
\pgfpathmoveto{\pgfqpoint{3.867735in}{1.733753in}}%
\pgfpathlineto{\pgfqpoint{3.867735in}{1.733753in}}%
\pgfpathlineto{\pgfqpoint{3.867735in}{1.738011in}}%
\pgfpathlineto{\pgfqpoint{3.871993in}{1.738011in}}%
\pgfpathlineto{\pgfqpoint{3.871993in}{1.733753in}}%
\pgfpathmoveto{\pgfqpoint{3.867735in}{1.738011in}}%
\pgfpathlineto{\pgfqpoint{3.867735in}{1.738011in}}%
\pgfpathlineto{\pgfqpoint{3.867735in}{1.742269in}}%
\pgfpathlineto{\pgfqpoint{3.871993in}{1.742269in}}%
\pgfpathlineto{\pgfqpoint{3.871993in}{1.738011in}}%
\pgfpathmoveto{\pgfqpoint{3.871993in}{1.733753in}}%
\pgfpathlineto{\pgfqpoint{3.871993in}{1.733753in}}%
\pgfpathlineto{\pgfqpoint{3.871993in}{1.738011in}}%
\pgfpathlineto{\pgfqpoint{3.876251in}{1.738011in}}%
\pgfpathlineto{\pgfqpoint{3.876251in}{1.733753in}}%
\pgfpathmoveto{\pgfqpoint{3.871993in}{1.738011in}}%
\pgfpathlineto{\pgfqpoint{3.871993in}{1.738011in}}%
\pgfpathlineto{\pgfqpoint{3.871993in}{1.742269in}}%
\pgfpathlineto{\pgfqpoint{3.876251in}{1.742269in}}%
\pgfpathlineto{\pgfqpoint{3.876251in}{1.738011in}}%
\pgfpathmoveto{\pgfqpoint{3.867735in}{1.742269in}}%
\pgfpathlineto{\pgfqpoint{3.867735in}{1.742269in}}%
\pgfpathlineto{\pgfqpoint{3.867735in}{1.746526in}}%
\pgfpathlineto{\pgfqpoint{3.871993in}{1.746526in}}%
\pgfpathlineto{\pgfqpoint{3.871993in}{1.742269in}}%
\pgfpathmoveto{\pgfqpoint{3.867735in}{1.746526in}}%
\pgfpathlineto{\pgfqpoint{3.867735in}{1.746526in}}%
\pgfpathlineto{\pgfqpoint{3.867735in}{1.750784in}}%
\pgfpathlineto{\pgfqpoint{3.871993in}{1.750784in}}%
\pgfpathlineto{\pgfqpoint{3.871993in}{1.746526in}}%
\pgfpathmoveto{\pgfqpoint{3.871993in}{1.742269in}}%
\pgfpathlineto{\pgfqpoint{3.871993in}{1.742269in}}%
\pgfpathlineto{\pgfqpoint{3.871993in}{1.746526in}}%
\pgfpathlineto{\pgfqpoint{3.876251in}{1.746526in}}%
\pgfpathlineto{\pgfqpoint{3.876251in}{1.742269in}}%
\pgfpathmoveto{\pgfqpoint{3.871993in}{1.746526in}}%
\pgfpathlineto{\pgfqpoint{3.871993in}{1.746526in}}%
\pgfpathlineto{\pgfqpoint{3.871993in}{1.750784in}}%
\pgfpathlineto{\pgfqpoint{3.876251in}{1.750784in}}%
\pgfpathlineto{\pgfqpoint{3.876251in}{1.746526in}}%
\pgfpathmoveto{\pgfqpoint{3.863477in}{1.755041in}}%
\pgfpathlineto{\pgfqpoint{3.863477in}{1.755041in}}%
\pgfpathlineto{\pgfqpoint{3.863477in}{1.759299in}}%
\pgfpathlineto{\pgfqpoint{3.867735in}{1.759299in}}%
\pgfpathlineto{\pgfqpoint{3.867735in}{1.755041in}}%
\pgfpathmoveto{\pgfqpoint{3.863477in}{1.759299in}}%
\pgfpathlineto{\pgfqpoint{3.863477in}{1.759299in}}%
\pgfpathlineto{\pgfqpoint{3.863477in}{1.763557in}}%
\pgfpathlineto{\pgfqpoint{3.867735in}{1.763557in}}%
\pgfpathlineto{\pgfqpoint{3.867735in}{1.759299in}}%
\pgfpathmoveto{\pgfqpoint{3.863477in}{1.763557in}}%
\pgfpathlineto{\pgfqpoint{3.863477in}{1.763557in}}%
\pgfpathlineto{\pgfqpoint{3.863477in}{1.767814in}}%
\pgfpathlineto{\pgfqpoint{3.867735in}{1.767814in}}%
\pgfpathlineto{\pgfqpoint{3.867735in}{1.763557in}}%
\pgfpathmoveto{\pgfqpoint{3.867735in}{1.750784in}}%
\pgfpathlineto{\pgfqpoint{3.867735in}{1.750784in}}%
\pgfpathlineto{\pgfqpoint{3.867735in}{1.755041in}}%
\pgfpathlineto{\pgfqpoint{3.871993in}{1.755041in}}%
\pgfpathlineto{\pgfqpoint{3.871993in}{1.750784in}}%
\pgfpathmoveto{\pgfqpoint{3.867735in}{1.755041in}}%
\pgfpathlineto{\pgfqpoint{3.867735in}{1.755041in}}%
\pgfpathlineto{\pgfqpoint{3.867735in}{1.759299in}}%
\pgfpathlineto{\pgfqpoint{3.871993in}{1.759299in}}%
\pgfpathlineto{\pgfqpoint{3.871993in}{1.755041in}}%
\pgfpathmoveto{\pgfqpoint{3.871993in}{1.750784in}}%
\pgfpathlineto{\pgfqpoint{3.871993in}{1.750784in}}%
\pgfpathlineto{\pgfqpoint{3.871993in}{1.755041in}}%
\pgfpathlineto{\pgfqpoint{3.876251in}{1.755041in}}%
\pgfpathlineto{\pgfqpoint{3.876251in}{1.750784in}}%
\pgfpathmoveto{\pgfqpoint{3.871993in}{1.755041in}}%
\pgfpathlineto{\pgfqpoint{3.871993in}{1.755041in}}%
\pgfpathlineto{\pgfqpoint{3.871993in}{1.759299in}}%
\pgfpathlineto{\pgfqpoint{3.876251in}{1.759299in}}%
\pgfpathlineto{\pgfqpoint{3.876251in}{1.755041in}}%
\pgfpathmoveto{\pgfqpoint{3.867735in}{1.759299in}}%
\pgfpathlineto{\pgfqpoint{3.867735in}{1.759299in}}%
\pgfpathlineto{\pgfqpoint{3.867735in}{1.763557in}}%
\pgfpathlineto{\pgfqpoint{3.871993in}{1.763557in}}%
\pgfpathlineto{\pgfqpoint{3.871993in}{1.759299in}}%
\pgfpathmoveto{\pgfqpoint{3.867735in}{1.763557in}}%
\pgfpathlineto{\pgfqpoint{3.867735in}{1.763557in}}%
\pgfpathlineto{\pgfqpoint{3.867735in}{1.767814in}}%
\pgfpathlineto{\pgfqpoint{3.871993in}{1.767814in}}%
\pgfpathlineto{\pgfqpoint{3.871993in}{1.763557in}}%
\pgfpathmoveto{\pgfqpoint{3.863477in}{1.767814in}}%
\pgfpathlineto{\pgfqpoint{3.863477in}{1.767814in}}%
\pgfpathlineto{\pgfqpoint{3.863477in}{1.772072in}}%
\pgfpathlineto{\pgfqpoint{3.867735in}{1.772072in}}%
\pgfpathlineto{\pgfqpoint{3.867735in}{1.767814in}}%
\pgfpathmoveto{\pgfqpoint{3.863477in}{1.772072in}}%
\pgfpathlineto{\pgfqpoint{3.863477in}{1.772072in}}%
\pgfpathlineto{\pgfqpoint{3.863477in}{1.776329in}}%
\pgfpathlineto{\pgfqpoint{3.867735in}{1.776329in}}%
\pgfpathlineto{\pgfqpoint{3.867735in}{1.772072in}}%
\pgfpathmoveto{\pgfqpoint{3.863477in}{1.776329in}}%
\pgfpathlineto{\pgfqpoint{3.863477in}{1.776329in}}%
\pgfpathlineto{\pgfqpoint{3.863477in}{1.780587in}}%
\pgfpathlineto{\pgfqpoint{3.867735in}{1.780587in}}%
\pgfpathlineto{\pgfqpoint{3.867735in}{1.776329in}}%
\pgfpathmoveto{\pgfqpoint{3.863477in}{1.780587in}}%
\pgfpathlineto{\pgfqpoint{3.863477in}{1.780587in}}%
\pgfpathlineto{\pgfqpoint{3.863477in}{1.784844in}}%
\pgfpathlineto{\pgfqpoint{3.867735in}{1.784844in}}%
\pgfpathlineto{\pgfqpoint{3.867735in}{1.780587in}}%
\pgfpathmoveto{\pgfqpoint{3.867735in}{1.767814in}}%
\pgfpathlineto{\pgfqpoint{3.867735in}{1.767814in}}%
\pgfpathlineto{\pgfqpoint{3.867735in}{1.772072in}}%
\pgfpathlineto{\pgfqpoint{3.871993in}{1.772072in}}%
\pgfpathlineto{\pgfqpoint{3.871993in}{1.767814in}}%
\pgfpathmoveto{\pgfqpoint{3.867735in}{1.772072in}}%
\pgfpathlineto{\pgfqpoint{3.867735in}{1.772072in}}%
\pgfpathlineto{\pgfqpoint{3.867735in}{1.776329in}}%
\pgfpathlineto{\pgfqpoint{3.871993in}{1.776329in}}%
\pgfpathlineto{\pgfqpoint{3.871993in}{1.772072in}}%
\pgfpathmoveto{\pgfqpoint{3.867735in}{1.776329in}}%
\pgfpathlineto{\pgfqpoint{3.867735in}{1.776329in}}%
\pgfpathlineto{\pgfqpoint{3.867735in}{1.780587in}}%
\pgfpathlineto{\pgfqpoint{3.871993in}{1.780587in}}%
\pgfpathlineto{\pgfqpoint{3.871993in}{1.776329in}}%
\pgfpathmoveto{\pgfqpoint{3.867735in}{1.780587in}}%
\pgfpathlineto{\pgfqpoint{3.867735in}{1.780587in}}%
\pgfpathlineto{\pgfqpoint{3.867735in}{1.784844in}}%
\pgfpathlineto{\pgfqpoint{3.871993in}{1.784844in}}%
\pgfpathlineto{\pgfqpoint{3.871993in}{1.780587in}}%
\pgfpathmoveto{\pgfqpoint{3.859219in}{1.784844in}}%
\pgfpathlineto{\pgfqpoint{3.859219in}{1.784844in}}%
\pgfpathlineto{\pgfqpoint{3.859219in}{1.789102in}}%
\pgfpathlineto{\pgfqpoint{3.863477in}{1.789102in}}%
\pgfpathlineto{\pgfqpoint{3.863477in}{1.784844in}}%
\pgfpathmoveto{\pgfqpoint{3.859219in}{1.789102in}}%
\pgfpathlineto{\pgfqpoint{3.859219in}{1.789102in}}%
\pgfpathlineto{\pgfqpoint{3.859219in}{1.793360in}}%
\pgfpathlineto{\pgfqpoint{3.863477in}{1.793360in}}%
\pgfpathlineto{\pgfqpoint{3.863477in}{1.789102in}}%
\pgfpathmoveto{\pgfqpoint{3.863477in}{1.784844in}}%
\pgfpathlineto{\pgfqpoint{3.863477in}{1.784844in}}%
\pgfpathlineto{\pgfqpoint{3.863477in}{1.789102in}}%
\pgfpathlineto{\pgfqpoint{3.867735in}{1.789102in}}%
\pgfpathlineto{\pgfqpoint{3.867735in}{1.784844in}}%
\pgfpathmoveto{\pgfqpoint{3.863477in}{1.789102in}}%
\pgfpathlineto{\pgfqpoint{3.863477in}{1.789102in}}%
\pgfpathlineto{\pgfqpoint{3.863477in}{1.793360in}}%
\pgfpathlineto{\pgfqpoint{3.867735in}{1.793360in}}%
\pgfpathlineto{\pgfqpoint{3.867735in}{1.789102in}}%
\pgfpathmoveto{\pgfqpoint{3.859219in}{1.793360in}}%
\pgfpathlineto{\pgfqpoint{3.859219in}{1.793360in}}%
\pgfpathlineto{\pgfqpoint{3.859219in}{1.797617in}}%
\pgfpathlineto{\pgfqpoint{3.863477in}{1.797617in}}%
\pgfpathlineto{\pgfqpoint{3.863477in}{1.793360in}}%
\pgfpathmoveto{\pgfqpoint{3.859219in}{1.797617in}}%
\pgfpathlineto{\pgfqpoint{3.859219in}{1.797617in}}%
\pgfpathlineto{\pgfqpoint{3.859219in}{1.801875in}}%
\pgfpathlineto{\pgfqpoint{3.863477in}{1.801875in}}%
\pgfpathlineto{\pgfqpoint{3.863477in}{1.797617in}}%
\pgfpathmoveto{\pgfqpoint{3.863477in}{1.793360in}}%
\pgfpathlineto{\pgfqpoint{3.863477in}{1.793360in}}%
\pgfpathlineto{\pgfqpoint{3.863477in}{1.797617in}}%
\pgfpathlineto{\pgfqpoint{3.867735in}{1.797617in}}%
\pgfpathlineto{\pgfqpoint{3.867735in}{1.793360in}}%
\pgfpathmoveto{\pgfqpoint{3.863477in}{1.797617in}}%
\pgfpathlineto{\pgfqpoint{3.863477in}{1.797617in}}%
\pgfpathlineto{\pgfqpoint{3.863477in}{1.801875in}}%
\pgfpathlineto{\pgfqpoint{3.867735in}{1.801875in}}%
\pgfpathlineto{\pgfqpoint{3.867735in}{1.797617in}}%
\pgfpathmoveto{\pgfqpoint{3.867735in}{1.784844in}}%
\pgfpathlineto{\pgfqpoint{3.867735in}{1.784844in}}%
\pgfpathlineto{\pgfqpoint{3.867735in}{1.789102in}}%
\pgfpathlineto{\pgfqpoint{3.871993in}{1.789102in}}%
\pgfpathlineto{\pgfqpoint{3.871993in}{1.784844in}}%
\pgfpathmoveto{\pgfqpoint{3.867735in}{1.789102in}}%
\pgfpathlineto{\pgfqpoint{3.867735in}{1.789102in}}%
\pgfpathlineto{\pgfqpoint{3.867735in}{1.793360in}}%
\pgfpathlineto{\pgfqpoint{3.871993in}{1.793360in}}%
\pgfpathlineto{\pgfqpoint{3.871993in}{1.789102in}}%
\pgfpathmoveto{\pgfqpoint{3.854961in}{1.818905in}}%
\pgfpathlineto{\pgfqpoint{3.854961in}{1.818905in}}%
\pgfpathlineto{\pgfqpoint{3.854961in}{1.823163in}}%
\pgfpathlineto{\pgfqpoint{3.859219in}{1.823163in}}%
\pgfpathlineto{\pgfqpoint{3.859219in}{1.818905in}}%
\pgfpathmoveto{\pgfqpoint{3.854961in}{1.823163in}}%
\pgfpathlineto{\pgfqpoint{3.854961in}{1.823163in}}%
\pgfpathlineto{\pgfqpoint{3.854961in}{1.827420in}}%
\pgfpathlineto{\pgfqpoint{3.859219in}{1.827420in}}%
\pgfpathlineto{\pgfqpoint{3.859219in}{1.823163in}}%
\pgfpathmoveto{\pgfqpoint{3.854961in}{1.827420in}}%
\pgfpathlineto{\pgfqpoint{3.854961in}{1.827420in}}%
\pgfpathlineto{\pgfqpoint{3.854961in}{1.831678in}}%
\pgfpathlineto{\pgfqpoint{3.859219in}{1.831678in}}%
\pgfpathlineto{\pgfqpoint{3.859219in}{1.827420in}}%
\pgfpathmoveto{\pgfqpoint{3.854961in}{1.831678in}}%
\pgfpathlineto{\pgfqpoint{3.854961in}{1.831678in}}%
\pgfpathlineto{\pgfqpoint{3.854961in}{1.835935in}}%
\pgfpathlineto{\pgfqpoint{3.859219in}{1.835935in}}%
\pgfpathlineto{\pgfqpoint{3.859219in}{1.831678in}}%
\pgfpathmoveto{\pgfqpoint{3.859219in}{1.801875in}}%
\pgfpathlineto{\pgfqpoint{3.859219in}{1.801875in}}%
\pgfpathlineto{\pgfqpoint{3.859219in}{1.806132in}}%
\pgfpathlineto{\pgfqpoint{3.863477in}{1.806132in}}%
\pgfpathlineto{\pgfqpoint{3.863477in}{1.801875in}}%
\pgfpathmoveto{\pgfqpoint{3.859219in}{1.806132in}}%
\pgfpathlineto{\pgfqpoint{3.859219in}{1.806132in}}%
\pgfpathlineto{\pgfqpoint{3.859219in}{1.810390in}}%
\pgfpathlineto{\pgfqpoint{3.863477in}{1.810390in}}%
\pgfpathlineto{\pgfqpoint{3.863477in}{1.806132in}}%
\pgfpathmoveto{\pgfqpoint{3.863477in}{1.801875in}}%
\pgfpathlineto{\pgfqpoint{3.863477in}{1.801875in}}%
\pgfpathlineto{\pgfqpoint{3.863477in}{1.806132in}}%
\pgfpathlineto{\pgfqpoint{3.867735in}{1.806132in}}%
\pgfpathlineto{\pgfqpoint{3.867735in}{1.801875in}}%
\pgfpathmoveto{\pgfqpoint{3.863477in}{1.806132in}}%
\pgfpathlineto{\pgfqpoint{3.863477in}{1.806132in}}%
\pgfpathlineto{\pgfqpoint{3.863477in}{1.810390in}}%
\pgfpathlineto{\pgfqpoint{3.867735in}{1.810390in}}%
\pgfpathlineto{\pgfqpoint{3.867735in}{1.806132in}}%
\pgfpathmoveto{\pgfqpoint{3.859219in}{1.810390in}}%
\pgfpathlineto{\pgfqpoint{3.859219in}{1.810390in}}%
\pgfpathlineto{\pgfqpoint{3.859219in}{1.814647in}}%
\pgfpathlineto{\pgfqpoint{3.863477in}{1.814647in}}%
\pgfpathlineto{\pgfqpoint{3.863477in}{1.810390in}}%
\pgfpathmoveto{\pgfqpoint{3.859219in}{1.814647in}}%
\pgfpathlineto{\pgfqpoint{3.859219in}{1.814647in}}%
\pgfpathlineto{\pgfqpoint{3.859219in}{1.818905in}}%
\pgfpathlineto{\pgfqpoint{3.863477in}{1.818905in}}%
\pgfpathlineto{\pgfqpoint{3.863477in}{1.814647in}}%
\pgfpathmoveto{\pgfqpoint{3.863477in}{1.810390in}}%
\pgfpathlineto{\pgfqpoint{3.863477in}{1.810390in}}%
\pgfpathlineto{\pgfqpoint{3.863477in}{1.814647in}}%
\pgfpathlineto{\pgfqpoint{3.867735in}{1.814647in}}%
\pgfpathlineto{\pgfqpoint{3.867735in}{1.810390in}}%
\pgfpathmoveto{\pgfqpoint{3.863477in}{1.814647in}}%
\pgfpathlineto{\pgfqpoint{3.863477in}{1.814647in}}%
\pgfpathlineto{\pgfqpoint{3.863477in}{1.818905in}}%
\pgfpathlineto{\pgfqpoint{3.867735in}{1.818905in}}%
\pgfpathlineto{\pgfqpoint{3.867735in}{1.814647in}}%
\pgfpathmoveto{\pgfqpoint{3.859219in}{1.818905in}}%
\pgfpathlineto{\pgfqpoint{3.859219in}{1.818905in}}%
\pgfpathlineto{\pgfqpoint{3.859219in}{1.823163in}}%
\pgfpathlineto{\pgfqpoint{3.863477in}{1.823163in}}%
\pgfpathlineto{\pgfqpoint{3.863477in}{1.818905in}}%
\pgfpathmoveto{\pgfqpoint{3.859219in}{1.823163in}}%
\pgfpathlineto{\pgfqpoint{3.859219in}{1.823163in}}%
\pgfpathlineto{\pgfqpoint{3.859219in}{1.827420in}}%
\pgfpathlineto{\pgfqpoint{3.863477in}{1.827420in}}%
\pgfpathlineto{\pgfqpoint{3.863477in}{1.823163in}}%
\pgfpathmoveto{\pgfqpoint{3.863477in}{1.818905in}}%
\pgfpathlineto{\pgfqpoint{3.863477in}{1.818905in}}%
\pgfpathlineto{\pgfqpoint{3.863477in}{1.823163in}}%
\pgfpathlineto{\pgfqpoint{3.867735in}{1.823163in}}%
\pgfpathlineto{\pgfqpoint{3.867735in}{1.818905in}}%
\pgfpathmoveto{\pgfqpoint{3.859219in}{1.827420in}}%
\pgfpathlineto{\pgfqpoint{3.859219in}{1.827420in}}%
\pgfpathlineto{\pgfqpoint{3.859219in}{1.831678in}}%
\pgfpathlineto{\pgfqpoint{3.863477in}{1.831678in}}%
\pgfpathlineto{\pgfqpoint{3.863477in}{1.827420in}}%
\pgfpathmoveto{\pgfqpoint{3.859219in}{1.831678in}}%
\pgfpathlineto{\pgfqpoint{3.859219in}{1.831678in}}%
\pgfpathlineto{\pgfqpoint{3.859219in}{1.835935in}}%
\pgfpathlineto{\pgfqpoint{3.863477in}{1.835935in}}%
\pgfpathlineto{\pgfqpoint{3.863477in}{1.831678in}}%
\pgfpathmoveto{\pgfqpoint{3.854961in}{1.835935in}}%
\pgfpathlineto{\pgfqpoint{3.854961in}{1.835935in}}%
\pgfpathlineto{\pgfqpoint{3.854961in}{1.840193in}}%
\pgfpathlineto{\pgfqpoint{3.859219in}{1.840193in}}%
\pgfpathlineto{\pgfqpoint{3.859219in}{1.835935in}}%
\pgfpathmoveto{\pgfqpoint{3.854961in}{1.840193in}}%
\pgfpathlineto{\pgfqpoint{3.854961in}{1.840193in}}%
\pgfpathlineto{\pgfqpoint{3.854961in}{1.844451in}}%
\pgfpathlineto{\pgfqpoint{3.859219in}{1.844451in}}%
\pgfpathlineto{\pgfqpoint{3.859219in}{1.840193in}}%
\pgfpathmoveto{\pgfqpoint{3.850703in}{1.848708in}}%
\pgfpathlineto{\pgfqpoint{3.850703in}{1.848708in}}%
\pgfpathlineto{\pgfqpoint{3.850703in}{1.852966in}}%
\pgfpathlineto{\pgfqpoint{3.854961in}{1.852966in}}%
\pgfpathlineto{\pgfqpoint{3.854961in}{1.848708in}}%
\pgfpathmoveto{\pgfqpoint{3.854961in}{1.844451in}}%
\pgfpathlineto{\pgfqpoint{3.854961in}{1.844451in}}%
\pgfpathlineto{\pgfqpoint{3.854961in}{1.848708in}}%
\pgfpathlineto{\pgfqpoint{3.859219in}{1.848708in}}%
\pgfpathlineto{\pgfqpoint{3.859219in}{1.844451in}}%
\pgfpathmoveto{\pgfqpoint{3.854961in}{1.848708in}}%
\pgfpathlineto{\pgfqpoint{3.854961in}{1.848708in}}%
\pgfpathlineto{\pgfqpoint{3.854961in}{1.852966in}}%
\pgfpathlineto{\pgfqpoint{3.859219in}{1.852966in}}%
\pgfpathlineto{\pgfqpoint{3.859219in}{1.848708in}}%
\pgfpathmoveto{\pgfqpoint{3.850703in}{1.852966in}}%
\pgfpathlineto{\pgfqpoint{3.850703in}{1.852966in}}%
\pgfpathlineto{\pgfqpoint{3.850703in}{1.857223in}}%
\pgfpathlineto{\pgfqpoint{3.854961in}{1.857223in}}%
\pgfpathlineto{\pgfqpoint{3.854961in}{1.852966in}}%
\pgfpathmoveto{\pgfqpoint{3.850703in}{1.857223in}}%
\pgfpathlineto{\pgfqpoint{3.850703in}{1.857223in}}%
\pgfpathlineto{\pgfqpoint{3.850703in}{1.861481in}}%
\pgfpathlineto{\pgfqpoint{3.854961in}{1.861481in}}%
\pgfpathlineto{\pgfqpoint{3.854961in}{1.857223in}}%
\pgfpathmoveto{\pgfqpoint{3.854961in}{1.852966in}}%
\pgfpathlineto{\pgfqpoint{3.854961in}{1.852966in}}%
\pgfpathlineto{\pgfqpoint{3.854961in}{1.857223in}}%
\pgfpathlineto{\pgfqpoint{3.859219in}{1.857223in}}%
\pgfpathlineto{\pgfqpoint{3.859219in}{1.852966in}}%
\pgfpathmoveto{\pgfqpoint{3.854961in}{1.857223in}}%
\pgfpathlineto{\pgfqpoint{3.854961in}{1.857223in}}%
\pgfpathlineto{\pgfqpoint{3.854961in}{1.861481in}}%
\pgfpathlineto{\pgfqpoint{3.859219in}{1.861481in}}%
\pgfpathlineto{\pgfqpoint{3.859219in}{1.857223in}}%
\pgfpathmoveto{\pgfqpoint{3.850703in}{1.861481in}}%
\pgfpathlineto{\pgfqpoint{3.850703in}{1.861481in}}%
\pgfpathlineto{\pgfqpoint{3.850703in}{1.865738in}}%
\pgfpathlineto{\pgfqpoint{3.854961in}{1.865738in}}%
\pgfpathlineto{\pgfqpoint{3.854961in}{1.861481in}}%
\pgfpathmoveto{\pgfqpoint{3.850703in}{1.865738in}}%
\pgfpathlineto{\pgfqpoint{3.850703in}{1.865738in}}%
\pgfpathlineto{\pgfqpoint{3.850703in}{1.869996in}}%
\pgfpathlineto{\pgfqpoint{3.854961in}{1.869996in}}%
\pgfpathlineto{\pgfqpoint{3.854961in}{1.865738in}}%
\pgfpathmoveto{\pgfqpoint{3.854961in}{1.861481in}}%
\pgfpathlineto{\pgfqpoint{3.854961in}{1.861481in}}%
\pgfpathlineto{\pgfqpoint{3.854961in}{1.865738in}}%
\pgfpathlineto{\pgfqpoint{3.859219in}{1.865738in}}%
\pgfpathlineto{\pgfqpoint{3.859219in}{1.861481in}}%
\pgfpathmoveto{\pgfqpoint{3.854961in}{1.865738in}}%
\pgfpathlineto{\pgfqpoint{3.854961in}{1.865738in}}%
\pgfpathlineto{\pgfqpoint{3.854961in}{1.869996in}}%
\pgfpathlineto{\pgfqpoint{3.859219in}{1.869996in}}%
\pgfpathlineto{\pgfqpoint{3.859219in}{1.865738in}}%
\pgfpathmoveto{\pgfqpoint{3.859219in}{1.835935in}}%
\pgfpathlineto{\pgfqpoint{3.859219in}{1.835935in}}%
\pgfpathlineto{\pgfqpoint{3.859219in}{1.840193in}}%
\pgfpathlineto{\pgfqpoint{3.863477in}{1.840193in}}%
\pgfpathlineto{\pgfqpoint{3.863477in}{1.835935in}}%
\pgfpathmoveto{\pgfqpoint{3.859219in}{1.840193in}}%
\pgfpathlineto{\pgfqpoint{3.859219in}{1.840193in}}%
\pgfpathlineto{\pgfqpoint{3.859219in}{1.844451in}}%
\pgfpathlineto{\pgfqpoint{3.863477in}{1.844451in}}%
\pgfpathlineto{\pgfqpoint{3.863477in}{1.840193in}}%
\pgfpathmoveto{\pgfqpoint{3.859219in}{1.844451in}}%
\pgfpathlineto{\pgfqpoint{3.859219in}{1.844451in}}%
\pgfpathlineto{\pgfqpoint{3.859219in}{1.848708in}}%
\pgfpathlineto{\pgfqpoint{3.863477in}{1.848708in}}%
\pgfpathlineto{\pgfqpoint{3.863477in}{1.844451in}}%
\pgfpathmoveto{\pgfqpoint{3.859219in}{1.848708in}}%
\pgfpathlineto{\pgfqpoint{3.859219in}{1.848708in}}%
\pgfpathlineto{\pgfqpoint{3.859219in}{1.852966in}}%
\pgfpathlineto{\pgfqpoint{3.863477in}{1.852966in}}%
\pgfpathlineto{\pgfqpoint{3.863477in}{1.848708in}}%
\pgfpathmoveto{\pgfqpoint{3.859219in}{1.852966in}}%
\pgfpathlineto{\pgfqpoint{3.859219in}{1.852966in}}%
\pgfpathlineto{\pgfqpoint{3.859219in}{1.857223in}}%
\pgfpathlineto{\pgfqpoint{3.863477in}{1.857223in}}%
\pgfpathlineto{\pgfqpoint{3.863477in}{1.852966in}}%
\pgfpathmoveto{\pgfqpoint{3.837929in}{1.946637in}}%
\pgfpathlineto{\pgfqpoint{3.837929in}{1.946637in}}%
\pgfpathlineto{\pgfqpoint{3.837929in}{1.950895in}}%
\pgfpathlineto{\pgfqpoint{3.842187in}{1.950895in}}%
\pgfpathlineto{\pgfqpoint{3.842187in}{1.946637in}}%
\pgfpathmoveto{\pgfqpoint{3.837929in}{1.950895in}}%
\pgfpathlineto{\pgfqpoint{3.837929in}{1.950895in}}%
\pgfpathlineto{\pgfqpoint{3.837929in}{1.955152in}}%
\pgfpathlineto{\pgfqpoint{3.842187in}{1.955152in}}%
\pgfpathlineto{\pgfqpoint{3.842187in}{1.950895in}}%
\pgfpathmoveto{\pgfqpoint{3.837929in}{1.955152in}}%
\pgfpathlineto{\pgfqpoint{3.837929in}{1.955152in}}%
\pgfpathlineto{\pgfqpoint{3.837929in}{1.959410in}}%
\pgfpathlineto{\pgfqpoint{3.842187in}{1.959410in}}%
\pgfpathlineto{\pgfqpoint{3.842187in}{1.955152in}}%
\pgfpathmoveto{\pgfqpoint{3.837929in}{1.959410in}}%
\pgfpathlineto{\pgfqpoint{3.837929in}{1.959410in}}%
\pgfpathlineto{\pgfqpoint{3.837929in}{1.963668in}}%
\pgfpathlineto{\pgfqpoint{3.842187in}{1.963668in}}%
\pgfpathlineto{\pgfqpoint{3.842187in}{1.959410in}}%
\pgfpathmoveto{\pgfqpoint{3.837929in}{1.963668in}}%
\pgfpathlineto{\pgfqpoint{3.837929in}{1.963668in}}%
\pgfpathlineto{\pgfqpoint{3.837929in}{1.967926in}}%
\pgfpathlineto{\pgfqpoint{3.842187in}{1.967926in}}%
\pgfpathlineto{\pgfqpoint{3.842187in}{1.963668in}}%
\pgfpathmoveto{\pgfqpoint{3.837929in}{1.967926in}}%
\pgfpathlineto{\pgfqpoint{3.837929in}{1.967926in}}%
\pgfpathlineto{\pgfqpoint{3.837929in}{1.972184in}}%
\pgfpathlineto{\pgfqpoint{3.842187in}{1.972184in}}%
\pgfpathlineto{\pgfqpoint{3.842187in}{1.967926in}}%
\pgfpathmoveto{\pgfqpoint{3.833671in}{1.976441in}}%
\pgfpathlineto{\pgfqpoint{3.833671in}{1.976441in}}%
\pgfpathlineto{\pgfqpoint{3.833671in}{1.980699in}}%
\pgfpathlineto{\pgfqpoint{3.837929in}{1.980699in}}%
\pgfpathlineto{\pgfqpoint{3.837929in}{1.976441in}}%
\pgfpathmoveto{\pgfqpoint{3.837929in}{1.972184in}}%
\pgfpathlineto{\pgfqpoint{3.837929in}{1.972184in}}%
\pgfpathlineto{\pgfqpoint{3.837929in}{1.976441in}}%
\pgfpathlineto{\pgfqpoint{3.842187in}{1.976441in}}%
\pgfpathlineto{\pgfqpoint{3.842187in}{1.972184in}}%
\pgfpathmoveto{\pgfqpoint{3.837929in}{1.976441in}}%
\pgfpathlineto{\pgfqpoint{3.837929in}{1.976441in}}%
\pgfpathlineto{\pgfqpoint{3.837929in}{1.980699in}}%
\pgfpathlineto{\pgfqpoint{3.842187in}{1.980699in}}%
\pgfpathlineto{\pgfqpoint{3.842187in}{1.976441in}}%
\pgfpathmoveto{\pgfqpoint{3.833671in}{1.980699in}}%
\pgfpathlineto{\pgfqpoint{3.833671in}{1.980699in}}%
\pgfpathlineto{\pgfqpoint{3.833671in}{1.984957in}}%
\pgfpathlineto{\pgfqpoint{3.837929in}{1.984957in}}%
\pgfpathlineto{\pgfqpoint{3.837929in}{1.980699in}}%
\pgfpathmoveto{\pgfqpoint{3.833671in}{1.984957in}}%
\pgfpathlineto{\pgfqpoint{3.833671in}{1.984957in}}%
\pgfpathlineto{\pgfqpoint{3.833671in}{1.989215in}}%
\pgfpathlineto{\pgfqpoint{3.837929in}{1.989215in}}%
\pgfpathlineto{\pgfqpoint{3.837929in}{1.984957in}}%
\pgfpathmoveto{\pgfqpoint{3.837929in}{1.980699in}}%
\pgfpathlineto{\pgfqpoint{3.837929in}{1.980699in}}%
\pgfpathlineto{\pgfqpoint{3.837929in}{1.984957in}}%
\pgfpathlineto{\pgfqpoint{3.842187in}{1.984957in}}%
\pgfpathlineto{\pgfqpoint{3.842187in}{1.980699in}}%
\pgfpathmoveto{\pgfqpoint{3.837929in}{1.984957in}}%
\pgfpathlineto{\pgfqpoint{3.837929in}{1.984957in}}%
\pgfpathlineto{\pgfqpoint{3.837929in}{1.989215in}}%
\pgfpathlineto{\pgfqpoint{3.842187in}{1.989215in}}%
\pgfpathlineto{\pgfqpoint{3.842187in}{1.984957in}}%
\pgfpathmoveto{\pgfqpoint{3.833671in}{1.989215in}}%
\pgfpathlineto{\pgfqpoint{3.833671in}{1.989215in}}%
\pgfpathlineto{\pgfqpoint{3.833671in}{1.993473in}}%
\pgfpathlineto{\pgfqpoint{3.837929in}{1.993473in}}%
\pgfpathlineto{\pgfqpoint{3.837929in}{1.989215in}}%
\pgfpathmoveto{\pgfqpoint{3.833671in}{1.993473in}}%
\pgfpathlineto{\pgfqpoint{3.833671in}{1.993473in}}%
\pgfpathlineto{\pgfqpoint{3.833671in}{1.997731in}}%
\pgfpathlineto{\pgfqpoint{3.837929in}{1.997731in}}%
\pgfpathlineto{\pgfqpoint{3.837929in}{1.993473in}}%
\pgfpathmoveto{\pgfqpoint{3.837929in}{1.989215in}}%
\pgfpathlineto{\pgfqpoint{3.837929in}{1.989215in}}%
\pgfpathlineto{\pgfqpoint{3.837929in}{1.993473in}}%
\pgfpathlineto{\pgfqpoint{3.842187in}{1.993473in}}%
\pgfpathlineto{\pgfqpoint{3.842187in}{1.989215in}}%
\pgfpathmoveto{\pgfqpoint{3.837929in}{1.993473in}}%
\pgfpathlineto{\pgfqpoint{3.837929in}{1.993473in}}%
\pgfpathlineto{\pgfqpoint{3.837929in}{1.997731in}}%
\pgfpathlineto{\pgfqpoint{3.842187in}{1.997731in}}%
\pgfpathlineto{\pgfqpoint{3.842187in}{1.993473in}}%
\pgfpathmoveto{\pgfqpoint{3.833671in}{1.997731in}}%
\pgfpathlineto{\pgfqpoint{3.833671in}{1.997731in}}%
\pgfpathlineto{\pgfqpoint{3.833671in}{2.001988in}}%
\pgfpathlineto{\pgfqpoint{3.837929in}{2.001988in}}%
\pgfpathlineto{\pgfqpoint{3.837929in}{1.997731in}}%
\pgfpathmoveto{\pgfqpoint{3.833671in}{2.001988in}}%
\pgfpathlineto{\pgfqpoint{3.833671in}{2.001988in}}%
\pgfpathlineto{\pgfqpoint{3.833671in}{2.006246in}}%
\pgfpathlineto{\pgfqpoint{3.837929in}{2.006246in}}%
\pgfpathlineto{\pgfqpoint{3.837929in}{2.001988in}}%
\pgfpathmoveto{\pgfqpoint{3.837929in}{1.997731in}}%
\pgfpathlineto{\pgfqpoint{3.837929in}{1.997731in}}%
\pgfpathlineto{\pgfqpoint{3.837929in}{2.001988in}}%
\pgfpathlineto{\pgfqpoint{3.842187in}{2.001988in}}%
\pgfpathlineto{\pgfqpoint{3.842187in}{1.997731in}}%
\pgfpathmoveto{\pgfqpoint{3.837929in}{2.001988in}}%
\pgfpathlineto{\pgfqpoint{3.837929in}{2.001988in}}%
\pgfpathlineto{\pgfqpoint{3.837929in}{2.006246in}}%
\pgfpathlineto{\pgfqpoint{3.842187in}{2.006246in}}%
\pgfpathlineto{\pgfqpoint{3.842187in}{2.001988in}}%
\pgfpathmoveto{\pgfqpoint{3.846445in}{1.882769in}}%
\pgfpathlineto{\pgfqpoint{3.846445in}{1.882769in}}%
\pgfpathlineto{\pgfqpoint{3.846445in}{1.887027in}}%
\pgfpathlineto{\pgfqpoint{3.850703in}{1.887027in}}%
\pgfpathlineto{\pgfqpoint{3.850703in}{1.882769in}}%
\pgfpathmoveto{\pgfqpoint{3.850703in}{1.869996in}}%
\pgfpathlineto{\pgfqpoint{3.850703in}{1.869996in}}%
\pgfpathlineto{\pgfqpoint{3.850703in}{1.874254in}}%
\pgfpathlineto{\pgfqpoint{3.854961in}{1.874254in}}%
\pgfpathlineto{\pgfqpoint{3.854961in}{1.869996in}}%
\pgfpathmoveto{\pgfqpoint{3.850703in}{1.874254in}}%
\pgfpathlineto{\pgfqpoint{3.850703in}{1.874254in}}%
\pgfpathlineto{\pgfqpoint{3.850703in}{1.878512in}}%
\pgfpathlineto{\pgfqpoint{3.854961in}{1.878512in}}%
\pgfpathlineto{\pgfqpoint{3.854961in}{1.874254in}}%
\pgfpathmoveto{\pgfqpoint{3.854961in}{1.869996in}}%
\pgfpathlineto{\pgfqpoint{3.854961in}{1.869996in}}%
\pgfpathlineto{\pgfqpoint{3.854961in}{1.874254in}}%
\pgfpathlineto{\pgfqpoint{3.859219in}{1.874254in}}%
\pgfpathlineto{\pgfqpoint{3.859219in}{1.869996in}}%
\pgfpathmoveto{\pgfqpoint{3.854961in}{1.874254in}}%
\pgfpathlineto{\pgfqpoint{3.854961in}{1.874254in}}%
\pgfpathlineto{\pgfqpoint{3.854961in}{1.878512in}}%
\pgfpathlineto{\pgfqpoint{3.859219in}{1.878512in}}%
\pgfpathlineto{\pgfqpoint{3.859219in}{1.874254in}}%
\pgfpathmoveto{\pgfqpoint{3.850703in}{1.878512in}}%
\pgfpathlineto{\pgfqpoint{3.850703in}{1.878512in}}%
\pgfpathlineto{\pgfqpoint{3.850703in}{1.882769in}}%
\pgfpathlineto{\pgfqpoint{3.854961in}{1.882769in}}%
\pgfpathlineto{\pgfqpoint{3.854961in}{1.878512in}}%
\pgfpathmoveto{\pgfqpoint{3.850703in}{1.882769in}}%
\pgfpathlineto{\pgfqpoint{3.850703in}{1.882769in}}%
\pgfpathlineto{\pgfqpoint{3.850703in}{1.887027in}}%
\pgfpathlineto{\pgfqpoint{3.854961in}{1.887027in}}%
\pgfpathlineto{\pgfqpoint{3.854961in}{1.882769in}}%
\pgfpathmoveto{\pgfqpoint{3.854961in}{1.878512in}}%
\pgfpathlineto{\pgfqpoint{3.854961in}{1.878512in}}%
\pgfpathlineto{\pgfqpoint{3.854961in}{1.882769in}}%
\pgfpathlineto{\pgfqpoint{3.859219in}{1.882769in}}%
\pgfpathlineto{\pgfqpoint{3.859219in}{1.878512in}}%
\pgfpathmoveto{\pgfqpoint{3.854961in}{1.882769in}}%
\pgfpathlineto{\pgfqpoint{3.854961in}{1.882769in}}%
\pgfpathlineto{\pgfqpoint{3.854961in}{1.887027in}}%
\pgfpathlineto{\pgfqpoint{3.859219in}{1.887027in}}%
\pgfpathlineto{\pgfqpoint{3.859219in}{1.882769in}}%
\pgfpathmoveto{\pgfqpoint{3.846445in}{1.887027in}}%
\pgfpathlineto{\pgfqpoint{3.846445in}{1.887027in}}%
\pgfpathlineto{\pgfqpoint{3.846445in}{1.891285in}}%
\pgfpathlineto{\pgfqpoint{3.850703in}{1.891285in}}%
\pgfpathlineto{\pgfqpoint{3.850703in}{1.887027in}}%
\pgfpathmoveto{\pgfqpoint{3.846445in}{1.891285in}}%
\pgfpathlineto{\pgfqpoint{3.846445in}{1.891285in}}%
\pgfpathlineto{\pgfqpoint{3.846445in}{1.895543in}}%
\pgfpathlineto{\pgfqpoint{3.850703in}{1.895543in}}%
\pgfpathlineto{\pgfqpoint{3.850703in}{1.891285in}}%
\pgfpathmoveto{\pgfqpoint{3.846445in}{1.895543in}}%
\pgfpathlineto{\pgfqpoint{3.846445in}{1.895543in}}%
\pgfpathlineto{\pgfqpoint{3.846445in}{1.899801in}}%
\pgfpathlineto{\pgfqpoint{3.850703in}{1.899801in}}%
\pgfpathlineto{\pgfqpoint{3.850703in}{1.895543in}}%
\pgfpathmoveto{\pgfqpoint{3.846445in}{1.899801in}}%
\pgfpathlineto{\pgfqpoint{3.846445in}{1.899801in}}%
\pgfpathlineto{\pgfqpoint{3.846445in}{1.904059in}}%
\pgfpathlineto{\pgfqpoint{3.850703in}{1.904059in}}%
\pgfpathlineto{\pgfqpoint{3.850703in}{1.899801in}}%
\pgfpathmoveto{\pgfqpoint{3.850703in}{1.887027in}}%
\pgfpathlineto{\pgfqpoint{3.850703in}{1.887027in}}%
\pgfpathlineto{\pgfqpoint{3.850703in}{1.891285in}}%
\pgfpathlineto{\pgfqpoint{3.854961in}{1.891285in}}%
\pgfpathlineto{\pgfqpoint{3.854961in}{1.887027in}}%
\pgfpathmoveto{\pgfqpoint{3.850703in}{1.891285in}}%
\pgfpathlineto{\pgfqpoint{3.850703in}{1.891285in}}%
\pgfpathlineto{\pgfqpoint{3.850703in}{1.895543in}}%
\pgfpathlineto{\pgfqpoint{3.854961in}{1.895543in}}%
\pgfpathlineto{\pgfqpoint{3.854961in}{1.891285in}}%
\pgfpathmoveto{\pgfqpoint{3.850703in}{1.895543in}}%
\pgfpathlineto{\pgfqpoint{3.850703in}{1.895543in}}%
\pgfpathlineto{\pgfqpoint{3.850703in}{1.899801in}}%
\pgfpathlineto{\pgfqpoint{3.854961in}{1.899801in}}%
\pgfpathlineto{\pgfqpoint{3.854961in}{1.895543in}}%
\pgfpathmoveto{\pgfqpoint{3.850703in}{1.899801in}}%
\pgfpathlineto{\pgfqpoint{3.850703in}{1.899801in}}%
\pgfpathlineto{\pgfqpoint{3.850703in}{1.904059in}}%
\pgfpathlineto{\pgfqpoint{3.854961in}{1.904059in}}%
\pgfpathlineto{\pgfqpoint{3.854961in}{1.899801in}}%
\pgfpathmoveto{\pgfqpoint{3.846445in}{1.904059in}}%
\pgfpathlineto{\pgfqpoint{3.846445in}{1.904059in}}%
\pgfpathlineto{\pgfqpoint{3.846445in}{1.908316in}}%
\pgfpathlineto{\pgfqpoint{3.850703in}{1.908316in}}%
\pgfpathlineto{\pgfqpoint{3.850703in}{1.904059in}}%
\pgfpathmoveto{\pgfqpoint{3.846445in}{1.908316in}}%
\pgfpathlineto{\pgfqpoint{3.846445in}{1.908316in}}%
\pgfpathlineto{\pgfqpoint{3.846445in}{1.912574in}}%
\pgfpathlineto{\pgfqpoint{3.850703in}{1.912574in}}%
\pgfpathlineto{\pgfqpoint{3.850703in}{1.908316in}}%
\pgfpathmoveto{\pgfqpoint{3.842187in}{1.912574in}}%
\pgfpathlineto{\pgfqpoint{3.842187in}{1.912574in}}%
\pgfpathlineto{\pgfqpoint{3.842187in}{1.916832in}}%
\pgfpathlineto{\pgfqpoint{3.846445in}{1.916832in}}%
\pgfpathlineto{\pgfqpoint{3.846445in}{1.912574in}}%
\pgfpathmoveto{\pgfqpoint{3.842187in}{1.916832in}}%
\pgfpathlineto{\pgfqpoint{3.842187in}{1.916832in}}%
\pgfpathlineto{\pgfqpoint{3.842187in}{1.921090in}}%
\pgfpathlineto{\pgfqpoint{3.846445in}{1.921090in}}%
\pgfpathlineto{\pgfqpoint{3.846445in}{1.916832in}}%
\pgfpathmoveto{\pgfqpoint{3.846445in}{1.912574in}}%
\pgfpathlineto{\pgfqpoint{3.846445in}{1.912574in}}%
\pgfpathlineto{\pgfqpoint{3.846445in}{1.916832in}}%
\pgfpathlineto{\pgfqpoint{3.850703in}{1.916832in}}%
\pgfpathlineto{\pgfqpoint{3.850703in}{1.912574in}}%
\pgfpathmoveto{\pgfqpoint{3.846445in}{1.916832in}}%
\pgfpathlineto{\pgfqpoint{3.846445in}{1.916832in}}%
\pgfpathlineto{\pgfqpoint{3.846445in}{1.921090in}}%
\pgfpathlineto{\pgfqpoint{3.850703in}{1.921090in}}%
\pgfpathlineto{\pgfqpoint{3.850703in}{1.916832in}}%
\pgfpathmoveto{\pgfqpoint{3.850703in}{1.904059in}}%
\pgfpathlineto{\pgfqpoint{3.850703in}{1.904059in}}%
\pgfpathlineto{\pgfqpoint{3.850703in}{1.908316in}}%
\pgfpathlineto{\pgfqpoint{3.854961in}{1.908316in}}%
\pgfpathlineto{\pgfqpoint{3.854961in}{1.904059in}}%
\pgfpathmoveto{\pgfqpoint{3.850703in}{1.908316in}}%
\pgfpathlineto{\pgfqpoint{3.850703in}{1.908316in}}%
\pgfpathlineto{\pgfqpoint{3.850703in}{1.912574in}}%
\pgfpathlineto{\pgfqpoint{3.854961in}{1.912574in}}%
\pgfpathlineto{\pgfqpoint{3.854961in}{1.908316in}}%
\pgfpathmoveto{\pgfqpoint{3.850703in}{1.912574in}}%
\pgfpathlineto{\pgfqpoint{3.850703in}{1.912574in}}%
\pgfpathlineto{\pgfqpoint{3.850703in}{1.916832in}}%
\pgfpathlineto{\pgfqpoint{3.854961in}{1.916832in}}%
\pgfpathlineto{\pgfqpoint{3.854961in}{1.912574in}}%
\pgfpathmoveto{\pgfqpoint{3.850703in}{1.916832in}}%
\pgfpathlineto{\pgfqpoint{3.850703in}{1.916832in}}%
\pgfpathlineto{\pgfqpoint{3.850703in}{1.921090in}}%
\pgfpathlineto{\pgfqpoint{3.854961in}{1.921090in}}%
\pgfpathlineto{\pgfqpoint{3.854961in}{1.916832in}}%
\pgfpathmoveto{\pgfqpoint{3.842187in}{1.921090in}}%
\pgfpathlineto{\pgfqpoint{3.842187in}{1.921090in}}%
\pgfpathlineto{\pgfqpoint{3.842187in}{1.925348in}}%
\pgfpathlineto{\pgfqpoint{3.846445in}{1.925348in}}%
\pgfpathlineto{\pgfqpoint{3.846445in}{1.921090in}}%
\pgfpathmoveto{\pgfqpoint{3.842187in}{1.925348in}}%
\pgfpathlineto{\pgfqpoint{3.842187in}{1.925348in}}%
\pgfpathlineto{\pgfqpoint{3.842187in}{1.929605in}}%
\pgfpathlineto{\pgfqpoint{3.846445in}{1.929605in}}%
\pgfpathlineto{\pgfqpoint{3.846445in}{1.925348in}}%
\pgfpathmoveto{\pgfqpoint{3.846445in}{1.921090in}}%
\pgfpathlineto{\pgfqpoint{3.846445in}{1.921090in}}%
\pgfpathlineto{\pgfqpoint{3.846445in}{1.925348in}}%
\pgfpathlineto{\pgfqpoint{3.850703in}{1.925348in}}%
\pgfpathlineto{\pgfqpoint{3.850703in}{1.921090in}}%
\pgfpathmoveto{\pgfqpoint{3.846445in}{1.925348in}}%
\pgfpathlineto{\pgfqpoint{3.846445in}{1.925348in}}%
\pgfpathlineto{\pgfqpoint{3.846445in}{1.929605in}}%
\pgfpathlineto{\pgfqpoint{3.850703in}{1.929605in}}%
\pgfpathlineto{\pgfqpoint{3.850703in}{1.925348in}}%
\pgfpathmoveto{\pgfqpoint{3.842187in}{1.929605in}}%
\pgfpathlineto{\pgfqpoint{3.842187in}{1.929605in}}%
\pgfpathlineto{\pgfqpoint{3.842187in}{1.933863in}}%
\pgfpathlineto{\pgfqpoint{3.846445in}{1.933863in}}%
\pgfpathlineto{\pgfqpoint{3.846445in}{1.929605in}}%
\pgfpathmoveto{\pgfqpoint{3.842187in}{1.933863in}}%
\pgfpathlineto{\pgfqpoint{3.842187in}{1.933863in}}%
\pgfpathlineto{\pgfqpoint{3.842187in}{1.938121in}}%
\pgfpathlineto{\pgfqpoint{3.846445in}{1.938121in}}%
\pgfpathlineto{\pgfqpoint{3.846445in}{1.933863in}}%
\pgfpathmoveto{\pgfqpoint{3.846445in}{1.929605in}}%
\pgfpathlineto{\pgfqpoint{3.846445in}{1.929605in}}%
\pgfpathlineto{\pgfqpoint{3.846445in}{1.933863in}}%
\pgfpathlineto{\pgfqpoint{3.850703in}{1.933863in}}%
\pgfpathlineto{\pgfqpoint{3.850703in}{1.929605in}}%
\pgfpathmoveto{\pgfqpoint{3.846445in}{1.933863in}}%
\pgfpathlineto{\pgfqpoint{3.846445in}{1.933863in}}%
\pgfpathlineto{\pgfqpoint{3.846445in}{1.938121in}}%
\pgfpathlineto{\pgfqpoint{3.850703in}{1.938121in}}%
\pgfpathlineto{\pgfqpoint{3.850703in}{1.933863in}}%
\pgfpathmoveto{\pgfqpoint{3.842187in}{1.938121in}}%
\pgfpathlineto{\pgfqpoint{3.842187in}{1.938121in}}%
\pgfpathlineto{\pgfqpoint{3.842187in}{1.942379in}}%
\pgfpathlineto{\pgfqpoint{3.846445in}{1.942379in}}%
\pgfpathlineto{\pgfqpoint{3.846445in}{1.938121in}}%
\pgfpathmoveto{\pgfqpoint{3.842187in}{1.942379in}}%
\pgfpathlineto{\pgfqpoint{3.842187in}{1.942379in}}%
\pgfpathlineto{\pgfqpoint{3.842187in}{1.946637in}}%
\pgfpathlineto{\pgfqpoint{3.846445in}{1.946637in}}%
\pgfpathlineto{\pgfqpoint{3.846445in}{1.942379in}}%
\pgfpathmoveto{\pgfqpoint{3.846445in}{1.938121in}}%
\pgfpathlineto{\pgfqpoint{3.846445in}{1.938121in}}%
\pgfpathlineto{\pgfqpoint{3.846445in}{1.942379in}}%
\pgfpathlineto{\pgfqpoint{3.850703in}{1.942379in}}%
\pgfpathlineto{\pgfqpoint{3.850703in}{1.938121in}}%
\pgfpathmoveto{\pgfqpoint{3.846445in}{1.942379in}}%
\pgfpathlineto{\pgfqpoint{3.846445in}{1.942379in}}%
\pgfpathlineto{\pgfqpoint{3.846445in}{1.946637in}}%
\pgfpathlineto{\pgfqpoint{3.850703in}{1.946637in}}%
\pgfpathlineto{\pgfqpoint{3.850703in}{1.942379in}}%
\pgfpathmoveto{\pgfqpoint{3.842187in}{1.946637in}}%
\pgfpathlineto{\pgfqpoint{3.842187in}{1.946637in}}%
\pgfpathlineto{\pgfqpoint{3.842187in}{1.950895in}}%
\pgfpathlineto{\pgfqpoint{3.846445in}{1.950895in}}%
\pgfpathlineto{\pgfqpoint{3.846445in}{1.946637in}}%
\pgfpathmoveto{\pgfqpoint{3.842187in}{1.950895in}}%
\pgfpathlineto{\pgfqpoint{3.842187in}{1.950895in}}%
\pgfpathlineto{\pgfqpoint{3.842187in}{1.955152in}}%
\pgfpathlineto{\pgfqpoint{3.846445in}{1.955152in}}%
\pgfpathlineto{\pgfqpoint{3.846445in}{1.950895in}}%
\pgfpathmoveto{\pgfqpoint{3.846445in}{1.946637in}}%
\pgfpathlineto{\pgfqpoint{3.846445in}{1.946637in}}%
\pgfpathlineto{\pgfqpoint{3.846445in}{1.950895in}}%
\pgfpathlineto{\pgfqpoint{3.850703in}{1.950895in}}%
\pgfpathlineto{\pgfqpoint{3.850703in}{1.946637in}}%
\pgfpathmoveto{\pgfqpoint{3.842187in}{1.955152in}}%
\pgfpathlineto{\pgfqpoint{3.842187in}{1.955152in}}%
\pgfpathlineto{\pgfqpoint{3.842187in}{1.959410in}}%
\pgfpathlineto{\pgfqpoint{3.846445in}{1.959410in}}%
\pgfpathlineto{\pgfqpoint{3.846445in}{1.955152in}}%
\pgfpathmoveto{\pgfqpoint{3.842187in}{1.959410in}}%
\pgfpathlineto{\pgfqpoint{3.842187in}{1.959410in}}%
\pgfpathlineto{\pgfqpoint{3.842187in}{1.963668in}}%
\pgfpathlineto{\pgfqpoint{3.846445in}{1.963668in}}%
\pgfpathlineto{\pgfqpoint{3.846445in}{1.959410in}}%
\pgfpathmoveto{\pgfqpoint{3.842187in}{1.963668in}}%
\pgfpathlineto{\pgfqpoint{3.842187in}{1.963668in}}%
\pgfpathlineto{\pgfqpoint{3.842187in}{1.967926in}}%
\pgfpathlineto{\pgfqpoint{3.846445in}{1.967926in}}%
\pgfpathlineto{\pgfqpoint{3.846445in}{1.963668in}}%
\pgfpathmoveto{\pgfqpoint{3.842187in}{1.967926in}}%
\pgfpathlineto{\pgfqpoint{3.842187in}{1.967926in}}%
\pgfpathlineto{\pgfqpoint{3.842187in}{1.972184in}}%
\pgfpathlineto{\pgfqpoint{3.846445in}{1.972184in}}%
\pgfpathlineto{\pgfqpoint{3.846445in}{1.967926in}}%
\pgfpathmoveto{\pgfqpoint{3.842187in}{1.972184in}}%
\pgfpathlineto{\pgfqpoint{3.842187in}{1.972184in}}%
\pgfpathlineto{\pgfqpoint{3.842187in}{1.976441in}}%
\pgfpathlineto{\pgfqpoint{3.846445in}{1.976441in}}%
\pgfpathlineto{\pgfqpoint{3.846445in}{1.972184in}}%
\pgfpathmoveto{\pgfqpoint{3.842187in}{1.976441in}}%
\pgfpathlineto{\pgfqpoint{3.842187in}{1.976441in}}%
\pgfpathlineto{\pgfqpoint{3.842187in}{1.980699in}}%
\pgfpathlineto{\pgfqpoint{3.846445in}{1.980699in}}%
\pgfpathlineto{\pgfqpoint{3.846445in}{1.976441in}}%
\pgfpathmoveto{\pgfqpoint{3.842187in}{1.980699in}}%
\pgfpathlineto{\pgfqpoint{3.842187in}{1.980699in}}%
\pgfpathlineto{\pgfqpoint{3.842187in}{1.984957in}}%
\pgfpathlineto{\pgfqpoint{3.846445in}{1.984957in}}%
\pgfpathlineto{\pgfqpoint{3.846445in}{1.980699in}}%
\pgfpathmoveto{\pgfqpoint{3.829413in}{2.006246in}}%
\pgfpathlineto{\pgfqpoint{3.829413in}{2.006246in}}%
\pgfpathlineto{\pgfqpoint{3.829413in}{2.010504in}}%
\pgfpathlineto{\pgfqpoint{3.833671in}{2.010504in}}%
\pgfpathlineto{\pgfqpoint{3.833671in}{2.006246in}}%
\pgfpathmoveto{\pgfqpoint{3.829413in}{2.010504in}}%
\pgfpathlineto{\pgfqpoint{3.829413in}{2.010504in}}%
\pgfpathlineto{\pgfqpoint{3.829413in}{2.014762in}}%
\pgfpathlineto{\pgfqpoint{3.833671in}{2.014762in}}%
\pgfpathlineto{\pgfqpoint{3.833671in}{2.010504in}}%
\pgfpathmoveto{\pgfqpoint{3.829413in}{2.014762in}}%
\pgfpathlineto{\pgfqpoint{3.829413in}{2.014762in}}%
\pgfpathlineto{\pgfqpoint{3.829413in}{2.019020in}}%
\pgfpathlineto{\pgfqpoint{3.833671in}{2.019020in}}%
\pgfpathlineto{\pgfqpoint{3.833671in}{2.014762in}}%
\pgfpathmoveto{\pgfqpoint{3.829413in}{2.019020in}}%
\pgfpathlineto{\pgfqpoint{3.829413in}{2.019020in}}%
\pgfpathlineto{\pgfqpoint{3.829413in}{2.023278in}}%
\pgfpathlineto{\pgfqpoint{3.833671in}{2.023278in}}%
\pgfpathlineto{\pgfqpoint{3.833671in}{2.019020in}}%
\pgfpathmoveto{\pgfqpoint{3.833671in}{2.006246in}}%
\pgfpathlineto{\pgfqpoint{3.833671in}{2.006246in}}%
\pgfpathlineto{\pgfqpoint{3.833671in}{2.010504in}}%
\pgfpathlineto{\pgfqpoint{3.837929in}{2.010504in}}%
\pgfpathlineto{\pgfqpoint{3.837929in}{2.006246in}}%
\pgfpathmoveto{\pgfqpoint{3.833671in}{2.010504in}}%
\pgfpathlineto{\pgfqpoint{3.833671in}{2.010504in}}%
\pgfpathlineto{\pgfqpoint{3.833671in}{2.014762in}}%
\pgfpathlineto{\pgfqpoint{3.837929in}{2.014762in}}%
\pgfpathlineto{\pgfqpoint{3.837929in}{2.010504in}}%
\pgfpathmoveto{\pgfqpoint{3.837929in}{2.006246in}}%
\pgfpathlineto{\pgfqpoint{3.837929in}{2.006246in}}%
\pgfpathlineto{\pgfqpoint{3.837929in}{2.010504in}}%
\pgfpathlineto{\pgfqpoint{3.842187in}{2.010504in}}%
\pgfpathlineto{\pgfqpoint{3.842187in}{2.006246in}}%
\pgfpathmoveto{\pgfqpoint{3.837929in}{2.010504in}}%
\pgfpathlineto{\pgfqpoint{3.837929in}{2.010504in}}%
\pgfpathlineto{\pgfqpoint{3.837929in}{2.014762in}}%
\pgfpathlineto{\pgfqpoint{3.842187in}{2.014762in}}%
\pgfpathlineto{\pgfqpoint{3.842187in}{2.010504in}}%
\pgfpathmoveto{\pgfqpoint{3.833671in}{2.014762in}}%
\pgfpathlineto{\pgfqpoint{3.833671in}{2.014762in}}%
\pgfpathlineto{\pgfqpoint{3.833671in}{2.019020in}}%
\pgfpathlineto{\pgfqpoint{3.837929in}{2.019020in}}%
\pgfpathlineto{\pgfqpoint{3.837929in}{2.014762in}}%
\pgfpathmoveto{\pgfqpoint{3.833671in}{2.019020in}}%
\pgfpathlineto{\pgfqpoint{3.833671in}{2.019020in}}%
\pgfpathlineto{\pgfqpoint{3.833671in}{2.023278in}}%
\pgfpathlineto{\pgfqpoint{3.837929in}{2.023278in}}%
\pgfpathlineto{\pgfqpoint{3.837929in}{2.019020in}}%
\pgfpathmoveto{\pgfqpoint{3.829413in}{2.023278in}}%
\pgfpathlineto{\pgfqpoint{3.829413in}{2.023278in}}%
\pgfpathlineto{\pgfqpoint{3.829413in}{2.027536in}}%
\pgfpathlineto{\pgfqpoint{3.833671in}{2.027536in}}%
\pgfpathlineto{\pgfqpoint{3.833671in}{2.023278in}}%
\pgfpathmoveto{\pgfqpoint{3.829413in}{2.027536in}}%
\pgfpathlineto{\pgfqpoint{3.829413in}{2.027536in}}%
\pgfpathlineto{\pgfqpoint{3.829413in}{2.031793in}}%
\pgfpathlineto{\pgfqpoint{3.833671in}{2.031793in}}%
\pgfpathlineto{\pgfqpoint{3.833671in}{2.027536in}}%
\pgfpathmoveto{\pgfqpoint{3.825156in}{2.036051in}}%
\pgfpathlineto{\pgfqpoint{3.825156in}{2.036051in}}%
\pgfpathlineto{\pgfqpoint{3.825156in}{2.040309in}}%
\pgfpathlineto{\pgfqpoint{3.829413in}{2.040309in}}%
\pgfpathlineto{\pgfqpoint{3.829413in}{2.036051in}}%
\pgfpathmoveto{\pgfqpoint{3.829413in}{2.031793in}}%
\pgfpathlineto{\pgfqpoint{3.829413in}{2.031793in}}%
\pgfpathlineto{\pgfqpoint{3.829413in}{2.036051in}}%
\pgfpathlineto{\pgfqpoint{3.833671in}{2.036051in}}%
\pgfpathlineto{\pgfqpoint{3.833671in}{2.031793in}}%
\pgfpathmoveto{\pgfqpoint{3.829413in}{2.036051in}}%
\pgfpathlineto{\pgfqpoint{3.829413in}{2.036051in}}%
\pgfpathlineto{\pgfqpoint{3.829413in}{2.040309in}}%
\pgfpathlineto{\pgfqpoint{3.833671in}{2.040309in}}%
\pgfpathlineto{\pgfqpoint{3.833671in}{2.036051in}}%
\pgfpathmoveto{\pgfqpoint{3.833671in}{2.023278in}}%
\pgfpathlineto{\pgfqpoint{3.833671in}{2.023278in}}%
\pgfpathlineto{\pgfqpoint{3.833671in}{2.027536in}}%
\pgfpathlineto{\pgfqpoint{3.837929in}{2.027536in}}%
\pgfpathlineto{\pgfqpoint{3.837929in}{2.023278in}}%
\pgfpathmoveto{\pgfqpoint{3.833671in}{2.027536in}}%
\pgfpathlineto{\pgfqpoint{3.833671in}{2.027536in}}%
\pgfpathlineto{\pgfqpoint{3.833671in}{2.031793in}}%
\pgfpathlineto{\pgfqpoint{3.837929in}{2.031793in}}%
\pgfpathlineto{\pgfqpoint{3.837929in}{2.027536in}}%
\pgfpathmoveto{\pgfqpoint{3.833671in}{2.031793in}}%
\pgfpathlineto{\pgfqpoint{3.833671in}{2.031793in}}%
\pgfpathlineto{\pgfqpoint{3.833671in}{2.036051in}}%
\pgfpathlineto{\pgfqpoint{3.837929in}{2.036051in}}%
\pgfpathlineto{\pgfqpoint{3.837929in}{2.031793in}}%
\pgfpathmoveto{\pgfqpoint{3.833671in}{2.036051in}}%
\pgfpathlineto{\pgfqpoint{3.833671in}{2.036051in}}%
\pgfpathlineto{\pgfqpoint{3.833671in}{2.040309in}}%
\pgfpathlineto{\pgfqpoint{3.837929in}{2.040309in}}%
\pgfpathlineto{\pgfqpoint{3.837929in}{2.036051in}}%
\pgfpathmoveto{\pgfqpoint{3.820898in}{2.065856in}}%
\pgfpathlineto{\pgfqpoint{3.820898in}{2.065856in}}%
\pgfpathlineto{\pgfqpoint{3.820898in}{2.070114in}}%
\pgfpathlineto{\pgfqpoint{3.825156in}{2.070114in}}%
\pgfpathlineto{\pgfqpoint{3.825156in}{2.065856in}}%
\pgfpathmoveto{\pgfqpoint{3.820898in}{2.070114in}}%
\pgfpathlineto{\pgfqpoint{3.820898in}{2.070114in}}%
\pgfpathlineto{\pgfqpoint{3.820898in}{2.074372in}}%
\pgfpathlineto{\pgfqpoint{3.825156in}{2.074372in}}%
\pgfpathlineto{\pgfqpoint{3.825156in}{2.070114in}}%
\pgfpathmoveto{\pgfqpoint{3.825156in}{2.040309in}}%
\pgfpathlineto{\pgfqpoint{3.825156in}{2.040309in}}%
\pgfpathlineto{\pgfqpoint{3.825156in}{2.044567in}}%
\pgfpathlineto{\pgfqpoint{3.829413in}{2.044567in}}%
\pgfpathlineto{\pgfqpoint{3.829413in}{2.040309in}}%
\pgfpathmoveto{\pgfqpoint{3.825156in}{2.044567in}}%
\pgfpathlineto{\pgfqpoint{3.825156in}{2.044567in}}%
\pgfpathlineto{\pgfqpoint{3.825156in}{2.048825in}}%
\pgfpathlineto{\pgfqpoint{3.829413in}{2.048825in}}%
\pgfpathlineto{\pgfqpoint{3.829413in}{2.044567in}}%
\pgfpathmoveto{\pgfqpoint{3.829413in}{2.040309in}}%
\pgfpathlineto{\pgfqpoint{3.829413in}{2.040309in}}%
\pgfpathlineto{\pgfqpoint{3.829413in}{2.044567in}}%
\pgfpathlineto{\pgfqpoint{3.833671in}{2.044567in}}%
\pgfpathlineto{\pgfqpoint{3.833671in}{2.040309in}}%
\pgfpathmoveto{\pgfqpoint{3.829413in}{2.044567in}}%
\pgfpathlineto{\pgfqpoint{3.829413in}{2.044567in}}%
\pgfpathlineto{\pgfqpoint{3.829413in}{2.048825in}}%
\pgfpathlineto{\pgfqpoint{3.833671in}{2.048825in}}%
\pgfpathlineto{\pgfqpoint{3.833671in}{2.044567in}}%
\pgfpathmoveto{\pgfqpoint{3.825156in}{2.048825in}}%
\pgfpathlineto{\pgfqpoint{3.825156in}{2.048825in}}%
\pgfpathlineto{\pgfqpoint{3.825156in}{2.053083in}}%
\pgfpathlineto{\pgfqpoint{3.829413in}{2.053083in}}%
\pgfpathlineto{\pgfqpoint{3.829413in}{2.048825in}}%
\pgfpathmoveto{\pgfqpoint{3.825156in}{2.053083in}}%
\pgfpathlineto{\pgfqpoint{3.825156in}{2.053083in}}%
\pgfpathlineto{\pgfqpoint{3.825156in}{2.057341in}}%
\pgfpathlineto{\pgfqpoint{3.829413in}{2.057341in}}%
\pgfpathlineto{\pgfqpoint{3.829413in}{2.053083in}}%
\pgfpathmoveto{\pgfqpoint{3.829413in}{2.048825in}}%
\pgfpathlineto{\pgfqpoint{3.829413in}{2.048825in}}%
\pgfpathlineto{\pgfqpoint{3.829413in}{2.053083in}}%
\pgfpathlineto{\pgfqpoint{3.833671in}{2.053083in}}%
\pgfpathlineto{\pgfqpoint{3.833671in}{2.048825in}}%
\pgfpathmoveto{\pgfqpoint{3.829413in}{2.053083in}}%
\pgfpathlineto{\pgfqpoint{3.829413in}{2.053083in}}%
\pgfpathlineto{\pgfqpoint{3.829413in}{2.057341in}}%
\pgfpathlineto{\pgfqpoint{3.833671in}{2.057341in}}%
\pgfpathlineto{\pgfqpoint{3.833671in}{2.053083in}}%
\pgfpathmoveto{\pgfqpoint{3.833671in}{2.040309in}}%
\pgfpathlineto{\pgfqpoint{3.833671in}{2.040309in}}%
\pgfpathlineto{\pgfqpoint{3.833671in}{2.044567in}}%
\pgfpathlineto{\pgfqpoint{3.837929in}{2.044567in}}%
\pgfpathlineto{\pgfqpoint{3.837929in}{2.040309in}}%
\pgfpathmoveto{\pgfqpoint{3.825156in}{2.057341in}}%
\pgfpathlineto{\pgfqpoint{3.825156in}{2.057341in}}%
\pgfpathlineto{\pgfqpoint{3.825156in}{2.061599in}}%
\pgfpathlineto{\pgfqpoint{3.829413in}{2.061599in}}%
\pgfpathlineto{\pgfqpoint{3.829413in}{2.057341in}}%
\pgfpathmoveto{\pgfqpoint{3.825156in}{2.061599in}}%
\pgfpathlineto{\pgfqpoint{3.825156in}{2.061599in}}%
\pgfpathlineto{\pgfqpoint{3.825156in}{2.065856in}}%
\pgfpathlineto{\pgfqpoint{3.829413in}{2.065856in}}%
\pgfpathlineto{\pgfqpoint{3.829413in}{2.061599in}}%
\pgfpathmoveto{\pgfqpoint{3.829413in}{2.057341in}}%
\pgfpathlineto{\pgfqpoint{3.829413in}{2.057341in}}%
\pgfpathlineto{\pgfqpoint{3.829413in}{2.061599in}}%
\pgfpathlineto{\pgfqpoint{3.833671in}{2.061599in}}%
\pgfpathlineto{\pgfqpoint{3.833671in}{2.057341in}}%
\pgfpathmoveto{\pgfqpoint{3.829413in}{2.061599in}}%
\pgfpathlineto{\pgfqpoint{3.829413in}{2.061599in}}%
\pgfpathlineto{\pgfqpoint{3.829413in}{2.065856in}}%
\pgfpathlineto{\pgfqpoint{3.833671in}{2.065856in}}%
\pgfpathlineto{\pgfqpoint{3.833671in}{2.061599in}}%
\pgfpathmoveto{\pgfqpoint{3.825156in}{2.065856in}}%
\pgfpathlineto{\pgfqpoint{3.825156in}{2.065856in}}%
\pgfpathlineto{\pgfqpoint{3.825156in}{2.070114in}}%
\pgfpathlineto{\pgfqpoint{3.829413in}{2.070114in}}%
\pgfpathlineto{\pgfqpoint{3.829413in}{2.065856in}}%
\pgfpathmoveto{\pgfqpoint{3.825156in}{2.070114in}}%
\pgfpathlineto{\pgfqpoint{3.825156in}{2.070114in}}%
\pgfpathlineto{\pgfqpoint{3.825156in}{2.074372in}}%
\pgfpathlineto{\pgfqpoint{3.829413in}{2.074372in}}%
\pgfpathlineto{\pgfqpoint{3.829413in}{2.070114in}}%
\pgfpathmoveto{\pgfqpoint{3.829413in}{2.065856in}}%
\pgfpathlineto{\pgfqpoint{3.829413in}{2.065856in}}%
\pgfpathlineto{\pgfqpoint{3.829413in}{2.070114in}}%
\pgfpathlineto{\pgfqpoint{3.833671in}{2.070114in}}%
\pgfpathlineto{\pgfqpoint{3.833671in}{2.065856in}}%
\pgfpathmoveto{\pgfqpoint{3.829413in}{2.070114in}}%
\pgfpathlineto{\pgfqpoint{3.829413in}{2.070114in}}%
\pgfpathlineto{\pgfqpoint{3.829413in}{2.074372in}}%
\pgfpathlineto{\pgfqpoint{3.833671in}{2.074372in}}%
\pgfpathlineto{\pgfqpoint{3.833671in}{2.070114in}}%
\pgfpathmoveto{\pgfqpoint{3.820898in}{2.074372in}}%
\pgfpathlineto{\pgfqpoint{3.820898in}{2.074372in}}%
\pgfpathlineto{\pgfqpoint{3.820898in}{2.078630in}}%
\pgfpathlineto{\pgfqpoint{3.825156in}{2.078630in}}%
\pgfpathlineto{\pgfqpoint{3.825156in}{2.074372in}}%
\pgfpathmoveto{\pgfqpoint{3.820898in}{2.078630in}}%
\pgfpathlineto{\pgfqpoint{3.820898in}{2.078630in}}%
\pgfpathlineto{\pgfqpoint{3.820898in}{2.082888in}}%
\pgfpathlineto{\pgfqpoint{3.825156in}{2.082888in}}%
\pgfpathlineto{\pgfqpoint{3.825156in}{2.078630in}}%
\pgfpathmoveto{\pgfqpoint{3.820898in}{2.082888in}}%
\pgfpathlineto{\pgfqpoint{3.820898in}{2.082888in}}%
\pgfpathlineto{\pgfqpoint{3.820898in}{2.087146in}}%
\pgfpathlineto{\pgfqpoint{3.825156in}{2.087146in}}%
\pgfpathlineto{\pgfqpoint{3.825156in}{2.082888in}}%
\pgfpathmoveto{\pgfqpoint{3.820898in}{2.087146in}}%
\pgfpathlineto{\pgfqpoint{3.820898in}{2.087146in}}%
\pgfpathlineto{\pgfqpoint{3.820898in}{2.091404in}}%
\pgfpathlineto{\pgfqpoint{3.825156in}{2.091404in}}%
\pgfpathlineto{\pgfqpoint{3.825156in}{2.087146in}}%
\pgfpathmoveto{\pgfqpoint{3.816640in}{2.095661in}}%
\pgfpathlineto{\pgfqpoint{3.816640in}{2.095661in}}%
\pgfpathlineto{\pgfqpoint{3.816640in}{2.099919in}}%
\pgfpathlineto{\pgfqpoint{3.820898in}{2.099919in}}%
\pgfpathlineto{\pgfqpoint{3.820898in}{2.095661in}}%
\pgfpathmoveto{\pgfqpoint{3.820898in}{2.091404in}}%
\pgfpathlineto{\pgfqpoint{3.820898in}{2.091404in}}%
\pgfpathlineto{\pgfqpoint{3.820898in}{2.095661in}}%
\pgfpathlineto{\pgfqpoint{3.825156in}{2.095661in}}%
\pgfpathlineto{\pgfqpoint{3.825156in}{2.091404in}}%
\pgfpathmoveto{\pgfqpoint{3.820898in}{2.095661in}}%
\pgfpathlineto{\pgfqpoint{3.820898in}{2.095661in}}%
\pgfpathlineto{\pgfqpoint{3.820898in}{2.099919in}}%
\pgfpathlineto{\pgfqpoint{3.825156in}{2.099919in}}%
\pgfpathlineto{\pgfqpoint{3.825156in}{2.095661in}}%
\pgfpathmoveto{\pgfqpoint{3.816640in}{2.099919in}}%
\pgfpathlineto{\pgfqpoint{3.816640in}{2.099919in}}%
\pgfpathlineto{\pgfqpoint{3.816640in}{2.104177in}}%
\pgfpathlineto{\pgfqpoint{3.820898in}{2.104177in}}%
\pgfpathlineto{\pgfqpoint{3.820898in}{2.099919in}}%
\pgfpathmoveto{\pgfqpoint{3.816640in}{2.104177in}}%
\pgfpathlineto{\pgfqpoint{3.816640in}{2.104177in}}%
\pgfpathlineto{\pgfqpoint{3.816640in}{2.108435in}}%
\pgfpathlineto{\pgfqpoint{3.820898in}{2.108435in}}%
\pgfpathlineto{\pgfqpoint{3.820898in}{2.104177in}}%
\pgfpathmoveto{\pgfqpoint{3.820898in}{2.099919in}}%
\pgfpathlineto{\pgfqpoint{3.820898in}{2.099919in}}%
\pgfpathlineto{\pgfqpoint{3.820898in}{2.104177in}}%
\pgfpathlineto{\pgfqpoint{3.825156in}{2.104177in}}%
\pgfpathlineto{\pgfqpoint{3.825156in}{2.099919in}}%
\pgfpathmoveto{\pgfqpoint{3.820898in}{2.104177in}}%
\pgfpathlineto{\pgfqpoint{3.820898in}{2.104177in}}%
\pgfpathlineto{\pgfqpoint{3.820898in}{2.108435in}}%
\pgfpathlineto{\pgfqpoint{3.825156in}{2.108435in}}%
\pgfpathlineto{\pgfqpoint{3.825156in}{2.104177in}}%
\pgfpathmoveto{\pgfqpoint{3.825156in}{2.074372in}}%
\pgfpathlineto{\pgfqpoint{3.825156in}{2.074372in}}%
\pgfpathlineto{\pgfqpoint{3.825156in}{2.078630in}}%
\pgfpathlineto{\pgfqpoint{3.829413in}{2.078630in}}%
\pgfpathlineto{\pgfqpoint{3.829413in}{2.074372in}}%
\pgfpathmoveto{\pgfqpoint{3.825156in}{2.078630in}}%
\pgfpathlineto{\pgfqpoint{3.825156in}{2.078630in}}%
\pgfpathlineto{\pgfqpoint{3.825156in}{2.082888in}}%
\pgfpathlineto{\pgfqpoint{3.829413in}{2.082888in}}%
\pgfpathlineto{\pgfqpoint{3.829413in}{2.078630in}}%
\pgfpathmoveto{\pgfqpoint{3.825156in}{2.082888in}}%
\pgfpathlineto{\pgfqpoint{3.825156in}{2.082888in}}%
\pgfpathlineto{\pgfqpoint{3.825156in}{2.087146in}}%
\pgfpathlineto{\pgfqpoint{3.829413in}{2.087146in}}%
\pgfpathlineto{\pgfqpoint{3.829413in}{2.082888in}}%
\pgfpathmoveto{\pgfqpoint{3.825156in}{2.087146in}}%
\pgfpathlineto{\pgfqpoint{3.825156in}{2.087146in}}%
\pgfpathlineto{\pgfqpoint{3.825156in}{2.091404in}}%
\pgfpathlineto{\pgfqpoint{3.829413in}{2.091404in}}%
\pgfpathlineto{\pgfqpoint{3.829413in}{2.087146in}}%
\pgfpathmoveto{\pgfqpoint{3.825156in}{2.091404in}}%
\pgfpathlineto{\pgfqpoint{3.825156in}{2.091404in}}%
\pgfpathlineto{\pgfqpoint{3.825156in}{2.095661in}}%
\pgfpathlineto{\pgfqpoint{3.829413in}{2.095661in}}%
\pgfpathlineto{\pgfqpoint{3.829413in}{2.091404in}}%
\pgfpathmoveto{\pgfqpoint{3.825156in}{2.095661in}}%
\pgfpathlineto{\pgfqpoint{3.825156in}{2.095661in}}%
\pgfpathlineto{\pgfqpoint{3.825156in}{2.099919in}}%
\pgfpathlineto{\pgfqpoint{3.829413in}{2.099919in}}%
\pgfpathlineto{\pgfqpoint{3.829413in}{2.095661in}}%
\pgfpathmoveto{\pgfqpoint{3.825156in}{2.099919in}}%
\pgfpathlineto{\pgfqpoint{3.825156in}{2.099919in}}%
\pgfpathlineto{\pgfqpoint{3.825156in}{2.104177in}}%
\pgfpathlineto{\pgfqpoint{3.829413in}{2.104177in}}%
\pgfpathlineto{\pgfqpoint{3.829413in}{2.099919in}}%
\pgfpathmoveto{\pgfqpoint{3.816640in}{2.108435in}}%
\pgfpathlineto{\pgfqpoint{3.816640in}{2.108435in}}%
\pgfpathlineto{\pgfqpoint{3.816640in}{2.112693in}}%
\pgfpathlineto{\pgfqpoint{3.820898in}{2.112693in}}%
\pgfpathlineto{\pgfqpoint{3.820898in}{2.108435in}}%
\pgfpathmoveto{\pgfqpoint{3.816640in}{2.112693in}}%
\pgfpathlineto{\pgfqpoint{3.816640in}{2.112693in}}%
\pgfpathlineto{\pgfqpoint{3.816640in}{2.116951in}}%
\pgfpathlineto{\pgfqpoint{3.820898in}{2.116951in}}%
\pgfpathlineto{\pgfqpoint{3.820898in}{2.112693in}}%
\pgfpathmoveto{\pgfqpoint{3.820898in}{2.108435in}}%
\pgfpathlineto{\pgfqpoint{3.820898in}{2.108435in}}%
\pgfpathlineto{\pgfqpoint{3.820898in}{2.112693in}}%
\pgfpathlineto{\pgfqpoint{3.825156in}{2.112693in}}%
\pgfpathlineto{\pgfqpoint{3.825156in}{2.108435in}}%
\pgfpathmoveto{\pgfqpoint{3.820898in}{2.112693in}}%
\pgfpathlineto{\pgfqpoint{3.820898in}{2.112693in}}%
\pgfpathlineto{\pgfqpoint{3.820898in}{2.116951in}}%
\pgfpathlineto{\pgfqpoint{3.825156in}{2.116951in}}%
\pgfpathlineto{\pgfqpoint{3.825156in}{2.112693in}}%
\pgfpathmoveto{\pgfqpoint{3.816640in}{2.116951in}}%
\pgfpathlineto{\pgfqpoint{3.816640in}{2.116951in}}%
\pgfpathlineto{\pgfqpoint{3.816640in}{2.121209in}}%
\pgfpathlineto{\pgfqpoint{3.820898in}{2.121209in}}%
\pgfpathlineto{\pgfqpoint{3.820898in}{2.116951in}}%
\pgfpathmoveto{\pgfqpoint{3.816640in}{2.121209in}}%
\pgfpathlineto{\pgfqpoint{3.816640in}{2.121209in}}%
\pgfpathlineto{\pgfqpoint{3.816640in}{2.125467in}}%
\pgfpathlineto{\pgfqpoint{3.820898in}{2.125467in}}%
\pgfpathlineto{\pgfqpoint{3.820898in}{2.121209in}}%
\pgfpathmoveto{\pgfqpoint{3.820898in}{2.116951in}}%
\pgfpathlineto{\pgfqpoint{3.820898in}{2.116951in}}%
\pgfpathlineto{\pgfqpoint{3.820898in}{2.121209in}}%
\pgfpathlineto{\pgfqpoint{3.825156in}{2.121209in}}%
\pgfpathlineto{\pgfqpoint{3.825156in}{2.116951in}}%
\pgfpathmoveto{\pgfqpoint{3.820898in}{2.121209in}}%
\pgfpathlineto{\pgfqpoint{3.820898in}{2.121209in}}%
\pgfpathlineto{\pgfqpoint{3.820898in}{2.125467in}}%
\pgfpathlineto{\pgfqpoint{3.825156in}{2.125467in}}%
\pgfpathlineto{\pgfqpoint{3.825156in}{2.121209in}}%
\pgfpathmoveto{\pgfqpoint{3.812382in}{2.125467in}}%
\pgfpathlineto{\pgfqpoint{3.812382in}{2.125467in}}%
\pgfpathlineto{\pgfqpoint{3.812382in}{2.129724in}}%
\pgfpathlineto{\pgfqpoint{3.816640in}{2.129724in}}%
\pgfpathlineto{\pgfqpoint{3.816640in}{2.125467in}}%
\pgfpathmoveto{\pgfqpoint{3.812382in}{2.129724in}}%
\pgfpathlineto{\pgfqpoint{3.812382in}{2.129724in}}%
\pgfpathlineto{\pgfqpoint{3.812382in}{2.133982in}}%
\pgfpathlineto{\pgfqpoint{3.816640in}{2.133982in}}%
\pgfpathlineto{\pgfqpoint{3.816640in}{2.129724in}}%
\pgfpathmoveto{\pgfqpoint{3.812382in}{2.133982in}}%
\pgfpathlineto{\pgfqpoint{3.812382in}{2.133982in}}%
\pgfpathlineto{\pgfqpoint{3.812382in}{2.138240in}}%
\pgfpathlineto{\pgfqpoint{3.816640in}{2.138240in}}%
\pgfpathlineto{\pgfqpoint{3.816640in}{2.133982in}}%
\pgfpathmoveto{\pgfqpoint{3.812382in}{2.138240in}}%
\pgfpathlineto{\pgfqpoint{3.812382in}{2.138240in}}%
\pgfpathlineto{\pgfqpoint{3.812382in}{2.142498in}}%
\pgfpathlineto{\pgfqpoint{3.816640in}{2.142498in}}%
\pgfpathlineto{\pgfqpoint{3.816640in}{2.138240in}}%
\pgfpathmoveto{\pgfqpoint{3.816640in}{2.125467in}}%
\pgfpathlineto{\pgfqpoint{3.816640in}{2.125467in}}%
\pgfpathlineto{\pgfqpoint{3.816640in}{2.129724in}}%
\pgfpathlineto{\pgfqpoint{3.820898in}{2.129724in}}%
\pgfpathlineto{\pgfqpoint{3.820898in}{2.125467in}}%
\pgfpathmoveto{\pgfqpoint{3.816640in}{2.129724in}}%
\pgfpathlineto{\pgfqpoint{3.816640in}{2.129724in}}%
\pgfpathlineto{\pgfqpoint{3.816640in}{2.133982in}}%
\pgfpathlineto{\pgfqpoint{3.820898in}{2.133982in}}%
\pgfpathlineto{\pgfqpoint{3.820898in}{2.129724in}}%
\pgfpathmoveto{\pgfqpoint{3.820898in}{2.125467in}}%
\pgfpathlineto{\pgfqpoint{3.820898in}{2.125467in}}%
\pgfpathlineto{\pgfqpoint{3.820898in}{2.129724in}}%
\pgfpathlineto{\pgfqpoint{3.825156in}{2.129724in}}%
\pgfpathlineto{\pgfqpoint{3.825156in}{2.125467in}}%
\pgfpathmoveto{\pgfqpoint{3.820898in}{2.129724in}}%
\pgfpathlineto{\pgfqpoint{3.820898in}{2.129724in}}%
\pgfpathlineto{\pgfqpoint{3.820898in}{2.133982in}}%
\pgfpathlineto{\pgfqpoint{3.825156in}{2.133982in}}%
\pgfpathlineto{\pgfqpoint{3.825156in}{2.129724in}}%
\pgfpathmoveto{\pgfqpoint{3.816640in}{2.133982in}}%
\pgfpathlineto{\pgfqpoint{3.816640in}{2.133982in}}%
\pgfpathlineto{\pgfqpoint{3.816640in}{2.138240in}}%
\pgfpathlineto{\pgfqpoint{3.820898in}{2.138240in}}%
\pgfpathlineto{\pgfqpoint{3.820898in}{2.133982in}}%
\pgfpathmoveto{\pgfqpoint{3.816640in}{2.138240in}}%
\pgfpathlineto{\pgfqpoint{3.816640in}{2.138240in}}%
\pgfpathlineto{\pgfqpoint{3.816640in}{2.142498in}}%
\pgfpathlineto{\pgfqpoint{3.820898in}{2.142498in}}%
\pgfpathlineto{\pgfqpoint{3.820898in}{2.138240in}}%
\pgfpathmoveto{\pgfqpoint{3.803866in}{2.180819in}}%
\pgfpathlineto{\pgfqpoint{3.803866in}{2.180819in}}%
\pgfpathlineto{\pgfqpoint{3.803866in}{2.185077in}}%
\pgfpathlineto{\pgfqpoint{3.808124in}{2.185077in}}%
\pgfpathlineto{\pgfqpoint{3.808124in}{2.180819in}}%
\pgfpathmoveto{\pgfqpoint{3.803866in}{2.185077in}}%
\pgfpathlineto{\pgfqpoint{3.803866in}{2.185077in}}%
\pgfpathlineto{\pgfqpoint{3.803866in}{2.189335in}}%
\pgfpathlineto{\pgfqpoint{3.808124in}{2.189335in}}%
\pgfpathlineto{\pgfqpoint{3.808124in}{2.185077in}}%
\pgfpathmoveto{\pgfqpoint{3.803866in}{2.189335in}}%
\pgfpathlineto{\pgfqpoint{3.803866in}{2.189335in}}%
\pgfpathlineto{\pgfqpoint{3.803866in}{2.193592in}}%
\pgfpathlineto{\pgfqpoint{3.808124in}{2.193592in}}%
\pgfpathlineto{\pgfqpoint{3.808124in}{2.189335in}}%
\pgfpathmoveto{\pgfqpoint{3.803866in}{2.193592in}}%
\pgfpathlineto{\pgfqpoint{3.803866in}{2.193592in}}%
\pgfpathlineto{\pgfqpoint{3.803866in}{2.197850in}}%
\pgfpathlineto{\pgfqpoint{3.808124in}{2.197850in}}%
\pgfpathlineto{\pgfqpoint{3.808124in}{2.193592in}}%
\pgfpathmoveto{\pgfqpoint{3.803866in}{2.197850in}}%
\pgfpathlineto{\pgfqpoint{3.803866in}{2.197850in}}%
\pgfpathlineto{\pgfqpoint{3.803866in}{2.202108in}}%
\pgfpathlineto{\pgfqpoint{3.808124in}{2.202108in}}%
\pgfpathlineto{\pgfqpoint{3.808124in}{2.197850in}}%
\pgfpathmoveto{\pgfqpoint{3.803866in}{2.202108in}}%
\pgfpathlineto{\pgfqpoint{3.803866in}{2.202108in}}%
\pgfpathlineto{\pgfqpoint{3.803866in}{2.206366in}}%
\pgfpathlineto{\pgfqpoint{3.808124in}{2.206366in}}%
\pgfpathlineto{\pgfqpoint{3.808124in}{2.202108in}}%
\pgfpathmoveto{\pgfqpoint{3.803866in}{2.206366in}}%
\pgfpathlineto{\pgfqpoint{3.803866in}{2.206366in}}%
\pgfpathlineto{\pgfqpoint{3.803866in}{2.210624in}}%
\pgfpathlineto{\pgfqpoint{3.808124in}{2.210624in}}%
\pgfpathlineto{\pgfqpoint{3.808124in}{2.206366in}}%
\pgfpathmoveto{\pgfqpoint{3.812382in}{2.142498in}}%
\pgfpathlineto{\pgfqpoint{3.812382in}{2.142498in}}%
\pgfpathlineto{\pgfqpoint{3.812382in}{2.146756in}}%
\pgfpathlineto{\pgfqpoint{3.816640in}{2.146756in}}%
\pgfpathlineto{\pgfqpoint{3.816640in}{2.142498in}}%
\pgfpathmoveto{\pgfqpoint{3.812382in}{2.146756in}}%
\pgfpathlineto{\pgfqpoint{3.812382in}{2.146756in}}%
\pgfpathlineto{\pgfqpoint{3.812382in}{2.151014in}}%
\pgfpathlineto{\pgfqpoint{3.816640in}{2.151014in}}%
\pgfpathlineto{\pgfqpoint{3.816640in}{2.146756in}}%
\pgfpathmoveto{\pgfqpoint{3.808124in}{2.151014in}}%
\pgfpathlineto{\pgfqpoint{3.808124in}{2.151014in}}%
\pgfpathlineto{\pgfqpoint{3.808124in}{2.155272in}}%
\pgfpathlineto{\pgfqpoint{3.812382in}{2.155272in}}%
\pgfpathlineto{\pgfqpoint{3.812382in}{2.151014in}}%
\pgfpathmoveto{\pgfqpoint{3.808124in}{2.155272in}}%
\pgfpathlineto{\pgfqpoint{3.808124in}{2.155272in}}%
\pgfpathlineto{\pgfqpoint{3.808124in}{2.159530in}}%
\pgfpathlineto{\pgfqpoint{3.812382in}{2.159530in}}%
\pgfpathlineto{\pgfqpoint{3.812382in}{2.155272in}}%
\pgfpathmoveto{\pgfqpoint{3.812382in}{2.151014in}}%
\pgfpathlineto{\pgfqpoint{3.812382in}{2.151014in}}%
\pgfpathlineto{\pgfqpoint{3.812382in}{2.155272in}}%
\pgfpathlineto{\pgfqpoint{3.816640in}{2.155272in}}%
\pgfpathlineto{\pgfqpoint{3.816640in}{2.151014in}}%
\pgfpathmoveto{\pgfqpoint{3.812382in}{2.155272in}}%
\pgfpathlineto{\pgfqpoint{3.812382in}{2.155272in}}%
\pgfpathlineto{\pgfqpoint{3.812382in}{2.159530in}}%
\pgfpathlineto{\pgfqpoint{3.816640in}{2.159530in}}%
\pgfpathlineto{\pgfqpoint{3.816640in}{2.155272in}}%
\pgfpathmoveto{\pgfqpoint{3.816640in}{2.142498in}}%
\pgfpathlineto{\pgfqpoint{3.816640in}{2.142498in}}%
\pgfpathlineto{\pgfqpoint{3.816640in}{2.146756in}}%
\pgfpathlineto{\pgfqpoint{3.820898in}{2.146756in}}%
\pgfpathlineto{\pgfqpoint{3.820898in}{2.142498in}}%
\pgfpathmoveto{\pgfqpoint{3.816640in}{2.146756in}}%
\pgfpathlineto{\pgfqpoint{3.816640in}{2.146756in}}%
\pgfpathlineto{\pgfqpoint{3.816640in}{2.151014in}}%
\pgfpathlineto{\pgfqpoint{3.820898in}{2.151014in}}%
\pgfpathlineto{\pgfqpoint{3.820898in}{2.146756in}}%
\pgfpathmoveto{\pgfqpoint{3.816640in}{2.151014in}}%
\pgfpathlineto{\pgfqpoint{3.816640in}{2.151014in}}%
\pgfpathlineto{\pgfqpoint{3.816640in}{2.155272in}}%
\pgfpathlineto{\pgfqpoint{3.820898in}{2.155272in}}%
\pgfpathlineto{\pgfqpoint{3.820898in}{2.151014in}}%
\pgfpathmoveto{\pgfqpoint{3.816640in}{2.155272in}}%
\pgfpathlineto{\pgfqpoint{3.816640in}{2.155272in}}%
\pgfpathlineto{\pgfqpoint{3.816640in}{2.159530in}}%
\pgfpathlineto{\pgfqpoint{3.820898in}{2.159530in}}%
\pgfpathlineto{\pgfqpoint{3.820898in}{2.155272in}}%
\pgfpathmoveto{\pgfqpoint{3.808124in}{2.159530in}}%
\pgfpathlineto{\pgfqpoint{3.808124in}{2.159530in}}%
\pgfpathlineto{\pgfqpoint{3.808124in}{2.163787in}}%
\pgfpathlineto{\pgfqpoint{3.812382in}{2.163787in}}%
\pgfpathlineto{\pgfqpoint{3.812382in}{2.159530in}}%
\pgfpathmoveto{\pgfqpoint{3.808124in}{2.163787in}}%
\pgfpathlineto{\pgfqpoint{3.808124in}{2.163787in}}%
\pgfpathlineto{\pgfqpoint{3.808124in}{2.168045in}}%
\pgfpathlineto{\pgfqpoint{3.812382in}{2.168045in}}%
\pgfpathlineto{\pgfqpoint{3.812382in}{2.163787in}}%
\pgfpathmoveto{\pgfqpoint{3.812382in}{2.159530in}}%
\pgfpathlineto{\pgfqpoint{3.812382in}{2.159530in}}%
\pgfpathlineto{\pgfqpoint{3.812382in}{2.163787in}}%
\pgfpathlineto{\pgfqpoint{3.816640in}{2.163787in}}%
\pgfpathlineto{\pgfqpoint{3.816640in}{2.159530in}}%
\pgfpathmoveto{\pgfqpoint{3.812382in}{2.163787in}}%
\pgfpathlineto{\pgfqpoint{3.812382in}{2.163787in}}%
\pgfpathlineto{\pgfqpoint{3.812382in}{2.168045in}}%
\pgfpathlineto{\pgfqpoint{3.816640in}{2.168045in}}%
\pgfpathlineto{\pgfqpoint{3.816640in}{2.163787in}}%
\pgfpathmoveto{\pgfqpoint{3.808124in}{2.168045in}}%
\pgfpathlineto{\pgfqpoint{3.808124in}{2.168045in}}%
\pgfpathlineto{\pgfqpoint{3.808124in}{2.172303in}}%
\pgfpathlineto{\pgfqpoint{3.812382in}{2.172303in}}%
\pgfpathlineto{\pgfqpoint{3.812382in}{2.168045in}}%
\pgfpathmoveto{\pgfqpoint{3.808124in}{2.172303in}}%
\pgfpathlineto{\pgfqpoint{3.808124in}{2.172303in}}%
\pgfpathlineto{\pgfqpoint{3.808124in}{2.176561in}}%
\pgfpathlineto{\pgfqpoint{3.812382in}{2.176561in}}%
\pgfpathlineto{\pgfqpoint{3.812382in}{2.172303in}}%
\pgfpathmoveto{\pgfqpoint{3.812382in}{2.168045in}}%
\pgfpathlineto{\pgfqpoint{3.812382in}{2.168045in}}%
\pgfpathlineto{\pgfqpoint{3.812382in}{2.172303in}}%
\pgfpathlineto{\pgfqpoint{3.816640in}{2.172303in}}%
\pgfpathlineto{\pgfqpoint{3.816640in}{2.168045in}}%
\pgfpathmoveto{\pgfqpoint{3.812382in}{2.172303in}}%
\pgfpathlineto{\pgfqpoint{3.812382in}{2.172303in}}%
\pgfpathlineto{\pgfqpoint{3.812382in}{2.176561in}}%
\pgfpathlineto{\pgfqpoint{3.816640in}{2.176561in}}%
\pgfpathlineto{\pgfqpoint{3.816640in}{2.172303in}}%
\pgfpathmoveto{\pgfqpoint{3.816640in}{2.159530in}}%
\pgfpathlineto{\pgfqpoint{3.816640in}{2.159530in}}%
\pgfpathlineto{\pgfqpoint{3.816640in}{2.163787in}}%
\pgfpathlineto{\pgfqpoint{3.820898in}{2.163787in}}%
\pgfpathlineto{\pgfqpoint{3.820898in}{2.159530in}}%
\pgfpathmoveto{\pgfqpoint{3.808124in}{2.176561in}}%
\pgfpathlineto{\pgfqpoint{3.808124in}{2.176561in}}%
\pgfpathlineto{\pgfqpoint{3.808124in}{2.180819in}}%
\pgfpathlineto{\pgfqpoint{3.812382in}{2.180819in}}%
\pgfpathlineto{\pgfqpoint{3.812382in}{2.176561in}}%
\pgfpathmoveto{\pgfqpoint{3.808124in}{2.180819in}}%
\pgfpathlineto{\pgfqpoint{3.808124in}{2.180819in}}%
\pgfpathlineto{\pgfqpoint{3.808124in}{2.185077in}}%
\pgfpathlineto{\pgfqpoint{3.812382in}{2.185077in}}%
\pgfpathlineto{\pgfqpoint{3.812382in}{2.180819in}}%
\pgfpathmoveto{\pgfqpoint{3.812382in}{2.176561in}}%
\pgfpathlineto{\pgfqpoint{3.812382in}{2.176561in}}%
\pgfpathlineto{\pgfqpoint{3.812382in}{2.180819in}}%
\pgfpathlineto{\pgfqpoint{3.816640in}{2.180819in}}%
\pgfpathlineto{\pgfqpoint{3.816640in}{2.176561in}}%
\pgfpathmoveto{\pgfqpoint{3.812382in}{2.180819in}}%
\pgfpathlineto{\pgfqpoint{3.812382in}{2.180819in}}%
\pgfpathlineto{\pgfqpoint{3.812382in}{2.185077in}}%
\pgfpathlineto{\pgfqpoint{3.816640in}{2.185077in}}%
\pgfpathlineto{\pgfqpoint{3.816640in}{2.180819in}}%
\pgfpathmoveto{\pgfqpoint{3.808124in}{2.185077in}}%
\pgfpathlineto{\pgfqpoint{3.808124in}{2.185077in}}%
\pgfpathlineto{\pgfqpoint{3.808124in}{2.189335in}}%
\pgfpathlineto{\pgfqpoint{3.812382in}{2.189335in}}%
\pgfpathlineto{\pgfqpoint{3.812382in}{2.185077in}}%
\pgfpathmoveto{\pgfqpoint{3.808124in}{2.189335in}}%
\pgfpathlineto{\pgfqpoint{3.808124in}{2.189335in}}%
\pgfpathlineto{\pgfqpoint{3.808124in}{2.193592in}}%
\pgfpathlineto{\pgfqpoint{3.812382in}{2.193592in}}%
\pgfpathlineto{\pgfqpoint{3.812382in}{2.189335in}}%
\pgfpathmoveto{\pgfqpoint{3.812382in}{2.185077in}}%
\pgfpathlineto{\pgfqpoint{3.812382in}{2.185077in}}%
\pgfpathlineto{\pgfqpoint{3.812382in}{2.189335in}}%
\pgfpathlineto{\pgfqpoint{3.816640in}{2.189335in}}%
\pgfpathlineto{\pgfqpoint{3.816640in}{2.185077in}}%
\pgfpathmoveto{\pgfqpoint{3.808124in}{2.193592in}}%
\pgfpathlineto{\pgfqpoint{3.808124in}{2.193592in}}%
\pgfpathlineto{\pgfqpoint{3.808124in}{2.197850in}}%
\pgfpathlineto{\pgfqpoint{3.812382in}{2.197850in}}%
\pgfpathlineto{\pgfqpoint{3.812382in}{2.193592in}}%
\pgfpathmoveto{\pgfqpoint{3.808124in}{2.197850in}}%
\pgfpathlineto{\pgfqpoint{3.808124in}{2.197850in}}%
\pgfpathlineto{\pgfqpoint{3.808124in}{2.202108in}}%
\pgfpathlineto{\pgfqpoint{3.812382in}{2.202108in}}%
\pgfpathlineto{\pgfqpoint{3.812382in}{2.197850in}}%
\pgfpathmoveto{\pgfqpoint{3.808124in}{2.202108in}}%
\pgfpathlineto{\pgfqpoint{3.808124in}{2.202108in}}%
\pgfpathlineto{\pgfqpoint{3.808124in}{2.206366in}}%
\pgfpathlineto{\pgfqpoint{3.812382in}{2.206366in}}%
\pgfpathlineto{\pgfqpoint{3.812382in}{2.202108in}}%
\pgfpathmoveto{\pgfqpoint{3.808124in}{2.206366in}}%
\pgfpathlineto{\pgfqpoint{3.808124in}{2.206366in}}%
\pgfpathlineto{\pgfqpoint{3.808124in}{2.210624in}}%
\pgfpathlineto{\pgfqpoint{3.812382in}{2.210624in}}%
\pgfpathlineto{\pgfqpoint{3.812382in}{2.206366in}}%
\pgfpathmoveto{\pgfqpoint{3.799608in}{2.210624in}}%
\pgfpathlineto{\pgfqpoint{3.799608in}{2.210624in}}%
\pgfpathlineto{\pgfqpoint{3.799608in}{2.214882in}}%
\pgfpathlineto{\pgfqpoint{3.803866in}{2.214882in}}%
\pgfpathlineto{\pgfqpoint{3.803866in}{2.210624in}}%
\pgfpathmoveto{\pgfqpoint{3.799608in}{2.214882in}}%
\pgfpathlineto{\pgfqpoint{3.799608in}{2.214882in}}%
\pgfpathlineto{\pgfqpoint{3.799608in}{2.219140in}}%
\pgfpathlineto{\pgfqpoint{3.803866in}{2.219140in}}%
\pgfpathlineto{\pgfqpoint{3.803866in}{2.214882in}}%
\pgfpathmoveto{\pgfqpoint{3.803866in}{2.210624in}}%
\pgfpathlineto{\pgfqpoint{3.803866in}{2.210624in}}%
\pgfpathlineto{\pgfqpoint{3.803866in}{2.214882in}}%
\pgfpathlineto{\pgfqpoint{3.808124in}{2.214882in}}%
\pgfpathlineto{\pgfqpoint{3.808124in}{2.210624in}}%
\pgfpathmoveto{\pgfqpoint{3.803866in}{2.214882in}}%
\pgfpathlineto{\pgfqpoint{3.803866in}{2.214882in}}%
\pgfpathlineto{\pgfqpoint{3.803866in}{2.219140in}}%
\pgfpathlineto{\pgfqpoint{3.808124in}{2.219140in}}%
\pgfpathlineto{\pgfqpoint{3.808124in}{2.214882in}}%
\pgfpathmoveto{\pgfqpoint{3.799608in}{2.219140in}}%
\pgfpathlineto{\pgfqpoint{3.799608in}{2.219140in}}%
\pgfpathlineto{\pgfqpoint{3.799608in}{2.223397in}}%
\pgfpathlineto{\pgfqpoint{3.803866in}{2.223397in}}%
\pgfpathlineto{\pgfqpoint{3.803866in}{2.219140in}}%
\pgfpathmoveto{\pgfqpoint{3.799608in}{2.223397in}}%
\pgfpathlineto{\pgfqpoint{3.799608in}{2.223397in}}%
\pgfpathlineto{\pgfqpoint{3.799608in}{2.227655in}}%
\pgfpathlineto{\pgfqpoint{3.803866in}{2.227655in}}%
\pgfpathlineto{\pgfqpoint{3.803866in}{2.223397in}}%
\pgfpathmoveto{\pgfqpoint{3.803866in}{2.219140in}}%
\pgfpathlineto{\pgfqpoint{3.803866in}{2.219140in}}%
\pgfpathlineto{\pgfqpoint{3.803866in}{2.223397in}}%
\pgfpathlineto{\pgfqpoint{3.808124in}{2.223397in}}%
\pgfpathlineto{\pgfqpoint{3.808124in}{2.219140in}}%
\pgfpathmoveto{\pgfqpoint{3.803866in}{2.223397in}}%
\pgfpathlineto{\pgfqpoint{3.803866in}{2.223397in}}%
\pgfpathlineto{\pgfqpoint{3.803866in}{2.227655in}}%
\pgfpathlineto{\pgfqpoint{3.808124in}{2.227655in}}%
\pgfpathlineto{\pgfqpoint{3.808124in}{2.223397in}}%
\pgfpathmoveto{\pgfqpoint{3.795350in}{2.236171in}}%
\pgfpathlineto{\pgfqpoint{3.795350in}{2.236171in}}%
\pgfpathlineto{\pgfqpoint{3.795350in}{2.240429in}}%
\pgfpathlineto{\pgfqpoint{3.799608in}{2.240429in}}%
\pgfpathlineto{\pgfqpoint{3.799608in}{2.236171in}}%
\pgfpathmoveto{\pgfqpoint{3.795350in}{2.240429in}}%
\pgfpathlineto{\pgfqpoint{3.795350in}{2.240429in}}%
\pgfpathlineto{\pgfqpoint{3.795350in}{2.244687in}}%
\pgfpathlineto{\pgfqpoint{3.799608in}{2.244687in}}%
\pgfpathlineto{\pgfqpoint{3.799608in}{2.240429in}}%
\pgfpathmoveto{\pgfqpoint{3.799608in}{2.227655in}}%
\pgfpathlineto{\pgfqpoint{3.799608in}{2.227655in}}%
\pgfpathlineto{\pgfqpoint{3.799608in}{2.231913in}}%
\pgfpathlineto{\pgfqpoint{3.803866in}{2.231913in}}%
\pgfpathlineto{\pgfqpoint{3.803866in}{2.227655in}}%
\pgfpathmoveto{\pgfqpoint{3.799608in}{2.231913in}}%
\pgfpathlineto{\pgfqpoint{3.799608in}{2.231913in}}%
\pgfpathlineto{\pgfqpoint{3.799608in}{2.236171in}}%
\pgfpathlineto{\pgfqpoint{3.803866in}{2.236171in}}%
\pgfpathlineto{\pgfqpoint{3.803866in}{2.231913in}}%
\pgfpathmoveto{\pgfqpoint{3.803866in}{2.227655in}}%
\pgfpathlineto{\pgfqpoint{3.803866in}{2.227655in}}%
\pgfpathlineto{\pgfqpoint{3.803866in}{2.231913in}}%
\pgfpathlineto{\pgfqpoint{3.808124in}{2.231913in}}%
\pgfpathlineto{\pgfqpoint{3.808124in}{2.227655in}}%
\pgfpathmoveto{\pgfqpoint{3.803866in}{2.231913in}}%
\pgfpathlineto{\pgfqpoint{3.803866in}{2.231913in}}%
\pgfpathlineto{\pgfqpoint{3.803866in}{2.236171in}}%
\pgfpathlineto{\pgfqpoint{3.808124in}{2.236171in}}%
\pgfpathlineto{\pgfqpoint{3.808124in}{2.231913in}}%
\pgfpathmoveto{\pgfqpoint{3.799608in}{2.236171in}}%
\pgfpathlineto{\pgfqpoint{3.799608in}{2.236171in}}%
\pgfpathlineto{\pgfqpoint{3.799608in}{2.240429in}}%
\pgfpathlineto{\pgfqpoint{3.803866in}{2.240429in}}%
\pgfpathlineto{\pgfqpoint{3.803866in}{2.236171in}}%
\pgfpathmoveto{\pgfqpoint{3.799608in}{2.240429in}}%
\pgfpathlineto{\pgfqpoint{3.799608in}{2.240429in}}%
\pgfpathlineto{\pgfqpoint{3.799608in}{2.244687in}}%
\pgfpathlineto{\pgfqpoint{3.803866in}{2.244687in}}%
\pgfpathlineto{\pgfqpoint{3.803866in}{2.240429in}}%
\pgfpathmoveto{\pgfqpoint{3.803866in}{2.236171in}}%
\pgfpathlineto{\pgfqpoint{3.803866in}{2.236171in}}%
\pgfpathlineto{\pgfqpoint{3.803866in}{2.240429in}}%
\pgfpathlineto{\pgfqpoint{3.808124in}{2.240429in}}%
\pgfpathlineto{\pgfqpoint{3.808124in}{2.236171in}}%
\pgfpathmoveto{\pgfqpoint{3.803866in}{2.240429in}}%
\pgfpathlineto{\pgfqpoint{3.803866in}{2.240429in}}%
\pgfpathlineto{\pgfqpoint{3.803866in}{2.244687in}}%
\pgfpathlineto{\pgfqpoint{3.808124in}{2.244687in}}%
\pgfpathlineto{\pgfqpoint{3.808124in}{2.240429in}}%
\pgfpathmoveto{\pgfqpoint{3.795350in}{2.244687in}}%
\pgfpathlineto{\pgfqpoint{3.795350in}{2.244687in}}%
\pgfpathlineto{\pgfqpoint{3.795350in}{2.248945in}}%
\pgfpathlineto{\pgfqpoint{3.799608in}{2.248945in}}%
\pgfpathlineto{\pgfqpoint{3.799608in}{2.244687in}}%
\pgfpathmoveto{\pgfqpoint{3.795350in}{2.248945in}}%
\pgfpathlineto{\pgfqpoint{3.795350in}{2.248945in}}%
\pgfpathlineto{\pgfqpoint{3.795350in}{2.253203in}}%
\pgfpathlineto{\pgfqpoint{3.799608in}{2.253203in}}%
\pgfpathlineto{\pgfqpoint{3.799608in}{2.248945in}}%
\pgfpathmoveto{\pgfqpoint{3.795350in}{2.253203in}}%
\pgfpathlineto{\pgfqpoint{3.795350in}{2.253203in}}%
\pgfpathlineto{\pgfqpoint{3.795350in}{2.257460in}}%
\pgfpathlineto{\pgfqpoint{3.799608in}{2.257460in}}%
\pgfpathlineto{\pgfqpoint{3.799608in}{2.253203in}}%
\pgfpathmoveto{\pgfqpoint{3.795350in}{2.257460in}}%
\pgfpathlineto{\pgfqpoint{3.795350in}{2.257460in}}%
\pgfpathlineto{\pgfqpoint{3.795350in}{2.261718in}}%
\pgfpathlineto{\pgfqpoint{3.799608in}{2.261718in}}%
\pgfpathlineto{\pgfqpoint{3.799608in}{2.257460in}}%
\pgfpathmoveto{\pgfqpoint{3.799608in}{2.244687in}}%
\pgfpathlineto{\pgfqpoint{3.799608in}{2.244687in}}%
\pgfpathlineto{\pgfqpoint{3.799608in}{2.248945in}}%
\pgfpathlineto{\pgfqpoint{3.803866in}{2.248945in}}%
\pgfpathlineto{\pgfqpoint{3.803866in}{2.244687in}}%
\pgfpathmoveto{\pgfqpoint{3.799608in}{2.248945in}}%
\pgfpathlineto{\pgfqpoint{3.799608in}{2.248945in}}%
\pgfpathlineto{\pgfqpoint{3.799608in}{2.253203in}}%
\pgfpathlineto{\pgfqpoint{3.803866in}{2.253203in}}%
\pgfpathlineto{\pgfqpoint{3.803866in}{2.248945in}}%
\pgfpathmoveto{\pgfqpoint{3.803866in}{2.244687in}}%
\pgfpathlineto{\pgfqpoint{3.803866in}{2.244687in}}%
\pgfpathlineto{\pgfqpoint{3.803866in}{2.248945in}}%
\pgfpathlineto{\pgfqpoint{3.808124in}{2.248945in}}%
\pgfpathlineto{\pgfqpoint{3.808124in}{2.244687in}}%
\pgfpathmoveto{\pgfqpoint{3.799608in}{2.253203in}}%
\pgfpathlineto{\pgfqpoint{3.799608in}{2.253203in}}%
\pgfpathlineto{\pgfqpoint{3.799608in}{2.257460in}}%
\pgfpathlineto{\pgfqpoint{3.803866in}{2.257460in}}%
\pgfpathlineto{\pgfqpoint{3.803866in}{2.253203in}}%
\pgfpathmoveto{\pgfqpoint{3.799608in}{2.257460in}}%
\pgfpathlineto{\pgfqpoint{3.799608in}{2.257460in}}%
\pgfpathlineto{\pgfqpoint{3.799608in}{2.261718in}}%
\pgfpathlineto{\pgfqpoint{3.803866in}{2.261718in}}%
\pgfpathlineto{\pgfqpoint{3.803866in}{2.257460in}}%
\pgfpathmoveto{\pgfqpoint{3.791092in}{2.265976in}}%
\pgfpathlineto{\pgfqpoint{3.791092in}{2.265976in}}%
\pgfpathlineto{\pgfqpoint{3.791092in}{2.270234in}}%
\pgfpathlineto{\pgfqpoint{3.795350in}{2.270234in}}%
\pgfpathlineto{\pgfqpoint{3.795350in}{2.265976in}}%
\pgfpathmoveto{\pgfqpoint{3.795350in}{2.261718in}}%
\pgfpathlineto{\pgfqpoint{3.795350in}{2.261718in}}%
\pgfpathlineto{\pgfqpoint{3.795350in}{2.265976in}}%
\pgfpathlineto{\pgfqpoint{3.799608in}{2.265976in}}%
\pgfpathlineto{\pgfqpoint{3.799608in}{2.261718in}}%
\pgfpathmoveto{\pgfqpoint{3.795350in}{2.265976in}}%
\pgfpathlineto{\pgfqpoint{3.795350in}{2.265976in}}%
\pgfpathlineto{\pgfqpoint{3.795350in}{2.270234in}}%
\pgfpathlineto{\pgfqpoint{3.799608in}{2.270234in}}%
\pgfpathlineto{\pgfqpoint{3.799608in}{2.265976in}}%
\pgfpathmoveto{\pgfqpoint{3.791092in}{2.270234in}}%
\pgfpathlineto{\pgfqpoint{3.791092in}{2.270234in}}%
\pgfpathlineto{\pgfqpoint{3.791092in}{2.274492in}}%
\pgfpathlineto{\pgfqpoint{3.795350in}{2.274492in}}%
\pgfpathlineto{\pgfqpoint{3.795350in}{2.270234in}}%
\pgfpathmoveto{\pgfqpoint{3.791092in}{2.274492in}}%
\pgfpathlineto{\pgfqpoint{3.791092in}{2.274492in}}%
\pgfpathlineto{\pgfqpoint{3.791092in}{2.278750in}}%
\pgfpathlineto{\pgfqpoint{3.795350in}{2.278750in}}%
\pgfpathlineto{\pgfqpoint{3.795350in}{2.274492in}}%
\pgfpathmoveto{\pgfqpoint{3.795350in}{2.270234in}}%
\pgfpathlineto{\pgfqpoint{3.795350in}{2.270234in}}%
\pgfpathlineto{\pgfqpoint{3.795350in}{2.274492in}}%
\pgfpathlineto{\pgfqpoint{3.799608in}{2.274492in}}%
\pgfpathlineto{\pgfqpoint{3.799608in}{2.270234in}}%
\pgfpathmoveto{\pgfqpoint{3.795350in}{2.274492in}}%
\pgfpathlineto{\pgfqpoint{3.795350in}{2.274492in}}%
\pgfpathlineto{\pgfqpoint{3.795350in}{2.278750in}}%
\pgfpathlineto{\pgfqpoint{3.799608in}{2.278750in}}%
\pgfpathlineto{\pgfqpoint{3.799608in}{2.274492in}}%
\pgfpathmoveto{\pgfqpoint{3.799608in}{2.261718in}}%
\pgfpathlineto{\pgfqpoint{3.799608in}{2.261718in}}%
\pgfpathlineto{\pgfqpoint{3.799608in}{2.265976in}}%
\pgfpathlineto{\pgfqpoint{3.803866in}{2.265976in}}%
\pgfpathlineto{\pgfqpoint{3.803866in}{2.261718in}}%
\pgfpathmoveto{\pgfqpoint{3.799608in}{2.265976in}}%
\pgfpathlineto{\pgfqpoint{3.799608in}{2.265976in}}%
\pgfpathlineto{\pgfqpoint{3.799608in}{2.270234in}}%
\pgfpathlineto{\pgfqpoint{3.803866in}{2.270234in}}%
\pgfpathlineto{\pgfqpoint{3.803866in}{2.265976in}}%
\pgfpathmoveto{\pgfqpoint{3.799608in}{2.270234in}}%
\pgfpathlineto{\pgfqpoint{3.799608in}{2.270234in}}%
\pgfpathlineto{\pgfqpoint{3.799608in}{2.274492in}}%
\pgfpathlineto{\pgfqpoint{3.803866in}{2.274492in}}%
\pgfpathlineto{\pgfqpoint{3.803866in}{2.270234in}}%
\pgfpathmoveto{\pgfqpoint{3.808124in}{2.210624in}}%
\pgfpathlineto{\pgfqpoint{3.808124in}{2.210624in}}%
\pgfpathlineto{\pgfqpoint{3.808124in}{2.214882in}}%
\pgfpathlineto{\pgfqpoint{3.812382in}{2.214882in}}%
\pgfpathlineto{\pgfqpoint{3.812382in}{2.210624in}}%
\pgfpathmoveto{\pgfqpoint{3.808124in}{2.214882in}}%
\pgfpathlineto{\pgfqpoint{3.808124in}{2.214882in}}%
\pgfpathlineto{\pgfqpoint{3.808124in}{2.219140in}}%
\pgfpathlineto{\pgfqpoint{3.812382in}{2.219140in}}%
\pgfpathlineto{\pgfqpoint{3.812382in}{2.214882in}}%
\pgfpathmoveto{\pgfqpoint{3.786834in}{2.291523in}}%
\pgfpathlineto{\pgfqpoint{3.786834in}{2.291523in}}%
\pgfpathlineto{\pgfqpoint{3.786834in}{2.295781in}}%
\pgfpathlineto{\pgfqpoint{3.791092in}{2.295781in}}%
\pgfpathlineto{\pgfqpoint{3.791092in}{2.291523in}}%
\pgfpathmoveto{\pgfqpoint{3.786834in}{2.295781in}}%
\pgfpathlineto{\pgfqpoint{3.786834in}{2.295781in}}%
\pgfpathlineto{\pgfqpoint{3.786834in}{2.300039in}}%
\pgfpathlineto{\pgfqpoint{3.791092in}{2.300039in}}%
\pgfpathlineto{\pgfqpoint{3.791092in}{2.295781in}}%
\pgfpathmoveto{\pgfqpoint{3.786834in}{2.300039in}}%
\pgfpathlineto{\pgfqpoint{3.786834in}{2.300039in}}%
\pgfpathlineto{\pgfqpoint{3.786834in}{2.304296in}}%
\pgfpathlineto{\pgfqpoint{3.791092in}{2.304296in}}%
\pgfpathlineto{\pgfqpoint{3.791092in}{2.300039in}}%
\pgfpathmoveto{\pgfqpoint{3.786834in}{2.304296in}}%
\pgfpathlineto{\pgfqpoint{3.786834in}{2.304296in}}%
\pgfpathlineto{\pgfqpoint{3.786834in}{2.308554in}}%
\pgfpathlineto{\pgfqpoint{3.791092in}{2.308554in}}%
\pgfpathlineto{\pgfqpoint{3.791092in}{2.304296in}}%
\pgfpathmoveto{\pgfqpoint{3.786834in}{2.308554in}}%
\pgfpathlineto{\pgfqpoint{3.786834in}{2.308554in}}%
\pgfpathlineto{\pgfqpoint{3.786834in}{2.312812in}}%
\pgfpathlineto{\pgfqpoint{3.791092in}{2.312812in}}%
\pgfpathlineto{\pgfqpoint{3.791092in}{2.308554in}}%
\pgfpathmoveto{\pgfqpoint{3.791092in}{2.278750in}}%
\pgfpathlineto{\pgfqpoint{3.791092in}{2.278750in}}%
\pgfpathlineto{\pgfqpoint{3.791092in}{2.283007in}}%
\pgfpathlineto{\pgfqpoint{3.795350in}{2.283007in}}%
\pgfpathlineto{\pgfqpoint{3.795350in}{2.278750in}}%
\pgfpathmoveto{\pgfqpoint{3.791092in}{2.283007in}}%
\pgfpathlineto{\pgfqpoint{3.791092in}{2.283007in}}%
\pgfpathlineto{\pgfqpoint{3.791092in}{2.287265in}}%
\pgfpathlineto{\pgfqpoint{3.795350in}{2.287265in}}%
\pgfpathlineto{\pgfqpoint{3.795350in}{2.283007in}}%
\pgfpathmoveto{\pgfqpoint{3.795350in}{2.278750in}}%
\pgfpathlineto{\pgfqpoint{3.795350in}{2.278750in}}%
\pgfpathlineto{\pgfqpoint{3.795350in}{2.283007in}}%
\pgfpathlineto{\pgfqpoint{3.799608in}{2.283007in}}%
\pgfpathlineto{\pgfqpoint{3.799608in}{2.278750in}}%
\pgfpathmoveto{\pgfqpoint{3.795350in}{2.283007in}}%
\pgfpathlineto{\pgfqpoint{3.795350in}{2.283007in}}%
\pgfpathlineto{\pgfqpoint{3.795350in}{2.287265in}}%
\pgfpathlineto{\pgfqpoint{3.799608in}{2.287265in}}%
\pgfpathlineto{\pgfqpoint{3.799608in}{2.283007in}}%
\pgfpathmoveto{\pgfqpoint{3.791092in}{2.287265in}}%
\pgfpathlineto{\pgfqpoint{3.791092in}{2.287265in}}%
\pgfpathlineto{\pgfqpoint{3.791092in}{2.291523in}}%
\pgfpathlineto{\pgfqpoint{3.795350in}{2.291523in}}%
\pgfpathlineto{\pgfqpoint{3.795350in}{2.287265in}}%
\pgfpathmoveto{\pgfqpoint{3.791092in}{2.291523in}}%
\pgfpathlineto{\pgfqpoint{3.791092in}{2.291523in}}%
\pgfpathlineto{\pgfqpoint{3.791092in}{2.295781in}}%
\pgfpathlineto{\pgfqpoint{3.795350in}{2.295781in}}%
\pgfpathlineto{\pgfqpoint{3.795350in}{2.291523in}}%
\pgfpathmoveto{\pgfqpoint{3.795350in}{2.287265in}}%
\pgfpathlineto{\pgfqpoint{3.795350in}{2.287265in}}%
\pgfpathlineto{\pgfqpoint{3.795350in}{2.291523in}}%
\pgfpathlineto{\pgfqpoint{3.799608in}{2.291523in}}%
\pgfpathlineto{\pgfqpoint{3.799608in}{2.287265in}}%
\pgfpathmoveto{\pgfqpoint{3.795350in}{2.291523in}}%
\pgfpathlineto{\pgfqpoint{3.795350in}{2.291523in}}%
\pgfpathlineto{\pgfqpoint{3.795350in}{2.295781in}}%
\pgfpathlineto{\pgfqpoint{3.799608in}{2.295781in}}%
\pgfpathlineto{\pgfqpoint{3.799608in}{2.291523in}}%
\pgfpathmoveto{\pgfqpoint{3.791092in}{2.295781in}}%
\pgfpathlineto{\pgfqpoint{3.791092in}{2.295781in}}%
\pgfpathlineto{\pgfqpoint{3.791092in}{2.300039in}}%
\pgfpathlineto{\pgfqpoint{3.795350in}{2.300039in}}%
\pgfpathlineto{\pgfqpoint{3.795350in}{2.295781in}}%
\pgfpathmoveto{\pgfqpoint{3.791092in}{2.300039in}}%
\pgfpathlineto{\pgfqpoint{3.791092in}{2.300039in}}%
\pgfpathlineto{\pgfqpoint{3.791092in}{2.304296in}}%
\pgfpathlineto{\pgfqpoint{3.795350in}{2.304296in}}%
\pgfpathlineto{\pgfqpoint{3.795350in}{2.300039in}}%
\pgfpathmoveto{\pgfqpoint{3.795350in}{2.295781in}}%
\pgfpathlineto{\pgfqpoint{3.795350in}{2.295781in}}%
\pgfpathlineto{\pgfqpoint{3.795350in}{2.300039in}}%
\pgfpathlineto{\pgfqpoint{3.799608in}{2.300039in}}%
\pgfpathlineto{\pgfqpoint{3.799608in}{2.295781in}}%
\pgfpathmoveto{\pgfqpoint{3.795350in}{2.300039in}}%
\pgfpathlineto{\pgfqpoint{3.795350in}{2.300039in}}%
\pgfpathlineto{\pgfqpoint{3.795350in}{2.304296in}}%
\pgfpathlineto{\pgfqpoint{3.799608in}{2.304296in}}%
\pgfpathlineto{\pgfqpoint{3.799608in}{2.300039in}}%
\pgfpathmoveto{\pgfqpoint{3.791092in}{2.304296in}}%
\pgfpathlineto{\pgfqpoint{3.791092in}{2.304296in}}%
\pgfpathlineto{\pgfqpoint{3.791092in}{2.308554in}}%
\pgfpathlineto{\pgfqpoint{3.795350in}{2.308554in}}%
\pgfpathlineto{\pgfqpoint{3.795350in}{2.304296in}}%
\pgfpathmoveto{\pgfqpoint{3.791092in}{2.308554in}}%
\pgfpathlineto{\pgfqpoint{3.791092in}{2.308554in}}%
\pgfpathlineto{\pgfqpoint{3.791092in}{2.312812in}}%
\pgfpathlineto{\pgfqpoint{3.795350in}{2.312812in}}%
\pgfpathlineto{\pgfqpoint{3.795350in}{2.308554in}}%
\pgfpathmoveto{\pgfqpoint{3.782576in}{2.317070in}}%
\pgfpathlineto{\pgfqpoint{3.782576in}{2.317070in}}%
\pgfpathlineto{\pgfqpoint{3.782576in}{2.321327in}}%
\pgfpathlineto{\pgfqpoint{3.786834in}{2.321327in}}%
\pgfpathlineto{\pgfqpoint{3.786834in}{2.317070in}}%
\pgfpathmoveto{\pgfqpoint{3.786834in}{2.312812in}}%
\pgfpathlineto{\pgfqpoint{3.786834in}{2.312812in}}%
\pgfpathlineto{\pgfqpoint{3.786834in}{2.317070in}}%
\pgfpathlineto{\pgfqpoint{3.791092in}{2.317070in}}%
\pgfpathlineto{\pgfqpoint{3.791092in}{2.312812in}}%
\pgfpathmoveto{\pgfqpoint{3.786834in}{2.317070in}}%
\pgfpathlineto{\pgfqpoint{3.786834in}{2.317070in}}%
\pgfpathlineto{\pgfqpoint{3.786834in}{2.321327in}}%
\pgfpathlineto{\pgfqpoint{3.791092in}{2.321327in}}%
\pgfpathlineto{\pgfqpoint{3.791092in}{2.317070in}}%
\pgfpathmoveto{\pgfqpoint{3.782576in}{2.321327in}}%
\pgfpathlineto{\pgfqpoint{3.782576in}{2.321327in}}%
\pgfpathlineto{\pgfqpoint{3.782576in}{2.325585in}}%
\pgfpathlineto{\pgfqpoint{3.786834in}{2.325585in}}%
\pgfpathlineto{\pgfqpoint{3.786834in}{2.321327in}}%
\pgfpathmoveto{\pgfqpoint{3.782576in}{2.325585in}}%
\pgfpathlineto{\pgfqpoint{3.782576in}{2.325585in}}%
\pgfpathlineto{\pgfqpoint{3.782576in}{2.329843in}}%
\pgfpathlineto{\pgfqpoint{3.786834in}{2.329843in}}%
\pgfpathlineto{\pgfqpoint{3.786834in}{2.325585in}}%
\pgfpathmoveto{\pgfqpoint{3.786834in}{2.321327in}}%
\pgfpathlineto{\pgfqpoint{3.786834in}{2.321327in}}%
\pgfpathlineto{\pgfqpoint{3.786834in}{2.325585in}}%
\pgfpathlineto{\pgfqpoint{3.791092in}{2.325585in}}%
\pgfpathlineto{\pgfqpoint{3.791092in}{2.321327in}}%
\pgfpathmoveto{\pgfqpoint{3.786834in}{2.325585in}}%
\pgfpathlineto{\pgfqpoint{3.786834in}{2.325585in}}%
\pgfpathlineto{\pgfqpoint{3.786834in}{2.329843in}}%
\pgfpathlineto{\pgfqpoint{3.791092in}{2.329843in}}%
\pgfpathlineto{\pgfqpoint{3.791092in}{2.325585in}}%
\pgfpathmoveto{\pgfqpoint{3.778318in}{2.342616in}}%
\pgfpathlineto{\pgfqpoint{3.778318in}{2.342616in}}%
\pgfpathlineto{\pgfqpoint{3.778318in}{2.346874in}}%
\pgfpathlineto{\pgfqpoint{3.782576in}{2.346874in}}%
\pgfpathlineto{\pgfqpoint{3.782576in}{2.342616in}}%
\pgfpathmoveto{\pgfqpoint{3.782576in}{2.329843in}}%
\pgfpathlineto{\pgfqpoint{3.782576in}{2.329843in}}%
\pgfpathlineto{\pgfqpoint{3.782576in}{2.334101in}}%
\pgfpathlineto{\pgfqpoint{3.786834in}{2.334101in}}%
\pgfpathlineto{\pgfqpoint{3.786834in}{2.329843in}}%
\pgfpathmoveto{\pgfqpoint{3.782576in}{2.334101in}}%
\pgfpathlineto{\pgfqpoint{3.782576in}{2.334101in}}%
\pgfpathlineto{\pgfqpoint{3.782576in}{2.338359in}}%
\pgfpathlineto{\pgfqpoint{3.786834in}{2.338359in}}%
\pgfpathlineto{\pgfqpoint{3.786834in}{2.334101in}}%
\pgfpathmoveto{\pgfqpoint{3.786834in}{2.329843in}}%
\pgfpathlineto{\pgfqpoint{3.786834in}{2.329843in}}%
\pgfpathlineto{\pgfqpoint{3.786834in}{2.334101in}}%
\pgfpathlineto{\pgfqpoint{3.791092in}{2.334101in}}%
\pgfpathlineto{\pgfqpoint{3.791092in}{2.329843in}}%
\pgfpathmoveto{\pgfqpoint{3.786834in}{2.334101in}}%
\pgfpathlineto{\pgfqpoint{3.786834in}{2.334101in}}%
\pgfpathlineto{\pgfqpoint{3.786834in}{2.338359in}}%
\pgfpathlineto{\pgfqpoint{3.791092in}{2.338359in}}%
\pgfpathlineto{\pgfqpoint{3.791092in}{2.334101in}}%
\pgfpathmoveto{\pgfqpoint{3.782576in}{2.338359in}}%
\pgfpathlineto{\pgfqpoint{3.782576in}{2.338359in}}%
\pgfpathlineto{\pgfqpoint{3.782576in}{2.342616in}}%
\pgfpathlineto{\pgfqpoint{3.786834in}{2.342616in}}%
\pgfpathlineto{\pgfqpoint{3.786834in}{2.338359in}}%
\pgfpathmoveto{\pgfqpoint{3.782576in}{2.342616in}}%
\pgfpathlineto{\pgfqpoint{3.782576in}{2.342616in}}%
\pgfpathlineto{\pgfqpoint{3.782576in}{2.346874in}}%
\pgfpathlineto{\pgfqpoint{3.786834in}{2.346874in}}%
\pgfpathlineto{\pgfqpoint{3.786834in}{2.342616in}}%
\pgfpathmoveto{\pgfqpoint{3.786834in}{2.338359in}}%
\pgfpathlineto{\pgfqpoint{3.786834in}{2.338359in}}%
\pgfpathlineto{\pgfqpoint{3.786834in}{2.342616in}}%
\pgfpathlineto{\pgfqpoint{3.791092in}{2.342616in}}%
\pgfpathlineto{\pgfqpoint{3.791092in}{2.338359in}}%
\pgfpathmoveto{\pgfqpoint{3.786834in}{2.342616in}}%
\pgfpathlineto{\pgfqpoint{3.786834in}{2.342616in}}%
\pgfpathlineto{\pgfqpoint{3.786834in}{2.346874in}}%
\pgfpathlineto{\pgfqpoint{3.791092in}{2.346874in}}%
\pgfpathlineto{\pgfqpoint{3.791092in}{2.342616in}}%
\pgfpathmoveto{\pgfqpoint{3.791092in}{2.312812in}}%
\pgfpathlineto{\pgfqpoint{3.791092in}{2.312812in}}%
\pgfpathlineto{\pgfqpoint{3.791092in}{2.317070in}}%
\pgfpathlineto{\pgfqpoint{3.795350in}{2.317070in}}%
\pgfpathlineto{\pgfqpoint{3.795350in}{2.312812in}}%
\pgfpathmoveto{\pgfqpoint{3.791092in}{2.317070in}}%
\pgfpathlineto{\pgfqpoint{3.791092in}{2.317070in}}%
\pgfpathlineto{\pgfqpoint{3.791092in}{2.321327in}}%
\pgfpathlineto{\pgfqpoint{3.795350in}{2.321327in}}%
\pgfpathlineto{\pgfqpoint{3.795350in}{2.317070in}}%
\pgfpathmoveto{\pgfqpoint{3.791092in}{2.321327in}}%
\pgfpathlineto{\pgfqpoint{3.791092in}{2.321327in}}%
\pgfpathlineto{\pgfqpoint{3.791092in}{2.325585in}}%
\pgfpathlineto{\pgfqpoint{3.795350in}{2.325585in}}%
\pgfpathlineto{\pgfqpoint{3.795350in}{2.321327in}}%
\pgfpathmoveto{\pgfqpoint{3.791092in}{2.325585in}}%
\pgfpathlineto{\pgfqpoint{3.791092in}{2.325585in}}%
\pgfpathlineto{\pgfqpoint{3.791092in}{2.329843in}}%
\pgfpathlineto{\pgfqpoint{3.795350in}{2.329843in}}%
\pgfpathlineto{\pgfqpoint{3.795350in}{2.325585in}}%
\pgfpathmoveto{\pgfqpoint{3.778318in}{2.346874in}}%
\pgfpathlineto{\pgfqpoint{3.778318in}{2.346874in}}%
\pgfpathlineto{\pgfqpoint{3.778318in}{2.351132in}}%
\pgfpathlineto{\pgfqpoint{3.782576in}{2.351132in}}%
\pgfpathlineto{\pgfqpoint{3.782576in}{2.346874in}}%
\pgfpathmoveto{\pgfqpoint{3.778318in}{2.351132in}}%
\pgfpathlineto{\pgfqpoint{3.778318in}{2.351132in}}%
\pgfpathlineto{\pgfqpoint{3.778318in}{2.355390in}}%
\pgfpathlineto{\pgfqpoint{3.782576in}{2.355390in}}%
\pgfpathlineto{\pgfqpoint{3.782576in}{2.351132in}}%
\pgfpathmoveto{\pgfqpoint{3.778318in}{2.355390in}}%
\pgfpathlineto{\pgfqpoint{3.778318in}{2.355390in}}%
\pgfpathlineto{\pgfqpoint{3.778318in}{2.359647in}}%
\pgfpathlineto{\pgfqpoint{3.782576in}{2.359647in}}%
\pgfpathlineto{\pgfqpoint{3.782576in}{2.355390in}}%
\pgfpathmoveto{\pgfqpoint{3.778318in}{2.359647in}}%
\pgfpathlineto{\pgfqpoint{3.778318in}{2.359647in}}%
\pgfpathlineto{\pgfqpoint{3.778318in}{2.363905in}}%
\pgfpathlineto{\pgfqpoint{3.782576in}{2.363905in}}%
\pgfpathlineto{\pgfqpoint{3.782576in}{2.359647in}}%
\pgfpathmoveto{\pgfqpoint{3.782576in}{2.346874in}}%
\pgfpathlineto{\pgfqpoint{3.782576in}{2.346874in}}%
\pgfpathlineto{\pgfqpoint{3.782576in}{2.351132in}}%
\pgfpathlineto{\pgfqpoint{3.786834in}{2.351132in}}%
\pgfpathlineto{\pgfqpoint{3.786834in}{2.346874in}}%
\pgfpathmoveto{\pgfqpoint{3.782576in}{2.351132in}}%
\pgfpathlineto{\pgfqpoint{3.782576in}{2.351132in}}%
\pgfpathlineto{\pgfqpoint{3.782576in}{2.355390in}}%
\pgfpathlineto{\pgfqpoint{3.786834in}{2.355390in}}%
\pgfpathlineto{\pgfqpoint{3.786834in}{2.351132in}}%
\pgfpathmoveto{\pgfqpoint{3.786834in}{2.346874in}}%
\pgfpathlineto{\pgfqpoint{3.786834in}{2.346874in}}%
\pgfpathlineto{\pgfqpoint{3.786834in}{2.351132in}}%
\pgfpathlineto{\pgfqpoint{3.791092in}{2.351132in}}%
\pgfpathlineto{\pgfqpoint{3.791092in}{2.346874in}}%
\pgfpathmoveto{\pgfqpoint{3.786834in}{2.351132in}}%
\pgfpathlineto{\pgfqpoint{3.786834in}{2.351132in}}%
\pgfpathlineto{\pgfqpoint{3.786834in}{2.355390in}}%
\pgfpathlineto{\pgfqpoint{3.791092in}{2.355390in}}%
\pgfpathlineto{\pgfqpoint{3.791092in}{2.351132in}}%
\pgfpathmoveto{\pgfqpoint{3.782576in}{2.355390in}}%
\pgfpathlineto{\pgfqpoint{3.782576in}{2.355390in}}%
\pgfpathlineto{\pgfqpoint{3.782576in}{2.359647in}}%
\pgfpathlineto{\pgfqpoint{3.786834in}{2.359647in}}%
\pgfpathlineto{\pgfqpoint{3.786834in}{2.355390in}}%
\pgfpathmoveto{\pgfqpoint{3.782576in}{2.359647in}}%
\pgfpathlineto{\pgfqpoint{3.782576in}{2.359647in}}%
\pgfpathlineto{\pgfqpoint{3.782576in}{2.363905in}}%
\pgfpathlineto{\pgfqpoint{3.786834in}{2.363905in}}%
\pgfpathlineto{\pgfqpoint{3.786834in}{2.359647in}}%
\pgfpathmoveto{\pgfqpoint{3.774061in}{2.368163in}}%
\pgfpathlineto{\pgfqpoint{3.774061in}{2.368163in}}%
\pgfpathlineto{\pgfqpoint{3.774061in}{2.372421in}}%
\pgfpathlineto{\pgfqpoint{3.778318in}{2.372421in}}%
\pgfpathlineto{\pgfqpoint{3.778318in}{2.368163in}}%
\pgfpathmoveto{\pgfqpoint{3.778318in}{2.363905in}}%
\pgfpathlineto{\pgfqpoint{3.778318in}{2.363905in}}%
\pgfpathlineto{\pgfqpoint{3.778318in}{2.368163in}}%
\pgfpathlineto{\pgfqpoint{3.782576in}{2.368163in}}%
\pgfpathlineto{\pgfqpoint{3.782576in}{2.363905in}}%
\pgfpathmoveto{\pgfqpoint{3.778318in}{2.368163in}}%
\pgfpathlineto{\pgfqpoint{3.778318in}{2.368163in}}%
\pgfpathlineto{\pgfqpoint{3.778318in}{2.372421in}}%
\pgfpathlineto{\pgfqpoint{3.782576in}{2.372421in}}%
\pgfpathlineto{\pgfqpoint{3.782576in}{2.368163in}}%
\pgfpathmoveto{\pgfqpoint{3.774061in}{2.372421in}}%
\pgfpathlineto{\pgfqpoint{3.774061in}{2.372421in}}%
\pgfpathlineto{\pgfqpoint{3.774061in}{2.376679in}}%
\pgfpathlineto{\pgfqpoint{3.778318in}{2.376679in}}%
\pgfpathlineto{\pgfqpoint{3.778318in}{2.372421in}}%
\pgfpathmoveto{\pgfqpoint{3.774061in}{2.376679in}}%
\pgfpathlineto{\pgfqpoint{3.774061in}{2.376679in}}%
\pgfpathlineto{\pgfqpoint{3.774061in}{2.380936in}}%
\pgfpathlineto{\pgfqpoint{3.778318in}{2.380936in}}%
\pgfpathlineto{\pgfqpoint{3.778318in}{2.376679in}}%
\pgfpathmoveto{\pgfqpoint{3.778318in}{2.372421in}}%
\pgfpathlineto{\pgfqpoint{3.778318in}{2.372421in}}%
\pgfpathlineto{\pgfqpoint{3.778318in}{2.376679in}}%
\pgfpathlineto{\pgfqpoint{3.782576in}{2.376679in}}%
\pgfpathlineto{\pgfqpoint{3.782576in}{2.372421in}}%
\pgfpathmoveto{\pgfqpoint{3.778318in}{2.376679in}}%
\pgfpathlineto{\pgfqpoint{3.778318in}{2.376679in}}%
\pgfpathlineto{\pgfqpoint{3.778318in}{2.380936in}}%
\pgfpathlineto{\pgfqpoint{3.782576in}{2.380936in}}%
\pgfpathlineto{\pgfqpoint{3.782576in}{2.376679in}}%
\pgfpathmoveto{\pgfqpoint{3.782576in}{2.363905in}}%
\pgfpathlineto{\pgfqpoint{3.782576in}{2.363905in}}%
\pgfpathlineto{\pgfqpoint{3.782576in}{2.368163in}}%
\pgfpathlineto{\pgfqpoint{3.786834in}{2.368163in}}%
\pgfpathlineto{\pgfqpoint{3.786834in}{2.363905in}}%
\pgfpathmoveto{\pgfqpoint{3.782576in}{2.368163in}}%
\pgfpathlineto{\pgfqpoint{3.782576in}{2.368163in}}%
\pgfpathlineto{\pgfqpoint{3.782576in}{2.372421in}}%
\pgfpathlineto{\pgfqpoint{3.786834in}{2.372421in}}%
\pgfpathlineto{\pgfqpoint{3.786834in}{2.368163in}}%
\pgfpathmoveto{\pgfqpoint{3.782576in}{2.372421in}}%
\pgfpathlineto{\pgfqpoint{3.782576in}{2.372421in}}%
\pgfpathlineto{\pgfqpoint{3.782576in}{2.376679in}}%
\pgfpathlineto{\pgfqpoint{3.786834in}{2.376679in}}%
\pgfpathlineto{\pgfqpoint{3.786834in}{2.372421in}}%
\pgfpathmoveto{\pgfqpoint{3.782576in}{2.376679in}}%
\pgfpathlineto{\pgfqpoint{3.782576in}{2.376679in}}%
\pgfpathlineto{\pgfqpoint{3.782576in}{2.380936in}}%
\pgfpathlineto{\pgfqpoint{3.786834in}{2.380936in}}%
\pgfpathlineto{\pgfqpoint{3.786834in}{2.376679in}}%
\pgfpathmoveto{\pgfqpoint{3.774061in}{2.380936in}}%
\pgfpathlineto{\pgfqpoint{3.774061in}{2.380936in}}%
\pgfpathlineto{\pgfqpoint{3.774061in}{2.385194in}}%
\pgfpathlineto{\pgfqpoint{3.778318in}{2.385194in}}%
\pgfpathlineto{\pgfqpoint{3.778318in}{2.380936in}}%
\pgfpathmoveto{\pgfqpoint{3.774061in}{2.385194in}}%
\pgfpathlineto{\pgfqpoint{3.774061in}{2.385194in}}%
\pgfpathlineto{\pgfqpoint{3.774061in}{2.389452in}}%
\pgfpathlineto{\pgfqpoint{3.778318in}{2.389452in}}%
\pgfpathlineto{\pgfqpoint{3.778318in}{2.385194in}}%
\pgfpathmoveto{\pgfqpoint{3.778318in}{2.380936in}}%
\pgfpathlineto{\pgfqpoint{3.778318in}{2.380936in}}%
\pgfpathlineto{\pgfqpoint{3.778318in}{2.385194in}}%
\pgfpathlineto{\pgfqpoint{3.782576in}{2.385194in}}%
\pgfpathlineto{\pgfqpoint{3.782576in}{2.380936in}}%
\pgfpathmoveto{\pgfqpoint{3.778318in}{2.385194in}}%
\pgfpathlineto{\pgfqpoint{3.778318in}{2.385194in}}%
\pgfpathlineto{\pgfqpoint{3.778318in}{2.389452in}}%
\pgfpathlineto{\pgfqpoint{3.782576in}{2.389452in}}%
\pgfpathlineto{\pgfqpoint{3.782576in}{2.385194in}}%
\pgfpathmoveto{\pgfqpoint{3.774061in}{2.389452in}}%
\pgfpathlineto{\pgfqpoint{3.774061in}{2.389452in}}%
\pgfpathlineto{\pgfqpoint{3.774061in}{2.393710in}}%
\pgfpathlineto{\pgfqpoint{3.778318in}{2.393710in}}%
\pgfpathlineto{\pgfqpoint{3.778318in}{2.389452in}}%
\pgfpathmoveto{\pgfqpoint{3.774061in}{2.393710in}}%
\pgfpathlineto{\pgfqpoint{3.774061in}{2.393710in}}%
\pgfpathlineto{\pgfqpoint{3.774061in}{2.397967in}}%
\pgfpathlineto{\pgfqpoint{3.778318in}{2.397967in}}%
\pgfpathlineto{\pgfqpoint{3.778318in}{2.393710in}}%
\pgfpathmoveto{\pgfqpoint{3.778318in}{2.389452in}}%
\pgfpathlineto{\pgfqpoint{3.778318in}{2.389452in}}%
\pgfpathlineto{\pgfqpoint{3.778318in}{2.393710in}}%
\pgfpathlineto{\pgfqpoint{3.782576in}{2.393710in}}%
\pgfpathlineto{\pgfqpoint{3.782576in}{2.389452in}}%
\pgfpathmoveto{\pgfqpoint{3.778318in}{2.393710in}}%
\pgfpathlineto{\pgfqpoint{3.778318in}{2.393710in}}%
\pgfpathlineto{\pgfqpoint{3.778318in}{2.397967in}}%
\pgfpathlineto{\pgfqpoint{3.782576in}{2.397967in}}%
\pgfpathlineto{\pgfqpoint{3.782576in}{2.393710in}}%
\pgfpathmoveto{\pgfqpoint{3.774061in}{2.397967in}}%
\pgfpathlineto{\pgfqpoint{3.774061in}{2.397967in}}%
\pgfpathlineto{\pgfqpoint{3.774061in}{2.402225in}}%
\pgfpathlineto{\pgfqpoint{3.778318in}{2.402225in}}%
\pgfpathlineto{\pgfqpoint{3.778318in}{2.397967in}}%
\pgfpathmoveto{\pgfqpoint{3.774061in}{2.402225in}}%
\pgfpathlineto{\pgfqpoint{3.774061in}{2.402225in}}%
\pgfpathlineto{\pgfqpoint{3.774061in}{2.406483in}}%
\pgfpathlineto{\pgfqpoint{3.778318in}{2.406483in}}%
\pgfpathlineto{\pgfqpoint{3.778318in}{2.402225in}}%
\pgfpathmoveto{\pgfqpoint{3.778318in}{2.397967in}}%
\pgfpathlineto{\pgfqpoint{3.778318in}{2.397967in}}%
\pgfpathlineto{\pgfqpoint{3.778318in}{2.402225in}}%
\pgfpathlineto{\pgfqpoint{3.782576in}{2.402225in}}%
\pgfpathlineto{\pgfqpoint{3.782576in}{2.397967in}}%
\pgfpathmoveto{\pgfqpoint{3.778318in}{2.402225in}}%
\pgfpathlineto{\pgfqpoint{3.778318in}{2.402225in}}%
\pgfpathlineto{\pgfqpoint{3.778318in}{2.406483in}}%
\pgfpathlineto{\pgfqpoint{3.782576in}{2.406483in}}%
\pgfpathlineto{\pgfqpoint{3.782576in}{2.402225in}}%
\pgfpathmoveto{\pgfqpoint{3.774061in}{2.406483in}}%
\pgfpathlineto{\pgfqpoint{3.774061in}{2.406483in}}%
\pgfpathlineto{\pgfqpoint{3.774061in}{2.410741in}}%
\pgfpathlineto{\pgfqpoint{3.778318in}{2.410741in}}%
\pgfpathlineto{\pgfqpoint{3.778318in}{2.406483in}}%
\pgfpathmoveto{\pgfqpoint{3.774061in}{2.410741in}}%
\pgfpathlineto{\pgfqpoint{3.774061in}{2.410741in}}%
\pgfpathlineto{\pgfqpoint{3.774061in}{2.414998in}}%
\pgfpathlineto{\pgfqpoint{3.778318in}{2.414998in}}%
\pgfpathlineto{\pgfqpoint{3.778318in}{2.410741in}}%
\pgfpathmoveto{\pgfqpoint{3.774061in}{2.414998in}}%
\pgfpathlineto{\pgfqpoint{3.774061in}{2.414998in}}%
\pgfpathlineto{\pgfqpoint{3.774061in}{2.419256in}}%
\pgfpathlineto{\pgfqpoint{3.778318in}{2.419256in}}%
\pgfpathlineto{\pgfqpoint{3.778318in}{2.414998in}}%
\pgfpathmoveto{\pgfqpoint{3.774061in}{2.419256in}}%
\pgfpathlineto{\pgfqpoint{3.774061in}{2.419256in}}%
\pgfpathlineto{\pgfqpoint{3.774061in}{2.423514in}}%
\pgfpathlineto{\pgfqpoint{3.778318in}{2.423514in}}%
\pgfpathlineto{\pgfqpoint{3.778318in}{2.419256in}}%
\pgfpathmoveto{\pgfqpoint{3.774061in}{2.423514in}}%
\pgfpathlineto{\pgfqpoint{3.774061in}{2.423514in}}%
\pgfpathlineto{\pgfqpoint{3.774061in}{2.427772in}}%
\pgfpathlineto{\pgfqpoint{3.778318in}{2.427772in}}%
\pgfpathlineto{\pgfqpoint{3.778318in}{2.423514in}}%
\pgfpathmoveto{\pgfqpoint{3.774061in}{2.427772in}}%
\pgfpathlineto{\pgfqpoint{3.774061in}{2.427772in}}%
\pgfpathlineto{\pgfqpoint{3.774061in}{2.432030in}}%
\pgfpathlineto{\pgfqpoint{3.778318in}{2.432030in}}%
\pgfpathlineto{\pgfqpoint{3.778318in}{2.427772in}}%
\pgfpathmoveto{\pgfqpoint{3.774061in}{2.432030in}}%
\pgfpathlineto{\pgfqpoint{3.774061in}{2.432030in}}%
\pgfpathlineto{\pgfqpoint{3.774061in}{2.436288in}}%
\pgfpathlineto{\pgfqpoint{3.778318in}{2.436288in}}%
\pgfpathlineto{\pgfqpoint{3.778318in}{2.432030in}}%
\pgfpathmoveto{\pgfqpoint{4.025272in}{0.264804in}}%
\pgfpathlineto{\pgfqpoint{4.025272in}{0.264804in}}%
\pgfpathlineto{\pgfqpoint{4.025272in}{0.269062in}}%
\pgfpathlineto{\pgfqpoint{4.029529in}{0.269062in}}%
\pgfpathlineto{\pgfqpoint{4.029529in}{0.264804in}}%
\pgfpathmoveto{\pgfqpoint{4.029529in}{0.234999in}}%
\pgfpathlineto{\pgfqpoint{4.029529in}{0.234999in}}%
\pgfpathlineto{\pgfqpoint{4.029529in}{0.239257in}}%
\pgfpathlineto{\pgfqpoint{4.033787in}{0.239257in}}%
\pgfpathlineto{\pgfqpoint{4.033787in}{0.234999in}}%
\pgfpathmoveto{\pgfqpoint{4.029529in}{0.239257in}}%
\pgfpathlineto{\pgfqpoint{4.029529in}{0.239257in}}%
\pgfpathlineto{\pgfqpoint{4.029529in}{0.243515in}}%
\pgfpathlineto{\pgfqpoint{4.033787in}{0.243515in}}%
\pgfpathlineto{\pgfqpoint{4.033787in}{0.239257in}}%
\pgfpathmoveto{\pgfqpoint{4.033787in}{0.234999in}}%
\pgfpathlineto{\pgfqpoint{4.033787in}{0.234999in}}%
\pgfpathlineto{\pgfqpoint{4.033787in}{0.239257in}}%
\pgfpathlineto{\pgfqpoint{4.038045in}{0.239257in}}%
\pgfpathlineto{\pgfqpoint{4.038045in}{0.234999in}}%
\pgfpathmoveto{\pgfqpoint{4.033787in}{0.239257in}}%
\pgfpathlineto{\pgfqpoint{4.033787in}{0.239257in}}%
\pgfpathlineto{\pgfqpoint{4.033787in}{0.243515in}}%
\pgfpathlineto{\pgfqpoint{4.038045in}{0.243515in}}%
\pgfpathlineto{\pgfqpoint{4.038045in}{0.239257in}}%
\pgfpathmoveto{\pgfqpoint{4.029529in}{0.243515in}}%
\pgfpathlineto{\pgfqpoint{4.029529in}{0.243515in}}%
\pgfpathlineto{\pgfqpoint{4.029529in}{0.247772in}}%
\pgfpathlineto{\pgfqpoint{4.033787in}{0.247772in}}%
\pgfpathlineto{\pgfqpoint{4.033787in}{0.243515in}}%
\pgfpathmoveto{\pgfqpoint{4.029529in}{0.247772in}}%
\pgfpathlineto{\pgfqpoint{4.029529in}{0.247772in}}%
\pgfpathlineto{\pgfqpoint{4.029529in}{0.252030in}}%
\pgfpathlineto{\pgfqpoint{4.033787in}{0.252030in}}%
\pgfpathlineto{\pgfqpoint{4.033787in}{0.247772in}}%
\pgfpathmoveto{\pgfqpoint{4.033787in}{0.243515in}}%
\pgfpathlineto{\pgfqpoint{4.033787in}{0.243515in}}%
\pgfpathlineto{\pgfqpoint{4.033787in}{0.247772in}}%
\pgfpathlineto{\pgfqpoint{4.038045in}{0.247772in}}%
\pgfpathlineto{\pgfqpoint{4.038045in}{0.243515in}}%
\pgfpathmoveto{\pgfqpoint{4.033787in}{0.247772in}}%
\pgfpathlineto{\pgfqpoint{4.033787in}{0.247772in}}%
\pgfpathlineto{\pgfqpoint{4.033787in}{0.252030in}}%
\pgfpathlineto{\pgfqpoint{4.038045in}{0.252030in}}%
\pgfpathlineto{\pgfqpoint{4.038045in}{0.247772in}}%
\pgfpathmoveto{\pgfqpoint{4.029529in}{0.252030in}}%
\pgfpathlineto{\pgfqpoint{4.029529in}{0.252030in}}%
\pgfpathlineto{\pgfqpoint{4.029529in}{0.256288in}}%
\pgfpathlineto{\pgfqpoint{4.033787in}{0.256288in}}%
\pgfpathlineto{\pgfqpoint{4.033787in}{0.252030in}}%
\pgfpathmoveto{\pgfqpoint{4.029529in}{0.256288in}}%
\pgfpathlineto{\pgfqpoint{4.029529in}{0.256288in}}%
\pgfpathlineto{\pgfqpoint{4.029529in}{0.260546in}}%
\pgfpathlineto{\pgfqpoint{4.033787in}{0.260546in}}%
\pgfpathlineto{\pgfqpoint{4.033787in}{0.256288in}}%
\pgfpathmoveto{\pgfqpoint{4.033787in}{0.252030in}}%
\pgfpathlineto{\pgfqpoint{4.033787in}{0.252030in}}%
\pgfpathlineto{\pgfqpoint{4.033787in}{0.256288in}}%
\pgfpathlineto{\pgfqpoint{4.038045in}{0.256288in}}%
\pgfpathlineto{\pgfqpoint{4.038045in}{0.252030in}}%
\pgfpathmoveto{\pgfqpoint{4.029529in}{0.260546in}}%
\pgfpathlineto{\pgfqpoint{4.029529in}{0.260546in}}%
\pgfpathlineto{\pgfqpoint{4.029529in}{0.264804in}}%
\pgfpathlineto{\pgfqpoint{4.033787in}{0.264804in}}%
\pgfpathlineto{\pgfqpoint{4.033787in}{0.260546in}}%
\pgfpathmoveto{\pgfqpoint{4.029529in}{0.264804in}}%
\pgfpathlineto{\pgfqpoint{4.029529in}{0.264804in}}%
\pgfpathlineto{\pgfqpoint{4.029529in}{0.269062in}}%
\pgfpathlineto{\pgfqpoint{4.033787in}{0.269062in}}%
\pgfpathlineto{\pgfqpoint{4.033787in}{0.264804in}}%
\pgfpathmoveto{\pgfqpoint{4.025272in}{0.269062in}}%
\pgfpathlineto{\pgfqpoint{4.025272in}{0.269062in}}%
\pgfpathlineto{\pgfqpoint{4.025272in}{0.273320in}}%
\pgfpathlineto{\pgfqpoint{4.029529in}{0.273320in}}%
\pgfpathlineto{\pgfqpoint{4.029529in}{0.269062in}}%
\pgfpathmoveto{\pgfqpoint{4.025272in}{0.273320in}}%
\pgfpathlineto{\pgfqpoint{4.025272in}{0.273320in}}%
\pgfpathlineto{\pgfqpoint{4.025272in}{0.277578in}}%
\pgfpathlineto{\pgfqpoint{4.029529in}{0.277578in}}%
\pgfpathlineto{\pgfqpoint{4.029529in}{0.273320in}}%
\pgfpathmoveto{\pgfqpoint{4.025272in}{0.277578in}}%
\pgfpathlineto{\pgfqpoint{4.025272in}{0.277578in}}%
\pgfpathlineto{\pgfqpoint{4.025272in}{0.281835in}}%
\pgfpathlineto{\pgfqpoint{4.029529in}{0.281835in}}%
\pgfpathlineto{\pgfqpoint{4.029529in}{0.277578in}}%
\pgfpathmoveto{\pgfqpoint{4.025272in}{0.281835in}}%
\pgfpathlineto{\pgfqpoint{4.025272in}{0.281835in}}%
\pgfpathlineto{\pgfqpoint{4.025272in}{0.286093in}}%
\pgfpathlineto{\pgfqpoint{4.029529in}{0.286093in}}%
\pgfpathlineto{\pgfqpoint{4.029529in}{0.281835in}}%
\pgfpathmoveto{\pgfqpoint{4.025272in}{0.286093in}}%
\pgfpathlineto{\pgfqpoint{4.025272in}{0.286093in}}%
\pgfpathlineto{\pgfqpoint{4.025272in}{0.290351in}}%
\pgfpathlineto{\pgfqpoint{4.029529in}{0.290351in}}%
\pgfpathlineto{\pgfqpoint{4.029529in}{0.286093in}}%
\pgfpathmoveto{\pgfqpoint{4.025272in}{0.290351in}}%
\pgfpathlineto{\pgfqpoint{4.025272in}{0.290351in}}%
\pgfpathlineto{\pgfqpoint{4.025272in}{0.294609in}}%
\pgfpathlineto{\pgfqpoint{4.029529in}{0.294609in}}%
\pgfpathlineto{\pgfqpoint{4.029529in}{0.290351in}}%
\pgfpathmoveto{\pgfqpoint{4.025272in}{0.294609in}}%
\pgfpathlineto{\pgfqpoint{4.025272in}{0.294609in}}%
\pgfpathlineto{\pgfqpoint{4.025272in}{0.298867in}}%
\pgfpathlineto{\pgfqpoint{4.029529in}{0.298867in}}%
\pgfpathlineto{\pgfqpoint{4.029529in}{0.294609in}}%
\pgfpathmoveto{\pgfqpoint{4.025272in}{0.298867in}}%
\pgfpathlineto{\pgfqpoint{4.025272in}{0.298867in}}%
\pgfpathlineto{\pgfqpoint{4.025272in}{0.303125in}}%
\pgfpathlineto{\pgfqpoint{4.029529in}{0.303125in}}%
\pgfpathlineto{\pgfqpoint{4.029529in}{0.298867in}}%
\pgfpathmoveto{\pgfqpoint{4.029529in}{0.269062in}}%
\pgfpathlineto{\pgfqpoint{4.029529in}{0.269062in}}%
\pgfpathlineto{\pgfqpoint{4.029529in}{0.273320in}}%
\pgfpathlineto{\pgfqpoint{4.033787in}{0.273320in}}%
\pgfpathlineto{\pgfqpoint{4.033787in}{0.269062in}}%
\pgfpathmoveto{\pgfqpoint{4.029529in}{0.273320in}}%
\pgfpathlineto{\pgfqpoint{4.029529in}{0.273320in}}%
\pgfpathlineto{\pgfqpoint{4.029529in}{0.277578in}}%
\pgfpathlineto{\pgfqpoint{4.033787in}{0.277578in}}%
\pgfpathlineto{\pgfqpoint{4.033787in}{0.273320in}}%
\pgfpathmoveto{\pgfqpoint{4.029529in}{0.277578in}}%
\pgfpathlineto{\pgfqpoint{4.029529in}{0.277578in}}%
\pgfpathlineto{\pgfqpoint{4.029529in}{0.281835in}}%
\pgfpathlineto{\pgfqpoint{4.033787in}{0.281835in}}%
\pgfpathlineto{\pgfqpoint{4.033787in}{0.277578in}}%
\pgfpathmoveto{\pgfqpoint{4.029529in}{0.281835in}}%
\pgfpathlineto{\pgfqpoint{4.029529in}{0.281835in}}%
\pgfpathlineto{\pgfqpoint{4.029529in}{0.286093in}}%
\pgfpathlineto{\pgfqpoint{4.033787in}{0.286093in}}%
\pgfpathlineto{\pgfqpoint{4.033787in}{0.281835in}}%
\pgfpathmoveto{\pgfqpoint{4.029529in}{0.286093in}}%
\pgfpathlineto{\pgfqpoint{4.029529in}{0.286093in}}%
\pgfpathlineto{\pgfqpoint{4.029529in}{0.290351in}}%
\pgfpathlineto{\pgfqpoint{4.033787in}{0.290351in}}%
\pgfpathlineto{\pgfqpoint{4.033787in}{0.286093in}}%
\pgfpathmoveto{\pgfqpoint{4.029529in}{0.290351in}}%
\pgfpathlineto{\pgfqpoint{4.029529in}{0.290351in}}%
\pgfpathlineto{\pgfqpoint{4.029529in}{0.294609in}}%
\pgfpathlineto{\pgfqpoint{4.033787in}{0.294609in}}%
\pgfpathlineto{\pgfqpoint{4.033787in}{0.290351in}}%
\pgfpathmoveto{\pgfqpoint{4.029529in}{0.294609in}}%
\pgfpathlineto{\pgfqpoint{4.029529in}{0.294609in}}%
\pgfpathlineto{\pgfqpoint{4.029529in}{0.298867in}}%
\pgfpathlineto{\pgfqpoint{4.033787in}{0.298867in}}%
\pgfpathlineto{\pgfqpoint{4.033787in}{0.294609in}}%
\pgfpathmoveto{\pgfqpoint{4.029529in}{0.298867in}}%
\pgfpathlineto{\pgfqpoint{4.029529in}{0.298867in}}%
\pgfpathlineto{\pgfqpoint{4.029529in}{0.303125in}}%
\pgfpathlineto{\pgfqpoint{4.033787in}{0.303125in}}%
\pgfpathlineto{\pgfqpoint{4.033787in}{0.298867in}}%
\pgfpathmoveto{\pgfqpoint{4.025272in}{0.303125in}}%
\pgfpathlineto{\pgfqpoint{4.025272in}{0.303125in}}%
\pgfpathlineto{\pgfqpoint{4.025272in}{0.307383in}}%
\pgfpathlineto{\pgfqpoint{4.029529in}{0.307383in}}%
\pgfpathlineto{\pgfqpoint{4.029529in}{0.303125in}}%
\pgfpathmoveto{\pgfqpoint{4.025272in}{0.307383in}}%
\pgfpathlineto{\pgfqpoint{4.025272in}{0.307383in}}%
\pgfpathlineto{\pgfqpoint{4.025272in}{0.311641in}}%
\pgfpathlineto{\pgfqpoint{4.029529in}{0.311641in}}%
\pgfpathlineto{\pgfqpoint{4.029529in}{0.307383in}}%
\pgfpathmoveto{\pgfqpoint{4.021014in}{0.311641in}}%
\pgfpathlineto{\pgfqpoint{4.021014in}{0.311641in}}%
\pgfpathlineto{\pgfqpoint{4.021014in}{0.315898in}}%
\pgfpathlineto{\pgfqpoint{4.025272in}{0.315898in}}%
\pgfpathlineto{\pgfqpoint{4.025272in}{0.311641in}}%
\pgfpathmoveto{\pgfqpoint{4.021014in}{0.315898in}}%
\pgfpathlineto{\pgfqpoint{4.021014in}{0.315898in}}%
\pgfpathlineto{\pgfqpoint{4.021014in}{0.320156in}}%
\pgfpathlineto{\pgfqpoint{4.025272in}{0.320156in}}%
\pgfpathlineto{\pgfqpoint{4.025272in}{0.315898in}}%
\pgfpathmoveto{\pgfqpoint{4.025272in}{0.311641in}}%
\pgfpathlineto{\pgfqpoint{4.025272in}{0.311641in}}%
\pgfpathlineto{\pgfqpoint{4.025272in}{0.315898in}}%
\pgfpathlineto{\pgfqpoint{4.029529in}{0.315898in}}%
\pgfpathlineto{\pgfqpoint{4.029529in}{0.311641in}}%
\pgfpathmoveto{\pgfqpoint{4.025272in}{0.315898in}}%
\pgfpathlineto{\pgfqpoint{4.025272in}{0.315898in}}%
\pgfpathlineto{\pgfqpoint{4.025272in}{0.320156in}}%
\pgfpathlineto{\pgfqpoint{4.029529in}{0.320156in}}%
\pgfpathlineto{\pgfqpoint{4.029529in}{0.315898in}}%
\pgfpathmoveto{\pgfqpoint{4.021014in}{0.320156in}}%
\pgfpathlineto{\pgfqpoint{4.021014in}{0.320156in}}%
\pgfpathlineto{\pgfqpoint{4.021014in}{0.324414in}}%
\pgfpathlineto{\pgfqpoint{4.025272in}{0.324414in}}%
\pgfpathlineto{\pgfqpoint{4.025272in}{0.320156in}}%
\pgfpathmoveto{\pgfqpoint{4.021014in}{0.324414in}}%
\pgfpathlineto{\pgfqpoint{4.021014in}{0.324414in}}%
\pgfpathlineto{\pgfqpoint{4.021014in}{0.328672in}}%
\pgfpathlineto{\pgfqpoint{4.025272in}{0.328672in}}%
\pgfpathlineto{\pgfqpoint{4.025272in}{0.324414in}}%
\pgfpathmoveto{\pgfqpoint{4.025272in}{0.320156in}}%
\pgfpathlineto{\pgfqpoint{4.025272in}{0.320156in}}%
\pgfpathlineto{\pgfqpoint{4.025272in}{0.324414in}}%
\pgfpathlineto{\pgfqpoint{4.029529in}{0.324414in}}%
\pgfpathlineto{\pgfqpoint{4.029529in}{0.320156in}}%
\pgfpathmoveto{\pgfqpoint{4.025272in}{0.324414in}}%
\pgfpathlineto{\pgfqpoint{4.025272in}{0.324414in}}%
\pgfpathlineto{\pgfqpoint{4.025272in}{0.328672in}}%
\pgfpathlineto{\pgfqpoint{4.029529in}{0.328672in}}%
\pgfpathlineto{\pgfqpoint{4.029529in}{0.324414in}}%
\pgfpathmoveto{\pgfqpoint{4.021014in}{0.328672in}}%
\pgfpathlineto{\pgfqpoint{4.021014in}{0.328672in}}%
\pgfpathlineto{\pgfqpoint{4.021014in}{0.332930in}}%
\pgfpathlineto{\pgfqpoint{4.025272in}{0.332930in}}%
\pgfpathlineto{\pgfqpoint{4.025272in}{0.328672in}}%
\pgfpathmoveto{\pgfqpoint{4.021014in}{0.332930in}}%
\pgfpathlineto{\pgfqpoint{4.021014in}{0.332930in}}%
\pgfpathlineto{\pgfqpoint{4.021014in}{0.337188in}}%
\pgfpathlineto{\pgfqpoint{4.025272in}{0.337188in}}%
\pgfpathlineto{\pgfqpoint{4.025272in}{0.332930in}}%
\pgfpathmoveto{\pgfqpoint{4.025272in}{0.328672in}}%
\pgfpathlineto{\pgfqpoint{4.025272in}{0.328672in}}%
\pgfpathlineto{\pgfqpoint{4.025272in}{0.332930in}}%
\pgfpathlineto{\pgfqpoint{4.029529in}{0.332930in}}%
\pgfpathlineto{\pgfqpoint{4.029529in}{0.328672in}}%
\pgfpathmoveto{\pgfqpoint{4.025272in}{0.332930in}}%
\pgfpathlineto{\pgfqpoint{4.025272in}{0.332930in}}%
\pgfpathlineto{\pgfqpoint{4.025272in}{0.337188in}}%
\pgfpathlineto{\pgfqpoint{4.029529in}{0.337188in}}%
\pgfpathlineto{\pgfqpoint{4.029529in}{0.332930in}}%
\pgfpathmoveto{\pgfqpoint{4.021014in}{0.337188in}}%
\pgfpathlineto{\pgfqpoint{4.021014in}{0.337188in}}%
\pgfpathlineto{\pgfqpoint{4.021014in}{0.341446in}}%
\pgfpathlineto{\pgfqpoint{4.025272in}{0.341446in}}%
\pgfpathlineto{\pgfqpoint{4.025272in}{0.337188in}}%
\pgfpathmoveto{\pgfqpoint{4.021014in}{0.341446in}}%
\pgfpathlineto{\pgfqpoint{4.021014in}{0.341446in}}%
\pgfpathlineto{\pgfqpoint{4.021014in}{0.345704in}}%
\pgfpathlineto{\pgfqpoint{4.025272in}{0.345704in}}%
\pgfpathlineto{\pgfqpoint{4.025272in}{0.341446in}}%
\pgfpathmoveto{\pgfqpoint{4.025272in}{0.337188in}}%
\pgfpathlineto{\pgfqpoint{4.025272in}{0.337188in}}%
\pgfpathlineto{\pgfqpoint{4.025272in}{0.341446in}}%
\pgfpathlineto{\pgfqpoint{4.029529in}{0.341446in}}%
\pgfpathlineto{\pgfqpoint{4.029529in}{0.337188in}}%
\pgfpathmoveto{\pgfqpoint{4.025272in}{0.341446in}}%
\pgfpathlineto{\pgfqpoint{4.025272in}{0.341446in}}%
\pgfpathlineto{\pgfqpoint{4.025272in}{0.345704in}}%
\pgfpathlineto{\pgfqpoint{4.029529in}{0.345704in}}%
\pgfpathlineto{\pgfqpoint{4.029529in}{0.341446in}}%
\pgfpathmoveto{\pgfqpoint{4.021014in}{0.345704in}}%
\pgfpathlineto{\pgfqpoint{4.021014in}{0.345704in}}%
\pgfpathlineto{\pgfqpoint{4.021014in}{0.349961in}}%
\pgfpathlineto{\pgfqpoint{4.025272in}{0.349961in}}%
\pgfpathlineto{\pgfqpoint{4.025272in}{0.345704in}}%
\pgfpathmoveto{\pgfqpoint{4.021014in}{0.349961in}}%
\pgfpathlineto{\pgfqpoint{4.021014in}{0.349961in}}%
\pgfpathlineto{\pgfqpoint{4.021014in}{0.354219in}}%
\pgfpathlineto{\pgfqpoint{4.025272in}{0.354219in}}%
\pgfpathlineto{\pgfqpoint{4.025272in}{0.349961in}}%
\pgfpathmoveto{\pgfqpoint{4.025272in}{0.345704in}}%
\pgfpathlineto{\pgfqpoint{4.025272in}{0.345704in}}%
\pgfpathlineto{\pgfqpoint{4.025272in}{0.349961in}}%
\pgfpathlineto{\pgfqpoint{4.029529in}{0.349961in}}%
\pgfpathlineto{\pgfqpoint{4.029529in}{0.345704in}}%
\pgfpathmoveto{\pgfqpoint{4.016756in}{0.354219in}}%
\pgfpathlineto{\pgfqpoint{4.016756in}{0.354219in}}%
\pgfpathlineto{\pgfqpoint{4.016756in}{0.358477in}}%
\pgfpathlineto{\pgfqpoint{4.021014in}{0.358477in}}%
\pgfpathlineto{\pgfqpoint{4.021014in}{0.354219in}}%
\pgfpathmoveto{\pgfqpoint{4.016756in}{0.358477in}}%
\pgfpathlineto{\pgfqpoint{4.016756in}{0.358477in}}%
\pgfpathlineto{\pgfqpoint{4.016756in}{0.362735in}}%
\pgfpathlineto{\pgfqpoint{4.021014in}{0.362735in}}%
\pgfpathlineto{\pgfqpoint{4.021014in}{0.358477in}}%
\pgfpathmoveto{\pgfqpoint{4.016756in}{0.362735in}}%
\pgfpathlineto{\pgfqpoint{4.016756in}{0.362735in}}%
\pgfpathlineto{\pgfqpoint{4.016756in}{0.366993in}}%
\pgfpathlineto{\pgfqpoint{4.021014in}{0.366993in}}%
\pgfpathlineto{\pgfqpoint{4.021014in}{0.362735in}}%
\pgfpathmoveto{\pgfqpoint{4.016756in}{0.366993in}}%
\pgfpathlineto{\pgfqpoint{4.016756in}{0.366993in}}%
\pgfpathlineto{\pgfqpoint{4.016756in}{0.371251in}}%
\pgfpathlineto{\pgfqpoint{4.021014in}{0.371251in}}%
\pgfpathlineto{\pgfqpoint{4.021014in}{0.366993in}}%
\pgfpathmoveto{\pgfqpoint{4.021014in}{0.354219in}}%
\pgfpathlineto{\pgfqpoint{4.021014in}{0.354219in}}%
\pgfpathlineto{\pgfqpoint{4.021014in}{0.358477in}}%
\pgfpathlineto{\pgfqpoint{4.025272in}{0.358477in}}%
\pgfpathlineto{\pgfqpoint{4.025272in}{0.354219in}}%
\pgfpathmoveto{\pgfqpoint{4.021014in}{0.358477in}}%
\pgfpathlineto{\pgfqpoint{4.021014in}{0.358477in}}%
\pgfpathlineto{\pgfqpoint{4.021014in}{0.362735in}}%
\pgfpathlineto{\pgfqpoint{4.025272in}{0.362735in}}%
\pgfpathlineto{\pgfqpoint{4.025272in}{0.358477in}}%
\pgfpathmoveto{\pgfqpoint{4.021014in}{0.362735in}}%
\pgfpathlineto{\pgfqpoint{4.021014in}{0.362735in}}%
\pgfpathlineto{\pgfqpoint{4.021014in}{0.366993in}}%
\pgfpathlineto{\pgfqpoint{4.025272in}{0.366993in}}%
\pgfpathlineto{\pgfqpoint{4.025272in}{0.362735in}}%
\pgfpathmoveto{\pgfqpoint{4.021014in}{0.366993in}}%
\pgfpathlineto{\pgfqpoint{4.021014in}{0.366993in}}%
\pgfpathlineto{\pgfqpoint{4.021014in}{0.371251in}}%
\pgfpathlineto{\pgfqpoint{4.025272in}{0.371251in}}%
\pgfpathlineto{\pgfqpoint{4.025272in}{0.366993in}}%
\pgfpathmoveto{\pgfqpoint{4.016756in}{0.371251in}}%
\pgfpathlineto{\pgfqpoint{4.016756in}{0.371251in}}%
\pgfpathlineto{\pgfqpoint{4.016756in}{0.375509in}}%
\pgfpathlineto{\pgfqpoint{4.021014in}{0.375509in}}%
\pgfpathlineto{\pgfqpoint{4.021014in}{0.371251in}}%
\pgfpathmoveto{\pgfqpoint{4.016756in}{0.375509in}}%
\pgfpathlineto{\pgfqpoint{4.016756in}{0.375509in}}%
\pgfpathlineto{\pgfqpoint{4.016756in}{0.379767in}}%
\pgfpathlineto{\pgfqpoint{4.021014in}{0.379767in}}%
\pgfpathlineto{\pgfqpoint{4.021014in}{0.375509in}}%
\pgfpathmoveto{\pgfqpoint{4.016756in}{0.379767in}}%
\pgfpathlineto{\pgfqpoint{4.016756in}{0.379767in}}%
\pgfpathlineto{\pgfqpoint{4.016756in}{0.384024in}}%
\pgfpathlineto{\pgfqpoint{4.021014in}{0.384024in}}%
\pgfpathlineto{\pgfqpoint{4.021014in}{0.379767in}}%
\pgfpathmoveto{\pgfqpoint{4.016756in}{0.384024in}}%
\pgfpathlineto{\pgfqpoint{4.016756in}{0.384024in}}%
\pgfpathlineto{\pgfqpoint{4.016756in}{0.388282in}}%
\pgfpathlineto{\pgfqpoint{4.021014in}{0.388282in}}%
\pgfpathlineto{\pgfqpoint{4.021014in}{0.384024in}}%
\pgfpathmoveto{\pgfqpoint{4.021014in}{0.371251in}}%
\pgfpathlineto{\pgfqpoint{4.021014in}{0.371251in}}%
\pgfpathlineto{\pgfqpoint{4.021014in}{0.375509in}}%
\pgfpathlineto{\pgfqpoint{4.025272in}{0.375509in}}%
\pgfpathlineto{\pgfqpoint{4.025272in}{0.371251in}}%
\pgfpathmoveto{\pgfqpoint{4.021014in}{0.375509in}}%
\pgfpathlineto{\pgfqpoint{4.021014in}{0.375509in}}%
\pgfpathlineto{\pgfqpoint{4.021014in}{0.379767in}}%
\pgfpathlineto{\pgfqpoint{4.025272in}{0.379767in}}%
\pgfpathlineto{\pgfqpoint{4.025272in}{0.375509in}}%
\pgfpathmoveto{\pgfqpoint{4.021014in}{0.379767in}}%
\pgfpathlineto{\pgfqpoint{4.021014in}{0.379767in}}%
\pgfpathlineto{\pgfqpoint{4.021014in}{0.384024in}}%
\pgfpathlineto{\pgfqpoint{4.025272in}{0.384024in}}%
\pgfpathlineto{\pgfqpoint{4.025272in}{0.379767in}}%
\pgfpathmoveto{\pgfqpoint{4.021014in}{0.384024in}}%
\pgfpathlineto{\pgfqpoint{4.021014in}{0.384024in}}%
\pgfpathlineto{\pgfqpoint{4.021014in}{0.388282in}}%
\pgfpathlineto{\pgfqpoint{4.025272in}{0.388282in}}%
\pgfpathlineto{\pgfqpoint{4.025272in}{0.384024in}}%
\pgfpathmoveto{\pgfqpoint{4.016756in}{0.388282in}}%
\pgfpathlineto{\pgfqpoint{4.016756in}{0.388282in}}%
\pgfpathlineto{\pgfqpoint{4.016756in}{0.392540in}}%
\pgfpathlineto{\pgfqpoint{4.021014in}{0.392540in}}%
\pgfpathlineto{\pgfqpoint{4.021014in}{0.388282in}}%
\pgfpathmoveto{\pgfqpoint{4.016756in}{0.392540in}}%
\pgfpathlineto{\pgfqpoint{4.016756in}{0.392540in}}%
\pgfpathlineto{\pgfqpoint{4.016756in}{0.396798in}}%
\pgfpathlineto{\pgfqpoint{4.021014in}{0.396798in}}%
\pgfpathlineto{\pgfqpoint{4.021014in}{0.392540in}}%
\pgfpathmoveto{\pgfqpoint{4.012499in}{0.401056in}}%
\pgfpathlineto{\pgfqpoint{4.012499in}{0.401056in}}%
\pgfpathlineto{\pgfqpoint{4.012499in}{0.405313in}}%
\pgfpathlineto{\pgfqpoint{4.016756in}{0.405313in}}%
\pgfpathlineto{\pgfqpoint{4.016756in}{0.401056in}}%
\pgfpathmoveto{\pgfqpoint{4.016756in}{0.396798in}}%
\pgfpathlineto{\pgfqpoint{4.016756in}{0.396798in}}%
\pgfpathlineto{\pgfqpoint{4.016756in}{0.401056in}}%
\pgfpathlineto{\pgfqpoint{4.021014in}{0.401056in}}%
\pgfpathlineto{\pgfqpoint{4.021014in}{0.396798in}}%
\pgfpathmoveto{\pgfqpoint{4.016756in}{0.401056in}}%
\pgfpathlineto{\pgfqpoint{4.016756in}{0.401056in}}%
\pgfpathlineto{\pgfqpoint{4.016756in}{0.405313in}}%
\pgfpathlineto{\pgfqpoint{4.021014in}{0.405313in}}%
\pgfpathlineto{\pgfqpoint{4.021014in}{0.401056in}}%
\pgfpathmoveto{\pgfqpoint{4.021014in}{0.388282in}}%
\pgfpathlineto{\pgfqpoint{4.021014in}{0.388282in}}%
\pgfpathlineto{\pgfqpoint{4.021014in}{0.392540in}}%
\pgfpathlineto{\pgfqpoint{4.025272in}{0.392540in}}%
\pgfpathlineto{\pgfqpoint{4.025272in}{0.388282in}}%
\pgfpathmoveto{\pgfqpoint{4.012499in}{0.405313in}}%
\pgfpathlineto{\pgfqpoint{4.012499in}{0.405313in}}%
\pgfpathlineto{\pgfqpoint{4.012499in}{0.409571in}}%
\pgfpathlineto{\pgfqpoint{4.016756in}{0.409571in}}%
\pgfpathlineto{\pgfqpoint{4.016756in}{0.405313in}}%
\pgfpathmoveto{\pgfqpoint{4.012499in}{0.409571in}}%
\pgfpathlineto{\pgfqpoint{4.012499in}{0.409571in}}%
\pgfpathlineto{\pgfqpoint{4.012499in}{0.413829in}}%
\pgfpathlineto{\pgfqpoint{4.016756in}{0.413829in}}%
\pgfpathlineto{\pgfqpoint{4.016756in}{0.409571in}}%
\pgfpathmoveto{\pgfqpoint{4.016756in}{0.405313in}}%
\pgfpathlineto{\pgfqpoint{4.016756in}{0.405313in}}%
\pgfpathlineto{\pgfqpoint{4.016756in}{0.409571in}}%
\pgfpathlineto{\pgfqpoint{4.021014in}{0.409571in}}%
\pgfpathlineto{\pgfqpoint{4.021014in}{0.405313in}}%
\pgfpathmoveto{\pgfqpoint{4.016756in}{0.409571in}}%
\pgfpathlineto{\pgfqpoint{4.016756in}{0.409571in}}%
\pgfpathlineto{\pgfqpoint{4.016756in}{0.413829in}}%
\pgfpathlineto{\pgfqpoint{4.021014in}{0.413829in}}%
\pgfpathlineto{\pgfqpoint{4.021014in}{0.409571in}}%
\pgfpathmoveto{\pgfqpoint{4.012499in}{0.413829in}}%
\pgfpathlineto{\pgfqpoint{4.012499in}{0.413829in}}%
\pgfpathlineto{\pgfqpoint{4.012499in}{0.418087in}}%
\pgfpathlineto{\pgfqpoint{4.016756in}{0.418087in}}%
\pgfpathlineto{\pgfqpoint{4.016756in}{0.413829in}}%
\pgfpathmoveto{\pgfqpoint{4.012499in}{0.418087in}}%
\pgfpathlineto{\pgfqpoint{4.012499in}{0.418087in}}%
\pgfpathlineto{\pgfqpoint{4.012499in}{0.422345in}}%
\pgfpathlineto{\pgfqpoint{4.016756in}{0.422345in}}%
\pgfpathlineto{\pgfqpoint{4.016756in}{0.418087in}}%
\pgfpathmoveto{\pgfqpoint{4.016756in}{0.413829in}}%
\pgfpathlineto{\pgfqpoint{4.016756in}{0.413829in}}%
\pgfpathlineto{\pgfqpoint{4.016756in}{0.418087in}}%
\pgfpathlineto{\pgfqpoint{4.021014in}{0.418087in}}%
\pgfpathlineto{\pgfqpoint{4.021014in}{0.413829in}}%
\pgfpathmoveto{\pgfqpoint{4.016756in}{0.418087in}}%
\pgfpathlineto{\pgfqpoint{4.016756in}{0.418087in}}%
\pgfpathlineto{\pgfqpoint{4.016756in}{0.422345in}}%
\pgfpathlineto{\pgfqpoint{4.021014in}{0.422345in}}%
\pgfpathlineto{\pgfqpoint{4.021014in}{0.418087in}}%
\pgfpathmoveto{\pgfqpoint{4.012499in}{0.422345in}}%
\pgfpathlineto{\pgfqpoint{4.012499in}{0.422345in}}%
\pgfpathlineto{\pgfqpoint{4.012499in}{0.426603in}}%
\pgfpathlineto{\pgfqpoint{4.016756in}{0.426603in}}%
\pgfpathlineto{\pgfqpoint{4.016756in}{0.422345in}}%
\pgfpathmoveto{\pgfqpoint{4.012499in}{0.426603in}}%
\pgfpathlineto{\pgfqpoint{4.012499in}{0.426603in}}%
\pgfpathlineto{\pgfqpoint{4.012499in}{0.430860in}}%
\pgfpathlineto{\pgfqpoint{4.016756in}{0.430860in}}%
\pgfpathlineto{\pgfqpoint{4.016756in}{0.426603in}}%
\pgfpathmoveto{\pgfqpoint{4.016756in}{0.422345in}}%
\pgfpathlineto{\pgfqpoint{4.016756in}{0.422345in}}%
\pgfpathlineto{\pgfqpoint{4.016756in}{0.426603in}}%
\pgfpathlineto{\pgfqpoint{4.021014in}{0.426603in}}%
\pgfpathlineto{\pgfqpoint{4.021014in}{0.422345in}}%
\pgfpathmoveto{\pgfqpoint{4.016756in}{0.426603in}}%
\pgfpathlineto{\pgfqpoint{4.016756in}{0.426603in}}%
\pgfpathlineto{\pgfqpoint{4.016756in}{0.430860in}}%
\pgfpathlineto{\pgfqpoint{4.021014in}{0.430860in}}%
\pgfpathlineto{\pgfqpoint{4.021014in}{0.426603in}}%
\pgfpathmoveto{\pgfqpoint{4.012499in}{0.430860in}}%
\pgfpathlineto{\pgfqpoint{4.012499in}{0.430860in}}%
\pgfpathlineto{\pgfqpoint{4.012499in}{0.435118in}}%
\pgfpathlineto{\pgfqpoint{4.016756in}{0.435118in}}%
\pgfpathlineto{\pgfqpoint{4.016756in}{0.430860in}}%
\pgfpathmoveto{\pgfqpoint{4.012499in}{0.435118in}}%
\pgfpathlineto{\pgfqpoint{4.012499in}{0.435118in}}%
\pgfpathlineto{\pgfqpoint{4.012499in}{0.439376in}}%
\pgfpathlineto{\pgfqpoint{4.016756in}{0.439376in}}%
\pgfpathlineto{\pgfqpoint{4.016756in}{0.435118in}}%
\pgfpathmoveto{\pgfqpoint{4.016756in}{0.430860in}}%
\pgfpathlineto{\pgfqpoint{4.016756in}{0.430860in}}%
\pgfpathlineto{\pgfqpoint{4.016756in}{0.435118in}}%
\pgfpathlineto{\pgfqpoint{4.021014in}{0.435118in}}%
\pgfpathlineto{\pgfqpoint{4.021014in}{0.430860in}}%
\pgfpathmoveto{\pgfqpoint{4.016756in}{0.435118in}}%
\pgfpathlineto{\pgfqpoint{4.016756in}{0.435118in}}%
\pgfpathlineto{\pgfqpoint{4.016756in}{0.439376in}}%
\pgfpathlineto{\pgfqpoint{4.021014in}{0.439376in}}%
\pgfpathlineto{\pgfqpoint{4.021014in}{0.435118in}}%
\pgfpathmoveto{\pgfqpoint{4.008241in}{0.443634in}}%
\pgfpathlineto{\pgfqpoint{4.008241in}{0.443634in}}%
\pgfpathlineto{\pgfqpoint{4.008241in}{0.447892in}}%
\pgfpathlineto{\pgfqpoint{4.012499in}{0.447892in}}%
\pgfpathlineto{\pgfqpoint{4.012499in}{0.443634in}}%
\pgfpathmoveto{\pgfqpoint{4.008241in}{0.447892in}}%
\pgfpathlineto{\pgfqpoint{4.008241in}{0.447892in}}%
\pgfpathlineto{\pgfqpoint{4.008241in}{0.452150in}}%
\pgfpathlineto{\pgfqpoint{4.012499in}{0.452150in}}%
\pgfpathlineto{\pgfqpoint{4.012499in}{0.447892in}}%
\pgfpathmoveto{\pgfqpoint{4.008241in}{0.452150in}}%
\pgfpathlineto{\pgfqpoint{4.008241in}{0.452150in}}%
\pgfpathlineto{\pgfqpoint{4.008241in}{0.456407in}}%
\pgfpathlineto{\pgfqpoint{4.012499in}{0.456407in}}%
\pgfpathlineto{\pgfqpoint{4.012499in}{0.452150in}}%
\pgfpathmoveto{\pgfqpoint{4.008241in}{0.456407in}}%
\pgfpathlineto{\pgfqpoint{4.008241in}{0.456407in}}%
\pgfpathlineto{\pgfqpoint{4.008241in}{0.460665in}}%
\pgfpathlineto{\pgfqpoint{4.012499in}{0.460665in}}%
\pgfpathlineto{\pgfqpoint{4.012499in}{0.456407in}}%
\pgfpathmoveto{\pgfqpoint{4.008241in}{0.460665in}}%
\pgfpathlineto{\pgfqpoint{4.008241in}{0.460665in}}%
\pgfpathlineto{\pgfqpoint{4.008241in}{0.464923in}}%
\pgfpathlineto{\pgfqpoint{4.012499in}{0.464923in}}%
\pgfpathlineto{\pgfqpoint{4.012499in}{0.460665in}}%
\pgfpathmoveto{\pgfqpoint{4.008241in}{0.464923in}}%
\pgfpathlineto{\pgfqpoint{4.008241in}{0.464923in}}%
\pgfpathlineto{\pgfqpoint{4.008241in}{0.469181in}}%
\pgfpathlineto{\pgfqpoint{4.012499in}{0.469181in}}%
\pgfpathlineto{\pgfqpoint{4.012499in}{0.464923in}}%
\pgfpathmoveto{\pgfqpoint{4.008241in}{0.469181in}}%
\pgfpathlineto{\pgfqpoint{4.008241in}{0.469181in}}%
\pgfpathlineto{\pgfqpoint{4.008241in}{0.473439in}}%
\pgfpathlineto{\pgfqpoint{4.012499in}{0.473439in}}%
\pgfpathlineto{\pgfqpoint{4.012499in}{0.469181in}}%
\pgfpathmoveto{\pgfqpoint{4.008241in}{0.473439in}}%
\pgfpathlineto{\pgfqpoint{4.008241in}{0.473439in}}%
\pgfpathlineto{\pgfqpoint{4.008241in}{0.477696in}}%
\pgfpathlineto{\pgfqpoint{4.012499in}{0.477696in}}%
\pgfpathlineto{\pgfqpoint{4.012499in}{0.473439in}}%
\pgfpathmoveto{\pgfqpoint{4.008241in}{0.477696in}}%
\pgfpathlineto{\pgfqpoint{4.008241in}{0.477696in}}%
\pgfpathlineto{\pgfqpoint{4.008241in}{0.481954in}}%
\pgfpathlineto{\pgfqpoint{4.012499in}{0.481954in}}%
\pgfpathlineto{\pgfqpoint{4.012499in}{0.477696in}}%
\pgfpathmoveto{\pgfqpoint{4.003983in}{0.486212in}}%
\pgfpathlineto{\pgfqpoint{4.003983in}{0.486212in}}%
\pgfpathlineto{\pgfqpoint{4.003983in}{0.490470in}}%
\pgfpathlineto{\pgfqpoint{4.008241in}{0.490470in}}%
\pgfpathlineto{\pgfqpoint{4.008241in}{0.486212in}}%
\pgfpathmoveto{\pgfqpoint{4.008241in}{0.481954in}}%
\pgfpathlineto{\pgfqpoint{4.008241in}{0.481954in}}%
\pgfpathlineto{\pgfqpoint{4.008241in}{0.486212in}}%
\pgfpathlineto{\pgfqpoint{4.012499in}{0.486212in}}%
\pgfpathlineto{\pgfqpoint{4.012499in}{0.481954in}}%
\pgfpathmoveto{\pgfqpoint{4.008241in}{0.486212in}}%
\pgfpathlineto{\pgfqpoint{4.008241in}{0.486212in}}%
\pgfpathlineto{\pgfqpoint{4.008241in}{0.490470in}}%
\pgfpathlineto{\pgfqpoint{4.012499in}{0.490470in}}%
\pgfpathlineto{\pgfqpoint{4.012499in}{0.486212in}}%
\pgfpathmoveto{\pgfqpoint{4.003983in}{0.490470in}}%
\pgfpathlineto{\pgfqpoint{4.003983in}{0.490470in}}%
\pgfpathlineto{\pgfqpoint{4.003983in}{0.494728in}}%
\pgfpathlineto{\pgfqpoint{4.008241in}{0.494728in}}%
\pgfpathlineto{\pgfqpoint{4.008241in}{0.490470in}}%
\pgfpathmoveto{\pgfqpoint{4.003983in}{0.494728in}}%
\pgfpathlineto{\pgfqpoint{4.003983in}{0.494728in}}%
\pgfpathlineto{\pgfqpoint{4.003983in}{0.498986in}}%
\pgfpathlineto{\pgfqpoint{4.008241in}{0.498986in}}%
\pgfpathlineto{\pgfqpoint{4.008241in}{0.494728in}}%
\pgfpathmoveto{\pgfqpoint{4.008241in}{0.490470in}}%
\pgfpathlineto{\pgfqpoint{4.008241in}{0.490470in}}%
\pgfpathlineto{\pgfqpoint{4.008241in}{0.494728in}}%
\pgfpathlineto{\pgfqpoint{4.012499in}{0.494728in}}%
\pgfpathlineto{\pgfqpoint{4.012499in}{0.490470in}}%
\pgfpathmoveto{\pgfqpoint{4.008241in}{0.494728in}}%
\pgfpathlineto{\pgfqpoint{4.008241in}{0.494728in}}%
\pgfpathlineto{\pgfqpoint{4.008241in}{0.498986in}}%
\pgfpathlineto{\pgfqpoint{4.012499in}{0.498986in}}%
\pgfpathlineto{\pgfqpoint{4.012499in}{0.494728in}}%
\pgfpathmoveto{\pgfqpoint{4.003983in}{0.498986in}}%
\pgfpathlineto{\pgfqpoint{4.003983in}{0.498986in}}%
\pgfpathlineto{\pgfqpoint{4.003983in}{0.503243in}}%
\pgfpathlineto{\pgfqpoint{4.008241in}{0.503243in}}%
\pgfpathlineto{\pgfqpoint{4.008241in}{0.498986in}}%
\pgfpathmoveto{\pgfqpoint{4.003983in}{0.503243in}}%
\pgfpathlineto{\pgfqpoint{4.003983in}{0.503243in}}%
\pgfpathlineto{\pgfqpoint{4.003983in}{0.507501in}}%
\pgfpathlineto{\pgfqpoint{4.008241in}{0.507501in}}%
\pgfpathlineto{\pgfqpoint{4.008241in}{0.503243in}}%
\pgfpathmoveto{\pgfqpoint{4.008241in}{0.498986in}}%
\pgfpathlineto{\pgfqpoint{4.008241in}{0.498986in}}%
\pgfpathlineto{\pgfqpoint{4.008241in}{0.503243in}}%
\pgfpathlineto{\pgfqpoint{4.012499in}{0.503243in}}%
\pgfpathlineto{\pgfqpoint{4.012499in}{0.498986in}}%
\pgfpathmoveto{\pgfqpoint{4.008241in}{0.503243in}}%
\pgfpathlineto{\pgfqpoint{4.008241in}{0.503243in}}%
\pgfpathlineto{\pgfqpoint{4.008241in}{0.507501in}}%
\pgfpathlineto{\pgfqpoint{4.012499in}{0.507501in}}%
\pgfpathlineto{\pgfqpoint{4.012499in}{0.503243in}}%
\pgfpathmoveto{\pgfqpoint{4.012499in}{0.439376in}}%
\pgfpathlineto{\pgfqpoint{4.012499in}{0.439376in}}%
\pgfpathlineto{\pgfqpoint{4.012499in}{0.443634in}}%
\pgfpathlineto{\pgfqpoint{4.016756in}{0.443634in}}%
\pgfpathlineto{\pgfqpoint{4.016756in}{0.439376in}}%
\pgfpathmoveto{\pgfqpoint{4.012499in}{0.443634in}}%
\pgfpathlineto{\pgfqpoint{4.012499in}{0.443634in}}%
\pgfpathlineto{\pgfqpoint{4.012499in}{0.447892in}}%
\pgfpathlineto{\pgfqpoint{4.016756in}{0.447892in}}%
\pgfpathlineto{\pgfqpoint{4.016756in}{0.443634in}}%
\pgfpathmoveto{\pgfqpoint{4.012499in}{0.447892in}}%
\pgfpathlineto{\pgfqpoint{4.012499in}{0.447892in}}%
\pgfpathlineto{\pgfqpoint{4.012499in}{0.452150in}}%
\pgfpathlineto{\pgfqpoint{4.016756in}{0.452150in}}%
\pgfpathlineto{\pgfqpoint{4.016756in}{0.447892in}}%
\pgfpathmoveto{\pgfqpoint{4.012499in}{0.452150in}}%
\pgfpathlineto{\pgfqpoint{4.012499in}{0.452150in}}%
\pgfpathlineto{\pgfqpoint{4.012499in}{0.456407in}}%
\pgfpathlineto{\pgfqpoint{4.016756in}{0.456407in}}%
\pgfpathlineto{\pgfqpoint{4.016756in}{0.452150in}}%
\pgfpathmoveto{\pgfqpoint{4.012499in}{0.456407in}}%
\pgfpathlineto{\pgfqpoint{4.012499in}{0.456407in}}%
\pgfpathlineto{\pgfqpoint{4.012499in}{0.460665in}}%
\pgfpathlineto{\pgfqpoint{4.016756in}{0.460665in}}%
\pgfpathlineto{\pgfqpoint{4.016756in}{0.456407in}}%
\pgfpathmoveto{\pgfqpoint{4.012499in}{0.460665in}}%
\pgfpathlineto{\pgfqpoint{4.012499in}{0.460665in}}%
\pgfpathlineto{\pgfqpoint{4.012499in}{0.464923in}}%
\pgfpathlineto{\pgfqpoint{4.016756in}{0.464923in}}%
\pgfpathlineto{\pgfqpoint{4.016756in}{0.460665in}}%
\pgfpathmoveto{\pgfqpoint{4.012499in}{0.464923in}}%
\pgfpathlineto{\pgfqpoint{4.012499in}{0.464923in}}%
\pgfpathlineto{\pgfqpoint{4.012499in}{0.469181in}}%
\pgfpathlineto{\pgfqpoint{4.016756in}{0.469181in}}%
\pgfpathlineto{\pgfqpoint{4.016756in}{0.464923in}}%
\pgfpathmoveto{\pgfqpoint{4.012499in}{0.469181in}}%
\pgfpathlineto{\pgfqpoint{4.012499in}{0.469181in}}%
\pgfpathlineto{\pgfqpoint{4.012499in}{0.473439in}}%
\pgfpathlineto{\pgfqpoint{4.016756in}{0.473439in}}%
\pgfpathlineto{\pgfqpoint{4.016756in}{0.469181in}}%
\pgfpathmoveto{\pgfqpoint{4.012499in}{0.473439in}}%
\pgfpathlineto{\pgfqpoint{4.012499in}{0.473439in}}%
\pgfpathlineto{\pgfqpoint{4.012499in}{0.477696in}}%
\pgfpathlineto{\pgfqpoint{4.016756in}{0.477696in}}%
\pgfpathlineto{\pgfqpoint{4.016756in}{0.473439in}}%
\pgfpathmoveto{\pgfqpoint{4.012499in}{0.477696in}}%
\pgfpathlineto{\pgfqpoint{4.012499in}{0.477696in}}%
\pgfpathlineto{\pgfqpoint{4.012499in}{0.481954in}}%
\pgfpathlineto{\pgfqpoint{4.016756in}{0.481954in}}%
\pgfpathlineto{\pgfqpoint{4.016756in}{0.477696in}}%
\pgfpathmoveto{\pgfqpoint{4.003983in}{0.507501in}}%
\pgfpathlineto{\pgfqpoint{4.003983in}{0.507501in}}%
\pgfpathlineto{\pgfqpoint{4.003983in}{0.511759in}}%
\pgfpathlineto{\pgfqpoint{4.008241in}{0.511759in}}%
\pgfpathlineto{\pgfqpoint{4.008241in}{0.507501in}}%
\pgfpathmoveto{\pgfqpoint{4.003983in}{0.511759in}}%
\pgfpathlineto{\pgfqpoint{4.003983in}{0.511759in}}%
\pgfpathlineto{\pgfqpoint{4.003983in}{0.516017in}}%
\pgfpathlineto{\pgfqpoint{4.008241in}{0.516017in}}%
\pgfpathlineto{\pgfqpoint{4.008241in}{0.511759in}}%
\pgfpathmoveto{\pgfqpoint{4.008241in}{0.507501in}}%
\pgfpathlineto{\pgfqpoint{4.008241in}{0.507501in}}%
\pgfpathlineto{\pgfqpoint{4.008241in}{0.511759in}}%
\pgfpathlineto{\pgfqpoint{4.012499in}{0.511759in}}%
\pgfpathlineto{\pgfqpoint{4.012499in}{0.507501in}}%
\pgfpathmoveto{\pgfqpoint{4.008241in}{0.511759in}}%
\pgfpathlineto{\pgfqpoint{4.008241in}{0.511759in}}%
\pgfpathlineto{\pgfqpoint{4.008241in}{0.516017in}}%
\pgfpathlineto{\pgfqpoint{4.012499in}{0.516017in}}%
\pgfpathlineto{\pgfqpoint{4.012499in}{0.511759in}}%
\pgfpathmoveto{\pgfqpoint{4.003983in}{0.516017in}}%
\pgfpathlineto{\pgfqpoint{4.003983in}{0.516017in}}%
\pgfpathlineto{\pgfqpoint{4.003983in}{0.520274in}}%
\pgfpathlineto{\pgfqpoint{4.008241in}{0.520274in}}%
\pgfpathlineto{\pgfqpoint{4.008241in}{0.516017in}}%
\pgfpathmoveto{\pgfqpoint{4.003983in}{0.520274in}}%
\pgfpathlineto{\pgfqpoint{4.003983in}{0.520274in}}%
\pgfpathlineto{\pgfqpoint{4.003983in}{0.524532in}}%
\pgfpathlineto{\pgfqpoint{4.008241in}{0.524532in}}%
\pgfpathlineto{\pgfqpoint{4.008241in}{0.520274in}}%
\pgfpathmoveto{\pgfqpoint{4.008241in}{0.516017in}}%
\pgfpathlineto{\pgfqpoint{4.008241in}{0.516017in}}%
\pgfpathlineto{\pgfqpoint{4.008241in}{0.520274in}}%
\pgfpathlineto{\pgfqpoint{4.012499in}{0.520274in}}%
\pgfpathlineto{\pgfqpoint{4.012499in}{0.516017in}}%
\pgfpathmoveto{\pgfqpoint{4.008241in}{0.520274in}}%
\pgfpathlineto{\pgfqpoint{4.008241in}{0.520274in}}%
\pgfpathlineto{\pgfqpoint{4.008241in}{0.524532in}}%
\pgfpathlineto{\pgfqpoint{4.012499in}{0.524532in}}%
\pgfpathlineto{\pgfqpoint{4.012499in}{0.520274in}}%
\pgfpathmoveto{\pgfqpoint{3.999726in}{0.533047in}}%
\pgfpathlineto{\pgfqpoint{3.999726in}{0.533047in}}%
\pgfpathlineto{\pgfqpoint{3.999726in}{0.537305in}}%
\pgfpathlineto{\pgfqpoint{4.003983in}{0.537305in}}%
\pgfpathlineto{\pgfqpoint{4.003983in}{0.533047in}}%
\pgfpathmoveto{\pgfqpoint{3.999726in}{0.537305in}}%
\pgfpathlineto{\pgfqpoint{3.999726in}{0.537305in}}%
\pgfpathlineto{\pgfqpoint{3.999726in}{0.541563in}}%
\pgfpathlineto{\pgfqpoint{4.003983in}{0.541563in}}%
\pgfpathlineto{\pgfqpoint{4.003983in}{0.537305in}}%
\pgfpathmoveto{\pgfqpoint{4.003983in}{0.524532in}}%
\pgfpathlineto{\pgfqpoint{4.003983in}{0.524532in}}%
\pgfpathlineto{\pgfqpoint{4.003983in}{0.528790in}}%
\pgfpathlineto{\pgfqpoint{4.008241in}{0.528790in}}%
\pgfpathlineto{\pgfqpoint{4.008241in}{0.524532in}}%
\pgfpathmoveto{\pgfqpoint{4.003983in}{0.528790in}}%
\pgfpathlineto{\pgfqpoint{4.003983in}{0.528790in}}%
\pgfpathlineto{\pgfqpoint{4.003983in}{0.533047in}}%
\pgfpathlineto{\pgfqpoint{4.008241in}{0.533047in}}%
\pgfpathlineto{\pgfqpoint{4.008241in}{0.528790in}}%
\pgfpathmoveto{\pgfqpoint{4.003983in}{0.533047in}}%
\pgfpathlineto{\pgfqpoint{4.003983in}{0.533047in}}%
\pgfpathlineto{\pgfqpoint{4.003983in}{0.537305in}}%
\pgfpathlineto{\pgfqpoint{4.008241in}{0.537305in}}%
\pgfpathlineto{\pgfqpoint{4.008241in}{0.533047in}}%
\pgfpathmoveto{\pgfqpoint{4.003983in}{0.537305in}}%
\pgfpathlineto{\pgfqpoint{4.003983in}{0.537305in}}%
\pgfpathlineto{\pgfqpoint{4.003983in}{0.541563in}}%
\pgfpathlineto{\pgfqpoint{4.008241in}{0.541563in}}%
\pgfpathlineto{\pgfqpoint{4.008241in}{0.537305in}}%
\pgfpathmoveto{\pgfqpoint{3.999726in}{0.541563in}}%
\pgfpathlineto{\pgfqpoint{3.999726in}{0.541563in}}%
\pgfpathlineto{\pgfqpoint{3.999726in}{0.545820in}}%
\pgfpathlineto{\pgfqpoint{4.003983in}{0.545820in}}%
\pgfpathlineto{\pgfqpoint{4.003983in}{0.541563in}}%
\pgfpathmoveto{\pgfqpoint{3.999726in}{0.545820in}}%
\pgfpathlineto{\pgfqpoint{3.999726in}{0.545820in}}%
\pgfpathlineto{\pgfqpoint{3.999726in}{0.550078in}}%
\pgfpathlineto{\pgfqpoint{4.003983in}{0.550078in}}%
\pgfpathlineto{\pgfqpoint{4.003983in}{0.545820in}}%
\pgfpathmoveto{\pgfqpoint{3.999726in}{0.550078in}}%
\pgfpathlineto{\pgfqpoint{3.999726in}{0.550078in}}%
\pgfpathlineto{\pgfqpoint{3.999726in}{0.554336in}}%
\pgfpathlineto{\pgfqpoint{4.003983in}{0.554336in}}%
\pgfpathlineto{\pgfqpoint{4.003983in}{0.550078in}}%
\pgfpathmoveto{\pgfqpoint{3.999726in}{0.554336in}}%
\pgfpathlineto{\pgfqpoint{3.999726in}{0.554336in}}%
\pgfpathlineto{\pgfqpoint{3.999726in}{0.558593in}}%
\pgfpathlineto{\pgfqpoint{4.003983in}{0.558593in}}%
\pgfpathlineto{\pgfqpoint{4.003983in}{0.554336in}}%
\pgfpathmoveto{\pgfqpoint{4.003983in}{0.541563in}}%
\pgfpathlineto{\pgfqpoint{4.003983in}{0.541563in}}%
\pgfpathlineto{\pgfqpoint{4.003983in}{0.545820in}}%
\pgfpathlineto{\pgfqpoint{4.008241in}{0.545820in}}%
\pgfpathlineto{\pgfqpoint{4.008241in}{0.541563in}}%
\pgfpathmoveto{\pgfqpoint{4.003983in}{0.545820in}}%
\pgfpathlineto{\pgfqpoint{4.003983in}{0.545820in}}%
\pgfpathlineto{\pgfqpoint{4.003983in}{0.550078in}}%
\pgfpathlineto{\pgfqpoint{4.008241in}{0.550078in}}%
\pgfpathlineto{\pgfqpoint{4.008241in}{0.545820in}}%
\pgfpathmoveto{\pgfqpoint{4.003983in}{0.550078in}}%
\pgfpathlineto{\pgfqpoint{4.003983in}{0.550078in}}%
\pgfpathlineto{\pgfqpoint{4.003983in}{0.554336in}}%
\pgfpathlineto{\pgfqpoint{4.008241in}{0.554336in}}%
\pgfpathlineto{\pgfqpoint{4.008241in}{0.550078in}}%
\pgfpathmoveto{\pgfqpoint{4.003983in}{0.554336in}}%
\pgfpathlineto{\pgfqpoint{4.003983in}{0.554336in}}%
\pgfpathlineto{\pgfqpoint{4.003983in}{0.558593in}}%
\pgfpathlineto{\pgfqpoint{4.008241in}{0.558593in}}%
\pgfpathlineto{\pgfqpoint{4.008241in}{0.554336in}}%
\pgfpathmoveto{\pgfqpoint{3.999726in}{0.558593in}}%
\pgfpathlineto{\pgfqpoint{3.999726in}{0.558593in}}%
\pgfpathlineto{\pgfqpoint{3.999726in}{0.562851in}}%
\pgfpathlineto{\pgfqpoint{4.003983in}{0.562851in}}%
\pgfpathlineto{\pgfqpoint{4.003983in}{0.558593in}}%
\pgfpathmoveto{\pgfqpoint{3.999726in}{0.562851in}}%
\pgfpathlineto{\pgfqpoint{3.999726in}{0.562851in}}%
\pgfpathlineto{\pgfqpoint{3.999726in}{0.567109in}}%
\pgfpathlineto{\pgfqpoint{4.003983in}{0.567109in}}%
\pgfpathlineto{\pgfqpoint{4.003983in}{0.562851in}}%
\pgfpathmoveto{\pgfqpoint{3.999726in}{0.567109in}}%
\pgfpathlineto{\pgfqpoint{3.999726in}{0.567109in}}%
\pgfpathlineto{\pgfqpoint{3.999726in}{0.571366in}}%
\pgfpathlineto{\pgfqpoint{4.003983in}{0.571366in}}%
\pgfpathlineto{\pgfqpoint{4.003983in}{0.567109in}}%
\pgfpathmoveto{\pgfqpoint{3.999726in}{0.571366in}}%
\pgfpathlineto{\pgfqpoint{3.999726in}{0.571366in}}%
\pgfpathlineto{\pgfqpoint{3.999726in}{0.575624in}}%
\pgfpathlineto{\pgfqpoint{4.003983in}{0.575624in}}%
\pgfpathlineto{\pgfqpoint{4.003983in}{0.571366in}}%
\pgfpathmoveto{\pgfqpoint{4.003983in}{0.558593in}}%
\pgfpathlineto{\pgfqpoint{4.003983in}{0.558593in}}%
\pgfpathlineto{\pgfqpoint{4.003983in}{0.562851in}}%
\pgfpathlineto{\pgfqpoint{4.008241in}{0.562851in}}%
\pgfpathlineto{\pgfqpoint{4.008241in}{0.558593in}}%
\pgfpathmoveto{\pgfqpoint{4.003983in}{0.562851in}}%
\pgfpathlineto{\pgfqpoint{4.003983in}{0.562851in}}%
\pgfpathlineto{\pgfqpoint{4.003983in}{0.567109in}}%
\pgfpathlineto{\pgfqpoint{4.008241in}{0.567109in}}%
\pgfpathlineto{\pgfqpoint{4.008241in}{0.562851in}}%
\pgfpathmoveto{\pgfqpoint{4.003983in}{0.567109in}}%
\pgfpathlineto{\pgfqpoint{4.003983in}{0.567109in}}%
\pgfpathlineto{\pgfqpoint{4.003983in}{0.571366in}}%
\pgfpathlineto{\pgfqpoint{4.008241in}{0.571366in}}%
\pgfpathlineto{\pgfqpoint{4.008241in}{0.567109in}}%
\pgfpathmoveto{\pgfqpoint{3.995468in}{0.575624in}}%
\pgfpathlineto{\pgfqpoint{3.995468in}{0.575624in}}%
\pgfpathlineto{\pgfqpoint{3.995468in}{0.579882in}}%
\pgfpathlineto{\pgfqpoint{3.999726in}{0.579882in}}%
\pgfpathlineto{\pgfqpoint{3.999726in}{0.575624in}}%
\pgfpathmoveto{\pgfqpoint{3.995468in}{0.579882in}}%
\pgfpathlineto{\pgfqpoint{3.995468in}{0.579882in}}%
\pgfpathlineto{\pgfqpoint{3.995468in}{0.584139in}}%
\pgfpathlineto{\pgfqpoint{3.999726in}{0.584139in}}%
\pgfpathlineto{\pgfqpoint{3.999726in}{0.579882in}}%
\pgfpathmoveto{\pgfqpoint{3.999726in}{0.575624in}}%
\pgfpathlineto{\pgfqpoint{3.999726in}{0.575624in}}%
\pgfpathlineto{\pgfqpoint{3.999726in}{0.579882in}}%
\pgfpathlineto{\pgfqpoint{4.003983in}{0.579882in}}%
\pgfpathlineto{\pgfqpoint{4.003983in}{0.575624in}}%
\pgfpathmoveto{\pgfqpoint{3.999726in}{0.579882in}}%
\pgfpathlineto{\pgfqpoint{3.999726in}{0.579882in}}%
\pgfpathlineto{\pgfqpoint{3.999726in}{0.584139in}}%
\pgfpathlineto{\pgfqpoint{4.003983in}{0.584139in}}%
\pgfpathlineto{\pgfqpoint{4.003983in}{0.579882in}}%
\pgfpathmoveto{\pgfqpoint{3.995468in}{0.584139in}}%
\pgfpathlineto{\pgfqpoint{3.995468in}{0.584139in}}%
\pgfpathlineto{\pgfqpoint{3.995468in}{0.588397in}}%
\pgfpathlineto{\pgfqpoint{3.999726in}{0.588397in}}%
\pgfpathlineto{\pgfqpoint{3.999726in}{0.584139in}}%
\pgfpathmoveto{\pgfqpoint{3.995468in}{0.588397in}}%
\pgfpathlineto{\pgfqpoint{3.995468in}{0.588397in}}%
\pgfpathlineto{\pgfqpoint{3.995468in}{0.592655in}}%
\pgfpathlineto{\pgfqpoint{3.999726in}{0.592655in}}%
\pgfpathlineto{\pgfqpoint{3.999726in}{0.588397in}}%
\pgfpathmoveto{\pgfqpoint{3.999726in}{0.584139in}}%
\pgfpathlineto{\pgfqpoint{3.999726in}{0.584139in}}%
\pgfpathlineto{\pgfqpoint{3.999726in}{0.588397in}}%
\pgfpathlineto{\pgfqpoint{4.003983in}{0.588397in}}%
\pgfpathlineto{\pgfqpoint{4.003983in}{0.584139in}}%
\pgfpathmoveto{\pgfqpoint{3.999726in}{0.588397in}}%
\pgfpathlineto{\pgfqpoint{3.999726in}{0.588397in}}%
\pgfpathlineto{\pgfqpoint{3.999726in}{0.592655in}}%
\pgfpathlineto{\pgfqpoint{4.003983in}{0.592655in}}%
\pgfpathlineto{\pgfqpoint{4.003983in}{0.588397in}}%
\pgfpathmoveto{\pgfqpoint{3.995468in}{0.592655in}}%
\pgfpathlineto{\pgfqpoint{3.995468in}{0.592655in}}%
\pgfpathlineto{\pgfqpoint{3.995468in}{0.596912in}}%
\pgfpathlineto{\pgfqpoint{3.999726in}{0.596912in}}%
\pgfpathlineto{\pgfqpoint{3.999726in}{0.592655in}}%
\pgfpathmoveto{\pgfqpoint{3.995468in}{0.596912in}}%
\pgfpathlineto{\pgfqpoint{3.995468in}{0.596912in}}%
\pgfpathlineto{\pgfqpoint{3.995468in}{0.601170in}}%
\pgfpathlineto{\pgfqpoint{3.999726in}{0.601170in}}%
\pgfpathlineto{\pgfqpoint{3.999726in}{0.596912in}}%
\pgfpathmoveto{\pgfqpoint{3.999726in}{0.592655in}}%
\pgfpathlineto{\pgfqpoint{3.999726in}{0.592655in}}%
\pgfpathlineto{\pgfqpoint{3.999726in}{0.596912in}}%
\pgfpathlineto{\pgfqpoint{4.003983in}{0.596912in}}%
\pgfpathlineto{\pgfqpoint{4.003983in}{0.592655in}}%
\pgfpathmoveto{\pgfqpoint{3.999726in}{0.596912in}}%
\pgfpathlineto{\pgfqpoint{3.999726in}{0.596912in}}%
\pgfpathlineto{\pgfqpoint{3.999726in}{0.601170in}}%
\pgfpathlineto{\pgfqpoint{4.003983in}{0.601170in}}%
\pgfpathlineto{\pgfqpoint{4.003983in}{0.596912in}}%
\pgfpathmoveto{\pgfqpoint{3.995468in}{0.601170in}}%
\pgfpathlineto{\pgfqpoint{3.995468in}{0.601170in}}%
\pgfpathlineto{\pgfqpoint{3.995468in}{0.605428in}}%
\pgfpathlineto{\pgfqpoint{3.999726in}{0.605428in}}%
\pgfpathlineto{\pgfqpoint{3.999726in}{0.601170in}}%
\pgfpathmoveto{\pgfqpoint{3.995468in}{0.605428in}}%
\pgfpathlineto{\pgfqpoint{3.995468in}{0.605428in}}%
\pgfpathlineto{\pgfqpoint{3.995468in}{0.609685in}}%
\pgfpathlineto{\pgfqpoint{3.999726in}{0.609685in}}%
\pgfpathlineto{\pgfqpoint{3.999726in}{0.605428in}}%
\pgfpathmoveto{\pgfqpoint{3.999726in}{0.601170in}}%
\pgfpathlineto{\pgfqpoint{3.999726in}{0.601170in}}%
\pgfpathlineto{\pgfqpoint{3.999726in}{0.605428in}}%
\pgfpathlineto{\pgfqpoint{4.003983in}{0.605428in}}%
\pgfpathlineto{\pgfqpoint{4.003983in}{0.601170in}}%
\pgfpathmoveto{\pgfqpoint{3.999726in}{0.605428in}}%
\pgfpathlineto{\pgfqpoint{3.999726in}{0.605428in}}%
\pgfpathlineto{\pgfqpoint{3.999726in}{0.609685in}}%
\pgfpathlineto{\pgfqpoint{4.003983in}{0.609685in}}%
\pgfpathlineto{\pgfqpoint{4.003983in}{0.605428in}}%
\pgfpathmoveto{\pgfqpoint{3.991210in}{0.618201in}}%
\pgfpathlineto{\pgfqpoint{3.991210in}{0.618201in}}%
\pgfpathlineto{\pgfqpoint{3.991210in}{0.622458in}}%
\pgfpathlineto{\pgfqpoint{3.995468in}{0.622458in}}%
\pgfpathlineto{\pgfqpoint{3.995468in}{0.618201in}}%
\pgfpathmoveto{\pgfqpoint{3.991210in}{0.622458in}}%
\pgfpathlineto{\pgfqpoint{3.991210in}{0.622458in}}%
\pgfpathlineto{\pgfqpoint{3.991210in}{0.626716in}}%
\pgfpathlineto{\pgfqpoint{3.995468in}{0.626716in}}%
\pgfpathlineto{\pgfqpoint{3.995468in}{0.622458in}}%
\pgfpathmoveto{\pgfqpoint{3.991210in}{0.626716in}}%
\pgfpathlineto{\pgfqpoint{3.991210in}{0.626716in}}%
\pgfpathlineto{\pgfqpoint{3.991210in}{0.630974in}}%
\pgfpathlineto{\pgfqpoint{3.995468in}{0.630974in}}%
\pgfpathlineto{\pgfqpoint{3.995468in}{0.626716in}}%
\pgfpathmoveto{\pgfqpoint{3.991210in}{0.630974in}}%
\pgfpathlineto{\pgfqpoint{3.991210in}{0.630974in}}%
\pgfpathlineto{\pgfqpoint{3.991210in}{0.635231in}}%
\pgfpathlineto{\pgfqpoint{3.995468in}{0.635231in}}%
\pgfpathlineto{\pgfqpoint{3.995468in}{0.630974in}}%
\pgfpathmoveto{\pgfqpoint{3.991210in}{0.635231in}}%
\pgfpathlineto{\pgfqpoint{3.991210in}{0.635231in}}%
\pgfpathlineto{\pgfqpoint{3.991210in}{0.639489in}}%
\pgfpathlineto{\pgfqpoint{3.995468in}{0.639489in}}%
\pgfpathlineto{\pgfqpoint{3.995468in}{0.635231in}}%
\pgfpathmoveto{\pgfqpoint{3.991210in}{0.639489in}}%
\pgfpathlineto{\pgfqpoint{3.991210in}{0.639489in}}%
\pgfpathlineto{\pgfqpoint{3.991210in}{0.643747in}}%
\pgfpathlineto{\pgfqpoint{3.995468in}{0.643747in}}%
\pgfpathlineto{\pgfqpoint{3.995468in}{0.639489in}}%
\pgfpathmoveto{\pgfqpoint{3.995468in}{0.609685in}}%
\pgfpathlineto{\pgfqpoint{3.995468in}{0.609685in}}%
\pgfpathlineto{\pgfqpoint{3.995468in}{0.613943in}}%
\pgfpathlineto{\pgfqpoint{3.999726in}{0.613943in}}%
\pgfpathlineto{\pgfqpoint{3.999726in}{0.609685in}}%
\pgfpathmoveto{\pgfqpoint{3.995468in}{0.613943in}}%
\pgfpathlineto{\pgfqpoint{3.995468in}{0.613943in}}%
\pgfpathlineto{\pgfqpoint{3.995468in}{0.618201in}}%
\pgfpathlineto{\pgfqpoint{3.999726in}{0.618201in}}%
\pgfpathlineto{\pgfqpoint{3.999726in}{0.613943in}}%
\pgfpathmoveto{\pgfqpoint{3.999726in}{0.609685in}}%
\pgfpathlineto{\pgfqpoint{3.999726in}{0.609685in}}%
\pgfpathlineto{\pgfqpoint{3.999726in}{0.613943in}}%
\pgfpathlineto{\pgfqpoint{4.003983in}{0.613943in}}%
\pgfpathlineto{\pgfqpoint{4.003983in}{0.609685in}}%
\pgfpathmoveto{\pgfqpoint{3.995468in}{0.618201in}}%
\pgfpathlineto{\pgfqpoint{3.995468in}{0.618201in}}%
\pgfpathlineto{\pgfqpoint{3.995468in}{0.622458in}}%
\pgfpathlineto{\pgfqpoint{3.999726in}{0.622458in}}%
\pgfpathlineto{\pgfqpoint{3.999726in}{0.618201in}}%
\pgfpathmoveto{\pgfqpoint{3.995468in}{0.622458in}}%
\pgfpathlineto{\pgfqpoint{3.995468in}{0.622458in}}%
\pgfpathlineto{\pgfqpoint{3.995468in}{0.626716in}}%
\pgfpathlineto{\pgfqpoint{3.999726in}{0.626716in}}%
\pgfpathlineto{\pgfqpoint{3.999726in}{0.622458in}}%
\pgfpathmoveto{\pgfqpoint{3.995468in}{0.626716in}}%
\pgfpathlineto{\pgfqpoint{3.995468in}{0.626716in}}%
\pgfpathlineto{\pgfqpoint{3.995468in}{0.630974in}}%
\pgfpathlineto{\pgfqpoint{3.999726in}{0.630974in}}%
\pgfpathlineto{\pgfqpoint{3.999726in}{0.626716in}}%
\pgfpathmoveto{\pgfqpoint{3.995468in}{0.630974in}}%
\pgfpathlineto{\pgfqpoint{3.995468in}{0.630974in}}%
\pgfpathlineto{\pgfqpoint{3.995468in}{0.635231in}}%
\pgfpathlineto{\pgfqpoint{3.999726in}{0.635231in}}%
\pgfpathlineto{\pgfqpoint{3.999726in}{0.630974in}}%
\pgfpathmoveto{\pgfqpoint{3.995468in}{0.635231in}}%
\pgfpathlineto{\pgfqpoint{3.995468in}{0.635231in}}%
\pgfpathlineto{\pgfqpoint{3.995468in}{0.639489in}}%
\pgfpathlineto{\pgfqpoint{3.999726in}{0.639489in}}%
\pgfpathlineto{\pgfqpoint{3.999726in}{0.635231in}}%
\pgfpathmoveto{\pgfqpoint{3.995468in}{0.639489in}}%
\pgfpathlineto{\pgfqpoint{3.995468in}{0.639489in}}%
\pgfpathlineto{\pgfqpoint{3.995468in}{0.643747in}}%
\pgfpathlineto{\pgfqpoint{3.999726in}{0.643747in}}%
\pgfpathlineto{\pgfqpoint{3.999726in}{0.639489in}}%
\pgfpathmoveto{\pgfqpoint{3.991210in}{0.643747in}}%
\pgfpathlineto{\pgfqpoint{3.991210in}{0.643747in}}%
\pgfpathlineto{\pgfqpoint{3.991210in}{0.648005in}}%
\pgfpathlineto{\pgfqpoint{3.995468in}{0.648005in}}%
\pgfpathlineto{\pgfqpoint{3.995468in}{0.643747in}}%
\pgfpathmoveto{\pgfqpoint{3.991210in}{0.648005in}}%
\pgfpathlineto{\pgfqpoint{3.991210in}{0.648005in}}%
\pgfpathlineto{\pgfqpoint{3.991210in}{0.652263in}}%
\pgfpathlineto{\pgfqpoint{3.995468in}{0.652263in}}%
\pgfpathlineto{\pgfqpoint{3.995468in}{0.648005in}}%
\pgfpathmoveto{\pgfqpoint{3.991210in}{0.652263in}}%
\pgfpathlineto{\pgfqpoint{3.991210in}{0.652263in}}%
\pgfpathlineto{\pgfqpoint{3.991210in}{0.656521in}}%
\pgfpathlineto{\pgfqpoint{3.995468in}{0.656521in}}%
\pgfpathlineto{\pgfqpoint{3.995468in}{0.652263in}}%
\pgfpathmoveto{\pgfqpoint{3.991210in}{0.656521in}}%
\pgfpathlineto{\pgfqpoint{3.991210in}{0.656521in}}%
\pgfpathlineto{\pgfqpoint{3.991210in}{0.660779in}}%
\pgfpathlineto{\pgfqpoint{3.995468in}{0.660779in}}%
\pgfpathlineto{\pgfqpoint{3.995468in}{0.656521in}}%
\pgfpathmoveto{\pgfqpoint{3.986952in}{0.660779in}}%
\pgfpathlineto{\pgfqpoint{3.986952in}{0.660779in}}%
\pgfpathlineto{\pgfqpoint{3.986952in}{0.665037in}}%
\pgfpathlineto{\pgfqpoint{3.991210in}{0.665037in}}%
\pgfpathlineto{\pgfqpoint{3.991210in}{0.660779in}}%
\pgfpathmoveto{\pgfqpoint{3.986952in}{0.665037in}}%
\pgfpathlineto{\pgfqpoint{3.986952in}{0.665037in}}%
\pgfpathlineto{\pgfqpoint{3.986952in}{0.669295in}}%
\pgfpathlineto{\pgfqpoint{3.991210in}{0.669295in}}%
\pgfpathlineto{\pgfqpoint{3.991210in}{0.665037in}}%
\pgfpathmoveto{\pgfqpoint{3.991210in}{0.660779in}}%
\pgfpathlineto{\pgfqpoint{3.991210in}{0.660779in}}%
\pgfpathlineto{\pgfqpoint{3.991210in}{0.665037in}}%
\pgfpathlineto{\pgfqpoint{3.995468in}{0.665037in}}%
\pgfpathlineto{\pgfqpoint{3.995468in}{0.660779in}}%
\pgfpathmoveto{\pgfqpoint{3.991210in}{0.665037in}}%
\pgfpathlineto{\pgfqpoint{3.991210in}{0.665037in}}%
\pgfpathlineto{\pgfqpoint{3.991210in}{0.669295in}}%
\pgfpathlineto{\pgfqpoint{3.995468in}{0.669295in}}%
\pgfpathlineto{\pgfqpoint{3.995468in}{0.665037in}}%
\pgfpathmoveto{\pgfqpoint{3.986952in}{0.669295in}}%
\pgfpathlineto{\pgfqpoint{3.986952in}{0.669295in}}%
\pgfpathlineto{\pgfqpoint{3.986952in}{0.673553in}}%
\pgfpathlineto{\pgfqpoint{3.991210in}{0.673553in}}%
\pgfpathlineto{\pgfqpoint{3.991210in}{0.669295in}}%
\pgfpathmoveto{\pgfqpoint{3.986952in}{0.673553in}}%
\pgfpathlineto{\pgfqpoint{3.986952in}{0.673553in}}%
\pgfpathlineto{\pgfqpoint{3.986952in}{0.677811in}}%
\pgfpathlineto{\pgfqpoint{3.991210in}{0.677811in}}%
\pgfpathlineto{\pgfqpoint{3.991210in}{0.673553in}}%
\pgfpathmoveto{\pgfqpoint{3.991210in}{0.669295in}}%
\pgfpathlineto{\pgfqpoint{3.991210in}{0.669295in}}%
\pgfpathlineto{\pgfqpoint{3.991210in}{0.673553in}}%
\pgfpathlineto{\pgfqpoint{3.995468in}{0.673553in}}%
\pgfpathlineto{\pgfqpoint{3.995468in}{0.669295in}}%
\pgfpathmoveto{\pgfqpoint{3.991210in}{0.673553in}}%
\pgfpathlineto{\pgfqpoint{3.991210in}{0.673553in}}%
\pgfpathlineto{\pgfqpoint{3.991210in}{0.677811in}}%
\pgfpathlineto{\pgfqpoint{3.995468in}{0.677811in}}%
\pgfpathlineto{\pgfqpoint{3.995468in}{0.673553in}}%
\pgfpathmoveto{\pgfqpoint{3.995468in}{0.643747in}}%
\pgfpathlineto{\pgfqpoint{3.995468in}{0.643747in}}%
\pgfpathlineto{\pgfqpoint{3.995468in}{0.648005in}}%
\pgfpathlineto{\pgfqpoint{3.999726in}{0.648005in}}%
\pgfpathlineto{\pgfqpoint{3.999726in}{0.643747in}}%
\pgfpathmoveto{\pgfqpoint{3.995468in}{0.648005in}}%
\pgfpathlineto{\pgfqpoint{3.995468in}{0.648005in}}%
\pgfpathlineto{\pgfqpoint{3.995468in}{0.652263in}}%
\pgfpathlineto{\pgfqpoint{3.999726in}{0.652263in}}%
\pgfpathlineto{\pgfqpoint{3.999726in}{0.648005in}}%
\pgfpathmoveto{\pgfqpoint{3.995468in}{0.652263in}}%
\pgfpathlineto{\pgfqpoint{3.995468in}{0.652263in}}%
\pgfpathlineto{\pgfqpoint{3.995468in}{0.656521in}}%
\pgfpathlineto{\pgfqpoint{3.999726in}{0.656521in}}%
\pgfpathlineto{\pgfqpoint{3.999726in}{0.652263in}}%
\pgfpathmoveto{\pgfqpoint{3.986952in}{0.677811in}}%
\pgfpathlineto{\pgfqpoint{3.986952in}{0.677811in}}%
\pgfpathlineto{\pgfqpoint{3.986952in}{0.682069in}}%
\pgfpathlineto{\pgfqpoint{3.991210in}{0.682069in}}%
\pgfpathlineto{\pgfqpoint{3.991210in}{0.677811in}}%
\pgfpathmoveto{\pgfqpoint{3.986952in}{0.682069in}}%
\pgfpathlineto{\pgfqpoint{3.986952in}{0.682069in}}%
\pgfpathlineto{\pgfqpoint{3.986952in}{0.686327in}}%
\pgfpathlineto{\pgfqpoint{3.991210in}{0.686327in}}%
\pgfpathlineto{\pgfqpoint{3.991210in}{0.682069in}}%
\pgfpathmoveto{\pgfqpoint{3.991210in}{0.677811in}}%
\pgfpathlineto{\pgfqpoint{3.991210in}{0.677811in}}%
\pgfpathlineto{\pgfqpoint{3.991210in}{0.682069in}}%
\pgfpathlineto{\pgfqpoint{3.995468in}{0.682069in}}%
\pgfpathlineto{\pgfqpoint{3.995468in}{0.677811in}}%
\pgfpathmoveto{\pgfqpoint{3.991210in}{0.682069in}}%
\pgfpathlineto{\pgfqpoint{3.991210in}{0.682069in}}%
\pgfpathlineto{\pgfqpoint{3.991210in}{0.686327in}}%
\pgfpathlineto{\pgfqpoint{3.995468in}{0.686327in}}%
\pgfpathlineto{\pgfqpoint{3.995468in}{0.682069in}}%
\pgfpathmoveto{\pgfqpoint{3.986952in}{0.686327in}}%
\pgfpathlineto{\pgfqpoint{3.986952in}{0.686327in}}%
\pgfpathlineto{\pgfqpoint{3.986952in}{0.690585in}}%
\pgfpathlineto{\pgfqpoint{3.991210in}{0.690585in}}%
\pgfpathlineto{\pgfqpoint{3.991210in}{0.686327in}}%
\pgfpathmoveto{\pgfqpoint{3.986952in}{0.690585in}}%
\pgfpathlineto{\pgfqpoint{3.986952in}{0.690585in}}%
\pgfpathlineto{\pgfqpoint{3.986952in}{0.694843in}}%
\pgfpathlineto{\pgfqpoint{3.991210in}{0.694843in}}%
\pgfpathlineto{\pgfqpoint{3.991210in}{0.690585in}}%
\pgfpathmoveto{\pgfqpoint{3.991210in}{0.686327in}}%
\pgfpathlineto{\pgfqpoint{3.991210in}{0.686327in}}%
\pgfpathlineto{\pgfqpoint{3.991210in}{0.690585in}}%
\pgfpathlineto{\pgfqpoint{3.995468in}{0.690585in}}%
\pgfpathlineto{\pgfqpoint{3.995468in}{0.686327in}}%
\pgfpathmoveto{\pgfqpoint{3.991210in}{0.690585in}}%
\pgfpathlineto{\pgfqpoint{3.991210in}{0.690585in}}%
\pgfpathlineto{\pgfqpoint{3.991210in}{0.694843in}}%
\pgfpathlineto{\pgfqpoint{3.995468in}{0.694843in}}%
\pgfpathlineto{\pgfqpoint{3.995468in}{0.690585in}}%
\pgfpathmoveto{\pgfqpoint{3.982695in}{0.703359in}}%
\pgfpathlineto{\pgfqpoint{3.982695in}{0.703359in}}%
\pgfpathlineto{\pgfqpoint{3.982695in}{0.707617in}}%
\pgfpathlineto{\pgfqpoint{3.986952in}{0.707617in}}%
\pgfpathlineto{\pgfqpoint{3.986952in}{0.703359in}}%
\pgfpathmoveto{\pgfqpoint{3.982695in}{0.707617in}}%
\pgfpathlineto{\pgfqpoint{3.982695in}{0.707617in}}%
\pgfpathlineto{\pgfqpoint{3.982695in}{0.711875in}}%
\pgfpathlineto{\pgfqpoint{3.986952in}{0.711875in}}%
\pgfpathlineto{\pgfqpoint{3.986952in}{0.707617in}}%
\pgfpathmoveto{\pgfqpoint{3.986952in}{0.694843in}}%
\pgfpathlineto{\pgfqpoint{3.986952in}{0.694843in}}%
\pgfpathlineto{\pgfqpoint{3.986952in}{0.699101in}}%
\pgfpathlineto{\pgfqpoint{3.991210in}{0.699101in}}%
\pgfpathlineto{\pgfqpoint{3.991210in}{0.694843in}}%
\pgfpathmoveto{\pgfqpoint{3.986952in}{0.699101in}}%
\pgfpathlineto{\pgfqpoint{3.986952in}{0.699101in}}%
\pgfpathlineto{\pgfqpoint{3.986952in}{0.703359in}}%
\pgfpathlineto{\pgfqpoint{3.991210in}{0.703359in}}%
\pgfpathlineto{\pgfqpoint{3.991210in}{0.699101in}}%
\pgfpathmoveto{\pgfqpoint{3.991210in}{0.694843in}}%
\pgfpathlineto{\pgfqpoint{3.991210in}{0.694843in}}%
\pgfpathlineto{\pgfqpoint{3.991210in}{0.699101in}}%
\pgfpathlineto{\pgfqpoint{3.995468in}{0.699101in}}%
\pgfpathlineto{\pgfqpoint{3.995468in}{0.694843in}}%
\pgfpathmoveto{\pgfqpoint{3.986952in}{0.703359in}}%
\pgfpathlineto{\pgfqpoint{3.986952in}{0.703359in}}%
\pgfpathlineto{\pgfqpoint{3.986952in}{0.707617in}}%
\pgfpathlineto{\pgfqpoint{3.991210in}{0.707617in}}%
\pgfpathlineto{\pgfqpoint{3.991210in}{0.703359in}}%
\pgfpathmoveto{\pgfqpoint{3.986952in}{0.707617in}}%
\pgfpathlineto{\pgfqpoint{3.986952in}{0.707617in}}%
\pgfpathlineto{\pgfqpoint{3.986952in}{0.711875in}}%
\pgfpathlineto{\pgfqpoint{3.991210in}{0.711875in}}%
\pgfpathlineto{\pgfqpoint{3.991210in}{0.707617in}}%
\pgfpathmoveto{\pgfqpoint{3.982695in}{0.711875in}}%
\pgfpathlineto{\pgfqpoint{3.982695in}{0.711875in}}%
\pgfpathlineto{\pgfqpoint{3.982695in}{0.716133in}}%
\pgfpathlineto{\pgfqpoint{3.986952in}{0.716133in}}%
\pgfpathlineto{\pgfqpoint{3.986952in}{0.711875in}}%
\pgfpathmoveto{\pgfqpoint{3.982695in}{0.716133in}}%
\pgfpathlineto{\pgfqpoint{3.982695in}{0.716133in}}%
\pgfpathlineto{\pgfqpoint{3.982695in}{0.720391in}}%
\pgfpathlineto{\pgfqpoint{3.986952in}{0.720391in}}%
\pgfpathlineto{\pgfqpoint{3.986952in}{0.716133in}}%
\pgfpathmoveto{\pgfqpoint{3.982695in}{0.720391in}}%
\pgfpathlineto{\pgfqpoint{3.982695in}{0.720391in}}%
\pgfpathlineto{\pgfqpoint{3.982695in}{0.724649in}}%
\pgfpathlineto{\pgfqpoint{3.986952in}{0.724649in}}%
\pgfpathlineto{\pgfqpoint{3.986952in}{0.720391in}}%
\pgfpathmoveto{\pgfqpoint{3.982695in}{0.724649in}}%
\pgfpathlineto{\pgfqpoint{3.982695in}{0.724649in}}%
\pgfpathlineto{\pgfqpoint{3.982695in}{0.728907in}}%
\pgfpathlineto{\pgfqpoint{3.986952in}{0.728907in}}%
\pgfpathlineto{\pgfqpoint{3.986952in}{0.724649in}}%
\pgfpathmoveto{\pgfqpoint{3.986952in}{0.711875in}}%
\pgfpathlineto{\pgfqpoint{3.986952in}{0.711875in}}%
\pgfpathlineto{\pgfqpoint{3.986952in}{0.716133in}}%
\pgfpathlineto{\pgfqpoint{3.991210in}{0.716133in}}%
\pgfpathlineto{\pgfqpoint{3.991210in}{0.711875in}}%
\pgfpathmoveto{\pgfqpoint{3.986952in}{0.716133in}}%
\pgfpathlineto{\pgfqpoint{3.986952in}{0.716133in}}%
\pgfpathlineto{\pgfqpoint{3.986952in}{0.720391in}}%
\pgfpathlineto{\pgfqpoint{3.991210in}{0.720391in}}%
\pgfpathlineto{\pgfqpoint{3.991210in}{0.716133in}}%
\pgfpathmoveto{\pgfqpoint{3.986952in}{0.720391in}}%
\pgfpathlineto{\pgfqpoint{3.986952in}{0.720391in}}%
\pgfpathlineto{\pgfqpoint{3.986952in}{0.724649in}}%
\pgfpathlineto{\pgfqpoint{3.991210in}{0.724649in}}%
\pgfpathlineto{\pgfqpoint{3.991210in}{0.720391in}}%
\pgfpathmoveto{\pgfqpoint{3.986952in}{0.724649in}}%
\pgfpathlineto{\pgfqpoint{3.986952in}{0.724649in}}%
\pgfpathlineto{\pgfqpoint{3.986952in}{0.728907in}}%
\pgfpathlineto{\pgfqpoint{3.991210in}{0.728907in}}%
\pgfpathlineto{\pgfqpoint{3.991210in}{0.724649in}}%
\pgfpathmoveto{\pgfqpoint{3.982695in}{0.728907in}}%
\pgfpathlineto{\pgfqpoint{3.982695in}{0.728907in}}%
\pgfpathlineto{\pgfqpoint{3.982695in}{0.733165in}}%
\pgfpathlineto{\pgfqpoint{3.986952in}{0.733165in}}%
\pgfpathlineto{\pgfqpoint{3.986952in}{0.728907in}}%
\pgfpathmoveto{\pgfqpoint{3.982695in}{0.733165in}}%
\pgfpathlineto{\pgfqpoint{3.982695in}{0.733165in}}%
\pgfpathlineto{\pgfqpoint{3.982695in}{0.737424in}}%
\pgfpathlineto{\pgfqpoint{3.986952in}{0.737424in}}%
\pgfpathlineto{\pgfqpoint{3.986952in}{0.733165in}}%
\pgfpathmoveto{\pgfqpoint{3.982695in}{0.737424in}}%
\pgfpathlineto{\pgfqpoint{3.982695in}{0.737424in}}%
\pgfpathlineto{\pgfqpoint{3.982695in}{0.741682in}}%
\pgfpathlineto{\pgfqpoint{3.986952in}{0.741682in}}%
\pgfpathlineto{\pgfqpoint{3.986952in}{0.737424in}}%
\pgfpathmoveto{\pgfqpoint{3.982695in}{0.741682in}}%
\pgfpathlineto{\pgfqpoint{3.982695in}{0.741682in}}%
\pgfpathlineto{\pgfqpoint{3.982695in}{0.745940in}}%
\pgfpathlineto{\pgfqpoint{3.986952in}{0.745940in}}%
\pgfpathlineto{\pgfqpoint{3.986952in}{0.741682in}}%
\pgfpathmoveto{\pgfqpoint{3.986952in}{0.728907in}}%
\pgfpathlineto{\pgfqpoint{3.986952in}{0.728907in}}%
\pgfpathlineto{\pgfqpoint{3.986952in}{0.733165in}}%
\pgfpathlineto{\pgfqpoint{3.991210in}{0.733165in}}%
\pgfpathlineto{\pgfqpoint{3.991210in}{0.728907in}}%
\pgfpathmoveto{\pgfqpoint{3.986952in}{0.733165in}}%
\pgfpathlineto{\pgfqpoint{3.986952in}{0.733165in}}%
\pgfpathlineto{\pgfqpoint{3.986952in}{0.737424in}}%
\pgfpathlineto{\pgfqpoint{3.991210in}{0.737424in}}%
\pgfpathlineto{\pgfqpoint{3.991210in}{0.733165in}}%
\pgfpathmoveto{\pgfqpoint{3.986952in}{0.737424in}}%
\pgfpathlineto{\pgfqpoint{3.986952in}{0.737424in}}%
\pgfpathlineto{\pgfqpoint{3.986952in}{0.741682in}}%
\pgfpathlineto{\pgfqpoint{3.991210in}{0.741682in}}%
\pgfpathlineto{\pgfqpoint{3.991210in}{0.737424in}}%
\pgfpathmoveto{\pgfqpoint{3.978437in}{0.745940in}}%
\pgfpathlineto{\pgfqpoint{3.978437in}{0.745940in}}%
\pgfpathlineto{\pgfqpoint{3.978437in}{0.750198in}}%
\pgfpathlineto{\pgfqpoint{3.982695in}{0.750198in}}%
\pgfpathlineto{\pgfqpoint{3.982695in}{0.745940in}}%
\pgfpathmoveto{\pgfqpoint{3.978437in}{0.750198in}}%
\pgfpathlineto{\pgfqpoint{3.978437in}{0.750198in}}%
\pgfpathlineto{\pgfqpoint{3.978437in}{0.754456in}}%
\pgfpathlineto{\pgfqpoint{3.982695in}{0.754456in}}%
\pgfpathlineto{\pgfqpoint{3.982695in}{0.750198in}}%
\pgfpathmoveto{\pgfqpoint{3.982695in}{0.745940in}}%
\pgfpathlineto{\pgfqpoint{3.982695in}{0.745940in}}%
\pgfpathlineto{\pgfqpoint{3.982695in}{0.750198in}}%
\pgfpathlineto{\pgfqpoint{3.986952in}{0.750198in}}%
\pgfpathlineto{\pgfqpoint{3.986952in}{0.745940in}}%
\pgfpathmoveto{\pgfqpoint{3.982695in}{0.750198in}}%
\pgfpathlineto{\pgfqpoint{3.982695in}{0.750198in}}%
\pgfpathlineto{\pgfqpoint{3.982695in}{0.754456in}}%
\pgfpathlineto{\pgfqpoint{3.986952in}{0.754456in}}%
\pgfpathlineto{\pgfqpoint{3.986952in}{0.750198in}}%
\pgfpathmoveto{\pgfqpoint{3.978437in}{0.754456in}}%
\pgfpathlineto{\pgfqpoint{3.978437in}{0.754456in}}%
\pgfpathlineto{\pgfqpoint{3.978437in}{0.758714in}}%
\pgfpathlineto{\pgfqpoint{3.982695in}{0.758714in}}%
\pgfpathlineto{\pgfqpoint{3.982695in}{0.754456in}}%
\pgfpathmoveto{\pgfqpoint{3.978437in}{0.758714in}}%
\pgfpathlineto{\pgfqpoint{3.978437in}{0.758714in}}%
\pgfpathlineto{\pgfqpoint{3.978437in}{0.762972in}}%
\pgfpathlineto{\pgfqpoint{3.982695in}{0.762972in}}%
\pgfpathlineto{\pgfqpoint{3.982695in}{0.758714in}}%
\pgfpathmoveto{\pgfqpoint{3.982695in}{0.754456in}}%
\pgfpathlineto{\pgfqpoint{3.982695in}{0.754456in}}%
\pgfpathlineto{\pgfqpoint{3.982695in}{0.758714in}}%
\pgfpathlineto{\pgfqpoint{3.986952in}{0.758714in}}%
\pgfpathlineto{\pgfqpoint{3.986952in}{0.754456in}}%
\pgfpathmoveto{\pgfqpoint{3.982695in}{0.758714in}}%
\pgfpathlineto{\pgfqpoint{3.982695in}{0.758714in}}%
\pgfpathlineto{\pgfqpoint{3.982695in}{0.762972in}}%
\pgfpathlineto{\pgfqpoint{3.986952in}{0.762972in}}%
\pgfpathlineto{\pgfqpoint{3.986952in}{0.758714in}}%
\pgfpathmoveto{\pgfqpoint{3.978437in}{0.762972in}}%
\pgfpathlineto{\pgfqpoint{3.978437in}{0.762972in}}%
\pgfpathlineto{\pgfqpoint{3.978437in}{0.767230in}}%
\pgfpathlineto{\pgfqpoint{3.982695in}{0.767230in}}%
\pgfpathlineto{\pgfqpoint{3.982695in}{0.762972in}}%
\pgfpathmoveto{\pgfqpoint{3.978437in}{0.767230in}}%
\pgfpathlineto{\pgfqpoint{3.978437in}{0.767230in}}%
\pgfpathlineto{\pgfqpoint{3.978437in}{0.771488in}}%
\pgfpathlineto{\pgfqpoint{3.982695in}{0.771488in}}%
\pgfpathlineto{\pgfqpoint{3.982695in}{0.767230in}}%
\pgfpathmoveto{\pgfqpoint{3.982695in}{0.762972in}}%
\pgfpathlineto{\pgfqpoint{3.982695in}{0.762972in}}%
\pgfpathlineto{\pgfqpoint{3.982695in}{0.767230in}}%
\pgfpathlineto{\pgfqpoint{3.986952in}{0.767230in}}%
\pgfpathlineto{\pgfqpoint{3.986952in}{0.762972in}}%
\pgfpathmoveto{\pgfqpoint{3.982695in}{0.767230in}}%
\pgfpathlineto{\pgfqpoint{3.982695in}{0.767230in}}%
\pgfpathlineto{\pgfqpoint{3.982695in}{0.771488in}}%
\pgfpathlineto{\pgfqpoint{3.986952in}{0.771488in}}%
\pgfpathlineto{\pgfqpoint{3.986952in}{0.767230in}}%
\pgfpathmoveto{\pgfqpoint{3.978437in}{0.771488in}}%
\pgfpathlineto{\pgfqpoint{3.978437in}{0.771488in}}%
\pgfpathlineto{\pgfqpoint{3.978437in}{0.775746in}}%
\pgfpathlineto{\pgfqpoint{3.982695in}{0.775746in}}%
\pgfpathlineto{\pgfqpoint{3.982695in}{0.771488in}}%
\pgfpathmoveto{\pgfqpoint{3.978437in}{0.775746in}}%
\pgfpathlineto{\pgfqpoint{3.978437in}{0.775746in}}%
\pgfpathlineto{\pgfqpoint{3.978437in}{0.780004in}}%
\pgfpathlineto{\pgfqpoint{3.982695in}{0.780004in}}%
\pgfpathlineto{\pgfqpoint{3.982695in}{0.775746in}}%
\pgfpathmoveto{\pgfqpoint{3.982695in}{0.771488in}}%
\pgfpathlineto{\pgfqpoint{3.982695in}{0.771488in}}%
\pgfpathlineto{\pgfqpoint{3.982695in}{0.775746in}}%
\pgfpathlineto{\pgfqpoint{3.986952in}{0.775746in}}%
\pgfpathlineto{\pgfqpoint{3.986952in}{0.771488in}}%
\pgfpathmoveto{\pgfqpoint{3.982695in}{0.775746in}}%
\pgfpathlineto{\pgfqpoint{3.982695in}{0.775746in}}%
\pgfpathlineto{\pgfqpoint{3.982695in}{0.780004in}}%
\pgfpathlineto{\pgfqpoint{3.986952in}{0.780004in}}%
\pgfpathlineto{\pgfqpoint{3.986952in}{0.775746in}}%
\pgfpathmoveto{\pgfqpoint{3.974179in}{0.788519in}}%
\pgfpathlineto{\pgfqpoint{3.974179in}{0.788519in}}%
\pgfpathlineto{\pgfqpoint{3.974179in}{0.792777in}}%
\pgfpathlineto{\pgfqpoint{3.978437in}{0.792777in}}%
\pgfpathlineto{\pgfqpoint{3.978437in}{0.788519in}}%
\pgfpathmoveto{\pgfqpoint{3.974179in}{0.792777in}}%
\pgfpathlineto{\pgfqpoint{3.974179in}{0.792777in}}%
\pgfpathlineto{\pgfqpoint{3.974179in}{0.797034in}}%
\pgfpathlineto{\pgfqpoint{3.978437in}{0.797034in}}%
\pgfpathlineto{\pgfqpoint{3.978437in}{0.792777in}}%
\pgfpathmoveto{\pgfqpoint{3.974179in}{0.797034in}}%
\pgfpathlineto{\pgfqpoint{3.974179in}{0.797034in}}%
\pgfpathlineto{\pgfqpoint{3.974179in}{0.801292in}}%
\pgfpathlineto{\pgfqpoint{3.978437in}{0.801292in}}%
\pgfpathlineto{\pgfqpoint{3.978437in}{0.797034in}}%
\pgfpathmoveto{\pgfqpoint{3.974179in}{0.801292in}}%
\pgfpathlineto{\pgfqpoint{3.974179in}{0.801292in}}%
\pgfpathlineto{\pgfqpoint{3.974179in}{0.805550in}}%
\pgfpathlineto{\pgfqpoint{3.978437in}{0.805550in}}%
\pgfpathlineto{\pgfqpoint{3.978437in}{0.801292in}}%
\pgfpathmoveto{\pgfqpoint{3.974179in}{0.805550in}}%
\pgfpathlineto{\pgfqpoint{3.974179in}{0.805550in}}%
\pgfpathlineto{\pgfqpoint{3.974179in}{0.809807in}}%
\pgfpathlineto{\pgfqpoint{3.978437in}{0.809807in}}%
\pgfpathlineto{\pgfqpoint{3.978437in}{0.805550in}}%
\pgfpathmoveto{\pgfqpoint{3.974179in}{0.809807in}}%
\pgfpathlineto{\pgfqpoint{3.974179in}{0.809807in}}%
\pgfpathlineto{\pgfqpoint{3.974179in}{0.814065in}}%
\pgfpathlineto{\pgfqpoint{3.978437in}{0.814065in}}%
\pgfpathlineto{\pgfqpoint{3.978437in}{0.809807in}}%
\pgfpathmoveto{\pgfqpoint{3.974179in}{0.814065in}}%
\pgfpathlineto{\pgfqpoint{3.974179in}{0.814065in}}%
\pgfpathlineto{\pgfqpoint{3.974179in}{0.818322in}}%
\pgfpathlineto{\pgfqpoint{3.978437in}{0.818322in}}%
\pgfpathlineto{\pgfqpoint{3.978437in}{0.814065in}}%
\pgfpathmoveto{\pgfqpoint{3.974179in}{0.818322in}}%
\pgfpathlineto{\pgfqpoint{3.974179in}{0.818322in}}%
\pgfpathlineto{\pgfqpoint{3.974179in}{0.822580in}}%
\pgfpathlineto{\pgfqpoint{3.978437in}{0.822580in}}%
\pgfpathlineto{\pgfqpoint{3.978437in}{0.818322in}}%
\pgfpathmoveto{\pgfqpoint{3.969922in}{0.826838in}}%
\pgfpathlineto{\pgfqpoint{3.969922in}{0.826838in}}%
\pgfpathlineto{\pgfqpoint{3.969922in}{0.831095in}}%
\pgfpathlineto{\pgfqpoint{3.974179in}{0.831095in}}%
\pgfpathlineto{\pgfqpoint{3.974179in}{0.826838in}}%
\pgfpathmoveto{\pgfqpoint{3.974179in}{0.822580in}}%
\pgfpathlineto{\pgfqpoint{3.974179in}{0.822580in}}%
\pgfpathlineto{\pgfqpoint{3.974179in}{0.826838in}}%
\pgfpathlineto{\pgfqpoint{3.978437in}{0.826838in}}%
\pgfpathlineto{\pgfqpoint{3.978437in}{0.822580in}}%
\pgfpathmoveto{\pgfqpoint{3.974179in}{0.826838in}}%
\pgfpathlineto{\pgfqpoint{3.974179in}{0.826838in}}%
\pgfpathlineto{\pgfqpoint{3.974179in}{0.831095in}}%
\pgfpathlineto{\pgfqpoint{3.978437in}{0.831095in}}%
\pgfpathlineto{\pgfqpoint{3.978437in}{0.826838in}}%
\pgfpathmoveto{\pgfqpoint{3.969922in}{0.831095in}}%
\pgfpathlineto{\pgfqpoint{3.969922in}{0.831095in}}%
\pgfpathlineto{\pgfqpoint{3.969922in}{0.835353in}}%
\pgfpathlineto{\pgfqpoint{3.974179in}{0.835353in}}%
\pgfpathlineto{\pgfqpoint{3.974179in}{0.831095in}}%
\pgfpathmoveto{\pgfqpoint{3.969922in}{0.835353in}}%
\pgfpathlineto{\pgfqpoint{3.969922in}{0.835353in}}%
\pgfpathlineto{\pgfqpoint{3.969922in}{0.839610in}}%
\pgfpathlineto{\pgfqpoint{3.974179in}{0.839610in}}%
\pgfpathlineto{\pgfqpoint{3.974179in}{0.835353in}}%
\pgfpathmoveto{\pgfqpoint{3.974179in}{0.831095in}}%
\pgfpathlineto{\pgfqpoint{3.974179in}{0.831095in}}%
\pgfpathlineto{\pgfqpoint{3.974179in}{0.835353in}}%
\pgfpathlineto{\pgfqpoint{3.978437in}{0.835353in}}%
\pgfpathlineto{\pgfqpoint{3.978437in}{0.831095in}}%
\pgfpathmoveto{\pgfqpoint{3.974179in}{0.835353in}}%
\pgfpathlineto{\pgfqpoint{3.974179in}{0.835353in}}%
\pgfpathlineto{\pgfqpoint{3.974179in}{0.839610in}}%
\pgfpathlineto{\pgfqpoint{3.978437in}{0.839610in}}%
\pgfpathlineto{\pgfqpoint{3.978437in}{0.835353in}}%
\pgfpathmoveto{\pgfqpoint{3.969922in}{0.839610in}}%
\pgfpathlineto{\pgfqpoint{3.969922in}{0.839610in}}%
\pgfpathlineto{\pgfqpoint{3.969922in}{0.843868in}}%
\pgfpathlineto{\pgfqpoint{3.974179in}{0.843868in}}%
\pgfpathlineto{\pgfqpoint{3.974179in}{0.839610in}}%
\pgfpathmoveto{\pgfqpoint{3.969922in}{0.843868in}}%
\pgfpathlineto{\pgfqpoint{3.969922in}{0.843868in}}%
\pgfpathlineto{\pgfqpoint{3.969922in}{0.848126in}}%
\pgfpathlineto{\pgfqpoint{3.974179in}{0.848126in}}%
\pgfpathlineto{\pgfqpoint{3.974179in}{0.843868in}}%
\pgfpathmoveto{\pgfqpoint{3.974179in}{0.839610in}}%
\pgfpathlineto{\pgfqpoint{3.974179in}{0.839610in}}%
\pgfpathlineto{\pgfqpoint{3.974179in}{0.843868in}}%
\pgfpathlineto{\pgfqpoint{3.978437in}{0.843868in}}%
\pgfpathlineto{\pgfqpoint{3.978437in}{0.839610in}}%
\pgfpathmoveto{\pgfqpoint{3.974179in}{0.843868in}}%
\pgfpathlineto{\pgfqpoint{3.974179in}{0.843868in}}%
\pgfpathlineto{\pgfqpoint{3.974179in}{0.848126in}}%
\pgfpathlineto{\pgfqpoint{3.978437in}{0.848126in}}%
\pgfpathlineto{\pgfqpoint{3.978437in}{0.843868in}}%
\pgfpathmoveto{\pgfqpoint{3.969922in}{0.848126in}}%
\pgfpathlineto{\pgfqpoint{3.969922in}{0.848126in}}%
\pgfpathlineto{\pgfqpoint{3.969922in}{0.852383in}}%
\pgfpathlineto{\pgfqpoint{3.974179in}{0.852383in}}%
\pgfpathlineto{\pgfqpoint{3.974179in}{0.848126in}}%
\pgfpathmoveto{\pgfqpoint{3.969922in}{0.852383in}}%
\pgfpathlineto{\pgfqpoint{3.969922in}{0.852383in}}%
\pgfpathlineto{\pgfqpoint{3.969922in}{0.856641in}}%
\pgfpathlineto{\pgfqpoint{3.974179in}{0.856641in}}%
\pgfpathlineto{\pgfqpoint{3.974179in}{0.852383in}}%
\pgfpathmoveto{\pgfqpoint{3.974179in}{0.848126in}}%
\pgfpathlineto{\pgfqpoint{3.974179in}{0.848126in}}%
\pgfpathlineto{\pgfqpoint{3.974179in}{0.852383in}}%
\pgfpathlineto{\pgfqpoint{3.978437in}{0.852383in}}%
\pgfpathlineto{\pgfqpoint{3.978437in}{0.848126in}}%
\pgfpathmoveto{\pgfqpoint{3.974179in}{0.852383in}}%
\pgfpathlineto{\pgfqpoint{3.974179in}{0.852383in}}%
\pgfpathlineto{\pgfqpoint{3.974179in}{0.856641in}}%
\pgfpathlineto{\pgfqpoint{3.978437in}{0.856641in}}%
\pgfpathlineto{\pgfqpoint{3.978437in}{0.852383in}}%
\pgfpathmoveto{\pgfqpoint{3.969922in}{0.856641in}}%
\pgfpathlineto{\pgfqpoint{3.969922in}{0.856641in}}%
\pgfpathlineto{\pgfqpoint{3.969922in}{0.860898in}}%
\pgfpathlineto{\pgfqpoint{3.974179in}{0.860898in}}%
\pgfpathlineto{\pgfqpoint{3.974179in}{0.856641in}}%
\pgfpathmoveto{\pgfqpoint{3.969922in}{0.860898in}}%
\pgfpathlineto{\pgfqpoint{3.969922in}{0.860898in}}%
\pgfpathlineto{\pgfqpoint{3.969922in}{0.865156in}}%
\pgfpathlineto{\pgfqpoint{3.974179in}{0.865156in}}%
\pgfpathlineto{\pgfqpoint{3.974179in}{0.860898in}}%
\pgfpathmoveto{\pgfqpoint{3.974179in}{0.856641in}}%
\pgfpathlineto{\pgfqpoint{3.974179in}{0.856641in}}%
\pgfpathlineto{\pgfqpoint{3.974179in}{0.860898in}}%
\pgfpathlineto{\pgfqpoint{3.978437in}{0.860898in}}%
\pgfpathlineto{\pgfqpoint{3.978437in}{0.856641in}}%
\pgfpathmoveto{\pgfqpoint{3.974179in}{0.860898in}}%
\pgfpathlineto{\pgfqpoint{3.974179in}{0.860898in}}%
\pgfpathlineto{\pgfqpoint{3.974179in}{0.865156in}}%
\pgfpathlineto{\pgfqpoint{3.978437in}{0.865156in}}%
\pgfpathlineto{\pgfqpoint{3.978437in}{0.860898in}}%
\pgfpathmoveto{\pgfqpoint{3.965664in}{0.869414in}}%
\pgfpathlineto{\pgfqpoint{3.965664in}{0.869414in}}%
\pgfpathlineto{\pgfqpoint{3.965664in}{0.873671in}}%
\pgfpathlineto{\pgfqpoint{3.969922in}{0.873671in}}%
\pgfpathlineto{\pgfqpoint{3.969922in}{0.869414in}}%
\pgfpathmoveto{\pgfqpoint{3.965664in}{0.873671in}}%
\pgfpathlineto{\pgfqpoint{3.965664in}{0.873671in}}%
\pgfpathlineto{\pgfqpoint{3.965664in}{0.877929in}}%
\pgfpathlineto{\pgfqpoint{3.969922in}{0.877929in}}%
\pgfpathlineto{\pgfqpoint{3.969922in}{0.873671in}}%
\pgfpathmoveto{\pgfqpoint{3.965664in}{0.877929in}}%
\pgfpathlineto{\pgfqpoint{3.965664in}{0.877929in}}%
\pgfpathlineto{\pgfqpoint{3.965664in}{0.882186in}}%
\pgfpathlineto{\pgfqpoint{3.969922in}{0.882186in}}%
\pgfpathlineto{\pgfqpoint{3.969922in}{0.877929in}}%
\pgfpathmoveto{\pgfqpoint{3.969922in}{0.865156in}}%
\pgfpathlineto{\pgfqpoint{3.969922in}{0.865156in}}%
\pgfpathlineto{\pgfqpoint{3.969922in}{0.869414in}}%
\pgfpathlineto{\pgfqpoint{3.974179in}{0.869414in}}%
\pgfpathlineto{\pgfqpoint{3.974179in}{0.865156in}}%
\pgfpathmoveto{\pgfqpoint{3.969922in}{0.869414in}}%
\pgfpathlineto{\pgfqpoint{3.969922in}{0.869414in}}%
\pgfpathlineto{\pgfqpoint{3.969922in}{0.873671in}}%
\pgfpathlineto{\pgfqpoint{3.974179in}{0.873671in}}%
\pgfpathlineto{\pgfqpoint{3.974179in}{0.869414in}}%
\pgfpathmoveto{\pgfqpoint{3.969922in}{0.873671in}}%
\pgfpathlineto{\pgfqpoint{3.969922in}{0.873671in}}%
\pgfpathlineto{\pgfqpoint{3.969922in}{0.877929in}}%
\pgfpathlineto{\pgfqpoint{3.974179in}{0.877929in}}%
\pgfpathlineto{\pgfqpoint{3.974179in}{0.873671in}}%
\pgfpathmoveto{\pgfqpoint{3.969922in}{0.877929in}}%
\pgfpathlineto{\pgfqpoint{3.969922in}{0.877929in}}%
\pgfpathlineto{\pgfqpoint{3.969922in}{0.882186in}}%
\pgfpathlineto{\pgfqpoint{3.974179in}{0.882186in}}%
\pgfpathlineto{\pgfqpoint{3.974179in}{0.877929in}}%
\pgfpathmoveto{\pgfqpoint{3.965664in}{0.882186in}}%
\pgfpathlineto{\pgfqpoint{3.965664in}{0.882186in}}%
\pgfpathlineto{\pgfqpoint{3.965664in}{0.886444in}}%
\pgfpathlineto{\pgfqpoint{3.969922in}{0.886444in}}%
\pgfpathlineto{\pgfqpoint{3.969922in}{0.882186in}}%
\pgfpathmoveto{\pgfqpoint{3.965664in}{0.886444in}}%
\pgfpathlineto{\pgfqpoint{3.965664in}{0.886444in}}%
\pgfpathlineto{\pgfqpoint{3.965664in}{0.890702in}}%
\pgfpathlineto{\pgfqpoint{3.969922in}{0.890702in}}%
\pgfpathlineto{\pgfqpoint{3.969922in}{0.886444in}}%
\pgfpathmoveto{\pgfqpoint{3.965664in}{0.890702in}}%
\pgfpathlineto{\pgfqpoint{3.965664in}{0.890702in}}%
\pgfpathlineto{\pgfqpoint{3.965664in}{0.894959in}}%
\pgfpathlineto{\pgfqpoint{3.969922in}{0.894959in}}%
\pgfpathlineto{\pgfqpoint{3.969922in}{0.890702in}}%
\pgfpathmoveto{\pgfqpoint{3.965664in}{0.894959in}}%
\pgfpathlineto{\pgfqpoint{3.965664in}{0.894959in}}%
\pgfpathlineto{\pgfqpoint{3.965664in}{0.899217in}}%
\pgfpathlineto{\pgfqpoint{3.969922in}{0.899217in}}%
\pgfpathlineto{\pgfqpoint{3.969922in}{0.894959in}}%
\pgfpathmoveto{\pgfqpoint{3.969922in}{0.882186in}}%
\pgfpathlineto{\pgfqpoint{3.969922in}{0.882186in}}%
\pgfpathlineto{\pgfqpoint{3.969922in}{0.886444in}}%
\pgfpathlineto{\pgfqpoint{3.974179in}{0.886444in}}%
\pgfpathlineto{\pgfqpoint{3.974179in}{0.882186in}}%
\pgfpathmoveto{\pgfqpoint{3.969922in}{0.886444in}}%
\pgfpathlineto{\pgfqpoint{3.969922in}{0.886444in}}%
\pgfpathlineto{\pgfqpoint{3.969922in}{0.890702in}}%
\pgfpathlineto{\pgfqpoint{3.974179in}{0.890702in}}%
\pgfpathlineto{\pgfqpoint{3.974179in}{0.886444in}}%
\pgfpathmoveto{\pgfqpoint{3.969922in}{0.890702in}}%
\pgfpathlineto{\pgfqpoint{3.969922in}{0.890702in}}%
\pgfpathlineto{\pgfqpoint{3.969922in}{0.894959in}}%
\pgfpathlineto{\pgfqpoint{3.974179in}{0.894959in}}%
\pgfpathlineto{\pgfqpoint{3.974179in}{0.890702in}}%
\pgfpathmoveto{\pgfqpoint{3.969922in}{0.894959in}}%
\pgfpathlineto{\pgfqpoint{3.969922in}{0.894959in}}%
\pgfpathlineto{\pgfqpoint{3.969922in}{0.899217in}}%
\pgfpathlineto{\pgfqpoint{3.974179in}{0.899217in}}%
\pgfpathlineto{\pgfqpoint{3.974179in}{0.894959in}}%
\pgfpathmoveto{\pgfqpoint{3.965664in}{0.899217in}}%
\pgfpathlineto{\pgfqpoint{3.965664in}{0.899217in}}%
\pgfpathlineto{\pgfqpoint{3.965664in}{0.903474in}}%
\pgfpathlineto{\pgfqpoint{3.969922in}{0.903474in}}%
\pgfpathlineto{\pgfqpoint{3.969922in}{0.899217in}}%
\pgfpathmoveto{\pgfqpoint{3.965664in}{0.903474in}}%
\pgfpathlineto{\pgfqpoint{3.965664in}{0.903474in}}%
\pgfpathlineto{\pgfqpoint{3.965664in}{0.907732in}}%
\pgfpathlineto{\pgfqpoint{3.969922in}{0.907732in}}%
\pgfpathlineto{\pgfqpoint{3.969922in}{0.903474in}}%
\pgfpathmoveto{\pgfqpoint{3.961406in}{0.907732in}}%
\pgfpathlineto{\pgfqpoint{3.961406in}{0.907732in}}%
\pgfpathlineto{\pgfqpoint{3.961406in}{0.911990in}}%
\pgfpathlineto{\pgfqpoint{3.965664in}{0.911990in}}%
\pgfpathlineto{\pgfqpoint{3.965664in}{0.907732in}}%
\pgfpathmoveto{\pgfqpoint{3.961406in}{0.911990in}}%
\pgfpathlineto{\pgfqpoint{3.961406in}{0.911990in}}%
\pgfpathlineto{\pgfqpoint{3.961406in}{0.916247in}}%
\pgfpathlineto{\pgfqpoint{3.965664in}{0.916247in}}%
\pgfpathlineto{\pgfqpoint{3.965664in}{0.911990in}}%
\pgfpathmoveto{\pgfqpoint{3.965664in}{0.907732in}}%
\pgfpathlineto{\pgfqpoint{3.965664in}{0.907732in}}%
\pgfpathlineto{\pgfqpoint{3.965664in}{0.911990in}}%
\pgfpathlineto{\pgfqpoint{3.969922in}{0.911990in}}%
\pgfpathlineto{\pgfqpoint{3.969922in}{0.907732in}}%
\pgfpathmoveto{\pgfqpoint{3.965664in}{0.911990in}}%
\pgfpathlineto{\pgfqpoint{3.965664in}{0.911990in}}%
\pgfpathlineto{\pgfqpoint{3.965664in}{0.916247in}}%
\pgfpathlineto{\pgfqpoint{3.969922in}{0.916247in}}%
\pgfpathlineto{\pgfqpoint{3.969922in}{0.911990in}}%
\pgfpathmoveto{\pgfqpoint{3.969922in}{0.899217in}}%
\pgfpathlineto{\pgfqpoint{3.969922in}{0.899217in}}%
\pgfpathlineto{\pgfqpoint{3.969922in}{0.903474in}}%
\pgfpathlineto{\pgfqpoint{3.974179in}{0.903474in}}%
\pgfpathlineto{\pgfqpoint{3.974179in}{0.899217in}}%
\pgfpathmoveto{\pgfqpoint{3.969922in}{0.903474in}}%
\pgfpathlineto{\pgfqpoint{3.969922in}{0.903474in}}%
\pgfpathlineto{\pgfqpoint{3.969922in}{0.907732in}}%
\pgfpathlineto{\pgfqpoint{3.974179in}{0.907732in}}%
\pgfpathlineto{\pgfqpoint{3.974179in}{0.903474in}}%
\pgfpathmoveto{\pgfqpoint{3.978437in}{0.780004in}}%
\pgfpathlineto{\pgfqpoint{3.978437in}{0.780004in}}%
\pgfpathlineto{\pgfqpoint{3.978437in}{0.784262in}}%
\pgfpathlineto{\pgfqpoint{3.982695in}{0.784262in}}%
\pgfpathlineto{\pgfqpoint{3.982695in}{0.780004in}}%
\pgfpathmoveto{\pgfqpoint{3.978437in}{0.784262in}}%
\pgfpathlineto{\pgfqpoint{3.978437in}{0.784262in}}%
\pgfpathlineto{\pgfqpoint{3.978437in}{0.788519in}}%
\pgfpathlineto{\pgfqpoint{3.982695in}{0.788519in}}%
\pgfpathlineto{\pgfqpoint{3.982695in}{0.784262in}}%
\pgfpathmoveto{\pgfqpoint{3.982695in}{0.780004in}}%
\pgfpathlineto{\pgfqpoint{3.982695in}{0.780004in}}%
\pgfpathlineto{\pgfqpoint{3.982695in}{0.784262in}}%
\pgfpathlineto{\pgfqpoint{3.986952in}{0.784262in}}%
\pgfpathlineto{\pgfqpoint{3.986952in}{0.780004in}}%
\pgfpathmoveto{\pgfqpoint{3.978437in}{0.788519in}}%
\pgfpathlineto{\pgfqpoint{3.978437in}{0.788519in}}%
\pgfpathlineto{\pgfqpoint{3.978437in}{0.792777in}}%
\pgfpathlineto{\pgfqpoint{3.982695in}{0.792777in}}%
\pgfpathlineto{\pgfqpoint{3.982695in}{0.788519in}}%
\pgfpathmoveto{\pgfqpoint{3.978437in}{0.792777in}}%
\pgfpathlineto{\pgfqpoint{3.978437in}{0.792777in}}%
\pgfpathlineto{\pgfqpoint{3.978437in}{0.797034in}}%
\pgfpathlineto{\pgfqpoint{3.982695in}{0.797034in}}%
\pgfpathlineto{\pgfqpoint{3.982695in}{0.792777in}}%
\pgfpathmoveto{\pgfqpoint{3.978437in}{0.797034in}}%
\pgfpathlineto{\pgfqpoint{3.978437in}{0.797034in}}%
\pgfpathlineto{\pgfqpoint{3.978437in}{0.801292in}}%
\pgfpathlineto{\pgfqpoint{3.982695in}{0.801292in}}%
\pgfpathlineto{\pgfqpoint{3.982695in}{0.797034in}}%
\pgfpathmoveto{\pgfqpoint{3.978437in}{0.801292in}}%
\pgfpathlineto{\pgfqpoint{3.978437in}{0.801292in}}%
\pgfpathlineto{\pgfqpoint{3.978437in}{0.805550in}}%
\pgfpathlineto{\pgfqpoint{3.982695in}{0.805550in}}%
\pgfpathlineto{\pgfqpoint{3.982695in}{0.801292in}}%
\pgfpathmoveto{\pgfqpoint{3.978437in}{0.805550in}}%
\pgfpathlineto{\pgfqpoint{3.978437in}{0.805550in}}%
\pgfpathlineto{\pgfqpoint{3.978437in}{0.809807in}}%
\pgfpathlineto{\pgfqpoint{3.982695in}{0.809807in}}%
\pgfpathlineto{\pgfqpoint{3.982695in}{0.805550in}}%
\pgfpathmoveto{\pgfqpoint{3.978437in}{0.809807in}}%
\pgfpathlineto{\pgfqpoint{3.978437in}{0.809807in}}%
\pgfpathlineto{\pgfqpoint{3.978437in}{0.814065in}}%
\pgfpathlineto{\pgfqpoint{3.982695in}{0.814065in}}%
\pgfpathlineto{\pgfqpoint{3.982695in}{0.809807in}}%
\pgfpathmoveto{\pgfqpoint{3.978437in}{0.814065in}}%
\pgfpathlineto{\pgfqpoint{3.978437in}{0.814065in}}%
\pgfpathlineto{\pgfqpoint{3.978437in}{0.818322in}}%
\pgfpathlineto{\pgfqpoint{3.982695in}{0.818322in}}%
\pgfpathlineto{\pgfqpoint{3.982695in}{0.814065in}}%
\pgfpathmoveto{\pgfqpoint{3.978437in}{0.818322in}}%
\pgfpathlineto{\pgfqpoint{3.978437in}{0.818322in}}%
\pgfpathlineto{\pgfqpoint{3.978437in}{0.822580in}}%
\pgfpathlineto{\pgfqpoint{3.982695in}{0.822580in}}%
\pgfpathlineto{\pgfqpoint{3.982695in}{0.818322in}}%
\pgfpathmoveto{\pgfqpoint{3.978437in}{0.822580in}}%
\pgfpathlineto{\pgfqpoint{3.978437in}{0.822580in}}%
\pgfpathlineto{\pgfqpoint{3.978437in}{0.826838in}}%
\pgfpathlineto{\pgfqpoint{3.982695in}{0.826838in}}%
\pgfpathlineto{\pgfqpoint{3.982695in}{0.822580in}}%
\pgfpathmoveto{\pgfqpoint{3.961406in}{0.916247in}}%
\pgfpathlineto{\pgfqpoint{3.961406in}{0.916247in}}%
\pgfpathlineto{\pgfqpoint{3.961406in}{0.920505in}}%
\pgfpathlineto{\pgfqpoint{3.965664in}{0.920505in}}%
\pgfpathlineto{\pgfqpoint{3.965664in}{0.916247in}}%
\pgfpathmoveto{\pgfqpoint{3.961406in}{0.920505in}}%
\pgfpathlineto{\pgfqpoint{3.961406in}{0.920505in}}%
\pgfpathlineto{\pgfqpoint{3.961406in}{0.924763in}}%
\pgfpathlineto{\pgfqpoint{3.965664in}{0.924763in}}%
\pgfpathlineto{\pgfqpoint{3.965664in}{0.920505in}}%
\pgfpathmoveto{\pgfqpoint{3.965664in}{0.916247in}}%
\pgfpathlineto{\pgfqpoint{3.965664in}{0.916247in}}%
\pgfpathlineto{\pgfqpoint{3.965664in}{0.920505in}}%
\pgfpathlineto{\pgfqpoint{3.969922in}{0.920505in}}%
\pgfpathlineto{\pgfqpoint{3.969922in}{0.916247in}}%
\pgfpathmoveto{\pgfqpoint{3.965664in}{0.920505in}}%
\pgfpathlineto{\pgfqpoint{3.965664in}{0.920505in}}%
\pgfpathlineto{\pgfqpoint{3.965664in}{0.924763in}}%
\pgfpathlineto{\pgfqpoint{3.969922in}{0.924763in}}%
\pgfpathlineto{\pgfqpoint{3.969922in}{0.920505in}}%
\pgfpathmoveto{\pgfqpoint{3.961406in}{0.924763in}}%
\pgfpathlineto{\pgfqpoint{3.961406in}{0.924763in}}%
\pgfpathlineto{\pgfqpoint{3.961406in}{0.929021in}}%
\pgfpathlineto{\pgfqpoint{3.965664in}{0.929021in}}%
\pgfpathlineto{\pgfqpoint{3.965664in}{0.924763in}}%
\pgfpathmoveto{\pgfqpoint{3.961406in}{0.929021in}}%
\pgfpathlineto{\pgfqpoint{3.961406in}{0.929021in}}%
\pgfpathlineto{\pgfqpoint{3.961406in}{0.933279in}}%
\pgfpathlineto{\pgfqpoint{3.965664in}{0.933279in}}%
\pgfpathlineto{\pgfqpoint{3.965664in}{0.929021in}}%
\pgfpathmoveto{\pgfqpoint{3.965664in}{0.924763in}}%
\pgfpathlineto{\pgfqpoint{3.965664in}{0.924763in}}%
\pgfpathlineto{\pgfqpoint{3.965664in}{0.929021in}}%
\pgfpathlineto{\pgfqpoint{3.969922in}{0.929021in}}%
\pgfpathlineto{\pgfqpoint{3.969922in}{0.924763in}}%
\pgfpathmoveto{\pgfqpoint{3.965664in}{0.929021in}}%
\pgfpathlineto{\pgfqpoint{3.965664in}{0.929021in}}%
\pgfpathlineto{\pgfqpoint{3.965664in}{0.933279in}}%
\pgfpathlineto{\pgfqpoint{3.969922in}{0.933279in}}%
\pgfpathlineto{\pgfqpoint{3.969922in}{0.929021in}}%
\pgfpathmoveto{\pgfqpoint{3.961406in}{0.933279in}}%
\pgfpathlineto{\pgfqpoint{3.961406in}{0.933279in}}%
\pgfpathlineto{\pgfqpoint{3.961406in}{0.937537in}}%
\pgfpathlineto{\pgfqpoint{3.965664in}{0.937537in}}%
\pgfpathlineto{\pgfqpoint{3.965664in}{0.933279in}}%
\pgfpathmoveto{\pgfqpoint{3.961406in}{0.937537in}}%
\pgfpathlineto{\pgfqpoint{3.961406in}{0.937537in}}%
\pgfpathlineto{\pgfqpoint{3.961406in}{0.941795in}}%
\pgfpathlineto{\pgfqpoint{3.965664in}{0.941795in}}%
\pgfpathlineto{\pgfqpoint{3.965664in}{0.937537in}}%
\pgfpathmoveto{\pgfqpoint{3.965664in}{0.933279in}}%
\pgfpathlineto{\pgfqpoint{3.965664in}{0.933279in}}%
\pgfpathlineto{\pgfqpoint{3.965664in}{0.937537in}}%
\pgfpathlineto{\pgfqpoint{3.969922in}{0.937537in}}%
\pgfpathlineto{\pgfqpoint{3.969922in}{0.933279in}}%
\pgfpathmoveto{\pgfqpoint{3.965664in}{0.937537in}}%
\pgfpathlineto{\pgfqpoint{3.965664in}{0.937537in}}%
\pgfpathlineto{\pgfqpoint{3.965664in}{0.941795in}}%
\pgfpathlineto{\pgfqpoint{3.969922in}{0.941795in}}%
\pgfpathlineto{\pgfqpoint{3.969922in}{0.937537in}}%
\pgfpathmoveto{\pgfqpoint{3.961406in}{0.941795in}}%
\pgfpathlineto{\pgfqpoint{3.961406in}{0.941795in}}%
\pgfpathlineto{\pgfqpoint{3.961406in}{0.946053in}}%
\pgfpathlineto{\pgfqpoint{3.965664in}{0.946053in}}%
\pgfpathlineto{\pgfqpoint{3.965664in}{0.941795in}}%
\pgfpathmoveto{\pgfqpoint{3.961406in}{0.946053in}}%
\pgfpathlineto{\pgfqpoint{3.961406in}{0.946053in}}%
\pgfpathlineto{\pgfqpoint{3.961406in}{0.950311in}}%
\pgfpathlineto{\pgfqpoint{3.965664in}{0.950311in}}%
\pgfpathlineto{\pgfqpoint{3.965664in}{0.946053in}}%
\pgfpathmoveto{\pgfqpoint{3.965664in}{0.941795in}}%
\pgfpathlineto{\pgfqpoint{3.965664in}{0.941795in}}%
\pgfpathlineto{\pgfqpoint{3.965664in}{0.946053in}}%
\pgfpathlineto{\pgfqpoint{3.969922in}{0.946053in}}%
\pgfpathlineto{\pgfqpoint{3.969922in}{0.941795in}}%
\pgfpathmoveto{\pgfqpoint{3.957149in}{0.950311in}}%
\pgfpathlineto{\pgfqpoint{3.957149in}{0.950311in}}%
\pgfpathlineto{\pgfqpoint{3.957149in}{0.954569in}}%
\pgfpathlineto{\pgfqpoint{3.961406in}{0.954569in}}%
\pgfpathlineto{\pgfqpoint{3.961406in}{0.950311in}}%
\pgfpathmoveto{\pgfqpoint{3.957149in}{0.954569in}}%
\pgfpathlineto{\pgfqpoint{3.957149in}{0.954569in}}%
\pgfpathlineto{\pgfqpoint{3.957149in}{0.958827in}}%
\pgfpathlineto{\pgfqpoint{3.961406in}{0.958827in}}%
\pgfpathlineto{\pgfqpoint{3.961406in}{0.954569in}}%
\pgfpathmoveto{\pgfqpoint{3.957149in}{0.958827in}}%
\pgfpathlineto{\pgfqpoint{3.957149in}{0.958827in}}%
\pgfpathlineto{\pgfqpoint{3.957149in}{0.963085in}}%
\pgfpathlineto{\pgfqpoint{3.961406in}{0.963085in}}%
\pgfpathlineto{\pgfqpoint{3.961406in}{0.958827in}}%
\pgfpathmoveto{\pgfqpoint{3.957149in}{0.963085in}}%
\pgfpathlineto{\pgfqpoint{3.957149in}{0.963085in}}%
\pgfpathlineto{\pgfqpoint{3.957149in}{0.967343in}}%
\pgfpathlineto{\pgfqpoint{3.961406in}{0.967343in}}%
\pgfpathlineto{\pgfqpoint{3.961406in}{0.963085in}}%
\pgfpathmoveto{\pgfqpoint{3.957149in}{0.967343in}}%
\pgfpathlineto{\pgfqpoint{3.957149in}{0.967343in}}%
\pgfpathlineto{\pgfqpoint{3.957149in}{0.971601in}}%
\pgfpathlineto{\pgfqpoint{3.961406in}{0.971601in}}%
\pgfpathlineto{\pgfqpoint{3.961406in}{0.967343in}}%
\pgfpathmoveto{\pgfqpoint{3.957149in}{0.971601in}}%
\pgfpathlineto{\pgfqpoint{3.957149in}{0.971601in}}%
\pgfpathlineto{\pgfqpoint{3.957149in}{0.975859in}}%
\pgfpathlineto{\pgfqpoint{3.961406in}{0.975859in}}%
\pgfpathlineto{\pgfqpoint{3.961406in}{0.971601in}}%
\pgfpathmoveto{\pgfqpoint{3.957149in}{0.975859in}}%
\pgfpathlineto{\pgfqpoint{3.957149in}{0.975859in}}%
\pgfpathlineto{\pgfqpoint{3.957149in}{0.980117in}}%
\pgfpathlineto{\pgfqpoint{3.961406in}{0.980117in}}%
\pgfpathlineto{\pgfqpoint{3.961406in}{0.975859in}}%
\pgfpathmoveto{\pgfqpoint{3.957149in}{0.980117in}}%
\pgfpathlineto{\pgfqpoint{3.957149in}{0.980117in}}%
\pgfpathlineto{\pgfqpoint{3.957149in}{0.984375in}}%
\pgfpathlineto{\pgfqpoint{3.961406in}{0.984375in}}%
\pgfpathlineto{\pgfqpoint{3.961406in}{0.980117in}}%
\pgfpathmoveto{\pgfqpoint{3.961406in}{0.950311in}}%
\pgfpathlineto{\pgfqpoint{3.961406in}{0.950311in}}%
\pgfpathlineto{\pgfqpoint{3.961406in}{0.954569in}}%
\pgfpathlineto{\pgfqpoint{3.965664in}{0.954569in}}%
\pgfpathlineto{\pgfqpoint{3.965664in}{0.950311in}}%
\pgfpathmoveto{\pgfqpoint{3.961406in}{0.954569in}}%
\pgfpathlineto{\pgfqpoint{3.961406in}{0.954569in}}%
\pgfpathlineto{\pgfqpoint{3.961406in}{0.958827in}}%
\pgfpathlineto{\pgfqpoint{3.965664in}{0.958827in}}%
\pgfpathlineto{\pgfqpoint{3.965664in}{0.954569in}}%
\pgfpathmoveto{\pgfqpoint{3.961406in}{0.958827in}}%
\pgfpathlineto{\pgfqpoint{3.961406in}{0.958827in}}%
\pgfpathlineto{\pgfqpoint{3.961406in}{0.963085in}}%
\pgfpathlineto{\pgfqpoint{3.965664in}{0.963085in}}%
\pgfpathlineto{\pgfqpoint{3.965664in}{0.958827in}}%
\pgfpathmoveto{\pgfqpoint{3.961406in}{0.963085in}}%
\pgfpathlineto{\pgfqpoint{3.961406in}{0.963085in}}%
\pgfpathlineto{\pgfqpoint{3.961406in}{0.967343in}}%
\pgfpathlineto{\pgfqpoint{3.965664in}{0.967343in}}%
\pgfpathlineto{\pgfqpoint{3.965664in}{0.963085in}}%
\pgfpathmoveto{\pgfqpoint{3.961406in}{0.967343in}}%
\pgfpathlineto{\pgfqpoint{3.961406in}{0.967343in}}%
\pgfpathlineto{\pgfqpoint{3.961406in}{0.971601in}}%
\pgfpathlineto{\pgfqpoint{3.965664in}{0.971601in}}%
\pgfpathlineto{\pgfqpoint{3.965664in}{0.967343in}}%
\pgfpathmoveto{\pgfqpoint{3.961406in}{0.971601in}}%
\pgfpathlineto{\pgfqpoint{3.961406in}{0.971601in}}%
\pgfpathlineto{\pgfqpoint{3.961406in}{0.975859in}}%
\pgfpathlineto{\pgfqpoint{3.965664in}{0.975859in}}%
\pgfpathlineto{\pgfqpoint{3.965664in}{0.971601in}}%
\pgfpathmoveto{\pgfqpoint{3.961406in}{0.975859in}}%
\pgfpathlineto{\pgfqpoint{3.961406in}{0.975859in}}%
\pgfpathlineto{\pgfqpoint{3.961406in}{0.980117in}}%
\pgfpathlineto{\pgfqpoint{3.965664in}{0.980117in}}%
\pgfpathlineto{\pgfqpoint{3.965664in}{0.975859in}}%
\pgfpathmoveto{\pgfqpoint{3.961406in}{0.980117in}}%
\pgfpathlineto{\pgfqpoint{3.961406in}{0.980117in}}%
\pgfpathlineto{\pgfqpoint{3.961406in}{0.984375in}}%
\pgfpathlineto{\pgfqpoint{3.965664in}{0.984375in}}%
\pgfpathlineto{\pgfqpoint{3.965664in}{0.980117in}}%
\pgfpathmoveto{\pgfqpoint{3.952891in}{0.988633in}}%
\pgfpathlineto{\pgfqpoint{3.952891in}{0.988633in}}%
\pgfpathlineto{\pgfqpoint{3.952891in}{0.992891in}}%
\pgfpathlineto{\pgfqpoint{3.957149in}{0.992891in}}%
\pgfpathlineto{\pgfqpoint{3.957149in}{0.988633in}}%
\pgfpathmoveto{\pgfqpoint{3.957149in}{0.984375in}}%
\pgfpathlineto{\pgfqpoint{3.957149in}{0.984375in}}%
\pgfpathlineto{\pgfqpoint{3.957149in}{0.988633in}}%
\pgfpathlineto{\pgfqpoint{3.961406in}{0.988633in}}%
\pgfpathlineto{\pgfqpoint{3.961406in}{0.984375in}}%
\pgfpathmoveto{\pgfqpoint{3.957149in}{0.988633in}}%
\pgfpathlineto{\pgfqpoint{3.957149in}{0.988633in}}%
\pgfpathlineto{\pgfqpoint{3.957149in}{0.992891in}}%
\pgfpathlineto{\pgfqpoint{3.961406in}{0.992891in}}%
\pgfpathlineto{\pgfqpoint{3.961406in}{0.988633in}}%
\pgfpathmoveto{\pgfqpoint{3.952891in}{0.992891in}}%
\pgfpathlineto{\pgfqpoint{3.952891in}{0.992891in}}%
\pgfpathlineto{\pgfqpoint{3.952891in}{0.997148in}}%
\pgfpathlineto{\pgfqpoint{3.957149in}{0.997148in}}%
\pgfpathlineto{\pgfqpoint{3.957149in}{0.992891in}}%
\pgfpathmoveto{\pgfqpoint{3.952891in}{0.997148in}}%
\pgfpathlineto{\pgfqpoint{3.952891in}{0.997148in}}%
\pgfpathlineto{\pgfqpoint{3.952891in}{1.001406in}}%
\pgfpathlineto{\pgfqpoint{3.957149in}{1.001406in}}%
\pgfpathlineto{\pgfqpoint{3.957149in}{0.997148in}}%
\pgfpathmoveto{\pgfqpoint{3.957149in}{0.992891in}}%
\pgfpathlineto{\pgfqpoint{3.957149in}{0.992891in}}%
\pgfpathlineto{\pgfqpoint{3.957149in}{0.997148in}}%
\pgfpathlineto{\pgfqpoint{3.961406in}{0.997148in}}%
\pgfpathlineto{\pgfqpoint{3.961406in}{0.992891in}}%
\pgfpathmoveto{\pgfqpoint{3.957149in}{0.997148in}}%
\pgfpathlineto{\pgfqpoint{3.957149in}{0.997148in}}%
\pgfpathlineto{\pgfqpoint{3.957149in}{1.001406in}}%
\pgfpathlineto{\pgfqpoint{3.961406in}{1.001406in}}%
\pgfpathlineto{\pgfqpoint{3.961406in}{0.997148in}}%
\pgfpathmoveto{\pgfqpoint{3.952891in}{1.001406in}}%
\pgfpathlineto{\pgfqpoint{3.952891in}{1.001406in}}%
\pgfpathlineto{\pgfqpoint{3.952891in}{1.005664in}}%
\pgfpathlineto{\pgfqpoint{3.957149in}{1.005664in}}%
\pgfpathlineto{\pgfqpoint{3.957149in}{1.001406in}}%
\pgfpathmoveto{\pgfqpoint{3.952891in}{1.005664in}}%
\pgfpathlineto{\pgfqpoint{3.952891in}{1.005664in}}%
\pgfpathlineto{\pgfqpoint{3.952891in}{1.009922in}}%
\pgfpathlineto{\pgfqpoint{3.957149in}{1.009922in}}%
\pgfpathlineto{\pgfqpoint{3.957149in}{1.005664in}}%
\pgfpathmoveto{\pgfqpoint{3.957149in}{1.001406in}}%
\pgfpathlineto{\pgfqpoint{3.957149in}{1.001406in}}%
\pgfpathlineto{\pgfqpoint{3.957149in}{1.005664in}}%
\pgfpathlineto{\pgfqpoint{3.961406in}{1.005664in}}%
\pgfpathlineto{\pgfqpoint{3.961406in}{1.001406in}}%
\pgfpathmoveto{\pgfqpoint{3.957149in}{1.005664in}}%
\pgfpathlineto{\pgfqpoint{3.957149in}{1.005664in}}%
\pgfpathlineto{\pgfqpoint{3.957149in}{1.009922in}}%
\pgfpathlineto{\pgfqpoint{3.961406in}{1.009922in}}%
\pgfpathlineto{\pgfqpoint{3.961406in}{1.005664in}}%
\pgfpathmoveto{\pgfqpoint{3.952891in}{1.009922in}}%
\pgfpathlineto{\pgfqpoint{3.952891in}{1.009922in}}%
\pgfpathlineto{\pgfqpoint{3.952891in}{1.014180in}}%
\pgfpathlineto{\pgfqpoint{3.957149in}{1.014180in}}%
\pgfpathlineto{\pgfqpoint{3.957149in}{1.009922in}}%
\pgfpathmoveto{\pgfqpoint{3.952891in}{1.014180in}}%
\pgfpathlineto{\pgfqpoint{3.952891in}{1.014180in}}%
\pgfpathlineto{\pgfqpoint{3.952891in}{1.018438in}}%
\pgfpathlineto{\pgfqpoint{3.957149in}{1.018438in}}%
\pgfpathlineto{\pgfqpoint{3.957149in}{1.014180in}}%
\pgfpathmoveto{\pgfqpoint{3.957149in}{1.009922in}}%
\pgfpathlineto{\pgfqpoint{3.957149in}{1.009922in}}%
\pgfpathlineto{\pgfqpoint{3.957149in}{1.014180in}}%
\pgfpathlineto{\pgfqpoint{3.961406in}{1.014180in}}%
\pgfpathlineto{\pgfqpoint{3.961406in}{1.009922in}}%
\pgfpathmoveto{\pgfqpoint{3.957149in}{1.014180in}}%
\pgfpathlineto{\pgfqpoint{3.957149in}{1.014180in}}%
\pgfpathlineto{\pgfqpoint{3.957149in}{1.018438in}}%
\pgfpathlineto{\pgfqpoint{3.961406in}{1.018438in}}%
\pgfpathlineto{\pgfqpoint{3.961406in}{1.014180in}}%
\pgfpathmoveto{\pgfqpoint{3.961406in}{0.984375in}}%
\pgfpathlineto{\pgfqpoint{3.961406in}{0.984375in}}%
\pgfpathlineto{\pgfqpoint{3.961406in}{0.988633in}}%
\pgfpathlineto{\pgfqpoint{3.965664in}{0.988633in}}%
\pgfpathlineto{\pgfqpoint{3.965664in}{0.984375in}}%
\pgfpathmoveto{\pgfqpoint{3.948633in}{1.031212in}}%
\pgfpathlineto{\pgfqpoint{3.948633in}{1.031212in}}%
\pgfpathlineto{\pgfqpoint{3.948633in}{1.035470in}}%
\pgfpathlineto{\pgfqpoint{3.952891in}{1.035470in}}%
\pgfpathlineto{\pgfqpoint{3.952891in}{1.031212in}}%
\pgfpathmoveto{\pgfqpoint{3.952891in}{1.018438in}}%
\pgfpathlineto{\pgfqpoint{3.952891in}{1.018438in}}%
\pgfpathlineto{\pgfqpoint{3.952891in}{1.022696in}}%
\pgfpathlineto{\pgfqpoint{3.957149in}{1.022696in}}%
\pgfpathlineto{\pgfqpoint{3.957149in}{1.018438in}}%
\pgfpathmoveto{\pgfqpoint{3.952891in}{1.022696in}}%
\pgfpathlineto{\pgfqpoint{3.952891in}{1.022696in}}%
\pgfpathlineto{\pgfqpoint{3.952891in}{1.026954in}}%
\pgfpathlineto{\pgfqpoint{3.957149in}{1.026954in}}%
\pgfpathlineto{\pgfqpoint{3.957149in}{1.022696in}}%
\pgfpathmoveto{\pgfqpoint{3.957149in}{1.018438in}}%
\pgfpathlineto{\pgfqpoint{3.957149in}{1.018438in}}%
\pgfpathlineto{\pgfqpoint{3.957149in}{1.022696in}}%
\pgfpathlineto{\pgfqpoint{3.961406in}{1.022696in}}%
\pgfpathlineto{\pgfqpoint{3.961406in}{1.018438in}}%
\pgfpathmoveto{\pgfqpoint{3.957149in}{1.022696in}}%
\pgfpathlineto{\pgfqpoint{3.957149in}{1.022696in}}%
\pgfpathlineto{\pgfqpoint{3.957149in}{1.026954in}}%
\pgfpathlineto{\pgfqpoint{3.961406in}{1.026954in}}%
\pgfpathlineto{\pgfqpoint{3.961406in}{1.022696in}}%
\pgfpathmoveto{\pgfqpoint{3.952891in}{1.026954in}}%
\pgfpathlineto{\pgfqpoint{3.952891in}{1.026954in}}%
\pgfpathlineto{\pgfqpoint{3.952891in}{1.031212in}}%
\pgfpathlineto{\pgfqpoint{3.957149in}{1.031212in}}%
\pgfpathlineto{\pgfqpoint{3.957149in}{1.026954in}}%
\pgfpathmoveto{\pgfqpoint{3.952891in}{1.031212in}}%
\pgfpathlineto{\pgfqpoint{3.952891in}{1.031212in}}%
\pgfpathlineto{\pgfqpoint{3.952891in}{1.035470in}}%
\pgfpathlineto{\pgfqpoint{3.957149in}{1.035470in}}%
\pgfpathlineto{\pgfqpoint{3.957149in}{1.031212in}}%
\pgfpathmoveto{\pgfqpoint{3.948633in}{1.035470in}}%
\pgfpathlineto{\pgfqpoint{3.948633in}{1.035470in}}%
\pgfpathlineto{\pgfqpoint{3.948633in}{1.039728in}}%
\pgfpathlineto{\pgfqpoint{3.952891in}{1.039728in}}%
\pgfpathlineto{\pgfqpoint{3.952891in}{1.035470in}}%
\pgfpathmoveto{\pgfqpoint{3.948633in}{1.039728in}}%
\pgfpathlineto{\pgfqpoint{3.948633in}{1.039728in}}%
\pgfpathlineto{\pgfqpoint{3.948633in}{1.043986in}}%
\pgfpathlineto{\pgfqpoint{3.952891in}{1.043986in}}%
\pgfpathlineto{\pgfqpoint{3.952891in}{1.039728in}}%
\pgfpathmoveto{\pgfqpoint{3.948633in}{1.043986in}}%
\pgfpathlineto{\pgfqpoint{3.948633in}{1.043986in}}%
\pgfpathlineto{\pgfqpoint{3.948633in}{1.048244in}}%
\pgfpathlineto{\pgfqpoint{3.952891in}{1.048244in}}%
\pgfpathlineto{\pgfqpoint{3.952891in}{1.043986in}}%
\pgfpathmoveto{\pgfqpoint{3.948633in}{1.048244in}}%
\pgfpathlineto{\pgfqpoint{3.948633in}{1.048244in}}%
\pgfpathlineto{\pgfqpoint{3.948633in}{1.052502in}}%
\pgfpathlineto{\pgfqpoint{3.952891in}{1.052502in}}%
\pgfpathlineto{\pgfqpoint{3.952891in}{1.048244in}}%
\pgfpathmoveto{\pgfqpoint{3.952891in}{1.035470in}}%
\pgfpathlineto{\pgfqpoint{3.952891in}{1.035470in}}%
\pgfpathlineto{\pgfqpoint{3.952891in}{1.039728in}}%
\pgfpathlineto{\pgfqpoint{3.957149in}{1.039728in}}%
\pgfpathlineto{\pgfqpoint{3.957149in}{1.035470in}}%
\pgfpathmoveto{\pgfqpoint{3.952891in}{1.039728in}}%
\pgfpathlineto{\pgfqpoint{3.952891in}{1.039728in}}%
\pgfpathlineto{\pgfqpoint{3.952891in}{1.043986in}}%
\pgfpathlineto{\pgfqpoint{3.957149in}{1.043986in}}%
\pgfpathlineto{\pgfqpoint{3.957149in}{1.039728in}}%
\pgfpathmoveto{\pgfqpoint{3.952891in}{1.043986in}}%
\pgfpathlineto{\pgfqpoint{3.952891in}{1.043986in}}%
\pgfpathlineto{\pgfqpoint{3.952891in}{1.048244in}}%
\pgfpathlineto{\pgfqpoint{3.957149in}{1.048244in}}%
\pgfpathlineto{\pgfqpoint{3.957149in}{1.043986in}}%
\pgfpathmoveto{\pgfqpoint{3.952891in}{1.048244in}}%
\pgfpathlineto{\pgfqpoint{3.952891in}{1.048244in}}%
\pgfpathlineto{\pgfqpoint{3.952891in}{1.052502in}}%
\pgfpathlineto{\pgfqpoint{3.957149in}{1.052502in}}%
\pgfpathlineto{\pgfqpoint{3.957149in}{1.048244in}}%
\pgfpathmoveto{\pgfqpoint{3.940118in}{1.107854in}}%
\pgfpathlineto{\pgfqpoint{3.940118in}{1.107854in}}%
\pgfpathlineto{\pgfqpoint{3.940118in}{1.112112in}}%
\pgfpathlineto{\pgfqpoint{3.944375in}{1.112112in}}%
\pgfpathlineto{\pgfqpoint{3.944375in}{1.107854in}}%
\pgfpathmoveto{\pgfqpoint{3.940118in}{1.112112in}}%
\pgfpathlineto{\pgfqpoint{3.940118in}{1.112112in}}%
\pgfpathlineto{\pgfqpoint{3.940118in}{1.116370in}}%
\pgfpathlineto{\pgfqpoint{3.944375in}{1.116370in}}%
\pgfpathlineto{\pgfqpoint{3.944375in}{1.112112in}}%
\pgfpathmoveto{\pgfqpoint{3.940118in}{1.116370in}}%
\pgfpathlineto{\pgfqpoint{3.940118in}{1.116370in}}%
\pgfpathlineto{\pgfqpoint{3.940118in}{1.120628in}}%
\pgfpathlineto{\pgfqpoint{3.944375in}{1.120628in}}%
\pgfpathlineto{\pgfqpoint{3.944375in}{1.116370in}}%
\pgfpathmoveto{\pgfqpoint{3.948633in}{1.052502in}}%
\pgfpathlineto{\pgfqpoint{3.948633in}{1.052502in}}%
\pgfpathlineto{\pgfqpoint{3.948633in}{1.056760in}}%
\pgfpathlineto{\pgfqpoint{3.952891in}{1.056760in}}%
\pgfpathlineto{\pgfqpoint{3.952891in}{1.052502in}}%
\pgfpathmoveto{\pgfqpoint{3.948633in}{1.056760in}}%
\pgfpathlineto{\pgfqpoint{3.948633in}{1.056760in}}%
\pgfpathlineto{\pgfqpoint{3.948633in}{1.061018in}}%
\pgfpathlineto{\pgfqpoint{3.952891in}{1.061018in}}%
\pgfpathlineto{\pgfqpoint{3.952891in}{1.056760in}}%
\pgfpathmoveto{\pgfqpoint{3.948633in}{1.061018in}}%
\pgfpathlineto{\pgfqpoint{3.948633in}{1.061018in}}%
\pgfpathlineto{\pgfqpoint{3.948633in}{1.065276in}}%
\pgfpathlineto{\pgfqpoint{3.952891in}{1.065276in}}%
\pgfpathlineto{\pgfqpoint{3.952891in}{1.061018in}}%
\pgfpathmoveto{\pgfqpoint{3.948633in}{1.065276in}}%
\pgfpathlineto{\pgfqpoint{3.948633in}{1.065276in}}%
\pgfpathlineto{\pgfqpoint{3.948633in}{1.069533in}}%
\pgfpathlineto{\pgfqpoint{3.952891in}{1.069533in}}%
\pgfpathlineto{\pgfqpoint{3.952891in}{1.065276in}}%
\pgfpathmoveto{\pgfqpoint{3.952891in}{1.052502in}}%
\pgfpathlineto{\pgfqpoint{3.952891in}{1.052502in}}%
\pgfpathlineto{\pgfqpoint{3.952891in}{1.056760in}}%
\pgfpathlineto{\pgfqpoint{3.957149in}{1.056760in}}%
\pgfpathlineto{\pgfqpoint{3.957149in}{1.052502in}}%
\pgfpathmoveto{\pgfqpoint{3.952891in}{1.056760in}}%
\pgfpathlineto{\pgfqpoint{3.952891in}{1.056760in}}%
\pgfpathlineto{\pgfqpoint{3.952891in}{1.061018in}}%
\pgfpathlineto{\pgfqpoint{3.957149in}{1.061018in}}%
\pgfpathlineto{\pgfqpoint{3.957149in}{1.056760in}}%
\pgfpathmoveto{\pgfqpoint{3.952891in}{1.061018in}}%
\pgfpathlineto{\pgfqpoint{3.952891in}{1.061018in}}%
\pgfpathlineto{\pgfqpoint{3.952891in}{1.065276in}}%
\pgfpathlineto{\pgfqpoint{3.957149in}{1.065276in}}%
\pgfpathlineto{\pgfqpoint{3.957149in}{1.061018in}}%
\pgfpathmoveto{\pgfqpoint{3.952891in}{1.065276in}}%
\pgfpathlineto{\pgfqpoint{3.952891in}{1.065276in}}%
\pgfpathlineto{\pgfqpoint{3.952891in}{1.069533in}}%
\pgfpathlineto{\pgfqpoint{3.957149in}{1.069533in}}%
\pgfpathlineto{\pgfqpoint{3.957149in}{1.065276in}}%
\pgfpathmoveto{\pgfqpoint{3.944375in}{1.069533in}}%
\pgfpathlineto{\pgfqpoint{3.944375in}{1.069533in}}%
\pgfpathlineto{\pgfqpoint{3.944375in}{1.073791in}}%
\pgfpathlineto{\pgfqpoint{3.948633in}{1.073791in}}%
\pgfpathlineto{\pgfqpoint{3.948633in}{1.069533in}}%
\pgfpathmoveto{\pgfqpoint{3.944375in}{1.073791in}}%
\pgfpathlineto{\pgfqpoint{3.944375in}{1.073791in}}%
\pgfpathlineto{\pgfqpoint{3.944375in}{1.078049in}}%
\pgfpathlineto{\pgfqpoint{3.948633in}{1.078049in}}%
\pgfpathlineto{\pgfqpoint{3.948633in}{1.073791in}}%
\pgfpathmoveto{\pgfqpoint{3.948633in}{1.069533in}}%
\pgfpathlineto{\pgfqpoint{3.948633in}{1.069533in}}%
\pgfpathlineto{\pgfqpoint{3.948633in}{1.073791in}}%
\pgfpathlineto{\pgfqpoint{3.952891in}{1.073791in}}%
\pgfpathlineto{\pgfqpoint{3.952891in}{1.069533in}}%
\pgfpathmoveto{\pgfqpoint{3.948633in}{1.073791in}}%
\pgfpathlineto{\pgfqpoint{3.948633in}{1.073791in}}%
\pgfpathlineto{\pgfqpoint{3.948633in}{1.078049in}}%
\pgfpathlineto{\pgfqpoint{3.952891in}{1.078049in}}%
\pgfpathlineto{\pgfqpoint{3.952891in}{1.073791in}}%
\pgfpathmoveto{\pgfqpoint{3.944375in}{1.078049in}}%
\pgfpathlineto{\pgfqpoint{3.944375in}{1.078049in}}%
\pgfpathlineto{\pgfqpoint{3.944375in}{1.082307in}}%
\pgfpathlineto{\pgfqpoint{3.948633in}{1.082307in}}%
\pgfpathlineto{\pgfqpoint{3.948633in}{1.078049in}}%
\pgfpathmoveto{\pgfqpoint{3.944375in}{1.082307in}}%
\pgfpathlineto{\pgfqpoint{3.944375in}{1.082307in}}%
\pgfpathlineto{\pgfqpoint{3.944375in}{1.086565in}}%
\pgfpathlineto{\pgfqpoint{3.948633in}{1.086565in}}%
\pgfpathlineto{\pgfqpoint{3.948633in}{1.082307in}}%
\pgfpathmoveto{\pgfqpoint{3.948633in}{1.078049in}}%
\pgfpathlineto{\pgfqpoint{3.948633in}{1.078049in}}%
\pgfpathlineto{\pgfqpoint{3.948633in}{1.082307in}}%
\pgfpathlineto{\pgfqpoint{3.952891in}{1.082307in}}%
\pgfpathlineto{\pgfqpoint{3.952891in}{1.078049in}}%
\pgfpathmoveto{\pgfqpoint{3.948633in}{1.082307in}}%
\pgfpathlineto{\pgfqpoint{3.948633in}{1.082307in}}%
\pgfpathlineto{\pgfqpoint{3.948633in}{1.086565in}}%
\pgfpathlineto{\pgfqpoint{3.952891in}{1.086565in}}%
\pgfpathlineto{\pgfqpoint{3.952891in}{1.082307in}}%
\pgfpathmoveto{\pgfqpoint{3.944375in}{1.086565in}}%
\pgfpathlineto{\pgfqpoint{3.944375in}{1.086565in}}%
\pgfpathlineto{\pgfqpoint{3.944375in}{1.090823in}}%
\pgfpathlineto{\pgfqpoint{3.948633in}{1.090823in}}%
\pgfpathlineto{\pgfqpoint{3.948633in}{1.086565in}}%
\pgfpathmoveto{\pgfqpoint{3.944375in}{1.090823in}}%
\pgfpathlineto{\pgfqpoint{3.944375in}{1.090823in}}%
\pgfpathlineto{\pgfqpoint{3.944375in}{1.095081in}}%
\pgfpathlineto{\pgfqpoint{3.948633in}{1.095081in}}%
\pgfpathlineto{\pgfqpoint{3.948633in}{1.090823in}}%
\pgfpathmoveto{\pgfqpoint{3.948633in}{1.086565in}}%
\pgfpathlineto{\pgfqpoint{3.948633in}{1.086565in}}%
\pgfpathlineto{\pgfqpoint{3.948633in}{1.090823in}}%
\pgfpathlineto{\pgfqpoint{3.952891in}{1.090823in}}%
\pgfpathlineto{\pgfqpoint{3.952891in}{1.086565in}}%
\pgfpathmoveto{\pgfqpoint{3.948633in}{1.090823in}}%
\pgfpathlineto{\pgfqpoint{3.948633in}{1.090823in}}%
\pgfpathlineto{\pgfqpoint{3.948633in}{1.095081in}}%
\pgfpathlineto{\pgfqpoint{3.952891in}{1.095081in}}%
\pgfpathlineto{\pgfqpoint{3.952891in}{1.090823in}}%
\pgfpathmoveto{\pgfqpoint{3.944375in}{1.095081in}}%
\pgfpathlineto{\pgfqpoint{3.944375in}{1.095081in}}%
\pgfpathlineto{\pgfqpoint{3.944375in}{1.099339in}}%
\pgfpathlineto{\pgfqpoint{3.948633in}{1.099339in}}%
\pgfpathlineto{\pgfqpoint{3.948633in}{1.095081in}}%
\pgfpathmoveto{\pgfqpoint{3.944375in}{1.099339in}}%
\pgfpathlineto{\pgfqpoint{3.944375in}{1.099339in}}%
\pgfpathlineto{\pgfqpoint{3.944375in}{1.103597in}}%
\pgfpathlineto{\pgfqpoint{3.948633in}{1.103597in}}%
\pgfpathlineto{\pgfqpoint{3.948633in}{1.099339in}}%
\pgfpathmoveto{\pgfqpoint{3.948633in}{1.095081in}}%
\pgfpathlineto{\pgfqpoint{3.948633in}{1.095081in}}%
\pgfpathlineto{\pgfqpoint{3.948633in}{1.099339in}}%
\pgfpathlineto{\pgfqpoint{3.952891in}{1.099339in}}%
\pgfpathlineto{\pgfqpoint{3.952891in}{1.095081in}}%
\pgfpathmoveto{\pgfqpoint{3.948633in}{1.099339in}}%
\pgfpathlineto{\pgfqpoint{3.948633in}{1.099339in}}%
\pgfpathlineto{\pgfqpoint{3.948633in}{1.103597in}}%
\pgfpathlineto{\pgfqpoint{3.952891in}{1.103597in}}%
\pgfpathlineto{\pgfqpoint{3.952891in}{1.099339in}}%
\pgfpathmoveto{\pgfqpoint{3.944375in}{1.103597in}}%
\pgfpathlineto{\pgfqpoint{3.944375in}{1.103597in}}%
\pgfpathlineto{\pgfqpoint{3.944375in}{1.107854in}}%
\pgfpathlineto{\pgfqpoint{3.948633in}{1.107854in}}%
\pgfpathlineto{\pgfqpoint{3.948633in}{1.103597in}}%
\pgfpathmoveto{\pgfqpoint{3.944375in}{1.107854in}}%
\pgfpathlineto{\pgfqpoint{3.944375in}{1.107854in}}%
\pgfpathlineto{\pgfqpoint{3.944375in}{1.112112in}}%
\pgfpathlineto{\pgfqpoint{3.948633in}{1.112112in}}%
\pgfpathlineto{\pgfqpoint{3.948633in}{1.107854in}}%
\pgfpathmoveto{\pgfqpoint{3.948633in}{1.103597in}}%
\pgfpathlineto{\pgfqpoint{3.948633in}{1.103597in}}%
\pgfpathlineto{\pgfqpoint{3.948633in}{1.107854in}}%
\pgfpathlineto{\pgfqpoint{3.952891in}{1.107854in}}%
\pgfpathlineto{\pgfqpoint{3.952891in}{1.103597in}}%
\pgfpathmoveto{\pgfqpoint{3.944375in}{1.112112in}}%
\pgfpathlineto{\pgfqpoint{3.944375in}{1.112112in}}%
\pgfpathlineto{\pgfqpoint{3.944375in}{1.116370in}}%
\pgfpathlineto{\pgfqpoint{3.948633in}{1.116370in}}%
\pgfpathlineto{\pgfqpoint{3.948633in}{1.112112in}}%
\pgfpathmoveto{\pgfqpoint{3.944375in}{1.116370in}}%
\pgfpathlineto{\pgfqpoint{3.944375in}{1.116370in}}%
\pgfpathlineto{\pgfqpoint{3.944375in}{1.120628in}}%
\pgfpathlineto{\pgfqpoint{3.948633in}{1.120628in}}%
\pgfpathlineto{\pgfqpoint{3.948633in}{1.116370in}}%
\pgfpathmoveto{\pgfqpoint{3.940118in}{1.120628in}}%
\pgfpathlineto{\pgfqpoint{3.940118in}{1.120628in}}%
\pgfpathlineto{\pgfqpoint{3.940118in}{1.124886in}}%
\pgfpathlineto{\pgfqpoint{3.944375in}{1.124886in}}%
\pgfpathlineto{\pgfqpoint{3.944375in}{1.120628in}}%
\pgfpathmoveto{\pgfqpoint{3.940118in}{1.124886in}}%
\pgfpathlineto{\pgfqpoint{3.940118in}{1.124886in}}%
\pgfpathlineto{\pgfqpoint{3.940118in}{1.129144in}}%
\pgfpathlineto{\pgfqpoint{3.944375in}{1.129144in}}%
\pgfpathlineto{\pgfqpoint{3.944375in}{1.124886in}}%
\pgfpathmoveto{\pgfqpoint{3.940118in}{1.129144in}}%
\pgfpathlineto{\pgfqpoint{3.940118in}{1.129144in}}%
\pgfpathlineto{\pgfqpoint{3.940118in}{1.133402in}}%
\pgfpathlineto{\pgfqpoint{3.944375in}{1.133402in}}%
\pgfpathlineto{\pgfqpoint{3.944375in}{1.129144in}}%
\pgfpathmoveto{\pgfqpoint{3.940118in}{1.133402in}}%
\pgfpathlineto{\pgfqpoint{3.940118in}{1.133402in}}%
\pgfpathlineto{\pgfqpoint{3.940118in}{1.137660in}}%
\pgfpathlineto{\pgfqpoint{3.944375in}{1.137660in}}%
\pgfpathlineto{\pgfqpoint{3.944375in}{1.133402in}}%
\pgfpathmoveto{\pgfqpoint{3.940118in}{1.137660in}}%
\pgfpathlineto{\pgfqpoint{3.940118in}{1.137660in}}%
\pgfpathlineto{\pgfqpoint{3.940118in}{1.141917in}}%
\pgfpathlineto{\pgfqpoint{3.944375in}{1.141917in}}%
\pgfpathlineto{\pgfqpoint{3.944375in}{1.137660in}}%
\pgfpathmoveto{\pgfqpoint{3.940118in}{1.141917in}}%
\pgfpathlineto{\pgfqpoint{3.940118in}{1.141917in}}%
\pgfpathlineto{\pgfqpoint{3.940118in}{1.146175in}}%
\pgfpathlineto{\pgfqpoint{3.944375in}{1.146175in}}%
\pgfpathlineto{\pgfqpoint{3.944375in}{1.141917in}}%
\pgfpathmoveto{\pgfqpoint{3.935860in}{1.146175in}}%
\pgfpathlineto{\pgfqpoint{3.935860in}{1.146175in}}%
\pgfpathlineto{\pgfqpoint{3.935860in}{1.150433in}}%
\pgfpathlineto{\pgfqpoint{3.940118in}{1.150433in}}%
\pgfpathlineto{\pgfqpoint{3.940118in}{1.146175in}}%
\pgfpathmoveto{\pgfqpoint{3.935860in}{1.150433in}}%
\pgfpathlineto{\pgfqpoint{3.935860in}{1.150433in}}%
\pgfpathlineto{\pgfqpoint{3.935860in}{1.154691in}}%
\pgfpathlineto{\pgfqpoint{3.940118in}{1.154691in}}%
\pgfpathlineto{\pgfqpoint{3.940118in}{1.150433in}}%
\pgfpathmoveto{\pgfqpoint{3.940118in}{1.146175in}}%
\pgfpathlineto{\pgfqpoint{3.940118in}{1.146175in}}%
\pgfpathlineto{\pgfqpoint{3.940118in}{1.150433in}}%
\pgfpathlineto{\pgfqpoint{3.944375in}{1.150433in}}%
\pgfpathlineto{\pgfqpoint{3.944375in}{1.146175in}}%
\pgfpathmoveto{\pgfqpoint{3.940118in}{1.150433in}}%
\pgfpathlineto{\pgfqpoint{3.940118in}{1.150433in}}%
\pgfpathlineto{\pgfqpoint{3.940118in}{1.154691in}}%
\pgfpathlineto{\pgfqpoint{3.944375in}{1.154691in}}%
\pgfpathlineto{\pgfqpoint{3.944375in}{1.150433in}}%
\pgfpathmoveto{\pgfqpoint{3.935860in}{1.154691in}}%
\pgfpathlineto{\pgfqpoint{3.935860in}{1.154691in}}%
\pgfpathlineto{\pgfqpoint{3.935860in}{1.158949in}}%
\pgfpathlineto{\pgfqpoint{3.940118in}{1.158949in}}%
\pgfpathlineto{\pgfqpoint{3.940118in}{1.154691in}}%
\pgfpathmoveto{\pgfqpoint{3.935860in}{1.158949in}}%
\pgfpathlineto{\pgfqpoint{3.935860in}{1.158949in}}%
\pgfpathlineto{\pgfqpoint{3.935860in}{1.163207in}}%
\pgfpathlineto{\pgfqpoint{3.940118in}{1.163207in}}%
\pgfpathlineto{\pgfqpoint{3.940118in}{1.158949in}}%
\pgfpathmoveto{\pgfqpoint{3.940118in}{1.154691in}}%
\pgfpathlineto{\pgfqpoint{3.940118in}{1.154691in}}%
\pgfpathlineto{\pgfqpoint{3.940118in}{1.158949in}}%
\pgfpathlineto{\pgfqpoint{3.944375in}{1.158949in}}%
\pgfpathlineto{\pgfqpoint{3.944375in}{1.154691in}}%
\pgfpathmoveto{\pgfqpoint{3.940118in}{1.158949in}}%
\pgfpathlineto{\pgfqpoint{3.940118in}{1.158949in}}%
\pgfpathlineto{\pgfqpoint{3.940118in}{1.163207in}}%
\pgfpathlineto{\pgfqpoint{3.944375in}{1.163207in}}%
\pgfpathlineto{\pgfqpoint{3.944375in}{1.158949in}}%
\pgfpathmoveto{\pgfqpoint{3.935860in}{1.163207in}}%
\pgfpathlineto{\pgfqpoint{3.935860in}{1.163207in}}%
\pgfpathlineto{\pgfqpoint{3.935860in}{1.167465in}}%
\pgfpathlineto{\pgfqpoint{3.940118in}{1.167465in}}%
\pgfpathlineto{\pgfqpoint{3.940118in}{1.163207in}}%
\pgfpathmoveto{\pgfqpoint{3.935860in}{1.167465in}}%
\pgfpathlineto{\pgfqpoint{3.935860in}{1.167465in}}%
\pgfpathlineto{\pgfqpoint{3.935860in}{1.171723in}}%
\pgfpathlineto{\pgfqpoint{3.940118in}{1.171723in}}%
\pgfpathlineto{\pgfqpoint{3.940118in}{1.167465in}}%
\pgfpathmoveto{\pgfqpoint{3.940118in}{1.163207in}}%
\pgfpathlineto{\pgfqpoint{3.940118in}{1.163207in}}%
\pgfpathlineto{\pgfqpoint{3.940118in}{1.167465in}}%
\pgfpathlineto{\pgfqpoint{3.944375in}{1.167465in}}%
\pgfpathlineto{\pgfqpoint{3.944375in}{1.163207in}}%
\pgfpathmoveto{\pgfqpoint{3.940118in}{1.167465in}}%
\pgfpathlineto{\pgfqpoint{3.940118in}{1.167465in}}%
\pgfpathlineto{\pgfqpoint{3.940118in}{1.171723in}}%
\pgfpathlineto{\pgfqpoint{3.944375in}{1.171723in}}%
\pgfpathlineto{\pgfqpoint{3.944375in}{1.167465in}}%
\pgfpathmoveto{\pgfqpoint{3.931602in}{1.184496in}}%
\pgfpathlineto{\pgfqpoint{3.931602in}{1.184496in}}%
\pgfpathlineto{\pgfqpoint{3.931602in}{1.188754in}}%
\pgfpathlineto{\pgfqpoint{3.935860in}{1.188754in}}%
\pgfpathlineto{\pgfqpoint{3.935860in}{1.184496in}}%
\pgfpathmoveto{\pgfqpoint{3.935860in}{1.171723in}}%
\pgfpathlineto{\pgfqpoint{3.935860in}{1.171723in}}%
\pgfpathlineto{\pgfqpoint{3.935860in}{1.175981in}}%
\pgfpathlineto{\pgfqpoint{3.940118in}{1.175981in}}%
\pgfpathlineto{\pgfqpoint{3.940118in}{1.171723in}}%
\pgfpathmoveto{\pgfqpoint{3.935860in}{1.175981in}}%
\pgfpathlineto{\pgfqpoint{3.935860in}{1.175981in}}%
\pgfpathlineto{\pgfqpoint{3.935860in}{1.180238in}}%
\pgfpathlineto{\pgfqpoint{3.940118in}{1.180238in}}%
\pgfpathlineto{\pgfqpoint{3.940118in}{1.175981in}}%
\pgfpathmoveto{\pgfqpoint{3.940118in}{1.171723in}}%
\pgfpathlineto{\pgfqpoint{3.940118in}{1.171723in}}%
\pgfpathlineto{\pgfqpoint{3.940118in}{1.175981in}}%
\pgfpathlineto{\pgfqpoint{3.944375in}{1.175981in}}%
\pgfpathlineto{\pgfqpoint{3.944375in}{1.171723in}}%
\pgfpathmoveto{\pgfqpoint{3.940118in}{1.175981in}}%
\pgfpathlineto{\pgfqpoint{3.940118in}{1.175981in}}%
\pgfpathlineto{\pgfqpoint{3.940118in}{1.180238in}}%
\pgfpathlineto{\pgfqpoint{3.944375in}{1.180238in}}%
\pgfpathlineto{\pgfqpoint{3.944375in}{1.175981in}}%
\pgfpathmoveto{\pgfqpoint{3.935860in}{1.180238in}}%
\pgfpathlineto{\pgfqpoint{3.935860in}{1.180238in}}%
\pgfpathlineto{\pgfqpoint{3.935860in}{1.184496in}}%
\pgfpathlineto{\pgfqpoint{3.940118in}{1.184496in}}%
\pgfpathlineto{\pgfqpoint{3.940118in}{1.180238in}}%
\pgfpathmoveto{\pgfqpoint{3.935860in}{1.184496in}}%
\pgfpathlineto{\pgfqpoint{3.935860in}{1.184496in}}%
\pgfpathlineto{\pgfqpoint{3.935860in}{1.188754in}}%
\pgfpathlineto{\pgfqpoint{3.940118in}{1.188754in}}%
\pgfpathlineto{\pgfqpoint{3.940118in}{1.184496in}}%
\pgfpathmoveto{\pgfqpoint{3.940118in}{1.180238in}}%
\pgfpathlineto{\pgfqpoint{3.940118in}{1.180238in}}%
\pgfpathlineto{\pgfqpoint{3.940118in}{1.184496in}}%
\pgfpathlineto{\pgfqpoint{3.944375in}{1.184496in}}%
\pgfpathlineto{\pgfqpoint{3.944375in}{1.180238in}}%
\pgfpathmoveto{\pgfqpoint{3.944375in}{1.120628in}}%
\pgfpathlineto{\pgfqpoint{3.944375in}{1.120628in}}%
\pgfpathlineto{\pgfqpoint{3.944375in}{1.124886in}}%
\pgfpathlineto{\pgfqpoint{3.948633in}{1.124886in}}%
\pgfpathlineto{\pgfqpoint{3.948633in}{1.120628in}}%
\pgfpathmoveto{\pgfqpoint{3.944375in}{1.124886in}}%
\pgfpathlineto{\pgfqpoint{3.944375in}{1.124886in}}%
\pgfpathlineto{\pgfqpoint{3.944375in}{1.129144in}}%
\pgfpathlineto{\pgfqpoint{3.948633in}{1.129144in}}%
\pgfpathlineto{\pgfqpoint{3.948633in}{1.124886in}}%
\pgfpathmoveto{\pgfqpoint{3.944375in}{1.129144in}}%
\pgfpathlineto{\pgfqpoint{3.944375in}{1.129144in}}%
\pgfpathlineto{\pgfqpoint{3.944375in}{1.133402in}}%
\pgfpathlineto{\pgfqpoint{3.948633in}{1.133402in}}%
\pgfpathlineto{\pgfqpoint{3.948633in}{1.129144in}}%
\pgfpathmoveto{\pgfqpoint{3.944375in}{1.133402in}}%
\pgfpathlineto{\pgfqpoint{3.944375in}{1.133402in}}%
\pgfpathlineto{\pgfqpoint{3.944375in}{1.137660in}}%
\pgfpathlineto{\pgfqpoint{3.948633in}{1.137660in}}%
\pgfpathlineto{\pgfqpoint{3.948633in}{1.133402in}}%
\pgfpathmoveto{\pgfqpoint{3.944375in}{1.137660in}}%
\pgfpathlineto{\pgfqpoint{3.944375in}{1.137660in}}%
\pgfpathlineto{\pgfqpoint{3.944375in}{1.141917in}}%
\pgfpathlineto{\pgfqpoint{3.948633in}{1.141917in}}%
\pgfpathlineto{\pgfqpoint{3.948633in}{1.137660in}}%
\pgfpathmoveto{\pgfqpoint{3.944375in}{1.141917in}}%
\pgfpathlineto{\pgfqpoint{3.944375in}{1.141917in}}%
\pgfpathlineto{\pgfqpoint{3.944375in}{1.146175in}}%
\pgfpathlineto{\pgfqpoint{3.948633in}{1.146175in}}%
\pgfpathlineto{\pgfqpoint{3.948633in}{1.141917in}}%
\pgfpathmoveto{\pgfqpoint{3.931602in}{1.188754in}}%
\pgfpathlineto{\pgfqpoint{3.931602in}{1.188754in}}%
\pgfpathlineto{\pgfqpoint{3.931602in}{1.193012in}}%
\pgfpathlineto{\pgfqpoint{3.935860in}{1.193012in}}%
\pgfpathlineto{\pgfqpoint{3.935860in}{1.188754in}}%
\pgfpathmoveto{\pgfqpoint{3.931602in}{1.193012in}}%
\pgfpathlineto{\pgfqpoint{3.931602in}{1.193012in}}%
\pgfpathlineto{\pgfqpoint{3.931602in}{1.197269in}}%
\pgfpathlineto{\pgfqpoint{3.935860in}{1.197269in}}%
\pgfpathlineto{\pgfqpoint{3.935860in}{1.193012in}}%
\pgfpathmoveto{\pgfqpoint{3.931602in}{1.197269in}}%
\pgfpathlineto{\pgfqpoint{3.931602in}{1.197269in}}%
\pgfpathlineto{\pgfqpoint{3.931602in}{1.201527in}}%
\pgfpathlineto{\pgfqpoint{3.935860in}{1.201527in}}%
\pgfpathlineto{\pgfqpoint{3.935860in}{1.197269in}}%
\pgfpathmoveto{\pgfqpoint{3.931602in}{1.201527in}}%
\pgfpathlineto{\pgfqpoint{3.931602in}{1.201527in}}%
\pgfpathlineto{\pgfqpoint{3.931602in}{1.205784in}}%
\pgfpathlineto{\pgfqpoint{3.935860in}{1.205784in}}%
\pgfpathlineto{\pgfqpoint{3.935860in}{1.201527in}}%
\pgfpathmoveto{\pgfqpoint{3.935860in}{1.188754in}}%
\pgfpathlineto{\pgfqpoint{3.935860in}{1.188754in}}%
\pgfpathlineto{\pgfqpoint{3.935860in}{1.193012in}}%
\pgfpathlineto{\pgfqpoint{3.940118in}{1.193012in}}%
\pgfpathlineto{\pgfqpoint{3.940118in}{1.188754in}}%
\pgfpathmoveto{\pgfqpoint{3.935860in}{1.193012in}}%
\pgfpathlineto{\pgfqpoint{3.935860in}{1.193012in}}%
\pgfpathlineto{\pgfqpoint{3.935860in}{1.197269in}}%
\pgfpathlineto{\pgfqpoint{3.940118in}{1.197269in}}%
\pgfpathlineto{\pgfqpoint{3.940118in}{1.193012in}}%
\pgfpathmoveto{\pgfqpoint{3.935860in}{1.197269in}}%
\pgfpathlineto{\pgfqpoint{3.935860in}{1.197269in}}%
\pgfpathlineto{\pgfqpoint{3.935860in}{1.201527in}}%
\pgfpathlineto{\pgfqpoint{3.940118in}{1.201527in}}%
\pgfpathlineto{\pgfqpoint{3.940118in}{1.197269in}}%
\pgfpathmoveto{\pgfqpoint{3.935860in}{1.201527in}}%
\pgfpathlineto{\pgfqpoint{3.935860in}{1.201527in}}%
\pgfpathlineto{\pgfqpoint{3.935860in}{1.205784in}}%
\pgfpathlineto{\pgfqpoint{3.940118in}{1.205784in}}%
\pgfpathlineto{\pgfqpoint{3.940118in}{1.201527in}}%
\pgfpathmoveto{\pgfqpoint{3.931602in}{1.205784in}}%
\pgfpathlineto{\pgfqpoint{3.931602in}{1.205784in}}%
\pgfpathlineto{\pgfqpoint{3.931602in}{1.210042in}}%
\pgfpathlineto{\pgfqpoint{3.935860in}{1.210042in}}%
\pgfpathlineto{\pgfqpoint{3.935860in}{1.205784in}}%
\pgfpathmoveto{\pgfqpoint{3.931602in}{1.210042in}}%
\pgfpathlineto{\pgfqpoint{3.931602in}{1.210042in}}%
\pgfpathlineto{\pgfqpoint{3.931602in}{1.214300in}}%
\pgfpathlineto{\pgfqpoint{3.935860in}{1.214300in}}%
\pgfpathlineto{\pgfqpoint{3.935860in}{1.210042in}}%
\pgfpathmoveto{\pgfqpoint{3.931602in}{1.214300in}}%
\pgfpathlineto{\pgfqpoint{3.931602in}{1.214300in}}%
\pgfpathlineto{\pgfqpoint{3.931602in}{1.218557in}}%
\pgfpathlineto{\pgfqpoint{3.935860in}{1.218557in}}%
\pgfpathlineto{\pgfqpoint{3.935860in}{1.214300in}}%
\pgfpathmoveto{\pgfqpoint{3.931602in}{1.218557in}}%
\pgfpathlineto{\pgfqpoint{3.931602in}{1.218557in}}%
\pgfpathlineto{\pgfqpoint{3.931602in}{1.222815in}}%
\pgfpathlineto{\pgfqpoint{3.935860in}{1.222815in}}%
\pgfpathlineto{\pgfqpoint{3.935860in}{1.218557in}}%
\pgfpathmoveto{\pgfqpoint{3.935860in}{1.205784in}}%
\pgfpathlineto{\pgfqpoint{3.935860in}{1.205784in}}%
\pgfpathlineto{\pgfqpoint{3.935860in}{1.210042in}}%
\pgfpathlineto{\pgfqpoint{3.940118in}{1.210042in}}%
\pgfpathlineto{\pgfqpoint{3.940118in}{1.205784in}}%
\pgfpathmoveto{\pgfqpoint{3.935860in}{1.210042in}}%
\pgfpathlineto{\pgfqpoint{3.935860in}{1.210042in}}%
\pgfpathlineto{\pgfqpoint{3.935860in}{1.214300in}}%
\pgfpathlineto{\pgfqpoint{3.940118in}{1.214300in}}%
\pgfpathlineto{\pgfqpoint{3.940118in}{1.210042in}}%
\pgfpathmoveto{\pgfqpoint{3.935860in}{1.214300in}}%
\pgfpathlineto{\pgfqpoint{3.935860in}{1.214300in}}%
\pgfpathlineto{\pgfqpoint{3.935860in}{1.218557in}}%
\pgfpathlineto{\pgfqpoint{3.940118in}{1.218557in}}%
\pgfpathlineto{\pgfqpoint{3.940118in}{1.214300in}}%
\pgfpathmoveto{\pgfqpoint{3.935860in}{1.218557in}}%
\pgfpathlineto{\pgfqpoint{3.935860in}{1.218557in}}%
\pgfpathlineto{\pgfqpoint{3.935860in}{1.222815in}}%
\pgfpathlineto{\pgfqpoint{3.940118in}{1.222815in}}%
\pgfpathlineto{\pgfqpoint{3.940118in}{1.218557in}}%
\pgfpathmoveto{\pgfqpoint{3.927345in}{1.222815in}}%
\pgfpathlineto{\pgfqpoint{3.927345in}{1.222815in}}%
\pgfpathlineto{\pgfqpoint{3.927345in}{1.227072in}}%
\pgfpathlineto{\pgfqpoint{3.931602in}{1.227072in}}%
\pgfpathlineto{\pgfqpoint{3.931602in}{1.222815in}}%
\pgfpathmoveto{\pgfqpoint{3.927345in}{1.227072in}}%
\pgfpathlineto{\pgfqpoint{3.927345in}{1.227072in}}%
\pgfpathlineto{\pgfqpoint{3.927345in}{1.231330in}}%
\pgfpathlineto{\pgfqpoint{3.931602in}{1.231330in}}%
\pgfpathlineto{\pgfqpoint{3.931602in}{1.227072in}}%
\pgfpathmoveto{\pgfqpoint{3.931602in}{1.222815in}}%
\pgfpathlineto{\pgfqpoint{3.931602in}{1.222815in}}%
\pgfpathlineto{\pgfqpoint{3.931602in}{1.227072in}}%
\pgfpathlineto{\pgfqpoint{3.935860in}{1.227072in}}%
\pgfpathlineto{\pgfqpoint{3.935860in}{1.222815in}}%
\pgfpathmoveto{\pgfqpoint{3.931602in}{1.227072in}}%
\pgfpathlineto{\pgfqpoint{3.931602in}{1.227072in}}%
\pgfpathlineto{\pgfqpoint{3.931602in}{1.231330in}}%
\pgfpathlineto{\pgfqpoint{3.935860in}{1.231330in}}%
\pgfpathlineto{\pgfqpoint{3.935860in}{1.227072in}}%
\pgfpathmoveto{\pgfqpoint{3.927345in}{1.231330in}}%
\pgfpathlineto{\pgfqpoint{3.927345in}{1.231330in}}%
\pgfpathlineto{\pgfqpoint{3.927345in}{1.235587in}}%
\pgfpathlineto{\pgfqpoint{3.931602in}{1.235587in}}%
\pgfpathlineto{\pgfqpoint{3.931602in}{1.231330in}}%
\pgfpathmoveto{\pgfqpoint{3.927345in}{1.235587in}}%
\pgfpathlineto{\pgfqpoint{3.927345in}{1.235587in}}%
\pgfpathlineto{\pgfqpoint{3.927345in}{1.239845in}}%
\pgfpathlineto{\pgfqpoint{3.931602in}{1.239845in}}%
\pgfpathlineto{\pgfqpoint{3.931602in}{1.235587in}}%
\pgfpathmoveto{\pgfqpoint{3.931602in}{1.231330in}}%
\pgfpathlineto{\pgfqpoint{3.931602in}{1.231330in}}%
\pgfpathlineto{\pgfqpoint{3.931602in}{1.235587in}}%
\pgfpathlineto{\pgfqpoint{3.935860in}{1.235587in}}%
\pgfpathlineto{\pgfqpoint{3.935860in}{1.231330in}}%
\pgfpathmoveto{\pgfqpoint{3.931602in}{1.235587in}}%
\pgfpathlineto{\pgfqpoint{3.931602in}{1.235587in}}%
\pgfpathlineto{\pgfqpoint{3.931602in}{1.239845in}}%
\pgfpathlineto{\pgfqpoint{3.935860in}{1.239845in}}%
\pgfpathlineto{\pgfqpoint{3.935860in}{1.235587in}}%
\pgfpathmoveto{\pgfqpoint{3.927345in}{1.239845in}}%
\pgfpathlineto{\pgfqpoint{3.927345in}{1.239845in}}%
\pgfpathlineto{\pgfqpoint{3.927345in}{1.244103in}}%
\pgfpathlineto{\pgfqpoint{3.931602in}{1.244103in}}%
\pgfpathlineto{\pgfqpoint{3.931602in}{1.239845in}}%
\pgfpathmoveto{\pgfqpoint{3.927345in}{1.244103in}}%
\pgfpathlineto{\pgfqpoint{3.927345in}{1.244103in}}%
\pgfpathlineto{\pgfqpoint{3.927345in}{1.248360in}}%
\pgfpathlineto{\pgfqpoint{3.931602in}{1.248360in}}%
\pgfpathlineto{\pgfqpoint{3.931602in}{1.244103in}}%
\pgfpathmoveto{\pgfqpoint{3.931602in}{1.239845in}}%
\pgfpathlineto{\pgfqpoint{3.931602in}{1.239845in}}%
\pgfpathlineto{\pgfqpoint{3.931602in}{1.244103in}}%
\pgfpathlineto{\pgfqpoint{3.935860in}{1.244103in}}%
\pgfpathlineto{\pgfqpoint{3.935860in}{1.239845in}}%
\pgfpathmoveto{\pgfqpoint{3.931602in}{1.244103in}}%
\pgfpathlineto{\pgfqpoint{3.931602in}{1.244103in}}%
\pgfpathlineto{\pgfqpoint{3.931602in}{1.248360in}}%
\pgfpathlineto{\pgfqpoint{3.935860in}{1.248360in}}%
\pgfpathlineto{\pgfqpoint{3.935860in}{1.244103in}}%
\pgfpathmoveto{\pgfqpoint{3.927345in}{1.248360in}}%
\pgfpathlineto{\pgfqpoint{3.927345in}{1.248360in}}%
\pgfpathlineto{\pgfqpoint{3.927345in}{1.252618in}}%
\pgfpathlineto{\pgfqpoint{3.931602in}{1.252618in}}%
\pgfpathlineto{\pgfqpoint{3.931602in}{1.248360in}}%
\pgfpathmoveto{\pgfqpoint{3.927345in}{1.252618in}}%
\pgfpathlineto{\pgfqpoint{3.927345in}{1.252618in}}%
\pgfpathlineto{\pgfqpoint{3.927345in}{1.256875in}}%
\pgfpathlineto{\pgfqpoint{3.931602in}{1.256875in}}%
\pgfpathlineto{\pgfqpoint{3.931602in}{1.252618in}}%
\pgfpathmoveto{\pgfqpoint{3.931602in}{1.248360in}}%
\pgfpathlineto{\pgfqpoint{3.931602in}{1.248360in}}%
\pgfpathlineto{\pgfqpoint{3.931602in}{1.252618in}}%
\pgfpathlineto{\pgfqpoint{3.935860in}{1.252618in}}%
\pgfpathlineto{\pgfqpoint{3.935860in}{1.248360in}}%
\pgfpathmoveto{\pgfqpoint{3.931602in}{1.252618in}}%
\pgfpathlineto{\pgfqpoint{3.931602in}{1.252618in}}%
\pgfpathlineto{\pgfqpoint{3.931602in}{1.256875in}}%
\pgfpathlineto{\pgfqpoint{3.935860in}{1.256875in}}%
\pgfpathlineto{\pgfqpoint{3.935860in}{1.252618in}}%
\pgfpathmoveto{\pgfqpoint{3.923087in}{1.261133in}}%
\pgfpathlineto{\pgfqpoint{3.923087in}{1.261133in}}%
\pgfpathlineto{\pgfqpoint{3.923087in}{1.265390in}}%
\pgfpathlineto{\pgfqpoint{3.927345in}{1.265390in}}%
\pgfpathlineto{\pgfqpoint{3.927345in}{1.261133in}}%
\pgfpathmoveto{\pgfqpoint{3.923087in}{1.265390in}}%
\pgfpathlineto{\pgfqpoint{3.923087in}{1.265390in}}%
\pgfpathlineto{\pgfqpoint{3.923087in}{1.269648in}}%
\pgfpathlineto{\pgfqpoint{3.927345in}{1.269648in}}%
\pgfpathlineto{\pgfqpoint{3.927345in}{1.265390in}}%
\pgfpathmoveto{\pgfqpoint{3.923087in}{1.269648in}}%
\pgfpathlineto{\pgfqpoint{3.923087in}{1.269648in}}%
\pgfpathlineto{\pgfqpoint{3.923087in}{1.273906in}}%
\pgfpathlineto{\pgfqpoint{3.927345in}{1.273906in}}%
\pgfpathlineto{\pgfqpoint{3.927345in}{1.269648in}}%
\pgfpathmoveto{\pgfqpoint{3.923087in}{1.273906in}}%
\pgfpathlineto{\pgfqpoint{3.923087in}{1.273906in}}%
\pgfpathlineto{\pgfqpoint{3.923087in}{1.278163in}}%
\pgfpathlineto{\pgfqpoint{3.927345in}{1.278163in}}%
\pgfpathlineto{\pgfqpoint{3.927345in}{1.273906in}}%
\pgfpathmoveto{\pgfqpoint{3.923087in}{1.278163in}}%
\pgfpathlineto{\pgfqpoint{3.923087in}{1.278163in}}%
\pgfpathlineto{\pgfqpoint{3.923087in}{1.282421in}}%
\pgfpathlineto{\pgfqpoint{3.927345in}{1.282421in}}%
\pgfpathlineto{\pgfqpoint{3.927345in}{1.278163in}}%
\pgfpathmoveto{\pgfqpoint{3.923087in}{1.282421in}}%
\pgfpathlineto{\pgfqpoint{3.923087in}{1.282421in}}%
\pgfpathlineto{\pgfqpoint{3.923087in}{1.286678in}}%
\pgfpathlineto{\pgfqpoint{3.927345in}{1.286678in}}%
\pgfpathlineto{\pgfqpoint{3.927345in}{1.282421in}}%
\pgfpathmoveto{\pgfqpoint{3.923087in}{1.286678in}}%
\pgfpathlineto{\pgfqpoint{3.923087in}{1.286678in}}%
\pgfpathlineto{\pgfqpoint{3.923087in}{1.290936in}}%
\pgfpathlineto{\pgfqpoint{3.927345in}{1.290936in}}%
\pgfpathlineto{\pgfqpoint{3.927345in}{1.286678in}}%
\pgfpathmoveto{\pgfqpoint{3.927345in}{1.256875in}}%
\pgfpathlineto{\pgfqpoint{3.927345in}{1.256875in}}%
\pgfpathlineto{\pgfqpoint{3.927345in}{1.261133in}}%
\pgfpathlineto{\pgfqpoint{3.931602in}{1.261133in}}%
\pgfpathlineto{\pgfqpoint{3.931602in}{1.256875in}}%
\pgfpathmoveto{\pgfqpoint{3.927345in}{1.261133in}}%
\pgfpathlineto{\pgfqpoint{3.927345in}{1.261133in}}%
\pgfpathlineto{\pgfqpoint{3.927345in}{1.265390in}}%
\pgfpathlineto{\pgfqpoint{3.931602in}{1.265390in}}%
\pgfpathlineto{\pgfqpoint{3.931602in}{1.261133in}}%
\pgfpathmoveto{\pgfqpoint{3.931602in}{1.256875in}}%
\pgfpathlineto{\pgfqpoint{3.931602in}{1.256875in}}%
\pgfpathlineto{\pgfqpoint{3.931602in}{1.261133in}}%
\pgfpathlineto{\pgfqpoint{3.935860in}{1.261133in}}%
\pgfpathlineto{\pgfqpoint{3.935860in}{1.256875in}}%
\pgfpathmoveto{\pgfqpoint{3.927345in}{1.265390in}}%
\pgfpathlineto{\pgfqpoint{3.927345in}{1.265390in}}%
\pgfpathlineto{\pgfqpoint{3.927345in}{1.269648in}}%
\pgfpathlineto{\pgfqpoint{3.931602in}{1.269648in}}%
\pgfpathlineto{\pgfqpoint{3.931602in}{1.265390in}}%
\pgfpathmoveto{\pgfqpoint{3.927345in}{1.269648in}}%
\pgfpathlineto{\pgfqpoint{3.927345in}{1.269648in}}%
\pgfpathlineto{\pgfqpoint{3.927345in}{1.273906in}}%
\pgfpathlineto{\pgfqpoint{3.931602in}{1.273906in}}%
\pgfpathlineto{\pgfqpoint{3.931602in}{1.269648in}}%
\pgfpathmoveto{\pgfqpoint{3.927345in}{1.273906in}}%
\pgfpathlineto{\pgfqpoint{3.927345in}{1.273906in}}%
\pgfpathlineto{\pgfqpoint{3.927345in}{1.278163in}}%
\pgfpathlineto{\pgfqpoint{3.931602in}{1.278163in}}%
\pgfpathlineto{\pgfqpoint{3.931602in}{1.273906in}}%
\pgfpathmoveto{\pgfqpoint{3.927345in}{1.278163in}}%
\pgfpathlineto{\pgfqpoint{3.927345in}{1.278163in}}%
\pgfpathlineto{\pgfqpoint{3.927345in}{1.282421in}}%
\pgfpathlineto{\pgfqpoint{3.931602in}{1.282421in}}%
\pgfpathlineto{\pgfqpoint{3.931602in}{1.278163in}}%
\pgfpathmoveto{\pgfqpoint{3.927345in}{1.282421in}}%
\pgfpathlineto{\pgfqpoint{3.927345in}{1.282421in}}%
\pgfpathlineto{\pgfqpoint{3.927345in}{1.286678in}}%
\pgfpathlineto{\pgfqpoint{3.931602in}{1.286678in}}%
\pgfpathlineto{\pgfqpoint{3.931602in}{1.282421in}}%
\pgfpathmoveto{\pgfqpoint{3.927345in}{1.286678in}}%
\pgfpathlineto{\pgfqpoint{3.927345in}{1.286678in}}%
\pgfpathlineto{\pgfqpoint{3.927345in}{1.290936in}}%
\pgfpathlineto{\pgfqpoint{3.931602in}{1.290936in}}%
\pgfpathlineto{\pgfqpoint{3.931602in}{1.286678in}}%
\pgfpathmoveto{\pgfqpoint{3.923087in}{1.290936in}}%
\pgfpathlineto{\pgfqpoint{3.923087in}{1.290936in}}%
\pgfpathlineto{\pgfqpoint{3.923087in}{1.295193in}}%
\pgfpathlineto{\pgfqpoint{3.927345in}{1.295193in}}%
\pgfpathlineto{\pgfqpoint{3.927345in}{1.290936in}}%
\pgfpathmoveto{\pgfqpoint{3.923087in}{1.295193in}}%
\pgfpathlineto{\pgfqpoint{3.923087in}{1.295193in}}%
\pgfpathlineto{\pgfqpoint{3.923087in}{1.299451in}}%
\pgfpathlineto{\pgfqpoint{3.927345in}{1.299451in}}%
\pgfpathlineto{\pgfqpoint{3.927345in}{1.295193in}}%
\pgfpathmoveto{\pgfqpoint{3.918829in}{1.299451in}}%
\pgfpathlineto{\pgfqpoint{3.918829in}{1.299451in}}%
\pgfpathlineto{\pgfqpoint{3.918829in}{1.303709in}}%
\pgfpathlineto{\pgfqpoint{3.923087in}{1.303709in}}%
\pgfpathlineto{\pgfqpoint{3.923087in}{1.299451in}}%
\pgfpathmoveto{\pgfqpoint{3.918829in}{1.303709in}}%
\pgfpathlineto{\pgfqpoint{3.918829in}{1.303709in}}%
\pgfpathlineto{\pgfqpoint{3.918829in}{1.307966in}}%
\pgfpathlineto{\pgfqpoint{3.923087in}{1.307966in}}%
\pgfpathlineto{\pgfqpoint{3.923087in}{1.303709in}}%
\pgfpathmoveto{\pgfqpoint{3.923087in}{1.299451in}}%
\pgfpathlineto{\pgfqpoint{3.923087in}{1.299451in}}%
\pgfpathlineto{\pgfqpoint{3.923087in}{1.303709in}}%
\pgfpathlineto{\pgfqpoint{3.927345in}{1.303709in}}%
\pgfpathlineto{\pgfqpoint{3.927345in}{1.299451in}}%
\pgfpathmoveto{\pgfqpoint{3.923087in}{1.303709in}}%
\pgfpathlineto{\pgfqpoint{3.923087in}{1.303709in}}%
\pgfpathlineto{\pgfqpoint{3.923087in}{1.307966in}}%
\pgfpathlineto{\pgfqpoint{3.927345in}{1.307966in}}%
\pgfpathlineto{\pgfqpoint{3.927345in}{1.303709in}}%
\pgfpathmoveto{\pgfqpoint{3.918829in}{1.307966in}}%
\pgfpathlineto{\pgfqpoint{3.918829in}{1.307966in}}%
\pgfpathlineto{\pgfqpoint{3.918829in}{1.312224in}}%
\pgfpathlineto{\pgfqpoint{3.923087in}{1.312224in}}%
\pgfpathlineto{\pgfqpoint{3.923087in}{1.307966in}}%
\pgfpathmoveto{\pgfqpoint{3.918829in}{1.312224in}}%
\pgfpathlineto{\pgfqpoint{3.918829in}{1.312224in}}%
\pgfpathlineto{\pgfqpoint{3.918829in}{1.316481in}}%
\pgfpathlineto{\pgfqpoint{3.923087in}{1.316481in}}%
\pgfpathlineto{\pgfqpoint{3.923087in}{1.312224in}}%
\pgfpathmoveto{\pgfqpoint{3.923087in}{1.307966in}}%
\pgfpathlineto{\pgfqpoint{3.923087in}{1.307966in}}%
\pgfpathlineto{\pgfqpoint{3.923087in}{1.312224in}}%
\pgfpathlineto{\pgfqpoint{3.927345in}{1.312224in}}%
\pgfpathlineto{\pgfqpoint{3.927345in}{1.307966in}}%
\pgfpathmoveto{\pgfqpoint{3.923087in}{1.312224in}}%
\pgfpathlineto{\pgfqpoint{3.923087in}{1.312224in}}%
\pgfpathlineto{\pgfqpoint{3.923087in}{1.316481in}}%
\pgfpathlineto{\pgfqpoint{3.927345in}{1.316481in}}%
\pgfpathlineto{\pgfqpoint{3.927345in}{1.312224in}}%
\pgfpathmoveto{\pgfqpoint{3.918829in}{1.316481in}}%
\pgfpathlineto{\pgfqpoint{3.918829in}{1.316481in}}%
\pgfpathlineto{\pgfqpoint{3.918829in}{1.320739in}}%
\pgfpathlineto{\pgfqpoint{3.923087in}{1.320739in}}%
\pgfpathlineto{\pgfqpoint{3.923087in}{1.316481in}}%
\pgfpathmoveto{\pgfqpoint{3.918829in}{1.320739in}}%
\pgfpathlineto{\pgfqpoint{3.918829in}{1.320739in}}%
\pgfpathlineto{\pgfqpoint{3.918829in}{1.324996in}}%
\pgfpathlineto{\pgfqpoint{3.923087in}{1.324996in}}%
\pgfpathlineto{\pgfqpoint{3.923087in}{1.320739in}}%
\pgfpathmoveto{\pgfqpoint{3.923087in}{1.316481in}}%
\pgfpathlineto{\pgfqpoint{3.923087in}{1.316481in}}%
\pgfpathlineto{\pgfqpoint{3.923087in}{1.320739in}}%
\pgfpathlineto{\pgfqpoint{3.927345in}{1.320739in}}%
\pgfpathlineto{\pgfqpoint{3.927345in}{1.316481in}}%
\pgfpathmoveto{\pgfqpoint{3.923087in}{1.320739in}}%
\pgfpathlineto{\pgfqpoint{3.923087in}{1.320739in}}%
\pgfpathlineto{\pgfqpoint{3.923087in}{1.324996in}}%
\pgfpathlineto{\pgfqpoint{3.927345in}{1.324996in}}%
\pgfpathlineto{\pgfqpoint{3.927345in}{1.320739in}}%
\pgfpathmoveto{\pgfqpoint{3.927345in}{1.290936in}}%
\pgfpathlineto{\pgfqpoint{3.927345in}{1.290936in}}%
\pgfpathlineto{\pgfqpoint{3.927345in}{1.295193in}}%
\pgfpathlineto{\pgfqpoint{3.931602in}{1.295193in}}%
\pgfpathlineto{\pgfqpoint{3.931602in}{1.290936in}}%
\pgfpathmoveto{\pgfqpoint{3.927345in}{1.295193in}}%
\pgfpathlineto{\pgfqpoint{3.927345in}{1.295193in}}%
\pgfpathlineto{\pgfqpoint{3.927345in}{1.299451in}}%
\pgfpathlineto{\pgfqpoint{3.931602in}{1.299451in}}%
\pgfpathlineto{\pgfqpoint{3.931602in}{1.295193in}}%
\pgfpathmoveto{\pgfqpoint{3.914572in}{1.333512in}}%
\pgfpathlineto{\pgfqpoint{3.914572in}{1.333512in}}%
\pgfpathlineto{\pgfqpoint{3.914572in}{1.337770in}}%
\pgfpathlineto{\pgfqpoint{3.918829in}{1.337770in}}%
\pgfpathlineto{\pgfqpoint{3.918829in}{1.333512in}}%
\pgfpathmoveto{\pgfqpoint{3.914572in}{1.337770in}}%
\pgfpathlineto{\pgfqpoint{3.914572in}{1.337770in}}%
\pgfpathlineto{\pgfqpoint{3.914572in}{1.342028in}}%
\pgfpathlineto{\pgfqpoint{3.918829in}{1.342028in}}%
\pgfpathlineto{\pgfqpoint{3.918829in}{1.337770in}}%
\pgfpathmoveto{\pgfqpoint{3.918829in}{1.324996in}}%
\pgfpathlineto{\pgfqpoint{3.918829in}{1.324996in}}%
\pgfpathlineto{\pgfqpoint{3.918829in}{1.329254in}}%
\pgfpathlineto{\pgfqpoint{3.923087in}{1.329254in}}%
\pgfpathlineto{\pgfqpoint{3.923087in}{1.324996in}}%
\pgfpathmoveto{\pgfqpoint{3.918829in}{1.329254in}}%
\pgfpathlineto{\pgfqpoint{3.918829in}{1.329254in}}%
\pgfpathlineto{\pgfqpoint{3.918829in}{1.333512in}}%
\pgfpathlineto{\pgfqpoint{3.923087in}{1.333512in}}%
\pgfpathlineto{\pgfqpoint{3.923087in}{1.329254in}}%
\pgfpathmoveto{\pgfqpoint{3.923087in}{1.324996in}}%
\pgfpathlineto{\pgfqpoint{3.923087in}{1.324996in}}%
\pgfpathlineto{\pgfqpoint{3.923087in}{1.329254in}}%
\pgfpathlineto{\pgfqpoint{3.927345in}{1.329254in}}%
\pgfpathlineto{\pgfqpoint{3.927345in}{1.324996in}}%
\pgfpathmoveto{\pgfqpoint{3.923087in}{1.329254in}}%
\pgfpathlineto{\pgfqpoint{3.923087in}{1.329254in}}%
\pgfpathlineto{\pgfqpoint{3.923087in}{1.333512in}}%
\pgfpathlineto{\pgfqpoint{3.927345in}{1.333512in}}%
\pgfpathlineto{\pgfqpoint{3.927345in}{1.329254in}}%
\pgfpathmoveto{\pgfqpoint{3.918829in}{1.333512in}}%
\pgfpathlineto{\pgfqpoint{3.918829in}{1.333512in}}%
\pgfpathlineto{\pgfqpoint{3.918829in}{1.337770in}}%
\pgfpathlineto{\pgfqpoint{3.923087in}{1.337770in}}%
\pgfpathlineto{\pgfqpoint{3.923087in}{1.333512in}}%
\pgfpathmoveto{\pgfqpoint{3.918829in}{1.337770in}}%
\pgfpathlineto{\pgfqpoint{3.918829in}{1.337770in}}%
\pgfpathlineto{\pgfqpoint{3.918829in}{1.342028in}}%
\pgfpathlineto{\pgfqpoint{3.923087in}{1.342028in}}%
\pgfpathlineto{\pgfqpoint{3.923087in}{1.337770in}}%
\pgfpathmoveto{\pgfqpoint{3.923087in}{1.333512in}}%
\pgfpathlineto{\pgfqpoint{3.923087in}{1.333512in}}%
\pgfpathlineto{\pgfqpoint{3.923087in}{1.337770in}}%
\pgfpathlineto{\pgfqpoint{3.927345in}{1.337770in}}%
\pgfpathlineto{\pgfqpoint{3.927345in}{1.333512in}}%
\pgfpathmoveto{\pgfqpoint{3.914572in}{1.342028in}}%
\pgfpathlineto{\pgfqpoint{3.914572in}{1.342028in}}%
\pgfpathlineto{\pgfqpoint{3.914572in}{1.346286in}}%
\pgfpathlineto{\pgfqpoint{3.918829in}{1.346286in}}%
\pgfpathlineto{\pgfqpoint{3.918829in}{1.342028in}}%
\pgfpathmoveto{\pgfqpoint{3.914572in}{1.346286in}}%
\pgfpathlineto{\pgfqpoint{3.914572in}{1.346286in}}%
\pgfpathlineto{\pgfqpoint{3.914572in}{1.350544in}}%
\pgfpathlineto{\pgfqpoint{3.918829in}{1.350544in}}%
\pgfpathlineto{\pgfqpoint{3.918829in}{1.346286in}}%
\pgfpathmoveto{\pgfqpoint{3.914572in}{1.350544in}}%
\pgfpathlineto{\pgfqpoint{3.914572in}{1.350544in}}%
\pgfpathlineto{\pgfqpoint{3.914572in}{1.354802in}}%
\pgfpathlineto{\pgfqpoint{3.918829in}{1.354802in}}%
\pgfpathlineto{\pgfqpoint{3.918829in}{1.350544in}}%
\pgfpathmoveto{\pgfqpoint{3.914572in}{1.354802in}}%
\pgfpathlineto{\pgfqpoint{3.914572in}{1.354802in}}%
\pgfpathlineto{\pgfqpoint{3.914572in}{1.359059in}}%
\pgfpathlineto{\pgfqpoint{3.918829in}{1.359059in}}%
\pgfpathlineto{\pgfqpoint{3.918829in}{1.354802in}}%
\pgfpathmoveto{\pgfqpoint{3.918829in}{1.342028in}}%
\pgfpathlineto{\pgfqpoint{3.918829in}{1.342028in}}%
\pgfpathlineto{\pgfqpoint{3.918829in}{1.346286in}}%
\pgfpathlineto{\pgfqpoint{3.923087in}{1.346286in}}%
\pgfpathlineto{\pgfqpoint{3.923087in}{1.342028in}}%
\pgfpathmoveto{\pgfqpoint{3.918829in}{1.346286in}}%
\pgfpathlineto{\pgfqpoint{3.918829in}{1.346286in}}%
\pgfpathlineto{\pgfqpoint{3.918829in}{1.350544in}}%
\pgfpathlineto{\pgfqpoint{3.923087in}{1.350544in}}%
\pgfpathlineto{\pgfqpoint{3.923087in}{1.346286in}}%
\pgfpathmoveto{\pgfqpoint{3.918829in}{1.350544in}}%
\pgfpathlineto{\pgfqpoint{3.918829in}{1.350544in}}%
\pgfpathlineto{\pgfqpoint{3.918829in}{1.354802in}}%
\pgfpathlineto{\pgfqpoint{3.923087in}{1.354802in}}%
\pgfpathlineto{\pgfqpoint{3.923087in}{1.350544in}}%
\pgfpathmoveto{\pgfqpoint{3.918829in}{1.354802in}}%
\pgfpathlineto{\pgfqpoint{3.918829in}{1.354802in}}%
\pgfpathlineto{\pgfqpoint{3.918829in}{1.359059in}}%
\pgfpathlineto{\pgfqpoint{3.923087in}{1.359059in}}%
\pgfpathlineto{\pgfqpoint{3.923087in}{1.354802in}}%
\pgfpathmoveto{\pgfqpoint{3.914572in}{1.359059in}}%
\pgfpathlineto{\pgfqpoint{3.914572in}{1.359059in}}%
\pgfpathlineto{\pgfqpoint{3.914572in}{1.363317in}}%
\pgfpathlineto{\pgfqpoint{3.918829in}{1.363317in}}%
\pgfpathlineto{\pgfqpoint{3.918829in}{1.359059in}}%
\pgfpathmoveto{\pgfqpoint{3.914572in}{1.363317in}}%
\pgfpathlineto{\pgfqpoint{3.914572in}{1.363317in}}%
\pgfpathlineto{\pgfqpoint{3.914572in}{1.367575in}}%
\pgfpathlineto{\pgfqpoint{3.918829in}{1.367575in}}%
\pgfpathlineto{\pgfqpoint{3.918829in}{1.363317in}}%
\pgfpathmoveto{\pgfqpoint{3.910314in}{1.371833in}}%
\pgfpathlineto{\pgfqpoint{3.910314in}{1.371833in}}%
\pgfpathlineto{\pgfqpoint{3.910314in}{1.376091in}}%
\pgfpathlineto{\pgfqpoint{3.914572in}{1.376091in}}%
\pgfpathlineto{\pgfqpoint{3.914572in}{1.371833in}}%
\pgfpathmoveto{\pgfqpoint{3.914572in}{1.367575in}}%
\pgfpathlineto{\pgfqpoint{3.914572in}{1.367575in}}%
\pgfpathlineto{\pgfqpoint{3.914572in}{1.371833in}}%
\pgfpathlineto{\pgfqpoint{3.918829in}{1.371833in}}%
\pgfpathlineto{\pgfqpoint{3.918829in}{1.367575in}}%
\pgfpathmoveto{\pgfqpoint{3.914572in}{1.371833in}}%
\pgfpathlineto{\pgfqpoint{3.914572in}{1.371833in}}%
\pgfpathlineto{\pgfqpoint{3.914572in}{1.376091in}}%
\pgfpathlineto{\pgfqpoint{3.918829in}{1.376091in}}%
\pgfpathlineto{\pgfqpoint{3.918829in}{1.371833in}}%
\pgfpathmoveto{\pgfqpoint{3.918829in}{1.359059in}}%
\pgfpathlineto{\pgfqpoint{3.918829in}{1.359059in}}%
\pgfpathlineto{\pgfqpoint{3.918829in}{1.363317in}}%
\pgfpathlineto{\pgfqpoint{3.923087in}{1.363317in}}%
\pgfpathlineto{\pgfqpoint{3.923087in}{1.359059in}}%
\pgfpathmoveto{\pgfqpoint{3.918829in}{1.363317in}}%
\pgfpathlineto{\pgfqpoint{3.918829in}{1.363317in}}%
\pgfpathlineto{\pgfqpoint{3.918829in}{1.367575in}}%
\pgfpathlineto{\pgfqpoint{3.923087in}{1.367575in}}%
\pgfpathlineto{\pgfqpoint{3.923087in}{1.363317in}}%
\pgfpathmoveto{\pgfqpoint{3.918829in}{1.367575in}}%
\pgfpathlineto{\pgfqpoint{3.918829in}{1.367575in}}%
\pgfpathlineto{\pgfqpoint{3.918829in}{1.371833in}}%
\pgfpathlineto{\pgfqpoint{3.923087in}{1.371833in}}%
\pgfpathlineto{\pgfqpoint{3.923087in}{1.367575in}}%
\pgfpathmoveto{\pgfqpoint{3.910314in}{1.376091in}}%
\pgfpathlineto{\pgfqpoint{3.910314in}{1.376091in}}%
\pgfpathlineto{\pgfqpoint{3.910314in}{1.380349in}}%
\pgfpathlineto{\pgfqpoint{3.914572in}{1.380349in}}%
\pgfpathlineto{\pgfqpoint{3.914572in}{1.376091in}}%
\pgfpathmoveto{\pgfqpoint{3.910314in}{1.380349in}}%
\pgfpathlineto{\pgfqpoint{3.910314in}{1.380349in}}%
\pgfpathlineto{\pgfqpoint{3.910314in}{1.384607in}}%
\pgfpathlineto{\pgfqpoint{3.914572in}{1.384607in}}%
\pgfpathlineto{\pgfqpoint{3.914572in}{1.380349in}}%
\pgfpathmoveto{\pgfqpoint{3.914572in}{1.376091in}}%
\pgfpathlineto{\pgfqpoint{3.914572in}{1.376091in}}%
\pgfpathlineto{\pgfqpoint{3.914572in}{1.380349in}}%
\pgfpathlineto{\pgfqpoint{3.918829in}{1.380349in}}%
\pgfpathlineto{\pgfqpoint{3.918829in}{1.376091in}}%
\pgfpathmoveto{\pgfqpoint{3.914572in}{1.380349in}}%
\pgfpathlineto{\pgfqpoint{3.914572in}{1.380349in}}%
\pgfpathlineto{\pgfqpoint{3.914572in}{1.384607in}}%
\pgfpathlineto{\pgfqpoint{3.918829in}{1.384607in}}%
\pgfpathlineto{\pgfqpoint{3.918829in}{1.380349in}}%
\pgfpathmoveto{\pgfqpoint{3.910314in}{1.384607in}}%
\pgfpathlineto{\pgfqpoint{3.910314in}{1.384607in}}%
\pgfpathlineto{\pgfqpoint{3.910314in}{1.388865in}}%
\pgfpathlineto{\pgfqpoint{3.914572in}{1.388865in}}%
\pgfpathlineto{\pgfqpoint{3.914572in}{1.384607in}}%
\pgfpathmoveto{\pgfqpoint{3.910314in}{1.388865in}}%
\pgfpathlineto{\pgfqpoint{3.910314in}{1.388865in}}%
\pgfpathlineto{\pgfqpoint{3.910314in}{1.393123in}}%
\pgfpathlineto{\pgfqpoint{3.914572in}{1.393123in}}%
\pgfpathlineto{\pgfqpoint{3.914572in}{1.388865in}}%
\pgfpathmoveto{\pgfqpoint{3.914572in}{1.384607in}}%
\pgfpathlineto{\pgfqpoint{3.914572in}{1.384607in}}%
\pgfpathlineto{\pgfqpoint{3.914572in}{1.388865in}}%
\pgfpathlineto{\pgfqpoint{3.918829in}{1.388865in}}%
\pgfpathlineto{\pgfqpoint{3.918829in}{1.384607in}}%
\pgfpathmoveto{\pgfqpoint{3.914572in}{1.388865in}}%
\pgfpathlineto{\pgfqpoint{3.914572in}{1.388865in}}%
\pgfpathlineto{\pgfqpoint{3.914572in}{1.393123in}}%
\pgfpathlineto{\pgfqpoint{3.918829in}{1.393123in}}%
\pgfpathlineto{\pgfqpoint{3.918829in}{1.388865in}}%
\pgfpathmoveto{\pgfqpoint{3.910314in}{1.393123in}}%
\pgfpathlineto{\pgfqpoint{3.910314in}{1.393123in}}%
\pgfpathlineto{\pgfqpoint{3.910314in}{1.397380in}}%
\pgfpathlineto{\pgfqpoint{3.914572in}{1.397380in}}%
\pgfpathlineto{\pgfqpoint{3.914572in}{1.393123in}}%
\pgfpathmoveto{\pgfqpoint{3.910314in}{1.397380in}}%
\pgfpathlineto{\pgfqpoint{3.910314in}{1.397380in}}%
\pgfpathlineto{\pgfqpoint{3.910314in}{1.401638in}}%
\pgfpathlineto{\pgfqpoint{3.914572in}{1.401638in}}%
\pgfpathlineto{\pgfqpoint{3.914572in}{1.397380in}}%
\pgfpathmoveto{\pgfqpoint{3.914572in}{1.393123in}}%
\pgfpathlineto{\pgfqpoint{3.914572in}{1.393123in}}%
\pgfpathlineto{\pgfqpoint{3.914572in}{1.397380in}}%
\pgfpathlineto{\pgfqpoint{3.918829in}{1.397380in}}%
\pgfpathlineto{\pgfqpoint{3.918829in}{1.393123in}}%
\pgfpathmoveto{\pgfqpoint{3.914572in}{1.397380in}}%
\pgfpathlineto{\pgfqpoint{3.914572in}{1.397380in}}%
\pgfpathlineto{\pgfqpoint{3.914572in}{1.401638in}}%
\pgfpathlineto{\pgfqpoint{3.918829in}{1.401638in}}%
\pgfpathlineto{\pgfqpoint{3.918829in}{1.397380in}}%
\pgfpathmoveto{\pgfqpoint{3.910314in}{1.401638in}}%
\pgfpathlineto{\pgfqpoint{3.910314in}{1.401638in}}%
\pgfpathlineto{\pgfqpoint{3.910314in}{1.405896in}}%
\pgfpathlineto{\pgfqpoint{3.914572in}{1.405896in}}%
\pgfpathlineto{\pgfqpoint{3.914572in}{1.401638in}}%
\pgfpathmoveto{\pgfqpoint{3.910314in}{1.405896in}}%
\pgfpathlineto{\pgfqpoint{3.910314in}{1.405896in}}%
\pgfpathlineto{\pgfqpoint{3.910314in}{1.410154in}}%
\pgfpathlineto{\pgfqpoint{3.914572in}{1.410154in}}%
\pgfpathlineto{\pgfqpoint{3.914572in}{1.405896in}}%
\pgfpathmoveto{\pgfqpoint{3.914572in}{1.401638in}}%
\pgfpathlineto{\pgfqpoint{3.914572in}{1.401638in}}%
\pgfpathlineto{\pgfqpoint{3.914572in}{1.405896in}}%
\pgfpathlineto{\pgfqpoint{3.918829in}{1.405896in}}%
\pgfpathlineto{\pgfqpoint{3.918829in}{1.401638in}}%
\pgfpathmoveto{\pgfqpoint{3.914572in}{1.405896in}}%
\pgfpathlineto{\pgfqpoint{3.914572in}{1.405896in}}%
\pgfpathlineto{\pgfqpoint{3.914572in}{1.410154in}}%
\pgfpathlineto{\pgfqpoint{3.918829in}{1.410154in}}%
\pgfpathlineto{\pgfqpoint{3.918829in}{1.405896in}}%
\pgfpathmoveto{\pgfqpoint{3.910314in}{1.410154in}}%
\pgfpathlineto{\pgfqpoint{3.910314in}{1.410154in}}%
\pgfpathlineto{\pgfqpoint{3.910314in}{1.414412in}}%
\pgfpathlineto{\pgfqpoint{3.914572in}{1.414412in}}%
\pgfpathlineto{\pgfqpoint{3.914572in}{1.410154in}}%
\pgfpathmoveto{\pgfqpoint{3.910314in}{1.414412in}}%
\pgfpathlineto{\pgfqpoint{3.910314in}{1.414412in}}%
\pgfpathlineto{\pgfqpoint{3.910314in}{1.418670in}}%
\pgfpathlineto{\pgfqpoint{3.914572in}{1.418670in}}%
\pgfpathlineto{\pgfqpoint{3.914572in}{1.414412in}}%
\pgfpathmoveto{\pgfqpoint{3.910314in}{1.418670in}}%
\pgfpathlineto{\pgfqpoint{3.910314in}{1.418670in}}%
\pgfpathlineto{\pgfqpoint{3.910314in}{1.422928in}}%
\pgfpathlineto{\pgfqpoint{3.914572in}{1.422928in}}%
\pgfpathlineto{\pgfqpoint{3.914572in}{1.418670in}}%
\pgfpathmoveto{\pgfqpoint{3.910314in}{1.422928in}}%
\pgfpathlineto{\pgfqpoint{3.910314in}{1.422928in}}%
\pgfpathlineto{\pgfqpoint{3.910314in}{1.427186in}}%
\pgfpathlineto{\pgfqpoint{3.914572in}{1.427186in}}%
\pgfpathlineto{\pgfqpoint{3.914572in}{1.422928in}}%
\pgfpathmoveto{\pgfqpoint{3.910314in}{1.427186in}}%
\pgfpathlineto{\pgfqpoint{3.910314in}{1.427186in}}%
\pgfpathlineto{\pgfqpoint{3.910314in}{1.431443in}}%
\pgfpathlineto{\pgfqpoint{3.914572in}{1.431443in}}%
\pgfpathlineto{\pgfqpoint{3.914572in}{1.427186in}}%
\pgfpathmoveto{\pgfqpoint{3.910314in}{1.431443in}}%
\pgfpathlineto{\pgfqpoint{3.910314in}{1.431443in}}%
\pgfpathlineto{\pgfqpoint{3.910314in}{1.435701in}}%
\pgfpathlineto{\pgfqpoint{3.914572in}{1.435701in}}%
\pgfpathlineto{\pgfqpoint{3.914572in}{1.431443in}}%
\pgfpathmoveto{\pgfqpoint{3.910314in}{1.435701in}}%
\pgfpathlineto{\pgfqpoint{3.910314in}{1.435701in}}%
\pgfpathlineto{\pgfqpoint{3.910314in}{1.439959in}}%
\pgfpathlineto{\pgfqpoint{3.914572in}{1.439959in}}%
\pgfpathlineto{\pgfqpoint{3.914572in}{1.435701in}}%
\pgfpathmoveto{\pgfqpoint{3.910314in}{1.439959in}}%
\pgfpathlineto{\pgfqpoint{3.910314in}{1.439959in}}%
\pgfpathlineto{\pgfqpoint{3.910314in}{1.444217in}}%
\pgfpathlineto{\pgfqpoint{3.914572in}{1.444217in}}%
\pgfpathlineto{\pgfqpoint{3.914572in}{1.439959in}}%
\pgfpathclose%
\pgfusepath{fill}%
\end{pgfscope}%
\begin{pgfscope}%
\pgfpathrectangle{\pgfqpoint{1.049063in}{0.235000in}}{\pgfqpoint{4.360000in}{4.360000in}}%
\pgfusepath{clip}%
\pgfsetrectcap%
\pgfsetroundjoin%
\pgfsetlinewidth{0.803000pt}%
\definecolor{currentstroke}{rgb}{0.690196,0.690196,0.690196}%
\pgfsetstrokecolor{currentstroke}%
\pgfsetdash{}{0pt}%
\pgfpathmoveto{\pgfqpoint{1.049063in}{0.235000in}}%
\pgfpathlineto{\pgfqpoint{1.049063in}{4.595000in}}%
\pgfusepath{stroke}%
\end{pgfscope}%
\begin{pgfscope}%
\pgfsetbuttcap%
\pgfsetroundjoin%
\definecolor{currentfill}{rgb}{0.000000,0.000000,0.000000}%
\pgfsetfillcolor{currentfill}%
\pgfsetlinewidth{0.803000pt}%
\definecolor{currentstroke}{rgb}{0.000000,0.000000,0.000000}%
\pgfsetstrokecolor{currentstroke}%
\pgfsetdash{}{0pt}%
\pgfsys@defobject{currentmarker}{\pgfqpoint{0.000000in}{-0.048611in}}{\pgfqpoint{0.000000in}{0.000000in}}{%
\pgfpathmoveto{\pgfqpoint{0.000000in}{0.000000in}}%
\pgfpathlineto{\pgfqpoint{0.000000in}{-0.048611in}}%
\pgfusepath{stroke,fill}%
}%
\begin{pgfscope}%
\pgfsys@transformshift{1.049063in}{2.415000in}%
\pgfsys@useobject{currentmarker}{}%
\end{pgfscope}%
\end{pgfscope}%
\begin{pgfscope}%
\definecolor{textcolor}{rgb}{0.000000,0.000000,0.000000}%
\pgfsetstrokecolor{textcolor}%
\pgfsetfillcolor{textcolor}%
\pgftext[x=1.049063in,y=2.317778in,,top]{\color{textcolor}\sffamily\fontsize{10.000000}{12.000000}\selectfont −20}%
\end{pgfscope}%
\begin{pgfscope}%
\pgfpathrectangle{\pgfqpoint{1.049063in}{0.235000in}}{\pgfqpoint{4.360000in}{4.360000in}}%
\pgfusepath{clip}%
\pgfsetrectcap%
\pgfsetroundjoin%
\pgfsetlinewidth{0.803000pt}%
\definecolor{currentstroke}{rgb}{0.690196,0.690196,0.690196}%
\pgfsetstrokecolor{currentstroke}%
\pgfsetdash{}{0pt}%
\pgfpathmoveto{\pgfqpoint{1.594062in}{0.235000in}}%
\pgfpathlineto{\pgfqpoint{1.594062in}{4.595000in}}%
\pgfusepath{stroke}%
\end{pgfscope}%
\begin{pgfscope}%
\pgfsetbuttcap%
\pgfsetroundjoin%
\definecolor{currentfill}{rgb}{0.000000,0.000000,0.000000}%
\pgfsetfillcolor{currentfill}%
\pgfsetlinewidth{0.803000pt}%
\definecolor{currentstroke}{rgb}{0.000000,0.000000,0.000000}%
\pgfsetstrokecolor{currentstroke}%
\pgfsetdash{}{0pt}%
\pgfsys@defobject{currentmarker}{\pgfqpoint{0.000000in}{-0.048611in}}{\pgfqpoint{0.000000in}{0.000000in}}{%
\pgfpathmoveto{\pgfqpoint{0.000000in}{0.000000in}}%
\pgfpathlineto{\pgfqpoint{0.000000in}{-0.048611in}}%
\pgfusepath{stroke,fill}%
}%
\begin{pgfscope}%
\pgfsys@transformshift{1.594062in}{2.415000in}%
\pgfsys@useobject{currentmarker}{}%
\end{pgfscope}%
\end{pgfscope}%
\begin{pgfscope}%
\definecolor{textcolor}{rgb}{0.000000,0.000000,0.000000}%
\pgfsetstrokecolor{textcolor}%
\pgfsetfillcolor{textcolor}%
\pgftext[x=1.594062in,y=2.317778in,,top]{\color{textcolor}\sffamily\fontsize{10.000000}{12.000000}\selectfont −15}%
\end{pgfscope}%
\begin{pgfscope}%
\pgfpathrectangle{\pgfqpoint{1.049063in}{0.235000in}}{\pgfqpoint{4.360000in}{4.360000in}}%
\pgfusepath{clip}%
\pgfsetrectcap%
\pgfsetroundjoin%
\pgfsetlinewidth{0.803000pt}%
\definecolor{currentstroke}{rgb}{0.690196,0.690196,0.690196}%
\pgfsetstrokecolor{currentstroke}%
\pgfsetdash{}{0pt}%
\pgfpathmoveto{\pgfqpoint{2.139063in}{0.235000in}}%
\pgfpathlineto{\pgfqpoint{2.139063in}{4.595000in}}%
\pgfusepath{stroke}%
\end{pgfscope}%
\begin{pgfscope}%
\pgfsetbuttcap%
\pgfsetroundjoin%
\definecolor{currentfill}{rgb}{0.000000,0.000000,0.000000}%
\pgfsetfillcolor{currentfill}%
\pgfsetlinewidth{0.803000pt}%
\definecolor{currentstroke}{rgb}{0.000000,0.000000,0.000000}%
\pgfsetstrokecolor{currentstroke}%
\pgfsetdash{}{0pt}%
\pgfsys@defobject{currentmarker}{\pgfqpoint{0.000000in}{-0.048611in}}{\pgfqpoint{0.000000in}{0.000000in}}{%
\pgfpathmoveto{\pgfqpoint{0.000000in}{0.000000in}}%
\pgfpathlineto{\pgfqpoint{0.000000in}{-0.048611in}}%
\pgfusepath{stroke,fill}%
}%
\begin{pgfscope}%
\pgfsys@transformshift{2.139063in}{2.415000in}%
\pgfsys@useobject{currentmarker}{}%
\end{pgfscope}%
\end{pgfscope}%
\begin{pgfscope}%
\definecolor{textcolor}{rgb}{0.000000,0.000000,0.000000}%
\pgfsetstrokecolor{textcolor}%
\pgfsetfillcolor{textcolor}%
\pgftext[x=2.139063in,y=2.317778in,,top]{\color{textcolor}\sffamily\fontsize{10.000000}{12.000000}\selectfont −10}%
\end{pgfscope}%
\begin{pgfscope}%
\pgfpathrectangle{\pgfqpoint{1.049063in}{0.235000in}}{\pgfqpoint{4.360000in}{4.360000in}}%
\pgfusepath{clip}%
\pgfsetrectcap%
\pgfsetroundjoin%
\pgfsetlinewidth{0.803000pt}%
\definecolor{currentstroke}{rgb}{0.690196,0.690196,0.690196}%
\pgfsetstrokecolor{currentstroke}%
\pgfsetdash{}{0pt}%
\pgfpathmoveto{\pgfqpoint{2.684063in}{0.235000in}}%
\pgfpathlineto{\pgfqpoint{2.684063in}{4.595000in}}%
\pgfusepath{stroke}%
\end{pgfscope}%
\begin{pgfscope}%
\pgfsetbuttcap%
\pgfsetroundjoin%
\definecolor{currentfill}{rgb}{0.000000,0.000000,0.000000}%
\pgfsetfillcolor{currentfill}%
\pgfsetlinewidth{0.803000pt}%
\definecolor{currentstroke}{rgb}{0.000000,0.000000,0.000000}%
\pgfsetstrokecolor{currentstroke}%
\pgfsetdash{}{0pt}%
\pgfsys@defobject{currentmarker}{\pgfqpoint{0.000000in}{-0.048611in}}{\pgfqpoint{0.000000in}{0.000000in}}{%
\pgfpathmoveto{\pgfqpoint{0.000000in}{0.000000in}}%
\pgfpathlineto{\pgfqpoint{0.000000in}{-0.048611in}}%
\pgfusepath{stroke,fill}%
}%
\begin{pgfscope}%
\pgfsys@transformshift{2.684063in}{2.415000in}%
\pgfsys@useobject{currentmarker}{}%
\end{pgfscope}%
\end{pgfscope}%
\begin{pgfscope}%
\definecolor{textcolor}{rgb}{0.000000,0.000000,0.000000}%
\pgfsetstrokecolor{textcolor}%
\pgfsetfillcolor{textcolor}%
\pgftext[x=2.684063in,y=2.317778in,,top]{\color{textcolor}\sffamily\fontsize{10.000000}{12.000000}\selectfont −5}%
\end{pgfscope}%
\begin{pgfscope}%
\pgfpathrectangle{\pgfqpoint{1.049063in}{0.235000in}}{\pgfqpoint{4.360000in}{4.360000in}}%
\pgfusepath{clip}%
\pgfsetrectcap%
\pgfsetroundjoin%
\pgfsetlinewidth{0.803000pt}%
\definecolor{currentstroke}{rgb}{0.690196,0.690196,0.690196}%
\pgfsetstrokecolor{currentstroke}%
\pgfsetdash{}{0pt}%
\pgfpathmoveto{\pgfqpoint{3.229062in}{0.235000in}}%
\pgfpathlineto{\pgfqpoint{3.229062in}{4.595000in}}%
\pgfusepath{stroke}%
\end{pgfscope}%
\begin{pgfscope}%
\pgfsetbuttcap%
\pgfsetroundjoin%
\definecolor{currentfill}{rgb}{0.000000,0.000000,0.000000}%
\pgfsetfillcolor{currentfill}%
\pgfsetlinewidth{0.803000pt}%
\definecolor{currentstroke}{rgb}{0.000000,0.000000,0.000000}%
\pgfsetstrokecolor{currentstroke}%
\pgfsetdash{}{0pt}%
\pgfsys@defobject{currentmarker}{\pgfqpoint{0.000000in}{-0.048611in}}{\pgfqpoint{0.000000in}{0.000000in}}{%
\pgfpathmoveto{\pgfqpoint{0.000000in}{0.000000in}}%
\pgfpathlineto{\pgfqpoint{0.000000in}{-0.048611in}}%
\pgfusepath{stroke,fill}%
}%
\begin{pgfscope}%
\pgfsys@transformshift{3.229062in}{2.415000in}%
\pgfsys@useobject{currentmarker}{}%
\end{pgfscope}%
\end{pgfscope}%
\begin{pgfscope}%
\definecolor{textcolor}{rgb}{0.000000,0.000000,0.000000}%
\pgfsetstrokecolor{textcolor}%
\pgfsetfillcolor{textcolor}%
\pgftext[x=3.229062in,y=2.317778in,,top]{\color{textcolor}\sffamily\fontsize{10.000000}{12.000000}\selectfont 0}%
\end{pgfscope}%
\begin{pgfscope}%
\pgfpathrectangle{\pgfqpoint{1.049063in}{0.235000in}}{\pgfqpoint{4.360000in}{4.360000in}}%
\pgfusepath{clip}%
\pgfsetrectcap%
\pgfsetroundjoin%
\pgfsetlinewidth{0.803000pt}%
\definecolor{currentstroke}{rgb}{0.690196,0.690196,0.690196}%
\pgfsetstrokecolor{currentstroke}%
\pgfsetdash{}{0pt}%
\pgfpathmoveto{\pgfqpoint{3.774062in}{0.235000in}}%
\pgfpathlineto{\pgfqpoint{3.774062in}{4.595000in}}%
\pgfusepath{stroke}%
\end{pgfscope}%
\begin{pgfscope}%
\pgfsetbuttcap%
\pgfsetroundjoin%
\definecolor{currentfill}{rgb}{0.000000,0.000000,0.000000}%
\pgfsetfillcolor{currentfill}%
\pgfsetlinewidth{0.803000pt}%
\definecolor{currentstroke}{rgb}{0.000000,0.000000,0.000000}%
\pgfsetstrokecolor{currentstroke}%
\pgfsetdash{}{0pt}%
\pgfsys@defobject{currentmarker}{\pgfqpoint{0.000000in}{-0.048611in}}{\pgfqpoint{0.000000in}{0.000000in}}{%
\pgfpathmoveto{\pgfqpoint{0.000000in}{0.000000in}}%
\pgfpathlineto{\pgfqpoint{0.000000in}{-0.048611in}}%
\pgfusepath{stroke,fill}%
}%
\begin{pgfscope}%
\pgfsys@transformshift{3.774062in}{2.415000in}%
\pgfsys@useobject{currentmarker}{}%
\end{pgfscope}%
\end{pgfscope}%
\begin{pgfscope}%
\definecolor{textcolor}{rgb}{0.000000,0.000000,0.000000}%
\pgfsetstrokecolor{textcolor}%
\pgfsetfillcolor{textcolor}%
\pgftext[x=3.774062in,y=2.317778in,,top]{\color{textcolor}\sffamily\fontsize{10.000000}{12.000000}\selectfont 5}%
\end{pgfscope}%
\begin{pgfscope}%
\pgfpathrectangle{\pgfqpoint{1.049063in}{0.235000in}}{\pgfqpoint{4.360000in}{4.360000in}}%
\pgfusepath{clip}%
\pgfsetrectcap%
\pgfsetroundjoin%
\pgfsetlinewidth{0.803000pt}%
\definecolor{currentstroke}{rgb}{0.690196,0.690196,0.690196}%
\pgfsetstrokecolor{currentstroke}%
\pgfsetdash{}{0pt}%
\pgfpathmoveto{\pgfqpoint{4.319063in}{0.235000in}}%
\pgfpathlineto{\pgfqpoint{4.319063in}{4.595000in}}%
\pgfusepath{stroke}%
\end{pgfscope}%
\begin{pgfscope}%
\pgfsetbuttcap%
\pgfsetroundjoin%
\definecolor{currentfill}{rgb}{0.000000,0.000000,0.000000}%
\pgfsetfillcolor{currentfill}%
\pgfsetlinewidth{0.803000pt}%
\definecolor{currentstroke}{rgb}{0.000000,0.000000,0.000000}%
\pgfsetstrokecolor{currentstroke}%
\pgfsetdash{}{0pt}%
\pgfsys@defobject{currentmarker}{\pgfqpoint{0.000000in}{-0.048611in}}{\pgfqpoint{0.000000in}{0.000000in}}{%
\pgfpathmoveto{\pgfqpoint{0.000000in}{0.000000in}}%
\pgfpathlineto{\pgfqpoint{0.000000in}{-0.048611in}}%
\pgfusepath{stroke,fill}%
}%
\begin{pgfscope}%
\pgfsys@transformshift{4.319063in}{2.415000in}%
\pgfsys@useobject{currentmarker}{}%
\end{pgfscope}%
\end{pgfscope}%
\begin{pgfscope}%
\definecolor{textcolor}{rgb}{0.000000,0.000000,0.000000}%
\pgfsetstrokecolor{textcolor}%
\pgfsetfillcolor{textcolor}%
\pgftext[x=4.319063in,y=2.317778in,,top]{\color{textcolor}\sffamily\fontsize{10.000000}{12.000000}\selectfont 10}%
\end{pgfscope}%
\begin{pgfscope}%
\pgfpathrectangle{\pgfqpoint{1.049063in}{0.235000in}}{\pgfqpoint{4.360000in}{4.360000in}}%
\pgfusepath{clip}%
\pgfsetrectcap%
\pgfsetroundjoin%
\pgfsetlinewidth{0.803000pt}%
\definecolor{currentstroke}{rgb}{0.690196,0.690196,0.690196}%
\pgfsetstrokecolor{currentstroke}%
\pgfsetdash{}{0pt}%
\pgfpathmoveto{\pgfqpoint{4.864063in}{0.235000in}}%
\pgfpathlineto{\pgfqpoint{4.864063in}{4.595000in}}%
\pgfusepath{stroke}%
\end{pgfscope}%
\begin{pgfscope}%
\pgfsetbuttcap%
\pgfsetroundjoin%
\definecolor{currentfill}{rgb}{0.000000,0.000000,0.000000}%
\pgfsetfillcolor{currentfill}%
\pgfsetlinewidth{0.803000pt}%
\definecolor{currentstroke}{rgb}{0.000000,0.000000,0.000000}%
\pgfsetstrokecolor{currentstroke}%
\pgfsetdash{}{0pt}%
\pgfsys@defobject{currentmarker}{\pgfqpoint{0.000000in}{-0.048611in}}{\pgfqpoint{0.000000in}{0.000000in}}{%
\pgfpathmoveto{\pgfqpoint{0.000000in}{0.000000in}}%
\pgfpathlineto{\pgfqpoint{0.000000in}{-0.048611in}}%
\pgfusepath{stroke,fill}%
}%
\begin{pgfscope}%
\pgfsys@transformshift{4.864063in}{2.415000in}%
\pgfsys@useobject{currentmarker}{}%
\end{pgfscope}%
\end{pgfscope}%
\begin{pgfscope}%
\definecolor{textcolor}{rgb}{0.000000,0.000000,0.000000}%
\pgfsetstrokecolor{textcolor}%
\pgfsetfillcolor{textcolor}%
\pgftext[x=4.864063in,y=2.317778in,,top]{\color{textcolor}\sffamily\fontsize{10.000000}{12.000000}\selectfont 15}%
\end{pgfscope}%
\begin{pgfscope}%
\pgfpathrectangle{\pgfqpoint{1.049063in}{0.235000in}}{\pgfqpoint{4.360000in}{4.360000in}}%
\pgfusepath{clip}%
\pgfsetrectcap%
\pgfsetroundjoin%
\pgfsetlinewidth{0.803000pt}%
\definecolor{currentstroke}{rgb}{0.690196,0.690196,0.690196}%
\pgfsetstrokecolor{currentstroke}%
\pgfsetdash{}{0pt}%
\pgfpathmoveto{\pgfqpoint{5.409063in}{0.235000in}}%
\pgfpathlineto{\pgfqpoint{5.409063in}{4.595000in}}%
\pgfusepath{stroke}%
\end{pgfscope}%
\begin{pgfscope}%
\pgfsetbuttcap%
\pgfsetroundjoin%
\definecolor{currentfill}{rgb}{0.000000,0.000000,0.000000}%
\pgfsetfillcolor{currentfill}%
\pgfsetlinewidth{0.803000pt}%
\definecolor{currentstroke}{rgb}{0.000000,0.000000,0.000000}%
\pgfsetstrokecolor{currentstroke}%
\pgfsetdash{}{0pt}%
\pgfsys@defobject{currentmarker}{\pgfqpoint{0.000000in}{-0.048611in}}{\pgfqpoint{0.000000in}{0.000000in}}{%
\pgfpathmoveto{\pgfqpoint{0.000000in}{0.000000in}}%
\pgfpathlineto{\pgfqpoint{0.000000in}{-0.048611in}}%
\pgfusepath{stroke,fill}%
}%
\begin{pgfscope}%
\pgfsys@transformshift{5.409063in}{2.415000in}%
\pgfsys@useobject{currentmarker}{}%
\end{pgfscope}%
\end{pgfscope}%
\begin{pgfscope}%
\definecolor{textcolor}{rgb}{0.000000,0.000000,0.000000}%
\pgfsetstrokecolor{textcolor}%
\pgfsetfillcolor{textcolor}%
\pgftext[x=5.409063in,y=2.317778in,,top]{\color{textcolor}\sffamily\fontsize{10.000000}{12.000000}\selectfont 20}%
\end{pgfscope}%
\begin{pgfscope}%
\definecolor{textcolor}{rgb}{0.000000,0.000000,0.000000}%
\pgfsetstrokecolor{textcolor}%
\pgfsetfillcolor{textcolor}%
\pgftext[x=5.409063in,y=2.127809in,,top]{\color{textcolor}\sffamily\fontsize{10.000000}{12.000000}\selectfont x}%
\end{pgfscope}%
\begin{pgfscope}%
\pgfpathrectangle{\pgfqpoint{1.049063in}{0.235000in}}{\pgfqpoint{4.360000in}{4.360000in}}%
\pgfusepath{clip}%
\pgfsetrectcap%
\pgfsetroundjoin%
\pgfsetlinewidth{0.803000pt}%
\definecolor{currentstroke}{rgb}{0.690196,0.690196,0.690196}%
\pgfsetstrokecolor{currentstroke}%
\pgfsetdash{}{0pt}%
\pgfpathmoveto{\pgfqpoint{1.049063in}{0.235000in}}%
\pgfpathlineto{\pgfqpoint{5.409063in}{0.235000in}}%
\pgfusepath{stroke}%
\end{pgfscope}%
\begin{pgfscope}%
\pgfsetbuttcap%
\pgfsetroundjoin%
\definecolor{currentfill}{rgb}{0.000000,0.000000,0.000000}%
\pgfsetfillcolor{currentfill}%
\pgfsetlinewidth{0.803000pt}%
\definecolor{currentstroke}{rgb}{0.000000,0.000000,0.000000}%
\pgfsetstrokecolor{currentstroke}%
\pgfsetdash{}{0pt}%
\pgfsys@defobject{currentmarker}{\pgfqpoint{-0.048611in}{0.000000in}}{\pgfqpoint{-0.000000in}{0.000000in}}{%
\pgfpathmoveto{\pgfqpoint{-0.000000in}{0.000000in}}%
\pgfpathlineto{\pgfqpoint{-0.048611in}{0.000000in}}%
\pgfusepath{stroke,fill}%
}%
\begin{pgfscope}%
\pgfsys@transformshift{3.229062in}{0.235000in}%
\pgfsys@useobject{currentmarker}{}%
\end{pgfscope}%
\end{pgfscope}%
\begin{pgfscope}%
\definecolor{textcolor}{rgb}{0.000000,0.000000,0.000000}%
\pgfsetstrokecolor{textcolor}%
\pgfsetfillcolor{textcolor}%
\pgftext[x=2.838736in, y=0.182238in, left, base]{\color{textcolor}\sffamily\fontsize{10.000000}{12.000000}\selectfont −20}%
\end{pgfscope}%
\begin{pgfscope}%
\pgfpathrectangle{\pgfqpoint{1.049063in}{0.235000in}}{\pgfqpoint{4.360000in}{4.360000in}}%
\pgfusepath{clip}%
\pgfsetrectcap%
\pgfsetroundjoin%
\pgfsetlinewidth{0.803000pt}%
\definecolor{currentstroke}{rgb}{0.690196,0.690196,0.690196}%
\pgfsetstrokecolor{currentstroke}%
\pgfsetdash{}{0pt}%
\pgfpathmoveto{\pgfqpoint{1.049063in}{0.780000in}}%
\pgfpathlineto{\pgfqpoint{5.409063in}{0.780000in}}%
\pgfusepath{stroke}%
\end{pgfscope}%
\begin{pgfscope}%
\pgfsetbuttcap%
\pgfsetroundjoin%
\definecolor{currentfill}{rgb}{0.000000,0.000000,0.000000}%
\pgfsetfillcolor{currentfill}%
\pgfsetlinewidth{0.803000pt}%
\definecolor{currentstroke}{rgb}{0.000000,0.000000,0.000000}%
\pgfsetstrokecolor{currentstroke}%
\pgfsetdash{}{0pt}%
\pgfsys@defobject{currentmarker}{\pgfqpoint{-0.048611in}{0.000000in}}{\pgfqpoint{-0.000000in}{0.000000in}}{%
\pgfpathmoveto{\pgfqpoint{-0.000000in}{0.000000in}}%
\pgfpathlineto{\pgfqpoint{-0.048611in}{0.000000in}}%
\pgfusepath{stroke,fill}%
}%
\begin{pgfscope}%
\pgfsys@transformshift{3.229062in}{0.780000in}%
\pgfsys@useobject{currentmarker}{}%
\end{pgfscope}%
\end{pgfscope}%
\begin{pgfscope}%
\definecolor{textcolor}{rgb}{0.000000,0.000000,0.000000}%
\pgfsetstrokecolor{textcolor}%
\pgfsetfillcolor{textcolor}%
\pgftext[x=2.838736in, y=0.727238in, left, base]{\color{textcolor}\sffamily\fontsize{10.000000}{12.000000}\selectfont −15}%
\end{pgfscope}%
\begin{pgfscope}%
\pgfpathrectangle{\pgfqpoint{1.049063in}{0.235000in}}{\pgfqpoint{4.360000in}{4.360000in}}%
\pgfusepath{clip}%
\pgfsetrectcap%
\pgfsetroundjoin%
\pgfsetlinewidth{0.803000pt}%
\definecolor{currentstroke}{rgb}{0.690196,0.690196,0.690196}%
\pgfsetstrokecolor{currentstroke}%
\pgfsetdash{}{0pt}%
\pgfpathmoveto{\pgfqpoint{1.049063in}{1.325000in}}%
\pgfpathlineto{\pgfqpoint{5.409063in}{1.325000in}}%
\pgfusepath{stroke}%
\end{pgfscope}%
\begin{pgfscope}%
\pgfsetbuttcap%
\pgfsetroundjoin%
\definecolor{currentfill}{rgb}{0.000000,0.000000,0.000000}%
\pgfsetfillcolor{currentfill}%
\pgfsetlinewidth{0.803000pt}%
\definecolor{currentstroke}{rgb}{0.000000,0.000000,0.000000}%
\pgfsetstrokecolor{currentstroke}%
\pgfsetdash{}{0pt}%
\pgfsys@defobject{currentmarker}{\pgfqpoint{-0.048611in}{0.000000in}}{\pgfqpoint{-0.000000in}{0.000000in}}{%
\pgfpathmoveto{\pgfqpoint{-0.000000in}{0.000000in}}%
\pgfpathlineto{\pgfqpoint{-0.048611in}{0.000000in}}%
\pgfusepath{stroke,fill}%
}%
\begin{pgfscope}%
\pgfsys@transformshift{3.229062in}{1.325000in}%
\pgfsys@useobject{currentmarker}{}%
\end{pgfscope}%
\end{pgfscope}%
\begin{pgfscope}%
\definecolor{textcolor}{rgb}{0.000000,0.000000,0.000000}%
\pgfsetstrokecolor{textcolor}%
\pgfsetfillcolor{textcolor}%
\pgftext[x=2.838736in, y=1.272238in, left, base]{\color{textcolor}\sffamily\fontsize{10.000000}{12.000000}\selectfont −10}%
\end{pgfscope}%
\begin{pgfscope}%
\pgfpathrectangle{\pgfqpoint{1.049063in}{0.235000in}}{\pgfqpoint{4.360000in}{4.360000in}}%
\pgfusepath{clip}%
\pgfsetrectcap%
\pgfsetroundjoin%
\pgfsetlinewidth{0.803000pt}%
\definecolor{currentstroke}{rgb}{0.690196,0.690196,0.690196}%
\pgfsetstrokecolor{currentstroke}%
\pgfsetdash{}{0pt}%
\pgfpathmoveto{\pgfqpoint{1.049063in}{1.870000in}}%
\pgfpathlineto{\pgfqpoint{5.409063in}{1.870000in}}%
\pgfusepath{stroke}%
\end{pgfscope}%
\begin{pgfscope}%
\pgfsetbuttcap%
\pgfsetroundjoin%
\definecolor{currentfill}{rgb}{0.000000,0.000000,0.000000}%
\pgfsetfillcolor{currentfill}%
\pgfsetlinewidth{0.803000pt}%
\definecolor{currentstroke}{rgb}{0.000000,0.000000,0.000000}%
\pgfsetstrokecolor{currentstroke}%
\pgfsetdash{}{0pt}%
\pgfsys@defobject{currentmarker}{\pgfqpoint{-0.048611in}{0.000000in}}{\pgfqpoint{-0.000000in}{0.000000in}}{%
\pgfpathmoveto{\pgfqpoint{-0.000000in}{0.000000in}}%
\pgfpathlineto{\pgfqpoint{-0.048611in}{0.000000in}}%
\pgfusepath{stroke,fill}%
}%
\begin{pgfscope}%
\pgfsys@transformshift{3.229062in}{1.870000in}%
\pgfsys@useobject{currentmarker}{}%
\end{pgfscope}%
\end{pgfscope}%
\begin{pgfscope}%
\definecolor{textcolor}{rgb}{0.000000,0.000000,0.000000}%
\pgfsetstrokecolor{textcolor}%
\pgfsetfillcolor{textcolor}%
\pgftext[x=2.927101in, y=1.817238in, left, base]{\color{textcolor}\sffamily\fontsize{10.000000}{12.000000}\selectfont −5}%
\end{pgfscope}%
\begin{pgfscope}%
\pgfpathrectangle{\pgfqpoint{1.049063in}{0.235000in}}{\pgfqpoint{4.360000in}{4.360000in}}%
\pgfusepath{clip}%
\pgfsetrectcap%
\pgfsetroundjoin%
\pgfsetlinewidth{0.803000pt}%
\definecolor{currentstroke}{rgb}{0.690196,0.690196,0.690196}%
\pgfsetstrokecolor{currentstroke}%
\pgfsetdash{}{0pt}%
\pgfpathmoveto{\pgfqpoint{1.049063in}{2.415000in}}%
\pgfpathlineto{\pgfqpoint{5.409063in}{2.415000in}}%
\pgfusepath{stroke}%
\end{pgfscope}%
\begin{pgfscope}%
\pgfsetbuttcap%
\pgfsetroundjoin%
\definecolor{currentfill}{rgb}{0.000000,0.000000,0.000000}%
\pgfsetfillcolor{currentfill}%
\pgfsetlinewidth{0.803000pt}%
\definecolor{currentstroke}{rgb}{0.000000,0.000000,0.000000}%
\pgfsetstrokecolor{currentstroke}%
\pgfsetdash{}{0pt}%
\pgfsys@defobject{currentmarker}{\pgfqpoint{-0.048611in}{0.000000in}}{\pgfqpoint{-0.000000in}{0.000000in}}{%
\pgfpathmoveto{\pgfqpoint{-0.000000in}{0.000000in}}%
\pgfpathlineto{\pgfqpoint{-0.048611in}{0.000000in}}%
\pgfusepath{stroke,fill}%
}%
\begin{pgfscope}%
\pgfsys@transformshift{3.229062in}{2.415000in}%
\pgfsys@useobject{currentmarker}{}%
\end{pgfscope}%
\end{pgfscope}%
\begin{pgfscope}%
\definecolor{textcolor}{rgb}{0.000000,0.000000,0.000000}%
\pgfsetstrokecolor{textcolor}%
\pgfsetfillcolor{textcolor}%
\pgftext[x=3.043475in, y=2.362238in, left, base]{\color{textcolor}\sffamily\fontsize{10.000000}{12.000000}\selectfont 0}%
\end{pgfscope}%
\begin{pgfscope}%
\pgfpathrectangle{\pgfqpoint{1.049063in}{0.235000in}}{\pgfqpoint{4.360000in}{4.360000in}}%
\pgfusepath{clip}%
\pgfsetrectcap%
\pgfsetroundjoin%
\pgfsetlinewidth{0.803000pt}%
\definecolor{currentstroke}{rgb}{0.690196,0.690196,0.690196}%
\pgfsetstrokecolor{currentstroke}%
\pgfsetdash{}{0pt}%
\pgfpathmoveto{\pgfqpoint{1.049063in}{2.960000in}}%
\pgfpathlineto{\pgfqpoint{5.409063in}{2.960000in}}%
\pgfusepath{stroke}%
\end{pgfscope}%
\begin{pgfscope}%
\pgfsetbuttcap%
\pgfsetroundjoin%
\definecolor{currentfill}{rgb}{0.000000,0.000000,0.000000}%
\pgfsetfillcolor{currentfill}%
\pgfsetlinewidth{0.803000pt}%
\definecolor{currentstroke}{rgb}{0.000000,0.000000,0.000000}%
\pgfsetstrokecolor{currentstroke}%
\pgfsetdash{}{0pt}%
\pgfsys@defobject{currentmarker}{\pgfqpoint{-0.048611in}{0.000000in}}{\pgfqpoint{-0.000000in}{0.000000in}}{%
\pgfpathmoveto{\pgfqpoint{-0.000000in}{0.000000in}}%
\pgfpathlineto{\pgfqpoint{-0.048611in}{0.000000in}}%
\pgfusepath{stroke,fill}%
}%
\begin{pgfscope}%
\pgfsys@transformshift{3.229062in}{2.960000in}%
\pgfsys@useobject{currentmarker}{}%
\end{pgfscope}%
\end{pgfscope}%
\begin{pgfscope}%
\definecolor{textcolor}{rgb}{0.000000,0.000000,0.000000}%
\pgfsetstrokecolor{textcolor}%
\pgfsetfillcolor{textcolor}%
\pgftext[x=3.043475in, y=2.907238in, left, base]{\color{textcolor}\sffamily\fontsize{10.000000}{12.000000}\selectfont 5}%
\end{pgfscope}%
\begin{pgfscope}%
\pgfpathrectangle{\pgfqpoint{1.049063in}{0.235000in}}{\pgfqpoint{4.360000in}{4.360000in}}%
\pgfusepath{clip}%
\pgfsetrectcap%
\pgfsetroundjoin%
\pgfsetlinewidth{0.803000pt}%
\definecolor{currentstroke}{rgb}{0.690196,0.690196,0.690196}%
\pgfsetstrokecolor{currentstroke}%
\pgfsetdash{}{0pt}%
\pgfpathmoveto{\pgfqpoint{1.049063in}{3.505000in}}%
\pgfpathlineto{\pgfqpoint{5.409063in}{3.505000in}}%
\pgfusepath{stroke}%
\end{pgfscope}%
\begin{pgfscope}%
\pgfsetbuttcap%
\pgfsetroundjoin%
\definecolor{currentfill}{rgb}{0.000000,0.000000,0.000000}%
\pgfsetfillcolor{currentfill}%
\pgfsetlinewidth{0.803000pt}%
\definecolor{currentstroke}{rgb}{0.000000,0.000000,0.000000}%
\pgfsetstrokecolor{currentstroke}%
\pgfsetdash{}{0pt}%
\pgfsys@defobject{currentmarker}{\pgfqpoint{-0.048611in}{0.000000in}}{\pgfqpoint{-0.000000in}{0.000000in}}{%
\pgfpathmoveto{\pgfqpoint{-0.000000in}{0.000000in}}%
\pgfpathlineto{\pgfqpoint{-0.048611in}{0.000000in}}%
\pgfusepath{stroke,fill}%
}%
\begin{pgfscope}%
\pgfsys@transformshift{3.229062in}{3.505000in}%
\pgfsys@useobject{currentmarker}{}%
\end{pgfscope}%
\end{pgfscope}%
\begin{pgfscope}%
\definecolor{textcolor}{rgb}{0.000000,0.000000,0.000000}%
\pgfsetstrokecolor{textcolor}%
\pgfsetfillcolor{textcolor}%
\pgftext[x=2.955110in, y=3.452238in, left, base]{\color{textcolor}\sffamily\fontsize{10.000000}{12.000000}\selectfont 10}%
\end{pgfscope}%
\begin{pgfscope}%
\pgfpathrectangle{\pgfqpoint{1.049063in}{0.235000in}}{\pgfqpoint{4.360000in}{4.360000in}}%
\pgfusepath{clip}%
\pgfsetrectcap%
\pgfsetroundjoin%
\pgfsetlinewidth{0.803000pt}%
\definecolor{currentstroke}{rgb}{0.690196,0.690196,0.690196}%
\pgfsetstrokecolor{currentstroke}%
\pgfsetdash{}{0pt}%
\pgfpathmoveto{\pgfqpoint{1.049063in}{4.050000in}}%
\pgfpathlineto{\pgfqpoint{5.409063in}{4.050000in}}%
\pgfusepath{stroke}%
\end{pgfscope}%
\begin{pgfscope}%
\pgfsetbuttcap%
\pgfsetroundjoin%
\definecolor{currentfill}{rgb}{0.000000,0.000000,0.000000}%
\pgfsetfillcolor{currentfill}%
\pgfsetlinewidth{0.803000pt}%
\definecolor{currentstroke}{rgb}{0.000000,0.000000,0.000000}%
\pgfsetstrokecolor{currentstroke}%
\pgfsetdash{}{0pt}%
\pgfsys@defobject{currentmarker}{\pgfqpoint{-0.048611in}{0.000000in}}{\pgfqpoint{-0.000000in}{0.000000in}}{%
\pgfpathmoveto{\pgfqpoint{-0.000000in}{0.000000in}}%
\pgfpathlineto{\pgfqpoint{-0.048611in}{0.000000in}}%
\pgfusepath{stroke,fill}%
}%
\begin{pgfscope}%
\pgfsys@transformshift{3.229062in}{4.050000in}%
\pgfsys@useobject{currentmarker}{}%
\end{pgfscope}%
\end{pgfscope}%
\begin{pgfscope}%
\definecolor{textcolor}{rgb}{0.000000,0.000000,0.000000}%
\pgfsetstrokecolor{textcolor}%
\pgfsetfillcolor{textcolor}%
\pgftext[x=2.955110in, y=3.997238in, left, base]{\color{textcolor}\sffamily\fontsize{10.000000}{12.000000}\selectfont 15}%
\end{pgfscope}%
\begin{pgfscope}%
\pgfpathrectangle{\pgfqpoint{1.049063in}{0.235000in}}{\pgfqpoint{4.360000in}{4.360000in}}%
\pgfusepath{clip}%
\pgfsetrectcap%
\pgfsetroundjoin%
\pgfsetlinewidth{0.803000pt}%
\definecolor{currentstroke}{rgb}{0.690196,0.690196,0.690196}%
\pgfsetstrokecolor{currentstroke}%
\pgfsetdash{}{0pt}%
\pgfpathmoveto{\pgfqpoint{1.049063in}{4.595000in}}%
\pgfpathlineto{\pgfqpoint{5.409063in}{4.595000in}}%
\pgfusepath{stroke}%
\end{pgfscope}%
\begin{pgfscope}%
\pgfsetbuttcap%
\pgfsetroundjoin%
\definecolor{currentfill}{rgb}{0.000000,0.000000,0.000000}%
\pgfsetfillcolor{currentfill}%
\pgfsetlinewidth{0.803000pt}%
\definecolor{currentstroke}{rgb}{0.000000,0.000000,0.000000}%
\pgfsetstrokecolor{currentstroke}%
\pgfsetdash{}{0pt}%
\pgfsys@defobject{currentmarker}{\pgfqpoint{-0.048611in}{0.000000in}}{\pgfqpoint{-0.000000in}{0.000000in}}{%
\pgfpathmoveto{\pgfqpoint{-0.000000in}{0.000000in}}%
\pgfpathlineto{\pgfqpoint{-0.048611in}{0.000000in}}%
\pgfusepath{stroke,fill}%
}%
\begin{pgfscope}%
\pgfsys@transformshift{3.229062in}{4.595000in}%
\pgfsys@useobject{currentmarker}{}%
\end{pgfscope}%
\end{pgfscope}%
\begin{pgfscope}%
\definecolor{textcolor}{rgb}{0.000000,0.000000,0.000000}%
\pgfsetstrokecolor{textcolor}%
\pgfsetfillcolor{textcolor}%
\pgftext[x=2.955110in, y=4.542238in, left, base]{\color{textcolor}\sffamily\fontsize{10.000000}{12.000000}\selectfont 20}%
\end{pgfscope}%
\begin{pgfscope}%
\definecolor{textcolor}{rgb}{0.000000,0.000000,0.000000}%
\pgfsetstrokecolor{textcolor}%
\pgfsetfillcolor{textcolor}%
\pgftext[x=2.783180in,y=4.595000in,,bottom,rotate=90.000000]{\color{textcolor}\sffamily\fontsize{10.000000}{12.000000}\selectfont y}%
\end{pgfscope}%
\begin{pgfscope}%
\pgfsetrectcap%
\pgfsetmiterjoin%
\pgfsetlinewidth{0.803000pt}%
\definecolor{currentstroke}{rgb}{0.000000,0.000000,0.000000}%
\pgfsetstrokecolor{currentstroke}%
\pgfsetdash{}{0pt}%
\pgfpathmoveto{\pgfqpoint{3.229062in}{0.235000in}}%
\pgfpathlineto{\pgfqpoint{3.229062in}{4.595000in}}%
\pgfusepath{stroke}%
\end{pgfscope}%
\begin{pgfscope}%
\pgfsetrectcap%
\pgfsetmiterjoin%
\pgfsetlinewidth{0.000000pt}%
\definecolor{currentstroke}{rgb}{0.000000,0.000000,0.000000}%
\pgfsetstrokecolor{currentstroke}%
\pgfsetstrokeopacity{0.000000}%
\pgfsetdash{}{0pt}%
\pgfpathmoveto{\pgfqpoint{5.409063in}{0.235000in}}%
\pgfpathlineto{\pgfqpoint{5.409063in}{4.595000in}}%
\pgfusepath{}%
\end{pgfscope}%
\begin{pgfscope}%
\pgfsetrectcap%
\pgfsetmiterjoin%
\pgfsetlinewidth{0.803000pt}%
\definecolor{currentstroke}{rgb}{0.000000,0.000000,0.000000}%
\pgfsetstrokecolor{currentstroke}%
\pgfsetdash{}{0pt}%
\pgfpathmoveto{\pgfqpoint{1.049063in}{2.415000in}}%
\pgfpathlineto{\pgfqpoint{5.409063in}{2.415000in}}%
\pgfusepath{stroke}%
\end{pgfscope}%
\begin{pgfscope}%
\pgfsetrectcap%
\pgfsetmiterjoin%
\pgfsetlinewidth{0.000000pt}%
\definecolor{currentstroke}{rgb}{0.000000,0.000000,0.000000}%
\pgfsetstrokecolor{currentstroke}%
\pgfsetstrokeopacity{0.000000}%
\pgfsetdash{}{0pt}%
\pgfpathmoveto{\pgfqpoint{1.049063in}{4.595000in}}%
\pgfpathlineto{\pgfqpoint{5.409063in}{4.595000in}}%
\pgfusepath{}%
\end{pgfscope}%
\end{pgfpicture}%
\makeatother%
\endgroup%
}\end{solution} \end{parts}




\question Dadas las siguientes funciones, dadas por sus gráficas, obtén sus propiedades:

\includegraphics[width=0.61\columnwidth]{funcion2}




\question Representa las siguientes funciones, indicando sus propiedades:
\begin{multicols}{2}
\begin{parts}
\part[] $y=x^2-4x-5$ 
\begin{solution}
\begin{tikzpicture}[domain=-\xa + 2:\xa,>=triangle 45, scale=0.5]


\tikzmath{
			\a = 1; \b = -4; \c = -5; 
			\v = - \b / ( 2 * \a);
			\xmin = (-\xa + 2); \xmax = \xa;
          }
          
\draw[color=red]    plot (\x,{\a*(\x)^2 + \b *\x + \c})             node[right] {$f(x) =\a x^2 + \b x + \c $}; 

\draw[very thin,color=gray] (\xmin - 0.5, \a * \v^2 + \b * \v + \c - 0.5) grid (\xmax + 0.5, 0.5);

\draw[<->] (\xmin -1,0) -- (\xmax + 1,0) node[right] {$x$};
\draw[<->] (0,1) -- (0,\a * \v^2 + \b * \v + \c - 1) node[above] {$y$};

\end{tikzpicture} \end{solution}

\part[] $y=-x^2+4x+5$ 
\begin{solution}
\begin{tikzpicture}[domain=-\xa + 2:\xa,>=triangle 45, scale=0.5]


\tikzmath{
			\a = -1; \b = 4; \c = 5; 
			\v = - \b / ( 2 * \a);
			\xmin = (-\xa + 2); \xmax = \xa;
          }
          
\draw[color=red]    plot (\x,{\a*(\x)^2 + \b *\x + \c})             node[right] {$f(x) =\a x^2 + \b x + \c $}; 

\draw[very thin,color=gray] (\xmin - 0.5, \a * \v^2 + \b * \v + \c + 0.5) grid (\xmax, -0.5);

\draw[<->] (\xmin -1,0) -- (\xmax + 1,0) node[right] {$x$};
\draw[<->] (0,-0.5) -- (0,\a * \v^2 + \b * \v + \c + 0.5) node[above] {$y$};

\end{tikzpicture}
\end{solution}
 
\part[] $y=-x^2-5x+6$ 
\begin{solution} 

\begin{tikzpicture}[domain=-\xa - 3.1:\xa -1.9,>=triangle 45, scale=0.5]


\tikzmath{
			\a = -1; \b = -5; \c = 6; 
			\v = - \b / ( 2 * \a);
			\xmin = (-\xa - 3.1); \xmax = \xa - 1.9;
          }
          
\draw[color=red]    plot (\x,{\a*(\x)^2 + \b *\x + \c})             node[right] {$f(x) =\a x^2 + \b x + \c $}; 

\draw[very thin,color=gray] (\xmin - 0.5, \a * \v*\v + \b * \v + \c + 0.5) grid (\xmax, -0.5);

\draw[<->] (\xmin -1,0) -- (\xmax + 1,0) node[right] {$x$};
\draw[<->] (0,-0.5) -- (0, \a*\v*\v + \b* \v + \c) node[above] {$y$};

\end{tikzpicture}\end{solution}

\part[] $f(x) =
\left\{
	\begin{array}{clc}
		4  & \mbox{si } & x < -2 \\
		-x^2 & \mbox{si } & -2 \leq x < 4 \\
		2x-3 & \mbox{si } & x \geq 4
	\end{array}
\right.$ 
\begin{solution} \begin{tikzpicture}[domain=-4.1:6.1 ,>=triangle 45, scale=0.35]


\tikzmath{
			\a = -1; \b = 0; \c = 0; 
			\v = - \b / ( 2 * \a);
			\m1 = 0; \n1 = 4;
			\m3 = 2; \n3 = -3;
			\xmin1 = - 4.1; \xmax1 = -2;
			\xmin2 = - 2; \xmax2 = 4;
			\xmin3 = 4; \xmax3 = 6.1;
          }
          
 
\draw[color=red, domain=\xmin1 -0.5:\xmax1-0.25]    plot (\x,{\m1*(\x) + \n1}) node[right] {};

\draw [red] (\xmax1,{\m1*(\xmax1) + \n1}) circle (0.25) node [left] {};

\draw[color=red, domain=\xmin2:\xmax2]    plot (\x,{\a*(\x)^2 + \b *\x + \c})             node[right] {}; 
\draw [red, fill] (\xmin2,{\a*(\xmin2)^2 + \b *\xmin2 + \c}) circle (0.25) node [left] {};
\draw [red] (\xmax2,{\a*(\xmax2)^2 + \b *\xmax2 + \c}) circle (0.25) node [left] {};

\draw[color=red, domain=\xmin3  :\xmax3 + 0.1]    plot (\x,{\m3*(\x) + \n3}) node[right] {};
\draw [red] (\xmin3,{\m3*(\xmin3) + \n3}) circle (0.25) node [left] {};

\draw[very thin,color=lightgray,dash pattern=on 1pt off 1pt] (\xmin1 - 0.5, \a * \xmax2*\xmax2 + \b * \xmax2 + \c - 0.5) grid (\xmax3 + 0.5 , \m3 * \xmax3 + \n3);

\draw[<->] (\xmin1 -1,0) -- (\xmax3 + 1,0) node[right] {$x$};
\draw[<->] (0,\a * \xmax2*\xmax2 + \b * \xmax2 + \c - 0.5) -- (0, \m3 * \xmax3 + \n3 ) node[above] {$y$};

\end{tikzpicture} \end{solution}

\part[] $f(x) =
\left\{
	\begin{array}{clc}
		2x  & \mbox{si } & x < -3 \\
		x^2-2x-8 & \mbox{si } & -3 \leq x \leq 3 \\
		2x-3 & \mbox{si } & x \geq 3
	\end{array}
\right.$ 
\begin{solution}  Nótese, que en este caso la definición no sería una función ($x = 3$ tiene dos imágenes)

\begin{tikzpicture}[domain=-4.1:5.1 ,>=triangle 45, scale=0.35]

\tikzmath{
			\a = 1; \b = -2; \c = -8; 
			\v = - \b / ( 2 * \a);
			\m1 = 2; \n1 = 0;
			\m3 = 2; \n3 = -3;
			\xmin1 = - 4.1; \xmax1 = -3;
			\xmin2 = - 3; \xmax2 = 3;
			\xmin3 = 3; \xmax3 = 5.1;
          }
          
 
\draw[color=red, domain=\xmin1 -0.5:\xmax1]    plot (\x,{\m1*(\x) + \n1}) node[right] {};

\draw [red] (\xmax1,{\m1*(\xmax1) + \n1}) circle (0.25) node [left] {};

\draw[color=red, domain=\xmin2:\xmax2]    plot (\x,{\a*(\x)^2 + \b *\x + \c})             node[right] {}; 
\draw [red, fill] (\xmin2,{\a*(\xmin2)^2 + \b *\xmin2 + \c}) circle (0.25) node [left] {};
\draw [red, fill] (\xmax2,{\a*(\xmax2)^2 + \b *\xmax2 + \c}) circle (0.25) node [left] {};

\draw[color=red, domain=\xmin3  :\xmax3 + 0.1]    plot (\x,{\m3*(\x) + \n3}) node[right] {};
\draw [red, fill] (\xmin3,{\m3*(\xmin3) + \n3}) circle (0.25) node [left] {};

\draw[very thin,color=lightgray,dash pattern=on 1pt off 1pt] (\xmin1 - 0.5, \m1 * \xmin1 + \n1 - 0.5) grid (\xmax3 + 0.5 , \m3 * \xmax3 + \n3);

\draw[<->] (\xmin1 -1,0) -- (\xmax3 + 1,0) node[right] {$x$};
\draw[<->] (0,\m1 * \xmin1 + \n1 - 0.5) -- (0, \m3 * \xmax3 + \n3 ) node[above] {$y$};

\end{tikzpicture} \end{solution}




\part[] $f(x) =
\left\{
	\begin{array}{clc}
		x+1  & \mbox{si } & x \leq 0 \\
		x^2-4x+3 & \mbox{si } &  x > 0 
	\end{array}
\right.$ 
\begin{solution}
\begin{tikzpicture}[domain=-4.1:5.1 ,>=triangle 45, scale=0.35]

\tikzmath{
			\a = 1; \b = -4; \c = 3; 
			\v = - \b / ( 2 * \a);
			\m1 = 1; \n1 = 1;
			\xmin1 = - 4.1; \xmax1 = 0;
			\xmin2 = 0; \xmax2 = 5.1;
         }
          
 
\draw[color=red, domain=\xmin1 -0.5:\xmax1]    plot (\x,{\m1*(\x) + \n1}) node[right] {};

\draw [red, fill] (\xmax1,{\m1*(\xmax1) + \n1}) circle (0.25) node [left] {};

\draw[color=red, domain=\xmin2:\xmax2]    plot (\x,{\a*(\x)^2 + \b *\x + \c})             node[right] {}; 
\draw [red] (\xmin2,{\a*(\xmin2)^2 + \b *\xmin2 + \c}) circle (0.25) node [left] {};



\draw[very thin,color=lightgray,dash pattern=on 1pt off 1pt] (\xmin1 - 0.5, \a * \v * \v + \b * \v + \c - 0.5) grid (\xmax2 + 0.5 , \a * \xmax2 * \xmax2 + \b * \xmax2 + \c + 0.5);

\draw[<->] (\xmin1 -1,0) -- (\xmax2 + 1,0) node[right] {$x$};
\draw[<->] (0,\a * \v * \v + \b * \v + \c - 0.5) -- (0, \a * \xmax2 * \xmax2 + \b * \xmax2 + \c + 0.5 ) node[above] {$y$};

\end{tikzpicture}  
\end{solution}

\end{parts}
\end{multicols}


\begin{comment}
\question 
\begin{multicols}{3}
\begin{parts}
\part[]  
\begin{solution} \end{solution}
\end{parts}
\end{multicols}
\end{comment}


\end{questions}
\end{document}

