\documentclass[spanish, 11pt]{exam}

%These tell TeX which packages to use.
\usepackage{array,epsfig}
\usepackage{amsmath}
\usepackage{amsfonts}
\usepackage{amssymb}
\usepackage{amsxtra}
\usepackage{amsthm}
\usepackage{mathrsfs}
\usepackage{color}
\usepackage{multicol}
\usepackage{verbatim}

\usepackage[utf8]{inputenc}
\usepackage[spanish]{babel}
\usepackage{eurosym}

\usepackage{graphicx}
\graphicspath{{../img/}}


%\printanswers
\nopointsinmargin
\pointformat{}

%Pagination stuff.
%\setlength{\topmargin}{-.3 in}
%\setlength{\oddsidemargin}{0in}
%\setlength{\evensidemargin}{0in}
%\setlength{\textheight}{9.in}
%\setlength{\textwidth}{6.5in}
%\pagestyle{empty}

\renewcommand{\solutiontitle}{\noindent\textbf{Sol:}\enspace}

\newcommand{\class}{4º Académicas}
\newcommand{\examdate}{\today}
\newcommand{\examnum}{Repaso de inecuaciones y sistemas}
\newcommand{\tipo}{A}


\newcommand{\timelimit}{50 minutos}



\pagestyle{head}
\firstpageheader{\includegraphics[width=0.2\columnwidth]{header_left}}{\textbf{Departamento de Matemáticas\linebreak \class}\linebreak \examnum}{\includegraphics[width=0.1\columnwidth]{header_right}}
\runningheader{\class}{\examnum}{Página \thepage\ of \numpages}
\runningheadrule

\begin{document}



\begin{questions}

\question Resolver los siguientes ejercicios:
\begin{multicols}{2}
\begin{parts}
\part[]$x^4 + 2x^2- 3x < 0$  
% solve_univariate_inequality (x**4+2*x**2-3*x<0,x,relational =false )
\begin{solution} $ \left(0, 1\right)$ \end{solution}
\part[]$ 2x^2 - 4x - 6 \geqslant 0  $  
\begin{solution} $ \left(-\infty, -1\right] \cup \left[3, \infty\right)$ \end{solution}

\part[]$\dfrac{2x-2}{1-3x}<-\dfrac{2}{3} $  
%from sympy.solvers.inequalities import reduce_rational_inequalities
%reduce_rational_inequalities([[(2*x-2)/(1-3*x) < -2/3]], x,relational=0)
\begin{solution} $\left(-\infty, \frac{1}{3}\right)$\end{solution}

\part[]$\dfrac{x^2-3x+2}{x^2-x-6}\leqslant 0$  
\begin{solution} $\left(-2, 1\right] \cup \left[2, 3\right)$ \end{solution}

\part[]  $\left| {2x + 9} \right| > 3$ 
%reduce_abs_inequality(Abs(2*x + 9) - 3, '>', x)
\begin{solution} $\left(-\infty, -6\right) \cup \left(-3, \infty\right) $ \end{solution}

\end{parts}
\end{multicols}






\question Resolver los siguientes sistemas de inecuaciones:
\begin{multicols}{2}
\begin{parts}
\part[] $\left\{ \begin{gathered}
  \frac{{x - 4}}{2} + \frac{{x + 2}}{3} \leqslant 7 \hfill \\
  2\cdot\left(x-3\right) > x-5\\
\end{gathered}  \right.$  
 
\begin{solution} $\left(1, 10\right]$\end{solution}

\part[]  $\left\{ {\begin{matrix}
   {2x + y \leqslant 4}  \\ 
   {x \geqslant 0}  \\ 
   {y \geqslant 1}  \\ 

 \end{matrix} } \right.$

\end{parts}
\end{multicols}

\question Resuelve los siguientes problemas:
\begin{multicols}{2}
\begin{parts}
\part[] Se tienen dos cuadrados distintos. La suma de dos lados, uno de cada cuadrado, es de 62 centímetros, y la suma de sus áreas, de 1954 centímetros cuadrados. ¿Cuáles son sus medidas?
\begin{solution} $\left\{ {\begin{matrix}
   {x+y=62}  \\ 
   {x^2+y^2=1954}  \\ 
 \end{matrix} } \right. \to \\ s\left [ \left ( 27, \quad 35\right ), \quad \left ( 35, \quad 27\right )\right ]$ \end{solution}
\part[] En una clase hay 5 chicos más que chicas. Sabemos que en total son algo más de 20 alumnos, pero no llegan a 25. ¿Cuál puede ser la composición de la clase?
\begin{solution} $\left\{ {\begin{matrix}
   {y=x+5}  \\ 
   {20<x+y<25}  \\ 
 \end{matrix} } \right. \to$ 8 chicas y 13 chicos o 9 chicas y 14 chicos\end{solution}
\part[]  ¿Cuántos litros de vino de 5\euro/l se deben mezclar con 20 l de otro de 3,50\euro/l para que el
precio de la mezcla sea inferior a 4\euro /l ?
\begin{solution}$5x + 70 < 4\cdot(20 + x)\to\left(20, +\infty\right)\to x<10 \to$ Se deben mezclar menos de 10 l del vino caro  \end{solution}


\end{parts}
\end{multicols}

\begin{comment}
\question 
\begin{multicols}{3}
\begin{parts}
\part[]  
\begin{solution} \end{solution}
\end{parts}
\end{multicols}
\end{comment}



\end{questions}

\end{document}


