\documentclass[spanish, 11pt]{exam}

%These tell TeX which packages to use.
\usepackage{array,epsfig}
\usepackage{amsmath, textcomp}
\usepackage{amsfonts}
\usepackage{amssymb}
\usepackage{amsxtra}
\usepackage{amsthm}
\usepackage{mathrsfs}
\usepackage{color}
\usepackage{multicol}
\usepackage{verbatim}


\usepackage[utf8]{inputenc}
\usepackage[spanish]{babel}
\usepackage{eurosym}

\usepackage{graphicx}
\graphicspath{{../img/}}



%\printanswers
\nopointsinmargin
\pointformat{}

%Pagination stuff.
%\setlength{\topmargin}{-.3 in}
%\setlength{\oddsidemargin}{0in}
%\setlength{\evensidemargin}{0in}
%\setlength{\textheight}{9.in}
%\setlength{\textwidth}{6.5in}
%\pagestyle{empty}

\renewcommand{\solutiontitle}{\noindent\textbf{Sol:}\enspace}

\newcommand{\samedir}{\mathbin{\!/\mkern-5mu/\!}}

\newcommand{\class}{4º Académicas}
\newcommand{\examdate}{\today}
\newcommand{\examnum}{Geometría Analítica}
\newcommand{\tipo}{A}


\newcommand{\timelimit}{50 minutos}



\pagestyle{head}
\firstpageheader{\includegraphics[width=0.2\columnwidth]{header_left}}{\textbf{Departamento de Matemáticas\linebreak \class}\linebreak \examnum}{\includegraphics[width=0.1\columnwidth]{header_right}}
\runningheader{\class}{\examnum}{Página \thepage\ of \numpages}
\runningheadrule

\begin{document}

\begin{questions}

\question Calcula la distancia que hay entre los puntos  
$A(8,10)$ y $B(-2,14)$
%A, B = Point(8,10), Point(-2,14)
%A.distance(B)
\begin{solution}
$2\sqrt{29}$
\end{solution}
\question	Dados los siguientes vectores: $\overrightarrow u \left( {3,{\text{ }}2} \right){\text{ }}$ y ${\text{ }}\overrightarrow v \left( {1,{\text{ }}4} \right)$, calcula:
\begin{multicols}{4}
\begin{parts}
\part[] $\overrightarrow u \, + \,\overrightarrow v $ 
\begin{solution} $(4,6)$ \end{solution}
\part[]$\overrightarrow u \, - \,\overrightarrow v $  
\begin{solution}
$(2,-2) $ \end{solution}

\part[]$2\overrightarrow u \, + 3\,\overrightarrow v $ 
\begin{solution} $(9,16)$\end{solution}

\part[]$3\overrightarrow u \, - 4\,\overrightarrow v $ 
\begin{solution} $ (5,-10)$ \end{solution}

\end{parts}
\end{multicols}

\question Averigua el punto simétrico de $A(5, -1)$ con respecto a $B(4, -2)$.
%from sympy.geometry import Point, Line
%from sympy import pi
%p1, p2 = Point(5,-1), Point(4,-2)
%p3=p1.rotate(pi,p2)
%p3 
\begin{solution} B tiene que ser el punto medio \\
 $ \to (4,-2)=(\dfrac{5+x_1}{2},\dfrac{-1+x_2}{2})\to (3,-3) $\end{solution}

\question Halla el punto medio del segmento de extremos $A(5, -1)$ y $B(4, -2)$ 
%p1.midpoint(p2)
\begin{solution} $(9/2, -3/2)$\end{solution}

\question Dados los puntos $A(2, -3)$, $B(-1, 4)$ y $C(x, 3)$, determina el valor de x para que $A$, $B$ y $C$ estén alineados.
%p1, p2 = Point(2,-3), Point(-1,4)
%y3=3
%l1=Line(p1,p2)
%l1.direction.y
%solve((y3-p2.y)/l1.direction.y-(x-p2.x)/l1.direction.x) 
%comprobacion
%p3= Point(-4/7,3)
%p1.is_collinear(p2,p3)
\begin{solution} $ -\frac{4}{7}$\end{solution}

\question Halla las coordenadas del vértice $D$ del paralelogramo $ABCD$, sabiendo que $A(-1, -2)$, $B(3, 1)$ y $C(1, 3)$.
%A, B, C, D = Point(-1,-2), Point(3,1), Point(1,3), Point
%D=C-Line(A,B).direction
%D
\begin{solution} $(-3,0)$\end{solution}


\question Halla las coordenadas de los puntos medios de los lados del
triángulo de vértices $A(1,3)$, $B(2,5)$ y $C(1,-1)$
\begin{solution} $(3/2,4)$, $(3/2,2)$ y $(1,1)$ \end{solution}

\question Las coordenadas del punto medio del segmento $AB$ son $M(0,1)$. Si las coordenadas de $B$ son $(1,2)$, ¿cuáles son las del punto $A$?
\begin{solution} $A(-1,0)$ \end{solution}

\question Calcula el punto simétrico de $A(1,3)$ respecto de $B(-5,7)$ 
\begin{solution} $C(-11,11)$ \end{solution}

\question Sea un paralelogramo $ABCD$. Si $A(2,3)$, $B(5,1)$ y $C(4,0)$, halla
el vértice $D$
\begin{solution} $D(1,2)$ \end{solution}

\question Escribe la ecuación vectorial y las paramétricas de la recta que pasa por el punto P y tiene por vector direccional a $\overrightarrow{v}$:
\begin{multicols}{2}
\begin{parts}
\part[] $P(2,1)$, $\overrightarrow{v}(1,1)$  
\begin{solution} (x,y)=(2,1)+t(1,1); x=2+t, y=1+t \end{solution}
\part[] $P(2,2)$,$\overrightarrow{v}=[\overrightarrow{CD}]$, siendo $C(2,1)$ y $D(1,0)$  
\begin{solution} (x,y)=(2,2)+t(-1,-1); x=2-t, y=2-t \end{solution}
\part[] $P(0,1)$, $\overrightarrow{v}(2,5)$  
\begin{solution} (x,y)=(0,1)+t(2,5); x=2t, y=1+5t\end{solution}
\part[] $P(8,1)$,$\overrightarrow{v}=[\overrightarrow{PO}]$, siendo $O$ el origen de coordenadas  
\begin{solution} (x,y)=(8,1)+t(-8,-1); x=8-8t, y=1-t \end{solution}
\end{parts}
\end{multicols}

\question Escribe la ecuación continua y general de las recta que pasa por el punto P y tiene por vector direccional a $\overrightarrow{v}$:
\begin{multicols}{2}
\begin{parts}
\part[] $P(2,1)$, $\overrightarrow{v}(1,1)$  
\begin{solution} (x-2)/1=(y-1)/1, x-y-1=0\end{solution}
\part[] $P(2,2)$,$\overrightarrow{v}=[\overrightarrow{CD}]$, siendo $C(2,1)$ y $D(1,0)$  
\begin{solution} (x-2)/-1=(y-2)/-1, x-y=0\end{solution}
\part[] $P(0,1)$, $\overrightarrow{v}(2,5)$  
\begin{solution} (x-0)/2=(y-1)/5, 5x-2y+2=0\end{solution}
\part[] $P(8,1)$,$\overrightarrow{v}=[\overrightarrow{PO}]$, siendo $O$ el origen de coordenadas  
\begin{solution} (x-8)/-8=(y-1)/-1, x-8y=0\end{solution}
\end{parts}
\end{multicols}


\question Dada la recta $r\equiv 3x + y = 2$, halla una recta $s$, paralela a $r$, y otra perpendicular $t$, que pasen por el punto $P(2, – 1)$.
\begin{solution} $r\equiv y=2-3x \to m_1=-3 \to m_2=\frac{1}{3}\\ s\equiv y=-3x+n_1 \land P(2,-1) \in s \to n_1=5 \to s\equiv y=5-3x \\
t\equiv y=\frac{1}{3}x+n_2 \land P(2,-1) \in s \to n_2=-\frac{5}{3} \to t\equiv y=\frac{1}{3}x-\frac{5}{3} \\ $\end{solution}

\question Halla el coeficiente $a$ para que la recta $ax + 4y = 11$ pase por el punto $P(1, 2)$
\begin{solution} $ a+8=11 \to a=3$ \end{solution}



\question Halla las ecuaciones paramétricas de la recta paralela a $2x - y + 3 = 0 $ y que pasa por el punto $P(4, 3)$.
\begin{solution} Paramétrica: $(-3t/2 + 4, -3t + 3)$, general: $3x - 3y/2 - 15/2=0$\end{solution}


\question Dadas las rectas:  $r \equiv \left\{ 
    {\begin{matrix}
	   {x = 2 - 4\lambda }  \\ 
	   {y =  - 2 + \lambda}  \\ 
    \end{matrix} } \right.$ 	y 	
    $s \equiv \left\{ {\begin{matrix}
   {x = 3 + 8\lambda }  \\ 
   {y =  - 1 - 2\lambda }  \\ 
\end{matrix} } \right.$  averigua su posición relativa. Si se cortan, di cuál es el punto de corte

%P1, V1, P2, V2 = Point(2,-2), Point(-4,1), Point(3,-1), Point(8,-2)
%r = Line(P1, P1+V1)
%print(r.direction)
%s = Line(P2, P2+V2)
%print(s.direction)
%print(r.is_parallel(s))
%r.intersection(s)

\begin{solution} $\overrightarrow{d}(-4,1)$ y $\overrightarrow{d'}(8,-2)$, como $\overrightarrow{d'}=-2\cdot\overrightarrow{d} \to \overrightarrow{d}\samedir\overrightarrow{d'}$  \end{solution}

\question ¿Cuál ha de ser el valor de $k$ para que estas dos rectas sean paralelas? $$x + 3y -2 = 0 \ \ \ \  kx + 2y + 3 = 0$$ 
\begin{solution} $m=-1/3$ y $m'=-k/2$ entonces $m=m' \to -1/3=-k/2 \to k=2/3$ \end{solution}

\question Halla el valor de $k$ para que las rectas $2x - 3y + 4 = 0 $,  $-3x + ky -1 = 0$ 
sean perpendiculares \begin{solution} $m=2/3$ y $m'=3/k$ entonces $m\cdot m'=-1 \to 2/k=-1 \to k=-2$\end{solution}



\question Dados los puntos A(-1, -1), B(1, 4) y C(5, 2), hallar: 
\begin{multicols}{2}
\begin{parts}
\part[] Si están alineados   
%(sin(rad(60))-sin(rad(30)))/(sin(rad(60))+sin(rad(30)))
\begin{solution} No \end{solution}
\part[] Mediana trazada desde B    
\begin{solution} Punto medio AC: $(2,1/2)$, Mediana B:$-7x/2 - y + 15/2=0$\end{solution}
\part[] Altura trazada desde A      
\begin{solution} Recta BC: $2x + 4y - 18=0$, Mediana B:$-4x + 2y - 2=0$\end{solution}
\part[] Mediatriz del lado AB 
\begin{solution} Recta AB: $-5x + 2y - 3=0$, Mediatriz AB:$-2x - 5y + 13/2=0$\end{solution}
\end{parts}
\end{multicols}

\question Sean $A(1,0)$, $B(4,-3)$ y $C(5,2)$ los tres vértices de un triángulo. Hallar: 

\begin{parts}
\part[] La ecuación de la recta que pasando por A es paralela a la que pasa por B y C
\begin{solution} Recta BC: $-5x + y + 23=0$, Paralela por A:$-5*x + y + 5=0$\end{solution}
\part[] La ecuación de la mediana que pasa por C.     
\begin{solution} Punto medio AB: $(5/2, -3/2)$, Mediana C:$-7x/2 + 5y/2 + 25/2=0$\end{solution}
\end{parts}


\begin{comment}
\question 
\begin{multicols}{3}
\begin{parts}
\part[]  
\begin{solution} \end{solution}
\end{parts}
\end{multicols}
\end{comment}


\end{questions}
\end{document}

