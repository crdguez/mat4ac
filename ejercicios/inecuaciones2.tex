\documentclass[spanish, 11pt]{exam}

%These tell TeX which packages to use.
\usepackage{array,epsfig}
\usepackage{amsmath}
\usepackage{amsfonts}
\usepackage{amssymb}
\usepackage{amsxtra}
\usepackage{amsthm}
\usepackage{mathrsfs}
\usepackage{color}
\usepackage{multicol}
\usepackage{verbatim}

\usepackage[utf8]{inputenc}
\usepackage[spanish]{babel}
\usepackage{eurosym}

\usepackage{graphicx}
\graphicspath{{../img/}}

%\printanswers
\nopointsinmargin
\pointformat{}

%Pagination stuff.
%\setlength{\topmargin}{-.3 in}
%\setlength{\oddsidemargin}{0in}
%\setlength{\evensidemargin}{0in}
%\setlength{\textheight}{9.in}
%\setlength{\textwidth}{6.5in}
%\pagestyle{empty}

\renewcommand{\solutiontitle}{\noindent\textbf{Sol:}\enspace}

\newcommand{\class}{4º Académicas}
\newcommand{\examdate}{\today}
\newcommand{\examnum}{Inecuaciones 2}
\newcommand{\tipo}{A}


\newcommand{\timelimit}{50 minutos}



\pagestyle{head}
\firstpageheader{\includegraphics[width=0.2\columnwidth]{header_left}}{\textbf{Departamento de Matemáticas\linebreak \class}\linebreak \examnum}{\includegraphics[width=0.1\columnwidth]{header_right}}
\runningheader{\class}{\examnum}{Página \thepage\ of \numpages}
\runningheadrule

\begin{document}



\begin{questions}

\question Resolver las siguientes ecuaciones polinómicas:
\begin{multicols}{2}
\begin{parts}
\part[]$x^3-5x^2+6x\leqslant 0$  
\begin{solution} $ \left(-\infty, 0\right] \cup	\left[2, 3\right]$ \end{solution}
\part[]$ 2x^3+4x^2+2x \geqslant 0 $  
\begin{solution} $ \left\{-1\right\}\cup\left[0, \infty\right)$ \end{solution}
\begin{comment}
\part[]$ x^3 - 2x^2-5x+6 \geqslant 0$  
\begin{solution} $ \left[-2, 1\right]\cup\left[3, \infty\right)$ \end{solution}
\part[]$x^4+3x^3-3x^2\leqslant11x+6 $  
\begin{solution}$\left[-3, 2\right]$ \end{solution}
\end{comment}
\part[]$x^4+6x^3\leqslant-9x^2+4x+12 $  
\begin{solution} $\left[-3, 1\right]$\end{solution}
\part[]$x^4+2x^3-12x^2+14x-5>0 $  
\begin{solution} $\left(-\infty, -5\right)\cup\left(1, \infty\right) $ \end{solution}

\end{parts}
\end{multicols}

\question Resolver las siguientes inecuaciones racionales
\begin{multicols}{3}
\begin{parts}
\part[]  $\frac{{5x - 4}}{{x + 3}}\, - \,2\, \geqslant \,\frac{{2x}}{{x + 3}}$
\begin{solution} $ $ \end{solution}
\part[]  $\frac{x}{{4 - 2x}}\, > \,\frac{3}{{4 - 2x}}$
\begin{solution} $ $ \end{solution}
\part[]  $ \frac{{{x^2} - 9}}{{x - 1}}\, \leqslant \,0$
\begin{solution} $ $ \end{solution}
\part[]  $\frac{{{x^2} - 1}}{{{x^2}}}\, \geqslant \,0$
\begin{solution} $ $ \end{solution}
\part[]  $ \frac{{{x^2} - 3x + 2}}{{4 - {x^2}}}\, \leqslant \,0$
\begin{solution} $ $ \end{solution}
\part[]  $\frac{{{x^2} - 3x - 2}}{{{x^2} + 2x + 6}}\, < \,3$
\begin{solution} $ $ \end{solution}


\end{parts}
\end{multicols}

\question Resolver las siguientes inecuaciones con valor absoluto:
\begin{multicols}{2}
\begin{parts}
\part[]  $\left| {3x - 1} \right|\, \leqslant \,5$
\begin{solution} \end{solution}
\part[]  $\left| {4x + 3} \right|\,\, > \,2$ 
\begin{solution} \end{solution}
\part[]  $\left| {3 - 4x} \right|\, \geqslant \,5$
\begin{solution} \end{solution}
\part[]  $\left| {5x - 3} \right|\, \leqslant 2$
\begin{solution} \end{solution}
\end{parts}
\end{multicols}


\question Resolver los siguientes sistemas de inecuaciones:
\begin{multicols}{2}
\begin{parts}
\part[]  $\left\{ {\begin{matrix}
   { - 2x - 6 \leqslant 0} \hfill  \\ 
   {3x + 3 \leqslant 0} \hfill  \\ 

 \end{matrix} } \right.$ 
\begin{solution} \end{solution}
\part[] $\left\{ {\begin{matrix}
   {3x - 5 \geqslant 2x - 6} \hfill  \\ 
   {4x + 1 < 2x + 7} \hfill  \\ 

 \end{matrix} } \right.$ 
\begin{solution} \end{solution}
\part[] $\left\{ {\begin{matrix}
   {2\left( {3x - 1} \right) - \left( {2 + 4x} \right) > x} \hfill  \\ 
   {2 - \frac{{3x + 1}}{2} \leqslant x\, - \,\frac{{x + 2}}{3}} \hfill  \\ 

 \end{matrix} } \right.$
\begin{solution} \end{solution}
\part[]  $\left\{ {\begin{matrix}
   {\frac{{2x - 1}}{2}\, - \,\frac{{2\left( {x + 1} \right)}}{5}\, \geqslant \, - 1} \hfill  \\ 
   {\frac{{3x + 1}}{4}\, + \,\frac{x}{6}\, < 2} \hfill  \\ 

 \end{matrix} } \right.$
\begin{solution} \end{solution}
\end{parts}
\end{multicols}

\question Resuelve los siguientes problemas
\begin{multicols}{2}
\begin{parts}
\part[] La  base  de  un  rectángulo  excede  en 3 cm a  la altura. Si  elevamos  al  cuadrado  su  diagonal  el  resultado  es menor que 29. Determina las posibles dimensiones del rectángulo. 
\begin{solution} \end{solution}
\part[]  Una fábrica A paga a sus viajantes 10 euros por artículo vendido más una cantidad fija de 400 euros. Otra fábrica B paga  15  euros  por  artículo  vendido  más  una  cantidad  fija  de  300  euros.  ¿Cuántos artículos  debe vender el viajante de fábrica B para ganar más dinero que el viajante de la fábrica A?
\begin{solution} \end{solution}
\part[]  Se ha medido un campo rectangular con un error menor que un metro. Los valores de los lados, $x$ e $y$, cumplen $70 < x < 71$ y $47 < y < 48$. Determina entre qué valores está el perímetro y el área.
\begin{solution} \end{solution}
\part[]  Un alumno ha realizado dos exámenes de Matemáticas obteniendo calificaciones respectivas de 4,5 y 5,7 puntos. ¿Cuánto ha de sacar como mínimo en el tercero para aprobar, si la nota final es la media aritmética de las tres notas? ¿Y si el primer examen cuenta un 15\%, el segundo un 35\% y el tercero un 50\%?
\begin{solution} \end{solution}
\part[]  Entre los triángulos isósceles de lado desigual igual a 20 cm., ¿cuáles tienen perímetro inferior a 200 cm.?
\begin{solution} \end{solution}
\part[]  Un padre tiene 33 años más que su hijo y el abuelo 33 años más que el padre. Hace tres años, sus edades sumaban menos de un siglo. ¿Qué edad puede tener cada uno?
\begin{solution} \end{solution}

\end{parts}
\end{multicols}

\begin{comment}
\question 
\begin{multicols}{3}
\begin{parts}
\part[]  
\begin{solution} \end{solution}
\end{parts}
\end{multicols}
\end{comment}



\end{questions}

\end{document}


