\documentclass[spanish, 11pt]{exam}

%These tell TeX which packages to use.
\usepackage{array,epsfig}
\usepackage{amsmath}
\usepackage{amsfonts}
\usepackage{amssymb}
\usepackage{amsxtra}
\usepackage{amsthm}
\usepackage{mathrsfs}
\usepackage{color}
\usepackage{multicol}
\usepackage{verbatim}
\usepackage{tikz}

\usepackage[utf8]{inputenc}
\usepackage[spanish]{babel}
\usepackage{eurosym}

\usepackage{graphicx}
\graphicspath{{../img/}}


\printanswers
\nopointsinmargin
\pointformat{}

%Pagination stuff.
%\setlength{\topmargin}{-.3 in}
%\setlength{\oddsidemargin}{0in}
%\setlength{\evensidemargin}{0in}
%\setlength{\textheight}{9.in}
%\setlength{\textwidth}{6.5in}
%\pagestyle{empty}

\renewcommand{\solutiontitle}{\noindent\textbf{Sol:}\enspace}

\newcommand{\class}{4º Académicas}
\newcommand{\examdate}{\today}
\newcommand{\examnum}{Inecuaciones 2}
\newcommand{\tipo}{A}


\newcommand{\timelimit}{50 minutos}



\pagestyle{head}
\firstpageheader{\includegraphics[width=0.2\columnwidth]{header_left}}{\textbf{Departamento de Matemáticas\linebreak \class}\linebreak \examnum}{\includegraphics[width=0.1\columnwidth]{header_right}}
\runningheader{\class}{\examnum}{Página \thepage\ of \numpages}
\runningheadrule

\DeclareUnicodeCharacter{2212}{-}

\begin{document}



\begin{questions}

\question Resolver las siguientes ecuaciones polinómicas:
\begin{multicols}{2}
\begin{parts}
\part[]$x^3-5x^2+6x\leqslant 0$  
\begin{solution} $ \left(-\infty, 0\right] \cup	\left[2, 3\right]$ \end{solution}
\part[]$ 2x^3+4x^2+2x \geqslant 0 $  
\begin{solution} $ \left\{-1\right\}\cup\left[0, \infty\right)$ \end{solution}
\begin{comment}
\part[]$ x^3 - 2x^2-5x+6 \geqslant 0$  
\begin{solution} $ \left[-2, 1\right]\cup\left[3, \infty\right)$ \end{solution}
\part[]$x^4+3x^3-3x^2\leqslant11x+6 $  
\begin{solution}$\left[-3, 2\right]$ \end{solution}
\end{comment}
\part[]$x^4+6x^3\leqslant-9x^2+4x+12 $  
\begin{solution} $\left[-3, 1\right]$\end{solution}
\part[]$x^4+2x^3-12x^2+14x-5>0 $  
\begin{solution} $\left(-\infty, -5\right)\cup\left(1, \infty\right) $ \end{solution}

\end{parts}
\end{multicols}

\question Resolver las siguientes inecuaciones racionales
\begin{multicols}{3}
\begin{parts}
\part[]  $\frac{{5x - 4}}{{x + 3}}\, - \,2\, \geqslant \,\frac{{2x}}{{x + 3}}$
%from sympy.solvers.inequalities import reduce_rational_inequalities
%reduce_rational_inequalities([[(5*x-4)/(x+3)-2 >= 2*x/(x+3)]], x,relational=0)
\begin{solution} $ \left(-\infty, -3\right) \cup \left[10, \infty\right)$ \end{solution}
\part[]  $\frac{x}{{4 - 2x}}\, > \,\frac{3}{{4 - 2x}}$
\begin{solution} $ \left(2, 3\right)$ \end{solution}
\part[]  $ \frac{{{x^2} - 9}}{{x - 1}}\, \leqslant \,0$
\begin{solution} $ \left(-\infty, -3\right] \cup \left(1, 3\right]$ \end{solution}
\part[]  $\frac{{{x^2} - 1}}{{{x^2}}}\, \geqslant \,0$
\begin{solution} $\left(-\infty, -1\right] \cup \left[1, \infty\right) $ \end{solution}
\part[]  $ \frac{{{x^2} - 3x + 2}}{{4 - {x^2}}}\, \leqslant \,0$
\begin{solution} $ \left(-\infty, -2\right) \cup \left[1, 2\right) \cup \left(2, \infty\right)$ \end{solution}
\part[]  $\frac{{{x^2} - 3x - 2}}{{{x^2} + 2x + 6}}\, < \,3$
\begin{solution} $ \left(-\infty, \infty\right)$ \end{solution}


\end{parts}
\end{multicols}

\question Resolver las siguientes inecuaciones con valor absoluto:
\begin{multicols}{2}
\begin{parts}

\part[]  $\left| {3x - 1} \right|\, \leqslant \,5$
%reduce_abs_inequalities([(Abs(3*x - 1) - 5, '<=')], x)
\begin{solution} $\left[-\frac{4}{3},2\right]$\end{solution}
\part[]  $\left| {4x + 3} \right|\,\, > \,2$ 
\begin{solution} $\left(-\infty,-\frac{5}{4}\right)\cup\left(-\frac{1}{4},+\infty\right) $ \end{solution}
\part[]  $\left| {3 - 4x} \right|\, \geqslant \,5$
\begin{solution} $\left(-\infty,-\frac{1}{2}\right]\cup\left[2,+\infty\right) $ \end{solution}
\part[]  $\left| {5x - 3} \right|\, \leqslant 2$
\begin{solution} $\left[\frac{1}{5},1\right]$ \end{solution}
\end{parts}
\end{multicols}


\question Resolver los siguientes sistemas de inecuaciones con dos incógnitas:
%\begin{multicols}{1}
\begin{parts}
\part[]  $\left\{ {\begin{matrix}
   {y <  - 2x + 4}  \\ 
   {y \geqslant x}  \\ 

 \end{matrix} } \right.$
 



\begin{solution} \scalebox{.6}{%% Creator: Matplotlib, PGF backend
%%
%% To include the figure in your LaTeX document, write
%%   \input{<filename>.pgf}
%%
%% Make sure the required packages are loaded in your preamble
%%   \usepackage{pgf}
%%
%% and, on pdftex
%%   \usepackage[utf8]{inputenc}\DeclareUnicodeCharacter{2212}{-}
%%
%% or, on luatex and xetex
%%   \usepackage{unicode-math}
%%
%% Figures using additional raster images can only be included by \input if
%% they are in the same directory as the main LaTeX file. For loading figures
%% from other directories you can use the `import` package
%%   \usepackage{import}
%%
%% and then include the figures with
%%   \import{<path to file>}{<filename>.pgf}
%%
%% Matplotlib used the following preamble
%%   \usepackage{fontspec}
%%   \setmainfont{DejaVuSerif.ttf}[Path=/home/hp/Mis_aplicaciones/anaconda3/lib/python3.6/site-packages/matplotlib/mpl-data/fonts/ttf/]
%%   \setsansfont{DejaVuSans.ttf}[Path=/home/hp/Mis_aplicaciones/anaconda3/lib/python3.6/site-packages/matplotlib/mpl-data/fonts/ttf/]
%%   \setmonofont{DejaVuSansMono.ttf}[Path=/home/hp/Mis_aplicaciones/anaconda3/lib/python3.6/site-packages/matplotlib/mpl-data/fonts/ttf/]
%%
\begingroup%
\makeatletter%
\begin{pgfpicture}%
\pgfpathrectangle{\pgfpointorigin}{\pgfqpoint{6.000000in}{4.000000in}}%
\pgfusepath{use as bounding box, clip}%
\begin{pgfscope}%
\pgfsetbuttcap%
\pgfsetmiterjoin%
\pgfsetlinewidth{0.000000pt}%
\definecolor{currentstroke}{rgb}{1.000000,1.000000,1.000000}%
\pgfsetstrokecolor{currentstroke}%
\pgfsetstrokeopacity{0.000000}%
\pgfsetdash{}{0pt}%
\pgfpathmoveto{\pgfqpoint{0.000000in}{0.000000in}}%
\pgfpathlineto{\pgfqpoint{6.000000in}{0.000000in}}%
\pgfpathlineto{\pgfqpoint{6.000000in}{4.000000in}}%
\pgfpathlineto{\pgfqpoint{0.000000in}{4.000000in}}%
\pgfpathclose%
\pgfusepath{}%
\end{pgfscope}%
\begin{pgfscope}%
\pgfsetbuttcap%
\pgfsetmiterjoin%
\definecolor{currentfill}{rgb}{1.000000,1.000000,1.000000}%
\pgfsetfillcolor{currentfill}%
\pgfsetlinewidth{0.000000pt}%
\definecolor{currentstroke}{rgb}{0.000000,0.000000,0.000000}%
\pgfsetstrokecolor{currentstroke}%
\pgfsetstrokeopacity{0.000000}%
\pgfsetdash{}{0pt}%
\pgfpathmoveto{\pgfqpoint{0.750000in}{0.500000in}}%
\pgfpathlineto{\pgfqpoint{5.400000in}{0.500000in}}%
\pgfpathlineto{\pgfqpoint{5.400000in}{3.520000in}}%
\pgfpathlineto{\pgfqpoint{0.750000in}{3.520000in}}%
\pgfpathclose%
\pgfusepath{fill}%
\end{pgfscope}%
\begin{pgfscope}%
\pgfpathrectangle{\pgfqpoint{0.750000in}{0.500000in}}{\pgfqpoint{4.650000in}{3.020000in}}%
\pgfusepath{clip}%
\pgfsetbuttcap%
\pgfsetmiterjoin%
\definecolor{currentfill}{rgb}{0.000000,0.000000,1.000000}%
\pgfsetfillcolor{currentfill}%
\pgfsetlinewidth{0.000000pt}%
\definecolor{currentstroke}{rgb}{0.000000,0.000000,0.000000}%
\pgfsetstrokecolor{currentstroke}%
\pgfsetstrokeopacity{0.000000}%
\pgfsetdash{}{0pt}%
\pgfpathmoveto{\pgfqpoint{0.750004in}{0.594376in}}%
\pgfpathlineto{\pgfqpoint{0.750004in}{0.688752in}}%
\pgfpathlineto{\pgfqpoint{0.895314in}{0.688752in}}%
\pgfpathlineto{\pgfqpoint{0.895314in}{0.594376in}}%
\pgfpathmoveto{\pgfqpoint{0.750004in}{0.688752in}}%
\pgfpathlineto{\pgfqpoint{0.750004in}{0.688752in}}%
\pgfpathlineto{\pgfqpoint{0.750004in}{0.783123in}}%
\pgfpathlineto{\pgfqpoint{0.895314in}{0.783123in}}%
\pgfpathlineto{\pgfqpoint{0.895314in}{0.688752in}}%
\pgfpathmoveto{\pgfqpoint{0.750004in}{0.783123in}}%
\pgfpathlineto{\pgfqpoint{0.750004in}{0.783123in}}%
\pgfpathlineto{\pgfqpoint{0.750004in}{0.877498in}}%
\pgfpathlineto{\pgfqpoint{0.895314in}{0.877498in}}%
\pgfpathlineto{\pgfqpoint{0.895314in}{0.783123in}}%
\pgfpathmoveto{\pgfqpoint{0.750004in}{0.877498in}}%
\pgfpathlineto{\pgfqpoint{0.750004in}{0.877498in}}%
\pgfpathlineto{\pgfqpoint{0.750004in}{0.971875in}}%
\pgfpathlineto{\pgfqpoint{0.895314in}{0.971875in}}%
\pgfpathlineto{\pgfqpoint{0.895314in}{0.877498in}}%
\pgfpathmoveto{\pgfqpoint{0.750004in}{0.971875in}}%
\pgfpathlineto{\pgfqpoint{0.750004in}{0.971875in}}%
\pgfpathlineto{\pgfqpoint{0.750004in}{1.066250in}}%
\pgfpathlineto{\pgfqpoint{0.895314in}{1.066250in}}%
\pgfpathlineto{\pgfqpoint{0.895314in}{0.971875in}}%
\pgfpathmoveto{\pgfqpoint{0.750004in}{1.066250in}}%
\pgfpathlineto{\pgfqpoint{0.750004in}{1.066250in}}%
\pgfpathlineto{\pgfqpoint{0.750004in}{1.160623in}}%
\pgfpathlineto{\pgfqpoint{0.895314in}{1.160623in}}%
\pgfpathlineto{\pgfqpoint{0.895314in}{1.066250in}}%
\pgfpathmoveto{\pgfqpoint{0.750004in}{1.160623in}}%
\pgfpathlineto{\pgfqpoint{0.750004in}{1.160623in}}%
\pgfpathlineto{\pgfqpoint{0.750004in}{1.254999in}}%
\pgfpathlineto{\pgfqpoint{0.895314in}{1.254999in}}%
\pgfpathlineto{\pgfqpoint{0.895314in}{1.160623in}}%
\pgfpathmoveto{\pgfqpoint{0.750004in}{1.254999in}}%
\pgfpathlineto{\pgfqpoint{0.750004in}{1.254999in}}%
\pgfpathlineto{\pgfqpoint{0.750004in}{1.349376in}}%
\pgfpathlineto{\pgfqpoint{0.895314in}{1.349376in}}%
\pgfpathlineto{\pgfqpoint{0.895314in}{1.254999in}}%
\pgfpathmoveto{\pgfqpoint{0.750004in}{1.349376in}}%
\pgfpathlineto{\pgfqpoint{0.750004in}{1.349376in}}%
\pgfpathlineto{\pgfqpoint{0.750004in}{1.443752in}}%
\pgfpathlineto{\pgfqpoint{0.895314in}{1.443752in}}%
\pgfpathlineto{\pgfqpoint{0.895314in}{1.349376in}}%
\pgfpathmoveto{\pgfqpoint{0.750004in}{1.443752in}}%
\pgfpathlineto{\pgfqpoint{0.750004in}{1.443752in}}%
\pgfpathlineto{\pgfqpoint{0.750004in}{1.538125in}}%
\pgfpathlineto{\pgfqpoint{0.895314in}{1.538125in}}%
\pgfpathlineto{\pgfqpoint{0.895314in}{1.443752in}}%
\pgfpathmoveto{\pgfqpoint{0.750004in}{1.538125in}}%
\pgfpathlineto{\pgfqpoint{0.750004in}{1.538125in}}%
\pgfpathlineto{\pgfqpoint{0.750004in}{1.632498in}}%
\pgfpathlineto{\pgfqpoint{0.895314in}{1.632498in}}%
\pgfpathlineto{\pgfqpoint{0.895314in}{1.538125in}}%
\pgfpathmoveto{\pgfqpoint{0.750004in}{1.632498in}}%
\pgfpathlineto{\pgfqpoint{0.750004in}{1.632498in}}%
\pgfpathlineto{\pgfqpoint{0.750004in}{1.726874in}}%
\pgfpathlineto{\pgfqpoint{0.895314in}{1.726874in}}%
\pgfpathlineto{\pgfqpoint{0.895314in}{1.632498in}}%
\pgfpathmoveto{\pgfqpoint{0.750004in}{1.726874in}}%
\pgfpathlineto{\pgfqpoint{0.750004in}{1.726874in}}%
\pgfpathlineto{\pgfqpoint{0.750004in}{1.821250in}}%
\pgfpathlineto{\pgfqpoint{0.895314in}{1.821250in}}%
\pgfpathlineto{\pgfqpoint{0.895314in}{1.726874in}}%
\pgfpathmoveto{\pgfqpoint{0.750004in}{1.821250in}}%
\pgfpathlineto{\pgfqpoint{0.750004in}{1.821250in}}%
\pgfpathlineto{\pgfqpoint{0.750004in}{1.915626in}}%
\pgfpathlineto{\pgfqpoint{0.895314in}{1.915626in}}%
\pgfpathlineto{\pgfqpoint{0.895314in}{1.821250in}}%
\pgfpathmoveto{\pgfqpoint{0.750004in}{1.915626in}}%
\pgfpathlineto{\pgfqpoint{0.750004in}{1.915626in}}%
\pgfpathlineto{\pgfqpoint{0.750004in}{2.009997in}}%
\pgfpathlineto{\pgfqpoint{0.895314in}{2.009997in}}%
\pgfpathlineto{\pgfqpoint{0.895314in}{1.915626in}}%
\pgfpathmoveto{\pgfqpoint{0.750004in}{2.009997in}}%
\pgfpathlineto{\pgfqpoint{0.750004in}{2.009997in}}%
\pgfpathlineto{\pgfqpoint{0.750004in}{2.104378in}}%
\pgfpathlineto{\pgfqpoint{0.895314in}{2.104378in}}%
\pgfpathlineto{\pgfqpoint{0.895314in}{2.009997in}}%
\pgfpathmoveto{\pgfqpoint{0.750004in}{2.104378in}}%
\pgfpathlineto{\pgfqpoint{0.750004in}{2.104378in}}%
\pgfpathlineto{\pgfqpoint{0.750004in}{2.198752in}}%
\pgfpathlineto{\pgfqpoint{0.895314in}{2.198752in}}%
\pgfpathlineto{\pgfqpoint{0.895314in}{2.104378in}}%
\pgfpathmoveto{\pgfqpoint{0.750004in}{2.198752in}}%
\pgfpathlineto{\pgfqpoint{0.750004in}{2.198752in}}%
\pgfpathlineto{\pgfqpoint{0.750004in}{2.293123in}}%
\pgfpathlineto{\pgfqpoint{0.895314in}{2.293123in}}%
\pgfpathlineto{\pgfqpoint{0.895314in}{2.198752in}}%
\pgfpathmoveto{\pgfqpoint{0.750004in}{2.293123in}}%
\pgfpathlineto{\pgfqpoint{0.750004in}{2.293123in}}%
\pgfpathlineto{\pgfqpoint{0.750004in}{2.387497in}}%
\pgfpathlineto{\pgfqpoint{0.895314in}{2.387497in}}%
\pgfpathlineto{\pgfqpoint{0.895314in}{2.293123in}}%
\pgfpathmoveto{\pgfqpoint{0.750004in}{2.387497in}}%
\pgfpathlineto{\pgfqpoint{0.750004in}{2.387497in}}%
\pgfpathlineto{\pgfqpoint{0.750004in}{2.481873in}}%
\pgfpathlineto{\pgfqpoint{0.895314in}{2.481873in}}%
\pgfpathlineto{\pgfqpoint{0.895314in}{2.387497in}}%
\pgfpathmoveto{\pgfqpoint{0.750004in}{2.481873in}}%
\pgfpathlineto{\pgfqpoint{0.750004in}{2.481873in}}%
\pgfpathlineto{\pgfqpoint{0.750004in}{2.576252in}}%
\pgfpathlineto{\pgfqpoint{0.895314in}{2.576252in}}%
\pgfpathlineto{\pgfqpoint{0.895314in}{2.481873in}}%
\pgfpathmoveto{\pgfqpoint{0.750004in}{2.576252in}}%
\pgfpathlineto{\pgfqpoint{0.750004in}{2.576252in}}%
\pgfpathlineto{\pgfqpoint{0.750004in}{2.670624in}}%
\pgfpathlineto{\pgfqpoint{0.895314in}{2.670624in}}%
\pgfpathlineto{\pgfqpoint{0.895314in}{2.576252in}}%
\pgfpathmoveto{\pgfqpoint{0.750004in}{2.670624in}}%
\pgfpathlineto{\pgfqpoint{0.750004in}{2.670624in}}%
\pgfpathlineto{\pgfqpoint{0.750004in}{2.765002in}}%
\pgfpathlineto{\pgfqpoint{0.895314in}{2.765002in}}%
\pgfpathlineto{\pgfqpoint{0.895314in}{2.670624in}}%
\pgfpathmoveto{\pgfqpoint{0.750004in}{2.765002in}}%
\pgfpathlineto{\pgfqpoint{0.750004in}{2.765002in}}%
\pgfpathlineto{\pgfqpoint{0.750004in}{2.859376in}}%
\pgfpathlineto{\pgfqpoint{0.895314in}{2.859376in}}%
\pgfpathlineto{\pgfqpoint{0.895314in}{2.765002in}}%
\pgfpathmoveto{\pgfqpoint{0.750004in}{2.859376in}}%
\pgfpathlineto{\pgfqpoint{0.750004in}{2.859376in}}%
\pgfpathlineto{\pgfqpoint{0.750004in}{2.953748in}}%
\pgfpathlineto{\pgfqpoint{0.895314in}{2.953748in}}%
\pgfpathlineto{\pgfqpoint{0.895314in}{2.859376in}}%
\pgfpathmoveto{\pgfqpoint{0.750004in}{2.953748in}}%
\pgfpathlineto{\pgfqpoint{0.750004in}{2.953748in}}%
\pgfpathlineto{\pgfqpoint{0.750004in}{3.048126in}}%
\pgfpathlineto{\pgfqpoint{0.895314in}{3.048126in}}%
\pgfpathlineto{\pgfqpoint{0.895314in}{2.953748in}}%
\pgfpathmoveto{\pgfqpoint{0.750004in}{3.048126in}}%
\pgfpathlineto{\pgfqpoint{0.750004in}{3.048126in}}%
\pgfpathlineto{\pgfqpoint{0.750004in}{3.142501in}}%
\pgfpathlineto{\pgfqpoint{0.895314in}{3.142501in}}%
\pgfpathlineto{\pgfqpoint{0.895314in}{3.048126in}}%
\pgfpathmoveto{\pgfqpoint{0.750004in}{3.142501in}}%
\pgfpathlineto{\pgfqpoint{0.750004in}{3.142501in}}%
\pgfpathlineto{\pgfqpoint{0.750004in}{3.236873in}}%
\pgfpathlineto{\pgfqpoint{0.895314in}{3.236873in}}%
\pgfpathlineto{\pgfqpoint{0.895314in}{3.142501in}}%
\pgfpathmoveto{\pgfqpoint{0.750004in}{3.236873in}}%
\pgfpathlineto{\pgfqpoint{0.750004in}{3.236873in}}%
\pgfpathlineto{\pgfqpoint{0.750004in}{3.331249in}}%
\pgfpathlineto{\pgfqpoint{0.895314in}{3.331249in}}%
\pgfpathlineto{\pgfqpoint{0.895314in}{3.236873in}}%
\pgfpathmoveto{\pgfqpoint{0.750004in}{3.331249in}}%
\pgfpathlineto{\pgfqpoint{0.750004in}{3.331249in}}%
\pgfpathlineto{\pgfqpoint{0.750004in}{3.425625in}}%
\pgfpathlineto{\pgfqpoint{0.895314in}{3.425625in}}%
\pgfpathlineto{\pgfqpoint{0.895314in}{3.331249in}}%
\pgfpathmoveto{\pgfqpoint{0.750004in}{3.425625in}}%
\pgfpathlineto{\pgfqpoint{0.750004in}{3.425625in}}%
\pgfpathlineto{\pgfqpoint{0.750004in}{3.519999in}}%
\pgfpathlineto{\pgfqpoint{0.895314in}{3.519999in}}%
\pgfpathlineto{\pgfqpoint{0.895314in}{3.425625in}}%
\pgfpathmoveto{\pgfqpoint{0.895314in}{0.688752in}}%
\pgfpathlineto{\pgfqpoint{0.895314in}{0.688752in}}%
\pgfpathlineto{\pgfqpoint{0.895314in}{0.783123in}}%
\pgfpathlineto{\pgfqpoint{1.040623in}{0.783123in}}%
\pgfpathlineto{\pgfqpoint{1.040623in}{0.688752in}}%
\pgfpathmoveto{\pgfqpoint{0.895314in}{0.783123in}}%
\pgfpathlineto{\pgfqpoint{0.895314in}{0.783123in}}%
\pgfpathlineto{\pgfqpoint{0.895314in}{0.877498in}}%
\pgfpathlineto{\pgfqpoint{1.040623in}{0.877498in}}%
\pgfpathlineto{\pgfqpoint{1.040623in}{0.783123in}}%
\pgfpathmoveto{\pgfqpoint{0.895314in}{0.877498in}}%
\pgfpathlineto{\pgfqpoint{0.895314in}{0.877498in}}%
\pgfpathlineto{\pgfqpoint{0.895314in}{0.971875in}}%
\pgfpathlineto{\pgfqpoint{1.040623in}{0.971875in}}%
\pgfpathlineto{\pgfqpoint{1.040623in}{0.877498in}}%
\pgfpathmoveto{\pgfqpoint{0.895314in}{0.971875in}}%
\pgfpathlineto{\pgfqpoint{0.895314in}{0.971875in}}%
\pgfpathlineto{\pgfqpoint{0.895314in}{1.066250in}}%
\pgfpathlineto{\pgfqpoint{1.040623in}{1.066250in}}%
\pgfpathlineto{\pgfqpoint{1.040623in}{0.971875in}}%
\pgfpathmoveto{\pgfqpoint{0.895314in}{1.066250in}}%
\pgfpathlineto{\pgfqpoint{0.895314in}{1.066250in}}%
\pgfpathlineto{\pgfqpoint{0.895314in}{1.160623in}}%
\pgfpathlineto{\pgfqpoint{1.040623in}{1.160623in}}%
\pgfpathlineto{\pgfqpoint{1.040623in}{1.066250in}}%
\pgfpathmoveto{\pgfqpoint{0.895314in}{1.160623in}}%
\pgfpathlineto{\pgfqpoint{0.895314in}{1.160623in}}%
\pgfpathlineto{\pgfqpoint{0.895314in}{1.254999in}}%
\pgfpathlineto{\pgfqpoint{1.040623in}{1.254999in}}%
\pgfpathlineto{\pgfqpoint{1.040623in}{1.160623in}}%
\pgfpathmoveto{\pgfqpoint{0.895314in}{1.254999in}}%
\pgfpathlineto{\pgfqpoint{0.895314in}{1.254999in}}%
\pgfpathlineto{\pgfqpoint{0.895314in}{1.349376in}}%
\pgfpathlineto{\pgfqpoint{1.040623in}{1.349376in}}%
\pgfpathlineto{\pgfqpoint{1.040623in}{1.254999in}}%
\pgfpathmoveto{\pgfqpoint{0.895314in}{1.349376in}}%
\pgfpathlineto{\pgfqpoint{0.895314in}{1.349376in}}%
\pgfpathlineto{\pgfqpoint{0.895314in}{1.443752in}}%
\pgfpathlineto{\pgfqpoint{1.040623in}{1.443752in}}%
\pgfpathlineto{\pgfqpoint{1.040623in}{1.349376in}}%
\pgfpathmoveto{\pgfqpoint{0.895314in}{1.443752in}}%
\pgfpathlineto{\pgfqpoint{0.895314in}{1.443752in}}%
\pgfpathlineto{\pgfqpoint{0.895314in}{1.538125in}}%
\pgfpathlineto{\pgfqpoint{1.040623in}{1.538125in}}%
\pgfpathlineto{\pgfqpoint{1.040623in}{1.443752in}}%
\pgfpathmoveto{\pgfqpoint{0.895314in}{1.538125in}}%
\pgfpathlineto{\pgfqpoint{0.895314in}{1.538125in}}%
\pgfpathlineto{\pgfqpoint{0.895314in}{1.632498in}}%
\pgfpathlineto{\pgfqpoint{1.040623in}{1.632498in}}%
\pgfpathlineto{\pgfqpoint{1.040623in}{1.538125in}}%
\pgfpathmoveto{\pgfqpoint{0.895314in}{1.632498in}}%
\pgfpathlineto{\pgfqpoint{0.895314in}{1.632498in}}%
\pgfpathlineto{\pgfqpoint{0.895314in}{1.726874in}}%
\pgfpathlineto{\pgfqpoint{1.040623in}{1.726874in}}%
\pgfpathlineto{\pgfqpoint{1.040623in}{1.632498in}}%
\pgfpathmoveto{\pgfqpoint{0.895314in}{1.726874in}}%
\pgfpathlineto{\pgfqpoint{0.895314in}{1.726874in}}%
\pgfpathlineto{\pgfqpoint{0.895314in}{1.821250in}}%
\pgfpathlineto{\pgfqpoint{1.040623in}{1.821250in}}%
\pgfpathlineto{\pgfqpoint{1.040623in}{1.726874in}}%
\pgfpathmoveto{\pgfqpoint{0.895314in}{1.821250in}}%
\pgfpathlineto{\pgfqpoint{0.895314in}{1.821250in}}%
\pgfpathlineto{\pgfqpoint{0.895314in}{1.915626in}}%
\pgfpathlineto{\pgfqpoint{1.040623in}{1.915626in}}%
\pgfpathlineto{\pgfqpoint{1.040623in}{1.821250in}}%
\pgfpathmoveto{\pgfqpoint{0.895314in}{1.915626in}}%
\pgfpathlineto{\pgfqpoint{0.895314in}{1.915626in}}%
\pgfpathlineto{\pgfqpoint{0.895314in}{2.009997in}}%
\pgfpathlineto{\pgfqpoint{1.040623in}{2.009997in}}%
\pgfpathlineto{\pgfqpoint{1.040623in}{1.915626in}}%
\pgfpathmoveto{\pgfqpoint{0.895314in}{2.009997in}}%
\pgfpathlineto{\pgfqpoint{0.895314in}{2.009997in}}%
\pgfpathlineto{\pgfqpoint{0.895314in}{2.104378in}}%
\pgfpathlineto{\pgfqpoint{1.040623in}{2.104378in}}%
\pgfpathlineto{\pgfqpoint{1.040623in}{2.009997in}}%
\pgfpathmoveto{\pgfqpoint{0.895314in}{2.104378in}}%
\pgfpathlineto{\pgfqpoint{0.895314in}{2.104378in}}%
\pgfpathlineto{\pgfqpoint{0.895314in}{2.198752in}}%
\pgfpathlineto{\pgfqpoint{1.040623in}{2.198752in}}%
\pgfpathlineto{\pgfqpoint{1.040623in}{2.104378in}}%
\pgfpathmoveto{\pgfqpoint{0.895314in}{2.198752in}}%
\pgfpathlineto{\pgfqpoint{0.895314in}{2.198752in}}%
\pgfpathlineto{\pgfqpoint{0.895314in}{2.293123in}}%
\pgfpathlineto{\pgfqpoint{1.040623in}{2.293123in}}%
\pgfpathlineto{\pgfqpoint{1.040623in}{2.198752in}}%
\pgfpathmoveto{\pgfqpoint{0.895314in}{2.293123in}}%
\pgfpathlineto{\pgfqpoint{0.895314in}{2.293123in}}%
\pgfpathlineto{\pgfqpoint{0.895314in}{2.387497in}}%
\pgfpathlineto{\pgfqpoint{1.040623in}{2.387497in}}%
\pgfpathlineto{\pgfqpoint{1.040623in}{2.293123in}}%
\pgfpathmoveto{\pgfqpoint{0.895314in}{2.387497in}}%
\pgfpathlineto{\pgfqpoint{0.895314in}{2.387497in}}%
\pgfpathlineto{\pgfqpoint{0.895314in}{2.481873in}}%
\pgfpathlineto{\pgfqpoint{1.040623in}{2.481873in}}%
\pgfpathlineto{\pgfqpoint{1.040623in}{2.387497in}}%
\pgfpathmoveto{\pgfqpoint{0.895314in}{2.481873in}}%
\pgfpathlineto{\pgfqpoint{0.895314in}{2.481873in}}%
\pgfpathlineto{\pgfqpoint{0.895314in}{2.576252in}}%
\pgfpathlineto{\pgfqpoint{1.040623in}{2.576252in}}%
\pgfpathlineto{\pgfqpoint{1.040623in}{2.481873in}}%
\pgfpathmoveto{\pgfqpoint{0.895314in}{2.576252in}}%
\pgfpathlineto{\pgfqpoint{0.895314in}{2.576252in}}%
\pgfpathlineto{\pgfqpoint{0.895314in}{2.670624in}}%
\pgfpathlineto{\pgfqpoint{1.040623in}{2.670624in}}%
\pgfpathlineto{\pgfqpoint{1.040623in}{2.576252in}}%
\pgfpathmoveto{\pgfqpoint{0.895314in}{2.670624in}}%
\pgfpathlineto{\pgfqpoint{0.895314in}{2.670624in}}%
\pgfpathlineto{\pgfqpoint{0.895314in}{2.765002in}}%
\pgfpathlineto{\pgfqpoint{1.040623in}{2.765002in}}%
\pgfpathlineto{\pgfqpoint{1.040623in}{2.670624in}}%
\pgfpathmoveto{\pgfqpoint{0.895314in}{2.765002in}}%
\pgfpathlineto{\pgfqpoint{0.895314in}{2.765002in}}%
\pgfpathlineto{\pgfqpoint{0.895314in}{2.859376in}}%
\pgfpathlineto{\pgfqpoint{1.040623in}{2.859376in}}%
\pgfpathlineto{\pgfqpoint{1.040623in}{2.765002in}}%
\pgfpathmoveto{\pgfqpoint{0.895314in}{2.859376in}}%
\pgfpathlineto{\pgfqpoint{0.895314in}{2.859376in}}%
\pgfpathlineto{\pgfqpoint{0.895314in}{2.953748in}}%
\pgfpathlineto{\pgfqpoint{1.040623in}{2.953748in}}%
\pgfpathlineto{\pgfqpoint{1.040623in}{2.859376in}}%
\pgfpathmoveto{\pgfqpoint{0.895314in}{2.953748in}}%
\pgfpathlineto{\pgfqpoint{0.895314in}{2.953748in}}%
\pgfpathlineto{\pgfqpoint{0.895314in}{3.048126in}}%
\pgfpathlineto{\pgfqpoint{1.040623in}{3.048126in}}%
\pgfpathlineto{\pgfqpoint{1.040623in}{2.953748in}}%
\pgfpathmoveto{\pgfqpoint{0.895314in}{3.048126in}}%
\pgfpathlineto{\pgfqpoint{0.895314in}{3.048126in}}%
\pgfpathlineto{\pgfqpoint{0.895314in}{3.142501in}}%
\pgfpathlineto{\pgfqpoint{1.040623in}{3.142501in}}%
\pgfpathlineto{\pgfqpoint{1.040623in}{3.048126in}}%
\pgfpathmoveto{\pgfqpoint{0.895314in}{3.142501in}}%
\pgfpathlineto{\pgfqpoint{0.895314in}{3.142501in}}%
\pgfpathlineto{\pgfqpoint{0.895314in}{3.236873in}}%
\pgfpathlineto{\pgfqpoint{1.040623in}{3.236873in}}%
\pgfpathlineto{\pgfqpoint{1.040623in}{3.142501in}}%
\pgfpathmoveto{\pgfqpoint{0.895314in}{3.236873in}}%
\pgfpathlineto{\pgfqpoint{0.895314in}{3.236873in}}%
\pgfpathlineto{\pgfqpoint{0.895314in}{3.331249in}}%
\pgfpathlineto{\pgfqpoint{1.040623in}{3.331249in}}%
\pgfpathlineto{\pgfqpoint{1.040623in}{3.236873in}}%
\pgfpathmoveto{\pgfqpoint{0.895314in}{3.331249in}}%
\pgfpathlineto{\pgfqpoint{0.895314in}{3.331249in}}%
\pgfpathlineto{\pgfqpoint{0.895314in}{3.425625in}}%
\pgfpathlineto{\pgfqpoint{1.040623in}{3.425625in}}%
\pgfpathlineto{\pgfqpoint{1.040623in}{3.331249in}}%
\pgfpathmoveto{\pgfqpoint{0.895314in}{3.425625in}}%
\pgfpathlineto{\pgfqpoint{0.895314in}{3.425625in}}%
\pgfpathlineto{\pgfqpoint{0.895314in}{3.519999in}}%
\pgfpathlineto{\pgfqpoint{1.040623in}{3.519999in}}%
\pgfpathlineto{\pgfqpoint{1.040623in}{3.425625in}}%
\pgfpathmoveto{\pgfqpoint{1.040623in}{0.877498in}}%
\pgfpathlineto{\pgfqpoint{1.040623in}{0.877498in}}%
\pgfpathlineto{\pgfqpoint{1.040623in}{0.971875in}}%
\pgfpathlineto{\pgfqpoint{1.185936in}{0.971875in}}%
\pgfpathlineto{\pgfqpoint{1.185936in}{0.877498in}}%
\pgfpathmoveto{\pgfqpoint{1.040623in}{0.971875in}}%
\pgfpathlineto{\pgfqpoint{1.040623in}{0.971875in}}%
\pgfpathlineto{\pgfqpoint{1.040623in}{1.066250in}}%
\pgfpathlineto{\pgfqpoint{1.185936in}{1.066250in}}%
\pgfpathlineto{\pgfqpoint{1.185936in}{0.971875in}}%
\pgfpathmoveto{\pgfqpoint{1.040623in}{1.066250in}}%
\pgfpathlineto{\pgfqpoint{1.040623in}{1.066250in}}%
\pgfpathlineto{\pgfqpoint{1.040623in}{1.160623in}}%
\pgfpathlineto{\pgfqpoint{1.185936in}{1.160623in}}%
\pgfpathlineto{\pgfqpoint{1.185936in}{1.066250in}}%
\pgfpathmoveto{\pgfqpoint{1.040623in}{1.160623in}}%
\pgfpathlineto{\pgfqpoint{1.040623in}{1.160623in}}%
\pgfpathlineto{\pgfqpoint{1.040623in}{1.254999in}}%
\pgfpathlineto{\pgfqpoint{1.185936in}{1.254999in}}%
\pgfpathlineto{\pgfqpoint{1.185936in}{1.160623in}}%
\pgfpathmoveto{\pgfqpoint{1.040623in}{1.254999in}}%
\pgfpathlineto{\pgfqpoint{1.040623in}{1.254999in}}%
\pgfpathlineto{\pgfqpoint{1.040623in}{1.349376in}}%
\pgfpathlineto{\pgfqpoint{1.185936in}{1.349376in}}%
\pgfpathlineto{\pgfqpoint{1.185936in}{1.254999in}}%
\pgfpathmoveto{\pgfqpoint{1.040623in}{1.349376in}}%
\pgfpathlineto{\pgfqpoint{1.040623in}{1.349376in}}%
\pgfpathlineto{\pgfqpoint{1.040623in}{1.443752in}}%
\pgfpathlineto{\pgfqpoint{1.185936in}{1.443752in}}%
\pgfpathlineto{\pgfqpoint{1.185936in}{1.349376in}}%
\pgfpathmoveto{\pgfqpoint{1.040623in}{1.443752in}}%
\pgfpathlineto{\pgfqpoint{1.040623in}{1.443752in}}%
\pgfpathlineto{\pgfqpoint{1.040623in}{1.538125in}}%
\pgfpathlineto{\pgfqpoint{1.185936in}{1.538125in}}%
\pgfpathlineto{\pgfqpoint{1.185936in}{1.443752in}}%
\pgfpathmoveto{\pgfqpoint{1.040623in}{1.538125in}}%
\pgfpathlineto{\pgfqpoint{1.040623in}{1.538125in}}%
\pgfpathlineto{\pgfqpoint{1.040623in}{1.632498in}}%
\pgfpathlineto{\pgfqpoint{1.185936in}{1.632498in}}%
\pgfpathlineto{\pgfqpoint{1.185936in}{1.538125in}}%
\pgfpathmoveto{\pgfqpoint{1.040623in}{1.632498in}}%
\pgfpathlineto{\pgfqpoint{1.040623in}{1.632498in}}%
\pgfpathlineto{\pgfqpoint{1.040623in}{1.726874in}}%
\pgfpathlineto{\pgfqpoint{1.185936in}{1.726874in}}%
\pgfpathlineto{\pgfqpoint{1.185936in}{1.632498in}}%
\pgfpathmoveto{\pgfqpoint{1.040623in}{1.726874in}}%
\pgfpathlineto{\pgfqpoint{1.040623in}{1.726874in}}%
\pgfpathlineto{\pgfqpoint{1.040623in}{1.821250in}}%
\pgfpathlineto{\pgfqpoint{1.185936in}{1.821250in}}%
\pgfpathlineto{\pgfqpoint{1.185936in}{1.726874in}}%
\pgfpathmoveto{\pgfqpoint{1.040623in}{1.821250in}}%
\pgfpathlineto{\pgfqpoint{1.040623in}{1.821250in}}%
\pgfpathlineto{\pgfqpoint{1.040623in}{1.915626in}}%
\pgfpathlineto{\pgfqpoint{1.185936in}{1.915626in}}%
\pgfpathlineto{\pgfqpoint{1.185936in}{1.821250in}}%
\pgfpathmoveto{\pgfqpoint{1.040623in}{1.915626in}}%
\pgfpathlineto{\pgfqpoint{1.040623in}{1.915626in}}%
\pgfpathlineto{\pgfqpoint{1.040623in}{2.009997in}}%
\pgfpathlineto{\pgfqpoint{1.185936in}{2.009997in}}%
\pgfpathlineto{\pgfqpoint{1.185936in}{1.915626in}}%
\pgfpathmoveto{\pgfqpoint{1.040623in}{2.009997in}}%
\pgfpathlineto{\pgfqpoint{1.040623in}{2.009997in}}%
\pgfpathlineto{\pgfqpoint{1.040623in}{2.104378in}}%
\pgfpathlineto{\pgfqpoint{1.185936in}{2.104378in}}%
\pgfpathlineto{\pgfqpoint{1.185936in}{2.009997in}}%
\pgfpathmoveto{\pgfqpoint{1.040623in}{2.104378in}}%
\pgfpathlineto{\pgfqpoint{1.040623in}{2.104378in}}%
\pgfpathlineto{\pgfqpoint{1.040623in}{2.198752in}}%
\pgfpathlineto{\pgfqpoint{1.185936in}{2.198752in}}%
\pgfpathlineto{\pgfqpoint{1.185936in}{2.104378in}}%
\pgfpathmoveto{\pgfqpoint{1.040623in}{2.198752in}}%
\pgfpathlineto{\pgfqpoint{1.040623in}{2.198752in}}%
\pgfpathlineto{\pgfqpoint{1.040623in}{2.293123in}}%
\pgfpathlineto{\pgfqpoint{1.185936in}{2.293123in}}%
\pgfpathlineto{\pgfqpoint{1.185936in}{2.198752in}}%
\pgfpathmoveto{\pgfqpoint{1.040623in}{2.293123in}}%
\pgfpathlineto{\pgfqpoint{1.040623in}{2.293123in}}%
\pgfpathlineto{\pgfqpoint{1.040623in}{2.387497in}}%
\pgfpathlineto{\pgfqpoint{1.185936in}{2.387497in}}%
\pgfpathlineto{\pgfqpoint{1.185936in}{2.293123in}}%
\pgfpathmoveto{\pgfqpoint{1.040623in}{2.387497in}}%
\pgfpathlineto{\pgfqpoint{1.040623in}{2.387497in}}%
\pgfpathlineto{\pgfqpoint{1.040623in}{2.481873in}}%
\pgfpathlineto{\pgfqpoint{1.185936in}{2.481873in}}%
\pgfpathlineto{\pgfqpoint{1.185936in}{2.387497in}}%
\pgfpathmoveto{\pgfqpoint{1.040623in}{2.481873in}}%
\pgfpathlineto{\pgfqpoint{1.040623in}{2.481873in}}%
\pgfpathlineto{\pgfqpoint{1.040623in}{2.576252in}}%
\pgfpathlineto{\pgfqpoint{1.185936in}{2.576252in}}%
\pgfpathlineto{\pgfqpoint{1.185936in}{2.481873in}}%
\pgfpathmoveto{\pgfqpoint{1.040623in}{2.576252in}}%
\pgfpathlineto{\pgfqpoint{1.040623in}{2.576252in}}%
\pgfpathlineto{\pgfqpoint{1.040623in}{2.670624in}}%
\pgfpathlineto{\pgfqpoint{1.185936in}{2.670624in}}%
\pgfpathlineto{\pgfqpoint{1.185936in}{2.576252in}}%
\pgfpathmoveto{\pgfqpoint{1.040623in}{2.670624in}}%
\pgfpathlineto{\pgfqpoint{1.040623in}{2.670624in}}%
\pgfpathlineto{\pgfqpoint{1.040623in}{2.765002in}}%
\pgfpathlineto{\pgfqpoint{1.185936in}{2.765002in}}%
\pgfpathlineto{\pgfqpoint{1.185936in}{2.670624in}}%
\pgfpathmoveto{\pgfqpoint{1.040623in}{2.765002in}}%
\pgfpathlineto{\pgfqpoint{1.040623in}{2.765002in}}%
\pgfpathlineto{\pgfqpoint{1.040623in}{2.859376in}}%
\pgfpathlineto{\pgfqpoint{1.185936in}{2.859376in}}%
\pgfpathlineto{\pgfqpoint{1.185936in}{2.765002in}}%
\pgfpathmoveto{\pgfqpoint{1.040623in}{2.859376in}}%
\pgfpathlineto{\pgfqpoint{1.040623in}{2.859376in}}%
\pgfpathlineto{\pgfqpoint{1.040623in}{2.953748in}}%
\pgfpathlineto{\pgfqpoint{1.185936in}{2.953748in}}%
\pgfpathlineto{\pgfqpoint{1.185936in}{2.859376in}}%
\pgfpathmoveto{\pgfqpoint{1.040623in}{2.953748in}}%
\pgfpathlineto{\pgfqpoint{1.040623in}{2.953748in}}%
\pgfpathlineto{\pgfqpoint{1.040623in}{3.048126in}}%
\pgfpathlineto{\pgfqpoint{1.185936in}{3.048126in}}%
\pgfpathlineto{\pgfqpoint{1.185936in}{2.953748in}}%
\pgfpathmoveto{\pgfqpoint{1.040623in}{3.048126in}}%
\pgfpathlineto{\pgfqpoint{1.040623in}{3.048126in}}%
\pgfpathlineto{\pgfqpoint{1.040623in}{3.142501in}}%
\pgfpathlineto{\pgfqpoint{1.185936in}{3.142501in}}%
\pgfpathlineto{\pgfqpoint{1.185936in}{3.048126in}}%
\pgfpathmoveto{\pgfqpoint{1.040623in}{3.142501in}}%
\pgfpathlineto{\pgfqpoint{1.040623in}{3.142501in}}%
\pgfpathlineto{\pgfqpoint{1.040623in}{3.236873in}}%
\pgfpathlineto{\pgfqpoint{1.185936in}{3.236873in}}%
\pgfpathlineto{\pgfqpoint{1.185936in}{3.142501in}}%
\pgfpathmoveto{\pgfqpoint{1.040623in}{3.236873in}}%
\pgfpathlineto{\pgfqpoint{1.040623in}{3.236873in}}%
\pgfpathlineto{\pgfqpoint{1.040623in}{3.331249in}}%
\pgfpathlineto{\pgfqpoint{1.185936in}{3.331249in}}%
\pgfpathlineto{\pgfqpoint{1.185936in}{3.236873in}}%
\pgfpathmoveto{\pgfqpoint{1.040623in}{3.331249in}}%
\pgfpathlineto{\pgfqpoint{1.040623in}{3.331249in}}%
\pgfpathlineto{\pgfqpoint{1.040623in}{3.425625in}}%
\pgfpathlineto{\pgfqpoint{1.185936in}{3.425625in}}%
\pgfpathlineto{\pgfqpoint{1.185936in}{3.331249in}}%
\pgfpathmoveto{\pgfqpoint{1.040623in}{3.425625in}}%
\pgfpathlineto{\pgfqpoint{1.040623in}{3.425625in}}%
\pgfpathlineto{\pgfqpoint{1.040623in}{3.519999in}}%
\pgfpathlineto{\pgfqpoint{1.185936in}{3.519999in}}%
\pgfpathlineto{\pgfqpoint{1.185936in}{3.425625in}}%
\pgfpathmoveto{\pgfqpoint{1.185936in}{0.971875in}}%
\pgfpathlineto{\pgfqpoint{1.185936in}{0.971875in}}%
\pgfpathlineto{\pgfqpoint{1.185936in}{1.066250in}}%
\pgfpathlineto{\pgfqpoint{1.331249in}{1.066250in}}%
\pgfpathlineto{\pgfqpoint{1.331249in}{0.971875in}}%
\pgfpathmoveto{\pgfqpoint{1.185936in}{1.066250in}}%
\pgfpathlineto{\pgfqpoint{1.185936in}{1.066250in}}%
\pgfpathlineto{\pgfqpoint{1.185936in}{1.160623in}}%
\pgfpathlineto{\pgfqpoint{1.331249in}{1.160623in}}%
\pgfpathlineto{\pgfqpoint{1.331249in}{1.066250in}}%
\pgfpathmoveto{\pgfqpoint{1.185936in}{1.160623in}}%
\pgfpathlineto{\pgfqpoint{1.185936in}{1.160623in}}%
\pgfpathlineto{\pgfqpoint{1.185936in}{1.254999in}}%
\pgfpathlineto{\pgfqpoint{1.331249in}{1.254999in}}%
\pgfpathlineto{\pgfqpoint{1.331249in}{1.160623in}}%
\pgfpathmoveto{\pgfqpoint{1.185936in}{1.254999in}}%
\pgfpathlineto{\pgfqpoint{1.185936in}{1.254999in}}%
\pgfpathlineto{\pgfqpoint{1.185936in}{1.349376in}}%
\pgfpathlineto{\pgfqpoint{1.331249in}{1.349376in}}%
\pgfpathlineto{\pgfqpoint{1.331249in}{1.254999in}}%
\pgfpathmoveto{\pgfqpoint{1.185936in}{1.349376in}}%
\pgfpathlineto{\pgfqpoint{1.185936in}{1.349376in}}%
\pgfpathlineto{\pgfqpoint{1.185936in}{1.443752in}}%
\pgfpathlineto{\pgfqpoint{1.331249in}{1.443752in}}%
\pgfpathlineto{\pgfqpoint{1.331249in}{1.349376in}}%
\pgfpathmoveto{\pgfqpoint{1.185936in}{1.443752in}}%
\pgfpathlineto{\pgfqpoint{1.185936in}{1.443752in}}%
\pgfpathlineto{\pgfqpoint{1.185936in}{1.538125in}}%
\pgfpathlineto{\pgfqpoint{1.331249in}{1.538125in}}%
\pgfpathlineto{\pgfqpoint{1.331249in}{1.443752in}}%
\pgfpathmoveto{\pgfqpoint{1.185936in}{1.538125in}}%
\pgfpathlineto{\pgfqpoint{1.185936in}{1.538125in}}%
\pgfpathlineto{\pgfqpoint{1.185936in}{1.632498in}}%
\pgfpathlineto{\pgfqpoint{1.331249in}{1.632498in}}%
\pgfpathlineto{\pgfqpoint{1.331249in}{1.538125in}}%
\pgfpathmoveto{\pgfqpoint{1.185936in}{1.632498in}}%
\pgfpathlineto{\pgfqpoint{1.185936in}{1.632498in}}%
\pgfpathlineto{\pgfqpoint{1.185936in}{1.726874in}}%
\pgfpathlineto{\pgfqpoint{1.331249in}{1.726874in}}%
\pgfpathlineto{\pgfqpoint{1.331249in}{1.632498in}}%
\pgfpathmoveto{\pgfqpoint{1.185936in}{1.726874in}}%
\pgfpathlineto{\pgfqpoint{1.185936in}{1.726874in}}%
\pgfpathlineto{\pgfqpoint{1.185936in}{1.821250in}}%
\pgfpathlineto{\pgfqpoint{1.331249in}{1.821250in}}%
\pgfpathlineto{\pgfqpoint{1.331249in}{1.726874in}}%
\pgfpathmoveto{\pgfqpoint{1.185936in}{1.821250in}}%
\pgfpathlineto{\pgfqpoint{1.185936in}{1.821250in}}%
\pgfpathlineto{\pgfqpoint{1.185936in}{1.915626in}}%
\pgfpathlineto{\pgfqpoint{1.331249in}{1.915626in}}%
\pgfpathlineto{\pgfqpoint{1.331249in}{1.821250in}}%
\pgfpathmoveto{\pgfqpoint{1.185936in}{1.915626in}}%
\pgfpathlineto{\pgfqpoint{1.185936in}{1.915626in}}%
\pgfpathlineto{\pgfqpoint{1.185936in}{2.009997in}}%
\pgfpathlineto{\pgfqpoint{1.331249in}{2.009997in}}%
\pgfpathlineto{\pgfqpoint{1.331249in}{1.915626in}}%
\pgfpathmoveto{\pgfqpoint{1.185936in}{2.009997in}}%
\pgfpathlineto{\pgfqpoint{1.185936in}{2.009997in}}%
\pgfpathlineto{\pgfqpoint{1.185936in}{2.104378in}}%
\pgfpathlineto{\pgfqpoint{1.331249in}{2.104378in}}%
\pgfpathlineto{\pgfqpoint{1.331249in}{2.009997in}}%
\pgfpathmoveto{\pgfqpoint{1.185936in}{2.104378in}}%
\pgfpathlineto{\pgfqpoint{1.185936in}{2.104378in}}%
\pgfpathlineto{\pgfqpoint{1.185936in}{2.198752in}}%
\pgfpathlineto{\pgfqpoint{1.331249in}{2.198752in}}%
\pgfpathlineto{\pgfqpoint{1.331249in}{2.104378in}}%
\pgfpathmoveto{\pgfqpoint{1.185936in}{2.198752in}}%
\pgfpathlineto{\pgfqpoint{1.185936in}{2.198752in}}%
\pgfpathlineto{\pgfqpoint{1.185936in}{2.293123in}}%
\pgfpathlineto{\pgfqpoint{1.331249in}{2.293123in}}%
\pgfpathlineto{\pgfqpoint{1.331249in}{2.198752in}}%
\pgfpathmoveto{\pgfqpoint{1.185936in}{2.293123in}}%
\pgfpathlineto{\pgfqpoint{1.185936in}{2.293123in}}%
\pgfpathlineto{\pgfqpoint{1.185936in}{2.387497in}}%
\pgfpathlineto{\pgfqpoint{1.331249in}{2.387497in}}%
\pgfpathlineto{\pgfqpoint{1.331249in}{2.293123in}}%
\pgfpathmoveto{\pgfqpoint{1.185936in}{2.387497in}}%
\pgfpathlineto{\pgfqpoint{1.185936in}{2.387497in}}%
\pgfpathlineto{\pgfqpoint{1.185936in}{2.481873in}}%
\pgfpathlineto{\pgfqpoint{1.331249in}{2.481873in}}%
\pgfpathlineto{\pgfqpoint{1.331249in}{2.387497in}}%
\pgfpathmoveto{\pgfqpoint{1.185936in}{2.481873in}}%
\pgfpathlineto{\pgfqpoint{1.185936in}{2.481873in}}%
\pgfpathlineto{\pgfqpoint{1.185936in}{2.576252in}}%
\pgfpathlineto{\pgfqpoint{1.331249in}{2.576252in}}%
\pgfpathlineto{\pgfqpoint{1.331249in}{2.481873in}}%
\pgfpathmoveto{\pgfqpoint{1.185936in}{2.576252in}}%
\pgfpathlineto{\pgfqpoint{1.185936in}{2.576252in}}%
\pgfpathlineto{\pgfqpoint{1.185936in}{2.670624in}}%
\pgfpathlineto{\pgfqpoint{1.331249in}{2.670624in}}%
\pgfpathlineto{\pgfqpoint{1.331249in}{2.576252in}}%
\pgfpathmoveto{\pgfqpoint{1.185936in}{2.670624in}}%
\pgfpathlineto{\pgfqpoint{1.185936in}{2.670624in}}%
\pgfpathlineto{\pgfqpoint{1.185936in}{2.765002in}}%
\pgfpathlineto{\pgfqpoint{1.331249in}{2.765002in}}%
\pgfpathlineto{\pgfqpoint{1.331249in}{2.670624in}}%
\pgfpathmoveto{\pgfqpoint{1.185936in}{2.765002in}}%
\pgfpathlineto{\pgfqpoint{1.185936in}{2.765002in}}%
\pgfpathlineto{\pgfqpoint{1.185936in}{2.859376in}}%
\pgfpathlineto{\pgfqpoint{1.331249in}{2.859376in}}%
\pgfpathlineto{\pgfqpoint{1.331249in}{2.765002in}}%
\pgfpathmoveto{\pgfqpoint{1.185936in}{2.859376in}}%
\pgfpathlineto{\pgfqpoint{1.185936in}{2.859376in}}%
\pgfpathlineto{\pgfqpoint{1.185936in}{2.953748in}}%
\pgfpathlineto{\pgfqpoint{1.331249in}{2.953748in}}%
\pgfpathlineto{\pgfqpoint{1.331249in}{2.859376in}}%
\pgfpathmoveto{\pgfqpoint{1.185936in}{2.953748in}}%
\pgfpathlineto{\pgfqpoint{1.185936in}{2.953748in}}%
\pgfpathlineto{\pgfqpoint{1.185936in}{3.048126in}}%
\pgfpathlineto{\pgfqpoint{1.331249in}{3.048126in}}%
\pgfpathlineto{\pgfqpoint{1.331249in}{2.953748in}}%
\pgfpathmoveto{\pgfqpoint{1.185936in}{3.048126in}}%
\pgfpathlineto{\pgfqpoint{1.185936in}{3.048126in}}%
\pgfpathlineto{\pgfqpoint{1.185936in}{3.142501in}}%
\pgfpathlineto{\pgfqpoint{1.331249in}{3.142501in}}%
\pgfpathlineto{\pgfqpoint{1.331249in}{3.048126in}}%
\pgfpathmoveto{\pgfqpoint{1.185936in}{3.142501in}}%
\pgfpathlineto{\pgfqpoint{1.185936in}{3.142501in}}%
\pgfpathlineto{\pgfqpoint{1.185936in}{3.236873in}}%
\pgfpathlineto{\pgfqpoint{1.331249in}{3.236873in}}%
\pgfpathlineto{\pgfqpoint{1.331249in}{3.142501in}}%
\pgfpathmoveto{\pgfqpoint{1.185936in}{3.236873in}}%
\pgfpathlineto{\pgfqpoint{1.185936in}{3.236873in}}%
\pgfpathlineto{\pgfqpoint{1.185936in}{3.331249in}}%
\pgfpathlineto{\pgfqpoint{1.331249in}{3.331249in}}%
\pgfpathlineto{\pgfqpoint{1.331249in}{3.236873in}}%
\pgfpathmoveto{\pgfqpoint{1.185936in}{3.331249in}}%
\pgfpathlineto{\pgfqpoint{1.185936in}{3.331249in}}%
\pgfpathlineto{\pgfqpoint{1.185936in}{3.425625in}}%
\pgfpathlineto{\pgfqpoint{1.331249in}{3.425625in}}%
\pgfpathlineto{\pgfqpoint{1.331249in}{3.331249in}}%
\pgfpathmoveto{\pgfqpoint{1.185936in}{3.425625in}}%
\pgfpathlineto{\pgfqpoint{1.185936in}{3.425625in}}%
\pgfpathlineto{\pgfqpoint{1.185936in}{3.519999in}}%
\pgfpathlineto{\pgfqpoint{1.331249in}{3.519999in}}%
\pgfpathlineto{\pgfqpoint{1.331249in}{3.425625in}}%
\pgfpathmoveto{\pgfqpoint{1.331249in}{0.971875in}}%
\pgfpathlineto{\pgfqpoint{1.331249in}{0.971875in}}%
\pgfpathlineto{\pgfqpoint{1.331249in}{1.066250in}}%
\pgfpathlineto{\pgfqpoint{1.476559in}{1.066250in}}%
\pgfpathlineto{\pgfqpoint{1.476559in}{0.971875in}}%
\pgfpathmoveto{\pgfqpoint{1.331249in}{1.066250in}}%
\pgfpathlineto{\pgfqpoint{1.331249in}{1.066250in}}%
\pgfpathlineto{\pgfqpoint{1.331249in}{1.160623in}}%
\pgfpathlineto{\pgfqpoint{1.476559in}{1.160623in}}%
\pgfpathlineto{\pgfqpoint{1.476559in}{1.066250in}}%
\pgfpathmoveto{\pgfqpoint{1.331249in}{1.160623in}}%
\pgfpathlineto{\pgfqpoint{1.331249in}{1.160623in}}%
\pgfpathlineto{\pgfqpoint{1.331249in}{1.254999in}}%
\pgfpathlineto{\pgfqpoint{1.476559in}{1.254999in}}%
\pgfpathlineto{\pgfqpoint{1.476559in}{1.160623in}}%
\pgfpathmoveto{\pgfqpoint{1.331249in}{1.254999in}}%
\pgfpathlineto{\pgfqpoint{1.331249in}{1.254999in}}%
\pgfpathlineto{\pgfqpoint{1.331249in}{1.349376in}}%
\pgfpathlineto{\pgfqpoint{1.476559in}{1.349376in}}%
\pgfpathlineto{\pgfqpoint{1.476559in}{1.254999in}}%
\pgfpathmoveto{\pgfqpoint{1.331249in}{1.349376in}}%
\pgfpathlineto{\pgfqpoint{1.331249in}{1.349376in}}%
\pgfpathlineto{\pgfqpoint{1.331249in}{1.443752in}}%
\pgfpathlineto{\pgfqpoint{1.476559in}{1.443752in}}%
\pgfpathlineto{\pgfqpoint{1.476559in}{1.349376in}}%
\pgfpathmoveto{\pgfqpoint{1.331249in}{1.443752in}}%
\pgfpathlineto{\pgfqpoint{1.331249in}{1.443752in}}%
\pgfpathlineto{\pgfqpoint{1.331249in}{1.538125in}}%
\pgfpathlineto{\pgfqpoint{1.476559in}{1.538125in}}%
\pgfpathlineto{\pgfqpoint{1.476559in}{1.443752in}}%
\pgfpathmoveto{\pgfqpoint{1.331249in}{1.538125in}}%
\pgfpathlineto{\pgfqpoint{1.331249in}{1.538125in}}%
\pgfpathlineto{\pgfqpoint{1.331249in}{1.632498in}}%
\pgfpathlineto{\pgfqpoint{1.476559in}{1.632498in}}%
\pgfpathlineto{\pgfqpoint{1.476559in}{1.538125in}}%
\pgfpathmoveto{\pgfqpoint{1.331249in}{1.632498in}}%
\pgfpathlineto{\pgfqpoint{1.331249in}{1.632498in}}%
\pgfpathlineto{\pgfqpoint{1.331249in}{1.726874in}}%
\pgfpathlineto{\pgfqpoint{1.476559in}{1.726874in}}%
\pgfpathlineto{\pgfqpoint{1.476559in}{1.632498in}}%
\pgfpathmoveto{\pgfqpoint{1.331249in}{1.726874in}}%
\pgfpathlineto{\pgfqpoint{1.331249in}{1.726874in}}%
\pgfpathlineto{\pgfqpoint{1.331249in}{1.821250in}}%
\pgfpathlineto{\pgfqpoint{1.476559in}{1.821250in}}%
\pgfpathlineto{\pgfqpoint{1.476559in}{1.726874in}}%
\pgfpathmoveto{\pgfqpoint{1.331249in}{1.821250in}}%
\pgfpathlineto{\pgfqpoint{1.331249in}{1.821250in}}%
\pgfpathlineto{\pgfqpoint{1.331249in}{1.915626in}}%
\pgfpathlineto{\pgfqpoint{1.476559in}{1.915626in}}%
\pgfpathlineto{\pgfqpoint{1.476559in}{1.821250in}}%
\pgfpathmoveto{\pgfqpoint{1.331249in}{1.915626in}}%
\pgfpathlineto{\pgfqpoint{1.331249in}{1.915626in}}%
\pgfpathlineto{\pgfqpoint{1.331249in}{2.009997in}}%
\pgfpathlineto{\pgfqpoint{1.476559in}{2.009997in}}%
\pgfpathlineto{\pgfqpoint{1.476559in}{1.915626in}}%
\pgfpathmoveto{\pgfqpoint{1.331249in}{2.009997in}}%
\pgfpathlineto{\pgfqpoint{1.331249in}{2.009997in}}%
\pgfpathlineto{\pgfqpoint{1.331249in}{2.104378in}}%
\pgfpathlineto{\pgfqpoint{1.476559in}{2.104378in}}%
\pgfpathlineto{\pgfqpoint{1.476559in}{2.009997in}}%
\pgfpathmoveto{\pgfqpoint{1.331249in}{2.104378in}}%
\pgfpathlineto{\pgfqpoint{1.331249in}{2.104378in}}%
\pgfpathlineto{\pgfqpoint{1.331249in}{2.198752in}}%
\pgfpathlineto{\pgfqpoint{1.476559in}{2.198752in}}%
\pgfpathlineto{\pgfqpoint{1.476559in}{2.104378in}}%
\pgfpathmoveto{\pgfqpoint{1.331249in}{2.198752in}}%
\pgfpathlineto{\pgfqpoint{1.331249in}{2.198752in}}%
\pgfpathlineto{\pgfqpoint{1.331249in}{2.293123in}}%
\pgfpathlineto{\pgfqpoint{1.476559in}{2.293123in}}%
\pgfpathlineto{\pgfqpoint{1.476559in}{2.198752in}}%
\pgfpathmoveto{\pgfqpoint{1.331249in}{2.293123in}}%
\pgfpathlineto{\pgfqpoint{1.331249in}{2.293123in}}%
\pgfpathlineto{\pgfqpoint{1.331249in}{2.387497in}}%
\pgfpathlineto{\pgfqpoint{1.476559in}{2.387497in}}%
\pgfpathlineto{\pgfqpoint{1.476559in}{2.293123in}}%
\pgfpathmoveto{\pgfqpoint{1.331249in}{2.387497in}}%
\pgfpathlineto{\pgfqpoint{1.331249in}{2.387497in}}%
\pgfpathlineto{\pgfqpoint{1.331249in}{2.481873in}}%
\pgfpathlineto{\pgfqpoint{1.476559in}{2.481873in}}%
\pgfpathlineto{\pgfqpoint{1.476559in}{2.387497in}}%
\pgfpathmoveto{\pgfqpoint{1.331249in}{2.481873in}}%
\pgfpathlineto{\pgfqpoint{1.331249in}{2.481873in}}%
\pgfpathlineto{\pgfqpoint{1.331249in}{2.576252in}}%
\pgfpathlineto{\pgfqpoint{1.476559in}{2.576252in}}%
\pgfpathlineto{\pgfqpoint{1.476559in}{2.481873in}}%
\pgfpathmoveto{\pgfqpoint{1.331249in}{2.576252in}}%
\pgfpathlineto{\pgfqpoint{1.331249in}{2.576252in}}%
\pgfpathlineto{\pgfqpoint{1.331249in}{2.670624in}}%
\pgfpathlineto{\pgfqpoint{1.476559in}{2.670624in}}%
\pgfpathlineto{\pgfqpoint{1.476559in}{2.576252in}}%
\pgfpathmoveto{\pgfqpoint{1.331249in}{2.670624in}}%
\pgfpathlineto{\pgfqpoint{1.331249in}{2.670624in}}%
\pgfpathlineto{\pgfqpoint{1.331249in}{2.765002in}}%
\pgfpathlineto{\pgfqpoint{1.476559in}{2.765002in}}%
\pgfpathlineto{\pgfqpoint{1.476559in}{2.670624in}}%
\pgfpathmoveto{\pgfqpoint{1.331249in}{2.765002in}}%
\pgfpathlineto{\pgfqpoint{1.331249in}{2.765002in}}%
\pgfpathlineto{\pgfqpoint{1.331249in}{2.859376in}}%
\pgfpathlineto{\pgfqpoint{1.476559in}{2.859376in}}%
\pgfpathlineto{\pgfqpoint{1.476559in}{2.765002in}}%
\pgfpathmoveto{\pgfqpoint{1.331249in}{2.859376in}}%
\pgfpathlineto{\pgfqpoint{1.331249in}{2.859376in}}%
\pgfpathlineto{\pgfqpoint{1.331249in}{2.953748in}}%
\pgfpathlineto{\pgfqpoint{1.476559in}{2.953748in}}%
\pgfpathlineto{\pgfqpoint{1.476559in}{2.859376in}}%
\pgfpathmoveto{\pgfqpoint{1.331249in}{2.953748in}}%
\pgfpathlineto{\pgfqpoint{1.331249in}{2.953748in}}%
\pgfpathlineto{\pgfqpoint{1.331249in}{3.048126in}}%
\pgfpathlineto{\pgfqpoint{1.476559in}{3.048126in}}%
\pgfpathlineto{\pgfqpoint{1.476559in}{2.953748in}}%
\pgfpathmoveto{\pgfqpoint{1.331249in}{3.048126in}}%
\pgfpathlineto{\pgfqpoint{1.331249in}{3.048126in}}%
\pgfpathlineto{\pgfqpoint{1.331249in}{3.142501in}}%
\pgfpathlineto{\pgfqpoint{1.476559in}{3.142501in}}%
\pgfpathlineto{\pgfqpoint{1.476559in}{3.048126in}}%
\pgfpathmoveto{\pgfqpoint{1.331249in}{3.142501in}}%
\pgfpathlineto{\pgfqpoint{1.331249in}{3.142501in}}%
\pgfpathlineto{\pgfqpoint{1.331249in}{3.236873in}}%
\pgfpathlineto{\pgfqpoint{1.476559in}{3.236873in}}%
\pgfpathlineto{\pgfqpoint{1.476559in}{3.142501in}}%
\pgfpathmoveto{\pgfqpoint{1.331249in}{3.236873in}}%
\pgfpathlineto{\pgfqpoint{1.331249in}{3.236873in}}%
\pgfpathlineto{\pgfqpoint{1.331249in}{3.331249in}}%
\pgfpathlineto{\pgfqpoint{1.476559in}{3.331249in}}%
\pgfpathlineto{\pgfqpoint{1.476559in}{3.236873in}}%
\pgfpathmoveto{\pgfqpoint{1.331249in}{3.331249in}}%
\pgfpathlineto{\pgfqpoint{1.331249in}{3.331249in}}%
\pgfpathlineto{\pgfqpoint{1.331249in}{3.425625in}}%
\pgfpathlineto{\pgfqpoint{1.476559in}{3.425625in}}%
\pgfpathlineto{\pgfqpoint{1.476559in}{3.331249in}}%
\pgfpathmoveto{\pgfqpoint{1.331249in}{3.425625in}}%
\pgfpathlineto{\pgfqpoint{1.331249in}{3.425625in}}%
\pgfpathlineto{\pgfqpoint{1.331249in}{3.519999in}}%
\pgfpathlineto{\pgfqpoint{1.476559in}{3.519999in}}%
\pgfpathlineto{\pgfqpoint{1.476559in}{3.425625in}}%
\pgfpathmoveto{\pgfqpoint{1.476559in}{1.066250in}}%
\pgfpathlineto{\pgfqpoint{1.476559in}{1.066250in}}%
\pgfpathlineto{\pgfqpoint{1.476559in}{1.160623in}}%
\pgfpathlineto{\pgfqpoint{1.621874in}{1.160623in}}%
\pgfpathlineto{\pgfqpoint{1.621874in}{1.066250in}}%
\pgfpathmoveto{\pgfqpoint{1.476559in}{1.160623in}}%
\pgfpathlineto{\pgfqpoint{1.476559in}{1.160623in}}%
\pgfpathlineto{\pgfqpoint{1.476559in}{1.254999in}}%
\pgfpathlineto{\pgfqpoint{1.621874in}{1.254999in}}%
\pgfpathlineto{\pgfqpoint{1.621874in}{1.160623in}}%
\pgfpathmoveto{\pgfqpoint{1.476559in}{1.254999in}}%
\pgfpathlineto{\pgfqpoint{1.476559in}{1.254999in}}%
\pgfpathlineto{\pgfqpoint{1.476559in}{1.349376in}}%
\pgfpathlineto{\pgfqpoint{1.621874in}{1.349376in}}%
\pgfpathlineto{\pgfqpoint{1.621874in}{1.254999in}}%
\pgfpathmoveto{\pgfqpoint{1.476559in}{1.349376in}}%
\pgfpathlineto{\pgfqpoint{1.476559in}{1.349376in}}%
\pgfpathlineto{\pgfqpoint{1.476559in}{1.443752in}}%
\pgfpathlineto{\pgfqpoint{1.621874in}{1.443752in}}%
\pgfpathlineto{\pgfqpoint{1.621874in}{1.349376in}}%
\pgfpathmoveto{\pgfqpoint{1.476559in}{1.443752in}}%
\pgfpathlineto{\pgfqpoint{1.476559in}{1.443752in}}%
\pgfpathlineto{\pgfqpoint{1.476559in}{1.538125in}}%
\pgfpathlineto{\pgfqpoint{1.621874in}{1.538125in}}%
\pgfpathlineto{\pgfqpoint{1.621874in}{1.443752in}}%
\pgfpathmoveto{\pgfqpoint{1.476559in}{1.538125in}}%
\pgfpathlineto{\pgfqpoint{1.476559in}{1.538125in}}%
\pgfpathlineto{\pgfqpoint{1.476559in}{1.632498in}}%
\pgfpathlineto{\pgfqpoint{1.621874in}{1.632498in}}%
\pgfpathlineto{\pgfqpoint{1.621874in}{1.538125in}}%
\pgfpathmoveto{\pgfqpoint{1.476559in}{1.632498in}}%
\pgfpathlineto{\pgfqpoint{1.476559in}{1.632498in}}%
\pgfpathlineto{\pgfqpoint{1.476559in}{1.726874in}}%
\pgfpathlineto{\pgfqpoint{1.621874in}{1.726874in}}%
\pgfpathlineto{\pgfqpoint{1.621874in}{1.632498in}}%
\pgfpathmoveto{\pgfqpoint{1.476559in}{1.726874in}}%
\pgfpathlineto{\pgfqpoint{1.476559in}{1.726874in}}%
\pgfpathlineto{\pgfqpoint{1.476559in}{1.821250in}}%
\pgfpathlineto{\pgfqpoint{1.621874in}{1.821250in}}%
\pgfpathlineto{\pgfqpoint{1.621874in}{1.726874in}}%
\pgfpathmoveto{\pgfqpoint{1.476559in}{1.821250in}}%
\pgfpathlineto{\pgfqpoint{1.476559in}{1.821250in}}%
\pgfpathlineto{\pgfqpoint{1.476559in}{1.915626in}}%
\pgfpathlineto{\pgfqpoint{1.621874in}{1.915626in}}%
\pgfpathlineto{\pgfqpoint{1.621874in}{1.821250in}}%
\pgfpathmoveto{\pgfqpoint{1.476559in}{1.915626in}}%
\pgfpathlineto{\pgfqpoint{1.476559in}{1.915626in}}%
\pgfpathlineto{\pgfqpoint{1.476559in}{2.009997in}}%
\pgfpathlineto{\pgfqpoint{1.621874in}{2.009997in}}%
\pgfpathlineto{\pgfqpoint{1.621874in}{1.915626in}}%
\pgfpathmoveto{\pgfqpoint{1.476559in}{2.009997in}}%
\pgfpathlineto{\pgfqpoint{1.476559in}{2.009997in}}%
\pgfpathlineto{\pgfqpoint{1.476559in}{2.104378in}}%
\pgfpathlineto{\pgfqpoint{1.621874in}{2.104378in}}%
\pgfpathlineto{\pgfqpoint{1.621874in}{2.009997in}}%
\pgfpathmoveto{\pgfqpoint{1.476559in}{2.104378in}}%
\pgfpathlineto{\pgfqpoint{1.476559in}{2.104378in}}%
\pgfpathlineto{\pgfqpoint{1.476559in}{2.198752in}}%
\pgfpathlineto{\pgfqpoint{1.621874in}{2.198752in}}%
\pgfpathlineto{\pgfqpoint{1.621874in}{2.104378in}}%
\pgfpathmoveto{\pgfqpoint{1.476559in}{2.198752in}}%
\pgfpathlineto{\pgfqpoint{1.476559in}{2.198752in}}%
\pgfpathlineto{\pgfqpoint{1.476559in}{2.293123in}}%
\pgfpathlineto{\pgfqpoint{1.621874in}{2.293123in}}%
\pgfpathlineto{\pgfqpoint{1.621874in}{2.198752in}}%
\pgfpathmoveto{\pgfqpoint{1.476559in}{2.293123in}}%
\pgfpathlineto{\pgfqpoint{1.476559in}{2.293123in}}%
\pgfpathlineto{\pgfqpoint{1.476559in}{2.387497in}}%
\pgfpathlineto{\pgfqpoint{1.621874in}{2.387497in}}%
\pgfpathlineto{\pgfqpoint{1.621874in}{2.293123in}}%
\pgfpathmoveto{\pgfqpoint{1.476559in}{2.387497in}}%
\pgfpathlineto{\pgfqpoint{1.476559in}{2.387497in}}%
\pgfpathlineto{\pgfqpoint{1.476559in}{2.481873in}}%
\pgfpathlineto{\pgfqpoint{1.621874in}{2.481873in}}%
\pgfpathlineto{\pgfqpoint{1.621874in}{2.387497in}}%
\pgfpathmoveto{\pgfqpoint{1.476559in}{2.481873in}}%
\pgfpathlineto{\pgfqpoint{1.476559in}{2.481873in}}%
\pgfpathlineto{\pgfqpoint{1.476559in}{2.576252in}}%
\pgfpathlineto{\pgfqpoint{1.621874in}{2.576252in}}%
\pgfpathlineto{\pgfqpoint{1.621874in}{2.481873in}}%
\pgfpathmoveto{\pgfqpoint{1.476559in}{2.576252in}}%
\pgfpathlineto{\pgfqpoint{1.476559in}{2.576252in}}%
\pgfpathlineto{\pgfqpoint{1.476559in}{2.670624in}}%
\pgfpathlineto{\pgfqpoint{1.621874in}{2.670624in}}%
\pgfpathlineto{\pgfqpoint{1.621874in}{2.576252in}}%
\pgfpathmoveto{\pgfqpoint{1.476559in}{2.670624in}}%
\pgfpathlineto{\pgfqpoint{1.476559in}{2.670624in}}%
\pgfpathlineto{\pgfqpoint{1.476559in}{2.765002in}}%
\pgfpathlineto{\pgfqpoint{1.621874in}{2.765002in}}%
\pgfpathlineto{\pgfqpoint{1.621874in}{2.670624in}}%
\pgfpathmoveto{\pgfqpoint{1.476559in}{2.765002in}}%
\pgfpathlineto{\pgfqpoint{1.476559in}{2.765002in}}%
\pgfpathlineto{\pgfqpoint{1.476559in}{2.859376in}}%
\pgfpathlineto{\pgfqpoint{1.621874in}{2.859376in}}%
\pgfpathlineto{\pgfqpoint{1.621874in}{2.765002in}}%
\pgfpathmoveto{\pgfqpoint{1.476559in}{2.859376in}}%
\pgfpathlineto{\pgfqpoint{1.476559in}{2.859376in}}%
\pgfpathlineto{\pgfqpoint{1.476559in}{2.953748in}}%
\pgfpathlineto{\pgfqpoint{1.621874in}{2.953748in}}%
\pgfpathlineto{\pgfqpoint{1.621874in}{2.859376in}}%
\pgfpathmoveto{\pgfqpoint{1.476559in}{2.953748in}}%
\pgfpathlineto{\pgfqpoint{1.476559in}{2.953748in}}%
\pgfpathlineto{\pgfqpoint{1.476559in}{3.048126in}}%
\pgfpathlineto{\pgfqpoint{1.621874in}{3.048126in}}%
\pgfpathlineto{\pgfqpoint{1.621874in}{2.953748in}}%
\pgfpathmoveto{\pgfqpoint{1.476559in}{3.048126in}}%
\pgfpathlineto{\pgfqpoint{1.476559in}{3.048126in}}%
\pgfpathlineto{\pgfqpoint{1.476559in}{3.142501in}}%
\pgfpathlineto{\pgfqpoint{1.621874in}{3.142501in}}%
\pgfpathlineto{\pgfqpoint{1.621874in}{3.048126in}}%
\pgfpathmoveto{\pgfqpoint{1.476559in}{3.142501in}}%
\pgfpathlineto{\pgfqpoint{1.476559in}{3.142501in}}%
\pgfpathlineto{\pgfqpoint{1.476559in}{3.236873in}}%
\pgfpathlineto{\pgfqpoint{1.621874in}{3.236873in}}%
\pgfpathlineto{\pgfqpoint{1.621874in}{3.142501in}}%
\pgfpathmoveto{\pgfqpoint{1.476559in}{3.236873in}}%
\pgfpathlineto{\pgfqpoint{1.476559in}{3.236873in}}%
\pgfpathlineto{\pgfqpoint{1.476559in}{3.331249in}}%
\pgfpathlineto{\pgfqpoint{1.621874in}{3.331249in}}%
\pgfpathlineto{\pgfqpoint{1.621874in}{3.236873in}}%
\pgfpathmoveto{\pgfqpoint{1.476559in}{3.331249in}}%
\pgfpathlineto{\pgfqpoint{1.476559in}{3.331249in}}%
\pgfpathlineto{\pgfqpoint{1.476559in}{3.425625in}}%
\pgfpathlineto{\pgfqpoint{1.621874in}{3.425625in}}%
\pgfpathlineto{\pgfqpoint{1.621874in}{3.331249in}}%
\pgfpathmoveto{\pgfqpoint{1.476559in}{3.425625in}}%
\pgfpathlineto{\pgfqpoint{1.476559in}{3.425625in}}%
\pgfpathlineto{\pgfqpoint{1.476559in}{3.519999in}}%
\pgfpathlineto{\pgfqpoint{1.621874in}{3.519999in}}%
\pgfpathlineto{\pgfqpoint{1.621874in}{3.425625in}}%
\pgfpathmoveto{\pgfqpoint{1.621874in}{1.254999in}}%
\pgfpathlineto{\pgfqpoint{1.621874in}{1.254999in}}%
\pgfpathlineto{\pgfqpoint{1.621874in}{1.349376in}}%
\pgfpathlineto{\pgfqpoint{1.767187in}{1.349376in}}%
\pgfpathlineto{\pgfqpoint{1.767187in}{1.254999in}}%
\pgfpathmoveto{\pgfqpoint{1.621874in}{1.349376in}}%
\pgfpathlineto{\pgfqpoint{1.621874in}{1.349376in}}%
\pgfpathlineto{\pgfqpoint{1.621874in}{1.443752in}}%
\pgfpathlineto{\pgfqpoint{1.767187in}{1.443752in}}%
\pgfpathlineto{\pgfqpoint{1.767187in}{1.349376in}}%
\pgfpathmoveto{\pgfqpoint{1.621874in}{1.443752in}}%
\pgfpathlineto{\pgfqpoint{1.621874in}{1.443752in}}%
\pgfpathlineto{\pgfqpoint{1.621874in}{1.538125in}}%
\pgfpathlineto{\pgfqpoint{1.767187in}{1.538125in}}%
\pgfpathlineto{\pgfqpoint{1.767187in}{1.443752in}}%
\pgfpathmoveto{\pgfqpoint{1.621874in}{1.538125in}}%
\pgfpathlineto{\pgfqpoint{1.621874in}{1.538125in}}%
\pgfpathlineto{\pgfqpoint{1.621874in}{1.632498in}}%
\pgfpathlineto{\pgfqpoint{1.767187in}{1.632498in}}%
\pgfpathlineto{\pgfqpoint{1.767187in}{1.538125in}}%
\pgfpathmoveto{\pgfqpoint{1.621874in}{1.632498in}}%
\pgfpathlineto{\pgfqpoint{1.621874in}{1.632498in}}%
\pgfpathlineto{\pgfqpoint{1.621874in}{1.726874in}}%
\pgfpathlineto{\pgfqpoint{1.767187in}{1.726874in}}%
\pgfpathlineto{\pgfqpoint{1.767187in}{1.632498in}}%
\pgfpathmoveto{\pgfqpoint{1.621874in}{1.726874in}}%
\pgfpathlineto{\pgfqpoint{1.621874in}{1.726874in}}%
\pgfpathlineto{\pgfqpoint{1.621874in}{1.821250in}}%
\pgfpathlineto{\pgfqpoint{1.767187in}{1.821250in}}%
\pgfpathlineto{\pgfqpoint{1.767187in}{1.726874in}}%
\pgfpathmoveto{\pgfqpoint{1.621874in}{1.821250in}}%
\pgfpathlineto{\pgfqpoint{1.621874in}{1.821250in}}%
\pgfpathlineto{\pgfqpoint{1.621874in}{1.915626in}}%
\pgfpathlineto{\pgfqpoint{1.767187in}{1.915626in}}%
\pgfpathlineto{\pgfqpoint{1.767187in}{1.821250in}}%
\pgfpathmoveto{\pgfqpoint{1.621874in}{1.915626in}}%
\pgfpathlineto{\pgfqpoint{1.621874in}{1.915626in}}%
\pgfpathlineto{\pgfqpoint{1.621874in}{2.009997in}}%
\pgfpathlineto{\pgfqpoint{1.767187in}{2.009997in}}%
\pgfpathlineto{\pgfqpoint{1.767187in}{1.915626in}}%
\pgfpathmoveto{\pgfqpoint{1.621874in}{2.009997in}}%
\pgfpathlineto{\pgfqpoint{1.621874in}{2.009997in}}%
\pgfpathlineto{\pgfqpoint{1.621874in}{2.104378in}}%
\pgfpathlineto{\pgfqpoint{1.767187in}{2.104378in}}%
\pgfpathlineto{\pgfqpoint{1.767187in}{2.009997in}}%
\pgfpathmoveto{\pgfqpoint{1.621874in}{2.104378in}}%
\pgfpathlineto{\pgfqpoint{1.621874in}{2.104378in}}%
\pgfpathlineto{\pgfqpoint{1.621874in}{2.198752in}}%
\pgfpathlineto{\pgfqpoint{1.767187in}{2.198752in}}%
\pgfpathlineto{\pgfqpoint{1.767187in}{2.104378in}}%
\pgfpathmoveto{\pgfqpoint{1.621874in}{2.198752in}}%
\pgfpathlineto{\pgfqpoint{1.621874in}{2.198752in}}%
\pgfpathlineto{\pgfqpoint{1.621874in}{2.293123in}}%
\pgfpathlineto{\pgfqpoint{1.767187in}{2.293123in}}%
\pgfpathlineto{\pgfqpoint{1.767187in}{2.198752in}}%
\pgfpathmoveto{\pgfqpoint{1.621874in}{2.293123in}}%
\pgfpathlineto{\pgfqpoint{1.621874in}{2.293123in}}%
\pgfpathlineto{\pgfqpoint{1.621874in}{2.387497in}}%
\pgfpathlineto{\pgfqpoint{1.767187in}{2.387497in}}%
\pgfpathlineto{\pgfqpoint{1.767187in}{2.293123in}}%
\pgfpathmoveto{\pgfqpoint{1.621874in}{2.387497in}}%
\pgfpathlineto{\pgfqpoint{1.621874in}{2.387497in}}%
\pgfpathlineto{\pgfqpoint{1.621874in}{2.481873in}}%
\pgfpathlineto{\pgfqpoint{1.767187in}{2.481873in}}%
\pgfpathlineto{\pgfqpoint{1.767187in}{2.387497in}}%
\pgfpathmoveto{\pgfqpoint{1.621874in}{2.481873in}}%
\pgfpathlineto{\pgfqpoint{1.621874in}{2.481873in}}%
\pgfpathlineto{\pgfqpoint{1.621874in}{2.576252in}}%
\pgfpathlineto{\pgfqpoint{1.767187in}{2.576252in}}%
\pgfpathlineto{\pgfqpoint{1.767187in}{2.481873in}}%
\pgfpathmoveto{\pgfqpoint{1.621874in}{2.576252in}}%
\pgfpathlineto{\pgfqpoint{1.621874in}{2.576252in}}%
\pgfpathlineto{\pgfqpoint{1.621874in}{2.670624in}}%
\pgfpathlineto{\pgfqpoint{1.767187in}{2.670624in}}%
\pgfpathlineto{\pgfqpoint{1.767187in}{2.576252in}}%
\pgfpathmoveto{\pgfqpoint{1.621874in}{2.670624in}}%
\pgfpathlineto{\pgfqpoint{1.621874in}{2.670624in}}%
\pgfpathlineto{\pgfqpoint{1.621874in}{2.765002in}}%
\pgfpathlineto{\pgfqpoint{1.767187in}{2.765002in}}%
\pgfpathlineto{\pgfqpoint{1.767187in}{2.670624in}}%
\pgfpathmoveto{\pgfqpoint{1.621874in}{2.765002in}}%
\pgfpathlineto{\pgfqpoint{1.621874in}{2.765002in}}%
\pgfpathlineto{\pgfqpoint{1.621874in}{2.859376in}}%
\pgfpathlineto{\pgfqpoint{1.767187in}{2.859376in}}%
\pgfpathlineto{\pgfqpoint{1.767187in}{2.765002in}}%
\pgfpathmoveto{\pgfqpoint{1.621874in}{2.859376in}}%
\pgfpathlineto{\pgfqpoint{1.621874in}{2.859376in}}%
\pgfpathlineto{\pgfqpoint{1.621874in}{2.953748in}}%
\pgfpathlineto{\pgfqpoint{1.767187in}{2.953748in}}%
\pgfpathlineto{\pgfqpoint{1.767187in}{2.859376in}}%
\pgfpathmoveto{\pgfqpoint{1.621874in}{2.953748in}}%
\pgfpathlineto{\pgfqpoint{1.621874in}{2.953748in}}%
\pgfpathlineto{\pgfqpoint{1.621874in}{3.048126in}}%
\pgfpathlineto{\pgfqpoint{1.767187in}{3.048126in}}%
\pgfpathlineto{\pgfqpoint{1.767187in}{2.953748in}}%
\pgfpathmoveto{\pgfqpoint{1.621874in}{3.048126in}}%
\pgfpathlineto{\pgfqpoint{1.621874in}{3.048126in}}%
\pgfpathlineto{\pgfqpoint{1.621874in}{3.142501in}}%
\pgfpathlineto{\pgfqpoint{1.767187in}{3.142501in}}%
\pgfpathlineto{\pgfqpoint{1.767187in}{3.048126in}}%
\pgfpathmoveto{\pgfqpoint{1.621874in}{3.142501in}}%
\pgfpathlineto{\pgfqpoint{1.621874in}{3.142501in}}%
\pgfpathlineto{\pgfqpoint{1.621874in}{3.236873in}}%
\pgfpathlineto{\pgfqpoint{1.767187in}{3.236873in}}%
\pgfpathlineto{\pgfqpoint{1.767187in}{3.142501in}}%
\pgfpathmoveto{\pgfqpoint{1.621874in}{3.236873in}}%
\pgfpathlineto{\pgfqpoint{1.621874in}{3.236873in}}%
\pgfpathlineto{\pgfqpoint{1.621874in}{3.331249in}}%
\pgfpathlineto{\pgfqpoint{1.767187in}{3.331249in}}%
\pgfpathlineto{\pgfqpoint{1.767187in}{3.236873in}}%
\pgfpathmoveto{\pgfqpoint{1.621874in}{3.331249in}}%
\pgfpathlineto{\pgfqpoint{1.621874in}{3.331249in}}%
\pgfpathlineto{\pgfqpoint{1.621874in}{3.425625in}}%
\pgfpathlineto{\pgfqpoint{1.767187in}{3.425625in}}%
\pgfpathlineto{\pgfqpoint{1.767187in}{3.331249in}}%
\pgfpathmoveto{\pgfqpoint{1.621874in}{3.425625in}}%
\pgfpathlineto{\pgfqpoint{1.621874in}{3.425625in}}%
\pgfpathlineto{\pgfqpoint{1.621874in}{3.519999in}}%
\pgfpathlineto{\pgfqpoint{1.767187in}{3.519999in}}%
\pgfpathlineto{\pgfqpoint{1.767187in}{3.425625in}}%
\pgfpathmoveto{\pgfqpoint{1.767187in}{1.349376in}}%
\pgfpathlineto{\pgfqpoint{1.767187in}{1.349376in}}%
\pgfpathlineto{\pgfqpoint{1.767187in}{1.443752in}}%
\pgfpathlineto{\pgfqpoint{1.912503in}{1.443752in}}%
\pgfpathlineto{\pgfqpoint{1.912503in}{1.349376in}}%
\pgfpathmoveto{\pgfqpoint{1.767187in}{1.443752in}}%
\pgfpathlineto{\pgfqpoint{1.767187in}{1.443752in}}%
\pgfpathlineto{\pgfqpoint{1.767187in}{1.538125in}}%
\pgfpathlineto{\pgfqpoint{1.912503in}{1.538125in}}%
\pgfpathlineto{\pgfqpoint{1.912503in}{1.443752in}}%
\pgfpathmoveto{\pgfqpoint{1.767187in}{1.538125in}}%
\pgfpathlineto{\pgfqpoint{1.767187in}{1.538125in}}%
\pgfpathlineto{\pgfqpoint{1.767187in}{1.632498in}}%
\pgfpathlineto{\pgfqpoint{1.912503in}{1.632498in}}%
\pgfpathlineto{\pgfqpoint{1.912503in}{1.538125in}}%
\pgfpathmoveto{\pgfqpoint{1.767187in}{1.632498in}}%
\pgfpathlineto{\pgfqpoint{1.767187in}{1.632498in}}%
\pgfpathlineto{\pgfqpoint{1.767187in}{1.726874in}}%
\pgfpathlineto{\pgfqpoint{1.912503in}{1.726874in}}%
\pgfpathlineto{\pgfqpoint{1.912503in}{1.632498in}}%
\pgfpathmoveto{\pgfqpoint{1.767187in}{1.726874in}}%
\pgfpathlineto{\pgfqpoint{1.767187in}{1.726874in}}%
\pgfpathlineto{\pgfqpoint{1.767187in}{1.821250in}}%
\pgfpathlineto{\pgfqpoint{1.912503in}{1.821250in}}%
\pgfpathlineto{\pgfqpoint{1.912503in}{1.726874in}}%
\pgfpathmoveto{\pgfqpoint{1.767187in}{1.821250in}}%
\pgfpathlineto{\pgfqpoint{1.767187in}{1.821250in}}%
\pgfpathlineto{\pgfqpoint{1.767187in}{1.915626in}}%
\pgfpathlineto{\pgfqpoint{1.912503in}{1.915626in}}%
\pgfpathlineto{\pgfqpoint{1.912503in}{1.821250in}}%
\pgfpathmoveto{\pgfqpoint{1.767187in}{1.915626in}}%
\pgfpathlineto{\pgfqpoint{1.767187in}{1.915626in}}%
\pgfpathlineto{\pgfqpoint{1.767187in}{2.009997in}}%
\pgfpathlineto{\pgfqpoint{1.912503in}{2.009997in}}%
\pgfpathlineto{\pgfqpoint{1.912503in}{1.915626in}}%
\pgfpathmoveto{\pgfqpoint{1.767187in}{2.009997in}}%
\pgfpathlineto{\pgfqpoint{1.767187in}{2.009997in}}%
\pgfpathlineto{\pgfqpoint{1.767187in}{2.104378in}}%
\pgfpathlineto{\pgfqpoint{1.912503in}{2.104378in}}%
\pgfpathlineto{\pgfqpoint{1.912503in}{2.009997in}}%
\pgfpathmoveto{\pgfqpoint{1.767187in}{2.104378in}}%
\pgfpathlineto{\pgfqpoint{1.767187in}{2.104378in}}%
\pgfpathlineto{\pgfqpoint{1.767187in}{2.198752in}}%
\pgfpathlineto{\pgfqpoint{1.912503in}{2.198752in}}%
\pgfpathlineto{\pgfqpoint{1.912503in}{2.104378in}}%
\pgfpathmoveto{\pgfqpoint{1.767187in}{2.198752in}}%
\pgfpathlineto{\pgfqpoint{1.767187in}{2.198752in}}%
\pgfpathlineto{\pgfqpoint{1.767187in}{2.293123in}}%
\pgfpathlineto{\pgfqpoint{1.912503in}{2.293123in}}%
\pgfpathlineto{\pgfqpoint{1.912503in}{2.198752in}}%
\pgfpathmoveto{\pgfqpoint{1.767187in}{2.293123in}}%
\pgfpathlineto{\pgfqpoint{1.767187in}{2.293123in}}%
\pgfpathlineto{\pgfqpoint{1.767187in}{2.387497in}}%
\pgfpathlineto{\pgfqpoint{1.912503in}{2.387497in}}%
\pgfpathlineto{\pgfqpoint{1.912503in}{2.293123in}}%
\pgfpathmoveto{\pgfqpoint{1.767187in}{2.387497in}}%
\pgfpathlineto{\pgfqpoint{1.767187in}{2.387497in}}%
\pgfpathlineto{\pgfqpoint{1.767187in}{2.481873in}}%
\pgfpathlineto{\pgfqpoint{1.912503in}{2.481873in}}%
\pgfpathlineto{\pgfqpoint{1.912503in}{2.387497in}}%
\pgfpathmoveto{\pgfqpoint{1.767187in}{2.481873in}}%
\pgfpathlineto{\pgfqpoint{1.767187in}{2.481873in}}%
\pgfpathlineto{\pgfqpoint{1.767187in}{2.576252in}}%
\pgfpathlineto{\pgfqpoint{1.912503in}{2.576252in}}%
\pgfpathlineto{\pgfqpoint{1.912503in}{2.481873in}}%
\pgfpathmoveto{\pgfqpoint{1.767187in}{2.576252in}}%
\pgfpathlineto{\pgfqpoint{1.767187in}{2.576252in}}%
\pgfpathlineto{\pgfqpoint{1.767187in}{2.670624in}}%
\pgfpathlineto{\pgfqpoint{1.912503in}{2.670624in}}%
\pgfpathlineto{\pgfqpoint{1.912503in}{2.576252in}}%
\pgfpathmoveto{\pgfqpoint{1.767187in}{2.670624in}}%
\pgfpathlineto{\pgfqpoint{1.767187in}{2.670624in}}%
\pgfpathlineto{\pgfqpoint{1.767187in}{2.765002in}}%
\pgfpathlineto{\pgfqpoint{1.912503in}{2.765002in}}%
\pgfpathlineto{\pgfqpoint{1.912503in}{2.670624in}}%
\pgfpathmoveto{\pgfqpoint{1.767187in}{2.765002in}}%
\pgfpathlineto{\pgfqpoint{1.767187in}{2.765002in}}%
\pgfpathlineto{\pgfqpoint{1.767187in}{2.859376in}}%
\pgfpathlineto{\pgfqpoint{1.912503in}{2.859376in}}%
\pgfpathlineto{\pgfqpoint{1.912503in}{2.765002in}}%
\pgfpathmoveto{\pgfqpoint{1.767187in}{2.859376in}}%
\pgfpathlineto{\pgfqpoint{1.767187in}{2.859376in}}%
\pgfpathlineto{\pgfqpoint{1.767187in}{2.953748in}}%
\pgfpathlineto{\pgfqpoint{1.912503in}{2.953748in}}%
\pgfpathlineto{\pgfqpoint{1.912503in}{2.859376in}}%
\pgfpathmoveto{\pgfqpoint{1.767187in}{2.953748in}}%
\pgfpathlineto{\pgfqpoint{1.767187in}{2.953748in}}%
\pgfpathlineto{\pgfqpoint{1.767187in}{3.048126in}}%
\pgfpathlineto{\pgfqpoint{1.912503in}{3.048126in}}%
\pgfpathlineto{\pgfqpoint{1.912503in}{2.953748in}}%
\pgfpathmoveto{\pgfqpoint{1.767187in}{3.048126in}}%
\pgfpathlineto{\pgfqpoint{1.767187in}{3.048126in}}%
\pgfpathlineto{\pgfqpoint{1.767187in}{3.142501in}}%
\pgfpathlineto{\pgfqpoint{1.912503in}{3.142501in}}%
\pgfpathlineto{\pgfqpoint{1.912503in}{3.048126in}}%
\pgfpathmoveto{\pgfqpoint{1.767187in}{3.142501in}}%
\pgfpathlineto{\pgfqpoint{1.767187in}{3.142501in}}%
\pgfpathlineto{\pgfqpoint{1.767187in}{3.236873in}}%
\pgfpathlineto{\pgfqpoint{1.912503in}{3.236873in}}%
\pgfpathlineto{\pgfqpoint{1.912503in}{3.142501in}}%
\pgfpathmoveto{\pgfqpoint{1.767187in}{3.236873in}}%
\pgfpathlineto{\pgfqpoint{1.767187in}{3.236873in}}%
\pgfpathlineto{\pgfqpoint{1.767187in}{3.331249in}}%
\pgfpathlineto{\pgfqpoint{1.912503in}{3.331249in}}%
\pgfpathlineto{\pgfqpoint{1.912503in}{3.236873in}}%
\pgfpathmoveto{\pgfqpoint{1.767187in}{3.331249in}}%
\pgfpathlineto{\pgfqpoint{1.767187in}{3.331249in}}%
\pgfpathlineto{\pgfqpoint{1.767187in}{3.425625in}}%
\pgfpathlineto{\pgfqpoint{1.912503in}{3.425625in}}%
\pgfpathlineto{\pgfqpoint{1.912503in}{3.331249in}}%
\pgfpathmoveto{\pgfqpoint{1.767187in}{3.425625in}}%
\pgfpathlineto{\pgfqpoint{1.767187in}{3.425625in}}%
\pgfpathlineto{\pgfqpoint{1.767187in}{3.519999in}}%
\pgfpathlineto{\pgfqpoint{1.912503in}{3.519999in}}%
\pgfpathlineto{\pgfqpoint{1.912503in}{3.425625in}}%
\pgfpathmoveto{\pgfqpoint{1.912503in}{1.349376in}}%
\pgfpathlineto{\pgfqpoint{1.912503in}{1.349376in}}%
\pgfpathlineto{\pgfqpoint{1.912503in}{1.443752in}}%
\pgfpathlineto{\pgfqpoint{2.057810in}{1.443752in}}%
\pgfpathlineto{\pgfqpoint{2.057810in}{1.349376in}}%
\pgfpathmoveto{\pgfqpoint{1.912503in}{1.443752in}}%
\pgfpathlineto{\pgfqpoint{1.912503in}{1.443752in}}%
\pgfpathlineto{\pgfqpoint{1.912503in}{1.538125in}}%
\pgfpathlineto{\pgfqpoint{2.057810in}{1.538125in}}%
\pgfpathlineto{\pgfqpoint{2.057810in}{1.443752in}}%
\pgfpathmoveto{\pgfqpoint{1.912503in}{1.538125in}}%
\pgfpathlineto{\pgfqpoint{1.912503in}{1.538125in}}%
\pgfpathlineto{\pgfqpoint{1.912503in}{1.632498in}}%
\pgfpathlineto{\pgfqpoint{2.057810in}{1.632498in}}%
\pgfpathlineto{\pgfqpoint{2.057810in}{1.538125in}}%
\pgfpathmoveto{\pgfqpoint{1.912503in}{1.632498in}}%
\pgfpathlineto{\pgfqpoint{1.912503in}{1.632498in}}%
\pgfpathlineto{\pgfqpoint{1.912503in}{1.726874in}}%
\pgfpathlineto{\pgfqpoint{2.057810in}{1.726874in}}%
\pgfpathlineto{\pgfqpoint{2.057810in}{1.632498in}}%
\pgfpathmoveto{\pgfqpoint{1.912503in}{1.726874in}}%
\pgfpathlineto{\pgfqpoint{1.912503in}{1.726874in}}%
\pgfpathlineto{\pgfqpoint{1.912503in}{1.821250in}}%
\pgfpathlineto{\pgfqpoint{2.057810in}{1.821250in}}%
\pgfpathlineto{\pgfqpoint{2.057810in}{1.726874in}}%
\pgfpathmoveto{\pgfqpoint{1.912503in}{1.821250in}}%
\pgfpathlineto{\pgfqpoint{1.912503in}{1.821250in}}%
\pgfpathlineto{\pgfqpoint{1.912503in}{1.915626in}}%
\pgfpathlineto{\pgfqpoint{2.057810in}{1.915626in}}%
\pgfpathlineto{\pgfqpoint{2.057810in}{1.821250in}}%
\pgfpathmoveto{\pgfqpoint{1.912503in}{1.915626in}}%
\pgfpathlineto{\pgfqpoint{1.912503in}{1.915626in}}%
\pgfpathlineto{\pgfqpoint{1.912503in}{2.009997in}}%
\pgfpathlineto{\pgfqpoint{2.057810in}{2.009997in}}%
\pgfpathlineto{\pgfqpoint{2.057810in}{1.915626in}}%
\pgfpathmoveto{\pgfqpoint{1.912503in}{2.009997in}}%
\pgfpathlineto{\pgfqpoint{1.912503in}{2.009997in}}%
\pgfpathlineto{\pgfqpoint{1.912503in}{2.104378in}}%
\pgfpathlineto{\pgfqpoint{2.057810in}{2.104378in}}%
\pgfpathlineto{\pgfqpoint{2.057810in}{2.009997in}}%
\pgfpathmoveto{\pgfqpoint{1.912503in}{2.104378in}}%
\pgfpathlineto{\pgfqpoint{1.912503in}{2.104378in}}%
\pgfpathlineto{\pgfqpoint{1.912503in}{2.198752in}}%
\pgfpathlineto{\pgfqpoint{2.057810in}{2.198752in}}%
\pgfpathlineto{\pgfqpoint{2.057810in}{2.104378in}}%
\pgfpathmoveto{\pgfqpoint{1.912503in}{2.198752in}}%
\pgfpathlineto{\pgfqpoint{1.912503in}{2.198752in}}%
\pgfpathlineto{\pgfqpoint{1.912503in}{2.293123in}}%
\pgfpathlineto{\pgfqpoint{2.057810in}{2.293123in}}%
\pgfpathlineto{\pgfqpoint{2.057810in}{2.198752in}}%
\pgfpathmoveto{\pgfqpoint{1.912503in}{2.293123in}}%
\pgfpathlineto{\pgfqpoint{1.912503in}{2.293123in}}%
\pgfpathlineto{\pgfqpoint{1.912503in}{2.387497in}}%
\pgfpathlineto{\pgfqpoint{2.057810in}{2.387497in}}%
\pgfpathlineto{\pgfqpoint{2.057810in}{2.293123in}}%
\pgfpathmoveto{\pgfqpoint{1.912503in}{2.387497in}}%
\pgfpathlineto{\pgfqpoint{1.912503in}{2.387497in}}%
\pgfpathlineto{\pgfqpoint{1.912503in}{2.481873in}}%
\pgfpathlineto{\pgfqpoint{2.057810in}{2.481873in}}%
\pgfpathlineto{\pgfqpoint{2.057810in}{2.387497in}}%
\pgfpathmoveto{\pgfqpoint{1.912503in}{2.481873in}}%
\pgfpathlineto{\pgfqpoint{1.912503in}{2.481873in}}%
\pgfpathlineto{\pgfqpoint{1.912503in}{2.576252in}}%
\pgfpathlineto{\pgfqpoint{2.057810in}{2.576252in}}%
\pgfpathlineto{\pgfqpoint{2.057810in}{2.481873in}}%
\pgfpathmoveto{\pgfqpoint{1.912503in}{2.576252in}}%
\pgfpathlineto{\pgfqpoint{1.912503in}{2.576252in}}%
\pgfpathlineto{\pgfqpoint{1.912503in}{2.670624in}}%
\pgfpathlineto{\pgfqpoint{2.057810in}{2.670624in}}%
\pgfpathlineto{\pgfqpoint{2.057810in}{2.576252in}}%
\pgfpathmoveto{\pgfqpoint{1.912503in}{2.670624in}}%
\pgfpathlineto{\pgfqpoint{1.912503in}{2.670624in}}%
\pgfpathlineto{\pgfqpoint{1.912503in}{2.765002in}}%
\pgfpathlineto{\pgfqpoint{2.057810in}{2.765002in}}%
\pgfpathlineto{\pgfqpoint{2.057810in}{2.670624in}}%
\pgfpathmoveto{\pgfqpoint{1.912503in}{2.765002in}}%
\pgfpathlineto{\pgfqpoint{1.912503in}{2.765002in}}%
\pgfpathlineto{\pgfqpoint{1.912503in}{2.859376in}}%
\pgfpathlineto{\pgfqpoint{2.057810in}{2.859376in}}%
\pgfpathlineto{\pgfqpoint{2.057810in}{2.765002in}}%
\pgfpathmoveto{\pgfqpoint{1.912503in}{2.859376in}}%
\pgfpathlineto{\pgfqpoint{1.912503in}{2.859376in}}%
\pgfpathlineto{\pgfqpoint{1.912503in}{2.953748in}}%
\pgfpathlineto{\pgfqpoint{2.057810in}{2.953748in}}%
\pgfpathlineto{\pgfqpoint{2.057810in}{2.859376in}}%
\pgfpathmoveto{\pgfqpoint{1.912503in}{2.953748in}}%
\pgfpathlineto{\pgfqpoint{1.912503in}{2.953748in}}%
\pgfpathlineto{\pgfqpoint{1.912503in}{3.048126in}}%
\pgfpathlineto{\pgfqpoint{2.057810in}{3.048126in}}%
\pgfpathlineto{\pgfqpoint{2.057810in}{2.953748in}}%
\pgfpathmoveto{\pgfqpoint{1.912503in}{3.048126in}}%
\pgfpathlineto{\pgfqpoint{1.912503in}{3.048126in}}%
\pgfpathlineto{\pgfqpoint{1.912503in}{3.142501in}}%
\pgfpathlineto{\pgfqpoint{2.057810in}{3.142501in}}%
\pgfpathlineto{\pgfqpoint{2.057810in}{3.048126in}}%
\pgfpathmoveto{\pgfqpoint{1.912503in}{3.142501in}}%
\pgfpathlineto{\pgfqpoint{1.912503in}{3.142501in}}%
\pgfpathlineto{\pgfqpoint{1.912503in}{3.236873in}}%
\pgfpathlineto{\pgfqpoint{2.057810in}{3.236873in}}%
\pgfpathlineto{\pgfqpoint{2.057810in}{3.142501in}}%
\pgfpathmoveto{\pgfqpoint{1.912503in}{3.236873in}}%
\pgfpathlineto{\pgfqpoint{1.912503in}{3.236873in}}%
\pgfpathlineto{\pgfqpoint{1.912503in}{3.331249in}}%
\pgfpathlineto{\pgfqpoint{2.057810in}{3.331249in}}%
\pgfpathlineto{\pgfqpoint{2.057810in}{3.236873in}}%
\pgfpathmoveto{\pgfqpoint{1.912503in}{3.331249in}}%
\pgfpathlineto{\pgfqpoint{1.912503in}{3.331249in}}%
\pgfpathlineto{\pgfqpoint{1.912503in}{3.425625in}}%
\pgfpathlineto{\pgfqpoint{2.057810in}{3.425625in}}%
\pgfpathlineto{\pgfqpoint{2.057810in}{3.331249in}}%
\pgfpathmoveto{\pgfqpoint{1.912503in}{3.425625in}}%
\pgfpathlineto{\pgfqpoint{1.912503in}{3.425625in}}%
\pgfpathlineto{\pgfqpoint{1.912503in}{3.519999in}}%
\pgfpathlineto{\pgfqpoint{2.057810in}{3.519999in}}%
\pgfpathlineto{\pgfqpoint{2.057810in}{3.425625in}}%
\pgfpathmoveto{\pgfqpoint{2.057810in}{1.443752in}}%
\pgfpathlineto{\pgfqpoint{2.057810in}{1.443752in}}%
\pgfpathlineto{\pgfqpoint{2.057810in}{1.538125in}}%
\pgfpathlineto{\pgfqpoint{2.203121in}{1.538125in}}%
\pgfpathlineto{\pgfqpoint{2.203121in}{1.443752in}}%
\pgfpathmoveto{\pgfqpoint{2.057810in}{1.538125in}}%
\pgfpathlineto{\pgfqpoint{2.057810in}{1.538125in}}%
\pgfpathlineto{\pgfqpoint{2.057810in}{1.632498in}}%
\pgfpathlineto{\pgfqpoint{2.203121in}{1.632498in}}%
\pgfpathlineto{\pgfqpoint{2.203121in}{1.538125in}}%
\pgfpathmoveto{\pgfqpoint{2.057810in}{1.632498in}}%
\pgfpathlineto{\pgfqpoint{2.057810in}{1.632498in}}%
\pgfpathlineto{\pgfqpoint{2.057810in}{1.726874in}}%
\pgfpathlineto{\pgfqpoint{2.203121in}{1.726874in}}%
\pgfpathlineto{\pgfqpoint{2.203121in}{1.632498in}}%
\pgfpathmoveto{\pgfqpoint{2.057810in}{1.726874in}}%
\pgfpathlineto{\pgfqpoint{2.057810in}{1.726874in}}%
\pgfpathlineto{\pgfqpoint{2.057810in}{1.821250in}}%
\pgfpathlineto{\pgfqpoint{2.203121in}{1.821250in}}%
\pgfpathlineto{\pgfqpoint{2.203121in}{1.726874in}}%
\pgfpathmoveto{\pgfqpoint{2.057810in}{1.821250in}}%
\pgfpathlineto{\pgfqpoint{2.057810in}{1.821250in}}%
\pgfpathlineto{\pgfqpoint{2.057810in}{1.915626in}}%
\pgfpathlineto{\pgfqpoint{2.203121in}{1.915626in}}%
\pgfpathlineto{\pgfqpoint{2.203121in}{1.821250in}}%
\pgfpathmoveto{\pgfqpoint{2.057810in}{1.915626in}}%
\pgfpathlineto{\pgfqpoint{2.057810in}{1.915626in}}%
\pgfpathlineto{\pgfqpoint{2.057810in}{2.009997in}}%
\pgfpathlineto{\pgfqpoint{2.203121in}{2.009997in}}%
\pgfpathlineto{\pgfqpoint{2.203121in}{1.915626in}}%
\pgfpathmoveto{\pgfqpoint{2.057810in}{2.009997in}}%
\pgfpathlineto{\pgfqpoint{2.057810in}{2.009997in}}%
\pgfpathlineto{\pgfqpoint{2.057810in}{2.104378in}}%
\pgfpathlineto{\pgfqpoint{2.203121in}{2.104378in}}%
\pgfpathlineto{\pgfqpoint{2.203121in}{2.009997in}}%
\pgfpathmoveto{\pgfqpoint{2.057810in}{2.104378in}}%
\pgfpathlineto{\pgfqpoint{2.057810in}{2.104378in}}%
\pgfpathlineto{\pgfqpoint{2.057810in}{2.198752in}}%
\pgfpathlineto{\pgfqpoint{2.203121in}{2.198752in}}%
\pgfpathlineto{\pgfqpoint{2.203121in}{2.104378in}}%
\pgfpathmoveto{\pgfqpoint{2.057810in}{2.198752in}}%
\pgfpathlineto{\pgfqpoint{2.057810in}{2.198752in}}%
\pgfpathlineto{\pgfqpoint{2.057810in}{2.293123in}}%
\pgfpathlineto{\pgfqpoint{2.203121in}{2.293123in}}%
\pgfpathlineto{\pgfqpoint{2.203121in}{2.198752in}}%
\pgfpathmoveto{\pgfqpoint{2.057810in}{2.293123in}}%
\pgfpathlineto{\pgfqpoint{2.057810in}{2.293123in}}%
\pgfpathlineto{\pgfqpoint{2.057810in}{2.387497in}}%
\pgfpathlineto{\pgfqpoint{2.203121in}{2.387497in}}%
\pgfpathlineto{\pgfqpoint{2.203121in}{2.293123in}}%
\pgfpathmoveto{\pgfqpoint{2.057810in}{2.387497in}}%
\pgfpathlineto{\pgfqpoint{2.057810in}{2.387497in}}%
\pgfpathlineto{\pgfqpoint{2.057810in}{2.481873in}}%
\pgfpathlineto{\pgfqpoint{2.203121in}{2.481873in}}%
\pgfpathlineto{\pgfqpoint{2.203121in}{2.387497in}}%
\pgfpathmoveto{\pgfqpoint{2.057810in}{2.481873in}}%
\pgfpathlineto{\pgfqpoint{2.057810in}{2.481873in}}%
\pgfpathlineto{\pgfqpoint{2.057810in}{2.576252in}}%
\pgfpathlineto{\pgfqpoint{2.203121in}{2.576252in}}%
\pgfpathlineto{\pgfqpoint{2.203121in}{2.481873in}}%
\pgfpathmoveto{\pgfqpoint{2.057810in}{2.576252in}}%
\pgfpathlineto{\pgfqpoint{2.057810in}{2.576252in}}%
\pgfpathlineto{\pgfqpoint{2.057810in}{2.670624in}}%
\pgfpathlineto{\pgfqpoint{2.203121in}{2.670624in}}%
\pgfpathlineto{\pgfqpoint{2.203121in}{2.576252in}}%
\pgfpathmoveto{\pgfqpoint{2.057810in}{2.670624in}}%
\pgfpathlineto{\pgfqpoint{2.057810in}{2.670624in}}%
\pgfpathlineto{\pgfqpoint{2.057810in}{2.765002in}}%
\pgfpathlineto{\pgfqpoint{2.203121in}{2.765002in}}%
\pgfpathlineto{\pgfqpoint{2.203121in}{2.670624in}}%
\pgfpathmoveto{\pgfqpoint{2.057810in}{2.765002in}}%
\pgfpathlineto{\pgfqpoint{2.057810in}{2.765002in}}%
\pgfpathlineto{\pgfqpoint{2.057810in}{2.859376in}}%
\pgfpathlineto{\pgfqpoint{2.203121in}{2.859376in}}%
\pgfpathlineto{\pgfqpoint{2.203121in}{2.765002in}}%
\pgfpathmoveto{\pgfqpoint{2.057810in}{2.859376in}}%
\pgfpathlineto{\pgfqpoint{2.057810in}{2.859376in}}%
\pgfpathlineto{\pgfqpoint{2.057810in}{2.953748in}}%
\pgfpathlineto{\pgfqpoint{2.203121in}{2.953748in}}%
\pgfpathlineto{\pgfqpoint{2.203121in}{2.859376in}}%
\pgfpathmoveto{\pgfqpoint{2.057810in}{2.953748in}}%
\pgfpathlineto{\pgfqpoint{2.057810in}{2.953748in}}%
\pgfpathlineto{\pgfqpoint{2.057810in}{3.048126in}}%
\pgfpathlineto{\pgfqpoint{2.203121in}{3.048126in}}%
\pgfpathlineto{\pgfqpoint{2.203121in}{2.953748in}}%
\pgfpathmoveto{\pgfqpoint{2.057810in}{3.048126in}}%
\pgfpathlineto{\pgfqpoint{2.057810in}{3.048126in}}%
\pgfpathlineto{\pgfqpoint{2.057810in}{3.142501in}}%
\pgfpathlineto{\pgfqpoint{2.203121in}{3.142501in}}%
\pgfpathlineto{\pgfqpoint{2.203121in}{3.048126in}}%
\pgfpathmoveto{\pgfqpoint{2.057810in}{3.142501in}}%
\pgfpathlineto{\pgfqpoint{2.057810in}{3.142501in}}%
\pgfpathlineto{\pgfqpoint{2.057810in}{3.236873in}}%
\pgfpathlineto{\pgfqpoint{2.203121in}{3.236873in}}%
\pgfpathlineto{\pgfqpoint{2.203121in}{3.142501in}}%
\pgfpathmoveto{\pgfqpoint{2.057810in}{3.236873in}}%
\pgfpathlineto{\pgfqpoint{2.057810in}{3.236873in}}%
\pgfpathlineto{\pgfqpoint{2.057810in}{3.331249in}}%
\pgfpathlineto{\pgfqpoint{2.203121in}{3.331249in}}%
\pgfpathlineto{\pgfqpoint{2.203121in}{3.236873in}}%
\pgfpathmoveto{\pgfqpoint{2.057810in}{3.331249in}}%
\pgfpathlineto{\pgfqpoint{2.057810in}{3.331249in}}%
\pgfpathlineto{\pgfqpoint{2.057810in}{3.425625in}}%
\pgfpathlineto{\pgfqpoint{2.203121in}{3.425625in}}%
\pgfpathlineto{\pgfqpoint{2.203121in}{3.331249in}}%
\pgfpathmoveto{\pgfqpoint{2.057810in}{3.425625in}}%
\pgfpathlineto{\pgfqpoint{2.057810in}{3.425625in}}%
\pgfpathlineto{\pgfqpoint{2.057810in}{3.519999in}}%
\pgfpathlineto{\pgfqpoint{2.203121in}{3.519999in}}%
\pgfpathlineto{\pgfqpoint{2.203121in}{3.425625in}}%
\pgfpathmoveto{\pgfqpoint{2.203121in}{1.632498in}}%
\pgfpathlineto{\pgfqpoint{2.203121in}{1.632498in}}%
\pgfpathlineto{\pgfqpoint{2.203121in}{1.726874in}}%
\pgfpathlineto{\pgfqpoint{2.348439in}{1.726874in}}%
\pgfpathlineto{\pgfqpoint{2.348439in}{1.632498in}}%
\pgfpathmoveto{\pgfqpoint{2.203121in}{1.726874in}}%
\pgfpathlineto{\pgfqpoint{2.203121in}{1.726874in}}%
\pgfpathlineto{\pgfqpoint{2.203121in}{1.821250in}}%
\pgfpathlineto{\pgfqpoint{2.348439in}{1.821250in}}%
\pgfpathlineto{\pgfqpoint{2.348439in}{1.726874in}}%
\pgfpathmoveto{\pgfqpoint{2.203121in}{1.821250in}}%
\pgfpathlineto{\pgfqpoint{2.203121in}{1.821250in}}%
\pgfpathlineto{\pgfqpoint{2.203121in}{1.915626in}}%
\pgfpathlineto{\pgfqpoint{2.348439in}{1.915626in}}%
\pgfpathlineto{\pgfqpoint{2.348439in}{1.821250in}}%
\pgfpathmoveto{\pgfqpoint{2.203121in}{1.915626in}}%
\pgfpathlineto{\pgfqpoint{2.203121in}{1.915626in}}%
\pgfpathlineto{\pgfqpoint{2.203121in}{2.009997in}}%
\pgfpathlineto{\pgfqpoint{2.348439in}{2.009997in}}%
\pgfpathlineto{\pgfqpoint{2.348439in}{1.915626in}}%
\pgfpathmoveto{\pgfqpoint{2.203121in}{2.009997in}}%
\pgfpathlineto{\pgfqpoint{2.203121in}{2.009997in}}%
\pgfpathlineto{\pgfqpoint{2.203121in}{2.104378in}}%
\pgfpathlineto{\pgfqpoint{2.348439in}{2.104378in}}%
\pgfpathlineto{\pgfqpoint{2.348439in}{2.009997in}}%
\pgfpathmoveto{\pgfqpoint{2.203121in}{2.104378in}}%
\pgfpathlineto{\pgfqpoint{2.203121in}{2.104378in}}%
\pgfpathlineto{\pgfqpoint{2.203121in}{2.198752in}}%
\pgfpathlineto{\pgfqpoint{2.348439in}{2.198752in}}%
\pgfpathlineto{\pgfqpoint{2.348439in}{2.104378in}}%
\pgfpathmoveto{\pgfqpoint{2.203121in}{2.198752in}}%
\pgfpathlineto{\pgfqpoint{2.203121in}{2.198752in}}%
\pgfpathlineto{\pgfqpoint{2.203121in}{2.293123in}}%
\pgfpathlineto{\pgfqpoint{2.348439in}{2.293123in}}%
\pgfpathlineto{\pgfqpoint{2.348439in}{2.198752in}}%
\pgfpathmoveto{\pgfqpoint{2.203121in}{2.293123in}}%
\pgfpathlineto{\pgfqpoint{2.203121in}{2.293123in}}%
\pgfpathlineto{\pgfqpoint{2.203121in}{2.387497in}}%
\pgfpathlineto{\pgfqpoint{2.348439in}{2.387497in}}%
\pgfpathlineto{\pgfqpoint{2.348439in}{2.293123in}}%
\pgfpathmoveto{\pgfqpoint{2.203121in}{2.387497in}}%
\pgfpathlineto{\pgfqpoint{2.203121in}{2.387497in}}%
\pgfpathlineto{\pgfqpoint{2.203121in}{2.481873in}}%
\pgfpathlineto{\pgfqpoint{2.348439in}{2.481873in}}%
\pgfpathlineto{\pgfqpoint{2.348439in}{2.387497in}}%
\pgfpathmoveto{\pgfqpoint{2.203121in}{2.481873in}}%
\pgfpathlineto{\pgfqpoint{2.203121in}{2.481873in}}%
\pgfpathlineto{\pgfqpoint{2.203121in}{2.576252in}}%
\pgfpathlineto{\pgfqpoint{2.348439in}{2.576252in}}%
\pgfpathlineto{\pgfqpoint{2.348439in}{2.481873in}}%
\pgfpathmoveto{\pgfqpoint{2.203121in}{2.576252in}}%
\pgfpathlineto{\pgfqpoint{2.203121in}{2.576252in}}%
\pgfpathlineto{\pgfqpoint{2.203121in}{2.670624in}}%
\pgfpathlineto{\pgfqpoint{2.348439in}{2.670624in}}%
\pgfpathlineto{\pgfqpoint{2.348439in}{2.576252in}}%
\pgfpathmoveto{\pgfqpoint{2.203121in}{2.670624in}}%
\pgfpathlineto{\pgfqpoint{2.203121in}{2.670624in}}%
\pgfpathlineto{\pgfqpoint{2.203121in}{2.765002in}}%
\pgfpathlineto{\pgfqpoint{2.348439in}{2.765002in}}%
\pgfpathlineto{\pgfqpoint{2.348439in}{2.670624in}}%
\pgfpathmoveto{\pgfqpoint{2.203121in}{2.765002in}}%
\pgfpathlineto{\pgfqpoint{2.203121in}{2.765002in}}%
\pgfpathlineto{\pgfqpoint{2.203121in}{2.859376in}}%
\pgfpathlineto{\pgfqpoint{2.348439in}{2.859376in}}%
\pgfpathlineto{\pgfqpoint{2.348439in}{2.765002in}}%
\pgfpathmoveto{\pgfqpoint{2.203121in}{2.859376in}}%
\pgfpathlineto{\pgfqpoint{2.203121in}{2.859376in}}%
\pgfpathlineto{\pgfqpoint{2.203121in}{2.953748in}}%
\pgfpathlineto{\pgfqpoint{2.348439in}{2.953748in}}%
\pgfpathlineto{\pgfqpoint{2.348439in}{2.859376in}}%
\pgfpathmoveto{\pgfqpoint{2.203121in}{2.953748in}}%
\pgfpathlineto{\pgfqpoint{2.203121in}{2.953748in}}%
\pgfpathlineto{\pgfqpoint{2.203121in}{3.048126in}}%
\pgfpathlineto{\pgfqpoint{2.348439in}{3.048126in}}%
\pgfpathlineto{\pgfqpoint{2.348439in}{2.953748in}}%
\pgfpathmoveto{\pgfqpoint{2.203121in}{3.048126in}}%
\pgfpathlineto{\pgfqpoint{2.203121in}{3.048126in}}%
\pgfpathlineto{\pgfqpoint{2.203121in}{3.142501in}}%
\pgfpathlineto{\pgfqpoint{2.348439in}{3.142501in}}%
\pgfpathlineto{\pgfqpoint{2.348439in}{3.048126in}}%
\pgfpathmoveto{\pgfqpoint{2.203121in}{3.142501in}}%
\pgfpathlineto{\pgfqpoint{2.203121in}{3.142501in}}%
\pgfpathlineto{\pgfqpoint{2.203121in}{3.236873in}}%
\pgfpathlineto{\pgfqpoint{2.348439in}{3.236873in}}%
\pgfpathlineto{\pgfqpoint{2.348439in}{3.142501in}}%
\pgfpathmoveto{\pgfqpoint{2.203121in}{3.236873in}}%
\pgfpathlineto{\pgfqpoint{2.203121in}{3.236873in}}%
\pgfpathlineto{\pgfqpoint{2.203121in}{3.331249in}}%
\pgfpathlineto{\pgfqpoint{2.348439in}{3.331249in}}%
\pgfpathlineto{\pgfqpoint{2.348439in}{3.236873in}}%
\pgfpathmoveto{\pgfqpoint{2.203121in}{3.331249in}}%
\pgfpathlineto{\pgfqpoint{2.203121in}{3.331249in}}%
\pgfpathlineto{\pgfqpoint{2.203121in}{3.425625in}}%
\pgfpathlineto{\pgfqpoint{2.348439in}{3.425625in}}%
\pgfpathlineto{\pgfqpoint{2.348439in}{3.331249in}}%
\pgfpathmoveto{\pgfqpoint{2.203121in}{3.425625in}}%
\pgfpathlineto{\pgfqpoint{2.203121in}{3.425625in}}%
\pgfpathlineto{\pgfqpoint{2.203121in}{3.519999in}}%
\pgfpathlineto{\pgfqpoint{2.348439in}{3.519999in}}%
\pgfpathlineto{\pgfqpoint{2.348439in}{3.425625in}}%
\pgfpathmoveto{\pgfqpoint{2.348439in}{1.726874in}}%
\pgfpathlineto{\pgfqpoint{2.348439in}{1.726874in}}%
\pgfpathlineto{\pgfqpoint{2.348439in}{1.821250in}}%
\pgfpathlineto{\pgfqpoint{2.493747in}{1.821250in}}%
\pgfpathlineto{\pgfqpoint{2.493747in}{1.726874in}}%
\pgfpathmoveto{\pgfqpoint{2.348439in}{1.821250in}}%
\pgfpathlineto{\pgfqpoint{2.348439in}{1.821250in}}%
\pgfpathlineto{\pgfqpoint{2.348439in}{1.915626in}}%
\pgfpathlineto{\pgfqpoint{2.493747in}{1.915626in}}%
\pgfpathlineto{\pgfqpoint{2.493747in}{1.821250in}}%
\pgfpathmoveto{\pgfqpoint{2.348439in}{1.915626in}}%
\pgfpathlineto{\pgfqpoint{2.348439in}{1.915626in}}%
\pgfpathlineto{\pgfqpoint{2.348439in}{2.009997in}}%
\pgfpathlineto{\pgfqpoint{2.493747in}{2.009997in}}%
\pgfpathlineto{\pgfqpoint{2.493747in}{1.915626in}}%
\pgfpathmoveto{\pgfqpoint{2.348439in}{2.009997in}}%
\pgfpathlineto{\pgfqpoint{2.348439in}{2.009997in}}%
\pgfpathlineto{\pgfqpoint{2.348439in}{2.104378in}}%
\pgfpathlineto{\pgfqpoint{2.493747in}{2.104378in}}%
\pgfpathlineto{\pgfqpoint{2.493747in}{2.009997in}}%
\pgfpathmoveto{\pgfqpoint{2.348439in}{2.104378in}}%
\pgfpathlineto{\pgfqpoint{2.348439in}{2.104378in}}%
\pgfpathlineto{\pgfqpoint{2.348439in}{2.198752in}}%
\pgfpathlineto{\pgfqpoint{2.493747in}{2.198752in}}%
\pgfpathlineto{\pgfqpoint{2.493747in}{2.104378in}}%
\pgfpathmoveto{\pgfqpoint{2.348439in}{2.198752in}}%
\pgfpathlineto{\pgfqpoint{2.348439in}{2.198752in}}%
\pgfpathlineto{\pgfqpoint{2.348439in}{2.293123in}}%
\pgfpathlineto{\pgfqpoint{2.493747in}{2.293123in}}%
\pgfpathlineto{\pgfqpoint{2.493747in}{2.198752in}}%
\pgfpathmoveto{\pgfqpoint{2.348439in}{2.293123in}}%
\pgfpathlineto{\pgfqpoint{2.348439in}{2.293123in}}%
\pgfpathlineto{\pgfqpoint{2.348439in}{2.387497in}}%
\pgfpathlineto{\pgfqpoint{2.493747in}{2.387497in}}%
\pgfpathlineto{\pgfqpoint{2.493747in}{2.293123in}}%
\pgfpathmoveto{\pgfqpoint{2.348439in}{2.387497in}}%
\pgfpathlineto{\pgfqpoint{2.348439in}{2.387497in}}%
\pgfpathlineto{\pgfqpoint{2.348439in}{2.481873in}}%
\pgfpathlineto{\pgfqpoint{2.493747in}{2.481873in}}%
\pgfpathlineto{\pgfqpoint{2.493747in}{2.387497in}}%
\pgfpathmoveto{\pgfqpoint{2.348439in}{2.481873in}}%
\pgfpathlineto{\pgfqpoint{2.348439in}{2.481873in}}%
\pgfpathlineto{\pgfqpoint{2.348439in}{2.576252in}}%
\pgfpathlineto{\pgfqpoint{2.493747in}{2.576252in}}%
\pgfpathlineto{\pgfqpoint{2.493747in}{2.481873in}}%
\pgfpathmoveto{\pgfqpoint{2.348439in}{2.576252in}}%
\pgfpathlineto{\pgfqpoint{2.348439in}{2.576252in}}%
\pgfpathlineto{\pgfqpoint{2.348439in}{2.670624in}}%
\pgfpathlineto{\pgfqpoint{2.493747in}{2.670624in}}%
\pgfpathlineto{\pgfqpoint{2.493747in}{2.576252in}}%
\pgfpathmoveto{\pgfqpoint{2.348439in}{2.670624in}}%
\pgfpathlineto{\pgfqpoint{2.348439in}{2.670624in}}%
\pgfpathlineto{\pgfqpoint{2.348439in}{2.765002in}}%
\pgfpathlineto{\pgfqpoint{2.493747in}{2.765002in}}%
\pgfpathlineto{\pgfqpoint{2.493747in}{2.670624in}}%
\pgfpathmoveto{\pgfqpoint{2.348439in}{2.765002in}}%
\pgfpathlineto{\pgfqpoint{2.348439in}{2.765002in}}%
\pgfpathlineto{\pgfqpoint{2.348439in}{2.859376in}}%
\pgfpathlineto{\pgfqpoint{2.493747in}{2.859376in}}%
\pgfpathlineto{\pgfqpoint{2.493747in}{2.765002in}}%
\pgfpathmoveto{\pgfqpoint{2.348439in}{2.859376in}}%
\pgfpathlineto{\pgfqpoint{2.348439in}{2.859376in}}%
\pgfpathlineto{\pgfqpoint{2.348439in}{2.953748in}}%
\pgfpathlineto{\pgfqpoint{2.493747in}{2.953748in}}%
\pgfpathlineto{\pgfqpoint{2.493747in}{2.859376in}}%
\pgfpathmoveto{\pgfqpoint{2.348439in}{2.953748in}}%
\pgfpathlineto{\pgfqpoint{2.348439in}{2.953748in}}%
\pgfpathlineto{\pgfqpoint{2.348439in}{3.048126in}}%
\pgfpathlineto{\pgfqpoint{2.493747in}{3.048126in}}%
\pgfpathlineto{\pgfqpoint{2.493747in}{2.953748in}}%
\pgfpathmoveto{\pgfqpoint{2.348439in}{3.048126in}}%
\pgfpathlineto{\pgfqpoint{2.348439in}{3.048126in}}%
\pgfpathlineto{\pgfqpoint{2.348439in}{3.142501in}}%
\pgfpathlineto{\pgfqpoint{2.493747in}{3.142501in}}%
\pgfpathlineto{\pgfqpoint{2.493747in}{3.048126in}}%
\pgfpathmoveto{\pgfqpoint{2.348439in}{3.142501in}}%
\pgfpathlineto{\pgfqpoint{2.348439in}{3.142501in}}%
\pgfpathlineto{\pgfqpoint{2.348439in}{3.236873in}}%
\pgfpathlineto{\pgfqpoint{2.493747in}{3.236873in}}%
\pgfpathlineto{\pgfqpoint{2.493747in}{3.142501in}}%
\pgfpathmoveto{\pgfqpoint{2.348439in}{3.236873in}}%
\pgfpathlineto{\pgfqpoint{2.348439in}{3.236873in}}%
\pgfpathlineto{\pgfqpoint{2.348439in}{3.331249in}}%
\pgfpathlineto{\pgfqpoint{2.493747in}{3.331249in}}%
\pgfpathlineto{\pgfqpoint{2.493747in}{3.236873in}}%
\pgfpathmoveto{\pgfqpoint{2.348439in}{3.331249in}}%
\pgfpathlineto{\pgfqpoint{2.348439in}{3.331249in}}%
\pgfpathlineto{\pgfqpoint{2.348439in}{3.425625in}}%
\pgfpathlineto{\pgfqpoint{2.493747in}{3.425625in}}%
\pgfpathlineto{\pgfqpoint{2.493747in}{3.331249in}}%
\pgfpathmoveto{\pgfqpoint{2.348439in}{3.425625in}}%
\pgfpathlineto{\pgfqpoint{2.348439in}{3.425625in}}%
\pgfpathlineto{\pgfqpoint{2.348439in}{3.519999in}}%
\pgfpathlineto{\pgfqpoint{2.493747in}{3.519999in}}%
\pgfpathlineto{\pgfqpoint{2.493747in}{3.425625in}}%
\pgfpathmoveto{\pgfqpoint{2.493747in}{1.821250in}}%
\pgfpathlineto{\pgfqpoint{2.493747in}{1.821250in}}%
\pgfpathlineto{\pgfqpoint{2.493747in}{1.915626in}}%
\pgfpathlineto{\pgfqpoint{2.639065in}{1.915626in}}%
\pgfpathlineto{\pgfqpoint{2.639065in}{1.821250in}}%
\pgfpathmoveto{\pgfqpoint{2.493747in}{1.915626in}}%
\pgfpathlineto{\pgfqpoint{2.493747in}{1.915626in}}%
\pgfpathlineto{\pgfqpoint{2.493747in}{2.009997in}}%
\pgfpathlineto{\pgfqpoint{2.639065in}{2.009997in}}%
\pgfpathlineto{\pgfqpoint{2.639065in}{1.915626in}}%
\pgfpathmoveto{\pgfqpoint{2.493747in}{2.009997in}}%
\pgfpathlineto{\pgfqpoint{2.493747in}{2.009997in}}%
\pgfpathlineto{\pgfqpoint{2.493747in}{2.104378in}}%
\pgfpathlineto{\pgfqpoint{2.639065in}{2.104378in}}%
\pgfpathlineto{\pgfqpoint{2.639065in}{2.009997in}}%
\pgfpathmoveto{\pgfqpoint{2.493747in}{2.104378in}}%
\pgfpathlineto{\pgfqpoint{2.493747in}{2.104378in}}%
\pgfpathlineto{\pgfqpoint{2.493747in}{2.198752in}}%
\pgfpathlineto{\pgfqpoint{2.639065in}{2.198752in}}%
\pgfpathlineto{\pgfqpoint{2.639065in}{2.104378in}}%
\pgfpathmoveto{\pgfqpoint{2.493747in}{2.198752in}}%
\pgfpathlineto{\pgfqpoint{2.493747in}{2.198752in}}%
\pgfpathlineto{\pgfqpoint{2.493747in}{2.293123in}}%
\pgfpathlineto{\pgfqpoint{2.639065in}{2.293123in}}%
\pgfpathlineto{\pgfqpoint{2.639065in}{2.198752in}}%
\pgfpathmoveto{\pgfqpoint{2.493747in}{2.293123in}}%
\pgfpathlineto{\pgfqpoint{2.493747in}{2.293123in}}%
\pgfpathlineto{\pgfqpoint{2.493747in}{2.387497in}}%
\pgfpathlineto{\pgfqpoint{2.639065in}{2.387497in}}%
\pgfpathlineto{\pgfqpoint{2.639065in}{2.293123in}}%
\pgfpathmoveto{\pgfqpoint{2.493747in}{2.387497in}}%
\pgfpathlineto{\pgfqpoint{2.493747in}{2.387497in}}%
\pgfpathlineto{\pgfqpoint{2.493747in}{2.481873in}}%
\pgfpathlineto{\pgfqpoint{2.639065in}{2.481873in}}%
\pgfpathlineto{\pgfqpoint{2.639065in}{2.387497in}}%
\pgfpathmoveto{\pgfqpoint{2.493747in}{2.481873in}}%
\pgfpathlineto{\pgfqpoint{2.493747in}{2.481873in}}%
\pgfpathlineto{\pgfqpoint{2.493747in}{2.576252in}}%
\pgfpathlineto{\pgfqpoint{2.639065in}{2.576252in}}%
\pgfpathlineto{\pgfqpoint{2.639065in}{2.481873in}}%
\pgfpathmoveto{\pgfqpoint{2.493747in}{2.576252in}}%
\pgfpathlineto{\pgfqpoint{2.493747in}{2.576252in}}%
\pgfpathlineto{\pgfqpoint{2.493747in}{2.670624in}}%
\pgfpathlineto{\pgfqpoint{2.639065in}{2.670624in}}%
\pgfpathlineto{\pgfqpoint{2.639065in}{2.576252in}}%
\pgfpathmoveto{\pgfqpoint{2.493747in}{2.670624in}}%
\pgfpathlineto{\pgfqpoint{2.493747in}{2.670624in}}%
\pgfpathlineto{\pgfqpoint{2.493747in}{2.765002in}}%
\pgfpathlineto{\pgfqpoint{2.639065in}{2.765002in}}%
\pgfpathlineto{\pgfqpoint{2.639065in}{2.670624in}}%
\pgfpathmoveto{\pgfqpoint{2.493747in}{2.765002in}}%
\pgfpathlineto{\pgfqpoint{2.493747in}{2.765002in}}%
\pgfpathlineto{\pgfqpoint{2.493747in}{2.859376in}}%
\pgfpathlineto{\pgfqpoint{2.639065in}{2.859376in}}%
\pgfpathlineto{\pgfqpoint{2.639065in}{2.765002in}}%
\pgfpathmoveto{\pgfqpoint{2.493747in}{2.859376in}}%
\pgfpathlineto{\pgfqpoint{2.493747in}{2.859376in}}%
\pgfpathlineto{\pgfqpoint{2.493747in}{2.953748in}}%
\pgfpathlineto{\pgfqpoint{2.639065in}{2.953748in}}%
\pgfpathlineto{\pgfqpoint{2.639065in}{2.859376in}}%
\pgfpathmoveto{\pgfqpoint{2.493747in}{2.953748in}}%
\pgfpathlineto{\pgfqpoint{2.493747in}{2.953748in}}%
\pgfpathlineto{\pgfqpoint{2.493747in}{3.048126in}}%
\pgfpathlineto{\pgfqpoint{2.639065in}{3.048126in}}%
\pgfpathlineto{\pgfqpoint{2.639065in}{2.953748in}}%
\pgfpathmoveto{\pgfqpoint{2.493747in}{3.048126in}}%
\pgfpathlineto{\pgfqpoint{2.493747in}{3.048126in}}%
\pgfpathlineto{\pgfqpoint{2.493747in}{3.142501in}}%
\pgfpathlineto{\pgfqpoint{2.639065in}{3.142501in}}%
\pgfpathlineto{\pgfqpoint{2.639065in}{3.048126in}}%
\pgfpathmoveto{\pgfqpoint{2.493747in}{3.142501in}}%
\pgfpathlineto{\pgfqpoint{2.493747in}{3.142501in}}%
\pgfpathlineto{\pgfqpoint{2.493747in}{3.236873in}}%
\pgfpathlineto{\pgfqpoint{2.639065in}{3.236873in}}%
\pgfpathlineto{\pgfqpoint{2.639065in}{3.142501in}}%
\pgfpathmoveto{\pgfqpoint{2.493747in}{3.236873in}}%
\pgfpathlineto{\pgfqpoint{2.493747in}{3.236873in}}%
\pgfpathlineto{\pgfqpoint{2.493747in}{3.331249in}}%
\pgfpathlineto{\pgfqpoint{2.639065in}{3.331249in}}%
\pgfpathlineto{\pgfqpoint{2.639065in}{3.236873in}}%
\pgfpathmoveto{\pgfqpoint{2.493747in}{3.331249in}}%
\pgfpathlineto{\pgfqpoint{2.493747in}{3.331249in}}%
\pgfpathlineto{\pgfqpoint{2.493747in}{3.425625in}}%
\pgfpathlineto{\pgfqpoint{2.639065in}{3.425625in}}%
\pgfpathlineto{\pgfqpoint{2.639065in}{3.331249in}}%
\pgfpathmoveto{\pgfqpoint{2.493747in}{3.425625in}}%
\pgfpathlineto{\pgfqpoint{2.493747in}{3.425625in}}%
\pgfpathlineto{\pgfqpoint{2.493747in}{3.519999in}}%
\pgfpathlineto{\pgfqpoint{2.639065in}{3.519999in}}%
\pgfpathlineto{\pgfqpoint{2.639065in}{3.425625in}}%
\pgfpathmoveto{\pgfqpoint{2.639065in}{1.915626in}}%
\pgfpathlineto{\pgfqpoint{2.639065in}{1.915626in}}%
\pgfpathlineto{\pgfqpoint{2.639065in}{2.009997in}}%
\pgfpathlineto{\pgfqpoint{2.784377in}{2.009997in}}%
\pgfpathlineto{\pgfqpoint{2.784377in}{1.915626in}}%
\pgfpathmoveto{\pgfqpoint{2.639065in}{2.009997in}}%
\pgfpathlineto{\pgfqpoint{2.639065in}{2.009997in}}%
\pgfpathlineto{\pgfqpoint{2.639065in}{2.104378in}}%
\pgfpathlineto{\pgfqpoint{2.784377in}{2.104378in}}%
\pgfpathlineto{\pgfqpoint{2.784377in}{2.009997in}}%
\pgfpathmoveto{\pgfqpoint{2.639065in}{2.104378in}}%
\pgfpathlineto{\pgfqpoint{2.639065in}{2.104378in}}%
\pgfpathlineto{\pgfqpoint{2.639065in}{2.198752in}}%
\pgfpathlineto{\pgfqpoint{2.784377in}{2.198752in}}%
\pgfpathlineto{\pgfqpoint{2.784377in}{2.104378in}}%
\pgfpathmoveto{\pgfqpoint{2.639065in}{2.198752in}}%
\pgfpathlineto{\pgfqpoint{2.639065in}{2.198752in}}%
\pgfpathlineto{\pgfqpoint{2.639065in}{2.293123in}}%
\pgfpathlineto{\pgfqpoint{2.784377in}{2.293123in}}%
\pgfpathlineto{\pgfqpoint{2.784377in}{2.198752in}}%
\pgfpathmoveto{\pgfqpoint{2.639065in}{2.293123in}}%
\pgfpathlineto{\pgfqpoint{2.639065in}{2.293123in}}%
\pgfpathlineto{\pgfqpoint{2.639065in}{2.387497in}}%
\pgfpathlineto{\pgfqpoint{2.784377in}{2.387497in}}%
\pgfpathlineto{\pgfqpoint{2.784377in}{2.293123in}}%
\pgfpathmoveto{\pgfqpoint{2.639065in}{2.387497in}}%
\pgfpathlineto{\pgfqpoint{2.639065in}{2.387497in}}%
\pgfpathlineto{\pgfqpoint{2.639065in}{2.481873in}}%
\pgfpathlineto{\pgfqpoint{2.784377in}{2.481873in}}%
\pgfpathlineto{\pgfqpoint{2.784377in}{2.387497in}}%
\pgfpathmoveto{\pgfqpoint{2.639065in}{2.481873in}}%
\pgfpathlineto{\pgfqpoint{2.639065in}{2.481873in}}%
\pgfpathlineto{\pgfqpoint{2.639065in}{2.576252in}}%
\pgfpathlineto{\pgfqpoint{2.784377in}{2.576252in}}%
\pgfpathlineto{\pgfqpoint{2.784377in}{2.481873in}}%
\pgfpathmoveto{\pgfqpoint{2.639065in}{2.576252in}}%
\pgfpathlineto{\pgfqpoint{2.639065in}{2.576252in}}%
\pgfpathlineto{\pgfqpoint{2.639065in}{2.670624in}}%
\pgfpathlineto{\pgfqpoint{2.784377in}{2.670624in}}%
\pgfpathlineto{\pgfqpoint{2.784377in}{2.576252in}}%
\pgfpathmoveto{\pgfqpoint{2.639065in}{2.670624in}}%
\pgfpathlineto{\pgfqpoint{2.639065in}{2.670624in}}%
\pgfpathlineto{\pgfqpoint{2.639065in}{2.765002in}}%
\pgfpathlineto{\pgfqpoint{2.784377in}{2.765002in}}%
\pgfpathlineto{\pgfqpoint{2.784377in}{2.670624in}}%
\pgfpathmoveto{\pgfqpoint{2.639065in}{2.765002in}}%
\pgfpathlineto{\pgfqpoint{2.639065in}{2.765002in}}%
\pgfpathlineto{\pgfqpoint{2.639065in}{2.859376in}}%
\pgfpathlineto{\pgfqpoint{2.784377in}{2.859376in}}%
\pgfpathlineto{\pgfqpoint{2.784377in}{2.765002in}}%
\pgfpathmoveto{\pgfqpoint{2.639065in}{2.859376in}}%
\pgfpathlineto{\pgfqpoint{2.639065in}{2.859376in}}%
\pgfpathlineto{\pgfqpoint{2.639065in}{2.953748in}}%
\pgfpathlineto{\pgfqpoint{2.784377in}{2.953748in}}%
\pgfpathlineto{\pgfqpoint{2.784377in}{2.859376in}}%
\pgfpathmoveto{\pgfqpoint{2.639065in}{2.953748in}}%
\pgfpathlineto{\pgfqpoint{2.639065in}{2.953748in}}%
\pgfpathlineto{\pgfqpoint{2.639065in}{3.048126in}}%
\pgfpathlineto{\pgfqpoint{2.784377in}{3.048126in}}%
\pgfpathlineto{\pgfqpoint{2.784377in}{2.953748in}}%
\pgfpathmoveto{\pgfqpoint{2.639065in}{3.048126in}}%
\pgfpathlineto{\pgfqpoint{2.639065in}{3.048126in}}%
\pgfpathlineto{\pgfqpoint{2.639065in}{3.142501in}}%
\pgfpathlineto{\pgfqpoint{2.784377in}{3.142501in}}%
\pgfpathlineto{\pgfqpoint{2.784377in}{3.048126in}}%
\pgfpathmoveto{\pgfqpoint{2.639065in}{3.142501in}}%
\pgfpathlineto{\pgfqpoint{2.639065in}{3.142501in}}%
\pgfpathlineto{\pgfqpoint{2.639065in}{3.236873in}}%
\pgfpathlineto{\pgfqpoint{2.784377in}{3.236873in}}%
\pgfpathlineto{\pgfqpoint{2.784377in}{3.142501in}}%
\pgfpathmoveto{\pgfqpoint{2.639065in}{3.236873in}}%
\pgfpathlineto{\pgfqpoint{2.639065in}{3.236873in}}%
\pgfpathlineto{\pgfqpoint{2.639065in}{3.331249in}}%
\pgfpathlineto{\pgfqpoint{2.784377in}{3.331249in}}%
\pgfpathlineto{\pgfqpoint{2.784377in}{3.236873in}}%
\pgfpathmoveto{\pgfqpoint{2.639065in}{3.331249in}}%
\pgfpathlineto{\pgfqpoint{2.639065in}{3.331249in}}%
\pgfpathlineto{\pgfqpoint{2.639065in}{3.425625in}}%
\pgfpathlineto{\pgfqpoint{2.784377in}{3.425625in}}%
\pgfpathlineto{\pgfqpoint{2.784377in}{3.331249in}}%
\pgfpathmoveto{\pgfqpoint{2.639065in}{3.425625in}}%
\pgfpathlineto{\pgfqpoint{2.639065in}{3.425625in}}%
\pgfpathlineto{\pgfqpoint{2.639065in}{3.519999in}}%
\pgfpathlineto{\pgfqpoint{2.784377in}{3.519999in}}%
\pgfpathlineto{\pgfqpoint{2.784377in}{3.425625in}}%
\pgfpathmoveto{\pgfqpoint{2.784377in}{2.009997in}}%
\pgfpathlineto{\pgfqpoint{2.784377in}{2.009997in}}%
\pgfpathlineto{\pgfqpoint{2.784377in}{2.104378in}}%
\pgfpathlineto{\pgfqpoint{2.929691in}{2.104378in}}%
\pgfpathlineto{\pgfqpoint{2.929691in}{2.009997in}}%
\pgfpathmoveto{\pgfqpoint{2.784377in}{2.104378in}}%
\pgfpathlineto{\pgfqpoint{2.784377in}{2.104378in}}%
\pgfpathlineto{\pgfqpoint{2.784377in}{2.198752in}}%
\pgfpathlineto{\pgfqpoint{2.929691in}{2.198752in}}%
\pgfpathlineto{\pgfqpoint{2.929691in}{2.104378in}}%
\pgfpathmoveto{\pgfqpoint{2.784377in}{2.198752in}}%
\pgfpathlineto{\pgfqpoint{2.784377in}{2.198752in}}%
\pgfpathlineto{\pgfqpoint{2.784377in}{2.293123in}}%
\pgfpathlineto{\pgfqpoint{2.929691in}{2.293123in}}%
\pgfpathlineto{\pgfqpoint{2.929691in}{2.198752in}}%
\pgfpathmoveto{\pgfqpoint{2.784377in}{2.293123in}}%
\pgfpathlineto{\pgfqpoint{2.784377in}{2.293123in}}%
\pgfpathlineto{\pgfqpoint{2.784377in}{2.387497in}}%
\pgfpathlineto{\pgfqpoint{2.929691in}{2.387497in}}%
\pgfpathlineto{\pgfqpoint{2.929691in}{2.293123in}}%
\pgfpathmoveto{\pgfqpoint{2.784377in}{2.387497in}}%
\pgfpathlineto{\pgfqpoint{2.784377in}{2.387497in}}%
\pgfpathlineto{\pgfqpoint{2.784377in}{2.481873in}}%
\pgfpathlineto{\pgfqpoint{2.929691in}{2.481873in}}%
\pgfpathlineto{\pgfqpoint{2.929691in}{2.387497in}}%
\pgfpathmoveto{\pgfqpoint{2.784377in}{2.481873in}}%
\pgfpathlineto{\pgfqpoint{2.784377in}{2.481873in}}%
\pgfpathlineto{\pgfqpoint{2.784377in}{2.576252in}}%
\pgfpathlineto{\pgfqpoint{2.929691in}{2.576252in}}%
\pgfpathlineto{\pgfqpoint{2.929691in}{2.481873in}}%
\pgfpathmoveto{\pgfqpoint{2.784377in}{2.576252in}}%
\pgfpathlineto{\pgfqpoint{2.784377in}{2.576252in}}%
\pgfpathlineto{\pgfqpoint{2.784377in}{2.670624in}}%
\pgfpathlineto{\pgfqpoint{2.929691in}{2.670624in}}%
\pgfpathlineto{\pgfqpoint{2.929691in}{2.576252in}}%
\pgfpathmoveto{\pgfqpoint{2.784377in}{2.670624in}}%
\pgfpathlineto{\pgfqpoint{2.784377in}{2.670624in}}%
\pgfpathlineto{\pgfqpoint{2.784377in}{2.765002in}}%
\pgfpathlineto{\pgfqpoint{2.929691in}{2.765002in}}%
\pgfpathlineto{\pgfqpoint{2.929691in}{2.670624in}}%
\pgfpathmoveto{\pgfqpoint{2.784377in}{2.765002in}}%
\pgfpathlineto{\pgfqpoint{2.784377in}{2.765002in}}%
\pgfpathlineto{\pgfqpoint{2.784377in}{2.859376in}}%
\pgfpathlineto{\pgfqpoint{2.929691in}{2.859376in}}%
\pgfpathlineto{\pgfqpoint{2.929691in}{2.765002in}}%
\pgfpathmoveto{\pgfqpoint{2.784377in}{2.859376in}}%
\pgfpathlineto{\pgfqpoint{2.784377in}{2.859376in}}%
\pgfpathlineto{\pgfqpoint{2.784377in}{2.953748in}}%
\pgfpathlineto{\pgfqpoint{2.929691in}{2.953748in}}%
\pgfpathlineto{\pgfqpoint{2.929691in}{2.859376in}}%
\pgfpathmoveto{\pgfqpoint{2.784377in}{2.953748in}}%
\pgfpathlineto{\pgfqpoint{2.784377in}{2.953748in}}%
\pgfpathlineto{\pgfqpoint{2.784377in}{3.048126in}}%
\pgfpathlineto{\pgfqpoint{2.929691in}{3.048126in}}%
\pgfpathlineto{\pgfqpoint{2.929691in}{2.953748in}}%
\pgfpathmoveto{\pgfqpoint{2.784377in}{3.048126in}}%
\pgfpathlineto{\pgfqpoint{2.784377in}{3.048126in}}%
\pgfpathlineto{\pgfqpoint{2.784377in}{3.142501in}}%
\pgfpathlineto{\pgfqpoint{2.929691in}{3.142501in}}%
\pgfpathlineto{\pgfqpoint{2.929691in}{3.048126in}}%
\pgfpathmoveto{\pgfqpoint{2.784377in}{3.142501in}}%
\pgfpathlineto{\pgfqpoint{2.784377in}{3.142501in}}%
\pgfpathlineto{\pgfqpoint{2.784377in}{3.236873in}}%
\pgfpathlineto{\pgfqpoint{2.929691in}{3.236873in}}%
\pgfpathlineto{\pgfqpoint{2.929691in}{3.142501in}}%
\pgfpathmoveto{\pgfqpoint{2.784377in}{3.236873in}}%
\pgfpathlineto{\pgfqpoint{2.784377in}{3.236873in}}%
\pgfpathlineto{\pgfqpoint{2.784377in}{3.331249in}}%
\pgfpathlineto{\pgfqpoint{2.929691in}{3.331249in}}%
\pgfpathlineto{\pgfqpoint{2.929691in}{3.236873in}}%
\pgfpathmoveto{\pgfqpoint{2.929691in}{2.104378in}}%
\pgfpathlineto{\pgfqpoint{2.929691in}{2.104378in}}%
\pgfpathlineto{\pgfqpoint{2.929691in}{2.198752in}}%
\pgfpathlineto{\pgfqpoint{3.074996in}{2.198752in}}%
\pgfpathlineto{\pgfqpoint{3.074996in}{2.104378in}}%
\pgfpathmoveto{\pgfqpoint{2.929691in}{2.198752in}}%
\pgfpathlineto{\pgfqpoint{2.929691in}{2.198752in}}%
\pgfpathlineto{\pgfqpoint{2.929691in}{2.293123in}}%
\pgfpathlineto{\pgfqpoint{3.074996in}{2.293123in}}%
\pgfpathlineto{\pgfqpoint{3.074996in}{2.198752in}}%
\pgfpathmoveto{\pgfqpoint{2.929691in}{2.293123in}}%
\pgfpathlineto{\pgfqpoint{2.929691in}{2.293123in}}%
\pgfpathlineto{\pgfqpoint{2.929691in}{2.387497in}}%
\pgfpathlineto{\pgfqpoint{3.074996in}{2.387497in}}%
\pgfpathlineto{\pgfqpoint{3.074996in}{2.293123in}}%
\pgfpathmoveto{\pgfqpoint{2.929691in}{2.387497in}}%
\pgfpathlineto{\pgfqpoint{2.929691in}{2.387497in}}%
\pgfpathlineto{\pgfqpoint{2.929691in}{2.481873in}}%
\pgfpathlineto{\pgfqpoint{3.074996in}{2.481873in}}%
\pgfpathlineto{\pgfqpoint{3.074996in}{2.387497in}}%
\pgfpathmoveto{\pgfqpoint{2.929691in}{2.481873in}}%
\pgfpathlineto{\pgfqpoint{2.929691in}{2.481873in}}%
\pgfpathlineto{\pgfqpoint{2.929691in}{2.576252in}}%
\pgfpathlineto{\pgfqpoint{3.074996in}{2.576252in}}%
\pgfpathlineto{\pgfqpoint{3.074996in}{2.481873in}}%
\pgfpathmoveto{\pgfqpoint{2.929691in}{2.576252in}}%
\pgfpathlineto{\pgfqpoint{2.929691in}{2.576252in}}%
\pgfpathlineto{\pgfqpoint{2.929691in}{2.670624in}}%
\pgfpathlineto{\pgfqpoint{3.074996in}{2.670624in}}%
\pgfpathlineto{\pgfqpoint{3.074996in}{2.576252in}}%
\pgfpathmoveto{\pgfqpoint{2.929691in}{2.670624in}}%
\pgfpathlineto{\pgfqpoint{2.929691in}{2.670624in}}%
\pgfpathlineto{\pgfqpoint{2.929691in}{2.765002in}}%
\pgfpathlineto{\pgfqpoint{3.074996in}{2.765002in}}%
\pgfpathlineto{\pgfqpoint{3.074996in}{2.670624in}}%
\pgfpathmoveto{\pgfqpoint{2.929691in}{2.765002in}}%
\pgfpathlineto{\pgfqpoint{2.929691in}{2.765002in}}%
\pgfpathlineto{\pgfqpoint{2.929691in}{2.859376in}}%
\pgfpathlineto{\pgfqpoint{3.074996in}{2.859376in}}%
\pgfpathlineto{\pgfqpoint{3.074996in}{2.765002in}}%
\pgfpathmoveto{\pgfqpoint{2.929691in}{2.859376in}}%
\pgfpathlineto{\pgfqpoint{2.929691in}{2.859376in}}%
\pgfpathlineto{\pgfqpoint{2.929691in}{2.953748in}}%
\pgfpathlineto{\pgfqpoint{3.074996in}{2.953748in}}%
\pgfpathlineto{\pgfqpoint{3.074996in}{2.859376in}}%
\pgfpathmoveto{\pgfqpoint{2.929691in}{2.953748in}}%
\pgfpathlineto{\pgfqpoint{2.929691in}{2.953748in}}%
\pgfpathlineto{\pgfqpoint{2.929691in}{3.048126in}}%
\pgfpathlineto{\pgfqpoint{3.074996in}{3.048126in}}%
\pgfpathlineto{\pgfqpoint{3.074996in}{2.953748in}}%
\pgfpathmoveto{\pgfqpoint{2.929691in}{3.048126in}}%
\pgfpathlineto{\pgfqpoint{2.929691in}{3.048126in}}%
\pgfpathlineto{\pgfqpoint{2.929691in}{3.142501in}}%
\pgfpathlineto{\pgfqpoint{3.074996in}{3.142501in}}%
\pgfpathlineto{\pgfqpoint{3.074996in}{3.048126in}}%
\pgfpathmoveto{\pgfqpoint{3.074996in}{2.104378in}}%
\pgfpathlineto{\pgfqpoint{3.074996in}{2.104378in}}%
\pgfpathlineto{\pgfqpoint{3.074996in}{2.198752in}}%
\pgfpathlineto{\pgfqpoint{3.220316in}{2.198752in}}%
\pgfpathlineto{\pgfqpoint{3.220316in}{2.104378in}}%
\pgfpathmoveto{\pgfqpoint{3.074996in}{2.198752in}}%
\pgfpathlineto{\pgfqpoint{3.074996in}{2.198752in}}%
\pgfpathlineto{\pgfqpoint{3.074996in}{2.293123in}}%
\pgfpathlineto{\pgfqpoint{3.220316in}{2.293123in}}%
\pgfpathlineto{\pgfqpoint{3.220316in}{2.198752in}}%
\pgfpathmoveto{\pgfqpoint{3.074996in}{2.293123in}}%
\pgfpathlineto{\pgfqpoint{3.074996in}{2.293123in}}%
\pgfpathlineto{\pgfqpoint{3.074996in}{2.387497in}}%
\pgfpathlineto{\pgfqpoint{3.220316in}{2.387497in}}%
\pgfpathlineto{\pgfqpoint{3.220316in}{2.293123in}}%
\pgfpathmoveto{\pgfqpoint{3.074996in}{2.387497in}}%
\pgfpathlineto{\pgfqpoint{3.074996in}{2.387497in}}%
\pgfpathlineto{\pgfqpoint{3.074996in}{2.481873in}}%
\pgfpathlineto{\pgfqpoint{3.220316in}{2.481873in}}%
\pgfpathlineto{\pgfqpoint{3.220316in}{2.387497in}}%
\pgfpathmoveto{\pgfqpoint{3.074996in}{2.481873in}}%
\pgfpathlineto{\pgfqpoint{3.074996in}{2.481873in}}%
\pgfpathlineto{\pgfqpoint{3.074996in}{2.576252in}}%
\pgfpathlineto{\pgfqpoint{3.220316in}{2.576252in}}%
\pgfpathlineto{\pgfqpoint{3.220316in}{2.481873in}}%
\pgfpathmoveto{\pgfqpoint{3.074996in}{2.576252in}}%
\pgfpathlineto{\pgfqpoint{3.074996in}{2.576252in}}%
\pgfpathlineto{\pgfqpoint{3.074996in}{2.670624in}}%
\pgfpathlineto{\pgfqpoint{3.220316in}{2.670624in}}%
\pgfpathlineto{\pgfqpoint{3.220316in}{2.576252in}}%
\pgfpathmoveto{\pgfqpoint{3.074996in}{2.670624in}}%
\pgfpathlineto{\pgfqpoint{3.074996in}{2.670624in}}%
\pgfpathlineto{\pgfqpoint{3.074996in}{2.765002in}}%
\pgfpathlineto{\pgfqpoint{3.220316in}{2.765002in}}%
\pgfpathlineto{\pgfqpoint{3.220316in}{2.670624in}}%
\pgfpathmoveto{\pgfqpoint{3.074996in}{2.765002in}}%
\pgfpathlineto{\pgfqpoint{3.074996in}{2.765002in}}%
\pgfpathlineto{\pgfqpoint{3.074996in}{2.859376in}}%
\pgfpathlineto{\pgfqpoint{3.220316in}{2.859376in}}%
\pgfpathlineto{\pgfqpoint{3.220316in}{2.765002in}}%
\pgfpathmoveto{\pgfqpoint{3.074996in}{2.859376in}}%
\pgfpathlineto{\pgfqpoint{3.074996in}{2.859376in}}%
\pgfpathlineto{\pgfqpoint{3.074996in}{2.953748in}}%
\pgfpathlineto{\pgfqpoint{3.220316in}{2.953748in}}%
\pgfpathlineto{\pgfqpoint{3.220316in}{2.859376in}}%
\pgfpathmoveto{\pgfqpoint{3.220316in}{2.198752in}}%
\pgfpathlineto{\pgfqpoint{3.220316in}{2.198752in}}%
\pgfpathlineto{\pgfqpoint{3.220316in}{2.293123in}}%
\pgfpathlineto{\pgfqpoint{3.365622in}{2.293123in}}%
\pgfpathlineto{\pgfqpoint{3.365622in}{2.198752in}}%
\pgfpathmoveto{\pgfqpoint{3.220316in}{2.293123in}}%
\pgfpathlineto{\pgfqpoint{3.220316in}{2.293123in}}%
\pgfpathlineto{\pgfqpoint{3.220316in}{2.387497in}}%
\pgfpathlineto{\pgfqpoint{3.365622in}{2.387497in}}%
\pgfpathlineto{\pgfqpoint{3.365622in}{2.293123in}}%
\pgfpathmoveto{\pgfqpoint{3.220316in}{2.387497in}}%
\pgfpathlineto{\pgfqpoint{3.220316in}{2.387497in}}%
\pgfpathlineto{\pgfqpoint{3.220316in}{2.481873in}}%
\pgfpathlineto{\pgfqpoint{3.365622in}{2.481873in}}%
\pgfpathlineto{\pgfqpoint{3.365622in}{2.387497in}}%
\pgfpathmoveto{\pgfqpoint{3.220316in}{2.481873in}}%
\pgfpathlineto{\pgfqpoint{3.220316in}{2.481873in}}%
\pgfpathlineto{\pgfqpoint{3.220316in}{2.576252in}}%
\pgfpathlineto{\pgfqpoint{3.365622in}{2.576252in}}%
\pgfpathlineto{\pgfqpoint{3.365622in}{2.481873in}}%
\pgfpathmoveto{\pgfqpoint{3.220316in}{2.576252in}}%
\pgfpathlineto{\pgfqpoint{3.220316in}{2.576252in}}%
\pgfpathlineto{\pgfqpoint{3.220316in}{2.670624in}}%
\pgfpathlineto{\pgfqpoint{3.365622in}{2.670624in}}%
\pgfpathlineto{\pgfqpoint{3.365622in}{2.576252in}}%
\pgfpathmoveto{\pgfqpoint{3.220316in}{2.670624in}}%
\pgfpathlineto{\pgfqpoint{3.220316in}{2.670624in}}%
\pgfpathlineto{\pgfqpoint{3.220316in}{2.765002in}}%
\pgfpathlineto{\pgfqpoint{3.365622in}{2.765002in}}%
\pgfpathlineto{\pgfqpoint{3.365622in}{2.670624in}}%
\pgfpathmoveto{\pgfqpoint{3.365622in}{2.387497in}}%
\pgfpathlineto{\pgfqpoint{3.365622in}{2.387497in}}%
\pgfpathlineto{\pgfqpoint{3.365622in}{2.481873in}}%
\pgfpathlineto{\pgfqpoint{3.510941in}{2.481873in}}%
\pgfpathlineto{\pgfqpoint{3.510941in}{2.387497in}}%
\pgfpathmoveto{\pgfqpoint{3.365622in}{2.481873in}}%
\pgfpathlineto{\pgfqpoint{3.365622in}{2.481873in}}%
\pgfpathlineto{\pgfqpoint{3.365622in}{2.576252in}}%
\pgfpathlineto{\pgfqpoint{3.510941in}{2.576252in}}%
\pgfpathlineto{\pgfqpoint{3.510941in}{2.481873in}}%
\pgfpathmoveto{\pgfqpoint{0.895314in}{0.641564in}}%
\pgfpathlineto{\pgfqpoint{0.895314in}{0.641564in}}%
\pgfpathlineto{\pgfqpoint{0.895314in}{0.688752in}}%
\pgfpathlineto{\pgfqpoint{0.967968in}{0.688752in}}%
\pgfpathlineto{\pgfqpoint{0.967968in}{0.641564in}}%
\pgfpathmoveto{\pgfqpoint{1.040623in}{0.735937in}}%
\pgfpathlineto{\pgfqpoint{1.040623in}{0.735937in}}%
\pgfpathlineto{\pgfqpoint{1.040623in}{0.783123in}}%
\pgfpathlineto{\pgfqpoint{1.113279in}{0.783123in}}%
\pgfpathlineto{\pgfqpoint{1.113279in}{0.735937in}}%
\pgfpathmoveto{\pgfqpoint{1.040623in}{0.783123in}}%
\pgfpathlineto{\pgfqpoint{1.040623in}{0.783123in}}%
\pgfpathlineto{\pgfqpoint{1.040623in}{0.830311in}}%
\pgfpathlineto{\pgfqpoint{1.113279in}{0.830311in}}%
\pgfpathlineto{\pgfqpoint{1.113279in}{0.783123in}}%
\pgfpathmoveto{\pgfqpoint{1.040623in}{0.830311in}}%
\pgfpathlineto{\pgfqpoint{1.040623in}{0.830311in}}%
\pgfpathlineto{\pgfqpoint{1.040623in}{0.877498in}}%
\pgfpathlineto{\pgfqpoint{1.113279in}{0.877498in}}%
\pgfpathlineto{\pgfqpoint{1.113279in}{0.830311in}}%
\pgfpathmoveto{\pgfqpoint{1.113279in}{0.830311in}}%
\pgfpathlineto{\pgfqpoint{1.113279in}{0.830311in}}%
\pgfpathlineto{\pgfqpoint{1.113279in}{0.877498in}}%
\pgfpathlineto{\pgfqpoint{1.185936in}{0.877498in}}%
\pgfpathlineto{\pgfqpoint{1.185936in}{0.830311in}}%
\pgfpathmoveto{\pgfqpoint{1.185936in}{0.877498in}}%
\pgfpathlineto{\pgfqpoint{1.185936in}{0.877498in}}%
\pgfpathlineto{\pgfqpoint{1.185936in}{0.924687in}}%
\pgfpathlineto{\pgfqpoint{1.258592in}{0.924687in}}%
\pgfpathlineto{\pgfqpoint{1.258592in}{0.877498in}}%
\pgfpathmoveto{\pgfqpoint{1.185936in}{0.924687in}}%
\pgfpathlineto{\pgfqpoint{1.185936in}{0.924687in}}%
\pgfpathlineto{\pgfqpoint{1.185936in}{0.971875in}}%
\pgfpathlineto{\pgfqpoint{1.258592in}{0.971875in}}%
\pgfpathlineto{\pgfqpoint{1.258592in}{0.924687in}}%
\pgfpathmoveto{\pgfqpoint{1.258592in}{0.924687in}}%
\pgfpathlineto{\pgfqpoint{1.258592in}{0.924687in}}%
\pgfpathlineto{\pgfqpoint{1.258592in}{0.971875in}}%
\pgfpathlineto{\pgfqpoint{1.331249in}{0.971875in}}%
\pgfpathlineto{\pgfqpoint{1.331249in}{0.924687in}}%
\pgfpathmoveto{\pgfqpoint{1.331249in}{0.924687in}}%
\pgfpathlineto{\pgfqpoint{1.331249in}{0.924687in}}%
\pgfpathlineto{\pgfqpoint{1.331249in}{0.971875in}}%
\pgfpathlineto{\pgfqpoint{1.403904in}{0.971875in}}%
\pgfpathlineto{\pgfqpoint{1.403904in}{0.924687in}}%
\pgfpathmoveto{\pgfqpoint{1.476559in}{1.019062in}}%
\pgfpathlineto{\pgfqpoint{1.476559in}{1.019062in}}%
\pgfpathlineto{\pgfqpoint{1.476559in}{1.066250in}}%
\pgfpathlineto{\pgfqpoint{1.549216in}{1.066250in}}%
\pgfpathlineto{\pgfqpoint{1.549216in}{1.019062in}}%
\pgfpathmoveto{\pgfqpoint{1.621874in}{1.160623in}}%
\pgfpathlineto{\pgfqpoint{1.621874in}{1.160623in}}%
\pgfpathlineto{\pgfqpoint{1.621874in}{1.207811in}}%
\pgfpathlineto{\pgfqpoint{1.694530in}{1.207811in}}%
\pgfpathlineto{\pgfqpoint{1.694530in}{1.160623in}}%
\pgfpathmoveto{\pgfqpoint{1.621874in}{1.207811in}}%
\pgfpathlineto{\pgfqpoint{1.621874in}{1.207811in}}%
\pgfpathlineto{\pgfqpoint{1.621874in}{1.254999in}}%
\pgfpathlineto{\pgfqpoint{1.694530in}{1.254999in}}%
\pgfpathlineto{\pgfqpoint{1.694530in}{1.207811in}}%
\pgfpathmoveto{\pgfqpoint{1.694530in}{1.207811in}}%
\pgfpathlineto{\pgfqpoint{1.694530in}{1.207811in}}%
\pgfpathlineto{\pgfqpoint{1.694530in}{1.254999in}}%
\pgfpathlineto{\pgfqpoint{1.767187in}{1.254999in}}%
\pgfpathlineto{\pgfqpoint{1.767187in}{1.207811in}}%
\pgfpathmoveto{\pgfqpoint{1.767187in}{1.254999in}}%
\pgfpathlineto{\pgfqpoint{1.767187in}{1.254999in}}%
\pgfpathlineto{\pgfqpoint{1.767187in}{1.302188in}}%
\pgfpathlineto{\pgfqpoint{1.839845in}{1.302188in}}%
\pgfpathlineto{\pgfqpoint{1.839845in}{1.254999in}}%
\pgfpathmoveto{\pgfqpoint{1.767187in}{1.302188in}}%
\pgfpathlineto{\pgfqpoint{1.767187in}{1.302188in}}%
\pgfpathlineto{\pgfqpoint{1.767187in}{1.349376in}}%
\pgfpathlineto{\pgfqpoint{1.839845in}{1.349376in}}%
\pgfpathlineto{\pgfqpoint{1.839845in}{1.302188in}}%
\pgfpathmoveto{\pgfqpoint{1.839845in}{1.302188in}}%
\pgfpathlineto{\pgfqpoint{1.839845in}{1.302188in}}%
\pgfpathlineto{\pgfqpoint{1.839845in}{1.349376in}}%
\pgfpathlineto{\pgfqpoint{1.912503in}{1.349376in}}%
\pgfpathlineto{\pgfqpoint{1.912503in}{1.302188in}}%
\pgfpathmoveto{\pgfqpoint{2.057810in}{1.396564in}}%
\pgfpathlineto{\pgfqpoint{2.057810in}{1.396564in}}%
\pgfpathlineto{\pgfqpoint{2.057810in}{1.443752in}}%
\pgfpathlineto{\pgfqpoint{2.130466in}{1.443752in}}%
\pgfpathlineto{\pgfqpoint{2.130466in}{1.396564in}}%
\pgfpathmoveto{\pgfqpoint{2.203121in}{1.490938in}}%
\pgfpathlineto{\pgfqpoint{2.203121in}{1.490938in}}%
\pgfpathlineto{\pgfqpoint{2.203121in}{1.538125in}}%
\pgfpathlineto{\pgfqpoint{2.275780in}{1.538125in}}%
\pgfpathlineto{\pgfqpoint{2.275780in}{1.490938in}}%
\pgfpathmoveto{\pgfqpoint{2.203121in}{1.538125in}}%
\pgfpathlineto{\pgfqpoint{2.203121in}{1.538125in}}%
\pgfpathlineto{\pgfqpoint{2.203121in}{1.585311in}}%
\pgfpathlineto{\pgfqpoint{2.275780in}{1.585311in}}%
\pgfpathlineto{\pgfqpoint{2.275780in}{1.538125in}}%
\pgfpathmoveto{\pgfqpoint{2.203121in}{1.585311in}}%
\pgfpathlineto{\pgfqpoint{2.203121in}{1.585311in}}%
\pgfpathlineto{\pgfqpoint{2.203121in}{1.632498in}}%
\pgfpathlineto{\pgfqpoint{2.275780in}{1.632498in}}%
\pgfpathlineto{\pgfqpoint{2.275780in}{1.585311in}}%
\pgfpathmoveto{\pgfqpoint{2.275780in}{1.585311in}}%
\pgfpathlineto{\pgfqpoint{2.275780in}{1.585311in}}%
\pgfpathlineto{\pgfqpoint{2.275780in}{1.632498in}}%
\pgfpathlineto{\pgfqpoint{2.348439in}{1.632498in}}%
\pgfpathlineto{\pgfqpoint{2.348439in}{1.585311in}}%
\pgfpathmoveto{\pgfqpoint{2.348439in}{1.632498in}}%
\pgfpathlineto{\pgfqpoint{2.348439in}{1.632498in}}%
\pgfpathlineto{\pgfqpoint{2.348439in}{1.679686in}}%
\pgfpathlineto{\pgfqpoint{2.421093in}{1.679686in}}%
\pgfpathlineto{\pgfqpoint{2.421093in}{1.632498in}}%
\pgfpathmoveto{\pgfqpoint{2.348439in}{1.679686in}}%
\pgfpathlineto{\pgfqpoint{2.348439in}{1.679686in}}%
\pgfpathlineto{\pgfqpoint{2.348439in}{1.726874in}}%
\pgfpathlineto{\pgfqpoint{2.421093in}{1.726874in}}%
\pgfpathlineto{\pgfqpoint{2.421093in}{1.679686in}}%
\pgfpathmoveto{\pgfqpoint{2.421093in}{1.679686in}}%
\pgfpathlineto{\pgfqpoint{2.421093in}{1.679686in}}%
\pgfpathlineto{\pgfqpoint{2.421093in}{1.726874in}}%
\pgfpathlineto{\pgfqpoint{2.493747in}{1.726874in}}%
\pgfpathlineto{\pgfqpoint{2.493747in}{1.679686in}}%
\pgfpathmoveto{\pgfqpoint{2.493747in}{1.726874in}}%
\pgfpathlineto{\pgfqpoint{2.493747in}{1.726874in}}%
\pgfpathlineto{\pgfqpoint{2.493747in}{1.774062in}}%
\pgfpathlineto{\pgfqpoint{2.566406in}{1.774062in}}%
\pgfpathlineto{\pgfqpoint{2.566406in}{1.726874in}}%
\pgfpathmoveto{\pgfqpoint{2.493747in}{1.774062in}}%
\pgfpathlineto{\pgfqpoint{2.493747in}{1.774062in}}%
\pgfpathlineto{\pgfqpoint{2.493747in}{1.821250in}}%
\pgfpathlineto{\pgfqpoint{2.566406in}{1.821250in}}%
\pgfpathlineto{\pgfqpoint{2.566406in}{1.774062in}}%
\pgfpathmoveto{\pgfqpoint{2.566406in}{1.774062in}}%
\pgfpathlineto{\pgfqpoint{2.566406in}{1.774062in}}%
\pgfpathlineto{\pgfqpoint{2.566406in}{1.821250in}}%
\pgfpathlineto{\pgfqpoint{2.639065in}{1.821250in}}%
\pgfpathlineto{\pgfqpoint{2.639065in}{1.774062in}}%
\pgfpathmoveto{\pgfqpoint{2.639065in}{1.821250in}}%
\pgfpathlineto{\pgfqpoint{2.639065in}{1.821250in}}%
\pgfpathlineto{\pgfqpoint{2.639065in}{1.868438in}}%
\pgfpathlineto{\pgfqpoint{2.711721in}{1.868438in}}%
\pgfpathlineto{\pgfqpoint{2.711721in}{1.821250in}}%
\pgfpathmoveto{\pgfqpoint{2.639065in}{1.868438in}}%
\pgfpathlineto{\pgfqpoint{2.639065in}{1.868438in}}%
\pgfpathlineto{\pgfqpoint{2.639065in}{1.915626in}}%
\pgfpathlineto{\pgfqpoint{2.711721in}{1.915626in}}%
\pgfpathlineto{\pgfqpoint{2.711721in}{1.868438in}}%
\pgfpathmoveto{\pgfqpoint{2.711721in}{1.868438in}}%
\pgfpathlineto{\pgfqpoint{2.711721in}{1.868438in}}%
\pgfpathlineto{\pgfqpoint{2.711721in}{1.915626in}}%
\pgfpathlineto{\pgfqpoint{2.784377in}{1.915626in}}%
\pgfpathlineto{\pgfqpoint{2.784377in}{1.868438in}}%
\pgfpathmoveto{\pgfqpoint{2.784377in}{1.915626in}}%
\pgfpathlineto{\pgfqpoint{2.784377in}{1.915626in}}%
\pgfpathlineto{\pgfqpoint{2.784377in}{1.962812in}}%
\pgfpathlineto{\pgfqpoint{2.857034in}{1.962812in}}%
\pgfpathlineto{\pgfqpoint{2.857034in}{1.915626in}}%
\pgfpathmoveto{\pgfqpoint{2.784377in}{1.962812in}}%
\pgfpathlineto{\pgfqpoint{2.784377in}{1.962812in}}%
\pgfpathlineto{\pgfqpoint{2.784377in}{2.009997in}}%
\pgfpathlineto{\pgfqpoint{2.857034in}{2.009997in}}%
\pgfpathlineto{\pgfqpoint{2.857034in}{1.962812in}}%
\pgfpathmoveto{\pgfqpoint{2.857034in}{1.962812in}}%
\pgfpathlineto{\pgfqpoint{2.857034in}{1.962812in}}%
\pgfpathlineto{\pgfqpoint{2.857034in}{2.009997in}}%
\pgfpathlineto{\pgfqpoint{2.929691in}{2.009997in}}%
\pgfpathlineto{\pgfqpoint{2.929691in}{1.962812in}}%
\pgfpathmoveto{\pgfqpoint{2.784377in}{3.331249in}}%
\pgfpathlineto{\pgfqpoint{2.784377in}{3.331249in}}%
\pgfpathlineto{\pgfqpoint{2.784377in}{3.378437in}}%
\pgfpathlineto{\pgfqpoint{2.857034in}{3.378437in}}%
\pgfpathlineto{\pgfqpoint{2.857034in}{3.331249in}}%
\pgfpathmoveto{\pgfqpoint{2.784377in}{3.378437in}}%
\pgfpathlineto{\pgfqpoint{2.784377in}{3.378437in}}%
\pgfpathlineto{\pgfqpoint{2.784377in}{3.425625in}}%
\pgfpathlineto{\pgfqpoint{2.857034in}{3.425625in}}%
\pgfpathlineto{\pgfqpoint{2.857034in}{3.378437in}}%
\pgfpathmoveto{\pgfqpoint{2.857034in}{3.331249in}}%
\pgfpathlineto{\pgfqpoint{2.857034in}{3.331249in}}%
\pgfpathlineto{\pgfqpoint{2.857034in}{3.378437in}}%
\pgfpathlineto{\pgfqpoint{2.929691in}{3.378437in}}%
\pgfpathlineto{\pgfqpoint{2.929691in}{3.331249in}}%
\pgfpathmoveto{\pgfqpoint{2.784377in}{3.425625in}}%
\pgfpathlineto{\pgfqpoint{2.784377in}{3.425625in}}%
\pgfpathlineto{\pgfqpoint{2.784377in}{3.472812in}}%
\pgfpathlineto{\pgfqpoint{2.857034in}{3.472812in}}%
\pgfpathlineto{\pgfqpoint{2.857034in}{3.425625in}}%
\pgfpathmoveto{\pgfqpoint{2.929691in}{2.009997in}}%
\pgfpathlineto{\pgfqpoint{2.929691in}{2.009997in}}%
\pgfpathlineto{\pgfqpoint{2.929691in}{2.057188in}}%
\pgfpathlineto{\pgfqpoint{3.002344in}{2.057188in}}%
\pgfpathlineto{\pgfqpoint{3.002344in}{2.009997in}}%
\pgfpathmoveto{\pgfqpoint{2.929691in}{2.057188in}}%
\pgfpathlineto{\pgfqpoint{2.929691in}{2.057188in}}%
\pgfpathlineto{\pgfqpoint{2.929691in}{2.104378in}}%
\pgfpathlineto{\pgfqpoint{3.002344in}{2.104378in}}%
\pgfpathlineto{\pgfqpoint{3.002344in}{2.057188in}}%
\pgfpathmoveto{\pgfqpoint{3.002344in}{2.057188in}}%
\pgfpathlineto{\pgfqpoint{3.002344in}{2.057188in}}%
\pgfpathlineto{\pgfqpoint{3.002344in}{2.104378in}}%
\pgfpathlineto{\pgfqpoint{3.074996in}{2.104378in}}%
\pgfpathlineto{\pgfqpoint{3.074996in}{2.057188in}}%
\pgfpathmoveto{\pgfqpoint{2.929691in}{3.142501in}}%
\pgfpathlineto{\pgfqpoint{2.929691in}{3.142501in}}%
\pgfpathlineto{\pgfqpoint{2.929691in}{3.189687in}}%
\pgfpathlineto{\pgfqpoint{3.002344in}{3.189687in}}%
\pgfpathlineto{\pgfqpoint{3.002344in}{3.142501in}}%
\pgfpathmoveto{\pgfqpoint{2.929691in}{3.189687in}}%
\pgfpathlineto{\pgfqpoint{2.929691in}{3.189687in}}%
\pgfpathlineto{\pgfqpoint{2.929691in}{3.236873in}}%
\pgfpathlineto{\pgfqpoint{3.002344in}{3.236873in}}%
\pgfpathlineto{\pgfqpoint{3.002344in}{3.189687in}}%
\pgfpathmoveto{\pgfqpoint{3.002344in}{3.142501in}}%
\pgfpathlineto{\pgfqpoint{3.002344in}{3.142501in}}%
\pgfpathlineto{\pgfqpoint{3.002344in}{3.189687in}}%
\pgfpathlineto{\pgfqpoint{3.074996in}{3.189687in}}%
\pgfpathlineto{\pgfqpoint{3.074996in}{3.142501in}}%
\pgfpathmoveto{\pgfqpoint{2.929691in}{3.236873in}}%
\pgfpathlineto{\pgfqpoint{2.929691in}{3.236873in}}%
\pgfpathlineto{\pgfqpoint{2.929691in}{3.284061in}}%
\pgfpathlineto{\pgfqpoint{3.002344in}{3.284061in}}%
\pgfpathlineto{\pgfqpoint{3.002344in}{3.236873in}}%
\pgfpathmoveto{\pgfqpoint{3.074996in}{2.057188in}}%
\pgfpathlineto{\pgfqpoint{3.074996in}{2.057188in}}%
\pgfpathlineto{\pgfqpoint{3.074996in}{2.104378in}}%
\pgfpathlineto{\pgfqpoint{3.147656in}{2.104378in}}%
\pgfpathlineto{\pgfqpoint{3.147656in}{2.057188in}}%
\pgfpathmoveto{\pgfqpoint{3.074996in}{2.953748in}}%
\pgfpathlineto{\pgfqpoint{3.074996in}{2.953748in}}%
\pgfpathlineto{\pgfqpoint{3.074996in}{3.000937in}}%
\pgfpathlineto{\pgfqpoint{3.147656in}{3.000937in}}%
\pgfpathlineto{\pgfqpoint{3.147656in}{2.953748in}}%
\pgfpathmoveto{\pgfqpoint{3.074996in}{3.000937in}}%
\pgfpathlineto{\pgfqpoint{3.074996in}{3.000937in}}%
\pgfpathlineto{\pgfqpoint{3.074996in}{3.048126in}}%
\pgfpathlineto{\pgfqpoint{3.147656in}{3.048126in}}%
\pgfpathlineto{\pgfqpoint{3.147656in}{3.000937in}}%
\pgfpathmoveto{\pgfqpoint{3.147656in}{2.953748in}}%
\pgfpathlineto{\pgfqpoint{3.147656in}{2.953748in}}%
\pgfpathlineto{\pgfqpoint{3.147656in}{3.000937in}}%
\pgfpathlineto{\pgfqpoint{3.220316in}{3.000937in}}%
\pgfpathlineto{\pgfqpoint{3.220316in}{2.953748in}}%
\pgfpathmoveto{\pgfqpoint{3.074996in}{3.048126in}}%
\pgfpathlineto{\pgfqpoint{3.074996in}{3.048126in}}%
\pgfpathlineto{\pgfqpoint{3.074996in}{3.095314in}}%
\pgfpathlineto{\pgfqpoint{3.147656in}{3.095314in}}%
\pgfpathlineto{\pgfqpoint{3.147656in}{3.048126in}}%
\pgfpathmoveto{\pgfqpoint{3.220316in}{2.151565in}}%
\pgfpathlineto{\pgfqpoint{3.220316in}{2.151565in}}%
\pgfpathlineto{\pgfqpoint{3.220316in}{2.198752in}}%
\pgfpathlineto{\pgfqpoint{3.292969in}{2.198752in}}%
\pgfpathlineto{\pgfqpoint{3.292969in}{2.151565in}}%
\pgfpathmoveto{\pgfqpoint{3.220316in}{2.765002in}}%
\pgfpathlineto{\pgfqpoint{3.220316in}{2.765002in}}%
\pgfpathlineto{\pgfqpoint{3.220316in}{2.812189in}}%
\pgfpathlineto{\pgfqpoint{3.292969in}{2.812189in}}%
\pgfpathlineto{\pgfqpoint{3.292969in}{2.765002in}}%
\pgfpathmoveto{\pgfqpoint{3.220316in}{2.812189in}}%
\pgfpathlineto{\pgfqpoint{3.220316in}{2.812189in}}%
\pgfpathlineto{\pgfqpoint{3.220316in}{2.859376in}}%
\pgfpathlineto{\pgfqpoint{3.292969in}{2.859376in}}%
\pgfpathlineto{\pgfqpoint{3.292969in}{2.812189in}}%
\pgfpathmoveto{\pgfqpoint{3.292969in}{2.765002in}}%
\pgfpathlineto{\pgfqpoint{3.292969in}{2.765002in}}%
\pgfpathlineto{\pgfqpoint{3.292969in}{2.812189in}}%
\pgfpathlineto{\pgfqpoint{3.365622in}{2.812189in}}%
\pgfpathlineto{\pgfqpoint{3.365622in}{2.765002in}}%
\pgfpathmoveto{\pgfqpoint{3.220316in}{2.859376in}}%
\pgfpathlineto{\pgfqpoint{3.220316in}{2.859376in}}%
\pgfpathlineto{\pgfqpoint{3.220316in}{2.906562in}}%
\pgfpathlineto{\pgfqpoint{3.292969in}{2.906562in}}%
\pgfpathlineto{\pgfqpoint{3.292969in}{2.859376in}}%
\pgfpathmoveto{\pgfqpoint{3.365622in}{2.293123in}}%
\pgfpathlineto{\pgfqpoint{3.365622in}{2.293123in}}%
\pgfpathlineto{\pgfqpoint{3.365622in}{2.340310in}}%
\pgfpathlineto{\pgfqpoint{3.438281in}{2.340310in}}%
\pgfpathlineto{\pgfqpoint{3.438281in}{2.293123in}}%
\pgfpathmoveto{\pgfqpoint{3.365622in}{2.340310in}}%
\pgfpathlineto{\pgfqpoint{3.365622in}{2.340310in}}%
\pgfpathlineto{\pgfqpoint{3.365622in}{2.387497in}}%
\pgfpathlineto{\pgfqpoint{3.438281in}{2.387497in}}%
\pgfpathlineto{\pgfqpoint{3.438281in}{2.340310in}}%
\pgfpathmoveto{\pgfqpoint{3.438281in}{2.340310in}}%
\pgfpathlineto{\pgfqpoint{3.438281in}{2.340310in}}%
\pgfpathlineto{\pgfqpoint{3.438281in}{2.387497in}}%
\pgfpathlineto{\pgfqpoint{3.510941in}{2.387497in}}%
\pgfpathlineto{\pgfqpoint{3.510941in}{2.340310in}}%
\pgfpathmoveto{\pgfqpoint{3.365622in}{2.576252in}}%
\pgfpathlineto{\pgfqpoint{3.365622in}{2.576252in}}%
\pgfpathlineto{\pgfqpoint{3.365622in}{2.623438in}}%
\pgfpathlineto{\pgfqpoint{3.438281in}{2.623438in}}%
\pgfpathlineto{\pgfqpoint{3.438281in}{2.576252in}}%
\pgfpathmoveto{\pgfqpoint{3.365622in}{2.623438in}}%
\pgfpathlineto{\pgfqpoint{3.365622in}{2.623438in}}%
\pgfpathlineto{\pgfqpoint{3.365622in}{2.670624in}}%
\pgfpathlineto{\pgfqpoint{3.438281in}{2.670624in}}%
\pgfpathlineto{\pgfqpoint{3.438281in}{2.623438in}}%
\pgfpathmoveto{\pgfqpoint{3.438281in}{2.576252in}}%
\pgfpathlineto{\pgfqpoint{3.438281in}{2.576252in}}%
\pgfpathlineto{\pgfqpoint{3.438281in}{2.623438in}}%
\pgfpathlineto{\pgfqpoint{3.510941in}{2.623438in}}%
\pgfpathlineto{\pgfqpoint{3.510941in}{2.576252in}}%
\pgfpathmoveto{\pgfqpoint{3.365622in}{2.670624in}}%
\pgfpathlineto{\pgfqpoint{3.365622in}{2.670624in}}%
\pgfpathlineto{\pgfqpoint{3.365622in}{2.717813in}}%
\pgfpathlineto{\pgfqpoint{3.438281in}{2.717813in}}%
\pgfpathlineto{\pgfqpoint{3.438281in}{2.670624in}}%
\pgfpathmoveto{\pgfqpoint{3.510941in}{2.387497in}}%
\pgfpathlineto{\pgfqpoint{3.510941in}{2.387497in}}%
\pgfpathlineto{\pgfqpoint{3.510941in}{2.434685in}}%
\pgfpathlineto{\pgfqpoint{3.583593in}{2.434685in}}%
\pgfpathlineto{\pgfqpoint{3.583593in}{2.387497in}}%
\pgfpathmoveto{\pgfqpoint{3.510941in}{2.434685in}}%
\pgfpathlineto{\pgfqpoint{3.510941in}{2.434685in}}%
\pgfpathlineto{\pgfqpoint{3.510941in}{2.481873in}}%
\pgfpathlineto{\pgfqpoint{3.583593in}{2.481873in}}%
\pgfpathlineto{\pgfqpoint{3.583593in}{2.434685in}}%
\pgfpathmoveto{\pgfqpoint{3.510941in}{2.481873in}}%
\pgfpathlineto{\pgfqpoint{3.510941in}{2.481873in}}%
\pgfpathlineto{\pgfqpoint{3.510941in}{2.529062in}}%
\pgfpathlineto{\pgfqpoint{3.583593in}{2.529062in}}%
\pgfpathlineto{\pgfqpoint{3.583593in}{2.481873in}}%
\pgfpathmoveto{\pgfqpoint{0.750004in}{0.547188in}}%
\pgfpathlineto{\pgfqpoint{0.750004in}{0.547188in}}%
\pgfpathlineto{\pgfqpoint{0.750004in}{0.570782in}}%
\pgfpathlineto{\pgfqpoint{0.786332in}{0.570782in}}%
\pgfpathlineto{\pgfqpoint{0.786332in}{0.547188in}}%
\pgfpathmoveto{\pgfqpoint{0.750004in}{0.570782in}}%
\pgfpathlineto{\pgfqpoint{0.750004in}{0.570782in}}%
\pgfpathlineto{\pgfqpoint{0.750004in}{0.594376in}}%
\pgfpathlineto{\pgfqpoint{0.786332in}{0.594376in}}%
\pgfpathlineto{\pgfqpoint{0.786332in}{0.570782in}}%
\pgfpathmoveto{\pgfqpoint{0.786332in}{0.570782in}}%
\pgfpathlineto{\pgfqpoint{0.786332in}{0.570782in}}%
\pgfpathlineto{\pgfqpoint{0.786332in}{0.594376in}}%
\pgfpathlineto{\pgfqpoint{0.822659in}{0.594376in}}%
\pgfpathlineto{\pgfqpoint{0.822659in}{0.570782in}}%
\pgfpathmoveto{\pgfqpoint{0.895314in}{0.617970in}}%
\pgfpathlineto{\pgfqpoint{0.895314in}{0.617970in}}%
\pgfpathlineto{\pgfqpoint{0.895314in}{0.641564in}}%
\pgfpathlineto{\pgfqpoint{0.931641in}{0.641564in}}%
\pgfpathlineto{\pgfqpoint{0.931641in}{0.617970in}}%
\pgfpathmoveto{\pgfqpoint{0.967968in}{0.665158in}}%
\pgfpathlineto{\pgfqpoint{0.967968in}{0.665158in}}%
\pgfpathlineto{\pgfqpoint{0.967968in}{0.688752in}}%
\pgfpathlineto{\pgfqpoint{1.004295in}{0.688752in}}%
\pgfpathlineto{\pgfqpoint{1.004295in}{0.665158in}}%
\pgfpathmoveto{\pgfqpoint{1.040623in}{0.712344in}}%
\pgfpathlineto{\pgfqpoint{1.040623in}{0.712344in}}%
\pgfpathlineto{\pgfqpoint{1.040623in}{0.735937in}}%
\pgfpathlineto{\pgfqpoint{1.076951in}{0.735937in}}%
\pgfpathlineto{\pgfqpoint{1.076951in}{0.712344in}}%
\pgfpathmoveto{\pgfqpoint{1.113279in}{0.759530in}}%
\pgfpathlineto{\pgfqpoint{1.113279in}{0.759530in}}%
\pgfpathlineto{\pgfqpoint{1.113279in}{0.783123in}}%
\pgfpathlineto{\pgfqpoint{1.149607in}{0.783123in}}%
\pgfpathlineto{\pgfqpoint{1.149607in}{0.759530in}}%
\pgfpathmoveto{\pgfqpoint{1.113279in}{0.783123in}}%
\pgfpathlineto{\pgfqpoint{1.113279in}{0.783123in}}%
\pgfpathlineto{\pgfqpoint{1.113279in}{0.806717in}}%
\pgfpathlineto{\pgfqpoint{1.149607in}{0.806717in}}%
\pgfpathlineto{\pgfqpoint{1.149607in}{0.783123in}}%
\pgfpathmoveto{\pgfqpoint{1.113279in}{0.806717in}}%
\pgfpathlineto{\pgfqpoint{1.113279in}{0.806717in}}%
\pgfpathlineto{\pgfqpoint{1.113279in}{0.830311in}}%
\pgfpathlineto{\pgfqpoint{1.149607in}{0.830311in}}%
\pgfpathlineto{\pgfqpoint{1.149607in}{0.806717in}}%
\pgfpathmoveto{\pgfqpoint{1.149607in}{0.806717in}}%
\pgfpathlineto{\pgfqpoint{1.149607in}{0.806717in}}%
\pgfpathlineto{\pgfqpoint{1.149607in}{0.830311in}}%
\pgfpathlineto{\pgfqpoint{1.185936in}{0.830311in}}%
\pgfpathlineto{\pgfqpoint{1.185936in}{0.806717in}}%
\pgfpathmoveto{\pgfqpoint{1.185936in}{0.830311in}}%
\pgfpathlineto{\pgfqpoint{1.185936in}{0.830311in}}%
\pgfpathlineto{\pgfqpoint{1.185936in}{0.853905in}}%
\pgfpathlineto{\pgfqpoint{1.222264in}{0.853905in}}%
\pgfpathlineto{\pgfqpoint{1.222264in}{0.830311in}}%
\pgfpathmoveto{\pgfqpoint{1.185936in}{0.853905in}}%
\pgfpathlineto{\pgfqpoint{1.185936in}{0.853905in}}%
\pgfpathlineto{\pgfqpoint{1.185936in}{0.877498in}}%
\pgfpathlineto{\pgfqpoint{1.222264in}{0.877498in}}%
\pgfpathlineto{\pgfqpoint{1.222264in}{0.853905in}}%
\pgfpathmoveto{\pgfqpoint{1.222264in}{0.853905in}}%
\pgfpathlineto{\pgfqpoint{1.222264in}{0.853905in}}%
\pgfpathlineto{\pgfqpoint{1.222264in}{0.877498in}}%
\pgfpathlineto{\pgfqpoint{1.258592in}{0.877498in}}%
\pgfpathlineto{\pgfqpoint{1.258592in}{0.853905in}}%
\pgfpathmoveto{\pgfqpoint{1.258592in}{0.877498in}}%
\pgfpathlineto{\pgfqpoint{1.258592in}{0.877498in}}%
\pgfpathlineto{\pgfqpoint{1.258592in}{0.901093in}}%
\pgfpathlineto{\pgfqpoint{1.294921in}{0.901093in}}%
\pgfpathlineto{\pgfqpoint{1.294921in}{0.877498in}}%
\pgfpathmoveto{\pgfqpoint{1.258592in}{0.901093in}}%
\pgfpathlineto{\pgfqpoint{1.258592in}{0.901093in}}%
\pgfpathlineto{\pgfqpoint{1.258592in}{0.924687in}}%
\pgfpathlineto{\pgfqpoint{1.294921in}{0.924687in}}%
\pgfpathlineto{\pgfqpoint{1.294921in}{0.901093in}}%
\pgfpathmoveto{\pgfqpoint{1.294921in}{0.901093in}}%
\pgfpathlineto{\pgfqpoint{1.294921in}{0.901093in}}%
\pgfpathlineto{\pgfqpoint{1.294921in}{0.924687in}}%
\pgfpathlineto{\pgfqpoint{1.331249in}{0.924687in}}%
\pgfpathlineto{\pgfqpoint{1.331249in}{0.901093in}}%
\pgfpathmoveto{\pgfqpoint{1.403904in}{0.948281in}}%
\pgfpathlineto{\pgfqpoint{1.403904in}{0.948281in}}%
\pgfpathlineto{\pgfqpoint{1.403904in}{0.971875in}}%
\pgfpathlineto{\pgfqpoint{1.440231in}{0.971875in}}%
\pgfpathlineto{\pgfqpoint{1.440231in}{0.948281in}}%
\pgfpathmoveto{\pgfqpoint{1.476559in}{0.995469in}}%
\pgfpathlineto{\pgfqpoint{1.476559in}{0.995469in}}%
\pgfpathlineto{\pgfqpoint{1.476559in}{1.019062in}}%
\pgfpathlineto{\pgfqpoint{1.512887in}{1.019062in}}%
\pgfpathlineto{\pgfqpoint{1.512887in}{0.995469in}}%
\pgfpathmoveto{\pgfqpoint{1.549216in}{1.042656in}}%
\pgfpathlineto{\pgfqpoint{1.549216in}{1.042656in}}%
\pgfpathlineto{\pgfqpoint{1.549216in}{1.066250in}}%
\pgfpathlineto{\pgfqpoint{1.585545in}{1.066250in}}%
\pgfpathlineto{\pgfqpoint{1.585545in}{1.042656in}}%
\pgfpathmoveto{\pgfqpoint{1.621874in}{1.089843in}}%
\pgfpathlineto{\pgfqpoint{1.621874in}{1.089843in}}%
\pgfpathlineto{\pgfqpoint{1.621874in}{1.113436in}}%
\pgfpathlineto{\pgfqpoint{1.658202in}{1.113436in}}%
\pgfpathlineto{\pgfqpoint{1.658202in}{1.089843in}}%
\pgfpathmoveto{\pgfqpoint{1.621874in}{1.113436in}}%
\pgfpathlineto{\pgfqpoint{1.621874in}{1.113436in}}%
\pgfpathlineto{\pgfqpoint{1.621874in}{1.137030in}}%
\pgfpathlineto{\pgfqpoint{1.658202in}{1.137030in}}%
\pgfpathlineto{\pgfqpoint{1.658202in}{1.113436in}}%
\pgfpathmoveto{\pgfqpoint{1.621874in}{1.137030in}}%
\pgfpathlineto{\pgfqpoint{1.621874in}{1.137030in}}%
\pgfpathlineto{\pgfqpoint{1.621874in}{1.160623in}}%
\pgfpathlineto{\pgfqpoint{1.658202in}{1.160623in}}%
\pgfpathlineto{\pgfqpoint{1.658202in}{1.137030in}}%
\pgfpathmoveto{\pgfqpoint{1.658202in}{1.137030in}}%
\pgfpathlineto{\pgfqpoint{1.658202in}{1.137030in}}%
\pgfpathlineto{\pgfqpoint{1.658202in}{1.160623in}}%
\pgfpathlineto{\pgfqpoint{1.694530in}{1.160623in}}%
\pgfpathlineto{\pgfqpoint{1.694530in}{1.137030in}}%
\pgfpathmoveto{\pgfqpoint{1.694530in}{1.160623in}}%
\pgfpathlineto{\pgfqpoint{1.694530in}{1.160623in}}%
\pgfpathlineto{\pgfqpoint{1.694530in}{1.184217in}}%
\pgfpathlineto{\pgfqpoint{1.730858in}{1.184217in}}%
\pgfpathlineto{\pgfqpoint{1.730858in}{1.160623in}}%
\pgfpathmoveto{\pgfqpoint{1.694530in}{1.184217in}}%
\pgfpathlineto{\pgfqpoint{1.694530in}{1.184217in}}%
\pgfpathlineto{\pgfqpoint{1.694530in}{1.207811in}}%
\pgfpathlineto{\pgfqpoint{1.730858in}{1.207811in}}%
\pgfpathlineto{\pgfqpoint{1.730858in}{1.184217in}}%
\pgfpathmoveto{\pgfqpoint{1.730858in}{1.184217in}}%
\pgfpathlineto{\pgfqpoint{1.730858in}{1.184217in}}%
\pgfpathlineto{\pgfqpoint{1.730858in}{1.207811in}}%
\pgfpathlineto{\pgfqpoint{1.767187in}{1.207811in}}%
\pgfpathlineto{\pgfqpoint{1.767187in}{1.184217in}}%
\pgfpathmoveto{\pgfqpoint{1.767187in}{1.207811in}}%
\pgfpathlineto{\pgfqpoint{1.767187in}{1.207811in}}%
\pgfpathlineto{\pgfqpoint{1.767187in}{1.231405in}}%
\pgfpathlineto{\pgfqpoint{1.803516in}{1.231405in}}%
\pgfpathlineto{\pgfqpoint{1.803516in}{1.207811in}}%
\pgfpathmoveto{\pgfqpoint{1.767187in}{1.231405in}}%
\pgfpathlineto{\pgfqpoint{1.767187in}{1.231405in}}%
\pgfpathlineto{\pgfqpoint{1.767187in}{1.254999in}}%
\pgfpathlineto{\pgfqpoint{1.803516in}{1.254999in}}%
\pgfpathlineto{\pgfqpoint{1.803516in}{1.231405in}}%
\pgfpathmoveto{\pgfqpoint{1.803516in}{1.231405in}}%
\pgfpathlineto{\pgfqpoint{1.803516in}{1.231405in}}%
\pgfpathlineto{\pgfqpoint{1.803516in}{1.254999in}}%
\pgfpathlineto{\pgfqpoint{1.839845in}{1.254999in}}%
\pgfpathlineto{\pgfqpoint{1.839845in}{1.231405in}}%
\pgfpathmoveto{\pgfqpoint{1.839845in}{1.254999in}}%
\pgfpathlineto{\pgfqpoint{1.839845in}{1.254999in}}%
\pgfpathlineto{\pgfqpoint{1.839845in}{1.278594in}}%
\pgfpathlineto{\pgfqpoint{1.876174in}{1.278594in}}%
\pgfpathlineto{\pgfqpoint{1.876174in}{1.254999in}}%
\pgfpathmoveto{\pgfqpoint{1.839845in}{1.278594in}}%
\pgfpathlineto{\pgfqpoint{1.839845in}{1.278594in}}%
\pgfpathlineto{\pgfqpoint{1.839845in}{1.302188in}}%
\pgfpathlineto{\pgfqpoint{1.876174in}{1.302188in}}%
\pgfpathlineto{\pgfqpoint{1.876174in}{1.278594in}}%
\pgfpathmoveto{\pgfqpoint{1.876174in}{1.278594in}}%
\pgfpathlineto{\pgfqpoint{1.876174in}{1.278594in}}%
\pgfpathlineto{\pgfqpoint{1.876174in}{1.302188in}}%
\pgfpathlineto{\pgfqpoint{1.912503in}{1.302188in}}%
\pgfpathlineto{\pgfqpoint{1.912503in}{1.278594in}}%
\pgfpathmoveto{\pgfqpoint{1.912503in}{1.302188in}}%
\pgfpathlineto{\pgfqpoint{1.912503in}{1.302188in}}%
\pgfpathlineto{\pgfqpoint{1.912503in}{1.325782in}}%
\pgfpathlineto{\pgfqpoint{1.948830in}{1.325782in}}%
\pgfpathlineto{\pgfqpoint{1.948830in}{1.302188in}}%
\pgfpathmoveto{\pgfqpoint{1.912503in}{1.325782in}}%
\pgfpathlineto{\pgfqpoint{1.912503in}{1.325782in}}%
\pgfpathlineto{\pgfqpoint{1.912503in}{1.349376in}}%
\pgfpathlineto{\pgfqpoint{1.948830in}{1.349376in}}%
\pgfpathlineto{\pgfqpoint{1.948830in}{1.325782in}}%
\pgfpathmoveto{\pgfqpoint{1.948830in}{1.325782in}}%
\pgfpathlineto{\pgfqpoint{1.948830in}{1.325782in}}%
\pgfpathlineto{\pgfqpoint{1.948830in}{1.349376in}}%
\pgfpathlineto{\pgfqpoint{1.985157in}{1.349376in}}%
\pgfpathlineto{\pgfqpoint{1.985157in}{1.325782in}}%
\pgfpathmoveto{\pgfqpoint{1.985157in}{1.325782in}}%
\pgfpathlineto{\pgfqpoint{1.985157in}{1.325782in}}%
\pgfpathlineto{\pgfqpoint{1.985157in}{1.349376in}}%
\pgfpathlineto{\pgfqpoint{2.021483in}{1.349376in}}%
\pgfpathlineto{\pgfqpoint{2.021483in}{1.325782in}}%
\pgfpathmoveto{\pgfqpoint{2.057810in}{1.372970in}}%
\pgfpathlineto{\pgfqpoint{2.057810in}{1.372970in}}%
\pgfpathlineto{\pgfqpoint{2.057810in}{1.396564in}}%
\pgfpathlineto{\pgfqpoint{2.094138in}{1.396564in}}%
\pgfpathlineto{\pgfqpoint{2.094138in}{1.372970in}}%
\pgfpathmoveto{\pgfqpoint{2.130466in}{1.420158in}}%
\pgfpathlineto{\pgfqpoint{2.130466in}{1.420158in}}%
\pgfpathlineto{\pgfqpoint{2.130466in}{1.443752in}}%
\pgfpathlineto{\pgfqpoint{2.166793in}{1.443752in}}%
\pgfpathlineto{\pgfqpoint{2.166793in}{1.420158in}}%
\pgfpathmoveto{\pgfqpoint{2.203121in}{1.467345in}}%
\pgfpathlineto{\pgfqpoint{2.203121in}{1.467345in}}%
\pgfpathlineto{\pgfqpoint{2.203121in}{1.490938in}}%
\pgfpathlineto{\pgfqpoint{2.239451in}{1.490938in}}%
\pgfpathlineto{\pgfqpoint{2.239451in}{1.467345in}}%
\pgfpathmoveto{\pgfqpoint{2.275780in}{1.514531in}}%
\pgfpathlineto{\pgfqpoint{2.275780in}{1.514531in}}%
\pgfpathlineto{\pgfqpoint{2.275780in}{1.538125in}}%
\pgfpathlineto{\pgfqpoint{2.312110in}{1.538125in}}%
\pgfpathlineto{\pgfqpoint{2.312110in}{1.514531in}}%
\pgfpathmoveto{\pgfqpoint{2.275780in}{1.538125in}}%
\pgfpathlineto{\pgfqpoint{2.275780in}{1.538125in}}%
\pgfpathlineto{\pgfqpoint{2.275780in}{1.561718in}}%
\pgfpathlineto{\pgfqpoint{2.312110in}{1.561718in}}%
\pgfpathlineto{\pgfqpoint{2.312110in}{1.538125in}}%
\pgfpathmoveto{\pgfqpoint{2.275780in}{1.561718in}}%
\pgfpathlineto{\pgfqpoint{2.275780in}{1.561718in}}%
\pgfpathlineto{\pgfqpoint{2.275780in}{1.585311in}}%
\pgfpathlineto{\pgfqpoint{2.312110in}{1.585311in}}%
\pgfpathlineto{\pgfqpoint{2.312110in}{1.561718in}}%
\pgfpathmoveto{\pgfqpoint{2.312110in}{1.561718in}}%
\pgfpathlineto{\pgfqpoint{2.312110in}{1.561718in}}%
\pgfpathlineto{\pgfqpoint{2.312110in}{1.585311in}}%
\pgfpathlineto{\pgfqpoint{2.348439in}{1.585311in}}%
\pgfpathlineto{\pgfqpoint{2.348439in}{1.561718in}}%
\pgfpathmoveto{\pgfqpoint{2.348439in}{1.585311in}}%
\pgfpathlineto{\pgfqpoint{2.348439in}{1.585311in}}%
\pgfpathlineto{\pgfqpoint{2.348439in}{1.608905in}}%
\pgfpathlineto{\pgfqpoint{2.384766in}{1.608905in}}%
\pgfpathlineto{\pgfqpoint{2.384766in}{1.585311in}}%
\pgfpathmoveto{\pgfqpoint{2.348439in}{1.608905in}}%
\pgfpathlineto{\pgfqpoint{2.348439in}{1.608905in}}%
\pgfpathlineto{\pgfqpoint{2.348439in}{1.632498in}}%
\pgfpathlineto{\pgfqpoint{2.384766in}{1.632498in}}%
\pgfpathlineto{\pgfqpoint{2.384766in}{1.608905in}}%
\pgfpathmoveto{\pgfqpoint{2.384766in}{1.608905in}}%
\pgfpathlineto{\pgfqpoint{2.384766in}{1.608905in}}%
\pgfpathlineto{\pgfqpoint{2.384766in}{1.632498in}}%
\pgfpathlineto{\pgfqpoint{2.421093in}{1.632498in}}%
\pgfpathlineto{\pgfqpoint{2.421093in}{1.608905in}}%
\pgfpathmoveto{\pgfqpoint{2.421093in}{1.632498in}}%
\pgfpathlineto{\pgfqpoint{2.421093in}{1.632498in}}%
\pgfpathlineto{\pgfqpoint{2.421093in}{1.656092in}}%
\pgfpathlineto{\pgfqpoint{2.457420in}{1.656092in}}%
\pgfpathlineto{\pgfqpoint{2.457420in}{1.632498in}}%
\pgfpathmoveto{\pgfqpoint{2.421093in}{1.656092in}}%
\pgfpathlineto{\pgfqpoint{2.421093in}{1.656092in}}%
\pgfpathlineto{\pgfqpoint{2.421093in}{1.679686in}}%
\pgfpathlineto{\pgfqpoint{2.457420in}{1.679686in}}%
\pgfpathlineto{\pgfqpoint{2.457420in}{1.656092in}}%
\pgfpathmoveto{\pgfqpoint{2.457420in}{1.656092in}}%
\pgfpathlineto{\pgfqpoint{2.457420in}{1.656092in}}%
\pgfpathlineto{\pgfqpoint{2.457420in}{1.679686in}}%
\pgfpathlineto{\pgfqpoint{2.493747in}{1.679686in}}%
\pgfpathlineto{\pgfqpoint{2.493747in}{1.656092in}}%
\pgfpathmoveto{\pgfqpoint{2.493747in}{1.679686in}}%
\pgfpathlineto{\pgfqpoint{2.493747in}{1.679686in}}%
\pgfpathlineto{\pgfqpoint{2.493747in}{1.703280in}}%
\pgfpathlineto{\pgfqpoint{2.530077in}{1.703280in}}%
\pgfpathlineto{\pgfqpoint{2.530077in}{1.679686in}}%
\pgfpathmoveto{\pgfqpoint{2.493747in}{1.703280in}}%
\pgfpathlineto{\pgfqpoint{2.493747in}{1.703280in}}%
\pgfpathlineto{\pgfqpoint{2.493747in}{1.726874in}}%
\pgfpathlineto{\pgfqpoint{2.530077in}{1.726874in}}%
\pgfpathlineto{\pgfqpoint{2.530077in}{1.703280in}}%
\pgfpathmoveto{\pgfqpoint{2.530077in}{1.703280in}}%
\pgfpathlineto{\pgfqpoint{2.530077in}{1.703280in}}%
\pgfpathlineto{\pgfqpoint{2.530077in}{1.726874in}}%
\pgfpathlineto{\pgfqpoint{2.566406in}{1.726874in}}%
\pgfpathlineto{\pgfqpoint{2.566406in}{1.703280in}}%
\pgfpathmoveto{\pgfqpoint{2.566406in}{1.726874in}}%
\pgfpathlineto{\pgfqpoint{2.566406in}{1.726874in}}%
\pgfpathlineto{\pgfqpoint{2.566406in}{1.750468in}}%
\pgfpathlineto{\pgfqpoint{2.602735in}{1.750468in}}%
\pgfpathlineto{\pgfqpoint{2.602735in}{1.726874in}}%
\pgfpathmoveto{\pgfqpoint{2.566406in}{1.750468in}}%
\pgfpathlineto{\pgfqpoint{2.566406in}{1.750468in}}%
\pgfpathlineto{\pgfqpoint{2.566406in}{1.774062in}}%
\pgfpathlineto{\pgfqpoint{2.602735in}{1.774062in}}%
\pgfpathlineto{\pgfqpoint{2.602735in}{1.750468in}}%
\pgfpathmoveto{\pgfqpoint{2.602735in}{1.750468in}}%
\pgfpathlineto{\pgfqpoint{2.602735in}{1.750468in}}%
\pgfpathlineto{\pgfqpoint{2.602735in}{1.774062in}}%
\pgfpathlineto{\pgfqpoint{2.639065in}{1.774062in}}%
\pgfpathlineto{\pgfqpoint{2.639065in}{1.750468in}}%
\pgfpathmoveto{\pgfqpoint{2.639065in}{1.774062in}}%
\pgfpathlineto{\pgfqpoint{2.639065in}{1.774062in}}%
\pgfpathlineto{\pgfqpoint{2.639065in}{1.797656in}}%
\pgfpathlineto{\pgfqpoint{2.675393in}{1.797656in}}%
\pgfpathlineto{\pgfqpoint{2.675393in}{1.774062in}}%
\pgfpathmoveto{\pgfqpoint{2.639065in}{1.797656in}}%
\pgfpathlineto{\pgfqpoint{2.639065in}{1.797656in}}%
\pgfpathlineto{\pgfqpoint{2.639065in}{1.821250in}}%
\pgfpathlineto{\pgfqpoint{2.675393in}{1.821250in}}%
\pgfpathlineto{\pgfqpoint{2.675393in}{1.797656in}}%
\pgfpathmoveto{\pgfqpoint{2.675393in}{1.797656in}}%
\pgfpathlineto{\pgfqpoint{2.675393in}{1.797656in}}%
\pgfpathlineto{\pgfqpoint{2.675393in}{1.821250in}}%
\pgfpathlineto{\pgfqpoint{2.711721in}{1.821250in}}%
\pgfpathlineto{\pgfqpoint{2.711721in}{1.797656in}}%
\pgfpathmoveto{\pgfqpoint{2.711721in}{1.821250in}}%
\pgfpathlineto{\pgfqpoint{2.711721in}{1.821250in}}%
\pgfpathlineto{\pgfqpoint{2.711721in}{1.844844in}}%
\pgfpathlineto{\pgfqpoint{2.748049in}{1.844844in}}%
\pgfpathlineto{\pgfqpoint{2.748049in}{1.821250in}}%
\pgfpathmoveto{\pgfqpoint{2.711721in}{1.844844in}}%
\pgfpathlineto{\pgfqpoint{2.711721in}{1.844844in}}%
\pgfpathlineto{\pgfqpoint{2.711721in}{1.868438in}}%
\pgfpathlineto{\pgfqpoint{2.748049in}{1.868438in}}%
\pgfpathlineto{\pgfqpoint{2.748049in}{1.844844in}}%
\pgfpathmoveto{\pgfqpoint{2.748049in}{1.844844in}}%
\pgfpathlineto{\pgfqpoint{2.748049in}{1.844844in}}%
\pgfpathlineto{\pgfqpoint{2.748049in}{1.868438in}}%
\pgfpathlineto{\pgfqpoint{2.784377in}{1.868438in}}%
\pgfpathlineto{\pgfqpoint{2.784377in}{1.844844in}}%
\pgfpathmoveto{\pgfqpoint{2.784377in}{1.868438in}}%
\pgfpathlineto{\pgfqpoint{2.784377in}{1.868438in}}%
\pgfpathlineto{\pgfqpoint{2.784377in}{1.892032in}}%
\pgfpathlineto{\pgfqpoint{2.820706in}{1.892032in}}%
\pgfpathlineto{\pgfqpoint{2.820706in}{1.868438in}}%
\pgfpathmoveto{\pgfqpoint{2.784377in}{1.892032in}}%
\pgfpathlineto{\pgfqpoint{2.784377in}{1.892032in}}%
\pgfpathlineto{\pgfqpoint{2.784377in}{1.915626in}}%
\pgfpathlineto{\pgfqpoint{2.820706in}{1.915626in}}%
\pgfpathlineto{\pgfqpoint{2.820706in}{1.892032in}}%
\pgfpathmoveto{\pgfqpoint{2.820706in}{1.892032in}}%
\pgfpathlineto{\pgfqpoint{2.820706in}{1.892032in}}%
\pgfpathlineto{\pgfqpoint{2.820706in}{1.915626in}}%
\pgfpathlineto{\pgfqpoint{2.857034in}{1.915626in}}%
\pgfpathlineto{\pgfqpoint{2.857034in}{1.892032in}}%
\pgfpathmoveto{\pgfqpoint{2.857034in}{1.915626in}}%
\pgfpathlineto{\pgfqpoint{2.857034in}{1.915626in}}%
\pgfpathlineto{\pgfqpoint{2.857034in}{1.939219in}}%
\pgfpathlineto{\pgfqpoint{2.893363in}{1.939219in}}%
\pgfpathlineto{\pgfqpoint{2.893363in}{1.915626in}}%
\pgfpathmoveto{\pgfqpoint{2.857034in}{1.939219in}}%
\pgfpathlineto{\pgfqpoint{2.857034in}{1.939219in}}%
\pgfpathlineto{\pgfqpoint{2.857034in}{1.962812in}}%
\pgfpathlineto{\pgfqpoint{2.893363in}{1.962812in}}%
\pgfpathlineto{\pgfqpoint{2.893363in}{1.939219in}}%
\pgfpathmoveto{\pgfqpoint{2.893363in}{1.939219in}}%
\pgfpathlineto{\pgfqpoint{2.893363in}{1.939219in}}%
\pgfpathlineto{\pgfqpoint{2.893363in}{1.962812in}}%
\pgfpathlineto{\pgfqpoint{2.929691in}{1.962812in}}%
\pgfpathlineto{\pgfqpoint{2.929691in}{1.939219in}}%
\pgfpathmoveto{\pgfqpoint{2.857034in}{3.378437in}}%
\pgfpathlineto{\pgfqpoint{2.857034in}{3.378437in}}%
\pgfpathlineto{\pgfqpoint{2.857034in}{3.402031in}}%
\pgfpathlineto{\pgfqpoint{2.893363in}{3.402031in}}%
\pgfpathlineto{\pgfqpoint{2.893363in}{3.378437in}}%
\pgfpathmoveto{\pgfqpoint{2.857034in}{3.402031in}}%
\pgfpathlineto{\pgfqpoint{2.857034in}{3.402031in}}%
\pgfpathlineto{\pgfqpoint{2.857034in}{3.425625in}}%
\pgfpathlineto{\pgfqpoint{2.893363in}{3.425625in}}%
\pgfpathlineto{\pgfqpoint{2.893363in}{3.402031in}}%
\pgfpathmoveto{\pgfqpoint{2.893363in}{3.378437in}}%
\pgfpathlineto{\pgfqpoint{2.893363in}{3.378437in}}%
\pgfpathlineto{\pgfqpoint{2.893363in}{3.402031in}}%
\pgfpathlineto{\pgfqpoint{2.929691in}{3.402031in}}%
\pgfpathlineto{\pgfqpoint{2.929691in}{3.378437in}}%
\pgfpathmoveto{\pgfqpoint{2.784377in}{3.472812in}}%
\pgfpathlineto{\pgfqpoint{2.784377in}{3.472812in}}%
\pgfpathlineto{\pgfqpoint{2.784377in}{3.496405in}}%
\pgfpathlineto{\pgfqpoint{2.820706in}{3.496405in}}%
\pgfpathlineto{\pgfqpoint{2.820706in}{3.472812in}}%
\pgfpathmoveto{\pgfqpoint{2.784377in}{3.496405in}}%
\pgfpathlineto{\pgfqpoint{2.784377in}{3.496405in}}%
\pgfpathlineto{\pgfqpoint{2.784377in}{3.519999in}}%
\pgfpathlineto{\pgfqpoint{2.820706in}{3.519999in}}%
\pgfpathlineto{\pgfqpoint{2.820706in}{3.496405in}}%
\pgfpathmoveto{\pgfqpoint{2.820706in}{3.472812in}}%
\pgfpathlineto{\pgfqpoint{2.820706in}{3.472812in}}%
\pgfpathlineto{\pgfqpoint{2.820706in}{3.496405in}}%
\pgfpathlineto{\pgfqpoint{2.857034in}{3.496405in}}%
\pgfpathlineto{\pgfqpoint{2.857034in}{3.472812in}}%
\pgfpathmoveto{\pgfqpoint{2.857034in}{3.425625in}}%
\pgfpathlineto{\pgfqpoint{2.857034in}{3.425625in}}%
\pgfpathlineto{\pgfqpoint{2.857034in}{3.449218in}}%
\pgfpathlineto{\pgfqpoint{2.893363in}{3.449218in}}%
\pgfpathlineto{\pgfqpoint{2.893363in}{3.425625in}}%
\pgfpathmoveto{\pgfqpoint{2.929691in}{1.962812in}}%
\pgfpathlineto{\pgfqpoint{2.929691in}{1.962812in}}%
\pgfpathlineto{\pgfqpoint{2.929691in}{1.986405in}}%
\pgfpathlineto{\pgfqpoint{2.966017in}{1.986405in}}%
\pgfpathlineto{\pgfqpoint{2.966017in}{1.962812in}}%
\pgfpathmoveto{\pgfqpoint{2.929691in}{1.986405in}}%
\pgfpathlineto{\pgfqpoint{2.929691in}{1.986405in}}%
\pgfpathlineto{\pgfqpoint{2.929691in}{2.009997in}}%
\pgfpathlineto{\pgfqpoint{2.966017in}{2.009997in}}%
\pgfpathlineto{\pgfqpoint{2.966017in}{1.986405in}}%
\pgfpathmoveto{\pgfqpoint{2.966017in}{1.986405in}}%
\pgfpathlineto{\pgfqpoint{2.966017in}{1.986405in}}%
\pgfpathlineto{\pgfqpoint{2.966017in}{2.009997in}}%
\pgfpathlineto{\pgfqpoint{3.002344in}{2.009997in}}%
\pgfpathlineto{\pgfqpoint{3.002344in}{1.986405in}}%
\pgfpathmoveto{\pgfqpoint{3.002344in}{2.009997in}}%
\pgfpathlineto{\pgfqpoint{3.002344in}{2.009997in}}%
\pgfpathlineto{\pgfqpoint{3.002344in}{2.033593in}}%
\pgfpathlineto{\pgfqpoint{3.038670in}{2.033593in}}%
\pgfpathlineto{\pgfqpoint{3.038670in}{2.009997in}}%
\pgfpathmoveto{\pgfqpoint{3.002344in}{2.033593in}}%
\pgfpathlineto{\pgfqpoint{3.002344in}{2.033593in}}%
\pgfpathlineto{\pgfqpoint{3.002344in}{2.057188in}}%
\pgfpathlineto{\pgfqpoint{3.038670in}{2.057188in}}%
\pgfpathlineto{\pgfqpoint{3.038670in}{2.033593in}}%
\pgfpathmoveto{\pgfqpoint{3.038670in}{2.033593in}}%
\pgfpathlineto{\pgfqpoint{3.038670in}{2.033593in}}%
\pgfpathlineto{\pgfqpoint{3.038670in}{2.057188in}}%
\pgfpathlineto{\pgfqpoint{3.074996in}{2.057188in}}%
\pgfpathlineto{\pgfqpoint{3.074996in}{2.033593in}}%
\pgfpathmoveto{\pgfqpoint{3.002344in}{3.189687in}}%
\pgfpathlineto{\pgfqpoint{3.002344in}{3.189687in}}%
\pgfpathlineto{\pgfqpoint{3.002344in}{3.213280in}}%
\pgfpathlineto{\pgfqpoint{3.038670in}{3.213280in}}%
\pgfpathlineto{\pgfqpoint{3.038670in}{3.189687in}}%
\pgfpathmoveto{\pgfqpoint{3.002344in}{3.213280in}}%
\pgfpathlineto{\pgfqpoint{3.002344in}{3.213280in}}%
\pgfpathlineto{\pgfqpoint{3.002344in}{3.236873in}}%
\pgfpathlineto{\pgfqpoint{3.038670in}{3.236873in}}%
\pgfpathlineto{\pgfqpoint{3.038670in}{3.213280in}}%
\pgfpathmoveto{\pgfqpoint{3.038670in}{3.189687in}}%
\pgfpathlineto{\pgfqpoint{3.038670in}{3.189687in}}%
\pgfpathlineto{\pgfqpoint{3.038670in}{3.213280in}}%
\pgfpathlineto{\pgfqpoint{3.074996in}{3.213280in}}%
\pgfpathlineto{\pgfqpoint{3.074996in}{3.189687in}}%
\pgfpathmoveto{\pgfqpoint{2.929691in}{3.284061in}}%
\pgfpathlineto{\pgfqpoint{2.929691in}{3.284061in}}%
\pgfpathlineto{\pgfqpoint{2.929691in}{3.307655in}}%
\pgfpathlineto{\pgfqpoint{2.966017in}{3.307655in}}%
\pgfpathlineto{\pgfqpoint{2.966017in}{3.284061in}}%
\pgfpathmoveto{\pgfqpoint{2.929691in}{3.307655in}}%
\pgfpathlineto{\pgfqpoint{2.929691in}{3.307655in}}%
\pgfpathlineto{\pgfqpoint{2.929691in}{3.331249in}}%
\pgfpathlineto{\pgfqpoint{2.966017in}{3.331249in}}%
\pgfpathlineto{\pgfqpoint{2.966017in}{3.307655in}}%
\pgfpathmoveto{\pgfqpoint{2.966017in}{3.284061in}}%
\pgfpathlineto{\pgfqpoint{2.966017in}{3.284061in}}%
\pgfpathlineto{\pgfqpoint{2.966017in}{3.307655in}}%
\pgfpathlineto{\pgfqpoint{3.002344in}{3.307655in}}%
\pgfpathlineto{\pgfqpoint{3.002344in}{3.284061in}}%
\pgfpathmoveto{\pgfqpoint{3.002344in}{3.236873in}}%
\pgfpathlineto{\pgfqpoint{3.002344in}{3.236873in}}%
\pgfpathlineto{\pgfqpoint{3.002344in}{3.260467in}}%
\pgfpathlineto{\pgfqpoint{3.038670in}{3.260467in}}%
\pgfpathlineto{\pgfqpoint{3.038670in}{3.236873in}}%
\pgfpathmoveto{\pgfqpoint{2.929691in}{3.331249in}}%
\pgfpathlineto{\pgfqpoint{2.929691in}{3.331249in}}%
\pgfpathlineto{\pgfqpoint{2.929691in}{3.354843in}}%
\pgfpathlineto{\pgfqpoint{2.966017in}{3.354843in}}%
\pgfpathlineto{\pgfqpoint{2.966017in}{3.331249in}}%
\pgfpathmoveto{\pgfqpoint{3.074996in}{2.033593in}}%
\pgfpathlineto{\pgfqpoint{3.074996in}{2.033593in}}%
\pgfpathlineto{\pgfqpoint{3.074996in}{2.057188in}}%
\pgfpathlineto{\pgfqpoint{3.111326in}{2.057188in}}%
\pgfpathlineto{\pgfqpoint{3.111326in}{2.033593in}}%
\pgfpathmoveto{\pgfqpoint{3.147656in}{2.080783in}}%
\pgfpathlineto{\pgfqpoint{3.147656in}{2.080783in}}%
\pgfpathlineto{\pgfqpoint{3.147656in}{2.104378in}}%
\pgfpathlineto{\pgfqpoint{3.183986in}{2.104378in}}%
\pgfpathlineto{\pgfqpoint{3.183986in}{2.080783in}}%
\pgfpathmoveto{\pgfqpoint{3.147656in}{3.000937in}}%
\pgfpathlineto{\pgfqpoint{3.147656in}{3.000937in}}%
\pgfpathlineto{\pgfqpoint{3.147656in}{3.024531in}}%
\pgfpathlineto{\pgfqpoint{3.183986in}{3.024531in}}%
\pgfpathlineto{\pgfqpoint{3.183986in}{3.000937in}}%
\pgfpathmoveto{\pgfqpoint{3.147656in}{3.024531in}}%
\pgfpathlineto{\pgfqpoint{3.147656in}{3.024531in}}%
\pgfpathlineto{\pgfqpoint{3.147656in}{3.048126in}}%
\pgfpathlineto{\pgfqpoint{3.183986in}{3.048126in}}%
\pgfpathlineto{\pgfqpoint{3.183986in}{3.024531in}}%
\pgfpathmoveto{\pgfqpoint{3.183986in}{3.000937in}}%
\pgfpathlineto{\pgfqpoint{3.183986in}{3.000937in}}%
\pgfpathlineto{\pgfqpoint{3.183986in}{3.024531in}}%
\pgfpathlineto{\pgfqpoint{3.220316in}{3.024531in}}%
\pgfpathlineto{\pgfqpoint{3.220316in}{3.000937in}}%
\pgfpathmoveto{\pgfqpoint{3.074996in}{3.095314in}}%
\pgfpathlineto{\pgfqpoint{3.074996in}{3.095314in}}%
\pgfpathlineto{\pgfqpoint{3.074996in}{3.118907in}}%
\pgfpathlineto{\pgfqpoint{3.111326in}{3.118907in}}%
\pgfpathlineto{\pgfqpoint{3.111326in}{3.095314in}}%
\pgfpathmoveto{\pgfqpoint{3.074996in}{3.118907in}}%
\pgfpathlineto{\pgfqpoint{3.074996in}{3.118907in}}%
\pgfpathlineto{\pgfqpoint{3.074996in}{3.142501in}}%
\pgfpathlineto{\pgfqpoint{3.111326in}{3.142501in}}%
\pgfpathlineto{\pgfqpoint{3.111326in}{3.118907in}}%
\pgfpathmoveto{\pgfqpoint{3.111326in}{3.095314in}}%
\pgfpathlineto{\pgfqpoint{3.111326in}{3.095314in}}%
\pgfpathlineto{\pgfqpoint{3.111326in}{3.118907in}}%
\pgfpathlineto{\pgfqpoint{3.147656in}{3.118907in}}%
\pgfpathlineto{\pgfqpoint{3.147656in}{3.095314in}}%
\pgfpathmoveto{\pgfqpoint{3.147656in}{3.048126in}}%
\pgfpathlineto{\pgfqpoint{3.147656in}{3.048126in}}%
\pgfpathlineto{\pgfqpoint{3.147656in}{3.071720in}}%
\pgfpathlineto{\pgfqpoint{3.183986in}{3.071720in}}%
\pgfpathlineto{\pgfqpoint{3.183986in}{3.048126in}}%
\pgfpathmoveto{\pgfqpoint{3.074996in}{3.142501in}}%
\pgfpathlineto{\pgfqpoint{3.074996in}{3.142501in}}%
\pgfpathlineto{\pgfqpoint{3.074996in}{3.166094in}}%
\pgfpathlineto{\pgfqpoint{3.111326in}{3.166094in}}%
\pgfpathlineto{\pgfqpoint{3.111326in}{3.142501in}}%
\pgfpathmoveto{\pgfqpoint{3.220316in}{2.127971in}}%
\pgfpathlineto{\pgfqpoint{3.220316in}{2.127971in}}%
\pgfpathlineto{\pgfqpoint{3.220316in}{2.151565in}}%
\pgfpathlineto{\pgfqpoint{3.256643in}{2.151565in}}%
\pgfpathlineto{\pgfqpoint{3.256643in}{2.127971in}}%
\pgfpathmoveto{\pgfqpoint{3.292969in}{2.175158in}}%
\pgfpathlineto{\pgfqpoint{3.292969in}{2.175158in}}%
\pgfpathlineto{\pgfqpoint{3.292969in}{2.198752in}}%
\pgfpathlineto{\pgfqpoint{3.329296in}{2.198752in}}%
\pgfpathlineto{\pgfqpoint{3.329296in}{2.175158in}}%
\pgfpathmoveto{\pgfqpoint{3.292969in}{2.812189in}}%
\pgfpathlineto{\pgfqpoint{3.292969in}{2.812189in}}%
\pgfpathlineto{\pgfqpoint{3.292969in}{2.835782in}}%
\pgfpathlineto{\pgfqpoint{3.329296in}{2.835782in}}%
\pgfpathlineto{\pgfqpoint{3.329296in}{2.812189in}}%
\pgfpathmoveto{\pgfqpoint{3.292969in}{2.835782in}}%
\pgfpathlineto{\pgfqpoint{3.292969in}{2.835782in}}%
\pgfpathlineto{\pgfqpoint{3.292969in}{2.859376in}}%
\pgfpathlineto{\pgfqpoint{3.329296in}{2.859376in}}%
\pgfpathlineto{\pgfqpoint{3.329296in}{2.835782in}}%
\pgfpathmoveto{\pgfqpoint{3.329296in}{2.812189in}}%
\pgfpathlineto{\pgfqpoint{3.329296in}{2.812189in}}%
\pgfpathlineto{\pgfqpoint{3.329296in}{2.835782in}}%
\pgfpathlineto{\pgfqpoint{3.365622in}{2.835782in}}%
\pgfpathlineto{\pgfqpoint{3.365622in}{2.812189in}}%
\pgfpathmoveto{\pgfqpoint{3.220316in}{2.906562in}}%
\pgfpathlineto{\pgfqpoint{3.220316in}{2.906562in}}%
\pgfpathlineto{\pgfqpoint{3.220316in}{2.930155in}}%
\pgfpathlineto{\pgfqpoint{3.256643in}{2.930155in}}%
\pgfpathlineto{\pgfqpoint{3.256643in}{2.906562in}}%
\pgfpathmoveto{\pgfqpoint{3.220316in}{2.930155in}}%
\pgfpathlineto{\pgfqpoint{3.220316in}{2.930155in}}%
\pgfpathlineto{\pgfqpoint{3.220316in}{2.953748in}}%
\pgfpathlineto{\pgfqpoint{3.256643in}{2.953748in}}%
\pgfpathlineto{\pgfqpoint{3.256643in}{2.930155in}}%
\pgfpathmoveto{\pgfqpoint{3.256643in}{2.906562in}}%
\pgfpathlineto{\pgfqpoint{3.256643in}{2.906562in}}%
\pgfpathlineto{\pgfqpoint{3.256643in}{2.930155in}}%
\pgfpathlineto{\pgfqpoint{3.292969in}{2.930155in}}%
\pgfpathlineto{\pgfqpoint{3.292969in}{2.906562in}}%
\pgfpathmoveto{\pgfqpoint{3.292969in}{2.859376in}}%
\pgfpathlineto{\pgfqpoint{3.292969in}{2.859376in}}%
\pgfpathlineto{\pgfqpoint{3.292969in}{2.882969in}}%
\pgfpathlineto{\pgfqpoint{3.329296in}{2.882969in}}%
\pgfpathlineto{\pgfqpoint{3.329296in}{2.859376in}}%
\pgfpathmoveto{\pgfqpoint{3.220316in}{2.953748in}}%
\pgfpathlineto{\pgfqpoint{3.220316in}{2.953748in}}%
\pgfpathlineto{\pgfqpoint{3.220316in}{2.977343in}}%
\pgfpathlineto{\pgfqpoint{3.256643in}{2.977343in}}%
\pgfpathlineto{\pgfqpoint{3.256643in}{2.953748in}}%
\pgfpathmoveto{\pgfqpoint{3.365622in}{2.222345in}}%
\pgfpathlineto{\pgfqpoint{3.365622in}{2.222345in}}%
\pgfpathlineto{\pgfqpoint{3.365622in}{2.245937in}}%
\pgfpathlineto{\pgfqpoint{3.401952in}{2.245937in}}%
\pgfpathlineto{\pgfqpoint{3.401952in}{2.222345in}}%
\pgfpathmoveto{\pgfqpoint{3.365622in}{2.245937in}}%
\pgfpathlineto{\pgfqpoint{3.365622in}{2.245937in}}%
\pgfpathlineto{\pgfqpoint{3.365622in}{2.269530in}}%
\pgfpathlineto{\pgfqpoint{3.401952in}{2.269530in}}%
\pgfpathlineto{\pgfqpoint{3.401952in}{2.245937in}}%
\pgfpathmoveto{\pgfqpoint{3.365622in}{2.269530in}}%
\pgfpathlineto{\pgfqpoint{3.365622in}{2.269530in}}%
\pgfpathlineto{\pgfqpoint{3.365622in}{2.293123in}}%
\pgfpathlineto{\pgfqpoint{3.401952in}{2.293123in}}%
\pgfpathlineto{\pgfqpoint{3.401952in}{2.269530in}}%
\pgfpathmoveto{\pgfqpoint{3.401952in}{2.269530in}}%
\pgfpathlineto{\pgfqpoint{3.401952in}{2.269530in}}%
\pgfpathlineto{\pgfqpoint{3.401952in}{2.293123in}}%
\pgfpathlineto{\pgfqpoint{3.438281in}{2.293123in}}%
\pgfpathlineto{\pgfqpoint{3.438281in}{2.269530in}}%
\pgfpathmoveto{\pgfqpoint{3.438281in}{2.293123in}}%
\pgfpathlineto{\pgfqpoint{3.438281in}{2.293123in}}%
\pgfpathlineto{\pgfqpoint{3.438281in}{2.316716in}}%
\pgfpathlineto{\pgfqpoint{3.474611in}{2.316716in}}%
\pgfpathlineto{\pgfqpoint{3.474611in}{2.293123in}}%
\pgfpathmoveto{\pgfqpoint{3.438281in}{2.316716in}}%
\pgfpathlineto{\pgfqpoint{3.438281in}{2.316716in}}%
\pgfpathlineto{\pgfqpoint{3.438281in}{2.340310in}}%
\pgfpathlineto{\pgfqpoint{3.474611in}{2.340310in}}%
\pgfpathlineto{\pgfqpoint{3.474611in}{2.316716in}}%
\pgfpathmoveto{\pgfqpoint{3.474611in}{2.316716in}}%
\pgfpathlineto{\pgfqpoint{3.474611in}{2.316716in}}%
\pgfpathlineto{\pgfqpoint{3.474611in}{2.340310in}}%
\pgfpathlineto{\pgfqpoint{3.510941in}{2.340310in}}%
\pgfpathlineto{\pgfqpoint{3.510941in}{2.316716in}}%
\pgfpathmoveto{\pgfqpoint{3.438281in}{2.623438in}}%
\pgfpathlineto{\pgfqpoint{3.438281in}{2.623438in}}%
\pgfpathlineto{\pgfqpoint{3.438281in}{2.647031in}}%
\pgfpathlineto{\pgfqpoint{3.474611in}{2.647031in}}%
\pgfpathlineto{\pgfqpoint{3.474611in}{2.623438in}}%
\pgfpathmoveto{\pgfqpoint{3.438281in}{2.647031in}}%
\pgfpathlineto{\pgfqpoint{3.438281in}{2.647031in}}%
\pgfpathlineto{\pgfqpoint{3.438281in}{2.670624in}}%
\pgfpathlineto{\pgfqpoint{3.474611in}{2.670624in}}%
\pgfpathlineto{\pgfqpoint{3.474611in}{2.647031in}}%
\pgfpathmoveto{\pgfqpoint{3.474611in}{2.623438in}}%
\pgfpathlineto{\pgfqpoint{3.474611in}{2.623438in}}%
\pgfpathlineto{\pgfqpoint{3.474611in}{2.647031in}}%
\pgfpathlineto{\pgfqpoint{3.510941in}{2.647031in}}%
\pgfpathlineto{\pgfqpoint{3.510941in}{2.623438in}}%
\pgfpathmoveto{\pgfqpoint{3.365622in}{2.717813in}}%
\pgfpathlineto{\pgfqpoint{3.365622in}{2.717813in}}%
\pgfpathlineto{\pgfqpoint{3.365622in}{2.741408in}}%
\pgfpathlineto{\pgfqpoint{3.401952in}{2.741408in}}%
\pgfpathlineto{\pgfqpoint{3.401952in}{2.717813in}}%
\pgfpathmoveto{\pgfqpoint{3.365622in}{2.741408in}}%
\pgfpathlineto{\pgfqpoint{3.365622in}{2.741408in}}%
\pgfpathlineto{\pgfqpoint{3.365622in}{2.765002in}}%
\pgfpathlineto{\pgfqpoint{3.401952in}{2.765002in}}%
\pgfpathlineto{\pgfqpoint{3.401952in}{2.741408in}}%
\pgfpathmoveto{\pgfqpoint{3.401952in}{2.717813in}}%
\pgfpathlineto{\pgfqpoint{3.401952in}{2.717813in}}%
\pgfpathlineto{\pgfqpoint{3.401952in}{2.741408in}}%
\pgfpathlineto{\pgfqpoint{3.438281in}{2.741408in}}%
\pgfpathlineto{\pgfqpoint{3.438281in}{2.717813in}}%
\pgfpathmoveto{\pgfqpoint{3.438281in}{2.670624in}}%
\pgfpathlineto{\pgfqpoint{3.438281in}{2.670624in}}%
\pgfpathlineto{\pgfqpoint{3.438281in}{2.694219in}}%
\pgfpathlineto{\pgfqpoint{3.474611in}{2.694219in}}%
\pgfpathlineto{\pgfqpoint{3.474611in}{2.670624in}}%
\pgfpathmoveto{\pgfqpoint{3.365622in}{2.765002in}}%
\pgfpathlineto{\pgfqpoint{3.365622in}{2.765002in}}%
\pgfpathlineto{\pgfqpoint{3.365622in}{2.788596in}}%
\pgfpathlineto{\pgfqpoint{3.401952in}{2.788596in}}%
\pgfpathlineto{\pgfqpoint{3.401952in}{2.765002in}}%
\pgfpathmoveto{\pgfqpoint{3.510941in}{2.340310in}}%
\pgfpathlineto{\pgfqpoint{3.510941in}{2.340310in}}%
\pgfpathlineto{\pgfqpoint{3.510941in}{2.363904in}}%
\pgfpathlineto{\pgfqpoint{3.547267in}{2.363904in}}%
\pgfpathlineto{\pgfqpoint{3.547267in}{2.340310in}}%
\pgfpathmoveto{\pgfqpoint{3.510941in}{2.363904in}}%
\pgfpathlineto{\pgfqpoint{3.510941in}{2.363904in}}%
\pgfpathlineto{\pgfqpoint{3.510941in}{2.387497in}}%
\pgfpathlineto{\pgfqpoint{3.547267in}{2.387497in}}%
\pgfpathlineto{\pgfqpoint{3.547267in}{2.363904in}}%
\pgfpathmoveto{\pgfqpoint{3.547267in}{2.363904in}}%
\pgfpathlineto{\pgfqpoint{3.547267in}{2.363904in}}%
\pgfpathlineto{\pgfqpoint{3.547267in}{2.387497in}}%
\pgfpathlineto{\pgfqpoint{3.583593in}{2.387497in}}%
\pgfpathlineto{\pgfqpoint{3.583593in}{2.363904in}}%
\pgfpathmoveto{\pgfqpoint{3.583593in}{2.387497in}}%
\pgfpathlineto{\pgfqpoint{3.583593in}{2.387497in}}%
\pgfpathlineto{\pgfqpoint{3.583593in}{2.411091in}}%
\pgfpathlineto{\pgfqpoint{3.619920in}{2.411091in}}%
\pgfpathlineto{\pgfqpoint{3.619920in}{2.387497in}}%
\pgfpathmoveto{\pgfqpoint{3.583593in}{2.411091in}}%
\pgfpathlineto{\pgfqpoint{3.583593in}{2.411091in}}%
\pgfpathlineto{\pgfqpoint{3.583593in}{2.434685in}}%
\pgfpathlineto{\pgfqpoint{3.619920in}{2.434685in}}%
\pgfpathlineto{\pgfqpoint{3.619920in}{2.411091in}}%
\pgfpathmoveto{\pgfqpoint{3.619920in}{2.411091in}}%
\pgfpathlineto{\pgfqpoint{3.619920in}{2.411091in}}%
\pgfpathlineto{\pgfqpoint{3.619920in}{2.434685in}}%
\pgfpathlineto{\pgfqpoint{3.656246in}{2.434685in}}%
\pgfpathlineto{\pgfqpoint{3.656246in}{2.411091in}}%
\pgfpathmoveto{\pgfqpoint{3.583593in}{2.434685in}}%
\pgfpathlineto{\pgfqpoint{3.583593in}{2.434685in}}%
\pgfpathlineto{\pgfqpoint{3.583593in}{2.458279in}}%
\pgfpathlineto{\pgfqpoint{3.619920in}{2.458279in}}%
\pgfpathlineto{\pgfqpoint{3.619920in}{2.434685in}}%
\pgfpathmoveto{\pgfqpoint{3.583593in}{2.458279in}}%
\pgfpathlineto{\pgfqpoint{3.583593in}{2.458279in}}%
\pgfpathlineto{\pgfqpoint{3.583593in}{2.481873in}}%
\pgfpathlineto{\pgfqpoint{3.619920in}{2.481873in}}%
\pgfpathlineto{\pgfqpoint{3.619920in}{2.458279in}}%
\pgfpathmoveto{\pgfqpoint{3.619920in}{2.434685in}}%
\pgfpathlineto{\pgfqpoint{3.619920in}{2.434685in}}%
\pgfpathlineto{\pgfqpoint{3.619920in}{2.458279in}}%
\pgfpathlineto{\pgfqpoint{3.656246in}{2.458279in}}%
\pgfpathlineto{\pgfqpoint{3.656246in}{2.434685in}}%
\pgfpathmoveto{\pgfqpoint{3.510941in}{2.529062in}}%
\pgfpathlineto{\pgfqpoint{3.510941in}{2.529062in}}%
\pgfpathlineto{\pgfqpoint{3.510941in}{2.552657in}}%
\pgfpathlineto{\pgfqpoint{3.547267in}{2.552657in}}%
\pgfpathlineto{\pgfqpoint{3.547267in}{2.529062in}}%
\pgfpathmoveto{\pgfqpoint{3.510941in}{2.552657in}}%
\pgfpathlineto{\pgfqpoint{3.510941in}{2.552657in}}%
\pgfpathlineto{\pgfqpoint{3.510941in}{2.576252in}}%
\pgfpathlineto{\pgfqpoint{3.547267in}{2.576252in}}%
\pgfpathlineto{\pgfqpoint{3.547267in}{2.552657in}}%
\pgfpathmoveto{\pgfqpoint{3.547267in}{2.529062in}}%
\pgfpathlineto{\pgfqpoint{3.547267in}{2.529062in}}%
\pgfpathlineto{\pgfqpoint{3.547267in}{2.552657in}}%
\pgfpathlineto{\pgfqpoint{3.583593in}{2.552657in}}%
\pgfpathlineto{\pgfqpoint{3.583593in}{2.529062in}}%
\pgfpathmoveto{\pgfqpoint{3.583593in}{2.481873in}}%
\pgfpathlineto{\pgfqpoint{3.583593in}{2.481873in}}%
\pgfpathlineto{\pgfqpoint{3.583593in}{2.505467in}}%
\pgfpathlineto{\pgfqpoint{3.619920in}{2.505467in}}%
\pgfpathlineto{\pgfqpoint{3.619920in}{2.481873in}}%
\pgfpathmoveto{\pgfqpoint{3.510941in}{2.576252in}}%
\pgfpathlineto{\pgfqpoint{3.510941in}{2.576252in}}%
\pgfpathlineto{\pgfqpoint{3.510941in}{2.599845in}}%
\pgfpathlineto{\pgfqpoint{3.547267in}{2.599845in}}%
\pgfpathlineto{\pgfqpoint{3.547267in}{2.576252in}}%
\pgfpathmoveto{\pgfqpoint{0.750004in}{0.523593in}}%
\pgfpathlineto{\pgfqpoint{0.750004in}{0.523593in}}%
\pgfpathlineto{\pgfqpoint{0.750004in}{0.535390in}}%
\pgfpathlineto{\pgfqpoint{0.768168in}{0.535390in}}%
\pgfpathlineto{\pgfqpoint{0.768168in}{0.523593in}}%
\pgfpathmoveto{\pgfqpoint{0.750004in}{0.535390in}}%
\pgfpathlineto{\pgfqpoint{0.750004in}{0.535390in}}%
\pgfpathlineto{\pgfqpoint{0.750004in}{0.547188in}}%
\pgfpathlineto{\pgfqpoint{0.768168in}{0.547188in}}%
\pgfpathlineto{\pgfqpoint{0.768168in}{0.535390in}}%
\pgfpathmoveto{\pgfqpoint{0.768168in}{0.535390in}}%
\pgfpathlineto{\pgfqpoint{0.768168in}{0.535390in}}%
\pgfpathlineto{\pgfqpoint{0.768168in}{0.547188in}}%
\pgfpathlineto{\pgfqpoint{0.786332in}{0.547188in}}%
\pgfpathlineto{\pgfqpoint{0.786332in}{0.535390in}}%
\pgfpathmoveto{\pgfqpoint{0.786332in}{0.547188in}}%
\pgfpathlineto{\pgfqpoint{0.786332in}{0.547188in}}%
\pgfpathlineto{\pgfqpoint{0.786332in}{0.558985in}}%
\pgfpathlineto{\pgfqpoint{0.804495in}{0.558985in}}%
\pgfpathlineto{\pgfqpoint{0.804495in}{0.547188in}}%
\pgfpathmoveto{\pgfqpoint{0.786332in}{0.558985in}}%
\pgfpathlineto{\pgfqpoint{0.786332in}{0.558985in}}%
\pgfpathlineto{\pgfqpoint{0.786332in}{0.570782in}}%
\pgfpathlineto{\pgfqpoint{0.804495in}{0.570782in}}%
\pgfpathlineto{\pgfqpoint{0.804495in}{0.558985in}}%
\pgfpathmoveto{\pgfqpoint{0.804495in}{0.558985in}}%
\pgfpathlineto{\pgfqpoint{0.804495in}{0.558985in}}%
\pgfpathlineto{\pgfqpoint{0.804495in}{0.570782in}}%
\pgfpathlineto{\pgfqpoint{0.822659in}{0.570782in}}%
\pgfpathlineto{\pgfqpoint{0.822659in}{0.558985in}}%
\pgfpathmoveto{\pgfqpoint{0.822659in}{0.570782in}}%
\pgfpathlineto{\pgfqpoint{0.822659in}{0.570782in}}%
\pgfpathlineto{\pgfqpoint{0.822659in}{0.582579in}}%
\pgfpathlineto{\pgfqpoint{0.840823in}{0.582579in}}%
\pgfpathlineto{\pgfqpoint{0.840823in}{0.570782in}}%
\pgfpathmoveto{\pgfqpoint{0.822659in}{0.582579in}}%
\pgfpathlineto{\pgfqpoint{0.822659in}{0.582579in}}%
\pgfpathlineto{\pgfqpoint{0.822659in}{0.594376in}}%
\pgfpathlineto{\pgfqpoint{0.840823in}{0.594376in}}%
\pgfpathlineto{\pgfqpoint{0.840823in}{0.582579in}}%
\pgfpathmoveto{\pgfqpoint{0.840823in}{0.582579in}}%
\pgfpathlineto{\pgfqpoint{0.840823in}{0.582579in}}%
\pgfpathlineto{\pgfqpoint{0.840823in}{0.594376in}}%
\pgfpathlineto{\pgfqpoint{0.858987in}{0.594376in}}%
\pgfpathlineto{\pgfqpoint{0.858987in}{0.582579in}}%
\pgfpathmoveto{\pgfqpoint{0.895314in}{0.606173in}}%
\pgfpathlineto{\pgfqpoint{0.895314in}{0.606173in}}%
\pgfpathlineto{\pgfqpoint{0.895314in}{0.617970in}}%
\pgfpathlineto{\pgfqpoint{0.913478in}{0.617970in}}%
\pgfpathlineto{\pgfqpoint{0.913478in}{0.606173in}}%
\pgfpathmoveto{\pgfqpoint{0.931641in}{0.629767in}}%
\pgfpathlineto{\pgfqpoint{0.931641in}{0.629767in}}%
\pgfpathlineto{\pgfqpoint{0.931641in}{0.641564in}}%
\pgfpathlineto{\pgfqpoint{0.949805in}{0.641564in}}%
\pgfpathlineto{\pgfqpoint{0.949805in}{0.629767in}}%
\pgfpathmoveto{\pgfqpoint{0.967968in}{0.653361in}}%
\pgfpathlineto{\pgfqpoint{0.967968in}{0.653361in}}%
\pgfpathlineto{\pgfqpoint{0.967968in}{0.665158in}}%
\pgfpathlineto{\pgfqpoint{0.986132in}{0.665158in}}%
\pgfpathlineto{\pgfqpoint{0.986132in}{0.653361in}}%
\pgfpathmoveto{\pgfqpoint{1.004295in}{0.676955in}}%
\pgfpathlineto{\pgfqpoint{1.004295in}{0.676955in}}%
\pgfpathlineto{\pgfqpoint{1.004295in}{0.688752in}}%
\pgfpathlineto{\pgfqpoint{1.022459in}{0.688752in}}%
\pgfpathlineto{\pgfqpoint{1.022459in}{0.676955in}}%
\pgfpathmoveto{\pgfqpoint{1.040623in}{0.700548in}}%
\pgfpathlineto{\pgfqpoint{1.040623in}{0.700548in}}%
\pgfpathlineto{\pgfqpoint{1.040623in}{0.712344in}}%
\pgfpathlineto{\pgfqpoint{1.058787in}{0.712344in}}%
\pgfpathlineto{\pgfqpoint{1.058787in}{0.700548in}}%
\pgfpathmoveto{\pgfqpoint{1.076951in}{0.724141in}}%
\pgfpathlineto{\pgfqpoint{1.076951in}{0.724141in}}%
\pgfpathlineto{\pgfqpoint{1.076951in}{0.735937in}}%
\pgfpathlineto{\pgfqpoint{1.095115in}{0.735937in}}%
\pgfpathlineto{\pgfqpoint{1.095115in}{0.724141in}}%
\pgfpathmoveto{\pgfqpoint{1.113279in}{0.747734in}}%
\pgfpathlineto{\pgfqpoint{1.113279in}{0.747734in}}%
\pgfpathlineto{\pgfqpoint{1.113279in}{0.759530in}}%
\pgfpathlineto{\pgfqpoint{1.131443in}{0.759530in}}%
\pgfpathlineto{\pgfqpoint{1.131443in}{0.747734in}}%
\pgfpathmoveto{\pgfqpoint{1.149607in}{0.783123in}}%
\pgfpathlineto{\pgfqpoint{1.149607in}{0.783123in}}%
\pgfpathlineto{\pgfqpoint{1.149607in}{0.794920in}}%
\pgfpathlineto{\pgfqpoint{1.167771in}{0.794920in}}%
\pgfpathlineto{\pgfqpoint{1.167771in}{0.783123in}}%
\pgfpathmoveto{\pgfqpoint{1.149607in}{0.794920in}}%
\pgfpathlineto{\pgfqpoint{1.149607in}{0.794920in}}%
\pgfpathlineto{\pgfqpoint{1.149607in}{0.806717in}}%
\pgfpathlineto{\pgfqpoint{1.167771in}{0.806717in}}%
\pgfpathlineto{\pgfqpoint{1.167771in}{0.794920in}}%
\pgfpathmoveto{\pgfqpoint{1.167771in}{0.794920in}}%
\pgfpathlineto{\pgfqpoint{1.167771in}{0.794920in}}%
\pgfpathlineto{\pgfqpoint{1.167771in}{0.806717in}}%
\pgfpathlineto{\pgfqpoint{1.185936in}{0.806717in}}%
\pgfpathlineto{\pgfqpoint{1.185936in}{0.794920in}}%
\pgfpathmoveto{\pgfqpoint{1.185936in}{0.806717in}}%
\pgfpathlineto{\pgfqpoint{1.185936in}{0.806717in}}%
\pgfpathlineto{\pgfqpoint{1.185936in}{0.818514in}}%
\pgfpathlineto{\pgfqpoint{1.204100in}{0.818514in}}%
\pgfpathlineto{\pgfqpoint{1.204100in}{0.806717in}}%
\pgfpathmoveto{\pgfqpoint{1.185936in}{0.818514in}}%
\pgfpathlineto{\pgfqpoint{1.185936in}{0.818514in}}%
\pgfpathlineto{\pgfqpoint{1.185936in}{0.830311in}}%
\pgfpathlineto{\pgfqpoint{1.204100in}{0.830311in}}%
\pgfpathlineto{\pgfqpoint{1.204100in}{0.818514in}}%
\pgfpathmoveto{\pgfqpoint{1.204100in}{0.818514in}}%
\pgfpathlineto{\pgfqpoint{1.204100in}{0.818514in}}%
\pgfpathlineto{\pgfqpoint{1.204100in}{0.830311in}}%
\pgfpathlineto{\pgfqpoint{1.222264in}{0.830311in}}%
\pgfpathlineto{\pgfqpoint{1.222264in}{0.818514in}}%
\pgfpathmoveto{\pgfqpoint{1.222264in}{0.830311in}}%
\pgfpathlineto{\pgfqpoint{1.222264in}{0.830311in}}%
\pgfpathlineto{\pgfqpoint{1.222264in}{0.842108in}}%
\pgfpathlineto{\pgfqpoint{1.240428in}{0.842108in}}%
\pgfpathlineto{\pgfqpoint{1.240428in}{0.830311in}}%
\pgfpathmoveto{\pgfqpoint{1.222264in}{0.842108in}}%
\pgfpathlineto{\pgfqpoint{1.222264in}{0.842108in}}%
\pgfpathlineto{\pgfqpoint{1.222264in}{0.853905in}}%
\pgfpathlineto{\pgfqpoint{1.240428in}{0.853905in}}%
\pgfpathlineto{\pgfqpoint{1.240428in}{0.842108in}}%
\pgfpathmoveto{\pgfqpoint{1.240428in}{0.842108in}}%
\pgfpathlineto{\pgfqpoint{1.240428in}{0.842108in}}%
\pgfpathlineto{\pgfqpoint{1.240428in}{0.853905in}}%
\pgfpathlineto{\pgfqpoint{1.258592in}{0.853905in}}%
\pgfpathlineto{\pgfqpoint{1.258592in}{0.842108in}}%
\pgfpathmoveto{\pgfqpoint{1.258592in}{0.853905in}}%
\pgfpathlineto{\pgfqpoint{1.258592in}{0.853905in}}%
\pgfpathlineto{\pgfqpoint{1.258592in}{0.865701in}}%
\pgfpathlineto{\pgfqpoint{1.276757in}{0.865701in}}%
\pgfpathlineto{\pgfqpoint{1.276757in}{0.853905in}}%
\pgfpathmoveto{\pgfqpoint{1.258592in}{0.865701in}}%
\pgfpathlineto{\pgfqpoint{1.258592in}{0.865701in}}%
\pgfpathlineto{\pgfqpoint{1.258592in}{0.877498in}}%
\pgfpathlineto{\pgfqpoint{1.276757in}{0.877498in}}%
\pgfpathlineto{\pgfqpoint{1.276757in}{0.865701in}}%
\pgfpathmoveto{\pgfqpoint{1.276757in}{0.865701in}}%
\pgfpathlineto{\pgfqpoint{1.276757in}{0.865701in}}%
\pgfpathlineto{\pgfqpoint{1.276757in}{0.877498in}}%
\pgfpathlineto{\pgfqpoint{1.294921in}{0.877498in}}%
\pgfpathlineto{\pgfqpoint{1.294921in}{0.865701in}}%
\pgfpathmoveto{\pgfqpoint{1.294921in}{0.877498in}}%
\pgfpathlineto{\pgfqpoint{1.294921in}{0.877498in}}%
\pgfpathlineto{\pgfqpoint{1.294921in}{0.889296in}}%
\pgfpathlineto{\pgfqpoint{1.313085in}{0.889296in}}%
\pgfpathlineto{\pgfqpoint{1.313085in}{0.877498in}}%
\pgfpathmoveto{\pgfqpoint{1.294921in}{0.889296in}}%
\pgfpathlineto{\pgfqpoint{1.294921in}{0.889296in}}%
\pgfpathlineto{\pgfqpoint{1.294921in}{0.901093in}}%
\pgfpathlineto{\pgfqpoint{1.313085in}{0.901093in}}%
\pgfpathlineto{\pgfqpoint{1.313085in}{0.889296in}}%
\pgfpathmoveto{\pgfqpoint{1.313085in}{0.889296in}}%
\pgfpathlineto{\pgfqpoint{1.313085in}{0.889296in}}%
\pgfpathlineto{\pgfqpoint{1.313085in}{0.901093in}}%
\pgfpathlineto{\pgfqpoint{1.331249in}{0.901093in}}%
\pgfpathlineto{\pgfqpoint{1.331249in}{0.889296in}}%
\pgfpathmoveto{\pgfqpoint{1.331249in}{0.901093in}}%
\pgfpathlineto{\pgfqpoint{1.331249in}{0.901093in}}%
\pgfpathlineto{\pgfqpoint{1.331249in}{0.912890in}}%
\pgfpathlineto{\pgfqpoint{1.349413in}{0.912890in}}%
\pgfpathlineto{\pgfqpoint{1.349413in}{0.901093in}}%
\pgfpathmoveto{\pgfqpoint{1.331249in}{0.912890in}}%
\pgfpathlineto{\pgfqpoint{1.331249in}{0.912890in}}%
\pgfpathlineto{\pgfqpoint{1.331249in}{0.924687in}}%
\pgfpathlineto{\pgfqpoint{1.349413in}{0.924687in}}%
\pgfpathlineto{\pgfqpoint{1.349413in}{0.912890in}}%
\pgfpathmoveto{\pgfqpoint{1.349413in}{0.912890in}}%
\pgfpathlineto{\pgfqpoint{1.349413in}{0.912890in}}%
\pgfpathlineto{\pgfqpoint{1.349413in}{0.924687in}}%
\pgfpathlineto{\pgfqpoint{1.367577in}{0.924687in}}%
\pgfpathlineto{\pgfqpoint{1.367577in}{0.912890in}}%
\pgfpathmoveto{\pgfqpoint{1.367577in}{0.912890in}}%
\pgfpathlineto{\pgfqpoint{1.367577in}{0.912890in}}%
\pgfpathlineto{\pgfqpoint{1.367577in}{0.924687in}}%
\pgfpathlineto{\pgfqpoint{1.385740in}{0.924687in}}%
\pgfpathlineto{\pgfqpoint{1.385740in}{0.912890in}}%
\pgfpathmoveto{\pgfqpoint{1.403904in}{0.936484in}}%
\pgfpathlineto{\pgfqpoint{1.403904in}{0.936484in}}%
\pgfpathlineto{\pgfqpoint{1.403904in}{0.948281in}}%
\pgfpathlineto{\pgfqpoint{1.422068in}{0.948281in}}%
\pgfpathlineto{\pgfqpoint{1.422068in}{0.936484in}}%
\pgfpathmoveto{\pgfqpoint{1.440231in}{0.960078in}}%
\pgfpathlineto{\pgfqpoint{1.440231in}{0.960078in}}%
\pgfpathlineto{\pgfqpoint{1.440231in}{0.971875in}}%
\pgfpathlineto{\pgfqpoint{1.458395in}{0.971875in}}%
\pgfpathlineto{\pgfqpoint{1.458395in}{0.960078in}}%
\pgfpathmoveto{\pgfqpoint{1.476559in}{0.983672in}}%
\pgfpathlineto{\pgfqpoint{1.476559in}{0.983672in}}%
\pgfpathlineto{\pgfqpoint{1.476559in}{0.995469in}}%
\pgfpathlineto{\pgfqpoint{1.494723in}{0.995469in}}%
\pgfpathlineto{\pgfqpoint{1.494723in}{0.983672in}}%
\pgfpathmoveto{\pgfqpoint{1.512887in}{1.007266in}}%
\pgfpathlineto{\pgfqpoint{1.512887in}{1.007266in}}%
\pgfpathlineto{\pgfqpoint{1.512887in}{1.019062in}}%
\pgfpathlineto{\pgfqpoint{1.531052in}{1.019062in}}%
\pgfpathlineto{\pgfqpoint{1.531052in}{1.007266in}}%
\pgfpathmoveto{\pgfqpoint{1.549216in}{1.030859in}}%
\pgfpathlineto{\pgfqpoint{1.549216in}{1.030859in}}%
\pgfpathlineto{\pgfqpoint{1.549216in}{1.042656in}}%
\pgfpathlineto{\pgfqpoint{1.567380in}{1.042656in}}%
\pgfpathlineto{\pgfqpoint{1.567380in}{1.030859in}}%
\pgfpathmoveto{\pgfqpoint{1.585545in}{1.054453in}}%
\pgfpathlineto{\pgfqpoint{1.585545in}{1.054453in}}%
\pgfpathlineto{\pgfqpoint{1.585545in}{1.066250in}}%
\pgfpathlineto{\pgfqpoint{1.603709in}{1.066250in}}%
\pgfpathlineto{\pgfqpoint{1.603709in}{1.054453in}}%
\pgfpathmoveto{\pgfqpoint{1.621874in}{1.078046in}}%
\pgfpathlineto{\pgfqpoint{1.621874in}{1.078046in}}%
\pgfpathlineto{\pgfqpoint{1.621874in}{1.089843in}}%
\pgfpathlineto{\pgfqpoint{1.640038in}{1.089843in}}%
\pgfpathlineto{\pgfqpoint{1.640038in}{1.078046in}}%
\pgfpathmoveto{\pgfqpoint{1.658202in}{1.113436in}}%
\pgfpathlineto{\pgfqpoint{1.658202in}{1.113436in}}%
\pgfpathlineto{\pgfqpoint{1.658202in}{1.125233in}}%
\pgfpathlineto{\pgfqpoint{1.676366in}{1.125233in}}%
\pgfpathlineto{\pgfqpoint{1.676366in}{1.113436in}}%
\pgfpathmoveto{\pgfqpoint{1.658202in}{1.125233in}}%
\pgfpathlineto{\pgfqpoint{1.658202in}{1.125233in}}%
\pgfpathlineto{\pgfqpoint{1.658202in}{1.137030in}}%
\pgfpathlineto{\pgfqpoint{1.676366in}{1.137030in}}%
\pgfpathlineto{\pgfqpoint{1.676366in}{1.125233in}}%
\pgfpathmoveto{\pgfqpoint{1.676366in}{1.125233in}}%
\pgfpathlineto{\pgfqpoint{1.676366in}{1.125233in}}%
\pgfpathlineto{\pgfqpoint{1.676366in}{1.137030in}}%
\pgfpathlineto{\pgfqpoint{1.694530in}{1.137030in}}%
\pgfpathlineto{\pgfqpoint{1.694530in}{1.125233in}}%
\pgfpathmoveto{\pgfqpoint{1.694530in}{1.137030in}}%
\pgfpathlineto{\pgfqpoint{1.694530in}{1.137030in}}%
\pgfpathlineto{\pgfqpoint{1.694530in}{1.148826in}}%
\pgfpathlineto{\pgfqpoint{1.712694in}{1.148826in}}%
\pgfpathlineto{\pgfqpoint{1.712694in}{1.137030in}}%
\pgfpathmoveto{\pgfqpoint{1.694530in}{1.148826in}}%
\pgfpathlineto{\pgfqpoint{1.694530in}{1.148826in}}%
\pgfpathlineto{\pgfqpoint{1.694530in}{1.160623in}}%
\pgfpathlineto{\pgfqpoint{1.712694in}{1.160623in}}%
\pgfpathlineto{\pgfqpoint{1.712694in}{1.148826in}}%
\pgfpathmoveto{\pgfqpoint{1.712694in}{1.148826in}}%
\pgfpathlineto{\pgfqpoint{1.712694in}{1.148826in}}%
\pgfpathlineto{\pgfqpoint{1.712694in}{1.160623in}}%
\pgfpathlineto{\pgfqpoint{1.730858in}{1.160623in}}%
\pgfpathlineto{\pgfqpoint{1.730858in}{1.148826in}}%
\pgfpathmoveto{\pgfqpoint{1.730858in}{1.160623in}}%
\pgfpathlineto{\pgfqpoint{1.730858in}{1.160623in}}%
\pgfpathlineto{\pgfqpoint{1.730858in}{1.172420in}}%
\pgfpathlineto{\pgfqpoint{1.749022in}{1.172420in}}%
\pgfpathlineto{\pgfqpoint{1.749022in}{1.160623in}}%
\pgfpathmoveto{\pgfqpoint{1.730858in}{1.172420in}}%
\pgfpathlineto{\pgfqpoint{1.730858in}{1.172420in}}%
\pgfpathlineto{\pgfqpoint{1.730858in}{1.184217in}}%
\pgfpathlineto{\pgfqpoint{1.749022in}{1.184217in}}%
\pgfpathlineto{\pgfqpoint{1.749022in}{1.172420in}}%
\pgfpathmoveto{\pgfqpoint{1.749022in}{1.172420in}}%
\pgfpathlineto{\pgfqpoint{1.749022in}{1.172420in}}%
\pgfpathlineto{\pgfqpoint{1.749022in}{1.184217in}}%
\pgfpathlineto{\pgfqpoint{1.767187in}{1.184217in}}%
\pgfpathlineto{\pgfqpoint{1.767187in}{1.172420in}}%
\pgfpathmoveto{\pgfqpoint{1.767187in}{1.184217in}}%
\pgfpathlineto{\pgfqpoint{1.767187in}{1.184217in}}%
\pgfpathlineto{\pgfqpoint{1.767187in}{1.196014in}}%
\pgfpathlineto{\pgfqpoint{1.785351in}{1.196014in}}%
\pgfpathlineto{\pgfqpoint{1.785351in}{1.184217in}}%
\pgfpathmoveto{\pgfqpoint{1.767187in}{1.196014in}}%
\pgfpathlineto{\pgfqpoint{1.767187in}{1.196014in}}%
\pgfpathlineto{\pgfqpoint{1.767187in}{1.207811in}}%
\pgfpathlineto{\pgfqpoint{1.785351in}{1.207811in}}%
\pgfpathlineto{\pgfqpoint{1.785351in}{1.196014in}}%
\pgfpathmoveto{\pgfqpoint{1.785351in}{1.196014in}}%
\pgfpathlineto{\pgfqpoint{1.785351in}{1.196014in}}%
\pgfpathlineto{\pgfqpoint{1.785351in}{1.207811in}}%
\pgfpathlineto{\pgfqpoint{1.803516in}{1.207811in}}%
\pgfpathlineto{\pgfqpoint{1.803516in}{1.196014in}}%
\pgfpathmoveto{\pgfqpoint{1.803516in}{1.207811in}}%
\pgfpathlineto{\pgfqpoint{1.803516in}{1.207811in}}%
\pgfpathlineto{\pgfqpoint{1.803516in}{1.219608in}}%
\pgfpathlineto{\pgfqpoint{1.821680in}{1.219608in}}%
\pgfpathlineto{\pgfqpoint{1.821680in}{1.207811in}}%
\pgfpathmoveto{\pgfqpoint{1.803516in}{1.219608in}}%
\pgfpathlineto{\pgfqpoint{1.803516in}{1.219608in}}%
\pgfpathlineto{\pgfqpoint{1.803516in}{1.231405in}}%
\pgfpathlineto{\pgfqpoint{1.821680in}{1.231405in}}%
\pgfpathlineto{\pgfqpoint{1.821680in}{1.219608in}}%
\pgfpathmoveto{\pgfqpoint{1.821680in}{1.219608in}}%
\pgfpathlineto{\pgfqpoint{1.821680in}{1.219608in}}%
\pgfpathlineto{\pgfqpoint{1.821680in}{1.231405in}}%
\pgfpathlineto{\pgfqpoint{1.839845in}{1.231405in}}%
\pgfpathlineto{\pgfqpoint{1.839845in}{1.219608in}}%
\pgfpathmoveto{\pgfqpoint{1.839845in}{1.231405in}}%
\pgfpathlineto{\pgfqpoint{1.839845in}{1.231405in}}%
\pgfpathlineto{\pgfqpoint{1.839845in}{1.243202in}}%
\pgfpathlineto{\pgfqpoint{1.858009in}{1.243202in}}%
\pgfpathlineto{\pgfqpoint{1.858009in}{1.231405in}}%
\pgfpathmoveto{\pgfqpoint{1.839845in}{1.243202in}}%
\pgfpathlineto{\pgfqpoint{1.839845in}{1.243202in}}%
\pgfpathlineto{\pgfqpoint{1.839845in}{1.254999in}}%
\pgfpathlineto{\pgfqpoint{1.858009in}{1.254999in}}%
\pgfpathlineto{\pgfqpoint{1.858009in}{1.243202in}}%
\pgfpathmoveto{\pgfqpoint{1.858009in}{1.243202in}}%
\pgfpathlineto{\pgfqpoint{1.858009in}{1.243202in}}%
\pgfpathlineto{\pgfqpoint{1.858009in}{1.254999in}}%
\pgfpathlineto{\pgfqpoint{1.876174in}{1.254999in}}%
\pgfpathlineto{\pgfqpoint{1.876174in}{1.243202in}}%
\pgfpathmoveto{\pgfqpoint{1.876174in}{1.254999in}}%
\pgfpathlineto{\pgfqpoint{1.876174in}{1.254999in}}%
\pgfpathlineto{\pgfqpoint{1.876174in}{1.266796in}}%
\pgfpathlineto{\pgfqpoint{1.894339in}{1.266796in}}%
\pgfpathlineto{\pgfqpoint{1.894339in}{1.254999in}}%
\pgfpathmoveto{\pgfqpoint{1.876174in}{1.266796in}}%
\pgfpathlineto{\pgfqpoint{1.876174in}{1.266796in}}%
\pgfpathlineto{\pgfqpoint{1.876174in}{1.278594in}}%
\pgfpathlineto{\pgfqpoint{1.894339in}{1.278594in}}%
\pgfpathlineto{\pgfqpoint{1.894339in}{1.266796in}}%
\pgfpathmoveto{\pgfqpoint{1.894339in}{1.266796in}}%
\pgfpathlineto{\pgfqpoint{1.894339in}{1.266796in}}%
\pgfpathlineto{\pgfqpoint{1.894339in}{1.278594in}}%
\pgfpathlineto{\pgfqpoint{1.912503in}{1.278594in}}%
\pgfpathlineto{\pgfqpoint{1.912503in}{1.266796in}}%
\pgfpathmoveto{\pgfqpoint{1.912503in}{1.278594in}}%
\pgfpathlineto{\pgfqpoint{1.912503in}{1.278594in}}%
\pgfpathlineto{\pgfqpoint{1.912503in}{1.290391in}}%
\pgfpathlineto{\pgfqpoint{1.930667in}{1.290391in}}%
\pgfpathlineto{\pgfqpoint{1.930667in}{1.278594in}}%
\pgfpathmoveto{\pgfqpoint{1.912503in}{1.290391in}}%
\pgfpathlineto{\pgfqpoint{1.912503in}{1.290391in}}%
\pgfpathlineto{\pgfqpoint{1.912503in}{1.302188in}}%
\pgfpathlineto{\pgfqpoint{1.930667in}{1.302188in}}%
\pgfpathlineto{\pgfqpoint{1.930667in}{1.290391in}}%
\pgfpathmoveto{\pgfqpoint{1.930667in}{1.290391in}}%
\pgfpathlineto{\pgfqpoint{1.930667in}{1.290391in}}%
\pgfpathlineto{\pgfqpoint{1.930667in}{1.302188in}}%
\pgfpathlineto{\pgfqpoint{1.948830in}{1.302188in}}%
\pgfpathlineto{\pgfqpoint{1.948830in}{1.290391in}}%
\pgfpathmoveto{\pgfqpoint{1.948830in}{1.302188in}}%
\pgfpathlineto{\pgfqpoint{1.948830in}{1.302188in}}%
\pgfpathlineto{\pgfqpoint{1.948830in}{1.313985in}}%
\pgfpathlineto{\pgfqpoint{1.966993in}{1.313985in}}%
\pgfpathlineto{\pgfqpoint{1.966993in}{1.302188in}}%
\pgfpathmoveto{\pgfqpoint{1.948830in}{1.313985in}}%
\pgfpathlineto{\pgfqpoint{1.948830in}{1.313985in}}%
\pgfpathlineto{\pgfqpoint{1.948830in}{1.325782in}}%
\pgfpathlineto{\pgfqpoint{1.966993in}{1.325782in}}%
\pgfpathlineto{\pgfqpoint{1.966993in}{1.313985in}}%
\pgfpathmoveto{\pgfqpoint{1.966993in}{1.313985in}}%
\pgfpathlineto{\pgfqpoint{1.966993in}{1.313985in}}%
\pgfpathlineto{\pgfqpoint{1.966993in}{1.325782in}}%
\pgfpathlineto{\pgfqpoint{1.985157in}{1.325782in}}%
\pgfpathlineto{\pgfqpoint{1.985157in}{1.313985in}}%
\pgfpathmoveto{\pgfqpoint{1.985157in}{1.313985in}}%
\pgfpathlineto{\pgfqpoint{1.985157in}{1.313985in}}%
\pgfpathlineto{\pgfqpoint{1.985157in}{1.325782in}}%
\pgfpathlineto{\pgfqpoint{2.003320in}{1.325782in}}%
\pgfpathlineto{\pgfqpoint{2.003320in}{1.313985in}}%
\pgfpathmoveto{\pgfqpoint{2.021483in}{1.337579in}}%
\pgfpathlineto{\pgfqpoint{2.021483in}{1.337579in}}%
\pgfpathlineto{\pgfqpoint{2.021483in}{1.349376in}}%
\pgfpathlineto{\pgfqpoint{2.039647in}{1.349376in}}%
\pgfpathlineto{\pgfqpoint{2.039647in}{1.337579in}}%
\pgfpathmoveto{\pgfqpoint{2.057810in}{1.361173in}}%
\pgfpathlineto{\pgfqpoint{2.057810in}{1.361173in}}%
\pgfpathlineto{\pgfqpoint{2.057810in}{1.372970in}}%
\pgfpathlineto{\pgfqpoint{2.075974in}{1.372970in}}%
\pgfpathlineto{\pgfqpoint{2.075974in}{1.361173in}}%
\pgfpathmoveto{\pgfqpoint{2.094138in}{1.384767in}}%
\pgfpathlineto{\pgfqpoint{2.094138in}{1.384767in}}%
\pgfpathlineto{\pgfqpoint{2.094138in}{1.396564in}}%
\pgfpathlineto{\pgfqpoint{2.112302in}{1.396564in}}%
\pgfpathlineto{\pgfqpoint{2.112302in}{1.384767in}}%
\pgfpathmoveto{\pgfqpoint{2.130466in}{1.408361in}}%
\pgfpathlineto{\pgfqpoint{2.130466in}{1.408361in}}%
\pgfpathlineto{\pgfqpoint{2.130466in}{1.420158in}}%
\pgfpathlineto{\pgfqpoint{2.148629in}{1.420158in}}%
\pgfpathlineto{\pgfqpoint{2.148629in}{1.408361in}}%
\pgfpathmoveto{\pgfqpoint{2.166793in}{1.431955in}}%
\pgfpathlineto{\pgfqpoint{2.166793in}{1.431955in}}%
\pgfpathlineto{\pgfqpoint{2.166793in}{1.443752in}}%
\pgfpathlineto{\pgfqpoint{2.184957in}{1.443752in}}%
\pgfpathlineto{\pgfqpoint{2.184957in}{1.431955in}}%
\pgfpathmoveto{\pgfqpoint{2.203121in}{1.455549in}}%
\pgfpathlineto{\pgfqpoint{2.203121in}{1.455549in}}%
\pgfpathlineto{\pgfqpoint{2.203121in}{1.467345in}}%
\pgfpathlineto{\pgfqpoint{2.221286in}{1.467345in}}%
\pgfpathlineto{\pgfqpoint{2.221286in}{1.455549in}}%
\pgfpathmoveto{\pgfqpoint{2.239451in}{1.479142in}}%
\pgfpathlineto{\pgfqpoint{2.239451in}{1.479142in}}%
\pgfpathlineto{\pgfqpoint{2.239451in}{1.490938in}}%
\pgfpathlineto{\pgfqpoint{2.257615in}{1.490938in}}%
\pgfpathlineto{\pgfqpoint{2.257615in}{1.479142in}}%
\pgfpathmoveto{\pgfqpoint{2.275780in}{1.502735in}}%
\pgfpathlineto{\pgfqpoint{2.275780in}{1.502735in}}%
\pgfpathlineto{\pgfqpoint{2.275780in}{1.514531in}}%
\pgfpathlineto{\pgfqpoint{2.293945in}{1.514531in}}%
\pgfpathlineto{\pgfqpoint{2.293945in}{1.502735in}}%
\pgfpathmoveto{\pgfqpoint{2.312110in}{1.538125in}}%
\pgfpathlineto{\pgfqpoint{2.312110in}{1.538125in}}%
\pgfpathlineto{\pgfqpoint{2.312110in}{1.549921in}}%
\pgfpathlineto{\pgfqpoint{2.330274in}{1.549921in}}%
\pgfpathlineto{\pgfqpoint{2.330274in}{1.538125in}}%
\pgfpathmoveto{\pgfqpoint{2.312110in}{1.549921in}}%
\pgfpathlineto{\pgfqpoint{2.312110in}{1.549921in}}%
\pgfpathlineto{\pgfqpoint{2.312110in}{1.561718in}}%
\pgfpathlineto{\pgfqpoint{2.330274in}{1.561718in}}%
\pgfpathlineto{\pgfqpoint{2.330274in}{1.549921in}}%
\pgfpathmoveto{\pgfqpoint{2.330274in}{1.549921in}}%
\pgfpathlineto{\pgfqpoint{2.330274in}{1.549921in}}%
\pgfpathlineto{\pgfqpoint{2.330274in}{1.561718in}}%
\pgfpathlineto{\pgfqpoint{2.348439in}{1.561718in}}%
\pgfpathlineto{\pgfqpoint{2.348439in}{1.549921in}}%
\pgfpathmoveto{\pgfqpoint{2.348439in}{1.561718in}}%
\pgfpathlineto{\pgfqpoint{2.348439in}{1.561718in}}%
\pgfpathlineto{\pgfqpoint{2.348439in}{1.573515in}}%
\pgfpathlineto{\pgfqpoint{2.366603in}{1.573515in}}%
\pgfpathlineto{\pgfqpoint{2.366603in}{1.561718in}}%
\pgfpathmoveto{\pgfqpoint{2.348439in}{1.573515in}}%
\pgfpathlineto{\pgfqpoint{2.348439in}{1.573515in}}%
\pgfpathlineto{\pgfqpoint{2.348439in}{1.585311in}}%
\pgfpathlineto{\pgfqpoint{2.366603in}{1.585311in}}%
\pgfpathlineto{\pgfqpoint{2.366603in}{1.573515in}}%
\pgfpathmoveto{\pgfqpoint{2.366603in}{1.573515in}}%
\pgfpathlineto{\pgfqpoint{2.366603in}{1.573515in}}%
\pgfpathlineto{\pgfqpoint{2.366603in}{1.585311in}}%
\pgfpathlineto{\pgfqpoint{2.384766in}{1.585311in}}%
\pgfpathlineto{\pgfqpoint{2.384766in}{1.573515in}}%
\pgfpathmoveto{\pgfqpoint{2.384766in}{1.585311in}}%
\pgfpathlineto{\pgfqpoint{2.384766in}{1.585311in}}%
\pgfpathlineto{\pgfqpoint{2.384766in}{1.597108in}}%
\pgfpathlineto{\pgfqpoint{2.402930in}{1.597108in}}%
\pgfpathlineto{\pgfqpoint{2.402930in}{1.585311in}}%
\pgfpathmoveto{\pgfqpoint{2.384766in}{1.597108in}}%
\pgfpathlineto{\pgfqpoint{2.384766in}{1.597108in}}%
\pgfpathlineto{\pgfqpoint{2.384766in}{1.608905in}}%
\pgfpathlineto{\pgfqpoint{2.402930in}{1.608905in}}%
\pgfpathlineto{\pgfqpoint{2.402930in}{1.597108in}}%
\pgfpathmoveto{\pgfqpoint{2.402930in}{1.597108in}}%
\pgfpathlineto{\pgfqpoint{2.402930in}{1.597108in}}%
\pgfpathlineto{\pgfqpoint{2.402930in}{1.608905in}}%
\pgfpathlineto{\pgfqpoint{2.421093in}{1.608905in}}%
\pgfpathlineto{\pgfqpoint{2.421093in}{1.597108in}}%
\pgfpathmoveto{\pgfqpoint{2.421093in}{1.608905in}}%
\pgfpathlineto{\pgfqpoint{2.421093in}{1.608905in}}%
\pgfpathlineto{\pgfqpoint{2.421093in}{1.620701in}}%
\pgfpathlineto{\pgfqpoint{2.439257in}{1.620701in}}%
\pgfpathlineto{\pgfqpoint{2.439257in}{1.608905in}}%
\pgfpathmoveto{\pgfqpoint{2.421093in}{1.620701in}}%
\pgfpathlineto{\pgfqpoint{2.421093in}{1.620701in}}%
\pgfpathlineto{\pgfqpoint{2.421093in}{1.632498in}}%
\pgfpathlineto{\pgfqpoint{2.439257in}{1.632498in}}%
\pgfpathlineto{\pgfqpoint{2.439257in}{1.620701in}}%
\pgfpathmoveto{\pgfqpoint{2.439257in}{1.620701in}}%
\pgfpathlineto{\pgfqpoint{2.439257in}{1.620701in}}%
\pgfpathlineto{\pgfqpoint{2.439257in}{1.632498in}}%
\pgfpathlineto{\pgfqpoint{2.457420in}{1.632498in}}%
\pgfpathlineto{\pgfqpoint{2.457420in}{1.620701in}}%
\pgfpathmoveto{\pgfqpoint{2.457420in}{1.632498in}}%
\pgfpathlineto{\pgfqpoint{2.457420in}{1.632498in}}%
\pgfpathlineto{\pgfqpoint{2.457420in}{1.644295in}}%
\pgfpathlineto{\pgfqpoint{2.475584in}{1.644295in}}%
\pgfpathlineto{\pgfqpoint{2.475584in}{1.632498in}}%
\pgfpathmoveto{\pgfqpoint{2.457420in}{1.644295in}}%
\pgfpathlineto{\pgfqpoint{2.457420in}{1.644295in}}%
\pgfpathlineto{\pgfqpoint{2.457420in}{1.656092in}}%
\pgfpathlineto{\pgfqpoint{2.475584in}{1.656092in}}%
\pgfpathlineto{\pgfqpoint{2.475584in}{1.644295in}}%
\pgfpathmoveto{\pgfqpoint{2.475584in}{1.644295in}}%
\pgfpathlineto{\pgfqpoint{2.475584in}{1.644295in}}%
\pgfpathlineto{\pgfqpoint{2.475584in}{1.656092in}}%
\pgfpathlineto{\pgfqpoint{2.493747in}{1.656092in}}%
\pgfpathlineto{\pgfqpoint{2.493747in}{1.644295in}}%
\pgfpathmoveto{\pgfqpoint{2.493747in}{1.656092in}}%
\pgfpathlineto{\pgfqpoint{2.493747in}{1.656092in}}%
\pgfpathlineto{\pgfqpoint{2.493747in}{1.667889in}}%
\pgfpathlineto{\pgfqpoint{2.511912in}{1.667889in}}%
\pgfpathlineto{\pgfqpoint{2.511912in}{1.656092in}}%
\pgfpathmoveto{\pgfqpoint{2.493747in}{1.667889in}}%
\pgfpathlineto{\pgfqpoint{2.493747in}{1.667889in}}%
\pgfpathlineto{\pgfqpoint{2.493747in}{1.679686in}}%
\pgfpathlineto{\pgfqpoint{2.511912in}{1.679686in}}%
\pgfpathlineto{\pgfqpoint{2.511912in}{1.667889in}}%
\pgfpathmoveto{\pgfqpoint{2.511912in}{1.667889in}}%
\pgfpathlineto{\pgfqpoint{2.511912in}{1.667889in}}%
\pgfpathlineto{\pgfqpoint{2.511912in}{1.679686in}}%
\pgfpathlineto{\pgfqpoint{2.530077in}{1.679686in}}%
\pgfpathlineto{\pgfqpoint{2.530077in}{1.667889in}}%
\pgfpathmoveto{\pgfqpoint{2.530077in}{1.679686in}}%
\pgfpathlineto{\pgfqpoint{2.530077in}{1.679686in}}%
\pgfpathlineto{\pgfqpoint{2.530077in}{1.691483in}}%
\pgfpathlineto{\pgfqpoint{2.548241in}{1.691483in}}%
\pgfpathlineto{\pgfqpoint{2.548241in}{1.679686in}}%
\pgfpathmoveto{\pgfqpoint{2.530077in}{1.691483in}}%
\pgfpathlineto{\pgfqpoint{2.530077in}{1.691483in}}%
\pgfpathlineto{\pgfqpoint{2.530077in}{1.703280in}}%
\pgfpathlineto{\pgfqpoint{2.548241in}{1.703280in}}%
\pgfpathlineto{\pgfqpoint{2.548241in}{1.691483in}}%
\pgfpathmoveto{\pgfqpoint{2.548241in}{1.691483in}}%
\pgfpathlineto{\pgfqpoint{2.548241in}{1.691483in}}%
\pgfpathlineto{\pgfqpoint{2.548241in}{1.703280in}}%
\pgfpathlineto{\pgfqpoint{2.566406in}{1.703280in}}%
\pgfpathlineto{\pgfqpoint{2.566406in}{1.691483in}}%
\pgfpathmoveto{\pgfqpoint{2.566406in}{1.703280in}}%
\pgfpathlineto{\pgfqpoint{2.566406in}{1.703280in}}%
\pgfpathlineto{\pgfqpoint{2.566406in}{1.715077in}}%
\pgfpathlineto{\pgfqpoint{2.584571in}{1.715077in}}%
\pgfpathlineto{\pgfqpoint{2.584571in}{1.703280in}}%
\pgfpathmoveto{\pgfqpoint{2.566406in}{1.715077in}}%
\pgfpathlineto{\pgfqpoint{2.566406in}{1.715077in}}%
\pgfpathlineto{\pgfqpoint{2.566406in}{1.726874in}}%
\pgfpathlineto{\pgfqpoint{2.584571in}{1.726874in}}%
\pgfpathlineto{\pgfqpoint{2.584571in}{1.715077in}}%
\pgfpathmoveto{\pgfqpoint{2.584571in}{1.715077in}}%
\pgfpathlineto{\pgfqpoint{2.584571in}{1.715077in}}%
\pgfpathlineto{\pgfqpoint{2.584571in}{1.726874in}}%
\pgfpathlineto{\pgfqpoint{2.602735in}{1.726874in}}%
\pgfpathlineto{\pgfqpoint{2.602735in}{1.715077in}}%
\pgfpathmoveto{\pgfqpoint{2.602735in}{1.726874in}}%
\pgfpathlineto{\pgfqpoint{2.602735in}{1.726874in}}%
\pgfpathlineto{\pgfqpoint{2.602735in}{1.738671in}}%
\pgfpathlineto{\pgfqpoint{2.620900in}{1.738671in}}%
\pgfpathlineto{\pgfqpoint{2.620900in}{1.726874in}}%
\pgfpathmoveto{\pgfqpoint{2.602735in}{1.738671in}}%
\pgfpathlineto{\pgfqpoint{2.602735in}{1.738671in}}%
\pgfpathlineto{\pgfqpoint{2.602735in}{1.750468in}}%
\pgfpathlineto{\pgfqpoint{2.620900in}{1.750468in}}%
\pgfpathlineto{\pgfqpoint{2.620900in}{1.738671in}}%
\pgfpathmoveto{\pgfqpoint{2.620900in}{1.738671in}}%
\pgfpathlineto{\pgfqpoint{2.620900in}{1.738671in}}%
\pgfpathlineto{\pgfqpoint{2.620900in}{1.750468in}}%
\pgfpathlineto{\pgfqpoint{2.639065in}{1.750468in}}%
\pgfpathlineto{\pgfqpoint{2.639065in}{1.738671in}}%
\pgfpathmoveto{\pgfqpoint{2.639065in}{1.750468in}}%
\pgfpathlineto{\pgfqpoint{2.639065in}{1.750468in}}%
\pgfpathlineto{\pgfqpoint{2.639065in}{1.762265in}}%
\pgfpathlineto{\pgfqpoint{2.657229in}{1.762265in}}%
\pgfpathlineto{\pgfqpoint{2.657229in}{1.750468in}}%
\pgfpathmoveto{\pgfqpoint{2.639065in}{1.762265in}}%
\pgfpathlineto{\pgfqpoint{2.639065in}{1.762265in}}%
\pgfpathlineto{\pgfqpoint{2.639065in}{1.774062in}}%
\pgfpathlineto{\pgfqpoint{2.657229in}{1.774062in}}%
\pgfpathlineto{\pgfqpoint{2.657229in}{1.762265in}}%
\pgfpathmoveto{\pgfqpoint{2.657229in}{1.762265in}}%
\pgfpathlineto{\pgfqpoint{2.657229in}{1.762265in}}%
\pgfpathlineto{\pgfqpoint{2.657229in}{1.774062in}}%
\pgfpathlineto{\pgfqpoint{2.675393in}{1.774062in}}%
\pgfpathlineto{\pgfqpoint{2.675393in}{1.762265in}}%
\pgfpathmoveto{\pgfqpoint{2.675393in}{1.774062in}}%
\pgfpathlineto{\pgfqpoint{2.675393in}{1.774062in}}%
\pgfpathlineto{\pgfqpoint{2.675393in}{1.785859in}}%
\pgfpathlineto{\pgfqpoint{2.693557in}{1.785859in}}%
\pgfpathlineto{\pgfqpoint{2.693557in}{1.774062in}}%
\pgfpathmoveto{\pgfqpoint{2.675393in}{1.785859in}}%
\pgfpathlineto{\pgfqpoint{2.675393in}{1.785859in}}%
\pgfpathlineto{\pgfqpoint{2.675393in}{1.797656in}}%
\pgfpathlineto{\pgfqpoint{2.693557in}{1.797656in}}%
\pgfpathlineto{\pgfqpoint{2.693557in}{1.785859in}}%
\pgfpathmoveto{\pgfqpoint{2.693557in}{1.785859in}}%
\pgfpathlineto{\pgfqpoint{2.693557in}{1.785859in}}%
\pgfpathlineto{\pgfqpoint{2.693557in}{1.797656in}}%
\pgfpathlineto{\pgfqpoint{2.711721in}{1.797656in}}%
\pgfpathlineto{\pgfqpoint{2.711721in}{1.785859in}}%
\pgfpathmoveto{\pgfqpoint{2.711721in}{1.797656in}}%
\pgfpathlineto{\pgfqpoint{2.711721in}{1.797656in}}%
\pgfpathlineto{\pgfqpoint{2.711721in}{1.809453in}}%
\pgfpathlineto{\pgfqpoint{2.729885in}{1.809453in}}%
\pgfpathlineto{\pgfqpoint{2.729885in}{1.797656in}}%
\pgfpathmoveto{\pgfqpoint{2.711721in}{1.809453in}}%
\pgfpathlineto{\pgfqpoint{2.711721in}{1.809453in}}%
\pgfpathlineto{\pgfqpoint{2.711721in}{1.821250in}}%
\pgfpathlineto{\pgfqpoint{2.729885in}{1.821250in}}%
\pgfpathlineto{\pgfqpoint{2.729885in}{1.809453in}}%
\pgfpathmoveto{\pgfqpoint{2.729885in}{1.809453in}}%
\pgfpathlineto{\pgfqpoint{2.729885in}{1.809453in}}%
\pgfpathlineto{\pgfqpoint{2.729885in}{1.821250in}}%
\pgfpathlineto{\pgfqpoint{2.748049in}{1.821250in}}%
\pgfpathlineto{\pgfqpoint{2.748049in}{1.809453in}}%
\pgfpathmoveto{\pgfqpoint{2.748049in}{1.821250in}}%
\pgfpathlineto{\pgfqpoint{2.748049in}{1.821250in}}%
\pgfpathlineto{\pgfqpoint{2.748049in}{1.833047in}}%
\pgfpathlineto{\pgfqpoint{2.766213in}{1.833047in}}%
\pgfpathlineto{\pgfqpoint{2.766213in}{1.821250in}}%
\pgfpathmoveto{\pgfqpoint{2.748049in}{1.833047in}}%
\pgfpathlineto{\pgfqpoint{2.748049in}{1.833047in}}%
\pgfpathlineto{\pgfqpoint{2.748049in}{1.844844in}}%
\pgfpathlineto{\pgfqpoint{2.766213in}{1.844844in}}%
\pgfpathlineto{\pgfqpoint{2.766213in}{1.833047in}}%
\pgfpathmoveto{\pgfqpoint{2.766213in}{1.833047in}}%
\pgfpathlineto{\pgfqpoint{2.766213in}{1.833047in}}%
\pgfpathlineto{\pgfqpoint{2.766213in}{1.844844in}}%
\pgfpathlineto{\pgfqpoint{2.784377in}{1.844844in}}%
\pgfpathlineto{\pgfqpoint{2.784377in}{1.833047in}}%
\pgfpathmoveto{\pgfqpoint{2.784377in}{1.844844in}}%
\pgfpathlineto{\pgfqpoint{2.784377in}{1.844844in}}%
\pgfpathlineto{\pgfqpoint{2.784377in}{1.856641in}}%
\pgfpathlineto{\pgfqpoint{2.802541in}{1.856641in}}%
\pgfpathlineto{\pgfqpoint{2.802541in}{1.844844in}}%
\pgfpathmoveto{\pgfqpoint{2.784377in}{1.856641in}}%
\pgfpathlineto{\pgfqpoint{2.784377in}{1.856641in}}%
\pgfpathlineto{\pgfqpoint{2.784377in}{1.868438in}}%
\pgfpathlineto{\pgfqpoint{2.802541in}{1.868438in}}%
\pgfpathlineto{\pgfqpoint{2.802541in}{1.856641in}}%
\pgfpathmoveto{\pgfqpoint{2.802541in}{1.856641in}}%
\pgfpathlineto{\pgfqpoint{2.802541in}{1.856641in}}%
\pgfpathlineto{\pgfqpoint{2.802541in}{1.868438in}}%
\pgfpathlineto{\pgfqpoint{2.820706in}{1.868438in}}%
\pgfpathlineto{\pgfqpoint{2.820706in}{1.856641in}}%
\pgfpathmoveto{\pgfqpoint{2.820706in}{1.868438in}}%
\pgfpathlineto{\pgfqpoint{2.820706in}{1.868438in}}%
\pgfpathlineto{\pgfqpoint{2.820706in}{1.880235in}}%
\pgfpathlineto{\pgfqpoint{2.838870in}{1.880235in}}%
\pgfpathlineto{\pgfqpoint{2.838870in}{1.868438in}}%
\pgfpathmoveto{\pgfqpoint{2.820706in}{1.880235in}}%
\pgfpathlineto{\pgfqpoint{2.820706in}{1.880235in}}%
\pgfpathlineto{\pgfqpoint{2.820706in}{1.892032in}}%
\pgfpathlineto{\pgfqpoint{2.838870in}{1.892032in}}%
\pgfpathlineto{\pgfqpoint{2.838870in}{1.880235in}}%
\pgfpathmoveto{\pgfqpoint{2.838870in}{1.880235in}}%
\pgfpathlineto{\pgfqpoint{2.838870in}{1.880235in}}%
\pgfpathlineto{\pgfqpoint{2.838870in}{1.892032in}}%
\pgfpathlineto{\pgfqpoint{2.857034in}{1.892032in}}%
\pgfpathlineto{\pgfqpoint{2.857034in}{1.880235in}}%
\pgfpathmoveto{\pgfqpoint{2.857034in}{1.892032in}}%
\pgfpathlineto{\pgfqpoint{2.857034in}{1.892032in}}%
\pgfpathlineto{\pgfqpoint{2.857034in}{1.903829in}}%
\pgfpathlineto{\pgfqpoint{2.875198in}{1.903829in}}%
\pgfpathlineto{\pgfqpoint{2.875198in}{1.892032in}}%
\pgfpathmoveto{\pgfqpoint{2.857034in}{1.903829in}}%
\pgfpathlineto{\pgfqpoint{2.857034in}{1.903829in}}%
\pgfpathlineto{\pgfqpoint{2.857034in}{1.915626in}}%
\pgfpathlineto{\pgfqpoint{2.875198in}{1.915626in}}%
\pgfpathlineto{\pgfqpoint{2.875198in}{1.903829in}}%
\pgfpathmoveto{\pgfqpoint{2.875198in}{1.903829in}}%
\pgfpathlineto{\pgfqpoint{2.875198in}{1.903829in}}%
\pgfpathlineto{\pgfqpoint{2.875198in}{1.915626in}}%
\pgfpathlineto{\pgfqpoint{2.893363in}{1.915626in}}%
\pgfpathlineto{\pgfqpoint{2.893363in}{1.903829in}}%
\pgfpathmoveto{\pgfqpoint{2.893363in}{1.915626in}}%
\pgfpathlineto{\pgfqpoint{2.893363in}{1.915626in}}%
\pgfpathlineto{\pgfqpoint{2.893363in}{1.927422in}}%
\pgfpathlineto{\pgfqpoint{2.911527in}{1.927422in}}%
\pgfpathlineto{\pgfqpoint{2.911527in}{1.915626in}}%
\pgfpathmoveto{\pgfqpoint{2.893363in}{1.927422in}}%
\pgfpathlineto{\pgfqpoint{2.893363in}{1.927422in}}%
\pgfpathlineto{\pgfqpoint{2.893363in}{1.939219in}}%
\pgfpathlineto{\pgfqpoint{2.911527in}{1.939219in}}%
\pgfpathlineto{\pgfqpoint{2.911527in}{1.927422in}}%
\pgfpathmoveto{\pgfqpoint{2.911527in}{1.927422in}}%
\pgfpathlineto{\pgfqpoint{2.911527in}{1.927422in}}%
\pgfpathlineto{\pgfqpoint{2.911527in}{1.939219in}}%
\pgfpathlineto{\pgfqpoint{2.929691in}{1.939219in}}%
\pgfpathlineto{\pgfqpoint{2.929691in}{1.927422in}}%
\pgfpathmoveto{\pgfqpoint{2.893363in}{3.402031in}}%
\pgfpathlineto{\pgfqpoint{2.893363in}{3.402031in}}%
\pgfpathlineto{\pgfqpoint{2.893363in}{3.413828in}}%
\pgfpathlineto{\pgfqpoint{2.911527in}{3.413828in}}%
\pgfpathlineto{\pgfqpoint{2.911527in}{3.402031in}}%
\pgfpathmoveto{\pgfqpoint{2.893363in}{3.413828in}}%
\pgfpathlineto{\pgfqpoint{2.893363in}{3.413828in}}%
\pgfpathlineto{\pgfqpoint{2.893363in}{3.425625in}}%
\pgfpathlineto{\pgfqpoint{2.911527in}{3.425625in}}%
\pgfpathlineto{\pgfqpoint{2.911527in}{3.413828in}}%
\pgfpathmoveto{\pgfqpoint{2.820706in}{3.496405in}}%
\pgfpathlineto{\pgfqpoint{2.820706in}{3.496405in}}%
\pgfpathlineto{\pgfqpoint{2.820706in}{3.508202in}}%
\pgfpathlineto{\pgfqpoint{2.838870in}{3.508202in}}%
\pgfpathlineto{\pgfqpoint{2.838870in}{3.496405in}}%
\pgfpathmoveto{\pgfqpoint{2.820706in}{3.508202in}}%
\pgfpathlineto{\pgfqpoint{2.820706in}{3.508202in}}%
\pgfpathlineto{\pgfqpoint{2.820706in}{3.519999in}}%
\pgfpathlineto{\pgfqpoint{2.838870in}{3.519999in}}%
\pgfpathlineto{\pgfqpoint{2.838870in}{3.508202in}}%
\pgfpathmoveto{\pgfqpoint{2.857034in}{3.449218in}}%
\pgfpathlineto{\pgfqpoint{2.857034in}{3.449218in}}%
\pgfpathlineto{\pgfqpoint{2.857034in}{3.461015in}}%
\pgfpathlineto{\pgfqpoint{2.875198in}{3.461015in}}%
\pgfpathlineto{\pgfqpoint{2.875198in}{3.449218in}}%
\pgfpathmoveto{\pgfqpoint{2.857034in}{3.461015in}}%
\pgfpathlineto{\pgfqpoint{2.857034in}{3.461015in}}%
\pgfpathlineto{\pgfqpoint{2.857034in}{3.472812in}}%
\pgfpathlineto{\pgfqpoint{2.875198in}{3.472812in}}%
\pgfpathlineto{\pgfqpoint{2.875198in}{3.461015in}}%
\pgfpathmoveto{\pgfqpoint{2.929691in}{1.939219in}}%
\pgfpathlineto{\pgfqpoint{2.929691in}{1.939219in}}%
\pgfpathlineto{\pgfqpoint{2.929691in}{1.951015in}}%
\pgfpathlineto{\pgfqpoint{2.947854in}{1.951015in}}%
\pgfpathlineto{\pgfqpoint{2.947854in}{1.939219in}}%
\pgfpathmoveto{\pgfqpoint{2.929691in}{1.951015in}}%
\pgfpathlineto{\pgfqpoint{2.929691in}{1.951015in}}%
\pgfpathlineto{\pgfqpoint{2.929691in}{1.962812in}}%
\pgfpathlineto{\pgfqpoint{2.947854in}{1.962812in}}%
\pgfpathlineto{\pgfqpoint{2.947854in}{1.951015in}}%
\pgfpathmoveto{\pgfqpoint{2.947854in}{1.951015in}}%
\pgfpathlineto{\pgfqpoint{2.947854in}{1.951015in}}%
\pgfpathlineto{\pgfqpoint{2.947854in}{1.962812in}}%
\pgfpathlineto{\pgfqpoint{2.966017in}{1.962812in}}%
\pgfpathlineto{\pgfqpoint{2.966017in}{1.951015in}}%
\pgfpathmoveto{\pgfqpoint{2.966017in}{1.962812in}}%
\pgfpathlineto{\pgfqpoint{2.966017in}{1.962812in}}%
\pgfpathlineto{\pgfqpoint{2.966017in}{1.974608in}}%
\pgfpathlineto{\pgfqpoint{2.984180in}{1.974608in}}%
\pgfpathlineto{\pgfqpoint{2.984180in}{1.962812in}}%
\pgfpathmoveto{\pgfqpoint{2.966017in}{1.974608in}}%
\pgfpathlineto{\pgfqpoint{2.966017in}{1.974608in}}%
\pgfpathlineto{\pgfqpoint{2.966017in}{1.986405in}}%
\pgfpathlineto{\pgfqpoint{2.984180in}{1.986405in}}%
\pgfpathlineto{\pgfqpoint{2.984180in}{1.974608in}}%
\pgfpathmoveto{\pgfqpoint{2.984180in}{1.974608in}}%
\pgfpathlineto{\pgfqpoint{2.984180in}{1.974608in}}%
\pgfpathlineto{\pgfqpoint{2.984180in}{1.986405in}}%
\pgfpathlineto{\pgfqpoint{3.002344in}{1.986405in}}%
\pgfpathlineto{\pgfqpoint{3.002344in}{1.974608in}}%
\pgfpathmoveto{\pgfqpoint{3.002344in}{1.986405in}}%
\pgfpathlineto{\pgfqpoint{3.002344in}{1.986405in}}%
\pgfpathlineto{\pgfqpoint{3.002344in}{1.998201in}}%
\pgfpathlineto{\pgfqpoint{3.020507in}{1.998201in}}%
\pgfpathlineto{\pgfqpoint{3.020507in}{1.986405in}}%
\pgfpathmoveto{\pgfqpoint{3.002344in}{1.998201in}}%
\pgfpathlineto{\pgfqpoint{3.002344in}{1.998201in}}%
\pgfpathlineto{\pgfqpoint{3.002344in}{2.009997in}}%
\pgfpathlineto{\pgfqpoint{3.020507in}{2.009997in}}%
\pgfpathlineto{\pgfqpoint{3.020507in}{1.998201in}}%
\pgfpathmoveto{\pgfqpoint{3.020507in}{1.998201in}}%
\pgfpathlineto{\pgfqpoint{3.020507in}{1.998201in}}%
\pgfpathlineto{\pgfqpoint{3.020507in}{2.009997in}}%
\pgfpathlineto{\pgfqpoint{3.038670in}{2.009997in}}%
\pgfpathlineto{\pgfqpoint{3.038670in}{1.998201in}}%
\pgfpathmoveto{\pgfqpoint{3.038670in}{2.009997in}}%
\pgfpathlineto{\pgfqpoint{3.038670in}{2.009997in}}%
\pgfpathlineto{\pgfqpoint{3.038670in}{2.021795in}}%
\pgfpathlineto{\pgfqpoint{3.056833in}{2.021795in}}%
\pgfpathlineto{\pgfqpoint{3.056833in}{2.009997in}}%
\pgfpathmoveto{\pgfqpoint{3.038670in}{2.021795in}}%
\pgfpathlineto{\pgfqpoint{3.038670in}{2.021795in}}%
\pgfpathlineto{\pgfqpoint{3.038670in}{2.033593in}}%
\pgfpathlineto{\pgfqpoint{3.056833in}{2.033593in}}%
\pgfpathlineto{\pgfqpoint{3.056833in}{2.021795in}}%
\pgfpathmoveto{\pgfqpoint{3.056833in}{2.021795in}}%
\pgfpathlineto{\pgfqpoint{3.056833in}{2.021795in}}%
\pgfpathlineto{\pgfqpoint{3.056833in}{2.033593in}}%
\pgfpathlineto{\pgfqpoint{3.074996in}{2.033593in}}%
\pgfpathlineto{\pgfqpoint{3.074996in}{2.021795in}}%
\pgfpathmoveto{\pgfqpoint{3.038670in}{3.213280in}}%
\pgfpathlineto{\pgfqpoint{3.038670in}{3.213280in}}%
\pgfpathlineto{\pgfqpoint{3.038670in}{3.225077in}}%
\pgfpathlineto{\pgfqpoint{3.056833in}{3.225077in}}%
\pgfpathlineto{\pgfqpoint{3.056833in}{3.213280in}}%
\pgfpathmoveto{\pgfqpoint{3.038670in}{3.225077in}}%
\pgfpathlineto{\pgfqpoint{3.038670in}{3.225077in}}%
\pgfpathlineto{\pgfqpoint{3.038670in}{3.236873in}}%
\pgfpathlineto{\pgfqpoint{3.056833in}{3.236873in}}%
\pgfpathlineto{\pgfqpoint{3.056833in}{3.225077in}}%
\pgfpathmoveto{\pgfqpoint{2.966017in}{3.307655in}}%
\pgfpathlineto{\pgfqpoint{2.966017in}{3.307655in}}%
\pgfpathlineto{\pgfqpoint{2.966017in}{3.319452in}}%
\pgfpathlineto{\pgfqpoint{2.984180in}{3.319452in}}%
\pgfpathlineto{\pgfqpoint{2.984180in}{3.307655in}}%
\pgfpathmoveto{\pgfqpoint{2.966017in}{3.319452in}}%
\pgfpathlineto{\pgfqpoint{2.966017in}{3.319452in}}%
\pgfpathlineto{\pgfqpoint{2.966017in}{3.331249in}}%
\pgfpathlineto{\pgfqpoint{2.984180in}{3.331249in}}%
\pgfpathlineto{\pgfqpoint{2.984180in}{3.319452in}}%
\pgfpathmoveto{\pgfqpoint{3.002344in}{3.260467in}}%
\pgfpathlineto{\pgfqpoint{3.002344in}{3.260467in}}%
\pgfpathlineto{\pgfqpoint{3.002344in}{3.272264in}}%
\pgfpathlineto{\pgfqpoint{3.020507in}{3.272264in}}%
\pgfpathlineto{\pgfqpoint{3.020507in}{3.260467in}}%
\pgfpathmoveto{\pgfqpoint{3.002344in}{3.272264in}}%
\pgfpathlineto{\pgfqpoint{3.002344in}{3.272264in}}%
\pgfpathlineto{\pgfqpoint{3.002344in}{3.284061in}}%
\pgfpathlineto{\pgfqpoint{3.020507in}{3.284061in}}%
\pgfpathlineto{\pgfqpoint{3.020507in}{3.272264in}}%
\pgfpathmoveto{\pgfqpoint{2.929691in}{3.354843in}}%
\pgfpathlineto{\pgfqpoint{2.929691in}{3.354843in}}%
\pgfpathlineto{\pgfqpoint{2.929691in}{3.366640in}}%
\pgfpathlineto{\pgfqpoint{2.947854in}{3.366640in}}%
\pgfpathlineto{\pgfqpoint{2.947854in}{3.354843in}}%
\pgfpathmoveto{\pgfqpoint{2.929691in}{3.366640in}}%
\pgfpathlineto{\pgfqpoint{2.929691in}{3.366640in}}%
\pgfpathlineto{\pgfqpoint{2.929691in}{3.378437in}}%
\pgfpathlineto{\pgfqpoint{2.947854in}{3.378437in}}%
\pgfpathlineto{\pgfqpoint{2.947854in}{3.366640in}}%
\pgfpathmoveto{\pgfqpoint{3.111326in}{2.045390in}}%
\pgfpathlineto{\pgfqpoint{3.111326in}{2.045390in}}%
\pgfpathlineto{\pgfqpoint{3.111326in}{2.057188in}}%
\pgfpathlineto{\pgfqpoint{3.129491in}{2.057188in}}%
\pgfpathlineto{\pgfqpoint{3.129491in}{2.045390in}}%
\pgfpathmoveto{\pgfqpoint{3.147656in}{2.068985in}}%
\pgfpathlineto{\pgfqpoint{3.147656in}{2.068985in}}%
\pgfpathlineto{\pgfqpoint{3.147656in}{2.080783in}}%
\pgfpathlineto{\pgfqpoint{3.165821in}{2.080783in}}%
\pgfpathlineto{\pgfqpoint{3.165821in}{2.068985in}}%
\pgfpathmoveto{\pgfqpoint{3.183986in}{2.092580in}}%
\pgfpathlineto{\pgfqpoint{3.183986in}{2.092580in}}%
\pgfpathlineto{\pgfqpoint{3.183986in}{2.104378in}}%
\pgfpathlineto{\pgfqpoint{3.202151in}{2.104378in}}%
\pgfpathlineto{\pgfqpoint{3.202151in}{2.092580in}}%
\pgfpathmoveto{\pgfqpoint{3.183986in}{3.024531in}}%
\pgfpathlineto{\pgfqpoint{3.183986in}{3.024531in}}%
\pgfpathlineto{\pgfqpoint{3.183986in}{3.036329in}}%
\pgfpathlineto{\pgfqpoint{3.202151in}{3.036329in}}%
\pgfpathlineto{\pgfqpoint{3.202151in}{3.024531in}}%
\pgfpathmoveto{\pgfqpoint{3.183986in}{3.036329in}}%
\pgfpathlineto{\pgfqpoint{3.183986in}{3.036329in}}%
\pgfpathlineto{\pgfqpoint{3.183986in}{3.048126in}}%
\pgfpathlineto{\pgfqpoint{3.202151in}{3.048126in}}%
\pgfpathlineto{\pgfqpoint{3.202151in}{3.036329in}}%
\pgfpathmoveto{\pgfqpoint{3.111326in}{3.118907in}}%
\pgfpathlineto{\pgfqpoint{3.111326in}{3.118907in}}%
\pgfpathlineto{\pgfqpoint{3.111326in}{3.130704in}}%
\pgfpathlineto{\pgfqpoint{3.129491in}{3.130704in}}%
\pgfpathlineto{\pgfqpoint{3.129491in}{3.118907in}}%
\pgfpathmoveto{\pgfqpoint{3.111326in}{3.130704in}}%
\pgfpathlineto{\pgfqpoint{3.111326in}{3.130704in}}%
\pgfpathlineto{\pgfqpoint{3.111326in}{3.142501in}}%
\pgfpathlineto{\pgfqpoint{3.129491in}{3.142501in}}%
\pgfpathlineto{\pgfqpoint{3.129491in}{3.130704in}}%
\pgfpathmoveto{\pgfqpoint{3.147656in}{3.071720in}}%
\pgfpathlineto{\pgfqpoint{3.147656in}{3.071720in}}%
\pgfpathlineto{\pgfqpoint{3.147656in}{3.083517in}}%
\pgfpathlineto{\pgfqpoint{3.165821in}{3.083517in}}%
\pgfpathlineto{\pgfqpoint{3.165821in}{3.071720in}}%
\pgfpathmoveto{\pgfqpoint{3.147656in}{3.083517in}}%
\pgfpathlineto{\pgfqpoint{3.147656in}{3.083517in}}%
\pgfpathlineto{\pgfqpoint{3.147656in}{3.095314in}}%
\pgfpathlineto{\pgfqpoint{3.165821in}{3.095314in}}%
\pgfpathlineto{\pgfqpoint{3.165821in}{3.083517in}}%
\pgfpathmoveto{\pgfqpoint{3.074996in}{3.166094in}}%
\pgfpathlineto{\pgfqpoint{3.074996in}{3.166094in}}%
\pgfpathlineto{\pgfqpoint{3.074996in}{3.177891in}}%
\pgfpathlineto{\pgfqpoint{3.093161in}{3.177891in}}%
\pgfpathlineto{\pgfqpoint{3.093161in}{3.166094in}}%
\pgfpathmoveto{\pgfqpoint{3.074996in}{3.177891in}}%
\pgfpathlineto{\pgfqpoint{3.074996in}{3.177891in}}%
\pgfpathlineto{\pgfqpoint{3.074996in}{3.189687in}}%
\pgfpathlineto{\pgfqpoint{3.093161in}{3.189687in}}%
\pgfpathlineto{\pgfqpoint{3.093161in}{3.177891in}}%
\pgfpathmoveto{\pgfqpoint{3.220316in}{2.116175in}}%
\pgfpathlineto{\pgfqpoint{3.220316in}{2.116175in}}%
\pgfpathlineto{\pgfqpoint{3.220316in}{2.127971in}}%
\pgfpathlineto{\pgfqpoint{3.238480in}{2.127971in}}%
\pgfpathlineto{\pgfqpoint{3.238480in}{2.116175in}}%
\pgfpathmoveto{\pgfqpoint{3.256643in}{2.139768in}}%
\pgfpathlineto{\pgfqpoint{3.256643in}{2.139768in}}%
\pgfpathlineto{\pgfqpoint{3.256643in}{2.151565in}}%
\pgfpathlineto{\pgfqpoint{3.274806in}{2.151565in}}%
\pgfpathlineto{\pgfqpoint{3.274806in}{2.139768in}}%
\pgfpathmoveto{\pgfqpoint{3.292969in}{2.163362in}}%
\pgfpathlineto{\pgfqpoint{3.292969in}{2.163362in}}%
\pgfpathlineto{\pgfqpoint{3.292969in}{2.175158in}}%
\pgfpathlineto{\pgfqpoint{3.311132in}{2.175158in}}%
\pgfpathlineto{\pgfqpoint{3.311132in}{2.163362in}}%
\pgfpathmoveto{\pgfqpoint{3.329296in}{2.186955in}}%
\pgfpathlineto{\pgfqpoint{3.329296in}{2.186955in}}%
\pgfpathlineto{\pgfqpoint{3.329296in}{2.198752in}}%
\pgfpathlineto{\pgfqpoint{3.347459in}{2.198752in}}%
\pgfpathlineto{\pgfqpoint{3.347459in}{2.186955in}}%
\pgfpathmoveto{\pgfqpoint{3.329296in}{2.835782in}}%
\pgfpathlineto{\pgfqpoint{3.329296in}{2.835782in}}%
\pgfpathlineto{\pgfqpoint{3.329296in}{2.847579in}}%
\pgfpathlineto{\pgfqpoint{3.347459in}{2.847579in}}%
\pgfpathlineto{\pgfqpoint{3.347459in}{2.835782in}}%
\pgfpathmoveto{\pgfqpoint{3.329296in}{2.847579in}}%
\pgfpathlineto{\pgfqpoint{3.329296in}{2.847579in}}%
\pgfpathlineto{\pgfqpoint{3.329296in}{2.859376in}}%
\pgfpathlineto{\pgfqpoint{3.347459in}{2.859376in}}%
\pgfpathlineto{\pgfqpoint{3.347459in}{2.847579in}}%
\pgfpathmoveto{\pgfqpoint{3.256643in}{2.930155in}}%
\pgfpathlineto{\pgfqpoint{3.256643in}{2.930155in}}%
\pgfpathlineto{\pgfqpoint{3.256643in}{2.941952in}}%
\pgfpathlineto{\pgfqpoint{3.274806in}{2.941952in}}%
\pgfpathlineto{\pgfqpoint{3.274806in}{2.930155in}}%
\pgfpathmoveto{\pgfqpoint{3.256643in}{2.941952in}}%
\pgfpathlineto{\pgfqpoint{3.256643in}{2.941952in}}%
\pgfpathlineto{\pgfqpoint{3.256643in}{2.953748in}}%
\pgfpathlineto{\pgfqpoint{3.274806in}{2.953748in}}%
\pgfpathlineto{\pgfqpoint{3.274806in}{2.941952in}}%
\pgfpathmoveto{\pgfqpoint{3.292969in}{2.882969in}}%
\pgfpathlineto{\pgfqpoint{3.292969in}{2.882969in}}%
\pgfpathlineto{\pgfqpoint{3.292969in}{2.894765in}}%
\pgfpathlineto{\pgfqpoint{3.311132in}{2.894765in}}%
\pgfpathlineto{\pgfqpoint{3.311132in}{2.882969in}}%
\pgfpathmoveto{\pgfqpoint{3.292969in}{2.894765in}}%
\pgfpathlineto{\pgfqpoint{3.292969in}{2.894765in}}%
\pgfpathlineto{\pgfqpoint{3.292969in}{2.906562in}}%
\pgfpathlineto{\pgfqpoint{3.311132in}{2.906562in}}%
\pgfpathlineto{\pgfqpoint{3.311132in}{2.894765in}}%
\pgfpathmoveto{\pgfqpoint{3.220316in}{2.977343in}}%
\pgfpathlineto{\pgfqpoint{3.220316in}{2.977343in}}%
\pgfpathlineto{\pgfqpoint{3.220316in}{2.989140in}}%
\pgfpathlineto{\pgfqpoint{3.238480in}{2.989140in}}%
\pgfpathlineto{\pgfqpoint{3.238480in}{2.977343in}}%
\pgfpathmoveto{\pgfqpoint{3.220316in}{2.989140in}}%
\pgfpathlineto{\pgfqpoint{3.220316in}{2.989140in}}%
\pgfpathlineto{\pgfqpoint{3.220316in}{3.000937in}}%
\pgfpathlineto{\pgfqpoint{3.238480in}{3.000937in}}%
\pgfpathlineto{\pgfqpoint{3.238480in}{2.989140in}}%
\pgfpathmoveto{\pgfqpoint{3.365622in}{2.210548in}}%
\pgfpathlineto{\pgfqpoint{3.365622in}{2.210548in}}%
\pgfpathlineto{\pgfqpoint{3.365622in}{2.222345in}}%
\pgfpathlineto{\pgfqpoint{3.383787in}{2.222345in}}%
\pgfpathlineto{\pgfqpoint{3.383787in}{2.210548in}}%
\pgfpathmoveto{\pgfqpoint{3.401952in}{2.234141in}}%
\pgfpathlineto{\pgfqpoint{3.401952in}{2.234141in}}%
\pgfpathlineto{\pgfqpoint{3.401952in}{2.245937in}}%
\pgfpathlineto{\pgfqpoint{3.420116in}{2.245937in}}%
\pgfpathlineto{\pgfqpoint{3.420116in}{2.234141in}}%
\pgfpathmoveto{\pgfqpoint{3.401952in}{2.245937in}}%
\pgfpathlineto{\pgfqpoint{3.401952in}{2.245937in}}%
\pgfpathlineto{\pgfqpoint{3.401952in}{2.257734in}}%
\pgfpathlineto{\pgfqpoint{3.420116in}{2.257734in}}%
\pgfpathlineto{\pgfqpoint{3.420116in}{2.245937in}}%
\pgfpathmoveto{\pgfqpoint{3.401952in}{2.257734in}}%
\pgfpathlineto{\pgfqpoint{3.401952in}{2.257734in}}%
\pgfpathlineto{\pgfqpoint{3.401952in}{2.269530in}}%
\pgfpathlineto{\pgfqpoint{3.420116in}{2.269530in}}%
\pgfpathlineto{\pgfqpoint{3.420116in}{2.257734in}}%
\pgfpathmoveto{\pgfqpoint{3.420116in}{2.257734in}}%
\pgfpathlineto{\pgfqpoint{3.420116in}{2.257734in}}%
\pgfpathlineto{\pgfqpoint{3.420116in}{2.269530in}}%
\pgfpathlineto{\pgfqpoint{3.438281in}{2.269530in}}%
\pgfpathlineto{\pgfqpoint{3.438281in}{2.257734in}}%
\pgfpathmoveto{\pgfqpoint{3.438281in}{2.269530in}}%
\pgfpathlineto{\pgfqpoint{3.438281in}{2.269530in}}%
\pgfpathlineto{\pgfqpoint{3.438281in}{2.281326in}}%
\pgfpathlineto{\pgfqpoint{3.456446in}{2.281326in}}%
\pgfpathlineto{\pgfqpoint{3.456446in}{2.269530in}}%
\pgfpathmoveto{\pgfqpoint{3.438281in}{2.281326in}}%
\pgfpathlineto{\pgfqpoint{3.438281in}{2.281326in}}%
\pgfpathlineto{\pgfqpoint{3.438281in}{2.293123in}}%
\pgfpathlineto{\pgfqpoint{3.456446in}{2.293123in}}%
\pgfpathlineto{\pgfqpoint{3.456446in}{2.281326in}}%
\pgfpathmoveto{\pgfqpoint{3.456446in}{2.281326in}}%
\pgfpathlineto{\pgfqpoint{3.456446in}{2.281326in}}%
\pgfpathlineto{\pgfqpoint{3.456446in}{2.293123in}}%
\pgfpathlineto{\pgfqpoint{3.474611in}{2.293123in}}%
\pgfpathlineto{\pgfqpoint{3.474611in}{2.281326in}}%
\pgfpathmoveto{\pgfqpoint{3.474611in}{2.293123in}}%
\pgfpathlineto{\pgfqpoint{3.474611in}{2.293123in}}%
\pgfpathlineto{\pgfqpoint{3.474611in}{2.304919in}}%
\pgfpathlineto{\pgfqpoint{3.492776in}{2.304919in}}%
\pgfpathlineto{\pgfqpoint{3.492776in}{2.293123in}}%
\pgfpathmoveto{\pgfqpoint{3.474611in}{2.304919in}}%
\pgfpathlineto{\pgfqpoint{3.474611in}{2.304919in}}%
\pgfpathlineto{\pgfqpoint{3.474611in}{2.316716in}}%
\pgfpathlineto{\pgfqpoint{3.492776in}{2.316716in}}%
\pgfpathlineto{\pgfqpoint{3.492776in}{2.304919in}}%
\pgfpathmoveto{\pgfqpoint{3.492776in}{2.304919in}}%
\pgfpathlineto{\pgfqpoint{3.492776in}{2.304919in}}%
\pgfpathlineto{\pgfqpoint{3.492776in}{2.316716in}}%
\pgfpathlineto{\pgfqpoint{3.510941in}{2.316716in}}%
\pgfpathlineto{\pgfqpoint{3.510941in}{2.304919in}}%
\pgfpathmoveto{\pgfqpoint{3.474611in}{2.647031in}}%
\pgfpathlineto{\pgfqpoint{3.474611in}{2.647031in}}%
\pgfpathlineto{\pgfqpoint{3.474611in}{2.658828in}}%
\pgfpathlineto{\pgfqpoint{3.492776in}{2.658828in}}%
\pgfpathlineto{\pgfqpoint{3.492776in}{2.647031in}}%
\pgfpathmoveto{\pgfqpoint{3.474611in}{2.658828in}}%
\pgfpathlineto{\pgfqpoint{3.474611in}{2.658828in}}%
\pgfpathlineto{\pgfqpoint{3.474611in}{2.670624in}}%
\pgfpathlineto{\pgfqpoint{3.492776in}{2.670624in}}%
\pgfpathlineto{\pgfqpoint{3.492776in}{2.658828in}}%
\pgfpathmoveto{\pgfqpoint{3.401952in}{2.741408in}}%
\pgfpathlineto{\pgfqpoint{3.401952in}{2.741408in}}%
\pgfpathlineto{\pgfqpoint{3.401952in}{2.753205in}}%
\pgfpathlineto{\pgfqpoint{3.420116in}{2.753205in}}%
\pgfpathlineto{\pgfqpoint{3.420116in}{2.741408in}}%
\pgfpathmoveto{\pgfqpoint{3.401952in}{2.753205in}}%
\pgfpathlineto{\pgfqpoint{3.401952in}{2.753205in}}%
\pgfpathlineto{\pgfqpoint{3.401952in}{2.765002in}}%
\pgfpathlineto{\pgfqpoint{3.420116in}{2.765002in}}%
\pgfpathlineto{\pgfqpoint{3.420116in}{2.753205in}}%
\pgfpathmoveto{\pgfqpoint{3.438281in}{2.694219in}}%
\pgfpathlineto{\pgfqpoint{3.438281in}{2.694219in}}%
\pgfpathlineto{\pgfqpoint{3.438281in}{2.706016in}}%
\pgfpathlineto{\pgfqpoint{3.456446in}{2.706016in}}%
\pgfpathlineto{\pgfqpoint{3.456446in}{2.694219in}}%
\pgfpathmoveto{\pgfqpoint{3.438281in}{2.706016in}}%
\pgfpathlineto{\pgfqpoint{3.438281in}{2.706016in}}%
\pgfpathlineto{\pgfqpoint{3.438281in}{2.717813in}}%
\pgfpathlineto{\pgfqpoint{3.456446in}{2.717813in}}%
\pgfpathlineto{\pgfqpoint{3.456446in}{2.706016in}}%
\pgfpathmoveto{\pgfqpoint{3.365622in}{2.788596in}}%
\pgfpathlineto{\pgfqpoint{3.365622in}{2.788596in}}%
\pgfpathlineto{\pgfqpoint{3.365622in}{2.800392in}}%
\pgfpathlineto{\pgfqpoint{3.383787in}{2.800392in}}%
\pgfpathlineto{\pgfqpoint{3.383787in}{2.788596in}}%
\pgfpathmoveto{\pgfqpoint{3.365622in}{2.800392in}}%
\pgfpathlineto{\pgfqpoint{3.365622in}{2.800392in}}%
\pgfpathlineto{\pgfqpoint{3.365622in}{2.812189in}}%
\pgfpathlineto{\pgfqpoint{3.383787in}{2.812189in}}%
\pgfpathlineto{\pgfqpoint{3.383787in}{2.800392in}}%
\pgfpathmoveto{\pgfqpoint{3.510941in}{2.316716in}}%
\pgfpathlineto{\pgfqpoint{3.510941in}{2.316716in}}%
\pgfpathlineto{\pgfqpoint{3.510941in}{2.328513in}}%
\pgfpathlineto{\pgfqpoint{3.529104in}{2.328513in}}%
\pgfpathlineto{\pgfqpoint{3.529104in}{2.316716in}}%
\pgfpathmoveto{\pgfqpoint{3.510941in}{2.328513in}}%
\pgfpathlineto{\pgfqpoint{3.510941in}{2.328513in}}%
\pgfpathlineto{\pgfqpoint{3.510941in}{2.340310in}}%
\pgfpathlineto{\pgfqpoint{3.529104in}{2.340310in}}%
\pgfpathlineto{\pgfqpoint{3.529104in}{2.328513in}}%
\pgfpathmoveto{\pgfqpoint{3.529104in}{2.328513in}}%
\pgfpathlineto{\pgfqpoint{3.529104in}{2.328513in}}%
\pgfpathlineto{\pgfqpoint{3.529104in}{2.340310in}}%
\pgfpathlineto{\pgfqpoint{3.547267in}{2.340310in}}%
\pgfpathlineto{\pgfqpoint{3.547267in}{2.328513in}}%
\pgfpathmoveto{\pgfqpoint{3.547267in}{2.340310in}}%
\pgfpathlineto{\pgfqpoint{3.547267in}{2.340310in}}%
\pgfpathlineto{\pgfqpoint{3.547267in}{2.352107in}}%
\pgfpathlineto{\pgfqpoint{3.565430in}{2.352107in}}%
\pgfpathlineto{\pgfqpoint{3.565430in}{2.340310in}}%
\pgfpathmoveto{\pgfqpoint{3.547267in}{2.352107in}}%
\pgfpathlineto{\pgfqpoint{3.547267in}{2.352107in}}%
\pgfpathlineto{\pgfqpoint{3.547267in}{2.363904in}}%
\pgfpathlineto{\pgfqpoint{3.565430in}{2.363904in}}%
\pgfpathlineto{\pgfqpoint{3.565430in}{2.352107in}}%
\pgfpathmoveto{\pgfqpoint{3.565430in}{2.352107in}}%
\pgfpathlineto{\pgfqpoint{3.565430in}{2.352107in}}%
\pgfpathlineto{\pgfqpoint{3.565430in}{2.363904in}}%
\pgfpathlineto{\pgfqpoint{3.583593in}{2.363904in}}%
\pgfpathlineto{\pgfqpoint{3.583593in}{2.352107in}}%
\pgfpathmoveto{\pgfqpoint{3.583593in}{2.363904in}}%
\pgfpathlineto{\pgfqpoint{3.583593in}{2.363904in}}%
\pgfpathlineto{\pgfqpoint{3.583593in}{2.375701in}}%
\pgfpathlineto{\pgfqpoint{3.601757in}{2.375701in}}%
\pgfpathlineto{\pgfqpoint{3.601757in}{2.363904in}}%
\pgfpathmoveto{\pgfqpoint{3.583593in}{2.375701in}}%
\pgfpathlineto{\pgfqpoint{3.583593in}{2.375701in}}%
\pgfpathlineto{\pgfqpoint{3.583593in}{2.387497in}}%
\pgfpathlineto{\pgfqpoint{3.601757in}{2.387497in}}%
\pgfpathlineto{\pgfqpoint{3.601757in}{2.375701in}}%
\pgfpathmoveto{\pgfqpoint{3.601757in}{2.375701in}}%
\pgfpathlineto{\pgfqpoint{3.601757in}{2.375701in}}%
\pgfpathlineto{\pgfqpoint{3.601757in}{2.387497in}}%
\pgfpathlineto{\pgfqpoint{3.619920in}{2.387497in}}%
\pgfpathlineto{\pgfqpoint{3.619920in}{2.375701in}}%
\pgfpathmoveto{\pgfqpoint{3.619920in}{2.387497in}}%
\pgfpathlineto{\pgfqpoint{3.619920in}{2.387497in}}%
\pgfpathlineto{\pgfqpoint{3.619920in}{2.399294in}}%
\pgfpathlineto{\pgfqpoint{3.638083in}{2.399294in}}%
\pgfpathlineto{\pgfqpoint{3.638083in}{2.387497in}}%
\pgfpathmoveto{\pgfqpoint{3.619920in}{2.399294in}}%
\pgfpathlineto{\pgfqpoint{3.619920in}{2.399294in}}%
\pgfpathlineto{\pgfqpoint{3.619920in}{2.411091in}}%
\pgfpathlineto{\pgfqpoint{3.638083in}{2.411091in}}%
\pgfpathlineto{\pgfqpoint{3.638083in}{2.399294in}}%
\pgfpathmoveto{\pgfqpoint{3.638083in}{2.399294in}}%
\pgfpathlineto{\pgfqpoint{3.638083in}{2.399294in}}%
\pgfpathlineto{\pgfqpoint{3.638083in}{2.411091in}}%
\pgfpathlineto{\pgfqpoint{3.656246in}{2.411091in}}%
\pgfpathlineto{\pgfqpoint{3.656246in}{2.399294in}}%
\pgfpathmoveto{\pgfqpoint{3.619920in}{2.458279in}}%
\pgfpathlineto{\pgfqpoint{3.619920in}{2.458279in}}%
\pgfpathlineto{\pgfqpoint{3.619920in}{2.470076in}}%
\pgfpathlineto{\pgfqpoint{3.638083in}{2.470076in}}%
\pgfpathlineto{\pgfqpoint{3.638083in}{2.458279in}}%
\pgfpathmoveto{\pgfqpoint{3.619920in}{2.470076in}}%
\pgfpathlineto{\pgfqpoint{3.619920in}{2.470076in}}%
\pgfpathlineto{\pgfqpoint{3.619920in}{2.481873in}}%
\pgfpathlineto{\pgfqpoint{3.638083in}{2.481873in}}%
\pgfpathlineto{\pgfqpoint{3.638083in}{2.470076in}}%
\pgfpathmoveto{\pgfqpoint{3.547267in}{2.552657in}}%
\pgfpathlineto{\pgfqpoint{3.547267in}{2.552657in}}%
\pgfpathlineto{\pgfqpoint{3.547267in}{2.564454in}}%
\pgfpathlineto{\pgfqpoint{3.565430in}{2.564454in}}%
\pgfpathlineto{\pgfqpoint{3.565430in}{2.552657in}}%
\pgfpathmoveto{\pgfqpoint{3.547267in}{2.564454in}}%
\pgfpathlineto{\pgfqpoint{3.547267in}{2.564454in}}%
\pgfpathlineto{\pgfqpoint{3.547267in}{2.576252in}}%
\pgfpathlineto{\pgfqpoint{3.565430in}{2.576252in}}%
\pgfpathlineto{\pgfqpoint{3.565430in}{2.564454in}}%
\pgfpathmoveto{\pgfqpoint{3.583593in}{2.505467in}}%
\pgfpathlineto{\pgfqpoint{3.583593in}{2.505467in}}%
\pgfpathlineto{\pgfqpoint{3.583593in}{2.517265in}}%
\pgfpathlineto{\pgfqpoint{3.601757in}{2.517265in}}%
\pgfpathlineto{\pgfqpoint{3.601757in}{2.505467in}}%
\pgfpathmoveto{\pgfqpoint{3.583593in}{2.517265in}}%
\pgfpathlineto{\pgfqpoint{3.583593in}{2.517265in}}%
\pgfpathlineto{\pgfqpoint{3.583593in}{2.529062in}}%
\pgfpathlineto{\pgfqpoint{3.601757in}{2.529062in}}%
\pgfpathlineto{\pgfqpoint{3.601757in}{2.517265in}}%
\pgfpathmoveto{\pgfqpoint{3.510941in}{2.599845in}}%
\pgfpathlineto{\pgfqpoint{3.510941in}{2.599845in}}%
\pgfpathlineto{\pgfqpoint{3.510941in}{2.611642in}}%
\pgfpathlineto{\pgfqpoint{3.529104in}{2.611642in}}%
\pgfpathlineto{\pgfqpoint{3.529104in}{2.599845in}}%
\pgfpathmoveto{\pgfqpoint{3.510941in}{2.611642in}}%
\pgfpathlineto{\pgfqpoint{3.510941in}{2.611642in}}%
\pgfpathlineto{\pgfqpoint{3.510941in}{2.623438in}}%
\pgfpathlineto{\pgfqpoint{3.529104in}{2.623438in}}%
\pgfpathlineto{\pgfqpoint{3.529104in}{2.611642in}}%
\pgfpathmoveto{\pgfqpoint{3.656246in}{2.411091in}}%
\pgfpathlineto{\pgfqpoint{3.656246in}{2.411091in}}%
\pgfpathlineto{\pgfqpoint{3.656246in}{2.422888in}}%
\pgfpathlineto{\pgfqpoint{3.674411in}{2.422888in}}%
\pgfpathlineto{\pgfqpoint{3.674411in}{2.411091in}}%
\pgfpathmoveto{\pgfqpoint{3.656246in}{2.422888in}}%
\pgfpathlineto{\pgfqpoint{3.656246in}{2.422888in}}%
\pgfpathlineto{\pgfqpoint{3.656246in}{2.434685in}}%
\pgfpathlineto{\pgfqpoint{3.674411in}{2.434685in}}%
\pgfpathlineto{\pgfqpoint{3.674411in}{2.422888in}}%
\pgfpathmoveto{\pgfqpoint{0.750004in}{0.511796in}}%
\pgfpathlineto{\pgfqpoint{0.750004in}{0.511796in}}%
\pgfpathlineto{\pgfqpoint{0.750004in}{0.517695in}}%
\pgfpathlineto{\pgfqpoint{0.759086in}{0.517695in}}%
\pgfpathlineto{\pgfqpoint{0.759086in}{0.511796in}}%
\pgfpathmoveto{\pgfqpoint{0.750004in}{0.517695in}}%
\pgfpathlineto{\pgfqpoint{0.750004in}{0.517695in}}%
\pgfpathlineto{\pgfqpoint{0.750004in}{0.523593in}}%
\pgfpathlineto{\pgfqpoint{0.759086in}{0.523593in}}%
\pgfpathlineto{\pgfqpoint{0.759086in}{0.517695in}}%
\pgfpathmoveto{\pgfqpoint{0.759086in}{0.517695in}}%
\pgfpathlineto{\pgfqpoint{0.759086in}{0.517695in}}%
\pgfpathlineto{\pgfqpoint{0.759086in}{0.523593in}}%
\pgfpathlineto{\pgfqpoint{0.768168in}{0.523593in}}%
\pgfpathlineto{\pgfqpoint{0.768168in}{0.517695in}}%
\pgfpathmoveto{\pgfqpoint{0.768168in}{0.523593in}}%
\pgfpathlineto{\pgfqpoint{0.768168in}{0.523593in}}%
\pgfpathlineto{\pgfqpoint{0.768168in}{0.529492in}}%
\pgfpathlineto{\pgfqpoint{0.777250in}{0.529492in}}%
\pgfpathlineto{\pgfqpoint{0.777250in}{0.523593in}}%
\pgfpathmoveto{\pgfqpoint{0.768168in}{0.529492in}}%
\pgfpathlineto{\pgfqpoint{0.768168in}{0.529492in}}%
\pgfpathlineto{\pgfqpoint{0.768168in}{0.535390in}}%
\pgfpathlineto{\pgfqpoint{0.777250in}{0.535390in}}%
\pgfpathlineto{\pgfqpoint{0.777250in}{0.529492in}}%
\pgfpathmoveto{\pgfqpoint{0.777250in}{0.529492in}}%
\pgfpathlineto{\pgfqpoint{0.777250in}{0.529492in}}%
\pgfpathlineto{\pgfqpoint{0.777250in}{0.535390in}}%
\pgfpathlineto{\pgfqpoint{0.786332in}{0.535390in}}%
\pgfpathlineto{\pgfqpoint{0.786332in}{0.529492in}}%
\pgfpathmoveto{\pgfqpoint{0.786332in}{0.535390in}}%
\pgfpathlineto{\pgfqpoint{0.786332in}{0.535390in}}%
\pgfpathlineto{\pgfqpoint{0.786332in}{0.541289in}}%
\pgfpathlineto{\pgfqpoint{0.795414in}{0.541289in}}%
\pgfpathlineto{\pgfqpoint{0.795414in}{0.535390in}}%
\pgfpathmoveto{\pgfqpoint{0.786332in}{0.541289in}}%
\pgfpathlineto{\pgfqpoint{0.786332in}{0.541289in}}%
\pgfpathlineto{\pgfqpoint{0.786332in}{0.547188in}}%
\pgfpathlineto{\pgfqpoint{0.795414in}{0.547188in}}%
\pgfpathlineto{\pgfqpoint{0.795414in}{0.541289in}}%
\pgfpathmoveto{\pgfqpoint{0.795414in}{0.541289in}}%
\pgfpathlineto{\pgfqpoint{0.795414in}{0.541289in}}%
\pgfpathlineto{\pgfqpoint{0.795414in}{0.547188in}}%
\pgfpathlineto{\pgfqpoint{0.804495in}{0.547188in}}%
\pgfpathlineto{\pgfqpoint{0.804495in}{0.541289in}}%
\pgfpathmoveto{\pgfqpoint{0.804495in}{0.547188in}}%
\pgfpathlineto{\pgfqpoint{0.804495in}{0.547188in}}%
\pgfpathlineto{\pgfqpoint{0.804495in}{0.553086in}}%
\pgfpathlineto{\pgfqpoint{0.813577in}{0.553086in}}%
\pgfpathlineto{\pgfqpoint{0.813577in}{0.547188in}}%
\pgfpathmoveto{\pgfqpoint{0.804495in}{0.553086in}}%
\pgfpathlineto{\pgfqpoint{0.804495in}{0.553086in}}%
\pgfpathlineto{\pgfqpoint{0.804495in}{0.558985in}}%
\pgfpathlineto{\pgfqpoint{0.813577in}{0.558985in}}%
\pgfpathlineto{\pgfqpoint{0.813577in}{0.553086in}}%
\pgfpathmoveto{\pgfqpoint{0.813577in}{0.553086in}}%
\pgfpathlineto{\pgfqpoint{0.813577in}{0.553086in}}%
\pgfpathlineto{\pgfqpoint{0.813577in}{0.558985in}}%
\pgfpathlineto{\pgfqpoint{0.822659in}{0.558985in}}%
\pgfpathlineto{\pgfqpoint{0.822659in}{0.553086in}}%
\pgfpathmoveto{\pgfqpoint{0.822659in}{0.558985in}}%
\pgfpathlineto{\pgfqpoint{0.822659in}{0.558985in}}%
\pgfpathlineto{\pgfqpoint{0.822659in}{0.564883in}}%
\pgfpathlineto{\pgfqpoint{0.831741in}{0.564883in}}%
\pgfpathlineto{\pgfqpoint{0.831741in}{0.558985in}}%
\pgfpathmoveto{\pgfqpoint{0.822659in}{0.564883in}}%
\pgfpathlineto{\pgfqpoint{0.822659in}{0.564883in}}%
\pgfpathlineto{\pgfqpoint{0.822659in}{0.570782in}}%
\pgfpathlineto{\pgfqpoint{0.831741in}{0.570782in}}%
\pgfpathlineto{\pgfqpoint{0.831741in}{0.564883in}}%
\pgfpathmoveto{\pgfqpoint{0.831741in}{0.564883in}}%
\pgfpathlineto{\pgfqpoint{0.831741in}{0.564883in}}%
\pgfpathlineto{\pgfqpoint{0.831741in}{0.570782in}}%
\pgfpathlineto{\pgfqpoint{0.840823in}{0.570782in}}%
\pgfpathlineto{\pgfqpoint{0.840823in}{0.564883in}}%
\pgfpathmoveto{\pgfqpoint{0.840823in}{0.570782in}}%
\pgfpathlineto{\pgfqpoint{0.840823in}{0.570782in}}%
\pgfpathlineto{\pgfqpoint{0.840823in}{0.576680in}}%
\pgfpathlineto{\pgfqpoint{0.849905in}{0.576680in}}%
\pgfpathlineto{\pgfqpoint{0.849905in}{0.570782in}}%
\pgfpathmoveto{\pgfqpoint{0.840823in}{0.576680in}}%
\pgfpathlineto{\pgfqpoint{0.840823in}{0.576680in}}%
\pgfpathlineto{\pgfqpoint{0.840823in}{0.582579in}}%
\pgfpathlineto{\pgfqpoint{0.849905in}{0.582579in}}%
\pgfpathlineto{\pgfqpoint{0.849905in}{0.576680in}}%
\pgfpathmoveto{\pgfqpoint{0.849905in}{0.576680in}}%
\pgfpathlineto{\pgfqpoint{0.849905in}{0.576680in}}%
\pgfpathlineto{\pgfqpoint{0.849905in}{0.582579in}}%
\pgfpathlineto{\pgfqpoint{0.858987in}{0.582579in}}%
\pgfpathlineto{\pgfqpoint{0.858987in}{0.576680in}}%
\pgfpathmoveto{\pgfqpoint{0.858987in}{0.582579in}}%
\pgfpathlineto{\pgfqpoint{0.858987in}{0.582579in}}%
\pgfpathlineto{\pgfqpoint{0.858987in}{0.588478in}}%
\pgfpathlineto{\pgfqpoint{0.868069in}{0.588478in}}%
\pgfpathlineto{\pgfqpoint{0.868069in}{0.582579in}}%
\pgfpathmoveto{\pgfqpoint{0.858987in}{0.588478in}}%
\pgfpathlineto{\pgfqpoint{0.858987in}{0.588478in}}%
\pgfpathlineto{\pgfqpoint{0.858987in}{0.594376in}}%
\pgfpathlineto{\pgfqpoint{0.868069in}{0.594376in}}%
\pgfpathlineto{\pgfqpoint{0.868069in}{0.588478in}}%
\pgfpathmoveto{\pgfqpoint{0.868069in}{0.588478in}}%
\pgfpathlineto{\pgfqpoint{0.868069in}{0.588478in}}%
\pgfpathlineto{\pgfqpoint{0.868069in}{0.594376in}}%
\pgfpathlineto{\pgfqpoint{0.877150in}{0.594376in}}%
\pgfpathlineto{\pgfqpoint{0.877150in}{0.588478in}}%
\pgfpathmoveto{\pgfqpoint{0.895314in}{0.600275in}}%
\pgfpathlineto{\pgfqpoint{0.895314in}{0.600275in}}%
\pgfpathlineto{\pgfqpoint{0.895314in}{0.606173in}}%
\pgfpathlineto{\pgfqpoint{0.904396in}{0.606173in}}%
\pgfpathlineto{\pgfqpoint{0.904396in}{0.600275in}}%
\pgfpathmoveto{\pgfqpoint{0.913478in}{0.612072in}}%
\pgfpathlineto{\pgfqpoint{0.913478in}{0.612072in}}%
\pgfpathlineto{\pgfqpoint{0.913478in}{0.617970in}}%
\pgfpathlineto{\pgfqpoint{0.922560in}{0.617970in}}%
\pgfpathlineto{\pgfqpoint{0.922560in}{0.612072in}}%
\pgfpathmoveto{\pgfqpoint{0.931641in}{0.623869in}}%
\pgfpathlineto{\pgfqpoint{0.931641in}{0.623869in}}%
\pgfpathlineto{\pgfqpoint{0.931641in}{0.629767in}}%
\pgfpathlineto{\pgfqpoint{0.940723in}{0.629767in}}%
\pgfpathlineto{\pgfqpoint{0.940723in}{0.623869in}}%
\pgfpathmoveto{\pgfqpoint{0.949805in}{0.635665in}}%
\pgfpathlineto{\pgfqpoint{0.949805in}{0.635665in}}%
\pgfpathlineto{\pgfqpoint{0.949805in}{0.641564in}}%
\pgfpathlineto{\pgfqpoint{0.958887in}{0.641564in}}%
\pgfpathlineto{\pgfqpoint{0.958887in}{0.635665in}}%
\pgfpathmoveto{\pgfqpoint{0.967968in}{0.647462in}}%
\pgfpathlineto{\pgfqpoint{0.967968in}{0.647462in}}%
\pgfpathlineto{\pgfqpoint{0.967968in}{0.653361in}}%
\pgfpathlineto{\pgfqpoint{0.977050in}{0.653361in}}%
\pgfpathlineto{\pgfqpoint{0.977050in}{0.647462in}}%
\pgfpathmoveto{\pgfqpoint{0.986132in}{0.659259in}}%
\pgfpathlineto{\pgfqpoint{0.986132in}{0.659259in}}%
\pgfpathlineto{\pgfqpoint{0.986132in}{0.665158in}}%
\pgfpathlineto{\pgfqpoint{0.995214in}{0.665158in}}%
\pgfpathlineto{\pgfqpoint{0.995214in}{0.659259in}}%
\pgfpathmoveto{\pgfqpoint{1.004295in}{0.671056in}}%
\pgfpathlineto{\pgfqpoint{1.004295in}{0.671056in}}%
\pgfpathlineto{\pgfqpoint{1.004295in}{0.676955in}}%
\pgfpathlineto{\pgfqpoint{1.013377in}{0.676955in}}%
\pgfpathlineto{\pgfqpoint{1.013377in}{0.671056in}}%
\pgfpathmoveto{\pgfqpoint{1.022459in}{0.682853in}}%
\pgfpathlineto{\pgfqpoint{1.022459in}{0.682853in}}%
\pgfpathlineto{\pgfqpoint{1.022459in}{0.688752in}}%
\pgfpathlineto{\pgfqpoint{1.031541in}{0.688752in}}%
\pgfpathlineto{\pgfqpoint{1.031541in}{0.682853in}}%
\pgfpathmoveto{\pgfqpoint{1.040623in}{0.694650in}}%
\pgfpathlineto{\pgfqpoint{1.040623in}{0.694650in}}%
\pgfpathlineto{\pgfqpoint{1.040623in}{0.700548in}}%
\pgfpathlineto{\pgfqpoint{1.049705in}{0.700548in}}%
\pgfpathlineto{\pgfqpoint{1.049705in}{0.694650in}}%
\pgfpathmoveto{\pgfqpoint{1.058787in}{0.706446in}}%
\pgfpathlineto{\pgfqpoint{1.058787in}{0.706446in}}%
\pgfpathlineto{\pgfqpoint{1.058787in}{0.712344in}}%
\pgfpathlineto{\pgfqpoint{1.067869in}{0.712344in}}%
\pgfpathlineto{\pgfqpoint{1.067869in}{0.706446in}}%
\pgfpathmoveto{\pgfqpoint{1.076951in}{0.718243in}}%
\pgfpathlineto{\pgfqpoint{1.076951in}{0.718243in}}%
\pgfpathlineto{\pgfqpoint{1.076951in}{0.724141in}}%
\pgfpathlineto{\pgfqpoint{1.086033in}{0.724141in}}%
\pgfpathlineto{\pgfqpoint{1.086033in}{0.718243in}}%
\pgfpathmoveto{\pgfqpoint{1.095115in}{0.730039in}}%
\pgfpathlineto{\pgfqpoint{1.095115in}{0.730039in}}%
\pgfpathlineto{\pgfqpoint{1.095115in}{0.735937in}}%
\pgfpathlineto{\pgfqpoint{1.104197in}{0.735937in}}%
\pgfpathlineto{\pgfqpoint{1.104197in}{0.730039in}}%
\pgfpathmoveto{\pgfqpoint{1.113279in}{0.741836in}}%
\pgfpathlineto{\pgfqpoint{1.113279in}{0.741836in}}%
\pgfpathlineto{\pgfqpoint{1.113279in}{0.747734in}}%
\pgfpathlineto{\pgfqpoint{1.122361in}{0.747734in}}%
\pgfpathlineto{\pgfqpoint{1.122361in}{0.741836in}}%
\pgfpathmoveto{\pgfqpoint{1.131443in}{0.753632in}}%
\pgfpathlineto{\pgfqpoint{1.131443in}{0.753632in}}%
\pgfpathlineto{\pgfqpoint{1.131443in}{0.759530in}}%
\pgfpathlineto{\pgfqpoint{1.140525in}{0.759530in}}%
\pgfpathlineto{\pgfqpoint{1.140525in}{0.753632in}}%
\pgfpathmoveto{\pgfqpoint{1.149607in}{0.765428in}}%
\pgfpathlineto{\pgfqpoint{1.149607in}{0.765428in}}%
\pgfpathlineto{\pgfqpoint{1.149607in}{0.771327in}}%
\pgfpathlineto{\pgfqpoint{1.158689in}{0.771327in}}%
\pgfpathlineto{\pgfqpoint{1.158689in}{0.765428in}}%
\pgfpathmoveto{\pgfqpoint{1.149607in}{0.771327in}}%
\pgfpathlineto{\pgfqpoint{1.149607in}{0.771327in}}%
\pgfpathlineto{\pgfqpoint{1.149607in}{0.777225in}}%
\pgfpathlineto{\pgfqpoint{1.158689in}{0.777225in}}%
\pgfpathlineto{\pgfqpoint{1.158689in}{0.771327in}}%
\pgfpathmoveto{\pgfqpoint{1.149607in}{0.777225in}}%
\pgfpathlineto{\pgfqpoint{1.149607in}{0.777225in}}%
\pgfpathlineto{\pgfqpoint{1.149607in}{0.783123in}}%
\pgfpathlineto{\pgfqpoint{1.158689in}{0.783123in}}%
\pgfpathlineto{\pgfqpoint{1.158689in}{0.777225in}}%
\pgfpathmoveto{\pgfqpoint{1.158689in}{0.777225in}}%
\pgfpathlineto{\pgfqpoint{1.158689in}{0.777225in}}%
\pgfpathlineto{\pgfqpoint{1.158689in}{0.783123in}}%
\pgfpathlineto{\pgfqpoint{1.167771in}{0.783123in}}%
\pgfpathlineto{\pgfqpoint{1.167771in}{0.777225in}}%
\pgfpathmoveto{\pgfqpoint{1.167771in}{0.783123in}}%
\pgfpathlineto{\pgfqpoint{1.167771in}{0.783123in}}%
\pgfpathlineto{\pgfqpoint{1.167771in}{0.789021in}}%
\pgfpathlineto{\pgfqpoint{1.176853in}{0.789021in}}%
\pgfpathlineto{\pgfqpoint{1.176853in}{0.783123in}}%
\pgfpathmoveto{\pgfqpoint{1.167771in}{0.789021in}}%
\pgfpathlineto{\pgfqpoint{1.167771in}{0.789021in}}%
\pgfpathlineto{\pgfqpoint{1.167771in}{0.794920in}}%
\pgfpathlineto{\pgfqpoint{1.176853in}{0.794920in}}%
\pgfpathlineto{\pgfqpoint{1.176853in}{0.789021in}}%
\pgfpathmoveto{\pgfqpoint{1.176853in}{0.789021in}}%
\pgfpathlineto{\pgfqpoint{1.176853in}{0.789021in}}%
\pgfpathlineto{\pgfqpoint{1.176853in}{0.794920in}}%
\pgfpathlineto{\pgfqpoint{1.185936in}{0.794920in}}%
\pgfpathlineto{\pgfqpoint{1.185936in}{0.789021in}}%
\pgfpathmoveto{\pgfqpoint{1.185936in}{0.794920in}}%
\pgfpathlineto{\pgfqpoint{1.185936in}{0.794920in}}%
\pgfpathlineto{\pgfqpoint{1.185936in}{0.800818in}}%
\pgfpathlineto{\pgfqpoint{1.195018in}{0.800818in}}%
\pgfpathlineto{\pgfqpoint{1.195018in}{0.794920in}}%
\pgfpathmoveto{\pgfqpoint{1.185936in}{0.800818in}}%
\pgfpathlineto{\pgfqpoint{1.185936in}{0.800818in}}%
\pgfpathlineto{\pgfqpoint{1.185936in}{0.806717in}}%
\pgfpathlineto{\pgfqpoint{1.195018in}{0.806717in}}%
\pgfpathlineto{\pgfqpoint{1.195018in}{0.800818in}}%
\pgfpathmoveto{\pgfqpoint{1.195018in}{0.800818in}}%
\pgfpathlineto{\pgfqpoint{1.195018in}{0.800818in}}%
\pgfpathlineto{\pgfqpoint{1.195018in}{0.806717in}}%
\pgfpathlineto{\pgfqpoint{1.204100in}{0.806717in}}%
\pgfpathlineto{\pgfqpoint{1.204100in}{0.800818in}}%
\pgfpathmoveto{\pgfqpoint{1.204100in}{0.806717in}}%
\pgfpathlineto{\pgfqpoint{1.204100in}{0.806717in}}%
\pgfpathlineto{\pgfqpoint{1.204100in}{0.812615in}}%
\pgfpathlineto{\pgfqpoint{1.213182in}{0.812615in}}%
\pgfpathlineto{\pgfqpoint{1.213182in}{0.806717in}}%
\pgfpathmoveto{\pgfqpoint{1.204100in}{0.812615in}}%
\pgfpathlineto{\pgfqpoint{1.204100in}{0.812615in}}%
\pgfpathlineto{\pgfqpoint{1.204100in}{0.818514in}}%
\pgfpathlineto{\pgfqpoint{1.213182in}{0.818514in}}%
\pgfpathlineto{\pgfqpoint{1.213182in}{0.812615in}}%
\pgfpathmoveto{\pgfqpoint{1.213182in}{0.812615in}}%
\pgfpathlineto{\pgfqpoint{1.213182in}{0.812615in}}%
\pgfpathlineto{\pgfqpoint{1.213182in}{0.818514in}}%
\pgfpathlineto{\pgfqpoint{1.222264in}{0.818514in}}%
\pgfpathlineto{\pgfqpoint{1.222264in}{0.812615in}}%
\pgfpathmoveto{\pgfqpoint{1.222264in}{0.818514in}}%
\pgfpathlineto{\pgfqpoint{1.222264in}{0.818514in}}%
\pgfpathlineto{\pgfqpoint{1.222264in}{0.824412in}}%
\pgfpathlineto{\pgfqpoint{1.231346in}{0.824412in}}%
\pgfpathlineto{\pgfqpoint{1.231346in}{0.818514in}}%
\pgfpathmoveto{\pgfqpoint{1.222264in}{0.824412in}}%
\pgfpathlineto{\pgfqpoint{1.222264in}{0.824412in}}%
\pgfpathlineto{\pgfqpoint{1.222264in}{0.830311in}}%
\pgfpathlineto{\pgfqpoint{1.231346in}{0.830311in}}%
\pgfpathlineto{\pgfqpoint{1.231346in}{0.824412in}}%
\pgfpathmoveto{\pgfqpoint{1.231346in}{0.824412in}}%
\pgfpathlineto{\pgfqpoint{1.231346in}{0.824412in}}%
\pgfpathlineto{\pgfqpoint{1.231346in}{0.830311in}}%
\pgfpathlineto{\pgfqpoint{1.240428in}{0.830311in}}%
\pgfpathlineto{\pgfqpoint{1.240428in}{0.824412in}}%
\pgfpathmoveto{\pgfqpoint{1.240428in}{0.830311in}}%
\pgfpathlineto{\pgfqpoint{1.240428in}{0.830311in}}%
\pgfpathlineto{\pgfqpoint{1.240428in}{0.836209in}}%
\pgfpathlineto{\pgfqpoint{1.249510in}{0.836209in}}%
\pgfpathlineto{\pgfqpoint{1.249510in}{0.830311in}}%
\pgfpathmoveto{\pgfqpoint{1.240428in}{0.836209in}}%
\pgfpathlineto{\pgfqpoint{1.240428in}{0.836209in}}%
\pgfpathlineto{\pgfqpoint{1.240428in}{0.842108in}}%
\pgfpathlineto{\pgfqpoint{1.249510in}{0.842108in}}%
\pgfpathlineto{\pgfqpoint{1.249510in}{0.836209in}}%
\pgfpathmoveto{\pgfqpoint{1.249510in}{0.836209in}}%
\pgfpathlineto{\pgfqpoint{1.249510in}{0.836209in}}%
\pgfpathlineto{\pgfqpoint{1.249510in}{0.842108in}}%
\pgfpathlineto{\pgfqpoint{1.258592in}{0.842108in}}%
\pgfpathlineto{\pgfqpoint{1.258592in}{0.836209in}}%
\pgfpathmoveto{\pgfqpoint{1.258592in}{0.842108in}}%
\pgfpathlineto{\pgfqpoint{1.258592in}{0.842108in}}%
\pgfpathlineto{\pgfqpoint{1.258592in}{0.848006in}}%
\pgfpathlineto{\pgfqpoint{1.267674in}{0.848006in}}%
\pgfpathlineto{\pgfqpoint{1.267674in}{0.842108in}}%
\pgfpathmoveto{\pgfqpoint{1.258592in}{0.848006in}}%
\pgfpathlineto{\pgfqpoint{1.258592in}{0.848006in}}%
\pgfpathlineto{\pgfqpoint{1.258592in}{0.853905in}}%
\pgfpathlineto{\pgfqpoint{1.267674in}{0.853905in}}%
\pgfpathlineto{\pgfqpoint{1.267674in}{0.848006in}}%
\pgfpathmoveto{\pgfqpoint{1.267674in}{0.848006in}}%
\pgfpathlineto{\pgfqpoint{1.267674in}{0.848006in}}%
\pgfpathlineto{\pgfqpoint{1.267674in}{0.853905in}}%
\pgfpathlineto{\pgfqpoint{1.276757in}{0.853905in}}%
\pgfpathlineto{\pgfqpoint{1.276757in}{0.848006in}}%
\pgfpathmoveto{\pgfqpoint{1.276757in}{0.853905in}}%
\pgfpathlineto{\pgfqpoint{1.276757in}{0.853905in}}%
\pgfpathlineto{\pgfqpoint{1.276757in}{0.859803in}}%
\pgfpathlineto{\pgfqpoint{1.285839in}{0.859803in}}%
\pgfpathlineto{\pgfqpoint{1.285839in}{0.853905in}}%
\pgfpathmoveto{\pgfqpoint{1.276757in}{0.859803in}}%
\pgfpathlineto{\pgfqpoint{1.276757in}{0.859803in}}%
\pgfpathlineto{\pgfqpoint{1.276757in}{0.865701in}}%
\pgfpathlineto{\pgfqpoint{1.285839in}{0.865701in}}%
\pgfpathlineto{\pgfqpoint{1.285839in}{0.859803in}}%
\pgfpathmoveto{\pgfqpoint{1.285839in}{0.859803in}}%
\pgfpathlineto{\pgfqpoint{1.285839in}{0.859803in}}%
\pgfpathlineto{\pgfqpoint{1.285839in}{0.865701in}}%
\pgfpathlineto{\pgfqpoint{1.294921in}{0.865701in}}%
\pgfpathlineto{\pgfqpoint{1.294921in}{0.859803in}}%
\pgfpathmoveto{\pgfqpoint{1.294921in}{0.865701in}}%
\pgfpathlineto{\pgfqpoint{1.294921in}{0.865701in}}%
\pgfpathlineto{\pgfqpoint{1.294921in}{0.871600in}}%
\pgfpathlineto{\pgfqpoint{1.304003in}{0.871600in}}%
\pgfpathlineto{\pgfqpoint{1.304003in}{0.865701in}}%
\pgfpathmoveto{\pgfqpoint{1.294921in}{0.871600in}}%
\pgfpathlineto{\pgfqpoint{1.294921in}{0.871600in}}%
\pgfpathlineto{\pgfqpoint{1.294921in}{0.877498in}}%
\pgfpathlineto{\pgfqpoint{1.304003in}{0.877498in}}%
\pgfpathlineto{\pgfqpoint{1.304003in}{0.871600in}}%
\pgfpathmoveto{\pgfqpoint{1.304003in}{0.871600in}}%
\pgfpathlineto{\pgfqpoint{1.304003in}{0.871600in}}%
\pgfpathlineto{\pgfqpoint{1.304003in}{0.877498in}}%
\pgfpathlineto{\pgfqpoint{1.313085in}{0.877498in}}%
\pgfpathlineto{\pgfqpoint{1.313085in}{0.871600in}}%
\pgfpathmoveto{\pgfqpoint{1.313085in}{0.877498in}}%
\pgfpathlineto{\pgfqpoint{1.313085in}{0.877498in}}%
\pgfpathlineto{\pgfqpoint{1.313085in}{0.883397in}}%
\pgfpathlineto{\pgfqpoint{1.322167in}{0.883397in}}%
\pgfpathlineto{\pgfqpoint{1.322167in}{0.877498in}}%
\pgfpathmoveto{\pgfqpoint{1.313085in}{0.883397in}}%
\pgfpathlineto{\pgfqpoint{1.313085in}{0.883397in}}%
\pgfpathlineto{\pgfqpoint{1.313085in}{0.889296in}}%
\pgfpathlineto{\pgfqpoint{1.322167in}{0.889296in}}%
\pgfpathlineto{\pgfqpoint{1.322167in}{0.883397in}}%
\pgfpathmoveto{\pgfqpoint{1.322167in}{0.883397in}}%
\pgfpathlineto{\pgfqpoint{1.322167in}{0.883397in}}%
\pgfpathlineto{\pgfqpoint{1.322167in}{0.889296in}}%
\pgfpathlineto{\pgfqpoint{1.331249in}{0.889296in}}%
\pgfpathlineto{\pgfqpoint{1.331249in}{0.883397in}}%
\pgfpathmoveto{\pgfqpoint{1.331249in}{0.889296in}}%
\pgfpathlineto{\pgfqpoint{1.331249in}{0.889296in}}%
\pgfpathlineto{\pgfqpoint{1.331249in}{0.895194in}}%
\pgfpathlineto{\pgfqpoint{1.340331in}{0.895194in}}%
\pgfpathlineto{\pgfqpoint{1.340331in}{0.889296in}}%
\pgfpathmoveto{\pgfqpoint{1.331249in}{0.895194in}}%
\pgfpathlineto{\pgfqpoint{1.331249in}{0.895194in}}%
\pgfpathlineto{\pgfqpoint{1.331249in}{0.901093in}}%
\pgfpathlineto{\pgfqpoint{1.340331in}{0.901093in}}%
\pgfpathlineto{\pgfqpoint{1.340331in}{0.895194in}}%
\pgfpathmoveto{\pgfqpoint{1.340331in}{0.895194in}}%
\pgfpathlineto{\pgfqpoint{1.340331in}{0.895194in}}%
\pgfpathlineto{\pgfqpoint{1.340331in}{0.901093in}}%
\pgfpathlineto{\pgfqpoint{1.349413in}{0.901093in}}%
\pgfpathlineto{\pgfqpoint{1.349413in}{0.895194in}}%
\pgfpathmoveto{\pgfqpoint{1.349413in}{0.901093in}}%
\pgfpathlineto{\pgfqpoint{1.349413in}{0.901093in}}%
\pgfpathlineto{\pgfqpoint{1.349413in}{0.906991in}}%
\pgfpathlineto{\pgfqpoint{1.358495in}{0.906991in}}%
\pgfpathlineto{\pgfqpoint{1.358495in}{0.901093in}}%
\pgfpathmoveto{\pgfqpoint{1.349413in}{0.906991in}}%
\pgfpathlineto{\pgfqpoint{1.349413in}{0.906991in}}%
\pgfpathlineto{\pgfqpoint{1.349413in}{0.912890in}}%
\pgfpathlineto{\pgfqpoint{1.358495in}{0.912890in}}%
\pgfpathlineto{\pgfqpoint{1.358495in}{0.906991in}}%
\pgfpathmoveto{\pgfqpoint{1.358495in}{0.906991in}}%
\pgfpathlineto{\pgfqpoint{1.358495in}{0.906991in}}%
\pgfpathlineto{\pgfqpoint{1.358495in}{0.912890in}}%
\pgfpathlineto{\pgfqpoint{1.367577in}{0.912890in}}%
\pgfpathlineto{\pgfqpoint{1.367577in}{0.906991in}}%
\pgfpathmoveto{\pgfqpoint{1.367577in}{0.906991in}}%
\pgfpathlineto{\pgfqpoint{1.367577in}{0.906991in}}%
\pgfpathlineto{\pgfqpoint{1.367577in}{0.912890in}}%
\pgfpathlineto{\pgfqpoint{1.376658in}{0.912890in}}%
\pgfpathlineto{\pgfqpoint{1.376658in}{0.906991in}}%
\pgfpathmoveto{\pgfqpoint{1.385740in}{0.918788in}}%
\pgfpathlineto{\pgfqpoint{1.385740in}{0.918788in}}%
\pgfpathlineto{\pgfqpoint{1.385740in}{0.924687in}}%
\pgfpathlineto{\pgfqpoint{1.394822in}{0.924687in}}%
\pgfpathlineto{\pgfqpoint{1.394822in}{0.918788in}}%
\pgfpathmoveto{\pgfqpoint{1.403904in}{0.930585in}}%
\pgfpathlineto{\pgfqpoint{1.403904in}{0.930585in}}%
\pgfpathlineto{\pgfqpoint{1.403904in}{0.936484in}}%
\pgfpathlineto{\pgfqpoint{1.412986in}{0.936484in}}%
\pgfpathlineto{\pgfqpoint{1.412986in}{0.930585in}}%
\pgfpathmoveto{\pgfqpoint{1.422068in}{0.942383in}}%
\pgfpathlineto{\pgfqpoint{1.422068in}{0.942383in}}%
\pgfpathlineto{\pgfqpoint{1.422068in}{0.948281in}}%
\pgfpathlineto{\pgfqpoint{1.431149in}{0.948281in}}%
\pgfpathlineto{\pgfqpoint{1.431149in}{0.942383in}}%
\pgfpathmoveto{\pgfqpoint{1.440231in}{0.954180in}}%
\pgfpathlineto{\pgfqpoint{1.440231in}{0.954180in}}%
\pgfpathlineto{\pgfqpoint{1.440231in}{0.960078in}}%
\pgfpathlineto{\pgfqpoint{1.449313in}{0.960078in}}%
\pgfpathlineto{\pgfqpoint{1.449313in}{0.954180in}}%
\pgfpathmoveto{\pgfqpoint{1.458395in}{0.965977in}}%
\pgfpathlineto{\pgfqpoint{1.458395in}{0.965977in}}%
\pgfpathlineto{\pgfqpoint{1.458395in}{0.971875in}}%
\pgfpathlineto{\pgfqpoint{1.467477in}{0.971875in}}%
\pgfpathlineto{\pgfqpoint{1.467477in}{0.965977in}}%
\pgfpathmoveto{\pgfqpoint{1.476559in}{0.977774in}}%
\pgfpathlineto{\pgfqpoint{1.476559in}{0.977774in}}%
\pgfpathlineto{\pgfqpoint{1.476559in}{0.983672in}}%
\pgfpathlineto{\pgfqpoint{1.485641in}{0.983672in}}%
\pgfpathlineto{\pgfqpoint{1.485641in}{0.977774in}}%
\pgfpathmoveto{\pgfqpoint{1.494723in}{0.989570in}}%
\pgfpathlineto{\pgfqpoint{1.494723in}{0.989570in}}%
\pgfpathlineto{\pgfqpoint{1.494723in}{0.995469in}}%
\pgfpathlineto{\pgfqpoint{1.503805in}{0.995469in}}%
\pgfpathlineto{\pgfqpoint{1.503805in}{0.989570in}}%
\pgfpathmoveto{\pgfqpoint{1.512887in}{1.001367in}}%
\pgfpathlineto{\pgfqpoint{1.512887in}{1.001367in}}%
\pgfpathlineto{\pgfqpoint{1.512887in}{1.007266in}}%
\pgfpathlineto{\pgfqpoint{1.521970in}{1.007266in}}%
\pgfpathlineto{\pgfqpoint{1.521970in}{1.001367in}}%
\pgfpathmoveto{\pgfqpoint{1.531052in}{1.013164in}}%
\pgfpathlineto{\pgfqpoint{1.531052in}{1.013164in}}%
\pgfpathlineto{\pgfqpoint{1.531052in}{1.019062in}}%
\pgfpathlineto{\pgfqpoint{1.540134in}{1.019062in}}%
\pgfpathlineto{\pgfqpoint{1.540134in}{1.013164in}}%
\pgfpathmoveto{\pgfqpoint{1.549216in}{1.024961in}}%
\pgfpathlineto{\pgfqpoint{1.549216in}{1.024961in}}%
\pgfpathlineto{\pgfqpoint{1.549216in}{1.030859in}}%
\pgfpathlineto{\pgfqpoint{1.558298in}{1.030859in}}%
\pgfpathlineto{\pgfqpoint{1.558298in}{1.024961in}}%
\pgfpathmoveto{\pgfqpoint{1.567380in}{1.036758in}}%
\pgfpathlineto{\pgfqpoint{1.567380in}{1.036758in}}%
\pgfpathlineto{\pgfqpoint{1.567380in}{1.042656in}}%
\pgfpathlineto{\pgfqpoint{1.576463in}{1.042656in}}%
\pgfpathlineto{\pgfqpoint{1.576463in}{1.036758in}}%
\pgfpathmoveto{\pgfqpoint{1.585545in}{1.048554in}}%
\pgfpathlineto{\pgfqpoint{1.585545in}{1.048554in}}%
\pgfpathlineto{\pgfqpoint{1.585545in}{1.054453in}}%
\pgfpathlineto{\pgfqpoint{1.594627in}{1.054453in}}%
\pgfpathlineto{\pgfqpoint{1.594627in}{1.048554in}}%
\pgfpathmoveto{\pgfqpoint{1.603709in}{1.060351in}}%
\pgfpathlineto{\pgfqpoint{1.603709in}{1.060351in}}%
\pgfpathlineto{\pgfqpoint{1.603709in}{1.066250in}}%
\pgfpathlineto{\pgfqpoint{1.612791in}{1.066250in}}%
\pgfpathlineto{\pgfqpoint{1.612791in}{1.060351in}}%
\pgfpathmoveto{\pgfqpoint{1.621874in}{1.072148in}}%
\pgfpathlineto{\pgfqpoint{1.621874in}{1.072148in}}%
\pgfpathlineto{\pgfqpoint{1.621874in}{1.078046in}}%
\pgfpathlineto{\pgfqpoint{1.630956in}{1.078046in}}%
\pgfpathlineto{\pgfqpoint{1.630956in}{1.072148in}}%
\pgfpathmoveto{\pgfqpoint{1.640038in}{1.083945in}}%
\pgfpathlineto{\pgfqpoint{1.640038in}{1.083945in}}%
\pgfpathlineto{\pgfqpoint{1.640038in}{1.089843in}}%
\pgfpathlineto{\pgfqpoint{1.649120in}{1.089843in}}%
\pgfpathlineto{\pgfqpoint{1.649120in}{1.083945in}}%
\pgfpathmoveto{\pgfqpoint{1.658202in}{1.095741in}}%
\pgfpathlineto{\pgfqpoint{1.658202in}{1.095741in}}%
\pgfpathlineto{\pgfqpoint{1.658202in}{1.101640in}}%
\pgfpathlineto{\pgfqpoint{1.667284in}{1.101640in}}%
\pgfpathlineto{\pgfqpoint{1.667284in}{1.095741in}}%
\pgfpathmoveto{\pgfqpoint{1.658202in}{1.101640in}}%
\pgfpathlineto{\pgfqpoint{1.658202in}{1.101640in}}%
\pgfpathlineto{\pgfqpoint{1.658202in}{1.107538in}}%
\pgfpathlineto{\pgfqpoint{1.667284in}{1.107538in}}%
\pgfpathlineto{\pgfqpoint{1.667284in}{1.101640in}}%
\pgfpathmoveto{\pgfqpoint{1.658202in}{1.107538in}}%
\pgfpathlineto{\pgfqpoint{1.658202in}{1.107538in}}%
\pgfpathlineto{\pgfqpoint{1.658202in}{1.113436in}}%
\pgfpathlineto{\pgfqpoint{1.667284in}{1.113436in}}%
\pgfpathlineto{\pgfqpoint{1.667284in}{1.107538in}}%
\pgfpathmoveto{\pgfqpoint{1.667284in}{1.107538in}}%
\pgfpathlineto{\pgfqpoint{1.667284in}{1.107538in}}%
\pgfpathlineto{\pgfqpoint{1.667284in}{1.113436in}}%
\pgfpathlineto{\pgfqpoint{1.676366in}{1.113436in}}%
\pgfpathlineto{\pgfqpoint{1.676366in}{1.107538in}}%
\pgfpathmoveto{\pgfqpoint{1.676366in}{1.113436in}}%
\pgfpathlineto{\pgfqpoint{1.676366in}{1.113436in}}%
\pgfpathlineto{\pgfqpoint{1.676366in}{1.119335in}}%
\pgfpathlineto{\pgfqpoint{1.685448in}{1.119335in}}%
\pgfpathlineto{\pgfqpoint{1.685448in}{1.113436in}}%
\pgfpathmoveto{\pgfqpoint{1.676366in}{1.119335in}}%
\pgfpathlineto{\pgfqpoint{1.676366in}{1.119335in}}%
\pgfpathlineto{\pgfqpoint{1.676366in}{1.125233in}}%
\pgfpathlineto{\pgfqpoint{1.685448in}{1.125233in}}%
\pgfpathlineto{\pgfqpoint{1.685448in}{1.119335in}}%
\pgfpathmoveto{\pgfqpoint{1.685448in}{1.119335in}}%
\pgfpathlineto{\pgfqpoint{1.685448in}{1.119335in}}%
\pgfpathlineto{\pgfqpoint{1.685448in}{1.125233in}}%
\pgfpathlineto{\pgfqpoint{1.694530in}{1.125233in}}%
\pgfpathlineto{\pgfqpoint{1.694530in}{1.119335in}}%
\pgfpathmoveto{\pgfqpoint{1.694530in}{1.125233in}}%
\pgfpathlineto{\pgfqpoint{1.694530in}{1.125233in}}%
\pgfpathlineto{\pgfqpoint{1.694530in}{1.131131in}}%
\pgfpathlineto{\pgfqpoint{1.703612in}{1.131131in}}%
\pgfpathlineto{\pgfqpoint{1.703612in}{1.125233in}}%
\pgfpathmoveto{\pgfqpoint{1.694530in}{1.131131in}}%
\pgfpathlineto{\pgfqpoint{1.694530in}{1.131131in}}%
\pgfpathlineto{\pgfqpoint{1.694530in}{1.137030in}}%
\pgfpathlineto{\pgfqpoint{1.703612in}{1.137030in}}%
\pgfpathlineto{\pgfqpoint{1.703612in}{1.131131in}}%
\pgfpathmoveto{\pgfqpoint{1.703612in}{1.131131in}}%
\pgfpathlineto{\pgfqpoint{1.703612in}{1.131131in}}%
\pgfpathlineto{\pgfqpoint{1.703612in}{1.137030in}}%
\pgfpathlineto{\pgfqpoint{1.712694in}{1.137030in}}%
\pgfpathlineto{\pgfqpoint{1.712694in}{1.131131in}}%
\pgfpathmoveto{\pgfqpoint{1.712694in}{1.137030in}}%
\pgfpathlineto{\pgfqpoint{1.712694in}{1.137030in}}%
\pgfpathlineto{\pgfqpoint{1.712694in}{1.142928in}}%
\pgfpathlineto{\pgfqpoint{1.721776in}{1.142928in}}%
\pgfpathlineto{\pgfqpoint{1.721776in}{1.137030in}}%
\pgfpathmoveto{\pgfqpoint{1.712694in}{1.142928in}}%
\pgfpathlineto{\pgfqpoint{1.712694in}{1.142928in}}%
\pgfpathlineto{\pgfqpoint{1.712694in}{1.148826in}}%
\pgfpathlineto{\pgfqpoint{1.721776in}{1.148826in}}%
\pgfpathlineto{\pgfqpoint{1.721776in}{1.142928in}}%
\pgfpathmoveto{\pgfqpoint{1.721776in}{1.142928in}}%
\pgfpathlineto{\pgfqpoint{1.721776in}{1.142928in}}%
\pgfpathlineto{\pgfqpoint{1.721776in}{1.148826in}}%
\pgfpathlineto{\pgfqpoint{1.730858in}{1.148826in}}%
\pgfpathlineto{\pgfqpoint{1.730858in}{1.142928in}}%
\pgfpathmoveto{\pgfqpoint{1.730858in}{1.148826in}}%
\pgfpathlineto{\pgfqpoint{1.730858in}{1.148826in}}%
\pgfpathlineto{\pgfqpoint{1.730858in}{1.154725in}}%
\pgfpathlineto{\pgfqpoint{1.739940in}{1.154725in}}%
\pgfpathlineto{\pgfqpoint{1.739940in}{1.148826in}}%
\pgfpathmoveto{\pgfqpoint{1.730858in}{1.154725in}}%
\pgfpathlineto{\pgfqpoint{1.730858in}{1.154725in}}%
\pgfpathlineto{\pgfqpoint{1.730858in}{1.160623in}}%
\pgfpathlineto{\pgfqpoint{1.739940in}{1.160623in}}%
\pgfpathlineto{\pgfqpoint{1.739940in}{1.154725in}}%
\pgfpathmoveto{\pgfqpoint{1.739940in}{1.154725in}}%
\pgfpathlineto{\pgfqpoint{1.739940in}{1.154725in}}%
\pgfpathlineto{\pgfqpoint{1.739940in}{1.160623in}}%
\pgfpathlineto{\pgfqpoint{1.749022in}{1.160623in}}%
\pgfpathlineto{\pgfqpoint{1.749022in}{1.154725in}}%
\pgfpathmoveto{\pgfqpoint{1.749022in}{1.160623in}}%
\pgfpathlineto{\pgfqpoint{1.749022in}{1.160623in}}%
\pgfpathlineto{\pgfqpoint{1.749022in}{1.166522in}}%
\pgfpathlineto{\pgfqpoint{1.758104in}{1.166522in}}%
\pgfpathlineto{\pgfqpoint{1.758104in}{1.160623in}}%
\pgfpathmoveto{\pgfqpoint{1.749022in}{1.166522in}}%
\pgfpathlineto{\pgfqpoint{1.749022in}{1.166522in}}%
\pgfpathlineto{\pgfqpoint{1.749022in}{1.172420in}}%
\pgfpathlineto{\pgfqpoint{1.758104in}{1.172420in}}%
\pgfpathlineto{\pgfqpoint{1.758104in}{1.166522in}}%
\pgfpathmoveto{\pgfqpoint{1.758104in}{1.166522in}}%
\pgfpathlineto{\pgfqpoint{1.758104in}{1.166522in}}%
\pgfpathlineto{\pgfqpoint{1.758104in}{1.172420in}}%
\pgfpathlineto{\pgfqpoint{1.767187in}{1.172420in}}%
\pgfpathlineto{\pgfqpoint{1.767187in}{1.166522in}}%
\pgfpathmoveto{\pgfqpoint{1.767187in}{1.172420in}}%
\pgfpathlineto{\pgfqpoint{1.767187in}{1.172420in}}%
\pgfpathlineto{\pgfqpoint{1.767187in}{1.178319in}}%
\pgfpathlineto{\pgfqpoint{1.776269in}{1.178319in}}%
\pgfpathlineto{\pgfqpoint{1.776269in}{1.172420in}}%
\pgfpathmoveto{\pgfqpoint{1.767187in}{1.178319in}}%
\pgfpathlineto{\pgfqpoint{1.767187in}{1.178319in}}%
\pgfpathlineto{\pgfqpoint{1.767187in}{1.184217in}}%
\pgfpathlineto{\pgfqpoint{1.776269in}{1.184217in}}%
\pgfpathlineto{\pgfqpoint{1.776269in}{1.178319in}}%
\pgfpathmoveto{\pgfqpoint{1.776269in}{1.178319in}}%
\pgfpathlineto{\pgfqpoint{1.776269in}{1.178319in}}%
\pgfpathlineto{\pgfqpoint{1.776269in}{1.184217in}}%
\pgfpathlineto{\pgfqpoint{1.785351in}{1.184217in}}%
\pgfpathlineto{\pgfqpoint{1.785351in}{1.178319in}}%
\pgfpathmoveto{\pgfqpoint{1.785351in}{1.184217in}}%
\pgfpathlineto{\pgfqpoint{1.785351in}{1.184217in}}%
\pgfpathlineto{\pgfqpoint{1.785351in}{1.190116in}}%
\pgfpathlineto{\pgfqpoint{1.794433in}{1.190116in}}%
\pgfpathlineto{\pgfqpoint{1.794433in}{1.184217in}}%
\pgfpathmoveto{\pgfqpoint{1.785351in}{1.190116in}}%
\pgfpathlineto{\pgfqpoint{1.785351in}{1.190116in}}%
\pgfpathlineto{\pgfqpoint{1.785351in}{1.196014in}}%
\pgfpathlineto{\pgfqpoint{1.794433in}{1.196014in}}%
\pgfpathlineto{\pgfqpoint{1.794433in}{1.190116in}}%
\pgfpathmoveto{\pgfqpoint{1.794433in}{1.190116in}}%
\pgfpathlineto{\pgfqpoint{1.794433in}{1.190116in}}%
\pgfpathlineto{\pgfqpoint{1.794433in}{1.196014in}}%
\pgfpathlineto{\pgfqpoint{1.803516in}{1.196014in}}%
\pgfpathlineto{\pgfqpoint{1.803516in}{1.190116in}}%
\pgfpathmoveto{\pgfqpoint{1.803516in}{1.196014in}}%
\pgfpathlineto{\pgfqpoint{1.803516in}{1.196014in}}%
\pgfpathlineto{\pgfqpoint{1.803516in}{1.201913in}}%
\pgfpathlineto{\pgfqpoint{1.812598in}{1.201913in}}%
\pgfpathlineto{\pgfqpoint{1.812598in}{1.196014in}}%
\pgfpathmoveto{\pgfqpoint{1.803516in}{1.201913in}}%
\pgfpathlineto{\pgfqpoint{1.803516in}{1.201913in}}%
\pgfpathlineto{\pgfqpoint{1.803516in}{1.207811in}}%
\pgfpathlineto{\pgfqpoint{1.812598in}{1.207811in}}%
\pgfpathlineto{\pgfqpoint{1.812598in}{1.201913in}}%
\pgfpathmoveto{\pgfqpoint{1.812598in}{1.201913in}}%
\pgfpathlineto{\pgfqpoint{1.812598in}{1.201913in}}%
\pgfpathlineto{\pgfqpoint{1.812598in}{1.207811in}}%
\pgfpathlineto{\pgfqpoint{1.821680in}{1.207811in}}%
\pgfpathlineto{\pgfqpoint{1.821680in}{1.201913in}}%
\pgfpathmoveto{\pgfqpoint{1.821680in}{1.207811in}}%
\pgfpathlineto{\pgfqpoint{1.821680in}{1.207811in}}%
\pgfpathlineto{\pgfqpoint{1.821680in}{1.213710in}}%
\pgfpathlineto{\pgfqpoint{1.830763in}{1.213710in}}%
\pgfpathlineto{\pgfqpoint{1.830763in}{1.207811in}}%
\pgfpathmoveto{\pgfqpoint{1.821680in}{1.213710in}}%
\pgfpathlineto{\pgfqpoint{1.821680in}{1.213710in}}%
\pgfpathlineto{\pgfqpoint{1.821680in}{1.219608in}}%
\pgfpathlineto{\pgfqpoint{1.830763in}{1.219608in}}%
\pgfpathlineto{\pgfqpoint{1.830763in}{1.213710in}}%
\pgfpathmoveto{\pgfqpoint{1.830763in}{1.213710in}}%
\pgfpathlineto{\pgfqpoint{1.830763in}{1.213710in}}%
\pgfpathlineto{\pgfqpoint{1.830763in}{1.219608in}}%
\pgfpathlineto{\pgfqpoint{1.839845in}{1.219608in}}%
\pgfpathlineto{\pgfqpoint{1.839845in}{1.213710in}}%
\pgfpathmoveto{\pgfqpoint{1.839845in}{1.219608in}}%
\pgfpathlineto{\pgfqpoint{1.839845in}{1.219608in}}%
\pgfpathlineto{\pgfqpoint{1.839845in}{1.225507in}}%
\pgfpathlineto{\pgfqpoint{1.848927in}{1.225507in}}%
\pgfpathlineto{\pgfqpoint{1.848927in}{1.219608in}}%
\pgfpathmoveto{\pgfqpoint{1.839845in}{1.225507in}}%
\pgfpathlineto{\pgfqpoint{1.839845in}{1.225507in}}%
\pgfpathlineto{\pgfqpoint{1.839845in}{1.231405in}}%
\pgfpathlineto{\pgfqpoint{1.848927in}{1.231405in}}%
\pgfpathlineto{\pgfqpoint{1.848927in}{1.225507in}}%
\pgfpathmoveto{\pgfqpoint{1.848927in}{1.225507in}}%
\pgfpathlineto{\pgfqpoint{1.848927in}{1.225507in}}%
\pgfpathlineto{\pgfqpoint{1.848927in}{1.231405in}}%
\pgfpathlineto{\pgfqpoint{1.858009in}{1.231405in}}%
\pgfpathlineto{\pgfqpoint{1.858009in}{1.225507in}}%
\pgfpathmoveto{\pgfqpoint{1.858009in}{1.231405in}}%
\pgfpathlineto{\pgfqpoint{1.858009in}{1.231405in}}%
\pgfpathlineto{\pgfqpoint{1.858009in}{1.237304in}}%
\pgfpathlineto{\pgfqpoint{1.867092in}{1.237304in}}%
\pgfpathlineto{\pgfqpoint{1.867092in}{1.231405in}}%
\pgfpathmoveto{\pgfqpoint{1.858009in}{1.237304in}}%
\pgfpathlineto{\pgfqpoint{1.858009in}{1.237304in}}%
\pgfpathlineto{\pgfqpoint{1.858009in}{1.243202in}}%
\pgfpathlineto{\pgfqpoint{1.867092in}{1.243202in}}%
\pgfpathlineto{\pgfqpoint{1.867092in}{1.237304in}}%
\pgfpathmoveto{\pgfqpoint{1.867092in}{1.237304in}}%
\pgfpathlineto{\pgfqpoint{1.867092in}{1.237304in}}%
\pgfpathlineto{\pgfqpoint{1.867092in}{1.243202in}}%
\pgfpathlineto{\pgfqpoint{1.876174in}{1.243202in}}%
\pgfpathlineto{\pgfqpoint{1.876174in}{1.237304in}}%
\pgfpathmoveto{\pgfqpoint{1.876174in}{1.243202in}}%
\pgfpathlineto{\pgfqpoint{1.876174in}{1.243202in}}%
\pgfpathlineto{\pgfqpoint{1.876174in}{1.249101in}}%
\pgfpathlineto{\pgfqpoint{1.885256in}{1.249101in}}%
\pgfpathlineto{\pgfqpoint{1.885256in}{1.243202in}}%
\pgfpathmoveto{\pgfqpoint{1.876174in}{1.249101in}}%
\pgfpathlineto{\pgfqpoint{1.876174in}{1.249101in}}%
\pgfpathlineto{\pgfqpoint{1.876174in}{1.254999in}}%
\pgfpathlineto{\pgfqpoint{1.885256in}{1.254999in}}%
\pgfpathlineto{\pgfqpoint{1.885256in}{1.249101in}}%
\pgfpathmoveto{\pgfqpoint{1.885256in}{1.249101in}}%
\pgfpathlineto{\pgfqpoint{1.885256in}{1.249101in}}%
\pgfpathlineto{\pgfqpoint{1.885256in}{1.254999in}}%
\pgfpathlineto{\pgfqpoint{1.894339in}{1.254999in}}%
\pgfpathlineto{\pgfqpoint{1.894339in}{1.249101in}}%
\pgfpathmoveto{\pgfqpoint{1.894339in}{1.254999in}}%
\pgfpathlineto{\pgfqpoint{1.894339in}{1.254999in}}%
\pgfpathlineto{\pgfqpoint{1.894339in}{1.260898in}}%
\pgfpathlineto{\pgfqpoint{1.903421in}{1.260898in}}%
\pgfpathlineto{\pgfqpoint{1.903421in}{1.254999in}}%
\pgfpathmoveto{\pgfqpoint{1.894339in}{1.260898in}}%
\pgfpathlineto{\pgfqpoint{1.894339in}{1.260898in}}%
\pgfpathlineto{\pgfqpoint{1.894339in}{1.266796in}}%
\pgfpathlineto{\pgfqpoint{1.903421in}{1.266796in}}%
\pgfpathlineto{\pgfqpoint{1.903421in}{1.260898in}}%
\pgfpathmoveto{\pgfqpoint{1.903421in}{1.260898in}}%
\pgfpathlineto{\pgfqpoint{1.903421in}{1.260898in}}%
\pgfpathlineto{\pgfqpoint{1.903421in}{1.266796in}}%
\pgfpathlineto{\pgfqpoint{1.912503in}{1.266796in}}%
\pgfpathlineto{\pgfqpoint{1.912503in}{1.260898in}}%
\pgfpathmoveto{\pgfqpoint{1.912503in}{1.266796in}}%
\pgfpathlineto{\pgfqpoint{1.912503in}{1.266796in}}%
\pgfpathlineto{\pgfqpoint{1.912503in}{1.272695in}}%
\pgfpathlineto{\pgfqpoint{1.921585in}{1.272695in}}%
\pgfpathlineto{\pgfqpoint{1.921585in}{1.266796in}}%
\pgfpathmoveto{\pgfqpoint{1.912503in}{1.272695in}}%
\pgfpathlineto{\pgfqpoint{1.912503in}{1.272695in}}%
\pgfpathlineto{\pgfqpoint{1.912503in}{1.278594in}}%
\pgfpathlineto{\pgfqpoint{1.921585in}{1.278594in}}%
\pgfpathlineto{\pgfqpoint{1.921585in}{1.272695in}}%
\pgfpathmoveto{\pgfqpoint{1.921585in}{1.272695in}}%
\pgfpathlineto{\pgfqpoint{1.921585in}{1.272695in}}%
\pgfpathlineto{\pgfqpoint{1.921585in}{1.278594in}}%
\pgfpathlineto{\pgfqpoint{1.930667in}{1.278594in}}%
\pgfpathlineto{\pgfqpoint{1.930667in}{1.272695in}}%
\pgfpathmoveto{\pgfqpoint{1.930667in}{1.278594in}}%
\pgfpathlineto{\pgfqpoint{1.930667in}{1.278594in}}%
\pgfpathlineto{\pgfqpoint{1.930667in}{1.284492in}}%
\pgfpathlineto{\pgfqpoint{1.939748in}{1.284492in}}%
\pgfpathlineto{\pgfqpoint{1.939748in}{1.278594in}}%
\pgfpathmoveto{\pgfqpoint{1.930667in}{1.284492in}}%
\pgfpathlineto{\pgfqpoint{1.930667in}{1.284492in}}%
\pgfpathlineto{\pgfqpoint{1.930667in}{1.290391in}}%
\pgfpathlineto{\pgfqpoint{1.939748in}{1.290391in}}%
\pgfpathlineto{\pgfqpoint{1.939748in}{1.284492in}}%
\pgfpathmoveto{\pgfqpoint{1.939748in}{1.284492in}}%
\pgfpathlineto{\pgfqpoint{1.939748in}{1.284492in}}%
\pgfpathlineto{\pgfqpoint{1.939748in}{1.290391in}}%
\pgfpathlineto{\pgfqpoint{1.948830in}{1.290391in}}%
\pgfpathlineto{\pgfqpoint{1.948830in}{1.284492in}}%
\pgfpathmoveto{\pgfqpoint{1.948830in}{1.290391in}}%
\pgfpathlineto{\pgfqpoint{1.948830in}{1.290391in}}%
\pgfpathlineto{\pgfqpoint{1.948830in}{1.296289in}}%
\pgfpathlineto{\pgfqpoint{1.957912in}{1.296289in}}%
\pgfpathlineto{\pgfqpoint{1.957912in}{1.290391in}}%
\pgfpathmoveto{\pgfqpoint{1.948830in}{1.296289in}}%
\pgfpathlineto{\pgfqpoint{1.948830in}{1.296289in}}%
\pgfpathlineto{\pgfqpoint{1.948830in}{1.302188in}}%
\pgfpathlineto{\pgfqpoint{1.957912in}{1.302188in}}%
\pgfpathlineto{\pgfqpoint{1.957912in}{1.296289in}}%
\pgfpathmoveto{\pgfqpoint{1.957912in}{1.296289in}}%
\pgfpathlineto{\pgfqpoint{1.957912in}{1.296289in}}%
\pgfpathlineto{\pgfqpoint{1.957912in}{1.302188in}}%
\pgfpathlineto{\pgfqpoint{1.966993in}{1.302188in}}%
\pgfpathlineto{\pgfqpoint{1.966993in}{1.296289in}}%
\pgfpathmoveto{\pgfqpoint{1.966993in}{1.302188in}}%
\pgfpathlineto{\pgfqpoint{1.966993in}{1.302188in}}%
\pgfpathlineto{\pgfqpoint{1.966993in}{1.308086in}}%
\pgfpathlineto{\pgfqpoint{1.976075in}{1.308086in}}%
\pgfpathlineto{\pgfqpoint{1.976075in}{1.302188in}}%
\pgfpathmoveto{\pgfqpoint{1.966993in}{1.308086in}}%
\pgfpathlineto{\pgfqpoint{1.966993in}{1.308086in}}%
\pgfpathlineto{\pgfqpoint{1.966993in}{1.313985in}}%
\pgfpathlineto{\pgfqpoint{1.976075in}{1.313985in}}%
\pgfpathlineto{\pgfqpoint{1.976075in}{1.308086in}}%
\pgfpathmoveto{\pgfqpoint{1.976075in}{1.308086in}}%
\pgfpathlineto{\pgfqpoint{1.976075in}{1.308086in}}%
\pgfpathlineto{\pgfqpoint{1.976075in}{1.313985in}}%
\pgfpathlineto{\pgfqpoint{1.985157in}{1.313985in}}%
\pgfpathlineto{\pgfqpoint{1.985157in}{1.308086in}}%
\pgfpathmoveto{\pgfqpoint{1.985157in}{1.308086in}}%
\pgfpathlineto{\pgfqpoint{1.985157in}{1.308086in}}%
\pgfpathlineto{\pgfqpoint{1.985157in}{1.313985in}}%
\pgfpathlineto{\pgfqpoint{1.994238in}{1.313985in}}%
\pgfpathlineto{\pgfqpoint{1.994238in}{1.308086in}}%
\pgfpathmoveto{\pgfqpoint{2.003320in}{1.319883in}}%
\pgfpathlineto{\pgfqpoint{2.003320in}{1.319883in}}%
\pgfpathlineto{\pgfqpoint{2.003320in}{1.325782in}}%
\pgfpathlineto{\pgfqpoint{2.012402in}{1.325782in}}%
\pgfpathlineto{\pgfqpoint{2.012402in}{1.319883in}}%
\pgfpathmoveto{\pgfqpoint{2.021483in}{1.331680in}}%
\pgfpathlineto{\pgfqpoint{2.021483in}{1.331680in}}%
\pgfpathlineto{\pgfqpoint{2.021483in}{1.337579in}}%
\pgfpathlineto{\pgfqpoint{2.030565in}{1.337579in}}%
\pgfpathlineto{\pgfqpoint{2.030565in}{1.331680in}}%
\pgfpathmoveto{\pgfqpoint{2.039647in}{1.343477in}}%
\pgfpathlineto{\pgfqpoint{2.039647in}{1.343477in}}%
\pgfpathlineto{\pgfqpoint{2.039647in}{1.349376in}}%
\pgfpathlineto{\pgfqpoint{2.048728in}{1.349376in}}%
\pgfpathlineto{\pgfqpoint{2.048728in}{1.343477in}}%
\pgfpathmoveto{\pgfqpoint{2.057810in}{1.355274in}}%
\pgfpathlineto{\pgfqpoint{2.057810in}{1.355274in}}%
\pgfpathlineto{\pgfqpoint{2.057810in}{1.361173in}}%
\pgfpathlineto{\pgfqpoint{2.066892in}{1.361173in}}%
\pgfpathlineto{\pgfqpoint{2.066892in}{1.355274in}}%
\pgfpathmoveto{\pgfqpoint{2.075974in}{1.367071in}}%
\pgfpathlineto{\pgfqpoint{2.075974in}{1.367071in}}%
\pgfpathlineto{\pgfqpoint{2.075974in}{1.372970in}}%
\pgfpathlineto{\pgfqpoint{2.085056in}{1.372970in}}%
\pgfpathlineto{\pgfqpoint{2.085056in}{1.367071in}}%
\pgfpathmoveto{\pgfqpoint{2.094138in}{1.378868in}}%
\pgfpathlineto{\pgfqpoint{2.094138in}{1.378868in}}%
\pgfpathlineto{\pgfqpoint{2.094138in}{1.384767in}}%
\pgfpathlineto{\pgfqpoint{2.103220in}{1.384767in}}%
\pgfpathlineto{\pgfqpoint{2.103220in}{1.378868in}}%
\pgfpathmoveto{\pgfqpoint{2.112302in}{1.390665in}}%
\pgfpathlineto{\pgfqpoint{2.112302in}{1.390665in}}%
\pgfpathlineto{\pgfqpoint{2.112302in}{1.396564in}}%
\pgfpathlineto{\pgfqpoint{2.121384in}{1.396564in}}%
\pgfpathlineto{\pgfqpoint{2.121384in}{1.390665in}}%
\pgfpathmoveto{\pgfqpoint{2.130466in}{1.402462in}}%
\pgfpathlineto{\pgfqpoint{2.130466in}{1.402462in}}%
\pgfpathlineto{\pgfqpoint{2.130466in}{1.408361in}}%
\pgfpathlineto{\pgfqpoint{2.139547in}{1.408361in}}%
\pgfpathlineto{\pgfqpoint{2.139547in}{1.402462in}}%
\pgfpathmoveto{\pgfqpoint{2.148629in}{1.414260in}}%
\pgfpathlineto{\pgfqpoint{2.148629in}{1.414260in}}%
\pgfpathlineto{\pgfqpoint{2.148629in}{1.420158in}}%
\pgfpathlineto{\pgfqpoint{2.157711in}{1.420158in}}%
\pgfpathlineto{\pgfqpoint{2.157711in}{1.414260in}}%
\pgfpathmoveto{\pgfqpoint{2.166793in}{1.426057in}}%
\pgfpathlineto{\pgfqpoint{2.166793in}{1.426057in}}%
\pgfpathlineto{\pgfqpoint{2.166793in}{1.431955in}}%
\pgfpathlineto{\pgfqpoint{2.175875in}{1.431955in}}%
\pgfpathlineto{\pgfqpoint{2.175875in}{1.426057in}}%
\pgfpathmoveto{\pgfqpoint{2.184957in}{1.437854in}}%
\pgfpathlineto{\pgfqpoint{2.184957in}{1.437854in}}%
\pgfpathlineto{\pgfqpoint{2.184957in}{1.443752in}}%
\pgfpathlineto{\pgfqpoint{2.194039in}{1.443752in}}%
\pgfpathlineto{\pgfqpoint{2.194039in}{1.437854in}}%
\pgfpathmoveto{\pgfqpoint{2.203121in}{1.449651in}}%
\pgfpathlineto{\pgfqpoint{2.203121in}{1.449651in}}%
\pgfpathlineto{\pgfqpoint{2.203121in}{1.455549in}}%
\pgfpathlineto{\pgfqpoint{2.212203in}{1.455549in}}%
\pgfpathlineto{\pgfqpoint{2.212203in}{1.449651in}}%
\pgfpathmoveto{\pgfqpoint{2.221286in}{1.461447in}}%
\pgfpathlineto{\pgfqpoint{2.221286in}{1.461447in}}%
\pgfpathlineto{\pgfqpoint{2.221286in}{1.467345in}}%
\pgfpathlineto{\pgfqpoint{2.230368in}{1.467345in}}%
\pgfpathlineto{\pgfqpoint{2.230368in}{1.461447in}}%
\pgfpathmoveto{\pgfqpoint{2.239451in}{1.473244in}}%
\pgfpathlineto{\pgfqpoint{2.239451in}{1.473244in}}%
\pgfpathlineto{\pgfqpoint{2.239451in}{1.479142in}}%
\pgfpathlineto{\pgfqpoint{2.248533in}{1.479142in}}%
\pgfpathlineto{\pgfqpoint{2.248533in}{1.473244in}}%
\pgfpathmoveto{\pgfqpoint{2.257615in}{1.485040in}}%
\pgfpathlineto{\pgfqpoint{2.257615in}{1.485040in}}%
\pgfpathlineto{\pgfqpoint{2.257615in}{1.490938in}}%
\pgfpathlineto{\pgfqpoint{2.266698in}{1.490938in}}%
\pgfpathlineto{\pgfqpoint{2.266698in}{1.485040in}}%
\pgfpathmoveto{\pgfqpoint{2.275780in}{1.496837in}}%
\pgfpathlineto{\pgfqpoint{2.275780in}{1.496837in}}%
\pgfpathlineto{\pgfqpoint{2.275780in}{1.502735in}}%
\pgfpathlineto{\pgfqpoint{2.284863in}{1.502735in}}%
\pgfpathlineto{\pgfqpoint{2.284863in}{1.496837in}}%
\pgfpathmoveto{\pgfqpoint{2.293945in}{1.508633in}}%
\pgfpathlineto{\pgfqpoint{2.293945in}{1.508633in}}%
\pgfpathlineto{\pgfqpoint{2.293945in}{1.514531in}}%
\pgfpathlineto{\pgfqpoint{2.303027in}{1.514531in}}%
\pgfpathlineto{\pgfqpoint{2.303027in}{1.508633in}}%
\pgfpathmoveto{\pgfqpoint{2.312110in}{1.526328in}}%
\pgfpathlineto{\pgfqpoint{2.312110in}{1.526328in}}%
\pgfpathlineto{\pgfqpoint{2.312110in}{1.532226in}}%
\pgfpathlineto{\pgfqpoint{2.321192in}{1.532226in}}%
\pgfpathlineto{\pgfqpoint{2.321192in}{1.526328in}}%
\pgfpathmoveto{\pgfqpoint{2.312110in}{1.532226in}}%
\pgfpathlineto{\pgfqpoint{2.312110in}{1.532226in}}%
\pgfpathlineto{\pgfqpoint{2.312110in}{1.538125in}}%
\pgfpathlineto{\pgfqpoint{2.321192in}{1.538125in}}%
\pgfpathlineto{\pgfqpoint{2.321192in}{1.532226in}}%
\pgfpathmoveto{\pgfqpoint{2.321192in}{1.532226in}}%
\pgfpathlineto{\pgfqpoint{2.321192in}{1.532226in}}%
\pgfpathlineto{\pgfqpoint{2.321192in}{1.538125in}}%
\pgfpathlineto{\pgfqpoint{2.330274in}{1.538125in}}%
\pgfpathlineto{\pgfqpoint{2.330274in}{1.532226in}}%
\pgfpathmoveto{\pgfqpoint{2.330274in}{1.538125in}}%
\pgfpathlineto{\pgfqpoint{2.330274in}{1.538125in}}%
\pgfpathlineto{\pgfqpoint{2.330274in}{1.544023in}}%
\pgfpathlineto{\pgfqpoint{2.339357in}{1.544023in}}%
\pgfpathlineto{\pgfqpoint{2.339357in}{1.538125in}}%
\pgfpathmoveto{\pgfqpoint{2.330274in}{1.544023in}}%
\pgfpathlineto{\pgfqpoint{2.330274in}{1.544023in}}%
\pgfpathlineto{\pgfqpoint{2.330274in}{1.549921in}}%
\pgfpathlineto{\pgfqpoint{2.339357in}{1.549921in}}%
\pgfpathlineto{\pgfqpoint{2.339357in}{1.544023in}}%
\pgfpathmoveto{\pgfqpoint{2.339357in}{1.544023in}}%
\pgfpathlineto{\pgfqpoint{2.339357in}{1.544023in}}%
\pgfpathlineto{\pgfqpoint{2.339357in}{1.549921in}}%
\pgfpathlineto{\pgfqpoint{2.348439in}{1.549921in}}%
\pgfpathlineto{\pgfqpoint{2.348439in}{1.544023in}}%
\pgfpathmoveto{\pgfqpoint{2.348439in}{1.549921in}}%
\pgfpathlineto{\pgfqpoint{2.348439in}{1.549921in}}%
\pgfpathlineto{\pgfqpoint{2.348439in}{1.555820in}}%
\pgfpathlineto{\pgfqpoint{2.357521in}{1.555820in}}%
\pgfpathlineto{\pgfqpoint{2.357521in}{1.549921in}}%
\pgfpathmoveto{\pgfqpoint{2.348439in}{1.555820in}}%
\pgfpathlineto{\pgfqpoint{2.348439in}{1.555820in}}%
\pgfpathlineto{\pgfqpoint{2.348439in}{1.561718in}}%
\pgfpathlineto{\pgfqpoint{2.357521in}{1.561718in}}%
\pgfpathlineto{\pgfqpoint{2.357521in}{1.555820in}}%
\pgfpathmoveto{\pgfqpoint{2.357521in}{1.555820in}}%
\pgfpathlineto{\pgfqpoint{2.357521in}{1.555820in}}%
\pgfpathlineto{\pgfqpoint{2.357521in}{1.561718in}}%
\pgfpathlineto{\pgfqpoint{2.366603in}{1.561718in}}%
\pgfpathlineto{\pgfqpoint{2.366603in}{1.555820in}}%
\pgfpathmoveto{\pgfqpoint{2.366603in}{1.561718in}}%
\pgfpathlineto{\pgfqpoint{2.366603in}{1.561718in}}%
\pgfpathlineto{\pgfqpoint{2.366603in}{1.567616in}}%
\pgfpathlineto{\pgfqpoint{2.375684in}{1.567616in}}%
\pgfpathlineto{\pgfqpoint{2.375684in}{1.561718in}}%
\pgfpathmoveto{\pgfqpoint{2.366603in}{1.567616in}}%
\pgfpathlineto{\pgfqpoint{2.366603in}{1.567616in}}%
\pgfpathlineto{\pgfqpoint{2.366603in}{1.573515in}}%
\pgfpathlineto{\pgfqpoint{2.375684in}{1.573515in}}%
\pgfpathlineto{\pgfqpoint{2.375684in}{1.567616in}}%
\pgfpathmoveto{\pgfqpoint{2.375684in}{1.567616in}}%
\pgfpathlineto{\pgfqpoint{2.375684in}{1.567616in}}%
\pgfpathlineto{\pgfqpoint{2.375684in}{1.573515in}}%
\pgfpathlineto{\pgfqpoint{2.384766in}{1.573515in}}%
\pgfpathlineto{\pgfqpoint{2.384766in}{1.567616in}}%
\pgfpathmoveto{\pgfqpoint{2.384766in}{1.573515in}}%
\pgfpathlineto{\pgfqpoint{2.384766in}{1.573515in}}%
\pgfpathlineto{\pgfqpoint{2.384766in}{1.579413in}}%
\pgfpathlineto{\pgfqpoint{2.393848in}{1.579413in}}%
\pgfpathlineto{\pgfqpoint{2.393848in}{1.573515in}}%
\pgfpathmoveto{\pgfqpoint{2.384766in}{1.579413in}}%
\pgfpathlineto{\pgfqpoint{2.384766in}{1.579413in}}%
\pgfpathlineto{\pgfqpoint{2.384766in}{1.585311in}}%
\pgfpathlineto{\pgfqpoint{2.393848in}{1.585311in}}%
\pgfpathlineto{\pgfqpoint{2.393848in}{1.579413in}}%
\pgfpathmoveto{\pgfqpoint{2.393848in}{1.579413in}}%
\pgfpathlineto{\pgfqpoint{2.393848in}{1.579413in}}%
\pgfpathlineto{\pgfqpoint{2.393848in}{1.585311in}}%
\pgfpathlineto{\pgfqpoint{2.402930in}{1.585311in}}%
\pgfpathlineto{\pgfqpoint{2.402930in}{1.579413in}}%
\pgfpathmoveto{\pgfqpoint{2.402930in}{1.585311in}}%
\pgfpathlineto{\pgfqpoint{2.402930in}{1.585311in}}%
\pgfpathlineto{\pgfqpoint{2.402930in}{1.591210in}}%
\pgfpathlineto{\pgfqpoint{2.412011in}{1.591210in}}%
\pgfpathlineto{\pgfqpoint{2.412011in}{1.585311in}}%
\pgfpathmoveto{\pgfqpoint{2.402930in}{1.591210in}}%
\pgfpathlineto{\pgfqpoint{2.402930in}{1.591210in}}%
\pgfpathlineto{\pgfqpoint{2.402930in}{1.597108in}}%
\pgfpathlineto{\pgfqpoint{2.412011in}{1.597108in}}%
\pgfpathlineto{\pgfqpoint{2.412011in}{1.591210in}}%
\pgfpathmoveto{\pgfqpoint{2.412011in}{1.591210in}}%
\pgfpathlineto{\pgfqpoint{2.412011in}{1.591210in}}%
\pgfpathlineto{\pgfqpoint{2.412011in}{1.597108in}}%
\pgfpathlineto{\pgfqpoint{2.421093in}{1.597108in}}%
\pgfpathlineto{\pgfqpoint{2.421093in}{1.591210in}}%
\pgfpathmoveto{\pgfqpoint{2.421093in}{1.597108in}}%
\pgfpathlineto{\pgfqpoint{2.421093in}{1.597108in}}%
\pgfpathlineto{\pgfqpoint{2.421093in}{1.603006in}}%
\pgfpathlineto{\pgfqpoint{2.430175in}{1.603006in}}%
\pgfpathlineto{\pgfqpoint{2.430175in}{1.597108in}}%
\pgfpathmoveto{\pgfqpoint{2.421093in}{1.603006in}}%
\pgfpathlineto{\pgfqpoint{2.421093in}{1.603006in}}%
\pgfpathlineto{\pgfqpoint{2.421093in}{1.608905in}}%
\pgfpathlineto{\pgfqpoint{2.430175in}{1.608905in}}%
\pgfpathlineto{\pgfqpoint{2.430175in}{1.603006in}}%
\pgfpathmoveto{\pgfqpoint{2.430175in}{1.603006in}}%
\pgfpathlineto{\pgfqpoint{2.430175in}{1.603006in}}%
\pgfpathlineto{\pgfqpoint{2.430175in}{1.608905in}}%
\pgfpathlineto{\pgfqpoint{2.439257in}{1.608905in}}%
\pgfpathlineto{\pgfqpoint{2.439257in}{1.603006in}}%
\pgfpathmoveto{\pgfqpoint{2.439257in}{1.608905in}}%
\pgfpathlineto{\pgfqpoint{2.439257in}{1.608905in}}%
\pgfpathlineto{\pgfqpoint{2.439257in}{1.614803in}}%
\pgfpathlineto{\pgfqpoint{2.448338in}{1.614803in}}%
\pgfpathlineto{\pgfqpoint{2.448338in}{1.608905in}}%
\pgfpathmoveto{\pgfqpoint{2.439257in}{1.614803in}}%
\pgfpathlineto{\pgfqpoint{2.439257in}{1.614803in}}%
\pgfpathlineto{\pgfqpoint{2.439257in}{1.620701in}}%
\pgfpathlineto{\pgfqpoint{2.448338in}{1.620701in}}%
\pgfpathlineto{\pgfqpoint{2.448338in}{1.614803in}}%
\pgfpathmoveto{\pgfqpoint{2.448338in}{1.614803in}}%
\pgfpathlineto{\pgfqpoint{2.448338in}{1.614803in}}%
\pgfpathlineto{\pgfqpoint{2.448338in}{1.620701in}}%
\pgfpathlineto{\pgfqpoint{2.457420in}{1.620701in}}%
\pgfpathlineto{\pgfqpoint{2.457420in}{1.614803in}}%
\pgfpathmoveto{\pgfqpoint{2.457420in}{1.620701in}}%
\pgfpathlineto{\pgfqpoint{2.457420in}{1.620701in}}%
\pgfpathlineto{\pgfqpoint{2.457420in}{1.626600in}}%
\pgfpathlineto{\pgfqpoint{2.466502in}{1.626600in}}%
\pgfpathlineto{\pgfqpoint{2.466502in}{1.620701in}}%
\pgfpathmoveto{\pgfqpoint{2.457420in}{1.626600in}}%
\pgfpathlineto{\pgfqpoint{2.457420in}{1.626600in}}%
\pgfpathlineto{\pgfqpoint{2.457420in}{1.632498in}}%
\pgfpathlineto{\pgfqpoint{2.466502in}{1.632498in}}%
\pgfpathlineto{\pgfqpoint{2.466502in}{1.626600in}}%
\pgfpathmoveto{\pgfqpoint{2.466502in}{1.626600in}}%
\pgfpathlineto{\pgfqpoint{2.466502in}{1.626600in}}%
\pgfpathlineto{\pgfqpoint{2.466502in}{1.632498in}}%
\pgfpathlineto{\pgfqpoint{2.475584in}{1.632498in}}%
\pgfpathlineto{\pgfqpoint{2.475584in}{1.626600in}}%
\pgfpathmoveto{\pgfqpoint{2.475584in}{1.632498in}}%
\pgfpathlineto{\pgfqpoint{2.475584in}{1.632498in}}%
\pgfpathlineto{\pgfqpoint{2.475584in}{1.638396in}}%
\pgfpathlineto{\pgfqpoint{2.484665in}{1.638396in}}%
\pgfpathlineto{\pgfqpoint{2.484665in}{1.632498in}}%
\pgfpathmoveto{\pgfqpoint{2.475584in}{1.638396in}}%
\pgfpathlineto{\pgfqpoint{2.475584in}{1.638396in}}%
\pgfpathlineto{\pgfqpoint{2.475584in}{1.644295in}}%
\pgfpathlineto{\pgfqpoint{2.484665in}{1.644295in}}%
\pgfpathlineto{\pgfqpoint{2.484665in}{1.638396in}}%
\pgfpathmoveto{\pgfqpoint{2.484665in}{1.638396in}}%
\pgfpathlineto{\pgfqpoint{2.484665in}{1.638396in}}%
\pgfpathlineto{\pgfqpoint{2.484665in}{1.644295in}}%
\pgfpathlineto{\pgfqpoint{2.493747in}{1.644295in}}%
\pgfpathlineto{\pgfqpoint{2.493747in}{1.638396in}}%
\pgfpathmoveto{\pgfqpoint{2.493747in}{1.644295in}}%
\pgfpathlineto{\pgfqpoint{2.493747in}{1.644295in}}%
\pgfpathlineto{\pgfqpoint{2.493747in}{1.650193in}}%
\pgfpathlineto{\pgfqpoint{2.502829in}{1.650193in}}%
\pgfpathlineto{\pgfqpoint{2.502829in}{1.644295in}}%
\pgfpathmoveto{\pgfqpoint{2.493747in}{1.650193in}}%
\pgfpathlineto{\pgfqpoint{2.493747in}{1.650193in}}%
\pgfpathlineto{\pgfqpoint{2.493747in}{1.656092in}}%
\pgfpathlineto{\pgfqpoint{2.502829in}{1.656092in}}%
\pgfpathlineto{\pgfqpoint{2.502829in}{1.650193in}}%
\pgfpathmoveto{\pgfqpoint{2.502829in}{1.650193in}}%
\pgfpathlineto{\pgfqpoint{2.502829in}{1.650193in}}%
\pgfpathlineto{\pgfqpoint{2.502829in}{1.656092in}}%
\pgfpathlineto{\pgfqpoint{2.511912in}{1.656092in}}%
\pgfpathlineto{\pgfqpoint{2.511912in}{1.650193in}}%
\pgfpathmoveto{\pgfqpoint{2.511912in}{1.656092in}}%
\pgfpathlineto{\pgfqpoint{2.511912in}{1.656092in}}%
\pgfpathlineto{\pgfqpoint{2.511912in}{1.661990in}}%
\pgfpathlineto{\pgfqpoint{2.520994in}{1.661990in}}%
\pgfpathlineto{\pgfqpoint{2.520994in}{1.656092in}}%
\pgfpathmoveto{\pgfqpoint{2.511912in}{1.661990in}}%
\pgfpathlineto{\pgfqpoint{2.511912in}{1.661990in}}%
\pgfpathlineto{\pgfqpoint{2.511912in}{1.667889in}}%
\pgfpathlineto{\pgfqpoint{2.520994in}{1.667889in}}%
\pgfpathlineto{\pgfqpoint{2.520994in}{1.661990in}}%
\pgfpathmoveto{\pgfqpoint{2.520994in}{1.661990in}}%
\pgfpathlineto{\pgfqpoint{2.520994in}{1.661990in}}%
\pgfpathlineto{\pgfqpoint{2.520994in}{1.667889in}}%
\pgfpathlineto{\pgfqpoint{2.530077in}{1.667889in}}%
\pgfpathlineto{\pgfqpoint{2.530077in}{1.661990in}}%
\pgfpathmoveto{\pgfqpoint{2.530077in}{1.667889in}}%
\pgfpathlineto{\pgfqpoint{2.530077in}{1.667889in}}%
\pgfpathlineto{\pgfqpoint{2.530077in}{1.673787in}}%
\pgfpathlineto{\pgfqpoint{2.539159in}{1.673787in}}%
\pgfpathlineto{\pgfqpoint{2.539159in}{1.667889in}}%
\pgfpathmoveto{\pgfqpoint{2.530077in}{1.673787in}}%
\pgfpathlineto{\pgfqpoint{2.530077in}{1.673787in}}%
\pgfpathlineto{\pgfqpoint{2.530077in}{1.679686in}}%
\pgfpathlineto{\pgfqpoint{2.539159in}{1.679686in}}%
\pgfpathlineto{\pgfqpoint{2.539159in}{1.673787in}}%
\pgfpathmoveto{\pgfqpoint{2.539159in}{1.673787in}}%
\pgfpathlineto{\pgfqpoint{2.539159in}{1.673787in}}%
\pgfpathlineto{\pgfqpoint{2.539159in}{1.679686in}}%
\pgfpathlineto{\pgfqpoint{2.548241in}{1.679686in}}%
\pgfpathlineto{\pgfqpoint{2.548241in}{1.673787in}}%
\pgfpathmoveto{\pgfqpoint{2.548241in}{1.679686in}}%
\pgfpathlineto{\pgfqpoint{2.548241in}{1.679686in}}%
\pgfpathlineto{\pgfqpoint{2.548241in}{1.685584in}}%
\pgfpathlineto{\pgfqpoint{2.557324in}{1.685584in}}%
\pgfpathlineto{\pgfqpoint{2.557324in}{1.679686in}}%
\pgfpathmoveto{\pgfqpoint{2.548241in}{1.685584in}}%
\pgfpathlineto{\pgfqpoint{2.548241in}{1.685584in}}%
\pgfpathlineto{\pgfqpoint{2.548241in}{1.691483in}}%
\pgfpathlineto{\pgfqpoint{2.557324in}{1.691483in}}%
\pgfpathlineto{\pgfqpoint{2.557324in}{1.685584in}}%
\pgfpathmoveto{\pgfqpoint{2.557324in}{1.685584in}}%
\pgfpathlineto{\pgfqpoint{2.557324in}{1.685584in}}%
\pgfpathlineto{\pgfqpoint{2.557324in}{1.691483in}}%
\pgfpathlineto{\pgfqpoint{2.566406in}{1.691483in}}%
\pgfpathlineto{\pgfqpoint{2.566406in}{1.685584in}}%
\pgfpathmoveto{\pgfqpoint{2.566406in}{1.691483in}}%
\pgfpathlineto{\pgfqpoint{2.566406in}{1.691483in}}%
\pgfpathlineto{\pgfqpoint{2.566406in}{1.697381in}}%
\pgfpathlineto{\pgfqpoint{2.575488in}{1.697381in}}%
\pgfpathlineto{\pgfqpoint{2.575488in}{1.691483in}}%
\pgfpathmoveto{\pgfqpoint{2.566406in}{1.697381in}}%
\pgfpathlineto{\pgfqpoint{2.566406in}{1.697381in}}%
\pgfpathlineto{\pgfqpoint{2.566406in}{1.703280in}}%
\pgfpathlineto{\pgfqpoint{2.575488in}{1.703280in}}%
\pgfpathlineto{\pgfqpoint{2.575488in}{1.697381in}}%
\pgfpathmoveto{\pgfqpoint{2.575488in}{1.697381in}}%
\pgfpathlineto{\pgfqpoint{2.575488in}{1.697381in}}%
\pgfpathlineto{\pgfqpoint{2.575488in}{1.703280in}}%
\pgfpathlineto{\pgfqpoint{2.584571in}{1.703280in}}%
\pgfpathlineto{\pgfqpoint{2.584571in}{1.697381in}}%
\pgfpathmoveto{\pgfqpoint{2.584571in}{1.703280in}}%
\pgfpathlineto{\pgfqpoint{2.584571in}{1.703280in}}%
\pgfpathlineto{\pgfqpoint{2.584571in}{1.709178in}}%
\pgfpathlineto{\pgfqpoint{2.593653in}{1.709178in}}%
\pgfpathlineto{\pgfqpoint{2.593653in}{1.703280in}}%
\pgfpathmoveto{\pgfqpoint{2.584571in}{1.709178in}}%
\pgfpathlineto{\pgfqpoint{2.584571in}{1.709178in}}%
\pgfpathlineto{\pgfqpoint{2.584571in}{1.715077in}}%
\pgfpathlineto{\pgfqpoint{2.593653in}{1.715077in}}%
\pgfpathlineto{\pgfqpoint{2.593653in}{1.709178in}}%
\pgfpathmoveto{\pgfqpoint{2.593653in}{1.709178in}}%
\pgfpathlineto{\pgfqpoint{2.593653in}{1.709178in}}%
\pgfpathlineto{\pgfqpoint{2.593653in}{1.715077in}}%
\pgfpathlineto{\pgfqpoint{2.602735in}{1.715077in}}%
\pgfpathlineto{\pgfqpoint{2.602735in}{1.709178in}}%
\pgfpathmoveto{\pgfqpoint{2.602735in}{1.715077in}}%
\pgfpathlineto{\pgfqpoint{2.602735in}{1.715077in}}%
\pgfpathlineto{\pgfqpoint{2.602735in}{1.720976in}}%
\pgfpathlineto{\pgfqpoint{2.611818in}{1.720976in}}%
\pgfpathlineto{\pgfqpoint{2.611818in}{1.715077in}}%
\pgfpathmoveto{\pgfqpoint{2.602735in}{1.720976in}}%
\pgfpathlineto{\pgfqpoint{2.602735in}{1.720976in}}%
\pgfpathlineto{\pgfqpoint{2.602735in}{1.726874in}}%
\pgfpathlineto{\pgfqpoint{2.611818in}{1.726874in}}%
\pgfpathlineto{\pgfqpoint{2.611818in}{1.720976in}}%
\pgfpathmoveto{\pgfqpoint{2.611818in}{1.720976in}}%
\pgfpathlineto{\pgfqpoint{2.611818in}{1.720976in}}%
\pgfpathlineto{\pgfqpoint{2.611818in}{1.726874in}}%
\pgfpathlineto{\pgfqpoint{2.620900in}{1.726874in}}%
\pgfpathlineto{\pgfqpoint{2.620900in}{1.720976in}}%
\pgfpathmoveto{\pgfqpoint{2.620900in}{1.726874in}}%
\pgfpathlineto{\pgfqpoint{2.620900in}{1.726874in}}%
\pgfpathlineto{\pgfqpoint{2.620900in}{1.732773in}}%
\pgfpathlineto{\pgfqpoint{2.629982in}{1.732773in}}%
\pgfpathlineto{\pgfqpoint{2.629982in}{1.726874in}}%
\pgfpathmoveto{\pgfqpoint{2.620900in}{1.732773in}}%
\pgfpathlineto{\pgfqpoint{2.620900in}{1.732773in}}%
\pgfpathlineto{\pgfqpoint{2.620900in}{1.738671in}}%
\pgfpathlineto{\pgfqpoint{2.629982in}{1.738671in}}%
\pgfpathlineto{\pgfqpoint{2.629982in}{1.732773in}}%
\pgfpathmoveto{\pgfqpoint{2.629982in}{1.732773in}}%
\pgfpathlineto{\pgfqpoint{2.629982in}{1.732773in}}%
\pgfpathlineto{\pgfqpoint{2.629982in}{1.738671in}}%
\pgfpathlineto{\pgfqpoint{2.639065in}{1.738671in}}%
\pgfpathlineto{\pgfqpoint{2.639065in}{1.732773in}}%
\pgfpathmoveto{\pgfqpoint{2.639065in}{1.738671in}}%
\pgfpathlineto{\pgfqpoint{2.639065in}{1.738671in}}%
\pgfpathlineto{\pgfqpoint{2.639065in}{1.744570in}}%
\pgfpathlineto{\pgfqpoint{2.648147in}{1.744570in}}%
\pgfpathlineto{\pgfqpoint{2.648147in}{1.738671in}}%
\pgfpathmoveto{\pgfqpoint{2.639065in}{1.744570in}}%
\pgfpathlineto{\pgfqpoint{2.639065in}{1.744570in}}%
\pgfpathlineto{\pgfqpoint{2.639065in}{1.750468in}}%
\pgfpathlineto{\pgfqpoint{2.648147in}{1.750468in}}%
\pgfpathlineto{\pgfqpoint{2.648147in}{1.744570in}}%
\pgfpathmoveto{\pgfqpoint{2.648147in}{1.744570in}}%
\pgfpathlineto{\pgfqpoint{2.648147in}{1.744570in}}%
\pgfpathlineto{\pgfqpoint{2.648147in}{1.750468in}}%
\pgfpathlineto{\pgfqpoint{2.657229in}{1.750468in}}%
\pgfpathlineto{\pgfqpoint{2.657229in}{1.744570in}}%
\pgfpathmoveto{\pgfqpoint{2.657229in}{1.750468in}}%
\pgfpathlineto{\pgfqpoint{2.657229in}{1.750468in}}%
\pgfpathlineto{\pgfqpoint{2.657229in}{1.756367in}}%
\pgfpathlineto{\pgfqpoint{2.666311in}{1.756367in}}%
\pgfpathlineto{\pgfqpoint{2.666311in}{1.750468in}}%
\pgfpathmoveto{\pgfqpoint{2.657229in}{1.756367in}}%
\pgfpathlineto{\pgfqpoint{2.657229in}{1.756367in}}%
\pgfpathlineto{\pgfqpoint{2.657229in}{1.762265in}}%
\pgfpathlineto{\pgfqpoint{2.666311in}{1.762265in}}%
\pgfpathlineto{\pgfqpoint{2.666311in}{1.756367in}}%
\pgfpathmoveto{\pgfqpoint{2.666311in}{1.756367in}}%
\pgfpathlineto{\pgfqpoint{2.666311in}{1.756367in}}%
\pgfpathlineto{\pgfqpoint{2.666311in}{1.762265in}}%
\pgfpathlineto{\pgfqpoint{2.675393in}{1.762265in}}%
\pgfpathlineto{\pgfqpoint{2.675393in}{1.756367in}}%
\pgfpathmoveto{\pgfqpoint{2.675393in}{1.762265in}}%
\pgfpathlineto{\pgfqpoint{2.675393in}{1.762265in}}%
\pgfpathlineto{\pgfqpoint{2.675393in}{1.768164in}}%
\pgfpathlineto{\pgfqpoint{2.684475in}{1.768164in}}%
\pgfpathlineto{\pgfqpoint{2.684475in}{1.762265in}}%
\pgfpathmoveto{\pgfqpoint{2.675393in}{1.768164in}}%
\pgfpathlineto{\pgfqpoint{2.675393in}{1.768164in}}%
\pgfpathlineto{\pgfqpoint{2.675393in}{1.774062in}}%
\pgfpathlineto{\pgfqpoint{2.684475in}{1.774062in}}%
\pgfpathlineto{\pgfqpoint{2.684475in}{1.768164in}}%
\pgfpathmoveto{\pgfqpoint{2.684475in}{1.768164in}}%
\pgfpathlineto{\pgfqpoint{2.684475in}{1.768164in}}%
\pgfpathlineto{\pgfqpoint{2.684475in}{1.774062in}}%
\pgfpathlineto{\pgfqpoint{2.693557in}{1.774062in}}%
\pgfpathlineto{\pgfqpoint{2.693557in}{1.768164in}}%
\pgfpathmoveto{\pgfqpoint{2.693557in}{1.774062in}}%
\pgfpathlineto{\pgfqpoint{2.693557in}{1.774062in}}%
\pgfpathlineto{\pgfqpoint{2.693557in}{1.779961in}}%
\pgfpathlineto{\pgfqpoint{2.702639in}{1.779961in}}%
\pgfpathlineto{\pgfqpoint{2.702639in}{1.774062in}}%
\pgfpathmoveto{\pgfqpoint{2.693557in}{1.779961in}}%
\pgfpathlineto{\pgfqpoint{2.693557in}{1.779961in}}%
\pgfpathlineto{\pgfqpoint{2.693557in}{1.785859in}}%
\pgfpathlineto{\pgfqpoint{2.702639in}{1.785859in}}%
\pgfpathlineto{\pgfqpoint{2.702639in}{1.779961in}}%
\pgfpathmoveto{\pgfqpoint{2.702639in}{1.779961in}}%
\pgfpathlineto{\pgfqpoint{2.702639in}{1.779961in}}%
\pgfpathlineto{\pgfqpoint{2.702639in}{1.785859in}}%
\pgfpathlineto{\pgfqpoint{2.711721in}{1.785859in}}%
\pgfpathlineto{\pgfqpoint{2.711721in}{1.779961in}}%
\pgfpathmoveto{\pgfqpoint{2.711721in}{1.785859in}}%
\pgfpathlineto{\pgfqpoint{2.711721in}{1.785859in}}%
\pgfpathlineto{\pgfqpoint{2.711721in}{1.791758in}}%
\pgfpathlineto{\pgfqpoint{2.720803in}{1.791758in}}%
\pgfpathlineto{\pgfqpoint{2.720803in}{1.785859in}}%
\pgfpathmoveto{\pgfqpoint{2.711721in}{1.791758in}}%
\pgfpathlineto{\pgfqpoint{2.711721in}{1.791758in}}%
\pgfpathlineto{\pgfqpoint{2.711721in}{1.797656in}}%
\pgfpathlineto{\pgfqpoint{2.720803in}{1.797656in}}%
\pgfpathlineto{\pgfqpoint{2.720803in}{1.791758in}}%
\pgfpathmoveto{\pgfqpoint{2.720803in}{1.791758in}}%
\pgfpathlineto{\pgfqpoint{2.720803in}{1.791758in}}%
\pgfpathlineto{\pgfqpoint{2.720803in}{1.797656in}}%
\pgfpathlineto{\pgfqpoint{2.729885in}{1.797656in}}%
\pgfpathlineto{\pgfqpoint{2.729885in}{1.791758in}}%
\pgfpathmoveto{\pgfqpoint{2.729885in}{1.797656in}}%
\pgfpathlineto{\pgfqpoint{2.729885in}{1.797656in}}%
\pgfpathlineto{\pgfqpoint{2.729885in}{1.803555in}}%
\pgfpathlineto{\pgfqpoint{2.738967in}{1.803555in}}%
\pgfpathlineto{\pgfqpoint{2.738967in}{1.797656in}}%
\pgfpathmoveto{\pgfqpoint{2.729885in}{1.803555in}}%
\pgfpathlineto{\pgfqpoint{2.729885in}{1.803555in}}%
\pgfpathlineto{\pgfqpoint{2.729885in}{1.809453in}}%
\pgfpathlineto{\pgfqpoint{2.738967in}{1.809453in}}%
\pgfpathlineto{\pgfqpoint{2.738967in}{1.803555in}}%
\pgfpathmoveto{\pgfqpoint{2.738967in}{1.803555in}}%
\pgfpathlineto{\pgfqpoint{2.738967in}{1.803555in}}%
\pgfpathlineto{\pgfqpoint{2.738967in}{1.809453in}}%
\pgfpathlineto{\pgfqpoint{2.748049in}{1.809453in}}%
\pgfpathlineto{\pgfqpoint{2.748049in}{1.803555in}}%
\pgfpathmoveto{\pgfqpoint{2.748049in}{1.809453in}}%
\pgfpathlineto{\pgfqpoint{2.748049in}{1.809453in}}%
\pgfpathlineto{\pgfqpoint{2.748049in}{1.815352in}}%
\pgfpathlineto{\pgfqpoint{2.757131in}{1.815352in}}%
\pgfpathlineto{\pgfqpoint{2.757131in}{1.809453in}}%
\pgfpathmoveto{\pgfqpoint{2.748049in}{1.815352in}}%
\pgfpathlineto{\pgfqpoint{2.748049in}{1.815352in}}%
\pgfpathlineto{\pgfqpoint{2.748049in}{1.821250in}}%
\pgfpathlineto{\pgfqpoint{2.757131in}{1.821250in}}%
\pgfpathlineto{\pgfqpoint{2.757131in}{1.815352in}}%
\pgfpathmoveto{\pgfqpoint{2.757131in}{1.815352in}}%
\pgfpathlineto{\pgfqpoint{2.757131in}{1.815352in}}%
\pgfpathlineto{\pgfqpoint{2.757131in}{1.821250in}}%
\pgfpathlineto{\pgfqpoint{2.766213in}{1.821250in}}%
\pgfpathlineto{\pgfqpoint{2.766213in}{1.815352in}}%
\pgfpathmoveto{\pgfqpoint{2.766213in}{1.821250in}}%
\pgfpathlineto{\pgfqpoint{2.766213in}{1.821250in}}%
\pgfpathlineto{\pgfqpoint{2.766213in}{1.827149in}}%
\pgfpathlineto{\pgfqpoint{2.775295in}{1.827149in}}%
\pgfpathlineto{\pgfqpoint{2.775295in}{1.821250in}}%
\pgfpathmoveto{\pgfqpoint{2.766213in}{1.827149in}}%
\pgfpathlineto{\pgfqpoint{2.766213in}{1.827149in}}%
\pgfpathlineto{\pgfqpoint{2.766213in}{1.833047in}}%
\pgfpathlineto{\pgfqpoint{2.775295in}{1.833047in}}%
\pgfpathlineto{\pgfqpoint{2.775295in}{1.827149in}}%
\pgfpathmoveto{\pgfqpoint{2.775295in}{1.827149in}}%
\pgfpathlineto{\pgfqpoint{2.775295in}{1.827149in}}%
\pgfpathlineto{\pgfqpoint{2.775295in}{1.833047in}}%
\pgfpathlineto{\pgfqpoint{2.784377in}{1.833047in}}%
\pgfpathlineto{\pgfqpoint{2.784377in}{1.827149in}}%
\pgfpathmoveto{\pgfqpoint{2.784377in}{1.833047in}}%
\pgfpathlineto{\pgfqpoint{2.784377in}{1.833047in}}%
\pgfpathlineto{\pgfqpoint{2.784377in}{1.838946in}}%
\pgfpathlineto{\pgfqpoint{2.793459in}{1.838946in}}%
\pgfpathlineto{\pgfqpoint{2.793459in}{1.833047in}}%
\pgfpathmoveto{\pgfqpoint{2.784377in}{1.838946in}}%
\pgfpathlineto{\pgfqpoint{2.784377in}{1.838946in}}%
\pgfpathlineto{\pgfqpoint{2.784377in}{1.844844in}}%
\pgfpathlineto{\pgfqpoint{2.793459in}{1.844844in}}%
\pgfpathlineto{\pgfqpoint{2.793459in}{1.838946in}}%
\pgfpathmoveto{\pgfqpoint{2.793459in}{1.838946in}}%
\pgfpathlineto{\pgfqpoint{2.793459in}{1.838946in}}%
\pgfpathlineto{\pgfqpoint{2.793459in}{1.844844in}}%
\pgfpathlineto{\pgfqpoint{2.802541in}{1.844844in}}%
\pgfpathlineto{\pgfqpoint{2.802541in}{1.838946in}}%
\pgfpathmoveto{\pgfqpoint{2.802541in}{1.844844in}}%
\pgfpathlineto{\pgfqpoint{2.802541in}{1.844844in}}%
\pgfpathlineto{\pgfqpoint{2.802541in}{1.850743in}}%
\pgfpathlineto{\pgfqpoint{2.811623in}{1.850743in}}%
\pgfpathlineto{\pgfqpoint{2.811623in}{1.844844in}}%
\pgfpathmoveto{\pgfqpoint{2.802541in}{1.850743in}}%
\pgfpathlineto{\pgfqpoint{2.802541in}{1.850743in}}%
\pgfpathlineto{\pgfqpoint{2.802541in}{1.856641in}}%
\pgfpathlineto{\pgfqpoint{2.811623in}{1.856641in}}%
\pgfpathlineto{\pgfqpoint{2.811623in}{1.850743in}}%
\pgfpathmoveto{\pgfqpoint{2.811623in}{1.850743in}}%
\pgfpathlineto{\pgfqpoint{2.811623in}{1.850743in}}%
\pgfpathlineto{\pgfqpoint{2.811623in}{1.856641in}}%
\pgfpathlineto{\pgfqpoint{2.820706in}{1.856641in}}%
\pgfpathlineto{\pgfqpoint{2.820706in}{1.850743in}}%
\pgfpathmoveto{\pgfqpoint{2.820706in}{1.856641in}}%
\pgfpathlineto{\pgfqpoint{2.820706in}{1.856641in}}%
\pgfpathlineto{\pgfqpoint{2.820706in}{1.862540in}}%
\pgfpathlineto{\pgfqpoint{2.829788in}{1.862540in}}%
\pgfpathlineto{\pgfqpoint{2.829788in}{1.856641in}}%
\pgfpathmoveto{\pgfqpoint{2.820706in}{1.862540in}}%
\pgfpathlineto{\pgfqpoint{2.820706in}{1.862540in}}%
\pgfpathlineto{\pgfqpoint{2.820706in}{1.868438in}}%
\pgfpathlineto{\pgfqpoint{2.829788in}{1.868438in}}%
\pgfpathlineto{\pgfqpoint{2.829788in}{1.862540in}}%
\pgfpathmoveto{\pgfqpoint{2.829788in}{1.862540in}}%
\pgfpathlineto{\pgfqpoint{2.829788in}{1.862540in}}%
\pgfpathlineto{\pgfqpoint{2.829788in}{1.868438in}}%
\pgfpathlineto{\pgfqpoint{2.838870in}{1.868438in}}%
\pgfpathlineto{\pgfqpoint{2.838870in}{1.862540in}}%
\pgfpathmoveto{\pgfqpoint{2.838870in}{1.868438in}}%
\pgfpathlineto{\pgfqpoint{2.838870in}{1.868438in}}%
\pgfpathlineto{\pgfqpoint{2.838870in}{1.874337in}}%
\pgfpathlineto{\pgfqpoint{2.847952in}{1.874337in}}%
\pgfpathlineto{\pgfqpoint{2.847952in}{1.868438in}}%
\pgfpathmoveto{\pgfqpoint{2.838870in}{1.874337in}}%
\pgfpathlineto{\pgfqpoint{2.838870in}{1.874337in}}%
\pgfpathlineto{\pgfqpoint{2.838870in}{1.880235in}}%
\pgfpathlineto{\pgfqpoint{2.847952in}{1.880235in}}%
\pgfpathlineto{\pgfqpoint{2.847952in}{1.874337in}}%
\pgfpathmoveto{\pgfqpoint{2.847952in}{1.874337in}}%
\pgfpathlineto{\pgfqpoint{2.847952in}{1.874337in}}%
\pgfpathlineto{\pgfqpoint{2.847952in}{1.880235in}}%
\pgfpathlineto{\pgfqpoint{2.857034in}{1.880235in}}%
\pgfpathlineto{\pgfqpoint{2.857034in}{1.874337in}}%
\pgfpathmoveto{\pgfqpoint{2.857034in}{1.880235in}}%
\pgfpathlineto{\pgfqpoint{2.857034in}{1.880235in}}%
\pgfpathlineto{\pgfqpoint{2.857034in}{1.886133in}}%
\pgfpathlineto{\pgfqpoint{2.866116in}{1.886133in}}%
\pgfpathlineto{\pgfqpoint{2.866116in}{1.880235in}}%
\pgfpathmoveto{\pgfqpoint{2.857034in}{1.886133in}}%
\pgfpathlineto{\pgfqpoint{2.857034in}{1.886133in}}%
\pgfpathlineto{\pgfqpoint{2.857034in}{1.892032in}}%
\pgfpathlineto{\pgfqpoint{2.866116in}{1.892032in}}%
\pgfpathlineto{\pgfqpoint{2.866116in}{1.886133in}}%
\pgfpathmoveto{\pgfqpoint{2.866116in}{1.886133in}}%
\pgfpathlineto{\pgfqpoint{2.866116in}{1.886133in}}%
\pgfpathlineto{\pgfqpoint{2.866116in}{1.892032in}}%
\pgfpathlineto{\pgfqpoint{2.875198in}{1.892032in}}%
\pgfpathlineto{\pgfqpoint{2.875198in}{1.886133in}}%
\pgfpathmoveto{\pgfqpoint{2.875198in}{1.892032in}}%
\pgfpathlineto{\pgfqpoint{2.875198in}{1.892032in}}%
\pgfpathlineto{\pgfqpoint{2.875198in}{1.897930in}}%
\pgfpathlineto{\pgfqpoint{2.884280in}{1.897930in}}%
\pgfpathlineto{\pgfqpoint{2.884280in}{1.892032in}}%
\pgfpathmoveto{\pgfqpoint{2.875198in}{1.897930in}}%
\pgfpathlineto{\pgfqpoint{2.875198in}{1.897930in}}%
\pgfpathlineto{\pgfqpoint{2.875198in}{1.903829in}}%
\pgfpathlineto{\pgfqpoint{2.884280in}{1.903829in}}%
\pgfpathlineto{\pgfqpoint{2.884280in}{1.897930in}}%
\pgfpathmoveto{\pgfqpoint{2.884280in}{1.897930in}}%
\pgfpathlineto{\pgfqpoint{2.884280in}{1.897930in}}%
\pgfpathlineto{\pgfqpoint{2.884280in}{1.903829in}}%
\pgfpathlineto{\pgfqpoint{2.893363in}{1.903829in}}%
\pgfpathlineto{\pgfqpoint{2.893363in}{1.897930in}}%
\pgfpathmoveto{\pgfqpoint{2.893363in}{1.903829in}}%
\pgfpathlineto{\pgfqpoint{2.893363in}{1.903829in}}%
\pgfpathlineto{\pgfqpoint{2.893363in}{1.909727in}}%
\pgfpathlineto{\pgfqpoint{2.902445in}{1.909727in}}%
\pgfpathlineto{\pgfqpoint{2.902445in}{1.903829in}}%
\pgfpathmoveto{\pgfqpoint{2.893363in}{1.909727in}}%
\pgfpathlineto{\pgfqpoint{2.893363in}{1.909727in}}%
\pgfpathlineto{\pgfqpoint{2.893363in}{1.915626in}}%
\pgfpathlineto{\pgfqpoint{2.902445in}{1.915626in}}%
\pgfpathlineto{\pgfqpoint{2.902445in}{1.909727in}}%
\pgfpathmoveto{\pgfqpoint{2.902445in}{1.909727in}}%
\pgfpathlineto{\pgfqpoint{2.902445in}{1.909727in}}%
\pgfpathlineto{\pgfqpoint{2.902445in}{1.915626in}}%
\pgfpathlineto{\pgfqpoint{2.911527in}{1.915626in}}%
\pgfpathlineto{\pgfqpoint{2.911527in}{1.909727in}}%
\pgfpathmoveto{\pgfqpoint{2.911527in}{1.915626in}}%
\pgfpathlineto{\pgfqpoint{2.911527in}{1.915626in}}%
\pgfpathlineto{\pgfqpoint{2.911527in}{1.921524in}}%
\pgfpathlineto{\pgfqpoint{2.920609in}{1.921524in}}%
\pgfpathlineto{\pgfqpoint{2.920609in}{1.915626in}}%
\pgfpathmoveto{\pgfqpoint{2.911527in}{1.921524in}}%
\pgfpathlineto{\pgfqpoint{2.911527in}{1.921524in}}%
\pgfpathlineto{\pgfqpoint{2.911527in}{1.927422in}}%
\pgfpathlineto{\pgfqpoint{2.920609in}{1.927422in}}%
\pgfpathlineto{\pgfqpoint{2.920609in}{1.921524in}}%
\pgfpathmoveto{\pgfqpoint{2.920609in}{1.921524in}}%
\pgfpathlineto{\pgfqpoint{2.920609in}{1.921524in}}%
\pgfpathlineto{\pgfqpoint{2.920609in}{1.927422in}}%
\pgfpathlineto{\pgfqpoint{2.929691in}{1.927422in}}%
\pgfpathlineto{\pgfqpoint{2.929691in}{1.921524in}}%
\pgfpathmoveto{\pgfqpoint{2.911527in}{3.402031in}}%
\pgfpathlineto{\pgfqpoint{2.911527in}{3.402031in}}%
\pgfpathlineto{\pgfqpoint{2.911527in}{3.407929in}}%
\pgfpathlineto{\pgfqpoint{2.920609in}{3.407929in}}%
\pgfpathlineto{\pgfqpoint{2.920609in}{3.402031in}}%
\pgfpathmoveto{\pgfqpoint{2.911527in}{3.407929in}}%
\pgfpathlineto{\pgfqpoint{2.911527in}{3.407929in}}%
\pgfpathlineto{\pgfqpoint{2.911527in}{3.413828in}}%
\pgfpathlineto{\pgfqpoint{2.920609in}{3.413828in}}%
\pgfpathlineto{\pgfqpoint{2.920609in}{3.407929in}}%
\pgfpathmoveto{\pgfqpoint{2.838870in}{3.496405in}}%
\pgfpathlineto{\pgfqpoint{2.838870in}{3.496405in}}%
\pgfpathlineto{\pgfqpoint{2.838870in}{3.502304in}}%
\pgfpathlineto{\pgfqpoint{2.847952in}{3.502304in}}%
\pgfpathlineto{\pgfqpoint{2.847952in}{3.496405in}}%
\pgfpathmoveto{\pgfqpoint{2.838870in}{3.502304in}}%
\pgfpathlineto{\pgfqpoint{2.838870in}{3.502304in}}%
\pgfpathlineto{\pgfqpoint{2.838870in}{3.508202in}}%
\pgfpathlineto{\pgfqpoint{2.847952in}{3.508202in}}%
\pgfpathlineto{\pgfqpoint{2.847952in}{3.502304in}}%
\pgfpathmoveto{\pgfqpoint{2.875198in}{3.449218in}}%
\pgfpathlineto{\pgfqpoint{2.875198in}{3.449218in}}%
\pgfpathlineto{\pgfqpoint{2.875198in}{3.455117in}}%
\pgfpathlineto{\pgfqpoint{2.884280in}{3.455117in}}%
\pgfpathlineto{\pgfqpoint{2.884280in}{3.449218in}}%
\pgfpathmoveto{\pgfqpoint{2.875198in}{3.455117in}}%
\pgfpathlineto{\pgfqpoint{2.875198in}{3.455117in}}%
\pgfpathlineto{\pgfqpoint{2.875198in}{3.461015in}}%
\pgfpathlineto{\pgfqpoint{2.884280in}{3.461015in}}%
\pgfpathlineto{\pgfqpoint{2.884280in}{3.455117in}}%
\pgfpathmoveto{\pgfqpoint{2.893363in}{3.425625in}}%
\pgfpathlineto{\pgfqpoint{2.893363in}{3.425625in}}%
\pgfpathlineto{\pgfqpoint{2.893363in}{3.431523in}}%
\pgfpathlineto{\pgfqpoint{2.902445in}{3.431523in}}%
\pgfpathlineto{\pgfqpoint{2.902445in}{3.425625in}}%
\pgfpathmoveto{\pgfqpoint{2.893363in}{3.431523in}}%
\pgfpathlineto{\pgfqpoint{2.893363in}{3.431523in}}%
\pgfpathlineto{\pgfqpoint{2.893363in}{3.437421in}}%
\pgfpathlineto{\pgfqpoint{2.902445in}{3.437421in}}%
\pgfpathlineto{\pgfqpoint{2.902445in}{3.431523in}}%
\pgfpathmoveto{\pgfqpoint{2.857034in}{3.472812in}}%
\pgfpathlineto{\pgfqpoint{2.857034in}{3.472812in}}%
\pgfpathlineto{\pgfqpoint{2.857034in}{3.478710in}}%
\pgfpathlineto{\pgfqpoint{2.866116in}{3.478710in}}%
\pgfpathlineto{\pgfqpoint{2.866116in}{3.472812in}}%
\pgfpathmoveto{\pgfqpoint{2.857034in}{3.478710in}}%
\pgfpathlineto{\pgfqpoint{2.857034in}{3.478710in}}%
\pgfpathlineto{\pgfqpoint{2.857034in}{3.484609in}}%
\pgfpathlineto{\pgfqpoint{2.866116in}{3.484609in}}%
\pgfpathlineto{\pgfqpoint{2.866116in}{3.478710in}}%
\pgfpathmoveto{\pgfqpoint{2.929691in}{1.927422in}}%
\pgfpathlineto{\pgfqpoint{2.929691in}{1.927422in}}%
\pgfpathlineto{\pgfqpoint{2.929691in}{1.933320in}}%
\pgfpathlineto{\pgfqpoint{2.938773in}{1.933320in}}%
\pgfpathlineto{\pgfqpoint{2.938773in}{1.927422in}}%
\pgfpathmoveto{\pgfqpoint{2.929691in}{1.933320in}}%
\pgfpathlineto{\pgfqpoint{2.929691in}{1.933320in}}%
\pgfpathlineto{\pgfqpoint{2.929691in}{1.939219in}}%
\pgfpathlineto{\pgfqpoint{2.938773in}{1.939219in}}%
\pgfpathlineto{\pgfqpoint{2.938773in}{1.933320in}}%
\pgfpathmoveto{\pgfqpoint{2.938773in}{1.933320in}}%
\pgfpathlineto{\pgfqpoint{2.938773in}{1.933320in}}%
\pgfpathlineto{\pgfqpoint{2.938773in}{1.939219in}}%
\pgfpathlineto{\pgfqpoint{2.947854in}{1.939219in}}%
\pgfpathlineto{\pgfqpoint{2.947854in}{1.933320in}}%
\pgfpathmoveto{\pgfqpoint{2.947854in}{1.939219in}}%
\pgfpathlineto{\pgfqpoint{2.947854in}{1.939219in}}%
\pgfpathlineto{\pgfqpoint{2.947854in}{1.945117in}}%
\pgfpathlineto{\pgfqpoint{2.956936in}{1.945117in}}%
\pgfpathlineto{\pgfqpoint{2.956936in}{1.939219in}}%
\pgfpathmoveto{\pgfqpoint{2.947854in}{1.945117in}}%
\pgfpathlineto{\pgfqpoint{2.947854in}{1.945117in}}%
\pgfpathlineto{\pgfqpoint{2.947854in}{1.951015in}}%
\pgfpathlineto{\pgfqpoint{2.956936in}{1.951015in}}%
\pgfpathlineto{\pgfqpoint{2.956936in}{1.945117in}}%
\pgfpathmoveto{\pgfqpoint{2.956936in}{1.945117in}}%
\pgfpathlineto{\pgfqpoint{2.956936in}{1.945117in}}%
\pgfpathlineto{\pgfqpoint{2.956936in}{1.951015in}}%
\pgfpathlineto{\pgfqpoint{2.966017in}{1.951015in}}%
\pgfpathlineto{\pgfqpoint{2.966017in}{1.945117in}}%
\pgfpathmoveto{\pgfqpoint{2.966017in}{1.951015in}}%
\pgfpathlineto{\pgfqpoint{2.966017in}{1.951015in}}%
\pgfpathlineto{\pgfqpoint{2.966017in}{1.956913in}}%
\pgfpathlineto{\pgfqpoint{2.975099in}{1.956913in}}%
\pgfpathlineto{\pgfqpoint{2.975099in}{1.951015in}}%
\pgfpathmoveto{\pgfqpoint{2.966017in}{1.956913in}}%
\pgfpathlineto{\pgfqpoint{2.966017in}{1.956913in}}%
\pgfpathlineto{\pgfqpoint{2.966017in}{1.962812in}}%
\pgfpathlineto{\pgfqpoint{2.975099in}{1.962812in}}%
\pgfpathlineto{\pgfqpoint{2.975099in}{1.956913in}}%
\pgfpathmoveto{\pgfqpoint{2.975099in}{1.956913in}}%
\pgfpathlineto{\pgfqpoint{2.975099in}{1.956913in}}%
\pgfpathlineto{\pgfqpoint{2.975099in}{1.962812in}}%
\pgfpathlineto{\pgfqpoint{2.984180in}{1.962812in}}%
\pgfpathlineto{\pgfqpoint{2.984180in}{1.956913in}}%
\pgfpathmoveto{\pgfqpoint{2.984180in}{1.962812in}}%
\pgfpathlineto{\pgfqpoint{2.984180in}{1.962812in}}%
\pgfpathlineto{\pgfqpoint{2.984180in}{1.968710in}}%
\pgfpathlineto{\pgfqpoint{2.993262in}{1.968710in}}%
\pgfpathlineto{\pgfqpoint{2.993262in}{1.962812in}}%
\pgfpathmoveto{\pgfqpoint{2.984180in}{1.968710in}}%
\pgfpathlineto{\pgfqpoint{2.984180in}{1.968710in}}%
\pgfpathlineto{\pgfqpoint{2.984180in}{1.974608in}}%
\pgfpathlineto{\pgfqpoint{2.993262in}{1.974608in}}%
\pgfpathlineto{\pgfqpoint{2.993262in}{1.968710in}}%
\pgfpathmoveto{\pgfqpoint{2.993262in}{1.968710in}}%
\pgfpathlineto{\pgfqpoint{2.993262in}{1.968710in}}%
\pgfpathlineto{\pgfqpoint{2.993262in}{1.974608in}}%
\pgfpathlineto{\pgfqpoint{3.002344in}{1.974608in}}%
\pgfpathlineto{\pgfqpoint{3.002344in}{1.968710in}}%
\pgfpathmoveto{\pgfqpoint{3.002344in}{1.974608in}}%
\pgfpathlineto{\pgfqpoint{3.002344in}{1.974608in}}%
\pgfpathlineto{\pgfqpoint{3.002344in}{1.980506in}}%
\pgfpathlineto{\pgfqpoint{3.011425in}{1.980506in}}%
\pgfpathlineto{\pgfqpoint{3.011425in}{1.974608in}}%
\pgfpathmoveto{\pgfqpoint{3.002344in}{1.980506in}}%
\pgfpathlineto{\pgfqpoint{3.002344in}{1.980506in}}%
\pgfpathlineto{\pgfqpoint{3.002344in}{1.986405in}}%
\pgfpathlineto{\pgfqpoint{3.011425in}{1.986405in}}%
\pgfpathlineto{\pgfqpoint{3.011425in}{1.980506in}}%
\pgfpathmoveto{\pgfqpoint{3.011425in}{1.980506in}}%
\pgfpathlineto{\pgfqpoint{3.011425in}{1.980506in}}%
\pgfpathlineto{\pgfqpoint{3.011425in}{1.986405in}}%
\pgfpathlineto{\pgfqpoint{3.020507in}{1.986405in}}%
\pgfpathlineto{\pgfqpoint{3.020507in}{1.980506in}}%
\pgfpathmoveto{\pgfqpoint{3.020507in}{1.986405in}}%
\pgfpathlineto{\pgfqpoint{3.020507in}{1.986405in}}%
\pgfpathlineto{\pgfqpoint{3.020507in}{1.992303in}}%
\pgfpathlineto{\pgfqpoint{3.029588in}{1.992303in}}%
\pgfpathlineto{\pgfqpoint{3.029588in}{1.986405in}}%
\pgfpathmoveto{\pgfqpoint{3.020507in}{1.992303in}}%
\pgfpathlineto{\pgfqpoint{3.020507in}{1.992303in}}%
\pgfpathlineto{\pgfqpoint{3.020507in}{1.998201in}}%
\pgfpathlineto{\pgfqpoint{3.029588in}{1.998201in}}%
\pgfpathlineto{\pgfqpoint{3.029588in}{1.992303in}}%
\pgfpathmoveto{\pgfqpoint{3.029588in}{1.992303in}}%
\pgfpathlineto{\pgfqpoint{3.029588in}{1.992303in}}%
\pgfpathlineto{\pgfqpoint{3.029588in}{1.998201in}}%
\pgfpathlineto{\pgfqpoint{3.038670in}{1.998201in}}%
\pgfpathlineto{\pgfqpoint{3.038670in}{1.992303in}}%
\pgfpathmoveto{\pgfqpoint{3.038670in}{1.998201in}}%
\pgfpathlineto{\pgfqpoint{3.038670in}{1.998201in}}%
\pgfpathlineto{\pgfqpoint{3.038670in}{2.004099in}}%
\pgfpathlineto{\pgfqpoint{3.047752in}{2.004099in}}%
\pgfpathlineto{\pgfqpoint{3.047752in}{1.998201in}}%
\pgfpathmoveto{\pgfqpoint{3.038670in}{2.004099in}}%
\pgfpathlineto{\pgfqpoint{3.038670in}{2.004099in}}%
\pgfpathlineto{\pgfqpoint{3.038670in}{2.009997in}}%
\pgfpathlineto{\pgfqpoint{3.047752in}{2.009997in}}%
\pgfpathlineto{\pgfqpoint{3.047752in}{2.004099in}}%
\pgfpathmoveto{\pgfqpoint{3.047752in}{2.004099in}}%
\pgfpathlineto{\pgfqpoint{3.047752in}{2.004099in}}%
\pgfpathlineto{\pgfqpoint{3.047752in}{2.009997in}}%
\pgfpathlineto{\pgfqpoint{3.056833in}{2.009997in}}%
\pgfpathlineto{\pgfqpoint{3.056833in}{2.004099in}}%
\pgfpathmoveto{\pgfqpoint{3.056833in}{2.009997in}}%
\pgfpathlineto{\pgfqpoint{3.056833in}{2.009997in}}%
\pgfpathlineto{\pgfqpoint{3.056833in}{2.015896in}}%
\pgfpathlineto{\pgfqpoint{3.065915in}{2.015896in}}%
\pgfpathlineto{\pgfqpoint{3.065915in}{2.009997in}}%
\pgfpathmoveto{\pgfqpoint{3.056833in}{2.015896in}}%
\pgfpathlineto{\pgfqpoint{3.056833in}{2.015896in}}%
\pgfpathlineto{\pgfqpoint{3.056833in}{2.021795in}}%
\pgfpathlineto{\pgfqpoint{3.065915in}{2.021795in}}%
\pgfpathlineto{\pgfqpoint{3.065915in}{2.015896in}}%
\pgfpathmoveto{\pgfqpoint{3.065915in}{2.015896in}}%
\pgfpathlineto{\pgfqpoint{3.065915in}{2.015896in}}%
\pgfpathlineto{\pgfqpoint{3.065915in}{2.021795in}}%
\pgfpathlineto{\pgfqpoint{3.074996in}{2.021795in}}%
\pgfpathlineto{\pgfqpoint{3.074996in}{2.015896in}}%
\pgfpathmoveto{\pgfqpoint{3.056833in}{3.213280in}}%
\pgfpathlineto{\pgfqpoint{3.056833in}{3.213280in}}%
\pgfpathlineto{\pgfqpoint{3.056833in}{3.219178in}}%
\pgfpathlineto{\pgfqpoint{3.065915in}{3.219178in}}%
\pgfpathlineto{\pgfqpoint{3.065915in}{3.213280in}}%
\pgfpathmoveto{\pgfqpoint{3.056833in}{3.219178in}}%
\pgfpathlineto{\pgfqpoint{3.056833in}{3.219178in}}%
\pgfpathlineto{\pgfqpoint{3.056833in}{3.225077in}}%
\pgfpathlineto{\pgfqpoint{3.065915in}{3.225077in}}%
\pgfpathlineto{\pgfqpoint{3.065915in}{3.219178in}}%
\pgfpathmoveto{\pgfqpoint{2.984180in}{3.307655in}}%
\pgfpathlineto{\pgfqpoint{2.984180in}{3.307655in}}%
\pgfpathlineto{\pgfqpoint{2.984180in}{3.313554in}}%
\pgfpathlineto{\pgfqpoint{2.993262in}{3.313554in}}%
\pgfpathlineto{\pgfqpoint{2.993262in}{3.307655in}}%
\pgfpathmoveto{\pgfqpoint{2.984180in}{3.313554in}}%
\pgfpathlineto{\pgfqpoint{2.984180in}{3.313554in}}%
\pgfpathlineto{\pgfqpoint{2.984180in}{3.319452in}}%
\pgfpathlineto{\pgfqpoint{2.993262in}{3.319452in}}%
\pgfpathlineto{\pgfqpoint{2.993262in}{3.313554in}}%
\pgfpathmoveto{\pgfqpoint{3.020507in}{3.260467in}}%
\pgfpathlineto{\pgfqpoint{3.020507in}{3.260467in}}%
\pgfpathlineto{\pgfqpoint{3.020507in}{3.266365in}}%
\pgfpathlineto{\pgfqpoint{3.029588in}{3.266365in}}%
\pgfpathlineto{\pgfqpoint{3.029588in}{3.260467in}}%
\pgfpathmoveto{\pgfqpoint{3.020507in}{3.266365in}}%
\pgfpathlineto{\pgfqpoint{3.020507in}{3.266365in}}%
\pgfpathlineto{\pgfqpoint{3.020507in}{3.272264in}}%
\pgfpathlineto{\pgfqpoint{3.029588in}{3.272264in}}%
\pgfpathlineto{\pgfqpoint{3.029588in}{3.266365in}}%
\pgfpathmoveto{\pgfqpoint{3.038670in}{3.236873in}}%
\pgfpathlineto{\pgfqpoint{3.038670in}{3.236873in}}%
\pgfpathlineto{\pgfqpoint{3.038670in}{3.242771in}}%
\pgfpathlineto{\pgfqpoint{3.047752in}{3.242771in}}%
\pgfpathlineto{\pgfqpoint{3.047752in}{3.236873in}}%
\pgfpathmoveto{\pgfqpoint{3.038670in}{3.242771in}}%
\pgfpathlineto{\pgfqpoint{3.038670in}{3.242771in}}%
\pgfpathlineto{\pgfqpoint{3.038670in}{3.248670in}}%
\pgfpathlineto{\pgfqpoint{3.047752in}{3.248670in}}%
\pgfpathlineto{\pgfqpoint{3.047752in}{3.242771in}}%
\pgfpathmoveto{\pgfqpoint{3.002344in}{3.284061in}}%
\pgfpathlineto{\pgfqpoint{3.002344in}{3.284061in}}%
\pgfpathlineto{\pgfqpoint{3.002344in}{3.289960in}}%
\pgfpathlineto{\pgfqpoint{3.011425in}{3.289960in}}%
\pgfpathlineto{\pgfqpoint{3.011425in}{3.284061in}}%
\pgfpathmoveto{\pgfqpoint{3.002344in}{3.289960in}}%
\pgfpathlineto{\pgfqpoint{3.002344in}{3.289960in}}%
\pgfpathlineto{\pgfqpoint{3.002344in}{3.295858in}}%
\pgfpathlineto{\pgfqpoint{3.011425in}{3.295858in}}%
\pgfpathlineto{\pgfqpoint{3.011425in}{3.289960in}}%
\pgfpathmoveto{\pgfqpoint{2.947854in}{3.354843in}}%
\pgfpathlineto{\pgfqpoint{2.947854in}{3.354843in}}%
\pgfpathlineto{\pgfqpoint{2.947854in}{3.360741in}}%
\pgfpathlineto{\pgfqpoint{2.956936in}{3.360741in}}%
\pgfpathlineto{\pgfqpoint{2.956936in}{3.354843in}}%
\pgfpathmoveto{\pgfqpoint{2.947854in}{3.360741in}}%
\pgfpathlineto{\pgfqpoint{2.947854in}{3.360741in}}%
\pgfpathlineto{\pgfqpoint{2.947854in}{3.366640in}}%
\pgfpathlineto{\pgfqpoint{2.956936in}{3.366640in}}%
\pgfpathlineto{\pgfqpoint{2.956936in}{3.360741in}}%
\pgfpathmoveto{\pgfqpoint{2.966017in}{3.331249in}}%
\pgfpathlineto{\pgfqpoint{2.966017in}{3.331249in}}%
\pgfpathlineto{\pgfqpoint{2.966017in}{3.337148in}}%
\pgfpathlineto{\pgfqpoint{2.975099in}{3.337148in}}%
\pgfpathlineto{\pgfqpoint{2.975099in}{3.331249in}}%
\pgfpathmoveto{\pgfqpoint{2.966017in}{3.337148in}}%
\pgfpathlineto{\pgfqpoint{2.966017in}{3.337148in}}%
\pgfpathlineto{\pgfqpoint{2.966017in}{3.343046in}}%
\pgfpathlineto{\pgfqpoint{2.975099in}{3.343046in}}%
\pgfpathlineto{\pgfqpoint{2.975099in}{3.337148in}}%
\pgfpathmoveto{\pgfqpoint{2.929691in}{3.378437in}}%
\pgfpathlineto{\pgfqpoint{2.929691in}{3.378437in}}%
\pgfpathlineto{\pgfqpoint{2.929691in}{3.384335in}}%
\pgfpathlineto{\pgfqpoint{2.938773in}{3.384335in}}%
\pgfpathlineto{\pgfqpoint{2.938773in}{3.378437in}}%
\pgfpathmoveto{\pgfqpoint{2.929691in}{3.384335in}}%
\pgfpathlineto{\pgfqpoint{2.929691in}{3.384335in}}%
\pgfpathlineto{\pgfqpoint{2.929691in}{3.390234in}}%
\pgfpathlineto{\pgfqpoint{2.938773in}{3.390234in}}%
\pgfpathlineto{\pgfqpoint{2.938773in}{3.384335in}}%
\pgfpathmoveto{\pgfqpoint{3.074996in}{2.021795in}}%
\pgfpathlineto{\pgfqpoint{3.074996in}{2.021795in}}%
\pgfpathlineto{\pgfqpoint{3.074996in}{2.027694in}}%
\pgfpathlineto{\pgfqpoint{3.084079in}{2.027694in}}%
\pgfpathlineto{\pgfqpoint{3.084079in}{2.021795in}}%
\pgfpathmoveto{\pgfqpoint{3.074996in}{2.027694in}}%
\pgfpathlineto{\pgfqpoint{3.074996in}{2.027694in}}%
\pgfpathlineto{\pgfqpoint{3.074996in}{2.033593in}}%
\pgfpathlineto{\pgfqpoint{3.084079in}{2.033593in}}%
\pgfpathlineto{\pgfqpoint{3.084079in}{2.027694in}}%
\pgfpathmoveto{\pgfqpoint{3.084079in}{2.027694in}}%
\pgfpathlineto{\pgfqpoint{3.084079in}{2.027694in}}%
\pgfpathlineto{\pgfqpoint{3.084079in}{2.033593in}}%
\pgfpathlineto{\pgfqpoint{3.093161in}{2.033593in}}%
\pgfpathlineto{\pgfqpoint{3.093161in}{2.027694in}}%
\pgfpathmoveto{\pgfqpoint{3.093161in}{2.027694in}}%
\pgfpathlineto{\pgfqpoint{3.093161in}{2.027694in}}%
\pgfpathlineto{\pgfqpoint{3.093161in}{2.033593in}}%
\pgfpathlineto{\pgfqpoint{3.102244in}{2.033593in}}%
\pgfpathlineto{\pgfqpoint{3.102244in}{2.027694in}}%
\pgfpathmoveto{\pgfqpoint{3.111326in}{2.039491in}}%
\pgfpathlineto{\pgfqpoint{3.111326in}{2.039491in}}%
\pgfpathlineto{\pgfqpoint{3.111326in}{2.045390in}}%
\pgfpathlineto{\pgfqpoint{3.120409in}{2.045390in}}%
\pgfpathlineto{\pgfqpoint{3.120409in}{2.039491in}}%
\pgfpathmoveto{\pgfqpoint{3.129491in}{2.051289in}}%
\pgfpathlineto{\pgfqpoint{3.129491in}{2.051289in}}%
\pgfpathlineto{\pgfqpoint{3.129491in}{2.057188in}}%
\pgfpathlineto{\pgfqpoint{3.138574in}{2.057188in}}%
\pgfpathlineto{\pgfqpoint{3.138574in}{2.051289in}}%
\pgfpathmoveto{\pgfqpoint{3.147656in}{2.063086in}}%
\pgfpathlineto{\pgfqpoint{3.147656in}{2.063086in}}%
\pgfpathlineto{\pgfqpoint{3.147656in}{2.068985in}}%
\pgfpathlineto{\pgfqpoint{3.156739in}{2.068985in}}%
\pgfpathlineto{\pgfqpoint{3.156739in}{2.063086in}}%
\pgfpathmoveto{\pgfqpoint{3.165821in}{2.074884in}}%
\pgfpathlineto{\pgfqpoint{3.165821in}{2.074884in}}%
\pgfpathlineto{\pgfqpoint{3.165821in}{2.080783in}}%
\pgfpathlineto{\pgfqpoint{3.174904in}{2.080783in}}%
\pgfpathlineto{\pgfqpoint{3.174904in}{2.074884in}}%
\pgfpathmoveto{\pgfqpoint{3.183986in}{2.086682in}}%
\pgfpathlineto{\pgfqpoint{3.183986in}{2.086682in}}%
\pgfpathlineto{\pgfqpoint{3.183986in}{2.092580in}}%
\pgfpathlineto{\pgfqpoint{3.193069in}{2.092580in}}%
\pgfpathlineto{\pgfqpoint{3.193069in}{2.086682in}}%
\pgfpathmoveto{\pgfqpoint{3.202151in}{2.098479in}}%
\pgfpathlineto{\pgfqpoint{3.202151in}{2.098479in}}%
\pgfpathlineto{\pgfqpoint{3.202151in}{2.104378in}}%
\pgfpathlineto{\pgfqpoint{3.211234in}{2.104378in}}%
\pgfpathlineto{\pgfqpoint{3.211234in}{2.098479in}}%
\pgfpathmoveto{\pgfqpoint{3.202151in}{3.024531in}}%
\pgfpathlineto{\pgfqpoint{3.202151in}{3.024531in}}%
\pgfpathlineto{\pgfqpoint{3.202151in}{3.030430in}}%
\pgfpathlineto{\pgfqpoint{3.211234in}{3.030430in}}%
\pgfpathlineto{\pgfqpoint{3.211234in}{3.024531in}}%
\pgfpathmoveto{\pgfqpoint{3.202151in}{3.030430in}}%
\pgfpathlineto{\pgfqpoint{3.202151in}{3.030430in}}%
\pgfpathlineto{\pgfqpoint{3.202151in}{3.036329in}}%
\pgfpathlineto{\pgfqpoint{3.211234in}{3.036329in}}%
\pgfpathlineto{\pgfqpoint{3.211234in}{3.030430in}}%
\pgfpathmoveto{\pgfqpoint{3.129491in}{3.118907in}}%
\pgfpathlineto{\pgfqpoint{3.129491in}{3.118907in}}%
\pgfpathlineto{\pgfqpoint{3.129491in}{3.124806in}}%
\pgfpathlineto{\pgfqpoint{3.138574in}{3.124806in}}%
\pgfpathlineto{\pgfqpoint{3.138574in}{3.118907in}}%
\pgfpathmoveto{\pgfqpoint{3.129491in}{3.124806in}}%
\pgfpathlineto{\pgfqpoint{3.129491in}{3.124806in}}%
\pgfpathlineto{\pgfqpoint{3.129491in}{3.130704in}}%
\pgfpathlineto{\pgfqpoint{3.138574in}{3.130704in}}%
\pgfpathlineto{\pgfqpoint{3.138574in}{3.124806in}}%
\pgfpathmoveto{\pgfqpoint{3.165821in}{3.071720in}}%
\pgfpathlineto{\pgfqpoint{3.165821in}{3.071720in}}%
\pgfpathlineto{\pgfqpoint{3.165821in}{3.077618in}}%
\pgfpathlineto{\pgfqpoint{3.174904in}{3.077618in}}%
\pgfpathlineto{\pgfqpoint{3.174904in}{3.071720in}}%
\pgfpathmoveto{\pgfqpoint{3.165821in}{3.077618in}}%
\pgfpathlineto{\pgfqpoint{3.165821in}{3.077618in}}%
\pgfpathlineto{\pgfqpoint{3.165821in}{3.083517in}}%
\pgfpathlineto{\pgfqpoint{3.174904in}{3.083517in}}%
\pgfpathlineto{\pgfqpoint{3.174904in}{3.077618in}}%
\pgfpathmoveto{\pgfqpoint{3.183986in}{3.048126in}}%
\pgfpathlineto{\pgfqpoint{3.183986in}{3.048126in}}%
\pgfpathlineto{\pgfqpoint{3.183986in}{3.054024in}}%
\pgfpathlineto{\pgfqpoint{3.193069in}{3.054024in}}%
\pgfpathlineto{\pgfqpoint{3.193069in}{3.048126in}}%
\pgfpathmoveto{\pgfqpoint{3.183986in}{3.054024in}}%
\pgfpathlineto{\pgfqpoint{3.183986in}{3.054024in}}%
\pgfpathlineto{\pgfqpoint{3.183986in}{3.059923in}}%
\pgfpathlineto{\pgfqpoint{3.193069in}{3.059923in}}%
\pgfpathlineto{\pgfqpoint{3.193069in}{3.054024in}}%
\pgfpathmoveto{\pgfqpoint{3.147656in}{3.095314in}}%
\pgfpathlineto{\pgfqpoint{3.147656in}{3.095314in}}%
\pgfpathlineto{\pgfqpoint{3.147656in}{3.101212in}}%
\pgfpathlineto{\pgfqpoint{3.156739in}{3.101212in}}%
\pgfpathlineto{\pgfqpoint{3.156739in}{3.095314in}}%
\pgfpathmoveto{\pgfqpoint{3.147656in}{3.101212in}}%
\pgfpathlineto{\pgfqpoint{3.147656in}{3.101212in}}%
\pgfpathlineto{\pgfqpoint{3.147656in}{3.107110in}}%
\pgfpathlineto{\pgfqpoint{3.156739in}{3.107110in}}%
\pgfpathlineto{\pgfqpoint{3.156739in}{3.101212in}}%
\pgfpathmoveto{\pgfqpoint{3.093161in}{3.166094in}}%
\pgfpathlineto{\pgfqpoint{3.093161in}{3.166094in}}%
\pgfpathlineto{\pgfqpoint{3.093161in}{3.171992in}}%
\pgfpathlineto{\pgfqpoint{3.102244in}{3.171992in}}%
\pgfpathlineto{\pgfqpoint{3.102244in}{3.166094in}}%
\pgfpathmoveto{\pgfqpoint{3.093161in}{3.171992in}}%
\pgfpathlineto{\pgfqpoint{3.093161in}{3.171992in}}%
\pgfpathlineto{\pgfqpoint{3.093161in}{3.177891in}}%
\pgfpathlineto{\pgfqpoint{3.102244in}{3.177891in}}%
\pgfpathlineto{\pgfqpoint{3.102244in}{3.171992in}}%
\pgfpathmoveto{\pgfqpoint{3.111326in}{3.142501in}}%
\pgfpathlineto{\pgfqpoint{3.111326in}{3.142501in}}%
\pgfpathlineto{\pgfqpoint{3.111326in}{3.148399in}}%
\pgfpathlineto{\pgfqpoint{3.120409in}{3.148399in}}%
\pgfpathlineto{\pgfqpoint{3.120409in}{3.142501in}}%
\pgfpathmoveto{\pgfqpoint{3.111326in}{3.148399in}}%
\pgfpathlineto{\pgfqpoint{3.111326in}{3.148399in}}%
\pgfpathlineto{\pgfqpoint{3.111326in}{3.154298in}}%
\pgfpathlineto{\pgfqpoint{3.120409in}{3.154298in}}%
\pgfpathlineto{\pgfqpoint{3.120409in}{3.148399in}}%
\pgfpathmoveto{\pgfqpoint{3.074996in}{3.189687in}}%
\pgfpathlineto{\pgfqpoint{3.074996in}{3.189687in}}%
\pgfpathlineto{\pgfqpoint{3.074996in}{3.195585in}}%
\pgfpathlineto{\pgfqpoint{3.084079in}{3.195585in}}%
\pgfpathlineto{\pgfqpoint{3.084079in}{3.189687in}}%
\pgfpathmoveto{\pgfqpoint{3.074996in}{3.195585in}}%
\pgfpathlineto{\pgfqpoint{3.074996in}{3.195585in}}%
\pgfpathlineto{\pgfqpoint{3.074996in}{3.201484in}}%
\pgfpathlineto{\pgfqpoint{3.084079in}{3.201484in}}%
\pgfpathlineto{\pgfqpoint{3.084079in}{3.195585in}}%
\pgfpathmoveto{\pgfqpoint{3.220316in}{2.110276in}}%
\pgfpathlineto{\pgfqpoint{3.220316in}{2.110276in}}%
\pgfpathlineto{\pgfqpoint{3.220316in}{2.116175in}}%
\pgfpathlineto{\pgfqpoint{3.229398in}{2.116175in}}%
\pgfpathlineto{\pgfqpoint{3.229398in}{2.110276in}}%
\pgfpathmoveto{\pgfqpoint{3.238480in}{2.122073in}}%
\pgfpathlineto{\pgfqpoint{3.238480in}{2.122073in}}%
\pgfpathlineto{\pgfqpoint{3.238480in}{2.127971in}}%
\pgfpathlineto{\pgfqpoint{3.247561in}{2.127971in}}%
\pgfpathlineto{\pgfqpoint{3.247561in}{2.122073in}}%
\pgfpathmoveto{\pgfqpoint{3.256643in}{2.133870in}}%
\pgfpathlineto{\pgfqpoint{3.256643in}{2.133870in}}%
\pgfpathlineto{\pgfqpoint{3.256643in}{2.139768in}}%
\pgfpathlineto{\pgfqpoint{3.265724in}{2.139768in}}%
\pgfpathlineto{\pgfqpoint{3.265724in}{2.133870in}}%
\pgfpathmoveto{\pgfqpoint{3.274806in}{2.145667in}}%
\pgfpathlineto{\pgfqpoint{3.274806in}{2.145667in}}%
\pgfpathlineto{\pgfqpoint{3.274806in}{2.151565in}}%
\pgfpathlineto{\pgfqpoint{3.283888in}{2.151565in}}%
\pgfpathlineto{\pgfqpoint{3.283888in}{2.145667in}}%
\pgfpathmoveto{\pgfqpoint{3.292969in}{2.157463in}}%
\pgfpathlineto{\pgfqpoint{3.292969in}{2.157463in}}%
\pgfpathlineto{\pgfqpoint{3.292969in}{2.163362in}}%
\pgfpathlineto{\pgfqpoint{3.302051in}{2.163362in}}%
\pgfpathlineto{\pgfqpoint{3.302051in}{2.157463in}}%
\pgfpathmoveto{\pgfqpoint{3.311132in}{2.169260in}}%
\pgfpathlineto{\pgfqpoint{3.311132in}{2.169260in}}%
\pgfpathlineto{\pgfqpoint{3.311132in}{2.175158in}}%
\pgfpathlineto{\pgfqpoint{3.320214in}{2.175158in}}%
\pgfpathlineto{\pgfqpoint{3.320214in}{2.169260in}}%
\pgfpathmoveto{\pgfqpoint{3.329296in}{2.181057in}}%
\pgfpathlineto{\pgfqpoint{3.329296in}{2.181057in}}%
\pgfpathlineto{\pgfqpoint{3.329296in}{2.186955in}}%
\pgfpathlineto{\pgfqpoint{3.338377in}{2.186955in}}%
\pgfpathlineto{\pgfqpoint{3.338377in}{2.181057in}}%
\pgfpathmoveto{\pgfqpoint{3.347459in}{2.192854in}}%
\pgfpathlineto{\pgfqpoint{3.347459in}{2.192854in}}%
\pgfpathlineto{\pgfqpoint{3.347459in}{2.198752in}}%
\pgfpathlineto{\pgfqpoint{3.356540in}{2.198752in}}%
\pgfpathlineto{\pgfqpoint{3.356540in}{2.192854in}}%
\pgfpathmoveto{\pgfqpoint{3.347459in}{2.835782in}}%
\pgfpathlineto{\pgfqpoint{3.347459in}{2.835782in}}%
\pgfpathlineto{\pgfqpoint{3.347459in}{2.841681in}}%
\pgfpathlineto{\pgfqpoint{3.356540in}{2.841681in}}%
\pgfpathlineto{\pgfqpoint{3.356540in}{2.835782in}}%
\pgfpathmoveto{\pgfqpoint{3.347459in}{2.841681in}}%
\pgfpathlineto{\pgfqpoint{3.347459in}{2.841681in}}%
\pgfpathlineto{\pgfqpoint{3.347459in}{2.847579in}}%
\pgfpathlineto{\pgfqpoint{3.356540in}{2.847579in}}%
\pgfpathlineto{\pgfqpoint{3.356540in}{2.841681in}}%
\pgfpathmoveto{\pgfqpoint{3.274806in}{2.930155in}}%
\pgfpathlineto{\pgfqpoint{3.274806in}{2.930155in}}%
\pgfpathlineto{\pgfqpoint{3.274806in}{2.936053in}}%
\pgfpathlineto{\pgfqpoint{3.283888in}{2.936053in}}%
\pgfpathlineto{\pgfqpoint{3.283888in}{2.930155in}}%
\pgfpathmoveto{\pgfqpoint{3.274806in}{2.936053in}}%
\pgfpathlineto{\pgfqpoint{3.274806in}{2.936053in}}%
\pgfpathlineto{\pgfqpoint{3.274806in}{2.941952in}}%
\pgfpathlineto{\pgfqpoint{3.283888in}{2.941952in}}%
\pgfpathlineto{\pgfqpoint{3.283888in}{2.936053in}}%
\pgfpathmoveto{\pgfqpoint{3.311132in}{2.882969in}}%
\pgfpathlineto{\pgfqpoint{3.311132in}{2.882969in}}%
\pgfpathlineto{\pgfqpoint{3.311132in}{2.888867in}}%
\pgfpathlineto{\pgfqpoint{3.320214in}{2.888867in}}%
\pgfpathlineto{\pgfqpoint{3.320214in}{2.882969in}}%
\pgfpathmoveto{\pgfqpoint{3.311132in}{2.888867in}}%
\pgfpathlineto{\pgfqpoint{3.311132in}{2.888867in}}%
\pgfpathlineto{\pgfqpoint{3.311132in}{2.894765in}}%
\pgfpathlineto{\pgfqpoint{3.320214in}{2.894765in}}%
\pgfpathlineto{\pgfqpoint{3.320214in}{2.888867in}}%
\pgfpathmoveto{\pgfqpoint{3.329296in}{2.859376in}}%
\pgfpathlineto{\pgfqpoint{3.329296in}{2.859376in}}%
\pgfpathlineto{\pgfqpoint{3.329296in}{2.865274in}}%
\pgfpathlineto{\pgfqpoint{3.338377in}{2.865274in}}%
\pgfpathlineto{\pgfqpoint{3.338377in}{2.859376in}}%
\pgfpathmoveto{\pgfqpoint{3.329296in}{2.865274in}}%
\pgfpathlineto{\pgfqpoint{3.329296in}{2.865274in}}%
\pgfpathlineto{\pgfqpoint{3.329296in}{2.871172in}}%
\pgfpathlineto{\pgfqpoint{3.338377in}{2.871172in}}%
\pgfpathlineto{\pgfqpoint{3.338377in}{2.865274in}}%
\pgfpathmoveto{\pgfqpoint{3.292969in}{2.906562in}}%
\pgfpathlineto{\pgfqpoint{3.292969in}{2.906562in}}%
\pgfpathlineto{\pgfqpoint{3.292969in}{2.912460in}}%
\pgfpathlineto{\pgfqpoint{3.302051in}{2.912460in}}%
\pgfpathlineto{\pgfqpoint{3.302051in}{2.906562in}}%
\pgfpathmoveto{\pgfqpoint{3.292969in}{2.912460in}}%
\pgfpathlineto{\pgfqpoint{3.292969in}{2.912460in}}%
\pgfpathlineto{\pgfqpoint{3.292969in}{2.918359in}}%
\pgfpathlineto{\pgfqpoint{3.302051in}{2.918359in}}%
\pgfpathlineto{\pgfqpoint{3.302051in}{2.912460in}}%
\pgfpathmoveto{\pgfqpoint{3.238480in}{2.977343in}}%
\pgfpathlineto{\pgfqpoint{3.238480in}{2.977343in}}%
\pgfpathlineto{\pgfqpoint{3.238480in}{2.983241in}}%
\pgfpathlineto{\pgfqpoint{3.247561in}{2.983241in}}%
\pgfpathlineto{\pgfqpoint{3.247561in}{2.977343in}}%
\pgfpathmoveto{\pgfqpoint{3.238480in}{2.983241in}}%
\pgfpathlineto{\pgfqpoint{3.238480in}{2.983241in}}%
\pgfpathlineto{\pgfqpoint{3.238480in}{2.989140in}}%
\pgfpathlineto{\pgfqpoint{3.247561in}{2.989140in}}%
\pgfpathlineto{\pgfqpoint{3.247561in}{2.983241in}}%
\pgfpathmoveto{\pgfqpoint{3.256643in}{2.953748in}}%
\pgfpathlineto{\pgfqpoint{3.256643in}{2.953748in}}%
\pgfpathlineto{\pgfqpoint{3.256643in}{2.959647in}}%
\pgfpathlineto{\pgfqpoint{3.265724in}{2.959647in}}%
\pgfpathlineto{\pgfqpoint{3.265724in}{2.953748in}}%
\pgfpathmoveto{\pgfqpoint{3.256643in}{2.959647in}}%
\pgfpathlineto{\pgfqpoint{3.256643in}{2.959647in}}%
\pgfpathlineto{\pgfqpoint{3.256643in}{2.965545in}}%
\pgfpathlineto{\pgfqpoint{3.265724in}{2.965545in}}%
\pgfpathlineto{\pgfqpoint{3.265724in}{2.959647in}}%
\pgfpathmoveto{\pgfqpoint{3.220316in}{3.000937in}}%
\pgfpathlineto{\pgfqpoint{3.220316in}{3.000937in}}%
\pgfpathlineto{\pgfqpoint{3.220316in}{3.006836in}}%
\pgfpathlineto{\pgfqpoint{3.229398in}{3.006836in}}%
\pgfpathlineto{\pgfqpoint{3.229398in}{3.000937in}}%
\pgfpathmoveto{\pgfqpoint{3.220316in}{3.006836in}}%
\pgfpathlineto{\pgfqpoint{3.220316in}{3.006836in}}%
\pgfpathlineto{\pgfqpoint{3.220316in}{3.012734in}}%
\pgfpathlineto{\pgfqpoint{3.229398in}{3.012734in}}%
\pgfpathlineto{\pgfqpoint{3.229398in}{3.006836in}}%
\pgfpathmoveto{\pgfqpoint{3.365622in}{2.204650in}}%
\pgfpathlineto{\pgfqpoint{3.365622in}{2.204650in}}%
\pgfpathlineto{\pgfqpoint{3.365622in}{2.210548in}}%
\pgfpathlineto{\pgfqpoint{3.374704in}{2.210548in}}%
\pgfpathlineto{\pgfqpoint{3.374704in}{2.204650in}}%
\pgfpathmoveto{\pgfqpoint{3.383787in}{2.216446in}}%
\pgfpathlineto{\pgfqpoint{3.383787in}{2.216446in}}%
\pgfpathlineto{\pgfqpoint{3.383787in}{2.222345in}}%
\pgfpathlineto{\pgfqpoint{3.392869in}{2.222345in}}%
\pgfpathlineto{\pgfqpoint{3.392869in}{2.216446in}}%
\pgfpathmoveto{\pgfqpoint{3.401952in}{2.228243in}}%
\pgfpathlineto{\pgfqpoint{3.401952in}{2.228243in}}%
\pgfpathlineto{\pgfqpoint{3.401952in}{2.234141in}}%
\pgfpathlineto{\pgfqpoint{3.411034in}{2.234141in}}%
\pgfpathlineto{\pgfqpoint{3.411034in}{2.228243in}}%
\pgfpathmoveto{\pgfqpoint{3.420116in}{2.240039in}}%
\pgfpathlineto{\pgfqpoint{3.420116in}{2.240039in}}%
\pgfpathlineto{\pgfqpoint{3.420116in}{2.245937in}}%
\pgfpathlineto{\pgfqpoint{3.429199in}{2.245937in}}%
\pgfpathlineto{\pgfqpoint{3.429199in}{2.240039in}}%
\pgfpathmoveto{\pgfqpoint{3.420116in}{2.245937in}}%
\pgfpathlineto{\pgfqpoint{3.420116in}{2.245937in}}%
\pgfpathlineto{\pgfqpoint{3.420116in}{2.251835in}}%
\pgfpathlineto{\pgfqpoint{3.429199in}{2.251835in}}%
\pgfpathlineto{\pgfqpoint{3.429199in}{2.245937in}}%
\pgfpathmoveto{\pgfqpoint{3.420116in}{2.251835in}}%
\pgfpathlineto{\pgfqpoint{3.420116in}{2.251835in}}%
\pgfpathlineto{\pgfqpoint{3.420116in}{2.257734in}}%
\pgfpathlineto{\pgfqpoint{3.429199in}{2.257734in}}%
\pgfpathlineto{\pgfqpoint{3.429199in}{2.251835in}}%
\pgfpathmoveto{\pgfqpoint{3.429199in}{2.251835in}}%
\pgfpathlineto{\pgfqpoint{3.429199in}{2.251835in}}%
\pgfpathlineto{\pgfqpoint{3.429199in}{2.257734in}}%
\pgfpathlineto{\pgfqpoint{3.438281in}{2.257734in}}%
\pgfpathlineto{\pgfqpoint{3.438281in}{2.251835in}}%
\pgfpathmoveto{\pgfqpoint{3.438281in}{2.257734in}}%
\pgfpathlineto{\pgfqpoint{3.438281in}{2.257734in}}%
\pgfpathlineto{\pgfqpoint{3.438281in}{2.263632in}}%
\pgfpathlineto{\pgfqpoint{3.447364in}{2.263632in}}%
\pgfpathlineto{\pgfqpoint{3.447364in}{2.257734in}}%
\pgfpathmoveto{\pgfqpoint{3.438281in}{2.263632in}}%
\pgfpathlineto{\pgfqpoint{3.438281in}{2.263632in}}%
\pgfpathlineto{\pgfqpoint{3.438281in}{2.269530in}}%
\pgfpathlineto{\pgfqpoint{3.447364in}{2.269530in}}%
\pgfpathlineto{\pgfqpoint{3.447364in}{2.263632in}}%
\pgfpathmoveto{\pgfqpoint{3.447364in}{2.263632in}}%
\pgfpathlineto{\pgfqpoint{3.447364in}{2.263632in}}%
\pgfpathlineto{\pgfqpoint{3.447364in}{2.269530in}}%
\pgfpathlineto{\pgfqpoint{3.456446in}{2.269530in}}%
\pgfpathlineto{\pgfqpoint{3.456446in}{2.263632in}}%
\pgfpathmoveto{\pgfqpoint{3.456446in}{2.269530in}}%
\pgfpathlineto{\pgfqpoint{3.456446in}{2.269530in}}%
\pgfpathlineto{\pgfqpoint{3.456446in}{2.275428in}}%
\pgfpathlineto{\pgfqpoint{3.465529in}{2.275428in}}%
\pgfpathlineto{\pgfqpoint{3.465529in}{2.269530in}}%
\pgfpathmoveto{\pgfqpoint{3.456446in}{2.275428in}}%
\pgfpathlineto{\pgfqpoint{3.456446in}{2.275428in}}%
\pgfpathlineto{\pgfqpoint{3.456446in}{2.281326in}}%
\pgfpathlineto{\pgfqpoint{3.465529in}{2.281326in}}%
\pgfpathlineto{\pgfqpoint{3.465529in}{2.275428in}}%
\pgfpathmoveto{\pgfqpoint{3.465529in}{2.275428in}}%
\pgfpathlineto{\pgfqpoint{3.465529in}{2.275428in}}%
\pgfpathlineto{\pgfqpoint{3.465529in}{2.281326in}}%
\pgfpathlineto{\pgfqpoint{3.474611in}{2.281326in}}%
\pgfpathlineto{\pgfqpoint{3.474611in}{2.275428in}}%
\pgfpathmoveto{\pgfqpoint{3.474611in}{2.281326in}}%
\pgfpathlineto{\pgfqpoint{3.474611in}{2.281326in}}%
\pgfpathlineto{\pgfqpoint{3.474611in}{2.287224in}}%
\pgfpathlineto{\pgfqpoint{3.483693in}{2.287224in}}%
\pgfpathlineto{\pgfqpoint{3.483693in}{2.281326in}}%
\pgfpathmoveto{\pgfqpoint{3.474611in}{2.287224in}}%
\pgfpathlineto{\pgfqpoint{3.474611in}{2.287224in}}%
\pgfpathlineto{\pgfqpoint{3.474611in}{2.293123in}}%
\pgfpathlineto{\pgfqpoint{3.483693in}{2.293123in}}%
\pgfpathlineto{\pgfqpoint{3.483693in}{2.287224in}}%
\pgfpathmoveto{\pgfqpoint{3.483693in}{2.287224in}}%
\pgfpathlineto{\pgfqpoint{3.483693in}{2.287224in}}%
\pgfpathlineto{\pgfqpoint{3.483693in}{2.293123in}}%
\pgfpathlineto{\pgfqpoint{3.492776in}{2.293123in}}%
\pgfpathlineto{\pgfqpoint{3.492776in}{2.287224in}}%
\pgfpathmoveto{\pgfqpoint{3.492776in}{2.293123in}}%
\pgfpathlineto{\pgfqpoint{3.492776in}{2.293123in}}%
\pgfpathlineto{\pgfqpoint{3.492776in}{2.299021in}}%
\pgfpathlineto{\pgfqpoint{3.501858in}{2.299021in}}%
\pgfpathlineto{\pgfqpoint{3.501858in}{2.293123in}}%
\pgfpathmoveto{\pgfqpoint{3.492776in}{2.299021in}}%
\pgfpathlineto{\pgfqpoint{3.492776in}{2.299021in}}%
\pgfpathlineto{\pgfqpoint{3.492776in}{2.304919in}}%
\pgfpathlineto{\pgfqpoint{3.501858in}{2.304919in}}%
\pgfpathlineto{\pgfqpoint{3.501858in}{2.299021in}}%
\pgfpathmoveto{\pgfqpoint{3.501858in}{2.299021in}}%
\pgfpathlineto{\pgfqpoint{3.501858in}{2.299021in}}%
\pgfpathlineto{\pgfqpoint{3.501858in}{2.304919in}}%
\pgfpathlineto{\pgfqpoint{3.510941in}{2.304919in}}%
\pgfpathlineto{\pgfqpoint{3.510941in}{2.299021in}}%
\pgfpathmoveto{\pgfqpoint{3.492776in}{2.647031in}}%
\pgfpathlineto{\pgfqpoint{3.492776in}{2.647031in}}%
\pgfpathlineto{\pgfqpoint{3.492776in}{2.652930in}}%
\pgfpathlineto{\pgfqpoint{3.501858in}{2.652930in}}%
\pgfpathlineto{\pgfqpoint{3.501858in}{2.647031in}}%
\pgfpathmoveto{\pgfqpoint{3.492776in}{2.652930in}}%
\pgfpathlineto{\pgfqpoint{3.492776in}{2.652930in}}%
\pgfpathlineto{\pgfqpoint{3.492776in}{2.658828in}}%
\pgfpathlineto{\pgfqpoint{3.501858in}{2.658828in}}%
\pgfpathlineto{\pgfqpoint{3.501858in}{2.652930in}}%
\pgfpathmoveto{\pgfqpoint{3.420116in}{2.741408in}}%
\pgfpathlineto{\pgfqpoint{3.420116in}{2.741408in}}%
\pgfpathlineto{\pgfqpoint{3.420116in}{2.747306in}}%
\pgfpathlineto{\pgfqpoint{3.429199in}{2.747306in}}%
\pgfpathlineto{\pgfqpoint{3.429199in}{2.741408in}}%
\pgfpathmoveto{\pgfqpoint{3.420116in}{2.747306in}}%
\pgfpathlineto{\pgfqpoint{3.420116in}{2.747306in}}%
\pgfpathlineto{\pgfqpoint{3.420116in}{2.753205in}}%
\pgfpathlineto{\pgfqpoint{3.429199in}{2.753205in}}%
\pgfpathlineto{\pgfqpoint{3.429199in}{2.747306in}}%
\pgfpathmoveto{\pgfqpoint{3.456446in}{2.694219in}}%
\pgfpathlineto{\pgfqpoint{3.456446in}{2.694219in}}%
\pgfpathlineto{\pgfqpoint{3.456446in}{2.700118in}}%
\pgfpathlineto{\pgfqpoint{3.465529in}{2.700118in}}%
\pgfpathlineto{\pgfqpoint{3.465529in}{2.694219in}}%
\pgfpathmoveto{\pgfqpoint{3.456446in}{2.700118in}}%
\pgfpathlineto{\pgfqpoint{3.456446in}{2.700118in}}%
\pgfpathlineto{\pgfqpoint{3.456446in}{2.706016in}}%
\pgfpathlineto{\pgfqpoint{3.465529in}{2.706016in}}%
\pgfpathlineto{\pgfqpoint{3.465529in}{2.700118in}}%
\pgfpathmoveto{\pgfqpoint{3.474611in}{2.670624in}}%
\pgfpathlineto{\pgfqpoint{3.474611in}{2.670624in}}%
\pgfpathlineto{\pgfqpoint{3.474611in}{2.676523in}}%
\pgfpathlineto{\pgfqpoint{3.483693in}{2.676523in}}%
\pgfpathlineto{\pgfqpoint{3.483693in}{2.670624in}}%
\pgfpathmoveto{\pgfqpoint{3.474611in}{2.676523in}}%
\pgfpathlineto{\pgfqpoint{3.474611in}{2.676523in}}%
\pgfpathlineto{\pgfqpoint{3.474611in}{2.682422in}}%
\pgfpathlineto{\pgfqpoint{3.483693in}{2.682422in}}%
\pgfpathlineto{\pgfqpoint{3.483693in}{2.676523in}}%
\pgfpathmoveto{\pgfqpoint{3.438281in}{2.717813in}}%
\pgfpathlineto{\pgfqpoint{3.438281in}{2.717813in}}%
\pgfpathlineto{\pgfqpoint{3.438281in}{2.723712in}}%
\pgfpathlineto{\pgfqpoint{3.447364in}{2.723712in}}%
\pgfpathlineto{\pgfqpoint{3.447364in}{2.717813in}}%
\pgfpathmoveto{\pgfqpoint{3.438281in}{2.723712in}}%
\pgfpathlineto{\pgfqpoint{3.438281in}{2.723712in}}%
\pgfpathlineto{\pgfqpoint{3.438281in}{2.729611in}}%
\pgfpathlineto{\pgfqpoint{3.447364in}{2.729611in}}%
\pgfpathlineto{\pgfqpoint{3.447364in}{2.723712in}}%
\pgfpathmoveto{\pgfqpoint{3.383787in}{2.788596in}}%
\pgfpathlineto{\pgfqpoint{3.383787in}{2.788596in}}%
\pgfpathlineto{\pgfqpoint{3.383787in}{2.794494in}}%
\pgfpathlineto{\pgfqpoint{3.392869in}{2.794494in}}%
\pgfpathlineto{\pgfqpoint{3.392869in}{2.788596in}}%
\pgfpathmoveto{\pgfqpoint{3.383787in}{2.794494in}}%
\pgfpathlineto{\pgfqpoint{3.383787in}{2.794494in}}%
\pgfpathlineto{\pgfqpoint{3.383787in}{2.800392in}}%
\pgfpathlineto{\pgfqpoint{3.392869in}{2.800392in}}%
\pgfpathlineto{\pgfqpoint{3.392869in}{2.794494in}}%
\pgfpathmoveto{\pgfqpoint{3.401952in}{2.765002in}}%
\pgfpathlineto{\pgfqpoint{3.401952in}{2.765002in}}%
\pgfpathlineto{\pgfqpoint{3.401952in}{2.770901in}}%
\pgfpathlineto{\pgfqpoint{3.411034in}{2.770901in}}%
\pgfpathlineto{\pgfqpoint{3.411034in}{2.765002in}}%
\pgfpathmoveto{\pgfqpoint{3.401952in}{2.770901in}}%
\pgfpathlineto{\pgfqpoint{3.401952in}{2.770901in}}%
\pgfpathlineto{\pgfqpoint{3.401952in}{2.776799in}}%
\pgfpathlineto{\pgfqpoint{3.411034in}{2.776799in}}%
\pgfpathlineto{\pgfqpoint{3.411034in}{2.770901in}}%
\pgfpathmoveto{\pgfqpoint{3.365622in}{2.812189in}}%
\pgfpathlineto{\pgfqpoint{3.365622in}{2.812189in}}%
\pgfpathlineto{\pgfqpoint{3.365622in}{2.818087in}}%
\pgfpathlineto{\pgfqpoint{3.374704in}{2.818087in}}%
\pgfpathlineto{\pgfqpoint{3.374704in}{2.812189in}}%
\pgfpathmoveto{\pgfqpoint{3.365622in}{2.818087in}}%
\pgfpathlineto{\pgfqpoint{3.365622in}{2.818087in}}%
\pgfpathlineto{\pgfqpoint{3.365622in}{2.823986in}}%
\pgfpathlineto{\pgfqpoint{3.374704in}{2.823986in}}%
\pgfpathlineto{\pgfqpoint{3.374704in}{2.818087in}}%
\pgfpathmoveto{\pgfqpoint{3.510941in}{2.304919in}}%
\pgfpathlineto{\pgfqpoint{3.510941in}{2.304919in}}%
\pgfpathlineto{\pgfqpoint{3.510941in}{2.310818in}}%
\pgfpathlineto{\pgfqpoint{3.520022in}{2.310818in}}%
\pgfpathlineto{\pgfqpoint{3.520022in}{2.304919in}}%
\pgfpathmoveto{\pgfqpoint{3.510941in}{2.310818in}}%
\pgfpathlineto{\pgfqpoint{3.510941in}{2.310818in}}%
\pgfpathlineto{\pgfqpoint{3.510941in}{2.316716in}}%
\pgfpathlineto{\pgfqpoint{3.520022in}{2.316716in}}%
\pgfpathlineto{\pgfqpoint{3.520022in}{2.310818in}}%
\pgfpathmoveto{\pgfqpoint{3.520022in}{2.310818in}}%
\pgfpathlineto{\pgfqpoint{3.520022in}{2.310818in}}%
\pgfpathlineto{\pgfqpoint{3.520022in}{2.316716in}}%
\pgfpathlineto{\pgfqpoint{3.529104in}{2.316716in}}%
\pgfpathlineto{\pgfqpoint{3.529104in}{2.310818in}}%
\pgfpathmoveto{\pgfqpoint{3.529104in}{2.316716in}}%
\pgfpathlineto{\pgfqpoint{3.529104in}{2.316716in}}%
\pgfpathlineto{\pgfqpoint{3.529104in}{2.322615in}}%
\pgfpathlineto{\pgfqpoint{3.538185in}{2.322615in}}%
\pgfpathlineto{\pgfqpoint{3.538185in}{2.316716in}}%
\pgfpathmoveto{\pgfqpoint{3.529104in}{2.322615in}}%
\pgfpathlineto{\pgfqpoint{3.529104in}{2.322615in}}%
\pgfpathlineto{\pgfqpoint{3.529104in}{2.328513in}}%
\pgfpathlineto{\pgfqpoint{3.538185in}{2.328513in}}%
\pgfpathlineto{\pgfqpoint{3.538185in}{2.322615in}}%
\pgfpathmoveto{\pgfqpoint{3.538185in}{2.322615in}}%
\pgfpathlineto{\pgfqpoint{3.538185in}{2.322615in}}%
\pgfpathlineto{\pgfqpoint{3.538185in}{2.328513in}}%
\pgfpathlineto{\pgfqpoint{3.547267in}{2.328513in}}%
\pgfpathlineto{\pgfqpoint{3.547267in}{2.322615in}}%
\pgfpathmoveto{\pgfqpoint{3.547267in}{2.328513in}}%
\pgfpathlineto{\pgfqpoint{3.547267in}{2.328513in}}%
\pgfpathlineto{\pgfqpoint{3.547267in}{2.334412in}}%
\pgfpathlineto{\pgfqpoint{3.556349in}{2.334412in}}%
\pgfpathlineto{\pgfqpoint{3.556349in}{2.328513in}}%
\pgfpathmoveto{\pgfqpoint{3.547267in}{2.334412in}}%
\pgfpathlineto{\pgfqpoint{3.547267in}{2.334412in}}%
\pgfpathlineto{\pgfqpoint{3.547267in}{2.340310in}}%
\pgfpathlineto{\pgfqpoint{3.556349in}{2.340310in}}%
\pgfpathlineto{\pgfqpoint{3.556349in}{2.334412in}}%
\pgfpathmoveto{\pgfqpoint{3.556349in}{2.334412in}}%
\pgfpathlineto{\pgfqpoint{3.556349in}{2.334412in}}%
\pgfpathlineto{\pgfqpoint{3.556349in}{2.340310in}}%
\pgfpathlineto{\pgfqpoint{3.565430in}{2.340310in}}%
\pgfpathlineto{\pgfqpoint{3.565430in}{2.334412in}}%
\pgfpathmoveto{\pgfqpoint{3.565430in}{2.340310in}}%
\pgfpathlineto{\pgfqpoint{3.565430in}{2.340310in}}%
\pgfpathlineto{\pgfqpoint{3.565430in}{2.346208in}}%
\pgfpathlineto{\pgfqpoint{3.574512in}{2.346208in}}%
\pgfpathlineto{\pgfqpoint{3.574512in}{2.340310in}}%
\pgfpathmoveto{\pgfqpoint{3.565430in}{2.346208in}}%
\pgfpathlineto{\pgfqpoint{3.565430in}{2.346208in}}%
\pgfpathlineto{\pgfqpoint{3.565430in}{2.352107in}}%
\pgfpathlineto{\pgfqpoint{3.574512in}{2.352107in}}%
\pgfpathlineto{\pgfqpoint{3.574512in}{2.346208in}}%
\pgfpathmoveto{\pgfqpoint{3.574512in}{2.346208in}}%
\pgfpathlineto{\pgfqpoint{3.574512in}{2.346208in}}%
\pgfpathlineto{\pgfqpoint{3.574512in}{2.352107in}}%
\pgfpathlineto{\pgfqpoint{3.583593in}{2.352107in}}%
\pgfpathlineto{\pgfqpoint{3.583593in}{2.346208in}}%
\pgfpathmoveto{\pgfqpoint{3.583593in}{2.352107in}}%
\pgfpathlineto{\pgfqpoint{3.583593in}{2.352107in}}%
\pgfpathlineto{\pgfqpoint{3.583593in}{2.358005in}}%
\pgfpathlineto{\pgfqpoint{3.592675in}{2.358005in}}%
\pgfpathlineto{\pgfqpoint{3.592675in}{2.352107in}}%
\pgfpathmoveto{\pgfqpoint{3.583593in}{2.358005in}}%
\pgfpathlineto{\pgfqpoint{3.583593in}{2.358005in}}%
\pgfpathlineto{\pgfqpoint{3.583593in}{2.363904in}}%
\pgfpathlineto{\pgfqpoint{3.592675in}{2.363904in}}%
\pgfpathlineto{\pgfqpoint{3.592675in}{2.358005in}}%
\pgfpathmoveto{\pgfqpoint{3.592675in}{2.358005in}}%
\pgfpathlineto{\pgfqpoint{3.592675in}{2.358005in}}%
\pgfpathlineto{\pgfqpoint{3.592675in}{2.363904in}}%
\pgfpathlineto{\pgfqpoint{3.601757in}{2.363904in}}%
\pgfpathlineto{\pgfqpoint{3.601757in}{2.358005in}}%
\pgfpathmoveto{\pgfqpoint{3.601757in}{2.363904in}}%
\pgfpathlineto{\pgfqpoint{3.601757in}{2.363904in}}%
\pgfpathlineto{\pgfqpoint{3.601757in}{2.369802in}}%
\pgfpathlineto{\pgfqpoint{3.610838in}{2.369802in}}%
\pgfpathlineto{\pgfqpoint{3.610838in}{2.363904in}}%
\pgfpathmoveto{\pgfqpoint{3.601757in}{2.369802in}}%
\pgfpathlineto{\pgfqpoint{3.601757in}{2.369802in}}%
\pgfpathlineto{\pgfqpoint{3.601757in}{2.375701in}}%
\pgfpathlineto{\pgfqpoint{3.610838in}{2.375701in}}%
\pgfpathlineto{\pgfqpoint{3.610838in}{2.369802in}}%
\pgfpathmoveto{\pgfqpoint{3.610838in}{2.369802in}}%
\pgfpathlineto{\pgfqpoint{3.610838in}{2.369802in}}%
\pgfpathlineto{\pgfqpoint{3.610838in}{2.375701in}}%
\pgfpathlineto{\pgfqpoint{3.619920in}{2.375701in}}%
\pgfpathlineto{\pgfqpoint{3.619920in}{2.369802in}}%
\pgfpathmoveto{\pgfqpoint{3.619920in}{2.375701in}}%
\pgfpathlineto{\pgfqpoint{3.619920in}{2.375701in}}%
\pgfpathlineto{\pgfqpoint{3.619920in}{2.381599in}}%
\pgfpathlineto{\pgfqpoint{3.629001in}{2.381599in}}%
\pgfpathlineto{\pgfqpoint{3.629001in}{2.375701in}}%
\pgfpathmoveto{\pgfqpoint{3.619920in}{2.381599in}}%
\pgfpathlineto{\pgfqpoint{3.619920in}{2.381599in}}%
\pgfpathlineto{\pgfqpoint{3.619920in}{2.387497in}}%
\pgfpathlineto{\pgfqpoint{3.629001in}{2.387497in}}%
\pgfpathlineto{\pgfqpoint{3.629001in}{2.381599in}}%
\pgfpathmoveto{\pgfqpoint{3.629001in}{2.381599in}}%
\pgfpathlineto{\pgfqpoint{3.629001in}{2.381599in}}%
\pgfpathlineto{\pgfqpoint{3.629001in}{2.387497in}}%
\pgfpathlineto{\pgfqpoint{3.638083in}{2.387497in}}%
\pgfpathlineto{\pgfqpoint{3.638083in}{2.381599in}}%
\pgfpathmoveto{\pgfqpoint{3.638083in}{2.387497in}}%
\pgfpathlineto{\pgfqpoint{3.638083in}{2.387497in}}%
\pgfpathlineto{\pgfqpoint{3.638083in}{2.393396in}}%
\pgfpathlineto{\pgfqpoint{3.647165in}{2.393396in}}%
\pgfpathlineto{\pgfqpoint{3.647165in}{2.387497in}}%
\pgfpathmoveto{\pgfqpoint{3.638083in}{2.393396in}}%
\pgfpathlineto{\pgfqpoint{3.638083in}{2.393396in}}%
\pgfpathlineto{\pgfqpoint{3.638083in}{2.399294in}}%
\pgfpathlineto{\pgfqpoint{3.647165in}{2.399294in}}%
\pgfpathlineto{\pgfqpoint{3.647165in}{2.393396in}}%
\pgfpathmoveto{\pgfqpoint{3.647165in}{2.393396in}}%
\pgfpathlineto{\pgfqpoint{3.647165in}{2.393396in}}%
\pgfpathlineto{\pgfqpoint{3.647165in}{2.399294in}}%
\pgfpathlineto{\pgfqpoint{3.656246in}{2.399294in}}%
\pgfpathlineto{\pgfqpoint{3.656246in}{2.393396in}}%
\pgfpathmoveto{\pgfqpoint{3.638083in}{2.458279in}}%
\pgfpathlineto{\pgfqpoint{3.638083in}{2.458279in}}%
\pgfpathlineto{\pgfqpoint{3.638083in}{2.464177in}}%
\pgfpathlineto{\pgfqpoint{3.647165in}{2.464177in}}%
\pgfpathlineto{\pgfqpoint{3.647165in}{2.458279in}}%
\pgfpathmoveto{\pgfqpoint{3.638083in}{2.464177in}}%
\pgfpathlineto{\pgfqpoint{3.638083in}{2.464177in}}%
\pgfpathlineto{\pgfqpoint{3.638083in}{2.470076in}}%
\pgfpathlineto{\pgfqpoint{3.647165in}{2.470076in}}%
\pgfpathlineto{\pgfqpoint{3.647165in}{2.464177in}}%
\pgfpathmoveto{\pgfqpoint{3.565430in}{2.552657in}}%
\pgfpathlineto{\pgfqpoint{3.565430in}{2.552657in}}%
\pgfpathlineto{\pgfqpoint{3.565430in}{2.558556in}}%
\pgfpathlineto{\pgfqpoint{3.574512in}{2.558556in}}%
\pgfpathlineto{\pgfqpoint{3.574512in}{2.552657in}}%
\pgfpathmoveto{\pgfqpoint{3.565430in}{2.558556in}}%
\pgfpathlineto{\pgfqpoint{3.565430in}{2.558556in}}%
\pgfpathlineto{\pgfqpoint{3.565430in}{2.564454in}}%
\pgfpathlineto{\pgfqpoint{3.574512in}{2.564454in}}%
\pgfpathlineto{\pgfqpoint{3.574512in}{2.558556in}}%
\pgfpathmoveto{\pgfqpoint{3.601757in}{2.505467in}}%
\pgfpathlineto{\pgfqpoint{3.601757in}{2.505467in}}%
\pgfpathlineto{\pgfqpoint{3.601757in}{2.511366in}}%
\pgfpathlineto{\pgfqpoint{3.610838in}{2.511366in}}%
\pgfpathlineto{\pgfqpoint{3.610838in}{2.505467in}}%
\pgfpathmoveto{\pgfqpoint{3.601757in}{2.511366in}}%
\pgfpathlineto{\pgfqpoint{3.601757in}{2.511366in}}%
\pgfpathlineto{\pgfqpoint{3.601757in}{2.517265in}}%
\pgfpathlineto{\pgfqpoint{3.610838in}{2.517265in}}%
\pgfpathlineto{\pgfqpoint{3.610838in}{2.511366in}}%
\pgfpathmoveto{\pgfqpoint{3.619920in}{2.481873in}}%
\pgfpathlineto{\pgfqpoint{3.619920in}{2.481873in}}%
\pgfpathlineto{\pgfqpoint{3.619920in}{2.487771in}}%
\pgfpathlineto{\pgfqpoint{3.629001in}{2.487771in}}%
\pgfpathlineto{\pgfqpoint{3.629001in}{2.481873in}}%
\pgfpathmoveto{\pgfqpoint{3.619920in}{2.487771in}}%
\pgfpathlineto{\pgfqpoint{3.619920in}{2.487771in}}%
\pgfpathlineto{\pgfqpoint{3.619920in}{2.493670in}}%
\pgfpathlineto{\pgfqpoint{3.629001in}{2.493670in}}%
\pgfpathlineto{\pgfqpoint{3.629001in}{2.487771in}}%
\pgfpathmoveto{\pgfqpoint{3.583593in}{2.529062in}}%
\pgfpathlineto{\pgfqpoint{3.583593in}{2.529062in}}%
\pgfpathlineto{\pgfqpoint{3.583593in}{2.534961in}}%
\pgfpathlineto{\pgfqpoint{3.592675in}{2.534961in}}%
\pgfpathlineto{\pgfqpoint{3.592675in}{2.529062in}}%
\pgfpathmoveto{\pgfqpoint{3.583593in}{2.534961in}}%
\pgfpathlineto{\pgfqpoint{3.583593in}{2.534961in}}%
\pgfpathlineto{\pgfqpoint{3.583593in}{2.540860in}}%
\pgfpathlineto{\pgfqpoint{3.592675in}{2.540860in}}%
\pgfpathlineto{\pgfqpoint{3.592675in}{2.534961in}}%
\pgfpathmoveto{\pgfqpoint{3.529104in}{2.599845in}}%
\pgfpathlineto{\pgfqpoint{3.529104in}{2.599845in}}%
\pgfpathlineto{\pgfqpoint{3.529104in}{2.605743in}}%
\pgfpathlineto{\pgfqpoint{3.538185in}{2.605743in}}%
\pgfpathlineto{\pgfqpoint{3.538185in}{2.599845in}}%
\pgfpathmoveto{\pgfqpoint{3.529104in}{2.605743in}}%
\pgfpathlineto{\pgfqpoint{3.529104in}{2.605743in}}%
\pgfpathlineto{\pgfqpoint{3.529104in}{2.611642in}}%
\pgfpathlineto{\pgfqpoint{3.538185in}{2.611642in}}%
\pgfpathlineto{\pgfqpoint{3.538185in}{2.605743in}}%
\pgfpathmoveto{\pgfqpoint{3.547267in}{2.576252in}}%
\pgfpathlineto{\pgfqpoint{3.547267in}{2.576252in}}%
\pgfpathlineto{\pgfqpoint{3.547267in}{2.582150in}}%
\pgfpathlineto{\pgfqpoint{3.556349in}{2.582150in}}%
\pgfpathlineto{\pgfqpoint{3.556349in}{2.576252in}}%
\pgfpathmoveto{\pgfqpoint{3.547267in}{2.582150in}}%
\pgfpathlineto{\pgfqpoint{3.547267in}{2.582150in}}%
\pgfpathlineto{\pgfqpoint{3.547267in}{2.588048in}}%
\pgfpathlineto{\pgfqpoint{3.556349in}{2.588048in}}%
\pgfpathlineto{\pgfqpoint{3.556349in}{2.582150in}}%
\pgfpathmoveto{\pgfqpoint{3.510941in}{2.623438in}}%
\pgfpathlineto{\pgfqpoint{3.510941in}{2.623438in}}%
\pgfpathlineto{\pgfqpoint{3.510941in}{2.629336in}}%
\pgfpathlineto{\pgfqpoint{3.520022in}{2.629336in}}%
\pgfpathlineto{\pgfqpoint{3.520022in}{2.623438in}}%
\pgfpathmoveto{\pgfqpoint{3.510941in}{2.629336in}}%
\pgfpathlineto{\pgfqpoint{3.510941in}{2.629336in}}%
\pgfpathlineto{\pgfqpoint{3.510941in}{2.635235in}}%
\pgfpathlineto{\pgfqpoint{3.520022in}{2.635235in}}%
\pgfpathlineto{\pgfqpoint{3.520022in}{2.629336in}}%
\pgfpathmoveto{\pgfqpoint{3.656246in}{2.399294in}}%
\pgfpathlineto{\pgfqpoint{3.656246in}{2.399294in}}%
\pgfpathlineto{\pgfqpoint{3.656246in}{2.405193in}}%
\pgfpathlineto{\pgfqpoint{3.665329in}{2.405193in}}%
\pgfpathlineto{\pgfqpoint{3.665329in}{2.399294in}}%
\pgfpathmoveto{\pgfqpoint{3.656246in}{2.405193in}}%
\pgfpathlineto{\pgfqpoint{3.656246in}{2.405193in}}%
\pgfpathlineto{\pgfqpoint{3.656246in}{2.411091in}}%
\pgfpathlineto{\pgfqpoint{3.665329in}{2.411091in}}%
\pgfpathlineto{\pgfqpoint{3.665329in}{2.405193in}}%
\pgfpathmoveto{\pgfqpoint{3.665329in}{2.405193in}}%
\pgfpathlineto{\pgfqpoint{3.665329in}{2.405193in}}%
\pgfpathlineto{\pgfqpoint{3.665329in}{2.411091in}}%
\pgfpathlineto{\pgfqpoint{3.674411in}{2.411091in}}%
\pgfpathlineto{\pgfqpoint{3.674411in}{2.405193in}}%
\pgfpathmoveto{\pgfqpoint{3.674411in}{2.411091in}}%
\pgfpathlineto{\pgfqpoint{3.674411in}{2.411091in}}%
\pgfpathlineto{\pgfqpoint{3.674411in}{2.416990in}}%
\pgfpathlineto{\pgfqpoint{3.683494in}{2.416990in}}%
\pgfpathlineto{\pgfqpoint{3.683494in}{2.411091in}}%
\pgfpathmoveto{\pgfqpoint{3.674411in}{2.416990in}}%
\pgfpathlineto{\pgfqpoint{3.674411in}{2.416990in}}%
\pgfpathlineto{\pgfqpoint{3.674411in}{2.422888in}}%
\pgfpathlineto{\pgfqpoint{3.683494in}{2.422888in}}%
\pgfpathlineto{\pgfqpoint{3.683494in}{2.416990in}}%
\pgfpathmoveto{\pgfqpoint{3.656246in}{2.434685in}}%
\pgfpathlineto{\pgfqpoint{3.656246in}{2.434685in}}%
\pgfpathlineto{\pgfqpoint{3.656246in}{2.440583in}}%
\pgfpathlineto{\pgfqpoint{3.665329in}{2.440583in}}%
\pgfpathlineto{\pgfqpoint{3.665329in}{2.434685in}}%
\pgfpathmoveto{\pgfqpoint{3.656246in}{2.440583in}}%
\pgfpathlineto{\pgfqpoint{3.656246in}{2.440583in}}%
\pgfpathlineto{\pgfqpoint{3.656246in}{2.446482in}}%
\pgfpathlineto{\pgfqpoint{3.665329in}{2.446482in}}%
\pgfpathlineto{\pgfqpoint{3.665329in}{2.440583in}}%
\pgfpathmoveto{\pgfqpoint{0.750004in}{0.499999in}}%
\pgfpathlineto{\pgfqpoint{0.750004in}{0.499999in}}%
\pgfpathlineto{\pgfqpoint{0.750004in}{0.502948in}}%
\pgfpathlineto{\pgfqpoint{0.754545in}{0.502948in}}%
\pgfpathlineto{\pgfqpoint{0.754545in}{0.499999in}}%
\pgfpathmoveto{\pgfqpoint{0.750004in}{0.502948in}}%
\pgfpathlineto{\pgfqpoint{0.750004in}{0.502948in}}%
\pgfpathlineto{\pgfqpoint{0.750004in}{0.505898in}}%
\pgfpathlineto{\pgfqpoint{0.754545in}{0.505898in}}%
\pgfpathlineto{\pgfqpoint{0.754545in}{0.502948in}}%
\pgfpathmoveto{\pgfqpoint{0.754545in}{0.502948in}}%
\pgfpathlineto{\pgfqpoint{0.754545in}{0.502948in}}%
\pgfpathlineto{\pgfqpoint{0.754545in}{0.505898in}}%
\pgfpathlineto{\pgfqpoint{0.759086in}{0.505898in}}%
\pgfpathlineto{\pgfqpoint{0.759086in}{0.502948in}}%
\pgfpathmoveto{\pgfqpoint{0.750004in}{0.505898in}}%
\pgfpathlineto{\pgfqpoint{0.750004in}{0.505898in}}%
\pgfpathlineto{\pgfqpoint{0.750004in}{0.508847in}}%
\pgfpathlineto{\pgfqpoint{0.754545in}{0.508847in}}%
\pgfpathlineto{\pgfqpoint{0.754545in}{0.505898in}}%
\pgfpathmoveto{\pgfqpoint{0.750004in}{0.508847in}}%
\pgfpathlineto{\pgfqpoint{0.750004in}{0.508847in}}%
\pgfpathlineto{\pgfqpoint{0.750004in}{0.511796in}}%
\pgfpathlineto{\pgfqpoint{0.754545in}{0.511796in}}%
\pgfpathlineto{\pgfqpoint{0.754545in}{0.508847in}}%
\pgfpathmoveto{\pgfqpoint{0.754545in}{0.505898in}}%
\pgfpathlineto{\pgfqpoint{0.754545in}{0.505898in}}%
\pgfpathlineto{\pgfqpoint{0.754545in}{0.508847in}}%
\pgfpathlineto{\pgfqpoint{0.759086in}{0.508847in}}%
\pgfpathlineto{\pgfqpoint{0.759086in}{0.505898in}}%
\pgfpathmoveto{\pgfqpoint{0.754545in}{0.508847in}}%
\pgfpathlineto{\pgfqpoint{0.754545in}{0.508847in}}%
\pgfpathlineto{\pgfqpoint{0.754545in}{0.511796in}}%
\pgfpathlineto{\pgfqpoint{0.759086in}{0.511796in}}%
\pgfpathlineto{\pgfqpoint{0.759086in}{0.508847in}}%
\pgfpathmoveto{\pgfqpoint{0.759086in}{0.505898in}}%
\pgfpathlineto{\pgfqpoint{0.759086in}{0.505898in}}%
\pgfpathlineto{\pgfqpoint{0.759086in}{0.508847in}}%
\pgfpathlineto{\pgfqpoint{0.763627in}{0.508847in}}%
\pgfpathlineto{\pgfqpoint{0.763627in}{0.505898in}}%
\pgfpathmoveto{\pgfqpoint{0.759086in}{0.508847in}}%
\pgfpathlineto{\pgfqpoint{0.759086in}{0.508847in}}%
\pgfpathlineto{\pgfqpoint{0.759086in}{0.511796in}}%
\pgfpathlineto{\pgfqpoint{0.763627in}{0.511796in}}%
\pgfpathlineto{\pgfqpoint{0.763627in}{0.508847in}}%
\pgfpathmoveto{\pgfqpoint{0.763627in}{0.508847in}}%
\pgfpathlineto{\pgfqpoint{0.763627in}{0.508847in}}%
\pgfpathlineto{\pgfqpoint{0.763627in}{0.511796in}}%
\pgfpathlineto{\pgfqpoint{0.768168in}{0.511796in}}%
\pgfpathlineto{\pgfqpoint{0.768168in}{0.508847in}}%
\pgfpathmoveto{\pgfqpoint{0.759086in}{0.511796in}}%
\pgfpathlineto{\pgfqpoint{0.759086in}{0.511796in}}%
\pgfpathlineto{\pgfqpoint{0.759086in}{0.514745in}}%
\pgfpathlineto{\pgfqpoint{0.763627in}{0.514745in}}%
\pgfpathlineto{\pgfqpoint{0.763627in}{0.511796in}}%
\pgfpathmoveto{\pgfqpoint{0.759086in}{0.514745in}}%
\pgfpathlineto{\pgfqpoint{0.759086in}{0.514745in}}%
\pgfpathlineto{\pgfqpoint{0.759086in}{0.517695in}}%
\pgfpathlineto{\pgfqpoint{0.763627in}{0.517695in}}%
\pgfpathlineto{\pgfqpoint{0.763627in}{0.514745in}}%
\pgfpathmoveto{\pgfqpoint{0.763627in}{0.511796in}}%
\pgfpathlineto{\pgfqpoint{0.763627in}{0.511796in}}%
\pgfpathlineto{\pgfqpoint{0.763627in}{0.514745in}}%
\pgfpathlineto{\pgfqpoint{0.768168in}{0.514745in}}%
\pgfpathlineto{\pgfqpoint{0.768168in}{0.511796in}}%
\pgfpathmoveto{\pgfqpoint{0.763627in}{0.514745in}}%
\pgfpathlineto{\pgfqpoint{0.763627in}{0.514745in}}%
\pgfpathlineto{\pgfqpoint{0.763627in}{0.517695in}}%
\pgfpathlineto{\pgfqpoint{0.768168in}{0.517695in}}%
\pgfpathlineto{\pgfqpoint{0.768168in}{0.514745in}}%
\pgfpathmoveto{\pgfqpoint{0.768168in}{0.511796in}}%
\pgfpathlineto{\pgfqpoint{0.768168in}{0.511796in}}%
\pgfpathlineto{\pgfqpoint{0.768168in}{0.514745in}}%
\pgfpathlineto{\pgfqpoint{0.772709in}{0.514745in}}%
\pgfpathlineto{\pgfqpoint{0.772709in}{0.511796in}}%
\pgfpathmoveto{\pgfqpoint{0.768168in}{0.514745in}}%
\pgfpathlineto{\pgfqpoint{0.768168in}{0.514745in}}%
\pgfpathlineto{\pgfqpoint{0.768168in}{0.517695in}}%
\pgfpathlineto{\pgfqpoint{0.772709in}{0.517695in}}%
\pgfpathlineto{\pgfqpoint{0.772709in}{0.514745in}}%
\pgfpathmoveto{\pgfqpoint{0.772709in}{0.514745in}}%
\pgfpathlineto{\pgfqpoint{0.772709in}{0.514745in}}%
\pgfpathlineto{\pgfqpoint{0.772709in}{0.517695in}}%
\pgfpathlineto{\pgfqpoint{0.777250in}{0.517695in}}%
\pgfpathlineto{\pgfqpoint{0.777250in}{0.514745in}}%
\pgfpathmoveto{\pgfqpoint{0.768168in}{0.517695in}}%
\pgfpathlineto{\pgfqpoint{0.768168in}{0.517695in}}%
\pgfpathlineto{\pgfqpoint{0.768168in}{0.520644in}}%
\pgfpathlineto{\pgfqpoint{0.772709in}{0.520644in}}%
\pgfpathlineto{\pgfqpoint{0.772709in}{0.517695in}}%
\pgfpathmoveto{\pgfqpoint{0.768168in}{0.520644in}}%
\pgfpathlineto{\pgfqpoint{0.768168in}{0.520644in}}%
\pgfpathlineto{\pgfqpoint{0.768168in}{0.523593in}}%
\pgfpathlineto{\pgfqpoint{0.772709in}{0.523593in}}%
\pgfpathlineto{\pgfqpoint{0.772709in}{0.520644in}}%
\pgfpathmoveto{\pgfqpoint{0.772709in}{0.517695in}}%
\pgfpathlineto{\pgfqpoint{0.772709in}{0.517695in}}%
\pgfpathlineto{\pgfqpoint{0.772709in}{0.520644in}}%
\pgfpathlineto{\pgfqpoint{0.777250in}{0.520644in}}%
\pgfpathlineto{\pgfqpoint{0.777250in}{0.517695in}}%
\pgfpathmoveto{\pgfqpoint{0.772709in}{0.520644in}}%
\pgfpathlineto{\pgfqpoint{0.772709in}{0.520644in}}%
\pgfpathlineto{\pgfqpoint{0.772709in}{0.523593in}}%
\pgfpathlineto{\pgfqpoint{0.777250in}{0.523593in}}%
\pgfpathlineto{\pgfqpoint{0.777250in}{0.520644in}}%
\pgfpathmoveto{\pgfqpoint{0.777250in}{0.517695in}}%
\pgfpathlineto{\pgfqpoint{0.777250in}{0.517695in}}%
\pgfpathlineto{\pgfqpoint{0.777250in}{0.520644in}}%
\pgfpathlineto{\pgfqpoint{0.781791in}{0.520644in}}%
\pgfpathlineto{\pgfqpoint{0.781791in}{0.517695in}}%
\pgfpathmoveto{\pgfqpoint{0.777250in}{0.520644in}}%
\pgfpathlineto{\pgfqpoint{0.777250in}{0.520644in}}%
\pgfpathlineto{\pgfqpoint{0.777250in}{0.523593in}}%
\pgfpathlineto{\pgfqpoint{0.781791in}{0.523593in}}%
\pgfpathlineto{\pgfqpoint{0.781791in}{0.520644in}}%
\pgfpathmoveto{\pgfqpoint{0.781791in}{0.520644in}}%
\pgfpathlineto{\pgfqpoint{0.781791in}{0.520644in}}%
\pgfpathlineto{\pgfqpoint{0.781791in}{0.523593in}}%
\pgfpathlineto{\pgfqpoint{0.786332in}{0.523593in}}%
\pgfpathlineto{\pgfqpoint{0.786332in}{0.520644in}}%
\pgfpathmoveto{\pgfqpoint{0.777250in}{0.523593in}}%
\pgfpathlineto{\pgfqpoint{0.777250in}{0.523593in}}%
\pgfpathlineto{\pgfqpoint{0.777250in}{0.526543in}}%
\pgfpathlineto{\pgfqpoint{0.781791in}{0.526543in}}%
\pgfpathlineto{\pgfqpoint{0.781791in}{0.523593in}}%
\pgfpathmoveto{\pgfqpoint{0.777250in}{0.526543in}}%
\pgfpathlineto{\pgfqpoint{0.777250in}{0.526543in}}%
\pgfpathlineto{\pgfqpoint{0.777250in}{0.529492in}}%
\pgfpathlineto{\pgfqpoint{0.781791in}{0.529492in}}%
\pgfpathlineto{\pgfqpoint{0.781791in}{0.526543in}}%
\pgfpathmoveto{\pgfqpoint{0.781791in}{0.523593in}}%
\pgfpathlineto{\pgfqpoint{0.781791in}{0.523593in}}%
\pgfpathlineto{\pgfqpoint{0.781791in}{0.526543in}}%
\pgfpathlineto{\pgfqpoint{0.786332in}{0.526543in}}%
\pgfpathlineto{\pgfqpoint{0.786332in}{0.523593in}}%
\pgfpathmoveto{\pgfqpoint{0.781791in}{0.526543in}}%
\pgfpathlineto{\pgfqpoint{0.781791in}{0.526543in}}%
\pgfpathlineto{\pgfqpoint{0.781791in}{0.529492in}}%
\pgfpathlineto{\pgfqpoint{0.786332in}{0.529492in}}%
\pgfpathlineto{\pgfqpoint{0.786332in}{0.526543in}}%
\pgfpathmoveto{\pgfqpoint{0.786332in}{0.523593in}}%
\pgfpathlineto{\pgfqpoint{0.786332in}{0.523593in}}%
\pgfpathlineto{\pgfqpoint{0.786332in}{0.526543in}}%
\pgfpathlineto{\pgfqpoint{0.790873in}{0.526543in}}%
\pgfpathlineto{\pgfqpoint{0.790873in}{0.523593in}}%
\pgfpathmoveto{\pgfqpoint{0.786332in}{0.526543in}}%
\pgfpathlineto{\pgfqpoint{0.786332in}{0.526543in}}%
\pgfpathlineto{\pgfqpoint{0.786332in}{0.529492in}}%
\pgfpathlineto{\pgfqpoint{0.790873in}{0.529492in}}%
\pgfpathlineto{\pgfqpoint{0.790873in}{0.526543in}}%
\pgfpathmoveto{\pgfqpoint{0.790873in}{0.526543in}}%
\pgfpathlineto{\pgfqpoint{0.790873in}{0.526543in}}%
\pgfpathlineto{\pgfqpoint{0.790873in}{0.529492in}}%
\pgfpathlineto{\pgfqpoint{0.795414in}{0.529492in}}%
\pgfpathlineto{\pgfqpoint{0.795414in}{0.526543in}}%
\pgfpathmoveto{\pgfqpoint{0.786332in}{0.529492in}}%
\pgfpathlineto{\pgfqpoint{0.786332in}{0.529492in}}%
\pgfpathlineto{\pgfqpoint{0.786332in}{0.532441in}}%
\pgfpathlineto{\pgfqpoint{0.790873in}{0.532441in}}%
\pgfpathlineto{\pgfqpoint{0.790873in}{0.529492in}}%
\pgfpathmoveto{\pgfqpoint{0.786332in}{0.532441in}}%
\pgfpathlineto{\pgfqpoint{0.786332in}{0.532441in}}%
\pgfpathlineto{\pgfqpoint{0.786332in}{0.535390in}}%
\pgfpathlineto{\pgfqpoint{0.790873in}{0.535390in}}%
\pgfpathlineto{\pgfqpoint{0.790873in}{0.532441in}}%
\pgfpathmoveto{\pgfqpoint{0.790873in}{0.529492in}}%
\pgfpathlineto{\pgfqpoint{0.790873in}{0.529492in}}%
\pgfpathlineto{\pgfqpoint{0.790873in}{0.532441in}}%
\pgfpathlineto{\pgfqpoint{0.795414in}{0.532441in}}%
\pgfpathlineto{\pgfqpoint{0.795414in}{0.529492in}}%
\pgfpathmoveto{\pgfqpoint{0.790873in}{0.532441in}}%
\pgfpathlineto{\pgfqpoint{0.790873in}{0.532441in}}%
\pgfpathlineto{\pgfqpoint{0.790873in}{0.535390in}}%
\pgfpathlineto{\pgfqpoint{0.795414in}{0.535390in}}%
\pgfpathlineto{\pgfqpoint{0.795414in}{0.532441in}}%
\pgfpathmoveto{\pgfqpoint{0.795414in}{0.529492in}}%
\pgfpathlineto{\pgfqpoint{0.795414in}{0.529492in}}%
\pgfpathlineto{\pgfqpoint{0.795414in}{0.532441in}}%
\pgfpathlineto{\pgfqpoint{0.799954in}{0.532441in}}%
\pgfpathlineto{\pgfqpoint{0.799954in}{0.529492in}}%
\pgfpathmoveto{\pgfqpoint{0.795414in}{0.532441in}}%
\pgfpathlineto{\pgfqpoint{0.795414in}{0.532441in}}%
\pgfpathlineto{\pgfqpoint{0.795414in}{0.535390in}}%
\pgfpathlineto{\pgfqpoint{0.799954in}{0.535390in}}%
\pgfpathlineto{\pgfqpoint{0.799954in}{0.532441in}}%
\pgfpathmoveto{\pgfqpoint{0.799954in}{0.532441in}}%
\pgfpathlineto{\pgfqpoint{0.799954in}{0.532441in}}%
\pgfpathlineto{\pgfqpoint{0.799954in}{0.535390in}}%
\pgfpathlineto{\pgfqpoint{0.804495in}{0.535390in}}%
\pgfpathlineto{\pgfqpoint{0.804495in}{0.532441in}}%
\pgfpathmoveto{\pgfqpoint{0.795414in}{0.535390in}}%
\pgfpathlineto{\pgfqpoint{0.795414in}{0.535390in}}%
\pgfpathlineto{\pgfqpoint{0.795414in}{0.538340in}}%
\pgfpathlineto{\pgfqpoint{0.799954in}{0.538340in}}%
\pgfpathlineto{\pgfqpoint{0.799954in}{0.535390in}}%
\pgfpathmoveto{\pgfqpoint{0.795414in}{0.538340in}}%
\pgfpathlineto{\pgfqpoint{0.795414in}{0.538340in}}%
\pgfpathlineto{\pgfqpoint{0.795414in}{0.541289in}}%
\pgfpathlineto{\pgfqpoint{0.799954in}{0.541289in}}%
\pgfpathlineto{\pgfqpoint{0.799954in}{0.538340in}}%
\pgfpathmoveto{\pgfqpoint{0.799954in}{0.535390in}}%
\pgfpathlineto{\pgfqpoint{0.799954in}{0.535390in}}%
\pgfpathlineto{\pgfqpoint{0.799954in}{0.538340in}}%
\pgfpathlineto{\pgfqpoint{0.804495in}{0.538340in}}%
\pgfpathlineto{\pgfqpoint{0.804495in}{0.535390in}}%
\pgfpathmoveto{\pgfqpoint{0.799954in}{0.538340in}}%
\pgfpathlineto{\pgfqpoint{0.799954in}{0.538340in}}%
\pgfpathlineto{\pgfqpoint{0.799954in}{0.541289in}}%
\pgfpathlineto{\pgfqpoint{0.804495in}{0.541289in}}%
\pgfpathlineto{\pgfqpoint{0.804495in}{0.538340in}}%
\pgfpathmoveto{\pgfqpoint{0.804495in}{0.535390in}}%
\pgfpathlineto{\pgfqpoint{0.804495in}{0.535390in}}%
\pgfpathlineto{\pgfqpoint{0.804495in}{0.538340in}}%
\pgfpathlineto{\pgfqpoint{0.809036in}{0.538340in}}%
\pgfpathlineto{\pgfqpoint{0.809036in}{0.535390in}}%
\pgfpathmoveto{\pgfqpoint{0.804495in}{0.538340in}}%
\pgfpathlineto{\pgfqpoint{0.804495in}{0.538340in}}%
\pgfpathlineto{\pgfqpoint{0.804495in}{0.541289in}}%
\pgfpathlineto{\pgfqpoint{0.809036in}{0.541289in}}%
\pgfpathlineto{\pgfqpoint{0.809036in}{0.538340in}}%
\pgfpathmoveto{\pgfqpoint{0.809036in}{0.538340in}}%
\pgfpathlineto{\pgfqpoint{0.809036in}{0.538340in}}%
\pgfpathlineto{\pgfqpoint{0.809036in}{0.541289in}}%
\pgfpathlineto{\pgfqpoint{0.813577in}{0.541289in}}%
\pgfpathlineto{\pgfqpoint{0.813577in}{0.538340in}}%
\pgfpathmoveto{\pgfqpoint{0.804495in}{0.541289in}}%
\pgfpathlineto{\pgfqpoint{0.804495in}{0.541289in}}%
\pgfpathlineto{\pgfqpoint{0.804495in}{0.544238in}}%
\pgfpathlineto{\pgfqpoint{0.809036in}{0.544238in}}%
\pgfpathlineto{\pgfqpoint{0.809036in}{0.541289in}}%
\pgfpathmoveto{\pgfqpoint{0.804495in}{0.544238in}}%
\pgfpathlineto{\pgfqpoint{0.804495in}{0.544238in}}%
\pgfpathlineto{\pgfqpoint{0.804495in}{0.547188in}}%
\pgfpathlineto{\pgfqpoint{0.809036in}{0.547188in}}%
\pgfpathlineto{\pgfqpoint{0.809036in}{0.544238in}}%
\pgfpathmoveto{\pgfqpoint{0.809036in}{0.541289in}}%
\pgfpathlineto{\pgfqpoint{0.809036in}{0.541289in}}%
\pgfpathlineto{\pgfqpoint{0.809036in}{0.544238in}}%
\pgfpathlineto{\pgfqpoint{0.813577in}{0.544238in}}%
\pgfpathlineto{\pgfqpoint{0.813577in}{0.541289in}}%
\pgfpathmoveto{\pgfqpoint{0.809036in}{0.544238in}}%
\pgfpathlineto{\pgfqpoint{0.809036in}{0.544238in}}%
\pgfpathlineto{\pgfqpoint{0.809036in}{0.547188in}}%
\pgfpathlineto{\pgfqpoint{0.813577in}{0.547188in}}%
\pgfpathlineto{\pgfqpoint{0.813577in}{0.544238in}}%
\pgfpathmoveto{\pgfqpoint{0.813577in}{0.541289in}}%
\pgfpathlineto{\pgfqpoint{0.813577in}{0.541289in}}%
\pgfpathlineto{\pgfqpoint{0.813577in}{0.544238in}}%
\pgfpathlineto{\pgfqpoint{0.818118in}{0.544238in}}%
\pgfpathlineto{\pgfqpoint{0.818118in}{0.541289in}}%
\pgfpathmoveto{\pgfqpoint{0.813577in}{0.544238in}}%
\pgfpathlineto{\pgfqpoint{0.813577in}{0.544238in}}%
\pgfpathlineto{\pgfqpoint{0.813577in}{0.547188in}}%
\pgfpathlineto{\pgfqpoint{0.818118in}{0.547188in}}%
\pgfpathlineto{\pgfqpoint{0.818118in}{0.544238in}}%
\pgfpathmoveto{\pgfqpoint{0.818118in}{0.544238in}}%
\pgfpathlineto{\pgfqpoint{0.818118in}{0.544238in}}%
\pgfpathlineto{\pgfqpoint{0.818118in}{0.547188in}}%
\pgfpathlineto{\pgfqpoint{0.822659in}{0.547188in}}%
\pgfpathlineto{\pgfqpoint{0.822659in}{0.544238in}}%
\pgfpathmoveto{\pgfqpoint{0.813577in}{0.547188in}}%
\pgfpathlineto{\pgfqpoint{0.813577in}{0.547188in}}%
\pgfpathlineto{\pgfqpoint{0.813577in}{0.550137in}}%
\pgfpathlineto{\pgfqpoint{0.818118in}{0.550137in}}%
\pgfpathlineto{\pgfqpoint{0.818118in}{0.547188in}}%
\pgfpathmoveto{\pgfqpoint{0.813577in}{0.550137in}}%
\pgfpathlineto{\pgfqpoint{0.813577in}{0.550137in}}%
\pgfpathlineto{\pgfqpoint{0.813577in}{0.553086in}}%
\pgfpathlineto{\pgfqpoint{0.818118in}{0.553086in}}%
\pgfpathlineto{\pgfqpoint{0.818118in}{0.550137in}}%
\pgfpathmoveto{\pgfqpoint{0.818118in}{0.547188in}}%
\pgfpathlineto{\pgfqpoint{0.818118in}{0.547188in}}%
\pgfpathlineto{\pgfqpoint{0.818118in}{0.550137in}}%
\pgfpathlineto{\pgfqpoint{0.822659in}{0.550137in}}%
\pgfpathlineto{\pgfqpoint{0.822659in}{0.547188in}}%
\pgfpathmoveto{\pgfqpoint{0.818118in}{0.550137in}}%
\pgfpathlineto{\pgfqpoint{0.818118in}{0.550137in}}%
\pgfpathlineto{\pgfqpoint{0.818118in}{0.553086in}}%
\pgfpathlineto{\pgfqpoint{0.822659in}{0.553086in}}%
\pgfpathlineto{\pgfqpoint{0.822659in}{0.550137in}}%
\pgfpathmoveto{\pgfqpoint{0.822659in}{0.547188in}}%
\pgfpathlineto{\pgfqpoint{0.822659in}{0.547188in}}%
\pgfpathlineto{\pgfqpoint{0.822659in}{0.550137in}}%
\pgfpathlineto{\pgfqpoint{0.827200in}{0.550137in}}%
\pgfpathlineto{\pgfqpoint{0.827200in}{0.547188in}}%
\pgfpathmoveto{\pgfqpoint{0.822659in}{0.550137in}}%
\pgfpathlineto{\pgfqpoint{0.822659in}{0.550137in}}%
\pgfpathlineto{\pgfqpoint{0.822659in}{0.553086in}}%
\pgfpathlineto{\pgfqpoint{0.827200in}{0.553086in}}%
\pgfpathlineto{\pgfqpoint{0.827200in}{0.550137in}}%
\pgfpathmoveto{\pgfqpoint{0.827200in}{0.550137in}}%
\pgfpathlineto{\pgfqpoint{0.827200in}{0.550137in}}%
\pgfpathlineto{\pgfqpoint{0.827200in}{0.553086in}}%
\pgfpathlineto{\pgfqpoint{0.831741in}{0.553086in}}%
\pgfpathlineto{\pgfqpoint{0.831741in}{0.550137in}}%
\pgfpathmoveto{\pgfqpoint{0.822659in}{0.553086in}}%
\pgfpathlineto{\pgfqpoint{0.822659in}{0.553086in}}%
\pgfpathlineto{\pgfqpoint{0.822659in}{0.556035in}}%
\pgfpathlineto{\pgfqpoint{0.827200in}{0.556035in}}%
\pgfpathlineto{\pgfqpoint{0.827200in}{0.553086in}}%
\pgfpathmoveto{\pgfqpoint{0.822659in}{0.556035in}}%
\pgfpathlineto{\pgfqpoint{0.822659in}{0.556035in}}%
\pgfpathlineto{\pgfqpoint{0.822659in}{0.558985in}}%
\pgfpathlineto{\pgfqpoint{0.827200in}{0.558985in}}%
\pgfpathlineto{\pgfqpoint{0.827200in}{0.556035in}}%
\pgfpathmoveto{\pgfqpoint{0.827200in}{0.553086in}}%
\pgfpathlineto{\pgfqpoint{0.827200in}{0.553086in}}%
\pgfpathlineto{\pgfqpoint{0.827200in}{0.556035in}}%
\pgfpathlineto{\pgfqpoint{0.831741in}{0.556035in}}%
\pgfpathlineto{\pgfqpoint{0.831741in}{0.553086in}}%
\pgfpathmoveto{\pgfqpoint{0.827200in}{0.556035in}}%
\pgfpathlineto{\pgfqpoint{0.827200in}{0.556035in}}%
\pgfpathlineto{\pgfqpoint{0.827200in}{0.558985in}}%
\pgfpathlineto{\pgfqpoint{0.831741in}{0.558985in}}%
\pgfpathlineto{\pgfqpoint{0.831741in}{0.556035in}}%
\pgfpathmoveto{\pgfqpoint{0.831741in}{0.553086in}}%
\pgfpathlineto{\pgfqpoint{0.831741in}{0.553086in}}%
\pgfpathlineto{\pgfqpoint{0.831741in}{0.556035in}}%
\pgfpathlineto{\pgfqpoint{0.836282in}{0.556035in}}%
\pgfpathlineto{\pgfqpoint{0.836282in}{0.553086in}}%
\pgfpathmoveto{\pgfqpoint{0.831741in}{0.556035in}}%
\pgfpathlineto{\pgfqpoint{0.831741in}{0.556035in}}%
\pgfpathlineto{\pgfqpoint{0.831741in}{0.558985in}}%
\pgfpathlineto{\pgfqpoint{0.836282in}{0.558985in}}%
\pgfpathlineto{\pgfqpoint{0.836282in}{0.556035in}}%
\pgfpathmoveto{\pgfqpoint{0.836282in}{0.556035in}}%
\pgfpathlineto{\pgfqpoint{0.836282in}{0.556035in}}%
\pgfpathlineto{\pgfqpoint{0.836282in}{0.558985in}}%
\pgfpathlineto{\pgfqpoint{0.840823in}{0.558985in}}%
\pgfpathlineto{\pgfqpoint{0.840823in}{0.556035in}}%
\pgfpathmoveto{\pgfqpoint{0.831741in}{0.558985in}}%
\pgfpathlineto{\pgfqpoint{0.831741in}{0.558985in}}%
\pgfpathlineto{\pgfqpoint{0.831741in}{0.561934in}}%
\pgfpathlineto{\pgfqpoint{0.836282in}{0.561934in}}%
\pgfpathlineto{\pgfqpoint{0.836282in}{0.558985in}}%
\pgfpathmoveto{\pgfqpoint{0.831741in}{0.561934in}}%
\pgfpathlineto{\pgfqpoint{0.831741in}{0.561934in}}%
\pgfpathlineto{\pgfqpoint{0.831741in}{0.564883in}}%
\pgfpathlineto{\pgfqpoint{0.836282in}{0.564883in}}%
\pgfpathlineto{\pgfqpoint{0.836282in}{0.561934in}}%
\pgfpathmoveto{\pgfqpoint{0.836282in}{0.558985in}}%
\pgfpathlineto{\pgfqpoint{0.836282in}{0.558985in}}%
\pgfpathlineto{\pgfqpoint{0.836282in}{0.561934in}}%
\pgfpathlineto{\pgfqpoint{0.840823in}{0.561934in}}%
\pgfpathlineto{\pgfqpoint{0.840823in}{0.558985in}}%
\pgfpathmoveto{\pgfqpoint{0.836282in}{0.561934in}}%
\pgfpathlineto{\pgfqpoint{0.836282in}{0.561934in}}%
\pgfpathlineto{\pgfqpoint{0.836282in}{0.564883in}}%
\pgfpathlineto{\pgfqpoint{0.840823in}{0.564883in}}%
\pgfpathlineto{\pgfqpoint{0.840823in}{0.561934in}}%
\pgfpathmoveto{\pgfqpoint{0.840823in}{0.558985in}}%
\pgfpathlineto{\pgfqpoint{0.840823in}{0.558985in}}%
\pgfpathlineto{\pgfqpoint{0.840823in}{0.561934in}}%
\pgfpathlineto{\pgfqpoint{0.845364in}{0.561934in}}%
\pgfpathlineto{\pgfqpoint{0.845364in}{0.558985in}}%
\pgfpathmoveto{\pgfqpoint{0.840823in}{0.561934in}}%
\pgfpathlineto{\pgfqpoint{0.840823in}{0.561934in}}%
\pgfpathlineto{\pgfqpoint{0.840823in}{0.564883in}}%
\pgfpathlineto{\pgfqpoint{0.845364in}{0.564883in}}%
\pgfpathlineto{\pgfqpoint{0.845364in}{0.561934in}}%
\pgfpathmoveto{\pgfqpoint{0.845364in}{0.561934in}}%
\pgfpathlineto{\pgfqpoint{0.845364in}{0.561934in}}%
\pgfpathlineto{\pgfqpoint{0.845364in}{0.564883in}}%
\pgfpathlineto{\pgfqpoint{0.849905in}{0.564883in}}%
\pgfpathlineto{\pgfqpoint{0.849905in}{0.561934in}}%
\pgfpathmoveto{\pgfqpoint{0.840823in}{0.564883in}}%
\pgfpathlineto{\pgfqpoint{0.840823in}{0.564883in}}%
\pgfpathlineto{\pgfqpoint{0.840823in}{0.567833in}}%
\pgfpathlineto{\pgfqpoint{0.845364in}{0.567833in}}%
\pgfpathlineto{\pgfqpoint{0.845364in}{0.564883in}}%
\pgfpathmoveto{\pgfqpoint{0.840823in}{0.567833in}}%
\pgfpathlineto{\pgfqpoint{0.840823in}{0.567833in}}%
\pgfpathlineto{\pgfqpoint{0.840823in}{0.570782in}}%
\pgfpathlineto{\pgfqpoint{0.845364in}{0.570782in}}%
\pgfpathlineto{\pgfqpoint{0.845364in}{0.567833in}}%
\pgfpathmoveto{\pgfqpoint{0.845364in}{0.564883in}}%
\pgfpathlineto{\pgfqpoint{0.845364in}{0.564883in}}%
\pgfpathlineto{\pgfqpoint{0.845364in}{0.567833in}}%
\pgfpathlineto{\pgfqpoint{0.849905in}{0.567833in}}%
\pgfpathlineto{\pgfqpoint{0.849905in}{0.564883in}}%
\pgfpathmoveto{\pgfqpoint{0.845364in}{0.567833in}}%
\pgfpathlineto{\pgfqpoint{0.845364in}{0.567833in}}%
\pgfpathlineto{\pgfqpoint{0.845364in}{0.570782in}}%
\pgfpathlineto{\pgfqpoint{0.849905in}{0.570782in}}%
\pgfpathlineto{\pgfqpoint{0.849905in}{0.567833in}}%
\pgfpathmoveto{\pgfqpoint{0.849905in}{0.564883in}}%
\pgfpathlineto{\pgfqpoint{0.849905in}{0.564883in}}%
\pgfpathlineto{\pgfqpoint{0.849905in}{0.567833in}}%
\pgfpathlineto{\pgfqpoint{0.854446in}{0.567833in}}%
\pgfpathlineto{\pgfqpoint{0.854446in}{0.564883in}}%
\pgfpathmoveto{\pgfqpoint{0.849905in}{0.567833in}}%
\pgfpathlineto{\pgfqpoint{0.849905in}{0.567833in}}%
\pgfpathlineto{\pgfqpoint{0.849905in}{0.570782in}}%
\pgfpathlineto{\pgfqpoint{0.854446in}{0.570782in}}%
\pgfpathlineto{\pgfqpoint{0.854446in}{0.567833in}}%
\pgfpathmoveto{\pgfqpoint{0.854446in}{0.567833in}}%
\pgfpathlineto{\pgfqpoint{0.854446in}{0.567833in}}%
\pgfpathlineto{\pgfqpoint{0.854446in}{0.570782in}}%
\pgfpathlineto{\pgfqpoint{0.858987in}{0.570782in}}%
\pgfpathlineto{\pgfqpoint{0.858987in}{0.567833in}}%
\pgfpathmoveto{\pgfqpoint{0.849905in}{0.570782in}}%
\pgfpathlineto{\pgfqpoint{0.849905in}{0.570782in}}%
\pgfpathlineto{\pgfqpoint{0.849905in}{0.573731in}}%
\pgfpathlineto{\pgfqpoint{0.854446in}{0.573731in}}%
\pgfpathlineto{\pgfqpoint{0.854446in}{0.570782in}}%
\pgfpathmoveto{\pgfqpoint{0.849905in}{0.573731in}}%
\pgfpathlineto{\pgfqpoint{0.849905in}{0.573731in}}%
\pgfpathlineto{\pgfqpoint{0.849905in}{0.576680in}}%
\pgfpathlineto{\pgfqpoint{0.854446in}{0.576680in}}%
\pgfpathlineto{\pgfqpoint{0.854446in}{0.573731in}}%
\pgfpathmoveto{\pgfqpoint{0.854446in}{0.570782in}}%
\pgfpathlineto{\pgfqpoint{0.854446in}{0.570782in}}%
\pgfpathlineto{\pgfqpoint{0.854446in}{0.573731in}}%
\pgfpathlineto{\pgfqpoint{0.858987in}{0.573731in}}%
\pgfpathlineto{\pgfqpoint{0.858987in}{0.570782in}}%
\pgfpathmoveto{\pgfqpoint{0.854446in}{0.573731in}}%
\pgfpathlineto{\pgfqpoint{0.854446in}{0.573731in}}%
\pgfpathlineto{\pgfqpoint{0.854446in}{0.576680in}}%
\pgfpathlineto{\pgfqpoint{0.858987in}{0.576680in}}%
\pgfpathlineto{\pgfqpoint{0.858987in}{0.573731in}}%
\pgfpathmoveto{\pgfqpoint{0.858987in}{0.570782in}}%
\pgfpathlineto{\pgfqpoint{0.858987in}{0.570782in}}%
\pgfpathlineto{\pgfqpoint{0.858987in}{0.573731in}}%
\pgfpathlineto{\pgfqpoint{0.863528in}{0.573731in}}%
\pgfpathlineto{\pgfqpoint{0.863528in}{0.570782in}}%
\pgfpathmoveto{\pgfqpoint{0.858987in}{0.573731in}}%
\pgfpathlineto{\pgfqpoint{0.858987in}{0.573731in}}%
\pgfpathlineto{\pgfqpoint{0.858987in}{0.576680in}}%
\pgfpathlineto{\pgfqpoint{0.863528in}{0.576680in}}%
\pgfpathlineto{\pgfqpoint{0.863528in}{0.573731in}}%
\pgfpathmoveto{\pgfqpoint{0.863528in}{0.573731in}}%
\pgfpathlineto{\pgfqpoint{0.863528in}{0.573731in}}%
\pgfpathlineto{\pgfqpoint{0.863528in}{0.576680in}}%
\pgfpathlineto{\pgfqpoint{0.868069in}{0.576680in}}%
\pgfpathlineto{\pgfqpoint{0.868069in}{0.573731in}}%
\pgfpathmoveto{\pgfqpoint{0.858987in}{0.576680in}}%
\pgfpathlineto{\pgfqpoint{0.858987in}{0.576680in}}%
\pgfpathlineto{\pgfqpoint{0.858987in}{0.579630in}}%
\pgfpathlineto{\pgfqpoint{0.863528in}{0.579630in}}%
\pgfpathlineto{\pgfqpoint{0.863528in}{0.576680in}}%
\pgfpathmoveto{\pgfqpoint{0.858987in}{0.579630in}}%
\pgfpathlineto{\pgfqpoint{0.858987in}{0.579630in}}%
\pgfpathlineto{\pgfqpoint{0.858987in}{0.582579in}}%
\pgfpathlineto{\pgfqpoint{0.863528in}{0.582579in}}%
\pgfpathlineto{\pgfqpoint{0.863528in}{0.579630in}}%
\pgfpathmoveto{\pgfqpoint{0.863528in}{0.576680in}}%
\pgfpathlineto{\pgfqpoint{0.863528in}{0.576680in}}%
\pgfpathlineto{\pgfqpoint{0.863528in}{0.579630in}}%
\pgfpathlineto{\pgfqpoint{0.868069in}{0.579630in}}%
\pgfpathlineto{\pgfqpoint{0.868069in}{0.576680in}}%
\pgfpathmoveto{\pgfqpoint{0.863528in}{0.579630in}}%
\pgfpathlineto{\pgfqpoint{0.863528in}{0.579630in}}%
\pgfpathlineto{\pgfqpoint{0.863528in}{0.582579in}}%
\pgfpathlineto{\pgfqpoint{0.868069in}{0.582579in}}%
\pgfpathlineto{\pgfqpoint{0.868069in}{0.579630in}}%
\pgfpathmoveto{\pgfqpoint{0.868069in}{0.576680in}}%
\pgfpathlineto{\pgfqpoint{0.868069in}{0.576680in}}%
\pgfpathlineto{\pgfqpoint{0.868069in}{0.579630in}}%
\pgfpathlineto{\pgfqpoint{0.872610in}{0.579630in}}%
\pgfpathlineto{\pgfqpoint{0.872610in}{0.576680in}}%
\pgfpathmoveto{\pgfqpoint{0.868069in}{0.579630in}}%
\pgfpathlineto{\pgfqpoint{0.868069in}{0.579630in}}%
\pgfpathlineto{\pgfqpoint{0.868069in}{0.582579in}}%
\pgfpathlineto{\pgfqpoint{0.872610in}{0.582579in}}%
\pgfpathlineto{\pgfqpoint{0.872610in}{0.579630in}}%
\pgfpathmoveto{\pgfqpoint{0.872610in}{0.579630in}}%
\pgfpathlineto{\pgfqpoint{0.872610in}{0.579630in}}%
\pgfpathlineto{\pgfqpoint{0.872610in}{0.582579in}}%
\pgfpathlineto{\pgfqpoint{0.877150in}{0.582579in}}%
\pgfpathlineto{\pgfqpoint{0.877150in}{0.579630in}}%
\pgfpathmoveto{\pgfqpoint{0.868069in}{0.582579in}}%
\pgfpathlineto{\pgfqpoint{0.868069in}{0.582579in}}%
\pgfpathlineto{\pgfqpoint{0.868069in}{0.585528in}}%
\pgfpathlineto{\pgfqpoint{0.872610in}{0.585528in}}%
\pgfpathlineto{\pgfqpoint{0.872610in}{0.582579in}}%
\pgfpathmoveto{\pgfqpoint{0.868069in}{0.585528in}}%
\pgfpathlineto{\pgfqpoint{0.868069in}{0.585528in}}%
\pgfpathlineto{\pgfqpoint{0.868069in}{0.588478in}}%
\pgfpathlineto{\pgfqpoint{0.872610in}{0.588478in}}%
\pgfpathlineto{\pgfqpoint{0.872610in}{0.585528in}}%
\pgfpathmoveto{\pgfqpoint{0.872610in}{0.582579in}}%
\pgfpathlineto{\pgfqpoint{0.872610in}{0.582579in}}%
\pgfpathlineto{\pgfqpoint{0.872610in}{0.585528in}}%
\pgfpathlineto{\pgfqpoint{0.877150in}{0.585528in}}%
\pgfpathlineto{\pgfqpoint{0.877150in}{0.582579in}}%
\pgfpathmoveto{\pgfqpoint{0.872610in}{0.585528in}}%
\pgfpathlineto{\pgfqpoint{0.872610in}{0.585528in}}%
\pgfpathlineto{\pgfqpoint{0.872610in}{0.588478in}}%
\pgfpathlineto{\pgfqpoint{0.877150in}{0.588478in}}%
\pgfpathlineto{\pgfqpoint{0.877150in}{0.585528in}}%
\pgfpathmoveto{\pgfqpoint{0.877150in}{0.582579in}}%
\pgfpathlineto{\pgfqpoint{0.877150in}{0.582579in}}%
\pgfpathlineto{\pgfqpoint{0.877150in}{0.585528in}}%
\pgfpathlineto{\pgfqpoint{0.881691in}{0.585528in}}%
\pgfpathlineto{\pgfqpoint{0.881691in}{0.582579in}}%
\pgfpathmoveto{\pgfqpoint{0.877150in}{0.585528in}}%
\pgfpathlineto{\pgfqpoint{0.877150in}{0.585528in}}%
\pgfpathlineto{\pgfqpoint{0.877150in}{0.588478in}}%
\pgfpathlineto{\pgfqpoint{0.881691in}{0.588478in}}%
\pgfpathlineto{\pgfqpoint{0.881691in}{0.585528in}}%
\pgfpathmoveto{\pgfqpoint{0.881691in}{0.585528in}}%
\pgfpathlineto{\pgfqpoint{0.881691in}{0.585528in}}%
\pgfpathlineto{\pgfqpoint{0.881691in}{0.588478in}}%
\pgfpathlineto{\pgfqpoint{0.886232in}{0.588478in}}%
\pgfpathlineto{\pgfqpoint{0.886232in}{0.585528in}}%
\pgfpathmoveto{\pgfqpoint{0.877150in}{0.588478in}}%
\pgfpathlineto{\pgfqpoint{0.877150in}{0.588478in}}%
\pgfpathlineto{\pgfqpoint{0.877150in}{0.591427in}}%
\pgfpathlineto{\pgfqpoint{0.881691in}{0.591427in}}%
\pgfpathlineto{\pgfqpoint{0.881691in}{0.588478in}}%
\pgfpathmoveto{\pgfqpoint{0.877150in}{0.591427in}}%
\pgfpathlineto{\pgfqpoint{0.877150in}{0.591427in}}%
\pgfpathlineto{\pgfqpoint{0.877150in}{0.594376in}}%
\pgfpathlineto{\pgfqpoint{0.881691in}{0.594376in}}%
\pgfpathlineto{\pgfqpoint{0.881691in}{0.591427in}}%
\pgfpathmoveto{\pgfqpoint{0.881691in}{0.588478in}}%
\pgfpathlineto{\pgfqpoint{0.881691in}{0.588478in}}%
\pgfpathlineto{\pgfqpoint{0.881691in}{0.591427in}}%
\pgfpathlineto{\pgfqpoint{0.886232in}{0.591427in}}%
\pgfpathlineto{\pgfqpoint{0.886232in}{0.588478in}}%
\pgfpathmoveto{\pgfqpoint{0.881691in}{0.591427in}}%
\pgfpathlineto{\pgfqpoint{0.881691in}{0.591427in}}%
\pgfpathlineto{\pgfqpoint{0.881691in}{0.594376in}}%
\pgfpathlineto{\pgfqpoint{0.886232in}{0.594376in}}%
\pgfpathlineto{\pgfqpoint{0.886232in}{0.591427in}}%
\pgfpathmoveto{\pgfqpoint{0.886232in}{0.588478in}}%
\pgfpathlineto{\pgfqpoint{0.886232in}{0.588478in}}%
\pgfpathlineto{\pgfqpoint{0.886232in}{0.591427in}}%
\pgfpathlineto{\pgfqpoint{0.890773in}{0.591427in}}%
\pgfpathlineto{\pgfqpoint{0.890773in}{0.588478in}}%
\pgfpathmoveto{\pgfqpoint{0.886232in}{0.591427in}}%
\pgfpathlineto{\pgfqpoint{0.886232in}{0.591427in}}%
\pgfpathlineto{\pgfqpoint{0.886232in}{0.594376in}}%
\pgfpathlineto{\pgfqpoint{0.890773in}{0.594376in}}%
\pgfpathlineto{\pgfqpoint{0.890773in}{0.591427in}}%
\pgfpathmoveto{\pgfqpoint{0.890773in}{0.591427in}}%
\pgfpathlineto{\pgfqpoint{0.890773in}{0.591427in}}%
\pgfpathlineto{\pgfqpoint{0.890773in}{0.594376in}}%
\pgfpathlineto{\pgfqpoint{0.895314in}{0.594376in}}%
\pgfpathlineto{\pgfqpoint{0.895314in}{0.591427in}}%
\pgfpathmoveto{\pgfqpoint{0.895314in}{0.591427in}}%
\pgfpathlineto{\pgfqpoint{0.895314in}{0.591427in}}%
\pgfpathlineto{\pgfqpoint{0.895314in}{0.594376in}}%
\pgfpathlineto{\pgfqpoint{0.899855in}{0.594376in}}%
\pgfpathlineto{\pgfqpoint{0.899855in}{0.591427in}}%
\pgfpathmoveto{\pgfqpoint{0.895314in}{0.594376in}}%
\pgfpathlineto{\pgfqpoint{0.895314in}{0.594376in}}%
\pgfpathlineto{\pgfqpoint{0.895314in}{0.597325in}}%
\pgfpathlineto{\pgfqpoint{0.899855in}{0.597325in}}%
\pgfpathlineto{\pgfqpoint{0.899855in}{0.594376in}}%
\pgfpathmoveto{\pgfqpoint{0.895314in}{0.597325in}}%
\pgfpathlineto{\pgfqpoint{0.895314in}{0.597325in}}%
\pgfpathlineto{\pgfqpoint{0.895314in}{0.600275in}}%
\pgfpathlineto{\pgfqpoint{0.899855in}{0.600275in}}%
\pgfpathlineto{\pgfqpoint{0.899855in}{0.597325in}}%
\pgfpathmoveto{\pgfqpoint{0.899855in}{0.594376in}}%
\pgfpathlineto{\pgfqpoint{0.899855in}{0.594376in}}%
\pgfpathlineto{\pgfqpoint{0.899855in}{0.597325in}}%
\pgfpathlineto{\pgfqpoint{0.904396in}{0.597325in}}%
\pgfpathlineto{\pgfqpoint{0.904396in}{0.594376in}}%
\pgfpathmoveto{\pgfqpoint{0.899855in}{0.597325in}}%
\pgfpathlineto{\pgfqpoint{0.899855in}{0.597325in}}%
\pgfpathlineto{\pgfqpoint{0.899855in}{0.600275in}}%
\pgfpathlineto{\pgfqpoint{0.904396in}{0.600275in}}%
\pgfpathlineto{\pgfqpoint{0.904396in}{0.597325in}}%
\pgfpathmoveto{\pgfqpoint{0.904396in}{0.597325in}}%
\pgfpathlineto{\pgfqpoint{0.904396in}{0.597325in}}%
\pgfpathlineto{\pgfqpoint{0.904396in}{0.600275in}}%
\pgfpathlineto{\pgfqpoint{0.908937in}{0.600275in}}%
\pgfpathlineto{\pgfqpoint{0.908937in}{0.597325in}}%
\pgfpathmoveto{\pgfqpoint{0.904396in}{0.600275in}}%
\pgfpathlineto{\pgfqpoint{0.904396in}{0.600275in}}%
\pgfpathlineto{\pgfqpoint{0.904396in}{0.603224in}}%
\pgfpathlineto{\pgfqpoint{0.908937in}{0.603224in}}%
\pgfpathlineto{\pgfqpoint{0.908937in}{0.600275in}}%
\pgfpathmoveto{\pgfqpoint{0.904396in}{0.603224in}}%
\pgfpathlineto{\pgfqpoint{0.904396in}{0.603224in}}%
\pgfpathlineto{\pgfqpoint{0.904396in}{0.606173in}}%
\pgfpathlineto{\pgfqpoint{0.908937in}{0.606173in}}%
\pgfpathlineto{\pgfqpoint{0.908937in}{0.603224in}}%
\pgfpathmoveto{\pgfqpoint{0.908937in}{0.600275in}}%
\pgfpathlineto{\pgfqpoint{0.908937in}{0.600275in}}%
\pgfpathlineto{\pgfqpoint{0.908937in}{0.603224in}}%
\pgfpathlineto{\pgfqpoint{0.913478in}{0.603224in}}%
\pgfpathlineto{\pgfqpoint{0.913478in}{0.600275in}}%
\pgfpathmoveto{\pgfqpoint{0.908937in}{0.603224in}}%
\pgfpathlineto{\pgfqpoint{0.908937in}{0.603224in}}%
\pgfpathlineto{\pgfqpoint{0.908937in}{0.606173in}}%
\pgfpathlineto{\pgfqpoint{0.913478in}{0.606173in}}%
\pgfpathlineto{\pgfqpoint{0.913478in}{0.603224in}}%
\pgfpathmoveto{\pgfqpoint{0.913478in}{0.603224in}}%
\pgfpathlineto{\pgfqpoint{0.913478in}{0.603224in}}%
\pgfpathlineto{\pgfqpoint{0.913478in}{0.606173in}}%
\pgfpathlineto{\pgfqpoint{0.918019in}{0.606173in}}%
\pgfpathlineto{\pgfqpoint{0.918019in}{0.603224in}}%
\pgfpathmoveto{\pgfqpoint{0.913478in}{0.606173in}}%
\pgfpathlineto{\pgfqpoint{0.913478in}{0.606173in}}%
\pgfpathlineto{\pgfqpoint{0.913478in}{0.609122in}}%
\pgfpathlineto{\pgfqpoint{0.918019in}{0.609122in}}%
\pgfpathlineto{\pgfqpoint{0.918019in}{0.606173in}}%
\pgfpathmoveto{\pgfqpoint{0.913478in}{0.609122in}}%
\pgfpathlineto{\pgfqpoint{0.913478in}{0.609122in}}%
\pgfpathlineto{\pgfqpoint{0.913478in}{0.612072in}}%
\pgfpathlineto{\pgfqpoint{0.918019in}{0.612072in}}%
\pgfpathlineto{\pgfqpoint{0.918019in}{0.609122in}}%
\pgfpathmoveto{\pgfqpoint{0.918019in}{0.606173in}}%
\pgfpathlineto{\pgfqpoint{0.918019in}{0.606173in}}%
\pgfpathlineto{\pgfqpoint{0.918019in}{0.609122in}}%
\pgfpathlineto{\pgfqpoint{0.922560in}{0.609122in}}%
\pgfpathlineto{\pgfqpoint{0.922560in}{0.606173in}}%
\pgfpathmoveto{\pgfqpoint{0.918019in}{0.609122in}}%
\pgfpathlineto{\pgfqpoint{0.918019in}{0.609122in}}%
\pgfpathlineto{\pgfqpoint{0.918019in}{0.612072in}}%
\pgfpathlineto{\pgfqpoint{0.922560in}{0.612072in}}%
\pgfpathlineto{\pgfqpoint{0.922560in}{0.609122in}}%
\pgfpathmoveto{\pgfqpoint{0.922560in}{0.609122in}}%
\pgfpathlineto{\pgfqpoint{0.922560in}{0.609122in}}%
\pgfpathlineto{\pgfqpoint{0.922560in}{0.612072in}}%
\pgfpathlineto{\pgfqpoint{0.927100in}{0.612072in}}%
\pgfpathlineto{\pgfqpoint{0.927100in}{0.609122in}}%
\pgfpathmoveto{\pgfqpoint{0.922560in}{0.612072in}}%
\pgfpathlineto{\pgfqpoint{0.922560in}{0.612072in}}%
\pgfpathlineto{\pgfqpoint{0.922560in}{0.615021in}}%
\pgfpathlineto{\pgfqpoint{0.927100in}{0.615021in}}%
\pgfpathlineto{\pgfqpoint{0.927100in}{0.612072in}}%
\pgfpathmoveto{\pgfqpoint{0.922560in}{0.615021in}}%
\pgfpathlineto{\pgfqpoint{0.922560in}{0.615021in}}%
\pgfpathlineto{\pgfqpoint{0.922560in}{0.617970in}}%
\pgfpathlineto{\pgfqpoint{0.927100in}{0.617970in}}%
\pgfpathlineto{\pgfqpoint{0.927100in}{0.615021in}}%
\pgfpathmoveto{\pgfqpoint{0.927100in}{0.612072in}}%
\pgfpathlineto{\pgfqpoint{0.927100in}{0.612072in}}%
\pgfpathlineto{\pgfqpoint{0.927100in}{0.615021in}}%
\pgfpathlineto{\pgfqpoint{0.931641in}{0.615021in}}%
\pgfpathlineto{\pgfqpoint{0.931641in}{0.612072in}}%
\pgfpathmoveto{\pgfqpoint{0.927100in}{0.615021in}}%
\pgfpathlineto{\pgfqpoint{0.927100in}{0.615021in}}%
\pgfpathlineto{\pgfqpoint{0.927100in}{0.617970in}}%
\pgfpathlineto{\pgfqpoint{0.931641in}{0.617970in}}%
\pgfpathlineto{\pgfqpoint{0.931641in}{0.615021in}}%
\pgfpathmoveto{\pgfqpoint{0.931641in}{0.615021in}}%
\pgfpathlineto{\pgfqpoint{0.931641in}{0.615021in}}%
\pgfpathlineto{\pgfqpoint{0.931641in}{0.617970in}}%
\pgfpathlineto{\pgfqpoint{0.936182in}{0.617970in}}%
\pgfpathlineto{\pgfqpoint{0.936182in}{0.615021in}}%
\pgfpathmoveto{\pgfqpoint{0.931641in}{0.617970in}}%
\pgfpathlineto{\pgfqpoint{0.931641in}{0.617970in}}%
\pgfpathlineto{\pgfqpoint{0.931641in}{0.620919in}}%
\pgfpathlineto{\pgfqpoint{0.936182in}{0.620919in}}%
\pgfpathlineto{\pgfqpoint{0.936182in}{0.617970in}}%
\pgfpathmoveto{\pgfqpoint{0.931641in}{0.620919in}}%
\pgfpathlineto{\pgfqpoint{0.931641in}{0.620919in}}%
\pgfpathlineto{\pgfqpoint{0.931641in}{0.623869in}}%
\pgfpathlineto{\pgfqpoint{0.936182in}{0.623869in}}%
\pgfpathlineto{\pgfqpoint{0.936182in}{0.620919in}}%
\pgfpathmoveto{\pgfqpoint{0.936182in}{0.617970in}}%
\pgfpathlineto{\pgfqpoint{0.936182in}{0.617970in}}%
\pgfpathlineto{\pgfqpoint{0.936182in}{0.620919in}}%
\pgfpathlineto{\pgfqpoint{0.940723in}{0.620919in}}%
\pgfpathlineto{\pgfqpoint{0.940723in}{0.617970in}}%
\pgfpathmoveto{\pgfqpoint{0.936182in}{0.620919in}}%
\pgfpathlineto{\pgfqpoint{0.936182in}{0.620919in}}%
\pgfpathlineto{\pgfqpoint{0.936182in}{0.623869in}}%
\pgfpathlineto{\pgfqpoint{0.940723in}{0.623869in}}%
\pgfpathlineto{\pgfqpoint{0.940723in}{0.620919in}}%
\pgfpathmoveto{\pgfqpoint{0.940723in}{0.620919in}}%
\pgfpathlineto{\pgfqpoint{0.940723in}{0.620919in}}%
\pgfpathlineto{\pgfqpoint{0.940723in}{0.623869in}}%
\pgfpathlineto{\pgfqpoint{0.945264in}{0.623869in}}%
\pgfpathlineto{\pgfqpoint{0.945264in}{0.620919in}}%
\pgfpathmoveto{\pgfqpoint{0.940723in}{0.623869in}}%
\pgfpathlineto{\pgfqpoint{0.940723in}{0.623869in}}%
\pgfpathlineto{\pgfqpoint{0.940723in}{0.626818in}}%
\pgfpathlineto{\pgfqpoint{0.945264in}{0.626818in}}%
\pgfpathlineto{\pgfqpoint{0.945264in}{0.623869in}}%
\pgfpathmoveto{\pgfqpoint{0.940723in}{0.626818in}}%
\pgfpathlineto{\pgfqpoint{0.940723in}{0.626818in}}%
\pgfpathlineto{\pgfqpoint{0.940723in}{0.629767in}}%
\pgfpathlineto{\pgfqpoint{0.945264in}{0.629767in}}%
\pgfpathlineto{\pgfqpoint{0.945264in}{0.626818in}}%
\pgfpathmoveto{\pgfqpoint{0.945264in}{0.623869in}}%
\pgfpathlineto{\pgfqpoint{0.945264in}{0.623869in}}%
\pgfpathlineto{\pgfqpoint{0.945264in}{0.626818in}}%
\pgfpathlineto{\pgfqpoint{0.949805in}{0.626818in}}%
\pgfpathlineto{\pgfqpoint{0.949805in}{0.623869in}}%
\pgfpathmoveto{\pgfqpoint{0.945264in}{0.626818in}}%
\pgfpathlineto{\pgfqpoint{0.945264in}{0.626818in}}%
\pgfpathlineto{\pgfqpoint{0.945264in}{0.629767in}}%
\pgfpathlineto{\pgfqpoint{0.949805in}{0.629767in}}%
\pgfpathlineto{\pgfqpoint{0.949805in}{0.626818in}}%
\pgfpathmoveto{\pgfqpoint{0.949805in}{0.626818in}}%
\pgfpathlineto{\pgfqpoint{0.949805in}{0.626818in}}%
\pgfpathlineto{\pgfqpoint{0.949805in}{0.629767in}}%
\pgfpathlineto{\pgfqpoint{0.954346in}{0.629767in}}%
\pgfpathlineto{\pgfqpoint{0.954346in}{0.626818in}}%
\pgfpathmoveto{\pgfqpoint{0.949805in}{0.629767in}}%
\pgfpathlineto{\pgfqpoint{0.949805in}{0.629767in}}%
\pgfpathlineto{\pgfqpoint{0.949805in}{0.632716in}}%
\pgfpathlineto{\pgfqpoint{0.954346in}{0.632716in}}%
\pgfpathlineto{\pgfqpoint{0.954346in}{0.629767in}}%
\pgfpathmoveto{\pgfqpoint{0.949805in}{0.632716in}}%
\pgfpathlineto{\pgfqpoint{0.949805in}{0.632716in}}%
\pgfpathlineto{\pgfqpoint{0.949805in}{0.635665in}}%
\pgfpathlineto{\pgfqpoint{0.954346in}{0.635665in}}%
\pgfpathlineto{\pgfqpoint{0.954346in}{0.632716in}}%
\pgfpathmoveto{\pgfqpoint{0.954346in}{0.629767in}}%
\pgfpathlineto{\pgfqpoint{0.954346in}{0.629767in}}%
\pgfpathlineto{\pgfqpoint{0.954346in}{0.632716in}}%
\pgfpathlineto{\pgfqpoint{0.958887in}{0.632716in}}%
\pgfpathlineto{\pgfqpoint{0.958887in}{0.629767in}}%
\pgfpathmoveto{\pgfqpoint{0.954346in}{0.632716in}}%
\pgfpathlineto{\pgfqpoint{0.954346in}{0.632716in}}%
\pgfpathlineto{\pgfqpoint{0.954346in}{0.635665in}}%
\pgfpathlineto{\pgfqpoint{0.958887in}{0.635665in}}%
\pgfpathlineto{\pgfqpoint{0.958887in}{0.632716in}}%
\pgfpathmoveto{\pgfqpoint{0.958887in}{0.632716in}}%
\pgfpathlineto{\pgfqpoint{0.958887in}{0.632716in}}%
\pgfpathlineto{\pgfqpoint{0.958887in}{0.635665in}}%
\pgfpathlineto{\pgfqpoint{0.963428in}{0.635665in}}%
\pgfpathlineto{\pgfqpoint{0.963428in}{0.632716in}}%
\pgfpathmoveto{\pgfqpoint{0.958887in}{0.635665in}}%
\pgfpathlineto{\pgfqpoint{0.958887in}{0.635665in}}%
\pgfpathlineto{\pgfqpoint{0.958887in}{0.638615in}}%
\pgfpathlineto{\pgfqpoint{0.963428in}{0.638615in}}%
\pgfpathlineto{\pgfqpoint{0.963428in}{0.635665in}}%
\pgfpathmoveto{\pgfqpoint{0.958887in}{0.638615in}}%
\pgfpathlineto{\pgfqpoint{0.958887in}{0.638615in}}%
\pgfpathlineto{\pgfqpoint{0.958887in}{0.641564in}}%
\pgfpathlineto{\pgfqpoint{0.963428in}{0.641564in}}%
\pgfpathlineto{\pgfqpoint{0.963428in}{0.638615in}}%
\pgfpathmoveto{\pgfqpoint{0.963428in}{0.635665in}}%
\pgfpathlineto{\pgfqpoint{0.963428in}{0.635665in}}%
\pgfpathlineto{\pgfqpoint{0.963428in}{0.638615in}}%
\pgfpathlineto{\pgfqpoint{0.967968in}{0.638615in}}%
\pgfpathlineto{\pgfqpoint{0.967968in}{0.635665in}}%
\pgfpathmoveto{\pgfqpoint{0.963428in}{0.638615in}}%
\pgfpathlineto{\pgfqpoint{0.963428in}{0.638615in}}%
\pgfpathlineto{\pgfqpoint{0.963428in}{0.641564in}}%
\pgfpathlineto{\pgfqpoint{0.967968in}{0.641564in}}%
\pgfpathlineto{\pgfqpoint{0.967968in}{0.638615in}}%
\pgfpathmoveto{\pgfqpoint{0.967968in}{0.638615in}}%
\pgfpathlineto{\pgfqpoint{0.967968in}{0.638615in}}%
\pgfpathlineto{\pgfqpoint{0.967968in}{0.641564in}}%
\pgfpathlineto{\pgfqpoint{0.972509in}{0.641564in}}%
\pgfpathlineto{\pgfqpoint{0.972509in}{0.638615in}}%
\pgfpathmoveto{\pgfqpoint{0.967968in}{0.641564in}}%
\pgfpathlineto{\pgfqpoint{0.967968in}{0.641564in}}%
\pgfpathlineto{\pgfqpoint{0.967968in}{0.644513in}}%
\pgfpathlineto{\pgfqpoint{0.972509in}{0.644513in}}%
\pgfpathlineto{\pgfqpoint{0.972509in}{0.641564in}}%
\pgfpathmoveto{\pgfqpoint{0.967968in}{0.644513in}}%
\pgfpathlineto{\pgfqpoint{0.967968in}{0.644513in}}%
\pgfpathlineto{\pgfqpoint{0.967968in}{0.647462in}}%
\pgfpathlineto{\pgfqpoint{0.972509in}{0.647462in}}%
\pgfpathlineto{\pgfqpoint{0.972509in}{0.644513in}}%
\pgfpathmoveto{\pgfqpoint{0.972509in}{0.641564in}}%
\pgfpathlineto{\pgfqpoint{0.972509in}{0.641564in}}%
\pgfpathlineto{\pgfqpoint{0.972509in}{0.644513in}}%
\pgfpathlineto{\pgfqpoint{0.977050in}{0.644513in}}%
\pgfpathlineto{\pgfqpoint{0.977050in}{0.641564in}}%
\pgfpathmoveto{\pgfqpoint{0.972509in}{0.644513in}}%
\pgfpathlineto{\pgfqpoint{0.972509in}{0.644513in}}%
\pgfpathlineto{\pgfqpoint{0.972509in}{0.647462in}}%
\pgfpathlineto{\pgfqpoint{0.977050in}{0.647462in}}%
\pgfpathlineto{\pgfqpoint{0.977050in}{0.644513in}}%
\pgfpathmoveto{\pgfqpoint{0.977050in}{0.644513in}}%
\pgfpathlineto{\pgfqpoint{0.977050in}{0.644513in}}%
\pgfpathlineto{\pgfqpoint{0.977050in}{0.647462in}}%
\pgfpathlineto{\pgfqpoint{0.981591in}{0.647462in}}%
\pgfpathlineto{\pgfqpoint{0.981591in}{0.644513in}}%
\pgfpathmoveto{\pgfqpoint{0.977050in}{0.647462in}}%
\pgfpathlineto{\pgfqpoint{0.977050in}{0.647462in}}%
\pgfpathlineto{\pgfqpoint{0.977050in}{0.650412in}}%
\pgfpathlineto{\pgfqpoint{0.981591in}{0.650412in}}%
\pgfpathlineto{\pgfqpoint{0.981591in}{0.647462in}}%
\pgfpathmoveto{\pgfqpoint{0.977050in}{0.650412in}}%
\pgfpathlineto{\pgfqpoint{0.977050in}{0.650412in}}%
\pgfpathlineto{\pgfqpoint{0.977050in}{0.653361in}}%
\pgfpathlineto{\pgfqpoint{0.981591in}{0.653361in}}%
\pgfpathlineto{\pgfqpoint{0.981591in}{0.650412in}}%
\pgfpathmoveto{\pgfqpoint{0.981591in}{0.647462in}}%
\pgfpathlineto{\pgfqpoint{0.981591in}{0.647462in}}%
\pgfpathlineto{\pgfqpoint{0.981591in}{0.650412in}}%
\pgfpathlineto{\pgfqpoint{0.986132in}{0.650412in}}%
\pgfpathlineto{\pgfqpoint{0.986132in}{0.647462in}}%
\pgfpathmoveto{\pgfqpoint{0.981591in}{0.650412in}}%
\pgfpathlineto{\pgfqpoint{0.981591in}{0.650412in}}%
\pgfpathlineto{\pgfqpoint{0.981591in}{0.653361in}}%
\pgfpathlineto{\pgfqpoint{0.986132in}{0.653361in}}%
\pgfpathlineto{\pgfqpoint{0.986132in}{0.650412in}}%
\pgfpathmoveto{\pgfqpoint{0.986132in}{0.650412in}}%
\pgfpathlineto{\pgfqpoint{0.986132in}{0.650412in}}%
\pgfpathlineto{\pgfqpoint{0.986132in}{0.653361in}}%
\pgfpathlineto{\pgfqpoint{0.990673in}{0.653361in}}%
\pgfpathlineto{\pgfqpoint{0.990673in}{0.650412in}}%
\pgfpathmoveto{\pgfqpoint{0.986132in}{0.653361in}}%
\pgfpathlineto{\pgfqpoint{0.986132in}{0.653361in}}%
\pgfpathlineto{\pgfqpoint{0.986132in}{0.656310in}}%
\pgfpathlineto{\pgfqpoint{0.990673in}{0.656310in}}%
\pgfpathlineto{\pgfqpoint{0.990673in}{0.653361in}}%
\pgfpathmoveto{\pgfqpoint{0.986132in}{0.656310in}}%
\pgfpathlineto{\pgfqpoint{0.986132in}{0.656310in}}%
\pgfpathlineto{\pgfqpoint{0.986132in}{0.659259in}}%
\pgfpathlineto{\pgfqpoint{0.990673in}{0.659259in}}%
\pgfpathlineto{\pgfqpoint{0.990673in}{0.656310in}}%
\pgfpathmoveto{\pgfqpoint{0.990673in}{0.653361in}}%
\pgfpathlineto{\pgfqpoint{0.990673in}{0.653361in}}%
\pgfpathlineto{\pgfqpoint{0.990673in}{0.656310in}}%
\pgfpathlineto{\pgfqpoint{0.995214in}{0.656310in}}%
\pgfpathlineto{\pgfqpoint{0.995214in}{0.653361in}}%
\pgfpathmoveto{\pgfqpoint{0.990673in}{0.656310in}}%
\pgfpathlineto{\pgfqpoint{0.990673in}{0.656310in}}%
\pgfpathlineto{\pgfqpoint{0.990673in}{0.659259in}}%
\pgfpathlineto{\pgfqpoint{0.995214in}{0.659259in}}%
\pgfpathlineto{\pgfqpoint{0.995214in}{0.656310in}}%
\pgfpathmoveto{\pgfqpoint{0.995214in}{0.656310in}}%
\pgfpathlineto{\pgfqpoint{0.995214in}{0.656310in}}%
\pgfpathlineto{\pgfqpoint{0.995214in}{0.659259in}}%
\pgfpathlineto{\pgfqpoint{0.999755in}{0.659259in}}%
\pgfpathlineto{\pgfqpoint{0.999755in}{0.656310in}}%
\pgfpathmoveto{\pgfqpoint{0.995214in}{0.659259in}}%
\pgfpathlineto{\pgfqpoint{0.995214in}{0.659259in}}%
\pgfpathlineto{\pgfqpoint{0.995214in}{0.662209in}}%
\pgfpathlineto{\pgfqpoint{0.999755in}{0.662209in}}%
\pgfpathlineto{\pgfqpoint{0.999755in}{0.659259in}}%
\pgfpathmoveto{\pgfqpoint{0.995214in}{0.662209in}}%
\pgfpathlineto{\pgfqpoint{0.995214in}{0.662209in}}%
\pgfpathlineto{\pgfqpoint{0.995214in}{0.665158in}}%
\pgfpathlineto{\pgfqpoint{0.999755in}{0.665158in}}%
\pgfpathlineto{\pgfqpoint{0.999755in}{0.662209in}}%
\pgfpathmoveto{\pgfqpoint{0.999755in}{0.659259in}}%
\pgfpathlineto{\pgfqpoint{0.999755in}{0.659259in}}%
\pgfpathlineto{\pgfqpoint{0.999755in}{0.662209in}}%
\pgfpathlineto{\pgfqpoint{1.004295in}{0.662209in}}%
\pgfpathlineto{\pgfqpoint{1.004295in}{0.659259in}}%
\pgfpathmoveto{\pgfqpoint{0.999755in}{0.662209in}}%
\pgfpathlineto{\pgfqpoint{0.999755in}{0.662209in}}%
\pgfpathlineto{\pgfqpoint{0.999755in}{0.665158in}}%
\pgfpathlineto{\pgfqpoint{1.004295in}{0.665158in}}%
\pgfpathlineto{\pgfqpoint{1.004295in}{0.662209in}}%
\pgfpathmoveto{\pgfqpoint{1.004295in}{0.662209in}}%
\pgfpathlineto{\pgfqpoint{1.004295in}{0.662209in}}%
\pgfpathlineto{\pgfqpoint{1.004295in}{0.665158in}}%
\pgfpathlineto{\pgfqpoint{1.008836in}{0.665158in}}%
\pgfpathlineto{\pgfqpoint{1.008836in}{0.662209in}}%
\pgfpathmoveto{\pgfqpoint{1.004295in}{0.665158in}}%
\pgfpathlineto{\pgfqpoint{1.004295in}{0.665158in}}%
\pgfpathlineto{\pgfqpoint{1.004295in}{0.668107in}}%
\pgfpathlineto{\pgfqpoint{1.008836in}{0.668107in}}%
\pgfpathlineto{\pgfqpoint{1.008836in}{0.665158in}}%
\pgfpathmoveto{\pgfqpoint{1.004295in}{0.668107in}}%
\pgfpathlineto{\pgfqpoint{1.004295in}{0.668107in}}%
\pgfpathlineto{\pgfqpoint{1.004295in}{0.671056in}}%
\pgfpathlineto{\pgfqpoint{1.008836in}{0.671056in}}%
\pgfpathlineto{\pgfqpoint{1.008836in}{0.668107in}}%
\pgfpathmoveto{\pgfqpoint{1.008836in}{0.665158in}}%
\pgfpathlineto{\pgfqpoint{1.008836in}{0.665158in}}%
\pgfpathlineto{\pgfqpoint{1.008836in}{0.668107in}}%
\pgfpathlineto{\pgfqpoint{1.013377in}{0.668107in}}%
\pgfpathlineto{\pgfqpoint{1.013377in}{0.665158in}}%
\pgfpathmoveto{\pgfqpoint{1.008836in}{0.668107in}}%
\pgfpathlineto{\pgfqpoint{1.008836in}{0.668107in}}%
\pgfpathlineto{\pgfqpoint{1.008836in}{0.671056in}}%
\pgfpathlineto{\pgfqpoint{1.013377in}{0.671056in}}%
\pgfpathlineto{\pgfqpoint{1.013377in}{0.668107in}}%
\pgfpathmoveto{\pgfqpoint{1.013377in}{0.668107in}}%
\pgfpathlineto{\pgfqpoint{1.013377in}{0.668107in}}%
\pgfpathlineto{\pgfqpoint{1.013377in}{0.671056in}}%
\pgfpathlineto{\pgfqpoint{1.017918in}{0.671056in}}%
\pgfpathlineto{\pgfqpoint{1.017918in}{0.668107in}}%
\pgfpathmoveto{\pgfqpoint{1.013377in}{0.671056in}}%
\pgfpathlineto{\pgfqpoint{1.013377in}{0.671056in}}%
\pgfpathlineto{\pgfqpoint{1.013377in}{0.674005in}}%
\pgfpathlineto{\pgfqpoint{1.017918in}{0.674005in}}%
\pgfpathlineto{\pgfqpoint{1.017918in}{0.671056in}}%
\pgfpathmoveto{\pgfqpoint{1.013377in}{0.674005in}}%
\pgfpathlineto{\pgfqpoint{1.013377in}{0.674005in}}%
\pgfpathlineto{\pgfqpoint{1.013377in}{0.676955in}}%
\pgfpathlineto{\pgfqpoint{1.017918in}{0.676955in}}%
\pgfpathlineto{\pgfqpoint{1.017918in}{0.674005in}}%
\pgfpathmoveto{\pgfqpoint{1.017918in}{0.671056in}}%
\pgfpathlineto{\pgfqpoint{1.017918in}{0.671056in}}%
\pgfpathlineto{\pgfqpoint{1.017918in}{0.674005in}}%
\pgfpathlineto{\pgfqpoint{1.022459in}{0.674005in}}%
\pgfpathlineto{\pgfqpoint{1.022459in}{0.671056in}}%
\pgfpathmoveto{\pgfqpoint{1.017918in}{0.674005in}}%
\pgfpathlineto{\pgfqpoint{1.017918in}{0.674005in}}%
\pgfpathlineto{\pgfqpoint{1.017918in}{0.676955in}}%
\pgfpathlineto{\pgfqpoint{1.022459in}{0.676955in}}%
\pgfpathlineto{\pgfqpoint{1.022459in}{0.674005in}}%
\pgfpathmoveto{\pgfqpoint{1.022459in}{0.674005in}}%
\pgfpathlineto{\pgfqpoint{1.022459in}{0.674005in}}%
\pgfpathlineto{\pgfqpoint{1.022459in}{0.676955in}}%
\pgfpathlineto{\pgfqpoint{1.027000in}{0.676955in}}%
\pgfpathlineto{\pgfqpoint{1.027000in}{0.674005in}}%
\pgfpathmoveto{\pgfqpoint{1.022459in}{0.676955in}}%
\pgfpathlineto{\pgfqpoint{1.022459in}{0.676955in}}%
\pgfpathlineto{\pgfqpoint{1.022459in}{0.679904in}}%
\pgfpathlineto{\pgfqpoint{1.027000in}{0.679904in}}%
\pgfpathlineto{\pgfqpoint{1.027000in}{0.676955in}}%
\pgfpathmoveto{\pgfqpoint{1.022459in}{0.679904in}}%
\pgfpathlineto{\pgfqpoint{1.022459in}{0.679904in}}%
\pgfpathlineto{\pgfqpoint{1.022459in}{0.682853in}}%
\pgfpathlineto{\pgfqpoint{1.027000in}{0.682853in}}%
\pgfpathlineto{\pgfqpoint{1.027000in}{0.679904in}}%
\pgfpathmoveto{\pgfqpoint{1.027000in}{0.676955in}}%
\pgfpathlineto{\pgfqpoint{1.027000in}{0.676955in}}%
\pgfpathlineto{\pgfqpoint{1.027000in}{0.679904in}}%
\pgfpathlineto{\pgfqpoint{1.031541in}{0.679904in}}%
\pgfpathlineto{\pgfqpoint{1.031541in}{0.676955in}}%
\pgfpathmoveto{\pgfqpoint{1.027000in}{0.679904in}}%
\pgfpathlineto{\pgfqpoint{1.027000in}{0.679904in}}%
\pgfpathlineto{\pgfqpoint{1.027000in}{0.682853in}}%
\pgfpathlineto{\pgfqpoint{1.031541in}{0.682853in}}%
\pgfpathlineto{\pgfqpoint{1.031541in}{0.679904in}}%
\pgfpathmoveto{\pgfqpoint{1.031541in}{0.679904in}}%
\pgfpathlineto{\pgfqpoint{1.031541in}{0.679904in}}%
\pgfpathlineto{\pgfqpoint{1.031541in}{0.682853in}}%
\pgfpathlineto{\pgfqpoint{1.036082in}{0.682853in}}%
\pgfpathlineto{\pgfqpoint{1.036082in}{0.679904in}}%
\pgfpathmoveto{\pgfqpoint{1.031541in}{0.682853in}}%
\pgfpathlineto{\pgfqpoint{1.031541in}{0.682853in}}%
\pgfpathlineto{\pgfqpoint{1.031541in}{0.685802in}}%
\pgfpathlineto{\pgfqpoint{1.036082in}{0.685802in}}%
\pgfpathlineto{\pgfqpoint{1.036082in}{0.682853in}}%
\pgfpathmoveto{\pgfqpoint{1.031541in}{0.685802in}}%
\pgfpathlineto{\pgfqpoint{1.031541in}{0.685802in}}%
\pgfpathlineto{\pgfqpoint{1.031541in}{0.688752in}}%
\pgfpathlineto{\pgfqpoint{1.036082in}{0.688752in}}%
\pgfpathlineto{\pgfqpoint{1.036082in}{0.685802in}}%
\pgfpathmoveto{\pgfqpoint{1.036082in}{0.682853in}}%
\pgfpathlineto{\pgfqpoint{1.036082in}{0.682853in}}%
\pgfpathlineto{\pgfqpoint{1.036082in}{0.685802in}}%
\pgfpathlineto{\pgfqpoint{1.040623in}{0.685802in}}%
\pgfpathlineto{\pgfqpoint{1.040623in}{0.682853in}}%
\pgfpathmoveto{\pgfqpoint{1.036082in}{0.685802in}}%
\pgfpathlineto{\pgfqpoint{1.036082in}{0.685802in}}%
\pgfpathlineto{\pgfqpoint{1.036082in}{0.688752in}}%
\pgfpathlineto{\pgfqpoint{1.040623in}{0.688752in}}%
\pgfpathlineto{\pgfqpoint{1.040623in}{0.685802in}}%
\pgfpathmoveto{\pgfqpoint{1.040623in}{0.685802in}}%
\pgfpathlineto{\pgfqpoint{1.040623in}{0.685802in}}%
\pgfpathlineto{\pgfqpoint{1.040623in}{0.688752in}}%
\pgfpathlineto{\pgfqpoint{1.045164in}{0.688752in}}%
\pgfpathlineto{\pgfqpoint{1.045164in}{0.685802in}}%
\pgfpathmoveto{\pgfqpoint{1.040623in}{0.688752in}}%
\pgfpathlineto{\pgfqpoint{1.040623in}{0.688752in}}%
\pgfpathlineto{\pgfqpoint{1.040623in}{0.691701in}}%
\pgfpathlineto{\pgfqpoint{1.045164in}{0.691701in}}%
\pgfpathlineto{\pgfqpoint{1.045164in}{0.688752in}}%
\pgfpathmoveto{\pgfqpoint{1.040623in}{0.691701in}}%
\pgfpathlineto{\pgfqpoint{1.040623in}{0.691701in}}%
\pgfpathlineto{\pgfqpoint{1.040623in}{0.694650in}}%
\pgfpathlineto{\pgfqpoint{1.045164in}{0.694650in}}%
\pgfpathlineto{\pgfqpoint{1.045164in}{0.691701in}}%
\pgfpathmoveto{\pgfqpoint{1.045164in}{0.688752in}}%
\pgfpathlineto{\pgfqpoint{1.045164in}{0.688752in}}%
\pgfpathlineto{\pgfqpoint{1.045164in}{0.691701in}}%
\pgfpathlineto{\pgfqpoint{1.049705in}{0.691701in}}%
\pgfpathlineto{\pgfqpoint{1.049705in}{0.688752in}}%
\pgfpathmoveto{\pgfqpoint{1.045164in}{0.691701in}}%
\pgfpathlineto{\pgfqpoint{1.045164in}{0.691701in}}%
\pgfpathlineto{\pgfqpoint{1.045164in}{0.694650in}}%
\pgfpathlineto{\pgfqpoint{1.049705in}{0.694650in}}%
\pgfpathlineto{\pgfqpoint{1.049705in}{0.691701in}}%
\pgfpathmoveto{\pgfqpoint{1.049705in}{0.691701in}}%
\pgfpathlineto{\pgfqpoint{1.049705in}{0.691701in}}%
\pgfpathlineto{\pgfqpoint{1.049705in}{0.694650in}}%
\pgfpathlineto{\pgfqpoint{1.054246in}{0.694650in}}%
\pgfpathlineto{\pgfqpoint{1.054246in}{0.691701in}}%
\pgfpathmoveto{\pgfqpoint{1.049705in}{0.694650in}}%
\pgfpathlineto{\pgfqpoint{1.049705in}{0.694650in}}%
\pgfpathlineto{\pgfqpoint{1.049705in}{0.697599in}}%
\pgfpathlineto{\pgfqpoint{1.054246in}{0.697599in}}%
\pgfpathlineto{\pgfqpoint{1.054246in}{0.694650in}}%
\pgfpathmoveto{\pgfqpoint{1.049705in}{0.697599in}}%
\pgfpathlineto{\pgfqpoint{1.049705in}{0.697599in}}%
\pgfpathlineto{\pgfqpoint{1.049705in}{0.700548in}}%
\pgfpathlineto{\pgfqpoint{1.054246in}{0.700548in}}%
\pgfpathlineto{\pgfqpoint{1.054246in}{0.697599in}}%
\pgfpathmoveto{\pgfqpoint{1.054246in}{0.694650in}}%
\pgfpathlineto{\pgfqpoint{1.054246in}{0.694650in}}%
\pgfpathlineto{\pgfqpoint{1.054246in}{0.697599in}}%
\pgfpathlineto{\pgfqpoint{1.058787in}{0.697599in}}%
\pgfpathlineto{\pgfqpoint{1.058787in}{0.694650in}}%
\pgfpathmoveto{\pgfqpoint{1.054246in}{0.697599in}}%
\pgfpathlineto{\pgfqpoint{1.054246in}{0.697599in}}%
\pgfpathlineto{\pgfqpoint{1.054246in}{0.700548in}}%
\pgfpathlineto{\pgfqpoint{1.058787in}{0.700548in}}%
\pgfpathlineto{\pgfqpoint{1.058787in}{0.697599in}}%
\pgfpathmoveto{\pgfqpoint{1.058787in}{0.697599in}}%
\pgfpathlineto{\pgfqpoint{1.058787in}{0.697599in}}%
\pgfpathlineto{\pgfqpoint{1.058787in}{0.700548in}}%
\pgfpathlineto{\pgfqpoint{1.063328in}{0.700548in}}%
\pgfpathlineto{\pgfqpoint{1.063328in}{0.697599in}}%
\pgfpathmoveto{\pgfqpoint{1.058787in}{0.700548in}}%
\pgfpathlineto{\pgfqpoint{1.058787in}{0.700548in}}%
\pgfpathlineto{\pgfqpoint{1.058787in}{0.703497in}}%
\pgfpathlineto{\pgfqpoint{1.063328in}{0.703497in}}%
\pgfpathlineto{\pgfqpoint{1.063328in}{0.700548in}}%
\pgfpathmoveto{\pgfqpoint{1.058787in}{0.703497in}}%
\pgfpathlineto{\pgfqpoint{1.058787in}{0.703497in}}%
\pgfpathlineto{\pgfqpoint{1.058787in}{0.706446in}}%
\pgfpathlineto{\pgfqpoint{1.063328in}{0.706446in}}%
\pgfpathlineto{\pgfqpoint{1.063328in}{0.703497in}}%
\pgfpathmoveto{\pgfqpoint{1.063328in}{0.700548in}}%
\pgfpathlineto{\pgfqpoint{1.063328in}{0.700548in}}%
\pgfpathlineto{\pgfqpoint{1.063328in}{0.703497in}}%
\pgfpathlineto{\pgfqpoint{1.067869in}{0.703497in}}%
\pgfpathlineto{\pgfqpoint{1.067869in}{0.700548in}}%
\pgfpathmoveto{\pgfqpoint{1.063328in}{0.703497in}}%
\pgfpathlineto{\pgfqpoint{1.063328in}{0.703497in}}%
\pgfpathlineto{\pgfqpoint{1.063328in}{0.706446in}}%
\pgfpathlineto{\pgfqpoint{1.067869in}{0.706446in}}%
\pgfpathlineto{\pgfqpoint{1.067869in}{0.703497in}}%
\pgfpathmoveto{\pgfqpoint{1.067869in}{0.703497in}}%
\pgfpathlineto{\pgfqpoint{1.067869in}{0.703497in}}%
\pgfpathlineto{\pgfqpoint{1.067869in}{0.706446in}}%
\pgfpathlineto{\pgfqpoint{1.072410in}{0.706446in}}%
\pgfpathlineto{\pgfqpoint{1.072410in}{0.703497in}}%
\pgfpathmoveto{\pgfqpoint{1.067869in}{0.706446in}}%
\pgfpathlineto{\pgfqpoint{1.067869in}{0.706446in}}%
\pgfpathlineto{\pgfqpoint{1.067869in}{0.709395in}}%
\pgfpathlineto{\pgfqpoint{1.072410in}{0.709395in}}%
\pgfpathlineto{\pgfqpoint{1.072410in}{0.706446in}}%
\pgfpathmoveto{\pgfqpoint{1.067869in}{0.709395in}}%
\pgfpathlineto{\pgfqpoint{1.067869in}{0.709395in}}%
\pgfpathlineto{\pgfqpoint{1.067869in}{0.712344in}}%
\pgfpathlineto{\pgfqpoint{1.072410in}{0.712344in}}%
\pgfpathlineto{\pgfqpoint{1.072410in}{0.709395in}}%
\pgfpathmoveto{\pgfqpoint{1.072410in}{0.706446in}}%
\pgfpathlineto{\pgfqpoint{1.072410in}{0.706446in}}%
\pgfpathlineto{\pgfqpoint{1.072410in}{0.709395in}}%
\pgfpathlineto{\pgfqpoint{1.076951in}{0.709395in}}%
\pgfpathlineto{\pgfqpoint{1.076951in}{0.706446in}}%
\pgfpathmoveto{\pgfqpoint{1.072410in}{0.709395in}}%
\pgfpathlineto{\pgfqpoint{1.072410in}{0.709395in}}%
\pgfpathlineto{\pgfqpoint{1.072410in}{0.712344in}}%
\pgfpathlineto{\pgfqpoint{1.076951in}{0.712344in}}%
\pgfpathlineto{\pgfqpoint{1.076951in}{0.709395in}}%
\pgfpathmoveto{\pgfqpoint{1.076951in}{0.709395in}}%
\pgfpathlineto{\pgfqpoint{1.076951in}{0.709395in}}%
\pgfpathlineto{\pgfqpoint{1.076951in}{0.712344in}}%
\pgfpathlineto{\pgfqpoint{1.081492in}{0.712344in}}%
\pgfpathlineto{\pgfqpoint{1.081492in}{0.709395in}}%
\pgfpathmoveto{\pgfqpoint{1.076951in}{0.712344in}}%
\pgfpathlineto{\pgfqpoint{1.076951in}{0.712344in}}%
\pgfpathlineto{\pgfqpoint{1.076951in}{0.715294in}}%
\pgfpathlineto{\pgfqpoint{1.081492in}{0.715294in}}%
\pgfpathlineto{\pgfqpoint{1.081492in}{0.712344in}}%
\pgfpathmoveto{\pgfqpoint{1.076951in}{0.715294in}}%
\pgfpathlineto{\pgfqpoint{1.076951in}{0.715294in}}%
\pgfpathlineto{\pgfqpoint{1.076951in}{0.718243in}}%
\pgfpathlineto{\pgfqpoint{1.081492in}{0.718243in}}%
\pgfpathlineto{\pgfqpoint{1.081492in}{0.715294in}}%
\pgfpathmoveto{\pgfqpoint{1.081492in}{0.712344in}}%
\pgfpathlineto{\pgfqpoint{1.081492in}{0.712344in}}%
\pgfpathlineto{\pgfqpoint{1.081492in}{0.715294in}}%
\pgfpathlineto{\pgfqpoint{1.086033in}{0.715294in}}%
\pgfpathlineto{\pgfqpoint{1.086033in}{0.712344in}}%
\pgfpathmoveto{\pgfqpoint{1.081492in}{0.715294in}}%
\pgfpathlineto{\pgfqpoint{1.081492in}{0.715294in}}%
\pgfpathlineto{\pgfqpoint{1.081492in}{0.718243in}}%
\pgfpathlineto{\pgfqpoint{1.086033in}{0.718243in}}%
\pgfpathlineto{\pgfqpoint{1.086033in}{0.715294in}}%
\pgfpathmoveto{\pgfqpoint{1.086033in}{0.715294in}}%
\pgfpathlineto{\pgfqpoint{1.086033in}{0.715294in}}%
\pgfpathlineto{\pgfqpoint{1.086033in}{0.718243in}}%
\pgfpathlineto{\pgfqpoint{1.090574in}{0.718243in}}%
\pgfpathlineto{\pgfqpoint{1.090574in}{0.715294in}}%
\pgfpathmoveto{\pgfqpoint{1.086033in}{0.718243in}}%
\pgfpathlineto{\pgfqpoint{1.086033in}{0.718243in}}%
\pgfpathlineto{\pgfqpoint{1.086033in}{0.721192in}}%
\pgfpathlineto{\pgfqpoint{1.090574in}{0.721192in}}%
\pgfpathlineto{\pgfqpoint{1.090574in}{0.718243in}}%
\pgfpathmoveto{\pgfqpoint{1.086033in}{0.721192in}}%
\pgfpathlineto{\pgfqpoint{1.086033in}{0.721192in}}%
\pgfpathlineto{\pgfqpoint{1.086033in}{0.724141in}}%
\pgfpathlineto{\pgfqpoint{1.090574in}{0.724141in}}%
\pgfpathlineto{\pgfqpoint{1.090574in}{0.721192in}}%
\pgfpathmoveto{\pgfqpoint{1.090574in}{0.718243in}}%
\pgfpathlineto{\pgfqpoint{1.090574in}{0.718243in}}%
\pgfpathlineto{\pgfqpoint{1.090574in}{0.721192in}}%
\pgfpathlineto{\pgfqpoint{1.095115in}{0.721192in}}%
\pgfpathlineto{\pgfqpoint{1.095115in}{0.718243in}}%
\pgfpathmoveto{\pgfqpoint{1.090574in}{0.721192in}}%
\pgfpathlineto{\pgfqpoint{1.090574in}{0.721192in}}%
\pgfpathlineto{\pgfqpoint{1.090574in}{0.724141in}}%
\pgfpathlineto{\pgfqpoint{1.095115in}{0.724141in}}%
\pgfpathlineto{\pgfqpoint{1.095115in}{0.721192in}}%
\pgfpathmoveto{\pgfqpoint{1.095115in}{0.721192in}}%
\pgfpathlineto{\pgfqpoint{1.095115in}{0.721192in}}%
\pgfpathlineto{\pgfqpoint{1.095115in}{0.724141in}}%
\pgfpathlineto{\pgfqpoint{1.099656in}{0.724141in}}%
\pgfpathlineto{\pgfqpoint{1.099656in}{0.721192in}}%
\pgfpathmoveto{\pgfqpoint{1.095115in}{0.724141in}}%
\pgfpathlineto{\pgfqpoint{1.095115in}{0.724141in}}%
\pgfpathlineto{\pgfqpoint{1.095115in}{0.727090in}}%
\pgfpathlineto{\pgfqpoint{1.099656in}{0.727090in}}%
\pgfpathlineto{\pgfqpoint{1.099656in}{0.724141in}}%
\pgfpathmoveto{\pgfqpoint{1.095115in}{0.727090in}}%
\pgfpathlineto{\pgfqpoint{1.095115in}{0.727090in}}%
\pgfpathlineto{\pgfqpoint{1.095115in}{0.730039in}}%
\pgfpathlineto{\pgfqpoint{1.099656in}{0.730039in}}%
\pgfpathlineto{\pgfqpoint{1.099656in}{0.727090in}}%
\pgfpathmoveto{\pgfqpoint{1.099656in}{0.724141in}}%
\pgfpathlineto{\pgfqpoint{1.099656in}{0.724141in}}%
\pgfpathlineto{\pgfqpoint{1.099656in}{0.727090in}}%
\pgfpathlineto{\pgfqpoint{1.104197in}{0.727090in}}%
\pgfpathlineto{\pgfqpoint{1.104197in}{0.724141in}}%
\pgfpathmoveto{\pgfqpoint{1.099656in}{0.727090in}}%
\pgfpathlineto{\pgfqpoint{1.099656in}{0.727090in}}%
\pgfpathlineto{\pgfqpoint{1.099656in}{0.730039in}}%
\pgfpathlineto{\pgfqpoint{1.104197in}{0.730039in}}%
\pgfpathlineto{\pgfqpoint{1.104197in}{0.727090in}}%
\pgfpathmoveto{\pgfqpoint{1.104197in}{0.727090in}}%
\pgfpathlineto{\pgfqpoint{1.104197in}{0.727090in}}%
\pgfpathlineto{\pgfqpoint{1.104197in}{0.730039in}}%
\pgfpathlineto{\pgfqpoint{1.108738in}{0.730039in}}%
\pgfpathlineto{\pgfqpoint{1.108738in}{0.727090in}}%
\pgfpathmoveto{\pgfqpoint{1.104197in}{0.730039in}}%
\pgfpathlineto{\pgfqpoint{1.104197in}{0.730039in}}%
\pgfpathlineto{\pgfqpoint{1.104197in}{0.732988in}}%
\pgfpathlineto{\pgfqpoint{1.108738in}{0.732988in}}%
\pgfpathlineto{\pgfqpoint{1.108738in}{0.730039in}}%
\pgfpathmoveto{\pgfqpoint{1.104197in}{0.732988in}}%
\pgfpathlineto{\pgfqpoint{1.104197in}{0.732988in}}%
\pgfpathlineto{\pgfqpoint{1.104197in}{0.735937in}}%
\pgfpathlineto{\pgfqpoint{1.108738in}{0.735937in}}%
\pgfpathlineto{\pgfqpoint{1.108738in}{0.732988in}}%
\pgfpathmoveto{\pgfqpoint{1.108738in}{0.730039in}}%
\pgfpathlineto{\pgfqpoint{1.108738in}{0.730039in}}%
\pgfpathlineto{\pgfqpoint{1.108738in}{0.732988in}}%
\pgfpathlineto{\pgfqpoint{1.113279in}{0.732988in}}%
\pgfpathlineto{\pgfqpoint{1.113279in}{0.730039in}}%
\pgfpathmoveto{\pgfqpoint{1.108738in}{0.732988in}}%
\pgfpathlineto{\pgfqpoint{1.108738in}{0.732988in}}%
\pgfpathlineto{\pgfqpoint{1.108738in}{0.735937in}}%
\pgfpathlineto{\pgfqpoint{1.113279in}{0.735937in}}%
\pgfpathlineto{\pgfqpoint{1.113279in}{0.732988in}}%
\pgfpathmoveto{\pgfqpoint{1.113279in}{0.732988in}}%
\pgfpathlineto{\pgfqpoint{1.113279in}{0.732988in}}%
\pgfpathlineto{\pgfqpoint{1.113279in}{0.735937in}}%
\pgfpathlineto{\pgfqpoint{1.117820in}{0.735937in}}%
\pgfpathlineto{\pgfqpoint{1.117820in}{0.732988in}}%
\pgfpathmoveto{\pgfqpoint{1.113279in}{0.735937in}}%
\pgfpathlineto{\pgfqpoint{1.113279in}{0.735937in}}%
\pgfpathlineto{\pgfqpoint{1.113279in}{0.738886in}}%
\pgfpathlineto{\pgfqpoint{1.117820in}{0.738886in}}%
\pgfpathlineto{\pgfqpoint{1.117820in}{0.735937in}}%
\pgfpathmoveto{\pgfqpoint{1.113279in}{0.738886in}}%
\pgfpathlineto{\pgfqpoint{1.113279in}{0.738886in}}%
\pgfpathlineto{\pgfqpoint{1.113279in}{0.741836in}}%
\pgfpathlineto{\pgfqpoint{1.117820in}{0.741836in}}%
\pgfpathlineto{\pgfqpoint{1.117820in}{0.738886in}}%
\pgfpathmoveto{\pgfqpoint{1.117820in}{0.735937in}}%
\pgfpathlineto{\pgfqpoint{1.117820in}{0.735937in}}%
\pgfpathlineto{\pgfqpoint{1.117820in}{0.738886in}}%
\pgfpathlineto{\pgfqpoint{1.122361in}{0.738886in}}%
\pgfpathlineto{\pgfqpoint{1.122361in}{0.735937in}}%
\pgfpathmoveto{\pgfqpoint{1.117820in}{0.738886in}}%
\pgfpathlineto{\pgfqpoint{1.117820in}{0.738886in}}%
\pgfpathlineto{\pgfqpoint{1.117820in}{0.741836in}}%
\pgfpathlineto{\pgfqpoint{1.122361in}{0.741836in}}%
\pgfpathlineto{\pgfqpoint{1.122361in}{0.738886in}}%
\pgfpathmoveto{\pgfqpoint{1.122361in}{0.738886in}}%
\pgfpathlineto{\pgfqpoint{1.122361in}{0.738886in}}%
\pgfpathlineto{\pgfqpoint{1.122361in}{0.741836in}}%
\pgfpathlineto{\pgfqpoint{1.126902in}{0.741836in}}%
\pgfpathlineto{\pgfqpoint{1.126902in}{0.738886in}}%
\pgfpathmoveto{\pgfqpoint{1.122361in}{0.741836in}}%
\pgfpathlineto{\pgfqpoint{1.122361in}{0.741836in}}%
\pgfpathlineto{\pgfqpoint{1.122361in}{0.744785in}}%
\pgfpathlineto{\pgfqpoint{1.126902in}{0.744785in}}%
\pgfpathlineto{\pgfqpoint{1.126902in}{0.741836in}}%
\pgfpathmoveto{\pgfqpoint{1.122361in}{0.744785in}}%
\pgfpathlineto{\pgfqpoint{1.122361in}{0.744785in}}%
\pgfpathlineto{\pgfqpoint{1.122361in}{0.747734in}}%
\pgfpathlineto{\pgfqpoint{1.126902in}{0.747734in}}%
\pgfpathlineto{\pgfqpoint{1.126902in}{0.744785in}}%
\pgfpathmoveto{\pgfqpoint{1.126902in}{0.741836in}}%
\pgfpathlineto{\pgfqpoint{1.126902in}{0.741836in}}%
\pgfpathlineto{\pgfqpoint{1.126902in}{0.744785in}}%
\pgfpathlineto{\pgfqpoint{1.131443in}{0.744785in}}%
\pgfpathlineto{\pgfqpoint{1.131443in}{0.741836in}}%
\pgfpathmoveto{\pgfqpoint{1.126902in}{0.744785in}}%
\pgfpathlineto{\pgfqpoint{1.126902in}{0.744785in}}%
\pgfpathlineto{\pgfqpoint{1.126902in}{0.747734in}}%
\pgfpathlineto{\pgfqpoint{1.131443in}{0.747734in}}%
\pgfpathlineto{\pgfqpoint{1.131443in}{0.744785in}}%
\pgfpathmoveto{\pgfqpoint{1.131443in}{0.744785in}}%
\pgfpathlineto{\pgfqpoint{1.131443in}{0.744785in}}%
\pgfpathlineto{\pgfqpoint{1.131443in}{0.747734in}}%
\pgfpathlineto{\pgfqpoint{1.135984in}{0.747734in}}%
\pgfpathlineto{\pgfqpoint{1.135984in}{0.744785in}}%
\pgfpathmoveto{\pgfqpoint{1.131443in}{0.747734in}}%
\pgfpathlineto{\pgfqpoint{1.131443in}{0.747734in}}%
\pgfpathlineto{\pgfqpoint{1.131443in}{0.750683in}}%
\pgfpathlineto{\pgfqpoint{1.135984in}{0.750683in}}%
\pgfpathlineto{\pgfqpoint{1.135984in}{0.747734in}}%
\pgfpathmoveto{\pgfqpoint{1.131443in}{0.750683in}}%
\pgfpathlineto{\pgfqpoint{1.131443in}{0.750683in}}%
\pgfpathlineto{\pgfqpoint{1.131443in}{0.753632in}}%
\pgfpathlineto{\pgfqpoint{1.135984in}{0.753632in}}%
\pgfpathlineto{\pgfqpoint{1.135984in}{0.750683in}}%
\pgfpathmoveto{\pgfqpoint{1.135984in}{0.747734in}}%
\pgfpathlineto{\pgfqpoint{1.135984in}{0.747734in}}%
\pgfpathlineto{\pgfqpoint{1.135984in}{0.750683in}}%
\pgfpathlineto{\pgfqpoint{1.140525in}{0.750683in}}%
\pgfpathlineto{\pgfqpoint{1.140525in}{0.747734in}}%
\pgfpathmoveto{\pgfqpoint{1.135984in}{0.750683in}}%
\pgfpathlineto{\pgfqpoint{1.135984in}{0.750683in}}%
\pgfpathlineto{\pgfqpoint{1.135984in}{0.753632in}}%
\pgfpathlineto{\pgfqpoint{1.140525in}{0.753632in}}%
\pgfpathlineto{\pgfqpoint{1.140525in}{0.750683in}}%
\pgfpathmoveto{\pgfqpoint{1.140525in}{0.750683in}}%
\pgfpathlineto{\pgfqpoint{1.140525in}{0.750683in}}%
\pgfpathlineto{\pgfqpoint{1.140525in}{0.753632in}}%
\pgfpathlineto{\pgfqpoint{1.145066in}{0.753632in}}%
\pgfpathlineto{\pgfqpoint{1.145066in}{0.750683in}}%
\pgfpathmoveto{\pgfqpoint{1.140525in}{0.753632in}}%
\pgfpathlineto{\pgfqpoint{1.140525in}{0.753632in}}%
\pgfpathlineto{\pgfqpoint{1.140525in}{0.756581in}}%
\pgfpathlineto{\pgfqpoint{1.145066in}{0.756581in}}%
\pgfpathlineto{\pgfqpoint{1.145066in}{0.753632in}}%
\pgfpathmoveto{\pgfqpoint{1.140525in}{0.756581in}}%
\pgfpathlineto{\pgfqpoint{1.140525in}{0.756581in}}%
\pgfpathlineto{\pgfqpoint{1.140525in}{0.759530in}}%
\pgfpathlineto{\pgfqpoint{1.145066in}{0.759530in}}%
\pgfpathlineto{\pgfqpoint{1.145066in}{0.756581in}}%
\pgfpathmoveto{\pgfqpoint{1.145066in}{0.753632in}}%
\pgfpathlineto{\pgfqpoint{1.145066in}{0.753632in}}%
\pgfpathlineto{\pgfqpoint{1.145066in}{0.756581in}}%
\pgfpathlineto{\pgfqpoint{1.149607in}{0.756581in}}%
\pgfpathlineto{\pgfqpoint{1.149607in}{0.753632in}}%
\pgfpathmoveto{\pgfqpoint{1.145066in}{0.756581in}}%
\pgfpathlineto{\pgfqpoint{1.145066in}{0.756581in}}%
\pgfpathlineto{\pgfqpoint{1.145066in}{0.759530in}}%
\pgfpathlineto{\pgfqpoint{1.149607in}{0.759530in}}%
\pgfpathlineto{\pgfqpoint{1.149607in}{0.756581in}}%
\pgfpathmoveto{\pgfqpoint{1.149607in}{0.756581in}}%
\pgfpathlineto{\pgfqpoint{1.149607in}{0.756581in}}%
\pgfpathlineto{\pgfqpoint{1.149607in}{0.759530in}}%
\pgfpathlineto{\pgfqpoint{1.154148in}{0.759530in}}%
\pgfpathlineto{\pgfqpoint{1.154148in}{0.756581in}}%
\pgfpathmoveto{\pgfqpoint{1.149607in}{0.759530in}}%
\pgfpathlineto{\pgfqpoint{1.149607in}{0.759530in}}%
\pgfpathlineto{\pgfqpoint{1.149607in}{0.762479in}}%
\pgfpathlineto{\pgfqpoint{1.154148in}{0.762479in}}%
\pgfpathlineto{\pgfqpoint{1.154148in}{0.759530in}}%
\pgfpathmoveto{\pgfqpoint{1.149607in}{0.762479in}}%
\pgfpathlineto{\pgfqpoint{1.149607in}{0.762479in}}%
\pgfpathlineto{\pgfqpoint{1.149607in}{0.765428in}}%
\pgfpathlineto{\pgfqpoint{1.154148in}{0.765428in}}%
\pgfpathlineto{\pgfqpoint{1.154148in}{0.762479in}}%
\pgfpathmoveto{\pgfqpoint{1.154148in}{0.759530in}}%
\pgfpathlineto{\pgfqpoint{1.154148in}{0.759530in}}%
\pgfpathlineto{\pgfqpoint{1.154148in}{0.762479in}}%
\pgfpathlineto{\pgfqpoint{1.158689in}{0.762479in}}%
\pgfpathlineto{\pgfqpoint{1.158689in}{0.759530in}}%
\pgfpathmoveto{\pgfqpoint{1.154148in}{0.762479in}}%
\pgfpathlineto{\pgfqpoint{1.154148in}{0.762479in}}%
\pgfpathlineto{\pgfqpoint{1.154148in}{0.765428in}}%
\pgfpathlineto{\pgfqpoint{1.158689in}{0.765428in}}%
\pgfpathlineto{\pgfqpoint{1.158689in}{0.762479in}}%
\pgfpathmoveto{\pgfqpoint{1.158689in}{0.762479in}}%
\pgfpathlineto{\pgfqpoint{1.158689in}{0.762479in}}%
\pgfpathlineto{\pgfqpoint{1.158689in}{0.765428in}}%
\pgfpathlineto{\pgfqpoint{1.163230in}{0.765428in}}%
\pgfpathlineto{\pgfqpoint{1.163230in}{0.762479in}}%
\pgfpathmoveto{\pgfqpoint{1.158689in}{0.765428in}}%
\pgfpathlineto{\pgfqpoint{1.158689in}{0.765428in}}%
\pgfpathlineto{\pgfqpoint{1.158689in}{0.768377in}}%
\pgfpathlineto{\pgfqpoint{1.163230in}{0.768377in}}%
\pgfpathlineto{\pgfqpoint{1.163230in}{0.765428in}}%
\pgfpathmoveto{\pgfqpoint{1.158689in}{0.768377in}}%
\pgfpathlineto{\pgfqpoint{1.158689in}{0.768377in}}%
\pgfpathlineto{\pgfqpoint{1.158689in}{0.771327in}}%
\pgfpathlineto{\pgfqpoint{1.163230in}{0.771327in}}%
\pgfpathlineto{\pgfqpoint{1.163230in}{0.768377in}}%
\pgfpathmoveto{\pgfqpoint{1.163230in}{0.768377in}}%
\pgfpathlineto{\pgfqpoint{1.163230in}{0.768377in}}%
\pgfpathlineto{\pgfqpoint{1.163230in}{0.771327in}}%
\pgfpathlineto{\pgfqpoint{1.167771in}{0.771327in}}%
\pgfpathlineto{\pgfqpoint{1.167771in}{0.768377in}}%
\pgfpathmoveto{\pgfqpoint{1.158689in}{0.771327in}}%
\pgfpathlineto{\pgfqpoint{1.158689in}{0.771327in}}%
\pgfpathlineto{\pgfqpoint{1.158689in}{0.774276in}}%
\pgfpathlineto{\pgfqpoint{1.163230in}{0.774276in}}%
\pgfpathlineto{\pgfqpoint{1.163230in}{0.771327in}}%
\pgfpathmoveto{\pgfqpoint{1.158689in}{0.774276in}}%
\pgfpathlineto{\pgfqpoint{1.158689in}{0.774276in}}%
\pgfpathlineto{\pgfqpoint{1.158689in}{0.777225in}}%
\pgfpathlineto{\pgfqpoint{1.163230in}{0.777225in}}%
\pgfpathlineto{\pgfqpoint{1.163230in}{0.774276in}}%
\pgfpathmoveto{\pgfqpoint{1.163230in}{0.771327in}}%
\pgfpathlineto{\pgfqpoint{1.163230in}{0.771327in}}%
\pgfpathlineto{\pgfqpoint{1.163230in}{0.774276in}}%
\pgfpathlineto{\pgfqpoint{1.167771in}{0.774276in}}%
\pgfpathlineto{\pgfqpoint{1.167771in}{0.771327in}}%
\pgfpathmoveto{\pgfqpoint{1.163230in}{0.774276in}}%
\pgfpathlineto{\pgfqpoint{1.163230in}{0.774276in}}%
\pgfpathlineto{\pgfqpoint{1.163230in}{0.777225in}}%
\pgfpathlineto{\pgfqpoint{1.167771in}{0.777225in}}%
\pgfpathlineto{\pgfqpoint{1.167771in}{0.774276in}}%
\pgfpathmoveto{\pgfqpoint{1.167771in}{0.771327in}}%
\pgfpathlineto{\pgfqpoint{1.167771in}{0.771327in}}%
\pgfpathlineto{\pgfqpoint{1.167771in}{0.774276in}}%
\pgfpathlineto{\pgfqpoint{1.172312in}{0.774276in}}%
\pgfpathlineto{\pgfqpoint{1.172312in}{0.771327in}}%
\pgfpathmoveto{\pgfqpoint{1.167771in}{0.774276in}}%
\pgfpathlineto{\pgfqpoint{1.167771in}{0.774276in}}%
\pgfpathlineto{\pgfqpoint{1.167771in}{0.777225in}}%
\pgfpathlineto{\pgfqpoint{1.172312in}{0.777225in}}%
\pgfpathlineto{\pgfqpoint{1.172312in}{0.774276in}}%
\pgfpathmoveto{\pgfqpoint{1.172312in}{0.774276in}}%
\pgfpathlineto{\pgfqpoint{1.172312in}{0.774276in}}%
\pgfpathlineto{\pgfqpoint{1.172312in}{0.777225in}}%
\pgfpathlineto{\pgfqpoint{1.176853in}{0.777225in}}%
\pgfpathlineto{\pgfqpoint{1.176853in}{0.774276in}}%
\pgfpathmoveto{\pgfqpoint{1.167771in}{0.777225in}}%
\pgfpathlineto{\pgfqpoint{1.167771in}{0.777225in}}%
\pgfpathlineto{\pgfqpoint{1.167771in}{0.780174in}}%
\pgfpathlineto{\pgfqpoint{1.172312in}{0.780174in}}%
\pgfpathlineto{\pgfqpoint{1.172312in}{0.777225in}}%
\pgfpathmoveto{\pgfqpoint{1.167771in}{0.780174in}}%
\pgfpathlineto{\pgfqpoint{1.167771in}{0.780174in}}%
\pgfpathlineto{\pgfqpoint{1.167771in}{0.783123in}}%
\pgfpathlineto{\pgfqpoint{1.172312in}{0.783123in}}%
\pgfpathlineto{\pgfqpoint{1.172312in}{0.780174in}}%
\pgfpathmoveto{\pgfqpoint{1.172312in}{0.777225in}}%
\pgfpathlineto{\pgfqpoint{1.172312in}{0.777225in}}%
\pgfpathlineto{\pgfqpoint{1.172312in}{0.780174in}}%
\pgfpathlineto{\pgfqpoint{1.176853in}{0.780174in}}%
\pgfpathlineto{\pgfqpoint{1.176853in}{0.777225in}}%
\pgfpathmoveto{\pgfqpoint{1.172312in}{0.780174in}}%
\pgfpathlineto{\pgfqpoint{1.172312in}{0.780174in}}%
\pgfpathlineto{\pgfqpoint{1.172312in}{0.783123in}}%
\pgfpathlineto{\pgfqpoint{1.176853in}{0.783123in}}%
\pgfpathlineto{\pgfqpoint{1.176853in}{0.780174in}}%
\pgfpathmoveto{\pgfqpoint{1.176853in}{0.777225in}}%
\pgfpathlineto{\pgfqpoint{1.176853in}{0.777225in}}%
\pgfpathlineto{\pgfqpoint{1.176853in}{0.780174in}}%
\pgfpathlineto{\pgfqpoint{1.181394in}{0.780174in}}%
\pgfpathlineto{\pgfqpoint{1.181394in}{0.777225in}}%
\pgfpathmoveto{\pgfqpoint{1.176853in}{0.780174in}}%
\pgfpathlineto{\pgfqpoint{1.176853in}{0.780174in}}%
\pgfpathlineto{\pgfqpoint{1.176853in}{0.783123in}}%
\pgfpathlineto{\pgfqpoint{1.181394in}{0.783123in}}%
\pgfpathlineto{\pgfqpoint{1.181394in}{0.780174in}}%
\pgfpathmoveto{\pgfqpoint{1.181394in}{0.780174in}}%
\pgfpathlineto{\pgfqpoint{1.181394in}{0.780174in}}%
\pgfpathlineto{\pgfqpoint{1.181394in}{0.783123in}}%
\pgfpathlineto{\pgfqpoint{1.185936in}{0.783123in}}%
\pgfpathlineto{\pgfqpoint{1.185936in}{0.780174in}}%
\pgfpathmoveto{\pgfqpoint{1.176853in}{0.783123in}}%
\pgfpathlineto{\pgfqpoint{1.176853in}{0.783123in}}%
\pgfpathlineto{\pgfqpoint{1.176853in}{0.786072in}}%
\pgfpathlineto{\pgfqpoint{1.181394in}{0.786072in}}%
\pgfpathlineto{\pgfqpoint{1.181394in}{0.783123in}}%
\pgfpathmoveto{\pgfqpoint{1.176853in}{0.786072in}}%
\pgfpathlineto{\pgfqpoint{1.176853in}{0.786072in}}%
\pgfpathlineto{\pgfqpoint{1.176853in}{0.789021in}}%
\pgfpathlineto{\pgfqpoint{1.181394in}{0.789021in}}%
\pgfpathlineto{\pgfqpoint{1.181394in}{0.786072in}}%
\pgfpathmoveto{\pgfqpoint{1.181394in}{0.783123in}}%
\pgfpathlineto{\pgfqpoint{1.181394in}{0.783123in}}%
\pgfpathlineto{\pgfqpoint{1.181394in}{0.786072in}}%
\pgfpathlineto{\pgfqpoint{1.185936in}{0.786072in}}%
\pgfpathlineto{\pgfqpoint{1.185936in}{0.783123in}}%
\pgfpathmoveto{\pgfqpoint{1.181394in}{0.786072in}}%
\pgfpathlineto{\pgfqpoint{1.181394in}{0.786072in}}%
\pgfpathlineto{\pgfqpoint{1.181394in}{0.789021in}}%
\pgfpathlineto{\pgfqpoint{1.185936in}{0.789021in}}%
\pgfpathlineto{\pgfqpoint{1.185936in}{0.786072in}}%
\pgfpathmoveto{\pgfqpoint{1.185936in}{0.783123in}}%
\pgfpathlineto{\pgfqpoint{1.185936in}{0.783123in}}%
\pgfpathlineto{\pgfqpoint{1.185936in}{0.786072in}}%
\pgfpathlineto{\pgfqpoint{1.190477in}{0.786072in}}%
\pgfpathlineto{\pgfqpoint{1.190477in}{0.783123in}}%
\pgfpathmoveto{\pgfqpoint{1.185936in}{0.786072in}}%
\pgfpathlineto{\pgfqpoint{1.185936in}{0.786072in}}%
\pgfpathlineto{\pgfqpoint{1.185936in}{0.789021in}}%
\pgfpathlineto{\pgfqpoint{1.190477in}{0.789021in}}%
\pgfpathlineto{\pgfqpoint{1.190477in}{0.786072in}}%
\pgfpathmoveto{\pgfqpoint{1.190477in}{0.786072in}}%
\pgfpathlineto{\pgfqpoint{1.190477in}{0.786072in}}%
\pgfpathlineto{\pgfqpoint{1.190477in}{0.789021in}}%
\pgfpathlineto{\pgfqpoint{1.195018in}{0.789021in}}%
\pgfpathlineto{\pgfqpoint{1.195018in}{0.786072in}}%
\pgfpathmoveto{\pgfqpoint{1.185936in}{0.789021in}}%
\pgfpathlineto{\pgfqpoint{1.185936in}{0.789021in}}%
\pgfpathlineto{\pgfqpoint{1.185936in}{0.791971in}}%
\pgfpathlineto{\pgfqpoint{1.190477in}{0.791971in}}%
\pgfpathlineto{\pgfqpoint{1.190477in}{0.789021in}}%
\pgfpathmoveto{\pgfqpoint{1.185936in}{0.791971in}}%
\pgfpathlineto{\pgfqpoint{1.185936in}{0.791971in}}%
\pgfpathlineto{\pgfqpoint{1.185936in}{0.794920in}}%
\pgfpathlineto{\pgfqpoint{1.190477in}{0.794920in}}%
\pgfpathlineto{\pgfqpoint{1.190477in}{0.791971in}}%
\pgfpathmoveto{\pgfqpoint{1.190477in}{0.789021in}}%
\pgfpathlineto{\pgfqpoint{1.190477in}{0.789021in}}%
\pgfpathlineto{\pgfqpoint{1.190477in}{0.791971in}}%
\pgfpathlineto{\pgfqpoint{1.195018in}{0.791971in}}%
\pgfpathlineto{\pgfqpoint{1.195018in}{0.789021in}}%
\pgfpathmoveto{\pgfqpoint{1.190477in}{0.791971in}}%
\pgfpathlineto{\pgfqpoint{1.190477in}{0.791971in}}%
\pgfpathlineto{\pgfqpoint{1.190477in}{0.794920in}}%
\pgfpathlineto{\pgfqpoint{1.195018in}{0.794920in}}%
\pgfpathlineto{\pgfqpoint{1.195018in}{0.791971in}}%
\pgfpathmoveto{\pgfqpoint{1.195018in}{0.789021in}}%
\pgfpathlineto{\pgfqpoint{1.195018in}{0.789021in}}%
\pgfpathlineto{\pgfqpoint{1.195018in}{0.791971in}}%
\pgfpathlineto{\pgfqpoint{1.199559in}{0.791971in}}%
\pgfpathlineto{\pgfqpoint{1.199559in}{0.789021in}}%
\pgfpathmoveto{\pgfqpoint{1.195018in}{0.791971in}}%
\pgfpathlineto{\pgfqpoint{1.195018in}{0.791971in}}%
\pgfpathlineto{\pgfqpoint{1.195018in}{0.794920in}}%
\pgfpathlineto{\pgfqpoint{1.199559in}{0.794920in}}%
\pgfpathlineto{\pgfqpoint{1.199559in}{0.791971in}}%
\pgfpathmoveto{\pgfqpoint{1.199559in}{0.791971in}}%
\pgfpathlineto{\pgfqpoint{1.199559in}{0.791971in}}%
\pgfpathlineto{\pgfqpoint{1.199559in}{0.794920in}}%
\pgfpathlineto{\pgfqpoint{1.204100in}{0.794920in}}%
\pgfpathlineto{\pgfqpoint{1.204100in}{0.791971in}}%
\pgfpathmoveto{\pgfqpoint{1.195018in}{0.794920in}}%
\pgfpathlineto{\pgfqpoint{1.195018in}{0.794920in}}%
\pgfpathlineto{\pgfqpoint{1.195018in}{0.797869in}}%
\pgfpathlineto{\pgfqpoint{1.199559in}{0.797869in}}%
\pgfpathlineto{\pgfqpoint{1.199559in}{0.794920in}}%
\pgfpathmoveto{\pgfqpoint{1.195018in}{0.797869in}}%
\pgfpathlineto{\pgfqpoint{1.195018in}{0.797869in}}%
\pgfpathlineto{\pgfqpoint{1.195018in}{0.800818in}}%
\pgfpathlineto{\pgfqpoint{1.199559in}{0.800818in}}%
\pgfpathlineto{\pgfqpoint{1.199559in}{0.797869in}}%
\pgfpathmoveto{\pgfqpoint{1.199559in}{0.794920in}}%
\pgfpathlineto{\pgfqpoint{1.199559in}{0.794920in}}%
\pgfpathlineto{\pgfqpoint{1.199559in}{0.797869in}}%
\pgfpathlineto{\pgfqpoint{1.204100in}{0.797869in}}%
\pgfpathlineto{\pgfqpoint{1.204100in}{0.794920in}}%
\pgfpathmoveto{\pgfqpoint{1.199559in}{0.797869in}}%
\pgfpathlineto{\pgfqpoint{1.199559in}{0.797869in}}%
\pgfpathlineto{\pgfqpoint{1.199559in}{0.800818in}}%
\pgfpathlineto{\pgfqpoint{1.204100in}{0.800818in}}%
\pgfpathlineto{\pgfqpoint{1.204100in}{0.797869in}}%
\pgfpathmoveto{\pgfqpoint{1.204100in}{0.794920in}}%
\pgfpathlineto{\pgfqpoint{1.204100in}{0.794920in}}%
\pgfpathlineto{\pgfqpoint{1.204100in}{0.797869in}}%
\pgfpathlineto{\pgfqpoint{1.208641in}{0.797869in}}%
\pgfpathlineto{\pgfqpoint{1.208641in}{0.794920in}}%
\pgfpathmoveto{\pgfqpoint{1.204100in}{0.797869in}}%
\pgfpathlineto{\pgfqpoint{1.204100in}{0.797869in}}%
\pgfpathlineto{\pgfqpoint{1.204100in}{0.800818in}}%
\pgfpathlineto{\pgfqpoint{1.208641in}{0.800818in}}%
\pgfpathlineto{\pgfqpoint{1.208641in}{0.797869in}}%
\pgfpathmoveto{\pgfqpoint{1.208641in}{0.797869in}}%
\pgfpathlineto{\pgfqpoint{1.208641in}{0.797869in}}%
\pgfpathlineto{\pgfqpoint{1.208641in}{0.800818in}}%
\pgfpathlineto{\pgfqpoint{1.213182in}{0.800818in}}%
\pgfpathlineto{\pgfqpoint{1.213182in}{0.797869in}}%
\pgfpathmoveto{\pgfqpoint{1.204100in}{0.800818in}}%
\pgfpathlineto{\pgfqpoint{1.204100in}{0.800818in}}%
\pgfpathlineto{\pgfqpoint{1.204100in}{0.803768in}}%
\pgfpathlineto{\pgfqpoint{1.208641in}{0.803768in}}%
\pgfpathlineto{\pgfqpoint{1.208641in}{0.800818in}}%
\pgfpathmoveto{\pgfqpoint{1.204100in}{0.803768in}}%
\pgfpathlineto{\pgfqpoint{1.204100in}{0.803768in}}%
\pgfpathlineto{\pgfqpoint{1.204100in}{0.806717in}}%
\pgfpathlineto{\pgfqpoint{1.208641in}{0.806717in}}%
\pgfpathlineto{\pgfqpoint{1.208641in}{0.803768in}}%
\pgfpathmoveto{\pgfqpoint{1.208641in}{0.800818in}}%
\pgfpathlineto{\pgfqpoint{1.208641in}{0.800818in}}%
\pgfpathlineto{\pgfqpoint{1.208641in}{0.803768in}}%
\pgfpathlineto{\pgfqpoint{1.213182in}{0.803768in}}%
\pgfpathlineto{\pgfqpoint{1.213182in}{0.800818in}}%
\pgfpathmoveto{\pgfqpoint{1.208641in}{0.803768in}}%
\pgfpathlineto{\pgfqpoint{1.208641in}{0.803768in}}%
\pgfpathlineto{\pgfqpoint{1.208641in}{0.806717in}}%
\pgfpathlineto{\pgfqpoint{1.213182in}{0.806717in}}%
\pgfpathlineto{\pgfqpoint{1.213182in}{0.803768in}}%
\pgfpathmoveto{\pgfqpoint{1.213182in}{0.800818in}}%
\pgfpathlineto{\pgfqpoint{1.213182in}{0.800818in}}%
\pgfpathlineto{\pgfqpoint{1.213182in}{0.803768in}}%
\pgfpathlineto{\pgfqpoint{1.217723in}{0.803768in}}%
\pgfpathlineto{\pgfqpoint{1.217723in}{0.800818in}}%
\pgfpathmoveto{\pgfqpoint{1.213182in}{0.803768in}}%
\pgfpathlineto{\pgfqpoint{1.213182in}{0.803768in}}%
\pgfpathlineto{\pgfqpoint{1.213182in}{0.806717in}}%
\pgfpathlineto{\pgfqpoint{1.217723in}{0.806717in}}%
\pgfpathlineto{\pgfqpoint{1.217723in}{0.803768in}}%
\pgfpathmoveto{\pgfqpoint{1.217723in}{0.803768in}}%
\pgfpathlineto{\pgfqpoint{1.217723in}{0.803768in}}%
\pgfpathlineto{\pgfqpoint{1.217723in}{0.806717in}}%
\pgfpathlineto{\pgfqpoint{1.222264in}{0.806717in}}%
\pgfpathlineto{\pgfqpoint{1.222264in}{0.803768in}}%
\pgfpathmoveto{\pgfqpoint{1.213182in}{0.806717in}}%
\pgfpathlineto{\pgfqpoint{1.213182in}{0.806717in}}%
\pgfpathlineto{\pgfqpoint{1.213182in}{0.809666in}}%
\pgfpathlineto{\pgfqpoint{1.217723in}{0.809666in}}%
\pgfpathlineto{\pgfqpoint{1.217723in}{0.806717in}}%
\pgfpathmoveto{\pgfqpoint{1.213182in}{0.809666in}}%
\pgfpathlineto{\pgfqpoint{1.213182in}{0.809666in}}%
\pgfpathlineto{\pgfqpoint{1.213182in}{0.812615in}}%
\pgfpathlineto{\pgfqpoint{1.217723in}{0.812615in}}%
\pgfpathlineto{\pgfqpoint{1.217723in}{0.809666in}}%
\pgfpathmoveto{\pgfqpoint{1.217723in}{0.806717in}}%
\pgfpathlineto{\pgfqpoint{1.217723in}{0.806717in}}%
\pgfpathlineto{\pgfqpoint{1.217723in}{0.809666in}}%
\pgfpathlineto{\pgfqpoint{1.222264in}{0.809666in}}%
\pgfpathlineto{\pgfqpoint{1.222264in}{0.806717in}}%
\pgfpathmoveto{\pgfqpoint{1.217723in}{0.809666in}}%
\pgfpathlineto{\pgfqpoint{1.217723in}{0.809666in}}%
\pgfpathlineto{\pgfqpoint{1.217723in}{0.812615in}}%
\pgfpathlineto{\pgfqpoint{1.222264in}{0.812615in}}%
\pgfpathlineto{\pgfqpoint{1.222264in}{0.809666in}}%
\pgfpathmoveto{\pgfqpoint{1.222264in}{0.806717in}}%
\pgfpathlineto{\pgfqpoint{1.222264in}{0.806717in}}%
\pgfpathlineto{\pgfqpoint{1.222264in}{0.809666in}}%
\pgfpathlineto{\pgfqpoint{1.226805in}{0.809666in}}%
\pgfpathlineto{\pgfqpoint{1.226805in}{0.806717in}}%
\pgfpathmoveto{\pgfqpoint{1.222264in}{0.809666in}}%
\pgfpathlineto{\pgfqpoint{1.222264in}{0.809666in}}%
\pgfpathlineto{\pgfqpoint{1.222264in}{0.812615in}}%
\pgfpathlineto{\pgfqpoint{1.226805in}{0.812615in}}%
\pgfpathlineto{\pgfqpoint{1.226805in}{0.809666in}}%
\pgfpathmoveto{\pgfqpoint{1.226805in}{0.809666in}}%
\pgfpathlineto{\pgfqpoint{1.226805in}{0.809666in}}%
\pgfpathlineto{\pgfqpoint{1.226805in}{0.812615in}}%
\pgfpathlineto{\pgfqpoint{1.231346in}{0.812615in}}%
\pgfpathlineto{\pgfqpoint{1.231346in}{0.809666in}}%
\pgfpathmoveto{\pgfqpoint{1.222264in}{0.812615in}}%
\pgfpathlineto{\pgfqpoint{1.222264in}{0.812615in}}%
\pgfpathlineto{\pgfqpoint{1.222264in}{0.815565in}}%
\pgfpathlineto{\pgfqpoint{1.226805in}{0.815565in}}%
\pgfpathlineto{\pgfqpoint{1.226805in}{0.812615in}}%
\pgfpathmoveto{\pgfqpoint{1.222264in}{0.815565in}}%
\pgfpathlineto{\pgfqpoint{1.222264in}{0.815565in}}%
\pgfpathlineto{\pgfqpoint{1.222264in}{0.818514in}}%
\pgfpathlineto{\pgfqpoint{1.226805in}{0.818514in}}%
\pgfpathlineto{\pgfqpoint{1.226805in}{0.815565in}}%
\pgfpathmoveto{\pgfqpoint{1.226805in}{0.812615in}}%
\pgfpathlineto{\pgfqpoint{1.226805in}{0.812615in}}%
\pgfpathlineto{\pgfqpoint{1.226805in}{0.815565in}}%
\pgfpathlineto{\pgfqpoint{1.231346in}{0.815565in}}%
\pgfpathlineto{\pgfqpoint{1.231346in}{0.812615in}}%
\pgfpathmoveto{\pgfqpoint{1.226805in}{0.815565in}}%
\pgfpathlineto{\pgfqpoint{1.226805in}{0.815565in}}%
\pgfpathlineto{\pgfqpoint{1.226805in}{0.818514in}}%
\pgfpathlineto{\pgfqpoint{1.231346in}{0.818514in}}%
\pgfpathlineto{\pgfqpoint{1.231346in}{0.815565in}}%
\pgfpathmoveto{\pgfqpoint{1.231346in}{0.812615in}}%
\pgfpathlineto{\pgfqpoint{1.231346in}{0.812615in}}%
\pgfpathlineto{\pgfqpoint{1.231346in}{0.815565in}}%
\pgfpathlineto{\pgfqpoint{1.235887in}{0.815565in}}%
\pgfpathlineto{\pgfqpoint{1.235887in}{0.812615in}}%
\pgfpathmoveto{\pgfqpoint{1.231346in}{0.815565in}}%
\pgfpathlineto{\pgfqpoint{1.231346in}{0.815565in}}%
\pgfpathlineto{\pgfqpoint{1.231346in}{0.818514in}}%
\pgfpathlineto{\pgfqpoint{1.235887in}{0.818514in}}%
\pgfpathlineto{\pgfqpoint{1.235887in}{0.815565in}}%
\pgfpathmoveto{\pgfqpoint{1.235887in}{0.815565in}}%
\pgfpathlineto{\pgfqpoint{1.235887in}{0.815565in}}%
\pgfpathlineto{\pgfqpoint{1.235887in}{0.818514in}}%
\pgfpathlineto{\pgfqpoint{1.240428in}{0.818514in}}%
\pgfpathlineto{\pgfqpoint{1.240428in}{0.815565in}}%
\pgfpathmoveto{\pgfqpoint{1.231346in}{0.818514in}}%
\pgfpathlineto{\pgfqpoint{1.231346in}{0.818514in}}%
\pgfpathlineto{\pgfqpoint{1.231346in}{0.821463in}}%
\pgfpathlineto{\pgfqpoint{1.235887in}{0.821463in}}%
\pgfpathlineto{\pgfqpoint{1.235887in}{0.818514in}}%
\pgfpathmoveto{\pgfqpoint{1.231346in}{0.821463in}}%
\pgfpathlineto{\pgfqpoint{1.231346in}{0.821463in}}%
\pgfpathlineto{\pgfqpoint{1.231346in}{0.824412in}}%
\pgfpathlineto{\pgfqpoint{1.235887in}{0.824412in}}%
\pgfpathlineto{\pgfqpoint{1.235887in}{0.821463in}}%
\pgfpathmoveto{\pgfqpoint{1.235887in}{0.818514in}}%
\pgfpathlineto{\pgfqpoint{1.235887in}{0.818514in}}%
\pgfpathlineto{\pgfqpoint{1.235887in}{0.821463in}}%
\pgfpathlineto{\pgfqpoint{1.240428in}{0.821463in}}%
\pgfpathlineto{\pgfqpoint{1.240428in}{0.818514in}}%
\pgfpathmoveto{\pgfqpoint{1.235887in}{0.821463in}}%
\pgfpathlineto{\pgfqpoint{1.235887in}{0.821463in}}%
\pgfpathlineto{\pgfqpoint{1.235887in}{0.824412in}}%
\pgfpathlineto{\pgfqpoint{1.240428in}{0.824412in}}%
\pgfpathlineto{\pgfqpoint{1.240428in}{0.821463in}}%
\pgfpathmoveto{\pgfqpoint{1.240428in}{0.818514in}}%
\pgfpathlineto{\pgfqpoint{1.240428in}{0.818514in}}%
\pgfpathlineto{\pgfqpoint{1.240428in}{0.821463in}}%
\pgfpathlineto{\pgfqpoint{1.244969in}{0.821463in}}%
\pgfpathlineto{\pgfqpoint{1.244969in}{0.818514in}}%
\pgfpathmoveto{\pgfqpoint{1.240428in}{0.821463in}}%
\pgfpathlineto{\pgfqpoint{1.240428in}{0.821463in}}%
\pgfpathlineto{\pgfqpoint{1.240428in}{0.824412in}}%
\pgfpathlineto{\pgfqpoint{1.244969in}{0.824412in}}%
\pgfpathlineto{\pgfqpoint{1.244969in}{0.821463in}}%
\pgfpathmoveto{\pgfqpoint{1.244969in}{0.821463in}}%
\pgfpathlineto{\pgfqpoint{1.244969in}{0.821463in}}%
\pgfpathlineto{\pgfqpoint{1.244969in}{0.824412in}}%
\pgfpathlineto{\pgfqpoint{1.249510in}{0.824412in}}%
\pgfpathlineto{\pgfqpoint{1.249510in}{0.821463in}}%
\pgfpathmoveto{\pgfqpoint{1.240428in}{0.824412in}}%
\pgfpathlineto{\pgfqpoint{1.240428in}{0.824412in}}%
\pgfpathlineto{\pgfqpoint{1.240428in}{0.827361in}}%
\pgfpathlineto{\pgfqpoint{1.244969in}{0.827361in}}%
\pgfpathlineto{\pgfqpoint{1.244969in}{0.824412in}}%
\pgfpathmoveto{\pgfqpoint{1.240428in}{0.827361in}}%
\pgfpathlineto{\pgfqpoint{1.240428in}{0.827361in}}%
\pgfpathlineto{\pgfqpoint{1.240428in}{0.830311in}}%
\pgfpathlineto{\pgfqpoint{1.244969in}{0.830311in}}%
\pgfpathlineto{\pgfqpoint{1.244969in}{0.827361in}}%
\pgfpathmoveto{\pgfqpoint{1.244969in}{0.824412in}}%
\pgfpathlineto{\pgfqpoint{1.244969in}{0.824412in}}%
\pgfpathlineto{\pgfqpoint{1.244969in}{0.827361in}}%
\pgfpathlineto{\pgfqpoint{1.249510in}{0.827361in}}%
\pgfpathlineto{\pgfqpoint{1.249510in}{0.824412in}}%
\pgfpathmoveto{\pgfqpoint{1.244969in}{0.827361in}}%
\pgfpathlineto{\pgfqpoint{1.244969in}{0.827361in}}%
\pgfpathlineto{\pgfqpoint{1.244969in}{0.830311in}}%
\pgfpathlineto{\pgfqpoint{1.249510in}{0.830311in}}%
\pgfpathlineto{\pgfqpoint{1.249510in}{0.827361in}}%
\pgfpathmoveto{\pgfqpoint{1.249510in}{0.824412in}}%
\pgfpathlineto{\pgfqpoint{1.249510in}{0.824412in}}%
\pgfpathlineto{\pgfqpoint{1.249510in}{0.827361in}}%
\pgfpathlineto{\pgfqpoint{1.254051in}{0.827361in}}%
\pgfpathlineto{\pgfqpoint{1.254051in}{0.824412in}}%
\pgfpathmoveto{\pgfqpoint{1.249510in}{0.827361in}}%
\pgfpathlineto{\pgfqpoint{1.249510in}{0.827361in}}%
\pgfpathlineto{\pgfqpoint{1.249510in}{0.830311in}}%
\pgfpathlineto{\pgfqpoint{1.254051in}{0.830311in}}%
\pgfpathlineto{\pgfqpoint{1.254051in}{0.827361in}}%
\pgfpathmoveto{\pgfqpoint{1.254051in}{0.827361in}}%
\pgfpathlineto{\pgfqpoint{1.254051in}{0.827361in}}%
\pgfpathlineto{\pgfqpoint{1.254051in}{0.830311in}}%
\pgfpathlineto{\pgfqpoint{1.258592in}{0.830311in}}%
\pgfpathlineto{\pgfqpoint{1.258592in}{0.827361in}}%
\pgfpathmoveto{\pgfqpoint{1.249510in}{0.830311in}}%
\pgfpathlineto{\pgfqpoint{1.249510in}{0.830311in}}%
\pgfpathlineto{\pgfqpoint{1.249510in}{0.833260in}}%
\pgfpathlineto{\pgfqpoint{1.254051in}{0.833260in}}%
\pgfpathlineto{\pgfqpoint{1.254051in}{0.830311in}}%
\pgfpathmoveto{\pgfqpoint{1.249510in}{0.833260in}}%
\pgfpathlineto{\pgfqpoint{1.249510in}{0.833260in}}%
\pgfpathlineto{\pgfqpoint{1.249510in}{0.836209in}}%
\pgfpathlineto{\pgfqpoint{1.254051in}{0.836209in}}%
\pgfpathlineto{\pgfqpoint{1.254051in}{0.833260in}}%
\pgfpathmoveto{\pgfqpoint{1.254051in}{0.830311in}}%
\pgfpathlineto{\pgfqpoint{1.254051in}{0.830311in}}%
\pgfpathlineto{\pgfqpoint{1.254051in}{0.833260in}}%
\pgfpathlineto{\pgfqpoint{1.258592in}{0.833260in}}%
\pgfpathlineto{\pgfqpoint{1.258592in}{0.830311in}}%
\pgfpathmoveto{\pgfqpoint{1.254051in}{0.833260in}}%
\pgfpathlineto{\pgfqpoint{1.254051in}{0.833260in}}%
\pgfpathlineto{\pgfqpoint{1.254051in}{0.836209in}}%
\pgfpathlineto{\pgfqpoint{1.258592in}{0.836209in}}%
\pgfpathlineto{\pgfqpoint{1.258592in}{0.833260in}}%
\pgfpathmoveto{\pgfqpoint{1.258592in}{0.830311in}}%
\pgfpathlineto{\pgfqpoint{1.258592in}{0.830311in}}%
\pgfpathlineto{\pgfqpoint{1.258592in}{0.833260in}}%
\pgfpathlineto{\pgfqpoint{1.263133in}{0.833260in}}%
\pgfpathlineto{\pgfqpoint{1.263133in}{0.830311in}}%
\pgfpathmoveto{\pgfqpoint{1.258592in}{0.833260in}}%
\pgfpathlineto{\pgfqpoint{1.258592in}{0.833260in}}%
\pgfpathlineto{\pgfqpoint{1.258592in}{0.836209in}}%
\pgfpathlineto{\pgfqpoint{1.263133in}{0.836209in}}%
\pgfpathlineto{\pgfqpoint{1.263133in}{0.833260in}}%
\pgfpathmoveto{\pgfqpoint{1.263133in}{0.833260in}}%
\pgfpathlineto{\pgfqpoint{1.263133in}{0.833260in}}%
\pgfpathlineto{\pgfqpoint{1.263133in}{0.836209in}}%
\pgfpathlineto{\pgfqpoint{1.267674in}{0.836209in}}%
\pgfpathlineto{\pgfqpoint{1.267674in}{0.833260in}}%
\pgfpathmoveto{\pgfqpoint{1.258592in}{0.836209in}}%
\pgfpathlineto{\pgfqpoint{1.258592in}{0.836209in}}%
\pgfpathlineto{\pgfqpoint{1.258592in}{0.839158in}}%
\pgfpathlineto{\pgfqpoint{1.263133in}{0.839158in}}%
\pgfpathlineto{\pgfqpoint{1.263133in}{0.836209in}}%
\pgfpathmoveto{\pgfqpoint{1.258592in}{0.839158in}}%
\pgfpathlineto{\pgfqpoint{1.258592in}{0.839158in}}%
\pgfpathlineto{\pgfqpoint{1.258592in}{0.842108in}}%
\pgfpathlineto{\pgfqpoint{1.263133in}{0.842108in}}%
\pgfpathlineto{\pgfqpoint{1.263133in}{0.839158in}}%
\pgfpathmoveto{\pgfqpoint{1.263133in}{0.836209in}}%
\pgfpathlineto{\pgfqpoint{1.263133in}{0.836209in}}%
\pgfpathlineto{\pgfqpoint{1.263133in}{0.839158in}}%
\pgfpathlineto{\pgfqpoint{1.267674in}{0.839158in}}%
\pgfpathlineto{\pgfqpoint{1.267674in}{0.836209in}}%
\pgfpathmoveto{\pgfqpoint{1.263133in}{0.839158in}}%
\pgfpathlineto{\pgfqpoint{1.263133in}{0.839158in}}%
\pgfpathlineto{\pgfqpoint{1.263133in}{0.842108in}}%
\pgfpathlineto{\pgfqpoint{1.267674in}{0.842108in}}%
\pgfpathlineto{\pgfqpoint{1.267674in}{0.839158in}}%
\pgfpathmoveto{\pgfqpoint{1.267674in}{0.836209in}}%
\pgfpathlineto{\pgfqpoint{1.267674in}{0.836209in}}%
\pgfpathlineto{\pgfqpoint{1.267674in}{0.839158in}}%
\pgfpathlineto{\pgfqpoint{1.272215in}{0.839158in}}%
\pgfpathlineto{\pgfqpoint{1.272215in}{0.836209in}}%
\pgfpathmoveto{\pgfqpoint{1.267674in}{0.839158in}}%
\pgfpathlineto{\pgfqpoint{1.267674in}{0.839158in}}%
\pgfpathlineto{\pgfqpoint{1.267674in}{0.842108in}}%
\pgfpathlineto{\pgfqpoint{1.272215in}{0.842108in}}%
\pgfpathlineto{\pgfqpoint{1.272215in}{0.839158in}}%
\pgfpathmoveto{\pgfqpoint{1.272215in}{0.839158in}}%
\pgfpathlineto{\pgfqpoint{1.272215in}{0.839158in}}%
\pgfpathlineto{\pgfqpoint{1.272215in}{0.842108in}}%
\pgfpathlineto{\pgfqpoint{1.276757in}{0.842108in}}%
\pgfpathlineto{\pgfqpoint{1.276757in}{0.839158in}}%
\pgfpathmoveto{\pgfqpoint{1.267674in}{0.842108in}}%
\pgfpathlineto{\pgfqpoint{1.267674in}{0.842108in}}%
\pgfpathlineto{\pgfqpoint{1.267674in}{0.845057in}}%
\pgfpathlineto{\pgfqpoint{1.272215in}{0.845057in}}%
\pgfpathlineto{\pgfqpoint{1.272215in}{0.842108in}}%
\pgfpathmoveto{\pgfqpoint{1.267674in}{0.845057in}}%
\pgfpathlineto{\pgfqpoint{1.267674in}{0.845057in}}%
\pgfpathlineto{\pgfqpoint{1.267674in}{0.848006in}}%
\pgfpathlineto{\pgfqpoint{1.272215in}{0.848006in}}%
\pgfpathlineto{\pgfqpoint{1.272215in}{0.845057in}}%
\pgfpathmoveto{\pgfqpoint{1.272215in}{0.842108in}}%
\pgfpathlineto{\pgfqpoint{1.272215in}{0.842108in}}%
\pgfpathlineto{\pgfqpoint{1.272215in}{0.845057in}}%
\pgfpathlineto{\pgfqpoint{1.276757in}{0.845057in}}%
\pgfpathlineto{\pgfqpoint{1.276757in}{0.842108in}}%
\pgfpathmoveto{\pgfqpoint{1.272215in}{0.845057in}}%
\pgfpathlineto{\pgfqpoint{1.272215in}{0.845057in}}%
\pgfpathlineto{\pgfqpoint{1.272215in}{0.848006in}}%
\pgfpathlineto{\pgfqpoint{1.276757in}{0.848006in}}%
\pgfpathlineto{\pgfqpoint{1.276757in}{0.845057in}}%
\pgfpathmoveto{\pgfqpoint{1.276757in}{0.842108in}}%
\pgfpathlineto{\pgfqpoint{1.276757in}{0.842108in}}%
\pgfpathlineto{\pgfqpoint{1.276757in}{0.845057in}}%
\pgfpathlineto{\pgfqpoint{1.281298in}{0.845057in}}%
\pgfpathlineto{\pgfqpoint{1.281298in}{0.842108in}}%
\pgfpathmoveto{\pgfqpoint{1.276757in}{0.845057in}}%
\pgfpathlineto{\pgfqpoint{1.276757in}{0.845057in}}%
\pgfpathlineto{\pgfqpoint{1.276757in}{0.848006in}}%
\pgfpathlineto{\pgfqpoint{1.281298in}{0.848006in}}%
\pgfpathlineto{\pgfqpoint{1.281298in}{0.845057in}}%
\pgfpathmoveto{\pgfqpoint{1.281298in}{0.845057in}}%
\pgfpathlineto{\pgfqpoint{1.281298in}{0.845057in}}%
\pgfpathlineto{\pgfqpoint{1.281298in}{0.848006in}}%
\pgfpathlineto{\pgfqpoint{1.285839in}{0.848006in}}%
\pgfpathlineto{\pgfqpoint{1.285839in}{0.845057in}}%
\pgfpathmoveto{\pgfqpoint{1.276757in}{0.848006in}}%
\pgfpathlineto{\pgfqpoint{1.276757in}{0.848006in}}%
\pgfpathlineto{\pgfqpoint{1.276757in}{0.850955in}}%
\pgfpathlineto{\pgfqpoint{1.281298in}{0.850955in}}%
\pgfpathlineto{\pgfqpoint{1.281298in}{0.848006in}}%
\pgfpathmoveto{\pgfqpoint{1.276757in}{0.850955in}}%
\pgfpathlineto{\pgfqpoint{1.276757in}{0.850955in}}%
\pgfpathlineto{\pgfqpoint{1.276757in}{0.853905in}}%
\pgfpathlineto{\pgfqpoint{1.281298in}{0.853905in}}%
\pgfpathlineto{\pgfqpoint{1.281298in}{0.850955in}}%
\pgfpathmoveto{\pgfqpoint{1.281298in}{0.848006in}}%
\pgfpathlineto{\pgfqpoint{1.281298in}{0.848006in}}%
\pgfpathlineto{\pgfqpoint{1.281298in}{0.850955in}}%
\pgfpathlineto{\pgfqpoint{1.285839in}{0.850955in}}%
\pgfpathlineto{\pgfqpoint{1.285839in}{0.848006in}}%
\pgfpathmoveto{\pgfqpoint{1.281298in}{0.850955in}}%
\pgfpathlineto{\pgfqpoint{1.281298in}{0.850955in}}%
\pgfpathlineto{\pgfqpoint{1.281298in}{0.853905in}}%
\pgfpathlineto{\pgfqpoint{1.285839in}{0.853905in}}%
\pgfpathlineto{\pgfqpoint{1.285839in}{0.850955in}}%
\pgfpathmoveto{\pgfqpoint{1.285839in}{0.848006in}}%
\pgfpathlineto{\pgfqpoint{1.285839in}{0.848006in}}%
\pgfpathlineto{\pgfqpoint{1.285839in}{0.850955in}}%
\pgfpathlineto{\pgfqpoint{1.290380in}{0.850955in}}%
\pgfpathlineto{\pgfqpoint{1.290380in}{0.848006in}}%
\pgfpathmoveto{\pgfqpoint{1.285839in}{0.850955in}}%
\pgfpathlineto{\pgfqpoint{1.285839in}{0.850955in}}%
\pgfpathlineto{\pgfqpoint{1.285839in}{0.853905in}}%
\pgfpathlineto{\pgfqpoint{1.290380in}{0.853905in}}%
\pgfpathlineto{\pgfqpoint{1.290380in}{0.850955in}}%
\pgfpathmoveto{\pgfqpoint{1.290380in}{0.850955in}}%
\pgfpathlineto{\pgfqpoint{1.290380in}{0.850955in}}%
\pgfpathlineto{\pgfqpoint{1.290380in}{0.853905in}}%
\pgfpathlineto{\pgfqpoint{1.294921in}{0.853905in}}%
\pgfpathlineto{\pgfqpoint{1.294921in}{0.850955in}}%
\pgfpathmoveto{\pgfqpoint{1.285839in}{0.853905in}}%
\pgfpathlineto{\pgfqpoint{1.285839in}{0.853905in}}%
\pgfpathlineto{\pgfqpoint{1.285839in}{0.856854in}}%
\pgfpathlineto{\pgfqpoint{1.290380in}{0.856854in}}%
\pgfpathlineto{\pgfqpoint{1.290380in}{0.853905in}}%
\pgfpathmoveto{\pgfqpoint{1.285839in}{0.856854in}}%
\pgfpathlineto{\pgfqpoint{1.285839in}{0.856854in}}%
\pgfpathlineto{\pgfqpoint{1.285839in}{0.859803in}}%
\pgfpathlineto{\pgfqpoint{1.290380in}{0.859803in}}%
\pgfpathlineto{\pgfqpoint{1.290380in}{0.856854in}}%
\pgfpathmoveto{\pgfqpoint{1.290380in}{0.853905in}}%
\pgfpathlineto{\pgfqpoint{1.290380in}{0.853905in}}%
\pgfpathlineto{\pgfqpoint{1.290380in}{0.856854in}}%
\pgfpathlineto{\pgfqpoint{1.294921in}{0.856854in}}%
\pgfpathlineto{\pgfqpoint{1.294921in}{0.853905in}}%
\pgfpathmoveto{\pgfqpoint{1.290380in}{0.856854in}}%
\pgfpathlineto{\pgfqpoint{1.290380in}{0.856854in}}%
\pgfpathlineto{\pgfqpoint{1.290380in}{0.859803in}}%
\pgfpathlineto{\pgfqpoint{1.294921in}{0.859803in}}%
\pgfpathlineto{\pgfqpoint{1.294921in}{0.856854in}}%
\pgfpathmoveto{\pgfqpoint{1.294921in}{0.853905in}}%
\pgfpathlineto{\pgfqpoint{1.294921in}{0.853905in}}%
\pgfpathlineto{\pgfqpoint{1.294921in}{0.856854in}}%
\pgfpathlineto{\pgfqpoint{1.299462in}{0.856854in}}%
\pgfpathlineto{\pgfqpoint{1.299462in}{0.853905in}}%
\pgfpathmoveto{\pgfqpoint{1.294921in}{0.856854in}}%
\pgfpathlineto{\pgfqpoint{1.294921in}{0.856854in}}%
\pgfpathlineto{\pgfqpoint{1.294921in}{0.859803in}}%
\pgfpathlineto{\pgfqpoint{1.299462in}{0.859803in}}%
\pgfpathlineto{\pgfqpoint{1.299462in}{0.856854in}}%
\pgfpathmoveto{\pgfqpoint{1.299462in}{0.856854in}}%
\pgfpathlineto{\pgfqpoint{1.299462in}{0.856854in}}%
\pgfpathlineto{\pgfqpoint{1.299462in}{0.859803in}}%
\pgfpathlineto{\pgfqpoint{1.304003in}{0.859803in}}%
\pgfpathlineto{\pgfqpoint{1.304003in}{0.856854in}}%
\pgfpathmoveto{\pgfqpoint{1.294921in}{0.859803in}}%
\pgfpathlineto{\pgfqpoint{1.294921in}{0.859803in}}%
\pgfpathlineto{\pgfqpoint{1.294921in}{0.862752in}}%
\pgfpathlineto{\pgfqpoint{1.299462in}{0.862752in}}%
\pgfpathlineto{\pgfqpoint{1.299462in}{0.859803in}}%
\pgfpathmoveto{\pgfqpoint{1.294921in}{0.862752in}}%
\pgfpathlineto{\pgfqpoint{1.294921in}{0.862752in}}%
\pgfpathlineto{\pgfqpoint{1.294921in}{0.865701in}}%
\pgfpathlineto{\pgfqpoint{1.299462in}{0.865701in}}%
\pgfpathlineto{\pgfqpoint{1.299462in}{0.862752in}}%
\pgfpathmoveto{\pgfqpoint{1.299462in}{0.859803in}}%
\pgfpathlineto{\pgfqpoint{1.299462in}{0.859803in}}%
\pgfpathlineto{\pgfqpoint{1.299462in}{0.862752in}}%
\pgfpathlineto{\pgfqpoint{1.304003in}{0.862752in}}%
\pgfpathlineto{\pgfqpoint{1.304003in}{0.859803in}}%
\pgfpathmoveto{\pgfqpoint{1.299462in}{0.862752in}}%
\pgfpathlineto{\pgfqpoint{1.299462in}{0.862752in}}%
\pgfpathlineto{\pgfqpoint{1.299462in}{0.865701in}}%
\pgfpathlineto{\pgfqpoint{1.304003in}{0.865701in}}%
\pgfpathlineto{\pgfqpoint{1.304003in}{0.862752in}}%
\pgfpathmoveto{\pgfqpoint{1.304003in}{0.859803in}}%
\pgfpathlineto{\pgfqpoint{1.304003in}{0.859803in}}%
\pgfpathlineto{\pgfqpoint{1.304003in}{0.862752in}}%
\pgfpathlineto{\pgfqpoint{1.308544in}{0.862752in}}%
\pgfpathlineto{\pgfqpoint{1.308544in}{0.859803in}}%
\pgfpathmoveto{\pgfqpoint{1.304003in}{0.862752in}}%
\pgfpathlineto{\pgfqpoint{1.304003in}{0.862752in}}%
\pgfpathlineto{\pgfqpoint{1.304003in}{0.865701in}}%
\pgfpathlineto{\pgfqpoint{1.308544in}{0.865701in}}%
\pgfpathlineto{\pgfqpoint{1.308544in}{0.862752in}}%
\pgfpathmoveto{\pgfqpoint{1.308544in}{0.862752in}}%
\pgfpathlineto{\pgfqpoint{1.308544in}{0.862752in}}%
\pgfpathlineto{\pgfqpoint{1.308544in}{0.865701in}}%
\pgfpathlineto{\pgfqpoint{1.313085in}{0.865701in}}%
\pgfpathlineto{\pgfqpoint{1.313085in}{0.862752in}}%
\pgfpathmoveto{\pgfqpoint{1.304003in}{0.865701in}}%
\pgfpathlineto{\pgfqpoint{1.304003in}{0.865701in}}%
\pgfpathlineto{\pgfqpoint{1.304003in}{0.868651in}}%
\pgfpathlineto{\pgfqpoint{1.308544in}{0.868651in}}%
\pgfpathlineto{\pgfqpoint{1.308544in}{0.865701in}}%
\pgfpathmoveto{\pgfqpoint{1.304003in}{0.868651in}}%
\pgfpathlineto{\pgfqpoint{1.304003in}{0.868651in}}%
\pgfpathlineto{\pgfqpoint{1.304003in}{0.871600in}}%
\pgfpathlineto{\pgfqpoint{1.308544in}{0.871600in}}%
\pgfpathlineto{\pgfqpoint{1.308544in}{0.868651in}}%
\pgfpathmoveto{\pgfqpoint{1.308544in}{0.865701in}}%
\pgfpathlineto{\pgfqpoint{1.308544in}{0.865701in}}%
\pgfpathlineto{\pgfqpoint{1.308544in}{0.868651in}}%
\pgfpathlineto{\pgfqpoint{1.313085in}{0.868651in}}%
\pgfpathlineto{\pgfqpoint{1.313085in}{0.865701in}}%
\pgfpathmoveto{\pgfqpoint{1.308544in}{0.868651in}}%
\pgfpathlineto{\pgfqpoint{1.308544in}{0.868651in}}%
\pgfpathlineto{\pgfqpoint{1.308544in}{0.871600in}}%
\pgfpathlineto{\pgfqpoint{1.313085in}{0.871600in}}%
\pgfpathlineto{\pgfqpoint{1.313085in}{0.868651in}}%
\pgfpathmoveto{\pgfqpoint{1.313085in}{0.865701in}}%
\pgfpathlineto{\pgfqpoint{1.313085in}{0.865701in}}%
\pgfpathlineto{\pgfqpoint{1.313085in}{0.868651in}}%
\pgfpathlineto{\pgfqpoint{1.317626in}{0.868651in}}%
\pgfpathlineto{\pgfqpoint{1.317626in}{0.865701in}}%
\pgfpathmoveto{\pgfqpoint{1.313085in}{0.868651in}}%
\pgfpathlineto{\pgfqpoint{1.313085in}{0.868651in}}%
\pgfpathlineto{\pgfqpoint{1.313085in}{0.871600in}}%
\pgfpathlineto{\pgfqpoint{1.317626in}{0.871600in}}%
\pgfpathlineto{\pgfqpoint{1.317626in}{0.868651in}}%
\pgfpathmoveto{\pgfqpoint{1.317626in}{0.868651in}}%
\pgfpathlineto{\pgfqpoint{1.317626in}{0.868651in}}%
\pgfpathlineto{\pgfqpoint{1.317626in}{0.871600in}}%
\pgfpathlineto{\pgfqpoint{1.322167in}{0.871600in}}%
\pgfpathlineto{\pgfqpoint{1.322167in}{0.868651in}}%
\pgfpathmoveto{\pgfqpoint{1.313085in}{0.871600in}}%
\pgfpathlineto{\pgfqpoint{1.313085in}{0.871600in}}%
\pgfpathlineto{\pgfqpoint{1.313085in}{0.874549in}}%
\pgfpathlineto{\pgfqpoint{1.317626in}{0.874549in}}%
\pgfpathlineto{\pgfqpoint{1.317626in}{0.871600in}}%
\pgfpathmoveto{\pgfqpoint{1.313085in}{0.874549in}}%
\pgfpathlineto{\pgfqpoint{1.313085in}{0.874549in}}%
\pgfpathlineto{\pgfqpoint{1.313085in}{0.877498in}}%
\pgfpathlineto{\pgfqpoint{1.317626in}{0.877498in}}%
\pgfpathlineto{\pgfqpoint{1.317626in}{0.874549in}}%
\pgfpathmoveto{\pgfqpoint{1.317626in}{0.871600in}}%
\pgfpathlineto{\pgfqpoint{1.317626in}{0.871600in}}%
\pgfpathlineto{\pgfqpoint{1.317626in}{0.874549in}}%
\pgfpathlineto{\pgfqpoint{1.322167in}{0.874549in}}%
\pgfpathlineto{\pgfqpoint{1.322167in}{0.871600in}}%
\pgfpathmoveto{\pgfqpoint{1.317626in}{0.874549in}}%
\pgfpathlineto{\pgfqpoint{1.317626in}{0.874549in}}%
\pgfpathlineto{\pgfqpoint{1.317626in}{0.877498in}}%
\pgfpathlineto{\pgfqpoint{1.322167in}{0.877498in}}%
\pgfpathlineto{\pgfqpoint{1.322167in}{0.874549in}}%
\pgfpathmoveto{\pgfqpoint{1.322167in}{0.871600in}}%
\pgfpathlineto{\pgfqpoint{1.322167in}{0.871600in}}%
\pgfpathlineto{\pgfqpoint{1.322167in}{0.874549in}}%
\pgfpathlineto{\pgfqpoint{1.326708in}{0.874549in}}%
\pgfpathlineto{\pgfqpoint{1.326708in}{0.871600in}}%
\pgfpathmoveto{\pgfqpoint{1.322167in}{0.874549in}}%
\pgfpathlineto{\pgfqpoint{1.322167in}{0.874549in}}%
\pgfpathlineto{\pgfqpoint{1.322167in}{0.877498in}}%
\pgfpathlineto{\pgfqpoint{1.326708in}{0.877498in}}%
\pgfpathlineto{\pgfqpoint{1.326708in}{0.874549in}}%
\pgfpathmoveto{\pgfqpoint{1.326708in}{0.874549in}}%
\pgfpathlineto{\pgfqpoint{1.326708in}{0.874549in}}%
\pgfpathlineto{\pgfqpoint{1.326708in}{0.877498in}}%
\pgfpathlineto{\pgfqpoint{1.331249in}{0.877498in}}%
\pgfpathlineto{\pgfqpoint{1.331249in}{0.874549in}}%
\pgfpathmoveto{\pgfqpoint{1.322167in}{0.877498in}}%
\pgfpathlineto{\pgfqpoint{1.322167in}{0.877498in}}%
\pgfpathlineto{\pgfqpoint{1.322167in}{0.880448in}}%
\pgfpathlineto{\pgfqpoint{1.326708in}{0.880448in}}%
\pgfpathlineto{\pgfqpoint{1.326708in}{0.877498in}}%
\pgfpathmoveto{\pgfqpoint{1.322167in}{0.880448in}}%
\pgfpathlineto{\pgfqpoint{1.322167in}{0.880448in}}%
\pgfpathlineto{\pgfqpoint{1.322167in}{0.883397in}}%
\pgfpathlineto{\pgfqpoint{1.326708in}{0.883397in}}%
\pgfpathlineto{\pgfqpoint{1.326708in}{0.880448in}}%
\pgfpathmoveto{\pgfqpoint{1.326708in}{0.877498in}}%
\pgfpathlineto{\pgfqpoint{1.326708in}{0.877498in}}%
\pgfpathlineto{\pgfqpoint{1.326708in}{0.880448in}}%
\pgfpathlineto{\pgfqpoint{1.331249in}{0.880448in}}%
\pgfpathlineto{\pgfqpoint{1.331249in}{0.877498in}}%
\pgfpathmoveto{\pgfqpoint{1.326708in}{0.880448in}}%
\pgfpathlineto{\pgfqpoint{1.326708in}{0.880448in}}%
\pgfpathlineto{\pgfqpoint{1.326708in}{0.883397in}}%
\pgfpathlineto{\pgfqpoint{1.331249in}{0.883397in}}%
\pgfpathlineto{\pgfqpoint{1.331249in}{0.880448in}}%
\pgfpathmoveto{\pgfqpoint{1.331249in}{0.877498in}}%
\pgfpathlineto{\pgfqpoint{1.331249in}{0.877498in}}%
\pgfpathlineto{\pgfqpoint{1.331249in}{0.880448in}}%
\pgfpathlineto{\pgfqpoint{1.335790in}{0.880448in}}%
\pgfpathlineto{\pgfqpoint{1.335790in}{0.877498in}}%
\pgfpathmoveto{\pgfqpoint{1.331249in}{0.880448in}}%
\pgfpathlineto{\pgfqpoint{1.331249in}{0.880448in}}%
\pgfpathlineto{\pgfqpoint{1.331249in}{0.883397in}}%
\pgfpathlineto{\pgfqpoint{1.335790in}{0.883397in}}%
\pgfpathlineto{\pgfqpoint{1.335790in}{0.880448in}}%
\pgfpathmoveto{\pgfqpoint{1.335790in}{0.880448in}}%
\pgfpathlineto{\pgfqpoint{1.335790in}{0.880448in}}%
\pgfpathlineto{\pgfqpoint{1.335790in}{0.883397in}}%
\pgfpathlineto{\pgfqpoint{1.340331in}{0.883397in}}%
\pgfpathlineto{\pgfqpoint{1.340331in}{0.880448in}}%
\pgfpathmoveto{\pgfqpoint{1.331249in}{0.883397in}}%
\pgfpathlineto{\pgfqpoint{1.331249in}{0.883397in}}%
\pgfpathlineto{\pgfqpoint{1.331249in}{0.886346in}}%
\pgfpathlineto{\pgfqpoint{1.335790in}{0.886346in}}%
\pgfpathlineto{\pgfqpoint{1.335790in}{0.883397in}}%
\pgfpathmoveto{\pgfqpoint{1.331249in}{0.886346in}}%
\pgfpathlineto{\pgfqpoint{1.331249in}{0.886346in}}%
\pgfpathlineto{\pgfqpoint{1.331249in}{0.889296in}}%
\pgfpathlineto{\pgfqpoint{1.335790in}{0.889296in}}%
\pgfpathlineto{\pgfqpoint{1.335790in}{0.886346in}}%
\pgfpathmoveto{\pgfqpoint{1.335790in}{0.883397in}}%
\pgfpathlineto{\pgfqpoint{1.335790in}{0.883397in}}%
\pgfpathlineto{\pgfqpoint{1.335790in}{0.886346in}}%
\pgfpathlineto{\pgfqpoint{1.340331in}{0.886346in}}%
\pgfpathlineto{\pgfqpoint{1.340331in}{0.883397in}}%
\pgfpathmoveto{\pgfqpoint{1.335790in}{0.886346in}}%
\pgfpathlineto{\pgfqpoint{1.335790in}{0.886346in}}%
\pgfpathlineto{\pgfqpoint{1.335790in}{0.889296in}}%
\pgfpathlineto{\pgfqpoint{1.340331in}{0.889296in}}%
\pgfpathlineto{\pgfqpoint{1.340331in}{0.886346in}}%
\pgfpathmoveto{\pgfqpoint{1.340331in}{0.883397in}}%
\pgfpathlineto{\pgfqpoint{1.340331in}{0.883397in}}%
\pgfpathlineto{\pgfqpoint{1.340331in}{0.886346in}}%
\pgfpathlineto{\pgfqpoint{1.344872in}{0.886346in}}%
\pgfpathlineto{\pgfqpoint{1.344872in}{0.883397in}}%
\pgfpathmoveto{\pgfqpoint{1.340331in}{0.886346in}}%
\pgfpathlineto{\pgfqpoint{1.340331in}{0.886346in}}%
\pgfpathlineto{\pgfqpoint{1.340331in}{0.889296in}}%
\pgfpathlineto{\pgfqpoint{1.344872in}{0.889296in}}%
\pgfpathlineto{\pgfqpoint{1.344872in}{0.886346in}}%
\pgfpathmoveto{\pgfqpoint{1.344872in}{0.886346in}}%
\pgfpathlineto{\pgfqpoint{1.344872in}{0.886346in}}%
\pgfpathlineto{\pgfqpoint{1.344872in}{0.889296in}}%
\pgfpathlineto{\pgfqpoint{1.349413in}{0.889296in}}%
\pgfpathlineto{\pgfqpoint{1.349413in}{0.886346in}}%
\pgfpathmoveto{\pgfqpoint{1.340331in}{0.889296in}}%
\pgfpathlineto{\pgfqpoint{1.340331in}{0.889296in}}%
\pgfpathlineto{\pgfqpoint{1.340331in}{0.892245in}}%
\pgfpathlineto{\pgfqpoint{1.344872in}{0.892245in}}%
\pgfpathlineto{\pgfqpoint{1.344872in}{0.889296in}}%
\pgfpathmoveto{\pgfqpoint{1.340331in}{0.892245in}}%
\pgfpathlineto{\pgfqpoint{1.340331in}{0.892245in}}%
\pgfpathlineto{\pgfqpoint{1.340331in}{0.895194in}}%
\pgfpathlineto{\pgfqpoint{1.344872in}{0.895194in}}%
\pgfpathlineto{\pgfqpoint{1.344872in}{0.892245in}}%
\pgfpathmoveto{\pgfqpoint{1.344872in}{0.889296in}}%
\pgfpathlineto{\pgfqpoint{1.344872in}{0.889296in}}%
\pgfpathlineto{\pgfqpoint{1.344872in}{0.892245in}}%
\pgfpathlineto{\pgfqpoint{1.349413in}{0.892245in}}%
\pgfpathlineto{\pgfqpoint{1.349413in}{0.889296in}}%
\pgfpathmoveto{\pgfqpoint{1.344872in}{0.892245in}}%
\pgfpathlineto{\pgfqpoint{1.344872in}{0.892245in}}%
\pgfpathlineto{\pgfqpoint{1.344872in}{0.895194in}}%
\pgfpathlineto{\pgfqpoint{1.349413in}{0.895194in}}%
\pgfpathlineto{\pgfqpoint{1.349413in}{0.892245in}}%
\pgfpathmoveto{\pgfqpoint{1.349413in}{0.889296in}}%
\pgfpathlineto{\pgfqpoint{1.349413in}{0.889296in}}%
\pgfpathlineto{\pgfqpoint{1.349413in}{0.892245in}}%
\pgfpathlineto{\pgfqpoint{1.353954in}{0.892245in}}%
\pgfpathlineto{\pgfqpoint{1.353954in}{0.889296in}}%
\pgfpathmoveto{\pgfqpoint{1.349413in}{0.892245in}}%
\pgfpathlineto{\pgfqpoint{1.349413in}{0.892245in}}%
\pgfpathlineto{\pgfqpoint{1.349413in}{0.895194in}}%
\pgfpathlineto{\pgfqpoint{1.353954in}{0.895194in}}%
\pgfpathlineto{\pgfqpoint{1.353954in}{0.892245in}}%
\pgfpathmoveto{\pgfqpoint{1.353954in}{0.892245in}}%
\pgfpathlineto{\pgfqpoint{1.353954in}{0.892245in}}%
\pgfpathlineto{\pgfqpoint{1.353954in}{0.895194in}}%
\pgfpathlineto{\pgfqpoint{1.358495in}{0.895194in}}%
\pgfpathlineto{\pgfqpoint{1.358495in}{0.892245in}}%
\pgfpathmoveto{\pgfqpoint{1.349413in}{0.895194in}}%
\pgfpathlineto{\pgfqpoint{1.349413in}{0.895194in}}%
\pgfpathlineto{\pgfqpoint{1.349413in}{0.898143in}}%
\pgfpathlineto{\pgfqpoint{1.353954in}{0.898143in}}%
\pgfpathlineto{\pgfqpoint{1.353954in}{0.895194in}}%
\pgfpathmoveto{\pgfqpoint{1.349413in}{0.898143in}}%
\pgfpathlineto{\pgfqpoint{1.349413in}{0.898143in}}%
\pgfpathlineto{\pgfqpoint{1.349413in}{0.901093in}}%
\pgfpathlineto{\pgfqpoint{1.353954in}{0.901093in}}%
\pgfpathlineto{\pgfqpoint{1.353954in}{0.898143in}}%
\pgfpathmoveto{\pgfqpoint{1.353954in}{0.895194in}}%
\pgfpathlineto{\pgfqpoint{1.353954in}{0.895194in}}%
\pgfpathlineto{\pgfqpoint{1.353954in}{0.898143in}}%
\pgfpathlineto{\pgfqpoint{1.358495in}{0.898143in}}%
\pgfpathlineto{\pgfqpoint{1.358495in}{0.895194in}}%
\pgfpathmoveto{\pgfqpoint{1.353954in}{0.898143in}}%
\pgfpathlineto{\pgfqpoint{1.353954in}{0.898143in}}%
\pgfpathlineto{\pgfqpoint{1.353954in}{0.901093in}}%
\pgfpathlineto{\pgfqpoint{1.358495in}{0.901093in}}%
\pgfpathlineto{\pgfqpoint{1.358495in}{0.898143in}}%
\pgfpathmoveto{\pgfqpoint{1.358495in}{0.895194in}}%
\pgfpathlineto{\pgfqpoint{1.358495in}{0.895194in}}%
\pgfpathlineto{\pgfqpoint{1.358495in}{0.898143in}}%
\pgfpathlineto{\pgfqpoint{1.363036in}{0.898143in}}%
\pgfpathlineto{\pgfqpoint{1.363036in}{0.895194in}}%
\pgfpathmoveto{\pgfqpoint{1.358495in}{0.898143in}}%
\pgfpathlineto{\pgfqpoint{1.358495in}{0.898143in}}%
\pgfpathlineto{\pgfqpoint{1.358495in}{0.901093in}}%
\pgfpathlineto{\pgfqpoint{1.363036in}{0.901093in}}%
\pgfpathlineto{\pgfqpoint{1.363036in}{0.898143in}}%
\pgfpathmoveto{\pgfqpoint{1.363036in}{0.898143in}}%
\pgfpathlineto{\pgfqpoint{1.363036in}{0.898143in}}%
\pgfpathlineto{\pgfqpoint{1.363036in}{0.901093in}}%
\pgfpathlineto{\pgfqpoint{1.367577in}{0.901093in}}%
\pgfpathlineto{\pgfqpoint{1.367577in}{0.898143in}}%
\pgfpathmoveto{\pgfqpoint{1.358495in}{0.901093in}}%
\pgfpathlineto{\pgfqpoint{1.358495in}{0.901093in}}%
\pgfpathlineto{\pgfqpoint{1.358495in}{0.904042in}}%
\pgfpathlineto{\pgfqpoint{1.363036in}{0.904042in}}%
\pgfpathlineto{\pgfqpoint{1.363036in}{0.901093in}}%
\pgfpathmoveto{\pgfqpoint{1.358495in}{0.904042in}}%
\pgfpathlineto{\pgfqpoint{1.358495in}{0.904042in}}%
\pgfpathlineto{\pgfqpoint{1.358495in}{0.906991in}}%
\pgfpathlineto{\pgfqpoint{1.363036in}{0.906991in}}%
\pgfpathlineto{\pgfqpoint{1.363036in}{0.904042in}}%
\pgfpathmoveto{\pgfqpoint{1.363036in}{0.901093in}}%
\pgfpathlineto{\pgfqpoint{1.363036in}{0.901093in}}%
\pgfpathlineto{\pgfqpoint{1.363036in}{0.904042in}}%
\pgfpathlineto{\pgfqpoint{1.367577in}{0.904042in}}%
\pgfpathlineto{\pgfqpoint{1.367577in}{0.901093in}}%
\pgfpathmoveto{\pgfqpoint{1.363036in}{0.904042in}}%
\pgfpathlineto{\pgfqpoint{1.363036in}{0.904042in}}%
\pgfpathlineto{\pgfqpoint{1.363036in}{0.906991in}}%
\pgfpathlineto{\pgfqpoint{1.367577in}{0.906991in}}%
\pgfpathlineto{\pgfqpoint{1.367577in}{0.904042in}}%
\pgfpathmoveto{\pgfqpoint{1.367577in}{0.901093in}}%
\pgfpathlineto{\pgfqpoint{1.367577in}{0.901093in}}%
\pgfpathlineto{\pgfqpoint{1.367577in}{0.904042in}}%
\pgfpathlineto{\pgfqpoint{1.372117in}{0.904042in}}%
\pgfpathlineto{\pgfqpoint{1.372117in}{0.901093in}}%
\pgfpathmoveto{\pgfqpoint{1.367577in}{0.904042in}}%
\pgfpathlineto{\pgfqpoint{1.367577in}{0.904042in}}%
\pgfpathlineto{\pgfqpoint{1.367577in}{0.906991in}}%
\pgfpathlineto{\pgfqpoint{1.372117in}{0.906991in}}%
\pgfpathlineto{\pgfqpoint{1.372117in}{0.904042in}}%
\pgfpathmoveto{\pgfqpoint{1.372117in}{0.901093in}}%
\pgfpathlineto{\pgfqpoint{1.372117in}{0.901093in}}%
\pgfpathlineto{\pgfqpoint{1.372117in}{0.904042in}}%
\pgfpathlineto{\pgfqpoint{1.376658in}{0.904042in}}%
\pgfpathlineto{\pgfqpoint{1.376658in}{0.901093in}}%
\pgfpathmoveto{\pgfqpoint{1.372117in}{0.904042in}}%
\pgfpathlineto{\pgfqpoint{1.372117in}{0.904042in}}%
\pgfpathlineto{\pgfqpoint{1.372117in}{0.906991in}}%
\pgfpathlineto{\pgfqpoint{1.376658in}{0.906991in}}%
\pgfpathlineto{\pgfqpoint{1.376658in}{0.904042in}}%
\pgfpathmoveto{\pgfqpoint{1.376658in}{0.904042in}}%
\pgfpathlineto{\pgfqpoint{1.376658in}{0.904042in}}%
\pgfpathlineto{\pgfqpoint{1.376658in}{0.906991in}}%
\pgfpathlineto{\pgfqpoint{1.381199in}{0.906991in}}%
\pgfpathlineto{\pgfqpoint{1.381199in}{0.904042in}}%
\pgfpathmoveto{\pgfqpoint{1.376658in}{0.906991in}}%
\pgfpathlineto{\pgfqpoint{1.376658in}{0.906991in}}%
\pgfpathlineto{\pgfqpoint{1.376658in}{0.909940in}}%
\pgfpathlineto{\pgfqpoint{1.381199in}{0.909940in}}%
\pgfpathlineto{\pgfqpoint{1.381199in}{0.906991in}}%
\pgfpathmoveto{\pgfqpoint{1.376658in}{0.909940in}}%
\pgfpathlineto{\pgfqpoint{1.376658in}{0.909940in}}%
\pgfpathlineto{\pgfqpoint{1.376658in}{0.912890in}}%
\pgfpathlineto{\pgfqpoint{1.381199in}{0.912890in}}%
\pgfpathlineto{\pgfqpoint{1.381199in}{0.909940in}}%
\pgfpathmoveto{\pgfqpoint{1.381199in}{0.906991in}}%
\pgfpathlineto{\pgfqpoint{1.381199in}{0.906991in}}%
\pgfpathlineto{\pgfqpoint{1.381199in}{0.909940in}}%
\pgfpathlineto{\pgfqpoint{1.385740in}{0.909940in}}%
\pgfpathlineto{\pgfqpoint{1.385740in}{0.906991in}}%
\pgfpathmoveto{\pgfqpoint{1.381199in}{0.909940in}}%
\pgfpathlineto{\pgfqpoint{1.381199in}{0.909940in}}%
\pgfpathlineto{\pgfqpoint{1.381199in}{0.912890in}}%
\pgfpathlineto{\pgfqpoint{1.385740in}{0.912890in}}%
\pgfpathlineto{\pgfqpoint{1.385740in}{0.909940in}}%
\pgfpathmoveto{\pgfqpoint{1.385740in}{0.909940in}}%
\pgfpathlineto{\pgfqpoint{1.385740in}{0.909940in}}%
\pgfpathlineto{\pgfqpoint{1.385740in}{0.912890in}}%
\pgfpathlineto{\pgfqpoint{1.390281in}{0.912890in}}%
\pgfpathlineto{\pgfqpoint{1.390281in}{0.909940in}}%
\pgfpathmoveto{\pgfqpoint{1.385740in}{0.912890in}}%
\pgfpathlineto{\pgfqpoint{1.385740in}{0.912890in}}%
\pgfpathlineto{\pgfqpoint{1.385740in}{0.915839in}}%
\pgfpathlineto{\pgfqpoint{1.390281in}{0.915839in}}%
\pgfpathlineto{\pgfqpoint{1.390281in}{0.912890in}}%
\pgfpathmoveto{\pgfqpoint{1.385740in}{0.915839in}}%
\pgfpathlineto{\pgfqpoint{1.385740in}{0.915839in}}%
\pgfpathlineto{\pgfqpoint{1.385740in}{0.918788in}}%
\pgfpathlineto{\pgfqpoint{1.390281in}{0.918788in}}%
\pgfpathlineto{\pgfqpoint{1.390281in}{0.915839in}}%
\pgfpathmoveto{\pgfqpoint{1.390281in}{0.912890in}}%
\pgfpathlineto{\pgfqpoint{1.390281in}{0.912890in}}%
\pgfpathlineto{\pgfqpoint{1.390281in}{0.915839in}}%
\pgfpathlineto{\pgfqpoint{1.394822in}{0.915839in}}%
\pgfpathlineto{\pgfqpoint{1.394822in}{0.912890in}}%
\pgfpathmoveto{\pgfqpoint{1.390281in}{0.915839in}}%
\pgfpathlineto{\pgfqpoint{1.390281in}{0.915839in}}%
\pgfpathlineto{\pgfqpoint{1.390281in}{0.918788in}}%
\pgfpathlineto{\pgfqpoint{1.394822in}{0.918788in}}%
\pgfpathlineto{\pgfqpoint{1.394822in}{0.915839in}}%
\pgfpathmoveto{\pgfqpoint{1.394822in}{0.915839in}}%
\pgfpathlineto{\pgfqpoint{1.394822in}{0.915839in}}%
\pgfpathlineto{\pgfqpoint{1.394822in}{0.918788in}}%
\pgfpathlineto{\pgfqpoint{1.399363in}{0.918788in}}%
\pgfpathlineto{\pgfqpoint{1.399363in}{0.915839in}}%
\pgfpathmoveto{\pgfqpoint{1.394822in}{0.918788in}}%
\pgfpathlineto{\pgfqpoint{1.394822in}{0.918788in}}%
\pgfpathlineto{\pgfqpoint{1.394822in}{0.921738in}}%
\pgfpathlineto{\pgfqpoint{1.399363in}{0.921738in}}%
\pgfpathlineto{\pgfqpoint{1.399363in}{0.918788in}}%
\pgfpathmoveto{\pgfqpoint{1.394822in}{0.921738in}}%
\pgfpathlineto{\pgfqpoint{1.394822in}{0.921738in}}%
\pgfpathlineto{\pgfqpoint{1.394822in}{0.924687in}}%
\pgfpathlineto{\pgfqpoint{1.399363in}{0.924687in}}%
\pgfpathlineto{\pgfqpoint{1.399363in}{0.921738in}}%
\pgfpathmoveto{\pgfqpoint{1.399363in}{0.918788in}}%
\pgfpathlineto{\pgfqpoint{1.399363in}{0.918788in}}%
\pgfpathlineto{\pgfqpoint{1.399363in}{0.921738in}}%
\pgfpathlineto{\pgfqpoint{1.403904in}{0.921738in}}%
\pgfpathlineto{\pgfqpoint{1.403904in}{0.918788in}}%
\pgfpathmoveto{\pgfqpoint{1.399363in}{0.921738in}}%
\pgfpathlineto{\pgfqpoint{1.399363in}{0.921738in}}%
\pgfpathlineto{\pgfqpoint{1.399363in}{0.924687in}}%
\pgfpathlineto{\pgfqpoint{1.403904in}{0.924687in}}%
\pgfpathlineto{\pgfqpoint{1.403904in}{0.921738in}}%
\pgfpathmoveto{\pgfqpoint{1.403904in}{0.921738in}}%
\pgfpathlineto{\pgfqpoint{1.403904in}{0.921738in}}%
\pgfpathlineto{\pgfqpoint{1.403904in}{0.924687in}}%
\pgfpathlineto{\pgfqpoint{1.408445in}{0.924687in}}%
\pgfpathlineto{\pgfqpoint{1.408445in}{0.921738in}}%
\pgfpathmoveto{\pgfqpoint{1.403904in}{0.924687in}}%
\pgfpathlineto{\pgfqpoint{1.403904in}{0.924687in}}%
\pgfpathlineto{\pgfqpoint{1.403904in}{0.927636in}}%
\pgfpathlineto{\pgfqpoint{1.408445in}{0.927636in}}%
\pgfpathlineto{\pgfqpoint{1.408445in}{0.924687in}}%
\pgfpathmoveto{\pgfqpoint{1.403904in}{0.927636in}}%
\pgfpathlineto{\pgfqpoint{1.403904in}{0.927636in}}%
\pgfpathlineto{\pgfqpoint{1.403904in}{0.930585in}}%
\pgfpathlineto{\pgfqpoint{1.408445in}{0.930585in}}%
\pgfpathlineto{\pgfqpoint{1.408445in}{0.927636in}}%
\pgfpathmoveto{\pgfqpoint{1.408445in}{0.924687in}}%
\pgfpathlineto{\pgfqpoint{1.408445in}{0.924687in}}%
\pgfpathlineto{\pgfqpoint{1.408445in}{0.927636in}}%
\pgfpathlineto{\pgfqpoint{1.412986in}{0.927636in}}%
\pgfpathlineto{\pgfqpoint{1.412986in}{0.924687in}}%
\pgfpathmoveto{\pgfqpoint{1.408445in}{0.927636in}}%
\pgfpathlineto{\pgfqpoint{1.408445in}{0.927636in}}%
\pgfpathlineto{\pgfqpoint{1.408445in}{0.930585in}}%
\pgfpathlineto{\pgfqpoint{1.412986in}{0.930585in}}%
\pgfpathlineto{\pgfqpoint{1.412986in}{0.927636in}}%
\pgfpathmoveto{\pgfqpoint{1.412986in}{0.927636in}}%
\pgfpathlineto{\pgfqpoint{1.412986in}{0.927636in}}%
\pgfpathlineto{\pgfqpoint{1.412986in}{0.930585in}}%
\pgfpathlineto{\pgfqpoint{1.417527in}{0.930585in}}%
\pgfpathlineto{\pgfqpoint{1.417527in}{0.927636in}}%
\pgfpathmoveto{\pgfqpoint{1.412986in}{0.930585in}}%
\pgfpathlineto{\pgfqpoint{1.412986in}{0.930585in}}%
\pgfpathlineto{\pgfqpoint{1.412986in}{0.933535in}}%
\pgfpathlineto{\pgfqpoint{1.417527in}{0.933535in}}%
\pgfpathlineto{\pgfqpoint{1.417527in}{0.930585in}}%
\pgfpathmoveto{\pgfqpoint{1.412986in}{0.933535in}}%
\pgfpathlineto{\pgfqpoint{1.412986in}{0.933535in}}%
\pgfpathlineto{\pgfqpoint{1.412986in}{0.936484in}}%
\pgfpathlineto{\pgfqpoint{1.417527in}{0.936484in}}%
\pgfpathlineto{\pgfqpoint{1.417527in}{0.933535in}}%
\pgfpathmoveto{\pgfqpoint{1.417527in}{0.930585in}}%
\pgfpathlineto{\pgfqpoint{1.417527in}{0.930585in}}%
\pgfpathlineto{\pgfqpoint{1.417527in}{0.933535in}}%
\pgfpathlineto{\pgfqpoint{1.422068in}{0.933535in}}%
\pgfpathlineto{\pgfqpoint{1.422068in}{0.930585in}}%
\pgfpathmoveto{\pgfqpoint{1.417527in}{0.933535in}}%
\pgfpathlineto{\pgfqpoint{1.417527in}{0.933535in}}%
\pgfpathlineto{\pgfqpoint{1.417527in}{0.936484in}}%
\pgfpathlineto{\pgfqpoint{1.422068in}{0.936484in}}%
\pgfpathlineto{\pgfqpoint{1.422068in}{0.933535in}}%
\pgfpathmoveto{\pgfqpoint{1.422068in}{0.933535in}}%
\pgfpathlineto{\pgfqpoint{1.422068in}{0.933535in}}%
\pgfpathlineto{\pgfqpoint{1.422068in}{0.936484in}}%
\pgfpathlineto{\pgfqpoint{1.426609in}{0.936484in}}%
\pgfpathlineto{\pgfqpoint{1.426609in}{0.933535in}}%
\pgfpathmoveto{\pgfqpoint{1.422068in}{0.936484in}}%
\pgfpathlineto{\pgfqpoint{1.422068in}{0.936484in}}%
\pgfpathlineto{\pgfqpoint{1.422068in}{0.939433in}}%
\pgfpathlineto{\pgfqpoint{1.426609in}{0.939433in}}%
\pgfpathlineto{\pgfqpoint{1.426609in}{0.936484in}}%
\pgfpathmoveto{\pgfqpoint{1.422068in}{0.939433in}}%
\pgfpathlineto{\pgfqpoint{1.422068in}{0.939433in}}%
\pgfpathlineto{\pgfqpoint{1.422068in}{0.942383in}}%
\pgfpathlineto{\pgfqpoint{1.426609in}{0.942383in}}%
\pgfpathlineto{\pgfqpoint{1.426609in}{0.939433in}}%
\pgfpathmoveto{\pgfqpoint{1.426609in}{0.936484in}}%
\pgfpathlineto{\pgfqpoint{1.426609in}{0.936484in}}%
\pgfpathlineto{\pgfqpoint{1.426609in}{0.939433in}}%
\pgfpathlineto{\pgfqpoint{1.431149in}{0.939433in}}%
\pgfpathlineto{\pgfqpoint{1.431149in}{0.936484in}}%
\pgfpathmoveto{\pgfqpoint{1.426609in}{0.939433in}}%
\pgfpathlineto{\pgfqpoint{1.426609in}{0.939433in}}%
\pgfpathlineto{\pgfqpoint{1.426609in}{0.942383in}}%
\pgfpathlineto{\pgfqpoint{1.431149in}{0.942383in}}%
\pgfpathlineto{\pgfqpoint{1.431149in}{0.939433in}}%
\pgfpathmoveto{\pgfqpoint{1.431149in}{0.939433in}}%
\pgfpathlineto{\pgfqpoint{1.431149in}{0.939433in}}%
\pgfpathlineto{\pgfqpoint{1.431149in}{0.942383in}}%
\pgfpathlineto{\pgfqpoint{1.435690in}{0.942383in}}%
\pgfpathlineto{\pgfqpoint{1.435690in}{0.939433in}}%
\pgfpathmoveto{\pgfqpoint{1.431149in}{0.942383in}}%
\pgfpathlineto{\pgfqpoint{1.431149in}{0.942383in}}%
\pgfpathlineto{\pgfqpoint{1.431149in}{0.945332in}}%
\pgfpathlineto{\pgfqpoint{1.435690in}{0.945332in}}%
\pgfpathlineto{\pgfqpoint{1.435690in}{0.942383in}}%
\pgfpathmoveto{\pgfqpoint{1.431149in}{0.945332in}}%
\pgfpathlineto{\pgfqpoint{1.431149in}{0.945332in}}%
\pgfpathlineto{\pgfqpoint{1.431149in}{0.948281in}}%
\pgfpathlineto{\pgfqpoint{1.435690in}{0.948281in}}%
\pgfpathlineto{\pgfqpoint{1.435690in}{0.945332in}}%
\pgfpathmoveto{\pgfqpoint{1.435690in}{0.942383in}}%
\pgfpathlineto{\pgfqpoint{1.435690in}{0.942383in}}%
\pgfpathlineto{\pgfqpoint{1.435690in}{0.945332in}}%
\pgfpathlineto{\pgfqpoint{1.440231in}{0.945332in}}%
\pgfpathlineto{\pgfqpoint{1.440231in}{0.942383in}}%
\pgfpathmoveto{\pgfqpoint{1.435690in}{0.945332in}}%
\pgfpathlineto{\pgfqpoint{1.435690in}{0.945332in}}%
\pgfpathlineto{\pgfqpoint{1.435690in}{0.948281in}}%
\pgfpathlineto{\pgfqpoint{1.440231in}{0.948281in}}%
\pgfpathlineto{\pgfqpoint{1.440231in}{0.945332in}}%
\pgfpathmoveto{\pgfqpoint{1.440231in}{0.945332in}}%
\pgfpathlineto{\pgfqpoint{1.440231in}{0.945332in}}%
\pgfpathlineto{\pgfqpoint{1.440231in}{0.948281in}}%
\pgfpathlineto{\pgfqpoint{1.444772in}{0.948281in}}%
\pgfpathlineto{\pgfqpoint{1.444772in}{0.945332in}}%
\pgfpathmoveto{\pgfqpoint{1.440231in}{0.948281in}}%
\pgfpathlineto{\pgfqpoint{1.440231in}{0.948281in}}%
\pgfpathlineto{\pgfqpoint{1.440231in}{0.951230in}}%
\pgfpathlineto{\pgfqpoint{1.444772in}{0.951230in}}%
\pgfpathlineto{\pgfqpoint{1.444772in}{0.948281in}}%
\pgfpathmoveto{\pgfqpoint{1.440231in}{0.951230in}}%
\pgfpathlineto{\pgfqpoint{1.440231in}{0.951230in}}%
\pgfpathlineto{\pgfqpoint{1.440231in}{0.954180in}}%
\pgfpathlineto{\pgfqpoint{1.444772in}{0.954180in}}%
\pgfpathlineto{\pgfqpoint{1.444772in}{0.951230in}}%
\pgfpathmoveto{\pgfqpoint{1.444772in}{0.948281in}}%
\pgfpathlineto{\pgfqpoint{1.444772in}{0.948281in}}%
\pgfpathlineto{\pgfqpoint{1.444772in}{0.951230in}}%
\pgfpathlineto{\pgfqpoint{1.449313in}{0.951230in}}%
\pgfpathlineto{\pgfqpoint{1.449313in}{0.948281in}}%
\pgfpathmoveto{\pgfqpoint{1.444772in}{0.951230in}}%
\pgfpathlineto{\pgfqpoint{1.444772in}{0.951230in}}%
\pgfpathlineto{\pgfqpoint{1.444772in}{0.954180in}}%
\pgfpathlineto{\pgfqpoint{1.449313in}{0.954180in}}%
\pgfpathlineto{\pgfqpoint{1.449313in}{0.951230in}}%
\pgfpathmoveto{\pgfqpoint{1.449313in}{0.951230in}}%
\pgfpathlineto{\pgfqpoint{1.449313in}{0.951230in}}%
\pgfpathlineto{\pgfqpoint{1.449313in}{0.954180in}}%
\pgfpathlineto{\pgfqpoint{1.453854in}{0.954180in}}%
\pgfpathlineto{\pgfqpoint{1.453854in}{0.951230in}}%
\pgfpathmoveto{\pgfqpoint{1.449313in}{0.954180in}}%
\pgfpathlineto{\pgfqpoint{1.449313in}{0.954180in}}%
\pgfpathlineto{\pgfqpoint{1.449313in}{0.957129in}}%
\pgfpathlineto{\pgfqpoint{1.453854in}{0.957129in}}%
\pgfpathlineto{\pgfqpoint{1.453854in}{0.954180in}}%
\pgfpathmoveto{\pgfqpoint{1.449313in}{0.957129in}}%
\pgfpathlineto{\pgfqpoint{1.449313in}{0.957129in}}%
\pgfpathlineto{\pgfqpoint{1.449313in}{0.960078in}}%
\pgfpathlineto{\pgfqpoint{1.453854in}{0.960078in}}%
\pgfpathlineto{\pgfqpoint{1.453854in}{0.957129in}}%
\pgfpathmoveto{\pgfqpoint{1.453854in}{0.954180in}}%
\pgfpathlineto{\pgfqpoint{1.453854in}{0.954180in}}%
\pgfpathlineto{\pgfqpoint{1.453854in}{0.957129in}}%
\pgfpathlineto{\pgfqpoint{1.458395in}{0.957129in}}%
\pgfpathlineto{\pgfqpoint{1.458395in}{0.954180in}}%
\pgfpathmoveto{\pgfqpoint{1.453854in}{0.957129in}}%
\pgfpathlineto{\pgfqpoint{1.453854in}{0.957129in}}%
\pgfpathlineto{\pgfqpoint{1.453854in}{0.960078in}}%
\pgfpathlineto{\pgfqpoint{1.458395in}{0.960078in}}%
\pgfpathlineto{\pgfqpoint{1.458395in}{0.957129in}}%
\pgfpathmoveto{\pgfqpoint{1.458395in}{0.957129in}}%
\pgfpathlineto{\pgfqpoint{1.458395in}{0.957129in}}%
\pgfpathlineto{\pgfqpoint{1.458395in}{0.960078in}}%
\pgfpathlineto{\pgfqpoint{1.462936in}{0.960078in}}%
\pgfpathlineto{\pgfqpoint{1.462936in}{0.957129in}}%
\pgfpathmoveto{\pgfqpoint{1.458395in}{0.960078in}}%
\pgfpathlineto{\pgfqpoint{1.458395in}{0.960078in}}%
\pgfpathlineto{\pgfqpoint{1.458395in}{0.963027in}}%
\pgfpathlineto{\pgfqpoint{1.462936in}{0.963027in}}%
\pgfpathlineto{\pgfqpoint{1.462936in}{0.960078in}}%
\pgfpathmoveto{\pgfqpoint{1.458395in}{0.963027in}}%
\pgfpathlineto{\pgfqpoint{1.458395in}{0.963027in}}%
\pgfpathlineto{\pgfqpoint{1.458395in}{0.965977in}}%
\pgfpathlineto{\pgfqpoint{1.462936in}{0.965977in}}%
\pgfpathlineto{\pgfqpoint{1.462936in}{0.963027in}}%
\pgfpathmoveto{\pgfqpoint{1.462936in}{0.960078in}}%
\pgfpathlineto{\pgfqpoint{1.462936in}{0.960078in}}%
\pgfpathlineto{\pgfqpoint{1.462936in}{0.963027in}}%
\pgfpathlineto{\pgfqpoint{1.467477in}{0.963027in}}%
\pgfpathlineto{\pgfqpoint{1.467477in}{0.960078in}}%
\pgfpathmoveto{\pgfqpoint{1.462936in}{0.963027in}}%
\pgfpathlineto{\pgfqpoint{1.462936in}{0.963027in}}%
\pgfpathlineto{\pgfqpoint{1.462936in}{0.965977in}}%
\pgfpathlineto{\pgfqpoint{1.467477in}{0.965977in}}%
\pgfpathlineto{\pgfqpoint{1.467477in}{0.963027in}}%
\pgfpathmoveto{\pgfqpoint{1.467477in}{0.963027in}}%
\pgfpathlineto{\pgfqpoint{1.467477in}{0.963027in}}%
\pgfpathlineto{\pgfqpoint{1.467477in}{0.965977in}}%
\pgfpathlineto{\pgfqpoint{1.472018in}{0.965977in}}%
\pgfpathlineto{\pgfqpoint{1.472018in}{0.963027in}}%
\pgfpathmoveto{\pgfqpoint{1.467477in}{0.965977in}}%
\pgfpathlineto{\pgfqpoint{1.467477in}{0.965977in}}%
\pgfpathlineto{\pgfqpoint{1.467477in}{0.968926in}}%
\pgfpathlineto{\pgfqpoint{1.472018in}{0.968926in}}%
\pgfpathlineto{\pgfqpoint{1.472018in}{0.965977in}}%
\pgfpathmoveto{\pgfqpoint{1.467477in}{0.968926in}}%
\pgfpathlineto{\pgfqpoint{1.467477in}{0.968926in}}%
\pgfpathlineto{\pgfqpoint{1.467477in}{0.971875in}}%
\pgfpathlineto{\pgfqpoint{1.472018in}{0.971875in}}%
\pgfpathlineto{\pgfqpoint{1.472018in}{0.968926in}}%
\pgfpathmoveto{\pgfqpoint{1.472018in}{0.965977in}}%
\pgfpathlineto{\pgfqpoint{1.472018in}{0.965977in}}%
\pgfpathlineto{\pgfqpoint{1.472018in}{0.968926in}}%
\pgfpathlineto{\pgfqpoint{1.476559in}{0.968926in}}%
\pgfpathlineto{\pgfqpoint{1.476559in}{0.965977in}}%
\pgfpathmoveto{\pgfqpoint{1.472018in}{0.968926in}}%
\pgfpathlineto{\pgfqpoint{1.472018in}{0.968926in}}%
\pgfpathlineto{\pgfqpoint{1.472018in}{0.971875in}}%
\pgfpathlineto{\pgfqpoint{1.476559in}{0.971875in}}%
\pgfpathlineto{\pgfqpoint{1.476559in}{0.968926in}}%
\pgfpathmoveto{\pgfqpoint{1.476559in}{0.968926in}}%
\pgfpathlineto{\pgfqpoint{1.476559in}{0.968926in}}%
\pgfpathlineto{\pgfqpoint{1.476559in}{0.971875in}}%
\pgfpathlineto{\pgfqpoint{1.481100in}{0.971875in}}%
\pgfpathlineto{\pgfqpoint{1.481100in}{0.968926in}}%
\pgfpathmoveto{\pgfqpoint{1.476559in}{0.971875in}}%
\pgfpathlineto{\pgfqpoint{1.476559in}{0.971875in}}%
\pgfpathlineto{\pgfqpoint{1.476559in}{0.974824in}}%
\pgfpathlineto{\pgfqpoint{1.481100in}{0.974824in}}%
\pgfpathlineto{\pgfqpoint{1.481100in}{0.971875in}}%
\pgfpathmoveto{\pgfqpoint{1.476559in}{0.974824in}}%
\pgfpathlineto{\pgfqpoint{1.476559in}{0.974824in}}%
\pgfpathlineto{\pgfqpoint{1.476559in}{0.977774in}}%
\pgfpathlineto{\pgfqpoint{1.481100in}{0.977774in}}%
\pgfpathlineto{\pgfqpoint{1.481100in}{0.974824in}}%
\pgfpathmoveto{\pgfqpoint{1.481100in}{0.971875in}}%
\pgfpathlineto{\pgfqpoint{1.481100in}{0.971875in}}%
\pgfpathlineto{\pgfqpoint{1.481100in}{0.974824in}}%
\pgfpathlineto{\pgfqpoint{1.485641in}{0.974824in}}%
\pgfpathlineto{\pgfqpoint{1.485641in}{0.971875in}}%
\pgfpathmoveto{\pgfqpoint{1.481100in}{0.974824in}}%
\pgfpathlineto{\pgfqpoint{1.481100in}{0.974824in}}%
\pgfpathlineto{\pgfqpoint{1.481100in}{0.977774in}}%
\pgfpathlineto{\pgfqpoint{1.485641in}{0.977774in}}%
\pgfpathlineto{\pgfqpoint{1.485641in}{0.974824in}}%
\pgfpathmoveto{\pgfqpoint{1.485641in}{0.974824in}}%
\pgfpathlineto{\pgfqpoint{1.485641in}{0.974824in}}%
\pgfpathlineto{\pgfqpoint{1.485641in}{0.977774in}}%
\pgfpathlineto{\pgfqpoint{1.490182in}{0.977774in}}%
\pgfpathlineto{\pgfqpoint{1.490182in}{0.974824in}}%
\pgfpathmoveto{\pgfqpoint{1.485641in}{0.977774in}}%
\pgfpathlineto{\pgfqpoint{1.485641in}{0.977774in}}%
\pgfpathlineto{\pgfqpoint{1.485641in}{0.980723in}}%
\pgfpathlineto{\pgfqpoint{1.490182in}{0.980723in}}%
\pgfpathlineto{\pgfqpoint{1.490182in}{0.977774in}}%
\pgfpathmoveto{\pgfqpoint{1.485641in}{0.980723in}}%
\pgfpathlineto{\pgfqpoint{1.485641in}{0.980723in}}%
\pgfpathlineto{\pgfqpoint{1.485641in}{0.983672in}}%
\pgfpathlineto{\pgfqpoint{1.490182in}{0.983672in}}%
\pgfpathlineto{\pgfqpoint{1.490182in}{0.980723in}}%
\pgfpathmoveto{\pgfqpoint{1.490182in}{0.977774in}}%
\pgfpathlineto{\pgfqpoint{1.490182in}{0.977774in}}%
\pgfpathlineto{\pgfqpoint{1.490182in}{0.980723in}}%
\pgfpathlineto{\pgfqpoint{1.494723in}{0.980723in}}%
\pgfpathlineto{\pgfqpoint{1.494723in}{0.977774in}}%
\pgfpathmoveto{\pgfqpoint{1.490182in}{0.980723in}}%
\pgfpathlineto{\pgfqpoint{1.490182in}{0.980723in}}%
\pgfpathlineto{\pgfqpoint{1.490182in}{0.983672in}}%
\pgfpathlineto{\pgfqpoint{1.494723in}{0.983672in}}%
\pgfpathlineto{\pgfqpoint{1.494723in}{0.980723in}}%
\pgfpathmoveto{\pgfqpoint{1.494723in}{0.980723in}}%
\pgfpathlineto{\pgfqpoint{1.494723in}{0.980723in}}%
\pgfpathlineto{\pgfqpoint{1.494723in}{0.983672in}}%
\pgfpathlineto{\pgfqpoint{1.499264in}{0.983672in}}%
\pgfpathlineto{\pgfqpoint{1.499264in}{0.980723in}}%
\pgfpathmoveto{\pgfqpoint{1.494723in}{0.983672in}}%
\pgfpathlineto{\pgfqpoint{1.494723in}{0.983672in}}%
\pgfpathlineto{\pgfqpoint{1.494723in}{0.986621in}}%
\pgfpathlineto{\pgfqpoint{1.499264in}{0.986621in}}%
\pgfpathlineto{\pgfqpoint{1.499264in}{0.983672in}}%
\pgfpathmoveto{\pgfqpoint{1.494723in}{0.986621in}}%
\pgfpathlineto{\pgfqpoint{1.494723in}{0.986621in}}%
\pgfpathlineto{\pgfqpoint{1.494723in}{0.989570in}}%
\pgfpathlineto{\pgfqpoint{1.499264in}{0.989570in}}%
\pgfpathlineto{\pgfqpoint{1.499264in}{0.986621in}}%
\pgfpathmoveto{\pgfqpoint{1.499264in}{0.983672in}}%
\pgfpathlineto{\pgfqpoint{1.499264in}{0.983672in}}%
\pgfpathlineto{\pgfqpoint{1.499264in}{0.986621in}}%
\pgfpathlineto{\pgfqpoint{1.503805in}{0.986621in}}%
\pgfpathlineto{\pgfqpoint{1.503805in}{0.983672in}}%
\pgfpathmoveto{\pgfqpoint{1.499264in}{0.986621in}}%
\pgfpathlineto{\pgfqpoint{1.499264in}{0.986621in}}%
\pgfpathlineto{\pgfqpoint{1.499264in}{0.989570in}}%
\pgfpathlineto{\pgfqpoint{1.503805in}{0.989570in}}%
\pgfpathlineto{\pgfqpoint{1.503805in}{0.986621in}}%
\pgfpathmoveto{\pgfqpoint{1.503805in}{0.986621in}}%
\pgfpathlineto{\pgfqpoint{1.503805in}{0.986621in}}%
\pgfpathlineto{\pgfqpoint{1.503805in}{0.989570in}}%
\pgfpathlineto{\pgfqpoint{1.508346in}{0.989570in}}%
\pgfpathlineto{\pgfqpoint{1.508346in}{0.986621in}}%
\pgfpathmoveto{\pgfqpoint{1.503805in}{0.989570in}}%
\pgfpathlineto{\pgfqpoint{1.503805in}{0.989570in}}%
\pgfpathlineto{\pgfqpoint{1.503805in}{0.992520in}}%
\pgfpathlineto{\pgfqpoint{1.508346in}{0.992520in}}%
\pgfpathlineto{\pgfqpoint{1.508346in}{0.989570in}}%
\pgfpathmoveto{\pgfqpoint{1.503805in}{0.992520in}}%
\pgfpathlineto{\pgfqpoint{1.503805in}{0.992520in}}%
\pgfpathlineto{\pgfqpoint{1.503805in}{0.995469in}}%
\pgfpathlineto{\pgfqpoint{1.508346in}{0.995469in}}%
\pgfpathlineto{\pgfqpoint{1.508346in}{0.992520in}}%
\pgfpathmoveto{\pgfqpoint{1.508346in}{0.989570in}}%
\pgfpathlineto{\pgfqpoint{1.508346in}{0.989570in}}%
\pgfpathlineto{\pgfqpoint{1.508346in}{0.992520in}}%
\pgfpathlineto{\pgfqpoint{1.512887in}{0.992520in}}%
\pgfpathlineto{\pgfqpoint{1.512887in}{0.989570in}}%
\pgfpathmoveto{\pgfqpoint{1.508346in}{0.992520in}}%
\pgfpathlineto{\pgfqpoint{1.508346in}{0.992520in}}%
\pgfpathlineto{\pgfqpoint{1.508346in}{0.995469in}}%
\pgfpathlineto{\pgfqpoint{1.512887in}{0.995469in}}%
\pgfpathlineto{\pgfqpoint{1.512887in}{0.992520in}}%
\pgfpathmoveto{\pgfqpoint{1.512887in}{0.992520in}}%
\pgfpathlineto{\pgfqpoint{1.512887in}{0.992520in}}%
\pgfpathlineto{\pgfqpoint{1.512887in}{0.995469in}}%
\pgfpathlineto{\pgfqpoint{1.517428in}{0.995469in}}%
\pgfpathlineto{\pgfqpoint{1.517428in}{0.992520in}}%
\pgfpathmoveto{\pgfqpoint{1.512887in}{0.995469in}}%
\pgfpathlineto{\pgfqpoint{1.512887in}{0.995469in}}%
\pgfpathlineto{\pgfqpoint{1.512887in}{0.998418in}}%
\pgfpathlineto{\pgfqpoint{1.517428in}{0.998418in}}%
\pgfpathlineto{\pgfqpoint{1.517428in}{0.995469in}}%
\pgfpathmoveto{\pgfqpoint{1.512887in}{0.998418in}}%
\pgfpathlineto{\pgfqpoint{1.512887in}{0.998418in}}%
\pgfpathlineto{\pgfqpoint{1.512887in}{1.001367in}}%
\pgfpathlineto{\pgfqpoint{1.517428in}{1.001367in}}%
\pgfpathlineto{\pgfqpoint{1.517428in}{0.998418in}}%
\pgfpathmoveto{\pgfqpoint{1.517428in}{0.995469in}}%
\pgfpathlineto{\pgfqpoint{1.517428in}{0.995469in}}%
\pgfpathlineto{\pgfqpoint{1.517428in}{0.998418in}}%
\pgfpathlineto{\pgfqpoint{1.521970in}{0.998418in}}%
\pgfpathlineto{\pgfqpoint{1.521970in}{0.995469in}}%
\pgfpathmoveto{\pgfqpoint{1.517428in}{0.998418in}}%
\pgfpathlineto{\pgfqpoint{1.517428in}{0.998418in}}%
\pgfpathlineto{\pgfqpoint{1.517428in}{1.001367in}}%
\pgfpathlineto{\pgfqpoint{1.521970in}{1.001367in}}%
\pgfpathlineto{\pgfqpoint{1.521970in}{0.998418in}}%
\pgfpathmoveto{\pgfqpoint{1.521970in}{0.998418in}}%
\pgfpathlineto{\pgfqpoint{1.521970in}{0.998418in}}%
\pgfpathlineto{\pgfqpoint{1.521970in}{1.001367in}}%
\pgfpathlineto{\pgfqpoint{1.526511in}{1.001367in}}%
\pgfpathlineto{\pgfqpoint{1.526511in}{0.998418in}}%
\pgfpathmoveto{\pgfqpoint{1.521970in}{1.001367in}}%
\pgfpathlineto{\pgfqpoint{1.521970in}{1.001367in}}%
\pgfpathlineto{\pgfqpoint{1.521970in}{1.004316in}}%
\pgfpathlineto{\pgfqpoint{1.526511in}{1.004316in}}%
\pgfpathlineto{\pgfqpoint{1.526511in}{1.001367in}}%
\pgfpathmoveto{\pgfqpoint{1.521970in}{1.004316in}}%
\pgfpathlineto{\pgfqpoint{1.521970in}{1.004316in}}%
\pgfpathlineto{\pgfqpoint{1.521970in}{1.007266in}}%
\pgfpathlineto{\pgfqpoint{1.526511in}{1.007266in}}%
\pgfpathlineto{\pgfqpoint{1.526511in}{1.004316in}}%
\pgfpathmoveto{\pgfqpoint{1.526511in}{1.001367in}}%
\pgfpathlineto{\pgfqpoint{1.526511in}{1.001367in}}%
\pgfpathlineto{\pgfqpoint{1.526511in}{1.004316in}}%
\pgfpathlineto{\pgfqpoint{1.531052in}{1.004316in}}%
\pgfpathlineto{\pgfqpoint{1.531052in}{1.001367in}}%
\pgfpathmoveto{\pgfqpoint{1.526511in}{1.004316in}}%
\pgfpathlineto{\pgfqpoint{1.526511in}{1.004316in}}%
\pgfpathlineto{\pgfqpoint{1.526511in}{1.007266in}}%
\pgfpathlineto{\pgfqpoint{1.531052in}{1.007266in}}%
\pgfpathlineto{\pgfqpoint{1.531052in}{1.004316in}}%
\pgfpathmoveto{\pgfqpoint{1.531052in}{1.004316in}}%
\pgfpathlineto{\pgfqpoint{1.531052in}{1.004316in}}%
\pgfpathlineto{\pgfqpoint{1.531052in}{1.007266in}}%
\pgfpathlineto{\pgfqpoint{1.535593in}{1.007266in}}%
\pgfpathlineto{\pgfqpoint{1.535593in}{1.004316in}}%
\pgfpathmoveto{\pgfqpoint{1.531052in}{1.007266in}}%
\pgfpathlineto{\pgfqpoint{1.531052in}{1.007266in}}%
\pgfpathlineto{\pgfqpoint{1.531052in}{1.010215in}}%
\pgfpathlineto{\pgfqpoint{1.535593in}{1.010215in}}%
\pgfpathlineto{\pgfqpoint{1.535593in}{1.007266in}}%
\pgfpathmoveto{\pgfqpoint{1.531052in}{1.010215in}}%
\pgfpathlineto{\pgfqpoint{1.531052in}{1.010215in}}%
\pgfpathlineto{\pgfqpoint{1.531052in}{1.013164in}}%
\pgfpathlineto{\pgfqpoint{1.535593in}{1.013164in}}%
\pgfpathlineto{\pgfqpoint{1.535593in}{1.010215in}}%
\pgfpathmoveto{\pgfqpoint{1.535593in}{1.007266in}}%
\pgfpathlineto{\pgfqpoint{1.535593in}{1.007266in}}%
\pgfpathlineto{\pgfqpoint{1.535593in}{1.010215in}}%
\pgfpathlineto{\pgfqpoint{1.540134in}{1.010215in}}%
\pgfpathlineto{\pgfqpoint{1.540134in}{1.007266in}}%
\pgfpathmoveto{\pgfqpoint{1.535593in}{1.010215in}}%
\pgfpathlineto{\pgfqpoint{1.535593in}{1.010215in}}%
\pgfpathlineto{\pgfqpoint{1.535593in}{1.013164in}}%
\pgfpathlineto{\pgfqpoint{1.540134in}{1.013164in}}%
\pgfpathlineto{\pgfqpoint{1.540134in}{1.010215in}}%
\pgfpathmoveto{\pgfqpoint{1.540134in}{1.010215in}}%
\pgfpathlineto{\pgfqpoint{1.540134in}{1.010215in}}%
\pgfpathlineto{\pgfqpoint{1.540134in}{1.013164in}}%
\pgfpathlineto{\pgfqpoint{1.544675in}{1.013164in}}%
\pgfpathlineto{\pgfqpoint{1.544675in}{1.010215in}}%
\pgfpathmoveto{\pgfqpoint{1.540134in}{1.013164in}}%
\pgfpathlineto{\pgfqpoint{1.540134in}{1.013164in}}%
\pgfpathlineto{\pgfqpoint{1.540134in}{1.016113in}}%
\pgfpathlineto{\pgfqpoint{1.544675in}{1.016113in}}%
\pgfpathlineto{\pgfqpoint{1.544675in}{1.013164in}}%
\pgfpathmoveto{\pgfqpoint{1.540134in}{1.016113in}}%
\pgfpathlineto{\pgfqpoint{1.540134in}{1.016113in}}%
\pgfpathlineto{\pgfqpoint{1.540134in}{1.019062in}}%
\pgfpathlineto{\pgfqpoint{1.544675in}{1.019062in}}%
\pgfpathlineto{\pgfqpoint{1.544675in}{1.016113in}}%
\pgfpathmoveto{\pgfqpoint{1.544675in}{1.013164in}}%
\pgfpathlineto{\pgfqpoint{1.544675in}{1.013164in}}%
\pgfpathlineto{\pgfqpoint{1.544675in}{1.016113in}}%
\pgfpathlineto{\pgfqpoint{1.549216in}{1.016113in}}%
\pgfpathlineto{\pgfqpoint{1.549216in}{1.013164in}}%
\pgfpathmoveto{\pgfqpoint{1.544675in}{1.016113in}}%
\pgfpathlineto{\pgfqpoint{1.544675in}{1.016113in}}%
\pgfpathlineto{\pgfqpoint{1.544675in}{1.019062in}}%
\pgfpathlineto{\pgfqpoint{1.549216in}{1.019062in}}%
\pgfpathlineto{\pgfqpoint{1.549216in}{1.016113in}}%
\pgfpathmoveto{\pgfqpoint{1.549216in}{1.016113in}}%
\pgfpathlineto{\pgfqpoint{1.549216in}{1.016113in}}%
\pgfpathlineto{\pgfqpoint{1.549216in}{1.019062in}}%
\pgfpathlineto{\pgfqpoint{1.553757in}{1.019062in}}%
\pgfpathlineto{\pgfqpoint{1.553757in}{1.016113in}}%
\pgfpathmoveto{\pgfqpoint{1.549216in}{1.019062in}}%
\pgfpathlineto{\pgfqpoint{1.549216in}{1.019062in}}%
\pgfpathlineto{\pgfqpoint{1.549216in}{1.022012in}}%
\pgfpathlineto{\pgfqpoint{1.553757in}{1.022012in}}%
\pgfpathlineto{\pgfqpoint{1.553757in}{1.019062in}}%
\pgfpathmoveto{\pgfqpoint{1.549216in}{1.022012in}}%
\pgfpathlineto{\pgfqpoint{1.549216in}{1.022012in}}%
\pgfpathlineto{\pgfqpoint{1.549216in}{1.024961in}}%
\pgfpathlineto{\pgfqpoint{1.553757in}{1.024961in}}%
\pgfpathlineto{\pgfqpoint{1.553757in}{1.022012in}}%
\pgfpathmoveto{\pgfqpoint{1.553757in}{1.019062in}}%
\pgfpathlineto{\pgfqpoint{1.553757in}{1.019062in}}%
\pgfpathlineto{\pgfqpoint{1.553757in}{1.022012in}}%
\pgfpathlineto{\pgfqpoint{1.558298in}{1.022012in}}%
\pgfpathlineto{\pgfqpoint{1.558298in}{1.019062in}}%
\pgfpathmoveto{\pgfqpoint{1.553757in}{1.022012in}}%
\pgfpathlineto{\pgfqpoint{1.553757in}{1.022012in}}%
\pgfpathlineto{\pgfqpoint{1.553757in}{1.024961in}}%
\pgfpathlineto{\pgfqpoint{1.558298in}{1.024961in}}%
\pgfpathlineto{\pgfqpoint{1.558298in}{1.022012in}}%
\pgfpathmoveto{\pgfqpoint{1.558298in}{1.022012in}}%
\pgfpathlineto{\pgfqpoint{1.558298in}{1.022012in}}%
\pgfpathlineto{\pgfqpoint{1.558298in}{1.024961in}}%
\pgfpathlineto{\pgfqpoint{1.562839in}{1.024961in}}%
\pgfpathlineto{\pgfqpoint{1.562839in}{1.022012in}}%
\pgfpathmoveto{\pgfqpoint{1.558298in}{1.024961in}}%
\pgfpathlineto{\pgfqpoint{1.558298in}{1.024961in}}%
\pgfpathlineto{\pgfqpoint{1.558298in}{1.027910in}}%
\pgfpathlineto{\pgfqpoint{1.562839in}{1.027910in}}%
\pgfpathlineto{\pgfqpoint{1.562839in}{1.024961in}}%
\pgfpathmoveto{\pgfqpoint{1.558298in}{1.027910in}}%
\pgfpathlineto{\pgfqpoint{1.558298in}{1.027910in}}%
\pgfpathlineto{\pgfqpoint{1.558298in}{1.030859in}}%
\pgfpathlineto{\pgfqpoint{1.562839in}{1.030859in}}%
\pgfpathlineto{\pgfqpoint{1.562839in}{1.027910in}}%
\pgfpathmoveto{\pgfqpoint{1.562839in}{1.024961in}}%
\pgfpathlineto{\pgfqpoint{1.562839in}{1.024961in}}%
\pgfpathlineto{\pgfqpoint{1.562839in}{1.027910in}}%
\pgfpathlineto{\pgfqpoint{1.567380in}{1.027910in}}%
\pgfpathlineto{\pgfqpoint{1.567380in}{1.024961in}}%
\pgfpathmoveto{\pgfqpoint{1.562839in}{1.027910in}}%
\pgfpathlineto{\pgfqpoint{1.562839in}{1.027910in}}%
\pgfpathlineto{\pgfqpoint{1.562839in}{1.030859in}}%
\pgfpathlineto{\pgfqpoint{1.567380in}{1.030859in}}%
\pgfpathlineto{\pgfqpoint{1.567380in}{1.027910in}}%
\pgfpathmoveto{\pgfqpoint{1.567380in}{1.027910in}}%
\pgfpathlineto{\pgfqpoint{1.567380in}{1.027910in}}%
\pgfpathlineto{\pgfqpoint{1.567380in}{1.030859in}}%
\pgfpathlineto{\pgfqpoint{1.571922in}{1.030859in}}%
\pgfpathlineto{\pgfqpoint{1.571922in}{1.027910in}}%
\pgfpathmoveto{\pgfqpoint{1.567380in}{1.030859in}}%
\pgfpathlineto{\pgfqpoint{1.567380in}{1.030859in}}%
\pgfpathlineto{\pgfqpoint{1.567380in}{1.033808in}}%
\pgfpathlineto{\pgfqpoint{1.571922in}{1.033808in}}%
\pgfpathlineto{\pgfqpoint{1.571922in}{1.030859in}}%
\pgfpathmoveto{\pgfqpoint{1.567380in}{1.033808in}}%
\pgfpathlineto{\pgfqpoint{1.567380in}{1.033808in}}%
\pgfpathlineto{\pgfqpoint{1.567380in}{1.036758in}}%
\pgfpathlineto{\pgfqpoint{1.571922in}{1.036758in}}%
\pgfpathlineto{\pgfqpoint{1.571922in}{1.033808in}}%
\pgfpathmoveto{\pgfqpoint{1.571922in}{1.030859in}}%
\pgfpathlineto{\pgfqpoint{1.571922in}{1.030859in}}%
\pgfpathlineto{\pgfqpoint{1.571922in}{1.033808in}}%
\pgfpathlineto{\pgfqpoint{1.576463in}{1.033808in}}%
\pgfpathlineto{\pgfqpoint{1.576463in}{1.030859in}}%
\pgfpathmoveto{\pgfqpoint{1.571922in}{1.033808in}}%
\pgfpathlineto{\pgfqpoint{1.571922in}{1.033808in}}%
\pgfpathlineto{\pgfqpoint{1.571922in}{1.036758in}}%
\pgfpathlineto{\pgfqpoint{1.576463in}{1.036758in}}%
\pgfpathlineto{\pgfqpoint{1.576463in}{1.033808in}}%
\pgfpathmoveto{\pgfqpoint{1.576463in}{1.033808in}}%
\pgfpathlineto{\pgfqpoint{1.576463in}{1.033808in}}%
\pgfpathlineto{\pgfqpoint{1.576463in}{1.036758in}}%
\pgfpathlineto{\pgfqpoint{1.581004in}{1.036758in}}%
\pgfpathlineto{\pgfqpoint{1.581004in}{1.033808in}}%
\pgfpathmoveto{\pgfqpoint{1.576463in}{1.036758in}}%
\pgfpathlineto{\pgfqpoint{1.576463in}{1.036758in}}%
\pgfpathlineto{\pgfqpoint{1.576463in}{1.039707in}}%
\pgfpathlineto{\pgfqpoint{1.581004in}{1.039707in}}%
\pgfpathlineto{\pgfqpoint{1.581004in}{1.036758in}}%
\pgfpathmoveto{\pgfqpoint{1.576463in}{1.039707in}}%
\pgfpathlineto{\pgfqpoint{1.576463in}{1.039707in}}%
\pgfpathlineto{\pgfqpoint{1.576463in}{1.042656in}}%
\pgfpathlineto{\pgfqpoint{1.581004in}{1.042656in}}%
\pgfpathlineto{\pgfqpoint{1.581004in}{1.039707in}}%
\pgfpathmoveto{\pgfqpoint{1.581004in}{1.036758in}}%
\pgfpathlineto{\pgfqpoint{1.581004in}{1.036758in}}%
\pgfpathlineto{\pgfqpoint{1.581004in}{1.039707in}}%
\pgfpathlineto{\pgfqpoint{1.585545in}{1.039707in}}%
\pgfpathlineto{\pgfqpoint{1.585545in}{1.036758in}}%
\pgfpathmoveto{\pgfqpoint{1.581004in}{1.039707in}}%
\pgfpathlineto{\pgfqpoint{1.581004in}{1.039707in}}%
\pgfpathlineto{\pgfqpoint{1.581004in}{1.042656in}}%
\pgfpathlineto{\pgfqpoint{1.585545in}{1.042656in}}%
\pgfpathlineto{\pgfqpoint{1.585545in}{1.039707in}}%
\pgfpathmoveto{\pgfqpoint{1.585545in}{1.039707in}}%
\pgfpathlineto{\pgfqpoint{1.585545in}{1.039707in}}%
\pgfpathlineto{\pgfqpoint{1.585545in}{1.042656in}}%
\pgfpathlineto{\pgfqpoint{1.590086in}{1.042656in}}%
\pgfpathlineto{\pgfqpoint{1.590086in}{1.039707in}}%
\pgfpathmoveto{\pgfqpoint{1.585545in}{1.042656in}}%
\pgfpathlineto{\pgfqpoint{1.585545in}{1.042656in}}%
\pgfpathlineto{\pgfqpoint{1.585545in}{1.045605in}}%
\pgfpathlineto{\pgfqpoint{1.590086in}{1.045605in}}%
\pgfpathlineto{\pgfqpoint{1.590086in}{1.042656in}}%
\pgfpathmoveto{\pgfqpoint{1.585545in}{1.045605in}}%
\pgfpathlineto{\pgfqpoint{1.585545in}{1.045605in}}%
\pgfpathlineto{\pgfqpoint{1.585545in}{1.048554in}}%
\pgfpathlineto{\pgfqpoint{1.590086in}{1.048554in}}%
\pgfpathlineto{\pgfqpoint{1.590086in}{1.045605in}}%
\pgfpathmoveto{\pgfqpoint{1.590086in}{1.042656in}}%
\pgfpathlineto{\pgfqpoint{1.590086in}{1.042656in}}%
\pgfpathlineto{\pgfqpoint{1.590086in}{1.045605in}}%
\pgfpathlineto{\pgfqpoint{1.594627in}{1.045605in}}%
\pgfpathlineto{\pgfqpoint{1.594627in}{1.042656in}}%
\pgfpathmoveto{\pgfqpoint{1.590086in}{1.045605in}}%
\pgfpathlineto{\pgfqpoint{1.590086in}{1.045605in}}%
\pgfpathlineto{\pgfqpoint{1.590086in}{1.048554in}}%
\pgfpathlineto{\pgfqpoint{1.594627in}{1.048554in}}%
\pgfpathlineto{\pgfqpoint{1.594627in}{1.045605in}}%
\pgfpathmoveto{\pgfqpoint{1.594627in}{1.045605in}}%
\pgfpathlineto{\pgfqpoint{1.594627in}{1.045605in}}%
\pgfpathlineto{\pgfqpoint{1.594627in}{1.048554in}}%
\pgfpathlineto{\pgfqpoint{1.599168in}{1.048554in}}%
\pgfpathlineto{\pgfqpoint{1.599168in}{1.045605in}}%
\pgfpathmoveto{\pgfqpoint{1.594627in}{1.048554in}}%
\pgfpathlineto{\pgfqpoint{1.594627in}{1.048554in}}%
\pgfpathlineto{\pgfqpoint{1.594627in}{1.051504in}}%
\pgfpathlineto{\pgfqpoint{1.599168in}{1.051504in}}%
\pgfpathlineto{\pgfqpoint{1.599168in}{1.048554in}}%
\pgfpathmoveto{\pgfqpoint{1.594627in}{1.051504in}}%
\pgfpathlineto{\pgfqpoint{1.594627in}{1.051504in}}%
\pgfpathlineto{\pgfqpoint{1.594627in}{1.054453in}}%
\pgfpathlineto{\pgfqpoint{1.599168in}{1.054453in}}%
\pgfpathlineto{\pgfqpoint{1.599168in}{1.051504in}}%
\pgfpathmoveto{\pgfqpoint{1.599168in}{1.048554in}}%
\pgfpathlineto{\pgfqpoint{1.599168in}{1.048554in}}%
\pgfpathlineto{\pgfqpoint{1.599168in}{1.051504in}}%
\pgfpathlineto{\pgfqpoint{1.603709in}{1.051504in}}%
\pgfpathlineto{\pgfqpoint{1.603709in}{1.048554in}}%
\pgfpathmoveto{\pgfqpoint{1.599168in}{1.051504in}}%
\pgfpathlineto{\pgfqpoint{1.599168in}{1.051504in}}%
\pgfpathlineto{\pgfqpoint{1.599168in}{1.054453in}}%
\pgfpathlineto{\pgfqpoint{1.603709in}{1.054453in}}%
\pgfpathlineto{\pgfqpoint{1.603709in}{1.051504in}}%
\pgfpathmoveto{\pgfqpoint{1.603709in}{1.051504in}}%
\pgfpathlineto{\pgfqpoint{1.603709in}{1.051504in}}%
\pgfpathlineto{\pgfqpoint{1.603709in}{1.054453in}}%
\pgfpathlineto{\pgfqpoint{1.608250in}{1.054453in}}%
\pgfpathlineto{\pgfqpoint{1.608250in}{1.051504in}}%
\pgfpathmoveto{\pgfqpoint{1.603709in}{1.054453in}}%
\pgfpathlineto{\pgfqpoint{1.603709in}{1.054453in}}%
\pgfpathlineto{\pgfqpoint{1.603709in}{1.057402in}}%
\pgfpathlineto{\pgfqpoint{1.608250in}{1.057402in}}%
\pgfpathlineto{\pgfqpoint{1.608250in}{1.054453in}}%
\pgfpathmoveto{\pgfqpoint{1.603709in}{1.057402in}}%
\pgfpathlineto{\pgfqpoint{1.603709in}{1.057402in}}%
\pgfpathlineto{\pgfqpoint{1.603709in}{1.060351in}}%
\pgfpathlineto{\pgfqpoint{1.608250in}{1.060351in}}%
\pgfpathlineto{\pgfqpoint{1.608250in}{1.057402in}}%
\pgfpathmoveto{\pgfqpoint{1.608250in}{1.054453in}}%
\pgfpathlineto{\pgfqpoint{1.608250in}{1.054453in}}%
\pgfpathlineto{\pgfqpoint{1.608250in}{1.057402in}}%
\pgfpathlineto{\pgfqpoint{1.612791in}{1.057402in}}%
\pgfpathlineto{\pgfqpoint{1.612791in}{1.054453in}}%
\pgfpathmoveto{\pgfqpoint{1.608250in}{1.057402in}}%
\pgfpathlineto{\pgfqpoint{1.608250in}{1.057402in}}%
\pgfpathlineto{\pgfqpoint{1.608250in}{1.060351in}}%
\pgfpathlineto{\pgfqpoint{1.612791in}{1.060351in}}%
\pgfpathlineto{\pgfqpoint{1.612791in}{1.057402in}}%
\pgfpathmoveto{\pgfqpoint{1.612791in}{1.057402in}}%
\pgfpathlineto{\pgfqpoint{1.612791in}{1.057402in}}%
\pgfpathlineto{\pgfqpoint{1.612791in}{1.060351in}}%
\pgfpathlineto{\pgfqpoint{1.617332in}{1.060351in}}%
\pgfpathlineto{\pgfqpoint{1.617332in}{1.057402in}}%
\pgfpathmoveto{\pgfqpoint{1.612791in}{1.060351in}}%
\pgfpathlineto{\pgfqpoint{1.612791in}{1.060351in}}%
\pgfpathlineto{\pgfqpoint{1.612791in}{1.063300in}}%
\pgfpathlineto{\pgfqpoint{1.617332in}{1.063300in}}%
\pgfpathlineto{\pgfqpoint{1.617332in}{1.060351in}}%
\pgfpathmoveto{\pgfqpoint{1.612791in}{1.063300in}}%
\pgfpathlineto{\pgfqpoint{1.612791in}{1.063300in}}%
\pgfpathlineto{\pgfqpoint{1.612791in}{1.066250in}}%
\pgfpathlineto{\pgfqpoint{1.617332in}{1.066250in}}%
\pgfpathlineto{\pgfqpoint{1.617332in}{1.063300in}}%
\pgfpathmoveto{\pgfqpoint{1.617332in}{1.060351in}}%
\pgfpathlineto{\pgfqpoint{1.617332in}{1.060351in}}%
\pgfpathlineto{\pgfqpoint{1.617332in}{1.063300in}}%
\pgfpathlineto{\pgfqpoint{1.621874in}{1.063300in}}%
\pgfpathlineto{\pgfqpoint{1.621874in}{1.060351in}}%
\pgfpathmoveto{\pgfqpoint{1.617332in}{1.063300in}}%
\pgfpathlineto{\pgfqpoint{1.617332in}{1.063300in}}%
\pgfpathlineto{\pgfqpoint{1.617332in}{1.066250in}}%
\pgfpathlineto{\pgfqpoint{1.621874in}{1.066250in}}%
\pgfpathlineto{\pgfqpoint{1.621874in}{1.063300in}}%
\pgfpathmoveto{\pgfqpoint{1.621874in}{1.063300in}}%
\pgfpathlineto{\pgfqpoint{1.621874in}{1.063300in}}%
\pgfpathlineto{\pgfqpoint{1.621874in}{1.066250in}}%
\pgfpathlineto{\pgfqpoint{1.626415in}{1.066250in}}%
\pgfpathlineto{\pgfqpoint{1.626415in}{1.063300in}}%
\pgfpathmoveto{\pgfqpoint{1.621874in}{1.066250in}}%
\pgfpathlineto{\pgfqpoint{1.621874in}{1.066250in}}%
\pgfpathlineto{\pgfqpoint{1.621874in}{1.069199in}}%
\pgfpathlineto{\pgfqpoint{1.626415in}{1.069199in}}%
\pgfpathlineto{\pgfqpoint{1.626415in}{1.066250in}}%
\pgfpathmoveto{\pgfqpoint{1.621874in}{1.069199in}}%
\pgfpathlineto{\pgfqpoint{1.621874in}{1.069199in}}%
\pgfpathlineto{\pgfqpoint{1.621874in}{1.072148in}}%
\pgfpathlineto{\pgfqpoint{1.626415in}{1.072148in}}%
\pgfpathlineto{\pgfqpoint{1.626415in}{1.069199in}}%
\pgfpathmoveto{\pgfqpoint{1.626415in}{1.066250in}}%
\pgfpathlineto{\pgfqpoint{1.626415in}{1.066250in}}%
\pgfpathlineto{\pgfqpoint{1.626415in}{1.069199in}}%
\pgfpathlineto{\pgfqpoint{1.630956in}{1.069199in}}%
\pgfpathlineto{\pgfqpoint{1.630956in}{1.066250in}}%
\pgfpathmoveto{\pgfqpoint{1.626415in}{1.069199in}}%
\pgfpathlineto{\pgfqpoint{1.626415in}{1.069199in}}%
\pgfpathlineto{\pgfqpoint{1.626415in}{1.072148in}}%
\pgfpathlineto{\pgfqpoint{1.630956in}{1.072148in}}%
\pgfpathlineto{\pgfqpoint{1.630956in}{1.069199in}}%
\pgfpathmoveto{\pgfqpoint{1.630956in}{1.069199in}}%
\pgfpathlineto{\pgfqpoint{1.630956in}{1.069199in}}%
\pgfpathlineto{\pgfqpoint{1.630956in}{1.072148in}}%
\pgfpathlineto{\pgfqpoint{1.635497in}{1.072148in}}%
\pgfpathlineto{\pgfqpoint{1.635497in}{1.069199in}}%
\pgfpathmoveto{\pgfqpoint{1.630956in}{1.072148in}}%
\pgfpathlineto{\pgfqpoint{1.630956in}{1.072148in}}%
\pgfpathlineto{\pgfqpoint{1.630956in}{1.075097in}}%
\pgfpathlineto{\pgfqpoint{1.635497in}{1.075097in}}%
\pgfpathlineto{\pgfqpoint{1.635497in}{1.072148in}}%
\pgfpathmoveto{\pgfqpoint{1.630956in}{1.075097in}}%
\pgfpathlineto{\pgfqpoint{1.630956in}{1.075097in}}%
\pgfpathlineto{\pgfqpoint{1.630956in}{1.078046in}}%
\pgfpathlineto{\pgfqpoint{1.635497in}{1.078046in}}%
\pgfpathlineto{\pgfqpoint{1.635497in}{1.075097in}}%
\pgfpathmoveto{\pgfqpoint{1.635497in}{1.072148in}}%
\pgfpathlineto{\pgfqpoint{1.635497in}{1.072148in}}%
\pgfpathlineto{\pgfqpoint{1.635497in}{1.075097in}}%
\pgfpathlineto{\pgfqpoint{1.640038in}{1.075097in}}%
\pgfpathlineto{\pgfqpoint{1.640038in}{1.072148in}}%
\pgfpathmoveto{\pgfqpoint{1.635497in}{1.075097in}}%
\pgfpathlineto{\pgfqpoint{1.635497in}{1.075097in}}%
\pgfpathlineto{\pgfqpoint{1.635497in}{1.078046in}}%
\pgfpathlineto{\pgfqpoint{1.640038in}{1.078046in}}%
\pgfpathlineto{\pgfqpoint{1.640038in}{1.075097in}}%
\pgfpathmoveto{\pgfqpoint{1.640038in}{1.075097in}}%
\pgfpathlineto{\pgfqpoint{1.640038in}{1.075097in}}%
\pgfpathlineto{\pgfqpoint{1.640038in}{1.078046in}}%
\pgfpathlineto{\pgfqpoint{1.644579in}{1.078046in}}%
\pgfpathlineto{\pgfqpoint{1.644579in}{1.075097in}}%
\pgfpathmoveto{\pgfqpoint{1.640038in}{1.078046in}}%
\pgfpathlineto{\pgfqpoint{1.640038in}{1.078046in}}%
\pgfpathlineto{\pgfqpoint{1.640038in}{1.080996in}}%
\pgfpathlineto{\pgfqpoint{1.644579in}{1.080996in}}%
\pgfpathlineto{\pgfqpoint{1.644579in}{1.078046in}}%
\pgfpathmoveto{\pgfqpoint{1.640038in}{1.080996in}}%
\pgfpathlineto{\pgfqpoint{1.640038in}{1.080996in}}%
\pgfpathlineto{\pgfqpoint{1.640038in}{1.083945in}}%
\pgfpathlineto{\pgfqpoint{1.644579in}{1.083945in}}%
\pgfpathlineto{\pgfqpoint{1.644579in}{1.080996in}}%
\pgfpathmoveto{\pgfqpoint{1.644579in}{1.078046in}}%
\pgfpathlineto{\pgfqpoint{1.644579in}{1.078046in}}%
\pgfpathlineto{\pgfqpoint{1.644579in}{1.080996in}}%
\pgfpathlineto{\pgfqpoint{1.649120in}{1.080996in}}%
\pgfpathlineto{\pgfqpoint{1.649120in}{1.078046in}}%
\pgfpathmoveto{\pgfqpoint{1.644579in}{1.080996in}}%
\pgfpathlineto{\pgfqpoint{1.644579in}{1.080996in}}%
\pgfpathlineto{\pgfqpoint{1.644579in}{1.083945in}}%
\pgfpathlineto{\pgfqpoint{1.649120in}{1.083945in}}%
\pgfpathlineto{\pgfqpoint{1.649120in}{1.080996in}}%
\pgfpathmoveto{\pgfqpoint{1.649120in}{1.080996in}}%
\pgfpathlineto{\pgfqpoint{1.649120in}{1.080996in}}%
\pgfpathlineto{\pgfqpoint{1.649120in}{1.083945in}}%
\pgfpathlineto{\pgfqpoint{1.653661in}{1.083945in}}%
\pgfpathlineto{\pgfqpoint{1.653661in}{1.080996in}}%
\pgfpathmoveto{\pgfqpoint{1.649120in}{1.083945in}}%
\pgfpathlineto{\pgfqpoint{1.649120in}{1.083945in}}%
\pgfpathlineto{\pgfqpoint{1.649120in}{1.086894in}}%
\pgfpathlineto{\pgfqpoint{1.653661in}{1.086894in}}%
\pgfpathlineto{\pgfqpoint{1.653661in}{1.083945in}}%
\pgfpathmoveto{\pgfqpoint{1.649120in}{1.086894in}}%
\pgfpathlineto{\pgfqpoint{1.649120in}{1.086894in}}%
\pgfpathlineto{\pgfqpoint{1.649120in}{1.089843in}}%
\pgfpathlineto{\pgfqpoint{1.653661in}{1.089843in}}%
\pgfpathlineto{\pgfqpoint{1.653661in}{1.086894in}}%
\pgfpathmoveto{\pgfqpoint{1.653661in}{1.083945in}}%
\pgfpathlineto{\pgfqpoint{1.653661in}{1.083945in}}%
\pgfpathlineto{\pgfqpoint{1.653661in}{1.086894in}}%
\pgfpathlineto{\pgfqpoint{1.658202in}{1.086894in}}%
\pgfpathlineto{\pgfqpoint{1.658202in}{1.083945in}}%
\pgfpathmoveto{\pgfqpoint{1.653661in}{1.086894in}}%
\pgfpathlineto{\pgfqpoint{1.653661in}{1.086894in}}%
\pgfpathlineto{\pgfqpoint{1.653661in}{1.089843in}}%
\pgfpathlineto{\pgfqpoint{1.658202in}{1.089843in}}%
\pgfpathlineto{\pgfqpoint{1.658202in}{1.086894in}}%
\pgfpathmoveto{\pgfqpoint{1.658202in}{1.086894in}}%
\pgfpathlineto{\pgfqpoint{1.658202in}{1.086894in}}%
\pgfpathlineto{\pgfqpoint{1.658202in}{1.089843in}}%
\pgfpathlineto{\pgfqpoint{1.662743in}{1.089843in}}%
\pgfpathlineto{\pgfqpoint{1.662743in}{1.086894in}}%
\pgfpathmoveto{\pgfqpoint{1.658202in}{1.089843in}}%
\pgfpathlineto{\pgfqpoint{1.658202in}{1.089843in}}%
\pgfpathlineto{\pgfqpoint{1.658202in}{1.092792in}}%
\pgfpathlineto{\pgfqpoint{1.662743in}{1.092792in}}%
\pgfpathlineto{\pgfqpoint{1.662743in}{1.089843in}}%
\pgfpathmoveto{\pgfqpoint{1.658202in}{1.092792in}}%
\pgfpathlineto{\pgfqpoint{1.658202in}{1.092792in}}%
\pgfpathlineto{\pgfqpoint{1.658202in}{1.095741in}}%
\pgfpathlineto{\pgfqpoint{1.662743in}{1.095741in}}%
\pgfpathlineto{\pgfqpoint{1.662743in}{1.092792in}}%
\pgfpathmoveto{\pgfqpoint{1.662743in}{1.089843in}}%
\pgfpathlineto{\pgfqpoint{1.662743in}{1.089843in}}%
\pgfpathlineto{\pgfqpoint{1.662743in}{1.092792in}}%
\pgfpathlineto{\pgfqpoint{1.667284in}{1.092792in}}%
\pgfpathlineto{\pgfqpoint{1.667284in}{1.089843in}}%
\pgfpathmoveto{\pgfqpoint{1.662743in}{1.092792in}}%
\pgfpathlineto{\pgfqpoint{1.662743in}{1.092792in}}%
\pgfpathlineto{\pgfqpoint{1.662743in}{1.095741in}}%
\pgfpathlineto{\pgfqpoint{1.667284in}{1.095741in}}%
\pgfpathlineto{\pgfqpoint{1.667284in}{1.092792in}}%
\pgfpathmoveto{\pgfqpoint{1.667284in}{1.092792in}}%
\pgfpathlineto{\pgfqpoint{1.667284in}{1.092792in}}%
\pgfpathlineto{\pgfqpoint{1.667284in}{1.095741in}}%
\pgfpathlineto{\pgfqpoint{1.671825in}{1.095741in}}%
\pgfpathlineto{\pgfqpoint{1.671825in}{1.092792in}}%
\pgfpathmoveto{\pgfqpoint{1.667284in}{1.095741in}}%
\pgfpathlineto{\pgfqpoint{1.667284in}{1.095741in}}%
\pgfpathlineto{\pgfqpoint{1.667284in}{1.098691in}}%
\pgfpathlineto{\pgfqpoint{1.671825in}{1.098691in}}%
\pgfpathlineto{\pgfqpoint{1.671825in}{1.095741in}}%
\pgfpathmoveto{\pgfqpoint{1.667284in}{1.098691in}}%
\pgfpathlineto{\pgfqpoint{1.667284in}{1.098691in}}%
\pgfpathlineto{\pgfqpoint{1.667284in}{1.101640in}}%
\pgfpathlineto{\pgfqpoint{1.671825in}{1.101640in}}%
\pgfpathlineto{\pgfqpoint{1.671825in}{1.098691in}}%
\pgfpathmoveto{\pgfqpoint{1.671825in}{1.098691in}}%
\pgfpathlineto{\pgfqpoint{1.671825in}{1.098691in}}%
\pgfpathlineto{\pgfqpoint{1.671825in}{1.101640in}}%
\pgfpathlineto{\pgfqpoint{1.676366in}{1.101640in}}%
\pgfpathlineto{\pgfqpoint{1.676366in}{1.098691in}}%
\pgfpathmoveto{\pgfqpoint{1.667284in}{1.101640in}}%
\pgfpathlineto{\pgfqpoint{1.667284in}{1.101640in}}%
\pgfpathlineto{\pgfqpoint{1.667284in}{1.104589in}}%
\pgfpathlineto{\pgfqpoint{1.671825in}{1.104589in}}%
\pgfpathlineto{\pgfqpoint{1.671825in}{1.101640in}}%
\pgfpathmoveto{\pgfqpoint{1.667284in}{1.104589in}}%
\pgfpathlineto{\pgfqpoint{1.667284in}{1.104589in}}%
\pgfpathlineto{\pgfqpoint{1.667284in}{1.107538in}}%
\pgfpathlineto{\pgfqpoint{1.671825in}{1.107538in}}%
\pgfpathlineto{\pgfqpoint{1.671825in}{1.104589in}}%
\pgfpathmoveto{\pgfqpoint{1.671825in}{1.101640in}}%
\pgfpathlineto{\pgfqpoint{1.671825in}{1.101640in}}%
\pgfpathlineto{\pgfqpoint{1.671825in}{1.104589in}}%
\pgfpathlineto{\pgfqpoint{1.676366in}{1.104589in}}%
\pgfpathlineto{\pgfqpoint{1.676366in}{1.101640in}}%
\pgfpathmoveto{\pgfqpoint{1.671825in}{1.104589in}}%
\pgfpathlineto{\pgfqpoint{1.671825in}{1.104589in}}%
\pgfpathlineto{\pgfqpoint{1.671825in}{1.107538in}}%
\pgfpathlineto{\pgfqpoint{1.676366in}{1.107538in}}%
\pgfpathlineto{\pgfqpoint{1.676366in}{1.104589in}}%
\pgfpathmoveto{\pgfqpoint{1.676366in}{1.101640in}}%
\pgfpathlineto{\pgfqpoint{1.676366in}{1.101640in}}%
\pgfpathlineto{\pgfqpoint{1.676366in}{1.104589in}}%
\pgfpathlineto{\pgfqpoint{1.680907in}{1.104589in}}%
\pgfpathlineto{\pgfqpoint{1.680907in}{1.101640in}}%
\pgfpathmoveto{\pgfqpoint{1.676366in}{1.104589in}}%
\pgfpathlineto{\pgfqpoint{1.676366in}{1.104589in}}%
\pgfpathlineto{\pgfqpoint{1.676366in}{1.107538in}}%
\pgfpathlineto{\pgfqpoint{1.680907in}{1.107538in}}%
\pgfpathlineto{\pgfqpoint{1.680907in}{1.104589in}}%
\pgfpathmoveto{\pgfqpoint{1.680907in}{1.104589in}}%
\pgfpathlineto{\pgfqpoint{1.680907in}{1.104589in}}%
\pgfpathlineto{\pgfqpoint{1.680907in}{1.107538in}}%
\pgfpathlineto{\pgfqpoint{1.685448in}{1.107538in}}%
\pgfpathlineto{\pgfqpoint{1.685448in}{1.104589in}}%
\pgfpathmoveto{\pgfqpoint{1.676366in}{1.107538in}}%
\pgfpathlineto{\pgfqpoint{1.676366in}{1.107538in}}%
\pgfpathlineto{\pgfqpoint{1.676366in}{1.110487in}}%
\pgfpathlineto{\pgfqpoint{1.680907in}{1.110487in}}%
\pgfpathlineto{\pgfqpoint{1.680907in}{1.107538in}}%
\pgfpathmoveto{\pgfqpoint{1.676366in}{1.110487in}}%
\pgfpathlineto{\pgfqpoint{1.676366in}{1.110487in}}%
\pgfpathlineto{\pgfqpoint{1.676366in}{1.113436in}}%
\pgfpathlineto{\pgfqpoint{1.680907in}{1.113436in}}%
\pgfpathlineto{\pgfqpoint{1.680907in}{1.110487in}}%
\pgfpathmoveto{\pgfqpoint{1.680907in}{1.107538in}}%
\pgfpathlineto{\pgfqpoint{1.680907in}{1.107538in}}%
\pgfpathlineto{\pgfqpoint{1.680907in}{1.110487in}}%
\pgfpathlineto{\pgfqpoint{1.685448in}{1.110487in}}%
\pgfpathlineto{\pgfqpoint{1.685448in}{1.107538in}}%
\pgfpathmoveto{\pgfqpoint{1.680907in}{1.110487in}}%
\pgfpathlineto{\pgfqpoint{1.680907in}{1.110487in}}%
\pgfpathlineto{\pgfqpoint{1.680907in}{1.113436in}}%
\pgfpathlineto{\pgfqpoint{1.685448in}{1.113436in}}%
\pgfpathlineto{\pgfqpoint{1.685448in}{1.110487in}}%
\pgfpathmoveto{\pgfqpoint{1.685448in}{1.107538in}}%
\pgfpathlineto{\pgfqpoint{1.685448in}{1.107538in}}%
\pgfpathlineto{\pgfqpoint{1.685448in}{1.110487in}}%
\pgfpathlineto{\pgfqpoint{1.689989in}{1.110487in}}%
\pgfpathlineto{\pgfqpoint{1.689989in}{1.107538in}}%
\pgfpathmoveto{\pgfqpoint{1.685448in}{1.110487in}}%
\pgfpathlineto{\pgfqpoint{1.685448in}{1.110487in}}%
\pgfpathlineto{\pgfqpoint{1.685448in}{1.113436in}}%
\pgfpathlineto{\pgfqpoint{1.689989in}{1.113436in}}%
\pgfpathlineto{\pgfqpoint{1.689989in}{1.110487in}}%
\pgfpathmoveto{\pgfqpoint{1.689989in}{1.110487in}}%
\pgfpathlineto{\pgfqpoint{1.689989in}{1.110487in}}%
\pgfpathlineto{\pgfqpoint{1.689989in}{1.113436in}}%
\pgfpathlineto{\pgfqpoint{1.694530in}{1.113436in}}%
\pgfpathlineto{\pgfqpoint{1.694530in}{1.110487in}}%
\pgfpathmoveto{\pgfqpoint{1.685448in}{1.113436in}}%
\pgfpathlineto{\pgfqpoint{1.685448in}{1.113436in}}%
\pgfpathlineto{\pgfqpoint{1.685448in}{1.116386in}}%
\pgfpathlineto{\pgfqpoint{1.689989in}{1.116386in}}%
\pgfpathlineto{\pgfqpoint{1.689989in}{1.113436in}}%
\pgfpathmoveto{\pgfqpoint{1.685448in}{1.116386in}}%
\pgfpathlineto{\pgfqpoint{1.685448in}{1.116386in}}%
\pgfpathlineto{\pgfqpoint{1.685448in}{1.119335in}}%
\pgfpathlineto{\pgfqpoint{1.689989in}{1.119335in}}%
\pgfpathlineto{\pgfqpoint{1.689989in}{1.116386in}}%
\pgfpathmoveto{\pgfqpoint{1.689989in}{1.113436in}}%
\pgfpathlineto{\pgfqpoint{1.689989in}{1.113436in}}%
\pgfpathlineto{\pgfqpoint{1.689989in}{1.116386in}}%
\pgfpathlineto{\pgfqpoint{1.694530in}{1.116386in}}%
\pgfpathlineto{\pgfqpoint{1.694530in}{1.113436in}}%
\pgfpathmoveto{\pgfqpoint{1.689989in}{1.116386in}}%
\pgfpathlineto{\pgfqpoint{1.689989in}{1.116386in}}%
\pgfpathlineto{\pgfqpoint{1.689989in}{1.119335in}}%
\pgfpathlineto{\pgfqpoint{1.694530in}{1.119335in}}%
\pgfpathlineto{\pgfqpoint{1.694530in}{1.116386in}}%
\pgfpathmoveto{\pgfqpoint{1.694530in}{1.113436in}}%
\pgfpathlineto{\pgfqpoint{1.694530in}{1.113436in}}%
\pgfpathlineto{\pgfqpoint{1.694530in}{1.116386in}}%
\pgfpathlineto{\pgfqpoint{1.699071in}{1.116386in}}%
\pgfpathlineto{\pgfqpoint{1.699071in}{1.113436in}}%
\pgfpathmoveto{\pgfqpoint{1.694530in}{1.116386in}}%
\pgfpathlineto{\pgfqpoint{1.694530in}{1.116386in}}%
\pgfpathlineto{\pgfqpoint{1.694530in}{1.119335in}}%
\pgfpathlineto{\pgfqpoint{1.699071in}{1.119335in}}%
\pgfpathlineto{\pgfqpoint{1.699071in}{1.116386in}}%
\pgfpathmoveto{\pgfqpoint{1.699071in}{1.116386in}}%
\pgfpathlineto{\pgfqpoint{1.699071in}{1.116386in}}%
\pgfpathlineto{\pgfqpoint{1.699071in}{1.119335in}}%
\pgfpathlineto{\pgfqpoint{1.703612in}{1.119335in}}%
\pgfpathlineto{\pgfqpoint{1.703612in}{1.116386in}}%
\pgfpathmoveto{\pgfqpoint{1.694530in}{1.119335in}}%
\pgfpathlineto{\pgfqpoint{1.694530in}{1.119335in}}%
\pgfpathlineto{\pgfqpoint{1.694530in}{1.122284in}}%
\pgfpathlineto{\pgfqpoint{1.699071in}{1.122284in}}%
\pgfpathlineto{\pgfqpoint{1.699071in}{1.119335in}}%
\pgfpathmoveto{\pgfqpoint{1.694530in}{1.122284in}}%
\pgfpathlineto{\pgfqpoint{1.694530in}{1.122284in}}%
\pgfpathlineto{\pgfqpoint{1.694530in}{1.125233in}}%
\pgfpathlineto{\pgfqpoint{1.699071in}{1.125233in}}%
\pgfpathlineto{\pgfqpoint{1.699071in}{1.122284in}}%
\pgfpathmoveto{\pgfqpoint{1.699071in}{1.119335in}}%
\pgfpathlineto{\pgfqpoint{1.699071in}{1.119335in}}%
\pgfpathlineto{\pgfqpoint{1.699071in}{1.122284in}}%
\pgfpathlineto{\pgfqpoint{1.703612in}{1.122284in}}%
\pgfpathlineto{\pgfqpoint{1.703612in}{1.119335in}}%
\pgfpathmoveto{\pgfqpoint{1.699071in}{1.122284in}}%
\pgfpathlineto{\pgfqpoint{1.699071in}{1.122284in}}%
\pgfpathlineto{\pgfqpoint{1.699071in}{1.125233in}}%
\pgfpathlineto{\pgfqpoint{1.703612in}{1.125233in}}%
\pgfpathlineto{\pgfqpoint{1.703612in}{1.122284in}}%
\pgfpathmoveto{\pgfqpoint{1.703612in}{1.119335in}}%
\pgfpathlineto{\pgfqpoint{1.703612in}{1.119335in}}%
\pgfpathlineto{\pgfqpoint{1.703612in}{1.122284in}}%
\pgfpathlineto{\pgfqpoint{1.708153in}{1.122284in}}%
\pgfpathlineto{\pgfqpoint{1.708153in}{1.119335in}}%
\pgfpathmoveto{\pgfqpoint{1.703612in}{1.122284in}}%
\pgfpathlineto{\pgfqpoint{1.703612in}{1.122284in}}%
\pgfpathlineto{\pgfqpoint{1.703612in}{1.125233in}}%
\pgfpathlineto{\pgfqpoint{1.708153in}{1.125233in}}%
\pgfpathlineto{\pgfqpoint{1.708153in}{1.122284in}}%
\pgfpathmoveto{\pgfqpoint{1.708153in}{1.122284in}}%
\pgfpathlineto{\pgfqpoint{1.708153in}{1.122284in}}%
\pgfpathlineto{\pgfqpoint{1.708153in}{1.125233in}}%
\pgfpathlineto{\pgfqpoint{1.712694in}{1.125233in}}%
\pgfpathlineto{\pgfqpoint{1.712694in}{1.122284in}}%
\pgfpathmoveto{\pgfqpoint{1.703612in}{1.125233in}}%
\pgfpathlineto{\pgfqpoint{1.703612in}{1.125233in}}%
\pgfpathlineto{\pgfqpoint{1.703612in}{1.128182in}}%
\pgfpathlineto{\pgfqpoint{1.708153in}{1.128182in}}%
\pgfpathlineto{\pgfqpoint{1.708153in}{1.125233in}}%
\pgfpathmoveto{\pgfqpoint{1.703612in}{1.128182in}}%
\pgfpathlineto{\pgfqpoint{1.703612in}{1.128182in}}%
\pgfpathlineto{\pgfqpoint{1.703612in}{1.131131in}}%
\pgfpathlineto{\pgfqpoint{1.708153in}{1.131131in}}%
\pgfpathlineto{\pgfqpoint{1.708153in}{1.128182in}}%
\pgfpathmoveto{\pgfqpoint{1.708153in}{1.125233in}}%
\pgfpathlineto{\pgfqpoint{1.708153in}{1.125233in}}%
\pgfpathlineto{\pgfqpoint{1.708153in}{1.128182in}}%
\pgfpathlineto{\pgfqpoint{1.712694in}{1.128182in}}%
\pgfpathlineto{\pgfqpoint{1.712694in}{1.125233in}}%
\pgfpathmoveto{\pgfqpoint{1.708153in}{1.128182in}}%
\pgfpathlineto{\pgfqpoint{1.708153in}{1.128182in}}%
\pgfpathlineto{\pgfqpoint{1.708153in}{1.131131in}}%
\pgfpathlineto{\pgfqpoint{1.712694in}{1.131131in}}%
\pgfpathlineto{\pgfqpoint{1.712694in}{1.128182in}}%
\pgfpathmoveto{\pgfqpoint{1.712694in}{1.125233in}}%
\pgfpathlineto{\pgfqpoint{1.712694in}{1.125233in}}%
\pgfpathlineto{\pgfqpoint{1.712694in}{1.128182in}}%
\pgfpathlineto{\pgfqpoint{1.717235in}{1.128182in}}%
\pgfpathlineto{\pgfqpoint{1.717235in}{1.125233in}}%
\pgfpathmoveto{\pgfqpoint{1.712694in}{1.128182in}}%
\pgfpathlineto{\pgfqpoint{1.712694in}{1.128182in}}%
\pgfpathlineto{\pgfqpoint{1.712694in}{1.131131in}}%
\pgfpathlineto{\pgfqpoint{1.717235in}{1.131131in}}%
\pgfpathlineto{\pgfqpoint{1.717235in}{1.128182in}}%
\pgfpathmoveto{\pgfqpoint{1.717235in}{1.128182in}}%
\pgfpathlineto{\pgfqpoint{1.717235in}{1.128182in}}%
\pgfpathlineto{\pgfqpoint{1.717235in}{1.131131in}}%
\pgfpathlineto{\pgfqpoint{1.721776in}{1.131131in}}%
\pgfpathlineto{\pgfqpoint{1.721776in}{1.128182in}}%
\pgfpathmoveto{\pgfqpoint{1.712694in}{1.131131in}}%
\pgfpathlineto{\pgfqpoint{1.712694in}{1.131131in}}%
\pgfpathlineto{\pgfqpoint{1.712694in}{1.134080in}}%
\pgfpathlineto{\pgfqpoint{1.717235in}{1.134080in}}%
\pgfpathlineto{\pgfqpoint{1.717235in}{1.131131in}}%
\pgfpathmoveto{\pgfqpoint{1.712694in}{1.134080in}}%
\pgfpathlineto{\pgfqpoint{1.712694in}{1.134080in}}%
\pgfpathlineto{\pgfqpoint{1.712694in}{1.137030in}}%
\pgfpathlineto{\pgfqpoint{1.717235in}{1.137030in}}%
\pgfpathlineto{\pgfqpoint{1.717235in}{1.134080in}}%
\pgfpathmoveto{\pgfqpoint{1.717235in}{1.131131in}}%
\pgfpathlineto{\pgfqpoint{1.717235in}{1.131131in}}%
\pgfpathlineto{\pgfqpoint{1.717235in}{1.134080in}}%
\pgfpathlineto{\pgfqpoint{1.721776in}{1.134080in}}%
\pgfpathlineto{\pgfqpoint{1.721776in}{1.131131in}}%
\pgfpathmoveto{\pgfqpoint{1.717235in}{1.134080in}}%
\pgfpathlineto{\pgfqpoint{1.717235in}{1.134080in}}%
\pgfpathlineto{\pgfqpoint{1.717235in}{1.137030in}}%
\pgfpathlineto{\pgfqpoint{1.721776in}{1.137030in}}%
\pgfpathlineto{\pgfqpoint{1.721776in}{1.134080in}}%
\pgfpathmoveto{\pgfqpoint{1.721776in}{1.131131in}}%
\pgfpathlineto{\pgfqpoint{1.721776in}{1.131131in}}%
\pgfpathlineto{\pgfqpoint{1.721776in}{1.134080in}}%
\pgfpathlineto{\pgfqpoint{1.726317in}{1.134080in}}%
\pgfpathlineto{\pgfqpoint{1.726317in}{1.131131in}}%
\pgfpathmoveto{\pgfqpoint{1.721776in}{1.134080in}}%
\pgfpathlineto{\pgfqpoint{1.721776in}{1.134080in}}%
\pgfpathlineto{\pgfqpoint{1.721776in}{1.137030in}}%
\pgfpathlineto{\pgfqpoint{1.726317in}{1.137030in}}%
\pgfpathlineto{\pgfqpoint{1.726317in}{1.134080in}}%
\pgfpathmoveto{\pgfqpoint{1.726317in}{1.134080in}}%
\pgfpathlineto{\pgfqpoint{1.726317in}{1.134080in}}%
\pgfpathlineto{\pgfqpoint{1.726317in}{1.137030in}}%
\pgfpathlineto{\pgfqpoint{1.730858in}{1.137030in}}%
\pgfpathlineto{\pgfqpoint{1.730858in}{1.134080in}}%
\pgfpathmoveto{\pgfqpoint{1.721776in}{1.137030in}}%
\pgfpathlineto{\pgfqpoint{1.721776in}{1.137030in}}%
\pgfpathlineto{\pgfqpoint{1.721776in}{1.139979in}}%
\pgfpathlineto{\pgfqpoint{1.726317in}{1.139979in}}%
\pgfpathlineto{\pgfqpoint{1.726317in}{1.137030in}}%
\pgfpathmoveto{\pgfqpoint{1.721776in}{1.139979in}}%
\pgfpathlineto{\pgfqpoint{1.721776in}{1.139979in}}%
\pgfpathlineto{\pgfqpoint{1.721776in}{1.142928in}}%
\pgfpathlineto{\pgfqpoint{1.726317in}{1.142928in}}%
\pgfpathlineto{\pgfqpoint{1.726317in}{1.139979in}}%
\pgfpathmoveto{\pgfqpoint{1.726317in}{1.137030in}}%
\pgfpathlineto{\pgfqpoint{1.726317in}{1.137030in}}%
\pgfpathlineto{\pgfqpoint{1.726317in}{1.139979in}}%
\pgfpathlineto{\pgfqpoint{1.730858in}{1.139979in}}%
\pgfpathlineto{\pgfqpoint{1.730858in}{1.137030in}}%
\pgfpathmoveto{\pgfqpoint{1.726317in}{1.139979in}}%
\pgfpathlineto{\pgfqpoint{1.726317in}{1.139979in}}%
\pgfpathlineto{\pgfqpoint{1.726317in}{1.142928in}}%
\pgfpathlineto{\pgfqpoint{1.730858in}{1.142928in}}%
\pgfpathlineto{\pgfqpoint{1.730858in}{1.139979in}}%
\pgfpathmoveto{\pgfqpoint{1.730858in}{1.137030in}}%
\pgfpathlineto{\pgfqpoint{1.730858in}{1.137030in}}%
\pgfpathlineto{\pgfqpoint{1.730858in}{1.139979in}}%
\pgfpathlineto{\pgfqpoint{1.735399in}{1.139979in}}%
\pgfpathlineto{\pgfqpoint{1.735399in}{1.137030in}}%
\pgfpathmoveto{\pgfqpoint{1.730858in}{1.139979in}}%
\pgfpathlineto{\pgfqpoint{1.730858in}{1.139979in}}%
\pgfpathlineto{\pgfqpoint{1.730858in}{1.142928in}}%
\pgfpathlineto{\pgfqpoint{1.735399in}{1.142928in}}%
\pgfpathlineto{\pgfqpoint{1.735399in}{1.139979in}}%
\pgfpathmoveto{\pgfqpoint{1.735399in}{1.139979in}}%
\pgfpathlineto{\pgfqpoint{1.735399in}{1.139979in}}%
\pgfpathlineto{\pgfqpoint{1.735399in}{1.142928in}}%
\pgfpathlineto{\pgfqpoint{1.739940in}{1.142928in}}%
\pgfpathlineto{\pgfqpoint{1.739940in}{1.139979in}}%
\pgfpathmoveto{\pgfqpoint{1.730858in}{1.142928in}}%
\pgfpathlineto{\pgfqpoint{1.730858in}{1.142928in}}%
\pgfpathlineto{\pgfqpoint{1.730858in}{1.145877in}}%
\pgfpathlineto{\pgfqpoint{1.735399in}{1.145877in}}%
\pgfpathlineto{\pgfqpoint{1.735399in}{1.142928in}}%
\pgfpathmoveto{\pgfqpoint{1.730858in}{1.145877in}}%
\pgfpathlineto{\pgfqpoint{1.730858in}{1.145877in}}%
\pgfpathlineto{\pgfqpoint{1.730858in}{1.148826in}}%
\pgfpathlineto{\pgfqpoint{1.735399in}{1.148826in}}%
\pgfpathlineto{\pgfqpoint{1.735399in}{1.145877in}}%
\pgfpathmoveto{\pgfqpoint{1.735399in}{1.142928in}}%
\pgfpathlineto{\pgfqpoint{1.735399in}{1.142928in}}%
\pgfpathlineto{\pgfqpoint{1.735399in}{1.145877in}}%
\pgfpathlineto{\pgfqpoint{1.739940in}{1.145877in}}%
\pgfpathlineto{\pgfqpoint{1.739940in}{1.142928in}}%
\pgfpathmoveto{\pgfqpoint{1.735399in}{1.145877in}}%
\pgfpathlineto{\pgfqpoint{1.735399in}{1.145877in}}%
\pgfpathlineto{\pgfqpoint{1.735399in}{1.148826in}}%
\pgfpathlineto{\pgfqpoint{1.739940in}{1.148826in}}%
\pgfpathlineto{\pgfqpoint{1.739940in}{1.145877in}}%
\pgfpathmoveto{\pgfqpoint{1.739940in}{1.142928in}}%
\pgfpathlineto{\pgfqpoint{1.739940in}{1.142928in}}%
\pgfpathlineto{\pgfqpoint{1.739940in}{1.145877in}}%
\pgfpathlineto{\pgfqpoint{1.744481in}{1.145877in}}%
\pgfpathlineto{\pgfqpoint{1.744481in}{1.142928in}}%
\pgfpathmoveto{\pgfqpoint{1.739940in}{1.145877in}}%
\pgfpathlineto{\pgfqpoint{1.739940in}{1.145877in}}%
\pgfpathlineto{\pgfqpoint{1.739940in}{1.148826in}}%
\pgfpathlineto{\pgfqpoint{1.744481in}{1.148826in}}%
\pgfpathlineto{\pgfqpoint{1.744481in}{1.145877in}}%
\pgfpathmoveto{\pgfqpoint{1.744481in}{1.145877in}}%
\pgfpathlineto{\pgfqpoint{1.744481in}{1.145877in}}%
\pgfpathlineto{\pgfqpoint{1.744481in}{1.148826in}}%
\pgfpathlineto{\pgfqpoint{1.749022in}{1.148826in}}%
\pgfpathlineto{\pgfqpoint{1.749022in}{1.145877in}}%
\pgfpathmoveto{\pgfqpoint{1.739940in}{1.148826in}}%
\pgfpathlineto{\pgfqpoint{1.739940in}{1.148826in}}%
\pgfpathlineto{\pgfqpoint{1.739940in}{1.151775in}}%
\pgfpathlineto{\pgfqpoint{1.744481in}{1.151775in}}%
\pgfpathlineto{\pgfqpoint{1.744481in}{1.148826in}}%
\pgfpathmoveto{\pgfqpoint{1.739940in}{1.151775in}}%
\pgfpathlineto{\pgfqpoint{1.739940in}{1.151775in}}%
\pgfpathlineto{\pgfqpoint{1.739940in}{1.154725in}}%
\pgfpathlineto{\pgfqpoint{1.744481in}{1.154725in}}%
\pgfpathlineto{\pgfqpoint{1.744481in}{1.151775in}}%
\pgfpathmoveto{\pgfqpoint{1.744481in}{1.148826in}}%
\pgfpathlineto{\pgfqpoint{1.744481in}{1.148826in}}%
\pgfpathlineto{\pgfqpoint{1.744481in}{1.151775in}}%
\pgfpathlineto{\pgfqpoint{1.749022in}{1.151775in}}%
\pgfpathlineto{\pgfqpoint{1.749022in}{1.148826in}}%
\pgfpathmoveto{\pgfqpoint{1.744481in}{1.151775in}}%
\pgfpathlineto{\pgfqpoint{1.744481in}{1.151775in}}%
\pgfpathlineto{\pgfqpoint{1.744481in}{1.154725in}}%
\pgfpathlineto{\pgfqpoint{1.749022in}{1.154725in}}%
\pgfpathlineto{\pgfqpoint{1.749022in}{1.151775in}}%
\pgfpathmoveto{\pgfqpoint{1.749022in}{1.148826in}}%
\pgfpathlineto{\pgfqpoint{1.749022in}{1.148826in}}%
\pgfpathlineto{\pgfqpoint{1.749022in}{1.151775in}}%
\pgfpathlineto{\pgfqpoint{1.753563in}{1.151775in}}%
\pgfpathlineto{\pgfqpoint{1.753563in}{1.148826in}}%
\pgfpathmoveto{\pgfqpoint{1.749022in}{1.151775in}}%
\pgfpathlineto{\pgfqpoint{1.749022in}{1.151775in}}%
\pgfpathlineto{\pgfqpoint{1.749022in}{1.154725in}}%
\pgfpathlineto{\pgfqpoint{1.753563in}{1.154725in}}%
\pgfpathlineto{\pgfqpoint{1.753563in}{1.151775in}}%
\pgfpathmoveto{\pgfqpoint{1.753563in}{1.151775in}}%
\pgfpathlineto{\pgfqpoint{1.753563in}{1.151775in}}%
\pgfpathlineto{\pgfqpoint{1.753563in}{1.154725in}}%
\pgfpathlineto{\pgfqpoint{1.758104in}{1.154725in}}%
\pgfpathlineto{\pgfqpoint{1.758104in}{1.151775in}}%
\pgfpathmoveto{\pgfqpoint{1.749022in}{1.154725in}}%
\pgfpathlineto{\pgfqpoint{1.749022in}{1.154725in}}%
\pgfpathlineto{\pgfqpoint{1.749022in}{1.157674in}}%
\pgfpathlineto{\pgfqpoint{1.753563in}{1.157674in}}%
\pgfpathlineto{\pgfqpoint{1.753563in}{1.154725in}}%
\pgfpathmoveto{\pgfqpoint{1.749022in}{1.157674in}}%
\pgfpathlineto{\pgfqpoint{1.749022in}{1.157674in}}%
\pgfpathlineto{\pgfqpoint{1.749022in}{1.160623in}}%
\pgfpathlineto{\pgfqpoint{1.753563in}{1.160623in}}%
\pgfpathlineto{\pgfqpoint{1.753563in}{1.157674in}}%
\pgfpathmoveto{\pgfqpoint{1.753563in}{1.154725in}}%
\pgfpathlineto{\pgfqpoint{1.753563in}{1.154725in}}%
\pgfpathlineto{\pgfqpoint{1.753563in}{1.157674in}}%
\pgfpathlineto{\pgfqpoint{1.758104in}{1.157674in}}%
\pgfpathlineto{\pgfqpoint{1.758104in}{1.154725in}}%
\pgfpathmoveto{\pgfqpoint{1.753563in}{1.157674in}}%
\pgfpathlineto{\pgfqpoint{1.753563in}{1.157674in}}%
\pgfpathlineto{\pgfqpoint{1.753563in}{1.160623in}}%
\pgfpathlineto{\pgfqpoint{1.758104in}{1.160623in}}%
\pgfpathlineto{\pgfqpoint{1.758104in}{1.157674in}}%
\pgfpathmoveto{\pgfqpoint{1.758104in}{1.154725in}}%
\pgfpathlineto{\pgfqpoint{1.758104in}{1.154725in}}%
\pgfpathlineto{\pgfqpoint{1.758104in}{1.157674in}}%
\pgfpathlineto{\pgfqpoint{1.762645in}{1.157674in}}%
\pgfpathlineto{\pgfqpoint{1.762645in}{1.154725in}}%
\pgfpathmoveto{\pgfqpoint{1.758104in}{1.157674in}}%
\pgfpathlineto{\pgfqpoint{1.758104in}{1.157674in}}%
\pgfpathlineto{\pgfqpoint{1.758104in}{1.160623in}}%
\pgfpathlineto{\pgfqpoint{1.762645in}{1.160623in}}%
\pgfpathlineto{\pgfqpoint{1.762645in}{1.157674in}}%
\pgfpathmoveto{\pgfqpoint{1.762645in}{1.157674in}}%
\pgfpathlineto{\pgfqpoint{1.762645in}{1.157674in}}%
\pgfpathlineto{\pgfqpoint{1.762645in}{1.160623in}}%
\pgfpathlineto{\pgfqpoint{1.767187in}{1.160623in}}%
\pgfpathlineto{\pgfqpoint{1.767187in}{1.157674in}}%
\pgfpathmoveto{\pgfqpoint{1.758104in}{1.160623in}}%
\pgfpathlineto{\pgfqpoint{1.758104in}{1.160623in}}%
\pgfpathlineto{\pgfqpoint{1.758104in}{1.163572in}}%
\pgfpathlineto{\pgfqpoint{1.762645in}{1.163572in}}%
\pgfpathlineto{\pgfqpoint{1.762645in}{1.160623in}}%
\pgfpathmoveto{\pgfqpoint{1.758104in}{1.163572in}}%
\pgfpathlineto{\pgfqpoint{1.758104in}{1.163572in}}%
\pgfpathlineto{\pgfqpoint{1.758104in}{1.166522in}}%
\pgfpathlineto{\pgfqpoint{1.762645in}{1.166522in}}%
\pgfpathlineto{\pgfqpoint{1.762645in}{1.163572in}}%
\pgfpathmoveto{\pgfqpoint{1.762645in}{1.160623in}}%
\pgfpathlineto{\pgfqpoint{1.762645in}{1.160623in}}%
\pgfpathlineto{\pgfqpoint{1.762645in}{1.163572in}}%
\pgfpathlineto{\pgfqpoint{1.767187in}{1.163572in}}%
\pgfpathlineto{\pgfqpoint{1.767187in}{1.160623in}}%
\pgfpathmoveto{\pgfqpoint{1.762645in}{1.163572in}}%
\pgfpathlineto{\pgfqpoint{1.762645in}{1.163572in}}%
\pgfpathlineto{\pgfqpoint{1.762645in}{1.166522in}}%
\pgfpathlineto{\pgfqpoint{1.767187in}{1.166522in}}%
\pgfpathlineto{\pgfqpoint{1.767187in}{1.163572in}}%
\pgfpathmoveto{\pgfqpoint{1.767187in}{1.160623in}}%
\pgfpathlineto{\pgfqpoint{1.767187in}{1.160623in}}%
\pgfpathlineto{\pgfqpoint{1.767187in}{1.163572in}}%
\pgfpathlineto{\pgfqpoint{1.771728in}{1.163572in}}%
\pgfpathlineto{\pgfqpoint{1.771728in}{1.160623in}}%
\pgfpathmoveto{\pgfqpoint{1.767187in}{1.163572in}}%
\pgfpathlineto{\pgfqpoint{1.767187in}{1.163572in}}%
\pgfpathlineto{\pgfqpoint{1.767187in}{1.166522in}}%
\pgfpathlineto{\pgfqpoint{1.771728in}{1.166522in}}%
\pgfpathlineto{\pgfqpoint{1.771728in}{1.163572in}}%
\pgfpathmoveto{\pgfqpoint{1.771728in}{1.163572in}}%
\pgfpathlineto{\pgfqpoint{1.771728in}{1.163572in}}%
\pgfpathlineto{\pgfqpoint{1.771728in}{1.166522in}}%
\pgfpathlineto{\pgfqpoint{1.776269in}{1.166522in}}%
\pgfpathlineto{\pgfqpoint{1.776269in}{1.163572in}}%
\pgfpathmoveto{\pgfqpoint{1.767187in}{1.166522in}}%
\pgfpathlineto{\pgfqpoint{1.767187in}{1.166522in}}%
\pgfpathlineto{\pgfqpoint{1.767187in}{1.169471in}}%
\pgfpathlineto{\pgfqpoint{1.771728in}{1.169471in}}%
\pgfpathlineto{\pgfqpoint{1.771728in}{1.166522in}}%
\pgfpathmoveto{\pgfqpoint{1.767187in}{1.169471in}}%
\pgfpathlineto{\pgfqpoint{1.767187in}{1.169471in}}%
\pgfpathlineto{\pgfqpoint{1.767187in}{1.172420in}}%
\pgfpathlineto{\pgfqpoint{1.771728in}{1.172420in}}%
\pgfpathlineto{\pgfqpoint{1.771728in}{1.169471in}}%
\pgfpathmoveto{\pgfqpoint{1.771728in}{1.166522in}}%
\pgfpathlineto{\pgfqpoint{1.771728in}{1.166522in}}%
\pgfpathlineto{\pgfqpoint{1.771728in}{1.169471in}}%
\pgfpathlineto{\pgfqpoint{1.776269in}{1.169471in}}%
\pgfpathlineto{\pgfqpoint{1.776269in}{1.166522in}}%
\pgfpathmoveto{\pgfqpoint{1.771728in}{1.169471in}}%
\pgfpathlineto{\pgfqpoint{1.771728in}{1.169471in}}%
\pgfpathlineto{\pgfqpoint{1.771728in}{1.172420in}}%
\pgfpathlineto{\pgfqpoint{1.776269in}{1.172420in}}%
\pgfpathlineto{\pgfqpoint{1.776269in}{1.169471in}}%
\pgfpathmoveto{\pgfqpoint{1.776269in}{1.166522in}}%
\pgfpathlineto{\pgfqpoint{1.776269in}{1.166522in}}%
\pgfpathlineto{\pgfqpoint{1.776269in}{1.169471in}}%
\pgfpathlineto{\pgfqpoint{1.780810in}{1.169471in}}%
\pgfpathlineto{\pgfqpoint{1.780810in}{1.166522in}}%
\pgfpathmoveto{\pgfqpoint{1.776269in}{1.169471in}}%
\pgfpathlineto{\pgfqpoint{1.776269in}{1.169471in}}%
\pgfpathlineto{\pgfqpoint{1.776269in}{1.172420in}}%
\pgfpathlineto{\pgfqpoint{1.780810in}{1.172420in}}%
\pgfpathlineto{\pgfqpoint{1.780810in}{1.169471in}}%
\pgfpathmoveto{\pgfqpoint{1.780810in}{1.169471in}}%
\pgfpathlineto{\pgfqpoint{1.780810in}{1.169471in}}%
\pgfpathlineto{\pgfqpoint{1.780810in}{1.172420in}}%
\pgfpathlineto{\pgfqpoint{1.785351in}{1.172420in}}%
\pgfpathlineto{\pgfqpoint{1.785351in}{1.169471in}}%
\pgfpathmoveto{\pgfqpoint{1.776269in}{1.172420in}}%
\pgfpathlineto{\pgfqpoint{1.776269in}{1.172420in}}%
\pgfpathlineto{\pgfqpoint{1.776269in}{1.175369in}}%
\pgfpathlineto{\pgfqpoint{1.780810in}{1.175369in}}%
\pgfpathlineto{\pgfqpoint{1.780810in}{1.172420in}}%
\pgfpathmoveto{\pgfqpoint{1.776269in}{1.175369in}}%
\pgfpathlineto{\pgfqpoint{1.776269in}{1.175369in}}%
\pgfpathlineto{\pgfqpoint{1.776269in}{1.178319in}}%
\pgfpathlineto{\pgfqpoint{1.780810in}{1.178319in}}%
\pgfpathlineto{\pgfqpoint{1.780810in}{1.175369in}}%
\pgfpathmoveto{\pgfqpoint{1.780810in}{1.172420in}}%
\pgfpathlineto{\pgfqpoint{1.780810in}{1.172420in}}%
\pgfpathlineto{\pgfqpoint{1.780810in}{1.175369in}}%
\pgfpathlineto{\pgfqpoint{1.785351in}{1.175369in}}%
\pgfpathlineto{\pgfqpoint{1.785351in}{1.172420in}}%
\pgfpathmoveto{\pgfqpoint{1.780810in}{1.175369in}}%
\pgfpathlineto{\pgfqpoint{1.780810in}{1.175369in}}%
\pgfpathlineto{\pgfqpoint{1.780810in}{1.178319in}}%
\pgfpathlineto{\pgfqpoint{1.785351in}{1.178319in}}%
\pgfpathlineto{\pgfqpoint{1.785351in}{1.175369in}}%
\pgfpathmoveto{\pgfqpoint{1.785351in}{1.172420in}}%
\pgfpathlineto{\pgfqpoint{1.785351in}{1.172420in}}%
\pgfpathlineto{\pgfqpoint{1.785351in}{1.175369in}}%
\pgfpathlineto{\pgfqpoint{1.789892in}{1.175369in}}%
\pgfpathlineto{\pgfqpoint{1.789892in}{1.172420in}}%
\pgfpathmoveto{\pgfqpoint{1.785351in}{1.175369in}}%
\pgfpathlineto{\pgfqpoint{1.785351in}{1.175369in}}%
\pgfpathlineto{\pgfqpoint{1.785351in}{1.178319in}}%
\pgfpathlineto{\pgfqpoint{1.789892in}{1.178319in}}%
\pgfpathlineto{\pgfqpoint{1.789892in}{1.175369in}}%
\pgfpathmoveto{\pgfqpoint{1.789892in}{1.175369in}}%
\pgfpathlineto{\pgfqpoint{1.789892in}{1.175369in}}%
\pgfpathlineto{\pgfqpoint{1.789892in}{1.178319in}}%
\pgfpathlineto{\pgfqpoint{1.794433in}{1.178319in}}%
\pgfpathlineto{\pgfqpoint{1.794433in}{1.175369in}}%
\pgfpathmoveto{\pgfqpoint{1.785351in}{1.178319in}}%
\pgfpathlineto{\pgfqpoint{1.785351in}{1.178319in}}%
\pgfpathlineto{\pgfqpoint{1.785351in}{1.181268in}}%
\pgfpathlineto{\pgfqpoint{1.789892in}{1.181268in}}%
\pgfpathlineto{\pgfqpoint{1.789892in}{1.178319in}}%
\pgfpathmoveto{\pgfqpoint{1.785351in}{1.181268in}}%
\pgfpathlineto{\pgfqpoint{1.785351in}{1.181268in}}%
\pgfpathlineto{\pgfqpoint{1.785351in}{1.184217in}}%
\pgfpathlineto{\pgfqpoint{1.789892in}{1.184217in}}%
\pgfpathlineto{\pgfqpoint{1.789892in}{1.181268in}}%
\pgfpathmoveto{\pgfqpoint{1.789892in}{1.178319in}}%
\pgfpathlineto{\pgfqpoint{1.789892in}{1.178319in}}%
\pgfpathlineto{\pgfqpoint{1.789892in}{1.181268in}}%
\pgfpathlineto{\pgfqpoint{1.794433in}{1.181268in}}%
\pgfpathlineto{\pgfqpoint{1.794433in}{1.178319in}}%
\pgfpathmoveto{\pgfqpoint{1.789892in}{1.181268in}}%
\pgfpathlineto{\pgfqpoint{1.789892in}{1.181268in}}%
\pgfpathlineto{\pgfqpoint{1.789892in}{1.184217in}}%
\pgfpathlineto{\pgfqpoint{1.794433in}{1.184217in}}%
\pgfpathlineto{\pgfqpoint{1.794433in}{1.181268in}}%
\pgfpathmoveto{\pgfqpoint{1.794433in}{1.178319in}}%
\pgfpathlineto{\pgfqpoint{1.794433in}{1.178319in}}%
\pgfpathlineto{\pgfqpoint{1.794433in}{1.181268in}}%
\pgfpathlineto{\pgfqpoint{1.798975in}{1.181268in}}%
\pgfpathlineto{\pgfqpoint{1.798975in}{1.178319in}}%
\pgfpathmoveto{\pgfqpoint{1.794433in}{1.181268in}}%
\pgfpathlineto{\pgfqpoint{1.794433in}{1.181268in}}%
\pgfpathlineto{\pgfqpoint{1.794433in}{1.184217in}}%
\pgfpathlineto{\pgfqpoint{1.798975in}{1.184217in}}%
\pgfpathlineto{\pgfqpoint{1.798975in}{1.181268in}}%
\pgfpathmoveto{\pgfqpoint{1.798975in}{1.181268in}}%
\pgfpathlineto{\pgfqpoint{1.798975in}{1.181268in}}%
\pgfpathlineto{\pgfqpoint{1.798975in}{1.184217in}}%
\pgfpathlineto{\pgfqpoint{1.803516in}{1.184217in}}%
\pgfpathlineto{\pgfqpoint{1.803516in}{1.181268in}}%
\pgfpathmoveto{\pgfqpoint{1.794433in}{1.184217in}}%
\pgfpathlineto{\pgfqpoint{1.794433in}{1.184217in}}%
\pgfpathlineto{\pgfqpoint{1.794433in}{1.187166in}}%
\pgfpathlineto{\pgfqpoint{1.798975in}{1.187166in}}%
\pgfpathlineto{\pgfqpoint{1.798975in}{1.184217in}}%
\pgfpathmoveto{\pgfqpoint{1.794433in}{1.187166in}}%
\pgfpathlineto{\pgfqpoint{1.794433in}{1.187166in}}%
\pgfpathlineto{\pgfqpoint{1.794433in}{1.190116in}}%
\pgfpathlineto{\pgfqpoint{1.798975in}{1.190116in}}%
\pgfpathlineto{\pgfqpoint{1.798975in}{1.187166in}}%
\pgfpathmoveto{\pgfqpoint{1.798975in}{1.184217in}}%
\pgfpathlineto{\pgfqpoint{1.798975in}{1.184217in}}%
\pgfpathlineto{\pgfqpoint{1.798975in}{1.187166in}}%
\pgfpathlineto{\pgfqpoint{1.803516in}{1.187166in}}%
\pgfpathlineto{\pgfqpoint{1.803516in}{1.184217in}}%
\pgfpathmoveto{\pgfqpoint{1.798975in}{1.187166in}}%
\pgfpathlineto{\pgfqpoint{1.798975in}{1.187166in}}%
\pgfpathlineto{\pgfqpoint{1.798975in}{1.190116in}}%
\pgfpathlineto{\pgfqpoint{1.803516in}{1.190116in}}%
\pgfpathlineto{\pgfqpoint{1.803516in}{1.187166in}}%
\pgfpathmoveto{\pgfqpoint{1.803516in}{1.184217in}}%
\pgfpathlineto{\pgfqpoint{1.803516in}{1.184217in}}%
\pgfpathlineto{\pgfqpoint{1.803516in}{1.187166in}}%
\pgfpathlineto{\pgfqpoint{1.808057in}{1.187166in}}%
\pgfpathlineto{\pgfqpoint{1.808057in}{1.184217in}}%
\pgfpathmoveto{\pgfqpoint{1.803516in}{1.187166in}}%
\pgfpathlineto{\pgfqpoint{1.803516in}{1.187166in}}%
\pgfpathlineto{\pgfqpoint{1.803516in}{1.190116in}}%
\pgfpathlineto{\pgfqpoint{1.808057in}{1.190116in}}%
\pgfpathlineto{\pgfqpoint{1.808057in}{1.187166in}}%
\pgfpathmoveto{\pgfqpoint{1.808057in}{1.187166in}}%
\pgfpathlineto{\pgfqpoint{1.808057in}{1.187166in}}%
\pgfpathlineto{\pgfqpoint{1.808057in}{1.190116in}}%
\pgfpathlineto{\pgfqpoint{1.812598in}{1.190116in}}%
\pgfpathlineto{\pgfqpoint{1.812598in}{1.187166in}}%
\pgfpathmoveto{\pgfqpoint{1.803516in}{1.190116in}}%
\pgfpathlineto{\pgfqpoint{1.803516in}{1.190116in}}%
\pgfpathlineto{\pgfqpoint{1.803516in}{1.193065in}}%
\pgfpathlineto{\pgfqpoint{1.808057in}{1.193065in}}%
\pgfpathlineto{\pgfqpoint{1.808057in}{1.190116in}}%
\pgfpathmoveto{\pgfqpoint{1.803516in}{1.193065in}}%
\pgfpathlineto{\pgfqpoint{1.803516in}{1.193065in}}%
\pgfpathlineto{\pgfqpoint{1.803516in}{1.196014in}}%
\pgfpathlineto{\pgfqpoint{1.808057in}{1.196014in}}%
\pgfpathlineto{\pgfqpoint{1.808057in}{1.193065in}}%
\pgfpathmoveto{\pgfqpoint{1.808057in}{1.190116in}}%
\pgfpathlineto{\pgfqpoint{1.808057in}{1.190116in}}%
\pgfpathlineto{\pgfqpoint{1.808057in}{1.193065in}}%
\pgfpathlineto{\pgfqpoint{1.812598in}{1.193065in}}%
\pgfpathlineto{\pgfqpoint{1.812598in}{1.190116in}}%
\pgfpathmoveto{\pgfqpoint{1.808057in}{1.193065in}}%
\pgfpathlineto{\pgfqpoint{1.808057in}{1.193065in}}%
\pgfpathlineto{\pgfqpoint{1.808057in}{1.196014in}}%
\pgfpathlineto{\pgfqpoint{1.812598in}{1.196014in}}%
\pgfpathlineto{\pgfqpoint{1.812598in}{1.193065in}}%
\pgfpathmoveto{\pgfqpoint{1.812598in}{1.190116in}}%
\pgfpathlineto{\pgfqpoint{1.812598in}{1.190116in}}%
\pgfpathlineto{\pgfqpoint{1.812598in}{1.193065in}}%
\pgfpathlineto{\pgfqpoint{1.817139in}{1.193065in}}%
\pgfpathlineto{\pgfqpoint{1.817139in}{1.190116in}}%
\pgfpathmoveto{\pgfqpoint{1.812598in}{1.193065in}}%
\pgfpathlineto{\pgfqpoint{1.812598in}{1.193065in}}%
\pgfpathlineto{\pgfqpoint{1.812598in}{1.196014in}}%
\pgfpathlineto{\pgfqpoint{1.817139in}{1.196014in}}%
\pgfpathlineto{\pgfqpoint{1.817139in}{1.193065in}}%
\pgfpathmoveto{\pgfqpoint{1.817139in}{1.193065in}}%
\pgfpathlineto{\pgfqpoint{1.817139in}{1.193065in}}%
\pgfpathlineto{\pgfqpoint{1.817139in}{1.196014in}}%
\pgfpathlineto{\pgfqpoint{1.821680in}{1.196014in}}%
\pgfpathlineto{\pgfqpoint{1.821680in}{1.193065in}}%
\pgfpathmoveto{\pgfqpoint{1.812598in}{1.196014in}}%
\pgfpathlineto{\pgfqpoint{1.812598in}{1.196014in}}%
\pgfpathlineto{\pgfqpoint{1.812598in}{1.198963in}}%
\pgfpathlineto{\pgfqpoint{1.817139in}{1.198963in}}%
\pgfpathlineto{\pgfqpoint{1.817139in}{1.196014in}}%
\pgfpathmoveto{\pgfqpoint{1.812598in}{1.198963in}}%
\pgfpathlineto{\pgfqpoint{1.812598in}{1.198963in}}%
\pgfpathlineto{\pgfqpoint{1.812598in}{1.201913in}}%
\pgfpathlineto{\pgfqpoint{1.817139in}{1.201913in}}%
\pgfpathlineto{\pgfqpoint{1.817139in}{1.198963in}}%
\pgfpathmoveto{\pgfqpoint{1.817139in}{1.196014in}}%
\pgfpathlineto{\pgfqpoint{1.817139in}{1.196014in}}%
\pgfpathlineto{\pgfqpoint{1.817139in}{1.198963in}}%
\pgfpathlineto{\pgfqpoint{1.821680in}{1.198963in}}%
\pgfpathlineto{\pgfqpoint{1.821680in}{1.196014in}}%
\pgfpathmoveto{\pgfqpoint{1.817139in}{1.198963in}}%
\pgfpathlineto{\pgfqpoint{1.817139in}{1.198963in}}%
\pgfpathlineto{\pgfqpoint{1.817139in}{1.201913in}}%
\pgfpathlineto{\pgfqpoint{1.821680in}{1.201913in}}%
\pgfpathlineto{\pgfqpoint{1.821680in}{1.198963in}}%
\pgfpathmoveto{\pgfqpoint{1.821680in}{1.196014in}}%
\pgfpathlineto{\pgfqpoint{1.821680in}{1.196014in}}%
\pgfpathlineto{\pgfqpoint{1.821680in}{1.198963in}}%
\pgfpathlineto{\pgfqpoint{1.826221in}{1.198963in}}%
\pgfpathlineto{\pgfqpoint{1.826221in}{1.196014in}}%
\pgfpathmoveto{\pgfqpoint{1.821680in}{1.198963in}}%
\pgfpathlineto{\pgfqpoint{1.821680in}{1.198963in}}%
\pgfpathlineto{\pgfqpoint{1.821680in}{1.201913in}}%
\pgfpathlineto{\pgfqpoint{1.826221in}{1.201913in}}%
\pgfpathlineto{\pgfqpoint{1.826221in}{1.198963in}}%
\pgfpathmoveto{\pgfqpoint{1.826221in}{1.198963in}}%
\pgfpathlineto{\pgfqpoint{1.826221in}{1.198963in}}%
\pgfpathlineto{\pgfqpoint{1.826221in}{1.201913in}}%
\pgfpathlineto{\pgfqpoint{1.830763in}{1.201913in}}%
\pgfpathlineto{\pgfqpoint{1.830763in}{1.198963in}}%
\pgfpathmoveto{\pgfqpoint{1.821680in}{1.201913in}}%
\pgfpathlineto{\pgfqpoint{1.821680in}{1.201913in}}%
\pgfpathlineto{\pgfqpoint{1.821680in}{1.204862in}}%
\pgfpathlineto{\pgfqpoint{1.826221in}{1.204862in}}%
\pgfpathlineto{\pgfqpoint{1.826221in}{1.201913in}}%
\pgfpathmoveto{\pgfqpoint{1.821680in}{1.204862in}}%
\pgfpathlineto{\pgfqpoint{1.821680in}{1.204862in}}%
\pgfpathlineto{\pgfqpoint{1.821680in}{1.207811in}}%
\pgfpathlineto{\pgfqpoint{1.826221in}{1.207811in}}%
\pgfpathlineto{\pgfqpoint{1.826221in}{1.204862in}}%
\pgfpathmoveto{\pgfqpoint{1.826221in}{1.201913in}}%
\pgfpathlineto{\pgfqpoint{1.826221in}{1.201913in}}%
\pgfpathlineto{\pgfqpoint{1.826221in}{1.204862in}}%
\pgfpathlineto{\pgfqpoint{1.830763in}{1.204862in}}%
\pgfpathlineto{\pgfqpoint{1.830763in}{1.201913in}}%
\pgfpathmoveto{\pgfqpoint{1.826221in}{1.204862in}}%
\pgfpathlineto{\pgfqpoint{1.826221in}{1.204862in}}%
\pgfpathlineto{\pgfqpoint{1.826221in}{1.207811in}}%
\pgfpathlineto{\pgfqpoint{1.830763in}{1.207811in}}%
\pgfpathlineto{\pgfqpoint{1.830763in}{1.204862in}}%
\pgfpathmoveto{\pgfqpoint{1.830763in}{1.201913in}}%
\pgfpathlineto{\pgfqpoint{1.830763in}{1.201913in}}%
\pgfpathlineto{\pgfqpoint{1.830763in}{1.204862in}}%
\pgfpathlineto{\pgfqpoint{1.835304in}{1.204862in}}%
\pgfpathlineto{\pgfqpoint{1.835304in}{1.201913in}}%
\pgfpathmoveto{\pgfqpoint{1.830763in}{1.204862in}}%
\pgfpathlineto{\pgfqpoint{1.830763in}{1.204862in}}%
\pgfpathlineto{\pgfqpoint{1.830763in}{1.207811in}}%
\pgfpathlineto{\pgfqpoint{1.835304in}{1.207811in}}%
\pgfpathlineto{\pgfqpoint{1.835304in}{1.204862in}}%
\pgfpathmoveto{\pgfqpoint{1.835304in}{1.204862in}}%
\pgfpathlineto{\pgfqpoint{1.835304in}{1.204862in}}%
\pgfpathlineto{\pgfqpoint{1.835304in}{1.207811in}}%
\pgfpathlineto{\pgfqpoint{1.839845in}{1.207811in}}%
\pgfpathlineto{\pgfqpoint{1.839845in}{1.204862in}}%
\pgfpathmoveto{\pgfqpoint{1.830763in}{1.207811in}}%
\pgfpathlineto{\pgfqpoint{1.830763in}{1.207811in}}%
\pgfpathlineto{\pgfqpoint{1.830763in}{1.210760in}}%
\pgfpathlineto{\pgfqpoint{1.835304in}{1.210760in}}%
\pgfpathlineto{\pgfqpoint{1.835304in}{1.207811in}}%
\pgfpathmoveto{\pgfqpoint{1.830763in}{1.210760in}}%
\pgfpathlineto{\pgfqpoint{1.830763in}{1.210760in}}%
\pgfpathlineto{\pgfqpoint{1.830763in}{1.213710in}}%
\pgfpathlineto{\pgfqpoint{1.835304in}{1.213710in}}%
\pgfpathlineto{\pgfqpoint{1.835304in}{1.210760in}}%
\pgfpathmoveto{\pgfqpoint{1.835304in}{1.207811in}}%
\pgfpathlineto{\pgfqpoint{1.835304in}{1.207811in}}%
\pgfpathlineto{\pgfqpoint{1.835304in}{1.210760in}}%
\pgfpathlineto{\pgfqpoint{1.839845in}{1.210760in}}%
\pgfpathlineto{\pgfqpoint{1.839845in}{1.207811in}}%
\pgfpathmoveto{\pgfqpoint{1.835304in}{1.210760in}}%
\pgfpathlineto{\pgfqpoint{1.835304in}{1.210760in}}%
\pgfpathlineto{\pgfqpoint{1.835304in}{1.213710in}}%
\pgfpathlineto{\pgfqpoint{1.839845in}{1.213710in}}%
\pgfpathlineto{\pgfqpoint{1.839845in}{1.210760in}}%
\pgfpathmoveto{\pgfqpoint{1.839845in}{1.207811in}}%
\pgfpathlineto{\pgfqpoint{1.839845in}{1.207811in}}%
\pgfpathlineto{\pgfqpoint{1.839845in}{1.210760in}}%
\pgfpathlineto{\pgfqpoint{1.844386in}{1.210760in}}%
\pgfpathlineto{\pgfqpoint{1.844386in}{1.207811in}}%
\pgfpathmoveto{\pgfqpoint{1.839845in}{1.210760in}}%
\pgfpathlineto{\pgfqpoint{1.839845in}{1.210760in}}%
\pgfpathlineto{\pgfqpoint{1.839845in}{1.213710in}}%
\pgfpathlineto{\pgfqpoint{1.844386in}{1.213710in}}%
\pgfpathlineto{\pgfqpoint{1.844386in}{1.210760in}}%
\pgfpathmoveto{\pgfqpoint{1.844386in}{1.210760in}}%
\pgfpathlineto{\pgfqpoint{1.844386in}{1.210760in}}%
\pgfpathlineto{\pgfqpoint{1.844386in}{1.213710in}}%
\pgfpathlineto{\pgfqpoint{1.848927in}{1.213710in}}%
\pgfpathlineto{\pgfqpoint{1.848927in}{1.210760in}}%
\pgfpathmoveto{\pgfqpoint{1.839845in}{1.213710in}}%
\pgfpathlineto{\pgfqpoint{1.839845in}{1.213710in}}%
\pgfpathlineto{\pgfqpoint{1.839845in}{1.216659in}}%
\pgfpathlineto{\pgfqpoint{1.844386in}{1.216659in}}%
\pgfpathlineto{\pgfqpoint{1.844386in}{1.213710in}}%
\pgfpathmoveto{\pgfqpoint{1.839845in}{1.216659in}}%
\pgfpathlineto{\pgfqpoint{1.839845in}{1.216659in}}%
\pgfpathlineto{\pgfqpoint{1.839845in}{1.219608in}}%
\pgfpathlineto{\pgfqpoint{1.844386in}{1.219608in}}%
\pgfpathlineto{\pgfqpoint{1.844386in}{1.216659in}}%
\pgfpathmoveto{\pgfqpoint{1.844386in}{1.213710in}}%
\pgfpathlineto{\pgfqpoint{1.844386in}{1.213710in}}%
\pgfpathlineto{\pgfqpoint{1.844386in}{1.216659in}}%
\pgfpathlineto{\pgfqpoint{1.848927in}{1.216659in}}%
\pgfpathlineto{\pgfqpoint{1.848927in}{1.213710in}}%
\pgfpathmoveto{\pgfqpoint{1.844386in}{1.216659in}}%
\pgfpathlineto{\pgfqpoint{1.844386in}{1.216659in}}%
\pgfpathlineto{\pgfqpoint{1.844386in}{1.219608in}}%
\pgfpathlineto{\pgfqpoint{1.848927in}{1.219608in}}%
\pgfpathlineto{\pgfqpoint{1.848927in}{1.216659in}}%
\pgfpathmoveto{\pgfqpoint{1.848927in}{1.213710in}}%
\pgfpathlineto{\pgfqpoint{1.848927in}{1.213710in}}%
\pgfpathlineto{\pgfqpoint{1.848927in}{1.216659in}}%
\pgfpathlineto{\pgfqpoint{1.853468in}{1.216659in}}%
\pgfpathlineto{\pgfqpoint{1.853468in}{1.213710in}}%
\pgfpathmoveto{\pgfqpoint{1.848927in}{1.216659in}}%
\pgfpathlineto{\pgfqpoint{1.848927in}{1.216659in}}%
\pgfpathlineto{\pgfqpoint{1.848927in}{1.219608in}}%
\pgfpathlineto{\pgfqpoint{1.853468in}{1.219608in}}%
\pgfpathlineto{\pgfqpoint{1.853468in}{1.216659in}}%
\pgfpathmoveto{\pgfqpoint{1.853468in}{1.216659in}}%
\pgfpathlineto{\pgfqpoint{1.853468in}{1.216659in}}%
\pgfpathlineto{\pgfqpoint{1.853468in}{1.219608in}}%
\pgfpathlineto{\pgfqpoint{1.858009in}{1.219608in}}%
\pgfpathlineto{\pgfqpoint{1.858009in}{1.216659in}}%
\pgfpathmoveto{\pgfqpoint{1.848927in}{1.219608in}}%
\pgfpathlineto{\pgfqpoint{1.848927in}{1.219608in}}%
\pgfpathlineto{\pgfqpoint{1.848927in}{1.222558in}}%
\pgfpathlineto{\pgfqpoint{1.853468in}{1.222558in}}%
\pgfpathlineto{\pgfqpoint{1.853468in}{1.219608in}}%
\pgfpathmoveto{\pgfqpoint{1.848927in}{1.222558in}}%
\pgfpathlineto{\pgfqpoint{1.848927in}{1.222558in}}%
\pgfpathlineto{\pgfqpoint{1.848927in}{1.225507in}}%
\pgfpathlineto{\pgfqpoint{1.853468in}{1.225507in}}%
\pgfpathlineto{\pgfqpoint{1.853468in}{1.222558in}}%
\pgfpathmoveto{\pgfqpoint{1.853468in}{1.219608in}}%
\pgfpathlineto{\pgfqpoint{1.853468in}{1.219608in}}%
\pgfpathlineto{\pgfqpoint{1.853468in}{1.222558in}}%
\pgfpathlineto{\pgfqpoint{1.858009in}{1.222558in}}%
\pgfpathlineto{\pgfqpoint{1.858009in}{1.219608in}}%
\pgfpathmoveto{\pgfqpoint{1.853468in}{1.222558in}}%
\pgfpathlineto{\pgfqpoint{1.853468in}{1.222558in}}%
\pgfpathlineto{\pgfqpoint{1.853468in}{1.225507in}}%
\pgfpathlineto{\pgfqpoint{1.858009in}{1.225507in}}%
\pgfpathlineto{\pgfqpoint{1.858009in}{1.222558in}}%
\pgfpathmoveto{\pgfqpoint{1.858009in}{1.219608in}}%
\pgfpathlineto{\pgfqpoint{1.858009in}{1.219608in}}%
\pgfpathlineto{\pgfqpoint{1.858009in}{1.222558in}}%
\pgfpathlineto{\pgfqpoint{1.862551in}{1.222558in}}%
\pgfpathlineto{\pgfqpoint{1.862551in}{1.219608in}}%
\pgfpathmoveto{\pgfqpoint{1.858009in}{1.222558in}}%
\pgfpathlineto{\pgfqpoint{1.858009in}{1.222558in}}%
\pgfpathlineto{\pgfqpoint{1.858009in}{1.225507in}}%
\pgfpathlineto{\pgfqpoint{1.862551in}{1.225507in}}%
\pgfpathlineto{\pgfqpoint{1.862551in}{1.222558in}}%
\pgfpathmoveto{\pgfqpoint{1.862551in}{1.222558in}}%
\pgfpathlineto{\pgfqpoint{1.862551in}{1.222558in}}%
\pgfpathlineto{\pgfqpoint{1.862551in}{1.225507in}}%
\pgfpathlineto{\pgfqpoint{1.867092in}{1.225507in}}%
\pgfpathlineto{\pgfqpoint{1.867092in}{1.222558in}}%
\pgfpathmoveto{\pgfqpoint{1.858009in}{1.225507in}}%
\pgfpathlineto{\pgfqpoint{1.858009in}{1.225507in}}%
\pgfpathlineto{\pgfqpoint{1.858009in}{1.228456in}}%
\pgfpathlineto{\pgfqpoint{1.862551in}{1.228456in}}%
\pgfpathlineto{\pgfqpoint{1.862551in}{1.225507in}}%
\pgfpathmoveto{\pgfqpoint{1.858009in}{1.228456in}}%
\pgfpathlineto{\pgfqpoint{1.858009in}{1.228456in}}%
\pgfpathlineto{\pgfqpoint{1.858009in}{1.231405in}}%
\pgfpathlineto{\pgfqpoint{1.862551in}{1.231405in}}%
\pgfpathlineto{\pgfqpoint{1.862551in}{1.228456in}}%
\pgfpathmoveto{\pgfqpoint{1.862551in}{1.225507in}}%
\pgfpathlineto{\pgfqpoint{1.862551in}{1.225507in}}%
\pgfpathlineto{\pgfqpoint{1.862551in}{1.228456in}}%
\pgfpathlineto{\pgfqpoint{1.867092in}{1.228456in}}%
\pgfpathlineto{\pgfqpoint{1.867092in}{1.225507in}}%
\pgfpathmoveto{\pgfqpoint{1.862551in}{1.228456in}}%
\pgfpathlineto{\pgfqpoint{1.862551in}{1.228456in}}%
\pgfpathlineto{\pgfqpoint{1.862551in}{1.231405in}}%
\pgfpathlineto{\pgfqpoint{1.867092in}{1.231405in}}%
\pgfpathlineto{\pgfqpoint{1.867092in}{1.228456in}}%
\pgfpathmoveto{\pgfqpoint{1.867092in}{1.225507in}}%
\pgfpathlineto{\pgfqpoint{1.867092in}{1.225507in}}%
\pgfpathlineto{\pgfqpoint{1.867092in}{1.228456in}}%
\pgfpathlineto{\pgfqpoint{1.871633in}{1.228456in}}%
\pgfpathlineto{\pgfqpoint{1.871633in}{1.225507in}}%
\pgfpathmoveto{\pgfqpoint{1.867092in}{1.228456in}}%
\pgfpathlineto{\pgfqpoint{1.867092in}{1.228456in}}%
\pgfpathlineto{\pgfqpoint{1.867092in}{1.231405in}}%
\pgfpathlineto{\pgfqpoint{1.871633in}{1.231405in}}%
\pgfpathlineto{\pgfqpoint{1.871633in}{1.228456in}}%
\pgfpathmoveto{\pgfqpoint{1.871633in}{1.228456in}}%
\pgfpathlineto{\pgfqpoint{1.871633in}{1.228456in}}%
\pgfpathlineto{\pgfqpoint{1.871633in}{1.231405in}}%
\pgfpathlineto{\pgfqpoint{1.876174in}{1.231405in}}%
\pgfpathlineto{\pgfqpoint{1.876174in}{1.228456in}}%
\pgfpathmoveto{\pgfqpoint{1.867092in}{1.231405in}}%
\pgfpathlineto{\pgfqpoint{1.867092in}{1.231405in}}%
\pgfpathlineto{\pgfqpoint{1.867092in}{1.234355in}}%
\pgfpathlineto{\pgfqpoint{1.871633in}{1.234355in}}%
\pgfpathlineto{\pgfqpoint{1.871633in}{1.231405in}}%
\pgfpathmoveto{\pgfqpoint{1.867092in}{1.234355in}}%
\pgfpathlineto{\pgfqpoint{1.867092in}{1.234355in}}%
\pgfpathlineto{\pgfqpoint{1.867092in}{1.237304in}}%
\pgfpathlineto{\pgfqpoint{1.871633in}{1.237304in}}%
\pgfpathlineto{\pgfqpoint{1.871633in}{1.234355in}}%
\pgfpathmoveto{\pgfqpoint{1.871633in}{1.231405in}}%
\pgfpathlineto{\pgfqpoint{1.871633in}{1.231405in}}%
\pgfpathlineto{\pgfqpoint{1.871633in}{1.234355in}}%
\pgfpathlineto{\pgfqpoint{1.876174in}{1.234355in}}%
\pgfpathlineto{\pgfqpoint{1.876174in}{1.231405in}}%
\pgfpathmoveto{\pgfqpoint{1.871633in}{1.234355in}}%
\pgfpathlineto{\pgfqpoint{1.871633in}{1.234355in}}%
\pgfpathlineto{\pgfqpoint{1.871633in}{1.237304in}}%
\pgfpathlineto{\pgfqpoint{1.876174in}{1.237304in}}%
\pgfpathlineto{\pgfqpoint{1.876174in}{1.234355in}}%
\pgfpathmoveto{\pgfqpoint{1.876174in}{1.231405in}}%
\pgfpathlineto{\pgfqpoint{1.876174in}{1.231405in}}%
\pgfpathlineto{\pgfqpoint{1.876174in}{1.234355in}}%
\pgfpathlineto{\pgfqpoint{1.880715in}{1.234355in}}%
\pgfpathlineto{\pgfqpoint{1.880715in}{1.231405in}}%
\pgfpathmoveto{\pgfqpoint{1.876174in}{1.234355in}}%
\pgfpathlineto{\pgfqpoint{1.876174in}{1.234355in}}%
\pgfpathlineto{\pgfqpoint{1.876174in}{1.237304in}}%
\pgfpathlineto{\pgfqpoint{1.880715in}{1.237304in}}%
\pgfpathlineto{\pgfqpoint{1.880715in}{1.234355in}}%
\pgfpathmoveto{\pgfqpoint{1.880715in}{1.234355in}}%
\pgfpathlineto{\pgfqpoint{1.880715in}{1.234355in}}%
\pgfpathlineto{\pgfqpoint{1.880715in}{1.237304in}}%
\pgfpathlineto{\pgfqpoint{1.885256in}{1.237304in}}%
\pgfpathlineto{\pgfqpoint{1.885256in}{1.234355in}}%
\pgfpathmoveto{\pgfqpoint{1.876174in}{1.237304in}}%
\pgfpathlineto{\pgfqpoint{1.876174in}{1.237304in}}%
\pgfpathlineto{\pgfqpoint{1.876174in}{1.240253in}}%
\pgfpathlineto{\pgfqpoint{1.880715in}{1.240253in}}%
\pgfpathlineto{\pgfqpoint{1.880715in}{1.237304in}}%
\pgfpathmoveto{\pgfqpoint{1.876174in}{1.240253in}}%
\pgfpathlineto{\pgfqpoint{1.876174in}{1.240253in}}%
\pgfpathlineto{\pgfqpoint{1.876174in}{1.243202in}}%
\pgfpathlineto{\pgfqpoint{1.880715in}{1.243202in}}%
\pgfpathlineto{\pgfqpoint{1.880715in}{1.240253in}}%
\pgfpathmoveto{\pgfqpoint{1.880715in}{1.237304in}}%
\pgfpathlineto{\pgfqpoint{1.880715in}{1.237304in}}%
\pgfpathlineto{\pgfqpoint{1.880715in}{1.240253in}}%
\pgfpathlineto{\pgfqpoint{1.885256in}{1.240253in}}%
\pgfpathlineto{\pgfqpoint{1.885256in}{1.237304in}}%
\pgfpathmoveto{\pgfqpoint{1.880715in}{1.240253in}}%
\pgfpathlineto{\pgfqpoint{1.880715in}{1.240253in}}%
\pgfpathlineto{\pgfqpoint{1.880715in}{1.243202in}}%
\pgfpathlineto{\pgfqpoint{1.885256in}{1.243202in}}%
\pgfpathlineto{\pgfqpoint{1.885256in}{1.240253in}}%
\pgfpathmoveto{\pgfqpoint{1.885256in}{1.237304in}}%
\pgfpathlineto{\pgfqpoint{1.885256in}{1.237304in}}%
\pgfpathlineto{\pgfqpoint{1.885256in}{1.240253in}}%
\pgfpathlineto{\pgfqpoint{1.889797in}{1.240253in}}%
\pgfpathlineto{\pgfqpoint{1.889797in}{1.237304in}}%
\pgfpathmoveto{\pgfqpoint{1.885256in}{1.240253in}}%
\pgfpathlineto{\pgfqpoint{1.885256in}{1.240253in}}%
\pgfpathlineto{\pgfqpoint{1.885256in}{1.243202in}}%
\pgfpathlineto{\pgfqpoint{1.889797in}{1.243202in}}%
\pgfpathlineto{\pgfqpoint{1.889797in}{1.240253in}}%
\pgfpathmoveto{\pgfqpoint{1.889797in}{1.240253in}}%
\pgfpathlineto{\pgfqpoint{1.889797in}{1.240253in}}%
\pgfpathlineto{\pgfqpoint{1.889797in}{1.243202in}}%
\pgfpathlineto{\pgfqpoint{1.894339in}{1.243202in}}%
\pgfpathlineto{\pgfqpoint{1.894339in}{1.240253in}}%
\pgfpathmoveto{\pgfqpoint{1.885256in}{1.243202in}}%
\pgfpathlineto{\pgfqpoint{1.885256in}{1.243202in}}%
\pgfpathlineto{\pgfqpoint{1.885256in}{1.246152in}}%
\pgfpathlineto{\pgfqpoint{1.889797in}{1.246152in}}%
\pgfpathlineto{\pgfqpoint{1.889797in}{1.243202in}}%
\pgfpathmoveto{\pgfqpoint{1.885256in}{1.246152in}}%
\pgfpathlineto{\pgfqpoint{1.885256in}{1.246152in}}%
\pgfpathlineto{\pgfqpoint{1.885256in}{1.249101in}}%
\pgfpathlineto{\pgfqpoint{1.889797in}{1.249101in}}%
\pgfpathlineto{\pgfqpoint{1.889797in}{1.246152in}}%
\pgfpathmoveto{\pgfqpoint{1.889797in}{1.243202in}}%
\pgfpathlineto{\pgfqpoint{1.889797in}{1.243202in}}%
\pgfpathlineto{\pgfqpoint{1.889797in}{1.246152in}}%
\pgfpathlineto{\pgfqpoint{1.894339in}{1.246152in}}%
\pgfpathlineto{\pgfqpoint{1.894339in}{1.243202in}}%
\pgfpathmoveto{\pgfqpoint{1.889797in}{1.246152in}}%
\pgfpathlineto{\pgfqpoint{1.889797in}{1.246152in}}%
\pgfpathlineto{\pgfqpoint{1.889797in}{1.249101in}}%
\pgfpathlineto{\pgfqpoint{1.894339in}{1.249101in}}%
\pgfpathlineto{\pgfqpoint{1.894339in}{1.246152in}}%
\pgfpathmoveto{\pgfqpoint{1.894339in}{1.243202in}}%
\pgfpathlineto{\pgfqpoint{1.894339in}{1.243202in}}%
\pgfpathlineto{\pgfqpoint{1.894339in}{1.246152in}}%
\pgfpathlineto{\pgfqpoint{1.898880in}{1.246152in}}%
\pgfpathlineto{\pgfqpoint{1.898880in}{1.243202in}}%
\pgfpathmoveto{\pgfqpoint{1.894339in}{1.246152in}}%
\pgfpathlineto{\pgfqpoint{1.894339in}{1.246152in}}%
\pgfpathlineto{\pgfqpoint{1.894339in}{1.249101in}}%
\pgfpathlineto{\pgfqpoint{1.898880in}{1.249101in}}%
\pgfpathlineto{\pgfqpoint{1.898880in}{1.246152in}}%
\pgfpathmoveto{\pgfqpoint{1.898880in}{1.246152in}}%
\pgfpathlineto{\pgfqpoint{1.898880in}{1.246152in}}%
\pgfpathlineto{\pgfqpoint{1.898880in}{1.249101in}}%
\pgfpathlineto{\pgfqpoint{1.903421in}{1.249101in}}%
\pgfpathlineto{\pgfqpoint{1.903421in}{1.246152in}}%
\pgfpathmoveto{\pgfqpoint{1.894339in}{1.249101in}}%
\pgfpathlineto{\pgfqpoint{1.894339in}{1.249101in}}%
\pgfpathlineto{\pgfqpoint{1.894339in}{1.252050in}}%
\pgfpathlineto{\pgfqpoint{1.898880in}{1.252050in}}%
\pgfpathlineto{\pgfqpoint{1.898880in}{1.249101in}}%
\pgfpathmoveto{\pgfqpoint{1.894339in}{1.252050in}}%
\pgfpathlineto{\pgfqpoint{1.894339in}{1.252050in}}%
\pgfpathlineto{\pgfqpoint{1.894339in}{1.254999in}}%
\pgfpathlineto{\pgfqpoint{1.898880in}{1.254999in}}%
\pgfpathlineto{\pgfqpoint{1.898880in}{1.252050in}}%
\pgfpathmoveto{\pgfqpoint{1.898880in}{1.249101in}}%
\pgfpathlineto{\pgfqpoint{1.898880in}{1.249101in}}%
\pgfpathlineto{\pgfqpoint{1.898880in}{1.252050in}}%
\pgfpathlineto{\pgfqpoint{1.903421in}{1.252050in}}%
\pgfpathlineto{\pgfqpoint{1.903421in}{1.249101in}}%
\pgfpathmoveto{\pgfqpoint{1.898880in}{1.252050in}}%
\pgfpathlineto{\pgfqpoint{1.898880in}{1.252050in}}%
\pgfpathlineto{\pgfqpoint{1.898880in}{1.254999in}}%
\pgfpathlineto{\pgfqpoint{1.903421in}{1.254999in}}%
\pgfpathlineto{\pgfqpoint{1.903421in}{1.252050in}}%
\pgfpathmoveto{\pgfqpoint{1.903421in}{1.249101in}}%
\pgfpathlineto{\pgfqpoint{1.903421in}{1.249101in}}%
\pgfpathlineto{\pgfqpoint{1.903421in}{1.252050in}}%
\pgfpathlineto{\pgfqpoint{1.907962in}{1.252050in}}%
\pgfpathlineto{\pgfqpoint{1.907962in}{1.249101in}}%
\pgfpathmoveto{\pgfqpoint{1.903421in}{1.252050in}}%
\pgfpathlineto{\pgfqpoint{1.903421in}{1.252050in}}%
\pgfpathlineto{\pgfqpoint{1.903421in}{1.254999in}}%
\pgfpathlineto{\pgfqpoint{1.907962in}{1.254999in}}%
\pgfpathlineto{\pgfqpoint{1.907962in}{1.252050in}}%
\pgfpathmoveto{\pgfqpoint{1.907962in}{1.252050in}}%
\pgfpathlineto{\pgfqpoint{1.907962in}{1.252050in}}%
\pgfpathlineto{\pgfqpoint{1.907962in}{1.254999in}}%
\pgfpathlineto{\pgfqpoint{1.912503in}{1.254999in}}%
\pgfpathlineto{\pgfqpoint{1.912503in}{1.252050in}}%
\pgfpathmoveto{\pgfqpoint{1.903421in}{1.254999in}}%
\pgfpathlineto{\pgfqpoint{1.903421in}{1.254999in}}%
\pgfpathlineto{\pgfqpoint{1.903421in}{1.257949in}}%
\pgfpathlineto{\pgfqpoint{1.907962in}{1.257949in}}%
\pgfpathlineto{\pgfqpoint{1.907962in}{1.254999in}}%
\pgfpathmoveto{\pgfqpoint{1.903421in}{1.257949in}}%
\pgfpathlineto{\pgfqpoint{1.903421in}{1.257949in}}%
\pgfpathlineto{\pgfqpoint{1.903421in}{1.260898in}}%
\pgfpathlineto{\pgfqpoint{1.907962in}{1.260898in}}%
\pgfpathlineto{\pgfqpoint{1.907962in}{1.257949in}}%
\pgfpathmoveto{\pgfqpoint{1.907962in}{1.254999in}}%
\pgfpathlineto{\pgfqpoint{1.907962in}{1.254999in}}%
\pgfpathlineto{\pgfqpoint{1.907962in}{1.257949in}}%
\pgfpathlineto{\pgfqpoint{1.912503in}{1.257949in}}%
\pgfpathlineto{\pgfqpoint{1.912503in}{1.254999in}}%
\pgfpathmoveto{\pgfqpoint{1.907962in}{1.257949in}}%
\pgfpathlineto{\pgfqpoint{1.907962in}{1.257949in}}%
\pgfpathlineto{\pgfqpoint{1.907962in}{1.260898in}}%
\pgfpathlineto{\pgfqpoint{1.912503in}{1.260898in}}%
\pgfpathlineto{\pgfqpoint{1.912503in}{1.257949in}}%
\pgfpathmoveto{\pgfqpoint{1.912503in}{1.254999in}}%
\pgfpathlineto{\pgfqpoint{1.912503in}{1.254999in}}%
\pgfpathlineto{\pgfqpoint{1.912503in}{1.257949in}}%
\pgfpathlineto{\pgfqpoint{1.917044in}{1.257949in}}%
\pgfpathlineto{\pgfqpoint{1.917044in}{1.254999in}}%
\pgfpathmoveto{\pgfqpoint{1.912503in}{1.257949in}}%
\pgfpathlineto{\pgfqpoint{1.912503in}{1.257949in}}%
\pgfpathlineto{\pgfqpoint{1.912503in}{1.260898in}}%
\pgfpathlineto{\pgfqpoint{1.917044in}{1.260898in}}%
\pgfpathlineto{\pgfqpoint{1.917044in}{1.257949in}}%
\pgfpathmoveto{\pgfqpoint{1.917044in}{1.257949in}}%
\pgfpathlineto{\pgfqpoint{1.917044in}{1.257949in}}%
\pgfpathlineto{\pgfqpoint{1.917044in}{1.260898in}}%
\pgfpathlineto{\pgfqpoint{1.921585in}{1.260898in}}%
\pgfpathlineto{\pgfqpoint{1.921585in}{1.257949in}}%
\pgfpathmoveto{\pgfqpoint{1.912503in}{1.260898in}}%
\pgfpathlineto{\pgfqpoint{1.912503in}{1.260898in}}%
\pgfpathlineto{\pgfqpoint{1.912503in}{1.263847in}}%
\pgfpathlineto{\pgfqpoint{1.917044in}{1.263847in}}%
\pgfpathlineto{\pgfqpoint{1.917044in}{1.260898in}}%
\pgfpathmoveto{\pgfqpoint{1.912503in}{1.263847in}}%
\pgfpathlineto{\pgfqpoint{1.912503in}{1.263847in}}%
\pgfpathlineto{\pgfqpoint{1.912503in}{1.266796in}}%
\pgfpathlineto{\pgfqpoint{1.917044in}{1.266796in}}%
\pgfpathlineto{\pgfqpoint{1.917044in}{1.263847in}}%
\pgfpathmoveto{\pgfqpoint{1.917044in}{1.260898in}}%
\pgfpathlineto{\pgfqpoint{1.917044in}{1.260898in}}%
\pgfpathlineto{\pgfqpoint{1.917044in}{1.263847in}}%
\pgfpathlineto{\pgfqpoint{1.921585in}{1.263847in}}%
\pgfpathlineto{\pgfqpoint{1.921585in}{1.260898in}}%
\pgfpathmoveto{\pgfqpoint{1.917044in}{1.263847in}}%
\pgfpathlineto{\pgfqpoint{1.917044in}{1.263847in}}%
\pgfpathlineto{\pgfqpoint{1.917044in}{1.266796in}}%
\pgfpathlineto{\pgfqpoint{1.921585in}{1.266796in}}%
\pgfpathlineto{\pgfqpoint{1.921585in}{1.263847in}}%
\pgfpathmoveto{\pgfqpoint{1.921585in}{1.260898in}}%
\pgfpathlineto{\pgfqpoint{1.921585in}{1.260898in}}%
\pgfpathlineto{\pgfqpoint{1.921585in}{1.263847in}}%
\pgfpathlineto{\pgfqpoint{1.926126in}{1.263847in}}%
\pgfpathlineto{\pgfqpoint{1.926126in}{1.260898in}}%
\pgfpathmoveto{\pgfqpoint{1.921585in}{1.263847in}}%
\pgfpathlineto{\pgfqpoint{1.921585in}{1.263847in}}%
\pgfpathlineto{\pgfqpoint{1.921585in}{1.266796in}}%
\pgfpathlineto{\pgfqpoint{1.926126in}{1.266796in}}%
\pgfpathlineto{\pgfqpoint{1.926126in}{1.263847in}}%
\pgfpathmoveto{\pgfqpoint{1.926126in}{1.263847in}}%
\pgfpathlineto{\pgfqpoint{1.926126in}{1.263847in}}%
\pgfpathlineto{\pgfqpoint{1.926126in}{1.266796in}}%
\pgfpathlineto{\pgfqpoint{1.930667in}{1.266796in}}%
\pgfpathlineto{\pgfqpoint{1.930667in}{1.263847in}}%
\pgfpathmoveto{\pgfqpoint{1.921585in}{1.266796in}}%
\pgfpathlineto{\pgfqpoint{1.921585in}{1.266796in}}%
\pgfpathlineto{\pgfqpoint{1.921585in}{1.269746in}}%
\pgfpathlineto{\pgfqpoint{1.926126in}{1.269746in}}%
\pgfpathlineto{\pgfqpoint{1.926126in}{1.266796in}}%
\pgfpathmoveto{\pgfqpoint{1.921585in}{1.269746in}}%
\pgfpathlineto{\pgfqpoint{1.921585in}{1.269746in}}%
\pgfpathlineto{\pgfqpoint{1.921585in}{1.272695in}}%
\pgfpathlineto{\pgfqpoint{1.926126in}{1.272695in}}%
\pgfpathlineto{\pgfqpoint{1.926126in}{1.269746in}}%
\pgfpathmoveto{\pgfqpoint{1.926126in}{1.266796in}}%
\pgfpathlineto{\pgfqpoint{1.926126in}{1.266796in}}%
\pgfpathlineto{\pgfqpoint{1.926126in}{1.269746in}}%
\pgfpathlineto{\pgfqpoint{1.930667in}{1.269746in}}%
\pgfpathlineto{\pgfqpoint{1.930667in}{1.266796in}}%
\pgfpathmoveto{\pgfqpoint{1.926126in}{1.269746in}}%
\pgfpathlineto{\pgfqpoint{1.926126in}{1.269746in}}%
\pgfpathlineto{\pgfqpoint{1.926126in}{1.272695in}}%
\pgfpathlineto{\pgfqpoint{1.930667in}{1.272695in}}%
\pgfpathlineto{\pgfqpoint{1.930667in}{1.269746in}}%
\pgfpathmoveto{\pgfqpoint{1.930667in}{1.266796in}}%
\pgfpathlineto{\pgfqpoint{1.930667in}{1.266796in}}%
\pgfpathlineto{\pgfqpoint{1.930667in}{1.269746in}}%
\pgfpathlineto{\pgfqpoint{1.935207in}{1.269746in}}%
\pgfpathlineto{\pgfqpoint{1.935207in}{1.266796in}}%
\pgfpathmoveto{\pgfqpoint{1.930667in}{1.269746in}}%
\pgfpathlineto{\pgfqpoint{1.930667in}{1.269746in}}%
\pgfpathlineto{\pgfqpoint{1.930667in}{1.272695in}}%
\pgfpathlineto{\pgfqpoint{1.935207in}{1.272695in}}%
\pgfpathlineto{\pgfqpoint{1.935207in}{1.269746in}}%
\pgfpathmoveto{\pgfqpoint{1.935207in}{1.269746in}}%
\pgfpathlineto{\pgfqpoint{1.935207in}{1.269746in}}%
\pgfpathlineto{\pgfqpoint{1.935207in}{1.272695in}}%
\pgfpathlineto{\pgfqpoint{1.939748in}{1.272695in}}%
\pgfpathlineto{\pgfqpoint{1.939748in}{1.269746in}}%
\pgfpathmoveto{\pgfqpoint{1.930667in}{1.272695in}}%
\pgfpathlineto{\pgfqpoint{1.930667in}{1.272695in}}%
\pgfpathlineto{\pgfqpoint{1.930667in}{1.275644in}}%
\pgfpathlineto{\pgfqpoint{1.935207in}{1.275644in}}%
\pgfpathlineto{\pgfqpoint{1.935207in}{1.272695in}}%
\pgfpathmoveto{\pgfqpoint{1.930667in}{1.275644in}}%
\pgfpathlineto{\pgfqpoint{1.930667in}{1.275644in}}%
\pgfpathlineto{\pgfqpoint{1.930667in}{1.278594in}}%
\pgfpathlineto{\pgfqpoint{1.935207in}{1.278594in}}%
\pgfpathlineto{\pgfqpoint{1.935207in}{1.275644in}}%
\pgfpathmoveto{\pgfqpoint{1.935207in}{1.272695in}}%
\pgfpathlineto{\pgfqpoint{1.935207in}{1.272695in}}%
\pgfpathlineto{\pgfqpoint{1.935207in}{1.275644in}}%
\pgfpathlineto{\pgfqpoint{1.939748in}{1.275644in}}%
\pgfpathlineto{\pgfqpoint{1.939748in}{1.272695in}}%
\pgfpathmoveto{\pgfqpoint{1.935207in}{1.275644in}}%
\pgfpathlineto{\pgfqpoint{1.935207in}{1.275644in}}%
\pgfpathlineto{\pgfqpoint{1.935207in}{1.278594in}}%
\pgfpathlineto{\pgfqpoint{1.939748in}{1.278594in}}%
\pgfpathlineto{\pgfqpoint{1.939748in}{1.275644in}}%
\pgfpathmoveto{\pgfqpoint{1.939748in}{1.272695in}}%
\pgfpathlineto{\pgfqpoint{1.939748in}{1.272695in}}%
\pgfpathlineto{\pgfqpoint{1.939748in}{1.275644in}}%
\pgfpathlineto{\pgfqpoint{1.944289in}{1.275644in}}%
\pgfpathlineto{\pgfqpoint{1.944289in}{1.272695in}}%
\pgfpathmoveto{\pgfqpoint{1.939748in}{1.275644in}}%
\pgfpathlineto{\pgfqpoint{1.939748in}{1.275644in}}%
\pgfpathlineto{\pgfqpoint{1.939748in}{1.278594in}}%
\pgfpathlineto{\pgfqpoint{1.944289in}{1.278594in}}%
\pgfpathlineto{\pgfqpoint{1.944289in}{1.275644in}}%
\pgfpathmoveto{\pgfqpoint{1.944289in}{1.275644in}}%
\pgfpathlineto{\pgfqpoint{1.944289in}{1.275644in}}%
\pgfpathlineto{\pgfqpoint{1.944289in}{1.278594in}}%
\pgfpathlineto{\pgfqpoint{1.948830in}{1.278594in}}%
\pgfpathlineto{\pgfqpoint{1.948830in}{1.275644in}}%
\pgfpathmoveto{\pgfqpoint{1.939748in}{1.278594in}}%
\pgfpathlineto{\pgfqpoint{1.939748in}{1.278594in}}%
\pgfpathlineto{\pgfqpoint{1.939748in}{1.281543in}}%
\pgfpathlineto{\pgfqpoint{1.944289in}{1.281543in}}%
\pgfpathlineto{\pgfqpoint{1.944289in}{1.278594in}}%
\pgfpathmoveto{\pgfqpoint{1.939748in}{1.281543in}}%
\pgfpathlineto{\pgfqpoint{1.939748in}{1.281543in}}%
\pgfpathlineto{\pgfqpoint{1.939748in}{1.284492in}}%
\pgfpathlineto{\pgfqpoint{1.944289in}{1.284492in}}%
\pgfpathlineto{\pgfqpoint{1.944289in}{1.281543in}}%
\pgfpathmoveto{\pgfqpoint{1.944289in}{1.278594in}}%
\pgfpathlineto{\pgfqpoint{1.944289in}{1.278594in}}%
\pgfpathlineto{\pgfqpoint{1.944289in}{1.281543in}}%
\pgfpathlineto{\pgfqpoint{1.948830in}{1.281543in}}%
\pgfpathlineto{\pgfqpoint{1.948830in}{1.278594in}}%
\pgfpathmoveto{\pgfqpoint{1.944289in}{1.281543in}}%
\pgfpathlineto{\pgfqpoint{1.944289in}{1.281543in}}%
\pgfpathlineto{\pgfqpoint{1.944289in}{1.284492in}}%
\pgfpathlineto{\pgfqpoint{1.948830in}{1.284492in}}%
\pgfpathlineto{\pgfqpoint{1.948830in}{1.281543in}}%
\pgfpathmoveto{\pgfqpoint{1.948830in}{1.278594in}}%
\pgfpathlineto{\pgfqpoint{1.948830in}{1.278594in}}%
\pgfpathlineto{\pgfqpoint{1.948830in}{1.281543in}}%
\pgfpathlineto{\pgfqpoint{1.953371in}{1.281543in}}%
\pgfpathlineto{\pgfqpoint{1.953371in}{1.278594in}}%
\pgfpathmoveto{\pgfqpoint{1.948830in}{1.281543in}}%
\pgfpathlineto{\pgfqpoint{1.948830in}{1.281543in}}%
\pgfpathlineto{\pgfqpoint{1.948830in}{1.284492in}}%
\pgfpathlineto{\pgfqpoint{1.953371in}{1.284492in}}%
\pgfpathlineto{\pgfqpoint{1.953371in}{1.281543in}}%
\pgfpathmoveto{\pgfqpoint{1.953371in}{1.281543in}}%
\pgfpathlineto{\pgfqpoint{1.953371in}{1.281543in}}%
\pgfpathlineto{\pgfqpoint{1.953371in}{1.284492in}}%
\pgfpathlineto{\pgfqpoint{1.957912in}{1.284492in}}%
\pgfpathlineto{\pgfqpoint{1.957912in}{1.281543in}}%
\pgfpathmoveto{\pgfqpoint{1.948830in}{1.284492in}}%
\pgfpathlineto{\pgfqpoint{1.948830in}{1.284492in}}%
\pgfpathlineto{\pgfqpoint{1.948830in}{1.287441in}}%
\pgfpathlineto{\pgfqpoint{1.953371in}{1.287441in}}%
\pgfpathlineto{\pgfqpoint{1.953371in}{1.284492in}}%
\pgfpathmoveto{\pgfqpoint{1.948830in}{1.287441in}}%
\pgfpathlineto{\pgfqpoint{1.948830in}{1.287441in}}%
\pgfpathlineto{\pgfqpoint{1.948830in}{1.290391in}}%
\pgfpathlineto{\pgfqpoint{1.953371in}{1.290391in}}%
\pgfpathlineto{\pgfqpoint{1.953371in}{1.287441in}}%
\pgfpathmoveto{\pgfqpoint{1.953371in}{1.284492in}}%
\pgfpathlineto{\pgfqpoint{1.953371in}{1.284492in}}%
\pgfpathlineto{\pgfqpoint{1.953371in}{1.287441in}}%
\pgfpathlineto{\pgfqpoint{1.957912in}{1.287441in}}%
\pgfpathlineto{\pgfqpoint{1.957912in}{1.284492in}}%
\pgfpathmoveto{\pgfqpoint{1.953371in}{1.287441in}}%
\pgfpathlineto{\pgfqpoint{1.953371in}{1.287441in}}%
\pgfpathlineto{\pgfqpoint{1.953371in}{1.290391in}}%
\pgfpathlineto{\pgfqpoint{1.957912in}{1.290391in}}%
\pgfpathlineto{\pgfqpoint{1.957912in}{1.287441in}}%
\pgfpathmoveto{\pgfqpoint{1.957912in}{1.284492in}}%
\pgfpathlineto{\pgfqpoint{1.957912in}{1.284492in}}%
\pgfpathlineto{\pgfqpoint{1.957912in}{1.287441in}}%
\pgfpathlineto{\pgfqpoint{1.962452in}{1.287441in}}%
\pgfpathlineto{\pgfqpoint{1.962452in}{1.284492in}}%
\pgfpathmoveto{\pgfqpoint{1.957912in}{1.287441in}}%
\pgfpathlineto{\pgfqpoint{1.957912in}{1.287441in}}%
\pgfpathlineto{\pgfqpoint{1.957912in}{1.290391in}}%
\pgfpathlineto{\pgfqpoint{1.962452in}{1.290391in}}%
\pgfpathlineto{\pgfqpoint{1.962452in}{1.287441in}}%
\pgfpathmoveto{\pgfqpoint{1.962452in}{1.287441in}}%
\pgfpathlineto{\pgfqpoint{1.962452in}{1.287441in}}%
\pgfpathlineto{\pgfqpoint{1.962452in}{1.290391in}}%
\pgfpathlineto{\pgfqpoint{1.966993in}{1.290391in}}%
\pgfpathlineto{\pgfqpoint{1.966993in}{1.287441in}}%
\pgfpathmoveto{\pgfqpoint{1.957912in}{1.290391in}}%
\pgfpathlineto{\pgfqpoint{1.957912in}{1.290391in}}%
\pgfpathlineto{\pgfqpoint{1.957912in}{1.293340in}}%
\pgfpathlineto{\pgfqpoint{1.962452in}{1.293340in}}%
\pgfpathlineto{\pgfqpoint{1.962452in}{1.290391in}}%
\pgfpathmoveto{\pgfqpoint{1.957912in}{1.293340in}}%
\pgfpathlineto{\pgfqpoint{1.957912in}{1.293340in}}%
\pgfpathlineto{\pgfqpoint{1.957912in}{1.296289in}}%
\pgfpathlineto{\pgfqpoint{1.962452in}{1.296289in}}%
\pgfpathlineto{\pgfqpoint{1.962452in}{1.293340in}}%
\pgfpathmoveto{\pgfqpoint{1.962452in}{1.290391in}}%
\pgfpathlineto{\pgfqpoint{1.962452in}{1.290391in}}%
\pgfpathlineto{\pgfqpoint{1.962452in}{1.293340in}}%
\pgfpathlineto{\pgfqpoint{1.966993in}{1.293340in}}%
\pgfpathlineto{\pgfqpoint{1.966993in}{1.290391in}}%
\pgfpathmoveto{\pgfqpoint{1.962452in}{1.293340in}}%
\pgfpathlineto{\pgfqpoint{1.962452in}{1.293340in}}%
\pgfpathlineto{\pgfqpoint{1.962452in}{1.296289in}}%
\pgfpathlineto{\pgfqpoint{1.966993in}{1.296289in}}%
\pgfpathlineto{\pgfqpoint{1.966993in}{1.293340in}}%
\pgfpathmoveto{\pgfqpoint{1.966993in}{1.290391in}}%
\pgfpathlineto{\pgfqpoint{1.966993in}{1.290391in}}%
\pgfpathlineto{\pgfqpoint{1.966993in}{1.293340in}}%
\pgfpathlineto{\pgfqpoint{1.971534in}{1.293340in}}%
\pgfpathlineto{\pgfqpoint{1.971534in}{1.290391in}}%
\pgfpathmoveto{\pgfqpoint{1.966993in}{1.293340in}}%
\pgfpathlineto{\pgfqpoint{1.966993in}{1.293340in}}%
\pgfpathlineto{\pgfqpoint{1.966993in}{1.296289in}}%
\pgfpathlineto{\pgfqpoint{1.971534in}{1.296289in}}%
\pgfpathlineto{\pgfqpoint{1.971534in}{1.293340in}}%
\pgfpathmoveto{\pgfqpoint{1.971534in}{1.293340in}}%
\pgfpathlineto{\pgfqpoint{1.971534in}{1.293340in}}%
\pgfpathlineto{\pgfqpoint{1.971534in}{1.296289in}}%
\pgfpathlineto{\pgfqpoint{1.976075in}{1.296289in}}%
\pgfpathlineto{\pgfqpoint{1.976075in}{1.293340in}}%
\pgfpathmoveto{\pgfqpoint{1.966993in}{1.296289in}}%
\pgfpathlineto{\pgfqpoint{1.966993in}{1.296289in}}%
\pgfpathlineto{\pgfqpoint{1.966993in}{1.299238in}}%
\pgfpathlineto{\pgfqpoint{1.971534in}{1.299238in}}%
\pgfpathlineto{\pgfqpoint{1.971534in}{1.296289in}}%
\pgfpathmoveto{\pgfqpoint{1.966993in}{1.299238in}}%
\pgfpathlineto{\pgfqpoint{1.966993in}{1.299238in}}%
\pgfpathlineto{\pgfqpoint{1.966993in}{1.302188in}}%
\pgfpathlineto{\pgfqpoint{1.971534in}{1.302188in}}%
\pgfpathlineto{\pgfqpoint{1.971534in}{1.299238in}}%
\pgfpathmoveto{\pgfqpoint{1.971534in}{1.296289in}}%
\pgfpathlineto{\pgfqpoint{1.971534in}{1.296289in}}%
\pgfpathlineto{\pgfqpoint{1.971534in}{1.299238in}}%
\pgfpathlineto{\pgfqpoint{1.976075in}{1.299238in}}%
\pgfpathlineto{\pgfqpoint{1.976075in}{1.296289in}}%
\pgfpathmoveto{\pgfqpoint{1.971534in}{1.299238in}}%
\pgfpathlineto{\pgfqpoint{1.971534in}{1.299238in}}%
\pgfpathlineto{\pgfqpoint{1.971534in}{1.302188in}}%
\pgfpathlineto{\pgfqpoint{1.976075in}{1.302188in}}%
\pgfpathlineto{\pgfqpoint{1.976075in}{1.299238in}}%
\pgfpathmoveto{\pgfqpoint{1.976075in}{1.296289in}}%
\pgfpathlineto{\pgfqpoint{1.976075in}{1.296289in}}%
\pgfpathlineto{\pgfqpoint{1.976075in}{1.299238in}}%
\pgfpathlineto{\pgfqpoint{1.980616in}{1.299238in}}%
\pgfpathlineto{\pgfqpoint{1.980616in}{1.296289in}}%
\pgfpathmoveto{\pgfqpoint{1.976075in}{1.299238in}}%
\pgfpathlineto{\pgfqpoint{1.976075in}{1.299238in}}%
\pgfpathlineto{\pgfqpoint{1.976075in}{1.302188in}}%
\pgfpathlineto{\pgfqpoint{1.980616in}{1.302188in}}%
\pgfpathlineto{\pgfqpoint{1.980616in}{1.299238in}}%
\pgfpathmoveto{\pgfqpoint{1.980616in}{1.299238in}}%
\pgfpathlineto{\pgfqpoint{1.980616in}{1.299238in}}%
\pgfpathlineto{\pgfqpoint{1.980616in}{1.302188in}}%
\pgfpathlineto{\pgfqpoint{1.985157in}{1.302188in}}%
\pgfpathlineto{\pgfqpoint{1.985157in}{1.299238in}}%
\pgfpathmoveto{\pgfqpoint{1.976075in}{1.302188in}}%
\pgfpathlineto{\pgfqpoint{1.976075in}{1.302188in}}%
\pgfpathlineto{\pgfqpoint{1.976075in}{1.305137in}}%
\pgfpathlineto{\pgfqpoint{1.980616in}{1.305137in}}%
\pgfpathlineto{\pgfqpoint{1.980616in}{1.302188in}}%
\pgfpathmoveto{\pgfqpoint{1.976075in}{1.305137in}}%
\pgfpathlineto{\pgfqpoint{1.976075in}{1.305137in}}%
\pgfpathlineto{\pgfqpoint{1.976075in}{1.308086in}}%
\pgfpathlineto{\pgfqpoint{1.980616in}{1.308086in}}%
\pgfpathlineto{\pgfqpoint{1.980616in}{1.305137in}}%
\pgfpathmoveto{\pgfqpoint{1.980616in}{1.302188in}}%
\pgfpathlineto{\pgfqpoint{1.980616in}{1.302188in}}%
\pgfpathlineto{\pgfqpoint{1.980616in}{1.305137in}}%
\pgfpathlineto{\pgfqpoint{1.985157in}{1.305137in}}%
\pgfpathlineto{\pgfqpoint{1.985157in}{1.302188in}}%
\pgfpathmoveto{\pgfqpoint{1.980616in}{1.305137in}}%
\pgfpathlineto{\pgfqpoint{1.980616in}{1.305137in}}%
\pgfpathlineto{\pgfqpoint{1.980616in}{1.308086in}}%
\pgfpathlineto{\pgfqpoint{1.985157in}{1.308086in}}%
\pgfpathlineto{\pgfqpoint{1.985157in}{1.305137in}}%
\pgfpathmoveto{\pgfqpoint{1.985157in}{1.302188in}}%
\pgfpathlineto{\pgfqpoint{1.985157in}{1.302188in}}%
\pgfpathlineto{\pgfqpoint{1.985157in}{1.305137in}}%
\pgfpathlineto{\pgfqpoint{1.989697in}{1.305137in}}%
\pgfpathlineto{\pgfqpoint{1.989697in}{1.302188in}}%
\pgfpathmoveto{\pgfqpoint{1.985157in}{1.305137in}}%
\pgfpathlineto{\pgfqpoint{1.985157in}{1.305137in}}%
\pgfpathlineto{\pgfqpoint{1.985157in}{1.308086in}}%
\pgfpathlineto{\pgfqpoint{1.989697in}{1.308086in}}%
\pgfpathlineto{\pgfqpoint{1.989697in}{1.305137in}}%
\pgfpathmoveto{\pgfqpoint{1.989697in}{1.305137in}}%
\pgfpathlineto{\pgfqpoint{1.989697in}{1.305137in}}%
\pgfpathlineto{\pgfqpoint{1.989697in}{1.308086in}}%
\pgfpathlineto{\pgfqpoint{1.994238in}{1.308086in}}%
\pgfpathlineto{\pgfqpoint{1.994238in}{1.305137in}}%
\pgfpathmoveto{\pgfqpoint{1.994238in}{1.305137in}}%
\pgfpathlineto{\pgfqpoint{1.994238in}{1.305137in}}%
\pgfpathlineto{\pgfqpoint{1.994238in}{1.308086in}}%
\pgfpathlineto{\pgfqpoint{1.998779in}{1.308086in}}%
\pgfpathlineto{\pgfqpoint{1.998779in}{1.305137in}}%
\pgfpathmoveto{\pgfqpoint{1.994238in}{1.308086in}}%
\pgfpathlineto{\pgfqpoint{1.994238in}{1.308086in}}%
\pgfpathlineto{\pgfqpoint{1.994238in}{1.311035in}}%
\pgfpathlineto{\pgfqpoint{1.998779in}{1.311035in}}%
\pgfpathlineto{\pgfqpoint{1.998779in}{1.308086in}}%
\pgfpathmoveto{\pgfqpoint{1.994238in}{1.311035in}}%
\pgfpathlineto{\pgfqpoint{1.994238in}{1.311035in}}%
\pgfpathlineto{\pgfqpoint{1.994238in}{1.313985in}}%
\pgfpathlineto{\pgfqpoint{1.998779in}{1.313985in}}%
\pgfpathlineto{\pgfqpoint{1.998779in}{1.311035in}}%
\pgfpathmoveto{\pgfqpoint{1.998779in}{1.308086in}}%
\pgfpathlineto{\pgfqpoint{1.998779in}{1.308086in}}%
\pgfpathlineto{\pgfqpoint{1.998779in}{1.311035in}}%
\pgfpathlineto{\pgfqpoint{2.003320in}{1.311035in}}%
\pgfpathlineto{\pgfqpoint{2.003320in}{1.308086in}}%
\pgfpathmoveto{\pgfqpoint{1.998779in}{1.311035in}}%
\pgfpathlineto{\pgfqpoint{1.998779in}{1.311035in}}%
\pgfpathlineto{\pgfqpoint{1.998779in}{1.313985in}}%
\pgfpathlineto{\pgfqpoint{2.003320in}{1.313985in}}%
\pgfpathlineto{\pgfqpoint{2.003320in}{1.311035in}}%
\pgfpathmoveto{\pgfqpoint{2.003320in}{1.311035in}}%
\pgfpathlineto{\pgfqpoint{2.003320in}{1.311035in}}%
\pgfpathlineto{\pgfqpoint{2.003320in}{1.313985in}}%
\pgfpathlineto{\pgfqpoint{2.007861in}{1.313985in}}%
\pgfpathlineto{\pgfqpoint{2.007861in}{1.311035in}}%
\pgfpathmoveto{\pgfqpoint{2.003320in}{1.313985in}}%
\pgfpathlineto{\pgfqpoint{2.003320in}{1.313985in}}%
\pgfpathlineto{\pgfqpoint{2.003320in}{1.316934in}}%
\pgfpathlineto{\pgfqpoint{2.007861in}{1.316934in}}%
\pgfpathlineto{\pgfqpoint{2.007861in}{1.313985in}}%
\pgfpathmoveto{\pgfqpoint{2.003320in}{1.316934in}}%
\pgfpathlineto{\pgfqpoint{2.003320in}{1.316934in}}%
\pgfpathlineto{\pgfqpoint{2.003320in}{1.319883in}}%
\pgfpathlineto{\pgfqpoint{2.007861in}{1.319883in}}%
\pgfpathlineto{\pgfqpoint{2.007861in}{1.316934in}}%
\pgfpathmoveto{\pgfqpoint{2.007861in}{1.313985in}}%
\pgfpathlineto{\pgfqpoint{2.007861in}{1.313985in}}%
\pgfpathlineto{\pgfqpoint{2.007861in}{1.316934in}}%
\pgfpathlineto{\pgfqpoint{2.012402in}{1.316934in}}%
\pgfpathlineto{\pgfqpoint{2.012402in}{1.313985in}}%
\pgfpathmoveto{\pgfqpoint{2.007861in}{1.316934in}}%
\pgfpathlineto{\pgfqpoint{2.007861in}{1.316934in}}%
\pgfpathlineto{\pgfqpoint{2.007861in}{1.319883in}}%
\pgfpathlineto{\pgfqpoint{2.012402in}{1.319883in}}%
\pgfpathlineto{\pgfqpoint{2.012402in}{1.316934in}}%
\pgfpathmoveto{\pgfqpoint{2.012402in}{1.316934in}}%
\pgfpathlineto{\pgfqpoint{2.012402in}{1.316934in}}%
\pgfpathlineto{\pgfqpoint{2.012402in}{1.319883in}}%
\pgfpathlineto{\pgfqpoint{2.016943in}{1.319883in}}%
\pgfpathlineto{\pgfqpoint{2.016943in}{1.316934in}}%
\pgfpathmoveto{\pgfqpoint{2.012402in}{1.319883in}}%
\pgfpathlineto{\pgfqpoint{2.012402in}{1.319883in}}%
\pgfpathlineto{\pgfqpoint{2.012402in}{1.322832in}}%
\pgfpathlineto{\pgfqpoint{2.016943in}{1.322832in}}%
\pgfpathlineto{\pgfqpoint{2.016943in}{1.319883in}}%
\pgfpathmoveto{\pgfqpoint{2.012402in}{1.322832in}}%
\pgfpathlineto{\pgfqpoint{2.012402in}{1.322832in}}%
\pgfpathlineto{\pgfqpoint{2.012402in}{1.325782in}}%
\pgfpathlineto{\pgfqpoint{2.016943in}{1.325782in}}%
\pgfpathlineto{\pgfqpoint{2.016943in}{1.322832in}}%
\pgfpathmoveto{\pgfqpoint{2.016943in}{1.319883in}}%
\pgfpathlineto{\pgfqpoint{2.016943in}{1.319883in}}%
\pgfpathlineto{\pgfqpoint{2.016943in}{1.322832in}}%
\pgfpathlineto{\pgfqpoint{2.021483in}{1.322832in}}%
\pgfpathlineto{\pgfqpoint{2.021483in}{1.319883in}}%
\pgfpathmoveto{\pgfqpoint{2.016943in}{1.322832in}}%
\pgfpathlineto{\pgfqpoint{2.016943in}{1.322832in}}%
\pgfpathlineto{\pgfqpoint{2.016943in}{1.325782in}}%
\pgfpathlineto{\pgfqpoint{2.021483in}{1.325782in}}%
\pgfpathlineto{\pgfqpoint{2.021483in}{1.322832in}}%
\pgfpathmoveto{\pgfqpoint{2.021483in}{1.322832in}}%
\pgfpathlineto{\pgfqpoint{2.021483in}{1.322832in}}%
\pgfpathlineto{\pgfqpoint{2.021483in}{1.325782in}}%
\pgfpathlineto{\pgfqpoint{2.026024in}{1.325782in}}%
\pgfpathlineto{\pgfqpoint{2.026024in}{1.322832in}}%
\pgfpathmoveto{\pgfqpoint{2.021483in}{1.325782in}}%
\pgfpathlineto{\pgfqpoint{2.021483in}{1.325782in}}%
\pgfpathlineto{\pgfqpoint{2.021483in}{1.328731in}}%
\pgfpathlineto{\pgfqpoint{2.026024in}{1.328731in}}%
\pgfpathlineto{\pgfqpoint{2.026024in}{1.325782in}}%
\pgfpathmoveto{\pgfqpoint{2.021483in}{1.328731in}}%
\pgfpathlineto{\pgfqpoint{2.021483in}{1.328731in}}%
\pgfpathlineto{\pgfqpoint{2.021483in}{1.331680in}}%
\pgfpathlineto{\pgfqpoint{2.026024in}{1.331680in}}%
\pgfpathlineto{\pgfqpoint{2.026024in}{1.328731in}}%
\pgfpathmoveto{\pgfqpoint{2.026024in}{1.325782in}}%
\pgfpathlineto{\pgfqpoint{2.026024in}{1.325782in}}%
\pgfpathlineto{\pgfqpoint{2.026024in}{1.328731in}}%
\pgfpathlineto{\pgfqpoint{2.030565in}{1.328731in}}%
\pgfpathlineto{\pgfqpoint{2.030565in}{1.325782in}}%
\pgfpathmoveto{\pgfqpoint{2.026024in}{1.328731in}}%
\pgfpathlineto{\pgfqpoint{2.026024in}{1.328731in}}%
\pgfpathlineto{\pgfqpoint{2.026024in}{1.331680in}}%
\pgfpathlineto{\pgfqpoint{2.030565in}{1.331680in}}%
\pgfpathlineto{\pgfqpoint{2.030565in}{1.328731in}}%
\pgfpathmoveto{\pgfqpoint{2.030565in}{1.328731in}}%
\pgfpathlineto{\pgfqpoint{2.030565in}{1.328731in}}%
\pgfpathlineto{\pgfqpoint{2.030565in}{1.331680in}}%
\pgfpathlineto{\pgfqpoint{2.035106in}{1.331680in}}%
\pgfpathlineto{\pgfqpoint{2.035106in}{1.328731in}}%
\pgfpathmoveto{\pgfqpoint{2.030565in}{1.331680in}}%
\pgfpathlineto{\pgfqpoint{2.030565in}{1.331680in}}%
\pgfpathlineto{\pgfqpoint{2.030565in}{1.334629in}}%
\pgfpathlineto{\pgfqpoint{2.035106in}{1.334629in}}%
\pgfpathlineto{\pgfqpoint{2.035106in}{1.331680in}}%
\pgfpathmoveto{\pgfqpoint{2.030565in}{1.334629in}}%
\pgfpathlineto{\pgfqpoint{2.030565in}{1.334629in}}%
\pgfpathlineto{\pgfqpoint{2.030565in}{1.337579in}}%
\pgfpathlineto{\pgfqpoint{2.035106in}{1.337579in}}%
\pgfpathlineto{\pgfqpoint{2.035106in}{1.334629in}}%
\pgfpathmoveto{\pgfqpoint{2.035106in}{1.331680in}}%
\pgfpathlineto{\pgfqpoint{2.035106in}{1.331680in}}%
\pgfpathlineto{\pgfqpoint{2.035106in}{1.334629in}}%
\pgfpathlineto{\pgfqpoint{2.039647in}{1.334629in}}%
\pgfpathlineto{\pgfqpoint{2.039647in}{1.331680in}}%
\pgfpathmoveto{\pgfqpoint{2.035106in}{1.334629in}}%
\pgfpathlineto{\pgfqpoint{2.035106in}{1.334629in}}%
\pgfpathlineto{\pgfqpoint{2.035106in}{1.337579in}}%
\pgfpathlineto{\pgfqpoint{2.039647in}{1.337579in}}%
\pgfpathlineto{\pgfqpoint{2.039647in}{1.334629in}}%
\pgfpathmoveto{\pgfqpoint{2.039647in}{1.334629in}}%
\pgfpathlineto{\pgfqpoint{2.039647in}{1.334629in}}%
\pgfpathlineto{\pgfqpoint{2.039647in}{1.337579in}}%
\pgfpathlineto{\pgfqpoint{2.044188in}{1.337579in}}%
\pgfpathlineto{\pgfqpoint{2.044188in}{1.334629in}}%
\pgfpathmoveto{\pgfqpoint{2.039647in}{1.337579in}}%
\pgfpathlineto{\pgfqpoint{2.039647in}{1.337579in}}%
\pgfpathlineto{\pgfqpoint{2.039647in}{1.340528in}}%
\pgfpathlineto{\pgfqpoint{2.044188in}{1.340528in}}%
\pgfpathlineto{\pgfqpoint{2.044188in}{1.337579in}}%
\pgfpathmoveto{\pgfqpoint{2.039647in}{1.340528in}}%
\pgfpathlineto{\pgfqpoint{2.039647in}{1.340528in}}%
\pgfpathlineto{\pgfqpoint{2.039647in}{1.343477in}}%
\pgfpathlineto{\pgfqpoint{2.044188in}{1.343477in}}%
\pgfpathlineto{\pgfqpoint{2.044188in}{1.340528in}}%
\pgfpathmoveto{\pgfqpoint{2.044188in}{1.337579in}}%
\pgfpathlineto{\pgfqpoint{2.044188in}{1.337579in}}%
\pgfpathlineto{\pgfqpoint{2.044188in}{1.340528in}}%
\pgfpathlineto{\pgfqpoint{2.048728in}{1.340528in}}%
\pgfpathlineto{\pgfqpoint{2.048728in}{1.337579in}}%
\pgfpathmoveto{\pgfqpoint{2.044188in}{1.340528in}}%
\pgfpathlineto{\pgfqpoint{2.044188in}{1.340528in}}%
\pgfpathlineto{\pgfqpoint{2.044188in}{1.343477in}}%
\pgfpathlineto{\pgfqpoint{2.048728in}{1.343477in}}%
\pgfpathlineto{\pgfqpoint{2.048728in}{1.340528in}}%
\pgfpathmoveto{\pgfqpoint{2.048728in}{1.340528in}}%
\pgfpathlineto{\pgfqpoint{2.048728in}{1.340528in}}%
\pgfpathlineto{\pgfqpoint{2.048728in}{1.343477in}}%
\pgfpathlineto{\pgfqpoint{2.053269in}{1.343477in}}%
\pgfpathlineto{\pgfqpoint{2.053269in}{1.340528in}}%
\pgfpathmoveto{\pgfqpoint{2.048728in}{1.343477in}}%
\pgfpathlineto{\pgfqpoint{2.048728in}{1.343477in}}%
\pgfpathlineto{\pgfqpoint{2.048728in}{1.346426in}}%
\pgfpathlineto{\pgfqpoint{2.053269in}{1.346426in}}%
\pgfpathlineto{\pgfqpoint{2.053269in}{1.343477in}}%
\pgfpathmoveto{\pgfqpoint{2.048728in}{1.346426in}}%
\pgfpathlineto{\pgfqpoint{2.048728in}{1.346426in}}%
\pgfpathlineto{\pgfqpoint{2.048728in}{1.349376in}}%
\pgfpathlineto{\pgfqpoint{2.053269in}{1.349376in}}%
\pgfpathlineto{\pgfqpoint{2.053269in}{1.346426in}}%
\pgfpathmoveto{\pgfqpoint{2.053269in}{1.343477in}}%
\pgfpathlineto{\pgfqpoint{2.053269in}{1.343477in}}%
\pgfpathlineto{\pgfqpoint{2.053269in}{1.346426in}}%
\pgfpathlineto{\pgfqpoint{2.057810in}{1.346426in}}%
\pgfpathlineto{\pgfqpoint{2.057810in}{1.343477in}}%
\pgfpathmoveto{\pgfqpoint{2.053269in}{1.346426in}}%
\pgfpathlineto{\pgfqpoint{2.053269in}{1.346426in}}%
\pgfpathlineto{\pgfqpoint{2.053269in}{1.349376in}}%
\pgfpathlineto{\pgfqpoint{2.057810in}{1.349376in}}%
\pgfpathlineto{\pgfqpoint{2.057810in}{1.346426in}}%
\pgfpathmoveto{\pgfqpoint{2.057810in}{1.346426in}}%
\pgfpathlineto{\pgfqpoint{2.057810in}{1.346426in}}%
\pgfpathlineto{\pgfqpoint{2.057810in}{1.349376in}}%
\pgfpathlineto{\pgfqpoint{2.062351in}{1.349376in}}%
\pgfpathlineto{\pgfqpoint{2.062351in}{1.346426in}}%
\pgfpathmoveto{\pgfqpoint{2.057810in}{1.349376in}}%
\pgfpathlineto{\pgfqpoint{2.057810in}{1.349376in}}%
\pgfpathlineto{\pgfqpoint{2.057810in}{1.352325in}}%
\pgfpathlineto{\pgfqpoint{2.062351in}{1.352325in}}%
\pgfpathlineto{\pgfqpoint{2.062351in}{1.349376in}}%
\pgfpathmoveto{\pgfqpoint{2.057810in}{1.352325in}}%
\pgfpathlineto{\pgfqpoint{2.057810in}{1.352325in}}%
\pgfpathlineto{\pgfqpoint{2.057810in}{1.355274in}}%
\pgfpathlineto{\pgfqpoint{2.062351in}{1.355274in}}%
\pgfpathlineto{\pgfqpoint{2.062351in}{1.352325in}}%
\pgfpathmoveto{\pgfqpoint{2.062351in}{1.349376in}}%
\pgfpathlineto{\pgfqpoint{2.062351in}{1.349376in}}%
\pgfpathlineto{\pgfqpoint{2.062351in}{1.352325in}}%
\pgfpathlineto{\pgfqpoint{2.066892in}{1.352325in}}%
\pgfpathlineto{\pgfqpoint{2.066892in}{1.349376in}}%
\pgfpathmoveto{\pgfqpoint{2.062351in}{1.352325in}}%
\pgfpathlineto{\pgfqpoint{2.062351in}{1.352325in}}%
\pgfpathlineto{\pgfqpoint{2.062351in}{1.355274in}}%
\pgfpathlineto{\pgfqpoint{2.066892in}{1.355274in}}%
\pgfpathlineto{\pgfqpoint{2.066892in}{1.352325in}}%
\pgfpathmoveto{\pgfqpoint{2.066892in}{1.352325in}}%
\pgfpathlineto{\pgfqpoint{2.066892in}{1.352325in}}%
\pgfpathlineto{\pgfqpoint{2.066892in}{1.355274in}}%
\pgfpathlineto{\pgfqpoint{2.071433in}{1.355274in}}%
\pgfpathlineto{\pgfqpoint{2.071433in}{1.352325in}}%
\pgfpathmoveto{\pgfqpoint{2.066892in}{1.355274in}}%
\pgfpathlineto{\pgfqpoint{2.066892in}{1.355274in}}%
\pgfpathlineto{\pgfqpoint{2.066892in}{1.358223in}}%
\pgfpathlineto{\pgfqpoint{2.071433in}{1.358223in}}%
\pgfpathlineto{\pgfqpoint{2.071433in}{1.355274in}}%
\pgfpathmoveto{\pgfqpoint{2.066892in}{1.358223in}}%
\pgfpathlineto{\pgfqpoint{2.066892in}{1.358223in}}%
\pgfpathlineto{\pgfqpoint{2.066892in}{1.361173in}}%
\pgfpathlineto{\pgfqpoint{2.071433in}{1.361173in}}%
\pgfpathlineto{\pgfqpoint{2.071433in}{1.358223in}}%
\pgfpathmoveto{\pgfqpoint{2.071433in}{1.355274in}}%
\pgfpathlineto{\pgfqpoint{2.071433in}{1.355274in}}%
\pgfpathlineto{\pgfqpoint{2.071433in}{1.358223in}}%
\pgfpathlineto{\pgfqpoint{2.075974in}{1.358223in}}%
\pgfpathlineto{\pgfqpoint{2.075974in}{1.355274in}}%
\pgfpathmoveto{\pgfqpoint{2.071433in}{1.358223in}}%
\pgfpathlineto{\pgfqpoint{2.071433in}{1.358223in}}%
\pgfpathlineto{\pgfqpoint{2.071433in}{1.361173in}}%
\pgfpathlineto{\pgfqpoint{2.075974in}{1.361173in}}%
\pgfpathlineto{\pgfqpoint{2.075974in}{1.358223in}}%
\pgfpathmoveto{\pgfqpoint{2.075974in}{1.358223in}}%
\pgfpathlineto{\pgfqpoint{2.075974in}{1.358223in}}%
\pgfpathlineto{\pgfqpoint{2.075974in}{1.361173in}}%
\pgfpathlineto{\pgfqpoint{2.080515in}{1.361173in}}%
\pgfpathlineto{\pgfqpoint{2.080515in}{1.358223in}}%
\pgfpathmoveto{\pgfqpoint{2.075974in}{1.361173in}}%
\pgfpathlineto{\pgfqpoint{2.075974in}{1.361173in}}%
\pgfpathlineto{\pgfqpoint{2.075974in}{1.364122in}}%
\pgfpathlineto{\pgfqpoint{2.080515in}{1.364122in}}%
\pgfpathlineto{\pgfqpoint{2.080515in}{1.361173in}}%
\pgfpathmoveto{\pgfqpoint{2.075974in}{1.364122in}}%
\pgfpathlineto{\pgfqpoint{2.075974in}{1.364122in}}%
\pgfpathlineto{\pgfqpoint{2.075974in}{1.367071in}}%
\pgfpathlineto{\pgfqpoint{2.080515in}{1.367071in}}%
\pgfpathlineto{\pgfqpoint{2.080515in}{1.364122in}}%
\pgfpathmoveto{\pgfqpoint{2.080515in}{1.361173in}}%
\pgfpathlineto{\pgfqpoint{2.080515in}{1.361173in}}%
\pgfpathlineto{\pgfqpoint{2.080515in}{1.364122in}}%
\pgfpathlineto{\pgfqpoint{2.085056in}{1.364122in}}%
\pgfpathlineto{\pgfqpoint{2.085056in}{1.361173in}}%
\pgfpathmoveto{\pgfqpoint{2.080515in}{1.364122in}}%
\pgfpathlineto{\pgfqpoint{2.080515in}{1.364122in}}%
\pgfpathlineto{\pgfqpoint{2.080515in}{1.367071in}}%
\pgfpathlineto{\pgfqpoint{2.085056in}{1.367071in}}%
\pgfpathlineto{\pgfqpoint{2.085056in}{1.364122in}}%
\pgfpathmoveto{\pgfqpoint{2.085056in}{1.364122in}}%
\pgfpathlineto{\pgfqpoint{2.085056in}{1.364122in}}%
\pgfpathlineto{\pgfqpoint{2.085056in}{1.367071in}}%
\pgfpathlineto{\pgfqpoint{2.089597in}{1.367071in}}%
\pgfpathlineto{\pgfqpoint{2.089597in}{1.364122in}}%
\pgfpathmoveto{\pgfqpoint{2.085056in}{1.367071in}}%
\pgfpathlineto{\pgfqpoint{2.085056in}{1.367071in}}%
\pgfpathlineto{\pgfqpoint{2.085056in}{1.370020in}}%
\pgfpathlineto{\pgfqpoint{2.089597in}{1.370020in}}%
\pgfpathlineto{\pgfqpoint{2.089597in}{1.367071in}}%
\pgfpathmoveto{\pgfqpoint{2.085056in}{1.370020in}}%
\pgfpathlineto{\pgfqpoint{2.085056in}{1.370020in}}%
\pgfpathlineto{\pgfqpoint{2.085056in}{1.372970in}}%
\pgfpathlineto{\pgfqpoint{2.089597in}{1.372970in}}%
\pgfpathlineto{\pgfqpoint{2.089597in}{1.370020in}}%
\pgfpathmoveto{\pgfqpoint{2.089597in}{1.367071in}}%
\pgfpathlineto{\pgfqpoint{2.089597in}{1.367071in}}%
\pgfpathlineto{\pgfqpoint{2.089597in}{1.370020in}}%
\pgfpathlineto{\pgfqpoint{2.094138in}{1.370020in}}%
\pgfpathlineto{\pgfqpoint{2.094138in}{1.367071in}}%
\pgfpathmoveto{\pgfqpoint{2.089597in}{1.370020in}}%
\pgfpathlineto{\pgfqpoint{2.089597in}{1.370020in}}%
\pgfpathlineto{\pgfqpoint{2.089597in}{1.372970in}}%
\pgfpathlineto{\pgfqpoint{2.094138in}{1.372970in}}%
\pgfpathlineto{\pgfqpoint{2.094138in}{1.370020in}}%
\pgfpathmoveto{\pgfqpoint{2.094138in}{1.370020in}}%
\pgfpathlineto{\pgfqpoint{2.094138in}{1.370020in}}%
\pgfpathlineto{\pgfqpoint{2.094138in}{1.372970in}}%
\pgfpathlineto{\pgfqpoint{2.098679in}{1.372970in}}%
\pgfpathlineto{\pgfqpoint{2.098679in}{1.370020in}}%
\pgfpathmoveto{\pgfqpoint{2.094138in}{1.372970in}}%
\pgfpathlineto{\pgfqpoint{2.094138in}{1.372970in}}%
\pgfpathlineto{\pgfqpoint{2.094138in}{1.375919in}}%
\pgfpathlineto{\pgfqpoint{2.098679in}{1.375919in}}%
\pgfpathlineto{\pgfqpoint{2.098679in}{1.372970in}}%
\pgfpathmoveto{\pgfqpoint{2.094138in}{1.375919in}}%
\pgfpathlineto{\pgfqpoint{2.094138in}{1.375919in}}%
\pgfpathlineto{\pgfqpoint{2.094138in}{1.378868in}}%
\pgfpathlineto{\pgfqpoint{2.098679in}{1.378868in}}%
\pgfpathlineto{\pgfqpoint{2.098679in}{1.375919in}}%
\pgfpathmoveto{\pgfqpoint{2.098679in}{1.372970in}}%
\pgfpathlineto{\pgfqpoint{2.098679in}{1.372970in}}%
\pgfpathlineto{\pgfqpoint{2.098679in}{1.375919in}}%
\pgfpathlineto{\pgfqpoint{2.103220in}{1.375919in}}%
\pgfpathlineto{\pgfqpoint{2.103220in}{1.372970in}}%
\pgfpathmoveto{\pgfqpoint{2.098679in}{1.375919in}}%
\pgfpathlineto{\pgfqpoint{2.098679in}{1.375919in}}%
\pgfpathlineto{\pgfqpoint{2.098679in}{1.378868in}}%
\pgfpathlineto{\pgfqpoint{2.103220in}{1.378868in}}%
\pgfpathlineto{\pgfqpoint{2.103220in}{1.375919in}}%
\pgfpathmoveto{\pgfqpoint{2.103220in}{1.375919in}}%
\pgfpathlineto{\pgfqpoint{2.103220in}{1.375919in}}%
\pgfpathlineto{\pgfqpoint{2.103220in}{1.378868in}}%
\pgfpathlineto{\pgfqpoint{2.107761in}{1.378868in}}%
\pgfpathlineto{\pgfqpoint{2.107761in}{1.375919in}}%
\pgfpathmoveto{\pgfqpoint{2.103220in}{1.378868in}}%
\pgfpathlineto{\pgfqpoint{2.103220in}{1.378868in}}%
\pgfpathlineto{\pgfqpoint{2.103220in}{1.381818in}}%
\pgfpathlineto{\pgfqpoint{2.107761in}{1.381818in}}%
\pgfpathlineto{\pgfqpoint{2.107761in}{1.378868in}}%
\pgfpathmoveto{\pgfqpoint{2.103220in}{1.381818in}}%
\pgfpathlineto{\pgfqpoint{2.103220in}{1.381818in}}%
\pgfpathlineto{\pgfqpoint{2.103220in}{1.384767in}}%
\pgfpathlineto{\pgfqpoint{2.107761in}{1.384767in}}%
\pgfpathlineto{\pgfqpoint{2.107761in}{1.381818in}}%
\pgfpathmoveto{\pgfqpoint{2.107761in}{1.378868in}}%
\pgfpathlineto{\pgfqpoint{2.107761in}{1.378868in}}%
\pgfpathlineto{\pgfqpoint{2.107761in}{1.381818in}}%
\pgfpathlineto{\pgfqpoint{2.112302in}{1.381818in}}%
\pgfpathlineto{\pgfqpoint{2.112302in}{1.378868in}}%
\pgfpathmoveto{\pgfqpoint{2.107761in}{1.381818in}}%
\pgfpathlineto{\pgfqpoint{2.107761in}{1.381818in}}%
\pgfpathlineto{\pgfqpoint{2.107761in}{1.384767in}}%
\pgfpathlineto{\pgfqpoint{2.112302in}{1.384767in}}%
\pgfpathlineto{\pgfqpoint{2.112302in}{1.381818in}}%
\pgfpathmoveto{\pgfqpoint{2.112302in}{1.381818in}}%
\pgfpathlineto{\pgfqpoint{2.112302in}{1.381818in}}%
\pgfpathlineto{\pgfqpoint{2.112302in}{1.384767in}}%
\pgfpathlineto{\pgfqpoint{2.116843in}{1.384767in}}%
\pgfpathlineto{\pgfqpoint{2.116843in}{1.381818in}}%
\pgfpathmoveto{\pgfqpoint{2.112302in}{1.384767in}}%
\pgfpathlineto{\pgfqpoint{2.112302in}{1.384767in}}%
\pgfpathlineto{\pgfqpoint{2.112302in}{1.387716in}}%
\pgfpathlineto{\pgfqpoint{2.116843in}{1.387716in}}%
\pgfpathlineto{\pgfqpoint{2.116843in}{1.384767in}}%
\pgfpathmoveto{\pgfqpoint{2.112302in}{1.387716in}}%
\pgfpathlineto{\pgfqpoint{2.112302in}{1.387716in}}%
\pgfpathlineto{\pgfqpoint{2.112302in}{1.390665in}}%
\pgfpathlineto{\pgfqpoint{2.116843in}{1.390665in}}%
\pgfpathlineto{\pgfqpoint{2.116843in}{1.387716in}}%
\pgfpathmoveto{\pgfqpoint{2.116843in}{1.384767in}}%
\pgfpathlineto{\pgfqpoint{2.116843in}{1.384767in}}%
\pgfpathlineto{\pgfqpoint{2.116843in}{1.387716in}}%
\pgfpathlineto{\pgfqpoint{2.121384in}{1.387716in}}%
\pgfpathlineto{\pgfqpoint{2.121384in}{1.384767in}}%
\pgfpathmoveto{\pgfqpoint{2.116843in}{1.387716in}}%
\pgfpathlineto{\pgfqpoint{2.116843in}{1.387716in}}%
\pgfpathlineto{\pgfqpoint{2.116843in}{1.390665in}}%
\pgfpathlineto{\pgfqpoint{2.121384in}{1.390665in}}%
\pgfpathlineto{\pgfqpoint{2.121384in}{1.387716in}}%
\pgfpathmoveto{\pgfqpoint{2.121384in}{1.387716in}}%
\pgfpathlineto{\pgfqpoint{2.121384in}{1.387716in}}%
\pgfpathlineto{\pgfqpoint{2.121384in}{1.390665in}}%
\pgfpathlineto{\pgfqpoint{2.125925in}{1.390665in}}%
\pgfpathlineto{\pgfqpoint{2.125925in}{1.387716in}}%
\pgfpathmoveto{\pgfqpoint{2.121384in}{1.390665in}}%
\pgfpathlineto{\pgfqpoint{2.121384in}{1.390665in}}%
\pgfpathlineto{\pgfqpoint{2.121384in}{1.393615in}}%
\pgfpathlineto{\pgfqpoint{2.125925in}{1.393615in}}%
\pgfpathlineto{\pgfqpoint{2.125925in}{1.390665in}}%
\pgfpathmoveto{\pgfqpoint{2.121384in}{1.393615in}}%
\pgfpathlineto{\pgfqpoint{2.121384in}{1.393615in}}%
\pgfpathlineto{\pgfqpoint{2.121384in}{1.396564in}}%
\pgfpathlineto{\pgfqpoint{2.125925in}{1.396564in}}%
\pgfpathlineto{\pgfqpoint{2.125925in}{1.393615in}}%
\pgfpathmoveto{\pgfqpoint{2.125925in}{1.390665in}}%
\pgfpathlineto{\pgfqpoint{2.125925in}{1.390665in}}%
\pgfpathlineto{\pgfqpoint{2.125925in}{1.393615in}}%
\pgfpathlineto{\pgfqpoint{2.130466in}{1.393615in}}%
\pgfpathlineto{\pgfqpoint{2.130466in}{1.390665in}}%
\pgfpathmoveto{\pgfqpoint{2.125925in}{1.393615in}}%
\pgfpathlineto{\pgfqpoint{2.125925in}{1.393615in}}%
\pgfpathlineto{\pgfqpoint{2.125925in}{1.396564in}}%
\pgfpathlineto{\pgfqpoint{2.130466in}{1.396564in}}%
\pgfpathlineto{\pgfqpoint{2.130466in}{1.393615in}}%
\pgfpathmoveto{\pgfqpoint{2.130466in}{1.393615in}}%
\pgfpathlineto{\pgfqpoint{2.130466in}{1.393615in}}%
\pgfpathlineto{\pgfqpoint{2.130466in}{1.396564in}}%
\pgfpathlineto{\pgfqpoint{2.135007in}{1.396564in}}%
\pgfpathlineto{\pgfqpoint{2.135007in}{1.393615in}}%
\pgfpathmoveto{\pgfqpoint{2.130466in}{1.396564in}}%
\pgfpathlineto{\pgfqpoint{2.130466in}{1.396564in}}%
\pgfpathlineto{\pgfqpoint{2.130466in}{1.399513in}}%
\pgfpathlineto{\pgfqpoint{2.135007in}{1.399513in}}%
\pgfpathlineto{\pgfqpoint{2.135007in}{1.396564in}}%
\pgfpathmoveto{\pgfqpoint{2.130466in}{1.399513in}}%
\pgfpathlineto{\pgfqpoint{2.130466in}{1.399513in}}%
\pgfpathlineto{\pgfqpoint{2.130466in}{1.402462in}}%
\pgfpathlineto{\pgfqpoint{2.135007in}{1.402462in}}%
\pgfpathlineto{\pgfqpoint{2.135007in}{1.399513in}}%
\pgfpathmoveto{\pgfqpoint{2.135007in}{1.396564in}}%
\pgfpathlineto{\pgfqpoint{2.135007in}{1.396564in}}%
\pgfpathlineto{\pgfqpoint{2.135007in}{1.399513in}}%
\pgfpathlineto{\pgfqpoint{2.139547in}{1.399513in}}%
\pgfpathlineto{\pgfqpoint{2.139547in}{1.396564in}}%
\pgfpathmoveto{\pgfqpoint{2.135007in}{1.399513in}}%
\pgfpathlineto{\pgfqpoint{2.135007in}{1.399513in}}%
\pgfpathlineto{\pgfqpoint{2.135007in}{1.402462in}}%
\pgfpathlineto{\pgfqpoint{2.139547in}{1.402462in}}%
\pgfpathlineto{\pgfqpoint{2.139547in}{1.399513in}}%
\pgfpathmoveto{\pgfqpoint{2.139547in}{1.399513in}}%
\pgfpathlineto{\pgfqpoint{2.139547in}{1.399513in}}%
\pgfpathlineto{\pgfqpoint{2.139547in}{1.402462in}}%
\pgfpathlineto{\pgfqpoint{2.144088in}{1.402462in}}%
\pgfpathlineto{\pgfqpoint{2.144088in}{1.399513in}}%
\pgfpathmoveto{\pgfqpoint{2.139547in}{1.402462in}}%
\pgfpathlineto{\pgfqpoint{2.139547in}{1.402462in}}%
\pgfpathlineto{\pgfqpoint{2.139547in}{1.405412in}}%
\pgfpathlineto{\pgfqpoint{2.144088in}{1.405412in}}%
\pgfpathlineto{\pgfqpoint{2.144088in}{1.402462in}}%
\pgfpathmoveto{\pgfqpoint{2.139547in}{1.405412in}}%
\pgfpathlineto{\pgfqpoint{2.139547in}{1.405412in}}%
\pgfpathlineto{\pgfqpoint{2.139547in}{1.408361in}}%
\pgfpathlineto{\pgfqpoint{2.144088in}{1.408361in}}%
\pgfpathlineto{\pgfqpoint{2.144088in}{1.405412in}}%
\pgfpathmoveto{\pgfqpoint{2.144088in}{1.402462in}}%
\pgfpathlineto{\pgfqpoint{2.144088in}{1.402462in}}%
\pgfpathlineto{\pgfqpoint{2.144088in}{1.405412in}}%
\pgfpathlineto{\pgfqpoint{2.148629in}{1.405412in}}%
\pgfpathlineto{\pgfqpoint{2.148629in}{1.402462in}}%
\pgfpathmoveto{\pgfqpoint{2.144088in}{1.405412in}}%
\pgfpathlineto{\pgfqpoint{2.144088in}{1.405412in}}%
\pgfpathlineto{\pgfqpoint{2.144088in}{1.408361in}}%
\pgfpathlineto{\pgfqpoint{2.148629in}{1.408361in}}%
\pgfpathlineto{\pgfqpoint{2.148629in}{1.405412in}}%
\pgfpathmoveto{\pgfqpoint{2.148629in}{1.405412in}}%
\pgfpathlineto{\pgfqpoint{2.148629in}{1.405412in}}%
\pgfpathlineto{\pgfqpoint{2.148629in}{1.408361in}}%
\pgfpathlineto{\pgfqpoint{2.153170in}{1.408361in}}%
\pgfpathlineto{\pgfqpoint{2.153170in}{1.405412in}}%
\pgfpathmoveto{\pgfqpoint{2.148629in}{1.408361in}}%
\pgfpathlineto{\pgfqpoint{2.148629in}{1.408361in}}%
\pgfpathlineto{\pgfqpoint{2.148629in}{1.411310in}}%
\pgfpathlineto{\pgfqpoint{2.153170in}{1.411310in}}%
\pgfpathlineto{\pgfqpoint{2.153170in}{1.408361in}}%
\pgfpathmoveto{\pgfqpoint{2.148629in}{1.411310in}}%
\pgfpathlineto{\pgfqpoint{2.148629in}{1.411310in}}%
\pgfpathlineto{\pgfqpoint{2.148629in}{1.414260in}}%
\pgfpathlineto{\pgfqpoint{2.153170in}{1.414260in}}%
\pgfpathlineto{\pgfqpoint{2.153170in}{1.411310in}}%
\pgfpathmoveto{\pgfqpoint{2.153170in}{1.408361in}}%
\pgfpathlineto{\pgfqpoint{2.153170in}{1.408361in}}%
\pgfpathlineto{\pgfqpoint{2.153170in}{1.411310in}}%
\pgfpathlineto{\pgfqpoint{2.157711in}{1.411310in}}%
\pgfpathlineto{\pgfqpoint{2.157711in}{1.408361in}}%
\pgfpathmoveto{\pgfqpoint{2.153170in}{1.411310in}}%
\pgfpathlineto{\pgfqpoint{2.153170in}{1.411310in}}%
\pgfpathlineto{\pgfqpoint{2.153170in}{1.414260in}}%
\pgfpathlineto{\pgfqpoint{2.157711in}{1.414260in}}%
\pgfpathlineto{\pgfqpoint{2.157711in}{1.411310in}}%
\pgfpathmoveto{\pgfqpoint{2.157711in}{1.411310in}}%
\pgfpathlineto{\pgfqpoint{2.157711in}{1.411310in}}%
\pgfpathlineto{\pgfqpoint{2.157711in}{1.414260in}}%
\pgfpathlineto{\pgfqpoint{2.162252in}{1.414260in}}%
\pgfpathlineto{\pgfqpoint{2.162252in}{1.411310in}}%
\pgfpathmoveto{\pgfqpoint{2.157711in}{1.414260in}}%
\pgfpathlineto{\pgfqpoint{2.157711in}{1.414260in}}%
\pgfpathlineto{\pgfqpoint{2.157711in}{1.417209in}}%
\pgfpathlineto{\pgfqpoint{2.162252in}{1.417209in}}%
\pgfpathlineto{\pgfqpoint{2.162252in}{1.414260in}}%
\pgfpathmoveto{\pgfqpoint{2.157711in}{1.417209in}}%
\pgfpathlineto{\pgfqpoint{2.157711in}{1.417209in}}%
\pgfpathlineto{\pgfqpoint{2.157711in}{1.420158in}}%
\pgfpathlineto{\pgfqpoint{2.162252in}{1.420158in}}%
\pgfpathlineto{\pgfqpoint{2.162252in}{1.417209in}}%
\pgfpathmoveto{\pgfqpoint{2.162252in}{1.414260in}}%
\pgfpathlineto{\pgfqpoint{2.162252in}{1.414260in}}%
\pgfpathlineto{\pgfqpoint{2.162252in}{1.417209in}}%
\pgfpathlineto{\pgfqpoint{2.166793in}{1.417209in}}%
\pgfpathlineto{\pgfqpoint{2.166793in}{1.414260in}}%
\pgfpathmoveto{\pgfqpoint{2.162252in}{1.417209in}}%
\pgfpathlineto{\pgfqpoint{2.162252in}{1.417209in}}%
\pgfpathlineto{\pgfqpoint{2.162252in}{1.420158in}}%
\pgfpathlineto{\pgfqpoint{2.166793in}{1.420158in}}%
\pgfpathlineto{\pgfqpoint{2.166793in}{1.417209in}}%
\pgfpathmoveto{\pgfqpoint{2.166793in}{1.417209in}}%
\pgfpathlineto{\pgfqpoint{2.166793in}{1.417209in}}%
\pgfpathlineto{\pgfqpoint{2.166793in}{1.420158in}}%
\pgfpathlineto{\pgfqpoint{2.171334in}{1.420158in}}%
\pgfpathlineto{\pgfqpoint{2.171334in}{1.417209in}}%
\pgfpathmoveto{\pgfqpoint{2.166793in}{1.420158in}}%
\pgfpathlineto{\pgfqpoint{2.166793in}{1.420158in}}%
\pgfpathlineto{\pgfqpoint{2.166793in}{1.423107in}}%
\pgfpathlineto{\pgfqpoint{2.171334in}{1.423107in}}%
\pgfpathlineto{\pgfqpoint{2.171334in}{1.420158in}}%
\pgfpathmoveto{\pgfqpoint{2.166793in}{1.423107in}}%
\pgfpathlineto{\pgfqpoint{2.166793in}{1.423107in}}%
\pgfpathlineto{\pgfqpoint{2.166793in}{1.426057in}}%
\pgfpathlineto{\pgfqpoint{2.171334in}{1.426057in}}%
\pgfpathlineto{\pgfqpoint{2.171334in}{1.423107in}}%
\pgfpathmoveto{\pgfqpoint{2.171334in}{1.420158in}}%
\pgfpathlineto{\pgfqpoint{2.171334in}{1.420158in}}%
\pgfpathlineto{\pgfqpoint{2.171334in}{1.423107in}}%
\pgfpathlineto{\pgfqpoint{2.175875in}{1.423107in}}%
\pgfpathlineto{\pgfqpoint{2.175875in}{1.420158in}}%
\pgfpathmoveto{\pgfqpoint{2.171334in}{1.423107in}}%
\pgfpathlineto{\pgfqpoint{2.171334in}{1.423107in}}%
\pgfpathlineto{\pgfqpoint{2.171334in}{1.426057in}}%
\pgfpathlineto{\pgfqpoint{2.175875in}{1.426057in}}%
\pgfpathlineto{\pgfqpoint{2.175875in}{1.423107in}}%
\pgfpathmoveto{\pgfqpoint{2.175875in}{1.423107in}}%
\pgfpathlineto{\pgfqpoint{2.175875in}{1.423107in}}%
\pgfpathlineto{\pgfqpoint{2.175875in}{1.426057in}}%
\pgfpathlineto{\pgfqpoint{2.180416in}{1.426057in}}%
\pgfpathlineto{\pgfqpoint{2.180416in}{1.423107in}}%
\pgfpathmoveto{\pgfqpoint{2.175875in}{1.426057in}}%
\pgfpathlineto{\pgfqpoint{2.175875in}{1.426057in}}%
\pgfpathlineto{\pgfqpoint{2.175875in}{1.429006in}}%
\pgfpathlineto{\pgfqpoint{2.180416in}{1.429006in}}%
\pgfpathlineto{\pgfqpoint{2.180416in}{1.426057in}}%
\pgfpathmoveto{\pgfqpoint{2.175875in}{1.429006in}}%
\pgfpathlineto{\pgfqpoint{2.175875in}{1.429006in}}%
\pgfpathlineto{\pgfqpoint{2.175875in}{1.431955in}}%
\pgfpathlineto{\pgfqpoint{2.180416in}{1.431955in}}%
\pgfpathlineto{\pgfqpoint{2.180416in}{1.429006in}}%
\pgfpathmoveto{\pgfqpoint{2.180416in}{1.426057in}}%
\pgfpathlineto{\pgfqpoint{2.180416in}{1.426057in}}%
\pgfpathlineto{\pgfqpoint{2.180416in}{1.429006in}}%
\pgfpathlineto{\pgfqpoint{2.184957in}{1.429006in}}%
\pgfpathlineto{\pgfqpoint{2.184957in}{1.426057in}}%
\pgfpathmoveto{\pgfqpoint{2.180416in}{1.429006in}}%
\pgfpathlineto{\pgfqpoint{2.180416in}{1.429006in}}%
\pgfpathlineto{\pgfqpoint{2.180416in}{1.431955in}}%
\pgfpathlineto{\pgfqpoint{2.184957in}{1.431955in}}%
\pgfpathlineto{\pgfqpoint{2.184957in}{1.429006in}}%
\pgfpathmoveto{\pgfqpoint{2.184957in}{1.429006in}}%
\pgfpathlineto{\pgfqpoint{2.184957in}{1.429006in}}%
\pgfpathlineto{\pgfqpoint{2.184957in}{1.431955in}}%
\pgfpathlineto{\pgfqpoint{2.189498in}{1.431955in}}%
\pgfpathlineto{\pgfqpoint{2.189498in}{1.429006in}}%
\pgfpathmoveto{\pgfqpoint{2.184957in}{1.431955in}}%
\pgfpathlineto{\pgfqpoint{2.184957in}{1.431955in}}%
\pgfpathlineto{\pgfqpoint{2.184957in}{1.434904in}}%
\pgfpathlineto{\pgfqpoint{2.189498in}{1.434904in}}%
\pgfpathlineto{\pgfqpoint{2.189498in}{1.431955in}}%
\pgfpathmoveto{\pgfqpoint{2.184957in}{1.434904in}}%
\pgfpathlineto{\pgfqpoint{2.184957in}{1.434904in}}%
\pgfpathlineto{\pgfqpoint{2.184957in}{1.437854in}}%
\pgfpathlineto{\pgfqpoint{2.189498in}{1.437854in}}%
\pgfpathlineto{\pgfqpoint{2.189498in}{1.434904in}}%
\pgfpathmoveto{\pgfqpoint{2.189498in}{1.431955in}}%
\pgfpathlineto{\pgfqpoint{2.189498in}{1.431955in}}%
\pgfpathlineto{\pgfqpoint{2.189498in}{1.434904in}}%
\pgfpathlineto{\pgfqpoint{2.194039in}{1.434904in}}%
\pgfpathlineto{\pgfqpoint{2.194039in}{1.431955in}}%
\pgfpathmoveto{\pgfqpoint{2.189498in}{1.434904in}}%
\pgfpathlineto{\pgfqpoint{2.189498in}{1.434904in}}%
\pgfpathlineto{\pgfqpoint{2.189498in}{1.437854in}}%
\pgfpathlineto{\pgfqpoint{2.194039in}{1.437854in}}%
\pgfpathlineto{\pgfqpoint{2.194039in}{1.434904in}}%
\pgfpathmoveto{\pgfqpoint{2.194039in}{1.434904in}}%
\pgfpathlineto{\pgfqpoint{2.194039in}{1.434904in}}%
\pgfpathlineto{\pgfqpoint{2.194039in}{1.437854in}}%
\pgfpathlineto{\pgfqpoint{2.198580in}{1.437854in}}%
\pgfpathlineto{\pgfqpoint{2.198580in}{1.434904in}}%
\pgfpathmoveto{\pgfqpoint{2.194039in}{1.437854in}}%
\pgfpathlineto{\pgfqpoint{2.194039in}{1.437854in}}%
\pgfpathlineto{\pgfqpoint{2.194039in}{1.440803in}}%
\pgfpathlineto{\pgfqpoint{2.198580in}{1.440803in}}%
\pgfpathlineto{\pgfqpoint{2.198580in}{1.437854in}}%
\pgfpathmoveto{\pgfqpoint{2.194039in}{1.440803in}}%
\pgfpathlineto{\pgfqpoint{2.194039in}{1.440803in}}%
\pgfpathlineto{\pgfqpoint{2.194039in}{1.443752in}}%
\pgfpathlineto{\pgfqpoint{2.198580in}{1.443752in}}%
\pgfpathlineto{\pgfqpoint{2.198580in}{1.440803in}}%
\pgfpathmoveto{\pgfqpoint{2.198580in}{1.437854in}}%
\pgfpathlineto{\pgfqpoint{2.198580in}{1.437854in}}%
\pgfpathlineto{\pgfqpoint{2.198580in}{1.440803in}}%
\pgfpathlineto{\pgfqpoint{2.203121in}{1.440803in}}%
\pgfpathlineto{\pgfqpoint{2.203121in}{1.437854in}}%
\pgfpathmoveto{\pgfqpoint{2.198580in}{1.440803in}}%
\pgfpathlineto{\pgfqpoint{2.198580in}{1.440803in}}%
\pgfpathlineto{\pgfqpoint{2.198580in}{1.443752in}}%
\pgfpathlineto{\pgfqpoint{2.203121in}{1.443752in}}%
\pgfpathlineto{\pgfqpoint{2.203121in}{1.440803in}}%
\pgfpathmoveto{\pgfqpoint{2.203121in}{1.440803in}}%
\pgfpathlineto{\pgfqpoint{2.203121in}{1.440803in}}%
\pgfpathlineto{\pgfqpoint{2.203121in}{1.443752in}}%
\pgfpathlineto{\pgfqpoint{2.207662in}{1.443752in}}%
\pgfpathlineto{\pgfqpoint{2.207662in}{1.440803in}}%
\pgfpathmoveto{\pgfqpoint{2.203121in}{1.443752in}}%
\pgfpathlineto{\pgfqpoint{2.203121in}{1.443752in}}%
\pgfpathlineto{\pgfqpoint{2.203121in}{1.446701in}}%
\pgfpathlineto{\pgfqpoint{2.207662in}{1.446701in}}%
\pgfpathlineto{\pgfqpoint{2.207662in}{1.443752in}}%
\pgfpathmoveto{\pgfqpoint{2.203121in}{1.446701in}}%
\pgfpathlineto{\pgfqpoint{2.203121in}{1.446701in}}%
\pgfpathlineto{\pgfqpoint{2.203121in}{1.449651in}}%
\pgfpathlineto{\pgfqpoint{2.207662in}{1.449651in}}%
\pgfpathlineto{\pgfqpoint{2.207662in}{1.446701in}}%
\pgfpathmoveto{\pgfqpoint{2.207662in}{1.443752in}}%
\pgfpathlineto{\pgfqpoint{2.207662in}{1.443752in}}%
\pgfpathlineto{\pgfqpoint{2.207662in}{1.446701in}}%
\pgfpathlineto{\pgfqpoint{2.212203in}{1.446701in}}%
\pgfpathlineto{\pgfqpoint{2.212203in}{1.443752in}}%
\pgfpathmoveto{\pgfqpoint{2.207662in}{1.446701in}}%
\pgfpathlineto{\pgfqpoint{2.207662in}{1.446701in}}%
\pgfpathlineto{\pgfqpoint{2.207662in}{1.449651in}}%
\pgfpathlineto{\pgfqpoint{2.212203in}{1.449651in}}%
\pgfpathlineto{\pgfqpoint{2.212203in}{1.446701in}}%
\pgfpathmoveto{\pgfqpoint{2.212203in}{1.446701in}}%
\pgfpathlineto{\pgfqpoint{2.212203in}{1.446701in}}%
\pgfpathlineto{\pgfqpoint{2.212203in}{1.449651in}}%
\pgfpathlineto{\pgfqpoint{2.216745in}{1.449651in}}%
\pgfpathlineto{\pgfqpoint{2.216745in}{1.446701in}}%
\pgfpathmoveto{\pgfqpoint{2.212203in}{1.449651in}}%
\pgfpathlineto{\pgfqpoint{2.212203in}{1.449651in}}%
\pgfpathlineto{\pgfqpoint{2.212203in}{1.452600in}}%
\pgfpathlineto{\pgfqpoint{2.216745in}{1.452600in}}%
\pgfpathlineto{\pgfqpoint{2.216745in}{1.449651in}}%
\pgfpathmoveto{\pgfqpoint{2.212203in}{1.452600in}}%
\pgfpathlineto{\pgfqpoint{2.212203in}{1.452600in}}%
\pgfpathlineto{\pgfqpoint{2.212203in}{1.455549in}}%
\pgfpathlineto{\pgfqpoint{2.216745in}{1.455549in}}%
\pgfpathlineto{\pgfqpoint{2.216745in}{1.452600in}}%
\pgfpathmoveto{\pgfqpoint{2.216745in}{1.449651in}}%
\pgfpathlineto{\pgfqpoint{2.216745in}{1.449651in}}%
\pgfpathlineto{\pgfqpoint{2.216745in}{1.452600in}}%
\pgfpathlineto{\pgfqpoint{2.221286in}{1.452600in}}%
\pgfpathlineto{\pgfqpoint{2.221286in}{1.449651in}}%
\pgfpathmoveto{\pgfqpoint{2.216745in}{1.452600in}}%
\pgfpathlineto{\pgfqpoint{2.216745in}{1.452600in}}%
\pgfpathlineto{\pgfqpoint{2.216745in}{1.455549in}}%
\pgfpathlineto{\pgfqpoint{2.221286in}{1.455549in}}%
\pgfpathlineto{\pgfqpoint{2.221286in}{1.452600in}}%
\pgfpathmoveto{\pgfqpoint{2.221286in}{1.452600in}}%
\pgfpathlineto{\pgfqpoint{2.221286in}{1.452600in}}%
\pgfpathlineto{\pgfqpoint{2.221286in}{1.455549in}}%
\pgfpathlineto{\pgfqpoint{2.225827in}{1.455549in}}%
\pgfpathlineto{\pgfqpoint{2.225827in}{1.452600in}}%
\pgfpathmoveto{\pgfqpoint{2.221286in}{1.455549in}}%
\pgfpathlineto{\pgfqpoint{2.221286in}{1.455549in}}%
\pgfpathlineto{\pgfqpoint{2.221286in}{1.458498in}}%
\pgfpathlineto{\pgfqpoint{2.225827in}{1.458498in}}%
\pgfpathlineto{\pgfqpoint{2.225827in}{1.455549in}}%
\pgfpathmoveto{\pgfqpoint{2.221286in}{1.458498in}}%
\pgfpathlineto{\pgfqpoint{2.221286in}{1.458498in}}%
\pgfpathlineto{\pgfqpoint{2.221286in}{1.461447in}}%
\pgfpathlineto{\pgfqpoint{2.225827in}{1.461447in}}%
\pgfpathlineto{\pgfqpoint{2.225827in}{1.458498in}}%
\pgfpathmoveto{\pgfqpoint{2.225827in}{1.455549in}}%
\pgfpathlineto{\pgfqpoint{2.225827in}{1.455549in}}%
\pgfpathlineto{\pgfqpoint{2.225827in}{1.458498in}}%
\pgfpathlineto{\pgfqpoint{2.230368in}{1.458498in}}%
\pgfpathlineto{\pgfqpoint{2.230368in}{1.455549in}}%
\pgfpathmoveto{\pgfqpoint{2.225827in}{1.458498in}}%
\pgfpathlineto{\pgfqpoint{2.225827in}{1.458498in}}%
\pgfpathlineto{\pgfqpoint{2.225827in}{1.461447in}}%
\pgfpathlineto{\pgfqpoint{2.230368in}{1.461447in}}%
\pgfpathlineto{\pgfqpoint{2.230368in}{1.458498in}}%
\pgfpathmoveto{\pgfqpoint{2.230368in}{1.458498in}}%
\pgfpathlineto{\pgfqpoint{2.230368in}{1.458498in}}%
\pgfpathlineto{\pgfqpoint{2.230368in}{1.461447in}}%
\pgfpathlineto{\pgfqpoint{2.234909in}{1.461447in}}%
\pgfpathlineto{\pgfqpoint{2.234909in}{1.458498in}}%
\pgfpathmoveto{\pgfqpoint{2.230368in}{1.461447in}}%
\pgfpathlineto{\pgfqpoint{2.230368in}{1.461447in}}%
\pgfpathlineto{\pgfqpoint{2.230368in}{1.464396in}}%
\pgfpathlineto{\pgfqpoint{2.234909in}{1.464396in}}%
\pgfpathlineto{\pgfqpoint{2.234909in}{1.461447in}}%
\pgfpathmoveto{\pgfqpoint{2.230368in}{1.464396in}}%
\pgfpathlineto{\pgfqpoint{2.230368in}{1.464396in}}%
\pgfpathlineto{\pgfqpoint{2.230368in}{1.467345in}}%
\pgfpathlineto{\pgfqpoint{2.234909in}{1.467345in}}%
\pgfpathlineto{\pgfqpoint{2.234909in}{1.464396in}}%
\pgfpathmoveto{\pgfqpoint{2.234909in}{1.461447in}}%
\pgfpathlineto{\pgfqpoint{2.234909in}{1.461447in}}%
\pgfpathlineto{\pgfqpoint{2.234909in}{1.464396in}}%
\pgfpathlineto{\pgfqpoint{2.239451in}{1.464396in}}%
\pgfpathlineto{\pgfqpoint{2.239451in}{1.461447in}}%
\pgfpathmoveto{\pgfqpoint{2.234909in}{1.464396in}}%
\pgfpathlineto{\pgfqpoint{2.234909in}{1.464396in}}%
\pgfpathlineto{\pgfqpoint{2.234909in}{1.467345in}}%
\pgfpathlineto{\pgfqpoint{2.239451in}{1.467345in}}%
\pgfpathlineto{\pgfqpoint{2.239451in}{1.464396in}}%
\pgfpathmoveto{\pgfqpoint{2.239451in}{1.464396in}}%
\pgfpathlineto{\pgfqpoint{2.239451in}{1.464396in}}%
\pgfpathlineto{\pgfqpoint{2.239451in}{1.467345in}}%
\pgfpathlineto{\pgfqpoint{2.243992in}{1.467345in}}%
\pgfpathlineto{\pgfqpoint{2.243992in}{1.464396in}}%
\pgfpathmoveto{\pgfqpoint{2.239451in}{1.467345in}}%
\pgfpathlineto{\pgfqpoint{2.239451in}{1.467345in}}%
\pgfpathlineto{\pgfqpoint{2.239451in}{1.470294in}}%
\pgfpathlineto{\pgfqpoint{2.243992in}{1.470294in}}%
\pgfpathlineto{\pgfqpoint{2.243992in}{1.467345in}}%
\pgfpathmoveto{\pgfqpoint{2.239451in}{1.470294in}}%
\pgfpathlineto{\pgfqpoint{2.239451in}{1.470294in}}%
\pgfpathlineto{\pgfqpoint{2.239451in}{1.473244in}}%
\pgfpathlineto{\pgfqpoint{2.243992in}{1.473244in}}%
\pgfpathlineto{\pgfqpoint{2.243992in}{1.470294in}}%
\pgfpathmoveto{\pgfqpoint{2.243992in}{1.467345in}}%
\pgfpathlineto{\pgfqpoint{2.243992in}{1.467345in}}%
\pgfpathlineto{\pgfqpoint{2.243992in}{1.470294in}}%
\pgfpathlineto{\pgfqpoint{2.248533in}{1.470294in}}%
\pgfpathlineto{\pgfqpoint{2.248533in}{1.467345in}}%
\pgfpathmoveto{\pgfqpoint{2.243992in}{1.470294in}}%
\pgfpathlineto{\pgfqpoint{2.243992in}{1.470294in}}%
\pgfpathlineto{\pgfqpoint{2.243992in}{1.473244in}}%
\pgfpathlineto{\pgfqpoint{2.248533in}{1.473244in}}%
\pgfpathlineto{\pgfqpoint{2.248533in}{1.470294in}}%
\pgfpathmoveto{\pgfqpoint{2.248533in}{1.470294in}}%
\pgfpathlineto{\pgfqpoint{2.248533in}{1.470294in}}%
\pgfpathlineto{\pgfqpoint{2.248533in}{1.473244in}}%
\pgfpathlineto{\pgfqpoint{2.253074in}{1.473244in}}%
\pgfpathlineto{\pgfqpoint{2.253074in}{1.470294in}}%
\pgfpathmoveto{\pgfqpoint{2.248533in}{1.473244in}}%
\pgfpathlineto{\pgfqpoint{2.248533in}{1.473244in}}%
\pgfpathlineto{\pgfqpoint{2.248533in}{1.476193in}}%
\pgfpathlineto{\pgfqpoint{2.253074in}{1.476193in}}%
\pgfpathlineto{\pgfqpoint{2.253074in}{1.473244in}}%
\pgfpathmoveto{\pgfqpoint{2.248533in}{1.476193in}}%
\pgfpathlineto{\pgfqpoint{2.248533in}{1.476193in}}%
\pgfpathlineto{\pgfqpoint{2.248533in}{1.479142in}}%
\pgfpathlineto{\pgfqpoint{2.253074in}{1.479142in}}%
\pgfpathlineto{\pgfqpoint{2.253074in}{1.476193in}}%
\pgfpathmoveto{\pgfqpoint{2.253074in}{1.473244in}}%
\pgfpathlineto{\pgfqpoint{2.253074in}{1.473244in}}%
\pgfpathlineto{\pgfqpoint{2.253074in}{1.476193in}}%
\pgfpathlineto{\pgfqpoint{2.257615in}{1.476193in}}%
\pgfpathlineto{\pgfqpoint{2.257615in}{1.473244in}}%
\pgfpathmoveto{\pgfqpoint{2.253074in}{1.476193in}}%
\pgfpathlineto{\pgfqpoint{2.253074in}{1.476193in}}%
\pgfpathlineto{\pgfqpoint{2.253074in}{1.479142in}}%
\pgfpathlineto{\pgfqpoint{2.257615in}{1.479142in}}%
\pgfpathlineto{\pgfqpoint{2.257615in}{1.476193in}}%
\pgfpathmoveto{\pgfqpoint{2.257615in}{1.476193in}}%
\pgfpathlineto{\pgfqpoint{2.257615in}{1.476193in}}%
\pgfpathlineto{\pgfqpoint{2.257615in}{1.479142in}}%
\pgfpathlineto{\pgfqpoint{2.262157in}{1.479142in}}%
\pgfpathlineto{\pgfqpoint{2.262157in}{1.476193in}}%
\pgfpathmoveto{\pgfqpoint{2.257615in}{1.479142in}}%
\pgfpathlineto{\pgfqpoint{2.257615in}{1.479142in}}%
\pgfpathlineto{\pgfqpoint{2.257615in}{1.482091in}}%
\pgfpathlineto{\pgfqpoint{2.262157in}{1.482091in}}%
\pgfpathlineto{\pgfqpoint{2.262157in}{1.479142in}}%
\pgfpathmoveto{\pgfqpoint{2.257615in}{1.482091in}}%
\pgfpathlineto{\pgfqpoint{2.257615in}{1.482091in}}%
\pgfpathlineto{\pgfqpoint{2.257615in}{1.485040in}}%
\pgfpathlineto{\pgfqpoint{2.262157in}{1.485040in}}%
\pgfpathlineto{\pgfqpoint{2.262157in}{1.482091in}}%
\pgfpathmoveto{\pgfqpoint{2.262157in}{1.479142in}}%
\pgfpathlineto{\pgfqpoint{2.262157in}{1.479142in}}%
\pgfpathlineto{\pgfqpoint{2.262157in}{1.482091in}}%
\pgfpathlineto{\pgfqpoint{2.266698in}{1.482091in}}%
\pgfpathlineto{\pgfqpoint{2.266698in}{1.479142in}}%
\pgfpathmoveto{\pgfqpoint{2.262157in}{1.482091in}}%
\pgfpathlineto{\pgfqpoint{2.262157in}{1.482091in}}%
\pgfpathlineto{\pgfqpoint{2.262157in}{1.485040in}}%
\pgfpathlineto{\pgfqpoint{2.266698in}{1.485040in}}%
\pgfpathlineto{\pgfqpoint{2.266698in}{1.482091in}}%
\pgfpathmoveto{\pgfqpoint{2.266698in}{1.482091in}}%
\pgfpathlineto{\pgfqpoint{2.266698in}{1.482091in}}%
\pgfpathlineto{\pgfqpoint{2.266698in}{1.485040in}}%
\pgfpathlineto{\pgfqpoint{2.271239in}{1.485040in}}%
\pgfpathlineto{\pgfqpoint{2.271239in}{1.482091in}}%
\pgfpathmoveto{\pgfqpoint{2.266698in}{1.485040in}}%
\pgfpathlineto{\pgfqpoint{2.266698in}{1.485040in}}%
\pgfpathlineto{\pgfqpoint{2.266698in}{1.487989in}}%
\pgfpathlineto{\pgfqpoint{2.271239in}{1.487989in}}%
\pgfpathlineto{\pgfqpoint{2.271239in}{1.485040in}}%
\pgfpathmoveto{\pgfqpoint{2.266698in}{1.487989in}}%
\pgfpathlineto{\pgfqpoint{2.266698in}{1.487989in}}%
\pgfpathlineto{\pgfqpoint{2.266698in}{1.490938in}}%
\pgfpathlineto{\pgfqpoint{2.271239in}{1.490938in}}%
\pgfpathlineto{\pgfqpoint{2.271239in}{1.487989in}}%
\pgfpathmoveto{\pgfqpoint{2.271239in}{1.485040in}}%
\pgfpathlineto{\pgfqpoint{2.271239in}{1.485040in}}%
\pgfpathlineto{\pgfqpoint{2.271239in}{1.487989in}}%
\pgfpathlineto{\pgfqpoint{2.275780in}{1.487989in}}%
\pgfpathlineto{\pgfqpoint{2.275780in}{1.485040in}}%
\pgfpathmoveto{\pgfqpoint{2.271239in}{1.487989in}}%
\pgfpathlineto{\pgfqpoint{2.271239in}{1.487989in}}%
\pgfpathlineto{\pgfqpoint{2.271239in}{1.490938in}}%
\pgfpathlineto{\pgfqpoint{2.275780in}{1.490938in}}%
\pgfpathlineto{\pgfqpoint{2.275780in}{1.487989in}}%
\pgfpathmoveto{\pgfqpoint{2.275780in}{1.487989in}}%
\pgfpathlineto{\pgfqpoint{2.275780in}{1.487989in}}%
\pgfpathlineto{\pgfqpoint{2.275780in}{1.490938in}}%
\pgfpathlineto{\pgfqpoint{2.280321in}{1.490938in}}%
\pgfpathlineto{\pgfqpoint{2.280321in}{1.487989in}}%
\pgfpathmoveto{\pgfqpoint{2.275780in}{1.490938in}}%
\pgfpathlineto{\pgfqpoint{2.275780in}{1.490938in}}%
\pgfpathlineto{\pgfqpoint{2.275780in}{1.493888in}}%
\pgfpathlineto{\pgfqpoint{2.280321in}{1.493888in}}%
\pgfpathlineto{\pgfqpoint{2.280321in}{1.490938in}}%
\pgfpathmoveto{\pgfqpoint{2.275780in}{1.493888in}}%
\pgfpathlineto{\pgfqpoint{2.275780in}{1.493888in}}%
\pgfpathlineto{\pgfqpoint{2.275780in}{1.496837in}}%
\pgfpathlineto{\pgfqpoint{2.280321in}{1.496837in}}%
\pgfpathlineto{\pgfqpoint{2.280321in}{1.493888in}}%
\pgfpathmoveto{\pgfqpoint{2.280321in}{1.490938in}}%
\pgfpathlineto{\pgfqpoint{2.280321in}{1.490938in}}%
\pgfpathlineto{\pgfqpoint{2.280321in}{1.493888in}}%
\pgfpathlineto{\pgfqpoint{2.284863in}{1.493888in}}%
\pgfpathlineto{\pgfqpoint{2.284863in}{1.490938in}}%
\pgfpathmoveto{\pgfqpoint{2.280321in}{1.493888in}}%
\pgfpathlineto{\pgfqpoint{2.280321in}{1.493888in}}%
\pgfpathlineto{\pgfqpoint{2.280321in}{1.496837in}}%
\pgfpathlineto{\pgfqpoint{2.284863in}{1.496837in}}%
\pgfpathlineto{\pgfqpoint{2.284863in}{1.493888in}}%
\pgfpathmoveto{\pgfqpoint{2.284863in}{1.493888in}}%
\pgfpathlineto{\pgfqpoint{2.284863in}{1.493888in}}%
\pgfpathlineto{\pgfqpoint{2.284863in}{1.496837in}}%
\pgfpathlineto{\pgfqpoint{2.289404in}{1.496837in}}%
\pgfpathlineto{\pgfqpoint{2.289404in}{1.493888in}}%
\pgfpathmoveto{\pgfqpoint{2.284863in}{1.496837in}}%
\pgfpathlineto{\pgfqpoint{2.284863in}{1.496837in}}%
\pgfpathlineto{\pgfqpoint{2.284863in}{1.499786in}}%
\pgfpathlineto{\pgfqpoint{2.289404in}{1.499786in}}%
\pgfpathlineto{\pgfqpoint{2.289404in}{1.496837in}}%
\pgfpathmoveto{\pgfqpoint{2.284863in}{1.499786in}}%
\pgfpathlineto{\pgfqpoint{2.284863in}{1.499786in}}%
\pgfpathlineto{\pgfqpoint{2.284863in}{1.502735in}}%
\pgfpathlineto{\pgfqpoint{2.289404in}{1.502735in}}%
\pgfpathlineto{\pgfqpoint{2.289404in}{1.499786in}}%
\pgfpathmoveto{\pgfqpoint{2.289404in}{1.496837in}}%
\pgfpathlineto{\pgfqpoint{2.289404in}{1.496837in}}%
\pgfpathlineto{\pgfqpoint{2.289404in}{1.499786in}}%
\pgfpathlineto{\pgfqpoint{2.293945in}{1.499786in}}%
\pgfpathlineto{\pgfqpoint{2.293945in}{1.496837in}}%
\pgfpathmoveto{\pgfqpoint{2.289404in}{1.499786in}}%
\pgfpathlineto{\pgfqpoint{2.289404in}{1.499786in}}%
\pgfpathlineto{\pgfqpoint{2.289404in}{1.502735in}}%
\pgfpathlineto{\pgfqpoint{2.293945in}{1.502735in}}%
\pgfpathlineto{\pgfqpoint{2.293945in}{1.499786in}}%
\pgfpathmoveto{\pgfqpoint{2.293945in}{1.499786in}}%
\pgfpathlineto{\pgfqpoint{2.293945in}{1.499786in}}%
\pgfpathlineto{\pgfqpoint{2.293945in}{1.502735in}}%
\pgfpathlineto{\pgfqpoint{2.298486in}{1.502735in}}%
\pgfpathlineto{\pgfqpoint{2.298486in}{1.499786in}}%
\pgfpathmoveto{\pgfqpoint{2.293945in}{1.502735in}}%
\pgfpathlineto{\pgfqpoint{2.293945in}{1.502735in}}%
\pgfpathlineto{\pgfqpoint{2.293945in}{1.505684in}}%
\pgfpathlineto{\pgfqpoint{2.298486in}{1.505684in}}%
\pgfpathlineto{\pgfqpoint{2.298486in}{1.502735in}}%
\pgfpathmoveto{\pgfqpoint{2.293945in}{1.505684in}}%
\pgfpathlineto{\pgfqpoint{2.293945in}{1.505684in}}%
\pgfpathlineto{\pgfqpoint{2.293945in}{1.508633in}}%
\pgfpathlineto{\pgfqpoint{2.298486in}{1.508633in}}%
\pgfpathlineto{\pgfqpoint{2.298486in}{1.505684in}}%
\pgfpathmoveto{\pgfqpoint{2.298486in}{1.502735in}}%
\pgfpathlineto{\pgfqpoint{2.298486in}{1.502735in}}%
\pgfpathlineto{\pgfqpoint{2.298486in}{1.505684in}}%
\pgfpathlineto{\pgfqpoint{2.303027in}{1.505684in}}%
\pgfpathlineto{\pgfqpoint{2.303027in}{1.502735in}}%
\pgfpathmoveto{\pgfqpoint{2.298486in}{1.505684in}}%
\pgfpathlineto{\pgfqpoint{2.298486in}{1.505684in}}%
\pgfpathlineto{\pgfqpoint{2.298486in}{1.508633in}}%
\pgfpathlineto{\pgfqpoint{2.303027in}{1.508633in}}%
\pgfpathlineto{\pgfqpoint{2.303027in}{1.505684in}}%
\pgfpathmoveto{\pgfqpoint{2.303027in}{1.505684in}}%
\pgfpathlineto{\pgfqpoint{2.303027in}{1.505684in}}%
\pgfpathlineto{\pgfqpoint{2.303027in}{1.508633in}}%
\pgfpathlineto{\pgfqpoint{2.307568in}{1.508633in}}%
\pgfpathlineto{\pgfqpoint{2.307568in}{1.505684in}}%
\pgfpathmoveto{\pgfqpoint{2.303027in}{1.508633in}}%
\pgfpathlineto{\pgfqpoint{2.303027in}{1.508633in}}%
\pgfpathlineto{\pgfqpoint{2.303027in}{1.511582in}}%
\pgfpathlineto{\pgfqpoint{2.307568in}{1.511582in}}%
\pgfpathlineto{\pgfqpoint{2.307568in}{1.508633in}}%
\pgfpathmoveto{\pgfqpoint{2.303027in}{1.511582in}}%
\pgfpathlineto{\pgfqpoint{2.303027in}{1.511582in}}%
\pgfpathlineto{\pgfqpoint{2.303027in}{1.514531in}}%
\pgfpathlineto{\pgfqpoint{2.307568in}{1.514531in}}%
\pgfpathlineto{\pgfqpoint{2.307568in}{1.511582in}}%
\pgfpathmoveto{\pgfqpoint{2.307568in}{1.508633in}}%
\pgfpathlineto{\pgfqpoint{2.307568in}{1.508633in}}%
\pgfpathlineto{\pgfqpoint{2.307568in}{1.511582in}}%
\pgfpathlineto{\pgfqpoint{2.312110in}{1.511582in}}%
\pgfpathlineto{\pgfqpoint{2.312110in}{1.508633in}}%
\pgfpathmoveto{\pgfqpoint{2.307568in}{1.511582in}}%
\pgfpathlineto{\pgfqpoint{2.307568in}{1.511582in}}%
\pgfpathlineto{\pgfqpoint{2.307568in}{1.514531in}}%
\pgfpathlineto{\pgfqpoint{2.312110in}{1.514531in}}%
\pgfpathlineto{\pgfqpoint{2.312110in}{1.511582in}}%
\pgfpathmoveto{\pgfqpoint{2.312110in}{1.511582in}}%
\pgfpathlineto{\pgfqpoint{2.312110in}{1.511582in}}%
\pgfpathlineto{\pgfqpoint{2.312110in}{1.514531in}}%
\pgfpathlineto{\pgfqpoint{2.316651in}{1.514531in}}%
\pgfpathlineto{\pgfqpoint{2.316651in}{1.511582in}}%
\pgfpathmoveto{\pgfqpoint{2.312110in}{1.514531in}}%
\pgfpathlineto{\pgfqpoint{2.312110in}{1.514531in}}%
\pgfpathlineto{\pgfqpoint{2.312110in}{1.517481in}}%
\pgfpathlineto{\pgfqpoint{2.316651in}{1.517481in}}%
\pgfpathlineto{\pgfqpoint{2.316651in}{1.514531in}}%
\pgfpathmoveto{\pgfqpoint{2.312110in}{1.517481in}}%
\pgfpathlineto{\pgfqpoint{2.312110in}{1.517481in}}%
\pgfpathlineto{\pgfqpoint{2.312110in}{1.520430in}}%
\pgfpathlineto{\pgfqpoint{2.316651in}{1.520430in}}%
\pgfpathlineto{\pgfqpoint{2.316651in}{1.517481in}}%
\pgfpathmoveto{\pgfqpoint{2.316651in}{1.517481in}}%
\pgfpathlineto{\pgfqpoint{2.316651in}{1.517481in}}%
\pgfpathlineto{\pgfqpoint{2.316651in}{1.520430in}}%
\pgfpathlineto{\pgfqpoint{2.321192in}{1.520430in}}%
\pgfpathlineto{\pgfqpoint{2.321192in}{1.517481in}}%
\pgfpathmoveto{\pgfqpoint{2.312110in}{1.520430in}}%
\pgfpathlineto{\pgfqpoint{2.312110in}{1.520430in}}%
\pgfpathlineto{\pgfqpoint{2.312110in}{1.523379in}}%
\pgfpathlineto{\pgfqpoint{2.316651in}{1.523379in}}%
\pgfpathlineto{\pgfqpoint{2.316651in}{1.520430in}}%
\pgfpathmoveto{\pgfqpoint{2.312110in}{1.523379in}}%
\pgfpathlineto{\pgfqpoint{2.312110in}{1.523379in}}%
\pgfpathlineto{\pgfqpoint{2.312110in}{1.526328in}}%
\pgfpathlineto{\pgfqpoint{2.316651in}{1.526328in}}%
\pgfpathlineto{\pgfqpoint{2.316651in}{1.523379in}}%
\pgfpathmoveto{\pgfqpoint{2.316651in}{1.520430in}}%
\pgfpathlineto{\pgfqpoint{2.316651in}{1.520430in}}%
\pgfpathlineto{\pgfqpoint{2.316651in}{1.523379in}}%
\pgfpathlineto{\pgfqpoint{2.321192in}{1.523379in}}%
\pgfpathlineto{\pgfqpoint{2.321192in}{1.520430in}}%
\pgfpathmoveto{\pgfqpoint{2.316651in}{1.523379in}}%
\pgfpathlineto{\pgfqpoint{2.316651in}{1.523379in}}%
\pgfpathlineto{\pgfqpoint{2.316651in}{1.526328in}}%
\pgfpathlineto{\pgfqpoint{2.321192in}{1.526328in}}%
\pgfpathlineto{\pgfqpoint{2.321192in}{1.523379in}}%
\pgfpathmoveto{\pgfqpoint{2.321192in}{1.520430in}}%
\pgfpathlineto{\pgfqpoint{2.321192in}{1.520430in}}%
\pgfpathlineto{\pgfqpoint{2.321192in}{1.523379in}}%
\pgfpathlineto{\pgfqpoint{2.325733in}{1.523379in}}%
\pgfpathlineto{\pgfqpoint{2.325733in}{1.520430in}}%
\pgfpathmoveto{\pgfqpoint{2.321192in}{1.523379in}}%
\pgfpathlineto{\pgfqpoint{2.321192in}{1.523379in}}%
\pgfpathlineto{\pgfqpoint{2.321192in}{1.526328in}}%
\pgfpathlineto{\pgfqpoint{2.325733in}{1.526328in}}%
\pgfpathlineto{\pgfqpoint{2.325733in}{1.523379in}}%
\pgfpathmoveto{\pgfqpoint{2.325733in}{1.523379in}}%
\pgfpathlineto{\pgfqpoint{2.325733in}{1.523379in}}%
\pgfpathlineto{\pgfqpoint{2.325733in}{1.526328in}}%
\pgfpathlineto{\pgfqpoint{2.330274in}{1.526328in}}%
\pgfpathlineto{\pgfqpoint{2.330274in}{1.523379in}}%
\pgfpathmoveto{\pgfqpoint{2.321192in}{1.526328in}}%
\pgfpathlineto{\pgfqpoint{2.321192in}{1.526328in}}%
\pgfpathlineto{\pgfqpoint{2.321192in}{1.529277in}}%
\pgfpathlineto{\pgfqpoint{2.325733in}{1.529277in}}%
\pgfpathlineto{\pgfqpoint{2.325733in}{1.526328in}}%
\pgfpathmoveto{\pgfqpoint{2.321192in}{1.529277in}}%
\pgfpathlineto{\pgfqpoint{2.321192in}{1.529277in}}%
\pgfpathlineto{\pgfqpoint{2.321192in}{1.532226in}}%
\pgfpathlineto{\pgfqpoint{2.325733in}{1.532226in}}%
\pgfpathlineto{\pgfqpoint{2.325733in}{1.529277in}}%
\pgfpathmoveto{\pgfqpoint{2.325733in}{1.526328in}}%
\pgfpathlineto{\pgfqpoint{2.325733in}{1.526328in}}%
\pgfpathlineto{\pgfqpoint{2.325733in}{1.529277in}}%
\pgfpathlineto{\pgfqpoint{2.330274in}{1.529277in}}%
\pgfpathlineto{\pgfqpoint{2.330274in}{1.526328in}}%
\pgfpathmoveto{\pgfqpoint{2.325733in}{1.529277in}}%
\pgfpathlineto{\pgfqpoint{2.325733in}{1.529277in}}%
\pgfpathlineto{\pgfqpoint{2.325733in}{1.532226in}}%
\pgfpathlineto{\pgfqpoint{2.330274in}{1.532226in}}%
\pgfpathlineto{\pgfqpoint{2.330274in}{1.529277in}}%
\pgfpathmoveto{\pgfqpoint{2.330274in}{1.526328in}}%
\pgfpathlineto{\pgfqpoint{2.330274in}{1.526328in}}%
\pgfpathlineto{\pgfqpoint{2.330274in}{1.529277in}}%
\pgfpathlineto{\pgfqpoint{2.334816in}{1.529277in}}%
\pgfpathlineto{\pgfqpoint{2.334816in}{1.526328in}}%
\pgfpathmoveto{\pgfqpoint{2.330274in}{1.529277in}}%
\pgfpathlineto{\pgfqpoint{2.330274in}{1.529277in}}%
\pgfpathlineto{\pgfqpoint{2.330274in}{1.532226in}}%
\pgfpathlineto{\pgfqpoint{2.334816in}{1.532226in}}%
\pgfpathlineto{\pgfqpoint{2.334816in}{1.529277in}}%
\pgfpathmoveto{\pgfqpoint{2.334816in}{1.529277in}}%
\pgfpathlineto{\pgfqpoint{2.334816in}{1.529277in}}%
\pgfpathlineto{\pgfqpoint{2.334816in}{1.532226in}}%
\pgfpathlineto{\pgfqpoint{2.339357in}{1.532226in}}%
\pgfpathlineto{\pgfqpoint{2.339357in}{1.529277in}}%
\pgfpathmoveto{\pgfqpoint{2.330274in}{1.532226in}}%
\pgfpathlineto{\pgfqpoint{2.330274in}{1.532226in}}%
\pgfpathlineto{\pgfqpoint{2.330274in}{1.535175in}}%
\pgfpathlineto{\pgfqpoint{2.334816in}{1.535175in}}%
\pgfpathlineto{\pgfqpoint{2.334816in}{1.532226in}}%
\pgfpathmoveto{\pgfqpoint{2.330274in}{1.535175in}}%
\pgfpathlineto{\pgfqpoint{2.330274in}{1.535175in}}%
\pgfpathlineto{\pgfqpoint{2.330274in}{1.538125in}}%
\pgfpathlineto{\pgfqpoint{2.334816in}{1.538125in}}%
\pgfpathlineto{\pgfqpoint{2.334816in}{1.535175in}}%
\pgfpathmoveto{\pgfqpoint{2.334816in}{1.532226in}}%
\pgfpathlineto{\pgfqpoint{2.334816in}{1.532226in}}%
\pgfpathlineto{\pgfqpoint{2.334816in}{1.535175in}}%
\pgfpathlineto{\pgfqpoint{2.339357in}{1.535175in}}%
\pgfpathlineto{\pgfqpoint{2.339357in}{1.532226in}}%
\pgfpathmoveto{\pgfqpoint{2.334816in}{1.535175in}}%
\pgfpathlineto{\pgfqpoint{2.334816in}{1.535175in}}%
\pgfpathlineto{\pgfqpoint{2.334816in}{1.538125in}}%
\pgfpathlineto{\pgfqpoint{2.339357in}{1.538125in}}%
\pgfpathlineto{\pgfqpoint{2.339357in}{1.535175in}}%
\pgfpathmoveto{\pgfqpoint{2.339357in}{1.532226in}}%
\pgfpathlineto{\pgfqpoint{2.339357in}{1.532226in}}%
\pgfpathlineto{\pgfqpoint{2.339357in}{1.535175in}}%
\pgfpathlineto{\pgfqpoint{2.343898in}{1.535175in}}%
\pgfpathlineto{\pgfqpoint{2.343898in}{1.532226in}}%
\pgfpathmoveto{\pgfqpoint{2.339357in}{1.535175in}}%
\pgfpathlineto{\pgfqpoint{2.339357in}{1.535175in}}%
\pgfpathlineto{\pgfqpoint{2.339357in}{1.538125in}}%
\pgfpathlineto{\pgfqpoint{2.343898in}{1.538125in}}%
\pgfpathlineto{\pgfqpoint{2.343898in}{1.535175in}}%
\pgfpathmoveto{\pgfqpoint{2.343898in}{1.535175in}}%
\pgfpathlineto{\pgfqpoint{2.343898in}{1.535175in}}%
\pgfpathlineto{\pgfqpoint{2.343898in}{1.538125in}}%
\pgfpathlineto{\pgfqpoint{2.348439in}{1.538125in}}%
\pgfpathlineto{\pgfqpoint{2.348439in}{1.535175in}}%
\pgfpathmoveto{\pgfqpoint{2.339357in}{1.538125in}}%
\pgfpathlineto{\pgfqpoint{2.339357in}{1.538125in}}%
\pgfpathlineto{\pgfqpoint{2.339357in}{1.541074in}}%
\pgfpathlineto{\pgfqpoint{2.343898in}{1.541074in}}%
\pgfpathlineto{\pgfqpoint{2.343898in}{1.538125in}}%
\pgfpathmoveto{\pgfqpoint{2.339357in}{1.541074in}}%
\pgfpathlineto{\pgfqpoint{2.339357in}{1.541074in}}%
\pgfpathlineto{\pgfqpoint{2.339357in}{1.544023in}}%
\pgfpathlineto{\pgfqpoint{2.343898in}{1.544023in}}%
\pgfpathlineto{\pgfqpoint{2.343898in}{1.541074in}}%
\pgfpathmoveto{\pgfqpoint{2.343898in}{1.538125in}}%
\pgfpathlineto{\pgfqpoint{2.343898in}{1.538125in}}%
\pgfpathlineto{\pgfqpoint{2.343898in}{1.541074in}}%
\pgfpathlineto{\pgfqpoint{2.348439in}{1.541074in}}%
\pgfpathlineto{\pgfqpoint{2.348439in}{1.538125in}}%
\pgfpathmoveto{\pgfqpoint{2.343898in}{1.541074in}}%
\pgfpathlineto{\pgfqpoint{2.343898in}{1.541074in}}%
\pgfpathlineto{\pgfqpoint{2.343898in}{1.544023in}}%
\pgfpathlineto{\pgfqpoint{2.348439in}{1.544023in}}%
\pgfpathlineto{\pgfqpoint{2.348439in}{1.541074in}}%
\pgfpathmoveto{\pgfqpoint{2.348439in}{1.538125in}}%
\pgfpathlineto{\pgfqpoint{2.348439in}{1.538125in}}%
\pgfpathlineto{\pgfqpoint{2.348439in}{1.541074in}}%
\pgfpathlineto{\pgfqpoint{2.352980in}{1.541074in}}%
\pgfpathlineto{\pgfqpoint{2.352980in}{1.538125in}}%
\pgfpathmoveto{\pgfqpoint{2.348439in}{1.541074in}}%
\pgfpathlineto{\pgfqpoint{2.348439in}{1.541074in}}%
\pgfpathlineto{\pgfqpoint{2.348439in}{1.544023in}}%
\pgfpathlineto{\pgfqpoint{2.352980in}{1.544023in}}%
\pgfpathlineto{\pgfqpoint{2.352980in}{1.541074in}}%
\pgfpathmoveto{\pgfqpoint{2.352980in}{1.541074in}}%
\pgfpathlineto{\pgfqpoint{2.352980in}{1.541074in}}%
\pgfpathlineto{\pgfqpoint{2.352980in}{1.544023in}}%
\pgfpathlineto{\pgfqpoint{2.357521in}{1.544023in}}%
\pgfpathlineto{\pgfqpoint{2.357521in}{1.541074in}}%
\pgfpathmoveto{\pgfqpoint{2.348439in}{1.544023in}}%
\pgfpathlineto{\pgfqpoint{2.348439in}{1.544023in}}%
\pgfpathlineto{\pgfqpoint{2.348439in}{1.546972in}}%
\pgfpathlineto{\pgfqpoint{2.352980in}{1.546972in}}%
\pgfpathlineto{\pgfqpoint{2.352980in}{1.544023in}}%
\pgfpathmoveto{\pgfqpoint{2.348439in}{1.546972in}}%
\pgfpathlineto{\pgfqpoint{2.348439in}{1.546972in}}%
\pgfpathlineto{\pgfqpoint{2.348439in}{1.549921in}}%
\pgfpathlineto{\pgfqpoint{2.352980in}{1.549921in}}%
\pgfpathlineto{\pgfqpoint{2.352980in}{1.546972in}}%
\pgfpathmoveto{\pgfqpoint{2.352980in}{1.544023in}}%
\pgfpathlineto{\pgfqpoint{2.352980in}{1.544023in}}%
\pgfpathlineto{\pgfqpoint{2.352980in}{1.546972in}}%
\pgfpathlineto{\pgfqpoint{2.357521in}{1.546972in}}%
\pgfpathlineto{\pgfqpoint{2.357521in}{1.544023in}}%
\pgfpathmoveto{\pgfqpoint{2.352980in}{1.546972in}}%
\pgfpathlineto{\pgfqpoint{2.352980in}{1.546972in}}%
\pgfpathlineto{\pgfqpoint{2.352980in}{1.549921in}}%
\pgfpathlineto{\pgfqpoint{2.357521in}{1.549921in}}%
\pgfpathlineto{\pgfqpoint{2.357521in}{1.546972in}}%
\pgfpathmoveto{\pgfqpoint{2.357521in}{1.544023in}}%
\pgfpathlineto{\pgfqpoint{2.357521in}{1.544023in}}%
\pgfpathlineto{\pgfqpoint{2.357521in}{1.546972in}}%
\pgfpathlineto{\pgfqpoint{2.362062in}{1.546972in}}%
\pgfpathlineto{\pgfqpoint{2.362062in}{1.544023in}}%
\pgfpathmoveto{\pgfqpoint{2.357521in}{1.546972in}}%
\pgfpathlineto{\pgfqpoint{2.357521in}{1.546972in}}%
\pgfpathlineto{\pgfqpoint{2.357521in}{1.549921in}}%
\pgfpathlineto{\pgfqpoint{2.362062in}{1.549921in}}%
\pgfpathlineto{\pgfqpoint{2.362062in}{1.546972in}}%
\pgfpathmoveto{\pgfqpoint{2.362062in}{1.546972in}}%
\pgfpathlineto{\pgfqpoint{2.362062in}{1.546972in}}%
\pgfpathlineto{\pgfqpoint{2.362062in}{1.549921in}}%
\pgfpathlineto{\pgfqpoint{2.366603in}{1.549921in}}%
\pgfpathlineto{\pgfqpoint{2.366603in}{1.546972in}}%
\pgfpathmoveto{\pgfqpoint{2.357521in}{1.549921in}}%
\pgfpathlineto{\pgfqpoint{2.357521in}{1.549921in}}%
\pgfpathlineto{\pgfqpoint{2.357521in}{1.552870in}}%
\pgfpathlineto{\pgfqpoint{2.362062in}{1.552870in}}%
\pgfpathlineto{\pgfqpoint{2.362062in}{1.549921in}}%
\pgfpathmoveto{\pgfqpoint{2.357521in}{1.552870in}}%
\pgfpathlineto{\pgfqpoint{2.357521in}{1.552870in}}%
\pgfpathlineto{\pgfqpoint{2.357521in}{1.555820in}}%
\pgfpathlineto{\pgfqpoint{2.362062in}{1.555820in}}%
\pgfpathlineto{\pgfqpoint{2.362062in}{1.552870in}}%
\pgfpathmoveto{\pgfqpoint{2.362062in}{1.549921in}}%
\pgfpathlineto{\pgfqpoint{2.362062in}{1.549921in}}%
\pgfpathlineto{\pgfqpoint{2.362062in}{1.552870in}}%
\pgfpathlineto{\pgfqpoint{2.366603in}{1.552870in}}%
\pgfpathlineto{\pgfqpoint{2.366603in}{1.549921in}}%
\pgfpathmoveto{\pgfqpoint{2.362062in}{1.552870in}}%
\pgfpathlineto{\pgfqpoint{2.362062in}{1.552870in}}%
\pgfpathlineto{\pgfqpoint{2.362062in}{1.555820in}}%
\pgfpathlineto{\pgfqpoint{2.366603in}{1.555820in}}%
\pgfpathlineto{\pgfqpoint{2.366603in}{1.552870in}}%
\pgfpathmoveto{\pgfqpoint{2.366603in}{1.549921in}}%
\pgfpathlineto{\pgfqpoint{2.366603in}{1.549921in}}%
\pgfpathlineto{\pgfqpoint{2.366603in}{1.552870in}}%
\pgfpathlineto{\pgfqpoint{2.371144in}{1.552870in}}%
\pgfpathlineto{\pgfqpoint{2.371144in}{1.549921in}}%
\pgfpathmoveto{\pgfqpoint{2.366603in}{1.552870in}}%
\pgfpathlineto{\pgfqpoint{2.366603in}{1.552870in}}%
\pgfpathlineto{\pgfqpoint{2.366603in}{1.555820in}}%
\pgfpathlineto{\pgfqpoint{2.371144in}{1.555820in}}%
\pgfpathlineto{\pgfqpoint{2.371144in}{1.552870in}}%
\pgfpathmoveto{\pgfqpoint{2.371144in}{1.552870in}}%
\pgfpathlineto{\pgfqpoint{2.371144in}{1.552870in}}%
\pgfpathlineto{\pgfqpoint{2.371144in}{1.555820in}}%
\pgfpathlineto{\pgfqpoint{2.375684in}{1.555820in}}%
\pgfpathlineto{\pgfqpoint{2.375684in}{1.552870in}}%
\pgfpathmoveto{\pgfqpoint{2.366603in}{1.555820in}}%
\pgfpathlineto{\pgfqpoint{2.366603in}{1.555820in}}%
\pgfpathlineto{\pgfqpoint{2.366603in}{1.558769in}}%
\pgfpathlineto{\pgfqpoint{2.371144in}{1.558769in}}%
\pgfpathlineto{\pgfqpoint{2.371144in}{1.555820in}}%
\pgfpathmoveto{\pgfqpoint{2.366603in}{1.558769in}}%
\pgfpathlineto{\pgfqpoint{2.366603in}{1.558769in}}%
\pgfpathlineto{\pgfqpoint{2.366603in}{1.561718in}}%
\pgfpathlineto{\pgfqpoint{2.371144in}{1.561718in}}%
\pgfpathlineto{\pgfqpoint{2.371144in}{1.558769in}}%
\pgfpathmoveto{\pgfqpoint{2.371144in}{1.555820in}}%
\pgfpathlineto{\pgfqpoint{2.371144in}{1.555820in}}%
\pgfpathlineto{\pgfqpoint{2.371144in}{1.558769in}}%
\pgfpathlineto{\pgfqpoint{2.375684in}{1.558769in}}%
\pgfpathlineto{\pgfqpoint{2.375684in}{1.555820in}}%
\pgfpathmoveto{\pgfqpoint{2.371144in}{1.558769in}}%
\pgfpathlineto{\pgfqpoint{2.371144in}{1.558769in}}%
\pgfpathlineto{\pgfqpoint{2.371144in}{1.561718in}}%
\pgfpathlineto{\pgfqpoint{2.375684in}{1.561718in}}%
\pgfpathlineto{\pgfqpoint{2.375684in}{1.558769in}}%
\pgfpathmoveto{\pgfqpoint{2.375684in}{1.555820in}}%
\pgfpathlineto{\pgfqpoint{2.375684in}{1.555820in}}%
\pgfpathlineto{\pgfqpoint{2.375684in}{1.558769in}}%
\pgfpathlineto{\pgfqpoint{2.380225in}{1.558769in}}%
\pgfpathlineto{\pgfqpoint{2.380225in}{1.555820in}}%
\pgfpathmoveto{\pgfqpoint{2.375684in}{1.558769in}}%
\pgfpathlineto{\pgfqpoint{2.375684in}{1.558769in}}%
\pgfpathlineto{\pgfqpoint{2.375684in}{1.561718in}}%
\pgfpathlineto{\pgfqpoint{2.380225in}{1.561718in}}%
\pgfpathlineto{\pgfqpoint{2.380225in}{1.558769in}}%
\pgfpathmoveto{\pgfqpoint{2.380225in}{1.558769in}}%
\pgfpathlineto{\pgfqpoint{2.380225in}{1.558769in}}%
\pgfpathlineto{\pgfqpoint{2.380225in}{1.561718in}}%
\pgfpathlineto{\pgfqpoint{2.384766in}{1.561718in}}%
\pgfpathlineto{\pgfqpoint{2.384766in}{1.558769in}}%
\pgfpathmoveto{\pgfqpoint{2.375684in}{1.561718in}}%
\pgfpathlineto{\pgfqpoint{2.375684in}{1.561718in}}%
\pgfpathlineto{\pgfqpoint{2.375684in}{1.564667in}}%
\pgfpathlineto{\pgfqpoint{2.380225in}{1.564667in}}%
\pgfpathlineto{\pgfqpoint{2.380225in}{1.561718in}}%
\pgfpathmoveto{\pgfqpoint{2.375684in}{1.564667in}}%
\pgfpathlineto{\pgfqpoint{2.375684in}{1.564667in}}%
\pgfpathlineto{\pgfqpoint{2.375684in}{1.567616in}}%
\pgfpathlineto{\pgfqpoint{2.380225in}{1.567616in}}%
\pgfpathlineto{\pgfqpoint{2.380225in}{1.564667in}}%
\pgfpathmoveto{\pgfqpoint{2.380225in}{1.561718in}}%
\pgfpathlineto{\pgfqpoint{2.380225in}{1.561718in}}%
\pgfpathlineto{\pgfqpoint{2.380225in}{1.564667in}}%
\pgfpathlineto{\pgfqpoint{2.384766in}{1.564667in}}%
\pgfpathlineto{\pgfqpoint{2.384766in}{1.561718in}}%
\pgfpathmoveto{\pgfqpoint{2.380225in}{1.564667in}}%
\pgfpathlineto{\pgfqpoint{2.380225in}{1.564667in}}%
\pgfpathlineto{\pgfqpoint{2.380225in}{1.567616in}}%
\pgfpathlineto{\pgfqpoint{2.384766in}{1.567616in}}%
\pgfpathlineto{\pgfqpoint{2.384766in}{1.564667in}}%
\pgfpathmoveto{\pgfqpoint{2.384766in}{1.561718in}}%
\pgfpathlineto{\pgfqpoint{2.384766in}{1.561718in}}%
\pgfpathlineto{\pgfqpoint{2.384766in}{1.564667in}}%
\pgfpathlineto{\pgfqpoint{2.389307in}{1.564667in}}%
\pgfpathlineto{\pgfqpoint{2.389307in}{1.561718in}}%
\pgfpathmoveto{\pgfqpoint{2.384766in}{1.564667in}}%
\pgfpathlineto{\pgfqpoint{2.384766in}{1.564667in}}%
\pgfpathlineto{\pgfqpoint{2.384766in}{1.567616in}}%
\pgfpathlineto{\pgfqpoint{2.389307in}{1.567616in}}%
\pgfpathlineto{\pgfqpoint{2.389307in}{1.564667in}}%
\pgfpathmoveto{\pgfqpoint{2.389307in}{1.564667in}}%
\pgfpathlineto{\pgfqpoint{2.389307in}{1.564667in}}%
\pgfpathlineto{\pgfqpoint{2.389307in}{1.567616in}}%
\pgfpathlineto{\pgfqpoint{2.393848in}{1.567616in}}%
\pgfpathlineto{\pgfqpoint{2.393848in}{1.564667in}}%
\pgfpathmoveto{\pgfqpoint{2.384766in}{1.567616in}}%
\pgfpathlineto{\pgfqpoint{2.384766in}{1.567616in}}%
\pgfpathlineto{\pgfqpoint{2.384766in}{1.570565in}}%
\pgfpathlineto{\pgfqpoint{2.389307in}{1.570565in}}%
\pgfpathlineto{\pgfqpoint{2.389307in}{1.567616in}}%
\pgfpathmoveto{\pgfqpoint{2.384766in}{1.570565in}}%
\pgfpathlineto{\pgfqpoint{2.384766in}{1.570565in}}%
\pgfpathlineto{\pgfqpoint{2.384766in}{1.573515in}}%
\pgfpathlineto{\pgfqpoint{2.389307in}{1.573515in}}%
\pgfpathlineto{\pgfqpoint{2.389307in}{1.570565in}}%
\pgfpathmoveto{\pgfqpoint{2.389307in}{1.567616in}}%
\pgfpathlineto{\pgfqpoint{2.389307in}{1.567616in}}%
\pgfpathlineto{\pgfqpoint{2.389307in}{1.570565in}}%
\pgfpathlineto{\pgfqpoint{2.393848in}{1.570565in}}%
\pgfpathlineto{\pgfqpoint{2.393848in}{1.567616in}}%
\pgfpathmoveto{\pgfqpoint{2.389307in}{1.570565in}}%
\pgfpathlineto{\pgfqpoint{2.389307in}{1.570565in}}%
\pgfpathlineto{\pgfqpoint{2.389307in}{1.573515in}}%
\pgfpathlineto{\pgfqpoint{2.393848in}{1.573515in}}%
\pgfpathlineto{\pgfqpoint{2.393848in}{1.570565in}}%
\pgfpathmoveto{\pgfqpoint{2.393848in}{1.567616in}}%
\pgfpathlineto{\pgfqpoint{2.393848in}{1.567616in}}%
\pgfpathlineto{\pgfqpoint{2.393848in}{1.570565in}}%
\pgfpathlineto{\pgfqpoint{2.398389in}{1.570565in}}%
\pgfpathlineto{\pgfqpoint{2.398389in}{1.567616in}}%
\pgfpathmoveto{\pgfqpoint{2.393848in}{1.570565in}}%
\pgfpathlineto{\pgfqpoint{2.393848in}{1.570565in}}%
\pgfpathlineto{\pgfqpoint{2.393848in}{1.573515in}}%
\pgfpathlineto{\pgfqpoint{2.398389in}{1.573515in}}%
\pgfpathlineto{\pgfqpoint{2.398389in}{1.570565in}}%
\pgfpathmoveto{\pgfqpoint{2.398389in}{1.570565in}}%
\pgfpathlineto{\pgfqpoint{2.398389in}{1.570565in}}%
\pgfpathlineto{\pgfqpoint{2.398389in}{1.573515in}}%
\pgfpathlineto{\pgfqpoint{2.402930in}{1.573515in}}%
\pgfpathlineto{\pgfqpoint{2.402930in}{1.570565in}}%
\pgfpathmoveto{\pgfqpoint{2.393848in}{1.573515in}}%
\pgfpathlineto{\pgfqpoint{2.393848in}{1.573515in}}%
\pgfpathlineto{\pgfqpoint{2.393848in}{1.576464in}}%
\pgfpathlineto{\pgfqpoint{2.398389in}{1.576464in}}%
\pgfpathlineto{\pgfqpoint{2.398389in}{1.573515in}}%
\pgfpathmoveto{\pgfqpoint{2.393848in}{1.576464in}}%
\pgfpathlineto{\pgfqpoint{2.393848in}{1.576464in}}%
\pgfpathlineto{\pgfqpoint{2.393848in}{1.579413in}}%
\pgfpathlineto{\pgfqpoint{2.398389in}{1.579413in}}%
\pgfpathlineto{\pgfqpoint{2.398389in}{1.576464in}}%
\pgfpathmoveto{\pgfqpoint{2.398389in}{1.573515in}}%
\pgfpathlineto{\pgfqpoint{2.398389in}{1.573515in}}%
\pgfpathlineto{\pgfqpoint{2.398389in}{1.576464in}}%
\pgfpathlineto{\pgfqpoint{2.402930in}{1.576464in}}%
\pgfpathlineto{\pgfqpoint{2.402930in}{1.573515in}}%
\pgfpathmoveto{\pgfqpoint{2.398389in}{1.576464in}}%
\pgfpathlineto{\pgfqpoint{2.398389in}{1.576464in}}%
\pgfpathlineto{\pgfqpoint{2.398389in}{1.579413in}}%
\pgfpathlineto{\pgfqpoint{2.402930in}{1.579413in}}%
\pgfpathlineto{\pgfqpoint{2.402930in}{1.576464in}}%
\pgfpathmoveto{\pgfqpoint{2.402930in}{1.573515in}}%
\pgfpathlineto{\pgfqpoint{2.402930in}{1.573515in}}%
\pgfpathlineto{\pgfqpoint{2.402930in}{1.576464in}}%
\pgfpathlineto{\pgfqpoint{2.407471in}{1.576464in}}%
\pgfpathlineto{\pgfqpoint{2.407471in}{1.573515in}}%
\pgfpathmoveto{\pgfqpoint{2.402930in}{1.576464in}}%
\pgfpathlineto{\pgfqpoint{2.402930in}{1.576464in}}%
\pgfpathlineto{\pgfqpoint{2.402930in}{1.579413in}}%
\pgfpathlineto{\pgfqpoint{2.407471in}{1.579413in}}%
\pgfpathlineto{\pgfqpoint{2.407471in}{1.576464in}}%
\pgfpathmoveto{\pgfqpoint{2.407471in}{1.576464in}}%
\pgfpathlineto{\pgfqpoint{2.407471in}{1.576464in}}%
\pgfpathlineto{\pgfqpoint{2.407471in}{1.579413in}}%
\pgfpathlineto{\pgfqpoint{2.412011in}{1.579413in}}%
\pgfpathlineto{\pgfqpoint{2.412011in}{1.576464in}}%
\pgfpathmoveto{\pgfqpoint{2.402930in}{1.579413in}}%
\pgfpathlineto{\pgfqpoint{2.402930in}{1.579413in}}%
\pgfpathlineto{\pgfqpoint{2.402930in}{1.582362in}}%
\pgfpathlineto{\pgfqpoint{2.407471in}{1.582362in}}%
\pgfpathlineto{\pgfqpoint{2.407471in}{1.579413in}}%
\pgfpathmoveto{\pgfqpoint{2.402930in}{1.582362in}}%
\pgfpathlineto{\pgfqpoint{2.402930in}{1.582362in}}%
\pgfpathlineto{\pgfqpoint{2.402930in}{1.585311in}}%
\pgfpathlineto{\pgfqpoint{2.407471in}{1.585311in}}%
\pgfpathlineto{\pgfqpoint{2.407471in}{1.582362in}}%
\pgfpathmoveto{\pgfqpoint{2.407471in}{1.579413in}}%
\pgfpathlineto{\pgfqpoint{2.407471in}{1.579413in}}%
\pgfpathlineto{\pgfqpoint{2.407471in}{1.582362in}}%
\pgfpathlineto{\pgfqpoint{2.412011in}{1.582362in}}%
\pgfpathlineto{\pgfqpoint{2.412011in}{1.579413in}}%
\pgfpathmoveto{\pgfqpoint{2.407471in}{1.582362in}}%
\pgfpathlineto{\pgfqpoint{2.407471in}{1.582362in}}%
\pgfpathlineto{\pgfqpoint{2.407471in}{1.585311in}}%
\pgfpathlineto{\pgfqpoint{2.412011in}{1.585311in}}%
\pgfpathlineto{\pgfqpoint{2.412011in}{1.582362in}}%
\pgfpathmoveto{\pgfqpoint{2.412011in}{1.579413in}}%
\pgfpathlineto{\pgfqpoint{2.412011in}{1.579413in}}%
\pgfpathlineto{\pgfqpoint{2.412011in}{1.582362in}}%
\pgfpathlineto{\pgfqpoint{2.416552in}{1.582362in}}%
\pgfpathlineto{\pgfqpoint{2.416552in}{1.579413in}}%
\pgfpathmoveto{\pgfqpoint{2.412011in}{1.582362in}}%
\pgfpathlineto{\pgfqpoint{2.412011in}{1.582362in}}%
\pgfpathlineto{\pgfqpoint{2.412011in}{1.585311in}}%
\pgfpathlineto{\pgfqpoint{2.416552in}{1.585311in}}%
\pgfpathlineto{\pgfqpoint{2.416552in}{1.582362in}}%
\pgfpathmoveto{\pgfqpoint{2.416552in}{1.582362in}}%
\pgfpathlineto{\pgfqpoint{2.416552in}{1.582362in}}%
\pgfpathlineto{\pgfqpoint{2.416552in}{1.585311in}}%
\pgfpathlineto{\pgfqpoint{2.421093in}{1.585311in}}%
\pgfpathlineto{\pgfqpoint{2.421093in}{1.582362in}}%
\pgfpathmoveto{\pgfqpoint{2.412011in}{1.585311in}}%
\pgfpathlineto{\pgfqpoint{2.412011in}{1.585311in}}%
\pgfpathlineto{\pgfqpoint{2.412011in}{1.588260in}}%
\pgfpathlineto{\pgfqpoint{2.416552in}{1.588260in}}%
\pgfpathlineto{\pgfqpoint{2.416552in}{1.585311in}}%
\pgfpathmoveto{\pgfqpoint{2.412011in}{1.588260in}}%
\pgfpathlineto{\pgfqpoint{2.412011in}{1.588260in}}%
\pgfpathlineto{\pgfqpoint{2.412011in}{1.591210in}}%
\pgfpathlineto{\pgfqpoint{2.416552in}{1.591210in}}%
\pgfpathlineto{\pgfqpoint{2.416552in}{1.588260in}}%
\pgfpathmoveto{\pgfqpoint{2.416552in}{1.585311in}}%
\pgfpathlineto{\pgfqpoint{2.416552in}{1.585311in}}%
\pgfpathlineto{\pgfqpoint{2.416552in}{1.588260in}}%
\pgfpathlineto{\pgfqpoint{2.421093in}{1.588260in}}%
\pgfpathlineto{\pgfqpoint{2.421093in}{1.585311in}}%
\pgfpathmoveto{\pgfqpoint{2.416552in}{1.588260in}}%
\pgfpathlineto{\pgfqpoint{2.416552in}{1.588260in}}%
\pgfpathlineto{\pgfqpoint{2.416552in}{1.591210in}}%
\pgfpathlineto{\pgfqpoint{2.421093in}{1.591210in}}%
\pgfpathlineto{\pgfqpoint{2.421093in}{1.588260in}}%
\pgfpathmoveto{\pgfqpoint{2.421093in}{1.585311in}}%
\pgfpathlineto{\pgfqpoint{2.421093in}{1.585311in}}%
\pgfpathlineto{\pgfqpoint{2.421093in}{1.588260in}}%
\pgfpathlineto{\pgfqpoint{2.425634in}{1.588260in}}%
\pgfpathlineto{\pgfqpoint{2.425634in}{1.585311in}}%
\pgfpathmoveto{\pgfqpoint{2.421093in}{1.588260in}}%
\pgfpathlineto{\pgfqpoint{2.421093in}{1.588260in}}%
\pgfpathlineto{\pgfqpoint{2.421093in}{1.591210in}}%
\pgfpathlineto{\pgfqpoint{2.425634in}{1.591210in}}%
\pgfpathlineto{\pgfqpoint{2.425634in}{1.588260in}}%
\pgfpathmoveto{\pgfqpoint{2.425634in}{1.588260in}}%
\pgfpathlineto{\pgfqpoint{2.425634in}{1.588260in}}%
\pgfpathlineto{\pgfqpoint{2.425634in}{1.591210in}}%
\pgfpathlineto{\pgfqpoint{2.430175in}{1.591210in}}%
\pgfpathlineto{\pgfqpoint{2.430175in}{1.588260in}}%
\pgfpathmoveto{\pgfqpoint{2.421093in}{1.591210in}}%
\pgfpathlineto{\pgfqpoint{2.421093in}{1.591210in}}%
\pgfpathlineto{\pgfqpoint{2.421093in}{1.594159in}}%
\pgfpathlineto{\pgfqpoint{2.425634in}{1.594159in}}%
\pgfpathlineto{\pgfqpoint{2.425634in}{1.591210in}}%
\pgfpathmoveto{\pgfqpoint{2.421093in}{1.594159in}}%
\pgfpathlineto{\pgfqpoint{2.421093in}{1.594159in}}%
\pgfpathlineto{\pgfqpoint{2.421093in}{1.597108in}}%
\pgfpathlineto{\pgfqpoint{2.425634in}{1.597108in}}%
\pgfpathlineto{\pgfqpoint{2.425634in}{1.594159in}}%
\pgfpathmoveto{\pgfqpoint{2.425634in}{1.591210in}}%
\pgfpathlineto{\pgfqpoint{2.425634in}{1.591210in}}%
\pgfpathlineto{\pgfqpoint{2.425634in}{1.594159in}}%
\pgfpathlineto{\pgfqpoint{2.430175in}{1.594159in}}%
\pgfpathlineto{\pgfqpoint{2.430175in}{1.591210in}}%
\pgfpathmoveto{\pgfqpoint{2.425634in}{1.594159in}}%
\pgfpathlineto{\pgfqpoint{2.425634in}{1.594159in}}%
\pgfpathlineto{\pgfqpoint{2.425634in}{1.597108in}}%
\pgfpathlineto{\pgfqpoint{2.430175in}{1.597108in}}%
\pgfpathlineto{\pgfqpoint{2.430175in}{1.594159in}}%
\pgfpathmoveto{\pgfqpoint{2.430175in}{1.591210in}}%
\pgfpathlineto{\pgfqpoint{2.430175in}{1.591210in}}%
\pgfpathlineto{\pgfqpoint{2.430175in}{1.594159in}}%
\pgfpathlineto{\pgfqpoint{2.434716in}{1.594159in}}%
\pgfpathlineto{\pgfqpoint{2.434716in}{1.591210in}}%
\pgfpathmoveto{\pgfqpoint{2.430175in}{1.594159in}}%
\pgfpathlineto{\pgfqpoint{2.430175in}{1.594159in}}%
\pgfpathlineto{\pgfqpoint{2.430175in}{1.597108in}}%
\pgfpathlineto{\pgfqpoint{2.434716in}{1.597108in}}%
\pgfpathlineto{\pgfqpoint{2.434716in}{1.594159in}}%
\pgfpathmoveto{\pgfqpoint{2.434716in}{1.594159in}}%
\pgfpathlineto{\pgfqpoint{2.434716in}{1.594159in}}%
\pgfpathlineto{\pgfqpoint{2.434716in}{1.597108in}}%
\pgfpathlineto{\pgfqpoint{2.439257in}{1.597108in}}%
\pgfpathlineto{\pgfqpoint{2.439257in}{1.594159in}}%
\pgfpathmoveto{\pgfqpoint{2.430175in}{1.597108in}}%
\pgfpathlineto{\pgfqpoint{2.430175in}{1.597108in}}%
\pgfpathlineto{\pgfqpoint{2.430175in}{1.600057in}}%
\pgfpathlineto{\pgfqpoint{2.434716in}{1.600057in}}%
\pgfpathlineto{\pgfqpoint{2.434716in}{1.597108in}}%
\pgfpathmoveto{\pgfqpoint{2.430175in}{1.600057in}}%
\pgfpathlineto{\pgfqpoint{2.430175in}{1.600057in}}%
\pgfpathlineto{\pgfqpoint{2.430175in}{1.603006in}}%
\pgfpathlineto{\pgfqpoint{2.434716in}{1.603006in}}%
\pgfpathlineto{\pgfqpoint{2.434716in}{1.600057in}}%
\pgfpathmoveto{\pgfqpoint{2.434716in}{1.597108in}}%
\pgfpathlineto{\pgfqpoint{2.434716in}{1.597108in}}%
\pgfpathlineto{\pgfqpoint{2.434716in}{1.600057in}}%
\pgfpathlineto{\pgfqpoint{2.439257in}{1.600057in}}%
\pgfpathlineto{\pgfqpoint{2.439257in}{1.597108in}}%
\pgfpathmoveto{\pgfqpoint{2.434716in}{1.600057in}}%
\pgfpathlineto{\pgfqpoint{2.434716in}{1.600057in}}%
\pgfpathlineto{\pgfqpoint{2.434716in}{1.603006in}}%
\pgfpathlineto{\pgfqpoint{2.439257in}{1.603006in}}%
\pgfpathlineto{\pgfqpoint{2.439257in}{1.600057in}}%
\pgfpathmoveto{\pgfqpoint{2.439257in}{1.597108in}}%
\pgfpathlineto{\pgfqpoint{2.439257in}{1.597108in}}%
\pgfpathlineto{\pgfqpoint{2.439257in}{1.600057in}}%
\pgfpathlineto{\pgfqpoint{2.443798in}{1.600057in}}%
\pgfpathlineto{\pgfqpoint{2.443798in}{1.597108in}}%
\pgfpathmoveto{\pgfqpoint{2.439257in}{1.600057in}}%
\pgfpathlineto{\pgfqpoint{2.439257in}{1.600057in}}%
\pgfpathlineto{\pgfqpoint{2.439257in}{1.603006in}}%
\pgfpathlineto{\pgfqpoint{2.443798in}{1.603006in}}%
\pgfpathlineto{\pgfqpoint{2.443798in}{1.600057in}}%
\pgfpathmoveto{\pgfqpoint{2.443798in}{1.600057in}}%
\pgfpathlineto{\pgfqpoint{2.443798in}{1.600057in}}%
\pgfpathlineto{\pgfqpoint{2.443798in}{1.603006in}}%
\pgfpathlineto{\pgfqpoint{2.448338in}{1.603006in}}%
\pgfpathlineto{\pgfqpoint{2.448338in}{1.600057in}}%
\pgfpathmoveto{\pgfqpoint{2.439257in}{1.603006in}}%
\pgfpathlineto{\pgfqpoint{2.439257in}{1.603006in}}%
\pgfpathlineto{\pgfqpoint{2.439257in}{1.605955in}}%
\pgfpathlineto{\pgfqpoint{2.443798in}{1.605955in}}%
\pgfpathlineto{\pgfqpoint{2.443798in}{1.603006in}}%
\pgfpathmoveto{\pgfqpoint{2.439257in}{1.605955in}}%
\pgfpathlineto{\pgfqpoint{2.439257in}{1.605955in}}%
\pgfpathlineto{\pgfqpoint{2.439257in}{1.608905in}}%
\pgfpathlineto{\pgfqpoint{2.443798in}{1.608905in}}%
\pgfpathlineto{\pgfqpoint{2.443798in}{1.605955in}}%
\pgfpathmoveto{\pgfqpoint{2.443798in}{1.603006in}}%
\pgfpathlineto{\pgfqpoint{2.443798in}{1.603006in}}%
\pgfpathlineto{\pgfqpoint{2.443798in}{1.605955in}}%
\pgfpathlineto{\pgfqpoint{2.448338in}{1.605955in}}%
\pgfpathlineto{\pgfqpoint{2.448338in}{1.603006in}}%
\pgfpathmoveto{\pgfqpoint{2.443798in}{1.605955in}}%
\pgfpathlineto{\pgfqpoint{2.443798in}{1.605955in}}%
\pgfpathlineto{\pgfqpoint{2.443798in}{1.608905in}}%
\pgfpathlineto{\pgfqpoint{2.448338in}{1.608905in}}%
\pgfpathlineto{\pgfqpoint{2.448338in}{1.605955in}}%
\pgfpathmoveto{\pgfqpoint{2.448338in}{1.603006in}}%
\pgfpathlineto{\pgfqpoint{2.448338in}{1.603006in}}%
\pgfpathlineto{\pgfqpoint{2.448338in}{1.605955in}}%
\pgfpathlineto{\pgfqpoint{2.452879in}{1.605955in}}%
\pgfpathlineto{\pgfqpoint{2.452879in}{1.603006in}}%
\pgfpathmoveto{\pgfqpoint{2.448338in}{1.605955in}}%
\pgfpathlineto{\pgfqpoint{2.448338in}{1.605955in}}%
\pgfpathlineto{\pgfqpoint{2.448338in}{1.608905in}}%
\pgfpathlineto{\pgfqpoint{2.452879in}{1.608905in}}%
\pgfpathlineto{\pgfqpoint{2.452879in}{1.605955in}}%
\pgfpathmoveto{\pgfqpoint{2.452879in}{1.605955in}}%
\pgfpathlineto{\pgfqpoint{2.452879in}{1.605955in}}%
\pgfpathlineto{\pgfqpoint{2.452879in}{1.608905in}}%
\pgfpathlineto{\pgfqpoint{2.457420in}{1.608905in}}%
\pgfpathlineto{\pgfqpoint{2.457420in}{1.605955in}}%
\pgfpathmoveto{\pgfqpoint{2.448338in}{1.608905in}}%
\pgfpathlineto{\pgfqpoint{2.448338in}{1.608905in}}%
\pgfpathlineto{\pgfqpoint{2.448338in}{1.611854in}}%
\pgfpathlineto{\pgfqpoint{2.452879in}{1.611854in}}%
\pgfpathlineto{\pgfqpoint{2.452879in}{1.608905in}}%
\pgfpathmoveto{\pgfqpoint{2.448338in}{1.611854in}}%
\pgfpathlineto{\pgfqpoint{2.448338in}{1.611854in}}%
\pgfpathlineto{\pgfqpoint{2.448338in}{1.614803in}}%
\pgfpathlineto{\pgfqpoint{2.452879in}{1.614803in}}%
\pgfpathlineto{\pgfqpoint{2.452879in}{1.611854in}}%
\pgfpathmoveto{\pgfqpoint{2.452879in}{1.608905in}}%
\pgfpathlineto{\pgfqpoint{2.452879in}{1.608905in}}%
\pgfpathlineto{\pgfqpoint{2.452879in}{1.611854in}}%
\pgfpathlineto{\pgfqpoint{2.457420in}{1.611854in}}%
\pgfpathlineto{\pgfqpoint{2.457420in}{1.608905in}}%
\pgfpathmoveto{\pgfqpoint{2.452879in}{1.611854in}}%
\pgfpathlineto{\pgfqpoint{2.452879in}{1.611854in}}%
\pgfpathlineto{\pgfqpoint{2.452879in}{1.614803in}}%
\pgfpathlineto{\pgfqpoint{2.457420in}{1.614803in}}%
\pgfpathlineto{\pgfqpoint{2.457420in}{1.611854in}}%
\pgfpathmoveto{\pgfqpoint{2.457420in}{1.608905in}}%
\pgfpathlineto{\pgfqpoint{2.457420in}{1.608905in}}%
\pgfpathlineto{\pgfqpoint{2.457420in}{1.611854in}}%
\pgfpathlineto{\pgfqpoint{2.461961in}{1.611854in}}%
\pgfpathlineto{\pgfqpoint{2.461961in}{1.608905in}}%
\pgfpathmoveto{\pgfqpoint{2.457420in}{1.611854in}}%
\pgfpathlineto{\pgfqpoint{2.457420in}{1.611854in}}%
\pgfpathlineto{\pgfqpoint{2.457420in}{1.614803in}}%
\pgfpathlineto{\pgfqpoint{2.461961in}{1.614803in}}%
\pgfpathlineto{\pgfqpoint{2.461961in}{1.611854in}}%
\pgfpathmoveto{\pgfqpoint{2.461961in}{1.611854in}}%
\pgfpathlineto{\pgfqpoint{2.461961in}{1.611854in}}%
\pgfpathlineto{\pgfqpoint{2.461961in}{1.614803in}}%
\pgfpathlineto{\pgfqpoint{2.466502in}{1.614803in}}%
\pgfpathlineto{\pgfqpoint{2.466502in}{1.611854in}}%
\pgfpathmoveto{\pgfqpoint{2.457420in}{1.614803in}}%
\pgfpathlineto{\pgfqpoint{2.457420in}{1.614803in}}%
\pgfpathlineto{\pgfqpoint{2.457420in}{1.617752in}}%
\pgfpathlineto{\pgfqpoint{2.461961in}{1.617752in}}%
\pgfpathlineto{\pgfqpoint{2.461961in}{1.614803in}}%
\pgfpathmoveto{\pgfqpoint{2.457420in}{1.617752in}}%
\pgfpathlineto{\pgfqpoint{2.457420in}{1.617752in}}%
\pgfpathlineto{\pgfqpoint{2.457420in}{1.620701in}}%
\pgfpathlineto{\pgfqpoint{2.461961in}{1.620701in}}%
\pgfpathlineto{\pgfqpoint{2.461961in}{1.617752in}}%
\pgfpathmoveto{\pgfqpoint{2.461961in}{1.614803in}}%
\pgfpathlineto{\pgfqpoint{2.461961in}{1.614803in}}%
\pgfpathlineto{\pgfqpoint{2.461961in}{1.617752in}}%
\pgfpathlineto{\pgfqpoint{2.466502in}{1.617752in}}%
\pgfpathlineto{\pgfqpoint{2.466502in}{1.614803in}}%
\pgfpathmoveto{\pgfqpoint{2.461961in}{1.617752in}}%
\pgfpathlineto{\pgfqpoint{2.461961in}{1.617752in}}%
\pgfpathlineto{\pgfqpoint{2.461961in}{1.620701in}}%
\pgfpathlineto{\pgfqpoint{2.466502in}{1.620701in}}%
\pgfpathlineto{\pgfqpoint{2.466502in}{1.617752in}}%
\pgfpathmoveto{\pgfqpoint{2.466502in}{1.614803in}}%
\pgfpathlineto{\pgfqpoint{2.466502in}{1.614803in}}%
\pgfpathlineto{\pgfqpoint{2.466502in}{1.617752in}}%
\pgfpathlineto{\pgfqpoint{2.471043in}{1.617752in}}%
\pgfpathlineto{\pgfqpoint{2.471043in}{1.614803in}}%
\pgfpathmoveto{\pgfqpoint{2.466502in}{1.617752in}}%
\pgfpathlineto{\pgfqpoint{2.466502in}{1.617752in}}%
\pgfpathlineto{\pgfqpoint{2.466502in}{1.620701in}}%
\pgfpathlineto{\pgfqpoint{2.471043in}{1.620701in}}%
\pgfpathlineto{\pgfqpoint{2.471043in}{1.617752in}}%
\pgfpathmoveto{\pgfqpoint{2.471043in}{1.617752in}}%
\pgfpathlineto{\pgfqpoint{2.471043in}{1.617752in}}%
\pgfpathlineto{\pgfqpoint{2.471043in}{1.620701in}}%
\pgfpathlineto{\pgfqpoint{2.475584in}{1.620701in}}%
\pgfpathlineto{\pgfqpoint{2.475584in}{1.617752in}}%
\pgfpathmoveto{\pgfqpoint{2.466502in}{1.620701in}}%
\pgfpathlineto{\pgfqpoint{2.466502in}{1.620701in}}%
\pgfpathlineto{\pgfqpoint{2.466502in}{1.623650in}}%
\pgfpathlineto{\pgfqpoint{2.471043in}{1.623650in}}%
\pgfpathlineto{\pgfqpoint{2.471043in}{1.620701in}}%
\pgfpathmoveto{\pgfqpoint{2.466502in}{1.623650in}}%
\pgfpathlineto{\pgfqpoint{2.466502in}{1.623650in}}%
\pgfpathlineto{\pgfqpoint{2.466502in}{1.626600in}}%
\pgfpathlineto{\pgfqpoint{2.471043in}{1.626600in}}%
\pgfpathlineto{\pgfqpoint{2.471043in}{1.623650in}}%
\pgfpathmoveto{\pgfqpoint{2.471043in}{1.620701in}}%
\pgfpathlineto{\pgfqpoint{2.471043in}{1.620701in}}%
\pgfpathlineto{\pgfqpoint{2.471043in}{1.623650in}}%
\pgfpathlineto{\pgfqpoint{2.475584in}{1.623650in}}%
\pgfpathlineto{\pgfqpoint{2.475584in}{1.620701in}}%
\pgfpathmoveto{\pgfqpoint{2.471043in}{1.623650in}}%
\pgfpathlineto{\pgfqpoint{2.471043in}{1.623650in}}%
\pgfpathlineto{\pgfqpoint{2.471043in}{1.626600in}}%
\pgfpathlineto{\pgfqpoint{2.475584in}{1.626600in}}%
\pgfpathlineto{\pgfqpoint{2.475584in}{1.623650in}}%
\pgfpathmoveto{\pgfqpoint{2.475584in}{1.620701in}}%
\pgfpathlineto{\pgfqpoint{2.475584in}{1.620701in}}%
\pgfpathlineto{\pgfqpoint{2.475584in}{1.623650in}}%
\pgfpathlineto{\pgfqpoint{2.480124in}{1.623650in}}%
\pgfpathlineto{\pgfqpoint{2.480124in}{1.620701in}}%
\pgfpathmoveto{\pgfqpoint{2.475584in}{1.623650in}}%
\pgfpathlineto{\pgfqpoint{2.475584in}{1.623650in}}%
\pgfpathlineto{\pgfqpoint{2.475584in}{1.626600in}}%
\pgfpathlineto{\pgfqpoint{2.480124in}{1.626600in}}%
\pgfpathlineto{\pgfqpoint{2.480124in}{1.623650in}}%
\pgfpathmoveto{\pgfqpoint{2.480124in}{1.623650in}}%
\pgfpathlineto{\pgfqpoint{2.480124in}{1.623650in}}%
\pgfpathlineto{\pgfqpoint{2.480124in}{1.626600in}}%
\pgfpathlineto{\pgfqpoint{2.484665in}{1.626600in}}%
\pgfpathlineto{\pgfqpoint{2.484665in}{1.623650in}}%
\pgfpathmoveto{\pgfqpoint{2.475584in}{1.626600in}}%
\pgfpathlineto{\pgfqpoint{2.475584in}{1.626600in}}%
\pgfpathlineto{\pgfqpoint{2.475584in}{1.629549in}}%
\pgfpathlineto{\pgfqpoint{2.480124in}{1.629549in}}%
\pgfpathlineto{\pgfqpoint{2.480124in}{1.626600in}}%
\pgfpathmoveto{\pgfqpoint{2.475584in}{1.629549in}}%
\pgfpathlineto{\pgfqpoint{2.475584in}{1.629549in}}%
\pgfpathlineto{\pgfqpoint{2.475584in}{1.632498in}}%
\pgfpathlineto{\pgfqpoint{2.480124in}{1.632498in}}%
\pgfpathlineto{\pgfqpoint{2.480124in}{1.629549in}}%
\pgfpathmoveto{\pgfqpoint{2.480124in}{1.626600in}}%
\pgfpathlineto{\pgfqpoint{2.480124in}{1.626600in}}%
\pgfpathlineto{\pgfqpoint{2.480124in}{1.629549in}}%
\pgfpathlineto{\pgfqpoint{2.484665in}{1.629549in}}%
\pgfpathlineto{\pgfqpoint{2.484665in}{1.626600in}}%
\pgfpathmoveto{\pgfqpoint{2.480124in}{1.629549in}}%
\pgfpathlineto{\pgfqpoint{2.480124in}{1.629549in}}%
\pgfpathlineto{\pgfqpoint{2.480124in}{1.632498in}}%
\pgfpathlineto{\pgfqpoint{2.484665in}{1.632498in}}%
\pgfpathlineto{\pgfqpoint{2.484665in}{1.629549in}}%
\pgfpathmoveto{\pgfqpoint{2.484665in}{1.626600in}}%
\pgfpathlineto{\pgfqpoint{2.484665in}{1.626600in}}%
\pgfpathlineto{\pgfqpoint{2.484665in}{1.629549in}}%
\pgfpathlineto{\pgfqpoint{2.489206in}{1.629549in}}%
\pgfpathlineto{\pgfqpoint{2.489206in}{1.626600in}}%
\pgfpathmoveto{\pgfqpoint{2.484665in}{1.629549in}}%
\pgfpathlineto{\pgfqpoint{2.484665in}{1.629549in}}%
\pgfpathlineto{\pgfqpoint{2.484665in}{1.632498in}}%
\pgfpathlineto{\pgfqpoint{2.489206in}{1.632498in}}%
\pgfpathlineto{\pgfqpoint{2.489206in}{1.629549in}}%
\pgfpathmoveto{\pgfqpoint{2.489206in}{1.629549in}}%
\pgfpathlineto{\pgfqpoint{2.489206in}{1.629549in}}%
\pgfpathlineto{\pgfqpoint{2.489206in}{1.632498in}}%
\pgfpathlineto{\pgfqpoint{2.493747in}{1.632498in}}%
\pgfpathlineto{\pgfqpoint{2.493747in}{1.629549in}}%
\pgfpathmoveto{\pgfqpoint{2.484665in}{1.632498in}}%
\pgfpathlineto{\pgfqpoint{2.484665in}{1.632498in}}%
\pgfpathlineto{\pgfqpoint{2.484665in}{1.635447in}}%
\pgfpathlineto{\pgfqpoint{2.489206in}{1.635447in}}%
\pgfpathlineto{\pgfqpoint{2.489206in}{1.632498in}}%
\pgfpathmoveto{\pgfqpoint{2.484665in}{1.635447in}}%
\pgfpathlineto{\pgfqpoint{2.484665in}{1.635447in}}%
\pgfpathlineto{\pgfqpoint{2.484665in}{1.638396in}}%
\pgfpathlineto{\pgfqpoint{2.489206in}{1.638396in}}%
\pgfpathlineto{\pgfqpoint{2.489206in}{1.635447in}}%
\pgfpathmoveto{\pgfqpoint{2.489206in}{1.632498in}}%
\pgfpathlineto{\pgfqpoint{2.489206in}{1.632498in}}%
\pgfpathlineto{\pgfqpoint{2.489206in}{1.635447in}}%
\pgfpathlineto{\pgfqpoint{2.493747in}{1.635447in}}%
\pgfpathlineto{\pgfqpoint{2.493747in}{1.632498in}}%
\pgfpathmoveto{\pgfqpoint{2.489206in}{1.635447in}}%
\pgfpathlineto{\pgfqpoint{2.489206in}{1.635447in}}%
\pgfpathlineto{\pgfqpoint{2.489206in}{1.638396in}}%
\pgfpathlineto{\pgfqpoint{2.493747in}{1.638396in}}%
\pgfpathlineto{\pgfqpoint{2.493747in}{1.635447in}}%
\pgfpathmoveto{\pgfqpoint{2.493747in}{1.632498in}}%
\pgfpathlineto{\pgfqpoint{2.493747in}{1.632498in}}%
\pgfpathlineto{\pgfqpoint{2.493747in}{1.635447in}}%
\pgfpathlineto{\pgfqpoint{2.498288in}{1.635447in}}%
\pgfpathlineto{\pgfqpoint{2.498288in}{1.632498in}}%
\pgfpathmoveto{\pgfqpoint{2.493747in}{1.635447in}}%
\pgfpathlineto{\pgfqpoint{2.493747in}{1.635447in}}%
\pgfpathlineto{\pgfqpoint{2.493747in}{1.638396in}}%
\pgfpathlineto{\pgfqpoint{2.498288in}{1.638396in}}%
\pgfpathlineto{\pgfqpoint{2.498288in}{1.635447in}}%
\pgfpathmoveto{\pgfqpoint{2.498288in}{1.635447in}}%
\pgfpathlineto{\pgfqpoint{2.498288in}{1.635447in}}%
\pgfpathlineto{\pgfqpoint{2.498288in}{1.638396in}}%
\pgfpathlineto{\pgfqpoint{2.502829in}{1.638396in}}%
\pgfpathlineto{\pgfqpoint{2.502829in}{1.635447in}}%
\pgfpathmoveto{\pgfqpoint{2.493747in}{1.638396in}}%
\pgfpathlineto{\pgfqpoint{2.493747in}{1.638396in}}%
\pgfpathlineto{\pgfqpoint{2.493747in}{1.641346in}}%
\pgfpathlineto{\pgfqpoint{2.498288in}{1.641346in}}%
\pgfpathlineto{\pgfqpoint{2.498288in}{1.638396in}}%
\pgfpathmoveto{\pgfqpoint{2.493747in}{1.641346in}}%
\pgfpathlineto{\pgfqpoint{2.493747in}{1.641346in}}%
\pgfpathlineto{\pgfqpoint{2.493747in}{1.644295in}}%
\pgfpathlineto{\pgfqpoint{2.498288in}{1.644295in}}%
\pgfpathlineto{\pgfqpoint{2.498288in}{1.641346in}}%
\pgfpathmoveto{\pgfqpoint{2.498288in}{1.638396in}}%
\pgfpathlineto{\pgfqpoint{2.498288in}{1.638396in}}%
\pgfpathlineto{\pgfqpoint{2.498288in}{1.641346in}}%
\pgfpathlineto{\pgfqpoint{2.502829in}{1.641346in}}%
\pgfpathlineto{\pgfqpoint{2.502829in}{1.638396in}}%
\pgfpathmoveto{\pgfqpoint{2.498288in}{1.641346in}}%
\pgfpathlineto{\pgfqpoint{2.498288in}{1.641346in}}%
\pgfpathlineto{\pgfqpoint{2.498288in}{1.644295in}}%
\pgfpathlineto{\pgfqpoint{2.502829in}{1.644295in}}%
\pgfpathlineto{\pgfqpoint{2.502829in}{1.641346in}}%
\pgfpathmoveto{\pgfqpoint{2.502829in}{1.638396in}}%
\pgfpathlineto{\pgfqpoint{2.502829in}{1.638396in}}%
\pgfpathlineto{\pgfqpoint{2.502829in}{1.641346in}}%
\pgfpathlineto{\pgfqpoint{2.507371in}{1.641346in}}%
\pgfpathlineto{\pgfqpoint{2.507371in}{1.638396in}}%
\pgfpathmoveto{\pgfqpoint{2.502829in}{1.641346in}}%
\pgfpathlineto{\pgfqpoint{2.502829in}{1.641346in}}%
\pgfpathlineto{\pgfqpoint{2.502829in}{1.644295in}}%
\pgfpathlineto{\pgfqpoint{2.507371in}{1.644295in}}%
\pgfpathlineto{\pgfqpoint{2.507371in}{1.641346in}}%
\pgfpathmoveto{\pgfqpoint{2.507371in}{1.641346in}}%
\pgfpathlineto{\pgfqpoint{2.507371in}{1.641346in}}%
\pgfpathlineto{\pgfqpoint{2.507371in}{1.644295in}}%
\pgfpathlineto{\pgfqpoint{2.511912in}{1.644295in}}%
\pgfpathlineto{\pgfqpoint{2.511912in}{1.641346in}}%
\pgfpathmoveto{\pgfqpoint{2.502829in}{1.644295in}}%
\pgfpathlineto{\pgfqpoint{2.502829in}{1.644295in}}%
\pgfpathlineto{\pgfqpoint{2.502829in}{1.647244in}}%
\pgfpathlineto{\pgfqpoint{2.507371in}{1.647244in}}%
\pgfpathlineto{\pgfqpoint{2.507371in}{1.644295in}}%
\pgfpathmoveto{\pgfqpoint{2.502829in}{1.647244in}}%
\pgfpathlineto{\pgfqpoint{2.502829in}{1.647244in}}%
\pgfpathlineto{\pgfqpoint{2.502829in}{1.650193in}}%
\pgfpathlineto{\pgfqpoint{2.507371in}{1.650193in}}%
\pgfpathlineto{\pgfqpoint{2.507371in}{1.647244in}}%
\pgfpathmoveto{\pgfqpoint{2.507371in}{1.644295in}}%
\pgfpathlineto{\pgfqpoint{2.507371in}{1.644295in}}%
\pgfpathlineto{\pgfqpoint{2.507371in}{1.647244in}}%
\pgfpathlineto{\pgfqpoint{2.511912in}{1.647244in}}%
\pgfpathlineto{\pgfqpoint{2.511912in}{1.644295in}}%
\pgfpathmoveto{\pgfqpoint{2.507371in}{1.647244in}}%
\pgfpathlineto{\pgfqpoint{2.507371in}{1.647244in}}%
\pgfpathlineto{\pgfqpoint{2.507371in}{1.650193in}}%
\pgfpathlineto{\pgfqpoint{2.511912in}{1.650193in}}%
\pgfpathlineto{\pgfqpoint{2.511912in}{1.647244in}}%
\pgfpathmoveto{\pgfqpoint{2.511912in}{1.644295in}}%
\pgfpathlineto{\pgfqpoint{2.511912in}{1.644295in}}%
\pgfpathlineto{\pgfqpoint{2.511912in}{1.647244in}}%
\pgfpathlineto{\pgfqpoint{2.516453in}{1.647244in}}%
\pgfpathlineto{\pgfqpoint{2.516453in}{1.644295in}}%
\pgfpathmoveto{\pgfqpoint{2.511912in}{1.647244in}}%
\pgfpathlineto{\pgfqpoint{2.511912in}{1.647244in}}%
\pgfpathlineto{\pgfqpoint{2.511912in}{1.650193in}}%
\pgfpathlineto{\pgfqpoint{2.516453in}{1.650193in}}%
\pgfpathlineto{\pgfqpoint{2.516453in}{1.647244in}}%
\pgfpathmoveto{\pgfqpoint{2.516453in}{1.647244in}}%
\pgfpathlineto{\pgfqpoint{2.516453in}{1.647244in}}%
\pgfpathlineto{\pgfqpoint{2.516453in}{1.650193in}}%
\pgfpathlineto{\pgfqpoint{2.520994in}{1.650193in}}%
\pgfpathlineto{\pgfqpoint{2.520994in}{1.647244in}}%
\pgfpathmoveto{\pgfqpoint{2.511912in}{1.650193in}}%
\pgfpathlineto{\pgfqpoint{2.511912in}{1.650193in}}%
\pgfpathlineto{\pgfqpoint{2.511912in}{1.653143in}}%
\pgfpathlineto{\pgfqpoint{2.516453in}{1.653143in}}%
\pgfpathlineto{\pgfqpoint{2.516453in}{1.650193in}}%
\pgfpathmoveto{\pgfqpoint{2.511912in}{1.653143in}}%
\pgfpathlineto{\pgfqpoint{2.511912in}{1.653143in}}%
\pgfpathlineto{\pgfqpoint{2.511912in}{1.656092in}}%
\pgfpathlineto{\pgfqpoint{2.516453in}{1.656092in}}%
\pgfpathlineto{\pgfqpoint{2.516453in}{1.653143in}}%
\pgfpathmoveto{\pgfqpoint{2.516453in}{1.650193in}}%
\pgfpathlineto{\pgfqpoint{2.516453in}{1.650193in}}%
\pgfpathlineto{\pgfqpoint{2.516453in}{1.653143in}}%
\pgfpathlineto{\pgfqpoint{2.520994in}{1.653143in}}%
\pgfpathlineto{\pgfqpoint{2.520994in}{1.650193in}}%
\pgfpathmoveto{\pgfqpoint{2.516453in}{1.653143in}}%
\pgfpathlineto{\pgfqpoint{2.516453in}{1.653143in}}%
\pgfpathlineto{\pgfqpoint{2.516453in}{1.656092in}}%
\pgfpathlineto{\pgfqpoint{2.520994in}{1.656092in}}%
\pgfpathlineto{\pgfqpoint{2.520994in}{1.653143in}}%
\pgfpathmoveto{\pgfqpoint{2.520994in}{1.650193in}}%
\pgfpathlineto{\pgfqpoint{2.520994in}{1.650193in}}%
\pgfpathlineto{\pgfqpoint{2.520994in}{1.653143in}}%
\pgfpathlineto{\pgfqpoint{2.525535in}{1.653143in}}%
\pgfpathlineto{\pgfqpoint{2.525535in}{1.650193in}}%
\pgfpathmoveto{\pgfqpoint{2.520994in}{1.653143in}}%
\pgfpathlineto{\pgfqpoint{2.520994in}{1.653143in}}%
\pgfpathlineto{\pgfqpoint{2.520994in}{1.656092in}}%
\pgfpathlineto{\pgfqpoint{2.525535in}{1.656092in}}%
\pgfpathlineto{\pgfqpoint{2.525535in}{1.653143in}}%
\pgfpathmoveto{\pgfqpoint{2.525535in}{1.653143in}}%
\pgfpathlineto{\pgfqpoint{2.525535in}{1.653143in}}%
\pgfpathlineto{\pgfqpoint{2.525535in}{1.656092in}}%
\pgfpathlineto{\pgfqpoint{2.530077in}{1.656092in}}%
\pgfpathlineto{\pgfqpoint{2.530077in}{1.653143in}}%
\pgfpathmoveto{\pgfqpoint{2.520994in}{1.656092in}}%
\pgfpathlineto{\pgfqpoint{2.520994in}{1.656092in}}%
\pgfpathlineto{\pgfqpoint{2.520994in}{1.659041in}}%
\pgfpathlineto{\pgfqpoint{2.525535in}{1.659041in}}%
\pgfpathlineto{\pgfqpoint{2.525535in}{1.656092in}}%
\pgfpathmoveto{\pgfqpoint{2.520994in}{1.659041in}}%
\pgfpathlineto{\pgfqpoint{2.520994in}{1.659041in}}%
\pgfpathlineto{\pgfqpoint{2.520994in}{1.661990in}}%
\pgfpathlineto{\pgfqpoint{2.525535in}{1.661990in}}%
\pgfpathlineto{\pgfqpoint{2.525535in}{1.659041in}}%
\pgfpathmoveto{\pgfqpoint{2.525535in}{1.656092in}}%
\pgfpathlineto{\pgfqpoint{2.525535in}{1.656092in}}%
\pgfpathlineto{\pgfqpoint{2.525535in}{1.659041in}}%
\pgfpathlineto{\pgfqpoint{2.530077in}{1.659041in}}%
\pgfpathlineto{\pgfqpoint{2.530077in}{1.656092in}}%
\pgfpathmoveto{\pgfqpoint{2.525535in}{1.659041in}}%
\pgfpathlineto{\pgfqpoint{2.525535in}{1.659041in}}%
\pgfpathlineto{\pgfqpoint{2.525535in}{1.661990in}}%
\pgfpathlineto{\pgfqpoint{2.530077in}{1.661990in}}%
\pgfpathlineto{\pgfqpoint{2.530077in}{1.659041in}}%
\pgfpathmoveto{\pgfqpoint{2.530077in}{1.656092in}}%
\pgfpathlineto{\pgfqpoint{2.530077in}{1.656092in}}%
\pgfpathlineto{\pgfqpoint{2.530077in}{1.659041in}}%
\pgfpathlineto{\pgfqpoint{2.534618in}{1.659041in}}%
\pgfpathlineto{\pgfqpoint{2.534618in}{1.656092in}}%
\pgfpathmoveto{\pgfqpoint{2.530077in}{1.659041in}}%
\pgfpathlineto{\pgfqpoint{2.530077in}{1.659041in}}%
\pgfpathlineto{\pgfqpoint{2.530077in}{1.661990in}}%
\pgfpathlineto{\pgfqpoint{2.534618in}{1.661990in}}%
\pgfpathlineto{\pgfqpoint{2.534618in}{1.659041in}}%
\pgfpathmoveto{\pgfqpoint{2.534618in}{1.659041in}}%
\pgfpathlineto{\pgfqpoint{2.534618in}{1.659041in}}%
\pgfpathlineto{\pgfqpoint{2.534618in}{1.661990in}}%
\pgfpathlineto{\pgfqpoint{2.539159in}{1.661990in}}%
\pgfpathlineto{\pgfqpoint{2.539159in}{1.659041in}}%
\pgfpathmoveto{\pgfqpoint{2.530077in}{1.661990in}}%
\pgfpathlineto{\pgfqpoint{2.530077in}{1.661990in}}%
\pgfpathlineto{\pgfqpoint{2.530077in}{1.664940in}}%
\pgfpathlineto{\pgfqpoint{2.534618in}{1.664940in}}%
\pgfpathlineto{\pgfqpoint{2.534618in}{1.661990in}}%
\pgfpathmoveto{\pgfqpoint{2.530077in}{1.664940in}}%
\pgfpathlineto{\pgfqpoint{2.530077in}{1.664940in}}%
\pgfpathlineto{\pgfqpoint{2.530077in}{1.667889in}}%
\pgfpathlineto{\pgfqpoint{2.534618in}{1.667889in}}%
\pgfpathlineto{\pgfqpoint{2.534618in}{1.664940in}}%
\pgfpathmoveto{\pgfqpoint{2.534618in}{1.661990in}}%
\pgfpathlineto{\pgfqpoint{2.534618in}{1.661990in}}%
\pgfpathlineto{\pgfqpoint{2.534618in}{1.664940in}}%
\pgfpathlineto{\pgfqpoint{2.539159in}{1.664940in}}%
\pgfpathlineto{\pgfqpoint{2.539159in}{1.661990in}}%
\pgfpathmoveto{\pgfqpoint{2.534618in}{1.664940in}}%
\pgfpathlineto{\pgfqpoint{2.534618in}{1.664940in}}%
\pgfpathlineto{\pgfqpoint{2.534618in}{1.667889in}}%
\pgfpathlineto{\pgfqpoint{2.539159in}{1.667889in}}%
\pgfpathlineto{\pgfqpoint{2.539159in}{1.664940in}}%
\pgfpathmoveto{\pgfqpoint{2.539159in}{1.661990in}}%
\pgfpathlineto{\pgfqpoint{2.539159in}{1.661990in}}%
\pgfpathlineto{\pgfqpoint{2.539159in}{1.664940in}}%
\pgfpathlineto{\pgfqpoint{2.543700in}{1.664940in}}%
\pgfpathlineto{\pgfqpoint{2.543700in}{1.661990in}}%
\pgfpathmoveto{\pgfqpoint{2.539159in}{1.664940in}}%
\pgfpathlineto{\pgfqpoint{2.539159in}{1.664940in}}%
\pgfpathlineto{\pgfqpoint{2.539159in}{1.667889in}}%
\pgfpathlineto{\pgfqpoint{2.543700in}{1.667889in}}%
\pgfpathlineto{\pgfqpoint{2.543700in}{1.664940in}}%
\pgfpathmoveto{\pgfqpoint{2.543700in}{1.664940in}}%
\pgfpathlineto{\pgfqpoint{2.543700in}{1.664940in}}%
\pgfpathlineto{\pgfqpoint{2.543700in}{1.667889in}}%
\pgfpathlineto{\pgfqpoint{2.548241in}{1.667889in}}%
\pgfpathlineto{\pgfqpoint{2.548241in}{1.664940in}}%
\pgfpathmoveto{\pgfqpoint{2.539159in}{1.667889in}}%
\pgfpathlineto{\pgfqpoint{2.539159in}{1.667889in}}%
\pgfpathlineto{\pgfqpoint{2.539159in}{1.670838in}}%
\pgfpathlineto{\pgfqpoint{2.543700in}{1.670838in}}%
\pgfpathlineto{\pgfqpoint{2.543700in}{1.667889in}}%
\pgfpathmoveto{\pgfqpoint{2.539159in}{1.670838in}}%
\pgfpathlineto{\pgfqpoint{2.539159in}{1.670838in}}%
\pgfpathlineto{\pgfqpoint{2.539159in}{1.673787in}}%
\pgfpathlineto{\pgfqpoint{2.543700in}{1.673787in}}%
\pgfpathlineto{\pgfqpoint{2.543700in}{1.670838in}}%
\pgfpathmoveto{\pgfqpoint{2.543700in}{1.667889in}}%
\pgfpathlineto{\pgfqpoint{2.543700in}{1.667889in}}%
\pgfpathlineto{\pgfqpoint{2.543700in}{1.670838in}}%
\pgfpathlineto{\pgfqpoint{2.548241in}{1.670838in}}%
\pgfpathlineto{\pgfqpoint{2.548241in}{1.667889in}}%
\pgfpathmoveto{\pgfqpoint{2.543700in}{1.670838in}}%
\pgfpathlineto{\pgfqpoint{2.543700in}{1.670838in}}%
\pgfpathlineto{\pgfqpoint{2.543700in}{1.673787in}}%
\pgfpathlineto{\pgfqpoint{2.548241in}{1.673787in}}%
\pgfpathlineto{\pgfqpoint{2.548241in}{1.670838in}}%
\pgfpathmoveto{\pgfqpoint{2.548241in}{1.667889in}}%
\pgfpathlineto{\pgfqpoint{2.548241in}{1.667889in}}%
\pgfpathlineto{\pgfqpoint{2.548241in}{1.670838in}}%
\pgfpathlineto{\pgfqpoint{2.552782in}{1.670838in}}%
\pgfpathlineto{\pgfqpoint{2.552782in}{1.667889in}}%
\pgfpathmoveto{\pgfqpoint{2.548241in}{1.670838in}}%
\pgfpathlineto{\pgfqpoint{2.548241in}{1.670838in}}%
\pgfpathlineto{\pgfqpoint{2.548241in}{1.673787in}}%
\pgfpathlineto{\pgfqpoint{2.552782in}{1.673787in}}%
\pgfpathlineto{\pgfqpoint{2.552782in}{1.670838in}}%
\pgfpathmoveto{\pgfqpoint{2.552782in}{1.670838in}}%
\pgfpathlineto{\pgfqpoint{2.552782in}{1.670838in}}%
\pgfpathlineto{\pgfqpoint{2.552782in}{1.673787in}}%
\pgfpathlineto{\pgfqpoint{2.557324in}{1.673787in}}%
\pgfpathlineto{\pgfqpoint{2.557324in}{1.670838in}}%
\pgfpathmoveto{\pgfqpoint{2.548241in}{1.673787in}}%
\pgfpathlineto{\pgfqpoint{2.548241in}{1.673787in}}%
\pgfpathlineto{\pgfqpoint{2.548241in}{1.676737in}}%
\pgfpathlineto{\pgfqpoint{2.552782in}{1.676737in}}%
\pgfpathlineto{\pgfqpoint{2.552782in}{1.673787in}}%
\pgfpathmoveto{\pgfqpoint{2.548241in}{1.676737in}}%
\pgfpathlineto{\pgfqpoint{2.548241in}{1.676737in}}%
\pgfpathlineto{\pgfqpoint{2.548241in}{1.679686in}}%
\pgfpathlineto{\pgfqpoint{2.552782in}{1.679686in}}%
\pgfpathlineto{\pgfqpoint{2.552782in}{1.676737in}}%
\pgfpathmoveto{\pgfqpoint{2.552782in}{1.673787in}}%
\pgfpathlineto{\pgfqpoint{2.552782in}{1.673787in}}%
\pgfpathlineto{\pgfqpoint{2.552782in}{1.676737in}}%
\pgfpathlineto{\pgfqpoint{2.557324in}{1.676737in}}%
\pgfpathlineto{\pgfqpoint{2.557324in}{1.673787in}}%
\pgfpathmoveto{\pgfqpoint{2.552782in}{1.676737in}}%
\pgfpathlineto{\pgfqpoint{2.552782in}{1.676737in}}%
\pgfpathlineto{\pgfqpoint{2.552782in}{1.679686in}}%
\pgfpathlineto{\pgfqpoint{2.557324in}{1.679686in}}%
\pgfpathlineto{\pgfqpoint{2.557324in}{1.676737in}}%
\pgfpathmoveto{\pgfqpoint{2.557324in}{1.673787in}}%
\pgfpathlineto{\pgfqpoint{2.557324in}{1.673787in}}%
\pgfpathlineto{\pgfqpoint{2.557324in}{1.676737in}}%
\pgfpathlineto{\pgfqpoint{2.561865in}{1.676737in}}%
\pgfpathlineto{\pgfqpoint{2.561865in}{1.673787in}}%
\pgfpathmoveto{\pgfqpoint{2.557324in}{1.676737in}}%
\pgfpathlineto{\pgfqpoint{2.557324in}{1.676737in}}%
\pgfpathlineto{\pgfqpoint{2.557324in}{1.679686in}}%
\pgfpathlineto{\pgfqpoint{2.561865in}{1.679686in}}%
\pgfpathlineto{\pgfqpoint{2.561865in}{1.676737in}}%
\pgfpathmoveto{\pgfqpoint{2.561865in}{1.676737in}}%
\pgfpathlineto{\pgfqpoint{2.561865in}{1.676737in}}%
\pgfpathlineto{\pgfqpoint{2.561865in}{1.679686in}}%
\pgfpathlineto{\pgfqpoint{2.566406in}{1.679686in}}%
\pgfpathlineto{\pgfqpoint{2.566406in}{1.676737in}}%
\pgfpathmoveto{\pgfqpoint{2.557324in}{1.679686in}}%
\pgfpathlineto{\pgfqpoint{2.557324in}{1.679686in}}%
\pgfpathlineto{\pgfqpoint{2.557324in}{1.682635in}}%
\pgfpathlineto{\pgfqpoint{2.561865in}{1.682635in}}%
\pgfpathlineto{\pgfqpoint{2.561865in}{1.679686in}}%
\pgfpathmoveto{\pgfqpoint{2.557324in}{1.682635in}}%
\pgfpathlineto{\pgfqpoint{2.557324in}{1.682635in}}%
\pgfpathlineto{\pgfqpoint{2.557324in}{1.685584in}}%
\pgfpathlineto{\pgfqpoint{2.561865in}{1.685584in}}%
\pgfpathlineto{\pgfqpoint{2.561865in}{1.682635in}}%
\pgfpathmoveto{\pgfqpoint{2.561865in}{1.679686in}}%
\pgfpathlineto{\pgfqpoint{2.561865in}{1.679686in}}%
\pgfpathlineto{\pgfqpoint{2.561865in}{1.682635in}}%
\pgfpathlineto{\pgfqpoint{2.566406in}{1.682635in}}%
\pgfpathlineto{\pgfqpoint{2.566406in}{1.679686in}}%
\pgfpathmoveto{\pgfqpoint{2.561865in}{1.682635in}}%
\pgfpathlineto{\pgfqpoint{2.561865in}{1.682635in}}%
\pgfpathlineto{\pgfqpoint{2.561865in}{1.685584in}}%
\pgfpathlineto{\pgfqpoint{2.566406in}{1.685584in}}%
\pgfpathlineto{\pgfqpoint{2.566406in}{1.682635in}}%
\pgfpathmoveto{\pgfqpoint{2.566406in}{1.679686in}}%
\pgfpathlineto{\pgfqpoint{2.566406in}{1.679686in}}%
\pgfpathlineto{\pgfqpoint{2.566406in}{1.682635in}}%
\pgfpathlineto{\pgfqpoint{2.570947in}{1.682635in}}%
\pgfpathlineto{\pgfqpoint{2.570947in}{1.679686in}}%
\pgfpathmoveto{\pgfqpoint{2.566406in}{1.682635in}}%
\pgfpathlineto{\pgfqpoint{2.566406in}{1.682635in}}%
\pgfpathlineto{\pgfqpoint{2.566406in}{1.685584in}}%
\pgfpathlineto{\pgfqpoint{2.570947in}{1.685584in}}%
\pgfpathlineto{\pgfqpoint{2.570947in}{1.682635in}}%
\pgfpathmoveto{\pgfqpoint{2.570947in}{1.682635in}}%
\pgfpathlineto{\pgfqpoint{2.570947in}{1.682635in}}%
\pgfpathlineto{\pgfqpoint{2.570947in}{1.685584in}}%
\pgfpathlineto{\pgfqpoint{2.575488in}{1.685584in}}%
\pgfpathlineto{\pgfqpoint{2.575488in}{1.682635in}}%
\pgfpathmoveto{\pgfqpoint{2.566406in}{1.685584in}}%
\pgfpathlineto{\pgfqpoint{2.566406in}{1.685584in}}%
\pgfpathlineto{\pgfqpoint{2.566406in}{1.688534in}}%
\pgfpathlineto{\pgfqpoint{2.570947in}{1.688534in}}%
\pgfpathlineto{\pgfqpoint{2.570947in}{1.685584in}}%
\pgfpathmoveto{\pgfqpoint{2.566406in}{1.688534in}}%
\pgfpathlineto{\pgfqpoint{2.566406in}{1.688534in}}%
\pgfpathlineto{\pgfqpoint{2.566406in}{1.691483in}}%
\pgfpathlineto{\pgfqpoint{2.570947in}{1.691483in}}%
\pgfpathlineto{\pgfqpoint{2.570947in}{1.688534in}}%
\pgfpathmoveto{\pgfqpoint{2.570947in}{1.685584in}}%
\pgfpathlineto{\pgfqpoint{2.570947in}{1.685584in}}%
\pgfpathlineto{\pgfqpoint{2.570947in}{1.688534in}}%
\pgfpathlineto{\pgfqpoint{2.575488in}{1.688534in}}%
\pgfpathlineto{\pgfqpoint{2.575488in}{1.685584in}}%
\pgfpathmoveto{\pgfqpoint{2.570947in}{1.688534in}}%
\pgfpathlineto{\pgfqpoint{2.570947in}{1.688534in}}%
\pgfpathlineto{\pgfqpoint{2.570947in}{1.691483in}}%
\pgfpathlineto{\pgfqpoint{2.575488in}{1.691483in}}%
\pgfpathlineto{\pgfqpoint{2.575488in}{1.688534in}}%
\pgfpathmoveto{\pgfqpoint{2.575488in}{1.685584in}}%
\pgfpathlineto{\pgfqpoint{2.575488in}{1.685584in}}%
\pgfpathlineto{\pgfqpoint{2.575488in}{1.688534in}}%
\pgfpathlineto{\pgfqpoint{2.580029in}{1.688534in}}%
\pgfpathlineto{\pgfqpoint{2.580029in}{1.685584in}}%
\pgfpathmoveto{\pgfqpoint{2.575488in}{1.688534in}}%
\pgfpathlineto{\pgfqpoint{2.575488in}{1.688534in}}%
\pgfpathlineto{\pgfqpoint{2.575488in}{1.691483in}}%
\pgfpathlineto{\pgfqpoint{2.580029in}{1.691483in}}%
\pgfpathlineto{\pgfqpoint{2.580029in}{1.688534in}}%
\pgfpathmoveto{\pgfqpoint{2.580029in}{1.688534in}}%
\pgfpathlineto{\pgfqpoint{2.580029in}{1.688534in}}%
\pgfpathlineto{\pgfqpoint{2.580029in}{1.691483in}}%
\pgfpathlineto{\pgfqpoint{2.584571in}{1.691483in}}%
\pgfpathlineto{\pgfqpoint{2.584571in}{1.688534in}}%
\pgfpathmoveto{\pgfqpoint{2.575488in}{1.691483in}}%
\pgfpathlineto{\pgfqpoint{2.575488in}{1.691483in}}%
\pgfpathlineto{\pgfqpoint{2.575488in}{1.694432in}}%
\pgfpathlineto{\pgfqpoint{2.580029in}{1.694432in}}%
\pgfpathlineto{\pgfqpoint{2.580029in}{1.691483in}}%
\pgfpathmoveto{\pgfqpoint{2.575488in}{1.694432in}}%
\pgfpathlineto{\pgfqpoint{2.575488in}{1.694432in}}%
\pgfpathlineto{\pgfqpoint{2.575488in}{1.697381in}}%
\pgfpathlineto{\pgfqpoint{2.580029in}{1.697381in}}%
\pgfpathlineto{\pgfqpoint{2.580029in}{1.694432in}}%
\pgfpathmoveto{\pgfqpoint{2.580029in}{1.691483in}}%
\pgfpathlineto{\pgfqpoint{2.580029in}{1.691483in}}%
\pgfpathlineto{\pgfqpoint{2.580029in}{1.694432in}}%
\pgfpathlineto{\pgfqpoint{2.584571in}{1.694432in}}%
\pgfpathlineto{\pgfqpoint{2.584571in}{1.691483in}}%
\pgfpathmoveto{\pgfqpoint{2.580029in}{1.694432in}}%
\pgfpathlineto{\pgfqpoint{2.580029in}{1.694432in}}%
\pgfpathlineto{\pgfqpoint{2.580029in}{1.697381in}}%
\pgfpathlineto{\pgfqpoint{2.584571in}{1.697381in}}%
\pgfpathlineto{\pgfqpoint{2.584571in}{1.694432in}}%
\pgfpathmoveto{\pgfqpoint{2.584571in}{1.691483in}}%
\pgfpathlineto{\pgfqpoint{2.584571in}{1.691483in}}%
\pgfpathlineto{\pgfqpoint{2.584571in}{1.694432in}}%
\pgfpathlineto{\pgfqpoint{2.589112in}{1.694432in}}%
\pgfpathlineto{\pgfqpoint{2.589112in}{1.691483in}}%
\pgfpathmoveto{\pgfqpoint{2.584571in}{1.694432in}}%
\pgfpathlineto{\pgfqpoint{2.584571in}{1.694432in}}%
\pgfpathlineto{\pgfqpoint{2.584571in}{1.697381in}}%
\pgfpathlineto{\pgfqpoint{2.589112in}{1.697381in}}%
\pgfpathlineto{\pgfqpoint{2.589112in}{1.694432in}}%
\pgfpathmoveto{\pgfqpoint{2.589112in}{1.694432in}}%
\pgfpathlineto{\pgfqpoint{2.589112in}{1.694432in}}%
\pgfpathlineto{\pgfqpoint{2.589112in}{1.697381in}}%
\pgfpathlineto{\pgfqpoint{2.593653in}{1.697381in}}%
\pgfpathlineto{\pgfqpoint{2.593653in}{1.694432in}}%
\pgfpathmoveto{\pgfqpoint{2.584571in}{1.697381in}}%
\pgfpathlineto{\pgfqpoint{2.584571in}{1.697381in}}%
\pgfpathlineto{\pgfqpoint{2.584571in}{1.700331in}}%
\pgfpathlineto{\pgfqpoint{2.589112in}{1.700331in}}%
\pgfpathlineto{\pgfqpoint{2.589112in}{1.697381in}}%
\pgfpathmoveto{\pgfqpoint{2.584571in}{1.700331in}}%
\pgfpathlineto{\pgfqpoint{2.584571in}{1.700331in}}%
\pgfpathlineto{\pgfqpoint{2.584571in}{1.703280in}}%
\pgfpathlineto{\pgfqpoint{2.589112in}{1.703280in}}%
\pgfpathlineto{\pgfqpoint{2.589112in}{1.700331in}}%
\pgfpathmoveto{\pgfqpoint{2.589112in}{1.697381in}}%
\pgfpathlineto{\pgfqpoint{2.589112in}{1.697381in}}%
\pgfpathlineto{\pgfqpoint{2.589112in}{1.700331in}}%
\pgfpathlineto{\pgfqpoint{2.593653in}{1.700331in}}%
\pgfpathlineto{\pgfqpoint{2.593653in}{1.697381in}}%
\pgfpathmoveto{\pgfqpoint{2.589112in}{1.700331in}}%
\pgfpathlineto{\pgfqpoint{2.589112in}{1.700331in}}%
\pgfpathlineto{\pgfqpoint{2.589112in}{1.703280in}}%
\pgfpathlineto{\pgfqpoint{2.593653in}{1.703280in}}%
\pgfpathlineto{\pgfqpoint{2.593653in}{1.700331in}}%
\pgfpathmoveto{\pgfqpoint{2.593653in}{1.697381in}}%
\pgfpathlineto{\pgfqpoint{2.593653in}{1.697381in}}%
\pgfpathlineto{\pgfqpoint{2.593653in}{1.700331in}}%
\pgfpathlineto{\pgfqpoint{2.598194in}{1.700331in}}%
\pgfpathlineto{\pgfqpoint{2.598194in}{1.697381in}}%
\pgfpathmoveto{\pgfqpoint{2.593653in}{1.700331in}}%
\pgfpathlineto{\pgfqpoint{2.593653in}{1.700331in}}%
\pgfpathlineto{\pgfqpoint{2.593653in}{1.703280in}}%
\pgfpathlineto{\pgfqpoint{2.598194in}{1.703280in}}%
\pgfpathlineto{\pgfqpoint{2.598194in}{1.700331in}}%
\pgfpathmoveto{\pgfqpoint{2.598194in}{1.700331in}}%
\pgfpathlineto{\pgfqpoint{2.598194in}{1.700331in}}%
\pgfpathlineto{\pgfqpoint{2.598194in}{1.703280in}}%
\pgfpathlineto{\pgfqpoint{2.602735in}{1.703280in}}%
\pgfpathlineto{\pgfqpoint{2.602735in}{1.700331in}}%
\pgfpathmoveto{\pgfqpoint{2.593653in}{1.703280in}}%
\pgfpathlineto{\pgfqpoint{2.593653in}{1.703280in}}%
\pgfpathlineto{\pgfqpoint{2.593653in}{1.706229in}}%
\pgfpathlineto{\pgfqpoint{2.598194in}{1.706229in}}%
\pgfpathlineto{\pgfqpoint{2.598194in}{1.703280in}}%
\pgfpathmoveto{\pgfqpoint{2.593653in}{1.706229in}}%
\pgfpathlineto{\pgfqpoint{2.593653in}{1.706229in}}%
\pgfpathlineto{\pgfqpoint{2.593653in}{1.709178in}}%
\pgfpathlineto{\pgfqpoint{2.598194in}{1.709178in}}%
\pgfpathlineto{\pgfqpoint{2.598194in}{1.706229in}}%
\pgfpathmoveto{\pgfqpoint{2.598194in}{1.703280in}}%
\pgfpathlineto{\pgfqpoint{2.598194in}{1.703280in}}%
\pgfpathlineto{\pgfqpoint{2.598194in}{1.706229in}}%
\pgfpathlineto{\pgfqpoint{2.602735in}{1.706229in}}%
\pgfpathlineto{\pgfqpoint{2.602735in}{1.703280in}}%
\pgfpathmoveto{\pgfqpoint{2.598194in}{1.706229in}}%
\pgfpathlineto{\pgfqpoint{2.598194in}{1.706229in}}%
\pgfpathlineto{\pgfqpoint{2.598194in}{1.709178in}}%
\pgfpathlineto{\pgfqpoint{2.602735in}{1.709178in}}%
\pgfpathlineto{\pgfqpoint{2.602735in}{1.706229in}}%
\pgfpathmoveto{\pgfqpoint{2.602735in}{1.703280in}}%
\pgfpathlineto{\pgfqpoint{2.602735in}{1.703280in}}%
\pgfpathlineto{\pgfqpoint{2.602735in}{1.706229in}}%
\pgfpathlineto{\pgfqpoint{2.607277in}{1.706229in}}%
\pgfpathlineto{\pgfqpoint{2.607277in}{1.703280in}}%
\pgfpathmoveto{\pgfqpoint{2.602735in}{1.706229in}}%
\pgfpathlineto{\pgfqpoint{2.602735in}{1.706229in}}%
\pgfpathlineto{\pgfqpoint{2.602735in}{1.709178in}}%
\pgfpathlineto{\pgfqpoint{2.607277in}{1.709178in}}%
\pgfpathlineto{\pgfqpoint{2.607277in}{1.706229in}}%
\pgfpathmoveto{\pgfqpoint{2.607277in}{1.706229in}}%
\pgfpathlineto{\pgfqpoint{2.607277in}{1.706229in}}%
\pgfpathlineto{\pgfqpoint{2.607277in}{1.709178in}}%
\pgfpathlineto{\pgfqpoint{2.611818in}{1.709178in}}%
\pgfpathlineto{\pgfqpoint{2.611818in}{1.706229in}}%
\pgfpathmoveto{\pgfqpoint{2.602735in}{1.709178in}}%
\pgfpathlineto{\pgfqpoint{2.602735in}{1.709178in}}%
\pgfpathlineto{\pgfqpoint{2.602735in}{1.712128in}}%
\pgfpathlineto{\pgfqpoint{2.607277in}{1.712128in}}%
\pgfpathlineto{\pgfqpoint{2.607277in}{1.709178in}}%
\pgfpathmoveto{\pgfqpoint{2.602735in}{1.712128in}}%
\pgfpathlineto{\pgfqpoint{2.602735in}{1.712128in}}%
\pgfpathlineto{\pgfqpoint{2.602735in}{1.715077in}}%
\pgfpathlineto{\pgfqpoint{2.607277in}{1.715077in}}%
\pgfpathlineto{\pgfqpoint{2.607277in}{1.712128in}}%
\pgfpathmoveto{\pgfqpoint{2.607277in}{1.709178in}}%
\pgfpathlineto{\pgfqpoint{2.607277in}{1.709178in}}%
\pgfpathlineto{\pgfqpoint{2.607277in}{1.712128in}}%
\pgfpathlineto{\pgfqpoint{2.611818in}{1.712128in}}%
\pgfpathlineto{\pgfqpoint{2.611818in}{1.709178in}}%
\pgfpathmoveto{\pgfqpoint{2.607277in}{1.712128in}}%
\pgfpathlineto{\pgfqpoint{2.607277in}{1.712128in}}%
\pgfpathlineto{\pgfqpoint{2.607277in}{1.715077in}}%
\pgfpathlineto{\pgfqpoint{2.611818in}{1.715077in}}%
\pgfpathlineto{\pgfqpoint{2.611818in}{1.712128in}}%
\pgfpathmoveto{\pgfqpoint{2.611818in}{1.709178in}}%
\pgfpathlineto{\pgfqpoint{2.611818in}{1.709178in}}%
\pgfpathlineto{\pgfqpoint{2.611818in}{1.712128in}}%
\pgfpathlineto{\pgfqpoint{2.616359in}{1.712128in}}%
\pgfpathlineto{\pgfqpoint{2.616359in}{1.709178in}}%
\pgfpathmoveto{\pgfqpoint{2.611818in}{1.712128in}}%
\pgfpathlineto{\pgfqpoint{2.611818in}{1.712128in}}%
\pgfpathlineto{\pgfqpoint{2.611818in}{1.715077in}}%
\pgfpathlineto{\pgfqpoint{2.616359in}{1.715077in}}%
\pgfpathlineto{\pgfqpoint{2.616359in}{1.712128in}}%
\pgfpathmoveto{\pgfqpoint{2.616359in}{1.712128in}}%
\pgfpathlineto{\pgfqpoint{2.616359in}{1.712128in}}%
\pgfpathlineto{\pgfqpoint{2.616359in}{1.715077in}}%
\pgfpathlineto{\pgfqpoint{2.620900in}{1.715077in}}%
\pgfpathlineto{\pgfqpoint{2.620900in}{1.712128in}}%
\pgfpathmoveto{\pgfqpoint{2.611818in}{1.715077in}}%
\pgfpathlineto{\pgfqpoint{2.611818in}{1.715077in}}%
\pgfpathlineto{\pgfqpoint{2.611818in}{1.718026in}}%
\pgfpathlineto{\pgfqpoint{2.616359in}{1.718026in}}%
\pgfpathlineto{\pgfqpoint{2.616359in}{1.715077in}}%
\pgfpathmoveto{\pgfqpoint{2.611818in}{1.718026in}}%
\pgfpathlineto{\pgfqpoint{2.611818in}{1.718026in}}%
\pgfpathlineto{\pgfqpoint{2.611818in}{1.720976in}}%
\pgfpathlineto{\pgfqpoint{2.616359in}{1.720976in}}%
\pgfpathlineto{\pgfqpoint{2.616359in}{1.718026in}}%
\pgfpathmoveto{\pgfqpoint{2.616359in}{1.715077in}}%
\pgfpathlineto{\pgfqpoint{2.616359in}{1.715077in}}%
\pgfpathlineto{\pgfqpoint{2.616359in}{1.718026in}}%
\pgfpathlineto{\pgfqpoint{2.620900in}{1.718026in}}%
\pgfpathlineto{\pgfqpoint{2.620900in}{1.715077in}}%
\pgfpathmoveto{\pgfqpoint{2.616359in}{1.718026in}}%
\pgfpathlineto{\pgfqpoint{2.616359in}{1.718026in}}%
\pgfpathlineto{\pgfqpoint{2.616359in}{1.720976in}}%
\pgfpathlineto{\pgfqpoint{2.620900in}{1.720976in}}%
\pgfpathlineto{\pgfqpoint{2.620900in}{1.718026in}}%
\pgfpathmoveto{\pgfqpoint{2.620900in}{1.715077in}}%
\pgfpathlineto{\pgfqpoint{2.620900in}{1.715077in}}%
\pgfpathlineto{\pgfqpoint{2.620900in}{1.718026in}}%
\pgfpathlineto{\pgfqpoint{2.625441in}{1.718026in}}%
\pgfpathlineto{\pgfqpoint{2.625441in}{1.715077in}}%
\pgfpathmoveto{\pgfqpoint{2.620900in}{1.718026in}}%
\pgfpathlineto{\pgfqpoint{2.620900in}{1.718026in}}%
\pgfpathlineto{\pgfqpoint{2.620900in}{1.720976in}}%
\pgfpathlineto{\pgfqpoint{2.625441in}{1.720976in}}%
\pgfpathlineto{\pgfqpoint{2.625441in}{1.718026in}}%
\pgfpathmoveto{\pgfqpoint{2.625441in}{1.718026in}}%
\pgfpathlineto{\pgfqpoint{2.625441in}{1.718026in}}%
\pgfpathlineto{\pgfqpoint{2.625441in}{1.720976in}}%
\pgfpathlineto{\pgfqpoint{2.629982in}{1.720976in}}%
\pgfpathlineto{\pgfqpoint{2.629982in}{1.718026in}}%
\pgfpathmoveto{\pgfqpoint{2.620900in}{1.720976in}}%
\pgfpathlineto{\pgfqpoint{2.620900in}{1.720976in}}%
\pgfpathlineto{\pgfqpoint{2.620900in}{1.723925in}}%
\pgfpathlineto{\pgfqpoint{2.625441in}{1.723925in}}%
\pgfpathlineto{\pgfqpoint{2.625441in}{1.720976in}}%
\pgfpathmoveto{\pgfqpoint{2.620900in}{1.723925in}}%
\pgfpathlineto{\pgfqpoint{2.620900in}{1.723925in}}%
\pgfpathlineto{\pgfqpoint{2.620900in}{1.726874in}}%
\pgfpathlineto{\pgfqpoint{2.625441in}{1.726874in}}%
\pgfpathlineto{\pgfqpoint{2.625441in}{1.723925in}}%
\pgfpathmoveto{\pgfqpoint{2.625441in}{1.720976in}}%
\pgfpathlineto{\pgfqpoint{2.625441in}{1.720976in}}%
\pgfpathlineto{\pgfqpoint{2.625441in}{1.723925in}}%
\pgfpathlineto{\pgfqpoint{2.629982in}{1.723925in}}%
\pgfpathlineto{\pgfqpoint{2.629982in}{1.720976in}}%
\pgfpathmoveto{\pgfqpoint{2.625441in}{1.723925in}}%
\pgfpathlineto{\pgfqpoint{2.625441in}{1.723925in}}%
\pgfpathlineto{\pgfqpoint{2.625441in}{1.726874in}}%
\pgfpathlineto{\pgfqpoint{2.629982in}{1.726874in}}%
\pgfpathlineto{\pgfqpoint{2.629982in}{1.723925in}}%
\pgfpathmoveto{\pgfqpoint{2.629982in}{1.720976in}}%
\pgfpathlineto{\pgfqpoint{2.629982in}{1.720976in}}%
\pgfpathlineto{\pgfqpoint{2.629982in}{1.723925in}}%
\pgfpathlineto{\pgfqpoint{2.634524in}{1.723925in}}%
\pgfpathlineto{\pgfqpoint{2.634524in}{1.720976in}}%
\pgfpathmoveto{\pgfqpoint{2.629982in}{1.723925in}}%
\pgfpathlineto{\pgfqpoint{2.629982in}{1.723925in}}%
\pgfpathlineto{\pgfqpoint{2.629982in}{1.726874in}}%
\pgfpathlineto{\pgfqpoint{2.634524in}{1.726874in}}%
\pgfpathlineto{\pgfqpoint{2.634524in}{1.723925in}}%
\pgfpathmoveto{\pgfqpoint{2.634524in}{1.723925in}}%
\pgfpathlineto{\pgfqpoint{2.634524in}{1.723925in}}%
\pgfpathlineto{\pgfqpoint{2.634524in}{1.726874in}}%
\pgfpathlineto{\pgfqpoint{2.639065in}{1.726874in}}%
\pgfpathlineto{\pgfqpoint{2.639065in}{1.723925in}}%
\pgfpathmoveto{\pgfqpoint{2.629982in}{1.726874in}}%
\pgfpathlineto{\pgfqpoint{2.629982in}{1.726874in}}%
\pgfpathlineto{\pgfqpoint{2.629982in}{1.729823in}}%
\pgfpathlineto{\pgfqpoint{2.634524in}{1.729823in}}%
\pgfpathlineto{\pgfqpoint{2.634524in}{1.726874in}}%
\pgfpathmoveto{\pgfqpoint{2.629982in}{1.729823in}}%
\pgfpathlineto{\pgfqpoint{2.629982in}{1.729823in}}%
\pgfpathlineto{\pgfqpoint{2.629982in}{1.732773in}}%
\pgfpathlineto{\pgfqpoint{2.634524in}{1.732773in}}%
\pgfpathlineto{\pgfqpoint{2.634524in}{1.729823in}}%
\pgfpathmoveto{\pgfqpoint{2.634524in}{1.726874in}}%
\pgfpathlineto{\pgfqpoint{2.634524in}{1.726874in}}%
\pgfpathlineto{\pgfqpoint{2.634524in}{1.729823in}}%
\pgfpathlineto{\pgfqpoint{2.639065in}{1.729823in}}%
\pgfpathlineto{\pgfqpoint{2.639065in}{1.726874in}}%
\pgfpathmoveto{\pgfqpoint{2.634524in}{1.729823in}}%
\pgfpathlineto{\pgfqpoint{2.634524in}{1.729823in}}%
\pgfpathlineto{\pgfqpoint{2.634524in}{1.732773in}}%
\pgfpathlineto{\pgfqpoint{2.639065in}{1.732773in}}%
\pgfpathlineto{\pgfqpoint{2.639065in}{1.729823in}}%
\pgfpathmoveto{\pgfqpoint{2.639065in}{1.726874in}}%
\pgfpathlineto{\pgfqpoint{2.639065in}{1.726874in}}%
\pgfpathlineto{\pgfqpoint{2.639065in}{1.729823in}}%
\pgfpathlineto{\pgfqpoint{2.643606in}{1.729823in}}%
\pgfpathlineto{\pgfqpoint{2.643606in}{1.726874in}}%
\pgfpathmoveto{\pgfqpoint{2.639065in}{1.729823in}}%
\pgfpathlineto{\pgfqpoint{2.639065in}{1.729823in}}%
\pgfpathlineto{\pgfqpoint{2.639065in}{1.732773in}}%
\pgfpathlineto{\pgfqpoint{2.643606in}{1.732773in}}%
\pgfpathlineto{\pgfqpoint{2.643606in}{1.729823in}}%
\pgfpathmoveto{\pgfqpoint{2.643606in}{1.729823in}}%
\pgfpathlineto{\pgfqpoint{2.643606in}{1.729823in}}%
\pgfpathlineto{\pgfqpoint{2.643606in}{1.732773in}}%
\pgfpathlineto{\pgfqpoint{2.648147in}{1.732773in}}%
\pgfpathlineto{\pgfqpoint{2.648147in}{1.729823in}}%
\pgfpathmoveto{\pgfqpoint{2.639065in}{1.732773in}}%
\pgfpathlineto{\pgfqpoint{2.639065in}{1.732773in}}%
\pgfpathlineto{\pgfqpoint{2.639065in}{1.735722in}}%
\pgfpathlineto{\pgfqpoint{2.643606in}{1.735722in}}%
\pgfpathlineto{\pgfqpoint{2.643606in}{1.732773in}}%
\pgfpathmoveto{\pgfqpoint{2.639065in}{1.735722in}}%
\pgfpathlineto{\pgfqpoint{2.639065in}{1.735722in}}%
\pgfpathlineto{\pgfqpoint{2.639065in}{1.738671in}}%
\pgfpathlineto{\pgfqpoint{2.643606in}{1.738671in}}%
\pgfpathlineto{\pgfqpoint{2.643606in}{1.735722in}}%
\pgfpathmoveto{\pgfqpoint{2.643606in}{1.732773in}}%
\pgfpathlineto{\pgfqpoint{2.643606in}{1.732773in}}%
\pgfpathlineto{\pgfqpoint{2.643606in}{1.735722in}}%
\pgfpathlineto{\pgfqpoint{2.648147in}{1.735722in}}%
\pgfpathlineto{\pgfqpoint{2.648147in}{1.732773in}}%
\pgfpathmoveto{\pgfqpoint{2.643606in}{1.735722in}}%
\pgfpathlineto{\pgfqpoint{2.643606in}{1.735722in}}%
\pgfpathlineto{\pgfqpoint{2.643606in}{1.738671in}}%
\pgfpathlineto{\pgfqpoint{2.648147in}{1.738671in}}%
\pgfpathlineto{\pgfqpoint{2.648147in}{1.735722in}}%
\pgfpathmoveto{\pgfqpoint{2.648147in}{1.732773in}}%
\pgfpathlineto{\pgfqpoint{2.648147in}{1.732773in}}%
\pgfpathlineto{\pgfqpoint{2.648147in}{1.735722in}}%
\pgfpathlineto{\pgfqpoint{2.652688in}{1.735722in}}%
\pgfpathlineto{\pgfqpoint{2.652688in}{1.732773in}}%
\pgfpathmoveto{\pgfqpoint{2.648147in}{1.735722in}}%
\pgfpathlineto{\pgfqpoint{2.648147in}{1.735722in}}%
\pgfpathlineto{\pgfqpoint{2.648147in}{1.738671in}}%
\pgfpathlineto{\pgfqpoint{2.652688in}{1.738671in}}%
\pgfpathlineto{\pgfqpoint{2.652688in}{1.735722in}}%
\pgfpathmoveto{\pgfqpoint{2.652688in}{1.735722in}}%
\pgfpathlineto{\pgfqpoint{2.652688in}{1.735722in}}%
\pgfpathlineto{\pgfqpoint{2.652688in}{1.738671in}}%
\pgfpathlineto{\pgfqpoint{2.657229in}{1.738671in}}%
\pgfpathlineto{\pgfqpoint{2.657229in}{1.735722in}}%
\pgfpathmoveto{\pgfqpoint{2.648147in}{1.738671in}}%
\pgfpathlineto{\pgfqpoint{2.648147in}{1.738671in}}%
\pgfpathlineto{\pgfqpoint{2.648147in}{1.741620in}}%
\pgfpathlineto{\pgfqpoint{2.652688in}{1.741620in}}%
\pgfpathlineto{\pgfqpoint{2.652688in}{1.738671in}}%
\pgfpathmoveto{\pgfqpoint{2.648147in}{1.741620in}}%
\pgfpathlineto{\pgfqpoint{2.648147in}{1.741620in}}%
\pgfpathlineto{\pgfqpoint{2.648147in}{1.744570in}}%
\pgfpathlineto{\pgfqpoint{2.652688in}{1.744570in}}%
\pgfpathlineto{\pgfqpoint{2.652688in}{1.741620in}}%
\pgfpathmoveto{\pgfqpoint{2.652688in}{1.738671in}}%
\pgfpathlineto{\pgfqpoint{2.652688in}{1.738671in}}%
\pgfpathlineto{\pgfqpoint{2.652688in}{1.741620in}}%
\pgfpathlineto{\pgfqpoint{2.657229in}{1.741620in}}%
\pgfpathlineto{\pgfqpoint{2.657229in}{1.738671in}}%
\pgfpathmoveto{\pgfqpoint{2.652688in}{1.741620in}}%
\pgfpathlineto{\pgfqpoint{2.652688in}{1.741620in}}%
\pgfpathlineto{\pgfqpoint{2.652688in}{1.744570in}}%
\pgfpathlineto{\pgfqpoint{2.657229in}{1.744570in}}%
\pgfpathlineto{\pgfqpoint{2.657229in}{1.741620in}}%
\pgfpathmoveto{\pgfqpoint{2.657229in}{1.738671in}}%
\pgfpathlineto{\pgfqpoint{2.657229in}{1.738671in}}%
\pgfpathlineto{\pgfqpoint{2.657229in}{1.741620in}}%
\pgfpathlineto{\pgfqpoint{2.661770in}{1.741620in}}%
\pgfpathlineto{\pgfqpoint{2.661770in}{1.738671in}}%
\pgfpathmoveto{\pgfqpoint{2.657229in}{1.741620in}}%
\pgfpathlineto{\pgfqpoint{2.657229in}{1.741620in}}%
\pgfpathlineto{\pgfqpoint{2.657229in}{1.744570in}}%
\pgfpathlineto{\pgfqpoint{2.661770in}{1.744570in}}%
\pgfpathlineto{\pgfqpoint{2.661770in}{1.741620in}}%
\pgfpathmoveto{\pgfqpoint{2.661770in}{1.741620in}}%
\pgfpathlineto{\pgfqpoint{2.661770in}{1.741620in}}%
\pgfpathlineto{\pgfqpoint{2.661770in}{1.744570in}}%
\pgfpathlineto{\pgfqpoint{2.666311in}{1.744570in}}%
\pgfpathlineto{\pgfqpoint{2.666311in}{1.741620in}}%
\pgfpathmoveto{\pgfqpoint{2.657229in}{1.744570in}}%
\pgfpathlineto{\pgfqpoint{2.657229in}{1.744570in}}%
\pgfpathlineto{\pgfqpoint{2.657229in}{1.747519in}}%
\pgfpathlineto{\pgfqpoint{2.661770in}{1.747519in}}%
\pgfpathlineto{\pgfqpoint{2.661770in}{1.744570in}}%
\pgfpathmoveto{\pgfqpoint{2.657229in}{1.747519in}}%
\pgfpathlineto{\pgfqpoint{2.657229in}{1.747519in}}%
\pgfpathlineto{\pgfqpoint{2.657229in}{1.750468in}}%
\pgfpathlineto{\pgfqpoint{2.661770in}{1.750468in}}%
\pgfpathlineto{\pgfqpoint{2.661770in}{1.747519in}}%
\pgfpathmoveto{\pgfqpoint{2.661770in}{1.744570in}}%
\pgfpathlineto{\pgfqpoint{2.661770in}{1.744570in}}%
\pgfpathlineto{\pgfqpoint{2.661770in}{1.747519in}}%
\pgfpathlineto{\pgfqpoint{2.666311in}{1.747519in}}%
\pgfpathlineto{\pgfqpoint{2.666311in}{1.744570in}}%
\pgfpathmoveto{\pgfqpoint{2.661770in}{1.747519in}}%
\pgfpathlineto{\pgfqpoint{2.661770in}{1.747519in}}%
\pgfpathlineto{\pgfqpoint{2.661770in}{1.750468in}}%
\pgfpathlineto{\pgfqpoint{2.666311in}{1.750468in}}%
\pgfpathlineto{\pgfqpoint{2.666311in}{1.747519in}}%
\pgfpathmoveto{\pgfqpoint{2.666311in}{1.744570in}}%
\pgfpathlineto{\pgfqpoint{2.666311in}{1.744570in}}%
\pgfpathlineto{\pgfqpoint{2.666311in}{1.747519in}}%
\pgfpathlineto{\pgfqpoint{2.670852in}{1.747519in}}%
\pgfpathlineto{\pgfqpoint{2.670852in}{1.744570in}}%
\pgfpathmoveto{\pgfqpoint{2.666311in}{1.747519in}}%
\pgfpathlineto{\pgfqpoint{2.666311in}{1.747519in}}%
\pgfpathlineto{\pgfqpoint{2.666311in}{1.750468in}}%
\pgfpathlineto{\pgfqpoint{2.670852in}{1.750468in}}%
\pgfpathlineto{\pgfqpoint{2.670852in}{1.747519in}}%
\pgfpathmoveto{\pgfqpoint{2.670852in}{1.747519in}}%
\pgfpathlineto{\pgfqpoint{2.670852in}{1.747519in}}%
\pgfpathlineto{\pgfqpoint{2.670852in}{1.750468in}}%
\pgfpathlineto{\pgfqpoint{2.675393in}{1.750468in}}%
\pgfpathlineto{\pgfqpoint{2.675393in}{1.747519in}}%
\pgfpathmoveto{\pgfqpoint{2.666311in}{1.750468in}}%
\pgfpathlineto{\pgfqpoint{2.666311in}{1.750468in}}%
\pgfpathlineto{\pgfqpoint{2.666311in}{1.753417in}}%
\pgfpathlineto{\pgfqpoint{2.670852in}{1.753417in}}%
\pgfpathlineto{\pgfqpoint{2.670852in}{1.750468in}}%
\pgfpathmoveto{\pgfqpoint{2.666311in}{1.753417in}}%
\pgfpathlineto{\pgfqpoint{2.666311in}{1.753417in}}%
\pgfpathlineto{\pgfqpoint{2.666311in}{1.756367in}}%
\pgfpathlineto{\pgfqpoint{2.670852in}{1.756367in}}%
\pgfpathlineto{\pgfqpoint{2.670852in}{1.753417in}}%
\pgfpathmoveto{\pgfqpoint{2.670852in}{1.750468in}}%
\pgfpathlineto{\pgfqpoint{2.670852in}{1.750468in}}%
\pgfpathlineto{\pgfqpoint{2.670852in}{1.753417in}}%
\pgfpathlineto{\pgfqpoint{2.675393in}{1.753417in}}%
\pgfpathlineto{\pgfqpoint{2.675393in}{1.750468in}}%
\pgfpathmoveto{\pgfqpoint{2.670852in}{1.753417in}}%
\pgfpathlineto{\pgfqpoint{2.670852in}{1.753417in}}%
\pgfpathlineto{\pgfqpoint{2.670852in}{1.756367in}}%
\pgfpathlineto{\pgfqpoint{2.675393in}{1.756367in}}%
\pgfpathlineto{\pgfqpoint{2.675393in}{1.753417in}}%
\pgfpathmoveto{\pgfqpoint{2.675393in}{1.750468in}}%
\pgfpathlineto{\pgfqpoint{2.675393in}{1.750468in}}%
\pgfpathlineto{\pgfqpoint{2.675393in}{1.753417in}}%
\pgfpathlineto{\pgfqpoint{2.679934in}{1.753417in}}%
\pgfpathlineto{\pgfqpoint{2.679934in}{1.750468in}}%
\pgfpathmoveto{\pgfqpoint{2.675393in}{1.753417in}}%
\pgfpathlineto{\pgfqpoint{2.675393in}{1.753417in}}%
\pgfpathlineto{\pgfqpoint{2.675393in}{1.756367in}}%
\pgfpathlineto{\pgfqpoint{2.679934in}{1.756367in}}%
\pgfpathlineto{\pgfqpoint{2.679934in}{1.753417in}}%
\pgfpathmoveto{\pgfqpoint{2.679934in}{1.753417in}}%
\pgfpathlineto{\pgfqpoint{2.679934in}{1.753417in}}%
\pgfpathlineto{\pgfqpoint{2.679934in}{1.756367in}}%
\pgfpathlineto{\pgfqpoint{2.684475in}{1.756367in}}%
\pgfpathlineto{\pgfqpoint{2.684475in}{1.753417in}}%
\pgfpathmoveto{\pgfqpoint{2.675393in}{1.756367in}}%
\pgfpathlineto{\pgfqpoint{2.675393in}{1.756367in}}%
\pgfpathlineto{\pgfqpoint{2.675393in}{1.759316in}}%
\pgfpathlineto{\pgfqpoint{2.679934in}{1.759316in}}%
\pgfpathlineto{\pgfqpoint{2.679934in}{1.756367in}}%
\pgfpathmoveto{\pgfqpoint{2.675393in}{1.759316in}}%
\pgfpathlineto{\pgfqpoint{2.675393in}{1.759316in}}%
\pgfpathlineto{\pgfqpoint{2.675393in}{1.762265in}}%
\pgfpathlineto{\pgfqpoint{2.679934in}{1.762265in}}%
\pgfpathlineto{\pgfqpoint{2.679934in}{1.759316in}}%
\pgfpathmoveto{\pgfqpoint{2.679934in}{1.756367in}}%
\pgfpathlineto{\pgfqpoint{2.679934in}{1.756367in}}%
\pgfpathlineto{\pgfqpoint{2.679934in}{1.759316in}}%
\pgfpathlineto{\pgfqpoint{2.684475in}{1.759316in}}%
\pgfpathlineto{\pgfqpoint{2.684475in}{1.756367in}}%
\pgfpathmoveto{\pgfqpoint{2.679934in}{1.759316in}}%
\pgfpathlineto{\pgfqpoint{2.679934in}{1.759316in}}%
\pgfpathlineto{\pgfqpoint{2.679934in}{1.762265in}}%
\pgfpathlineto{\pgfqpoint{2.684475in}{1.762265in}}%
\pgfpathlineto{\pgfqpoint{2.684475in}{1.759316in}}%
\pgfpathmoveto{\pgfqpoint{2.684475in}{1.756367in}}%
\pgfpathlineto{\pgfqpoint{2.684475in}{1.756367in}}%
\pgfpathlineto{\pgfqpoint{2.684475in}{1.759316in}}%
\pgfpathlineto{\pgfqpoint{2.689016in}{1.759316in}}%
\pgfpathlineto{\pgfqpoint{2.689016in}{1.756367in}}%
\pgfpathmoveto{\pgfqpoint{2.684475in}{1.759316in}}%
\pgfpathlineto{\pgfqpoint{2.684475in}{1.759316in}}%
\pgfpathlineto{\pgfqpoint{2.684475in}{1.762265in}}%
\pgfpathlineto{\pgfqpoint{2.689016in}{1.762265in}}%
\pgfpathlineto{\pgfqpoint{2.689016in}{1.759316in}}%
\pgfpathmoveto{\pgfqpoint{2.689016in}{1.759316in}}%
\pgfpathlineto{\pgfqpoint{2.689016in}{1.759316in}}%
\pgfpathlineto{\pgfqpoint{2.689016in}{1.762265in}}%
\pgfpathlineto{\pgfqpoint{2.693557in}{1.762265in}}%
\pgfpathlineto{\pgfqpoint{2.693557in}{1.759316in}}%
\pgfpathmoveto{\pgfqpoint{2.684475in}{1.762265in}}%
\pgfpathlineto{\pgfqpoint{2.684475in}{1.762265in}}%
\pgfpathlineto{\pgfqpoint{2.684475in}{1.765214in}}%
\pgfpathlineto{\pgfqpoint{2.689016in}{1.765214in}}%
\pgfpathlineto{\pgfqpoint{2.689016in}{1.762265in}}%
\pgfpathmoveto{\pgfqpoint{2.684475in}{1.765214in}}%
\pgfpathlineto{\pgfqpoint{2.684475in}{1.765214in}}%
\pgfpathlineto{\pgfqpoint{2.684475in}{1.768164in}}%
\pgfpathlineto{\pgfqpoint{2.689016in}{1.768164in}}%
\pgfpathlineto{\pgfqpoint{2.689016in}{1.765214in}}%
\pgfpathmoveto{\pgfqpoint{2.689016in}{1.762265in}}%
\pgfpathlineto{\pgfqpoint{2.689016in}{1.762265in}}%
\pgfpathlineto{\pgfqpoint{2.689016in}{1.765214in}}%
\pgfpathlineto{\pgfqpoint{2.693557in}{1.765214in}}%
\pgfpathlineto{\pgfqpoint{2.693557in}{1.762265in}}%
\pgfpathmoveto{\pgfqpoint{2.689016in}{1.765214in}}%
\pgfpathlineto{\pgfqpoint{2.689016in}{1.765214in}}%
\pgfpathlineto{\pgfqpoint{2.689016in}{1.768164in}}%
\pgfpathlineto{\pgfqpoint{2.693557in}{1.768164in}}%
\pgfpathlineto{\pgfqpoint{2.693557in}{1.765214in}}%
\pgfpathmoveto{\pgfqpoint{2.693557in}{1.762265in}}%
\pgfpathlineto{\pgfqpoint{2.693557in}{1.762265in}}%
\pgfpathlineto{\pgfqpoint{2.693557in}{1.765214in}}%
\pgfpathlineto{\pgfqpoint{2.698098in}{1.765214in}}%
\pgfpathlineto{\pgfqpoint{2.698098in}{1.762265in}}%
\pgfpathmoveto{\pgfqpoint{2.693557in}{1.765214in}}%
\pgfpathlineto{\pgfqpoint{2.693557in}{1.765214in}}%
\pgfpathlineto{\pgfqpoint{2.693557in}{1.768164in}}%
\pgfpathlineto{\pgfqpoint{2.698098in}{1.768164in}}%
\pgfpathlineto{\pgfqpoint{2.698098in}{1.765214in}}%
\pgfpathmoveto{\pgfqpoint{2.698098in}{1.765214in}}%
\pgfpathlineto{\pgfqpoint{2.698098in}{1.765214in}}%
\pgfpathlineto{\pgfqpoint{2.698098in}{1.768164in}}%
\pgfpathlineto{\pgfqpoint{2.702639in}{1.768164in}}%
\pgfpathlineto{\pgfqpoint{2.702639in}{1.765214in}}%
\pgfpathmoveto{\pgfqpoint{2.693557in}{1.768164in}}%
\pgfpathlineto{\pgfqpoint{2.693557in}{1.768164in}}%
\pgfpathlineto{\pgfqpoint{2.693557in}{1.771113in}}%
\pgfpathlineto{\pgfqpoint{2.698098in}{1.771113in}}%
\pgfpathlineto{\pgfqpoint{2.698098in}{1.768164in}}%
\pgfpathmoveto{\pgfqpoint{2.693557in}{1.771113in}}%
\pgfpathlineto{\pgfqpoint{2.693557in}{1.771113in}}%
\pgfpathlineto{\pgfqpoint{2.693557in}{1.774062in}}%
\pgfpathlineto{\pgfqpoint{2.698098in}{1.774062in}}%
\pgfpathlineto{\pgfqpoint{2.698098in}{1.771113in}}%
\pgfpathmoveto{\pgfqpoint{2.698098in}{1.768164in}}%
\pgfpathlineto{\pgfqpoint{2.698098in}{1.768164in}}%
\pgfpathlineto{\pgfqpoint{2.698098in}{1.771113in}}%
\pgfpathlineto{\pgfqpoint{2.702639in}{1.771113in}}%
\pgfpathlineto{\pgfqpoint{2.702639in}{1.768164in}}%
\pgfpathmoveto{\pgfqpoint{2.698098in}{1.771113in}}%
\pgfpathlineto{\pgfqpoint{2.698098in}{1.771113in}}%
\pgfpathlineto{\pgfqpoint{2.698098in}{1.774062in}}%
\pgfpathlineto{\pgfqpoint{2.702639in}{1.774062in}}%
\pgfpathlineto{\pgfqpoint{2.702639in}{1.771113in}}%
\pgfpathmoveto{\pgfqpoint{2.702639in}{1.768164in}}%
\pgfpathlineto{\pgfqpoint{2.702639in}{1.768164in}}%
\pgfpathlineto{\pgfqpoint{2.702639in}{1.771113in}}%
\pgfpathlineto{\pgfqpoint{2.707180in}{1.771113in}}%
\pgfpathlineto{\pgfqpoint{2.707180in}{1.768164in}}%
\pgfpathmoveto{\pgfqpoint{2.702639in}{1.771113in}}%
\pgfpathlineto{\pgfqpoint{2.702639in}{1.771113in}}%
\pgfpathlineto{\pgfqpoint{2.702639in}{1.774062in}}%
\pgfpathlineto{\pgfqpoint{2.707180in}{1.774062in}}%
\pgfpathlineto{\pgfqpoint{2.707180in}{1.771113in}}%
\pgfpathmoveto{\pgfqpoint{2.707180in}{1.771113in}}%
\pgfpathlineto{\pgfqpoint{2.707180in}{1.771113in}}%
\pgfpathlineto{\pgfqpoint{2.707180in}{1.774062in}}%
\pgfpathlineto{\pgfqpoint{2.711721in}{1.774062in}}%
\pgfpathlineto{\pgfqpoint{2.711721in}{1.771113in}}%
\pgfpathmoveto{\pgfqpoint{2.702639in}{1.774062in}}%
\pgfpathlineto{\pgfqpoint{2.702639in}{1.774062in}}%
\pgfpathlineto{\pgfqpoint{2.702639in}{1.777011in}}%
\pgfpathlineto{\pgfqpoint{2.707180in}{1.777011in}}%
\pgfpathlineto{\pgfqpoint{2.707180in}{1.774062in}}%
\pgfpathmoveto{\pgfqpoint{2.702639in}{1.777011in}}%
\pgfpathlineto{\pgfqpoint{2.702639in}{1.777011in}}%
\pgfpathlineto{\pgfqpoint{2.702639in}{1.779961in}}%
\pgfpathlineto{\pgfqpoint{2.707180in}{1.779961in}}%
\pgfpathlineto{\pgfqpoint{2.707180in}{1.777011in}}%
\pgfpathmoveto{\pgfqpoint{2.707180in}{1.774062in}}%
\pgfpathlineto{\pgfqpoint{2.707180in}{1.774062in}}%
\pgfpathlineto{\pgfqpoint{2.707180in}{1.777011in}}%
\pgfpathlineto{\pgfqpoint{2.711721in}{1.777011in}}%
\pgfpathlineto{\pgfqpoint{2.711721in}{1.774062in}}%
\pgfpathmoveto{\pgfqpoint{2.707180in}{1.777011in}}%
\pgfpathlineto{\pgfqpoint{2.707180in}{1.777011in}}%
\pgfpathlineto{\pgfqpoint{2.707180in}{1.779961in}}%
\pgfpathlineto{\pgfqpoint{2.711721in}{1.779961in}}%
\pgfpathlineto{\pgfqpoint{2.711721in}{1.777011in}}%
\pgfpathmoveto{\pgfqpoint{2.711721in}{1.774062in}}%
\pgfpathlineto{\pgfqpoint{2.711721in}{1.774062in}}%
\pgfpathlineto{\pgfqpoint{2.711721in}{1.777011in}}%
\pgfpathlineto{\pgfqpoint{2.716262in}{1.777011in}}%
\pgfpathlineto{\pgfqpoint{2.716262in}{1.774062in}}%
\pgfpathmoveto{\pgfqpoint{2.711721in}{1.777011in}}%
\pgfpathlineto{\pgfqpoint{2.711721in}{1.777011in}}%
\pgfpathlineto{\pgfqpoint{2.711721in}{1.779961in}}%
\pgfpathlineto{\pgfqpoint{2.716262in}{1.779961in}}%
\pgfpathlineto{\pgfqpoint{2.716262in}{1.777011in}}%
\pgfpathmoveto{\pgfqpoint{2.716262in}{1.777011in}}%
\pgfpathlineto{\pgfqpoint{2.716262in}{1.777011in}}%
\pgfpathlineto{\pgfqpoint{2.716262in}{1.779961in}}%
\pgfpathlineto{\pgfqpoint{2.720803in}{1.779961in}}%
\pgfpathlineto{\pgfqpoint{2.720803in}{1.777011in}}%
\pgfpathmoveto{\pgfqpoint{2.711721in}{1.779961in}}%
\pgfpathlineto{\pgfqpoint{2.711721in}{1.779961in}}%
\pgfpathlineto{\pgfqpoint{2.711721in}{1.782910in}}%
\pgfpathlineto{\pgfqpoint{2.716262in}{1.782910in}}%
\pgfpathlineto{\pgfqpoint{2.716262in}{1.779961in}}%
\pgfpathmoveto{\pgfqpoint{2.711721in}{1.782910in}}%
\pgfpathlineto{\pgfqpoint{2.711721in}{1.782910in}}%
\pgfpathlineto{\pgfqpoint{2.711721in}{1.785859in}}%
\pgfpathlineto{\pgfqpoint{2.716262in}{1.785859in}}%
\pgfpathlineto{\pgfqpoint{2.716262in}{1.782910in}}%
\pgfpathmoveto{\pgfqpoint{2.716262in}{1.779961in}}%
\pgfpathlineto{\pgfqpoint{2.716262in}{1.779961in}}%
\pgfpathlineto{\pgfqpoint{2.716262in}{1.782910in}}%
\pgfpathlineto{\pgfqpoint{2.720803in}{1.782910in}}%
\pgfpathlineto{\pgfqpoint{2.720803in}{1.779961in}}%
\pgfpathmoveto{\pgfqpoint{2.716262in}{1.782910in}}%
\pgfpathlineto{\pgfqpoint{2.716262in}{1.782910in}}%
\pgfpathlineto{\pgfqpoint{2.716262in}{1.785859in}}%
\pgfpathlineto{\pgfqpoint{2.720803in}{1.785859in}}%
\pgfpathlineto{\pgfqpoint{2.720803in}{1.782910in}}%
\pgfpathmoveto{\pgfqpoint{2.720803in}{1.779961in}}%
\pgfpathlineto{\pgfqpoint{2.720803in}{1.779961in}}%
\pgfpathlineto{\pgfqpoint{2.720803in}{1.782910in}}%
\pgfpathlineto{\pgfqpoint{2.725344in}{1.782910in}}%
\pgfpathlineto{\pgfqpoint{2.725344in}{1.779961in}}%
\pgfpathmoveto{\pgfqpoint{2.720803in}{1.782910in}}%
\pgfpathlineto{\pgfqpoint{2.720803in}{1.782910in}}%
\pgfpathlineto{\pgfqpoint{2.720803in}{1.785859in}}%
\pgfpathlineto{\pgfqpoint{2.725344in}{1.785859in}}%
\pgfpathlineto{\pgfqpoint{2.725344in}{1.782910in}}%
\pgfpathmoveto{\pgfqpoint{2.725344in}{1.782910in}}%
\pgfpathlineto{\pgfqpoint{2.725344in}{1.782910in}}%
\pgfpathlineto{\pgfqpoint{2.725344in}{1.785859in}}%
\pgfpathlineto{\pgfqpoint{2.729885in}{1.785859in}}%
\pgfpathlineto{\pgfqpoint{2.729885in}{1.782910in}}%
\pgfpathmoveto{\pgfqpoint{2.720803in}{1.785859in}}%
\pgfpathlineto{\pgfqpoint{2.720803in}{1.785859in}}%
\pgfpathlineto{\pgfqpoint{2.720803in}{1.788809in}}%
\pgfpathlineto{\pgfqpoint{2.725344in}{1.788809in}}%
\pgfpathlineto{\pgfqpoint{2.725344in}{1.785859in}}%
\pgfpathmoveto{\pgfqpoint{2.720803in}{1.788809in}}%
\pgfpathlineto{\pgfqpoint{2.720803in}{1.788809in}}%
\pgfpathlineto{\pgfqpoint{2.720803in}{1.791758in}}%
\pgfpathlineto{\pgfqpoint{2.725344in}{1.791758in}}%
\pgfpathlineto{\pgfqpoint{2.725344in}{1.788809in}}%
\pgfpathmoveto{\pgfqpoint{2.725344in}{1.785859in}}%
\pgfpathlineto{\pgfqpoint{2.725344in}{1.785859in}}%
\pgfpathlineto{\pgfqpoint{2.725344in}{1.788809in}}%
\pgfpathlineto{\pgfqpoint{2.729885in}{1.788809in}}%
\pgfpathlineto{\pgfqpoint{2.729885in}{1.785859in}}%
\pgfpathmoveto{\pgfqpoint{2.725344in}{1.788809in}}%
\pgfpathlineto{\pgfqpoint{2.725344in}{1.788809in}}%
\pgfpathlineto{\pgfqpoint{2.725344in}{1.791758in}}%
\pgfpathlineto{\pgfqpoint{2.729885in}{1.791758in}}%
\pgfpathlineto{\pgfqpoint{2.729885in}{1.788809in}}%
\pgfpathmoveto{\pgfqpoint{2.729885in}{1.785859in}}%
\pgfpathlineto{\pgfqpoint{2.729885in}{1.785859in}}%
\pgfpathlineto{\pgfqpoint{2.729885in}{1.788809in}}%
\pgfpathlineto{\pgfqpoint{2.734426in}{1.788809in}}%
\pgfpathlineto{\pgfqpoint{2.734426in}{1.785859in}}%
\pgfpathmoveto{\pgfqpoint{2.729885in}{1.788809in}}%
\pgfpathlineto{\pgfqpoint{2.729885in}{1.788809in}}%
\pgfpathlineto{\pgfqpoint{2.729885in}{1.791758in}}%
\pgfpathlineto{\pgfqpoint{2.734426in}{1.791758in}}%
\pgfpathlineto{\pgfqpoint{2.734426in}{1.788809in}}%
\pgfpathmoveto{\pgfqpoint{2.734426in}{1.788809in}}%
\pgfpathlineto{\pgfqpoint{2.734426in}{1.788809in}}%
\pgfpathlineto{\pgfqpoint{2.734426in}{1.791758in}}%
\pgfpathlineto{\pgfqpoint{2.738967in}{1.791758in}}%
\pgfpathlineto{\pgfqpoint{2.738967in}{1.788809in}}%
\pgfpathmoveto{\pgfqpoint{2.729885in}{1.791758in}}%
\pgfpathlineto{\pgfqpoint{2.729885in}{1.791758in}}%
\pgfpathlineto{\pgfqpoint{2.729885in}{1.794707in}}%
\pgfpathlineto{\pgfqpoint{2.734426in}{1.794707in}}%
\pgfpathlineto{\pgfqpoint{2.734426in}{1.791758in}}%
\pgfpathmoveto{\pgfqpoint{2.729885in}{1.794707in}}%
\pgfpathlineto{\pgfqpoint{2.729885in}{1.794707in}}%
\pgfpathlineto{\pgfqpoint{2.729885in}{1.797656in}}%
\pgfpathlineto{\pgfqpoint{2.734426in}{1.797656in}}%
\pgfpathlineto{\pgfqpoint{2.734426in}{1.794707in}}%
\pgfpathmoveto{\pgfqpoint{2.734426in}{1.791758in}}%
\pgfpathlineto{\pgfqpoint{2.734426in}{1.791758in}}%
\pgfpathlineto{\pgfqpoint{2.734426in}{1.794707in}}%
\pgfpathlineto{\pgfqpoint{2.738967in}{1.794707in}}%
\pgfpathlineto{\pgfqpoint{2.738967in}{1.791758in}}%
\pgfpathmoveto{\pgfqpoint{2.734426in}{1.794707in}}%
\pgfpathlineto{\pgfqpoint{2.734426in}{1.794707in}}%
\pgfpathlineto{\pgfqpoint{2.734426in}{1.797656in}}%
\pgfpathlineto{\pgfqpoint{2.738967in}{1.797656in}}%
\pgfpathlineto{\pgfqpoint{2.738967in}{1.794707in}}%
\pgfpathmoveto{\pgfqpoint{2.738967in}{1.791758in}}%
\pgfpathlineto{\pgfqpoint{2.738967in}{1.791758in}}%
\pgfpathlineto{\pgfqpoint{2.738967in}{1.794707in}}%
\pgfpathlineto{\pgfqpoint{2.743508in}{1.794707in}}%
\pgfpathlineto{\pgfqpoint{2.743508in}{1.791758in}}%
\pgfpathmoveto{\pgfqpoint{2.738967in}{1.794707in}}%
\pgfpathlineto{\pgfqpoint{2.738967in}{1.794707in}}%
\pgfpathlineto{\pgfqpoint{2.738967in}{1.797656in}}%
\pgfpathlineto{\pgfqpoint{2.743508in}{1.797656in}}%
\pgfpathlineto{\pgfqpoint{2.743508in}{1.794707in}}%
\pgfpathmoveto{\pgfqpoint{2.743508in}{1.794707in}}%
\pgfpathlineto{\pgfqpoint{2.743508in}{1.794707in}}%
\pgfpathlineto{\pgfqpoint{2.743508in}{1.797656in}}%
\pgfpathlineto{\pgfqpoint{2.748049in}{1.797656in}}%
\pgfpathlineto{\pgfqpoint{2.748049in}{1.794707in}}%
\pgfpathmoveto{\pgfqpoint{2.738967in}{1.797656in}}%
\pgfpathlineto{\pgfqpoint{2.738967in}{1.797656in}}%
\pgfpathlineto{\pgfqpoint{2.738967in}{1.800606in}}%
\pgfpathlineto{\pgfqpoint{2.743508in}{1.800606in}}%
\pgfpathlineto{\pgfqpoint{2.743508in}{1.797656in}}%
\pgfpathmoveto{\pgfqpoint{2.738967in}{1.800606in}}%
\pgfpathlineto{\pgfqpoint{2.738967in}{1.800606in}}%
\pgfpathlineto{\pgfqpoint{2.738967in}{1.803555in}}%
\pgfpathlineto{\pgfqpoint{2.743508in}{1.803555in}}%
\pgfpathlineto{\pgfqpoint{2.743508in}{1.800606in}}%
\pgfpathmoveto{\pgfqpoint{2.743508in}{1.797656in}}%
\pgfpathlineto{\pgfqpoint{2.743508in}{1.797656in}}%
\pgfpathlineto{\pgfqpoint{2.743508in}{1.800606in}}%
\pgfpathlineto{\pgfqpoint{2.748049in}{1.800606in}}%
\pgfpathlineto{\pgfqpoint{2.748049in}{1.797656in}}%
\pgfpathmoveto{\pgfqpoint{2.743508in}{1.800606in}}%
\pgfpathlineto{\pgfqpoint{2.743508in}{1.800606in}}%
\pgfpathlineto{\pgfqpoint{2.743508in}{1.803555in}}%
\pgfpathlineto{\pgfqpoint{2.748049in}{1.803555in}}%
\pgfpathlineto{\pgfqpoint{2.748049in}{1.800606in}}%
\pgfpathmoveto{\pgfqpoint{2.748049in}{1.797656in}}%
\pgfpathlineto{\pgfqpoint{2.748049in}{1.797656in}}%
\pgfpathlineto{\pgfqpoint{2.748049in}{1.800606in}}%
\pgfpathlineto{\pgfqpoint{2.752590in}{1.800606in}}%
\pgfpathlineto{\pgfqpoint{2.752590in}{1.797656in}}%
\pgfpathmoveto{\pgfqpoint{2.748049in}{1.800606in}}%
\pgfpathlineto{\pgfqpoint{2.748049in}{1.800606in}}%
\pgfpathlineto{\pgfqpoint{2.748049in}{1.803555in}}%
\pgfpathlineto{\pgfqpoint{2.752590in}{1.803555in}}%
\pgfpathlineto{\pgfqpoint{2.752590in}{1.800606in}}%
\pgfpathmoveto{\pgfqpoint{2.752590in}{1.800606in}}%
\pgfpathlineto{\pgfqpoint{2.752590in}{1.800606in}}%
\pgfpathlineto{\pgfqpoint{2.752590in}{1.803555in}}%
\pgfpathlineto{\pgfqpoint{2.757131in}{1.803555in}}%
\pgfpathlineto{\pgfqpoint{2.757131in}{1.800606in}}%
\pgfpathmoveto{\pgfqpoint{2.748049in}{1.803555in}}%
\pgfpathlineto{\pgfqpoint{2.748049in}{1.803555in}}%
\pgfpathlineto{\pgfqpoint{2.748049in}{1.806504in}}%
\pgfpathlineto{\pgfqpoint{2.752590in}{1.806504in}}%
\pgfpathlineto{\pgfqpoint{2.752590in}{1.803555in}}%
\pgfpathmoveto{\pgfqpoint{2.748049in}{1.806504in}}%
\pgfpathlineto{\pgfqpoint{2.748049in}{1.806504in}}%
\pgfpathlineto{\pgfqpoint{2.748049in}{1.809453in}}%
\pgfpathlineto{\pgfqpoint{2.752590in}{1.809453in}}%
\pgfpathlineto{\pgfqpoint{2.752590in}{1.806504in}}%
\pgfpathmoveto{\pgfqpoint{2.752590in}{1.803555in}}%
\pgfpathlineto{\pgfqpoint{2.752590in}{1.803555in}}%
\pgfpathlineto{\pgfqpoint{2.752590in}{1.806504in}}%
\pgfpathlineto{\pgfqpoint{2.757131in}{1.806504in}}%
\pgfpathlineto{\pgfqpoint{2.757131in}{1.803555in}}%
\pgfpathmoveto{\pgfqpoint{2.752590in}{1.806504in}}%
\pgfpathlineto{\pgfqpoint{2.752590in}{1.806504in}}%
\pgfpathlineto{\pgfqpoint{2.752590in}{1.809453in}}%
\pgfpathlineto{\pgfqpoint{2.757131in}{1.809453in}}%
\pgfpathlineto{\pgfqpoint{2.757131in}{1.806504in}}%
\pgfpathmoveto{\pgfqpoint{2.757131in}{1.803555in}}%
\pgfpathlineto{\pgfqpoint{2.757131in}{1.803555in}}%
\pgfpathlineto{\pgfqpoint{2.757131in}{1.806504in}}%
\pgfpathlineto{\pgfqpoint{2.761672in}{1.806504in}}%
\pgfpathlineto{\pgfqpoint{2.761672in}{1.803555in}}%
\pgfpathmoveto{\pgfqpoint{2.757131in}{1.806504in}}%
\pgfpathlineto{\pgfqpoint{2.757131in}{1.806504in}}%
\pgfpathlineto{\pgfqpoint{2.757131in}{1.809453in}}%
\pgfpathlineto{\pgfqpoint{2.761672in}{1.809453in}}%
\pgfpathlineto{\pgfqpoint{2.761672in}{1.806504in}}%
\pgfpathmoveto{\pgfqpoint{2.761672in}{1.806504in}}%
\pgfpathlineto{\pgfqpoint{2.761672in}{1.806504in}}%
\pgfpathlineto{\pgfqpoint{2.761672in}{1.809453in}}%
\pgfpathlineto{\pgfqpoint{2.766213in}{1.809453in}}%
\pgfpathlineto{\pgfqpoint{2.766213in}{1.806504in}}%
\pgfpathmoveto{\pgfqpoint{2.757131in}{1.809453in}}%
\pgfpathlineto{\pgfqpoint{2.757131in}{1.809453in}}%
\pgfpathlineto{\pgfqpoint{2.757131in}{1.812403in}}%
\pgfpathlineto{\pgfqpoint{2.761672in}{1.812403in}}%
\pgfpathlineto{\pgfqpoint{2.761672in}{1.809453in}}%
\pgfpathmoveto{\pgfqpoint{2.757131in}{1.812403in}}%
\pgfpathlineto{\pgfqpoint{2.757131in}{1.812403in}}%
\pgfpathlineto{\pgfqpoint{2.757131in}{1.815352in}}%
\pgfpathlineto{\pgfqpoint{2.761672in}{1.815352in}}%
\pgfpathlineto{\pgfqpoint{2.761672in}{1.812403in}}%
\pgfpathmoveto{\pgfqpoint{2.761672in}{1.809453in}}%
\pgfpathlineto{\pgfqpoint{2.761672in}{1.809453in}}%
\pgfpathlineto{\pgfqpoint{2.761672in}{1.812403in}}%
\pgfpathlineto{\pgfqpoint{2.766213in}{1.812403in}}%
\pgfpathlineto{\pgfqpoint{2.766213in}{1.809453in}}%
\pgfpathmoveto{\pgfqpoint{2.761672in}{1.812403in}}%
\pgfpathlineto{\pgfqpoint{2.761672in}{1.812403in}}%
\pgfpathlineto{\pgfqpoint{2.761672in}{1.815352in}}%
\pgfpathlineto{\pgfqpoint{2.766213in}{1.815352in}}%
\pgfpathlineto{\pgfqpoint{2.766213in}{1.812403in}}%
\pgfpathmoveto{\pgfqpoint{2.766213in}{1.809453in}}%
\pgfpathlineto{\pgfqpoint{2.766213in}{1.809453in}}%
\pgfpathlineto{\pgfqpoint{2.766213in}{1.812403in}}%
\pgfpathlineto{\pgfqpoint{2.770754in}{1.812403in}}%
\pgfpathlineto{\pgfqpoint{2.770754in}{1.809453in}}%
\pgfpathmoveto{\pgfqpoint{2.766213in}{1.812403in}}%
\pgfpathlineto{\pgfqpoint{2.766213in}{1.812403in}}%
\pgfpathlineto{\pgfqpoint{2.766213in}{1.815352in}}%
\pgfpathlineto{\pgfqpoint{2.770754in}{1.815352in}}%
\pgfpathlineto{\pgfqpoint{2.770754in}{1.812403in}}%
\pgfpathmoveto{\pgfqpoint{2.770754in}{1.812403in}}%
\pgfpathlineto{\pgfqpoint{2.770754in}{1.812403in}}%
\pgfpathlineto{\pgfqpoint{2.770754in}{1.815352in}}%
\pgfpathlineto{\pgfqpoint{2.775295in}{1.815352in}}%
\pgfpathlineto{\pgfqpoint{2.775295in}{1.812403in}}%
\pgfpathmoveto{\pgfqpoint{2.766213in}{1.815352in}}%
\pgfpathlineto{\pgfqpoint{2.766213in}{1.815352in}}%
\pgfpathlineto{\pgfqpoint{2.766213in}{1.818301in}}%
\pgfpathlineto{\pgfqpoint{2.770754in}{1.818301in}}%
\pgfpathlineto{\pgfqpoint{2.770754in}{1.815352in}}%
\pgfpathmoveto{\pgfqpoint{2.766213in}{1.818301in}}%
\pgfpathlineto{\pgfqpoint{2.766213in}{1.818301in}}%
\pgfpathlineto{\pgfqpoint{2.766213in}{1.821250in}}%
\pgfpathlineto{\pgfqpoint{2.770754in}{1.821250in}}%
\pgfpathlineto{\pgfqpoint{2.770754in}{1.818301in}}%
\pgfpathmoveto{\pgfqpoint{2.770754in}{1.815352in}}%
\pgfpathlineto{\pgfqpoint{2.770754in}{1.815352in}}%
\pgfpathlineto{\pgfqpoint{2.770754in}{1.818301in}}%
\pgfpathlineto{\pgfqpoint{2.775295in}{1.818301in}}%
\pgfpathlineto{\pgfqpoint{2.775295in}{1.815352in}}%
\pgfpathmoveto{\pgfqpoint{2.770754in}{1.818301in}}%
\pgfpathlineto{\pgfqpoint{2.770754in}{1.818301in}}%
\pgfpathlineto{\pgfqpoint{2.770754in}{1.821250in}}%
\pgfpathlineto{\pgfqpoint{2.775295in}{1.821250in}}%
\pgfpathlineto{\pgfqpoint{2.775295in}{1.818301in}}%
\pgfpathmoveto{\pgfqpoint{2.775295in}{1.815352in}}%
\pgfpathlineto{\pgfqpoint{2.775295in}{1.815352in}}%
\pgfpathlineto{\pgfqpoint{2.775295in}{1.818301in}}%
\pgfpathlineto{\pgfqpoint{2.779836in}{1.818301in}}%
\pgfpathlineto{\pgfqpoint{2.779836in}{1.815352in}}%
\pgfpathmoveto{\pgfqpoint{2.775295in}{1.818301in}}%
\pgfpathlineto{\pgfqpoint{2.775295in}{1.818301in}}%
\pgfpathlineto{\pgfqpoint{2.775295in}{1.821250in}}%
\pgfpathlineto{\pgfqpoint{2.779836in}{1.821250in}}%
\pgfpathlineto{\pgfqpoint{2.779836in}{1.818301in}}%
\pgfpathmoveto{\pgfqpoint{2.779836in}{1.818301in}}%
\pgfpathlineto{\pgfqpoint{2.779836in}{1.818301in}}%
\pgfpathlineto{\pgfqpoint{2.779836in}{1.821250in}}%
\pgfpathlineto{\pgfqpoint{2.784377in}{1.821250in}}%
\pgfpathlineto{\pgfqpoint{2.784377in}{1.818301in}}%
\pgfpathmoveto{\pgfqpoint{2.775295in}{1.821250in}}%
\pgfpathlineto{\pgfqpoint{2.775295in}{1.821250in}}%
\pgfpathlineto{\pgfqpoint{2.775295in}{1.824200in}}%
\pgfpathlineto{\pgfqpoint{2.779836in}{1.824200in}}%
\pgfpathlineto{\pgfqpoint{2.779836in}{1.821250in}}%
\pgfpathmoveto{\pgfqpoint{2.775295in}{1.824200in}}%
\pgfpathlineto{\pgfqpoint{2.775295in}{1.824200in}}%
\pgfpathlineto{\pgfqpoint{2.775295in}{1.827149in}}%
\pgfpathlineto{\pgfqpoint{2.779836in}{1.827149in}}%
\pgfpathlineto{\pgfqpoint{2.779836in}{1.824200in}}%
\pgfpathmoveto{\pgfqpoint{2.779836in}{1.821250in}}%
\pgfpathlineto{\pgfqpoint{2.779836in}{1.821250in}}%
\pgfpathlineto{\pgfqpoint{2.779836in}{1.824200in}}%
\pgfpathlineto{\pgfqpoint{2.784377in}{1.824200in}}%
\pgfpathlineto{\pgfqpoint{2.784377in}{1.821250in}}%
\pgfpathmoveto{\pgfqpoint{2.779836in}{1.824200in}}%
\pgfpathlineto{\pgfqpoint{2.779836in}{1.824200in}}%
\pgfpathlineto{\pgfqpoint{2.779836in}{1.827149in}}%
\pgfpathlineto{\pgfqpoint{2.784377in}{1.827149in}}%
\pgfpathlineto{\pgfqpoint{2.784377in}{1.824200in}}%
\pgfpathmoveto{\pgfqpoint{2.784377in}{1.821250in}}%
\pgfpathlineto{\pgfqpoint{2.784377in}{1.821250in}}%
\pgfpathlineto{\pgfqpoint{2.784377in}{1.824200in}}%
\pgfpathlineto{\pgfqpoint{2.788918in}{1.824200in}}%
\pgfpathlineto{\pgfqpoint{2.788918in}{1.821250in}}%
\pgfpathmoveto{\pgfqpoint{2.784377in}{1.824200in}}%
\pgfpathlineto{\pgfqpoint{2.784377in}{1.824200in}}%
\pgfpathlineto{\pgfqpoint{2.784377in}{1.827149in}}%
\pgfpathlineto{\pgfqpoint{2.788918in}{1.827149in}}%
\pgfpathlineto{\pgfqpoint{2.788918in}{1.824200in}}%
\pgfpathmoveto{\pgfqpoint{2.788918in}{1.824200in}}%
\pgfpathlineto{\pgfqpoint{2.788918in}{1.824200in}}%
\pgfpathlineto{\pgfqpoint{2.788918in}{1.827149in}}%
\pgfpathlineto{\pgfqpoint{2.793459in}{1.827149in}}%
\pgfpathlineto{\pgfqpoint{2.793459in}{1.824200in}}%
\pgfpathmoveto{\pgfqpoint{2.784377in}{1.827149in}}%
\pgfpathlineto{\pgfqpoint{2.784377in}{1.827149in}}%
\pgfpathlineto{\pgfqpoint{2.784377in}{1.830098in}}%
\pgfpathlineto{\pgfqpoint{2.788918in}{1.830098in}}%
\pgfpathlineto{\pgfqpoint{2.788918in}{1.827149in}}%
\pgfpathmoveto{\pgfqpoint{2.784377in}{1.830098in}}%
\pgfpathlineto{\pgfqpoint{2.784377in}{1.830098in}}%
\pgfpathlineto{\pgfqpoint{2.784377in}{1.833047in}}%
\pgfpathlineto{\pgfqpoint{2.788918in}{1.833047in}}%
\pgfpathlineto{\pgfqpoint{2.788918in}{1.830098in}}%
\pgfpathmoveto{\pgfqpoint{2.788918in}{1.827149in}}%
\pgfpathlineto{\pgfqpoint{2.788918in}{1.827149in}}%
\pgfpathlineto{\pgfqpoint{2.788918in}{1.830098in}}%
\pgfpathlineto{\pgfqpoint{2.793459in}{1.830098in}}%
\pgfpathlineto{\pgfqpoint{2.793459in}{1.827149in}}%
\pgfpathmoveto{\pgfqpoint{2.788918in}{1.830098in}}%
\pgfpathlineto{\pgfqpoint{2.788918in}{1.830098in}}%
\pgfpathlineto{\pgfqpoint{2.788918in}{1.833047in}}%
\pgfpathlineto{\pgfqpoint{2.793459in}{1.833047in}}%
\pgfpathlineto{\pgfqpoint{2.793459in}{1.830098in}}%
\pgfpathmoveto{\pgfqpoint{2.793459in}{1.827149in}}%
\pgfpathlineto{\pgfqpoint{2.793459in}{1.827149in}}%
\pgfpathlineto{\pgfqpoint{2.793459in}{1.830098in}}%
\pgfpathlineto{\pgfqpoint{2.798000in}{1.830098in}}%
\pgfpathlineto{\pgfqpoint{2.798000in}{1.827149in}}%
\pgfpathmoveto{\pgfqpoint{2.793459in}{1.830098in}}%
\pgfpathlineto{\pgfqpoint{2.793459in}{1.830098in}}%
\pgfpathlineto{\pgfqpoint{2.793459in}{1.833047in}}%
\pgfpathlineto{\pgfqpoint{2.798000in}{1.833047in}}%
\pgfpathlineto{\pgfqpoint{2.798000in}{1.830098in}}%
\pgfpathmoveto{\pgfqpoint{2.798000in}{1.830098in}}%
\pgfpathlineto{\pgfqpoint{2.798000in}{1.830098in}}%
\pgfpathlineto{\pgfqpoint{2.798000in}{1.833047in}}%
\pgfpathlineto{\pgfqpoint{2.802541in}{1.833047in}}%
\pgfpathlineto{\pgfqpoint{2.802541in}{1.830098in}}%
\pgfpathmoveto{\pgfqpoint{2.793459in}{1.833047in}}%
\pgfpathlineto{\pgfqpoint{2.793459in}{1.833047in}}%
\pgfpathlineto{\pgfqpoint{2.793459in}{1.835997in}}%
\pgfpathlineto{\pgfqpoint{2.798000in}{1.835997in}}%
\pgfpathlineto{\pgfqpoint{2.798000in}{1.833047in}}%
\pgfpathmoveto{\pgfqpoint{2.793459in}{1.835997in}}%
\pgfpathlineto{\pgfqpoint{2.793459in}{1.835997in}}%
\pgfpathlineto{\pgfqpoint{2.793459in}{1.838946in}}%
\pgfpathlineto{\pgfqpoint{2.798000in}{1.838946in}}%
\pgfpathlineto{\pgfqpoint{2.798000in}{1.835997in}}%
\pgfpathmoveto{\pgfqpoint{2.798000in}{1.833047in}}%
\pgfpathlineto{\pgfqpoint{2.798000in}{1.833047in}}%
\pgfpathlineto{\pgfqpoint{2.798000in}{1.835997in}}%
\pgfpathlineto{\pgfqpoint{2.802541in}{1.835997in}}%
\pgfpathlineto{\pgfqpoint{2.802541in}{1.833047in}}%
\pgfpathmoveto{\pgfqpoint{2.798000in}{1.835997in}}%
\pgfpathlineto{\pgfqpoint{2.798000in}{1.835997in}}%
\pgfpathlineto{\pgfqpoint{2.798000in}{1.838946in}}%
\pgfpathlineto{\pgfqpoint{2.802541in}{1.838946in}}%
\pgfpathlineto{\pgfqpoint{2.802541in}{1.835997in}}%
\pgfpathmoveto{\pgfqpoint{2.802541in}{1.833047in}}%
\pgfpathlineto{\pgfqpoint{2.802541in}{1.833047in}}%
\pgfpathlineto{\pgfqpoint{2.802541in}{1.835997in}}%
\pgfpathlineto{\pgfqpoint{2.807082in}{1.835997in}}%
\pgfpathlineto{\pgfqpoint{2.807082in}{1.833047in}}%
\pgfpathmoveto{\pgfqpoint{2.802541in}{1.835997in}}%
\pgfpathlineto{\pgfqpoint{2.802541in}{1.835997in}}%
\pgfpathlineto{\pgfqpoint{2.802541in}{1.838946in}}%
\pgfpathlineto{\pgfqpoint{2.807082in}{1.838946in}}%
\pgfpathlineto{\pgfqpoint{2.807082in}{1.835997in}}%
\pgfpathmoveto{\pgfqpoint{2.807082in}{1.835997in}}%
\pgfpathlineto{\pgfqpoint{2.807082in}{1.835997in}}%
\pgfpathlineto{\pgfqpoint{2.807082in}{1.838946in}}%
\pgfpathlineto{\pgfqpoint{2.811623in}{1.838946in}}%
\pgfpathlineto{\pgfqpoint{2.811623in}{1.835997in}}%
\pgfpathmoveto{\pgfqpoint{2.802541in}{1.838946in}}%
\pgfpathlineto{\pgfqpoint{2.802541in}{1.838946in}}%
\pgfpathlineto{\pgfqpoint{2.802541in}{1.841895in}}%
\pgfpathlineto{\pgfqpoint{2.807082in}{1.841895in}}%
\pgfpathlineto{\pgfqpoint{2.807082in}{1.838946in}}%
\pgfpathmoveto{\pgfqpoint{2.802541in}{1.841895in}}%
\pgfpathlineto{\pgfqpoint{2.802541in}{1.841895in}}%
\pgfpathlineto{\pgfqpoint{2.802541in}{1.844844in}}%
\pgfpathlineto{\pgfqpoint{2.807082in}{1.844844in}}%
\pgfpathlineto{\pgfqpoint{2.807082in}{1.841895in}}%
\pgfpathmoveto{\pgfqpoint{2.807082in}{1.838946in}}%
\pgfpathlineto{\pgfqpoint{2.807082in}{1.838946in}}%
\pgfpathlineto{\pgfqpoint{2.807082in}{1.841895in}}%
\pgfpathlineto{\pgfqpoint{2.811623in}{1.841895in}}%
\pgfpathlineto{\pgfqpoint{2.811623in}{1.838946in}}%
\pgfpathmoveto{\pgfqpoint{2.807082in}{1.841895in}}%
\pgfpathlineto{\pgfqpoint{2.807082in}{1.841895in}}%
\pgfpathlineto{\pgfqpoint{2.807082in}{1.844844in}}%
\pgfpathlineto{\pgfqpoint{2.811623in}{1.844844in}}%
\pgfpathlineto{\pgfqpoint{2.811623in}{1.841895in}}%
\pgfpathmoveto{\pgfqpoint{2.811623in}{1.838946in}}%
\pgfpathlineto{\pgfqpoint{2.811623in}{1.838946in}}%
\pgfpathlineto{\pgfqpoint{2.811623in}{1.841895in}}%
\pgfpathlineto{\pgfqpoint{2.816165in}{1.841895in}}%
\pgfpathlineto{\pgfqpoint{2.816165in}{1.838946in}}%
\pgfpathmoveto{\pgfqpoint{2.811623in}{1.841895in}}%
\pgfpathlineto{\pgfqpoint{2.811623in}{1.841895in}}%
\pgfpathlineto{\pgfqpoint{2.811623in}{1.844844in}}%
\pgfpathlineto{\pgfqpoint{2.816165in}{1.844844in}}%
\pgfpathlineto{\pgfqpoint{2.816165in}{1.841895in}}%
\pgfpathmoveto{\pgfqpoint{2.816165in}{1.841895in}}%
\pgfpathlineto{\pgfqpoint{2.816165in}{1.841895in}}%
\pgfpathlineto{\pgfqpoint{2.816165in}{1.844844in}}%
\pgfpathlineto{\pgfqpoint{2.820706in}{1.844844in}}%
\pgfpathlineto{\pgfqpoint{2.820706in}{1.841895in}}%
\pgfpathmoveto{\pgfqpoint{2.811623in}{1.844844in}}%
\pgfpathlineto{\pgfqpoint{2.811623in}{1.844844in}}%
\pgfpathlineto{\pgfqpoint{2.811623in}{1.847793in}}%
\pgfpathlineto{\pgfqpoint{2.816165in}{1.847793in}}%
\pgfpathlineto{\pgfqpoint{2.816165in}{1.844844in}}%
\pgfpathmoveto{\pgfqpoint{2.811623in}{1.847793in}}%
\pgfpathlineto{\pgfqpoint{2.811623in}{1.847793in}}%
\pgfpathlineto{\pgfqpoint{2.811623in}{1.850743in}}%
\pgfpathlineto{\pgfqpoint{2.816165in}{1.850743in}}%
\pgfpathlineto{\pgfqpoint{2.816165in}{1.847793in}}%
\pgfpathmoveto{\pgfqpoint{2.816165in}{1.844844in}}%
\pgfpathlineto{\pgfqpoint{2.816165in}{1.844844in}}%
\pgfpathlineto{\pgfqpoint{2.816165in}{1.847793in}}%
\pgfpathlineto{\pgfqpoint{2.820706in}{1.847793in}}%
\pgfpathlineto{\pgfqpoint{2.820706in}{1.844844in}}%
\pgfpathmoveto{\pgfqpoint{2.816165in}{1.847793in}}%
\pgfpathlineto{\pgfqpoint{2.816165in}{1.847793in}}%
\pgfpathlineto{\pgfqpoint{2.816165in}{1.850743in}}%
\pgfpathlineto{\pgfqpoint{2.820706in}{1.850743in}}%
\pgfpathlineto{\pgfqpoint{2.820706in}{1.847793in}}%
\pgfpathmoveto{\pgfqpoint{2.820706in}{1.844844in}}%
\pgfpathlineto{\pgfqpoint{2.820706in}{1.844844in}}%
\pgfpathlineto{\pgfqpoint{2.820706in}{1.847793in}}%
\pgfpathlineto{\pgfqpoint{2.825247in}{1.847793in}}%
\pgfpathlineto{\pgfqpoint{2.825247in}{1.844844in}}%
\pgfpathmoveto{\pgfqpoint{2.820706in}{1.847793in}}%
\pgfpathlineto{\pgfqpoint{2.820706in}{1.847793in}}%
\pgfpathlineto{\pgfqpoint{2.820706in}{1.850743in}}%
\pgfpathlineto{\pgfqpoint{2.825247in}{1.850743in}}%
\pgfpathlineto{\pgfqpoint{2.825247in}{1.847793in}}%
\pgfpathmoveto{\pgfqpoint{2.825247in}{1.847793in}}%
\pgfpathlineto{\pgfqpoint{2.825247in}{1.847793in}}%
\pgfpathlineto{\pgfqpoint{2.825247in}{1.850743in}}%
\pgfpathlineto{\pgfqpoint{2.829788in}{1.850743in}}%
\pgfpathlineto{\pgfqpoint{2.829788in}{1.847793in}}%
\pgfpathmoveto{\pgfqpoint{2.820706in}{1.850743in}}%
\pgfpathlineto{\pgfqpoint{2.820706in}{1.850743in}}%
\pgfpathlineto{\pgfqpoint{2.820706in}{1.853692in}}%
\pgfpathlineto{\pgfqpoint{2.825247in}{1.853692in}}%
\pgfpathlineto{\pgfqpoint{2.825247in}{1.850743in}}%
\pgfpathmoveto{\pgfqpoint{2.820706in}{1.853692in}}%
\pgfpathlineto{\pgfqpoint{2.820706in}{1.853692in}}%
\pgfpathlineto{\pgfqpoint{2.820706in}{1.856641in}}%
\pgfpathlineto{\pgfqpoint{2.825247in}{1.856641in}}%
\pgfpathlineto{\pgfqpoint{2.825247in}{1.853692in}}%
\pgfpathmoveto{\pgfqpoint{2.825247in}{1.850743in}}%
\pgfpathlineto{\pgfqpoint{2.825247in}{1.850743in}}%
\pgfpathlineto{\pgfqpoint{2.825247in}{1.853692in}}%
\pgfpathlineto{\pgfqpoint{2.829788in}{1.853692in}}%
\pgfpathlineto{\pgfqpoint{2.829788in}{1.850743in}}%
\pgfpathmoveto{\pgfqpoint{2.825247in}{1.853692in}}%
\pgfpathlineto{\pgfqpoint{2.825247in}{1.853692in}}%
\pgfpathlineto{\pgfqpoint{2.825247in}{1.856641in}}%
\pgfpathlineto{\pgfqpoint{2.829788in}{1.856641in}}%
\pgfpathlineto{\pgfqpoint{2.829788in}{1.853692in}}%
\pgfpathmoveto{\pgfqpoint{2.829788in}{1.850743in}}%
\pgfpathlineto{\pgfqpoint{2.829788in}{1.850743in}}%
\pgfpathlineto{\pgfqpoint{2.829788in}{1.853692in}}%
\pgfpathlineto{\pgfqpoint{2.834329in}{1.853692in}}%
\pgfpathlineto{\pgfqpoint{2.834329in}{1.850743in}}%
\pgfpathmoveto{\pgfqpoint{2.829788in}{1.853692in}}%
\pgfpathlineto{\pgfqpoint{2.829788in}{1.853692in}}%
\pgfpathlineto{\pgfqpoint{2.829788in}{1.856641in}}%
\pgfpathlineto{\pgfqpoint{2.834329in}{1.856641in}}%
\pgfpathlineto{\pgfqpoint{2.834329in}{1.853692in}}%
\pgfpathmoveto{\pgfqpoint{2.834329in}{1.853692in}}%
\pgfpathlineto{\pgfqpoint{2.834329in}{1.853692in}}%
\pgfpathlineto{\pgfqpoint{2.834329in}{1.856641in}}%
\pgfpathlineto{\pgfqpoint{2.838870in}{1.856641in}}%
\pgfpathlineto{\pgfqpoint{2.838870in}{1.853692in}}%
\pgfpathmoveto{\pgfqpoint{2.829788in}{1.856641in}}%
\pgfpathlineto{\pgfqpoint{2.829788in}{1.856641in}}%
\pgfpathlineto{\pgfqpoint{2.829788in}{1.859590in}}%
\pgfpathlineto{\pgfqpoint{2.834329in}{1.859590in}}%
\pgfpathlineto{\pgfqpoint{2.834329in}{1.856641in}}%
\pgfpathmoveto{\pgfqpoint{2.829788in}{1.859590in}}%
\pgfpathlineto{\pgfqpoint{2.829788in}{1.859590in}}%
\pgfpathlineto{\pgfqpoint{2.829788in}{1.862540in}}%
\pgfpathlineto{\pgfqpoint{2.834329in}{1.862540in}}%
\pgfpathlineto{\pgfqpoint{2.834329in}{1.859590in}}%
\pgfpathmoveto{\pgfqpoint{2.834329in}{1.856641in}}%
\pgfpathlineto{\pgfqpoint{2.834329in}{1.856641in}}%
\pgfpathlineto{\pgfqpoint{2.834329in}{1.859590in}}%
\pgfpathlineto{\pgfqpoint{2.838870in}{1.859590in}}%
\pgfpathlineto{\pgfqpoint{2.838870in}{1.856641in}}%
\pgfpathmoveto{\pgfqpoint{2.834329in}{1.859590in}}%
\pgfpathlineto{\pgfqpoint{2.834329in}{1.859590in}}%
\pgfpathlineto{\pgfqpoint{2.834329in}{1.862540in}}%
\pgfpathlineto{\pgfqpoint{2.838870in}{1.862540in}}%
\pgfpathlineto{\pgfqpoint{2.838870in}{1.859590in}}%
\pgfpathmoveto{\pgfqpoint{2.838870in}{1.856641in}}%
\pgfpathlineto{\pgfqpoint{2.838870in}{1.856641in}}%
\pgfpathlineto{\pgfqpoint{2.838870in}{1.859590in}}%
\pgfpathlineto{\pgfqpoint{2.843411in}{1.859590in}}%
\pgfpathlineto{\pgfqpoint{2.843411in}{1.856641in}}%
\pgfpathmoveto{\pgfqpoint{2.838870in}{1.859590in}}%
\pgfpathlineto{\pgfqpoint{2.838870in}{1.859590in}}%
\pgfpathlineto{\pgfqpoint{2.838870in}{1.862540in}}%
\pgfpathlineto{\pgfqpoint{2.843411in}{1.862540in}}%
\pgfpathlineto{\pgfqpoint{2.843411in}{1.859590in}}%
\pgfpathmoveto{\pgfqpoint{2.843411in}{1.859590in}}%
\pgfpathlineto{\pgfqpoint{2.843411in}{1.859590in}}%
\pgfpathlineto{\pgfqpoint{2.843411in}{1.862540in}}%
\pgfpathlineto{\pgfqpoint{2.847952in}{1.862540in}}%
\pgfpathlineto{\pgfqpoint{2.847952in}{1.859590in}}%
\pgfpathmoveto{\pgfqpoint{2.838870in}{1.862540in}}%
\pgfpathlineto{\pgfqpoint{2.838870in}{1.862540in}}%
\pgfpathlineto{\pgfqpoint{2.838870in}{1.865489in}}%
\pgfpathlineto{\pgfqpoint{2.843411in}{1.865489in}}%
\pgfpathlineto{\pgfqpoint{2.843411in}{1.862540in}}%
\pgfpathmoveto{\pgfqpoint{2.838870in}{1.865489in}}%
\pgfpathlineto{\pgfqpoint{2.838870in}{1.865489in}}%
\pgfpathlineto{\pgfqpoint{2.838870in}{1.868438in}}%
\pgfpathlineto{\pgfqpoint{2.843411in}{1.868438in}}%
\pgfpathlineto{\pgfqpoint{2.843411in}{1.865489in}}%
\pgfpathmoveto{\pgfqpoint{2.843411in}{1.862540in}}%
\pgfpathlineto{\pgfqpoint{2.843411in}{1.862540in}}%
\pgfpathlineto{\pgfqpoint{2.843411in}{1.865489in}}%
\pgfpathlineto{\pgfqpoint{2.847952in}{1.865489in}}%
\pgfpathlineto{\pgfqpoint{2.847952in}{1.862540in}}%
\pgfpathmoveto{\pgfqpoint{2.843411in}{1.865489in}}%
\pgfpathlineto{\pgfqpoint{2.843411in}{1.865489in}}%
\pgfpathlineto{\pgfqpoint{2.843411in}{1.868438in}}%
\pgfpathlineto{\pgfqpoint{2.847952in}{1.868438in}}%
\pgfpathlineto{\pgfqpoint{2.847952in}{1.865489in}}%
\pgfpathmoveto{\pgfqpoint{2.847952in}{1.862540in}}%
\pgfpathlineto{\pgfqpoint{2.847952in}{1.862540in}}%
\pgfpathlineto{\pgfqpoint{2.847952in}{1.865489in}}%
\pgfpathlineto{\pgfqpoint{2.852493in}{1.865489in}}%
\pgfpathlineto{\pgfqpoint{2.852493in}{1.862540in}}%
\pgfpathmoveto{\pgfqpoint{2.847952in}{1.865489in}}%
\pgfpathlineto{\pgfqpoint{2.847952in}{1.865489in}}%
\pgfpathlineto{\pgfqpoint{2.847952in}{1.868438in}}%
\pgfpathlineto{\pgfqpoint{2.852493in}{1.868438in}}%
\pgfpathlineto{\pgfqpoint{2.852493in}{1.865489in}}%
\pgfpathmoveto{\pgfqpoint{2.852493in}{1.865489in}}%
\pgfpathlineto{\pgfqpoint{2.852493in}{1.865489in}}%
\pgfpathlineto{\pgfqpoint{2.852493in}{1.868438in}}%
\pgfpathlineto{\pgfqpoint{2.857034in}{1.868438in}}%
\pgfpathlineto{\pgfqpoint{2.857034in}{1.865489in}}%
\pgfpathmoveto{\pgfqpoint{2.847952in}{1.868438in}}%
\pgfpathlineto{\pgfqpoint{2.847952in}{1.868438in}}%
\pgfpathlineto{\pgfqpoint{2.847952in}{1.871387in}}%
\pgfpathlineto{\pgfqpoint{2.852493in}{1.871387in}}%
\pgfpathlineto{\pgfqpoint{2.852493in}{1.868438in}}%
\pgfpathmoveto{\pgfqpoint{2.847952in}{1.871387in}}%
\pgfpathlineto{\pgfqpoint{2.847952in}{1.871387in}}%
\pgfpathlineto{\pgfqpoint{2.847952in}{1.874337in}}%
\pgfpathlineto{\pgfqpoint{2.852493in}{1.874337in}}%
\pgfpathlineto{\pgfqpoint{2.852493in}{1.871387in}}%
\pgfpathmoveto{\pgfqpoint{2.852493in}{1.868438in}}%
\pgfpathlineto{\pgfqpoint{2.852493in}{1.868438in}}%
\pgfpathlineto{\pgfqpoint{2.852493in}{1.871387in}}%
\pgfpathlineto{\pgfqpoint{2.857034in}{1.871387in}}%
\pgfpathlineto{\pgfqpoint{2.857034in}{1.868438in}}%
\pgfpathmoveto{\pgfqpoint{2.852493in}{1.871387in}}%
\pgfpathlineto{\pgfqpoint{2.852493in}{1.871387in}}%
\pgfpathlineto{\pgfqpoint{2.852493in}{1.874337in}}%
\pgfpathlineto{\pgfqpoint{2.857034in}{1.874337in}}%
\pgfpathlineto{\pgfqpoint{2.857034in}{1.871387in}}%
\pgfpathmoveto{\pgfqpoint{2.857034in}{1.868438in}}%
\pgfpathlineto{\pgfqpoint{2.857034in}{1.868438in}}%
\pgfpathlineto{\pgfqpoint{2.857034in}{1.871387in}}%
\pgfpathlineto{\pgfqpoint{2.861575in}{1.871387in}}%
\pgfpathlineto{\pgfqpoint{2.861575in}{1.868438in}}%
\pgfpathmoveto{\pgfqpoint{2.857034in}{1.871387in}}%
\pgfpathlineto{\pgfqpoint{2.857034in}{1.871387in}}%
\pgfpathlineto{\pgfqpoint{2.857034in}{1.874337in}}%
\pgfpathlineto{\pgfqpoint{2.861575in}{1.874337in}}%
\pgfpathlineto{\pgfqpoint{2.861575in}{1.871387in}}%
\pgfpathmoveto{\pgfqpoint{2.861575in}{1.871387in}}%
\pgfpathlineto{\pgfqpoint{2.861575in}{1.871387in}}%
\pgfpathlineto{\pgfqpoint{2.861575in}{1.874337in}}%
\pgfpathlineto{\pgfqpoint{2.866116in}{1.874337in}}%
\pgfpathlineto{\pgfqpoint{2.866116in}{1.871387in}}%
\pgfpathmoveto{\pgfqpoint{2.857034in}{1.874337in}}%
\pgfpathlineto{\pgfqpoint{2.857034in}{1.874337in}}%
\pgfpathlineto{\pgfqpoint{2.857034in}{1.877286in}}%
\pgfpathlineto{\pgfqpoint{2.861575in}{1.877286in}}%
\pgfpathlineto{\pgfqpoint{2.861575in}{1.874337in}}%
\pgfpathmoveto{\pgfqpoint{2.857034in}{1.877286in}}%
\pgfpathlineto{\pgfqpoint{2.857034in}{1.877286in}}%
\pgfpathlineto{\pgfqpoint{2.857034in}{1.880235in}}%
\pgfpathlineto{\pgfqpoint{2.861575in}{1.880235in}}%
\pgfpathlineto{\pgfqpoint{2.861575in}{1.877286in}}%
\pgfpathmoveto{\pgfqpoint{2.861575in}{1.874337in}}%
\pgfpathlineto{\pgfqpoint{2.861575in}{1.874337in}}%
\pgfpathlineto{\pgfqpoint{2.861575in}{1.877286in}}%
\pgfpathlineto{\pgfqpoint{2.866116in}{1.877286in}}%
\pgfpathlineto{\pgfqpoint{2.866116in}{1.874337in}}%
\pgfpathmoveto{\pgfqpoint{2.861575in}{1.877286in}}%
\pgfpathlineto{\pgfqpoint{2.861575in}{1.877286in}}%
\pgfpathlineto{\pgfqpoint{2.861575in}{1.880235in}}%
\pgfpathlineto{\pgfqpoint{2.866116in}{1.880235in}}%
\pgfpathlineto{\pgfqpoint{2.866116in}{1.877286in}}%
\pgfpathmoveto{\pgfqpoint{2.866116in}{1.874337in}}%
\pgfpathlineto{\pgfqpoint{2.866116in}{1.874337in}}%
\pgfpathlineto{\pgfqpoint{2.866116in}{1.877286in}}%
\pgfpathlineto{\pgfqpoint{2.870657in}{1.877286in}}%
\pgfpathlineto{\pgfqpoint{2.870657in}{1.874337in}}%
\pgfpathmoveto{\pgfqpoint{2.866116in}{1.877286in}}%
\pgfpathlineto{\pgfqpoint{2.866116in}{1.877286in}}%
\pgfpathlineto{\pgfqpoint{2.866116in}{1.880235in}}%
\pgfpathlineto{\pgfqpoint{2.870657in}{1.880235in}}%
\pgfpathlineto{\pgfqpoint{2.870657in}{1.877286in}}%
\pgfpathmoveto{\pgfqpoint{2.870657in}{1.877286in}}%
\pgfpathlineto{\pgfqpoint{2.870657in}{1.877286in}}%
\pgfpathlineto{\pgfqpoint{2.870657in}{1.880235in}}%
\pgfpathlineto{\pgfqpoint{2.875198in}{1.880235in}}%
\pgfpathlineto{\pgfqpoint{2.875198in}{1.877286in}}%
\pgfpathmoveto{\pgfqpoint{2.866116in}{1.880235in}}%
\pgfpathlineto{\pgfqpoint{2.866116in}{1.880235in}}%
\pgfpathlineto{\pgfqpoint{2.866116in}{1.883184in}}%
\pgfpathlineto{\pgfqpoint{2.870657in}{1.883184in}}%
\pgfpathlineto{\pgfqpoint{2.870657in}{1.880235in}}%
\pgfpathmoveto{\pgfqpoint{2.866116in}{1.883184in}}%
\pgfpathlineto{\pgfqpoint{2.866116in}{1.883184in}}%
\pgfpathlineto{\pgfqpoint{2.866116in}{1.886133in}}%
\pgfpathlineto{\pgfqpoint{2.870657in}{1.886133in}}%
\pgfpathlineto{\pgfqpoint{2.870657in}{1.883184in}}%
\pgfpathmoveto{\pgfqpoint{2.870657in}{1.880235in}}%
\pgfpathlineto{\pgfqpoint{2.870657in}{1.880235in}}%
\pgfpathlineto{\pgfqpoint{2.870657in}{1.883184in}}%
\pgfpathlineto{\pgfqpoint{2.875198in}{1.883184in}}%
\pgfpathlineto{\pgfqpoint{2.875198in}{1.880235in}}%
\pgfpathmoveto{\pgfqpoint{2.870657in}{1.883184in}}%
\pgfpathlineto{\pgfqpoint{2.870657in}{1.883184in}}%
\pgfpathlineto{\pgfqpoint{2.870657in}{1.886133in}}%
\pgfpathlineto{\pgfqpoint{2.875198in}{1.886133in}}%
\pgfpathlineto{\pgfqpoint{2.875198in}{1.883184in}}%
\pgfpathmoveto{\pgfqpoint{2.875198in}{1.880235in}}%
\pgfpathlineto{\pgfqpoint{2.875198in}{1.880235in}}%
\pgfpathlineto{\pgfqpoint{2.875198in}{1.883184in}}%
\pgfpathlineto{\pgfqpoint{2.879739in}{1.883184in}}%
\pgfpathlineto{\pgfqpoint{2.879739in}{1.880235in}}%
\pgfpathmoveto{\pgfqpoint{2.875198in}{1.883184in}}%
\pgfpathlineto{\pgfqpoint{2.875198in}{1.883184in}}%
\pgfpathlineto{\pgfqpoint{2.875198in}{1.886133in}}%
\pgfpathlineto{\pgfqpoint{2.879739in}{1.886133in}}%
\pgfpathlineto{\pgfqpoint{2.879739in}{1.883184in}}%
\pgfpathmoveto{\pgfqpoint{2.879739in}{1.883184in}}%
\pgfpathlineto{\pgfqpoint{2.879739in}{1.883184in}}%
\pgfpathlineto{\pgfqpoint{2.879739in}{1.886133in}}%
\pgfpathlineto{\pgfqpoint{2.884280in}{1.886133in}}%
\pgfpathlineto{\pgfqpoint{2.884280in}{1.883184in}}%
\pgfpathmoveto{\pgfqpoint{2.875198in}{1.886133in}}%
\pgfpathlineto{\pgfqpoint{2.875198in}{1.886133in}}%
\pgfpathlineto{\pgfqpoint{2.875198in}{1.889083in}}%
\pgfpathlineto{\pgfqpoint{2.879739in}{1.889083in}}%
\pgfpathlineto{\pgfqpoint{2.879739in}{1.886133in}}%
\pgfpathmoveto{\pgfqpoint{2.875198in}{1.889083in}}%
\pgfpathlineto{\pgfqpoint{2.875198in}{1.889083in}}%
\pgfpathlineto{\pgfqpoint{2.875198in}{1.892032in}}%
\pgfpathlineto{\pgfqpoint{2.879739in}{1.892032in}}%
\pgfpathlineto{\pgfqpoint{2.879739in}{1.889083in}}%
\pgfpathmoveto{\pgfqpoint{2.879739in}{1.886133in}}%
\pgfpathlineto{\pgfqpoint{2.879739in}{1.886133in}}%
\pgfpathlineto{\pgfqpoint{2.879739in}{1.889083in}}%
\pgfpathlineto{\pgfqpoint{2.884280in}{1.889083in}}%
\pgfpathlineto{\pgfqpoint{2.884280in}{1.886133in}}%
\pgfpathmoveto{\pgfqpoint{2.879739in}{1.889083in}}%
\pgfpathlineto{\pgfqpoint{2.879739in}{1.889083in}}%
\pgfpathlineto{\pgfqpoint{2.879739in}{1.892032in}}%
\pgfpathlineto{\pgfqpoint{2.884280in}{1.892032in}}%
\pgfpathlineto{\pgfqpoint{2.884280in}{1.889083in}}%
\pgfpathmoveto{\pgfqpoint{2.884280in}{1.886133in}}%
\pgfpathlineto{\pgfqpoint{2.884280in}{1.886133in}}%
\pgfpathlineto{\pgfqpoint{2.884280in}{1.889083in}}%
\pgfpathlineto{\pgfqpoint{2.888821in}{1.889083in}}%
\pgfpathlineto{\pgfqpoint{2.888821in}{1.886133in}}%
\pgfpathmoveto{\pgfqpoint{2.884280in}{1.889083in}}%
\pgfpathlineto{\pgfqpoint{2.884280in}{1.889083in}}%
\pgfpathlineto{\pgfqpoint{2.884280in}{1.892032in}}%
\pgfpathlineto{\pgfqpoint{2.888821in}{1.892032in}}%
\pgfpathlineto{\pgfqpoint{2.888821in}{1.889083in}}%
\pgfpathmoveto{\pgfqpoint{2.888821in}{1.889083in}}%
\pgfpathlineto{\pgfqpoint{2.888821in}{1.889083in}}%
\pgfpathlineto{\pgfqpoint{2.888821in}{1.892032in}}%
\pgfpathlineto{\pgfqpoint{2.893363in}{1.892032in}}%
\pgfpathlineto{\pgfqpoint{2.893363in}{1.889083in}}%
\pgfpathmoveto{\pgfqpoint{2.884280in}{1.892032in}}%
\pgfpathlineto{\pgfqpoint{2.884280in}{1.892032in}}%
\pgfpathlineto{\pgfqpoint{2.884280in}{1.894981in}}%
\pgfpathlineto{\pgfqpoint{2.888821in}{1.894981in}}%
\pgfpathlineto{\pgfqpoint{2.888821in}{1.892032in}}%
\pgfpathmoveto{\pgfqpoint{2.884280in}{1.894981in}}%
\pgfpathlineto{\pgfqpoint{2.884280in}{1.894981in}}%
\pgfpathlineto{\pgfqpoint{2.884280in}{1.897930in}}%
\pgfpathlineto{\pgfqpoint{2.888821in}{1.897930in}}%
\pgfpathlineto{\pgfqpoint{2.888821in}{1.894981in}}%
\pgfpathmoveto{\pgfqpoint{2.888821in}{1.892032in}}%
\pgfpathlineto{\pgfqpoint{2.888821in}{1.892032in}}%
\pgfpathlineto{\pgfqpoint{2.888821in}{1.894981in}}%
\pgfpathlineto{\pgfqpoint{2.893363in}{1.894981in}}%
\pgfpathlineto{\pgfqpoint{2.893363in}{1.892032in}}%
\pgfpathmoveto{\pgfqpoint{2.888821in}{1.894981in}}%
\pgfpathlineto{\pgfqpoint{2.888821in}{1.894981in}}%
\pgfpathlineto{\pgfqpoint{2.888821in}{1.897930in}}%
\pgfpathlineto{\pgfqpoint{2.893363in}{1.897930in}}%
\pgfpathlineto{\pgfqpoint{2.893363in}{1.894981in}}%
\pgfpathmoveto{\pgfqpoint{2.893363in}{1.892032in}}%
\pgfpathlineto{\pgfqpoint{2.893363in}{1.892032in}}%
\pgfpathlineto{\pgfqpoint{2.893363in}{1.894981in}}%
\pgfpathlineto{\pgfqpoint{2.897904in}{1.894981in}}%
\pgfpathlineto{\pgfqpoint{2.897904in}{1.892032in}}%
\pgfpathmoveto{\pgfqpoint{2.893363in}{1.894981in}}%
\pgfpathlineto{\pgfqpoint{2.893363in}{1.894981in}}%
\pgfpathlineto{\pgfqpoint{2.893363in}{1.897930in}}%
\pgfpathlineto{\pgfqpoint{2.897904in}{1.897930in}}%
\pgfpathlineto{\pgfqpoint{2.897904in}{1.894981in}}%
\pgfpathmoveto{\pgfqpoint{2.897904in}{1.894981in}}%
\pgfpathlineto{\pgfqpoint{2.897904in}{1.894981in}}%
\pgfpathlineto{\pgfqpoint{2.897904in}{1.897930in}}%
\pgfpathlineto{\pgfqpoint{2.902445in}{1.897930in}}%
\pgfpathlineto{\pgfqpoint{2.902445in}{1.894981in}}%
\pgfpathmoveto{\pgfqpoint{2.893363in}{1.897930in}}%
\pgfpathlineto{\pgfqpoint{2.893363in}{1.897930in}}%
\pgfpathlineto{\pgfqpoint{2.893363in}{1.900880in}}%
\pgfpathlineto{\pgfqpoint{2.897904in}{1.900880in}}%
\pgfpathlineto{\pgfqpoint{2.897904in}{1.897930in}}%
\pgfpathmoveto{\pgfqpoint{2.893363in}{1.900880in}}%
\pgfpathlineto{\pgfqpoint{2.893363in}{1.900880in}}%
\pgfpathlineto{\pgfqpoint{2.893363in}{1.903829in}}%
\pgfpathlineto{\pgfqpoint{2.897904in}{1.903829in}}%
\pgfpathlineto{\pgfqpoint{2.897904in}{1.900880in}}%
\pgfpathmoveto{\pgfqpoint{2.897904in}{1.897930in}}%
\pgfpathlineto{\pgfqpoint{2.897904in}{1.897930in}}%
\pgfpathlineto{\pgfqpoint{2.897904in}{1.900880in}}%
\pgfpathlineto{\pgfqpoint{2.902445in}{1.900880in}}%
\pgfpathlineto{\pgfqpoint{2.902445in}{1.897930in}}%
\pgfpathmoveto{\pgfqpoint{2.897904in}{1.900880in}}%
\pgfpathlineto{\pgfqpoint{2.897904in}{1.900880in}}%
\pgfpathlineto{\pgfqpoint{2.897904in}{1.903829in}}%
\pgfpathlineto{\pgfqpoint{2.902445in}{1.903829in}}%
\pgfpathlineto{\pgfqpoint{2.902445in}{1.900880in}}%
\pgfpathmoveto{\pgfqpoint{2.902445in}{1.897930in}}%
\pgfpathlineto{\pgfqpoint{2.902445in}{1.897930in}}%
\pgfpathlineto{\pgfqpoint{2.902445in}{1.900880in}}%
\pgfpathlineto{\pgfqpoint{2.906986in}{1.900880in}}%
\pgfpathlineto{\pgfqpoint{2.906986in}{1.897930in}}%
\pgfpathmoveto{\pgfqpoint{2.902445in}{1.900880in}}%
\pgfpathlineto{\pgfqpoint{2.902445in}{1.900880in}}%
\pgfpathlineto{\pgfqpoint{2.902445in}{1.903829in}}%
\pgfpathlineto{\pgfqpoint{2.906986in}{1.903829in}}%
\pgfpathlineto{\pgfqpoint{2.906986in}{1.900880in}}%
\pgfpathmoveto{\pgfqpoint{2.906986in}{1.900880in}}%
\pgfpathlineto{\pgfqpoint{2.906986in}{1.900880in}}%
\pgfpathlineto{\pgfqpoint{2.906986in}{1.903829in}}%
\pgfpathlineto{\pgfqpoint{2.911527in}{1.903829in}}%
\pgfpathlineto{\pgfqpoint{2.911527in}{1.900880in}}%
\pgfpathmoveto{\pgfqpoint{2.902445in}{1.903829in}}%
\pgfpathlineto{\pgfqpoint{2.902445in}{1.903829in}}%
\pgfpathlineto{\pgfqpoint{2.902445in}{1.906778in}}%
\pgfpathlineto{\pgfqpoint{2.906986in}{1.906778in}}%
\pgfpathlineto{\pgfqpoint{2.906986in}{1.903829in}}%
\pgfpathmoveto{\pgfqpoint{2.902445in}{1.906778in}}%
\pgfpathlineto{\pgfqpoint{2.902445in}{1.906778in}}%
\pgfpathlineto{\pgfqpoint{2.902445in}{1.909727in}}%
\pgfpathlineto{\pgfqpoint{2.906986in}{1.909727in}}%
\pgfpathlineto{\pgfqpoint{2.906986in}{1.906778in}}%
\pgfpathmoveto{\pgfqpoint{2.906986in}{1.903829in}}%
\pgfpathlineto{\pgfqpoint{2.906986in}{1.903829in}}%
\pgfpathlineto{\pgfqpoint{2.906986in}{1.906778in}}%
\pgfpathlineto{\pgfqpoint{2.911527in}{1.906778in}}%
\pgfpathlineto{\pgfqpoint{2.911527in}{1.903829in}}%
\pgfpathmoveto{\pgfqpoint{2.906986in}{1.906778in}}%
\pgfpathlineto{\pgfqpoint{2.906986in}{1.906778in}}%
\pgfpathlineto{\pgfqpoint{2.906986in}{1.909727in}}%
\pgfpathlineto{\pgfqpoint{2.911527in}{1.909727in}}%
\pgfpathlineto{\pgfqpoint{2.911527in}{1.906778in}}%
\pgfpathmoveto{\pgfqpoint{2.911527in}{1.903829in}}%
\pgfpathlineto{\pgfqpoint{2.911527in}{1.903829in}}%
\pgfpathlineto{\pgfqpoint{2.911527in}{1.906778in}}%
\pgfpathlineto{\pgfqpoint{2.916068in}{1.906778in}}%
\pgfpathlineto{\pgfqpoint{2.916068in}{1.903829in}}%
\pgfpathmoveto{\pgfqpoint{2.911527in}{1.906778in}}%
\pgfpathlineto{\pgfqpoint{2.911527in}{1.906778in}}%
\pgfpathlineto{\pgfqpoint{2.911527in}{1.909727in}}%
\pgfpathlineto{\pgfqpoint{2.916068in}{1.909727in}}%
\pgfpathlineto{\pgfqpoint{2.916068in}{1.906778in}}%
\pgfpathmoveto{\pgfqpoint{2.916068in}{1.906778in}}%
\pgfpathlineto{\pgfqpoint{2.916068in}{1.906778in}}%
\pgfpathlineto{\pgfqpoint{2.916068in}{1.909727in}}%
\pgfpathlineto{\pgfqpoint{2.920609in}{1.909727in}}%
\pgfpathlineto{\pgfqpoint{2.920609in}{1.906778in}}%
\pgfpathmoveto{\pgfqpoint{2.911527in}{1.909727in}}%
\pgfpathlineto{\pgfqpoint{2.911527in}{1.909727in}}%
\pgfpathlineto{\pgfqpoint{2.911527in}{1.912677in}}%
\pgfpathlineto{\pgfqpoint{2.916068in}{1.912677in}}%
\pgfpathlineto{\pgfqpoint{2.916068in}{1.909727in}}%
\pgfpathmoveto{\pgfqpoint{2.911527in}{1.912677in}}%
\pgfpathlineto{\pgfqpoint{2.911527in}{1.912677in}}%
\pgfpathlineto{\pgfqpoint{2.911527in}{1.915626in}}%
\pgfpathlineto{\pgfqpoint{2.916068in}{1.915626in}}%
\pgfpathlineto{\pgfqpoint{2.916068in}{1.912677in}}%
\pgfpathmoveto{\pgfqpoint{2.916068in}{1.909727in}}%
\pgfpathlineto{\pgfqpoint{2.916068in}{1.909727in}}%
\pgfpathlineto{\pgfqpoint{2.916068in}{1.912677in}}%
\pgfpathlineto{\pgfqpoint{2.920609in}{1.912677in}}%
\pgfpathlineto{\pgfqpoint{2.920609in}{1.909727in}}%
\pgfpathmoveto{\pgfqpoint{2.916068in}{1.912677in}}%
\pgfpathlineto{\pgfqpoint{2.916068in}{1.912677in}}%
\pgfpathlineto{\pgfqpoint{2.916068in}{1.915626in}}%
\pgfpathlineto{\pgfqpoint{2.920609in}{1.915626in}}%
\pgfpathlineto{\pgfqpoint{2.920609in}{1.912677in}}%
\pgfpathmoveto{\pgfqpoint{2.920609in}{1.909727in}}%
\pgfpathlineto{\pgfqpoint{2.920609in}{1.909727in}}%
\pgfpathlineto{\pgfqpoint{2.920609in}{1.912677in}}%
\pgfpathlineto{\pgfqpoint{2.925150in}{1.912677in}}%
\pgfpathlineto{\pgfqpoint{2.925150in}{1.909727in}}%
\pgfpathmoveto{\pgfqpoint{2.920609in}{1.912677in}}%
\pgfpathlineto{\pgfqpoint{2.920609in}{1.912677in}}%
\pgfpathlineto{\pgfqpoint{2.920609in}{1.915626in}}%
\pgfpathlineto{\pgfqpoint{2.925150in}{1.915626in}}%
\pgfpathlineto{\pgfqpoint{2.925150in}{1.912677in}}%
\pgfpathmoveto{\pgfqpoint{2.925150in}{1.912677in}}%
\pgfpathlineto{\pgfqpoint{2.925150in}{1.912677in}}%
\pgfpathlineto{\pgfqpoint{2.925150in}{1.915626in}}%
\pgfpathlineto{\pgfqpoint{2.929691in}{1.915626in}}%
\pgfpathlineto{\pgfqpoint{2.929691in}{1.912677in}}%
\pgfpathmoveto{\pgfqpoint{2.920609in}{1.915626in}}%
\pgfpathlineto{\pgfqpoint{2.920609in}{1.915626in}}%
\pgfpathlineto{\pgfqpoint{2.920609in}{1.918575in}}%
\pgfpathlineto{\pgfqpoint{2.925150in}{1.918575in}}%
\pgfpathlineto{\pgfqpoint{2.925150in}{1.915626in}}%
\pgfpathmoveto{\pgfqpoint{2.920609in}{1.918575in}}%
\pgfpathlineto{\pgfqpoint{2.920609in}{1.918575in}}%
\pgfpathlineto{\pgfqpoint{2.920609in}{1.921524in}}%
\pgfpathlineto{\pgfqpoint{2.925150in}{1.921524in}}%
\pgfpathlineto{\pgfqpoint{2.925150in}{1.918575in}}%
\pgfpathmoveto{\pgfqpoint{2.925150in}{1.915626in}}%
\pgfpathlineto{\pgfqpoint{2.925150in}{1.915626in}}%
\pgfpathlineto{\pgfqpoint{2.925150in}{1.918575in}}%
\pgfpathlineto{\pgfqpoint{2.929691in}{1.918575in}}%
\pgfpathlineto{\pgfqpoint{2.929691in}{1.915626in}}%
\pgfpathmoveto{\pgfqpoint{2.925150in}{1.918575in}}%
\pgfpathlineto{\pgfqpoint{2.925150in}{1.918575in}}%
\pgfpathlineto{\pgfqpoint{2.925150in}{1.921524in}}%
\pgfpathlineto{\pgfqpoint{2.929691in}{1.921524in}}%
\pgfpathlineto{\pgfqpoint{2.929691in}{1.918575in}}%
\pgfpathmoveto{\pgfqpoint{2.920609in}{3.402031in}}%
\pgfpathlineto{\pgfqpoint{2.920609in}{3.402031in}}%
\pgfpathlineto{\pgfqpoint{2.920609in}{3.404980in}}%
\pgfpathlineto{\pgfqpoint{2.925150in}{3.404980in}}%
\pgfpathlineto{\pgfqpoint{2.925150in}{3.402031in}}%
\pgfpathmoveto{\pgfqpoint{2.920609in}{3.404980in}}%
\pgfpathlineto{\pgfqpoint{2.920609in}{3.404980in}}%
\pgfpathlineto{\pgfqpoint{2.920609in}{3.407929in}}%
\pgfpathlineto{\pgfqpoint{2.925150in}{3.407929in}}%
\pgfpathlineto{\pgfqpoint{2.925150in}{3.404980in}}%
\pgfpathmoveto{\pgfqpoint{2.925150in}{3.402031in}}%
\pgfpathlineto{\pgfqpoint{2.925150in}{3.402031in}}%
\pgfpathlineto{\pgfqpoint{2.925150in}{3.404980in}}%
\pgfpathlineto{\pgfqpoint{2.929691in}{3.404980in}}%
\pgfpathlineto{\pgfqpoint{2.929691in}{3.402031in}}%
\pgfpathmoveto{\pgfqpoint{2.925150in}{3.404980in}}%
\pgfpathlineto{\pgfqpoint{2.925150in}{3.404980in}}%
\pgfpathlineto{\pgfqpoint{2.925150in}{3.407929in}}%
\pgfpathlineto{\pgfqpoint{2.929691in}{3.407929in}}%
\pgfpathlineto{\pgfqpoint{2.929691in}{3.404980in}}%
\pgfpathmoveto{\pgfqpoint{2.920609in}{3.407929in}}%
\pgfpathlineto{\pgfqpoint{2.920609in}{3.407929in}}%
\pgfpathlineto{\pgfqpoint{2.920609in}{3.410878in}}%
\pgfpathlineto{\pgfqpoint{2.925150in}{3.410878in}}%
\pgfpathlineto{\pgfqpoint{2.925150in}{3.407929in}}%
\pgfpathmoveto{\pgfqpoint{2.920609in}{3.410878in}}%
\pgfpathlineto{\pgfqpoint{2.920609in}{3.410878in}}%
\pgfpathlineto{\pgfqpoint{2.920609in}{3.413828in}}%
\pgfpathlineto{\pgfqpoint{2.925150in}{3.413828in}}%
\pgfpathlineto{\pgfqpoint{2.925150in}{3.410878in}}%
\pgfpathmoveto{\pgfqpoint{2.925150in}{3.407929in}}%
\pgfpathlineto{\pgfqpoint{2.925150in}{3.407929in}}%
\pgfpathlineto{\pgfqpoint{2.925150in}{3.410878in}}%
\pgfpathlineto{\pgfqpoint{2.929691in}{3.410878in}}%
\pgfpathlineto{\pgfqpoint{2.929691in}{3.407929in}}%
\pgfpathmoveto{\pgfqpoint{2.925150in}{3.410878in}}%
\pgfpathlineto{\pgfqpoint{2.925150in}{3.410878in}}%
\pgfpathlineto{\pgfqpoint{2.925150in}{3.413828in}}%
\pgfpathlineto{\pgfqpoint{2.929691in}{3.413828in}}%
\pgfpathlineto{\pgfqpoint{2.929691in}{3.410878in}}%
\pgfpathmoveto{\pgfqpoint{2.911527in}{3.413828in}}%
\pgfpathlineto{\pgfqpoint{2.911527in}{3.413828in}}%
\pgfpathlineto{\pgfqpoint{2.911527in}{3.416777in}}%
\pgfpathlineto{\pgfqpoint{2.916068in}{3.416777in}}%
\pgfpathlineto{\pgfqpoint{2.916068in}{3.413828in}}%
\pgfpathmoveto{\pgfqpoint{2.911527in}{3.416777in}}%
\pgfpathlineto{\pgfqpoint{2.911527in}{3.416777in}}%
\pgfpathlineto{\pgfqpoint{2.911527in}{3.419726in}}%
\pgfpathlineto{\pgfqpoint{2.916068in}{3.419726in}}%
\pgfpathlineto{\pgfqpoint{2.916068in}{3.416777in}}%
\pgfpathmoveto{\pgfqpoint{2.916068in}{3.413828in}}%
\pgfpathlineto{\pgfqpoint{2.916068in}{3.413828in}}%
\pgfpathlineto{\pgfqpoint{2.916068in}{3.416777in}}%
\pgfpathlineto{\pgfqpoint{2.920609in}{3.416777in}}%
\pgfpathlineto{\pgfqpoint{2.920609in}{3.413828in}}%
\pgfpathmoveto{\pgfqpoint{2.916068in}{3.416777in}}%
\pgfpathlineto{\pgfqpoint{2.916068in}{3.416777in}}%
\pgfpathlineto{\pgfqpoint{2.916068in}{3.419726in}}%
\pgfpathlineto{\pgfqpoint{2.920609in}{3.419726in}}%
\pgfpathlineto{\pgfqpoint{2.920609in}{3.416777in}}%
\pgfpathmoveto{\pgfqpoint{2.911527in}{3.419726in}}%
\pgfpathlineto{\pgfqpoint{2.911527in}{3.419726in}}%
\pgfpathlineto{\pgfqpoint{2.911527in}{3.422675in}}%
\pgfpathlineto{\pgfqpoint{2.916068in}{3.422675in}}%
\pgfpathlineto{\pgfqpoint{2.916068in}{3.419726in}}%
\pgfpathmoveto{\pgfqpoint{2.911527in}{3.422675in}}%
\pgfpathlineto{\pgfqpoint{2.911527in}{3.422675in}}%
\pgfpathlineto{\pgfqpoint{2.911527in}{3.425625in}}%
\pgfpathlineto{\pgfqpoint{2.916068in}{3.425625in}}%
\pgfpathlineto{\pgfqpoint{2.916068in}{3.422675in}}%
\pgfpathmoveto{\pgfqpoint{2.916068in}{3.419726in}}%
\pgfpathlineto{\pgfqpoint{2.916068in}{3.419726in}}%
\pgfpathlineto{\pgfqpoint{2.916068in}{3.422675in}}%
\pgfpathlineto{\pgfqpoint{2.920609in}{3.422675in}}%
\pgfpathlineto{\pgfqpoint{2.920609in}{3.419726in}}%
\pgfpathmoveto{\pgfqpoint{2.916068in}{3.422675in}}%
\pgfpathlineto{\pgfqpoint{2.916068in}{3.422675in}}%
\pgfpathlineto{\pgfqpoint{2.916068in}{3.425625in}}%
\pgfpathlineto{\pgfqpoint{2.920609in}{3.425625in}}%
\pgfpathlineto{\pgfqpoint{2.920609in}{3.422675in}}%
\pgfpathmoveto{\pgfqpoint{2.920609in}{3.413828in}}%
\pgfpathlineto{\pgfqpoint{2.920609in}{3.413828in}}%
\pgfpathlineto{\pgfqpoint{2.920609in}{3.416777in}}%
\pgfpathlineto{\pgfqpoint{2.925150in}{3.416777in}}%
\pgfpathlineto{\pgfqpoint{2.925150in}{3.413828in}}%
\pgfpathmoveto{\pgfqpoint{2.920609in}{3.416777in}}%
\pgfpathlineto{\pgfqpoint{2.920609in}{3.416777in}}%
\pgfpathlineto{\pgfqpoint{2.920609in}{3.419726in}}%
\pgfpathlineto{\pgfqpoint{2.925150in}{3.419726in}}%
\pgfpathlineto{\pgfqpoint{2.925150in}{3.416777in}}%
\pgfpathmoveto{\pgfqpoint{2.847952in}{3.496405in}}%
\pgfpathlineto{\pgfqpoint{2.847952in}{3.496405in}}%
\pgfpathlineto{\pgfqpoint{2.847952in}{3.499355in}}%
\pgfpathlineto{\pgfqpoint{2.852493in}{3.499355in}}%
\pgfpathlineto{\pgfqpoint{2.852493in}{3.496405in}}%
\pgfpathmoveto{\pgfqpoint{2.847952in}{3.499355in}}%
\pgfpathlineto{\pgfqpoint{2.847952in}{3.499355in}}%
\pgfpathlineto{\pgfqpoint{2.847952in}{3.502304in}}%
\pgfpathlineto{\pgfqpoint{2.852493in}{3.502304in}}%
\pgfpathlineto{\pgfqpoint{2.852493in}{3.499355in}}%
\pgfpathmoveto{\pgfqpoint{2.852493in}{3.496405in}}%
\pgfpathlineto{\pgfqpoint{2.852493in}{3.496405in}}%
\pgfpathlineto{\pgfqpoint{2.852493in}{3.499355in}}%
\pgfpathlineto{\pgfqpoint{2.857034in}{3.499355in}}%
\pgfpathlineto{\pgfqpoint{2.857034in}{3.496405in}}%
\pgfpathmoveto{\pgfqpoint{2.852493in}{3.499355in}}%
\pgfpathlineto{\pgfqpoint{2.852493in}{3.499355in}}%
\pgfpathlineto{\pgfqpoint{2.852493in}{3.502304in}}%
\pgfpathlineto{\pgfqpoint{2.857034in}{3.502304in}}%
\pgfpathlineto{\pgfqpoint{2.857034in}{3.499355in}}%
\pgfpathmoveto{\pgfqpoint{2.847952in}{3.502304in}}%
\pgfpathlineto{\pgfqpoint{2.847952in}{3.502304in}}%
\pgfpathlineto{\pgfqpoint{2.847952in}{3.505253in}}%
\pgfpathlineto{\pgfqpoint{2.852493in}{3.505253in}}%
\pgfpathlineto{\pgfqpoint{2.852493in}{3.502304in}}%
\pgfpathmoveto{\pgfqpoint{2.847952in}{3.505253in}}%
\pgfpathlineto{\pgfqpoint{2.847952in}{3.505253in}}%
\pgfpathlineto{\pgfqpoint{2.847952in}{3.508202in}}%
\pgfpathlineto{\pgfqpoint{2.852493in}{3.508202in}}%
\pgfpathlineto{\pgfqpoint{2.852493in}{3.505253in}}%
\pgfpathmoveto{\pgfqpoint{2.852493in}{3.502304in}}%
\pgfpathlineto{\pgfqpoint{2.852493in}{3.502304in}}%
\pgfpathlineto{\pgfqpoint{2.852493in}{3.505253in}}%
\pgfpathlineto{\pgfqpoint{2.857034in}{3.505253in}}%
\pgfpathlineto{\pgfqpoint{2.857034in}{3.502304in}}%
\pgfpathmoveto{\pgfqpoint{2.852493in}{3.505253in}}%
\pgfpathlineto{\pgfqpoint{2.852493in}{3.505253in}}%
\pgfpathlineto{\pgfqpoint{2.852493in}{3.508202in}}%
\pgfpathlineto{\pgfqpoint{2.857034in}{3.508202in}}%
\pgfpathlineto{\pgfqpoint{2.857034in}{3.505253in}}%
\pgfpathmoveto{\pgfqpoint{2.838870in}{3.508202in}}%
\pgfpathlineto{\pgfqpoint{2.838870in}{3.508202in}}%
\pgfpathlineto{\pgfqpoint{2.838870in}{3.511151in}}%
\pgfpathlineto{\pgfqpoint{2.843411in}{3.511151in}}%
\pgfpathlineto{\pgfqpoint{2.843411in}{3.508202in}}%
\pgfpathmoveto{\pgfqpoint{2.838870in}{3.511151in}}%
\pgfpathlineto{\pgfqpoint{2.838870in}{3.511151in}}%
\pgfpathlineto{\pgfqpoint{2.838870in}{3.514101in}}%
\pgfpathlineto{\pgfqpoint{2.843411in}{3.514101in}}%
\pgfpathlineto{\pgfqpoint{2.843411in}{3.511151in}}%
\pgfpathmoveto{\pgfqpoint{2.843411in}{3.508202in}}%
\pgfpathlineto{\pgfqpoint{2.843411in}{3.508202in}}%
\pgfpathlineto{\pgfqpoint{2.843411in}{3.511151in}}%
\pgfpathlineto{\pgfqpoint{2.847952in}{3.511151in}}%
\pgfpathlineto{\pgfqpoint{2.847952in}{3.508202in}}%
\pgfpathmoveto{\pgfqpoint{2.843411in}{3.511151in}}%
\pgfpathlineto{\pgfqpoint{2.843411in}{3.511151in}}%
\pgfpathlineto{\pgfqpoint{2.843411in}{3.514101in}}%
\pgfpathlineto{\pgfqpoint{2.847952in}{3.514101in}}%
\pgfpathlineto{\pgfqpoint{2.847952in}{3.511151in}}%
\pgfpathmoveto{\pgfqpoint{2.838870in}{3.514101in}}%
\pgfpathlineto{\pgfqpoint{2.838870in}{3.514101in}}%
\pgfpathlineto{\pgfqpoint{2.838870in}{3.517050in}}%
\pgfpathlineto{\pgfqpoint{2.843411in}{3.517050in}}%
\pgfpathlineto{\pgfqpoint{2.843411in}{3.514101in}}%
\pgfpathmoveto{\pgfqpoint{2.838870in}{3.517050in}}%
\pgfpathlineto{\pgfqpoint{2.838870in}{3.517050in}}%
\pgfpathlineto{\pgfqpoint{2.838870in}{3.519999in}}%
\pgfpathlineto{\pgfqpoint{2.843411in}{3.519999in}}%
\pgfpathlineto{\pgfqpoint{2.843411in}{3.517050in}}%
\pgfpathmoveto{\pgfqpoint{2.843411in}{3.514101in}}%
\pgfpathlineto{\pgfqpoint{2.843411in}{3.514101in}}%
\pgfpathlineto{\pgfqpoint{2.843411in}{3.517050in}}%
\pgfpathlineto{\pgfqpoint{2.847952in}{3.517050in}}%
\pgfpathlineto{\pgfqpoint{2.847952in}{3.514101in}}%
\pgfpathmoveto{\pgfqpoint{2.843411in}{3.517050in}}%
\pgfpathlineto{\pgfqpoint{2.843411in}{3.517050in}}%
\pgfpathlineto{\pgfqpoint{2.843411in}{3.519999in}}%
\pgfpathlineto{\pgfqpoint{2.847952in}{3.519999in}}%
\pgfpathlineto{\pgfqpoint{2.847952in}{3.517050in}}%
\pgfpathmoveto{\pgfqpoint{2.847952in}{3.508202in}}%
\pgfpathlineto{\pgfqpoint{2.847952in}{3.508202in}}%
\pgfpathlineto{\pgfqpoint{2.847952in}{3.511151in}}%
\pgfpathlineto{\pgfqpoint{2.852493in}{3.511151in}}%
\pgfpathlineto{\pgfqpoint{2.852493in}{3.508202in}}%
\pgfpathmoveto{\pgfqpoint{2.847952in}{3.511151in}}%
\pgfpathlineto{\pgfqpoint{2.847952in}{3.511151in}}%
\pgfpathlineto{\pgfqpoint{2.847952in}{3.514101in}}%
\pgfpathlineto{\pgfqpoint{2.852493in}{3.514101in}}%
\pgfpathlineto{\pgfqpoint{2.852493in}{3.511151in}}%
\pgfpathmoveto{\pgfqpoint{2.884280in}{3.449218in}}%
\pgfpathlineto{\pgfqpoint{2.884280in}{3.449218in}}%
\pgfpathlineto{\pgfqpoint{2.884280in}{3.452167in}}%
\pgfpathlineto{\pgfqpoint{2.888821in}{3.452167in}}%
\pgfpathlineto{\pgfqpoint{2.888821in}{3.449218in}}%
\pgfpathmoveto{\pgfqpoint{2.884280in}{3.452167in}}%
\pgfpathlineto{\pgfqpoint{2.884280in}{3.452167in}}%
\pgfpathlineto{\pgfqpoint{2.884280in}{3.455117in}}%
\pgfpathlineto{\pgfqpoint{2.888821in}{3.455117in}}%
\pgfpathlineto{\pgfqpoint{2.888821in}{3.452167in}}%
\pgfpathmoveto{\pgfqpoint{2.888821in}{3.449218in}}%
\pgfpathlineto{\pgfqpoint{2.888821in}{3.449218in}}%
\pgfpathlineto{\pgfqpoint{2.888821in}{3.452167in}}%
\pgfpathlineto{\pgfqpoint{2.893363in}{3.452167in}}%
\pgfpathlineto{\pgfqpoint{2.893363in}{3.449218in}}%
\pgfpathmoveto{\pgfqpoint{2.888821in}{3.452167in}}%
\pgfpathlineto{\pgfqpoint{2.888821in}{3.452167in}}%
\pgfpathlineto{\pgfqpoint{2.888821in}{3.455117in}}%
\pgfpathlineto{\pgfqpoint{2.893363in}{3.455117in}}%
\pgfpathlineto{\pgfqpoint{2.893363in}{3.452167in}}%
\pgfpathmoveto{\pgfqpoint{2.884280in}{3.455117in}}%
\pgfpathlineto{\pgfqpoint{2.884280in}{3.455117in}}%
\pgfpathlineto{\pgfqpoint{2.884280in}{3.458066in}}%
\pgfpathlineto{\pgfqpoint{2.888821in}{3.458066in}}%
\pgfpathlineto{\pgfqpoint{2.888821in}{3.455117in}}%
\pgfpathmoveto{\pgfqpoint{2.884280in}{3.458066in}}%
\pgfpathlineto{\pgfqpoint{2.884280in}{3.458066in}}%
\pgfpathlineto{\pgfqpoint{2.884280in}{3.461015in}}%
\pgfpathlineto{\pgfqpoint{2.888821in}{3.461015in}}%
\pgfpathlineto{\pgfqpoint{2.888821in}{3.458066in}}%
\pgfpathmoveto{\pgfqpoint{2.888821in}{3.455117in}}%
\pgfpathlineto{\pgfqpoint{2.888821in}{3.455117in}}%
\pgfpathlineto{\pgfqpoint{2.888821in}{3.458066in}}%
\pgfpathlineto{\pgfqpoint{2.893363in}{3.458066in}}%
\pgfpathlineto{\pgfqpoint{2.893363in}{3.455117in}}%
\pgfpathmoveto{\pgfqpoint{2.888821in}{3.458066in}}%
\pgfpathlineto{\pgfqpoint{2.888821in}{3.458066in}}%
\pgfpathlineto{\pgfqpoint{2.888821in}{3.461015in}}%
\pgfpathlineto{\pgfqpoint{2.893363in}{3.461015in}}%
\pgfpathlineto{\pgfqpoint{2.893363in}{3.458066in}}%
\pgfpathmoveto{\pgfqpoint{2.875198in}{3.461015in}}%
\pgfpathlineto{\pgfqpoint{2.875198in}{3.461015in}}%
\pgfpathlineto{\pgfqpoint{2.875198in}{3.463964in}}%
\pgfpathlineto{\pgfqpoint{2.879739in}{3.463964in}}%
\pgfpathlineto{\pgfqpoint{2.879739in}{3.461015in}}%
\pgfpathmoveto{\pgfqpoint{2.875198in}{3.463964in}}%
\pgfpathlineto{\pgfqpoint{2.875198in}{3.463964in}}%
\pgfpathlineto{\pgfqpoint{2.875198in}{3.466913in}}%
\pgfpathlineto{\pgfqpoint{2.879739in}{3.466913in}}%
\pgfpathlineto{\pgfqpoint{2.879739in}{3.463964in}}%
\pgfpathmoveto{\pgfqpoint{2.879739in}{3.461015in}}%
\pgfpathlineto{\pgfqpoint{2.879739in}{3.461015in}}%
\pgfpathlineto{\pgfqpoint{2.879739in}{3.463964in}}%
\pgfpathlineto{\pgfqpoint{2.884280in}{3.463964in}}%
\pgfpathlineto{\pgfqpoint{2.884280in}{3.461015in}}%
\pgfpathmoveto{\pgfqpoint{2.879739in}{3.463964in}}%
\pgfpathlineto{\pgfqpoint{2.879739in}{3.463964in}}%
\pgfpathlineto{\pgfqpoint{2.879739in}{3.466913in}}%
\pgfpathlineto{\pgfqpoint{2.884280in}{3.466913in}}%
\pgfpathlineto{\pgfqpoint{2.884280in}{3.463964in}}%
\pgfpathmoveto{\pgfqpoint{2.875198in}{3.466913in}}%
\pgfpathlineto{\pgfqpoint{2.875198in}{3.466913in}}%
\pgfpathlineto{\pgfqpoint{2.875198in}{3.469863in}}%
\pgfpathlineto{\pgfqpoint{2.879739in}{3.469863in}}%
\pgfpathlineto{\pgfqpoint{2.879739in}{3.466913in}}%
\pgfpathmoveto{\pgfqpoint{2.875198in}{3.469863in}}%
\pgfpathlineto{\pgfqpoint{2.875198in}{3.469863in}}%
\pgfpathlineto{\pgfqpoint{2.875198in}{3.472812in}}%
\pgfpathlineto{\pgfqpoint{2.879739in}{3.472812in}}%
\pgfpathlineto{\pgfqpoint{2.879739in}{3.469863in}}%
\pgfpathmoveto{\pgfqpoint{2.879739in}{3.466913in}}%
\pgfpathlineto{\pgfqpoint{2.879739in}{3.466913in}}%
\pgfpathlineto{\pgfqpoint{2.879739in}{3.469863in}}%
\pgfpathlineto{\pgfqpoint{2.884280in}{3.469863in}}%
\pgfpathlineto{\pgfqpoint{2.884280in}{3.466913in}}%
\pgfpathmoveto{\pgfqpoint{2.879739in}{3.469863in}}%
\pgfpathlineto{\pgfqpoint{2.879739in}{3.469863in}}%
\pgfpathlineto{\pgfqpoint{2.879739in}{3.472812in}}%
\pgfpathlineto{\pgfqpoint{2.884280in}{3.472812in}}%
\pgfpathlineto{\pgfqpoint{2.884280in}{3.469863in}}%
\pgfpathmoveto{\pgfqpoint{2.884280in}{3.461015in}}%
\pgfpathlineto{\pgfqpoint{2.884280in}{3.461015in}}%
\pgfpathlineto{\pgfqpoint{2.884280in}{3.463964in}}%
\pgfpathlineto{\pgfqpoint{2.888821in}{3.463964in}}%
\pgfpathlineto{\pgfqpoint{2.888821in}{3.461015in}}%
\pgfpathmoveto{\pgfqpoint{2.884280in}{3.463964in}}%
\pgfpathlineto{\pgfqpoint{2.884280in}{3.463964in}}%
\pgfpathlineto{\pgfqpoint{2.884280in}{3.466913in}}%
\pgfpathlineto{\pgfqpoint{2.888821in}{3.466913in}}%
\pgfpathlineto{\pgfqpoint{2.888821in}{3.463964in}}%
\pgfpathmoveto{\pgfqpoint{2.902445in}{3.425625in}}%
\pgfpathlineto{\pgfqpoint{2.902445in}{3.425625in}}%
\pgfpathlineto{\pgfqpoint{2.902445in}{3.428574in}}%
\pgfpathlineto{\pgfqpoint{2.906986in}{3.428574in}}%
\pgfpathlineto{\pgfqpoint{2.906986in}{3.425625in}}%
\pgfpathmoveto{\pgfqpoint{2.902445in}{3.428574in}}%
\pgfpathlineto{\pgfqpoint{2.902445in}{3.428574in}}%
\pgfpathlineto{\pgfqpoint{2.902445in}{3.431523in}}%
\pgfpathlineto{\pgfqpoint{2.906986in}{3.431523in}}%
\pgfpathlineto{\pgfqpoint{2.906986in}{3.428574in}}%
\pgfpathmoveto{\pgfqpoint{2.906986in}{3.425625in}}%
\pgfpathlineto{\pgfqpoint{2.906986in}{3.425625in}}%
\pgfpathlineto{\pgfqpoint{2.906986in}{3.428574in}}%
\pgfpathlineto{\pgfqpoint{2.911527in}{3.428574in}}%
\pgfpathlineto{\pgfqpoint{2.911527in}{3.425625in}}%
\pgfpathmoveto{\pgfqpoint{2.906986in}{3.428574in}}%
\pgfpathlineto{\pgfqpoint{2.906986in}{3.428574in}}%
\pgfpathlineto{\pgfqpoint{2.906986in}{3.431523in}}%
\pgfpathlineto{\pgfqpoint{2.911527in}{3.431523in}}%
\pgfpathlineto{\pgfqpoint{2.911527in}{3.428574in}}%
\pgfpathmoveto{\pgfqpoint{2.902445in}{3.431523in}}%
\pgfpathlineto{\pgfqpoint{2.902445in}{3.431523in}}%
\pgfpathlineto{\pgfqpoint{2.902445in}{3.434472in}}%
\pgfpathlineto{\pgfqpoint{2.906986in}{3.434472in}}%
\pgfpathlineto{\pgfqpoint{2.906986in}{3.431523in}}%
\pgfpathmoveto{\pgfqpoint{2.902445in}{3.434472in}}%
\pgfpathlineto{\pgfqpoint{2.902445in}{3.434472in}}%
\pgfpathlineto{\pgfqpoint{2.902445in}{3.437421in}}%
\pgfpathlineto{\pgfqpoint{2.906986in}{3.437421in}}%
\pgfpathlineto{\pgfqpoint{2.906986in}{3.434472in}}%
\pgfpathmoveto{\pgfqpoint{2.906986in}{3.431523in}}%
\pgfpathlineto{\pgfqpoint{2.906986in}{3.431523in}}%
\pgfpathlineto{\pgfqpoint{2.906986in}{3.434472in}}%
\pgfpathlineto{\pgfqpoint{2.911527in}{3.434472in}}%
\pgfpathlineto{\pgfqpoint{2.911527in}{3.431523in}}%
\pgfpathmoveto{\pgfqpoint{2.906986in}{3.434472in}}%
\pgfpathlineto{\pgfqpoint{2.906986in}{3.434472in}}%
\pgfpathlineto{\pgfqpoint{2.906986in}{3.437421in}}%
\pgfpathlineto{\pgfqpoint{2.911527in}{3.437421in}}%
\pgfpathlineto{\pgfqpoint{2.911527in}{3.434472in}}%
\pgfpathmoveto{\pgfqpoint{2.893363in}{3.437421in}}%
\pgfpathlineto{\pgfqpoint{2.893363in}{3.437421in}}%
\pgfpathlineto{\pgfqpoint{2.893363in}{3.440371in}}%
\pgfpathlineto{\pgfqpoint{2.897904in}{3.440371in}}%
\pgfpathlineto{\pgfqpoint{2.897904in}{3.437421in}}%
\pgfpathmoveto{\pgfqpoint{2.893363in}{3.440371in}}%
\pgfpathlineto{\pgfqpoint{2.893363in}{3.440371in}}%
\pgfpathlineto{\pgfqpoint{2.893363in}{3.443320in}}%
\pgfpathlineto{\pgfqpoint{2.897904in}{3.443320in}}%
\pgfpathlineto{\pgfqpoint{2.897904in}{3.440371in}}%
\pgfpathmoveto{\pgfqpoint{2.897904in}{3.437421in}}%
\pgfpathlineto{\pgfqpoint{2.897904in}{3.437421in}}%
\pgfpathlineto{\pgfqpoint{2.897904in}{3.440371in}}%
\pgfpathlineto{\pgfqpoint{2.902445in}{3.440371in}}%
\pgfpathlineto{\pgfqpoint{2.902445in}{3.437421in}}%
\pgfpathmoveto{\pgfqpoint{2.897904in}{3.440371in}}%
\pgfpathlineto{\pgfqpoint{2.897904in}{3.440371in}}%
\pgfpathlineto{\pgfqpoint{2.897904in}{3.443320in}}%
\pgfpathlineto{\pgfqpoint{2.902445in}{3.443320in}}%
\pgfpathlineto{\pgfqpoint{2.902445in}{3.440371in}}%
\pgfpathmoveto{\pgfqpoint{2.893363in}{3.443320in}}%
\pgfpathlineto{\pgfqpoint{2.893363in}{3.443320in}}%
\pgfpathlineto{\pgfqpoint{2.893363in}{3.446269in}}%
\pgfpathlineto{\pgfqpoint{2.897904in}{3.446269in}}%
\pgfpathlineto{\pgfqpoint{2.897904in}{3.443320in}}%
\pgfpathmoveto{\pgfqpoint{2.893363in}{3.446269in}}%
\pgfpathlineto{\pgfqpoint{2.893363in}{3.446269in}}%
\pgfpathlineto{\pgfqpoint{2.893363in}{3.449218in}}%
\pgfpathlineto{\pgfqpoint{2.897904in}{3.449218in}}%
\pgfpathlineto{\pgfqpoint{2.897904in}{3.446269in}}%
\pgfpathmoveto{\pgfqpoint{2.897904in}{3.443320in}}%
\pgfpathlineto{\pgfqpoint{2.897904in}{3.443320in}}%
\pgfpathlineto{\pgfqpoint{2.897904in}{3.446269in}}%
\pgfpathlineto{\pgfqpoint{2.902445in}{3.446269in}}%
\pgfpathlineto{\pgfqpoint{2.902445in}{3.443320in}}%
\pgfpathmoveto{\pgfqpoint{2.897904in}{3.446269in}}%
\pgfpathlineto{\pgfqpoint{2.897904in}{3.446269in}}%
\pgfpathlineto{\pgfqpoint{2.897904in}{3.449218in}}%
\pgfpathlineto{\pgfqpoint{2.902445in}{3.449218in}}%
\pgfpathlineto{\pgfqpoint{2.902445in}{3.446269in}}%
\pgfpathmoveto{\pgfqpoint{2.902445in}{3.437421in}}%
\pgfpathlineto{\pgfqpoint{2.902445in}{3.437421in}}%
\pgfpathlineto{\pgfqpoint{2.902445in}{3.440371in}}%
\pgfpathlineto{\pgfqpoint{2.906986in}{3.440371in}}%
\pgfpathlineto{\pgfqpoint{2.906986in}{3.437421in}}%
\pgfpathmoveto{\pgfqpoint{2.902445in}{3.440371in}}%
\pgfpathlineto{\pgfqpoint{2.902445in}{3.440371in}}%
\pgfpathlineto{\pgfqpoint{2.902445in}{3.443320in}}%
\pgfpathlineto{\pgfqpoint{2.906986in}{3.443320in}}%
\pgfpathlineto{\pgfqpoint{2.906986in}{3.440371in}}%
\pgfpathmoveto{\pgfqpoint{2.911527in}{3.425625in}}%
\pgfpathlineto{\pgfqpoint{2.911527in}{3.425625in}}%
\pgfpathlineto{\pgfqpoint{2.911527in}{3.428574in}}%
\pgfpathlineto{\pgfqpoint{2.916068in}{3.428574in}}%
\pgfpathlineto{\pgfqpoint{2.916068in}{3.425625in}}%
\pgfpathmoveto{\pgfqpoint{2.911527in}{3.428574in}}%
\pgfpathlineto{\pgfqpoint{2.911527in}{3.428574in}}%
\pgfpathlineto{\pgfqpoint{2.911527in}{3.431523in}}%
\pgfpathlineto{\pgfqpoint{2.916068in}{3.431523in}}%
\pgfpathlineto{\pgfqpoint{2.916068in}{3.428574in}}%
\pgfpathmoveto{\pgfqpoint{2.893363in}{3.449218in}}%
\pgfpathlineto{\pgfqpoint{2.893363in}{3.449218in}}%
\pgfpathlineto{\pgfqpoint{2.893363in}{3.452167in}}%
\pgfpathlineto{\pgfqpoint{2.897904in}{3.452167in}}%
\pgfpathlineto{\pgfqpoint{2.897904in}{3.449218in}}%
\pgfpathmoveto{\pgfqpoint{2.893363in}{3.452167in}}%
\pgfpathlineto{\pgfqpoint{2.893363in}{3.452167in}}%
\pgfpathlineto{\pgfqpoint{2.893363in}{3.455117in}}%
\pgfpathlineto{\pgfqpoint{2.897904in}{3.455117in}}%
\pgfpathlineto{\pgfqpoint{2.897904in}{3.452167in}}%
\pgfpathmoveto{\pgfqpoint{2.866116in}{3.472812in}}%
\pgfpathlineto{\pgfqpoint{2.866116in}{3.472812in}}%
\pgfpathlineto{\pgfqpoint{2.866116in}{3.475761in}}%
\pgfpathlineto{\pgfqpoint{2.870657in}{3.475761in}}%
\pgfpathlineto{\pgfqpoint{2.870657in}{3.472812in}}%
\pgfpathmoveto{\pgfqpoint{2.866116in}{3.475761in}}%
\pgfpathlineto{\pgfqpoint{2.866116in}{3.475761in}}%
\pgfpathlineto{\pgfqpoint{2.866116in}{3.478710in}}%
\pgfpathlineto{\pgfqpoint{2.870657in}{3.478710in}}%
\pgfpathlineto{\pgfqpoint{2.870657in}{3.475761in}}%
\pgfpathmoveto{\pgfqpoint{2.870657in}{3.472812in}}%
\pgfpathlineto{\pgfqpoint{2.870657in}{3.472812in}}%
\pgfpathlineto{\pgfqpoint{2.870657in}{3.475761in}}%
\pgfpathlineto{\pgfqpoint{2.875198in}{3.475761in}}%
\pgfpathlineto{\pgfqpoint{2.875198in}{3.472812in}}%
\pgfpathmoveto{\pgfqpoint{2.870657in}{3.475761in}}%
\pgfpathlineto{\pgfqpoint{2.870657in}{3.475761in}}%
\pgfpathlineto{\pgfqpoint{2.870657in}{3.478710in}}%
\pgfpathlineto{\pgfqpoint{2.875198in}{3.478710in}}%
\pgfpathlineto{\pgfqpoint{2.875198in}{3.475761in}}%
\pgfpathmoveto{\pgfqpoint{2.866116in}{3.478710in}}%
\pgfpathlineto{\pgfqpoint{2.866116in}{3.478710in}}%
\pgfpathlineto{\pgfqpoint{2.866116in}{3.481659in}}%
\pgfpathlineto{\pgfqpoint{2.870657in}{3.481659in}}%
\pgfpathlineto{\pgfqpoint{2.870657in}{3.478710in}}%
\pgfpathmoveto{\pgfqpoint{2.866116in}{3.481659in}}%
\pgfpathlineto{\pgfqpoint{2.866116in}{3.481659in}}%
\pgfpathlineto{\pgfqpoint{2.866116in}{3.484609in}}%
\pgfpathlineto{\pgfqpoint{2.870657in}{3.484609in}}%
\pgfpathlineto{\pgfqpoint{2.870657in}{3.481659in}}%
\pgfpathmoveto{\pgfqpoint{2.870657in}{3.478710in}}%
\pgfpathlineto{\pgfqpoint{2.870657in}{3.478710in}}%
\pgfpathlineto{\pgfqpoint{2.870657in}{3.481659in}}%
\pgfpathlineto{\pgfqpoint{2.875198in}{3.481659in}}%
\pgfpathlineto{\pgfqpoint{2.875198in}{3.478710in}}%
\pgfpathmoveto{\pgfqpoint{2.870657in}{3.481659in}}%
\pgfpathlineto{\pgfqpoint{2.870657in}{3.481659in}}%
\pgfpathlineto{\pgfqpoint{2.870657in}{3.484609in}}%
\pgfpathlineto{\pgfqpoint{2.875198in}{3.484609in}}%
\pgfpathlineto{\pgfqpoint{2.875198in}{3.481659in}}%
\pgfpathmoveto{\pgfqpoint{2.857034in}{3.484609in}}%
\pgfpathlineto{\pgfqpoint{2.857034in}{3.484609in}}%
\pgfpathlineto{\pgfqpoint{2.857034in}{3.487558in}}%
\pgfpathlineto{\pgfqpoint{2.861575in}{3.487558in}}%
\pgfpathlineto{\pgfqpoint{2.861575in}{3.484609in}}%
\pgfpathmoveto{\pgfqpoint{2.857034in}{3.487558in}}%
\pgfpathlineto{\pgfqpoint{2.857034in}{3.487558in}}%
\pgfpathlineto{\pgfqpoint{2.857034in}{3.490507in}}%
\pgfpathlineto{\pgfqpoint{2.861575in}{3.490507in}}%
\pgfpathlineto{\pgfqpoint{2.861575in}{3.487558in}}%
\pgfpathmoveto{\pgfqpoint{2.861575in}{3.484609in}}%
\pgfpathlineto{\pgfqpoint{2.861575in}{3.484609in}}%
\pgfpathlineto{\pgfqpoint{2.861575in}{3.487558in}}%
\pgfpathlineto{\pgfqpoint{2.866116in}{3.487558in}}%
\pgfpathlineto{\pgfqpoint{2.866116in}{3.484609in}}%
\pgfpathmoveto{\pgfqpoint{2.861575in}{3.487558in}}%
\pgfpathlineto{\pgfqpoint{2.861575in}{3.487558in}}%
\pgfpathlineto{\pgfqpoint{2.861575in}{3.490507in}}%
\pgfpathlineto{\pgfqpoint{2.866116in}{3.490507in}}%
\pgfpathlineto{\pgfqpoint{2.866116in}{3.487558in}}%
\pgfpathmoveto{\pgfqpoint{2.857034in}{3.490507in}}%
\pgfpathlineto{\pgfqpoint{2.857034in}{3.490507in}}%
\pgfpathlineto{\pgfqpoint{2.857034in}{3.493456in}}%
\pgfpathlineto{\pgfqpoint{2.861575in}{3.493456in}}%
\pgfpathlineto{\pgfqpoint{2.861575in}{3.490507in}}%
\pgfpathmoveto{\pgfqpoint{2.857034in}{3.493456in}}%
\pgfpathlineto{\pgfqpoint{2.857034in}{3.493456in}}%
\pgfpathlineto{\pgfqpoint{2.857034in}{3.496405in}}%
\pgfpathlineto{\pgfqpoint{2.861575in}{3.496405in}}%
\pgfpathlineto{\pgfqpoint{2.861575in}{3.493456in}}%
\pgfpathmoveto{\pgfqpoint{2.861575in}{3.490507in}}%
\pgfpathlineto{\pgfqpoint{2.861575in}{3.490507in}}%
\pgfpathlineto{\pgfqpoint{2.861575in}{3.493456in}}%
\pgfpathlineto{\pgfqpoint{2.866116in}{3.493456in}}%
\pgfpathlineto{\pgfqpoint{2.866116in}{3.490507in}}%
\pgfpathmoveto{\pgfqpoint{2.861575in}{3.493456in}}%
\pgfpathlineto{\pgfqpoint{2.861575in}{3.493456in}}%
\pgfpathlineto{\pgfqpoint{2.861575in}{3.496405in}}%
\pgfpathlineto{\pgfqpoint{2.866116in}{3.496405in}}%
\pgfpathlineto{\pgfqpoint{2.866116in}{3.493456in}}%
\pgfpathmoveto{\pgfqpoint{2.866116in}{3.484609in}}%
\pgfpathlineto{\pgfqpoint{2.866116in}{3.484609in}}%
\pgfpathlineto{\pgfqpoint{2.866116in}{3.487558in}}%
\pgfpathlineto{\pgfqpoint{2.870657in}{3.487558in}}%
\pgfpathlineto{\pgfqpoint{2.870657in}{3.484609in}}%
\pgfpathmoveto{\pgfqpoint{2.866116in}{3.487558in}}%
\pgfpathlineto{\pgfqpoint{2.866116in}{3.487558in}}%
\pgfpathlineto{\pgfqpoint{2.866116in}{3.490507in}}%
\pgfpathlineto{\pgfqpoint{2.870657in}{3.490507in}}%
\pgfpathlineto{\pgfqpoint{2.870657in}{3.487558in}}%
\pgfpathmoveto{\pgfqpoint{2.875198in}{3.472812in}}%
\pgfpathlineto{\pgfqpoint{2.875198in}{3.472812in}}%
\pgfpathlineto{\pgfqpoint{2.875198in}{3.475761in}}%
\pgfpathlineto{\pgfqpoint{2.879739in}{3.475761in}}%
\pgfpathlineto{\pgfqpoint{2.879739in}{3.472812in}}%
\pgfpathmoveto{\pgfqpoint{2.875198in}{3.475761in}}%
\pgfpathlineto{\pgfqpoint{2.875198in}{3.475761in}}%
\pgfpathlineto{\pgfqpoint{2.875198in}{3.478710in}}%
\pgfpathlineto{\pgfqpoint{2.879739in}{3.478710in}}%
\pgfpathlineto{\pgfqpoint{2.879739in}{3.475761in}}%
\pgfpathmoveto{\pgfqpoint{2.857034in}{3.496405in}}%
\pgfpathlineto{\pgfqpoint{2.857034in}{3.496405in}}%
\pgfpathlineto{\pgfqpoint{2.857034in}{3.499355in}}%
\pgfpathlineto{\pgfqpoint{2.861575in}{3.499355in}}%
\pgfpathlineto{\pgfqpoint{2.861575in}{3.496405in}}%
\pgfpathmoveto{\pgfqpoint{2.857034in}{3.499355in}}%
\pgfpathlineto{\pgfqpoint{2.857034in}{3.499355in}}%
\pgfpathlineto{\pgfqpoint{2.857034in}{3.502304in}}%
\pgfpathlineto{\pgfqpoint{2.861575in}{3.502304in}}%
\pgfpathlineto{\pgfqpoint{2.861575in}{3.499355in}}%
\pgfpathmoveto{\pgfqpoint{2.929691in}{1.915626in}}%
\pgfpathlineto{\pgfqpoint{2.929691in}{1.915626in}}%
\pgfpathlineto{\pgfqpoint{2.929691in}{1.918575in}}%
\pgfpathlineto{\pgfqpoint{2.934232in}{1.918575in}}%
\pgfpathlineto{\pgfqpoint{2.934232in}{1.915626in}}%
\pgfpathmoveto{\pgfqpoint{2.929691in}{1.918575in}}%
\pgfpathlineto{\pgfqpoint{2.929691in}{1.918575in}}%
\pgfpathlineto{\pgfqpoint{2.929691in}{1.921524in}}%
\pgfpathlineto{\pgfqpoint{2.934232in}{1.921524in}}%
\pgfpathlineto{\pgfqpoint{2.934232in}{1.918575in}}%
\pgfpathmoveto{\pgfqpoint{2.934232in}{1.918575in}}%
\pgfpathlineto{\pgfqpoint{2.934232in}{1.918575in}}%
\pgfpathlineto{\pgfqpoint{2.934232in}{1.921524in}}%
\pgfpathlineto{\pgfqpoint{2.938773in}{1.921524in}}%
\pgfpathlineto{\pgfqpoint{2.938773in}{1.918575in}}%
\pgfpathmoveto{\pgfqpoint{2.929691in}{1.921524in}}%
\pgfpathlineto{\pgfqpoint{2.929691in}{1.921524in}}%
\pgfpathlineto{\pgfqpoint{2.929691in}{1.924473in}}%
\pgfpathlineto{\pgfqpoint{2.934232in}{1.924473in}}%
\pgfpathlineto{\pgfqpoint{2.934232in}{1.921524in}}%
\pgfpathmoveto{\pgfqpoint{2.929691in}{1.924473in}}%
\pgfpathlineto{\pgfqpoint{2.929691in}{1.924473in}}%
\pgfpathlineto{\pgfqpoint{2.929691in}{1.927422in}}%
\pgfpathlineto{\pgfqpoint{2.934232in}{1.927422in}}%
\pgfpathlineto{\pgfqpoint{2.934232in}{1.924473in}}%
\pgfpathmoveto{\pgfqpoint{2.934232in}{1.921524in}}%
\pgfpathlineto{\pgfqpoint{2.934232in}{1.921524in}}%
\pgfpathlineto{\pgfqpoint{2.934232in}{1.924473in}}%
\pgfpathlineto{\pgfqpoint{2.938773in}{1.924473in}}%
\pgfpathlineto{\pgfqpoint{2.938773in}{1.921524in}}%
\pgfpathmoveto{\pgfqpoint{2.934232in}{1.924473in}}%
\pgfpathlineto{\pgfqpoint{2.934232in}{1.924473in}}%
\pgfpathlineto{\pgfqpoint{2.934232in}{1.927422in}}%
\pgfpathlineto{\pgfqpoint{2.938773in}{1.927422in}}%
\pgfpathlineto{\pgfqpoint{2.938773in}{1.924473in}}%
\pgfpathmoveto{\pgfqpoint{2.938773in}{1.921524in}}%
\pgfpathlineto{\pgfqpoint{2.938773in}{1.921524in}}%
\pgfpathlineto{\pgfqpoint{2.938773in}{1.924473in}}%
\pgfpathlineto{\pgfqpoint{2.943313in}{1.924473in}}%
\pgfpathlineto{\pgfqpoint{2.943313in}{1.921524in}}%
\pgfpathmoveto{\pgfqpoint{2.938773in}{1.924473in}}%
\pgfpathlineto{\pgfqpoint{2.938773in}{1.924473in}}%
\pgfpathlineto{\pgfqpoint{2.938773in}{1.927422in}}%
\pgfpathlineto{\pgfqpoint{2.943313in}{1.927422in}}%
\pgfpathlineto{\pgfqpoint{2.943313in}{1.924473in}}%
\pgfpathmoveto{\pgfqpoint{2.943313in}{1.924473in}}%
\pgfpathlineto{\pgfqpoint{2.943313in}{1.924473in}}%
\pgfpathlineto{\pgfqpoint{2.943313in}{1.927422in}}%
\pgfpathlineto{\pgfqpoint{2.947854in}{1.927422in}}%
\pgfpathlineto{\pgfqpoint{2.947854in}{1.924473in}}%
\pgfpathmoveto{\pgfqpoint{2.938773in}{1.927422in}}%
\pgfpathlineto{\pgfqpoint{2.938773in}{1.927422in}}%
\pgfpathlineto{\pgfqpoint{2.938773in}{1.930371in}}%
\pgfpathlineto{\pgfqpoint{2.943313in}{1.930371in}}%
\pgfpathlineto{\pgfqpoint{2.943313in}{1.927422in}}%
\pgfpathmoveto{\pgfqpoint{2.938773in}{1.930371in}}%
\pgfpathlineto{\pgfqpoint{2.938773in}{1.930371in}}%
\pgfpathlineto{\pgfqpoint{2.938773in}{1.933320in}}%
\pgfpathlineto{\pgfqpoint{2.943313in}{1.933320in}}%
\pgfpathlineto{\pgfqpoint{2.943313in}{1.930371in}}%
\pgfpathmoveto{\pgfqpoint{2.943313in}{1.927422in}}%
\pgfpathlineto{\pgfqpoint{2.943313in}{1.927422in}}%
\pgfpathlineto{\pgfqpoint{2.943313in}{1.930371in}}%
\pgfpathlineto{\pgfqpoint{2.947854in}{1.930371in}}%
\pgfpathlineto{\pgfqpoint{2.947854in}{1.927422in}}%
\pgfpathmoveto{\pgfqpoint{2.943313in}{1.930371in}}%
\pgfpathlineto{\pgfqpoint{2.943313in}{1.930371in}}%
\pgfpathlineto{\pgfqpoint{2.943313in}{1.933320in}}%
\pgfpathlineto{\pgfqpoint{2.947854in}{1.933320in}}%
\pgfpathlineto{\pgfqpoint{2.947854in}{1.930371in}}%
\pgfpathmoveto{\pgfqpoint{2.947854in}{1.927422in}}%
\pgfpathlineto{\pgfqpoint{2.947854in}{1.927422in}}%
\pgfpathlineto{\pgfqpoint{2.947854in}{1.930371in}}%
\pgfpathlineto{\pgfqpoint{2.952395in}{1.930371in}}%
\pgfpathlineto{\pgfqpoint{2.952395in}{1.927422in}}%
\pgfpathmoveto{\pgfqpoint{2.947854in}{1.930371in}}%
\pgfpathlineto{\pgfqpoint{2.947854in}{1.930371in}}%
\pgfpathlineto{\pgfqpoint{2.947854in}{1.933320in}}%
\pgfpathlineto{\pgfqpoint{2.952395in}{1.933320in}}%
\pgfpathlineto{\pgfqpoint{2.952395in}{1.930371in}}%
\pgfpathmoveto{\pgfqpoint{2.952395in}{1.930371in}}%
\pgfpathlineto{\pgfqpoint{2.952395in}{1.930371in}}%
\pgfpathlineto{\pgfqpoint{2.952395in}{1.933320in}}%
\pgfpathlineto{\pgfqpoint{2.956936in}{1.933320in}}%
\pgfpathlineto{\pgfqpoint{2.956936in}{1.930371in}}%
\pgfpathmoveto{\pgfqpoint{2.947854in}{1.933320in}}%
\pgfpathlineto{\pgfqpoint{2.947854in}{1.933320in}}%
\pgfpathlineto{\pgfqpoint{2.947854in}{1.936270in}}%
\pgfpathlineto{\pgfqpoint{2.952395in}{1.936270in}}%
\pgfpathlineto{\pgfqpoint{2.952395in}{1.933320in}}%
\pgfpathmoveto{\pgfqpoint{2.947854in}{1.936270in}}%
\pgfpathlineto{\pgfqpoint{2.947854in}{1.936270in}}%
\pgfpathlineto{\pgfqpoint{2.947854in}{1.939219in}}%
\pgfpathlineto{\pgfqpoint{2.952395in}{1.939219in}}%
\pgfpathlineto{\pgfqpoint{2.952395in}{1.936270in}}%
\pgfpathmoveto{\pgfqpoint{2.952395in}{1.933320in}}%
\pgfpathlineto{\pgfqpoint{2.952395in}{1.933320in}}%
\pgfpathlineto{\pgfqpoint{2.952395in}{1.936270in}}%
\pgfpathlineto{\pgfqpoint{2.956936in}{1.936270in}}%
\pgfpathlineto{\pgfqpoint{2.956936in}{1.933320in}}%
\pgfpathmoveto{\pgfqpoint{2.952395in}{1.936270in}}%
\pgfpathlineto{\pgfqpoint{2.952395in}{1.936270in}}%
\pgfpathlineto{\pgfqpoint{2.952395in}{1.939219in}}%
\pgfpathlineto{\pgfqpoint{2.956936in}{1.939219in}}%
\pgfpathlineto{\pgfqpoint{2.956936in}{1.936270in}}%
\pgfpathmoveto{\pgfqpoint{2.956936in}{1.933320in}}%
\pgfpathlineto{\pgfqpoint{2.956936in}{1.933320in}}%
\pgfpathlineto{\pgfqpoint{2.956936in}{1.936270in}}%
\pgfpathlineto{\pgfqpoint{2.961477in}{1.936270in}}%
\pgfpathlineto{\pgfqpoint{2.961477in}{1.933320in}}%
\pgfpathmoveto{\pgfqpoint{2.956936in}{1.936270in}}%
\pgfpathlineto{\pgfqpoint{2.956936in}{1.936270in}}%
\pgfpathlineto{\pgfqpoint{2.956936in}{1.939219in}}%
\pgfpathlineto{\pgfqpoint{2.961477in}{1.939219in}}%
\pgfpathlineto{\pgfqpoint{2.961477in}{1.936270in}}%
\pgfpathmoveto{\pgfqpoint{2.961477in}{1.936270in}}%
\pgfpathlineto{\pgfqpoint{2.961477in}{1.936270in}}%
\pgfpathlineto{\pgfqpoint{2.961477in}{1.939219in}}%
\pgfpathlineto{\pgfqpoint{2.966017in}{1.939219in}}%
\pgfpathlineto{\pgfqpoint{2.966017in}{1.936270in}}%
\pgfpathmoveto{\pgfqpoint{2.956936in}{1.939219in}}%
\pgfpathlineto{\pgfqpoint{2.956936in}{1.939219in}}%
\pgfpathlineto{\pgfqpoint{2.956936in}{1.942168in}}%
\pgfpathlineto{\pgfqpoint{2.961477in}{1.942168in}}%
\pgfpathlineto{\pgfqpoint{2.961477in}{1.939219in}}%
\pgfpathmoveto{\pgfqpoint{2.956936in}{1.942168in}}%
\pgfpathlineto{\pgfqpoint{2.956936in}{1.942168in}}%
\pgfpathlineto{\pgfqpoint{2.956936in}{1.945117in}}%
\pgfpathlineto{\pgfqpoint{2.961477in}{1.945117in}}%
\pgfpathlineto{\pgfqpoint{2.961477in}{1.942168in}}%
\pgfpathmoveto{\pgfqpoint{2.961477in}{1.939219in}}%
\pgfpathlineto{\pgfqpoint{2.961477in}{1.939219in}}%
\pgfpathlineto{\pgfqpoint{2.961477in}{1.942168in}}%
\pgfpathlineto{\pgfqpoint{2.966017in}{1.942168in}}%
\pgfpathlineto{\pgfqpoint{2.966017in}{1.939219in}}%
\pgfpathmoveto{\pgfqpoint{2.961477in}{1.942168in}}%
\pgfpathlineto{\pgfqpoint{2.961477in}{1.942168in}}%
\pgfpathlineto{\pgfqpoint{2.961477in}{1.945117in}}%
\pgfpathlineto{\pgfqpoint{2.966017in}{1.945117in}}%
\pgfpathlineto{\pgfqpoint{2.966017in}{1.942168in}}%
\pgfpathmoveto{\pgfqpoint{2.966017in}{1.939219in}}%
\pgfpathlineto{\pgfqpoint{2.966017in}{1.939219in}}%
\pgfpathlineto{\pgfqpoint{2.966017in}{1.942168in}}%
\pgfpathlineto{\pgfqpoint{2.970558in}{1.942168in}}%
\pgfpathlineto{\pgfqpoint{2.970558in}{1.939219in}}%
\pgfpathmoveto{\pgfqpoint{2.966017in}{1.942168in}}%
\pgfpathlineto{\pgfqpoint{2.966017in}{1.942168in}}%
\pgfpathlineto{\pgfqpoint{2.966017in}{1.945117in}}%
\pgfpathlineto{\pgfqpoint{2.970558in}{1.945117in}}%
\pgfpathlineto{\pgfqpoint{2.970558in}{1.942168in}}%
\pgfpathmoveto{\pgfqpoint{2.970558in}{1.942168in}}%
\pgfpathlineto{\pgfqpoint{2.970558in}{1.942168in}}%
\pgfpathlineto{\pgfqpoint{2.970558in}{1.945117in}}%
\pgfpathlineto{\pgfqpoint{2.975099in}{1.945117in}}%
\pgfpathlineto{\pgfqpoint{2.975099in}{1.942168in}}%
\pgfpathmoveto{\pgfqpoint{2.966017in}{1.945117in}}%
\pgfpathlineto{\pgfqpoint{2.966017in}{1.945117in}}%
\pgfpathlineto{\pgfqpoint{2.966017in}{1.948066in}}%
\pgfpathlineto{\pgfqpoint{2.970558in}{1.948066in}}%
\pgfpathlineto{\pgfqpoint{2.970558in}{1.945117in}}%
\pgfpathmoveto{\pgfqpoint{2.966017in}{1.948066in}}%
\pgfpathlineto{\pgfqpoint{2.966017in}{1.948066in}}%
\pgfpathlineto{\pgfqpoint{2.966017in}{1.951015in}}%
\pgfpathlineto{\pgfqpoint{2.970558in}{1.951015in}}%
\pgfpathlineto{\pgfqpoint{2.970558in}{1.948066in}}%
\pgfpathmoveto{\pgfqpoint{2.970558in}{1.945117in}}%
\pgfpathlineto{\pgfqpoint{2.970558in}{1.945117in}}%
\pgfpathlineto{\pgfqpoint{2.970558in}{1.948066in}}%
\pgfpathlineto{\pgfqpoint{2.975099in}{1.948066in}}%
\pgfpathlineto{\pgfqpoint{2.975099in}{1.945117in}}%
\pgfpathmoveto{\pgfqpoint{2.970558in}{1.948066in}}%
\pgfpathlineto{\pgfqpoint{2.970558in}{1.948066in}}%
\pgfpathlineto{\pgfqpoint{2.970558in}{1.951015in}}%
\pgfpathlineto{\pgfqpoint{2.975099in}{1.951015in}}%
\pgfpathlineto{\pgfqpoint{2.975099in}{1.948066in}}%
\pgfpathmoveto{\pgfqpoint{2.975099in}{1.945117in}}%
\pgfpathlineto{\pgfqpoint{2.975099in}{1.945117in}}%
\pgfpathlineto{\pgfqpoint{2.975099in}{1.948066in}}%
\pgfpathlineto{\pgfqpoint{2.979640in}{1.948066in}}%
\pgfpathlineto{\pgfqpoint{2.979640in}{1.945117in}}%
\pgfpathmoveto{\pgfqpoint{2.975099in}{1.948066in}}%
\pgfpathlineto{\pgfqpoint{2.975099in}{1.948066in}}%
\pgfpathlineto{\pgfqpoint{2.975099in}{1.951015in}}%
\pgfpathlineto{\pgfqpoint{2.979640in}{1.951015in}}%
\pgfpathlineto{\pgfqpoint{2.979640in}{1.948066in}}%
\pgfpathmoveto{\pgfqpoint{2.979640in}{1.948066in}}%
\pgfpathlineto{\pgfqpoint{2.979640in}{1.948066in}}%
\pgfpathlineto{\pgfqpoint{2.979640in}{1.951015in}}%
\pgfpathlineto{\pgfqpoint{2.984180in}{1.951015in}}%
\pgfpathlineto{\pgfqpoint{2.984180in}{1.948066in}}%
\pgfpathmoveto{\pgfqpoint{2.975099in}{1.951015in}}%
\pgfpathlineto{\pgfqpoint{2.975099in}{1.951015in}}%
\pgfpathlineto{\pgfqpoint{2.975099in}{1.953964in}}%
\pgfpathlineto{\pgfqpoint{2.979640in}{1.953964in}}%
\pgfpathlineto{\pgfqpoint{2.979640in}{1.951015in}}%
\pgfpathmoveto{\pgfqpoint{2.975099in}{1.953964in}}%
\pgfpathlineto{\pgfqpoint{2.975099in}{1.953964in}}%
\pgfpathlineto{\pgfqpoint{2.975099in}{1.956913in}}%
\pgfpathlineto{\pgfqpoint{2.979640in}{1.956913in}}%
\pgfpathlineto{\pgfqpoint{2.979640in}{1.953964in}}%
\pgfpathmoveto{\pgfqpoint{2.979640in}{1.951015in}}%
\pgfpathlineto{\pgfqpoint{2.979640in}{1.951015in}}%
\pgfpathlineto{\pgfqpoint{2.979640in}{1.953964in}}%
\pgfpathlineto{\pgfqpoint{2.984180in}{1.953964in}}%
\pgfpathlineto{\pgfqpoint{2.984180in}{1.951015in}}%
\pgfpathmoveto{\pgfqpoint{2.979640in}{1.953964in}}%
\pgfpathlineto{\pgfqpoint{2.979640in}{1.953964in}}%
\pgfpathlineto{\pgfqpoint{2.979640in}{1.956913in}}%
\pgfpathlineto{\pgfqpoint{2.984180in}{1.956913in}}%
\pgfpathlineto{\pgfqpoint{2.984180in}{1.953964in}}%
\pgfpathmoveto{\pgfqpoint{2.984180in}{1.951015in}}%
\pgfpathlineto{\pgfqpoint{2.984180in}{1.951015in}}%
\pgfpathlineto{\pgfqpoint{2.984180in}{1.953964in}}%
\pgfpathlineto{\pgfqpoint{2.988721in}{1.953964in}}%
\pgfpathlineto{\pgfqpoint{2.988721in}{1.951015in}}%
\pgfpathmoveto{\pgfqpoint{2.984180in}{1.953964in}}%
\pgfpathlineto{\pgfqpoint{2.984180in}{1.953964in}}%
\pgfpathlineto{\pgfqpoint{2.984180in}{1.956913in}}%
\pgfpathlineto{\pgfqpoint{2.988721in}{1.956913in}}%
\pgfpathlineto{\pgfqpoint{2.988721in}{1.953964in}}%
\pgfpathmoveto{\pgfqpoint{2.988721in}{1.953964in}}%
\pgfpathlineto{\pgfqpoint{2.988721in}{1.953964in}}%
\pgfpathlineto{\pgfqpoint{2.988721in}{1.956913in}}%
\pgfpathlineto{\pgfqpoint{2.993262in}{1.956913in}}%
\pgfpathlineto{\pgfqpoint{2.993262in}{1.953964in}}%
\pgfpathmoveto{\pgfqpoint{2.984180in}{1.956913in}}%
\pgfpathlineto{\pgfqpoint{2.984180in}{1.956913in}}%
\pgfpathlineto{\pgfqpoint{2.984180in}{1.959863in}}%
\pgfpathlineto{\pgfqpoint{2.988721in}{1.959863in}}%
\pgfpathlineto{\pgfqpoint{2.988721in}{1.956913in}}%
\pgfpathmoveto{\pgfqpoint{2.984180in}{1.959863in}}%
\pgfpathlineto{\pgfqpoint{2.984180in}{1.959863in}}%
\pgfpathlineto{\pgfqpoint{2.984180in}{1.962812in}}%
\pgfpathlineto{\pgfqpoint{2.988721in}{1.962812in}}%
\pgfpathlineto{\pgfqpoint{2.988721in}{1.959863in}}%
\pgfpathmoveto{\pgfqpoint{2.988721in}{1.956913in}}%
\pgfpathlineto{\pgfqpoint{2.988721in}{1.956913in}}%
\pgfpathlineto{\pgfqpoint{2.988721in}{1.959863in}}%
\pgfpathlineto{\pgfqpoint{2.993262in}{1.959863in}}%
\pgfpathlineto{\pgfqpoint{2.993262in}{1.956913in}}%
\pgfpathmoveto{\pgfqpoint{2.988721in}{1.959863in}}%
\pgfpathlineto{\pgfqpoint{2.988721in}{1.959863in}}%
\pgfpathlineto{\pgfqpoint{2.988721in}{1.962812in}}%
\pgfpathlineto{\pgfqpoint{2.993262in}{1.962812in}}%
\pgfpathlineto{\pgfqpoint{2.993262in}{1.959863in}}%
\pgfpathmoveto{\pgfqpoint{2.993262in}{1.956913in}}%
\pgfpathlineto{\pgfqpoint{2.993262in}{1.956913in}}%
\pgfpathlineto{\pgfqpoint{2.993262in}{1.959863in}}%
\pgfpathlineto{\pgfqpoint{2.997803in}{1.959863in}}%
\pgfpathlineto{\pgfqpoint{2.997803in}{1.956913in}}%
\pgfpathmoveto{\pgfqpoint{2.993262in}{1.959863in}}%
\pgfpathlineto{\pgfqpoint{2.993262in}{1.959863in}}%
\pgfpathlineto{\pgfqpoint{2.993262in}{1.962812in}}%
\pgfpathlineto{\pgfqpoint{2.997803in}{1.962812in}}%
\pgfpathlineto{\pgfqpoint{2.997803in}{1.959863in}}%
\pgfpathmoveto{\pgfqpoint{2.997803in}{1.959863in}}%
\pgfpathlineto{\pgfqpoint{2.997803in}{1.959863in}}%
\pgfpathlineto{\pgfqpoint{2.997803in}{1.962812in}}%
\pgfpathlineto{\pgfqpoint{3.002344in}{1.962812in}}%
\pgfpathlineto{\pgfqpoint{3.002344in}{1.959863in}}%
\pgfpathmoveto{\pgfqpoint{2.993262in}{1.962812in}}%
\pgfpathlineto{\pgfqpoint{2.993262in}{1.962812in}}%
\pgfpathlineto{\pgfqpoint{2.993262in}{1.965761in}}%
\pgfpathlineto{\pgfqpoint{2.997803in}{1.965761in}}%
\pgfpathlineto{\pgfqpoint{2.997803in}{1.962812in}}%
\pgfpathmoveto{\pgfqpoint{2.993262in}{1.965761in}}%
\pgfpathlineto{\pgfqpoint{2.993262in}{1.965761in}}%
\pgfpathlineto{\pgfqpoint{2.993262in}{1.968710in}}%
\pgfpathlineto{\pgfqpoint{2.997803in}{1.968710in}}%
\pgfpathlineto{\pgfqpoint{2.997803in}{1.965761in}}%
\pgfpathmoveto{\pgfqpoint{2.997803in}{1.962812in}}%
\pgfpathlineto{\pgfqpoint{2.997803in}{1.962812in}}%
\pgfpathlineto{\pgfqpoint{2.997803in}{1.965761in}}%
\pgfpathlineto{\pgfqpoint{3.002344in}{1.965761in}}%
\pgfpathlineto{\pgfqpoint{3.002344in}{1.962812in}}%
\pgfpathmoveto{\pgfqpoint{2.997803in}{1.965761in}}%
\pgfpathlineto{\pgfqpoint{2.997803in}{1.965761in}}%
\pgfpathlineto{\pgfqpoint{2.997803in}{1.968710in}}%
\pgfpathlineto{\pgfqpoint{3.002344in}{1.968710in}}%
\pgfpathlineto{\pgfqpoint{3.002344in}{1.965761in}}%
\pgfpathmoveto{\pgfqpoint{3.002344in}{1.962812in}}%
\pgfpathlineto{\pgfqpoint{3.002344in}{1.962812in}}%
\pgfpathlineto{\pgfqpoint{3.002344in}{1.965761in}}%
\pgfpathlineto{\pgfqpoint{3.006884in}{1.965761in}}%
\pgfpathlineto{\pgfqpoint{3.006884in}{1.962812in}}%
\pgfpathmoveto{\pgfqpoint{3.002344in}{1.965761in}}%
\pgfpathlineto{\pgfqpoint{3.002344in}{1.965761in}}%
\pgfpathlineto{\pgfqpoint{3.002344in}{1.968710in}}%
\pgfpathlineto{\pgfqpoint{3.006884in}{1.968710in}}%
\pgfpathlineto{\pgfqpoint{3.006884in}{1.965761in}}%
\pgfpathmoveto{\pgfqpoint{3.006884in}{1.965761in}}%
\pgfpathlineto{\pgfqpoint{3.006884in}{1.965761in}}%
\pgfpathlineto{\pgfqpoint{3.006884in}{1.968710in}}%
\pgfpathlineto{\pgfqpoint{3.011425in}{1.968710in}}%
\pgfpathlineto{\pgfqpoint{3.011425in}{1.965761in}}%
\pgfpathmoveto{\pgfqpoint{3.002344in}{1.968710in}}%
\pgfpathlineto{\pgfqpoint{3.002344in}{1.968710in}}%
\pgfpathlineto{\pgfqpoint{3.002344in}{1.971659in}}%
\pgfpathlineto{\pgfqpoint{3.006884in}{1.971659in}}%
\pgfpathlineto{\pgfqpoint{3.006884in}{1.968710in}}%
\pgfpathmoveto{\pgfqpoint{3.002344in}{1.971659in}}%
\pgfpathlineto{\pgfqpoint{3.002344in}{1.971659in}}%
\pgfpathlineto{\pgfqpoint{3.002344in}{1.974608in}}%
\pgfpathlineto{\pgfqpoint{3.006884in}{1.974608in}}%
\pgfpathlineto{\pgfqpoint{3.006884in}{1.971659in}}%
\pgfpathmoveto{\pgfqpoint{3.006884in}{1.968710in}}%
\pgfpathlineto{\pgfqpoint{3.006884in}{1.968710in}}%
\pgfpathlineto{\pgfqpoint{3.006884in}{1.971659in}}%
\pgfpathlineto{\pgfqpoint{3.011425in}{1.971659in}}%
\pgfpathlineto{\pgfqpoint{3.011425in}{1.968710in}}%
\pgfpathmoveto{\pgfqpoint{3.006884in}{1.971659in}}%
\pgfpathlineto{\pgfqpoint{3.006884in}{1.971659in}}%
\pgfpathlineto{\pgfqpoint{3.006884in}{1.974608in}}%
\pgfpathlineto{\pgfqpoint{3.011425in}{1.974608in}}%
\pgfpathlineto{\pgfqpoint{3.011425in}{1.971659in}}%
\pgfpathmoveto{\pgfqpoint{3.011425in}{1.968710in}}%
\pgfpathlineto{\pgfqpoint{3.011425in}{1.968710in}}%
\pgfpathlineto{\pgfqpoint{3.011425in}{1.971659in}}%
\pgfpathlineto{\pgfqpoint{3.015966in}{1.971659in}}%
\pgfpathlineto{\pgfqpoint{3.015966in}{1.968710in}}%
\pgfpathmoveto{\pgfqpoint{3.011425in}{1.971659in}}%
\pgfpathlineto{\pgfqpoint{3.011425in}{1.971659in}}%
\pgfpathlineto{\pgfqpoint{3.011425in}{1.974608in}}%
\pgfpathlineto{\pgfqpoint{3.015966in}{1.974608in}}%
\pgfpathlineto{\pgfqpoint{3.015966in}{1.971659in}}%
\pgfpathmoveto{\pgfqpoint{3.015966in}{1.971659in}}%
\pgfpathlineto{\pgfqpoint{3.015966in}{1.971659in}}%
\pgfpathlineto{\pgfqpoint{3.015966in}{1.974608in}}%
\pgfpathlineto{\pgfqpoint{3.020507in}{1.974608in}}%
\pgfpathlineto{\pgfqpoint{3.020507in}{1.971659in}}%
\pgfpathmoveto{\pgfqpoint{3.011425in}{1.974608in}}%
\pgfpathlineto{\pgfqpoint{3.011425in}{1.974608in}}%
\pgfpathlineto{\pgfqpoint{3.011425in}{1.977557in}}%
\pgfpathlineto{\pgfqpoint{3.015966in}{1.977557in}}%
\pgfpathlineto{\pgfqpoint{3.015966in}{1.974608in}}%
\pgfpathmoveto{\pgfqpoint{3.011425in}{1.977557in}}%
\pgfpathlineto{\pgfqpoint{3.011425in}{1.977557in}}%
\pgfpathlineto{\pgfqpoint{3.011425in}{1.980506in}}%
\pgfpathlineto{\pgfqpoint{3.015966in}{1.980506in}}%
\pgfpathlineto{\pgfqpoint{3.015966in}{1.977557in}}%
\pgfpathmoveto{\pgfqpoint{3.015966in}{1.974608in}}%
\pgfpathlineto{\pgfqpoint{3.015966in}{1.974608in}}%
\pgfpathlineto{\pgfqpoint{3.015966in}{1.977557in}}%
\pgfpathlineto{\pgfqpoint{3.020507in}{1.977557in}}%
\pgfpathlineto{\pgfqpoint{3.020507in}{1.974608in}}%
\pgfpathmoveto{\pgfqpoint{3.015966in}{1.977557in}}%
\pgfpathlineto{\pgfqpoint{3.015966in}{1.977557in}}%
\pgfpathlineto{\pgfqpoint{3.015966in}{1.980506in}}%
\pgfpathlineto{\pgfqpoint{3.020507in}{1.980506in}}%
\pgfpathlineto{\pgfqpoint{3.020507in}{1.977557in}}%
\pgfpathmoveto{\pgfqpoint{3.020507in}{1.974608in}}%
\pgfpathlineto{\pgfqpoint{3.020507in}{1.974608in}}%
\pgfpathlineto{\pgfqpoint{3.020507in}{1.977557in}}%
\pgfpathlineto{\pgfqpoint{3.025048in}{1.977557in}}%
\pgfpathlineto{\pgfqpoint{3.025048in}{1.974608in}}%
\pgfpathmoveto{\pgfqpoint{3.020507in}{1.977557in}}%
\pgfpathlineto{\pgfqpoint{3.020507in}{1.977557in}}%
\pgfpathlineto{\pgfqpoint{3.020507in}{1.980506in}}%
\pgfpathlineto{\pgfqpoint{3.025048in}{1.980506in}}%
\pgfpathlineto{\pgfqpoint{3.025048in}{1.977557in}}%
\pgfpathmoveto{\pgfqpoint{3.025048in}{1.977557in}}%
\pgfpathlineto{\pgfqpoint{3.025048in}{1.977557in}}%
\pgfpathlineto{\pgfqpoint{3.025048in}{1.980506in}}%
\pgfpathlineto{\pgfqpoint{3.029588in}{1.980506in}}%
\pgfpathlineto{\pgfqpoint{3.029588in}{1.977557in}}%
\pgfpathmoveto{\pgfqpoint{3.020507in}{1.980506in}}%
\pgfpathlineto{\pgfqpoint{3.020507in}{1.980506in}}%
\pgfpathlineto{\pgfqpoint{3.020507in}{1.983455in}}%
\pgfpathlineto{\pgfqpoint{3.025048in}{1.983455in}}%
\pgfpathlineto{\pgfqpoint{3.025048in}{1.980506in}}%
\pgfpathmoveto{\pgfqpoint{3.020507in}{1.983455in}}%
\pgfpathlineto{\pgfqpoint{3.020507in}{1.983455in}}%
\pgfpathlineto{\pgfqpoint{3.020507in}{1.986405in}}%
\pgfpathlineto{\pgfqpoint{3.025048in}{1.986405in}}%
\pgfpathlineto{\pgfqpoint{3.025048in}{1.983455in}}%
\pgfpathmoveto{\pgfqpoint{3.025048in}{1.980506in}}%
\pgfpathlineto{\pgfqpoint{3.025048in}{1.980506in}}%
\pgfpathlineto{\pgfqpoint{3.025048in}{1.983455in}}%
\pgfpathlineto{\pgfqpoint{3.029588in}{1.983455in}}%
\pgfpathlineto{\pgfqpoint{3.029588in}{1.980506in}}%
\pgfpathmoveto{\pgfqpoint{3.025048in}{1.983455in}}%
\pgfpathlineto{\pgfqpoint{3.025048in}{1.983455in}}%
\pgfpathlineto{\pgfqpoint{3.025048in}{1.986405in}}%
\pgfpathlineto{\pgfqpoint{3.029588in}{1.986405in}}%
\pgfpathlineto{\pgfqpoint{3.029588in}{1.983455in}}%
\pgfpathmoveto{\pgfqpoint{3.029588in}{1.980506in}}%
\pgfpathlineto{\pgfqpoint{3.029588in}{1.980506in}}%
\pgfpathlineto{\pgfqpoint{3.029588in}{1.983455in}}%
\pgfpathlineto{\pgfqpoint{3.034129in}{1.983455in}}%
\pgfpathlineto{\pgfqpoint{3.034129in}{1.980506in}}%
\pgfpathmoveto{\pgfqpoint{3.029588in}{1.983455in}}%
\pgfpathlineto{\pgfqpoint{3.029588in}{1.983455in}}%
\pgfpathlineto{\pgfqpoint{3.029588in}{1.986405in}}%
\pgfpathlineto{\pgfqpoint{3.034129in}{1.986405in}}%
\pgfpathlineto{\pgfqpoint{3.034129in}{1.983455in}}%
\pgfpathmoveto{\pgfqpoint{3.034129in}{1.983455in}}%
\pgfpathlineto{\pgfqpoint{3.034129in}{1.983455in}}%
\pgfpathlineto{\pgfqpoint{3.034129in}{1.986405in}}%
\pgfpathlineto{\pgfqpoint{3.038670in}{1.986405in}}%
\pgfpathlineto{\pgfqpoint{3.038670in}{1.983455in}}%
\pgfpathmoveto{\pgfqpoint{3.029588in}{1.986405in}}%
\pgfpathlineto{\pgfqpoint{3.029588in}{1.986405in}}%
\pgfpathlineto{\pgfqpoint{3.029588in}{1.989354in}}%
\pgfpathlineto{\pgfqpoint{3.034129in}{1.989354in}}%
\pgfpathlineto{\pgfqpoint{3.034129in}{1.986405in}}%
\pgfpathmoveto{\pgfqpoint{3.029588in}{1.989354in}}%
\pgfpathlineto{\pgfqpoint{3.029588in}{1.989354in}}%
\pgfpathlineto{\pgfqpoint{3.029588in}{1.992303in}}%
\pgfpathlineto{\pgfqpoint{3.034129in}{1.992303in}}%
\pgfpathlineto{\pgfqpoint{3.034129in}{1.989354in}}%
\pgfpathmoveto{\pgfqpoint{3.034129in}{1.986405in}}%
\pgfpathlineto{\pgfqpoint{3.034129in}{1.986405in}}%
\pgfpathlineto{\pgfqpoint{3.034129in}{1.989354in}}%
\pgfpathlineto{\pgfqpoint{3.038670in}{1.989354in}}%
\pgfpathlineto{\pgfqpoint{3.038670in}{1.986405in}}%
\pgfpathmoveto{\pgfqpoint{3.034129in}{1.989354in}}%
\pgfpathlineto{\pgfqpoint{3.034129in}{1.989354in}}%
\pgfpathlineto{\pgfqpoint{3.034129in}{1.992303in}}%
\pgfpathlineto{\pgfqpoint{3.038670in}{1.992303in}}%
\pgfpathlineto{\pgfqpoint{3.038670in}{1.989354in}}%
\pgfpathmoveto{\pgfqpoint{3.038670in}{1.986405in}}%
\pgfpathlineto{\pgfqpoint{3.038670in}{1.986405in}}%
\pgfpathlineto{\pgfqpoint{3.038670in}{1.989354in}}%
\pgfpathlineto{\pgfqpoint{3.043211in}{1.989354in}}%
\pgfpathlineto{\pgfqpoint{3.043211in}{1.986405in}}%
\pgfpathmoveto{\pgfqpoint{3.038670in}{1.989354in}}%
\pgfpathlineto{\pgfqpoint{3.038670in}{1.989354in}}%
\pgfpathlineto{\pgfqpoint{3.038670in}{1.992303in}}%
\pgfpathlineto{\pgfqpoint{3.043211in}{1.992303in}}%
\pgfpathlineto{\pgfqpoint{3.043211in}{1.989354in}}%
\pgfpathmoveto{\pgfqpoint{3.043211in}{1.989354in}}%
\pgfpathlineto{\pgfqpoint{3.043211in}{1.989354in}}%
\pgfpathlineto{\pgfqpoint{3.043211in}{1.992303in}}%
\pgfpathlineto{\pgfqpoint{3.047752in}{1.992303in}}%
\pgfpathlineto{\pgfqpoint{3.047752in}{1.989354in}}%
\pgfpathmoveto{\pgfqpoint{3.038670in}{1.992303in}}%
\pgfpathlineto{\pgfqpoint{3.038670in}{1.992303in}}%
\pgfpathlineto{\pgfqpoint{3.038670in}{1.995252in}}%
\pgfpathlineto{\pgfqpoint{3.043211in}{1.995252in}}%
\pgfpathlineto{\pgfqpoint{3.043211in}{1.992303in}}%
\pgfpathmoveto{\pgfqpoint{3.038670in}{1.995252in}}%
\pgfpathlineto{\pgfqpoint{3.038670in}{1.995252in}}%
\pgfpathlineto{\pgfqpoint{3.038670in}{1.998201in}}%
\pgfpathlineto{\pgfqpoint{3.043211in}{1.998201in}}%
\pgfpathlineto{\pgfqpoint{3.043211in}{1.995252in}}%
\pgfpathmoveto{\pgfqpoint{3.043211in}{1.992303in}}%
\pgfpathlineto{\pgfqpoint{3.043211in}{1.992303in}}%
\pgfpathlineto{\pgfqpoint{3.043211in}{1.995252in}}%
\pgfpathlineto{\pgfqpoint{3.047752in}{1.995252in}}%
\pgfpathlineto{\pgfqpoint{3.047752in}{1.992303in}}%
\pgfpathmoveto{\pgfqpoint{3.043211in}{1.995252in}}%
\pgfpathlineto{\pgfqpoint{3.043211in}{1.995252in}}%
\pgfpathlineto{\pgfqpoint{3.043211in}{1.998201in}}%
\pgfpathlineto{\pgfqpoint{3.047752in}{1.998201in}}%
\pgfpathlineto{\pgfqpoint{3.047752in}{1.995252in}}%
\pgfpathmoveto{\pgfqpoint{3.047752in}{1.992303in}}%
\pgfpathlineto{\pgfqpoint{3.047752in}{1.992303in}}%
\pgfpathlineto{\pgfqpoint{3.047752in}{1.995252in}}%
\pgfpathlineto{\pgfqpoint{3.052292in}{1.995252in}}%
\pgfpathlineto{\pgfqpoint{3.052292in}{1.992303in}}%
\pgfpathmoveto{\pgfqpoint{3.047752in}{1.995252in}}%
\pgfpathlineto{\pgfqpoint{3.047752in}{1.995252in}}%
\pgfpathlineto{\pgfqpoint{3.047752in}{1.998201in}}%
\pgfpathlineto{\pgfqpoint{3.052292in}{1.998201in}}%
\pgfpathlineto{\pgfqpoint{3.052292in}{1.995252in}}%
\pgfpathmoveto{\pgfqpoint{3.052292in}{1.995252in}}%
\pgfpathlineto{\pgfqpoint{3.052292in}{1.995252in}}%
\pgfpathlineto{\pgfqpoint{3.052292in}{1.998201in}}%
\pgfpathlineto{\pgfqpoint{3.056833in}{1.998201in}}%
\pgfpathlineto{\pgfqpoint{3.056833in}{1.995252in}}%
\pgfpathmoveto{\pgfqpoint{3.047752in}{1.998201in}}%
\pgfpathlineto{\pgfqpoint{3.047752in}{1.998201in}}%
\pgfpathlineto{\pgfqpoint{3.047752in}{2.001150in}}%
\pgfpathlineto{\pgfqpoint{3.052292in}{2.001150in}}%
\pgfpathlineto{\pgfqpoint{3.052292in}{1.998201in}}%
\pgfpathmoveto{\pgfqpoint{3.047752in}{2.001150in}}%
\pgfpathlineto{\pgfqpoint{3.047752in}{2.001150in}}%
\pgfpathlineto{\pgfqpoint{3.047752in}{2.004099in}}%
\pgfpathlineto{\pgfqpoint{3.052292in}{2.004099in}}%
\pgfpathlineto{\pgfqpoint{3.052292in}{2.001150in}}%
\pgfpathmoveto{\pgfqpoint{3.052292in}{1.998201in}}%
\pgfpathlineto{\pgfqpoint{3.052292in}{1.998201in}}%
\pgfpathlineto{\pgfqpoint{3.052292in}{2.001150in}}%
\pgfpathlineto{\pgfqpoint{3.056833in}{2.001150in}}%
\pgfpathlineto{\pgfqpoint{3.056833in}{1.998201in}}%
\pgfpathmoveto{\pgfqpoint{3.052292in}{2.001150in}}%
\pgfpathlineto{\pgfqpoint{3.052292in}{2.001150in}}%
\pgfpathlineto{\pgfqpoint{3.052292in}{2.004099in}}%
\pgfpathlineto{\pgfqpoint{3.056833in}{2.004099in}}%
\pgfpathlineto{\pgfqpoint{3.056833in}{2.001150in}}%
\pgfpathmoveto{\pgfqpoint{3.056833in}{1.998201in}}%
\pgfpathlineto{\pgfqpoint{3.056833in}{1.998201in}}%
\pgfpathlineto{\pgfqpoint{3.056833in}{2.001150in}}%
\pgfpathlineto{\pgfqpoint{3.061374in}{2.001150in}}%
\pgfpathlineto{\pgfqpoint{3.061374in}{1.998201in}}%
\pgfpathmoveto{\pgfqpoint{3.056833in}{2.001150in}}%
\pgfpathlineto{\pgfqpoint{3.056833in}{2.001150in}}%
\pgfpathlineto{\pgfqpoint{3.056833in}{2.004099in}}%
\pgfpathlineto{\pgfqpoint{3.061374in}{2.004099in}}%
\pgfpathlineto{\pgfqpoint{3.061374in}{2.001150in}}%
\pgfpathmoveto{\pgfqpoint{3.061374in}{2.001150in}}%
\pgfpathlineto{\pgfqpoint{3.061374in}{2.001150in}}%
\pgfpathlineto{\pgfqpoint{3.061374in}{2.004099in}}%
\pgfpathlineto{\pgfqpoint{3.065915in}{2.004099in}}%
\pgfpathlineto{\pgfqpoint{3.065915in}{2.001150in}}%
\pgfpathmoveto{\pgfqpoint{3.056833in}{2.004099in}}%
\pgfpathlineto{\pgfqpoint{3.056833in}{2.004099in}}%
\pgfpathlineto{\pgfqpoint{3.056833in}{2.007048in}}%
\pgfpathlineto{\pgfqpoint{3.061374in}{2.007048in}}%
\pgfpathlineto{\pgfqpoint{3.061374in}{2.004099in}}%
\pgfpathmoveto{\pgfqpoint{3.056833in}{2.007048in}}%
\pgfpathlineto{\pgfqpoint{3.056833in}{2.007048in}}%
\pgfpathlineto{\pgfqpoint{3.056833in}{2.009997in}}%
\pgfpathlineto{\pgfqpoint{3.061374in}{2.009997in}}%
\pgfpathlineto{\pgfqpoint{3.061374in}{2.007048in}}%
\pgfpathmoveto{\pgfqpoint{3.061374in}{2.004099in}}%
\pgfpathlineto{\pgfqpoint{3.061374in}{2.004099in}}%
\pgfpathlineto{\pgfqpoint{3.061374in}{2.007048in}}%
\pgfpathlineto{\pgfqpoint{3.065915in}{2.007048in}}%
\pgfpathlineto{\pgfqpoint{3.065915in}{2.004099in}}%
\pgfpathmoveto{\pgfqpoint{3.061374in}{2.007048in}}%
\pgfpathlineto{\pgfqpoint{3.061374in}{2.007048in}}%
\pgfpathlineto{\pgfqpoint{3.061374in}{2.009997in}}%
\pgfpathlineto{\pgfqpoint{3.065915in}{2.009997in}}%
\pgfpathlineto{\pgfqpoint{3.065915in}{2.007048in}}%
\pgfpathmoveto{\pgfqpoint{3.065915in}{2.004099in}}%
\pgfpathlineto{\pgfqpoint{3.065915in}{2.004099in}}%
\pgfpathlineto{\pgfqpoint{3.065915in}{2.007048in}}%
\pgfpathlineto{\pgfqpoint{3.070455in}{2.007048in}}%
\pgfpathlineto{\pgfqpoint{3.070455in}{2.004099in}}%
\pgfpathmoveto{\pgfqpoint{3.065915in}{2.007048in}}%
\pgfpathlineto{\pgfqpoint{3.065915in}{2.007048in}}%
\pgfpathlineto{\pgfqpoint{3.065915in}{2.009997in}}%
\pgfpathlineto{\pgfqpoint{3.070455in}{2.009997in}}%
\pgfpathlineto{\pgfqpoint{3.070455in}{2.007048in}}%
\pgfpathmoveto{\pgfqpoint{3.070455in}{2.007048in}}%
\pgfpathlineto{\pgfqpoint{3.070455in}{2.007048in}}%
\pgfpathlineto{\pgfqpoint{3.070455in}{2.009997in}}%
\pgfpathlineto{\pgfqpoint{3.074996in}{2.009997in}}%
\pgfpathlineto{\pgfqpoint{3.074996in}{2.007048in}}%
\pgfpathmoveto{\pgfqpoint{3.065915in}{2.009997in}}%
\pgfpathlineto{\pgfqpoint{3.065915in}{2.009997in}}%
\pgfpathlineto{\pgfqpoint{3.065915in}{2.012947in}}%
\pgfpathlineto{\pgfqpoint{3.070455in}{2.012947in}}%
\pgfpathlineto{\pgfqpoint{3.070455in}{2.009997in}}%
\pgfpathmoveto{\pgfqpoint{3.065915in}{2.012947in}}%
\pgfpathlineto{\pgfqpoint{3.065915in}{2.012947in}}%
\pgfpathlineto{\pgfqpoint{3.065915in}{2.015896in}}%
\pgfpathlineto{\pgfqpoint{3.070455in}{2.015896in}}%
\pgfpathlineto{\pgfqpoint{3.070455in}{2.012947in}}%
\pgfpathmoveto{\pgfqpoint{3.070455in}{2.009997in}}%
\pgfpathlineto{\pgfqpoint{3.070455in}{2.009997in}}%
\pgfpathlineto{\pgfqpoint{3.070455in}{2.012947in}}%
\pgfpathlineto{\pgfqpoint{3.074996in}{2.012947in}}%
\pgfpathlineto{\pgfqpoint{3.074996in}{2.009997in}}%
\pgfpathmoveto{\pgfqpoint{3.070455in}{2.012947in}}%
\pgfpathlineto{\pgfqpoint{3.070455in}{2.012947in}}%
\pgfpathlineto{\pgfqpoint{3.070455in}{2.015896in}}%
\pgfpathlineto{\pgfqpoint{3.074996in}{2.015896in}}%
\pgfpathlineto{\pgfqpoint{3.074996in}{2.012947in}}%
\pgfpathmoveto{\pgfqpoint{3.065915in}{3.213280in}}%
\pgfpathlineto{\pgfqpoint{3.065915in}{3.213280in}}%
\pgfpathlineto{\pgfqpoint{3.065915in}{3.216229in}}%
\pgfpathlineto{\pgfqpoint{3.070455in}{3.216229in}}%
\pgfpathlineto{\pgfqpoint{3.070455in}{3.213280in}}%
\pgfpathmoveto{\pgfqpoint{3.065915in}{3.216229in}}%
\pgfpathlineto{\pgfqpoint{3.065915in}{3.216229in}}%
\pgfpathlineto{\pgfqpoint{3.065915in}{3.219178in}}%
\pgfpathlineto{\pgfqpoint{3.070455in}{3.219178in}}%
\pgfpathlineto{\pgfqpoint{3.070455in}{3.216229in}}%
\pgfpathmoveto{\pgfqpoint{3.070455in}{3.213280in}}%
\pgfpathlineto{\pgfqpoint{3.070455in}{3.213280in}}%
\pgfpathlineto{\pgfqpoint{3.070455in}{3.216229in}}%
\pgfpathlineto{\pgfqpoint{3.074996in}{3.216229in}}%
\pgfpathlineto{\pgfqpoint{3.074996in}{3.213280in}}%
\pgfpathmoveto{\pgfqpoint{3.070455in}{3.216229in}}%
\pgfpathlineto{\pgfqpoint{3.070455in}{3.216229in}}%
\pgfpathlineto{\pgfqpoint{3.070455in}{3.219178in}}%
\pgfpathlineto{\pgfqpoint{3.074996in}{3.219178in}}%
\pgfpathlineto{\pgfqpoint{3.074996in}{3.216229in}}%
\pgfpathmoveto{\pgfqpoint{3.065915in}{3.219178in}}%
\pgfpathlineto{\pgfqpoint{3.065915in}{3.219178in}}%
\pgfpathlineto{\pgfqpoint{3.065915in}{3.222127in}}%
\pgfpathlineto{\pgfqpoint{3.070455in}{3.222127in}}%
\pgfpathlineto{\pgfqpoint{3.070455in}{3.219178in}}%
\pgfpathmoveto{\pgfqpoint{3.065915in}{3.222127in}}%
\pgfpathlineto{\pgfqpoint{3.065915in}{3.222127in}}%
\pgfpathlineto{\pgfqpoint{3.065915in}{3.225077in}}%
\pgfpathlineto{\pgfqpoint{3.070455in}{3.225077in}}%
\pgfpathlineto{\pgfqpoint{3.070455in}{3.222127in}}%
\pgfpathmoveto{\pgfqpoint{3.070455in}{3.219178in}}%
\pgfpathlineto{\pgfqpoint{3.070455in}{3.219178in}}%
\pgfpathlineto{\pgfqpoint{3.070455in}{3.222127in}}%
\pgfpathlineto{\pgfqpoint{3.074996in}{3.222127in}}%
\pgfpathlineto{\pgfqpoint{3.074996in}{3.219178in}}%
\pgfpathmoveto{\pgfqpoint{3.070455in}{3.222127in}}%
\pgfpathlineto{\pgfqpoint{3.070455in}{3.222127in}}%
\pgfpathlineto{\pgfqpoint{3.070455in}{3.225077in}}%
\pgfpathlineto{\pgfqpoint{3.074996in}{3.225077in}}%
\pgfpathlineto{\pgfqpoint{3.074996in}{3.222127in}}%
\pgfpathmoveto{\pgfqpoint{3.056833in}{3.225077in}}%
\pgfpathlineto{\pgfqpoint{3.056833in}{3.225077in}}%
\pgfpathlineto{\pgfqpoint{3.056833in}{3.228026in}}%
\pgfpathlineto{\pgfqpoint{3.061374in}{3.228026in}}%
\pgfpathlineto{\pgfqpoint{3.061374in}{3.225077in}}%
\pgfpathmoveto{\pgfqpoint{3.056833in}{3.228026in}}%
\pgfpathlineto{\pgfqpoint{3.056833in}{3.228026in}}%
\pgfpathlineto{\pgfqpoint{3.056833in}{3.230975in}}%
\pgfpathlineto{\pgfqpoint{3.061374in}{3.230975in}}%
\pgfpathlineto{\pgfqpoint{3.061374in}{3.228026in}}%
\pgfpathmoveto{\pgfqpoint{3.061374in}{3.225077in}}%
\pgfpathlineto{\pgfqpoint{3.061374in}{3.225077in}}%
\pgfpathlineto{\pgfqpoint{3.061374in}{3.228026in}}%
\pgfpathlineto{\pgfqpoint{3.065915in}{3.228026in}}%
\pgfpathlineto{\pgfqpoint{3.065915in}{3.225077in}}%
\pgfpathmoveto{\pgfqpoint{3.061374in}{3.228026in}}%
\pgfpathlineto{\pgfqpoint{3.061374in}{3.228026in}}%
\pgfpathlineto{\pgfqpoint{3.061374in}{3.230975in}}%
\pgfpathlineto{\pgfqpoint{3.065915in}{3.230975in}}%
\pgfpathlineto{\pgfqpoint{3.065915in}{3.228026in}}%
\pgfpathmoveto{\pgfqpoint{3.056833in}{3.230975in}}%
\pgfpathlineto{\pgfqpoint{3.056833in}{3.230975in}}%
\pgfpathlineto{\pgfqpoint{3.056833in}{3.233924in}}%
\pgfpathlineto{\pgfqpoint{3.061374in}{3.233924in}}%
\pgfpathlineto{\pgfqpoint{3.061374in}{3.230975in}}%
\pgfpathmoveto{\pgfqpoint{3.056833in}{3.233924in}}%
\pgfpathlineto{\pgfqpoint{3.056833in}{3.233924in}}%
\pgfpathlineto{\pgfqpoint{3.056833in}{3.236873in}}%
\pgfpathlineto{\pgfqpoint{3.061374in}{3.236873in}}%
\pgfpathlineto{\pgfqpoint{3.061374in}{3.233924in}}%
\pgfpathmoveto{\pgfqpoint{3.061374in}{3.230975in}}%
\pgfpathlineto{\pgfqpoint{3.061374in}{3.230975in}}%
\pgfpathlineto{\pgfqpoint{3.061374in}{3.233924in}}%
\pgfpathlineto{\pgfqpoint{3.065915in}{3.233924in}}%
\pgfpathlineto{\pgfqpoint{3.065915in}{3.230975in}}%
\pgfpathmoveto{\pgfqpoint{3.061374in}{3.233924in}}%
\pgfpathlineto{\pgfqpoint{3.061374in}{3.233924in}}%
\pgfpathlineto{\pgfqpoint{3.061374in}{3.236873in}}%
\pgfpathlineto{\pgfqpoint{3.065915in}{3.236873in}}%
\pgfpathlineto{\pgfqpoint{3.065915in}{3.233924in}}%
\pgfpathmoveto{\pgfqpoint{3.065915in}{3.225077in}}%
\pgfpathlineto{\pgfqpoint{3.065915in}{3.225077in}}%
\pgfpathlineto{\pgfqpoint{3.065915in}{3.228026in}}%
\pgfpathlineto{\pgfqpoint{3.070455in}{3.228026in}}%
\pgfpathlineto{\pgfqpoint{3.070455in}{3.225077in}}%
\pgfpathmoveto{\pgfqpoint{3.065915in}{3.228026in}}%
\pgfpathlineto{\pgfqpoint{3.065915in}{3.228026in}}%
\pgfpathlineto{\pgfqpoint{3.065915in}{3.230975in}}%
\pgfpathlineto{\pgfqpoint{3.070455in}{3.230975in}}%
\pgfpathlineto{\pgfqpoint{3.070455in}{3.228026in}}%
\pgfpathmoveto{\pgfqpoint{2.993262in}{3.307655in}}%
\pgfpathlineto{\pgfqpoint{2.993262in}{3.307655in}}%
\pgfpathlineto{\pgfqpoint{2.993262in}{3.310604in}}%
\pgfpathlineto{\pgfqpoint{2.997803in}{3.310604in}}%
\pgfpathlineto{\pgfqpoint{2.997803in}{3.307655in}}%
\pgfpathmoveto{\pgfqpoint{2.993262in}{3.310604in}}%
\pgfpathlineto{\pgfqpoint{2.993262in}{3.310604in}}%
\pgfpathlineto{\pgfqpoint{2.993262in}{3.313554in}}%
\pgfpathlineto{\pgfqpoint{2.997803in}{3.313554in}}%
\pgfpathlineto{\pgfqpoint{2.997803in}{3.310604in}}%
\pgfpathmoveto{\pgfqpoint{2.997803in}{3.307655in}}%
\pgfpathlineto{\pgfqpoint{2.997803in}{3.307655in}}%
\pgfpathlineto{\pgfqpoint{2.997803in}{3.310604in}}%
\pgfpathlineto{\pgfqpoint{3.002344in}{3.310604in}}%
\pgfpathlineto{\pgfqpoint{3.002344in}{3.307655in}}%
\pgfpathmoveto{\pgfqpoint{2.997803in}{3.310604in}}%
\pgfpathlineto{\pgfqpoint{2.997803in}{3.310604in}}%
\pgfpathlineto{\pgfqpoint{2.997803in}{3.313554in}}%
\pgfpathlineto{\pgfqpoint{3.002344in}{3.313554in}}%
\pgfpathlineto{\pgfqpoint{3.002344in}{3.310604in}}%
\pgfpathmoveto{\pgfqpoint{2.993262in}{3.313554in}}%
\pgfpathlineto{\pgfqpoint{2.993262in}{3.313554in}}%
\pgfpathlineto{\pgfqpoint{2.993262in}{3.316503in}}%
\pgfpathlineto{\pgfqpoint{2.997803in}{3.316503in}}%
\pgfpathlineto{\pgfqpoint{2.997803in}{3.313554in}}%
\pgfpathmoveto{\pgfqpoint{2.993262in}{3.316503in}}%
\pgfpathlineto{\pgfqpoint{2.993262in}{3.316503in}}%
\pgfpathlineto{\pgfqpoint{2.993262in}{3.319452in}}%
\pgfpathlineto{\pgfqpoint{2.997803in}{3.319452in}}%
\pgfpathlineto{\pgfqpoint{2.997803in}{3.316503in}}%
\pgfpathmoveto{\pgfqpoint{2.997803in}{3.313554in}}%
\pgfpathlineto{\pgfqpoint{2.997803in}{3.313554in}}%
\pgfpathlineto{\pgfqpoint{2.997803in}{3.316503in}}%
\pgfpathlineto{\pgfqpoint{3.002344in}{3.316503in}}%
\pgfpathlineto{\pgfqpoint{3.002344in}{3.313554in}}%
\pgfpathmoveto{\pgfqpoint{2.997803in}{3.316503in}}%
\pgfpathlineto{\pgfqpoint{2.997803in}{3.316503in}}%
\pgfpathlineto{\pgfqpoint{2.997803in}{3.319452in}}%
\pgfpathlineto{\pgfqpoint{3.002344in}{3.319452in}}%
\pgfpathlineto{\pgfqpoint{3.002344in}{3.316503in}}%
\pgfpathmoveto{\pgfqpoint{2.984180in}{3.319452in}}%
\pgfpathlineto{\pgfqpoint{2.984180in}{3.319452in}}%
\pgfpathlineto{\pgfqpoint{2.984180in}{3.322401in}}%
\pgfpathlineto{\pgfqpoint{2.988721in}{3.322401in}}%
\pgfpathlineto{\pgfqpoint{2.988721in}{3.319452in}}%
\pgfpathmoveto{\pgfqpoint{2.984180in}{3.322401in}}%
\pgfpathlineto{\pgfqpoint{2.984180in}{3.322401in}}%
\pgfpathlineto{\pgfqpoint{2.984180in}{3.325351in}}%
\pgfpathlineto{\pgfqpoint{2.988721in}{3.325351in}}%
\pgfpathlineto{\pgfqpoint{2.988721in}{3.322401in}}%
\pgfpathmoveto{\pgfqpoint{2.988721in}{3.319452in}}%
\pgfpathlineto{\pgfqpoint{2.988721in}{3.319452in}}%
\pgfpathlineto{\pgfqpoint{2.988721in}{3.322401in}}%
\pgfpathlineto{\pgfqpoint{2.993262in}{3.322401in}}%
\pgfpathlineto{\pgfqpoint{2.993262in}{3.319452in}}%
\pgfpathmoveto{\pgfqpoint{2.988721in}{3.322401in}}%
\pgfpathlineto{\pgfqpoint{2.988721in}{3.322401in}}%
\pgfpathlineto{\pgfqpoint{2.988721in}{3.325351in}}%
\pgfpathlineto{\pgfqpoint{2.993262in}{3.325351in}}%
\pgfpathlineto{\pgfqpoint{2.993262in}{3.322401in}}%
\pgfpathmoveto{\pgfqpoint{2.984180in}{3.325351in}}%
\pgfpathlineto{\pgfqpoint{2.984180in}{3.325351in}}%
\pgfpathlineto{\pgfqpoint{2.984180in}{3.328300in}}%
\pgfpathlineto{\pgfqpoint{2.988721in}{3.328300in}}%
\pgfpathlineto{\pgfqpoint{2.988721in}{3.325351in}}%
\pgfpathmoveto{\pgfqpoint{2.984180in}{3.328300in}}%
\pgfpathlineto{\pgfqpoint{2.984180in}{3.328300in}}%
\pgfpathlineto{\pgfqpoint{2.984180in}{3.331249in}}%
\pgfpathlineto{\pgfqpoint{2.988721in}{3.331249in}}%
\pgfpathlineto{\pgfqpoint{2.988721in}{3.328300in}}%
\pgfpathmoveto{\pgfqpoint{2.988721in}{3.325351in}}%
\pgfpathlineto{\pgfqpoint{2.988721in}{3.325351in}}%
\pgfpathlineto{\pgfqpoint{2.988721in}{3.328300in}}%
\pgfpathlineto{\pgfqpoint{2.993262in}{3.328300in}}%
\pgfpathlineto{\pgfqpoint{2.993262in}{3.325351in}}%
\pgfpathmoveto{\pgfqpoint{2.988721in}{3.328300in}}%
\pgfpathlineto{\pgfqpoint{2.988721in}{3.328300in}}%
\pgfpathlineto{\pgfqpoint{2.988721in}{3.331249in}}%
\pgfpathlineto{\pgfqpoint{2.993262in}{3.331249in}}%
\pgfpathlineto{\pgfqpoint{2.993262in}{3.328300in}}%
\pgfpathmoveto{\pgfqpoint{2.993262in}{3.319452in}}%
\pgfpathlineto{\pgfqpoint{2.993262in}{3.319452in}}%
\pgfpathlineto{\pgfqpoint{2.993262in}{3.322401in}}%
\pgfpathlineto{\pgfqpoint{2.997803in}{3.322401in}}%
\pgfpathlineto{\pgfqpoint{2.997803in}{3.319452in}}%
\pgfpathmoveto{\pgfqpoint{2.993262in}{3.322401in}}%
\pgfpathlineto{\pgfqpoint{2.993262in}{3.322401in}}%
\pgfpathlineto{\pgfqpoint{2.993262in}{3.325351in}}%
\pgfpathlineto{\pgfqpoint{2.997803in}{3.325351in}}%
\pgfpathlineto{\pgfqpoint{2.997803in}{3.322401in}}%
\pgfpathmoveto{\pgfqpoint{3.029588in}{3.260467in}}%
\pgfpathlineto{\pgfqpoint{3.029588in}{3.260467in}}%
\pgfpathlineto{\pgfqpoint{3.029588in}{3.263416in}}%
\pgfpathlineto{\pgfqpoint{3.034129in}{3.263416in}}%
\pgfpathlineto{\pgfqpoint{3.034129in}{3.260467in}}%
\pgfpathmoveto{\pgfqpoint{3.029588in}{3.263416in}}%
\pgfpathlineto{\pgfqpoint{3.029588in}{3.263416in}}%
\pgfpathlineto{\pgfqpoint{3.029588in}{3.266365in}}%
\pgfpathlineto{\pgfqpoint{3.034129in}{3.266365in}}%
\pgfpathlineto{\pgfqpoint{3.034129in}{3.263416in}}%
\pgfpathmoveto{\pgfqpoint{3.034129in}{3.260467in}}%
\pgfpathlineto{\pgfqpoint{3.034129in}{3.260467in}}%
\pgfpathlineto{\pgfqpoint{3.034129in}{3.263416in}}%
\pgfpathlineto{\pgfqpoint{3.038670in}{3.263416in}}%
\pgfpathlineto{\pgfqpoint{3.038670in}{3.260467in}}%
\pgfpathmoveto{\pgfqpoint{3.034129in}{3.263416in}}%
\pgfpathlineto{\pgfqpoint{3.034129in}{3.263416in}}%
\pgfpathlineto{\pgfqpoint{3.034129in}{3.266365in}}%
\pgfpathlineto{\pgfqpoint{3.038670in}{3.266365in}}%
\pgfpathlineto{\pgfqpoint{3.038670in}{3.263416in}}%
\pgfpathmoveto{\pgfqpoint{3.029588in}{3.266365in}}%
\pgfpathlineto{\pgfqpoint{3.029588in}{3.266365in}}%
\pgfpathlineto{\pgfqpoint{3.029588in}{3.269315in}}%
\pgfpathlineto{\pgfqpoint{3.034129in}{3.269315in}}%
\pgfpathlineto{\pgfqpoint{3.034129in}{3.266365in}}%
\pgfpathmoveto{\pgfqpoint{3.029588in}{3.269315in}}%
\pgfpathlineto{\pgfqpoint{3.029588in}{3.269315in}}%
\pgfpathlineto{\pgfqpoint{3.029588in}{3.272264in}}%
\pgfpathlineto{\pgfqpoint{3.034129in}{3.272264in}}%
\pgfpathlineto{\pgfqpoint{3.034129in}{3.269315in}}%
\pgfpathmoveto{\pgfqpoint{3.034129in}{3.266365in}}%
\pgfpathlineto{\pgfqpoint{3.034129in}{3.266365in}}%
\pgfpathlineto{\pgfqpoint{3.034129in}{3.269315in}}%
\pgfpathlineto{\pgfqpoint{3.038670in}{3.269315in}}%
\pgfpathlineto{\pgfqpoint{3.038670in}{3.266365in}}%
\pgfpathmoveto{\pgfqpoint{3.034129in}{3.269315in}}%
\pgfpathlineto{\pgfqpoint{3.034129in}{3.269315in}}%
\pgfpathlineto{\pgfqpoint{3.034129in}{3.272264in}}%
\pgfpathlineto{\pgfqpoint{3.038670in}{3.272264in}}%
\pgfpathlineto{\pgfqpoint{3.038670in}{3.269315in}}%
\pgfpathmoveto{\pgfqpoint{3.020507in}{3.272264in}}%
\pgfpathlineto{\pgfqpoint{3.020507in}{3.272264in}}%
\pgfpathlineto{\pgfqpoint{3.020507in}{3.275213in}}%
\pgfpathlineto{\pgfqpoint{3.025048in}{3.275213in}}%
\pgfpathlineto{\pgfqpoint{3.025048in}{3.272264in}}%
\pgfpathmoveto{\pgfqpoint{3.020507in}{3.275213in}}%
\pgfpathlineto{\pgfqpoint{3.020507in}{3.275213in}}%
\pgfpathlineto{\pgfqpoint{3.020507in}{3.278163in}}%
\pgfpathlineto{\pgfqpoint{3.025048in}{3.278163in}}%
\pgfpathlineto{\pgfqpoint{3.025048in}{3.275213in}}%
\pgfpathmoveto{\pgfqpoint{3.025048in}{3.272264in}}%
\pgfpathlineto{\pgfqpoint{3.025048in}{3.272264in}}%
\pgfpathlineto{\pgfqpoint{3.025048in}{3.275213in}}%
\pgfpathlineto{\pgfqpoint{3.029588in}{3.275213in}}%
\pgfpathlineto{\pgfqpoint{3.029588in}{3.272264in}}%
\pgfpathmoveto{\pgfqpoint{3.025048in}{3.275213in}}%
\pgfpathlineto{\pgfqpoint{3.025048in}{3.275213in}}%
\pgfpathlineto{\pgfqpoint{3.025048in}{3.278163in}}%
\pgfpathlineto{\pgfqpoint{3.029588in}{3.278163in}}%
\pgfpathlineto{\pgfqpoint{3.029588in}{3.275213in}}%
\pgfpathmoveto{\pgfqpoint{3.020507in}{3.278163in}}%
\pgfpathlineto{\pgfqpoint{3.020507in}{3.278163in}}%
\pgfpathlineto{\pgfqpoint{3.020507in}{3.281112in}}%
\pgfpathlineto{\pgfqpoint{3.025048in}{3.281112in}}%
\pgfpathlineto{\pgfqpoint{3.025048in}{3.278163in}}%
\pgfpathmoveto{\pgfqpoint{3.020507in}{3.281112in}}%
\pgfpathlineto{\pgfqpoint{3.020507in}{3.281112in}}%
\pgfpathlineto{\pgfqpoint{3.020507in}{3.284061in}}%
\pgfpathlineto{\pgfqpoint{3.025048in}{3.284061in}}%
\pgfpathlineto{\pgfqpoint{3.025048in}{3.281112in}}%
\pgfpathmoveto{\pgfqpoint{3.025048in}{3.278163in}}%
\pgfpathlineto{\pgfqpoint{3.025048in}{3.278163in}}%
\pgfpathlineto{\pgfqpoint{3.025048in}{3.281112in}}%
\pgfpathlineto{\pgfqpoint{3.029588in}{3.281112in}}%
\pgfpathlineto{\pgfqpoint{3.029588in}{3.278163in}}%
\pgfpathmoveto{\pgfqpoint{3.025048in}{3.281112in}}%
\pgfpathlineto{\pgfqpoint{3.025048in}{3.281112in}}%
\pgfpathlineto{\pgfqpoint{3.025048in}{3.284061in}}%
\pgfpathlineto{\pgfqpoint{3.029588in}{3.284061in}}%
\pgfpathlineto{\pgfqpoint{3.029588in}{3.281112in}}%
\pgfpathmoveto{\pgfqpoint{3.029588in}{3.272264in}}%
\pgfpathlineto{\pgfqpoint{3.029588in}{3.272264in}}%
\pgfpathlineto{\pgfqpoint{3.029588in}{3.275213in}}%
\pgfpathlineto{\pgfqpoint{3.034129in}{3.275213in}}%
\pgfpathlineto{\pgfqpoint{3.034129in}{3.272264in}}%
\pgfpathmoveto{\pgfqpoint{3.029588in}{3.275213in}}%
\pgfpathlineto{\pgfqpoint{3.029588in}{3.275213in}}%
\pgfpathlineto{\pgfqpoint{3.029588in}{3.278163in}}%
\pgfpathlineto{\pgfqpoint{3.034129in}{3.278163in}}%
\pgfpathlineto{\pgfqpoint{3.034129in}{3.275213in}}%
\pgfpathmoveto{\pgfqpoint{3.047752in}{3.236873in}}%
\pgfpathlineto{\pgfqpoint{3.047752in}{3.236873in}}%
\pgfpathlineto{\pgfqpoint{3.047752in}{3.239822in}}%
\pgfpathlineto{\pgfqpoint{3.052292in}{3.239822in}}%
\pgfpathlineto{\pgfqpoint{3.052292in}{3.236873in}}%
\pgfpathmoveto{\pgfqpoint{3.047752in}{3.239822in}}%
\pgfpathlineto{\pgfqpoint{3.047752in}{3.239822in}}%
\pgfpathlineto{\pgfqpoint{3.047752in}{3.242771in}}%
\pgfpathlineto{\pgfqpoint{3.052292in}{3.242771in}}%
\pgfpathlineto{\pgfqpoint{3.052292in}{3.239822in}}%
\pgfpathmoveto{\pgfqpoint{3.052292in}{3.236873in}}%
\pgfpathlineto{\pgfqpoint{3.052292in}{3.236873in}}%
\pgfpathlineto{\pgfqpoint{3.052292in}{3.239822in}}%
\pgfpathlineto{\pgfqpoint{3.056833in}{3.239822in}}%
\pgfpathlineto{\pgfqpoint{3.056833in}{3.236873in}}%
\pgfpathmoveto{\pgfqpoint{3.052292in}{3.239822in}}%
\pgfpathlineto{\pgfqpoint{3.052292in}{3.239822in}}%
\pgfpathlineto{\pgfqpoint{3.052292in}{3.242771in}}%
\pgfpathlineto{\pgfqpoint{3.056833in}{3.242771in}}%
\pgfpathlineto{\pgfqpoint{3.056833in}{3.239822in}}%
\pgfpathmoveto{\pgfqpoint{3.047752in}{3.242771in}}%
\pgfpathlineto{\pgfqpoint{3.047752in}{3.242771in}}%
\pgfpathlineto{\pgfqpoint{3.047752in}{3.245721in}}%
\pgfpathlineto{\pgfqpoint{3.052292in}{3.245721in}}%
\pgfpathlineto{\pgfqpoint{3.052292in}{3.242771in}}%
\pgfpathmoveto{\pgfqpoint{3.047752in}{3.245721in}}%
\pgfpathlineto{\pgfqpoint{3.047752in}{3.245721in}}%
\pgfpathlineto{\pgfqpoint{3.047752in}{3.248670in}}%
\pgfpathlineto{\pgfqpoint{3.052292in}{3.248670in}}%
\pgfpathlineto{\pgfqpoint{3.052292in}{3.245721in}}%
\pgfpathmoveto{\pgfqpoint{3.052292in}{3.242771in}}%
\pgfpathlineto{\pgfqpoint{3.052292in}{3.242771in}}%
\pgfpathlineto{\pgfqpoint{3.052292in}{3.245721in}}%
\pgfpathlineto{\pgfqpoint{3.056833in}{3.245721in}}%
\pgfpathlineto{\pgfqpoint{3.056833in}{3.242771in}}%
\pgfpathmoveto{\pgfqpoint{3.052292in}{3.245721in}}%
\pgfpathlineto{\pgfqpoint{3.052292in}{3.245721in}}%
\pgfpathlineto{\pgfqpoint{3.052292in}{3.248670in}}%
\pgfpathlineto{\pgfqpoint{3.056833in}{3.248670in}}%
\pgfpathlineto{\pgfqpoint{3.056833in}{3.245721in}}%
\pgfpathmoveto{\pgfqpoint{3.038670in}{3.248670in}}%
\pgfpathlineto{\pgfqpoint{3.038670in}{3.248670in}}%
\pgfpathlineto{\pgfqpoint{3.038670in}{3.251619in}}%
\pgfpathlineto{\pgfqpoint{3.043211in}{3.251619in}}%
\pgfpathlineto{\pgfqpoint{3.043211in}{3.248670in}}%
\pgfpathmoveto{\pgfqpoint{3.038670in}{3.251619in}}%
\pgfpathlineto{\pgfqpoint{3.038670in}{3.251619in}}%
\pgfpathlineto{\pgfqpoint{3.038670in}{3.254568in}}%
\pgfpathlineto{\pgfqpoint{3.043211in}{3.254568in}}%
\pgfpathlineto{\pgfqpoint{3.043211in}{3.251619in}}%
\pgfpathmoveto{\pgfqpoint{3.043211in}{3.248670in}}%
\pgfpathlineto{\pgfqpoint{3.043211in}{3.248670in}}%
\pgfpathlineto{\pgfqpoint{3.043211in}{3.251619in}}%
\pgfpathlineto{\pgfqpoint{3.047752in}{3.251619in}}%
\pgfpathlineto{\pgfqpoint{3.047752in}{3.248670in}}%
\pgfpathmoveto{\pgfqpoint{3.043211in}{3.251619in}}%
\pgfpathlineto{\pgfqpoint{3.043211in}{3.251619in}}%
\pgfpathlineto{\pgfqpoint{3.043211in}{3.254568in}}%
\pgfpathlineto{\pgfqpoint{3.047752in}{3.254568in}}%
\pgfpathlineto{\pgfqpoint{3.047752in}{3.251619in}}%
\pgfpathmoveto{\pgfqpoint{3.038670in}{3.254568in}}%
\pgfpathlineto{\pgfqpoint{3.038670in}{3.254568in}}%
\pgfpathlineto{\pgfqpoint{3.038670in}{3.257518in}}%
\pgfpathlineto{\pgfqpoint{3.043211in}{3.257518in}}%
\pgfpathlineto{\pgfqpoint{3.043211in}{3.254568in}}%
\pgfpathmoveto{\pgfqpoint{3.038670in}{3.257518in}}%
\pgfpathlineto{\pgfqpoint{3.038670in}{3.257518in}}%
\pgfpathlineto{\pgfqpoint{3.038670in}{3.260467in}}%
\pgfpathlineto{\pgfqpoint{3.043211in}{3.260467in}}%
\pgfpathlineto{\pgfqpoint{3.043211in}{3.257518in}}%
\pgfpathmoveto{\pgfqpoint{3.043211in}{3.254568in}}%
\pgfpathlineto{\pgfqpoint{3.043211in}{3.254568in}}%
\pgfpathlineto{\pgfqpoint{3.043211in}{3.257518in}}%
\pgfpathlineto{\pgfqpoint{3.047752in}{3.257518in}}%
\pgfpathlineto{\pgfqpoint{3.047752in}{3.254568in}}%
\pgfpathmoveto{\pgfqpoint{3.043211in}{3.257518in}}%
\pgfpathlineto{\pgfqpoint{3.043211in}{3.257518in}}%
\pgfpathlineto{\pgfqpoint{3.043211in}{3.260467in}}%
\pgfpathlineto{\pgfqpoint{3.047752in}{3.260467in}}%
\pgfpathlineto{\pgfqpoint{3.047752in}{3.257518in}}%
\pgfpathmoveto{\pgfqpoint{3.047752in}{3.248670in}}%
\pgfpathlineto{\pgfqpoint{3.047752in}{3.248670in}}%
\pgfpathlineto{\pgfqpoint{3.047752in}{3.251619in}}%
\pgfpathlineto{\pgfqpoint{3.052292in}{3.251619in}}%
\pgfpathlineto{\pgfqpoint{3.052292in}{3.248670in}}%
\pgfpathmoveto{\pgfqpoint{3.047752in}{3.251619in}}%
\pgfpathlineto{\pgfqpoint{3.047752in}{3.251619in}}%
\pgfpathlineto{\pgfqpoint{3.047752in}{3.254568in}}%
\pgfpathlineto{\pgfqpoint{3.052292in}{3.254568in}}%
\pgfpathlineto{\pgfqpoint{3.052292in}{3.251619in}}%
\pgfpathmoveto{\pgfqpoint{3.056833in}{3.236873in}}%
\pgfpathlineto{\pgfqpoint{3.056833in}{3.236873in}}%
\pgfpathlineto{\pgfqpoint{3.056833in}{3.239822in}}%
\pgfpathlineto{\pgfqpoint{3.061374in}{3.239822in}}%
\pgfpathlineto{\pgfqpoint{3.061374in}{3.236873in}}%
\pgfpathmoveto{\pgfqpoint{3.056833in}{3.239822in}}%
\pgfpathlineto{\pgfqpoint{3.056833in}{3.239822in}}%
\pgfpathlineto{\pgfqpoint{3.056833in}{3.242771in}}%
\pgfpathlineto{\pgfqpoint{3.061374in}{3.242771in}}%
\pgfpathlineto{\pgfqpoint{3.061374in}{3.239822in}}%
\pgfpathmoveto{\pgfqpoint{3.038670in}{3.260467in}}%
\pgfpathlineto{\pgfqpoint{3.038670in}{3.260467in}}%
\pgfpathlineto{\pgfqpoint{3.038670in}{3.263416in}}%
\pgfpathlineto{\pgfqpoint{3.043211in}{3.263416in}}%
\pgfpathlineto{\pgfqpoint{3.043211in}{3.260467in}}%
\pgfpathmoveto{\pgfqpoint{3.038670in}{3.263416in}}%
\pgfpathlineto{\pgfqpoint{3.038670in}{3.263416in}}%
\pgfpathlineto{\pgfqpoint{3.038670in}{3.266365in}}%
\pgfpathlineto{\pgfqpoint{3.043211in}{3.266365in}}%
\pgfpathlineto{\pgfqpoint{3.043211in}{3.263416in}}%
\pgfpathmoveto{\pgfqpoint{3.011425in}{3.284061in}}%
\pgfpathlineto{\pgfqpoint{3.011425in}{3.284061in}}%
\pgfpathlineto{\pgfqpoint{3.011425in}{3.287010in}}%
\pgfpathlineto{\pgfqpoint{3.015966in}{3.287010in}}%
\pgfpathlineto{\pgfqpoint{3.015966in}{3.284061in}}%
\pgfpathmoveto{\pgfqpoint{3.011425in}{3.287010in}}%
\pgfpathlineto{\pgfqpoint{3.011425in}{3.287010in}}%
\pgfpathlineto{\pgfqpoint{3.011425in}{3.289960in}}%
\pgfpathlineto{\pgfqpoint{3.015966in}{3.289960in}}%
\pgfpathlineto{\pgfqpoint{3.015966in}{3.287010in}}%
\pgfpathmoveto{\pgfqpoint{3.015966in}{3.284061in}}%
\pgfpathlineto{\pgfqpoint{3.015966in}{3.284061in}}%
\pgfpathlineto{\pgfqpoint{3.015966in}{3.287010in}}%
\pgfpathlineto{\pgfqpoint{3.020507in}{3.287010in}}%
\pgfpathlineto{\pgfqpoint{3.020507in}{3.284061in}}%
\pgfpathmoveto{\pgfqpoint{3.015966in}{3.287010in}}%
\pgfpathlineto{\pgfqpoint{3.015966in}{3.287010in}}%
\pgfpathlineto{\pgfqpoint{3.015966in}{3.289960in}}%
\pgfpathlineto{\pgfqpoint{3.020507in}{3.289960in}}%
\pgfpathlineto{\pgfqpoint{3.020507in}{3.287010in}}%
\pgfpathmoveto{\pgfqpoint{3.011425in}{3.289960in}}%
\pgfpathlineto{\pgfqpoint{3.011425in}{3.289960in}}%
\pgfpathlineto{\pgfqpoint{3.011425in}{3.292909in}}%
\pgfpathlineto{\pgfqpoint{3.015966in}{3.292909in}}%
\pgfpathlineto{\pgfqpoint{3.015966in}{3.289960in}}%
\pgfpathmoveto{\pgfqpoint{3.011425in}{3.292909in}}%
\pgfpathlineto{\pgfqpoint{3.011425in}{3.292909in}}%
\pgfpathlineto{\pgfqpoint{3.011425in}{3.295858in}}%
\pgfpathlineto{\pgfqpoint{3.015966in}{3.295858in}}%
\pgfpathlineto{\pgfqpoint{3.015966in}{3.292909in}}%
\pgfpathmoveto{\pgfqpoint{3.015966in}{3.289960in}}%
\pgfpathlineto{\pgfqpoint{3.015966in}{3.289960in}}%
\pgfpathlineto{\pgfqpoint{3.015966in}{3.292909in}}%
\pgfpathlineto{\pgfqpoint{3.020507in}{3.292909in}}%
\pgfpathlineto{\pgfqpoint{3.020507in}{3.289960in}}%
\pgfpathmoveto{\pgfqpoint{3.015966in}{3.292909in}}%
\pgfpathlineto{\pgfqpoint{3.015966in}{3.292909in}}%
\pgfpathlineto{\pgfqpoint{3.015966in}{3.295858in}}%
\pgfpathlineto{\pgfqpoint{3.020507in}{3.295858in}}%
\pgfpathlineto{\pgfqpoint{3.020507in}{3.292909in}}%
\pgfpathmoveto{\pgfqpoint{3.002344in}{3.295858in}}%
\pgfpathlineto{\pgfqpoint{3.002344in}{3.295858in}}%
\pgfpathlineto{\pgfqpoint{3.002344in}{3.298807in}}%
\pgfpathlineto{\pgfqpoint{3.006884in}{3.298807in}}%
\pgfpathlineto{\pgfqpoint{3.006884in}{3.295858in}}%
\pgfpathmoveto{\pgfqpoint{3.002344in}{3.298807in}}%
\pgfpathlineto{\pgfqpoint{3.002344in}{3.298807in}}%
\pgfpathlineto{\pgfqpoint{3.002344in}{3.301757in}}%
\pgfpathlineto{\pgfqpoint{3.006884in}{3.301757in}}%
\pgfpathlineto{\pgfqpoint{3.006884in}{3.298807in}}%
\pgfpathmoveto{\pgfqpoint{3.006884in}{3.295858in}}%
\pgfpathlineto{\pgfqpoint{3.006884in}{3.295858in}}%
\pgfpathlineto{\pgfqpoint{3.006884in}{3.298807in}}%
\pgfpathlineto{\pgfqpoint{3.011425in}{3.298807in}}%
\pgfpathlineto{\pgfqpoint{3.011425in}{3.295858in}}%
\pgfpathmoveto{\pgfqpoint{3.006884in}{3.298807in}}%
\pgfpathlineto{\pgfqpoint{3.006884in}{3.298807in}}%
\pgfpathlineto{\pgfqpoint{3.006884in}{3.301757in}}%
\pgfpathlineto{\pgfqpoint{3.011425in}{3.301757in}}%
\pgfpathlineto{\pgfqpoint{3.011425in}{3.298807in}}%
\pgfpathmoveto{\pgfqpoint{3.002344in}{3.301757in}}%
\pgfpathlineto{\pgfqpoint{3.002344in}{3.301757in}}%
\pgfpathlineto{\pgfqpoint{3.002344in}{3.304706in}}%
\pgfpathlineto{\pgfqpoint{3.006884in}{3.304706in}}%
\pgfpathlineto{\pgfqpoint{3.006884in}{3.301757in}}%
\pgfpathmoveto{\pgfqpoint{3.002344in}{3.304706in}}%
\pgfpathlineto{\pgfqpoint{3.002344in}{3.304706in}}%
\pgfpathlineto{\pgfqpoint{3.002344in}{3.307655in}}%
\pgfpathlineto{\pgfqpoint{3.006884in}{3.307655in}}%
\pgfpathlineto{\pgfqpoint{3.006884in}{3.304706in}}%
\pgfpathmoveto{\pgfqpoint{3.006884in}{3.301757in}}%
\pgfpathlineto{\pgfqpoint{3.006884in}{3.301757in}}%
\pgfpathlineto{\pgfqpoint{3.006884in}{3.304706in}}%
\pgfpathlineto{\pgfqpoint{3.011425in}{3.304706in}}%
\pgfpathlineto{\pgfqpoint{3.011425in}{3.301757in}}%
\pgfpathmoveto{\pgfqpoint{3.006884in}{3.304706in}}%
\pgfpathlineto{\pgfqpoint{3.006884in}{3.304706in}}%
\pgfpathlineto{\pgfqpoint{3.006884in}{3.307655in}}%
\pgfpathlineto{\pgfqpoint{3.011425in}{3.307655in}}%
\pgfpathlineto{\pgfqpoint{3.011425in}{3.304706in}}%
\pgfpathmoveto{\pgfqpoint{3.011425in}{3.295858in}}%
\pgfpathlineto{\pgfqpoint{3.011425in}{3.295858in}}%
\pgfpathlineto{\pgfqpoint{3.011425in}{3.298807in}}%
\pgfpathlineto{\pgfqpoint{3.015966in}{3.298807in}}%
\pgfpathlineto{\pgfqpoint{3.015966in}{3.295858in}}%
\pgfpathmoveto{\pgfqpoint{3.011425in}{3.298807in}}%
\pgfpathlineto{\pgfqpoint{3.011425in}{3.298807in}}%
\pgfpathlineto{\pgfqpoint{3.011425in}{3.301757in}}%
\pgfpathlineto{\pgfqpoint{3.015966in}{3.301757in}}%
\pgfpathlineto{\pgfqpoint{3.015966in}{3.298807in}}%
\pgfpathmoveto{\pgfqpoint{3.020507in}{3.284061in}}%
\pgfpathlineto{\pgfqpoint{3.020507in}{3.284061in}}%
\pgfpathlineto{\pgfqpoint{3.020507in}{3.287010in}}%
\pgfpathlineto{\pgfqpoint{3.025048in}{3.287010in}}%
\pgfpathlineto{\pgfqpoint{3.025048in}{3.284061in}}%
\pgfpathmoveto{\pgfqpoint{3.020507in}{3.287010in}}%
\pgfpathlineto{\pgfqpoint{3.020507in}{3.287010in}}%
\pgfpathlineto{\pgfqpoint{3.020507in}{3.289960in}}%
\pgfpathlineto{\pgfqpoint{3.025048in}{3.289960in}}%
\pgfpathlineto{\pgfqpoint{3.025048in}{3.287010in}}%
\pgfpathmoveto{\pgfqpoint{3.002344in}{3.307655in}}%
\pgfpathlineto{\pgfqpoint{3.002344in}{3.307655in}}%
\pgfpathlineto{\pgfqpoint{3.002344in}{3.310604in}}%
\pgfpathlineto{\pgfqpoint{3.006884in}{3.310604in}}%
\pgfpathlineto{\pgfqpoint{3.006884in}{3.307655in}}%
\pgfpathmoveto{\pgfqpoint{3.002344in}{3.310604in}}%
\pgfpathlineto{\pgfqpoint{3.002344in}{3.310604in}}%
\pgfpathlineto{\pgfqpoint{3.002344in}{3.313554in}}%
\pgfpathlineto{\pgfqpoint{3.006884in}{3.313554in}}%
\pgfpathlineto{\pgfqpoint{3.006884in}{3.310604in}}%
\pgfpathmoveto{\pgfqpoint{2.956936in}{3.354843in}}%
\pgfpathlineto{\pgfqpoint{2.956936in}{3.354843in}}%
\pgfpathlineto{\pgfqpoint{2.956936in}{3.357792in}}%
\pgfpathlineto{\pgfqpoint{2.961477in}{3.357792in}}%
\pgfpathlineto{\pgfqpoint{2.961477in}{3.354843in}}%
\pgfpathmoveto{\pgfqpoint{2.956936in}{3.357792in}}%
\pgfpathlineto{\pgfqpoint{2.956936in}{3.357792in}}%
\pgfpathlineto{\pgfqpoint{2.956936in}{3.360741in}}%
\pgfpathlineto{\pgfqpoint{2.961477in}{3.360741in}}%
\pgfpathlineto{\pgfqpoint{2.961477in}{3.357792in}}%
\pgfpathmoveto{\pgfqpoint{2.961477in}{3.354843in}}%
\pgfpathlineto{\pgfqpoint{2.961477in}{3.354843in}}%
\pgfpathlineto{\pgfqpoint{2.961477in}{3.357792in}}%
\pgfpathlineto{\pgfqpoint{2.966017in}{3.357792in}}%
\pgfpathlineto{\pgfqpoint{2.966017in}{3.354843in}}%
\pgfpathmoveto{\pgfqpoint{2.961477in}{3.357792in}}%
\pgfpathlineto{\pgfqpoint{2.961477in}{3.357792in}}%
\pgfpathlineto{\pgfqpoint{2.961477in}{3.360741in}}%
\pgfpathlineto{\pgfqpoint{2.966017in}{3.360741in}}%
\pgfpathlineto{\pgfqpoint{2.966017in}{3.357792in}}%
\pgfpathmoveto{\pgfqpoint{2.956936in}{3.360741in}}%
\pgfpathlineto{\pgfqpoint{2.956936in}{3.360741in}}%
\pgfpathlineto{\pgfqpoint{2.956936in}{3.363691in}}%
\pgfpathlineto{\pgfqpoint{2.961477in}{3.363691in}}%
\pgfpathlineto{\pgfqpoint{2.961477in}{3.360741in}}%
\pgfpathmoveto{\pgfqpoint{2.956936in}{3.363691in}}%
\pgfpathlineto{\pgfqpoint{2.956936in}{3.363691in}}%
\pgfpathlineto{\pgfqpoint{2.956936in}{3.366640in}}%
\pgfpathlineto{\pgfqpoint{2.961477in}{3.366640in}}%
\pgfpathlineto{\pgfqpoint{2.961477in}{3.363691in}}%
\pgfpathmoveto{\pgfqpoint{2.961477in}{3.360741in}}%
\pgfpathlineto{\pgfqpoint{2.961477in}{3.360741in}}%
\pgfpathlineto{\pgfqpoint{2.961477in}{3.363691in}}%
\pgfpathlineto{\pgfqpoint{2.966017in}{3.363691in}}%
\pgfpathlineto{\pgfqpoint{2.966017in}{3.360741in}}%
\pgfpathmoveto{\pgfqpoint{2.961477in}{3.363691in}}%
\pgfpathlineto{\pgfqpoint{2.961477in}{3.363691in}}%
\pgfpathlineto{\pgfqpoint{2.961477in}{3.366640in}}%
\pgfpathlineto{\pgfqpoint{2.966017in}{3.366640in}}%
\pgfpathlineto{\pgfqpoint{2.966017in}{3.363691in}}%
\pgfpathmoveto{\pgfqpoint{2.947854in}{3.366640in}}%
\pgfpathlineto{\pgfqpoint{2.947854in}{3.366640in}}%
\pgfpathlineto{\pgfqpoint{2.947854in}{3.369589in}}%
\pgfpathlineto{\pgfqpoint{2.952395in}{3.369589in}}%
\pgfpathlineto{\pgfqpoint{2.952395in}{3.366640in}}%
\pgfpathmoveto{\pgfqpoint{2.947854in}{3.369589in}}%
\pgfpathlineto{\pgfqpoint{2.947854in}{3.369589in}}%
\pgfpathlineto{\pgfqpoint{2.947854in}{3.372538in}}%
\pgfpathlineto{\pgfqpoint{2.952395in}{3.372538in}}%
\pgfpathlineto{\pgfqpoint{2.952395in}{3.369589in}}%
\pgfpathmoveto{\pgfqpoint{2.952395in}{3.366640in}}%
\pgfpathlineto{\pgfqpoint{2.952395in}{3.366640in}}%
\pgfpathlineto{\pgfqpoint{2.952395in}{3.369589in}}%
\pgfpathlineto{\pgfqpoint{2.956936in}{3.369589in}}%
\pgfpathlineto{\pgfqpoint{2.956936in}{3.366640in}}%
\pgfpathmoveto{\pgfqpoint{2.952395in}{3.369589in}}%
\pgfpathlineto{\pgfqpoint{2.952395in}{3.369589in}}%
\pgfpathlineto{\pgfqpoint{2.952395in}{3.372538in}}%
\pgfpathlineto{\pgfqpoint{2.956936in}{3.372538in}}%
\pgfpathlineto{\pgfqpoint{2.956936in}{3.369589in}}%
\pgfpathmoveto{\pgfqpoint{2.947854in}{3.372538in}}%
\pgfpathlineto{\pgfqpoint{2.947854in}{3.372538in}}%
\pgfpathlineto{\pgfqpoint{2.947854in}{3.375488in}}%
\pgfpathlineto{\pgfqpoint{2.952395in}{3.375488in}}%
\pgfpathlineto{\pgfqpoint{2.952395in}{3.372538in}}%
\pgfpathmoveto{\pgfqpoint{2.947854in}{3.375488in}}%
\pgfpathlineto{\pgfqpoint{2.947854in}{3.375488in}}%
\pgfpathlineto{\pgfqpoint{2.947854in}{3.378437in}}%
\pgfpathlineto{\pgfqpoint{2.952395in}{3.378437in}}%
\pgfpathlineto{\pgfqpoint{2.952395in}{3.375488in}}%
\pgfpathmoveto{\pgfqpoint{2.952395in}{3.372538in}}%
\pgfpathlineto{\pgfqpoint{2.952395in}{3.372538in}}%
\pgfpathlineto{\pgfqpoint{2.952395in}{3.375488in}}%
\pgfpathlineto{\pgfqpoint{2.956936in}{3.375488in}}%
\pgfpathlineto{\pgfqpoint{2.956936in}{3.372538in}}%
\pgfpathmoveto{\pgfqpoint{2.952395in}{3.375488in}}%
\pgfpathlineto{\pgfqpoint{2.952395in}{3.375488in}}%
\pgfpathlineto{\pgfqpoint{2.952395in}{3.378437in}}%
\pgfpathlineto{\pgfqpoint{2.956936in}{3.378437in}}%
\pgfpathlineto{\pgfqpoint{2.956936in}{3.375488in}}%
\pgfpathmoveto{\pgfqpoint{2.956936in}{3.366640in}}%
\pgfpathlineto{\pgfqpoint{2.956936in}{3.366640in}}%
\pgfpathlineto{\pgfqpoint{2.956936in}{3.369589in}}%
\pgfpathlineto{\pgfqpoint{2.961477in}{3.369589in}}%
\pgfpathlineto{\pgfqpoint{2.961477in}{3.366640in}}%
\pgfpathmoveto{\pgfqpoint{2.956936in}{3.369589in}}%
\pgfpathlineto{\pgfqpoint{2.956936in}{3.369589in}}%
\pgfpathlineto{\pgfqpoint{2.956936in}{3.372538in}}%
\pgfpathlineto{\pgfqpoint{2.961477in}{3.372538in}}%
\pgfpathlineto{\pgfqpoint{2.961477in}{3.369589in}}%
\pgfpathmoveto{\pgfqpoint{2.975099in}{3.331249in}}%
\pgfpathlineto{\pgfqpoint{2.975099in}{3.331249in}}%
\pgfpathlineto{\pgfqpoint{2.975099in}{3.334198in}}%
\pgfpathlineto{\pgfqpoint{2.979640in}{3.334198in}}%
\pgfpathlineto{\pgfqpoint{2.979640in}{3.331249in}}%
\pgfpathmoveto{\pgfqpoint{2.975099in}{3.334198in}}%
\pgfpathlineto{\pgfqpoint{2.975099in}{3.334198in}}%
\pgfpathlineto{\pgfqpoint{2.975099in}{3.337148in}}%
\pgfpathlineto{\pgfqpoint{2.979640in}{3.337148in}}%
\pgfpathlineto{\pgfqpoint{2.979640in}{3.334198in}}%
\pgfpathmoveto{\pgfqpoint{2.979640in}{3.331249in}}%
\pgfpathlineto{\pgfqpoint{2.979640in}{3.331249in}}%
\pgfpathlineto{\pgfqpoint{2.979640in}{3.334198in}}%
\pgfpathlineto{\pgfqpoint{2.984180in}{3.334198in}}%
\pgfpathlineto{\pgfqpoint{2.984180in}{3.331249in}}%
\pgfpathmoveto{\pgfqpoint{2.979640in}{3.334198in}}%
\pgfpathlineto{\pgfqpoint{2.979640in}{3.334198in}}%
\pgfpathlineto{\pgfqpoint{2.979640in}{3.337148in}}%
\pgfpathlineto{\pgfqpoint{2.984180in}{3.337148in}}%
\pgfpathlineto{\pgfqpoint{2.984180in}{3.334198in}}%
\pgfpathmoveto{\pgfqpoint{2.975099in}{3.337148in}}%
\pgfpathlineto{\pgfqpoint{2.975099in}{3.337148in}}%
\pgfpathlineto{\pgfqpoint{2.975099in}{3.340097in}}%
\pgfpathlineto{\pgfqpoint{2.979640in}{3.340097in}}%
\pgfpathlineto{\pgfqpoint{2.979640in}{3.337148in}}%
\pgfpathmoveto{\pgfqpoint{2.975099in}{3.340097in}}%
\pgfpathlineto{\pgfqpoint{2.975099in}{3.340097in}}%
\pgfpathlineto{\pgfqpoint{2.975099in}{3.343046in}}%
\pgfpathlineto{\pgfqpoint{2.979640in}{3.343046in}}%
\pgfpathlineto{\pgfqpoint{2.979640in}{3.340097in}}%
\pgfpathmoveto{\pgfqpoint{2.979640in}{3.337148in}}%
\pgfpathlineto{\pgfqpoint{2.979640in}{3.337148in}}%
\pgfpathlineto{\pgfqpoint{2.979640in}{3.340097in}}%
\pgfpathlineto{\pgfqpoint{2.984180in}{3.340097in}}%
\pgfpathlineto{\pgfqpoint{2.984180in}{3.337148in}}%
\pgfpathmoveto{\pgfqpoint{2.979640in}{3.340097in}}%
\pgfpathlineto{\pgfqpoint{2.979640in}{3.340097in}}%
\pgfpathlineto{\pgfqpoint{2.979640in}{3.343046in}}%
\pgfpathlineto{\pgfqpoint{2.984180in}{3.343046in}}%
\pgfpathlineto{\pgfqpoint{2.984180in}{3.340097in}}%
\pgfpathmoveto{\pgfqpoint{2.966017in}{3.343046in}}%
\pgfpathlineto{\pgfqpoint{2.966017in}{3.343046in}}%
\pgfpathlineto{\pgfqpoint{2.966017in}{3.345995in}}%
\pgfpathlineto{\pgfqpoint{2.970558in}{3.345995in}}%
\pgfpathlineto{\pgfqpoint{2.970558in}{3.343046in}}%
\pgfpathmoveto{\pgfqpoint{2.966017in}{3.345995in}}%
\pgfpathlineto{\pgfqpoint{2.966017in}{3.345995in}}%
\pgfpathlineto{\pgfqpoint{2.966017in}{3.348944in}}%
\pgfpathlineto{\pgfqpoint{2.970558in}{3.348944in}}%
\pgfpathlineto{\pgfqpoint{2.970558in}{3.345995in}}%
\pgfpathmoveto{\pgfqpoint{2.970558in}{3.343046in}}%
\pgfpathlineto{\pgfqpoint{2.970558in}{3.343046in}}%
\pgfpathlineto{\pgfqpoint{2.970558in}{3.345995in}}%
\pgfpathlineto{\pgfqpoint{2.975099in}{3.345995in}}%
\pgfpathlineto{\pgfqpoint{2.975099in}{3.343046in}}%
\pgfpathmoveto{\pgfqpoint{2.970558in}{3.345995in}}%
\pgfpathlineto{\pgfqpoint{2.970558in}{3.345995in}}%
\pgfpathlineto{\pgfqpoint{2.970558in}{3.348944in}}%
\pgfpathlineto{\pgfqpoint{2.975099in}{3.348944in}}%
\pgfpathlineto{\pgfqpoint{2.975099in}{3.345995in}}%
\pgfpathmoveto{\pgfqpoint{2.966017in}{3.348944in}}%
\pgfpathlineto{\pgfqpoint{2.966017in}{3.348944in}}%
\pgfpathlineto{\pgfqpoint{2.966017in}{3.351894in}}%
\pgfpathlineto{\pgfqpoint{2.970558in}{3.351894in}}%
\pgfpathlineto{\pgfqpoint{2.970558in}{3.348944in}}%
\pgfpathmoveto{\pgfqpoint{2.966017in}{3.351894in}}%
\pgfpathlineto{\pgfqpoint{2.966017in}{3.351894in}}%
\pgfpathlineto{\pgfqpoint{2.966017in}{3.354843in}}%
\pgfpathlineto{\pgfqpoint{2.970558in}{3.354843in}}%
\pgfpathlineto{\pgfqpoint{2.970558in}{3.351894in}}%
\pgfpathmoveto{\pgfqpoint{2.970558in}{3.348944in}}%
\pgfpathlineto{\pgfqpoint{2.970558in}{3.348944in}}%
\pgfpathlineto{\pgfqpoint{2.970558in}{3.351894in}}%
\pgfpathlineto{\pgfqpoint{2.975099in}{3.351894in}}%
\pgfpathlineto{\pgfqpoint{2.975099in}{3.348944in}}%
\pgfpathmoveto{\pgfqpoint{2.970558in}{3.351894in}}%
\pgfpathlineto{\pgfqpoint{2.970558in}{3.351894in}}%
\pgfpathlineto{\pgfqpoint{2.970558in}{3.354843in}}%
\pgfpathlineto{\pgfqpoint{2.975099in}{3.354843in}}%
\pgfpathlineto{\pgfqpoint{2.975099in}{3.351894in}}%
\pgfpathmoveto{\pgfqpoint{2.975099in}{3.343046in}}%
\pgfpathlineto{\pgfqpoint{2.975099in}{3.343046in}}%
\pgfpathlineto{\pgfqpoint{2.975099in}{3.345995in}}%
\pgfpathlineto{\pgfqpoint{2.979640in}{3.345995in}}%
\pgfpathlineto{\pgfqpoint{2.979640in}{3.343046in}}%
\pgfpathmoveto{\pgfqpoint{2.975099in}{3.345995in}}%
\pgfpathlineto{\pgfqpoint{2.975099in}{3.345995in}}%
\pgfpathlineto{\pgfqpoint{2.975099in}{3.348944in}}%
\pgfpathlineto{\pgfqpoint{2.979640in}{3.348944in}}%
\pgfpathlineto{\pgfqpoint{2.979640in}{3.345995in}}%
\pgfpathmoveto{\pgfqpoint{2.984180in}{3.331249in}}%
\pgfpathlineto{\pgfqpoint{2.984180in}{3.331249in}}%
\pgfpathlineto{\pgfqpoint{2.984180in}{3.334198in}}%
\pgfpathlineto{\pgfqpoint{2.988721in}{3.334198in}}%
\pgfpathlineto{\pgfqpoint{2.988721in}{3.331249in}}%
\pgfpathmoveto{\pgfqpoint{2.984180in}{3.334198in}}%
\pgfpathlineto{\pgfqpoint{2.984180in}{3.334198in}}%
\pgfpathlineto{\pgfqpoint{2.984180in}{3.337148in}}%
\pgfpathlineto{\pgfqpoint{2.988721in}{3.337148in}}%
\pgfpathlineto{\pgfqpoint{2.988721in}{3.334198in}}%
\pgfpathmoveto{\pgfqpoint{2.966017in}{3.354843in}}%
\pgfpathlineto{\pgfqpoint{2.966017in}{3.354843in}}%
\pgfpathlineto{\pgfqpoint{2.966017in}{3.357792in}}%
\pgfpathlineto{\pgfqpoint{2.970558in}{3.357792in}}%
\pgfpathlineto{\pgfqpoint{2.970558in}{3.354843in}}%
\pgfpathmoveto{\pgfqpoint{2.966017in}{3.357792in}}%
\pgfpathlineto{\pgfqpoint{2.966017in}{3.357792in}}%
\pgfpathlineto{\pgfqpoint{2.966017in}{3.360741in}}%
\pgfpathlineto{\pgfqpoint{2.970558in}{3.360741in}}%
\pgfpathlineto{\pgfqpoint{2.970558in}{3.357792in}}%
\pgfpathmoveto{\pgfqpoint{2.938773in}{3.378437in}}%
\pgfpathlineto{\pgfqpoint{2.938773in}{3.378437in}}%
\pgfpathlineto{\pgfqpoint{2.938773in}{3.381386in}}%
\pgfpathlineto{\pgfqpoint{2.943313in}{3.381386in}}%
\pgfpathlineto{\pgfqpoint{2.943313in}{3.378437in}}%
\pgfpathmoveto{\pgfqpoint{2.938773in}{3.381386in}}%
\pgfpathlineto{\pgfqpoint{2.938773in}{3.381386in}}%
\pgfpathlineto{\pgfqpoint{2.938773in}{3.384335in}}%
\pgfpathlineto{\pgfqpoint{2.943313in}{3.384335in}}%
\pgfpathlineto{\pgfqpoint{2.943313in}{3.381386in}}%
\pgfpathmoveto{\pgfqpoint{2.943313in}{3.378437in}}%
\pgfpathlineto{\pgfqpoint{2.943313in}{3.378437in}}%
\pgfpathlineto{\pgfqpoint{2.943313in}{3.381386in}}%
\pgfpathlineto{\pgfqpoint{2.947854in}{3.381386in}}%
\pgfpathlineto{\pgfqpoint{2.947854in}{3.378437in}}%
\pgfpathmoveto{\pgfqpoint{2.943313in}{3.381386in}}%
\pgfpathlineto{\pgfqpoint{2.943313in}{3.381386in}}%
\pgfpathlineto{\pgfqpoint{2.943313in}{3.384335in}}%
\pgfpathlineto{\pgfqpoint{2.947854in}{3.384335in}}%
\pgfpathlineto{\pgfqpoint{2.947854in}{3.381386in}}%
\pgfpathmoveto{\pgfqpoint{2.938773in}{3.384335in}}%
\pgfpathlineto{\pgfqpoint{2.938773in}{3.384335in}}%
\pgfpathlineto{\pgfqpoint{2.938773in}{3.387285in}}%
\pgfpathlineto{\pgfqpoint{2.943313in}{3.387285in}}%
\pgfpathlineto{\pgfqpoint{2.943313in}{3.384335in}}%
\pgfpathmoveto{\pgfqpoint{2.938773in}{3.387285in}}%
\pgfpathlineto{\pgfqpoint{2.938773in}{3.387285in}}%
\pgfpathlineto{\pgfqpoint{2.938773in}{3.390234in}}%
\pgfpathlineto{\pgfqpoint{2.943313in}{3.390234in}}%
\pgfpathlineto{\pgfqpoint{2.943313in}{3.387285in}}%
\pgfpathmoveto{\pgfqpoint{2.943313in}{3.384335in}}%
\pgfpathlineto{\pgfqpoint{2.943313in}{3.384335in}}%
\pgfpathlineto{\pgfqpoint{2.943313in}{3.387285in}}%
\pgfpathlineto{\pgfqpoint{2.947854in}{3.387285in}}%
\pgfpathlineto{\pgfqpoint{2.947854in}{3.384335in}}%
\pgfpathmoveto{\pgfqpoint{2.943313in}{3.387285in}}%
\pgfpathlineto{\pgfqpoint{2.943313in}{3.387285in}}%
\pgfpathlineto{\pgfqpoint{2.943313in}{3.390234in}}%
\pgfpathlineto{\pgfqpoint{2.947854in}{3.390234in}}%
\pgfpathlineto{\pgfqpoint{2.947854in}{3.387285in}}%
\pgfpathmoveto{\pgfqpoint{2.929691in}{3.390234in}}%
\pgfpathlineto{\pgfqpoint{2.929691in}{3.390234in}}%
\pgfpathlineto{\pgfqpoint{2.929691in}{3.393183in}}%
\pgfpathlineto{\pgfqpoint{2.934232in}{3.393183in}}%
\pgfpathlineto{\pgfqpoint{2.934232in}{3.390234in}}%
\pgfpathmoveto{\pgfqpoint{2.929691in}{3.393183in}}%
\pgfpathlineto{\pgfqpoint{2.929691in}{3.393183in}}%
\pgfpathlineto{\pgfqpoint{2.929691in}{3.396132in}}%
\pgfpathlineto{\pgfqpoint{2.934232in}{3.396132in}}%
\pgfpathlineto{\pgfqpoint{2.934232in}{3.393183in}}%
\pgfpathmoveto{\pgfqpoint{2.934232in}{3.390234in}}%
\pgfpathlineto{\pgfqpoint{2.934232in}{3.390234in}}%
\pgfpathlineto{\pgfqpoint{2.934232in}{3.393183in}}%
\pgfpathlineto{\pgfqpoint{2.938773in}{3.393183in}}%
\pgfpathlineto{\pgfqpoint{2.938773in}{3.390234in}}%
\pgfpathmoveto{\pgfqpoint{2.934232in}{3.393183in}}%
\pgfpathlineto{\pgfqpoint{2.934232in}{3.393183in}}%
\pgfpathlineto{\pgfqpoint{2.934232in}{3.396132in}}%
\pgfpathlineto{\pgfqpoint{2.938773in}{3.396132in}}%
\pgfpathlineto{\pgfqpoint{2.938773in}{3.393183in}}%
\pgfpathmoveto{\pgfqpoint{2.929691in}{3.396132in}}%
\pgfpathlineto{\pgfqpoint{2.929691in}{3.396132in}}%
\pgfpathlineto{\pgfqpoint{2.929691in}{3.399081in}}%
\pgfpathlineto{\pgfqpoint{2.934232in}{3.399081in}}%
\pgfpathlineto{\pgfqpoint{2.934232in}{3.396132in}}%
\pgfpathmoveto{\pgfqpoint{2.929691in}{3.399081in}}%
\pgfpathlineto{\pgfqpoint{2.929691in}{3.399081in}}%
\pgfpathlineto{\pgfqpoint{2.929691in}{3.402031in}}%
\pgfpathlineto{\pgfqpoint{2.934232in}{3.402031in}}%
\pgfpathlineto{\pgfqpoint{2.934232in}{3.399081in}}%
\pgfpathmoveto{\pgfqpoint{2.934232in}{3.396132in}}%
\pgfpathlineto{\pgfqpoint{2.934232in}{3.396132in}}%
\pgfpathlineto{\pgfqpoint{2.934232in}{3.399081in}}%
\pgfpathlineto{\pgfqpoint{2.938773in}{3.399081in}}%
\pgfpathlineto{\pgfqpoint{2.938773in}{3.396132in}}%
\pgfpathmoveto{\pgfqpoint{2.934232in}{3.399081in}}%
\pgfpathlineto{\pgfqpoint{2.934232in}{3.399081in}}%
\pgfpathlineto{\pgfqpoint{2.934232in}{3.402031in}}%
\pgfpathlineto{\pgfqpoint{2.938773in}{3.402031in}}%
\pgfpathlineto{\pgfqpoint{2.938773in}{3.399081in}}%
\pgfpathmoveto{\pgfqpoint{2.938773in}{3.390234in}}%
\pgfpathlineto{\pgfqpoint{2.938773in}{3.390234in}}%
\pgfpathlineto{\pgfqpoint{2.938773in}{3.393183in}}%
\pgfpathlineto{\pgfqpoint{2.943313in}{3.393183in}}%
\pgfpathlineto{\pgfqpoint{2.943313in}{3.390234in}}%
\pgfpathmoveto{\pgfqpoint{2.938773in}{3.393183in}}%
\pgfpathlineto{\pgfqpoint{2.938773in}{3.393183in}}%
\pgfpathlineto{\pgfqpoint{2.938773in}{3.396132in}}%
\pgfpathlineto{\pgfqpoint{2.943313in}{3.396132in}}%
\pgfpathlineto{\pgfqpoint{2.943313in}{3.393183in}}%
\pgfpathmoveto{\pgfqpoint{2.947854in}{3.378437in}}%
\pgfpathlineto{\pgfqpoint{2.947854in}{3.378437in}}%
\pgfpathlineto{\pgfqpoint{2.947854in}{3.381386in}}%
\pgfpathlineto{\pgfqpoint{2.952395in}{3.381386in}}%
\pgfpathlineto{\pgfqpoint{2.952395in}{3.378437in}}%
\pgfpathmoveto{\pgfqpoint{2.947854in}{3.381386in}}%
\pgfpathlineto{\pgfqpoint{2.947854in}{3.381386in}}%
\pgfpathlineto{\pgfqpoint{2.947854in}{3.384335in}}%
\pgfpathlineto{\pgfqpoint{2.952395in}{3.384335in}}%
\pgfpathlineto{\pgfqpoint{2.952395in}{3.381386in}}%
\pgfpathmoveto{\pgfqpoint{2.929691in}{3.402031in}}%
\pgfpathlineto{\pgfqpoint{2.929691in}{3.402031in}}%
\pgfpathlineto{\pgfqpoint{2.929691in}{3.404980in}}%
\pgfpathlineto{\pgfqpoint{2.934232in}{3.404980in}}%
\pgfpathlineto{\pgfqpoint{2.934232in}{3.402031in}}%
\pgfpathmoveto{\pgfqpoint{2.929691in}{3.404980in}}%
\pgfpathlineto{\pgfqpoint{2.929691in}{3.404980in}}%
\pgfpathlineto{\pgfqpoint{2.929691in}{3.407929in}}%
\pgfpathlineto{\pgfqpoint{2.934232in}{3.407929in}}%
\pgfpathlineto{\pgfqpoint{2.934232in}{3.404980in}}%
\pgfpathmoveto{\pgfqpoint{3.074996in}{2.009997in}}%
\pgfpathlineto{\pgfqpoint{3.074996in}{2.009997in}}%
\pgfpathlineto{\pgfqpoint{3.074996in}{2.012947in}}%
\pgfpathlineto{\pgfqpoint{3.079538in}{2.012947in}}%
\pgfpathlineto{\pgfqpoint{3.079538in}{2.009997in}}%
\pgfpathmoveto{\pgfqpoint{3.074996in}{2.012947in}}%
\pgfpathlineto{\pgfqpoint{3.074996in}{2.012947in}}%
\pgfpathlineto{\pgfqpoint{3.074996in}{2.015896in}}%
\pgfpathlineto{\pgfqpoint{3.079538in}{2.015896in}}%
\pgfpathlineto{\pgfqpoint{3.079538in}{2.012947in}}%
\pgfpathmoveto{\pgfqpoint{3.079538in}{2.012947in}}%
\pgfpathlineto{\pgfqpoint{3.079538in}{2.012947in}}%
\pgfpathlineto{\pgfqpoint{3.079538in}{2.015896in}}%
\pgfpathlineto{\pgfqpoint{3.084079in}{2.015896in}}%
\pgfpathlineto{\pgfqpoint{3.084079in}{2.012947in}}%
\pgfpathmoveto{\pgfqpoint{3.074996in}{2.015896in}}%
\pgfpathlineto{\pgfqpoint{3.074996in}{2.015896in}}%
\pgfpathlineto{\pgfqpoint{3.074996in}{2.018846in}}%
\pgfpathlineto{\pgfqpoint{3.079538in}{2.018846in}}%
\pgfpathlineto{\pgfqpoint{3.079538in}{2.015896in}}%
\pgfpathmoveto{\pgfqpoint{3.074996in}{2.018846in}}%
\pgfpathlineto{\pgfqpoint{3.074996in}{2.018846in}}%
\pgfpathlineto{\pgfqpoint{3.074996in}{2.021795in}}%
\pgfpathlineto{\pgfqpoint{3.079538in}{2.021795in}}%
\pgfpathlineto{\pgfqpoint{3.079538in}{2.018846in}}%
\pgfpathmoveto{\pgfqpoint{3.079538in}{2.015896in}}%
\pgfpathlineto{\pgfqpoint{3.079538in}{2.015896in}}%
\pgfpathlineto{\pgfqpoint{3.079538in}{2.018846in}}%
\pgfpathlineto{\pgfqpoint{3.084079in}{2.018846in}}%
\pgfpathlineto{\pgfqpoint{3.084079in}{2.015896in}}%
\pgfpathmoveto{\pgfqpoint{3.079538in}{2.018846in}}%
\pgfpathlineto{\pgfqpoint{3.079538in}{2.018846in}}%
\pgfpathlineto{\pgfqpoint{3.079538in}{2.021795in}}%
\pgfpathlineto{\pgfqpoint{3.084079in}{2.021795in}}%
\pgfpathlineto{\pgfqpoint{3.084079in}{2.018846in}}%
\pgfpathmoveto{\pgfqpoint{3.084079in}{2.015896in}}%
\pgfpathlineto{\pgfqpoint{3.084079in}{2.015896in}}%
\pgfpathlineto{\pgfqpoint{3.084079in}{2.018846in}}%
\pgfpathlineto{\pgfqpoint{3.088620in}{2.018846in}}%
\pgfpathlineto{\pgfqpoint{3.088620in}{2.015896in}}%
\pgfpathmoveto{\pgfqpoint{3.084079in}{2.018846in}}%
\pgfpathlineto{\pgfqpoint{3.084079in}{2.018846in}}%
\pgfpathlineto{\pgfqpoint{3.084079in}{2.021795in}}%
\pgfpathlineto{\pgfqpoint{3.088620in}{2.021795in}}%
\pgfpathlineto{\pgfqpoint{3.088620in}{2.018846in}}%
\pgfpathmoveto{\pgfqpoint{3.088620in}{2.018846in}}%
\pgfpathlineto{\pgfqpoint{3.088620in}{2.018846in}}%
\pgfpathlineto{\pgfqpoint{3.088620in}{2.021795in}}%
\pgfpathlineto{\pgfqpoint{3.093161in}{2.021795in}}%
\pgfpathlineto{\pgfqpoint{3.093161in}{2.018846in}}%
\pgfpathmoveto{\pgfqpoint{3.084079in}{2.021795in}}%
\pgfpathlineto{\pgfqpoint{3.084079in}{2.021795in}}%
\pgfpathlineto{\pgfqpoint{3.084079in}{2.024744in}}%
\pgfpathlineto{\pgfqpoint{3.088620in}{2.024744in}}%
\pgfpathlineto{\pgfqpoint{3.088620in}{2.021795in}}%
\pgfpathmoveto{\pgfqpoint{3.084079in}{2.024744in}}%
\pgfpathlineto{\pgfqpoint{3.084079in}{2.024744in}}%
\pgfpathlineto{\pgfqpoint{3.084079in}{2.027694in}}%
\pgfpathlineto{\pgfqpoint{3.088620in}{2.027694in}}%
\pgfpathlineto{\pgfqpoint{3.088620in}{2.024744in}}%
\pgfpathmoveto{\pgfqpoint{3.088620in}{2.021795in}}%
\pgfpathlineto{\pgfqpoint{3.088620in}{2.021795in}}%
\pgfpathlineto{\pgfqpoint{3.088620in}{2.024744in}}%
\pgfpathlineto{\pgfqpoint{3.093161in}{2.024744in}}%
\pgfpathlineto{\pgfqpoint{3.093161in}{2.021795in}}%
\pgfpathmoveto{\pgfqpoint{3.088620in}{2.024744in}}%
\pgfpathlineto{\pgfqpoint{3.088620in}{2.024744in}}%
\pgfpathlineto{\pgfqpoint{3.088620in}{2.027694in}}%
\pgfpathlineto{\pgfqpoint{3.093161in}{2.027694in}}%
\pgfpathlineto{\pgfqpoint{3.093161in}{2.024744in}}%
\pgfpathmoveto{\pgfqpoint{3.093161in}{2.021795in}}%
\pgfpathlineto{\pgfqpoint{3.093161in}{2.021795in}}%
\pgfpathlineto{\pgfqpoint{3.093161in}{2.024744in}}%
\pgfpathlineto{\pgfqpoint{3.097703in}{2.024744in}}%
\pgfpathlineto{\pgfqpoint{3.097703in}{2.021795in}}%
\pgfpathmoveto{\pgfqpoint{3.093161in}{2.024744in}}%
\pgfpathlineto{\pgfqpoint{3.093161in}{2.024744in}}%
\pgfpathlineto{\pgfqpoint{3.093161in}{2.027694in}}%
\pgfpathlineto{\pgfqpoint{3.097703in}{2.027694in}}%
\pgfpathlineto{\pgfqpoint{3.097703in}{2.024744in}}%
\pgfpathmoveto{\pgfqpoint{3.097703in}{2.024744in}}%
\pgfpathlineto{\pgfqpoint{3.097703in}{2.024744in}}%
\pgfpathlineto{\pgfqpoint{3.097703in}{2.027694in}}%
\pgfpathlineto{\pgfqpoint{3.102244in}{2.027694in}}%
\pgfpathlineto{\pgfqpoint{3.102244in}{2.024744in}}%
\pgfpathmoveto{\pgfqpoint{3.102244in}{2.024744in}}%
\pgfpathlineto{\pgfqpoint{3.102244in}{2.024744in}}%
\pgfpathlineto{\pgfqpoint{3.102244in}{2.027694in}}%
\pgfpathlineto{\pgfqpoint{3.106785in}{2.027694in}}%
\pgfpathlineto{\pgfqpoint{3.106785in}{2.024744in}}%
\pgfpathmoveto{\pgfqpoint{3.102244in}{2.027694in}}%
\pgfpathlineto{\pgfqpoint{3.102244in}{2.027694in}}%
\pgfpathlineto{\pgfqpoint{3.102244in}{2.030643in}}%
\pgfpathlineto{\pgfqpoint{3.106785in}{2.030643in}}%
\pgfpathlineto{\pgfqpoint{3.106785in}{2.027694in}}%
\pgfpathmoveto{\pgfqpoint{3.102244in}{2.030643in}}%
\pgfpathlineto{\pgfqpoint{3.102244in}{2.030643in}}%
\pgfpathlineto{\pgfqpoint{3.102244in}{2.033593in}}%
\pgfpathlineto{\pgfqpoint{3.106785in}{2.033593in}}%
\pgfpathlineto{\pgfqpoint{3.106785in}{2.030643in}}%
\pgfpathmoveto{\pgfqpoint{3.106785in}{2.027694in}}%
\pgfpathlineto{\pgfqpoint{3.106785in}{2.027694in}}%
\pgfpathlineto{\pgfqpoint{3.106785in}{2.030643in}}%
\pgfpathlineto{\pgfqpoint{3.111326in}{2.030643in}}%
\pgfpathlineto{\pgfqpoint{3.111326in}{2.027694in}}%
\pgfpathmoveto{\pgfqpoint{3.106785in}{2.030643in}}%
\pgfpathlineto{\pgfqpoint{3.106785in}{2.030643in}}%
\pgfpathlineto{\pgfqpoint{3.106785in}{2.033593in}}%
\pgfpathlineto{\pgfqpoint{3.111326in}{2.033593in}}%
\pgfpathlineto{\pgfqpoint{3.111326in}{2.030643in}}%
\pgfpathmoveto{\pgfqpoint{3.111326in}{2.030643in}}%
\pgfpathlineto{\pgfqpoint{3.111326in}{2.030643in}}%
\pgfpathlineto{\pgfqpoint{3.111326in}{2.033593in}}%
\pgfpathlineto{\pgfqpoint{3.115868in}{2.033593in}}%
\pgfpathlineto{\pgfqpoint{3.115868in}{2.030643in}}%
\pgfpathmoveto{\pgfqpoint{3.111326in}{2.033593in}}%
\pgfpathlineto{\pgfqpoint{3.111326in}{2.033593in}}%
\pgfpathlineto{\pgfqpoint{3.111326in}{2.036542in}}%
\pgfpathlineto{\pgfqpoint{3.115868in}{2.036542in}}%
\pgfpathlineto{\pgfqpoint{3.115868in}{2.033593in}}%
\pgfpathmoveto{\pgfqpoint{3.111326in}{2.036542in}}%
\pgfpathlineto{\pgfqpoint{3.111326in}{2.036542in}}%
\pgfpathlineto{\pgfqpoint{3.111326in}{2.039491in}}%
\pgfpathlineto{\pgfqpoint{3.115868in}{2.039491in}}%
\pgfpathlineto{\pgfqpoint{3.115868in}{2.036542in}}%
\pgfpathmoveto{\pgfqpoint{3.115868in}{2.033593in}}%
\pgfpathlineto{\pgfqpoint{3.115868in}{2.033593in}}%
\pgfpathlineto{\pgfqpoint{3.115868in}{2.036542in}}%
\pgfpathlineto{\pgfqpoint{3.120409in}{2.036542in}}%
\pgfpathlineto{\pgfqpoint{3.120409in}{2.033593in}}%
\pgfpathmoveto{\pgfqpoint{3.115868in}{2.036542in}}%
\pgfpathlineto{\pgfqpoint{3.115868in}{2.036542in}}%
\pgfpathlineto{\pgfqpoint{3.115868in}{2.039491in}}%
\pgfpathlineto{\pgfqpoint{3.120409in}{2.039491in}}%
\pgfpathlineto{\pgfqpoint{3.120409in}{2.036542in}}%
\pgfpathmoveto{\pgfqpoint{3.120409in}{2.036542in}}%
\pgfpathlineto{\pgfqpoint{3.120409in}{2.036542in}}%
\pgfpathlineto{\pgfqpoint{3.120409in}{2.039491in}}%
\pgfpathlineto{\pgfqpoint{3.124950in}{2.039491in}}%
\pgfpathlineto{\pgfqpoint{3.124950in}{2.036542in}}%
\pgfpathmoveto{\pgfqpoint{3.120409in}{2.039491in}}%
\pgfpathlineto{\pgfqpoint{3.120409in}{2.039491in}}%
\pgfpathlineto{\pgfqpoint{3.120409in}{2.042441in}}%
\pgfpathlineto{\pgfqpoint{3.124950in}{2.042441in}}%
\pgfpathlineto{\pgfqpoint{3.124950in}{2.039491in}}%
\pgfpathmoveto{\pgfqpoint{3.120409in}{2.042441in}}%
\pgfpathlineto{\pgfqpoint{3.120409in}{2.042441in}}%
\pgfpathlineto{\pgfqpoint{3.120409in}{2.045390in}}%
\pgfpathlineto{\pgfqpoint{3.124950in}{2.045390in}}%
\pgfpathlineto{\pgfqpoint{3.124950in}{2.042441in}}%
\pgfpathmoveto{\pgfqpoint{3.124950in}{2.039491in}}%
\pgfpathlineto{\pgfqpoint{3.124950in}{2.039491in}}%
\pgfpathlineto{\pgfqpoint{3.124950in}{2.042441in}}%
\pgfpathlineto{\pgfqpoint{3.129491in}{2.042441in}}%
\pgfpathlineto{\pgfqpoint{3.129491in}{2.039491in}}%
\pgfpathmoveto{\pgfqpoint{3.124950in}{2.042441in}}%
\pgfpathlineto{\pgfqpoint{3.124950in}{2.042441in}}%
\pgfpathlineto{\pgfqpoint{3.124950in}{2.045390in}}%
\pgfpathlineto{\pgfqpoint{3.129491in}{2.045390in}}%
\pgfpathlineto{\pgfqpoint{3.129491in}{2.042441in}}%
\pgfpathmoveto{\pgfqpoint{3.129491in}{2.042441in}}%
\pgfpathlineto{\pgfqpoint{3.129491in}{2.042441in}}%
\pgfpathlineto{\pgfqpoint{3.129491in}{2.045390in}}%
\pgfpathlineto{\pgfqpoint{3.134033in}{2.045390in}}%
\pgfpathlineto{\pgfqpoint{3.134033in}{2.042441in}}%
\pgfpathmoveto{\pgfqpoint{3.129491in}{2.045390in}}%
\pgfpathlineto{\pgfqpoint{3.129491in}{2.045390in}}%
\pgfpathlineto{\pgfqpoint{3.129491in}{2.048340in}}%
\pgfpathlineto{\pgfqpoint{3.134033in}{2.048340in}}%
\pgfpathlineto{\pgfqpoint{3.134033in}{2.045390in}}%
\pgfpathmoveto{\pgfqpoint{3.129491in}{2.048340in}}%
\pgfpathlineto{\pgfqpoint{3.129491in}{2.048340in}}%
\pgfpathlineto{\pgfqpoint{3.129491in}{2.051289in}}%
\pgfpathlineto{\pgfqpoint{3.134033in}{2.051289in}}%
\pgfpathlineto{\pgfqpoint{3.134033in}{2.048340in}}%
\pgfpathmoveto{\pgfqpoint{3.134033in}{2.045390in}}%
\pgfpathlineto{\pgfqpoint{3.134033in}{2.045390in}}%
\pgfpathlineto{\pgfqpoint{3.134033in}{2.048340in}}%
\pgfpathlineto{\pgfqpoint{3.138574in}{2.048340in}}%
\pgfpathlineto{\pgfqpoint{3.138574in}{2.045390in}}%
\pgfpathmoveto{\pgfqpoint{3.134033in}{2.048340in}}%
\pgfpathlineto{\pgfqpoint{3.134033in}{2.048340in}}%
\pgfpathlineto{\pgfqpoint{3.134033in}{2.051289in}}%
\pgfpathlineto{\pgfqpoint{3.138574in}{2.051289in}}%
\pgfpathlineto{\pgfqpoint{3.138574in}{2.048340in}}%
\pgfpathmoveto{\pgfqpoint{3.138574in}{2.048340in}}%
\pgfpathlineto{\pgfqpoint{3.138574in}{2.048340in}}%
\pgfpathlineto{\pgfqpoint{3.138574in}{2.051289in}}%
\pgfpathlineto{\pgfqpoint{3.143115in}{2.051289in}}%
\pgfpathlineto{\pgfqpoint{3.143115in}{2.048340in}}%
\pgfpathmoveto{\pgfqpoint{3.138574in}{2.051289in}}%
\pgfpathlineto{\pgfqpoint{3.138574in}{2.051289in}}%
\pgfpathlineto{\pgfqpoint{3.138574in}{2.054238in}}%
\pgfpathlineto{\pgfqpoint{3.143115in}{2.054238in}}%
\pgfpathlineto{\pgfqpoint{3.143115in}{2.051289in}}%
\pgfpathmoveto{\pgfqpoint{3.138574in}{2.054238in}}%
\pgfpathlineto{\pgfqpoint{3.138574in}{2.054238in}}%
\pgfpathlineto{\pgfqpoint{3.138574in}{2.057188in}}%
\pgfpathlineto{\pgfqpoint{3.143115in}{2.057188in}}%
\pgfpathlineto{\pgfqpoint{3.143115in}{2.054238in}}%
\pgfpathmoveto{\pgfqpoint{3.143115in}{2.051289in}}%
\pgfpathlineto{\pgfqpoint{3.143115in}{2.051289in}}%
\pgfpathlineto{\pgfqpoint{3.143115in}{2.054238in}}%
\pgfpathlineto{\pgfqpoint{3.147656in}{2.054238in}}%
\pgfpathlineto{\pgfqpoint{3.147656in}{2.051289in}}%
\pgfpathmoveto{\pgfqpoint{3.143115in}{2.054238in}}%
\pgfpathlineto{\pgfqpoint{3.143115in}{2.054238in}}%
\pgfpathlineto{\pgfqpoint{3.143115in}{2.057188in}}%
\pgfpathlineto{\pgfqpoint{3.147656in}{2.057188in}}%
\pgfpathlineto{\pgfqpoint{3.147656in}{2.054238in}}%
\pgfpathmoveto{\pgfqpoint{3.147656in}{2.054238in}}%
\pgfpathlineto{\pgfqpoint{3.147656in}{2.054238in}}%
\pgfpathlineto{\pgfqpoint{3.147656in}{2.057188in}}%
\pgfpathlineto{\pgfqpoint{3.152198in}{2.057188in}}%
\pgfpathlineto{\pgfqpoint{3.152198in}{2.054238in}}%
\pgfpathmoveto{\pgfqpoint{3.147656in}{2.057188in}}%
\pgfpathlineto{\pgfqpoint{3.147656in}{2.057188in}}%
\pgfpathlineto{\pgfqpoint{3.147656in}{2.060137in}}%
\pgfpathlineto{\pgfqpoint{3.152198in}{2.060137in}}%
\pgfpathlineto{\pgfqpoint{3.152198in}{2.057188in}}%
\pgfpathmoveto{\pgfqpoint{3.147656in}{2.060137in}}%
\pgfpathlineto{\pgfqpoint{3.147656in}{2.060137in}}%
\pgfpathlineto{\pgfqpoint{3.147656in}{2.063086in}}%
\pgfpathlineto{\pgfqpoint{3.152198in}{2.063086in}}%
\pgfpathlineto{\pgfqpoint{3.152198in}{2.060137in}}%
\pgfpathmoveto{\pgfqpoint{3.152198in}{2.057188in}}%
\pgfpathlineto{\pgfqpoint{3.152198in}{2.057188in}}%
\pgfpathlineto{\pgfqpoint{3.152198in}{2.060137in}}%
\pgfpathlineto{\pgfqpoint{3.156739in}{2.060137in}}%
\pgfpathlineto{\pgfqpoint{3.156739in}{2.057188in}}%
\pgfpathmoveto{\pgfqpoint{3.152198in}{2.060137in}}%
\pgfpathlineto{\pgfqpoint{3.152198in}{2.060137in}}%
\pgfpathlineto{\pgfqpoint{3.152198in}{2.063086in}}%
\pgfpathlineto{\pgfqpoint{3.156739in}{2.063086in}}%
\pgfpathlineto{\pgfqpoint{3.156739in}{2.060137in}}%
\pgfpathmoveto{\pgfqpoint{3.156739in}{2.060137in}}%
\pgfpathlineto{\pgfqpoint{3.156739in}{2.060137in}}%
\pgfpathlineto{\pgfqpoint{3.156739in}{2.063086in}}%
\pgfpathlineto{\pgfqpoint{3.161280in}{2.063086in}}%
\pgfpathlineto{\pgfqpoint{3.161280in}{2.060137in}}%
\pgfpathmoveto{\pgfqpoint{3.156739in}{2.063086in}}%
\pgfpathlineto{\pgfqpoint{3.156739in}{2.063086in}}%
\pgfpathlineto{\pgfqpoint{3.156739in}{2.066036in}}%
\pgfpathlineto{\pgfqpoint{3.161280in}{2.066036in}}%
\pgfpathlineto{\pgfqpoint{3.161280in}{2.063086in}}%
\pgfpathmoveto{\pgfqpoint{3.156739in}{2.066036in}}%
\pgfpathlineto{\pgfqpoint{3.156739in}{2.066036in}}%
\pgfpathlineto{\pgfqpoint{3.156739in}{2.068985in}}%
\pgfpathlineto{\pgfqpoint{3.161280in}{2.068985in}}%
\pgfpathlineto{\pgfqpoint{3.161280in}{2.066036in}}%
\pgfpathmoveto{\pgfqpoint{3.161280in}{2.063086in}}%
\pgfpathlineto{\pgfqpoint{3.161280in}{2.063086in}}%
\pgfpathlineto{\pgfqpoint{3.161280in}{2.066036in}}%
\pgfpathlineto{\pgfqpoint{3.165821in}{2.066036in}}%
\pgfpathlineto{\pgfqpoint{3.165821in}{2.063086in}}%
\pgfpathmoveto{\pgfqpoint{3.161280in}{2.066036in}}%
\pgfpathlineto{\pgfqpoint{3.161280in}{2.066036in}}%
\pgfpathlineto{\pgfqpoint{3.161280in}{2.068985in}}%
\pgfpathlineto{\pgfqpoint{3.165821in}{2.068985in}}%
\pgfpathlineto{\pgfqpoint{3.165821in}{2.066036in}}%
\pgfpathmoveto{\pgfqpoint{3.165821in}{2.066036in}}%
\pgfpathlineto{\pgfqpoint{3.165821in}{2.066036in}}%
\pgfpathlineto{\pgfqpoint{3.165821in}{2.068985in}}%
\pgfpathlineto{\pgfqpoint{3.170363in}{2.068985in}}%
\pgfpathlineto{\pgfqpoint{3.170363in}{2.066036in}}%
\pgfpathmoveto{\pgfqpoint{3.165821in}{2.068985in}}%
\pgfpathlineto{\pgfqpoint{3.165821in}{2.068985in}}%
\pgfpathlineto{\pgfqpoint{3.165821in}{2.071935in}}%
\pgfpathlineto{\pgfqpoint{3.170363in}{2.071935in}}%
\pgfpathlineto{\pgfqpoint{3.170363in}{2.068985in}}%
\pgfpathmoveto{\pgfqpoint{3.165821in}{2.071935in}}%
\pgfpathlineto{\pgfqpoint{3.165821in}{2.071935in}}%
\pgfpathlineto{\pgfqpoint{3.165821in}{2.074884in}}%
\pgfpathlineto{\pgfqpoint{3.170363in}{2.074884in}}%
\pgfpathlineto{\pgfqpoint{3.170363in}{2.071935in}}%
\pgfpathmoveto{\pgfqpoint{3.170363in}{2.068985in}}%
\pgfpathlineto{\pgfqpoint{3.170363in}{2.068985in}}%
\pgfpathlineto{\pgfqpoint{3.170363in}{2.071935in}}%
\pgfpathlineto{\pgfqpoint{3.174904in}{2.071935in}}%
\pgfpathlineto{\pgfqpoint{3.174904in}{2.068985in}}%
\pgfpathmoveto{\pgfqpoint{3.170363in}{2.071935in}}%
\pgfpathlineto{\pgfqpoint{3.170363in}{2.071935in}}%
\pgfpathlineto{\pgfqpoint{3.170363in}{2.074884in}}%
\pgfpathlineto{\pgfqpoint{3.174904in}{2.074884in}}%
\pgfpathlineto{\pgfqpoint{3.174904in}{2.071935in}}%
\pgfpathmoveto{\pgfqpoint{3.174904in}{2.071935in}}%
\pgfpathlineto{\pgfqpoint{3.174904in}{2.071935in}}%
\pgfpathlineto{\pgfqpoint{3.174904in}{2.074884in}}%
\pgfpathlineto{\pgfqpoint{3.179445in}{2.074884in}}%
\pgfpathlineto{\pgfqpoint{3.179445in}{2.071935in}}%
\pgfpathmoveto{\pgfqpoint{3.174904in}{2.074884in}}%
\pgfpathlineto{\pgfqpoint{3.174904in}{2.074884in}}%
\pgfpathlineto{\pgfqpoint{3.174904in}{2.077833in}}%
\pgfpathlineto{\pgfqpoint{3.179445in}{2.077833in}}%
\pgfpathlineto{\pgfqpoint{3.179445in}{2.074884in}}%
\pgfpathmoveto{\pgfqpoint{3.174904in}{2.077833in}}%
\pgfpathlineto{\pgfqpoint{3.174904in}{2.077833in}}%
\pgfpathlineto{\pgfqpoint{3.174904in}{2.080783in}}%
\pgfpathlineto{\pgfqpoint{3.179445in}{2.080783in}}%
\pgfpathlineto{\pgfqpoint{3.179445in}{2.077833in}}%
\pgfpathmoveto{\pgfqpoint{3.179445in}{2.074884in}}%
\pgfpathlineto{\pgfqpoint{3.179445in}{2.074884in}}%
\pgfpathlineto{\pgfqpoint{3.179445in}{2.077833in}}%
\pgfpathlineto{\pgfqpoint{3.183986in}{2.077833in}}%
\pgfpathlineto{\pgfqpoint{3.183986in}{2.074884in}}%
\pgfpathmoveto{\pgfqpoint{3.179445in}{2.077833in}}%
\pgfpathlineto{\pgfqpoint{3.179445in}{2.077833in}}%
\pgfpathlineto{\pgfqpoint{3.179445in}{2.080783in}}%
\pgfpathlineto{\pgfqpoint{3.183986in}{2.080783in}}%
\pgfpathlineto{\pgfqpoint{3.183986in}{2.077833in}}%
\pgfpathmoveto{\pgfqpoint{3.183986in}{2.077833in}}%
\pgfpathlineto{\pgfqpoint{3.183986in}{2.077833in}}%
\pgfpathlineto{\pgfqpoint{3.183986in}{2.080783in}}%
\pgfpathlineto{\pgfqpoint{3.188528in}{2.080783in}}%
\pgfpathlineto{\pgfqpoint{3.188528in}{2.077833in}}%
\pgfpathmoveto{\pgfqpoint{3.183986in}{2.080783in}}%
\pgfpathlineto{\pgfqpoint{3.183986in}{2.080783in}}%
\pgfpathlineto{\pgfqpoint{3.183986in}{2.083732in}}%
\pgfpathlineto{\pgfqpoint{3.188528in}{2.083732in}}%
\pgfpathlineto{\pgfqpoint{3.188528in}{2.080783in}}%
\pgfpathmoveto{\pgfqpoint{3.183986in}{2.083732in}}%
\pgfpathlineto{\pgfqpoint{3.183986in}{2.083732in}}%
\pgfpathlineto{\pgfqpoint{3.183986in}{2.086682in}}%
\pgfpathlineto{\pgfqpoint{3.188528in}{2.086682in}}%
\pgfpathlineto{\pgfqpoint{3.188528in}{2.083732in}}%
\pgfpathmoveto{\pgfqpoint{3.188528in}{2.080783in}}%
\pgfpathlineto{\pgfqpoint{3.188528in}{2.080783in}}%
\pgfpathlineto{\pgfqpoint{3.188528in}{2.083732in}}%
\pgfpathlineto{\pgfqpoint{3.193069in}{2.083732in}}%
\pgfpathlineto{\pgfqpoint{3.193069in}{2.080783in}}%
\pgfpathmoveto{\pgfqpoint{3.188528in}{2.083732in}}%
\pgfpathlineto{\pgfqpoint{3.188528in}{2.083732in}}%
\pgfpathlineto{\pgfqpoint{3.188528in}{2.086682in}}%
\pgfpathlineto{\pgfqpoint{3.193069in}{2.086682in}}%
\pgfpathlineto{\pgfqpoint{3.193069in}{2.083732in}}%
\pgfpathmoveto{\pgfqpoint{3.193069in}{2.083732in}}%
\pgfpathlineto{\pgfqpoint{3.193069in}{2.083732in}}%
\pgfpathlineto{\pgfqpoint{3.193069in}{2.086682in}}%
\pgfpathlineto{\pgfqpoint{3.197610in}{2.086682in}}%
\pgfpathlineto{\pgfqpoint{3.197610in}{2.083732in}}%
\pgfpathmoveto{\pgfqpoint{3.193069in}{2.086682in}}%
\pgfpathlineto{\pgfqpoint{3.193069in}{2.086682in}}%
\pgfpathlineto{\pgfqpoint{3.193069in}{2.089631in}}%
\pgfpathlineto{\pgfqpoint{3.197610in}{2.089631in}}%
\pgfpathlineto{\pgfqpoint{3.197610in}{2.086682in}}%
\pgfpathmoveto{\pgfqpoint{3.193069in}{2.089631in}}%
\pgfpathlineto{\pgfqpoint{3.193069in}{2.089631in}}%
\pgfpathlineto{\pgfqpoint{3.193069in}{2.092580in}}%
\pgfpathlineto{\pgfqpoint{3.197610in}{2.092580in}}%
\pgfpathlineto{\pgfqpoint{3.197610in}{2.089631in}}%
\pgfpathmoveto{\pgfqpoint{3.197610in}{2.086682in}}%
\pgfpathlineto{\pgfqpoint{3.197610in}{2.086682in}}%
\pgfpathlineto{\pgfqpoint{3.197610in}{2.089631in}}%
\pgfpathlineto{\pgfqpoint{3.202151in}{2.089631in}}%
\pgfpathlineto{\pgfqpoint{3.202151in}{2.086682in}}%
\pgfpathmoveto{\pgfqpoint{3.197610in}{2.089631in}}%
\pgfpathlineto{\pgfqpoint{3.197610in}{2.089631in}}%
\pgfpathlineto{\pgfqpoint{3.197610in}{2.092580in}}%
\pgfpathlineto{\pgfqpoint{3.202151in}{2.092580in}}%
\pgfpathlineto{\pgfqpoint{3.202151in}{2.089631in}}%
\pgfpathmoveto{\pgfqpoint{3.202151in}{2.089631in}}%
\pgfpathlineto{\pgfqpoint{3.202151in}{2.089631in}}%
\pgfpathlineto{\pgfqpoint{3.202151in}{2.092580in}}%
\pgfpathlineto{\pgfqpoint{3.206693in}{2.092580in}}%
\pgfpathlineto{\pgfqpoint{3.206693in}{2.089631in}}%
\pgfpathmoveto{\pgfqpoint{3.202151in}{2.092580in}}%
\pgfpathlineto{\pgfqpoint{3.202151in}{2.092580in}}%
\pgfpathlineto{\pgfqpoint{3.202151in}{2.095530in}}%
\pgfpathlineto{\pgfqpoint{3.206693in}{2.095530in}}%
\pgfpathlineto{\pgfqpoint{3.206693in}{2.092580in}}%
\pgfpathmoveto{\pgfqpoint{3.202151in}{2.095530in}}%
\pgfpathlineto{\pgfqpoint{3.202151in}{2.095530in}}%
\pgfpathlineto{\pgfqpoint{3.202151in}{2.098479in}}%
\pgfpathlineto{\pgfqpoint{3.206693in}{2.098479in}}%
\pgfpathlineto{\pgfqpoint{3.206693in}{2.095530in}}%
\pgfpathmoveto{\pgfqpoint{3.206693in}{2.092580in}}%
\pgfpathlineto{\pgfqpoint{3.206693in}{2.092580in}}%
\pgfpathlineto{\pgfqpoint{3.206693in}{2.095530in}}%
\pgfpathlineto{\pgfqpoint{3.211234in}{2.095530in}}%
\pgfpathlineto{\pgfqpoint{3.211234in}{2.092580in}}%
\pgfpathmoveto{\pgfqpoint{3.206693in}{2.095530in}}%
\pgfpathlineto{\pgfqpoint{3.206693in}{2.095530in}}%
\pgfpathlineto{\pgfqpoint{3.206693in}{2.098479in}}%
\pgfpathlineto{\pgfqpoint{3.211234in}{2.098479in}}%
\pgfpathlineto{\pgfqpoint{3.211234in}{2.095530in}}%
\pgfpathmoveto{\pgfqpoint{3.211234in}{2.095530in}}%
\pgfpathlineto{\pgfqpoint{3.211234in}{2.095530in}}%
\pgfpathlineto{\pgfqpoint{3.211234in}{2.098479in}}%
\pgfpathlineto{\pgfqpoint{3.215775in}{2.098479in}}%
\pgfpathlineto{\pgfqpoint{3.215775in}{2.095530in}}%
\pgfpathmoveto{\pgfqpoint{3.211234in}{2.098479in}}%
\pgfpathlineto{\pgfqpoint{3.211234in}{2.098479in}}%
\pgfpathlineto{\pgfqpoint{3.211234in}{2.101428in}}%
\pgfpathlineto{\pgfqpoint{3.215775in}{2.101428in}}%
\pgfpathlineto{\pgfqpoint{3.215775in}{2.098479in}}%
\pgfpathmoveto{\pgfqpoint{3.211234in}{2.101428in}}%
\pgfpathlineto{\pgfqpoint{3.211234in}{2.101428in}}%
\pgfpathlineto{\pgfqpoint{3.211234in}{2.104378in}}%
\pgfpathlineto{\pgfqpoint{3.215775in}{2.104378in}}%
\pgfpathlineto{\pgfqpoint{3.215775in}{2.101428in}}%
\pgfpathmoveto{\pgfqpoint{3.215775in}{2.098479in}}%
\pgfpathlineto{\pgfqpoint{3.215775in}{2.098479in}}%
\pgfpathlineto{\pgfqpoint{3.215775in}{2.101428in}}%
\pgfpathlineto{\pgfqpoint{3.220316in}{2.101428in}}%
\pgfpathlineto{\pgfqpoint{3.220316in}{2.098479in}}%
\pgfpathmoveto{\pgfqpoint{3.215775in}{2.101428in}}%
\pgfpathlineto{\pgfqpoint{3.215775in}{2.101428in}}%
\pgfpathlineto{\pgfqpoint{3.215775in}{2.104378in}}%
\pgfpathlineto{\pgfqpoint{3.220316in}{2.104378in}}%
\pgfpathlineto{\pgfqpoint{3.220316in}{2.101428in}}%
\pgfpathmoveto{\pgfqpoint{3.211234in}{3.024531in}}%
\pgfpathlineto{\pgfqpoint{3.211234in}{3.024531in}}%
\pgfpathlineto{\pgfqpoint{3.211234in}{3.027481in}}%
\pgfpathlineto{\pgfqpoint{3.215775in}{3.027481in}}%
\pgfpathlineto{\pgfqpoint{3.215775in}{3.024531in}}%
\pgfpathmoveto{\pgfqpoint{3.211234in}{3.027481in}}%
\pgfpathlineto{\pgfqpoint{3.211234in}{3.027481in}}%
\pgfpathlineto{\pgfqpoint{3.211234in}{3.030430in}}%
\pgfpathlineto{\pgfqpoint{3.215775in}{3.030430in}}%
\pgfpathlineto{\pgfqpoint{3.215775in}{3.027481in}}%
\pgfpathmoveto{\pgfqpoint{3.215775in}{3.024531in}}%
\pgfpathlineto{\pgfqpoint{3.215775in}{3.024531in}}%
\pgfpathlineto{\pgfqpoint{3.215775in}{3.027481in}}%
\pgfpathlineto{\pgfqpoint{3.220316in}{3.027481in}}%
\pgfpathlineto{\pgfqpoint{3.220316in}{3.024531in}}%
\pgfpathmoveto{\pgfqpoint{3.215775in}{3.027481in}}%
\pgfpathlineto{\pgfqpoint{3.215775in}{3.027481in}}%
\pgfpathlineto{\pgfqpoint{3.215775in}{3.030430in}}%
\pgfpathlineto{\pgfqpoint{3.220316in}{3.030430in}}%
\pgfpathlineto{\pgfqpoint{3.220316in}{3.027481in}}%
\pgfpathmoveto{\pgfqpoint{3.211234in}{3.030430in}}%
\pgfpathlineto{\pgfqpoint{3.211234in}{3.030430in}}%
\pgfpathlineto{\pgfqpoint{3.211234in}{3.033379in}}%
\pgfpathlineto{\pgfqpoint{3.215775in}{3.033379in}}%
\pgfpathlineto{\pgfqpoint{3.215775in}{3.030430in}}%
\pgfpathmoveto{\pgfqpoint{3.211234in}{3.033379in}}%
\pgfpathlineto{\pgfqpoint{3.211234in}{3.033379in}}%
\pgfpathlineto{\pgfqpoint{3.211234in}{3.036329in}}%
\pgfpathlineto{\pgfqpoint{3.215775in}{3.036329in}}%
\pgfpathlineto{\pgfqpoint{3.215775in}{3.033379in}}%
\pgfpathmoveto{\pgfqpoint{3.215775in}{3.030430in}}%
\pgfpathlineto{\pgfqpoint{3.215775in}{3.030430in}}%
\pgfpathlineto{\pgfqpoint{3.215775in}{3.033379in}}%
\pgfpathlineto{\pgfqpoint{3.220316in}{3.033379in}}%
\pgfpathlineto{\pgfqpoint{3.220316in}{3.030430in}}%
\pgfpathmoveto{\pgfqpoint{3.215775in}{3.033379in}}%
\pgfpathlineto{\pgfqpoint{3.215775in}{3.033379in}}%
\pgfpathlineto{\pgfqpoint{3.215775in}{3.036329in}}%
\pgfpathlineto{\pgfqpoint{3.220316in}{3.036329in}}%
\pgfpathlineto{\pgfqpoint{3.220316in}{3.033379in}}%
\pgfpathmoveto{\pgfqpoint{3.202151in}{3.036329in}}%
\pgfpathlineto{\pgfqpoint{3.202151in}{3.036329in}}%
\pgfpathlineto{\pgfqpoint{3.202151in}{3.039278in}}%
\pgfpathlineto{\pgfqpoint{3.206693in}{3.039278in}}%
\pgfpathlineto{\pgfqpoint{3.206693in}{3.036329in}}%
\pgfpathmoveto{\pgfqpoint{3.202151in}{3.039278in}}%
\pgfpathlineto{\pgfqpoint{3.202151in}{3.039278in}}%
\pgfpathlineto{\pgfqpoint{3.202151in}{3.042227in}}%
\pgfpathlineto{\pgfqpoint{3.206693in}{3.042227in}}%
\pgfpathlineto{\pgfqpoint{3.206693in}{3.039278in}}%
\pgfpathmoveto{\pgfqpoint{3.206693in}{3.036329in}}%
\pgfpathlineto{\pgfqpoint{3.206693in}{3.036329in}}%
\pgfpathlineto{\pgfqpoint{3.206693in}{3.039278in}}%
\pgfpathlineto{\pgfqpoint{3.211234in}{3.039278in}}%
\pgfpathlineto{\pgfqpoint{3.211234in}{3.036329in}}%
\pgfpathmoveto{\pgfqpoint{3.206693in}{3.039278in}}%
\pgfpathlineto{\pgfqpoint{3.206693in}{3.039278in}}%
\pgfpathlineto{\pgfqpoint{3.206693in}{3.042227in}}%
\pgfpathlineto{\pgfqpoint{3.211234in}{3.042227in}}%
\pgfpathlineto{\pgfqpoint{3.211234in}{3.039278in}}%
\pgfpathmoveto{\pgfqpoint{3.202151in}{3.042227in}}%
\pgfpathlineto{\pgfqpoint{3.202151in}{3.042227in}}%
\pgfpathlineto{\pgfqpoint{3.202151in}{3.045177in}}%
\pgfpathlineto{\pgfqpoint{3.206693in}{3.045177in}}%
\pgfpathlineto{\pgfqpoint{3.206693in}{3.042227in}}%
\pgfpathmoveto{\pgfqpoint{3.202151in}{3.045177in}}%
\pgfpathlineto{\pgfqpoint{3.202151in}{3.045177in}}%
\pgfpathlineto{\pgfqpoint{3.202151in}{3.048126in}}%
\pgfpathlineto{\pgfqpoint{3.206693in}{3.048126in}}%
\pgfpathlineto{\pgfqpoint{3.206693in}{3.045177in}}%
\pgfpathmoveto{\pgfqpoint{3.206693in}{3.042227in}}%
\pgfpathlineto{\pgfqpoint{3.206693in}{3.042227in}}%
\pgfpathlineto{\pgfqpoint{3.206693in}{3.045177in}}%
\pgfpathlineto{\pgfqpoint{3.211234in}{3.045177in}}%
\pgfpathlineto{\pgfqpoint{3.211234in}{3.042227in}}%
\pgfpathmoveto{\pgfqpoint{3.206693in}{3.045177in}}%
\pgfpathlineto{\pgfqpoint{3.206693in}{3.045177in}}%
\pgfpathlineto{\pgfqpoint{3.206693in}{3.048126in}}%
\pgfpathlineto{\pgfqpoint{3.211234in}{3.048126in}}%
\pgfpathlineto{\pgfqpoint{3.211234in}{3.045177in}}%
\pgfpathmoveto{\pgfqpoint{3.211234in}{3.036329in}}%
\pgfpathlineto{\pgfqpoint{3.211234in}{3.036329in}}%
\pgfpathlineto{\pgfqpoint{3.211234in}{3.039278in}}%
\pgfpathlineto{\pgfqpoint{3.215775in}{3.039278in}}%
\pgfpathlineto{\pgfqpoint{3.215775in}{3.036329in}}%
\pgfpathmoveto{\pgfqpoint{3.211234in}{3.039278in}}%
\pgfpathlineto{\pgfqpoint{3.211234in}{3.039278in}}%
\pgfpathlineto{\pgfqpoint{3.211234in}{3.042227in}}%
\pgfpathlineto{\pgfqpoint{3.215775in}{3.042227in}}%
\pgfpathlineto{\pgfqpoint{3.215775in}{3.039278in}}%
\pgfpathmoveto{\pgfqpoint{3.138574in}{3.118907in}}%
\pgfpathlineto{\pgfqpoint{3.138574in}{3.118907in}}%
\pgfpathlineto{\pgfqpoint{3.138574in}{3.121857in}}%
\pgfpathlineto{\pgfqpoint{3.143115in}{3.121857in}}%
\pgfpathlineto{\pgfqpoint{3.143115in}{3.118907in}}%
\pgfpathmoveto{\pgfqpoint{3.138574in}{3.121857in}}%
\pgfpathlineto{\pgfqpoint{3.138574in}{3.121857in}}%
\pgfpathlineto{\pgfqpoint{3.138574in}{3.124806in}}%
\pgfpathlineto{\pgfqpoint{3.143115in}{3.124806in}}%
\pgfpathlineto{\pgfqpoint{3.143115in}{3.121857in}}%
\pgfpathmoveto{\pgfqpoint{3.143115in}{3.118907in}}%
\pgfpathlineto{\pgfqpoint{3.143115in}{3.118907in}}%
\pgfpathlineto{\pgfqpoint{3.143115in}{3.121857in}}%
\pgfpathlineto{\pgfqpoint{3.147656in}{3.121857in}}%
\pgfpathlineto{\pgfqpoint{3.147656in}{3.118907in}}%
\pgfpathmoveto{\pgfqpoint{3.143115in}{3.121857in}}%
\pgfpathlineto{\pgfqpoint{3.143115in}{3.121857in}}%
\pgfpathlineto{\pgfqpoint{3.143115in}{3.124806in}}%
\pgfpathlineto{\pgfqpoint{3.147656in}{3.124806in}}%
\pgfpathlineto{\pgfqpoint{3.147656in}{3.121857in}}%
\pgfpathmoveto{\pgfqpoint{3.138574in}{3.124806in}}%
\pgfpathlineto{\pgfqpoint{3.138574in}{3.124806in}}%
\pgfpathlineto{\pgfqpoint{3.138574in}{3.127755in}}%
\pgfpathlineto{\pgfqpoint{3.143115in}{3.127755in}}%
\pgfpathlineto{\pgfqpoint{3.143115in}{3.124806in}}%
\pgfpathmoveto{\pgfqpoint{3.138574in}{3.127755in}}%
\pgfpathlineto{\pgfqpoint{3.138574in}{3.127755in}}%
\pgfpathlineto{\pgfqpoint{3.138574in}{3.130704in}}%
\pgfpathlineto{\pgfqpoint{3.143115in}{3.130704in}}%
\pgfpathlineto{\pgfqpoint{3.143115in}{3.127755in}}%
\pgfpathmoveto{\pgfqpoint{3.143115in}{3.124806in}}%
\pgfpathlineto{\pgfqpoint{3.143115in}{3.124806in}}%
\pgfpathlineto{\pgfqpoint{3.143115in}{3.127755in}}%
\pgfpathlineto{\pgfqpoint{3.147656in}{3.127755in}}%
\pgfpathlineto{\pgfqpoint{3.147656in}{3.124806in}}%
\pgfpathmoveto{\pgfqpoint{3.143115in}{3.127755in}}%
\pgfpathlineto{\pgfqpoint{3.143115in}{3.127755in}}%
\pgfpathlineto{\pgfqpoint{3.143115in}{3.130704in}}%
\pgfpathlineto{\pgfqpoint{3.147656in}{3.130704in}}%
\pgfpathlineto{\pgfqpoint{3.147656in}{3.127755in}}%
\pgfpathmoveto{\pgfqpoint{3.129491in}{3.130704in}}%
\pgfpathlineto{\pgfqpoint{3.129491in}{3.130704in}}%
\pgfpathlineto{\pgfqpoint{3.129491in}{3.133654in}}%
\pgfpathlineto{\pgfqpoint{3.134033in}{3.133654in}}%
\pgfpathlineto{\pgfqpoint{3.134033in}{3.130704in}}%
\pgfpathmoveto{\pgfqpoint{3.129491in}{3.133654in}}%
\pgfpathlineto{\pgfqpoint{3.129491in}{3.133654in}}%
\pgfpathlineto{\pgfqpoint{3.129491in}{3.136603in}}%
\pgfpathlineto{\pgfqpoint{3.134033in}{3.136603in}}%
\pgfpathlineto{\pgfqpoint{3.134033in}{3.133654in}}%
\pgfpathmoveto{\pgfqpoint{3.134033in}{3.130704in}}%
\pgfpathlineto{\pgfqpoint{3.134033in}{3.130704in}}%
\pgfpathlineto{\pgfqpoint{3.134033in}{3.133654in}}%
\pgfpathlineto{\pgfqpoint{3.138574in}{3.133654in}}%
\pgfpathlineto{\pgfqpoint{3.138574in}{3.130704in}}%
\pgfpathmoveto{\pgfqpoint{3.134033in}{3.133654in}}%
\pgfpathlineto{\pgfqpoint{3.134033in}{3.133654in}}%
\pgfpathlineto{\pgfqpoint{3.134033in}{3.136603in}}%
\pgfpathlineto{\pgfqpoint{3.138574in}{3.136603in}}%
\pgfpathlineto{\pgfqpoint{3.138574in}{3.133654in}}%
\pgfpathmoveto{\pgfqpoint{3.129491in}{3.136603in}}%
\pgfpathlineto{\pgfqpoint{3.129491in}{3.136603in}}%
\pgfpathlineto{\pgfqpoint{3.129491in}{3.139552in}}%
\pgfpathlineto{\pgfqpoint{3.134033in}{3.139552in}}%
\pgfpathlineto{\pgfqpoint{3.134033in}{3.136603in}}%
\pgfpathmoveto{\pgfqpoint{3.129491in}{3.139552in}}%
\pgfpathlineto{\pgfqpoint{3.129491in}{3.139552in}}%
\pgfpathlineto{\pgfqpoint{3.129491in}{3.142501in}}%
\pgfpathlineto{\pgfqpoint{3.134033in}{3.142501in}}%
\pgfpathlineto{\pgfqpoint{3.134033in}{3.139552in}}%
\pgfpathmoveto{\pgfqpoint{3.134033in}{3.136603in}}%
\pgfpathlineto{\pgfqpoint{3.134033in}{3.136603in}}%
\pgfpathlineto{\pgfqpoint{3.134033in}{3.139552in}}%
\pgfpathlineto{\pgfqpoint{3.138574in}{3.139552in}}%
\pgfpathlineto{\pgfqpoint{3.138574in}{3.136603in}}%
\pgfpathmoveto{\pgfqpoint{3.134033in}{3.139552in}}%
\pgfpathlineto{\pgfqpoint{3.134033in}{3.139552in}}%
\pgfpathlineto{\pgfqpoint{3.134033in}{3.142501in}}%
\pgfpathlineto{\pgfqpoint{3.138574in}{3.142501in}}%
\pgfpathlineto{\pgfqpoint{3.138574in}{3.139552in}}%
\pgfpathmoveto{\pgfqpoint{3.138574in}{3.130704in}}%
\pgfpathlineto{\pgfqpoint{3.138574in}{3.130704in}}%
\pgfpathlineto{\pgfqpoint{3.138574in}{3.133654in}}%
\pgfpathlineto{\pgfqpoint{3.143115in}{3.133654in}}%
\pgfpathlineto{\pgfqpoint{3.143115in}{3.130704in}}%
\pgfpathmoveto{\pgfqpoint{3.138574in}{3.133654in}}%
\pgfpathlineto{\pgfqpoint{3.138574in}{3.133654in}}%
\pgfpathlineto{\pgfqpoint{3.138574in}{3.136603in}}%
\pgfpathlineto{\pgfqpoint{3.143115in}{3.136603in}}%
\pgfpathlineto{\pgfqpoint{3.143115in}{3.133654in}}%
\pgfpathmoveto{\pgfqpoint{3.174904in}{3.071720in}}%
\pgfpathlineto{\pgfqpoint{3.174904in}{3.071720in}}%
\pgfpathlineto{\pgfqpoint{3.174904in}{3.074669in}}%
\pgfpathlineto{\pgfqpoint{3.179445in}{3.074669in}}%
\pgfpathlineto{\pgfqpoint{3.179445in}{3.071720in}}%
\pgfpathmoveto{\pgfqpoint{3.174904in}{3.074669in}}%
\pgfpathlineto{\pgfqpoint{3.174904in}{3.074669in}}%
\pgfpathlineto{\pgfqpoint{3.174904in}{3.077618in}}%
\pgfpathlineto{\pgfqpoint{3.179445in}{3.077618in}}%
\pgfpathlineto{\pgfqpoint{3.179445in}{3.074669in}}%
\pgfpathmoveto{\pgfqpoint{3.179445in}{3.071720in}}%
\pgfpathlineto{\pgfqpoint{3.179445in}{3.071720in}}%
\pgfpathlineto{\pgfqpoint{3.179445in}{3.074669in}}%
\pgfpathlineto{\pgfqpoint{3.183986in}{3.074669in}}%
\pgfpathlineto{\pgfqpoint{3.183986in}{3.071720in}}%
\pgfpathmoveto{\pgfqpoint{3.179445in}{3.074669in}}%
\pgfpathlineto{\pgfqpoint{3.179445in}{3.074669in}}%
\pgfpathlineto{\pgfqpoint{3.179445in}{3.077618in}}%
\pgfpathlineto{\pgfqpoint{3.183986in}{3.077618in}}%
\pgfpathlineto{\pgfqpoint{3.183986in}{3.074669in}}%
\pgfpathmoveto{\pgfqpoint{3.174904in}{3.077618in}}%
\pgfpathlineto{\pgfqpoint{3.174904in}{3.077618in}}%
\pgfpathlineto{\pgfqpoint{3.174904in}{3.080567in}}%
\pgfpathlineto{\pgfqpoint{3.179445in}{3.080567in}}%
\pgfpathlineto{\pgfqpoint{3.179445in}{3.077618in}}%
\pgfpathmoveto{\pgfqpoint{3.174904in}{3.080567in}}%
\pgfpathlineto{\pgfqpoint{3.174904in}{3.080567in}}%
\pgfpathlineto{\pgfqpoint{3.174904in}{3.083517in}}%
\pgfpathlineto{\pgfqpoint{3.179445in}{3.083517in}}%
\pgfpathlineto{\pgfqpoint{3.179445in}{3.080567in}}%
\pgfpathmoveto{\pgfqpoint{3.179445in}{3.077618in}}%
\pgfpathlineto{\pgfqpoint{3.179445in}{3.077618in}}%
\pgfpathlineto{\pgfqpoint{3.179445in}{3.080567in}}%
\pgfpathlineto{\pgfqpoint{3.183986in}{3.080567in}}%
\pgfpathlineto{\pgfqpoint{3.183986in}{3.077618in}}%
\pgfpathmoveto{\pgfqpoint{3.179445in}{3.080567in}}%
\pgfpathlineto{\pgfqpoint{3.179445in}{3.080567in}}%
\pgfpathlineto{\pgfqpoint{3.179445in}{3.083517in}}%
\pgfpathlineto{\pgfqpoint{3.183986in}{3.083517in}}%
\pgfpathlineto{\pgfqpoint{3.183986in}{3.080567in}}%
\pgfpathmoveto{\pgfqpoint{3.165821in}{3.083517in}}%
\pgfpathlineto{\pgfqpoint{3.165821in}{3.083517in}}%
\pgfpathlineto{\pgfqpoint{3.165821in}{3.086466in}}%
\pgfpathlineto{\pgfqpoint{3.170363in}{3.086466in}}%
\pgfpathlineto{\pgfqpoint{3.170363in}{3.083517in}}%
\pgfpathmoveto{\pgfqpoint{3.165821in}{3.086466in}}%
\pgfpathlineto{\pgfqpoint{3.165821in}{3.086466in}}%
\pgfpathlineto{\pgfqpoint{3.165821in}{3.089415in}}%
\pgfpathlineto{\pgfqpoint{3.170363in}{3.089415in}}%
\pgfpathlineto{\pgfqpoint{3.170363in}{3.086466in}}%
\pgfpathmoveto{\pgfqpoint{3.170363in}{3.083517in}}%
\pgfpathlineto{\pgfqpoint{3.170363in}{3.083517in}}%
\pgfpathlineto{\pgfqpoint{3.170363in}{3.086466in}}%
\pgfpathlineto{\pgfqpoint{3.174904in}{3.086466in}}%
\pgfpathlineto{\pgfqpoint{3.174904in}{3.083517in}}%
\pgfpathmoveto{\pgfqpoint{3.170363in}{3.086466in}}%
\pgfpathlineto{\pgfqpoint{3.170363in}{3.086466in}}%
\pgfpathlineto{\pgfqpoint{3.170363in}{3.089415in}}%
\pgfpathlineto{\pgfqpoint{3.174904in}{3.089415in}}%
\pgfpathlineto{\pgfqpoint{3.174904in}{3.086466in}}%
\pgfpathmoveto{\pgfqpoint{3.165821in}{3.089415in}}%
\pgfpathlineto{\pgfqpoint{3.165821in}{3.089415in}}%
\pgfpathlineto{\pgfqpoint{3.165821in}{3.092364in}}%
\pgfpathlineto{\pgfqpoint{3.170363in}{3.092364in}}%
\pgfpathlineto{\pgfqpoint{3.170363in}{3.089415in}}%
\pgfpathmoveto{\pgfqpoint{3.165821in}{3.092364in}}%
\pgfpathlineto{\pgfqpoint{3.165821in}{3.092364in}}%
\pgfpathlineto{\pgfqpoint{3.165821in}{3.095314in}}%
\pgfpathlineto{\pgfqpoint{3.170363in}{3.095314in}}%
\pgfpathlineto{\pgfqpoint{3.170363in}{3.092364in}}%
\pgfpathmoveto{\pgfqpoint{3.170363in}{3.089415in}}%
\pgfpathlineto{\pgfqpoint{3.170363in}{3.089415in}}%
\pgfpathlineto{\pgfqpoint{3.170363in}{3.092364in}}%
\pgfpathlineto{\pgfqpoint{3.174904in}{3.092364in}}%
\pgfpathlineto{\pgfqpoint{3.174904in}{3.089415in}}%
\pgfpathmoveto{\pgfqpoint{3.170363in}{3.092364in}}%
\pgfpathlineto{\pgfqpoint{3.170363in}{3.092364in}}%
\pgfpathlineto{\pgfqpoint{3.170363in}{3.095314in}}%
\pgfpathlineto{\pgfqpoint{3.174904in}{3.095314in}}%
\pgfpathlineto{\pgfqpoint{3.174904in}{3.092364in}}%
\pgfpathmoveto{\pgfqpoint{3.174904in}{3.083517in}}%
\pgfpathlineto{\pgfqpoint{3.174904in}{3.083517in}}%
\pgfpathlineto{\pgfqpoint{3.174904in}{3.086466in}}%
\pgfpathlineto{\pgfqpoint{3.179445in}{3.086466in}}%
\pgfpathlineto{\pgfqpoint{3.179445in}{3.083517in}}%
\pgfpathmoveto{\pgfqpoint{3.174904in}{3.086466in}}%
\pgfpathlineto{\pgfqpoint{3.174904in}{3.086466in}}%
\pgfpathlineto{\pgfqpoint{3.174904in}{3.089415in}}%
\pgfpathlineto{\pgfqpoint{3.179445in}{3.089415in}}%
\pgfpathlineto{\pgfqpoint{3.179445in}{3.086466in}}%
\pgfpathmoveto{\pgfqpoint{3.193069in}{3.048126in}}%
\pgfpathlineto{\pgfqpoint{3.193069in}{3.048126in}}%
\pgfpathlineto{\pgfqpoint{3.193069in}{3.051075in}}%
\pgfpathlineto{\pgfqpoint{3.197610in}{3.051075in}}%
\pgfpathlineto{\pgfqpoint{3.197610in}{3.048126in}}%
\pgfpathmoveto{\pgfqpoint{3.193069in}{3.051075in}}%
\pgfpathlineto{\pgfqpoint{3.193069in}{3.051075in}}%
\pgfpathlineto{\pgfqpoint{3.193069in}{3.054024in}}%
\pgfpathlineto{\pgfqpoint{3.197610in}{3.054024in}}%
\pgfpathlineto{\pgfqpoint{3.197610in}{3.051075in}}%
\pgfpathmoveto{\pgfqpoint{3.197610in}{3.048126in}}%
\pgfpathlineto{\pgfqpoint{3.197610in}{3.048126in}}%
\pgfpathlineto{\pgfqpoint{3.197610in}{3.051075in}}%
\pgfpathlineto{\pgfqpoint{3.202151in}{3.051075in}}%
\pgfpathlineto{\pgfqpoint{3.202151in}{3.048126in}}%
\pgfpathmoveto{\pgfqpoint{3.197610in}{3.051075in}}%
\pgfpathlineto{\pgfqpoint{3.197610in}{3.051075in}}%
\pgfpathlineto{\pgfqpoint{3.197610in}{3.054024in}}%
\pgfpathlineto{\pgfqpoint{3.202151in}{3.054024in}}%
\pgfpathlineto{\pgfqpoint{3.202151in}{3.051075in}}%
\pgfpathmoveto{\pgfqpoint{3.193069in}{3.054024in}}%
\pgfpathlineto{\pgfqpoint{3.193069in}{3.054024in}}%
\pgfpathlineto{\pgfqpoint{3.193069in}{3.056974in}}%
\pgfpathlineto{\pgfqpoint{3.197610in}{3.056974in}}%
\pgfpathlineto{\pgfqpoint{3.197610in}{3.054024in}}%
\pgfpathmoveto{\pgfqpoint{3.193069in}{3.056974in}}%
\pgfpathlineto{\pgfqpoint{3.193069in}{3.056974in}}%
\pgfpathlineto{\pgfqpoint{3.193069in}{3.059923in}}%
\pgfpathlineto{\pgfqpoint{3.197610in}{3.059923in}}%
\pgfpathlineto{\pgfqpoint{3.197610in}{3.056974in}}%
\pgfpathmoveto{\pgfqpoint{3.197610in}{3.054024in}}%
\pgfpathlineto{\pgfqpoint{3.197610in}{3.054024in}}%
\pgfpathlineto{\pgfqpoint{3.197610in}{3.056974in}}%
\pgfpathlineto{\pgfqpoint{3.202151in}{3.056974in}}%
\pgfpathlineto{\pgfqpoint{3.202151in}{3.054024in}}%
\pgfpathmoveto{\pgfqpoint{3.197610in}{3.056974in}}%
\pgfpathlineto{\pgfqpoint{3.197610in}{3.056974in}}%
\pgfpathlineto{\pgfqpoint{3.197610in}{3.059923in}}%
\pgfpathlineto{\pgfqpoint{3.202151in}{3.059923in}}%
\pgfpathlineto{\pgfqpoint{3.202151in}{3.056974in}}%
\pgfpathmoveto{\pgfqpoint{3.183986in}{3.059923in}}%
\pgfpathlineto{\pgfqpoint{3.183986in}{3.059923in}}%
\pgfpathlineto{\pgfqpoint{3.183986in}{3.062872in}}%
\pgfpathlineto{\pgfqpoint{3.188528in}{3.062872in}}%
\pgfpathlineto{\pgfqpoint{3.188528in}{3.059923in}}%
\pgfpathmoveto{\pgfqpoint{3.183986in}{3.062872in}}%
\pgfpathlineto{\pgfqpoint{3.183986in}{3.062872in}}%
\pgfpathlineto{\pgfqpoint{3.183986in}{3.065821in}}%
\pgfpathlineto{\pgfqpoint{3.188528in}{3.065821in}}%
\pgfpathlineto{\pgfqpoint{3.188528in}{3.062872in}}%
\pgfpathmoveto{\pgfqpoint{3.188528in}{3.059923in}}%
\pgfpathlineto{\pgfqpoint{3.188528in}{3.059923in}}%
\pgfpathlineto{\pgfqpoint{3.188528in}{3.062872in}}%
\pgfpathlineto{\pgfqpoint{3.193069in}{3.062872in}}%
\pgfpathlineto{\pgfqpoint{3.193069in}{3.059923in}}%
\pgfpathmoveto{\pgfqpoint{3.188528in}{3.062872in}}%
\pgfpathlineto{\pgfqpoint{3.188528in}{3.062872in}}%
\pgfpathlineto{\pgfqpoint{3.188528in}{3.065821in}}%
\pgfpathlineto{\pgfqpoint{3.193069in}{3.065821in}}%
\pgfpathlineto{\pgfqpoint{3.193069in}{3.062872in}}%
\pgfpathmoveto{\pgfqpoint{3.183986in}{3.065821in}}%
\pgfpathlineto{\pgfqpoint{3.183986in}{3.065821in}}%
\pgfpathlineto{\pgfqpoint{3.183986in}{3.068771in}}%
\pgfpathlineto{\pgfqpoint{3.188528in}{3.068771in}}%
\pgfpathlineto{\pgfqpoint{3.188528in}{3.065821in}}%
\pgfpathmoveto{\pgfqpoint{3.183986in}{3.068771in}}%
\pgfpathlineto{\pgfqpoint{3.183986in}{3.068771in}}%
\pgfpathlineto{\pgfqpoint{3.183986in}{3.071720in}}%
\pgfpathlineto{\pgfqpoint{3.188528in}{3.071720in}}%
\pgfpathlineto{\pgfqpoint{3.188528in}{3.068771in}}%
\pgfpathmoveto{\pgfqpoint{3.188528in}{3.065821in}}%
\pgfpathlineto{\pgfqpoint{3.188528in}{3.065821in}}%
\pgfpathlineto{\pgfqpoint{3.188528in}{3.068771in}}%
\pgfpathlineto{\pgfqpoint{3.193069in}{3.068771in}}%
\pgfpathlineto{\pgfqpoint{3.193069in}{3.065821in}}%
\pgfpathmoveto{\pgfqpoint{3.188528in}{3.068771in}}%
\pgfpathlineto{\pgfqpoint{3.188528in}{3.068771in}}%
\pgfpathlineto{\pgfqpoint{3.188528in}{3.071720in}}%
\pgfpathlineto{\pgfqpoint{3.193069in}{3.071720in}}%
\pgfpathlineto{\pgfqpoint{3.193069in}{3.068771in}}%
\pgfpathmoveto{\pgfqpoint{3.193069in}{3.059923in}}%
\pgfpathlineto{\pgfqpoint{3.193069in}{3.059923in}}%
\pgfpathlineto{\pgfqpoint{3.193069in}{3.062872in}}%
\pgfpathlineto{\pgfqpoint{3.197610in}{3.062872in}}%
\pgfpathlineto{\pgfqpoint{3.197610in}{3.059923in}}%
\pgfpathmoveto{\pgfqpoint{3.193069in}{3.062872in}}%
\pgfpathlineto{\pgfqpoint{3.193069in}{3.062872in}}%
\pgfpathlineto{\pgfqpoint{3.193069in}{3.065821in}}%
\pgfpathlineto{\pgfqpoint{3.197610in}{3.065821in}}%
\pgfpathlineto{\pgfqpoint{3.197610in}{3.062872in}}%
\pgfpathmoveto{\pgfqpoint{3.202151in}{3.048126in}}%
\pgfpathlineto{\pgfqpoint{3.202151in}{3.048126in}}%
\pgfpathlineto{\pgfqpoint{3.202151in}{3.051075in}}%
\pgfpathlineto{\pgfqpoint{3.206693in}{3.051075in}}%
\pgfpathlineto{\pgfqpoint{3.206693in}{3.048126in}}%
\pgfpathmoveto{\pgfqpoint{3.202151in}{3.051075in}}%
\pgfpathlineto{\pgfqpoint{3.202151in}{3.051075in}}%
\pgfpathlineto{\pgfqpoint{3.202151in}{3.054024in}}%
\pgfpathlineto{\pgfqpoint{3.206693in}{3.054024in}}%
\pgfpathlineto{\pgfqpoint{3.206693in}{3.051075in}}%
\pgfpathmoveto{\pgfqpoint{3.183986in}{3.071720in}}%
\pgfpathlineto{\pgfqpoint{3.183986in}{3.071720in}}%
\pgfpathlineto{\pgfqpoint{3.183986in}{3.074669in}}%
\pgfpathlineto{\pgfqpoint{3.188528in}{3.074669in}}%
\pgfpathlineto{\pgfqpoint{3.188528in}{3.071720in}}%
\pgfpathmoveto{\pgfqpoint{3.183986in}{3.074669in}}%
\pgfpathlineto{\pgfqpoint{3.183986in}{3.074669in}}%
\pgfpathlineto{\pgfqpoint{3.183986in}{3.077618in}}%
\pgfpathlineto{\pgfqpoint{3.188528in}{3.077618in}}%
\pgfpathlineto{\pgfqpoint{3.188528in}{3.074669in}}%
\pgfpathmoveto{\pgfqpoint{3.156739in}{3.095314in}}%
\pgfpathlineto{\pgfqpoint{3.156739in}{3.095314in}}%
\pgfpathlineto{\pgfqpoint{3.156739in}{3.098263in}}%
\pgfpathlineto{\pgfqpoint{3.161280in}{3.098263in}}%
\pgfpathlineto{\pgfqpoint{3.161280in}{3.095314in}}%
\pgfpathmoveto{\pgfqpoint{3.156739in}{3.098263in}}%
\pgfpathlineto{\pgfqpoint{3.156739in}{3.098263in}}%
\pgfpathlineto{\pgfqpoint{3.156739in}{3.101212in}}%
\pgfpathlineto{\pgfqpoint{3.161280in}{3.101212in}}%
\pgfpathlineto{\pgfqpoint{3.161280in}{3.098263in}}%
\pgfpathmoveto{\pgfqpoint{3.161280in}{3.095314in}}%
\pgfpathlineto{\pgfqpoint{3.161280in}{3.095314in}}%
\pgfpathlineto{\pgfqpoint{3.161280in}{3.098263in}}%
\pgfpathlineto{\pgfqpoint{3.165821in}{3.098263in}}%
\pgfpathlineto{\pgfqpoint{3.165821in}{3.095314in}}%
\pgfpathmoveto{\pgfqpoint{3.161280in}{3.098263in}}%
\pgfpathlineto{\pgfqpoint{3.161280in}{3.098263in}}%
\pgfpathlineto{\pgfqpoint{3.161280in}{3.101212in}}%
\pgfpathlineto{\pgfqpoint{3.165821in}{3.101212in}}%
\pgfpathlineto{\pgfqpoint{3.165821in}{3.098263in}}%
\pgfpathmoveto{\pgfqpoint{3.156739in}{3.101212in}}%
\pgfpathlineto{\pgfqpoint{3.156739in}{3.101212in}}%
\pgfpathlineto{\pgfqpoint{3.156739in}{3.104161in}}%
\pgfpathlineto{\pgfqpoint{3.161280in}{3.104161in}}%
\pgfpathlineto{\pgfqpoint{3.161280in}{3.101212in}}%
\pgfpathmoveto{\pgfqpoint{3.156739in}{3.104161in}}%
\pgfpathlineto{\pgfqpoint{3.156739in}{3.104161in}}%
\pgfpathlineto{\pgfqpoint{3.156739in}{3.107110in}}%
\pgfpathlineto{\pgfqpoint{3.161280in}{3.107110in}}%
\pgfpathlineto{\pgfqpoint{3.161280in}{3.104161in}}%
\pgfpathmoveto{\pgfqpoint{3.161280in}{3.101212in}}%
\pgfpathlineto{\pgfqpoint{3.161280in}{3.101212in}}%
\pgfpathlineto{\pgfqpoint{3.161280in}{3.104161in}}%
\pgfpathlineto{\pgfqpoint{3.165821in}{3.104161in}}%
\pgfpathlineto{\pgfqpoint{3.165821in}{3.101212in}}%
\pgfpathmoveto{\pgfqpoint{3.161280in}{3.104161in}}%
\pgfpathlineto{\pgfqpoint{3.161280in}{3.104161in}}%
\pgfpathlineto{\pgfqpoint{3.161280in}{3.107110in}}%
\pgfpathlineto{\pgfqpoint{3.165821in}{3.107110in}}%
\pgfpathlineto{\pgfqpoint{3.165821in}{3.104161in}}%
\pgfpathmoveto{\pgfqpoint{3.147656in}{3.107110in}}%
\pgfpathlineto{\pgfqpoint{3.147656in}{3.107110in}}%
\pgfpathlineto{\pgfqpoint{3.147656in}{3.110060in}}%
\pgfpathlineto{\pgfqpoint{3.152198in}{3.110060in}}%
\pgfpathlineto{\pgfqpoint{3.152198in}{3.107110in}}%
\pgfpathmoveto{\pgfqpoint{3.147656in}{3.110060in}}%
\pgfpathlineto{\pgfqpoint{3.147656in}{3.110060in}}%
\pgfpathlineto{\pgfqpoint{3.147656in}{3.113009in}}%
\pgfpathlineto{\pgfqpoint{3.152198in}{3.113009in}}%
\pgfpathlineto{\pgfqpoint{3.152198in}{3.110060in}}%
\pgfpathmoveto{\pgfqpoint{3.152198in}{3.107110in}}%
\pgfpathlineto{\pgfqpoint{3.152198in}{3.107110in}}%
\pgfpathlineto{\pgfqpoint{3.152198in}{3.110060in}}%
\pgfpathlineto{\pgfqpoint{3.156739in}{3.110060in}}%
\pgfpathlineto{\pgfqpoint{3.156739in}{3.107110in}}%
\pgfpathmoveto{\pgfqpoint{3.152198in}{3.110060in}}%
\pgfpathlineto{\pgfqpoint{3.152198in}{3.110060in}}%
\pgfpathlineto{\pgfqpoint{3.152198in}{3.113009in}}%
\pgfpathlineto{\pgfqpoint{3.156739in}{3.113009in}}%
\pgfpathlineto{\pgfqpoint{3.156739in}{3.110060in}}%
\pgfpathmoveto{\pgfqpoint{3.147656in}{3.113009in}}%
\pgfpathlineto{\pgfqpoint{3.147656in}{3.113009in}}%
\pgfpathlineto{\pgfqpoint{3.147656in}{3.115958in}}%
\pgfpathlineto{\pgfqpoint{3.152198in}{3.115958in}}%
\pgfpathlineto{\pgfqpoint{3.152198in}{3.113009in}}%
\pgfpathmoveto{\pgfqpoint{3.147656in}{3.115958in}}%
\pgfpathlineto{\pgfqpoint{3.147656in}{3.115958in}}%
\pgfpathlineto{\pgfqpoint{3.147656in}{3.118907in}}%
\pgfpathlineto{\pgfqpoint{3.152198in}{3.118907in}}%
\pgfpathlineto{\pgfqpoint{3.152198in}{3.115958in}}%
\pgfpathmoveto{\pgfqpoint{3.152198in}{3.113009in}}%
\pgfpathlineto{\pgfqpoint{3.152198in}{3.113009in}}%
\pgfpathlineto{\pgfqpoint{3.152198in}{3.115958in}}%
\pgfpathlineto{\pgfqpoint{3.156739in}{3.115958in}}%
\pgfpathlineto{\pgfqpoint{3.156739in}{3.113009in}}%
\pgfpathmoveto{\pgfqpoint{3.152198in}{3.115958in}}%
\pgfpathlineto{\pgfqpoint{3.152198in}{3.115958in}}%
\pgfpathlineto{\pgfqpoint{3.152198in}{3.118907in}}%
\pgfpathlineto{\pgfqpoint{3.156739in}{3.118907in}}%
\pgfpathlineto{\pgfqpoint{3.156739in}{3.115958in}}%
\pgfpathmoveto{\pgfqpoint{3.156739in}{3.107110in}}%
\pgfpathlineto{\pgfqpoint{3.156739in}{3.107110in}}%
\pgfpathlineto{\pgfqpoint{3.156739in}{3.110060in}}%
\pgfpathlineto{\pgfqpoint{3.161280in}{3.110060in}}%
\pgfpathlineto{\pgfqpoint{3.161280in}{3.107110in}}%
\pgfpathmoveto{\pgfqpoint{3.156739in}{3.110060in}}%
\pgfpathlineto{\pgfqpoint{3.156739in}{3.110060in}}%
\pgfpathlineto{\pgfqpoint{3.156739in}{3.113009in}}%
\pgfpathlineto{\pgfqpoint{3.161280in}{3.113009in}}%
\pgfpathlineto{\pgfqpoint{3.161280in}{3.110060in}}%
\pgfpathmoveto{\pgfqpoint{3.165821in}{3.095314in}}%
\pgfpathlineto{\pgfqpoint{3.165821in}{3.095314in}}%
\pgfpathlineto{\pgfqpoint{3.165821in}{3.098263in}}%
\pgfpathlineto{\pgfqpoint{3.170363in}{3.098263in}}%
\pgfpathlineto{\pgfqpoint{3.170363in}{3.095314in}}%
\pgfpathmoveto{\pgfqpoint{3.165821in}{3.098263in}}%
\pgfpathlineto{\pgfqpoint{3.165821in}{3.098263in}}%
\pgfpathlineto{\pgfqpoint{3.165821in}{3.101212in}}%
\pgfpathlineto{\pgfqpoint{3.170363in}{3.101212in}}%
\pgfpathlineto{\pgfqpoint{3.170363in}{3.098263in}}%
\pgfpathmoveto{\pgfqpoint{3.147656in}{3.118907in}}%
\pgfpathlineto{\pgfqpoint{3.147656in}{3.118907in}}%
\pgfpathlineto{\pgfqpoint{3.147656in}{3.121857in}}%
\pgfpathlineto{\pgfqpoint{3.152198in}{3.121857in}}%
\pgfpathlineto{\pgfqpoint{3.152198in}{3.118907in}}%
\pgfpathmoveto{\pgfqpoint{3.147656in}{3.121857in}}%
\pgfpathlineto{\pgfqpoint{3.147656in}{3.121857in}}%
\pgfpathlineto{\pgfqpoint{3.147656in}{3.124806in}}%
\pgfpathlineto{\pgfqpoint{3.152198in}{3.124806in}}%
\pgfpathlineto{\pgfqpoint{3.152198in}{3.121857in}}%
\pgfpathmoveto{\pgfqpoint{3.102244in}{3.166094in}}%
\pgfpathlineto{\pgfqpoint{3.102244in}{3.166094in}}%
\pgfpathlineto{\pgfqpoint{3.102244in}{3.169043in}}%
\pgfpathlineto{\pgfqpoint{3.106785in}{3.169043in}}%
\pgfpathlineto{\pgfqpoint{3.106785in}{3.166094in}}%
\pgfpathmoveto{\pgfqpoint{3.102244in}{3.169043in}}%
\pgfpathlineto{\pgfqpoint{3.102244in}{3.169043in}}%
\pgfpathlineto{\pgfqpoint{3.102244in}{3.171992in}}%
\pgfpathlineto{\pgfqpoint{3.106785in}{3.171992in}}%
\pgfpathlineto{\pgfqpoint{3.106785in}{3.169043in}}%
\pgfpathmoveto{\pgfqpoint{3.106785in}{3.166094in}}%
\pgfpathlineto{\pgfqpoint{3.106785in}{3.166094in}}%
\pgfpathlineto{\pgfqpoint{3.106785in}{3.169043in}}%
\pgfpathlineto{\pgfqpoint{3.111326in}{3.169043in}}%
\pgfpathlineto{\pgfqpoint{3.111326in}{3.166094in}}%
\pgfpathmoveto{\pgfqpoint{3.106785in}{3.169043in}}%
\pgfpathlineto{\pgfqpoint{3.106785in}{3.169043in}}%
\pgfpathlineto{\pgfqpoint{3.106785in}{3.171992in}}%
\pgfpathlineto{\pgfqpoint{3.111326in}{3.171992in}}%
\pgfpathlineto{\pgfqpoint{3.111326in}{3.169043in}}%
\pgfpathmoveto{\pgfqpoint{3.102244in}{3.171992in}}%
\pgfpathlineto{\pgfqpoint{3.102244in}{3.171992in}}%
\pgfpathlineto{\pgfqpoint{3.102244in}{3.174942in}}%
\pgfpathlineto{\pgfqpoint{3.106785in}{3.174942in}}%
\pgfpathlineto{\pgfqpoint{3.106785in}{3.171992in}}%
\pgfpathmoveto{\pgfqpoint{3.102244in}{3.174942in}}%
\pgfpathlineto{\pgfqpoint{3.102244in}{3.174942in}}%
\pgfpathlineto{\pgfqpoint{3.102244in}{3.177891in}}%
\pgfpathlineto{\pgfqpoint{3.106785in}{3.177891in}}%
\pgfpathlineto{\pgfqpoint{3.106785in}{3.174942in}}%
\pgfpathmoveto{\pgfqpoint{3.106785in}{3.171992in}}%
\pgfpathlineto{\pgfqpoint{3.106785in}{3.171992in}}%
\pgfpathlineto{\pgfqpoint{3.106785in}{3.174942in}}%
\pgfpathlineto{\pgfqpoint{3.111326in}{3.174942in}}%
\pgfpathlineto{\pgfqpoint{3.111326in}{3.171992in}}%
\pgfpathmoveto{\pgfqpoint{3.106785in}{3.174942in}}%
\pgfpathlineto{\pgfqpoint{3.106785in}{3.174942in}}%
\pgfpathlineto{\pgfqpoint{3.106785in}{3.177891in}}%
\pgfpathlineto{\pgfqpoint{3.111326in}{3.177891in}}%
\pgfpathlineto{\pgfqpoint{3.111326in}{3.174942in}}%
\pgfpathmoveto{\pgfqpoint{3.093161in}{3.177891in}}%
\pgfpathlineto{\pgfqpoint{3.093161in}{3.177891in}}%
\pgfpathlineto{\pgfqpoint{3.093161in}{3.180840in}}%
\pgfpathlineto{\pgfqpoint{3.097703in}{3.180840in}}%
\pgfpathlineto{\pgfqpoint{3.097703in}{3.177891in}}%
\pgfpathmoveto{\pgfqpoint{3.093161in}{3.180840in}}%
\pgfpathlineto{\pgfqpoint{3.093161in}{3.180840in}}%
\pgfpathlineto{\pgfqpoint{3.093161in}{3.183789in}}%
\pgfpathlineto{\pgfqpoint{3.097703in}{3.183789in}}%
\pgfpathlineto{\pgfqpoint{3.097703in}{3.180840in}}%
\pgfpathmoveto{\pgfqpoint{3.097703in}{3.177891in}}%
\pgfpathlineto{\pgfqpoint{3.097703in}{3.177891in}}%
\pgfpathlineto{\pgfqpoint{3.097703in}{3.180840in}}%
\pgfpathlineto{\pgfqpoint{3.102244in}{3.180840in}}%
\pgfpathlineto{\pgfqpoint{3.102244in}{3.177891in}}%
\pgfpathmoveto{\pgfqpoint{3.097703in}{3.180840in}}%
\pgfpathlineto{\pgfqpoint{3.097703in}{3.180840in}}%
\pgfpathlineto{\pgfqpoint{3.097703in}{3.183789in}}%
\pgfpathlineto{\pgfqpoint{3.102244in}{3.183789in}}%
\pgfpathlineto{\pgfqpoint{3.102244in}{3.180840in}}%
\pgfpathmoveto{\pgfqpoint{3.093161in}{3.183789in}}%
\pgfpathlineto{\pgfqpoint{3.093161in}{3.183789in}}%
\pgfpathlineto{\pgfqpoint{3.093161in}{3.186738in}}%
\pgfpathlineto{\pgfqpoint{3.097703in}{3.186738in}}%
\pgfpathlineto{\pgfqpoint{3.097703in}{3.183789in}}%
\pgfpathmoveto{\pgfqpoint{3.093161in}{3.186738in}}%
\pgfpathlineto{\pgfqpoint{3.093161in}{3.186738in}}%
\pgfpathlineto{\pgfqpoint{3.093161in}{3.189687in}}%
\pgfpathlineto{\pgfqpoint{3.097703in}{3.189687in}}%
\pgfpathlineto{\pgfqpoint{3.097703in}{3.186738in}}%
\pgfpathmoveto{\pgfqpoint{3.097703in}{3.183789in}}%
\pgfpathlineto{\pgfqpoint{3.097703in}{3.183789in}}%
\pgfpathlineto{\pgfqpoint{3.097703in}{3.186738in}}%
\pgfpathlineto{\pgfqpoint{3.102244in}{3.186738in}}%
\pgfpathlineto{\pgfqpoint{3.102244in}{3.183789in}}%
\pgfpathmoveto{\pgfqpoint{3.097703in}{3.186738in}}%
\pgfpathlineto{\pgfqpoint{3.097703in}{3.186738in}}%
\pgfpathlineto{\pgfqpoint{3.097703in}{3.189687in}}%
\pgfpathlineto{\pgfqpoint{3.102244in}{3.189687in}}%
\pgfpathlineto{\pgfqpoint{3.102244in}{3.186738in}}%
\pgfpathmoveto{\pgfqpoint{3.102244in}{3.177891in}}%
\pgfpathlineto{\pgfqpoint{3.102244in}{3.177891in}}%
\pgfpathlineto{\pgfqpoint{3.102244in}{3.180840in}}%
\pgfpathlineto{\pgfqpoint{3.106785in}{3.180840in}}%
\pgfpathlineto{\pgfqpoint{3.106785in}{3.177891in}}%
\pgfpathmoveto{\pgfqpoint{3.102244in}{3.180840in}}%
\pgfpathlineto{\pgfqpoint{3.102244in}{3.180840in}}%
\pgfpathlineto{\pgfqpoint{3.102244in}{3.183789in}}%
\pgfpathlineto{\pgfqpoint{3.106785in}{3.183789in}}%
\pgfpathlineto{\pgfqpoint{3.106785in}{3.180840in}}%
\pgfpathmoveto{\pgfqpoint{3.120409in}{3.142501in}}%
\pgfpathlineto{\pgfqpoint{3.120409in}{3.142501in}}%
\pgfpathlineto{\pgfqpoint{3.120409in}{3.145450in}}%
\pgfpathlineto{\pgfqpoint{3.124950in}{3.145450in}}%
\pgfpathlineto{\pgfqpoint{3.124950in}{3.142501in}}%
\pgfpathmoveto{\pgfqpoint{3.120409in}{3.145450in}}%
\pgfpathlineto{\pgfqpoint{3.120409in}{3.145450in}}%
\pgfpathlineto{\pgfqpoint{3.120409in}{3.148399in}}%
\pgfpathlineto{\pgfqpoint{3.124950in}{3.148399in}}%
\pgfpathlineto{\pgfqpoint{3.124950in}{3.145450in}}%
\pgfpathmoveto{\pgfqpoint{3.124950in}{3.142501in}}%
\pgfpathlineto{\pgfqpoint{3.124950in}{3.142501in}}%
\pgfpathlineto{\pgfqpoint{3.124950in}{3.145450in}}%
\pgfpathlineto{\pgfqpoint{3.129491in}{3.145450in}}%
\pgfpathlineto{\pgfqpoint{3.129491in}{3.142501in}}%
\pgfpathmoveto{\pgfqpoint{3.124950in}{3.145450in}}%
\pgfpathlineto{\pgfqpoint{3.124950in}{3.145450in}}%
\pgfpathlineto{\pgfqpoint{3.124950in}{3.148399in}}%
\pgfpathlineto{\pgfqpoint{3.129491in}{3.148399in}}%
\pgfpathlineto{\pgfqpoint{3.129491in}{3.145450in}}%
\pgfpathmoveto{\pgfqpoint{3.120409in}{3.148399in}}%
\pgfpathlineto{\pgfqpoint{3.120409in}{3.148399in}}%
\pgfpathlineto{\pgfqpoint{3.120409in}{3.151349in}}%
\pgfpathlineto{\pgfqpoint{3.124950in}{3.151349in}}%
\pgfpathlineto{\pgfqpoint{3.124950in}{3.148399in}}%
\pgfpathmoveto{\pgfqpoint{3.120409in}{3.151349in}}%
\pgfpathlineto{\pgfqpoint{3.120409in}{3.151349in}}%
\pgfpathlineto{\pgfqpoint{3.120409in}{3.154298in}}%
\pgfpathlineto{\pgfqpoint{3.124950in}{3.154298in}}%
\pgfpathlineto{\pgfqpoint{3.124950in}{3.151349in}}%
\pgfpathmoveto{\pgfqpoint{3.124950in}{3.148399in}}%
\pgfpathlineto{\pgfqpoint{3.124950in}{3.148399in}}%
\pgfpathlineto{\pgfqpoint{3.124950in}{3.151349in}}%
\pgfpathlineto{\pgfqpoint{3.129491in}{3.151349in}}%
\pgfpathlineto{\pgfqpoint{3.129491in}{3.148399in}}%
\pgfpathmoveto{\pgfqpoint{3.124950in}{3.151349in}}%
\pgfpathlineto{\pgfqpoint{3.124950in}{3.151349in}}%
\pgfpathlineto{\pgfqpoint{3.124950in}{3.154298in}}%
\pgfpathlineto{\pgfqpoint{3.129491in}{3.154298in}}%
\pgfpathlineto{\pgfqpoint{3.129491in}{3.151349in}}%
\pgfpathmoveto{\pgfqpoint{3.111326in}{3.154298in}}%
\pgfpathlineto{\pgfqpoint{3.111326in}{3.154298in}}%
\pgfpathlineto{\pgfqpoint{3.111326in}{3.157247in}}%
\pgfpathlineto{\pgfqpoint{3.115868in}{3.157247in}}%
\pgfpathlineto{\pgfqpoint{3.115868in}{3.154298in}}%
\pgfpathmoveto{\pgfqpoint{3.111326in}{3.157247in}}%
\pgfpathlineto{\pgfqpoint{3.111326in}{3.157247in}}%
\pgfpathlineto{\pgfqpoint{3.111326in}{3.160196in}}%
\pgfpathlineto{\pgfqpoint{3.115868in}{3.160196in}}%
\pgfpathlineto{\pgfqpoint{3.115868in}{3.157247in}}%
\pgfpathmoveto{\pgfqpoint{3.115868in}{3.154298in}}%
\pgfpathlineto{\pgfqpoint{3.115868in}{3.154298in}}%
\pgfpathlineto{\pgfqpoint{3.115868in}{3.157247in}}%
\pgfpathlineto{\pgfqpoint{3.120409in}{3.157247in}}%
\pgfpathlineto{\pgfqpoint{3.120409in}{3.154298in}}%
\pgfpathmoveto{\pgfqpoint{3.115868in}{3.157247in}}%
\pgfpathlineto{\pgfqpoint{3.115868in}{3.157247in}}%
\pgfpathlineto{\pgfqpoint{3.115868in}{3.160196in}}%
\pgfpathlineto{\pgfqpoint{3.120409in}{3.160196in}}%
\pgfpathlineto{\pgfqpoint{3.120409in}{3.157247in}}%
\pgfpathmoveto{\pgfqpoint{3.111326in}{3.160196in}}%
\pgfpathlineto{\pgfqpoint{3.111326in}{3.160196in}}%
\pgfpathlineto{\pgfqpoint{3.111326in}{3.163145in}}%
\pgfpathlineto{\pgfqpoint{3.115868in}{3.163145in}}%
\pgfpathlineto{\pgfqpoint{3.115868in}{3.160196in}}%
\pgfpathmoveto{\pgfqpoint{3.111326in}{3.163145in}}%
\pgfpathlineto{\pgfqpoint{3.111326in}{3.163145in}}%
\pgfpathlineto{\pgfqpoint{3.111326in}{3.166094in}}%
\pgfpathlineto{\pgfqpoint{3.115868in}{3.166094in}}%
\pgfpathlineto{\pgfqpoint{3.115868in}{3.163145in}}%
\pgfpathmoveto{\pgfqpoint{3.115868in}{3.160196in}}%
\pgfpathlineto{\pgfqpoint{3.115868in}{3.160196in}}%
\pgfpathlineto{\pgfqpoint{3.115868in}{3.163145in}}%
\pgfpathlineto{\pgfqpoint{3.120409in}{3.163145in}}%
\pgfpathlineto{\pgfqpoint{3.120409in}{3.160196in}}%
\pgfpathmoveto{\pgfqpoint{3.115868in}{3.163145in}}%
\pgfpathlineto{\pgfqpoint{3.115868in}{3.163145in}}%
\pgfpathlineto{\pgfqpoint{3.115868in}{3.166094in}}%
\pgfpathlineto{\pgfqpoint{3.120409in}{3.166094in}}%
\pgfpathlineto{\pgfqpoint{3.120409in}{3.163145in}}%
\pgfpathmoveto{\pgfqpoint{3.120409in}{3.154298in}}%
\pgfpathlineto{\pgfqpoint{3.120409in}{3.154298in}}%
\pgfpathlineto{\pgfqpoint{3.120409in}{3.157247in}}%
\pgfpathlineto{\pgfqpoint{3.124950in}{3.157247in}}%
\pgfpathlineto{\pgfqpoint{3.124950in}{3.154298in}}%
\pgfpathmoveto{\pgfqpoint{3.120409in}{3.157247in}}%
\pgfpathlineto{\pgfqpoint{3.120409in}{3.157247in}}%
\pgfpathlineto{\pgfqpoint{3.120409in}{3.160196in}}%
\pgfpathlineto{\pgfqpoint{3.124950in}{3.160196in}}%
\pgfpathlineto{\pgfqpoint{3.124950in}{3.157247in}}%
\pgfpathmoveto{\pgfqpoint{3.129491in}{3.142501in}}%
\pgfpathlineto{\pgfqpoint{3.129491in}{3.142501in}}%
\pgfpathlineto{\pgfqpoint{3.129491in}{3.145450in}}%
\pgfpathlineto{\pgfqpoint{3.134033in}{3.145450in}}%
\pgfpathlineto{\pgfqpoint{3.134033in}{3.142501in}}%
\pgfpathmoveto{\pgfqpoint{3.129491in}{3.145450in}}%
\pgfpathlineto{\pgfqpoint{3.129491in}{3.145450in}}%
\pgfpathlineto{\pgfqpoint{3.129491in}{3.148399in}}%
\pgfpathlineto{\pgfqpoint{3.134033in}{3.148399in}}%
\pgfpathlineto{\pgfqpoint{3.134033in}{3.145450in}}%
\pgfpathmoveto{\pgfqpoint{3.111326in}{3.166094in}}%
\pgfpathlineto{\pgfqpoint{3.111326in}{3.166094in}}%
\pgfpathlineto{\pgfqpoint{3.111326in}{3.169043in}}%
\pgfpathlineto{\pgfqpoint{3.115868in}{3.169043in}}%
\pgfpathlineto{\pgfqpoint{3.115868in}{3.166094in}}%
\pgfpathmoveto{\pgfqpoint{3.111326in}{3.169043in}}%
\pgfpathlineto{\pgfqpoint{3.111326in}{3.169043in}}%
\pgfpathlineto{\pgfqpoint{3.111326in}{3.171992in}}%
\pgfpathlineto{\pgfqpoint{3.115868in}{3.171992in}}%
\pgfpathlineto{\pgfqpoint{3.115868in}{3.169043in}}%
\pgfpathmoveto{\pgfqpoint{3.084079in}{3.189687in}}%
\pgfpathlineto{\pgfqpoint{3.084079in}{3.189687in}}%
\pgfpathlineto{\pgfqpoint{3.084079in}{3.192636in}}%
\pgfpathlineto{\pgfqpoint{3.088620in}{3.192636in}}%
\pgfpathlineto{\pgfqpoint{3.088620in}{3.189687in}}%
\pgfpathmoveto{\pgfqpoint{3.084079in}{3.192636in}}%
\pgfpathlineto{\pgfqpoint{3.084079in}{3.192636in}}%
\pgfpathlineto{\pgfqpoint{3.084079in}{3.195585in}}%
\pgfpathlineto{\pgfqpoint{3.088620in}{3.195585in}}%
\pgfpathlineto{\pgfqpoint{3.088620in}{3.192636in}}%
\pgfpathmoveto{\pgfqpoint{3.088620in}{3.189687in}}%
\pgfpathlineto{\pgfqpoint{3.088620in}{3.189687in}}%
\pgfpathlineto{\pgfqpoint{3.088620in}{3.192636in}}%
\pgfpathlineto{\pgfqpoint{3.093161in}{3.192636in}}%
\pgfpathlineto{\pgfqpoint{3.093161in}{3.189687in}}%
\pgfpathmoveto{\pgfqpoint{3.088620in}{3.192636in}}%
\pgfpathlineto{\pgfqpoint{3.088620in}{3.192636in}}%
\pgfpathlineto{\pgfqpoint{3.088620in}{3.195585in}}%
\pgfpathlineto{\pgfqpoint{3.093161in}{3.195585in}}%
\pgfpathlineto{\pgfqpoint{3.093161in}{3.192636in}}%
\pgfpathmoveto{\pgfqpoint{3.084079in}{3.195585in}}%
\pgfpathlineto{\pgfqpoint{3.084079in}{3.195585in}}%
\pgfpathlineto{\pgfqpoint{3.084079in}{3.198534in}}%
\pgfpathlineto{\pgfqpoint{3.088620in}{3.198534in}}%
\pgfpathlineto{\pgfqpoint{3.088620in}{3.195585in}}%
\pgfpathmoveto{\pgfqpoint{3.084079in}{3.198534in}}%
\pgfpathlineto{\pgfqpoint{3.084079in}{3.198534in}}%
\pgfpathlineto{\pgfqpoint{3.084079in}{3.201484in}}%
\pgfpathlineto{\pgfqpoint{3.088620in}{3.201484in}}%
\pgfpathlineto{\pgfqpoint{3.088620in}{3.198534in}}%
\pgfpathmoveto{\pgfqpoint{3.088620in}{3.195585in}}%
\pgfpathlineto{\pgfqpoint{3.088620in}{3.195585in}}%
\pgfpathlineto{\pgfqpoint{3.088620in}{3.198534in}}%
\pgfpathlineto{\pgfqpoint{3.093161in}{3.198534in}}%
\pgfpathlineto{\pgfqpoint{3.093161in}{3.195585in}}%
\pgfpathmoveto{\pgfqpoint{3.088620in}{3.198534in}}%
\pgfpathlineto{\pgfqpoint{3.088620in}{3.198534in}}%
\pgfpathlineto{\pgfqpoint{3.088620in}{3.201484in}}%
\pgfpathlineto{\pgfqpoint{3.093161in}{3.201484in}}%
\pgfpathlineto{\pgfqpoint{3.093161in}{3.198534in}}%
\pgfpathmoveto{\pgfqpoint{3.074996in}{3.201484in}}%
\pgfpathlineto{\pgfqpoint{3.074996in}{3.201484in}}%
\pgfpathlineto{\pgfqpoint{3.074996in}{3.204433in}}%
\pgfpathlineto{\pgfqpoint{3.079538in}{3.204433in}}%
\pgfpathlineto{\pgfqpoint{3.079538in}{3.201484in}}%
\pgfpathmoveto{\pgfqpoint{3.074996in}{3.204433in}}%
\pgfpathlineto{\pgfqpoint{3.074996in}{3.204433in}}%
\pgfpathlineto{\pgfqpoint{3.074996in}{3.207382in}}%
\pgfpathlineto{\pgfqpoint{3.079538in}{3.207382in}}%
\pgfpathlineto{\pgfqpoint{3.079538in}{3.204433in}}%
\pgfpathmoveto{\pgfqpoint{3.079538in}{3.201484in}}%
\pgfpathlineto{\pgfqpoint{3.079538in}{3.201484in}}%
\pgfpathlineto{\pgfqpoint{3.079538in}{3.204433in}}%
\pgfpathlineto{\pgfqpoint{3.084079in}{3.204433in}}%
\pgfpathlineto{\pgfqpoint{3.084079in}{3.201484in}}%
\pgfpathmoveto{\pgfqpoint{3.079538in}{3.204433in}}%
\pgfpathlineto{\pgfqpoint{3.079538in}{3.204433in}}%
\pgfpathlineto{\pgfqpoint{3.079538in}{3.207382in}}%
\pgfpathlineto{\pgfqpoint{3.084079in}{3.207382in}}%
\pgfpathlineto{\pgfqpoint{3.084079in}{3.204433in}}%
\pgfpathmoveto{\pgfqpoint{3.074996in}{3.207382in}}%
\pgfpathlineto{\pgfqpoint{3.074996in}{3.207382in}}%
\pgfpathlineto{\pgfqpoint{3.074996in}{3.210331in}}%
\pgfpathlineto{\pgfqpoint{3.079538in}{3.210331in}}%
\pgfpathlineto{\pgfqpoint{3.079538in}{3.207382in}}%
\pgfpathmoveto{\pgfqpoint{3.074996in}{3.210331in}}%
\pgfpathlineto{\pgfqpoint{3.074996in}{3.210331in}}%
\pgfpathlineto{\pgfqpoint{3.074996in}{3.213280in}}%
\pgfpathlineto{\pgfqpoint{3.079538in}{3.213280in}}%
\pgfpathlineto{\pgfqpoint{3.079538in}{3.210331in}}%
\pgfpathmoveto{\pgfqpoint{3.079538in}{3.207382in}}%
\pgfpathlineto{\pgfqpoint{3.079538in}{3.207382in}}%
\pgfpathlineto{\pgfqpoint{3.079538in}{3.210331in}}%
\pgfpathlineto{\pgfqpoint{3.084079in}{3.210331in}}%
\pgfpathlineto{\pgfqpoint{3.084079in}{3.207382in}}%
\pgfpathmoveto{\pgfqpoint{3.079538in}{3.210331in}}%
\pgfpathlineto{\pgfqpoint{3.079538in}{3.210331in}}%
\pgfpathlineto{\pgfqpoint{3.079538in}{3.213280in}}%
\pgfpathlineto{\pgfqpoint{3.084079in}{3.213280in}}%
\pgfpathlineto{\pgfqpoint{3.084079in}{3.210331in}}%
\pgfpathmoveto{\pgfqpoint{3.084079in}{3.201484in}}%
\pgfpathlineto{\pgfqpoint{3.084079in}{3.201484in}}%
\pgfpathlineto{\pgfqpoint{3.084079in}{3.204433in}}%
\pgfpathlineto{\pgfqpoint{3.088620in}{3.204433in}}%
\pgfpathlineto{\pgfqpoint{3.088620in}{3.201484in}}%
\pgfpathmoveto{\pgfqpoint{3.084079in}{3.204433in}}%
\pgfpathlineto{\pgfqpoint{3.084079in}{3.204433in}}%
\pgfpathlineto{\pgfqpoint{3.084079in}{3.207382in}}%
\pgfpathlineto{\pgfqpoint{3.088620in}{3.207382in}}%
\pgfpathlineto{\pgfqpoint{3.088620in}{3.204433in}}%
\pgfpathmoveto{\pgfqpoint{3.093161in}{3.189687in}}%
\pgfpathlineto{\pgfqpoint{3.093161in}{3.189687in}}%
\pgfpathlineto{\pgfqpoint{3.093161in}{3.192636in}}%
\pgfpathlineto{\pgfqpoint{3.097703in}{3.192636in}}%
\pgfpathlineto{\pgfqpoint{3.097703in}{3.189687in}}%
\pgfpathmoveto{\pgfqpoint{3.093161in}{3.192636in}}%
\pgfpathlineto{\pgfqpoint{3.093161in}{3.192636in}}%
\pgfpathlineto{\pgfqpoint{3.093161in}{3.195585in}}%
\pgfpathlineto{\pgfqpoint{3.097703in}{3.195585in}}%
\pgfpathlineto{\pgfqpoint{3.097703in}{3.192636in}}%
\pgfpathmoveto{\pgfqpoint{3.074996in}{3.213280in}}%
\pgfpathlineto{\pgfqpoint{3.074996in}{3.213280in}}%
\pgfpathlineto{\pgfqpoint{3.074996in}{3.216229in}}%
\pgfpathlineto{\pgfqpoint{3.079538in}{3.216229in}}%
\pgfpathlineto{\pgfqpoint{3.079538in}{3.213280in}}%
\pgfpathmoveto{\pgfqpoint{3.074996in}{3.216229in}}%
\pgfpathlineto{\pgfqpoint{3.074996in}{3.216229in}}%
\pgfpathlineto{\pgfqpoint{3.074996in}{3.219178in}}%
\pgfpathlineto{\pgfqpoint{3.079538in}{3.219178in}}%
\pgfpathlineto{\pgfqpoint{3.079538in}{3.216229in}}%
\pgfpathmoveto{\pgfqpoint{3.220316in}{2.101428in}}%
\pgfpathlineto{\pgfqpoint{3.220316in}{2.101428in}}%
\pgfpathlineto{\pgfqpoint{3.220316in}{2.104378in}}%
\pgfpathlineto{\pgfqpoint{3.224857in}{2.104378in}}%
\pgfpathlineto{\pgfqpoint{3.224857in}{2.101428in}}%
\pgfpathmoveto{\pgfqpoint{3.220316in}{2.104378in}}%
\pgfpathlineto{\pgfqpoint{3.220316in}{2.104378in}}%
\pgfpathlineto{\pgfqpoint{3.220316in}{2.107327in}}%
\pgfpathlineto{\pgfqpoint{3.224857in}{2.107327in}}%
\pgfpathlineto{\pgfqpoint{3.224857in}{2.104378in}}%
\pgfpathmoveto{\pgfqpoint{3.220316in}{2.107327in}}%
\pgfpathlineto{\pgfqpoint{3.220316in}{2.107327in}}%
\pgfpathlineto{\pgfqpoint{3.220316in}{2.110276in}}%
\pgfpathlineto{\pgfqpoint{3.224857in}{2.110276in}}%
\pgfpathlineto{\pgfqpoint{3.224857in}{2.107327in}}%
\pgfpathmoveto{\pgfqpoint{3.224857in}{2.104378in}}%
\pgfpathlineto{\pgfqpoint{3.224857in}{2.104378in}}%
\pgfpathlineto{\pgfqpoint{3.224857in}{2.107327in}}%
\pgfpathlineto{\pgfqpoint{3.229398in}{2.107327in}}%
\pgfpathlineto{\pgfqpoint{3.229398in}{2.104378in}}%
\pgfpathmoveto{\pgfqpoint{3.224857in}{2.107327in}}%
\pgfpathlineto{\pgfqpoint{3.224857in}{2.107327in}}%
\pgfpathlineto{\pgfqpoint{3.224857in}{2.110276in}}%
\pgfpathlineto{\pgfqpoint{3.229398in}{2.110276in}}%
\pgfpathlineto{\pgfqpoint{3.229398in}{2.107327in}}%
\pgfpathmoveto{\pgfqpoint{3.229398in}{2.107327in}}%
\pgfpathlineto{\pgfqpoint{3.229398in}{2.107327in}}%
\pgfpathlineto{\pgfqpoint{3.229398in}{2.110276in}}%
\pgfpathlineto{\pgfqpoint{3.233939in}{2.110276in}}%
\pgfpathlineto{\pgfqpoint{3.233939in}{2.107327in}}%
\pgfpathmoveto{\pgfqpoint{3.229398in}{2.110276in}}%
\pgfpathlineto{\pgfqpoint{3.229398in}{2.110276in}}%
\pgfpathlineto{\pgfqpoint{3.229398in}{2.113225in}}%
\pgfpathlineto{\pgfqpoint{3.233939in}{2.113225in}}%
\pgfpathlineto{\pgfqpoint{3.233939in}{2.110276in}}%
\pgfpathmoveto{\pgfqpoint{3.229398in}{2.113225in}}%
\pgfpathlineto{\pgfqpoint{3.229398in}{2.113225in}}%
\pgfpathlineto{\pgfqpoint{3.229398in}{2.116175in}}%
\pgfpathlineto{\pgfqpoint{3.233939in}{2.116175in}}%
\pgfpathlineto{\pgfqpoint{3.233939in}{2.113225in}}%
\pgfpathmoveto{\pgfqpoint{3.233939in}{2.110276in}}%
\pgfpathlineto{\pgfqpoint{3.233939in}{2.110276in}}%
\pgfpathlineto{\pgfqpoint{3.233939in}{2.113225in}}%
\pgfpathlineto{\pgfqpoint{3.238480in}{2.113225in}}%
\pgfpathlineto{\pgfqpoint{3.238480in}{2.110276in}}%
\pgfpathmoveto{\pgfqpoint{3.233939in}{2.113225in}}%
\pgfpathlineto{\pgfqpoint{3.233939in}{2.113225in}}%
\pgfpathlineto{\pgfqpoint{3.233939in}{2.116175in}}%
\pgfpathlineto{\pgfqpoint{3.238480in}{2.116175in}}%
\pgfpathlineto{\pgfqpoint{3.238480in}{2.113225in}}%
\pgfpathmoveto{\pgfqpoint{3.238480in}{2.113225in}}%
\pgfpathlineto{\pgfqpoint{3.238480in}{2.113225in}}%
\pgfpathlineto{\pgfqpoint{3.238480in}{2.116175in}}%
\pgfpathlineto{\pgfqpoint{3.243020in}{2.116175in}}%
\pgfpathlineto{\pgfqpoint{3.243020in}{2.113225in}}%
\pgfpathmoveto{\pgfqpoint{3.238480in}{2.116175in}}%
\pgfpathlineto{\pgfqpoint{3.238480in}{2.116175in}}%
\pgfpathlineto{\pgfqpoint{3.238480in}{2.119124in}}%
\pgfpathlineto{\pgfqpoint{3.243020in}{2.119124in}}%
\pgfpathlineto{\pgfqpoint{3.243020in}{2.116175in}}%
\pgfpathmoveto{\pgfqpoint{3.238480in}{2.119124in}}%
\pgfpathlineto{\pgfqpoint{3.238480in}{2.119124in}}%
\pgfpathlineto{\pgfqpoint{3.238480in}{2.122073in}}%
\pgfpathlineto{\pgfqpoint{3.243020in}{2.122073in}}%
\pgfpathlineto{\pgfqpoint{3.243020in}{2.119124in}}%
\pgfpathmoveto{\pgfqpoint{3.243020in}{2.116175in}}%
\pgfpathlineto{\pgfqpoint{3.243020in}{2.116175in}}%
\pgfpathlineto{\pgfqpoint{3.243020in}{2.119124in}}%
\pgfpathlineto{\pgfqpoint{3.247561in}{2.119124in}}%
\pgfpathlineto{\pgfqpoint{3.247561in}{2.116175in}}%
\pgfpathmoveto{\pgfqpoint{3.243020in}{2.119124in}}%
\pgfpathlineto{\pgfqpoint{3.243020in}{2.119124in}}%
\pgfpathlineto{\pgfqpoint{3.243020in}{2.122073in}}%
\pgfpathlineto{\pgfqpoint{3.247561in}{2.122073in}}%
\pgfpathlineto{\pgfqpoint{3.247561in}{2.119124in}}%
\pgfpathmoveto{\pgfqpoint{3.247561in}{2.119124in}}%
\pgfpathlineto{\pgfqpoint{3.247561in}{2.119124in}}%
\pgfpathlineto{\pgfqpoint{3.247561in}{2.122073in}}%
\pgfpathlineto{\pgfqpoint{3.252102in}{2.122073in}}%
\pgfpathlineto{\pgfqpoint{3.252102in}{2.119124in}}%
\pgfpathmoveto{\pgfqpoint{3.247561in}{2.122073in}}%
\pgfpathlineto{\pgfqpoint{3.247561in}{2.122073in}}%
\pgfpathlineto{\pgfqpoint{3.247561in}{2.125022in}}%
\pgfpathlineto{\pgfqpoint{3.252102in}{2.125022in}}%
\pgfpathlineto{\pgfqpoint{3.252102in}{2.122073in}}%
\pgfpathmoveto{\pgfqpoint{3.247561in}{2.125022in}}%
\pgfpathlineto{\pgfqpoint{3.247561in}{2.125022in}}%
\pgfpathlineto{\pgfqpoint{3.247561in}{2.127971in}}%
\pgfpathlineto{\pgfqpoint{3.252102in}{2.127971in}}%
\pgfpathlineto{\pgfqpoint{3.252102in}{2.125022in}}%
\pgfpathmoveto{\pgfqpoint{3.252102in}{2.122073in}}%
\pgfpathlineto{\pgfqpoint{3.252102in}{2.122073in}}%
\pgfpathlineto{\pgfqpoint{3.252102in}{2.125022in}}%
\pgfpathlineto{\pgfqpoint{3.256643in}{2.125022in}}%
\pgfpathlineto{\pgfqpoint{3.256643in}{2.122073in}}%
\pgfpathmoveto{\pgfqpoint{3.252102in}{2.125022in}}%
\pgfpathlineto{\pgfqpoint{3.252102in}{2.125022in}}%
\pgfpathlineto{\pgfqpoint{3.252102in}{2.127971in}}%
\pgfpathlineto{\pgfqpoint{3.256643in}{2.127971in}}%
\pgfpathlineto{\pgfqpoint{3.256643in}{2.125022in}}%
\pgfpathmoveto{\pgfqpoint{3.256643in}{2.125022in}}%
\pgfpathlineto{\pgfqpoint{3.256643in}{2.125022in}}%
\pgfpathlineto{\pgfqpoint{3.256643in}{2.127971in}}%
\pgfpathlineto{\pgfqpoint{3.261184in}{2.127971in}}%
\pgfpathlineto{\pgfqpoint{3.261184in}{2.125022in}}%
\pgfpathmoveto{\pgfqpoint{3.256643in}{2.127971in}}%
\pgfpathlineto{\pgfqpoint{3.256643in}{2.127971in}}%
\pgfpathlineto{\pgfqpoint{3.256643in}{2.130921in}}%
\pgfpathlineto{\pgfqpoint{3.261184in}{2.130921in}}%
\pgfpathlineto{\pgfqpoint{3.261184in}{2.127971in}}%
\pgfpathmoveto{\pgfqpoint{3.256643in}{2.130921in}}%
\pgfpathlineto{\pgfqpoint{3.256643in}{2.130921in}}%
\pgfpathlineto{\pgfqpoint{3.256643in}{2.133870in}}%
\pgfpathlineto{\pgfqpoint{3.261184in}{2.133870in}}%
\pgfpathlineto{\pgfqpoint{3.261184in}{2.130921in}}%
\pgfpathmoveto{\pgfqpoint{3.261184in}{2.127971in}}%
\pgfpathlineto{\pgfqpoint{3.261184in}{2.127971in}}%
\pgfpathlineto{\pgfqpoint{3.261184in}{2.130921in}}%
\pgfpathlineto{\pgfqpoint{3.265724in}{2.130921in}}%
\pgfpathlineto{\pgfqpoint{3.265724in}{2.127971in}}%
\pgfpathmoveto{\pgfqpoint{3.261184in}{2.130921in}}%
\pgfpathlineto{\pgfqpoint{3.261184in}{2.130921in}}%
\pgfpathlineto{\pgfqpoint{3.261184in}{2.133870in}}%
\pgfpathlineto{\pgfqpoint{3.265724in}{2.133870in}}%
\pgfpathlineto{\pgfqpoint{3.265724in}{2.130921in}}%
\pgfpathmoveto{\pgfqpoint{3.265724in}{2.130921in}}%
\pgfpathlineto{\pgfqpoint{3.265724in}{2.130921in}}%
\pgfpathlineto{\pgfqpoint{3.265724in}{2.133870in}}%
\pgfpathlineto{\pgfqpoint{3.270265in}{2.133870in}}%
\pgfpathlineto{\pgfqpoint{3.270265in}{2.130921in}}%
\pgfpathmoveto{\pgfqpoint{3.265724in}{2.133870in}}%
\pgfpathlineto{\pgfqpoint{3.265724in}{2.133870in}}%
\pgfpathlineto{\pgfqpoint{3.265724in}{2.136819in}}%
\pgfpathlineto{\pgfqpoint{3.270265in}{2.136819in}}%
\pgfpathlineto{\pgfqpoint{3.270265in}{2.133870in}}%
\pgfpathmoveto{\pgfqpoint{3.265724in}{2.136819in}}%
\pgfpathlineto{\pgfqpoint{3.265724in}{2.136819in}}%
\pgfpathlineto{\pgfqpoint{3.265724in}{2.139768in}}%
\pgfpathlineto{\pgfqpoint{3.270265in}{2.139768in}}%
\pgfpathlineto{\pgfqpoint{3.270265in}{2.136819in}}%
\pgfpathmoveto{\pgfqpoint{3.270265in}{2.133870in}}%
\pgfpathlineto{\pgfqpoint{3.270265in}{2.133870in}}%
\pgfpathlineto{\pgfqpoint{3.270265in}{2.136819in}}%
\pgfpathlineto{\pgfqpoint{3.274806in}{2.136819in}}%
\pgfpathlineto{\pgfqpoint{3.274806in}{2.133870in}}%
\pgfpathmoveto{\pgfqpoint{3.270265in}{2.136819in}}%
\pgfpathlineto{\pgfqpoint{3.270265in}{2.136819in}}%
\pgfpathlineto{\pgfqpoint{3.270265in}{2.139768in}}%
\pgfpathlineto{\pgfqpoint{3.274806in}{2.139768in}}%
\pgfpathlineto{\pgfqpoint{3.274806in}{2.136819in}}%
\pgfpathmoveto{\pgfqpoint{3.274806in}{2.136819in}}%
\pgfpathlineto{\pgfqpoint{3.274806in}{2.136819in}}%
\pgfpathlineto{\pgfqpoint{3.274806in}{2.139768in}}%
\pgfpathlineto{\pgfqpoint{3.279347in}{2.139768in}}%
\pgfpathlineto{\pgfqpoint{3.279347in}{2.136819in}}%
\pgfpathmoveto{\pgfqpoint{3.274806in}{2.139768in}}%
\pgfpathlineto{\pgfqpoint{3.274806in}{2.139768in}}%
\pgfpathlineto{\pgfqpoint{3.274806in}{2.142717in}}%
\pgfpathlineto{\pgfqpoint{3.279347in}{2.142717in}}%
\pgfpathlineto{\pgfqpoint{3.279347in}{2.139768in}}%
\pgfpathmoveto{\pgfqpoint{3.274806in}{2.142717in}}%
\pgfpathlineto{\pgfqpoint{3.274806in}{2.142717in}}%
\pgfpathlineto{\pgfqpoint{3.274806in}{2.145667in}}%
\pgfpathlineto{\pgfqpoint{3.279347in}{2.145667in}}%
\pgfpathlineto{\pgfqpoint{3.279347in}{2.142717in}}%
\pgfpathmoveto{\pgfqpoint{3.279347in}{2.139768in}}%
\pgfpathlineto{\pgfqpoint{3.279347in}{2.139768in}}%
\pgfpathlineto{\pgfqpoint{3.279347in}{2.142717in}}%
\pgfpathlineto{\pgfqpoint{3.283888in}{2.142717in}}%
\pgfpathlineto{\pgfqpoint{3.283888in}{2.139768in}}%
\pgfpathmoveto{\pgfqpoint{3.279347in}{2.142717in}}%
\pgfpathlineto{\pgfqpoint{3.279347in}{2.142717in}}%
\pgfpathlineto{\pgfqpoint{3.279347in}{2.145667in}}%
\pgfpathlineto{\pgfqpoint{3.283888in}{2.145667in}}%
\pgfpathlineto{\pgfqpoint{3.283888in}{2.142717in}}%
\pgfpathmoveto{\pgfqpoint{3.283888in}{2.142717in}}%
\pgfpathlineto{\pgfqpoint{3.283888in}{2.142717in}}%
\pgfpathlineto{\pgfqpoint{3.283888in}{2.145667in}}%
\pgfpathlineto{\pgfqpoint{3.288428in}{2.145667in}}%
\pgfpathlineto{\pgfqpoint{3.288428in}{2.142717in}}%
\pgfpathmoveto{\pgfqpoint{3.283888in}{2.145667in}}%
\pgfpathlineto{\pgfqpoint{3.283888in}{2.145667in}}%
\pgfpathlineto{\pgfqpoint{3.283888in}{2.148616in}}%
\pgfpathlineto{\pgfqpoint{3.288428in}{2.148616in}}%
\pgfpathlineto{\pgfqpoint{3.288428in}{2.145667in}}%
\pgfpathmoveto{\pgfqpoint{3.283888in}{2.148616in}}%
\pgfpathlineto{\pgfqpoint{3.283888in}{2.148616in}}%
\pgfpathlineto{\pgfqpoint{3.283888in}{2.151565in}}%
\pgfpathlineto{\pgfqpoint{3.288428in}{2.151565in}}%
\pgfpathlineto{\pgfqpoint{3.288428in}{2.148616in}}%
\pgfpathmoveto{\pgfqpoint{3.288428in}{2.145667in}}%
\pgfpathlineto{\pgfqpoint{3.288428in}{2.145667in}}%
\pgfpathlineto{\pgfqpoint{3.288428in}{2.148616in}}%
\pgfpathlineto{\pgfqpoint{3.292969in}{2.148616in}}%
\pgfpathlineto{\pgfqpoint{3.292969in}{2.145667in}}%
\pgfpathmoveto{\pgfqpoint{3.288428in}{2.148616in}}%
\pgfpathlineto{\pgfqpoint{3.288428in}{2.148616in}}%
\pgfpathlineto{\pgfqpoint{3.288428in}{2.151565in}}%
\pgfpathlineto{\pgfqpoint{3.292969in}{2.151565in}}%
\pgfpathlineto{\pgfqpoint{3.292969in}{2.148616in}}%
\pgfpathmoveto{\pgfqpoint{3.292969in}{2.148616in}}%
\pgfpathlineto{\pgfqpoint{3.292969in}{2.148616in}}%
\pgfpathlineto{\pgfqpoint{3.292969in}{2.151565in}}%
\pgfpathlineto{\pgfqpoint{3.297510in}{2.151565in}}%
\pgfpathlineto{\pgfqpoint{3.297510in}{2.148616in}}%
\pgfpathmoveto{\pgfqpoint{3.292969in}{2.151565in}}%
\pgfpathlineto{\pgfqpoint{3.292969in}{2.151565in}}%
\pgfpathlineto{\pgfqpoint{3.292969in}{2.154514in}}%
\pgfpathlineto{\pgfqpoint{3.297510in}{2.154514in}}%
\pgfpathlineto{\pgfqpoint{3.297510in}{2.151565in}}%
\pgfpathmoveto{\pgfqpoint{3.292969in}{2.154514in}}%
\pgfpathlineto{\pgfqpoint{3.292969in}{2.154514in}}%
\pgfpathlineto{\pgfqpoint{3.292969in}{2.157463in}}%
\pgfpathlineto{\pgfqpoint{3.297510in}{2.157463in}}%
\pgfpathlineto{\pgfqpoint{3.297510in}{2.154514in}}%
\pgfpathmoveto{\pgfqpoint{3.297510in}{2.151565in}}%
\pgfpathlineto{\pgfqpoint{3.297510in}{2.151565in}}%
\pgfpathlineto{\pgfqpoint{3.297510in}{2.154514in}}%
\pgfpathlineto{\pgfqpoint{3.302051in}{2.154514in}}%
\pgfpathlineto{\pgfqpoint{3.302051in}{2.151565in}}%
\pgfpathmoveto{\pgfqpoint{3.297510in}{2.154514in}}%
\pgfpathlineto{\pgfqpoint{3.297510in}{2.154514in}}%
\pgfpathlineto{\pgfqpoint{3.297510in}{2.157463in}}%
\pgfpathlineto{\pgfqpoint{3.302051in}{2.157463in}}%
\pgfpathlineto{\pgfqpoint{3.302051in}{2.154514in}}%
\pgfpathmoveto{\pgfqpoint{3.302051in}{2.154514in}}%
\pgfpathlineto{\pgfqpoint{3.302051in}{2.154514in}}%
\pgfpathlineto{\pgfqpoint{3.302051in}{2.157463in}}%
\pgfpathlineto{\pgfqpoint{3.306592in}{2.157463in}}%
\pgfpathlineto{\pgfqpoint{3.306592in}{2.154514in}}%
\pgfpathmoveto{\pgfqpoint{3.302051in}{2.157463in}}%
\pgfpathlineto{\pgfqpoint{3.302051in}{2.157463in}}%
\pgfpathlineto{\pgfqpoint{3.302051in}{2.160412in}}%
\pgfpathlineto{\pgfqpoint{3.306592in}{2.160412in}}%
\pgfpathlineto{\pgfqpoint{3.306592in}{2.157463in}}%
\pgfpathmoveto{\pgfqpoint{3.302051in}{2.160412in}}%
\pgfpathlineto{\pgfqpoint{3.302051in}{2.160412in}}%
\pgfpathlineto{\pgfqpoint{3.302051in}{2.163362in}}%
\pgfpathlineto{\pgfqpoint{3.306592in}{2.163362in}}%
\pgfpathlineto{\pgfqpoint{3.306592in}{2.160412in}}%
\pgfpathmoveto{\pgfqpoint{3.306592in}{2.157463in}}%
\pgfpathlineto{\pgfqpoint{3.306592in}{2.157463in}}%
\pgfpathlineto{\pgfqpoint{3.306592in}{2.160412in}}%
\pgfpathlineto{\pgfqpoint{3.311132in}{2.160412in}}%
\pgfpathlineto{\pgfqpoint{3.311132in}{2.157463in}}%
\pgfpathmoveto{\pgfqpoint{3.306592in}{2.160412in}}%
\pgfpathlineto{\pgfqpoint{3.306592in}{2.160412in}}%
\pgfpathlineto{\pgfqpoint{3.306592in}{2.163362in}}%
\pgfpathlineto{\pgfqpoint{3.311132in}{2.163362in}}%
\pgfpathlineto{\pgfqpoint{3.311132in}{2.160412in}}%
\pgfpathmoveto{\pgfqpoint{3.311132in}{2.160412in}}%
\pgfpathlineto{\pgfqpoint{3.311132in}{2.160412in}}%
\pgfpathlineto{\pgfqpoint{3.311132in}{2.163362in}}%
\pgfpathlineto{\pgfqpoint{3.315673in}{2.163362in}}%
\pgfpathlineto{\pgfqpoint{3.315673in}{2.160412in}}%
\pgfpathmoveto{\pgfqpoint{3.311132in}{2.163362in}}%
\pgfpathlineto{\pgfqpoint{3.311132in}{2.163362in}}%
\pgfpathlineto{\pgfqpoint{3.311132in}{2.166311in}}%
\pgfpathlineto{\pgfqpoint{3.315673in}{2.166311in}}%
\pgfpathlineto{\pgfqpoint{3.315673in}{2.163362in}}%
\pgfpathmoveto{\pgfqpoint{3.311132in}{2.166311in}}%
\pgfpathlineto{\pgfqpoint{3.311132in}{2.166311in}}%
\pgfpathlineto{\pgfqpoint{3.311132in}{2.169260in}}%
\pgfpathlineto{\pgfqpoint{3.315673in}{2.169260in}}%
\pgfpathlineto{\pgfqpoint{3.315673in}{2.166311in}}%
\pgfpathmoveto{\pgfqpoint{3.315673in}{2.163362in}}%
\pgfpathlineto{\pgfqpoint{3.315673in}{2.163362in}}%
\pgfpathlineto{\pgfqpoint{3.315673in}{2.166311in}}%
\pgfpathlineto{\pgfqpoint{3.320214in}{2.166311in}}%
\pgfpathlineto{\pgfqpoint{3.320214in}{2.163362in}}%
\pgfpathmoveto{\pgfqpoint{3.315673in}{2.166311in}}%
\pgfpathlineto{\pgfqpoint{3.315673in}{2.166311in}}%
\pgfpathlineto{\pgfqpoint{3.315673in}{2.169260in}}%
\pgfpathlineto{\pgfqpoint{3.320214in}{2.169260in}}%
\pgfpathlineto{\pgfqpoint{3.320214in}{2.166311in}}%
\pgfpathmoveto{\pgfqpoint{3.320214in}{2.166311in}}%
\pgfpathlineto{\pgfqpoint{3.320214in}{2.166311in}}%
\pgfpathlineto{\pgfqpoint{3.320214in}{2.169260in}}%
\pgfpathlineto{\pgfqpoint{3.324755in}{2.169260in}}%
\pgfpathlineto{\pgfqpoint{3.324755in}{2.166311in}}%
\pgfpathmoveto{\pgfqpoint{3.320214in}{2.169260in}}%
\pgfpathlineto{\pgfqpoint{3.320214in}{2.169260in}}%
\pgfpathlineto{\pgfqpoint{3.320214in}{2.172209in}}%
\pgfpathlineto{\pgfqpoint{3.324755in}{2.172209in}}%
\pgfpathlineto{\pgfqpoint{3.324755in}{2.169260in}}%
\pgfpathmoveto{\pgfqpoint{3.320214in}{2.172209in}}%
\pgfpathlineto{\pgfqpoint{3.320214in}{2.172209in}}%
\pgfpathlineto{\pgfqpoint{3.320214in}{2.175158in}}%
\pgfpathlineto{\pgfqpoint{3.324755in}{2.175158in}}%
\pgfpathlineto{\pgfqpoint{3.324755in}{2.172209in}}%
\pgfpathmoveto{\pgfqpoint{3.324755in}{2.169260in}}%
\pgfpathlineto{\pgfqpoint{3.324755in}{2.169260in}}%
\pgfpathlineto{\pgfqpoint{3.324755in}{2.172209in}}%
\pgfpathlineto{\pgfqpoint{3.329296in}{2.172209in}}%
\pgfpathlineto{\pgfqpoint{3.329296in}{2.169260in}}%
\pgfpathmoveto{\pgfqpoint{3.324755in}{2.172209in}}%
\pgfpathlineto{\pgfqpoint{3.324755in}{2.172209in}}%
\pgfpathlineto{\pgfqpoint{3.324755in}{2.175158in}}%
\pgfpathlineto{\pgfqpoint{3.329296in}{2.175158in}}%
\pgfpathlineto{\pgfqpoint{3.329296in}{2.172209in}}%
\pgfpathmoveto{\pgfqpoint{3.329296in}{2.172209in}}%
\pgfpathlineto{\pgfqpoint{3.329296in}{2.172209in}}%
\pgfpathlineto{\pgfqpoint{3.329296in}{2.175158in}}%
\pgfpathlineto{\pgfqpoint{3.333836in}{2.175158in}}%
\pgfpathlineto{\pgfqpoint{3.333836in}{2.172209in}}%
\pgfpathmoveto{\pgfqpoint{3.329296in}{2.175158in}}%
\pgfpathlineto{\pgfqpoint{3.329296in}{2.175158in}}%
\pgfpathlineto{\pgfqpoint{3.329296in}{2.178108in}}%
\pgfpathlineto{\pgfqpoint{3.333836in}{2.178108in}}%
\pgfpathlineto{\pgfqpoint{3.333836in}{2.175158in}}%
\pgfpathmoveto{\pgfqpoint{3.329296in}{2.178108in}}%
\pgfpathlineto{\pgfqpoint{3.329296in}{2.178108in}}%
\pgfpathlineto{\pgfqpoint{3.329296in}{2.181057in}}%
\pgfpathlineto{\pgfqpoint{3.333836in}{2.181057in}}%
\pgfpathlineto{\pgfqpoint{3.333836in}{2.178108in}}%
\pgfpathmoveto{\pgfqpoint{3.333836in}{2.175158in}}%
\pgfpathlineto{\pgfqpoint{3.333836in}{2.175158in}}%
\pgfpathlineto{\pgfqpoint{3.333836in}{2.178108in}}%
\pgfpathlineto{\pgfqpoint{3.338377in}{2.178108in}}%
\pgfpathlineto{\pgfqpoint{3.338377in}{2.175158in}}%
\pgfpathmoveto{\pgfqpoint{3.333836in}{2.178108in}}%
\pgfpathlineto{\pgfqpoint{3.333836in}{2.178108in}}%
\pgfpathlineto{\pgfqpoint{3.333836in}{2.181057in}}%
\pgfpathlineto{\pgfqpoint{3.338377in}{2.181057in}}%
\pgfpathlineto{\pgfqpoint{3.338377in}{2.178108in}}%
\pgfpathmoveto{\pgfqpoint{3.338377in}{2.178108in}}%
\pgfpathlineto{\pgfqpoint{3.338377in}{2.178108in}}%
\pgfpathlineto{\pgfqpoint{3.338377in}{2.181057in}}%
\pgfpathlineto{\pgfqpoint{3.342918in}{2.181057in}}%
\pgfpathlineto{\pgfqpoint{3.342918in}{2.178108in}}%
\pgfpathmoveto{\pgfqpoint{3.338377in}{2.181057in}}%
\pgfpathlineto{\pgfqpoint{3.338377in}{2.181057in}}%
\pgfpathlineto{\pgfqpoint{3.338377in}{2.184006in}}%
\pgfpathlineto{\pgfqpoint{3.342918in}{2.184006in}}%
\pgfpathlineto{\pgfqpoint{3.342918in}{2.181057in}}%
\pgfpathmoveto{\pgfqpoint{3.338377in}{2.184006in}}%
\pgfpathlineto{\pgfqpoint{3.338377in}{2.184006in}}%
\pgfpathlineto{\pgfqpoint{3.338377in}{2.186955in}}%
\pgfpathlineto{\pgfqpoint{3.342918in}{2.186955in}}%
\pgfpathlineto{\pgfqpoint{3.342918in}{2.184006in}}%
\pgfpathmoveto{\pgfqpoint{3.342918in}{2.181057in}}%
\pgfpathlineto{\pgfqpoint{3.342918in}{2.181057in}}%
\pgfpathlineto{\pgfqpoint{3.342918in}{2.184006in}}%
\pgfpathlineto{\pgfqpoint{3.347459in}{2.184006in}}%
\pgfpathlineto{\pgfqpoint{3.347459in}{2.181057in}}%
\pgfpathmoveto{\pgfqpoint{3.342918in}{2.184006in}}%
\pgfpathlineto{\pgfqpoint{3.342918in}{2.184006in}}%
\pgfpathlineto{\pgfqpoint{3.342918in}{2.186955in}}%
\pgfpathlineto{\pgfqpoint{3.347459in}{2.186955in}}%
\pgfpathlineto{\pgfqpoint{3.347459in}{2.184006in}}%
\pgfpathmoveto{\pgfqpoint{3.347459in}{2.184006in}}%
\pgfpathlineto{\pgfqpoint{3.347459in}{2.184006in}}%
\pgfpathlineto{\pgfqpoint{3.347459in}{2.186955in}}%
\pgfpathlineto{\pgfqpoint{3.352000in}{2.186955in}}%
\pgfpathlineto{\pgfqpoint{3.352000in}{2.184006in}}%
\pgfpathmoveto{\pgfqpoint{3.347459in}{2.186955in}}%
\pgfpathlineto{\pgfqpoint{3.347459in}{2.186955in}}%
\pgfpathlineto{\pgfqpoint{3.347459in}{2.189904in}}%
\pgfpathlineto{\pgfqpoint{3.352000in}{2.189904in}}%
\pgfpathlineto{\pgfqpoint{3.352000in}{2.186955in}}%
\pgfpathmoveto{\pgfqpoint{3.347459in}{2.189904in}}%
\pgfpathlineto{\pgfqpoint{3.347459in}{2.189904in}}%
\pgfpathlineto{\pgfqpoint{3.347459in}{2.192854in}}%
\pgfpathlineto{\pgfqpoint{3.352000in}{2.192854in}}%
\pgfpathlineto{\pgfqpoint{3.352000in}{2.189904in}}%
\pgfpathmoveto{\pgfqpoint{3.352000in}{2.186955in}}%
\pgfpathlineto{\pgfqpoint{3.352000in}{2.186955in}}%
\pgfpathlineto{\pgfqpoint{3.352000in}{2.189904in}}%
\pgfpathlineto{\pgfqpoint{3.356540in}{2.189904in}}%
\pgfpathlineto{\pgfqpoint{3.356540in}{2.186955in}}%
\pgfpathmoveto{\pgfqpoint{3.352000in}{2.189904in}}%
\pgfpathlineto{\pgfqpoint{3.352000in}{2.189904in}}%
\pgfpathlineto{\pgfqpoint{3.352000in}{2.192854in}}%
\pgfpathlineto{\pgfqpoint{3.356540in}{2.192854in}}%
\pgfpathlineto{\pgfqpoint{3.356540in}{2.189904in}}%
\pgfpathmoveto{\pgfqpoint{3.356540in}{2.189904in}}%
\pgfpathlineto{\pgfqpoint{3.356540in}{2.189904in}}%
\pgfpathlineto{\pgfqpoint{3.356540in}{2.192854in}}%
\pgfpathlineto{\pgfqpoint{3.361081in}{2.192854in}}%
\pgfpathlineto{\pgfqpoint{3.361081in}{2.189904in}}%
\pgfpathmoveto{\pgfqpoint{3.356540in}{2.192854in}}%
\pgfpathlineto{\pgfqpoint{3.356540in}{2.192854in}}%
\pgfpathlineto{\pgfqpoint{3.356540in}{2.195803in}}%
\pgfpathlineto{\pgfqpoint{3.361081in}{2.195803in}}%
\pgfpathlineto{\pgfqpoint{3.361081in}{2.192854in}}%
\pgfpathmoveto{\pgfqpoint{3.356540in}{2.195803in}}%
\pgfpathlineto{\pgfqpoint{3.356540in}{2.195803in}}%
\pgfpathlineto{\pgfqpoint{3.356540in}{2.198752in}}%
\pgfpathlineto{\pgfqpoint{3.361081in}{2.198752in}}%
\pgfpathlineto{\pgfqpoint{3.361081in}{2.195803in}}%
\pgfpathmoveto{\pgfqpoint{3.361081in}{2.192854in}}%
\pgfpathlineto{\pgfqpoint{3.361081in}{2.192854in}}%
\pgfpathlineto{\pgfqpoint{3.361081in}{2.195803in}}%
\pgfpathlineto{\pgfqpoint{3.365622in}{2.195803in}}%
\pgfpathlineto{\pgfqpoint{3.365622in}{2.192854in}}%
\pgfpathmoveto{\pgfqpoint{3.361081in}{2.195803in}}%
\pgfpathlineto{\pgfqpoint{3.361081in}{2.195803in}}%
\pgfpathlineto{\pgfqpoint{3.361081in}{2.198752in}}%
\pgfpathlineto{\pgfqpoint{3.365622in}{2.198752in}}%
\pgfpathlineto{\pgfqpoint{3.365622in}{2.195803in}}%
\pgfpathmoveto{\pgfqpoint{3.356540in}{2.835782in}}%
\pgfpathlineto{\pgfqpoint{3.356540in}{2.835782in}}%
\pgfpathlineto{\pgfqpoint{3.356540in}{2.838732in}}%
\pgfpathlineto{\pgfqpoint{3.361081in}{2.838732in}}%
\pgfpathlineto{\pgfqpoint{3.361081in}{2.835782in}}%
\pgfpathmoveto{\pgfqpoint{3.356540in}{2.838732in}}%
\pgfpathlineto{\pgfqpoint{3.356540in}{2.838732in}}%
\pgfpathlineto{\pgfqpoint{3.356540in}{2.841681in}}%
\pgfpathlineto{\pgfqpoint{3.361081in}{2.841681in}}%
\pgfpathlineto{\pgfqpoint{3.361081in}{2.838732in}}%
\pgfpathmoveto{\pgfqpoint{3.361081in}{2.835782in}}%
\pgfpathlineto{\pgfqpoint{3.361081in}{2.835782in}}%
\pgfpathlineto{\pgfqpoint{3.361081in}{2.838732in}}%
\pgfpathlineto{\pgfqpoint{3.365622in}{2.838732in}}%
\pgfpathlineto{\pgfqpoint{3.365622in}{2.835782in}}%
\pgfpathmoveto{\pgfqpoint{3.361081in}{2.838732in}}%
\pgfpathlineto{\pgfqpoint{3.361081in}{2.838732in}}%
\pgfpathlineto{\pgfqpoint{3.361081in}{2.841681in}}%
\pgfpathlineto{\pgfqpoint{3.365622in}{2.841681in}}%
\pgfpathlineto{\pgfqpoint{3.365622in}{2.838732in}}%
\pgfpathmoveto{\pgfqpoint{3.356540in}{2.841681in}}%
\pgfpathlineto{\pgfqpoint{3.356540in}{2.841681in}}%
\pgfpathlineto{\pgfqpoint{3.356540in}{2.844630in}}%
\pgfpathlineto{\pgfqpoint{3.361081in}{2.844630in}}%
\pgfpathlineto{\pgfqpoint{3.361081in}{2.841681in}}%
\pgfpathmoveto{\pgfqpoint{3.356540in}{2.844630in}}%
\pgfpathlineto{\pgfqpoint{3.356540in}{2.844630in}}%
\pgfpathlineto{\pgfqpoint{3.356540in}{2.847579in}}%
\pgfpathlineto{\pgfqpoint{3.361081in}{2.847579in}}%
\pgfpathlineto{\pgfqpoint{3.361081in}{2.844630in}}%
\pgfpathmoveto{\pgfqpoint{3.361081in}{2.841681in}}%
\pgfpathlineto{\pgfqpoint{3.361081in}{2.841681in}}%
\pgfpathlineto{\pgfqpoint{3.361081in}{2.844630in}}%
\pgfpathlineto{\pgfqpoint{3.365622in}{2.844630in}}%
\pgfpathlineto{\pgfqpoint{3.365622in}{2.841681in}}%
\pgfpathmoveto{\pgfqpoint{3.361081in}{2.844630in}}%
\pgfpathlineto{\pgfqpoint{3.361081in}{2.844630in}}%
\pgfpathlineto{\pgfqpoint{3.361081in}{2.847579in}}%
\pgfpathlineto{\pgfqpoint{3.365622in}{2.847579in}}%
\pgfpathlineto{\pgfqpoint{3.365622in}{2.844630in}}%
\pgfpathmoveto{\pgfqpoint{3.347459in}{2.847579in}}%
\pgfpathlineto{\pgfqpoint{3.347459in}{2.847579in}}%
\pgfpathlineto{\pgfqpoint{3.347459in}{2.850528in}}%
\pgfpathlineto{\pgfqpoint{3.352000in}{2.850528in}}%
\pgfpathlineto{\pgfqpoint{3.352000in}{2.847579in}}%
\pgfpathmoveto{\pgfqpoint{3.347459in}{2.850528in}}%
\pgfpathlineto{\pgfqpoint{3.347459in}{2.850528in}}%
\pgfpathlineto{\pgfqpoint{3.347459in}{2.853477in}}%
\pgfpathlineto{\pgfqpoint{3.352000in}{2.853477in}}%
\pgfpathlineto{\pgfqpoint{3.352000in}{2.850528in}}%
\pgfpathmoveto{\pgfqpoint{3.352000in}{2.847579in}}%
\pgfpathlineto{\pgfqpoint{3.352000in}{2.847579in}}%
\pgfpathlineto{\pgfqpoint{3.352000in}{2.850528in}}%
\pgfpathlineto{\pgfqpoint{3.356540in}{2.850528in}}%
\pgfpathlineto{\pgfqpoint{3.356540in}{2.847579in}}%
\pgfpathmoveto{\pgfqpoint{3.352000in}{2.850528in}}%
\pgfpathlineto{\pgfqpoint{3.352000in}{2.850528in}}%
\pgfpathlineto{\pgfqpoint{3.352000in}{2.853477in}}%
\pgfpathlineto{\pgfqpoint{3.356540in}{2.853477in}}%
\pgfpathlineto{\pgfqpoint{3.356540in}{2.850528in}}%
\pgfpathmoveto{\pgfqpoint{3.347459in}{2.853477in}}%
\pgfpathlineto{\pgfqpoint{3.347459in}{2.853477in}}%
\pgfpathlineto{\pgfqpoint{3.347459in}{2.856427in}}%
\pgfpathlineto{\pgfqpoint{3.352000in}{2.856427in}}%
\pgfpathlineto{\pgfqpoint{3.352000in}{2.853477in}}%
\pgfpathmoveto{\pgfqpoint{3.347459in}{2.856427in}}%
\pgfpathlineto{\pgfqpoint{3.347459in}{2.856427in}}%
\pgfpathlineto{\pgfqpoint{3.347459in}{2.859376in}}%
\pgfpathlineto{\pgfqpoint{3.352000in}{2.859376in}}%
\pgfpathlineto{\pgfqpoint{3.352000in}{2.856427in}}%
\pgfpathmoveto{\pgfqpoint{3.352000in}{2.853477in}}%
\pgfpathlineto{\pgfqpoint{3.352000in}{2.853477in}}%
\pgfpathlineto{\pgfqpoint{3.352000in}{2.856427in}}%
\pgfpathlineto{\pgfqpoint{3.356540in}{2.856427in}}%
\pgfpathlineto{\pgfqpoint{3.356540in}{2.853477in}}%
\pgfpathmoveto{\pgfqpoint{3.352000in}{2.856427in}}%
\pgfpathlineto{\pgfqpoint{3.352000in}{2.856427in}}%
\pgfpathlineto{\pgfqpoint{3.352000in}{2.859376in}}%
\pgfpathlineto{\pgfqpoint{3.356540in}{2.859376in}}%
\pgfpathlineto{\pgfqpoint{3.356540in}{2.856427in}}%
\pgfpathmoveto{\pgfqpoint{3.356540in}{2.847579in}}%
\pgfpathlineto{\pgfqpoint{3.356540in}{2.847579in}}%
\pgfpathlineto{\pgfqpoint{3.356540in}{2.850528in}}%
\pgfpathlineto{\pgfqpoint{3.361081in}{2.850528in}}%
\pgfpathlineto{\pgfqpoint{3.361081in}{2.847579in}}%
\pgfpathmoveto{\pgfqpoint{3.356540in}{2.850528in}}%
\pgfpathlineto{\pgfqpoint{3.356540in}{2.850528in}}%
\pgfpathlineto{\pgfqpoint{3.356540in}{2.853477in}}%
\pgfpathlineto{\pgfqpoint{3.361081in}{2.853477in}}%
\pgfpathlineto{\pgfqpoint{3.361081in}{2.850528in}}%
\pgfpathmoveto{\pgfqpoint{3.283888in}{2.930155in}}%
\pgfpathlineto{\pgfqpoint{3.283888in}{2.930155in}}%
\pgfpathlineto{\pgfqpoint{3.283888in}{2.933104in}}%
\pgfpathlineto{\pgfqpoint{3.288428in}{2.933104in}}%
\pgfpathlineto{\pgfqpoint{3.288428in}{2.930155in}}%
\pgfpathmoveto{\pgfqpoint{3.283888in}{2.933104in}}%
\pgfpathlineto{\pgfqpoint{3.283888in}{2.933104in}}%
\pgfpathlineto{\pgfqpoint{3.283888in}{2.936053in}}%
\pgfpathlineto{\pgfqpoint{3.288428in}{2.936053in}}%
\pgfpathlineto{\pgfqpoint{3.288428in}{2.933104in}}%
\pgfpathmoveto{\pgfqpoint{3.288428in}{2.930155in}}%
\pgfpathlineto{\pgfqpoint{3.288428in}{2.930155in}}%
\pgfpathlineto{\pgfqpoint{3.288428in}{2.933104in}}%
\pgfpathlineto{\pgfqpoint{3.292969in}{2.933104in}}%
\pgfpathlineto{\pgfqpoint{3.292969in}{2.930155in}}%
\pgfpathmoveto{\pgfqpoint{3.288428in}{2.933104in}}%
\pgfpathlineto{\pgfqpoint{3.288428in}{2.933104in}}%
\pgfpathlineto{\pgfqpoint{3.288428in}{2.936053in}}%
\pgfpathlineto{\pgfqpoint{3.292969in}{2.936053in}}%
\pgfpathlineto{\pgfqpoint{3.292969in}{2.933104in}}%
\pgfpathmoveto{\pgfqpoint{3.283888in}{2.936053in}}%
\pgfpathlineto{\pgfqpoint{3.283888in}{2.936053in}}%
\pgfpathlineto{\pgfqpoint{3.283888in}{2.939002in}}%
\pgfpathlineto{\pgfqpoint{3.288428in}{2.939002in}}%
\pgfpathlineto{\pgfqpoint{3.288428in}{2.936053in}}%
\pgfpathmoveto{\pgfqpoint{3.283888in}{2.939002in}}%
\pgfpathlineto{\pgfqpoint{3.283888in}{2.939002in}}%
\pgfpathlineto{\pgfqpoint{3.283888in}{2.941952in}}%
\pgfpathlineto{\pgfqpoint{3.288428in}{2.941952in}}%
\pgfpathlineto{\pgfqpoint{3.288428in}{2.939002in}}%
\pgfpathmoveto{\pgfqpoint{3.288428in}{2.936053in}}%
\pgfpathlineto{\pgfqpoint{3.288428in}{2.936053in}}%
\pgfpathlineto{\pgfqpoint{3.288428in}{2.939002in}}%
\pgfpathlineto{\pgfqpoint{3.292969in}{2.939002in}}%
\pgfpathlineto{\pgfqpoint{3.292969in}{2.936053in}}%
\pgfpathmoveto{\pgfqpoint{3.288428in}{2.939002in}}%
\pgfpathlineto{\pgfqpoint{3.288428in}{2.939002in}}%
\pgfpathlineto{\pgfqpoint{3.288428in}{2.941952in}}%
\pgfpathlineto{\pgfqpoint{3.292969in}{2.941952in}}%
\pgfpathlineto{\pgfqpoint{3.292969in}{2.939002in}}%
\pgfpathmoveto{\pgfqpoint{3.274806in}{2.941952in}}%
\pgfpathlineto{\pgfqpoint{3.274806in}{2.941952in}}%
\pgfpathlineto{\pgfqpoint{3.274806in}{2.944901in}}%
\pgfpathlineto{\pgfqpoint{3.279347in}{2.944901in}}%
\pgfpathlineto{\pgfqpoint{3.279347in}{2.941952in}}%
\pgfpathmoveto{\pgfqpoint{3.274806in}{2.944901in}}%
\pgfpathlineto{\pgfqpoint{3.274806in}{2.944901in}}%
\pgfpathlineto{\pgfqpoint{3.274806in}{2.947850in}}%
\pgfpathlineto{\pgfqpoint{3.279347in}{2.947850in}}%
\pgfpathlineto{\pgfqpoint{3.279347in}{2.944901in}}%
\pgfpathmoveto{\pgfqpoint{3.279347in}{2.941952in}}%
\pgfpathlineto{\pgfqpoint{3.279347in}{2.941952in}}%
\pgfpathlineto{\pgfqpoint{3.279347in}{2.944901in}}%
\pgfpathlineto{\pgfqpoint{3.283888in}{2.944901in}}%
\pgfpathlineto{\pgfqpoint{3.283888in}{2.941952in}}%
\pgfpathmoveto{\pgfqpoint{3.279347in}{2.944901in}}%
\pgfpathlineto{\pgfqpoint{3.279347in}{2.944901in}}%
\pgfpathlineto{\pgfqpoint{3.279347in}{2.947850in}}%
\pgfpathlineto{\pgfqpoint{3.283888in}{2.947850in}}%
\pgfpathlineto{\pgfqpoint{3.283888in}{2.944901in}}%
\pgfpathmoveto{\pgfqpoint{3.274806in}{2.947850in}}%
\pgfpathlineto{\pgfqpoint{3.274806in}{2.947850in}}%
\pgfpathlineto{\pgfqpoint{3.274806in}{2.950799in}}%
\pgfpathlineto{\pgfqpoint{3.279347in}{2.950799in}}%
\pgfpathlineto{\pgfqpoint{3.279347in}{2.947850in}}%
\pgfpathmoveto{\pgfqpoint{3.274806in}{2.950799in}}%
\pgfpathlineto{\pgfqpoint{3.274806in}{2.950799in}}%
\pgfpathlineto{\pgfqpoint{3.274806in}{2.953748in}}%
\pgfpathlineto{\pgfqpoint{3.279347in}{2.953748in}}%
\pgfpathlineto{\pgfqpoint{3.279347in}{2.950799in}}%
\pgfpathmoveto{\pgfqpoint{3.279347in}{2.947850in}}%
\pgfpathlineto{\pgfqpoint{3.279347in}{2.947850in}}%
\pgfpathlineto{\pgfqpoint{3.279347in}{2.950799in}}%
\pgfpathlineto{\pgfqpoint{3.283888in}{2.950799in}}%
\pgfpathlineto{\pgfqpoint{3.283888in}{2.947850in}}%
\pgfpathmoveto{\pgfqpoint{3.279347in}{2.950799in}}%
\pgfpathlineto{\pgfqpoint{3.279347in}{2.950799in}}%
\pgfpathlineto{\pgfqpoint{3.279347in}{2.953748in}}%
\pgfpathlineto{\pgfqpoint{3.283888in}{2.953748in}}%
\pgfpathlineto{\pgfqpoint{3.283888in}{2.950799in}}%
\pgfpathmoveto{\pgfqpoint{3.283888in}{2.941952in}}%
\pgfpathlineto{\pgfqpoint{3.283888in}{2.941952in}}%
\pgfpathlineto{\pgfqpoint{3.283888in}{2.944901in}}%
\pgfpathlineto{\pgfqpoint{3.288428in}{2.944901in}}%
\pgfpathlineto{\pgfqpoint{3.288428in}{2.941952in}}%
\pgfpathmoveto{\pgfqpoint{3.283888in}{2.944901in}}%
\pgfpathlineto{\pgfqpoint{3.283888in}{2.944901in}}%
\pgfpathlineto{\pgfqpoint{3.283888in}{2.947850in}}%
\pgfpathlineto{\pgfqpoint{3.288428in}{2.947850in}}%
\pgfpathlineto{\pgfqpoint{3.288428in}{2.944901in}}%
\pgfpathmoveto{\pgfqpoint{3.320214in}{2.882969in}}%
\pgfpathlineto{\pgfqpoint{3.320214in}{2.882969in}}%
\pgfpathlineto{\pgfqpoint{3.320214in}{2.885918in}}%
\pgfpathlineto{\pgfqpoint{3.324755in}{2.885918in}}%
\pgfpathlineto{\pgfqpoint{3.324755in}{2.882969in}}%
\pgfpathmoveto{\pgfqpoint{3.320214in}{2.885918in}}%
\pgfpathlineto{\pgfqpoint{3.320214in}{2.885918in}}%
\pgfpathlineto{\pgfqpoint{3.320214in}{2.888867in}}%
\pgfpathlineto{\pgfqpoint{3.324755in}{2.888867in}}%
\pgfpathlineto{\pgfqpoint{3.324755in}{2.885918in}}%
\pgfpathmoveto{\pgfqpoint{3.324755in}{2.882969in}}%
\pgfpathlineto{\pgfqpoint{3.324755in}{2.882969in}}%
\pgfpathlineto{\pgfqpoint{3.324755in}{2.885918in}}%
\pgfpathlineto{\pgfqpoint{3.329296in}{2.885918in}}%
\pgfpathlineto{\pgfqpoint{3.329296in}{2.882969in}}%
\pgfpathmoveto{\pgfqpoint{3.324755in}{2.885918in}}%
\pgfpathlineto{\pgfqpoint{3.324755in}{2.885918in}}%
\pgfpathlineto{\pgfqpoint{3.324755in}{2.888867in}}%
\pgfpathlineto{\pgfqpoint{3.329296in}{2.888867in}}%
\pgfpathlineto{\pgfqpoint{3.329296in}{2.885918in}}%
\pgfpathmoveto{\pgfqpoint{3.320214in}{2.888867in}}%
\pgfpathlineto{\pgfqpoint{3.320214in}{2.888867in}}%
\pgfpathlineto{\pgfqpoint{3.320214in}{2.891816in}}%
\pgfpathlineto{\pgfqpoint{3.324755in}{2.891816in}}%
\pgfpathlineto{\pgfqpoint{3.324755in}{2.888867in}}%
\pgfpathmoveto{\pgfqpoint{3.320214in}{2.891816in}}%
\pgfpathlineto{\pgfqpoint{3.320214in}{2.891816in}}%
\pgfpathlineto{\pgfqpoint{3.320214in}{2.894765in}}%
\pgfpathlineto{\pgfqpoint{3.324755in}{2.894765in}}%
\pgfpathlineto{\pgfqpoint{3.324755in}{2.891816in}}%
\pgfpathmoveto{\pgfqpoint{3.324755in}{2.888867in}}%
\pgfpathlineto{\pgfqpoint{3.324755in}{2.888867in}}%
\pgfpathlineto{\pgfqpoint{3.324755in}{2.891816in}}%
\pgfpathlineto{\pgfqpoint{3.329296in}{2.891816in}}%
\pgfpathlineto{\pgfqpoint{3.329296in}{2.888867in}}%
\pgfpathmoveto{\pgfqpoint{3.324755in}{2.891816in}}%
\pgfpathlineto{\pgfqpoint{3.324755in}{2.891816in}}%
\pgfpathlineto{\pgfqpoint{3.324755in}{2.894765in}}%
\pgfpathlineto{\pgfqpoint{3.329296in}{2.894765in}}%
\pgfpathlineto{\pgfqpoint{3.329296in}{2.891816in}}%
\pgfpathmoveto{\pgfqpoint{3.311132in}{2.894765in}}%
\pgfpathlineto{\pgfqpoint{3.311132in}{2.894765in}}%
\pgfpathlineto{\pgfqpoint{3.311132in}{2.897715in}}%
\pgfpathlineto{\pgfqpoint{3.315673in}{2.897715in}}%
\pgfpathlineto{\pgfqpoint{3.315673in}{2.894765in}}%
\pgfpathmoveto{\pgfqpoint{3.311132in}{2.897715in}}%
\pgfpathlineto{\pgfqpoint{3.311132in}{2.897715in}}%
\pgfpathlineto{\pgfqpoint{3.311132in}{2.900664in}}%
\pgfpathlineto{\pgfqpoint{3.315673in}{2.900664in}}%
\pgfpathlineto{\pgfqpoint{3.315673in}{2.897715in}}%
\pgfpathmoveto{\pgfqpoint{3.315673in}{2.894765in}}%
\pgfpathlineto{\pgfqpoint{3.315673in}{2.894765in}}%
\pgfpathlineto{\pgfqpoint{3.315673in}{2.897715in}}%
\pgfpathlineto{\pgfqpoint{3.320214in}{2.897715in}}%
\pgfpathlineto{\pgfqpoint{3.320214in}{2.894765in}}%
\pgfpathmoveto{\pgfqpoint{3.315673in}{2.897715in}}%
\pgfpathlineto{\pgfqpoint{3.315673in}{2.897715in}}%
\pgfpathlineto{\pgfqpoint{3.315673in}{2.900664in}}%
\pgfpathlineto{\pgfqpoint{3.320214in}{2.900664in}}%
\pgfpathlineto{\pgfqpoint{3.320214in}{2.897715in}}%
\pgfpathmoveto{\pgfqpoint{3.311132in}{2.900664in}}%
\pgfpathlineto{\pgfqpoint{3.311132in}{2.900664in}}%
\pgfpathlineto{\pgfqpoint{3.311132in}{2.903613in}}%
\pgfpathlineto{\pgfqpoint{3.315673in}{2.903613in}}%
\pgfpathlineto{\pgfqpoint{3.315673in}{2.900664in}}%
\pgfpathmoveto{\pgfqpoint{3.311132in}{2.903613in}}%
\pgfpathlineto{\pgfqpoint{3.311132in}{2.903613in}}%
\pgfpathlineto{\pgfqpoint{3.311132in}{2.906562in}}%
\pgfpathlineto{\pgfqpoint{3.315673in}{2.906562in}}%
\pgfpathlineto{\pgfqpoint{3.315673in}{2.903613in}}%
\pgfpathmoveto{\pgfqpoint{3.315673in}{2.900664in}}%
\pgfpathlineto{\pgfqpoint{3.315673in}{2.900664in}}%
\pgfpathlineto{\pgfqpoint{3.315673in}{2.903613in}}%
\pgfpathlineto{\pgfqpoint{3.320214in}{2.903613in}}%
\pgfpathlineto{\pgfqpoint{3.320214in}{2.900664in}}%
\pgfpathmoveto{\pgfqpoint{3.315673in}{2.903613in}}%
\pgfpathlineto{\pgfqpoint{3.315673in}{2.903613in}}%
\pgfpathlineto{\pgfqpoint{3.315673in}{2.906562in}}%
\pgfpathlineto{\pgfqpoint{3.320214in}{2.906562in}}%
\pgfpathlineto{\pgfqpoint{3.320214in}{2.903613in}}%
\pgfpathmoveto{\pgfqpoint{3.320214in}{2.894765in}}%
\pgfpathlineto{\pgfqpoint{3.320214in}{2.894765in}}%
\pgfpathlineto{\pgfqpoint{3.320214in}{2.897715in}}%
\pgfpathlineto{\pgfqpoint{3.324755in}{2.897715in}}%
\pgfpathlineto{\pgfqpoint{3.324755in}{2.894765in}}%
\pgfpathmoveto{\pgfqpoint{3.320214in}{2.897715in}}%
\pgfpathlineto{\pgfqpoint{3.320214in}{2.897715in}}%
\pgfpathlineto{\pgfqpoint{3.320214in}{2.900664in}}%
\pgfpathlineto{\pgfqpoint{3.324755in}{2.900664in}}%
\pgfpathlineto{\pgfqpoint{3.324755in}{2.897715in}}%
\pgfpathmoveto{\pgfqpoint{3.338377in}{2.859376in}}%
\pgfpathlineto{\pgfqpoint{3.338377in}{2.859376in}}%
\pgfpathlineto{\pgfqpoint{3.338377in}{2.862325in}}%
\pgfpathlineto{\pgfqpoint{3.342918in}{2.862325in}}%
\pgfpathlineto{\pgfqpoint{3.342918in}{2.859376in}}%
\pgfpathmoveto{\pgfqpoint{3.338377in}{2.862325in}}%
\pgfpathlineto{\pgfqpoint{3.338377in}{2.862325in}}%
\pgfpathlineto{\pgfqpoint{3.338377in}{2.865274in}}%
\pgfpathlineto{\pgfqpoint{3.342918in}{2.865274in}}%
\pgfpathlineto{\pgfqpoint{3.342918in}{2.862325in}}%
\pgfpathmoveto{\pgfqpoint{3.342918in}{2.859376in}}%
\pgfpathlineto{\pgfqpoint{3.342918in}{2.859376in}}%
\pgfpathlineto{\pgfqpoint{3.342918in}{2.862325in}}%
\pgfpathlineto{\pgfqpoint{3.347459in}{2.862325in}}%
\pgfpathlineto{\pgfqpoint{3.347459in}{2.859376in}}%
\pgfpathmoveto{\pgfqpoint{3.342918in}{2.862325in}}%
\pgfpathlineto{\pgfqpoint{3.342918in}{2.862325in}}%
\pgfpathlineto{\pgfqpoint{3.342918in}{2.865274in}}%
\pgfpathlineto{\pgfqpoint{3.347459in}{2.865274in}}%
\pgfpathlineto{\pgfqpoint{3.347459in}{2.862325in}}%
\pgfpathmoveto{\pgfqpoint{3.338377in}{2.865274in}}%
\pgfpathlineto{\pgfqpoint{3.338377in}{2.865274in}}%
\pgfpathlineto{\pgfqpoint{3.338377in}{2.868223in}}%
\pgfpathlineto{\pgfqpoint{3.342918in}{2.868223in}}%
\pgfpathlineto{\pgfqpoint{3.342918in}{2.865274in}}%
\pgfpathmoveto{\pgfqpoint{3.338377in}{2.868223in}}%
\pgfpathlineto{\pgfqpoint{3.338377in}{2.868223in}}%
\pgfpathlineto{\pgfqpoint{3.338377in}{2.871172in}}%
\pgfpathlineto{\pgfqpoint{3.342918in}{2.871172in}}%
\pgfpathlineto{\pgfqpoint{3.342918in}{2.868223in}}%
\pgfpathmoveto{\pgfqpoint{3.342918in}{2.865274in}}%
\pgfpathlineto{\pgfqpoint{3.342918in}{2.865274in}}%
\pgfpathlineto{\pgfqpoint{3.342918in}{2.868223in}}%
\pgfpathlineto{\pgfqpoint{3.347459in}{2.868223in}}%
\pgfpathlineto{\pgfqpoint{3.347459in}{2.865274in}}%
\pgfpathmoveto{\pgfqpoint{3.342918in}{2.868223in}}%
\pgfpathlineto{\pgfqpoint{3.342918in}{2.868223in}}%
\pgfpathlineto{\pgfqpoint{3.342918in}{2.871172in}}%
\pgfpathlineto{\pgfqpoint{3.347459in}{2.871172in}}%
\pgfpathlineto{\pgfqpoint{3.347459in}{2.868223in}}%
\pgfpathmoveto{\pgfqpoint{3.329296in}{2.871172in}}%
\pgfpathlineto{\pgfqpoint{3.329296in}{2.871172in}}%
\pgfpathlineto{\pgfqpoint{3.329296in}{2.874121in}}%
\pgfpathlineto{\pgfqpoint{3.333836in}{2.874121in}}%
\pgfpathlineto{\pgfqpoint{3.333836in}{2.871172in}}%
\pgfpathmoveto{\pgfqpoint{3.329296in}{2.874121in}}%
\pgfpathlineto{\pgfqpoint{3.329296in}{2.874121in}}%
\pgfpathlineto{\pgfqpoint{3.329296in}{2.877071in}}%
\pgfpathlineto{\pgfqpoint{3.333836in}{2.877071in}}%
\pgfpathlineto{\pgfqpoint{3.333836in}{2.874121in}}%
\pgfpathmoveto{\pgfqpoint{3.333836in}{2.871172in}}%
\pgfpathlineto{\pgfqpoint{3.333836in}{2.871172in}}%
\pgfpathlineto{\pgfqpoint{3.333836in}{2.874121in}}%
\pgfpathlineto{\pgfqpoint{3.338377in}{2.874121in}}%
\pgfpathlineto{\pgfqpoint{3.338377in}{2.871172in}}%
\pgfpathmoveto{\pgfqpoint{3.333836in}{2.874121in}}%
\pgfpathlineto{\pgfqpoint{3.333836in}{2.874121in}}%
\pgfpathlineto{\pgfqpoint{3.333836in}{2.877071in}}%
\pgfpathlineto{\pgfqpoint{3.338377in}{2.877071in}}%
\pgfpathlineto{\pgfqpoint{3.338377in}{2.874121in}}%
\pgfpathmoveto{\pgfqpoint{3.329296in}{2.877071in}}%
\pgfpathlineto{\pgfqpoint{3.329296in}{2.877071in}}%
\pgfpathlineto{\pgfqpoint{3.329296in}{2.880020in}}%
\pgfpathlineto{\pgfqpoint{3.333836in}{2.880020in}}%
\pgfpathlineto{\pgfqpoint{3.333836in}{2.877071in}}%
\pgfpathmoveto{\pgfqpoint{3.329296in}{2.880020in}}%
\pgfpathlineto{\pgfqpoint{3.329296in}{2.880020in}}%
\pgfpathlineto{\pgfqpoint{3.329296in}{2.882969in}}%
\pgfpathlineto{\pgfqpoint{3.333836in}{2.882969in}}%
\pgfpathlineto{\pgfqpoint{3.333836in}{2.880020in}}%
\pgfpathmoveto{\pgfqpoint{3.333836in}{2.877071in}}%
\pgfpathlineto{\pgfqpoint{3.333836in}{2.877071in}}%
\pgfpathlineto{\pgfqpoint{3.333836in}{2.880020in}}%
\pgfpathlineto{\pgfqpoint{3.338377in}{2.880020in}}%
\pgfpathlineto{\pgfqpoint{3.338377in}{2.877071in}}%
\pgfpathmoveto{\pgfqpoint{3.333836in}{2.880020in}}%
\pgfpathlineto{\pgfqpoint{3.333836in}{2.880020in}}%
\pgfpathlineto{\pgfqpoint{3.333836in}{2.882969in}}%
\pgfpathlineto{\pgfqpoint{3.338377in}{2.882969in}}%
\pgfpathlineto{\pgfqpoint{3.338377in}{2.880020in}}%
\pgfpathmoveto{\pgfqpoint{3.338377in}{2.871172in}}%
\pgfpathlineto{\pgfqpoint{3.338377in}{2.871172in}}%
\pgfpathlineto{\pgfqpoint{3.338377in}{2.874121in}}%
\pgfpathlineto{\pgfqpoint{3.342918in}{2.874121in}}%
\pgfpathlineto{\pgfqpoint{3.342918in}{2.871172in}}%
\pgfpathmoveto{\pgfqpoint{3.338377in}{2.874121in}}%
\pgfpathlineto{\pgfqpoint{3.338377in}{2.874121in}}%
\pgfpathlineto{\pgfqpoint{3.338377in}{2.877071in}}%
\pgfpathlineto{\pgfqpoint{3.342918in}{2.877071in}}%
\pgfpathlineto{\pgfqpoint{3.342918in}{2.874121in}}%
\pgfpathmoveto{\pgfqpoint{3.347459in}{2.859376in}}%
\pgfpathlineto{\pgfqpoint{3.347459in}{2.859376in}}%
\pgfpathlineto{\pgfqpoint{3.347459in}{2.862325in}}%
\pgfpathlineto{\pgfqpoint{3.352000in}{2.862325in}}%
\pgfpathlineto{\pgfqpoint{3.352000in}{2.859376in}}%
\pgfpathmoveto{\pgfqpoint{3.347459in}{2.862325in}}%
\pgfpathlineto{\pgfqpoint{3.347459in}{2.862325in}}%
\pgfpathlineto{\pgfqpoint{3.347459in}{2.865274in}}%
\pgfpathlineto{\pgfqpoint{3.352000in}{2.865274in}}%
\pgfpathlineto{\pgfqpoint{3.352000in}{2.862325in}}%
\pgfpathmoveto{\pgfqpoint{3.329296in}{2.882969in}}%
\pgfpathlineto{\pgfqpoint{3.329296in}{2.882969in}}%
\pgfpathlineto{\pgfqpoint{3.329296in}{2.885918in}}%
\pgfpathlineto{\pgfqpoint{3.333836in}{2.885918in}}%
\pgfpathlineto{\pgfqpoint{3.333836in}{2.882969in}}%
\pgfpathmoveto{\pgfqpoint{3.329296in}{2.885918in}}%
\pgfpathlineto{\pgfqpoint{3.329296in}{2.885918in}}%
\pgfpathlineto{\pgfqpoint{3.329296in}{2.888867in}}%
\pgfpathlineto{\pgfqpoint{3.333836in}{2.888867in}}%
\pgfpathlineto{\pgfqpoint{3.333836in}{2.885918in}}%
\pgfpathmoveto{\pgfqpoint{3.302051in}{2.906562in}}%
\pgfpathlineto{\pgfqpoint{3.302051in}{2.906562in}}%
\pgfpathlineto{\pgfqpoint{3.302051in}{2.909511in}}%
\pgfpathlineto{\pgfqpoint{3.306592in}{2.909511in}}%
\pgfpathlineto{\pgfqpoint{3.306592in}{2.906562in}}%
\pgfpathmoveto{\pgfqpoint{3.302051in}{2.909511in}}%
\pgfpathlineto{\pgfqpoint{3.302051in}{2.909511in}}%
\pgfpathlineto{\pgfqpoint{3.302051in}{2.912460in}}%
\pgfpathlineto{\pgfqpoint{3.306592in}{2.912460in}}%
\pgfpathlineto{\pgfqpoint{3.306592in}{2.909511in}}%
\pgfpathmoveto{\pgfqpoint{3.306592in}{2.906562in}}%
\pgfpathlineto{\pgfqpoint{3.306592in}{2.906562in}}%
\pgfpathlineto{\pgfqpoint{3.306592in}{2.909511in}}%
\pgfpathlineto{\pgfqpoint{3.311132in}{2.909511in}}%
\pgfpathlineto{\pgfqpoint{3.311132in}{2.906562in}}%
\pgfpathmoveto{\pgfqpoint{3.306592in}{2.909511in}}%
\pgfpathlineto{\pgfqpoint{3.306592in}{2.909511in}}%
\pgfpathlineto{\pgfqpoint{3.306592in}{2.912460in}}%
\pgfpathlineto{\pgfqpoint{3.311132in}{2.912460in}}%
\pgfpathlineto{\pgfqpoint{3.311132in}{2.909511in}}%
\pgfpathmoveto{\pgfqpoint{3.302051in}{2.912460in}}%
\pgfpathlineto{\pgfqpoint{3.302051in}{2.912460in}}%
\pgfpathlineto{\pgfqpoint{3.302051in}{2.915409in}}%
\pgfpathlineto{\pgfqpoint{3.306592in}{2.915409in}}%
\pgfpathlineto{\pgfqpoint{3.306592in}{2.912460in}}%
\pgfpathmoveto{\pgfqpoint{3.302051in}{2.915409in}}%
\pgfpathlineto{\pgfqpoint{3.302051in}{2.915409in}}%
\pgfpathlineto{\pgfqpoint{3.302051in}{2.918359in}}%
\pgfpathlineto{\pgfqpoint{3.306592in}{2.918359in}}%
\pgfpathlineto{\pgfqpoint{3.306592in}{2.915409in}}%
\pgfpathmoveto{\pgfqpoint{3.306592in}{2.912460in}}%
\pgfpathlineto{\pgfqpoint{3.306592in}{2.912460in}}%
\pgfpathlineto{\pgfqpoint{3.306592in}{2.915409in}}%
\pgfpathlineto{\pgfqpoint{3.311132in}{2.915409in}}%
\pgfpathlineto{\pgfqpoint{3.311132in}{2.912460in}}%
\pgfpathmoveto{\pgfqpoint{3.306592in}{2.915409in}}%
\pgfpathlineto{\pgfqpoint{3.306592in}{2.915409in}}%
\pgfpathlineto{\pgfqpoint{3.306592in}{2.918359in}}%
\pgfpathlineto{\pgfqpoint{3.311132in}{2.918359in}}%
\pgfpathlineto{\pgfqpoint{3.311132in}{2.915409in}}%
\pgfpathmoveto{\pgfqpoint{3.292969in}{2.918359in}}%
\pgfpathlineto{\pgfqpoint{3.292969in}{2.918359in}}%
\pgfpathlineto{\pgfqpoint{3.292969in}{2.921308in}}%
\pgfpathlineto{\pgfqpoint{3.297510in}{2.921308in}}%
\pgfpathlineto{\pgfqpoint{3.297510in}{2.918359in}}%
\pgfpathmoveto{\pgfqpoint{3.292969in}{2.921308in}}%
\pgfpathlineto{\pgfqpoint{3.292969in}{2.921308in}}%
\pgfpathlineto{\pgfqpoint{3.292969in}{2.924257in}}%
\pgfpathlineto{\pgfqpoint{3.297510in}{2.924257in}}%
\pgfpathlineto{\pgfqpoint{3.297510in}{2.921308in}}%
\pgfpathmoveto{\pgfqpoint{3.297510in}{2.918359in}}%
\pgfpathlineto{\pgfqpoint{3.297510in}{2.918359in}}%
\pgfpathlineto{\pgfqpoint{3.297510in}{2.921308in}}%
\pgfpathlineto{\pgfqpoint{3.302051in}{2.921308in}}%
\pgfpathlineto{\pgfqpoint{3.302051in}{2.918359in}}%
\pgfpathmoveto{\pgfqpoint{3.297510in}{2.921308in}}%
\pgfpathlineto{\pgfqpoint{3.297510in}{2.921308in}}%
\pgfpathlineto{\pgfqpoint{3.297510in}{2.924257in}}%
\pgfpathlineto{\pgfqpoint{3.302051in}{2.924257in}}%
\pgfpathlineto{\pgfqpoint{3.302051in}{2.921308in}}%
\pgfpathmoveto{\pgfqpoint{3.292969in}{2.924257in}}%
\pgfpathlineto{\pgfqpoint{3.292969in}{2.924257in}}%
\pgfpathlineto{\pgfqpoint{3.292969in}{2.927206in}}%
\pgfpathlineto{\pgfqpoint{3.297510in}{2.927206in}}%
\pgfpathlineto{\pgfqpoint{3.297510in}{2.924257in}}%
\pgfpathmoveto{\pgfqpoint{3.292969in}{2.927206in}}%
\pgfpathlineto{\pgfqpoint{3.292969in}{2.927206in}}%
\pgfpathlineto{\pgfqpoint{3.292969in}{2.930155in}}%
\pgfpathlineto{\pgfqpoint{3.297510in}{2.930155in}}%
\pgfpathlineto{\pgfqpoint{3.297510in}{2.927206in}}%
\pgfpathmoveto{\pgfqpoint{3.297510in}{2.924257in}}%
\pgfpathlineto{\pgfqpoint{3.297510in}{2.924257in}}%
\pgfpathlineto{\pgfqpoint{3.297510in}{2.927206in}}%
\pgfpathlineto{\pgfqpoint{3.302051in}{2.927206in}}%
\pgfpathlineto{\pgfqpoint{3.302051in}{2.924257in}}%
\pgfpathmoveto{\pgfqpoint{3.297510in}{2.927206in}}%
\pgfpathlineto{\pgfqpoint{3.297510in}{2.927206in}}%
\pgfpathlineto{\pgfqpoint{3.297510in}{2.930155in}}%
\pgfpathlineto{\pgfqpoint{3.302051in}{2.930155in}}%
\pgfpathlineto{\pgfqpoint{3.302051in}{2.927206in}}%
\pgfpathmoveto{\pgfqpoint{3.302051in}{2.918359in}}%
\pgfpathlineto{\pgfqpoint{3.302051in}{2.918359in}}%
\pgfpathlineto{\pgfqpoint{3.302051in}{2.921308in}}%
\pgfpathlineto{\pgfqpoint{3.306592in}{2.921308in}}%
\pgfpathlineto{\pgfqpoint{3.306592in}{2.918359in}}%
\pgfpathmoveto{\pgfqpoint{3.302051in}{2.921308in}}%
\pgfpathlineto{\pgfqpoint{3.302051in}{2.921308in}}%
\pgfpathlineto{\pgfqpoint{3.302051in}{2.924257in}}%
\pgfpathlineto{\pgfqpoint{3.306592in}{2.924257in}}%
\pgfpathlineto{\pgfqpoint{3.306592in}{2.921308in}}%
\pgfpathmoveto{\pgfqpoint{3.311132in}{2.906562in}}%
\pgfpathlineto{\pgfqpoint{3.311132in}{2.906562in}}%
\pgfpathlineto{\pgfqpoint{3.311132in}{2.909511in}}%
\pgfpathlineto{\pgfqpoint{3.315673in}{2.909511in}}%
\pgfpathlineto{\pgfqpoint{3.315673in}{2.906562in}}%
\pgfpathmoveto{\pgfqpoint{3.311132in}{2.909511in}}%
\pgfpathlineto{\pgfqpoint{3.311132in}{2.909511in}}%
\pgfpathlineto{\pgfqpoint{3.311132in}{2.912460in}}%
\pgfpathlineto{\pgfqpoint{3.315673in}{2.912460in}}%
\pgfpathlineto{\pgfqpoint{3.315673in}{2.909511in}}%
\pgfpathmoveto{\pgfqpoint{3.292969in}{2.930155in}}%
\pgfpathlineto{\pgfqpoint{3.292969in}{2.930155in}}%
\pgfpathlineto{\pgfqpoint{3.292969in}{2.933104in}}%
\pgfpathlineto{\pgfqpoint{3.297510in}{2.933104in}}%
\pgfpathlineto{\pgfqpoint{3.297510in}{2.930155in}}%
\pgfpathmoveto{\pgfqpoint{3.292969in}{2.933104in}}%
\pgfpathlineto{\pgfqpoint{3.292969in}{2.933104in}}%
\pgfpathlineto{\pgfqpoint{3.292969in}{2.936053in}}%
\pgfpathlineto{\pgfqpoint{3.297510in}{2.936053in}}%
\pgfpathlineto{\pgfqpoint{3.297510in}{2.933104in}}%
\pgfpathmoveto{\pgfqpoint{3.247561in}{2.977343in}}%
\pgfpathlineto{\pgfqpoint{3.247561in}{2.977343in}}%
\pgfpathlineto{\pgfqpoint{3.247561in}{2.980292in}}%
\pgfpathlineto{\pgfqpoint{3.252102in}{2.980292in}}%
\pgfpathlineto{\pgfqpoint{3.252102in}{2.977343in}}%
\pgfpathmoveto{\pgfqpoint{3.247561in}{2.980292in}}%
\pgfpathlineto{\pgfqpoint{3.247561in}{2.980292in}}%
\pgfpathlineto{\pgfqpoint{3.247561in}{2.983241in}}%
\pgfpathlineto{\pgfqpoint{3.252102in}{2.983241in}}%
\pgfpathlineto{\pgfqpoint{3.252102in}{2.980292in}}%
\pgfpathmoveto{\pgfqpoint{3.252102in}{2.977343in}}%
\pgfpathlineto{\pgfqpoint{3.252102in}{2.977343in}}%
\pgfpathlineto{\pgfqpoint{3.252102in}{2.980292in}}%
\pgfpathlineto{\pgfqpoint{3.256643in}{2.980292in}}%
\pgfpathlineto{\pgfqpoint{3.256643in}{2.977343in}}%
\pgfpathmoveto{\pgfqpoint{3.252102in}{2.980292in}}%
\pgfpathlineto{\pgfqpoint{3.252102in}{2.980292in}}%
\pgfpathlineto{\pgfqpoint{3.252102in}{2.983241in}}%
\pgfpathlineto{\pgfqpoint{3.256643in}{2.983241in}}%
\pgfpathlineto{\pgfqpoint{3.256643in}{2.980292in}}%
\pgfpathmoveto{\pgfqpoint{3.247561in}{2.983241in}}%
\pgfpathlineto{\pgfqpoint{3.247561in}{2.983241in}}%
\pgfpathlineto{\pgfqpoint{3.247561in}{2.986191in}}%
\pgfpathlineto{\pgfqpoint{3.252102in}{2.986191in}}%
\pgfpathlineto{\pgfqpoint{3.252102in}{2.983241in}}%
\pgfpathmoveto{\pgfqpoint{3.247561in}{2.986191in}}%
\pgfpathlineto{\pgfqpoint{3.247561in}{2.986191in}}%
\pgfpathlineto{\pgfqpoint{3.247561in}{2.989140in}}%
\pgfpathlineto{\pgfqpoint{3.252102in}{2.989140in}}%
\pgfpathlineto{\pgfqpoint{3.252102in}{2.986191in}}%
\pgfpathmoveto{\pgfqpoint{3.252102in}{2.983241in}}%
\pgfpathlineto{\pgfqpoint{3.252102in}{2.983241in}}%
\pgfpathlineto{\pgfqpoint{3.252102in}{2.986191in}}%
\pgfpathlineto{\pgfqpoint{3.256643in}{2.986191in}}%
\pgfpathlineto{\pgfqpoint{3.256643in}{2.983241in}}%
\pgfpathmoveto{\pgfqpoint{3.252102in}{2.986191in}}%
\pgfpathlineto{\pgfqpoint{3.252102in}{2.986191in}}%
\pgfpathlineto{\pgfqpoint{3.252102in}{2.989140in}}%
\pgfpathlineto{\pgfqpoint{3.256643in}{2.989140in}}%
\pgfpathlineto{\pgfqpoint{3.256643in}{2.986191in}}%
\pgfpathmoveto{\pgfqpoint{3.238480in}{2.989140in}}%
\pgfpathlineto{\pgfqpoint{3.238480in}{2.989140in}}%
\pgfpathlineto{\pgfqpoint{3.238480in}{2.992089in}}%
\pgfpathlineto{\pgfqpoint{3.243020in}{2.992089in}}%
\pgfpathlineto{\pgfqpoint{3.243020in}{2.989140in}}%
\pgfpathmoveto{\pgfqpoint{3.238480in}{2.992089in}}%
\pgfpathlineto{\pgfqpoint{3.238480in}{2.992089in}}%
\pgfpathlineto{\pgfqpoint{3.238480in}{2.995038in}}%
\pgfpathlineto{\pgfqpoint{3.243020in}{2.995038in}}%
\pgfpathlineto{\pgfqpoint{3.243020in}{2.992089in}}%
\pgfpathmoveto{\pgfqpoint{3.243020in}{2.989140in}}%
\pgfpathlineto{\pgfqpoint{3.243020in}{2.989140in}}%
\pgfpathlineto{\pgfqpoint{3.243020in}{2.992089in}}%
\pgfpathlineto{\pgfqpoint{3.247561in}{2.992089in}}%
\pgfpathlineto{\pgfqpoint{3.247561in}{2.989140in}}%
\pgfpathmoveto{\pgfqpoint{3.243020in}{2.992089in}}%
\pgfpathlineto{\pgfqpoint{3.243020in}{2.992089in}}%
\pgfpathlineto{\pgfqpoint{3.243020in}{2.995038in}}%
\pgfpathlineto{\pgfqpoint{3.247561in}{2.995038in}}%
\pgfpathlineto{\pgfqpoint{3.247561in}{2.992089in}}%
\pgfpathmoveto{\pgfqpoint{3.238480in}{2.995038in}}%
\pgfpathlineto{\pgfqpoint{3.238480in}{2.995038in}}%
\pgfpathlineto{\pgfqpoint{3.238480in}{2.997988in}}%
\pgfpathlineto{\pgfqpoint{3.243020in}{2.997988in}}%
\pgfpathlineto{\pgfqpoint{3.243020in}{2.995038in}}%
\pgfpathmoveto{\pgfqpoint{3.238480in}{2.997988in}}%
\pgfpathlineto{\pgfqpoint{3.238480in}{2.997988in}}%
\pgfpathlineto{\pgfqpoint{3.238480in}{3.000937in}}%
\pgfpathlineto{\pgfqpoint{3.243020in}{3.000937in}}%
\pgfpathlineto{\pgfqpoint{3.243020in}{2.997988in}}%
\pgfpathmoveto{\pgfqpoint{3.243020in}{2.995038in}}%
\pgfpathlineto{\pgfqpoint{3.243020in}{2.995038in}}%
\pgfpathlineto{\pgfqpoint{3.243020in}{2.997988in}}%
\pgfpathlineto{\pgfqpoint{3.247561in}{2.997988in}}%
\pgfpathlineto{\pgfqpoint{3.247561in}{2.995038in}}%
\pgfpathmoveto{\pgfqpoint{3.243020in}{2.997988in}}%
\pgfpathlineto{\pgfqpoint{3.243020in}{2.997988in}}%
\pgfpathlineto{\pgfqpoint{3.243020in}{3.000937in}}%
\pgfpathlineto{\pgfqpoint{3.247561in}{3.000937in}}%
\pgfpathlineto{\pgfqpoint{3.247561in}{2.997988in}}%
\pgfpathmoveto{\pgfqpoint{3.247561in}{2.989140in}}%
\pgfpathlineto{\pgfqpoint{3.247561in}{2.989140in}}%
\pgfpathlineto{\pgfqpoint{3.247561in}{2.992089in}}%
\pgfpathlineto{\pgfqpoint{3.252102in}{2.992089in}}%
\pgfpathlineto{\pgfqpoint{3.252102in}{2.989140in}}%
\pgfpathmoveto{\pgfqpoint{3.247561in}{2.992089in}}%
\pgfpathlineto{\pgfqpoint{3.247561in}{2.992089in}}%
\pgfpathlineto{\pgfqpoint{3.247561in}{2.995038in}}%
\pgfpathlineto{\pgfqpoint{3.252102in}{2.995038in}}%
\pgfpathlineto{\pgfqpoint{3.252102in}{2.992089in}}%
\pgfpathmoveto{\pgfqpoint{3.265724in}{2.953748in}}%
\pgfpathlineto{\pgfqpoint{3.265724in}{2.953748in}}%
\pgfpathlineto{\pgfqpoint{3.265724in}{2.956697in}}%
\pgfpathlineto{\pgfqpoint{3.270265in}{2.956697in}}%
\pgfpathlineto{\pgfqpoint{3.270265in}{2.953748in}}%
\pgfpathmoveto{\pgfqpoint{3.265724in}{2.956697in}}%
\pgfpathlineto{\pgfqpoint{3.265724in}{2.956697in}}%
\pgfpathlineto{\pgfqpoint{3.265724in}{2.959647in}}%
\pgfpathlineto{\pgfqpoint{3.270265in}{2.959647in}}%
\pgfpathlineto{\pgfqpoint{3.270265in}{2.956697in}}%
\pgfpathmoveto{\pgfqpoint{3.270265in}{2.953748in}}%
\pgfpathlineto{\pgfqpoint{3.270265in}{2.953748in}}%
\pgfpathlineto{\pgfqpoint{3.270265in}{2.956697in}}%
\pgfpathlineto{\pgfqpoint{3.274806in}{2.956697in}}%
\pgfpathlineto{\pgfqpoint{3.274806in}{2.953748in}}%
\pgfpathmoveto{\pgfqpoint{3.270265in}{2.956697in}}%
\pgfpathlineto{\pgfqpoint{3.270265in}{2.956697in}}%
\pgfpathlineto{\pgfqpoint{3.270265in}{2.959647in}}%
\pgfpathlineto{\pgfqpoint{3.274806in}{2.959647in}}%
\pgfpathlineto{\pgfqpoint{3.274806in}{2.956697in}}%
\pgfpathmoveto{\pgfqpoint{3.265724in}{2.959647in}}%
\pgfpathlineto{\pgfqpoint{3.265724in}{2.959647in}}%
\pgfpathlineto{\pgfqpoint{3.265724in}{2.962596in}}%
\pgfpathlineto{\pgfqpoint{3.270265in}{2.962596in}}%
\pgfpathlineto{\pgfqpoint{3.270265in}{2.959647in}}%
\pgfpathmoveto{\pgfqpoint{3.265724in}{2.962596in}}%
\pgfpathlineto{\pgfqpoint{3.265724in}{2.962596in}}%
\pgfpathlineto{\pgfqpoint{3.265724in}{2.965545in}}%
\pgfpathlineto{\pgfqpoint{3.270265in}{2.965545in}}%
\pgfpathlineto{\pgfqpoint{3.270265in}{2.962596in}}%
\pgfpathmoveto{\pgfqpoint{3.270265in}{2.959647in}}%
\pgfpathlineto{\pgfqpoint{3.270265in}{2.959647in}}%
\pgfpathlineto{\pgfqpoint{3.270265in}{2.962596in}}%
\pgfpathlineto{\pgfqpoint{3.274806in}{2.962596in}}%
\pgfpathlineto{\pgfqpoint{3.274806in}{2.959647in}}%
\pgfpathmoveto{\pgfqpoint{3.270265in}{2.962596in}}%
\pgfpathlineto{\pgfqpoint{3.270265in}{2.962596in}}%
\pgfpathlineto{\pgfqpoint{3.270265in}{2.965545in}}%
\pgfpathlineto{\pgfqpoint{3.274806in}{2.965545in}}%
\pgfpathlineto{\pgfqpoint{3.274806in}{2.962596in}}%
\pgfpathmoveto{\pgfqpoint{3.256643in}{2.965545in}}%
\pgfpathlineto{\pgfqpoint{3.256643in}{2.965545in}}%
\pgfpathlineto{\pgfqpoint{3.256643in}{2.968495in}}%
\pgfpathlineto{\pgfqpoint{3.261184in}{2.968495in}}%
\pgfpathlineto{\pgfqpoint{3.261184in}{2.965545in}}%
\pgfpathmoveto{\pgfqpoint{3.256643in}{2.968495in}}%
\pgfpathlineto{\pgfqpoint{3.256643in}{2.968495in}}%
\pgfpathlineto{\pgfqpoint{3.256643in}{2.971444in}}%
\pgfpathlineto{\pgfqpoint{3.261184in}{2.971444in}}%
\pgfpathlineto{\pgfqpoint{3.261184in}{2.968495in}}%
\pgfpathmoveto{\pgfqpoint{3.261184in}{2.965545in}}%
\pgfpathlineto{\pgfqpoint{3.261184in}{2.965545in}}%
\pgfpathlineto{\pgfqpoint{3.261184in}{2.968495in}}%
\pgfpathlineto{\pgfqpoint{3.265724in}{2.968495in}}%
\pgfpathlineto{\pgfqpoint{3.265724in}{2.965545in}}%
\pgfpathmoveto{\pgfqpoint{3.261184in}{2.968495in}}%
\pgfpathlineto{\pgfqpoint{3.261184in}{2.968495in}}%
\pgfpathlineto{\pgfqpoint{3.261184in}{2.971444in}}%
\pgfpathlineto{\pgfqpoint{3.265724in}{2.971444in}}%
\pgfpathlineto{\pgfqpoint{3.265724in}{2.968495in}}%
\pgfpathmoveto{\pgfqpoint{3.256643in}{2.971444in}}%
\pgfpathlineto{\pgfqpoint{3.256643in}{2.971444in}}%
\pgfpathlineto{\pgfqpoint{3.256643in}{2.974393in}}%
\pgfpathlineto{\pgfqpoint{3.261184in}{2.974393in}}%
\pgfpathlineto{\pgfqpoint{3.261184in}{2.971444in}}%
\pgfpathmoveto{\pgfqpoint{3.256643in}{2.974393in}}%
\pgfpathlineto{\pgfqpoint{3.256643in}{2.974393in}}%
\pgfpathlineto{\pgfqpoint{3.256643in}{2.977343in}}%
\pgfpathlineto{\pgfqpoint{3.261184in}{2.977343in}}%
\pgfpathlineto{\pgfqpoint{3.261184in}{2.974393in}}%
\pgfpathmoveto{\pgfqpoint{3.261184in}{2.971444in}}%
\pgfpathlineto{\pgfqpoint{3.261184in}{2.971444in}}%
\pgfpathlineto{\pgfqpoint{3.261184in}{2.974393in}}%
\pgfpathlineto{\pgfqpoint{3.265724in}{2.974393in}}%
\pgfpathlineto{\pgfqpoint{3.265724in}{2.971444in}}%
\pgfpathmoveto{\pgfqpoint{3.261184in}{2.974393in}}%
\pgfpathlineto{\pgfqpoint{3.261184in}{2.974393in}}%
\pgfpathlineto{\pgfqpoint{3.261184in}{2.977343in}}%
\pgfpathlineto{\pgfqpoint{3.265724in}{2.977343in}}%
\pgfpathlineto{\pgfqpoint{3.265724in}{2.974393in}}%
\pgfpathmoveto{\pgfqpoint{3.265724in}{2.965545in}}%
\pgfpathlineto{\pgfqpoint{3.265724in}{2.965545in}}%
\pgfpathlineto{\pgfqpoint{3.265724in}{2.968495in}}%
\pgfpathlineto{\pgfqpoint{3.270265in}{2.968495in}}%
\pgfpathlineto{\pgfqpoint{3.270265in}{2.965545in}}%
\pgfpathmoveto{\pgfqpoint{3.265724in}{2.968495in}}%
\pgfpathlineto{\pgfqpoint{3.265724in}{2.968495in}}%
\pgfpathlineto{\pgfqpoint{3.265724in}{2.971444in}}%
\pgfpathlineto{\pgfqpoint{3.270265in}{2.971444in}}%
\pgfpathlineto{\pgfqpoint{3.270265in}{2.968495in}}%
\pgfpathmoveto{\pgfqpoint{3.274806in}{2.953748in}}%
\pgfpathlineto{\pgfqpoint{3.274806in}{2.953748in}}%
\pgfpathlineto{\pgfqpoint{3.274806in}{2.956697in}}%
\pgfpathlineto{\pgfqpoint{3.279347in}{2.956697in}}%
\pgfpathlineto{\pgfqpoint{3.279347in}{2.953748in}}%
\pgfpathmoveto{\pgfqpoint{3.274806in}{2.956697in}}%
\pgfpathlineto{\pgfqpoint{3.274806in}{2.956697in}}%
\pgfpathlineto{\pgfqpoint{3.274806in}{2.959647in}}%
\pgfpathlineto{\pgfqpoint{3.279347in}{2.959647in}}%
\pgfpathlineto{\pgfqpoint{3.279347in}{2.956697in}}%
\pgfpathmoveto{\pgfqpoint{3.256643in}{2.977343in}}%
\pgfpathlineto{\pgfqpoint{3.256643in}{2.977343in}}%
\pgfpathlineto{\pgfqpoint{3.256643in}{2.980292in}}%
\pgfpathlineto{\pgfqpoint{3.261184in}{2.980292in}}%
\pgfpathlineto{\pgfqpoint{3.261184in}{2.977343in}}%
\pgfpathmoveto{\pgfqpoint{3.256643in}{2.980292in}}%
\pgfpathlineto{\pgfqpoint{3.256643in}{2.980292in}}%
\pgfpathlineto{\pgfqpoint{3.256643in}{2.983241in}}%
\pgfpathlineto{\pgfqpoint{3.261184in}{2.983241in}}%
\pgfpathlineto{\pgfqpoint{3.261184in}{2.980292in}}%
\pgfpathmoveto{\pgfqpoint{3.229398in}{3.000937in}}%
\pgfpathlineto{\pgfqpoint{3.229398in}{3.000937in}}%
\pgfpathlineto{\pgfqpoint{3.229398in}{3.003886in}}%
\pgfpathlineto{\pgfqpoint{3.233939in}{3.003886in}}%
\pgfpathlineto{\pgfqpoint{3.233939in}{3.000937in}}%
\pgfpathmoveto{\pgfqpoint{3.229398in}{3.003886in}}%
\pgfpathlineto{\pgfqpoint{3.229398in}{3.003886in}}%
\pgfpathlineto{\pgfqpoint{3.229398in}{3.006836in}}%
\pgfpathlineto{\pgfqpoint{3.233939in}{3.006836in}}%
\pgfpathlineto{\pgfqpoint{3.233939in}{3.003886in}}%
\pgfpathmoveto{\pgfqpoint{3.233939in}{3.000937in}}%
\pgfpathlineto{\pgfqpoint{3.233939in}{3.000937in}}%
\pgfpathlineto{\pgfqpoint{3.233939in}{3.003886in}}%
\pgfpathlineto{\pgfqpoint{3.238480in}{3.003886in}}%
\pgfpathlineto{\pgfqpoint{3.238480in}{3.000937in}}%
\pgfpathmoveto{\pgfqpoint{3.233939in}{3.003886in}}%
\pgfpathlineto{\pgfqpoint{3.233939in}{3.003886in}}%
\pgfpathlineto{\pgfqpoint{3.233939in}{3.006836in}}%
\pgfpathlineto{\pgfqpoint{3.238480in}{3.006836in}}%
\pgfpathlineto{\pgfqpoint{3.238480in}{3.003886in}}%
\pgfpathmoveto{\pgfqpoint{3.229398in}{3.006836in}}%
\pgfpathlineto{\pgfqpoint{3.229398in}{3.006836in}}%
\pgfpathlineto{\pgfqpoint{3.229398in}{3.009785in}}%
\pgfpathlineto{\pgfqpoint{3.233939in}{3.009785in}}%
\pgfpathlineto{\pgfqpoint{3.233939in}{3.006836in}}%
\pgfpathmoveto{\pgfqpoint{3.229398in}{3.009785in}}%
\pgfpathlineto{\pgfqpoint{3.229398in}{3.009785in}}%
\pgfpathlineto{\pgfqpoint{3.229398in}{3.012734in}}%
\pgfpathlineto{\pgfqpoint{3.233939in}{3.012734in}}%
\pgfpathlineto{\pgfqpoint{3.233939in}{3.009785in}}%
\pgfpathmoveto{\pgfqpoint{3.233939in}{3.006836in}}%
\pgfpathlineto{\pgfqpoint{3.233939in}{3.006836in}}%
\pgfpathlineto{\pgfqpoint{3.233939in}{3.009785in}}%
\pgfpathlineto{\pgfqpoint{3.238480in}{3.009785in}}%
\pgfpathlineto{\pgfqpoint{3.238480in}{3.006836in}}%
\pgfpathmoveto{\pgfqpoint{3.233939in}{3.009785in}}%
\pgfpathlineto{\pgfqpoint{3.233939in}{3.009785in}}%
\pgfpathlineto{\pgfqpoint{3.233939in}{3.012734in}}%
\pgfpathlineto{\pgfqpoint{3.238480in}{3.012734in}}%
\pgfpathlineto{\pgfqpoint{3.238480in}{3.009785in}}%
\pgfpathmoveto{\pgfqpoint{3.220316in}{3.012734in}}%
\pgfpathlineto{\pgfqpoint{3.220316in}{3.012734in}}%
\pgfpathlineto{\pgfqpoint{3.220316in}{3.015684in}}%
\pgfpathlineto{\pgfqpoint{3.224857in}{3.015684in}}%
\pgfpathlineto{\pgfqpoint{3.224857in}{3.012734in}}%
\pgfpathmoveto{\pgfqpoint{3.220316in}{3.015684in}}%
\pgfpathlineto{\pgfqpoint{3.220316in}{3.015684in}}%
\pgfpathlineto{\pgfqpoint{3.220316in}{3.018633in}}%
\pgfpathlineto{\pgfqpoint{3.224857in}{3.018633in}}%
\pgfpathlineto{\pgfqpoint{3.224857in}{3.015684in}}%
\pgfpathmoveto{\pgfqpoint{3.224857in}{3.012734in}}%
\pgfpathlineto{\pgfqpoint{3.224857in}{3.012734in}}%
\pgfpathlineto{\pgfqpoint{3.224857in}{3.015684in}}%
\pgfpathlineto{\pgfqpoint{3.229398in}{3.015684in}}%
\pgfpathlineto{\pgfqpoint{3.229398in}{3.012734in}}%
\pgfpathmoveto{\pgfqpoint{3.224857in}{3.015684in}}%
\pgfpathlineto{\pgfqpoint{3.224857in}{3.015684in}}%
\pgfpathlineto{\pgfqpoint{3.224857in}{3.018633in}}%
\pgfpathlineto{\pgfqpoint{3.229398in}{3.018633in}}%
\pgfpathlineto{\pgfqpoint{3.229398in}{3.015684in}}%
\pgfpathmoveto{\pgfqpoint{3.220316in}{3.018633in}}%
\pgfpathlineto{\pgfqpoint{3.220316in}{3.018633in}}%
\pgfpathlineto{\pgfqpoint{3.220316in}{3.021582in}}%
\pgfpathlineto{\pgfqpoint{3.224857in}{3.021582in}}%
\pgfpathlineto{\pgfqpoint{3.224857in}{3.018633in}}%
\pgfpathmoveto{\pgfqpoint{3.220316in}{3.021582in}}%
\pgfpathlineto{\pgfqpoint{3.220316in}{3.021582in}}%
\pgfpathlineto{\pgfqpoint{3.220316in}{3.024531in}}%
\pgfpathlineto{\pgfqpoint{3.224857in}{3.024531in}}%
\pgfpathlineto{\pgfqpoint{3.224857in}{3.021582in}}%
\pgfpathmoveto{\pgfqpoint{3.224857in}{3.018633in}}%
\pgfpathlineto{\pgfqpoint{3.224857in}{3.018633in}}%
\pgfpathlineto{\pgfqpoint{3.224857in}{3.021582in}}%
\pgfpathlineto{\pgfqpoint{3.229398in}{3.021582in}}%
\pgfpathlineto{\pgfqpoint{3.229398in}{3.018633in}}%
\pgfpathmoveto{\pgfqpoint{3.224857in}{3.021582in}}%
\pgfpathlineto{\pgfqpoint{3.224857in}{3.021582in}}%
\pgfpathlineto{\pgfqpoint{3.224857in}{3.024531in}}%
\pgfpathlineto{\pgfqpoint{3.229398in}{3.024531in}}%
\pgfpathlineto{\pgfqpoint{3.229398in}{3.021582in}}%
\pgfpathmoveto{\pgfqpoint{3.229398in}{3.012734in}}%
\pgfpathlineto{\pgfqpoint{3.229398in}{3.012734in}}%
\pgfpathlineto{\pgfqpoint{3.229398in}{3.015684in}}%
\pgfpathlineto{\pgfqpoint{3.233939in}{3.015684in}}%
\pgfpathlineto{\pgfqpoint{3.233939in}{3.012734in}}%
\pgfpathmoveto{\pgfqpoint{3.229398in}{3.015684in}}%
\pgfpathlineto{\pgfqpoint{3.229398in}{3.015684in}}%
\pgfpathlineto{\pgfqpoint{3.229398in}{3.018633in}}%
\pgfpathlineto{\pgfqpoint{3.233939in}{3.018633in}}%
\pgfpathlineto{\pgfqpoint{3.233939in}{3.015684in}}%
\pgfpathmoveto{\pgfqpoint{3.238480in}{3.000937in}}%
\pgfpathlineto{\pgfqpoint{3.238480in}{3.000937in}}%
\pgfpathlineto{\pgfqpoint{3.238480in}{3.003886in}}%
\pgfpathlineto{\pgfqpoint{3.243020in}{3.003886in}}%
\pgfpathlineto{\pgfqpoint{3.243020in}{3.000937in}}%
\pgfpathmoveto{\pgfqpoint{3.238480in}{3.003886in}}%
\pgfpathlineto{\pgfqpoint{3.238480in}{3.003886in}}%
\pgfpathlineto{\pgfqpoint{3.238480in}{3.006836in}}%
\pgfpathlineto{\pgfqpoint{3.243020in}{3.006836in}}%
\pgfpathlineto{\pgfqpoint{3.243020in}{3.003886in}}%
\pgfpathmoveto{\pgfqpoint{3.220316in}{3.024531in}}%
\pgfpathlineto{\pgfqpoint{3.220316in}{3.024531in}}%
\pgfpathlineto{\pgfqpoint{3.220316in}{3.027481in}}%
\pgfpathlineto{\pgfqpoint{3.224857in}{3.027481in}}%
\pgfpathlineto{\pgfqpoint{3.224857in}{3.024531in}}%
\pgfpathmoveto{\pgfqpoint{3.220316in}{3.027481in}}%
\pgfpathlineto{\pgfqpoint{3.220316in}{3.027481in}}%
\pgfpathlineto{\pgfqpoint{3.220316in}{3.030430in}}%
\pgfpathlineto{\pgfqpoint{3.224857in}{3.030430in}}%
\pgfpathlineto{\pgfqpoint{3.224857in}{3.027481in}}%
\pgfpathmoveto{\pgfqpoint{3.365622in}{2.195803in}}%
\pgfpathlineto{\pgfqpoint{3.365622in}{2.195803in}}%
\pgfpathlineto{\pgfqpoint{3.365622in}{2.198752in}}%
\pgfpathlineto{\pgfqpoint{3.370163in}{2.198752in}}%
\pgfpathlineto{\pgfqpoint{3.370163in}{2.195803in}}%
\pgfpathmoveto{\pgfqpoint{3.365622in}{2.198752in}}%
\pgfpathlineto{\pgfqpoint{3.365622in}{2.198752in}}%
\pgfpathlineto{\pgfqpoint{3.365622in}{2.201701in}}%
\pgfpathlineto{\pgfqpoint{3.370163in}{2.201701in}}%
\pgfpathlineto{\pgfqpoint{3.370163in}{2.198752in}}%
\pgfpathmoveto{\pgfqpoint{3.365622in}{2.201701in}}%
\pgfpathlineto{\pgfqpoint{3.365622in}{2.201701in}}%
\pgfpathlineto{\pgfqpoint{3.365622in}{2.204650in}}%
\pgfpathlineto{\pgfqpoint{3.370163in}{2.204650in}}%
\pgfpathlineto{\pgfqpoint{3.370163in}{2.201701in}}%
\pgfpathmoveto{\pgfqpoint{3.370163in}{2.198752in}}%
\pgfpathlineto{\pgfqpoint{3.370163in}{2.198752in}}%
\pgfpathlineto{\pgfqpoint{3.370163in}{2.201701in}}%
\pgfpathlineto{\pgfqpoint{3.374704in}{2.201701in}}%
\pgfpathlineto{\pgfqpoint{3.374704in}{2.198752in}}%
\pgfpathmoveto{\pgfqpoint{3.370163in}{2.201701in}}%
\pgfpathlineto{\pgfqpoint{3.370163in}{2.201701in}}%
\pgfpathlineto{\pgfqpoint{3.370163in}{2.204650in}}%
\pgfpathlineto{\pgfqpoint{3.374704in}{2.204650in}}%
\pgfpathlineto{\pgfqpoint{3.374704in}{2.201701in}}%
\pgfpathmoveto{\pgfqpoint{3.374704in}{2.201701in}}%
\pgfpathlineto{\pgfqpoint{3.374704in}{2.201701in}}%
\pgfpathlineto{\pgfqpoint{3.374704in}{2.204650in}}%
\pgfpathlineto{\pgfqpoint{3.379246in}{2.204650in}}%
\pgfpathlineto{\pgfqpoint{3.379246in}{2.201701in}}%
\pgfpathmoveto{\pgfqpoint{3.374704in}{2.204650in}}%
\pgfpathlineto{\pgfqpoint{3.374704in}{2.204650in}}%
\pgfpathlineto{\pgfqpoint{3.374704in}{2.207599in}}%
\pgfpathlineto{\pgfqpoint{3.379246in}{2.207599in}}%
\pgfpathlineto{\pgfqpoint{3.379246in}{2.204650in}}%
\pgfpathmoveto{\pgfqpoint{3.374704in}{2.207599in}}%
\pgfpathlineto{\pgfqpoint{3.374704in}{2.207599in}}%
\pgfpathlineto{\pgfqpoint{3.374704in}{2.210548in}}%
\pgfpathlineto{\pgfqpoint{3.379246in}{2.210548in}}%
\pgfpathlineto{\pgfqpoint{3.379246in}{2.207599in}}%
\pgfpathmoveto{\pgfqpoint{3.379246in}{2.204650in}}%
\pgfpathlineto{\pgfqpoint{3.379246in}{2.204650in}}%
\pgfpathlineto{\pgfqpoint{3.379246in}{2.207599in}}%
\pgfpathlineto{\pgfqpoint{3.383787in}{2.207599in}}%
\pgfpathlineto{\pgfqpoint{3.383787in}{2.204650in}}%
\pgfpathmoveto{\pgfqpoint{3.379246in}{2.207599in}}%
\pgfpathlineto{\pgfqpoint{3.379246in}{2.207599in}}%
\pgfpathlineto{\pgfqpoint{3.379246in}{2.210548in}}%
\pgfpathlineto{\pgfqpoint{3.383787in}{2.210548in}}%
\pgfpathlineto{\pgfqpoint{3.383787in}{2.207599in}}%
\pgfpathmoveto{\pgfqpoint{3.383787in}{2.207599in}}%
\pgfpathlineto{\pgfqpoint{3.383787in}{2.207599in}}%
\pgfpathlineto{\pgfqpoint{3.383787in}{2.210548in}}%
\pgfpathlineto{\pgfqpoint{3.388328in}{2.210548in}}%
\pgfpathlineto{\pgfqpoint{3.388328in}{2.207599in}}%
\pgfpathmoveto{\pgfqpoint{3.383787in}{2.210548in}}%
\pgfpathlineto{\pgfqpoint{3.383787in}{2.210548in}}%
\pgfpathlineto{\pgfqpoint{3.383787in}{2.213497in}}%
\pgfpathlineto{\pgfqpoint{3.388328in}{2.213497in}}%
\pgfpathlineto{\pgfqpoint{3.388328in}{2.210548in}}%
\pgfpathmoveto{\pgfqpoint{3.383787in}{2.213497in}}%
\pgfpathlineto{\pgfqpoint{3.383787in}{2.213497in}}%
\pgfpathlineto{\pgfqpoint{3.383787in}{2.216446in}}%
\pgfpathlineto{\pgfqpoint{3.388328in}{2.216446in}}%
\pgfpathlineto{\pgfqpoint{3.388328in}{2.213497in}}%
\pgfpathmoveto{\pgfqpoint{3.388328in}{2.210548in}}%
\pgfpathlineto{\pgfqpoint{3.388328in}{2.210548in}}%
\pgfpathlineto{\pgfqpoint{3.388328in}{2.213497in}}%
\pgfpathlineto{\pgfqpoint{3.392869in}{2.213497in}}%
\pgfpathlineto{\pgfqpoint{3.392869in}{2.210548in}}%
\pgfpathmoveto{\pgfqpoint{3.388328in}{2.213497in}}%
\pgfpathlineto{\pgfqpoint{3.388328in}{2.213497in}}%
\pgfpathlineto{\pgfqpoint{3.388328in}{2.216446in}}%
\pgfpathlineto{\pgfqpoint{3.392869in}{2.216446in}}%
\pgfpathlineto{\pgfqpoint{3.392869in}{2.213497in}}%
\pgfpathmoveto{\pgfqpoint{3.392869in}{2.213497in}}%
\pgfpathlineto{\pgfqpoint{3.392869in}{2.213497in}}%
\pgfpathlineto{\pgfqpoint{3.392869in}{2.216446in}}%
\pgfpathlineto{\pgfqpoint{3.397410in}{2.216446in}}%
\pgfpathlineto{\pgfqpoint{3.397410in}{2.213497in}}%
\pgfpathmoveto{\pgfqpoint{3.392869in}{2.216446in}}%
\pgfpathlineto{\pgfqpoint{3.392869in}{2.216446in}}%
\pgfpathlineto{\pgfqpoint{3.392869in}{2.219396in}}%
\pgfpathlineto{\pgfqpoint{3.397410in}{2.219396in}}%
\pgfpathlineto{\pgfqpoint{3.397410in}{2.216446in}}%
\pgfpathmoveto{\pgfqpoint{3.392869in}{2.219396in}}%
\pgfpathlineto{\pgfqpoint{3.392869in}{2.219396in}}%
\pgfpathlineto{\pgfqpoint{3.392869in}{2.222345in}}%
\pgfpathlineto{\pgfqpoint{3.397410in}{2.222345in}}%
\pgfpathlineto{\pgfqpoint{3.397410in}{2.219396in}}%
\pgfpathmoveto{\pgfqpoint{3.397410in}{2.216446in}}%
\pgfpathlineto{\pgfqpoint{3.397410in}{2.216446in}}%
\pgfpathlineto{\pgfqpoint{3.397410in}{2.219396in}}%
\pgfpathlineto{\pgfqpoint{3.401952in}{2.219396in}}%
\pgfpathlineto{\pgfqpoint{3.401952in}{2.216446in}}%
\pgfpathmoveto{\pgfqpoint{3.397410in}{2.219396in}}%
\pgfpathlineto{\pgfqpoint{3.397410in}{2.219396in}}%
\pgfpathlineto{\pgfqpoint{3.397410in}{2.222345in}}%
\pgfpathlineto{\pgfqpoint{3.401952in}{2.222345in}}%
\pgfpathlineto{\pgfqpoint{3.401952in}{2.219396in}}%
\pgfpathmoveto{\pgfqpoint{3.401952in}{2.219396in}}%
\pgfpathlineto{\pgfqpoint{3.401952in}{2.219396in}}%
\pgfpathlineto{\pgfqpoint{3.401952in}{2.222345in}}%
\pgfpathlineto{\pgfqpoint{3.406493in}{2.222345in}}%
\pgfpathlineto{\pgfqpoint{3.406493in}{2.219396in}}%
\pgfpathmoveto{\pgfqpoint{3.401952in}{2.222345in}}%
\pgfpathlineto{\pgfqpoint{3.401952in}{2.222345in}}%
\pgfpathlineto{\pgfqpoint{3.401952in}{2.225294in}}%
\pgfpathlineto{\pgfqpoint{3.406493in}{2.225294in}}%
\pgfpathlineto{\pgfqpoint{3.406493in}{2.222345in}}%
\pgfpathmoveto{\pgfqpoint{3.401952in}{2.225294in}}%
\pgfpathlineto{\pgfqpoint{3.401952in}{2.225294in}}%
\pgfpathlineto{\pgfqpoint{3.401952in}{2.228243in}}%
\pgfpathlineto{\pgfqpoint{3.406493in}{2.228243in}}%
\pgfpathlineto{\pgfqpoint{3.406493in}{2.225294in}}%
\pgfpathmoveto{\pgfqpoint{3.406493in}{2.222345in}}%
\pgfpathlineto{\pgfqpoint{3.406493in}{2.222345in}}%
\pgfpathlineto{\pgfqpoint{3.406493in}{2.225294in}}%
\pgfpathlineto{\pgfqpoint{3.411034in}{2.225294in}}%
\pgfpathlineto{\pgfqpoint{3.411034in}{2.222345in}}%
\pgfpathmoveto{\pgfqpoint{3.406493in}{2.225294in}}%
\pgfpathlineto{\pgfqpoint{3.406493in}{2.225294in}}%
\pgfpathlineto{\pgfqpoint{3.406493in}{2.228243in}}%
\pgfpathlineto{\pgfqpoint{3.411034in}{2.228243in}}%
\pgfpathlineto{\pgfqpoint{3.411034in}{2.225294in}}%
\pgfpathmoveto{\pgfqpoint{3.411034in}{2.225294in}}%
\pgfpathlineto{\pgfqpoint{3.411034in}{2.225294in}}%
\pgfpathlineto{\pgfqpoint{3.411034in}{2.228243in}}%
\pgfpathlineto{\pgfqpoint{3.415575in}{2.228243in}}%
\pgfpathlineto{\pgfqpoint{3.415575in}{2.225294in}}%
\pgfpathmoveto{\pgfqpoint{3.411034in}{2.228243in}}%
\pgfpathlineto{\pgfqpoint{3.411034in}{2.228243in}}%
\pgfpathlineto{\pgfqpoint{3.411034in}{2.231192in}}%
\pgfpathlineto{\pgfqpoint{3.415575in}{2.231192in}}%
\pgfpathlineto{\pgfqpoint{3.415575in}{2.228243in}}%
\pgfpathmoveto{\pgfqpoint{3.411034in}{2.231192in}}%
\pgfpathlineto{\pgfqpoint{3.411034in}{2.231192in}}%
\pgfpathlineto{\pgfqpoint{3.411034in}{2.234141in}}%
\pgfpathlineto{\pgfqpoint{3.415575in}{2.234141in}}%
\pgfpathlineto{\pgfqpoint{3.415575in}{2.231192in}}%
\pgfpathmoveto{\pgfqpoint{3.415575in}{2.228243in}}%
\pgfpathlineto{\pgfqpoint{3.415575in}{2.228243in}}%
\pgfpathlineto{\pgfqpoint{3.415575in}{2.231192in}}%
\pgfpathlineto{\pgfqpoint{3.420116in}{2.231192in}}%
\pgfpathlineto{\pgfqpoint{3.420116in}{2.228243in}}%
\pgfpathmoveto{\pgfqpoint{3.415575in}{2.231192in}}%
\pgfpathlineto{\pgfqpoint{3.415575in}{2.231192in}}%
\pgfpathlineto{\pgfqpoint{3.415575in}{2.234141in}}%
\pgfpathlineto{\pgfqpoint{3.420116in}{2.234141in}}%
\pgfpathlineto{\pgfqpoint{3.420116in}{2.231192in}}%
\pgfpathmoveto{\pgfqpoint{3.420116in}{2.231192in}}%
\pgfpathlineto{\pgfqpoint{3.420116in}{2.231192in}}%
\pgfpathlineto{\pgfqpoint{3.420116in}{2.234141in}}%
\pgfpathlineto{\pgfqpoint{3.424658in}{2.234141in}}%
\pgfpathlineto{\pgfqpoint{3.424658in}{2.231192in}}%
\pgfpathmoveto{\pgfqpoint{3.420116in}{2.234141in}}%
\pgfpathlineto{\pgfqpoint{3.420116in}{2.234141in}}%
\pgfpathlineto{\pgfqpoint{3.420116in}{2.237090in}}%
\pgfpathlineto{\pgfqpoint{3.424658in}{2.237090in}}%
\pgfpathlineto{\pgfqpoint{3.424658in}{2.234141in}}%
\pgfpathmoveto{\pgfqpoint{3.420116in}{2.237090in}}%
\pgfpathlineto{\pgfqpoint{3.420116in}{2.237090in}}%
\pgfpathlineto{\pgfqpoint{3.420116in}{2.240039in}}%
\pgfpathlineto{\pgfqpoint{3.424658in}{2.240039in}}%
\pgfpathlineto{\pgfqpoint{3.424658in}{2.237090in}}%
\pgfpathmoveto{\pgfqpoint{3.424658in}{2.234141in}}%
\pgfpathlineto{\pgfqpoint{3.424658in}{2.234141in}}%
\pgfpathlineto{\pgfqpoint{3.424658in}{2.237090in}}%
\pgfpathlineto{\pgfqpoint{3.429199in}{2.237090in}}%
\pgfpathlineto{\pgfqpoint{3.429199in}{2.234141in}}%
\pgfpathmoveto{\pgfqpoint{3.424658in}{2.237090in}}%
\pgfpathlineto{\pgfqpoint{3.424658in}{2.237090in}}%
\pgfpathlineto{\pgfqpoint{3.424658in}{2.240039in}}%
\pgfpathlineto{\pgfqpoint{3.429199in}{2.240039in}}%
\pgfpathlineto{\pgfqpoint{3.429199in}{2.237090in}}%
\pgfpathmoveto{\pgfqpoint{3.429199in}{2.237090in}}%
\pgfpathlineto{\pgfqpoint{3.429199in}{2.237090in}}%
\pgfpathlineto{\pgfqpoint{3.429199in}{2.240039in}}%
\pgfpathlineto{\pgfqpoint{3.433740in}{2.240039in}}%
\pgfpathlineto{\pgfqpoint{3.433740in}{2.237090in}}%
\pgfpathmoveto{\pgfqpoint{3.429199in}{2.240039in}}%
\pgfpathlineto{\pgfqpoint{3.429199in}{2.240039in}}%
\pgfpathlineto{\pgfqpoint{3.429199in}{2.242988in}}%
\pgfpathlineto{\pgfqpoint{3.433740in}{2.242988in}}%
\pgfpathlineto{\pgfqpoint{3.433740in}{2.240039in}}%
\pgfpathmoveto{\pgfqpoint{3.429199in}{2.242988in}}%
\pgfpathlineto{\pgfqpoint{3.429199in}{2.242988in}}%
\pgfpathlineto{\pgfqpoint{3.429199in}{2.245937in}}%
\pgfpathlineto{\pgfqpoint{3.433740in}{2.245937in}}%
\pgfpathlineto{\pgfqpoint{3.433740in}{2.242988in}}%
\pgfpathmoveto{\pgfqpoint{3.433740in}{2.240039in}}%
\pgfpathlineto{\pgfqpoint{3.433740in}{2.240039in}}%
\pgfpathlineto{\pgfqpoint{3.433740in}{2.242988in}}%
\pgfpathlineto{\pgfqpoint{3.438281in}{2.242988in}}%
\pgfpathlineto{\pgfqpoint{3.438281in}{2.240039in}}%
\pgfpathmoveto{\pgfqpoint{3.433740in}{2.242988in}}%
\pgfpathlineto{\pgfqpoint{3.433740in}{2.242988in}}%
\pgfpathlineto{\pgfqpoint{3.433740in}{2.245937in}}%
\pgfpathlineto{\pgfqpoint{3.438281in}{2.245937in}}%
\pgfpathlineto{\pgfqpoint{3.438281in}{2.242988in}}%
\pgfpathmoveto{\pgfqpoint{3.429199in}{2.245937in}}%
\pgfpathlineto{\pgfqpoint{3.429199in}{2.245937in}}%
\pgfpathlineto{\pgfqpoint{3.429199in}{2.248886in}}%
\pgfpathlineto{\pgfqpoint{3.433740in}{2.248886in}}%
\pgfpathlineto{\pgfqpoint{3.433740in}{2.245937in}}%
\pgfpathmoveto{\pgfqpoint{3.429199in}{2.248886in}}%
\pgfpathlineto{\pgfqpoint{3.429199in}{2.248886in}}%
\pgfpathlineto{\pgfqpoint{3.429199in}{2.251835in}}%
\pgfpathlineto{\pgfqpoint{3.433740in}{2.251835in}}%
\pgfpathlineto{\pgfqpoint{3.433740in}{2.248886in}}%
\pgfpathmoveto{\pgfqpoint{3.433740in}{2.245937in}}%
\pgfpathlineto{\pgfqpoint{3.433740in}{2.245937in}}%
\pgfpathlineto{\pgfqpoint{3.433740in}{2.248886in}}%
\pgfpathlineto{\pgfqpoint{3.438281in}{2.248886in}}%
\pgfpathlineto{\pgfqpoint{3.438281in}{2.245937in}}%
\pgfpathmoveto{\pgfqpoint{3.433740in}{2.248886in}}%
\pgfpathlineto{\pgfqpoint{3.433740in}{2.248886in}}%
\pgfpathlineto{\pgfqpoint{3.433740in}{2.251835in}}%
\pgfpathlineto{\pgfqpoint{3.438281in}{2.251835in}}%
\pgfpathlineto{\pgfqpoint{3.438281in}{2.248886in}}%
\pgfpathmoveto{\pgfqpoint{3.438281in}{2.245937in}}%
\pgfpathlineto{\pgfqpoint{3.438281in}{2.245937in}}%
\pgfpathlineto{\pgfqpoint{3.438281in}{2.248886in}}%
\pgfpathlineto{\pgfqpoint{3.442822in}{2.248886in}}%
\pgfpathlineto{\pgfqpoint{3.442822in}{2.245937in}}%
\pgfpathmoveto{\pgfqpoint{3.438281in}{2.248886in}}%
\pgfpathlineto{\pgfqpoint{3.438281in}{2.248886in}}%
\pgfpathlineto{\pgfqpoint{3.438281in}{2.251835in}}%
\pgfpathlineto{\pgfqpoint{3.442822in}{2.251835in}}%
\pgfpathlineto{\pgfqpoint{3.442822in}{2.248886in}}%
\pgfpathmoveto{\pgfqpoint{3.442822in}{2.248886in}}%
\pgfpathlineto{\pgfqpoint{3.442822in}{2.248886in}}%
\pgfpathlineto{\pgfqpoint{3.442822in}{2.251835in}}%
\pgfpathlineto{\pgfqpoint{3.447364in}{2.251835in}}%
\pgfpathlineto{\pgfqpoint{3.447364in}{2.248886in}}%
\pgfpathmoveto{\pgfqpoint{3.438281in}{2.251835in}}%
\pgfpathlineto{\pgfqpoint{3.438281in}{2.251835in}}%
\pgfpathlineto{\pgfqpoint{3.438281in}{2.254784in}}%
\pgfpathlineto{\pgfqpoint{3.442822in}{2.254784in}}%
\pgfpathlineto{\pgfqpoint{3.442822in}{2.251835in}}%
\pgfpathmoveto{\pgfqpoint{3.438281in}{2.254784in}}%
\pgfpathlineto{\pgfqpoint{3.438281in}{2.254784in}}%
\pgfpathlineto{\pgfqpoint{3.438281in}{2.257734in}}%
\pgfpathlineto{\pgfqpoint{3.442822in}{2.257734in}}%
\pgfpathlineto{\pgfqpoint{3.442822in}{2.254784in}}%
\pgfpathmoveto{\pgfqpoint{3.442822in}{2.251835in}}%
\pgfpathlineto{\pgfqpoint{3.442822in}{2.251835in}}%
\pgfpathlineto{\pgfqpoint{3.442822in}{2.254784in}}%
\pgfpathlineto{\pgfqpoint{3.447364in}{2.254784in}}%
\pgfpathlineto{\pgfqpoint{3.447364in}{2.251835in}}%
\pgfpathmoveto{\pgfqpoint{3.442822in}{2.254784in}}%
\pgfpathlineto{\pgfqpoint{3.442822in}{2.254784in}}%
\pgfpathlineto{\pgfqpoint{3.442822in}{2.257734in}}%
\pgfpathlineto{\pgfqpoint{3.447364in}{2.257734in}}%
\pgfpathlineto{\pgfqpoint{3.447364in}{2.254784in}}%
\pgfpathmoveto{\pgfqpoint{3.447364in}{2.251835in}}%
\pgfpathlineto{\pgfqpoint{3.447364in}{2.251835in}}%
\pgfpathlineto{\pgfqpoint{3.447364in}{2.254784in}}%
\pgfpathlineto{\pgfqpoint{3.451905in}{2.254784in}}%
\pgfpathlineto{\pgfqpoint{3.451905in}{2.251835in}}%
\pgfpathmoveto{\pgfqpoint{3.447364in}{2.254784in}}%
\pgfpathlineto{\pgfqpoint{3.447364in}{2.254784in}}%
\pgfpathlineto{\pgfqpoint{3.447364in}{2.257734in}}%
\pgfpathlineto{\pgfqpoint{3.451905in}{2.257734in}}%
\pgfpathlineto{\pgfqpoint{3.451905in}{2.254784in}}%
\pgfpathmoveto{\pgfqpoint{3.451905in}{2.254784in}}%
\pgfpathlineto{\pgfqpoint{3.451905in}{2.254784in}}%
\pgfpathlineto{\pgfqpoint{3.451905in}{2.257734in}}%
\pgfpathlineto{\pgfqpoint{3.456446in}{2.257734in}}%
\pgfpathlineto{\pgfqpoint{3.456446in}{2.254784in}}%
\pgfpathmoveto{\pgfqpoint{3.447364in}{2.257734in}}%
\pgfpathlineto{\pgfqpoint{3.447364in}{2.257734in}}%
\pgfpathlineto{\pgfqpoint{3.447364in}{2.260683in}}%
\pgfpathlineto{\pgfqpoint{3.451905in}{2.260683in}}%
\pgfpathlineto{\pgfqpoint{3.451905in}{2.257734in}}%
\pgfpathmoveto{\pgfqpoint{3.447364in}{2.260683in}}%
\pgfpathlineto{\pgfqpoint{3.447364in}{2.260683in}}%
\pgfpathlineto{\pgfqpoint{3.447364in}{2.263632in}}%
\pgfpathlineto{\pgfqpoint{3.451905in}{2.263632in}}%
\pgfpathlineto{\pgfqpoint{3.451905in}{2.260683in}}%
\pgfpathmoveto{\pgfqpoint{3.451905in}{2.257734in}}%
\pgfpathlineto{\pgfqpoint{3.451905in}{2.257734in}}%
\pgfpathlineto{\pgfqpoint{3.451905in}{2.260683in}}%
\pgfpathlineto{\pgfqpoint{3.456446in}{2.260683in}}%
\pgfpathlineto{\pgfqpoint{3.456446in}{2.257734in}}%
\pgfpathmoveto{\pgfqpoint{3.451905in}{2.260683in}}%
\pgfpathlineto{\pgfqpoint{3.451905in}{2.260683in}}%
\pgfpathlineto{\pgfqpoint{3.451905in}{2.263632in}}%
\pgfpathlineto{\pgfqpoint{3.456446in}{2.263632in}}%
\pgfpathlineto{\pgfqpoint{3.456446in}{2.260683in}}%
\pgfpathmoveto{\pgfqpoint{3.456446in}{2.257734in}}%
\pgfpathlineto{\pgfqpoint{3.456446in}{2.257734in}}%
\pgfpathlineto{\pgfqpoint{3.456446in}{2.260683in}}%
\pgfpathlineto{\pgfqpoint{3.460987in}{2.260683in}}%
\pgfpathlineto{\pgfqpoint{3.460987in}{2.257734in}}%
\pgfpathmoveto{\pgfqpoint{3.456446in}{2.260683in}}%
\pgfpathlineto{\pgfqpoint{3.456446in}{2.260683in}}%
\pgfpathlineto{\pgfqpoint{3.456446in}{2.263632in}}%
\pgfpathlineto{\pgfqpoint{3.460987in}{2.263632in}}%
\pgfpathlineto{\pgfqpoint{3.460987in}{2.260683in}}%
\pgfpathmoveto{\pgfqpoint{3.460987in}{2.260683in}}%
\pgfpathlineto{\pgfqpoint{3.460987in}{2.260683in}}%
\pgfpathlineto{\pgfqpoint{3.460987in}{2.263632in}}%
\pgfpathlineto{\pgfqpoint{3.465529in}{2.263632in}}%
\pgfpathlineto{\pgfqpoint{3.465529in}{2.260683in}}%
\pgfpathmoveto{\pgfqpoint{3.456446in}{2.263632in}}%
\pgfpathlineto{\pgfqpoint{3.456446in}{2.263632in}}%
\pgfpathlineto{\pgfqpoint{3.456446in}{2.266581in}}%
\pgfpathlineto{\pgfqpoint{3.460987in}{2.266581in}}%
\pgfpathlineto{\pgfqpoint{3.460987in}{2.263632in}}%
\pgfpathmoveto{\pgfqpoint{3.456446in}{2.266581in}}%
\pgfpathlineto{\pgfqpoint{3.456446in}{2.266581in}}%
\pgfpathlineto{\pgfqpoint{3.456446in}{2.269530in}}%
\pgfpathlineto{\pgfqpoint{3.460987in}{2.269530in}}%
\pgfpathlineto{\pgfqpoint{3.460987in}{2.266581in}}%
\pgfpathmoveto{\pgfqpoint{3.460987in}{2.263632in}}%
\pgfpathlineto{\pgfqpoint{3.460987in}{2.263632in}}%
\pgfpathlineto{\pgfqpoint{3.460987in}{2.266581in}}%
\pgfpathlineto{\pgfqpoint{3.465529in}{2.266581in}}%
\pgfpathlineto{\pgfqpoint{3.465529in}{2.263632in}}%
\pgfpathmoveto{\pgfqpoint{3.460987in}{2.266581in}}%
\pgfpathlineto{\pgfqpoint{3.460987in}{2.266581in}}%
\pgfpathlineto{\pgfqpoint{3.460987in}{2.269530in}}%
\pgfpathlineto{\pgfqpoint{3.465529in}{2.269530in}}%
\pgfpathlineto{\pgfqpoint{3.465529in}{2.266581in}}%
\pgfpathmoveto{\pgfqpoint{3.465529in}{2.263632in}}%
\pgfpathlineto{\pgfqpoint{3.465529in}{2.263632in}}%
\pgfpathlineto{\pgfqpoint{3.465529in}{2.266581in}}%
\pgfpathlineto{\pgfqpoint{3.470070in}{2.266581in}}%
\pgfpathlineto{\pgfqpoint{3.470070in}{2.263632in}}%
\pgfpathmoveto{\pgfqpoint{3.465529in}{2.266581in}}%
\pgfpathlineto{\pgfqpoint{3.465529in}{2.266581in}}%
\pgfpathlineto{\pgfqpoint{3.465529in}{2.269530in}}%
\pgfpathlineto{\pgfqpoint{3.470070in}{2.269530in}}%
\pgfpathlineto{\pgfqpoint{3.470070in}{2.266581in}}%
\pgfpathmoveto{\pgfqpoint{3.470070in}{2.266581in}}%
\pgfpathlineto{\pgfqpoint{3.470070in}{2.266581in}}%
\pgfpathlineto{\pgfqpoint{3.470070in}{2.269530in}}%
\pgfpathlineto{\pgfqpoint{3.474611in}{2.269530in}}%
\pgfpathlineto{\pgfqpoint{3.474611in}{2.266581in}}%
\pgfpathmoveto{\pgfqpoint{3.465529in}{2.269530in}}%
\pgfpathlineto{\pgfqpoint{3.465529in}{2.269530in}}%
\pgfpathlineto{\pgfqpoint{3.465529in}{2.272479in}}%
\pgfpathlineto{\pgfqpoint{3.470070in}{2.272479in}}%
\pgfpathlineto{\pgfqpoint{3.470070in}{2.269530in}}%
\pgfpathmoveto{\pgfqpoint{3.465529in}{2.272479in}}%
\pgfpathlineto{\pgfqpoint{3.465529in}{2.272479in}}%
\pgfpathlineto{\pgfqpoint{3.465529in}{2.275428in}}%
\pgfpathlineto{\pgfqpoint{3.470070in}{2.275428in}}%
\pgfpathlineto{\pgfqpoint{3.470070in}{2.272479in}}%
\pgfpathmoveto{\pgfqpoint{3.470070in}{2.269530in}}%
\pgfpathlineto{\pgfqpoint{3.470070in}{2.269530in}}%
\pgfpathlineto{\pgfqpoint{3.470070in}{2.272479in}}%
\pgfpathlineto{\pgfqpoint{3.474611in}{2.272479in}}%
\pgfpathlineto{\pgfqpoint{3.474611in}{2.269530in}}%
\pgfpathmoveto{\pgfqpoint{3.470070in}{2.272479in}}%
\pgfpathlineto{\pgfqpoint{3.470070in}{2.272479in}}%
\pgfpathlineto{\pgfqpoint{3.470070in}{2.275428in}}%
\pgfpathlineto{\pgfqpoint{3.474611in}{2.275428in}}%
\pgfpathlineto{\pgfqpoint{3.474611in}{2.272479in}}%
\pgfpathmoveto{\pgfqpoint{3.474611in}{2.269530in}}%
\pgfpathlineto{\pgfqpoint{3.474611in}{2.269530in}}%
\pgfpathlineto{\pgfqpoint{3.474611in}{2.272479in}}%
\pgfpathlineto{\pgfqpoint{3.479152in}{2.272479in}}%
\pgfpathlineto{\pgfqpoint{3.479152in}{2.269530in}}%
\pgfpathmoveto{\pgfqpoint{3.474611in}{2.272479in}}%
\pgfpathlineto{\pgfqpoint{3.474611in}{2.272479in}}%
\pgfpathlineto{\pgfqpoint{3.474611in}{2.275428in}}%
\pgfpathlineto{\pgfqpoint{3.479152in}{2.275428in}}%
\pgfpathlineto{\pgfqpoint{3.479152in}{2.272479in}}%
\pgfpathmoveto{\pgfqpoint{3.479152in}{2.272479in}}%
\pgfpathlineto{\pgfqpoint{3.479152in}{2.272479in}}%
\pgfpathlineto{\pgfqpoint{3.479152in}{2.275428in}}%
\pgfpathlineto{\pgfqpoint{3.483693in}{2.275428in}}%
\pgfpathlineto{\pgfqpoint{3.483693in}{2.272479in}}%
\pgfpathmoveto{\pgfqpoint{3.474611in}{2.275428in}}%
\pgfpathlineto{\pgfqpoint{3.474611in}{2.275428in}}%
\pgfpathlineto{\pgfqpoint{3.474611in}{2.278377in}}%
\pgfpathlineto{\pgfqpoint{3.479152in}{2.278377in}}%
\pgfpathlineto{\pgfqpoint{3.479152in}{2.275428in}}%
\pgfpathmoveto{\pgfqpoint{3.474611in}{2.278377in}}%
\pgfpathlineto{\pgfqpoint{3.474611in}{2.278377in}}%
\pgfpathlineto{\pgfqpoint{3.474611in}{2.281326in}}%
\pgfpathlineto{\pgfqpoint{3.479152in}{2.281326in}}%
\pgfpathlineto{\pgfqpoint{3.479152in}{2.278377in}}%
\pgfpathmoveto{\pgfqpoint{3.479152in}{2.275428in}}%
\pgfpathlineto{\pgfqpoint{3.479152in}{2.275428in}}%
\pgfpathlineto{\pgfqpoint{3.479152in}{2.278377in}}%
\pgfpathlineto{\pgfqpoint{3.483693in}{2.278377in}}%
\pgfpathlineto{\pgfqpoint{3.483693in}{2.275428in}}%
\pgfpathmoveto{\pgfqpoint{3.479152in}{2.278377in}}%
\pgfpathlineto{\pgfqpoint{3.479152in}{2.278377in}}%
\pgfpathlineto{\pgfqpoint{3.479152in}{2.281326in}}%
\pgfpathlineto{\pgfqpoint{3.483693in}{2.281326in}}%
\pgfpathlineto{\pgfqpoint{3.483693in}{2.278377in}}%
\pgfpathmoveto{\pgfqpoint{3.483693in}{2.275428in}}%
\pgfpathlineto{\pgfqpoint{3.483693in}{2.275428in}}%
\pgfpathlineto{\pgfqpoint{3.483693in}{2.278377in}}%
\pgfpathlineto{\pgfqpoint{3.488235in}{2.278377in}}%
\pgfpathlineto{\pgfqpoint{3.488235in}{2.275428in}}%
\pgfpathmoveto{\pgfqpoint{3.483693in}{2.278377in}}%
\pgfpathlineto{\pgfqpoint{3.483693in}{2.278377in}}%
\pgfpathlineto{\pgfqpoint{3.483693in}{2.281326in}}%
\pgfpathlineto{\pgfqpoint{3.488235in}{2.281326in}}%
\pgfpathlineto{\pgfqpoint{3.488235in}{2.278377in}}%
\pgfpathmoveto{\pgfqpoint{3.488235in}{2.278377in}}%
\pgfpathlineto{\pgfqpoint{3.488235in}{2.278377in}}%
\pgfpathlineto{\pgfqpoint{3.488235in}{2.281326in}}%
\pgfpathlineto{\pgfqpoint{3.492776in}{2.281326in}}%
\pgfpathlineto{\pgfqpoint{3.492776in}{2.278377in}}%
\pgfpathmoveto{\pgfqpoint{3.483693in}{2.281326in}}%
\pgfpathlineto{\pgfqpoint{3.483693in}{2.281326in}}%
\pgfpathlineto{\pgfqpoint{3.483693in}{2.284275in}}%
\pgfpathlineto{\pgfqpoint{3.488235in}{2.284275in}}%
\pgfpathlineto{\pgfqpoint{3.488235in}{2.281326in}}%
\pgfpathmoveto{\pgfqpoint{3.483693in}{2.284275in}}%
\pgfpathlineto{\pgfqpoint{3.483693in}{2.284275in}}%
\pgfpathlineto{\pgfqpoint{3.483693in}{2.287224in}}%
\pgfpathlineto{\pgfqpoint{3.488235in}{2.287224in}}%
\pgfpathlineto{\pgfqpoint{3.488235in}{2.284275in}}%
\pgfpathmoveto{\pgfqpoint{3.488235in}{2.281326in}}%
\pgfpathlineto{\pgfqpoint{3.488235in}{2.281326in}}%
\pgfpathlineto{\pgfqpoint{3.488235in}{2.284275in}}%
\pgfpathlineto{\pgfqpoint{3.492776in}{2.284275in}}%
\pgfpathlineto{\pgfqpoint{3.492776in}{2.281326in}}%
\pgfpathmoveto{\pgfqpoint{3.488235in}{2.284275in}}%
\pgfpathlineto{\pgfqpoint{3.488235in}{2.284275in}}%
\pgfpathlineto{\pgfqpoint{3.488235in}{2.287224in}}%
\pgfpathlineto{\pgfqpoint{3.492776in}{2.287224in}}%
\pgfpathlineto{\pgfqpoint{3.492776in}{2.284275in}}%
\pgfpathmoveto{\pgfqpoint{3.492776in}{2.281326in}}%
\pgfpathlineto{\pgfqpoint{3.492776in}{2.281326in}}%
\pgfpathlineto{\pgfqpoint{3.492776in}{2.284275in}}%
\pgfpathlineto{\pgfqpoint{3.497317in}{2.284275in}}%
\pgfpathlineto{\pgfqpoint{3.497317in}{2.281326in}}%
\pgfpathmoveto{\pgfqpoint{3.492776in}{2.284275in}}%
\pgfpathlineto{\pgfqpoint{3.492776in}{2.284275in}}%
\pgfpathlineto{\pgfqpoint{3.492776in}{2.287224in}}%
\pgfpathlineto{\pgfqpoint{3.497317in}{2.287224in}}%
\pgfpathlineto{\pgfqpoint{3.497317in}{2.284275in}}%
\pgfpathmoveto{\pgfqpoint{3.497317in}{2.284275in}}%
\pgfpathlineto{\pgfqpoint{3.497317in}{2.284275in}}%
\pgfpathlineto{\pgfqpoint{3.497317in}{2.287224in}}%
\pgfpathlineto{\pgfqpoint{3.501858in}{2.287224in}}%
\pgfpathlineto{\pgfqpoint{3.501858in}{2.284275in}}%
\pgfpathmoveto{\pgfqpoint{3.492776in}{2.287224in}}%
\pgfpathlineto{\pgfqpoint{3.492776in}{2.287224in}}%
\pgfpathlineto{\pgfqpoint{3.492776in}{2.290173in}}%
\pgfpathlineto{\pgfqpoint{3.497317in}{2.290173in}}%
\pgfpathlineto{\pgfqpoint{3.497317in}{2.287224in}}%
\pgfpathmoveto{\pgfqpoint{3.492776in}{2.290173in}}%
\pgfpathlineto{\pgfqpoint{3.492776in}{2.290173in}}%
\pgfpathlineto{\pgfqpoint{3.492776in}{2.293123in}}%
\pgfpathlineto{\pgfqpoint{3.497317in}{2.293123in}}%
\pgfpathlineto{\pgfqpoint{3.497317in}{2.290173in}}%
\pgfpathmoveto{\pgfqpoint{3.497317in}{2.287224in}}%
\pgfpathlineto{\pgfqpoint{3.497317in}{2.287224in}}%
\pgfpathlineto{\pgfqpoint{3.497317in}{2.290173in}}%
\pgfpathlineto{\pgfqpoint{3.501858in}{2.290173in}}%
\pgfpathlineto{\pgfqpoint{3.501858in}{2.287224in}}%
\pgfpathmoveto{\pgfqpoint{3.497317in}{2.290173in}}%
\pgfpathlineto{\pgfqpoint{3.497317in}{2.290173in}}%
\pgfpathlineto{\pgfqpoint{3.497317in}{2.293123in}}%
\pgfpathlineto{\pgfqpoint{3.501858in}{2.293123in}}%
\pgfpathlineto{\pgfqpoint{3.501858in}{2.290173in}}%
\pgfpathmoveto{\pgfqpoint{3.501858in}{2.287224in}}%
\pgfpathlineto{\pgfqpoint{3.501858in}{2.287224in}}%
\pgfpathlineto{\pgfqpoint{3.501858in}{2.290173in}}%
\pgfpathlineto{\pgfqpoint{3.506399in}{2.290173in}}%
\pgfpathlineto{\pgfqpoint{3.506399in}{2.287224in}}%
\pgfpathmoveto{\pgfqpoint{3.501858in}{2.290173in}}%
\pgfpathlineto{\pgfqpoint{3.501858in}{2.290173in}}%
\pgfpathlineto{\pgfqpoint{3.501858in}{2.293123in}}%
\pgfpathlineto{\pgfqpoint{3.506399in}{2.293123in}}%
\pgfpathlineto{\pgfqpoint{3.506399in}{2.290173in}}%
\pgfpathmoveto{\pgfqpoint{3.506399in}{2.290173in}}%
\pgfpathlineto{\pgfqpoint{3.506399in}{2.290173in}}%
\pgfpathlineto{\pgfqpoint{3.506399in}{2.293123in}}%
\pgfpathlineto{\pgfqpoint{3.510941in}{2.293123in}}%
\pgfpathlineto{\pgfqpoint{3.510941in}{2.290173in}}%
\pgfpathmoveto{\pgfqpoint{3.501858in}{2.293123in}}%
\pgfpathlineto{\pgfqpoint{3.501858in}{2.293123in}}%
\pgfpathlineto{\pgfqpoint{3.501858in}{2.296072in}}%
\pgfpathlineto{\pgfqpoint{3.506399in}{2.296072in}}%
\pgfpathlineto{\pgfqpoint{3.506399in}{2.293123in}}%
\pgfpathmoveto{\pgfqpoint{3.501858in}{2.296072in}}%
\pgfpathlineto{\pgfqpoint{3.501858in}{2.296072in}}%
\pgfpathlineto{\pgfqpoint{3.501858in}{2.299021in}}%
\pgfpathlineto{\pgfqpoint{3.506399in}{2.299021in}}%
\pgfpathlineto{\pgfqpoint{3.506399in}{2.296072in}}%
\pgfpathmoveto{\pgfqpoint{3.506399in}{2.293123in}}%
\pgfpathlineto{\pgfqpoint{3.506399in}{2.293123in}}%
\pgfpathlineto{\pgfqpoint{3.506399in}{2.296072in}}%
\pgfpathlineto{\pgfqpoint{3.510941in}{2.296072in}}%
\pgfpathlineto{\pgfqpoint{3.510941in}{2.293123in}}%
\pgfpathmoveto{\pgfqpoint{3.506399in}{2.296072in}}%
\pgfpathlineto{\pgfqpoint{3.506399in}{2.296072in}}%
\pgfpathlineto{\pgfqpoint{3.506399in}{2.299021in}}%
\pgfpathlineto{\pgfqpoint{3.510941in}{2.299021in}}%
\pgfpathlineto{\pgfqpoint{3.510941in}{2.296072in}}%
\pgfpathmoveto{\pgfqpoint{3.501858in}{2.647031in}}%
\pgfpathlineto{\pgfqpoint{3.501858in}{2.647031in}}%
\pgfpathlineto{\pgfqpoint{3.501858in}{2.649980in}}%
\pgfpathlineto{\pgfqpoint{3.506399in}{2.649980in}}%
\pgfpathlineto{\pgfqpoint{3.506399in}{2.647031in}}%
\pgfpathmoveto{\pgfqpoint{3.501858in}{2.649980in}}%
\pgfpathlineto{\pgfqpoint{3.501858in}{2.649980in}}%
\pgfpathlineto{\pgfqpoint{3.501858in}{2.652930in}}%
\pgfpathlineto{\pgfqpoint{3.506399in}{2.652930in}}%
\pgfpathlineto{\pgfqpoint{3.506399in}{2.649980in}}%
\pgfpathmoveto{\pgfqpoint{3.506399in}{2.647031in}}%
\pgfpathlineto{\pgfqpoint{3.506399in}{2.647031in}}%
\pgfpathlineto{\pgfqpoint{3.506399in}{2.649980in}}%
\pgfpathlineto{\pgfqpoint{3.510941in}{2.649980in}}%
\pgfpathlineto{\pgfqpoint{3.510941in}{2.647031in}}%
\pgfpathmoveto{\pgfqpoint{3.506399in}{2.649980in}}%
\pgfpathlineto{\pgfqpoint{3.506399in}{2.649980in}}%
\pgfpathlineto{\pgfqpoint{3.506399in}{2.652930in}}%
\pgfpathlineto{\pgfqpoint{3.510941in}{2.652930in}}%
\pgfpathlineto{\pgfqpoint{3.510941in}{2.649980in}}%
\pgfpathmoveto{\pgfqpoint{3.501858in}{2.652930in}}%
\pgfpathlineto{\pgfqpoint{3.501858in}{2.652930in}}%
\pgfpathlineto{\pgfqpoint{3.501858in}{2.655879in}}%
\pgfpathlineto{\pgfqpoint{3.506399in}{2.655879in}}%
\pgfpathlineto{\pgfqpoint{3.506399in}{2.652930in}}%
\pgfpathmoveto{\pgfqpoint{3.501858in}{2.655879in}}%
\pgfpathlineto{\pgfqpoint{3.501858in}{2.655879in}}%
\pgfpathlineto{\pgfqpoint{3.501858in}{2.658828in}}%
\pgfpathlineto{\pgfqpoint{3.506399in}{2.658828in}}%
\pgfpathlineto{\pgfqpoint{3.506399in}{2.655879in}}%
\pgfpathmoveto{\pgfqpoint{3.506399in}{2.652930in}}%
\pgfpathlineto{\pgfqpoint{3.506399in}{2.652930in}}%
\pgfpathlineto{\pgfqpoint{3.506399in}{2.655879in}}%
\pgfpathlineto{\pgfqpoint{3.510941in}{2.655879in}}%
\pgfpathlineto{\pgfqpoint{3.510941in}{2.652930in}}%
\pgfpathmoveto{\pgfqpoint{3.506399in}{2.655879in}}%
\pgfpathlineto{\pgfqpoint{3.506399in}{2.655879in}}%
\pgfpathlineto{\pgfqpoint{3.506399in}{2.658828in}}%
\pgfpathlineto{\pgfqpoint{3.510941in}{2.658828in}}%
\pgfpathlineto{\pgfqpoint{3.510941in}{2.655879in}}%
\pgfpathmoveto{\pgfqpoint{3.492776in}{2.658828in}}%
\pgfpathlineto{\pgfqpoint{3.492776in}{2.658828in}}%
\pgfpathlineto{\pgfqpoint{3.492776in}{2.661777in}}%
\pgfpathlineto{\pgfqpoint{3.497317in}{2.661777in}}%
\pgfpathlineto{\pgfqpoint{3.497317in}{2.658828in}}%
\pgfpathmoveto{\pgfqpoint{3.492776in}{2.661777in}}%
\pgfpathlineto{\pgfqpoint{3.492776in}{2.661777in}}%
\pgfpathlineto{\pgfqpoint{3.492776in}{2.664726in}}%
\pgfpathlineto{\pgfqpoint{3.497317in}{2.664726in}}%
\pgfpathlineto{\pgfqpoint{3.497317in}{2.661777in}}%
\pgfpathmoveto{\pgfqpoint{3.497317in}{2.658828in}}%
\pgfpathlineto{\pgfqpoint{3.497317in}{2.658828in}}%
\pgfpathlineto{\pgfqpoint{3.497317in}{2.661777in}}%
\pgfpathlineto{\pgfqpoint{3.501858in}{2.661777in}}%
\pgfpathlineto{\pgfqpoint{3.501858in}{2.658828in}}%
\pgfpathmoveto{\pgfqpoint{3.497317in}{2.661777in}}%
\pgfpathlineto{\pgfqpoint{3.497317in}{2.661777in}}%
\pgfpathlineto{\pgfqpoint{3.497317in}{2.664726in}}%
\pgfpathlineto{\pgfqpoint{3.501858in}{2.664726in}}%
\pgfpathlineto{\pgfqpoint{3.501858in}{2.661777in}}%
\pgfpathmoveto{\pgfqpoint{3.492776in}{2.664726in}}%
\pgfpathlineto{\pgfqpoint{3.492776in}{2.664726in}}%
\pgfpathlineto{\pgfqpoint{3.492776in}{2.667675in}}%
\pgfpathlineto{\pgfqpoint{3.497317in}{2.667675in}}%
\pgfpathlineto{\pgfqpoint{3.497317in}{2.664726in}}%
\pgfpathmoveto{\pgfqpoint{3.492776in}{2.667675in}}%
\pgfpathlineto{\pgfqpoint{3.492776in}{2.667675in}}%
\pgfpathlineto{\pgfqpoint{3.492776in}{2.670624in}}%
\pgfpathlineto{\pgfqpoint{3.497317in}{2.670624in}}%
\pgfpathlineto{\pgfqpoint{3.497317in}{2.667675in}}%
\pgfpathmoveto{\pgfqpoint{3.497317in}{2.664726in}}%
\pgfpathlineto{\pgfqpoint{3.497317in}{2.664726in}}%
\pgfpathlineto{\pgfqpoint{3.497317in}{2.667675in}}%
\pgfpathlineto{\pgfqpoint{3.501858in}{2.667675in}}%
\pgfpathlineto{\pgfqpoint{3.501858in}{2.664726in}}%
\pgfpathmoveto{\pgfqpoint{3.497317in}{2.667675in}}%
\pgfpathlineto{\pgfqpoint{3.497317in}{2.667675in}}%
\pgfpathlineto{\pgfqpoint{3.497317in}{2.670624in}}%
\pgfpathlineto{\pgfqpoint{3.501858in}{2.670624in}}%
\pgfpathlineto{\pgfqpoint{3.501858in}{2.667675in}}%
\pgfpathmoveto{\pgfqpoint{3.501858in}{2.658828in}}%
\pgfpathlineto{\pgfqpoint{3.501858in}{2.658828in}}%
\pgfpathlineto{\pgfqpoint{3.501858in}{2.661777in}}%
\pgfpathlineto{\pgfqpoint{3.506399in}{2.661777in}}%
\pgfpathlineto{\pgfqpoint{3.506399in}{2.658828in}}%
\pgfpathmoveto{\pgfqpoint{3.501858in}{2.661777in}}%
\pgfpathlineto{\pgfqpoint{3.501858in}{2.661777in}}%
\pgfpathlineto{\pgfqpoint{3.501858in}{2.664726in}}%
\pgfpathlineto{\pgfqpoint{3.506399in}{2.664726in}}%
\pgfpathlineto{\pgfqpoint{3.506399in}{2.661777in}}%
\pgfpathmoveto{\pgfqpoint{3.429199in}{2.741408in}}%
\pgfpathlineto{\pgfqpoint{3.429199in}{2.741408in}}%
\pgfpathlineto{\pgfqpoint{3.429199in}{2.744357in}}%
\pgfpathlineto{\pgfqpoint{3.433740in}{2.744357in}}%
\pgfpathlineto{\pgfqpoint{3.433740in}{2.741408in}}%
\pgfpathmoveto{\pgfqpoint{3.429199in}{2.744357in}}%
\pgfpathlineto{\pgfqpoint{3.429199in}{2.744357in}}%
\pgfpathlineto{\pgfqpoint{3.429199in}{2.747306in}}%
\pgfpathlineto{\pgfqpoint{3.433740in}{2.747306in}}%
\pgfpathlineto{\pgfqpoint{3.433740in}{2.744357in}}%
\pgfpathmoveto{\pgfqpoint{3.433740in}{2.741408in}}%
\pgfpathlineto{\pgfqpoint{3.433740in}{2.741408in}}%
\pgfpathlineto{\pgfqpoint{3.433740in}{2.744357in}}%
\pgfpathlineto{\pgfqpoint{3.438281in}{2.744357in}}%
\pgfpathlineto{\pgfqpoint{3.438281in}{2.741408in}}%
\pgfpathmoveto{\pgfqpoint{3.433740in}{2.744357in}}%
\pgfpathlineto{\pgfqpoint{3.433740in}{2.744357in}}%
\pgfpathlineto{\pgfqpoint{3.433740in}{2.747306in}}%
\pgfpathlineto{\pgfqpoint{3.438281in}{2.747306in}}%
\pgfpathlineto{\pgfqpoint{3.438281in}{2.744357in}}%
\pgfpathmoveto{\pgfqpoint{3.429199in}{2.747306in}}%
\pgfpathlineto{\pgfqpoint{3.429199in}{2.747306in}}%
\pgfpathlineto{\pgfqpoint{3.429199in}{2.750256in}}%
\pgfpathlineto{\pgfqpoint{3.433740in}{2.750256in}}%
\pgfpathlineto{\pgfqpoint{3.433740in}{2.747306in}}%
\pgfpathmoveto{\pgfqpoint{3.429199in}{2.750256in}}%
\pgfpathlineto{\pgfqpoint{3.429199in}{2.750256in}}%
\pgfpathlineto{\pgfqpoint{3.429199in}{2.753205in}}%
\pgfpathlineto{\pgfqpoint{3.433740in}{2.753205in}}%
\pgfpathlineto{\pgfqpoint{3.433740in}{2.750256in}}%
\pgfpathmoveto{\pgfqpoint{3.433740in}{2.747306in}}%
\pgfpathlineto{\pgfqpoint{3.433740in}{2.747306in}}%
\pgfpathlineto{\pgfqpoint{3.433740in}{2.750256in}}%
\pgfpathlineto{\pgfqpoint{3.438281in}{2.750256in}}%
\pgfpathlineto{\pgfqpoint{3.438281in}{2.747306in}}%
\pgfpathmoveto{\pgfqpoint{3.433740in}{2.750256in}}%
\pgfpathlineto{\pgfqpoint{3.433740in}{2.750256in}}%
\pgfpathlineto{\pgfqpoint{3.433740in}{2.753205in}}%
\pgfpathlineto{\pgfqpoint{3.438281in}{2.753205in}}%
\pgfpathlineto{\pgfqpoint{3.438281in}{2.750256in}}%
\pgfpathmoveto{\pgfqpoint{3.420116in}{2.753205in}}%
\pgfpathlineto{\pgfqpoint{3.420116in}{2.753205in}}%
\pgfpathlineto{\pgfqpoint{3.420116in}{2.756154in}}%
\pgfpathlineto{\pgfqpoint{3.424658in}{2.756154in}}%
\pgfpathlineto{\pgfqpoint{3.424658in}{2.753205in}}%
\pgfpathmoveto{\pgfqpoint{3.420116in}{2.756154in}}%
\pgfpathlineto{\pgfqpoint{3.420116in}{2.756154in}}%
\pgfpathlineto{\pgfqpoint{3.420116in}{2.759104in}}%
\pgfpathlineto{\pgfqpoint{3.424658in}{2.759104in}}%
\pgfpathlineto{\pgfqpoint{3.424658in}{2.756154in}}%
\pgfpathmoveto{\pgfqpoint{3.424658in}{2.753205in}}%
\pgfpathlineto{\pgfqpoint{3.424658in}{2.753205in}}%
\pgfpathlineto{\pgfqpoint{3.424658in}{2.756154in}}%
\pgfpathlineto{\pgfqpoint{3.429199in}{2.756154in}}%
\pgfpathlineto{\pgfqpoint{3.429199in}{2.753205in}}%
\pgfpathmoveto{\pgfqpoint{3.424658in}{2.756154in}}%
\pgfpathlineto{\pgfqpoint{3.424658in}{2.756154in}}%
\pgfpathlineto{\pgfqpoint{3.424658in}{2.759104in}}%
\pgfpathlineto{\pgfqpoint{3.429199in}{2.759104in}}%
\pgfpathlineto{\pgfqpoint{3.429199in}{2.756154in}}%
\pgfpathmoveto{\pgfqpoint{3.420116in}{2.759104in}}%
\pgfpathlineto{\pgfqpoint{3.420116in}{2.759104in}}%
\pgfpathlineto{\pgfqpoint{3.420116in}{2.762053in}}%
\pgfpathlineto{\pgfqpoint{3.424658in}{2.762053in}}%
\pgfpathlineto{\pgfqpoint{3.424658in}{2.759104in}}%
\pgfpathmoveto{\pgfqpoint{3.420116in}{2.762053in}}%
\pgfpathlineto{\pgfqpoint{3.420116in}{2.762053in}}%
\pgfpathlineto{\pgfqpoint{3.420116in}{2.765002in}}%
\pgfpathlineto{\pgfqpoint{3.424658in}{2.765002in}}%
\pgfpathlineto{\pgfqpoint{3.424658in}{2.762053in}}%
\pgfpathmoveto{\pgfqpoint{3.424658in}{2.759104in}}%
\pgfpathlineto{\pgfqpoint{3.424658in}{2.759104in}}%
\pgfpathlineto{\pgfqpoint{3.424658in}{2.762053in}}%
\pgfpathlineto{\pgfqpoint{3.429199in}{2.762053in}}%
\pgfpathlineto{\pgfqpoint{3.429199in}{2.759104in}}%
\pgfpathmoveto{\pgfqpoint{3.424658in}{2.762053in}}%
\pgfpathlineto{\pgfqpoint{3.424658in}{2.762053in}}%
\pgfpathlineto{\pgfqpoint{3.424658in}{2.765002in}}%
\pgfpathlineto{\pgfqpoint{3.429199in}{2.765002in}}%
\pgfpathlineto{\pgfqpoint{3.429199in}{2.762053in}}%
\pgfpathmoveto{\pgfqpoint{3.429199in}{2.753205in}}%
\pgfpathlineto{\pgfqpoint{3.429199in}{2.753205in}}%
\pgfpathlineto{\pgfqpoint{3.429199in}{2.756154in}}%
\pgfpathlineto{\pgfqpoint{3.433740in}{2.756154in}}%
\pgfpathlineto{\pgfqpoint{3.433740in}{2.753205in}}%
\pgfpathmoveto{\pgfqpoint{3.429199in}{2.756154in}}%
\pgfpathlineto{\pgfqpoint{3.429199in}{2.756154in}}%
\pgfpathlineto{\pgfqpoint{3.429199in}{2.759104in}}%
\pgfpathlineto{\pgfqpoint{3.433740in}{2.759104in}}%
\pgfpathlineto{\pgfqpoint{3.433740in}{2.756154in}}%
\pgfpathmoveto{\pgfqpoint{3.465529in}{2.694219in}}%
\pgfpathlineto{\pgfqpoint{3.465529in}{2.694219in}}%
\pgfpathlineto{\pgfqpoint{3.465529in}{2.697168in}}%
\pgfpathlineto{\pgfqpoint{3.470070in}{2.697168in}}%
\pgfpathlineto{\pgfqpoint{3.470070in}{2.694219in}}%
\pgfpathmoveto{\pgfqpoint{3.465529in}{2.697168in}}%
\pgfpathlineto{\pgfqpoint{3.465529in}{2.697168in}}%
\pgfpathlineto{\pgfqpoint{3.465529in}{2.700118in}}%
\pgfpathlineto{\pgfqpoint{3.470070in}{2.700118in}}%
\pgfpathlineto{\pgfqpoint{3.470070in}{2.697168in}}%
\pgfpathmoveto{\pgfqpoint{3.470070in}{2.694219in}}%
\pgfpathlineto{\pgfqpoint{3.470070in}{2.694219in}}%
\pgfpathlineto{\pgfqpoint{3.470070in}{2.697168in}}%
\pgfpathlineto{\pgfqpoint{3.474611in}{2.697168in}}%
\pgfpathlineto{\pgfqpoint{3.474611in}{2.694219in}}%
\pgfpathmoveto{\pgfqpoint{3.470070in}{2.697168in}}%
\pgfpathlineto{\pgfqpoint{3.470070in}{2.697168in}}%
\pgfpathlineto{\pgfqpoint{3.470070in}{2.700118in}}%
\pgfpathlineto{\pgfqpoint{3.474611in}{2.700118in}}%
\pgfpathlineto{\pgfqpoint{3.474611in}{2.697168in}}%
\pgfpathmoveto{\pgfqpoint{3.465529in}{2.700118in}}%
\pgfpathlineto{\pgfqpoint{3.465529in}{2.700118in}}%
\pgfpathlineto{\pgfqpoint{3.465529in}{2.703067in}}%
\pgfpathlineto{\pgfqpoint{3.470070in}{2.703067in}}%
\pgfpathlineto{\pgfqpoint{3.470070in}{2.700118in}}%
\pgfpathmoveto{\pgfqpoint{3.465529in}{2.703067in}}%
\pgfpathlineto{\pgfqpoint{3.465529in}{2.703067in}}%
\pgfpathlineto{\pgfqpoint{3.465529in}{2.706016in}}%
\pgfpathlineto{\pgfqpoint{3.470070in}{2.706016in}}%
\pgfpathlineto{\pgfqpoint{3.470070in}{2.703067in}}%
\pgfpathmoveto{\pgfqpoint{3.470070in}{2.700118in}}%
\pgfpathlineto{\pgfqpoint{3.470070in}{2.700118in}}%
\pgfpathlineto{\pgfqpoint{3.470070in}{2.703067in}}%
\pgfpathlineto{\pgfqpoint{3.474611in}{2.703067in}}%
\pgfpathlineto{\pgfqpoint{3.474611in}{2.700118in}}%
\pgfpathmoveto{\pgfqpoint{3.470070in}{2.703067in}}%
\pgfpathlineto{\pgfqpoint{3.470070in}{2.703067in}}%
\pgfpathlineto{\pgfqpoint{3.470070in}{2.706016in}}%
\pgfpathlineto{\pgfqpoint{3.474611in}{2.706016in}}%
\pgfpathlineto{\pgfqpoint{3.474611in}{2.703067in}}%
\pgfpathmoveto{\pgfqpoint{3.456446in}{2.706016in}}%
\pgfpathlineto{\pgfqpoint{3.456446in}{2.706016in}}%
\pgfpathlineto{\pgfqpoint{3.456446in}{2.708965in}}%
\pgfpathlineto{\pgfqpoint{3.460987in}{2.708965in}}%
\pgfpathlineto{\pgfqpoint{3.460987in}{2.706016in}}%
\pgfpathmoveto{\pgfqpoint{3.456446in}{2.708965in}}%
\pgfpathlineto{\pgfqpoint{3.456446in}{2.708965in}}%
\pgfpathlineto{\pgfqpoint{3.456446in}{2.711915in}}%
\pgfpathlineto{\pgfqpoint{3.460987in}{2.711915in}}%
\pgfpathlineto{\pgfqpoint{3.460987in}{2.708965in}}%
\pgfpathmoveto{\pgfqpoint{3.460987in}{2.706016in}}%
\pgfpathlineto{\pgfqpoint{3.460987in}{2.706016in}}%
\pgfpathlineto{\pgfqpoint{3.460987in}{2.708965in}}%
\pgfpathlineto{\pgfqpoint{3.465529in}{2.708965in}}%
\pgfpathlineto{\pgfqpoint{3.465529in}{2.706016in}}%
\pgfpathmoveto{\pgfqpoint{3.460987in}{2.708965in}}%
\pgfpathlineto{\pgfqpoint{3.460987in}{2.708965in}}%
\pgfpathlineto{\pgfqpoint{3.460987in}{2.711915in}}%
\pgfpathlineto{\pgfqpoint{3.465529in}{2.711915in}}%
\pgfpathlineto{\pgfqpoint{3.465529in}{2.708965in}}%
\pgfpathmoveto{\pgfqpoint{3.456446in}{2.711915in}}%
\pgfpathlineto{\pgfqpoint{3.456446in}{2.711915in}}%
\pgfpathlineto{\pgfqpoint{3.456446in}{2.714864in}}%
\pgfpathlineto{\pgfqpoint{3.460987in}{2.714864in}}%
\pgfpathlineto{\pgfqpoint{3.460987in}{2.711915in}}%
\pgfpathmoveto{\pgfqpoint{3.456446in}{2.714864in}}%
\pgfpathlineto{\pgfqpoint{3.456446in}{2.714864in}}%
\pgfpathlineto{\pgfqpoint{3.456446in}{2.717813in}}%
\pgfpathlineto{\pgfqpoint{3.460987in}{2.717813in}}%
\pgfpathlineto{\pgfqpoint{3.460987in}{2.714864in}}%
\pgfpathmoveto{\pgfqpoint{3.460987in}{2.711915in}}%
\pgfpathlineto{\pgfqpoint{3.460987in}{2.711915in}}%
\pgfpathlineto{\pgfqpoint{3.460987in}{2.714864in}}%
\pgfpathlineto{\pgfqpoint{3.465529in}{2.714864in}}%
\pgfpathlineto{\pgfqpoint{3.465529in}{2.711915in}}%
\pgfpathmoveto{\pgfqpoint{3.460987in}{2.714864in}}%
\pgfpathlineto{\pgfqpoint{3.460987in}{2.714864in}}%
\pgfpathlineto{\pgfqpoint{3.460987in}{2.717813in}}%
\pgfpathlineto{\pgfqpoint{3.465529in}{2.717813in}}%
\pgfpathlineto{\pgfqpoint{3.465529in}{2.714864in}}%
\pgfpathmoveto{\pgfqpoint{3.465529in}{2.706016in}}%
\pgfpathlineto{\pgfqpoint{3.465529in}{2.706016in}}%
\pgfpathlineto{\pgfqpoint{3.465529in}{2.708965in}}%
\pgfpathlineto{\pgfqpoint{3.470070in}{2.708965in}}%
\pgfpathlineto{\pgfqpoint{3.470070in}{2.706016in}}%
\pgfpathmoveto{\pgfqpoint{3.465529in}{2.708965in}}%
\pgfpathlineto{\pgfqpoint{3.465529in}{2.708965in}}%
\pgfpathlineto{\pgfqpoint{3.465529in}{2.711915in}}%
\pgfpathlineto{\pgfqpoint{3.470070in}{2.711915in}}%
\pgfpathlineto{\pgfqpoint{3.470070in}{2.708965in}}%
\pgfpathmoveto{\pgfqpoint{3.483693in}{2.670624in}}%
\pgfpathlineto{\pgfqpoint{3.483693in}{2.670624in}}%
\pgfpathlineto{\pgfqpoint{3.483693in}{2.673574in}}%
\pgfpathlineto{\pgfqpoint{3.488235in}{2.673574in}}%
\pgfpathlineto{\pgfqpoint{3.488235in}{2.670624in}}%
\pgfpathmoveto{\pgfqpoint{3.483693in}{2.673574in}}%
\pgfpathlineto{\pgfqpoint{3.483693in}{2.673574in}}%
\pgfpathlineto{\pgfqpoint{3.483693in}{2.676523in}}%
\pgfpathlineto{\pgfqpoint{3.488235in}{2.676523in}}%
\pgfpathlineto{\pgfqpoint{3.488235in}{2.673574in}}%
\pgfpathmoveto{\pgfqpoint{3.488235in}{2.670624in}}%
\pgfpathlineto{\pgfqpoint{3.488235in}{2.670624in}}%
\pgfpathlineto{\pgfqpoint{3.488235in}{2.673574in}}%
\pgfpathlineto{\pgfqpoint{3.492776in}{2.673574in}}%
\pgfpathlineto{\pgfqpoint{3.492776in}{2.670624in}}%
\pgfpathmoveto{\pgfqpoint{3.488235in}{2.673574in}}%
\pgfpathlineto{\pgfqpoint{3.488235in}{2.673574in}}%
\pgfpathlineto{\pgfqpoint{3.488235in}{2.676523in}}%
\pgfpathlineto{\pgfqpoint{3.492776in}{2.676523in}}%
\pgfpathlineto{\pgfqpoint{3.492776in}{2.673574in}}%
\pgfpathmoveto{\pgfqpoint{3.483693in}{2.676523in}}%
\pgfpathlineto{\pgfqpoint{3.483693in}{2.676523in}}%
\pgfpathlineto{\pgfqpoint{3.483693in}{2.679472in}}%
\pgfpathlineto{\pgfqpoint{3.488235in}{2.679472in}}%
\pgfpathlineto{\pgfqpoint{3.488235in}{2.676523in}}%
\pgfpathmoveto{\pgfqpoint{3.483693in}{2.679472in}}%
\pgfpathlineto{\pgfqpoint{3.483693in}{2.679472in}}%
\pgfpathlineto{\pgfqpoint{3.483693in}{2.682422in}}%
\pgfpathlineto{\pgfqpoint{3.488235in}{2.682422in}}%
\pgfpathlineto{\pgfqpoint{3.488235in}{2.679472in}}%
\pgfpathmoveto{\pgfqpoint{3.488235in}{2.676523in}}%
\pgfpathlineto{\pgfqpoint{3.488235in}{2.676523in}}%
\pgfpathlineto{\pgfqpoint{3.488235in}{2.679472in}}%
\pgfpathlineto{\pgfqpoint{3.492776in}{2.679472in}}%
\pgfpathlineto{\pgfqpoint{3.492776in}{2.676523in}}%
\pgfpathmoveto{\pgfqpoint{3.488235in}{2.679472in}}%
\pgfpathlineto{\pgfqpoint{3.488235in}{2.679472in}}%
\pgfpathlineto{\pgfqpoint{3.488235in}{2.682422in}}%
\pgfpathlineto{\pgfqpoint{3.492776in}{2.682422in}}%
\pgfpathlineto{\pgfqpoint{3.492776in}{2.679472in}}%
\pgfpathmoveto{\pgfqpoint{3.474611in}{2.682422in}}%
\pgfpathlineto{\pgfqpoint{3.474611in}{2.682422in}}%
\pgfpathlineto{\pgfqpoint{3.474611in}{2.685371in}}%
\pgfpathlineto{\pgfqpoint{3.479152in}{2.685371in}}%
\pgfpathlineto{\pgfqpoint{3.479152in}{2.682422in}}%
\pgfpathmoveto{\pgfqpoint{3.474611in}{2.685371in}}%
\pgfpathlineto{\pgfqpoint{3.474611in}{2.685371in}}%
\pgfpathlineto{\pgfqpoint{3.474611in}{2.688320in}}%
\pgfpathlineto{\pgfqpoint{3.479152in}{2.688320in}}%
\pgfpathlineto{\pgfqpoint{3.479152in}{2.685371in}}%
\pgfpathmoveto{\pgfqpoint{3.479152in}{2.682422in}}%
\pgfpathlineto{\pgfqpoint{3.479152in}{2.682422in}}%
\pgfpathlineto{\pgfqpoint{3.479152in}{2.685371in}}%
\pgfpathlineto{\pgfqpoint{3.483693in}{2.685371in}}%
\pgfpathlineto{\pgfqpoint{3.483693in}{2.682422in}}%
\pgfpathmoveto{\pgfqpoint{3.479152in}{2.685371in}}%
\pgfpathlineto{\pgfqpoint{3.479152in}{2.685371in}}%
\pgfpathlineto{\pgfqpoint{3.479152in}{2.688320in}}%
\pgfpathlineto{\pgfqpoint{3.483693in}{2.688320in}}%
\pgfpathlineto{\pgfqpoint{3.483693in}{2.685371in}}%
\pgfpathmoveto{\pgfqpoint{3.474611in}{2.688320in}}%
\pgfpathlineto{\pgfqpoint{3.474611in}{2.688320in}}%
\pgfpathlineto{\pgfqpoint{3.474611in}{2.691270in}}%
\pgfpathlineto{\pgfqpoint{3.479152in}{2.691270in}}%
\pgfpathlineto{\pgfqpoint{3.479152in}{2.688320in}}%
\pgfpathmoveto{\pgfqpoint{3.474611in}{2.691270in}}%
\pgfpathlineto{\pgfqpoint{3.474611in}{2.691270in}}%
\pgfpathlineto{\pgfqpoint{3.474611in}{2.694219in}}%
\pgfpathlineto{\pgfqpoint{3.479152in}{2.694219in}}%
\pgfpathlineto{\pgfqpoint{3.479152in}{2.691270in}}%
\pgfpathmoveto{\pgfqpoint{3.479152in}{2.688320in}}%
\pgfpathlineto{\pgfqpoint{3.479152in}{2.688320in}}%
\pgfpathlineto{\pgfqpoint{3.479152in}{2.691270in}}%
\pgfpathlineto{\pgfqpoint{3.483693in}{2.691270in}}%
\pgfpathlineto{\pgfqpoint{3.483693in}{2.688320in}}%
\pgfpathmoveto{\pgfqpoint{3.479152in}{2.691270in}}%
\pgfpathlineto{\pgfqpoint{3.479152in}{2.691270in}}%
\pgfpathlineto{\pgfqpoint{3.479152in}{2.694219in}}%
\pgfpathlineto{\pgfqpoint{3.483693in}{2.694219in}}%
\pgfpathlineto{\pgfqpoint{3.483693in}{2.691270in}}%
\pgfpathmoveto{\pgfqpoint{3.483693in}{2.682422in}}%
\pgfpathlineto{\pgfqpoint{3.483693in}{2.682422in}}%
\pgfpathlineto{\pgfqpoint{3.483693in}{2.685371in}}%
\pgfpathlineto{\pgfqpoint{3.488235in}{2.685371in}}%
\pgfpathlineto{\pgfqpoint{3.488235in}{2.682422in}}%
\pgfpathmoveto{\pgfqpoint{3.483693in}{2.685371in}}%
\pgfpathlineto{\pgfqpoint{3.483693in}{2.685371in}}%
\pgfpathlineto{\pgfqpoint{3.483693in}{2.688320in}}%
\pgfpathlineto{\pgfqpoint{3.488235in}{2.688320in}}%
\pgfpathlineto{\pgfqpoint{3.488235in}{2.685371in}}%
\pgfpathmoveto{\pgfqpoint{3.492776in}{2.670624in}}%
\pgfpathlineto{\pgfqpoint{3.492776in}{2.670624in}}%
\pgfpathlineto{\pgfqpoint{3.492776in}{2.673574in}}%
\pgfpathlineto{\pgfqpoint{3.497317in}{2.673574in}}%
\pgfpathlineto{\pgfqpoint{3.497317in}{2.670624in}}%
\pgfpathmoveto{\pgfqpoint{3.492776in}{2.673574in}}%
\pgfpathlineto{\pgfqpoint{3.492776in}{2.673574in}}%
\pgfpathlineto{\pgfqpoint{3.492776in}{2.676523in}}%
\pgfpathlineto{\pgfqpoint{3.497317in}{2.676523in}}%
\pgfpathlineto{\pgfqpoint{3.497317in}{2.673574in}}%
\pgfpathmoveto{\pgfqpoint{3.474611in}{2.694219in}}%
\pgfpathlineto{\pgfqpoint{3.474611in}{2.694219in}}%
\pgfpathlineto{\pgfqpoint{3.474611in}{2.697168in}}%
\pgfpathlineto{\pgfqpoint{3.479152in}{2.697168in}}%
\pgfpathlineto{\pgfqpoint{3.479152in}{2.694219in}}%
\pgfpathmoveto{\pgfqpoint{3.474611in}{2.697168in}}%
\pgfpathlineto{\pgfqpoint{3.474611in}{2.697168in}}%
\pgfpathlineto{\pgfqpoint{3.474611in}{2.700118in}}%
\pgfpathlineto{\pgfqpoint{3.479152in}{2.700118in}}%
\pgfpathlineto{\pgfqpoint{3.479152in}{2.697168in}}%
\pgfpathmoveto{\pgfqpoint{3.447364in}{2.717813in}}%
\pgfpathlineto{\pgfqpoint{3.447364in}{2.717813in}}%
\pgfpathlineto{\pgfqpoint{3.447364in}{2.720763in}}%
\pgfpathlineto{\pgfqpoint{3.451905in}{2.720763in}}%
\pgfpathlineto{\pgfqpoint{3.451905in}{2.717813in}}%
\pgfpathmoveto{\pgfqpoint{3.447364in}{2.720763in}}%
\pgfpathlineto{\pgfqpoint{3.447364in}{2.720763in}}%
\pgfpathlineto{\pgfqpoint{3.447364in}{2.723712in}}%
\pgfpathlineto{\pgfqpoint{3.451905in}{2.723712in}}%
\pgfpathlineto{\pgfqpoint{3.451905in}{2.720763in}}%
\pgfpathmoveto{\pgfqpoint{3.451905in}{2.717813in}}%
\pgfpathlineto{\pgfqpoint{3.451905in}{2.717813in}}%
\pgfpathlineto{\pgfqpoint{3.451905in}{2.720763in}}%
\pgfpathlineto{\pgfqpoint{3.456446in}{2.720763in}}%
\pgfpathlineto{\pgfqpoint{3.456446in}{2.717813in}}%
\pgfpathmoveto{\pgfqpoint{3.451905in}{2.720763in}}%
\pgfpathlineto{\pgfqpoint{3.451905in}{2.720763in}}%
\pgfpathlineto{\pgfqpoint{3.451905in}{2.723712in}}%
\pgfpathlineto{\pgfqpoint{3.456446in}{2.723712in}}%
\pgfpathlineto{\pgfqpoint{3.456446in}{2.720763in}}%
\pgfpathmoveto{\pgfqpoint{3.447364in}{2.723712in}}%
\pgfpathlineto{\pgfqpoint{3.447364in}{2.723712in}}%
\pgfpathlineto{\pgfqpoint{3.447364in}{2.726661in}}%
\pgfpathlineto{\pgfqpoint{3.451905in}{2.726661in}}%
\pgfpathlineto{\pgfqpoint{3.451905in}{2.723712in}}%
\pgfpathmoveto{\pgfqpoint{3.447364in}{2.726661in}}%
\pgfpathlineto{\pgfqpoint{3.447364in}{2.726661in}}%
\pgfpathlineto{\pgfqpoint{3.447364in}{2.729611in}}%
\pgfpathlineto{\pgfqpoint{3.451905in}{2.729611in}}%
\pgfpathlineto{\pgfqpoint{3.451905in}{2.726661in}}%
\pgfpathmoveto{\pgfqpoint{3.451905in}{2.723712in}}%
\pgfpathlineto{\pgfqpoint{3.451905in}{2.723712in}}%
\pgfpathlineto{\pgfqpoint{3.451905in}{2.726661in}}%
\pgfpathlineto{\pgfqpoint{3.456446in}{2.726661in}}%
\pgfpathlineto{\pgfqpoint{3.456446in}{2.723712in}}%
\pgfpathmoveto{\pgfqpoint{3.451905in}{2.726661in}}%
\pgfpathlineto{\pgfqpoint{3.451905in}{2.726661in}}%
\pgfpathlineto{\pgfqpoint{3.451905in}{2.729611in}}%
\pgfpathlineto{\pgfqpoint{3.456446in}{2.729611in}}%
\pgfpathlineto{\pgfqpoint{3.456446in}{2.726661in}}%
\pgfpathmoveto{\pgfqpoint{3.438281in}{2.729611in}}%
\pgfpathlineto{\pgfqpoint{3.438281in}{2.729611in}}%
\pgfpathlineto{\pgfqpoint{3.438281in}{2.732560in}}%
\pgfpathlineto{\pgfqpoint{3.442822in}{2.732560in}}%
\pgfpathlineto{\pgfqpoint{3.442822in}{2.729611in}}%
\pgfpathmoveto{\pgfqpoint{3.438281in}{2.732560in}}%
\pgfpathlineto{\pgfqpoint{3.438281in}{2.732560in}}%
\pgfpathlineto{\pgfqpoint{3.438281in}{2.735509in}}%
\pgfpathlineto{\pgfqpoint{3.442822in}{2.735509in}}%
\pgfpathlineto{\pgfqpoint{3.442822in}{2.732560in}}%
\pgfpathmoveto{\pgfqpoint{3.442822in}{2.729611in}}%
\pgfpathlineto{\pgfqpoint{3.442822in}{2.729611in}}%
\pgfpathlineto{\pgfqpoint{3.442822in}{2.732560in}}%
\pgfpathlineto{\pgfqpoint{3.447364in}{2.732560in}}%
\pgfpathlineto{\pgfqpoint{3.447364in}{2.729611in}}%
\pgfpathmoveto{\pgfqpoint{3.442822in}{2.732560in}}%
\pgfpathlineto{\pgfqpoint{3.442822in}{2.732560in}}%
\pgfpathlineto{\pgfqpoint{3.442822in}{2.735509in}}%
\pgfpathlineto{\pgfqpoint{3.447364in}{2.735509in}}%
\pgfpathlineto{\pgfqpoint{3.447364in}{2.732560in}}%
\pgfpathmoveto{\pgfqpoint{3.438281in}{2.735509in}}%
\pgfpathlineto{\pgfqpoint{3.438281in}{2.735509in}}%
\pgfpathlineto{\pgfqpoint{3.438281in}{2.738459in}}%
\pgfpathlineto{\pgfqpoint{3.442822in}{2.738459in}}%
\pgfpathlineto{\pgfqpoint{3.442822in}{2.735509in}}%
\pgfpathmoveto{\pgfqpoint{3.438281in}{2.738459in}}%
\pgfpathlineto{\pgfqpoint{3.438281in}{2.738459in}}%
\pgfpathlineto{\pgfqpoint{3.438281in}{2.741408in}}%
\pgfpathlineto{\pgfqpoint{3.442822in}{2.741408in}}%
\pgfpathlineto{\pgfqpoint{3.442822in}{2.738459in}}%
\pgfpathmoveto{\pgfqpoint{3.442822in}{2.735509in}}%
\pgfpathlineto{\pgfqpoint{3.442822in}{2.735509in}}%
\pgfpathlineto{\pgfqpoint{3.442822in}{2.738459in}}%
\pgfpathlineto{\pgfqpoint{3.447364in}{2.738459in}}%
\pgfpathlineto{\pgfqpoint{3.447364in}{2.735509in}}%
\pgfpathmoveto{\pgfqpoint{3.442822in}{2.738459in}}%
\pgfpathlineto{\pgfqpoint{3.442822in}{2.738459in}}%
\pgfpathlineto{\pgfqpoint{3.442822in}{2.741408in}}%
\pgfpathlineto{\pgfqpoint{3.447364in}{2.741408in}}%
\pgfpathlineto{\pgfqpoint{3.447364in}{2.738459in}}%
\pgfpathmoveto{\pgfqpoint{3.447364in}{2.729611in}}%
\pgfpathlineto{\pgfqpoint{3.447364in}{2.729611in}}%
\pgfpathlineto{\pgfqpoint{3.447364in}{2.732560in}}%
\pgfpathlineto{\pgfqpoint{3.451905in}{2.732560in}}%
\pgfpathlineto{\pgfqpoint{3.451905in}{2.729611in}}%
\pgfpathmoveto{\pgfqpoint{3.447364in}{2.732560in}}%
\pgfpathlineto{\pgfqpoint{3.447364in}{2.732560in}}%
\pgfpathlineto{\pgfqpoint{3.447364in}{2.735509in}}%
\pgfpathlineto{\pgfqpoint{3.451905in}{2.735509in}}%
\pgfpathlineto{\pgfqpoint{3.451905in}{2.732560in}}%
\pgfpathmoveto{\pgfqpoint{3.456446in}{2.717813in}}%
\pgfpathlineto{\pgfqpoint{3.456446in}{2.717813in}}%
\pgfpathlineto{\pgfqpoint{3.456446in}{2.720763in}}%
\pgfpathlineto{\pgfqpoint{3.460987in}{2.720763in}}%
\pgfpathlineto{\pgfqpoint{3.460987in}{2.717813in}}%
\pgfpathmoveto{\pgfqpoint{3.456446in}{2.720763in}}%
\pgfpathlineto{\pgfqpoint{3.456446in}{2.720763in}}%
\pgfpathlineto{\pgfqpoint{3.456446in}{2.723712in}}%
\pgfpathlineto{\pgfqpoint{3.460987in}{2.723712in}}%
\pgfpathlineto{\pgfqpoint{3.460987in}{2.720763in}}%
\pgfpathmoveto{\pgfqpoint{3.438281in}{2.741408in}}%
\pgfpathlineto{\pgfqpoint{3.438281in}{2.741408in}}%
\pgfpathlineto{\pgfqpoint{3.438281in}{2.744357in}}%
\pgfpathlineto{\pgfqpoint{3.442822in}{2.744357in}}%
\pgfpathlineto{\pgfqpoint{3.442822in}{2.741408in}}%
\pgfpathmoveto{\pgfqpoint{3.438281in}{2.744357in}}%
\pgfpathlineto{\pgfqpoint{3.438281in}{2.744357in}}%
\pgfpathlineto{\pgfqpoint{3.438281in}{2.747306in}}%
\pgfpathlineto{\pgfqpoint{3.442822in}{2.747306in}}%
\pgfpathlineto{\pgfqpoint{3.442822in}{2.744357in}}%
\pgfpathmoveto{\pgfqpoint{3.392869in}{2.788596in}}%
\pgfpathlineto{\pgfqpoint{3.392869in}{2.788596in}}%
\pgfpathlineto{\pgfqpoint{3.392869in}{2.791545in}}%
\pgfpathlineto{\pgfqpoint{3.397410in}{2.791545in}}%
\pgfpathlineto{\pgfqpoint{3.397410in}{2.788596in}}%
\pgfpathmoveto{\pgfqpoint{3.392869in}{2.791545in}}%
\pgfpathlineto{\pgfqpoint{3.392869in}{2.791545in}}%
\pgfpathlineto{\pgfqpoint{3.392869in}{2.794494in}}%
\pgfpathlineto{\pgfqpoint{3.397410in}{2.794494in}}%
\pgfpathlineto{\pgfqpoint{3.397410in}{2.791545in}}%
\pgfpathmoveto{\pgfqpoint{3.397410in}{2.788596in}}%
\pgfpathlineto{\pgfqpoint{3.397410in}{2.788596in}}%
\pgfpathlineto{\pgfqpoint{3.397410in}{2.791545in}}%
\pgfpathlineto{\pgfqpoint{3.401952in}{2.791545in}}%
\pgfpathlineto{\pgfqpoint{3.401952in}{2.788596in}}%
\pgfpathmoveto{\pgfqpoint{3.397410in}{2.791545in}}%
\pgfpathlineto{\pgfqpoint{3.397410in}{2.791545in}}%
\pgfpathlineto{\pgfqpoint{3.397410in}{2.794494in}}%
\pgfpathlineto{\pgfqpoint{3.401952in}{2.794494in}}%
\pgfpathlineto{\pgfqpoint{3.401952in}{2.791545in}}%
\pgfpathmoveto{\pgfqpoint{3.392869in}{2.794494in}}%
\pgfpathlineto{\pgfqpoint{3.392869in}{2.794494in}}%
\pgfpathlineto{\pgfqpoint{3.392869in}{2.797443in}}%
\pgfpathlineto{\pgfqpoint{3.397410in}{2.797443in}}%
\pgfpathlineto{\pgfqpoint{3.397410in}{2.794494in}}%
\pgfpathmoveto{\pgfqpoint{3.392869in}{2.797443in}}%
\pgfpathlineto{\pgfqpoint{3.392869in}{2.797443in}}%
\pgfpathlineto{\pgfqpoint{3.392869in}{2.800392in}}%
\pgfpathlineto{\pgfqpoint{3.397410in}{2.800392in}}%
\pgfpathlineto{\pgfqpoint{3.397410in}{2.797443in}}%
\pgfpathmoveto{\pgfqpoint{3.397410in}{2.794494in}}%
\pgfpathlineto{\pgfqpoint{3.397410in}{2.794494in}}%
\pgfpathlineto{\pgfqpoint{3.397410in}{2.797443in}}%
\pgfpathlineto{\pgfqpoint{3.401952in}{2.797443in}}%
\pgfpathlineto{\pgfqpoint{3.401952in}{2.794494in}}%
\pgfpathmoveto{\pgfqpoint{3.397410in}{2.797443in}}%
\pgfpathlineto{\pgfqpoint{3.397410in}{2.797443in}}%
\pgfpathlineto{\pgfqpoint{3.397410in}{2.800392in}}%
\pgfpathlineto{\pgfqpoint{3.401952in}{2.800392in}}%
\pgfpathlineto{\pgfqpoint{3.401952in}{2.797443in}}%
\pgfpathmoveto{\pgfqpoint{3.383787in}{2.800392in}}%
\pgfpathlineto{\pgfqpoint{3.383787in}{2.800392in}}%
\pgfpathlineto{\pgfqpoint{3.383787in}{2.803341in}}%
\pgfpathlineto{\pgfqpoint{3.388328in}{2.803341in}}%
\pgfpathlineto{\pgfqpoint{3.388328in}{2.800392in}}%
\pgfpathmoveto{\pgfqpoint{3.383787in}{2.803341in}}%
\pgfpathlineto{\pgfqpoint{3.383787in}{2.803341in}}%
\pgfpathlineto{\pgfqpoint{3.383787in}{2.806291in}}%
\pgfpathlineto{\pgfqpoint{3.388328in}{2.806291in}}%
\pgfpathlineto{\pgfqpoint{3.388328in}{2.803341in}}%
\pgfpathmoveto{\pgfqpoint{3.388328in}{2.800392in}}%
\pgfpathlineto{\pgfqpoint{3.388328in}{2.800392in}}%
\pgfpathlineto{\pgfqpoint{3.388328in}{2.803341in}}%
\pgfpathlineto{\pgfqpoint{3.392869in}{2.803341in}}%
\pgfpathlineto{\pgfqpoint{3.392869in}{2.800392in}}%
\pgfpathmoveto{\pgfqpoint{3.388328in}{2.803341in}}%
\pgfpathlineto{\pgfqpoint{3.388328in}{2.803341in}}%
\pgfpathlineto{\pgfqpoint{3.388328in}{2.806291in}}%
\pgfpathlineto{\pgfqpoint{3.392869in}{2.806291in}}%
\pgfpathlineto{\pgfqpoint{3.392869in}{2.803341in}}%
\pgfpathmoveto{\pgfqpoint{3.383787in}{2.806291in}}%
\pgfpathlineto{\pgfqpoint{3.383787in}{2.806291in}}%
\pgfpathlineto{\pgfqpoint{3.383787in}{2.809240in}}%
\pgfpathlineto{\pgfqpoint{3.388328in}{2.809240in}}%
\pgfpathlineto{\pgfqpoint{3.388328in}{2.806291in}}%
\pgfpathmoveto{\pgfqpoint{3.383787in}{2.809240in}}%
\pgfpathlineto{\pgfqpoint{3.383787in}{2.809240in}}%
\pgfpathlineto{\pgfqpoint{3.383787in}{2.812189in}}%
\pgfpathlineto{\pgfqpoint{3.388328in}{2.812189in}}%
\pgfpathlineto{\pgfqpoint{3.388328in}{2.809240in}}%
\pgfpathmoveto{\pgfqpoint{3.388328in}{2.806291in}}%
\pgfpathlineto{\pgfqpoint{3.388328in}{2.806291in}}%
\pgfpathlineto{\pgfqpoint{3.388328in}{2.809240in}}%
\pgfpathlineto{\pgfqpoint{3.392869in}{2.809240in}}%
\pgfpathlineto{\pgfqpoint{3.392869in}{2.806291in}}%
\pgfpathmoveto{\pgfqpoint{3.388328in}{2.809240in}}%
\pgfpathlineto{\pgfqpoint{3.388328in}{2.809240in}}%
\pgfpathlineto{\pgfqpoint{3.388328in}{2.812189in}}%
\pgfpathlineto{\pgfqpoint{3.392869in}{2.812189in}}%
\pgfpathlineto{\pgfqpoint{3.392869in}{2.809240in}}%
\pgfpathmoveto{\pgfqpoint{3.392869in}{2.800392in}}%
\pgfpathlineto{\pgfqpoint{3.392869in}{2.800392in}}%
\pgfpathlineto{\pgfqpoint{3.392869in}{2.803341in}}%
\pgfpathlineto{\pgfqpoint{3.397410in}{2.803341in}}%
\pgfpathlineto{\pgfqpoint{3.397410in}{2.800392in}}%
\pgfpathmoveto{\pgfqpoint{3.392869in}{2.803341in}}%
\pgfpathlineto{\pgfqpoint{3.392869in}{2.803341in}}%
\pgfpathlineto{\pgfqpoint{3.392869in}{2.806291in}}%
\pgfpathlineto{\pgfqpoint{3.397410in}{2.806291in}}%
\pgfpathlineto{\pgfqpoint{3.397410in}{2.803341in}}%
\pgfpathmoveto{\pgfqpoint{3.411034in}{2.765002in}}%
\pgfpathlineto{\pgfqpoint{3.411034in}{2.765002in}}%
\pgfpathlineto{\pgfqpoint{3.411034in}{2.767951in}}%
\pgfpathlineto{\pgfqpoint{3.415575in}{2.767951in}}%
\pgfpathlineto{\pgfqpoint{3.415575in}{2.765002in}}%
\pgfpathmoveto{\pgfqpoint{3.411034in}{2.767951in}}%
\pgfpathlineto{\pgfqpoint{3.411034in}{2.767951in}}%
\pgfpathlineto{\pgfqpoint{3.411034in}{2.770901in}}%
\pgfpathlineto{\pgfqpoint{3.415575in}{2.770901in}}%
\pgfpathlineto{\pgfqpoint{3.415575in}{2.767951in}}%
\pgfpathmoveto{\pgfqpoint{3.415575in}{2.765002in}}%
\pgfpathlineto{\pgfqpoint{3.415575in}{2.765002in}}%
\pgfpathlineto{\pgfqpoint{3.415575in}{2.767951in}}%
\pgfpathlineto{\pgfqpoint{3.420116in}{2.767951in}}%
\pgfpathlineto{\pgfqpoint{3.420116in}{2.765002in}}%
\pgfpathmoveto{\pgfqpoint{3.415575in}{2.767951in}}%
\pgfpathlineto{\pgfqpoint{3.415575in}{2.767951in}}%
\pgfpathlineto{\pgfqpoint{3.415575in}{2.770901in}}%
\pgfpathlineto{\pgfqpoint{3.420116in}{2.770901in}}%
\pgfpathlineto{\pgfqpoint{3.420116in}{2.767951in}}%
\pgfpathmoveto{\pgfqpoint{3.411034in}{2.770901in}}%
\pgfpathlineto{\pgfqpoint{3.411034in}{2.770901in}}%
\pgfpathlineto{\pgfqpoint{3.411034in}{2.773850in}}%
\pgfpathlineto{\pgfqpoint{3.415575in}{2.773850in}}%
\pgfpathlineto{\pgfqpoint{3.415575in}{2.770901in}}%
\pgfpathmoveto{\pgfqpoint{3.411034in}{2.773850in}}%
\pgfpathlineto{\pgfqpoint{3.411034in}{2.773850in}}%
\pgfpathlineto{\pgfqpoint{3.411034in}{2.776799in}}%
\pgfpathlineto{\pgfqpoint{3.415575in}{2.776799in}}%
\pgfpathlineto{\pgfqpoint{3.415575in}{2.773850in}}%
\pgfpathmoveto{\pgfqpoint{3.415575in}{2.770901in}}%
\pgfpathlineto{\pgfqpoint{3.415575in}{2.770901in}}%
\pgfpathlineto{\pgfqpoint{3.415575in}{2.773850in}}%
\pgfpathlineto{\pgfqpoint{3.420116in}{2.773850in}}%
\pgfpathlineto{\pgfqpoint{3.420116in}{2.770901in}}%
\pgfpathmoveto{\pgfqpoint{3.415575in}{2.773850in}}%
\pgfpathlineto{\pgfqpoint{3.415575in}{2.773850in}}%
\pgfpathlineto{\pgfqpoint{3.415575in}{2.776799in}}%
\pgfpathlineto{\pgfqpoint{3.420116in}{2.776799in}}%
\pgfpathlineto{\pgfqpoint{3.420116in}{2.773850in}}%
\pgfpathmoveto{\pgfqpoint{3.401952in}{2.776799in}}%
\pgfpathlineto{\pgfqpoint{3.401952in}{2.776799in}}%
\pgfpathlineto{\pgfqpoint{3.401952in}{2.779748in}}%
\pgfpathlineto{\pgfqpoint{3.406493in}{2.779748in}}%
\pgfpathlineto{\pgfqpoint{3.406493in}{2.776799in}}%
\pgfpathmoveto{\pgfqpoint{3.401952in}{2.779748in}}%
\pgfpathlineto{\pgfqpoint{3.401952in}{2.779748in}}%
\pgfpathlineto{\pgfqpoint{3.401952in}{2.782697in}}%
\pgfpathlineto{\pgfqpoint{3.406493in}{2.782697in}}%
\pgfpathlineto{\pgfqpoint{3.406493in}{2.779748in}}%
\pgfpathmoveto{\pgfqpoint{3.406493in}{2.776799in}}%
\pgfpathlineto{\pgfqpoint{3.406493in}{2.776799in}}%
\pgfpathlineto{\pgfqpoint{3.406493in}{2.779748in}}%
\pgfpathlineto{\pgfqpoint{3.411034in}{2.779748in}}%
\pgfpathlineto{\pgfqpoint{3.411034in}{2.776799in}}%
\pgfpathmoveto{\pgfqpoint{3.406493in}{2.779748in}}%
\pgfpathlineto{\pgfqpoint{3.406493in}{2.779748in}}%
\pgfpathlineto{\pgfqpoint{3.406493in}{2.782697in}}%
\pgfpathlineto{\pgfqpoint{3.411034in}{2.782697in}}%
\pgfpathlineto{\pgfqpoint{3.411034in}{2.779748in}}%
\pgfpathmoveto{\pgfqpoint{3.401952in}{2.782697in}}%
\pgfpathlineto{\pgfqpoint{3.401952in}{2.782697in}}%
\pgfpathlineto{\pgfqpoint{3.401952in}{2.785646in}}%
\pgfpathlineto{\pgfqpoint{3.406493in}{2.785646in}}%
\pgfpathlineto{\pgfqpoint{3.406493in}{2.782697in}}%
\pgfpathmoveto{\pgfqpoint{3.401952in}{2.785646in}}%
\pgfpathlineto{\pgfqpoint{3.401952in}{2.785646in}}%
\pgfpathlineto{\pgfqpoint{3.401952in}{2.788596in}}%
\pgfpathlineto{\pgfqpoint{3.406493in}{2.788596in}}%
\pgfpathlineto{\pgfqpoint{3.406493in}{2.785646in}}%
\pgfpathmoveto{\pgfqpoint{3.406493in}{2.782697in}}%
\pgfpathlineto{\pgfqpoint{3.406493in}{2.782697in}}%
\pgfpathlineto{\pgfqpoint{3.406493in}{2.785646in}}%
\pgfpathlineto{\pgfqpoint{3.411034in}{2.785646in}}%
\pgfpathlineto{\pgfqpoint{3.411034in}{2.782697in}}%
\pgfpathmoveto{\pgfqpoint{3.406493in}{2.785646in}}%
\pgfpathlineto{\pgfqpoint{3.406493in}{2.785646in}}%
\pgfpathlineto{\pgfqpoint{3.406493in}{2.788596in}}%
\pgfpathlineto{\pgfqpoint{3.411034in}{2.788596in}}%
\pgfpathlineto{\pgfqpoint{3.411034in}{2.785646in}}%
\pgfpathmoveto{\pgfqpoint{3.411034in}{2.776799in}}%
\pgfpathlineto{\pgfqpoint{3.411034in}{2.776799in}}%
\pgfpathlineto{\pgfqpoint{3.411034in}{2.779748in}}%
\pgfpathlineto{\pgfqpoint{3.415575in}{2.779748in}}%
\pgfpathlineto{\pgfqpoint{3.415575in}{2.776799in}}%
\pgfpathmoveto{\pgfqpoint{3.411034in}{2.779748in}}%
\pgfpathlineto{\pgfqpoint{3.411034in}{2.779748in}}%
\pgfpathlineto{\pgfqpoint{3.411034in}{2.782697in}}%
\pgfpathlineto{\pgfqpoint{3.415575in}{2.782697in}}%
\pgfpathlineto{\pgfqpoint{3.415575in}{2.779748in}}%
\pgfpathmoveto{\pgfqpoint{3.420116in}{2.765002in}}%
\pgfpathlineto{\pgfqpoint{3.420116in}{2.765002in}}%
\pgfpathlineto{\pgfqpoint{3.420116in}{2.767951in}}%
\pgfpathlineto{\pgfqpoint{3.424658in}{2.767951in}}%
\pgfpathlineto{\pgfqpoint{3.424658in}{2.765002in}}%
\pgfpathmoveto{\pgfqpoint{3.420116in}{2.767951in}}%
\pgfpathlineto{\pgfqpoint{3.420116in}{2.767951in}}%
\pgfpathlineto{\pgfqpoint{3.420116in}{2.770901in}}%
\pgfpathlineto{\pgfqpoint{3.424658in}{2.770901in}}%
\pgfpathlineto{\pgfqpoint{3.424658in}{2.767951in}}%
\pgfpathmoveto{\pgfqpoint{3.401952in}{2.788596in}}%
\pgfpathlineto{\pgfqpoint{3.401952in}{2.788596in}}%
\pgfpathlineto{\pgfqpoint{3.401952in}{2.791545in}}%
\pgfpathlineto{\pgfqpoint{3.406493in}{2.791545in}}%
\pgfpathlineto{\pgfqpoint{3.406493in}{2.788596in}}%
\pgfpathmoveto{\pgfqpoint{3.401952in}{2.791545in}}%
\pgfpathlineto{\pgfqpoint{3.401952in}{2.791545in}}%
\pgfpathlineto{\pgfqpoint{3.401952in}{2.794494in}}%
\pgfpathlineto{\pgfqpoint{3.406493in}{2.794494in}}%
\pgfpathlineto{\pgfqpoint{3.406493in}{2.791545in}}%
\pgfpathmoveto{\pgfqpoint{3.374704in}{2.812189in}}%
\pgfpathlineto{\pgfqpoint{3.374704in}{2.812189in}}%
\pgfpathlineto{\pgfqpoint{3.374704in}{2.815138in}}%
\pgfpathlineto{\pgfqpoint{3.379246in}{2.815138in}}%
\pgfpathlineto{\pgfqpoint{3.379246in}{2.812189in}}%
\pgfpathmoveto{\pgfqpoint{3.374704in}{2.815138in}}%
\pgfpathlineto{\pgfqpoint{3.374704in}{2.815138in}}%
\pgfpathlineto{\pgfqpoint{3.374704in}{2.818087in}}%
\pgfpathlineto{\pgfqpoint{3.379246in}{2.818087in}}%
\pgfpathlineto{\pgfqpoint{3.379246in}{2.815138in}}%
\pgfpathmoveto{\pgfqpoint{3.379246in}{2.812189in}}%
\pgfpathlineto{\pgfqpoint{3.379246in}{2.812189in}}%
\pgfpathlineto{\pgfqpoint{3.379246in}{2.815138in}}%
\pgfpathlineto{\pgfqpoint{3.383787in}{2.815138in}}%
\pgfpathlineto{\pgfqpoint{3.383787in}{2.812189in}}%
\pgfpathmoveto{\pgfqpoint{3.379246in}{2.815138in}}%
\pgfpathlineto{\pgfqpoint{3.379246in}{2.815138in}}%
\pgfpathlineto{\pgfqpoint{3.379246in}{2.818087in}}%
\pgfpathlineto{\pgfqpoint{3.383787in}{2.818087in}}%
\pgfpathlineto{\pgfqpoint{3.383787in}{2.815138in}}%
\pgfpathmoveto{\pgfqpoint{3.374704in}{2.818087in}}%
\pgfpathlineto{\pgfqpoint{3.374704in}{2.818087in}}%
\pgfpathlineto{\pgfqpoint{3.374704in}{2.821037in}}%
\pgfpathlineto{\pgfqpoint{3.379246in}{2.821037in}}%
\pgfpathlineto{\pgfqpoint{3.379246in}{2.818087in}}%
\pgfpathmoveto{\pgfqpoint{3.374704in}{2.821037in}}%
\pgfpathlineto{\pgfqpoint{3.374704in}{2.821037in}}%
\pgfpathlineto{\pgfqpoint{3.374704in}{2.823986in}}%
\pgfpathlineto{\pgfqpoint{3.379246in}{2.823986in}}%
\pgfpathlineto{\pgfqpoint{3.379246in}{2.821037in}}%
\pgfpathmoveto{\pgfqpoint{3.379246in}{2.818087in}}%
\pgfpathlineto{\pgfqpoint{3.379246in}{2.818087in}}%
\pgfpathlineto{\pgfqpoint{3.379246in}{2.821037in}}%
\pgfpathlineto{\pgfqpoint{3.383787in}{2.821037in}}%
\pgfpathlineto{\pgfqpoint{3.383787in}{2.818087in}}%
\pgfpathmoveto{\pgfqpoint{3.379246in}{2.821037in}}%
\pgfpathlineto{\pgfqpoint{3.379246in}{2.821037in}}%
\pgfpathlineto{\pgfqpoint{3.379246in}{2.823986in}}%
\pgfpathlineto{\pgfqpoint{3.383787in}{2.823986in}}%
\pgfpathlineto{\pgfqpoint{3.383787in}{2.821037in}}%
\pgfpathmoveto{\pgfqpoint{3.365622in}{2.823986in}}%
\pgfpathlineto{\pgfqpoint{3.365622in}{2.823986in}}%
\pgfpathlineto{\pgfqpoint{3.365622in}{2.826935in}}%
\pgfpathlineto{\pgfqpoint{3.370163in}{2.826935in}}%
\pgfpathlineto{\pgfqpoint{3.370163in}{2.823986in}}%
\pgfpathmoveto{\pgfqpoint{3.365622in}{2.826935in}}%
\pgfpathlineto{\pgfqpoint{3.365622in}{2.826935in}}%
\pgfpathlineto{\pgfqpoint{3.365622in}{2.829884in}}%
\pgfpathlineto{\pgfqpoint{3.370163in}{2.829884in}}%
\pgfpathlineto{\pgfqpoint{3.370163in}{2.826935in}}%
\pgfpathmoveto{\pgfqpoint{3.370163in}{2.823986in}}%
\pgfpathlineto{\pgfqpoint{3.370163in}{2.823986in}}%
\pgfpathlineto{\pgfqpoint{3.370163in}{2.826935in}}%
\pgfpathlineto{\pgfqpoint{3.374704in}{2.826935in}}%
\pgfpathlineto{\pgfqpoint{3.374704in}{2.823986in}}%
\pgfpathmoveto{\pgfqpoint{3.370163in}{2.826935in}}%
\pgfpathlineto{\pgfqpoint{3.370163in}{2.826935in}}%
\pgfpathlineto{\pgfqpoint{3.370163in}{2.829884in}}%
\pgfpathlineto{\pgfqpoint{3.374704in}{2.829884in}}%
\pgfpathlineto{\pgfqpoint{3.374704in}{2.826935in}}%
\pgfpathmoveto{\pgfqpoint{3.365622in}{2.829884in}}%
\pgfpathlineto{\pgfqpoint{3.365622in}{2.829884in}}%
\pgfpathlineto{\pgfqpoint{3.365622in}{2.832833in}}%
\pgfpathlineto{\pgfqpoint{3.370163in}{2.832833in}}%
\pgfpathlineto{\pgfqpoint{3.370163in}{2.829884in}}%
\pgfpathmoveto{\pgfqpoint{3.365622in}{2.832833in}}%
\pgfpathlineto{\pgfqpoint{3.365622in}{2.832833in}}%
\pgfpathlineto{\pgfqpoint{3.365622in}{2.835782in}}%
\pgfpathlineto{\pgfqpoint{3.370163in}{2.835782in}}%
\pgfpathlineto{\pgfqpoint{3.370163in}{2.832833in}}%
\pgfpathmoveto{\pgfqpoint{3.370163in}{2.829884in}}%
\pgfpathlineto{\pgfqpoint{3.370163in}{2.829884in}}%
\pgfpathlineto{\pgfqpoint{3.370163in}{2.832833in}}%
\pgfpathlineto{\pgfqpoint{3.374704in}{2.832833in}}%
\pgfpathlineto{\pgfqpoint{3.374704in}{2.829884in}}%
\pgfpathmoveto{\pgfqpoint{3.370163in}{2.832833in}}%
\pgfpathlineto{\pgfqpoint{3.370163in}{2.832833in}}%
\pgfpathlineto{\pgfqpoint{3.370163in}{2.835782in}}%
\pgfpathlineto{\pgfqpoint{3.374704in}{2.835782in}}%
\pgfpathlineto{\pgfqpoint{3.374704in}{2.832833in}}%
\pgfpathmoveto{\pgfqpoint{3.374704in}{2.823986in}}%
\pgfpathlineto{\pgfqpoint{3.374704in}{2.823986in}}%
\pgfpathlineto{\pgfqpoint{3.374704in}{2.826935in}}%
\pgfpathlineto{\pgfqpoint{3.379246in}{2.826935in}}%
\pgfpathlineto{\pgfqpoint{3.379246in}{2.823986in}}%
\pgfpathmoveto{\pgfqpoint{3.374704in}{2.826935in}}%
\pgfpathlineto{\pgfqpoint{3.374704in}{2.826935in}}%
\pgfpathlineto{\pgfqpoint{3.374704in}{2.829884in}}%
\pgfpathlineto{\pgfqpoint{3.379246in}{2.829884in}}%
\pgfpathlineto{\pgfqpoint{3.379246in}{2.826935in}}%
\pgfpathmoveto{\pgfqpoint{3.383787in}{2.812189in}}%
\pgfpathlineto{\pgfqpoint{3.383787in}{2.812189in}}%
\pgfpathlineto{\pgfqpoint{3.383787in}{2.815138in}}%
\pgfpathlineto{\pgfqpoint{3.388328in}{2.815138in}}%
\pgfpathlineto{\pgfqpoint{3.388328in}{2.812189in}}%
\pgfpathmoveto{\pgfqpoint{3.383787in}{2.815138in}}%
\pgfpathlineto{\pgfqpoint{3.383787in}{2.815138in}}%
\pgfpathlineto{\pgfqpoint{3.383787in}{2.818087in}}%
\pgfpathlineto{\pgfqpoint{3.388328in}{2.818087in}}%
\pgfpathlineto{\pgfqpoint{3.388328in}{2.815138in}}%
\pgfpathmoveto{\pgfqpoint{3.365622in}{2.835782in}}%
\pgfpathlineto{\pgfqpoint{3.365622in}{2.835782in}}%
\pgfpathlineto{\pgfqpoint{3.365622in}{2.838732in}}%
\pgfpathlineto{\pgfqpoint{3.370163in}{2.838732in}}%
\pgfpathlineto{\pgfqpoint{3.370163in}{2.835782in}}%
\pgfpathmoveto{\pgfqpoint{3.365622in}{2.838732in}}%
\pgfpathlineto{\pgfqpoint{3.365622in}{2.838732in}}%
\pgfpathlineto{\pgfqpoint{3.365622in}{2.841681in}}%
\pgfpathlineto{\pgfqpoint{3.370163in}{2.841681in}}%
\pgfpathlineto{\pgfqpoint{3.370163in}{2.838732in}}%
\pgfpathmoveto{\pgfqpoint{3.510941in}{2.293123in}}%
\pgfpathlineto{\pgfqpoint{3.510941in}{2.293123in}}%
\pgfpathlineto{\pgfqpoint{3.510941in}{2.296072in}}%
\pgfpathlineto{\pgfqpoint{3.515481in}{2.296072in}}%
\pgfpathlineto{\pgfqpoint{3.515481in}{2.293123in}}%
\pgfpathmoveto{\pgfqpoint{3.510941in}{2.296072in}}%
\pgfpathlineto{\pgfqpoint{3.510941in}{2.296072in}}%
\pgfpathlineto{\pgfqpoint{3.510941in}{2.299021in}}%
\pgfpathlineto{\pgfqpoint{3.515481in}{2.299021in}}%
\pgfpathlineto{\pgfqpoint{3.515481in}{2.296072in}}%
\pgfpathmoveto{\pgfqpoint{3.515481in}{2.296072in}}%
\pgfpathlineto{\pgfqpoint{3.515481in}{2.296072in}}%
\pgfpathlineto{\pgfqpoint{3.515481in}{2.299021in}}%
\pgfpathlineto{\pgfqpoint{3.520022in}{2.299021in}}%
\pgfpathlineto{\pgfqpoint{3.520022in}{2.296072in}}%
\pgfpathmoveto{\pgfqpoint{3.510941in}{2.299021in}}%
\pgfpathlineto{\pgfqpoint{3.510941in}{2.299021in}}%
\pgfpathlineto{\pgfqpoint{3.510941in}{2.301970in}}%
\pgfpathlineto{\pgfqpoint{3.515481in}{2.301970in}}%
\pgfpathlineto{\pgfqpoint{3.515481in}{2.299021in}}%
\pgfpathmoveto{\pgfqpoint{3.510941in}{2.301970in}}%
\pgfpathlineto{\pgfqpoint{3.510941in}{2.301970in}}%
\pgfpathlineto{\pgfqpoint{3.510941in}{2.304919in}}%
\pgfpathlineto{\pgfqpoint{3.515481in}{2.304919in}}%
\pgfpathlineto{\pgfqpoint{3.515481in}{2.301970in}}%
\pgfpathmoveto{\pgfqpoint{3.515481in}{2.299021in}}%
\pgfpathlineto{\pgfqpoint{3.515481in}{2.299021in}}%
\pgfpathlineto{\pgfqpoint{3.515481in}{2.301970in}}%
\pgfpathlineto{\pgfqpoint{3.520022in}{2.301970in}}%
\pgfpathlineto{\pgfqpoint{3.520022in}{2.299021in}}%
\pgfpathmoveto{\pgfqpoint{3.515481in}{2.301970in}}%
\pgfpathlineto{\pgfqpoint{3.515481in}{2.301970in}}%
\pgfpathlineto{\pgfqpoint{3.515481in}{2.304919in}}%
\pgfpathlineto{\pgfqpoint{3.520022in}{2.304919in}}%
\pgfpathlineto{\pgfqpoint{3.520022in}{2.301970in}}%
\pgfpathmoveto{\pgfqpoint{3.520022in}{2.299021in}}%
\pgfpathlineto{\pgfqpoint{3.520022in}{2.299021in}}%
\pgfpathlineto{\pgfqpoint{3.520022in}{2.301970in}}%
\pgfpathlineto{\pgfqpoint{3.524563in}{2.301970in}}%
\pgfpathlineto{\pgfqpoint{3.524563in}{2.299021in}}%
\pgfpathmoveto{\pgfqpoint{3.520022in}{2.301970in}}%
\pgfpathlineto{\pgfqpoint{3.520022in}{2.301970in}}%
\pgfpathlineto{\pgfqpoint{3.520022in}{2.304919in}}%
\pgfpathlineto{\pgfqpoint{3.524563in}{2.304919in}}%
\pgfpathlineto{\pgfqpoint{3.524563in}{2.301970in}}%
\pgfpathmoveto{\pgfqpoint{3.524563in}{2.301970in}}%
\pgfpathlineto{\pgfqpoint{3.524563in}{2.301970in}}%
\pgfpathlineto{\pgfqpoint{3.524563in}{2.304919in}}%
\pgfpathlineto{\pgfqpoint{3.529104in}{2.304919in}}%
\pgfpathlineto{\pgfqpoint{3.529104in}{2.301970in}}%
\pgfpathmoveto{\pgfqpoint{3.520022in}{2.304919in}}%
\pgfpathlineto{\pgfqpoint{3.520022in}{2.304919in}}%
\pgfpathlineto{\pgfqpoint{3.520022in}{2.307869in}}%
\pgfpathlineto{\pgfqpoint{3.524563in}{2.307869in}}%
\pgfpathlineto{\pgfqpoint{3.524563in}{2.304919in}}%
\pgfpathmoveto{\pgfqpoint{3.520022in}{2.307869in}}%
\pgfpathlineto{\pgfqpoint{3.520022in}{2.307869in}}%
\pgfpathlineto{\pgfqpoint{3.520022in}{2.310818in}}%
\pgfpathlineto{\pgfqpoint{3.524563in}{2.310818in}}%
\pgfpathlineto{\pgfqpoint{3.524563in}{2.307869in}}%
\pgfpathmoveto{\pgfqpoint{3.524563in}{2.304919in}}%
\pgfpathlineto{\pgfqpoint{3.524563in}{2.304919in}}%
\pgfpathlineto{\pgfqpoint{3.524563in}{2.307869in}}%
\pgfpathlineto{\pgfqpoint{3.529104in}{2.307869in}}%
\pgfpathlineto{\pgfqpoint{3.529104in}{2.304919in}}%
\pgfpathmoveto{\pgfqpoint{3.524563in}{2.307869in}}%
\pgfpathlineto{\pgfqpoint{3.524563in}{2.307869in}}%
\pgfpathlineto{\pgfqpoint{3.524563in}{2.310818in}}%
\pgfpathlineto{\pgfqpoint{3.529104in}{2.310818in}}%
\pgfpathlineto{\pgfqpoint{3.529104in}{2.307869in}}%
\pgfpathmoveto{\pgfqpoint{3.529104in}{2.304919in}}%
\pgfpathlineto{\pgfqpoint{3.529104in}{2.304919in}}%
\pgfpathlineto{\pgfqpoint{3.529104in}{2.307869in}}%
\pgfpathlineto{\pgfqpoint{3.533645in}{2.307869in}}%
\pgfpathlineto{\pgfqpoint{3.533645in}{2.304919in}}%
\pgfpathmoveto{\pgfqpoint{3.529104in}{2.307869in}}%
\pgfpathlineto{\pgfqpoint{3.529104in}{2.307869in}}%
\pgfpathlineto{\pgfqpoint{3.529104in}{2.310818in}}%
\pgfpathlineto{\pgfqpoint{3.533645in}{2.310818in}}%
\pgfpathlineto{\pgfqpoint{3.533645in}{2.307869in}}%
\pgfpathmoveto{\pgfqpoint{3.533645in}{2.307869in}}%
\pgfpathlineto{\pgfqpoint{3.533645in}{2.307869in}}%
\pgfpathlineto{\pgfqpoint{3.533645in}{2.310818in}}%
\pgfpathlineto{\pgfqpoint{3.538185in}{2.310818in}}%
\pgfpathlineto{\pgfqpoint{3.538185in}{2.307869in}}%
\pgfpathmoveto{\pgfqpoint{3.529104in}{2.310818in}}%
\pgfpathlineto{\pgfqpoint{3.529104in}{2.310818in}}%
\pgfpathlineto{\pgfqpoint{3.529104in}{2.313767in}}%
\pgfpathlineto{\pgfqpoint{3.533645in}{2.313767in}}%
\pgfpathlineto{\pgfqpoint{3.533645in}{2.310818in}}%
\pgfpathmoveto{\pgfqpoint{3.529104in}{2.313767in}}%
\pgfpathlineto{\pgfqpoint{3.529104in}{2.313767in}}%
\pgfpathlineto{\pgfqpoint{3.529104in}{2.316716in}}%
\pgfpathlineto{\pgfqpoint{3.533645in}{2.316716in}}%
\pgfpathlineto{\pgfqpoint{3.533645in}{2.313767in}}%
\pgfpathmoveto{\pgfqpoint{3.533645in}{2.310818in}}%
\pgfpathlineto{\pgfqpoint{3.533645in}{2.310818in}}%
\pgfpathlineto{\pgfqpoint{3.533645in}{2.313767in}}%
\pgfpathlineto{\pgfqpoint{3.538185in}{2.313767in}}%
\pgfpathlineto{\pgfqpoint{3.538185in}{2.310818in}}%
\pgfpathmoveto{\pgfqpoint{3.533645in}{2.313767in}}%
\pgfpathlineto{\pgfqpoint{3.533645in}{2.313767in}}%
\pgfpathlineto{\pgfqpoint{3.533645in}{2.316716in}}%
\pgfpathlineto{\pgfqpoint{3.538185in}{2.316716in}}%
\pgfpathlineto{\pgfqpoint{3.538185in}{2.313767in}}%
\pgfpathmoveto{\pgfqpoint{3.538185in}{2.310818in}}%
\pgfpathlineto{\pgfqpoint{3.538185in}{2.310818in}}%
\pgfpathlineto{\pgfqpoint{3.538185in}{2.313767in}}%
\pgfpathlineto{\pgfqpoint{3.542726in}{2.313767in}}%
\pgfpathlineto{\pgfqpoint{3.542726in}{2.310818in}}%
\pgfpathmoveto{\pgfqpoint{3.538185in}{2.313767in}}%
\pgfpathlineto{\pgfqpoint{3.538185in}{2.313767in}}%
\pgfpathlineto{\pgfqpoint{3.538185in}{2.316716in}}%
\pgfpathlineto{\pgfqpoint{3.542726in}{2.316716in}}%
\pgfpathlineto{\pgfqpoint{3.542726in}{2.313767in}}%
\pgfpathmoveto{\pgfqpoint{3.542726in}{2.313767in}}%
\pgfpathlineto{\pgfqpoint{3.542726in}{2.313767in}}%
\pgfpathlineto{\pgfqpoint{3.542726in}{2.316716in}}%
\pgfpathlineto{\pgfqpoint{3.547267in}{2.316716in}}%
\pgfpathlineto{\pgfqpoint{3.547267in}{2.313767in}}%
\pgfpathmoveto{\pgfqpoint{3.538185in}{2.316716in}}%
\pgfpathlineto{\pgfqpoint{3.538185in}{2.316716in}}%
\pgfpathlineto{\pgfqpoint{3.538185in}{2.319665in}}%
\pgfpathlineto{\pgfqpoint{3.542726in}{2.319665in}}%
\pgfpathlineto{\pgfqpoint{3.542726in}{2.316716in}}%
\pgfpathmoveto{\pgfqpoint{3.538185in}{2.319665in}}%
\pgfpathlineto{\pgfqpoint{3.538185in}{2.319665in}}%
\pgfpathlineto{\pgfqpoint{3.538185in}{2.322615in}}%
\pgfpathlineto{\pgfqpoint{3.542726in}{2.322615in}}%
\pgfpathlineto{\pgfqpoint{3.542726in}{2.319665in}}%
\pgfpathmoveto{\pgfqpoint{3.542726in}{2.316716in}}%
\pgfpathlineto{\pgfqpoint{3.542726in}{2.316716in}}%
\pgfpathlineto{\pgfqpoint{3.542726in}{2.319665in}}%
\pgfpathlineto{\pgfqpoint{3.547267in}{2.319665in}}%
\pgfpathlineto{\pgfqpoint{3.547267in}{2.316716in}}%
\pgfpathmoveto{\pgfqpoint{3.542726in}{2.319665in}}%
\pgfpathlineto{\pgfqpoint{3.542726in}{2.319665in}}%
\pgfpathlineto{\pgfqpoint{3.542726in}{2.322615in}}%
\pgfpathlineto{\pgfqpoint{3.547267in}{2.322615in}}%
\pgfpathlineto{\pgfqpoint{3.547267in}{2.319665in}}%
\pgfpathmoveto{\pgfqpoint{3.547267in}{2.316716in}}%
\pgfpathlineto{\pgfqpoint{3.547267in}{2.316716in}}%
\pgfpathlineto{\pgfqpoint{3.547267in}{2.319665in}}%
\pgfpathlineto{\pgfqpoint{3.551808in}{2.319665in}}%
\pgfpathlineto{\pgfqpoint{3.551808in}{2.316716in}}%
\pgfpathmoveto{\pgfqpoint{3.547267in}{2.319665in}}%
\pgfpathlineto{\pgfqpoint{3.547267in}{2.319665in}}%
\pgfpathlineto{\pgfqpoint{3.547267in}{2.322615in}}%
\pgfpathlineto{\pgfqpoint{3.551808in}{2.322615in}}%
\pgfpathlineto{\pgfqpoint{3.551808in}{2.319665in}}%
\pgfpathmoveto{\pgfqpoint{3.551808in}{2.319665in}}%
\pgfpathlineto{\pgfqpoint{3.551808in}{2.319665in}}%
\pgfpathlineto{\pgfqpoint{3.551808in}{2.322615in}}%
\pgfpathlineto{\pgfqpoint{3.556349in}{2.322615in}}%
\pgfpathlineto{\pgfqpoint{3.556349in}{2.319665in}}%
\pgfpathmoveto{\pgfqpoint{3.547267in}{2.322615in}}%
\pgfpathlineto{\pgfqpoint{3.547267in}{2.322615in}}%
\pgfpathlineto{\pgfqpoint{3.547267in}{2.325564in}}%
\pgfpathlineto{\pgfqpoint{3.551808in}{2.325564in}}%
\pgfpathlineto{\pgfqpoint{3.551808in}{2.322615in}}%
\pgfpathmoveto{\pgfqpoint{3.547267in}{2.325564in}}%
\pgfpathlineto{\pgfqpoint{3.547267in}{2.325564in}}%
\pgfpathlineto{\pgfqpoint{3.547267in}{2.328513in}}%
\pgfpathlineto{\pgfqpoint{3.551808in}{2.328513in}}%
\pgfpathlineto{\pgfqpoint{3.551808in}{2.325564in}}%
\pgfpathmoveto{\pgfqpoint{3.551808in}{2.322615in}}%
\pgfpathlineto{\pgfqpoint{3.551808in}{2.322615in}}%
\pgfpathlineto{\pgfqpoint{3.551808in}{2.325564in}}%
\pgfpathlineto{\pgfqpoint{3.556349in}{2.325564in}}%
\pgfpathlineto{\pgfqpoint{3.556349in}{2.322615in}}%
\pgfpathmoveto{\pgfqpoint{3.551808in}{2.325564in}}%
\pgfpathlineto{\pgfqpoint{3.551808in}{2.325564in}}%
\pgfpathlineto{\pgfqpoint{3.551808in}{2.328513in}}%
\pgfpathlineto{\pgfqpoint{3.556349in}{2.328513in}}%
\pgfpathlineto{\pgfqpoint{3.556349in}{2.325564in}}%
\pgfpathmoveto{\pgfqpoint{3.556349in}{2.322615in}}%
\pgfpathlineto{\pgfqpoint{3.556349in}{2.322615in}}%
\pgfpathlineto{\pgfqpoint{3.556349in}{2.325564in}}%
\pgfpathlineto{\pgfqpoint{3.560889in}{2.325564in}}%
\pgfpathlineto{\pgfqpoint{3.560889in}{2.322615in}}%
\pgfpathmoveto{\pgfqpoint{3.556349in}{2.325564in}}%
\pgfpathlineto{\pgfqpoint{3.556349in}{2.325564in}}%
\pgfpathlineto{\pgfqpoint{3.556349in}{2.328513in}}%
\pgfpathlineto{\pgfqpoint{3.560889in}{2.328513in}}%
\pgfpathlineto{\pgfqpoint{3.560889in}{2.325564in}}%
\pgfpathmoveto{\pgfqpoint{3.560889in}{2.325564in}}%
\pgfpathlineto{\pgfqpoint{3.560889in}{2.325564in}}%
\pgfpathlineto{\pgfqpoint{3.560889in}{2.328513in}}%
\pgfpathlineto{\pgfqpoint{3.565430in}{2.328513in}}%
\pgfpathlineto{\pgfqpoint{3.565430in}{2.325564in}}%
\pgfpathmoveto{\pgfqpoint{3.556349in}{2.328513in}}%
\pgfpathlineto{\pgfqpoint{3.556349in}{2.328513in}}%
\pgfpathlineto{\pgfqpoint{3.556349in}{2.331462in}}%
\pgfpathlineto{\pgfqpoint{3.560889in}{2.331462in}}%
\pgfpathlineto{\pgfqpoint{3.560889in}{2.328513in}}%
\pgfpathmoveto{\pgfqpoint{3.556349in}{2.331462in}}%
\pgfpathlineto{\pgfqpoint{3.556349in}{2.331462in}}%
\pgfpathlineto{\pgfqpoint{3.556349in}{2.334412in}}%
\pgfpathlineto{\pgfqpoint{3.560889in}{2.334412in}}%
\pgfpathlineto{\pgfqpoint{3.560889in}{2.331462in}}%
\pgfpathmoveto{\pgfqpoint{3.560889in}{2.328513in}}%
\pgfpathlineto{\pgfqpoint{3.560889in}{2.328513in}}%
\pgfpathlineto{\pgfqpoint{3.560889in}{2.331462in}}%
\pgfpathlineto{\pgfqpoint{3.565430in}{2.331462in}}%
\pgfpathlineto{\pgfqpoint{3.565430in}{2.328513in}}%
\pgfpathmoveto{\pgfqpoint{3.560889in}{2.331462in}}%
\pgfpathlineto{\pgfqpoint{3.560889in}{2.331462in}}%
\pgfpathlineto{\pgfqpoint{3.560889in}{2.334412in}}%
\pgfpathlineto{\pgfqpoint{3.565430in}{2.334412in}}%
\pgfpathlineto{\pgfqpoint{3.565430in}{2.331462in}}%
\pgfpathmoveto{\pgfqpoint{3.565430in}{2.328513in}}%
\pgfpathlineto{\pgfqpoint{3.565430in}{2.328513in}}%
\pgfpathlineto{\pgfqpoint{3.565430in}{2.331462in}}%
\pgfpathlineto{\pgfqpoint{3.569971in}{2.331462in}}%
\pgfpathlineto{\pgfqpoint{3.569971in}{2.328513in}}%
\pgfpathmoveto{\pgfqpoint{3.565430in}{2.331462in}}%
\pgfpathlineto{\pgfqpoint{3.565430in}{2.331462in}}%
\pgfpathlineto{\pgfqpoint{3.565430in}{2.334412in}}%
\pgfpathlineto{\pgfqpoint{3.569971in}{2.334412in}}%
\pgfpathlineto{\pgfqpoint{3.569971in}{2.331462in}}%
\pgfpathmoveto{\pgfqpoint{3.569971in}{2.331462in}}%
\pgfpathlineto{\pgfqpoint{3.569971in}{2.331462in}}%
\pgfpathlineto{\pgfqpoint{3.569971in}{2.334412in}}%
\pgfpathlineto{\pgfqpoint{3.574512in}{2.334412in}}%
\pgfpathlineto{\pgfqpoint{3.574512in}{2.331462in}}%
\pgfpathmoveto{\pgfqpoint{3.565430in}{2.334412in}}%
\pgfpathlineto{\pgfqpoint{3.565430in}{2.334412in}}%
\pgfpathlineto{\pgfqpoint{3.565430in}{2.337361in}}%
\pgfpathlineto{\pgfqpoint{3.569971in}{2.337361in}}%
\pgfpathlineto{\pgfqpoint{3.569971in}{2.334412in}}%
\pgfpathmoveto{\pgfqpoint{3.565430in}{2.337361in}}%
\pgfpathlineto{\pgfqpoint{3.565430in}{2.337361in}}%
\pgfpathlineto{\pgfqpoint{3.565430in}{2.340310in}}%
\pgfpathlineto{\pgfqpoint{3.569971in}{2.340310in}}%
\pgfpathlineto{\pgfqpoint{3.569971in}{2.337361in}}%
\pgfpathmoveto{\pgfqpoint{3.569971in}{2.334412in}}%
\pgfpathlineto{\pgfqpoint{3.569971in}{2.334412in}}%
\pgfpathlineto{\pgfqpoint{3.569971in}{2.337361in}}%
\pgfpathlineto{\pgfqpoint{3.574512in}{2.337361in}}%
\pgfpathlineto{\pgfqpoint{3.574512in}{2.334412in}}%
\pgfpathmoveto{\pgfqpoint{3.569971in}{2.337361in}}%
\pgfpathlineto{\pgfqpoint{3.569971in}{2.337361in}}%
\pgfpathlineto{\pgfqpoint{3.569971in}{2.340310in}}%
\pgfpathlineto{\pgfqpoint{3.574512in}{2.340310in}}%
\pgfpathlineto{\pgfqpoint{3.574512in}{2.337361in}}%
\pgfpathmoveto{\pgfqpoint{3.574512in}{2.334412in}}%
\pgfpathlineto{\pgfqpoint{3.574512in}{2.334412in}}%
\pgfpathlineto{\pgfqpoint{3.574512in}{2.337361in}}%
\pgfpathlineto{\pgfqpoint{3.579053in}{2.337361in}}%
\pgfpathlineto{\pgfqpoint{3.579053in}{2.334412in}}%
\pgfpathmoveto{\pgfqpoint{3.574512in}{2.337361in}}%
\pgfpathlineto{\pgfqpoint{3.574512in}{2.337361in}}%
\pgfpathlineto{\pgfqpoint{3.574512in}{2.340310in}}%
\pgfpathlineto{\pgfqpoint{3.579053in}{2.340310in}}%
\pgfpathlineto{\pgfqpoint{3.579053in}{2.337361in}}%
\pgfpathmoveto{\pgfqpoint{3.579053in}{2.337361in}}%
\pgfpathlineto{\pgfqpoint{3.579053in}{2.337361in}}%
\pgfpathlineto{\pgfqpoint{3.579053in}{2.340310in}}%
\pgfpathlineto{\pgfqpoint{3.583593in}{2.340310in}}%
\pgfpathlineto{\pgfqpoint{3.583593in}{2.337361in}}%
\pgfpathmoveto{\pgfqpoint{3.574512in}{2.340310in}}%
\pgfpathlineto{\pgfqpoint{3.574512in}{2.340310in}}%
\pgfpathlineto{\pgfqpoint{3.574512in}{2.343259in}}%
\pgfpathlineto{\pgfqpoint{3.579053in}{2.343259in}}%
\pgfpathlineto{\pgfqpoint{3.579053in}{2.340310in}}%
\pgfpathmoveto{\pgfqpoint{3.574512in}{2.343259in}}%
\pgfpathlineto{\pgfqpoint{3.574512in}{2.343259in}}%
\pgfpathlineto{\pgfqpoint{3.574512in}{2.346208in}}%
\pgfpathlineto{\pgfqpoint{3.579053in}{2.346208in}}%
\pgfpathlineto{\pgfqpoint{3.579053in}{2.343259in}}%
\pgfpathmoveto{\pgfqpoint{3.579053in}{2.340310in}}%
\pgfpathlineto{\pgfqpoint{3.579053in}{2.340310in}}%
\pgfpathlineto{\pgfqpoint{3.579053in}{2.343259in}}%
\pgfpathlineto{\pgfqpoint{3.583593in}{2.343259in}}%
\pgfpathlineto{\pgfqpoint{3.583593in}{2.340310in}}%
\pgfpathmoveto{\pgfqpoint{3.579053in}{2.343259in}}%
\pgfpathlineto{\pgfqpoint{3.579053in}{2.343259in}}%
\pgfpathlineto{\pgfqpoint{3.579053in}{2.346208in}}%
\pgfpathlineto{\pgfqpoint{3.583593in}{2.346208in}}%
\pgfpathlineto{\pgfqpoint{3.583593in}{2.343259in}}%
\pgfpathmoveto{\pgfqpoint{3.583593in}{2.340310in}}%
\pgfpathlineto{\pgfqpoint{3.583593in}{2.340310in}}%
\pgfpathlineto{\pgfqpoint{3.583593in}{2.343259in}}%
\pgfpathlineto{\pgfqpoint{3.588134in}{2.343259in}}%
\pgfpathlineto{\pgfqpoint{3.588134in}{2.340310in}}%
\pgfpathmoveto{\pgfqpoint{3.583593in}{2.343259in}}%
\pgfpathlineto{\pgfqpoint{3.583593in}{2.343259in}}%
\pgfpathlineto{\pgfqpoint{3.583593in}{2.346208in}}%
\pgfpathlineto{\pgfqpoint{3.588134in}{2.346208in}}%
\pgfpathlineto{\pgfqpoint{3.588134in}{2.343259in}}%
\pgfpathmoveto{\pgfqpoint{3.588134in}{2.343259in}}%
\pgfpathlineto{\pgfqpoint{3.588134in}{2.343259in}}%
\pgfpathlineto{\pgfqpoint{3.588134in}{2.346208in}}%
\pgfpathlineto{\pgfqpoint{3.592675in}{2.346208in}}%
\pgfpathlineto{\pgfqpoint{3.592675in}{2.343259in}}%
\pgfpathmoveto{\pgfqpoint{3.583593in}{2.346208in}}%
\pgfpathlineto{\pgfqpoint{3.583593in}{2.346208in}}%
\pgfpathlineto{\pgfqpoint{3.583593in}{2.349158in}}%
\pgfpathlineto{\pgfqpoint{3.588134in}{2.349158in}}%
\pgfpathlineto{\pgfqpoint{3.588134in}{2.346208in}}%
\pgfpathmoveto{\pgfqpoint{3.583593in}{2.349158in}}%
\pgfpathlineto{\pgfqpoint{3.583593in}{2.349158in}}%
\pgfpathlineto{\pgfqpoint{3.583593in}{2.352107in}}%
\pgfpathlineto{\pgfqpoint{3.588134in}{2.352107in}}%
\pgfpathlineto{\pgfqpoint{3.588134in}{2.349158in}}%
\pgfpathmoveto{\pgfqpoint{3.588134in}{2.346208in}}%
\pgfpathlineto{\pgfqpoint{3.588134in}{2.346208in}}%
\pgfpathlineto{\pgfqpoint{3.588134in}{2.349158in}}%
\pgfpathlineto{\pgfqpoint{3.592675in}{2.349158in}}%
\pgfpathlineto{\pgfqpoint{3.592675in}{2.346208in}}%
\pgfpathmoveto{\pgfqpoint{3.588134in}{2.349158in}}%
\pgfpathlineto{\pgfqpoint{3.588134in}{2.349158in}}%
\pgfpathlineto{\pgfqpoint{3.588134in}{2.352107in}}%
\pgfpathlineto{\pgfqpoint{3.592675in}{2.352107in}}%
\pgfpathlineto{\pgfqpoint{3.592675in}{2.349158in}}%
\pgfpathmoveto{\pgfqpoint{3.592675in}{2.346208in}}%
\pgfpathlineto{\pgfqpoint{3.592675in}{2.346208in}}%
\pgfpathlineto{\pgfqpoint{3.592675in}{2.349158in}}%
\pgfpathlineto{\pgfqpoint{3.597216in}{2.349158in}}%
\pgfpathlineto{\pgfqpoint{3.597216in}{2.346208in}}%
\pgfpathmoveto{\pgfqpoint{3.592675in}{2.349158in}}%
\pgfpathlineto{\pgfqpoint{3.592675in}{2.349158in}}%
\pgfpathlineto{\pgfqpoint{3.592675in}{2.352107in}}%
\pgfpathlineto{\pgfqpoint{3.597216in}{2.352107in}}%
\pgfpathlineto{\pgfqpoint{3.597216in}{2.349158in}}%
\pgfpathmoveto{\pgfqpoint{3.597216in}{2.349158in}}%
\pgfpathlineto{\pgfqpoint{3.597216in}{2.349158in}}%
\pgfpathlineto{\pgfqpoint{3.597216in}{2.352107in}}%
\pgfpathlineto{\pgfqpoint{3.601757in}{2.352107in}}%
\pgfpathlineto{\pgfqpoint{3.601757in}{2.349158in}}%
\pgfpathmoveto{\pgfqpoint{3.592675in}{2.352107in}}%
\pgfpathlineto{\pgfqpoint{3.592675in}{2.352107in}}%
\pgfpathlineto{\pgfqpoint{3.592675in}{2.355056in}}%
\pgfpathlineto{\pgfqpoint{3.597216in}{2.355056in}}%
\pgfpathlineto{\pgfqpoint{3.597216in}{2.352107in}}%
\pgfpathmoveto{\pgfqpoint{3.592675in}{2.355056in}}%
\pgfpathlineto{\pgfqpoint{3.592675in}{2.355056in}}%
\pgfpathlineto{\pgfqpoint{3.592675in}{2.358005in}}%
\pgfpathlineto{\pgfqpoint{3.597216in}{2.358005in}}%
\pgfpathlineto{\pgfqpoint{3.597216in}{2.355056in}}%
\pgfpathmoveto{\pgfqpoint{3.597216in}{2.352107in}}%
\pgfpathlineto{\pgfqpoint{3.597216in}{2.352107in}}%
\pgfpathlineto{\pgfqpoint{3.597216in}{2.355056in}}%
\pgfpathlineto{\pgfqpoint{3.601757in}{2.355056in}}%
\pgfpathlineto{\pgfqpoint{3.601757in}{2.352107in}}%
\pgfpathmoveto{\pgfqpoint{3.597216in}{2.355056in}}%
\pgfpathlineto{\pgfqpoint{3.597216in}{2.355056in}}%
\pgfpathlineto{\pgfqpoint{3.597216in}{2.358005in}}%
\pgfpathlineto{\pgfqpoint{3.601757in}{2.358005in}}%
\pgfpathlineto{\pgfqpoint{3.601757in}{2.355056in}}%
\pgfpathmoveto{\pgfqpoint{3.601757in}{2.352107in}}%
\pgfpathlineto{\pgfqpoint{3.601757in}{2.352107in}}%
\pgfpathlineto{\pgfqpoint{3.601757in}{2.355056in}}%
\pgfpathlineto{\pgfqpoint{3.606297in}{2.355056in}}%
\pgfpathlineto{\pgfqpoint{3.606297in}{2.352107in}}%
\pgfpathmoveto{\pgfqpoint{3.601757in}{2.355056in}}%
\pgfpathlineto{\pgfqpoint{3.601757in}{2.355056in}}%
\pgfpathlineto{\pgfqpoint{3.601757in}{2.358005in}}%
\pgfpathlineto{\pgfqpoint{3.606297in}{2.358005in}}%
\pgfpathlineto{\pgfqpoint{3.606297in}{2.355056in}}%
\pgfpathmoveto{\pgfqpoint{3.606297in}{2.355056in}}%
\pgfpathlineto{\pgfqpoint{3.606297in}{2.355056in}}%
\pgfpathlineto{\pgfqpoint{3.606297in}{2.358005in}}%
\pgfpathlineto{\pgfqpoint{3.610838in}{2.358005in}}%
\pgfpathlineto{\pgfqpoint{3.610838in}{2.355056in}}%
\pgfpathmoveto{\pgfqpoint{3.601757in}{2.358005in}}%
\pgfpathlineto{\pgfqpoint{3.601757in}{2.358005in}}%
\pgfpathlineto{\pgfqpoint{3.601757in}{2.360955in}}%
\pgfpathlineto{\pgfqpoint{3.606297in}{2.360955in}}%
\pgfpathlineto{\pgfqpoint{3.606297in}{2.358005in}}%
\pgfpathmoveto{\pgfqpoint{3.601757in}{2.360955in}}%
\pgfpathlineto{\pgfqpoint{3.601757in}{2.360955in}}%
\pgfpathlineto{\pgfqpoint{3.601757in}{2.363904in}}%
\pgfpathlineto{\pgfqpoint{3.606297in}{2.363904in}}%
\pgfpathlineto{\pgfqpoint{3.606297in}{2.360955in}}%
\pgfpathmoveto{\pgfqpoint{3.606297in}{2.358005in}}%
\pgfpathlineto{\pgfqpoint{3.606297in}{2.358005in}}%
\pgfpathlineto{\pgfqpoint{3.606297in}{2.360955in}}%
\pgfpathlineto{\pgfqpoint{3.610838in}{2.360955in}}%
\pgfpathlineto{\pgfqpoint{3.610838in}{2.358005in}}%
\pgfpathmoveto{\pgfqpoint{3.606297in}{2.360955in}}%
\pgfpathlineto{\pgfqpoint{3.606297in}{2.360955in}}%
\pgfpathlineto{\pgfqpoint{3.606297in}{2.363904in}}%
\pgfpathlineto{\pgfqpoint{3.610838in}{2.363904in}}%
\pgfpathlineto{\pgfqpoint{3.610838in}{2.360955in}}%
\pgfpathmoveto{\pgfqpoint{3.610838in}{2.358005in}}%
\pgfpathlineto{\pgfqpoint{3.610838in}{2.358005in}}%
\pgfpathlineto{\pgfqpoint{3.610838in}{2.360955in}}%
\pgfpathlineto{\pgfqpoint{3.615379in}{2.360955in}}%
\pgfpathlineto{\pgfqpoint{3.615379in}{2.358005in}}%
\pgfpathmoveto{\pgfqpoint{3.610838in}{2.360955in}}%
\pgfpathlineto{\pgfqpoint{3.610838in}{2.360955in}}%
\pgfpathlineto{\pgfqpoint{3.610838in}{2.363904in}}%
\pgfpathlineto{\pgfqpoint{3.615379in}{2.363904in}}%
\pgfpathlineto{\pgfqpoint{3.615379in}{2.360955in}}%
\pgfpathmoveto{\pgfqpoint{3.615379in}{2.360955in}}%
\pgfpathlineto{\pgfqpoint{3.615379in}{2.360955in}}%
\pgfpathlineto{\pgfqpoint{3.615379in}{2.363904in}}%
\pgfpathlineto{\pgfqpoint{3.619920in}{2.363904in}}%
\pgfpathlineto{\pgfqpoint{3.619920in}{2.360955in}}%
\pgfpathmoveto{\pgfqpoint{3.610838in}{2.363904in}}%
\pgfpathlineto{\pgfqpoint{3.610838in}{2.363904in}}%
\pgfpathlineto{\pgfqpoint{3.610838in}{2.366853in}}%
\pgfpathlineto{\pgfqpoint{3.615379in}{2.366853in}}%
\pgfpathlineto{\pgfqpoint{3.615379in}{2.363904in}}%
\pgfpathmoveto{\pgfqpoint{3.610838in}{2.366853in}}%
\pgfpathlineto{\pgfqpoint{3.610838in}{2.366853in}}%
\pgfpathlineto{\pgfqpoint{3.610838in}{2.369802in}}%
\pgfpathlineto{\pgfqpoint{3.615379in}{2.369802in}}%
\pgfpathlineto{\pgfqpoint{3.615379in}{2.366853in}}%
\pgfpathmoveto{\pgfqpoint{3.615379in}{2.363904in}}%
\pgfpathlineto{\pgfqpoint{3.615379in}{2.363904in}}%
\pgfpathlineto{\pgfqpoint{3.615379in}{2.366853in}}%
\pgfpathlineto{\pgfqpoint{3.619920in}{2.366853in}}%
\pgfpathlineto{\pgfqpoint{3.619920in}{2.363904in}}%
\pgfpathmoveto{\pgfqpoint{3.615379in}{2.366853in}}%
\pgfpathlineto{\pgfqpoint{3.615379in}{2.366853in}}%
\pgfpathlineto{\pgfqpoint{3.615379in}{2.369802in}}%
\pgfpathlineto{\pgfqpoint{3.619920in}{2.369802in}}%
\pgfpathlineto{\pgfqpoint{3.619920in}{2.366853in}}%
\pgfpathmoveto{\pgfqpoint{3.619920in}{2.363904in}}%
\pgfpathlineto{\pgfqpoint{3.619920in}{2.363904in}}%
\pgfpathlineto{\pgfqpoint{3.619920in}{2.366853in}}%
\pgfpathlineto{\pgfqpoint{3.624461in}{2.366853in}}%
\pgfpathlineto{\pgfqpoint{3.624461in}{2.363904in}}%
\pgfpathmoveto{\pgfqpoint{3.619920in}{2.366853in}}%
\pgfpathlineto{\pgfqpoint{3.619920in}{2.366853in}}%
\pgfpathlineto{\pgfqpoint{3.619920in}{2.369802in}}%
\pgfpathlineto{\pgfqpoint{3.624461in}{2.369802in}}%
\pgfpathlineto{\pgfqpoint{3.624461in}{2.366853in}}%
\pgfpathmoveto{\pgfqpoint{3.624461in}{2.366853in}}%
\pgfpathlineto{\pgfqpoint{3.624461in}{2.366853in}}%
\pgfpathlineto{\pgfqpoint{3.624461in}{2.369802in}}%
\pgfpathlineto{\pgfqpoint{3.629001in}{2.369802in}}%
\pgfpathlineto{\pgfqpoint{3.629001in}{2.366853in}}%
\pgfpathmoveto{\pgfqpoint{3.619920in}{2.369802in}}%
\pgfpathlineto{\pgfqpoint{3.619920in}{2.369802in}}%
\pgfpathlineto{\pgfqpoint{3.619920in}{2.372751in}}%
\pgfpathlineto{\pgfqpoint{3.624461in}{2.372751in}}%
\pgfpathlineto{\pgfqpoint{3.624461in}{2.369802in}}%
\pgfpathmoveto{\pgfqpoint{3.619920in}{2.372751in}}%
\pgfpathlineto{\pgfqpoint{3.619920in}{2.372751in}}%
\pgfpathlineto{\pgfqpoint{3.619920in}{2.375701in}}%
\pgfpathlineto{\pgfqpoint{3.624461in}{2.375701in}}%
\pgfpathlineto{\pgfqpoint{3.624461in}{2.372751in}}%
\pgfpathmoveto{\pgfqpoint{3.624461in}{2.369802in}}%
\pgfpathlineto{\pgfqpoint{3.624461in}{2.369802in}}%
\pgfpathlineto{\pgfqpoint{3.624461in}{2.372751in}}%
\pgfpathlineto{\pgfqpoint{3.629001in}{2.372751in}}%
\pgfpathlineto{\pgfqpoint{3.629001in}{2.369802in}}%
\pgfpathmoveto{\pgfqpoint{3.624461in}{2.372751in}}%
\pgfpathlineto{\pgfqpoint{3.624461in}{2.372751in}}%
\pgfpathlineto{\pgfqpoint{3.624461in}{2.375701in}}%
\pgfpathlineto{\pgfqpoint{3.629001in}{2.375701in}}%
\pgfpathlineto{\pgfqpoint{3.629001in}{2.372751in}}%
\pgfpathmoveto{\pgfqpoint{3.629001in}{2.369802in}}%
\pgfpathlineto{\pgfqpoint{3.629001in}{2.369802in}}%
\pgfpathlineto{\pgfqpoint{3.629001in}{2.372751in}}%
\pgfpathlineto{\pgfqpoint{3.633542in}{2.372751in}}%
\pgfpathlineto{\pgfqpoint{3.633542in}{2.369802in}}%
\pgfpathmoveto{\pgfqpoint{3.629001in}{2.372751in}}%
\pgfpathlineto{\pgfqpoint{3.629001in}{2.372751in}}%
\pgfpathlineto{\pgfqpoint{3.629001in}{2.375701in}}%
\pgfpathlineto{\pgfqpoint{3.633542in}{2.375701in}}%
\pgfpathlineto{\pgfqpoint{3.633542in}{2.372751in}}%
\pgfpathmoveto{\pgfqpoint{3.633542in}{2.372751in}}%
\pgfpathlineto{\pgfqpoint{3.633542in}{2.372751in}}%
\pgfpathlineto{\pgfqpoint{3.633542in}{2.375701in}}%
\pgfpathlineto{\pgfqpoint{3.638083in}{2.375701in}}%
\pgfpathlineto{\pgfqpoint{3.638083in}{2.372751in}}%
\pgfpathmoveto{\pgfqpoint{3.629001in}{2.375701in}}%
\pgfpathlineto{\pgfqpoint{3.629001in}{2.375701in}}%
\pgfpathlineto{\pgfqpoint{3.629001in}{2.378650in}}%
\pgfpathlineto{\pgfqpoint{3.633542in}{2.378650in}}%
\pgfpathlineto{\pgfqpoint{3.633542in}{2.375701in}}%
\pgfpathmoveto{\pgfqpoint{3.629001in}{2.378650in}}%
\pgfpathlineto{\pgfqpoint{3.629001in}{2.378650in}}%
\pgfpathlineto{\pgfqpoint{3.629001in}{2.381599in}}%
\pgfpathlineto{\pgfqpoint{3.633542in}{2.381599in}}%
\pgfpathlineto{\pgfqpoint{3.633542in}{2.378650in}}%
\pgfpathmoveto{\pgfqpoint{3.633542in}{2.375701in}}%
\pgfpathlineto{\pgfqpoint{3.633542in}{2.375701in}}%
\pgfpathlineto{\pgfqpoint{3.633542in}{2.378650in}}%
\pgfpathlineto{\pgfqpoint{3.638083in}{2.378650in}}%
\pgfpathlineto{\pgfqpoint{3.638083in}{2.375701in}}%
\pgfpathmoveto{\pgfqpoint{3.633542in}{2.378650in}}%
\pgfpathlineto{\pgfqpoint{3.633542in}{2.378650in}}%
\pgfpathlineto{\pgfqpoint{3.633542in}{2.381599in}}%
\pgfpathlineto{\pgfqpoint{3.638083in}{2.381599in}}%
\pgfpathlineto{\pgfqpoint{3.638083in}{2.378650in}}%
\pgfpathmoveto{\pgfqpoint{3.638083in}{2.375701in}}%
\pgfpathlineto{\pgfqpoint{3.638083in}{2.375701in}}%
\pgfpathlineto{\pgfqpoint{3.638083in}{2.378650in}}%
\pgfpathlineto{\pgfqpoint{3.642624in}{2.378650in}}%
\pgfpathlineto{\pgfqpoint{3.642624in}{2.375701in}}%
\pgfpathmoveto{\pgfqpoint{3.638083in}{2.378650in}}%
\pgfpathlineto{\pgfqpoint{3.638083in}{2.378650in}}%
\pgfpathlineto{\pgfqpoint{3.638083in}{2.381599in}}%
\pgfpathlineto{\pgfqpoint{3.642624in}{2.381599in}}%
\pgfpathlineto{\pgfqpoint{3.642624in}{2.378650in}}%
\pgfpathmoveto{\pgfqpoint{3.642624in}{2.378650in}}%
\pgfpathlineto{\pgfqpoint{3.642624in}{2.378650in}}%
\pgfpathlineto{\pgfqpoint{3.642624in}{2.381599in}}%
\pgfpathlineto{\pgfqpoint{3.647165in}{2.381599in}}%
\pgfpathlineto{\pgfqpoint{3.647165in}{2.378650in}}%
\pgfpathmoveto{\pgfqpoint{3.638083in}{2.381599in}}%
\pgfpathlineto{\pgfqpoint{3.638083in}{2.381599in}}%
\pgfpathlineto{\pgfqpoint{3.638083in}{2.384548in}}%
\pgfpathlineto{\pgfqpoint{3.642624in}{2.384548in}}%
\pgfpathlineto{\pgfqpoint{3.642624in}{2.381599in}}%
\pgfpathmoveto{\pgfqpoint{3.638083in}{2.384548in}}%
\pgfpathlineto{\pgfqpoint{3.638083in}{2.384548in}}%
\pgfpathlineto{\pgfqpoint{3.638083in}{2.387497in}}%
\pgfpathlineto{\pgfqpoint{3.642624in}{2.387497in}}%
\pgfpathlineto{\pgfqpoint{3.642624in}{2.384548in}}%
\pgfpathmoveto{\pgfqpoint{3.642624in}{2.381599in}}%
\pgfpathlineto{\pgfqpoint{3.642624in}{2.381599in}}%
\pgfpathlineto{\pgfqpoint{3.642624in}{2.384548in}}%
\pgfpathlineto{\pgfqpoint{3.647165in}{2.384548in}}%
\pgfpathlineto{\pgfqpoint{3.647165in}{2.381599in}}%
\pgfpathmoveto{\pgfqpoint{3.642624in}{2.384548in}}%
\pgfpathlineto{\pgfqpoint{3.642624in}{2.384548in}}%
\pgfpathlineto{\pgfqpoint{3.642624in}{2.387497in}}%
\pgfpathlineto{\pgfqpoint{3.647165in}{2.387497in}}%
\pgfpathlineto{\pgfqpoint{3.647165in}{2.384548in}}%
\pgfpathmoveto{\pgfqpoint{3.647165in}{2.381599in}}%
\pgfpathlineto{\pgfqpoint{3.647165in}{2.381599in}}%
\pgfpathlineto{\pgfqpoint{3.647165in}{2.384548in}}%
\pgfpathlineto{\pgfqpoint{3.651705in}{2.384548in}}%
\pgfpathlineto{\pgfqpoint{3.651705in}{2.381599in}}%
\pgfpathmoveto{\pgfqpoint{3.647165in}{2.384548in}}%
\pgfpathlineto{\pgfqpoint{3.647165in}{2.384548in}}%
\pgfpathlineto{\pgfqpoint{3.647165in}{2.387497in}}%
\pgfpathlineto{\pgfqpoint{3.651705in}{2.387497in}}%
\pgfpathlineto{\pgfqpoint{3.651705in}{2.384548in}}%
\pgfpathmoveto{\pgfqpoint{3.651705in}{2.384548in}}%
\pgfpathlineto{\pgfqpoint{3.651705in}{2.384548in}}%
\pgfpathlineto{\pgfqpoint{3.651705in}{2.387497in}}%
\pgfpathlineto{\pgfqpoint{3.656246in}{2.387497in}}%
\pgfpathlineto{\pgfqpoint{3.656246in}{2.384548in}}%
\pgfpathmoveto{\pgfqpoint{3.647165in}{2.387497in}}%
\pgfpathlineto{\pgfqpoint{3.647165in}{2.387497in}}%
\pgfpathlineto{\pgfqpoint{3.647165in}{2.390447in}}%
\pgfpathlineto{\pgfqpoint{3.651705in}{2.390447in}}%
\pgfpathlineto{\pgfqpoint{3.651705in}{2.387497in}}%
\pgfpathmoveto{\pgfqpoint{3.647165in}{2.390447in}}%
\pgfpathlineto{\pgfqpoint{3.647165in}{2.390447in}}%
\pgfpathlineto{\pgfqpoint{3.647165in}{2.393396in}}%
\pgfpathlineto{\pgfqpoint{3.651705in}{2.393396in}}%
\pgfpathlineto{\pgfqpoint{3.651705in}{2.390447in}}%
\pgfpathmoveto{\pgfqpoint{3.651705in}{2.387497in}}%
\pgfpathlineto{\pgfqpoint{3.651705in}{2.387497in}}%
\pgfpathlineto{\pgfqpoint{3.651705in}{2.390447in}}%
\pgfpathlineto{\pgfqpoint{3.656246in}{2.390447in}}%
\pgfpathlineto{\pgfqpoint{3.656246in}{2.387497in}}%
\pgfpathmoveto{\pgfqpoint{3.651705in}{2.390447in}}%
\pgfpathlineto{\pgfqpoint{3.651705in}{2.390447in}}%
\pgfpathlineto{\pgfqpoint{3.651705in}{2.393396in}}%
\pgfpathlineto{\pgfqpoint{3.656246in}{2.393396in}}%
\pgfpathlineto{\pgfqpoint{3.656246in}{2.390447in}}%
\pgfpathmoveto{\pgfqpoint{3.647165in}{2.458279in}}%
\pgfpathlineto{\pgfqpoint{3.647165in}{2.458279in}}%
\pgfpathlineto{\pgfqpoint{3.647165in}{2.461228in}}%
\pgfpathlineto{\pgfqpoint{3.651705in}{2.461228in}}%
\pgfpathlineto{\pgfqpoint{3.651705in}{2.458279in}}%
\pgfpathmoveto{\pgfqpoint{3.647165in}{2.461228in}}%
\pgfpathlineto{\pgfqpoint{3.647165in}{2.461228in}}%
\pgfpathlineto{\pgfqpoint{3.647165in}{2.464177in}}%
\pgfpathlineto{\pgfqpoint{3.651705in}{2.464177in}}%
\pgfpathlineto{\pgfqpoint{3.651705in}{2.461228in}}%
\pgfpathmoveto{\pgfqpoint{3.651705in}{2.458279in}}%
\pgfpathlineto{\pgfqpoint{3.651705in}{2.458279in}}%
\pgfpathlineto{\pgfqpoint{3.651705in}{2.461228in}}%
\pgfpathlineto{\pgfqpoint{3.656246in}{2.461228in}}%
\pgfpathlineto{\pgfqpoint{3.656246in}{2.458279in}}%
\pgfpathmoveto{\pgfqpoint{3.651705in}{2.461228in}}%
\pgfpathlineto{\pgfqpoint{3.651705in}{2.461228in}}%
\pgfpathlineto{\pgfqpoint{3.651705in}{2.464177in}}%
\pgfpathlineto{\pgfqpoint{3.656246in}{2.464177in}}%
\pgfpathlineto{\pgfqpoint{3.656246in}{2.461228in}}%
\pgfpathmoveto{\pgfqpoint{3.647165in}{2.464177in}}%
\pgfpathlineto{\pgfqpoint{3.647165in}{2.464177in}}%
\pgfpathlineto{\pgfqpoint{3.647165in}{2.467127in}}%
\pgfpathlineto{\pgfqpoint{3.651705in}{2.467127in}}%
\pgfpathlineto{\pgfqpoint{3.651705in}{2.464177in}}%
\pgfpathmoveto{\pgfqpoint{3.647165in}{2.467127in}}%
\pgfpathlineto{\pgfqpoint{3.647165in}{2.467127in}}%
\pgfpathlineto{\pgfqpoint{3.647165in}{2.470076in}}%
\pgfpathlineto{\pgfqpoint{3.651705in}{2.470076in}}%
\pgfpathlineto{\pgfqpoint{3.651705in}{2.467127in}}%
\pgfpathmoveto{\pgfqpoint{3.651705in}{2.464177in}}%
\pgfpathlineto{\pgfqpoint{3.651705in}{2.464177in}}%
\pgfpathlineto{\pgfqpoint{3.651705in}{2.467127in}}%
\pgfpathlineto{\pgfqpoint{3.656246in}{2.467127in}}%
\pgfpathlineto{\pgfqpoint{3.656246in}{2.464177in}}%
\pgfpathmoveto{\pgfqpoint{3.651705in}{2.467127in}}%
\pgfpathlineto{\pgfqpoint{3.651705in}{2.467127in}}%
\pgfpathlineto{\pgfqpoint{3.651705in}{2.470076in}}%
\pgfpathlineto{\pgfqpoint{3.656246in}{2.470076in}}%
\pgfpathlineto{\pgfqpoint{3.656246in}{2.467127in}}%
\pgfpathmoveto{\pgfqpoint{3.638083in}{2.470076in}}%
\pgfpathlineto{\pgfqpoint{3.638083in}{2.470076in}}%
\pgfpathlineto{\pgfqpoint{3.638083in}{2.473025in}}%
\pgfpathlineto{\pgfqpoint{3.642624in}{2.473025in}}%
\pgfpathlineto{\pgfqpoint{3.642624in}{2.470076in}}%
\pgfpathmoveto{\pgfqpoint{3.638083in}{2.473025in}}%
\pgfpathlineto{\pgfqpoint{3.638083in}{2.473025in}}%
\pgfpathlineto{\pgfqpoint{3.638083in}{2.475974in}}%
\pgfpathlineto{\pgfqpoint{3.642624in}{2.475974in}}%
\pgfpathlineto{\pgfqpoint{3.642624in}{2.473025in}}%
\pgfpathmoveto{\pgfqpoint{3.642624in}{2.470076in}}%
\pgfpathlineto{\pgfqpoint{3.642624in}{2.470076in}}%
\pgfpathlineto{\pgfqpoint{3.642624in}{2.473025in}}%
\pgfpathlineto{\pgfqpoint{3.647165in}{2.473025in}}%
\pgfpathlineto{\pgfqpoint{3.647165in}{2.470076in}}%
\pgfpathmoveto{\pgfqpoint{3.642624in}{2.473025in}}%
\pgfpathlineto{\pgfqpoint{3.642624in}{2.473025in}}%
\pgfpathlineto{\pgfqpoint{3.642624in}{2.475974in}}%
\pgfpathlineto{\pgfqpoint{3.647165in}{2.475974in}}%
\pgfpathlineto{\pgfqpoint{3.647165in}{2.473025in}}%
\pgfpathmoveto{\pgfqpoint{3.638083in}{2.475974in}}%
\pgfpathlineto{\pgfqpoint{3.638083in}{2.475974in}}%
\pgfpathlineto{\pgfqpoint{3.638083in}{2.478923in}}%
\pgfpathlineto{\pgfqpoint{3.642624in}{2.478923in}}%
\pgfpathlineto{\pgfqpoint{3.642624in}{2.475974in}}%
\pgfpathmoveto{\pgfqpoint{3.638083in}{2.478923in}}%
\pgfpathlineto{\pgfqpoint{3.638083in}{2.478923in}}%
\pgfpathlineto{\pgfqpoint{3.638083in}{2.481873in}}%
\pgfpathlineto{\pgfqpoint{3.642624in}{2.481873in}}%
\pgfpathlineto{\pgfqpoint{3.642624in}{2.478923in}}%
\pgfpathmoveto{\pgfqpoint{3.642624in}{2.475974in}}%
\pgfpathlineto{\pgfqpoint{3.642624in}{2.475974in}}%
\pgfpathlineto{\pgfqpoint{3.642624in}{2.478923in}}%
\pgfpathlineto{\pgfqpoint{3.647165in}{2.478923in}}%
\pgfpathlineto{\pgfqpoint{3.647165in}{2.475974in}}%
\pgfpathmoveto{\pgfqpoint{3.642624in}{2.478923in}}%
\pgfpathlineto{\pgfqpoint{3.642624in}{2.478923in}}%
\pgfpathlineto{\pgfqpoint{3.642624in}{2.481873in}}%
\pgfpathlineto{\pgfqpoint{3.647165in}{2.481873in}}%
\pgfpathlineto{\pgfqpoint{3.647165in}{2.478923in}}%
\pgfpathmoveto{\pgfqpoint{3.647165in}{2.470076in}}%
\pgfpathlineto{\pgfqpoint{3.647165in}{2.470076in}}%
\pgfpathlineto{\pgfqpoint{3.647165in}{2.473025in}}%
\pgfpathlineto{\pgfqpoint{3.651705in}{2.473025in}}%
\pgfpathlineto{\pgfqpoint{3.651705in}{2.470076in}}%
\pgfpathmoveto{\pgfqpoint{3.647165in}{2.473025in}}%
\pgfpathlineto{\pgfqpoint{3.647165in}{2.473025in}}%
\pgfpathlineto{\pgfqpoint{3.647165in}{2.475974in}}%
\pgfpathlineto{\pgfqpoint{3.651705in}{2.475974in}}%
\pgfpathlineto{\pgfqpoint{3.651705in}{2.473025in}}%
\pgfpathmoveto{\pgfqpoint{3.574512in}{2.552657in}}%
\pgfpathlineto{\pgfqpoint{3.574512in}{2.552657in}}%
\pgfpathlineto{\pgfqpoint{3.574512in}{2.555606in}}%
\pgfpathlineto{\pgfqpoint{3.579053in}{2.555606in}}%
\pgfpathlineto{\pgfqpoint{3.579053in}{2.552657in}}%
\pgfpathmoveto{\pgfqpoint{3.574512in}{2.555606in}}%
\pgfpathlineto{\pgfqpoint{3.574512in}{2.555606in}}%
\pgfpathlineto{\pgfqpoint{3.574512in}{2.558556in}}%
\pgfpathlineto{\pgfqpoint{3.579053in}{2.558556in}}%
\pgfpathlineto{\pgfqpoint{3.579053in}{2.555606in}}%
\pgfpathmoveto{\pgfqpoint{3.579053in}{2.552657in}}%
\pgfpathlineto{\pgfqpoint{3.579053in}{2.552657in}}%
\pgfpathlineto{\pgfqpoint{3.579053in}{2.555606in}}%
\pgfpathlineto{\pgfqpoint{3.583593in}{2.555606in}}%
\pgfpathlineto{\pgfqpoint{3.583593in}{2.552657in}}%
\pgfpathmoveto{\pgfqpoint{3.579053in}{2.555606in}}%
\pgfpathlineto{\pgfqpoint{3.579053in}{2.555606in}}%
\pgfpathlineto{\pgfqpoint{3.579053in}{2.558556in}}%
\pgfpathlineto{\pgfqpoint{3.583593in}{2.558556in}}%
\pgfpathlineto{\pgfqpoint{3.583593in}{2.555606in}}%
\pgfpathmoveto{\pgfqpoint{3.574512in}{2.558556in}}%
\pgfpathlineto{\pgfqpoint{3.574512in}{2.558556in}}%
\pgfpathlineto{\pgfqpoint{3.574512in}{2.561505in}}%
\pgfpathlineto{\pgfqpoint{3.579053in}{2.561505in}}%
\pgfpathlineto{\pgfqpoint{3.579053in}{2.558556in}}%
\pgfpathmoveto{\pgfqpoint{3.574512in}{2.561505in}}%
\pgfpathlineto{\pgfqpoint{3.574512in}{2.561505in}}%
\pgfpathlineto{\pgfqpoint{3.574512in}{2.564454in}}%
\pgfpathlineto{\pgfqpoint{3.579053in}{2.564454in}}%
\pgfpathlineto{\pgfqpoint{3.579053in}{2.561505in}}%
\pgfpathmoveto{\pgfqpoint{3.579053in}{2.558556in}}%
\pgfpathlineto{\pgfqpoint{3.579053in}{2.558556in}}%
\pgfpathlineto{\pgfqpoint{3.579053in}{2.561505in}}%
\pgfpathlineto{\pgfqpoint{3.583593in}{2.561505in}}%
\pgfpathlineto{\pgfqpoint{3.583593in}{2.558556in}}%
\pgfpathmoveto{\pgfqpoint{3.579053in}{2.561505in}}%
\pgfpathlineto{\pgfqpoint{3.579053in}{2.561505in}}%
\pgfpathlineto{\pgfqpoint{3.579053in}{2.564454in}}%
\pgfpathlineto{\pgfqpoint{3.583593in}{2.564454in}}%
\pgfpathlineto{\pgfqpoint{3.583593in}{2.561505in}}%
\pgfpathmoveto{\pgfqpoint{3.565430in}{2.564454in}}%
\pgfpathlineto{\pgfqpoint{3.565430in}{2.564454in}}%
\pgfpathlineto{\pgfqpoint{3.565430in}{2.567404in}}%
\pgfpathlineto{\pgfqpoint{3.569971in}{2.567404in}}%
\pgfpathlineto{\pgfqpoint{3.569971in}{2.564454in}}%
\pgfpathmoveto{\pgfqpoint{3.565430in}{2.567404in}}%
\pgfpathlineto{\pgfqpoint{3.565430in}{2.567404in}}%
\pgfpathlineto{\pgfqpoint{3.565430in}{2.570353in}}%
\pgfpathlineto{\pgfqpoint{3.569971in}{2.570353in}}%
\pgfpathlineto{\pgfqpoint{3.569971in}{2.567404in}}%
\pgfpathmoveto{\pgfqpoint{3.569971in}{2.564454in}}%
\pgfpathlineto{\pgfqpoint{3.569971in}{2.564454in}}%
\pgfpathlineto{\pgfqpoint{3.569971in}{2.567404in}}%
\pgfpathlineto{\pgfqpoint{3.574512in}{2.567404in}}%
\pgfpathlineto{\pgfqpoint{3.574512in}{2.564454in}}%
\pgfpathmoveto{\pgfqpoint{3.569971in}{2.567404in}}%
\pgfpathlineto{\pgfqpoint{3.569971in}{2.567404in}}%
\pgfpathlineto{\pgfqpoint{3.569971in}{2.570353in}}%
\pgfpathlineto{\pgfqpoint{3.574512in}{2.570353in}}%
\pgfpathlineto{\pgfqpoint{3.574512in}{2.567404in}}%
\pgfpathmoveto{\pgfqpoint{3.565430in}{2.570353in}}%
\pgfpathlineto{\pgfqpoint{3.565430in}{2.570353in}}%
\pgfpathlineto{\pgfqpoint{3.565430in}{2.573303in}}%
\pgfpathlineto{\pgfqpoint{3.569971in}{2.573303in}}%
\pgfpathlineto{\pgfqpoint{3.569971in}{2.570353in}}%
\pgfpathmoveto{\pgfqpoint{3.565430in}{2.573303in}}%
\pgfpathlineto{\pgfqpoint{3.565430in}{2.573303in}}%
\pgfpathlineto{\pgfqpoint{3.565430in}{2.576252in}}%
\pgfpathlineto{\pgfqpoint{3.569971in}{2.576252in}}%
\pgfpathlineto{\pgfqpoint{3.569971in}{2.573303in}}%
\pgfpathmoveto{\pgfqpoint{3.569971in}{2.570353in}}%
\pgfpathlineto{\pgfqpoint{3.569971in}{2.570353in}}%
\pgfpathlineto{\pgfqpoint{3.569971in}{2.573303in}}%
\pgfpathlineto{\pgfqpoint{3.574512in}{2.573303in}}%
\pgfpathlineto{\pgfqpoint{3.574512in}{2.570353in}}%
\pgfpathmoveto{\pgfqpoint{3.569971in}{2.573303in}}%
\pgfpathlineto{\pgfqpoint{3.569971in}{2.573303in}}%
\pgfpathlineto{\pgfqpoint{3.569971in}{2.576252in}}%
\pgfpathlineto{\pgfqpoint{3.574512in}{2.576252in}}%
\pgfpathlineto{\pgfqpoint{3.574512in}{2.573303in}}%
\pgfpathmoveto{\pgfqpoint{3.574512in}{2.564454in}}%
\pgfpathlineto{\pgfqpoint{3.574512in}{2.564454in}}%
\pgfpathlineto{\pgfqpoint{3.574512in}{2.567404in}}%
\pgfpathlineto{\pgfqpoint{3.579053in}{2.567404in}}%
\pgfpathlineto{\pgfqpoint{3.579053in}{2.564454in}}%
\pgfpathmoveto{\pgfqpoint{3.574512in}{2.567404in}}%
\pgfpathlineto{\pgfqpoint{3.574512in}{2.567404in}}%
\pgfpathlineto{\pgfqpoint{3.574512in}{2.570353in}}%
\pgfpathlineto{\pgfqpoint{3.579053in}{2.570353in}}%
\pgfpathlineto{\pgfqpoint{3.579053in}{2.567404in}}%
\pgfpathmoveto{\pgfqpoint{3.610838in}{2.505467in}}%
\pgfpathlineto{\pgfqpoint{3.610838in}{2.505467in}}%
\pgfpathlineto{\pgfqpoint{3.610838in}{2.508417in}}%
\pgfpathlineto{\pgfqpoint{3.615379in}{2.508417in}}%
\pgfpathlineto{\pgfqpoint{3.615379in}{2.505467in}}%
\pgfpathmoveto{\pgfqpoint{3.610838in}{2.508417in}}%
\pgfpathlineto{\pgfqpoint{3.610838in}{2.508417in}}%
\pgfpathlineto{\pgfqpoint{3.610838in}{2.511366in}}%
\pgfpathlineto{\pgfqpoint{3.615379in}{2.511366in}}%
\pgfpathlineto{\pgfqpoint{3.615379in}{2.508417in}}%
\pgfpathmoveto{\pgfqpoint{3.615379in}{2.505467in}}%
\pgfpathlineto{\pgfqpoint{3.615379in}{2.505467in}}%
\pgfpathlineto{\pgfqpoint{3.615379in}{2.508417in}}%
\pgfpathlineto{\pgfqpoint{3.619920in}{2.508417in}}%
\pgfpathlineto{\pgfqpoint{3.619920in}{2.505467in}}%
\pgfpathmoveto{\pgfqpoint{3.615379in}{2.508417in}}%
\pgfpathlineto{\pgfqpoint{3.615379in}{2.508417in}}%
\pgfpathlineto{\pgfqpoint{3.615379in}{2.511366in}}%
\pgfpathlineto{\pgfqpoint{3.619920in}{2.511366in}}%
\pgfpathlineto{\pgfqpoint{3.619920in}{2.508417in}}%
\pgfpathmoveto{\pgfqpoint{3.610838in}{2.511366in}}%
\pgfpathlineto{\pgfqpoint{3.610838in}{2.511366in}}%
\pgfpathlineto{\pgfqpoint{3.610838in}{2.514316in}}%
\pgfpathlineto{\pgfqpoint{3.615379in}{2.514316in}}%
\pgfpathlineto{\pgfqpoint{3.615379in}{2.511366in}}%
\pgfpathmoveto{\pgfqpoint{3.610838in}{2.514316in}}%
\pgfpathlineto{\pgfqpoint{3.610838in}{2.514316in}}%
\pgfpathlineto{\pgfqpoint{3.610838in}{2.517265in}}%
\pgfpathlineto{\pgfqpoint{3.615379in}{2.517265in}}%
\pgfpathlineto{\pgfqpoint{3.615379in}{2.514316in}}%
\pgfpathmoveto{\pgfqpoint{3.615379in}{2.511366in}}%
\pgfpathlineto{\pgfqpoint{3.615379in}{2.511366in}}%
\pgfpathlineto{\pgfqpoint{3.615379in}{2.514316in}}%
\pgfpathlineto{\pgfqpoint{3.619920in}{2.514316in}}%
\pgfpathlineto{\pgfqpoint{3.619920in}{2.511366in}}%
\pgfpathmoveto{\pgfqpoint{3.615379in}{2.514316in}}%
\pgfpathlineto{\pgfqpoint{3.615379in}{2.514316in}}%
\pgfpathlineto{\pgfqpoint{3.615379in}{2.517265in}}%
\pgfpathlineto{\pgfqpoint{3.619920in}{2.517265in}}%
\pgfpathlineto{\pgfqpoint{3.619920in}{2.514316in}}%
\pgfpathmoveto{\pgfqpoint{3.601757in}{2.517265in}}%
\pgfpathlineto{\pgfqpoint{3.601757in}{2.517265in}}%
\pgfpathlineto{\pgfqpoint{3.601757in}{2.520214in}}%
\pgfpathlineto{\pgfqpoint{3.606297in}{2.520214in}}%
\pgfpathlineto{\pgfqpoint{3.606297in}{2.517265in}}%
\pgfpathmoveto{\pgfqpoint{3.601757in}{2.520214in}}%
\pgfpathlineto{\pgfqpoint{3.601757in}{2.520214in}}%
\pgfpathlineto{\pgfqpoint{3.601757in}{2.523164in}}%
\pgfpathlineto{\pgfqpoint{3.606297in}{2.523164in}}%
\pgfpathlineto{\pgfqpoint{3.606297in}{2.520214in}}%
\pgfpathmoveto{\pgfqpoint{3.606297in}{2.517265in}}%
\pgfpathlineto{\pgfqpoint{3.606297in}{2.517265in}}%
\pgfpathlineto{\pgfqpoint{3.606297in}{2.520214in}}%
\pgfpathlineto{\pgfqpoint{3.610838in}{2.520214in}}%
\pgfpathlineto{\pgfqpoint{3.610838in}{2.517265in}}%
\pgfpathmoveto{\pgfqpoint{3.606297in}{2.520214in}}%
\pgfpathlineto{\pgfqpoint{3.606297in}{2.520214in}}%
\pgfpathlineto{\pgfqpoint{3.606297in}{2.523164in}}%
\pgfpathlineto{\pgfqpoint{3.610838in}{2.523164in}}%
\pgfpathlineto{\pgfqpoint{3.610838in}{2.520214in}}%
\pgfpathmoveto{\pgfqpoint{3.601757in}{2.523164in}}%
\pgfpathlineto{\pgfqpoint{3.601757in}{2.523164in}}%
\pgfpathlineto{\pgfqpoint{3.601757in}{2.526113in}}%
\pgfpathlineto{\pgfqpoint{3.606297in}{2.526113in}}%
\pgfpathlineto{\pgfqpoint{3.606297in}{2.523164in}}%
\pgfpathmoveto{\pgfqpoint{3.601757in}{2.526113in}}%
\pgfpathlineto{\pgfqpoint{3.601757in}{2.526113in}}%
\pgfpathlineto{\pgfqpoint{3.601757in}{2.529062in}}%
\pgfpathlineto{\pgfqpoint{3.606297in}{2.529062in}}%
\pgfpathlineto{\pgfqpoint{3.606297in}{2.526113in}}%
\pgfpathmoveto{\pgfqpoint{3.606297in}{2.523164in}}%
\pgfpathlineto{\pgfqpoint{3.606297in}{2.523164in}}%
\pgfpathlineto{\pgfqpoint{3.606297in}{2.526113in}}%
\pgfpathlineto{\pgfqpoint{3.610838in}{2.526113in}}%
\pgfpathlineto{\pgfqpoint{3.610838in}{2.523164in}}%
\pgfpathmoveto{\pgfqpoint{3.606297in}{2.526113in}}%
\pgfpathlineto{\pgfqpoint{3.606297in}{2.526113in}}%
\pgfpathlineto{\pgfqpoint{3.606297in}{2.529062in}}%
\pgfpathlineto{\pgfqpoint{3.610838in}{2.529062in}}%
\pgfpathlineto{\pgfqpoint{3.610838in}{2.526113in}}%
\pgfpathmoveto{\pgfqpoint{3.610838in}{2.517265in}}%
\pgfpathlineto{\pgfqpoint{3.610838in}{2.517265in}}%
\pgfpathlineto{\pgfqpoint{3.610838in}{2.520214in}}%
\pgfpathlineto{\pgfqpoint{3.615379in}{2.520214in}}%
\pgfpathlineto{\pgfqpoint{3.615379in}{2.517265in}}%
\pgfpathmoveto{\pgfqpoint{3.610838in}{2.520214in}}%
\pgfpathlineto{\pgfqpoint{3.610838in}{2.520214in}}%
\pgfpathlineto{\pgfqpoint{3.610838in}{2.523164in}}%
\pgfpathlineto{\pgfqpoint{3.615379in}{2.523164in}}%
\pgfpathlineto{\pgfqpoint{3.615379in}{2.520214in}}%
\pgfpathmoveto{\pgfqpoint{3.629001in}{2.481873in}}%
\pgfpathlineto{\pgfqpoint{3.629001in}{2.481873in}}%
\pgfpathlineto{\pgfqpoint{3.629001in}{2.484822in}}%
\pgfpathlineto{\pgfqpoint{3.633542in}{2.484822in}}%
\pgfpathlineto{\pgfqpoint{3.633542in}{2.481873in}}%
\pgfpathmoveto{\pgfqpoint{3.629001in}{2.484822in}}%
\pgfpathlineto{\pgfqpoint{3.629001in}{2.484822in}}%
\pgfpathlineto{\pgfqpoint{3.629001in}{2.487771in}}%
\pgfpathlineto{\pgfqpoint{3.633542in}{2.487771in}}%
\pgfpathlineto{\pgfqpoint{3.633542in}{2.484822in}}%
\pgfpathmoveto{\pgfqpoint{3.633542in}{2.481873in}}%
\pgfpathlineto{\pgfqpoint{3.633542in}{2.481873in}}%
\pgfpathlineto{\pgfqpoint{3.633542in}{2.484822in}}%
\pgfpathlineto{\pgfqpoint{3.638083in}{2.484822in}}%
\pgfpathlineto{\pgfqpoint{3.638083in}{2.481873in}}%
\pgfpathmoveto{\pgfqpoint{3.633542in}{2.484822in}}%
\pgfpathlineto{\pgfqpoint{3.633542in}{2.484822in}}%
\pgfpathlineto{\pgfqpoint{3.633542in}{2.487771in}}%
\pgfpathlineto{\pgfqpoint{3.638083in}{2.487771in}}%
\pgfpathlineto{\pgfqpoint{3.638083in}{2.484822in}}%
\pgfpathmoveto{\pgfqpoint{3.629001in}{2.487771in}}%
\pgfpathlineto{\pgfqpoint{3.629001in}{2.487771in}}%
\pgfpathlineto{\pgfqpoint{3.629001in}{2.490721in}}%
\pgfpathlineto{\pgfqpoint{3.633542in}{2.490721in}}%
\pgfpathlineto{\pgfqpoint{3.633542in}{2.487771in}}%
\pgfpathmoveto{\pgfqpoint{3.629001in}{2.490721in}}%
\pgfpathlineto{\pgfqpoint{3.629001in}{2.490721in}}%
\pgfpathlineto{\pgfqpoint{3.629001in}{2.493670in}}%
\pgfpathlineto{\pgfqpoint{3.633542in}{2.493670in}}%
\pgfpathlineto{\pgfqpoint{3.633542in}{2.490721in}}%
\pgfpathmoveto{\pgfqpoint{3.633542in}{2.487771in}}%
\pgfpathlineto{\pgfqpoint{3.633542in}{2.487771in}}%
\pgfpathlineto{\pgfqpoint{3.633542in}{2.490721in}}%
\pgfpathlineto{\pgfqpoint{3.638083in}{2.490721in}}%
\pgfpathlineto{\pgfqpoint{3.638083in}{2.487771in}}%
\pgfpathmoveto{\pgfqpoint{3.633542in}{2.490721in}}%
\pgfpathlineto{\pgfqpoint{3.633542in}{2.490721in}}%
\pgfpathlineto{\pgfqpoint{3.633542in}{2.493670in}}%
\pgfpathlineto{\pgfqpoint{3.638083in}{2.493670in}}%
\pgfpathlineto{\pgfqpoint{3.638083in}{2.490721in}}%
\pgfpathmoveto{\pgfqpoint{3.619920in}{2.493670in}}%
\pgfpathlineto{\pgfqpoint{3.619920in}{2.493670in}}%
\pgfpathlineto{\pgfqpoint{3.619920in}{2.496619in}}%
\pgfpathlineto{\pgfqpoint{3.624461in}{2.496619in}}%
\pgfpathlineto{\pgfqpoint{3.624461in}{2.493670in}}%
\pgfpathmoveto{\pgfqpoint{3.619920in}{2.496619in}}%
\pgfpathlineto{\pgfqpoint{3.619920in}{2.496619in}}%
\pgfpathlineto{\pgfqpoint{3.619920in}{2.499569in}}%
\pgfpathlineto{\pgfqpoint{3.624461in}{2.499569in}}%
\pgfpathlineto{\pgfqpoint{3.624461in}{2.496619in}}%
\pgfpathmoveto{\pgfqpoint{3.624461in}{2.493670in}}%
\pgfpathlineto{\pgfqpoint{3.624461in}{2.493670in}}%
\pgfpathlineto{\pgfqpoint{3.624461in}{2.496619in}}%
\pgfpathlineto{\pgfqpoint{3.629001in}{2.496619in}}%
\pgfpathlineto{\pgfqpoint{3.629001in}{2.493670in}}%
\pgfpathmoveto{\pgfqpoint{3.624461in}{2.496619in}}%
\pgfpathlineto{\pgfqpoint{3.624461in}{2.496619in}}%
\pgfpathlineto{\pgfqpoint{3.624461in}{2.499569in}}%
\pgfpathlineto{\pgfqpoint{3.629001in}{2.499569in}}%
\pgfpathlineto{\pgfqpoint{3.629001in}{2.496619in}}%
\pgfpathmoveto{\pgfqpoint{3.619920in}{2.499569in}}%
\pgfpathlineto{\pgfqpoint{3.619920in}{2.499569in}}%
\pgfpathlineto{\pgfqpoint{3.619920in}{2.502518in}}%
\pgfpathlineto{\pgfqpoint{3.624461in}{2.502518in}}%
\pgfpathlineto{\pgfqpoint{3.624461in}{2.499569in}}%
\pgfpathmoveto{\pgfqpoint{3.619920in}{2.502518in}}%
\pgfpathlineto{\pgfqpoint{3.619920in}{2.502518in}}%
\pgfpathlineto{\pgfqpoint{3.619920in}{2.505467in}}%
\pgfpathlineto{\pgfqpoint{3.624461in}{2.505467in}}%
\pgfpathlineto{\pgfqpoint{3.624461in}{2.502518in}}%
\pgfpathmoveto{\pgfqpoint{3.624461in}{2.499569in}}%
\pgfpathlineto{\pgfqpoint{3.624461in}{2.499569in}}%
\pgfpathlineto{\pgfqpoint{3.624461in}{2.502518in}}%
\pgfpathlineto{\pgfqpoint{3.629001in}{2.502518in}}%
\pgfpathlineto{\pgfqpoint{3.629001in}{2.499569in}}%
\pgfpathmoveto{\pgfqpoint{3.624461in}{2.502518in}}%
\pgfpathlineto{\pgfqpoint{3.624461in}{2.502518in}}%
\pgfpathlineto{\pgfqpoint{3.624461in}{2.505467in}}%
\pgfpathlineto{\pgfqpoint{3.629001in}{2.505467in}}%
\pgfpathlineto{\pgfqpoint{3.629001in}{2.502518in}}%
\pgfpathmoveto{\pgfqpoint{3.629001in}{2.493670in}}%
\pgfpathlineto{\pgfqpoint{3.629001in}{2.493670in}}%
\pgfpathlineto{\pgfqpoint{3.629001in}{2.496619in}}%
\pgfpathlineto{\pgfqpoint{3.633542in}{2.496619in}}%
\pgfpathlineto{\pgfqpoint{3.633542in}{2.493670in}}%
\pgfpathmoveto{\pgfqpoint{3.629001in}{2.496619in}}%
\pgfpathlineto{\pgfqpoint{3.629001in}{2.496619in}}%
\pgfpathlineto{\pgfqpoint{3.629001in}{2.499569in}}%
\pgfpathlineto{\pgfqpoint{3.633542in}{2.499569in}}%
\pgfpathlineto{\pgfqpoint{3.633542in}{2.496619in}}%
\pgfpathmoveto{\pgfqpoint{3.638083in}{2.481873in}}%
\pgfpathlineto{\pgfqpoint{3.638083in}{2.481873in}}%
\pgfpathlineto{\pgfqpoint{3.638083in}{2.484822in}}%
\pgfpathlineto{\pgfqpoint{3.642624in}{2.484822in}}%
\pgfpathlineto{\pgfqpoint{3.642624in}{2.481873in}}%
\pgfpathmoveto{\pgfqpoint{3.638083in}{2.484822in}}%
\pgfpathlineto{\pgfqpoint{3.638083in}{2.484822in}}%
\pgfpathlineto{\pgfqpoint{3.638083in}{2.487771in}}%
\pgfpathlineto{\pgfqpoint{3.642624in}{2.487771in}}%
\pgfpathlineto{\pgfqpoint{3.642624in}{2.484822in}}%
\pgfpathmoveto{\pgfqpoint{3.619920in}{2.505467in}}%
\pgfpathlineto{\pgfqpoint{3.619920in}{2.505467in}}%
\pgfpathlineto{\pgfqpoint{3.619920in}{2.508417in}}%
\pgfpathlineto{\pgfqpoint{3.624461in}{2.508417in}}%
\pgfpathlineto{\pgfqpoint{3.624461in}{2.505467in}}%
\pgfpathmoveto{\pgfqpoint{3.619920in}{2.508417in}}%
\pgfpathlineto{\pgfqpoint{3.619920in}{2.508417in}}%
\pgfpathlineto{\pgfqpoint{3.619920in}{2.511366in}}%
\pgfpathlineto{\pgfqpoint{3.624461in}{2.511366in}}%
\pgfpathlineto{\pgfqpoint{3.624461in}{2.508417in}}%
\pgfpathmoveto{\pgfqpoint{3.592675in}{2.529062in}}%
\pgfpathlineto{\pgfqpoint{3.592675in}{2.529062in}}%
\pgfpathlineto{\pgfqpoint{3.592675in}{2.532012in}}%
\pgfpathlineto{\pgfqpoint{3.597216in}{2.532012in}}%
\pgfpathlineto{\pgfqpoint{3.597216in}{2.529062in}}%
\pgfpathmoveto{\pgfqpoint{3.592675in}{2.532012in}}%
\pgfpathlineto{\pgfqpoint{3.592675in}{2.532012in}}%
\pgfpathlineto{\pgfqpoint{3.592675in}{2.534961in}}%
\pgfpathlineto{\pgfqpoint{3.597216in}{2.534961in}}%
\pgfpathlineto{\pgfqpoint{3.597216in}{2.532012in}}%
\pgfpathmoveto{\pgfqpoint{3.597216in}{2.529062in}}%
\pgfpathlineto{\pgfqpoint{3.597216in}{2.529062in}}%
\pgfpathlineto{\pgfqpoint{3.597216in}{2.532012in}}%
\pgfpathlineto{\pgfqpoint{3.601757in}{2.532012in}}%
\pgfpathlineto{\pgfqpoint{3.601757in}{2.529062in}}%
\pgfpathmoveto{\pgfqpoint{3.597216in}{2.532012in}}%
\pgfpathlineto{\pgfqpoint{3.597216in}{2.532012in}}%
\pgfpathlineto{\pgfqpoint{3.597216in}{2.534961in}}%
\pgfpathlineto{\pgfqpoint{3.601757in}{2.534961in}}%
\pgfpathlineto{\pgfqpoint{3.601757in}{2.532012in}}%
\pgfpathmoveto{\pgfqpoint{3.592675in}{2.534961in}}%
\pgfpathlineto{\pgfqpoint{3.592675in}{2.534961in}}%
\pgfpathlineto{\pgfqpoint{3.592675in}{2.537910in}}%
\pgfpathlineto{\pgfqpoint{3.597216in}{2.537910in}}%
\pgfpathlineto{\pgfqpoint{3.597216in}{2.534961in}}%
\pgfpathmoveto{\pgfqpoint{3.592675in}{2.537910in}}%
\pgfpathlineto{\pgfqpoint{3.592675in}{2.537910in}}%
\pgfpathlineto{\pgfqpoint{3.592675in}{2.540860in}}%
\pgfpathlineto{\pgfqpoint{3.597216in}{2.540860in}}%
\pgfpathlineto{\pgfqpoint{3.597216in}{2.537910in}}%
\pgfpathmoveto{\pgfqpoint{3.597216in}{2.534961in}}%
\pgfpathlineto{\pgfqpoint{3.597216in}{2.534961in}}%
\pgfpathlineto{\pgfqpoint{3.597216in}{2.537910in}}%
\pgfpathlineto{\pgfqpoint{3.601757in}{2.537910in}}%
\pgfpathlineto{\pgfqpoint{3.601757in}{2.534961in}}%
\pgfpathmoveto{\pgfqpoint{3.597216in}{2.537910in}}%
\pgfpathlineto{\pgfqpoint{3.597216in}{2.537910in}}%
\pgfpathlineto{\pgfqpoint{3.597216in}{2.540860in}}%
\pgfpathlineto{\pgfqpoint{3.601757in}{2.540860in}}%
\pgfpathlineto{\pgfqpoint{3.601757in}{2.537910in}}%
\pgfpathmoveto{\pgfqpoint{3.583593in}{2.540860in}}%
\pgfpathlineto{\pgfqpoint{3.583593in}{2.540860in}}%
\pgfpathlineto{\pgfqpoint{3.583593in}{2.543809in}}%
\pgfpathlineto{\pgfqpoint{3.588134in}{2.543809in}}%
\pgfpathlineto{\pgfqpoint{3.588134in}{2.540860in}}%
\pgfpathmoveto{\pgfqpoint{3.583593in}{2.543809in}}%
\pgfpathlineto{\pgfqpoint{3.583593in}{2.543809in}}%
\pgfpathlineto{\pgfqpoint{3.583593in}{2.546758in}}%
\pgfpathlineto{\pgfqpoint{3.588134in}{2.546758in}}%
\pgfpathlineto{\pgfqpoint{3.588134in}{2.543809in}}%
\pgfpathmoveto{\pgfqpoint{3.588134in}{2.540860in}}%
\pgfpathlineto{\pgfqpoint{3.588134in}{2.540860in}}%
\pgfpathlineto{\pgfqpoint{3.588134in}{2.543809in}}%
\pgfpathlineto{\pgfqpoint{3.592675in}{2.543809in}}%
\pgfpathlineto{\pgfqpoint{3.592675in}{2.540860in}}%
\pgfpathmoveto{\pgfqpoint{3.588134in}{2.543809in}}%
\pgfpathlineto{\pgfqpoint{3.588134in}{2.543809in}}%
\pgfpathlineto{\pgfqpoint{3.588134in}{2.546758in}}%
\pgfpathlineto{\pgfqpoint{3.592675in}{2.546758in}}%
\pgfpathlineto{\pgfqpoint{3.592675in}{2.543809in}}%
\pgfpathmoveto{\pgfqpoint{3.583593in}{2.546758in}}%
\pgfpathlineto{\pgfqpoint{3.583593in}{2.546758in}}%
\pgfpathlineto{\pgfqpoint{3.583593in}{2.549708in}}%
\pgfpathlineto{\pgfqpoint{3.588134in}{2.549708in}}%
\pgfpathlineto{\pgfqpoint{3.588134in}{2.546758in}}%
\pgfpathmoveto{\pgfqpoint{3.583593in}{2.549708in}}%
\pgfpathlineto{\pgfqpoint{3.583593in}{2.549708in}}%
\pgfpathlineto{\pgfqpoint{3.583593in}{2.552657in}}%
\pgfpathlineto{\pgfqpoint{3.588134in}{2.552657in}}%
\pgfpathlineto{\pgfqpoint{3.588134in}{2.549708in}}%
\pgfpathmoveto{\pgfqpoint{3.588134in}{2.546758in}}%
\pgfpathlineto{\pgfqpoint{3.588134in}{2.546758in}}%
\pgfpathlineto{\pgfqpoint{3.588134in}{2.549708in}}%
\pgfpathlineto{\pgfqpoint{3.592675in}{2.549708in}}%
\pgfpathlineto{\pgfqpoint{3.592675in}{2.546758in}}%
\pgfpathmoveto{\pgfqpoint{3.588134in}{2.549708in}}%
\pgfpathlineto{\pgfqpoint{3.588134in}{2.549708in}}%
\pgfpathlineto{\pgfqpoint{3.588134in}{2.552657in}}%
\pgfpathlineto{\pgfqpoint{3.592675in}{2.552657in}}%
\pgfpathlineto{\pgfqpoint{3.592675in}{2.549708in}}%
\pgfpathmoveto{\pgfqpoint{3.592675in}{2.540860in}}%
\pgfpathlineto{\pgfqpoint{3.592675in}{2.540860in}}%
\pgfpathlineto{\pgfqpoint{3.592675in}{2.543809in}}%
\pgfpathlineto{\pgfqpoint{3.597216in}{2.543809in}}%
\pgfpathlineto{\pgfqpoint{3.597216in}{2.540860in}}%
\pgfpathmoveto{\pgfqpoint{3.592675in}{2.543809in}}%
\pgfpathlineto{\pgfqpoint{3.592675in}{2.543809in}}%
\pgfpathlineto{\pgfqpoint{3.592675in}{2.546758in}}%
\pgfpathlineto{\pgfqpoint{3.597216in}{2.546758in}}%
\pgfpathlineto{\pgfqpoint{3.597216in}{2.543809in}}%
\pgfpathmoveto{\pgfqpoint{3.601757in}{2.529062in}}%
\pgfpathlineto{\pgfqpoint{3.601757in}{2.529062in}}%
\pgfpathlineto{\pgfqpoint{3.601757in}{2.532012in}}%
\pgfpathlineto{\pgfqpoint{3.606297in}{2.532012in}}%
\pgfpathlineto{\pgfqpoint{3.606297in}{2.529062in}}%
\pgfpathmoveto{\pgfqpoint{3.601757in}{2.532012in}}%
\pgfpathlineto{\pgfqpoint{3.601757in}{2.532012in}}%
\pgfpathlineto{\pgfqpoint{3.601757in}{2.534961in}}%
\pgfpathlineto{\pgfqpoint{3.606297in}{2.534961in}}%
\pgfpathlineto{\pgfqpoint{3.606297in}{2.532012in}}%
\pgfpathmoveto{\pgfqpoint{3.583593in}{2.552657in}}%
\pgfpathlineto{\pgfqpoint{3.583593in}{2.552657in}}%
\pgfpathlineto{\pgfqpoint{3.583593in}{2.555606in}}%
\pgfpathlineto{\pgfqpoint{3.588134in}{2.555606in}}%
\pgfpathlineto{\pgfqpoint{3.588134in}{2.552657in}}%
\pgfpathmoveto{\pgfqpoint{3.583593in}{2.555606in}}%
\pgfpathlineto{\pgfqpoint{3.583593in}{2.555606in}}%
\pgfpathlineto{\pgfqpoint{3.583593in}{2.558556in}}%
\pgfpathlineto{\pgfqpoint{3.588134in}{2.558556in}}%
\pgfpathlineto{\pgfqpoint{3.588134in}{2.555606in}}%
\pgfpathmoveto{\pgfqpoint{3.538185in}{2.599845in}}%
\pgfpathlineto{\pgfqpoint{3.538185in}{2.599845in}}%
\pgfpathlineto{\pgfqpoint{3.538185in}{2.602794in}}%
\pgfpathlineto{\pgfqpoint{3.542726in}{2.602794in}}%
\pgfpathlineto{\pgfqpoint{3.542726in}{2.599845in}}%
\pgfpathmoveto{\pgfqpoint{3.538185in}{2.602794in}}%
\pgfpathlineto{\pgfqpoint{3.538185in}{2.602794in}}%
\pgfpathlineto{\pgfqpoint{3.538185in}{2.605743in}}%
\pgfpathlineto{\pgfqpoint{3.542726in}{2.605743in}}%
\pgfpathlineto{\pgfqpoint{3.542726in}{2.602794in}}%
\pgfpathmoveto{\pgfqpoint{3.542726in}{2.599845in}}%
\pgfpathlineto{\pgfqpoint{3.542726in}{2.599845in}}%
\pgfpathlineto{\pgfqpoint{3.542726in}{2.602794in}}%
\pgfpathlineto{\pgfqpoint{3.547267in}{2.602794in}}%
\pgfpathlineto{\pgfqpoint{3.547267in}{2.599845in}}%
\pgfpathmoveto{\pgfqpoint{3.542726in}{2.602794in}}%
\pgfpathlineto{\pgfqpoint{3.542726in}{2.602794in}}%
\pgfpathlineto{\pgfqpoint{3.542726in}{2.605743in}}%
\pgfpathlineto{\pgfqpoint{3.547267in}{2.605743in}}%
\pgfpathlineto{\pgfqpoint{3.547267in}{2.602794in}}%
\pgfpathmoveto{\pgfqpoint{3.538185in}{2.605743in}}%
\pgfpathlineto{\pgfqpoint{3.538185in}{2.605743in}}%
\pgfpathlineto{\pgfqpoint{3.538185in}{2.608692in}}%
\pgfpathlineto{\pgfqpoint{3.542726in}{2.608692in}}%
\pgfpathlineto{\pgfqpoint{3.542726in}{2.605743in}}%
\pgfpathmoveto{\pgfqpoint{3.538185in}{2.608692in}}%
\pgfpathlineto{\pgfqpoint{3.538185in}{2.608692in}}%
\pgfpathlineto{\pgfqpoint{3.538185in}{2.611642in}}%
\pgfpathlineto{\pgfqpoint{3.542726in}{2.611642in}}%
\pgfpathlineto{\pgfqpoint{3.542726in}{2.608692in}}%
\pgfpathmoveto{\pgfqpoint{3.542726in}{2.605743in}}%
\pgfpathlineto{\pgfqpoint{3.542726in}{2.605743in}}%
\pgfpathlineto{\pgfqpoint{3.542726in}{2.608692in}}%
\pgfpathlineto{\pgfqpoint{3.547267in}{2.608692in}}%
\pgfpathlineto{\pgfqpoint{3.547267in}{2.605743in}}%
\pgfpathmoveto{\pgfqpoint{3.542726in}{2.608692in}}%
\pgfpathlineto{\pgfqpoint{3.542726in}{2.608692in}}%
\pgfpathlineto{\pgfqpoint{3.542726in}{2.611642in}}%
\pgfpathlineto{\pgfqpoint{3.547267in}{2.611642in}}%
\pgfpathlineto{\pgfqpoint{3.547267in}{2.608692in}}%
\pgfpathmoveto{\pgfqpoint{3.529104in}{2.611642in}}%
\pgfpathlineto{\pgfqpoint{3.529104in}{2.611642in}}%
\pgfpathlineto{\pgfqpoint{3.529104in}{2.614591in}}%
\pgfpathlineto{\pgfqpoint{3.533645in}{2.614591in}}%
\pgfpathlineto{\pgfqpoint{3.533645in}{2.611642in}}%
\pgfpathmoveto{\pgfqpoint{3.529104in}{2.614591in}}%
\pgfpathlineto{\pgfqpoint{3.529104in}{2.614591in}}%
\pgfpathlineto{\pgfqpoint{3.529104in}{2.617540in}}%
\pgfpathlineto{\pgfqpoint{3.533645in}{2.617540in}}%
\pgfpathlineto{\pgfqpoint{3.533645in}{2.614591in}}%
\pgfpathmoveto{\pgfqpoint{3.533645in}{2.611642in}}%
\pgfpathlineto{\pgfqpoint{3.533645in}{2.611642in}}%
\pgfpathlineto{\pgfqpoint{3.533645in}{2.614591in}}%
\pgfpathlineto{\pgfqpoint{3.538185in}{2.614591in}}%
\pgfpathlineto{\pgfqpoint{3.538185in}{2.611642in}}%
\pgfpathmoveto{\pgfqpoint{3.533645in}{2.614591in}}%
\pgfpathlineto{\pgfqpoint{3.533645in}{2.614591in}}%
\pgfpathlineto{\pgfqpoint{3.533645in}{2.617540in}}%
\pgfpathlineto{\pgfqpoint{3.538185in}{2.617540in}}%
\pgfpathlineto{\pgfqpoint{3.538185in}{2.614591in}}%
\pgfpathmoveto{\pgfqpoint{3.529104in}{2.617540in}}%
\pgfpathlineto{\pgfqpoint{3.529104in}{2.617540in}}%
\pgfpathlineto{\pgfqpoint{3.529104in}{2.620489in}}%
\pgfpathlineto{\pgfqpoint{3.533645in}{2.620489in}}%
\pgfpathlineto{\pgfqpoint{3.533645in}{2.617540in}}%
\pgfpathmoveto{\pgfqpoint{3.529104in}{2.620489in}}%
\pgfpathlineto{\pgfqpoint{3.529104in}{2.620489in}}%
\pgfpathlineto{\pgfqpoint{3.529104in}{2.623438in}}%
\pgfpathlineto{\pgfqpoint{3.533645in}{2.623438in}}%
\pgfpathlineto{\pgfqpoint{3.533645in}{2.620489in}}%
\pgfpathmoveto{\pgfqpoint{3.533645in}{2.617540in}}%
\pgfpathlineto{\pgfqpoint{3.533645in}{2.617540in}}%
\pgfpathlineto{\pgfqpoint{3.533645in}{2.620489in}}%
\pgfpathlineto{\pgfqpoint{3.538185in}{2.620489in}}%
\pgfpathlineto{\pgfqpoint{3.538185in}{2.617540in}}%
\pgfpathmoveto{\pgfqpoint{3.533645in}{2.620489in}}%
\pgfpathlineto{\pgfqpoint{3.533645in}{2.620489in}}%
\pgfpathlineto{\pgfqpoint{3.533645in}{2.623438in}}%
\pgfpathlineto{\pgfqpoint{3.538185in}{2.623438in}}%
\pgfpathlineto{\pgfqpoint{3.538185in}{2.620489in}}%
\pgfpathmoveto{\pgfqpoint{3.538185in}{2.611642in}}%
\pgfpathlineto{\pgfqpoint{3.538185in}{2.611642in}}%
\pgfpathlineto{\pgfqpoint{3.538185in}{2.614591in}}%
\pgfpathlineto{\pgfqpoint{3.542726in}{2.614591in}}%
\pgfpathlineto{\pgfqpoint{3.542726in}{2.611642in}}%
\pgfpathmoveto{\pgfqpoint{3.538185in}{2.614591in}}%
\pgfpathlineto{\pgfqpoint{3.538185in}{2.614591in}}%
\pgfpathlineto{\pgfqpoint{3.538185in}{2.617540in}}%
\pgfpathlineto{\pgfqpoint{3.542726in}{2.617540in}}%
\pgfpathlineto{\pgfqpoint{3.542726in}{2.614591in}}%
\pgfpathmoveto{\pgfqpoint{3.556349in}{2.576252in}}%
\pgfpathlineto{\pgfqpoint{3.556349in}{2.576252in}}%
\pgfpathlineto{\pgfqpoint{3.556349in}{2.579201in}}%
\pgfpathlineto{\pgfqpoint{3.560889in}{2.579201in}}%
\pgfpathlineto{\pgfqpoint{3.560889in}{2.576252in}}%
\pgfpathmoveto{\pgfqpoint{3.556349in}{2.579201in}}%
\pgfpathlineto{\pgfqpoint{3.556349in}{2.579201in}}%
\pgfpathlineto{\pgfqpoint{3.556349in}{2.582150in}}%
\pgfpathlineto{\pgfqpoint{3.560889in}{2.582150in}}%
\pgfpathlineto{\pgfqpoint{3.560889in}{2.579201in}}%
\pgfpathmoveto{\pgfqpoint{3.560889in}{2.576252in}}%
\pgfpathlineto{\pgfqpoint{3.560889in}{2.576252in}}%
\pgfpathlineto{\pgfqpoint{3.560889in}{2.579201in}}%
\pgfpathlineto{\pgfqpoint{3.565430in}{2.579201in}}%
\pgfpathlineto{\pgfqpoint{3.565430in}{2.576252in}}%
\pgfpathmoveto{\pgfqpoint{3.560889in}{2.579201in}}%
\pgfpathlineto{\pgfqpoint{3.560889in}{2.579201in}}%
\pgfpathlineto{\pgfqpoint{3.560889in}{2.582150in}}%
\pgfpathlineto{\pgfqpoint{3.565430in}{2.582150in}}%
\pgfpathlineto{\pgfqpoint{3.565430in}{2.579201in}}%
\pgfpathmoveto{\pgfqpoint{3.556349in}{2.582150in}}%
\pgfpathlineto{\pgfqpoint{3.556349in}{2.582150in}}%
\pgfpathlineto{\pgfqpoint{3.556349in}{2.585099in}}%
\pgfpathlineto{\pgfqpoint{3.560889in}{2.585099in}}%
\pgfpathlineto{\pgfqpoint{3.560889in}{2.582150in}}%
\pgfpathmoveto{\pgfqpoint{3.556349in}{2.585099in}}%
\pgfpathlineto{\pgfqpoint{3.556349in}{2.585099in}}%
\pgfpathlineto{\pgfqpoint{3.556349in}{2.588048in}}%
\pgfpathlineto{\pgfqpoint{3.560889in}{2.588048in}}%
\pgfpathlineto{\pgfqpoint{3.560889in}{2.585099in}}%
\pgfpathmoveto{\pgfqpoint{3.560889in}{2.582150in}}%
\pgfpathlineto{\pgfqpoint{3.560889in}{2.582150in}}%
\pgfpathlineto{\pgfqpoint{3.560889in}{2.585099in}}%
\pgfpathlineto{\pgfqpoint{3.565430in}{2.585099in}}%
\pgfpathlineto{\pgfqpoint{3.565430in}{2.582150in}}%
\pgfpathmoveto{\pgfqpoint{3.560889in}{2.585099in}}%
\pgfpathlineto{\pgfqpoint{3.560889in}{2.585099in}}%
\pgfpathlineto{\pgfqpoint{3.560889in}{2.588048in}}%
\pgfpathlineto{\pgfqpoint{3.565430in}{2.588048in}}%
\pgfpathlineto{\pgfqpoint{3.565430in}{2.585099in}}%
\pgfpathmoveto{\pgfqpoint{3.547267in}{2.588048in}}%
\pgfpathlineto{\pgfqpoint{3.547267in}{2.588048in}}%
\pgfpathlineto{\pgfqpoint{3.547267in}{2.590998in}}%
\pgfpathlineto{\pgfqpoint{3.551808in}{2.590998in}}%
\pgfpathlineto{\pgfqpoint{3.551808in}{2.588048in}}%
\pgfpathmoveto{\pgfqpoint{3.547267in}{2.590998in}}%
\pgfpathlineto{\pgfqpoint{3.547267in}{2.590998in}}%
\pgfpathlineto{\pgfqpoint{3.547267in}{2.593947in}}%
\pgfpathlineto{\pgfqpoint{3.551808in}{2.593947in}}%
\pgfpathlineto{\pgfqpoint{3.551808in}{2.590998in}}%
\pgfpathmoveto{\pgfqpoint{3.551808in}{2.588048in}}%
\pgfpathlineto{\pgfqpoint{3.551808in}{2.588048in}}%
\pgfpathlineto{\pgfqpoint{3.551808in}{2.590998in}}%
\pgfpathlineto{\pgfqpoint{3.556349in}{2.590998in}}%
\pgfpathlineto{\pgfqpoint{3.556349in}{2.588048in}}%
\pgfpathmoveto{\pgfqpoint{3.551808in}{2.590998in}}%
\pgfpathlineto{\pgfqpoint{3.551808in}{2.590998in}}%
\pgfpathlineto{\pgfqpoint{3.551808in}{2.593947in}}%
\pgfpathlineto{\pgfqpoint{3.556349in}{2.593947in}}%
\pgfpathlineto{\pgfqpoint{3.556349in}{2.590998in}}%
\pgfpathmoveto{\pgfqpoint{3.547267in}{2.593947in}}%
\pgfpathlineto{\pgfqpoint{3.547267in}{2.593947in}}%
\pgfpathlineto{\pgfqpoint{3.547267in}{2.596896in}}%
\pgfpathlineto{\pgfqpoint{3.551808in}{2.596896in}}%
\pgfpathlineto{\pgfqpoint{3.551808in}{2.593947in}}%
\pgfpathmoveto{\pgfqpoint{3.547267in}{2.596896in}}%
\pgfpathlineto{\pgfqpoint{3.547267in}{2.596896in}}%
\pgfpathlineto{\pgfqpoint{3.547267in}{2.599845in}}%
\pgfpathlineto{\pgfqpoint{3.551808in}{2.599845in}}%
\pgfpathlineto{\pgfqpoint{3.551808in}{2.596896in}}%
\pgfpathmoveto{\pgfqpoint{3.551808in}{2.593947in}}%
\pgfpathlineto{\pgfqpoint{3.551808in}{2.593947in}}%
\pgfpathlineto{\pgfqpoint{3.551808in}{2.596896in}}%
\pgfpathlineto{\pgfqpoint{3.556349in}{2.596896in}}%
\pgfpathlineto{\pgfqpoint{3.556349in}{2.593947in}}%
\pgfpathmoveto{\pgfqpoint{3.551808in}{2.596896in}}%
\pgfpathlineto{\pgfqpoint{3.551808in}{2.596896in}}%
\pgfpathlineto{\pgfqpoint{3.551808in}{2.599845in}}%
\pgfpathlineto{\pgfqpoint{3.556349in}{2.599845in}}%
\pgfpathlineto{\pgfqpoint{3.556349in}{2.596896in}}%
\pgfpathmoveto{\pgfqpoint{3.556349in}{2.588048in}}%
\pgfpathlineto{\pgfqpoint{3.556349in}{2.588048in}}%
\pgfpathlineto{\pgfqpoint{3.556349in}{2.590998in}}%
\pgfpathlineto{\pgfqpoint{3.560889in}{2.590998in}}%
\pgfpathlineto{\pgfqpoint{3.560889in}{2.588048in}}%
\pgfpathmoveto{\pgfqpoint{3.556349in}{2.590998in}}%
\pgfpathlineto{\pgfqpoint{3.556349in}{2.590998in}}%
\pgfpathlineto{\pgfqpoint{3.556349in}{2.593947in}}%
\pgfpathlineto{\pgfqpoint{3.560889in}{2.593947in}}%
\pgfpathlineto{\pgfqpoint{3.560889in}{2.590998in}}%
\pgfpathmoveto{\pgfqpoint{3.565430in}{2.576252in}}%
\pgfpathlineto{\pgfqpoint{3.565430in}{2.576252in}}%
\pgfpathlineto{\pgfqpoint{3.565430in}{2.579201in}}%
\pgfpathlineto{\pgfqpoint{3.569971in}{2.579201in}}%
\pgfpathlineto{\pgfqpoint{3.569971in}{2.576252in}}%
\pgfpathmoveto{\pgfqpoint{3.565430in}{2.579201in}}%
\pgfpathlineto{\pgfqpoint{3.565430in}{2.579201in}}%
\pgfpathlineto{\pgfqpoint{3.565430in}{2.582150in}}%
\pgfpathlineto{\pgfqpoint{3.569971in}{2.582150in}}%
\pgfpathlineto{\pgfqpoint{3.569971in}{2.579201in}}%
\pgfpathmoveto{\pgfqpoint{3.547267in}{2.599845in}}%
\pgfpathlineto{\pgfqpoint{3.547267in}{2.599845in}}%
\pgfpathlineto{\pgfqpoint{3.547267in}{2.602794in}}%
\pgfpathlineto{\pgfqpoint{3.551808in}{2.602794in}}%
\pgfpathlineto{\pgfqpoint{3.551808in}{2.599845in}}%
\pgfpathmoveto{\pgfqpoint{3.547267in}{2.602794in}}%
\pgfpathlineto{\pgfqpoint{3.547267in}{2.602794in}}%
\pgfpathlineto{\pgfqpoint{3.547267in}{2.605743in}}%
\pgfpathlineto{\pgfqpoint{3.551808in}{2.605743in}}%
\pgfpathlineto{\pgfqpoint{3.551808in}{2.602794in}}%
\pgfpathmoveto{\pgfqpoint{3.520022in}{2.623438in}}%
\pgfpathlineto{\pgfqpoint{3.520022in}{2.623438in}}%
\pgfpathlineto{\pgfqpoint{3.520022in}{2.626387in}}%
\pgfpathlineto{\pgfqpoint{3.524563in}{2.626387in}}%
\pgfpathlineto{\pgfqpoint{3.524563in}{2.623438in}}%
\pgfpathmoveto{\pgfqpoint{3.520022in}{2.626387in}}%
\pgfpathlineto{\pgfqpoint{3.520022in}{2.626387in}}%
\pgfpathlineto{\pgfqpoint{3.520022in}{2.629336in}}%
\pgfpathlineto{\pgfqpoint{3.524563in}{2.629336in}}%
\pgfpathlineto{\pgfqpoint{3.524563in}{2.626387in}}%
\pgfpathmoveto{\pgfqpoint{3.524563in}{2.623438in}}%
\pgfpathlineto{\pgfqpoint{3.524563in}{2.623438in}}%
\pgfpathlineto{\pgfqpoint{3.524563in}{2.626387in}}%
\pgfpathlineto{\pgfqpoint{3.529104in}{2.626387in}}%
\pgfpathlineto{\pgfqpoint{3.529104in}{2.623438in}}%
\pgfpathmoveto{\pgfqpoint{3.524563in}{2.626387in}}%
\pgfpathlineto{\pgfqpoint{3.524563in}{2.626387in}}%
\pgfpathlineto{\pgfqpoint{3.524563in}{2.629336in}}%
\pgfpathlineto{\pgfqpoint{3.529104in}{2.629336in}}%
\pgfpathlineto{\pgfqpoint{3.529104in}{2.626387in}}%
\pgfpathmoveto{\pgfqpoint{3.520022in}{2.629336in}}%
\pgfpathlineto{\pgfqpoint{3.520022in}{2.629336in}}%
\pgfpathlineto{\pgfqpoint{3.520022in}{2.632286in}}%
\pgfpathlineto{\pgfqpoint{3.524563in}{2.632286in}}%
\pgfpathlineto{\pgfqpoint{3.524563in}{2.629336in}}%
\pgfpathmoveto{\pgfqpoint{3.520022in}{2.632286in}}%
\pgfpathlineto{\pgfqpoint{3.520022in}{2.632286in}}%
\pgfpathlineto{\pgfqpoint{3.520022in}{2.635235in}}%
\pgfpathlineto{\pgfqpoint{3.524563in}{2.635235in}}%
\pgfpathlineto{\pgfqpoint{3.524563in}{2.632286in}}%
\pgfpathmoveto{\pgfqpoint{3.524563in}{2.629336in}}%
\pgfpathlineto{\pgfqpoint{3.524563in}{2.629336in}}%
\pgfpathlineto{\pgfqpoint{3.524563in}{2.632286in}}%
\pgfpathlineto{\pgfqpoint{3.529104in}{2.632286in}}%
\pgfpathlineto{\pgfqpoint{3.529104in}{2.629336in}}%
\pgfpathmoveto{\pgfqpoint{3.524563in}{2.632286in}}%
\pgfpathlineto{\pgfqpoint{3.524563in}{2.632286in}}%
\pgfpathlineto{\pgfqpoint{3.524563in}{2.635235in}}%
\pgfpathlineto{\pgfqpoint{3.529104in}{2.635235in}}%
\pgfpathlineto{\pgfqpoint{3.529104in}{2.632286in}}%
\pgfpathmoveto{\pgfqpoint{3.510941in}{2.635235in}}%
\pgfpathlineto{\pgfqpoint{3.510941in}{2.635235in}}%
\pgfpathlineto{\pgfqpoint{3.510941in}{2.638184in}}%
\pgfpathlineto{\pgfqpoint{3.515481in}{2.638184in}}%
\pgfpathlineto{\pgfqpoint{3.515481in}{2.635235in}}%
\pgfpathmoveto{\pgfqpoint{3.510941in}{2.638184in}}%
\pgfpathlineto{\pgfqpoint{3.510941in}{2.638184in}}%
\pgfpathlineto{\pgfqpoint{3.510941in}{2.641133in}}%
\pgfpathlineto{\pgfqpoint{3.515481in}{2.641133in}}%
\pgfpathlineto{\pgfqpoint{3.515481in}{2.638184in}}%
\pgfpathmoveto{\pgfqpoint{3.515481in}{2.635235in}}%
\pgfpathlineto{\pgfqpoint{3.515481in}{2.635235in}}%
\pgfpathlineto{\pgfqpoint{3.515481in}{2.638184in}}%
\pgfpathlineto{\pgfqpoint{3.520022in}{2.638184in}}%
\pgfpathlineto{\pgfqpoint{3.520022in}{2.635235in}}%
\pgfpathmoveto{\pgfqpoint{3.515481in}{2.638184in}}%
\pgfpathlineto{\pgfqpoint{3.515481in}{2.638184in}}%
\pgfpathlineto{\pgfqpoint{3.515481in}{2.641133in}}%
\pgfpathlineto{\pgfqpoint{3.520022in}{2.641133in}}%
\pgfpathlineto{\pgfqpoint{3.520022in}{2.638184in}}%
\pgfpathmoveto{\pgfqpoint{3.510941in}{2.641133in}}%
\pgfpathlineto{\pgfqpoint{3.510941in}{2.641133in}}%
\pgfpathlineto{\pgfqpoint{3.510941in}{2.644082in}}%
\pgfpathlineto{\pgfqpoint{3.515481in}{2.644082in}}%
\pgfpathlineto{\pgfqpoint{3.515481in}{2.641133in}}%
\pgfpathmoveto{\pgfqpoint{3.510941in}{2.644082in}}%
\pgfpathlineto{\pgfqpoint{3.510941in}{2.644082in}}%
\pgfpathlineto{\pgfqpoint{3.510941in}{2.647031in}}%
\pgfpathlineto{\pgfqpoint{3.515481in}{2.647031in}}%
\pgfpathlineto{\pgfqpoint{3.515481in}{2.644082in}}%
\pgfpathmoveto{\pgfqpoint{3.515481in}{2.641133in}}%
\pgfpathlineto{\pgfqpoint{3.515481in}{2.641133in}}%
\pgfpathlineto{\pgfqpoint{3.515481in}{2.644082in}}%
\pgfpathlineto{\pgfqpoint{3.520022in}{2.644082in}}%
\pgfpathlineto{\pgfqpoint{3.520022in}{2.641133in}}%
\pgfpathmoveto{\pgfqpoint{3.515481in}{2.644082in}}%
\pgfpathlineto{\pgfqpoint{3.515481in}{2.644082in}}%
\pgfpathlineto{\pgfqpoint{3.515481in}{2.647031in}}%
\pgfpathlineto{\pgfqpoint{3.520022in}{2.647031in}}%
\pgfpathlineto{\pgfqpoint{3.520022in}{2.644082in}}%
\pgfpathmoveto{\pgfqpoint{3.520022in}{2.635235in}}%
\pgfpathlineto{\pgfqpoint{3.520022in}{2.635235in}}%
\pgfpathlineto{\pgfqpoint{3.520022in}{2.638184in}}%
\pgfpathlineto{\pgfqpoint{3.524563in}{2.638184in}}%
\pgfpathlineto{\pgfqpoint{3.524563in}{2.635235in}}%
\pgfpathmoveto{\pgfqpoint{3.520022in}{2.638184in}}%
\pgfpathlineto{\pgfqpoint{3.520022in}{2.638184in}}%
\pgfpathlineto{\pgfqpoint{3.520022in}{2.641133in}}%
\pgfpathlineto{\pgfqpoint{3.524563in}{2.641133in}}%
\pgfpathlineto{\pgfqpoint{3.524563in}{2.638184in}}%
\pgfpathmoveto{\pgfqpoint{3.529104in}{2.623438in}}%
\pgfpathlineto{\pgfqpoint{3.529104in}{2.623438in}}%
\pgfpathlineto{\pgfqpoint{3.529104in}{2.626387in}}%
\pgfpathlineto{\pgfqpoint{3.533645in}{2.626387in}}%
\pgfpathlineto{\pgfqpoint{3.533645in}{2.623438in}}%
\pgfpathmoveto{\pgfqpoint{3.529104in}{2.626387in}}%
\pgfpathlineto{\pgfqpoint{3.529104in}{2.626387in}}%
\pgfpathlineto{\pgfqpoint{3.529104in}{2.629336in}}%
\pgfpathlineto{\pgfqpoint{3.533645in}{2.629336in}}%
\pgfpathlineto{\pgfqpoint{3.533645in}{2.626387in}}%
\pgfpathmoveto{\pgfqpoint{3.510941in}{2.647031in}}%
\pgfpathlineto{\pgfqpoint{3.510941in}{2.647031in}}%
\pgfpathlineto{\pgfqpoint{3.510941in}{2.649980in}}%
\pgfpathlineto{\pgfqpoint{3.515481in}{2.649980in}}%
\pgfpathlineto{\pgfqpoint{3.515481in}{2.647031in}}%
\pgfpathmoveto{\pgfqpoint{3.510941in}{2.649980in}}%
\pgfpathlineto{\pgfqpoint{3.510941in}{2.649980in}}%
\pgfpathlineto{\pgfqpoint{3.510941in}{2.652930in}}%
\pgfpathlineto{\pgfqpoint{3.515481in}{2.652930in}}%
\pgfpathlineto{\pgfqpoint{3.515481in}{2.649980in}}%
\pgfpathmoveto{\pgfqpoint{3.656246in}{2.387497in}}%
\pgfpathlineto{\pgfqpoint{3.656246in}{2.387497in}}%
\pgfpathlineto{\pgfqpoint{3.656246in}{2.390447in}}%
\pgfpathlineto{\pgfqpoint{3.660787in}{2.390447in}}%
\pgfpathlineto{\pgfqpoint{3.660787in}{2.387497in}}%
\pgfpathmoveto{\pgfqpoint{3.656246in}{2.390447in}}%
\pgfpathlineto{\pgfqpoint{3.656246in}{2.390447in}}%
\pgfpathlineto{\pgfqpoint{3.656246in}{2.393396in}}%
\pgfpathlineto{\pgfqpoint{3.660787in}{2.393396in}}%
\pgfpathlineto{\pgfqpoint{3.660787in}{2.390447in}}%
\pgfpathmoveto{\pgfqpoint{3.660787in}{2.390447in}}%
\pgfpathlineto{\pgfqpoint{3.660787in}{2.390447in}}%
\pgfpathlineto{\pgfqpoint{3.660787in}{2.393396in}}%
\pgfpathlineto{\pgfqpoint{3.665329in}{2.393396in}}%
\pgfpathlineto{\pgfqpoint{3.665329in}{2.390447in}}%
\pgfpathmoveto{\pgfqpoint{3.656246in}{2.393396in}}%
\pgfpathlineto{\pgfqpoint{3.656246in}{2.393396in}}%
\pgfpathlineto{\pgfqpoint{3.656246in}{2.396345in}}%
\pgfpathlineto{\pgfqpoint{3.660787in}{2.396345in}}%
\pgfpathlineto{\pgfqpoint{3.660787in}{2.393396in}}%
\pgfpathmoveto{\pgfqpoint{3.656246in}{2.396345in}}%
\pgfpathlineto{\pgfqpoint{3.656246in}{2.396345in}}%
\pgfpathlineto{\pgfqpoint{3.656246in}{2.399294in}}%
\pgfpathlineto{\pgfqpoint{3.660787in}{2.399294in}}%
\pgfpathlineto{\pgfqpoint{3.660787in}{2.396345in}}%
\pgfpathmoveto{\pgfqpoint{3.660787in}{2.393396in}}%
\pgfpathlineto{\pgfqpoint{3.660787in}{2.393396in}}%
\pgfpathlineto{\pgfqpoint{3.660787in}{2.396345in}}%
\pgfpathlineto{\pgfqpoint{3.665329in}{2.396345in}}%
\pgfpathlineto{\pgfqpoint{3.665329in}{2.393396in}}%
\pgfpathmoveto{\pgfqpoint{3.660787in}{2.396345in}}%
\pgfpathlineto{\pgfqpoint{3.660787in}{2.396345in}}%
\pgfpathlineto{\pgfqpoint{3.660787in}{2.399294in}}%
\pgfpathlineto{\pgfqpoint{3.665329in}{2.399294in}}%
\pgfpathlineto{\pgfqpoint{3.665329in}{2.396345in}}%
\pgfpathmoveto{\pgfqpoint{3.665329in}{2.393396in}}%
\pgfpathlineto{\pgfqpoint{3.665329in}{2.393396in}}%
\pgfpathlineto{\pgfqpoint{3.665329in}{2.396345in}}%
\pgfpathlineto{\pgfqpoint{3.669870in}{2.396345in}}%
\pgfpathlineto{\pgfqpoint{3.669870in}{2.393396in}}%
\pgfpathmoveto{\pgfqpoint{3.665329in}{2.396345in}}%
\pgfpathlineto{\pgfqpoint{3.665329in}{2.396345in}}%
\pgfpathlineto{\pgfqpoint{3.665329in}{2.399294in}}%
\pgfpathlineto{\pgfqpoint{3.669870in}{2.399294in}}%
\pgfpathlineto{\pgfqpoint{3.669870in}{2.396345in}}%
\pgfpathmoveto{\pgfqpoint{3.669870in}{2.396345in}}%
\pgfpathlineto{\pgfqpoint{3.669870in}{2.396345in}}%
\pgfpathlineto{\pgfqpoint{3.669870in}{2.399294in}}%
\pgfpathlineto{\pgfqpoint{3.674411in}{2.399294in}}%
\pgfpathlineto{\pgfqpoint{3.674411in}{2.396345in}}%
\pgfpathmoveto{\pgfqpoint{3.665329in}{2.399294in}}%
\pgfpathlineto{\pgfqpoint{3.665329in}{2.399294in}}%
\pgfpathlineto{\pgfqpoint{3.665329in}{2.402244in}}%
\pgfpathlineto{\pgfqpoint{3.669870in}{2.402244in}}%
\pgfpathlineto{\pgfqpoint{3.669870in}{2.399294in}}%
\pgfpathmoveto{\pgfqpoint{3.665329in}{2.402244in}}%
\pgfpathlineto{\pgfqpoint{3.665329in}{2.402244in}}%
\pgfpathlineto{\pgfqpoint{3.665329in}{2.405193in}}%
\pgfpathlineto{\pgfqpoint{3.669870in}{2.405193in}}%
\pgfpathlineto{\pgfqpoint{3.669870in}{2.402244in}}%
\pgfpathmoveto{\pgfqpoint{3.669870in}{2.399294in}}%
\pgfpathlineto{\pgfqpoint{3.669870in}{2.399294in}}%
\pgfpathlineto{\pgfqpoint{3.669870in}{2.402244in}}%
\pgfpathlineto{\pgfqpoint{3.674411in}{2.402244in}}%
\pgfpathlineto{\pgfqpoint{3.674411in}{2.399294in}}%
\pgfpathmoveto{\pgfqpoint{3.669870in}{2.402244in}}%
\pgfpathlineto{\pgfqpoint{3.669870in}{2.402244in}}%
\pgfpathlineto{\pgfqpoint{3.669870in}{2.405193in}}%
\pgfpathlineto{\pgfqpoint{3.674411in}{2.405193in}}%
\pgfpathlineto{\pgfqpoint{3.674411in}{2.402244in}}%
\pgfpathmoveto{\pgfqpoint{3.674411in}{2.399294in}}%
\pgfpathlineto{\pgfqpoint{3.674411in}{2.399294in}}%
\pgfpathlineto{\pgfqpoint{3.674411in}{2.402244in}}%
\pgfpathlineto{\pgfqpoint{3.678952in}{2.402244in}}%
\pgfpathlineto{\pgfqpoint{3.678952in}{2.399294in}}%
\pgfpathmoveto{\pgfqpoint{3.674411in}{2.402244in}}%
\pgfpathlineto{\pgfqpoint{3.674411in}{2.402244in}}%
\pgfpathlineto{\pgfqpoint{3.674411in}{2.405193in}}%
\pgfpathlineto{\pgfqpoint{3.678952in}{2.405193in}}%
\pgfpathlineto{\pgfqpoint{3.678952in}{2.402244in}}%
\pgfpathmoveto{\pgfqpoint{3.678952in}{2.402244in}}%
\pgfpathlineto{\pgfqpoint{3.678952in}{2.402244in}}%
\pgfpathlineto{\pgfqpoint{3.678952in}{2.405193in}}%
\pgfpathlineto{\pgfqpoint{3.683494in}{2.405193in}}%
\pgfpathlineto{\pgfqpoint{3.683494in}{2.402244in}}%
\pgfpathmoveto{\pgfqpoint{3.674411in}{2.405193in}}%
\pgfpathlineto{\pgfqpoint{3.674411in}{2.405193in}}%
\pgfpathlineto{\pgfqpoint{3.674411in}{2.408142in}}%
\pgfpathlineto{\pgfqpoint{3.678952in}{2.408142in}}%
\pgfpathlineto{\pgfqpoint{3.678952in}{2.405193in}}%
\pgfpathmoveto{\pgfqpoint{3.674411in}{2.408142in}}%
\pgfpathlineto{\pgfqpoint{3.674411in}{2.408142in}}%
\pgfpathlineto{\pgfqpoint{3.674411in}{2.411091in}}%
\pgfpathlineto{\pgfqpoint{3.678952in}{2.411091in}}%
\pgfpathlineto{\pgfqpoint{3.678952in}{2.408142in}}%
\pgfpathmoveto{\pgfqpoint{3.678952in}{2.405193in}}%
\pgfpathlineto{\pgfqpoint{3.678952in}{2.405193in}}%
\pgfpathlineto{\pgfqpoint{3.678952in}{2.408142in}}%
\pgfpathlineto{\pgfqpoint{3.683494in}{2.408142in}}%
\pgfpathlineto{\pgfqpoint{3.683494in}{2.405193in}}%
\pgfpathmoveto{\pgfqpoint{3.678952in}{2.408142in}}%
\pgfpathlineto{\pgfqpoint{3.678952in}{2.408142in}}%
\pgfpathlineto{\pgfqpoint{3.678952in}{2.411091in}}%
\pgfpathlineto{\pgfqpoint{3.683494in}{2.411091in}}%
\pgfpathlineto{\pgfqpoint{3.683494in}{2.408142in}}%
\pgfpathmoveto{\pgfqpoint{3.683494in}{2.405193in}}%
\pgfpathlineto{\pgfqpoint{3.683494in}{2.405193in}}%
\pgfpathlineto{\pgfqpoint{3.683494in}{2.408142in}}%
\pgfpathlineto{\pgfqpoint{3.688035in}{2.408142in}}%
\pgfpathlineto{\pgfqpoint{3.688035in}{2.405193in}}%
\pgfpathmoveto{\pgfqpoint{3.683494in}{2.408142in}}%
\pgfpathlineto{\pgfqpoint{3.683494in}{2.408142in}}%
\pgfpathlineto{\pgfqpoint{3.683494in}{2.411091in}}%
\pgfpathlineto{\pgfqpoint{3.688035in}{2.411091in}}%
\pgfpathlineto{\pgfqpoint{3.688035in}{2.408142in}}%
\pgfpathmoveto{\pgfqpoint{3.688035in}{2.408142in}}%
\pgfpathlineto{\pgfqpoint{3.688035in}{2.408142in}}%
\pgfpathlineto{\pgfqpoint{3.688035in}{2.411091in}}%
\pgfpathlineto{\pgfqpoint{3.692576in}{2.411091in}}%
\pgfpathlineto{\pgfqpoint{3.692576in}{2.408142in}}%
\pgfpathmoveto{\pgfqpoint{3.683494in}{2.411091in}}%
\pgfpathlineto{\pgfqpoint{3.683494in}{2.411091in}}%
\pgfpathlineto{\pgfqpoint{3.683494in}{2.414040in}}%
\pgfpathlineto{\pgfqpoint{3.688035in}{2.414040in}}%
\pgfpathlineto{\pgfqpoint{3.688035in}{2.411091in}}%
\pgfpathmoveto{\pgfqpoint{3.683494in}{2.414040in}}%
\pgfpathlineto{\pgfqpoint{3.683494in}{2.414040in}}%
\pgfpathlineto{\pgfqpoint{3.683494in}{2.416990in}}%
\pgfpathlineto{\pgfqpoint{3.688035in}{2.416990in}}%
\pgfpathlineto{\pgfqpoint{3.688035in}{2.414040in}}%
\pgfpathmoveto{\pgfqpoint{3.688035in}{2.411091in}}%
\pgfpathlineto{\pgfqpoint{3.688035in}{2.411091in}}%
\pgfpathlineto{\pgfqpoint{3.688035in}{2.414040in}}%
\pgfpathlineto{\pgfqpoint{3.692576in}{2.414040in}}%
\pgfpathlineto{\pgfqpoint{3.692576in}{2.411091in}}%
\pgfpathmoveto{\pgfqpoint{3.688035in}{2.414040in}}%
\pgfpathlineto{\pgfqpoint{3.688035in}{2.414040in}}%
\pgfpathlineto{\pgfqpoint{3.688035in}{2.416990in}}%
\pgfpathlineto{\pgfqpoint{3.692576in}{2.416990in}}%
\pgfpathlineto{\pgfqpoint{3.692576in}{2.414040in}}%
\pgfpathmoveto{\pgfqpoint{3.683494in}{2.416990in}}%
\pgfpathlineto{\pgfqpoint{3.683494in}{2.416990in}}%
\pgfpathlineto{\pgfqpoint{3.683494in}{2.419939in}}%
\pgfpathlineto{\pgfqpoint{3.688035in}{2.419939in}}%
\pgfpathlineto{\pgfqpoint{3.688035in}{2.416990in}}%
\pgfpathmoveto{\pgfqpoint{3.683494in}{2.419939in}}%
\pgfpathlineto{\pgfqpoint{3.683494in}{2.419939in}}%
\pgfpathlineto{\pgfqpoint{3.683494in}{2.422888in}}%
\pgfpathlineto{\pgfqpoint{3.688035in}{2.422888in}}%
\pgfpathlineto{\pgfqpoint{3.688035in}{2.419939in}}%
\pgfpathmoveto{\pgfqpoint{3.688035in}{2.416990in}}%
\pgfpathlineto{\pgfqpoint{3.688035in}{2.416990in}}%
\pgfpathlineto{\pgfqpoint{3.688035in}{2.419939in}}%
\pgfpathlineto{\pgfqpoint{3.692576in}{2.419939in}}%
\pgfpathlineto{\pgfqpoint{3.692576in}{2.416990in}}%
\pgfpathmoveto{\pgfqpoint{3.688035in}{2.419939in}}%
\pgfpathlineto{\pgfqpoint{3.688035in}{2.419939in}}%
\pgfpathlineto{\pgfqpoint{3.688035in}{2.422888in}}%
\pgfpathlineto{\pgfqpoint{3.692576in}{2.422888in}}%
\pgfpathlineto{\pgfqpoint{3.692576in}{2.419939in}}%
\pgfpathmoveto{\pgfqpoint{3.674411in}{2.422888in}}%
\pgfpathlineto{\pgfqpoint{3.674411in}{2.422888in}}%
\pgfpathlineto{\pgfqpoint{3.674411in}{2.425837in}}%
\pgfpathlineto{\pgfqpoint{3.678952in}{2.425837in}}%
\pgfpathlineto{\pgfqpoint{3.678952in}{2.422888in}}%
\pgfpathmoveto{\pgfqpoint{3.674411in}{2.425837in}}%
\pgfpathlineto{\pgfqpoint{3.674411in}{2.425837in}}%
\pgfpathlineto{\pgfqpoint{3.674411in}{2.428787in}}%
\pgfpathlineto{\pgfqpoint{3.678952in}{2.428787in}}%
\pgfpathlineto{\pgfqpoint{3.678952in}{2.425837in}}%
\pgfpathmoveto{\pgfqpoint{3.678952in}{2.422888in}}%
\pgfpathlineto{\pgfqpoint{3.678952in}{2.422888in}}%
\pgfpathlineto{\pgfqpoint{3.678952in}{2.425837in}}%
\pgfpathlineto{\pgfqpoint{3.683494in}{2.425837in}}%
\pgfpathlineto{\pgfqpoint{3.683494in}{2.422888in}}%
\pgfpathmoveto{\pgfqpoint{3.678952in}{2.425837in}}%
\pgfpathlineto{\pgfqpoint{3.678952in}{2.425837in}}%
\pgfpathlineto{\pgfqpoint{3.678952in}{2.428787in}}%
\pgfpathlineto{\pgfqpoint{3.683494in}{2.428787in}}%
\pgfpathlineto{\pgfqpoint{3.683494in}{2.425837in}}%
\pgfpathmoveto{\pgfqpoint{3.674411in}{2.428787in}}%
\pgfpathlineto{\pgfqpoint{3.674411in}{2.428787in}}%
\pgfpathlineto{\pgfqpoint{3.674411in}{2.431736in}}%
\pgfpathlineto{\pgfqpoint{3.678952in}{2.431736in}}%
\pgfpathlineto{\pgfqpoint{3.678952in}{2.428787in}}%
\pgfpathmoveto{\pgfqpoint{3.674411in}{2.431736in}}%
\pgfpathlineto{\pgfqpoint{3.674411in}{2.431736in}}%
\pgfpathlineto{\pgfqpoint{3.674411in}{2.434685in}}%
\pgfpathlineto{\pgfqpoint{3.678952in}{2.434685in}}%
\pgfpathlineto{\pgfqpoint{3.678952in}{2.431736in}}%
\pgfpathmoveto{\pgfqpoint{3.678952in}{2.428787in}}%
\pgfpathlineto{\pgfqpoint{3.678952in}{2.428787in}}%
\pgfpathlineto{\pgfqpoint{3.678952in}{2.431736in}}%
\pgfpathlineto{\pgfqpoint{3.683494in}{2.431736in}}%
\pgfpathlineto{\pgfqpoint{3.683494in}{2.428787in}}%
\pgfpathmoveto{\pgfqpoint{3.678952in}{2.431736in}}%
\pgfpathlineto{\pgfqpoint{3.678952in}{2.431736in}}%
\pgfpathlineto{\pgfqpoint{3.678952in}{2.434685in}}%
\pgfpathlineto{\pgfqpoint{3.683494in}{2.434685in}}%
\pgfpathlineto{\pgfqpoint{3.683494in}{2.431736in}}%
\pgfpathmoveto{\pgfqpoint{3.683494in}{2.422888in}}%
\pgfpathlineto{\pgfqpoint{3.683494in}{2.422888in}}%
\pgfpathlineto{\pgfqpoint{3.683494in}{2.425837in}}%
\pgfpathlineto{\pgfqpoint{3.688035in}{2.425837in}}%
\pgfpathlineto{\pgfqpoint{3.688035in}{2.422888in}}%
\pgfpathmoveto{\pgfqpoint{3.683494in}{2.425837in}}%
\pgfpathlineto{\pgfqpoint{3.683494in}{2.425837in}}%
\pgfpathlineto{\pgfqpoint{3.683494in}{2.428787in}}%
\pgfpathlineto{\pgfqpoint{3.688035in}{2.428787in}}%
\pgfpathlineto{\pgfqpoint{3.688035in}{2.425837in}}%
\pgfpathmoveto{\pgfqpoint{3.692576in}{2.411091in}}%
\pgfpathlineto{\pgfqpoint{3.692576in}{2.411091in}}%
\pgfpathlineto{\pgfqpoint{3.692576in}{2.414040in}}%
\pgfpathlineto{\pgfqpoint{3.697117in}{2.414040in}}%
\pgfpathlineto{\pgfqpoint{3.697117in}{2.411091in}}%
\pgfpathmoveto{\pgfqpoint{3.692576in}{2.414040in}}%
\pgfpathlineto{\pgfqpoint{3.692576in}{2.414040in}}%
\pgfpathlineto{\pgfqpoint{3.692576in}{2.416990in}}%
\pgfpathlineto{\pgfqpoint{3.697117in}{2.416990in}}%
\pgfpathlineto{\pgfqpoint{3.697117in}{2.414040in}}%
\pgfpathmoveto{\pgfqpoint{3.665329in}{2.434685in}}%
\pgfpathlineto{\pgfqpoint{3.665329in}{2.434685in}}%
\pgfpathlineto{\pgfqpoint{3.665329in}{2.437634in}}%
\pgfpathlineto{\pgfqpoint{3.669870in}{2.437634in}}%
\pgfpathlineto{\pgfqpoint{3.669870in}{2.434685in}}%
\pgfpathmoveto{\pgfqpoint{3.665329in}{2.437634in}}%
\pgfpathlineto{\pgfqpoint{3.665329in}{2.437634in}}%
\pgfpathlineto{\pgfqpoint{3.665329in}{2.440583in}}%
\pgfpathlineto{\pgfqpoint{3.669870in}{2.440583in}}%
\pgfpathlineto{\pgfqpoint{3.669870in}{2.437634in}}%
\pgfpathmoveto{\pgfqpoint{3.669870in}{2.434685in}}%
\pgfpathlineto{\pgfqpoint{3.669870in}{2.434685in}}%
\pgfpathlineto{\pgfqpoint{3.669870in}{2.437634in}}%
\pgfpathlineto{\pgfqpoint{3.674411in}{2.437634in}}%
\pgfpathlineto{\pgfqpoint{3.674411in}{2.434685in}}%
\pgfpathmoveto{\pgfqpoint{3.669870in}{2.437634in}}%
\pgfpathlineto{\pgfqpoint{3.669870in}{2.437634in}}%
\pgfpathlineto{\pgfqpoint{3.669870in}{2.440583in}}%
\pgfpathlineto{\pgfqpoint{3.674411in}{2.440583in}}%
\pgfpathlineto{\pgfqpoint{3.674411in}{2.437634in}}%
\pgfpathmoveto{\pgfqpoint{3.665329in}{2.440583in}}%
\pgfpathlineto{\pgfqpoint{3.665329in}{2.440583in}}%
\pgfpathlineto{\pgfqpoint{3.665329in}{2.443533in}}%
\pgfpathlineto{\pgfqpoint{3.669870in}{2.443533in}}%
\pgfpathlineto{\pgfqpoint{3.669870in}{2.440583in}}%
\pgfpathmoveto{\pgfqpoint{3.665329in}{2.443533in}}%
\pgfpathlineto{\pgfqpoint{3.665329in}{2.443533in}}%
\pgfpathlineto{\pgfqpoint{3.665329in}{2.446482in}}%
\pgfpathlineto{\pgfqpoint{3.669870in}{2.446482in}}%
\pgfpathlineto{\pgfqpoint{3.669870in}{2.443533in}}%
\pgfpathmoveto{\pgfqpoint{3.669870in}{2.440583in}}%
\pgfpathlineto{\pgfqpoint{3.669870in}{2.440583in}}%
\pgfpathlineto{\pgfqpoint{3.669870in}{2.443533in}}%
\pgfpathlineto{\pgfqpoint{3.674411in}{2.443533in}}%
\pgfpathlineto{\pgfqpoint{3.674411in}{2.440583in}}%
\pgfpathmoveto{\pgfqpoint{3.669870in}{2.443533in}}%
\pgfpathlineto{\pgfqpoint{3.669870in}{2.443533in}}%
\pgfpathlineto{\pgfqpoint{3.669870in}{2.446482in}}%
\pgfpathlineto{\pgfqpoint{3.674411in}{2.446482in}}%
\pgfpathlineto{\pgfqpoint{3.674411in}{2.443533in}}%
\pgfpathmoveto{\pgfqpoint{3.656246in}{2.446482in}}%
\pgfpathlineto{\pgfqpoint{3.656246in}{2.446482in}}%
\pgfpathlineto{\pgfqpoint{3.656246in}{2.449431in}}%
\pgfpathlineto{\pgfqpoint{3.660787in}{2.449431in}}%
\pgfpathlineto{\pgfqpoint{3.660787in}{2.446482in}}%
\pgfpathmoveto{\pgfqpoint{3.656246in}{2.449431in}}%
\pgfpathlineto{\pgfqpoint{3.656246in}{2.449431in}}%
\pgfpathlineto{\pgfqpoint{3.656246in}{2.452380in}}%
\pgfpathlineto{\pgfqpoint{3.660787in}{2.452380in}}%
\pgfpathlineto{\pgfqpoint{3.660787in}{2.449431in}}%
\pgfpathmoveto{\pgfqpoint{3.660787in}{2.446482in}}%
\pgfpathlineto{\pgfqpoint{3.660787in}{2.446482in}}%
\pgfpathlineto{\pgfqpoint{3.660787in}{2.449431in}}%
\pgfpathlineto{\pgfqpoint{3.665329in}{2.449431in}}%
\pgfpathlineto{\pgfqpoint{3.665329in}{2.446482in}}%
\pgfpathmoveto{\pgfqpoint{3.660787in}{2.449431in}}%
\pgfpathlineto{\pgfqpoint{3.660787in}{2.449431in}}%
\pgfpathlineto{\pgfqpoint{3.660787in}{2.452380in}}%
\pgfpathlineto{\pgfqpoint{3.665329in}{2.452380in}}%
\pgfpathlineto{\pgfqpoint{3.665329in}{2.449431in}}%
\pgfpathmoveto{\pgfqpoint{3.656246in}{2.452380in}}%
\pgfpathlineto{\pgfqpoint{3.656246in}{2.452380in}}%
\pgfpathlineto{\pgfqpoint{3.656246in}{2.455330in}}%
\pgfpathlineto{\pgfqpoint{3.660787in}{2.455330in}}%
\pgfpathlineto{\pgfqpoint{3.660787in}{2.452380in}}%
\pgfpathmoveto{\pgfqpoint{3.656246in}{2.455330in}}%
\pgfpathlineto{\pgfqpoint{3.656246in}{2.455330in}}%
\pgfpathlineto{\pgfqpoint{3.656246in}{2.458279in}}%
\pgfpathlineto{\pgfqpoint{3.660787in}{2.458279in}}%
\pgfpathlineto{\pgfqpoint{3.660787in}{2.455330in}}%
\pgfpathmoveto{\pgfqpoint{3.660787in}{2.452380in}}%
\pgfpathlineto{\pgfqpoint{3.660787in}{2.452380in}}%
\pgfpathlineto{\pgfqpoint{3.660787in}{2.455330in}}%
\pgfpathlineto{\pgfqpoint{3.665329in}{2.455330in}}%
\pgfpathlineto{\pgfqpoint{3.665329in}{2.452380in}}%
\pgfpathmoveto{\pgfqpoint{3.660787in}{2.455330in}}%
\pgfpathlineto{\pgfqpoint{3.660787in}{2.455330in}}%
\pgfpathlineto{\pgfqpoint{3.660787in}{2.458279in}}%
\pgfpathlineto{\pgfqpoint{3.665329in}{2.458279in}}%
\pgfpathlineto{\pgfqpoint{3.665329in}{2.455330in}}%
\pgfpathmoveto{\pgfqpoint{3.665329in}{2.446482in}}%
\pgfpathlineto{\pgfqpoint{3.665329in}{2.446482in}}%
\pgfpathlineto{\pgfqpoint{3.665329in}{2.449431in}}%
\pgfpathlineto{\pgfqpoint{3.669870in}{2.449431in}}%
\pgfpathlineto{\pgfqpoint{3.669870in}{2.446482in}}%
\pgfpathmoveto{\pgfqpoint{3.665329in}{2.449431in}}%
\pgfpathlineto{\pgfqpoint{3.665329in}{2.449431in}}%
\pgfpathlineto{\pgfqpoint{3.665329in}{2.452380in}}%
\pgfpathlineto{\pgfqpoint{3.669870in}{2.452380in}}%
\pgfpathlineto{\pgfqpoint{3.669870in}{2.449431in}}%
\pgfpathmoveto{\pgfqpoint{3.674411in}{2.434685in}}%
\pgfpathlineto{\pgfqpoint{3.674411in}{2.434685in}}%
\pgfpathlineto{\pgfqpoint{3.674411in}{2.437634in}}%
\pgfpathlineto{\pgfqpoint{3.678952in}{2.437634in}}%
\pgfpathlineto{\pgfqpoint{3.678952in}{2.434685in}}%
\pgfpathmoveto{\pgfqpoint{3.674411in}{2.437634in}}%
\pgfpathlineto{\pgfqpoint{3.674411in}{2.437634in}}%
\pgfpathlineto{\pgfqpoint{3.674411in}{2.440583in}}%
\pgfpathlineto{\pgfqpoint{3.678952in}{2.440583in}}%
\pgfpathlineto{\pgfqpoint{3.678952in}{2.437634in}}%
\pgfpathmoveto{\pgfqpoint{3.656246in}{2.458279in}}%
\pgfpathlineto{\pgfqpoint{3.656246in}{2.458279in}}%
\pgfpathlineto{\pgfqpoint{3.656246in}{2.461228in}}%
\pgfpathlineto{\pgfqpoint{3.660787in}{2.461228in}}%
\pgfpathlineto{\pgfqpoint{3.660787in}{2.458279in}}%
\pgfpathmoveto{\pgfqpoint{3.656246in}{2.461228in}}%
\pgfpathlineto{\pgfqpoint{3.656246in}{2.461228in}}%
\pgfpathlineto{\pgfqpoint{3.656246in}{2.464177in}}%
\pgfpathlineto{\pgfqpoint{3.660787in}{2.464177in}}%
\pgfpathlineto{\pgfqpoint{3.660787in}{2.461228in}}%
\pgfpathclose%
\pgfusepath{fill}%
\end{pgfscope}%
\begin{pgfscope}%
\pgfpathrectangle{\pgfqpoint{0.750000in}{0.500000in}}{\pgfqpoint{4.650000in}{3.020000in}}%
\pgfusepath{clip}%
\pgfsetbuttcap%
\pgfsetmiterjoin%
\definecolor{currentfill}{rgb}{1.000000,0.000000,0.000000}%
\pgfsetfillcolor{currentfill}%
\pgfsetlinewidth{0.000000pt}%
\definecolor{currentstroke}{rgb}{0.000000,0.000000,0.000000}%
\pgfsetstrokecolor{currentstroke}%
\pgfsetstrokeopacity{0.000000}%
\pgfsetdash{}{0pt}%
\pgfpathmoveto{\pgfqpoint{2.925146in}{3.404982in}}%
\pgfpathlineto{\pgfqpoint{2.925146in}{3.407931in}}%
\pgfpathlineto{\pgfqpoint{2.929687in}{3.407931in}}%
\pgfpathlineto{\pgfqpoint{2.929687in}{3.404982in}}%
\pgfpathmoveto{\pgfqpoint{2.920605in}{3.410880in}}%
\pgfpathlineto{\pgfqpoint{2.920605in}{3.410880in}}%
\pgfpathlineto{\pgfqpoint{2.920605in}{3.413830in}}%
\pgfpathlineto{\pgfqpoint{2.925146in}{3.413830in}}%
\pgfpathlineto{\pgfqpoint{2.925146in}{3.410880in}}%
\pgfpathmoveto{\pgfqpoint{2.925146in}{3.407931in}}%
\pgfpathlineto{\pgfqpoint{2.925146in}{3.407931in}}%
\pgfpathlineto{\pgfqpoint{2.925146in}{3.410880in}}%
\pgfpathlineto{\pgfqpoint{2.929687in}{3.410880in}}%
\pgfpathlineto{\pgfqpoint{2.929687in}{3.407931in}}%
\pgfpathmoveto{\pgfqpoint{2.925146in}{3.410880in}}%
\pgfpathlineto{\pgfqpoint{2.925146in}{3.410880in}}%
\pgfpathlineto{\pgfqpoint{2.925146in}{3.413830in}}%
\pgfpathlineto{\pgfqpoint{2.929687in}{3.413830in}}%
\pgfpathlineto{\pgfqpoint{2.929687in}{3.410880in}}%
\pgfpathmoveto{\pgfqpoint{2.916064in}{3.416779in}}%
\pgfpathlineto{\pgfqpoint{2.916064in}{3.416779in}}%
\pgfpathlineto{\pgfqpoint{2.916064in}{3.419728in}}%
\pgfpathlineto{\pgfqpoint{2.920605in}{3.419728in}}%
\pgfpathlineto{\pgfqpoint{2.920605in}{3.416779in}}%
\pgfpathmoveto{\pgfqpoint{2.911523in}{3.422678in}}%
\pgfpathlineto{\pgfqpoint{2.911523in}{3.422678in}}%
\pgfpathlineto{\pgfqpoint{2.911523in}{3.425627in}}%
\pgfpathlineto{\pgfqpoint{2.916064in}{3.425627in}}%
\pgfpathlineto{\pgfqpoint{2.916064in}{3.422678in}}%
\pgfpathmoveto{\pgfqpoint{2.916064in}{3.419728in}}%
\pgfpathlineto{\pgfqpoint{2.916064in}{3.419728in}}%
\pgfpathlineto{\pgfqpoint{2.916064in}{3.422678in}}%
\pgfpathlineto{\pgfqpoint{2.920605in}{3.422678in}}%
\pgfpathlineto{\pgfqpoint{2.920605in}{3.419728in}}%
\pgfpathmoveto{\pgfqpoint{2.916064in}{3.422678in}}%
\pgfpathlineto{\pgfqpoint{2.916064in}{3.422678in}}%
\pgfpathlineto{\pgfqpoint{2.916064in}{3.425627in}}%
\pgfpathlineto{\pgfqpoint{2.920605in}{3.425627in}}%
\pgfpathlineto{\pgfqpoint{2.920605in}{3.422678in}}%
\pgfpathmoveto{\pgfqpoint{2.920605in}{3.413830in}}%
\pgfpathlineto{\pgfqpoint{2.920605in}{3.413830in}}%
\pgfpathlineto{\pgfqpoint{2.920605in}{3.416779in}}%
\pgfpathlineto{\pgfqpoint{2.925146in}{3.416779in}}%
\pgfpathlineto{\pgfqpoint{2.925146in}{3.413830in}}%
\pgfpathmoveto{\pgfqpoint{2.920605in}{3.416779in}}%
\pgfpathlineto{\pgfqpoint{2.920605in}{3.416779in}}%
\pgfpathlineto{\pgfqpoint{2.920605in}{3.419728in}}%
\pgfpathlineto{\pgfqpoint{2.925146in}{3.419728in}}%
\pgfpathlineto{\pgfqpoint{2.925146in}{3.416779in}}%
\pgfpathmoveto{\pgfqpoint{2.852491in}{3.499358in}}%
\pgfpathlineto{\pgfqpoint{2.852491in}{3.499358in}}%
\pgfpathlineto{\pgfqpoint{2.852491in}{3.502307in}}%
\pgfpathlineto{\pgfqpoint{2.857032in}{3.502307in}}%
\pgfpathlineto{\pgfqpoint{2.857032in}{3.499358in}}%
\pgfpathmoveto{\pgfqpoint{2.847950in}{3.505256in}}%
\pgfpathlineto{\pgfqpoint{2.847950in}{3.505256in}}%
\pgfpathlineto{\pgfqpoint{2.847950in}{3.508206in}}%
\pgfpathlineto{\pgfqpoint{2.852491in}{3.508206in}}%
\pgfpathlineto{\pgfqpoint{2.852491in}{3.505256in}}%
\pgfpathmoveto{\pgfqpoint{2.852491in}{3.502307in}}%
\pgfpathlineto{\pgfqpoint{2.852491in}{3.502307in}}%
\pgfpathlineto{\pgfqpoint{2.852491in}{3.505256in}}%
\pgfpathlineto{\pgfqpoint{2.857032in}{3.505256in}}%
\pgfpathlineto{\pgfqpoint{2.857032in}{3.502307in}}%
\pgfpathmoveto{\pgfqpoint{2.852491in}{3.505256in}}%
\pgfpathlineto{\pgfqpoint{2.852491in}{3.505256in}}%
\pgfpathlineto{\pgfqpoint{2.852491in}{3.508206in}}%
\pgfpathlineto{\pgfqpoint{2.857032in}{3.508206in}}%
\pgfpathlineto{\pgfqpoint{2.857032in}{3.505256in}}%
\pgfpathmoveto{\pgfqpoint{2.843409in}{3.511155in}}%
\pgfpathlineto{\pgfqpoint{2.843409in}{3.511155in}}%
\pgfpathlineto{\pgfqpoint{2.843409in}{3.514104in}}%
\pgfpathlineto{\pgfqpoint{2.847950in}{3.514104in}}%
\pgfpathlineto{\pgfqpoint{2.847950in}{3.511155in}}%
\pgfpathmoveto{\pgfqpoint{2.838868in}{3.517053in}}%
\pgfpathlineto{\pgfqpoint{2.838868in}{3.517053in}}%
\pgfpathlineto{\pgfqpoint{2.838868in}{3.520003in}}%
\pgfpathlineto{\pgfqpoint{2.843409in}{3.520003in}}%
\pgfpathlineto{\pgfqpoint{2.843409in}{3.517053in}}%
\pgfpathmoveto{\pgfqpoint{2.843409in}{3.514104in}}%
\pgfpathlineto{\pgfqpoint{2.843409in}{3.514104in}}%
\pgfpathlineto{\pgfqpoint{2.843409in}{3.517053in}}%
\pgfpathlineto{\pgfqpoint{2.847950in}{3.517053in}}%
\pgfpathlineto{\pgfqpoint{2.847950in}{3.514104in}}%
\pgfpathmoveto{\pgfqpoint{2.843409in}{3.517053in}}%
\pgfpathlineto{\pgfqpoint{2.843409in}{3.517053in}}%
\pgfpathlineto{\pgfqpoint{2.843409in}{3.520003in}}%
\pgfpathlineto{\pgfqpoint{2.847950in}{3.520003in}}%
\pgfpathlineto{\pgfqpoint{2.847950in}{3.517053in}}%
\pgfpathmoveto{\pgfqpoint{2.847950in}{3.508206in}}%
\pgfpathlineto{\pgfqpoint{2.847950in}{3.508206in}}%
\pgfpathlineto{\pgfqpoint{2.847950in}{3.511155in}}%
\pgfpathlineto{\pgfqpoint{2.852491in}{3.511155in}}%
\pgfpathlineto{\pgfqpoint{2.852491in}{3.508206in}}%
\pgfpathmoveto{\pgfqpoint{2.847950in}{3.511155in}}%
\pgfpathlineto{\pgfqpoint{2.847950in}{3.511155in}}%
\pgfpathlineto{\pgfqpoint{2.847950in}{3.514104in}}%
\pgfpathlineto{\pgfqpoint{2.852491in}{3.514104in}}%
\pgfpathlineto{\pgfqpoint{2.852491in}{3.511155in}}%
\pgfpathmoveto{\pgfqpoint{2.888819in}{3.452170in}}%
\pgfpathlineto{\pgfqpoint{2.888819in}{3.452170in}}%
\pgfpathlineto{\pgfqpoint{2.888819in}{3.455119in}}%
\pgfpathlineto{\pgfqpoint{2.893360in}{3.455119in}}%
\pgfpathlineto{\pgfqpoint{2.893360in}{3.452170in}}%
\pgfpathmoveto{\pgfqpoint{2.884278in}{3.458069in}}%
\pgfpathlineto{\pgfqpoint{2.884278in}{3.458069in}}%
\pgfpathlineto{\pgfqpoint{2.884278in}{3.461018in}}%
\pgfpathlineto{\pgfqpoint{2.888819in}{3.461018in}}%
\pgfpathlineto{\pgfqpoint{2.888819in}{3.458069in}}%
\pgfpathmoveto{\pgfqpoint{2.888819in}{3.455119in}}%
\pgfpathlineto{\pgfqpoint{2.888819in}{3.455119in}}%
\pgfpathlineto{\pgfqpoint{2.888819in}{3.458069in}}%
\pgfpathlineto{\pgfqpoint{2.893360in}{3.458069in}}%
\pgfpathlineto{\pgfqpoint{2.893360in}{3.455119in}}%
\pgfpathmoveto{\pgfqpoint{2.888819in}{3.458069in}}%
\pgfpathlineto{\pgfqpoint{2.888819in}{3.458069in}}%
\pgfpathlineto{\pgfqpoint{2.888819in}{3.461018in}}%
\pgfpathlineto{\pgfqpoint{2.893360in}{3.461018in}}%
\pgfpathlineto{\pgfqpoint{2.893360in}{3.458069in}}%
\pgfpathmoveto{\pgfqpoint{2.879737in}{3.463967in}}%
\pgfpathlineto{\pgfqpoint{2.879737in}{3.463967in}}%
\pgfpathlineto{\pgfqpoint{2.879737in}{3.466916in}}%
\pgfpathlineto{\pgfqpoint{2.884278in}{3.466916in}}%
\pgfpathlineto{\pgfqpoint{2.884278in}{3.463967in}}%
\pgfpathmoveto{\pgfqpoint{2.875196in}{3.469866in}}%
\pgfpathlineto{\pgfqpoint{2.875196in}{3.469866in}}%
\pgfpathlineto{\pgfqpoint{2.875196in}{3.472815in}}%
\pgfpathlineto{\pgfqpoint{2.879737in}{3.472815in}}%
\pgfpathlineto{\pgfqpoint{2.879737in}{3.469866in}}%
\pgfpathmoveto{\pgfqpoint{2.879737in}{3.466916in}}%
\pgfpathlineto{\pgfqpoint{2.879737in}{3.466916in}}%
\pgfpathlineto{\pgfqpoint{2.879737in}{3.469866in}}%
\pgfpathlineto{\pgfqpoint{2.884278in}{3.469866in}}%
\pgfpathlineto{\pgfqpoint{2.884278in}{3.466916in}}%
\pgfpathmoveto{\pgfqpoint{2.879737in}{3.469866in}}%
\pgfpathlineto{\pgfqpoint{2.879737in}{3.469866in}}%
\pgfpathlineto{\pgfqpoint{2.879737in}{3.472815in}}%
\pgfpathlineto{\pgfqpoint{2.884278in}{3.472815in}}%
\pgfpathlineto{\pgfqpoint{2.884278in}{3.469866in}}%
\pgfpathmoveto{\pgfqpoint{2.884278in}{3.461018in}}%
\pgfpathlineto{\pgfqpoint{2.884278in}{3.461018in}}%
\pgfpathlineto{\pgfqpoint{2.884278in}{3.463967in}}%
\pgfpathlineto{\pgfqpoint{2.888819in}{3.463967in}}%
\pgfpathlineto{\pgfqpoint{2.888819in}{3.461018in}}%
\pgfpathmoveto{\pgfqpoint{2.884278in}{3.463967in}}%
\pgfpathlineto{\pgfqpoint{2.884278in}{3.463967in}}%
\pgfpathlineto{\pgfqpoint{2.884278in}{3.466916in}}%
\pgfpathlineto{\pgfqpoint{2.888819in}{3.466916in}}%
\pgfpathlineto{\pgfqpoint{2.888819in}{3.463967in}}%
\pgfpathmoveto{\pgfqpoint{2.906982in}{3.428576in}}%
\pgfpathlineto{\pgfqpoint{2.906982in}{3.428576in}}%
\pgfpathlineto{\pgfqpoint{2.906982in}{3.431525in}}%
\pgfpathlineto{\pgfqpoint{2.911523in}{3.431525in}}%
\pgfpathlineto{\pgfqpoint{2.911523in}{3.428576in}}%
\pgfpathmoveto{\pgfqpoint{2.902441in}{3.434475in}}%
\pgfpathlineto{\pgfqpoint{2.902441in}{3.434475in}}%
\pgfpathlineto{\pgfqpoint{2.902441in}{3.437424in}}%
\pgfpathlineto{\pgfqpoint{2.906982in}{3.437424in}}%
\pgfpathlineto{\pgfqpoint{2.906982in}{3.434475in}}%
\pgfpathmoveto{\pgfqpoint{2.906982in}{3.431525in}}%
\pgfpathlineto{\pgfqpoint{2.906982in}{3.431525in}}%
\pgfpathlineto{\pgfqpoint{2.906982in}{3.434475in}}%
\pgfpathlineto{\pgfqpoint{2.911523in}{3.434475in}}%
\pgfpathlineto{\pgfqpoint{2.911523in}{3.431525in}}%
\pgfpathmoveto{\pgfqpoint{2.906982in}{3.434475in}}%
\pgfpathlineto{\pgfqpoint{2.906982in}{3.434475in}}%
\pgfpathlineto{\pgfqpoint{2.906982in}{3.437424in}}%
\pgfpathlineto{\pgfqpoint{2.911523in}{3.437424in}}%
\pgfpathlineto{\pgfqpoint{2.911523in}{3.434475in}}%
\pgfpathmoveto{\pgfqpoint{2.897900in}{3.440373in}}%
\pgfpathlineto{\pgfqpoint{2.897900in}{3.440373in}}%
\pgfpathlineto{\pgfqpoint{2.897900in}{3.443322in}}%
\pgfpathlineto{\pgfqpoint{2.902441in}{3.443322in}}%
\pgfpathlineto{\pgfqpoint{2.902441in}{3.440373in}}%
\pgfpathmoveto{\pgfqpoint{2.893360in}{3.446272in}}%
\pgfpathlineto{\pgfqpoint{2.893360in}{3.446272in}}%
\pgfpathlineto{\pgfqpoint{2.893360in}{3.449221in}}%
\pgfpathlineto{\pgfqpoint{2.897900in}{3.449221in}}%
\pgfpathlineto{\pgfqpoint{2.897900in}{3.446272in}}%
\pgfpathmoveto{\pgfqpoint{2.897900in}{3.443322in}}%
\pgfpathlineto{\pgfqpoint{2.897900in}{3.443322in}}%
\pgfpathlineto{\pgfqpoint{2.897900in}{3.446272in}}%
\pgfpathlineto{\pgfqpoint{2.902441in}{3.446272in}}%
\pgfpathlineto{\pgfqpoint{2.902441in}{3.443322in}}%
\pgfpathmoveto{\pgfqpoint{2.897900in}{3.446272in}}%
\pgfpathlineto{\pgfqpoint{2.897900in}{3.446272in}}%
\pgfpathlineto{\pgfqpoint{2.897900in}{3.449221in}}%
\pgfpathlineto{\pgfqpoint{2.902441in}{3.449221in}}%
\pgfpathlineto{\pgfqpoint{2.902441in}{3.446272in}}%
\pgfpathmoveto{\pgfqpoint{2.902441in}{3.437424in}}%
\pgfpathlineto{\pgfqpoint{2.902441in}{3.437424in}}%
\pgfpathlineto{\pgfqpoint{2.902441in}{3.440373in}}%
\pgfpathlineto{\pgfqpoint{2.906982in}{3.440373in}}%
\pgfpathlineto{\pgfqpoint{2.906982in}{3.437424in}}%
\pgfpathmoveto{\pgfqpoint{2.902441in}{3.440373in}}%
\pgfpathlineto{\pgfqpoint{2.902441in}{3.440373in}}%
\pgfpathlineto{\pgfqpoint{2.902441in}{3.443322in}}%
\pgfpathlineto{\pgfqpoint{2.906982in}{3.443322in}}%
\pgfpathlineto{\pgfqpoint{2.906982in}{3.440373in}}%
\pgfpathmoveto{\pgfqpoint{2.911523in}{3.425627in}}%
\pgfpathlineto{\pgfqpoint{2.911523in}{3.425627in}}%
\pgfpathlineto{\pgfqpoint{2.911523in}{3.428576in}}%
\pgfpathlineto{\pgfqpoint{2.916064in}{3.428576in}}%
\pgfpathlineto{\pgfqpoint{2.916064in}{3.425627in}}%
\pgfpathmoveto{\pgfqpoint{2.911523in}{3.428576in}}%
\pgfpathlineto{\pgfqpoint{2.911523in}{3.428576in}}%
\pgfpathlineto{\pgfqpoint{2.911523in}{3.431525in}}%
\pgfpathlineto{\pgfqpoint{2.916064in}{3.431525in}}%
\pgfpathlineto{\pgfqpoint{2.916064in}{3.428576in}}%
\pgfpathmoveto{\pgfqpoint{2.893360in}{3.449221in}}%
\pgfpathlineto{\pgfqpoint{2.893360in}{3.449221in}}%
\pgfpathlineto{\pgfqpoint{2.893360in}{3.452170in}}%
\pgfpathlineto{\pgfqpoint{2.897900in}{3.452170in}}%
\pgfpathlineto{\pgfqpoint{2.897900in}{3.449221in}}%
\pgfpathmoveto{\pgfqpoint{2.893360in}{3.452170in}}%
\pgfpathlineto{\pgfqpoint{2.893360in}{3.452170in}}%
\pgfpathlineto{\pgfqpoint{2.893360in}{3.455119in}}%
\pgfpathlineto{\pgfqpoint{2.897900in}{3.455119in}}%
\pgfpathlineto{\pgfqpoint{2.897900in}{3.452170in}}%
\pgfpathmoveto{\pgfqpoint{2.870655in}{3.475764in}}%
\pgfpathlineto{\pgfqpoint{2.870655in}{3.475764in}}%
\pgfpathlineto{\pgfqpoint{2.870655in}{3.478713in}}%
\pgfpathlineto{\pgfqpoint{2.875196in}{3.478713in}}%
\pgfpathlineto{\pgfqpoint{2.875196in}{3.475764in}}%
\pgfpathmoveto{\pgfqpoint{2.866114in}{3.481663in}}%
\pgfpathlineto{\pgfqpoint{2.866114in}{3.481663in}}%
\pgfpathlineto{\pgfqpoint{2.866114in}{3.484612in}}%
\pgfpathlineto{\pgfqpoint{2.870655in}{3.484612in}}%
\pgfpathlineto{\pgfqpoint{2.870655in}{3.481663in}}%
\pgfpathmoveto{\pgfqpoint{2.870655in}{3.478713in}}%
\pgfpathlineto{\pgfqpoint{2.870655in}{3.478713in}}%
\pgfpathlineto{\pgfqpoint{2.870655in}{3.481663in}}%
\pgfpathlineto{\pgfqpoint{2.875196in}{3.481663in}}%
\pgfpathlineto{\pgfqpoint{2.875196in}{3.478713in}}%
\pgfpathmoveto{\pgfqpoint{2.870655in}{3.481663in}}%
\pgfpathlineto{\pgfqpoint{2.870655in}{3.481663in}}%
\pgfpathlineto{\pgfqpoint{2.870655in}{3.484612in}}%
\pgfpathlineto{\pgfqpoint{2.875196in}{3.484612in}}%
\pgfpathlineto{\pgfqpoint{2.875196in}{3.481663in}}%
\pgfpathmoveto{\pgfqpoint{2.861573in}{3.487561in}}%
\pgfpathlineto{\pgfqpoint{2.861573in}{3.487561in}}%
\pgfpathlineto{\pgfqpoint{2.861573in}{3.490510in}}%
\pgfpathlineto{\pgfqpoint{2.866114in}{3.490510in}}%
\pgfpathlineto{\pgfqpoint{2.866114in}{3.487561in}}%
\pgfpathmoveto{\pgfqpoint{2.857032in}{3.493460in}}%
\pgfpathlineto{\pgfqpoint{2.857032in}{3.493460in}}%
\pgfpathlineto{\pgfqpoint{2.857032in}{3.496409in}}%
\pgfpathlineto{\pgfqpoint{2.861573in}{3.496409in}}%
\pgfpathlineto{\pgfqpoint{2.861573in}{3.493460in}}%
\pgfpathmoveto{\pgfqpoint{2.861573in}{3.490510in}}%
\pgfpathlineto{\pgfqpoint{2.861573in}{3.490510in}}%
\pgfpathlineto{\pgfqpoint{2.861573in}{3.493460in}}%
\pgfpathlineto{\pgfqpoint{2.866114in}{3.493460in}}%
\pgfpathlineto{\pgfqpoint{2.866114in}{3.490510in}}%
\pgfpathmoveto{\pgfqpoint{2.861573in}{3.493460in}}%
\pgfpathlineto{\pgfqpoint{2.861573in}{3.493460in}}%
\pgfpathlineto{\pgfqpoint{2.861573in}{3.496409in}}%
\pgfpathlineto{\pgfqpoint{2.866114in}{3.496409in}}%
\pgfpathlineto{\pgfqpoint{2.866114in}{3.493460in}}%
\pgfpathmoveto{\pgfqpoint{2.866114in}{3.484612in}}%
\pgfpathlineto{\pgfqpoint{2.866114in}{3.484612in}}%
\pgfpathlineto{\pgfqpoint{2.866114in}{3.487561in}}%
\pgfpathlineto{\pgfqpoint{2.870655in}{3.487561in}}%
\pgfpathlineto{\pgfqpoint{2.870655in}{3.484612in}}%
\pgfpathmoveto{\pgfqpoint{2.866114in}{3.487561in}}%
\pgfpathlineto{\pgfqpoint{2.866114in}{3.487561in}}%
\pgfpathlineto{\pgfqpoint{2.866114in}{3.490510in}}%
\pgfpathlineto{\pgfqpoint{2.870655in}{3.490510in}}%
\pgfpathlineto{\pgfqpoint{2.870655in}{3.487561in}}%
\pgfpathmoveto{\pgfqpoint{2.875196in}{3.472815in}}%
\pgfpathlineto{\pgfqpoint{2.875196in}{3.472815in}}%
\pgfpathlineto{\pgfqpoint{2.875196in}{3.475764in}}%
\pgfpathlineto{\pgfqpoint{2.879737in}{3.475764in}}%
\pgfpathlineto{\pgfqpoint{2.879737in}{3.472815in}}%
\pgfpathmoveto{\pgfqpoint{2.875196in}{3.475764in}}%
\pgfpathlineto{\pgfqpoint{2.875196in}{3.475764in}}%
\pgfpathlineto{\pgfqpoint{2.875196in}{3.478713in}}%
\pgfpathlineto{\pgfqpoint{2.879737in}{3.478713in}}%
\pgfpathlineto{\pgfqpoint{2.879737in}{3.475764in}}%
\pgfpathmoveto{\pgfqpoint{2.857032in}{3.496409in}}%
\pgfpathlineto{\pgfqpoint{2.857032in}{3.496409in}}%
\pgfpathlineto{\pgfqpoint{2.857032in}{3.499358in}}%
\pgfpathlineto{\pgfqpoint{2.861573in}{3.499358in}}%
\pgfpathlineto{\pgfqpoint{2.861573in}{3.496409in}}%
\pgfpathmoveto{\pgfqpoint{2.857032in}{3.499358in}}%
\pgfpathlineto{\pgfqpoint{2.857032in}{3.499358in}}%
\pgfpathlineto{\pgfqpoint{2.857032in}{3.502307in}}%
\pgfpathlineto{\pgfqpoint{2.861573in}{3.502307in}}%
\pgfpathlineto{\pgfqpoint{2.861573in}{3.499358in}}%
\pgfpathmoveto{\pgfqpoint{3.070462in}{3.216230in}}%
\pgfpathlineto{\pgfqpoint{3.070462in}{3.216230in}}%
\pgfpathlineto{\pgfqpoint{3.070462in}{3.219179in}}%
\pgfpathlineto{\pgfqpoint{3.075003in}{3.219179in}}%
\pgfpathlineto{\pgfqpoint{3.075003in}{3.216230in}}%
\pgfpathmoveto{\pgfqpoint{3.065920in}{3.222128in}}%
\pgfpathlineto{\pgfqpoint{3.065920in}{3.222128in}}%
\pgfpathlineto{\pgfqpoint{3.065920in}{3.225077in}}%
\pgfpathlineto{\pgfqpoint{3.070462in}{3.225077in}}%
\pgfpathlineto{\pgfqpoint{3.070462in}{3.222128in}}%
\pgfpathmoveto{\pgfqpoint{3.070462in}{3.219179in}}%
\pgfpathlineto{\pgfqpoint{3.070462in}{3.219179in}}%
\pgfpathlineto{\pgfqpoint{3.070462in}{3.222128in}}%
\pgfpathlineto{\pgfqpoint{3.075003in}{3.222128in}}%
\pgfpathlineto{\pgfqpoint{3.075003in}{3.219179in}}%
\pgfpathmoveto{\pgfqpoint{3.070462in}{3.222128in}}%
\pgfpathlineto{\pgfqpoint{3.070462in}{3.222128in}}%
\pgfpathlineto{\pgfqpoint{3.070462in}{3.225077in}}%
\pgfpathlineto{\pgfqpoint{3.075003in}{3.225077in}}%
\pgfpathlineto{\pgfqpoint{3.075003in}{3.222128in}}%
\pgfpathmoveto{\pgfqpoint{3.061379in}{3.228027in}}%
\pgfpathlineto{\pgfqpoint{3.061379in}{3.228027in}}%
\pgfpathlineto{\pgfqpoint{3.061379in}{3.230976in}}%
\pgfpathlineto{\pgfqpoint{3.065920in}{3.230976in}}%
\pgfpathlineto{\pgfqpoint{3.065920in}{3.228027in}}%
\pgfpathmoveto{\pgfqpoint{3.056838in}{3.233925in}}%
\pgfpathlineto{\pgfqpoint{3.056838in}{3.233925in}}%
\pgfpathlineto{\pgfqpoint{3.056838in}{3.236874in}}%
\pgfpathlineto{\pgfqpoint{3.061379in}{3.236874in}}%
\pgfpathlineto{\pgfqpoint{3.061379in}{3.233925in}}%
\pgfpathmoveto{\pgfqpoint{3.061379in}{3.230976in}}%
\pgfpathlineto{\pgfqpoint{3.061379in}{3.230976in}}%
\pgfpathlineto{\pgfqpoint{3.061379in}{3.233925in}}%
\pgfpathlineto{\pgfqpoint{3.065920in}{3.233925in}}%
\pgfpathlineto{\pgfqpoint{3.065920in}{3.230976in}}%
\pgfpathmoveto{\pgfqpoint{3.061379in}{3.233925in}}%
\pgfpathlineto{\pgfqpoint{3.061379in}{3.233925in}}%
\pgfpathlineto{\pgfqpoint{3.061379in}{3.236874in}}%
\pgfpathlineto{\pgfqpoint{3.065920in}{3.236874in}}%
\pgfpathlineto{\pgfqpoint{3.065920in}{3.233925in}}%
\pgfpathmoveto{\pgfqpoint{3.065920in}{3.225077in}}%
\pgfpathlineto{\pgfqpoint{3.065920in}{3.225077in}}%
\pgfpathlineto{\pgfqpoint{3.065920in}{3.228027in}}%
\pgfpathlineto{\pgfqpoint{3.070462in}{3.228027in}}%
\pgfpathlineto{\pgfqpoint{3.070462in}{3.225077in}}%
\pgfpathmoveto{\pgfqpoint{3.065920in}{3.228027in}}%
\pgfpathlineto{\pgfqpoint{3.065920in}{3.228027in}}%
\pgfpathlineto{\pgfqpoint{3.065920in}{3.230976in}}%
\pgfpathlineto{\pgfqpoint{3.070462in}{3.230976in}}%
\pgfpathlineto{\pgfqpoint{3.070462in}{3.228027in}}%
\pgfpathmoveto{\pgfqpoint{2.997804in}{3.310605in}}%
\pgfpathlineto{\pgfqpoint{2.997804in}{3.310605in}}%
\pgfpathlineto{\pgfqpoint{2.997804in}{3.313554in}}%
\pgfpathlineto{\pgfqpoint{3.002345in}{3.313554in}}%
\pgfpathlineto{\pgfqpoint{3.002345in}{3.310605in}}%
\pgfpathmoveto{\pgfqpoint{2.993263in}{3.316503in}}%
\pgfpathlineto{\pgfqpoint{2.993263in}{3.316503in}}%
\pgfpathlineto{\pgfqpoint{2.993263in}{3.319452in}}%
\pgfpathlineto{\pgfqpoint{2.997804in}{3.319452in}}%
\pgfpathlineto{\pgfqpoint{2.997804in}{3.316503in}}%
\pgfpathmoveto{\pgfqpoint{2.997804in}{3.313554in}}%
\pgfpathlineto{\pgfqpoint{2.997804in}{3.313554in}}%
\pgfpathlineto{\pgfqpoint{2.997804in}{3.316503in}}%
\pgfpathlineto{\pgfqpoint{3.002345in}{3.316503in}}%
\pgfpathlineto{\pgfqpoint{3.002345in}{3.313554in}}%
\pgfpathmoveto{\pgfqpoint{2.997804in}{3.316503in}}%
\pgfpathlineto{\pgfqpoint{2.997804in}{3.316503in}}%
\pgfpathlineto{\pgfqpoint{2.997804in}{3.319452in}}%
\pgfpathlineto{\pgfqpoint{3.002345in}{3.319452in}}%
\pgfpathlineto{\pgfqpoint{3.002345in}{3.316503in}}%
\pgfpathmoveto{\pgfqpoint{2.988722in}{3.322402in}}%
\pgfpathlineto{\pgfqpoint{2.988722in}{3.322402in}}%
\pgfpathlineto{\pgfqpoint{2.988722in}{3.325351in}}%
\pgfpathlineto{\pgfqpoint{2.993263in}{3.325351in}}%
\pgfpathlineto{\pgfqpoint{2.993263in}{3.322402in}}%
\pgfpathmoveto{\pgfqpoint{2.984180in}{3.328300in}}%
\pgfpathlineto{\pgfqpoint{2.984180in}{3.328300in}}%
\pgfpathlineto{\pgfqpoint{2.984180in}{3.331249in}}%
\pgfpathlineto{\pgfqpoint{2.988722in}{3.331249in}}%
\pgfpathlineto{\pgfqpoint{2.988722in}{3.328300in}}%
\pgfpathmoveto{\pgfqpoint{2.988722in}{3.325351in}}%
\pgfpathlineto{\pgfqpoint{2.988722in}{3.325351in}}%
\pgfpathlineto{\pgfqpoint{2.988722in}{3.328300in}}%
\pgfpathlineto{\pgfqpoint{2.993263in}{3.328300in}}%
\pgfpathlineto{\pgfqpoint{2.993263in}{3.325351in}}%
\pgfpathmoveto{\pgfqpoint{2.988722in}{3.328300in}}%
\pgfpathlineto{\pgfqpoint{2.988722in}{3.328300in}}%
\pgfpathlineto{\pgfqpoint{2.988722in}{3.331249in}}%
\pgfpathlineto{\pgfqpoint{2.993263in}{3.331249in}}%
\pgfpathlineto{\pgfqpoint{2.993263in}{3.328300in}}%
\pgfpathmoveto{\pgfqpoint{2.993263in}{3.319452in}}%
\pgfpathlineto{\pgfqpoint{2.993263in}{3.319452in}}%
\pgfpathlineto{\pgfqpoint{2.993263in}{3.322402in}}%
\pgfpathlineto{\pgfqpoint{2.997804in}{3.322402in}}%
\pgfpathlineto{\pgfqpoint{2.997804in}{3.319452in}}%
\pgfpathmoveto{\pgfqpoint{2.993263in}{3.322402in}}%
\pgfpathlineto{\pgfqpoint{2.993263in}{3.322402in}}%
\pgfpathlineto{\pgfqpoint{2.993263in}{3.325351in}}%
\pgfpathlineto{\pgfqpoint{2.997804in}{3.325351in}}%
\pgfpathlineto{\pgfqpoint{2.997804in}{3.322402in}}%
\pgfpathmoveto{\pgfqpoint{3.034133in}{3.263417in}}%
\pgfpathlineto{\pgfqpoint{3.034133in}{3.263417in}}%
\pgfpathlineto{\pgfqpoint{3.034133in}{3.266366in}}%
\pgfpathlineto{\pgfqpoint{3.038674in}{3.266366in}}%
\pgfpathlineto{\pgfqpoint{3.038674in}{3.263417in}}%
\pgfpathmoveto{\pgfqpoint{3.029592in}{3.269316in}}%
\pgfpathlineto{\pgfqpoint{3.029592in}{3.269316in}}%
\pgfpathlineto{\pgfqpoint{3.029592in}{3.272265in}}%
\pgfpathlineto{\pgfqpoint{3.034133in}{3.272265in}}%
\pgfpathlineto{\pgfqpoint{3.034133in}{3.269316in}}%
\pgfpathmoveto{\pgfqpoint{3.034133in}{3.266366in}}%
\pgfpathlineto{\pgfqpoint{3.034133in}{3.266366in}}%
\pgfpathlineto{\pgfqpoint{3.034133in}{3.269316in}}%
\pgfpathlineto{\pgfqpoint{3.038674in}{3.269316in}}%
\pgfpathlineto{\pgfqpoint{3.038674in}{3.266366in}}%
\pgfpathmoveto{\pgfqpoint{3.034133in}{3.269316in}}%
\pgfpathlineto{\pgfqpoint{3.034133in}{3.269316in}}%
\pgfpathlineto{\pgfqpoint{3.034133in}{3.272265in}}%
\pgfpathlineto{\pgfqpoint{3.038674in}{3.272265in}}%
\pgfpathlineto{\pgfqpoint{3.038674in}{3.269316in}}%
\pgfpathmoveto{\pgfqpoint{3.025050in}{3.275214in}}%
\pgfpathlineto{\pgfqpoint{3.025050in}{3.275214in}}%
\pgfpathlineto{\pgfqpoint{3.025050in}{3.278163in}}%
\pgfpathlineto{\pgfqpoint{3.029592in}{3.278163in}}%
\pgfpathlineto{\pgfqpoint{3.029592in}{3.275214in}}%
\pgfpathmoveto{\pgfqpoint{3.020509in}{3.281113in}}%
\pgfpathlineto{\pgfqpoint{3.020509in}{3.281113in}}%
\pgfpathlineto{\pgfqpoint{3.020509in}{3.284062in}}%
\pgfpathlineto{\pgfqpoint{3.025050in}{3.284062in}}%
\pgfpathlineto{\pgfqpoint{3.025050in}{3.281113in}}%
\pgfpathmoveto{\pgfqpoint{3.025050in}{3.278163in}}%
\pgfpathlineto{\pgfqpoint{3.025050in}{3.278163in}}%
\pgfpathlineto{\pgfqpoint{3.025050in}{3.281113in}}%
\pgfpathlineto{\pgfqpoint{3.029592in}{3.281113in}}%
\pgfpathlineto{\pgfqpoint{3.029592in}{3.278163in}}%
\pgfpathmoveto{\pgfqpoint{3.025050in}{3.281113in}}%
\pgfpathlineto{\pgfqpoint{3.025050in}{3.281113in}}%
\pgfpathlineto{\pgfqpoint{3.025050in}{3.284062in}}%
\pgfpathlineto{\pgfqpoint{3.029592in}{3.284062in}}%
\pgfpathlineto{\pgfqpoint{3.029592in}{3.281113in}}%
\pgfpathmoveto{\pgfqpoint{3.029592in}{3.272265in}}%
\pgfpathlineto{\pgfqpoint{3.029592in}{3.272265in}}%
\pgfpathlineto{\pgfqpoint{3.029592in}{3.275214in}}%
\pgfpathlineto{\pgfqpoint{3.034133in}{3.275214in}}%
\pgfpathlineto{\pgfqpoint{3.034133in}{3.272265in}}%
\pgfpathmoveto{\pgfqpoint{3.029592in}{3.275214in}}%
\pgfpathlineto{\pgfqpoint{3.029592in}{3.275214in}}%
\pgfpathlineto{\pgfqpoint{3.029592in}{3.278163in}}%
\pgfpathlineto{\pgfqpoint{3.034133in}{3.278163in}}%
\pgfpathlineto{\pgfqpoint{3.034133in}{3.275214in}}%
\pgfpathmoveto{\pgfqpoint{3.052297in}{3.239823in}}%
\pgfpathlineto{\pgfqpoint{3.052297in}{3.239823in}}%
\pgfpathlineto{\pgfqpoint{3.052297in}{3.242773in}}%
\pgfpathlineto{\pgfqpoint{3.056838in}{3.242773in}}%
\pgfpathlineto{\pgfqpoint{3.056838in}{3.239823in}}%
\pgfpathmoveto{\pgfqpoint{3.047756in}{3.245722in}}%
\pgfpathlineto{\pgfqpoint{3.047756in}{3.245722in}}%
\pgfpathlineto{\pgfqpoint{3.047756in}{3.248671in}}%
\pgfpathlineto{\pgfqpoint{3.052297in}{3.248671in}}%
\pgfpathlineto{\pgfqpoint{3.052297in}{3.245722in}}%
\pgfpathmoveto{\pgfqpoint{3.052297in}{3.242773in}}%
\pgfpathlineto{\pgfqpoint{3.052297in}{3.242773in}}%
\pgfpathlineto{\pgfqpoint{3.052297in}{3.245722in}}%
\pgfpathlineto{\pgfqpoint{3.056838in}{3.245722in}}%
\pgfpathlineto{\pgfqpoint{3.056838in}{3.242773in}}%
\pgfpathmoveto{\pgfqpoint{3.052297in}{3.245722in}}%
\pgfpathlineto{\pgfqpoint{3.052297in}{3.245722in}}%
\pgfpathlineto{\pgfqpoint{3.052297in}{3.248671in}}%
\pgfpathlineto{\pgfqpoint{3.056838in}{3.248671in}}%
\pgfpathlineto{\pgfqpoint{3.056838in}{3.245722in}}%
\pgfpathmoveto{\pgfqpoint{3.043215in}{3.251620in}}%
\pgfpathlineto{\pgfqpoint{3.043215in}{3.251620in}}%
\pgfpathlineto{\pgfqpoint{3.043215in}{3.254570in}}%
\pgfpathlineto{\pgfqpoint{3.047756in}{3.254570in}}%
\pgfpathlineto{\pgfqpoint{3.047756in}{3.251620in}}%
\pgfpathmoveto{\pgfqpoint{3.038674in}{3.257519in}}%
\pgfpathlineto{\pgfqpoint{3.038674in}{3.257519in}}%
\pgfpathlineto{\pgfqpoint{3.038674in}{3.260468in}}%
\pgfpathlineto{\pgfqpoint{3.043215in}{3.260468in}}%
\pgfpathlineto{\pgfqpoint{3.043215in}{3.257519in}}%
\pgfpathmoveto{\pgfqpoint{3.043215in}{3.254570in}}%
\pgfpathlineto{\pgfqpoint{3.043215in}{3.254570in}}%
\pgfpathlineto{\pgfqpoint{3.043215in}{3.257519in}}%
\pgfpathlineto{\pgfqpoint{3.047756in}{3.257519in}}%
\pgfpathlineto{\pgfqpoint{3.047756in}{3.254570in}}%
\pgfpathmoveto{\pgfqpoint{3.043215in}{3.257519in}}%
\pgfpathlineto{\pgfqpoint{3.043215in}{3.257519in}}%
\pgfpathlineto{\pgfqpoint{3.043215in}{3.260468in}}%
\pgfpathlineto{\pgfqpoint{3.047756in}{3.260468in}}%
\pgfpathlineto{\pgfqpoint{3.047756in}{3.257519in}}%
\pgfpathmoveto{\pgfqpoint{3.047756in}{3.248671in}}%
\pgfpathlineto{\pgfqpoint{3.047756in}{3.248671in}}%
\pgfpathlineto{\pgfqpoint{3.047756in}{3.251620in}}%
\pgfpathlineto{\pgfqpoint{3.052297in}{3.251620in}}%
\pgfpathlineto{\pgfqpoint{3.052297in}{3.248671in}}%
\pgfpathmoveto{\pgfqpoint{3.047756in}{3.251620in}}%
\pgfpathlineto{\pgfqpoint{3.047756in}{3.251620in}}%
\pgfpathlineto{\pgfqpoint{3.047756in}{3.254570in}}%
\pgfpathlineto{\pgfqpoint{3.052297in}{3.254570in}}%
\pgfpathlineto{\pgfqpoint{3.052297in}{3.251620in}}%
\pgfpathmoveto{\pgfqpoint{3.056838in}{3.236874in}}%
\pgfpathlineto{\pgfqpoint{3.056838in}{3.236874in}}%
\pgfpathlineto{\pgfqpoint{3.056838in}{3.239823in}}%
\pgfpathlineto{\pgfqpoint{3.061379in}{3.239823in}}%
\pgfpathlineto{\pgfqpoint{3.061379in}{3.236874in}}%
\pgfpathmoveto{\pgfqpoint{3.056838in}{3.239823in}}%
\pgfpathlineto{\pgfqpoint{3.056838in}{3.239823in}}%
\pgfpathlineto{\pgfqpoint{3.056838in}{3.242773in}}%
\pgfpathlineto{\pgfqpoint{3.061379in}{3.242773in}}%
\pgfpathlineto{\pgfqpoint{3.061379in}{3.239823in}}%
\pgfpathmoveto{\pgfqpoint{3.038674in}{3.260468in}}%
\pgfpathlineto{\pgfqpoint{3.038674in}{3.260468in}}%
\pgfpathlineto{\pgfqpoint{3.038674in}{3.263417in}}%
\pgfpathlineto{\pgfqpoint{3.043215in}{3.263417in}}%
\pgfpathlineto{\pgfqpoint{3.043215in}{3.260468in}}%
\pgfpathmoveto{\pgfqpoint{3.038674in}{3.263417in}}%
\pgfpathlineto{\pgfqpoint{3.038674in}{3.263417in}}%
\pgfpathlineto{\pgfqpoint{3.038674in}{3.266366in}}%
\pgfpathlineto{\pgfqpoint{3.043215in}{3.266366in}}%
\pgfpathlineto{\pgfqpoint{3.043215in}{3.263417in}}%
\pgfpathmoveto{\pgfqpoint{3.015968in}{3.287011in}}%
\pgfpathlineto{\pgfqpoint{3.015968in}{3.287011in}}%
\pgfpathlineto{\pgfqpoint{3.015968in}{3.289960in}}%
\pgfpathlineto{\pgfqpoint{3.020509in}{3.289960in}}%
\pgfpathlineto{\pgfqpoint{3.020509in}{3.287011in}}%
\pgfpathmoveto{\pgfqpoint{3.011427in}{3.292909in}}%
\pgfpathlineto{\pgfqpoint{3.011427in}{3.292909in}}%
\pgfpathlineto{\pgfqpoint{3.011427in}{3.295859in}}%
\pgfpathlineto{\pgfqpoint{3.015968in}{3.295859in}}%
\pgfpathlineto{\pgfqpoint{3.015968in}{3.292909in}}%
\pgfpathmoveto{\pgfqpoint{3.015968in}{3.289960in}}%
\pgfpathlineto{\pgfqpoint{3.015968in}{3.289960in}}%
\pgfpathlineto{\pgfqpoint{3.015968in}{3.292909in}}%
\pgfpathlineto{\pgfqpoint{3.020509in}{3.292909in}}%
\pgfpathlineto{\pgfqpoint{3.020509in}{3.289960in}}%
\pgfpathmoveto{\pgfqpoint{3.015968in}{3.292909in}}%
\pgfpathlineto{\pgfqpoint{3.015968in}{3.292909in}}%
\pgfpathlineto{\pgfqpoint{3.015968in}{3.295859in}}%
\pgfpathlineto{\pgfqpoint{3.020509in}{3.295859in}}%
\pgfpathlineto{\pgfqpoint{3.020509in}{3.292909in}}%
\pgfpathmoveto{\pgfqpoint{3.006886in}{3.298808in}}%
\pgfpathlineto{\pgfqpoint{3.006886in}{3.298808in}}%
\pgfpathlineto{\pgfqpoint{3.006886in}{3.301757in}}%
\pgfpathlineto{\pgfqpoint{3.011427in}{3.301757in}}%
\pgfpathlineto{\pgfqpoint{3.011427in}{3.298808in}}%
\pgfpathmoveto{\pgfqpoint{3.002345in}{3.304706in}}%
\pgfpathlineto{\pgfqpoint{3.002345in}{3.304706in}}%
\pgfpathlineto{\pgfqpoint{3.002345in}{3.307656in}}%
\pgfpathlineto{\pgfqpoint{3.006886in}{3.307656in}}%
\pgfpathlineto{\pgfqpoint{3.006886in}{3.304706in}}%
\pgfpathmoveto{\pgfqpoint{3.006886in}{3.301757in}}%
\pgfpathlineto{\pgfqpoint{3.006886in}{3.301757in}}%
\pgfpathlineto{\pgfqpoint{3.006886in}{3.304706in}}%
\pgfpathlineto{\pgfqpoint{3.011427in}{3.304706in}}%
\pgfpathlineto{\pgfqpoint{3.011427in}{3.301757in}}%
\pgfpathmoveto{\pgfqpoint{3.006886in}{3.304706in}}%
\pgfpathlineto{\pgfqpoint{3.006886in}{3.304706in}}%
\pgfpathlineto{\pgfqpoint{3.006886in}{3.307656in}}%
\pgfpathlineto{\pgfqpoint{3.011427in}{3.307656in}}%
\pgfpathlineto{\pgfqpoint{3.011427in}{3.304706in}}%
\pgfpathmoveto{\pgfqpoint{3.011427in}{3.295859in}}%
\pgfpathlineto{\pgfqpoint{3.011427in}{3.295859in}}%
\pgfpathlineto{\pgfqpoint{3.011427in}{3.298808in}}%
\pgfpathlineto{\pgfqpoint{3.015968in}{3.298808in}}%
\pgfpathlineto{\pgfqpoint{3.015968in}{3.295859in}}%
\pgfpathmoveto{\pgfqpoint{3.011427in}{3.298808in}}%
\pgfpathlineto{\pgfqpoint{3.011427in}{3.298808in}}%
\pgfpathlineto{\pgfqpoint{3.011427in}{3.301757in}}%
\pgfpathlineto{\pgfqpoint{3.015968in}{3.301757in}}%
\pgfpathlineto{\pgfqpoint{3.015968in}{3.298808in}}%
\pgfpathmoveto{\pgfqpoint{3.020509in}{3.284062in}}%
\pgfpathlineto{\pgfqpoint{3.020509in}{3.284062in}}%
\pgfpathlineto{\pgfqpoint{3.020509in}{3.287011in}}%
\pgfpathlineto{\pgfqpoint{3.025050in}{3.287011in}}%
\pgfpathlineto{\pgfqpoint{3.025050in}{3.284062in}}%
\pgfpathmoveto{\pgfqpoint{3.020509in}{3.287011in}}%
\pgfpathlineto{\pgfqpoint{3.020509in}{3.287011in}}%
\pgfpathlineto{\pgfqpoint{3.020509in}{3.289960in}}%
\pgfpathlineto{\pgfqpoint{3.025050in}{3.289960in}}%
\pgfpathlineto{\pgfqpoint{3.025050in}{3.287011in}}%
\pgfpathmoveto{\pgfqpoint{3.002345in}{3.307656in}}%
\pgfpathlineto{\pgfqpoint{3.002345in}{3.307656in}}%
\pgfpathlineto{\pgfqpoint{3.002345in}{3.310605in}}%
\pgfpathlineto{\pgfqpoint{3.006886in}{3.310605in}}%
\pgfpathlineto{\pgfqpoint{3.006886in}{3.307656in}}%
\pgfpathmoveto{\pgfqpoint{3.002345in}{3.310605in}}%
\pgfpathlineto{\pgfqpoint{3.002345in}{3.310605in}}%
\pgfpathlineto{\pgfqpoint{3.002345in}{3.313554in}}%
\pgfpathlineto{\pgfqpoint{3.006886in}{3.313554in}}%
\pgfpathlineto{\pgfqpoint{3.006886in}{3.310605in}}%
\pgfpathmoveto{\pgfqpoint{2.961475in}{3.357793in}}%
\pgfpathlineto{\pgfqpoint{2.961475in}{3.357793in}}%
\pgfpathlineto{\pgfqpoint{2.961475in}{3.360742in}}%
\pgfpathlineto{\pgfqpoint{2.966016in}{3.360742in}}%
\pgfpathlineto{\pgfqpoint{2.966016in}{3.357793in}}%
\pgfpathmoveto{\pgfqpoint{2.956934in}{3.363692in}}%
\pgfpathlineto{\pgfqpoint{2.956934in}{3.363692in}}%
\pgfpathlineto{\pgfqpoint{2.956934in}{3.366641in}}%
\pgfpathlineto{\pgfqpoint{2.961475in}{3.366641in}}%
\pgfpathlineto{\pgfqpoint{2.961475in}{3.363692in}}%
\pgfpathmoveto{\pgfqpoint{2.961475in}{3.360742in}}%
\pgfpathlineto{\pgfqpoint{2.961475in}{3.360742in}}%
\pgfpathlineto{\pgfqpoint{2.961475in}{3.363692in}}%
\pgfpathlineto{\pgfqpoint{2.966016in}{3.363692in}}%
\pgfpathlineto{\pgfqpoint{2.966016in}{3.360742in}}%
\pgfpathmoveto{\pgfqpoint{2.961475in}{3.363692in}}%
\pgfpathlineto{\pgfqpoint{2.961475in}{3.363692in}}%
\pgfpathlineto{\pgfqpoint{2.961475in}{3.366641in}}%
\pgfpathlineto{\pgfqpoint{2.966016in}{3.366641in}}%
\pgfpathlineto{\pgfqpoint{2.966016in}{3.363692in}}%
\pgfpathmoveto{\pgfqpoint{2.952393in}{3.369590in}}%
\pgfpathlineto{\pgfqpoint{2.952393in}{3.369590in}}%
\pgfpathlineto{\pgfqpoint{2.952393in}{3.372540in}}%
\pgfpathlineto{\pgfqpoint{2.956934in}{3.372540in}}%
\pgfpathlineto{\pgfqpoint{2.956934in}{3.369590in}}%
\pgfpathmoveto{\pgfqpoint{2.947852in}{3.375489in}}%
\pgfpathlineto{\pgfqpoint{2.947852in}{3.375489in}}%
\pgfpathlineto{\pgfqpoint{2.947852in}{3.378438in}}%
\pgfpathlineto{\pgfqpoint{2.952393in}{3.378438in}}%
\pgfpathlineto{\pgfqpoint{2.952393in}{3.375489in}}%
\pgfpathmoveto{\pgfqpoint{2.952393in}{3.372540in}}%
\pgfpathlineto{\pgfqpoint{2.952393in}{3.372540in}}%
\pgfpathlineto{\pgfqpoint{2.952393in}{3.375489in}}%
\pgfpathlineto{\pgfqpoint{2.956934in}{3.375489in}}%
\pgfpathlineto{\pgfqpoint{2.956934in}{3.372540in}}%
\pgfpathmoveto{\pgfqpoint{2.952393in}{3.375489in}}%
\pgfpathlineto{\pgfqpoint{2.952393in}{3.375489in}}%
\pgfpathlineto{\pgfqpoint{2.952393in}{3.378438in}}%
\pgfpathlineto{\pgfqpoint{2.956934in}{3.378438in}}%
\pgfpathlineto{\pgfqpoint{2.956934in}{3.375489in}}%
\pgfpathmoveto{\pgfqpoint{2.956934in}{3.366641in}}%
\pgfpathlineto{\pgfqpoint{2.956934in}{3.366641in}}%
\pgfpathlineto{\pgfqpoint{2.956934in}{3.369590in}}%
\pgfpathlineto{\pgfqpoint{2.961475in}{3.369590in}}%
\pgfpathlineto{\pgfqpoint{2.961475in}{3.366641in}}%
\pgfpathmoveto{\pgfqpoint{2.956934in}{3.369590in}}%
\pgfpathlineto{\pgfqpoint{2.956934in}{3.369590in}}%
\pgfpathlineto{\pgfqpoint{2.956934in}{3.372540in}}%
\pgfpathlineto{\pgfqpoint{2.961475in}{3.372540in}}%
\pgfpathlineto{\pgfqpoint{2.961475in}{3.369590in}}%
\pgfpathmoveto{\pgfqpoint{2.979639in}{3.334199in}}%
\pgfpathlineto{\pgfqpoint{2.979639in}{3.334199in}}%
\pgfpathlineto{\pgfqpoint{2.979639in}{3.337148in}}%
\pgfpathlineto{\pgfqpoint{2.984180in}{3.337148in}}%
\pgfpathlineto{\pgfqpoint{2.984180in}{3.334199in}}%
\pgfpathmoveto{\pgfqpoint{2.975098in}{3.340097in}}%
\pgfpathlineto{\pgfqpoint{2.975098in}{3.340097in}}%
\pgfpathlineto{\pgfqpoint{2.975098in}{3.343047in}}%
\pgfpathlineto{\pgfqpoint{2.979639in}{3.343047in}}%
\pgfpathlineto{\pgfqpoint{2.979639in}{3.340097in}}%
\pgfpathmoveto{\pgfqpoint{2.979639in}{3.337148in}}%
\pgfpathlineto{\pgfqpoint{2.979639in}{3.337148in}}%
\pgfpathlineto{\pgfqpoint{2.979639in}{3.340097in}}%
\pgfpathlineto{\pgfqpoint{2.984180in}{3.340097in}}%
\pgfpathlineto{\pgfqpoint{2.984180in}{3.337148in}}%
\pgfpathmoveto{\pgfqpoint{2.979639in}{3.340097in}}%
\pgfpathlineto{\pgfqpoint{2.979639in}{3.340097in}}%
\pgfpathlineto{\pgfqpoint{2.979639in}{3.343047in}}%
\pgfpathlineto{\pgfqpoint{2.984180in}{3.343047in}}%
\pgfpathlineto{\pgfqpoint{2.984180in}{3.340097in}}%
\pgfpathmoveto{\pgfqpoint{2.970557in}{3.345996in}}%
\pgfpathlineto{\pgfqpoint{2.970557in}{3.345996in}}%
\pgfpathlineto{\pgfqpoint{2.970557in}{3.348945in}}%
\pgfpathlineto{\pgfqpoint{2.975098in}{3.348945in}}%
\pgfpathlineto{\pgfqpoint{2.975098in}{3.345996in}}%
\pgfpathmoveto{\pgfqpoint{2.966016in}{3.351894in}}%
\pgfpathlineto{\pgfqpoint{2.966016in}{3.351894in}}%
\pgfpathlineto{\pgfqpoint{2.966016in}{3.354844in}}%
\pgfpathlineto{\pgfqpoint{2.970557in}{3.354844in}}%
\pgfpathlineto{\pgfqpoint{2.970557in}{3.351894in}}%
\pgfpathmoveto{\pgfqpoint{2.970557in}{3.348945in}}%
\pgfpathlineto{\pgfqpoint{2.970557in}{3.348945in}}%
\pgfpathlineto{\pgfqpoint{2.970557in}{3.351894in}}%
\pgfpathlineto{\pgfqpoint{2.975098in}{3.351894in}}%
\pgfpathlineto{\pgfqpoint{2.975098in}{3.348945in}}%
\pgfpathmoveto{\pgfqpoint{2.970557in}{3.351894in}}%
\pgfpathlineto{\pgfqpoint{2.970557in}{3.351894in}}%
\pgfpathlineto{\pgfqpoint{2.970557in}{3.354844in}}%
\pgfpathlineto{\pgfqpoint{2.975098in}{3.354844in}}%
\pgfpathlineto{\pgfqpoint{2.975098in}{3.351894in}}%
\pgfpathmoveto{\pgfqpoint{2.975098in}{3.343047in}}%
\pgfpathlineto{\pgfqpoint{2.975098in}{3.343047in}}%
\pgfpathlineto{\pgfqpoint{2.975098in}{3.345996in}}%
\pgfpathlineto{\pgfqpoint{2.979639in}{3.345996in}}%
\pgfpathlineto{\pgfqpoint{2.979639in}{3.343047in}}%
\pgfpathmoveto{\pgfqpoint{2.975098in}{3.345996in}}%
\pgfpathlineto{\pgfqpoint{2.975098in}{3.345996in}}%
\pgfpathlineto{\pgfqpoint{2.975098in}{3.348945in}}%
\pgfpathlineto{\pgfqpoint{2.979639in}{3.348945in}}%
\pgfpathlineto{\pgfqpoint{2.979639in}{3.345996in}}%
\pgfpathmoveto{\pgfqpoint{2.984180in}{3.331249in}}%
\pgfpathlineto{\pgfqpoint{2.984180in}{3.331249in}}%
\pgfpathlineto{\pgfqpoint{2.984180in}{3.334199in}}%
\pgfpathlineto{\pgfqpoint{2.988722in}{3.334199in}}%
\pgfpathlineto{\pgfqpoint{2.988722in}{3.331249in}}%
\pgfpathmoveto{\pgfqpoint{2.984180in}{3.334199in}}%
\pgfpathlineto{\pgfqpoint{2.984180in}{3.334199in}}%
\pgfpathlineto{\pgfqpoint{2.984180in}{3.337148in}}%
\pgfpathlineto{\pgfqpoint{2.988722in}{3.337148in}}%
\pgfpathlineto{\pgfqpoint{2.988722in}{3.334199in}}%
\pgfpathmoveto{\pgfqpoint{2.966016in}{3.354844in}}%
\pgfpathlineto{\pgfqpoint{2.966016in}{3.354844in}}%
\pgfpathlineto{\pgfqpoint{2.966016in}{3.357793in}}%
\pgfpathlineto{\pgfqpoint{2.970557in}{3.357793in}}%
\pgfpathlineto{\pgfqpoint{2.970557in}{3.354844in}}%
\pgfpathmoveto{\pgfqpoint{2.966016in}{3.357793in}}%
\pgfpathlineto{\pgfqpoint{2.966016in}{3.357793in}}%
\pgfpathlineto{\pgfqpoint{2.966016in}{3.360742in}}%
\pgfpathlineto{\pgfqpoint{2.970557in}{3.360742in}}%
\pgfpathlineto{\pgfqpoint{2.970557in}{3.357793in}}%
\pgfpathmoveto{\pgfqpoint{2.943310in}{3.381387in}}%
\pgfpathlineto{\pgfqpoint{2.943310in}{3.381387in}}%
\pgfpathlineto{\pgfqpoint{2.943310in}{3.384337in}}%
\pgfpathlineto{\pgfqpoint{2.947852in}{3.384337in}}%
\pgfpathlineto{\pgfqpoint{2.947852in}{3.381387in}}%
\pgfpathmoveto{\pgfqpoint{2.938769in}{3.387286in}}%
\pgfpathlineto{\pgfqpoint{2.938769in}{3.387286in}}%
\pgfpathlineto{\pgfqpoint{2.938769in}{3.390235in}}%
\pgfpathlineto{\pgfqpoint{2.943310in}{3.390235in}}%
\pgfpathlineto{\pgfqpoint{2.943310in}{3.387286in}}%
\pgfpathmoveto{\pgfqpoint{2.943310in}{3.384337in}}%
\pgfpathlineto{\pgfqpoint{2.943310in}{3.384337in}}%
\pgfpathlineto{\pgfqpoint{2.943310in}{3.387286in}}%
\pgfpathlineto{\pgfqpoint{2.947852in}{3.387286in}}%
\pgfpathlineto{\pgfqpoint{2.947852in}{3.384337in}}%
\pgfpathmoveto{\pgfqpoint{2.943310in}{3.387286in}}%
\pgfpathlineto{\pgfqpoint{2.943310in}{3.387286in}}%
\pgfpathlineto{\pgfqpoint{2.943310in}{3.390235in}}%
\pgfpathlineto{\pgfqpoint{2.947852in}{3.390235in}}%
\pgfpathlineto{\pgfqpoint{2.947852in}{3.387286in}}%
\pgfpathmoveto{\pgfqpoint{2.934228in}{3.393185in}}%
\pgfpathlineto{\pgfqpoint{2.934228in}{3.393185in}}%
\pgfpathlineto{\pgfqpoint{2.934228in}{3.396134in}}%
\pgfpathlineto{\pgfqpoint{2.938769in}{3.396134in}}%
\pgfpathlineto{\pgfqpoint{2.938769in}{3.393185in}}%
\pgfpathmoveto{\pgfqpoint{2.929687in}{3.399083in}}%
\pgfpathlineto{\pgfqpoint{2.929687in}{3.399083in}}%
\pgfpathlineto{\pgfqpoint{2.929687in}{3.402033in}}%
\pgfpathlineto{\pgfqpoint{2.934228in}{3.402033in}}%
\pgfpathlineto{\pgfqpoint{2.934228in}{3.399083in}}%
\pgfpathmoveto{\pgfqpoint{2.934228in}{3.396134in}}%
\pgfpathlineto{\pgfqpoint{2.934228in}{3.396134in}}%
\pgfpathlineto{\pgfqpoint{2.934228in}{3.399083in}}%
\pgfpathlineto{\pgfqpoint{2.938769in}{3.399083in}}%
\pgfpathlineto{\pgfqpoint{2.938769in}{3.396134in}}%
\pgfpathmoveto{\pgfqpoint{2.934228in}{3.399083in}}%
\pgfpathlineto{\pgfqpoint{2.934228in}{3.399083in}}%
\pgfpathlineto{\pgfqpoint{2.934228in}{3.402033in}}%
\pgfpathlineto{\pgfqpoint{2.938769in}{3.402033in}}%
\pgfpathlineto{\pgfqpoint{2.938769in}{3.399083in}}%
\pgfpathmoveto{\pgfqpoint{2.938769in}{3.390235in}}%
\pgfpathlineto{\pgfqpoint{2.938769in}{3.390235in}}%
\pgfpathlineto{\pgfqpoint{2.938769in}{3.393185in}}%
\pgfpathlineto{\pgfqpoint{2.943310in}{3.393185in}}%
\pgfpathlineto{\pgfqpoint{2.943310in}{3.390235in}}%
\pgfpathmoveto{\pgfqpoint{2.938769in}{3.393185in}}%
\pgfpathlineto{\pgfqpoint{2.938769in}{3.393185in}}%
\pgfpathlineto{\pgfqpoint{2.938769in}{3.396134in}}%
\pgfpathlineto{\pgfqpoint{2.943310in}{3.396134in}}%
\pgfpathlineto{\pgfqpoint{2.943310in}{3.393185in}}%
\pgfpathmoveto{\pgfqpoint{2.947852in}{3.378438in}}%
\pgfpathlineto{\pgfqpoint{2.947852in}{3.378438in}}%
\pgfpathlineto{\pgfqpoint{2.947852in}{3.381387in}}%
\pgfpathlineto{\pgfqpoint{2.952393in}{3.381387in}}%
\pgfpathlineto{\pgfqpoint{2.952393in}{3.378438in}}%
\pgfpathmoveto{\pgfqpoint{2.947852in}{3.381387in}}%
\pgfpathlineto{\pgfqpoint{2.947852in}{3.381387in}}%
\pgfpathlineto{\pgfqpoint{2.947852in}{3.384337in}}%
\pgfpathlineto{\pgfqpoint{2.952393in}{3.384337in}}%
\pgfpathlineto{\pgfqpoint{2.952393in}{3.381387in}}%
\pgfpathmoveto{\pgfqpoint{2.929687in}{3.402033in}}%
\pgfpathlineto{\pgfqpoint{2.929687in}{3.402033in}}%
\pgfpathlineto{\pgfqpoint{2.929687in}{3.404982in}}%
\pgfpathlineto{\pgfqpoint{2.934228in}{3.404982in}}%
\pgfpathlineto{\pgfqpoint{2.934228in}{3.402033in}}%
\pgfpathmoveto{\pgfqpoint{2.929687in}{3.404982in}}%
\pgfpathlineto{\pgfqpoint{2.929687in}{3.404982in}}%
\pgfpathlineto{\pgfqpoint{2.929687in}{3.407931in}}%
\pgfpathlineto{\pgfqpoint{2.934228in}{3.407931in}}%
\pgfpathlineto{\pgfqpoint{2.934228in}{3.404982in}}%
\pgfpathmoveto{\pgfqpoint{3.215772in}{3.027482in}}%
\pgfpathlineto{\pgfqpoint{3.215772in}{3.027482in}}%
\pgfpathlineto{\pgfqpoint{3.215772in}{3.030431in}}%
\pgfpathlineto{\pgfqpoint{3.220313in}{3.030431in}}%
\pgfpathlineto{\pgfqpoint{3.220313in}{3.027482in}}%
\pgfpathmoveto{\pgfqpoint{3.211231in}{3.033380in}}%
\pgfpathlineto{\pgfqpoint{3.211231in}{3.033380in}}%
\pgfpathlineto{\pgfqpoint{3.211231in}{3.036329in}}%
\pgfpathlineto{\pgfqpoint{3.215772in}{3.036329in}}%
\pgfpathlineto{\pgfqpoint{3.215772in}{3.033380in}}%
\pgfpathmoveto{\pgfqpoint{3.215772in}{3.030431in}}%
\pgfpathlineto{\pgfqpoint{3.215772in}{3.030431in}}%
\pgfpathlineto{\pgfqpoint{3.215772in}{3.033380in}}%
\pgfpathlineto{\pgfqpoint{3.220313in}{3.033380in}}%
\pgfpathlineto{\pgfqpoint{3.220313in}{3.030431in}}%
\pgfpathmoveto{\pgfqpoint{3.215772in}{3.033380in}}%
\pgfpathlineto{\pgfqpoint{3.215772in}{3.033380in}}%
\pgfpathlineto{\pgfqpoint{3.215772in}{3.036329in}}%
\pgfpathlineto{\pgfqpoint{3.220313in}{3.036329in}}%
\pgfpathlineto{\pgfqpoint{3.220313in}{3.033380in}}%
\pgfpathmoveto{\pgfqpoint{3.206690in}{3.039279in}}%
\pgfpathlineto{\pgfqpoint{3.206690in}{3.039279in}}%
\pgfpathlineto{\pgfqpoint{3.206690in}{3.042228in}}%
\pgfpathlineto{\pgfqpoint{3.211231in}{3.042228in}}%
\pgfpathlineto{\pgfqpoint{3.211231in}{3.039279in}}%
\pgfpathmoveto{\pgfqpoint{3.202149in}{3.045177in}}%
\pgfpathlineto{\pgfqpoint{3.202149in}{3.045177in}}%
\pgfpathlineto{\pgfqpoint{3.202149in}{3.048126in}}%
\pgfpathlineto{\pgfqpoint{3.206690in}{3.048126in}}%
\pgfpathlineto{\pgfqpoint{3.206690in}{3.045177in}}%
\pgfpathmoveto{\pgfqpoint{3.206690in}{3.042228in}}%
\pgfpathlineto{\pgfqpoint{3.206690in}{3.042228in}}%
\pgfpathlineto{\pgfqpoint{3.206690in}{3.045177in}}%
\pgfpathlineto{\pgfqpoint{3.211231in}{3.045177in}}%
\pgfpathlineto{\pgfqpoint{3.211231in}{3.042228in}}%
\pgfpathmoveto{\pgfqpoint{3.206690in}{3.045177in}}%
\pgfpathlineto{\pgfqpoint{3.206690in}{3.045177in}}%
\pgfpathlineto{\pgfqpoint{3.206690in}{3.048126in}}%
\pgfpathlineto{\pgfqpoint{3.211231in}{3.048126in}}%
\pgfpathlineto{\pgfqpoint{3.211231in}{3.045177in}}%
\pgfpathmoveto{\pgfqpoint{3.211231in}{3.036329in}}%
\pgfpathlineto{\pgfqpoint{3.211231in}{3.036329in}}%
\pgfpathlineto{\pgfqpoint{3.211231in}{3.039279in}}%
\pgfpathlineto{\pgfqpoint{3.215772in}{3.039279in}}%
\pgfpathlineto{\pgfqpoint{3.215772in}{3.036329in}}%
\pgfpathmoveto{\pgfqpoint{3.211231in}{3.039279in}}%
\pgfpathlineto{\pgfqpoint{3.211231in}{3.039279in}}%
\pgfpathlineto{\pgfqpoint{3.211231in}{3.042228in}}%
\pgfpathlineto{\pgfqpoint{3.215772in}{3.042228in}}%
\pgfpathlineto{\pgfqpoint{3.215772in}{3.039279in}}%
\pgfpathmoveto{\pgfqpoint{3.143117in}{3.121855in}}%
\pgfpathlineto{\pgfqpoint{3.143117in}{3.121855in}}%
\pgfpathlineto{\pgfqpoint{3.143117in}{3.124804in}}%
\pgfpathlineto{\pgfqpoint{3.147658in}{3.124804in}}%
\pgfpathlineto{\pgfqpoint{3.147658in}{3.121855in}}%
\pgfpathmoveto{\pgfqpoint{3.138576in}{3.127753in}}%
\pgfpathlineto{\pgfqpoint{3.138576in}{3.127753in}}%
\pgfpathlineto{\pgfqpoint{3.138576in}{3.130702in}}%
\pgfpathlineto{\pgfqpoint{3.143117in}{3.130702in}}%
\pgfpathlineto{\pgfqpoint{3.143117in}{3.127753in}}%
\pgfpathmoveto{\pgfqpoint{3.143117in}{3.124804in}}%
\pgfpathlineto{\pgfqpoint{3.143117in}{3.124804in}}%
\pgfpathlineto{\pgfqpoint{3.143117in}{3.127753in}}%
\pgfpathlineto{\pgfqpoint{3.147658in}{3.127753in}}%
\pgfpathlineto{\pgfqpoint{3.147658in}{3.124804in}}%
\pgfpathmoveto{\pgfqpoint{3.143117in}{3.127753in}}%
\pgfpathlineto{\pgfqpoint{3.143117in}{3.127753in}}%
\pgfpathlineto{\pgfqpoint{3.143117in}{3.130702in}}%
\pgfpathlineto{\pgfqpoint{3.147658in}{3.130702in}}%
\pgfpathlineto{\pgfqpoint{3.147658in}{3.127753in}}%
\pgfpathmoveto{\pgfqpoint{3.134035in}{3.133651in}}%
\pgfpathlineto{\pgfqpoint{3.134035in}{3.133651in}}%
\pgfpathlineto{\pgfqpoint{3.134035in}{3.136601in}}%
\pgfpathlineto{\pgfqpoint{3.138576in}{3.136601in}}%
\pgfpathlineto{\pgfqpoint{3.138576in}{3.133651in}}%
\pgfpathmoveto{\pgfqpoint{3.129494in}{3.139550in}}%
\pgfpathlineto{\pgfqpoint{3.129494in}{3.139550in}}%
\pgfpathlineto{\pgfqpoint{3.129494in}{3.142499in}}%
\pgfpathlineto{\pgfqpoint{3.134035in}{3.142499in}}%
\pgfpathlineto{\pgfqpoint{3.134035in}{3.139550in}}%
\pgfpathmoveto{\pgfqpoint{3.134035in}{3.136601in}}%
\pgfpathlineto{\pgfqpoint{3.134035in}{3.136601in}}%
\pgfpathlineto{\pgfqpoint{3.134035in}{3.139550in}}%
\pgfpathlineto{\pgfqpoint{3.138576in}{3.139550in}}%
\pgfpathlineto{\pgfqpoint{3.138576in}{3.136601in}}%
\pgfpathmoveto{\pgfqpoint{3.134035in}{3.139550in}}%
\pgfpathlineto{\pgfqpoint{3.134035in}{3.139550in}}%
\pgfpathlineto{\pgfqpoint{3.134035in}{3.142499in}}%
\pgfpathlineto{\pgfqpoint{3.138576in}{3.142499in}}%
\pgfpathlineto{\pgfqpoint{3.138576in}{3.139550in}}%
\pgfpathmoveto{\pgfqpoint{3.138576in}{3.130702in}}%
\pgfpathlineto{\pgfqpoint{3.138576in}{3.130702in}}%
\pgfpathlineto{\pgfqpoint{3.138576in}{3.133651in}}%
\pgfpathlineto{\pgfqpoint{3.143117in}{3.133651in}}%
\pgfpathlineto{\pgfqpoint{3.143117in}{3.130702in}}%
\pgfpathmoveto{\pgfqpoint{3.138576in}{3.133651in}}%
\pgfpathlineto{\pgfqpoint{3.138576in}{3.133651in}}%
\pgfpathlineto{\pgfqpoint{3.138576in}{3.136601in}}%
\pgfpathlineto{\pgfqpoint{3.143117in}{3.136601in}}%
\pgfpathlineto{\pgfqpoint{3.143117in}{3.133651in}}%
\pgfpathmoveto{\pgfqpoint{3.179445in}{3.074669in}}%
\pgfpathlineto{\pgfqpoint{3.179445in}{3.074669in}}%
\pgfpathlineto{\pgfqpoint{3.179445in}{3.077618in}}%
\pgfpathlineto{\pgfqpoint{3.183986in}{3.077618in}}%
\pgfpathlineto{\pgfqpoint{3.183986in}{3.074669in}}%
\pgfpathmoveto{\pgfqpoint{3.174904in}{3.080567in}}%
\pgfpathlineto{\pgfqpoint{3.174904in}{3.080567in}}%
\pgfpathlineto{\pgfqpoint{3.174904in}{3.083516in}}%
\pgfpathlineto{\pgfqpoint{3.179445in}{3.083516in}}%
\pgfpathlineto{\pgfqpoint{3.179445in}{3.080567in}}%
\pgfpathmoveto{\pgfqpoint{3.179445in}{3.077618in}}%
\pgfpathlineto{\pgfqpoint{3.179445in}{3.077618in}}%
\pgfpathlineto{\pgfqpoint{3.179445in}{3.080567in}}%
\pgfpathlineto{\pgfqpoint{3.183986in}{3.080567in}}%
\pgfpathlineto{\pgfqpoint{3.183986in}{3.077618in}}%
\pgfpathmoveto{\pgfqpoint{3.179445in}{3.080567in}}%
\pgfpathlineto{\pgfqpoint{3.179445in}{3.080567in}}%
\pgfpathlineto{\pgfqpoint{3.179445in}{3.083516in}}%
\pgfpathlineto{\pgfqpoint{3.183986in}{3.083516in}}%
\pgfpathlineto{\pgfqpoint{3.183986in}{3.080567in}}%
\pgfpathmoveto{\pgfqpoint{3.170363in}{3.086465in}}%
\pgfpathlineto{\pgfqpoint{3.170363in}{3.086465in}}%
\pgfpathlineto{\pgfqpoint{3.170363in}{3.089414in}}%
\pgfpathlineto{\pgfqpoint{3.174904in}{3.089414in}}%
\pgfpathlineto{\pgfqpoint{3.174904in}{3.086465in}}%
\pgfpathmoveto{\pgfqpoint{3.165822in}{3.092363in}}%
\pgfpathlineto{\pgfqpoint{3.165822in}{3.092363in}}%
\pgfpathlineto{\pgfqpoint{3.165822in}{3.095313in}}%
\pgfpathlineto{\pgfqpoint{3.170363in}{3.095313in}}%
\pgfpathlineto{\pgfqpoint{3.170363in}{3.092363in}}%
\pgfpathmoveto{\pgfqpoint{3.170363in}{3.089414in}}%
\pgfpathlineto{\pgfqpoint{3.170363in}{3.089414in}}%
\pgfpathlineto{\pgfqpoint{3.170363in}{3.092363in}}%
\pgfpathlineto{\pgfqpoint{3.174904in}{3.092363in}}%
\pgfpathlineto{\pgfqpoint{3.174904in}{3.089414in}}%
\pgfpathmoveto{\pgfqpoint{3.170363in}{3.092363in}}%
\pgfpathlineto{\pgfqpoint{3.170363in}{3.092363in}}%
\pgfpathlineto{\pgfqpoint{3.170363in}{3.095313in}}%
\pgfpathlineto{\pgfqpoint{3.174904in}{3.095313in}}%
\pgfpathlineto{\pgfqpoint{3.174904in}{3.092363in}}%
\pgfpathmoveto{\pgfqpoint{3.174904in}{3.083516in}}%
\pgfpathlineto{\pgfqpoint{3.174904in}{3.083516in}}%
\pgfpathlineto{\pgfqpoint{3.174904in}{3.086465in}}%
\pgfpathlineto{\pgfqpoint{3.179445in}{3.086465in}}%
\pgfpathlineto{\pgfqpoint{3.179445in}{3.083516in}}%
\pgfpathmoveto{\pgfqpoint{3.174904in}{3.086465in}}%
\pgfpathlineto{\pgfqpoint{3.174904in}{3.086465in}}%
\pgfpathlineto{\pgfqpoint{3.174904in}{3.089414in}}%
\pgfpathlineto{\pgfqpoint{3.179445in}{3.089414in}}%
\pgfpathlineto{\pgfqpoint{3.179445in}{3.086465in}}%
\pgfpathmoveto{\pgfqpoint{3.197608in}{3.051075in}}%
\pgfpathlineto{\pgfqpoint{3.197608in}{3.051075in}}%
\pgfpathlineto{\pgfqpoint{3.197608in}{3.054025in}}%
\pgfpathlineto{\pgfqpoint{3.202149in}{3.054025in}}%
\pgfpathlineto{\pgfqpoint{3.202149in}{3.051075in}}%
\pgfpathmoveto{\pgfqpoint{3.193067in}{3.056974in}}%
\pgfpathlineto{\pgfqpoint{3.193067in}{3.056974in}}%
\pgfpathlineto{\pgfqpoint{3.193067in}{3.059923in}}%
\pgfpathlineto{\pgfqpoint{3.197608in}{3.059923in}}%
\pgfpathlineto{\pgfqpoint{3.197608in}{3.056974in}}%
\pgfpathmoveto{\pgfqpoint{3.197608in}{3.054025in}}%
\pgfpathlineto{\pgfqpoint{3.197608in}{3.054025in}}%
\pgfpathlineto{\pgfqpoint{3.197608in}{3.056974in}}%
\pgfpathlineto{\pgfqpoint{3.202149in}{3.056974in}}%
\pgfpathlineto{\pgfqpoint{3.202149in}{3.054025in}}%
\pgfpathmoveto{\pgfqpoint{3.197608in}{3.056974in}}%
\pgfpathlineto{\pgfqpoint{3.197608in}{3.056974in}}%
\pgfpathlineto{\pgfqpoint{3.197608in}{3.059923in}}%
\pgfpathlineto{\pgfqpoint{3.202149in}{3.059923in}}%
\pgfpathlineto{\pgfqpoint{3.202149in}{3.056974in}}%
\pgfpathmoveto{\pgfqpoint{3.188527in}{3.062872in}}%
\pgfpathlineto{\pgfqpoint{3.188527in}{3.062872in}}%
\pgfpathlineto{\pgfqpoint{3.188527in}{3.065821in}}%
\pgfpathlineto{\pgfqpoint{3.193067in}{3.065821in}}%
\pgfpathlineto{\pgfqpoint{3.193067in}{3.062872in}}%
\pgfpathmoveto{\pgfqpoint{3.183986in}{3.068770in}}%
\pgfpathlineto{\pgfqpoint{3.183986in}{3.068770in}}%
\pgfpathlineto{\pgfqpoint{3.183986in}{3.071719in}}%
\pgfpathlineto{\pgfqpoint{3.188527in}{3.071719in}}%
\pgfpathlineto{\pgfqpoint{3.188527in}{3.068770in}}%
\pgfpathmoveto{\pgfqpoint{3.188527in}{3.065821in}}%
\pgfpathlineto{\pgfqpoint{3.188527in}{3.065821in}}%
\pgfpathlineto{\pgfqpoint{3.188527in}{3.068770in}}%
\pgfpathlineto{\pgfqpoint{3.193067in}{3.068770in}}%
\pgfpathlineto{\pgfqpoint{3.193067in}{3.065821in}}%
\pgfpathmoveto{\pgfqpoint{3.188527in}{3.068770in}}%
\pgfpathlineto{\pgfqpoint{3.188527in}{3.068770in}}%
\pgfpathlineto{\pgfqpoint{3.188527in}{3.071719in}}%
\pgfpathlineto{\pgfqpoint{3.193067in}{3.071719in}}%
\pgfpathlineto{\pgfqpoint{3.193067in}{3.068770in}}%
\pgfpathmoveto{\pgfqpoint{3.193067in}{3.059923in}}%
\pgfpathlineto{\pgfqpoint{3.193067in}{3.059923in}}%
\pgfpathlineto{\pgfqpoint{3.193067in}{3.062872in}}%
\pgfpathlineto{\pgfqpoint{3.197608in}{3.062872in}}%
\pgfpathlineto{\pgfqpoint{3.197608in}{3.059923in}}%
\pgfpathmoveto{\pgfqpoint{3.193067in}{3.062872in}}%
\pgfpathlineto{\pgfqpoint{3.193067in}{3.062872in}}%
\pgfpathlineto{\pgfqpoint{3.193067in}{3.065821in}}%
\pgfpathlineto{\pgfqpoint{3.197608in}{3.065821in}}%
\pgfpathlineto{\pgfqpoint{3.197608in}{3.062872in}}%
\pgfpathmoveto{\pgfqpoint{3.202149in}{3.048126in}}%
\pgfpathlineto{\pgfqpoint{3.202149in}{3.048126in}}%
\pgfpathlineto{\pgfqpoint{3.202149in}{3.051075in}}%
\pgfpathlineto{\pgfqpoint{3.206690in}{3.051075in}}%
\pgfpathlineto{\pgfqpoint{3.206690in}{3.048126in}}%
\pgfpathmoveto{\pgfqpoint{3.202149in}{3.051075in}}%
\pgfpathlineto{\pgfqpoint{3.202149in}{3.051075in}}%
\pgfpathlineto{\pgfqpoint{3.202149in}{3.054025in}}%
\pgfpathlineto{\pgfqpoint{3.206690in}{3.054025in}}%
\pgfpathlineto{\pgfqpoint{3.206690in}{3.051075in}}%
\pgfpathmoveto{\pgfqpoint{3.183986in}{3.071719in}}%
\pgfpathlineto{\pgfqpoint{3.183986in}{3.071719in}}%
\pgfpathlineto{\pgfqpoint{3.183986in}{3.074669in}}%
\pgfpathlineto{\pgfqpoint{3.188527in}{3.074669in}}%
\pgfpathlineto{\pgfqpoint{3.188527in}{3.071719in}}%
\pgfpathmoveto{\pgfqpoint{3.183986in}{3.074669in}}%
\pgfpathlineto{\pgfqpoint{3.183986in}{3.074669in}}%
\pgfpathlineto{\pgfqpoint{3.183986in}{3.077618in}}%
\pgfpathlineto{\pgfqpoint{3.188527in}{3.077618in}}%
\pgfpathlineto{\pgfqpoint{3.188527in}{3.074669in}}%
\pgfpathmoveto{\pgfqpoint{3.161281in}{3.098262in}}%
\pgfpathlineto{\pgfqpoint{3.161281in}{3.098262in}}%
\pgfpathlineto{\pgfqpoint{3.161281in}{3.101211in}}%
\pgfpathlineto{\pgfqpoint{3.165822in}{3.101211in}}%
\pgfpathlineto{\pgfqpoint{3.165822in}{3.098262in}}%
\pgfpathmoveto{\pgfqpoint{3.156740in}{3.104160in}}%
\pgfpathlineto{\pgfqpoint{3.156740in}{3.104160in}}%
\pgfpathlineto{\pgfqpoint{3.156740in}{3.107109in}}%
\pgfpathlineto{\pgfqpoint{3.161281in}{3.107109in}}%
\pgfpathlineto{\pgfqpoint{3.161281in}{3.104160in}}%
\pgfpathmoveto{\pgfqpoint{3.161281in}{3.101211in}}%
\pgfpathlineto{\pgfqpoint{3.161281in}{3.101211in}}%
\pgfpathlineto{\pgfqpoint{3.161281in}{3.104160in}}%
\pgfpathlineto{\pgfqpoint{3.165822in}{3.104160in}}%
\pgfpathlineto{\pgfqpoint{3.165822in}{3.101211in}}%
\pgfpathmoveto{\pgfqpoint{3.161281in}{3.104160in}}%
\pgfpathlineto{\pgfqpoint{3.161281in}{3.104160in}}%
\pgfpathlineto{\pgfqpoint{3.161281in}{3.107109in}}%
\pgfpathlineto{\pgfqpoint{3.165822in}{3.107109in}}%
\pgfpathlineto{\pgfqpoint{3.165822in}{3.104160in}}%
\pgfpathmoveto{\pgfqpoint{3.152199in}{3.110058in}}%
\pgfpathlineto{\pgfqpoint{3.152199in}{3.110058in}}%
\pgfpathlineto{\pgfqpoint{3.152199in}{3.113007in}}%
\pgfpathlineto{\pgfqpoint{3.156740in}{3.113007in}}%
\pgfpathlineto{\pgfqpoint{3.156740in}{3.110058in}}%
\pgfpathmoveto{\pgfqpoint{3.147658in}{3.115957in}}%
\pgfpathlineto{\pgfqpoint{3.147658in}{3.115957in}}%
\pgfpathlineto{\pgfqpoint{3.147658in}{3.118906in}}%
\pgfpathlineto{\pgfqpoint{3.152199in}{3.118906in}}%
\pgfpathlineto{\pgfqpoint{3.152199in}{3.115957in}}%
\pgfpathmoveto{\pgfqpoint{3.152199in}{3.113007in}}%
\pgfpathlineto{\pgfqpoint{3.152199in}{3.113007in}}%
\pgfpathlineto{\pgfqpoint{3.152199in}{3.115957in}}%
\pgfpathlineto{\pgfqpoint{3.156740in}{3.115957in}}%
\pgfpathlineto{\pgfqpoint{3.156740in}{3.113007in}}%
\pgfpathmoveto{\pgfqpoint{3.152199in}{3.115957in}}%
\pgfpathlineto{\pgfqpoint{3.152199in}{3.115957in}}%
\pgfpathlineto{\pgfqpoint{3.152199in}{3.118906in}}%
\pgfpathlineto{\pgfqpoint{3.156740in}{3.118906in}}%
\pgfpathlineto{\pgfqpoint{3.156740in}{3.115957in}}%
\pgfpathmoveto{\pgfqpoint{3.156740in}{3.107109in}}%
\pgfpathlineto{\pgfqpoint{3.156740in}{3.107109in}}%
\pgfpathlineto{\pgfqpoint{3.156740in}{3.110058in}}%
\pgfpathlineto{\pgfqpoint{3.161281in}{3.110058in}}%
\pgfpathlineto{\pgfqpoint{3.161281in}{3.107109in}}%
\pgfpathmoveto{\pgfqpoint{3.156740in}{3.110058in}}%
\pgfpathlineto{\pgfqpoint{3.156740in}{3.110058in}}%
\pgfpathlineto{\pgfqpoint{3.156740in}{3.113007in}}%
\pgfpathlineto{\pgfqpoint{3.161281in}{3.113007in}}%
\pgfpathlineto{\pgfqpoint{3.161281in}{3.110058in}}%
\pgfpathmoveto{\pgfqpoint{3.165822in}{3.095313in}}%
\pgfpathlineto{\pgfqpoint{3.165822in}{3.095313in}}%
\pgfpathlineto{\pgfqpoint{3.165822in}{3.098262in}}%
\pgfpathlineto{\pgfqpoint{3.170363in}{3.098262in}}%
\pgfpathlineto{\pgfqpoint{3.170363in}{3.095313in}}%
\pgfpathmoveto{\pgfqpoint{3.165822in}{3.098262in}}%
\pgfpathlineto{\pgfqpoint{3.165822in}{3.098262in}}%
\pgfpathlineto{\pgfqpoint{3.165822in}{3.101211in}}%
\pgfpathlineto{\pgfqpoint{3.170363in}{3.101211in}}%
\pgfpathlineto{\pgfqpoint{3.170363in}{3.098262in}}%
\pgfpathmoveto{\pgfqpoint{3.147658in}{3.118906in}}%
\pgfpathlineto{\pgfqpoint{3.147658in}{3.118906in}}%
\pgfpathlineto{\pgfqpoint{3.147658in}{3.121855in}}%
\pgfpathlineto{\pgfqpoint{3.152199in}{3.121855in}}%
\pgfpathlineto{\pgfqpoint{3.152199in}{3.118906in}}%
\pgfpathmoveto{\pgfqpoint{3.147658in}{3.121855in}}%
\pgfpathlineto{\pgfqpoint{3.147658in}{3.121855in}}%
\pgfpathlineto{\pgfqpoint{3.147658in}{3.124804in}}%
\pgfpathlineto{\pgfqpoint{3.152199in}{3.124804in}}%
\pgfpathlineto{\pgfqpoint{3.152199in}{3.121855in}}%
\pgfpathmoveto{\pgfqpoint{3.106789in}{3.169042in}}%
\pgfpathlineto{\pgfqpoint{3.106789in}{3.169042in}}%
\pgfpathlineto{\pgfqpoint{3.106789in}{3.171991in}}%
\pgfpathlineto{\pgfqpoint{3.111330in}{3.171991in}}%
\pgfpathlineto{\pgfqpoint{3.111330in}{3.169042in}}%
\pgfpathmoveto{\pgfqpoint{3.102248in}{3.174940in}}%
\pgfpathlineto{\pgfqpoint{3.102248in}{3.174940in}}%
\pgfpathlineto{\pgfqpoint{3.102248in}{3.177890in}}%
\pgfpathlineto{\pgfqpoint{3.106789in}{3.177890in}}%
\pgfpathlineto{\pgfqpoint{3.106789in}{3.174940in}}%
\pgfpathmoveto{\pgfqpoint{3.106789in}{3.171991in}}%
\pgfpathlineto{\pgfqpoint{3.106789in}{3.171991in}}%
\pgfpathlineto{\pgfqpoint{3.106789in}{3.174940in}}%
\pgfpathlineto{\pgfqpoint{3.111330in}{3.174940in}}%
\pgfpathlineto{\pgfqpoint{3.111330in}{3.171991in}}%
\pgfpathmoveto{\pgfqpoint{3.106789in}{3.174940in}}%
\pgfpathlineto{\pgfqpoint{3.106789in}{3.174940in}}%
\pgfpathlineto{\pgfqpoint{3.106789in}{3.177890in}}%
\pgfpathlineto{\pgfqpoint{3.111330in}{3.177890in}}%
\pgfpathlineto{\pgfqpoint{3.111330in}{3.174940in}}%
\pgfpathmoveto{\pgfqpoint{3.097707in}{3.180839in}}%
\pgfpathlineto{\pgfqpoint{3.097707in}{3.180839in}}%
\pgfpathlineto{\pgfqpoint{3.097707in}{3.183788in}}%
\pgfpathlineto{\pgfqpoint{3.102248in}{3.183788in}}%
\pgfpathlineto{\pgfqpoint{3.102248in}{3.180839in}}%
\pgfpathmoveto{\pgfqpoint{3.093166in}{3.186737in}}%
\pgfpathlineto{\pgfqpoint{3.093166in}{3.186737in}}%
\pgfpathlineto{\pgfqpoint{3.093166in}{3.189687in}}%
\pgfpathlineto{\pgfqpoint{3.097707in}{3.189687in}}%
\pgfpathlineto{\pgfqpoint{3.097707in}{3.186737in}}%
\pgfpathmoveto{\pgfqpoint{3.097707in}{3.183788in}}%
\pgfpathlineto{\pgfqpoint{3.097707in}{3.183788in}}%
\pgfpathlineto{\pgfqpoint{3.097707in}{3.186737in}}%
\pgfpathlineto{\pgfqpoint{3.102248in}{3.186737in}}%
\pgfpathlineto{\pgfqpoint{3.102248in}{3.183788in}}%
\pgfpathmoveto{\pgfqpoint{3.097707in}{3.186737in}}%
\pgfpathlineto{\pgfqpoint{3.097707in}{3.186737in}}%
\pgfpathlineto{\pgfqpoint{3.097707in}{3.189687in}}%
\pgfpathlineto{\pgfqpoint{3.102248in}{3.189687in}}%
\pgfpathlineto{\pgfqpoint{3.102248in}{3.186737in}}%
\pgfpathmoveto{\pgfqpoint{3.102248in}{3.177890in}}%
\pgfpathlineto{\pgfqpoint{3.102248in}{3.177890in}}%
\pgfpathlineto{\pgfqpoint{3.102248in}{3.180839in}}%
\pgfpathlineto{\pgfqpoint{3.106789in}{3.180839in}}%
\pgfpathlineto{\pgfqpoint{3.106789in}{3.177890in}}%
\pgfpathmoveto{\pgfqpoint{3.102248in}{3.180839in}}%
\pgfpathlineto{\pgfqpoint{3.102248in}{3.180839in}}%
\pgfpathlineto{\pgfqpoint{3.102248in}{3.183788in}}%
\pgfpathlineto{\pgfqpoint{3.106789in}{3.183788in}}%
\pgfpathlineto{\pgfqpoint{3.106789in}{3.180839in}}%
\pgfpathmoveto{\pgfqpoint{3.124953in}{3.145448in}}%
\pgfpathlineto{\pgfqpoint{3.124953in}{3.145448in}}%
\pgfpathlineto{\pgfqpoint{3.124953in}{3.148397in}}%
\pgfpathlineto{\pgfqpoint{3.129494in}{3.148397in}}%
\pgfpathlineto{\pgfqpoint{3.129494in}{3.145448in}}%
\pgfpathmoveto{\pgfqpoint{3.120412in}{3.151347in}}%
\pgfpathlineto{\pgfqpoint{3.120412in}{3.151347in}}%
\pgfpathlineto{\pgfqpoint{3.120412in}{3.154296in}}%
\pgfpathlineto{\pgfqpoint{3.124953in}{3.154296in}}%
\pgfpathlineto{\pgfqpoint{3.124953in}{3.151347in}}%
\pgfpathmoveto{\pgfqpoint{3.124953in}{3.148397in}}%
\pgfpathlineto{\pgfqpoint{3.124953in}{3.148397in}}%
\pgfpathlineto{\pgfqpoint{3.124953in}{3.151347in}}%
\pgfpathlineto{\pgfqpoint{3.129494in}{3.151347in}}%
\pgfpathlineto{\pgfqpoint{3.129494in}{3.148397in}}%
\pgfpathmoveto{\pgfqpoint{3.124953in}{3.151347in}}%
\pgfpathlineto{\pgfqpoint{3.124953in}{3.151347in}}%
\pgfpathlineto{\pgfqpoint{3.124953in}{3.154296in}}%
\pgfpathlineto{\pgfqpoint{3.129494in}{3.154296in}}%
\pgfpathlineto{\pgfqpoint{3.129494in}{3.151347in}}%
\pgfpathmoveto{\pgfqpoint{3.115871in}{3.157245in}}%
\pgfpathlineto{\pgfqpoint{3.115871in}{3.157245in}}%
\pgfpathlineto{\pgfqpoint{3.115871in}{3.160194in}}%
\pgfpathlineto{\pgfqpoint{3.120412in}{3.160194in}}%
\pgfpathlineto{\pgfqpoint{3.120412in}{3.157245in}}%
\pgfpathmoveto{\pgfqpoint{3.111330in}{3.163143in}}%
\pgfpathlineto{\pgfqpoint{3.111330in}{3.163143in}}%
\pgfpathlineto{\pgfqpoint{3.111330in}{3.166093in}}%
\pgfpathlineto{\pgfqpoint{3.115871in}{3.166093in}}%
\pgfpathlineto{\pgfqpoint{3.115871in}{3.163143in}}%
\pgfpathmoveto{\pgfqpoint{3.115871in}{3.160194in}}%
\pgfpathlineto{\pgfqpoint{3.115871in}{3.160194in}}%
\pgfpathlineto{\pgfqpoint{3.115871in}{3.163143in}}%
\pgfpathlineto{\pgfqpoint{3.120412in}{3.163143in}}%
\pgfpathlineto{\pgfqpoint{3.120412in}{3.160194in}}%
\pgfpathmoveto{\pgfqpoint{3.115871in}{3.163143in}}%
\pgfpathlineto{\pgfqpoint{3.115871in}{3.163143in}}%
\pgfpathlineto{\pgfqpoint{3.115871in}{3.166093in}}%
\pgfpathlineto{\pgfqpoint{3.120412in}{3.166093in}}%
\pgfpathlineto{\pgfqpoint{3.120412in}{3.163143in}}%
\pgfpathmoveto{\pgfqpoint{3.120412in}{3.154296in}}%
\pgfpathlineto{\pgfqpoint{3.120412in}{3.154296in}}%
\pgfpathlineto{\pgfqpoint{3.120412in}{3.157245in}}%
\pgfpathlineto{\pgfqpoint{3.124953in}{3.157245in}}%
\pgfpathlineto{\pgfqpoint{3.124953in}{3.154296in}}%
\pgfpathmoveto{\pgfqpoint{3.120412in}{3.157245in}}%
\pgfpathlineto{\pgfqpoint{3.120412in}{3.157245in}}%
\pgfpathlineto{\pgfqpoint{3.120412in}{3.160194in}}%
\pgfpathlineto{\pgfqpoint{3.124953in}{3.160194in}}%
\pgfpathlineto{\pgfqpoint{3.124953in}{3.157245in}}%
\pgfpathmoveto{\pgfqpoint{3.129494in}{3.142499in}}%
\pgfpathlineto{\pgfqpoint{3.129494in}{3.142499in}}%
\pgfpathlineto{\pgfqpoint{3.129494in}{3.145448in}}%
\pgfpathlineto{\pgfqpoint{3.134035in}{3.145448in}}%
\pgfpathlineto{\pgfqpoint{3.134035in}{3.142499in}}%
\pgfpathmoveto{\pgfqpoint{3.129494in}{3.145448in}}%
\pgfpathlineto{\pgfqpoint{3.129494in}{3.145448in}}%
\pgfpathlineto{\pgfqpoint{3.129494in}{3.148397in}}%
\pgfpathlineto{\pgfqpoint{3.134035in}{3.148397in}}%
\pgfpathlineto{\pgfqpoint{3.134035in}{3.145448in}}%
\pgfpathmoveto{\pgfqpoint{3.111330in}{3.166093in}}%
\pgfpathlineto{\pgfqpoint{3.111330in}{3.166093in}}%
\pgfpathlineto{\pgfqpoint{3.111330in}{3.169042in}}%
\pgfpathlineto{\pgfqpoint{3.115871in}{3.169042in}}%
\pgfpathlineto{\pgfqpoint{3.115871in}{3.166093in}}%
\pgfpathmoveto{\pgfqpoint{3.111330in}{3.169042in}}%
\pgfpathlineto{\pgfqpoint{3.111330in}{3.169042in}}%
\pgfpathlineto{\pgfqpoint{3.111330in}{3.171991in}}%
\pgfpathlineto{\pgfqpoint{3.115871in}{3.171991in}}%
\pgfpathlineto{\pgfqpoint{3.115871in}{3.169042in}}%
\pgfpathmoveto{\pgfqpoint{3.088625in}{3.192636in}}%
\pgfpathlineto{\pgfqpoint{3.088625in}{3.192636in}}%
\pgfpathlineto{\pgfqpoint{3.088625in}{3.195585in}}%
\pgfpathlineto{\pgfqpoint{3.093166in}{3.195585in}}%
\pgfpathlineto{\pgfqpoint{3.093166in}{3.192636in}}%
\pgfpathmoveto{\pgfqpoint{3.084085in}{3.198534in}}%
\pgfpathlineto{\pgfqpoint{3.084085in}{3.198534in}}%
\pgfpathlineto{\pgfqpoint{3.084085in}{3.201483in}}%
\pgfpathlineto{\pgfqpoint{3.088625in}{3.201483in}}%
\pgfpathlineto{\pgfqpoint{3.088625in}{3.198534in}}%
\pgfpathmoveto{\pgfqpoint{3.088625in}{3.195585in}}%
\pgfpathlineto{\pgfqpoint{3.088625in}{3.195585in}}%
\pgfpathlineto{\pgfqpoint{3.088625in}{3.198534in}}%
\pgfpathlineto{\pgfqpoint{3.093166in}{3.198534in}}%
\pgfpathlineto{\pgfqpoint{3.093166in}{3.195585in}}%
\pgfpathmoveto{\pgfqpoint{3.088625in}{3.198534in}}%
\pgfpathlineto{\pgfqpoint{3.088625in}{3.198534in}}%
\pgfpathlineto{\pgfqpoint{3.088625in}{3.201483in}}%
\pgfpathlineto{\pgfqpoint{3.093166in}{3.201483in}}%
\pgfpathlineto{\pgfqpoint{3.093166in}{3.198534in}}%
\pgfpathmoveto{\pgfqpoint{3.079544in}{3.204433in}}%
\pgfpathlineto{\pgfqpoint{3.079544in}{3.204433in}}%
\pgfpathlineto{\pgfqpoint{3.079544in}{3.207382in}}%
\pgfpathlineto{\pgfqpoint{3.084085in}{3.207382in}}%
\pgfpathlineto{\pgfqpoint{3.084085in}{3.204433in}}%
\pgfpathmoveto{\pgfqpoint{3.075003in}{3.210331in}}%
\pgfpathlineto{\pgfqpoint{3.075003in}{3.210331in}}%
\pgfpathlineto{\pgfqpoint{3.075003in}{3.213280in}}%
\pgfpathlineto{\pgfqpoint{3.079544in}{3.213280in}}%
\pgfpathlineto{\pgfqpoint{3.079544in}{3.210331in}}%
\pgfpathmoveto{\pgfqpoint{3.079544in}{3.207382in}}%
\pgfpathlineto{\pgfqpoint{3.079544in}{3.207382in}}%
\pgfpathlineto{\pgfqpoint{3.079544in}{3.210331in}}%
\pgfpathlineto{\pgfqpoint{3.084085in}{3.210331in}}%
\pgfpathlineto{\pgfqpoint{3.084085in}{3.207382in}}%
\pgfpathmoveto{\pgfqpoint{3.079544in}{3.210331in}}%
\pgfpathlineto{\pgfqpoint{3.079544in}{3.210331in}}%
\pgfpathlineto{\pgfqpoint{3.079544in}{3.213280in}}%
\pgfpathlineto{\pgfqpoint{3.084085in}{3.213280in}}%
\pgfpathlineto{\pgfqpoint{3.084085in}{3.210331in}}%
\pgfpathmoveto{\pgfqpoint{3.084085in}{3.201483in}}%
\pgfpathlineto{\pgfqpoint{3.084085in}{3.201483in}}%
\pgfpathlineto{\pgfqpoint{3.084085in}{3.204433in}}%
\pgfpathlineto{\pgfqpoint{3.088625in}{3.204433in}}%
\pgfpathlineto{\pgfqpoint{3.088625in}{3.201483in}}%
\pgfpathmoveto{\pgfqpoint{3.084085in}{3.204433in}}%
\pgfpathlineto{\pgfqpoint{3.084085in}{3.204433in}}%
\pgfpathlineto{\pgfqpoint{3.084085in}{3.207382in}}%
\pgfpathlineto{\pgfqpoint{3.088625in}{3.207382in}}%
\pgfpathlineto{\pgfqpoint{3.088625in}{3.204433in}}%
\pgfpathmoveto{\pgfqpoint{3.093166in}{3.189687in}}%
\pgfpathlineto{\pgfqpoint{3.093166in}{3.189687in}}%
\pgfpathlineto{\pgfqpoint{3.093166in}{3.192636in}}%
\pgfpathlineto{\pgfqpoint{3.097707in}{3.192636in}}%
\pgfpathlineto{\pgfqpoint{3.097707in}{3.189687in}}%
\pgfpathmoveto{\pgfqpoint{3.093166in}{3.192636in}}%
\pgfpathlineto{\pgfqpoint{3.093166in}{3.192636in}}%
\pgfpathlineto{\pgfqpoint{3.093166in}{3.195585in}}%
\pgfpathlineto{\pgfqpoint{3.097707in}{3.195585in}}%
\pgfpathlineto{\pgfqpoint{3.097707in}{3.192636in}}%
\pgfpathmoveto{\pgfqpoint{3.075003in}{3.213280in}}%
\pgfpathlineto{\pgfqpoint{3.075003in}{3.213280in}}%
\pgfpathlineto{\pgfqpoint{3.075003in}{3.216230in}}%
\pgfpathlineto{\pgfqpoint{3.079544in}{3.216230in}}%
\pgfpathlineto{\pgfqpoint{3.079544in}{3.213280in}}%
\pgfpathmoveto{\pgfqpoint{3.075003in}{3.216230in}}%
\pgfpathlineto{\pgfqpoint{3.075003in}{3.216230in}}%
\pgfpathlineto{\pgfqpoint{3.075003in}{3.219179in}}%
\pgfpathlineto{\pgfqpoint{3.079544in}{3.219179in}}%
\pgfpathlineto{\pgfqpoint{3.079544in}{3.216230in}}%
\pgfpathmoveto{\pgfqpoint{3.361086in}{2.838730in}}%
\pgfpathlineto{\pgfqpoint{3.361086in}{2.838730in}}%
\pgfpathlineto{\pgfqpoint{3.361086in}{2.841679in}}%
\pgfpathlineto{\pgfqpoint{3.365627in}{2.841679in}}%
\pgfpathlineto{\pgfqpoint{3.365627in}{2.838730in}}%
\pgfpathmoveto{\pgfqpoint{3.356545in}{2.844628in}}%
\pgfpathlineto{\pgfqpoint{3.356545in}{2.844628in}}%
\pgfpathlineto{\pgfqpoint{3.356545in}{2.847578in}}%
\pgfpathlineto{\pgfqpoint{3.361086in}{2.847578in}}%
\pgfpathlineto{\pgfqpoint{3.361086in}{2.844628in}}%
\pgfpathmoveto{\pgfqpoint{3.361086in}{2.841679in}}%
\pgfpathlineto{\pgfqpoint{3.361086in}{2.841679in}}%
\pgfpathlineto{\pgfqpoint{3.361086in}{2.844628in}}%
\pgfpathlineto{\pgfqpoint{3.365627in}{2.844628in}}%
\pgfpathlineto{\pgfqpoint{3.365627in}{2.841679in}}%
\pgfpathmoveto{\pgfqpoint{3.361086in}{2.844628in}}%
\pgfpathlineto{\pgfqpoint{3.361086in}{2.844628in}}%
\pgfpathlineto{\pgfqpoint{3.361086in}{2.847578in}}%
\pgfpathlineto{\pgfqpoint{3.365627in}{2.847578in}}%
\pgfpathlineto{\pgfqpoint{3.365627in}{2.844628in}}%
\pgfpathmoveto{\pgfqpoint{3.352004in}{2.850527in}}%
\pgfpathlineto{\pgfqpoint{3.352004in}{2.850527in}}%
\pgfpathlineto{\pgfqpoint{3.352004in}{2.853476in}}%
\pgfpathlineto{\pgfqpoint{3.356545in}{2.853476in}}%
\pgfpathlineto{\pgfqpoint{3.356545in}{2.850527in}}%
\pgfpathmoveto{\pgfqpoint{3.347463in}{2.856426in}}%
\pgfpathlineto{\pgfqpoint{3.347463in}{2.856426in}}%
\pgfpathlineto{\pgfqpoint{3.347463in}{2.859375in}}%
\pgfpathlineto{\pgfqpoint{3.352004in}{2.859375in}}%
\pgfpathlineto{\pgfqpoint{3.352004in}{2.856426in}}%
\pgfpathmoveto{\pgfqpoint{3.352004in}{2.853476in}}%
\pgfpathlineto{\pgfqpoint{3.352004in}{2.853476in}}%
\pgfpathlineto{\pgfqpoint{3.352004in}{2.856426in}}%
\pgfpathlineto{\pgfqpoint{3.356545in}{2.856426in}}%
\pgfpathlineto{\pgfqpoint{3.356545in}{2.853476in}}%
\pgfpathmoveto{\pgfqpoint{3.352004in}{2.856426in}}%
\pgfpathlineto{\pgfqpoint{3.352004in}{2.856426in}}%
\pgfpathlineto{\pgfqpoint{3.352004in}{2.859375in}}%
\pgfpathlineto{\pgfqpoint{3.356545in}{2.859375in}}%
\pgfpathlineto{\pgfqpoint{3.356545in}{2.856426in}}%
\pgfpathmoveto{\pgfqpoint{3.356545in}{2.847578in}}%
\pgfpathlineto{\pgfqpoint{3.356545in}{2.847578in}}%
\pgfpathlineto{\pgfqpoint{3.356545in}{2.850527in}}%
\pgfpathlineto{\pgfqpoint{3.361086in}{2.850527in}}%
\pgfpathlineto{\pgfqpoint{3.361086in}{2.847578in}}%
\pgfpathmoveto{\pgfqpoint{3.356545in}{2.850527in}}%
\pgfpathlineto{\pgfqpoint{3.356545in}{2.850527in}}%
\pgfpathlineto{\pgfqpoint{3.356545in}{2.853476in}}%
\pgfpathlineto{\pgfqpoint{3.361086in}{2.853476in}}%
\pgfpathlineto{\pgfqpoint{3.361086in}{2.850527in}}%
\pgfpathmoveto{\pgfqpoint{3.288429in}{2.933106in}}%
\pgfpathlineto{\pgfqpoint{3.288429in}{2.933106in}}%
\pgfpathlineto{\pgfqpoint{3.288429in}{2.936056in}}%
\pgfpathlineto{\pgfqpoint{3.292970in}{2.936056in}}%
\pgfpathlineto{\pgfqpoint{3.292970in}{2.933106in}}%
\pgfpathmoveto{\pgfqpoint{3.283888in}{2.939005in}}%
\pgfpathlineto{\pgfqpoint{3.283888in}{2.939005in}}%
\pgfpathlineto{\pgfqpoint{3.283888in}{2.941954in}}%
\pgfpathlineto{\pgfqpoint{3.288429in}{2.941954in}}%
\pgfpathlineto{\pgfqpoint{3.288429in}{2.939005in}}%
\pgfpathmoveto{\pgfqpoint{3.288429in}{2.936056in}}%
\pgfpathlineto{\pgfqpoint{3.288429in}{2.936056in}}%
\pgfpathlineto{\pgfqpoint{3.288429in}{2.939005in}}%
\pgfpathlineto{\pgfqpoint{3.292970in}{2.939005in}}%
\pgfpathlineto{\pgfqpoint{3.292970in}{2.936056in}}%
\pgfpathmoveto{\pgfqpoint{3.288429in}{2.939005in}}%
\pgfpathlineto{\pgfqpoint{3.288429in}{2.939005in}}%
\pgfpathlineto{\pgfqpoint{3.288429in}{2.941954in}}%
\pgfpathlineto{\pgfqpoint{3.292970in}{2.941954in}}%
\pgfpathlineto{\pgfqpoint{3.292970in}{2.939005in}}%
\pgfpathmoveto{\pgfqpoint{3.279347in}{2.944904in}}%
\pgfpathlineto{\pgfqpoint{3.279347in}{2.944904in}}%
\pgfpathlineto{\pgfqpoint{3.279347in}{2.947853in}}%
\pgfpathlineto{\pgfqpoint{3.283888in}{2.947853in}}%
\pgfpathlineto{\pgfqpoint{3.283888in}{2.944904in}}%
\pgfpathmoveto{\pgfqpoint{3.274806in}{2.950802in}}%
\pgfpathlineto{\pgfqpoint{3.274806in}{2.950802in}}%
\pgfpathlineto{\pgfqpoint{3.274806in}{2.953751in}}%
\pgfpathlineto{\pgfqpoint{3.279347in}{2.953751in}}%
\pgfpathlineto{\pgfqpoint{3.279347in}{2.950802in}}%
\pgfpathmoveto{\pgfqpoint{3.279347in}{2.947853in}}%
\pgfpathlineto{\pgfqpoint{3.279347in}{2.947853in}}%
\pgfpathlineto{\pgfqpoint{3.279347in}{2.950802in}}%
\pgfpathlineto{\pgfqpoint{3.283888in}{2.950802in}}%
\pgfpathlineto{\pgfqpoint{3.283888in}{2.947853in}}%
\pgfpathmoveto{\pgfqpoint{3.279347in}{2.950802in}}%
\pgfpathlineto{\pgfqpoint{3.279347in}{2.950802in}}%
\pgfpathlineto{\pgfqpoint{3.279347in}{2.953751in}}%
\pgfpathlineto{\pgfqpoint{3.283888in}{2.953751in}}%
\pgfpathlineto{\pgfqpoint{3.283888in}{2.950802in}}%
\pgfpathmoveto{\pgfqpoint{3.283888in}{2.941954in}}%
\pgfpathlineto{\pgfqpoint{3.283888in}{2.941954in}}%
\pgfpathlineto{\pgfqpoint{3.283888in}{2.944904in}}%
\pgfpathlineto{\pgfqpoint{3.288429in}{2.944904in}}%
\pgfpathlineto{\pgfqpoint{3.288429in}{2.941954in}}%
\pgfpathmoveto{\pgfqpoint{3.283888in}{2.944904in}}%
\pgfpathlineto{\pgfqpoint{3.283888in}{2.944904in}}%
\pgfpathlineto{\pgfqpoint{3.283888in}{2.947853in}}%
\pgfpathlineto{\pgfqpoint{3.288429in}{2.947853in}}%
\pgfpathlineto{\pgfqpoint{3.288429in}{2.944904in}}%
\pgfpathmoveto{\pgfqpoint{3.324758in}{2.885918in}}%
\pgfpathlineto{\pgfqpoint{3.324758in}{2.885918in}}%
\pgfpathlineto{\pgfqpoint{3.324758in}{2.888868in}}%
\pgfpathlineto{\pgfqpoint{3.329299in}{2.888868in}}%
\pgfpathlineto{\pgfqpoint{3.329299in}{2.885918in}}%
\pgfpathmoveto{\pgfqpoint{3.320217in}{2.891817in}}%
\pgfpathlineto{\pgfqpoint{3.320217in}{2.891817in}}%
\pgfpathlineto{\pgfqpoint{3.320217in}{2.894766in}}%
\pgfpathlineto{\pgfqpoint{3.324758in}{2.894766in}}%
\pgfpathlineto{\pgfqpoint{3.324758in}{2.891817in}}%
\pgfpathmoveto{\pgfqpoint{3.324758in}{2.888868in}}%
\pgfpathlineto{\pgfqpoint{3.324758in}{2.888868in}}%
\pgfpathlineto{\pgfqpoint{3.324758in}{2.891817in}}%
\pgfpathlineto{\pgfqpoint{3.329299in}{2.891817in}}%
\pgfpathlineto{\pgfqpoint{3.329299in}{2.888868in}}%
\pgfpathmoveto{\pgfqpoint{3.324758in}{2.891817in}}%
\pgfpathlineto{\pgfqpoint{3.324758in}{2.891817in}}%
\pgfpathlineto{\pgfqpoint{3.324758in}{2.894766in}}%
\pgfpathlineto{\pgfqpoint{3.329299in}{2.894766in}}%
\pgfpathlineto{\pgfqpoint{3.329299in}{2.891817in}}%
\pgfpathmoveto{\pgfqpoint{3.315676in}{2.897715in}}%
\pgfpathlineto{\pgfqpoint{3.315676in}{2.897715in}}%
\pgfpathlineto{\pgfqpoint{3.315676in}{2.900665in}}%
\pgfpathlineto{\pgfqpoint{3.320217in}{2.900665in}}%
\pgfpathlineto{\pgfqpoint{3.320217in}{2.897715in}}%
\pgfpathmoveto{\pgfqpoint{3.311135in}{2.903614in}}%
\pgfpathlineto{\pgfqpoint{3.311135in}{2.903614in}}%
\pgfpathlineto{\pgfqpoint{3.311135in}{2.906563in}}%
\pgfpathlineto{\pgfqpoint{3.315676in}{2.906563in}}%
\pgfpathlineto{\pgfqpoint{3.315676in}{2.903614in}}%
\pgfpathmoveto{\pgfqpoint{3.315676in}{2.900665in}}%
\pgfpathlineto{\pgfqpoint{3.315676in}{2.900665in}}%
\pgfpathlineto{\pgfqpoint{3.315676in}{2.903614in}}%
\pgfpathlineto{\pgfqpoint{3.320217in}{2.903614in}}%
\pgfpathlineto{\pgfqpoint{3.320217in}{2.900665in}}%
\pgfpathmoveto{\pgfqpoint{3.315676in}{2.903614in}}%
\pgfpathlineto{\pgfqpoint{3.315676in}{2.903614in}}%
\pgfpathlineto{\pgfqpoint{3.315676in}{2.906563in}}%
\pgfpathlineto{\pgfqpoint{3.320217in}{2.906563in}}%
\pgfpathlineto{\pgfqpoint{3.320217in}{2.903614in}}%
\pgfpathmoveto{\pgfqpoint{3.320217in}{2.894766in}}%
\pgfpathlineto{\pgfqpoint{3.320217in}{2.894766in}}%
\pgfpathlineto{\pgfqpoint{3.320217in}{2.897715in}}%
\pgfpathlineto{\pgfqpoint{3.324758in}{2.897715in}}%
\pgfpathlineto{\pgfqpoint{3.324758in}{2.894766in}}%
\pgfpathmoveto{\pgfqpoint{3.320217in}{2.897715in}}%
\pgfpathlineto{\pgfqpoint{3.320217in}{2.897715in}}%
\pgfpathlineto{\pgfqpoint{3.320217in}{2.900665in}}%
\pgfpathlineto{\pgfqpoint{3.324758in}{2.900665in}}%
\pgfpathlineto{\pgfqpoint{3.324758in}{2.897715in}}%
\pgfpathmoveto{\pgfqpoint{3.342922in}{2.862324in}}%
\pgfpathlineto{\pgfqpoint{3.342922in}{2.862324in}}%
\pgfpathlineto{\pgfqpoint{3.342922in}{2.865273in}}%
\pgfpathlineto{\pgfqpoint{3.347463in}{2.865273in}}%
\pgfpathlineto{\pgfqpoint{3.347463in}{2.862324in}}%
\pgfpathmoveto{\pgfqpoint{3.338381in}{2.868223in}}%
\pgfpathlineto{\pgfqpoint{3.338381in}{2.868223in}}%
\pgfpathlineto{\pgfqpoint{3.338381in}{2.871172in}}%
\pgfpathlineto{\pgfqpoint{3.342922in}{2.871172in}}%
\pgfpathlineto{\pgfqpoint{3.342922in}{2.868223in}}%
\pgfpathmoveto{\pgfqpoint{3.342922in}{2.865273in}}%
\pgfpathlineto{\pgfqpoint{3.342922in}{2.865273in}}%
\pgfpathlineto{\pgfqpoint{3.342922in}{2.868223in}}%
\pgfpathlineto{\pgfqpoint{3.347463in}{2.868223in}}%
\pgfpathlineto{\pgfqpoint{3.347463in}{2.865273in}}%
\pgfpathmoveto{\pgfqpoint{3.342922in}{2.868223in}}%
\pgfpathlineto{\pgfqpoint{3.342922in}{2.868223in}}%
\pgfpathlineto{\pgfqpoint{3.342922in}{2.871172in}}%
\pgfpathlineto{\pgfqpoint{3.347463in}{2.871172in}}%
\pgfpathlineto{\pgfqpoint{3.347463in}{2.868223in}}%
\pgfpathmoveto{\pgfqpoint{3.333840in}{2.874121in}}%
\pgfpathlineto{\pgfqpoint{3.333840in}{2.874121in}}%
\pgfpathlineto{\pgfqpoint{3.333840in}{2.877070in}}%
\pgfpathlineto{\pgfqpoint{3.338381in}{2.877070in}}%
\pgfpathlineto{\pgfqpoint{3.338381in}{2.874121in}}%
\pgfpathmoveto{\pgfqpoint{3.329299in}{2.880020in}}%
\pgfpathlineto{\pgfqpoint{3.329299in}{2.880020in}}%
\pgfpathlineto{\pgfqpoint{3.329299in}{2.882969in}}%
\pgfpathlineto{\pgfqpoint{3.333840in}{2.882969in}}%
\pgfpathlineto{\pgfqpoint{3.333840in}{2.880020in}}%
\pgfpathmoveto{\pgfqpoint{3.333840in}{2.877070in}}%
\pgfpathlineto{\pgfqpoint{3.333840in}{2.877070in}}%
\pgfpathlineto{\pgfqpoint{3.333840in}{2.880020in}}%
\pgfpathlineto{\pgfqpoint{3.338381in}{2.880020in}}%
\pgfpathlineto{\pgfqpoint{3.338381in}{2.877070in}}%
\pgfpathmoveto{\pgfqpoint{3.333840in}{2.880020in}}%
\pgfpathlineto{\pgfqpoint{3.333840in}{2.880020in}}%
\pgfpathlineto{\pgfqpoint{3.333840in}{2.882969in}}%
\pgfpathlineto{\pgfqpoint{3.338381in}{2.882969in}}%
\pgfpathlineto{\pgfqpoint{3.338381in}{2.880020in}}%
\pgfpathmoveto{\pgfqpoint{3.338381in}{2.871172in}}%
\pgfpathlineto{\pgfqpoint{3.338381in}{2.871172in}}%
\pgfpathlineto{\pgfqpoint{3.338381in}{2.874121in}}%
\pgfpathlineto{\pgfqpoint{3.342922in}{2.874121in}}%
\pgfpathlineto{\pgfqpoint{3.342922in}{2.871172in}}%
\pgfpathmoveto{\pgfqpoint{3.338381in}{2.874121in}}%
\pgfpathlineto{\pgfqpoint{3.338381in}{2.874121in}}%
\pgfpathlineto{\pgfqpoint{3.338381in}{2.877070in}}%
\pgfpathlineto{\pgfqpoint{3.342922in}{2.877070in}}%
\pgfpathlineto{\pgfqpoint{3.342922in}{2.874121in}}%
\pgfpathmoveto{\pgfqpoint{3.347463in}{2.859375in}}%
\pgfpathlineto{\pgfqpoint{3.347463in}{2.859375in}}%
\pgfpathlineto{\pgfqpoint{3.347463in}{2.862324in}}%
\pgfpathlineto{\pgfqpoint{3.352004in}{2.862324in}}%
\pgfpathlineto{\pgfqpoint{3.352004in}{2.859375in}}%
\pgfpathmoveto{\pgfqpoint{3.347463in}{2.862324in}}%
\pgfpathlineto{\pgfqpoint{3.347463in}{2.862324in}}%
\pgfpathlineto{\pgfqpoint{3.347463in}{2.865273in}}%
\pgfpathlineto{\pgfqpoint{3.352004in}{2.865273in}}%
\pgfpathlineto{\pgfqpoint{3.352004in}{2.862324in}}%
\pgfpathmoveto{\pgfqpoint{3.329299in}{2.882969in}}%
\pgfpathlineto{\pgfqpoint{3.329299in}{2.882969in}}%
\pgfpathlineto{\pgfqpoint{3.329299in}{2.885918in}}%
\pgfpathlineto{\pgfqpoint{3.333840in}{2.885918in}}%
\pgfpathlineto{\pgfqpoint{3.333840in}{2.882969in}}%
\pgfpathmoveto{\pgfqpoint{3.329299in}{2.885918in}}%
\pgfpathlineto{\pgfqpoint{3.329299in}{2.885918in}}%
\pgfpathlineto{\pgfqpoint{3.329299in}{2.888868in}}%
\pgfpathlineto{\pgfqpoint{3.333840in}{2.888868in}}%
\pgfpathlineto{\pgfqpoint{3.333840in}{2.885918in}}%
\pgfpathmoveto{\pgfqpoint{3.306593in}{2.909512in}}%
\pgfpathlineto{\pgfqpoint{3.306593in}{2.909512in}}%
\pgfpathlineto{\pgfqpoint{3.306593in}{2.912462in}}%
\pgfpathlineto{\pgfqpoint{3.311135in}{2.912462in}}%
\pgfpathlineto{\pgfqpoint{3.311135in}{2.909512in}}%
\pgfpathmoveto{\pgfqpoint{3.302052in}{2.915411in}}%
\pgfpathlineto{\pgfqpoint{3.302052in}{2.915411in}}%
\pgfpathlineto{\pgfqpoint{3.302052in}{2.918360in}}%
\pgfpathlineto{\pgfqpoint{3.306593in}{2.918360in}}%
\pgfpathlineto{\pgfqpoint{3.306593in}{2.915411in}}%
\pgfpathmoveto{\pgfqpoint{3.306593in}{2.912462in}}%
\pgfpathlineto{\pgfqpoint{3.306593in}{2.912462in}}%
\pgfpathlineto{\pgfqpoint{3.306593in}{2.915411in}}%
\pgfpathlineto{\pgfqpoint{3.311135in}{2.915411in}}%
\pgfpathlineto{\pgfqpoint{3.311135in}{2.912462in}}%
\pgfpathmoveto{\pgfqpoint{3.306593in}{2.915411in}}%
\pgfpathlineto{\pgfqpoint{3.306593in}{2.915411in}}%
\pgfpathlineto{\pgfqpoint{3.306593in}{2.918360in}}%
\pgfpathlineto{\pgfqpoint{3.311135in}{2.918360in}}%
\pgfpathlineto{\pgfqpoint{3.311135in}{2.915411in}}%
\pgfpathmoveto{\pgfqpoint{3.297511in}{2.921309in}}%
\pgfpathlineto{\pgfqpoint{3.297511in}{2.921309in}}%
\pgfpathlineto{\pgfqpoint{3.297511in}{2.924259in}}%
\pgfpathlineto{\pgfqpoint{3.302052in}{2.924259in}}%
\pgfpathlineto{\pgfqpoint{3.302052in}{2.921309in}}%
\pgfpathmoveto{\pgfqpoint{3.292970in}{2.927208in}}%
\pgfpathlineto{\pgfqpoint{3.292970in}{2.927208in}}%
\pgfpathlineto{\pgfqpoint{3.292970in}{2.930157in}}%
\pgfpathlineto{\pgfqpoint{3.297511in}{2.930157in}}%
\pgfpathlineto{\pgfqpoint{3.297511in}{2.927208in}}%
\pgfpathmoveto{\pgfqpoint{3.297511in}{2.924259in}}%
\pgfpathlineto{\pgfqpoint{3.297511in}{2.924259in}}%
\pgfpathlineto{\pgfqpoint{3.297511in}{2.927208in}}%
\pgfpathlineto{\pgfqpoint{3.302052in}{2.927208in}}%
\pgfpathlineto{\pgfqpoint{3.302052in}{2.924259in}}%
\pgfpathmoveto{\pgfqpoint{3.297511in}{2.927208in}}%
\pgfpathlineto{\pgfqpoint{3.297511in}{2.927208in}}%
\pgfpathlineto{\pgfqpoint{3.297511in}{2.930157in}}%
\pgfpathlineto{\pgfqpoint{3.302052in}{2.930157in}}%
\pgfpathlineto{\pgfqpoint{3.302052in}{2.927208in}}%
\pgfpathmoveto{\pgfqpoint{3.302052in}{2.918360in}}%
\pgfpathlineto{\pgfqpoint{3.302052in}{2.918360in}}%
\pgfpathlineto{\pgfqpoint{3.302052in}{2.921309in}}%
\pgfpathlineto{\pgfqpoint{3.306593in}{2.921309in}}%
\pgfpathlineto{\pgfqpoint{3.306593in}{2.918360in}}%
\pgfpathmoveto{\pgfqpoint{3.302052in}{2.921309in}}%
\pgfpathlineto{\pgfqpoint{3.302052in}{2.921309in}}%
\pgfpathlineto{\pgfqpoint{3.302052in}{2.924259in}}%
\pgfpathlineto{\pgfqpoint{3.306593in}{2.924259in}}%
\pgfpathlineto{\pgfqpoint{3.306593in}{2.921309in}}%
\pgfpathmoveto{\pgfqpoint{3.311135in}{2.906563in}}%
\pgfpathlineto{\pgfqpoint{3.311135in}{2.906563in}}%
\pgfpathlineto{\pgfqpoint{3.311135in}{2.909512in}}%
\pgfpathlineto{\pgfqpoint{3.315676in}{2.909512in}}%
\pgfpathlineto{\pgfqpoint{3.315676in}{2.906563in}}%
\pgfpathmoveto{\pgfqpoint{3.311135in}{2.909512in}}%
\pgfpathlineto{\pgfqpoint{3.311135in}{2.909512in}}%
\pgfpathlineto{\pgfqpoint{3.311135in}{2.912462in}}%
\pgfpathlineto{\pgfqpoint{3.315676in}{2.912462in}}%
\pgfpathlineto{\pgfqpoint{3.315676in}{2.909512in}}%
\pgfpathmoveto{\pgfqpoint{3.292970in}{2.930157in}}%
\pgfpathlineto{\pgfqpoint{3.292970in}{2.930157in}}%
\pgfpathlineto{\pgfqpoint{3.292970in}{2.933106in}}%
\pgfpathlineto{\pgfqpoint{3.297511in}{2.933106in}}%
\pgfpathlineto{\pgfqpoint{3.297511in}{2.930157in}}%
\pgfpathmoveto{\pgfqpoint{3.292970in}{2.933106in}}%
\pgfpathlineto{\pgfqpoint{3.292970in}{2.933106in}}%
\pgfpathlineto{\pgfqpoint{3.292970in}{2.936056in}}%
\pgfpathlineto{\pgfqpoint{3.297511in}{2.936056in}}%
\pgfpathlineto{\pgfqpoint{3.297511in}{2.933106in}}%
\pgfpathmoveto{\pgfqpoint{3.252101in}{2.980294in}}%
\pgfpathlineto{\pgfqpoint{3.252101in}{2.980294in}}%
\pgfpathlineto{\pgfqpoint{3.252101in}{2.983243in}}%
\pgfpathlineto{\pgfqpoint{3.256642in}{2.983243in}}%
\pgfpathlineto{\pgfqpoint{3.256642in}{2.980294in}}%
\pgfpathmoveto{\pgfqpoint{3.247560in}{2.986193in}}%
\pgfpathlineto{\pgfqpoint{3.247560in}{2.986193in}}%
\pgfpathlineto{\pgfqpoint{3.247560in}{2.989142in}}%
\pgfpathlineto{\pgfqpoint{3.252101in}{2.989142in}}%
\pgfpathlineto{\pgfqpoint{3.252101in}{2.986193in}}%
\pgfpathmoveto{\pgfqpoint{3.252101in}{2.983243in}}%
\pgfpathlineto{\pgfqpoint{3.252101in}{2.983243in}}%
\pgfpathlineto{\pgfqpoint{3.252101in}{2.986193in}}%
\pgfpathlineto{\pgfqpoint{3.256642in}{2.986193in}}%
\pgfpathlineto{\pgfqpoint{3.256642in}{2.983243in}}%
\pgfpathmoveto{\pgfqpoint{3.252101in}{2.986193in}}%
\pgfpathlineto{\pgfqpoint{3.252101in}{2.986193in}}%
\pgfpathlineto{\pgfqpoint{3.252101in}{2.989142in}}%
\pgfpathlineto{\pgfqpoint{3.256642in}{2.989142in}}%
\pgfpathlineto{\pgfqpoint{3.256642in}{2.986193in}}%
\pgfpathmoveto{\pgfqpoint{3.243019in}{2.992091in}}%
\pgfpathlineto{\pgfqpoint{3.243019in}{2.992091in}}%
\pgfpathlineto{\pgfqpoint{3.243019in}{2.995040in}}%
\pgfpathlineto{\pgfqpoint{3.247560in}{2.995040in}}%
\pgfpathlineto{\pgfqpoint{3.247560in}{2.992091in}}%
\pgfpathmoveto{\pgfqpoint{3.238477in}{2.997990in}}%
\pgfpathlineto{\pgfqpoint{3.238477in}{2.997990in}}%
\pgfpathlineto{\pgfqpoint{3.238477in}{3.000939in}}%
\pgfpathlineto{\pgfqpoint{3.243019in}{3.000939in}}%
\pgfpathlineto{\pgfqpoint{3.243019in}{2.997990in}}%
\pgfpathmoveto{\pgfqpoint{3.243019in}{2.995040in}}%
\pgfpathlineto{\pgfqpoint{3.243019in}{2.995040in}}%
\pgfpathlineto{\pgfqpoint{3.243019in}{2.997990in}}%
\pgfpathlineto{\pgfqpoint{3.247560in}{2.997990in}}%
\pgfpathlineto{\pgfqpoint{3.247560in}{2.995040in}}%
\pgfpathmoveto{\pgfqpoint{3.243019in}{2.997990in}}%
\pgfpathlineto{\pgfqpoint{3.243019in}{2.997990in}}%
\pgfpathlineto{\pgfqpoint{3.243019in}{3.000939in}}%
\pgfpathlineto{\pgfqpoint{3.247560in}{3.000939in}}%
\pgfpathlineto{\pgfqpoint{3.247560in}{2.997990in}}%
\pgfpathmoveto{\pgfqpoint{3.247560in}{2.989142in}}%
\pgfpathlineto{\pgfqpoint{3.247560in}{2.989142in}}%
\pgfpathlineto{\pgfqpoint{3.247560in}{2.992091in}}%
\pgfpathlineto{\pgfqpoint{3.252101in}{2.992091in}}%
\pgfpathlineto{\pgfqpoint{3.252101in}{2.989142in}}%
\pgfpathmoveto{\pgfqpoint{3.247560in}{2.992091in}}%
\pgfpathlineto{\pgfqpoint{3.247560in}{2.992091in}}%
\pgfpathlineto{\pgfqpoint{3.247560in}{2.995040in}}%
\pgfpathlineto{\pgfqpoint{3.252101in}{2.995040in}}%
\pgfpathlineto{\pgfqpoint{3.252101in}{2.992091in}}%
\pgfpathmoveto{\pgfqpoint{3.270265in}{2.956701in}}%
\pgfpathlineto{\pgfqpoint{3.270265in}{2.956701in}}%
\pgfpathlineto{\pgfqpoint{3.270265in}{2.959650in}}%
\pgfpathlineto{\pgfqpoint{3.274806in}{2.959650in}}%
\pgfpathlineto{\pgfqpoint{3.274806in}{2.956701in}}%
\pgfpathmoveto{\pgfqpoint{3.265724in}{2.962599in}}%
\pgfpathlineto{\pgfqpoint{3.265724in}{2.962599in}}%
\pgfpathlineto{\pgfqpoint{3.265724in}{2.965548in}}%
\pgfpathlineto{\pgfqpoint{3.270265in}{2.965548in}}%
\pgfpathlineto{\pgfqpoint{3.270265in}{2.962599in}}%
\pgfpathmoveto{\pgfqpoint{3.270265in}{2.959650in}}%
\pgfpathlineto{\pgfqpoint{3.270265in}{2.959650in}}%
\pgfpathlineto{\pgfqpoint{3.270265in}{2.962599in}}%
\pgfpathlineto{\pgfqpoint{3.274806in}{2.962599in}}%
\pgfpathlineto{\pgfqpoint{3.274806in}{2.959650in}}%
\pgfpathmoveto{\pgfqpoint{3.270265in}{2.962599in}}%
\pgfpathlineto{\pgfqpoint{3.270265in}{2.962599in}}%
\pgfpathlineto{\pgfqpoint{3.270265in}{2.965548in}}%
\pgfpathlineto{\pgfqpoint{3.274806in}{2.965548in}}%
\pgfpathlineto{\pgfqpoint{3.274806in}{2.962599in}}%
\pgfpathmoveto{\pgfqpoint{3.261183in}{2.968497in}}%
\pgfpathlineto{\pgfqpoint{3.261183in}{2.968497in}}%
\pgfpathlineto{\pgfqpoint{3.261183in}{2.971447in}}%
\pgfpathlineto{\pgfqpoint{3.265724in}{2.971447in}}%
\pgfpathlineto{\pgfqpoint{3.265724in}{2.968497in}}%
\pgfpathmoveto{\pgfqpoint{3.256642in}{2.974396in}}%
\pgfpathlineto{\pgfqpoint{3.256642in}{2.974396in}}%
\pgfpathlineto{\pgfqpoint{3.256642in}{2.977345in}}%
\pgfpathlineto{\pgfqpoint{3.261183in}{2.977345in}}%
\pgfpathlineto{\pgfqpoint{3.261183in}{2.974396in}}%
\pgfpathmoveto{\pgfqpoint{3.261183in}{2.971447in}}%
\pgfpathlineto{\pgfqpoint{3.261183in}{2.971447in}}%
\pgfpathlineto{\pgfqpoint{3.261183in}{2.974396in}}%
\pgfpathlineto{\pgfqpoint{3.265724in}{2.974396in}}%
\pgfpathlineto{\pgfqpoint{3.265724in}{2.971447in}}%
\pgfpathmoveto{\pgfqpoint{3.261183in}{2.974396in}}%
\pgfpathlineto{\pgfqpoint{3.261183in}{2.974396in}}%
\pgfpathlineto{\pgfqpoint{3.261183in}{2.977345in}}%
\pgfpathlineto{\pgfqpoint{3.265724in}{2.977345in}}%
\pgfpathlineto{\pgfqpoint{3.265724in}{2.974396in}}%
\pgfpathmoveto{\pgfqpoint{3.265724in}{2.965548in}}%
\pgfpathlineto{\pgfqpoint{3.265724in}{2.965548in}}%
\pgfpathlineto{\pgfqpoint{3.265724in}{2.968497in}}%
\pgfpathlineto{\pgfqpoint{3.270265in}{2.968497in}}%
\pgfpathlineto{\pgfqpoint{3.270265in}{2.965548in}}%
\pgfpathmoveto{\pgfqpoint{3.265724in}{2.968497in}}%
\pgfpathlineto{\pgfqpoint{3.265724in}{2.968497in}}%
\pgfpathlineto{\pgfqpoint{3.265724in}{2.971447in}}%
\pgfpathlineto{\pgfqpoint{3.270265in}{2.971447in}}%
\pgfpathlineto{\pgfqpoint{3.270265in}{2.968497in}}%
\pgfpathmoveto{\pgfqpoint{3.274806in}{2.953751in}}%
\pgfpathlineto{\pgfqpoint{3.274806in}{2.953751in}}%
\pgfpathlineto{\pgfqpoint{3.274806in}{2.956701in}}%
\pgfpathlineto{\pgfqpoint{3.279347in}{2.956701in}}%
\pgfpathlineto{\pgfqpoint{3.279347in}{2.953751in}}%
\pgfpathmoveto{\pgfqpoint{3.274806in}{2.956701in}}%
\pgfpathlineto{\pgfqpoint{3.274806in}{2.956701in}}%
\pgfpathlineto{\pgfqpoint{3.274806in}{2.959650in}}%
\pgfpathlineto{\pgfqpoint{3.279347in}{2.959650in}}%
\pgfpathlineto{\pgfqpoint{3.279347in}{2.956701in}}%
\pgfpathmoveto{\pgfqpoint{3.256642in}{2.977345in}}%
\pgfpathlineto{\pgfqpoint{3.256642in}{2.977345in}}%
\pgfpathlineto{\pgfqpoint{3.256642in}{2.980294in}}%
\pgfpathlineto{\pgfqpoint{3.261183in}{2.980294in}}%
\pgfpathlineto{\pgfqpoint{3.261183in}{2.977345in}}%
\pgfpathmoveto{\pgfqpoint{3.256642in}{2.980294in}}%
\pgfpathlineto{\pgfqpoint{3.256642in}{2.980294in}}%
\pgfpathlineto{\pgfqpoint{3.256642in}{2.983243in}}%
\pgfpathlineto{\pgfqpoint{3.261183in}{2.983243in}}%
\pgfpathlineto{\pgfqpoint{3.261183in}{2.980294in}}%
\pgfpathmoveto{\pgfqpoint{3.233936in}{3.003888in}}%
\pgfpathlineto{\pgfqpoint{3.233936in}{3.003888in}}%
\pgfpathlineto{\pgfqpoint{3.233936in}{3.006837in}}%
\pgfpathlineto{\pgfqpoint{3.238477in}{3.006837in}}%
\pgfpathlineto{\pgfqpoint{3.238477in}{3.003888in}}%
\pgfpathmoveto{\pgfqpoint{3.229395in}{3.009786in}}%
\pgfpathlineto{\pgfqpoint{3.229395in}{3.009786in}}%
\pgfpathlineto{\pgfqpoint{3.229395in}{3.012736in}}%
\pgfpathlineto{\pgfqpoint{3.233936in}{3.012736in}}%
\pgfpathlineto{\pgfqpoint{3.233936in}{3.009786in}}%
\pgfpathmoveto{\pgfqpoint{3.233936in}{3.006837in}}%
\pgfpathlineto{\pgfqpoint{3.233936in}{3.006837in}}%
\pgfpathlineto{\pgfqpoint{3.233936in}{3.009786in}}%
\pgfpathlineto{\pgfqpoint{3.238477in}{3.009786in}}%
\pgfpathlineto{\pgfqpoint{3.238477in}{3.006837in}}%
\pgfpathmoveto{\pgfqpoint{3.233936in}{3.009786in}}%
\pgfpathlineto{\pgfqpoint{3.233936in}{3.009786in}}%
\pgfpathlineto{\pgfqpoint{3.233936in}{3.012736in}}%
\pgfpathlineto{\pgfqpoint{3.238477in}{3.012736in}}%
\pgfpathlineto{\pgfqpoint{3.238477in}{3.009786in}}%
\pgfpathmoveto{\pgfqpoint{3.224854in}{3.015685in}}%
\pgfpathlineto{\pgfqpoint{3.224854in}{3.015685in}}%
\pgfpathlineto{\pgfqpoint{3.224854in}{3.018634in}}%
\pgfpathlineto{\pgfqpoint{3.229395in}{3.018634in}}%
\pgfpathlineto{\pgfqpoint{3.229395in}{3.015685in}}%
\pgfpathmoveto{\pgfqpoint{3.220313in}{3.021583in}}%
\pgfpathlineto{\pgfqpoint{3.220313in}{3.021583in}}%
\pgfpathlineto{\pgfqpoint{3.220313in}{3.024533in}}%
\pgfpathlineto{\pgfqpoint{3.224854in}{3.024533in}}%
\pgfpathlineto{\pgfqpoint{3.224854in}{3.021583in}}%
\pgfpathmoveto{\pgfqpoint{3.224854in}{3.018634in}}%
\pgfpathlineto{\pgfqpoint{3.224854in}{3.018634in}}%
\pgfpathlineto{\pgfqpoint{3.224854in}{3.021583in}}%
\pgfpathlineto{\pgfqpoint{3.229395in}{3.021583in}}%
\pgfpathlineto{\pgfqpoint{3.229395in}{3.018634in}}%
\pgfpathmoveto{\pgfqpoint{3.224854in}{3.021583in}}%
\pgfpathlineto{\pgfqpoint{3.224854in}{3.021583in}}%
\pgfpathlineto{\pgfqpoint{3.224854in}{3.024533in}}%
\pgfpathlineto{\pgfqpoint{3.229395in}{3.024533in}}%
\pgfpathlineto{\pgfqpoint{3.229395in}{3.021583in}}%
\pgfpathmoveto{\pgfqpoint{3.229395in}{3.012736in}}%
\pgfpathlineto{\pgfqpoint{3.229395in}{3.012736in}}%
\pgfpathlineto{\pgfqpoint{3.229395in}{3.015685in}}%
\pgfpathlineto{\pgfqpoint{3.233936in}{3.015685in}}%
\pgfpathlineto{\pgfqpoint{3.233936in}{3.012736in}}%
\pgfpathmoveto{\pgfqpoint{3.229395in}{3.015685in}}%
\pgfpathlineto{\pgfqpoint{3.229395in}{3.015685in}}%
\pgfpathlineto{\pgfqpoint{3.229395in}{3.018634in}}%
\pgfpathlineto{\pgfqpoint{3.233936in}{3.018634in}}%
\pgfpathlineto{\pgfqpoint{3.233936in}{3.015685in}}%
\pgfpathmoveto{\pgfqpoint{3.238477in}{3.000939in}}%
\pgfpathlineto{\pgfqpoint{3.238477in}{3.000939in}}%
\pgfpathlineto{\pgfqpoint{3.238477in}{3.003888in}}%
\pgfpathlineto{\pgfqpoint{3.243019in}{3.003888in}}%
\pgfpathlineto{\pgfqpoint{3.243019in}{3.000939in}}%
\pgfpathmoveto{\pgfqpoint{3.238477in}{3.003888in}}%
\pgfpathlineto{\pgfqpoint{3.238477in}{3.003888in}}%
\pgfpathlineto{\pgfqpoint{3.238477in}{3.006837in}}%
\pgfpathlineto{\pgfqpoint{3.243019in}{3.006837in}}%
\pgfpathlineto{\pgfqpoint{3.243019in}{3.003888in}}%
\pgfpathmoveto{\pgfqpoint{3.220313in}{3.024533in}}%
\pgfpathlineto{\pgfqpoint{3.220313in}{3.024533in}}%
\pgfpathlineto{\pgfqpoint{3.220313in}{3.027482in}}%
\pgfpathlineto{\pgfqpoint{3.224854in}{3.027482in}}%
\pgfpathlineto{\pgfqpoint{3.224854in}{3.024533in}}%
\pgfpathmoveto{\pgfqpoint{3.220313in}{3.027482in}}%
\pgfpathlineto{\pgfqpoint{3.220313in}{3.027482in}}%
\pgfpathlineto{\pgfqpoint{3.220313in}{3.030431in}}%
\pgfpathlineto{\pgfqpoint{3.224854in}{3.030431in}}%
\pgfpathlineto{\pgfqpoint{3.224854in}{3.027482in}}%
\pgfpathmoveto{\pgfqpoint{3.506392in}{2.649980in}}%
\pgfpathlineto{\pgfqpoint{3.506392in}{2.649980in}}%
\pgfpathlineto{\pgfqpoint{3.506392in}{2.652929in}}%
\pgfpathlineto{\pgfqpoint{3.510933in}{2.652929in}}%
\pgfpathlineto{\pgfqpoint{3.510933in}{2.649980in}}%
\pgfpathmoveto{\pgfqpoint{3.501852in}{2.655878in}}%
\pgfpathlineto{\pgfqpoint{3.501852in}{2.655878in}}%
\pgfpathlineto{\pgfqpoint{3.501852in}{2.658827in}}%
\pgfpathlineto{\pgfqpoint{3.506392in}{2.658827in}}%
\pgfpathlineto{\pgfqpoint{3.506392in}{2.655878in}}%
\pgfpathmoveto{\pgfqpoint{3.506392in}{2.652929in}}%
\pgfpathlineto{\pgfqpoint{3.506392in}{2.652929in}}%
\pgfpathlineto{\pgfqpoint{3.506392in}{2.655878in}}%
\pgfpathlineto{\pgfqpoint{3.510933in}{2.655878in}}%
\pgfpathlineto{\pgfqpoint{3.510933in}{2.652929in}}%
\pgfpathmoveto{\pgfqpoint{3.506392in}{2.655878in}}%
\pgfpathlineto{\pgfqpoint{3.506392in}{2.655878in}}%
\pgfpathlineto{\pgfqpoint{3.506392in}{2.658827in}}%
\pgfpathlineto{\pgfqpoint{3.510933in}{2.658827in}}%
\pgfpathlineto{\pgfqpoint{3.510933in}{2.655878in}}%
\pgfpathmoveto{\pgfqpoint{3.497311in}{2.661776in}}%
\pgfpathlineto{\pgfqpoint{3.497311in}{2.661776in}}%
\pgfpathlineto{\pgfqpoint{3.497311in}{2.664725in}}%
\pgfpathlineto{\pgfqpoint{3.501852in}{2.664725in}}%
\pgfpathlineto{\pgfqpoint{3.501852in}{2.661776in}}%
\pgfpathmoveto{\pgfqpoint{3.492770in}{2.667675in}}%
\pgfpathlineto{\pgfqpoint{3.492770in}{2.667675in}}%
\pgfpathlineto{\pgfqpoint{3.492770in}{2.670624in}}%
\pgfpathlineto{\pgfqpoint{3.497311in}{2.670624in}}%
\pgfpathlineto{\pgfqpoint{3.497311in}{2.667675in}}%
\pgfpathmoveto{\pgfqpoint{3.497311in}{2.664725in}}%
\pgfpathlineto{\pgfqpoint{3.497311in}{2.664725in}}%
\pgfpathlineto{\pgfqpoint{3.497311in}{2.667675in}}%
\pgfpathlineto{\pgfqpoint{3.501852in}{2.667675in}}%
\pgfpathlineto{\pgfqpoint{3.501852in}{2.664725in}}%
\pgfpathmoveto{\pgfqpoint{3.497311in}{2.667675in}}%
\pgfpathlineto{\pgfqpoint{3.497311in}{2.667675in}}%
\pgfpathlineto{\pgfqpoint{3.497311in}{2.670624in}}%
\pgfpathlineto{\pgfqpoint{3.501852in}{2.670624in}}%
\pgfpathlineto{\pgfqpoint{3.501852in}{2.667675in}}%
\pgfpathmoveto{\pgfqpoint{3.501852in}{2.658827in}}%
\pgfpathlineto{\pgfqpoint{3.501852in}{2.658827in}}%
\pgfpathlineto{\pgfqpoint{3.501852in}{2.661776in}}%
\pgfpathlineto{\pgfqpoint{3.506392in}{2.661776in}}%
\pgfpathlineto{\pgfqpoint{3.506392in}{2.658827in}}%
\pgfpathmoveto{\pgfqpoint{3.501852in}{2.661776in}}%
\pgfpathlineto{\pgfqpoint{3.501852in}{2.661776in}}%
\pgfpathlineto{\pgfqpoint{3.501852in}{2.664725in}}%
\pgfpathlineto{\pgfqpoint{3.506392in}{2.664725in}}%
\pgfpathlineto{\pgfqpoint{3.506392in}{2.661776in}}%
\pgfpathmoveto{\pgfqpoint{3.433739in}{2.744353in}}%
\pgfpathlineto{\pgfqpoint{3.433739in}{2.744353in}}%
\pgfpathlineto{\pgfqpoint{3.433739in}{2.747303in}}%
\pgfpathlineto{\pgfqpoint{3.438280in}{2.747303in}}%
\pgfpathlineto{\pgfqpoint{3.438280in}{2.744353in}}%
\pgfpathmoveto{\pgfqpoint{3.429199in}{2.750252in}}%
\pgfpathlineto{\pgfqpoint{3.429199in}{2.750252in}}%
\pgfpathlineto{\pgfqpoint{3.429199in}{2.753201in}}%
\pgfpathlineto{\pgfqpoint{3.433739in}{2.753201in}}%
\pgfpathlineto{\pgfqpoint{3.433739in}{2.750252in}}%
\pgfpathmoveto{\pgfqpoint{3.433739in}{2.747303in}}%
\pgfpathlineto{\pgfqpoint{3.433739in}{2.747303in}}%
\pgfpathlineto{\pgfqpoint{3.433739in}{2.750252in}}%
\pgfpathlineto{\pgfqpoint{3.438280in}{2.750252in}}%
\pgfpathlineto{\pgfqpoint{3.438280in}{2.747303in}}%
\pgfpathmoveto{\pgfqpoint{3.433739in}{2.750252in}}%
\pgfpathlineto{\pgfqpoint{3.433739in}{2.750252in}}%
\pgfpathlineto{\pgfqpoint{3.433739in}{2.753201in}}%
\pgfpathlineto{\pgfqpoint{3.438280in}{2.753201in}}%
\pgfpathlineto{\pgfqpoint{3.438280in}{2.750252in}}%
\pgfpathmoveto{\pgfqpoint{3.424658in}{2.756150in}}%
\pgfpathlineto{\pgfqpoint{3.424658in}{2.756150in}}%
\pgfpathlineto{\pgfqpoint{3.424658in}{2.759099in}}%
\pgfpathlineto{\pgfqpoint{3.429199in}{2.759099in}}%
\pgfpathlineto{\pgfqpoint{3.429199in}{2.756150in}}%
\pgfpathmoveto{\pgfqpoint{3.420117in}{2.762048in}}%
\pgfpathlineto{\pgfqpoint{3.420117in}{2.762048in}}%
\pgfpathlineto{\pgfqpoint{3.420117in}{2.764998in}}%
\pgfpathlineto{\pgfqpoint{3.424658in}{2.764998in}}%
\pgfpathlineto{\pgfqpoint{3.424658in}{2.762048in}}%
\pgfpathmoveto{\pgfqpoint{3.424658in}{2.759099in}}%
\pgfpathlineto{\pgfqpoint{3.424658in}{2.759099in}}%
\pgfpathlineto{\pgfqpoint{3.424658in}{2.762048in}}%
\pgfpathlineto{\pgfqpoint{3.429199in}{2.762048in}}%
\pgfpathlineto{\pgfqpoint{3.429199in}{2.759099in}}%
\pgfpathmoveto{\pgfqpoint{3.424658in}{2.762048in}}%
\pgfpathlineto{\pgfqpoint{3.424658in}{2.762048in}}%
\pgfpathlineto{\pgfqpoint{3.424658in}{2.764998in}}%
\pgfpathlineto{\pgfqpoint{3.429199in}{2.764998in}}%
\pgfpathlineto{\pgfqpoint{3.429199in}{2.762048in}}%
\pgfpathmoveto{\pgfqpoint{3.429199in}{2.753201in}}%
\pgfpathlineto{\pgfqpoint{3.429199in}{2.753201in}}%
\pgfpathlineto{\pgfqpoint{3.429199in}{2.756150in}}%
\pgfpathlineto{\pgfqpoint{3.433739in}{2.756150in}}%
\pgfpathlineto{\pgfqpoint{3.433739in}{2.753201in}}%
\pgfpathmoveto{\pgfqpoint{3.429199in}{2.756150in}}%
\pgfpathlineto{\pgfqpoint{3.429199in}{2.756150in}}%
\pgfpathlineto{\pgfqpoint{3.429199in}{2.759099in}}%
\pgfpathlineto{\pgfqpoint{3.433739in}{2.759099in}}%
\pgfpathlineto{\pgfqpoint{3.433739in}{2.756150in}}%
\pgfpathmoveto{\pgfqpoint{3.470066in}{2.697166in}}%
\pgfpathlineto{\pgfqpoint{3.470066in}{2.697166in}}%
\pgfpathlineto{\pgfqpoint{3.470066in}{2.700116in}}%
\pgfpathlineto{\pgfqpoint{3.474607in}{2.700116in}}%
\pgfpathlineto{\pgfqpoint{3.474607in}{2.697166in}}%
\pgfpathmoveto{\pgfqpoint{3.465525in}{2.703065in}}%
\pgfpathlineto{\pgfqpoint{3.465525in}{2.703065in}}%
\pgfpathlineto{\pgfqpoint{3.465525in}{2.706014in}}%
\pgfpathlineto{\pgfqpoint{3.470066in}{2.706014in}}%
\pgfpathlineto{\pgfqpoint{3.470066in}{2.703065in}}%
\pgfpathmoveto{\pgfqpoint{3.470066in}{2.700116in}}%
\pgfpathlineto{\pgfqpoint{3.470066in}{2.700116in}}%
\pgfpathlineto{\pgfqpoint{3.470066in}{2.703065in}}%
\pgfpathlineto{\pgfqpoint{3.474607in}{2.703065in}}%
\pgfpathlineto{\pgfqpoint{3.474607in}{2.700116in}}%
\pgfpathmoveto{\pgfqpoint{3.470066in}{2.703065in}}%
\pgfpathlineto{\pgfqpoint{3.470066in}{2.703065in}}%
\pgfpathlineto{\pgfqpoint{3.470066in}{2.706014in}}%
\pgfpathlineto{\pgfqpoint{3.474607in}{2.706014in}}%
\pgfpathlineto{\pgfqpoint{3.474607in}{2.703065in}}%
\pgfpathmoveto{\pgfqpoint{3.460984in}{2.708963in}}%
\pgfpathlineto{\pgfqpoint{3.460984in}{2.708963in}}%
\pgfpathlineto{\pgfqpoint{3.460984in}{2.711912in}}%
\pgfpathlineto{\pgfqpoint{3.465525in}{2.711912in}}%
\pgfpathlineto{\pgfqpoint{3.465525in}{2.708963in}}%
\pgfpathmoveto{\pgfqpoint{3.456443in}{2.714861in}}%
\pgfpathlineto{\pgfqpoint{3.456443in}{2.714861in}}%
\pgfpathlineto{\pgfqpoint{3.456443in}{2.717811in}}%
\pgfpathlineto{\pgfqpoint{3.460984in}{2.717811in}}%
\pgfpathlineto{\pgfqpoint{3.460984in}{2.714861in}}%
\pgfpathmoveto{\pgfqpoint{3.460984in}{2.711912in}}%
\pgfpathlineto{\pgfqpoint{3.460984in}{2.711912in}}%
\pgfpathlineto{\pgfqpoint{3.460984in}{2.714861in}}%
\pgfpathlineto{\pgfqpoint{3.465525in}{2.714861in}}%
\pgfpathlineto{\pgfqpoint{3.465525in}{2.711912in}}%
\pgfpathmoveto{\pgfqpoint{3.460984in}{2.714861in}}%
\pgfpathlineto{\pgfqpoint{3.460984in}{2.714861in}}%
\pgfpathlineto{\pgfqpoint{3.460984in}{2.717811in}}%
\pgfpathlineto{\pgfqpoint{3.465525in}{2.717811in}}%
\pgfpathlineto{\pgfqpoint{3.465525in}{2.714861in}}%
\pgfpathmoveto{\pgfqpoint{3.465525in}{2.706014in}}%
\pgfpathlineto{\pgfqpoint{3.465525in}{2.706014in}}%
\pgfpathlineto{\pgfqpoint{3.465525in}{2.708963in}}%
\pgfpathlineto{\pgfqpoint{3.470066in}{2.708963in}}%
\pgfpathlineto{\pgfqpoint{3.470066in}{2.706014in}}%
\pgfpathmoveto{\pgfqpoint{3.465525in}{2.708963in}}%
\pgfpathlineto{\pgfqpoint{3.465525in}{2.708963in}}%
\pgfpathlineto{\pgfqpoint{3.465525in}{2.711912in}}%
\pgfpathlineto{\pgfqpoint{3.470066in}{2.711912in}}%
\pgfpathlineto{\pgfqpoint{3.470066in}{2.708963in}}%
\pgfpathmoveto{\pgfqpoint{3.488229in}{2.673573in}}%
\pgfpathlineto{\pgfqpoint{3.488229in}{2.673573in}}%
\pgfpathlineto{\pgfqpoint{3.488229in}{2.676522in}}%
\pgfpathlineto{\pgfqpoint{3.492770in}{2.676522in}}%
\pgfpathlineto{\pgfqpoint{3.492770in}{2.673573in}}%
\pgfpathmoveto{\pgfqpoint{3.483688in}{2.679471in}}%
\pgfpathlineto{\pgfqpoint{3.483688in}{2.679471in}}%
\pgfpathlineto{\pgfqpoint{3.483688in}{2.682420in}}%
\pgfpathlineto{\pgfqpoint{3.488229in}{2.682420in}}%
\pgfpathlineto{\pgfqpoint{3.488229in}{2.679471in}}%
\pgfpathmoveto{\pgfqpoint{3.488229in}{2.676522in}}%
\pgfpathlineto{\pgfqpoint{3.488229in}{2.676522in}}%
\pgfpathlineto{\pgfqpoint{3.488229in}{2.679471in}}%
\pgfpathlineto{\pgfqpoint{3.492770in}{2.679471in}}%
\pgfpathlineto{\pgfqpoint{3.492770in}{2.676522in}}%
\pgfpathmoveto{\pgfqpoint{3.488229in}{2.679471in}}%
\pgfpathlineto{\pgfqpoint{3.488229in}{2.679471in}}%
\pgfpathlineto{\pgfqpoint{3.488229in}{2.682420in}}%
\pgfpathlineto{\pgfqpoint{3.492770in}{2.682420in}}%
\pgfpathlineto{\pgfqpoint{3.492770in}{2.679471in}}%
\pgfpathmoveto{\pgfqpoint{3.479148in}{2.685370in}}%
\pgfpathlineto{\pgfqpoint{3.479148in}{2.685370in}}%
\pgfpathlineto{\pgfqpoint{3.479148in}{2.688319in}}%
\pgfpathlineto{\pgfqpoint{3.483688in}{2.688319in}}%
\pgfpathlineto{\pgfqpoint{3.483688in}{2.685370in}}%
\pgfpathmoveto{\pgfqpoint{3.474607in}{2.691268in}}%
\pgfpathlineto{\pgfqpoint{3.474607in}{2.691268in}}%
\pgfpathlineto{\pgfqpoint{3.474607in}{2.694217in}}%
\pgfpathlineto{\pgfqpoint{3.479148in}{2.694217in}}%
\pgfpathlineto{\pgfqpoint{3.479148in}{2.691268in}}%
\pgfpathmoveto{\pgfqpoint{3.479148in}{2.688319in}}%
\pgfpathlineto{\pgfqpoint{3.479148in}{2.688319in}}%
\pgfpathlineto{\pgfqpoint{3.479148in}{2.691268in}}%
\pgfpathlineto{\pgfqpoint{3.483688in}{2.691268in}}%
\pgfpathlineto{\pgfqpoint{3.483688in}{2.688319in}}%
\pgfpathmoveto{\pgfqpoint{3.479148in}{2.691268in}}%
\pgfpathlineto{\pgfqpoint{3.479148in}{2.691268in}}%
\pgfpathlineto{\pgfqpoint{3.479148in}{2.694217in}}%
\pgfpathlineto{\pgfqpoint{3.483688in}{2.694217in}}%
\pgfpathlineto{\pgfqpoint{3.483688in}{2.691268in}}%
\pgfpathmoveto{\pgfqpoint{3.483688in}{2.682420in}}%
\pgfpathlineto{\pgfqpoint{3.483688in}{2.682420in}}%
\pgfpathlineto{\pgfqpoint{3.483688in}{2.685370in}}%
\pgfpathlineto{\pgfqpoint{3.488229in}{2.685370in}}%
\pgfpathlineto{\pgfqpoint{3.488229in}{2.682420in}}%
\pgfpathmoveto{\pgfqpoint{3.483688in}{2.685370in}}%
\pgfpathlineto{\pgfqpoint{3.483688in}{2.685370in}}%
\pgfpathlineto{\pgfqpoint{3.483688in}{2.688319in}}%
\pgfpathlineto{\pgfqpoint{3.488229in}{2.688319in}}%
\pgfpathlineto{\pgfqpoint{3.488229in}{2.685370in}}%
\pgfpathmoveto{\pgfqpoint{3.492770in}{2.670624in}}%
\pgfpathlineto{\pgfqpoint{3.492770in}{2.670624in}}%
\pgfpathlineto{\pgfqpoint{3.492770in}{2.673573in}}%
\pgfpathlineto{\pgfqpoint{3.497311in}{2.673573in}}%
\pgfpathlineto{\pgfqpoint{3.497311in}{2.670624in}}%
\pgfpathmoveto{\pgfqpoint{3.492770in}{2.673573in}}%
\pgfpathlineto{\pgfqpoint{3.492770in}{2.673573in}}%
\pgfpathlineto{\pgfqpoint{3.492770in}{2.676522in}}%
\pgfpathlineto{\pgfqpoint{3.497311in}{2.676522in}}%
\pgfpathlineto{\pgfqpoint{3.497311in}{2.673573in}}%
\pgfpathmoveto{\pgfqpoint{3.474607in}{2.694217in}}%
\pgfpathlineto{\pgfqpoint{3.474607in}{2.694217in}}%
\pgfpathlineto{\pgfqpoint{3.474607in}{2.697166in}}%
\pgfpathlineto{\pgfqpoint{3.479148in}{2.697166in}}%
\pgfpathlineto{\pgfqpoint{3.479148in}{2.694217in}}%
\pgfpathmoveto{\pgfqpoint{3.474607in}{2.697166in}}%
\pgfpathlineto{\pgfqpoint{3.474607in}{2.697166in}}%
\pgfpathlineto{\pgfqpoint{3.474607in}{2.700116in}}%
\pgfpathlineto{\pgfqpoint{3.479148in}{2.700116in}}%
\pgfpathlineto{\pgfqpoint{3.479148in}{2.697166in}}%
\pgfpathmoveto{\pgfqpoint{3.451903in}{2.720760in}}%
\pgfpathlineto{\pgfqpoint{3.451903in}{2.720760in}}%
\pgfpathlineto{\pgfqpoint{3.451903in}{2.723709in}}%
\pgfpathlineto{\pgfqpoint{3.456443in}{2.723709in}}%
\pgfpathlineto{\pgfqpoint{3.456443in}{2.720760in}}%
\pgfpathmoveto{\pgfqpoint{3.447362in}{2.726658in}}%
\pgfpathlineto{\pgfqpoint{3.447362in}{2.726658in}}%
\pgfpathlineto{\pgfqpoint{3.447362in}{2.729607in}}%
\pgfpathlineto{\pgfqpoint{3.451903in}{2.729607in}}%
\pgfpathlineto{\pgfqpoint{3.451903in}{2.726658in}}%
\pgfpathmoveto{\pgfqpoint{3.451903in}{2.723709in}}%
\pgfpathlineto{\pgfqpoint{3.451903in}{2.723709in}}%
\pgfpathlineto{\pgfqpoint{3.451903in}{2.726658in}}%
\pgfpathlineto{\pgfqpoint{3.456443in}{2.726658in}}%
\pgfpathlineto{\pgfqpoint{3.456443in}{2.723709in}}%
\pgfpathmoveto{\pgfqpoint{3.451903in}{2.726658in}}%
\pgfpathlineto{\pgfqpoint{3.451903in}{2.726658in}}%
\pgfpathlineto{\pgfqpoint{3.451903in}{2.729607in}}%
\pgfpathlineto{\pgfqpoint{3.456443in}{2.729607in}}%
\pgfpathlineto{\pgfqpoint{3.456443in}{2.726658in}}%
\pgfpathmoveto{\pgfqpoint{3.442821in}{2.732557in}}%
\pgfpathlineto{\pgfqpoint{3.442821in}{2.732557in}}%
\pgfpathlineto{\pgfqpoint{3.442821in}{2.735506in}}%
\pgfpathlineto{\pgfqpoint{3.447362in}{2.735506in}}%
\pgfpathlineto{\pgfqpoint{3.447362in}{2.732557in}}%
\pgfpathmoveto{\pgfqpoint{3.438280in}{2.738455in}}%
\pgfpathlineto{\pgfqpoint{3.438280in}{2.738455in}}%
\pgfpathlineto{\pgfqpoint{3.438280in}{2.741404in}}%
\pgfpathlineto{\pgfqpoint{3.442821in}{2.741404in}}%
\pgfpathlineto{\pgfqpoint{3.442821in}{2.738455in}}%
\pgfpathmoveto{\pgfqpoint{3.442821in}{2.735506in}}%
\pgfpathlineto{\pgfqpoint{3.442821in}{2.735506in}}%
\pgfpathlineto{\pgfqpoint{3.442821in}{2.738455in}}%
\pgfpathlineto{\pgfqpoint{3.447362in}{2.738455in}}%
\pgfpathlineto{\pgfqpoint{3.447362in}{2.735506in}}%
\pgfpathmoveto{\pgfqpoint{3.442821in}{2.738455in}}%
\pgfpathlineto{\pgfqpoint{3.442821in}{2.738455in}}%
\pgfpathlineto{\pgfqpoint{3.442821in}{2.741404in}}%
\pgfpathlineto{\pgfqpoint{3.447362in}{2.741404in}}%
\pgfpathlineto{\pgfqpoint{3.447362in}{2.738455in}}%
\pgfpathmoveto{\pgfqpoint{3.447362in}{2.729607in}}%
\pgfpathlineto{\pgfqpoint{3.447362in}{2.729607in}}%
\pgfpathlineto{\pgfqpoint{3.447362in}{2.732557in}}%
\pgfpathlineto{\pgfqpoint{3.451903in}{2.732557in}}%
\pgfpathlineto{\pgfqpoint{3.451903in}{2.729607in}}%
\pgfpathmoveto{\pgfqpoint{3.447362in}{2.732557in}}%
\pgfpathlineto{\pgfqpoint{3.447362in}{2.732557in}}%
\pgfpathlineto{\pgfqpoint{3.447362in}{2.735506in}}%
\pgfpathlineto{\pgfqpoint{3.451903in}{2.735506in}}%
\pgfpathlineto{\pgfqpoint{3.451903in}{2.732557in}}%
\pgfpathmoveto{\pgfqpoint{3.456443in}{2.717811in}}%
\pgfpathlineto{\pgfqpoint{3.456443in}{2.717811in}}%
\pgfpathlineto{\pgfqpoint{3.456443in}{2.720760in}}%
\pgfpathlineto{\pgfqpoint{3.460984in}{2.720760in}}%
\pgfpathlineto{\pgfqpoint{3.460984in}{2.717811in}}%
\pgfpathmoveto{\pgfqpoint{3.456443in}{2.720760in}}%
\pgfpathlineto{\pgfqpoint{3.456443in}{2.720760in}}%
\pgfpathlineto{\pgfqpoint{3.456443in}{2.723709in}}%
\pgfpathlineto{\pgfqpoint{3.460984in}{2.723709in}}%
\pgfpathlineto{\pgfqpoint{3.460984in}{2.720760in}}%
\pgfpathmoveto{\pgfqpoint{3.438280in}{2.741404in}}%
\pgfpathlineto{\pgfqpoint{3.438280in}{2.741404in}}%
\pgfpathlineto{\pgfqpoint{3.438280in}{2.744353in}}%
\pgfpathlineto{\pgfqpoint{3.442821in}{2.744353in}}%
\pgfpathlineto{\pgfqpoint{3.442821in}{2.741404in}}%
\pgfpathmoveto{\pgfqpoint{3.438280in}{2.744353in}}%
\pgfpathlineto{\pgfqpoint{3.438280in}{2.744353in}}%
\pgfpathlineto{\pgfqpoint{3.438280in}{2.747303in}}%
\pgfpathlineto{\pgfqpoint{3.442821in}{2.747303in}}%
\pgfpathlineto{\pgfqpoint{3.442821in}{2.744353in}}%
\pgfpathmoveto{\pgfqpoint{3.397413in}{2.791541in}}%
\pgfpathlineto{\pgfqpoint{3.397413in}{2.791541in}}%
\pgfpathlineto{\pgfqpoint{3.397413in}{2.794491in}}%
\pgfpathlineto{\pgfqpoint{3.401954in}{2.794491in}}%
\pgfpathlineto{\pgfqpoint{3.401954in}{2.791541in}}%
\pgfpathmoveto{\pgfqpoint{3.392872in}{2.797440in}}%
\pgfpathlineto{\pgfqpoint{3.392872in}{2.797440in}}%
\pgfpathlineto{\pgfqpoint{3.392872in}{2.800389in}}%
\pgfpathlineto{\pgfqpoint{3.397413in}{2.800389in}}%
\pgfpathlineto{\pgfqpoint{3.397413in}{2.797440in}}%
\pgfpathmoveto{\pgfqpoint{3.397413in}{2.794491in}}%
\pgfpathlineto{\pgfqpoint{3.397413in}{2.794491in}}%
\pgfpathlineto{\pgfqpoint{3.397413in}{2.797440in}}%
\pgfpathlineto{\pgfqpoint{3.401954in}{2.797440in}}%
\pgfpathlineto{\pgfqpoint{3.401954in}{2.794491in}}%
\pgfpathmoveto{\pgfqpoint{3.397413in}{2.797440in}}%
\pgfpathlineto{\pgfqpoint{3.397413in}{2.797440in}}%
\pgfpathlineto{\pgfqpoint{3.397413in}{2.800389in}}%
\pgfpathlineto{\pgfqpoint{3.401954in}{2.800389in}}%
\pgfpathlineto{\pgfqpoint{3.401954in}{2.797440in}}%
\pgfpathmoveto{\pgfqpoint{3.388331in}{2.803338in}}%
\pgfpathlineto{\pgfqpoint{3.388331in}{2.803338in}}%
\pgfpathlineto{\pgfqpoint{3.388331in}{2.806288in}}%
\pgfpathlineto{\pgfqpoint{3.392872in}{2.806288in}}%
\pgfpathlineto{\pgfqpoint{3.392872in}{2.803338in}}%
\pgfpathmoveto{\pgfqpoint{3.383791in}{2.809237in}}%
\pgfpathlineto{\pgfqpoint{3.383791in}{2.809237in}}%
\pgfpathlineto{\pgfqpoint{3.383791in}{2.812186in}}%
\pgfpathlineto{\pgfqpoint{3.388331in}{2.812186in}}%
\pgfpathlineto{\pgfqpoint{3.388331in}{2.809237in}}%
\pgfpathmoveto{\pgfqpoint{3.388331in}{2.806288in}}%
\pgfpathlineto{\pgfqpoint{3.388331in}{2.806288in}}%
\pgfpathlineto{\pgfqpoint{3.388331in}{2.809237in}}%
\pgfpathlineto{\pgfqpoint{3.392872in}{2.809237in}}%
\pgfpathlineto{\pgfqpoint{3.392872in}{2.806288in}}%
\pgfpathmoveto{\pgfqpoint{3.388331in}{2.809237in}}%
\pgfpathlineto{\pgfqpoint{3.388331in}{2.809237in}}%
\pgfpathlineto{\pgfqpoint{3.388331in}{2.812186in}}%
\pgfpathlineto{\pgfqpoint{3.392872in}{2.812186in}}%
\pgfpathlineto{\pgfqpoint{3.392872in}{2.809237in}}%
\pgfpathmoveto{\pgfqpoint{3.392872in}{2.800389in}}%
\pgfpathlineto{\pgfqpoint{3.392872in}{2.800389in}}%
\pgfpathlineto{\pgfqpoint{3.392872in}{2.803338in}}%
\pgfpathlineto{\pgfqpoint{3.397413in}{2.803338in}}%
\pgfpathlineto{\pgfqpoint{3.397413in}{2.800389in}}%
\pgfpathmoveto{\pgfqpoint{3.392872in}{2.803338in}}%
\pgfpathlineto{\pgfqpoint{3.392872in}{2.803338in}}%
\pgfpathlineto{\pgfqpoint{3.392872in}{2.806288in}}%
\pgfpathlineto{\pgfqpoint{3.397413in}{2.806288in}}%
\pgfpathlineto{\pgfqpoint{3.397413in}{2.803338in}}%
\pgfpathmoveto{\pgfqpoint{3.415576in}{2.767947in}}%
\pgfpathlineto{\pgfqpoint{3.415576in}{2.767947in}}%
\pgfpathlineto{\pgfqpoint{3.415576in}{2.770896in}}%
\pgfpathlineto{\pgfqpoint{3.420117in}{2.770896in}}%
\pgfpathlineto{\pgfqpoint{3.420117in}{2.767947in}}%
\pgfpathmoveto{\pgfqpoint{3.411035in}{2.773845in}}%
\pgfpathlineto{\pgfqpoint{3.411035in}{2.773845in}}%
\pgfpathlineto{\pgfqpoint{3.411035in}{2.776795in}}%
\pgfpathlineto{\pgfqpoint{3.415576in}{2.776795in}}%
\pgfpathlineto{\pgfqpoint{3.415576in}{2.773845in}}%
\pgfpathmoveto{\pgfqpoint{3.415576in}{2.770896in}}%
\pgfpathlineto{\pgfqpoint{3.415576in}{2.770896in}}%
\pgfpathlineto{\pgfqpoint{3.415576in}{2.773845in}}%
\pgfpathlineto{\pgfqpoint{3.420117in}{2.773845in}}%
\pgfpathlineto{\pgfqpoint{3.420117in}{2.770896in}}%
\pgfpathmoveto{\pgfqpoint{3.415576in}{2.773845in}}%
\pgfpathlineto{\pgfqpoint{3.415576in}{2.773845in}}%
\pgfpathlineto{\pgfqpoint{3.415576in}{2.776795in}}%
\pgfpathlineto{\pgfqpoint{3.420117in}{2.776795in}}%
\pgfpathlineto{\pgfqpoint{3.420117in}{2.773845in}}%
\pgfpathmoveto{\pgfqpoint{3.406495in}{2.779744in}}%
\pgfpathlineto{\pgfqpoint{3.406495in}{2.779744in}}%
\pgfpathlineto{\pgfqpoint{3.406495in}{2.782693in}}%
\pgfpathlineto{\pgfqpoint{3.411035in}{2.782693in}}%
\pgfpathlineto{\pgfqpoint{3.411035in}{2.779744in}}%
\pgfpathmoveto{\pgfqpoint{3.401954in}{2.785643in}}%
\pgfpathlineto{\pgfqpoint{3.401954in}{2.785643in}}%
\pgfpathlineto{\pgfqpoint{3.401954in}{2.788592in}}%
\pgfpathlineto{\pgfqpoint{3.406495in}{2.788592in}}%
\pgfpathlineto{\pgfqpoint{3.406495in}{2.785643in}}%
\pgfpathmoveto{\pgfqpoint{3.406495in}{2.782693in}}%
\pgfpathlineto{\pgfqpoint{3.406495in}{2.782693in}}%
\pgfpathlineto{\pgfqpoint{3.406495in}{2.785643in}}%
\pgfpathlineto{\pgfqpoint{3.411035in}{2.785643in}}%
\pgfpathlineto{\pgfqpoint{3.411035in}{2.782693in}}%
\pgfpathmoveto{\pgfqpoint{3.406495in}{2.785643in}}%
\pgfpathlineto{\pgfqpoint{3.406495in}{2.785643in}}%
\pgfpathlineto{\pgfqpoint{3.406495in}{2.788592in}}%
\pgfpathlineto{\pgfqpoint{3.411035in}{2.788592in}}%
\pgfpathlineto{\pgfqpoint{3.411035in}{2.785643in}}%
\pgfpathmoveto{\pgfqpoint{3.411035in}{2.776795in}}%
\pgfpathlineto{\pgfqpoint{3.411035in}{2.776795in}}%
\pgfpathlineto{\pgfqpoint{3.411035in}{2.779744in}}%
\pgfpathlineto{\pgfqpoint{3.415576in}{2.779744in}}%
\pgfpathlineto{\pgfqpoint{3.415576in}{2.776795in}}%
\pgfpathmoveto{\pgfqpoint{3.411035in}{2.779744in}}%
\pgfpathlineto{\pgfqpoint{3.411035in}{2.779744in}}%
\pgfpathlineto{\pgfqpoint{3.411035in}{2.782693in}}%
\pgfpathlineto{\pgfqpoint{3.415576in}{2.782693in}}%
\pgfpathlineto{\pgfqpoint{3.415576in}{2.779744in}}%
\pgfpathmoveto{\pgfqpoint{3.420117in}{2.764998in}}%
\pgfpathlineto{\pgfqpoint{3.420117in}{2.764998in}}%
\pgfpathlineto{\pgfqpoint{3.420117in}{2.767947in}}%
\pgfpathlineto{\pgfqpoint{3.424658in}{2.767947in}}%
\pgfpathlineto{\pgfqpoint{3.424658in}{2.764998in}}%
\pgfpathmoveto{\pgfqpoint{3.420117in}{2.767947in}}%
\pgfpathlineto{\pgfqpoint{3.420117in}{2.767947in}}%
\pgfpathlineto{\pgfqpoint{3.420117in}{2.770896in}}%
\pgfpathlineto{\pgfqpoint{3.424658in}{2.770896in}}%
\pgfpathlineto{\pgfqpoint{3.424658in}{2.767947in}}%
\pgfpathmoveto{\pgfqpoint{3.401954in}{2.788592in}}%
\pgfpathlineto{\pgfqpoint{3.401954in}{2.788592in}}%
\pgfpathlineto{\pgfqpoint{3.401954in}{2.791541in}}%
\pgfpathlineto{\pgfqpoint{3.406495in}{2.791541in}}%
\pgfpathlineto{\pgfqpoint{3.406495in}{2.788592in}}%
\pgfpathmoveto{\pgfqpoint{3.401954in}{2.791541in}}%
\pgfpathlineto{\pgfqpoint{3.401954in}{2.791541in}}%
\pgfpathlineto{\pgfqpoint{3.401954in}{2.794491in}}%
\pgfpathlineto{\pgfqpoint{3.406495in}{2.794491in}}%
\pgfpathlineto{\pgfqpoint{3.406495in}{2.791541in}}%
\pgfpathmoveto{\pgfqpoint{3.379250in}{2.815136in}}%
\pgfpathlineto{\pgfqpoint{3.379250in}{2.815136in}}%
\pgfpathlineto{\pgfqpoint{3.379250in}{2.818085in}}%
\pgfpathlineto{\pgfqpoint{3.383791in}{2.818085in}}%
\pgfpathlineto{\pgfqpoint{3.383791in}{2.815136in}}%
\pgfpathmoveto{\pgfqpoint{3.374709in}{2.821034in}}%
\pgfpathlineto{\pgfqpoint{3.374709in}{2.821034in}}%
\pgfpathlineto{\pgfqpoint{3.374709in}{2.823983in}}%
\pgfpathlineto{\pgfqpoint{3.379250in}{2.823983in}}%
\pgfpathlineto{\pgfqpoint{3.379250in}{2.821034in}}%
\pgfpathmoveto{\pgfqpoint{3.379250in}{2.818085in}}%
\pgfpathlineto{\pgfqpoint{3.379250in}{2.818085in}}%
\pgfpathlineto{\pgfqpoint{3.379250in}{2.821034in}}%
\pgfpathlineto{\pgfqpoint{3.383791in}{2.821034in}}%
\pgfpathlineto{\pgfqpoint{3.383791in}{2.818085in}}%
\pgfpathmoveto{\pgfqpoint{3.379250in}{2.821034in}}%
\pgfpathlineto{\pgfqpoint{3.379250in}{2.821034in}}%
\pgfpathlineto{\pgfqpoint{3.379250in}{2.823983in}}%
\pgfpathlineto{\pgfqpoint{3.383791in}{2.823983in}}%
\pgfpathlineto{\pgfqpoint{3.383791in}{2.821034in}}%
\pgfpathmoveto{\pgfqpoint{3.370168in}{2.826933in}}%
\pgfpathlineto{\pgfqpoint{3.370168in}{2.826933in}}%
\pgfpathlineto{\pgfqpoint{3.370168in}{2.829882in}}%
\pgfpathlineto{\pgfqpoint{3.374709in}{2.829882in}}%
\pgfpathlineto{\pgfqpoint{3.374709in}{2.826933in}}%
\pgfpathmoveto{\pgfqpoint{3.365627in}{2.832831in}}%
\pgfpathlineto{\pgfqpoint{3.365627in}{2.832831in}}%
\pgfpathlineto{\pgfqpoint{3.365627in}{2.835781in}}%
\pgfpathlineto{\pgfqpoint{3.370168in}{2.835781in}}%
\pgfpathlineto{\pgfqpoint{3.370168in}{2.832831in}}%
\pgfpathmoveto{\pgfqpoint{3.370168in}{2.829882in}}%
\pgfpathlineto{\pgfqpoint{3.370168in}{2.829882in}}%
\pgfpathlineto{\pgfqpoint{3.370168in}{2.832831in}}%
\pgfpathlineto{\pgfqpoint{3.374709in}{2.832831in}}%
\pgfpathlineto{\pgfqpoint{3.374709in}{2.829882in}}%
\pgfpathmoveto{\pgfqpoint{3.370168in}{2.832831in}}%
\pgfpathlineto{\pgfqpoint{3.370168in}{2.832831in}}%
\pgfpathlineto{\pgfqpoint{3.370168in}{2.835781in}}%
\pgfpathlineto{\pgfqpoint{3.374709in}{2.835781in}}%
\pgfpathlineto{\pgfqpoint{3.374709in}{2.832831in}}%
\pgfpathmoveto{\pgfqpoint{3.374709in}{2.823983in}}%
\pgfpathlineto{\pgfqpoint{3.374709in}{2.823983in}}%
\pgfpathlineto{\pgfqpoint{3.374709in}{2.826933in}}%
\pgfpathlineto{\pgfqpoint{3.379250in}{2.826933in}}%
\pgfpathlineto{\pgfqpoint{3.379250in}{2.823983in}}%
\pgfpathmoveto{\pgfqpoint{3.374709in}{2.826933in}}%
\pgfpathlineto{\pgfqpoint{3.374709in}{2.826933in}}%
\pgfpathlineto{\pgfqpoint{3.374709in}{2.829882in}}%
\pgfpathlineto{\pgfqpoint{3.379250in}{2.829882in}}%
\pgfpathlineto{\pgfqpoint{3.379250in}{2.826933in}}%
\pgfpathmoveto{\pgfqpoint{3.383791in}{2.812186in}}%
\pgfpathlineto{\pgfqpoint{3.383791in}{2.812186in}}%
\pgfpathlineto{\pgfqpoint{3.383791in}{2.815136in}}%
\pgfpathlineto{\pgfqpoint{3.388331in}{2.815136in}}%
\pgfpathlineto{\pgfqpoint{3.388331in}{2.812186in}}%
\pgfpathmoveto{\pgfqpoint{3.383791in}{2.815136in}}%
\pgfpathlineto{\pgfqpoint{3.383791in}{2.815136in}}%
\pgfpathlineto{\pgfqpoint{3.383791in}{2.818085in}}%
\pgfpathlineto{\pgfqpoint{3.388331in}{2.818085in}}%
\pgfpathlineto{\pgfqpoint{3.388331in}{2.815136in}}%
\pgfpathmoveto{\pgfqpoint{3.365627in}{2.835781in}}%
\pgfpathlineto{\pgfqpoint{3.365627in}{2.835781in}}%
\pgfpathlineto{\pgfqpoint{3.365627in}{2.838730in}}%
\pgfpathlineto{\pgfqpoint{3.370168in}{2.838730in}}%
\pgfpathlineto{\pgfqpoint{3.370168in}{2.835781in}}%
\pgfpathmoveto{\pgfqpoint{3.365627in}{2.838730in}}%
\pgfpathlineto{\pgfqpoint{3.365627in}{2.838730in}}%
\pgfpathlineto{\pgfqpoint{3.365627in}{2.841679in}}%
\pgfpathlineto{\pgfqpoint{3.370168in}{2.841679in}}%
\pgfpathlineto{\pgfqpoint{3.370168in}{2.838730in}}%
\pgfpathmoveto{\pgfqpoint{3.651706in}{2.461229in}}%
\pgfpathlineto{\pgfqpoint{3.651706in}{2.461229in}}%
\pgfpathlineto{\pgfqpoint{3.651706in}{2.464178in}}%
\pgfpathlineto{\pgfqpoint{3.656247in}{2.464178in}}%
\pgfpathlineto{\pgfqpoint{3.656247in}{2.461229in}}%
\pgfpathmoveto{\pgfqpoint{3.647165in}{2.467127in}}%
\pgfpathlineto{\pgfqpoint{3.647165in}{2.467127in}}%
\pgfpathlineto{\pgfqpoint{3.647165in}{2.470076in}}%
\pgfpathlineto{\pgfqpoint{3.651706in}{2.470076in}}%
\pgfpathlineto{\pgfqpoint{3.651706in}{2.467127in}}%
\pgfpathmoveto{\pgfqpoint{3.651706in}{2.464178in}}%
\pgfpathlineto{\pgfqpoint{3.651706in}{2.464178in}}%
\pgfpathlineto{\pgfqpoint{3.651706in}{2.467127in}}%
\pgfpathlineto{\pgfqpoint{3.656247in}{2.467127in}}%
\pgfpathlineto{\pgfqpoint{3.656247in}{2.464178in}}%
\pgfpathmoveto{\pgfqpoint{3.651706in}{2.467127in}}%
\pgfpathlineto{\pgfqpoint{3.651706in}{2.467127in}}%
\pgfpathlineto{\pgfqpoint{3.651706in}{2.470076in}}%
\pgfpathlineto{\pgfqpoint{3.656247in}{2.470076in}}%
\pgfpathlineto{\pgfqpoint{3.656247in}{2.467127in}}%
\pgfpathmoveto{\pgfqpoint{3.642624in}{2.473025in}}%
\pgfpathlineto{\pgfqpoint{3.642624in}{2.473025in}}%
\pgfpathlineto{\pgfqpoint{3.642624in}{2.475974in}}%
\pgfpathlineto{\pgfqpoint{3.647165in}{2.475974in}}%
\pgfpathlineto{\pgfqpoint{3.647165in}{2.473025in}}%
\pgfpathmoveto{\pgfqpoint{3.638083in}{2.478923in}}%
\pgfpathlineto{\pgfqpoint{3.638083in}{2.478923in}}%
\pgfpathlineto{\pgfqpoint{3.638083in}{2.481872in}}%
\pgfpathlineto{\pgfqpoint{3.642624in}{2.481872in}}%
\pgfpathlineto{\pgfqpoint{3.642624in}{2.478923in}}%
\pgfpathmoveto{\pgfqpoint{3.642624in}{2.475974in}}%
\pgfpathlineto{\pgfqpoint{3.642624in}{2.475974in}}%
\pgfpathlineto{\pgfqpoint{3.642624in}{2.478923in}}%
\pgfpathlineto{\pgfqpoint{3.647165in}{2.478923in}}%
\pgfpathlineto{\pgfqpoint{3.647165in}{2.475974in}}%
\pgfpathmoveto{\pgfqpoint{3.642624in}{2.478923in}}%
\pgfpathlineto{\pgfqpoint{3.642624in}{2.478923in}}%
\pgfpathlineto{\pgfqpoint{3.642624in}{2.481872in}}%
\pgfpathlineto{\pgfqpoint{3.647165in}{2.481872in}}%
\pgfpathlineto{\pgfqpoint{3.647165in}{2.478923in}}%
\pgfpathmoveto{\pgfqpoint{3.647165in}{2.470076in}}%
\pgfpathlineto{\pgfqpoint{3.647165in}{2.470076in}}%
\pgfpathlineto{\pgfqpoint{3.647165in}{2.473025in}}%
\pgfpathlineto{\pgfqpoint{3.651706in}{2.473025in}}%
\pgfpathlineto{\pgfqpoint{3.651706in}{2.470076in}}%
\pgfpathmoveto{\pgfqpoint{3.647165in}{2.473025in}}%
\pgfpathlineto{\pgfqpoint{3.647165in}{2.473025in}}%
\pgfpathlineto{\pgfqpoint{3.647165in}{2.475974in}}%
\pgfpathlineto{\pgfqpoint{3.651706in}{2.475974in}}%
\pgfpathlineto{\pgfqpoint{3.651706in}{2.473025in}}%
\pgfpathmoveto{\pgfqpoint{3.579049in}{2.555606in}}%
\pgfpathlineto{\pgfqpoint{3.579049in}{2.555606in}}%
\pgfpathlineto{\pgfqpoint{3.579049in}{2.558556in}}%
\pgfpathlineto{\pgfqpoint{3.583590in}{2.558556in}}%
\pgfpathlineto{\pgfqpoint{3.583590in}{2.555606in}}%
\pgfpathmoveto{\pgfqpoint{3.574508in}{2.561505in}}%
\pgfpathlineto{\pgfqpoint{3.574508in}{2.561505in}}%
\pgfpathlineto{\pgfqpoint{3.574508in}{2.564454in}}%
\pgfpathlineto{\pgfqpoint{3.579049in}{2.564454in}}%
\pgfpathlineto{\pgfqpoint{3.579049in}{2.561505in}}%
\pgfpathmoveto{\pgfqpoint{3.579049in}{2.558556in}}%
\pgfpathlineto{\pgfqpoint{3.579049in}{2.558556in}}%
\pgfpathlineto{\pgfqpoint{3.579049in}{2.561505in}}%
\pgfpathlineto{\pgfqpoint{3.583590in}{2.561505in}}%
\pgfpathlineto{\pgfqpoint{3.583590in}{2.558556in}}%
\pgfpathmoveto{\pgfqpoint{3.579049in}{2.561505in}}%
\pgfpathlineto{\pgfqpoint{3.579049in}{2.561505in}}%
\pgfpathlineto{\pgfqpoint{3.579049in}{2.564454in}}%
\pgfpathlineto{\pgfqpoint{3.583590in}{2.564454in}}%
\pgfpathlineto{\pgfqpoint{3.583590in}{2.561505in}}%
\pgfpathmoveto{\pgfqpoint{3.569967in}{2.567404in}}%
\pgfpathlineto{\pgfqpoint{3.569967in}{2.567404in}}%
\pgfpathlineto{\pgfqpoint{3.569967in}{2.570353in}}%
\pgfpathlineto{\pgfqpoint{3.574508in}{2.570353in}}%
\pgfpathlineto{\pgfqpoint{3.574508in}{2.567404in}}%
\pgfpathmoveto{\pgfqpoint{3.565426in}{2.573302in}}%
\pgfpathlineto{\pgfqpoint{3.565426in}{2.573302in}}%
\pgfpathlineto{\pgfqpoint{3.565426in}{2.576252in}}%
\pgfpathlineto{\pgfqpoint{3.569967in}{2.576252in}}%
\pgfpathlineto{\pgfqpoint{3.569967in}{2.573302in}}%
\pgfpathmoveto{\pgfqpoint{3.569967in}{2.570353in}}%
\pgfpathlineto{\pgfqpoint{3.569967in}{2.570353in}}%
\pgfpathlineto{\pgfqpoint{3.569967in}{2.573302in}}%
\pgfpathlineto{\pgfqpoint{3.574508in}{2.573302in}}%
\pgfpathlineto{\pgfqpoint{3.574508in}{2.570353in}}%
\pgfpathmoveto{\pgfqpoint{3.569967in}{2.573302in}}%
\pgfpathlineto{\pgfqpoint{3.569967in}{2.573302in}}%
\pgfpathlineto{\pgfqpoint{3.569967in}{2.576252in}}%
\pgfpathlineto{\pgfqpoint{3.574508in}{2.576252in}}%
\pgfpathlineto{\pgfqpoint{3.574508in}{2.573302in}}%
\pgfpathmoveto{\pgfqpoint{3.574508in}{2.564454in}}%
\pgfpathlineto{\pgfqpoint{3.574508in}{2.564454in}}%
\pgfpathlineto{\pgfqpoint{3.574508in}{2.567404in}}%
\pgfpathlineto{\pgfqpoint{3.579049in}{2.567404in}}%
\pgfpathlineto{\pgfqpoint{3.579049in}{2.564454in}}%
\pgfpathmoveto{\pgfqpoint{3.574508in}{2.567404in}}%
\pgfpathlineto{\pgfqpoint{3.574508in}{2.567404in}}%
\pgfpathlineto{\pgfqpoint{3.574508in}{2.570353in}}%
\pgfpathlineto{\pgfqpoint{3.579049in}{2.570353in}}%
\pgfpathlineto{\pgfqpoint{3.579049in}{2.567404in}}%
\pgfpathmoveto{\pgfqpoint{3.615378in}{2.508416in}}%
\pgfpathlineto{\pgfqpoint{3.615378in}{2.508416in}}%
\pgfpathlineto{\pgfqpoint{3.615378in}{2.511366in}}%
\pgfpathlineto{\pgfqpoint{3.619919in}{2.511366in}}%
\pgfpathlineto{\pgfqpoint{3.619919in}{2.508416in}}%
\pgfpathmoveto{\pgfqpoint{3.610837in}{2.514315in}}%
\pgfpathlineto{\pgfqpoint{3.610837in}{2.514315in}}%
\pgfpathlineto{\pgfqpoint{3.610837in}{2.517265in}}%
\pgfpathlineto{\pgfqpoint{3.615378in}{2.517265in}}%
\pgfpathlineto{\pgfqpoint{3.615378in}{2.514315in}}%
\pgfpathmoveto{\pgfqpoint{3.615378in}{2.511366in}}%
\pgfpathlineto{\pgfqpoint{3.615378in}{2.511366in}}%
\pgfpathlineto{\pgfqpoint{3.615378in}{2.514315in}}%
\pgfpathlineto{\pgfqpoint{3.619919in}{2.514315in}}%
\pgfpathlineto{\pgfqpoint{3.619919in}{2.511366in}}%
\pgfpathmoveto{\pgfqpoint{3.615378in}{2.514315in}}%
\pgfpathlineto{\pgfqpoint{3.615378in}{2.514315in}}%
\pgfpathlineto{\pgfqpoint{3.615378in}{2.517265in}}%
\pgfpathlineto{\pgfqpoint{3.619919in}{2.517265in}}%
\pgfpathlineto{\pgfqpoint{3.619919in}{2.514315in}}%
\pgfpathmoveto{\pgfqpoint{3.606296in}{2.520214in}}%
\pgfpathlineto{\pgfqpoint{3.606296in}{2.520214in}}%
\pgfpathlineto{\pgfqpoint{3.606296in}{2.523163in}}%
\pgfpathlineto{\pgfqpoint{3.610837in}{2.523163in}}%
\pgfpathlineto{\pgfqpoint{3.610837in}{2.520214in}}%
\pgfpathmoveto{\pgfqpoint{3.601754in}{2.526113in}}%
\pgfpathlineto{\pgfqpoint{3.601754in}{2.526113in}}%
\pgfpathlineto{\pgfqpoint{3.601754in}{2.529062in}}%
\pgfpathlineto{\pgfqpoint{3.606296in}{2.529062in}}%
\pgfpathlineto{\pgfqpoint{3.606296in}{2.526113in}}%
\pgfpathmoveto{\pgfqpoint{3.606296in}{2.523163in}}%
\pgfpathlineto{\pgfqpoint{3.606296in}{2.523163in}}%
\pgfpathlineto{\pgfqpoint{3.606296in}{2.526113in}}%
\pgfpathlineto{\pgfqpoint{3.610837in}{2.526113in}}%
\pgfpathlineto{\pgfqpoint{3.610837in}{2.523163in}}%
\pgfpathmoveto{\pgfqpoint{3.606296in}{2.526113in}}%
\pgfpathlineto{\pgfqpoint{3.606296in}{2.526113in}}%
\pgfpathlineto{\pgfqpoint{3.606296in}{2.529062in}}%
\pgfpathlineto{\pgfqpoint{3.610837in}{2.529062in}}%
\pgfpathlineto{\pgfqpoint{3.610837in}{2.526113in}}%
\pgfpathmoveto{\pgfqpoint{3.610837in}{2.517265in}}%
\pgfpathlineto{\pgfqpoint{3.610837in}{2.517265in}}%
\pgfpathlineto{\pgfqpoint{3.610837in}{2.520214in}}%
\pgfpathlineto{\pgfqpoint{3.615378in}{2.520214in}}%
\pgfpathlineto{\pgfqpoint{3.615378in}{2.517265in}}%
\pgfpathmoveto{\pgfqpoint{3.610837in}{2.520214in}}%
\pgfpathlineto{\pgfqpoint{3.610837in}{2.520214in}}%
\pgfpathlineto{\pgfqpoint{3.610837in}{2.523163in}}%
\pgfpathlineto{\pgfqpoint{3.615378in}{2.523163in}}%
\pgfpathlineto{\pgfqpoint{3.615378in}{2.520214in}}%
\pgfpathmoveto{\pgfqpoint{3.633542in}{2.484822in}}%
\pgfpathlineto{\pgfqpoint{3.633542in}{2.484822in}}%
\pgfpathlineto{\pgfqpoint{3.633542in}{2.487771in}}%
\pgfpathlineto{\pgfqpoint{3.638083in}{2.487771in}}%
\pgfpathlineto{\pgfqpoint{3.638083in}{2.484822in}}%
\pgfpathmoveto{\pgfqpoint{3.629001in}{2.490720in}}%
\pgfpathlineto{\pgfqpoint{3.629001in}{2.490720in}}%
\pgfpathlineto{\pgfqpoint{3.629001in}{2.493670in}}%
\pgfpathlineto{\pgfqpoint{3.633542in}{2.493670in}}%
\pgfpathlineto{\pgfqpoint{3.633542in}{2.490720in}}%
\pgfpathmoveto{\pgfqpoint{3.633542in}{2.487771in}}%
\pgfpathlineto{\pgfqpoint{3.633542in}{2.487771in}}%
\pgfpathlineto{\pgfqpoint{3.633542in}{2.490720in}}%
\pgfpathlineto{\pgfqpoint{3.638083in}{2.490720in}}%
\pgfpathlineto{\pgfqpoint{3.638083in}{2.487771in}}%
\pgfpathmoveto{\pgfqpoint{3.633542in}{2.490720in}}%
\pgfpathlineto{\pgfqpoint{3.633542in}{2.490720in}}%
\pgfpathlineto{\pgfqpoint{3.633542in}{2.493670in}}%
\pgfpathlineto{\pgfqpoint{3.638083in}{2.493670in}}%
\pgfpathlineto{\pgfqpoint{3.638083in}{2.490720in}}%
\pgfpathmoveto{\pgfqpoint{3.624460in}{2.496619in}}%
\pgfpathlineto{\pgfqpoint{3.624460in}{2.496619in}}%
\pgfpathlineto{\pgfqpoint{3.624460in}{2.499568in}}%
\pgfpathlineto{\pgfqpoint{3.629001in}{2.499568in}}%
\pgfpathlineto{\pgfqpoint{3.629001in}{2.496619in}}%
\pgfpathmoveto{\pgfqpoint{3.619919in}{2.502518in}}%
\pgfpathlineto{\pgfqpoint{3.619919in}{2.502518in}}%
\pgfpathlineto{\pgfqpoint{3.619919in}{2.505467in}}%
\pgfpathlineto{\pgfqpoint{3.624460in}{2.505467in}}%
\pgfpathlineto{\pgfqpoint{3.624460in}{2.502518in}}%
\pgfpathmoveto{\pgfqpoint{3.624460in}{2.499568in}}%
\pgfpathlineto{\pgfqpoint{3.624460in}{2.499568in}}%
\pgfpathlineto{\pgfqpoint{3.624460in}{2.502518in}}%
\pgfpathlineto{\pgfqpoint{3.629001in}{2.502518in}}%
\pgfpathlineto{\pgfqpoint{3.629001in}{2.499568in}}%
\pgfpathmoveto{\pgfqpoint{3.624460in}{2.502518in}}%
\pgfpathlineto{\pgfqpoint{3.624460in}{2.502518in}}%
\pgfpathlineto{\pgfqpoint{3.624460in}{2.505467in}}%
\pgfpathlineto{\pgfqpoint{3.629001in}{2.505467in}}%
\pgfpathlineto{\pgfqpoint{3.629001in}{2.502518in}}%
\pgfpathmoveto{\pgfqpoint{3.629001in}{2.493670in}}%
\pgfpathlineto{\pgfqpoint{3.629001in}{2.493670in}}%
\pgfpathlineto{\pgfqpoint{3.629001in}{2.496619in}}%
\pgfpathlineto{\pgfqpoint{3.633542in}{2.496619in}}%
\pgfpathlineto{\pgfqpoint{3.633542in}{2.493670in}}%
\pgfpathmoveto{\pgfqpoint{3.629001in}{2.496619in}}%
\pgfpathlineto{\pgfqpoint{3.629001in}{2.496619in}}%
\pgfpathlineto{\pgfqpoint{3.629001in}{2.499568in}}%
\pgfpathlineto{\pgfqpoint{3.633542in}{2.499568in}}%
\pgfpathlineto{\pgfqpoint{3.633542in}{2.496619in}}%
\pgfpathmoveto{\pgfqpoint{3.638083in}{2.481872in}}%
\pgfpathlineto{\pgfqpoint{3.638083in}{2.481872in}}%
\pgfpathlineto{\pgfqpoint{3.638083in}{2.484822in}}%
\pgfpathlineto{\pgfqpoint{3.642624in}{2.484822in}}%
\pgfpathlineto{\pgfqpoint{3.642624in}{2.481872in}}%
\pgfpathmoveto{\pgfqpoint{3.638083in}{2.484822in}}%
\pgfpathlineto{\pgfqpoint{3.638083in}{2.484822in}}%
\pgfpathlineto{\pgfqpoint{3.638083in}{2.487771in}}%
\pgfpathlineto{\pgfqpoint{3.642624in}{2.487771in}}%
\pgfpathlineto{\pgfqpoint{3.642624in}{2.484822in}}%
\pgfpathmoveto{\pgfqpoint{3.619919in}{2.505467in}}%
\pgfpathlineto{\pgfqpoint{3.619919in}{2.505467in}}%
\pgfpathlineto{\pgfqpoint{3.619919in}{2.508416in}}%
\pgfpathlineto{\pgfqpoint{3.624460in}{2.508416in}}%
\pgfpathlineto{\pgfqpoint{3.624460in}{2.505467in}}%
\pgfpathmoveto{\pgfqpoint{3.619919in}{2.508416in}}%
\pgfpathlineto{\pgfqpoint{3.619919in}{2.508416in}}%
\pgfpathlineto{\pgfqpoint{3.619919in}{2.511366in}}%
\pgfpathlineto{\pgfqpoint{3.624460in}{2.511366in}}%
\pgfpathlineto{\pgfqpoint{3.624460in}{2.508416in}}%
\pgfpathmoveto{\pgfqpoint{3.597213in}{2.532011in}}%
\pgfpathlineto{\pgfqpoint{3.597213in}{2.532011in}}%
\pgfpathlineto{\pgfqpoint{3.597213in}{2.534961in}}%
\pgfpathlineto{\pgfqpoint{3.601754in}{2.534961in}}%
\pgfpathlineto{\pgfqpoint{3.601754in}{2.532011in}}%
\pgfpathmoveto{\pgfqpoint{3.592672in}{2.537910in}}%
\pgfpathlineto{\pgfqpoint{3.592672in}{2.537910in}}%
\pgfpathlineto{\pgfqpoint{3.592672in}{2.540859in}}%
\pgfpathlineto{\pgfqpoint{3.597213in}{2.540859in}}%
\pgfpathlineto{\pgfqpoint{3.597213in}{2.537910in}}%
\pgfpathmoveto{\pgfqpoint{3.597213in}{2.534961in}}%
\pgfpathlineto{\pgfqpoint{3.597213in}{2.534961in}}%
\pgfpathlineto{\pgfqpoint{3.597213in}{2.537910in}}%
\pgfpathlineto{\pgfqpoint{3.601754in}{2.537910in}}%
\pgfpathlineto{\pgfqpoint{3.601754in}{2.534961in}}%
\pgfpathmoveto{\pgfqpoint{3.597213in}{2.537910in}}%
\pgfpathlineto{\pgfqpoint{3.597213in}{2.537910in}}%
\pgfpathlineto{\pgfqpoint{3.597213in}{2.540859in}}%
\pgfpathlineto{\pgfqpoint{3.601754in}{2.540859in}}%
\pgfpathlineto{\pgfqpoint{3.601754in}{2.537910in}}%
\pgfpathmoveto{\pgfqpoint{3.588131in}{2.543809in}}%
\pgfpathlineto{\pgfqpoint{3.588131in}{2.543809in}}%
\pgfpathlineto{\pgfqpoint{3.588131in}{2.546758in}}%
\pgfpathlineto{\pgfqpoint{3.592672in}{2.546758in}}%
\pgfpathlineto{\pgfqpoint{3.592672in}{2.543809in}}%
\pgfpathmoveto{\pgfqpoint{3.583590in}{2.549707in}}%
\pgfpathlineto{\pgfqpoint{3.583590in}{2.549707in}}%
\pgfpathlineto{\pgfqpoint{3.583590in}{2.552657in}}%
\pgfpathlineto{\pgfqpoint{3.588131in}{2.552657in}}%
\pgfpathlineto{\pgfqpoint{3.588131in}{2.549707in}}%
\pgfpathmoveto{\pgfqpoint{3.588131in}{2.546758in}}%
\pgfpathlineto{\pgfqpoint{3.588131in}{2.546758in}}%
\pgfpathlineto{\pgfqpoint{3.588131in}{2.549707in}}%
\pgfpathlineto{\pgfqpoint{3.592672in}{2.549707in}}%
\pgfpathlineto{\pgfqpoint{3.592672in}{2.546758in}}%
\pgfpathmoveto{\pgfqpoint{3.588131in}{2.549707in}}%
\pgfpathlineto{\pgfqpoint{3.588131in}{2.549707in}}%
\pgfpathlineto{\pgfqpoint{3.588131in}{2.552657in}}%
\pgfpathlineto{\pgfqpoint{3.592672in}{2.552657in}}%
\pgfpathlineto{\pgfqpoint{3.592672in}{2.549707in}}%
\pgfpathmoveto{\pgfqpoint{3.592672in}{2.540859in}}%
\pgfpathlineto{\pgfqpoint{3.592672in}{2.540859in}}%
\pgfpathlineto{\pgfqpoint{3.592672in}{2.543809in}}%
\pgfpathlineto{\pgfqpoint{3.597213in}{2.543809in}}%
\pgfpathlineto{\pgfqpoint{3.597213in}{2.540859in}}%
\pgfpathmoveto{\pgfqpoint{3.592672in}{2.543809in}}%
\pgfpathlineto{\pgfqpoint{3.592672in}{2.543809in}}%
\pgfpathlineto{\pgfqpoint{3.592672in}{2.546758in}}%
\pgfpathlineto{\pgfqpoint{3.597213in}{2.546758in}}%
\pgfpathlineto{\pgfqpoint{3.597213in}{2.543809in}}%
\pgfpathmoveto{\pgfqpoint{3.601754in}{2.529062in}}%
\pgfpathlineto{\pgfqpoint{3.601754in}{2.529062in}}%
\pgfpathlineto{\pgfqpoint{3.601754in}{2.532011in}}%
\pgfpathlineto{\pgfqpoint{3.606296in}{2.532011in}}%
\pgfpathlineto{\pgfqpoint{3.606296in}{2.529062in}}%
\pgfpathmoveto{\pgfqpoint{3.601754in}{2.532011in}}%
\pgfpathlineto{\pgfqpoint{3.601754in}{2.532011in}}%
\pgfpathlineto{\pgfqpoint{3.601754in}{2.534961in}}%
\pgfpathlineto{\pgfqpoint{3.606296in}{2.534961in}}%
\pgfpathlineto{\pgfqpoint{3.606296in}{2.532011in}}%
\pgfpathmoveto{\pgfqpoint{3.583590in}{2.552657in}}%
\pgfpathlineto{\pgfqpoint{3.583590in}{2.552657in}}%
\pgfpathlineto{\pgfqpoint{3.583590in}{2.555606in}}%
\pgfpathlineto{\pgfqpoint{3.588131in}{2.555606in}}%
\pgfpathlineto{\pgfqpoint{3.588131in}{2.552657in}}%
\pgfpathmoveto{\pgfqpoint{3.583590in}{2.555606in}}%
\pgfpathlineto{\pgfqpoint{3.583590in}{2.555606in}}%
\pgfpathlineto{\pgfqpoint{3.583590in}{2.558556in}}%
\pgfpathlineto{\pgfqpoint{3.588131in}{2.558556in}}%
\pgfpathlineto{\pgfqpoint{3.588131in}{2.555606in}}%
\pgfpathmoveto{\pgfqpoint{3.542721in}{2.602794in}}%
\pgfpathlineto{\pgfqpoint{3.542721in}{2.602794in}}%
\pgfpathlineto{\pgfqpoint{3.542721in}{2.605743in}}%
\pgfpathlineto{\pgfqpoint{3.547262in}{2.605743in}}%
\pgfpathlineto{\pgfqpoint{3.547262in}{2.602794in}}%
\pgfpathmoveto{\pgfqpoint{3.538180in}{2.608692in}}%
\pgfpathlineto{\pgfqpoint{3.538180in}{2.608692in}}%
\pgfpathlineto{\pgfqpoint{3.538180in}{2.611641in}}%
\pgfpathlineto{\pgfqpoint{3.542721in}{2.611641in}}%
\pgfpathlineto{\pgfqpoint{3.542721in}{2.608692in}}%
\pgfpathmoveto{\pgfqpoint{3.542721in}{2.605743in}}%
\pgfpathlineto{\pgfqpoint{3.542721in}{2.605743in}}%
\pgfpathlineto{\pgfqpoint{3.542721in}{2.608692in}}%
\pgfpathlineto{\pgfqpoint{3.547262in}{2.608692in}}%
\pgfpathlineto{\pgfqpoint{3.547262in}{2.605743in}}%
\pgfpathmoveto{\pgfqpoint{3.542721in}{2.608692in}}%
\pgfpathlineto{\pgfqpoint{3.542721in}{2.608692in}}%
\pgfpathlineto{\pgfqpoint{3.542721in}{2.611641in}}%
\pgfpathlineto{\pgfqpoint{3.547262in}{2.611641in}}%
\pgfpathlineto{\pgfqpoint{3.547262in}{2.608692in}}%
\pgfpathmoveto{\pgfqpoint{3.533638in}{2.614590in}}%
\pgfpathlineto{\pgfqpoint{3.533638in}{2.614590in}}%
\pgfpathlineto{\pgfqpoint{3.533638in}{2.617539in}}%
\pgfpathlineto{\pgfqpoint{3.538180in}{2.617539in}}%
\pgfpathlineto{\pgfqpoint{3.538180in}{2.614590in}}%
\pgfpathmoveto{\pgfqpoint{3.529097in}{2.620489in}}%
\pgfpathlineto{\pgfqpoint{3.529097in}{2.620489in}}%
\pgfpathlineto{\pgfqpoint{3.529097in}{2.623438in}}%
\pgfpathlineto{\pgfqpoint{3.533638in}{2.623438in}}%
\pgfpathlineto{\pgfqpoint{3.533638in}{2.620489in}}%
\pgfpathmoveto{\pgfqpoint{3.533638in}{2.617539in}}%
\pgfpathlineto{\pgfqpoint{3.533638in}{2.617539in}}%
\pgfpathlineto{\pgfqpoint{3.533638in}{2.620489in}}%
\pgfpathlineto{\pgfqpoint{3.538180in}{2.620489in}}%
\pgfpathlineto{\pgfqpoint{3.538180in}{2.617539in}}%
\pgfpathmoveto{\pgfqpoint{3.533638in}{2.620489in}}%
\pgfpathlineto{\pgfqpoint{3.533638in}{2.620489in}}%
\pgfpathlineto{\pgfqpoint{3.533638in}{2.623438in}}%
\pgfpathlineto{\pgfqpoint{3.538180in}{2.623438in}}%
\pgfpathlineto{\pgfqpoint{3.538180in}{2.620489in}}%
\pgfpathmoveto{\pgfqpoint{3.538180in}{2.611641in}}%
\pgfpathlineto{\pgfqpoint{3.538180in}{2.611641in}}%
\pgfpathlineto{\pgfqpoint{3.538180in}{2.614590in}}%
\pgfpathlineto{\pgfqpoint{3.542721in}{2.614590in}}%
\pgfpathlineto{\pgfqpoint{3.542721in}{2.611641in}}%
\pgfpathmoveto{\pgfqpoint{3.538180in}{2.614590in}}%
\pgfpathlineto{\pgfqpoint{3.538180in}{2.614590in}}%
\pgfpathlineto{\pgfqpoint{3.538180in}{2.617539in}}%
\pgfpathlineto{\pgfqpoint{3.542721in}{2.617539in}}%
\pgfpathlineto{\pgfqpoint{3.542721in}{2.614590in}}%
\pgfpathmoveto{\pgfqpoint{3.560885in}{2.579201in}}%
\pgfpathlineto{\pgfqpoint{3.560885in}{2.579201in}}%
\pgfpathlineto{\pgfqpoint{3.560885in}{2.582150in}}%
\pgfpathlineto{\pgfqpoint{3.565426in}{2.582150in}}%
\pgfpathlineto{\pgfqpoint{3.565426in}{2.579201in}}%
\pgfpathmoveto{\pgfqpoint{3.556344in}{2.585099in}}%
\pgfpathlineto{\pgfqpoint{3.556344in}{2.585099in}}%
\pgfpathlineto{\pgfqpoint{3.556344in}{2.588048in}}%
\pgfpathlineto{\pgfqpoint{3.560885in}{2.588048in}}%
\pgfpathlineto{\pgfqpoint{3.560885in}{2.585099in}}%
\pgfpathmoveto{\pgfqpoint{3.560885in}{2.582150in}}%
\pgfpathlineto{\pgfqpoint{3.560885in}{2.582150in}}%
\pgfpathlineto{\pgfqpoint{3.560885in}{2.585099in}}%
\pgfpathlineto{\pgfqpoint{3.565426in}{2.585099in}}%
\pgfpathlineto{\pgfqpoint{3.565426in}{2.582150in}}%
\pgfpathmoveto{\pgfqpoint{3.560885in}{2.585099in}}%
\pgfpathlineto{\pgfqpoint{3.560885in}{2.585099in}}%
\pgfpathlineto{\pgfqpoint{3.560885in}{2.588048in}}%
\pgfpathlineto{\pgfqpoint{3.565426in}{2.588048in}}%
\pgfpathlineto{\pgfqpoint{3.565426in}{2.585099in}}%
\pgfpathmoveto{\pgfqpoint{3.551803in}{2.590997in}}%
\pgfpathlineto{\pgfqpoint{3.551803in}{2.590997in}}%
\pgfpathlineto{\pgfqpoint{3.551803in}{2.593946in}}%
\pgfpathlineto{\pgfqpoint{3.556344in}{2.593946in}}%
\pgfpathlineto{\pgfqpoint{3.556344in}{2.590997in}}%
\pgfpathmoveto{\pgfqpoint{3.547262in}{2.596896in}}%
\pgfpathlineto{\pgfqpoint{3.547262in}{2.596896in}}%
\pgfpathlineto{\pgfqpoint{3.547262in}{2.599845in}}%
\pgfpathlineto{\pgfqpoint{3.551803in}{2.599845in}}%
\pgfpathlineto{\pgfqpoint{3.551803in}{2.596896in}}%
\pgfpathmoveto{\pgfqpoint{3.551803in}{2.593946in}}%
\pgfpathlineto{\pgfqpoint{3.551803in}{2.593946in}}%
\pgfpathlineto{\pgfqpoint{3.551803in}{2.596896in}}%
\pgfpathlineto{\pgfqpoint{3.556344in}{2.596896in}}%
\pgfpathlineto{\pgfqpoint{3.556344in}{2.593946in}}%
\pgfpathmoveto{\pgfqpoint{3.551803in}{2.596896in}}%
\pgfpathlineto{\pgfqpoint{3.551803in}{2.596896in}}%
\pgfpathlineto{\pgfqpoint{3.551803in}{2.599845in}}%
\pgfpathlineto{\pgfqpoint{3.556344in}{2.599845in}}%
\pgfpathlineto{\pgfqpoint{3.556344in}{2.596896in}}%
\pgfpathmoveto{\pgfqpoint{3.556344in}{2.588048in}}%
\pgfpathlineto{\pgfqpoint{3.556344in}{2.588048in}}%
\pgfpathlineto{\pgfqpoint{3.556344in}{2.590997in}}%
\pgfpathlineto{\pgfqpoint{3.560885in}{2.590997in}}%
\pgfpathlineto{\pgfqpoint{3.560885in}{2.588048in}}%
\pgfpathmoveto{\pgfqpoint{3.556344in}{2.590997in}}%
\pgfpathlineto{\pgfqpoint{3.556344in}{2.590997in}}%
\pgfpathlineto{\pgfqpoint{3.556344in}{2.593946in}}%
\pgfpathlineto{\pgfqpoint{3.560885in}{2.593946in}}%
\pgfpathlineto{\pgfqpoint{3.560885in}{2.590997in}}%
\pgfpathmoveto{\pgfqpoint{3.565426in}{2.576252in}}%
\pgfpathlineto{\pgfqpoint{3.565426in}{2.576252in}}%
\pgfpathlineto{\pgfqpoint{3.565426in}{2.579201in}}%
\pgfpathlineto{\pgfqpoint{3.569967in}{2.579201in}}%
\pgfpathlineto{\pgfqpoint{3.569967in}{2.576252in}}%
\pgfpathmoveto{\pgfqpoint{3.565426in}{2.579201in}}%
\pgfpathlineto{\pgfqpoint{3.565426in}{2.579201in}}%
\pgfpathlineto{\pgfqpoint{3.565426in}{2.582150in}}%
\pgfpathlineto{\pgfqpoint{3.569967in}{2.582150in}}%
\pgfpathlineto{\pgfqpoint{3.569967in}{2.579201in}}%
\pgfpathmoveto{\pgfqpoint{3.547262in}{2.599845in}}%
\pgfpathlineto{\pgfqpoint{3.547262in}{2.599845in}}%
\pgfpathlineto{\pgfqpoint{3.547262in}{2.602794in}}%
\pgfpathlineto{\pgfqpoint{3.551803in}{2.602794in}}%
\pgfpathlineto{\pgfqpoint{3.551803in}{2.599845in}}%
\pgfpathmoveto{\pgfqpoint{3.547262in}{2.602794in}}%
\pgfpathlineto{\pgfqpoint{3.547262in}{2.602794in}}%
\pgfpathlineto{\pgfqpoint{3.547262in}{2.605743in}}%
\pgfpathlineto{\pgfqpoint{3.551803in}{2.605743in}}%
\pgfpathlineto{\pgfqpoint{3.551803in}{2.602794in}}%
\pgfpathmoveto{\pgfqpoint{3.524556in}{2.626387in}}%
\pgfpathlineto{\pgfqpoint{3.524556in}{2.626387in}}%
\pgfpathlineto{\pgfqpoint{3.524556in}{2.629336in}}%
\pgfpathlineto{\pgfqpoint{3.529097in}{2.629336in}}%
\pgfpathlineto{\pgfqpoint{3.529097in}{2.626387in}}%
\pgfpathmoveto{\pgfqpoint{3.520015in}{2.632285in}}%
\pgfpathlineto{\pgfqpoint{3.520015in}{2.632285in}}%
\pgfpathlineto{\pgfqpoint{3.520015in}{2.635234in}}%
\pgfpathlineto{\pgfqpoint{3.524556in}{2.635234in}}%
\pgfpathlineto{\pgfqpoint{3.524556in}{2.632285in}}%
\pgfpathmoveto{\pgfqpoint{3.524556in}{2.629336in}}%
\pgfpathlineto{\pgfqpoint{3.524556in}{2.629336in}}%
\pgfpathlineto{\pgfqpoint{3.524556in}{2.632285in}}%
\pgfpathlineto{\pgfqpoint{3.529097in}{2.632285in}}%
\pgfpathlineto{\pgfqpoint{3.529097in}{2.629336in}}%
\pgfpathmoveto{\pgfqpoint{3.524556in}{2.632285in}}%
\pgfpathlineto{\pgfqpoint{3.524556in}{2.632285in}}%
\pgfpathlineto{\pgfqpoint{3.524556in}{2.635234in}}%
\pgfpathlineto{\pgfqpoint{3.529097in}{2.635234in}}%
\pgfpathlineto{\pgfqpoint{3.529097in}{2.632285in}}%
\pgfpathmoveto{\pgfqpoint{3.515474in}{2.638183in}}%
\pgfpathlineto{\pgfqpoint{3.515474in}{2.638183in}}%
\pgfpathlineto{\pgfqpoint{3.515474in}{2.641132in}}%
\pgfpathlineto{\pgfqpoint{3.520015in}{2.641132in}}%
\pgfpathlineto{\pgfqpoint{3.520015in}{2.638183in}}%
\pgfpathmoveto{\pgfqpoint{3.510933in}{2.644082in}}%
\pgfpathlineto{\pgfqpoint{3.510933in}{2.644082in}}%
\pgfpathlineto{\pgfqpoint{3.510933in}{2.647031in}}%
\pgfpathlineto{\pgfqpoint{3.515474in}{2.647031in}}%
\pgfpathlineto{\pgfqpoint{3.515474in}{2.644082in}}%
\pgfpathmoveto{\pgfqpoint{3.515474in}{2.641132in}}%
\pgfpathlineto{\pgfqpoint{3.515474in}{2.641132in}}%
\pgfpathlineto{\pgfqpoint{3.515474in}{2.644082in}}%
\pgfpathlineto{\pgfqpoint{3.520015in}{2.644082in}}%
\pgfpathlineto{\pgfqpoint{3.520015in}{2.641132in}}%
\pgfpathmoveto{\pgfqpoint{3.515474in}{2.644082in}}%
\pgfpathlineto{\pgfqpoint{3.515474in}{2.644082in}}%
\pgfpathlineto{\pgfqpoint{3.515474in}{2.647031in}}%
\pgfpathlineto{\pgfqpoint{3.520015in}{2.647031in}}%
\pgfpathlineto{\pgfqpoint{3.520015in}{2.644082in}}%
\pgfpathmoveto{\pgfqpoint{3.520015in}{2.635234in}}%
\pgfpathlineto{\pgfqpoint{3.520015in}{2.635234in}}%
\pgfpathlineto{\pgfqpoint{3.520015in}{2.638183in}}%
\pgfpathlineto{\pgfqpoint{3.524556in}{2.638183in}}%
\pgfpathlineto{\pgfqpoint{3.524556in}{2.635234in}}%
\pgfpathmoveto{\pgfqpoint{3.520015in}{2.638183in}}%
\pgfpathlineto{\pgfqpoint{3.520015in}{2.638183in}}%
\pgfpathlineto{\pgfqpoint{3.520015in}{2.641132in}}%
\pgfpathlineto{\pgfqpoint{3.524556in}{2.641132in}}%
\pgfpathlineto{\pgfqpoint{3.524556in}{2.638183in}}%
\pgfpathmoveto{\pgfqpoint{3.529097in}{2.623438in}}%
\pgfpathlineto{\pgfqpoint{3.529097in}{2.623438in}}%
\pgfpathlineto{\pgfqpoint{3.529097in}{2.626387in}}%
\pgfpathlineto{\pgfqpoint{3.533638in}{2.626387in}}%
\pgfpathlineto{\pgfqpoint{3.533638in}{2.623438in}}%
\pgfpathmoveto{\pgfqpoint{3.529097in}{2.626387in}}%
\pgfpathlineto{\pgfqpoint{3.529097in}{2.626387in}}%
\pgfpathlineto{\pgfqpoint{3.529097in}{2.629336in}}%
\pgfpathlineto{\pgfqpoint{3.533638in}{2.629336in}}%
\pgfpathlineto{\pgfqpoint{3.533638in}{2.626387in}}%
\pgfpathmoveto{\pgfqpoint{3.510933in}{2.647031in}}%
\pgfpathlineto{\pgfqpoint{3.510933in}{2.647031in}}%
\pgfpathlineto{\pgfqpoint{3.510933in}{2.649980in}}%
\pgfpathlineto{\pgfqpoint{3.515474in}{2.649980in}}%
\pgfpathlineto{\pgfqpoint{3.515474in}{2.647031in}}%
\pgfpathmoveto{\pgfqpoint{3.510933in}{2.649980in}}%
\pgfpathlineto{\pgfqpoint{3.510933in}{2.649980in}}%
\pgfpathlineto{\pgfqpoint{3.510933in}{2.652929in}}%
\pgfpathlineto{\pgfqpoint{3.515474in}{2.652929in}}%
\pgfpathlineto{\pgfqpoint{3.515474in}{2.649980in}}%
\pgfpathmoveto{\pgfqpoint{3.797018in}{2.272482in}}%
\pgfpathlineto{\pgfqpoint{3.797018in}{2.272482in}}%
\pgfpathlineto{\pgfqpoint{3.797018in}{2.275431in}}%
\pgfpathlineto{\pgfqpoint{3.801559in}{2.275431in}}%
\pgfpathlineto{\pgfqpoint{3.801559in}{2.272482in}}%
\pgfpathmoveto{\pgfqpoint{3.792477in}{2.278381in}}%
\pgfpathlineto{\pgfqpoint{3.792477in}{2.278381in}}%
\pgfpathlineto{\pgfqpoint{3.792477in}{2.281330in}}%
\pgfpathlineto{\pgfqpoint{3.797018in}{2.281330in}}%
\pgfpathlineto{\pgfqpoint{3.797018in}{2.278381in}}%
\pgfpathmoveto{\pgfqpoint{3.797018in}{2.275431in}}%
\pgfpathlineto{\pgfqpoint{3.797018in}{2.275431in}}%
\pgfpathlineto{\pgfqpoint{3.797018in}{2.278381in}}%
\pgfpathlineto{\pgfqpoint{3.801559in}{2.278381in}}%
\pgfpathlineto{\pgfqpoint{3.801559in}{2.275431in}}%
\pgfpathmoveto{\pgfqpoint{3.797018in}{2.278381in}}%
\pgfpathlineto{\pgfqpoint{3.797018in}{2.278381in}}%
\pgfpathlineto{\pgfqpoint{3.797018in}{2.281330in}}%
\pgfpathlineto{\pgfqpoint{3.801559in}{2.281330in}}%
\pgfpathlineto{\pgfqpoint{3.801559in}{2.278381in}}%
\pgfpathmoveto{\pgfqpoint{3.787936in}{2.284279in}}%
\pgfpathlineto{\pgfqpoint{3.787936in}{2.284279in}}%
\pgfpathlineto{\pgfqpoint{3.787936in}{2.287229in}}%
\pgfpathlineto{\pgfqpoint{3.792477in}{2.287229in}}%
\pgfpathlineto{\pgfqpoint{3.792477in}{2.284279in}}%
\pgfpathmoveto{\pgfqpoint{3.783395in}{2.290178in}}%
\pgfpathlineto{\pgfqpoint{3.783395in}{2.290178in}}%
\pgfpathlineto{\pgfqpoint{3.783395in}{2.293127in}}%
\pgfpathlineto{\pgfqpoint{3.787936in}{2.293127in}}%
\pgfpathlineto{\pgfqpoint{3.787936in}{2.290178in}}%
\pgfpathmoveto{\pgfqpoint{3.787936in}{2.287229in}}%
\pgfpathlineto{\pgfqpoint{3.787936in}{2.287229in}}%
\pgfpathlineto{\pgfqpoint{3.787936in}{2.290178in}}%
\pgfpathlineto{\pgfqpoint{3.792477in}{2.290178in}}%
\pgfpathlineto{\pgfqpoint{3.792477in}{2.287229in}}%
\pgfpathmoveto{\pgfqpoint{3.787936in}{2.290178in}}%
\pgfpathlineto{\pgfqpoint{3.787936in}{2.290178in}}%
\pgfpathlineto{\pgfqpoint{3.787936in}{2.293127in}}%
\pgfpathlineto{\pgfqpoint{3.792477in}{2.293127in}}%
\pgfpathlineto{\pgfqpoint{3.792477in}{2.290178in}}%
\pgfpathmoveto{\pgfqpoint{3.792477in}{2.281330in}}%
\pgfpathlineto{\pgfqpoint{3.792477in}{2.281330in}}%
\pgfpathlineto{\pgfqpoint{3.792477in}{2.284279in}}%
\pgfpathlineto{\pgfqpoint{3.797018in}{2.284279in}}%
\pgfpathlineto{\pgfqpoint{3.797018in}{2.281330in}}%
\pgfpathmoveto{\pgfqpoint{3.792477in}{2.284279in}}%
\pgfpathlineto{\pgfqpoint{3.792477in}{2.284279in}}%
\pgfpathlineto{\pgfqpoint{3.792477in}{2.287229in}}%
\pgfpathlineto{\pgfqpoint{3.797018in}{2.287229in}}%
\pgfpathlineto{\pgfqpoint{3.797018in}{2.284279in}}%
\pgfpathmoveto{\pgfqpoint{3.724362in}{2.366858in}}%
\pgfpathlineto{\pgfqpoint{3.724362in}{2.366858in}}%
\pgfpathlineto{\pgfqpoint{3.724362in}{2.369807in}}%
\pgfpathlineto{\pgfqpoint{3.728903in}{2.369807in}}%
\pgfpathlineto{\pgfqpoint{3.728903in}{2.366858in}}%
\pgfpathmoveto{\pgfqpoint{3.719821in}{2.372757in}}%
\pgfpathlineto{\pgfqpoint{3.719821in}{2.372757in}}%
\pgfpathlineto{\pgfqpoint{3.719821in}{2.375706in}}%
\pgfpathlineto{\pgfqpoint{3.724362in}{2.375706in}}%
\pgfpathlineto{\pgfqpoint{3.724362in}{2.372757in}}%
\pgfpathmoveto{\pgfqpoint{3.724362in}{2.369807in}}%
\pgfpathlineto{\pgfqpoint{3.724362in}{2.369807in}}%
\pgfpathlineto{\pgfqpoint{3.724362in}{2.372757in}}%
\pgfpathlineto{\pgfqpoint{3.728903in}{2.372757in}}%
\pgfpathlineto{\pgfqpoint{3.728903in}{2.369807in}}%
\pgfpathmoveto{\pgfqpoint{3.724362in}{2.372757in}}%
\pgfpathlineto{\pgfqpoint{3.724362in}{2.372757in}}%
\pgfpathlineto{\pgfqpoint{3.724362in}{2.375706in}}%
\pgfpathlineto{\pgfqpoint{3.728903in}{2.375706in}}%
\pgfpathlineto{\pgfqpoint{3.728903in}{2.372757in}}%
\pgfpathmoveto{\pgfqpoint{3.715280in}{2.378655in}}%
\pgfpathlineto{\pgfqpoint{3.715280in}{2.378655in}}%
\pgfpathlineto{\pgfqpoint{3.715280in}{2.381604in}}%
\pgfpathlineto{\pgfqpoint{3.719821in}{2.381604in}}%
\pgfpathlineto{\pgfqpoint{3.719821in}{2.378655in}}%
\pgfpathmoveto{\pgfqpoint{3.710739in}{2.384554in}}%
\pgfpathlineto{\pgfqpoint{3.710739in}{2.384554in}}%
\pgfpathlineto{\pgfqpoint{3.710739in}{2.387503in}}%
\pgfpathlineto{\pgfqpoint{3.715280in}{2.387503in}}%
\pgfpathlineto{\pgfqpoint{3.715280in}{2.384554in}}%
\pgfpathmoveto{\pgfqpoint{3.715280in}{2.381604in}}%
\pgfpathlineto{\pgfqpoint{3.715280in}{2.381604in}}%
\pgfpathlineto{\pgfqpoint{3.715280in}{2.384554in}}%
\pgfpathlineto{\pgfqpoint{3.719821in}{2.384554in}}%
\pgfpathlineto{\pgfqpoint{3.719821in}{2.381604in}}%
\pgfpathmoveto{\pgfqpoint{3.715280in}{2.384554in}}%
\pgfpathlineto{\pgfqpoint{3.715280in}{2.384554in}}%
\pgfpathlineto{\pgfqpoint{3.715280in}{2.387503in}}%
\pgfpathlineto{\pgfqpoint{3.719821in}{2.387503in}}%
\pgfpathlineto{\pgfqpoint{3.719821in}{2.384554in}}%
\pgfpathmoveto{\pgfqpoint{3.719821in}{2.375706in}}%
\pgfpathlineto{\pgfqpoint{3.719821in}{2.375706in}}%
\pgfpathlineto{\pgfqpoint{3.719821in}{2.378655in}}%
\pgfpathlineto{\pgfqpoint{3.724362in}{2.378655in}}%
\pgfpathlineto{\pgfqpoint{3.724362in}{2.375706in}}%
\pgfpathmoveto{\pgfqpoint{3.719821in}{2.378655in}}%
\pgfpathlineto{\pgfqpoint{3.719821in}{2.378655in}}%
\pgfpathlineto{\pgfqpoint{3.719821in}{2.381604in}}%
\pgfpathlineto{\pgfqpoint{3.724362in}{2.381604in}}%
\pgfpathlineto{\pgfqpoint{3.724362in}{2.378655in}}%
\pgfpathmoveto{\pgfqpoint{3.760690in}{2.319670in}}%
\pgfpathlineto{\pgfqpoint{3.760690in}{2.319670in}}%
\pgfpathlineto{\pgfqpoint{3.760690in}{2.322620in}}%
\pgfpathlineto{\pgfqpoint{3.765231in}{2.322620in}}%
\pgfpathlineto{\pgfqpoint{3.765231in}{2.319670in}}%
\pgfpathmoveto{\pgfqpoint{3.756149in}{2.325569in}}%
\pgfpathlineto{\pgfqpoint{3.756149in}{2.325569in}}%
\pgfpathlineto{\pgfqpoint{3.756149in}{2.328518in}}%
\pgfpathlineto{\pgfqpoint{3.760690in}{2.328518in}}%
\pgfpathlineto{\pgfqpoint{3.760690in}{2.325569in}}%
\pgfpathmoveto{\pgfqpoint{3.760690in}{2.322620in}}%
\pgfpathlineto{\pgfqpoint{3.760690in}{2.322620in}}%
\pgfpathlineto{\pgfqpoint{3.760690in}{2.325569in}}%
\pgfpathlineto{\pgfqpoint{3.765231in}{2.325569in}}%
\pgfpathlineto{\pgfqpoint{3.765231in}{2.322620in}}%
\pgfpathmoveto{\pgfqpoint{3.760690in}{2.325569in}}%
\pgfpathlineto{\pgfqpoint{3.760690in}{2.325569in}}%
\pgfpathlineto{\pgfqpoint{3.760690in}{2.328518in}}%
\pgfpathlineto{\pgfqpoint{3.765231in}{2.328518in}}%
\pgfpathlineto{\pgfqpoint{3.765231in}{2.325569in}}%
\pgfpathmoveto{\pgfqpoint{3.751608in}{2.331467in}}%
\pgfpathlineto{\pgfqpoint{3.751608in}{2.331467in}}%
\pgfpathlineto{\pgfqpoint{3.751608in}{2.334417in}}%
\pgfpathlineto{\pgfqpoint{3.756149in}{2.334417in}}%
\pgfpathlineto{\pgfqpoint{3.756149in}{2.331467in}}%
\pgfpathmoveto{\pgfqpoint{3.747067in}{2.337366in}}%
\pgfpathlineto{\pgfqpoint{3.747067in}{2.337366in}}%
\pgfpathlineto{\pgfqpoint{3.747067in}{2.340315in}}%
\pgfpathlineto{\pgfqpoint{3.751608in}{2.340315in}}%
\pgfpathlineto{\pgfqpoint{3.751608in}{2.337366in}}%
\pgfpathmoveto{\pgfqpoint{3.751608in}{2.334417in}}%
\pgfpathlineto{\pgfqpoint{3.751608in}{2.334417in}}%
\pgfpathlineto{\pgfqpoint{3.751608in}{2.337366in}}%
\pgfpathlineto{\pgfqpoint{3.756149in}{2.337366in}}%
\pgfpathlineto{\pgfqpoint{3.756149in}{2.334417in}}%
\pgfpathmoveto{\pgfqpoint{3.751608in}{2.337366in}}%
\pgfpathlineto{\pgfqpoint{3.751608in}{2.337366in}}%
\pgfpathlineto{\pgfqpoint{3.751608in}{2.340315in}}%
\pgfpathlineto{\pgfqpoint{3.756149in}{2.340315in}}%
\pgfpathlineto{\pgfqpoint{3.756149in}{2.337366in}}%
\pgfpathmoveto{\pgfqpoint{3.756149in}{2.328518in}}%
\pgfpathlineto{\pgfqpoint{3.756149in}{2.328518in}}%
\pgfpathlineto{\pgfqpoint{3.756149in}{2.331467in}}%
\pgfpathlineto{\pgfqpoint{3.760690in}{2.331467in}}%
\pgfpathlineto{\pgfqpoint{3.760690in}{2.328518in}}%
\pgfpathmoveto{\pgfqpoint{3.756149in}{2.331467in}}%
\pgfpathlineto{\pgfqpoint{3.756149in}{2.331467in}}%
\pgfpathlineto{\pgfqpoint{3.756149in}{2.334417in}}%
\pgfpathlineto{\pgfqpoint{3.760690in}{2.334417in}}%
\pgfpathlineto{\pgfqpoint{3.760690in}{2.331467in}}%
\pgfpathmoveto{\pgfqpoint{3.778854in}{2.296077in}}%
\pgfpathlineto{\pgfqpoint{3.778854in}{2.296077in}}%
\pgfpathlineto{\pgfqpoint{3.778854in}{2.299026in}}%
\pgfpathlineto{\pgfqpoint{3.783395in}{2.299026in}}%
\pgfpathlineto{\pgfqpoint{3.783395in}{2.296077in}}%
\pgfpathmoveto{\pgfqpoint{3.774313in}{2.301975in}}%
\pgfpathlineto{\pgfqpoint{3.774313in}{2.301975in}}%
\pgfpathlineto{\pgfqpoint{3.774313in}{2.304924in}}%
\pgfpathlineto{\pgfqpoint{3.778854in}{2.304924in}}%
\pgfpathlineto{\pgfqpoint{3.778854in}{2.301975in}}%
\pgfpathmoveto{\pgfqpoint{3.778854in}{2.299026in}}%
\pgfpathlineto{\pgfqpoint{3.778854in}{2.299026in}}%
\pgfpathlineto{\pgfqpoint{3.778854in}{2.301975in}}%
\pgfpathlineto{\pgfqpoint{3.783395in}{2.301975in}}%
\pgfpathlineto{\pgfqpoint{3.783395in}{2.299026in}}%
\pgfpathmoveto{\pgfqpoint{3.778854in}{2.301975in}}%
\pgfpathlineto{\pgfqpoint{3.778854in}{2.301975in}}%
\pgfpathlineto{\pgfqpoint{3.778854in}{2.304924in}}%
\pgfpathlineto{\pgfqpoint{3.783395in}{2.304924in}}%
\pgfpathlineto{\pgfqpoint{3.783395in}{2.301975in}}%
\pgfpathmoveto{\pgfqpoint{3.769772in}{2.307874in}}%
\pgfpathlineto{\pgfqpoint{3.769772in}{2.307874in}}%
\pgfpathlineto{\pgfqpoint{3.769772in}{2.310823in}}%
\pgfpathlineto{\pgfqpoint{3.774313in}{2.310823in}}%
\pgfpathlineto{\pgfqpoint{3.774313in}{2.307874in}}%
\pgfpathmoveto{\pgfqpoint{3.765231in}{2.313772in}}%
\pgfpathlineto{\pgfqpoint{3.765231in}{2.313772in}}%
\pgfpathlineto{\pgfqpoint{3.765231in}{2.316721in}}%
\pgfpathlineto{\pgfqpoint{3.769772in}{2.316721in}}%
\pgfpathlineto{\pgfqpoint{3.769772in}{2.313772in}}%
\pgfpathmoveto{\pgfqpoint{3.769772in}{2.310823in}}%
\pgfpathlineto{\pgfqpoint{3.769772in}{2.310823in}}%
\pgfpathlineto{\pgfqpoint{3.769772in}{2.313772in}}%
\pgfpathlineto{\pgfqpoint{3.774313in}{2.313772in}}%
\pgfpathlineto{\pgfqpoint{3.774313in}{2.310823in}}%
\pgfpathmoveto{\pgfqpoint{3.769772in}{2.313772in}}%
\pgfpathlineto{\pgfqpoint{3.769772in}{2.313772in}}%
\pgfpathlineto{\pgfqpoint{3.769772in}{2.316721in}}%
\pgfpathlineto{\pgfqpoint{3.774313in}{2.316721in}}%
\pgfpathlineto{\pgfqpoint{3.774313in}{2.313772in}}%
\pgfpathmoveto{\pgfqpoint{3.774313in}{2.304924in}}%
\pgfpathlineto{\pgfqpoint{3.774313in}{2.304924in}}%
\pgfpathlineto{\pgfqpoint{3.774313in}{2.307874in}}%
\pgfpathlineto{\pgfqpoint{3.778854in}{2.307874in}}%
\pgfpathlineto{\pgfqpoint{3.778854in}{2.304924in}}%
\pgfpathmoveto{\pgfqpoint{3.774313in}{2.307874in}}%
\pgfpathlineto{\pgfqpoint{3.774313in}{2.307874in}}%
\pgfpathlineto{\pgfqpoint{3.774313in}{2.310823in}}%
\pgfpathlineto{\pgfqpoint{3.778854in}{2.310823in}}%
\pgfpathlineto{\pgfqpoint{3.778854in}{2.307874in}}%
\pgfpathmoveto{\pgfqpoint{3.783395in}{2.293127in}}%
\pgfpathlineto{\pgfqpoint{3.783395in}{2.293127in}}%
\pgfpathlineto{\pgfqpoint{3.783395in}{2.296077in}}%
\pgfpathlineto{\pgfqpoint{3.787936in}{2.296077in}}%
\pgfpathlineto{\pgfqpoint{3.787936in}{2.293127in}}%
\pgfpathmoveto{\pgfqpoint{3.783395in}{2.296077in}}%
\pgfpathlineto{\pgfqpoint{3.783395in}{2.296077in}}%
\pgfpathlineto{\pgfqpoint{3.783395in}{2.299026in}}%
\pgfpathlineto{\pgfqpoint{3.787936in}{2.299026in}}%
\pgfpathlineto{\pgfqpoint{3.787936in}{2.296077in}}%
\pgfpathmoveto{\pgfqpoint{3.765231in}{2.316721in}}%
\pgfpathlineto{\pgfqpoint{3.765231in}{2.316721in}}%
\pgfpathlineto{\pgfqpoint{3.765231in}{2.319670in}}%
\pgfpathlineto{\pgfqpoint{3.769772in}{2.319670in}}%
\pgfpathlineto{\pgfqpoint{3.769772in}{2.316721in}}%
\pgfpathmoveto{\pgfqpoint{3.765231in}{2.319670in}}%
\pgfpathlineto{\pgfqpoint{3.765231in}{2.319670in}}%
\pgfpathlineto{\pgfqpoint{3.765231in}{2.322620in}}%
\pgfpathlineto{\pgfqpoint{3.769772in}{2.322620in}}%
\pgfpathlineto{\pgfqpoint{3.769772in}{2.319670in}}%
\pgfpathmoveto{\pgfqpoint{3.742526in}{2.343264in}}%
\pgfpathlineto{\pgfqpoint{3.742526in}{2.343264in}}%
\pgfpathlineto{\pgfqpoint{3.742526in}{2.346214in}}%
\pgfpathlineto{\pgfqpoint{3.747067in}{2.346214in}}%
\pgfpathlineto{\pgfqpoint{3.747067in}{2.343264in}}%
\pgfpathmoveto{\pgfqpoint{3.737985in}{2.349163in}}%
\pgfpathlineto{\pgfqpoint{3.737985in}{2.349163in}}%
\pgfpathlineto{\pgfqpoint{3.737985in}{2.352112in}}%
\pgfpathlineto{\pgfqpoint{3.742526in}{2.352112in}}%
\pgfpathlineto{\pgfqpoint{3.742526in}{2.349163in}}%
\pgfpathmoveto{\pgfqpoint{3.742526in}{2.346214in}}%
\pgfpathlineto{\pgfqpoint{3.742526in}{2.346214in}}%
\pgfpathlineto{\pgfqpoint{3.742526in}{2.349163in}}%
\pgfpathlineto{\pgfqpoint{3.747067in}{2.349163in}}%
\pgfpathlineto{\pgfqpoint{3.747067in}{2.346214in}}%
\pgfpathmoveto{\pgfqpoint{3.742526in}{2.349163in}}%
\pgfpathlineto{\pgfqpoint{3.742526in}{2.349163in}}%
\pgfpathlineto{\pgfqpoint{3.742526in}{2.352112in}}%
\pgfpathlineto{\pgfqpoint{3.747067in}{2.352112in}}%
\pgfpathlineto{\pgfqpoint{3.747067in}{2.349163in}}%
\pgfpathmoveto{\pgfqpoint{3.733444in}{2.355061in}}%
\pgfpathlineto{\pgfqpoint{3.733444in}{2.355061in}}%
\pgfpathlineto{\pgfqpoint{3.733444in}{2.358011in}}%
\pgfpathlineto{\pgfqpoint{3.737985in}{2.358011in}}%
\pgfpathlineto{\pgfqpoint{3.737985in}{2.355061in}}%
\pgfpathmoveto{\pgfqpoint{3.728903in}{2.360960in}}%
\pgfpathlineto{\pgfqpoint{3.728903in}{2.360960in}}%
\pgfpathlineto{\pgfqpoint{3.728903in}{2.363909in}}%
\pgfpathlineto{\pgfqpoint{3.733444in}{2.363909in}}%
\pgfpathlineto{\pgfqpoint{3.733444in}{2.360960in}}%
\pgfpathmoveto{\pgfqpoint{3.733444in}{2.358011in}}%
\pgfpathlineto{\pgfqpoint{3.733444in}{2.358011in}}%
\pgfpathlineto{\pgfqpoint{3.733444in}{2.360960in}}%
\pgfpathlineto{\pgfqpoint{3.737985in}{2.360960in}}%
\pgfpathlineto{\pgfqpoint{3.737985in}{2.358011in}}%
\pgfpathmoveto{\pgfqpoint{3.733444in}{2.360960in}}%
\pgfpathlineto{\pgfqpoint{3.733444in}{2.360960in}}%
\pgfpathlineto{\pgfqpoint{3.733444in}{2.363909in}}%
\pgfpathlineto{\pgfqpoint{3.737985in}{2.363909in}}%
\pgfpathlineto{\pgfqpoint{3.737985in}{2.360960in}}%
\pgfpathmoveto{\pgfqpoint{3.737985in}{2.352112in}}%
\pgfpathlineto{\pgfqpoint{3.737985in}{2.352112in}}%
\pgfpathlineto{\pgfqpoint{3.737985in}{2.355061in}}%
\pgfpathlineto{\pgfqpoint{3.742526in}{2.355061in}}%
\pgfpathlineto{\pgfqpoint{3.742526in}{2.352112in}}%
\pgfpathmoveto{\pgfqpoint{3.737985in}{2.355061in}}%
\pgfpathlineto{\pgfqpoint{3.737985in}{2.355061in}}%
\pgfpathlineto{\pgfqpoint{3.737985in}{2.358011in}}%
\pgfpathlineto{\pgfqpoint{3.742526in}{2.358011in}}%
\pgfpathlineto{\pgfqpoint{3.742526in}{2.355061in}}%
\pgfpathmoveto{\pgfqpoint{3.747067in}{2.340315in}}%
\pgfpathlineto{\pgfqpoint{3.747067in}{2.340315in}}%
\pgfpathlineto{\pgfqpoint{3.747067in}{2.343264in}}%
\pgfpathlineto{\pgfqpoint{3.751608in}{2.343264in}}%
\pgfpathlineto{\pgfqpoint{3.751608in}{2.340315in}}%
\pgfpathmoveto{\pgfqpoint{3.747067in}{2.343264in}}%
\pgfpathlineto{\pgfqpoint{3.747067in}{2.343264in}}%
\pgfpathlineto{\pgfqpoint{3.747067in}{2.346214in}}%
\pgfpathlineto{\pgfqpoint{3.751608in}{2.346214in}}%
\pgfpathlineto{\pgfqpoint{3.751608in}{2.343264in}}%
\pgfpathmoveto{\pgfqpoint{3.728903in}{2.363909in}}%
\pgfpathlineto{\pgfqpoint{3.728903in}{2.363909in}}%
\pgfpathlineto{\pgfqpoint{3.728903in}{2.366858in}}%
\pgfpathlineto{\pgfqpoint{3.733444in}{2.366858in}}%
\pgfpathlineto{\pgfqpoint{3.733444in}{2.363909in}}%
\pgfpathmoveto{\pgfqpoint{3.728903in}{2.366858in}}%
\pgfpathlineto{\pgfqpoint{3.728903in}{2.366858in}}%
\pgfpathlineto{\pgfqpoint{3.728903in}{2.369807in}}%
\pgfpathlineto{\pgfqpoint{3.733444in}{2.369807in}}%
\pgfpathlineto{\pgfqpoint{3.733444in}{2.366858in}}%
\pgfpathmoveto{\pgfqpoint{3.688034in}{2.414044in}}%
\pgfpathlineto{\pgfqpoint{3.688034in}{2.414044in}}%
\pgfpathlineto{\pgfqpoint{3.688034in}{2.416993in}}%
\pgfpathlineto{\pgfqpoint{3.692575in}{2.416993in}}%
\pgfpathlineto{\pgfqpoint{3.692575in}{2.414044in}}%
\pgfpathmoveto{\pgfqpoint{3.683493in}{2.419942in}}%
\pgfpathlineto{\pgfqpoint{3.683493in}{2.419942in}}%
\pgfpathlineto{\pgfqpoint{3.683493in}{2.422891in}}%
\pgfpathlineto{\pgfqpoint{3.688034in}{2.422891in}}%
\pgfpathlineto{\pgfqpoint{3.688034in}{2.419942in}}%
\pgfpathmoveto{\pgfqpoint{3.688034in}{2.416993in}}%
\pgfpathlineto{\pgfqpoint{3.688034in}{2.416993in}}%
\pgfpathlineto{\pgfqpoint{3.688034in}{2.419942in}}%
\pgfpathlineto{\pgfqpoint{3.692575in}{2.419942in}}%
\pgfpathlineto{\pgfqpoint{3.692575in}{2.416993in}}%
\pgfpathmoveto{\pgfqpoint{3.688034in}{2.419942in}}%
\pgfpathlineto{\pgfqpoint{3.688034in}{2.419942in}}%
\pgfpathlineto{\pgfqpoint{3.688034in}{2.422891in}}%
\pgfpathlineto{\pgfqpoint{3.692575in}{2.422891in}}%
\pgfpathlineto{\pgfqpoint{3.692575in}{2.419942in}}%
\pgfpathmoveto{\pgfqpoint{3.678952in}{2.425840in}}%
\pgfpathlineto{\pgfqpoint{3.678952in}{2.425840in}}%
\pgfpathlineto{\pgfqpoint{3.678952in}{2.428789in}}%
\pgfpathlineto{\pgfqpoint{3.683493in}{2.428789in}}%
\pgfpathlineto{\pgfqpoint{3.683493in}{2.425840in}}%
\pgfpathmoveto{\pgfqpoint{3.674411in}{2.431738in}}%
\pgfpathlineto{\pgfqpoint{3.674411in}{2.431738in}}%
\pgfpathlineto{\pgfqpoint{3.674411in}{2.434688in}}%
\pgfpathlineto{\pgfqpoint{3.678952in}{2.434688in}}%
\pgfpathlineto{\pgfqpoint{3.678952in}{2.431738in}}%
\pgfpathmoveto{\pgfqpoint{3.678952in}{2.428789in}}%
\pgfpathlineto{\pgfqpoint{3.678952in}{2.428789in}}%
\pgfpathlineto{\pgfqpoint{3.678952in}{2.431738in}}%
\pgfpathlineto{\pgfqpoint{3.683493in}{2.431738in}}%
\pgfpathlineto{\pgfqpoint{3.683493in}{2.428789in}}%
\pgfpathmoveto{\pgfqpoint{3.678952in}{2.431738in}}%
\pgfpathlineto{\pgfqpoint{3.678952in}{2.431738in}}%
\pgfpathlineto{\pgfqpoint{3.678952in}{2.434688in}}%
\pgfpathlineto{\pgfqpoint{3.683493in}{2.434688in}}%
\pgfpathlineto{\pgfqpoint{3.683493in}{2.431738in}}%
\pgfpathmoveto{\pgfqpoint{3.683493in}{2.422891in}}%
\pgfpathlineto{\pgfqpoint{3.683493in}{2.422891in}}%
\pgfpathlineto{\pgfqpoint{3.683493in}{2.425840in}}%
\pgfpathlineto{\pgfqpoint{3.688034in}{2.425840in}}%
\pgfpathlineto{\pgfqpoint{3.688034in}{2.422891in}}%
\pgfpathmoveto{\pgfqpoint{3.683493in}{2.425840in}}%
\pgfpathlineto{\pgfqpoint{3.683493in}{2.425840in}}%
\pgfpathlineto{\pgfqpoint{3.683493in}{2.428789in}}%
\pgfpathlineto{\pgfqpoint{3.688034in}{2.428789in}}%
\pgfpathlineto{\pgfqpoint{3.688034in}{2.425840in}}%
\pgfpathmoveto{\pgfqpoint{3.706198in}{2.390452in}}%
\pgfpathlineto{\pgfqpoint{3.706198in}{2.390452in}}%
\pgfpathlineto{\pgfqpoint{3.706198in}{2.393401in}}%
\pgfpathlineto{\pgfqpoint{3.710739in}{2.393401in}}%
\pgfpathlineto{\pgfqpoint{3.710739in}{2.390452in}}%
\pgfpathmoveto{\pgfqpoint{3.701657in}{2.396350in}}%
\pgfpathlineto{\pgfqpoint{3.701657in}{2.396350in}}%
\pgfpathlineto{\pgfqpoint{3.701657in}{2.399299in}}%
\pgfpathlineto{\pgfqpoint{3.706198in}{2.399299in}}%
\pgfpathlineto{\pgfqpoint{3.706198in}{2.396350in}}%
\pgfpathmoveto{\pgfqpoint{3.706198in}{2.393401in}}%
\pgfpathlineto{\pgfqpoint{3.706198in}{2.393401in}}%
\pgfpathlineto{\pgfqpoint{3.706198in}{2.396350in}}%
\pgfpathlineto{\pgfqpoint{3.710739in}{2.396350in}}%
\pgfpathlineto{\pgfqpoint{3.710739in}{2.393401in}}%
\pgfpathmoveto{\pgfqpoint{3.706198in}{2.396350in}}%
\pgfpathlineto{\pgfqpoint{3.706198in}{2.396350in}}%
\pgfpathlineto{\pgfqpoint{3.706198in}{2.399299in}}%
\pgfpathlineto{\pgfqpoint{3.710739in}{2.399299in}}%
\pgfpathlineto{\pgfqpoint{3.710739in}{2.396350in}}%
\pgfpathmoveto{\pgfqpoint{3.697116in}{2.402248in}}%
\pgfpathlineto{\pgfqpoint{3.697116in}{2.402248in}}%
\pgfpathlineto{\pgfqpoint{3.697116in}{2.405197in}}%
\pgfpathlineto{\pgfqpoint{3.701657in}{2.405197in}}%
\pgfpathlineto{\pgfqpoint{3.701657in}{2.402248in}}%
\pgfpathmoveto{\pgfqpoint{3.692575in}{2.408146in}}%
\pgfpathlineto{\pgfqpoint{3.692575in}{2.408146in}}%
\pgfpathlineto{\pgfqpoint{3.692575in}{2.411095in}}%
\pgfpathlineto{\pgfqpoint{3.697116in}{2.411095in}}%
\pgfpathlineto{\pgfqpoint{3.697116in}{2.408146in}}%
\pgfpathmoveto{\pgfqpoint{3.697116in}{2.405197in}}%
\pgfpathlineto{\pgfqpoint{3.697116in}{2.405197in}}%
\pgfpathlineto{\pgfqpoint{3.697116in}{2.408146in}}%
\pgfpathlineto{\pgfqpoint{3.701657in}{2.408146in}}%
\pgfpathlineto{\pgfqpoint{3.701657in}{2.405197in}}%
\pgfpathmoveto{\pgfqpoint{3.697116in}{2.408146in}}%
\pgfpathlineto{\pgfqpoint{3.697116in}{2.408146in}}%
\pgfpathlineto{\pgfqpoint{3.697116in}{2.411095in}}%
\pgfpathlineto{\pgfqpoint{3.701657in}{2.411095in}}%
\pgfpathlineto{\pgfqpoint{3.701657in}{2.408146in}}%
\pgfpathmoveto{\pgfqpoint{3.701657in}{2.399299in}}%
\pgfpathlineto{\pgfqpoint{3.701657in}{2.399299in}}%
\pgfpathlineto{\pgfqpoint{3.701657in}{2.402248in}}%
\pgfpathlineto{\pgfqpoint{3.706198in}{2.402248in}}%
\pgfpathlineto{\pgfqpoint{3.706198in}{2.399299in}}%
\pgfpathmoveto{\pgfqpoint{3.701657in}{2.402248in}}%
\pgfpathlineto{\pgfqpoint{3.701657in}{2.402248in}}%
\pgfpathlineto{\pgfqpoint{3.701657in}{2.405197in}}%
\pgfpathlineto{\pgfqpoint{3.706198in}{2.405197in}}%
\pgfpathlineto{\pgfqpoint{3.706198in}{2.402248in}}%
\pgfpathmoveto{\pgfqpoint{3.710739in}{2.387503in}}%
\pgfpathlineto{\pgfqpoint{3.710739in}{2.387503in}}%
\pgfpathlineto{\pgfqpoint{3.710739in}{2.390452in}}%
\pgfpathlineto{\pgfqpoint{3.715280in}{2.390452in}}%
\pgfpathlineto{\pgfqpoint{3.715280in}{2.387503in}}%
\pgfpathmoveto{\pgfqpoint{3.710739in}{2.390452in}}%
\pgfpathlineto{\pgfqpoint{3.710739in}{2.390452in}}%
\pgfpathlineto{\pgfqpoint{3.710739in}{2.393401in}}%
\pgfpathlineto{\pgfqpoint{3.715280in}{2.393401in}}%
\pgfpathlineto{\pgfqpoint{3.715280in}{2.390452in}}%
\pgfpathmoveto{\pgfqpoint{3.692575in}{2.411095in}}%
\pgfpathlineto{\pgfqpoint{3.692575in}{2.411095in}}%
\pgfpathlineto{\pgfqpoint{3.692575in}{2.414044in}}%
\pgfpathlineto{\pgfqpoint{3.697116in}{2.414044in}}%
\pgfpathlineto{\pgfqpoint{3.697116in}{2.411095in}}%
\pgfpathmoveto{\pgfqpoint{3.692575in}{2.414044in}}%
\pgfpathlineto{\pgfqpoint{3.692575in}{2.414044in}}%
\pgfpathlineto{\pgfqpoint{3.692575in}{2.416993in}}%
\pgfpathlineto{\pgfqpoint{3.697116in}{2.416993in}}%
\pgfpathlineto{\pgfqpoint{3.697116in}{2.414044in}}%
\pgfpathmoveto{\pgfqpoint{3.669870in}{2.437637in}}%
\pgfpathlineto{\pgfqpoint{3.669870in}{2.437637in}}%
\pgfpathlineto{\pgfqpoint{3.669870in}{2.440586in}}%
\pgfpathlineto{\pgfqpoint{3.674411in}{2.440586in}}%
\pgfpathlineto{\pgfqpoint{3.674411in}{2.437637in}}%
\pgfpathmoveto{\pgfqpoint{3.665329in}{2.443535in}}%
\pgfpathlineto{\pgfqpoint{3.665329in}{2.443535in}}%
\pgfpathlineto{\pgfqpoint{3.665329in}{2.446484in}}%
\pgfpathlineto{\pgfqpoint{3.669870in}{2.446484in}}%
\pgfpathlineto{\pgfqpoint{3.669870in}{2.443535in}}%
\pgfpathmoveto{\pgfqpoint{3.669870in}{2.440586in}}%
\pgfpathlineto{\pgfqpoint{3.669870in}{2.440586in}}%
\pgfpathlineto{\pgfqpoint{3.669870in}{2.443535in}}%
\pgfpathlineto{\pgfqpoint{3.674411in}{2.443535in}}%
\pgfpathlineto{\pgfqpoint{3.674411in}{2.440586in}}%
\pgfpathmoveto{\pgfqpoint{3.669870in}{2.443535in}}%
\pgfpathlineto{\pgfqpoint{3.669870in}{2.443535in}}%
\pgfpathlineto{\pgfqpoint{3.669870in}{2.446484in}}%
\pgfpathlineto{\pgfqpoint{3.674411in}{2.446484in}}%
\pgfpathlineto{\pgfqpoint{3.674411in}{2.443535in}}%
\pgfpathmoveto{\pgfqpoint{3.660788in}{2.449433in}}%
\pgfpathlineto{\pgfqpoint{3.660788in}{2.449433in}}%
\pgfpathlineto{\pgfqpoint{3.660788in}{2.452382in}}%
\pgfpathlineto{\pgfqpoint{3.665329in}{2.452382in}}%
\pgfpathlineto{\pgfqpoint{3.665329in}{2.449433in}}%
\pgfpathmoveto{\pgfqpoint{3.656247in}{2.455331in}}%
\pgfpathlineto{\pgfqpoint{3.656247in}{2.455331in}}%
\pgfpathlineto{\pgfqpoint{3.656247in}{2.458280in}}%
\pgfpathlineto{\pgfqpoint{3.660788in}{2.458280in}}%
\pgfpathlineto{\pgfqpoint{3.660788in}{2.455331in}}%
\pgfpathmoveto{\pgfqpoint{3.660788in}{2.452382in}}%
\pgfpathlineto{\pgfqpoint{3.660788in}{2.452382in}}%
\pgfpathlineto{\pgfqpoint{3.660788in}{2.455331in}}%
\pgfpathlineto{\pgfqpoint{3.665329in}{2.455331in}}%
\pgfpathlineto{\pgfqpoint{3.665329in}{2.452382in}}%
\pgfpathmoveto{\pgfqpoint{3.660788in}{2.455331in}}%
\pgfpathlineto{\pgfqpoint{3.660788in}{2.455331in}}%
\pgfpathlineto{\pgfqpoint{3.660788in}{2.458280in}}%
\pgfpathlineto{\pgfqpoint{3.665329in}{2.458280in}}%
\pgfpathlineto{\pgfqpoint{3.665329in}{2.455331in}}%
\pgfpathmoveto{\pgfqpoint{3.665329in}{2.446484in}}%
\pgfpathlineto{\pgfqpoint{3.665329in}{2.446484in}}%
\pgfpathlineto{\pgfqpoint{3.665329in}{2.449433in}}%
\pgfpathlineto{\pgfqpoint{3.669870in}{2.449433in}}%
\pgfpathlineto{\pgfqpoint{3.669870in}{2.446484in}}%
\pgfpathmoveto{\pgfqpoint{3.665329in}{2.449433in}}%
\pgfpathlineto{\pgfqpoint{3.665329in}{2.449433in}}%
\pgfpathlineto{\pgfqpoint{3.665329in}{2.452382in}}%
\pgfpathlineto{\pgfqpoint{3.669870in}{2.452382in}}%
\pgfpathlineto{\pgfqpoint{3.669870in}{2.449433in}}%
\pgfpathmoveto{\pgfqpoint{3.674411in}{2.434688in}}%
\pgfpathlineto{\pgfqpoint{3.674411in}{2.434688in}}%
\pgfpathlineto{\pgfqpoint{3.674411in}{2.437637in}}%
\pgfpathlineto{\pgfqpoint{3.678952in}{2.437637in}}%
\pgfpathlineto{\pgfqpoint{3.678952in}{2.434688in}}%
\pgfpathmoveto{\pgfqpoint{3.674411in}{2.437637in}}%
\pgfpathlineto{\pgfqpoint{3.674411in}{2.437637in}}%
\pgfpathlineto{\pgfqpoint{3.674411in}{2.440586in}}%
\pgfpathlineto{\pgfqpoint{3.678952in}{2.440586in}}%
\pgfpathlineto{\pgfqpoint{3.678952in}{2.437637in}}%
\pgfpathmoveto{\pgfqpoint{3.656247in}{2.458280in}}%
\pgfpathlineto{\pgfqpoint{3.656247in}{2.458280in}}%
\pgfpathlineto{\pgfqpoint{3.656247in}{2.461229in}}%
\pgfpathlineto{\pgfqpoint{3.660788in}{2.461229in}}%
\pgfpathlineto{\pgfqpoint{3.660788in}{2.458280in}}%
\pgfpathmoveto{\pgfqpoint{3.656247in}{2.461229in}}%
\pgfpathlineto{\pgfqpoint{3.656247in}{2.461229in}}%
\pgfpathlineto{\pgfqpoint{3.656247in}{2.464178in}}%
\pgfpathlineto{\pgfqpoint{3.660788in}{2.464178in}}%
\pgfpathlineto{\pgfqpoint{3.660788in}{2.461229in}}%
\pgfpathmoveto{\pgfqpoint{3.942330in}{2.083730in}}%
\pgfpathlineto{\pgfqpoint{3.942330in}{2.083730in}}%
\pgfpathlineto{\pgfqpoint{3.942330in}{2.086679in}}%
\pgfpathlineto{\pgfqpoint{3.946871in}{2.086679in}}%
\pgfpathlineto{\pgfqpoint{3.946871in}{2.083730in}}%
\pgfpathmoveto{\pgfqpoint{3.937789in}{2.089628in}}%
\pgfpathlineto{\pgfqpoint{3.937789in}{2.089628in}}%
\pgfpathlineto{\pgfqpoint{3.937789in}{2.092577in}}%
\pgfpathlineto{\pgfqpoint{3.942330in}{2.092577in}}%
\pgfpathlineto{\pgfqpoint{3.942330in}{2.089628in}}%
\pgfpathmoveto{\pgfqpoint{3.942330in}{2.086679in}}%
\pgfpathlineto{\pgfqpoint{3.942330in}{2.086679in}}%
\pgfpathlineto{\pgfqpoint{3.942330in}{2.089628in}}%
\pgfpathlineto{\pgfqpoint{3.946871in}{2.089628in}}%
\pgfpathlineto{\pgfqpoint{3.946871in}{2.086679in}}%
\pgfpathmoveto{\pgfqpoint{3.942330in}{2.089628in}}%
\pgfpathlineto{\pgfqpoint{3.942330in}{2.089628in}}%
\pgfpathlineto{\pgfqpoint{3.942330in}{2.092577in}}%
\pgfpathlineto{\pgfqpoint{3.946871in}{2.092577in}}%
\pgfpathlineto{\pgfqpoint{3.946871in}{2.089628in}}%
\pgfpathmoveto{\pgfqpoint{3.933248in}{2.095526in}}%
\pgfpathlineto{\pgfqpoint{3.933248in}{2.095526in}}%
\pgfpathlineto{\pgfqpoint{3.933248in}{2.098475in}}%
\pgfpathlineto{\pgfqpoint{3.937789in}{2.098475in}}%
\pgfpathlineto{\pgfqpoint{3.937789in}{2.095526in}}%
\pgfpathmoveto{\pgfqpoint{3.928707in}{2.101424in}}%
\pgfpathlineto{\pgfqpoint{3.928707in}{2.101424in}}%
\pgfpathlineto{\pgfqpoint{3.928707in}{2.104373in}}%
\pgfpathlineto{\pgfqpoint{3.933248in}{2.104373in}}%
\pgfpathlineto{\pgfqpoint{3.933248in}{2.101424in}}%
\pgfpathmoveto{\pgfqpoint{3.933248in}{2.098475in}}%
\pgfpathlineto{\pgfqpoint{3.933248in}{2.098475in}}%
\pgfpathlineto{\pgfqpoint{3.933248in}{2.101424in}}%
\pgfpathlineto{\pgfqpoint{3.937789in}{2.101424in}}%
\pgfpathlineto{\pgfqpoint{3.937789in}{2.098475in}}%
\pgfpathmoveto{\pgfqpoint{3.933248in}{2.101424in}}%
\pgfpathlineto{\pgfqpoint{3.933248in}{2.101424in}}%
\pgfpathlineto{\pgfqpoint{3.933248in}{2.104373in}}%
\pgfpathlineto{\pgfqpoint{3.937789in}{2.104373in}}%
\pgfpathlineto{\pgfqpoint{3.937789in}{2.101424in}}%
\pgfpathmoveto{\pgfqpoint{3.937789in}{2.092577in}}%
\pgfpathlineto{\pgfqpoint{3.937789in}{2.092577in}}%
\pgfpathlineto{\pgfqpoint{3.937789in}{2.095526in}}%
\pgfpathlineto{\pgfqpoint{3.942330in}{2.095526in}}%
\pgfpathlineto{\pgfqpoint{3.942330in}{2.092577in}}%
\pgfpathmoveto{\pgfqpoint{3.937789in}{2.095526in}}%
\pgfpathlineto{\pgfqpoint{3.937789in}{2.095526in}}%
\pgfpathlineto{\pgfqpoint{3.937789in}{2.098475in}}%
\pgfpathlineto{\pgfqpoint{3.942330in}{2.098475in}}%
\pgfpathlineto{\pgfqpoint{3.942330in}{2.095526in}}%
\pgfpathmoveto{\pgfqpoint{3.869674in}{2.178104in}}%
\pgfpathlineto{\pgfqpoint{3.869674in}{2.178104in}}%
\pgfpathlineto{\pgfqpoint{3.869674in}{2.181053in}}%
\pgfpathlineto{\pgfqpoint{3.874215in}{2.181053in}}%
\pgfpathlineto{\pgfqpoint{3.874215in}{2.178104in}}%
\pgfpathmoveto{\pgfqpoint{3.865133in}{2.184003in}}%
\pgfpathlineto{\pgfqpoint{3.865133in}{2.184003in}}%
\pgfpathlineto{\pgfqpoint{3.865133in}{2.186952in}}%
\pgfpathlineto{\pgfqpoint{3.869674in}{2.186952in}}%
\pgfpathlineto{\pgfqpoint{3.869674in}{2.184003in}}%
\pgfpathmoveto{\pgfqpoint{3.869674in}{2.181053in}}%
\pgfpathlineto{\pgfqpoint{3.869674in}{2.181053in}}%
\pgfpathlineto{\pgfqpoint{3.869674in}{2.184003in}}%
\pgfpathlineto{\pgfqpoint{3.874215in}{2.184003in}}%
\pgfpathlineto{\pgfqpoint{3.874215in}{2.181053in}}%
\pgfpathmoveto{\pgfqpoint{3.869674in}{2.184003in}}%
\pgfpathlineto{\pgfqpoint{3.869674in}{2.184003in}}%
\pgfpathlineto{\pgfqpoint{3.869674in}{2.186952in}}%
\pgfpathlineto{\pgfqpoint{3.874215in}{2.186952in}}%
\pgfpathlineto{\pgfqpoint{3.874215in}{2.184003in}}%
\pgfpathmoveto{\pgfqpoint{3.860592in}{2.189901in}}%
\pgfpathlineto{\pgfqpoint{3.860592in}{2.189901in}}%
\pgfpathlineto{\pgfqpoint{3.860592in}{2.192850in}}%
\pgfpathlineto{\pgfqpoint{3.865133in}{2.192850in}}%
\pgfpathlineto{\pgfqpoint{3.865133in}{2.189901in}}%
\pgfpathmoveto{\pgfqpoint{3.856051in}{2.195800in}}%
\pgfpathlineto{\pgfqpoint{3.856051in}{2.195800in}}%
\pgfpathlineto{\pgfqpoint{3.856051in}{2.198749in}}%
\pgfpathlineto{\pgfqpoint{3.860592in}{2.198749in}}%
\pgfpathlineto{\pgfqpoint{3.860592in}{2.195800in}}%
\pgfpathmoveto{\pgfqpoint{3.860592in}{2.192850in}}%
\pgfpathlineto{\pgfqpoint{3.860592in}{2.192850in}}%
\pgfpathlineto{\pgfqpoint{3.860592in}{2.195800in}}%
\pgfpathlineto{\pgfqpoint{3.865133in}{2.195800in}}%
\pgfpathlineto{\pgfqpoint{3.865133in}{2.192850in}}%
\pgfpathmoveto{\pgfqpoint{3.860592in}{2.195800in}}%
\pgfpathlineto{\pgfqpoint{3.860592in}{2.195800in}}%
\pgfpathlineto{\pgfqpoint{3.860592in}{2.198749in}}%
\pgfpathlineto{\pgfqpoint{3.865133in}{2.198749in}}%
\pgfpathlineto{\pgfqpoint{3.865133in}{2.195800in}}%
\pgfpathmoveto{\pgfqpoint{3.865133in}{2.186952in}}%
\pgfpathlineto{\pgfqpoint{3.865133in}{2.186952in}}%
\pgfpathlineto{\pgfqpoint{3.865133in}{2.189901in}}%
\pgfpathlineto{\pgfqpoint{3.869674in}{2.189901in}}%
\pgfpathlineto{\pgfqpoint{3.869674in}{2.186952in}}%
\pgfpathmoveto{\pgfqpoint{3.865133in}{2.189901in}}%
\pgfpathlineto{\pgfqpoint{3.865133in}{2.189901in}}%
\pgfpathlineto{\pgfqpoint{3.865133in}{2.192850in}}%
\pgfpathlineto{\pgfqpoint{3.869674in}{2.192850in}}%
\pgfpathlineto{\pgfqpoint{3.869674in}{2.189901in}}%
\pgfpathmoveto{\pgfqpoint{3.906002in}{2.130916in}}%
\pgfpathlineto{\pgfqpoint{3.906002in}{2.130916in}}%
\pgfpathlineto{\pgfqpoint{3.906002in}{2.133866in}}%
\pgfpathlineto{\pgfqpoint{3.910543in}{2.133866in}}%
\pgfpathlineto{\pgfqpoint{3.910543in}{2.130916in}}%
\pgfpathmoveto{\pgfqpoint{3.901461in}{2.136815in}}%
\pgfpathlineto{\pgfqpoint{3.901461in}{2.136815in}}%
\pgfpathlineto{\pgfqpoint{3.901461in}{2.139764in}}%
\pgfpathlineto{\pgfqpoint{3.906002in}{2.139764in}}%
\pgfpathlineto{\pgfqpoint{3.906002in}{2.136815in}}%
\pgfpathmoveto{\pgfqpoint{3.906002in}{2.133866in}}%
\pgfpathlineto{\pgfqpoint{3.906002in}{2.133866in}}%
\pgfpathlineto{\pgfqpoint{3.906002in}{2.136815in}}%
\pgfpathlineto{\pgfqpoint{3.910543in}{2.136815in}}%
\pgfpathlineto{\pgfqpoint{3.910543in}{2.133866in}}%
\pgfpathmoveto{\pgfqpoint{3.906002in}{2.136815in}}%
\pgfpathlineto{\pgfqpoint{3.906002in}{2.136815in}}%
\pgfpathlineto{\pgfqpoint{3.906002in}{2.139764in}}%
\pgfpathlineto{\pgfqpoint{3.910543in}{2.139764in}}%
\pgfpathlineto{\pgfqpoint{3.910543in}{2.136815in}}%
\pgfpathmoveto{\pgfqpoint{3.896920in}{2.142713in}}%
\pgfpathlineto{\pgfqpoint{3.896920in}{2.142713in}}%
\pgfpathlineto{\pgfqpoint{3.896920in}{2.145663in}}%
\pgfpathlineto{\pgfqpoint{3.901461in}{2.145663in}}%
\pgfpathlineto{\pgfqpoint{3.901461in}{2.142713in}}%
\pgfpathmoveto{\pgfqpoint{3.892379in}{2.148612in}}%
\pgfpathlineto{\pgfqpoint{3.892379in}{2.148612in}}%
\pgfpathlineto{\pgfqpoint{3.892379in}{2.151561in}}%
\pgfpathlineto{\pgfqpoint{3.896920in}{2.151561in}}%
\pgfpathlineto{\pgfqpoint{3.896920in}{2.148612in}}%
\pgfpathmoveto{\pgfqpoint{3.896920in}{2.145663in}}%
\pgfpathlineto{\pgfqpoint{3.896920in}{2.145663in}}%
\pgfpathlineto{\pgfqpoint{3.896920in}{2.148612in}}%
\pgfpathlineto{\pgfqpoint{3.901461in}{2.148612in}}%
\pgfpathlineto{\pgfqpoint{3.901461in}{2.145663in}}%
\pgfpathmoveto{\pgfqpoint{3.896920in}{2.148612in}}%
\pgfpathlineto{\pgfqpoint{3.896920in}{2.148612in}}%
\pgfpathlineto{\pgfqpoint{3.896920in}{2.151561in}}%
\pgfpathlineto{\pgfqpoint{3.901461in}{2.151561in}}%
\pgfpathlineto{\pgfqpoint{3.901461in}{2.148612in}}%
\pgfpathmoveto{\pgfqpoint{3.901461in}{2.139764in}}%
\pgfpathlineto{\pgfqpoint{3.901461in}{2.139764in}}%
\pgfpathlineto{\pgfqpoint{3.901461in}{2.142713in}}%
\pgfpathlineto{\pgfqpoint{3.906002in}{2.142713in}}%
\pgfpathlineto{\pgfqpoint{3.906002in}{2.139764in}}%
\pgfpathmoveto{\pgfqpoint{3.901461in}{2.142713in}}%
\pgfpathlineto{\pgfqpoint{3.901461in}{2.142713in}}%
\pgfpathlineto{\pgfqpoint{3.901461in}{2.145663in}}%
\pgfpathlineto{\pgfqpoint{3.906002in}{2.145663in}}%
\pgfpathlineto{\pgfqpoint{3.906002in}{2.142713in}}%
\pgfpathmoveto{\pgfqpoint{3.924166in}{2.107323in}}%
\pgfpathlineto{\pgfqpoint{3.924166in}{2.107323in}}%
\pgfpathlineto{\pgfqpoint{3.924166in}{2.110272in}}%
\pgfpathlineto{\pgfqpoint{3.928707in}{2.110272in}}%
\pgfpathlineto{\pgfqpoint{3.928707in}{2.107323in}}%
\pgfpathmoveto{\pgfqpoint{3.919625in}{2.113221in}}%
\pgfpathlineto{\pgfqpoint{3.919625in}{2.113221in}}%
\pgfpathlineto{\pgfqpoint{3.919625in}{2.116170in}}%
\pgfpathlineto{\pgfqpoint{3.924166in}{2.116170in}}%
\pgfpathlineto{\pgfqpoint{3.924166in}{2.113221in}}%
\pgfpathmoveto{\pgfqpoint{3.924166in}{2.110272in}}%
\pgfpathlineto{\pgfqpoint{3.924166in}{2.110272in}}%
\pgfpathlineto{\pgfqpoint{3.924166in}{2.113221in}}%
\pgfpathlineto{\pgfqpoint{3.928707in}{2.113221in}}%
\pgfpathlineto{\pgfqpoint{3.928707in}{2.110272in}}%
\pgfpathmoveto{\pgfqpoint{3.924166in}{2.113221in}}%
\pgfpathlineto{\pgfqpoint{3.924166in}{2.113221in}}%
\pgfpathlineto{\pgfqpoint{3.924166in}{2.116170in}}%
\pgfpathlineto{\pgfqpoint{3.928707in}{2.116170in}}%
\pgfpathlineto{\pgfqpoint{3.928707in}{2.113221in}}%
\pgfpathmoveto{\pgfqpoint{3.915084in}{2.119119in}}%
\pgfpathlineto{\pgfqpoint{3.915084in}{2.119119in}}%
\pgfpathlineto{\pgfqpoint{3.915084in}{2.122069in}}%
\pgfpathlineto{\pgfqpoint{3.919625in}{2.122069in}}%
\pgfpathlineto{\pgfqpoint{3.919625in}{2.119119in}}%
\pgfpathmoveto{\pgfqpoint{3.910543in}{2.125018in}}%
\pgfpathlineto{\pgfqpoint{3.910543in}{2.125018in}}%
\pgfpathlineto{\pgfqpoint{3.910543in}{2.127967in}}%
\pgfpathlineto{\pgfqpoint{3.915084in}{2.127967in}}%
\pgfpathlineto{\pgfqpoint{3.915084in}{2.125018in}}%
\pgfpathmoveto{\pgfqpoint{3.915084in}{2.122069in}}%
\pgfpathlineto{\pgfqpoint{3.915084in}{2.122069in}}%
\pgfpathlineto{\pgfqpoint{3.915084in}{2.125018in}}%
\pgfpathlineto{\pgfqpoint{3.919625in}{2.125018in}}%
\pgfpathlineto{\pgfqpoint{3.919625in}{2.122069in}}%
\pgfpathmoveto{\pgfqpoint{3.915084in}{2.125018in}}%
\pgfpathlineto{\pgfqpoint{3.915084in}{2.125018in}}%
\pgfpathlineto{\pgfqpoint{3.915084in}{2.127967in}}%
\pgfpathlineto{\pgfqpoint{3.919625in}{2.127967in}}%
\pgfpathlineto{\pgfqpoint{3.919625in}{2.125018in}}%
\pgfpathmoveto{\pgfqpoint{3.919625in}{2.116170in}}%
\pgfpathlineto{\pgfqpoint{3.919625in}{2.116170in}}%
\pgfpathlineto{\pgfqpoint{3.919625in}{2.119119in}}%
\pgfpathlineto{\pgfqpoint{3.924166in}{2.119119in}}%
\pgfpathlineto{\pgfqpoint{3.924166in}{2.116170in}}%
\pgfpathmoveto{\pgfqpoint{3.919625in}{2.119119in}}%
\pgfpathlineto{\pgfqpoint{3.919625in}{2.119119in}}%
\pgfpathlineto{\pgfqpoint{3.919625in}{2.122069in}}%
\pgfpathlineto{\pgfqpoint{3.924166in}{2.122069in}}%
\pgfpathlineto{\pgfqpoint{3.924166in}{2.119119in}}%
\pgfpathmoveto{\pgfqpoint{3.928707in}{2.104373in}}%
\pgfpathlineto{\pgfqpoint{3.928707in}{2.104373in}}%
\pgfpathlineto{\pgfqpoint{3.928707in}{2.107323in}}%
\pgfpathlineto{\pgfqpoint{3.933248in}{2.107323in}}%
\pgfpathlineto{\pgfqpoint{3.933248in}{2.104373in}}%
\pgfpathmoveto{\pgfqpoint{3.928707in}{2.107323in}}%
\pgfpathlineto{\pgfqpoint{3.928707in}{2.107323in}}%
\pgfpathlineto{\pgfqpoint{3.928707in}{2.110272in}}%
\pgfpathlineto{\pgfqpoint{3.933248in}{2.110272in}}%
\pgfpathlineto{\pgfqpoint{3.933248in}{2.107323in}}%
\pgfpathmoveto{\pgfqpoint{3.910543in}{2.127967in}}%
\pgfpathlineto{\pgfqpoint{3.910543in}{2.127967in}}%
\pgfpathlineto{\pgfqpoint{3.910543in}{2.130916in}}%
\pgfpathlineto{\pgfqpoint{3.915084in}{2.130916in}}%
\pgfpathlineto{\pgfqpoint{3.915084in}{2.127967in}}%
\pgfpathmoveto{\pgfqpoint{3.910543in}{2.130916in}}%
\pgfpathlineto{\pgfqpoint{3.910543in}{2.130916in}}%
\pgfpathlineto{\pgfqpoint{3.910543in}{2.133866in}}%
\pgfpathlineto{\pgfqpoint{3.915084in}{2.133866in}}%
\pgfpathlineto{\pgfqpoint{3.915084in}{2.130916in}}%
\pgfpathmoveto{\pgfqpoint{3.887838in}{2.154510in}}%
\pgfpathlineto{\pgfqpoint{3.887838in}{2.154510in}}%
\pgfpathlineto{\pgfqpoint{3.887838in}{2.157459in}}%
\pgfpathlineto{\pgfqpoint{3.892379in}{2.157459in}}%
\pgfpathlineto{\pgfqpoint{3.892379in}{2.154510in}}%
\pgfpathmoveto{\pgfqpoint{3.883297in}{2.160409in}}%
\pgfpathlineto{\pgfqpoint{3.883297in}{2.160409in}}%
\pgfpathlineto{\pgfqpoint{3.883297in}{2.163358in}}%
\pgfpathlineto{\pgfqpoint{3.887838in}{2.163358in}}%
\pgfpathlineto{\pgfqpoint{3.887838in}{2.160409in}}%
\pgfpathmoveto{\pgfqpoint{3.887838in}{2.157459in}}%
\pgfpathlineto{\pgfqpoint{3.887838in}{2.157459in}}%
\pgfpathlineto{\pgfqpoint{3.887838in}{2.160409in}}%
\pgfpathlineto{\pgfqpoint{3.892379in}{2.160409in}}%
\pgfpathlineto{\pgfqpoint{3.892379in}{2.157459in}}%
\pgfpathmoveto{\pgfqpoint{3.887838in}{2.160409in}}%
\pgfpathlineto{\pgfqpoint{3.887838in}{2.160409in}}%
\pgfpathlineto{\pgfqpoint{3.887838in}{2.163358in}}%
\pgfpathlineto{\pgfqpoint{3.892379in}{2.163358in}}%
\pgfpathlineto{\pgfqpoint{3.892379in}{2.160409in}}%
\pgfpathmoveto{\pgfqpoint{3.878756in}{2.166307in}}%
\pgfpathlineto{\pgfqpoint{3.878756in}{2.166307in}}%
\pgfpathlineto{\pgfqpoint{3.878756in}{2.169256in}}%
\pgfpathlineto{\pgfqpoint{3.883297in}{2.169256in}}%
\pgfpathlineto{\pgfqpoint{3.883297in}{2.166307in}}%
\pgfpathmoveto{\pgfqpoint{3.874215in}{2.172206in}}%
\pgfpathlineto{\pgfqpoint{3.874215in}{2.172206in}}%
\pgfpathlineto{\pgfqpoint{3.874215in}{2.175155in}}%
\pgfpathlineto{\pgfqpoint{3.878756in}{2.175155in}}%
\pgfpathlineto{\pgfqpoint{3.878756in}{2.172206in}}%
\pgfpathmoveto{\pgfqpoint{3.878756in}{2.169256in}}%
\pgfpathlineto{\pgfqpoint{3.878756in}{2.169256in}}%
\pgfpathlineto{\pgfqpoint{3.878756in}{2.172206in}}%
\pgfpathlineto{\pgfqpoint{3.883297in}{2.172206in}}%
\pgfpathlineto{\pgfqpoint{3.883297in}{2.169256in}}%
\pgfpathmoveto{\pgfqpoint{3.878756in}{2.172206in}}%
\pgfpathlineto{\pgfqpoint{3.878756in}{2.172206in}}%
\pgfpathlineto{\pgfqpoint{3.878756in}{2.175155in}}%
\pgfpathlineto{\pgfqpoint{3.883297in}{2.175155in}}%
\pgfpathlineto{\pgfqpoint{3.883297in}{2.172206in}}%
\pgfpathmoveto{\pgfqpoint{3.883297in}{2.163358in}}%
\pgfpathlineto{\pgfqpoint{3.883297in}{2.163358in}}%
\pgfpathlineto{\pgfqpoint{3.883297in}{2.166307in}}%
\pgfpathlineto{\pgfqpoint{3.887838in}{2.166307in}}%
\pgfpathlineto{\pgfqpoint{3.887838in}{2.163358in}}%
\pgfpathmoveto{\pgfqpoint{3.883297in}{2.166307in}}%
\pgfpathlineto{\pgfqpoint{3.883297in}{2.166307in}}%
\pgfpathlineto{\pgfqpoint{3.883297in}{2.169256in}}%
\pgfpathlineto{\pgfqpoint{3.887838in}{2.169256in}}%
\pgfpathlineto{\pgfqpoint{3.887838in}{2.166307in}}%
\pgfpathmoveto{\pgfqpoint{3.892379in}{2.151561in}}%
\pgfpathlineto{\pgfqpoint{3.892379in}{2.151561in}}%
\pgfpathlineto{\pgfqpoint{3.892379in}{2.154510in}}%
\pgfpathlineto{\pgfqpoint{3.896920in}{2.154510in}}%
\pgfpathlineto{\pgfqpoint{3.896920in}{2.151561in}}%
\pgfpathmoveto{\pgfqpoint{3.892379in}{2.154510in}}%
\pgfpathlineto{\pgfqpoint{3.892379in}{2.154510in}}%
\pgfpathlineto{\pgfqpoint{3.892379in}{2.157459in}}%
\pgfpathlineto{\pgfqpoint{3.896920in}{2.157459in}}%
\pgfpathlineto{\pgfqpoint{3.896920in}{2.154510in}}%
\pgfpathmoveto{\pgfqpoint{3.874215in}{2.175155in}}%
\pgfpathlineto{\pgfqpoint{3.874215in}{2.175155in}}%
\pgfpathlineto{\pgfqpoint{3.874215in}{2.178104in}}%
\pgfpathlineto{\pgfqpoint{3.878756in}{2.178104in}}%
\pgfpathlineto{\pgfqpoint{3.878756in}{2.175155in}}%
\pgfpathmoveto{\pgfqpoint{3.874215in}{2.178104in}}%
\pgfpathlineto{\pgfqpoint{3.874215in}{2.178104in}}%
\pgfpathlineto{\pgfqpoint{3.874215in}{2.181053in}}%
\pgfpathlineto{\pgfqpoint{3.878756in}{2.181053in}}%
\pgfpathlineto{\pgfqpoint{3.878756in}{2.178104in}}%
\pgfpathmoveto{\pgfqpoint{3.833346in}{2.225293in}}%
\pgfpathlineto{\pgfqpoint{3.833346in}{2.225293in}}%
\pgfpathlineto{\pgfqpoint{3.833346in}{2.228242in}}%
\pgfpathlineto{\pgfqpoint{3.837887in}{2.228242in}}%
\pgfpathlineto{\pgfqpoint{3.837887in}{2.225293in}}%
\pgfpathmoveto{\pgfqpoint{3.828805in}{2.231191in}}%
\pgfpathlineto{\pgfqpoint{3.828805in}{2.231191in}}%
\pgfpathlineto{\pgfqpoint{3.828805in}{2.234141in}}%
\pgfpathlineto{\pgfqpoint{3.833346in}{2.234141in}}%
\pgfpathlineto{\pgfqpoint{3.833346in}{2.231191in}}%
\pgfpathmoveto{\pgfqpoint{3.833346in}{2.228242in}}%
\pgfpathlineto{\pgfqpoint{3.833346in}{2.228242in}}%
\pgfpathlineto{\pgfqpoint{3.833346in}{2.231191in}}%
\pgfpathlineto{\pgfqpoint{3.837887in}{2.231191in}}%
\pgfpathlineto{\pgfqpoint{3.837887in}{2.228242in}}%
\pgfpathmoveto{\pgfqpoint{3.833346in}{2.231191in}}%
\pgfpathlineto{\pgfqpoint{3.833346in}{2.231191in}}%
\pgfpathlineto{\pgfqpoint{3.833346in}{2.234141in}}%
\pgfpathlineto{\pgfqpoint{3.837887in}{2.234141in}}%
\pgfpathlineto{\pgfqpoint{3.837887in}{2.231191in}}%
\pgfpathmoveto{\pgfqpoint{3.824264in}{2.237090in}}%
\pgfpathlineto{\pgfqpoint{3.824264in}{2.237090in}}%
\pgfpathlineto{\pgfqpoint{3.824264in}{2.240039in}}%
\pgfpathlineto{\pgfqpoint{3.828805in}{2.240039in}}%
\pgfpathlineto{\pgfqpoint{3.828805in}{2.237090in}}%
\pgfpathmoveto{\pgfqpoint{3.819723in}{2.242989in}}%
\pgfpathlineto{\pgfqpoint{3.819723in}{2.242989in}}%
\pgfpathlineto{\pgfqpoint{3.819723in}{2.245938in}}%
\pgfpathlineto{\pgfqpoint{3.824264in}{2.245938in}}%
\pgfpathlineto{\pgfqpoint{3.824264in}{2.242989in}}%
\pgfpathmoveto{\pgfqpoint{3.824264in}{2.240039in}}%
\pgfpathlineto{\pgfqpoint{3.824264in}{2.240039in}}%
\pgfpathlineto{\pgfqpoint{3.824264in}{2.242989in}}%
\pgfpathlineto{\pgfqpoint{3.828805in}{2.242989in}}%
\pgfpathlineto{\pgfqpoint{3.828805in}{2.240039in}}%
\pgfpathmoveto{\pgfqpoint{3.824264in}{2.242989in}}%
\pgfpathlineto{\pgfqpoint{3.824264in}{2.242989in}}%
\pgfpathlineto{\pgfqpoint{3.824264in}{2.245938in}}%
\pgfpathlineto{\pgfqpoint{3.828805in}{2.245938in}}%
\pgfpathlineto{\pgfqpoint{3.828805in}{2.242989in}}%
\pgfpathmoveto{\pgfqpoint{3.828805in}{2.234141in}}%
\pgfpathlineto{\pgfqpoint{3.828805in}{2.234141in}}%
\pgfpathlineto{\pgfqpoint{3.828805in}{2.237090in}}%
\pgfpathlineto{\pgfqpoint{3.833346in}{2.237090in}}%
\pgfpathlineto{\pgfqpoint{3.833346in}{2.234141in}}%
\pgfpathmoveto{\pgfqpoint{3.828805in}{2.237090in}}%
\pgfpathlineto{\pgfqpoint{3.828805in}{2.237090in}}%
\pgfpathlineto{\pgfqpoint{3.828805in}{2.240039in}}%
\pgfpathlineto{\pgfqpoint{3.833346in}{2.240039in}}%
\pgfpathlineto{\pgfqpoint{3.833346in}{2.237090in}}%
\pgfpathmoveto{\pgfqpoint{3.851510in}{2.201698in}}%
\pgfpathlineto{\pgfqpoint{3.851510in}{2.201698in}}%
\pgfpathlineto{\pgfqpoint{3.851510in}{2.204647in}}%
\pgfpathlineto{\pgfqpoint{3.856051in}{2.204647in}}%
\pgfpathlineto{\pgfqpoint{3.856051in}{2.201698in}}%
\pgfpathmoveto{\pgfqpoint{3.846969in}{2.207597in}}%
\pgfpathlineto{\pgfqpoint{3.846969in}{2.207597in}}%
\pgfpathlineto{\pgfqpoint{3.846969in}{2.210546in}}%
\pgfpathlineto{\pgfqpoint{3.851510in}{2.210546in}}%
\pgfpathlineto{\pgfqpoint{3.851510in}{2.207597in}}%
\pgfpathmoveto{\pgfqpoint{3.851510in}{2.204647in}}%
\pgfpathlineto{\pgfqpoint{3.851510in}{2.204647in}}%
\pgfpathlineto{\pgfqpoint{3.851510in}{2.207597in}}%
\pgfpathlineto{\pgfqpoint{3.856051in}{2.207597in}}%
\pgfpathlineto{\pgfqpoint{3.856051in}{2.204647in}}%
\pgfpathmoveto{\pgfqpoint{3.851510in}{2.207597in}}%
\pgfpathlineto{\pgfqpoint{3.851510in}{2.207597in}}%
\pgfpathlineto{\pgfqpoint{3.851510in}{2.210546in}}%
\pgfpathlineto{\pgfqpoint{3.856051in}{2.210546in}}%
\pgfpathlineto{\pgfqpoint{3.856051in}{2.207597in}}%
\pgfpathmoveto{\pgfqpoint{3.842428in}{2.213495in}}%
\pgfpathlineto{\pgfqpoint{3.842428in}{2.213495in}}%
\pgfpathlineto{\pgfqpoint{3.842428in}{2.216445in}}%
\pgfpathlineto{\pgfqpoint{3.846969in}{2.216445in}}%
\pgfpathlineto{\pgfqpoint{3.846969in}{2.213495in}}%
\pgfpathmoveto{\pgfqpoint{3.837887in}{2.219394in}}%
\pgfpathlineto{\pgfqpoint{3.837887in}{2.219394in}}%
\pgfpathlineto{\pgfqpoint{3.837887in}{2.222343in}}%
\pgfpathlineto{\pgfqpoint{3.842428in}{2.222343in}}%
\pgfpathlineto{\pgfqpoint{3.842428in}{2.219394in}}%
\pgfpathmoveto{\pgfqpoint{3.842428in}{2.216445in}}%
\pgfpathlineto{\pgfqpoint{3.842428in}{2.216445in}}%
\pgfpathlineto{\pgfqpoint{3.842428in}{2.219394in}}%
\pgfpathlineto{\pgfqpoint{3.846969in}{2.219394in}}%
\pgfpathlineto{\pgfqpoint{3.846969in}{2.216445in}}%
\pgfpathmoveto{\pgfqpoint{3.842428in}{2.219394in}}%
\pgfpathlineto{\pgfqpoint{3.842428in}{2.219394in}}%
\pgfpathlineto{\pgfqpoint{3.842428in}{2.222343in}}%
\pgfpathlineto{\pgfqpoint{3.846969in}{2.222343in}}%
\pgfpathlineto{\pgfqpoint{3.846969in}{2.219394in}}%
\pgfpathmoveto{\pgfqpoint{3.846969in}{2.210546in}}%
\pgfpathlineto{\pgfqpoint{3.846969in}{2.210546in}}%
\pgfpathlineto{\pgfqpoint{3.846969in}{2.213495in}}%
\pgfpathlineto{\pgfqpoint{3.851510in}{2.213495in}}%
\pgfpathlineto{\pgfqpoint{3.851510in}{2.210546in}}%
\pgfpathmoveto{\pgfqpoint{3.846969in}{2.213495in}}%
\pgfpathlineto{\pgfqpoint{3.846969in}{2.213495in}}%
\pgfpathlineto{\pgfqpoint{3.846969in}{2.216445in}}%
\pgfpathlineto{\pgfqpoint{3.851510in}{2.216445in}}%
\pgfpathlineto{\pgfqpoint{3.851510in}{2.213495in}}%
\pgfpathmoveto{\pgfqpoint{3.856051in}{2.198749in}}%
\pgfpathlineto{\pgfqpoint{3.856051in}{2.198749in}}%
\pgfpathlineto{\pgfqpoint{3.856051in}{2.201698in}}%
\pgfpathlineto{\pgfqpoint{3.860592in}{2.201698in}}%
\pgfpathlineto{\pgfqpoint{3.860592in}{2.198749in}}%
\pgfpathmoveto{\pgfqpoint{3.856051in}{2.201698in}}%
\pgfpathlineto{\pgfqpoint{3.856051in}{2.201698in}}%
\pgfpathlineto{\pgfqpoint{3.856051in}{2.204647in}}%
\pgfpathlineto{\pgfqpoint{3.860592in}{2.204647in}}%
\pgfpathlineto{\pgfqpoint{3.860592in}{2.201698in}}%
\pgfpathmoveto{\pgfqpoint{3.837887in}{2.222343in}}%
\pgfpathlineto{\pgfqpoint{3.837887in}{2.222343in}}%
\pgfpathlineto{\pgfqpoint{3.837887in}{2.225293in}}%
\pgfpathlineto{\pgfqpoint{3.842428in}{2.225293in}}%
\pgfpathlineto{\pgfqpoint{3.842428in}{2.222343in}}%
\pgfpathmoveto{\pgfqpoint{3.837887in}{2.225293in}}%
\pgfpathlineto{\pgfqpoint{3.837887in}{2.225293in}}%
\pgfpathlineto{\pgfqpoint{3.837887in}{2.228242in}}%
\pgfpathlineto{\pgfqpoint{3.842428in}{2.228242in}}%
\pgfpathlineto{\pgfqpoint{3.842428in}{2.225293in}}%
\pgfpathmoveto{\pgfqpoint{3.815182in}{2.248887in}}%
\pgfpathlineto{\pgfqpoint{3.815182in}{2.248887in}}%
\pgfpathlineto{\pgfqpoint{3.815182in}{2.251837in}}%
\pgfpathlineto{\pgfqpoint{3.819723in}{2.251837in}}%
\pgfpathlineto{\pgfqpoint{3.819723in}{2.248887in}}%
\pgfpathmoveto{\pgfqpoint{3.810641in}{2.254786in}}%
\pgfpathlineto{\pgfqpoint{3.810641in}{2.254786in}}%
\pgfpathlineto{\pgfqpoint{3.810641in}{2.257735in}}%
\pgfpathlineto{\pgfqpoint{3.815182in}{2.257735in}}%
\pgfpathlineto{\pgfqpoint{3.815182in}{2.254786in}}%
\pgfpathmoveto{\pgfqpoint{3.815182in}{2.251837in}}%
\pgfpathlineto{\pgfqpoint{3.815182in}{2.251837in}}%
\pgfpathlineto{\pgfqpoint{3.815182in}{2.254786in}}%
\pgfpathlineto{\pgfqpoint{3.819723in}{2.254786in}}%
\pgfpathlineto{\pgfqpoint{3.819723in}{2.251837in}}%
\pgfpathmoveto{\pgfqpoint{3.815182in}{2.254786in}}%
\pgfpathlineto{\pgfqpoint{3.815182in}{2.254786in}}%
\pgfpathlineto{\pgfqpoint{3.815182in}{2.257735in}}%
\pgfpathlineto{\pgfqpoint{3.819723in}{2.257735in}}%
\pgfpathlineto{\pgfqpoint{3.819723in}{2.254786in}}%
\pgfpathmoveto{\pgfqpoint{3.806100in}{2.260685in}}%
\pgfpathlineto{\pgfqpoint{3.806100in}{2.260685in}}%
\pgfpathlineto{\pgfqpoint{3.806100in}{2.263634in}}%
\pgfpathlineto{\pgfqpoint{3.810641in}{2.263634in}}%
\pgfpathlineto{\pgfqpoint{3.810641in}{2.260685in}}%
\pgfpathmoveto{\pgfqpoint{3.801559in}{2.266583in}}%
\pgfpathlineto{\pgfqpoint{3.801559in}{2.266583in}}%
\pgfpathlineto{\pgfqpoint{3.801559in}{2.269533in}}%
\pgfpathlineto{\pgfqpoint{3.806100in}{2.269533in}}%
\pgfpathlineto{\pgfqpoint{3.806100in}{2.266583in}}%
\pgfpathmoveto{\pgfqpoint{3.806100in}{2.263634in}}%
\pgfpathlineto{\pgfqpoint{3.806100in}{2.263634in}}%
\pgfpathlineto{\pgfqpoint{3.806100in}{2.266583in}}%
\pgfpathlineto{\pgfqpoint{3.810641in}{2.266583in}}%
\pgfpathlineto{\pgfqpoint{3.810641in}{2.263634in}}%
\pgfpathmoveto{\pgfqpoint{3.806100in}{2.266583in}}%
\pgfpathlineto{\pgfqpoint{3.806100in}{2.266583in}}%
\pgfpathlineto{\pgfqpoint{3.806100in}{2.269533in}}%
\pgfpathlineto{\pgfqpoint{3.810641in}{2.269533in}}%
\pgfpathlineto{\pgfqpoint{3.810641in}{2.266583in}}%
\pgfpathmoveto{\pgfqpoint{3.810641in}{2.257735in}}%
\pgfpathlineto{\pgfqpoint{3.810641in}{2.257735in}}%
\pgfpathlineto{\pgfqpoint{3.810641in}{2.260685in}}%
\pgfpathlineto{\pgfqpoint{3.815182in}{2.260685in}}%
\pgfpathlineto{\pgfqpoint{3.815182in}{2.257735in}}%
\pgfpathmoveto{\pgfqpoint{3.810641in}{2.260685in}}%
\pgfpathlineto{\pgfqpoint{3.810641in}{2.260685in}}%
\pgfpathlineto{\pgfqpoint{3.810641in}{2.263634in}}%
\pgfpathlineto{\pgfqpoint{3.815182in}{2.263634in}}%
\pgfpathlineto{\pgfqpoint{3.815182in}{2.260685in}}%
\pgfpathmoveto{\pgfqpoint{3.819723in}{2.245938in}}%
\pgfpathlineto{\pgfqpoint{3.819723in}{2.245938in}}%
\pgfpathlineto{\pgfqpoint{3.819723in}{2.248887in}}%
\pgfpathlineto{\pgfqpoint{3.824264in}{2.248887in}}%
\pgfpathlineto{\pgfqpoint{3.824264in}{2.245938in}}%
\pgfpathmoveto{\pgfqpoint{3.819723in}{2.248887in}}%
\pgfpathlineto{\pgfqpoint{3.819723in}{2.248887in}}%
\pgfpathlineto{\pgfqpoint{3.819723in}{2.251837in}}%
\pgfpathlineto{\pgfqpoint{3.824264in}{2.251837in}}%
\pgfpathlineto{\pgfqpoint{3.824264in}{2.248887in}}%
\pgfpathmoveto{\pgfqpoint{3.801559in}{2.269533in}}%
\pgfpathlineto{\pgfqpoint{3.801559in}{2.269533in}}%
\pgfpathlineto{\pgfqpoint{3.801559in}{2.272482in}}%
\pgfpathlineto{\pgfqpoint{3.806100in}{2.272482in}}%
\pgfpathlineto{\pgfqpoint{3.806100in}{2.269533in}}%
\pgfpathmoveto{\pgfqpoint{3.801559in}{2.272482in}}%
\pgfpathlineto{\pgfqpoint{3.801559in}{2.272482in}}%
\pgfpathlineto{\pgfqpoint{3.801559in}{2.275431in}}%
\pgfpathlineto{\pgfqpoint{3.806100in}{2.275431in}}%
\pgfpathlineto{\pgfqpoint{3.806100in}{2.272482in}}%
\pgfpathmoveto{\pgfqpoint{4.087644in}{1.894979in}}%
\pgfpathlineto{\pgfqpoint{4.087644in}{1.894979in}}%
\pgfpathlineto{\pgfqpoint{4.087644in}{1.897928in}}%
\pgfpathlineto{\pgfqpoint{4.092185in}{1.897928in}}%
\pgfpathlineto{\pgfqpoint{4.092185in}{1.894979in}}%
\pgfpathmoveto{\pgfqpoint{4.083103in}{1.900877in}}%
\pgfpathlineto{\pgfqpoint{4.083103in}{1.900877in}}%
\pgfpathlineto{\pgfqpoint{4.083103in}{1.903826in}}%
\pgfpathlineto{\pgfqpoint{4.087644in}{1.903826in}}%
\pgfpathlineto{\pgfqpoint{4.087644in}{1.900877in}}%
\pgfpathmoveto{\pgfqpoint{4.087644in}{1.897928in}}%
\pgfpathlineto{\pgfqpoint{4.087644in}{1.897928in}}%
\pgfpathlineto{\pgfqpoint{4.087644in}{1.900877in}}%
\pgfpathlineto{\pgfqpoint{4.092185in}{1.900877in}}%
\pgfpathlineto{\pgfqpoint{4.092185in}{1.897928in}}%
\pgfpathmoveto{\pgfqpoint{4.087644in}{1.900877in}}%
\pgfpathlineto{\pgfqpoint{4.087644in}{1.900877in}}%
\pgfpathlineto{\pgfqpoint{4.087644in}{1.903826in}}%
\pgfpathlineto{\pgfqpoint{4.092185in}{1.903826in}}%
\pgfpathlineto{\pgfqpoint{4.092185in}{1.900877in}}%
\pgfpathmoveto{\pgfqpoint{4.078562in}{1.906775in}}%
\pgfpathlineto{\pgfqpoint{4.078562in}{1.906775in}}%
\pgfpathlineto{\pgfqpoint{4.078562in}{1.909725in}}%
\pgfpathlineto{\pgfqpoint{4.083103in}{1.909725in}}%
\pgfpathlineto{\pgfqpoint{4.083103in}{1.906775in}}%
\pgfpathmoveto{\pgfqpoint{4.074021in}{1.912674in}}%
\pgfpathlineto{\pgfqpoint{4.074021in}{1.912674in}}%
\pgfpathlineto{\pgfqpoint{4.074021in}{1.915623in}}%
\pgfpathlineto{\pgfqpoint{4.078562in}{1.915623in}}%
\pgfpathlineto{\pgfqpoint{4.078562in}{1.912674in}}%
\pgfpathmoveto{\pgfqpoint{4.078562in}{1.909725in}}%
\pgfpathlineto{\pgfqpoint{4.078562in}{1.909725in}}%
\pgfpathlineto{\pgfqpoint{4.078562in}{1.912674in}}%
\pgfpathlineto{\pgfqpoint{4.083103in}{1.912674in}}%
\pgfpathlineto{\pgfqpoint{4.083103in}{1.909725in}}%
\pgfpathmoveto{\pgfqpoint{4.078562in}{1.912674in}}%
\pgfpathlineto{\pgfqpoint{4.078562in}{1.912674in}}%
\pgfpathlineto{\pgfqpoint{4.078562in}{1.915623in}}%
\pgfpathlineto{\pgfqpoint{4.083103in}{1.915623in}}%
\pgfpathlineto{\pgfqpoint{4.083103in}{1.912674in}}%
\pgfpathmoveto{\pgfqpoint{4.083103in}{1.903826in}}%
\pgfpathlineto{\pgfqpoint{4.083103in}{1.903826in}}%
\pgfpathlineto{\pgfqpoint{4.083103in}{1.906775in}}%
\pgfpathlineto{\pgfqpoint{4.087644in}{1.906775in}}%
\pgfpathlineto{\pgfqpoint{4.087644in}{1.903826in}}%
\pgfpathmoveto{\pgfqpoint{4.083103in}{1.906775in}}%
\pgfpathlineto{\pgfqpoint{4.083103in}{1.906775in}}%
\pgfpathlineto{\pgfqpoint{4.083103in}{1.909725in}}%
\pgfpathlineto{\pgfqpoint{4.087644in}{1.909725in}}%
\pgfpathlineto{\pgfqpoint{4.087644in}{1.906775in}}%
\pgfpathmoveto{\pgfqpoint{4.014987in}{1.989356in}}%
\pgfpathlineto{\pgfqpoint{4.014987in}{1.989356in}}%
\pgfpathlineto{\pgfqpoint{4.014987in}{1.992306in}}%
\pgfpathlineto{\pgfqpoint{4.019528in}{1.992306in}}%
\pgfpathlineto{\pgfqpoint{4.019528in}{1.989356in}}%
\pgfpathmoveto{\pgfqpoint{4.010446in}{1.995255in}}%
\pgfpathlineto{\pgfqpoint{4.010446in}{1.995255in}}%
\pgfpathlineto{\pgfqpoint{4.010446in}{1.998204in}}%
\pgfpathlineto{\pgfqpoint{4.014987in}{1.998204in}}%
\pgfpathlineto{\pgfqpoint{4.014987in}{1.995255in}}%
\pgfpathmoveto{\pgfqpoint{4.014987in}{1.992306in}}%
\pgfpathlineto{\pgfqpoint{4.014987in}{1.992306in}}%
\pgfpathlineto{\pgfqpoint{4.014987in}{1.995255in}}%
\pgfpathlineto{\pgfqpoint{4.019528in}{1.995255in}}%
\pgfpathlineto{\pgfqpoint{4.019528in}{1.992306in}}%
\pgfpathmoveto{\pgfqpoint{4.014987in}{1.995255in}}%
\pgfpathlineto{\pgfqpoint{4.014987in}{1.995255in}}%
\pgfpathlineto{\pgfqpoint{4.014987in}{1.998204in}}%
\pgfpathlineto{\pgfqpoint{4.019528in}{1.998204in}}%
\pgfpathlineto{\pgfqpoint{4.019528in}{1.995255in}}%
\pgfpathmoveto{\pgfqpoint{4.005905in}{2.001154in}}%
\pgfpathlineto{\pgfqpoint{4.005905in}{2.001154in}}%
\pgfpathlineto{\pgfqpoint{4.005905in}{2.004103in}}%
\pgfpathlineto{\pgfqpoint{4.010446in}{2.004103in}}%
\pgfpathlineto{\pgfqpoint{4.010446in}{2.001154in}}%
\pgfpathmoveto{\pgfqpoint{4.001364in}{2.007052in}}%
\pgfpathlineto{\pgfqpoint{4.001364in}{2.007052in}}%
\pgfpathlineto{\pgfqpoint{4.001364in}{2.010002in}}%
\pgfpathlineto{\pgfqpoint{4.005905in}{2.010002in}}%
\pgfpathlineto{\pgfqpoint{4.005905in}{2.007052in}}%
\pgfpathmoveto{\pgfqpoint{4.005905in}{2.004103in}}%
\pgfpathlineto{\pgfqpoint{4.005905in}{2.004103in}}%
\pgfpathlineto{\pgfqpoint{4.005905in}{2.007052in}}%
\pgfpathlineto{\pgfqpoint{4.010446in}{2.007052in}}%
\pgfpathlineto{\pgfqpoint{4.010446in}{2.004103in}}%
\pgfpathmoveto{\pgfqpoint{4.005905in}{2.007052in}}%
\pgfpathlineto{\pgfqpoint{4.005905in}{2.007052in}}%
\pgfpathlineto{\pgfqpoint{4.005905in}{2.010002in}}%
\pgfpathlineto{\pgfqpoint{4.010446in}{2.010002in}}%
\pgfpathlineto{\pgfqpoint{4.010446in}{2.007052in}}%
\pgfpathmoveto{\pgfqpoint{4.010446in}{1.998204in}}%
\pgfpathlineto{\pgfqpoint{4.010446in}{1.998204in}}%
\pgfpathlineto{\pgfqpoint{4.010446in}{2.001154in}}%
\pgfpathlineto{\pgfqpoint{4.014987in}{2.001154in}}%
\pgfpathlineto{\pgfqpoint{4.014987in}{1.998204in}}%
\pgfpathmoveto{\pgfqpoint{4.010446in}{2.001154in}}%
\pgfpathlineto{\pgfqpoint{4.010446in}{2.001154in}}%
\pgfpathlineto{\pgfqpoint{4.010446in}{2.004103in}}%
\pgfpathlineto{\pgfqpoint{4.014987in}{2.004103in}}%
\pgfpathlineto{\pgfqpoint{4.014987in}{2.001154in}}%
\pgfpathmoveto{\pgfqpoint{4.051315in}{1.942167in}}%
\pgfpathlineto{\pgfqpoint{4.051315in}{1.942167in}}%
\pgfpathlineto{\pgfqpoint{4.051315in}{1.945116in}}%
\pgfpathlineto{\pgfqpoint{4.055856in}{1.945116in}}%
\pgfpathlineto{\pgfqpoint{4.055856in}{1.942167in}}%
\pgfpathmoveto{\pgfqpoint{4.046774in}{1.948066in}}%
\pgfpathlineto{\pgfqpoint{4.046774in}{1.948066in}}%
\pgfpathlineto{\pgfqpoint{4.046774in}{1.951015in}}%
\pgfpathlineto{\pgfqpoint{4.051315in}{1.951015in}}%
\pgfpathlineto{\pgfqpoint{4.051315in}{1.948066in}}%
\pgfpathmoveto{\pgfqpoint{4.051315in}{1.945116in}}%
\pgfpathlineto{\pgfqpoint{4.051315in}{1.945116in}}%
\pgfpathlineto{\pgfqpoint{4.051315in}{1.948066in}}%
\pgfpathlineto{\pgfqpoint{4.055856in}{1.948066in}}%
\pgfpathlineto{\pgfqpoint{4.055856in}{1.945116in}}%
\pgfpathmoveto{\pgfqpoint{4.051315in}{1.948066in}}%
\pgfpathlineto{\pgfqpoint{4.051315in}{1.948066in}}%
\pgfpathlineto{\pgfqpoint{4.051315in}{1.951015in}}%
\pgfpathlineto{\pgfqpoint{4.055856in}{1.951015in}}%
\pgfpathlineto{\pgfqpoint{4.055856in}{1.948066in}}%
\pgfpathmoveto{\pgfqpoint{4.042233in}{1.953964in}}%
\pgfpathlineto{\pgfqpoint{4.042233in}{1.953964in}}%
\pgfpathlineto{\pgfqpoint{4.042233in}{1.956914in}}%
\pgfpathlineto{\pgfqpoint{4.046774in}{1.956914in}}%
\pgfpathlineto{\pgfqpoint{4.046774in}{1.953964in}}%
\pgfpathmoveto{\pgfqpoint{4.037692in}{1.959863in}}%
\pgfpathlineto{\pgfqpoint{4.037692in}{1.959863in}}%
\pgfpathlineto{\pgfqpoint{4.037692in}{1.962812in}}%
\pgfpathlineto{\pgfqpoint{4.042233in}{1.962812in}}%
\pgfpathlineto{\pgfqpoint{4.042233in}{1.959863in}}%
\pgfpathmoveto{\pgfqpoint{4.042233in}{1.956914in}}%
\pgfpathlineto{\pgfqpoint{4.042233in}{1.956914in}}%
\pgfpathlineto{\pgfqpoint{4.042233in}{1.959863in}}%
\pgfpathlineto{\pgfqpoint{4.046774in}{1.959863in}}%
\pgfpathlineto{\pgfqpoint{4.046774in}{1.956914in}}%
\pgfpathmoveto{\pgfqpoint{4.042233in}{1.959863in}}%
\pgfpathlineto{\pgfqpoint{4.042233in}{1.959863in}}%
\pgfpathlineto{\pgfqpoint{4.042233in}{1.962812in}}%
\pgfpathlineto{\pgfqpoint{4.046774in}{1.962812in}}%
\pgfpathlineto{\pgfqpoint{4.046774in}{1.959863in}}%
\pgfpathmoveto{\pgfqpoint{4.046774in}{1.951015in}}%
\pgfpathlineto{\pgfqpoint{4.046774in}{1.951015in}}%
\pgfpathlineto{\pgfqpoint{4.046774in}{1.953964in}}%
\pgfpathlineto{\pgfqpoint{4.051315in}{1.953964in}}%
\pgfpathlineto{\pgfqpoint{4.051315in}{1.951015in}}%
\pgfpathmoveto{\pgfqpoint{4.046774in}{1.953964in}}%
\pgfpathlineto{\pgfqpoint{4.046774in}{1.953964in}}%
\pgfpathlineto{\pgfqpoint{4.046774in}{1.956914in}}%
\pgfpathlineto{\pgfqpoint{4.051315in}{1.956914in}}%
\pgfpathlineto{\pgfqpoint{4.051315in}{1.953964in}}%
\pgfpathmoveto{\pgfqpoint{4.069479in}{1.918572in}}%
\pgfpathlineto{\pgfqpoint{4.069479in}{1.918572in}}%
\pgfpathlineto{\pgfqpoint{4.069479in}{1.921522in}}%
\pgfpathlineto{\pgfqpoint{4.074021in}{1.921522in}}%
\pgfpathlineto{\pgfqpoint{4.074021in}{1.918572in}}%
\pgfpathmoveto{\pgfqpoint{4.064938in}{1.924471in}}%
\pgfpathlineto{\pgfqpoint{4.064938in}{1.924471in}}%
\pgfpathlineto{\pgfqpoint{4.064938in}{1.927420in}}%
\pgfpathlineto{\pgfqpoint{4.069479in}{1.927420in}}%
\pgfpathlineto{\pgfqpoint{4.069479in}{1.924471in}}%
\pgfpathmoveto{\pgfqpoint{4.069479in}{1.921522in}}%
\pgfpathlineto{\pgfqpoint{4.069479in}{1.921522in}}%
\pgfpathlineto{\pgfqpoint{4.069479in}{1.924471in}}%
\pgfpathlineto{\pgfqpoint{4.074021in}{1.924471in}}%
\pgfpathlineto{\pgfqpoint{4.074021in}{1.921522in}}%
\pgfpathmoveto{\pgfqpoint{4.069479in}{1.924471in}}%
\pgfpathlineto{\pgfqpoint{4.069479in}{1.924471in}}%
\pgfpathlineto{\pgfqpoint{4.069479in}{1.927420in}}%
\pgfpathlineto{\pgfqpoint{4.074021in}{1.927420in}}%
\pgfpathlineto{\pgfqpoint{4.074021in}{1.924471in}}%
\pgfpathmoveto{\pgfqpoint{4.060397in}{1.930370in}}%
\pgfpathlineto{\pgfqpoint{4.060397in}{1.930370in}}%
\pgfpathlineto{\pgfqpoint{4.060397in}{1.933319in}}%
\pgfpathlineto{\pgfqpoint{4.064938in}{1.933319in}}%
\pgfpathlineto{\pgfqpoint{4.064938in}{1.930370in}}%
\pgfpathmoveto{\pgfqpoint{4.055856in}{1.936268in}}%
\pgfpathlineto{\pgfqpoint{4.055856in}{1.936268in}}%
\pgfpathlineto{\pgfqpoint{4.055856in}{1.939218in}}%
\pgfpathlineto{\pgfqpoint{4.060397in}{1.939218in}}%
\pgfpathlineto{\pgfqpoint{4.060397in}{1.936268in}}%
\pgfpathmoveto{\pgfqpoint{4.060397in}{1.933319in}}%
\pgfpathlineto{\pgfqpoint{4.060397in}{1.933319in}}%
\pgfpathlineto{\pgfqpoint{4.060397in}{1.936268in}}%
\pgfpathlineto{\pgfqpoint{4.064938in}{1.936268in}}%
\pgfpathlineto{\pgfqpoint{4.064938in}{1.933319in}}%
\pgfpathmoveto{\pgfqpoint{4.060397in}{1.936268in}}%
\pgfpathlineto{\pgfqpoint{4.060397in}{1.936268in}}%
\pgfpathlineto{\pgfqpoint{4.060397in}{1.939218in}}%
\pgfpathlineto{\pgfqpoint{4.064938in}{1.939218in}}%
\pgfpathlineto{\pgfqpoint{4.064938in}{1.936268in}}%
\pgfpathmoveto{\pgfqpoint{4.064938in}{1.927420in}}%
\pgfpathlineto{\pgfqpoint{4.064938in}{1.927420in}}%
\pgfpathlineto{\pgfqpoint{4.064938in}{1.930370in}}%
\pgfpathlineto{\pgfqpoint{4.069479in}{1.930370in}}%
\pgfpathlineto{\pgfqpoint{4.069479in}{1.927420in}}%
\pgfpathmoveto{\pgfqpoint{4.064938in}{1.930370in}}%
\pgfpathlineto{\pgfqpoint{4.064938in}{1.930370in}}%
\pgfpathlineto{\pgfqpoint{4.064938in}{1.933319in}}%
\pgfpathlineto{\pgfqpoint{4.069479in}{1.933319in}}%
\pgfpathlineto{\pgfqpoint{4.069479in}{1.930370in}}%
\pgfpathmoveto{\pgfqpoint{4.074021in}{1.915623in}}%
\pgfpathlineto{\pgfqpoint{4.074021in}{1.915623in}}%
\pgfpathlineto{\pgfqpoint{4.074021in}{1.918572in}}%
\pgfpathlineto{\pgfqpoint{4.078562in}{1.918572in}}%
\pgfpathlineto{\pgfqpoint{4.078562in}{1.915623in}}%
\pgfpathmoveto{\pgfqpoint{4.074021in}{1.918572in}}%
\pgfpathlineto{\pgfqpoint{4.074021in}{1.918572in}}%
\pgfpathlineto{\pgfqpoint{4.074021in}{1.921522in}}%
\pgfpathlineto{\pgfqpoint{4.078562in}{1.921522in}}%
\pgfpathlineto{\pgfqpoint{4.078562in}{1.918572in}}%
\pgfpathmoveto{\pgfqpoint{4.055856in}{1.939218in}}%
\pgfpathlineto{\pgfqpoint{4.055856in}{1.939218in}}%
\pgfpathlineto{\pgfqpoint{4.055856in}{1.942167in}}%
\pgfpathlineto{\pgfqpoint{4.060397in}{1.942167in}}%
\pgfpathlineto{\pgfqpoint{4.060397in}{1.939218in}}%
\pgfpathmoveto{\pgfqpoint{4.055856in}{1.942167in}}%
\pgfpathlineto{\pgfqpoint{4.055856in}{1.942167in}}%
\pgfpathlineto{\pgfqpoint{4.055856in}{1.945116in}}%
\pgfpathlineto{\pgfqpoint{4.060397in}{1.945116in}}%
\pgfpathlineto{\pgfqpoint{4.060397in}{1.942167in}}%
\pgfpathmoveto{\pgfqpoint{4.033151in}{1.965762in}}%
\pgfpathlineto{\pgfqpoint{4.033151in}{1.965762in}}%
\pgfpathlineto{\pgfqpoint{4.033151in}{1.968711in}}%
\pgfpathlineto{\pgfqpoint{4.037692in}{1.968711in}}%
\pgfpathlineto{\pgfqpoint{4.037692in}{1.965762in}}%
\pgfpathmoveto{\pgfqpoint{4.028610in}{1.971660in}}%
\pgfpathlineto{\pgfqpoint{4.028610in}{1.971660in}}%
\pgfpathlineto{\pgfqpoint{4.028610in}{1.974610in}}%
\pgfpathlineto{\pgfqpoint{4.033151in}{1.974610in}}%
\pgfpathlineto{\pgfqpoint{4.033151in}{1.971660in}}%
\pgfpathmoveto{\pgfqpoint{4.033151in}{1.968711in}}%
\pgfpathlineto{\pgfqpoint{4.033151in}{1.968711in}}%
\pgfpathlineto{\pgfqpoint{4.033151in}{1.971660in}}%
\pgfpathlineto{\pgfqpoint{4.037692in}{1.971660in}}%
\pgfpathlineto{\pgfqpoint{4.037692in}{1.968711in}}%
\pgfpathmoveto{\pgfqpoint{4.033151in}{1.971660in}}%
\pgfpathlineto{\pgfqpoint{4.033151in}{1.971660in}}%
\pgfpathlineto{\pgfqpoint{4.033151in}{1.974610in}}%
\pgfpathlineto{\pgfqpoint{4.037692in}{1.974610in}}%
\pgfpathlineto{\pgfqpoint{4.037692in}{1.971660in}}%
\pgfpathmoveto{\pgfqpoint{4.024069in}{1.977559in}}%
\pgfpathlineto{\pgfqpoint{4.024069in}{1.977559in}}%
\pgfpathlineto{\pgfqpoint{4.024069in}{1.980508in}}%
\pgfpathlineto{\pgfqpoint{4.028610in}{1.980508in}}%
\pgfpathlineto{\pgfqpoint{4.028610in}{1.977559in}}%
\pgfpathmoveto{\pgfqpoint{4.019528in}{1.983458in}}%
\pgfpathlineto{\pgfqpoint{4.019528in}{1.983458in}}%
\pgfpathlineto{\pgfqpoint{4.019528in}{1.986407in}}%
\pgfpathlineto{\pgfqpoint{4.024069in}{1.986407in}}%
\pgfpathlineto{\pgfqpoint{4.024069in}{1.983458in}}%
\pgfpathmoveto{\pgfqpoint{4.024069in}{1.980508in}}%
\pgfpathlineto{\pgfqpoint{4.024069in}{1.980508in}}%
\pgfpathlineto{\pgfqpoint{4.024069in}{1.983458in}}%
\pgfpathlineto{\pgfqpoint{4.028610in}{1.983458in}}%
\pgfpathlineto{\pgfqpoint{4.028610in}{1.980508in}}%
\pgfpathmoveto{\pgfqpoint{4.024069in}{1.983458in}}%
\pgfpathlineto{\pgfqpoint{4.024069in}{1.983458in}}%
\pgfpathlineto{\pgfqpoint{4.024069in}{1.986407in}}%
\pgfpathlineto{\pgfqpoint{4.028610in}{1.986407in}}%
\pgfpathlineto{\pgfqpoint{4.028610in}{1.983458in}}%
\pgfpathmoveto{\pgfqpoint{4.028610in}{1.974610in}}%
\pgfpathlineto{\pgfqpoint{4.028610in}{1.974610in}}%
\pgfpathlineto{\pgfqpoint{4.028610in}{1.977559in}}%
\pgfpathlineto{\pgfqpoint{4.033151in}{1.977559in}}%
\pgfpathlineto{\pgfqpoint{4.033151in}{1.974610in}}%
\pgfpathmoveto{\pgfqpoint{4.028610in}{1.977559in}}%
\pgfpathlineto{\pgfqpoint{4.028610in}{1.977559in}}%
\pgfpathlineto{\pgfqpoint{4.028610in}{1.980508in}}%
\pgfpathlineto{\pgfqpoint{4.033151in}{1.980508in}}%
\pgfpathlineto{\pgfqpoint{4.033151in}{1.977559in}}%
\pgfpathmoveto{\pgfqpoint{4.037692in}{1.962812in}}%
\pgfpathlineto{\pgfqpoint{4.037692in}{1.962812in}}%
\pgfpathlineto{\pgfqpoint{4.037692in}{1.965762in}}%
\pgfpathlineto{\pgfqpoint{4.042233in}{1.965762in}}%
\pgfpathlineto{\pgfqpoint{4.042233in}{1.962812in}}%
\pgfpathmoveto{\pgfqpoint{4.037692in}{1.965762in}}%
\pgfpathlineto{\pgfqpoint{4.037692in}{1.965762in}}%
\pgfpathlineto{\pgfqpoint{4.037692in}{1.968711in}}%
\pgfpathlineto{\pgfqpoint{4.042233in}{1.968711in}}%
\pgfpathlineto{\pgfqpoint{4.042233in}{1.965762in}}%
\pgfpathmoveto{\pgfqpoint{4.019528in}{1.986407in}}%
\pgfpathlineto{\pgfqpoint{4.019528in}{1.986407in}}%
\pgfpathlineto{\pgfqpoint{4.019528in}{1.989356in}}%
\pgfpathlineto{\pgfqpoint{4.024069in}{1.989356in}}%
\pgfpathlineto{\pgfqpoint{4.024069in}{1.986407in}}%
\pgfpathmoveto{\pgfqpoint{4.019528in}{1.989356in}}%
\pgfpathlineto{\pgfqpoint{4.019528in}{1.989356in}}%
\pgfpathlineto{\pgfqpoint{4.019528in}{1.992306in}}%
\pgfpathlineto{\pgfqpoint{4.024069in}{1.992306in}}%
\pgfpathlineto{\pgfqpoint{4.024069in}{1.989356in}}%
\pgfpathmoveto{\pgfqpoint{3.978659in}{2.036544in}}%
\pgfpathlineto{\pgfqpoint{3.978659in}{2.036544in}}%
\pgfpathlineto{\pgfqpoint{3.978659in}{2.039493in}}%
\pgfpathlineto{\pgfqpoint{3.983200in}{2.039493in}}%
\pgfpathlineto{\pgfqpoint{3.983200in}{2.036544in}}%
\pgfpathmoveto{\pgfqpoint{3.974117in}{2.042442in}}%
\pgfpathlineto{\pgfqpoint{3.974117in}{2.042442in}}%
\pgfpathlineto{\pgfqpoint{3.974117in}{2.045391in}}%
\pgfpathlineto{\pgfqpoint{3.978659in}{2.045391in}}%
\pgfpathlineto{\pgfqpoint{3.978659in}{2.042442in}}%
\pgfpathmoveto{\pgfqpoint{3.978659in}{2.039493in}}%
\pgfpathlineto{\pgfqpoint{3.978659in}{2.039493in}}%
\pgfpathlineto{\pgfqpoint{3.978659in}{2.042442in}}%
\pgfpathlineto{\pgfqpoint{3.983200in}{2.042442in}}%
\pgfpathlineto{\pgfqpoint{3.983200in}{2.039493in}}%
\pgfpathmoveto{\pgfqpoint{3.978659in}{2.042442in}}%
\pgfpathlineto{\pgfqpoint{3.978659in}{2.042442in}}%
\pgfpathlineto{\pgfqpoint{3.978659in}{2.045391in}}%
\pgfpathlineto{\pgfqpoint{3.983200in}{2.045391in}}%
\pgfpathlineto{\pgfqpoint{3.983200in}{2.042442in}}%
\pgfpathmoveto{\pgfqpoint{3.969576in}{2.048340in}}%
\pgfpathlineto{\pgfqpoint{3.969576in}{2.048340in}}%
\pgfpathlineto{\pgfqpoint{3.969576in}{2.051289in}}%
\pgfpathlineto{\pgfqpoint{3.974117in}{2.051289in}}%
\pgfpathlineto{\pgfqpoint{3.974117in}{2.048340in}}%
\pgfpathmoveto{\pgfqpoint{3.965035in}{2.054238in}}%
\pgfpathlineto{\pgfqpoint{3.965035in}{2.054238in}}%
\pgfpathlineto{\pgfqpoint{3.965035in}{2.057188in}}%
\pgfpathlineto{\pgfqpoint{3.969576in}{2.057188in}}%
\pgfpathlineto{\pgfqpoint{3.969576in}{2.054238in}}%
\pgfpathmoveto{\pgfqpoint{3.969576in}{2.051289in}}%
\pgfpathlineto{\pgfqpoint{3.969576in}{2.051289in}}%
\pgfpathlineto{\pgfqpoint{3.969576in}{2.054238in}}%
\pgfpathlineto{\pgfqpoint{3.974117in}{2.054238in}}%
\pgfpathlineto{\pgfqpoint{3.974117in}{2.051289in}}%
\pgfpathmoveto{\pgfqpoint{3.969576in}{2.054238in}}%
\pgfpathlineto{\pgfqpoint{3.969576in}{2.054238in}}%
\pgfpathlineto{\pgfqpoint{3.969576in}{2.057188in}}%
\pgfpathlineto{\pgfqpoint{3.974117in}{2.057188in}}%
\pgfpathlineto{\pgfqpoint{3.974117in}{2.054238in}}%
\pgfpathmoveto{\pgfqpoint{3.974117in}{2.045391in}}%
\pgfpathlineto{\pgfqpoint{3.974117in}{2.045391in}}%
\pgfpathlineto{\pgfqpoint{3.974117in}{2.048340in}}%
\pgfpathlineto{\pgfqpoint{3.978659in}{2.048340in}}%
\pgfpathlineto{\pgfqpoint{3.978659in}{2.045391in}}%
\pgfpathmoveto{\pgfqpoint{3.974117in}{2.048340in}}%
\pgfpathlineto{\pgfqpoint{3.974117in}{2.048340in}}%
\pgfpathlineto{\pgfqpoint{3.974117in}{2.051289in}}%
\pgfpathlineto{\pgfqpoint{3.978659in}{2.051289in}}%
\pgfpathlineto{\pgfqpoint{3.978659in}{2.048340in}}%
\pgfpathmoveto{\pgfqpoint{3.996823in}{2.012951in}}%
\pgfpathlineto{\pgfqpoint{3.996823in}{2.012951in}}%
\pgfpathlineto{\pgfqpoint{3.996823in}{2.015900in}}%
\pgfpathlineto{\pgfqpoint{4.001364in}{2.015900in}}%
\pgfpathlineto{\pgfqpoint{4.001364in}{2.012951in}}%
\pgfpathmoveto{\pgfqpoint{3.992282in}{2.018849in}}%
\pgfpathlineto{\pgfqpoint{3.992282in}{2.018849in}}%
\pgfpathlineto{\pgfqpoint{3.992282in}{2.021798in}}%
\pgfpathlineto{\pgfqpoint{3.996823in}{2.021798in}}%
\pgfpathlineto{\pgfqpoint{3.996823in}{2.018849in}}%
\pgfpathmoveto{\pgfqpoint{3.996823in}{2.015900in}}%
\pgfpathlineto{\pgfqpoint{3.996823in}{2.015900in}}%
\pgfpathlineto{\pgfqpoint{3.996823in}{2.018849in}}%
\pgfpathlineto{\pgfqpoint{4.001364in}{2.018849in}}%
\pgfpathlineto{\pgfqpoint{4.001364in}{2.015900in}}%
\pgfpathmoveto{\pgfqpoint{3.996823in}{2.018849in}}%
\pgfpathlineto{\pgfqpoint{3.996823in}{2.018849in}}%
\pgfpathlineto{\pgfqpoint{3.996823in}{2.021798in}}%
\pgfpathlineto{\pgfqpoint{4.001364in}{2.021798in}}%
\pgfpathlineto{\pgfqpoint{4.001364in}{2.018849in}}%
\pgfpathmoveto{\pgfqpoint{3.987741in}{2.024747in}}%
\pgfpathlineto{\pgfqpoint{3.987741in}{2.024747in}}%
\pgfpathlineto{\pgfqpoint{3.987741in}{2.027696in}}%
\pgfpathlineto{\pgfqpoint{3.992282in}{2.027696in}}%
\pgfpathlineto{\pgfqpoint{3.992282in}{2.024747in}}%
\pgfpathmoveto{\pgfqpoint{3.983200in}{2.030646in}}%
\pgfpathlineto{\pgfqpoint{3.983200in}{2.030646in}}%
\pgfpathlineto{\pgfqpoint{3.983200in}{2.033595in}}%
\pgfpathlineto{\pgfqpoint{3.987741in}{2.033595in}}%
\pgfpathlineto{\pgfqpoint{3.987741in}{2.030646in}}%
\pgfpathmoveto{\pgfqpoint{3.987741in}{2.027696in}}%
\pgfpathlineto{\pgfqpoint{3.987741in}{2.027696in}}%
\pgfpathlineto{\pgfqpoint{3.987741in}{2.030646in}}%
\pgfpathlineto{\pgfqpoint{3.992282in}{2.030646in}}%
\pgfpathlineto{\pgfqpoint{3.992282in}{2.027696in}}%
\pgfpathmoveto{\pgfqpoint{3.987741in}{2.030646in}}%
\pgfpathlineto{\pgfqpoint{3.987741in}{2.030646in}}%
\pgfpathlineto{\pgfqpoint{3.987741in}{2.033595in}}%
\pgfpathlineto{\pgfqpoint{3.992282in}{2.033595in}}%
\pgfpathlineto{\pgfqpoint{3.992282in}{2.030646in}}%
\pgfpathmoveto{\pgfqpoint{3.992282in}{2.021798in}}%
\pgfpathlineto{\pgfqpoint{3.992282in}{2.021798in}}%
\pgfpathlineto{\pgfqpoint{3.992282in}{2.024747in}}%
\pgfpathlineto{\pgfqpoint{3.996823in}{2.024747in}}%
\pgfpathlineto{\pgfqpoint{3.996823in}{2.021798in}}%
\pgfpathmoveto{\pgfqpoint{3.992282in}{2.024747in}}%
\pgfpathlineto{\pgfqpoint{3.992282in}{2.024747in}}%
\pgfpathlineto{\pgfqpoint{3.992282in}{2.027696in}}%
\pgfpathlineto{\pgfqpoint{3.996823in}{2.027696in}}%
\pgfpathlineto{\pgfqpoint{3.996823in}{2.024747in}}%
\pgfpathmoveto{\pgfqpoint{4.001364in}{2.010002in}}%
\pgfpathlineto{\pgfqpoint{4.001364in}{2.010002in}}%
\pgfpathlineto{\pgfqpoint{4.001364in}{2.012951in}}%
\pgfpathlineto{\pgfqpoint{4.005905in}{2.012951in}}%
\pgfpathlineto{\pgfqpoint{4.005905in}{2.010002in}}%
\pgfpathmoveto{\pgfqpoint{4.001364in}{2.012951in}}%
\pgfpathlineto{\pgfqpoint{4.001364in}{2.012951in}}%
\pgfpathlineto{\pgfqpoint{4.001364in}{2.015900in}}%
\pgfpathlineto{\pgfqpoint{4.005905in}{2.015900in}}%
\pgfpathlineto{\pgfqpoint{4.005905in}{2.012951in}}%
\pgfpathmoveto{\pgfqpoint{3.983200in}{2.033595in}}%
\pgfpathlineto{\pgfqpoint{3.983200in}{2.033595in}}%
\pgfpathlineto{\pgfqpoint{3.983200in}{2.036544in}}%
\pgfpathlineto{\pgfqpoint{3.987741in}{2.036544in}}%
\pgfpathlineto{\pgfqpoint{3.987741in}{2.033595in}}%
\pgfpathmoveto{\pgfqpoint{3.983200in}{2.036544in}}%
\pgfpathlineto{\pgfqpoint{3.983200in}{2.036544in}}%
\pgfpathlineto{\pgfqpoint{3.983200in}{2.039493in}}%
\pgfpathlineto{\pgfqpoint{3.987741in}{2.039493in}}%
\pgfpathlineto{\pgfqpoint{3.987741in}{2.036544in}}%
\pgfpathmoveto{\pgfqpoint{3.960494in}{2.060137in}}%
\pgfpathlineto{\pgfqpoint{3.960494in}{2.060137in}}%
\pgfpathlineto{\pgfqpoint{3.960494in}{2.063086in}}%
\pgfpathlineto{\pgfqpoint{3.965035in}{2.063086in}}%
\pgfpathlineto{\pgfqpoint{3.965035in}{2.060137in}}%
\pgfpathmoveto{\pgfqpoint{3.955953in}{2.066035in}}%
\pgfpathlineto{\pgfqpoint{3.955953in}{2.066035in}}%
\pgfpathlineto{\pgfqpoint{3.955953in}{2.068984in}}%
\pgfpathlineto{\pgfqpoint{3.960494in}{2.068984in}}%
\pgfpathlineto{\pgfqpoint{3.960494in}{2.066035in}}%
\pgfpathmoveto{\pgfqpoint{3.960494in}{2.063086in}}%
\pgfpathlineto{\pgfqpoint{3.960494in}{2.063086in}}%
\pgfpathlineto{\pgfqpoint{3.960494in}{2.066035in}}%
\pgfpathlineto{\pgfqpoint{3.965035in}{2.066035in}}%
\pgfpathlineto{\pgfqpoint{3.965035in}{2.063086in}}%
\pgfpathmoveto{\pgfqpoint{3.960494in}{2.066035in}}%
\pgfpathlineto{\pgfqpoint{3.960494in}{2.066035in}}%
\pgfpathlineto{\pgfqpoint{3.960494in}{2.068984in}}%
\pgfpathlineto{\pgfqpoint{3.965035in}{2.068984in}}%
\pgfpathlineto{\pgfqpoint{3.965035in}{2.066035in}}%
\pgfpathmoveto{\pgfqpoint{3.951412in}{2.071933in}}%
\pgfpathlineto{\pgfqpoint{3.951412in}{2.071933in}}%
\pgfpathlineto{\pgfqpoint{3.951412in}{2.074882in}}%
\pgfpathlineto{\pgfqpoint{3.955953in}{2.074882in}}%
\pgfpathlineto{\pgfqpoint{3.955953in}{2.071933in}}%
\pgfpathmoveto{\pgfqpoint{3.946871in}{2.077831in}}%
\pgfpathlineto{\pgfqpoint{3.946871in}{2.077831in}}%
\pgfpathlineto{\pgfqpoint{3.946871in}{2.080780in}}%
\pgfpathlineto{\pgfqpoint{3.951412in}{2.080780in}}%
\pgfpathlineto{\pgfqpoint{3.951412in}{2.077831in}}%
\pgfpathmoveto{\pgfqpoint{3.951412in}{2.074882in}}%
\pgfpathlineto{\pgfqpoint{3.951412in}{2.074882in}}%
\pgfpathlineto{\pgfqpoint{3.951412in}{2.077831in}}%
\pgfpathlineto{\pgfqpoint{3.955953in}{2.077831in}}%
\pgfpathlineto{\pgfqpoint{3.955953in}{2.074882in}}%
\pgfpathmoveto{\pgfqpoint{3.951412in}{2.077831in}}%
\pgfpathlineto{\pgfqpoint{3.951412in}{2.077831in}}%
\pgfpathlineto{\pgfqpoint{3.951412in}{2.080780in}}%
\pgfpathlineto{\pgfqpoint{3.955953in}{2.080780in}}%
\pgfpathlineto{\pgfqpoint{3.955953in}{2.077831in}}%
\pgfpathmoveto{\pgfqpoint{3.955953in}{2.068984in}}%
\pgfpathlineto{\pgfqpoint{3.955953in}{2.068984in}}%
\pgfpathlineto{\pgfqpoint{3.955953in}{2.071933in}}%
\pgfpathlineto{\pgfqpoint{3.960494in}{2.071933in}}%
\pgfpathlineto{\pgfqpoint{3.960494in}{2.068984in}}%
\pgfpathmoveto{\pgfqpoint{3.955953in}{2.071933in}}%
\pgfpathlineto{\pgfqpoint{3.955953in}{2.071933in}}%
\pgfpathlineto{\pgfqpoint{3.955953in}{2.074882in}}%
\pgfpathlineto{\pgfqpoint{3.960494in}{2.074882in}}%
\pgfpathlineto{\pgfqpoint{3.960494in}{2.071933in}}%
\pgfpathmoveto{\pgfqpoint{3.965035in}{2.057188in}}%
\pgfpathlineto{\pgfqpoint{3.965035in}{2.057188in}}%
\pgfpathlineto{\pgfqpoint{3.965035in}{2.060137in}}%
\pgfpathlineto{\pgfqpoint{3.969576in}{2.060137in}}%
\pgfpathlineto{\pgfqpoint{3.969576in}{2.057188in}}%
\pgfpathmoveto{\pgfqpoint{3.965035in}{2.060137in}}%
\pgfpathlineto{\pgfqpoint{3.965035in}{2.060137in}}%
\pgfpathlineto{\pgfqpoint{3.965035in}{2.063086in}}%
\pgfpathlineto{\pgfqpoint{3.969576in}{2.063086in}}%
\pgfpathlineto{\pgfqpoint{3.969576in}{2.060137in}}%
\pgfpathmoveto{\pgfqpoint{3.946871in}{2.080780in}}%
\pgfpathlineto{\pgfqpoint{3.946871in}{2.080780in}}%
\pgfpathlineto{\pgfqpoint{3.946871in}{2.083730in}}%
\pgfpathlineto{\pgfqpoint{3.951412in}{2.083730in}}%
\pgfpathlineto{\pgfqpoint{3.951412in}{2.080780in}}%
\pgfpathmoveto{\pgfqpoint{3.946871in}{2.083730in}}%
\pgfpathlineto{\pgfqpoint{3.946871in}{2.083730in}}%
\pgfpathlineto{\pgfqpoint{3.946871in}{2.086679in}}%
\pgfpathlineto{\pgfqpoint{3.951412in}{2.086679in}}%
\pgfpathlineto{\pgfqpoint{3.951412in}{2.083730in}}%
\pgfpathmoveto{\pgfqpoint{4.232961in}{1.706231in}}%
\pgfpathlineto{\pgfqpoint{4.232961in}{1.706231in}}%
\pgfpathlineto{\pgfqpoint{4.232961in}{1.709180in}}%
\pgfpathlineto{\pgfqpoint{4.237503in}{1.709180in}}%
\pgfpathlineto{\pgfqpoint{4.237503in}{1.706231in}}%
\pgfpathmoveto{\pgfqpoint{4.228420in}{1.712130in}}%
\pgfpathlineto{\pgfqpoint{4.228420in}{1.712130in}}%
\pgfpathlineto{\pgfqpoint{4.228420in}{1.715079in}}%
\pgfpathlineto{\pgfqpoint{4.232961in}{1.715079in}}%
\pgfpathlineto{\pgfqpoint{4.232961in}{1.712130in}}%
\pgfpathmoveto{\pgfqpoint{4.232961in}{1.709180in}}%
\pgfpathlineto{\pgfqpoint{4.232961in}{1.709180in}}%
\pgfpathlineto{\pgfqpoint{4.232961in}{1.712130in}}%
\pgfpathlineto{\pgfqpoint{4.237503in}{1.712130in}}%
\pgfpathlineto{\pgfqpoint{4.237503in}{1.709180in}}%
\pgfpathmoveto{\pgfqpoint{4.232961in}{1.712130in}}%
\pgfpathlineto{\pgfqpoint{4.232961in}{1.712130in}}%
\pgfpathlineto{\pgfqpoint{4.232961in}{1.715079in}}%
\pgfpathlineto{\pgfqpoint{4.237503in}{1.715079in}}%
\pgfpathlineto{\pgfqpoint{4.237503in}{1.712130in}}%
\pgfpathmoveto{\pgfqpoint{4.223879in}{1.718028in}}%
\pgfpathlineto{\pgfqpoint{4.223879in}{1.718028in}}%
\pgfpathlineto{\pgfqpoint{4.223879in}{1.720977in}}%
\pgfpathlineto{\pgfqpoint{4.228420in}{1.720977in}}%
\pgfpathlineto{\pgfqpoint{4.228420in}{1.718028in}}%
\pgfpathmoveto{\pgfqpoint{4.219338in}{1.723927in}}%
\pgfpathlineto{\pgfqpoint{4.219338in}{1.723927in}}%
\pgfpathlineto{\pgfqpoint{4.219338in}{1.726876in}}%
\pgfpathlineto{\pgfqpoint{4.223879in}{1.726876in}}%
\pgfpathlineto{\pgfqpoint{4.223879in}{1.723927in}}%
\pgfpathmoveto{\pgfqpoint{4.223879in}{1.720977in}}%
\pgfpathlineto{\pgfqpoint{4.223879in}{1.720977in}}%
\pgfpathlineto{\pgfqpoint{4.223879in}{1.723927in}}%
\pgfpathlineto{\pgfqpoint{4.228420in}{1.723927in}}%
\pgfpathlineto{\pgfqpoint{4.228420in}{1.720977in}}%
\pgfpathmoveto{\pgfqpoint{4.223879in}{1.723927in}}%
\pgfpathlineto{\pgfqpoint{4.223879in}{1.723927in}}%
\pgfpathlineto{\pgfqpoint{4.223879in}{1.726876in}}%
\pgfpathlineto{\pgfqpoint{4.228420in}{1.726876in}}%
\pgfpathlineto{\pgfqpoint{4.228420in}{1.723927in}}%
\pgfpathmoveto{\pgfqpoint{4.228420in}{1.715079in}}%
\pgfpathlineto{\pgfqpoint{4.228420in}{1.715079in}}%
\pgfpathlineto{\pgfqpoint{4.228420in}{1.718028in}}%
\pgfpathlineto{\pgfqpoint{4.232961in}{1.718028in}}%
\pgfpathlineto{\pgfqpoint{4.232961in}{1.715079in}}%
\pgfpathmoveto{\pgfqpoint{4.228420in}{1.718028in}}%
\pgfpathlineto{\pgfqpoint{4.228420in}{1.718028in}}%
\pgfpathlineto{\pgfqpoint{4.228420in}{1.720977in}}%
\pgfpathlineto{\pgfqpoint{4.232961in}{1.720977in}}%
\pgfpathlineto{\pgfqpoint{4.232961in}{1.718028in}}%
\pgfpathmoveto{\pgfqpoint{4.160302in}{1.800605in}}%
\pgfpathlineto{\pgfqpoint{4.160302in}{1.800605in}}%
\pgfpathlineto{\pgfqpoint{4.160302in}{1.803554in}}%
\pgfpathlineto{\pgfqpoint{4.164844in}{1.803554in}}%
\pgfpathlineto{\pgfqpoint{4.164844in}{1.800605in}}%
\pgfpathmoveto{\pgfqpoint{4.155761in}{1.806503in}}%
\pgfpathlineto{\pgfqpoint{4.155761in}{1.806503in}}%
\pgfpathlineto{\pgfqpoint{4.155761in}{1.809453in}}%
\pgfpathlineto{\pgfqpoint{4.160302in}{1.809453in}}%
\pgfpathlineto{\pgfqpoint{4.160302in}{1.806503in}}%
\pgfpathmoveto{\pgfqpoint{4.160302in}{1.803554in}}%
\pgfpathlineto{\pgfqpoint{4.160302in}{1.803554in}}%
\pgfpathlineto{\pgfqpoint{4.160302in}{1.806503in}}%
\pgfpathlineto{\pgfqpoint{4.164844in}{1.806503in}}%
\pgfpathlineto{\pgfqpoint{4.164844in}{1.803554in}}%
\pgfpathmoveto{\pgfqpoint{4.160302in}{1.806503in}}%
\pgfpathlineto{\pgfqpoint{4.160302in}{1.806503in}}%
\pgfpathlineto{\pgfqpoint{4.160302in}{1.809453in}}%
\pgfpathlineto{\pgfqpoint{4.164844in}{1.809453in}}%
\pgfpathlineto{\pgfqpoint{4.164844in}{1.806503in}}%
\pgfpathmoveto{\pgfqpoint{4.151220in}{1.812402in}}%
\pgfpathlineto{\pgfqpoint{4.151220in}{1.812402in}}%
\pgfpathlineto{\pgfqpoint{4.151220in}{1.815351in}}%
\pgfpathlineto{\pgfqpoint{4.155761in}{1.815351in}}%
\pgfpathlineto{\pgfqpoint{4.155761in}{1.812402in}}%
\pgfpathmoveto{\pgfqpoint{4.146679in}{1.818300in}}%
\pgfpathlineto{\pgfqpoint{4.146679in}{1.818300in}}%
\pgfpathlineto{\pgfqpoint{4.146679in}{1.821249in}}%
\pgfpathlineto{\pgfqpoint{4.151220in}{1.821249in}}%
\pgfpathlineto{\pgfqpoint{4.151220in}{1.818300in}}%
\pgfpathmoveto{\pgfqpoint{4.151220in}{1.815351in}}%
\pgfpathlineto{\pgfqpoint{4.151220in}{1.815351in}}%
\pgfpathlineto{\pgfqpoint{4.151220in}{1.818300in}}%
\pgfpathlineto{\pgfqpoint{4.155761in}{1.818300in}}%
\pgfpathlineto{\pgfqpoint{4.155761in}{1.815351in}}%
\pgfpathmoveto{\pgfqpoint{4.151220in}{1.818300in}}%
\pgfpathlineto{\pgfqpoint{4.151220in}{1.818300in}}%
\pgfpathlineto{\pgfqpoint{4.151220in}{1.821249in}}%
\pgfpathlineto{\pgfqpoint{4.155761in}{1.821249in}}%
\pgfpathlineto{\pgfqpoint{4.155761in}{1.818300in}}%
\pgfpathmoveto{\pgfqpoint{4.155761in}{1.809453in}}%
\pgfpathlineto{\pgfqpoint{4.155761in}{1.809453in}}%
\pgfpathlineto{\pgfqpoint{4.155761in}{1.812402in}}%
\pgfpathlineto{\pgfqpoint{4.160302in}{1.812402in}}%
\pgfpathlineto{\pgfqpoint{4.160302in}{1.809453in}}%
\pgfpathmoveto{\pgfqpoint{4.155761in}{1.812402in}}%
\pgfpathlineto{\pgfqpoint{4.155761in}{1.812402in}}%
\pgfpathlineto{\pgfqpoint{4.155761in}{1.815351in}}%
\pgfpathlineto{\pgfqpoint{4.160302in}{1.815351in}}%
\pgfpathlineto{\pgfqpoint{4.160302in}{1.812402in}}%
\pgfpathmoveto{\pgfqpoint{4.196632in}{1.753418in}}%
\pgfpathlineto{\pgfqpoint{4.196632in}{1.753418in}}%
\pgfpathlineto{\pgfqpoint{4.196632in}{1.756368in}}%
\pgfpathlineto{\pgfqpoint{4.201173in}{1.756368in}}%
\pgfpathlineto{\pgfqpoint{4.201173in}{1.753418in}}%
\pgfpathmoveto{\pgfqpoint{4.192091in}{1.759317in}}%
\pgfpathlineto{\pgfqpoint{4.192091in}{1.759317in}}%
\pgfpathlineto{\pgfqpoint{4.192091in}{1.762266in}}%
\pgfpathlineto{\pgfqpoint{4.196632in}{1.762266in}}%
\pgfpathlineto{\pgfqpoint{4.196632in}{1.759317in}}%
\pgfpathmoveto{\pgfqpoint{4.196632in}{1.756368in}}%
\pgfpathlineto{\pgfqpoint{4.196632in}{1.756368in}}%
\pgfpathlineto{\pgfqpoint{4.196632in}{1.759317in}}%
\pgfpathlineto{\pgfqpoint{4.201173in}{1.759317in}}%
\pgfpathlineto{\pgfqpoint{4.201173in}{1.756368in}}%
\pgfpathmoveto{\pgfqpoint{4.196632in}{1.759317in}}%
\pgfpathlineto{\pgfqpoint{4.196632in}{1.759317in}}%
\pgfpathlineto{\pgfqpoint{4.196632in}{1.762266in}}%
\pgfpathlineto{\pgfqpoint{4.201173in}{1.762266in}}%
\pgfpathlineto{\pgfqpoint{4.201173in}{1.759317in}}%
\pgfpathmoveto{\pgfqpoint{4.187550in}{1.765215in}}%
\pgfpathlineto{\pgfqpoint{4.187550in}{1.765215in}}%
\pgfpathlineto{\pgfqpoint{4.187550in}{1.768164in}}%
\pgfpathlineto{\pgfqpoint{4.192091in}{1.768164in}}%
\pgfpathlineto{\pgfqpoint{4.192091in}{1.765215in}}%
\pgfpathmoveto{\pgfqpoint{4.183008in}{1.771113in}}%
\pgfpathlineto{\pgfqpoint{4.183008in}{1.771113in}}%
\pgfpathlineto{\pgfqpoint{4.183008in}{1.774063in}}%
\pgfpathlineto{\pgfqpoint{4.187550in}{1.774063in}}%
\pgfpathlineto{\pgfqpoint{4.187550in}{1.771113in}}%
\pgfpathmoveto{\pgfqpoint{4.187550in}{1.768164in}}%
\pgfpathlineto{\pgfqpoint{4.187550in}{1.768164in}}%
\pgfpathlineto{\pgfqpoint{4.187550in}{1.771113in}}%
\pgfpathlineto{\pgfqpoint{4.192091in}{1.771113in}}%
\pgfpathlineto{\pgfqpoint{4.192091in}{1.768164in}}%
\pgfpathmoveto{\pgfqpoint{4.187550in}{1.771113in}}%
\pgfpathlineto{\pgfqpoint{4.187550in}{1.771113in}}%
\pgfpathlineto{\pgfqpoint{4.187550in}{1.774063in}}%
\pgfpathlineto{\pgfqpoint{4.192091in}{1.774063in}}%
\pgfpathlineto{\pgfqpoint{4.192091in}{1.771113in}}%
\pgfpathmoveto{\pgfqpoint{4.192091in}{1.762266in}}%
\pgfpathlineto{\pgfqpoint{4.192091in}{1.762266in}}%
\pgfpathlineto{\pgfqpoint{4.192091in}{1.765215in}}%
\pgfpathlineto{\pgfqpoint{4.196632in}{1.765215in}}%
\pgfpathlineto{\pgfqpoint{4.196632in}{1.762266in}}%
\pgfpathmoveto{\pgfqpoint{4.192091in}{1.765215in}}%
\pgfpathlineto{\pgfqpoint{4.192091in}{1.765215in}}%
\pgfpathlineto{\pgfqpoint{4.192091in}{1.768164in}}%
\pgfpathlineto{\pgfqpoint{4.196632in}{1.768164in}}%
\pgfpathlineto{\pgfqpoint{4.196632in}{1.765215in}}%
\pgfpathmoveto{\pgfqpoint{4.214797in}{1.729825in}}%
\pgfpathlineto{\pgfqpoint{4.214797in}{1.729825in}}%
\pgfpathlineto{\pgfqpoint{4.214797in}{1.732774in}}%
\pgfpathlineto{\pgfqpoint{4.219338in}{1.732774in}}%
\pgfpathlineto{\pgfqpoint{4.219338in}{1.729825in}}%
\pgfpathmoveto{\pgfqpoint{4.210255in}{1.735723in}}%
\pgfpathlineto{\pgfqpoint{4.210255in}{1.735723in}}%
\pgfpathlineto{\pgfqpoint{4.210255in}{1.738672in}}%
\pgfpathlineto{\pgfqpoint{4.214797in}{1.738672in}}%
\pgfpathlineto{\pgfqpoint{4.214797in}{1.735723in}}%
\pgfpathmoveto{\pgfqpoint{4.214797in}{1.732774in}}%
\pgfpathlineto{\pgfqpoint{4.214797in}{1.732774in}}%
\pgfpathlineto{\pgfqpoint{4.214797in}{1.735723in}}%
\pgfpathlineto{\pgfqpoint{4.219338in}{1.735723in}}%
\pgfpathlineto{\pgfqpoint{4.219338in}{1.732774in}}%
\pgfpathmoveto{\pgfqpoint{4.214797in}{1.735723in}}%
\pgfpathlineto{\pgfqpoint{4.214797in}{1.735723in}}%
\pgfpathlineto{\pgfqpoint{4.214797in}{1.738672in}}%
\pgfpathlineto{\pgfqpoint{4.219338in}{1.738672in}}%
\pgfpathlineto{\pgfqpoint{4.219338in}{1.735723in}}%
\pgfpathmoveto{\pgfqpoint{4.205714in}{1.741622in}}%
\pgfpathlineto{\pgfqpoint{4.205714in}{1.741622in}}%
\pgfpathlineto{\pgfqpoint{4.205714in}{1.744571in}}%
\pgfpathlineto{\pgfqpoint{4.210255in}{1.744571in}}%
\pgfpathlineto{\pgfqpoint{4.210255in}{1.741622in}}%
\pgfpathmoveto{\pgfqpoint{4.201173in}{1.747520in}}%
\pgfpathlineto{\pgfqpoint{4.201173in}{1.747520in}}%
\pgfpathlineto{\pgfqpoint{4.201173in}{1.750469in}}%
\pgfpathlineto{\pgfqpoint{4.205714in}{1.750469in}}%
\pgfpathlineto{\pgfqpoint{4.205714in}{1.747520in}}%
\pgfpathmoveto{\pgfqpoint{4.205714in}{1.744571in}}%
\pgfpathlineto{\pgfqpoint{4.205714in}{1.744571in}}%
\pgfpathlineto{\pgfqpoint{4.205714in}{1.747520in}}%
\pgfpathlineto{\pgfqpoint{4.210255in}{1.747520in}}%
\pgfpathlineto{\pgfqpoint{4.210255in}{1.744571in}}%
\pgfpathmoveto{\pgfqpoint{4.205714in}{1.747520in}}%
\pgfpathlineto{\pgfqpoint{4.205714in}{1.747520in}}%
\pgfpathlineto{\pgfqpoint{4.205714in}{1.750469in}}%
\pgfpathlineto{\pgfqpoint{4.210255in}{1.750469in}}%
\pgfpathlineto{\pgfqpoint{4.210255in}{1.747520in}}%
\pgfpathmoveto{\pgfqpoint{4.210255in}{1.738672in}}%
\pgfpathlineto{\pgfqpoint{4.210255in}{1.738672in}}%
\pgfpathlineto{\pgfqpoint{4.210255in}{1.741622in}}%
\pgfpathlineto{\pgfqpoint{4.214797in}{1.741622in}}%
\pgfpathlineto{\pgfqpoint{4.214797in}{1.738672in}}%
\pgfpathmoveto{\pgfqpoint{4.210255in}{1.741622in}}%
\pgfpathlineto{\pgfqpoint{4.210255in}{1.741622in}}%
\pgfpathlineto{\pgfqpoint{4.210255in}{1.744571in}}%
\pgfpathlineto{\pgfqpoint{4.214797in}{1.744571in}}%
\pgfpathlineto{\pgfqpoint{4.214797in}{1.741622in}}%
\pgfpathmoveto{\pgfqpoint{4.219338in}{1.726876in}}%
\pgfpathlineto{\pgfqpoint{4.219338in}{1.726876in}}%
\pgfpathlineto{\pgfqpoint{4.219338in}{1.729825in}}%
\pgfpathlineto{\pgfqpoint{4.223879in}{1.729825in}}%
\pgfpathlineto{\pgfqpoint{4.223879in}{1.726876in}}%
\pgfpathmoveto{\pgfqpoint{4.219338in}{1.729825in}}%
\pgfpathlineto{\pgfqpoint{4.219338in}{1.729825in}}%
\pgfpathlineto{\pgfqpoint{4.219338in}{1.732774in}}%
\pgfpathlineto{\pgfqpoint{4.223879in}{1.732774in}}%
\pgfpathlineto{\pgfqpoint{4.223879in}{1.729825in}}%
\pgfpathmoveto{\pgfqpoint{4.201173in}{1.750469in}}%
\pgfpathlineto{\pgfqpoint{4.201173in}{1.750469in}}%
\pgfpathlineto{\pgfqpoint{4.201173in}{1.753418in}}%
\pgfpathlineto{\pgfqpoint{4.205714in}{1.753418in}}%
\pgfpathlineto{\pgfqpoint{4.205714in}{1.750469in}}%
\pgfpathmoveto{\pgfqpoint{4.201173in}{1.753418in}}%
\pgfpathlineto{\pgfqpoint{4.201173in}{1.753418in}}%
\pgfpathlineto{\pgfqpoint{4.201173in}{1.756368in}}%
\pgfpathlineto{\pgfqpoint{4.205714in}{1.756368in}}%
\pgfpathlineto{\pgfqpoint{4.205714in}{1.753418in}}%
\pgfpathmoveto{\pgfqpoint{4.178467in}{1.777012in}}%
\pgfpathlineto{\pgfqpoint{4.178467in}{1.777012in}}%
\pgfpathlineto{\pgfqpoint{4.178467in}{1.779961in}}%
\pgfpathlineto{\pgfqpoint{4.183008in}{1.779961in}}%
\pgfpathlineto{\pgfqpoint{4.183008in}{1.777012in}}%
\pgfpathmoveto{\pgfqpoint{4.173926in}{1.782910in}}%
\pgfpathlineto{\pgfqpoint{4.173926in}{1.782910in}}%
\pgfpathlineto{\pgfqpoint{4.173926in}{1.785859in}}%
\pgfpathlineto{\pgfqpoint{4.178467in}{1.785859in}}%
\pgfpathlineto{\pgfqpoint{4.178467in}{1.782910in}}%
\pgfpathmoveto{\pgfqpoint{4.178467in}{1.779961in}}%
\pgfpathlineto{\pgfqpoint{4.178467in}{1.779961in}}%
\pgfpathlineto{\pgfqpoint{4.178467in}{1.782910in}}%
\pgfpathlineto{\pgfqpoint{4.183008in}{1.782910in}}%
\pgfpathlineto{\pgfqpoint{4.183008in}{1.779961in}}%
\pgfpathmoveto{\pgfqpoint{4.178467in}{1.782910in}}%
\pgfpathlineto{\pgfqpoint{4.178467in}{1.782910in}}%
\pgfpathlineto{\pgfqpoint{4.178467in}{1.785859in}}%
\pgfpathlineto{\pgfqpoint{4.183008in}{1.785859in}}%
\pgfpathlineto{\pgfqpoint{4.183008in}{1.782910in}}%
\pgfpathmoveto{\pgfqpoint{4.169385in}{1.788808in}}%
\pgfpathlineto{\pgfqpoint{4.169385in}{1.788808in}}%
\pgfpathlineto{\pgfqpoint{4.169385in}{1.791758in}}%
\pgfpathlineto{\pgfqpoint{4.173926in}{1.791758in}}%
\pgfpathlineto{\pgfqpoint{4.173926in}{1.788808in}}%
\pgfpathmoveto{\pgfqpoint{4.164844in}{1.794707in}}%
\pgfpathlineto{\pgfqpoint{4.164844in}{1.794707in}}%
\pgfpathlineto{\pgfqpoint{4.164844in}{1.797656in}}%
\pgfpathlineto{\pgfqpoint{4.169385in}{1.797656in}}%
\pgfpathlineto{\pgfqpoint{4.169385in}{1.794707in}}%
\pgfpathmoveto{\pgfqpoint{4.169385in}{1.791758in}}%
\pgfpathlineto{\pgfqpoint{4.169385in}{1.791758in}}%
\pgfpathlineto{\pgfqpoint{4.169385in}{1.794707in}}%
\pgfpathlineto{\pgfqpoint{4.173926in}{1.794707in}}%
\pgfpathlineto{\pgfqpoint{4.173926in}{1.791758in}}%
\pgfpathmoveto{\pgfqpoint{4.169385in}{1.794707in}}%
\pgfpathlineto{\pgfqpoint{4.169385in}{1.794707in}}%
\pgfpathlineto{\pgfqpoint{4.169385in}{1.797656in}}%
\pgfpathlineto{\pgfqpoint{4.173926in}{1.797656in}}%
\pgfpathlineto{\pgfqpoint{4.173926in}{1.794707in}}%
\pgfpathmoveto{\pgfqpoint{4.173926in}{1.785859in}}%
\pgfpathlineto{\pgfqpoint{4.173926in}{1.785859in}}%
\pgfpathlineto{\pgfqpoint{4.173926in}{1.788808in}}%
\pgfpathlineto{\pgfqpoint{4.178467in}{1.788808in}}%
\pgfpathlineto{\pgfqpoint{4.178467in}{1.785859in}}%
\pgfpathmoveto{\pgfqpoint{4.173926in}{1.788808in}}%
\pgfpathlineto{\pgfqpoint{4.173926in}{1.788808in}}%
\pgfpathlineto{\pgfqpoint{4.173926in}{1.791758in}}%
\pgfpathlineto{\pgfqpoint{4.178467in}{1.791758in}}%
\pgfpathlineto{\pgfqpoint{4.178467in}{1.788808in}}%
\pgfpathmoveto{\pgfqpoint{4.183008in}{1.774063in}}%
\pgfpathlineto{\pgfqpoint{4.183008in}{1.774063in}}%
\pgfpathlineto{\pgfqpoint{4.183008in}{1.777012in}}%
\pgfpathlineto{\pgfqpoint{4.187550in}{1.777012in}}%
\pgfpathlineto{\pgfqpoint{4.187550in}{1.774063in}}%
\pgfpathmoveto{\pgfqpoint{4.183008in}{1.777012in}}%
\pgfpathlineto{\pgfqpoint{4.183008in}{1.777012in}}%
\pgfpathlineto{\pgfqpoint{4.183008in}{1.779961in}}%
\pgfpathlineto{\pgfqpoint{4.187550in}{1.779961in}}%
\pgfpathlineto{\pgfqpoint{4.187550in}{1.777012in}}%
\pgfpathmoveto{\pgfqpoint{4.164844in}{1.797656in}}%
\pgfpathlineto{\pgfqpoint{4.164844in}{1.797656in}}%
\pgfpathlineto{\pgfqpoint{4.164844in}{1.800605in}}%
\pgfpathlineto{\pgfqpoint{4.169385in}{1.800605in}}%
\pgfpathlineto{\pgfqpoint{4.169385in}{1.797656in}}%
\pgfpathmoveto{\pgfqpoint{4.164844in}{1.800605in}}%
\pgfpathlineto{\pgfqpoint{4.164844in}{1.800605in}}%
\pgfpathlineto{\pgfqpoint{4.164844in}{1.803554in}}%
\pgfpathlineto{\pgfqpoint{4.169385in}{1.803554in}}%
\pgfpathlineto{\pgfqpoint{4.169385in}{1.800605in}}%
\pgfpathmoveto{\pgfqpoint{4.123973in}{1.847792in}}%
\pgfpathlineto{\pgfqpoint{4.123973in}{1.847792in}}%
\pgfpathlineto{\pgfqpoint{4.123973in}{1.850741in}}%
\pgfpathlineto{\pgfqpoint{4.128514in}{1.850741in}}%
\pgfpathlineto{\pgfqpoint{4.128514in}{1.847792in}}%
\pgfpathmoveto{\pgfqpoint{4.119432in}{1.853690in}}%
\pgfpathlineto{\pgfqpoint{4.119432in}{1.853690in}}%
\pgfpathlineto{\pgfqpoint{4.119432in}{1.856639in}}%
\pgfpathlineto{\pgfqpoint{4.123973in}{1.856639in}}%
\pgfpathlineto{\pgfqpoint{4.123973in}{1.853690in}}%
\pgfpathmoveto{\pgfqpoint{4.123973in}{1.850741in}}%
\pgfpathlineto{\pgfqpoint{4.123973in}{1.850741in}}%
\pgfpathlineto{\pgfqpoint{4.123973in}{1.853690in}}%
\pgfpathlineto{\pgfqpoint{4.128514in}{1.853690in}}%
\pgfpathlineto{\pgfqpoint{4.128514in}{1.850741in}}%
\pgfpathmoveto{\pgfqpoint{4.123973in}{1.853690in}}%
\pgfpathlineto{\pgfqpoint{4.123973in}{1.853690in}}%
\pgfpathlineto{\pgfqpoint{4.123973in}{1.856639in}}%
\pgfpathlineto{\pgfqpoint{4.128514in}{1.856639in}}%
\pgfpathlineto{\pgfqpoint{4.128514in}{1.853690in}}%
\pgfpathmoveto{\pgfqpoint{4.114891in}{1.859589in}}%
\pgfpathlineto{\pgfqpoint{4.114891in}{1.859589in}}%
\pgfpathlineto{\pgfqpoint{4.114891in}{1.862538in}}%
\pgfpathlineto{\pgfqpoint{4.119432in}{1.862538in}}%
\pgfpathlineto{\pgfqpoint{4.119432in}{1.859589in}}%
\pgfpathmoveto{\pgfqpoint{4.110349in}{1.865487in}}%
\pgfpathlineto{\pgfqpoint{4.110349in}{1.865487in}}%
\pgfpathlineto{\pgfqpoint{4.110349in}{1.868436in}}%
\pgfpathlineto{\pgfqpoint{4.114891in}{1.868436in}}%
\pgfpathlineto{\pgfqpoint{4.114891in}{1.865487in}}%
\pgfpathmoveto{\pgfqpoint{4.114891in}{1.862538in}}%
\pgfpathlineto{\pgfqpoint{4.114891in}{1.862538in}}%
\pgfpathlineto{\pgfqpoint{4.114891in}{1.865487in}}%
\pgfpathlineto{\pgfqpoint{4.119432in}{1.865487in}}%
\pgfpathlineto{\pgfqpoint{4.119432in}{1.862538in}}%
\pgfpathmoveto{\pgfqpoint{4.114891in}{1.865487in}}%
\pgfpathlineto{\pgfqpoint{4.114891in}{1.865487in}}%
\pgfpathlineto{\pgfqpoint{4.114891in}{1.868436in}}%
\pgfpathlineto{\pgfqpoint{4.119432in}{1.868436in}}%
\pgfpathlineto{\pgfqpoint{4.119432in}{1.865487in}}%
\pgfpathmoveto{\pgfqpoint{4.119432in}{1.856639in}}%
\pgfpathlineto{\pgfqpoint{4.119432in}{1.856639in}}%
\pgfpathlineto{\pgfqpoint{4.119432in}{1.859589in}}%
\pgfpathlineto{\pgfqpoint{4.123973in}{1.859589in}}%
\pgfpathlineto{\pgfqpoint{4.123973in}{1.856639in}}%
\pgfpathmoveto{\pgfqpoint{4.119432in}{1.859589in}}%
\pgfpathlineto{\pgfqpoint{4.119432in}{1.859589in}}%
\pgfpathlineto{\pgfqpoint{4.119432in}{1.862538in}}%
\pgfpathlineto{\pgfqpoint{4.123973in}{1.862538in}}%
\pgfpathlineto{\pgfqpoint{4.123973in}{1.859589in}}%
\pgfpathmoveto{\pgfqpoint{4.142138in}{1.824198in}}%
\pgfpathlineto{\pgfqpoint{4.142138in}{1.824198in}}%
\pgfpathlineto{\pgfqpoint{4.142138in}{1.827148in}}%
\pgfpathlineto{\pgfqpoint{4.146679in}{1.827148in}}%
\pgfpathlineto{\pgfqpoint{4.146679in}{1.824198in}}%
\pgfpathmoveto{\pgfqpoint{4.137597in}{1.830097in}}%
\pgfpathlineto{\pgfqpoint{4.137597in}{1.830097in}}%
\pgfpathlineto{\pgfqpoint{4.137597in}{1.833046in}}%
\pgfpathlineto{\pgfqpoint{4.142138in}{1.833046in}}%
\pgfpathlineto{\pgfqpoint{4.142138in}{1.830097in}}%
\pgfpathmoveto{\pgfqpoint{4.142138in}{1.827148in}}%
\pgfpathlineto{\pgfqpoint{4.142138in}{1.827148in}}%
\pgfpathlineto{\pgfqpoint{4.142138in}{1.830097in}}%
\pgfpathlineto{\pgfqpoint{4.146679in}{1.830097in}}%
\pgfpathlineto{\pgfqpoint{4.146679in}{1.827148in}}%
\pgfpathmoveto{\pgfqpoint{4.142138in}{1.830097in}}%
\pgfpathlineto{\pgfqpoint{4.142138in}{1.830097in}}%
\pgfpathlineto{\pgfqpoint{4.142138in}{1.833046in}}%
\pgfpathlineto{\pgfqpoint{4.146679in}{1.833046in}}%
\pgfpathlineto{\pgfqpoint{4.146679in}{1.830097in}}%
\pgfpathmoveto{\pgfqpoint{4.133055in}{1.835995in}}%
\pgfpathlineto{\pgfqpoint{4.133055in}{1.835995in}}%
\pgfpathlineto{\pgfqpoint{4.133055in}{1.838944in}}%
\pgfpathlineto{\pgfqpoint{4.137597in}{1.838944in}}%
\pgfpathlineto{\pgfqpoint{4.137597in}{1.835995in}}%
\pgfpathmoveto{\pgfqpoint{4.128514in}{1.841894in}}%
\pgfpathlineto{\pgfqpoint{4.128514in}{1.841894in}}%
\pgfpathlineto{\pgfqpoint{4.128514in}{1.844843in}}%
\pgfpathlineto{\pgfqpoint{4.133055in}{1.844843in}}%
\pgfpathlineto{\pgfqpoint{4.133055in}{1.841894in}}%
\pgfpathmoveto{\pgfqpoint{4.133055in}{1.838944in}}%
\pgfpathlineto{\pgfqpoint{4.133055in}{1.838944in}}%
\pgfpathlineto{\pgfqpoint{4.133055in}{1.841894in}}%
\pgfpathlineto{\pgfqpoint{4.137597in}{1.841894in}}%
\pgfpathlineto{\pgfqpoint{4.137597in}{1.838944in}}%
\pgfpathmoveto{\pgfqpoint{4.133055in}{1.841894in}}%
\pgfpathlineto{\pgfqpoint{4.133055in}{1.841894in}}%
\pgfpathlineto{\pgfqpoint{4.133055in}{1.844843in}}%
\pgfpathlineto{\pgfqpoint{4.137597in}{1.844843in}}%
\pgfpathlineto{\pgfqpoint{4.137597in}{1.841894in}}%
\pgfpathmoveto{\pgfqpoint{4.137597in}{1.833046in}}%
\pgfpathlineto{\pgfqpoint{4.137597in}{1.833046in}}%
\pgfpathlineto{\pgfqpoint{4.137597in}{1.835995in}}%
\pgfpathlineto{\pgfqpoint{4.142138in}{1.835995in}}%
\pgfpathlineto{\pgfqpoint{4.142138in}{1.833046in}}%
\pgfpathmoveto{\pgfqpoint{4.137597in}{1.835995in}}%
\pgfpathlineto{\pgfqpoint{4.137597in}{1.835995in}}%
\pgfpathlineto{\pgfqpoint{4.137597in}{1.838944in}}%
\pgfpathlineto{\pgfqpoint{4.142138in}{1.838944in}}%
\pgfpathlineto{\pgfqpoint{4.142138in}{1.835995in}}%
\pgfpathmoveto{\pgfqpoint{4.146679in}{1.821249in}}%
\pgfpathlineto{\pgfqpoint{4.146679in}{1.821249in}}%
\pgfpathlineto{\pgfqpoint{4.146679in}{1.824198in}}%
\pgfpathlineto{\pgfqpoint{4.151220in}{1.824198in}}%
\pgfpathlineto{\pgfqpoint{4.151220in}{1.821249in}}%
\pgfpathmoveto{\pgfqpoint{4.146679in}{1.824198in}}%
\pgfpathlineto{\pgfqpoint{4.146679in}{1.824198in}}%
\pgfpathlineto{\pgfqpoint{4.146679in}{1.827148in}}%
\pgfpathlineto{\pgfqpoint{4.151220in}{1.827148in}}%
\pgfpathlineto{\pgfqpoint{4.151220in}{1.824198in}}%
\pgfpathmoveto{\pgfqpoint{4.128514in}{1.844843in}}%
\pgfpathlineto{\pgfqpoint{4.128514in}{1.844843in}}%
\pgfpathlineto{\pgfqpoint{4.128514in}{1.847792in}}%
\pgfpathlineto{\pgfqpoint{4.133055in}{1.847792in}}%
\pgfpathlineto{\pgfqpoint{4.133055in}{1.844843in}}%
\pgfpathmoveto{\pgfqpoint{4.128514in}{1.847792in}}%
\pgfpathlineto{\pgfqpoint{4.128514in}{1.847792in}}%
\pgfpathlineto{\pgfqpoint{4.128514in}{1.850741in}}%
\pgfpathlineto{\pgfqpoint{4.133055in}{1.850741in}}%
\pgfpathlineto{\pgfqpoint{4.133055in}{1.847792in}}%
\pgfpathmoveto{\pgfqpoint{4.105808in}{1.871385in}}%
\pgfpathlineto{\pgfqpoint{4.105808in}{1.871385in}}%
\pgfpathlineto{\pgfqpoint{4.105808in}{1.874335in}}%
\pgfpathlineto{\pgfqpoint{4.110349in}{1.874335in}}%
\pgfpathlineto{\pgfqpoint{4.110349in}{1.871385in}}%
\pgfpathmoveto{\pgfqpoint{4.101267in}{1.877284in}}%
\pgfpathlineto{\pgfqpoint{4.101267in}{1.877284in}}%
\pgfpathlineto{\pgfqpoint{4.101267in}{1.880233in}}%
\pgfpathlineto{\pgfqpoint{4.105808in}{1.880233in}}%
\pgfpathlineto{\pgfqpoint{4.105808in}{1.877284in}}%
\pgfpathmoveto{\pgfqpoint{4.105808in}{1.874335in}}%
\pgfpathlineto{\pgfqpoint{4.105808in}{1.874335in}}%
\pgfpathlineto{\pgfqpoint{4.105808in}{1.877284in}}%
\pgfpathlineto{\pgfqpoint{4.110349in}{1.877284in}}%
\pgfpathlineto{\pgfqpoint{4.110349in}{1.874335in}}%
\pgfpathmoveto{\pgfqpoint{4.105808in}{1.877284in}}%
\pgfpathlineto{\pgfqpoint{4.105808in}{1.877284in}}%
\pgfpathlineto{\pgfqpoint{4.105808in}{1.880233in}}%
\pgfpathlineto{\pgfqpoint{4.110349in}{1.880233in}}%
\pgfpathlineto{\pgfqpoint{4.110349in}{1.877284in}}%
\pgfpathmoveto{\pgfqpoint{4.096726in}{1.883182in}}%
\pgfpathlineto{\pgfqpoint{4.096726in}{1.883182in}}%
\pgfpathlineto{\pgfqpoint{4.096726in}{1.886131in}}%
\pgfpathlineto{\pgfqpoint{4.101267in}{1.886131in}}%
\pgfpathlineto{\pgfqpoint{4.101267in}{1.883182in}}%
\pgfpathmoveto{\pgfqpoint{4.092185in}{1.889080in}}%
\pgfpathlineto{\pgfqpoint{4.092185in}{1.889080in}}%
\pgfpathlineto{\pgfqpoint{4.092185in}{1.892030in}}%
\pgfpathlineto{\pgfqpoint{4.096726in}{1.892030in}}%
\pgfpathlineto{\pgfqpoint{4.096726in}{1.889080in}}%
\pgfpathmoveto{\pgfqpoint{4.096726in}{1.886131in}}%
\pgfpathlineto{\pgfqpoint{4.096726in}{1.886131in}}%
\pgfpathlineto{\pgfqpoint{4.096726in}{1.889080in}}%
\pgfpathlineto{\pgfqpoint{4.101267in}{1.889080in}}%
\pgfpathlineto{\pgfqpoint{4.101267in}{1.886131in}}%
\pgfpathmoveto{\pgfqpoint{4.096726in}{1.889080in}}%
\pgfpathlineto{\pgfqpoint{4.096726in}{1.889080in}}%
\pgfpathlineto{\pgfqpoint{4.096726in}{1.892030in}}%
\pgfpathlineto{\pgfqpoint{4.101267in}{1.892030in}}%
\pgfpathlineto{\pgfqpoint{4.101267in}{1.889080in}}%
\pgfpathmoveto{\pgfqpoint{4.101267in}{1.880233in}}%
\pgfpathlineto{\pgfqpoint{4.101267in}{1.880233in}}%
\pgfpathlineto{\pgfqpoint{4.101267in}{1.883182in}}%
\pgfpathlineto{\pgfqpoint{4.105808in}{1.883182in}}%
\pgfpathlineto{\pgfqpoint{4.105808in}{1.880233in}}%
\pgfpathmoveto{\pgfqpoint{4.101267in}{1.883182in}}%
\pgfpathlineto{\pgfqpoint{4.101267in}{1.883182in}}%
\pgfpathlineto{\pgfqpoint{4.101267in}{1.886131in}}%
\pgfpathlineto{\pgfqpoint{4.105808in}{1.886131in}}%
\pgfpathlineto{\pgfqpoint{4.105808in}{1.883182in}}%
\pgfpathmoveto{\pgfqpoint{4.110349in}{1.868436in}}%
\pgfpathlineto{\pgfqpoint{4.110349in}{1.868436in}}%
\pgfpathlineto{\pgfqpoint{4.110349in}{1.871385in}}%
\pgfpathlineto{\pgfqpoint{4.114891in}{1.871385in}}%
\pgfpathlineto{\pgfqpoint{4.114891in}{1.868436in}}%
\pgfpathmoveto{\pgfqpoint{4.110349in}{1.871385in}}%
\pgfpathlineto{\pgfqpoint{4.110349in}{1.871385in}}%
\pgfpathlineto{\pgfqpoint{4.110349in}{1.874335in}}%
\pgfpathlineto{\pgfqpoint{4.114891in}{1.874335in}}%
\pgfpathlineto{\pgfqpoint{4.114891in}{1.871385in}}%
\pgfpathmoveto{\pgfqpoint{4.092185in}{1.892030in}}%
\pgfpathlineto{\pgfqpoint{4.092185in}{1.892030in}}%
\pgfpathlineto{\pgfqpoint{4.092185in}{1.894979in}}%
\pgfpathlineto{\pgfqpoint{4.096726in}{1.894979in}}%
\pgfpathlineto{\pgfqpoint{4.096726in}{1.892030in}}%
\pgfpathmoveto{\pgfqpoint{4.092185in}{1.894979in}}%
\pgfpathlineto{\pgfqpoint{4.092185in}{1.894979in}}%
\pgfpathlineto{\pgfqpoint{4.092185in}{1.897928in}}%
\pgfpathlineto{\pgfqpoint{4.096726in}{1.897928in}}%
\pgfpathlineto{\pgfqpoint{4.096726in}{1.894979in}}%
\pgfpathmoveto{\pgfqpoint{4.378270in}{1.517479in}}%
\pgfpathlineto{\pgfqpoint{4.378270in}{1.517479in}}%
\pgfpathlineto{\pgfqpoint{4.378270in}{1.520428in}}%
\pgfpathlineto{\pgfqpoint{4.382811in}{1.520428in}}%
\pgfpathlineto{\pgfqpoint{4.382811in}{1.517479in}}%
\pgfpathmoveto{\pgfqpoint{4.373729in}{1.523377in}}%
\pgfpathlineto{\pgfqpoint{4.373729in}{1.523377in}}%
\pgfpathlineto{\pgfqpoint{4.373729in}{1.526326in}}%
\pgfpathlineto{\pgfqpoint{4.378270in}{1.526326in}}%
\pgfpathlineto{\pgfqpoint{4.378270in}{1.523377in}}%
\pgfpathmoveto{\pgfqpoint{4.378270in}{1.520428in}}%
\pgfpathlineto{\pgfqpoint{4.378270in}{1.520428in}}%
\pgfpathlineto{\pgfqpoint{4.378270in}{1.523377in}}%
\pgfpathlineto{\pgfqpoint{4.382811in}{1.523377in}}%
\pgfpathlineto{\pgfqpoint{4.382811in}{1.520428in}}%
\pgfpathmoveto{\pgfqpoint{4.378270in}{1.523377in}}%
\pgfpathlineto{\pgfqpoint{4.378270in}{1.523377in}}%
\pgfpathlineto{\pgfqpoint{4.378270in}{1.526326in}}%
\pgfpathlineto{\pgfqpoint{4.382811in}{1.526326in}}%
\pgfpathlineto{\pgfqpoint{4.382811in}{1.523377in}}%
\pgfpathmoveto{\pgfqpoint{4.369188in}{1.529275in}}%
\pgfpathlineto{\pgfqpoint{4.369188in}{1.529275in}}%
\pgfpathlineto{\pgfqpoint{4.369188in}{1.532225in}}%
\pgfpathlineto{\pgfqpoint{4.373729in}{1.532225in}}%
\pgfpathlineto{\pgfqpoint{4.373729in}{1.529275in}}%
\pgfpathmoveto{\pgfqpoint{4.364647in}{1.535174in}}%
\pgfpathlineto{\pgfqpoint{4.364647in}{1.535174in}}%
\pgfpathlineto{\pgfqpoint{4.364647in}{1.538123in}}%
\pgfpathlineto{\pgfqpoint{4.369188in}{1.538123in}}%
\pgfpathlineto{\pgfqpoint{4.369188in}{1.535174in}}%
\pgfpathmoveto{\pgfqpoint{4.369188in}{1.532225in}}%
\pgfpathlineto{\pgfqpoint{4.369188in}{1.532225in}}%
\pgfpathlineto{\pgfqpoint{4.369188in}{1.535174in}}%
\pgfpathlineto{\pgfqpoint{4.373729in}{1.535174in}}%
\pgfpathlineto{\pgfqpoint{4.373729in}{1.532225in}}%
\pgfpathmoveto{\pgfqpoint{4.369188in}{1.535174in}}%
\pgfpathlineto{\pgfqpoint{4.369188in}{1.535174in}}%
\pgfpathlineto{\pgfqpoint{4.369188in}{1.538123in}}%
\pgfpathlineto{\pgfqpoint{4.373729in}{1.538123in}}%
\pgfpathlineto{\pgfqpoint{4.373729in}{1.535174in}}%
\pgfpathmoveto{\pgfqpoint{4.373729in}{1.526326in}}%
\pgfpathlineto{\pgfqpoint{4.373729in}{1.526326in}}%
\pgfpathlineto{\pgfqpoint{4.373729in}{1.529275in}}%
\pgfpathlineto{\pgfqpoint{4.378270in}{1.529275in}}%
\pgfpathlineto{\pgfqpoint{4.378270in}{1.526326in}}%
\pgfpathmoveto{\pgfqpoint{4.373729in}{1.529275in}}%
\pgfpathlineto{\pgfqpoint{4.373729in}{1.529275in}}%
\pgfpathlineto{\pgfqpoint{4.373729in}{1.532225in}}%
\pgfpathlineto{\pgfqpoint{4.378270in}{1.532225in}}%
\pgfpathlineto{\pgfqpoint{4.378270in}{1.529275in}}%
\pgfpathmoveto{\pgfqpoint{4.305616in}{1.611855in}}%
\pgfpathlineto{\pgfqpoint{4.305616in}{1.611855in}}%
\pgfpathlineto{\pgfqpoint{4.305616in}{1.614804in}}%
\pgfpathlineto{\pgfqpoint{4.310157in}{1.614804in}}%
\pgfpathlineto{\pgfqpoint{4.310157in}{1.611855in}}%
\pgfpathmoveto{\pgfqpoint{4.301075in}{1.617753in}}%
\pgfpathlineto{\pgfqpoint{4.301075in}{1.617753in}}%
\pgfpathlineto{\pgfqpoint{4.301075in}{1.620703in}}%
\pgfpathlineto{\pgfqpoint{4.305616in}{1.620703in}}%
\pgfpathlineto{\pgfqpoint{4.305616in}{1.617753in}}%
\pgfpathmoveto{\pgfqpoint{4.305616in}{1.614804in}}%
\pgfpathlineto{\pgfqpoint{4.305616in}{1.614804in}}%
\pgfpathlineto{\pgfqpoint{4.305616in}{1.617753in}}%
\pgfpathlineto{\pgfqpoint{4.310157in}{1.617753in}}%
\pgfpathlineto{\pgfqpoint{4.310157in}{1.614804in}}%
\pgfpathmoveto{\pgfqpoint{4.305616in}{1.617753in}}%
\pgfpathlineto{\pgfqpoint{4.305616in}{1.617753in}}%
\pgfpathlineto{\pgfqpoint{4.305616in}{1.620703in}}%
\pgfpathlineto{\pgfqpoint{4.310157in}{1.620703in}}%
\pgfpathlineto{\pgfqpoint{4.310157in}{1.617753in}}%
\pgfpathmoveto{\pgfqpoint{4.296534in}{1.623652in}}%
\pgfpathlineto{\pgfqpoint{4.296534in}{1.623652in}}%
\pgfpathlineto{\pgfqpoint{4.296534in}{1.626601in}}%
\pgfpathlineto{\pgfqpoint{4.301075in}{1.626601in}}%
\pgfpathlineto{\pgfqpoint{4.301075in}{1.623652in}}%
\pgfpathmoveto{\pgfqpoint{4.291993in}{1.629551in}}%
\pgfpathlineto{\pgfqpoint{4.291993in}{1.629551in}}%
\pgfpathlineto{\pgfqpoint{4.291993in}{1.632500in}}%
\pgfpathlineto{\pgfqpoint{4.296534in}{1.632500in}}%
\pgfpathlineto{\pgfqpoint{4.296534in}{1.629551in}}%
\pgfpathmoveto{\pgfqpoint{4.296534in}{1.626601in}}%
\pgfpathlineto{\pgfqpoint{4.296534in}{1.626601in}}%
\pgfpathlineto{\pgfqpoint{4.296534in}{1.629551in}}%
\pgfpathlineto{\pgfqpoint{4.301075in}{1.629551in}}%
\pgfpathlineto{\pgfqpoint{4.301075in}{1.626601in}}%
\pgfpathmoveto{\pgfqpoint{4.296534in}{1.629551in}}%
\pgfpathlineto{\pgfqpoint{4.296534in}{1.629551in}}%
\pgfpathlineto{\pgfqpoint{4.296534in}{1.632500in}}%
\pgfpathlineto{\pgfqpoint{4.301075in}{1.632500in}}%
\pgfpathlineto{\pgfqpoint{4.301075in}{1.629551in}}%
\pgfpathmoveto{\pgfqpoint{4.301075in}{1.620703in}}%
\pgfpathlineto{\pgfqpoint{4.301075in}{1.620703in}}%
\pgfpathlineto{\pgfqpoint{4.301075in}{1.623652in}}%
\pgfpathlineto{\pgfqpoint{4.305616in}{1.623652in}}%
\pgfpathlineto{\pgfqpoint{4.305616in}{1.620703in}}%
\pgfpathmoveto{\pgfqpoint{4.301075in}{1.623652in}}%
\pgfpathlineto{\pgfqpoint{4.301075in}{1.623652in}}%
\pgfpathlineto{\pgfqpoint{4.301075in}{1.626601in}}%
\pgfpathlineto{\pgfqpoint{4.305616in}{1.626601in}}%
\pgfpathlineto{\pgfqpoint{4.305616in}{1.623652in}}%
\pgfpathmoveto{\pgfqpoint{4.341943in}{1.564666in}}%
\pgfpathlineto{\pgfqpoint{4.341943in}{1.564666in}}%
\pgfpathlineto{\pgfqpoint{4.341943in}{1.567616in}}%
\pgfpathlineto{\pgfqpoint{4.346484in}{1.567616in}}%
\pgfpathlineto{\pgfqpoint{4.346484in}{1.564666in}}%
\pgfpathmoveto{\pgfqpoint{4.337402in}{1.570565in}}%
\pgfpathlineto{\pgfqpoint{4.337402in}{1.570565in}}%
\pgfpathlineto{\pgfqpoint{4.337402in}{1.573514in}}%
\pgfpathlineto{\pgfqpoint{4.341943in}{1.573514in}}%
\pgfpathlineto{\pgfqpoint{4.341943in}{1.570565in}}%
\pgfpathmoveto{\pgfqpoint{4.341943in}{1.567616in}}%
\pgfpathlineto{\pgfqpoint{4.341943in}{1.567616in}}%
\pgfpathlineto{\pgfqpoint{4.341943in}{1.570565in}}%
\pgfpathlineto{\pgfqpoint{4.346484in}{1.570565in}}%
\pgfpathlineto{\pgfqpoint{4.346484in}{1.567616in}}%
\pgfpathmoveto{\pgfqpoint{4.341943in}{1.570565in}}%
\pgfpathlineto{\pgfqpoint{4.341943in}{1.570565in}}%
\pgfpathlineto{\pgfqpoint{4.341943in}{1.573514in}}%
\pgfpathlineto{\pgfqpoint{4.346484in}{1.573514in}}%
\pgfpathlineto{\pgfqpoint{4.346484in}{1.570565in}}%
\pgfpathmoveto{\pgfqpoint{4.332861in}{1.576464in}}%
\pgfpathlineto{\pgfqpoint{4.332861in}{1.576464in}}%
\pgfpathlineto{\pgfqpoint{4.332861in}{1.579413in}}%
\pgfpathlineto{\pgfqpoint{4.337402in}{1.579413in}}%
\pgfpathlineto{\pgfqpoint{4.337402in}{1.576464in}}%
\pgfpathmoveto{\pgfqpoint{4.328320in}{1.582362in}}%
\pgfpathlineto{\pgfqpoint{4.328320in}{1.582362in}}%
\pgfpathlineto{\pgfqpoint{4.328320in}{1.585311in}}%
\pgfpathlineto{\pgfqpoint{4.332861in}{1.585311in}}%
\pgfpathlineto{\pgfqpoint{4.332861in}{1.582362in}}%
\pgfpathmoveto{\pgfqpoint{4.332861in}{1.579413in}}%
\pgfpathlineto{\pgfqpoint{4.332861in}{1.579413in}}%
\pgfpathlineto{\pgfqpoint{4.332861in}{1.582362in}}%
\pgfpathlineto{\pgfqpoint{4.337402in}{1.582362in}}%
\pgfpathlineto{\pgfqpoint{4.337402in}{1.579413in}}%
\pgfpathmoveto{\pgfqpoint{4.332861in}{1.582362in}}%
\pgfpathlineto{\pgfqpoint{4.332861in}{1.582362in}}%
\pgfpathlineto{\pgfqpoint{4.332861in}{1.585311in}}%
\pgfpathlineto{\pgfqpoint{4.337402in}{1.585311in}}%
\pgfpathlineto{\pgfqpoint{4.337402in}{1.582362in}}%
\pgfpathmoveto{\pgfqpoint{4.337402in}{1.573514in}}%
\pgfpathlineto{\pgfqpoint{4.337402in}{1.573514in}}%
\pgfpathlineto{\pgfqpoint{4.337402in}{1.576464in}}%
\pgfpathlineto{\pgfqpoint{4.341943in}{1.576464in}}%
\pgfpathlineto{\pgfqpoint{4.341943in}{1.573514in}}%
\pgfpathmoveto{\pgfqpoint{4.337402in}{1.576464in}}%
\pgfpathlineto{\pgfqpoint{4.337402in}{1.576464in}}%
\pgfpathlineto{\pgfqpoint{4.337402in}{1.579413in}}%
\pgfpathlineto{\pgfqpoint{4.341943in}{1.579413in}}%
\pgfpathlineto{\pgfqpoint{4.341943in}{1.576464in}}%
\pgfpathmoveto{\pgfqpoint{4.360106in}{1.541072in}}%
\pgfpathlineto{\pgfqpoint{4.360106in}{1.541072in}}%
\pgfpathlineto{\pgfqpoint{4.360106in}{1.544021in}}%
\pgfpathlineto{\pgfqpoint{4.364647in}{1.544021in}}%
\pgfpathlineto{\pgfqpoint{4.364647in}{1.541072in}}%
\pgfpathmoveto{\pgfqpoint{4.355566in}{1.546971in}}%
\pgfpathlineto{\pgfqpoint{4.355566in}{1.546971in}}%
\pgfpathlineto{\pgfqpoint{4.355566in}{1.549920in}}%
\pgfpathlineto{\pgfqpoint{4.360106in}{1.549920in}}%
\pgfpathlineto{\pgfqpoint{4.360106in}{1.546971in}}%
\pgfpathmoveto{\pgfqpoint{4.360106in}{1.544021in}}%
\pgfpathlineto{\pgfqpoint{4.360106in}{1.544021in}}%
\pgfpathlineto{\pgfqpoint{4.360106in}{1.546971in}}%
\pgfpathlineto{\pgfqpoint{4.364647in}{1.546971in}}%
\pgfpathlineto{\pgfqpoint{4.364647in}{1.544021in}}%
\pgfpathmoveto{\pgfqpoint{4.360106in}{1.546971in}}%
\pgfpathlineto{\pgfqpoint{4.360106in}{1.546971in}}%
\pgfpathlineto{\pgfqpoint{4.360106in}{1.549920in}}%
\pgfpathlineto{\pgfqpoint{4.364647in}{1.549920in}}%
\pgfpathlineto{\pgfqpoint{4.364647in}{1.546971in}}%
\pgfpathmoveto{\pgfqpoint{4.351025in}{1.552869in}}%
\pgfpathlineto{\pgfqpoint{4.351025in}{1.552869in}}%
\pgfpathlineto{\pgfqpoint{4.351025in}{1.555819in}}%
\pgfpathlineto{\pgfqpoint{4.355566in}{1.555819in}}%
\pgfpathlineto{\pgfqpoint{4.355566in}{1.552869in}}%
\pgfpathmoveto{\pgfqpoint{4.346484in}{1.558768in}}%
\pgfpathlineto{\pgfqpoint{4.346484in}{1.558768in}}%
\pgfpathlineto{\pgfqpoint{4.346484in}{1.561717in}}%
\pgfpathlineto{\pgfqpoint{4.351025in}{1.561717in}}%
\pgfpathlineto{\pgfqpoint{4.351025in}{1.558768in}}%
\pgfpathmoveto{\pgfqpoint{4.351025in}{1.555819in}}%
\pgfpathlineto{\pgfqpoint{4.351025in}{1.555819in}}%
\pgfpathlineto{\pgfqpoint{4.351025in}{1.558768in}}%
\pgfpathlineto{\pgfqpoint{4.355566in}{1.558768in}}%
\pgfpathlineto{\pgfqpoint{4.355566in}{1.555819in}}%
\pgfpathmoveto{\pgfqpoint{4.351025in}{1.558768in}}%
\pgfpathlineto{\pgfqpoint{4.351025in}{1.558768in}}%
\pgfpathlineto{\pgfqpoint{4.351025in}{1.561717in}}%
\pgfpathlineto{\pgfqpoint{4.355566in}{1.561717in}}%
\pgfpathlineto{\pgfqpoint{4.355566in}{1.558768in}}%
\pgfpathmoveto{\pgfqpoint{4.355566in}{1.549920in}}%
\pgfpathlineto{\pgfqpoint{4.355566in}{1.549920in}}%
\pgfpathlineto{\pgfqpoint{4.355566in}{1.552869in}}%
\pgfpathlineto{\pgfqpoint{4.360106in}{1.552869in}}%
\pgfpathlineto{\pgfqpoint{4.360106in}{1.549920in}}%
\pgfpathmoveto{\pgfqpoint{4.355566in}{1.552869in}}%
\pgfpathlineto{\pgfqpoint{4.355566in}{1.552869in}}%
\pgfpathlineto{\pgfqpoint{4.355566in}{1.555819in}}%
\pgfpathlineto{\pgfqpoint{4.360106in}{1.555819in}}%
\pgfpathlineto{\pgfqpoint{4.360106in}{1.552869in}}%
\pgfpathmoveto{\pgfqpoint{4.364647in}{1.538123in}}%
\pgfpathlineto{\pgfqpoint{4.364647in}{1.538123in}}%
\pgfpathlineto{\pgfqpoint{4.364647in}{1.541072in}}%
\pgfpathlineto{\pgfqpoint{4.369188in}{1.541072in}}%
\pgfpathlineto{\pgfqpoint{4.369188in}{1.538123in}}%
\pgfpathmoveto{\pgfqpoint{4.364647in}{1.541072in}}%
\pgfpathlineto{\pgfqpoint{4.364647in}{1.541072in}}%
\pgfpathlineto{\pgfqpoint{4.364647in}{1.544021in}}%
\pgfpathlineto{\pgfqpoint{4.369188in}{1.544021in}}%
\pgfpathlineto{\pgfqpoint{4.369188in}{1.541072in}}%
\pgfpathmoveto{\pgfqpoint{4.346484in}{1.561717in}}%
\pgfpathlineto{\pgfqpoint{4.346484in}{1.561717in}}%
\pgfpathlineto{\pgfqpoint{4.346484in}{1.564666in}}%
\pgfpathlineto{\pgfqpoint{4.351025in}{1.564666in}}%
\pgfpathlineto{\pgfqpoint{4.351025in}{1.561717in}}%
\pgfpathmoveto{\pgfqpoint{4.346484in}{1.564666in}}%
\pgfpathlineto{\pgfqpoint{4.346484in}{1.564666in}}%
\pgfpathlineto{\pgfqpoint{4.346484in}{1.567616in}}%
\pgfpathlineto{\pgfqpoint{4.351025in}{1.567616in}}%
\pgfpathlineto{\pgfqpoint{4.351025in}{1.564666in}}%
\pgfpathmoveto{\pgfqpoint{4.323779in}{1.588261in}}%
\pgfpathlineto{\pgfqpoint{4.323779in}{1.588261in}}%
\pgfpathlineto{\pgfqpoint{4.323779in}{1.591210in}}%
\pgfpathlineto{\pgfqpoint{4.328320in}{1.591210in}}%
\pgfpathlineto{\pgfqpoint{4.328320in}{1.588261in}}%
\pgfpathmoveto{\pgfqpoint{4.319238in}{1.594159in}}%
\pgfpathlineto{\pgfqpoint{4.319238in}{1.594159in}}%
\pgfpathlineto{\pgfqpoint{4.319238in}{1.597108in}}%
\pgfpathlineto{\pgfqpoint{4.323779in}{1.597108in}}%
\pgfpathlineto{\pgfqpoint{4.323779in}{1.594159in}}%
\pgfpathmoveto{\pgfqpoint{4.323779in}{1.591210in}}%
\pgfpathlineto{\pgfqpoint{4.323779in}{1.591210in}}%
\pgfpathlineto{\pgfqpoint{4.323779in}{1.594159in}}%
\pgfpathlineto{\pgfqpoint{4.328320in}{1.594159in}}%
\pgfpathlineto{\pgfqpoint{4.328320in}{1.591210in}}%
\pgfpathmoveto{\pgfqpoint{4.323779in}{1.594159in}}%
\pgfpathlineto{\pgfqpoint{4.323779in}{1.594159in}}%
\pgfpathlineto{\pgfqpoint{4.323779in}{1.597108in}}%
\pgfpathlineto{\pgfqpoint{4.328320in}{1.597108in}}%
\pgfpathlineto{\pgfqpoint{4.328320in}{1.594159in}}%
\pgfpathmoveto{\pgfqpoint{4.314698in}{1.600058in}}%
\pgfpathlineto{\pgfqpoint{4.314698in}{1.600058in}}%
\pgfpathlineto{\pgfqpoint{4.314698in}{1.603007in}}%
\pgfpathlineto{\pgfqpoint{4.319238in}{1.603007in}}%
\pgfpathlineto{\pgfqpoint{4.319238in}{1.600058in}}%
\pgfpathmoveto{\pgfqpoint{4.310157in}{1.605956in}}%
\pgfpathlineto{\pgfqpoint{4.310157in}{1.605956in}}%
\pgfpathlineto{\pgfqpoint{4.310157in}{1.608906in}}%
\pgfpathlineto{\pgfqpoint{4.314698in}{1.608906in}}%
\pgfpathlineto{\pgfqpoint{4.314698in}{1.605956in}}%
\pgfpathmoveto{\pgfqpoint{4.314698in}{1.603007in}}%
\pgfpathlineto{\pgfqpoint{4.314698in}{1.603007in}}%
\pgfpathlineto{\pgfqpoint{4.314698in}{1.605956in}}%
\pgfpathlineto{\pgfqpoint{4.319238in}{1.605956in}}%
\pgfpathlineto{\pgfqpoint{4.319238in}{1.603007in}}%
\pgfpathmoveto{\pgfqpoint{4.314698in}{1.605956in}}%
\pgfpathlineto{\pgfqpoint{4.314698in}{1.605956in}}%
\pgfpathlineto{\pgfqpoint{4.314698in}{1.608906in}}%
\pgfpathlineto{\pgfqpoint{4.319238in}{1.608906in}}%
\pgfpathlineto{\pgfqpoint{4.319238in}{1.605956in}}%
\pgfpathmoveto{\pgfqpoint{4.319238in}{1.597108in}}%
\pgfpathlineto{\pgfqpoint{4.319238in}{1.597108in}}%
\pgfpathlineto{\pgfqpoint{4.319238in}{1.600058in}}%
\pgfpathlineto{\pgfqpoint{4.323779in}{1.600058in}}%
\pgfpathlineto{\pgfqpoint{4.323779in}{1.597108in}}%
\pgfpathmoveto{\pgfqpoint{4.319238in}{1.600058in}}%
\pgfpathlineto{\pgfqpoint{4.319238in}{1.600058in}}%
\pgfpathlineto{\pgfqpoint{4.319238in}{1.603007in}}%
\pgfpathlineto{\pgfqpoint{4.323779in}{1.603007in}}%
\pgfpathlineto{\pgfqpoint{4.323779in}{1.600058in}}%
\pgfpathmoveto{\pgfqpoint{4.328320in}{1.585311in}}%
\pgfpathlineto{\pgfqpoint{4.328320in}{1.585311in}}%
\pgfpathlineto{\pgfqpoint{4.328320in}{1.588261in}}%
\pgfpathlineto{\pgfqpoint{4.332861in}{1.588261in}}%
\pgfpathlineto{\pgfqpoint{4.332861in}{1.585311in}}%
\pgfpathmoveto{\pgfqpoint{4.328320in}{1.588261in}}%
\pgfpathlineto{\pgfqpoint{4.328320in}{1.588261in}}%
\pgfpathlineto{\pgfqpoint{4.328320in}{1.591210in}}%
\pgfpathlineto{\pgfqpoint{4.332861in}{1.591210in}}%
\pgfpathlineto{\pgfqpoint{4.332861in}{1.588261in}}%
\pgfpathmoveto{\pgfqpoint{4.310157in}{1.608906in}}%
\pgfpathlineto{\pgfqpoint{4.310157in}{1.608906in}}%
\pgfpathlineto{\pgfqpoint{4.310157in}{1.611855in}}%
\pgfpathlineto{\pgfqpoint{4.314698in}{1.611855in}}%
\pgfpathlineto{\pgfqpoint{4.314698in}{1.608906in}}%
\pgfpathmoveto{\pgfqpoint{4.310157in}{1.611855in}}%
\pgfpathlineto{\pgfqpoint{4.310157in}{1.611855in}}%
\pgfpathlineto{\pgfqpoint{4.310157in}{1.614804in}}%
\pgfpathlineto{\pgfqpoint{4.314698in}{1.614804in}}%
\pgfpathlineto{\pgfqpoint{4.314698in}{1.611855in}}%
\pgfpathmoveto{\pgfqpoint{4.269289in}{1.659043in}}%
\pgfpathlineto{\pgfqpoint{4.269289in}{1.659043in}}%
\pgfpathlineto{\pgfqpoint{4.269289in}{1.661992in}}%
\pgfpathlineto{\pgfqpoint{4.273830in}{1.661992in}}%
\pgfpathlineto{\pgfqpoint{4.273830in}{1.659043in}}%
\pgfpathmoveto{\pgfqpoint{4.264748in}{1.664942in}}%
\pgfpathlineto{\pgfqpoint{4.264748in}{1.664942in}}%
\pgfpathlineto{\pgfqpoint{4.264748in}{1.667891in}}%
\pgfpathlineto{\pgfqpoint{4.269289in}{1.667891in}}%
\pgfpathlineto{\pgfqpoint{4.269289in}{1.664942in}}%
\pgfpathmoveto{\pgfqpoint{4.269289in}{1.661992in}}%
\pgfpathlineto{\pgfqpoint{4.269289in}{1.661992in}}%
\pgfpathlineto{\pgfqpoint{4.269289in}{1.664942in}}%
\pgfpathlineto{\pgfqpoint{4.273830in}{1.664942in}}%
\pgfpathlineto{\pgfqpoint{4.273830in}{1.661992in}}%
\pgfpathmoveto{\pgfqpoint{4.269289in}{1.664942in}}%
\pgfpathlineto{\pgfqpoint{4.269289in}{1.664942in}}%
\pgfpathlineto{\pgfqpoint{4.269289in}{1.667891in}}%
\pgfpathlineto{\pgfqpoint{4.273830in}{1.667891in}}%
\pgfpathlineto{\pgfqpoint{4.273830in}{1.664942in}}%
\pgfpathmoveto{\pgfqpoint{4.260207in}{1.670840in}}%
\pgfpathlineto{\pgfqpoint{4.260207in}{1.670840in}}%
\pgfpathlineto{\pgfqpoint{4.260207in}{1.673789in}}%
\pgfpathlineto{\pgfqpoint{4.264748in}{1.673789in}}%
\pgfpathlineto{\pgfqpoint{4.264748in}{1.670840in}}%
\pgfpathmoveto{\pgfqpoint{4.255666in}{1.676739in}}%
\pgfpathlineto{\pgfqpoint{4.255666in}{1.676739in}}%
\pgfpathlineto{\pgfqpoint{4.255666in}{1.679688in}}%
\pgfpathlineto{\pgfqpoint{4.260207in}{1.679688in}}%
\pgfpathlineto{\pgfqpoint{4.260207in}{1.676739in}}%
\pgfpathmoveto{\pgfqpoint{4.260207in}{1.673789in}}%
\pgfpathlineto{\pgfqpoint{4.260207in}{1.673789in}}%
\pgfpathlineto{\pgfqpoint{4.260207in}{1.676739in}}%
\pgfpathlineto{\pgfqpoint{4.264748in}{1.676739in}}%
\pgfpathlineto{\pgfqpoint{4.264748in}{1.673789in}}%
\pgfpathmoveto{\pgfqpoint{4.260207in}{1.676739in}}%
\pgfpathlineto{\pgfqpoint{4.260207in}{1.676739in}}%
\pgfpathlineto{\pgfqpoint{4.260207in}{1.679688in}}%
\pgfpathlineto{\pgfqpoint{4.264748in}{1.679688in}}%
\pgfpathlineto{\pgfqpoint{4.264748in}{1.676739in}}%
\pgfpathmoveto{\pgfqpoint{4.264748in}{1.667891in}}%
\pgfpathlineto{\pgfqpoint{4.264748in}{1.667891in}}%
\pgfpathlineto{\pgfqpoint{4.264748in}{1.670840in}}%
\pgfpathlineto{\pgfqpoint{4.269289in}{1.670840in}}%
\pgfpathlineto{\pgfqpoint{4.269289in}{1.667891in}}%
\pgfpathmoveto{\pgfqpoint{4.264748in}{1.670840in}}%
\pgfpathlineto{\pgfqpoint{4.264748in}{1.670840in}}%
\pgfpathlineto{\pgfqpoint{4.264748in}{1.673789in}}%
\pgfpathlineto{\pgfqpoint{4.269289in}{1.673789in}}%
\pgfpathlineto{\pgfqpoint{4.269289in}{1.670840in}}%
\pgfpathmoveto{\pgfqpoint{4.287452in}{1.635449in}}%
\pgfpathlineto{\pgfqpoint{4.287452in}{1.635449in}}%
\pgfpathlineto{\pgfqpoint{4.287452in}{1.638398in}}%
\pgfpathlineto{\pgfqpoint{4.291993in}{1.638398in}}%
\pgfpathlineto{\pgfqpoint{4.291993in}{1.635449in}}%
\pgfpathmoveto{\pgfqpoint{4.282911in}{1.641348in}}%
\pgfpathlineto{\pgfqpoint{4.282911in}{1.641348in}}%
\pgfpathlineto{\pgfqpoint{4.282911in}{1.644297in}}%
\pgfpathlineto{\pgfqpoint{4.287452in}{1.644297in}}%
\pgfpathlineto{\pgfqpoint{4.287452in}{1.641348in}}%
\pgfpathmoveto{\pgfqpoint{4.287452in}{1.638398in}}%
\pgfpathlineto{\pgfqpoint{4.287452in}{1.638398in}}%
\pgfpathlineto{\pgfqpoint{4.287452in}{1.641348in}}%
\pgfpathlineto{\pgfqpoint{4.291993in}{1.641348in}}%
\pgfpathlineto{\pgfqpoint{4.291993in}{1.638398in}}%
\pgfpathmoveto{\pgfqpoint{4.287452in}{1.641348in}}%
\pgfpathlineto{\pgfqpoint{4.287452in}{1.641348in}}%
\pgfpathlineto{\pgfqpoint{4.287452in}{1.644297in}}%
\pgfpathlineto{\pgfqpoint{4.291993in}{1.644297in}}%
\pgfpathlineto{\pgfqpoint{4.291993in}{1.641348in}}%
\pgfpathmoveto{\pgfqpoint{4.278371in}{1.647246in}}%
\pgfpathlineto{\pgfqpoint{4.278371in}{1.647246in}}%
\pgfpathlineto{\pgfqpoint{4.278371in}{1.650195in}}%
\pgfpathlineto{\pgfqpoint{4.282911in}{1.650195in}}%
\pgfpathlineto{\pgfqpoint{4.282911in}{1.647246in}}%
\pgfpathmoveto{\pgfqpoint{4.273830in}{1.653145in}}%
\pgfpathlineto{\pgfqpoint{4.273830in}{1.653145in}}%
\pgfpathlineto{\pgfqpoint{4.273830in}{1.656094in}}%
\pgfpathlineto{\pgfqpoint{4.278371in}{1.656094in}}%
\pgfpathlineto{\pgfqpoint{4.278371in}{1.653145in}}%
\pgfpathmoveto{\pgfqpoint{4.278371in}{1.650195in}}%
\pgfpathlineto{\pgfqpoint{4.278371in}{1.650195in}}%
\pgfpathlineto{\pgfqpoint{4.278371in}{1.653145in}}%
\pgfpathlineto{\pgfqpoint{4.282911in}{1.653145in}}%
\pgfpathlineto{\pgfqpoint{4.282911in}{1.650195in}}%
\pgfpathmoveto{\pgfqpoint{4.278371in}{1.653145in}}%
\pgfpathlineto{\pgfqpoint{4.278371in}{1.653145in}}%
\pgfpathlineto{\pgfqpoint{4.278371in}{1.656094in}}%
\pgfpathlineto{\pgfqpoint{4.282911in}{1.656094in}}%
\pgfpathlineto{\pgfqpoint{4.282911in}{1.653145in}}%
\pgfpathmoveto{\pgfqpoint{4.282911in}{1.644297in}}%
\pgfpathlineto{\pgfqpoint{4.282911in}{1.644297in}}%
\pgfpathlineto{\pgfqpoint{4.282911in}{1.647246in}}%
\pgfpathlineto{\pgfqpoint{4.287452in}{1.647246in}}%
\pgfpathlineto{\pgfqpoint{4.287452in}{1.644297in}}%
\pgfpathmoveto{\pgfqpoint{4.282911in}{1.647246in}}%
\pgfpathlineto{\pgfqpoint{4.282911in}{1.647246in}}%
\pgfpathlineto{\pgfqpoint{4.282911in}{1.650195in}}%
\pgfpathlineto{\pgfqpoint{4.287452in}{1.650195in}}%
\pgfpathlineto{\pgfqpoint{4.287452in}{1.647246in}}%
\pgfpathmoveto{\pgfqpoint{4.291993in}{1.632500in}}%
\pgfpathlineto{\pgfqpoint{4.291993in}{1.632500in}}%
\pgfpathlineto{\pgfqpoint{4.291993in}{1.635449in}}%
\pgfpathlineto{\pgfqpoint{4.296534in}{1.635449in}}%
\pgfpathlineto{\pgfqpoint{4.296534in}{1.632500in}}%
\pgfpathmoveto{\pgfqpoint{4.291993in}{1.635449in}}%
\pgfpathlineto{\pgfqpoint{4.291993in}{1.635449in}}%
\pgfpathlineto{\pgfqpoint{4.291993in}{1.638398in}}%
\pgfpathlineto{\pgfqpoint{4.296534in}{1.638398in}}%
\pgfpathlineto{\pgfqpoint{4.296534in}{1.635449in}}%
\pgfpathmoveto{\pgfqpoint{4.273830in}{1.656094in}}%
\pgfpathlineto{\pgfqpoint{4.273830in}{1.656094in}}%
\pgfpathlineto{\pgfqpoint{4.273830in}{1.659043in}}%
\pgfpathlineto{\pgfqpoint{4.278371in}{1.659043in}}%
\pgfpathlineto{\pgfqpoint{4.278371in}{1.656094in}}%
\pgfpathmoveto{\pgfqpoint{4.273830in}{1.659043in}}%
\pgfpathlineto{\pgfqpoint{4.273830in}{1.659043in}}%
\pgfpathlineto{\pgfqpoint{4.273830in}{1.661992in}}%
\pgfpathlineto{\pgfqpoint{4.278371in}{1.661992in}}%
\pgfpathlineto{\pgfqpoint{4.278371in}{1.659043in}}%
\pgfpathmoveto{\pgfqpoint{4.251125in}{1.682637in}}%
\pgfpathlineto{\pgfqpoint{4.251125in}{1.682637in}}%
\pgfpathlineto{\pgfqpoint{4.251125in}{1.685586in}}%
\pgfpathlineto{\pgfqpoint{4.255666in}{1.685586in}}%
\pgfpathlineto{\pgfqpoint{4.255666in}{1.682637in}}%
\pgfpathmoveto{\pgfqpoint{4.246584in}{1.688536in}}%
\pgfpathlineto{\pgfqpoint{4.246584in}{1.688536in}}%
\pgfpathlineto{\pgfqpoint{4.246584in}{1.691485in}}%
\pgfpathlineto{\pgfqpoint{4.251125in}{1.691485in}}%
\pgfpathlineto{\pgfqpoint{4.251125in}{1.688536in}}%
\pgfpathmoveto{\pgfqpoint{4.251125in}{1.685586in}}%
\pgfpathlineto{\pgfqpoint{4.251125in}{1.685586in}}%
\pgfpathlineto{\pgfqpoint{4.251125in}{1.688536in}}%
\pgfpathlineto{\pgfqpoint{4.255666in}{1.688536in}}%
\pgfpathlineto{\pgfqpoint{4.255666in}{1.685586in}}%
\pgfpathmoveto{\pgfqpoint{4.251125in}{1.688536in}}%
\pgfpathlineto{\pgfqpoint{4.251125in}{1.688536in}}%
\pgfpathlineto{\pgfqpoint{4.251125in}{1.691485in}}%
\pgfpathlineto{\pgfqpoint{4.255666in}{1.691485in}}%
\pgfpathlineto{\pgfqpoint{4.255666in}{1.688536in}}%
\pgfpathmoveto{\pgfqpoint{4.242043in}{1.694434in}}%
\pgfpathlineto{\pgfqpoint{4.242043in}{1.694434in}}%
\pgfpathlineto{\pgfqpoint{4.242043in}{1.697383in}}%
\pgfpathlineto{\pgfqpoint{4.246584in}{1.697383in}}%
\pgfpathlineto{\pgfqpoint{4.246584in}{1.694434in}}%
\pgfpathmoveto{\pgfqpoint{4.237503in}{1.700333in}}%
\pgfpathlineto{\pgfqpoint{4.237503in}{1.700333in}}%
\pgfpathlineto{\pgfqpoint{4.237503in}{1.703282in}}%
\pgfpathlineto{\pgfqpoint{4.242043in}{1.703282in}}%
\pgfpathlineto{\pgfqpoint{4.242043in}{1.700333in}}%
\pgfpathmoveto{\pgfqpoint{4.242043in}{1.697383in}}%
\pgfpathlineto{\pgfqpoint{4.242043in}{1.697383in}}%
\pgfpathlineto{\pgfqpoint{4.242043in}{1.700333in}}%
\pgfpathlineto{\pgfqpoint{4.246584in}{1.700333in}}%
\pgfpathlineto{\pgfqpoint{4.246584in}{1.697383in}}%
\pgfpathmoveto{\pgfqpoint{4.242043in}{1.700333in}}%
\pgfpathlineto{\pgfqpoint{4.242043in}{1.700333in}}%
\pgfpathlineto{\pgfqpoint{4.242043in}{1.703282in}}%
\pgfpathlineto{\pgfqpoint{4.246584in}{1.703282in}}%
\pgfpathlineto{\pgfqpoint{4.246584in}{1.700333in}}%
\pgfpathmoveto{\pgfqpoint{4.246584in}{1.691485in}}%
\pgfpathlineto{\pgfqpoint{4.246584in}{1.691485in}}%
\pgfpathlineto{\pgfqpoint{4.246584in}{1.694434in}}%
\pgfpathlineto{\pgfqpoint{4.251125in}{1.694434in}}%
\pgfpathlineto{\pgfqpoint{4.251125in}{1.691485in}}%
\pgfpathmoveto{\pgfqpoint{4.246584in}{1.694434in}}%
\pgfpathlineto{\pgfqpoint{4.246584in}{1.694434in}}%
\pgfpathlineto{\pgfqpoint{4.246584in}{1.697383in}}%
\pgfpathlineto{\pgfqpoint{4.251125in}{1.697383in}}%
\pgfpathlineto{\pgfqpoint{4.251125in}{1.694434in}}%
\pgfpathmoveto{\pgfqpoint{4.255666in}{1.679688in}}%
\pgfpathlineto{\pgfqpoint{4.255666in}{1.679688in}}%
\pgfpathlineto{\pgfqpoint{4.255666in}{1.682637in}}%
\pgfpathlineto{\pgfqpoint{4.260207in}{1.682637in}}%
\pgfpathlineto{\pgfqpoint{4.260207in}{1.679688in}}%
\pgfpathmoveto{\pgfqpoint{4.255666in}{1.682637in}}%
\pgfpathlineto{\pgfqpoint{4.255666in}{1.682637in}}%
\pgfpathlineto{\pgfqpoint{4.255666in}{1.685586in}}%
\pgfpathlineto{\pgfqpoint{4.260207in}{1.685586in}}%
\pgfpathlineto{\pgfqpoint{4.260207in}{1.682637in}}%
\pgfpathmoveto{\pgfqpoint{4.237503in}{1.703282in}}%
\pgfpathlineto{\pgfqpoint{4.237503in}{1.703282in}}%
\pgfpathlineto{\pgfqpoint{4.237503in}{1.706231in}}%
\pgfpathlineto{\pgfqpoint{4.242043in}{1.706231in}}%
\pgfpathlineto{\pgfqpoint{4.242043in}{1.703282in}}%
\pgfpathmoveto{\pgfqpoint{4.237503in}{1.706231in}}%
\pgfpathlineto{\pgfqpoint{4.237503in}{1.706231in}}%
\pgfpathlineto{\pgfqpoint{4.237503in}{1.709180in}}%
\pgfpathlineto{\pgfqpoint{4.242043in}{1.709180in}}%
\pgfpathlineto{\pgfqpoint{4.242043in}{1.706231in}}%
\pgfpathmoveto{\pgfqpoint{4.523583in}{1.328729in}}%
\pgfpathlineto{\pgfqpoint{4.523583in}{1.328729in}}%
\pgfpathlineto{\pgfqpoint{4.523583in}{1.331678in}}%
\pgfpathlineto{\pgfqpoint{4.528124in}{1.331678in}}%
\pgfpathlineto{\pgfqpoint{4.528124in}{1.328729in}}%
\pgfpathmoveto{\pgfqpoint{4.519042in}{1.334627in}}%
\pgfpathlineto{\pgfqpoint{4.519042in}{1.334627in}}%
\pgfpathlineto{\pgfqpoint{4.519042in}{1.337576in}}%
\pgfpathlineto{\pgfqpoint{4.523583in}{1.337576in}}%
\pgfpathlineto{\pgfqpoint{4.523583in}{1.334627in}}%
\pgfpathmoveto{\pgfqpoint{4.523583in}{1.331678in}}%
\pgfpathlineto{\pgfqpoint{4.523583in}{1.331678in}}%
\pgfpathlineto{\pgfqpoint{4.523583in}{1.334627in}}%
\pgfpathlineto{\pgfqpoint{4.528124in}{1.334627in}}%
\pgfpathlineto{\pgfqpoint{4.528124in}{1.331678in}}%
\pgfpathmoveto{\pgfqpoint{4.523583in}{1.334627in}}%
\pgfpathlineto{\pgfqpoint{4.523583in}{1.334627in}}%
\pgfpathlineto{\pgfqpoint{4.523583in}{1.337576in}}%
\pgfpathlineto{\pgfqpoint{4.528124in}{1.337576in}}%
\pgfpathlineto{\pgfqpoint{4.528124in}{1.334627in}}%
\pgfpathmoveto{\pgfqpoint{4.514501in}{1.340525in}}%
\pgfpathlineto{\pgfqpoint{4.514501in}{1.340525in}}%
\pgfpathlineto{\pgfqpoint{4.514501in}{1.343474in}}%
\pgfpathlineto{\pgfqpoint{4.519042in}{1.343474in}}%
\pgfpathlineto{\pgfqpoint{4.519042in}{1.340525in}}%
\pgfpathmoveto{\pgfqpoint{4.509960in}{1.346423in}}%
\pgfpathlineto{\pgfqpoint{4.509960in}{1.346423in}}%
\pgfpathlineto{\pgfqpoint{4.509960in}{1.349372in}}%
\pgfpathlineto{\pgfqpoint{4.514501in}{1.349372in}}%
\pgfpathlineto{\pgfqpoint{4.514501in}{1.346423in}}%
\pgfpathmoveto{\pgfqpoint{4.514501in}{1.343474in}}%
\pgfpathlineto{\pgfqpoint{4.514501in}{1.343474in}}%
\pgfpathlineto{\pgfqpoint{4.514501in}{1.346423in}}%
\pgfpathlineto{\pgfqpoint{4.519042in}{1.346423in}}%
\pgfpathlineto{\pgfqpoint{4.519042in}{1.343474in}}%
\pgfpathmoveto{\pgfqpoint{4.514501in}{1.346423in}}%
\pgfpathlineto{\pgfqpoint{4.514501in}{1.346423in}}%
\pgfpathlineto{\pgfqpoint{4.514501in}{1.349372in}}%
\pgfpathlineto{\pgfqpoint{4.519042in}{1.349372in}}%
\pgfpathlineto{\pgfqpoint{4.519042in}{1.346423in}}%
\pgfpathmoveto{\pgfqpoint{4.519042in}{1.337576in}}%
\pgfpathlineto{\pgfqpoint{4.519042in}{1.337576in}}%
\pgfpathlineto{\pgfqpoint{4.519042in}{1.340525in}}%
\pgfpathlineto{\pgfqpoint{4.523583in}{1.340525in}}%
\pgfpathlineto{\pgfqpoint{4.523583in}{1.337576in}}%
\pgfpathmoveto{\pgfqpoint{4.519042in}{1.340525in}}%
\pgfpathlineto{\pgfqpoint{4.519042in}{1.340525in}}%
\pgfpathlineto{\pgfqpoint{4.519042in}{1.343474in}}%
\pgfpathlineto{\pgfqpoint{4.523583in}{1.343474in}}%
\pgfpathlineto{\pgfqpoint{4.523583in}{1.340525in}}%
\pgfpathmoveto{\pgfqpoint{4.450926in}{1.423104in}}%
\pgfpathlineto{\pgfqpoint{4.450926in}{1.423104in}}%
\pgfpathlineto{\pgfqpoint{4.450926in}{1.426053in}}%
\pgfpathlineto{\pgfqpoint{4.455467in}{1.426053in}}%
\pgfpathlineto{\pgfqpoint{4.455467in}{1.423104in}}%
\pgfpathmoveto{\pgfqpoint{4.446385in}{1.429003in}}%
\pgfpathlineto{\pgfqpoint{4.446385in}{1.429003in}}%
\pgfpathlineto{\pgfqpoint{4.446385in}{1.431952in}}%
\pgfpathlineto{\pgfqpoint{4.450926in}{1.431952in}}%
\pgfpathlineto{\pgfqpoint{4.450926in}{1.429003in}}%
\pgfpathmoveto{\pgfqpoint{4.450926in}{1.426053in}}%
\pgfpathlineto{\pgfqpoint{4.450926in}{1.426053in}}%
\pgfpathlineto{\pgfqpoint{4.450926in}{1.429003in}}%
\pgfpathlineto{\pgfqpoint{4.455467in}{1.429003in}}%
\pgfpathlineto{\pgfqpoint{4.455467in}{1.426053in}}%
\pgfpathmoveto{\pgfqpoint{4.450926in}{1.429003in}}%
\pgfpathlineto{\pgfqpoint{4.450926in}{1.429003in}}%
\pgfpathlineto{\pgfqpoint{4.450926in}{1.431952in}}%
\pgfpathlineto{\pgfqpoint{4.455467in}{1.431952in}}%
\pgfpathlineto{\pgfqpoint{4.455467in}{1.429003in}}%
\pgfpathmoveto{\pgfqpoint{4.441844in}{1.434901in}}%
\pgfpathlineto{\pgfqpoint{4.441844in}{1.434901in}}%
\pgfpathlineto{\pgfqpoint{4.441844in}{1.437850in}}%
\pgfpathlineto{\pgfqpoint{4.446385in}{1.437850in}}%
\pgfpathlineto{\pgfqpoint{4.446385in}{1.434901in}}%
\pgfpathmoveto{\pgfqpoint{4.437303in}{1.440800in}}%
\pgfpathlineto{\pgfqpoint{4.437303in}{1.440800in}}%
\pgfpathlineto{\pgfqpoint{4.437303in}{1.443749in}}%
\pgfpathlineto{\pgfqpoint{4.441844in}{1.443749in}}%
\pgfpathlineto{\pgfqpoint{4.441844in}{1.440800in}}%
\pgfpathmoveto{\pgfqpoint{4.441844in}{1.437850in}}%
\pgfpathlineto{\pgfqpoint{4.441844in}{1.437850in}}%
\pgfpathlineto{\pgfqpoint{4.441844in}{1.440800in}}%
\pgfpathlineto{\pgfqpoint{4.446385in}{1.440800in}}%
\pgfpathlineto{\pgfqpoint{4.446385in}{1.437850in}}%
\pgfpathmoveto{\pgfqpoint{4.441844in}{1.440800in}}%
\pgfpathlineto{\pgfqpoint{4.441844in}{1.440800in}}%
\pgfpathlineto{\pgfqpoint{4.441844in}{1.443749in}}%
\pgfpathlineto{\pgfqpoint{4.446385in}{1.443749in}}%
\pgfpathlineto{\pgfqpoint{4.446385in}{1.440800in}}%
\pgfpathmoveto{\pgfqpoint{4.446385in}{1.431952in}}%
\pgfpathlineto{\pgfqpoint{4.446385in}{1.431952in}}%
\pgfpathlineto{\pgfqpoint{4.446385in}{1.434901in}}%
\pgfpathlineto{\pgfqpoint{4.450926in}{1.434901in}}%
\pgfpathlineto{\pgfqpoint{4.450926in}{1.431952in}}%
\pgfpathmoveto{\pgfqpoint{4.446385in}{1.434901in}}%
\pgfpathlineto{\pgfqpoint{4.446385in}{1.434901in}}%
\pgfpathlineto{\pgfqpoint{4.446385in}{1.437850in}}%
\pgfpathlineto{\pgfqpoint{4.450926in}{1.437850in}}%
\pgfpathlineto{\pgfqpoint{4.450926in}{1.434901in}}%
\pgfpathmoveto{\pgfqpoint{4.487255in}{1.375916in}}%
\pgfpathlineto{\pgfqpoint{4.487255in}{1.375916in}}%
\pgfpathlineto{\pgfqpoint{4.487255in}{1.378865in}}%
\pgfpathlineto{\pgfqpoint{4.491796in}{1.378865in}}%
\pgfpathlineto{\pgfqpoint{4.491796in}{1.375916in}}%
\pgfpathmoveto{\pgfqpoint{4.482714in}{1.381814in}}%
\pgfpathlineto{\pgfqpoint{4.482714in}{1.381814in}}%
\pgfpathlineto{\pgfqpoint{4.482714in}{1.384764in}}%
\pgfpathlineto{\pgfqpoint{4.487255in}{1.384764in}}%
\pgfpathlineto{\pgfqpoint{4.487255in}{1.381814in}}%
\pgfpathmoveto{\pgfqpoint{4.487255in}{1.378865in}}%
\pgfpathlineto{\pgfqpoint{4.487255in}{1.378865in}}%
\pgfpathlineto{\pgfqpoint{4.487255in}{1.381814in}}%
\pgfpathlineto{\pgfqpoint{4.491796in}{1.381814in}}%
\pgfpathlineto{\pgfqpoint{4.491796in}{1.378865in}}%
\pgfpathmoveto{\pgfqpoint{4.487255in}{1.381814in}}%
\pgfpathlineto{\pgfqpoint{4.487255in}{1.381814in}}%
\pgfpathlineto{\pgfqpoint{4.487255in}{1.384764in}}%
\pgfpathlineto{\pgfqpoint{4.491796in}{1.384764in}}%
\pgfpathlineto{\pgfqpoint{4.491796in}{1.381814in}}%
\pgfpathmoveto{\pgfqpoint{4.478173in}{1.387713in}}%
\pgfpathlineto{\pgfqpoint{4.478173in}{1.387713in}}%
\pgfpathlineto{\pgfqpoint{4.478173in}{1.390662in}}%
\pgfpathlineto{\pgfqpoint{4.482714in}{1.390662in}}%
\pgfpathlineto{\pgfqpoint{4.482714in}{1.387713in}}%
\pgfpathmoveto{\pgfqpoint{4.473632in}{1.393611in}}%
\pgfpathlineto{\pgfqpoint{4.473632in}{1.393611in}}%
\pgfpathlineto{\pgfqpoint{4.473632in}{1.396561in}}%
\pgfpathlineto{\pgfqpoint{4.478173in}{1.396561in}}%
\pgfpathlineto{\pgfqpoint{4.478173in}{1.393611in}}%
\pgfpathmoveto{\pgfqpoint{4.478173in}{1.390662in}}%
\pgfpathlineto{\pgfqpoint{4.478173in}{1.390662in}}%
\pgfpathlineto{\pgfqpoint{4.478173in}{1.393611in}}%
\pgfpathlineto{\pgfqpoint{4.482714in}{1.393611in}}%
\pgfpathlineto{\pgfqpoint{4.482714in}{1.390662in}}%
\pgfpathmoveto{\pgfqpoint{4.478173in}{1.393611in}}%
\pgfpathlineto{\pgfqpoint{4.478173in}{1.393611in}}%
\pgfpathlineto{\pgfqpoint{4.478173in}{1.396561in}}%
\pgfpathlineto{\pgfqpoint{4.482714in}{1.396561in}}%
\pgfpathlineto{\pgfqpoint{4.482714in}{1.393611in}}%
\pgfpathmoveto{\pgfqpoint{4.482714in}{1.384764in}}%
\pgfpathlineto{\pgfqpoint{4.482714in}{1.384764in}}%
\pgfpathlineto{\pgfqpoint{4.482714in}{1.387713in}}%
\pgfpathlineto{\pgfqpoint{4.487255in}{1.387713in}}%
\pgfpathlineto{\pgfqpoint{4.487255in}{1.384764in}}%
\pgfpathmoveto{\pgfqpoint{4.482714in}{1.387713in}}%
\pgfpathlineto{\pgfqpoint{4.482714in}{1.387713in}}%
\pgfpathlineto{\pgfqpoint{4.482714in}{1.390662in}}%
\pgfpathlineto{\pgfqpoint{4.487255in}{1.390662in}}%
\pgfpathlineto{\pgfqpoint{4.487255in}{1.387713in}}%
\pgfpathmoveto{\pgfqpoint{4.505419in}{1.352322in}}%
\pgfpathlineto{\pgfqpoint{4.505419in}{1.352322in}}%
\pgfpathlineto{\pgfqpoint{4.505419in}{1.355271in}}%
\pgfpathlineto{\pgfqpoint{4.509960in}{1.355271in}}%
\pgfpathlineto{\pgfqpoint{4.509960in}{1.352322in}}%
\pgfpathmoveto{\pgfqpoint{4.500878in}{1.358220in}}%
\pgfpathlineto{\pgfqpoint{4.500878in}{1.358220in}}%
\pgfpathlineto{\pgfqpoint{4.500878in}{1.361170in}}%
\pgfpathlineto{\pgfqpoint{4.505419in}{1.361170in}}%
\pgfpathlineto{\pgfqpoint{4.505419in}{1.358220in}}%
\pgfpathmoveto{\pgfqpoint{4.505419in}{1.355271in}}%
\pgfpathlineto{\pgfqpoint{4.505419in}{1.355271in}}%
\pgfpathlineto{\pgfqpoint{4.505419in}{1.358220in}}%
\pgfpathlineto{\pgfqpoint{4.509960in}{1.358220in}}%
\pgfpathlineto{\pgfqpoint{4.509960in}{1.355271in}}%
\pgfpathmoveto{\pgfqpoint{4.505419in}{1.358220in}}%
\pgfpathlineto{\pgfqpoint{4.505419in}{1.358220in}}%
\pgfpathlineto{\pgfqpoint{4.505419in}{1.361170in}}%
\pgfpathlineto{\pgfqpoint{4.509960in}{1.361170in}}%
\pgfpathlineto{\pgfqpoint{4.509960in}{1.358220in}}%
\pgfpathmoveto{\pgfqpoint{4.496337in}{1.364119in}}%
\pgfpathlineto{\pgfqpoint{4.496337in}{1.364119in}}%
\pgfpathlineto{\pgfqpoint{4.496337in}{1.367068in}}%
\pgfpathlineto{\pgfqpoint{4.500878in}{1.367068in}}%
\pgfpathlineto{\pgfqpoint{4.500878in}{1.364119in}}%
\pgfpathmoveto{\pgfqpoint{4.491796in}{1.370017in}}%
\pgfpathlineto{\pgfqpoint{4.491796in}{1.370017in}}%
\pgfpathlineto{\pgfqpoint{4.491796in}{1.372967in}}%
\pgfpathlineto{\pgfqpoint{4.496337in}{1.372967in}}%
\pgfpathlineto{\pgfqpoint{4.496337in}{1.370017in}}%
\pgfpathmoveto{\pgfqpoint{4.496337in}{1.367068in}}%
\pgfpathlineto{\pgfqpoint{4.496337in}{1.367068in}}%
\pgfpathlineto{\pgfqpoint{4.496337in}{1.370017in}}%
\pgfpathlineto{\pgfqpoint{4.500878in}{1.370017in}}%
\pgfpathlineto{\pgfqpoint{4.500878in}{1.367068in}}%
\pgfpathmoveto{\pgfqpoint{4.496337in}{1.370017in}}%
\pgfpathlineto{\pgfqpoint{4.496337in}{1.370017in}}%
\pgfpathlineto{\pgfqpoint{4.496337in}{1.372967in}}%
\pgfpathlineto{\pgfqpoint{4.500878in}{1.372967in}}%
\pgfpathlineto{\pgfqpoint{4.500878in}{1.370017in}}%
\pgfpathmoveto{\pgfqpoint{4.500878in}{1.361170in}}%
\pgfpathlineto{\pgfqpoint{4.500878in}{1.361170in}}%
\pgfpathlineto{\pgfqpoint{4.500878in}{1.364119in}}%
\pgfpathlineto{\pgfqpoint{4.505419in}{1.364119in}}%
\pgfpathlineto{\pgfqpoint{4.505419in}{1.361170in}}%
\pgfpathmoveto{\pgfqpoint{4.500878in}{1.364119in}}%
\pgfpathlineto{\pgfqpoint{4.500878in}{1.364119in}}%
\pgfpathlineto{\pgfqpoint{4.500878in}{1.367068in}}%
\pgfpathlineto{\pgfqpoint{4.505419in}{1.367068in}}%
\pgfpathlineto{\pgfqpoint{4.505419in}{1.364119in}}%
\pgfpathmoveto{\pgfqpoint{4.509960in}{1.349372in}}%
\pgfpathlineto{\pgfqpoint{4.509960in}{1.349372in}}%
\pgfpathlineto{\pgfqpoint{4.509960in}{1.352322in}}%
\pgfpathlineto{\pgfqpoint{4.514501in}{1.352322in}}%
\pgfpathlineto{\pgfqpoint{4.514501in}{1.349372in}}%
\pgfpathmoveto{\pgfqpoint{4.509960in}{1.352322in}}%
\pgfpathlineto{\pgfqpoint{4.509960in}{1.352322in}}%
\pgfpathlineto{\pgfqpoint{4.509960in}{1.355271in}}%
\pgfpathlineto{\pgfqpoint{4.514501in}{1.355271in}}%
\pgfpathlineto{\pgfqpoint{4.514501in}{1.352322in}}%
\pgfpathmoveto{\pgfqpoint{4.491796in}{1.372967in}}%
\pgfpathlineto{\pgfqpoint{4.491796in}{1.372967in}}%
\pgfpathlineto{\pgfqpoint{4.491796in}{1.375916in}}%
\pgfpathlineto{\pgfqpoint{4.496337in}{1.375916in}}%
\pgfpathlineto{\pgfqpoint{4.496337in}{1.372967in}}%
\pgfpathmoveto{\pgfqpoint{4.491796in}{1.375916in}}%
\pgfpathlineto{\pgfqpoint{4.491796in}{1.375916in}}%
\pgfpathlineto{\pgfqpoint{4.491796in}{1.378865in}}%
\pgfpathlineto{\pgfqpoint{4.496337in}{1.378865in}}%
\pgfpathlineto{\pgfqpoint{4.496337in}{1.375916in}}%
\pgfpathmoveto{\pgfqpoint{4.469091in}{1.399510in}}%
\pgfpathlineto{\pgfqpoint{4.469091in}{1.399510in}}%
\pgfpathlineto{\pgfqpoint{4.469091in}{1.402459in}}%
\pgfpathlineto{\pgfqpoint{4.473632in}{1.402459in}}%
\pgfpathlineto{\pgfqpoint{4.473632in}{1.399510in}}%
\pgfpathmoveto{\pgfqpoint{4.464550in}{1.405409in}}%
\pgfpathlineto{\pgfqpoint{4.464550in}{1.405409in}}%
\pgfpathlineto{\pgfqpoint{4.464550in}{1.408358in}}%
\pgfpathlineto{\pgfqpoint{4.469091in}{1.408358in}}%
\pgfpathlineto{\pgfqpoint{4.469091in}{1.405409in}}%
\pgfpathmoveto{\pgfqpoint{4.469091in}{1.402459in}}%
\pgfpathlineto{\pgfqpoint{4.469091in}{1.402459in}}%
\pgfpathlineto{\pgfqpoint{4.469091in}{1.405409in}}%
\pgfpathlineto{\pgfqpoint{4.473632in}{1.405409in}}%
\pgfpathlineto{\pgfqpoint{4.473632in}{1.402459in}}%
\pgfpathmoveto{\pgfqpoint{4.469091in}{1.405409in}}%
\pgfpathlineto{\pgfqpoint{4.469091in}{1.405409in}}%
\pgfpathlineto{\pgfqpoint{4.469091in}{1.408358in}}%
\pgfpathlineto{\pgfqpoint{4.473632in}{1.408358in}}%
\pgfpathlineto{\pgfqpoint{4.473632in}{1.405409in}}%
\pgfpathmoveto{\pgfqpoint{4.460009in}{1.411307in}}%
\pgfpathlineto{\pgfqpoint{4.460009in}{1.411307in}}%
\pgfpathlineto{\pgfqpoint{4.460009in}{1.414256in}}%
\pgfpathlineto{\pgfqpoint{4.464550in}{1.414256in}}%
\pgfpathlineto{\pgfqpoint{4.464550in}{1.411307in}}%
\pgfpathmoveto{\pgfqpoint{4.455467in}{1.417206in}}%
\pgfpathlineto{\pgfqpoint{4.455467in}{1.417206in}}%
\pgfpathlineto{\pgfqpoint{4.455467in}{1.420155in}}%
\pgfpathlineto{\pgfqpoint{4.460009in}{1.420155in}}%
\pgfpathlineto{\pgfqpoint{4.460009in}{1.417206in}}%
\pgfpathmoveto{\pgfqpoint{4.460009in}{1.414256in}}%
\pgfpathlineto{\pgfqpoint{4.460009in}{1.414256in}}%
\pgfpathlineto{\pgfqpoint{4.460009in}{1.417206in}}%
\pgfpathlineto{\pgfqpoint{4.464550in}{1.417206in}}%
\pgfpathlineto{\pgfqpoint{4.464550in}{1.414256in}}%
\pgfpathmoveto{\pgfqpoint{4.460009in}{1.417206in}}%
\pgfpathlineto{\pgfqpoint{4.460009in}{1.417206in}}%
\pgfpathlineto{\pgfqpoint{4.460009in}{1.420155in}}%
\pgfpathlineto{\pgfqpoint{4.464550in}{1.420155in}}%
\pgfpathlineto{\pgfqpoint{4.464550in}{1.417206in}}%
\pgfpathmoveto{\pgfqpoint{4.464550in}{1.408358in}}%
\pgfpathlineto{\pgfqpoint{4.464550in}{1.408358in}}%
\pgfpathlineto{\pgfqpoint{4.464550in}{1.411307in}}%
\pgfpathlineto{\pgfqpoint{4.469091in}{1.411307in}}%
\pgfpathlineto{\pgfqpoint{4.469091in}{1.408358in}}%
\pgfpathmoveto{\pgfqpoint{4.464550in}{1.411307in}}%
\pgfpathlineto{\pgfqpoint{4.464550in}{1.411307in}}%
\pgfpathlineto{\pgfqpoint{4.464550in}{1.414256in}}%
\pgfpathlineto{\pgfqpoint{4.469091in}{1.414256in}}%
\pgfpathlineto{\pgfqpoint{4.469091in}{1.411307in}}%
\pgfpathmoveto{\pgfqpoint{4.473632in}{1.396561in}}%
\pgfpathlineto{\pgfqpoint{4.473632in}{1.396561in}}%
\pgfpathlineto{\pgfqpoint{4.473632in}{1.399510in}}%
\pgfpathlineto{\pgfqpoint{4.478173in}{1.399510in}}%
\pgfpathlineto{\pgfqpoint{4.478173in}{1.396561in}}%
\pgfpathmoveto{\pgfqpoint{4.473632in}{1.399510in}}%
\pgfpathlineto{\pgfqpoint{4.473632in}{1.399510in}}%
\pgfpathlineto{\pgfqpoint{4.473632in}{1.402459in}}%
\pgfpathlineto{\pgfqpoint{4.478173in}{1.402459in}}%
\pgfpathlineto{\pgfqpoint{4.478173in}{1.399510in}}%
\pgfpathmoveto{\pgfqpoint{4.455467in}{1.420155in}}%
\pgfpathlineto{\pgfqpoint{4.455467in}{1.420155in}}%
\pgfpathlineto{\pgfqpoint{4.455467in}{1.423104in}}%
\pgfpathlineto{\pgfqpoint{4.460009in}{1.423104in}}%
\pgfpathlineto{\pgfqpoint{4.460009in}{1.420155in}}%
\pgfpathmoveto{\pgfqpoint{4.455467in}{1.423104in}}%
\pgfpathlineto{\pgfqpoint{4.455467in}{1.423104in}}%
\pgfpathlineto{\pgfqpoint{4.455467in}{1.426053in}}%
\pgfpathlineto{\pgfqpoint{4.460009in}{1.426053in}}%
\pgfpathlineto{\pgfqpoint{4.460009in}{1.423104in}}%
\pgfpathmoveto{\pgfqpoint{4.414598in}{1.470292in}}%
\pgfpathlineto{\pgfqpoint{4.414598in}{1.470292in}}%
\pgfpathlineto{\pgfqpoint{4.414598in}{1.473241in}}%
\pgfpathlineto{\pgfqpoint{4.419139in}{1.473241in}}%
\pgfpathlineto{\pgfqpoint{4.419139in}{1.470292in}}%
\pgfpathmoveto{\pgfqpoint{4.410057in}{1.476190in}}%
\pgfpathlineto{\pgfqpoint{4.410057in}{1.476190in}}%
\pgfpathlineto{\pgfqpoint{4.410057in}{1.479139in}}%
\pgfpathlineto{\pgfqpoint{4.414598in}{1.479139in}}%
\pgfpathlineto{\pgfqpoint{4.414598in}{1.476190in}}%
\pgfpathmoveto{\pgfqpoint{4.414598in}{1.473241in}}%
\pgfpathlineto{\pgfqpoint{4.414598in}{1.473241in}}%
\pgfpathlineto{\pgfqpoint{4.414598in}{1.476190in}}%
\pgfpathlineto{\pgfqpoint{4.419139in}{1.476190in}}%
\pgfpathlineto{\pgfqpoint{4.419139in}{1.473241in}}%
\pgfpathmoveto{\pgfqpoint{4.414598in}{1.476190in}}%
\pgfpathlineto{\pgfqpoint{4.414598in}{1.476190in}}%
\pgfpathlineto{\pgfqpoint{4.414598in}{1.479139in}}%
\pgfpathlineto{\pgfqpoint{4.419139in}{1.479139in}}%
\pgfpathlineto{\pgfqpoint{4.419139in}{1.476190in}}%
\pgfpathmoveto{\pgfqpoint{4.405516in}{1.482088in}}%
\pgfpathlineto{\pgfqpoint{4.405516in}{1.482088in}}%
\pgfpathlineto{\pgfqpoint{4.405516in}{1.485038in}}%
\pgfpathlineto{\pgfqpoint{4.410057in}{1.485038in}}%
\pgfpathlineto{\pgfqpoint{4.410057in}{1.482088in}}%
\pgfpathmoveto{\pgfqpoint{4.400975in}{1.487987in}}%
\pgfpathlineto{\pgfqpoint{4.400975in}{1.487987in}}%
\pgfpathlineto{\pgfqpoint{4.400975in}{1.490936in}}%
\pgfpathlineto{\pgfqpoint{4.405516in}{1.490936in}}%
\pgfpathlineto{\pgfqpoint{4.405516in}{1.487987in}}%
\pgfpathmoveto{\pgfqpoint{4.405516in}{1.485038in}}%
\pgfpathlineto{\pgfqpoint{4.405516in}{1.485038in}}%
\pgfpathlineto{\pgfqpoint{4.405516in}{1.487987in}}%
\pgfpathlineto{\pgfqpoint{4.410057in}{1.487987in}}%
\pgfpathlineto{\pgfqpoint{4.410057in}{1.485038in}}%
\pgfpathmoveto{\pgfqpoint{4.405516in}{1.487987in}}%
\pgfpathlineto{\pgfqpoint{4.405516in}{1.487987in}}%
\pgfpathlineto{\pgfqpoint{4.405516in}{1.490936in}}%
\pgfpathlineto{\pgfqpoint{4.410057in}{1.490936in}}%
\pgfpathlineto{\pgfqpoint{4.410057in}{1.487987in}}%
\pgfpathmoveto{\pgfqpoint{4.410057in}{1.479139in}}%
\pgfpathlineto{\pgfqpoint{4.410057in}{1.479139in}}%
\pgfpathlineto{\pgfqpoint{4.410057in}{1.482088in}}%
\pgfpathlineto{\pgfqpoint{4.414598in}{1.482088in}}%
\pgfpathlineto{\pgfqpoint{4.414598in}{1.479139in}}%
\pgfpathmoveto{\pgfqpoint{4.410057in}{1.482088in}}%
\pgfpathlineto{\pgfqpoint{4.410057in}{1.482088in}}%
\pgfpathlineto{\pgfqpoint{4.410057in}{1.485038in}}%
\pgfpathlineto{\pgfqpoint{4.414598in}{1.485038in}}%
\pgfpathlineto{\pgfqpoint{4.414598in}{1.482088in}}%
\pgfpathmoveto{\pgfqpoint{4.432762in}{1.446698in}}%
\pgfpathlineto{\pgfqpoint{4.432762in}{1.446698in}}%
\pgfpathlineto{\pgfqpoint{4.432762in}{1.449647in}}%
\pgfpathlineto{\pgfqpoint{4.437303in}{1.449647in}}%
\pgfpathlineto{\pgfqpoint{4.437303in}{1.446698in}}%
\pgfpathmoveto{\pgfqpoint{4.428221in}{1.452597in}}%
\pgfpathlineto{\pgfqpoint{4.428221in}{1.452597in}}%
\pgfpathlineto{\pgfqpoint{4.428221in}{1.455546in}}%
\pgfpathlineto{\pgfqpoint{4.432762in}{1.455546in}}%
\pgfpathlineto{\pgfqpoint{4.432762in}{1.452597in}}%
\pgfpathmoveto{\pgfqpoint{4.432762in}{1.449647in}}%
\pgfpathlineto{\pgfqpoint{4.432762in}{1.449647in}}%
\pgfpathlineto{\pgfqpoint{4.432762in}{1.452597in}}%
\pgfpathlineto{\pgfqpoint{4.437303in}{1.452597in}}%
\pgfpathlineto{\pgfqpoint{4.437303in}{1.449647in}}%
\pgfpathmoveto{\pgfqpoint{4.432762in}{1.452597in}}%
\pgfpathlineto{\pgfqpoint{4.432762in}{1.452597in}}%
\pgfpathlineto{\pgfqpoint{4.432762in}{1.455546in}}%
\pgfpathlineto{\pgfqpoint{4.437303in}{1.455546in}}%
\pgfpathlineto{\pgfqpoint{4.437303in}{1.452597in}}%
\pgfpathmoveto{\pgfqpoint{4.423680in}{1.458495in}}%
\pgfpathlineto{\pgfqpoint{4.423680in}{1.458495in}}%
\pgfpathlineto{\pgfqpoint{4.423680in}{1.461444in}}%
\pgfpathlineto{\pgfqpoint{4.428221in}{1.461444in}}%
\pgfpathlineto{\pgfqpoint{4.428221in}{1.458495in}}%
\pgfpathmoveto{\pgfqpoint{4.419139in}{1.464393in}}%
\pgfpathlineto{\pgfqpoint{4.419139in}{1.464393in}}%
\pgfpathlineto{\pgfqpoint{4.419139in}{1.467342in}}%
\pgfpathlineto{\pgfqpoint{4.423680in}{1.467342in}}%
\pgfpathlineto{\pgfqpoint{4.423680in}{1.464393in}}%
\pgfpathmoveto{\pgfqpoint{4.423680in}{1.461444in}}%
\pgfpathlineto{\pgfqpoint{4.423680in}{1.461444in}}%
\pgfpathlineto{\pgfqpoint{4.423680in}{1.464393in}}%
\pgfpathlineto{\pgfqpoint{4.428221in}{1.464393in}}%
\pgfpathlineto{\pgfqpoint{4.428221in}{1.461444in}}%
\pgfpathmoveto{\pgfqpoint{4.423680in}{1.464393in}}%
\pgfpathlineto{\pgfqpoint{4.423680in}{1.464393in}}%
\pgfpathlineto{\pgfqpoint{4.423680in}{1.467342in}}%
\pgfpathlineto{\pgfqpoint{4.428221in}{1.467342in}}%
\pgfpathlineto{\pgfqpoint{4.428221in}{1.464393in}}%
\pgfpathmoveto{\pgfqpoint{4.428221in}{1.455546in}}%
\pgfpathlineto{\pgfqpoint{4.428221in}{1.455546in}}%
\pgfpathlineto{\pgfqpoint{4.428221in}{1.458495in}}%
\pgfpathlineto{\pgfqpoint{4.432762in}{1.458495in}}%
\pgfpathlineto{\pgfqpoint{4.432762in}{1.455546in}}%
\pgfpathmoveto{\pgfqpoint{4.428221in}{1.458495in}}%
\pgfpathlineto{\pgfqpoint{4.428221in}{1.458495in}}%
\pgfpathlineto{\pgfqpoint{4.428221in}{1.461444in}}%
\pgfpathlineto{\pgfqpoint{4.432762in}{1.461444in}}%
\pgfpathlineto{\pgfqpoint{4.432762in}{1.458495in}}%
\pgfpathmoveto{\pgfqpoint{4.437303in}{1.443749in}}%
\pgfpathlineto{\pgfqpoint{4.437303in}{1.443749in}}%
\pgfpathlineto{\pgfqpoint{4.437303in}{1.446698in}}%
\pgfpathlineto{\pgfqpoint{4.441844in}{1.446698in}}%
\pgfpathlineto{\pgfqpoint{4.441844in}{1.443749in}}%
\pgfpathmoveto{\pgfqpoint{4.437303in}{1.446698in}}%
\pgfpathlineto{\pgfqpoint{4.437303in}{1.446698in}}%
\pgfpathlineto{\pgfqpoint{4.437303in}{1.449647in}}%
\pgfpathlineto{\pgfqpoint{4.441844in}{1.449647in}}%
\pgfpathlineto{\pgfqpoint{4.441844in}{1.446698in}}%
\pgfpathmoveto{\pgfqpoint{4.419139in}{1.467342in}}%
\pgfpathlineto{\pgfqpoint{4.419139in}{1.467342in}}%
\pgfpathlineto{\pgfqpoint{4.419139in}{1.470292in}}%
\pgfpathlineto{\pgfqpoint{4.423680in}{1.470292in}}%
\pgfpathlineto{\pgfqpoint{4.423680in}{1.467342in}}%
\pgfpathmoveto{\pgfqpoint{4.419139in}{1.470292in}}%
\pgfpathlineto{\pgfqpoint{4.419139in}{1.470292in}}%
\pgfpathlineto{\pgfqpoint{4.419139in}{1.473241in}}%
\pgfpathlineto{\pgfqpoint{4.423680in}{1.473241in}}%
\pgfpathlineto{\pgfqpoint{4.423680in}{1.470292in}}%
\pgfpathmoveto{\pgfqpoint{4.396434in}{1.493885in}}%
\pgfpathlineto{\pgfqpoint{4.396434in}{1.493885in}}%
\pgfpathlineto{\pgfqpoint{4.396434in}{1.496834in}}%
\pgfpathlineto{\pgfqpoint{4.400975in}{1.496834in}}%
\pgfpathlineto{\pgfqpoint{4.400975in}{1.493885in}}%
\pgfpathmoveto{\pgfqpoint{4.391893in}{1.499784in}}%
\pgfpathlineto{\pgfqpoint{4.391893in}{1.499784in}}%
\pgfpathlineto{\pgfqpoint{4.391893in}{1.502733in}}%
\pgfpathlineto{\pgfqpoint{4.396434in}{1.502733in}}%
\pgfpathlineto{\pgfqpoint{4.396434in}{1.499784in}}%
\pgfpathmoveto{\pgfqpoint{4.396434in}{1.496834in}}%
\pgfpathlineto{\pgfqpoint{4.396434in}{1.496834in}}%
\pgfpathlineto{\pgfqpoint{4.396434in}{1.499784in}}%
\pgfpathlineto{\pgfqpoint{4.400975in}{1.499784in}}%
\pgfpathlineto{\pgfqpoint{4.400975in}{1.496834in}}%
\pgfpathmoveto{\pgfqpoint{4.396434in}{1.499784in}}%
\pgfpathlineto{\pgfqpoint{4.396434in}{1.499784in}}%
\pgfpathlineto{\pgfqpoint{4.396434in}{1.502733in}}%
\pgfpathlineto{\pgfqpoint{4.400975in}{1.502733in}}%
\pgfpathlineto{\pgfqpoint{4.400975in}{1.499784in}}%
\pgfpathmoveto{\pgfqpoint{4.387352in}{1.505682in}}%
\pgfpathlineto{\pgfqpoint{4.387352in}{1.505682in}}%
\pgfpathlineto{\pgfqpoint{4.387352in}{1.508631in}}%
\pgfpathlineto{\pgfqpoint{4.391893in}{1.508631in}}%
\pgfpathlineto{\pgfqpoint{4.391893in}{1.505682in}}%
\pgfpathmoveto{\pgfqpoint{4.382811in}{1.511580in}}%
\pgfpathlineto{\pgfqpoint{4.382811in}{1.511580in}}%
\pgfpathlineto{\pgfqpoint{4.382811in}{1.514529in}}%
\pgfpathlineto{\pgfqpoint{4.387352in}{1.514529in}}%
\pgfpathlineto{\pgfqpoint{4.387352in}{1.511580in}}%
\pgfpathmoveto{\pgfqpoint{4.387352in}{1.508631in}}%
\pgfpathlineto{\pgfqpoint{4.387352in}{1.508631in}}%
\pgfpathlineto{\pgfqpoint{4.387352in}{1.511580in}}%
\pgfpathlineto{\pgfqpoint{4.391893in}{1.511580in}}%
\pgfpathlineto{\pgfqpoint{4.391893in}{1.508631in}}%
\pgfpathmoveto{\pgfqpoint{4.387352in}{1.511580in}}%
\pgfpathlineto{\pgfqpoint{4.387352in}{1.511580in}}%
\pgfpathlineto{\pgfqpoint{4.387352in}{1.514529in}}%
\pgfpathlineto{\pgfqpoint{4.391893in}{1.514529in}}%
\pgfpathlineto{\pgfqpoint{4.391893in}{1.511580in}}%
\pgfpathmoveto{\pgfqpoint{4.391893in}{1.502733in}}%
\pgfpathlineto{\pgfqpoint{4.391893in}{1.502733in}}%
\pgfpathlineto{\pgfqpoint{4.391893in}{1.505682in}}%
\pgfpathlineto{\pgfqpoint{4.396434in}{1.505682in}}%
\pgfpathlineto{\pgfqpoint{4.396434in}{1.502733in}}%
\pgfpathmoveto{\pgfqpoint{4.391893in}{1.505682in}}%
\pgfpathlineto{\pgfqpoint{4.391893in}{1.505682in}}%
\pgfpathlineto{\pgfqpoint{4.391893in}{1.508631in}}%
\pgfpathlineto{\pgfqpoint{4.396434in}{1.508631in}}%
\pgfpathlineto{\pgfqpoint{4.396434in}{1.505682in}}%
\pgfpathmoveto{\pgfqpoint{4.400975in}{1.490936in}}%
\pgfpathlineto{\pgfqpoint{4.400975in}{1.490936in}}%
\pgfpathlineto{\pgfqpoint{4.400975in}{1.493885in}}%
\pgfpathlineto{\pgfqpoint{4.405516in}{1.493885in}}%
\pgfpathlineto{\pgfqpoint{4.405516in}{1.490936in}}%
\pgfpathmoveto{\pgfqpoint{4.400975in}{1.493885in}}%
\pgfpathlineto{\pgfqpoint{4.400975in}{1.493885in}}%
\pgfpathlineto{\pgfqpoint{4.400975in}{1.496834in}}%
\pgfpathlineto{\pgfqpoint{4.405516in}{1.496834in}}%
\pgfpathlineto{\pgfqpoint{4.405516in}{1.493885in}}%
\pgfpathmoveto{\pgfqpoint{4.382811in}{1.514529in}}%
\pgfpathlineto{\pgfqpoint{4.382811in}{1.514529in}}%
\pgfpathlineto{\pgfqpoint{4.382811in}{1.517479in}}%
\pgfpathlineto{\pgfqpoint{4.387352in}{1.517479in}}%
\pgfpathlineto{\pgfqpoint{4.387352in}{1.514529in}}%
\pgfpathmoveto{\pgfqpoint{4.382811in}{1.517479in}}%
\pgfpathlineto{\pgfqpoint{4.382811in}{1.517479in}}%
\pgfpathlineto{\pgfqpoint{4.382811in}{1.520428in}}%
\pgfpathlineto{\pgfqpoint{4.387352in}{1.520428in}}%
\pgfpathlineto{\pgfqpoint{4.387352in}{1.517479in}}%
\pgfpathmoveto{\pgfqpoint{4.668899in}{1.139982in}}%
\pgfpathlineto{\pgfqpoint{4.668899in}{1.139982in}}%
\pgfpathlineto{\pgfqpoint{4.668899in}{1.142931in}}%
\pgfpathlineto{\pgfqpoint{4.673441in}{1.142931in}}%
\pgfpathlineto{\pgfqpoint{4.673441in}{1.139982in}}%
\pgfpathmoveto{\pgfqpoint{4.664358in}{1.145880in}}%
\pgfpathlineto{\pgfqpoint{4.664358in}{1.145880in}}%
\pgfpathlineto{\pgfqpoint{4.664358in}{1.148829in}}%
\pgfpathlineto{\pgfqpoint{4.668899in}{1.148829in}}%
\pgfpathlineto{\pgfqpoint{4.668899in}{1.145880in}}%
\pgfpathmoveto{\pgfqpoint{4.668899in}{1.142931in}}%
\pgfpathlineto{\pgfqpoint{4.668899in}{1.142931in}}%
\pgfpathlineto{\pgfqpoint{4.668899in}{1.145880in}}%
\pgfpathlineto{\pgfqpoint{4.673441in}{1.145880in}}%
\pgfpathlineto{\pgfqpoint{4.673441in}{1.142931in}}%
\pgfpathmoveto{\pgfqpoint{4.668899in}{1.145880in}}%
\pgfpathlineto{\pgfqpoint{4.668899in}{1.145880in}}%
\pgfpathlineto{\pgfqpoint{4.668899in}{1.148829in}}%
\pgfpathlineto{\pgfqpoint{4.673441in}{1.148829in}}%
\pgfpathlineto{\pgfqpoint{4.673441in}{1.145880in}}%
\pgfpathmoveto{\pgfqpoint{4.659817in}{1.151779in}}%
\pgfpathlineto{\pgfqpoint{4.659817in}{1.151779in}}%
\pgfpathlineto{\pgfqpoint{4.659817in}{1.154728in}}%
\pgfpathlineto{\pgfqpoint{4.664358in}{1.154728in}}%
\pgfpathlineto{\pgfqpoint{4.664358in}{1.151779in}}%
\pgfpathmoveto{\pgfqpoint{4.655276in}{1.157677in}}%
\pgfpathlineto{\pgfqpoint{4.655276in}{1.157677in}}%
\pgfpathlineto{\pgfqpoint{4.655276in}{1.160626in}}%
\pgfpathlineto{\pgfqpoint{4.659817in}{1.160626in}}%
\pgfpathlineto{\pgfqpoint{4.659817in}{1.157677in}}%
\pgfpathmoveto{\pgfqpoint{4.659817in}{1.154728in}}%
\pgfpathlineto{\pgfqpoint{4.659817in}{1.154728in}}%
\pgfpathlineto{\pgfqpoint{4.659817in}{1.157677in}}%
\pgfpathlineto{\pgfqpoint{4.664358in}{1.157677in}}%
\pgfpathlineto{\pgfqpoint{4.664358in}{1.154728in}}%
\pgfpathmoveto{\pgfqpoint{4.659817in}{1.157677in}}%
\pgfpathlineto{\pgfqpoint{4.659817in}{1.157677in}}%
\pgfpathlineto{\pgfqpoint{4.659817in}{1.160626in}}%
\pgfpathlineto{\pgfqpoint{4.664358in}{1.160626in}}%
\pgfpathlineto{\pgfqpoint{4.664358in}{1.157677in}}%
\pgfpathmoveto{\pgfqpoint{4.664358in}{1.148829in}}%
\pgfpathlineto{\pgfqpoint{4.664358in}{1.148829in}}%
\pgfpathlineto{\pgfqpoint{4.664358in}{1.151779in}}%
\pgfpathlineto{\pgfqpoint{4.668899in}{1.151779in}}%
\pgfpathlineto{\pgfqpoint{4.668899in}{1.148829in}}%
\pgfpathmoveto{\pgfqpoint{4.664358in}{1.151779in}}%
\pgfpathlineto{\pgfqpoint{4.664358in}{1.151779in}}%
\pgfpathlineto{\pgfqpoint{4.664358in}{1.154728in}}%
\pgfpathlineto{\pgfqpoint{4.668899in}{1.154728in}}%
\pgfpathlineto{\pgfqpoint{4.668899in}{1.151779in}}%
\pgfpathmoveto{\pgfqpoint{4.596241in}{1.234356in}}%
\pgfpathlineto{\pgfqpoint{4.596241in}{1.234356in}}%
\pgfpathlineto{\pgfqpoint{4.596241in}{1.237306in}}%
\pgfpathlineto{\pgfqpoint{4.600782in}{1.237306in}}%
\pgfpathlineto{\pgfqpoint{4.600782in}{1.234356in}}%
\pgfpathmoveto{\pgfqpoint{4.591700in}{1.240255in}}%
\pgfpathlineto{\pgfqpoint{4.591700in}{1.240255in}}%
\pgfpathlineto{\pgfqpoint{4.591700in}{1.243204in}}%
\pgfpathlineto{\pgfqpoint{4.596241in}{1.243204in}}%
\pgfpathlineto{\pgfqpoint{4.596241in}{1.240255in}}%
\pgfpathmoveto{\pgfqpoint{4.596241in}{1.237306in}}%
\pgfpathlineto{\pgfqpoint{4.596241in}{1.237306in}}%
\pgfpathlineto{\pgfqpoint{4.596241in}{1.240255in}}%
\pgfpathlineto{\pgfqpoint{4.600782in}{1.240255in}}%
\pgfpathlineto{\pgfqpoint{4.600782in}{1.237306in}}%
\pgfpathmoveto{\pgfqpoint{4.596241in}{1.240255in}}%
\pgfpathlineto{\pgfqpoint{4.596241in}{1.240255in}}%
\pgfpathlineto{\pgfqpoint{4.596241in}{1.243204in}}%
\pgfpathlineto{\pgfqpoint{4.600782in}{1.243204in}}%
\pgfpathlineto{\pgfqpoint{4.600782in}{1.240255in}}%
\pgfpathmoveto{\pgfqpoint{4.587159in}{1.246153in}}%
\pgfpathlineto{\pgfqpoint{4.587159in}{1.246153in}}%
\pgfpathlineto{\pgfqpoint{4.587159in}{1.249102in}}%
\pgfpathlineto{\pgfqpoint{4.591700in}{1.249102in}}%
\pgfpathlineto{\pgfqpoint{4.591700in}{1.246153in}}%
\pgfpathmoveto{\pgfqpoint{4.582618in}{1.252052in}}%
\pgfpathlineto{\pgfqpoint{4.582618in}{1.252052in}}%
\pgfpathlineto{\pgfqpoint{4.582618in}{1.255001in}}%
\pgfpathlineto{\pgfqpoint{4.587159in}{1.255001in}}%
\pgfpathlineto{\pgfqpoint{4.587159in}{1.252052in}}%
\pgfpathmoveto{\pgfqpoint{4.587159in}{1.249102in}}%
\pgfpathlineto{\pgfqpoint{4.587159in}{1.249102in}}%
\pgfpathlineto{\pgfqpoint{4.587159in}{1.252052in}}%
\pgfpathlineto{\pgfqpoint{4.591700in}{1.252052in}}%
\pgfpathlineto{\pgfqpoint{4.591700in}{1.249102in}}%
\pgfpathmoveto{\pgfqpoint{4.587159in}{1.252052in}}%
\pgfpathlineto{\pgfqpoint{4.587159in}{1.252052in}}%
\pgfpathlineto{\pgfqpoint{4.587159in}{1.255001in}}%
\pgfpathlineto{\pgfqpoint{4.591700in}{1.255001in}}%
\pgfpathlineto{\pgfqpoint{4.591700in}{1.252052in}}%
\pgfpathmoveto{\pgfqpoint{4.591700in}{1.243204in}}%
\pgfpathlineto{\pgfqpoint{4.591700in}{1.243204in}}%
\pgfpathlineto{\pgfqpoint{4.591700in}{1.246153in}}%
\pgfpathlineto{\pgfqpoint{4.596241in}{1.246153in}}%
\pgfpathlineto{\pgfqpoint{4.596241in}{1.243204in}}%
\pgfpathmoveto{\pgfqpoint{4.591700in}{1.246153in}}%
\pgfpathlineto{\pgfqpoint{4.591700in}{1.246153in}}%
\pgfpathlineto{\pgfqpoint{4.591700in}{1.249102in}}%
\pgfpathlineto{\pgfqpoint{4.596241in}{1.249102in}}%
\pgfpathlineto{\pgfqpoint{4.596241in}{1.246153in}}%
\pgfpathmoveto{\pgfqpoint{4.632570in}{1.187169in}}%
\pgfpathlineto{\pgfqpoint{4.632570in}{1.187169in}}%
\pgfpathlineto{\pgfqpoint{4.632570in}{1.190118in}}%
\pgfpathlineto{\pgfqpoint{4.637111in}{1.190118in}}%
\pgfpathlineto{\pgfqpoint{4.637111in}{1.187169in}}%
\pgfpathmoveto{\pgfqpoint{4.628029in}{1.193068in}}%
\pgfpathlineto{\pgfqpoint{4.628029in}{1.193068in}}%
\pgfpathlineto{\pgfqpoint{4.628029in}{1.196017in}}%
\pgfpathlineto{\pgfqpoint{4.632570in}{1.196017in}}%
\pgfpathlineto{\pgfqpoint{4.632570in}{1.193068in}}%
\pgfpathmoveto{\pgfqpoint{4.632570in}{1.190118in}}%
\pgfpathlineto{\pgfqpoint{4.632570in}{1.190118in}}%
\pgfpathlineto{\pgfqpoint{4.632570in}{1.193068in}}%
\pgfpathlineto{\pgfqpoint{4.637111in}{1.193068in}}%
\pgfpathlineto{\pgfqpoint{4.637111in}{1.190118in}}%
\pgfpathmoveto{\pgfqpoint{4.632570in}{1.193068in}}%
\pgfpathlineto{\pgfqpoint{4.632570in}{1.193068in}}%
\pgfpathlineto{\pgfqpoint{4.632570in}{1.196017in}}%
\pgfpathlineto{\pgfqpoint{4.637111in}{1.196017in}}%
\pgfpathlineto{\pgfqpoint{4.637111in}{1.193068in}}%
\pgfpathmoveto{\pgfqpoint{4.623488in}{1.198966in}}%
\pgfpathlineto{\pgfqpoint{4.623488in}{1.198966in}}%
\pgfpathlineto{\pgfqpoint{4.623488in}{1.201915in}}%
\pgfpathlineto{\pgfqpoint{4.628029in}{1.201915in}}%
\pgfpathlineto{\pgfqpoint{4.628029in}{1.198966in}}%
\pgfpathmoveto{\pgfqpoint{4.618947in}{1.204864in}}%
\pgfpathlineto{\pgfqpoint{4.618947in}{1.204864in}}%
\pgfpathlineto{\pgfqpoint{4.618947in}{1.207814in}}%
\pgfpathlineto{\pgfqpoint{4.623488in}{1.207814in}}%
\pgfpathlineto{\pgfqpoint{4.623488in}{1.204864in}}%
\pgfpathmoveto{\pgfqpoint{4.623488in}{1.201915in}}%
\pgfpathlineto{\pgfqpoint{4.623488in}{1.201915in}}%
\pgfpathlineto{\pgfqpoint{4.623488in}{1.204864in}}%
\pgfpathlineto{\pgfqpoint{4.628029in}{1.204864in}}%
\pgfpathlineto{\pgfqpoint{4.628029in}{1.201915in}}%
\pgfpathmoveto{\pgfqpoint{4.623488in}{1.204864in}}%
\pgfpathlineto{\pgfqpoint{4.623488in}{1.204864in}}%
\pgfpathlineto{\pgfqpoint{4.623488in}{1.207814in}}%
\pgfpathlineto{\pgfqpoint{4.628029in}{1.207814in}}%
\pgfpathlineto{\pgfqpoint{4.628029in}{1.204864in}}%
\pgfpathmoveto{\pgfqpoint{4.628029in}{1.196017in}}%
\pgfpathlineto{\pgfqpoint{4.628029in}{1.196017in}}%
\pgfpathlineto{\pgfqpoint{4.628029in}{1.198966in}}%
\pgfpathlineto{\pgfqpoint{4.632570in}{1.198966in}}%
\pgfpathlineto{\pgfqpoint{4.632570in}{1.196017in}}%
\pgfpathmoveto{\pgfqpoint{4.628029in}{1.198966in}}%
\pgfpathlineto{\pgfqpoint{4.628029in}{1.198966in}}%
\pgfpathlineto{\pgfqpoint{4.628029in}{1.201915in}}%
\pgfpathlineto{\pgfqpoint{4.632570in}{1.201915in}}%
\pgfpathlineto{\pgfqpoint{4.632570in}{1.198966in}}%
\pgfpathmoveto{\pgfqpoint{4.650735in}{1.163576in}}%
\pgfpathlineto{\pgfqpoint{4.650735in}{1.163576in}}%
\pgfpathlineto{\pgfqpoint{4.650735in}{1.166525in}}%
\pgfpathlineto{\pgfqpoint{4.655276in}{1.166525in}}%
\pgfpathlineto{\pgfqpoint{4.655276in}{1.163576in}}%
\pgfpathmoveto{\pgfqpoint{4.646194in}{1.169474in}}%
\pgfpathlineto{\pgfqpoint{4.646194in}{1.169474in}}%
\pgfpathlineto{\pgfqpoint{4.646194in}{1.172423in}}%
\pgfpathlineto{\pgfqpoint{4.650735in}{1.172423in}}%
\pgfpathlineto{\pgfqpoint{4.650735in}{1.169474in}}%
\pgfpathmoveto{\pgfqpoint{4.650735in}{1.166525in}}%
\pgfpathlineto{\pgfqpoint{4.650735in}{1.166525in}}%
\pgfpathlineto{\pgfqpoint{4.650735in}{1.169474in}}%
\pgfpathlineto{\pgfqpoint{4.655276in}{1.169474in}}%
\pgfpathlineto{\pgfqpoint{4.655276in}{1.166525in}}%
\pgfpathmoveto{\pgfqpoint{4.650735in}{1.169474in}}%
\pgfpathlineto{\pgfqpoint{4.650735in}{1.169474in}}%
\pgfpathlineto{\pgfqpoint{4.650735in}{1.172423in}}%
\pgfpathlineto{\pgfqpoint{4.655276in}{1.172423in}}%
\pgfpathlineto{\pgfqpoint{4.655276in}{1.169474in}}%
\pgfpathmoveto{\pgfqpoint{4.641653in}{1.175372in}}%
\pgfpathlineto{\pgfqpoint{4.641653in}{1.175372in}}%
\pgfpathlineto{\pgfqpoint{4.641653in}{1.178322in}}%
\pgfpathlineto{\pgfqpoint{4.646194in}{1.178322in}}%
\pgfpathlineto{\pgfqpoint{4.646194in}{1.175372in}}%
\pgfpathmoveto{\pgfqpoint{4.637111in}{1.181271in}}%
\pgfpathlineto{\pgfqpoint{4.637111in}{1.181271in}}%
\pgfpathlineto{\pgfqpoint{4.637111in}{1.184220in}}%
\pgfpathlineto{\pgfqpoint{4.641653in}{1.184220in}}%
\pgfpathlineto{\pgfqpoint{4.641653in}{1.181271in}}%
\pgfpathmoveto{\pgfqpoint{4.641653in}{1.178322in}}%
\pgfpathlineto{\pgfqpoint{4.641653in}{1.178322in}}%
\pgfpathlineto{\pgfqpoint{4.641653in}{1.181271in}}%
\pgfpathlineto{\pgfqpoint{4.646194in}{1.181271in}}%
\pgfpathlineto{\pgfqpoint{4.646194in}{1.178322in}}%
\pgfpathmoveto{\pgfqpoint{4.641653in}{1.181271in}}%
\pgfpathlineto{\pgfqpoint{4.641653in}{1.181271in}}%
\pgfpathlineto{\pgfqpoint{4.641653in}{1.184220in}}%
\pgfpathlineto{\pgfqpoint{4.646194in}{1.184220in}}%
\pgfpathlineto{\pgfqpoint{4.646194in}{1.181271in}}%
\pgfpathmoveto{\pgfqpoint{4.646194in}{1.172423in}}%
\pgfpathlineto{\pgfqpoint{4.646194in}{1.172423in}}%
\pgfpathlineto{\pgfqpoint{4.646194in}{1.175372in}}%
\pgfpathlineto{\pgfqpoint{4.650735in}{1.175372in}}%
\pgfpathlineto{\pgfqpoint{4.650735in}{1.172423in}}%
\pgfpathmoveto{\pgfqpoint{4.646194in}{1.175372in}}%
\pgfpathlineto{\pgfqpoint{4.646194in}{1.175372in}}%
\pgfpathlineto{\pgfqpoint{4.646194in}{1.178322in}}%
\pgfpathlineto{\pgfqpoint{4.650735in}{1.178322in}}%
\pgfpathlineto{\pgfqpoint{4.650735in}{1.175372in}}%
\pgfpathmoveto{\pgfqpoint{4.655276in}{1.160626in}}%
\pgfpathlineto{\pgfqpoint{4.655276in}{1.160626in}}%
\pgfpathlineto{\pgfqpoint{4.655276in}{1.163576in}}%
\pgfpathlineto{\pgfqpoint{4.659817in}{1.163576in}}%
\pgfpathlineto{\pgfqpoint{4.659817in}{1.160626in}}%
\pgfpathmoveto{\pgfqpoint{4.655276in}{1.163576in}}%
\pgfpathlineto{\pgfqpoint{4.655276in}{1.163576in}}%
\pgfpathlineto{\pgfqpoint{4.655276in}{1.166525in}}%
\pgfpathlineto{\pgfqpoint{4.659817in}{1.166525in}}%
\pgfpathlineto{\pgfqpoint{4.659817in}{1.163576in}}%
\pgfpathmoveto{\pgfqpoint{4.637111in}{1.184220in}}%
\pgfpathlineto{\pgfqpoint{4.637111in}{1.184220in}}%
\pgfpathlineto{\pgfqpoint{4.637111in}{1.187169in}}%
\pgfpathlineto{\pgfqpoint{4.641653in}{1.187169in}}%
\pgfpathlineto{\pgfqpoint{4.641653in}{1.184220in}}%
\pgfpathmoveto{\pgfqpoint{4.637111in}{1.187169in}}%
\pgfpathlineto{\pgfqpoint{4.637111in}{1.187169in}}%
\pgfpathlineto{\pgfqpoint{4.637111in}{1.190118in}}%
\pgfpathlineto{\pgfqpoint{4.641653in}{1.190118in}}%
\pgfpathlineto{\pgfqpoint{4.641653in}{1.187169in}}%
\pgfpathmoveto{\pgfqpoint{4.614406in}{1.210763in}}%
\pgfpathlineto{\pgfqpoint{4.614406in}{1.210763in}}%
\pgfpathlineto{\pgfqpoint{4.614406in}{1.213712in}}%
\pgfpathlineto{\pgfqpoint{4.618947in}{1.213712in}}%
\pgfpathlineto{\pgfqpoint{4.618947in}{1.210763in}}%
\pgfpathmoveto{\pgfqpoint{4.609865in}{1.216661in}}%
\pgfpathlineto{\pgfqpoint{4.609865in}{1.216661in}}%
\pgfpathlineto{\pgfqpoint{4.609865in}{1.219610in}}%
\pgfpathlineto{\pgfqpoint{4.614406in}{1.219610in}}%
\pgfpathlineto{\pgfqpoint{4.614406in}{1.216661in}}%
\pgfpathmoveto{\pgfqpoint{4.614406in}{1.213712in}}%
\pgfpathlineto{\pgfqpoint{4.614406in}{1.213712in}}%
\pgfpathlineto{\pgfqpoint{4.614406in}{1.216661in}}%
\pgfpathlineto{\pgfqpoint{4.618947in}{1.216661in}}%
\pgfpathlineto{\pgfqpoint{4.618947in}{1.213712in}}%
\pgfpathmoveto{\pgfqpoint{4.614406in}{1.216661in}}%
\pgfpathlineto{\pgfqpoint{4.614406in}{1.216661in}}%
\pgfpathlineto{\pgfqpoint{4.614406in}{1.219610in}}%
\pgfpathlineto{\pgfqpoint{4.618947in}{1.219610in}}%
\pgfpathlineto{\pgfqpoint{4.618947in}{1.216661in}}%
\pgfpathmoveto{\pgfqpoint{4.605323in}{1.222560in}}%
\pgfpathlineto{\pgfqpoint{4.605323in}{1.222560in}}%
\pgfpathlineto{\pgfqpoint{4.605323in}{1.225509in}}%
\pgfpathlineto{\pgfqpoint{4.609865in}{1.225509in}}%
\pgfpathlineto{\pgfqpoint{4.609865in}{1.222560in}}%
\pgfpathmoveto{\pgfqpoint{4.600782in}{1.228458in}}%
\pgfpathlineto{\pgfqpoint{4.600782in}{1.228458in}}%
\pgfpathlineto{\pgfqpoint{4.600782in}{1.231407in}}%
\pgfpathlineto{\pgfqpoint{4.605323in}{1.231407in}}%
\pgfpathlineto{\pgfqpoint{4.605323in}{1.228458in}}%
\pgfpathmoveto{\pgfqpoint{4.605323in}{1.225509in}}%
\pgfpathlineto{\pgfqpoint{4.605323in}{1.225509in}}%
\pgfpathlineto{\pgfqpoint{4.605323in}{1.228458in}}%
\pgfpathlineto{\pgfqpoint{4.609865in}{1.228458in}}%
\pgfpathlineto{\pgfqpoint{4.609865in}{1.225509in}}%
\pgfpathmoveto{\pgfqpoint{4.605323in}{1.228458in}}%
\pgfpathlineto{\pgfqpoint{4.605323in}{1.228458in}}%
\pgfpathlineto{\pgfqpoint{4.605323in}{1.231407in}}%
\pgfpathlineto{\pgfqpoint{4.609865in}{1.231407in}}%
\pgfpathlineto{\pgfqpoint{4.609865in}{1.228458in}}%
\pgfpathmoveto{\pgfqpoint{4.609865in}{1.219610in}}%
\pgfpathlineto{\pgfqpoint{4.609865in}{1.219610in}}%
\pgfpathlineto{\pgfqpoint{4.609865in}{1.222560in}}%
\pgfpathlineto{\pgfqpoint{4.614406in}{1.222560in}}%
\pgfpathlineto{\pgfqpoint{4.614406in}{1.219610in}}%
\pgfpathmoveto{\pgfqpoint{4.609865in}{1.222560in}}%
\pgfpathlineto{\pgfqpoint{4.609865in}{1.222560in}}%
\pgfpathlineto{\pgfqpoint{4.609865in}{1.225509in}}%
\pgfpathlineto{\pgfqpoint{4.614406in}{1.225509in}}%
\pgfpathlineto{\pgfqpoint{4.614406in}{1.222560in}}%
\pgfpathmoveto{\pgfqpoint{4.618947in}{1.207814in}}%
\pgfpathlineto{\pgfqpoint{4.618947in}{1.207814in}}%
\pgfpathlineto{\pgfqpoint{4.618947in}{1.210763in}}%
\pgfpathlineto{\pgfqpoint{4.623488in}{1.210763in}}%
\pgfpathlineto{\pgfqpoint{4.623488in}{1.207814in}}%
\pgfpathmoveto{\pgfqpoint{4.618947in}{1.210763in}}%
\pgfpathlineto{\pgfqpoint{4.618947in}{1.210763in}}%
\pgfpathlineto{\pgfqpoint{4.618947in}{1.213712in}}%
\pgfpathlineto{\pgfqpoint{4.623488in}{1.213712in}}%
\pgfpathlineto{\pgfqpoint{4.623488in}{1.210763in}}%
\pgfpathmoveto{\pgfqpoint{4.600782in}{1.231407in}}%
\pgfpathlineto{\pgfqpoint{4.600782in}{1.231407in}}%
\pgfpathlineto{\pgfqpoint{4.600782in}{1.234356in}}%
\pgfpathlineto{\pgfqpoint{4.605323in}{1.234356in}}%
\pgfpathlineto{\pgfqpoint{4.605323in}{1.231407in}}%
\pgfpathmoveto{\pgfqpoint{4.600782in}{1.234356in}}%
\pgfpathlineto{\pgfqpoint{4.600782in}{1.234356in}}%
\pgfpathlineto{\pgfqpoint{4.600782in}{1.237306in}}%
\pgfpathlineto{\pgfqpoint{4.605323in}{1.237306in}}%
\pgfpathlineto{\pgfqpoint{4.605323in}{1.234356in}}%
\pgfpathmoveto{\pgfqpoint{4.559912in}{1.281543in}}%
\pgfpathlineto{\pgfqpoint{4.559912in}{1.281543in}}%
\pgfpathlineto{\pgfqpoint{4.559912in}{1.284492in}}%
\pgfpathlineto{\pgfqpoint{4.564453in}{1.284492in}}%
\pgfpathlineto{\pgfqpoint{4.564453in}{1.281543in}}%
\pgfpathmoveto{\pgfqpoint{4.555371in}{1.287441in}}%
\pgfpathlineto{\pgfqpoint{4.555371in}{1.287441in}}%
\pgfpathlineto{\pgfqpoint{4.555371in}{1.290390in}}%
\pgfpathlineto{\pgfqpoint{4.559912in}{1.290390in}}%
\pgfpathlineto{\pgfqpoint{4.559912in}{1.287441in}}%
\pgfpathmoveto{\pgfqpoint{4.559912in}{1.284492in}}%
\pgfpathlineto{\pgfqpoint{4.559912in}{1.284492in}}%
\pgfpathlineto{\pgfqpoint{4.559912in}{1.287441in}}%
\pgfpathlineto{\pgfqpoint{4.564453in}{1.287441in}}%
\pgfpathlineto{\pgfqpoint{4.564453in}{1.284492in}}%
\pgfpathmoveto{\pgfqpoint{4.559912in}{1.287441in}}%
\pgfpathlineto{\pgfqpoint{4.559912in}{1.287441in}}%
\pgfpathlineto{\pgfqpoint{4.559912in}{1.290390in}}%
\pgfpathlineto{\pgfqpoint{4.564453in}{1.290390in}}%
\pgfpathlineto{\pgfqpoint{4.564453in}{1.287441in}}%
\pgfpathmoveto{\pgfqpoint{4.550830in}{1.293339in}}%
\pgfpathlineto{\pgfqpoint{4.550830in}{1.293339in}}%
\pgfpathlineto{\pgfqpoint{4.550830in}{1.296288in}}%
\pgfpathlineto{\pgfqpoint{4.555371in}{1.296288in}}%
\pgfpathlineto{\pgfqpoint{4.555371in}{1.293339in}}%
\pgfpathmoveto{\pgfqpoint{4.546289in}{1.299237in}}%
\pgfpathlineto{\pgfqpoint{4.546289in}{1.299237in}}%
\pgfpathlineto{\pgfqpoint{4.546289in}{1.302187in}}%
\pgfpathlineto{\pgfqpoint{4.550830in}{1.302187in}}%
\pgfpathlineto{\pgfqpoint{4.550830in}{1.299237in}}%
\pgfpathmoveto{\pgfqpoint{4.550830in}{1.296288in}}%
\pgfpathlineto{\pgfqpoint{4.550830in}{1.296288in}}%
\pgfpathlineto{\pgfqpoint{4.550830in}{1.299237in}}%
\pgfpathlineto{\pgfqpoint{4.555371in}{1.299237in}}%
\pgfpathlineto{\pgfqpoint{4.555371in}{1.296288in}}%
\pgfpathmoveto{\pgfqpoint{4.550830in}{1.299237in}}%
\pgfpathlineto{\pgfqpoint{4.550830in}{1.299237in}}%
\pgfpathlineto{\pgfqpoint{4.550830in}{1.302187in}}%
\pgfpathlineto{\pgfqpoint{4.555371in}{1.302187in}}%
\pgfpathlineto{\pgfqpoint{4.555371in}{1.299237in}}%
\pgfpathmoveto{\pgfqpoint{4.555371in}{1.290390in}}%
\pgfpathlineto{\pgfqpoint{4.555371in}{1.290390in}}%
\pgfpathlineto{\pgfqpoint{4.555371in}{1.293339in}}%
\pgfpathlineto{\pgfqpoint{4.559912in}{1.293339in}}%
\pgfpathlineto{\pgfqpoint{4.559912in}{1.290390in}}%
\pgfpathmoveto{\pgfqpoint{4.555371in}{1.293339in}}%
\pgfpathlineto{\pgfqpoint{4.555371in}{1.293339in}}%
\pgfpathlineto{\pgfqpoint{4.555371in}{1.296288in}}%
\pgfpathlineto{\pgfqpoint{4.559912in}{1.296288in}}%
\pgfpathlineto{\pgfqpoint{4.559912in}{1.293339in}}%
\pgfpathmoveto{\pgfqpoint{4.578077in}{1.257950in}}%
\pgfpathlineto{\pgfqpoint{4.578077in}{1.257950in}}%
\pgfpathlineto{\pgfqpoint{4.578077in}{1.260899in}}%
\pgfpathlineto{\pgfqpoint{4.582618in}{1.260899in}}%
\pgfpathlineto{\pgfqpoint{4.582618in}{1.257950in}}%
\pgfpathmoveto{\pgfqpoint{4.573536in}{1.263848in}}%
\pgfpathlineto{\pgfqpoint{4.573536in}{1.263848in}}%
\pgfpathlineto{\pgfqpoint{4.573536in}{1.266797in}}%
\pgfpathlineto{\pgfqpoint{4.578077in}{1.266797in}}%
\pgfpathlineto{\pgfqpoint{4.578077in}{1.263848in}}%
\pgfpathmoveto{\pgfqpoint{4.578077in}{1.260899in}}%
\pgfpathlineto{\pgfqpoint{4.578077in}{1.260899in}}%
\pgfpathlineto{\pgfqpoint{4.578077in}{1.263848in}}%
\pgfpathlineto{\pgfqpoint{4.582618in}{1.263848in}}%
\pgfpathlineto{\pgfqpoint{4.582618in}{1.260899in}}%
\pgfpathmoveto{\pgfqpoint{4.578077in}{1.263848in}}%
\pgfpathlineto{\pgfqpoint{4.578077in}{1.263848in}}%
\pgfpathlineto{\pgfqpoint{4.578077in}{1.266797in}}%
\pgfpathlineto{\pgfqpoint{4.582618in}{1.266797in}}%
\pgfpathlineto{\pgfqpoint{4.582618in}{1.263848in}}%
\pgfpathmoveto{\pgfqpoint{4.568994in}{1.269746in}}%
\pgfpathlineto{\pgfqpoint{4.568994in}{1.269746in}}%
\pgfpathlineto{\pgfqpoint{4.568994in}{1.272695in}}%
\pgfpathlineto{\pgfqpoint{4.573536in}{1.272695in}}%
\pgfpathlineto{\pgfqpoint{4.573536in}{1.269746in}}%
\pgfpathmoveto{\pgfqpoint{4.564453in}{1.275645in}}%
\pgfpathlineto{\pgfqpoint{4.564453in}{1.275645in}}%
\pgfpathlineto{\pgfqpoint{4.564453in}{1.278594in}}%
\pgfpathlineto{\pgfqpoint{4.568994in}{1.278594in}}%
\pgfpathlineto{\pgfqpoint{4.568994in}{1.275645in}}%
\pgfpathmoveto{\pgfqpoint{4.568994in}{1.272695in}}%
\pgfpathlineto{\pgfqpoint{4.568994in}{1.272695in}}%
\pgfpathlineto{\pgfqpoint{4.568994in}{1.275645in}}%
\pgfpathlineto{\pgfqpoint{4.573536in}{1.275645in}}%
\pgfpathlineto{\pgfqpoint{4.573536in}{1.272695in}}%
\pgfpathmoveto{\pgfqpoint{4.568994in}{1.275645in}}%
\pgfpathlineto{\pgfqpoint{4.568994in}{1.275645in}}%
\pgfpathlineto{\pgfqpoint{4.568994in}{1.278594in}}%
\pgfpathlineto{\pgfqpoint{4.573536in}{1.278594in}}%
\pgfpathlineto{\pgfqpoint{4.573536in}{1.275645in}}%
\pgfpathmoveto{\pgfqpoint{4.573536in}{1.266797in}}%
\pgfpathlineto{\pgfqpoint{4.573536in}{1.266797in}}%
\pgfpathlineto{\pgfqpoint{4.573536in}{1.269746in}}%
\pgfpathlineto{\pgfqpoint{4.578077in}{1.269746in}}%
\pgfpathlineto{\pgfqpoint{4.578077in}{1.266797in}}%
\pgfpathmoveto{\pgfqpoint{4.573536in}{1.269746in}}%
\pgfpathlineto{\pgfqpoint{4.573536in}{1.269746in}}%
\pgfpathlineto{\pgfqpoint{4.573536in}{1.272695in}}%
\pgfpathlineto{\pgfqpoint{4.578077in}{1.272695in}}%
\pgfpathlineto{\pgfqpoint{4.578077in}{1.269746in}}%
\pgfpathmoveto{\pgfqpoint{4.582618in}{1.255001in}}%
\pgfpathlineto{\pgfqpoint{4.582618in}{1.255001in}}%
\pgfpathlineto{\pgfqpoint{4.582618in}{1.257950in}}%
\pgfpathlineto{\pgfqpoint{4.587159in}{1.257950in}}%
\pgfpathlineto{\pgfqpoint{4.587159in}{1.255001in}}%
\pgfpathmoveto{\pgfqpoint{4.582618in}{1.257950in}}%
\pgfpathlineto{\pgfqpoint{4.582618in}{1.257950in}}%
\pgfpathlineto{\pgfqpoint{4.582618in}{1.260899in}}%
\pgfpathlineto{\pgfqpoint{4.587159in}{1.260899in}}%
\pgfpathlineto{\pgfqpoint{4.587159in}{1.257950in}}%
\pgfpathmoveto{\pgfqpoint{4.564453in}{1.278594in}}%
\pgfpathlineto{\pgfqpoint{4.564453in}{1.278594in}}%
\pgfpathlineto{\pgfqpoint{4.564453in}{1.281543in}}%
\pgfpathlineto{\pgfqpoint{4.568994in}{1.281543in}}%
\pgfpathlineto{\pgfqpoint{4.568994in}{1.278594in}}%
\pgfpathmoveto{\pgfqpoint{4.564453in}{1.281543in}}%
\pgfpathlineto{\pgfqpoint{4.564453in}{1.281543in}}%
\pgfpathlineto{\pgfqpoint{4.564453in}{1.284492in}}%
\pgfpathlineto{\pgfqpoint{4.568994in}{1.284492in}}%
\pgfpathlineto{\pgfqpoint{4.568994in}{1.281543in}}%
\pgfpathmoveto{\pgfqpoint{4.541748in}{1.305136in}}%
\pgfpathlineto{\pgfqpoint{4.541748in}{1.305136in}}%
\pgfpathlineto{\pgfqpoint{4.541748in}{1.308085in}}%
\pgfpathlineto{\pgfqpoint{4.546289in}{1.308085in}}%
\pgfpathlineto{\pgfqpoint{4.546289in}{1.305136in}}%
\pgfpathmoveto{\pgfqpoint{4.537206in}{1.311034in}}%
\pgfpathlineto{\pgfqpoint{4.537206in}{1.311034in}}%
\pgfpathlineto{\pgfqpoint{4.537206in}{1.313983in}}%
\pgfpathlineto{\pgfqpoint{4.541748in}{1.313983in}}%
\pgfpathlineto{\pgfqpoint{4.541748in}{1.311034in}}%
\pgfpathmoveto{\pgfqpoint{4.541748in}{1.308085in}}%
\pgfpathlineto{\pgfqpoint{4.541748in}{1.308085in}}%
\pgfpathlineto{\pgfqpoint{4.541748in}{1.311034in}}%
\pgfpathlineto{\pgfqpoint{4.546289in}{1.311034in}}%
\pgfpathlineto{\pgfqpoint{4.546289in}{1.308085in}}%
\pgfpathmoveto{\pgfqpoint{4.541748in}{1.311034in}}%
\pgfpathlineto{\pgfqpoint{4.541748in}{1.311034in}}%
\pgfpathlineto{\pgfqpoint{4.541748in}{1.313983in}}%
\pgfpathlineto{\pgfqpoint{4.546289in}{1.313983in}}%
\pgfpathlineto{\pgfqpoint{4.546289in}{1.311034in}}%
\pgfpathmoveto{\pgfqpoint{4.532665in}{1.316932in}}%
\pgfpathlineto{\pgfqpoint{4.532665in}{1.316932in}}%
\pgfpathlineto{\pgfqpoint{4.532665in}{1.319881in}}%
\pgfpathlineto{\pgfqpoint{4.537206in}{1.319881in}}%
\pgfpathlineto{\pgfqpoint{4.537206in}{1.316932in}}%
\pgfpathmoveto{\pgfqpoint{4.528124in}{1.322830in}}%
\pgfpathlineto{\pgfqpoint{4.528124in}{1.322830in}}%
\pgfpathlineto{\pgfqpoint{4.528124in}{1.325780in}}%
\pgfpathlineto{\pgfqpoint{4.532665in}{1.325780in}}%
\pgfpathlineto{\pgfqpoint{4.532665in}{1.322830in}}%
\pgfpathmoveto{\pgfqpoint{4.532665in}{1.319881in}}%
\pgfpathlineto{\pgfqpoint{4.532665in}{1.319881in}}%
\pgfpathlineto{\pgfqpoint{4.532665in}{1.322830in}}%
\pgfpathlineto{\pgfqpoint{4.537206in}{1.322830in}}%
\pgfpathlineto{\pgfqpoint{4.537206in}{1.319881in}}%
\pgfpathmoveto{\pgfqpoint{4.532665in}{1.322830in}}%
\pgfpathlineto{\pgfqpoint{4.532665in}{1.322830in}}%
\pgfpathlineto{\pgfqpoint{4.532665in}{1.325780in}}%
\pgfpathlineto{\pgfqpoint{4.537206in}{1.325780in}}%
\pgfpathlineto{\pgfqpoint{4.537206in}{1.322830in}}%
\pgfpathmoveto{\pgfqpoint{4.537206in}{1.313983in}}%
\pgfpathlineto{\pgfqpoint{4.537206in}{1.313983in}}%
\pgfpathlineto{\pgfqpoint{4.537206in}{1.316932in}}%
\pgfpathlineto{\pgfqpoint{4.541748in}{1.316932in}}%
\pgfpathlineto{\pgfqpoint{4.541748in}{1.313983in}}%
\pgfpathmoveto{\pgfqpoint{4.537206in}{1.316932in}}%
\pgfpathlineto{\pgfqpoint{4.537206in}{1.316932in}}%
\pgfpathlineto{\pgfqpoint{4.537206in}{1.319881in}}%
\pgfpathlineto{\pgfqpoint{4.541748in}{1.319881in}}%
\pgfpathlineto{\pgfqpoint{4.541748in}{1.316932in}}%
\pgfpathmoveto{\pgfqpoint{4.546289in}{1.302187in}}%
\pgfpathlineto{\pgfqpoint{4.546289in}{1.302187in}}%
\pgfpathlineto{\pgfqpoint{4.546289in}{1.305136in}}%
\pgfpathlineto{\pgfqpoint{4.550830in}{1.305136in}}%
\pgfpathlineto{\pgfqpoint{4.550830in}{1.302187in}}%
\pgfpathmoveto{\pgfqpoint{4.546289in}{1.305136in}}%
\pgfpathlineto{\pgfqpoint{4.546289in}{1.305136in}}%
\pgfpathlineto{\pgfqpoint{4.546289in}{1.308085in}}%
\pgfpathlineto{\pgfqpoint{4.550830in}{1.308085in}}%
\pgfpathlineto{\pgfqpoint{4.550830in}{1.305136in}}%
\pgfpathmoveto{\pgfqpoint{4.528124in}{1.325780in}}%
\pgfpathlineto{\pgfqpoint{4.528124in}{1.325780in}}%
\pgfpathlineto{\pgfqpoint{4.528124in}{1.328729in}}%
\pgfpathlineto{\pgfqpoint{4.532665in}{1.328729in}}%
\pgfpathlineto{\pgfqpoint{4.532665in}{1.325780in}}%
\pgfpathmoveto{\pgfqpoint{4.528124in}{1.328729in}}%
\pgfpathlineto{\pgfqpoint{4.528124in}{1.328729in}}%
\pgfpathlineto{\pgfqpoint{4.528124in}{1.331678in}}%
\pgfpathlineto{\pgfqpoint{4.532665in}{1.331678in}}%
\pgfpathlineto{\pgfqpoint{4.532665in}{1.328729in}}%
\pgfpathmoveto{\pgfqpoint{4.814211in}{0.951232in}}%
\pgfpathlineto{\pgfqpoint{4.814211in}{0.951232in}}%
\pgfpathlineto{\pgfqpoint{4.814211in}{0.954181in}}%
\pgfpathlineto{\pgfqpoint{4.818752in}{0.954181in}}%
\pgfpathlineto{\pgfqpoint{4.818752in}{0.951232in}}%
\pgfpathmoveto{\pgfqpoint{4.809670in}{0.957130in}}%
\pgfpathlineto{\pgfqpoint{4.809670in}{0.957130in}}%
\pgfpathlineto{\pgfqpoint{4.809670in}{0.960079in}}%
\pgfpathlineto{\pgfqpoint{4.814211in}{0.960079in}}%
\pgfpathlineto{\pgfqpoint{4.814211in}{0.957130in}}%
\pgfpathmoveto{\pgfqpoint{4.814211in}{0.954181in}}%
\pgfpathlineto{\pgfqpoint{4.814211in}{0.954181in}}%
\pgfpathlineto{\pgfqpoint{4.814211in}{0.957130in}}%
\pgfpathlineto{\pgfqpoint{4.818752in}{0.957130in}}%
\pgfpathlineto{\pgfqpoint{4.818752in}{0.954181in}}%
\pgfpathmoveto{\pgfqpoint{4.814211in}{0.957130in}}%
\pgfpathlineto{\pgfqpoint{4.814211in}{0.957130in}}%
\pgfpathlineto{\pgfqpoint{4.814211in}{0.960079in}}%
\pgfpathlineto{\pgfqpoint{4.818752in}{0.960079in}}%
\pgfpathlineto{\pgfqpoint{4.818752in}{0.957130in}}%
\pgfpathmoveto{\pgfqpoint{4.805129in}{0.963028in}}%
\pgfpathlineto{\pgfqpoint{4.805129in}{0.963028in}}%
\pgfpathlineto{\pgfqpoint{4.805129in}{0.965977in}}%
\pgfpathlineto{\pgfqpoint{4.809670in}{0.965977in}}%
\pgfpathlineto{\pgfqpoint{4.809670in}{0.963028in}}%
\pgfpathmoveto{\pgfqpoint{4.800588in}{0.968927in}}%
\pgfpathlineto{\pgfqpoint{4.800588in}{0.968927in}}%
\pgfpathlineto{\pgfqpoint{4.800588in}{0.971876in}}%
\pgfpathlineto{\pgfqpoint{4.805129in}{0.971876in}}%
\pgfpathlineto{\pgfqpoint{4.805129in}{0.968927in}}%
\pgfpathmoveto{\pgfqpoint{4.805129in}{0.965977in}}%
\pgfpathlineto{\pgfqpoint{4.805129in}{0.965977in}}%
\pgfpathlineto{\pgfqpoint{4.805129in}{0.968927in}}%
\pgfpathlineto{\pgfqpoint{4.809670in}{0.968927in}}%
\pgfpathlineto{\pgfqpoint{4.809670in}{0.965977in}}%
\pgfpathmoveto{\pgfqpoint{4.805129in}{0.968927in}}%
\pgfpathlineto{\pgfqpoint{4.805129in}{0.968927in}}%
\pgfpathlineto{\pgfqpoint{4.805129in}{0.971876in}}%
\pgfpathlineto{\pgfqpoint{4.809670in}{0.971876in}}%
\pgfpathlineto{\pgfqpoint{4.809670in}{0.968927in}}%
\pgfpathmoveto{\pgfqpoint{4.809670in}{0.960079in}}%
\pgfpathlineto{\pgfqpoint{4.809670in}{0.960079in}}%
\pgfpathlineto{\pgfqpoint{4.809670in}{0.963028in}}%
\pgfpathlineto{\pgfqpoint{4.814211in}{0.963028in}}%
\pgfpathlineto{\pgfqpoint{4.814211in}{0.960079in}}%
\pgfpathmoveto{\pgfqpoint{4.809670in}{0.963028in}}%
\pgfpathlineto{\pgfqpoint{4.809670in}{0.963028in}}%
\pgfpathlineto{\pgfqpoint{4.809670in}{0.965977in}}%
\pgfpathlineto{\pgfqpoint{4.814211in}{0.965977in}}%
\pgfpathlineto{\pgfqpoint{4.814211in}{0.963028in}}%
\pgfpathmoveto{\pgfqpoint{4.741555in}{1.045606in}}%
\pgfpathlineto{\pgfqpoint{4.741555in}{1.045606in}}%
\pgfpathlineto{\pgfqpoint{4.741555in}{1.048555in}}%
\pgfpathlineto{\pgfqpoint{4.746096in}{1.048555in}}%
\pgfpathlineto{\pgfqpoint{4.746096in}{1.045606in}}%
\pgfpathmoveto{\pgfqpoint{4.737014in}{1.051505in}}%
\pgfpathlineto{\pgfqpoint{4.737014in}{1.051505in}}%
\pgfpathlineto{\pgfqpoint{4.737014in}{1.054454in}}%
\pgfpathlineto{\pgfqpoint{4.741555in}{1.054454in}}%
\pgfpathlineto{\pgfqpoint{4.741555in}{1.051505in}}%
\pgfpathmoveto{\pgfqpoint{4.741555in}{1.048555in}}%
\pgfpathlineto{\pgfqpoint{4.741555in}{1.048555in}}%
\pgfpathlineto{\pgfqpoint{4.741555in}{1.051505in}}%
\pgfpathlineto{\pgfqpoint{4.746096in}{1.051505in}}%
\pgfpathlineto{\pgfqpoint{4.746096in}{1.048555in}}%
\pgfpathmoveto{\pgfqpoint{4.741555in}{1.051505in}}%
\pgfpathlineto{\pgfqpoint{4.741555in}{1.051505in}}%
\pgfpathlineto{\pgfqpoint{4.741555in}{1.054454in}}%
\pgfpathlineto{\pgfqpoint{4.746096in}{1.054454in}}%
\pgfpathlineto{\pgfqpoint{4.746096in}{1.051505in}}%
\pgfpathmoveto{\pgfqpoint{4.732473in}{1.057403in}}%
\pgfpathlineto{\pgfqpoint{4.732473in}{1.057403in}}%
\pgfpathlineto{\pgfqpoint{4.732473in}{1.060352in}}%
\pgfpathlineto{\pgfqpoint{4.737014in}{1.060352in}}%
\pgfpathlineto{\pgfqpoint{4.737014in}{1.057403in}}%
\pgfpathmoveto{\pgfqpoint{4.727932in}{1.063302in}}%
\pgfpathlineto{\pgfqpoint{4.727932in}{1.063302in}}%
\pgfpathlineto{\pgfqpoint{4.727932in}{1.066251in}}%
\pgfpathlineto{\pgfqpoint{4.732473in}{1.066251in}}%
\pgfpathlineto{\pgfqpoint{4.732473in}{1.063302in}}%
\pgfpathmoveto{\pgfqpoint{4.732473in}{1.060352in}}%
\pgfpathlineto{\pgfqpoint{4.732473in}{1.060352in}}%
\pgfpathlineto{\pgfqpoint{4.732473in}{1.063302in}}%
\pgfpathlineto{\pgfqpoint{4.737014in}{1.063302in}}%
\pgfpathlineto{\pgfqpoint{4.737014in}{1.060352in}}%
\pgfpathmoveto{\pgfqpoint{4.732473in}{1.063302in}}%
\pgfpathlineto{\pgfqpoint{4.732473in}{1.063302in}}%
\pgfpathlineto{\pgfqpoint{4.732473in}{1.066251in}}%
\pgfpathlineto{\pgfqpoint{4.737014in}{1.066251in}}%
\pgfpathlineto{\pgfqpoint{4.737014in}{1.063302in}}%
\pgfpathmoveto{\pgfqpoint{4.737014in}{1.054454in}}%
\pgfpathlineto{\pgfqpoint{4.737014in}{1.054454in}}%
\pgfpathlineto{\pgfqpoint{4.737014in}{1.057403in}}%
\pgfpathlineto{\pgfqpoint{4.741555in}{1.057403in}}%
\pgfpathlineto{\pgfqpoint{4.741555in}{1.054454in}}%
\pgfpathmoveto{\pgfqpoint{4.737014in}{1.057403in}}%
\pgfpathlineto{\pgfqpoint{4.737014in}{1.057403in}}%
\pgfpathlineto{\pgfqpoint{4.737014in}{1.060352in}}%
\pgfpathlineto{\pgfqpoint{4.741555in}{1.060352in}}%
\pgfpathlineto{\pgfqpoint{4.741555in}{1.057403in}}%
\pgfpathmoveto{\pgfqpoint{4.777883in}{0.998419in}}%
\pgfpathlineto{\pgfqpoint{4.777883in}{0.998419in}}%
\pgfpathlineto{\pgfqpoint{4.777883in}{1.001368in}}%
\pgfpathlineto{\pgfqpoint{4.782424in}{1.001368in}}%
\pgfpathlineto{\pgfqpoint{4.782424in}{0.998419in}}%
\pgfpathmoveto{\pgfqpoint{4.773342in}{1.004317in}}%
\pgfpathlineto{\pgfqpoint{4.773342in}{1.004317in}}%
\pgfpathlineto{\pgfqpoint{4.773342in}{1.007266in}}%
\pgfpathlineto{\pgfqpoint{4.777883in}{1.007266in}}%
\pgfpathlineto{\pgfqpoint{4.777883in}{1.004317in}}%
\pgfpathmoveto{\pgfqpoint{4.777883in}{1.001368in}}%
\pgfpathlineto{\pgfqpoint{4.777883in}{1.001368in}}%
\pgfpathlineto{\pgfqpoint{4.777883in}{1.004317in}}%
\pgfpathlineto{\pgfqpoint{4.782424in}{1.004317in}}%
\pgfpathlineto{\pgfqpoint{4.782424in}{1.001368in}}%
\pgfpathmoveto{\pgfqpoint{4.777883in}{1.004317in}}%
\pgfpathlineto{\pgfqpoint{4.777883in}{1.004317in}}%
\pgfpathlineto{\pgfqpoint{4.777883in}{1.007266in}}%
\pgfpathlineto{\pgfqpoint{4.782424in}{1.007266in}}%
\pgfpathlineto{\pgfqpoint{4.782424in}{1.004317in}}%
\pgfpathmoveto{\pgfqpoint{4.768801in}{1.010216in}}%
\pgfpathlineto{\pgfqpoint{4.768801in}{1.010216in}}%
\pgfpathlineto{\pgfqpoint{4.768801in}{1.013165in}}%
\pgfpathlineto{\pgfqpoint{4.773342in}{1.013165in}}%
\pgfpathlineto{\pgfqpoint{4.773342in}{1.010216in}}%
\pgfpathmoveto{\pgfqpoint{4.764260in}{1.016114in}}%
\pgfpathlineto{\pgfqpoint{4.764260in}{1.016114in}}%
\pgfpathlineto{\pgfqpoint{4.764260in}{1.019063in}}%
\pgfpathlineto{\pgfqpoint{4.768801in}{1.019063in}}%
\pgfpathlineto{\pgfqpoint{4.768801in}{1.016114in}}%
\pgfpathmoveto{\pgfqpoint{4.768801in}{1.013165in}}%
\pgfpathlineto{\pgfqpoint{4.768801in}{1.013165in}}%
\pgfpathlineto{\pgfqpoint{4.768801in}{1.016114in}}%
\pgfpathlineto{\pgfqpoint{4.773342in}{1.016114in}}%
\pgfpathlineto{\pgfqpoint{4.773342in}{1.013165in}}%
\pgfpathmoveto{\pgfqpoint{4.768801in}{1.016114in}}%
\pgfpathlineto{\pgfqpoint{4.768801in}{1.016114in}}%
\pgfpathlineto{\pgfqpoint{4.768801in}{1.019063in}}%
\pgfpathlineto{\pgfqpoint{4.773342in}{1.019063in}}%
\pgfpathlineto{\pgfqpoint{4.773342in}{1.016114in}}%
\pgfpathmoveto{\pgfqpoint{4.773342in}{1.007266in}}%
\pgfpathlineto{\pgfqpoint{4.773342in}{1.007266in}}%
\pgfpathlineto{\pgfqpoint{4.773342in}{1.010216in}}%
\pgfpathlineto{\pgfqpoint{4.777883in}{1.010216in}}%
\pgfpathlineto{\pgfqpoint{4.777883in}{1.007266in}}%
\pgfpathmoveto{\pgfqpoint{4.773342in}{1.010216in}}%
\pgfpathlineto{\pgfqpoint{4.773342in}{1.010216in}}%
\pgfpathlineto{\pgfqpoint{4.773342in}{1.013165in}}%
\pgfpathlineto{\pgfqpoint{4.777883in}{1.013165in}}%
\pgfpathlineto{\pgfqpoint{4.777883in}{1.010216in}}%
\pgfpathmoveto{\pgfqpoint{4.796047in}{0.974825in}}%
\pgfpathlineto{\pgfqpoint{4.796047in}{0.974825in}}%
\pgfpathlineto{\pgfqpoint{4.796047in}{0.977774in}}%
\pgfpathlineto{\pgfqpoint{4.800588in}{0.977774in}}%
\pgfpathlineto{\pgfqpoint{4.800588in}{0.974825in}}%
\pgfpathmoveto{\pgfqpoint{4.791506in}{0.980723in}}%
\pgfpathlineto{\pgfqpoint{4.791506in}{0.980723in}}%
\pgfpathlineto{\pgfqpoint{4.791506in}{0.983673in}}%
\pgfpathlineto{\pgfqpoint{4.796047in}{0.983673in}}%
\pgfpathlineto{\pgfqpoint{4.796047in}{0.980723in}}%
\pgfpathmoveto{\pgfqpoint{4.796047in}{0.977774in}}%
\pgfpathlineto{\pgfqpoint{4.796047in}{0.977774in}}%
\pgfpathlineto{\pgfqpoint{4.796047in}{0.980723in}}%
\pgfpathlineto{\pgfqpoint{4.800588in}{0.980723in}}%
\pgfpathlineto{\pgfqpoint{4.800588in}{0.977774in}}%
\pgfpathmoveto{\pgfqpoint{4.796047in}{0.980723in}}%
\pgfpathlineto{\pgfqpoint{4.796047in}{0.980723in}}%
\pgfpathlineto{\pgfqpoint{4.796047in}{0.983673in}}%
\pgfpathlineto{\pgfqpoint{4.800588in}{0.983673in}}%
\pgfpathlineto{\pgfqpoint{4.800588in}{0.980723in}}%
\pgfpathmoveto{\pgfqpoint{4.786965in}{0.986622in}}%
\pgfpathlineto{\pgfqpoint{4.786965in}{0.986622in}}%
\pgfpathlineto{\pgfqpoint{4.786965in}{0.989571in}}%
\pgfpathlineto{\pgfqpoint{4.791506in}{0.989571in}}%
\pgfpathlineto{\pgfqpoint{4.791506in}{0.986622in}}%
\pgfpathmoveto{\pgfqpoint{4.782424in}{0.992520in}}%
\pgfpathlineto{\pgfqpoint{4.782424in}{0.992520in}}%
\pgfpathlineto{\pgfqpoint{4.782424in}{0.995470in}}%
\pgfpathlineto{\pgfqpoint{4.786965in}{0.995470in}}%
\pgfpathlineto{\pgfqpoint{4.786965in}{0.992520in}}%
\pgfpathmoveto{\pgfqpoint{4.786965in}{0.989571in}}%
\pgfpathlineto{\pgfqpoint{4.786965in}{0.989571in}}%
\pgfpathlineto{\pgfqpoint{4.786965in}{0.992520in}}%
\pgfpathlineto{\pgfqpoint{4.791506in}{0.992520in}}%
\pgfpathlineto{\pgfqpoint{4.791506in}{0.989571in}}%
\pgfpathmoveto{\pgfqpoint{4.786965in}{0.992520in}}%
\pgfpathlineto{\pgfqpoint{4.786965in}{0.992520in}}%
\pgfpathlineto{\pgfqpoint{4.786965in}{0.995470in}}%
\pgfpathlineto{\pgfqpoint{4.791506in}{0.995470in}}%
\pgfpathlineto{\pgfqpoint{4.791506in}{0.992520in}}%
\pgfpathmoveto{\pgfqpoint{4.791506in}{0.983673in}}%
\pgfpathlineto{\pgfqpoint{4.791506in}{0.983673in}}%
\pgfpathlineto{\pgfqpoint{4.791506in}{0.986622in}}%
\pgfpathlineto{\pgfqpoint{4.796047in}{0.986622in}}%
\pgfpathlineto{\pgfqpoint{4.796047in}{0.983673in}}%
\pgfpathmoveto{\pgfqpoint{4.791506in}{0.986622in}}%
\pgfpathlineto{\pgfqpoint{4.791506in}{0.986622in}}%
\pgfpathlineto{\pgfqpoint{4.791506in}{0.989571in}}%
\pgfpathlineto{\pgfqpoint{4.796047in}{0.989571in}}%
\pgfpathlineto{\pgfqpoint{4.796047in}{0.986622in}}%
\pgfpathmoveto{\pgfqpoint{4.800588in}{0.971876in}}%
\pgfpathlineto{\pgfqpoint{4.800588in}{0.971876in}}%
\pgfpathlineto{\pgfqpoint{4.800588in}{0.974825in}}%
\pgfpathlineto{\pgfqpoint{4.805129in}{0.974825in}}%
\pgfpathlineto{\pgfqpoint{4.805129in}{0.971876in}}%
\pgfpathmoveto{\pgfqpoint{4.800588in}{0.974825in}}%
\pgfpathlineto{\pgfqpoint{4.800588in}{0.974825in}}%
\pgfpathlineto{\pgfqpoint{4.800588in}{0.977774in}}%
\pgfpathlineto{\pgfqpoint{4.805129in}{0.977774in}}%
\pgfpathlineto{\pgfqpoint{4.805129in}{0.974825in}}%
\pgfpathmoveto{\pgfqpoint{4.782424in}{0.995470in}}%
\pgfpathlineto{\pgfqpoint{4.782424in}{0.995470in}}%
\pgfpathlineto{\pgfqpoint{4.782424in}{0.998419in}}%
\pgfpathlineto{\pgfqpoint{4.786965in}{0.998419in}}%
\pgfpathlineto{\pgfqpoint{4.786965in}{0.995470in}}%
\pgfpathmoveto{\pgfqpoint{4.782424in}{0.998419in}}%
\pgfpathlineto{\pgfqpoint{4.782424in}{0.998419in}}%
\pgfpathlineto{\pgfqpoint{4.782424in}{1.001368in}}%
\pgfpathlineto{\pgfqpoint{4.786965in}{1.001368in}}%
\pgfpathlineto{\pgfqpoint{4.786965in}{0.998419in}}%
\pgfpathmoveto{\pgfqpoint{4.759719in}{1.022013in}}%
\pgfpathlineto{\pgfqpoint{4.759719in}{1.022013in}}%
\pgfpathlineto{\pgfqpoint{4.759719in}{1.024962in}}%
\pgfpathlineto{\pgfqpoint{4.764260in}{1.024962in}}%
\pgfpathlineto{\pgfqpoint{4.764260in}{1.022013in}}%
\pgfpathmoveto{\pgfqpoint{4.755178in}{1.027911in}}%
\pgfpathlineto{\pgfqpoint{4.755178in}{1.027911in}}%
\pgfpathlineto{\pgfqpoint{4.755178in}{1.030860in}}%
\pgfpathlineto{\pgfqpoint{4.759719in}{1.030860in}}%
\pgfpathlineto{\pgfqpoint{4.759719in}{1.027911in}}%
\pgfpathmoveto{\pgfqpoint{4.759719in}{1.024962in}}%
\pgfpathlineto{\pgfqpoint{4.759719in}{1.024962in}}%
\pgfpathlineto{\pgfqpoint{4.759719in}{1.027911in}}%
\pgfpathlineto{\pgfqpoint{4.764260in}{1.027911in}}%
\pgfpathlineto{\pgfqpoint{4.764260in}{1.024962in}}%
\pgfpathmoveto{\pgfqpoint{4.759719in}{1.027911in}}%
\pgfpathlineto{\pgfqpoint{4.759719in}{1.027911in}}%
\pgfpathlineto{\pgfqpoint{4.759719in}{1.030860in}}%
\pgfpathlineto{\pgfqpoint{4.764260in}{1.030860in}}%
\pgfpathlineto{\pgfqpoint{4.764260in}{1.027911in}}%
\pgfpathmoveto{\pgfqpoint{4.750637in}{1.033809in}}%
\pgfpathlineto{\pgfqpoint{4.750637in}{1.033809in}}%
\pgfpathlineto{\pgfqpoint{4.750637in}{1.036759in}}%
\pgfpathlineto{\pgfqpoint{4.755178in}{1.036759in}}%
\pgfpathlineto{\pgfqpoint{4.755178in}{1.033809in}}%
\pgfpathmoveto{\pgfqpoint{4.746096in}{1.039708in}}%
\pgfpathlineto{\pgfqpoint{4.746096in}{1.039708in}}%
\pgfpathlineto{\pgfqpoint{4.746096in}{1.042657in}}%
\pgfpathlineto{\pgfqpoint{4.750637in}{1.042657in}}%
\pgfpathlineto{\pgfqpoint{4.750637in}{1.039708in}}%
\pgfpathmoveto{\pgfqpoint{4.750637in}{1.036759in}}%
\pgfpathlineto{\pgfqpoint{4.750637in}{1.036759in}}%
\pgfpathlineto{\pgfqpoint{4.750637in}{1.039708in}}%
\pgfpathlineto{\pgfqpoint{4.755178in}{1.039708in}}%
\pgfpathlineto{\pgfqpoint{4.755178in}{1.036759in}}%
\pgfpathmoveto{\pgfqpoint{4.750637in}{1.039708in}}%
\pgfpathlineto{\pgfqpoint{4.750637in}{1.039708in}}%
\pgfpathlineto{\pgfqpoint{4.750637in}{1.042657in}}%
\pgfpathlineto{\pgfqpoint{4.755178in}{1.042657in}}%
\pgfpathlineto{\pgfqpoint{4.755178in}{1.039708in}}%
\pgfpathmoveto{\pgfqpoint{4.755178in}{1.030860in}}%
\pgfpathlineto{\pgfqpoint{4.755178in}{1.030860in}}%
\pgfpathlineto{\pgfqpoint{4.755178in}{1.033809in}}%
\pgfpathlineto{\pgfqpoint{4.759719in}{1.033809in}}%
\pgfpathlineto{\pgfqpoint{4.759719in}{1.030860in}}%
\pgfpathmoveto{\pgfqpoint{4.755178in}{1.033809in}}%
\pgfpathlineto{\pgfqpoint{4.755178in}{1.033809in}}%
\pgfpathlineto{\pgfqpoint{4.755178in}{1.036759in}}%
\pgfpathlineto{\pgfqpoint{4.759719in}{1.036759in}}%
\pgfpathlineto{\pgfqpoint{4.759719in}{1.033809in}}%
\pgfpathmoveto{\pgfqpoint{4.764260in}{1.019063in}}%
\pgfpathlineto{\pgfqpoint{4.764260in}{1.019063in}}%
\pgfpathlineto{\pgfqpoint{4.764260in}{1.022013in}}%
\pgfpathlineto{\pgfqpoint{4.768801in}{1.022013in}}%
\pgfpathlineto{\pgfqpoint{4.768801in}{1.019063in}}%
\pgfpathmoveto{\pgfqpoint{4.764260in}{1.022013in}}%
\pgfpathlineto{\pgfqpoint{4.764260in}{1.022013in}}%
\pgfpathlineto{\pgfqpoint{4.764260in}{1.024962in}}%
\pgfpathlineto{\pgfqpoint{4.768801in}{1.024962in}}%
\pgfpathlineto{\pgfqpoint{4.768801in}{1.022013in}}%
\pgfpathmoveto{\pgfqpoint{4.746096in}{1.042657in}}%
\pgfpathlineto{\pgfqpoint{4.746096in}{1.042657in}}%
\pgfpathlineto{\pgfqpoint{4.746096in}{1.045606in}}%
\pgfpathlineto{\pgfqpoint{4.750637in}{1.045606in}}%
\pgfpathlineto{\pgfqpoint{4.750637in}{1.042657in}}%
\pgfpathmoveto{\pgfqpoint{4.746096in}{1.045606in}}%
\pgfpathlineto{\pgfqpoint{4.746096in}{1.045606in}}%
\pgfpathlineto{\pgfqpoint{4.746096in}{1.048555in}}%
\pgfpathlineto{\pgfqpoint{4.750637in}{1.048555in}}%
\pgfpathlineto{\pgfqpoint{4.750637in}{1.045606in}}%
\pgfpathmoveto{\pgfqpoint{4.705227in}{1.092794in}}%
\pgfpathlineto{\pgfqpoint{4.705227in}{1.092794in}}%
\pgfpathlineto{\pgfqpoint{4.705227in}{1.095743in}}%
\pgfpathlineto{\pgfqpoint{4.709768in}{1.095743in}}%
\pgfpathlineto{\pgfqpoint{4.709768in}{1.092794in}}%
\pgfpathmoveto{\pgfqpoint{4.700686in}{1.098692in}}%
\pgfpathlineto{\pgfqpoint{4.700686in}{1.098692in}}%
\pgfpathlineto{\pgfqpoint{4.700686in}{1.101642in}}%
\pgfpathlineto{\pgfqpoint{4.705227in}{1.101642in}}%
\pgfpathlineto{\pgfqpoint{4.705227in}{1.098692in}}%
\pgfpathmoveto{\pgfqpoint{4.705227in}{1.095743in}}%
\pgfpathlineto{\pgfqpoint{4.705227in}{1.095743in}}%
\pgfpathlineto{\pgfqpoint{4.705227in}{1.098692in}}%
\pgfpathlineto{\pgfqpoint{4.709768in}{1.098692in}}%
\pgfpathlineto{\pgfqpoint{4.709768in}{1.095743in}}%
\pgfpathmoveto{\pgfqpoint{4.705227in}{1.098692in}}%
\pgfpathlineto{\pgfqpoint{4.705227in}{1.098692in}}%
\pgfpathlineto{\pgfqpoint{4.705227in}{1.101642in}}%
\pgfpathlineto{\pgfqpoint{4.709768in}{1.101642in}}%
\pgfpathlineto{\pgfqpoint{4.709768in}{1.098692in}}%
\pgfpathmoveto{\pgfqpoint{4.696145in}{1.104591in}}%
\pgfpathlineto{\pgfqpoint{4.696145in}{1.104591in}}%
\pgfpathlineto{\pgfqpoint{4.696145in}{1.107540in}}%
\pgfpathlineto{\pgfqpoint{4.700686in}{1.107540in}}%
\pgfpathlineto{\pgfqpoint{4.700686in}{1.104591in}}%
\pgfpathmoveto{\pgfqpoint{4.691604in}{1.110489in}}%
\pgfpathlineto{\pgfqpoint{4.691604in}{1.110489in}}%
\pgfpathlineto{\pgfqpoint{4.691604in}{1.113439in}}%
\pgfpathlineto{\pgfqpoint{4.696145in}{1.113439in}}%
\pgfpathlineto{\pgfqpoint{4.696145in}{1.110489in}}%
\pgfpathmoveto{\pgfqpoint{4.696145in}{1.107540in}}%
\pgfpathlineto{\pgfqpoint{4.696145in}{1.107540in}}%
\pgfpathlineto{\pgfqpoint{4.696145in}{1.110489in}}%
\pgfpathlineto{\pgfqpoint{4.700686in}{1.110489in}}%
\pgfpathlineto{\pgfqpoint{4.700686in}{1.107540in}}%
\pgfpathmoveto{\pgfqpoint{4.696145in}{1.110489in}}%
\pgfpathlineto{\pgfqpoint{4.696145in}{1.110489in}}%
\pgfpathlineto{\pgfqpoint{4.696145in}{1.113439in}}%
\pgfpathlineto{\pgfqpoint{4.700686in}{1.113439in}}%
\pgfpathlineto{\pgfqpoint{4.700686in}{1.110489in}}%
\pgfpathmoveto{\pgfqpoint{4.700686in}{1.101642in}}%
\pgfpathlineto{\pgfqpoint{4.700686in}{1.101642in}}%
\pgfpathlineto{\pgfqpoint{4.700686in}{1.104591in}}%
\pgfpathlineto{\pgfqpoint{4.705227in}{1.104591in}}%
\pgfpathlineto{\pgfqpoint{4.705227in}{1.101642in}}%
\pgfpathmoveto{\pgfqpoint{4.700686in}{1.104591in}}%
\pgfpathlineto{\pgfqpoint{4.700686in}{1.104591in}}%
\pgfpathlineto{\pgfqpoint{4.700686in}{1.107540in}}%
\pgfpathlineto{\pgfqpoint{4.705227in}{1.107540in}}%
\pgfpathlineto{\pgfqpoint{4.705227in}{1.104591in}}%
\pgfpathmoveto{\pgfqpoint{4.723391in}{1.069200in}}%
\pgfpathlineto{\pgfqpoint{4.723391in}{1.069200in}}%
\pgfpathlineto{\pgfqpoint{4.723391in}{1.072149in}}%
\pgfpathlineto{\pgfqpoint{4.727932in}{1.072149in}}%
\pgfpathlineto{\pgfqpoint{4.727932in}{1.069200in}}%
\pgfpathmoveto{\pgfqpoint{4.718850in}{1.075099in}}%
\pgfpathlineto{\pgfqpoint{4.718850in}{1.075099in}}%
\pgfpathlineto{\pgfqpoint{4.718850in}{1.078048in}}%
\pgfpathlineto{\pgfqpoint{4.723391in}{1.078048in}}%
\pgfpathlineto{\pgfqpoint{4.723391in}{1.075099in}}%
\pgfpathmoveto{\pgfqpoint{4.723391in}{1.072149in}}%
\pgfpathlineto{\pgfqpoint{4.723391in}{1.072149in}}%
\pgfpathlineto{\pgfqpoint{4.723391in}{1.075099in}}%
\pgfpathlineto{\pgfqpoint{4.727932in}{1.075099in}}%
\pgfpathlineto{\pgfqpoint{4.727932in}{1.072149in}}%
\pgfpathmoveto{\pgfqpoint{4.723391in}{1.075099in}}%
\pgfpathlineto{\pgfqpoint{4.723391in}{1.075099in}}%
\pgfpathlineto{\pgfqpoint{4.723391in}{1.078048in}}%
\pgfpathlineto{\pgfqpoint{4.727932in}{1.078048in}}%
\pgfpathlineto{\pgfqpoint{4.727932in}{1.075099in}}%
\pgfpathmoveto{\pgfqpoint{4.714309in}{1.080997in}}%
\pgfpathlineto{\pgfqpoint{4.714309in}{1.080997in}}%
\pgfpathlineto{\pgfqpoint{4.714309in}{1.083946in}}%
\pgfpathlineto{\pgfqpoint{4.718850in}{1.083946in}}%
\pgfpathlineto{\pgfqpoint{4.718850in}{1.080997in}}%
\pgfpathmoveto{\pgfqpoint{4.709768in}{1.086895in}}%
\pgfpathlineto{\pgfqpoint{4.709768in}{1.086895in}}%
\pgfpathlineto{\pgfqpoint{4.709768in}{1.089845in}}%
\pgfpathlineto{\pgfqpoint{4.714309in}{1.089845in}}%
\pgfpathlineto{\pgfqpoint{4.714309in}{1.086895in}}%
\pgfpathmoveto{\pgfqpoint{4.714309in}{1.083946in}}%
\pgfpathlineto{\pgfqpoint{4.714309in}{1.083946in}}%
\pgfpathlineto{\pgfqpoint{4.714309in}{1.086895in}}%
\pgfpathlineto{\pgfqpoint{4.718850in}{1.086895in}}%
\pgfpathlineto{\pgfqpoint{4.718850in}{1.083946in}}%
\pgfpathmoveto{\pgfqpoint{4.714309in}{1.086895in}}%
\pgfpathlineto{\pgfqpoint{4.714309in}{1.086895in}}%
\pgfpathlineto{\pgfqpoint{4.714309in}{1.089845in}}%
\pgfpathlineto{\pgfqpoint{4.718850in}{1.089845in}}%
\pgfpathlineto{\pgfqpoint{4.718850in}{1.086895in}}%
\pgfpathmoveto{\pgfqpoint{4.718850in}{1.078048in}}%
\pgfpathlineto{\pgfqpoint{4.718850in}{1.078048in}}%
\pgfpathlineto{\pgfqpoint{4.718850in}{1.080997in}}%
\pgfpathlineto{\pgfqpoint{4.723391in}{1.080997in}}%
\pgfpathlineto{\pgfqpoint{4.723391in}{1.078048in}}%
\pgfpathmoveto{\pgfqpoint{4.718850in}{1.080997in}}%
\pgfpathlineto{\pgfqpoint{4.718850in}{1.080997in}}%
\pgfpathlineto{\pgfqpoint{4.718850in}{1.083946in}}%
\pgfpathlineto{\pgfqpoint{4.723391in}{1.083946in}}%
\pgfpathlineto{\pgfqpoint{4.723391in}{1.080997in}}%
\pgfpathmoveto{\pgfqpoint{4.727932in}{1.066251in}}%
\pgfpathlineto{\pgfqpoint{4.727932in}{1.066251in}}%
\pgfpathlineto{\pgfqpoint{4.727932in}{1.069200in}}%
\pgfpathlineto{\pgfqpoint{4.732473in}{1.069200in}}%
\pgfpathlineto{\pgfqpoint{4.732473in}{1.066251in}}%
\pgfpathmoveto{\pgfqpoint{4.727932in}{1.069200in}}%
\pgfpathlineto{\pgfqpoint{4.727932in}{1.069200in}}%
\pgfpathlineto{\pgfqpoint{4.727932in}{1.072149in}}%
\pgfpathlineto{\pgfqpoint{4.732473in}{1.072149in}}%
\pgfpathlineto{\pgfqpoint{4.732473in}{1.069200in}}%
\pgfpathmoveto{\pgfqpoint{4.709768in}{1.089845in}}%
\pgfpathlineto{\pgfqpoint{4.709768in}{1.089845in}}%
\pgfpathlineto{\pgfqpoint{4.709768in}{1.092794in}}%
\pgfpathlineto{\pgfqpoint{4.714309in}{1.092794in}}%
\pgfpathlineto{\pgfqpoint{4.714309in}{1.089845in}}%
\pgfpathmoveto{\pgfqpoint{4.709768in}{1.092794in}}%
\pgfpathlineto{\pgfqpoint{4.709768in}{1.092794in}}%
\pgfpathlineto{\pgfqpoint{4.709768in}{1.095743in}}%
\pgfpathlineto{\pgfqpoint{4.714309in}{1.095743in}}%
\pgfpathlineto{\pgfqpoint{4.714309in}{1.092794in}}%
\pgfpathmoveto{\pgfqpoint{4.687064in}{1.116388in}}%
\pgfpathlineto{\pgfqpoint{4.687064in}{1.116388in}}%
\pgfpathlineto{\pgfqpoint{4.687064in}{1.119337in}}%
\pgfpathlineto{\pgfqpoint{4.691604in}{1.119337in}}%
\pgfpathlineto{\pgfqpoint{4.691604in}{1.116388in}}%
\pgfpathmoveto{\pgfqpoint{4.682523in}{1.122286in}}%
\pgfpathlineto{\pgfqpoint{4.682523in}{1.122286in}}%
\pgfpathlineto{\pgfqpoint{4.682523in}{1.125236in}}%
\pgfpathlineto{\pgfqpoint{4.687064in}{1.125236in}}%
\pgfpathlineto{\pgfqpoint{4.687064in}{1.122286in}}%
\pgfpathmoveto{\pgfqpoint{4.687064in}{1.119337in}}%
\pgfpathlineto{\pgfqpoint{4.687064in}{1.119337in}}%
\pgfpathlineto{\pgfqpoint{4.687064in}{1.122286in}}%
\pgfpathlineto{\pgfqpoint{4.691604in}{1.122286in}}%
\pgfpathlineto{\pgfqpoint{4.691604in}{1.119337in}}%
\pgfpathmoveto{\pgfqpoint{4.687064in}{1.122286in}}%
\pgfpathlineto{\pgfqpoint{4.687064in}{1.122286in}}%
\pgfpathlineto{\pgfqpoint{4.687064in}{1.125236in}}%
\pgfpathlineto{\pgfqpoint{4.691604in}{1.125236in}}%
\pgfpathlineto{\pgfqpoint{4.691604in}{1.122286in}}%
\pgfpathmoveto{\pgfqpoint{4.677982in}{1.128185in}}%
\pgfpathlineto{\pgfqpoint{4.677982in}{1.128185in}}%
\pgfpathlineto{\pgfqpoint{4.677982in}{1.131134in}}%
\pgfpathlineto{\pgfqpoint{4.682523in}{1.131134in}}%
\pgfpathlineto{\pgfqpoint{4.682523in}{1.128185in}}%
\pgfpathmoveto{\pgfqpoint{4.673441in}{1.134083in}}%
\pgfpathlineto{\pgfqpoint{4.673441in}{1.134083in}}%
\pgfpathlineto{\pgfqpoint{4.673441in}{1.137033in}}%
\pgfpathlineto{\pgfqpoint{4.677982in}{1.137033in}}%
\pgfpathlineto{\pgfqpoint{4.677982in}{1.134083in}}%
\pgfpathmoveto{\pgfqpoint{4.677982in}{1.131134in}}%
\pgfpathlineto{\pgfqpoint{4.677982in}{1.131134in}}%
\pgfpathlineto{\pgfqpoint{4.677982in}{1.134083in}}%
\pgfpathlineto{\pgfqpoint{4.682523in}{1.134083in}}%
\pgfpathlineto{\pgfqpoint{4.682523in}{1.131134in}}%
\pgfpathmoveto{\pgfqpoint{4.677982in}{1.134083in}}%
\pgfpathlineto{\pgfqpoint{4.677982in}{1.134083in}}%
\pgfpathlineto{\pgfqpoint{4.677982in}{1.137033in}}%
\pgfpathlineto{\pgfqpoint{4.682523in}{1.137033in}}%
\pgfpathlineto{\pgfqpoint{4.682523in}{1.134083in}}%
\pgfpathmoveto{\pgfqpoint{4.682523in}{1.125236in}}%
\pgfpathlineto{\pgfqpoint{4.682523in}{1.125236in}}%
\pgfpathlineto{\pgfqpoint{4.682523in}{1.128185in}}%
\pgfpathlineto{\pgfqpoint{4.687064in}{1.128185in}}%
\pgfpathlineto{\pgfqpoint{4.687064in}{1.125236in}}%
\pgfpathmoveto{\pgfqpoint{4.682523in}{1.128185in}}%
\pgfpathlineto{\pgfqpoint{4.682523in}{1.128185in}}%
\pgfpathlineto{\pgfqpoint{4.682523in}{1.131134in}}%
\pgfpathlineto{\pgfqpoint{4.687064in}{1.131134in}}%
\pgfpathlineto{\pgfqpoint{4.687064in}{1.128185in}}%
\pgfpathmoveto{\pgfqpoint{4.691604in}{1.113439in}}%
\pgfpathlineto{\pgfqpoint{4.691604in}{1.113439in}}%
\pgfpathlineto{\pgfqpoint{4.691604in}{1.116388in}}%
\pgfpathlineto{\pgfqpoint{4.696145in}{1.116388in}}%
\pgfpathlineto{\pgfqpoint{4.696145in}{1.113439in}}%
\pgfpathmoveto{\pgfqpoint{4.691604in}{1.116388in}}%
\pgfpathlineto{\pgfqpoint{4.691604in}{1.116388in}}%
\pgfpathlineto{\pgfqpoint{4.691604in}{1.119337in}}%
\pgfpathlineto{\pgfqpoint{4.696145in}{1.119337in}}%
\pgfpathlineto{\pgfqpoint{4.696145in}{1.116388in}}%
\pgfpathmoveto{\pgfqpoint{4.673441in}{1.137033in}}%
\pgfpathlineto{\pgfqpoint{4.673441in}{1.137033in}}%
\pgfpathlineto{\pgfqpoint{4.673441in}{1.139982in}}%
\pgfpathlineto{\pgfqpoint{4.677982in}{1.139982in}}%
\pgfpathlineto{\pgfqpoint{4.677982in}{1.137033in}}%
\pgfpathmoveto{\pgfqpoint{4.673441in}{1.139982in}}%
\pgfpathlineto{\pgfqpoint{4.673441in}{1.139982in}}%
\pgfpathlineto{\pgfqpoint{4.673441in}{1.142931in}}%
\pgfpathlineto{\pgfqpoint{4.677982in}{1.142931in}}%
\pgfpathlineto{\pgfqpoint{4.677982in}{1.139982in}}%
\pgfpathmoveto{\pgfqpoint{4.959520in}{0.762479in}}%
\pgfpathlineto{\pgfqpoint{4.959520in}{0.762479in}}%
\pgfpathlineto{\pgfqpoint{4.959520in}{0.765428in}}%
\pgfpathlineto{\pgfqpoint{4.964061in}{0.765428in}}%
\pgfpathlineto{\pgfqpoint{4.964061in}{0.762479in}}%
\pgfpathmoveto{\pgfqpoint{4.954979in}{0.768377in}}%
\pgfpathlineto{\pgfqpoint{4.954979in}{0.768377in}}%
\pgfpathlineto{\pgfqpoint{4.954979in}{0.771327in}}%
\pgfpathlineto{\pgfqpoint{4.959520in}{0.771327in}}%
\pgfpathlineto{\pgfqpoint{4.959520in}{0.768377in}}%
\pgfpathmoveto{\pgfqpoint{4.959520in}{0.765428in}}%
\pgfpathlineto{\pgfqpoint{4.959520in}{0.765428in}}%
\pgfpathlineto{\pgfqpoint{4.959520in}{0.768377in}}%
\pgfpathlineto{\pgfqpoint{4.964061in}{0.768377in}}%
\pgfpathlineto{\pgfqpoint{4.964061in}{0.765428in}}%
\pgfpathmoveto{\pgfqpoint{4.959520in}{0.768377in}}%
\pgfpathlineto{\pgfqpoint{4.959520in}{0.768377in}}%
\pgfpathlineto{\pgfqpoint{4.959520in}{0.771327in}}%
\pgfpathlineto{\pgfqpoint{4.964061in}{0.771327in}}%
\pgfpathlineto{\pgfqpoint{4.964061in}{0.768377in}}%
\pgfpathmoveto{\pgfqpoint{4.950439in}{0.774276in}}%
\pgfpathlineto{\pgfqpoint{4.950439in}{0.774276in}}%
\pgfpathlineto{\pgfqpoint{4.950439in}{0.777225in}}%
\pgfpathlineto{\pgfqpoint{4.954979in}{0.777225in}}%
\pgfpathlineto{\pgfqpoint{4.954979in}{0.774276in}}%
\pgfpathmoveto{\pgfqpoint{4.945898in}{0.780174in}}%
\pgfpathlineto{\pgfqpoint{4.945898in}{0.780174in}}%
\pgfpathlineto{\pgfqpoint{4.945898in}{0.783123in}}%
\pgfpathlineto{\pgfqpoint{4.950439in}{0.783123in}}%
\pgfpathlineto{\pgfqpoint{4.950439in}{0.780174in}}%
\pgfpathmoveto{\pgfqpoint{4.950439in}{0.777225in}}%
\pgfpathlineto{\pgfqpoint{4.950439in}{0.777225in}}%
\pgfpathlineto{\pgfqpoint{4.950439in}{0.780174in}}%
\pgfpathlineto{\pgfqpoint{4.954979in}{0.780174in}}%
\pgfpathlineto{\pgfqpoint{4.954979in}{0.777225in}}%
\pgfpathmoveto{\pgfqpoint{4.950439in}{0.780174in}}%
\pgfpathlineto{\pgfqpoint{4.950439in}{0.780174in}}%
\pgfpathlineto{\pgfqpoint{4.950439in}{0.783123in}}%
\pgfpathlineto{\pgfqpoint{4.954979in}{0.783123in}}%
\pgfpathlineto{\pgfqpoint{4.954979in}{0.780174in}}%
\pgfpathmoveto{\pgfqpoint{4.954979in}{0.771327in}}%
\pgfpathlineto{\pgfqpoint{4.954979in}{0.771327in}}%
\pgfpathlineto{\pgfqpoint{4.954979in}{0.774276in}}%
\pgfpathlineto{\pgfqpoint{4.959520in}{0.774276in}}%
\pgfpathlineto{\pgfqpoint{4.959520in}{0.771327in}}%
\pgfpathmoveto{\pgfqpoint{4.954979in}{0.774276in}}%
\pgfpathlineto{\pgfqpoint{4.954979in}{0.774276in}}%
\pgfpathlineto{\pgfqpoint{4.954979in}{0.777225in}}%
\pgfpathlineto{\pgfqpoint{4.959520in}{0.777225in}}%
\pgfpathlineto{\pgfqpoint{4.959520in}{0.774276in}}%
\pgfpathmoveto{\pgfqpoint{4.886866in}{0.856857in}}%
\pgfpathlineto{\pgfqpoint{4.886866in}{0.856857in}}%
\pgfpathlineto{\pgfqpoint{4.886866in}{0.859806in}}%
\pgfpathlineto{\pgfqpoint{4.891406in}{0.859806in}}%
\pgfpathlineto{\pgfqpoint{4.891406in}{0.856857in}}%
\pgfpathmoveto{\pgfqpoint{4.882325in}{0.862755in}}%
\pgfpathlineto{\pgfqpoint{4.882325in}{0.862755in}}%
\pgfpathlineto{\pgfqpoint{4.882325in}{0.865705in}}%
\pgfpathlineto{\pgfqpoint{4.886866in}{0.865705in}}%
\pgfpathlineto{\pgfqpoint{4.886866in}{0.862755in}}%
\pgfpathmoveto{\pgfqpoint{4.886866in}{0.859806in}}%
\pgfpathlineto{\pgfqpoint{4.886866in}{0.859806in}}%
\pgfpathlineto{\pgfqpoint{4.886866in}{0.862755in}}%
\pgfpathlineto{\pgfqpoint{4.891406in}{0.862755in}}%
\pgfpathlineto{\pgfqpoint{4.891406in}{0.859806in}}%
\pgfpathmoveto{\pgfqpoint{4.886866in}{0.862755in}}%
\pgfpathlineto{\pgfqpoint{4.886866in}{0.862755in}}%
\pgfpathlineto{\pgfqpoint{4.886866in}{0.865705in}}%
\pgfpathlineto{\pgfqpoint{4.891406in}{0.865705in}}%
\pgfpathlineto{\pgfqpoint{4.891406in}{0.862755in}}%
\pgfpathmoveto{\pgfqpoint{4.877784in}{0.868654in}}%
\pgfpathlineto{\pgfqpoint{4.877784in}{0.868654in}}%
\pgfpathlineto{\pgfqpoint{4.877784in}{0.871603in}}%
\pgfpathlineto{\pgfqpoint{4.882325in}{0.871603in}}%
\pgfpathlineto{\pgfqpoint{4.882325in}{0.868654in}}%
\pgfpathmoveto{\pgfqpoint{4.873243in}{0.874553in}}%
\pgfpathlineto{\pgfqpoint{4.873243in}{0.874553in}}%
\pgfpathlineto{\pgfqpoint{4.873243in}{0.877502in}}%
\pgfpathlineto{\pgfqpoint{4.877784in}{0.877502in}}%
\pgfpathlineto{\pgfqpoint{4.877784in}{0.874553in}}%
\pgfpathmoveto{\pgfqpoint{4.877784in}{0.871603in}}%
\pgfpathlineto{\pgfqpoint{4.877784in}{0.871603in}}%
\pgfpathlineto{\pgfqpoint{4.877784in}{0.874553in}}%
\pgfpathlineto{\pgfqpoint{4.882325in}{0.874553in}}%
\pgfpathlineto{\pgfqpoint{4.882325in}{0.871603in}}%
\pgfpathmoveto{\pgfqpoint{4.877784in}{0.874553in}}%
\pgfpathlineto{\pgfqpoint{4.877784in}{0.874553in}}%
\pgfpathlineto{\pgfqpoint{4.877784in}{0.877502in}}%
\pgfpathlineto{\pgfqpoint{4.882325in}{0.877502in}}%
\pgfpathlineto{\pgfqpoint{4.882325in}{0.874553in}}%
\pgfpathmoveto{\pgfqpoint{4.882325in}{0.865705in}}%
\pgfpathlineto{\pgfqpoint{4.882325in}{0.865705in}}%
\pgfpathlineto{\pgfqpoint{4.882325in}{0.868654in}}%
\pgfpathlineto{\pgfqpoint{4.886866in}{0.868654in}}%
\pgfpathlineto{\pgfqpoint{4.886866in}{0.865705in}}%
\pgfpathmoveto{\pgfqpoint{4.882325in}{0.868654in}}%
\pgfpathlineto{\pgfqpoint{4.882325in}{0.868654in}}%
\pgfpathlineto{\pgfqpoint{4.882325in}{0.871603in}}%
\pgfpathlineto{\pgfqpoint{4.886866in}{0.871603in}}%
\pgfpathlineto{\pgfqpoint{4.886866in}{0.868654in}}%
\pgfpathmoveto{\pgfqpoint{4.923193in}{0.809667in}}%
\pgfpathlineto{\pgfqpoint{4.923193in}{0.809667in}}%
\pgfpathlineto{\pgfqpoint{4.923193in}{0.812617in}}%
\pgfpathlineto{\pgfqpoint{4.927734in}{0.812617in}}%
\pgfpathlineto{\pgfqpoint{4.927734in}{0.809667in}}%
\pgfpathmoveto{\pgfqpoint{4.918652in}{0.815566in}}%
\pgfpathlineto{\pgfqpoint{4.918652in}{0.815566in}}%
\pgfpathlineto{\pgfqpoint{4.918652in}{0.818515in}}%
\pgfpathlineto{\pgfqpoint{4.923193in}{0.818515in}}%
\pgfpathlineto{\pgfqpoint{4.923193in}{0.815566in}}%
\pgfpathmoveto{\pgfqpoint{4.923193in}{0.812617in}}%
\pgfpathlineto{\pgfqpoint{4.923193in}{0.812617in}}%
\pgfpathlineto{\pgfqpoint{4.923193in}{0.815566in}}%
\pgfpathlineto{\pgfqpoint{4.927734in}{0.815566in}}%
\pgfpathlineto{\pgfqpoint{4.927734in}{0.812617in}}%
\pgfpathmoveto{\pgfqpoint{4.923193in}{0.815566in}}%
\pgfpathlineto{\pgfqpoint{4.923193in}{0.815566in}}%
\pgfpathlineto{\pgfqpoint{4.923193in}{0.818515in}}%
\pgfpathlineto{\pgfqpoint{4.927734in}{0.818515in}}%
\pgfpathlineto{\pgfqpoint{4.927734in}{0.815566in}}%
\pgfpathmoveto{\pgfqpoint{4.914111in}{0.821465in}}%
\pgfpathlineto{\pgfqpoint{4.914111in}{0.821465in}}%
\pgfpathlineto{\pgfqpoint{4.914111in}{0.824414in}}%
\pgfpathlineto{\pgfqpoint{4.918652in}{0.824414in}}%
\pgfpathlineto{\pgfqpoint{4.918652in}{0.821465in}}%
\pgfpathmoveto{\pgfqpoint{4.909570in}{0.827363in}}%
\pgfpathlineto{\pgfqpoint{4.909570in}{0.827363in}}%
\pgfpathlineto{\pgfqpoint{4.909570in}{0.830313in}}%
\pgfpathlineto{\pgfqpoint{4.914111in}{0.830313in}}%
\pgfpathlineto{\pgfqpoint{4.914111in}{0.827363in}}%
\pgfpathmoveto{\pgfqpoint{4.914111in}{0.824414in}}%
\pgfpathlineto{\pgfqpoint{4.914111in}{0.824414in}}%
\pgfpathlineto{\pgfqpoint{4.914111in}{0.827363in}}%
\pgfpathlineto{\pgfqpoint{4.918652in}{0.827363in}}%
\pgfpathlineto{\pgfqpoint{4.918652in}{0.824414in}}%
\pgfpathmoveto{\pgfqpoint{4.914111in}{0.827363in}}%
\pgfpathlineto{\pgfqpoint{4.914111in}{0.827363in}}%
\pgfpathlineto{\pgfqpoint{4.914111in}{0.830313in}}%
\pgfpathlineto{\pgfqpoint{4.918652in}{0.830313in}}%
\pgfpathlineto{\pgfqpoint{4.918652in}{0.827363in}}%
\pgfpathmoveto{\pgfqpoint{4.918652in}{0.818515in}}%
\pgfpathlineto{\pgfqpoint{4.918652in}{0.818515in}}%
\pgfpathlineto{\pgfqpoint{4.918652in}{0.821465in}}%
\pgfpathlineto{\pgfqpoint{4.923193in}{0.821465in}}%
\pgfpathlineto{\pgfqpoint{4.923193in}{0.818515in}}%
\pgfpathmoveto{\pgfqpoint{4.918652in}{0.821465in}}%
\pgfpathlineto{\pgfqpoint{4.918652in}{0.821465in}}%
\pgfpathlineto{\pgfqpoint{4.918652in}{0.824414in}}%
\pgfpathlineto{\pgfqpoint{4.923193in}{0.824414in}}%
\pgfpathlineto{\pgfqpoint{4.923193in}{0.821465in}}%
\pgfpathmoveto{\pgfqpoint{4.941357in}{0.786073in}}%
\pgfpathlineto{\pgfqpoint{4.941357in}{0.786073in}}%
\pgfpathlineto{\pgfqpoint{4.941357in}{0.789022in}}%
\pgfpathlineto{\pgfqpoint{4.945898in}{0.789022in}}%
\pgfpathlineto{\pgfqpoint{4.945898in}{0.786073in}}%
\pgfpathmoveto{\pgfqpoint{4.936816in}{0.791971in}}%
\pgfpathlineto{\pgfqpoint{4.936816in}{0.791971in}}%
\pgfpathlineto{\pgfqpoint{4.936816in}{0.794921in}}%
\pgfpathlineto{\pgfqpoint{4.941357in}{0.794921in}}%
\pgfpathlineto{\pgfqpoint{4.941357in}{0.791971in}}%
\pgfpathmoveto{\pgfqpoint{4.941357in}{0.789022in}}%
\pgfpathlineto{\pgfqpoint{4.941357in}{0.789022in}}%
\pgfpathlineto{\pgfqpoint{4.941357in}{0.791971in}}%
\pgfpathlineto{\pgfqpoint{4.945898in}{0.791971in}}%
\pgfpathlineto{\pgfqpoint{4.945898in}{0.789022in}}%
\pgfpathmoveto{\pgfqpoint{4.941357in}{0.791971in}}%
\pgfpathlineto{\pgfqpoint{4.941357in}{0.791971in}}%
\pgfpathlineto{\pgfqpoint{4.941357in}{0.794921in}}%
\pgfpathlineto{\pgfqpoint{4.945898in}{0.794921in}}%
\pgfpathlineto{\pgfqpoint{4.945898in}{0.791971in}}%
\pgfpathmoveto{\pgfqpoint{4.932275in}{0.797870in}}%
\pgfpathlineto{\pgfqpoint{4.932275in}{0.797870in}}%
\pgfpathlineto{\pgfqpoint{4.932275in}{0.800819in}}%
\pgfpathlineto{\pgfqpoint{4.936816in}{0.800819in}}%
\pgfpathlineto{\pgfqpoint{4.936816in}{0.797870in}}%
\pgfpathmoveto{\pgfqpoint{4.927734in}{0.803769in}}%
\pgfpathlineto{\pgfqpoint{4.927734in}{0.803769in}}%
\pgfpathlineto{\pgfqpoint{4.927734in}{0.806718in}}%
\pgfpathlineto{\pgfqpoint{4.932275in}{0.806718in}}%
\pgfpathlineto{\pgfqpoint{4.932275in}{0.803769in}}%
\pgfpathmoveto{\pgfqpoint{4.932275in}{0.800819in}}%
\pgfpathlineto{\pgfqpoint{4.932275in}{0.800819in}}%
\pgfpathlineto{\pgfqpoint{4.932275in}{0.803769in}}%
\pgfpathlineto{\pgfqpoint{4.936816in}{0.803769in}}%
\pgfpathlineto{\pgfqpoint{4.936816in}{0.800819in}}%
\pgfpathmoveto{\pgfqpoint{4.932275in}{0.803769in}}%
\pgfpathlineto{\pgfqpoint{4.932275in}{0.803769in}}%
\pgfpathlineto{\pgfqpoint{4.932275in}{0.806718in}}%
\pgfpathlineto{\pgfqpoint{4.936816in}{0.806718in}}%
\pgfpathlineto{\pgfqpoint{4.936816in}{0.803769in}}%
\pgfpathmoveto{\pgfqpoint{4.936816in}{0.794921in}}%
\pgfpathlineto{\pgfqpoint{4.936816in}{0.794921in}}%
\pgfpathlineto{\pgfqpoint{4.936816in}{0.797870in}}%
\pgfpathlineto{\pgfqpoint{4.941357in}{0.797870in}}%
\pgfpathlineto{\pgfqpoint{4.941357in}{0.794921in}}%
\pgfpathmoveto{\pgfqpoint{4.936816in}{0.797870in}}%
\pgfpathlineto{\pgfqpoint{4.936816in}{0.797870in}}%
\pgfpathlineto{\pgfqpoint{4.936816in}{0.800819in}}%
\pgfpathlineto{\pgfqpoint{4.941357in}{0.800819in}}%
\pgfpathlineto{\pgfqpoint{4.941357in}{0.797870in}}%
\pgfpathmoveto{\pgfqpoint{4.945898in}{0.783123in}}%
\pgfpathlineto{\pgfqpoint{4.945898in}{0.783123in}}%
\pgfpathlineto{\pgfqpoint{4.945898in}{0.786073in}}%
\pgfpathlineto{\pgfqpoint{4.950439in}{0.786073in}}%
\pgfpathlineto{\pgfqpoint{4.950439in}{0.783123in}}%
\pgfpathmoveto{\pgfqpoint{4.945898in}{0.786073in}}%
\pgfpathlineto{\pgfqpoint{4.945898in}{0.786073in}}%
\pgfpathlineto{\pgfqpoint{4.945898in}{0.789022in}}%
\pgfpathlineto{\pgfqpoint{4.950439in}{0.789022in}}%
\pgfpathlineto{\pgfqpoint{4.950439in}{0.786073in}}%
\pgfpathmoveto{\pgfqpoint{4.927734in}{0.806718in}}%
\pgfpathlineto{\pgfqpoint{4.927734in}{0.806718in}}%
\pgfpathlineto{\pgfqpoint{4.927734in}{0.809667in}}%
\pgfpathlineto{\pgfqpoint{4.932275in}{0.809667in}}%
\pgfpathlineto{\pgfqpoint{4.932275in}{0.806718in}}%
\pgfpathmoveto{\pgfqpoint{4.927734in}{0.809667in}}%
\pgfpathlineto{\pgfqpoint{4.927734in}{0.809667in}}%
\pgfpathlineto{\pgfqpoint{4.927734in}{0.812617in}}%
\pgfpathlineto{\pgfqpoint{4.932275in}{0.812617in}}%
\pgfpathlineto{\pgfqpoint{4.932275in}{0.809667in}}%
\pgfpathmoveto{\pgfqpoint{4.905029in}{0.833262in}}%
\pgfpathlineto{\pgfqpoint{4.905029in}{0.833262in}}%
\pgfpathlineto{\pgfqpoint{4.905029in}{0.836211in}}%
\pgfpathlineto{\pgfqpoint{4.909570in}{0.836211in}}%
\pgfpathlineto{\pgfqpoint{4.909570in}{0.833262in}}%
\pgfpathmoveto{\pgfqpoint{4.900488in}{0.839161in}}%
\pgfpathlineto{\pgfqpoint{4.900488in}{0.839161in}}%
\pgfpathlineto{\pgfqpoint{4.900488in}{0.842110in}}%
\pgfpathlineto{\pgfqpoint{4.905029in}{0.842110in}}%
\pgfpathlineto{\pgfqpoint{4.905029in}{0.839161in}}%
\pgfpathmoveto{\pgfqpoint{4.905029in}{0.836211in}}%
\pgfpathlineto{\pgfqpoint{4.905029in}{0.836211in}}%
\pgfpathlineto{\pgfqpoint{4.905029in}{0.839161in}}%
\pgfpathlineto{\pgfqpoint{4.909570in}{0.839161in}}%
\pgfpathlineto{\pgfqpoint{4.909570in}{0.836211in}}%
\pgfpathmoveto{\pgfqpoint{4.905029in}{0.839161in}}%
\pgfpathlineto{\pgfqpoint{4.905029in}{0.839161in}}%
\pgfpathlineto{\pgfqpoint{4.905029in}{0.842110in}}%
\pgfpathlineto{\pgfqpoint{4.909570in}{0.842110in}}%
\pgfpathlineto{\pgfqpoint{4.909570in}{0.839161in}}%
\pgfpathmoveto{\pgfqpoint{4.895947in}{0.845059in}}%
\pgfpathlineto{\pgfqpoint{4.895947in}{0.845059in}}%
\pgfpathlineto{\pgfqpoint{4.895947in}{0.848009in}}%
\pgfpathlineto{\pgfqpoint{4.900488in}{0.848009in}}%
\pgfpathlineto{\pgfqpoint{4.900488in}{0.845059in}}%
\pgfpathmoveto{\pgfqpoint{4.891406in}{0.850958in}}%
\pgfpathlineto{\pgfqpoint{4.891406in}{0.850958in}}%
\pgfpathlineto{\pgfqpoint{4.891406in}{0.853907in}}%
\pgfpathlineto{\pgfqpoint{4.895947in}{0.853907in}}%
\pgfpathlineto{\pgfqpoint{4.895947in}{0.850958in}}%
\pgfpathmoveto{\pgfqpoint{4.895947in}{0.848009in}}%
\pgfpathlineto{\pgfqpoint{4.895947in}{0.848009in}}%
\pgfpathlineto{\pgfqpoint{4.895947in}{0.850958in}}%
\pgfpathlineto{\pgfqpoint{4.900488in}{0.850958in}}%
\pgfpathlineto{\pgfqpoint{4.900488in}{0.848009in}}%
\pgfpathmoveto{\pgfqpoint{4.895947in}{0.850958in}}%
\pgfpathlineto{\pgfqpoint{4.895947in}{0.850958in}}%
\pgfpathlineto{\pgfqpoint{4.895947in}{0.853907in}}%
\pgfpathlineto{\pgfqpoint{4.900488in}{0.853907in}}%
\pgfpathlineto{\pgfqpoint{4.900488in}{0.850958in}}%
\pgfpathmoveto{\pgfqpoint{4.900488in}{0.842110in}}%
\pgfpathlineto{\pgfqpoint{4.900488in}{0.842110in}}%
\pgfpathlineto{\pgfqpoint{4.900488in}{0.845059in}}%
\pgfpathlineto{\pgfqpoint{4.905029in}{0.845059in}}%
\pgfpathlineto{\pgfqpoint{4.905029in}{0.842110in}}%
\pgfpathmoveto{\pgfqpoint{4.900488in}{0.845059in}}%
\pgfpathlineto{\pgfqpoint{4.900488in}{0.845059in}}%
\pgfpathlineto{\pgfqpoint{4.900488in}{0.848009in}}%
\pgfpathlineto{\pgfqpoint{4.905029in}{0.848009in}}%
\pgfpathlineto{\pgfqpoint{4.905029in}{0.845059in}}%
\pgfpathmoveto{\pgfqpoint{4.909570in}{0.830313in}}%
\pgfpathlineto{\pgfqpoint{4.909570in}{0.830313in}}%
\pgfpathlineto{\pgfqpoint{4.909570in}{0.833262in}}%
\pgfpathlineto{\pgfqpoint{4.914111in}{0.833262in}}%
\pgfpathlineto{\pgfqpoint{4.914111in}{0.830313in}}%
\pgfpathmoveto{\pgfqpoint{4.909570in}{0.833262in}}%
\pgfpathlineto{\pgfqpoint{4.909570in}{0.833262in}}%
\pgfpathlineto{\pgfqpoint{4.909570in}{0.836211in}}%
\pgfpathlineto{\pgfqpoint{4.914111in}{0.836211in}}%
\pgfpathlineto{\pgfqpoint{4.914111in}{0.833262in}}%
\pgfpathmoveto{\pgfqpoint{4.891406in}{0.853907in}}%
\pgfpathlineto{\pgfqpoint{4.891406in}{0.853907in}}%
\pgfpathlineto{\pgfqpoint{4.891406in}{0.856857in}}%
\pgfpathlineto{\pgfqpoint{4.895947in}{0.856857in}}%
\pgfpathlineto{\pgfqpoint{4.895947in}{0.853907in}}%
\pgfpathmoveto{\pgfqpoint{4.891406in}{0.856857in}}%
\pgfpathlineto{\pgfqpoint{4.891406in}{0.856857in}}%
\pgfpathlineto{\pgfqpoint{4.891406in}{0.859806in}}%
\pgfpathlineto{\pgfqpoint{4.895947in}{0.859806in}}%
\pgfpathlineto{\pgfqpoint{4.895947in}{0.856857in}}%
\pgfpathmoveto{\pgfqpoint{4.850538in}{0.904045in}}%
\pgfpathlineto{\pgfqpoint{4.850538in}{0.904045in}}%
\pgfpathlineto{\pgfqpoint{4.850538in}{0.906994in}}%
\pgfpathlineto{\pgfqpoint{4.855079in}{0.906994in}}%
\pgfpathlineto{\pgfqpoint{4.855079in}{0.904045in}}%
\pgfpathmoveto{\pgfqpoint{4.845997in}{0.909943in}}%
\pgfpathlineto{\pgfqpoint{4.845997in}{0.909943in}}%
\pgfpathlineto{\pgfqpoint{4.845997in}{0.912892in}}%
\pgfpathlineto{\pgfqpoint{4.850538in}{0.912892in}}%
\pgfpathlineto{\pgfqpoint{4.850538in}{0.909943in}}%
\pgfpathmoveto{\pgfqpoint{4.850538in}{0.906994in}}%
\pgfpathlineto{\pgfqpoint{4.850538in}{0.906994in}}%
\pgfpathlineto{\pgfqpoint{4.850538in}{0.909943in}}%
\pgfpathlineto{\pgfqpoint{4.855079in}{0.909943in}}%
\pgfpathlineto{\pgfqpoint{4.855079in}{0.906994in}}%
\pgfpathmoveto{\pgfqpoint{4.850538in}{0.909943in}}%
\pgfpathlineto{\pgfqpoint{4.850538in}{0.909943in}}%
\pgfpathlineto{\pgfqpoint{4.850538in}{0.912892in}}%
\pgfpathlineto{\pgfqpoint{4.855079in}{0.912892in}}%
\pgfpathlineto{\pgfqpoint{4.855079in}{0.909943in}}%
\pgfpathmoveto{\pgfqpoint{4.841456in}{0.915841in}}%
\pgfpathlineto{\pgfqpoint{4.841456in}{0.915841in}}%
\pgfpathlineto{\pgfqpoint{4.841456in}{0.918791in}}%
\pgfpathlineto{\pgfqpoint{4.845997in}{0.918791in}}%
\pgfpathlineto{\pgfqpoint{4.845997in}{0.915841in}}%
\pgfpathmoveto{\pgfqpoint{4.836915in}{0.921740in}}%
\pgfpathlineto{\pgfqpoint{4.836915in}{0.921740in}}%
\pgfpathlineto{\pgfqpoint{4.836915in}{0.924689in}}%
\pgfpathlineto{\pgfqpoint{4.841456in}{0.924689in}}%
\pgfpathlineto{\pgfqpoint{4.841456in}{0.921740in}}%
\pgfpathmoveto{\pgfqpoint{4.841456in}{0.918791in}}%
\pgfpathlineto{\pgfqpoint{4.841456in}{0.918791in}}%
\pgfpathlineto{\pgfqpoint{4.841456in}{0.921740in}}%
\pgfpathlineto{\pgfqpoint{4.845997in}{0.921740in}}%
\pgfpathlineto{\pgfqpoint{4.845997in}{0.918791in}}%
\pgfpathmoveto{\pgfqpoint{4.841456in}{0.921740in}}%
\pgfpathlineto{\pgfqpoint{4.841456in}{0.921740in}}%
\pgfpathlineto{\pgfqpoint{4.841456in}{0.924689in}}%
\pgfpathlineto{\pgfqpoint{4.845997in}{0.924689in}}%
\pgfpathlineto{\pgfqpoint{4.845997in}{0.921740in}}%
\pgfpathmoveto{\pgfqpoint{4.845997in}{0.912892in}}%
\pgfpathlineto{\pgfqpoint{4.845997in}{0.912892in}}%
\pgfpathlineto{\pgfqpoint{4.845997in}{0.915841in}}%
\pgfpathlineto{\pgfqpoint{4.850538in}{0.915841in}}%
\pgfpathlineto{\pgfqpoint{4.850538in}{0.912892in}}%
\pgfpathmoveto{\pgfqpoint{4.845997in}{0.915841in}}%
\pgfpathlineto{\pgfqpoint{4.845997in}{0.915841in}}%
\pgfpathlineto{\pgfqpoint{4.845997in}{0.918791in}}%
\pgfpathlineto{\pgfqpoint{4.850538in}{0.918791in}}%
\pgfpathlineto{\pgfqpoint{4.850538in}{0.915841in}}%
\pgfpathmoveto{\pgfqpoint{4.868702in}{0.880451in}}%
\pgfpathlineto{\pgfqpoint{4.868702in}{0.880451in}}%
\pgfpathlineto{\pgfqpoint{4.868702in}{0.883400in}}%
\pgfpathlineto{\pgfqpoint{4.873243in}{0.883400in}}%
\pgfpathlineto{\pgfqpoint{4.873243in}{0.880451in}}%
\pgfpathmoveto{\pgfqpoint{4.864161in}{0.886350in}}%
\pgfpathlineto{\pgfqpoint{4.864161in}{0.886350in}}%
\pgfpathlineto{\pgfqpoint{4.864161in}{0.889299in}}%
\pgfpathlineto{\pgfqpoint{4.868702in}{0.889299in}}%
\pgfpathlineto{\pgfqpoint{4.868702in}{0.886350in}}%
\pgfpathmoveto{\pgfqpoint{4.868702in}{0.883400in}}%
\pgfpathlineto{\pgfqpoint{4.868702in}{0.883400in}}%
\pgfpathlineto{\pgfqpoint{4.868702in}{0.886350in}}%
\pgfpathlineto{\pgfqpoint{4.873243in}{0.886350in}}%
\pgfpathlineto{\pgfqpoint{4.873243in}{0.883400in}}%
\pgfpathmoveto{\pgfqpoint{4.868702in}{0.886350in}}%
\pgfpathlineto{\pgfqpoint{4.868702in}{0.886350in}}%
\pgfpathlineto{\pgfqpoint{4.868702in}{0.889299in}}%
\pgfpathlineto{\pgfqpoint{4.873243in}{0.889299in}}%
\pgfpathlineto{\pgfqpoint{4.873243in}{0.886350in}}%
\pgfpathmoveto{\pgfqpoint{4.859620in}{0.892248in}}%
\pgfpathlineto{\pgfqpoint{4.859620in}{0.892248in}}%
\pgfpathlineto{\pgfqpoint{4.859620in}{0.895197in}}%
\pgfpathlineto{\pgfqpoint{4.864161in}{0.895197in}}%
\pgfpathlineto{\pgfqpoint{4.864161in}{0.892248in}}%
\pgfpathmoveto{\pgfqpoint{4.855079in}{0.898146in}}%
\pgfpathlineto{\pgfqpoint{4.855079in}{0.898146in}}%
\pgfpathlineto{\pgfqpoint{4.855079in}{0.901095in}}%
\pgfpathlineto{\pgfqpoint{4.859620in}{0.901095in}}%
\pgfpathlineto{\pgfqpoint{4.859620in}{0.898146in}}%
\pgfpathmoveto{\pgfqpoint{4.859620in}{0.895197in}}%
\pgfpathlineto{\pgfqpoint{4.859620in}{0.895197in}}%
\pgfpathlineto{\pgfqpoint{4.859620in}{0.898146in}}%
\pgfpathlineto{\pgfqpoint{4.864161in}{0.898146in}}%
\pgfpathlineto{\pgfqpoint{4.864161in}{0.895197in}}%
\pgfpathmoveto{\pgfqpoint{4.859620in}{0.898146in}}%
\pgfpathlineto{\pgfqpoint{4.859620in}{0.898146in}}%
\pgfpathlineto{\pgfqpoint{4.859620in}{0.901095in}}%
\pgfpathlineto{\pgfqpoint{4.864161in}{0.901095in}}%
\pgfpathlineto{\pgfqpoint{4.864161in}{0.898146in}}%
\pgfpathmoveto{\pgfqpoint{4.864161in}{0.889299in}}%
\pgfpathlineto{\pgfqpoint{4.864161in}{0.889299in}}%
\pgfpathlineto{\pgfqpoint{4.864161in}{0.892248in}}%
\pgfpathlineto{\pgfqpoint{4.868702in}{0.892248in}}%
\pgfpathlineto{\pgfqpoint{4.868702in}{0.889299in}}%
\pgfpathmoveto{\pgfqpoint{4.864161in}{0.892248in}}%
\pgfpathlineto{\pgfqpoint{4.864161in}{0.892248in}}%
\pgfpathlineto{\pgfqpoint{4.864161in}{0.895197in}}%
\pgfpathlineto{\pgfqpoint{4.868702in}{0.895197in}}%
\pgfpathlineto{\pgfqpoint{4.868702in}{0.892248in}}%
\pgfpathmoveto{\pgfqpoint{4.873243in}{0.877502in}}%
\pgfpathlineto{\pgfqpoint{4.873243in}{0.877502in}}%
\pgfpathlineto{\pgfqpoint{4.873243in}{0.880451in}}%
\pgfpathlineto{\pgfqpoint{4.877784in}{0.880451in}}%
\pgfpathlineto{\pgfqpoint{4.877784in}{0.877502in}}%
\pgfpathmoveto{\pgfqpoint{4.873243in}{0.880451in}}%
\pgfpathlineto{\pgfqpoint{4.873243in}{0.880451in}}%
\pgfpathlineto{\pgfqpoint{4.873243in}{0.883400in}}%
\pgfpathlineto{\pgfqpoint{4.877784in}{0.883400in}}%
\pgfpathlineto{\pgfqpoint{4.877784in}{0.880451in}}%
\pgfpathmoveto{\pgfqpoint{4.855079in}{0.901095in}}%
\pgfpathlineto{\pgfqpoint{4.855079in}{0.901095in}}%
\pgfpathlineto{\pgfqpoint{4.855079in}{0.904045in}}%
\pgfpathlineto{\pgfqpoint{4.859620in}{0.904045in}}%
\pgfpathlineto{\pgfqpoint{4.859620in}{0.901095in}}%
\pgfpathmoveto{\pgfqpoint{4.855079in}{0.904045in}}%
\pgfpathlineto{\pgfqpoint{4.855079in}{0.904045in}}%
\pgfpathlineto{\pgfqpoint{4.855079in}{0.906994in}}%
\pgfpathlineto{\pgfqpoint{4.859620in}{0.906994in}}%
\pgfpathlineto{\pgfqpoint{4.859620in}{0.904045in}}%
\pgfpathmoveto{\pgfqpoint{4.832374in}{0.927638in}}%
\pgfpathlineto{\pgfqpoint{4.832374in}{0.927638in}}%
\pgfpathlineto{\pgfqpoint{4.832374in}{0.930587in}}%
\pgfpathlineto{\pgfqpoint{4.836915in}{0.930587in}}%
\pgfpathlineto{\pgfqpoint{4.836915in}{0.927638in}}%
\pgfpathmoveto{\pgfqpoint{4.827833in}{0.933536in}}%
\pgfpathlineto{\pgfqpoint{4.827833in}{0.933536in}}%
\pgfpathlineto{\pgfqpoint{4.827833in}{0.936486in}}%
\pgfpathlineto{\pgfqpoint{4.832374in}{0.936486in}}%
\pgfpathlineto{\pgfqpoint{4.832374in}{0.933536in}}%
\pgfpathmoveto{\pgfqpoint{4.832374in}{0.930587in}}%
\pgfpathlineto{\pgfqpoint{4.832374in}{0.930587in}}%
\pgfpathlineto{\pgfqpoint{4.832374in}{0.933536in}}%
\pgfpathlineto{\pgfqpoint{4.836915in}{0.933536in}}%
\pgfpathlineto{\pgfqpoint{4.836915in}{0.930587in}}%
\pgfpathmoveto{\pgfqpoint{4.832374in}{0.933536in}}%
\pgfpathlineto{\pgfqpoint{4.832374in}{0.933536in}}%
\pgfpathlineto{\pgfqpoint{4.832374in}{0.936486in}}%
\pgfpathlineto{\pgfqpoint{4.836915in}{0.936486in}}%
\pgfpathlineto{\pgfqpoint{4.836915in}{0.933536in}}%
\pgfpathmoveto{\pgfqpoint{4.823293in}{0.939435in}}%
\pgfpathlineto{\pgfqpoint{4.823293in}{0.939435in}}%
\pgfpathlineto{\pgfqpoint{4.823293in}{0.942384in}}%
\pgfpathlineto{\pgfqpoint{4.827833in}{0.942384in}}%
\pgfpathlineto{\pgfqpoint{4.827833in}{0.939435in}}%
\pgfpathmoveto{\pgfqpoint{4.818752in}{0.945333in}}%
\pgfpathlineto{\pgfqpoint{4.818752in}{0.945333in}}%
\pgfpathlineto{\pgfqpoint{4.818752in}{0.948282in}}%
\pgfpathlineto{\pgfqpoint{4.823293in}{0.948282in}}%
\pgfpathlineto{\pgfqpoint{4.823293in}{0.945333in}}%
\pgfpathmoveto{\pgfqpoint{4.823293in}{0.942384in}}%
\pgfpathlineto{\pgfqpoint{4.823293in}{0.942384in}}%
\pgfpathlineto{\pgfqpoint{4.823293in}{0.945333in}}%
\pgfpathlineto{\pgfqpoint{4.827833in}{0.945333in}}%
\pgfpathlineto{\pgfqpoint{4.827833in}{0.942384in}}%
\pgfpathmoveto{\pgfqpoint{4.823293in}{0.945333in}}%
\pgfpathlineto{\pgfqpoint{4.823293in}{0.945333in}}%
\pgfpathlineto{\pgfqpoint{4.823293in}{0.948282in}}%
\pgfpathlineto{\pgfqpoint{4.827833in}{0.948282in}}%
\pgfpathlineto{\pgfqpoint{4.827833in}{0.945333in}}%
\pgfpathmoveto{\pgfqpoint{4.827833in}{0.936486in}}%
\pgfpathlineto{\pgfqpoint{4.827833in}{0.936486in}}%
\pgfpathlineto{\pgfqpoint{4.827833in}{0.939435in}}%
\pgfpathlineto{\pgfqpoint{4.832374in}{0.939435in}}%
\pgfpathlineto{\pgfqpoint{4.832374in}{0.936486in}}%
\pgfpathmoveto{\pgfqpoint{4.827833in}{0.939435in}}%
\pgfpathlineto{\pgfqpoint{4.827833in}{0.939435in}}%
\pgfpathlineto{\pgfqpoint{4.827833in}{0.942384in}}%
\pgfpathlineto{\pgfqpoint{4.832374in}{0.942384in}}%
\pgfpathlineto{\pgfqpoint{4.832374in}{0.939435in}}%
\pgfpathmoveto{\pgfqpoint{4.836915in}{0.924689in}}%
\pgfpathlineto{\pgfqpoint{4.836915in}{0.924689in}}%
\pgfpathlineto{\pgfqpoint{4.836915in}{0.927638in}}%
\pgfpathlineto{\pgfqpoint{4.841456in}{0.927638in}}%
\pgfpathlineto{\pgfqpoint{4.841456in}{0.924689in}}%
\pgfpathmoveto{\pgfqpoint{4.836915in}{0.927638in}}%
\pgfpathlineto{\pgfqpoint{4.836915in}{0.927638in}}%
\pgfpathlineto{\pgfqpoint{4.836915in}{0.930587in}}%
\pgfpathlineto{\pgfqpoint{4.841456in}{0.930587in}}%
\pgfpathlineto{\pgfqpoint{4.841456in}{0.927638in}}%
\pgfpathmoveto{\pgfqpoint{4.818752in}{0.948282in}}%
\pgfpathlineto{\pgfqpoint{4.818752in}{0.948282in}}%
\pgfpathlineto{\pgfqpoint{4.818752in}{0.951232in}}%
\pgfpathlineto{\pgfqpoint{4.823293in}{0.951232in}}%
\pgfpathlineto{\pgfqpoint{4.823293in}{0.948282in}}%
\pgfpathmoveto{\pgfqpoint{4.818752in}{0.951232in}}%
\pgfpathlineto{\pgfqpoint{4.818752in}{0.951232in}}%
\pgfpathlineto{\pgfqpoint{4.818752in}{0.954181in}}%
\pgfpathlineto{\pgfqpoint{4.823293in}{0.954181in}}%
\pgfpathlineto{\pgfqpoint{4.823293in}{0.951232in}}%
\pgfpathmoveto{\pgfqpoint{5.104837in}{0.573732in}}%
\pgfpathlineto{\pgfqpoint{5.104837in}{0.573732in}}%
\pgfpathlineto{\pgfqpoint{5.104837in}{0.576681in}}%
\pgfpathlineto{\pgfqpoint{5.109378in}{0.576681in}}%
\pgfpathlineto{\pgfqpoint{5.109378in}{0.573732in}}%
\pgfpathmoveto{\pgfqpoint{5.100296in}{0.579630in}}%
\pgfpathlineto{\pgfqpoint{5.100296in}{0.579630in}}%
\pgfpathlineto{\pgfqpoint{5.100296in}{0.582580in}}%
\pgfpathlineto{\pgfqpoint{5.104837in}{0.582580in}}%
\pgfpathlineto{\pgfqpoint{5.104837in}{0.579630in}}%
\pgfpathmoveto{\pgfqpoint{5.104837in}{0.576681in}}%
\pgfpathlineto{\pgfqpoint{5.104837in}{0.576681in}}%
\pgfpathlineto{\pgfqpoint{5.104837in}{0.579630in}}%
\pgfpathlineto{\pgfqpoint{5.109378in}{0.579630in}}%
\pgfpathlineto{\pgfqpoint{5.109378in}{0.576681in}}%
\pgfpathmoveto{\pgfqpoint{5.104837in}{0.579630in}}%
\pgfpathlineto{\pgfqpoint{5.104837in}{0.579630in}}%
\pgfpathlineto{\pgfqpoint{5.104837in}{0.582580in}}%
\pgfpathlineto{\pgfqpoint{5.109378in}{0.582580in}}%
\pgfpathlineto{\pgfqpoint{5.109378in}{0.579630in}}%
\pgfpathmoveto{\pgfqpoint{5.095755in}{0.585529in}}%
\pgfpathlineto{\pgfqpoint{5.095755in}{0.585529in}}%
\pgfpathlineto{\pgfqpoint{5.095755in}{0.588478in}}%
\pgfpathlineto{\pgfqpoint{5.100296in}{0.588478in}}%
\pgfpathlineto{\pgfqpoint{5.100296in}{0.585529in}}%
\pgfpathmoveto{\pgfqpoint{5.091214in}{0.591428in}}%
\pgfpathlineto{\pgfqpoint{5.091214in}{0.591428in}}%
\pgfpathlineto{\pgfqpoint{5.091214in}{0.594377in}}%
\pgfpathlineto{\pgfqpoint{5.095755in}{0.594377in}}%
\pgfpathlineto{\pgfqpoint{5.095755in}{0.591428in}}%
\pgfpathmoveto{\pgfqpoint{5.095755in}{0.588478in}}%
\pgfpathlineto{\pgfqpoint{5.095755in}{0.588478in}}%
\pgfpathlineto{\pgfqpoint{5.095755in}{0.591428in}}%
\pgfpathlineto{\pgfqpoint{5.100296in}{0.591428in}}%
\pgfpathlineto{\pgfqpoint{5.100296in}{0.588478in}}%
\pgfpathmoveto{\pgfqpoint{5.095755in}{0.591428in}}%
\pgfpathlineto{\pgfqpoint{5.095755in}{0.591428in}}%
\pgfpathlineto{\pgfqpoint{5.095755in}{0.594377in}}%
\pgfpathlineto{\pgfqpoint{5.100296in}{0.594377in}}%
\pgfpathlineto{\pgfqpoint{5.100296in}{0.591428in}}%
\pgfpathmoveto{\pgfqpoint{5.100296in}{0.582580in}}%
\pgfpathlineto{\pgfqpoint{5.100296in}{0.582580in}}%
\pgfpathlineto{\pgfqpoint{5.100296in}{0.585529in}}%
\pgfpathlineto{\pgfqpoint{5.104837in}{0.585529in}}%
\pgfpathlineto{\pgfqpoint{5.104837in}{0.582580in}}%
\pgfpathmoveto{\pgfqpoint{5.100296in}{0.585529in}}%
\pgfpathlineto{\pgfqpoint{5.100296in}{0.585529in}}%
\pgfpathlineto{\pgfqpoint{5.100296in}{0.588478in}}%
\pgfpathlineto{\pgfqpoint{5.104837in}{0.588478in}}%
\pgfpathlineto{\pgfqpoint{5.104837in}{0.585529in}}%
\pgfpathmoveto{\pgfqpoint{5.032179in}{0.668105in}}%
\pgfpathlineto{\pgfqpoint{5.032179in}{0.668105in}}%
\pgfpathlineto{\pgfqpoint{5.032179in}{0.671054in}}%
\pgfpathlineto{\pgfqpoint{5.036720in}{0.671054in}}%
\pgfpathlineto{\pgfqpoint{5.036720in}{0.668105in}}%
\pgfpathmoveto{\pgfqpoint{5.027637in}{0.674003in}}%
\pgfpathlineto{\pgfqpoint{5.027637in}{0.674003in}}%
\pgfpathlineto{\pgfqpoint{5.027637in}{0.676952in}}%
\pgfpathlineto{\pgfqpoint{5.032179in}{0.676952in}}%
\pgfpathlineto{\pgfqpoint{5.032179in}{0.674003in}}%
\pgfpathmoveto{\pgfqpoint{5.032179in}{0.671054in}}%
\pgfpathlineto{\pgfqpoint{5.032179in}{0.671054in}}%
\pgfpathlineto{\pgfqpoint{5.032179in}{0.674003in}}%
\pgfpathlineto{\pgfqpoint{5.036720in}{0.674003in}}%
\pgfpathlineto{\pgfqpoint{5.036720in}{0.671054in}}%
\pgfpathmoveto{\pgfqpoint{5.032179in}{0.674003in}}%
\pgfpathlineto{\pgfqpoint{5.032179in}{0.674003in}}%
\pgfpathlineto{\pgfqpoint{5.032179in}{0.676952in}}%
\pgfpathlineto{\pgfqpoint{5.036720in}{0.676952in}}%
\pgfpathlineto{\pgfqpoint{5.036720in}{0.674003in}}%
\pgfpathmoveto{\pgfqpoint{5.023096in}{0.679901in}}%
\pgfpathlineto{\pgfqpoint{5.023096in}{0.679901in}}%
\pgfpathlineto{\pgfqpoint{5.023096in}{0.682850in}}%
\pgfpathlineto{\pgfqpoint{5.027637in}{0.682850in}}%
\pgfpathlineto{\pgfqpoint{5.027637in}{0.679901in}}%
\pgfpathmoveto{\pgfqpoint{5.018555in}{0.685799in}}%
\pgfpathlineto{\pgfqpoint{5.018555in}{0.685799in}}%
\pgfpathlineto{\pgfqpoint{5.018555in}{0.688749in}}%
\pgfpathlineto{\pgfqpoint{5.023096in}{0.688749in}}%
\pgfpathlineto{\pgfqpoint{5.023096in}{0.685799in}}%
\pgfpathmoveto{\pgfqpoint{5.023096in}{0.682850in}}%
\pgfpathlineto{\pgfqpoint{5.023096in}{0.682850in}}%
\pgfpathlineto{\pgfqpoint{5.023096in}{0.685799in}}%
\pgfpathlineto{\pgfqpoint{5.027637in}{0.685799in}}%
\pgfpathlineto{\pgfqpoint{5.027637in}{0.682850in}}%
\pgfpathmoveto{\pgfqpoint{5.023096in}{0.685799in}}%
\pgfpathlineto{\pgfqpoint{5.023096in}{0.685799in}}%
\pgfpathlineto{\pgfqpoint{5.023096in}{0.688749in}}%
\pgfpathlineto{\pgfqpoint{5.027637in}{0.688749in}}%
\pgfpathlineto{\pgfqpoint{5.027637in}{0.685799in}}%
\pgfpathmoveto{\pgfqpoint{5.027637in}{0.676952in}}%
\pgfpathlineto{\pgfqpoint{5.027637in}{0.676952in}}%
\pgfpathlineto{\pgfqpoint{5.027637in}{0.679901in}}%
\pgfpathlineto{\pgfqpoint{5.032179in}{0.679901in}}%
\pgfpathlineto{\pgfqpoint{5.032179in}{0.676952in}}%
\pgfpathmoveto{\pgfqpoint{5.027637in}{0.679901in}}%
\pgfpathlineto{\pgfqpoint{5.027637in}{0.679901in}}%
\pgfpathlineto{\pgfqpoint{5.027637in}{0.682850in}}%
\pgfpathlineto{\pgfqpoint{5.032179in}{0.682850in}}%
\pgfpathlineto{\pgfqpoint{5.032179in}{0.679901in}}%
\pgfpathmoveto{\pgfqpoint{5.068508in}{0.620919in}}%
\pgfpathlineto{\pgfqpoint{5.068508in}{0.620919in}}%
\pgfpathlineto{\pgfqpoint{5.068508in}{0.623868in}}%
\pgfpathlineto{\pgfqpoint{5.073049in}{0.623868in}}%
\pgfpathlineto{\pgfqpoint{5.073049in}{0.620919in}}%
\pgfpathmoveto{\pgfqpoint{5.063967in}{0.626817in}}%
\pgfpathlineto{\pgfqpoint{5.063967in}{0.626817in}}%
\pgfpathlineto{\pgfqpoint{5.063967in}{0.629766in}}%
\pgfpathlineto{\pgfqpoint{5.068508in}{0.629766in}}%
\pgfpathlineto{\pgfqpoint{5.068508in}{0.626817in}}%
\pgfpathmoveto{\pgfqpoint{5.068508in}{0.623868in}}%
\pgfpathlineto{\pgfqpoint{5.068508in}{0.623868in}}%
\pgfpathlineto{\pgfqpoint{5.068508in}{0.626817in}}%
\pgfpathlineto{\pgfqpoint{5.073049in}{0.626817in}}%
\pgfpathlineto{\pgfqpoint{5.073049in}{0.623868in}}%
\pgfpathmoveto{\pgfqpoint{5.068508in}{0.626817in}}%
\pgfpathlineto{\pgfqpoint{5.068508in}{0.626817in}}%
\pgfpathlineto{\pgfqpoint{5.068508in}{0.629766in}}%
\pgfpathlineto{\pgfqpoint{5.073049in}{0.629766in}}%
\pgfpathlineto{\pgfqpoint{5.073049in}{0.626817in}}%
\pgfpathmoveto{\pgfqpoint{5.059426in}{0.632716in}}%
\pgfpathlineto{\pgfqpoint{5.059426in}{0.632716in}}%
\pgfpathlineto{\pgfqpoint{5.059426in}{0.635665in}}%
\pgfpathlineto{\pgfqpoint{5.063967in}{0.635665in}}%
\pgfpathlineto{\pgfqpoint{5.063967in}{0.632716in}}%
\pgfpathmoveto{\pgfqpoint{5.054884in}{0.638614in}}%
\pgfpathlineto{\pgfqpoint{5.054884in}{0.638614in}}%
\pgfpathlineto{\pgfqpoint{5.054884in}{0.641563in}}%
\pgfpathlineto{\pgfqpoint{5.059426in}{0.641563in}}%
\pgfpathlineto{\pgfqpoint{5.059426in}{0.638614in}}%
\pgfpathmoveto{\pgfqpoint{5.059426in}{0.635665in}}%
\pgfpathlineto{\pgfqpoint{5.059426in}{0.635665in}}%
\pgfpathlineto{\pgfqpoint{5.059426in}{0.638614in}}%
\pgfpathlineto{\pgfqpoint{5.063967in}{0.638614in}}%
\pgfpathlineto{\pgfqpoint{5.063967in}{0.635665in}}%
\pgfpathmoveto{\pgfqpoint{5.059426in}{0.638614in}}%
\pgfpathlineto{\pgfqpoint{5.059426in}{0.638614in}}%
\pgfpathlineto{\pgfqpoint{5.059426in}{0.641563in}}%
\pgfpathlineto{\pgfqpoint{5.063967in}{0.641563in}}%
\pgfpathlineto{\pgfqpoint{5.063967in}{0.638614in}}%
\pgfpathmoveto{\pgfqpoint{5.063967in}{0.629766in}}%
\pgfpathlineto{\pgfqpoint{5.063967in}{0.629766in}}%
\pgfpathlineto{\pgfqpoint{5.063967in}{0.632716in}}%
\pgfpathlineto{\pgfqpoint{5.068508in}{0.632716in}}%
\pgfpathlineto{\pgfqpoint{5.068508in}{0.629766in}}%
\pgfpathmoveto{\pgfqpoint{5.063967in}{0.632716in}}%
\pgfpathlineto{\pgfqpoint{5.063967in}{0.632716in}}%
\pgfpathlineto{\pgfqpoint{5.063967in}{0.635665in}}%
\pgfpathlineto{\pgfqpoint{5.068508in}{0.635665in}}%
\pgfpathlineto{\pgfqpoint{5.068508in}{0.632716in}}%
\pgfpathmoveto{\pgfqpoint{5.086673in}{0.597326in}}%
\pgfpathlineto{\pgfqpoint{5.086673in}{0.597326in}}%
\pgfpathlineto{\pgfqpoint{5.086673in}{0.600275in}}%
\pgfpathlineto{\pgfqpoint{5.091214in}{0.600275in}}%
\pgfpathlineto{\pgfqpoint{5.091214in}{0.597326in}}%
\pgfpathmoveto{\pgfqpoint{5.082131in}{0.603224in}}%
\pgfpathlineto{\pgfqpoint{5.082131in}{0.603224in}}%
\pgfpathlineto{\pgfqpoint{5.082131in}{0.606174in}}%
\pgfpathlineto{\pgfqpoint{5.086673in}{0.606174in}}%
\pgfpathlineto{\pgfqpoint{5.086673in}{0.603224in}}%
\pgfpathmoveto{\pgfqpoint{5.086673in}{0.600275in}}%
\pgfpathlineto{\pgfqpoint{5.086673in}{0.600275in}}%
\pgfpathlineto{\pgfqpoint{5.086673in}{0.603224in}}%
\pgfpathlineto{\pgfqpoint{5.091214in}{0.603224in}}%
\pgfpathlineto{\pgfqpoint{5.091214in}{0.600275in}}%
\pgfpathmoveto{\pgfqpoint{5.086673in}{0.603224in}}%
\pgfpathlineto{\pgfqpoint{5.086673in}{0.603224in}}%
\pgfpathlineto{\pgfqpoint{5.086673in}{0.606174in}}%
\pgfpathlineto{\pgfqpoint{5.091214in}{0.606174in}}%
\pgfpathlineto{\pgfqpoint{5.091214in}{0.603224in}}%
\pgfpathmoveto{\pgfqpoint{5.077590in}{0.609123in}}%
\pgfpathlineto{\pgfqpoint{5.077590in}{0.609123in}}%
\pgfpathlineto{\pgfqpoint{5.077590in}{0.612072in}}%
\pgfpathlineto{\pgfqpoint{5.082131in}{0.612072in}}%
\pgfpathlineto{\pgfqpoint{5.082131in}{0.609123in}}%
\pgfpathmoveto{\pgfqpoint{5.073049in}{0.615021in}}%
\pgfpathlineto{\pgfqpoint{5.073049in}{0.615021in}}%
\pgfpathlineto{\pgfqpoint{5.073049in}{0.617970in}}%
\pgfpathlineto{\pgfqpoint{5.077590in}{0.617970in}}%
\pgfpathlineto{\pgfqpoint{5.077590in}{0.615021in}}%
\pgfpathmoveto{\pgfqpoint{5.077590in}{0.612072in}}%
\pgfpathlineto{\pgfqpoint{5.077590in}{0.612072in}}%
\pgfpathlineto{\pgfqpoint{5.077590in}{0.615021in}}%
\pgfpathlineto{\pgfqpoint{5.082131in}{0.615021in}}%
\pgfpathlineto{\pgfqpoint{5.082131in}{0.612072in}}%
\pgfpathmoveto{\pgfqpoint{5.077590in}{0.615021in}}%
\pgfpathlineto{\pgfqpoint{5.077590in}{0.615021in}}%
\pgfpathlineto{\pgfqpoint{5.077590in}{0.617970in}}%
\pgfpathlineto{\pgfqpoint{5.082131in}{0.617970in}}%
\pgfpathlineto{\pgfqpoint{5.082131in}{0.615021in}}%
\pgfpathmoveto{\pgfqpoint{5.082131in}{0.606174in}}%
\pgfpathlineto{\pgfqpoint{5.082131in}{0.606174in}}%
\pgfpathlineto{\pgfqpoint{5.082131in}{0.609123in}}%
\pgfpathlineto{\pgfqpoint{5.086673in}{0.609123in}}%
\pgfpathlineto{\pgfqpoint{5.086673in}{0.606174in}}%
\pgfpathmoveto{\pgfqpoint{5.082131in}{0.609123in}}%
\pgfpathlineto{\pgfqpoint{5.082131in}{0.609123in}}%
\pgfpathlineto{\pgfqpoint{5.082131in}{0.612072in}}%
\pgfpathlineto{\pgfqpoint{5.086673in}{0.612072in}}%
\pgfpathlineto{\pgfqpoint{5.086673in}{0.609123in}}%
\pgfpathmoveto{\pgfqpoint{5.091214in}{0.594377in}}%
\pgfpathlineto{\pgfqpoint{5.091214in}{0.594377in}}%
\pgfpathlineto{\pgfqpoint{5.091214in}{0.597326in}}%
\pgfpathlineto{\pgfqpoint{5.095755in}{0.597326in}}%
\pgfpathlineto{\pgfqpoint{5.095755in}{0.594377in}}%
\pgfpathmoveto{\pgfqpoint{5.091214in}{0.597326in}}%
\pgfpathlineto{\pgfqpoint{5.091214in}{0.597326in}}%
\pgfpathlineto{\pgfqpoint{5.091214in}{0.600275in}}%
\pgfpathlineto{\pgfqpoint{5.095755in}{0.600275in}}%
\pgfpathlineto{\pgfqpoint{5.095755in}{0.597326in}}%
\pgfpathmoveto{\pgfqpoint{5.073049in}{0.617970in}}%
\pgfpathlineto{\pgfqpoint{5.073049in}{0.617970in}}%
\pgfpathlineto{\pgfqpoint{5.073049in}{0.620919in}}%
\pgfpathlineto{\pgfqpoint{5.077590in}{0.620919in}}%
\pgfpathlineto{\pgfqpoint{5.077590in}{0.617970in}}%
\pgfpathmoveto{\pgfqpoint{5.073049in}{0.620919in}}%
\pgfpathlineto{\pgfqpoint{5.073049in}{0.620919in}}%
\pgfpathlineto{\pgfqpoint{5.073049in}{0.623868in}}%
\pgfpathlineto{\pgfqpoint{5.077590in}{0.623868in}}%
\pgfpathlineto{\pgfqpoint{5.077590in}{0.620919in}}%
\pgfpathmoveto{\pgfqpoint{5.050343in}{0.644512in}}%
\pgfpathlineto{\pgfqpoint{5.050343in}{0.644512in}}%
\pgfpathlineto{\pgfqpoint{5.050343in}{0.647461in}}%
\pgfpathlineto{\pgfqpoint{5.054884in}{0.647461in}}%
\pgfpathlineto{\pgfqpoint{5.054884in}{0.644512in}}%
\pgfpathmoveto{\pgfqpoint{5.045802in}{0.650410in}}%
\pgfpathlineto{\pgfqpoint{5.045802in}{0.650410in}}%
\pgfpathlineto{\pgfqpoint{5.045802in}{0.653359in}}%
\pgfpathlineto{\pgfqpoint{5.050343in}{0.653359in}}%
\pgfpathlineto{\pgfqpoint{5.050343in}{0.650410in}}%
\pgfpathmoveto{\pgfqpoint{5.050343in}{0.647461in}}%
\pgfpathlineto{\pgfqpoint{5.050343in}{0.647461in}}%
\pgfpathlineto{\pgfqpoint{5.050343in}{0.650410in}}%
\pgfpathlineto{\pgfqpoint{5.054884in}{0.650410in}}%
\pgfpathlineto{\pgfqpoint{5.054884in}{0.647461in}}%
\pgfpathmoveto{\pgfqpoint{5.050343in}{0.650410in}}%
\pgfpathlineto{\pgfqpoint{5.050343in}{0.650410in}}%
\pgfpathlineto{\pgfqpoint{5.050343in}{0.653359in}}%
\pgfpathlineto{\pgfqpoint{5.054884in}{0.653359in}}%
\pgfpathlineto{\pgfqpoint{5.054884in}{0.650410in}}%
\pgfpathmoveto{\pgfqpoint{5.041261in}{0.656308in}}%
\pgfpathlineto{\pgfqpoint{5.041261in}{0.656308in}}%
\pgfpathlineto{\pgfqpoint{5.041261in}{0.659258in}}%
\pgfpathlineto{\pgfqpoint{5.045802in}{0.659258in}}%
\pgfpathlineto{\pgfqpoint{5.045802in}{0.656308in}}%
\pgfpathmoveto{\pgfqpoint{5.036720in}{0.662207in}}%
\pgfpathlineto{\pgfqpoint{5.036720in}{0.662207in}}%
\pgfpathlineto{\pgfqpoint{5.036720in}{0.665156in}}%
\pgfpathlineto{\pgfqpoint{5.041261in}{0.665156in}}%
\pgfpathlineto{\pgfqpoint{5.041261in}{0.662207in}}%
\pgfpathmoveto{\pgfqpoint{5.041261in}{0.659258in}}%
\pgfpathlineto{\pgfqpoint{5.041261in}{0.659258in}}%
\pgfpathlineto{\pgfqpoint{5.041261in}{0.662207in}}%
\pgfpathlineto{\pgfqpoint{5.045802in}{0.662207in}}%
\pgfpathlineto{\pgfqpoint{5.045802in}{0.659258in}}%
\pgfpathmoveto{\pgfqpoint{5.041261in}{0.662207in}}%
\pgfpathlineto{\pgfqpoint{5.041261in}{0.662207in}}%
\pgfpathlineto{\pgfqpoint{5.041261in}{0.665156in}}%
\pgfpathlineto{\pgfqpoint{5.045802in}{0.665156in}}%
\pgfpathlineto{\pgfqpoint{5.045802in}{0.662207in}}%
\pgfpathmoveto{\pgfqpoint{5.045802in}{0.653359in}}%
\pgfpathlineto{\pgfqpoint{5.045802in}{0.653359in}}%
\pgfpathlineto{\pgfqpoint{5.045802in}{0.656308in}}%
\pgfpathlineto{\pgfqpoint{5.050343in}{0.656308in}}%
\pgfpathlineto{\pgfqpoint{5.050343in}{0.653359in}}%
\pgfpathmoveto{\pgfqpoint{5.045802in}{0.656308in}}%
\pgfpathlineto{\pgfqpoint{5.045802in}{0.656308in}}%
\pgfpathlineto{\pgfqpoint{5.045802in}{0.659258in}}%
\pgfpathlineto{\pgfqpoint{5.050343in}{0.659258in}}%
\pgfpathlineto{\pgfqpoint{5.050343in}{0.656308in}}%
\pgfpathmoveto{\pgfqpoint{5.054884in}{0.641563in}}%
\pgfpathlineto{\pgfqpoint{5.054884in}{0.641563in}}%
\pgfpathlineto{\pgfqpoint{5.054884in}{0.644512in}}%
\pgfpathlineto{\pgfqpoint{5.059426in}{0.644512in}}%
\pgfpathlineto{\pgfqpoint{5.059426in}{0.641563in}}%
\pgfpathmoveto{\pgfqpoint{5.054884in}{0.644512in}}%
\pgfpathlineto{\pgfqpoint{5.054884in}{0.644512in}}%
\pgfpathlineto{\pgfqpoint{5.054884in}{0.647461in}}%
\pgfpathlineto{\pgfqpoint{5.059426in}{0.647461in}}%
\pgfpathlineto{\pgfqpoint{5.059426in}{0.644512in}}%
\pgfpathmoveto{\pgfqpoint{5.036720in}{0.665156in}}%
\pgfpathlineto{\pgfqpoint{5.036720in}{0.665156in}}%
\pgfpathlineto{\pgfqpoint{5.036720in}{0.668105in}}%
\pgfpathlineto{\pgfqpoint{5.041261in}{0.668105in}}%
\pgfpathlineto{\pgfqpoint{5.041261in}{0.665156in}}%
\pgfpathmoveto{\pgfqpoint{5.036720in}{0.668105in}}%
\pgfpathlineto{\pgfqpoint{5.036720in}{0.668105in}}%
\pgfpathlineto{\pgfqpoint{5.036720in}{0.671054in}}%
\pgfpathlineto{\pgfqpoint{5.041261in}{0.671054in}}%
\pgfpathlineto{\pgfqpoint{5.041261in}{0.668105in}}%
\pgfpathmoveto{\pgfqpoint{4.995849in}{0.715292in}}%
\pgfpathlineto{\pgfqpoint{4.995849in}{0.715292in}}%
\pgfpathlineto{\pgfqpoint{4.995849in}{0.718241in}}%
\pgfpathlineto{\pgfqpoint{5.000391in}{0.718241in}}%
\pgfpathlineto{\pgfqpoint{5.000391in}{0.715292in}}%
\pgfpathmoveto{\pgfqpoint{4.991308in}{0.721190in}}%
\pgfpathlineto{\pgfqpoint{4.991308in}{0.721190in}}%
\pgfpathlineto{\pgfqpoint{4.991308in}{0.724139in}}%
\pgfpathlineto{\pgfqpoint{4.995849in}{0.724139in}}%
\pgfpathlineto{\pgfqpoint{4.995849in}{0.721190in}}%
\pgfpathmoveto{\pgfqpoint{4.995849in}{0.718241in}}%
\pgfpathlineto{\pgfqpoint{4.995849in}{0.718241in}}%
\pgfpathlineto{\pgfqpoint{4.995849in}{0.721190in}}%
\pgfpathlineto{\pgfqpoint{5.000391in}{0.721190in}}%
\pgfpathlineto{\pgfqpoint{5.000391in}{0.718241in}}%
\pgfpathmoveto{\pgfqpoint{4.995849in}{0.721190in}}%
\pgfpathlineto{\pgfqpoint{4.995849in}{0.721190in}}%
\pgfpathlineto{\pgfqpoint{4.995849in}{0.724139in}}%
\pgfpathlineto{\pgfqpoint{5.000391in}{0.724139in}}%
\pgfpathlineto{\pgfqpoint{5.000391in}{0.721190in}}%
\pgfpathmoveto{\pgfqpoint{4.986767in}{0.727088in}}%
\pgfpathlineto{\pgfqpoint{4.986767in}{0.727088in}}%
\pgfpathlineto{\pgfqpoint{4.986767in}{0.730038in}}%
\pgfpathlineto{\pgfqpoint{4.991308in}{0.730038in}}%
\pgfpathlineto{\pgfqpoint{4.991308in}{0.727088in}}%
\pgfpathmoveto{\pgfqpoint{4.982226in}{0.732987in}}%
\pgfpathlineto{\pgfqpoint{4.982226in}{0.732987in}}%
\pgfpathlineto{\pgfqpoint{4.982226in}{0.735936in}}%
\pgfpathlineto{\pgfqpoint{4.986767in}{0.735936in}}%
\pgfpathlineto{\pgfqpoint{4.986767in}{0.732987in}}%
\pgfpathmoveto{\pgfqpoint{4.986767in}{0.730038in}}%
\pgfpathlineto{\pgfqpoint{4.986767in}{0.730038in}}%
\pgfpathlineto{\pgfqpoint{4.986767in}{0.732987in}}%
\pgfpathlineto{\pgfqpoint{4.991308in}{0.732987in}}%
\pgfpathlineto{\pgfqpoint{4.991308in}{0.730038in}}%
\pgfpathmoveto{\pgfqpoint{4.986767in}{0.732987in}}%
\pgfpathlineto{\pgfqpoint{4.986767in}{0.732987in}}%
\pgfpathlineto{\pgfqpoint{4.986767in}{0.735936in}}%
\pgfpathlineto{\pgfqpoint{4.991308in}{0.735936in}}%
\pgfpathlineto{\pgfqpoint{4.991308in}{0.732987in}}%
\pgfpathmoveto{\pgfqpoint{4.991308in}{0.724139in}}%
\pgfpathlineto{\pgfqpoint{4.991308in}{0.724139in}}%
\pgfpathlineto{\pgfqpoint{4.991308in}{0.727088in}}%
\pgfpathlineto{\pgfqpoint{4.995849in}{0.727088in}}%
\pgfpathlineto{\pgfqpoint{4.995849in}{0.724139in}}%
\pgfpathmoveto{\pgfqpoint{4.991308in}{0.727088in}}%
\pgfpathlineto{\pgfqpoint{4.991308in}{0.727088in}}%
\pgfpathlineto{\pgfqpoint{4.991308in}{0.730038in}}%
\pgfpathlineto{\pgfqpoint{4.995849in}{0.730038in}}%
\pgfpathlineto{\pgfqpoint{4.995849in}{0.727088in}}%
\pgfpathmoveto{\pgfqpoint{5.014014in}{0.691698in}}%
\pgfpathlineto{\pgfqpoint{5.014014in}{0.691698in}}%
\pgfpathlineto{\pgfqpoint{5.014014in}{0.694647in}}%
\pgfpathlineto{\pgfqpoint{5.018555in}{0.694647in}}%
\pgfpathlineto{\pgfqpoint{5.018555in}{0.691698in}}%
\pgfpathmoveto{\pgfqpoint{5.009473in}{0.697596in}}%
\pgfpathlineto{\pgfqpoint{5.009473in}{0.697596in}}%
\pgfpathlineto{\pgfqpoint{5.009473in}{0.700545in}}%
\pgfpathlineto{\pgfqpoint{5.014014in}{0.700545in}}%
\pgfpathlineto{\pgfqpoint{5.014014in}{0.697596in}}%
\pgfpathmoveto{\pgfqpoint{5.014014in}{0.694647in}}%
\pgfpathlineto{\pgfqpoint{5.014014in}{0.694647in}}%
\pgfpathlineto{\pgfqpoint{5.014014in}{0.697596in}}%
\pgfpathlineto{\pgfqpoint{5.018555in}{0.697596in}}%
\pgfpathlineto{\pgfqpoint{5.018555in}{0.694647in}}%
\pgfpathmoveto{\pgfqpoint{5.014014in}{0.697596in}}%
\pgfpathlineto{\pgfqpoint{5.014014in}{0.697596in}}%
\pgfpathlineto{\pgfqpoint{5.014014in}{0.700545in}}%
\pgfpathlineto{\pgfqpoint{5.018555in}{0.700545in}}%
\pgfpathlineto{\pgfqpoint{5.018555in}{0.697596in}}%
\pgfpathmoveto{\pgfqpoint{5.004932in}{0.703495in}}%
\pgfpathlineto{\pgfqpoint{5.004932in}{0.703495in}}%
\pgfpathlineto{\pgfqpoint{5.004932in}{0.706444in}}%
\pgfpathlineto{\pgfqpoint{5.009473in}{0.706444in}}%
\pgfpathlineto{\pgfqpoint{5.009473in}{0.703495in}}%
\pgfpathmoveto{\pgfqpoint{5.000391in}{0.709393in}}%
\pgfpathlineto{\pgfqpoint{5.000391in}{0.709393in}}%
\pgfpathlineto{\pgfqpoint{5.000391in}{0.712342in}}%
\pgfpathlineto{\pgfqpoint{5.004932in}{0.712342in}}%
\pgfpathlineto{\pgfqpoint{5.004932in}{0.709393in}}%
\pgfpathmoveto{\pgfqpoint{5.004932in}{0.706444in}}%
\pgfpathlineto{\pgfqpoint{5.004932in}{0.706444in}}%
\pgfpathlineto{\pgfqpoint{5.004932in}{0.709393in}}%
\pgfpathlineto{\pgfqpoint{5.009473in}{0.709393in}}%
\pgfpathlineto{\pgfqpoint{5.009473in}{0.706444in}}%
\pgfpathmoveto{\pgfqpoint{5.004932in}{0.709393in}}%
\pgfpathlineto{\pgfqpoint{5.004932in}{0.709393in}}%
\pgfpathlineto{\pgfqpoint{5.004932in}{0.712342in}}%
\pgfpathlineto{\pgfqpoint{5.009473in}{0.712342in}}%
\pgfpathlineto{\pgfqpoint{5.009473in}{0.709393in}}%
\pgfpathmoveto{\pgfqpoint{5.009473in}{0.700545in}}%
\pgfpathlineto{\pgfqpoint{5.009473in}{0.700545in}}%
\pgfpathlineto{\pgfqpoint{5.009473in}{0.703495in}}%
\pgfpathlineto{\pgfqpoint{5.014014in}{0.703495in}}%
\pgfpathlineto{\pgfqpoint{5.014014in}{0.700545in}}%
\pgfpathmoveto{\pgfqpoint{5.009473in}{0.703495in}}%
\pgfpathlineto{\pgfqpoint{5.009473in}{0.703495in}}%
\pgfpathlineto{\pgfqpoint{5.009473in}{0.706444in}}%
\pgfpathlineto{\pgfqpoint{5.014014in}{0.706444in}}%
\pgfpathlineto{\pgfqpoint{5.014014in}{0.703495in}}%
\pgfpathmoveto{\pgfqpoint{5.018555in}{0.688749in}}%
\pgfpathlineto{\pgfqpoint{5.018555in}{0.688749in}}%
\pgfpathlineto{\pgfqpoint{5.018555in}{0.691698in}}%
\pgfpathlineto{\pgfqpoint{5.023096in}{0.691698in}}%
\pgfpathlineto{\pgfqpoint{5.023096in}{0.688749in}}%
\pgfpathmoveto{\pgfqpoint{5.018555in}{0.691698in}}%
\pgfpathlineto{\pgfqpoint{5.018555in}{0.691698in}}%
\pgfpathlineto{\pgfqpoint{5.018555in}{0.694647in}}%
\pgfpathlineto{\pgfqpoint{5.023096in}{0.694647in}}%
\pgfpathlineto{\pgfqpoint{5.023096in}{0.691698in}}%
\pgfpathmoveto{\pgfqpoint{5.000391in}{0.712342in}}%
\pgfpathlineto{\pgfqpoint{5.000391in}{0.712342in}}%
\pgfpathlineto{\pgfqpoint{5.000391in}{0.715292in}}%
\pgfpathlineto{\pgfqpoint{5.004932in}{0.715292in}}%
\pgfpathlineto{\pgfqpoint{5.004932in}{0.712342in}}%
\pgfpathmoveto{\pgfqpoint{5.000391in}{0.715292in}}%
\pgfpathlineto{\pgfqpoint{5.000391in}{0.715292in}}%
\pgfpathlineto{\pgfqpoint{5.000391in}{0.718241in}}%
\pgfpathlineto{\pgfqpoint{5.004932in}{0.718241in}}%
\pgfpathlineto{\pgfqpoint{5.004932in}{0.715292in}}%
\pgfpathmoveto{\pgfqpoint{4.977685in}{0.738885in}}%
\pgfpathlineto{\pgfqpoint{4.977685in}{0.738885in}}%
\pgfpathlineto{\pgfqpoint{4.977685in}{0.741834in}}%
\pgfpathlineto{\pgfqpoint{4.982226in}{0.741834in}}%
\pgfpathlineto{\pgfqpoint{4.982226in}{0.738885in}}%
\pgfpathmoveto{\pgfqpoint{4.973144in}{0.744784in}}%
\pgfpathlineto{\pgfqpoint{4.973144in}{0.744784in}}%
\pgfpathlineto{\pgfqpoint{4.973144in}{0.747733in}}%
\pgfpathlineto{\pgfqpoint{4.977685in}{0.747733in}}%
\pgfpathlineto{\pgfqpoint{4.977685in}{0.744784in}}%
\pgfpathmoveto{\pgfqpoint{4.977685in}{0.741834in}}%
\pgfpathlineto{\pgfqpoint{4.977685in}{0.741834in}}%
\pgfpathlineto{\pgfqpoint{4.977685in}{0.744784in}}%
\pgfpathlineto{\pgfqpoint{4.982226in}{0.744784in}}%
\pgfpathlineto{\pgfqpoint{4.982226in}{0.741834in}}%
\pgfpathmoveto{\pgfqpoint{4.977685in}{0.744784in}}%
\pgfpathlineto{\pgfqpoint{4.977685in}{0.744784in}}%
\pgfpathlineto{\pgfqpoint{4.977685in}{0.747733in}}%
\pgfpathlineto{\pgfqpoint{4.982226in}{0.747733in}}%
\pgfpathlineto{\pgfqpoint{4.982226in}{0.744784in}}%
\pgfpathmoveto{\pgfqpoint{4.968602in}{0.750682in}}%
\pgfpathlineto{\pgfqpoint{4.968602in}{0.750682in}}%
\pgfpathlineto{\pgfqpoint{4.968602in}{0.753631in}}%
\pgfpathlineto{\pgfqpoint{4.973144in}{0.753631in}}%
\pgfpathlineto{\pgfqpoint{4.973144in}{0.750682in}}%
\pgfpathmoveto{\pgfqpoint{4.964061in}{0.756581in}}%
\pgfpathlineto{\pgfqpoint{4.964061in}{0.756581in}}%
\pgfpathlineto{\pgfqpoint{4.964061in}{0.759530in}}%
\pgfpathlineto{\pgfqpoint{4.968602in}{0.759530in}}%
\pgfpathlineto{\pgfqpoint{4.968602in}{0.756581in}}%
\pgfpathmoveto{\pgfqpoint{4.968602in}{0.753631in}}%
\pgfpathlineto{\pgfqpoint{4.968602in}{0.753631in}}%
\pgfpathlineto{\pgfqpoint{4.968602in}{0.756581in}}%
\pgfpathlineto{\pgfqpoint{4.973144in}{0.756581in}}%
\pgfpathlineto{\pgfqpoint{4.973144in}{0.753631in}}%
\pgfpathmoveto{\pgfqpoint{4.968602in}{0.756581in}}%
\pgfpathlineto{\pgfqpoint{4.968602in}{0.756581in}}%
\pgfpathlineto{\pgfqpoint{4.968602in}{0.759530in}}%
\pgfpathlineto{\pgfqpoint{4.973144in}{0.759530in}}%
\pgfpathlineto{\pgfqpoint{4.973144in}{0.756581in}}%
\pgfpathmoveto{\pgfqpoint{4.973144in}{0.747733in}}%
\pgfpathlineto{\pgfqpoint{4.973144in}{0.747733in}}%
\pgfpathlineto{\pgfqpoint{4.973144in}{0.750682in}}%
\pgfpathlineto{\pgfqpoint{4.977685in}{0.750682in}}%
\pgfpathlineto{\pgfqpoint{4.977685in}{0.747733in}}%
\pgfpathmoveto{\pgfqpoint{4.973144in}{0.750682in}}%
\pgfpathlineto{\pgfqpoint{4.973144in}{0.750682in}}%
\pgfpathlineto{\pgfqpoint{4.973144in}{0.753631in}}%
\pgfpathlineto{\pgfqpoint{4.977685in}{0.753631in}}%
\pgfpathlineto{\pgfqpoint{4.977685in}{0.750682in}}%
\pgfpathmoveto{\pgfqpoint{4.982226in}{0.735936in}}%
\pgfpathlineto{\pgfqpoint{4.982226in}{0.735936in}}%
\pgfpathlineto{\pgfqpoint{4.982226in}{0.738885in}}%
\pgfpathlineto{\pgfqpoint{4.986767in}{0.738885in}}%
\pgfpathlineto{\pgfqpoint{4.986767in}{0.735936in}}%
\pgfpathmoveto{\pgfqpoint{4.982226in}{0.738885in}}%
\pgfpathlineto{\pgfqpoint{4.982226in}{0.738885in}}%
\pgfpathlineto{\pgfqpoint{4.982226in}{0.741834in}}%
\pgfpathlineto{\pgfqpoint{4.986767in}{0.741834in}}%
\pgfpathlineto{\pgfqpoint{4.986767in}{0.738885in}}%
\pgfpathmoveto{\pgfqpoint{4.964061in}{0.759530in}}%
\pgfpathlineto{\pgfqpoint{4.964061in}{0.759530in}}%
\pgfpathlineto{\pgfqpoint{4.964061in}{0.762479in}}%
\pgfpathlineto{\pgfqpoint{4.968602in}{0.762479in}}%
\pgfpathlineto{\pgfqpoint{4.968602in}{0.759530in}}%
\pgfpathmoveto{\pgfqpoint{4.964061in}{0.762479in}}%
\pgfpathlineto{\pgfqpoint{4.964061in}{0.762479in}}%
\pgfpathlineto{\pgfqpoint{4.964061in}{0.765428in}}%
\pgfpathlineto{\pgfqpoint{4.968602in}{0.765428in}}%
\pgfpathlineto{\pgfqpoint{4.968602in}{0.762479in}}%
\pgfpathmoveto{\pgfqpoint{5.141164in}{0.526542in}}%
\pgfpathlineto{\pgfqpoint{5.141164in}{0.526542in}}%
\pgfpathlineto{\pgfqpoint{5.141164in}{0.529491in}}%
\pgfpathlineto{\pgfqpoint{5.145705in}{0.529491in}}%
\pgfpathlineto{\pgfqpoint{5.145705in}{0.526542in}}%
\pgfpathmoveto{\pgfqpoint{5.136623in}{0.532440in}}%
\pgfpathlineto{\pgfqpoint{5.136623in}{0.532440in}}%
\pgfpathlineto{\pgfqpoint{5.136623in}{0.535390in}}%
\pgfpathlineto{\pgfqpoint{5.141164in}{0.535390in}}%
\pgfpathlineto{\pgfqpoint{5.141164in}{0.532440in}}%
\pgfpathmoveto{\pgfqpoint{5.141164in}{0.529491in}}%
\pgfpathlineto{\pgfqpoint{5.141164in}{0.529491in}}%
\pgfpathlineto{\pgfqpoint{5.141164in}{0.532440in}}%
\pgfpathlineto{\pgfqpoint{5.145705in}{0.532440in}}%
\pgfpathlineto{\pgfqpoint{5.145705in}{0.529491in}}%
\pgfpathmoveto{\pgfqpoint{5.141164in}{0.532440in}}%
\pgfpathlineto{\pgfqpoint{5.141164in}{0.532440in}}%
\pgfpathlineto{\pgfqpoint{5.141164in}{0.535390in}}%
\pgfpathlineto{\pgfqpoint{5.145705in}{0.535390in}}%
\pgfpathlineto{\pgfqpoint{5.145705in}{0.532440in}}%
\pgfpathmoveto{\pgfqpoint{5.132082in}{0.538339in}}%
\pgfpathlineto{\pgfqpoint{5.132082in}{0.538339in}}%
\pgfpathlineto{\pgfqpoint{5.132082in}{0.541288in}}%
\pgfpathlineto{\pgfqpoint{5.136623in}{0.541288in}}%
\pgfpathlineto{\pgfqpoint{5.136623in}{0.538339in}}%
\pgfpathmoveto{\pgfqpoint{5.127542in}{0.544238in}}%
\pgfpathlineto{\pgfqpoint{5.127542in}{0.544238in}}%
\pgfpathlineto{\pgfqpoint{5.127542in}{0.547187in}}%
\pgfpathlineto{\pgfqpoint{5.132082in}{0.547187in}}%
\pgfpathlineto{\pgfqpoint{5.132082in}{0.544238in}}%
\pgfpathmoveto{\pgfqpoint{5.132082in}{0.541288in}}%
\pgfpathlineto{\pgfqpoint{5.132082in}{0.541288in}}%
\pgfpathlineto{\pgfqpoint{5.132082in}{0.544238in}}%
\pgfpathlineto{\pgfqpoint{5.136623in}{0.544238in}}%
\pgfpathlineto{\pgfqpoint{5.136623in}{0.541288in}}%
\pgfpathmoveto{\pgfqpoint{5.132082in}{0.544238in}}%
\pgfpathlineto{\pgfqpoint{5.132082in}{0.544238in}}%
\pgfpathlineto{\pgfqpoint{5.132082in}{0.547187in}}%
\pgfpathlineto{\pgfqpoint{5.136623in}{0.547187in}}%
\pgfpathlineto{\pgfqpoint{5.136623in}{0.544238in}}%
\pgfpathmoveto{\pgfqpoint{5.136623in}{0.535390in}}%
\pgfpathlineto{\pgfqpoint{5.136623in}{0.535390in}}%
\pgfpathlineto{\pgfqpoint{5.136623in}{0.538339in}}%
\pgfpathlineto{\pgfqpoint{5.141164in}{0.538339in}}%
\pgfpathlineto{\pgfqpoint{5.141164in}{0.535390in}}%
\pgfpathmoveto{\pgfqpoint{5.136623in}{0.538339in}}%
\pgfpathlineto{\pgfqpoint{5.136623in}{0.538339in}}%
\pgfpathlineto{\pgfqpoint{5.136623in}{0.541288in}}%
\pgfpathlineto{\pgfqpoint{5.141164in}{0.541288in}}%
\pgfpathlineto{\pgfqpoint{5.141164in}{0.538339in}}%
\pgfpathmoveto{\pgfqpoint{5.159327in}{0.502947in}}%
\pgfpathlineto{\pgfqpoint{5.159327in}{0.502947in}}%
\pgfpathlineto{\pgfqpoint{5.159327in}{0.505896in}}%
\pgfpathlineto{\pgfqpoint{5.163868in}{0.505896in}}%
\pgfpathlineto{\pgfqpoint{5.163868in}{0.502947in}}%
\pgfpathmoveto{\pgfqpoint{5.154786in}{0.508845in}}%
\pgfpathlineto{\pgfqpoint{5.154786in}{0.508845in}}%
\pgfpathlineto{\pgfqpoint{5.154786in}{0.511795in}}%
\pgfpathlineto{\pgfqpoint{5.159327in}{0.511795in}}%
\pgfpathlineto{\pgfqpoint{5.159327in}{0.508845in}}%
\pgfpathmoveto{\pgfqpoint{5.159327in}{0.505896in}}%
\pgfpathlineto{\pgfqpoint{5.159327in}{0.505896in}}%
\pgfpathlineto{\pgfqpoint{5.159327in}{0.508845in}}%
\pgfpathlineto{\pgfqpoint{5.163868in}{0.508845in}}%
\pgfpathlineto{\pgfqpoint{5.163868in}{0.505896in}}%
\pgfpathmoveto{\pgfqpoint{5.159327in}{0.508845in}}%
\pgfpathlineto{\pgfqpoint{5.159327in}{0.508845in}}%
\pgfpathlineto{\pgfqpoint{5.159327in}{0.511795in}}%
\pgfpathlineto{\pgfqpoint{5.163868in}{0.511795in}}%
\pgfpathlineto{\pgfqpoint{5.163868in}{0.508845in}}%
\pgfpathmoveto{\pgfqpoint{5.150246in}{0.514744in}}%
\pgfpathlineto{\pgfqpoint{5.150246in}{0.514744in}}%
\pgfpathlineto{\pgfqpoint{5.150246in}{0.517693in}}%
\pgfpathlineto{\pgfqpoint{5.154786in}{0.517693in}}%
\pgfpathlineto{\pgfqpoint{5.154786in}{0.514744in}}%
\pgfpathmoveto{\pgfqpoint{5.145705in}{0.520643in}}%
\pgfpathlineto{\pgfqpoint{5.145705in}{0.520643in}}%
\pgfpathlineto{\pgfqpoint{5.145705in}{0.523592in}}%
\pgfpathlineto{\pgfqpoint{5.150246in}{0.523592in}}%
\pgfpathlineto{\pgfqpoint{5.150246in}{0.520643in}}%
\pgfpathmoveto{\pgfqpoint{5.150246in}{0.517693in}}%
\pgfpathlineto{\pgfqpoint{5.150246in}{0.517693in}}%
\pgfpathlineto{\pgfqpoint{5.150246in}{0.520643in}}%
\pgfpathlineto{\pgfqpoint{5.154786in}{0.520643in}}%
\pgfpathlineto{\pgfqpoint{5.154786in}{0.517693in}}%
\pgfpathmoveto{\pgfqpoint{5.150246in}{0.520643in}}%
\pgfpathlineto{\pgfqpoint{5.150246in}{0.520643in}}%
\pgfpathlineto{\pgfqpoint{5.150246in}{0.523592in}}%
\pgfpathlineto{\pgfqpoint{5.154786in}{0.523592in}}%
\pgfpathlineto{\pgfqpoint{5.154786in}{0.520643in}}%
\pgfpathmoveto{\pgfqpoint{5.154786in}{0.511795in}}%
\pgfpathlineto{\pgfqpoint{5.154786in}{0.511795in}}%
\pgfpathlineto{\pgfqpoint{5.154786in}{0.514744in}}%
\pgfpathlineto{\pgfqpoint{5.159327in}{0.514744in}}%
\pgfpathlineto{\pgfqpoint{5.159327in}{0.511795in}}%
\pgfpathmoveto{\pgfqpoint{5.154786in}{0.514744in}}%
\pgfpathlineto{\pgfqpoint{5.154786in}{0.514744in}}%
\pgfpathlineto{\pgfqpoint{5.154786in}{0.517693in}}%
\pgfpathlineto{\pgfqpoint{5.159327in}{0.517693in}}%
\pgfpathlineto{\pgfqpoint{5.159327in}{0.514744in}}%
\pgfpathmoveto{\pgfqpoint{5.163868in}{0.499997in}}%
\pgfpathlineto{\pgfqpoint{5.163868in}{0.499997in}}%
\pgfpathlineto{\pgfqpoint{5.163868in}{0.502947in}}%
\pgfpathlineto{\pgfqpoint{5.168409in}{0.502947in}}%
\pgfpathlineto{\pgfqpoint{5.168409in}{0.499997in}}%
\pgfpathmoveto{\pgfqpoint{5.163868in}{0.502947in}}%
\pgfpathlineto{\pgfqpoint{5.163868in}{0.502947in}}%
\pgfpathlineto{\pgfqpoint{5.163868in}{0.505896in}}%
\pgfpathlineto{\pgfqpoint{5.168409in}{0.505896in}}%
\pgfpathlineto{\pgfqpoint{5.168409in}{0.502947in}}%
\pgfpathmoveto{\pgfqpoint{5.145705in}{0.523592in}}%
\pgfpathlineto{\pgfqpoint{5.145705in}{0.523592in}}%
\pgfpathlineto{\pgfqpoint{5.145705in}{0.526542in}}%
\pgfpathlineto{\pgfqpoint{5.150246in}{0.526542in}}%
\pgfpathlineto{\pgfqpoint{5.150246in}{0.523592in}}%
\pgfpathmoveto{\pgfqpoint{5.145705in}{0.526542in}}%
\pgfpathlineto{\pgfqpoint{5.145705in}{0.526542in}}%
\pgfpathlineto{\pgfqpoint{5.145705in}{0.529491in}}%
\pgfpathlineto{\pgfqpoint{5.150246in}{0.529491in}}%
\pgfpathlineto{\pgfqpoint{5.150246in}{0.526542in}}%
\pgfpathmoveto{\pgfqpoint{5.123001in}{0.550137in}}%
\pgfpathlineto{\pgfqpoint{5.123001in}{0.550137in}}%
\pgfpathlineto{\pgfqpoint{5.123001in}{0.553086in}}%
\pgfpathlineto{\pgfqpoint{5.127542in}{0.553086in}}%
\pgfpathlineto{\pgfqpoint{5.127542in}{0.550137in}}%
\pgfpathmoveto{\pgfqpoint{5.118460in}{0.556035in}}%
\pgfpathlineto{\pgfqpoint{5.118460in}{0.556035in}}%
\pgfpathlineto{\pgfqpoint{5.118460in}{0.558985in}}%
\pgfpathlineto{\pgfqpoint{5.123001in}{0.558985in}}%
\pgfpathlineto{\pgfqpoint{5.123001in}{0.556035in}}%
\pgfpathmoveto{\pgfqpoint{5.123001in}{0.553086in}}%
\pgfpathlineto{\pgfqpoint{5.123001in}{0.553086in}}%
\pgfpathlineto{\pgfqpoint{5.123001in}{0.556035in}}%
\pgfpathlineto{\pgfqpoint{5.127542in}{0.556035in}}%
\pgfpathlineto{\pgfqpoint{5.127542in}{0.553086in}}%
\pgfpathmoveto{\pgfqpoint{5.123001in}{0.556035in}}%
\pgfpathlineto{\pgfqpoint{5.123001in}{0.556035in}}%
\pgfpathlineto{\pgfqpoint{5.123001in}{0.558985in}}%
\pgfpathlineto{\pgfqpoint{5.127542in}{0.558985in}}%
\pgfpathlineto{\pgfqpoint{5.127542in}{0.556035in}}%
\pgfpathmoveto{\pgfqpoint{5.113919in}{0.561934in}}%
\pgfpathlineto{\pgfqpoint{5.113919in}{0.561934in}}%
\pgfpathlineto{\pgfqpoint{5.113919in}{0.564883in}}%
\pgfpathlineto{\pgfqpoint{5.118460in}{0.564883in}}%
\pgfpathlineto{\pgfqpoint{5.118460in}{0.561934in}}%
\pgfpathmoveto{\pgfqpoint{5.109378in}{0.567833in}}%
\pgfpathlineto{\pgfqpoint{5.109378in}{0.567833in}}%
\pgfpathlineto{\pgfqpoint{5.109378in}{0.570782in}}%
\pgfpathlineto{\pgfqpoint{5.113919in}{0.570782in}}%
\pgfpathlineto{\pgfqpoint{5.113919in}{0.567833in}}%
\pgfpathmoveto{\pgfqpoint{5.113919in}{0.564883in}}%
\pgfpathlineto{\pgfqpoint{5.113919in}{0.564883in}}%
\pgfpathlineto{\pgfqpoint{5.113919in}{0.567833in}}%
\pgfpathlineto{\pgfqpoint{5.118460in}{0.567833in}}%
\pgfpathlineto{\pgfqpoint{5.118460in}{0.564883in}}%
\pgfpathmoveto{\pgfqpoint{5.113919in}{0.567833in}}%
\pgfpathlineto{\pgfqpoint{5.113919in}{0.567833in}}%
\pgfpathlineto{\pgfqpoint{5.113919in}{0.570782in}}%
\pgfpathlineto{\pgfqpoint{5.118460in}{0.570782in}}%
\pgfpathlineto{\pgfqpoint{5.118460in}{0.567833in}}%
\pgfpathmoveto{\pgfqpoint{5.118460in}{0.558985in}}%
\pgfpathlineto{\pgfqpoint{5.118460in}{0.558985in}}%
\pgfpathlineto{\pgfqpoint{5.118460in}{0.561934in}}%
\pgfpathlineto{\pgfqpoint{5.123001in}{0.561934in}}%
\pgfpathlineto{\pgfqpoint{5.123001in}{0.558985in}}%
\pgfpathmoveto{\pgfqpoint{5.118460in}{0.561934in}}%
\pgfpathlineto{\pgfqpoint{5.118460in}{0.561934in}}%
\pgfpathlineto{\pgfqpoint{5.118460in}{0.564883in}}%
\pgfpathlineto{\pgfqpoint{5.123001in}{0.564883in}}%
\pgfpathlineto{\pgfqpoint{5.123001in}{0.561934in}}%
\pgfpathmoveto{\pgfqpoint{5.127542in}{0.547187in}}%
\pgfpathlineto{\pgfqpoint{5.127542in}{0.547187in}}%
\pgfpathlineto{\pgfqpoint{5.127542in}{0.550137in}}%
\pgfpathlineto{\pgfqpoint{5.132082in}{0.550137in}}%
\pgfpathlineto{\pgfqpoint{5.132082in}{0.547187in}}%
\pgfpathmoveto{\pgfqpoint{5.127542in}{0.550137in}}%
\pgfpathlineto{\pgfqpoint{5.127542in}{0.550137in}}%
\pgfpathlineto{\pgfqpoint{5.127542in}{0.553086in}}%
\pgfpathlineto{\pgfqpoint{5.132082in}{0.553086in}}%
\pgfpathlineto{\pgfqpoint{5.132082in}{0.550137in}}%
\pgfpathmoveto{\pgfqpoint{5.109378in}{0.570782in}}%
\pgfpathlineto{\pgfqpoint{5.109378in}{0.570782in}}%
\pgfpathlineto{\pgfqpoint{5.109378in}{0.573732in}}%
\pgfpathlineto{\pgfqpoint{5.113919in}{0.573732in}}%
\pgfpathlineto{\pgfqpoint{5.113919in}{0.570782in}}%
\pgfpathmoveto{\pgfqpoint{5.109378in}{0.573732in}}%
\pgfpathlineto{\pgfqpoint{5.109378in}{0.573732in}}%
\pgfpathlineto{\pgfqpoint{5.109378in}{0.576681in}}%
\pgfpathlineto{\pgfqpoint{5.113919in}{0.576681in}}%
\pgfpathlineto{\pgfqpoint{5.113919in}{0.573732in}}%
\pgfpathclose%
\pgfusepath{fill}%
\end{pgfscope}%
\begin{pgfscope}%
\pgfpathrectangle{\pgfqpoint{0.750000in}{0.500000in}}{\pgfqpoint{4.650000in}{3.020000in}}%
\pgfusepath{clip}%
\pgfsetbuttcap%
\pgfsetmiterjoin%
\definecolor{currentfill}{rgb}{1.000000,0.000000,0.000000}%
\pgfsetfillcolor{currentfill}%
\pgfsetlinewidth{0.000000pt}%
\definecolor{currentstroke}{rgb}{0.000000,0.000000,0.000000}%
\pgfsetstrokecolor{currentstroke}%
\pgfsetstrokeopacity{0.000000}%
\pgfsetdash{}{0pt}%
\pgfpathmoveto{\pgfqpoint{0.750002in}{0.499998in}}%
\pgfpathlineto{\pgfqpoint{0.750002in}{0.502947in}}%
\pgfpathlineto{\pgfqpoint{0.754543in}{0.502947in}}%
\pgfpathlineto{\pgfqpoint{0.754543in}{0.499998in}}%
\pgfpathmoveto{\pgfqpoint{0.750002in}{0.502947in}}%
\pgfpathlineto{\pgfqpoint{0.750002in}{0.502947in}}%
\pgfpathlineto{\pgfqpoint{0.750002in}{0.505897in}}%
\pgfpathlineto{\pgfqpoint{0.754543in}{0.505897in}}%
\pgfpathlineto{\pgfqpoint{0.754543in}{0.502947in}}%
\pgfpathmoveto{\pgfqpoint{0.754543in}{0.502947in}}%
\pgfpathlineto{\pgfqpoint{0.754543in}{0.502947in}}%
\pgfpathlineto{\pgfqpoint{0.754543in}{0.505897in}}%
\pgfpathlineto{\pgfqpoint{0.759084in}{0.505897in}}%
\pgfpathlineto{\pgfqpoint{0.759084in}{0.502947in}}%
\pgfpathmoveto{\pgfqpoint{0.754543in}{0.505897in}}%
\pgfpathlineto{\pgfqpoint{0.754543in}{0.505897in}}%
\pgfpathlineto{\pgfqpoint{0.754543in}{0.508846in}}%
\pgfpathlineto{\pgfqpoint{0.759084in}{0.508846in}}%
\pgfpathlineto{\pgfqpoint{0.759084in}{0.505897in}}%
\pgfpathmoveto{\pgfqpoint{0.759084in}{0.505897in}}%
\pgfpathlineto{\pgfqpoint{0.759084in}{0.505897in}}%
\pgfpathlineto{\pgfqpoint{0.759084in}{0.508846in}}%
\pgfpathlineto{\pgfqpoint{0.763624in}{0.508846in}}%
\pgfpathlineto{\pgfqpoint{0.763624in}{0.505897in}}%
\pgfpathmoveto{\pgfqpoint{0.759084in}{0.508846in}}%
\pgfpathlineto{\pgfqpoint{0.759084in}{0.508846in}}%
\pgfpathlineto{\pgfqpoint{0.759084in}{0.511795in}}%
\pgfpathlineto{\pgfqpoint{0.763624in}{0.511795in}}%
\pgfpathlineto{\pgfqpoint{0.763624in}{0.508846in}}%
\pgfpathmoveto{\pgfqpoint{0.763624in}{0.508846in}}%
\pgfpathlineto{\pgfqpoint{0.763624in}{0.508846in}}%
\pgfpathlineto{\pgfqpoint{0.763624in}{0.511795in}}%
\pgfpathlineto{\pgfqpoint{0.768165in}{0.511795in}}%
\pgfpathlineto{\pgfqpoint{0.768165in}{0.508846in}}%
\pgfpathmoveto{\pgfqpoint{0.763624in}{0.511795in}}%
\pgfpathlineto{\pgfqpoint{0.763624in}{0.511795in}}%
\pgfpathlineto{\pgfqpoint{0.763624in}{0.514745in}}%
\pgfpathlineto{\pgfqpoint{0.768165in}{0.514745in}}%
\pgfpathlineto{\pgfqpoint{0.768165in}{0.511795in}}%
\pgfpathmoveto{\pgfqpoint{0.768165in}{0.511795in}}%
\pgfpathlineto{\pgfqpoint{0.768165in}{0.511795in}}%
\pgfpathlineto{\pgfqpoint{0.768165in}{0.514745in}}%
\pgfpathlineto{\pgfqpoint{0.772706in}{0.514745in}}%
\pgfpathlineto{\pgfqpoint{0.772706in}{0.511795in}}%
\pgfpathmoveto{\pgfqpoint{0.768165in}{0.514745in}}%
\pgfpathlineto{\pgfqpoint{0.768165in}{0.514745in}}%
\pgfpathlineto{\pgfqpoint{0.768165in}{0.517694in}}%
\pgfpathlineto{\pgfqpoint{0.772706in}{0.517694in}}%
\pgfpathlineto{\pgfqpoint{0.772706in}{0.514745in}}%
\pgfpathmoveto{\pgfqpoint{0.772706in}{0.514745in}}%
\pgfpathlineto{\pgfqpoint{0.772706in}{0.514745in}}%
\pgfpathlineto{\pgfqpoint{0.772706in}{0.517694in}}%
\pgfpathlineto{\pgfqpoint{0.777247in}{0.517694in}}%
\pgfpathlineto{\pgfqpoint{0.777247in}{0.514745in}}%
\pgfpathmoveto{\pgfqpoint{0.772706in}{0.517694in}}%
\pgfpathlineto{\pgfqpoint{0.772706in}{0.517694in}}%
\pgfpathlineto{\pgfqpoint{0.772706in}{0.520643in}}%
\pgfpathlineto{\pgfqpoint{0.777247in}{0.520643in}}%
\pgfpathlineto{\pgfqpoint{0.777247in}{0.517694in}}%
\pgfpathmoveto{\pgfqpoint{0.777247in}{0.517694in}}%
\pgfpathlineto{\pgfqpoint{0.777247in}{0.517694in}}%
\pgfpathlineto{\pgfqpoint{0.777247in}{0.520643in}}%
\pgfpathlineto{\pgfqpoint{0.781788in}{0.520643in}}%
\pgfpathlineto{\pgfqpoint{0.781788in}{0.517694in}}%
\pgfpathmoveto{\pgfqpoint{0.777247in}{0.520643in}}%
\pgfpathlineto{\pgfqpoint{0.777247in}{0.520643in}}%
\pgfpathlineto{\pgfqpoint{0.777247in}{0.523593in}}%
\pgfpathlineto{\pgfqpoint{0.781788in}{0.523593in}}%
\pgfpathlineto{\pgfqpoint{0.781788in}{0.520643in}}%
\pgfpathmoveto{\pgfqpoint{0.781788in}{0.520643in}}%
\pgfpathlineto{\pgfqpoint{0.781788in}{0.520643in}}%
\pgfpathlineto{\pgfqpoint{0.781788in}{0.523593in}}%
\pgfpathlineto{\pgfqpoint{0.786329in}{0.523593in}}%
\pgfpathlineto{\pgfqpoint{0.786329in}{0.520643in}}%
\pgfpathmoveto{\pgfqpoint{0.781788in}{0.523593in}}%
\pgfpathlineto{\pgfqpoint{0.781788in}{0.523593in}}%
\pgfpathlineto{\pgfqpoint{0.781788in}{0.526542in}}%
\pgfpathlineto{\pgfqpoint{0.786329in}{0.526542in}}%
\pgfpathlineto{\pgfqpoint{0.786329in}{0.523593in}}%
\pgfpathmoveto{\pgfqpoint{0.786329in}{0.523593in}}%
\pgfpathlineto{\pgfqpoint{0.786329in}{0.523593in}}%
\pgfpathlineto{\pgfqpoint{0.786329in}{0.526542in}}%
\pgfpathlineto{\pgfqpoint{0.790870in}{0.526542in}}%
\pgfpathlineto{\pgfqpoint{0.790870in}{0.523593in}}%
\pgfpathmoveto{\pgfqpoint{0.786329in}{0.526542in}}%
\pgfpathlineto{\pgfqpoint{0.786329in}{0.526542in}}%
\pgfpathlineto{\pgfqpoint{0.786329in}{0.529492in}}%
\pgfpathlineto{\pgfqpoint{0.790870in}{0.529492in}}%
\pgfpathlineto{\pgfqpoint{0.790870in}{0.526542in}}%
\pgfpathmoveto{\pgfqpoint{0.790870in}{0.526542in}}%
\pgfpathlineto{\pgfqpoint{0.790870in}{0.526542in}}%
\pgfpathlineto{\pgfqpoint{0.790870in}{0.529492in}}%
\pgfpathlineto{\pgfqpoint{0.795411in}{0.529492in}}%
\pgfpathlineto{\pgfqpoint{0.795411in}{0.526542in}}%
\pgfpathmoveto{\pgfqpoint{0.790870in}{0.529492in}}%
\pgfpathlineto{\pgfqpoint{0.790870in}{0.529492in}}%
\pgfpathlineto{\pgfqpoint{0.790870in}{0.532441in}}%
\pgfpathlineto{\pgfqpoint{0.795411in}{0.532441in}}%
\pgfpathlineto{\pgfqpoint{0.795411in}{0.529492in}}%
\pgfpathmoveto{\pgfqpoint{0.795411in}{0.529492in}}%
\pgfpathlineto{\pgfqpoint{0.795411in}{0.529492in}}%
\pgfpathlineto{\pgfqpoint{0.795411in}{0.532441in}}%
\pgfpathlineto{\pgfqpoint{0.799951in}{0.532441in}}%
\pgfpathlineto{\pgfqpoint{0.799951in}{0.529492in}}%
\pgfpathmoveto{\pgfqpoint{0.795411in}{0.532441in}}%
\pgfpathlineto{\pgfqpoint{0.795411in}{0.532441in}}%
\pgfpathlineto{\pgfqpoint{0.795411in}{0.535390in}}%
\pgfpathlineto{\pgfqpoint{0.799951in}{0.535390in}}%
\pgfpathlineto{\pgfqpoint{0.799951in}{0.532441in}}%
\pgfpathmoveto{\pgfqpoint{0.799951in}{0.532441in}}%
\pgfpathlineto{\pgfqpoint{0.799951in}{0.532441in}}%
\pgfpathlineto{\pgfqpoint{0.799951in}{0.535390in}}%
\pgfpathlineto{\pgfqpoint{0.804492in}{0.535390in}}%
\pgfpathlineto{\pgfqpoint{0.804492in}{0.532441in}}%
\pgfpathmoveto{\pgfqpoint{0.799951in}{0.535390in}}%
\pgfpathlineto{\pgfqpoint{0.799951in}{0.535390in}}%
\pgfpathlineto{\pgfqpoint{0.799951in}{0.538340in}}%
\pgfpathlineto{\pgfqpoint{0.804492in}{0.538340in}}%
\pgfpathlineto{\pgfqpoint{0.804492in}{0.535390in}}%
\pgfpathmoveto{\pgfqpoint{0.804492in}{0.535390in}}%
\pgfpathlineto{\pgfqpoint{0.804492in}{0.535390in}}%
\pgfpathlineto{\pgfqpoint{0.804492in}{0.538340in}}%
\pgfpathlineto{\pgfqpoint{0.809033in}{0.538340in}}%
\pgfpathlineto{\pgfqpoint{0.809033in}{0.535390in}}%
\pgfpathmoveto{\pgfqpoint{0.804492in}{0.538340in}}%
\pgfpathlineto{\pgfqpoint{0.804492in}{0.538340in}}%
\pgfpathlineto{\pgfqpoint{0.804492in}{0.541289in}}%
\pgfpathlineto{\pgfqpoint{0.809033in}{0.541289in}}%
\pgfpathlineto{\pgfqpoint{0.809033in}{0.538340in}}%
\pgfpathmoveto{\pgfqpoint{0.809033in}{0.538340in}}%
\pgfpathlineto{\pgfqpoint{0.809033in}{0.538340in}}%
\pgfpathlineto{\pgfqpoint{0.809033in}{0.541289in}}%
\pgfpathlineto{\pgfqpoint{0.813574in}{0.541289in}}%
\pgfpathlineto{\pgfqpoint{0.813574in}{0.538340in}}%
\pgfpathmoveto{\pgfqpoint{0.809033in}{0.541289in}}%
\pgfpathlineto{\pgfqpoint{0.809033in}{0.541289in}}%
\pgfpathlineto{\pgfqpoint{0.809033in}{0.544238in}}%
\pgfpathlineto{\pgfqpoint{0.813574in}{0.544238in}}%
\pgfpathlineto{\pgfqpoint{0.813574in}{0.541289in}}%
\pgfpathmoveto{\pgfqpoint{0.813574in}{0.541289in}}%
\pgfpathlineto{\pgfqpoint{0.813574in}{0.541289in}}%
\pgfpathlineto{\pgfqpoint{0.813574in}{0.544238in}}%
\pgfpathlineto{\pgfqpoint{0.818115in}{0.544238in}}%
\pgfpathlineto{\pgfqpoint{0.818115in}{0.541289in}}%
\pgfpathmoveto{\pgfqpoint{0.818115in}{0.541289in}}%
\pgfpathlineto{\pgfqpoint{0.818115in}{0.541289in}}%
\pgfpathlineto{\pgfqpoint{0.818115in}{0.544238in}}%
\pgfpathlineto{\pgfqpoint{0.822656in}{0.544238in}}%
\pgfpathlineto{\pgfqpoint{0.822656in}{0.541289in}}%
\pgfpathmoveto{\pgfqpoint{0.818115in}{0.544238in}}%
\pgfpathlineto{\pgfqpoint{0.818115in}{0.544238in}}%
\pgfpathlineto{\pgfqpoint{0.818115in}{0.547188in}}%
\pgfpathlineto{\pgfqpoint{0.822656in}{0.547188in}}%
\pgfpathlineto{\pgfqpoint{0.822656in}{0.544238in}}%
\pgfpathmoveto{\pgfqpoint{0.822656in}{0.544238in}}%
\pgfpathlineto{\pgfqpoint{0.822656in}{0.544238in}}%
\pgfpathlineto{\pgfqpoint{0.822656in}{0.547188in}}%
\pgfpathlineto{\pgfqpoint{0.827197in}{0.547188in}}%
\pgfpathlineto{\pgfqpoint{0.827197in}{0.544238in}}%
\pgfpathmoveto{\pgfqpoint{0.822656in}{0.547188in}}%
\pgfpathlineto{\pgfqpoint{0.822656in}{0.547188in}}%
\pgfpathlineto{\pgfqpoint{0.822656in}{0.550137in}}%
\pgfpathlineto{\pgfqpoint{0.827197in}{0.550137in}}%
\pgfpathlineto{\pgfqpoint{0.827197in}{0.547188in}}%
\pgfpathmoveto{\pgfqpoint{0.827197in}{0.547188in}}%
\pgfpathlineto{\pgfqpoint{0.827197in}{0.547188in}}%
\pgfpathlineto{\pgfqpoint{0.827197in}{0.550137in}}%
\pgfpathlineto{\pgfqpoint{0.831738in}{0.550137in}}%
\pgfpathlineto{\pgfqpoint{0.831738in}{0.547188in}}%
\pgfpathmoveto{\pgfqpoint{0.827197in}{0.550137in}}%
\pgfpathlineto{\pgfqpoint{0.827197in}{0.550137in}}%
\pgfpathlineto{\pgfqpoint{0.827197in}{0.553086in}}%
\pgfpathlineto{\pgfqpoint{0.831738in}{0.553086in}}%
\pgfpathlineto{\pgfqpoint{0.831738in}{0.550137in}}%
\pgfpathmoveto{\pgfqpoint{0.831738in}{0.550137in}}%
\pgfpathlineto{\pgfqpoint{0.831738in}{0.550137in}}%
\pgfpathlineto{\pgfqpoint{0.831738in}{0.553086in}}%
\pgfpathlineto{\pgfqpoint{0.836278in}{0.553086in}}%
\pgfpathlineto{\pgfqpoint{0.836278in}{0.550137in}}%
\pgfpathmoveto{\pgfqpoint{0.831738in}{0.553086in}}%
\pgfpathlineto{\pgfqpoint{0.831738in}{0.553086in}}%
\pgfpathlineto{\pgfqpoint{0.831738in}{0.556036in}}%
\pgfpathlineto{\pgfqpoint{0.836278in}{0.556036in}}%
\pgfpathlineto{\pgfqpoint{0.836278in}{0.553086in}}%
\pgfpathmoveto{\pgfqpoint{0.836278in}{0.553086in}}%
\pgfpathlineto{\pgfqpoint{0.836278in}{0.553086in}}%
\pgfpathlineto{\pgfqpoint{0.836278in}{0.556036in}}%
\pgfpathlineto{\pgfqpoint{0.840819in}{0.556036in}}%
\pgfpathlineto{\pgfqpoint{0.840819in}{0.553086in}}%
\pgfpathmoveto{\pgfqpoint{0.836278in}{0.556036in}}%
\pgfpathlineto{\pgfqpoint{0.836278in}{0.556036in}}%
\pgfpathlineto{\pgfqpoint{0.836278in}{0.558985in}}%
\pgfpathlineto{\pgfqpoint{0.840819in}{0.558985in}}%
\pgfpathlineto{\pgfqpoint{0.840819in}{0.556036in}}%
\pgfpathmoveto{\pgfqpoint{0.840819in}{0.556036in}}%
\pgfpathlineto{\pgfqpoint{0.840819in}{0.556036in}}%
\pgfpathlineto{\pgfqpoint{0.840819in}{0.558985in}}%
\pgfpathlineto{\pgfqpoint{0.845360in}{0.558985in}}%
\pgfpathlineto{\pgfqpoint{0.845360in}{0.556036in}}%
\pgfpathmoveto{\pgfqpoint{0.840819in}{0.558985in}}%
\pgfpathlineto{\pgfqpoint{0.840819in}{0.558985in}}%
\pgfpathlineto{\pgfqpoint{0.840819in}{0.561934in}}%
\pgfpathlineto{\pgfqpoint{0.845360in}{0.561934in}}%
\pgfpathlineto{\pgfqpoint{0.845360in}{0.558985in}}%
\pgfpathmoveto{\pgfqpoint{0.845360in}{0.558985in}}%
\pgfpathlineto{\pgfqpoint{0.845360in}{0.558985in}}%
\pgfpathlineto{\pgfqpoint{0.845360in}{0.561934in}}%
\pgfpathlineto{\pgfqpoint{0.849901in}{0.561934in}}%
\pgfpathlineto{\pgfqpoint{0.849901in}{0.558985in}}%
\pgfpathmoveto{\pgfqpoint{0.845360in}{0.561934in}}%
\pgfpathlineto{\pgfqpoint{0.845360in}{0.561934in}}%
\pgfpathlineto{\pgfqpoint{0.845360in}{0.564884in}}%
\pgfpathlineto{\pgfqpoint{0.849901in}{0.564884in}}%
\pgfpathlineto{\pgfqpoint{0.849901in}{0.561934in}}%
\pgfpathmoveto{\pgfqpoint{0.849901in}{0.561934in}}%
\pgfpathlineto{\pgfqpoint{0.849901in}{0.561934in}}%
\pgfpathlineto{\pgfqpoint{0.849901in}{0.564884in}}%
\pgfpathlineto{\pgfqpoint{0.854442in}{0.564884in}}%
\pgfpathlineto{\pgfqpoint{0.854442in}{0.561934in}}%
\pgfpathmoveto{\pgfqpoint{0.849901in}{0.564884in}}%
\pgfpathlineto{\pgfqpoint{0.849901in}{0.564884in}}%
\pgfpathlineto{\pgfqpoint{0.849901in}{0.567833in}}%
\pgfpathlineto{\pgfqpoint{0.854442in}{0.567833in}}%
\pgfpathlineto{\pgfqpoint{0.854442in}{0.564884in}}%
\pgfpathmoveto{\pgfqpoint{0.854442in}{0.564884in}}%
\pgfpathlineto{\pgfqpoint{0.854442in}{0.564884in}}%
\pgfpathlineto{\pgfqpoint{0.854442in}{0.567833in}}%
\pgfpathlineto{\pgfqpoint{0.858983in}{0.567833in}}%
\pgfpathlineto{\pgfqpoint{0.858983in}{0.564884in}}%
\pgfpathmoveto{\pgfqpoint{0.854442in}{0.567833in}}%
\pgfpathlineto{\pgfqpoint{0.854442in}{0.567833in}}%
\pgfpathlineto{\pgfqpoint{0.854442in}{0.570782in}}%
\pgfpathlineto{\pgfqpoint{0.858983in}{0.570782in}}%
\pgfpathlineto{\pgfqpoint{0.858983in}{0.567833in}}%
\pgfpathmoveto{\pgfqpoint{0.858983in}{0.567833in}}%
\pgfpathlineto{\pgfqpoint{0.858983in}{0.567833in}}%
\pgfpathlineto{\pgfqpoint{0.858983in}{0.570782in}}%
\pgfpathlineto{\pgfqpoint{0.863524in}{0.570782in}}%
\pgfpathlineto{\pgfqpoint{0.863524in}{0.567833in}}%
\pgfpathmoveto{\pgfqpoint{0.858983in}{0.570782in}}%
\pgfpathlineto{\pgfqpoint{0.858983in}{0.570782in}}%
\pgfpathlineto{\pgfqpoint{0.858983in}{0.573732in}}%
\pgfpathlineto{\pgfqpoint{0.863524in}{0.573732in}}%
\pgfpathlineto{\pgfqpoint{0.863524in}{0.570782in}}%
\pgfpathmoveto{\pgfqpoint{0.863524in}{0.570782in}}%
\pgfpathlineto{\pgfqpoint{0.863524in}{0.570782in}}%
\pgfpathlineto{\pgfqpoint{0.863524in}{0.573732in}}%
\pgfpathlineto{\pgfqpoint{0.868065in}{0.573732in}}%
\pgfpathlineto{\pgfqpoint{0.868065in}{0.570782in}}%
\pgfpathmoveto{\pgfqpoint{0.863524in}{0.573732in}}%
\pgfpathlineto{\pgfqpoint{0.863524in}{0.573732in}}%
\pgfpathlineto{\pgfqpoint{0.863524in}{0.576681in}}%
\pgfpathlineto{\pgfqpoint{0.868065in}{0.576681in}}%
\pgfpathlineto{\pgfqpoint{0.868065in}{0.573732in}}%
\pgfpathmoveto{\pgfqpoint{0.868065in}{0.573732in}}%
\pgfpathlineto{\pgfqpoint{0.868065in}{0.573732in}}%
\pgfpathlineto{\pgfqpoint{0.868065in}{0.576681in}}%
\pgfpathlineto{\pgfqpoint{0.872605in}{0.576681in}}%
\pgfpathlineto{\pgfqpoint{0.872605in}{0.573732in}}%
\pgfpathmoveto{\pgfqpoint{0.868065in}{0.576681in}}%
\pgfpathlineto{\pgfqpoint{0.868065in}{0.576681in}}%
\pgfpathlineto{\pgfqpoint{0.868065in}{0.579630in}}%
\pgfpathlineto{\pgfqpoint{0.872605in}{0.579630in}}%
\pgfpathlineto{\pgfqpoint{0.872605in}{0.576681in}}%
\pgfpathmoveto{\pgfqpoint{0.872605in}{0.576681in}}%
\pgfpathlineto{\pgfqpoint{0.872605in}{0.576681in}}%
\pgfpathlineto{\pgfqpoint{0.872605in}{0.579630in}}%
\pgfpathlineto{\pgfqpoint{0.877146in}{0.579630in}}%
\pgfpathlineto{\pgfqpoint{0.877146in}{0.576681in}}%
\pgfpathmoveto{\pgfqpoint{0.872605in}{0.579630in}}%
\pgfpathlineto{\pgfqpoint{0.872605in}{0.579630in}}%
\pgfpathlineto{\pgfqpoint{0.872605in}{0.582580in}}%
\pgfpathlineto{\pgfqpoint{0.877146in}{0.582580in}}%
\pgfpathlineto{\pgfqpoint{0.877146in}{0.579630in}}%
\pgfpathmoveto{\pgfqpoint{0.877146in}{0.579630in}}%
\pgfpathlineto{\pgfqpoint{0.877146in}{0.579630in}}%
\pgfpathlineto{\pgfqpoint{0.877146in}{0.582580in}}%
\pgfpathlineto{\pgfqpoint{0.881687in}{0.582580in}}%
\pgfpathlineto{\pgfqpoint{0.881687in}{0.579630in}}%
\pgfpathmoveto{\pgfqpoint{0.877146in}{0.582580in}}%
\pgfpathlineto{\pgfqpoint{0.877146in}{0.582580in}}%
\pgfpathlineto{\pgfqpoint{0.877146in}{0.585529in}}%
\pgfpathlineto{\pgfqpoint{0.881687in}{0.585529in}}%
\pgfpathlineto{\pgfqpoint{0.881687in}{0.582580in}}%
\pgfpathmoveto{\pgfqpoint{0.881687in}{0.582580in}}%
\pgfpathlineto{\pgfqpoint{0.881687in}{0.582580in}}%
\pgfpathlineto{\pgfqpoint{0.881687in}{0.585529in}}%
\pgfpathlineto{\pgfqpoint{0.886228in}{0.585529in}}%
\pgfpathlineto{\pgfqpoint{0.886228in}{0.582580in}}%
\pgfpathmoveto{\pgfqpoint{0.881687in}{0.585529in}}%
\pgfpathlineto{\pgfqpoint{0.881687in}{0.585529in}}%
\pgfpathlineto{\pgfqpoint{0.881687in}{0.588479in}}%
\pgfpathlineto{\pgfqpoint{0.886228in}{0.588479in}}%
\pgfpathlineto{\pgfqpoint{0.886228in}{0.585529in}}%
\pgfpathmoveto{\pgfqpoint{0.886228in}{0.585529in}}%
\pgfpathlineto{\pgfqpoint{0.886228in}{0.585529in}}%
\pgfpathlineto{\pgfqpoint{0.886228in}{0.588479in}}%
\pgfpathlineto{\pgfqpoint{0.890769in}{0.588479in}}%
\pgfpathlineto{\pgfqpoint{0.890769in}{0.585529in}}%
\pgfpathmoveto{\pgfqpoint{0.886228in}{0.588479in}}%
\pgfpathlineto{\pgfqpoint{0.886228in}{0.588479in}}%
\pgfpathlineto{\pgfqpoint{0.886228in}{0.591428in}}%
\pgfpathlineto{\pgfqpoint{0.890769in}{0.591428in}}%
\pgfpathlineto{\pgfqpoint{0.890769in}{0.588479in}}%
\pgfpathmoveto{\pgfqpoint{0.890769in}{0.588479in}}%
\pgfpathlineto{\pgfqpoint{0.890769in}{0.588479in}}%
\pgfpathlineto{\pgfqpoint{0.890769in}{0.591428in}}%
\pgfpathlineto{\pgfqpoint{0.895310in}{0.591428in}}%
\pgfpathlineto{\pgfqpoint{0.895310in}{0.588479in}}%
\pgfpathmoveto{\pgfqpoint{0.890769in}{0.591428in}}%
\pgfpathlineto{\pgfqpoint{0.890769in}{0.591428in}}%
\pgfpathlineto{\pgfqpoint{0.890769in}{0.594377in}}%
\pgfpathlineto{\pgfqpoint{0.895310in}{0.594377in}}%
\pgfpathlineto{\pgfqpoint{0.895310in}{0.591428in}}%
\pgfpathmoveto{\pgfqpoint{0.895310in}{0.591428in}}%
\pgfpathlineto{\pgfqpoint{0.895310in}{0.591428in}}%
\pgfpathlineto{\pgfqpoint{0.895310in}{0.594377in}}%
\pgfpathlineto{\pgfqpoint{0.899851in}{0.594377in}}%
\pgfpathlineto{\pgfqpoint{0.899851in}{0.591428in}}%
\pgfpathmoveto{\pgfqpoint{0.895310in}{0.594377in}}%
\pgfpathlineto{\pgfqpoint{0.895310in}{0.594377in}}%
\pgfpathlineto{\pgfqpoint{0.895310in}{0.597326in}}%
\pgfpathlineto{\pgfqpoint{0.899851in}{0.597326in}}%
\pgfpathlineto{\pgfqpoint{0.899851in}{0.594377in}}%
\pgfpathmoveto{\pgfqpoint{0.899851in}{0.594377in}}%
\pgfpathlineto{\pgfqpoint{0.899851in}{0.594377in}}%
\pgfpathlineto{\pgfqpoint{0.899851in}{0.597326in}}%
\pgfpathlineto{\pgfqpoint{0.904392in}{0.597326in}}%
\pgfpathlineto{\pgfqpoint{0.904392in}{0.594377in}}%
\pgfpathmoveto{\pgfqpoint{0.899851in}{0.597326in}}%
\pgfpathlineto{\pgfqpoint{0.899851in}{0.597326in}}%
\pgfpathlineto{\pgfqpoint{0.899851in}{0.600276in}}%
\pgfpathlineto{\pgfqpoint{0.904392in}{0.600276in}}%
\pgfpathlineto{\pgfqpoint{0.904392in}{0.597326in}}%
\pgfpathmoveto{\pgfqpoint{0.904392in}{0.597326in}}%
\pgfpathlineto{\pgfqpoint{0.904392in}{0.597326in}}%
\pgfpathlineto{\pgfqpoint{0.904392in}{0.600276in}}%
\pgfpathlineto{\pgfqpoint{0.908933in}{0.600276in}}%
\pgfpathlineto{\pgfqpoint{0.908933in}{0.597326in}}%
\pgfpathmoveto{\pgfqpoint{0.904392in}{0.600276in}}%
\pgfpathlineto{\pgfqpoint{0.904392in}{0.600276in}}%
\pgfpathlineto{\pgfqpoint{0.904392in}{0.603225in}}%
\pgfpathlineto{\pgfqpoint{0.908933in}{0.603225in}}%
\pgfpathlineto{\pgfqpoint{0.908933in}{0.600276in}}%
\pgfpathmoveto{\pgfqpoint{0.908933in}{0.600276in}}%
\pgfpathlineto{\pgfqpoint{0.908933in}{0.600276in}}%
\pgfpathlineto{\pgfqpoint{0.908933in}{0.603225in}}%
\pgfpathlineto{\pgfqpoint{0.913474in}{0.603225in}}%
\pgfpathlineto{\pgfqpoint{0.913474in}{0.600276in}}%
\pgfpathmoveto{\pgfqpoint{0.908933in}{0.603225in}}%
\pgfpathlineto{\pgfqpoint{0.908933in}{0.603225in}}%
\pgfpathlineto{\pgfqpoint{0.908933in}{0.606174in}}%
\pgfpathlineto{\pgfqpoint{0.913474in}{0.606174in}}%
\pgfpathlineto{\pgfqpoint{0.913474in}{0.603225in}}%
\pgfpathmoveto{\pgfqpoint{0.913474in}{0.603225in}}%
\pgfpathlineto{\pgfqpoint{0.913474in}{0.603225in}}%
\pgfpathlineto{\pgfqpoint{0.913474in}{0.606174in}}%
\pgfpathlineto{\pgfqpoint{0.918015in}{0.606174in}}%
\pgfpathlineto{\pgfqpoint{0.918015in}{0.603225in}}%
\pgfpathmoveto{\pgfqpoint{0.913474in}{0.606174in}}%
\pgfpathlineto{\pgfqpoint{0.913474in}{0.606174in}}%
\pgfpathlineto{\pgfqpoint{0.913474in}{0.609123in}}%
\pgfpathlineto{\pgfqpoint{0.918015in}{0.609123in}}%
\pgfpathlineto{\pgfqpoint{0.918015in}{0.606174in}}%
\pgfpathmoveto{\pgfqpoint{0.918015in}{0.606174in}}%
\pgfpathlineto{\pgfqpoint{0.918015in}{0.606174in}}%
\pgfpathlineto{\pgfqpoint{0.918015in}{0.609123in}}%
\pgfpathlineto{\pgfqpoint{0.922556in}{0.609123in}}%
\pgfpathlineto{\pgfqpoint{0.922556in}{0.606174in}}%
\pgfpathmoveto{\pgfqpoint{0.918015in}{0.609123in}}%
\pgfpathlineto{\pgfqpoint{0.918015in}{0.609123in}}%
\pgfpathlineto{\pgfqpoint{0.918015in}{0.612072in}}%
\pgfpathlineto{\pgfqpoint{0.922556in}{0.612072in}}%
\pgfpathlineto{\pgfqpoint{0.922556in}{0.609123in}}%
\pgfpathmoveto{\pgfqpoint{0.922556in}{0.609123in}}%
\pgfpathlineto{\pgfqpoint{0.922556in}{0.609123in}}%
\pgfpathlineto{\pgfqpoint{0.922556in}{0.612072in}}%
\pgfpathlineto{\pgfqpoint{0.927097in}{0.612072in}}%
\pgfpathlineto{\pgfqpoint{0.927097in}{0.609123in}}%
\pgfpathmoveto{\pgfqpoint{0.922556in}{0.612072in}}%
\pgfpathlineto{\pgfqpoint{0.922556in}{0.612072in}}%
\pgfpathlineto{\pgfqpoint{0.922556in}{0.615022in}}%
\pgfpathlineto{\pgfqpoint{0.927097in}{0.615022in}}%
\pgfpathlineto{\pgfqpoint{0.927097in}{0.612072in}}%
\pgfpathmoveto{\pgfqpoint{0.927097in}{0.612072in}}%
\pgfpathlineto{\pgfqpoint{0.927097in}{0.612072in}}%
\pgfpathlineto{\pgfqpoint{0.927097in}{0.615022in}}%
\pgfpathlineto{\pgfqpoint{0.931638in}{0.615022in}}%
\pgfpathlineto{\pgfqpoint{0.931638in}{0.612072in}}%
\pgfpathmoveto{\pgfqpoint{0.927097in}{0.615022in}}%
\pgfpathlineto{\pgfqpoint{0.927097in}{0.615022in}}%
\pgfpathlineto{\pgfqpoint{0.927097in}{0.617971in}}%
\pgfpathlineto{\pgfqpoint{0.931638in}{0.617971in}}%
\pgfpathlineto{\pgfqpoint{0.931638in}{0.615022in}}%
\pgfpathmoveto{\pgfqpoint{0.931638in}{0.615022in}}%
\pgfpathlineto{\pgfqpoint{0.931638in}{0.615022in}}%
\pgfpathlineto{\pgfqpoint{0.931638in}{0.617971in}}%
\pgfpathlineto{\pgfqpoint{0.936179in}{0.617971in}}%
\pgfpathlineto{\pgfqpoint{0.936179in}{0.615022in}}%
\pgfpathmoveto{\pgfqpoint{0.931638in}{0.617971in}}%
\pgfpathlineto{\pgfqpoint{0.931638in}{0.617971in}}%
\pgfpathlineto{\pgfqpoint{0.931638in}{0.620920in}}%
\pgfpathlineto{\pgfqpoint{0.936179in}{0.620920in}}%
\pgfpathlineto{\pgfqpoint{0.936179in}{0.617971in}}%
\pgfpathmoveto{\pgfqpoint{0.936179in}{0.617971in}}%
\pgfpathlineto{\pgfqpoint{0.936179in}{0.617971in}}%
\pgfpathlineto{\pgfqpoint{0.936179in}{0.620920in}}%
\pgfpathlineto{\pgfqpoint{0.940720in}{0.620920in}}%
\pgfpathlineto{\pgfqpoint{0.940720in}{0.617971in}}%
\pgfpathmoveto{\pgfqpoint{0.936179in}{0.620920in}}%
\pgfpathlineto{\pgfqpoint{0.936179in}{0.620920in}}%
\pgfpathlineto{\pgfqpoint{0.936179in}{0.623869in}}%
\pgfpathlineto{\pgfqpoint{0.940720in}{0.623869in}}%
\pgfpathlineto{\pgfqpoint{0.940720in}{0.620920in}}%
\pgfpathmoveto{\pgfqpoint{0.940720in}{0.620920in}}%
\pgfpathlineto{\pgfqpoint{0.940720in}{0.620920in}}%
\pgfpathlineto{\pgfqpoint{0.940720in}{0.623869in}}%
\pgfpathlineto{\pgfqpoint{0.945261in}{0.623869in}}%
\pgfpathlineto{\pgfqpoint{0.945261in}{0.620920in}}%
\pgfpathmoveto{\pgfqpoint{0.940720in}{0.623869in}}%
\pgfpathlineto{\pgfqpoint{0.940720in}{0.623869in}}%
\pgfpathlineto{\pgfqpoint{0.940720in}{0.626818in}}%
\pgfpathlineto{\pgfqpoint{0.945261in}{0.626818in}}%
\pgfpathlineto{\pgfqpoint{0.945261in}{0.623869in}}%
\pgfpathmoveto{\pgfqpoint{0.945261in}{0.623869in}}%
\pgfpathlineto{\pgfqpoint{0.945261in}{0.623869in}}%
\pgfpathlineto{\pgfqpoint{0.945261in}{0.626818in}}%
\pgfpathlineto{\pgfqpoint{0.949802in}{0.626818in}}%
\pgfpathlineto{\pgfqpoint{0.949802in}{0.623869in}}%
\pgfpathmoveto{\pgfqpoint{0.945261in}{0.626818in}}%
\pgfpathlineto{\pgfqpoint{0.945261in}{0.626818in}}%
\pgfpathlineto{\pgfqpoint{0.945261in}{0.629768in}}%
\pgfpathlineto{\pgfqpoint{0.949802in}{0.629768in}}%
\pgfpathlineto{\pgfqpoint{0.949802in}{0.626818in}}%
\pgfpathmoveto{\pgfqpoint{0.949802in}{0.626818in}}%
\pgfpathlineto{\pgfqpoint{0.949802in}{0.626818in}}%
\pgfpathlineto{\pgfqpoint{0.949802in}{0.629768in}}%
\pgfpathlineto{\pgfqpoint{0.954343in}{0.629768in}}%
\pgfpathlineto{\pgfqpoint{0.954343in}{0.626818in}}%
\pgfpathmoveto{\pgfqpoint{0.949802in}{0.629768in}}%
\pgfpathlineto{\pgfqpoint{0.949802in}{0.629768in}}%
\pgfpathlineto{\pgfqpoint{0.949802in}{0.632717in}}%
\pgfpathlineto{\pgfqpoint{0.954343in}{0.632717in}}%
\pgfpathlineto{\pgfqpoint{0.954343in}{0.629768in}}%
\pgfpathmoveto{\pgfqpoint{0.954343in}{0.629768in}}%
\pgfpathlineto{\pgfqpoint{0.954343in}{0.629768in}}%
\pgfpathlineto{\pgfqpoint{0.954343in}{0.632717in}}%
\pgfpathlineto{\pgfqpoint{0.958884in}{0.632717in}}%
\pgfpathlineto{\pgfqpoint{0.958884in}{0.629768in}}%
\pgfpathmoveto{\pgfqpoint{0.954343in}{0.632717in}}%
\pgfpathlineto{\pgfqpoint{0.954343in}{0.632717in}}%
\pgfpathlineto{\pgfqpoint{0.954343in}{0.635666in}}%
\pgfpathlineto{\pgfqpoint{0.958884in}{0.635666in}}%
\pgfpathlineto{\pgfqpoint{0.958884in}{0.632717in}}%
\pgfpathmoveto{\pgfqpoint{0.958884in}{0.632717in}}%
\pgfpathlineto{\pgfqpoint{0.958884in}{0.632717in}}%
\pgfpathlineto{\pgfqpoint{0.958884in}{0.635666in}}%
\pgfpathlineto{\pgfqpoint{0.963425in}{0.635666in}}%
\pgfpathlineto{\pgfqpoint{0.963425in}{0.632717in}}%
\pgfpathmoveto{\pgfqpoint{0.958884in}{0.635666in}}%
\pgfpathlineto{\pgfqpoint{0.958884in}{0.635666in}}%
\pgfpathlineto{\pgfqpoint{0.958884in}{0.638615in}}%
\pgfpathlineto{\pgfqpoint{0.963425in}{0.638615in}}%
\pgfpathlineto{\pgfqpoint{0.963425in}{0.635666in}}%
\pgfpathmoveto{\pgfqpoint{0.963425in}{0.635666in}}%
\pgfpathlineto{\pgfqpoint{0.963425in}{0.635666in}}%
\pgfpathlineto{\pgfqpoint{0.963425in}{0.638615in}}%
\pgfpathlineto{\pgfqpoint{0.967966in}{0.638615in}}%
\pgfpathlineto{\pgfqpoint{0.967966in}{0.635666in}}%
\pgfpathmoveto{\pgfqpoint{0.963425in}{0.638615in}}%
\pgfpathlineto{\pgfqpoint{0.963425in}{0.638615in}}%
\pgfpathlineto{\pgfqpoint{0.963425in}{0.641564in}}%
\pgfpathlineto{\pgfqpoint{0.967966in}{0.641564in}}%
\pgfpathlineto{\pgfqpoint{0.967966in}{0.638615in}}%
\pgfpathmoveto{\pgfqpoint{0.967966in}{0.638615in}}%
\pgfpathlineto{\pgfqpoint{0.967966in}{0.638615in}}%
\pgfpathlineto{\pgfqpoint{0.967966in}{0.641564in}}%
\pgfpathlineto{\pgfqpoint{0.972507in}{0.641564in}}%
\pgfpathlineto{\pgfqpoint{0.972507in}{0.638615in}}%
\pgfpathmoveto{\pgfqpoint{0.967966in}{0.641564in}}%
\pgfpathlineto{\pgfqpoint{0.967966in}{0.641564in}}%
\pgfpathlineto{\pgfqpoint{0.967966in}{0.644513in}}%
\pgfpathlineto{\pgfqpoint{0.972507in}{0.644513in}}%
\pgfpathlineto{\pgfqpoint{0.972507in}{0.641564in}}%
\pgfpathmoveto{\pgfqpoint{0.972507in}{0.641564in}}%
\pgfpathlineto{\pgfqpoint{0.972507in}{0.641564in}}%
\pgfpathlineto{\pgfqpoint{0.972507in}{0.644513in}}%
\pgfpathlineto{\pgfqpoint{0.977048in}{0.644513in}}%
\pgfpathlineto{\pgfqpoint{0.977048in}{0.641564in}}%
\pgfpathmoveto{\pgfqpoint{0.972507in}{0.644513in}}%
\pgfpathlineto{\pgfqpoint{0.972507in}{0.644513in}}%
\pgfpathlineto{\pgfqpoint{0.972507in}{0.647463in}}%
\pgfpathlineto{\pgfqpoint{0.977048in}{0.647463in}}%
\pgfpathlineto{\pgfqpoint{0.977048in}{0.644513in}}%
\pgfpathmoveto{\pgfqpoint{0.977048in}{0.644513in}}%
\pgfpathlineto{\pgfqpoint{0.977048in}{0.644513in}}%
\pgfpathlineto{\pgfqpoint{0.977048in}{0.647463in}}%
\pgfpathlineto{\pgfqpoint{0.981589in}{0.647463in}}%
\pgfpathlineto{\pgfqpoint{0.981589in}{0.644513in}}%
\pgfpathmoveto{\pgfqpoint{0.977048in}{0.647463in}}%
\pgfpathlineto{\pgfqpoint{0.977048in}{0.647463in}}%
\pgfpathlineto{\pgfqpoint{0.977048in}{0.650412in}}%
\pgfpathlineto{\pgfqpoint{0.981589in}{0.650412in}}%
\pgfpathlineto{\pgfqpoint{0.981589in}{0.647463in}}%
\pgfpathmoveto{\pgfqpoint{0.981589in}{0.647463in}}%
\pgfpathlineto{\pgfqpoint{0.981589in}{0.647463in}}%
\pgfpathlineto{\pgfqpoint{0.981589in}{0.650412in}}%
\pgfpathlineto{\pgfqpoint{0.986130in}{0.650412in}}%
\pgfpathlineto{\pgfqpoint{0.986130in}{0.647463in}}%
\pgfpathmoveto{\pgfqpoint{0.981589in}{0.650412in}}%
\pgfpathlineto{\pgfqpoint{0.981589in}{0.650412in}}%
\pgfpathlineto{\pgfqpoint{0.981589in}{0.653361in}}%
\pgfpathlineto{\pgfqpoint{0.986130in}{0.653361in}}%
\pgfpathlineto{\pgfqpoint{0.986130in}{0.650412in}}%
\pgfpathmoveto{\pgfqpoint{0.986130in}{0.650412in}}%
\pgfpathlineto{\pgfqpoint{0.986130in}{0.650412in}}%
\pgfpathlineto{\pgfqpoint{0.986130in}{0.653361in}}%
\pgfpathlineto{\pgfqpoint{0.990671in}{0.653361in}}%
\pgfpathlineto{\pgfqpoint{0.990671in}{0.650412in}}%
\pgfpathmoveto{\pgfqpoint{0.986130in}{0.653361in}}%
\pgfpathlineto{\pgfqpoint{0.986130in}{0.653361in}}%
\pgfpathlineto{\pgfqpoint{0.986130in}{0.656310in}}%
\pgfpathlineto{\pgfqpoint{0.990671in}{0.656310in}}%
\pgfpathlineto{\pgfqpoint{0.990671in}{0.653361in}}%
\pgfpathmoveto{\pgfqpoint{0.990671in}{0.653361in}}%
\pgfpathlineto{\pgfqpoint{0.990671in}{0.653361in}}%
\pgfpathlineto{\pgfqpoint{0.990671in}{0.656310in}}%
\pgfpathlineto{\pgfqpoint{0.995212in}{0.656310in}}%
\pgfpathlineto{\pgfqpoint{0.995212in}{0.653361in}}%
\pgfpathmoveto{\pgfqpoint{0.990671in}{0.656310in}}%
\pgfpathlineto{\pgfqpoint{0.990671in}{0.656310in}}%
\pgfpathlineto{\pgfqpoint{0.990671in}{0.659259in}}%
\pgfpathlineto{\pgfqpoint{0.995212in}{0.659259in}}%
\pgfpathlineto{\pgfqpoint{0.995212in}{0.656310in}}%
\pgfpathmoveto{\pgfqpoint{0.995212in}{0.656310in}}%
\pgfpathlineto{\pgfqpoint{0.995212in}{0.656310in}}%
\pgfpathlineto{\pgfqpoint{0.995212in}{0.659259in}}%
\pgfpathlineto{\pgfqpoint{0.999753in}{0.659259in}}%
\pgfpathlineto{\pgfqpoint{0.999753in}{0.656310in}}%
\pgfpathmoveto{\pgfqpoint{0.995212in}{0.659259in}}%
\pgfpathlineto{\pgfqpoint{0.995212in}{0.659259in}}%
\pgfpathlineto{\pgfqpoint{0.995212in}{0.662209in}}%
\pgfpathlineto{\pgfqpoint{0.999753in}{0.662209in}}%
\pgfpathlineto{\pgfqpoint{0.999753in}{0.659259in}}%
\pgfpathmoveto{\pgfqpoint{0.999753in}{0.659259in}}%
\pgfpathlineto{\pgfqpoint{0.999753in}{0.659259in}}%
\pgfpathlineto{\pgfqpoint{0.999753in}{0.662209in}}%
\pgfpathlineto{\pgfqpoint{1.004294in}{0.662209in}}%
\pgfpathlineto{\pgfqpoint{1.004294in}{0.659259in}}%
\pgfpathmoveto{\pgfqpoint{0.999753in}{0.662209in}}%
\pgfpathlineto{\pgfqpoint{0.999753in}{0.662209in}}%
\pgfpathlineto{\pgfqpoint{0.999753in}{0.665158in}}%
\pgfpathlineto{\pgfqpoint{1.004294in}{0.665158in}}%
\pgfpathlineto{\pgfqpoint{1.004294in}{0.662209in}}%
\pgfpathmoveto{\pgfqpoint{1.004294in}{0.662209in}}%
\pgfpathlineto{\pgfqpoint{1.004294in}{0.662209in}}%
\pgfpathlineto{\pgfqpoint{1.004294in}{0.665158in}}%
\pgfpathlineto{\pgfqpoint{1.008835in}{0.665158in}}%
\pgfpathlineto{\pgfqpoint{1.008835in}{0.662209in}}%
\pgfpathmoveto{\pgfqpoint{1.004294in}{0.665158in}}%
\pgfpathlineto{\pgfqpoint{1.004294in}{0.665158in}}%
\pgfpathlineto{\pgfqpoint{1.004294in}{0.668107in}}%
\pgfpathlineto{\pgfqpoint{1.008835in}{0.668107in}}%
\pgfpathlineto{\pgfqpoint{1.008835in}{0.665158in}}%
\pgfpathmoveto{\pgfqpoint{1.008835in}{0.665158in}}%
\pgfpathlineto{\pgfqpoint{1.008835in}{0.665158in}}%
\pgfpathlineto{\pgfqpoint{1.008835in}{0.668107in}}%
\pgfpathlineto{\pgfqpoint{1.013376in}{0.668107in}}%
\pgfpathlineto{\pgfqpoint{1.013376in}{0.665158in}}%
\pgfpathmoveto{\pgfqpoint{1.008835in}{0.668107in}}%
\pgfpathlineto{\pgfqpoint{1.008835in}{0.668107in}}%
\pgfpathlineto{\pgfqpoint{1.008835in}{0.671056in}}%
\pgfpathlineto{\pgfqpoint{1.013376in}{0.671056in}}%
\pgfpathlineto{\pgfqpoint{1.013376in}{0.668107in}}%
\pgfpathmoveto{\pgfqpoint{1.013376in}{0.668107in}}%
\pgfpathlineto{\pgfqpoint{1.013376in}{0.668107in}}%
\pgfpathlineto{\pgfqpoint{1.013376in}{0.671056in}}%
\pgfpathlineto{\pgfqpoint{1.017917in}{0.671056in}}%
\pgfpathlineto{\pgfqpoint{1.017917in}{0.668107in}}%
\pgfpathmoveto{\pgfqpoint{1.013376in}{0.671056in}}%
\pgfpathlineto{\pgfqpoint{1.013376in}{0.671056in}}%
\pgfpathlineto{\pgfqpoint{1.013376in}{0.674005in}}%
\pgfpathlineto{\pgfqpoint{1.017917in}{0.674005in}}%
\pgfpathlineto{\pgfqpoint{1.017917in}{0.671056in}}%
\pgfpathmoveto{\pgfqpoint{1.017917in}{0.671056in}}%
\pgfpathlineto{\pgfqpoint{1.017917in}{0.671056in}}%
\pgfpathlineto{\pgfqpoint{1.017917in}{0.674005in}}%
\pgfpathlineto{\pgfqpoint{1.022458in}{0.674005in}}%
\pgfpathlineto{\pgfqpoint{1.022458in}{0.671056in}}%
\pgfpathmoveto{\pgfqpoint{1.017917in}{0.674005in}}%
\pgfpathlineto{\pgfqpoint{1.017917in}{0.674005in}}%
\pgfpathlineto{\pgfqpoint{1.017917in}{0.676955in}}%
\pgfpathlineto{\pgfqpoint{1.022458in}{0.676955in}}%
\pgfpathlineto{\pgfqpoint{1.022458in}{0.674005in}}%
\pgfpathmoveto{\pgfqpoint{1.022458in}{0.674005in}}%
\pgfpathlineto{\pgfqpoint{1.022458in}{0.674005in}}%
\pgfpathlineto{\pgfqpoint{1.022458in}{0.676955in}}%
\pgfpathlineto{\pgfqpoint{1.026999in}{0.676955in}}%
\pgfpathlineto{\pgfqpoint{1.026999in}{0.674005in}}%
\pgfpathmoveto{\pgfqpoint{1.022458in}{0.676955in}}%
\pgfpathlineto{\pgfqpoint{1.022458in}{0.676955in}}%
\pgfpathlineto{\pgfqpoint{1.022458in}{0.679904in}}%
\pgfpathlineto{\pgfqpoint{1.026999in}{0.679904in}}%
\pgfpathlineto{\pgfqpoint{1.026999in}{0.676955in}}%
\pgfpathmoveto{\pgfqpoint{1.026999in}{0.676955in}}%
\pgfpathlineto{\pgfqpoint{1.026999in}{0.676955in}}%
\pgfpathlineto{\pgfqpoint{1.026999in}{0.679904in}}%
\pgfpathlineto{\pgfqpoint{1.031540in}{0.679904in}}%
\pgfpathlineto{\pgfqpoint{1.031540in}{0.676955in}}%
\pgfpathmoveto{\pgfqpoint{1.026999in}{0.679904in}}%
\pgfpathlineto{\pgfqpoint{1.026999in}{0.679904in}}%
\pgfpathlineto{\pgfqpoint{1.026999in}{0.682853in}}%
\pgfpathlineto{\pgfqpoint{1.031540in}{0.682853in}}%
\pgfpathlineto{\pgfqpoint{1.031540in}{0.679904in}}%
\pgfpathmoveto{\pgfqpoint{1.031540in}{0.679904in}}%
\pgfpathlineto{\pgfqpoint{1.031540in}{0.679904in}}%
\pgfpathlineto{\pgfqpoint{1.031540in}{0.682853in}}%
\pgfpathlineto{\pgfqpoint{1.036081in}{0.682853in}}%
\pgfpathlineto{\pgfqpoint{1.036081in}{0.679904in}}%
\pgfpathmoveto{\pgfqpoint{1.031540in}{0.682853in}}%
\pgfpathlineto{\pgfqpoint{1.031540in}{0.682853in}}%
\pgfpathlineto{\pgfqpoint{1.031540in}{0.685802in}}%
\pgfpathlineto{\pgfqpoint{1.036081in}{0.685802in}}%
\pgfpathlineto{\pgfqpoint{1.036081in}{0.682853in}}%
\pgfpathmoveto{\pgfqpoint{1.036081in}{0.682853in}}%
\pgfpathlineto{\pgfqpoint{1.036081in}{0.682853in}}%
\pgfpathlineto{\pgfqpoint{1.036081in}{0.685802in}}%
\pgfpathlineto{\pgfqpoint{1.040622in}{0.685802in}}%
\pgfpathlineto{\pgfqpoint{1.040622in}{0.682853in}}%
\pgfpathmoveto{\pgfqpoint{1.036081in}{0.685802in}}%
\pgfpathlineto{\pgfqpoint{1.036081in}{0.685802in}}%
\pgfpathlineto{\pgfqpoint{1.036081in}{0.688751in}}%
\pgfpathlineto{\pgfqpoint{1.040622in}{0.688751in}}%
\pgfpathlineto{\pgfqpoint{1.040622in}{0.685802in}}%
\pgfpathmoveto{\pgfqpoint{1.040622in}{0.685802in}}%
\pgfpathlineto{\pgfqpoint{1.040622in}{0.685802in}}%
\pgfpathlineto{\pgfqpoint{1.040622in}{0.688751in}}%
\pgfpathlineto{\pgfqpoint{1.045164in}{0.688751in}}%
\pgfpathlineto{\pgfqpoint{1.045164in}{0.685802in}}%
\pgfpathmoveto{\pgfqpoint{1.040622in}{0.688751in}}%
\pgfpathlineto{\pgfqpoint{1.040622in}{0.688751in}}%
\pgfpathlineto{\pgfqpoint{1.040622in}{0.691700in}}%
\pgfpathlineto{\pgfqpoint{1.045164in}{0.691700in}}%
\pgfpathlineto{\pgfqpoint{1.045164in}{0.688751in}}%
\pgfpathmoveto{\pgfqpoint{1.045164in}{0.688751in}}%
\pgfpathlineto{\pgfqpoint{1.045164in}{0.688751in}}%
\pgfpathlineto{\pgfqpoint{1.045164in}{0.691700in}}%
\pgfpathlineto{\pgfqpoint{1.049705in}{0.691700in}}%
\pgfpathlineto{\pgfqpoint{1.049705in}{0.688751in}}%
\pgfpathmoveto{\pgfqpoint{1.045164in}{0.691700in}}%
\pgfpathlineto{\pgfqpoint{1.045164in}{0.691700in}}%
\pgfpathlineto{\pgfqpoint{1.045164in}{0.694650in}}%
\pgfpathlineto{\pgfqpoint{1.049705in}{0.694650in}}%
\pgfpathlineto{\pgfqpoint{1.049705in}{0.691700in}}%
\pgfpathmoveto{\pgfqpoint{1.049705in}{0.691700in}}%
\pgfpathlineto{\pgfqpoint{1.049705in}{0.691700in}}%
\pgfpathlineto{\pgfqpoint{1.049705in}{0.694650in}}%
\pgfpathlineto{\pgfqpoint{1.054246in}{0.694650in}}%
\pgfpathlineto{\pgfqpoint{1.054246in}{0.691700in}}%
\pgfpathmoveto{\pgfqpoint{1.049705in}{0.694650in}}%
\pgfpathlineto{\pgfqpoint{1.049705in}{0.694650in}}%
\pgfpathlineto{\pgfqpoint{1.049705in}{0.697599in}}%
\pgfpathlineto{\pgfqpoint{1.054246in}{0.697599in}}%
\pgfpathlineto{\pgfqpoint{1.054246in}{0.694650in}}%
\pgfpathmoveto{\pgfqpoint{1.054246in}{0.694650in}}%
\pgfpathlineto{\pgfqpoint{1.054246in}{0.694650in}}%
\pgfpathlineto{\pgfqpoint{1.054246in}{0.697599in}}%
\pgfpathlineto{\pgfqpoint{1.058787in}{0.697599in}}%
\pgfpathlineto{\pgfqpoint{1.058787in}{0.694650in}}%
\pgfpathmoveto{\pgfqpoint{1.054246in}{0.697599in}}%
\pgfpathlineto{\pgfqpoint{1.054246in}{0.697599in}}%
\pgfpathlineto{\pgfqpoint{1.054246in}{0.700548in}}%
\pgfpathlineto{\pgfqpoint{1.058787in}{0.700548in}}%
\pgfpathlineto{\pgfqpoint{1.058787in}{0.697599in}}%
\pgfpathmoveto{\pgfqpoint{1.058787in}{0.697599in}}%
\pgfpathlineto{\pgfqpoint{1.058787in}{0.697599in}}%
\pgfpathlineto{\pgfqpoint{1.058787in}{0.700548in}}%
\pgfpathlineto{\pgfqpoint{1.063328in}{0.700548in}}%
\pgfpathlineto{\pgfqpoint{1.063328in}{0.697599in}}%
\pgfpathmoveto{\pgfqpoint{1.058787in}{0.700548in}}%
\pgfpathlineto{\pgfqpoint{1.058787in}{0.700548in}}%
\pgfpathlineto{\pgfqpoint{1.058787in}{0.703497in}}%
\pgfpathlineto{\pgfqpoint{1.063328in}{0.703497in}}%
\pgfpathlineto{\pgfqpoint{1.063328in}{0.700548in}}%
\pgfpathmoveto{\pgfqpoint{1.063328in}{0.700548in}}%
\pgfpathlineto{\pgfqpoint{1.063328in}{0.700548in}}%
\pgfpathlineto{\pgfqpoint{1.063328in}{0.703497in}}%
\pgfpathlineto{\pgfqpoint{1.067869in}{0.703497in}}%
\pgfpathlineto{\pgfqpoint{1.067869in}{0.700548in}}%
\pgfpathmoveto{\pgfqpoint{1.063328in}{0.703497in}}%
\pgfpathlineto{\pgfqpoint{1.063328in}{0.703497in}}%
\pgfpathlineto{\pgfqpoint{1.063328in}{0.706446in}}%
\pgfpathlineto{\pgfqpoint{1.067869in}{0.706446in}}%
\pgfpathlineto{\pgfqpoint{1.067869in}{0.703497in}}%
\pgfpathmoveto{\pgfqpoint{1.067869in}{0.703497in}}%
\pgfpathlineto{\pgfqpoint{1.067869in}{0.703497in}}%
\pgfpathlineto{\pgfqpoint{1.067869in}{0.706446in}}%
\pgfpathlineto{\pgfqpoint{1.072410in}{0.706446in}}%
\pgfpathlineto{\pgfqpoint{1.072410in}{0.703497in}}%
\pgfpathmoveto{\pgfqpoint{1.067869in}{0.706446in}}%
\pgfpathlineto{\pgfqpoint{1.067869in}{0.706446in}}%
\pgfpathlineto{\pgfqpoint{1.067869in}{0.709395in}}%
\pgfpathlineto{\pgfqpoint{1.072410in}{0.709395in}}%
\pgfpathlineto{\pgfqpoint{1.072410in}{0.706446in}}%
\pgfpathmoveto{\pgfqpoint{1.072410in}{0.706446in}}%
\pgfpathlineto{\pgfqpoint{1.072410in}{0.706446in}}%
\pgfpathlineto{\pgfqpoint{1.072410in}{0.709395in}}%
\pgfpathlineto{\pgfqpoint{1.076951in}{0.709395in}}%
\pgfpathlineto{\pgfqpoint{1.076951in}{0.706446in}}%
\pgfpathmoveto{\pgfqpoint{1.072410in}{0.709395in}}%
\pgfpathlineto{\pgfqpoint{1.072410in}{0.709395in}}%
\pgfpathlineto{\pgfqpoint{1.072410in}{0.712344in}}%
\pgfpathlineto{\pgfqpoint{1.076951in}{0.712344in}}%
\pgfpathlineto{\pgfqpoint{1.076951in}{0.709395in}}%
\pgfpathmoveto{\pgfqpoint{1.076951in}{0.709395in}}%
\pgfpathlineto{\pgfqpoint{1.076951in}{0.709395in}}%
\pgfpathlineto{\pgfqpoint{1.076951in}{0.712344in}}%
\pgfpathlineto{\pgfqpoint{1.081493in}{0.712344in}}%
\pgfpathlineto{\pgfqpoint{1.081493in}{0.709395in}}%
\pgfpathmoveto{\pgfqpoint{1.076951in}{0.712344in}}%
\pgfpathlineto{\pgfqpoint{1.076951in}{0.712344in}}%
\pgfpathlineto{\pgfqpoint{1.076951in}{0.715294in}}%
\pgfpathlineto{\pgfqpoint{1.081493in}{0.715294in}}%
\pgfpathlineto{\pgfqpoint{1.081493in}{0.712344in}}%
\pgfpathmoveto{\pgfqpoint{1.081493in}{0.712344in}}%
\pgfpathlineto{\pgfqpoint{1.081493in}{0.712344in}}%
\pgfpathlineto{\pgfqpoint{1.081493in}{0.715294in}}%
\pgfpathlineto{\pgfqpoint{1.086034in}{0.715294in}}%
\pgfpathlineto{\pgfqpoint{1.086034in}{0.712344in}}%
\pgfpathmoveto{\pgfqpoint{1.081493in}{0.715294in}}%
\pgfpathlineto{\pgfqpoint{1.081493in}{0.715294in}}%
\pgfpathlineto{\pgfqpoint{1.081493in}{0.718243in}}%
\pgfpathlineto{\pgfqpoint{1.086034in}{0.718243in}}%
\pgfpathlineto{\pgfqpoint{1.086034in}{0.715294in}}%
\pgfpathmoveto{\pgfqpoint{1.086034in}{0.715294in}}%
\pgfpathlineto{\pgfqpoint{1.086034in}{0.715294in}}%
\pgfpathlineto{\pgfqpoint{1.086034in}{0.718243in}}%
\pgfpathlineto{\pgfqpoint{1.090575in}{0.718243in}}%
\pgfpathlineto{\pgfqpoint{1.090575in}{0.715294in}}%
\pgfpathmoveto{\pgfqpoint{1.086034in}{0.718243in}}%
\pgfpathlineto{\pgfqpoint{1.086034in}{0.718243in}}%
\pgfpathlineto{\pgfqpoint{1.086034in}{0.721192in}}%
\pgfpathlineto{\pgfqpoint{1.090575in}{0.721192in}}%
\pgfpathlineto{\pgfqpoint{1.090575in}{0.718243in}}%
\pgfpathmoveto{\pgfqpoint{1.090575in}{0.718243in}}%
\pgfpathlineto{\pgfqpoint{1.090575in}{0.718243in}}%
\pgfpathlineto{\pgfqpoint{1.090575in}{0.721192in}}%
\pgfpathlineto{\pgfqpoint{1.095116in}{0.721192in}}%
\pgfpathlineto{\pgfqpoint{1.095116in}{0.718243in}}%
\pgfpathmoveto{\pgfqpoint{1.090575in}{0.721192in}}%
\pgfpathlineto{\pgfqpoint{1.090575in}{0.721192in}}%
\pgfpathlineto{\pgfqpoint{1.090575in}{0.724141in}}%
\pgfpathlineto{\pgfqpoint{1.095116in}{0.724141in}}%
\pgfpathlineto{\pgfqpoint{1.095116in}{0.721192in}}%
\pgfpathmoveto{\pgfqpoint{1.095116in}{0.721192in}}%
\pgfpathlineto{\pgfqpoint{1.095116in}{0.721192in}}%
\pgfpathlineto{\pgfqpoint{1.095116in}{0.724141in}}%
\pgfpathlineto{\pgfqpoint{1.099657in}{0.724141in}}%
\pgfpathlineto{\pgfqpoint{1.099657in}{0.721192in}}%
\pgfpathmoveto{\pgfqpoint{1.095116in}{0.724141in}}%
\pgfpathlineto{\pgfqpoint{1.095116in}{0.724141in}}%
\pgfpathlineto{\pgfqpoint{1.095116in}{0.727090in}}%
\pgfpathlineto{\pgfqpoint{1.099657in}{0.727090in}}%
\pgfpathlineto{\pgfqpoint{1.099657in}{0.724141in}}%
\pgfpathmoveto{\pgfqpoint{1.099657in}{0.724141in}}%
\pgfpathlineto{\pgfqpoint{1.099657in}{0.724141in}}%
\pgfpathlineto{\pgfqpoint{1.099657in}{0.727090in}}%
\pgfpathlineto{\pgfqpoint{1.104198in}{0.727090in}}%
\pgfpathlineto{\pgfqpoint{1.104198in}{0.724141in}}%
\pgfpathmoveto{\pgfqpoint{1.099657in}{0.727090in}}%
\pgfpathlineto{\pgfqpoint{1.099657in}{0.727090in}}%
\pgfpathlineto{\pgfqpoint{1.099657in}{0.730039in}}%
\pgfpathlineto{\pgfqpoint{1.104198in}{0.730039in}}%
\pgfpathlineto{\pgfqpoint{1.104198in}{0.727090in}}%
\pgfpathmoveto{\pgfqpoint{1.104198in}{0.727090in}}%
\pgfpathlineto{\pgfqpoint{1.104198in}{0.727090in}}%
\pgfpathlineto{\pgfqpoint{1.104198in}{0.730039in}}%
\pgfpathlineto{\pgfqpoint{1.108739in}{0.730039in}}%
\pgfpathlineto{\pgfqpoint{1.108739in}{0.727090in}}%
\pgfpathmoveto{\pgfqpoint{1.104198in}{0.730039in}}%
\pgfpathlineto{\pgfqpoint{1.104198in}{0.730039in}}%
\pgfpathlineto{\pgfqpoint{1.104198in}{0.732988in}}%
\pgfpathlineto{\pgfqpoint{1.108739in}{0.732988in}}%
\pgfpathlineto{\pgfqpoint{1.108739in}{0.730039in}}%
\pgfpathmoveto{\pgfqpoint{1.108739in}{0.730039in}}%
\pgfpathlineto{\pgfqpoint{1.108739in}{0.730039in}}%
\pgfpathlineto{\pgfqpoint{1.108739in}{0.732988in}}%
\pgfpathlineto{\pgfqpoint{1.113280in}{0.732988in}}%
\pgfpathlineto{\pgfqpoint{1.113280in}{0.730039in}}%
\pgfpathmoveto{\pgfqpoint{1.108739in}{0.732988in}}%
\pgfpathlineto{\pgfqpoint{1.108739in}{0.732988in}}%
\pgfpathlineto{\pgfqpoint{1.108739in}{0.735938in}}%
\pgfpathlineto{\pgfqpoint{1.113280in}{0.735938in}}%
\pgfpathlineto{\pgfqpoint{1.113280in}{0.732988in}}%
\pgfpathmoveto{\pgfqpoint{1.113280in}{0.732988in}}%
\pgfpathlineto{\pgfqpoint{1.113280in}{0.732988in}}%
\pgfpathlineto{\pgfqpoint{1.113280in}{0.735938in}}%
\pgfpathlineto{\pgfqpoint{1.117822in}{0.735938in}}%
\pgfpathlineto{\pgfqpoint{1.117822in}{0.732988in}}%
\pgfpathmoveto{\pgfqpoint{1.113280in}{0.735938in}}%
\pgfpathlineto{\pgfqpoint{1.113280in}{0.735938in}}%
\pgfpathlineto{\pgfqpoint{1.113280in}{0.738887in}}%
\pgfpathlineto{\pgfqpoint{1.117822in}{0.738887in}}%
\pgfpathlineto{\pgfqpoint{1.117822in}{0.735938in}}%
\pgfpathmoveto{\pgfqpoint{1.117822in}{0.735938in}}%
\pgfpathlineto{\pgfqpoint{1.117822in}{0.735938in}}%
\pgfpathlineto{\pgfqpoint{1.117822in}{0.738887in}}%
\pgfpathlineto{\pgfqpoint{1.122363in}{0.738887in}}%
\pgfpathlineto{\pgfqpoint{1.122363in}{0.735938in}}%
\pgfpathmoveto{\pgfqpoint{1.117822in}{0.738887in}}%
\pgfpathlineto{\pgfqpoint{1.117822in}{0.738887in}}%
\pgfpathlineto{\pgfqpoint{1.117822in}{0.741836in}}%
\pgfpathlineto{\pgfqpoint{1.122363in}{0.741836in}}%
\pgfpathlineto{\pgfqpoint{1.122363in}{0.738887in}}%
\pgfpathmoveto{\pgfqpoint{1.122363in}{0.738887in}}%
\pgfpathlineto{\pgfqpoint{1.122363in}{0.738887in}}%
\pgfpathlineto{\pgfqpoint{1.122363in}{0.741836in}}%
\pgfpathlineto{\pgfqpoint{1.126904in}{0.741836in}}%
\pgfpathlineto{\pgfqpoint{1.126904in}{0.738887in}}%
\pgfpathmoveto{\pgfqpoint{1.122363in}{0.741836in}}%
\pgfpathlineto{\pgfqpoint{1.122363in}{0.741836in}}%
\pgfpathlineto{\pgfqpoint{1.122363in}{0.744785in}}%
\pgfpathlineto{\pgfqpoint{1.126904in}{0.744785in}}%
\pgfpathlineto{\pgfqpoint{1.126904in}{0.741836in}}%
\pgfpathmoveto{\pgfqpoint{1.126904in}{0.741836in}}%
\pgfpathlineto{\pgfqpoint{1.126904in}{0.741836in}}%
\pgfpathlineto{\pgfqpoint{1.126904in}{0.744785in}}%
\pgfpathlineto{\pgfqpoint{1.131445in}{0.744785in}}%
\pgfpathlineto{\pgfqpoint{1.131445in}{0.741836in}}%
\pgfpathmoveto{\pgfqpoint{1.126904in}{0.744785in}}%
\pgfpathlineto{\pgfqpoint{1.126904in}{0.744785in}}%
\pgfpathlineto{\pgfqpoint{1.126904in}{0.747734in}}%
\pgfpathlineto{\pgfqpoint{1.131445in}{0.747734in}}%
\pgfpathlineto{\pgfqpoint{1.131445in}{0.744785in}}%
\pgfpathmoveto{\pgfqpoint{1.126904in}{0.747734in}}%
\pgfpathlineto{\pgfqpoint{1.126904in}{0.747734in}}%
\pgfpathlineto{\pgfqpoint{1.126904in}{0.750683in}}%
\pgfpathlineto{\pgfqpoint{1.131445in}{0.750683in}}%
\pgfpathlineto{\pgfqpoint{1.131445in}{0.747734in}}%
\pgfpathmoveto{\pgfqpoint{1.131445in}{0.747734in}}%
\pgfpathlineto{\pgfqpoint{1.131445in}{0.747734in}}%
\pgfpathlineto{\pgfqpoint{1.131445in}{0.750683in}}%
\pgfpathlineto{\pgfqpoint{1.135986in}{0.750683in}}%
\pgfpathlineto{\pgfqpoint{1.135986in}{0.747734in}}%
\pgfpathmoveto{\pgfqpoint{1.131445in}{0.750683in}}%
\pgfpathlineto{\pgfqpoint{1.131445in}{0.750683in}}%
\pgfpathlineto{\pgfqpoint{1.131445in}{0.753632in}}%
\pgfpathlineto{\pgfqpoint{1.135986in}{0.753632in}}%
\pgfpathlineto{\pgfqpoint{1.135986in}{0.750683in}}%
\pgfpathmoveto{\pgfqpoint{1.135986in}{0.750683in}}%
\pgfpathlineto{\pgfqpoint{1.135986in}{0.750683in}}%
\pgfpathlineto{\pgfqpoint{1.135986in}{0.753632in}}%
\pgfpathlineto{\pgfqpoint{1.140527in}{0.753632in}}%
\pgfpathlineto{\pgfqpoint{1.140527in}{0.750683in}}%
\pgfpathmoveto{\pgfqpoint{1.135986in}{0.753632in}}%
\pgfpathlineto{\pgfqpoint{1.135986in}{0.753632in}}%
\pgfpathlineto{\pgfqpoint{1.135986in}{0.756582in}}%
\pgfpathlineto{\pgfqpoint{1.140527in}{0.756582in}}%
\pgfpathlineto{\pgfqpoint{1.140527in}{0.753632in}}%
\pgfpathmoveto{\pgfqpoint{1.140527in}{0.753632in}}%
\pgfpathlineto{\pgfqpoint{1.140527in}{0.753632in}}%
\pgfpathlineto{\pgfqpoint{1.140527in}{0.756582in}}%
\pgfpathlineto{\pgfqpoint{1.145068in}{0.756582in}}%
\pgfpathlineto{\pgfqpoint{1.145068in}{0.753632in}}%
\pgfpathmoveto{\pgfqpoint{1.140527in}{0.756582in}}%
\pgfpathlineto{\pgfqpoint{1.140527in}{0.756582in}}%
\pgfpathlineto{\pgfqpoint{1.140527in}{0.759531in}}%
\pgfpathlineto{\pgfqpoint{1.145068in}{0.759531in}}%
\pgfpathlineto{\pgfqpoint{1.145068in}{0.756582in}}%
\pgfpathmoveto{\pgfqpoint{1.145068in}{0.756582in}}%
\pgfpathlineto{\pgfqpoint{1.145068in}{0.756582in}}%
\pgfpathlineto{\pgfqpoint{1.145068in}{0.759531in}}%
\pgfpathlineto{\pgfqpoint{1.149609in}{0.759531in}}%
\pgfpathlineto{\pgfqpoint{1.149609in}{0.756582in}}%
\pgfpathmoveto{\pgfqpoint{1.145068in}{0.759531in}}%
\pgfpathlineto{\pgfqpoint{1.145068in}{0.759531in}}%
\pgfpathlineto{\pgfqpoint{1.145068in}{0.762480in}}%
\pgfpathlineto{\pgfqpoint{1.149609in}{0.762480in}}%
\pgfpathlineto{\pgfqpoint{1.149609in}{0.759531in}}%
\pgfpathmoveto{\pgfqpoint{1.149609in}{0.759531in}}%
\pgfpathlineto{\pgfqpoint{1.149609in}{0.759531in}}%
\pgfpathlineto{\pgfqpoint{1.149609in}{0.762480in}}%
\pgfpathlineto{\pgfqpoint{1.154151in}{0.762480in}}%
\pgfpathlineto{\pgfqpoint{1.154151in}{0.759531in}}%
\pgfpathmoveto{\pgfqpoint{1.149609in}{0.762480in}}%
\pgfpathlineto{\pgfqpoint{1.149609in}{0.762480in}}%
\pgfpathlineto{\pgfqpoint{1.149609in}{0.765429in}}%
\pgfpathlineto{\pgfqpoint{1.154151in}{0.765429in}}%
\pgfpathlineto{\pgfqpoint{1.154151in}{0.762480in}}%
\pgfpathmoveto{\pgfqpoint{1.154151in}{0.762480in}}%
\pgfpathlineto{\pgfqpoint{1.154151in}{0.762480in}}%
\pgfpathlineto{\pgfqpoint{1.154151in}{0.765429in}}%
\pgfpathlineto{\pgfqpoint{1.158692in}{0.765429in}}%
\pgfpathlineto{\pgfqpoint{1.158692in}{0.762480in}}%
\pgfpathmoveto{\pgfqpoint{1.154151in}{0.765429in}}%
\pgfpathlineto{\pgfqpoint{1.154151in}{0.765429in}}%
\pgfpathlineto{\pgfqpoint{1.154151in}{0.768378in}}%
\pgfpathlineto{\pgfqpoint{1.158692in}{0.768378in}}%
\pgfpathlineto{\pgfqpoint{1.158692in}{0.765429in}}%
\pgfpathmoveto{\pgfqpoint{1.158692in}{0.765429in}}%
\pgfpathlineto{\pgfqpoint{1.158692in}{0.765429in}}%
\pgfpathlineto{\pgfqpoint{1.158692in}{0.768378in}}%
\pgfpathlineto{\pgfqpoint{1.163233in}{0.768378in}}%
\pgfpathlineto{\pgfqpoint{1.163233in}{0.765429in}}%
\pgfpathmoveto{\pgfqpoint{1.158692in}{0.768378in}}%
\pgfpathlineto{\pgfqpoint{1.158692in}{0.768378in}}%
\pgfpathlineto{\pgfqpoint{1.158692in}{0.771327in}}%
\pgfpathlineto{\pgfqpoint{1.163233in}{0.771327in}}%
\pgfpathlineto{\pgfqpoint{1.163233in}{0.768378in}}%
\pgfpathmoveto{\pgfqpoint{1.163233in}{0.768378in}}%
\pgfpathlineto{\pgfqpoint{1.163233in}{0.768378in}}%
\pgfpathlineto{\pgfqpoint{1.163233in}{0.771327in}}%
\pgfpathlineto{\pgfqpoint{1.167774in}{0.771327in}}%
\pgfpathlineto{\pgfqpoint{1.167774in}{0.768378in}}%
\pgfpathmoveto{\pgfqpoint{1.163233in}{0.771327in}}%
\pgfpathlineto{\pgfqpoint{1.163233in}{0.771327in}}%
\pgfpathlineto{\pgfqpoint{1.163233in}{0.774276in}}%
\pgfpathlineto{\pgfqpoint{1.167774in}{0.774276in}}%
\pgfpathlineto{\pgfqpoint{1.167774in}{0.771327in}}%
\pgfpathmoveto{\pgfqpoint{1.167774in}{0.771327in}}%
\pgfpathlineto{\pgfqpoint{1.167774in}{0.771327in}}%
\pgfpathlineto{\pgfqpoint{1.167774in}{0.774276in}}%
\pgfpathlineto{\pgfqpoint{1.172315in}{0.774276in}}%
\pgfpathlineto{\pgfqpoint{1.172315in}{0.771327in}}%
\pgfpathmoveto{\pgfqpoint{1.167774in}{0.774276in}}%
\pgfpathlineto{\pgfqpoint{1.167774in}{0.774276in}}%
\pgfpathlineto{\pgfqpoint{1.167774in}{0.777226in}}%
\pgfpathlineto{\pgfqpoint{1.172315in}{0.777226in}}%
\pgfpathlineto{\pgfqpoint{1.172315in}{0.774276in}}%
\pgfpathmoveto{\pgfqpoint{1.172315in}{0.774276in}}%
\pgfpathlineto{\pgfqpoint{1.172315in}{0.774276in}}%
\pgfpathlineto{\pgfqpoint{1.172315in}{0.777226in}}%
\pgfpathlineto{\pgfqpoint{1.176856in}{0.777226in}}%
\pgfpathlineto{\pgfqpoint{1.176856in}{0.774276in}}%
\pgfpathmoveto{\pgfqpoint{1.172315in}{0.777226in}}%
\pgfpathlineto{\pgfqpoint{1.172315in}{0.777226in}}%
\pgfpathlineto{\pgfqpoint{1.172315in}{0.780175in}}%
\pgfpathlineto{\pgfqpoint{1.176856in}{0.780175in}}%
\pgfpathlineto{\pgfqpoint{1.176856in}{0.777226in}}%
\pgfpathmoveto{\pgfqpoint{1.176856in}{0.777226in}}%
\pgfpathlineto{\pgfqpoint{1.176856in}{0.777226in}}%
\pgfpathlineto{\pgfqpoint{1.176856in}{0.780175in}}%
\pgfpathlineto{\pgfqpoint{1.181397in}{0.780175in}}%
\pgfpathlineto{\pgfqpoint{1.181397in}{0.777226in}}%
\pgfpathmoveto{\pgfqpoint{1.176856in}{0.780175in}}%
\pgfpathlineto{\pgfqpoint{1.176856in}{0.780175in}}%
\pgfpathlineto{\pgfqpoint{1.176856in}{0.783124in}}%
\pgfpathlineto{\pgfqpoint{1.181397in}{0.783124in}}%
\pgfpathlineto{\pgfqpoint{1.181397in}{0.780175in}}%
\pgfpathmoveto{\pgfqpoint{1.181397in}{0.780175in}}%
\pgfpathlineto{\pgfqpoint{1.181397in}{0.780175in}}%
\pgfpathlineto{\pgfqpoint{1.181397in}{0.783124in}}%
\pgfpathlineto{\pgfqpoint{1.185939in}{0.783124in}}%
\pgfpathlineto{\pgfqpoint{1.185939in}{0.780175in}}%
\pgfpathmoveto{\pgfqpoint{1.181397in}{0.783124in}}%
\pgfpathlineto{\pgfqpoint{1.181397in}{0.783124in}}%
\pgfpathlineto{\pgfqpoint{1.181397in}{0.786073in}}%
\pgfpathlineto{\pgfqpoint{1.185939in}{0.786073in}}%
\pgfpathlineto{\pgfqpoint{1.185939in}{0.783124in}}%
\pgfpathmoveto{\pgfqpoint{1.185939in}{0.783124in}}%
\pgfpathlineto{\pgfqpoint{1.185939in}{0.783124in}}%
\pgfpathlineto{\pgfqpoint{1.185939in}{0.786073in}}%
\pgfpathlineto{\pgfqpoint{1.190479in}{0.786073in}}%
\pgfpathlineto{\pgfqpoint{1.190479in}{0.783124in}}%
\pgfpathmoveto{\pgfqpoint{1.185939in}{0.786073in}}%
\pgfpathlineto{\pgfqpoint{1.185939in}{0.786073in}}%
\pgfpathlineto{\pgfqpoint{1.185939in}{0.789022in}}%
\pgfpathlineto{\pgfqpoint{1.190479in}{0.789022in}}%
\pgfpathlineto{\pgfqpoint{1.190479in}{0.786073in}}%
\pgfpathmoveto{\pgfqpoint{1.190479in}{0.786073in}}%
\pgfpathlineto{\pgfqpoint{1.190479in}{0.786073in}}%
\pgfpathlineto{\pgfqpoint{1.190479in}{0.789022in}}%
\pgfpathlineto{\pgfqpoint{1.195020in}{0.789022in}}%
\pgfpathlineto{\pgfqpoint{1.195020in}{0.786073in}}%
\pgfpathmoveto{\pgfqpoint{1.190479in}{0.789022in}}%
\pgfpathlineto{\pgfqpoint{1.190479in}{0.789022in}}%
\pgfpathlineto{\pgfqpoint{1.190479in}{0.791972in}}%
\pgfpathlineto{\pgfqpoint{1.195020in}{0.791972in}}%
\pgfpathlineto{\pgfqpoint{1.195020in}{0.789022in}}%
\pgfpathmoveto{\pgfqpoint{1.195020in}{0.789022in}}%
\pgfpathlineto{\pgfqpoint{1.195020in}{0.789022in}}%
\pgfpathlineto{\pgfqpoint{1.195020in}{0.791972in}}%
\pgfpathlineto{\pgfqpoint{1.199561in}{0.791972in}}%
\pgfpathlineto{\pgfqpoint{1.199561in}{0.789022in}}%
\pgfpathmoveto{\pgfqpoint{1.195020in}{0.791972in}}%
\pgfpathlineto{\pgfqpoint{1.195020in}{0.791972in}}%
\pgfpathlineto{\pgfqpoint{1.195020in}{0.794921in}}%
\pgfpathlineto{\pgfqpoint{1.199561in}{0.794921in}}%
\pgfpathlineto{\pgfqpoint{1.199561in}{0.791972in}}%
\pgfpathmoveto{\pgfqpoint{1.199561in}{0.791972in}}%
\pgfpathlineto{\pgfqpoint{1.199561in}{0.791972in}}%
\pgfpathlineto{\pgfqpoint{1.199561in}{0.794921in}}%
\pgfpathlineto{\pgfqpoint{1.204102in}{0.794921in}}%
\pgfpathlineto{\pgfqpoint{1.204102in}{0.791972in}}%
\pgfpathmoveto{\pgfqpoint{1.199561in}{0.794921in}}%
\pgfpathlineto{\pgfqpoint{1.199561in}{0.794921in}}%
\pgfpathlineto{\pgfqpoint{1.199561in}{0.797870in}}%
\pgfpathlineto{\pgfqpoint{1.204102in}{0.797870in}}%
\pgfpathlineto{\pgfqpoint{1.204102in}{0.794921in}}%
\pgfpathmoveto{\pgfqpoint{1.204102in}{0.794921in}}%
\pgfpathlineto{\pgfqpoint{1.204102in}{0.794921in}}%
\pgfpathlineto{\pgfqpoint{1.204102in}{0.797870in}}%
\pgfpathlineto{\pgfqpoint{1.208643in}{0.797870in}}%
\pgfpathlineto{\pgfqpoint{1.208643in}{0.794921in}}%
\pgfpathmoveto{\pgfqpoint{1.204102in}{0.797870in}}%
\pgfpathlineto{\pgfqpoint{1.204102in}{0.797870in}}%
\pgfpathlineto{\pgfqpoint{1.204102in}{0.800819in}}%
\pgfpathlineto{\pgfqpoint{1.208643in}{0.800819in}}%
\pgfpathlineto{\pgfqpoint{1.208643in}{0.797870in}}%
\pgfpathmoveto{\pgfqpoint{1.208643in}{0.797870in}}%
\pgfpathlineto{\pgfqpoint{1.208643in}{0.797870in}}%
\pgfpathlineto{\pgfqpoint{1.208643in}{0.800819in}}%
\pgfpathlineto{\pgfqpoint{1.213184in}{0.800819in}}%
\pgfpathlineto{\pgfqpoint{1.213184in}{0.797870in}}%
\pgfpathmoveto{\pgfqpoint{1.208643in}{0.800819in}}%
\pgfpathlineto{\pgfqpoint{1.208643in}{0.800819in}}%
\pgfpathlineto{\pgfqpoint{1.208643in}{0.803769in}}%
\pgfpathlineto{\pgfqpoint{1.213184in}{0.803769in}}%
\pgfpathlineto{\pgfqpoint{1.213184in}{0.800819in}}%
\pgfpathmoveto{\pgfqpoint{1.213184in}{0.800819in}}%
\pgfpathlineto{\pgfqpoint{1.213184in}{0.800819in}}%
\pgfpathlineto{\pgfqpoint{1.213184in}{0.803769in}}%
\pgfpathlineto{\pgfqpoint{1.217724in}{0.803769in}}%
\pgfpathlineto{\pgfqpoint{1.217724in}{0.800819in}}%
\pgfpathmoveto{\pgfqpoint{1.213184in}{0.803769in}}%
\pgfpathlineto{\pgfqpoint{1.213184in}{0.803769in}}%
\pgfpathlineto{\pgfqpoint{1.213184in}{0.806718in}}%
\pgfpathlineto{\pgfqpoint{1.217724in}{0.806718in}}%
\pgfpathlineto{\pgfqpoint{1.217724in}{0.803769in}}%
\pgfpathmoveto{\pgfqpoint{1.217724in}{0.803769in}}%
\pgfpathlineto{\pgfqpoint{1.217724in}{0.803769in}}%
\pgfpathlineto{\pgfqpoint{1.217724in}{0.806718in}}%
\pgfpathlineto{\pgfqpoint{1.222265in}{0.806718in}}%
\pgfpathlineto{\pgfqpoint{1.222265in}{0.803769in}}%
\pgfpathmoveto{\pgfqpoint{1.217724in}{0.806718in}}%
\pgfpathlineto{\pgfqpoint{1.217724in}{0.806718in}}%
\pgfpathlineto{\pgfqpoint{1.217724in}{0.809667in}}%
\pgfpathlineto{\pgfqpoint{1.222265in}{0.809667in}}%
\pgfpathlineto{\pgfqpoint{1.222265in}{0.806718in}}%
\pgfpathmoveto{\pgfqpoint{1.222265in}{0.806718in}}%
\pgfpathlineto{\pgfqpoint{1.222265in}{0.806718in}}%
\pgfpathlineto{\pgfqpoint{1.222265in}{0.809667in}}%
\pgfpathlineto{\pgfqpoint{1.226806in}{0.809667in}}%
\pgfpathlineto{\pgfqpoint{1.226806in}{0.806718in}}%
\pgfpathmoveto{\pgfqpoint{1.222265in}{0.809667in}}%
\pgfpathlineto{\pgfqpoint{1.222265in}{0.809667in}}%
\pgfpathlineto{\pgfqpoint{1.222265in}{0.812617in}}%
\pgfpathlineto{\pgfqpoint{1.226806in}{0.812617in}}%
\pgfpathlineto{\pgfqpoint{1.226806in}{0.809667in}}%
\pgfpathmoveto{\pgfqpoint{1.226806in}{0.809667in}}%
\pgfpathlineto{\pgfqpoint{1.226806in}{0.809667in}}%
\pgfpathlineto{\pgfqpoint{1.226806in}{0.812617in}}%
\pgfpathlineto{\pgfqpoint{1.231347in}{0.812617in}}%
\pgfpathlineto{\pgfqpoint{1.231347in}{0.809667in}}%
\pgfpathmoveto{\pgfqpoint{1.226806in}{0.812617in}}%
\pgfpathlineto{\pgfqpoint{1.226806in}{0.812617in}}%
\pgfpathlineto{\pgfqpoint{1.226806in}{0.815566in}}%
\pgfpathlineto{\pgfqpoint{1.231347in}{0.815566in}}%
\pgfpathlineto{\pgfqpoint{1.231347in}{0.812617in}}%
\pgfpathmoveto{\pgfqpoint{1.231347in}{0.812617in}}%
\pgfpathlineto{\pgfqpoint{1.231347in}{0.812617in}}%
\pgfpathlineto{\pgfqpoint{1.231347in}{0.815566in}}%
\pgfpathlineto{\pgfqpoint{1.235888in}{0.815566in}}%
\pgfpathlineto{\pgfqpoint{1.235888in}{0.812617in}}%
\pgfpathmoveto{\pgfqpoint{1.235888in}{0.812617in}}%
\pgfpathlineto{\pgfqpoint{1.235888in}{0.812617in}}%
\pgfpathlineto{\pgfqpoint{1.235888in}{0.815566in}}%
\pgfpathlineto{\pgfqpoint{1.240429in}{0.815566in}}%
\pgfpathlineto{\pgfqpoint{1.240429in}{0.812617in}}%
\pgfpathmoveto{\pgfqpoint{1.235888in}{0.815566in}}%
\pgfpathlineto{\pgfqpoint{1.235888in}{0.815566in}}%
\pgfpathlineto{\pgfqpoint{1.235888in}{0.818515in}}%
\pgfpathlineto{\pgfqpoint{1.240429in}{0.818515in}}%
\pgfpathlineto{\pgfqpoint{1.240429in}{0.815566in}}%
\pgfpathmoveto{\pgfqpoint{1.240429in}{0.815566in}}%
\pgfpathlineto{\pgfqpoint{1.240429in}{0.815566in}}%
\pgfpathlineto{\pgfqpoint{1.240429in}{0.818515in}}%
\pgfpathlineto{\pgfqpoint{1.244969in}{0.818515in}}%
\pgfpathlineto{\pgfqpoint{1.244969in}{0.815566in}}%
\pgfpathmoveto{\pgfqpoint{1.240429in}{0.818515in}}%
\pgfpathlineto{\pgfqpoint{1.240429in}{0.818515in}}%
\pgfpathlineto{\pgfqpoint{1.240429in}{0.821464in}}%
\pgfpathlineto{\pgfqpoint{1.244969in}{0.821464in}}%
\pgfpathlineto{\pgfqpoint{1.244969in}{0.818515in}}%
\pgfpathmoveto{\pgfqpoint{1.244969in}{0.818515in}}%
\pgfpathlineto{\pgfqpoint{1.244969in}{0.818515in}}%
\pgfpathlineto{\pgfqpoint{1.244969in}{0.821464in}}%
\pgfpathlineto{\pgfqpoint{1.249510in}{0.821464in}}%
\pgfpathlineto{\pgfqpoint{1.249510in}{0.818515in}}%
\pgfpathmoveto{\pgfqpoint{1.244969in}{0.821464in}}%
\pgfpathlineto{\pgfqpoint{1.244969in}{0.821464in}}%
\pgfpathlineto{\pgfqpoint{1.244969in}{0.824414in}}%
\pgfpathlineto{\pgfqpoint{1.249510in}{0.824414in}}%
\pgfpathlineto{\pgfqpoint{1.249510in}{0.821464in}}%
\pgfpathmoveto{\pgfqpoint{1.249510in}{0.821464in}}%
\pgfpathlineto{\pgfqpoint{1.249510in}{0.821464in}}%
\pgfpathlineto{\pgfqpoint{1.249510in}{0.824414in}}%
\pgfpathlineto{\pgfqpoint{1.254051in}{0.824414in}}%
\pgfpathlineto{\pgfqpoint{1.254051in}{0.821464in}}%
\pgfpathmoveto{\pgfqpoint{1.249510in}{0.824414in}}%
\pgfpathlineto{\pgfqpoint{1.249510in}{0.824414in}}%
\pgfpathlineto{\pgfqpoint{1.249510in}{0.827363in}}%
\pgfpathlineto{\pgfqpoint{1.254051in}{0.827363in}}%
\pgfpathlineto{\pgfqpoint{1.254051in}{0.824414in}}%
\pgfpathmoveto{\pgfqpoint{1.254051in}{0.824414in}}%
\pgfpathlineto{\pgfqpoint{1.254051in}{0.824414in}}%
\pgfpathlineto{\pgfqpoint{1.254051in}{0.827363in}}%
\pgfpathlineto{\pgfqpoint{1.258592in}{0.827363in}}%
\pgfpathlineto{\pgfqpoint{1.258592in}{0.824414in}}%
\pgfpathmoveto{\pgfqpoint{1.254051in}{0.827363in}}%
\pgfpathlineto{\pgfqpoint{1.254051in}{0.827363in}}%
\pgfpathlineto{\pgfqpoint{1.254051in}{0.830312in}}%
\pgfpathlineto{\pgfqpoint{1.258592in}{0.830312in}}%
\pgfpathlineto{\pgfqpoint{1.258592in}{0.827363in}}%
\pgfpathmoveto{\pgfqpoint{1.258592in}{0.827363in}}%
\pgfpathlineto{\pgfqpoint{1.258592in}{0.827363in}}%
\pgfpathlineto{\pgfqpoint{1.258592in}{0.830312in}}%
\pgfpathlineto{\pgfqpoint{1.263133in}{0.830312in}}%
\pgfpathlineto{\pgfqpoint{1.263133in}{0.827363in}}%
\pgfpathmoveto{\pgfqpoint{1.258592in}{0.830312in}}%
\pgfpathlineto{\pgfqpoint{1.258592in}{0.830312in}}%
\pgfpathlineto{\pgfqpoint{1.258592in}{0.833262in}}%
\pgfpathlineto{\pgfqpoint{1.263133in}{0.833262in}}%
\pgfpathlineto{\pgfqpoint{1.263133in}{0.830312in}}%
\pgfpathmoveto{\pgfqpoint{1.263133in}{0.830312in}}%
\pgfpathlineto{\pgfqpoint{1.263133in}{0.830312in}}%
\pgfpathlineto{\pgfqpoint{1.263133in}{0.833262in}}%
\pgfpathlineto{\pgfqpoint{1.267674in}{0.833262in}}%
\pgfpathlineto{\pgfqpoint{1.267674in}{0.830312in}}%
\pgfpathmoveto{\pgfqpoint{1.263133in}{0.833262in}}%
\pgfpathlineto{\pgfqpoint{1.263133in}{0.833262in}}%
\pgfpathlineto{\pgfqpoint{1.263133in}{0.836211in}}%
\pgfpathlineto{\pgfqpoint{1.267674in}{0.836211in}}%
\pgfpathlineto{\pgfqpoint{1.267674in}{0.833262in}}%
\pgfpathmoveto{\pgfqpoint{1.267674in}{0.833262in}}%
\pgfpathlineto{\pgfqpoint{1.267674in}{0.833262in}}%
\pgfpathlineto{\pgfqpoint{1.267674in}{0.836211in}}%
\pgfpathlineto{\pgfqpoint{1.272215in}{0.836211in}}%
\pgfpathlineto{\pgfqpoint{1.272215in}{0.833262in}}%
\pgfpathmoveto{\pgfqpoint{1.267674in}{0.836211in}}%
\pgfpathlineto{\pgfqpoint{1.267674in}{0.836211in}}%
\pgfpathlineto{\pgfqpoint{1.267674in}{0.839160in}}%
\pgfpathlineto{\pgfqpoint{1.272215in}{0.839160in}}%
\pgfpathlineto{\pgfqpoint{1.272215in}{0.836211in}}%
\pgfpathmoveto{\pgfqpoint{1.272215in}{0.836211in}}%
\pgfpathlineto{\pgfqpoint{1.272215in}{0.836211in}}%
\pgfpathlineto{\pgfqpoint{1.272215in}{0.839160in}}%
\pgfpathlineto{\pgfqpoint{1.276755in}{0.839160in}}%
\pgfpathlineto{\pgfqpoint{1.276755in}{0.836211in}}%
\pgfpathmoveto{\pgfqpoint{1.272215in}{0.839160in}}%
\pgfpathlineto{\pgfqpoint{1.272215in}{0.839160in}}%
\pgfpathlineto{\pgfqpoint{1.272215in}{0.842109in}}%
\pgfpathlineto{\pgfqpoint{1.276755in}{0.842109in}}%
\pgfpathlineto{\pgfqpoint{1.276755in}{0.839160in}}%
\pgfpathmoveto{\pgfqpoint{1.276755in}{0.839160in}}%
\pgfpathlineto{\pgfqpoint{1.276755in}{0.839160in}}%
\pgfpathlineto{\pgfqpoint{1.276755in}{0.842109in}}%
\pgfpathlineto{\pgfqpoint{1.281296in}{0.842109in}}%
\pgfpathlineto{\pgfqpoint{1.281296in}{0.839160in}}%
\pgfpathmoveto{\pgfqpoint{1.276755in}{0.842109in}}%
\pgfpathlineto{\pgfqpoint{1.276755in}{0.842109in}}%
\pgfpathlineto{\pgfqpoint{1.276755in}{0.845059in}}%
\pgfpathlineto{\pgfqpoint{1.281296in}{0.845059in}}%
\pgfpathlineto{\pgfqpoint{1.281296in}{0.842109in}}%
\pgfpathmoveto{\pgfqpoint{1.281296in}{0.842109in}}%
\pgfpathlineto{\pgfqpoint{1.281296in}{0.842109in}}%
\pgfpathlineto{\pgfqpoint{1.281296in}{0.845059in}}%
\pgfpathlineto{\pgfqpoint{1.285837in}{0.845059in}}%
\pgfpathlineto{\pgfqpoint{1.285837in}{0.842109in}}%
\pgfpathmoveto{\pgfqpoint{1.281296in}{0.845059in}}%
\pgfpathlineto{\pgfqpoint{1.281296in}{0.845059in}}%
\pgfpathlineto{\pgfqpoint{1.281296in}{0.848008in}}%
\pgfpathlineto{\pgfqpoint{1.285837in}{0.848008in}}%
\pgfpathlineto{\pgfqpoint{1.285837in}{0.845059in}}%
\pgfpathmoveto{\pgfqpoint{1.285837in}{0.845059in}}%
\pgfpathlineto{\pgfqpoint{1.285837in}{0.845059in}}%
\pgfpathlineto{\pgfqpoint{1.285837in}{0.848008in}}%
\pgfpathlineto{\pgfqpoint{1.290378in}{0.848008in}}%
\pgfpathlineto{\pgfqpoint{1.290378in}{0.845059in}}%
\pgfpathmoveto{\pgfqpoint{1.285837in}{0.848008in}}%
\pgfpathlineto{\pgfqpoint{1.285837in}{0.848008in}}%
\pgfpathlineto{\pgfqpoint{1.285837in}{0.850957in}}%
\pgfpathlineto{\pgfqpoint{1.290378in}{0.850957in}}%
\pgfpathlineto{\pgfqpoint{1.290378in}{0.848008in}}%
\pgfpathmoveto{\pgfqpoint{1.290378in}{0.848008in}}%
\pgfpathlineto{\pgfqpoint{1.290378in}{0.848008in}}%
\pgfpathlineto{\pgfqpoint{1.290378in}{0.850957in}}%
\pgfpathlineto{\pgfqpoint{1.294919in}{0.850957in}}%
\pgfpathlineto{\pgfqpoint{1.294919in}{0.848008in}}%
\pgfpathmoveto{\pgfqpoint{1.290378in}{0.850957in}}%
\pgfpathlineto{\pgfqpoint{1.290378in}{0.850957in}}%
\pgfpathlineto{\pgfqpoint{1.290378in}{0.853907in}}%
\pgfpathlineto{\pgfqpoint{1.294919in}{0.853907in}}%
\pgfpathlineto{\pgfqpoint{1.294919in}{0.850957in}}%
\pgfpathmoveto{\pgfqpoint{1.294919in}{0.850957in}}%
\pgfpathlineto{\pgfqpoint{1.294919in}{0.850957in}}%
\pgfpathlineto{\pgfqpoint{1.294919in}{0.853907in}}%
\pgfpathlineto{\pgfqpoint{1.299460in}{0.853907in}}%
\pgfpathlineto{\pgfqpoint{1.299460in}{0.850957in}}%
\pgfpathmoveto{\pgfqpoint{1.294919in}{0.853907in}}%
\pgfpathlineto{\pgfqpoint{1.294919in}{0.853907in}}%
\pgfpathlineto{\pgfqpoint{1.294919in}{0.856856in}}%
\pgfpathlineto{\pgfqpoint{1.299460in}{0.856856in}}%
\pgfpathlineto{\pgfqpoint{1.299460in}{0.853907in}}%
\pgfpathmoveto{\pgfqpoint{1.299460in}{0.853907in}}%
\pgfpathlineto{\pgfqpoint{1.299460in}{0.853907in}}%
\pgfpathlineto{\pgfqpoint{1.299460in}{0.856856in}}%
\pgfpathlineto{\pgfqpoint{1.304000in}{0.856856in}}%
\pgfpathlineto{\pgfqpoint{1.304000in}{0.853907in}}%
\pgfpathmoveto{\pgfqpoint{1.299460in}{0.856856in}}%
\pgfpathlineto{\pgfqpoint{1.299460in}{0.856856in}}%
\pgfpathlineto{\pgfqpoint{1.299460in}{0.859805in}}%
\pgfpathlineto{\pgfqpoint{1.304000in}{0.859805in}}%
\pgfpathlineto{\pgfqpoint{1.304000in}{0.856856in}}%
\pgfpathmoveto{\pgfqpoint{1.304000in}{0.856856in}}%
\pgfpathlineto{\pgfqpoint{1.304000in}{0.856856in}}%
\pgfpathlineto{\pgfqpoint{1.304000in}{0.859805in}}%
\pgfpathlineto{\pgfqpoint{1.308541in}{0.859805in}}%
\pgfpathlineto{\pgfqpoint{1.308541in}{0.856856in}}%
\pgfpathmoveto{\pgfqpoint{1.304000in}{0.859805in}}%
\pgfpathlineto{\pgfqpoint{1.304000in}{0.859805in}}%
\pgfpathlineto{\pgfqpoint{1.304000in}{0.862754in}}%
\pgfpathlineto{\pgfqpoint{1.308541in}{0.862754in}}%
\pgfpathlineto{\pgfqpoint{1.308541in}{0.859805in}}%
\pgfpathmoveto{\pgfqpoint{1.308541in}{0.859805in}}%
\pgfpathlineto{\pgfqpoint{1.308541in}{0.859805in}}%
\pgfpathlineto{\pgfqpoint{1.308541in}{0.862754in}}%
\pgfpathlineto{\pgfqpoint{1.313082in}{0.862754in}}%
\pgfpathlineto{\pgfqpoint{1.313082in}{0.859805in}}%
\pgfpathmoveto{\pgfqpoint{1.308541in}{0.862754in}}%
\pgfpathlineto{\pgfqpoint{1.308541in}{0.862754in}}%
\pgfpathlineto{\pgfqpoint{1.308541in}{0.865704in}}%
\pgfpathlineto{\pgfqpoint{1.313082in}{0.865704in}}%
\pgfpathlineto{\pgfqpoint{1.313082in}{0.862754in}}%
\pgfpathmoveto{\pgfqpoint{1.313082in}{0.862754in}}%
\pgfpathlineto{\pgfqpoint{1.313082in}{0.862754in}}%
\pgfpathlineto{\pgfqpoint{1.313082in}{0.865704in}}%
\pgfpathlineto{\pgfqpoint{1.317623in}{0.865704in}}%
\pgfpathlineto{\pgfqpoint{1.317623in}{0.862754in}}%
\pgfpathmoveto{\pgfqpoint{1.313082in}{0.865704in}}%
\pgfpathlineto{\pgfqpoint{1.313082in}{0.865704in}}%
\pgfpathlineto{\pgfqpoint{1.313082in}{0.868653in}}%
\pgfpathlineto{\pgfqpoint{1.317623in}{0.868653in}}%
\pgfpathlineto{\pgfqpoint{1.317623in}{0.865704in}}%
\pgfpathmoveto{\pgfqpoint{1.317623in}{0.865704in}}%
\pgfpathlineto{\pgfqpoint{1.317623in}{0.865704in}}%
\pgfpathlineto{\pgfqpoint{1.317623in}{0.868653in}}%
\pgfpathlineto{\pgfqpoint{1.322164in}{0.868653in}}%
\pgfpathlineto{\pgfqpoint{1.322164in}{0.865704in}}%
\pgfpathmoveto{\pgfqpoint{1.317623in}{0.868653in}}%
\pgfpathlineto{\pgfqpoint{1.317623in}{0.868653in}}%
\pgfpathlineto{\pgfqpoint{1.317623in}{0.871602in}}%
\pgfpathlineto{\pgfqpoint{1.322164in}{0.871602in}}%
\pgfpathlineto{\pgfqpoint{1.322164in}{0.868653in}}%
\pgfpathmoveto{\pgfqpoint{1.322164in}{0.868653in}}%
\pgfpathlineto{\pgfqpoint{1.322164in}{0.868653in}}%
\pgfpathlineto{\pgfqpoint{1.322164in}{0.871602in}}%
\pgfpathlineto{\pgfqpoint{1.326705in}{0.871602in}}%
\pgfpathlineto{\pgfqpoint{1.326705in}{0.868653in}}%
\pgfpathmoveto{\pgfqpoint{1.322164in}{0.871602in}}%
\pgfpathlineto{\pgfqpoint{1.322164in}{0.871602in}}%
\pgfpathlineto{\pgfqpoint{1.322164in}{0.874552in}}%
\pgfpathlineto{\pgfqpoint{1.326705in}{0.874552in}}%
\pgfpathlineto{\pgfqpoint{1.326705in}{0.871602in}}%
\pgfpathmoveto{\pgfqpoint{1.326705in}{0.871602in}}%
\pgfpathlineto{\pgfqpoint{1.326705in}{0.871602in}}%
\pgfpathlineto{\pgfqpoint{1.326705in}{0.874552in}}%
\pgfpathlineto{\pgfqpoint{1.331246in}{0.874552in}}%
\pgfpathlineto{\pgfqpoint{1.331246in}{0.871602in}}%
\pgfpathmoveto{\pgfqpoint{1.326705in}{0.874552in}}%
\pgfpathlineto{\pgfqpoint{1.326705in}{0.874552in}}%
\pgfpathlineto{\pgfqpoint{1.326705in}{0.877501in}}%
\pgfpathlineto{\pgfqpoint{1.331246in}{0.877501in}}%
\pgfpathlineto{\pgfqpoint{1.331246in}{0.874552in}}%
\pgfpathmoveto{\pgfqpoint{1.331246in}{0.874552in}}%
\pgfpathlineto{\pgfqpoint{1.331246in}{0.874552in}}%
\pgfpathlineto{\pgfqpoint{1.331246in}{0.877501in}}%
\pgfpathlineto{\pgfqpoint{1.335787in}{0.877501in}}%
\pgfpathlineto{\pgfqpoint{1.335787in}{0.874552in}}%
\pgfpathmoveto{\pgfqpoint{1.331246in}{0.877501in}}%
\pgfpathlineto{\pgfqpoint{1.331246in}{0.877501in}}%
\pgfpathlineto{\pgfqpoint{1.331246in}{0.880450in}}%
\pgfpathlineto{\pgfqpoint{1.335787in}{0.880450in}}%
\pgfpathlineto{\pgfqpoint{1.335787in}{0.877501in}}%
\pgfpathmoveto{\pgfqpoint{1.335787in}{0.877501in}}%
\pgfpathlineto{\pgfqpoint{1.335787in}{0.877501in}}%
\pgfpathlineto{\pgfqpoint{1.335787in}{0.880450in}}%
\pgfpathlineto{\pgfqpoint{1.340328in}{0.880450in}}%
\pgfpathlineto{\pgfqpoint{1.340328in}{0.877501in}}%
\pgfpathmoveto{\pgfqpoint{1.335787in}{0.880450in}}%
\pgfpathlineto{\pgfqpoint{1.335787in}{0.880450in}}%
\pgfpathlineto{\pgfqpoint{1.335787in}{0.883399in}}%
\pgfpathlineto{\pgfqpoint{1.340328in}{0.883399in}}%
\pgfpathlineto{\pgfqpoint{1.340328in}{0.880450in}}%
\pgfpathmoveto{\pgfqpoint{1.340328in}{0.880450in}}%
\pgfpathlineto{\pgfqpoint{1.340328in}{0.880450in}}%
\pgfpathlineto{\pgfqpoint{1.340328in}{0.883399in}}%
\pgfpathlineto{\pgfqpoint{1.344869in}{0.883399in}}%
\pgfpathlineto{\pgfqpoint{1.344869in}{0.880450in}}%
\pgfpathmoveto{\pgfqpoint{1.340328in}{0.883399in}}%
\pgfpathlineto{\pgfqpoint{1.340328in}{0.883399in}}%
\pgfpathlineto{\pgfqpoint{1.340328in}{0.886348in}}%
\pgfpathlineto{\pgfqpoint{1.344869in}{0.886348in}}%
\pgfpathlineto{\pgfqpoint{1.344869in}{0.883399in}}%
\pgfpathmoveto{\pgfqpoint{1.344869in}{0.883399in}}%
\pgfpathlineto{\pgfqpoint{1.344869in}{0.883399in}}%
\pgfpathlineto{\pgfqpoint{1.344869in}{0.886348in}}%
\pgfpathlineto{\pgfqpoint{1.349411in}{0.886348in}}%
\pgfpathlineto{\pgfqpoint{1.349411in}{0.883399in}}%
\pgfpathmoveto{\pgfqpoint{1.344869in}{0.886348in}}%
\pgfpathlineto{\pgfqpoint{1.344869in}{0.886348in}}%
\pgfpathlineto{\pgfqpoint{1.344869in}{0.889297in}}%
\pgfpathlineto{\pgfqpoint{1.349411in}{0.889297in}}%
\pgfpathlineto{\pgfqpoint{1.349411in}{0.886348in}}%
\pgfpathmoveto{\pgfqpoint{1.349411in}{0.886348in}}%
\pgfpathlineto{\pgfqpoint{1.349411in}{0.886348in}}%
\pgfpathlineto{\pgfqpoint{1.349411in}{0.889297in}}%
\pgfpathlineto{\pgfqpoint{1.353952in}{0.889297in}}%
\pgfpathlineto{\pgfqpoint{1.353952in}{0.886348in}}%
\pgfpathmoveto{\pgfqpoint{1.349411in}{0.889297in}}%
\pgfpathlineto{\pgfqpoint{1.349411in}{0.889297in}}%
\pgfpathlineto{\pgfqpoint{1.349411in}{0.892246in}}%
\pgfpathlineto{\pgfqpoint{1.353952in}{0.892246in}}%
\pgfpathlineto{\pgfqpoint{1.353952in}{0.889297in}}%
\pgfpathmoveto{\pgfqpoint{1.353952in}{0.889297in}}%
\pgfpathlineto{\pgfqpoint{1.353952in}{0.889297in}}%
\pgfpathlineto{\pgfqpoint{1.353952in}{0.892246in}}%
\pgfpathlineto{\pgfqpoint{1.358493in}{0.892246in}}%
\pgfpathlineto{\pgfqpoint{1.358493in}{0.889297in}}%
\pgfpathmoveto{\pgfqpoint{1.353952in}{0.892246in}}%
\pgfpathlineto{\pgfqpoint{1.353952in}{0.892246in}}%
\pgfpathlineto{\pgfqpoint{1.353952in}{0.895196in}}%
\pgfpathlineto{\pgfqpoint{1.358493in}{0.895196in}}%
\pgfpathlineto{\pgfqpoint{1.358493in}{0.892246in}}%
\pgfpathmoveto{\pgfqpoint{1.358493in}{0.892246in}}%
\pgfpathlineto{\pgfqpoint{1.358493in}{0.892246in}}%
\pgfpathlineto{\pgfqpoint{1.358493in}{0.895196in}}%
\pgfpathlineto{\pgfqpoint{1.363034in}{0.895196in}}%
\pgfpathlineto{\pgfqpoint{1.363034in}{0.892246in}}%
\pgfpathmoveto{\pgfqpoint{1.358493in}{0.895196in}}%
\pgfpathlineto{\pgfqpoint{1.358493in}{0.895196in}}%
\pgfpathlineto{\pgfqpoint{1.358493in}{0.898145in}}%
\pgfpathlineto{\pgfqpoint{1.363034in}{0.898145in}}%
\pgfpathlineto{\pgfqpoint{1.363034in}{0.895196in}}%
\pgfpathmoveto{\pgfqpoint{1.363034in}{0.895196in}}%
\pgfpathlineto{\pgfqpoint{1.363034in}{0.895196in}}%
\pgfpathlineto{\pgfqpoint{1.363034in}{0.898145in}}%
\pgfpathlineto{\pgfqpoint{1.367576in}{0.898145in}}%
\pgfpathlineto{\pgfqpoint{1.367576in}{0.895196in}}%
\pgfpathmoveto{\pgfqpoint{1.363034in}{0.898145in}}%
\pgfpathlineto{\pgfqpoint{1.363034in}{0.898145in}}%
\pgfpathlineto{\pgfqpoint{1.363034in}{0.901094in}}%
\pgfpathlineto{\pgfqpoint{1.367576in}{0.901094in}}%
\pgfpathlineto{\pgfqpoint{1.367576in}{0.898145in}}%
\pgfpathmoveto{\pgfqpoint{1.367576in}{0.898145in}}%
\pgfpathlineto{\pgfqpoint{1.367576in}{0.898145in}}%
\pgfpathlineto{\pgfqpoint{1.367576in}{0.901094in}}%
\pgfpathlineto{\pgfqpoint{1.372117in}{0.901094in}}%
\pgfpathlineto{\pgfqpoint{1.372117in}{0.898145in}}%
\pgfpathmoveto{\pgfqpoint{1.367576in}{0.901094in}}%
\pgfpathlineto{\pgfqpoint{1.367576in}{0.901094in}}%
\pgfpathlineto{\pgfqpoint{1.367576in}{0.904043in}}%
\pgfpathlineto{\pgfqpoint{1.372117in}{0.904043in}}%
\pgfpathlineto{\pgfqpoint{1.372117in}{0.901094in}}%
\pgfpathmoveto{\pgfqpoint{1.372117in}{0.901094in}}%
\pgfpathlineto{\pgfqpoint{1.372117in}{0.901094in}}%
\pgfpathlineto{\pgfqpoint{1.372117in}{0.904043in}}%
\pgfpathlineto{\pgfqpoint{1.376658in}{0.904043in}}%
\pgfpathlineto{\pgfqpoint{1.376658in}{0.901094in}}%
\pgfpathmoveto{\pgfqpoint{1.372117in}{0.904043in}}%
\pgfpathlineto{\pgfqpoint{1.372117in}{0.904043in}}%
\pgfpathlineto{\pgfqpoint{1.372117in}{0.906992in}}%
\pgfpathlineto{\pgfqpoint{1.376658in}{0.906992in}}%
\pgfpathlineto{\pgfqpoint{1.376658in}{0.904043in}}%
\pgfpathmoveto{\pgfqpoint{1.376658in}{0.904043in}}%
\pgfpathlineto{\pgfqpoint{1.376658in}{0.904043in}}%
\pgfpathlineto{\pgfqpoint{1.376658in}{0.906992in}}%
\pgfpathlineto{\pgfqpoint{1.381199in}{0.906992in}}%
\pgfpathlineto{\pgfqpoint{1.381199in}{0.904043in}}%
\pgfpathmoveto{\pgfqpoint{1.376658in}{0.906992in}}%
\pgfpathlineto{\pgfqpoint{1.376658in}{0.906992in}}%
\pgfpathlineto{\pgfqpoint{1.376658in}{0.909941in}}%
\pgfpathlineto{\pgfqpoint{1.381199in}{0.909941in}}%
\pgfpathlineto{\pgfqpoint{1.381199in}{0.906992in}}%
\pgfpathmoveto{\pgfqpoint{1.381199in}{0.906992in}}%
\pgfpathlineto{\pgfqpoint{1.381199in}{0.906992in}}%
\pgfpathlineto{\pgfqpoint{1.381199in}{0.909941in}}%
\pgfpathlineto{\pgfqpoint{1.385741in}{0.909941in}}%
\pgfpathlineto{\pgfqpoint{1.385741in}{0.906992in}}%
\pgfpathmoveto{\pgfqpoint{1.381199in}{0.909941in}}%
\pgfpathlineto{\pgfqpoint{1.381199in}{0.909941in}}%
\pgfpathlineto{\pgfqpoint{1.381199in}{0.912890in}}%
\pgfpathlineto{\pgfqpoint{1.385741in}{0.912890in}}%
\pgfpathlineto{\pgfqpoint{1.385741in}{0.909941in}}%
\pgfpathmoveto{\pgfqpoint{1.385741in}{0.909941in}}%
\pgfpathlineto{\pgfqpoint{1.385741in}{0.909941in}}%
\pgfpathlineto{\pgfqpoint{1.385741in}{0.912890in}}%
\pgfpathlineto{\pgfqpoint{1.390282in}{0.912890in}}%
\pgfpathlineto{\pgfqpoint{1.390282in}{0.909941in}}%
\pgfpathmoveto{\pgfqpoint{1.385741in}{0.912890in}}%
\pgfpathlineto{\pgfqpoint{1.385741in}{0.912890in}}%
\pgfpathlineto{\pgfqpoint{1.385741in}{0.915839in}}%
\pgfpathlineto{\pgfqpoint{1.390282in}{0.915839in}}%
\pgfpathlineto{\pgfqpoint{1.390282in}{0.912890in}}%
\pgfpathmoveto{\pgfqpoint{1.390282in}{0.912890in}}%
\pgfpathlineto{\pgfqpoint{1.390282in}{0.912890in}}%
\pgfpathlineto{\pgfqpoint{1.390282in}{0.915839in}}%
\pgfpathlineto{\pgfqpoint{1.394823in}{0.915839in}}%
\pgfpathlineto{\pgfqpoint{1.394823in}{0.912890in}}%
\pgfpathmoveto{\pgfqpoint{1.390282in}{0.915839in}}%
\pgfpathlineto{\pgfqpoint{1.390282in}{0.915839in}}%
\pgfpathlineto{\pgfqpoint{1.390282in}{0.918789in}}%
\pgfpathlineto{\pgfqpoint{1.394823in}{0.918789in}}%
\pgfpathlineto{\pgfqpoint{1.394823in}{0.915839in}}%
\pgfpathmoveto{\pgfqpoint{1.394823in}{0.915839in}}%
\pgfpathlineto{\pgfqpoint{1.394823in}{0.915839in}}%
\pgfpathlineto{\pgfqpoint{1.394823in}{0.918789in}}%
\pgfpathlineto{\pgfqpoint{1.399365in}{0.918789in}}%
\pgfpathlineto{\pgfqpoint{1.399365in}{0.915839in}}%
\pgfpathmoveto{\pgfqpoint{1.394823in}{0.918789in}}%
\pgfpathlineto{\pgfqpoint{1.394823in}{0.918789in}}%
\pgfpathlineto{\pgfqpoint{1.394823in}{0.921738in}}%
\pgfpathlineto{\pgfqpoint{1.399365in}{0.921738in}}%
\pgfpathlineto{\pgfqpoint{1.399365in}{0.918789in}}%
\pgfpathmoveto{\pgfqpoint{1.394823in}{0.921738in}}%
\pgfpathlineto{\pgfqpoint{1.394823in}{0.921738in}}%
\pgfpathlineto{\pgfqpoint{1.394823in}{0.924687in}}%
\pgfpathlineto{\pgfqpoint{1.399365in}{0.924687in}}%
\pgfpathlineto{\pgfqpoint{1.399365in}{0.921738in}}%
\pgfpathmoveto{\pgfqpoint{1.399365in}{0.921738in}}%
\pgfpathlineto{\pgfqpoint{1.399365in}{0.921738in}}%
\pgfpathlineto{\pgfqpoint{1.399365in}{0.924687in}}%
\pgfpathlineto{\pgfqpoint{1.403906in}{0.924687in}}%
\pgfpathlineto{\pgfqpoint{1.403906in}{0.921738in}}%
\pgfpathmoveto{\pgfqpoint{1.399365in}{0.924687in}}%
\pgfpathlineto{\pgfqpoint{1.399365in}{0.924687in}}%
\pgfpathlineto{\pgfqpoint{1.399365in}{0.927636in}}%
\pgfpathlineto{\pgfqpoint{1.403906in}{0.927636in}}%
\pgfpathlineto{\pgfqpoint{1.403906in}{0.924687in}}%
\pgfpathmoveto{\pgfqpoint{1.403906in}{0.924687in}}%
\pgfpathlineto{\pgfqpoint{1.403906in}{0.924687in}}%
\pgfpathlineto{\pgfqpoint{1.403906in}{0.927636in}}%
\pgfpathlineto{\pgfqpoint{1.408447in}{0.927636in}}%
\pgfpathlineto{\pgfqpoint{1.408447in}{0.924687in}}%
\pgfpathmoveto{\pgfqpoint{1.403906in}{0.927636in}}%
\pgfpathlineto{\pgfqpoint{1.403906in}{0.927636in}}%
\pgfpathlineto{\pgfqpoint{1.403906in}{0.930585in}}%
\pgfpathlineto{\pgfqpoint{1.408447in}{0.930585in}}%
\pgfpathlineto{\pgfqpoint{1.408447in}{0.927636in}}%
\pgfpathmoveto{\pgfqpoint{1.408447in}{0.927636in}}%
\pgfpathlineto{\pgfqpoint{1.408447in}{0.927636in}}%
\pgfpathlineto{\pgfqpoint{1.408447in}{0.930585in}}%
\pgfpathlineto{\pgfqpoint{1.412988in}{0.930585in}}%
\pgfpathlineto{\pgfqpoint{1.412988in}{0.927636in}}%
\pgfpathmoveto{\pgfqpoint{1.408447in}{0.930585in}}%
\pgfpathlineto{\pgfqpoint{1.408447in}{0.930585in}}%
\pgfpathlineto{\pgfqpoint{1.408447in}{0.933534in}}%
\pgfpathlineto{\pgfqpoint{1.412988in}{0.933534in}}%
\pgfpathlineto{\pgfqpoint{1.412988in}{0.930585in}}%
\pgfpathmoveto{\pgfqpoint{1.412988in}{0.930585in}}%
\pgfpathlineto{\pgfqpoint{1.412988in}{0.930585in}}%
\pgfpathlineto{\pgfqpoint{1.412988in}{0.933534in}}%
\pgfpathlineto{\pgfqpoint{1.417530in}{0.933534in}}%
\pgfpathlineto{\pgfqpoint{1.417530in}{0.930585in}}%
\pgfpathmoveto{\pgfqpoint{1.412988in}{0.933534in}}%
\pgfpathlineto{\pgfqpoint{1.412988in}{0.933534in}}%
\pgfpathlineto{\pgfqpoint{1.412988in}{0.936483in}}%
\pgfpathlineto{\pgfqpoint{1.417530in}{0.936483in}}%
\pgfpathlineto{\pgfqpoint{1.417530in}{0.933534in}}%
\pgfpathmoveto{\pgfqpoint{1.417530in}{0.933534in}}%
\pgfpathlineto{\pgfqpoint{1.417530in}{0.933534in}}%
\pgfpathlineto{\pgfqpoint{1.417530in}{0.936483in}}%
\pgfpathlineto{\pgfqpoint{1.422071in}{0.936483in}}%
\pgfpathlineto{\pgfqpoint{1.422071in}{0.933534in}}%
\pgfpathmoveto{\pgfqpoint{1.417530in}{0.936483in}}%
\pgfpathlineto{\pgfqpoint{1.417530in}{0.936483in}}%
\pgfpathlineto{\pgfqpoint{1.417530in}{0.939432in}}%
\pgfpathlineto{\pgfqpoint{1.422071in}{0.939432in}}%
\pgfpathlineto{\pgfqpoint{1.422071in}{0.936483in}}%
\pgfpathmoveto{\pgfqpoint{1.422071in}{0.936483in}}%
\pgfpathlineto{\pgfqpoint{1.422071in}{0.936483in}}%
\pgfpathlineto{\pgfqpoint{1.422071in}{0.939432in}}%
\pgfpathlineto{\pgfqpoint{1.426612in}{0.939432in}}%
\pgfpathlineto{\pgfqpoint{1.426612in}{0.936483in}}%
\pgfpathmoveto{\pgfqpoint{1.422071in}{0.939432in}}%
\pgfpathlineto{\pgfqpoint{1.422071in}{0.939432in}}%
\pgfpathlineto{\pgfqpoint{1.422071in}{0.942382in}}%
\pgfpathlineto{\pgfqpoint{1.426612in}{0.942382in}}%
\pgfpathlineto{\pgfqpoint{1.426612in}{0.939432in}}%
\pgfpathmoveto{\pgfqpoint{1.426612in}{0.939432in}}%
\pgfpathlineto{\pgfqpoint{1.426612in}{0.939432in}}%
\pgfpathlineto{\pgfqpoint{1.426612in}{0.942382in}}%
\pgfpathlineto{\pgfqpoint{1.431153in}{0.942382in}}%
\pgfpathlineto{\pgfqpoint{1.431153in}{0.939432in}}%
\pgfpathmoveto{\pgfqpoint{1.426612in}{0.942382in}}%
\pgfpathlineto{\pgfqpoint{1.426612in}{0.942382in}}%
\pgfpathlineto{\pgfqpoint{1.426612in}{0.945331in}}%
\pgfpathlineto{\pgfqpoint{1.431153in}{0.945331in}}%
\pgfpathlineto{\pgfqpoint{1.431153in}{0.942382in}}%
\pgfpathmoveto{\pgfqpoint{1.431153in}{0.942382in}}%
\pgfpathlineto{\pgfqpoint{1.431153in}{0.942382in}}%
\pgfpathlineto{\pgfqpoint{1.431153in}{0.945331in}}%
\pgfpathlineto{\pgfqpoint{1.435695in}{0.945331in}}%
\pgfpathlineto{\pgfqpoint{1.435695in}{0.942382in}}%
\pgfpathmoveto{\pgfqpoint{1.431153in}{0.945331in}}%
\pgfpathlineto{\pgfqpoint{1.431153in}{0.945331in}}%
\pgfpathlineto{\pgfqpoint{1.431153in}{0.948280in}}%
\pgfpathlineto{\pgfqpoint{1.435695in}{0.948280in}}%
\pgfpathlineto{\pgfqpoint{1.435695in}{0.945331in}}%
\pgfpathmoveto{\pgfqpoint{1.435695in}{0.945331in}}%
\pgfpathlineto{\pgfqpoint{1.435695in}{0.945331in}}%
\pgfpathlineto{\pgfqpoint{1.435695in}{0.948280in}}%
\pgfpathlineto{\pgfqpoint{1.440236in}{0.948280in}}%
\pgfpathlineto{\pgfqpoint{1.440236in}{0.945331in}}%
\pgfpathmoveto{\pgfqpoint{1.435695in}{0.948280in}}%
\pgfpathlineto{\pgfqpoint{1.435695in}{0.948280in}}%
\pgfpathlineto{\pgfqpoint{1.435695in}{0.951229in}}%
\pgfpathlineto{\pgfqpoint{1.440236in}{0.951229in}}%
\pgfpathlineto{\pgfqpoint{1.440236in}{0.948280in}}%
\pgfpathmoveto{\pgfqpoint{1.440236in}{0.948280in}}%
\pgfpathlineto{\pgfqpoint{1.440236in}{0.948280in}}%
\pgfpathlineto{\pgfqpoint{1.440236in}{0.951229in}}%
\pgfpathlineto{\pgfqpoint{1.444777in}{0.951229in}}%
\pgfpathlineto{\pgfqpoint{1.444777in}{0.948280in}}%
\pgfpathmoveto{\pgfqpoint{1.440236in}{0.951229in}}%
\pgfpathlineto{\pgfqpoint{1.440236in}{0.951229in}}%
\pgfpathlineto{\pgfqpoint{1.440236in}{0.954178in}}%
\pgfpathlineto{\pgfqpoint{1.444777in}{0.954178in}}%
\pgfpathlineto{\pgfqpoint{1.444777in}{0.951229in}}%
\pgfpathmoveto{\pgfqpoint{1.444777in}{0.951229in}}%
\pgfpathlineto{\pgfqpoint{1.444777in}{0.951229in}}%
\pgfpathlineto{\pgfqpoint{1.444777in}{0.954178in}}%
\pgfpathlineto{\pgfqpoint{1.449318in}{0.954178in}}%
\pgfpathlineto{\pgfqpoint{1.449318in}{0.951229in}}%
\pgfpathmoveto{\pgfqpoint{1.444777in}{0.954178in}}%
\pgfpathlineto{\pgfqpoint{1.444777in}{0.954178in}}%
\pgfpathlineto{\pgfqpoint{1.444777in}{0.957127in}}%
\pgfpathlineto{\pgfqpoint{1.449318in}{0.957127in}}%
\pgfpathlineto{\pgfqpoint{1.449318in}{0.954178in}}%
\pgfpathmoveto{\pgfqpoint{1.449318in}{0.954178in}}%
\pgfpathlineto{\pgfqpoint{1.449318in}{0.954178in}}%
\pgfpathlineto{\pgfqpoint{1.449318in}{0.957127in}}%
\pgfpathlineto{\pgfqpoint{1.453860in}{0.957127in}}%
\pgfpathlineto{\pgfqpoint{1.453860in}{0.954178in}}%
\pgfpathmoveto{\pgfqpoint{1.449318in}{0.957127in}}%
\pgfpathlineto{\pgfqpoint{1.449318in}{0.957127in}}%
\pgfpathlineto{\pgfqpoint{1.449318in}{0.960076in}}%
\pgfpathlineto{\pgfqpoint{1.453860in}{0.960076in}}%
\pgfpathlineto{\pgfqpoint{1.453860in}{0.957127in}}%
\pgfpathmoveto{\pgfqpoint{1.453860in}{0.957127in}}%
\pgfpathlineto{\pgfqpoint{1.453860in}{0.957127in}}%
\pgfpathlineto{\pgfqpoint{1.453860in}{0.960076in}}%
\pgfpathlineto{\pgfqpoint{1.458401in}{0.960076in}}%
\pgfpathlineto{\pgfqpoint{1.458401in}{0.957127in}}%
\pgfpathmoveto{\pgfqpoint{1.453860in}{0.960076in}}%
\pgfpathlineto{\pgfqpoint{1.453860in}{0.960076in}}%
\pgfpathlineto{\pgfqpoint{1.453860in}{0.963025in}}%
\pgfpathlineto{\pgfqpoint{1.458401in}{0.963025in}}%
\pgfpathlineto{\pgfqpoint{1.458401in}{0.960076in}}%
\pgfpathmoveto{\pgfqpoint{1.458401in}{0.960076in}}%
\pgfpathlineto{\pgfqpoint{1.458401in}{0.960076in}}%
\pgfpathlineto{\pgfqpoint{1.458401in}{0.963025in}}%
\pgfpathlineto{\pgfqpoint{1.462942in}{0.963025in}}%
\pgfpathlineto{\pgfqpoint{1.462942in}{0.960076in}}%
\pgfpathmoveto{\pgfqpoint{1.458401in}{0.963025in}}%
\pgfpathlineto{\pgfqpoint{1.458401in}{0.963025in}}%
\pgfpathlineto{\pgfqpoint{1.458401in}{0.965975in}}%
\pgfpathlineto{\pgfqpoint{1.462942in}{0.965975in}}%
\pgfpathlineto{\pgfqpoint{1.462942in}{0.963025in}}%
\pgfpathmoveto{\pgfqpoint{1.462942in}{0.963025in}}%
\pgfpathlineto{\pgfqpoint{1.462942in}{0.963025in}}%
\pgfpathlineto{\pgfqpoint{1.462942in}{0.965975in}}%
\pgfpathlineto{\pgfqpoint{1.467483in}{0.965975in}}%
\pgfpathlineto{\pgfqpoint{1.467483in}{0.963025in}}%
\pgfpathmoveto{\pgfqpoint{1.462942in}{0.965975in}}%
\pgfpathlineto{\pgfqpoint{1.462942in}{0.965975in}}%
\pgfpathlineto{\pgfqpoint{1.462942in}{0.968924in}}%
\pgfpathlineto{\pgfqpoint{1.467483in}{0.968924in}}%
\pgfpathlineto{\pgfqpoint{1.467483in}{0.965975in}}%
\pgfpathmoveto{\pgfqpoint{1.467483in}{0.965975in}}%
\pgfpathlineto{\pgfqpoint{1.467483in}{0.965975in}}%
\pgfpathlineto{\pgfqpoint{1.467483in}{0.968924in}}%
\pgfpathlineto{\pgfqpoint{1.472025in}{0.968924in}}%
\pgfpathlineto{\pgfqpoint{1.472025in}{0.965975in}}%
\pgfpathmoveto{\pgfqpoint{1.467483in}{0.968924in}}%
\pgfpathlineto{\pgfqpoint{1.467483in}{0.968924in}}%
\pgfpathlineto{\pgfqpoint{1.467483in}{0.971873in}}%
\pgfpathlineto{\pgfqpoint{1.472025in}{0.971873in}}%
\pgfpathlineto{\pgfqpoint{1.472025in}{0.968924in}}%
\pgfpathmoveto{\pgfqpoint{1.472025in}{0.968924in}}%
\pgfpathlineto{\pgfqpoint{1.472025in}{0.968924in}}%
\pgfpathlineto{\pgfqpoint{1.472025in}{0.971873in}}%
\pgfpathlineto{\pgfqpoint{1.476566in}{0.971873in}}%
\pgfpathlineto{\pgfqpoint{1.476566in}{0.968924in}}%
\pgfpathmoveto{\pgfqpoint{1.472025in}{0.971873in}}%
\pgfpathlineto{\pgfqpoint{1.472025in}{0.971873in}}%
\pgfpathlineto{\pgfqpoint{1.472025in}{0.974822in}}%
\pgfpathlineto{\pgfqpoint{1.476566in}{0.974822in}}%
\pgfpathlineto{\pgfqpoint{1.476566in}{0.971873in}}%
\pgfpathmoveto{\pgfqpoint{1.476566in}{0.971873in}}%
\pgfpathlineto{\pgfqpoint{1.476566in}{0.971873in}}%
\pgfpathlineto{\pgfqpoint{1.476566in}{0.974822in}}%
\pgfpathlineto{\pgfqpoint{1.481107in}{0.974822in}}%
\pgfpathlineto{\pgfqpoint{1.481107in}{0.971873in}}%
\pgfpathmoveto{\pgfqpoint{1.476566in}{0.974822in}}%
\pgfpathlineto{\pgfqpoint{1.476566in}{0.974822in}}%
\pgfpathlineto{\pgfqpoint{1.476566in}{0.977771in}}%
\pgfpathlineto{\pgfqpoint{1.481107in}{0.977771in}}%
\pgfpathlineto{\pgfqpoint{1.481107in}{0.974822in}}%
\pgfpathmoveto{\pgfqpoint{1.481107in}{0.974822in}}%
\pgfpathlineto{\pgfqpoint{1.481107in}{0.974822in}}%
\pgfpathlineto{\pgfqpoint{1.481107in}{0.977771in}}%
\pgfpathlineto{\pgfqpoint{1.485648in}{0.977771in}}%
\pgfpathlineto{\pgfqpoint{1.485648in}{0.974822in}}%
\pgfpathmoveto{\pgfqpoint{1.481107in}{0.977771in}}%
\pgfpathlineto{\pgfqpoint{1.481107in}{0.977771in}}%
\pgfpathlineto{\pgfqpoint{1.481107in}{0.980721in}}%
\pgfpathlineto{\pgfqpoint{1.485648in}{0.980721in}}%
\pgfpathlineto{\pgfqpoint{1.485648in}{0.977771in}}%
\pgfpathmoveto{\pgfqpoint{1.485648in}{0.977771in}}%
\pgfpathlineto{\pgfqpoint{1.485648in}{0.977771in}}%
\pgfpathlineto{\pgfqpoint{1.485648in}{0.980721in}}%
\pgfpathlineto{\pgfqpoint{1.490188in}{0.980721in}}%
\pgfpathlineto{\pgfqpoint{1.490188in}{0.977771in}}%
\pgfpathmoveto{\pgfqpoint{1.485648in}{0.980721in}}%
\pgfpathlineto{\pgfqpoint{1.485648in}{0.980721in}}%
\pgfpathlineto{\pgfqpoint{1.485648in}{0.983670in}}%
\pgfpathlineto{\pgfqpoint{1.490188in}{0.983670in}}%
\pgfpathlineto{\pgfqpoint{1.490188in}{0.980721in}}%
\pgfpathmoveto{\pgfqpoint{1.490188in}{0.980721in}}%
\pgfpathlineto{\pgfqpoint{1.490188in}{0.980721in}}%
\pgfpathlineto{\pgfqpoint{1.490188in}{0.983670in}}%
\pgfpathlineto{\pgfqpoint{1.494729in}{0.983670in}}%
\pgfpathlineto{\pgfqpoint{1.494729in}{0.980721in}}%
\pgfpathmoveto{\pgfqpoint{1.490188in}{0.983670in}}%
\pgfpathlineto{\pgfqpoint{1.490188in}{0.983670in}}%
\pgfpathlineto{\pgfqpoint{1.490188in}{0.986619in}}%
\pgfpathlineto{\pgfqpoint{1.494729in}{0.986619in}}%
\pgfpathlineto{\pgfqpoint{1.494729in}{0.983670in}}%
\pgfpathmoveto{\pgfqpoint{1.494729in}{0.983670in}}%
\pgfpathlineto{\pgfqpoint{1.494729in}{0.983670in}}%
\pgfpathlineto{\pgfqpoint{1.494729in}{0.986619in}}%
\pgfpathlineto{\pgfqpoint{1.499270in}{0.986619in}}%
\pgfpathlineto{\pgfqpoint{1.499270in}{0.983670in}}%
\pgfpathmoveto{\pgfqpoint{1.494729in}{0.986619in}}%
\pgfpathlineto{\pgfqpoint{1.494729in}{0.986619in}}%
\pgfpathlineto{\pgfqpoint{1.494729in}{0.989568in}}%
\pgfpathlineto{\pgfqpoint{1.499270in}{0.989568in}}%
\pgfpathlineto{\pgfqpoint{1.499270in}{0.986619in}}%
\pgfpathmoveto{\pgfqpoint{1.499270in}{0.986619in}}%
\pgfpathlineto{\pgfqpoint{1.499270in}{0.986619in}}%
\pgfpathlineto{\pgfqpoint{1.499270in}{0.989568in}}%
\pgfpathlineto{\pgfqpoint{1.503811in}{0.989568in}}%
\pgfpathlineto{\pgfqpoint{1.503811in}{0.986619in}}%
\pgfpathmoveto{\pgfqpoint{1.499270in}{0.989568in}}%
\pgfpathlineto{\pgfqpoint{1.499270in}{0.989568in}}%
\pgfpathlineto{\pgfqpoint{1.499270in}{0.992518in}}%
\pgfpathlineto{\pgfqpoint{1.503811in}{0.992518in}}%
\pgfpathlineto{\pgfqpoint{1.503811in}{0.989568in}}%
\pgfpathmoveto{\pgfqpoint{1.503811in}{0.989568in}}%
\pgfpathlineto{\pgfqpoint{1.503811in}{0.989568in}}%
\pgfpathlineto{\pgfqpoint{1.503811in}{0.992518in}}%
\pgfpathlineto{\pgfqpoint{1.508352in}{0.992518in}}%
\pgfpathlineto{\pgfqpoint{1.508352in}{0.989568in}}%
\pgfpathmoveto{\pgfqpoint{1.503811in}{0.992518in}}%
\pgfpathlineto{\pgfqpoint{1.503811in}{0.992518in}}%
\pgfpathlineto{\pgfqpoint{1.503811in}{0.995467in}}%
\pgfpathlineto{\pgfqpoint{1.508352in}{0.995467in}}%
\pgfpathlineto{\pgfqpoint{1.508352in}{0.992518in}}%
\pgfpathmoveto{\pgfqpoint{1.508352in}{0.992518in}}%
\pgfpathlineto{\pgfqpoint{1.508352in}{0.992518in}}%
\pgfpathlineto{\pgfqpoint{1.508352in}{0.995467in}}%
\pgfpathlineto{\pgfqpoint{1.512892in}{0.995467in}}%
\pgfpathlineto{\pgfqpoint{1.512892in}{0.992518in}}%
\pgfpathmoveto{\pgfqpoint{1.508352in}{0.995467in}}%
\pgfpathlineto{\pgfqpoint{1.508352in}{0.995467in}}%
\pgfpathlineto{\pgfqpoint{1.508352in}{0.998416in}}%
\pgfpathlineto{\pgfqpoint{1.512892in}{0.998416in}}%
\pgfpathlineto{\pgfqpoint{1.512892in}{0.995467in}}%
\pgfpathmoveto{\pgfqpoint{1.512892in}{0.995467in}}%
\pgfpathlineto{\pgfqpoint{1.512892in}{0.995467in}}%
\pgfpathlineto{\pgfqpoint{1.512892in}{0.998416in}}%
\pgfpathlineto{\pgfqpoint{1.517433in}{0.998416in}}%
\pgfpathlineto{\pgfqpoint{1.517433in}{0.995467in}}%
\pgfpathmoveto{\pgfqpoint{1.512892in}{0.998416in}}%
\pgfpathlineto{\pgfqpoint{1.512892in}{0.998416in}}%
\pgfpathlineto{\pgfqpoint{1.512892in}{1.001365in}}%
\pgfpathlineto{\pgfqpoint{1.517433in}{1.001365in}}%
\pgfpathlineto{\pgfqpoint{1.517433in}{0.998416in}}%
\pgfpathmoveto{\pgfqpoint{1.517433in}{0.998416in}}%
\pgfpathlineto{\pgfqpoint{1.517433in}{0.998416in}}%
\pgfpathlineto{\pgfqpoint{1.517433in}{1.001365in}}%
\pgfpathlineto{\pgfqpoint{1.521974in}{1.001365in}}%
\pgfpathlineto{\pgfqpoint{1.521974in}{0.998416in}}%
\pgfpathmoveto{\pgfqpoint{1.517433in}{1.001365in}}%
\pgfpathlineto{\pgfqpoint{1.517433in}{1.001365in}}%
\pgfpathlineto{\pgfqpoint{1.517433in}{1.004315in}}%
\pgfpathlineto{\pgfqpoint{1.521974in}{1.004315in}}%
\pgfpathlineto{\pgfqpoint{1.521974in}{1.001365in}}%
\pgfpathmoveto{\pgfqpoint{1.521974in}{1.001365in}}%
\pgfpathlineto{\pgfqpoint{1.521974in}{1.001365in}}%
\pgfpathlineto{\pgfqpoint{1.521974in}{1.004315in}}%
\pgfpathlineto{\pgfqpoint{1.526515in}{1.004315in}}%
\pgfpathlineto{\pgfqpoint{1.526515in}{1.001365in}}%
\pgfpathmoveto{\pgfqpoint{1.521974in}{1.004315in}}%
\pgfpathlineto{\pgfqpoint{1.521974in}{1.004315in}}%
\pgfpathlineto{\pgfqpoint{1.521974in}{1.007264in}}%
\pgfpathlineto{\pgfqpoint{1.526515in}{1.007264in}}%
\pgfpathlineto{\pgfqpoint{1.526515in}{1.004315in}}%
\pgfpathmoveto{\pgfqpoint{1.526515in}{1.004315in}}%
\pgfpathlineto{\pgfqpoint{1.526515in}{1.004315in}}%
\pgfpathlineto{\pgfqpoint{1.526515in}{1.007264in}}%
\pgfpathlineto{\pgfqpoint{1.531056in}{1.007264in}}%
\pgfpathlineto{\pgfqpoint{1.531056in}{1.004315in}}%
\pgfpathmoveto{\pgfqpoint{1.526515in}{1.007264in}}%
\pgfpathlineto{\pgfqpoint{1.526515in}{1.007264in}}%
\pgfpathlineto{\pgfqpoint{1.526515in}{1.010213in}}%
\pgfpathlineto{\pgfqpoint{1.531056in}{1.010213in}}%
\pgfpathlineto{\pgfqpoint{1.531056in}{1.007264in}}%
\pgfpathmoveto{\pgfqpoint{1.531056in}{1.007264in}}%
\pgfpathlineto{\pgfqpoint{1.531056in}{1.007264in}}%
\pgfpathlineto{\pgfqpoint{1.531056in}{1.010213in}}%
\pgfpathlineto{\pgfqpoint{1.535597in}{1.010213in}}%
\pgfpathlineto{\pgfqpoint{1.535597in}{1.007264in}}%
\pgfpathmoveto{\pgfqpoint{1.531056in}{1.010213in}}%
\pgfpathlineto{\pgfqpoint{1.531056in}{1.010213in}}%
\pgfpathlineto{\pgfqpoint{1.531056in}{1.013163in}}%
\pgfpathlineto{\pgfqpoint{1.535597in}{1.013163in}}%
\pgfpathlineto{\pgfqpoint{1.535597in}{1.010213in}}%
\pgfpathmoveto{\pgfqpoint{1.535597in}{1.010213in}}%
\pgfpathlineto{\pgfqpoint{1.535597in}{1.010213in}}%
\pgfpathlineto{\pgfqpoint{1.535597in}{1.013163in}}%
\pgfpathlineto{\pgfqpoint{1.540137in}{1.013163in}}%
\pgfpathlineto{\pgfqpoint{1.540137in}{1.010213in}}%
\pgfpathmoveto{\pgfqpoint{1.535597in}{1.013163in}}%
\pgfpathlineto{\pgfqpoint{1.535597in}{1.013163in}}%
\pgfpathlineto{\pgfqpoint{1.535597in}{1.016112in}}%
\pgfpathlineto{\pgfqpoint{1.540137in}{1.016112in}}%
\pgfpathlineto{\pgfqpoint{1.540137in}{1.013163in}}%
\pgfpathmoveto{\pgfqpoint{1.540137in}{1.013163in}}%
\pgfpathlineto{\pgfqpoint{1.540137in}{1.013163in}}%
\pgfpathlineto{\pgfqpoint{1.540137in}{1.016112in}}%
\pgfpathlineto{\pgfqpoint{1.544678in}{1.016112in}}%
\pgfpathlineto{\pgfqpoint{1.544678in}{1.013163in}}%
\pgfpathmoveto{\pgfqpoint{1.540137in}{1.016112in}}%
\pgfpathlineto{\pgfqpoint{1.540137in}{1.016112in}}%
\pgfpathlineto{\pgfqpoint{1.540137in}{1.019061in}}%
\pgfpathlineto{\pgfqpoint{1.544678in}{1.019061in}}%
\pgfpathlineto{\pgfqpoint{1.544678in}{1.016112in}}%
\pgfpathmoveto{\pgfqpoint{1.544678in}{1.016112in}}%
\pgfpathlineto{\pgfqpoint{1.544678in}{1.016112in}}%
\pgfpathlineto{\pgfqpoint{1.544678in}{1.019061in}}%
\pgfpathlineto{\pgfqpoint{1.549219in}{1.019061in}}%
\pgfpathlineto{\pgfqpoint{1.549219in}{1.016112in}}%
\pgfpathmoveto{\pgfqpoint{1.544678in}{1.019061in}}%
\pgfpathlineto{\pgfqpoint{1.544678in}{1.019061in}}%
\pgfpathlineto{\pgfqpoint{1.544678in}{1.022010in}}%
\pgfpathlineto{\pgfqpoint{1.549219in}{1.022010in}}%
\pgfpathlineto{\pgfqpoint{1.549219in}{1.019061in}}%
\pgfpathmoveto{\pgfqpoint{1.549219in}{1.019061in}}%
\pgfpathlineto{\pgfqpoint{1.549219in}{1.019061in}}%
\pgfpathlineto{\pgfqpoint{1.549219in}{1.022010in}}%
\pgfpathlineto{\pgfqpoint{1.553760in}{1.022010in}}%
\pgfpathlineto{\pgfqpoint{1.553760in}{1.019061in}}%
\pgfpathmoveto{\pgfqpoint{1.549219in}{1.022010in}}%
\pgfpathlineto{\pgfqpoint{1.549219in}{1.022010in}}%
\pgfpathlineto{\pgfqpoint{1.549219in}{1.024960in}}%
\pgfpathlineto{\pgfqpoint{1.553760in}{1.024960in}}%
\pgfpathlineto{\pgfqpoint{1.553760in}{1.022010in}}%
\pgfpathmoveto{\pgfqpoint{1.553760in}{1.022010in}}%
\pgfpathlineto{\pgfqpoint{1.553760in}{1.022010in}}%
\pgfpathlineto{\pgfqpoint{1.553760in}{1.024960in}}%
\pgfpathlineto{\pgfqpoint{1.558301in}{1.024960in}}%
\pgfpathlineto{\pgfqpoint{1.558301in}{1.022010in}}%
\pgfpathmoveto{\pgfqpoint{1.553760in}{1.024960in}}%
\pgfpathlineto{\pgfqpoint{1.553760in}{1.024960in}}%
\pgfpathlineto{\pgfqpoint{1.553760in}{1.027909in}}%
\pgfpathlineto{\pgfqpoint{1.558301in}{1.027909in}}%
\pgfpathlineto{\pgfqpoint{1.558301in}{1.024960in}}%
\pgfpathmoveto{\pgfqpoint{1.558301in}{1.024960in}}%
\pgfpathlineto{\pgfqpoint{1.558301in}{1.024960in}}%
\pgfpathlineto{\pgfqpoint{1.558301in}{1.027909in}}%
\pgfpathlineto{\pgfqpoint{1.562841in}{1.027909in}}%
\pgfpathlineto{\pgfqpoint{1.562841in}{1.024960in}}%
\pgfpathmoveto{\pgfqpoint{1.558301in}{1.027909in}}%
\pgfpathlineto{\pgfqpoint{1.558301in}{1.027909in}}%
\pgfpathlineto{\pgfqpoint{1.558301in}{1.030858in}}%
\pgfpathlineto{\pgfqpoint{1.562841in}{1.030858in}}%
\pgfpathlineto{\pgfqpoint{1.562841in}{1.027909in}}%
\pgfpathmoveto{\pgfqpoint{1.562841in}{1.027909in}}%
\pgfpathlineto{\pgfqpoint{1.562841in}{1.027909in}}%
\pgfpathlineto{\pgfqpoint{1.562841in}{1.030858in}}%
\pgfpathlineto{\pgfqpoint{1.567382in}{1.030858in}}%
\pgfpathlineto{\pgfqpoint{1.567382in}{1.027909in}}%
\pgfpathmoveto{\pgfqpoint{1.562841in}{1.030858in}}%
\pgfpathlineto{\pgfqpoint{1.562841in}{1.030858in}}%
\pgfpathlineto{\pgfqpoint{1.562841in}{1.033807in}}%
\pgfpathlineto{\pgfqpoint{1.567382in}{1.033807in}}%
\pgfpathlineto{\pgfqpoint{1.567382in}{1.030858in}}%
\pgfpathmoveto{\pgfqpoint{1.567382in}{1.030858in}}%
\pgfpathlineto{\pgfqpoint{1.567382in}{1.030858in}}%
\pgfpathlineto{\pgfqpoint{1.567382in}{1.033807in}}%
\pgfpathlineto{\pgfqpoint{1.571923in}{1.033807in}}%
\pgfpathlineto{\pgfqpoint{1.571923in}{1.030858in}}%
\pgfpathmoveto{\pgfqpoint{1.567382in}{1.033807in}}%
\pgfpathlineto{\pgfqpoint{1.567382in}{1.033807in}}%
\pgfpathlineto{\pgfqpoint{1.567382in}{1.036757in}}%
\pgfpathlineto{\pgfqpoint{1.571923in}{1.036757in}}%
\pgfpathlineto{\pgfqpoint{1.571923in}{1.033807in}}%
\pgfpathmoveto{\pgfqpoint{1.571923in}{1.033807in}}%
\pgfpathlineto{\pgfqpoint{1.571923in}{1.033807in}}%
\pgfpathlineto{\pgfqpoint{1.571923in}{1.036757in}}%
\pgfpathlineto{\pgfqpoint{1.576464in}{1.036757in}}%
\pgfpathlineto{\pgfqpoint{1.576464in}{1.033807in}}%
\pgfpathmoveto{\pgfqpoint{1.571923in}{1.036757in}}%
\pgfpathlineto{\pgfqpoint{1.571923in}{1.036757in}}%
\pgfpathlineto{\pgfqpoint{1.571923in}{1.039706in}}%
\pgfpathlineto{\pgfqpoint{1.576464in}{1.039706in}}%
\pgfpathlineto{\pgfqpoint{1.576464in}{1.036757in}}%
\pgfpathmoveto{\pgfqpoint{1.576464in}{1.036757in}}%
\pgfpathlineto{\pgfqpoint{1.576464in}{1.036757in}}%
\pgfpathlineto{\pgfqpoint{1.576464in}{1.039706in}}%
\pgfpathlineto{\pgfqpoint{1.581005in}{1.039706in}}%
\pgfpathlineto{\pgfqpoint{1.581005in}{1.036757in}}%
\pgfpathmoveto{\pgfqpoint{1.576464in}{1.039706in}}%
\pgfpathlineto{\pgfqpoint{1.576464in}{1.039706in}}%
\pgfpathlineto{\pgfqpoint{1.576464in}{1.042655in}}%
\pgfpathlineto{\pgfqpoint{1.581005in}{1.042655in}}%
\pgfpathlineto{\pgfqpoint{1.581005in}{1.039706in}}%
\pgfpathmoveto{\pgfqpoint{1.581005in}{1.039706in}}%
\pgfpathlineto{\pgfqpoint{1.581005in}{1.039706in}}%
\pgfpathlineto{\pgfqpoint{1.581005in}{1.042655in}}%
\pgfpathlineto{\pgfqpoint{1.585545in}{1.042655in}}%
\pgfpathlineto{\pgfqpoint{1.585545in}{1.039706in}}%
\pgfpathmoveto{\pgfqpoint{1.581005in}{1.042655in}}%
\pgfpathlineto{\pgfqpoint{1.581005in}{1.042655in}}%
\pgfpathlineto{\pgfqpoint{1.581005in}{1.045604in}}%
\pgfpathlineto{\pgfqpoint{1.585545in}{1.045604in}}%
\pgfpathlineto{\pgfqpoint{1.585545in}{1.042655in}}%
\pgfpathmoveto{\pgfqpoint{1.585545in}{1.042655in}}%
\pgfpathlineto{\pgfqpoint{1.585545in}{1.042655in}}%
\pgfpathlineto{\pgfqpoint{1.585545in}{1.045604in}}%
\pgfpathlineto{\pgfqpoint{1.590086in}{1.045604in}}%
\pgfpathlineto{\pgfqpoint{1.590086in}{1.042655in}}%
\pgfpathmoveto{\pgfqpoint{1.590086in}{1.042655in}}%
\pgfpathlineto{\pgfqpoint{1.590086in}{1.042655in}}%
\pgfpathlineto{\pgfqpoint{1.590086in}{1.045604in}}%
\pgfpathlineto{\pgfqpoint{1.594627in}{1.045604in}}%
\pgfpathlineto{\pgfqpoint{1.594627in}{1.042655in}}%
\pgfpathmoveto{\pgfqpoint{1.590086in}{1.045604in}}%
\pgfpathlineto{\pgfqpoint{1.590086in}{1.045604in}}%
\pgfpathlineto{\pgfqpoint{1.590086in}{1.048554in}}%
\pgfpathlineto{\pgfqpoint{1.594627in}{1.048554in}}%
\pgfpathlineto{\pgfqpoint{1.594627in}{1.045604in}}%
\pgfpathmoveto{\pgfqpoint{1.594627in}{1.045604in}}%
\pgfpathlineto{\pgfqpoint{1.594627in}{1.045604in}}%
\pgfpathlineto{\pgfqpoint{1.594627in}{1.048554in}}%
\pgfpathlineto{\pgfqpoint{1.599168in}{1.048554in}}%
\pgfpathlineto{\pgfqpoint{1.599168in}{1.045604in}}%
\pgfpathmoveto{\pgfqpoint{1.594627in}{1.048554in}}%
\pgfpathlineto{\pgfqpoint{1.594627in}{1.048554in}}%
\pgfpathlineto{\pgfqpoint{1.594627in}{1.051503in}}%
\pgfpathlineto{\pgfqpoint{1.599168in}{1.051503in}}%
\pgfpathlineto{\pgfqpoint{1.599168in}{1.048554in}}%
\pgfpathmoveto{\pgfqpoint{1.599168in}{1.048554in}}%
\pgfpathlineto{\pgfqpoint{1.599168in}{1.048554in}}%
\pgfpathlineto{\pgfqpoint{1.599168in}{1.051503in}}%
\pgfpathlineto{\pgfqpoint{1.603709in}{1.051503in}}%
\pgfpathlineto{\pgfqpoint{1.603709in}{1.048554in}}%
\pgfpathmoveto{\pgfqpoint{1.599168in}{1.051503in}}%
\pgfpathlineto{\pgfqpoint{1.599168in}{1.051503in}}%
\pgfpathlineto{\pgfqpoint{1.599168in}{1.054452in}}%
\pgfpathlineto{\pgfqpoint{1.603709in}{1.054452in}}%
\pgfpathlineto{\pgfqpoint{1.603709in}{1.051503in}}%
\pgfpathmoveto{\pgfqpoint{1.603709in}{1.051503in}}%
\pgfpathlineto{\pgfqpoint{1.603709in}{1.051503in}}%
\pgfpathlineto{\pgfqpoint{1.603709in}{1.054452in}}%
\pgfpathlineto{\pgfqpoint{1.608249in}{1.054452in}}%
\pgfpathlineto{\pgfqpoint{1.608249in}{1.051503in}}%
\pgfpathmoveto{\pgfqpoint{1.603709in}{1.054452in}}%
\pgfpathlineto{\pgfqpoint{1.603709in}{1.054452in}}%
\pgfpathlineto{\pgfqpoint{1.603709in}{1.057402in}}%
\pgfpathlineto{\pgfqpoint{1.608249in}{1.057402in}}%
\pgfpathlineto{\pgfqpoint{1.608249in}{1.054452in}}%
\pgfpathmoveto{\pgfqpoint{1.608249in}{1.054452in}}%
\pgfpathlineto{\pgfqpoint{1.608249in}{1.054452in}}%
\pgfpathlineto{\pgfqpoint{1.608249in}{1.057402in}}%
\pgfpathlineto{\pgfqpoint{1.612790in}{1.057402in}}%
\pgfpathlineto{\pgfqpoint{1.612790in}{1.054452in}}%
\pgfpathmoveto{\pgfqpoint{1.608249in}{1.057402in}}%
\pgfpathlineto{\pgfqpoint{1.608249in}{1.057402in}}%
\pgfpathlineto{\pgfqpoint{1.608249in}{1.060351in}}%
\pgfpathlineto{\pgfqpoint{1.612790in}{1.060351in}}%
\pgfpathlineto{\pgfqpoint{1.612790in}{1.057402in}}%
\pgfpathmoveto{\pgfqpoint{1.612790in}{1.057402in}}%
\pgfpathlineto{\pgfqpoint{1.612790in}{1.057402in}}%
\pgfpathlineto{\pgfqpoint{1.612790in}{1.060351in}}%
\pgfpathlineto{\pgfqpoint{1.617331in}{1.060351in}}%
\pgfpathlineto{\pgfqpoint{1.617331in}{1.057402in}}%
\pgfpathmoveto{\pgfqpoint{1.612790in}{1.060351in}}%
\pgfpathlineto{\pgfqpoint{1.612790in}{1.060351in}}%
\pgfpathlineto{\pgfqpoint{1.612790in}{1.063300in}}%
\pgfpathlineto{\pgfqpoint{1.617331in}{1.063300in}}%
\pgfpathlineto{\pgfqpoint{1.617331in}{1.060351in}}%
\pgfpathmoveto{\pgfqpoint{1.617331in}{1.060351in}}%
\pgfpathlineto{\pgfqpoint{1.617331in}{1.060351in}}%
\pgfpathlineto{\pgfqpoint{1.617331in}{1.063300in}}%
\pgfpathlineto{\pgfqpoint{1.621872in}{1.063300in}}%
\pgfpathlineto{\pgfqpoint{1.621872in}{1.060351in}}%
\pgfpathmoveto{\pgfqpoint{1.617331in}{1.063300in}}%
\pgfpathlineto{\pgfqpoint{1.617331in}{1.063300in}}%
\pgfpathlineto{\pgfqpoint{1.617331in}{1.066249in}}%
\pgfpathlineto{\pgfqpoint{1.621872in}{1.066249in}}%
\pgfpathlineto{\pgfqpoint{1.621872in}{1.063300in}}%
\pgfpathmoveto{\pgfqpoint{1.621872in}{1.063300in}}%
\pgfpathlineto{\pgfqpoint{1.621872in}{1.063300in}}%
\pgfpathlineto{\pgfqpoint{1.621872in}{1.066249in}}%
\pgfpathlineto{\pgfqpoint{1.626413in}{1.066249in}}%
\pgfpathlineto{\pgfqpoint{1.626413in}{1.063300in}}%
\pgfpathmoveto{\pgfqpoint{1.621872in}{1.066249in}}%
\pgfpathlineto{\pgfqpoint{1.621872in}{1.066249in}}%
\pgfpathlineto{\pgfqpoint{1.621872in}{1.069199in}}%
\pgfpathlineto{\pgfqpoint{1.626413in}{1.069199in}}%
\pgfpathlineto{\pgfqpoint{1.626413in}{1.066249in}}%
\pgfpathmoveto{\pgfqpoint{1.626413in}{1.066249in}}%
\pgfpathlineto{\pgfqpoint{1.626413in}{1.066249in}}%
\pgfpathlineto{\pgfqpoint{1.626413in}{1.069199in}}%
\pgfpathlineto{\pgfqpoint{1.630954in}{1.069199in}}%
\pgfpathlineto{\pgfqpoint{1.630954in}{1.066249in}}%
\pgfpathmoveto{\pgfqpoint{1.626413in}{1.069199in}}%
\pgfpathlineto{\pgfqpoint{1.626413in}{1.069199in}}%
\pgfpathlineto{\pgfqpoint{1.626413in}{1.072148in}}%
\pgfpathlineto{\pgfqpoint{1.630954in}{1.072148in}}%
\pgfpathlineto{\pgfqpoint{1.630954in}{1.069199in}}%
\pgfpathmoveto{\pgfqpoint{1.630954in}{1.069199in}}%
\pgfpathlineto{\pgfqpoint{1.630954in}{1.069199in}}%
\pgfpathlineto{\pgfqpoint{1.630954in}{1.072148in}}%
\pgfpathlineto{\pgfqpoint{1.635495in}{1.072148in}}%
\pgfpathlineto{\pgfqpoint{1.635495in}{1.069199in}}%
\pgfpathmoveto{\pgfqpoint{1.630954in}{1.072148in}}%
\pgfpathlineto{\pgfqpoint{1.630954in}{1.072148in}}%
\pgfpathlineto{\pgfqpoint{1.630954in}{1.075097in}}%
\pgfpathlineto{\pgfqpoint{1.635495in}{1.075097in}}%
\pgfpathlineto{\pgfqpoint{1.635495in}{1.072148in}}%
\pgfpathmoveto{\pgfqpoint{1.635495in}{1.072148in}}%
\pgfpathlineto{\pgfqpoint{1.635495in}{1.072148in}}%
\pgfpathlineto{\pgfqpoint{1.635495in}{1.075097in}}%
\pgfpathlineto{\pgfqpoint{1.640036in}{1.075097in}}%
\pgfpathlineto{\pgfqpoint{1.640036in}{1.072148in}}%
\pgfpathmoveto{\pgfqpoint{1.635495in}{1.075097in}}%
\pgfpathlineto{\pgfqpoint{1.635495in}{1.075097in}}%
\pgfpathlineto{\pgfqpoint{1.635495in}{1.078046in}}%
\pgfpathlineto{\pgfqpoint{1.640036in}{1.078046in}}%
\pgfpathlineto{\pgfqpoint{1.640036in}{1.075097in}}%
\pgfpathmoveto{\pgfqpoint{1.640036in}{1.075097in}}%
\pgfpathlineto{\pgfqpoint{1.640036in}{1.075097in}}%
\pgfpathlineto{\pgfqpoint{1.640036in}{1.078046in}}%
\pgfpathlineto{\pgfqpoint{1.644577in}{1.078046in}}%
\pgfpathlineto{\pgfqpoint{1.644577in}{1.075097in}}%
\pgfpathmoveto{\pgfqpoint{1.640036in}{1.078046in}}%
\pgfpathlineto{\pgfqpoint{1.640036in}{1.078046in}}%
\pgfpathlineto{\pgfqpoint{1.640036in}{1.080996in}}%
\pgfpathlineto{\pgfqpoint{1.644577in}{1.080996in}}%
\pgfpathlineto{\pgfqpoint{1.644577in}{1.078046in}}%
\pgfpathmoveto{\pgfqpoint{1.644577in}{1.078046in}}%
\pgfpathlineto{\pgfqpoint{1.644577in}{1.078046in}}%
\pgfpathlineto{\pgfqpoint{1.644577in}{1.080996in}}%
\pgfpathlineto{\pgfqpoint{1.649118in}{1.080996in}}%
\pgfpathlineto{\pgfqpoint{1.649118in}{1.078046in}}%
\pgfpathmoveto{\pgfqpoint{1.644577in}{1.080996in}}%
\pgfpathlineto{\pgfqpoint{1.644577in}{1.080996in}}%
\pgfpathlineto{\pgfqpoint{1.644577in}{1.083945in}}%
\pgfpathlineto{\pgfqpoint{1.649118in}{1.083945in}}%
\pgfpathlineto{\pgfqpoint{1.649118in}{1.080996in}}%
\pgfpathmoveto{\pgfqpoint{1.649118in}{1.080996in}}%
\pgfpathlineto{\pgfqpoint{1.649118in}{1.080996in}}%
\pgfpathlineto{\pgfqpoint{1.649118in}{1.083945in}}%
\pgfpathlineto{\pgfqpoint{1.653659in}{1.083945in}}%
\pgfpathlineto{\pgfqpoint{1.653659in}{1.080996in}}%
\pgfpathmoveto{\pgfqpoint{1.649118in}{1.083945in}}%
\pgfpathlineto{\pgfqpoint{1.649118in}{1.083945in}}%
\pgfpathlineto{\pgfqpoint{1.649118in}{1.086894in}}%
\pgfpathlineto{\pgfqpoint{1.653659in}{1.086894in}}%
\pgfpathlineto{\pgfqpoint{1.653659in}{1.083945in}}%
\pgfpathmoveto{\pgfqpoint{1.653659in}{1.083945in}}%
\pgfpathlineto{\pgfqpoint{1.653659in}{1.083945in}}%
\pgfpathlineto{\pgfqpoint{1.653659in}{1.086894in}}%
\pgfpathlineto{\pgfqpoint{1.658200in}{1.086894in}}%
\pgfpathlineto{\pgfqpoint{1.658200in}{1.083945in}}%
\pgfpathmoveto{\pgfqpoint{1.653659in}{1.086894in}}%
\pgfpathlineto{\pgfqpoint{1.653659in}{1.086894in}}%
\pgfpathlineto{\pgfqpoint{1.653659in}{1.089843in}}%
\pgfpathlineto{\pgfqpoint{1.658200in}{1.089843in}}%
\pgfpathlineto{\pgfqpoint{1.658200in}{1.086894in}}%
\pgfpathmoveto{\pgfqpoint{1.658200in}{1.086894in}}%
\pgfpathlineto{\pgfqpoint{1.658200in}{1.086894in}}%
\pgfpathlineto{\pgfqpoint{1.658200in}{1.089843in}}%
\pgfpathlineto{\pgfqpoint{1.662741in}{1.089843in}}%
\pgfpathlineto{\pgfqpoint{1.662741in}{1.086894in}}%
\pgfpathmoveto{\pgfqpoint{1.658200in}{1.089843in}}%
\pgfpathlineto{\pgfqpoint{1.658200in}{1.089843in}}%
\pgfpathlineto{\pgfqpoint{1.658200in}{1.092793in}}%
\pgfpathlineto{\pgfqpoint{1.662741in}{1.092793in}}%
\pgfpathlineto{\pgfqpoint{1.662741in}{1.089843in}}%
\pgfpathmoveto{\pgfqpoint{1.662741in}{1.089843in}}%
\pgfpathlineto{\pgfqpoint{1.662741in}{1.089843in}}%
\pgfpathlineto{\pgfqpoint{1.662741in}{1.092793in}}%
\pgfpathlineto{\pgfqpoint{1.667282in}{1.092793in}}%
\pgfpathlineto{\pgfqpoint{1.667282in}{1.089843in}}%
\pgfpathmoveto{\pgfqpoint{1.662741in}{1.092793in}}%
\pgfpathlineto{\pgfqpoint{1.662741in}{1.092793in}}%
\pgfpathlineto{\pgfqpoint{1.662741in}{1.095742in}}%
\pgfpathlineto{\pgfqpoint{1.667282in}{1.095742in}}%
\pgfpathlineto{\pgfqpoint{1.667282in}{1.092793in}}%
\pgfpathmoveto{\pgfqpoint{1.667282in}{1.092793in}}%
\pgfpathlineto{\pgfqpoint{1.667282in}{1.092793in}}%
\pgfpathlineto{\pgfqpoint{1.667282in}{1.095742in}}%
\pgfpathlineto{\pgfqpoint{1.671823in}{1.095742in}}%
\pgfpathlineto{\pgfqpoint{1.671823in}{1.092793in}}%
\pgfpathmoveto{\pgfqpoint{1.667282in}{1.095742in}}%
\pgfpathlineto{\pgfqpoint{1.667282in}{1.095742in}}%
\pgfpathlineto{\pgfqpoint{1.667282in}{1.098691in}}%
\pgfpathlineto{\pgfqpoint{1.671823in}{1.098691in}}%
\pgfpathlineto{\pgfqpoint{1.671823in}{1.095742in}}%
\pgfpathmoveto{\pgfqpoint{1.671823in}{1.095742in}}%
\pgfpathlineto{\pgfqpoint{1.671823in}{1.095742in}}%
\pgfpathlineto{\pgfqpoint{1.671823in}{1.098691in}}%
\pgfpathlineto{\pgfqpoint{1.676364in}{1.098691in}}%
\pgfpathlineto{\pgfqpoint{1.676364in}{1.095742in}}%
\pgfpathmoveto{\pgfqpoint{1.671823in}{1.098691in}}%
\pgfpathlineto{\pgfqpoint{1.671823in}{1.098691in}}%
\pgfpathlineto{\pgfqpoint{1.671823in}{1.101640in}}%
\pgfpathlineto{\pgfqpoint{1.676364in}{1.101640in}}%
\pgfpathlineto{\pgfqpoint{1.676364in}{1.098691in}}%
\pgfpathmoveto{\pgfqpoint{1.676364in}{1.098691in}}%
\pgfpathlineto{\pgfqpoint{1.676364in}{1.098691in}}%
\pgfpathlineto{\pgfqpoint{1.676364in}{1.101640in}}%
\pgfpathlineto{\pgfqpoint{1.680906in}{1.101640in}}%
\pgfpathlineto{\pgfqpoint{1.680906in}{1.098691in}}%
\pgfpathmoveto{\pgfqpoint{1.676364in}{1.101640in}}%
\pgfpathlineto{\pgfqpoint{1.676364in}{1.101640in}}%
\pgfpathlineto{\pgfqpoint{1.676364in}{1.104590in}}%
\pgfpathlineto{\pgfqpoint{1.680906in}{1.104590in}}%
\pgfpathlineto{\pgfqpoint{1.680906in}{1.101640in}}%
\pgfpathmoveto{\pgfqpoint{1.680906in}{1.101640in}}%
\pgfpathlineto{\pgfqpoint{1.680906in}{1.101640in}}%
\pgfpathlineto{\pgfqpoint{1.680906in}{1.104590in}}%
\pgfpathlineto{\pgfqpoint{1.685447in}{1.104590in}}%
\pgfpathlineto{\pgfqpoint{1.685447in}{1.101640in}}%
\pgfpathmoveto{\pgfqpoint{1.680906in}{1.104590in}}%
\pgfpathlineto{\pgfqpoint{1.680906in}{1.104590in}}%
\pgfpathlineto{\pgfqpoint{1.680906in}{1.107539in}}%
\pgfpathlineto{\pgfqpoint{1.685447in}{1.107539in}}%
\pgfpathlineto{\pgfqpoint{1.685447in}{1.104590in}}%
\pgfpathmoveto{\pgfqpoint{1.685447in}{1.104590in}}%
\pgfpathlineto{\pgfqpoint{1.685447in}{1.104590in}}%
\pgfpathlineto{\pgfqpoint{1.685447in}{1.107539in}}%
\pgfpathlineto{\pgfqpoint{1.689988in}{1.107539in}}%
\pgfpathlineto{\pgfqpoint{1.689988in}{1.104590in}}%
\pgfpathmoveto{\pgfqpoint{1.685447in}{1.107539in}}%
\pgfpathlineto{\pgfqpoint{1.685447in}{1.107539in}}%
\pgfpathlineto{\pgfqpoint{1.685447in}{1.110488in}}%
\pgfpathlineto{\pgfqpoint{1.689988in}{1.110488in}}%
\pgfpathlineto{\pgfqpoint{1.689988in}{1.107539in}}%
\pgfpathmoveto{\pgfqpoint{1.689988in}{1.107539in}}%
\pgfpathlineto{\pgfqpoint{1.689988in}{1.107539in}}%
\pgfpathlineto{\pgfqpoint{1.689988in}{1.110488in}}%
\pgfpathlineto{\pgfqpoint{1.694529in}{1.110488in}}%
\pgfpathlineto{\pgfqpoint{1.694529in}{1.107539in}}%
\pgfpathmoveto{\pgfqpoint{1.689988in}{1.110488in}}%
\pgfpathlineto{\pgfqpoint{1.689988in}{1.110488in}}%
\pgfpathlineto{\pgfqpoint{1.689988in}{1.113437in}}%
\pgfpathlineto{\pgfqpoint{1.694529in}{1.113437in}}%
\pgfpathlineto{\pgfqpoint{1.694529in}{1.110488in}}%
\pgfpathmoveto{\pgfqpoint{1.694529in}{1.110488in}}%
\pgfpathlineto{\pgfqpoint{1.694529in}{1.110488in}}%
\pgfpathlineto{\pgfqpoint{1.694529in}{1.113437in}}%
\pgfpathlineto{\pgfqpoint{1.699070in}{1.113437in}}%
\pgfpathlineto{\pgfqpoint{1.699070in}{1.110488in}}%
\pgfpathmoveto{\pgfqpoint{1.694529in}{1.113437in}}%
\pgfpathlineto{\pgfqpoint{1.694529in}{1.113437in}}%
\pgfpathlineto{\pgfqpoint{1.694529in}{1.116387in}}%
\pgfpathlineto{\pgfqpoint{1.699070in}{1.116387in}}%
\pgfpathlineto{\pgfqpoint{1.699070in}{1.113437in}}%
\pgfpathmoveto{\pgfqpoint{1.699070in}{1.113437in}}%
\pgfpathlineto{\pgfqpoint{1.699070in}{1.113437in}}%
\pgfpathlineto{\pgfqpoint{1.699070in}{1.116387in}}%
\pgfpathlineto{\pgfqpoint{1.703611in}{1.116387in}}%
\pgfpathlineto{\pgfqpoint{1.703611in}{1.113437in}}%
\pgfpathmoveto{\pgfqpoint{1.699070in}{1.116387in}}%
\pgfpathlineto{\pgfqpoint{1.699070in}{1.116387in}}%
\pgfpathlineto{\pgfqpoint{1.699070in}{1.119336in}}%
\pgfpathlineto{\pgfqpoint{1.703611in}{1.119336in}}%
\pgfpathlineto{\pgfqpoint{1.703611in}{1.116387in}}%
\pgfpathmoveto{\pgfqpoint{1.703611in}{1.116387in}}%
\pgfpathlineto{\pgfqpoint{1.703611in}{1.116387in}}%
\pgfpathlineto{\pgfqpoint{1.703611in}{1.119336in}}%
\pgfpathlineto{\pgfqpoint{1.708152in}{1.119336in}}%
\pgfpathlineto{\pgfqpoint{1.708152in}{1.116387in}}%
\pgfpathmoveto{\pgfqpoint{1.703611in}{1.119336in}}%
\pgfpathlineto{\pgfqpoint{1.703611in}{1.119336in}}%
\pgfpathlineto{\pgfqpoint{1.703611in}{1.122285in}}%
\pgfpathlineto{\pgfqpoint{1.708152in}{1.122285in}}%
\pgfpathlineto{\pgfqpoint{1.708152in}{1.119336in}}%
\pgfpathmoveto{\pgfqpoint{1.708152in}{1.119336in}}%
\pgfpathlineto{\pgfqpoint{1.708152in}{1.119336in}}%
\pgfpathlineto{\pgfqpoint{1.708152in}{1.122285in}}%
\pgfpathlineto{\pgfqpoint{1.712693in}{1.122285in}}%
\pgfpathlineto{\pgfqpoint{1.712693in}{1.119336in}}%
\pgfpathmoveto{\pgfqpoint{1.708152in}{1.122285in}}%
\pgfpathlineto{\pgfqpoint{1.708152in}{1.122285in}}%
\pgfpathlineto{\pgfqpoint{1.708152in}{1.125234in}}%
\pgfpathlineto{\pgfqpoint{1.712693in}{1.125234in}}%
\pgfpathlineto{\pgfqpoint{1.712693in}{1.122285in}}%
\pgfpathmoveto{\pgfqpoint{1.712693in}{1.122285in}}%
\pgfpathlineto{\pgfqpoint{1.712693in}{1.122285in}}%
\pgfpathlineto{\pgfqpoint{1.712693in}{1.125234in}}%
\pgfpathlineto{\pgfqpoint{1.717234in}{1.125234in}}%
\pgfpathlineto{\pgfqpoint{1.717234in}{1.122285in}}%
\pgfpathmoveto{\pgfqpoint{1.712693in}{1.125234in}}%
\pgfpathlineto{\pgfqpoint{1.712693in}{1.125234in}}%
\pgfpathlineto{\pgfqpoint{1.712693in}{1.128184in}}%
\pgfpathlineto{\pgfqpoint{1.717234in}{1.128184in}}%
\pgfpathlineto{\pgfqpoint{1.717234in}{1.125234in}}%
\pgfpathmoveto{\pgfqpoint{1.717234in}{1.125234in}}%
\pgfpathlineto{\pgfqpoint{1.717234in}{1.125234in}}%
\pgfpathlineto{\pgfqpoint{1.717234in}{1.128184in}}%
\pgfpathlineto{\pgfqpoint{1.721775in}{1.128184in}}%
\pgfpathlineto{\pgfqpoint{1.721775in}{1.125234in}}%
\pgfpathmoveto{\pgfqpoint{1.717234in}{1.128184in}}%
\pgfpathlineto{\pgfqpoint{1.717234in}{1.128184in}}%
\pgfpathlineto{\pgfqpoint{1.717234in}{1.131133in}}%
\pgfpathlineto{\pgfqpoint{1.721775in}{1.131133in}}%
\pgfpathlineto{\pgfqpoint{1.721775in}{1.128184in}}%
\pgfpathmoveto{\pgfqpoint{1.721775in}{1.128184in}}%
\pgfpathlineto{\pgfqpoint{1.721775in}{1.128184in}}%
\pgfpathlineto{\pgfqpoint{1.721775in}{1.131133in}}%
\pgfpathlineto{\pgfqpoint{1.726316in}{1.131133in}}%
\pgfpathlineto{\pgfqpoint{1.726316in}{1.128184in}}%
\pgfpathmoveto{\pgfqpoint{1.721775in}{1.131133in}}%
\pgfpathlineto{\pgfqpoint{1.721775in}{1.131133in}}%
\pgfpathlineto{\pgfqpoint{1.721775in}{1.134082in}}%
\pgfpathlineto{\pgfqpoint{1.726316in}{1.134082in}}%
\pgfpathlineto{\pgfqpoint{1.726316in}{1.131133in}}%
\pgfpathmoveto{\pgfqpoint{1.726316in}{1.131133in}}%
\pgfpathlineto{\pgfqpoint{1.726316in}{1.131133in}}%
\pgfpathlineto{\pgfqpoint{1.726316in}{1.134082in}}%
\pgfpathlineto{\pgfqpoint{1.730857in}{1.134082in}}%
\pgfpathlineto{\pgfqpoint{1.730857in}{1.131133in}}%
\pgfpathmoveto{\pgfqpoint{1.726316in}{1.134082in}}%
\pgfpathlineto{\pgfqpoint{1.726316in}{1.134082in}}%
\pgfpathlineto{\pgfqpoint{1.726316in}{1.137032in}}%
\pgfpathlineto{\pgfqpoint{1.730857in}{1.137032in}}%
\pgfpathlineto{\pgfqpoint{1.730857in}{1.134082in}}%
\pgfpathmoveto{\pgfqpoint{1.730857in}{1.134082in}}%
\pgfpathlineto{\pgfqpoint{1.730857in}{1.134082in}}%
\pgfpathlineto{\pgfqpoint{1.730857in}{1.137032in}}%
\pgfpathlineto{\pgfqpoint{1.735398in}{1.137032in}}%
\pgfpathlineto{\pgfqpoint{1.735398in}{1.134082in}}%
\pgfpathmoveto{\pgfqpoint{1.730857in}{1.137032in}}%
\pgfpathlineto{\pgfqpoint{1.730857in}{1.137032in}}%
\pgfpathlineto{\pgfqpoint{1.730857in}{1.139981in}}%
\pgfpathlineto{\pgfqpoint{1.735398in}{1.139981in}}%
\pgfpathlineto{\pgfqpoint{1.735398in}{1.137032in}}%
\pgfpathmoveto{\pgfqpoint{1.735398in}{1.137032in}}%
\pgfpathlineto{\pgfqpoint{1.735398in}{1.137032in}}%
\pgfpathlineto{\pgfqpoint{1.735398in}{1.139981in}}%
\pgfpathlineto{\pgfqpoint{1.739939in}{1.139981in}}%
\pgfpathlineto{\pgfqpoint{1.739939in}{1.137032in}}%
\pgfpathmoveto{\pgfqpoint{1.735398in}{1.139981in}}%
\pgfpathlineto{\pgfqpoint{1.735398in}{1.139981in}}%
\pgfpathlineto{\pgfqpoint{1.735398in}{1.142930in}}%
\pgfpathlineto{\pgfqpoint{1.739939in}{1.142930in}}%
\pgfpathlineto{\pgfqpoint{1.739939in}{1.139981in}}%
\pgfpathmoveto{\pgfqpoint{1.739939in}{1.139981in}}%
\pgfpathlineto{\pgfqpoint{1.739939in}{1.139981in}}%
\pgfpathlineto{\pgfqpoint{1.739939in}{1.142930in}}%
\pgfpathlineto{\pgfqpoint{1.744480in}{1.142930in}}%
\pgfpathlineto{\pgfqpoint{1.744480in}{1.139981in}}%
\pgfpathmoveto{\pgfqpoint{1.739939in}{1.142930in}}%
\pgfpathlineto{\pgfqpoint{1.739939in}{1.142930in}}%
\pgfpathlineto{\pgfqpoint{1.739939in}{1.145879in}}%
\pgfpathlineto{\pgfqpoint{1.744480in}{1.145879in}}%
\pgfpathlineto{\pgfqpoint{1.744480in}{1.142930in}}%
\pgfpathmoveto{\pgfqpoint{1.744480in}{1.142930in}}%
\pgfpathlineto{\pgfqpoint{1.744480in}{1.142930in}}%
\pgfpathlineto{\pgfqpoint{1.744480in}{1.145879in}}%
\pgfpathlineto{\pgfqpoint{1.749021in}{1.145879in}}%
\pgfpathlineto{\pgfqpoint{1.749021in}{1.142930in}}%
\pgfpathmoveto{\pgfqpoint{1.744480in}{1.145879in}}%
\pgfpathlineto{\pgfqpoint{1.744480in}{1.145879in}}%
\pgfpathlineto{\pgfqpoint{1.744480in}{1.148829in}}%
\pgfpathlineto{\pgfqpoint{1.749021in}{1.148829in}}%
\pgfpathlineto{\pgfqpoint{1.749021in}{1.145879in}}%
\pgfpathmoveto{\pgfqpoint{1.749021in}{1.145879in}}%
\pgfpathlineto{\pgfqpoint{1.749021in}{1.145879in}}%
\pgfpathlineto{\pgfqpoint{1.749021in}{1.148829in}}%
\pgfpathlineto{\pgfqpoint{1.753562in}{1.148829in}}%
\pgfpathlineto{\pgfqpoint{1.753562in}{1.145879in}}%
\pgfpathmoveto{\pgfqpoint{1.749021in}{1.148829in}}%
\pgfpathlineto{\pgfqpoint{1.749021in}{1.148829in}}%
\pgfpathlineto{\pgfqpoint{1.749021in}{1.151778in}}%
\pgfpathlineto{\pgfqpoint{1.753562in}{1.151778in}}%
\pgfpathlineto{\pgfqpoint{1.753562in}{1.148829in}}%
\pgfpathmoveto{\pgfqpoint{1.753562in}{1.148829in}}%
\pgfpathlineto{\pgfqpoint{1.753562in}{1.148829in}}%
\pgfpathlineto{\pgfqpoint{1.753562in}{1.151778in}}%
\pgfpathlineto{\pgfqpoint{1.758103in}{1.151778in}}%
\pgfpathlineto{\pgfqpoint{1.758103in}{1.148829in}}%
\pgfpathmoveto{\pgfqpoint{1.753562in}{1.151778in}}%
\pgfpathlineto{\pgfqpoint{1.753562in}{1.151778in}}%
\pgfpathlineto{\pgfqpoint{1.753562in}{1.154727in}}%
\pgfpathlineto{\pgfqpoint{1.758103in}{1.154727in}}%
\pgfpathlineto{\pgfqpoint{1.758103in}{1.151778in}}%
\pgfpathmoveto{\pgfqpoint{1.758103in}{1.151778in}}%
\pgfpathlineto{\pgfqpoint{1.758103in}{1.151778in}}%
\pgfpathlineto{\pgfqpoint{1.758103in}{1.154727in}}%
\pgfpathlineto{\pgfqpoint{1.762644in}{1.154727in}}%
\pgfpathlineto{\pgfqpoint{1.762644in}{1.151778in}}%
\pgfpathmoveto{\pgfqpoint{1.758103in}{1.154727in}}%
\pgfpathlineto{\pgfqpoint{1.758103in}{1.154727in}}%
\pgfpathlineto{\pgfqpoint{1.758103in}{1.157676in}}%
\pgfpathlineto{\pgfqpoint{1.762644in}{1.157676in}}%
\pgfpathlineto{\pgfqpoint{1.762644in}{1.154727in}}%
\pgfpathmoveto{\pgfqpoint{1.762644in}{1.154727in}}%
\pgfpathlineto{\pgfqpoint{1.762644in}{1.154727in}}%
\pgfpathlineto{\pgfqpoint{1.762644in}{1.157676in}}%
\pgfpathlineto{\pgfqpoint{1.767185in}{1.157676in}}%
\pgfpathlineto{\pgfqpoint{1.767185in}{1.154727in}}%
\pgfpathmoveto{\pgfqpoint{1.762644in}{1.157676in}}%
\pgfpathlineto{\pgfqpoint{1.762644in}{1.157676in}}%
\pgfpathlineto{\pgfqpoint{1.762644in}{1.160626in}}%
\pgfpathlineto{\pgfqpoint{1.767185in}{1.160626in}}%
\pgfpathlineto{\pgfqpoint{1.767185in}{1.157676in}}%
\pgfpathmoveto{\pgfqpoint{1.767185in}{1.157676in}}%
\pgfpathlineto{\pgfqpoint{1.767185in}{1.157676in}}%
\pgfpathlineto{\pgfqpoint{1.767185in}{1.160626in}}%
\pgfpathlineto{\pgfqpoint{1.771727in}{1.160626in}}%
\pgfpathlineto{\pgfqpoint{1.771727in}{1.157676in}}%
\pgfpathmoveto{\pgfqpoint{1.767185in}{1.160626in}}%
\pgfpathlineto{\pgfqpoint{1.767185in}{1.160626in}}%
\pgfpathlineto{\pgfqpoint{1.767185in}{1.163575in}}%
\pgfpathlineto{\pgfqpoint{1.771727in}{1.163575in}}%
\pgfpathlineto{\pgfqpoint{1.771727in}{1.160626in}}%
\pgfpathmoveto{\pgfqpoint{1.771727in}{1.160626in}}%
\pgfpathlineto{\pgfqpoint{1.771727in}{1.160626in}}%
\pgfpathlineto{\pgfqpoint{1.771727in}{1.163575in}}%
\pgfpathlineto{\pgfqpoint{1.776268in}{1.163575in}}%
\pgfpathlineto{\pgfqpoint{1.776268in}{1.160626in}}%
\pgfpathmoveto{\pgfqpoint{1.771727in}{1.163575in}}%
\pgfpathlineto{\pgfqpoint{1.771727in}{1.163575in}}%
\pgfpathlineto{\pgfqpoint{1.771727in}{1.166524in}}%
\pgfpathlineto{\pgfqpoint{1.776268in}{1.166524in}}%
\pgfpathlineto{\pgfqpoint{1.776268in}{1.163575in}}%
\pgfpathmoveto{\pgfqpoint{1.776268in}{1.163575in}}%
\pgfpathlineto{\pgfqpoint{1.776268in}{1.163575in}}%
\pgfpathlineto{\pgfqpoint{1.776268in}{1.166524in}}%
\pgfpathlineto{\pgfqpoint{1.780809in}{1.166524in}}%
\pgfpathlineto{\pgfqpoint{1.780809in}{1.163575in}}%
\pgfpathmoveto{\pgfqpoint{1.776268in}{1.166524in}}%
\pgfpathlineto{\pgfqpoint{1.776268in}{1.166524in}}%
\pgfpathlineto{\pgfqpoint{1.776268in}{1.169473in}}%
\pgfpathlineto{\pgfqpoint{1.780809in}{1.169473in}}%
\pgfpathlineto{\pgfqpoint{1.780809in}{1.166524in}}%
\pgfpathmoveto{\pgfqpoint{1.780809in}{1.166524in}}%
\pgfpathlineto{\pgfqpoint{1.780809in}{1.166524in}}%
\pgfpathlineto{\pgfqpoint{1.780809in}{1.169473in}}%
\pgfpathlineto{\pgfqpoint{1.785350in}{1.169473in}}%
\pgfpathlineto{\pgfqpoint{1.785350in}{1.166524in}}%
\pgfpathmoveto{\pgfqpoint{1.780809in}{1.169473in}}%
\pgfpathlineto{\pgfqpoint{1.780809in}{1.169473in}}%
\pgfpathlineto{\pgfqpoint{1.780809in}{1.172422in}}%
\pgfpathlineto{\pgfqpoint{1.785350in}{1.172422in}}%
\pgfpathlineto{\pgfqpoint{1.785350in}{1.169473in}}%
\pgfpathmoveto{\pgfqpoint{1.785350in}{1.169473in}}%
\pgfpathlineto{\pgfqpoint{1.785350in}{1.169473in}}%
\pgfpathlineto{\pgfqpoint{1.785350in}{1.172422in}}%
\pgfpathlineto{\pgfqpoint{1.789891in}{1.172422in}}%
\pgfpathlineto{\pgfqpoint{1.789891in}{1.169473in}}%
\pgfpathmoveto{\pgfqpoint{1.785350in}{1.172422in}}%
\pgfpathlineto{\pgfqpoint{1.785350in}{1.172422in}}%
\pgfpathlineto{\pgfqpoint{1.785350in}{1.175371in}}%
\pgfpathlineto{\pgfqpoint{1.789891in}{1.175371in}}%
\pgfpathlineto{\pgfqpoint{1.789891in}{1.172422in}}%
\pgfpathmoveto{\pgfqpoint{1.789891in}{1.172422in}}%
\pgfpathlineto{\pgfqpoint{1.789891in}{1.172422in}}%
\pgfpathlineto{\pgfqpoint{1.789891in}{1.175371in}}%
\pgfpathlineto{\pgfqpoint{1.794433in}{1.175371in}}%
\pgfpathlineto{\pgfqpoint{1.794433in}{1.172422in}}%
\pgfpathmoveto{\pgfqpoint{1.789891in}{1.175371in}}%
\pgfpathlineto{\pgfqpoint{1.789891in}{1.175371in}}%
\pgfpathlineto{\pgfqpoint{1.789891in}{1.178321in}}%
\pgfpathlineto{\pgfqpoint{1.794433in}{1.178321in}}%
\pgfpathlineto{\pgfqpoint{1.794433in}{1.175371in}}%
\pgfpathmoveto{\pgfqpoint{1.794433in}{1.175371in}}%
\pgfpathlineto{\pgfqpoint{1.794433in}{1.175371in}}%
\pgfpathlineto{\pgfqpoint{1.794433in}{1.178321in}}%
\pgfpathlineto{\pgfqpoint{1.798974in}{1.178321in}}%
\pgfpathlineto{\pgfqpoint{1.798974in}{1.175371in}}%
\pgfpathmoveto{\pgfqpoint{1.794433in}{1.178321in}}%
\pgfpathlineto{\pgfqpoint{1.794433in}{1.178321in}}%
\pgfpathlineto{\pgfqpoint{1.794433in}{1.181270in}}%
\pgfpathlineto{\pgfqpoint{1.798974in}{1.181270in}}%
\pgfpathlineto{\pgfqpoint{1.798974in}{1.178321in}}%
\pgfpathmoveto{\pgfqpoint{1.798974in}{1.178321in}}%
\pgfpathlineto{\pgfqpoint{1.798974in}{1.178321in}}%
\pgfpathlineto{\pgfqpoint{1.798974in}{1.181270in}}%
\pgfpathlineto{\pgfqpoint{1.803515in}{1.181270in}}%
\pgfpathlineto{\pgfqpoint{1.803515in}{1.178321in}}%
\pgfpathmoveto{\pgfqpoint{1.798974in}{1.181270in}}%
\pgfpathlineto{\pgfqpoint{1.798974in}{1.181270in}}%
\pgfpathlineto{\pgfqpoint{1.798974in}{1.184219in}}%
\pgfpathlineto{\pgfqpoint{1.803515in}{1.184219in}}%
\pgfpathlineto{\pgfqpoint{1.803515in}{1.181270in}}%
\pgfpathmoveto{\pgfqpoint{1.803515in}{1.181270in}}%
\pgfpathlineto{\pgfqpoint{1.803515in}{1.181270in}}%
\pgfpathlineto{\pgfqpoint{1.803515in}{1.184219in}}%
\pgfpathlineto{\pgfqpoint{1.808056in}{1.184219in}}%
\pgfpathlineto{\pgfqpoint{1.808056in}{1.181270in}}%
\pgfpathmoveto{\pgfqpoint{1.803515in}{1.184219in}}%
\pgfpathlineto{\pgfqpoint{1.803515in}{1.184219in}}%
\pgfpathlineto{\pgfqpoint{1.803515in}{1.187168in}}%
\pgfpathlineto{\pgfqpoint{1.808056in}{1.187168in}}%
\pgfpathlineto{\pgfqpoint{1.808056in}{1.184219in}}%
\pgfpathmoveto{\pgfqpoint{1.808056in}{1.184219in}}%
\pgfpathlineto{\pgfqpoint{1.808056in}{1.184219in}}%
\pgfpathlineto{\pgfqpoint{1.808056in}{1.187168in}}%
\pgfpathlineto{\pgfqpoint{1.812597in}{1.187168in}}%
\pgfpathlineto{\pgfqpoint{1.812597in}{1.184219in}}%
\pgfpathmoveto{\pgfqpoint{1.808056in}{1.187168in}}%
\pgfpathlineto{\pgfqpoint{1.808056in}{1.187168in}}%
\pgfpathlineto{\pgfqpoint{1.808056in}{1.190117in}}%
\pgfpathlineto{\pgfqpoint{1.812597in}{1.190117in}}%
\pgfpathlineto{\pgfqpoint{1.812597in}{1.187168in}}%
\pgfpathmoveto{\pgfqpoint{1.812597in}{1.187168in}}%
\pgfpathlineto{\pgfqpoint{1.812597in}{1.187168in}}%
\pgfpathlineto{\pgfqpoint{1.812597in}{1.190117in}}%
\pgfpathlineto{\pgfqpoint{1.817139in}{1.190117in}}%
\pgfpathlineto{\pgfqpoint{1.817139in}{1.187168in}}%
\pgfpathmoveto{\pgfqpoint{1.812597in}{1.190117in}}%
\pgfpathlineto{\pgfqpoint{1.812597in}{1.190117in}}%
\pgfpathlineto{\pgfqpoint{1.812597in}{1.193066in}}%
\pgfpathlineto{\pgfqpoint{1.817139in}{1.193066in}}%
\pgfpathlineto{\pgfqpoint{1.817139in}{1.190117in}}%
\pgfpathmoveto{\pgfqpoint{1.817139in}{1.190117in}}%
\pgfpathlineto{\pgfqpoint{1.817139in}{1.190117in}}%
\pgfpathlineto{\pgfqpoint{1.817139in}{1.193066in}}%
\pgfpathlineto{\pgfqpoint{1.821680in}{1.193066in}}%
\pgfpathlineto{\pgfqpoint{1.821680in}{1.190117in}}%
\pgfpathmoveto{\pgfqpoint{1.817139in}{1.193066in}}%
\pgfpathlineto{\pgfqpoint{1.817139in}{1.193066in}}%
\pgfpathlineto{\pgfqpoint{1.817139in}{1.196016in}}%
\pgfpathlineto{\pgfqpoint{1.821680in}{1.196016in}}%
\pgfpathlineto{\pgfqpoint{1.821680in}{1.193066in}}%
\pgfpathmoveto{\pgfqpoint{1.817139in}{1.196016in}}%
\pgfpathlineto{\pgfqpoint{1.817139in}{1.196016in}}%
\pgfpathlineto{\pgfqpoint{1.817139in}{1.198965in}}%
\pgfpathlineto{\pgfqpoint{1.821680in}{1.198965in}}%
\pgfpathlineto{\pgfqpoint{1.821680in}{1.196016in}}%
\pgfpathmoveto{\pgfqpoint{1.821680in}{1.196016in}}%
\pgfpathlineto{\pgfqpoint{1.821680in}{1.196016in}}%
\pgfpathlineto{\pgfqpoint{1.821680in}{1.198965in}}%
\pgfpathlineto{\pgfqpoint{1.826221in}{1.198965in}}%
\pgfpathlineto{\pgfqpoint{1.826221in}{1.196016in}}%
\pgfpathmoveto{\pgfqpoint{1.821680in}{1.198965in}}%
\pgfpathlineto{\pgfqpoint{1.821680in}{1.198965in}}%
\pgfpathlineto{\pgfqpoint{1.821680in}{1.201914in}}%
\pgfpathlineto{\pgfqpoint{1.826221in}{1.201914in}}%
\pgfpathlineto{\pgfqpoint{1.826221in}{1.198965in}}%
\pgfpathmoveto{\pgfqpoint{1.826221in}{1.198965in}}%
\pgfpathlineto{\pgfqpoint{1.826221in}{1.198965in}}%
\pgfpathlineto{\pgfqpoint{1.826221in}{1.201914in}}%
\pgfpathlineto{\pgfqpoint{1.830762in}{1.201914in}}%
\pgfpathlineto{\pgfqpoint{1.830762in}{1.198965in}}%
\pgfpathmoveto{\pgfqpoint{1.826221in}{1.201914in}}%
\pgfpathlineto{\pgfqpoint{1.826221in}{1.201914in}}%
\pgfpathlineto{\pgfqpoint{1.826221in}{1.204863in}}%
\pgfpathlineto{\pgfqpoint{1.830762in}{1.204863in}}%
\pgfpathlineto{\pgfqpoint{1.830762in}{1.201914in}}%
\pgfpathmoveto{\pgfqpoint{1.830762in}{1.201914in}}%
\pgfpathlineto{\pgfqpoint{1.830762in}{1.201914in}}%
\pgfpathlineto{\pgfqpoint{1.830762in}{1.204863in}}%
\pgfpathlineto{\pgfqpoint{1.835303in}{1.204863in}}%
\pgfpathlineto{\pgfqpoint{1.835303in}{1.201914in}}%
\pgfpathmoveto{\pgfqpoint{1.830762in}{1.204863in}}%
\pgfpathlineto{\pgfqpoint{1.830762in}{1.204863in}}%
\pgfpathlineto{\pgfqpoint{1.830762in}{1.207812in}}%
\pgfpathlineto{\pgfqpoint{1.835303in}{1.207812in}}%
\pgfpathlineto{\pgfqpoint{1.835303in}{1.204863in}}%
\pgfpathmoveto{\pgfqpoint{1.835303in}{1.204863in}}%
\pgfpathlineto{\pgfqpoint{1.835303in}{1.204863in}}%
\pgfpathlineto{\pgfqpoint{1.835303in}{1.207812in}}%
\pgfpathlineto{\pgfqpoint{1.839845in}{1.207812in}}%
\pgfpathlineto{\pgfqpoint{1.839845in}{1.204863in}}%
\pgfpathmoveto{\pgfqpoint{1.835303in}{1.207812in}}%
\pgfpathlineto{\pgfqpoint{1.835303in}{1.207812in}}%
\pgfpathlineto{\pgfqpoint{1.835303in}{1.210761in}}%
\pgfpathlineto{\pgfqpoint{1.839845in}{1.210761in}}%
\pgfpathlineto{\pgfqpoint{1.839845in}{1.207812in}}%
\pgfpathmoveto{\pgfqpoint{1.839845in}{1.207812in}}%
\pgfpathlineto{\pgfqpoint{1.839845in}{1.207812in}}%
\pgfpathlineto{\pgfqpoint{1.839845in}{1.210761in}}%
\pgfpathlineto{\pgfqpoint{1.844386in}{1.210761in}}%
\pgfpathlineto{\pgfqpoint{1.844386in}{1.207812in}}%
\pgfpathmoveto{\pgfqpoint{1.839845in}{1.210761in}}%
\pgfpathlineto{\pgfqpoint{1.839845in}{1.210761in}}%
\pgfpathlineto{\pgfqpoint{1.839845in}{1.213711in}}%
\pgfpathlineto{\pgfqpoint{1.844386in}{1.213711in}}%
\pgfpathlineto{\pgfqpoint{1.844386in}{1.210761in}}%
\pgfpathmoveto{\pgfqpoint{1.844386in}{1.210761in}}%
\pgfpathlineto{\pgfqpoint{1.844386in}{1.210761in}}%
\pgfpathlineto{\pgfqpoint{1.844386in}{1.213711in}}%
\pgfpathlineto{\pgfqpoint{1.848927in}{1.213711in}}%
\pgfpathlineto{\pgfqpoint{1.848927in}{1.210761in}}%
\pgfpathmoveto{\pgfqpoint{1.844386in}{1.213711in}}%
\pgfpathlineto{\pgfqpoint{1.844386in}{1.213711in}}%
\pgfpathlineto{\pgfqpoint{1.844386in}{1.216660in}}%
\pgfpathlineto{\pgfqpoint{1.848927in}{1.216660in}}%
\pgfpathlineto{\pgfqpoint{1.848927in}{1.213711in}}%
\pgfpathmoveto{\pgfqpoint{1.848927in}{1.213711in}}%
\pgfpathlineto{\pgfqpoint{1.848927in}{1.213711in}}%
\pgfpathlineto{\pgfqpoint{1.848927in}{1.216660in}}%
\pgfpathlineto{\pgfqpoint{1.853468in}{1.216660in}}%
\pgfpathlineto{\pgfqpoint{1.853468in}{1.213711in}}%
\pgfpathmoveto{\pgfqpoint{1.848927in}{1.216660in}}%
\pgfpathlineto{\pgfqpoint{1.848927in}{1.216660in}}%
\pgfpathlineto{\pgfqpoint{1.848927in}{1.219609in}}%
\pgfpathlineto{\pgfqpoint{1.853468in}{1.219609in}}%
\pgfpathlineto{\pgfqpoint{1.853468in}{1.216660in}}%
\pgfpathmoveto{\pgfqpoint{1.853468in}{1.216660in}}%
\pgfpathlineto{\pgfqpoint{1.853468in}{1.216660in}}%
\pgfpathlineto{\pgfqpoint{1.853468in}{1.219609in}}%
\pgfpathlineto{\pgfqpoint{1.858009in}{1.219609in}}%
\pgfpathlineto{\pgfqpoint{1.858009in}{1.216660in}}%
\pgfpathmoveto{\pgfqpoint{1.853468in}{1.219609in}}%
\pgfpathlineto{\pgfqpoint{1.853468in}{1.219609in}}%
\pgfpathlineto{\pgfqpoint{1.853468in}{1.222558in}}%
\pgfpathlineto{\pgfqpoint{1.858009in}{1.222558in}}%
\pgfpathlineto{\pgfqpoint{1.858009in}{1.219609in}}%
\pgfpathmoveto{\pgfqpoint{1.858009in}{1.219609in}}%
\pgfpathlineto{\pgfqpoint{1.858009in}{1.219609in}}%
\pgfpathlineto{\pgfqpoint{1.858009in}{1.222558in}}%
\pgfpathlineto{\pgfqpoint{1.862551in}{1.222558in}}%
\pgfpathlineto{\pgfqpoint{1.862551in}{1.219609in}}%
\pgfpathmoveto{\pgfqpoint{1.858009in}{1.222558in}}%
\pgfpathlineto{\pgfqpoint{1.858009in}{1.222558in}}%
\pgfpathlineto{\pgfqpoint{1.858009in}{1.225507in}}%
\pgfpathlineto{\pgfqpoint{1.862551in}{1.225507in}}%
\pgfpathlineto{\pgfqpoint{1.862551in}{1.222558in}}%
\pgfpathmoveto{\pgfqpoint{1.862551in}{1.222558in}}%
\pgfpathlineto{\pgfqpoint{1.862551in}{1.222558in}}%
\pgfpathlineto{\pgfqpoint{1.862551in}{1.225507in}}%
\pgfpathlineto{\pgfqpoint{1.867092in}{1.225507in}}%
\pgfpathlineto{\pgfqpoint{1.867092in}{1.222558in}}%
\pgfpathmoveto{\pgfqpoint{1.862551in}{1.225507in}}%
\pgfpathlineto{\pgfqpoint{1.862551in}{1.225507in}}%
\pgfpathlineto{\pgfqpoint{1.862551in}{1.228456in}}%
\pgfpathlineto{\pgfqpoint{1.867092in}{1.228456in}}%
\pgfpathlineto{\pgfqpoint{1.867092in}{1.225507in}}%
\pgfpathmoveto{\pgfqpoint{1.867092in}{1.225507in}}%
\pgfpathlineto{\pgfqpoint{1.867092in}{1.225507in}}%
\pgfpathlineto{\pgfqpoint{1.867092in}{1.228456in}}%
\pgfpathlineto{\pgfqpoint{1.871633in}{1.228456in}}%
\pgfpathlineto{\pgfqpoint{1.871633in}{1.225507in}}%
\pgfpathmoveto{\pgfqpoint{1.867092in}{1.228456in}}%
\pgfpathlineto{\pgfqpoint{1.867092in}{1.228456in}}%
\pgfpathlineto{\pgfqpoint{1.867092in}{1.231406in}}%
\pgfpathlineto{\pgfqpoint{1.871633in}{1.231406in}}%
\pgfpathlineto{\pgfqpoint{1.871633in}{1.228456in}}%
\pgfpathmoveto{\pgfqpoint{1.871633in}{1.228456in}}%
\pgfpathlineto{\pgfqpoint{1.871633in}{1.228456in}}%
\pgfpathlineto{\pgfqpoint{1.871633in}{1.231406in}}%
\pgfpathlineto{\pgfqpoint{1.876174in}{1.231406in}}%
\pgfpathlineto{\pgfqpoint{1.876174in}{1.228456in}}%
\pgfpathmoveto{\pgfqpoint{1.871633in}{1.231406in}}%
\pgfpathlineto{\pgfqpoint{1.871633in}{1.231406in}}%
\pgfpathlineto{\pgfqpoint{1.871633in}{1.234355in}}%
\pgfpathlineto{\pgfqpoint{1.876174in}{1.234355in}}%
\pgfpathlineto{\pgfqpoint{1.876174in}{1.231406in}}%
\pgfpathmoveto{\pgfqpoint{1.876174in}{1.231406in}}%
\pgfpathlineto{\pgfqpoint{1.876174in}{1.231406in}}%
\pgfpathlineto{\pgfqpoint{1.876174in}{1.234355in}}%
\pgfpathlineto{\pgfqpoint{1.880715in}{1.234355in}}%
\pgfpathlineto{\pgfqpoint{1.880715in}{1.231406in}}%
\pgfpathmoveto{\pgfqpoint{1.876174in}{1.234355in}}%
\pgfpathlineto{\pgfqpoint{1.876174in}{1.234355in}}%
\pgfpathlineto{\pgfqpoint{1.876174in}{1.237304in}}%
\pgfpathlineto{\pgfqpoint{1.880715in}{1.237304in}}%
\pgfpathlineto{\pgfqpoint{1.880715in}{1.234355in}}%
\pgfpathmoveto{\pgfqpoint{1.880715in}{1.234355in}}%
\pgfpathlineto{\pgfqpoint{1.880715in}{1.234355in}}%
\pgfpathlineto{\pgfqpoint{1.880715in}{1.237304in}}%
\pgfpathlineto{\pgfqpoint{1.885257in}{1.237304in}}%
\pgfpathlineto{\pgfqpoint{1.885257in}{1.234355in}}%
\pgfpathmoveto{\pgfqpoint{1.880715in}{1.237304in}}%
\pgfpathlineto{\pgfqpoint{1.880715in}{1.237304in}}%
\pgfpathlineto{\pgfqpoint{1.880715in}{1.240253in}}%
\pgfpathlineto{\pgfqpoint{1.885257in}{1.240253in}}%
\pgfpathlineto{\pgfqpoint{1.885257in}{1.237304in}}%
\pgfpathmoveto{\pgfqpoint{1.885257in}{1.237304in}}%
\pgfpathlineto{\pgfqpoint{1.885257in}{1.237304in}}%
\pgfpathlineto{\pgfqpoint{1.885257in}{1.240253in}}%
\pgfpathlineto{\pgfqpoint{1.889798in}{1.240253in}}%
\pgfpathlineto{\pgfqpoint{1.889798in}{1.237304in}}%
\pgfpathmoveto{\pgfqpoint{1.885257in}{1.240253in}}%
\pgfpathlineto{\pgfqpoint{1.885257in}{1.240253in}}%
\pgfpathlineto{\pgfqpoint{1.885257in}{1.243202in}}%
\pgfpathlineto{\pgfqpoint{1.889798in}{1.243202in}}%
\pgfpathlineto{\pgfqpoint{1.889798in}{1.240253in}}%
\pgfpathmoveto{\pgfqpoint{1.889798in}{1.240253in}}%
\pgfpathlineto{\pgfqpoint{1.889798in}{1.240253in}}%
\pgfpathlineto{\pgfqpoint{1.889798in}{1.243202in}}%
\pgfpathlineto{\pgfqpoint{1.894339in}{1.243202in}}%
\pgfpathlineto{\pgfqpoint{1.894339in}{1.240253in}}%
\pgfpathmoveto{\pgfqpoint{1.889798in}{1.243202in}}%
\pgfpathlineto{\pgfqpoint{1.889798in}{1.243202in}}%
\pgfpathlineto{\pgfqpoint{1.889798in}{1.246151in}}%
\pgfpathlineto{\pgfqpoint{1.894339in}{1.246151in}}%
\pgfpathlineto{\pgfqpoint{1.894339in}{1.243202in}}%
\pgfpathmoveto{\pgfqpoint{1.894339in}{1.243202in}}%
\pgfpathlineto{\pgfqpoint{1.894339in}{1.243202in}}%
\pgfpathlineto{\pgfqpoint{1.894339in}{1.246151in}}%
\pgfpathlineto{\pgfqpoint{1.898880in}{1.246151in}}%
\pgfpathlineto{\pgfqpoint{1.898880in}{1.243202in}}%
\pgfpathmoveto{\pgfqpoint{1.894339in}{1.246151in}}%
\pgfpathlineto{\pgfqpoint{1.894339in}{1.246151in}}%
\pgfpathlineto{\pgfqpoint{1.894339in}{1.249101in}}%
\pgfpathlineto{\pgfqpoint{1.898880in}{1.249101in}}%
\pgfpathlineto{\pgfqpoint{1.898880in}{1.246151in}}%
\pgfpathmoveto{\pgfqpoint{1.898880in}{1.246151in}}%
\pgfpathlineto{\pgfqpoint{1.898880in}{1.246151in}}%
\pgfpathlineto{\pgfqpoint{1.898880in}{1.249101in}}%
\pgfpathlineto{\pgfqpoint{1.903421in}{1.249101in}}%
\pgfpathlineto{\pgfqpoint{1.903421in}{1.246151in}}%
\pgfpathmoveto{\pgfqpoint{1.898880in}{1.249101in}}%
\pgfpathlineto{\pgfqpoint{1.898880in}{1.249101in}}%
\pgfpathlineto{\pgfqpoint{1.898880in}{1.252050in}}%
\pgfpathlineto{\pgfqpoint{1.903421in}{1.252050in}}%
\pgfpathlineto{\pgfqpoint{1.903421in}{1.249101in}}%
\pgfpathmoveto{\pgfqpoint{1.903421in}{1.249101in}}%
\pgfpathlineto{\pgfqpoint{1.903421in}{1.249101in}}%
\pgfpathlineto{\pgfqpoint{1.903421in}{1.252050in}}%
\pgfpathlineto{\pgfqpoint{1.907963in}{1.252050in}}%
\pgfpathlineto{\pgfqpoint{1.907963in}{1.249101in}}%
\pgfpathmoveto{\pgfqpoint{1.903421in}{1.252050in}}%
\pgfpathlineto{\pgfqpoint{1.903421in}{1.252050in}}%
\pgfpathlineto{\pgfqpoint{1.903421in}{1.254999in}}%
\pgfpathlineto{\pgfqpoint{1.907963in}{1.254999in}}%
\pgfpathlineto{\pgfqpoint{1.907963in}{1.252050in}}%
\pgfpathmoveto{\pgfqpoint{1.907963in}{1.252050in}}%
\pgfpathlineto{\pgfqpoint{1.907963in}{1.252050in}}%
\pgfpathlineto{\pgfqpoint{1.907963in}{1.254999in}}%
\pgfpathlineto{\pgfqpoint{1.912504in}{1.254999in}}%
\pgfpathlineto{\pgfqpoint{1.912504in}{1.252050in}}%
\pgfpathmoveto{\pgfqpoint{1.907963in}{1.254999in}}%
\pgfpathlineto{\pgfqpoint{1.907963in}{1.254999in}}%
\pgfpathlineto{\pgfqpoint{1.907963in}{1.257948in}}%
\pgfpathlineto{\pgfqpoint{1.912504in}{1.257948in}}%
\pgfpathlineto{\pgfqpoint{1.912504in}{1.254999in}}%
\pgfpathmoveto{\pgfqpoint{1.912504in}{1.254999in}}%
\pgfpathlineto{\pgfqpoint{1.912504in}{1.254999in}}%
\pgfpathlineto{\pgfqpoint{1.912504in}{1.257948in}}%
\pgfpathlineto{\pgfqpoint{1.917045in}{1.257948in}}%
\pgfpathlineto{\pgfqpoint{1.917045in}{1.254999in}}%
\pgfpathmoveto{\pgfqpoint{1.912504in}{1.257948in}}%
\pgfpathlineto{\pgfqpoint{1.912504in}{1.257948in}}%
\pgfpathlineto{\pgfqpoint{1.912504in}{1.260897in}}%
\pgfpathlineto{\pgfqpoint{1.917045in}{1.260897in}}%
\pgfpathlineto{\pgfqpoint{1.917045in}{1.257948in}}%
\pgfpathmoveto{\pgfqpoint{1.917045in}{1.257948in}}%
\pgfpathlineto{\pgfqpoint{1.917045in}{1.257948in}}%
\pgfpathlineto{\pgfqpoint{1.917045in}{1.260897in}}%
\pgfpathlineto{\pgfqpoint{1.921586in}{1.260897in}}%
\pgfpathlineto{\pgfqpoint{1.921586in}{1.257948in}}%
\pgfpathmoveto{\pgfqpoint{1.917045in}{1.260897in}}%
\pgfpathlineto{\pgfqpoint{1.917045in}{1.260897in}}%
\pgfpathlineto{\pgfqpoint{1.917045in}{1.263846in}}%
\pgfpathlineto{\pgfqpoint{1.921586in}{1.263846in}}%
\pgfpathlineto{\pgfqpoint{1.921586in}{1.260897in}}%
\pgfpathmoveto{\pgfqpoint{1.921586in}{1.260897in}}%
\pgfpathlineto{\pgfqpoint{1.921586in}{1.260897in}}%
\pgfpathlineto{\pgfqpoint{1.921586in}{1.263846in}}%
\pgfpathlineto{\pgfqpoint{1.926126in}{1.263846in}}%
\pgfpathlineto{\pgfqpoint{1.926126in}{1.260897in}}%
\pgfpathmoveto{\pgfqpoint{1.921586in}{1.263846in}}%
\pgfpathlineto{\pgfqpoint{1.921586in}{1.263846in}}%
\pgfpathlineto{\pgfqpoint{1.921586in}{1.266796in}}%
\pgfpathlineto{\pgfqpoint{1.926126in}{1.266796in}}%
\pgfpathlineto{\pgfqpoint{1.926126in}{1.263846in}}%
\pgfpathmoveto{\pgfqpoint{1.926126in}{1.263846in}}%
\pgfpathlineto{\pgfqpoint{1.926126in}{1.263846in}}%
\pgfpathlineto{\pgfqpoint{1.926126in}{1.266796in}}%
\pgfpathlineto{\pgfqpoint{1.930667in}{1.266796in}}%
\pgfpathlineto{\pgfqpoint{1.930667in}{1.263846in}}%
\pgfpathmoveto{\pgfqpoint{1.926126in}{1.266796in}}%
\pgfpathlineto{\pgfqpoint{1.926126in}{1.266796in}}%
\pgfpathlineto{\pgfqpoint{1.926126in}{1.269745in}}%
\pgfpathlineto{\pgfqpoint{1.930667in}{1.269745in}}%
\pgfpathlineto{\pgfqpoint{1.930667in}{1.266796in}}%
\pgfpathmoveto{\pgfqpoint{1.930667in}{1.266796in}}%
\pgfpathlineto{\pgfqpoint{1.930667in}{1.266796in}}%
\pgfpathlineto{\pgfqpoint{1.930667in}{1.269745in}}%
\pgfpathlineto{\pgfqpoint{1.935208in}{1.269745in}}%
\pgfpathlineto{\pgfqpoint{1.935208in}{1.266796in}}%
\pgfpathmoveto{\pgfqpoint{1.930667in}{1.269745in}}%
\pgfpathlineto{\pgfqpoint{1.930667in}{1.269745in}}%
\pgfpathlineto{\pgfqpoint{1.930667in}{1.272694in}}%
\pgfpathlineto{\pgfqpoint{1.935208in}{1.272694in}}%
\pgfpathlineto{\pgfqpoint{1.935208in}{1.269745in}}%
\pgfpathmoveto{\pgfqpoint{1.935208in}{1.269745in}}%
\pgfpathlineto{\pgfqpoint{1.935208in}{1.269745in}}%
\pgfpathlineto{\pgfqpoint{1.935208in}{1.272694in}}%
\pgfpathlineto{\pgfqpoint{1.939749in}{1.272694in}}%
\pgfpathlineto{\pgfqpoint{1.939749in}{1.269745in}}%
\pgfpathmoveto{\pgfqpoint{1.935208in}{1.272694in}}%
\pgfpathlineto{\pgfqpoint{1.935208in}{1.272694in}}%
\pgfpathlineto{\pgfqpoint{1.935208in}{1.275643in}}%
\pgfpathlineto{\pgfqpoint{1.939749in}{1.275643in}}%
\pgfpathlineto{\pgfqpoint{1.939749in}{1.272694in}}%
\pgfpathmoveto{\pgfqpoint{1.939749in}{1.272694in}}%
\pgfpathlineto{\pgfqpoint{1.939749in}{1.272694in}}%
\pgfpathlineto{\pgfqpoint{1.939749in}{1.275643in}}%
\pgfpathlineto{\pgfqpoint{1.944290in}{1.275643in}}%
\pgfpathlineto{\pgfqpoint{1.944290in}{1.272694in}}%
\pgfpathmoveto{\pgfqpoint{1.939749in}{1.275643in}}%
\pgfpathlineto{\pgfqpoint{1.939749in}{1.275643in}}%
\pgfpathlineto{\pgfqpoint{1.939749in}{1.278592in}}%
\pgfpathlineto{\pgfqpoint{1.944290in}{1.278592in}}%
\pgfpathlineto{\pgfqpoint{1.944290in}{1.275643in}}%
\pgfpathmoveto{\pgfqpoint{1.944290in}{1.275643in}}%
\pgfpathlineto{\pgfqpoint{1.944290in}{1.275643in}}%
\pgfpathlineto{\pgfqpoint{1.944290in}{1.278592in}}%
\pgfpathlineto{\pgfqpoint{1.948831in}{1.278592in}}%
\pgfpathlineto{\pgfqpoint{1.948831in}{1.275643in}}%
\pgfpathmoveto{\pgfqpoint{1.944290in}{1.278592in}}%
\pgfpathlineto{\pgfqpoint{1.944290in}{1.278592in}}%
\pgfpathlineto{\pgfqpoint{1.944290in}{1.281542in}}%
\pgfpathlineto{\pgfqpoint{1.948831in}{1.281542in}}%
\pgfpathlineto{\pgfqpoint{1.948831in}{1.278592in}}%
\pgfpathmoveto{\pgfqpoint{1.948831in}{1.278592in}}%
\pgfpathlineto{\pgfqpoint{1.948831in}{1.278592in}}%
\pgfpathlineto{\pgfqpoint{1.948831in}{1.281542in}}%
\pgfpathlineto{\pgfqpoint{1.953372in}{1.281542in}}%
\pgfpathlineto{\pgfqpoint{1.953372in}{1.278592in}}%
\pgfpathmoveto{\pgfqpoint{1.948831in}{1.281542in}}%
\pgfpathlineto{\pgfqpoint{1.948831in}{1.281542in}}%
\pgfpathlineto{\pgfqpoint{1.948831in}{1.284491in}}%
\pgfpathlineto{\pgfqpoint{1.953372in}{1.284491in}}%
\pgfpathlineto{\pgfqpoint{1.953372in}{1.281542in}}%
\pgfpathmoveto{\pgfqpoint{1.953372in}{1.281542in}}%
\pgfpathlineto{\pgfqpoint{1.953372in}{1.281542in}}%
\pgfpathlineto{\pgfqpoint{1.953372in}{1.284491in}}%
\pgfpathlineto{\pgfqpoint{1.957913in}{1.284491in}}%
\pgfpathlineto{\pgfqpoint{1.957913in}{1.281542in}}%
\pgfpathmoveto{\pgfqpoint{1.953372in}{1.284491in}}%
\pgfpathlineto{\pgfqpoint{1.953372in}{1.284491in}}%
\pgfpathlineto{\pgfqpoint{1.953372in}{1.287440in}}%
\pgfpathlineto{\pgfqpoint{1.957913in}{1.287440in}}%
\pgfpathlineto{\pgfqpoint{1.957913in}{1.284491in}}%
\pgfpathmoveto{\pgfqpoint{1.957913in}{1.284491in}}%
\pgfpathlineto{\pgfqpoint{1.957913in}{1.284491in}}%
\pgfpathlineto{\pgfqpoint{1.957913in}{1.287440in}}%
\pgfpathlineto{\pgfqpoint{1.962453in}{1.287440in}}%
\pgfpathlineto{\pgfqpoint{1.962453in}{1.284491in}}%
\pgfpathmoveto{\pgfqpoint{1.957913in}{1.287440in}}%
\pgfpathlineto{\pgfqpoint{1.957913in}{1.287440in}}%
\pgfpathlineto{\pgfqpoint{1.957913in}{1.290389in}}%
\pgfpathlineto{\pgfqpoint{1.962453in}{1.290389in}}%
\pgfpathlineto{\pgfqpoint{1.962453in}{1.287440in}}%
\pgfpathmoveto{\pgfqpoint{1.962453in}{1.287440in}}%
\pgfpathlineto{\pgfqpoint{1.962453in}{1.287440in}}%
\pgfpathlineto{\pgfqpoint{1.962453in}{1.290389in}}%
\pgfpathlineto{\pgfqpoint{1.966994in}{1.290389in}}%
\pgfpathlineto{\pgfqpoint{1.966994in}{1.287440in}}%
\pgfpathmoveto{\pgfqpoint{1.962453in}{1.290389in}}%
\pgfpathlineto{\pgfqpoint{1.962453in}{1.290389in}}%
\pgfpathlineto{\pgfqpoint{1.962453in}{1.293338in}}%
\pgfpathlineto{\pgfqpoint{1.966994in}{1.293338in}}%
\pgfpathlineto{\pgfqpoint{1.966994in}{1.290389in}}%
\pgfpathmoveto{\pgfqpoint{1.966994in}{1.290389in}}%
\pgfpathlineto{\pgfqpoint{1.966994in}{1.290389in}}%
\pgfpathlineto{\pgfqpoint{1.966994in}{1.293338in}}%
\pgfpathlineto{\pgfqpoint{1.971535in}{1.293338in}}%
\pgfpathlineto{\pgfqpoint{1.971535in}{1.290389in}}%
\pgfpathmoveto{\pgfqpoint{1.966994in}{1.293338in}}%
\pgfpathlineto{\pgfqpoint{1.966994in}{1.293338in}}%
\pgfpathlineto{\pgfqpoint{1.966994in}{1.296288in}}%
\pgfpathlineto{\pgfqpoint{1.971535in}{1.296288in}}%
\pgfpathlineto{\pgfqpoint{1.971535in}{1.293338in}}%
\pgfpathmoveto{\pgfqpoint{1.971535in}{1.293338in}}%
\pgfpathlineto{\pgfqpoint{1.971535in}{1.293338in}}%
\pgfpathlineto{\pgfqpoint{1.971535in}{1.296288in}}%
\pgfpathlineto{\pgfqpoint{1.976076in}{1.296288in}}%
\pgfpathlineto{\pgfqpoint{1.976076in}{1.293338in}}%
\pgfpathmoveto{\pgfqpoint{1.971535in}{1.296288in}}%
\pgfpathlineto{\pgfqpoint{1.971535in}{1.296288in}}%
\pgfpathlineto{\pgfqpoint{1.971535in}{1.299237in}}%
\pgfpathlineto{\pgfqpoint{1.976076in}{1.299237in}}%
\pgfpathlineto{\pgfqpoint{1.976076in}{1.296288in}}%
\pgfpathmoveto{\pgfqpoint{1.976076in}{1.296288in}}%
\pgfpathlineto{\pgfqpoint{1.976076in}{1.296288in}}%
\pgfpathlineto{\pgfqpoint{1.976076in}{1.299237in}}%
\pgfpathlineto{\pgfqpoint{1.980617in}{1.299237in}}%
\pgfpathlineto{\pgfqpoint{1.980617in}{1.296288in}}%
\pgfpathmoveto{\pgfqpoint{1.976076in}{1.299237in}}%
\pgfpathlineto{\pgfqpoint{1.976076in}{1.299237in}}%
\pgfpathlineto{\pgfqpoint{1.976076in}{1.302186in}}%
\pgfpathlineto{\pgfqpoint{1.980617in}{1.302186in}}%
\pgfpathlineto{\pgfqpoint{1.980617in}{1.299237in}}%
\pgfpathmoveto{\pgfqpoint{1.980617in}{1.299237in}}%
\pgfpathlineto{\pgfqpoint{1.980617in}{1.299237in}}%
\pgfpathlineto{\pgfqpoint{1.980617in}{1.302186in}}%
\pgfpathlineto{\pgfqpoint{1.985158in}{1.302186in}}%
\pgfpathlineto{\pgfqpoint{1.985158in}{1.299237in}}%
\pgfpathmoveto{\pgfqpoint{1.980617in}{1.302186in}}%
\pgfpathlineto{\pgfqpoint{1.980617in}{1.302186in}}%
\pgfpathlineto{\pgfqpoint{1.980617in}{1.305135in}}%
\pgfpathlineto{\pgfqpoint{1.985158in}{1.305135in}}%
\pgfpathlineto{\pgfqpoint{1.985158in}{1.302186in}}%
\pgfpathmoveto{\pgfqpoint{1.985158in}{1.302186in}}%
\pgfpathlineto{\pgfqpoint{1.985158in}{1.302186in}}%
\pgfpathlineto{\pgfqpoint{1.985158in}{1.305135in}}%
\pgfpathlineto{\pgfqpoint{1.989699in}{1.305135in}}%
\pgfpathlineto{\pgfqpoint{1.989699in}{1.302186in}}%
\pgfpathmoveto{\pgfqpoint{1.985158in}{1.305135in}}%
\pgfpathlineto{\pgfqpoint{1.985158in}{1.305135in}}%
\pgfpathlineto{\pgfqpoint{1.985158in}{1.308084in}}%
\pgfpathlineto{\pgfqpoint{1.989699in}{1.308084in}}%
\pgfpathlineto{\pgfqpoint{1.989699in}{1.305135in}}%
\pgfpathmoveto{\pgfqpoint{1.989699in}{1.305135in}}%
\pgfpathlineto{\pgfqpoint{1.989699in}{1.305135in}}%
\pgfpathlineto{\pgfqpoint{1.989699in}{1.308084in}}%
\pgfpathlineto{\pgfqpoint{1.994239in}{1.308084in}}%
\pgfpathlineto{\pgfqpoint{1.994239in}{1.305135in}}%
\pgfpathmoveto{\pgfqpoint{1.989699in}{1.308084in}}%
\pgfpathlineto{\pgfqpoint{1.989699in}{1.308084in}}%
\pgfpathlineto{\pgfqpoint{1.989699in}{1.311033in}}%
\pgfpathlineto{\pgfqpoint{1.994239in}{1.311033in}}%
\pgfpathlineto{\pgfqpoint{1.994239in}{1.308084in}}%
\pgfpathmoveto{\pgfqpoint{1.994239in}{1.308084in}}%
\pgfpathlineto{\pgfqpoint{1.994239in}{1.308084in}}%
\pgfpathlineto{\pgfqpoint{1.994239in}{1.311033in}}%
\pgfpathlineto{\pgfqpoint{1.998780in}{1.311033in}}%
\pgfpathlineto{\pgfqpoint{1.998780in}{1.308084in}}%
\pgfpathmoveto{\pgfqpoint{1.994239in}{1.311033in}}%
\pgfpathlineto{\pgfqpoint{1.994239in}{1.311033in}}%
\pgfpathlineto{\pgfqpoint{1.994239in}{1.313983in}}%
\pgfpathlineto{\pgfqpoint{1.998780in}{1.313983in}}%
\pgfpathlineto{\pgfqpoint{1.998780in}{1.311033in}}%
\pgfpathmoveto{\pgfqpoint{1.998780in}{1.311033in}}%
\pgfpathlineto{\pgfqpoint{1.998780in}{1.311033in}}%
\pgfpathlineto{\pgfqpoint{1.998780in}{1.313983in}}%
\pgfpathlineto{\pgfqpoint{2.003321in}{1.313983in}}%
\pgfpathlineto{\pgfqpoint{2.003321in}{1.311033in}}%
\pgfpathmoveto{\pgfqpoint{1.998780in}{1.313983in}}%
\pgfpathlineto{\pgfqpoint{1.998780in}{1.313983in}}%
\pgfpathlineto{\pgfqpoint{1.998780in}{1.316932in}}%
\pgfpathlineto{\pgfqpoint{2.003321in}{1.316932in}}%
\pgfpathlineto{\pgfqpoint{2.003321in}{1.313983in}}%
\pgfpathmoveto{\pgfqpoint{2.003321in}{1.313983in}}%
\pgfpathlineto{\pgfqpoint{2.003321in}{1.313983in}}%
\pgfpathlineto{\pgfqpoint{2.003321in}{1.316932in}}%
\pgfpathlineto{\pgfqpoint{2.007862in}{1.316932in}}%
\pgfpathlineto{\pgfqpoint{2.007862in}{1.313983in}}%
\pgfpathmoveto{\pgfqpoint{2.003321in}{1.316932in}}%
\pgfpathlineto{\pgfqpoint{2.003321in}{1.316932in}}%
\pgfpathlineto{\pgfqpoint{2.003321in}{1.319881in}}%
\pgfpathlineto{\pgfqpoint{2.007862in}{1.319881in}}%
\pgfpathlineto{\pgfqpoint{2.007862in}{1.316932in}}%
\pgfpathmoveto{\pgfqpoint{2.007862in}{1.316932in}}%
\pgfpathlineto{\pgfqpoint{2.007862in}{1.316932in}}%
\pgfpathlineto{\pgfqpoint{2.007862in}{1.319881in}}%
\pgfpathlineto{\pgfqpoint{2.012403in}{1.319881in}}%
\pgfpathlineto{\pgfqpoint{2.012403in}{1.316932in}}%
\pgfpathmoveto{\pgfqpoint{2.007862in}{1.319881in}}%
\pgfpathlineto{\pgfqpoint{2.007862in}{1.319881in}}%
\pgfpathlineto{\pgfqpoint{2.007862in}{1.322830in}}%
\pgfpathlineto{\pgfqpoint{2.012403in}{1.322830in}}%
\pgfpathlineto{\pgfqpoint{2.012403in}{1.319881in}}%
\pgfpathmoveto{\pgfqpoint{2.012403in}{1.319881in}}%
\pgfpathlineto{\pgfqpoint{2.012403in}{1.319881in}}%
\pgfpathlineto{\pgfqpoint{2.012403in}{1.322830in}}%
\pgfpathlineto{\pgfqpoint{2.016944in}{1.322830in}}%
\pgfpathlineto{\pgfqpoint{2.016944in}{1.319881in}}%
\pgfpathmoveto{\pgfqpoint{2.012403in}{1.322830in}}%
\pgfpathlineto{\pgfqpoint{2.012403in}{1.322830in}}%
\pgfpathlineto{\pgfqpoint{2.012403in}{1.325779in}}%
\pgfpathlineto{\pgfqpoint{2.016944in}{1.325779in}}%
\pgfpathlineto{\pgfqpoint{2.016944in}{1.322830in}}%
\pgfpathmoveto{\pgfqpoint{2.016944in}{1.322830in}}%
\pgfpathlineto{\pgfqpoint{2.016944in}{1.322830in}}%
\pgfpathlineto{\pgfqpoint{2.016944in}{1.325779in}}%
\pgfpathlineto{\pgfqpoint{2.021485in}{1.325779in}}%
\pgfpathlineto{\pgfqpoint{2.021485in}{1.322830in}}%
\pgfpathmoveto{\pgfqpoint{2.016944in}{1.325779in}}%
\pgfpathlineto{\pgfqpoint{2.016944in}{1.325779in}}%
\pgfpathlineto{\pgfqpoint{2.016944in}{1.328729in}}%
\pgfpathlineto{\pgfqpoint{2.021485in}{1.328729in}}%
\pgfpathlineto{\pgfqpoint{2.021485in}{1.325779in}}%
\pgfpathmoveto{\pgfqpoint{2.021485in}{1.325779in}}%
\pgfpathlineto{\pgfqpoint{2.021485in}{1.325779in}}%
\pgfpathlineto{\pgfqpoint{2.021485in}{1.328729in}}%
\pgfpathlineto{\pgfqpoint{2.026026in}{1.328729in}}%
\pgfpathlineto{\pgfqpoint{2.026026in}{1.325779in}}%
\pgfpathmoveto{\pgfqpoint{2.021485in}{1.328729in}}%
\pgfpathlineto{\pgfqpoint{2.021485in}{1.328729in}}%
\pgfpathlineto{\pgfqpoint{2.021485in}{1.331678in}}%
\pgfpathlineto{\pgfqpoint{2.026026in}{1.331678in}}%
\pgfpathlineto{\pgfqpoint{2.026026in}{1.328729in}}%
\pgfpathmoveto{\pgfqpoint{2.026026in}{1.328729in}}%
\pgfpathlineto{\pgfqpoint{2.026026in}{1.328729in}}%
\pgfpathlineto{\pgfqpoint{2.026026in}{1.331678in}}%
\pgfpathlineto{\pgfqpoint{2.030566in}{1.331678in}}%
\pgfpathlineto{\pgfqpoint{2.030566in}{1.328729in}}%
\pgfpathmoveto{\pgfqpoint{2.026026in}{1.331678in}}%
\pgfpathlineto{\pgfqpoint{2.026026in}{1.331678in}}%
\pgfpathlineto{\pgfqpoint{2.026026in}{1.334627in}}%
\pgfpathlineto{\pgfqpoint{2.030566in}{1.334627in}}%
\pgfpathlineto{\pgfqpoint{2.030566in}{1.331678in}}%
\pgfpathmoveto{\pgfqpoint{2.030566in}{1.331678in}}%
\pgfpathlineto{\pgfqpoint{2.030566in}{1.331678in}}%
\pgfpathlineto{\pgfqpoint{2.030566in}{1.334627in}}%
\pgfpathlineto{\pgfqpoint{2.035107in}{1.334627in}}%
\pgfpathlineto{\pgfqpoint{2.035107in}{1.331678in}}%
\pgfpathmoveto{\pgfqpoint{2.030566in}{1.334627in}}%
\pgfpathlineto{\pgfqpoint{2.030566in}{1.334627in}}%
\pgfpathlineto{\pgfqpoint{2.030566in}{1.337576in}}%
\pgfpathlineto{\pgfqpoint{2.035107in}{1.337576in}}%
\pgfpathlineto{\pgfqpoint{2.035107in}{1.334627in}}%
\pgfpathmoveto{\pgfqpoint{2.035107in}{1.334627in}}%
\pgfpathlineto{\pgfqpoint{2.035107in}{1.334627in}}%
\pgfpathlineto{\pgfqpoint{2.035107in}{1.337576in}}%
\pgfpathlineto{\pgfqpoint{2.039648in}{1.337576in}}%
\pgfpathlineto{\pgfqpoint{2.039648in}{1.334627in}}%
\pgfpathmoveto{\pgfqpoint{2.035107in}{1.337576in}}%
\pgfpathlineto{\pgfqpoint{2.035107in}{1.337576in}}%
\pgfpathlineto{\pgfqpoint{2.035107in}{1.340525in}}%
\pgfpathlineto{\pgfqpoint{2.039648in}{1.340525in}}%
\pgfpathlineto{\pgfqpoint{2.039648in}{1.337576in}}%
\pgfpathmoveto{\pgfqpoint{2.039648in}{1.337576in}}%
\pgfpathlineto{\pgfqpoint{2.039648in}{1.337576in}}%
\pgfpathlineto{\pgfqpoint{2.039648in}{1.340525in}}%
\pgfpathlineto{\pgfqpoint{2.044189in}{1.340525in}}%
\pgfpathlineto{\pgfqpoint{2.044189in}{1.337576in}}%
\pgfpathmoveto{\pgfqpoint{2.039648in}{1.340525in}}%
\pgfpathlineto{\pgfqpoint{2.039648in}{1.340525in}}%
\pgfpathlineto{\pgfqpoint{2.039648in}{1.343475in}}%
\pgfpathlineto{\pgfqpoint{2.044189in}{1.343475in}}%
\pgfpathlineto{\pgfqpoint{2.044189in}{1.340525in}}%
\pgfpathmoveto{\pgfqpoint{2.044189in}{1.340525in}}%
\pgfpathlineto{\pgfqpoint{2.044189in}{1.340525in}}%
\pgfpathlineto{\pgfqpoint{2.044189in}{1.343475in}}%
\pgfpathlineto{\pgfqpoint{2.048730in}{1.343475in}}%
\pgfpathlineto{\pgfqpoint{2.048730in}{1.340525in}}%
\pgfpathmoveto{\pgfqpoint{2.044189in}{1.343475in}}%
\pgfpathlineto{\pgfqpoint{2.044189in}{1.343475in}}%
\pgfpathlineto{\pgfqpoint{2.044189in}{1.346424in}}%
\pgfpathlineto{\pgfqpoint{2.048730in}{1.346424in}}%
\pgfpathlineto{\pgfqpoint{2.048730in}{1.343475in}}%
\pgfpathmoveto{\pgfqpoint{2.048730in}{1.343475in}}%
\pgfpathlineto{\pgfqpoint{2.048730in}{1.343475in}}%
\pgfpathlineto{\pgfqpoint{2.048730in}{1.346424in}}%
\pgfpathlineto{\pgfqpoint{2.053271in}{1.346424in}}%
\pgfpathlineto{\pgfqpoint{2.053271in}{1.343475in}}%
\pgfpathmoveto{\pgfqpoint{2.048730in}{1.346424in}}%
\pgfpathlineto{\pgfqpoint{2.048730in}{1.346424in}}%
\pgfpathlineto{\pgfqpoint{2.048730in}{1.349373in}}%
\pgfpathlineto{\pgfqpoint{2.053271in}{1.349373in}}%
\pgfpathlineto{\pgfqpoint{2.053271in}{1.346424in}}%
\pgfpathmoveto{\pgfqpoint{2.053271in}{1.346424in}}%
\pgfpathlineto{\pgfqpoint{2.053271in}{1.346424in}}%
\pgfpathlineto{\pgfqpoint{2.053271in}{1.349373in}}%
\pgfpathlineto{\pgfqpoint{2.057812in}{1.349373in}}%
\pgfpathlineto{\pgfqpoint{2.057812in}{1.346424in}}%
\pgfpathmoveto{\pgfqpoint{2.053271in}{1.349373in}}%
\pgfpathlineto{\pgfqpoint{2.053271in}{1.349373in}}%
\pgfpathlineto{\pgfqpoint{2.053271in}{1.352322in}}%
\pgfpathlineto{\pgfqpoint{2.057812in}{1.352322in}}%
\pgfpathlineto{\pgfqpoint{2.057812in}{1.349373in}}%
\pgfpathmoveto{\pgfqpoint{2.057812in}{1.349373in}}%
\pgfpathlineto{\pgfqpoint{2.057812in}{1.349373in}}%
\pgfpathlineto{\pgfqpoint{2.057812in}{1.352322in}}%
\pgfpathlineto{\pgfqpoint{2.062353in}{1.352322in}}%
\pgfpathlineto{\pgfqpoint{2.062353in}{1.349373in}}%
\pgfpathmoveto{\pgfqpoint{2.057812in}{1.352322in}}%
\pgfpathlineto{\pgfqpoint{2.057812in}{1.352322in}}%
\pgfpathlineto{\pgfqpoint{2.057812in}{1.355272in}}%
\pgfpathlineto{\pgfqpoint{2.062353in}{1.355272in}}%
\pgfpathlineto{\pgfqpoint{2.062353in}{1.352322in}}%
\pgfpathmoveto{\pgfqpoint{2.062353in}{1.352322in}}%
\pgfpathlineto{\pgfqpoint{2.062353in}{1.352322in}}%
\pgfpathlineto{\pgfqpoint{2.062353in}{1.355272in}}%
\pgfpathlineto{\pgfqpoint{2.066894in}{1.355272in}}%
\pgfpathlineto{\pgfqpoint{2.066894in}{1.352322in}}%
\pgfpathmoveto{\pgfqpoint{2.062353in}{1.355272in}}%
\pgfpathlineto{\pgfqpoint{2.062353in}{1.355272in}}%
\pgfpathlineto{\pgfqpoint{2.062353in}{1.358221in}}%
\pgfpathlineto{\pgfqpoint{2.066894in}{1.358221in}}%
\pgfpathlineto{\pgfqpoint{2.066894in}{1.355272in}}%
\pgfpathmoveto{\pgfqpoint{2.066894in}{1.355272in}}%
\pgfpathlineto{\pgfqpoint{2.066894in}{1.355272in}}%
\pgfpathlineto{\pgfqpoint{2.066894in}{1.358221in}}%
\pgfpathlineto{\pgfqpoint{2.071435in}{1.358221in}}%
\pgfpathlineto{\pgfqpoint{2.071435in}{1.355272in}}%
\pgfpathmoveto{\pgfqpoint{2.066894in}{1.358221in}}%
\pgfpathlineto{\pgfqpoint{2.066894in}{1.358221in}}%
\pgfpathlineto{\pgfqpoint{2.066894in}{1.361170in}}%
\pgfpathlineto{\pgfqpoint{2.071435in}{1.361170in}}%
\pgfpathlineto{\pgfqpoint{2.071435in}{1.358221in}}%
\pgfpathmoveto{\pgfqpoint{2.071435in}{1.358221in}}%
\pgfpathlineto{\pgfqpoint{2.071435in}{1.358221in}}%
\pgfpathlineto{\pgfqpoint{2.071435in}{1.361170in}}%
\pgfpathlineto{\pgfqpoint{2.075976in}{1.361170in}}%
\pgfpathlineto{\pgfqpoint{2.075976in}{1.358221in}}%
\pgfpathmoveto{\pgfqpoint{2.071435in}{1.361170in}}%
\pgfpathlineto{\pgfqpoint{2.071435in}{1.361170in}}%
\pgfpathlineto{\pgfqpoint{2.071435in}{1.364119in}}%
\pgfpathlineto{\pgfqpoint{2.075976in}{1.364119in}}%
\pgfpathlineto{\pgfqpoint{2.075976in}{1.361170in}}%
\pgfpathmoveto{\pgfqpoint{2.075976in}{1.361170in}}%
\pgfpathlineto{\pgfqpoint{2.075976in}{1.361170in}}%
\pgfpathlineto{\pgfqpoint{2.075976in}{1.364119in}}%
\pgfpathlineto{\pgfqpoint{2.080517in}{1.364119in}}%
\pgfpathlineto{\pgfqpoint{2.080517in}{1.361170in}}%
\pgfpathmoveto{\pgfqpoint{2.075976in}{1.364119in}}%
\pgfpathlineto{\pgfqpoint{2.075976in}{1.364119in}}%
\pgfpathlineto{\pgfqpoint{2.075976in}{1.367069in}}%
\pgfpathlineto{\pgfqpoint{2.080517in}{1.367069in}}%
\pgfpathlineto{\pgfqpoint{2.080517in}{1.364119in}}%
\pgfpathmoveto{\pgfqpoint{2.080517in}{1.364119in}}%
\pgfpathlineto{\pgfqpoint{2.080517in}{1.364119in}}%
\pgfpathlineto{\pgfqpoint{2.080517in}{1.367069in}}%
\pgfpathlineto{\pgfqpoint{2.085058in}{1.367069in}}%
\pgfpathlineto{\pgfqpoint{2.085058in}{1.364119in}}%
\pgfpathmoveto{\pgfqpoint{2.080517in}{1.367069in}}%
\pgfpathlineto{\pgfqpoint{2.080517in}{1.367069in}}%
\pgfpathlineto{\pgfqpoint{2.080517in}{1.370018in}}%
\pgfpathlineto{\pgfqpoint{2.085058in}{1.370018in}}%
\pgfpathlineto{\pgfqpoint{2.085058in}{1.367069in}}%
\pgfpathmoveto{\pgfqpoint{2.085058in}{1.367069in}}%
\pgfpathlineto{\pgfqpoint{2.085058in}{1.367069in}}%
\pgfpathlineto{\pgfqpoint{2.085058in}{1.370018in}}%
\pgfpathlineto{\pgfqpoint{2.089599in}{1.370018in}}%
\pgfpathlineto{\pgfqpoint{2.089599in}{1.367069in}}%
\pgfpathmoveto{\pgfqpoint{2.085058in}{1.370018in}}%
\pgfpathlineto{\pgfqpoint{2.085058in}{1.370018in}}%
\pgfpathlineto{\pgfqpoint{2.085058in}{1.372967in}}%
\pgfpathlineto{\pgfqpoint{2.089599in}{1.372967in}}%
\pgfpathlineto{\pgfqpoint{2.089599in}{1.370018in}}%
\pgfpathmoveto{\pgfqpoint{2.089599in}{1.370018in}}%
\pgfpathlineto{\pgfqpoint{2.089599in}{1.370018in}}%
\pgfpathlineto{\pgfqpoint{2.089599in}{1.372967in}}%
\pgfpathlineto{\pgfqpoint{2.094140in}{1.372967in}}%
\pgfpathlineto{\pgfqpoint{2.094140in}{1.370018in}}%
\pgfpathmoveto{\pgfqpoint{2.089599in}{1.372967in}}%
\pgfpathlineto{\pgfqpoint{2.089599in}{1.372967in}}%
\pgfpathlineto{\pgfqpoint{2.089599in}{1.375917in}}%
\pgfpathlineto{\pgfqpoint{2.094140in}{1.375917in}}%
\pgfpathlineto{\pgfqpoint{2.094140in}{1.372967in}}%
\pgfpathmoveto{\pgfqpoint{2.094140in}{1.372967in}}%
\pgfpathlineto{\pgfqpoint{2.094140in}{1.372967in}}%
\pgfpathlineto{\pgfqpoint{2.094140in}{1.375917in}}%
\pgfpathlineto{\pgfqpoint{2.098681in}{1.375917in}}%
\pgfpathlineto{\pgfqpoint{2.098681in}{1.372967in}}%
\pgfpathmoveto{\pgfqpoint{2.094140in}{1.375917in}}%
\pgfpathlineto{\pgfqpoint{2.094140in}{1.375917in}}%
\pgfpathlineto{\pgfqpoint{2.094140in}{1.378866in}}%
\pgfpathlineto{\pgfqpoint{2.098681in}{1.378866in}}%
\pgfpathlineto{\pgfqpoint{2.098681in}{1.375917in}}%
\pgfpathmoveto{\pgfqpoint{2.098681in}{1.375917in}}%
\pgfpathlineto{\pgfqpoint{2.098681in}{1.375917in}}%
\pgfpathlineto{\pgfqpoint{2.098681in}{1.378866in}}%
\pgfpathlineto{\pgfqpoint{2.103222in}{1.378866in}}%
\pgfpathlineto{\pgfqpoint{2.103222in}{1.375917in}}%
\pgfpathmoveto{\pgfqpoint{2.098681in}{1.378866in}}%
\pgfpathlineto{\pgfqpoint{2.098681in}{1.378866in}}%
\pgfpathlineto{\pgfqpoint{2.098681in}{1.381815in}}%
\pgfpathlineto{\pgfqpoint{2.103222in}{1.381815in}}%
\pgfpathlineto{\pgfqpoint{2.103222in}{1.378866in}}%
\pgfpathmoveto{\pgfqpoint{2.103222in}{1.378866in}}%
\pgfpathlineto{\pgfqpoint{2.103222in}{1.378866in}}%
\pgfpathlineto{\pgfqpoint{2.103222in}{1.381815in}}%
\pgfpathlineto{\pgfqpoint{2.107763in}{1.381815in}}%
\pgfpathlineto{\pgfqpoint{2.107763in}{1.378866in}}%
\pgfpathmoveto{\pgfqpoint{2.103222in}{1.381815in}}%
\pgfpathlineto{\pgfqpoint{2.103222in}{1.381815in}}%
\pgfpathlineto{\pgfqpoint{2.103222in}{1.384764in}}%
\pgfpathlineto{\pgfqpoint{2.107763in}{1.384764in}}%
\pgfpathlineto{\pgfqpoint{2.107763in}{1.381815in}}%
\pgfpathmoveto{\pgfqpoint{2.107763in}{1.381815in}}%
\pgfpathlineto{\pgfqpoint{2.107763in}{1.381815in}}%
\pgfpathlineto{\pgfqpoint{2.107763in}{1.384764in}}%
\pgfpathlineto{\pgfqpoint{2.112304in}{1.384764in}}%
\pgfpathlineto{\pgfqpoint{2.112304in}{1.381815in}}%
\pgfpathmoveto{\pgfqpoint{2.107763in}{1.384764in}}%
\pgfpathlineto{\pgfqpoint{2.107763in}{1.384764in}}%
\pgfpathlineto{\pgfqpoint{2.107763in}{1.387714in}}%
\pgfpathlineto{\pgfqpoint{2.112304in}{1.387714in}}%
\pgfpathlineto{\pgfqpoint{2.112304in}{1.384764in}}%
\pgfpathmoveto{\pgfqpoint{2.112304in}{1.384764in}}%
\pgfpathlineto{\pgfqpoint{2.112304in}{1.384764in}}%
\pgfpathlineto{\pgfqpoint{2.112304in}{1.387714in}}%
\pgfpathlineto{\pgfqpoint{2.116845in}{1.387714in}}%
\pgfpathlineto{\pgfqpoint{2.116845in}{1.384764in}}%
\pgfpathmoveto{\pgfqpoint{2.112304in}{1.387714in}}%
\pgfpathlineto{\pgfqpoint{2.112304in}{1.387714in}}%
\pgfpathlineto{\pgfqpoint{2.112304in}{1.390663in}}%
\pgfpathlineto{\pgfqpoint{2.116845in}{1.390663in}}%
\pgfpathlineto{\pgfqpoint{2.116845in}{1.387714in}}%
\pgfpathmoveto{\pgfqpoint{2.116845in}{1.387714in}}%
\pgfpathlineto{\pgfqpoint{2.116845in}{1.387714in}}%
\pgfpathlineto{\pgfqpoint{2.116845in}{1.390663in}}%
\pgfpathlineto{\pgfqpoint{2.121386in}{1.390663in}}%
\pgfpathlineto{\pgfqpoint{2.121386in}{1.387714in}}%
\pgfpathmoveto{\pgfqpoint{2.116845in}{1.390663in}}%
\pgfpathlineto{\pgfqpoint{2.116845in}{1.390663in}}%
\pgfpathlineto{\pgfqpoint{2.116845in}{1.393612in}}%
\pgfpathlineto{\pgfqpoint{2.121386in}{1.393612in}}%
\pgfpathlineto{\pgfqpoint{2.121386in}{1.390663in}}%
\pgfpathmoveto{\pgfqpoint{2.121386in}{1.390663in}}%
\pgfpathlineto{\pgfqpoint{2.121386in}{1.390663in}}%
\pgfpathlineto{\pgfqpoint{2.121386in}{1.393612in}}%
\pgfpathlineto{\pgfqpoint{2.125927in}{1.393612in}}%
\pgfpathlineto{\pgfqpoint{2.125927in}{1.390663in}}%
\pgfpathmoveto{\pgfqpoint{2.121386in}{1.393612in}}%
\pgfpathlineto{\pgfqpoint{2.121386in}{1.393612in}}%
\pgfpathlineto{\pgfqpoint{2.121386in}{1.396562in}}%
\pgfpathlineto{\pgfqpoint{2.125927in}{1.396562in}}%
\pgfpathlineto{\pgfqpoint{2.125927in}{1.393612in}}%
\pgfpathmoveto{\pgfqpoint{2.125927in}{1.393612in}}%
\pgfpathlineto{\pgfqpoint{2.125927in}{1.393612in}}%
\pgfpathlineto{\pgfqpoint{2.125927in}{1.396562in}}%
\pgfpathlineto{\pgfqpoint{2.130468in}{1.396562in}}%
\pgfpathlineto{\pgfqpoint{2.130468in}{1.393612in}}%
\pgfpathmoveto{\pgfqpoint{2.125927in}{1.396562in}}%
\pgfpathlineto{\pgfqpoint{2.125927in}{1.396562in}}%
\pgfpathlineto{\pgfqpoint{2.125927in}{1.399511in}}%
\pgfpathlineto{\pgfqpoint{2.130468in}{1.399511in}}%
\pgfpathlineto{\pgfqpoint{2.130468in}{1.396562in}}%
\pgfpathmoveto{\pgfqpoint{2.130468in}{1.396562in}}%
\pgfpathlineto{\pgfqpoint{2.130468in}{1.396562in}}%
\pgfpathlineto{\pgfqpoint{2.130468in}{1.399511in}}%
\pgfpathlineto{\pgfqpoint{2.135009in}{1.399511in}}%
\pgfpathlineto{\pgfqpoint{2.135009in}{1.396562in}}%
\pgfpathmoveto{\pgfqpoint{2.130468in}{1.399511in}}%
\pgfpathlineto{\pgfqpoint{2.130468in}{1.399511in}}%
\pgfpathlineto{\pgfqpoint{2.130468in}{1.402460in}}%
\pgfpathlineto{\pgfqpoint{2.135009in}{1.402460in}}%
\pgfpathlineto{\pgfqpoint{2.135009in}{1.399511in}}%
\pgfpathmoveto{\pgfqpoint{2.135009in}{1.399511in}}%
\pgfpathlineto{\pgfqpoint{2.135009in}{1.399511in}}%
\pgfpathlineto{\pgfqpoint{2.135009in}{1.402460in}}%
\pgfpathlineto{\pgfqpoint{2.139550in}{1.402460in}}%
\pgfpathlineto{\pgfqpoint{2.139550in}{1.399511in}}%
\pgfpathmoveto{\pgfqpoint{2.135009in}{1.402460in}}%
\pgfpathlineto{\pgfqpoint{2.135009in}{1.402460in}}%
\pgfpathlineto{\pgfqpoint{2.135009in}{1.405409in}}%
\pgfpathlineto{\pgfqpoint{2.139550in}{1.405409in}}%
\pgfpathlineto{\pgfqpoint{2.139550in}{1.402460in}}%
\pgfpathmoveto{\pgfqpoint{2.139550in}{1.402460in}}%
\pgfpathlineto{\pgfqpoint{2.139550in}{1.402460in}}%
\pgfpathlineto{\pgfqpoint{2.139550in}{1.405409in}}%
\pgfpathlineto{\pgfqpoint{2.144091in}{1.405409in}}%
\pgfpathlineto{\pgfqpoint{2.144091in}{1.402460in}}%
\pgfpathmoveto{\pgfqpoint{2.144091in}{1.402460in}}%
\pgfpathlineto{\pgfqpoint{2.144091in}{1.402460in}}%
\pgfpathlineto{\pgfqpoint{2.144091in}{1.405409in}}%
\pgfpathlineto{\pgfqpoint{2.148632in}{1.405409in}}%
\pgfpathlineto{\pgfqpoint{2.148632in}{1.402460in}}%
\pgfpathmoveto{\pgfqpoint{2.144091in}{1.405409in}}%
\pgfpathlineto{\pgfqpoint{2.144091in}{1.405409in}}%
\pgfpathlineto{\pgfqpoint{2.144091in}{1.408359in}}%
\pgfpathlineto{\pgfqpoint{2.148632in}{1.408359in}}%
\pgfpathlineto{\pgfqpoint{2.148632in}{1.405409in}}%
\pgfpathmoveto{\pgfqpoint{2.148632in}{1.405409in}}%
\pgfpathlineto{\pgfqpoint{2.148632in}{1.405409in}}%
\pgfpathlineto{\pgfqpoint{2.148632in}{1.408359in}}%
\pgfpathlineto{\pgfqpoint{2.153173in}{1.408359in}}%
\pgfpathlineto{\pgfqpoint{2.153173in}{1.405409in}}%
\pgfpathmoveto{\pgfqpoint{2.148632in}{1.408359in}}%
\pgfpathlineto{\pgfqpoint{2.148632in}{1.408359in}}%
\pgfpathlineto{\pgfqpoint{2.148632in}{1.411308in}}%
\pgfpathlineto{\pgfqpoint{2.153173in}{1.411308in}}%
\pgfpathlineto{\pgfqpoint{2.153173in}{1.408359in}}%
\pgfpathmoveto{\pgfqpoint{2.153173in}{1.408359in}}%
\pgfpathlineto{\pgfqpoint{2.153173in}{1.408359in}}%
\pgfpathlineto{\pgfqpoint{2.153173in}{1.411308in}}%
\pgfpathlineto{\pgfqpoint{2.157714in}{1.411308in}}%
\pgfpathlineto{\pgfqpoint{2.157714in}{1.408359in}}%
\pgfpathmoveto{\pgfqpoint{2.153173in}{1.411308in}}%
\pgfpathlineto{\pgfqpoint{2.153173in}{1.411308in}}%
\pgfpathlineto{\pgfqpoint{2.153173in}{1.414257in}}%
\pgfpathlineto{\pgfqpoint{2.157714in}{1.414257in}}%
\pgfpathlineto{\pgfqpoint{2.157714in}{1.411308in}}%
\pgfpathmoveto{\pgfqpoint{2.157714in}{1.411308in}}%
\pgfpathlineto{\pgfqpoint{2.157714in}{1.411308in}}%
\pgfpathlineto{\pgfqpoint{2.157714in}{1.414257in}}%
\pgfpathlineto{\pgfqpoint{2.162255in}{1.414257in}}%
\pgfpathlineto{\pgfqpoint{2.162255in}{1.411308in}}%
\pgfpathmoveto{\pgfqpoint{2.157714in}{1.414257in}}%
\pgfpathlineto{\pgfqpoint{2.157714in}{1.414257in}}%
\pgfpathlineto{\pgfqpoint{2.157714in}{1.417207in}}%
\pgfpathlineto{\pgfqpoint{2.162255in}{1.417207in}}%
\pgfpathlineto{\pgfqpoint{2.162255in}{1.414257in}}%
\pgfpathmoveto{\pgfqpoint{2.162255in}{1.414257in}}%
\pgfpathlineto{\pgfqpoint{2.162255in}{1.414257in}}%
\pgfpathlineto{\pgfqpoint{2.162255in}{1.417207in}}%
\pgfpathlineto{\pgfqpoint{2.166796in}{1.417207in}}%
\pgfpathlineto{\pgfqpoint{2.166796in}{1.414257in}}%
\pgfpathmoveto{\pgfqpoint{2.162255in}{1.417207in}}%
\pgfpathlineto{\pgfqpoint{2.162255in}{1.417207in}}%
\pgfpathlineto{\pgfqpoint{2.162255in}{1.420156in}}%
\pgfpathlineto{\pgfqpoint{2.166796in}{1.420156in}}%
\pgfpathlineto{\pgfqpoint{2.166796in}{1.417207in}}%
\pgfpathmoveto{\pgfqpoint{2.166796in}{1.417207in}}%
\pgfpathlineto{\pgfqpoint{2.166796in}{1.417207in}}%
\pgfpathlineto{\pgfqpoint{2.166796in}{1.420156in}}%
\pgfpathlineto{\pgfqpoint{2.171337in}{1.420156in}}%
\pgfpathlineto{\pgfqpoint{2.171337in}{1.417207in}}%
\pgfpathmoveto{\pgfqpoint{2.166796in}{1.420156in}}%
\pgfpathlineto{\pgfqpoint{2.166796in}{1.420156in}}%
\pgfpathlineto{\pgfqpoint{2.166796in}{1.423105in}}%
\pgfpathlineto{\pgfqpoint{2.171337in}{1.423105in}}%
\pgfpathlineto{\pgfqpoint{2.171337in}{1.420156in}}%
\pgfpathmoveto{\pgfqpoint{2.171337in}{1.420156in}}%
\pgfpathlineto{\pgfqpoint{2.171337in}{1.420156in}}%
\pgfpathlineto{\pgfqpoint{2.171337in}{1.423105in}}%
\pgfpathlineto{\pgfqpoint{2.175878in}{1.423105in}}%
\pgfpathlineto{\pgfqpoint{2.175878in}{1.420156in}}%
\pgfpathmoveto{\pgfqpoint{2.171337in}{1.423105in}}%
\pgfpathlineto{\pgfqpoint{2.171337in}{1.423105in}}%
\pgfpathlineto{\pgfqpoint{2.171337in}{1.426054in}}%
\pgfpathlineto{\pgfqpoint{2.175878in}{1.426054in}}%
\pgfpathlineto{\pgfqpoint{2.175878in}{1.423105in}}%
\pgfpathmoveto{\pgfqpoint{2.175878in}{1.423105in}}%
\pgfpathlineto{\pgfqpoint{2.175878in}{1.423105in}}%
\pgfpathlineto{\pgfqpoint{2.175878in}{1.426054in}}%
\pgfpathlineto{\pgfqpoint{2.180419in}{1.426054in}}%
\pgfpathlineto{\pgfqpoint{2.180419in}{1.423105in}}%
\pgfpathmoveto{\pgfqpoint{2.175878in}{1.426054in}}%
\pgfpathlineto{\pgfqpoint{2.175878in}{1.426054in}}%
\pgfpathlineto{\pgfqpoint{2.175878in}{1.429004in}}%
\pgfpathlineto{\pgfqpoint{2.180419in}{1.429004in}}%
\pgfpathlineto{\pgfqpoint{2.180419in}{1.426054in}}%
\pgfpathmoveto{\pgfqpoint{2.180419in}{1.426054in}}%
\pgfpathlineto{\pgfqpoint{2.180419in}{1.426054in}}%
\pgfpathlineto{\pgfqpoint{2.180419in}{1.429004in}}%
\pgfpathlineto{\pgfqpoint{2.184960in}{1.429004in}}%
\pgfpathlineto{\pgfqpoint{2.184960in}{1.426054in}}%
\pgfpathmoveto{\pgfqpoint{2.180419in}{1.429004in}}%
\pgfpathlineto{\pgfqpoint{2.180419in}{1.429004in}}%
\pgfpathlineto{\pgfqpoint{2.180419in}{1.431953in}}%
\pgfpathlineto{\pgfqpoint{2.184960in}{1.431953in}}%
\pgfpathlineto{\pgfqpoint{2.184960in}{1.429004in}}%
\pgfpathmoveto{\pgfqpoint{2.184960in}{1.429004in}}%
\pgfpathlineto{\pgfqpoint{2.184960in}{1.429004in}}%
\pgfpathlineto{\pgfqpoint{2.184960in}{1.431953in}}%
\pgfpathlineto{\pgfqpoint{2.189501in}{1.431953in}}%
\pgfpathlineto{\pgfqpoint{2.189501in}{1.429004in}}%
\pgfpathmoveto{\pgfqpoint{2.184960in}{1.431953in}}%
\pgfpathlineto{\pgfqpoint{2.184960in}{1.431953in}}%
\pgfpathlineto{\pgfqpoint{2.184960in}{1.434902in}}%
\pgfpathlineto{\pgfqpoint{2.189501in}{1.434902in}}%
\pgfpathlineto{\pgfqpoint{2.189501in}{1.431953in}}%
\pgfpathmoveto{\pgfqpoint{2.189501in}{1.431953in}}%
\pgfpathlineto{\pgfqpoint{2.189501in}{1.431953in}}%
\pgfpathlineto{\pgfqpoint{2.189501in}{1.434902in}}%
\pgfpathlineto{\pgfqpoint{2.194042in}{1.434902in}}%
\pgfpathlineto{\pgfqpoint{2.194042in}{1.431953in}}%
\pgfpathmoveto{\pgfqpoint{2.189501in}{1.434902in}}%
\pgfpathlineto{\pgfqpoint{2.189501in}{1.434902in}}%
\pgfpathlineto{\pgfqpoint{2.189501in}{1.437852in}}%
\pgfpathlineto{\pgfqpoint{2.194042in}{1.437852in}}%
\pgfpathlineto{\pgfqpoint{2.194042in}{1.434902in}}%
\pgfpathmoveto{\pgfqpoint{2.194042in}{1.434902in}}%
\pgfpathlineto{\pgfqpoint{2.194042in}{1.434902in}}%
\pgfpathlineto{\pgfqpoint{2.194042in}{1.437852in}}%
\pgfpathlineto{\pgfqpoint{2.198583in}{1.437852in}}%
\pgfpathlineto{\pgfqpoint{2.198583in}{1.434902in}}%
\pgfpathmoveto{\pgfqpoint{2.194042in}{1.437852in}}%
\pgfpathlineto{\pgfqpoint{2.194042in}{1.437852in}}%
\pgfpathlineto{\pgfqpoint{2.194042in}{1.440801in}}%
\pgfpathlineto{\pgfqpoint{2.198583in}{1.440801in}}%
\pgfpathlineto{\pgfqpoint{2.198583in}{1.437852in}}%
\pgfpathmoveto{\pgfqpoint{2.198583in}{1.437852in}}%
\pgfpathlineto{\pgfqpoint{2.198583in}{1.437852in}}%
\pgfpathlineto{\pgfqpoint{2.198583in}{1.440801in}}%
\pgfpathlineto{\pgfqpoint{2.203124in}{1.440801in}}%
\pgfpathlineto{\pgfqpoint{2.203124in}{1.437852in}}%
\pgfpathmoveto{\pgfqpoint{2.198583in}{1.440801in}}%
\pgfpathlineto{\pgfqpoint{2.198583in}{1.440801in}}%
\pgfpathlineto{\pgfqpoint{2.198583in}{1.443750in}}%
\pgfpathlineto{\pgfqpoint{2.203124in}{1.443750in}}%
\pgfpathlineto{\pgfqpoint{2.203124in}{1.440801in}}%
\pgfpathmoveto{\pgfqpoint{2.203124in}{1.440801in}}%
\pgfpathlineto{\pgfqpoint{2.203124in}{1.440801in}}%
\pgfpathlineto{\pgfqpoint{2.203124in}{1.443750in}}%
\pgfpathlineto{\pgfqpoint{2.207665in}{1.443750in}}%
\pgfpathlineto{\pgfqpoint{2.207665in}{1.440801in}}%
\pgfpathmoveto{\pgfqpoint{2.203124in}{1.443750in}}%
\pgfpathlineto{\pgfqpoint{2.203124in}{1.443750in}}%
\pgfpathlineto{\pgfqpoint{2.203124in}{1.446699in}}%
\pgfpathlineto{\pgfqpoint{2.207665in}{1.446699in}}%
\pgfpathlineto{\pgfqpoint{2.207665in}{1.443750in}}%
\pgfpathmoveto{\pgfqpoint{2.207665in}{1.443750in}}%
\pgfpathlineto{\pgfqpoint{2.207665in}{1.443750in}}%
\pgfpathlineto{\pgfqpoint{2.207665in}{1.446699in}}%
\pgfpathlineto{\pgfqpoint{2.212206in}{1.446699in}}%
\pgfpathlineto{\pgfqpoint{2.212206in}{1.443750in}}%
\pgfpathmoveto{\pgfqpoint{2.207665in}{1.446699in}}%
\pgfpathlineto{\pgfqpoint{2.207665in}{1.446699in}}%
\pgfpathlineto{\pgfqpoint{2.207665in}{1.449649in}}%
\pgfpathlineto{\pgfqpoint{2.212206in}{1.449649in}}%
\pgfpathlineto{\pgfqpoint{2.212206in}{1.446699in}}%
\pgfpathmoveto{\pgfqpoint{2.212206in}{1.446699in}}%
\pgfpathlineto{\pgfqpoint{2.212206in}{1.446699in}}%
\pgfpathlineto{\pgfqpoint{2.212206in}{1.449649in}}%
\pgfpathlineto{\pgfqpoint{2.216747in}{1.449649in}}%
\pgfpathlineto{\pgfqpoint{2.216747in}{1.446699in}}%
\pgfpathmoveto{\pgfqpoint{2.212206in}{1.449649in}}%
\pgfpathlineto{\pgfqpoint{2.212206in}{1.449649in}}%
\pgfpathlineto{\pgfqpoint{2.212206in}{1.452598in}}%
\pgfpathlineto{\pgfqpoint{2.216747in}{1.452598in}}%
\pgfpathlineto{\pgfqpoint{2.216747in}{1.449649in}}%
\pgfpathmoveto{\pgfqpoint{2.216747in}{1.449649in}}%
\pgfpathlineto{\pgfqpoint{2.216747in}{1.449649in}}%
\pgfpathlineto{\pgfqpoint{2.216747in}{1.452598in}}%
\pgfpathlineto{\pgfqpoint{2.221288in}{1.452598in}}%
\pgfpathlineto{\pgfqpoint{2.221288in}{1.449649in}}%
\pgfpathmoveto{\pgfqpoint{2.216747in}{1.452598in}}%
\pgfpathlineto{\pgfqpoint{2.216747in}{1.452598in}}%
\pgfpathlineto{\pgfqpoint{2.216747in}{1.455547in}}%
\pgfpathlineto{\pgfqpoint{2.221288in}{1.455547in}}%
\pgfpathlineto{\pgfqpoint{2.221288in}{1.452598in}}%
\pgfpathmoveto{\pgfqpoint{2.221288in}{1.452598in}}%
\pgfpathlineto{\pgfqpoint{2.221288in}{1.452598in}}%
\pgfpathlineto{\pgfqpoint{2.221288in}{1.455547in}}%
\pgfpathlineto{\pgfqpoint{2.225829in}{1.455547in}}%
\pgfpathlineto{\pgfqpoint{2.225829in}{1.452598in}}%
\pgfpathmoveto{\pgfqpoint{2.221288in}{1.455547in}}%
\pgfpathlineto{\pgfqpoint{2.221288in}{1.455547in}}%
\pgfpathlineto{\pgfqpoint{2.221288in}{1.458496in}}%
\pgfpathlineto{\pgfqpoint{2.225829in}{1.458496in}}%
\pgfpathlineto{\pgfqpoint{2.225829in}{1.455547in}}%
\pgfpathmoveto{\pgfqpoint{2.225829in}{1.455547in}}%
\pgfpathlineto{\pgfqpoint{2.225829in}{1.455547in}}%
\pgfpathlineto{\pgfqpoint{2.225829in}{1.458496in}}%
\pgfpathlineto{\pgfqpoint{2.230370in}{1.458496in}}%
\pgfpathlineto{\pgfqpoint{2.230370in}{1.455547in}}%
\pgfpathmoveto{\pgfqpoint{2.225829in}{1.458496in}}%
\pgfpathlineto{\pgfqpoint{2.225829in}{1.458496in}}%
\pgfpathlineto{\pgfqpoint{2.225829in}{1.461445in}}%
\pgfpathlineto{\pgfqpoint{2.230370in}{1.461445in}}%
\pgfpathlineto{\pgfqpoint{2.230370in}{1.458496in}}%
\pgfpathmoveto{\pgfqpoint{2.230370in}{1.458496in}}%
\pgfpathlineto{\pgfqpoint{2.230370in}{1.458496in}}%
\pgfpathlineto{\pgfqpoint{2.230370in}{1.461445in}}%
\pgfpathlineto{\pgfqpoint{2.234911in}{1.461445in}}%
\pgfpathlineto{\pgfqpoint{2.234911in}{1.458496in}}%
\pgfpathmoveto{\pgfqpoint{2.230370in}{1.461445in}}%
\pgfpathlineto{\pgfqpoint{2.230370in}{1.461445in}}%
\pgfpathlineto{\pgfqpoint{2.230370in}{1.464395in}}%
\pgfpathlineto{\pgfqpoint{2.234911in}{1.464395in}}%
\pgfpathlineto{\pgfqpoint{2.234911in}{1.461445in}}%
\pgfpathmoveto{\pgfqpoint{2.234911in}{1.461445in}}%
\pgfpathlineto{\pgfqpoint{2.234911in}{1.461445in}}%
\pgfpathlineto{\pgfqpoint{2.234911in}{1.464395in}}%
\pgfpathlineto{\pgfqpoint{2.239452in}{1.464395in}}%
\pgfpathlineto{\pgfqpoint{2.239452in}{1.461445in}}%
\pgfpathmoveto{\pgfqpoint{2.234911in}{1.464395in}}%
\pgfpathlineto{\pgfqpoint{2.234911in}{1.464395in}}%
\pgfpathlineto{\pgfqpoint{2.234911in}{1.467344in}}%
\pgfpathlineto{\pgfqpoint{2.239452in}{1.467344in}}%
\pgfpathlineto{\pgfqpoint{2.239452in}{1.464395in}}%
\pgfpathmoveto{\pgfqpoint{2.239452in}{1.464395in}}%
\pgfpathlineto{\pgfqpoint{2.239452in}{1.464395in}}%
\pgfpathlineto{\pgfqpoint{2.239452in}{1.467344in}}%
\pgfpathlineto{\pgfqpoint{2.243993in}{1.467344in}}%
\pgfpathlineto{\pgfqpoint{2.243993in}{1.464395in}}%
\pgfpathmoveto{\pgfqpoint{2.239452in}{1.467344in}}%
\pgfpathlineto{\pgfqpoint{2.239452in}{1.467344in}}%
\pgfpathlineto{\pgfqpoint{2.239452in}{1.470293in}}%
\pgfpathlineto{\pgfqpoint{2.243993in}{1.470293in}}%
\pgfpathlineto{\pgfqpoint{2.243993in}{1.467344in}}%
\pgfpathmoveto{\pgfqpoint{2.243993in}{1.467344in}}%
\pgfpathlineto{\pgfqpoint{2.243993in}{1.467344in}}%
\pgfpathlineto{\pgfqpoint{2.243993in}{1.470293in}}%
\pgfpathlineto{\pgfqpoint{2.248534in}{1.470293in}}%
\pgfpathlineto{\pgfqpoint{2.248534in}{1.467344in}}%
\pgfpathmoveto{\pgfqpoint{2.243993in}{1.470293in}}%
\pgfpathlineto{\pgfqpoint{2.243993in}{1.470293in}}%
\pgfpathlineto{\pgfqpoint{2.243993in}{1.473242in}}%
\pgfpathlineto{\pgfqpoint{2.248534in}{1.473242in}}%
\pgfpathlineto{\pgfqpoint{2.248534in}{1.470293in}}%
\pgfpathmoveto{\pgfqpoint{2.248534in}{1.470293in}}%
\pgfpathlineto{\pgfqpoint{2.248534in}{1.470293in}}%
\pgfpathlineto{\pgfqpoint{2.248534in}{1.473242in}}%
\pgfpathlineto{\pgfqpoint{2.253075in}{1.473242in}}%
\pgfpathlineto{\pgfqpoint{2.253075in}{1.470293in}}%
\pgfpathmoveto{\pgfqpoint{2.248534in}{1.473242in}}%
\pgfpathlineto{\pgfqpoint{2.248534in}{1.473242in}}%
\pgfpathlineto{\pgfqpoint{2.248534in}{1.476191in}}%
\pgfpathlineto{\pgfqpoint{2.253075in}{1.476191in}}%
\pgfpathlineto{\pgfqpoint{2.253075in}{1.473242in}}%
\pgfpathmoveto{\pgfqpoint{2.253075in}{1.473242in}}%
\pgfpathlineto{\pgfqpoint{2.253075in}{1.473242in}}%
\pgfpathlineto{\pgfqpoint{2.253075in}{1.476191in}}%
\pgfpathlineto{\pgfqpoint{2.257616in}{1.476191in}}%
\pgfpathlineto{\pgfqpoint{2.257616in}{1.473242in}}%
\pgfpathmoveto{\pgfqpoint{2.253075in}{1.476191in}}%
\pgfpathlineto{\pgfqpoint{2.253075in}{1.476191in}}%
\pgfpathlineto{\pgfqpoint{2.253075in}{1.479140in}}%
\pgfpathlineto{\pgfqpoint{2.257616in}{1.479140in}}%
\pgfpathlineto{\pgfqpoint{2.257616in}{1.476191in}}%
\pgfpathmoveto{\pgfqpoint{2.257616in}{1.476191in}}%
\pgfpathlineto{\pgfqpoint{2.257616in}{1.476191in}}%
\pgfpathlineto{\pgfqpoint{2.257616in}{1.479140in}}%
\pgfpathlineto{\pgfqpoint{2.262158in}{1.479140in}}%
\pgfpathlineto{\pgfqpoint{2.262158in}{1.476191in}}%
\pgfpathmoveto{\pgfqpoint{2.257616in}{1.479140in}}%
\pgfpathlineto{\pgfqpoint{2.257616in}{1.479140in}}%
\pgfpathlineto{\pgfqpoint{2.257616in}{1.482090in}}%
\pgfpathlineto{\pgfqpoint{2.262158in}{1.482090in}}%
\pgfpathlineto{\pgfqpoint{2.262158in}{1.479140in}}%
\pgfpathmoveto{\pgfqpoint{2.262158in}{1.479140in}}%
\pgfpathlineto{\pgfqpoint{2.262158in}{1.479140in}}%
\pgfpathlineto{\pgfqpoint{2.262158in}{1.482090in}}%
\pgfpathlineto{\pgfqpoint{2.266699in}{1.482090in}}%
\pgfpathlineto{\pgfqpoint{2.266699in}{1.479140in}}%
\pgfpathmoveto{\pgfqpoint{2.262158in}{1.482090in}}%
\pgfpathlineto{\pgfqpoint{2.262158in}{1.482090in}}%
\pgfpathlineto{\pgfqpoint{2.262158in}{1.485039in}}%
\pgfpathlineto{\pgfqpoint{2.266699in}{1.485039in}}%
\pgfpathlineto{\pgfqpoint{2.266699in}{1.482090in}}%
\pgfpathmoveto{\pgfqpoint{2.266699in}{1.482090in}}%
\pgfpathlineto{\pgfqpoint{2.266699in}{1.482090in}}%
\pgfpathlineto{\pgfqpoint{2.266699in}{1.485039in}}%
\pgfpathlineto{\pgfqpoint{2.271240in}{1.485039in}}%
\pgfpathlineto{\pgfqpoint{2.271240in}{1.482090in}}%
\pgfpathmoveto{\pgfqpoint{2.266699in}{1.485039in}}%
\pgfpathlineto{\pgfqpoint{2.266699in}{1.485039in}}%
\pgfpathlineto{\pgfqpoint{2.266699in}{1.487988in}}%
\pgfpathlineto{\pgfqpoint{2.271240in}{1.487988in}}%
\pgfpathlineto{\pgfqpoint{2.271240in}{1.485039in}}%
\pgfpathmoveto{\pgfqpoint{2.271240in}{1.485039in}}%
\pgfpathlineto{\pgfqpoint{2.271240in}{1.485039in}}%
\pgfpathlineto{\pgfqpoint{2.271240in}{1.487988in}}%
\pgfpathlineto{\pgfqpoint{2.275781in}{1.487988in}}%
\pgfpathlineto{\pgfqpoint{2.275781in}{1.485039in}}%
\pgfpathmoveto{\pgfqpoint{2.271240in}{1.487988in}}%
\pgfpathlineto{\pgfqpoint{2.271240in}{1.487988in}}%
\pgfpathlineto{\pgfqpoint{2.271240in}{1.490937in}}%
\pgfpathlineto{\pgfqpoint{2.275781in}{1.490937in}}%
\pgfpathlineto{\pgfqpoint{2.275781in}{1.487988in}}%
\pgfpathmoveto{\pgfqpoint{2.275781in}{1.487988in}}%
\pgfpathlineto{\pgfqpoint{2.275781in}{1.487988in}}%
\pgfpathlineto{\pgfqpoint{2.275781in}{1.490937in}}%
\pgfpathlineto{\pgfqpoint{2.280322in}{1.490937in}}%
\pgfpathlineto{\pgfqpoint{2.280322in}{1.487988in}}%
\pgfpathmoveto{\pgfqpoint{2.275781in}{1.490937in}}%
\pgfpathlineto{\pgfqpoint{2.275781in}{1.490937in}}%
\pgfpathlineto{\pgfqpoint{2.275781in}{1.493886in}}%
\pgfpathlineto{\pgfqpoint{2.280322in}{1.493886in}}%
\pgfpathlineto{\pgfqpoint{2.280322in}{1.490937in}}%
\pgfpathmoveto{\pgfqpoint{2.275781in}{1.493886in}}%
\pgfpathlineto{\pgfqpoint{2.275781in}{1.493886in}}%
\pgfpathlineto{\pgfqpoint{2.275781in}{1.496836in}}%
\pgfpathlineto{\pgfqpoint{2.280322in}{1.496836in}}%
\pgfpathlineto{\pgfqpoint{2.280322in}{1.493886in}}%
\pgfpathmoveto{\pgfqpoint{2.280322in}{1.493886in}}%
\pgfpathlineto{\pgfqpoint{2.280322in}{1.493886in}}%
\pgfpathlineto{\pgfqpoint{2.280322in}{1.496836in}}%
\pgfpathlineto{\pgfqpoint{2.284863in}{1.496836in}}%
\pgfpathlineto{\pgfqpoint{2.284863in}{1.493886in}}%
\pgfpathmoveto{\pgfqpoint{2.280322in}{1.496836in}}%
\pgfpathlineto{\pgfqpoint{2.280322in}{1.496836in}}%
\pgfpathlineto{\pgfqpoint{2.280322in}{1.499785in}}%
\pgfpathlineto{\pgfqpoint{2.284863in}{1.499785in}}%
\pgfpathlineto{\pgfqpoint{2.284863in}{1.496836in}}%
\pgfpathmoveto{\pgfqpoint{2.284863in}{1.496836in}}%
\pgfpathlineto{\pgfqpoint{2.284863in}{1.496836in}}%
\pgfpathlineto{\pgfqpoint{2.284863in}{1.499785in}}%
\pgfpathlineto{\pgfqpoint{2.289404in}{1.499785in}}%
\pgfpathlineto{\pgfqpoint{2.289404in}{1.496836in}}%
\pgfpathmoveto{\pgfqpoint{2.284863in}{1.499785in}}%
\pgfpathlineto{\pgfqpoint{2.284863in}{1.499785in}}%
\pgfpathlineto{\pgfqpoint{2.284863in}{1.502734in}}%
\pgfpathlineto{\pgfqpoint{2.289404in}{1.502734in}}%
\pgfpathlineto{\pgfqpoint{2.289404in}{1.499785in}}%
\pgfpathmoveto{\pgfqpoint{2.289404in}{1.499785in}}%
\pgfpathlineto{\pgfqpoint{2.289404in}{1.499785in}}%
\pgfpathlineto{\pgfqpoint{2.289404in}{1.502734in}}%
\pgfpathlineto{\pgfqpoint{2.293945in}{1.502734in}}%
\pgfpathlineto{\pgfqpoint{2.293945in}{1.499785in}}%
\pgfpathmoveto{\pgfqpoint{2.289404in}{1.502734in}}%
\pgfpathlineto{\pgfqpoint{2.289404in}{1.502734in}}%
\pgfpathlineto{\pgfqpoint{2.289404in}{1.505683in}}%
\pgfpathlineto{\pgfqpoint{2.293945in}{1.505683in}}%
\pgfpathlineto{\pgfqpoint{2.293945in}{1.502734in}}%
\pgfpathmoveto{\pgfqpoint{2.293945in}{1.502734in}}%
\pgfpathlineto{\pgfqpoint{2.293945in}{1.502734in}}%
\pgfpathlineto{\pgfqpoint{2.293945in}{1.505683in}}%
\pgfpathlineto{\pgfqpoint{2.298486in}{1.505683in}}%
\pgfpathlineto{\pgfqpoint{2.298486in}{1.502734in}}%
\pgfpathmoveto{\pgfqpoint{2.293945in}{1.505683in}}%
\pgfpathlineto{\pgfqpoint{2.293945in}{1.505683in}}%
\pgfpathlineto{\pgfqpoint{2.293945in}{1.508632in}}%
\pgfpathlineto{\pgfqpoint{2.298486in}{1.508632in}}%
\pgfpathlineto{\pgfqpoint{2.298486in}{1.505683in}}%
\pgfpathmoveto{\pgfqpoint{2.298486in}{1.505683in}}%
\pgfpathlineto{\pgfqpoint{2.298486in}{1.505683in}}%
\pgfpathlineto{\pgfqpoint{2.298486in}{1.508632in}}%
\pgfpathlineto{\pgfqpoint{2.303027in}{1.508632in}}%
\pgfpathlineto{\pgfqpoint{2.303027in}{1.505683in}}%
\pgfpathmoveto{\pgfqpoint{2.298486in}{1.508632in}}%
\pgfpathlineto{\pgfqpoint{2.298486in}{1.508632in}}%
\pgfpathlineto{\pgfqpoint{2.298486in}{1.511582in}}%
\pgfpathlineto{\pgfqpoint{2.303027in}{1.511582in}}%
\pgfpathlineto{\pgfqpoint{2.303027in}{1.508632in}}%
\pgfpathmoveto{\pgfqpoint{2.303027in}{1.508632in}}%
\pgfpathlineto{\pgfqpoint{2.303027in}{1.508632in}}%
\pgfpathlineto{\pgfqpoint{2.303027in}{1.511582in}}%
\pgfpathlineto{\pgfqpoint{2.307568in}{1.511582in}}%
\pgfpathlineto{\pgfqpoint{2.307568in}{1.508632in}}%
\pgfpathmoveto{\pgfqpoint{2.303027in}{1.511582in}}%
\pgfpathlineto{\pgfqpoint{2.303027in}{1.511582in}}%
\pgfpathlineto{\pgfqpoint{2.303027in}{1.514531in}}%
\pgfpathlineto{\pgfqpoint{2.307568in}{1.514531in}}%
\pgfpathlineto{\pgfqpoint{2.307568in}{1.511582in}}%
\pgfpathmoveto{\pgfqpoint{2.307568in}{1.511582in}}%
\pgfpathlineto{\pgfqpoint{2.307568in}{1.511582in}}%
\pgfpathlineto{\pgfqpoint{2.307568in}{1.514531in}}%
\pgfpathlineto{\pgfqpoint{2.312109in}{1.514531in}}%
\pgfpathlineto{\pgfqpoint{2.312109in}{1.511582in}}%
\pgfpathmoveto{\pgfqpoint{2.307568in}{1.514531in}}%
\pgfpathlineto{\pgfqpoint{2.307568in}{1.514531in}}%
\pgfpathlineto{\pgfqpoint{2.307568in}{1.517480in}}%
\pgfpathlineto{\pgfqpoint{2.312109in}{1.517480in}}%
\pgfpathlineto{\pgfqpoint{2.312109in}{1.514531in}}%
\pgfpathmoveto{\pgfqpoint{2.312109in}{1.514531in}}%
\pgfpathlineto{\pgfqpoint{2.312109in}{1.514531in}}%
\pgfpathlineto{\pgfqpoint{2.312109in}{1.517480in}}%
\pgfpathlineto{\pgfqpoint{2.316650in}{1.517480in}}%
\pgfpathlineto{\pgfqpoint{2.316650in}{1.514531in}}%
\pgfpathmoveto{\pgfqpoint{2.312109in}{1.517480in}}%
\pgfpathlineto{\pgfqpoint{2.312109in}{1.517480in}}%
\pgfpathlineto{\pgfqpoint{2.312109in}{1.520429in}}%
\pgfpathlineto{\pgfqpoint{2.316650in}{1.520429in}}%
\pgfpathlineto{\pgfqpoint{2.316650in}{1.517480in}}%
\pgfpathmoveto{\pgfqpoint{2.316650in}{1.517480in}}%
\pgfpathlineto{\pgfqpoint{2.316650in}{1.517480in}}%
\pgfpathlineto{\pgfqpoint{2.316650in}{1.520429in}}%
\pgfpathlineto{\pgfqpoint{2.321192in}{1.520429in}}%
\pgfpathlineto{\pgfqpoint{2.321192in}{1.517480in}}%
\pgfpathmoveto{\pgfqpoint{2.316650in}{1.520429in}}%
\pgfpathlineto{\pgfqpoint{2.316650in}{1.520429in}}%
\pgfpathlineto{\pgfqpoint{2.316650in}{1.523378in}}%
\pgfpathlineto{\pgfqpoint{2.321192in}{1.523378in}}%
\pgfpathlineto{\pgfqpoint{2.321192in}{1.520429in}}%
\pgfpathmoveto{\pgfqpoint{2.321192in}{1.520429in}}%
\pgfpathlineto{\pgfqpoint{2.321192in}{1.520429in}}%
\pgfpathlineto{\pgfqpoint{2.321192in}{1.523378in}}%
\pgfpathlineto{\pgfqpoint{2.325733in}{1.523378in}}%
\pgfpathlineto{\pgfqpoint{2.325733in}{1.520429in}}%
\pgfpathmoveto{\pgfqpoint{2.321192in}{1.523378in}}%
\pgfpathlineto{\pgfqpoint{2.321192in}{1.523378in}}%
\pgfpathlineto{\pgfqpoint{2.321192in}{1.526328in}}%
\pgfpathlineto{\pgfqpoint{2.325733in}{1.526328in}}%
\pgfpathlineto{\pgfqpoint{2.325733in}{1.523378in}}%
\pgfpathmoveto{\pgfqpoint{2.325733in}{1.523378in}}%
\pgfpathlineto{\pgfqpoint{2.325733in}{1.523378in}}%
\pgfpathlineto{\pgfqpoint{2.325733in}{1.526328in}}%
\pgfpathlineto{\pgfqpoint{2.330274in}{1.526328in}}%
\pgfpathlineto{\pgfqpoint{2.330274in}{1.523378in}}%
\pgfpathmoveto{\pgfqpoint{2.325733in}{1.526328in}}%
\pgfpathlineto{\pgfqpoint{2.325733in}{1.526328in}}%
\pgfpathlineto{\pgfqpoint{2.325733in}{1.529277in}}%
\pgfpathlineto{\pgfqpoint{2.330274in}{1.529277in}}%
\pgfpathlineto{\pgfqpoint{2.330274in}{1.526328in}}%
\pgfpathmoveto{\pgfqpoint{2.330274in}{1.526328in}}%
\pgfpathlineto{\pgfqpoint{2.330274in}{1.526328in}}%
\pgfpathlineto{\pgfqpoint{2.330274in}{1.529277in}}%
\pgfpathlineto{\pgfqpoint{2.334815in}{1.529277in}}%
\pgfpathlineto{\pgfqpoint{2.334815in}{1.526328in}}%
\pgfpathmoveto{\pgfqpoint{2.330274in}{1.529277in}}%
\pgfpathlineto{\pgfqpoint{2.330274in}{1.529277in}}%
\pgfpathlineto{\pgfqpoint{2.330274in}{1.532226in}}%
\pgfpathlineto{\pgfqpoint{2.334815in}{1.532226in}}%
\pgfpathlineto{\pgfqpoint{2.334815in}{1.529277in}}%
\pgfpathmoveto{\pgfqpoint{2.334815in}{1.529277in}}%
\pgfpathlineto{\pgfqpoint{2.334815in}{1.529277in}}%
\pgfpathlineto{\pgfqpoint{2.334815in}{1.532226in}}%
\pgfpathlineto{\pgfqpoint{2.339356in}{1.532226in}}%
\pgfpathlineto{\pgfqpoint{2.339356in}{1.529277in}}%
\pgfpathmoveto{\pgfqpoint{2.334815in}{1.532226in}}%
\pgfpathlineto{\pgfqpoint{2.334815in}{1.532226in}}%
\pgfpathlineto{\pgfqpoint{2.334815in}{1.535175in}}%
\pgfpathlineto{\pgfqpoint{2.339356in}{1.535175in}}%
\pgfpathlineto{\pgfqpoint{2.339356in}{1.532226in}}%
\pgfpathmoveto{\pgfqpoint{2.339356in}{1.532226in}}%
\pgfpathlineto{\pgfqpoint{2.339356in}{1.532226in}}%
\pgfpathlineto{\pgfqpoint{2.339356in}{1.535175in}}%
\pgfpathlineto{\pgfqpoint{2.343897in}{1.535175in}}%
\pgfpathlineto{\pgfqpoint{2.343897in}{1.532226in}}%
\pgfpathmoveto{\pgfqpoint{2.339356in}{1.535175in}}%
\pgfpathlineto{\pgfqpoint{2.339356in}{1.535175in}}%
\pgfpathlineto{\pgfqpoint{2.339356in}{1.538124in}}%
\pgfpathlineto{\pgfqpoint{2.343897in}{1.538124in}}%
\pgfpathlineto{\pgfqpoint{2.343897in}{1.535175in}}%
\pgfpathmoveto{\pgfqpoint{2.343897in}{1.535175in}}%
\pgfpathlineto{\pgfqpoint{2.343897in}{1.535175in}}%
\pgfpathlineto{\pgfqpoint{2.343897in}{1.538124in}}%
\pgfpathlineto{\pgfqpoint{2.348438in}{1.538124in}}%
\pgfpathlineto{\pgfqpoint{2.348438in}{1.535175in}}%
\pgfpathmoveto{\pgfqpoint{2.343897in}{1.538124in}}%
\pgfpathlineto{\pgfqpoint{2.343897in}{1.538124in}}%
\pgfpathlineto{\pgfqpoint{2.343897in}{1.541073in}}%
\pgfpathlineto{\pgfqpoint{2.348438in}{1.541073in}}%
\pgfpathlineto{\pgfqpoint{2.348438in}{1.538124in}}%
\pgfpathmoveto{\pgfqpoint{2.348438in}{1.538124in}}%
\pgfpathlineto{\pgfqpoint{2.348438in}{1.538124in}}%
\pgfpathlineto{\pgfqpoint{2.348438in}{1.541073in}}%
\pgfpathlineto{\pgfqpoint{2.352979in}{1.541073in}}%
\pgfpathlineto{\pgfqpoint{2.352979in}{1.538124in}}%
\pgfpathmoveto{\pgfqpoint{2.348438in}{1.541073in}}%
\pgfpathlineto{\pgfqpoint{2.348438in}{1.541073in}}%
\pgfpathlineto{\pgfqpoint{2.348438in}{1.544023in}}%
\pgfpathlineto{\pgfqpoint{2.352979in}{1.544023in}}%
\pgfpathlineto{\pgfqpoint{2.352979in}{1.541073in}}%
\pgfpathmoveto{\pgfqpoint{2.352979in}{1.541073in}}%
\pgfpathlineto{\pgfqpoint{2.352979in}{1.541073in}}%
\pgfpathlineto{\pgfqpoint{2.352979in}{1.544023in}}%
\pgfpathlineto{\pgfqpoint{2.357520in}{1.544023in}}%
\pgfpathlineto{\pgfqpoint{2.357520in}{1.541073in}}%
\pgfpathmoveto{\pgfqpoint{2.352979in}{1.544023in}}%
\pgfpathlineto{\pgfqpoint{2.352979in}{1.544023in}}%
\pgfpathlineto{\pgfqpoint{2.352979in}{1.546972in}}%
\pgfpathlineto{\pgfqpoint{2.357520in}{1.546972in}}%
\pgfpathlineto{\pgfqpoint{2.357520in}{1.544023in}}%
\pgfpathmoveto{\pgfqpoint{2.357520in}{1.544023in}}%
\pgfpathlineto{\pgfqpoint{2.357520in}{1.544023in}}%
\pgfpathlineto{\pgfqpoint{2.357520in}{1.546972in}}%
\pgfpathlineto{\pgfqpoint{2.362061in}{1.546972in}}%
\pgfpathlineto{\pgfqpoint{2.362061in}{1.544023in}}%
\pgfpathmoveto{\pgfqpoint{2.357520in}{1.546972in}}%
\pgfpathlineto{\pgfqpoint{2.357520in}{1.546972in}}%
\pgfpathlineto{\pgfqpoint{2.357520in}{1.549921in}}%
\pgfpathlineto{\pgfqpoint{2.362061in}{1.549921in}}%
\pgfpathlineto{\pgfqpoint{2.362061in}{1.546972in}}%
\pgfpathmoveto{\pgfqpoint{2.362061in}{1.546972in}}%
\pgfpathlineto{\pgfqpoint{2.362061in}{1.546972in}}%
\pgfpathlineto{\pgfqpoint{2.362061in}{1.549921in}}%
\pgfpathlineto{\pgfqpoint{2.366602in}{1.549921in}}%
\pgfpathlineto{\pgfqpoint{2.366602in}{1.546972in}}%
\pgfpathmoveto{\pgfqpoint{2.362061in}{1.549921in}}%
\pgfpathlineto{\pgfqpoint{2.362061in}{1.549921in}}%
\pgfpathlineto{\pgfqpoint{2.362061in}{1.552870in}}%
\pgfpathlineto{\pgfqpoint{2.366602in}{1.552870in}}%
\pgfpathlineto{\pgfqpoint{2.366602in}{1.549921in}}%
\pgfpathmoveto{\pgfqpoint{2.366602in}{1.549921in}}%
\pgfpathlineto{\pgfqpoint{2.366602in}{1.549921in}}%
\pgfpathlineto{\pgfqpoint{2.366602in}{1.552870in}}%
\pgfpathlineto{\pgfqpoint{2.371143in}{1.552870in}}%
\pgfpathlineto{\pgfqpoint{2.371143in}{1.549921in}}%
\pgfpathmoveto{\pgfqpoint{2.366602in}{1.552870in}}%
\pgfpathlineto{\pgfqpoint{2.366602in}{1.552870in}}%
\pgfpathlineto{\pgfqpoint{2.366602in}{1.555819in}}%
\pgfpathlineto{\pgfqpoint{2.371143in}{1.555819in}}%
\pgfpathlineto{\pgfqpoint{2.371143in}{1.552870in}}%
\pgfpathmoveto{\pgfqpoint{2.371143in}{1.552870in}}%
\pgfpathlineto{\pgfqpoint{2.371143in}{1.552870in}}%
\pgfpathlineto{\pgfqpoint{2.371143in}{1.555819in}}%
\pgfpathlineto{\pgfqpoint{2.375684in}{1.555819in}}%
\pgfpathlineto{\pgfqpoint{2.375684in}{1.552870in}}%
\pgfpathmoveto{\pgfqpoint{2.371143in}{1.555819in}}%
\pgfpathlineto{\pgfqpoint{2.371143in}{1.555819in}}%
\pgfpathlineto{\pgfqpoint{2.371143in}{1.558768in}}%
\pgfpathlineto{\pgfqpoint{2.375684in}{1.558768in}}%
\pgfpathlineto{\pgfqpoint{2.375684in}{1.555819in}}%
\pgfpathmoveto{\pgfqpoint{2.375684in}{1.555819in}}%
\pgfpathlineto{\pgfqpoint{2.375684in}{1.555819in}}%
\pgfpathlineto{\pgfqpoint{2.375684in}{1.558768in}}%
\pgfpathlineto{\pgfqpoint{2.380225in}{1.558768in}}%
\pgfpathlineto{\pgfqpoint{2.380225in}{1.555819in}}%
\pgfpathmoveto{\pgfqpoint{2.375684in}{1.558768in}}%
\pgfpathlineto{\pgfqpoint{2.375684in}{1.558768in}}%
\pgfpathlineto{\pgfqpoint{2.375684in}{1.561718in}}%
\pgfpathlineto{\pgfqpoint{2.380225in}{1.561718in}}%
\pgfpathlineto{\pgfqpoint{2.380225in}{1.558768in}}%
\pgfpathmoveto{\pgfqpoint{2.380225in}{1.558768in}}%
\pgfpathlineto{\pgfqpoint{2.380225in}{1.558768in}}%
\pgfpathlineto{\pgfqpoint{2.380225in}{1.561718in}}%
\pgfpathlineto{\pgfqpoint{2.384766in}{1.561718in}}%
\pgfpathlineto{\pgfqpoint{2.384766in}{1.558768in}}%
\pgfpathmoveto{\pgfqpoint{2.380225in}{1.561718in}}%
\pgfpathlineto{\pgfqpoint{2.380225in}{1.561718in}}%
\pgfpathlineto{\pgfqpoint{2.380225in}{1.564667in}}%
\pgfpathlineto{\pgfqpoint{2.384766in}{1.564667in}}%
\pgfpathlineto{\pgfqpoint{2.384766in}{1.561718in}}%
\pgfpathmoveto{\pgfqpoint{2.384766in}{1.561718in}}%
\pgfpathlineto{\pgfqpoint{2.384766in}{1.561718in}}%
\pgfpathlineto{\pgfqpoint{2.384766in}{1.564667in}}%
\pgfpathlineto{\pgfqpoint{2.389307in}{1.564667in}}%
\pgfpathlineto{\pgfqpoint{2.389307in}{1.561718in}}%
\pgfpathmoveto{\pgfqpoint{2.384766in}{1.564667in}}%
\pgfpathlineto{\pgfqpoint{2.384766in}{1.564667in}}%
\pgfpathlineto{\pgfqpoint{2.384766in}{1.567616in}}%
\pgfpathlineto{\pgfqpoint{2.389307in}{1.567616in}}%
\pgfpathlineto{\pgfqpoint{2.389307in}{1.564667in}}%
\pgfpathmoveto{\pgfqpoint{2.389307in}{1.564667in}}%
\pgfpathlineto{\pgfqpoint{2.389307in}{1.564667in}}%
\pgfpathlineto{\pgfqpoint{2.389307in}{1.567616in}}%
\pgfpathlineto{\pgfqpoint{2.393848in}{1.567616in}}%
\pgfpathlineto{\pgfqpoint{2.393848in}{1.564667in}}%
\pgfpathmoveto{\pgfqpoint{2.389307in}{1.567616in}}%
\pgfpathlineto{\pgfqpoint{2.389307in}{1.567616in}}%
\pgfpathlineto{\pgfqpoint{2.389307in}{1.570565in}}%
\pgfpathlineto{\pgfqpoint{2.393848in}{1.570565in}}%
\pgfpathlineto{\pgfqpoint{2.393848in}{1.567616in}}%
\pgfpathmoveto{\pgfqpoint{2.393848in}{1.567616in}}%
\pgfpathlineto{\pgfqpoint{2.393848in}{1.567616in}}%
\pgfpathlineto{\pgfqpoint{2.393848in}{1.570565in}}%
\pgfpathlineto{\pgfqpoint{2.398389in}{1.570565in}}%
\pgfpathlineto{\pgfqpoint{2.398389in}{1.567616in}}%
\pgfpathmoveto{\pgfqpoint{2.393848in}{1.570565in}}%
\pgfpathlineto{\pgfqpoint{2.393848in}{1.570565in}}%
\pgfpathlineto{\pgfqpoint{2.393848in}{1.573514in}}%
\pgfpathlineto{\pgfqpoint{2.398389in}{1.573514in}}%
\pgfpathlineto{\pgfqpoint{2.398389in}{1.570565in}}%
\pgfpathmoveto{\pgfqpoint{2.398389in}{1.570565in}}%
\pgfpathlineto{\pgfqpoint{2.398389in}{1.570565in}}%
\pgfpathlineto{\pgfqpoint{2.398389in}{1.573514in}}%
\pgfpathlineto{\pgfqpoint{2.402930in}{1.573514in}}%
\pgfpathlineto{\pgfqpoint{2.402930in}{1.570565in}}%
\pgfpathmoveto{\pgfqpoint{2.398389in}{1.573514in}}%
\pgfpathlineto{\pgfqpoint{2.398389in}{1.573514in}}%
\pgfpathlineto{\pgfqpoint{2.398389in}{1.576463in}}%
\pgfpathlineto{\pgfqpoint{2.402930in}{1.576463in}}%
\pgfpathlineto{\pgfqpoint{2.402930in}{1.573514in}}%
\pgfpathmoveto{\pgfqpoint{2.402930in}{1.573514in}}%
\pgfpathlineto{\pgfqpoint{2.402930in}{1.573514in}}%
\pgfpathlineto{\pgfqpoint{2.402930in}{1.576463in}}%
\pgfpathlineto{\pgfqpoint{2.407471in}{1.576463in}}%
\pgfpathlineto{\pgfqpoint{2.407471in}{1.573514in}}%
\pgfpathmoveto{\pgfqpoint{2.402930in}{1.576463in}}%
\pgfpathlineto{\pgfqpoint{2.402930in}{1.576463in}}%
\pgfpathlineto{\pgfqpoint{2.402930in}{1.579412in}}%
\pgfpathlineto{\pgfqpoint{2.407471in}{1.579412in}}%
\pgfpathlineto{\pgfqpoint{2.407471in}{1.576463in}}%
\pgfpathmoveto{\pgfqpoint{2.407471in}{1.576463in}}%
\pgfpathlineto{\pgfqpoint{2.407471in}{1.576463in}}%
\pgfpathlineto{\pgfqpoint{2.407471in}{1.579412in}}%
\pgfpathlineto{\pgfqpoint{2.412012in}{1.579412in}}%
\pgfpathlineto{\pgfqpoint{2.412012in}{1.576463in}}%
\pgfpathmoveto{\pgfqpoint{2.407471in}{1.579412in}}%
\pgfpathlineto{\pgfqpoint{2.407471in}{1.579412in}}%
\pgfpathlineto{\pgfqpoint{2.407471in}{1.582362in}}%
\pgfpathlineto{\pgfqpoint{2.412012in}{1.582362in}}%
\pgfpathlineto{\pgfqpoint{2.412012in}{1.579412in}}%
\pgfpathmoveto{\pgfqpoint{2.412012in}{1.579412in}}%
\pgfpathlineto{\pgfqpoint{2.412012in}{1.579412in}}%
\pgfpathlineto{\pgfqpoint{2.412012in}{1.582362in}}%
\pgfpathlineto{\pgfqpoint{2.416553in}{1.582362in}}%
\pgfpathlineto{\pgfqpoint{2.416553in}{1.579412in}}%
\pgfpathmoveto{\pgfqpoint{2.412012in}{1.582362in}}%
\pgfpathlineto{\pgfqpoint{2.412012in}{1.582362in}}%
\pgfpathlineto{\pgfqpoint{2.412012in}{1.585311in}}%
\pgfpathlineto{\pgfqpoint{2.416553in}{1.585311in}}%
\pgfpathlineto{\pgfqpoint{2.416553in}{1.582362in}}%
\pgfpathmoveto{\pgfqpoint{2.416553in}{1.582362in}}%
\pgfpathlineto{\pgfqpoint{2.416553in}{1.582362in}}%
\pgfpathlineto{\pgfqpoint{2.416553in}{1.585311in}}%
\pgfpathlineto{\pgfqpoint{2.421094in}{1.585311in}}%
\pgfpathlineto{\pgfqpoint{2.421094in}{1.582362in}}%
\pgfpathmoveto{\pgfqpoint{2.416553in}{1.585311in}}%
\pgfpathlineto{\pgfqpoint{2.416553in}{1.585311in}}%
\pgfpathlineto{\pgfqpoint{2.416553in}{1.588260in}}%
\pgfpathlineto{\pgfqpoint{2.421094in}{1.588260in}}%
\pgfpathlineto{\pgfqpoint{2.421094in}{1.585311in}}%
\pgfpathmoveto{\pgfqpoint{2.421094in}{1.585311in}}%
\pgfpathlineto{\pgfqpoint{2.421094in}{1.585311in}}%
\pgfpathlineto{\pgfqpoint{2.421094in}{1.588260in}}%
\pgfpathlineto{\pgfqpoint{2.425635in}{1.588260in}}%
\pgfpathlineto{\pgfqpoint{2.425635in}{1.585311in}}%
\pgfpathmoveto{\pgfqpoint{2.421094in}{1.588260in}}%
\pgfpathlineto{\pgfqpoint{2.421094in}{1.588260in}}%
\pgfpathlineto{\pgfqpoint{2.421094in}{1.591209in}}%
\pgfpathlineto{\pgfqpoint{2.425635in}{1.591209in}}%
\pgfpathlineto{\pgfqpoint{2.425635in}{1.588260in}}%
\pgfpathmoveto{\pgfqpoint{2.425635in}{1.588260in}}%
\pgfpathlineto{\pgfqpoint{2.425635in}{1.588260in}}%
\pgfpathlineto{\pgfqpoint{2.425635in}{1.591209in}}%
\pgfpathlineto{\pgfqpoint{2.430176in}{1.591209in}}%
\pgfpathlineto{\pgfqpoint{2.430176in}{1.588260in}}%
\pgfpathmoveto{\pgfqpoint{2.425635in}{1.591209in}}%
\pgfpathlineto{\pgfqpoint{2.425635in}{1.591209in}}%
\pgfpathlineto{\pgfqpoint{2.425635in}{1.594158in}}%
\pgfpathlineto{\pgfqpoint{2.430176in}{1.594158in}}%
\pgfpathlineto{\pgfqpoint{2.430176in}{1.591209in}}%
\pgfpathmoveto{\pgfqpoint{2.430176in}{1.591209in}}%
\pgfpathlineto{\pgfqpoint{2.430176in}{1.591209in}}%
\pgfpathlineto{\pgfqpoint{2.430176in}{1.594158in}}%
\pgfpathlineto{\pgfqpoint{2.434717in}{1.594158in}}%
\pgfpathlineto{\pgfqpoint{2.434717in}{1.591209in}}%
\pgfpathmoveto{\pgfqpoint{2.430176in}{1.594158in}}%
\pgfpathlineto{\pgfqpoint{2.430176in}{1.594158in}}%
\pgfpathlineto{\pgfqpoint{2.430176in}{1.597107in}}%
\pgfpathlineto{\pgfqpoint{2.434717in}{1.597107in}}%
\pgfpathlineto{\pgfqpoint{2.434717in}{1.594158in}}%
\pgfpathmoveto{\pgfqpoint{2.434717in}{1.594158in}}%
\pgfpathlineto{\pgfqpoint{2.434717in}{1.594158in}}%
\pgfpathlineto{\pgfqpoint{2.434717in}{1.597107in}}%
\pgfpathlineto{\pgfqpoint{2.439258in}{1.597107in}}%
\pgfpathlineto{\pgfqpoint{2.439258in}{1.594158in}}%
\pgfpathmoveto{\pgfqpoint{2.434717in}{1.597107in}}%
\pgfpathlineto{\pgfqpoint{2.434717in}{1.597107in}}%
\pgfpathlineto{\pgfqpoint{2.434717in}{1.600057in}}%
\pgfpathlineto{\pgfqpoint{2.439258in}{1.600057in}}%
\pgfpathlineto{\pgfqpoint{2.439258in}{1.597107in}}%
\pgfpathmoveto{\pgfqpoint{2.439258in}{1.597107in}}%
\pgfpathlineto{\pgfqpoint{2.439258in}{1.597107in}}%
\pgfpathlineto{\pgfqpoint{2.439258in}{1.600057in}}%
\pgfpathlineto{\pgfqpoint{2.443799in}{1.600057in}}%
\pgfpathlineto{\pgfqpoint{2.443799in}{1.597107in}}%
\pgfpathmoveto{\pgfqpoint{2.439258in}{1.600057in}}%
\pgfpathlineto{\pgfqpoint{2.439258in}{1.600057in}}%
\pgfpathlineto{\pgfqpoint{2.439258in}{1.603006in}}%
\pgfpathlineto{\pgfqpoint{2.443799in}{1.603006in}}%
\pgfpathlineto{\pgfqpoint{2.443799in}{1.600057in}}%
\pgfpathmoveto{\pgfqpoint{2.443799in}{1.600057in}}%
\pgfpathlineto{\pgfqpoint{2.443799in}{1.600057in}}%
\pgfpathlineto{\pgfqpoint{2.443799in}{1.603006in}}%
\pgfpathlineto{\pgfqpoint{2.448340in}{1.603006in}}%
\pgfpathlineto{\pgfqpoint{2.448340in}{1.600057in}}%
\pgfpathmoveto{\pgfqpoint{2.443799in}{1.603006in}}%
\pgfpathlineto{\pgfqpoint{2.443799in}{1.603006in}}%
\pgfpathlineto{\pgfqpoint{2.443799in}{1.605955in}}%
\pgfpathlineto{\pgfqpoint{2.448340in}{1.605955in}}%
\pgfpathlineto{\pgfqpoint{2.448340in}{1.603006in}}%
\pgfpathmoveto{\pgfqpoint{2.448340in}{1.603006in}}%
\pgfpathlineto{\pgfqpoint{2.448340in}{1.603006in}}%
\pgfpathlineto{\pgfqpoint{2.448340in}{1.605955in}}%
\pgfpathlineto{\pgfqpoint{2.452881in}{1.605955in}}%
\pgfpathlineto{\pgfqpoint{2.452881in}{1.603006in}}%
\pgfpathmoveto{\pgfqpoint{2.448340in}{1.605955in}}%
\pgfpathlineto{\pgfqpoint{2.448340in}{1.605955in}}%
\pgfpathlineto{\pgfqpoint{2.448340in}{1.608904in}}%
\pgfpathlineto{\pgfqpoint{2.452881in}{1.608904in}}%
\pgfpathlineto{\pgfqpoint{2.452881in}{1.605955in}}%
\pgfpathmoveto{\pgfqpoint{2.452881in}{1.605955in}}%
\pgfpathlineto{\pgfqpoint{2.452881in}{1.605955in}}%
\pgfpathlineto{\pgfqpoint{2.452881in}{1.608904in}}%
\pgfpathlineto{\pgfqpoint{2.457422in}{1.608904in}}%
\pgfpathlineto{\pgfqpoint{2.457422in}{1.605955in}}%
\pgfpathmoveto{\pgfqpoint{2.452881in}{1.608904in}}%
\pgfpathlineto{\pgfqpoint{2.452881in}{1.608904in}}%
\pgfpathlineto{\pgfqpoint{2.452881in}{1.611853in}}%
\pgfpathlineto{\pgfqpoint{2.457422in}{1.611853in}}%
\pgfpathlineto{\pgfqpoint{2.457422in}{1.608904in}}%
\pgfpathmoveto{\pgfqpoint{2.457422in}{1.608904in}}%
\pgfpathlineto{\pgfqpoint{2.457422in}{1.608904in}}%
\pgfpathlineto{\pgfqpoint{2.457422in}{1.611853in}}%
\pgfpathlineto{\pgfqpoint{2.461962in}{1.611853in}}%
\pgfpathlineto{\pgfqpoint{2.461962in}{1.608904in}}%
\pgfpathmoveto{\pgfqpoint{2.457422in}{1.611853in}}%
\pgfpathlineto{\pgfqpoint{2.457422in}{1.611853in}}%
\pgfpathlineto{\pgfqpoint{2.457422in}{1.614802in}}%
\pgfpathlineto{\pgfqpoint{2.461962in}{1.614802in}}%
\pgfpathlineto{\pgfqpoint{2.461962in}{1.611853in}}%
\pgfpathmoveto{\pgfqpoint{2.461962in}{1.611853in}}%
\pgfpathlineto{\pgfqpoint{2.461962in}{1.611853in}}%
\pgfpathlineto{\pgfqpoint{2.461962in}{1.614802in}}%
\pgfpathlineto{\pgfqpoint{2.466503in}{1.614802in}}%
\pgfpathlineto{\pgfqpoint{2.466503in}{1.611853in}}%
\pgfpathmoveto{\pgfqpoint{2.461962in}{1.614802in}}%
\pgfpathlineto{\pgfqpoint{2.461962in}{1.614802in}}%
\pgfpathlineto{\pgfqpoint{2.461962in}{1.617751in}}%
\pgfpathlineto{\pgfqpoint{2.466503in}{1.617751in}}%
\pgfpathlineto{\pgfqpoint{2.466503in}{1.614802in}}%
\pgfpathmoveto{\pgfqpoint{2.466503in}{1.614802in}}%
\pgfpathlineto{\pgfqpoint{2.466503in}{1.614802in}}%
\pgfpathlineto{\pgfqpoint{2.466503in}{1.617751in}}%
\pgfpathlineto{\pgfqpoint{2.471044in}{1.617751in}}%
\pgfpathlineto{\pgfqpoint{2.471044in}{1.614802in}}%
\pgfpathmoveto{\pgfqpoint{2.466503in}{1.617751in}}%
\pgfpathlineto{\pgfqpoint{2.466503in}{1.617751in}}%
\pgfpathlineto{\pgfqpoint{2.466503in}{1.620701in}}%
\pgfpathlineto{\pgfqpoint{2.471044in}{1.620701in}}%
\pgfpathlineto{\pgfqpoint{2.471044in}{1.617751in}}%
\pgfpathmoveto{\pgfqpoint{2.471044in}{1.617751in}}%
\pgfpathlineto{\pgfqpoint{2.471044in}{1.617751in}}%
\pgfpathlineto{\pgfqpoint{2.471044in}{1.620701in}}%
\pgfpathlineto{\pgfqpoint{2.475585in}{1.620701in}}%
\pgfpathlineto{\pgfqpoint{2.475585in}{1.617751in}}%
\pgfpathmoveto{\pgfqpoint{2.471044in}{1.620701in}}%
\pgfpathlineto{\pgfqpoint{2.471044in}{1.620701in}}%
\pgfpathlineto{\pgfqpoint{2.471044in}{1.623650in}}%
\pgfpathlineto{\pgfqpoint{2.475585in}{1.623650in}}%
\pgfpathlineto{\pgfqpoint{2.475585in}{1.620701in}}%
\pgfpathmoveto{\pgfqpoint{2.475585in}{1.620701in}}%
\pgfpathlineto{\pgfqpoint{2.475585in}{1.620701in}}%
\pgfpathlineto{\pgfqpoint{2.475585in}{1.623650in}}%
\pgfpathlineto{\pgfqpoint{2.480126in}{1.623650in}}%
\pgfpathlineto{\pgfqpoint{2.480126in}{1.620701in}}%
\pgfpathmoveto{\pgfqpoint{2.475585in}{1.623650in}}%
\pgfpathlineto{\pgfqpoint{2.475585in}{1.623650in}}%
\pgfpathlineto{\pgfqpoint{2.475585in}{1.626599in}}%
\pgfpathlineto{\pgfqpoint{2.480126in}{1.626599in}}%
\pgfpathlineto{\pgfqpoint{2.480126in}{1.623650in}}%
\pgfpathmoveto{\pgfqpoint{2.480126in}{1.623650in}}%
\pgfpathlineto{\pgfqpoint{2.480126in}{1.623650in}}%
\pgfpathlineto{\pgfqpoint{2.480126in}{1.626599in}}%
\pgfpathlineto{\pgfqpoint{2.484667in}{1.626599in}}%
\pgfpathlineto{\pgfqpoint{2.484667in}{1.623650in}}%
\pgfpathmoveto{\pgfqpoint{2.480126in}{1.626599in}}%
\pgfpathlineto{\pgfqpoint{2.480126in}{1.626599in}}%
\pgfpathlineto{\pgfqpoint{2.480126in}{1.629548in}}%
\pgfpathlineto{\pgfqpoint{2.484667in}{1.629548in}}%
\pgfpathlineto{\pgfqpoint{2.484667in}{1.626599in}}%
\pgfpathmoveto{\pgfqpoint{2.484667in}{1.626599in}}%
\pgfpathlineto{\pgfqpoint{2.484667in}{1.626599in}}%
\pgfpathlineto{\pgfqpoint{2.484667in}{1.629548in}}%
\pgfpathlineto{\pgfqpoint{2.489208in}{1.629548in}}%
\pgfpathlineto{\pgfqpoint{2.489208in}{1.626599in}}%
\pgfpathmoveto{\pgfqpoint{2.484667in}{1.629548in}}%
\pgfpathlineto{\pgfqpoint{2.484667in}{1.629548in}}%
\pgfpathlineto{\pgfqpoint{2.484667in}{1.632497in}}%
\pgfpathlineto{\pgfqpoint{2.489208in}{1.632497in}}%
\pgfpathlineto{\pgfqpoint{2.489208in}{1.629548in}}%
\pgfpathmoveto{\pgfqpoint{2.489208in}{1.629548in}}%
\pgfpathlineto{\pgfqpoint{2.489208in}{1.629548in}}%
\pgfpathlineto{\pgfqpoint{2.489208in}{1.632497in}}%
\pgfpathlineto{\pgfqpoint{2.493749in}{1.632497in}}%
\pgfpathlineto{\pgfqpoint{2.493749in}{1.629548in}}%
\pgfpathmoveto{\pgfqpoint{2.489208in}{1.632497in}}%
\pgfpathlineto{\pgfqpoint{2.489208in}{1.632497in}}%
\pgfpathlineto{\pgfqpoint{2.489208in}{1.635447in}}%
\pgfpathlineto{\pgfqpoint{2.493749in}{1.635447in}}%
\pgfpathlineto{\pgfqpoint{2.493749in}{1.632497in}}%
\pgfpathmoveto{\pgfqpoint{2.493749in}{1.632497in}}%
\pgfpathlineto{\pgfqpoint{2.493749in}{1.632497in}}%
\pgfpathlineto{\pgfqpoint{2.493749in}{1.635447in}}%
\pgfpathlineto{\pgfqpoint{2.498290in}{1.635447in}}%
\pgfpathlineto{\pgfqpoint{2.498290in}{1.632497in}}%
\pgfpathmoveto{\pgfqpoint{2.493749in}{1.635447in}}%
\pgfpathlineto{\pgfqpoint{2.493749in}{1.635447in}}%
\pgfpathlineto{\pgfqpoint{2.493749in}{1.638396in}}%
\pgfpathlineto{\pgfqpoint{2.498290in}{1.638396in}}%
\pgfpathlineto{\pgfqpoint{2.498290in}{1.635447in}}%
\pgfpathmoveto{\pgfqpoint{2.498290in}{1.635447in}}%
\pgfpathlineto{\pgfqpoint{2.498290in}{1.635447in}}%
\pgfpathlineto{\pgfqpoint{2.498290in}{1.638396in}}%
\pgfpathlineto{\pgfqpoint{2.502831in}{1.638396in}}%
\pgfpathlineto{\pgfqpoint{2.502831in}{1.635447in}}%
\pgfpathmoveto{\pgfqpoint{2.498290in}{1.638396in}}%
\pgfpathlineto{\pgfqpoint{2.498290in}{1.638396in}}%
\pgfpathlineto{\pgfqpoint{2.498290in}{1.641345in}}%
\pgfpathlineto{\pgfqpoint{2.502831in}{1.641345in}}%
\pgfpathlineto{\pgfqpoint{2.502831in}{1.638396in}}%
\pgfpathmoveto{\pgfqpoint{2.502831in}{1.638396in}}%
\pgfpathlineto{\pgfqpoint{2.502831in}{1.638396in}}%
\pgfpathlineto{\pgfqpoint{2.502831in}{1.641345in}}%
\pgfpathlineto{\pgfqpoint{2.507372in}{1.641345in}}%
\pgfpathlineto{\pgfqpoint{2.507372in}{1.638396in}}%
\pgfpathmoveto{\pgfqpoint{2.502831in}{1.641345in}}%
\pgfpathlineto{\pgfqpoint{2.502831in}{1.641345in}}%
\pgfpathlineto{\pgfqpoint{2.502831in}{1.644295in}}%
\pgfpathlineto{\pgfqpoint{2.507372in}{1.644295in}}%
\pgfpathlineto{\pgfqpoint{2.507372in}{1.641345in}}%
\pgfpathmoveto{\pgfqpoint{2.507372in}{1.641345in}}%
\pgfpathlineto{\pgfqpoint{2.507372in}{1.641345in}}%
\pgfpathlineto{\pgfqpoint{2.507372in}{1.644295in}}%
\pgfpathlineto{\pgfqpoint{2.511913in}{1.644295in}}%
\pgfpathlineto{\pgfqpoint{2.511913in}{1.641345in}}%
\pgfpathmoveto{\pgfqpoint{2.507372in}{1.644295in}}%
\pgfpathlineto{\pgfqpoint{2.507372in}{1.644295in}}%
\pgfpathlineto{\pgfqpoint{2.507372in}{1.647244in}}%
\pgfpathlineto{\pgfqpoint{2.511913in}{1.647244in}}%
\pgfpathlineto{\pgfqpoint{2.511913in}{1.644295in}}%
\pgfpathmoveto{\pgfqpoint{2.511913in}{1.644295in}}%
\pgfpathlineto{\pgfqpoint{2.511913in}{1.644295in}}%
\pgfpathlineto{\pgfqpoint{2.511913in}{1.647244in}}%
\pgfpathlineto{\pgfqpoint{2.516454in}{1.647244in}}%
\pgfpathlineto{\pgfqpoint{2.516454in}{1.644295in}}%
\pgfpathmoveto{\pgfqpoint{2.511913in}{1.647244in}}%
\pgfpathlineto{\pgfqpoint{2.511913in}{1.647244in}}%
\pgfpathlineto{\pgfqpoint{2.511913in}{1.650193in}}%
\pgfpathlineto{\pgfqpoint{2.516454in}{1.650193in}}%
\pgfpathlineto{\pgfqpoint{2.516454in}{1.647244in}}%
\pgfpathmoveto{\pgfqpoint{2.516454in}{1.647244in}}%
\pgfpathlineto{\pgfqpoint{2.516454in}{1.647244in}}%
\pgfpathlineto{\pgfqpoint{2.516454in}{1.650193in}}%
\pgfpathlineto{\pgfqpoint{2.520995in}{1.650193in}}%
\pgfpathlineto{\pgfqpoint{2.520995in}{1.647244in}}%
\pgfpathmoveto{\pgfqpoint{2.516454in}{1.650193in}}%
\pgfpathlineto{\pgfqpoint{2.516454in}{1.650193in}}%
\pgfpathlineto{\pgfqpoint{2.516454in}{1.653143in}}%
\pgfpathlineto{\pgfqpoint{2.520995in}{1.653143in}}%
\pgfpathlineto{\pgfqpoint{2.520995in}{1.650193in}}%
\pgfpathmoveto{\pgfqpoint{2.520995in}{1.650193in}}%
\pgfpathlineto{\pgfqpoint{2.520995in}{1.650193in}}%
\pgfpathlineto{\pgfqpoint{2.520995in}{1.653143in}}%
\pgfpathlineto{\pgfqpoint{2.525536in}{1.653143in}}%
\pgfpathlineto{\pgfqpoint{2.525536in}{1.650193in}}%
\pgfpathmoveto{\pgfqpoint{2.520995in}{1.653143in}}%
\pgfpathlineto{\pgfqpoint{2.520995in}{1.653143in}}%
\pgfpathlineto{\pgfqpoint{2.520995in}{1.656092in}}%
\pgfpathlineto{\pgfqpoint{2.525536in}{1.656092in}}%
\pgfpathlineto{\pgfqpoint{2.525536in}{1.653143in}}%
\pgfpathmoveto{\pgfqpoint{2.525536in}{1.653143in}}%
\pgfpathlineto{\pgfqpoint{2.525536in}{1.653143in}}%
\pgfpathlineto{\pgfqpoint{2.525536in}{1.656092in}}%
\pgfpathlineto{\pgfqpoint{2.530077in}{1.656092in}}%
\pgfpathlineto{\pgfqpoint{2.530077in}{1.653143in}}%
\pgfpathmoveto{\pgfqpoint{2.525536in}{1.656092in}}%
\pgfpathlineto{\pgfqpoint{2.525536in}{1.656092in}}%
\pgfpathlineto{\pgfqpoint{2.525536in}{1.659041in}}%
\pgfpathlineto{\pgfqpoint{2.530077in}{1.659041in}}%
\pgfpathlineto{\pgfqpoint{2.530077in}{1.656092in}}%
\pgfpathmoveto{\pgfqpoint{2.530077in}{1.656092in}}%
\pgfpathlineto{\pgfqpoint{2.530077in}{1.656092in}}%
\pgfpathlineto{\pgfqpoint{2.530077in}{1.659041in}}%
\pgfpathlineto{\pgfqpoint{2.534618in}{1.659041in}}%
\pgfpathlineto{\pgfqpoint{2.534618in}{1.656092in}}%
\pgfpathmoveto{\pgfqpoint{2.530077in}{1.659041in}}%
\pgfpathlineto{\pgfqpoint{2.530077in}{1.659041in}}%
\pgfpathlineto{\pgfqpoint{2.530077in}{1.661991in}}%
\pgfpathlineto{\pgfqpoint{2.534618in}{1.661991in}}%
\pgfpathlineto{\pgfqpoint{2.534618in}{1.659041in}}%
\pgfpathmoveto{\pgfqpoint{2.534618in}{1.659041in}}%
\pgfpathlineto{\pgfqpoint{2.534618in}{1.659041in}}%
\pgfpathlineto{\pgfqpoint{2.534618in}{1.661991in}}%
\pgfpathlineto{\pgfqpoint{2.539159in}{1.661991in}}%
\pgfpathlineto{\pgfqpoint{2.539159in}{1.659041in}}%
\pgfpathmoveto{\pgfqpoint{2.534618in}{1.661991in}}%
\pgfpathlineto{\pgfqpoint{2.534618in}{1.661991in}}%
\pgfpathlineto{\pgfqpoint{2.534618in}{1.664940in}}%
\pgfpathlineto{\pgfqpoint{2.539159in}{1.664940in}}%
\pgfpathlineto{\pgfqpoint{2.539159in}{1.661991in}}%
\pgfpathmoveto{\pgfqpoint{2.539159in}{1.661991in}}%
\pgfpathlineto{\pgfqpoint{2.539159in}{1.661991in}}%
\pgfpathlineto{\pgfqpoint{2.539159in}{1.664940in}}%
\pgfpathlineto{\pgfqpoint{2.543700in}{1.664940in}}%
\pgfpathlineto{\pgfqpoint{2.543700in}{1.661991in}}%
\pgfpathmoveto{\pgfqpoint{2.539159in}{1.664940in}}%
\pgfpathlineto{\pgfqpoint{2.539159in}{1.664940in}}%
\pgfpathlineto{\pgfqpoint{2.539159in}{1.667889in}}%
\pgfpathlineto{\pgfqpoint{2.543700in}{1.667889in}}%
\pgfpathlineto{\pgfqpoint{2.543700in}{1.664940in}}%
\pgfpathmoveto{\pgfqpoint{2.543700in}{1.664940in}}%
\pgfpathlineto{\pgfqpoint{2.543700in}{1.664940in}}%
\pgfpathlineto{\pgfqpoint{2.543700in}{1.667889in}}%
\pgfpathlineto{\pgfqpoint{2.548241in}{1.667889in}}%
\pgfpathlineto{\pgfqpoint{2.548241in}{1.664940in}}%
\pgfpathmoveto{\pgfqpoint{2.543700in}{1.667889in}}%
\pgfpathlineto{\pgfqpoint{2.543700in}{1.667889in}}%
\pgfpathlineto{\pgfqpoint{2.543700in}{1.670839in}}%
\pgfpathlineto{\pgfqpoint{2.548241in}{1.670839in}}%
\pgfpathlineto{\pgfqpoint{2.548241in}{1.667889in}}%
\pgfpathmoveto{\pgfqpoint{2.548241in}{1.667889in}}%
\pgfpathlineto{\pgfqpoint{2.548241in}{1.667889in}}%
\pgfpathlineto{\pgfqpoint{2.548241in}{1.670839in}}%
\pgfpathlineto{\pgfqpoint{2.552782in}{1.670839in}}%
\pgfpathlineto{\pgfqpoint{2.552782in}{1.667889in}}%
\pgfpathmoveto{\pgfqpoint{2.548241in}{1.670839in}}%
\pgfpathlineto{\pgfqpoint{2.548241in}{1.670839in}}%
\pgfpathlineto{\pgfqpoint{2.548241in}{1.673788in}}%
\pgfpathlineto{\pgfqpoint{2.552782in}{1.673788in}}%
\pgfpathlineto{\pgfqpoint{2.552782in}{1.670839in}}%
\pgfpathmoveto{\pgfqpoint{2.552782in}{1.670839in}}%
\pgfpathlineto{\pgfqpoint{2.552782in}{1.670839in}}%
\pgfpathlineto{\pgfqpoint{2.552782in}{1.673788in}}%
\pgfpathlineto{\pgfqpoint{2.557323in}{1.673788in}}%
\pgfpathlineto{\pgfqpoint{2.557323in}{1.670839in}}%
\pgfpathmoveto{\pgfqpoint{2.552782in}{1.673788in}}%
\pgfpathlineto{\pgfqpoint{2.552782in}{1.673788in}}%
\pgfpathlineto{\pgfqpoint{2.552782in}{1.676737in}}%
\pgfpathlineto{\pgfqpoint{2.557323in}{1.676737in}}%
\pgfpathlineto{\pgfqpoint{2.557323in}{1.673788in}}%
\pgfpathmoveto{\pgfqpoint{2.557323in}{1.673788in}}%
\pgfpathlineto{\pgfqpoint{2.557323in}{1.673788in}}%
\pgfpathlineto{\pgfqpoint{2.557323in}{1.676737in}}%
\pgfpathlineto{\pgfqpoint{2.561864in}{1.676737in}}%
\pgfpathlineto{\pgfqpoint{2.561864in}{1.673788in}}%
\pgfpathmoveto{\pgfqpoint{2.561864in}{1.673788in}}%
\pgfpathlineto{\pgfqpoint{2.561864in}{1.673788in}}%
\pgfpathlineto{\pgfqpoint{2.561864in}{1.676737in}}%
\pgfpathlineto{\pgfqpoint{2.566405in}{1.676737in}}%
\pgfpathlineto{\pgfqpoint{2.566405in}{1.673788in}}%
\pgfpathmoveto{\pgfqpoint{2.561864in}{1.676737in}}%
\pgfpathlineto{\pgfqpoint{2.561864in}{1.676737in}}%
\pgfpathlineto{\pgfqpoint{2.561864in}{1.679687in}}%
\pgfpathlineto{\pgfqpoint{2.566405in}{1.679687in}}%
\pgfpathlineto{\pgfqpoint{2.566405in}{1.676737in}}%
\pgfpathmoveto{\pgfqpoint{2.566405in}{1.676737in}}%
\pgfpathlineto{\pgfqpoint{2.566405in}{1.676737in}}%
\pgfpathlineto{\pgfqpoint{2.566405in}{1.679687in}}%
\pgfpathlineto{\pgfqpoint{2.570946in}{1.679687in}}%
\pgfpathlineto{\pgfqpoint{2.570946in}{1.676737in}}%
\pgfpathmoveto{\pgfqpoint{2.566405in}{1.679687in}}%
\pgfpathlineto{\pgfqpoint{2.566405in}{1.679687in}}%
\pgfpathlineto{\pgfqpoint{2.566405in}{1.682636in}}%
\pgfpathlineto{\pgfqpoint{2.570946in}{1.682636in}}%
\pgfpathlineto{\pgfqpoint{2.570946in}{1.679687in}}%
\pgfpathmoveto{\pgfqpoint{2.570946in}{1.679687in}}%
\pgfpathlineto{\pgfqpoint{2.570946in}{1.679687in}}%
\pgfpathlineto{\pgfqpoint{2.570946in}{1.682636in}}%
\pgfpathlineto{\pgfqpoint{2.575487in}{1.682636in}}%
\pgfpathlineto{\pgfqpoint{2.575487in}{1.679687in}}%
\pgfpathmoveto{\pgfqpoint{2.570946in}{1.682636in}}%
\pgfpathlineto{\pgfqpoint{2.570946in}{1.682636in}}%
\pgfpathlineto{\pgfqpoint{2.570946in}{1.685585in}}%
\pgfpathlineto{\pgfqpoint{2.575487in}{1.685585in}}%
\pgfpathlineto{\pgfqpoint{2.575487in}{1.682636in}}%
\pgfpathmoveto{\pgfqpoint{2.575487in}{1.682636in}}%
\pgfpathlineto{\pgfqpoint{2.575487in}{1.682636in}}%
\pgfpathlineto{\pgfqpoint{2.575487in}{1.685585in}}%
\pgfpathlineto{\pgfqpoint{2.580027in}{1.685585in}}%
\pgfpathlineto{\pgfqpoint{2.580027in}{1.682636in}}%
\pgfpathmoveto{\pgfqpoint{2.575487in}{1.685585in}}%
\pgfpathlineto{\pgfqpoint{2.575487in}{1.685585in}}%
\pgfpathlineto{\pgfqpoint{2.575487in}{1.688535in}}%
\pgfpathlineto{\pgfqpoint{2.580027in}{1.688535in}}%
\pgfpathlineto{\pgfqpoint{2.580027in}{1.685585in}}%
\pgfpathmoveto{\pgfqpoint{2.580027in}{1.685585in}}%
\pgfpathlineto{\pgfqpoint{2.580027in}{1.685585in}}%
\pgfpathlineto{\pgfqpoint{2.580027in}{1.688535in}}%
\pgfpathlineto{\pgfqpoint{2.584568in}{1.688535in}}%
\pgfpathlineto{\pgfqpoint{2.584568in}{1.685585in}}%
\pgfpathmoveto{\pgfqpoint{2.580027in}{1.688535in}}%
\pgfpathlineto{\pgfqpoint{2.580027in}{1.688535in}}%
\pgfpathlineto{\pgfqpoint{2.580027in}{1.691484in}}%
\pgfpathlineto{\pgfqpoint{2.584568in}{1.691484in}}%
\pgfpathlineto{\pgfqpoint{2.584568in}{1.688535in}}%
\pgfpathmoveto{\pgfqpoint{2.584568in}{1.688535in}}%
\pgfpathlineto{\pgfqpoint{2.584568in}{1.688535in}}%
\pgfpathlineto{\pgfqpoint{2.584568in}{1.691484in}}%
\pgfpathlineto{\pgfqpoint{2.589109in}{1.691484in}}%
\pgfpathlineto{\pgfqpoint{2.589109in}{1.688535in}}%
\pgfpathmoveto{\pgfqpoint{2.584568in}{1.691484in}}%
\pgfpathlineto{\pgfqpoint{2.584568in}{1.691484in}}%
\pgfpathlineto{\pgfqpoint{2.584568in}{1.694433in}}%
\pgfpathlineto{\pgfqpoint{2.589109in}{1.694433in}}%
\pgfpathlineto{\pgfqpoint{2.589109in}{1.691484in}}%
\pgfpathmoveto{\pgfqpoint{2.589109in}{1.691484in}}%
\pgfpathlineto{\pgfqpoint{2.589109in}{1.691484in}}%
\pgfpathlineto{\pgfqpoint{2.589109in}{1.694433in}}%
\pgfpathlineto{\pgfqpoint{2.593650in}{1.694433in}}%
\pgfpathlineto{\pgfqpoint{2.593650in}{1.691484in}}%
\pgfpathmoveto{\pgfqpoint{2.589109in}{1.694433in}}%
\pgfpathlineto{\pgfqpoint{2.589109in}{1.694433in}}%
\pgfpathlineto{\pgfqpoint{2.589109in}{1.697383in}}%
\pgfpathlineto{\pgfqpoint{2.593650in}{1.697383in}}%
\pgfpathlineto{\pgfqpoint{2.593650in}{1.694433in}}%
\pgfpathmoveto{\pgfqpoint{2.593650in}{1.694433in}}%
\pgfpathlineto{\pgfqpoint{2.593650in}{1.694433in}}%
\pgfpathlineto{\pgfqpoint{2.593650in}{1.697383in}}%
\pgfpathlineto{\pgfqpoint{2.598191in}{1.697383in}}%
\pgfpathlineto{\pgfqpoint{2.598191in}{1.694433in}}%
\pgfpathmoveto{\pgfqpoint{2.593650in}{1.697383in}}%
\pgfpathlineto{\pgfqpoint{2.593650in}{1.697383in}}%
\pgfpathlineto{\pgfqpoint{2.593650in}{1.700332in}}%
\pgfpathlineto{\pgfqpoint{2.598191in}{1.700332in}}%
\pgfpathlineto{\pgfqpoint{2.598191in}{1.697383in}}%
\pgfpathmoveto{\pgfqpoint{2.598191in}{1.697383in}}%
\pgfpathlineto{\pgfqpoint{2.598191in}{1.697383in}}%
\pgfpathlineto{\pgfqpoint{2.598191in}{1.700332in}}%
\pgfpathlineto{\pgfqpoint{2.602732in}{1.700332in}}%
\pgfpathlineto{\pgfqpoint{2.602732in}{1.697383in}}%
\pgfpathmoveto{\pgfqpoint{2.598191in}{1.700332in}}%
\pgfpathlineto{\pgfqpoint{2.598191in}{1.700332in}}%
\pgfpathlineto{\pgfqpoint{2.598191in}{1.703281in}}%
\pgfpathlineto{\pgfqpoint{2.602732in}{1.703281in}}%
\pgfpathlineto{\pgfqpoint{2.602732in}{1.700332in}}%
\pgfpathmoveto{\pgfqpoint{2.602732in}{1.700332in}}%
\pgfpathlineto{\pgfqpoint{2.602732in}{1.700332in}}%
\pgfpathlineto{\pgfqpoint{2.602732in}{1.703281in}}%
\pgfpathlineto{\pgfqpoint{2.607273in}{1.703281in}}%
\pgfpathlineto{\pgfqpoint{2.607273in}{1.700332in}}%
\pgfpathmoveto{\pgfqpoint{2.602732in}{1.703281in}}%
\pgfpathlineto{\pgfqpoint{2.602732in}{1.703281in}}%
\pgfpathlineto{\pgfqpoint{2.602732in}{1.706231in}}%
\pgfpathlineto{\pgfqpoint{2.607273in}{1.706231in}}%
\pgfpathlineto{\pgfqpoint{2.607273in}{1.703281in}}%
\pgfpathmoveto{\pgfqpoint{2.607273in}{1.703281in}}%
\pgfpathlineto{\pgfqpoint{2.607273in}{1.703281in}}%
\pgfpathlineto{\pgfqpoint{2.607273in}{1.706231in}}%
\pgfpathlineto{\pgfqpoint{2.611814in}{1.706231in}}%
\pgfpathlineto{\pgfqpoint{2.611814in}{1.703281in}}%
\pgfpathmoveto{\pgfqpoint{2.607273in}{1.706231in}}%
\pgfpathlineto{\pgfqpoint{2.607273in}{1.706231in}}%
\pgfpathlineto{\pgfqpoint{2.607273in}{1.709180in}}%
\pgfpathlineto{\pgfqpoint{2.611814in}{1.709180in}}%
\pgfpathlineto{\pgfqpoint{2.611814in}{1.706231in}}%
\pgfpathmoveto{\pgfqpoint{2.611814in}{1.706231in}}%
\pgfpathlineto{\pgfqpoint{2.611814in}{1.706231in}}%
\pgfpathlineto{\pgfqpoint{2.611814in}{1.709180in}}%
\pgfpathlineto{\pgfqpoint{2.616355in}{1.709180in}}%
\pgfpathlineto{\pgfqpoint{2.616355in}{1.706231in}}%
\pgfpathmoveto{\pgfqpoint{2.611814in}{1.709180in}}%
\pgfpathlineto{\pgfqpoint{2.611814in}{1.709180in}}%
\pgfpathlineto{\pgfqpoint{2.611814in}{1.712129in}}%
\pgfpathlineto{\pgfqpoint{2.616355in}{1.712129in}}%
\pgfpathlineto{\pgfqpoint{2.616355in}{1.709180in}}%
\pgfpathmoveto{\pgfqpoint{2.616355in}{1.709180in}}%
\pgfpathlineto{\pgfqpoint{2.616355in}{1.709180in}}%
\pgfpathlineto{\pgfqpoint{2.616355in}{1.712129in}}%
\pgfpathlineto{\pgfqpoint{2.620896in}{1.712129in}}%
\pgfpathlineto{\pgfqpoint{2.620896in}{1.709180in}}%
\pgfpathmoveto{\pgfqpoint{2.616355in}{1.712129in}}%
\pgfpathlineto{\pgfqpoint{2.616355in}{1.712129in}}%
\pgfpathlineto{\pgfqpoint{2.616355in}{1.715079in}}%
\pgfpathlineto{\pgfqpoint{2.620896in}{1.715079in}}%
\pgfpathlineto{\pgfqpoint{2.620896in}{1.712129in}}%
\pgfpathmoveto{\pgfqpoint{2.620896in}{1.712129in}}%
\pgfpathlineto{\pgfqpoint{2.620896in}{1.712129in}}%
\pgfpathlineto{\pgfqpoint{2.620896in}{1.715079in}}%
\pgfpathlineto{\pgfqpoint{2.625437in}{1.715079in}}%
\pgfpathlineto{\pgfqpoint{2.625437in}{1.712129in}}%
\pgfpathmoveto{\pgfqpoint{2.620896in}{1.715079in}}%
\pgfpathlineto{\pgfqpoint{2.620896in}{1.715079in}}%
\pgfpathlineto{\pgfqpoint{2.620896in}{1.718028in}}%
\pgfpathlineto{\pgfqpoint{2.625437in}{1.718028in}}%
\pgfpathlineto{\pgfqpoint{2.625437in}{1.715079in}}%
\pgfpathmoveto{\pgfqpoint{2.625437in}{1.715079in}}%
\pgfpathlineto{\pgfqpoint{2.625437in}{1.715079in}}%
\pgfpathlineto{\pgfqpoint{2.625437in}{1.718028in}}%
\pgfpathlineto{\pgfqpoint{2.629978in}{1.718028in}}%
\pgfpathlineto{\pgfqpoint{2.629978in}{1.715079in}}%
\pgfpathmoveto{\pgfqpoint{2.625437in}{1.718028in}}%
\pgfpathlineto{\pgfqpoint{2.625437in}{1.718028in}}%
\pgfpathlineto{\pgfqpoint{2.625437in}{1.720977in}}%
\pgfpathlineto{\pgfqpoint{2.629978in}{1.720977in}}%
\pgfpathlineto{\pgfqpoint{2.629978in}{1.718028in}}%
\pgfpathmoveto{\pgfqpoint{2.629978in}{1.718028in}}%
\pgfpathlineto{\pgfqpoint{2.629978in}{1.718028in}}%
\pgfpathlineto{\pgfqpoint{2.629978in}{1.720977in}}%
\pgfpathlineto{\pgfqpoint{2.634519in}{1.720977in}}%
\pgfpathlineto{\pgfqpoint{2.634519in}{1.718028in}}%
\pgfpathmoveto{\pgfqpoint{2.629978in}{1.720977in}}%
\pgfpathlineto{\pgfqpoint{2.629978in}{1.720977in}}%
\pgfpathlineto{\pgfqpoint{2.629978in}{1.723927in}}%
\pgfpathlineto{\pgfqpoint{2.634519in}{1.723927in}}%
\pgfpathlineto{\pgfqpoint{2.634519in}{1.720977in}}%
\pgfpathmoveto{\pgfqpoint{2.634519in}{1.720977in}}%
\pgfpathlineto{\pgfqpoint{2.634519in}{1.720977in}}%
\pgfpathlineto{\pgfqpoint{2.634519in}{1.723927in}}%
\pgfpathlineto{\pgfqpoint{2.639060in}{1.723927in}}%
\pgfpathlineto{\pgfqpoint{2.639060in}{1.720977in}}%
\pgfpathmoveto{\pgfqpoint{2.634519in}{1.723927in}}%
\pgfpathlineto{\pgfqpoint{2.634519in}{1.723927in}}%
\pgfpathlineto{\pgfqpoint{2.634519in}{1.726876in}}%
\pgfpathlineto{\pgfqpoint{2.639060in}{1.726876in}}%
\pgfpathlineto{\pgfqpoint{2.639060in}{1.723927in}}%
\pgfpathmoveto{\pgfqpoint{2.639060in}{1.723927in}}%
\pgfpathlineto{\pgfqpoint{2.639060in}{1.723927in}}%
\pgfpathlineto{\pgfqpoint{2.639060in}{1.726876in}}%
\pgfpathlineto{\pgfqpoint{2.643601in}{1.726876in}}%
\pgfpathlineto{\pgfqpoint{2.643601in}{1.723927in}}%
\pgfpathmoveto{\pgfqpoint{2.639060in}{1.726876in}}%
\pgfpathlineto{\pgfqpoint{2.639060in}{1.726876in}}%
\pgfpathlineto{\pgfqpoint{2.639060in}{1.729825in}}%
\pgfpathlineto{\pgfqpoint{2.643601in}{1.729825in}}%
\pgfpathlineto{\pgfqpoint{2.643601in}{1.726876in}}%
\pgfpathmoveto{\pgfqpoint{2.643601in}{1.726876in}}%
\pgfpathlineto{\pgfqpoint{2.643601in}{1.726876in}}%
\pgfpathlineto{\pgfqpoint{2.643601in}{1.729825in}}%
\pgfpathlineto{\pgfqpoint{2.648142in}{1.729825in}}%
\pgfpathlineto{\pgfqpoint{2.648142in}{1.726876in}}%
\pgfpathmoveto{\pgfqpoint{2.643601in}{1.729825in}}%
\pgfpathlineto{\pgfqpoint{2.643601in}{1.729825in}}%
\pgfpathlineto{\pgfqpoint{2.643601in}{1.732774in}}%
\pgfpathlineto{\pgfqpoint{2.648142in}{1.732774in}}%
\pgfpathlineto{\pgfqpoint{2.648142in}{1.729825in}}%
\pgfpathmoveto{\pgfqpoint{2.648142in}{1.729825in}}%
\pgfpathlineto{\pgfqpoint{2.648142in}{1.729825in}}%
\pgfpathlineto{\pgfqpoint{2.648142in}{1.732774in}}%
\pgfpathlineto{\pgfqpoint{2.652683in}{1.732774in}}%
\pgfpathlineto{\pgfqpoint{2.652683in}{1.729825in}}%
\pgfpathmoveto{\pgfqpoint{2.648142in}{1.732774in}}%
\pgfpathlineto{\pgfqpoint{2.648142in}{1.732774in}}%
\pgfpathlineto{\pgfqpoint{2.648142in}{1.735723in}}%
\pgfpathlineto{\pgfqpoint{2.652683in}{1.735723in}}%
\pgfpathlineto{\pgfqpoint{2.652683in}{1.732774in}}%
\pgfpathmoveto{\pgfqpoint{2.652683in}{1.732774in}}%
\pgfpathlineto{\pgfqpoint{2.652683in}{1.732774in}}%
\pgfpathlineto{\pgfqpoint{2.652683in}{1.735723in}}%
\pgfpathlineto{\pgfqpoint{2.657224in}{1.735723in}}%
\pgfpathlineto{\pgfqpoint{2.657224in}{1.732774in}}%
\pgfpathmoveto{\pgfqpoint{2.652683in}{1.735723in}}%
\pgfpathlineto{\pgfqpoint{2.652683in}{1.735723in}}%
\pgfpathlineto{\pgfqpoint{2.652683in}{1.738673in}}%
\pgfpathlineto{\pgfqpoint{2.657224in}{1.738673in}}%
\pgfpathlineto{\pgfqpoint{2.657224in}{1.735723in}}%
\pgfpathmoveto{\pgfqpoint{2.657224in}{1.735723in}}%
\pgfpathlineto{\pgfqpoint{2.657224in}{1.735723in}}%
\pgfpathlineto{\pgfqpoint{2.657224in}{1.738673in}}%
\pgfpathlineto{\pgfqpoint{2.661765in}{1.738673in}}%
\pgfpathlineto{\pgfqpoint{2.661765in}{1.735723in}}%
\pgfpathmoveto{\pgfqpoint{2.657224in}{1.738673in}}%
\pgfpathlineto{\pgfqpoint{2.657224in}{1.738673in}}%
\pgfpathlineto{\pgfqpoint{2.657224in}{1.741622in}}%
\pgfpathlineto{\pgfqpoint{2.661765in}{1.741622in}}%
\pgfpathlineto{\pgfqpoint{2.661765in}{1.738673in}}%
\pgfpathmoveto{\pgfqpoint{2.661765in}{1.738673in}}%
\pgfpathlineto{\pgfqpoint{2.661765in}{1.738673in}}%
\pgfpathlineto{\pgfqpoint{2.661765in}{1.741622in}}%
\pgfpathlineto{\pgfqpoint{2.666306in}{1.741622in}}%
\pgfpathlineto{\pgfqpoint{2.666306in}{1.738673in}}%
\pgfpathmoveto{\pgfqpoint{2.661765in}{1.741622in}}%
\pgfpathlineto{\pgfqpoint{2.661765in}{1.741622in}}%
\pgfpathlineto{\pgfqpoint{2.661765in}{1.744571in}}%
\pgfpathlineto{\pgfqpoint{2.666306in}{1.744571in}}%
\pgfpathlineto{\pgfqpoint{2.666306in}{1.741622in}}%
\pgfpathmoveto{\pgfqpoint{2.666306in}{1.741622in}}%
\pgfpathlineto{\pgfqpoint{2.666306in}{1.741622in}}%
\pgfpathlineto{\pgfqpoint{2.666306in}{1.744571in}}%
\pgfpathlineto{\pgfqpoint{2.670847in}{1.744571in}}%
\pgfpathlineto{\pgfqpoint{2.670847in}{1.741622in}}%
\pgfpathmoveto{\pgfqpoint{2.666306in}{1.744571in}}%
\pgfpathlineto{\pgfqpoint{2.666306in}{1.744571in}}%
\pgfpathlineto{\pgfqpoint{2.666306in}{1.747520in}}%
\pgfpathlineto{\pgfqpoint{2.670847in}{1.747520in}}%
\pgfpathlineto{\pgfqpoint{2.670847in}{1.744571in}}%
\pgfpathmoveto{\pgfqpoint{2.670847in}{1.744571in}}%
\pgfpathlineto{\pgfqpoint{2.670847in}{1.744571in}}%
\pgfpathlineto{\pgfqpoint{2.670847in}{1.747520in}}%
\pgfpathlineto{\pgfqpoint{2.675388in}{1.747520in}}%
\pgfpathlineto{\pgfqpoint{2.675388in}{1.744571in}}%
\pgfpathmoveto{\pgfqpoint{2.670847in}{1.747520in}}%
\pgfpathlineto{\pgfqpoint{2.670847in}{1.747520in}}%
\pgfpathlineto{\pgfqpoint{2.670847in}{1.750469in}}%
\pgfpathlineto{\pgfqpoint{2.675388in}{1.750469in}}%
\pgfpathlineto{\pgfqpoint{2.675388in}{1.747520in}}%
\pgfpathmoveto{\pgfqpoint{2.675388in}{1.747520in}}%
\pgfpathlineto{\pgfqpoint{2.675388in}{1.747520in}}%
\pgfpathlineto{\pgfqpoint{2.675388in}{1.750469in}}%
\pgfpathlineto{\pgfqpoint{2.679929in}{1.750469in}}%
\pgfpathlineto{\pgfqpoint{2.679929in}{1.747520in}}%
\pgfpathmoveto{\pgfqpoint{2.675388in}{1.750469in}}%
\pgfpathlineto{\pgfqpoint{2.675388in}{1.750469in}}%
\pgfpathlineto{\pgfqpoint{2.675388in}{1.753418in}}%
\pgfpathlineto{\pgfqpoint{2.679929in}{1.753418in}}%
\pgfpathlineto{\pgfqpoint{2.679929in}{1.750469in}}%
\pgfpathmoveto{\pgfqpoint{2.679929in}{1.750469in}}%
\pgfpathlineto{\pgfqpoint{2.679929in}{1.750469in}}%
\pgfpathlineto{\pgfqpoint{2.679929in}{1.753418in}}%
\pgfpathlineto{\pgfqpoint{2.684470in}{1.753418in}}%
\pgfpathlineto{\pgfqpoint{2.684470in}{1.750469in}}%
\pgfpathmoveto{\pgfqpoint{2.679929in}{1.753418in}}%
\pgfpathlineto{\pgfqpoint{2.679929in}{1.753418in}}%
\pgfpathlineto{\pgfqpoint{2.679929in}{1.756367in}}%
\pgfpathlineto{\pgfqpoint{2.684470in}{1.756367in}}%
\pgfpathlineto{\pgfqpoint{2.684470in}{1.753418in}}%
\pgfpathmoveto{\pgfqpoint{2.684470in}{1.753418in}}%
\pgfpathlineto{\pgfqpoint{2.684470in}{1.753418in}}%
\pgfpathlineto{\pgfqpoint{2.684470in}{1.756367in}}%
\pgfpathlineto{\pgfqpoint{2.689011in}{1.756367in}}%
\pgfpathlineto{\pgfqpoint{2.689011in}{1.753418in}}%
\pgfpathmoveto{\pgfqpoint{2.684470in}{1.756367in}}%
\pgfpathlineto{\pgfqpoint{2.684470in}{1.756367in}}%
\pgfpathlineto{\pgfqpoint{2.684470in}{1.759316in}}%
\pgfpathlineto{\pgfqpoint{2.689011in}{1.759316in}}%
\pgfpathlineto{\pgfqpoint{2.689011in}{1.756367in}}%
\pgfpathmoveto{\pgfqpoint{2.689011in}{1.756367in}}%
\pgfpathlineto{\pgfqpoint{2.689011in}{1.756367in}}%
\pgfpathlineto{\pgfqpoint{2.689011in}{1.759316in}}%
\pgfpathlineto{\pgfqpoint{2.693552in}{1.759316in}}%
\pgfpathlineto{\pgfqpoint{2.693552in}{1.756367in}}%
\pgfpathmoveto{\pgfqpoint{2.689011in}{1.759316in}}%
\pgfpathlineto{\pgfqpoint{2.689011in}{1.759316in}}%
\pgfpathlineto{\pgfqpoint{2.689011in}{1.762266in}}%
\pgfpathlineto{\pgfqpoint{2.693552in}{1.762266in}}%
\pgfpathlineto{\pgfqpoint{2.693552in}{1.759316in}}%
\pgfpathmoveto{\pgfqpoint{2.693552in}{1.759316in}}%
\pgfpathlineto{\pgfqpoint{2.693552in}{1.759316in}}%
\pgfpathlineto{\pgfqpoint{2.693552in}{1.762266in}}%
\pgfpathlineto{\pgfqpoint{2.698094in}{1.762266in}}%
\pgfpathlineto{\pgfqpoint{2.698094in}{1.759316in}}%
\pgfpathmoveto{\pgfqpoint{2.693552in}{1.762266in}}%
\pgfpathlineto{\pgfqpoint{2.693552in}{1.762266in}}%
\pgfpathlineto{\pgfqpoint{2.693552in}{1.765215in}}%
\pgfpathlineto{\pgfqpoint{2.698094in}{1.765215in}}%
\pgfpathlineto{\pgfqpoint{2.698094in}{1.762266in}}%
\pgfpathmoveto{\pgfqpoint{2.698094in}{1.762266in}}%
\pgfpathlineto{\pgfqpoint{2.698094in}{1.762266in}}%
\pgfpathlineto{\pgfqpoint{2.698094in}{1.765215in}}%
\pgfpathlineto{\pgfqpoint{2.702635in}{1.765215in}}%
\pgfpathlineto{\pgfqpoint{2.702635in}{1.762266in}}%
\pgfpathmoveto{\pgfqpoint{2.698094in}{1.765215in}}%
\pgfpathlineto{\pgfqpoint{2.698094in}{1.765215in}}%
\pgfpathlineto{\pgfqpoint{2.698094in}{1.768164in}}%
\pgfpathlineto{\pgfqpoint{2.702635in}{1.768164in}}%
\pgfpathlineto{\pgfqpoint{2.702635in}{1.765215in}}%
\pgfpathmoveto{\pgfqpoint{2.702635in}{1.765215in}}%
\pgfpathlineto{\pgfqpoint{2.702635in}{1.765215in}}%
\pgfpathlineto{\pgfqpoint{2.702635in}{1.768164in}}%
\pgfpathlineto{\pgfqpoint{2.707176in}{1.768164in}}%
\pgfpathlineto{\pgfqpoint{2.707176in}{1.765215in}}%
\pgfpathmoveto{\pgfqpoint{2.702635in}{1.768164in}}%
\pgfpathlineto{\pgfqpoint{2.702635in}{1.768164in}}%
\pgfpathlineto{\pgfqpoint{2.702635in}{1.771113in}}%
\pgfpathlineto{\pgfqpoint{2.707176in}{1.771113in}}%
\pgfpathlineto{\pgfqpoint{2.707176in}{1.768164in}}%
\pgfpathmoveto{\pgfqpoint{2.707176in}{1.768164in}}%
\pgfpathlineto{\pgfqpoint{2.707176in}{1.768164in}}%
\pgfpathlineto{\pgfqpoint{2.707176in}{1.771113in}}%
\pgfpathlineto{\pgfqpoint{2.711717in}{1.771113in}}%
\pgfpathlineto{\pgfqpoint{2.711717in}{1.768164in}}%
\pgfpathmoveto{\pgfqpoint{2.707176in}{1.771113in}}%
\pgfpathlineto{\pgfqpoint{2.707176in}{1.771113in}}%
\pgfpathlineto{\pgfqpoint{2.707176in}{1.774062in}}%
\pgfpathlineto{\pgfqpoint{2.711717in}{1.774062in}}%
\pgfpathlineto{\pgfqpoint{2.711717in}{1.771113in}}%
\pgfpathmoveto{\pgfqpoint{2.711717in}{1.771113in}}%
\pgfpathlineto{\pgfqpoint{2.711717in}{1.771113in}}%
\pgfpathlineto{\pgfqpoint{2.711717in}{1.774062in}}%
\pgfpathlineto{\pgfqpoint{2.716258in}{1.774062in}}%
\pgfpathlineto{\pgfqpoint{2.716258in}{1.771113in}}%
\pgfpathmoveto{\pgfqpoint{2.711717in}{1.774062in}}%
\pgfpathlineto{\pgfqpoint{2.711717in}{1.774062in}}%
\pgfpathlineto{\pgfqpoint{2.711717in}{1.777011in}}%
\pgfpathlineto{\pgfqpoint{2.716258in}{1.777011in}}%
\pgfpathlineto{\pgfqpoint{2.716258in}{1.774062in}}%
\pgfpathmoveto{\pgfqpoint{2.716258in}{1.774062in}}%
\pgfpathlineto{\pgfqpoint{2.716258in}{1.774062in}}%
\pgfpathlineto{\pgfqpoint{2.716258in}{1.777011in}}%
\pgfpathlineto{\pgfqpoint{2.720799in}{1.777011in}}%
\pgfpathlineto{\pgfqpoint{2.720799in}{1.774062in}}%
\pgfpathmoveto{\pgfqpoint{2.716258in}{1.777011in}}%
\pgfpathlineto{\pgfqpoint{2.716258in}{1.777011in}}%
\pgfpathlineto{\pgfqpoint{2.716258in}{1.779960in}}%
\pgfpathlineto{\pgfqpoint{2.720799in}{1.779960in}}%
\pgfpathlineto{\pgfqpoint{2.720799in}{1.777011in}}%
\pgfpathmoveto{\pgfqpoint{2.720799in}{1.777011in}}%
\pgfpathlineto{\pgfqpoint{2.720799in}{1.777011in}}%
\pgfpathlineto{\pgfqpoint{2.720799in}{1.779960in}}%
\pgfpathlineto{\pgfqpoint{2.725340in}{1.779960in}}%
\pgfpathlineto{\pgfqpoint{2.725340in}{1.777011in}}%
\pgfpathmoveto{\pgfqpoint{2.720799in}{1.779960in}}%
\pgfpathlineto{\pgfqpoint{2.720799in}{1.779960in}}%
\pgfpathlineto{\pgfqpoint{2.720799in}{1.782910in}}%
\pgfpathlineto{\pgfqpoint{2.725340in}{1.782910in}}%
\pgfpathlineto{\pgfqpoint{2.725340in}{1.779960in}}%
\pgfpathmoveto{\pgfqpoint{2.725340in}{1.779960in}}%
\pgfpathlineto{\pgfqpoint{2.725340in}{1.779960in}}%
\pgfpathlineto{\pgfqpoint{2.725340in}{1.782910in}}%
\pgfpathlineto{\pgfqpoint{2.729881in}{1.782910in}}%
\pgfpathlineto{\pgfqpoint{2.729881in}{1.779960in}}%
\pgfpathmoveto{\pgfqpoint{2.725340in}{1.782910in}}%
\pgfpathlineto{\pgfqpoint{2.725340in}{1.782910in}}%
\pgfpathlineto{\pgfqpoint{2.725340in}{1.785859in}}%
\pgfpathlineto{\pgfqpoint{2.729881in}{1.785859in}}%
\pgfpathlineto{\pgfqpoint{2.729881in}{1.782910in}}%
\pgfpathmoveto{\pgfqpoint{2.729881in}{1.782910in}}%
\pgfpathlineto{\pgfqpoint{2.729881in}{1.782910in}}%
\pgfpathlineto{\pgfqpoint{2.729881in}{1.785859in}}%
\pgfpathlineto{\pgfqpoint{2.734422in}{1.785859in}}%
\pgfpathlineto{\pgfqpoint{2.734422in}{1.782910in}}%
\pgfpathmoveto{\pgfqpoint{2.729881in}{1.785859in}}%
\pgfpathlineto{\pgfqpoint{2.729881in}{1.785859in}}%
\pgfpathlineto{\pgfqpoint{2.729881in}{1.788808in}}%
\pgfpathlineto{\pgfqpoint{2.734422in}{1.788808in}}%
\pgfpathlineto{\pgfqpoint{2.734422in}{1.785859in}}%
\pgfpathmoveto{\pgfqpoint{2.734422in}{1.785859in}}%
\pgfpathlineto{\pgfqpoint{2.734422in}{1.785859in}}%
\pgfpathlineto{\pgfqpoint{2.734422in}{1.788808in}}%
\pgfpathlineto{\pgfqpoint{2.738963in}{1.788808in}}%
\pgfpathlineto{\pgfqpoint{2.738963in}{1.785859in}}%
\pgfpathmoveto{\pgfqpoint{2.734422in}{1.788808in}}%
\pgfpathlineto{\pgfqpoint{2.734422in}{1.788808in}}%
\pgfpathlineto{\pgfqpoint{2.734422in}{1.791757in}}%
\pgfpathlineto{\pgfqpoint{2.738963in}{1.791757in}}%
\pgfpathlineto{\pgfqpoint{2.738963in}{1.788808in}}%
\pgfpathmoveto{\pgfqpoint{2.738963in}{1.788808in}}%
\pgfpathlineto{\pgfqpoint{2.738963in}{1.788808in}}%
\pgfpathlineto{\pgfqpoint{2.738963in}{1.791757in}}%
\pgfpathlineto{\pgfqpoint{2.743504in}{1.791757in}}%
\pgfpathlineto{\pgfqpoint{2.743504in}{1.788808in}}%
\pgfpathmoveto{\pgfqpoint{2.738963in}{1.791757in}}%
\pgfpathlineto{\pgfqpoint{2.738963in}{1.791757in}}%
\pgfpathlineto{\pgfqpoint{2.738963in}{1.794706in}}%
\pgfpathlineto{\pgfqpoint{2.743504in}{1.794706in}}%
\pgfpathlineto{\pgfqpoint{2.743504in}{1.791757in}}%
\pgfpathmoveto{\pgfqpoint{2.743504in}{1.791757in}}%
\pgfpathlineto{\pgfqpoint{2.743504in}{1.791757in}}%
\pgfpathlineto{\pgfqpoint{2.743504in}{1.794706in}}%
\pgfpathlineto{\pgfqpoint{2.748045in}{1.794706in}}%
\pgfpathlineto{\pgfqpoint{2.748045in}{1.791757in}}%
\pgfpathmoveto{\pgfqpoint{2.743504in}{1.794706in}}%
\pgfpathlineto{\pgfqpoint{2.743504in}{1.794706in}}%
\pgfpathlineto{\pgfqpoint{2.743504in}{1.797655in}}%
\pgfpathlineto{\pgfqpoint{2.748045in}{1.797655in}}%
\pgfpathlineto{\pgfqpoint{2.748045in}{1.794706in}}%
\pgfpathmoveto{\pgfqpoint{2.748045in}{1.794706in}}%
\pgfpathlineto{\pgfqpoint{2.748045in}{1.794706in}}%
\pgfpathlineto{\pgfqpoint{2.748045in}{1.797655in}}%
\pgfpathlineto{\pgfqpoint{2.752586in}{1.797655in}}%
\pgfpathlineto{\pgfqpoint{2.752586in}{1.794706in}}%
\pgfpathmoveto{\pgfqpoint{2.748045in}{1.797655in}}%
\pgfpathlineto{\pgfqpoint{2.748045in}{1.797655in}}%
\pgfpathlineto{\pgfqpoint{2.748045in}{1.800604in}}%
\pgfpathlineto{\pgfqpoint{2.752586in}{1.800604in}}%
\pgfpathlineto{\pgfqpoint{2.752586in}{1.797655in}}%
\pgfpathmoveto{\pgfqpoint{2.748045in}{1.800604in}}%
\pgfpathlineto{\pgfqpoint{2.748045in}{1.800604in}}%
\pgfpathlineto{\pgfqpoint{2.748045in}{1.803553in}}%
\pgfpathlineto{\pgfqpoint{2.752586in}{1.803553in}}%
\pgfpathlineto{\pgfqpoint{2.752586in}{1.800604in}}%
\pgfpathmoveto{\pgfqpoint{2.752586in}{1.800604in}}%
\pgfpathlineto{\pgfqpoint{2.752586in}{1.800604in}}%
\pgfpathlineto{\pgfqpoint{2.752586in}{1.803553in}}%
\pgfpathlineto{\pgfqpoint{2.757127in}{1.803553in}}%
\pgfpathlineto{\pgfqpoint{2.757127in}{1.800604in}}%
\pgfpathmoveto{\pgfqpoint{2.752586in}{1.803553in}}%
\pgfpathlineto{\pgfqpoint{2.752586in}{1.803553in}}%
\pgfpathlineto{\pgfqpoint{2.752586in}{1.806503in}}%
\pgfpathlineto{\pgfqpoint{2.757127in}{1.806503in}}%
\pgfpathlineto{\pgfqpoint{2.757127in}{1.803553in}}%
\pgfpathmoveto{\pgfqpoint{2.757127in}{1.803553in}}%
\pgfpathlineto{\pgfqpoint{2.757127in}{1.803553in}}%
\pgfpathlineto{\pgfqpoint{2.757127in}{1.806503in}}%
\pgfpathlineto{\pgfqpoint{2.761668in}{1.806503in}}%
\pgfpathlineto{\pgfqpoint{2.761668in}{1.803553in}}%
\pgfpathmoveto{\pgfqpoint{2.757127in}{1.806503in}}%
\pgfpathlineto{\pgfqpoint{2.757127in}{1.806503in}}%
\pgfpathlineto{\pgfqpoint{2.757127in}{1.809452in}}%
\pgfpathlineto{\pgfqpoint{2.761668in}{1.809452in}}%
\pgfpathlineto{\pgfqpoint{2.761668in}{1.806503in}}%
\pgfpathmoveto{\pgfqpoint{2.761668in}{1.806503in}}%
\pgfpathlineto{\pgfqpoint{2.761668in}{1.806503in}}%
\pgfpathlineto{\pgfqpoint{2.761668in}{1.809452in}}%
\pgfpathlineto{\pgfqpoint{2.766209in}{1.809452in}}%
\pgfpathlineto{\pgfqpoint{2.766209in}{1.806503in}}%
\pgfpathmoveto{\pgfqpoint{2.761668in}{1.809452in}}%
\pgfpathlineto{\pgfqpoint{2.761668in}{1.809452in}}%
\pgfpathlineto{\pgfqpoint{2.761668in}{1.812401in}}%
\pgfpathlineto{\pgfqpoint{2.766209in}{1.812401in}}%
\pgfpathlineto{\pgfqpoint{2.766209in}{1.809452in}}%
\pgfpathmoveto{\pgfqpoint{2.766209in}{1.809452in}}%
\pgfpathlineto{\pgfqpoint{2.766209in}{1.809452in}}%
\pgfpathlineto{\pgfqpoint{2.766209in}{1.812401in}}%
\pgfpathlineto{\pgfqpoint{2.770750in}{1.812401in}}%
\pgfpathlineto{\pgfqpoint{2.770750in}{1.809452in}}%
\pgfpathmoveto{\pgfqpoint{2.766209in}{1.812401in}}%
\pgfpathlineto{\pgfqpoint{2.766209in}{1.812401in}}%
\pgfpathlineto{\pgfqpoint{2.766209in}{1.815350in}}%
\pgfpathlineto{\pgfqpoint{2.770750in}{1.815350in}}%
\pgfpathlineto{\pgfqpoint{2.770750in}{1.812401in}}%
\pgfpathmoveto{\pgfqpoint{2.770750in}{1.812401in}}%
\pgfpathlineto{\pgfqpoint{2.770750in}{1.812401in}}%
\pgfpathlineto{\pgfqpoint{2.770750in}{1.815350in}}%
\pgfpathlineto{\pgfqpoint{2.775291in}{1.815350in}}%
\pgfpathlineto{\pgfqpoint{2.775291in}{1.812401in}}%
\pgfpathmoveto{\pgfqpoint{2.770750in}{1.815350in}}%
\pgfpathlineto{\pgfqpoint{2.770750in}{1.815350in}}%
\pgfpathlineto{\pgfqpoint{2.770750in}{1.818299in}}%
\pgfpathlineto{\pgfqpoint{2.775291in}{1.818299in}}%
\pgfpathlineto{\pgfqpoint{2.775291in}{1.815350in}}%
\pgfpathmoveto{\pgfqpoint{2.775291in}{1.815350in}}%
\pgfpathlineto{\pgfqpoint{2.775291in}{1.815350in}}%
\pgfpathlineto{\pgfqpoint{2.775291in}{1.818299in}}%
\pgfpathlineto{\pgfqpoint{2.779832in}{1.818299in}}%
\pgfpathlineto{\pgfqpoint{2.779832in}{1.815350in}}%
\pgfpathmoveto{\pgfqpoint{2.775291in}{1.818299in}}%
\pgfpathlineto{\pgfqpoint{2.775291in}{1.818299in}}%
\pgfpathlineto{\pgfqpoint{2.775291in}{1.821248in}}%
\pgfpathlineto{\pgfqpoint{2.779832in}{1.821248in}}%
\pgfpathlineto{\pgfqpoint{2.779832in}{1.818299in}}%
\pgfpathmoveto{\pgfqpoint{2.779832in}{1.818299in}}%
\pgfpathlineto{\pgfqpoint{2.779832in}{1.818299in}}%
\pgfpathlineto{\pgfqpoint{2.779832in}{1.821248in}}%
\pgfpathlineto{\pgfqpoint{2.784374in}{1.821248in}}%
\pgfpathlineto{\pgfqpoint{2.784374in}{1.818299in}}%
\pgfpathmoveto{\pgfqpoint{2.779832in}{1.821248in}}%
\pgfpathlineto{\pgfqpoint{2.779832in}{1.821248in}}%
\pgfpathlineto{\pgfqpoint{2.779832in}{1.824198in}}%
\pgfpathlineto{\pgfqpoint{2.784374in}{1.824198in}}%
\pgfpathlineto{\pgfqpoint{2.784374in}{1.821248in}}%
\pgfpathmoveto{\pgfqpoint{2.784374in}{1.821248in}}%
\pgfpathlineto{\pgfqpoint{2.784374in}{1.821248in}}%
\pgfpathlineto{\pgfqpoint{2.784374in}{1.824198in}}%
\pgfpathlineto{\pgfqpoint{2.788915in}{1.824198in}}%
\pgfpathlineto{\pgfqpoint{2.788915in}{1.821248in}}%
\pgfpathmoveto{\pgfqpoint{2.784374in}{1.824198in}}%
\pgfpathlineto{\pgfqpoint{2.784374in}{1.824198in}}%
\pgfpathlineto{\pgfqpoint{2.784374in}{1.827147in}}%
\pgfpathlineto{\pgfqpoint{2.788915in}{1.827147in}}%
\pgfpathlineto{\pgfqpoint{2.788915in}{1.824198in}}%
\pgfpathmoveto{\pgfqpoint{2.788915in}{1.824198in}}%
\pgfpathlineto{\pgfqpoint{2.788915in}{1.824198in}}%
\pgfpathlineto{\pgfqpoint{2.788915in}{1.827147in}}%
\pgfpathlineto{\pgfqpoint{2.793456in}{1.827147in}}%
\pgfpathlineto{\pgfqpoint{2.793456in}{1.824198in}}%
\pgfpathmoveto{\pgfqpoint{2.788915in}{1.827147in}}%
\pgfpathlineto{\pgfqpoint{2.788915in}{1.827147in}}%
\pgfpathlineto{\pgfqpoint{2.788915in}{1.830096in}}%
\pgfpathlineto{\pgfqpoint{2.793456in}{1.830096in}}%
\pgfpathlineto{\pgfqpoint{2.793456in}{1.827147in}}%
\pgfpathmoveto{\pgfqpoint{2.793456in}{1.827147in}}%
\pgfpathlineto{\pgfqpoint{2.793456in}{1.827147in}}%
\pgfpathlineto{\pgfqpoint{2.793456in}{1.830096in}}%
\pgfpathlineto{\pgfqpoint{2.797997in}{1.830096in}}%
\pgfpathlineto{\pgfqpoint{2.797997in}{1.827147in}}%
\pgfpathmoveto{\pgfqpoint{2.793456in}{1.830096in}}%
\pgfpathlineto{\pgfqpoint{2.793456in}{1.830096in}}%
\pgfpathlineto{\pgfqpoint{2.793456in}{1.833046in}}%
\pgfpathlineto{\pgfqpoint{2.797997in}{1.833046in}}%
\pgfpathlineto{\pgfqpoint{2.797997in}{1.830096in}}%
\pgfpathmoveto{\pgfqpoint{2.797997in}{1.830096in}}%
\pgfpathlineto{\pgfqpoint{2.797997in}{1.830096in}}%
\pgfpathlineto{\pgfqpoint{2.797997in}{1.833046in}}%
\pgfpathlineto{\pgfqpoint{2.802538in}{1.833046in}}%
\pgfpathlineto{\pgfqpoint{2.802538in}{1.830096in}}%
\pgfpathmoveto{\pgfqpoint{2.797997in}{1.833046in}}%
\pgfpathlineto{\pgfqpoint{2.797997in}{1.833046in}}%
\pgfpathlineto{\pgfqpoint{2.797997in}{1.835995in}}%
\pgfpathlineto{\pgfqpoint{2.802538in}{1.835995in}}%
\pgfpathlineto{\pgfqpoint{2.802538in}{1.833046in}}%
\pgfpathmoveto{\pgfqpoint{2.802538in}{1.833046in}}%
\pgfpathlineto{\pgfqpoint{2.802538in}{1.833046in}}%
\pgfpathlineto{\pgfqpoint{2.802538in}{1.835995in}}%
\pgfpathlineto{\pgfqpoint{2.807079in}{1.835995in}}%
\pgfpathlineto{\pgfqpoint{2.807079in}{1.833046in}}%
\pgfpathmoveto{\pgfqpoint{2.802538in}{1.835995in}}%
\pgfpathlineto{\pgfqpoint{2.802538in}{1.835995in}}%
\pgfpathlineto{\pgfqpoint{2.802538in}{1.838944in}}%
\pgfpathlineto{\pgfqpoint{2.807079in}{1.838944in}}%
\pgfpathlineto{\pgfqpoint{2.807079in}{1.835995in}}%
\pgfpathmoveto{\pgfqpoint{2.807079in}{1.835995in}}%
\pgfpathlineto{\pgfqpoint{2.807079in}{1.835995in}}%
\pgfpathlineto{\pgfqpoint{2.807079in}{1.838944in}}%
\pgfpathlineto{\pgfqpoint{2.811620in}{1.838944in}}%
\pgfpathlineto{\pgfqpoint{2.811620in}{1.835995in}}%
\pgfpathmoveto{\pgfqpoint{2.807079in}{1.838944in}}%
\pgfpathlineto{\pgfqpoint{2.807079in}{1.838944in}}%
\pgfpathlineto{\pgfqpoint{2.807079in}{1.841894in}}%
\pgfpathlineto{\pgfqpoint{2.811620in}{1.841894in}}%
\pgfpathlineto{\pgfqpoint{2.811620in}{1.838944in}}%
\pgfpathmoveto{\pgfqpoint{2.811620in}{1.838944in}}%
\pgfpathlineto{\pgfqpoint{2.811620in}{1.838944in}}%
\pgfpathlineto{\pgfqpoint{2.811620in}{1.841894in}}%
\pgfpathlineto{\pgfqpoint{2.816161in}{1.841894in}}%
\pgfpathlineto{\pgfqpoint{2.816161in}{1.838944in}}%
\pgfpathmoveto{\pgfqpoint{2.816161in}{1.838944in}}%
\pgfpathlineto{\pgfqpoint{2.816161in}{1.838944in}}%
\pgfpathlineto{\pgfqpoint{2.816161in}{1.841894in}}%
\pgfpathlineto{\pgfqpoint{2.820702in}{1.841894in}}%
\pgfpathlineto{\pgfqpoint{2.820702in}{1.838944in}}%
\pgfpathmoveto{\pgfqpoint{2.816161in}{1.841894in}}%
\pgfpathlineto{\pgfqpoint{2.816161in}{1.841894in}}%
\pgfpathlineto{\pgfqpoint{2.816161in}{1.844843in}}%
\pgfpathlineto{\pgfqpoint{2.820702in}{1.844843in}}%
\pgfpathlineto{\pgfqpoint{2.820702in}{1.841894in}}%
\pgfpathmoveto{\pgfqpoint{2.820702in}{1.841894in}}%
\pgfpathlineto{\pgfqpoint{2.820702in}{1.841894in}}%
\pgfpathlineto{\pgfqpoint{2.820702in}{1.844843in}}%
\pgfpathlineto{\pgfqpoint{2.825243in}{1.844843in}}%
\pgfpathlineto{\pgfqpoint{2.825243in}{1.841894in}}%
\pgfpathmoveto{\pgfqpoint{2.820702in}{1.844843in}}%
\pgfpathlineto{\pgfqpoint{2.820702in}{1.844843in}}%
\pgfpathlineto{\pgfqpoint{2.820702in}{1.847793in}}%
\pgfpathlineto{\pgfqpoint{2.825243in}{1.847793in}}%
\pgfpathlineto{\pgfqpoint{2.825243in}{1.844843in}}%
\pgfpathmoveto{\pgfqpoint{2.825243in}{1.844843in}}%
\pgfpathlineto{\pgfqpoint{2.825243in}{1.844843in}}%
\pgfpathlineto{\pgfqpoint{2.825243in}{1.847793in}}%
\pgfpathlineto{\pgfqpoint{2.829784in}{1.847793in}}%
\pgfpathlineto{\pgfqpoint{2.829784in}{1.844843in}}%
\pgfpathmoveto{\pgfqpoint{2.825243in}{1.847793in}}%
\pgfpathlineto{\pgfqpoint{2.825243in}{1.847793in}}%
\pgfpathlineto{\pgfqpoint{2.825243in}{1.850742in}}%
\pgfpathlineto{\pgfqpoint{2.829784in}{1.850742in}}%
\pgfpathlineto{\pgfqpoint{2.829784in}{1.847793in}}%
\pgfpathmoveto{\pgfqpoint{2.829784in}{1.847793in}}%
\pgfpathlineto{\pgfqpoint{2.829784in}{1.847793in}}%
\pgfpathlineto{\pgfqpoint{2.829784in}{1.850742in}}%
\pgfpathlineto{\pgfqpoint{2.834325in}{1.850742in}}%
\pgfpathlineto{\pgfqpoint{2.834325in}{1.847793in}}%
\pgfpathmoveto{\pgfqpoint{2.829784in}{1.850742in}}%
\pgfpathlineto{\pgfqpoint{2.829784in}{1.850742in}}%
\pgfpathlineto{\pgfqpoint{2.829784in}{1.853691in}}%
\pgfpathlineto{\pgfqpoint{2.834325in}{1.853691in}}%
\pgfpathlineto{\pgfqpoint{2.834325in}{1.850742in}}%
\pgfpathmoveto{\pgfqpoint{2.834325in}{1.850742in}}%
\pgfpathlineto{\pgfqpoint{2.834325in}{1.850742in}}%
\pgfpathlineto{\pgfqpoint{2.834325in}{1.853691in}}%
\pgfpathlineto{\pgfqpoint{2.838866in}{1.853691in}}%
\pgfpathlineto{\pgfqpoint{2.838866in}{1.850742in}}%
\pgfpathmoveto{\pgfqpoint{2.834325in}{1.853691in}}%
\pgfpathlineto{\pgfqpoint{2.834325in}{1.853691in}}%
\pgfpathlineto{\pgfqpoint{2.834325in}{1.856641in}}%
\pgfpathlineto{\pgfqpoint{2.838866in}{1.856641in}}%
\pgfpathlineto{\pgfqpoint{2.838866in}{1.853691in}}%
\pgfpathmoveto{\pgfqpoint{2.838866in}{1.853691in}}%
\pgfpathlineto{\pgfqpoint{2.838866in}{1.853691in}}%
\pgfpathlineto{\pgfqpoint{2.838866in}{1.856641in}}%
\pgfpathlineto{\pgfqpoint{2.843407in}{1.856641in}}%
\pgfpathlineto{\pgfqpoint{2.843407in}{1.853691in}}%
\pgfpathmoveto{\pgfqpoint{2.838866in}{1.856641in}}%
\pgfpathlineto{\pgfqpoint{2.838866in}{1.856641in}}%
\pgfpathlineto{\pgfqpoint{2.838866in}{1.859590in}}%
\pgfpathlineto{\pgfqpoint{2.843407in}{1.859590in}}%
\pgfpathlineto{\pgfqpoint{2.843407in}{1.856641in}}%
\pgfpathmoveto{\pgfqpoint{2.843407in}{1.856641in}}%
\pgfpathlineto{\pgfqpoint{2.843407in}{1.856641in}}%
\pgfpathlineto{\pgfqpoint{2.843407in}{1.859590in}}%
\pgfpathlineto{\pgfqpoint{2.847948in}{1.859590in}}%
\pgfpathlineto{\pgfqpoint{2.847948in}{1.856641in}}%
\pgfpathmoveto{\pgfqpoint{2.843407in}{1.859590in}}%
\pgfpathlineto{\pgfqpoint{2.843407in}{1.859590in}}%
\pgfpathlineto{\pgfqpoint{2.843407in}{1.862539in}}%
\pgfpathlineto{\pgfqpoint{2.847948in}{1.862539in}}%
\pgfpathlineto{\pgfqpoint{2.847948in}{1.859590in}}%
\pgfpathmoveto{\pgfqpoint{2.847948in}{1.859590in}}%
\pgfpathlineto{\pgfqpoint{2.847948in}{1.859590in}}%
\pgfpathlineto{\pgfqpoint{2.847948in}{1.862539in}}%
\pgfpathlineto{\pgfqpoint{2.852489in}{1.862539in}}%
\pgfpathlineto{\pgfqpoint{2.852489in}{1.859590in}}%
\pgfpathmoveto{\pgfqpoint{2.847948in}{1.862539in}}%
\pgfpathlineto{\pgfqpoint{2.847948in}{1.862539in}}%
\pgfpathlineto{\pgfqpoint{2.847948in}{1.865489in}}%
\pgfpathlineto{\pgfqpoint{2.852489in}{1.865489in}}%
\pgfpathlineto{\pgfqpoint{2.852489in}{1.862539in}}%
\pgfpathmoveto{\pgfqpoint{2.852489in}{1.862539in}}%
\pgfpathlineto{\pgfqpoint{2.852489in}{1.862539in}}%
\pgfpathlineto{\pgfqpoint{2.852489in}{1.865489in}}%
\pgfpathlineto{\pgfqpoint{2.857030in}{1.865489in}}%
\pgfpathlineto{\pgfqpoint{2.857030in}{1.862539in}}%
\pgfpathmoveto{\pgfqpoint{2.852489in}{1.865489in}}%
\pgfpathlineto{\pgfqpoint{2.852489in}{1.865489in}}%
\pgfpathlineto{\pgfqpoint{2.852489in}{1.868438in}}%
\pgfpathlineto{\pgfqpoint{2.857030in}{1.868438in}}%
\pgfpathlineto{\pgfqpoint{2.857030in}{1.865489in}}%
\pgfpathmoveto{\pgfqpoint{2.857030in}{1.865489in}}%
\pgfpathlineto{\pgfqpoint{2.857030in}{1.865489in}}%
\pgfpathlineto{\pgfqpoint{2.857030in}{1.868438in}}%
\pgfpathlineto{\pgfqpoint{2.861571in}{1.868438in}}%
\pgfpathlineto{\pgfqpoint{2.861571in}{1.865489in}}%
\pgfpathmoveto{\pgfqpoint{2.857030in}{1.868438in}}%
\pgfpathlineto{\pgfqpoint{2.857030in}{1.868438in}}%
\pgfpathlineto{\pgfqpoint{2.857030in}{1.871387in}}%
\pgfpathlineto{\pgfqpoint{2.861571in}{1.871387in}}%
\pgfpathlineto{\pgfqpoint{2.861571in}{1.868438in}}%
\pgfpathmoveto{\pgfqpoint{2.861571in}{1.868438in}}%
\pgfpathlineto{\pgfqpoint{2.861571in}{1.868438in}}%
\pgfpathlineto{\pgfqpoint{2.861571in}{1.871387in}}%
\pgfpathlineto{\pgfqpoint{2.866112in}{1.871387in}}%
\pgfpathlineto{\pgfqpoint{2.866112in}{1.868438in}}%
\pgfpathmoveto{\pgfqpoint{2.861571in}{1.871387in}}%
\pgfpathlineto{\pgfqpoint{2.861571in}{1.871387in}}%
\pgfpathlineto{\pgfqpoint{2.861571in}{1.874337in}}%
\pgfpathlineto{\pgfqpoint{2.866112in}{1.874337in}}%
\pgfpathlineto{\pgfqpoint{2.866112in}{1.871387in}}%
\pgfpathmoveto{\pgfqpoint{2.866112in}{1.871387in}}%
\pgfpathlineto{\pgfqpoint{2.866112in}{1.871387in}}%
\pgfpathlineto{\pgfqpoint{2.866112in}{1.874337in}}%
\pgfpathlineto{\pgfqpoint{2.870653in}{1.874337in}}%
\pgfpathlineto{\pgfqpoint{2.870653in}{1.871387in}}%
\pgfpathmoveto{\pgfqpoint{2.866112in}{1.874337in}}%
\pgfpathlineto{\pgfqpoint{2.866112in}{1.874337in}}%
\pgfpathlineto{\pgfqpoint{2.866112in}{1.877286in}}%
\pgfpathlineto{\pgfqpoint{2.870653in}{1.877286in}}%
\pgfpathlineto{\pgfqpoint{2.870653in}{1.874337in}}%
\pgfpathmoveto{\pgfqpoint{2.870653in}{1.874337in}}%
\pgfpathlineto{\pgfqpoint{2.870653in}{1.874337in}}%
\pgfpathlineto{\pgfqpoint{2.870653in}{1.877286in}}%
\pgfpathlineto{\pgfqpoint{2.875195in}{1.877286in}}%
\pgfpathlineto{\pgfqpoint{2.875195in}{1.874337in}}%
\pgfpathmoveto{\pgfqpoint{2.870653in}{1.877286in}}%
\pgfpathlineto{\pgfqpoint{2.870653in}{1.877286in}}%
\pgfpathlineto{\pgfqpoint{2.870653in}{1.880236in}}%
\pgfpathlineto{\pgfqpoint{2.875195in}{1.880236in}}%
\pgfpathlineto{\pgfqpoint{2.875195in}{1.877286in}}%
\pgfpathmoveto{\pgfqpoint{2.875195in}{1.877286in}}%
\pgfpathlineto{\pgfqpoint{2.875195in}{1.877286in}}%
\pgfpathlineto{\pgfqpoint{2.875195in}{1.880236in}}%
\pgfpathlineto{\pgfqpoint{2.879736in}{1.880236in}}%
\pgfpathlineto{\pgfqpoint{2.879736in}{1.877286in}}%
\pgfpathmoveto{\pgfqpoint{2.875195in}{1.880236in}}%
\pgfpathlineto{\pgfqpoint{2.875195in}{1.880236in}}%
\pgfpathlineto{\pgfqpoint{2.875195in}{1.883185in}}%
\pgfpathlineto{\pgfqpoint{2.879736in}{1.883185in}}%
\pgfpathlineto{\pgfqpoint{2.879736in}{1.880236in}}%
\pgfpathmoveto{\pgfqpoint{2.879736in}{1.880236in}}%
\pgfpathlineto{\pgfqpoint{2.879736in}{1.880236in}}%
\pgfpathlineto{\pgfqpoint{2.879736in}{1.883185in}}%
\pgfpathlineto{\pgfqpoint{2.884277in}{1.883185in}}%
\pgfpathlineto{\pgfqpoint{2.884277in}{1.880236in}}%
\pgfpathmoveto{\pgfqpoint{2.879736in}{1.883185in}}%
\pgfpathlineto{\pgfqpoint{2.879736in}{1.883185in}}%
\pgfpathlineto{\pgfqpoint{2.879736in}{1.886134in}}%
\pgfpathlineto{\pgfqpoint{2.884277in}{1.886134in}}%
\pgfpathlineto{\pgfqpoint{2.884277in}{1.883185in}}%
\pgfpathmoveto{\pgfqpoint{2.884277in}{1.883185in}}%
\pgfpathlineto{\pgfqpoint{2.884277in}{1.883185in}}%
\pgfpathlineto{\pgfqpoint{2.884277in}{1.886134in}}%
\pgfpathlineto{\pgfqpoint{2.888818in}{1.886134in}}%
\pgfpathlineto{\pgfqpoint{2.888818in}{1.883185in}}%
\pgfpathmoveto{\pgfqpoint{2.884277in}{1.886134in}}%
\pgfpathlineto{\pgfqpoint{2.884277in}{1.886134in}}%
\pgfpathlineto{\pgfqpoint{2.884277in}{1.889084in}}%
\pgfpathlineto{\pgfqpoint{2.888818in}{1.889084in}}%
\pgfpathlineto{\pgfqpoint{2.888818in}{1.886134in}}%
\pgfpathmoveto{\pgfqpoint{2.888818in}{1.886134in}}%
\pgfpathlineto{\pgfqpoint{2.888818in}{1.886134in}}%
\pgfpathlineto{\pgfqpoint{2.888818in}{1.889084in}}%
\pgfpathlineto{\pgfqpoint{2.893359in}{1.889084in}}%
\pgfpathlineto{\pgfqpoint{2.893359in}{1.886134in}}%
\pgfpathmoveto{\pgfqpoint{2.888818in}{1.889084in}}%
\pgfpathlineto{\pgfqpoint{2.888818in}{1.889084in}}%
\pgfpathlineto{\pgfqpoint{2.888818in}{1.892033in}}%
\pgfpathlineto{\pgfqpoint{2.893359in}{1.892033in}}%
\pgfpathlineto{\pgfqpoint{2.893359in}{1.889084in}}%
\pgfpathmoveto{\pgfqpoint{2.893359in}{1.889084in}}%
\pgfpathlineto{\pgfqpoint{2.893359in}{1.889084in}}%
\pgfpathlineto{\pgfqpoint{2.893359in}{1.892033in}}%
\pgfpathlineto{\pgfqpoint{2.897900in}{1.892033in}}%
\pgfpathlineto{\pgfqpoint{2.897900in}{1.889084in}}%
\pgfpathmoveto{\pgfqpoint{2.893359in}{1.892033in}}%
\pgfpathlineto{\pgfqpoint{2.893359in}{1.892033in}}%
\pgfpathlineto{\pgfqpoint{2.893359in}{1.894982in}}%
\pgfpathlineto{\pgfqpoint{2.897900in}{1.894982in}}%
\pgfpathlineto{\pgfqpoint{2.897900in}{1.892033in}}%
\pgfpathmoveto{\pgfqpoint{2.897900in}{1.892033in}}%
\pgfpathlineto{\pgfqpoint{2.897900in}{1.892033in}}%
\pgfpathlineto{\pgfqpoint{2.897900in}{1.894982in}}%
\pgfpathlineto{\pgfqpoint{2.902441in}{1.894982in}}%
\pgfpathlineto{\pgfqpoint{2.902441in}{1.892033in}}%
\pgfpathmoveto{\pgfqpoint{2.897900in}{1.894982in}}%
\pgfpathlineto{\pgfqpoint{2.897900in}{1.894982in}}%
\pgfpathlineto{\pgfqpoint{2.897900in}{1.897932in}}%
\pgfpathlineto{\pgfqpoint{2.902441in}{1.897932in}}%
\pgfpathlineto{\pgfqpoint{2.902441in}{1.894982in}}%
\pgfpathmoveto{\pgfqpoint{2.902441in}{1.894982in}}%
\pgfpathlineto{\pgfqpoint{2.902441in}{1.894982in}}%
\pgfpathlineto{\pgfqpoint{2.902441in}{1.897932in}}%
\pgfpathlineto{\pgfqpoint{2.906982in}{1.897932in}}%
\pgfpathlineto{\pgfqpoint{2.906982in}{1.894982in}}%
\pgfpathmoveto{\pgfqpoint{2.902441in}{1.897932in}}%
\pgfpathlineto{\pgfqpoint{2.902441in}{1.897932in}}%
\pgfpathlineto{\pgfqpoint{2.902441in}{1.900881in}}%
\pgfpathlineto{\pgfqpoint{2.906982in}{1.900881in}}%
\pgfpathlineto{\pgfqpoint{2.906982in}{1.897932in}}%
\pgfpathmoveto{\pgfqpoint{2.906982in}{1.897932in}}%
\pgfpathlineto{\pgfqpoint{2.906982in}{1.897932in}}%
\pgfpathlineto{\pgfqpoint{2.906982in}{1.900881in}}%
\pgfpathlineto{\pgfqpoint{2.911523in}{1.900881in}}%
\pgfpathlineto{\pgfqpoint{2.911523in}{1.897932in}}%
\pgfpathmoveto{\pgfqpoint{2.906982in}{1.900881in}}%
\pgfpathlineto{\pgfqpoint{2.906982in}{1.900881in}}%
\pgfpathlineto{\pgfqpoint{2.906982in}{1.903830in}}%
\pgfpathlineto{\pgfqpoint{2.911523in}{1.903830in}}%
\pgfpathlineto{\pgfqpoint{2.911523in}{1.900881in}}%
\pgfpathmoveto{\pgfqpoint{2.911523in}{1.900881in}}%
\pgfpathlineto{\pgfqpoint{2.911523in}{1.900881in}}%
\pgfpathlineto{\pgfqpoint{2.911523in}{1.903830in}}%
\pgfpathlineto{\pgfqpoint{2.916064in}{1.903830in}}%
\pgfpathlineto{\pgfqpoint{2.916064in}{1.900881in}}%
\pgfpathmoveto{\pgfqpoint{2.911523in}{1.903830in}}%
\pgfpathlineto{\pgfqpoint{2.911523in}{1.903830in}}%
\pgfpathlineto{\pgfqpoint{2.911523in}{1.906780in}}%
\pgfpathlineto{\pgfqpoint{2.916064in}{1.906780in}}%
\pgfpathlineto{\pgfqpoint{2.916064in}{1.903830in}}%
\pgfpathmoveto{\pgfqpoint{2.916064in}{1.903830in}}%
\pgfpathlineto{\pgfqpoint{2.916064in}{1.903830in}}%
\pgfpathlineto{\pgfqpoint{2.916064in}{1.906780in}}%
\pgfpathlineto{\pgfqpoint{2.920605in}{1.906780in}}%
\pgfpathlineto{\pgfqpoint{2.920605in}{1.903830in}}%
\pgfpathmoveto{\pgfqpoint{2.916064in}{1.906780in}}%
\pgfpathlineto{\pgfqpoint{2.916064in}{1.906780in}}%
\pgfpathlineto{\pgfqpoint{2.916064in}{1.909729in}}%
\pgfpathlineto{\pgfqpoint{2.920605in}{1.909729in}}%
\pgfpathlineto{\pgfqpoint{2.920605in}{1.906780in}}%
\pgfpathmoveto{\pgfqpoint{2.920605in}{1.906780in}}%
\pgfpathlineto{\pgfqpoint{2.920605in}{1.906780in}}%
\pgfpathlineto{\pgfqpoint{2.920605in}{1.909729in}}%
\pgfpathlineto{\pgfqpoint{2.925146in}{1.909729in}}%
\pgfpathlineto{\pgfqpoint{2.925146in}{1.906780in}}%
\pgfpathmoveto{\pgfqpoint{2.920605in}{1.909729in}}%
\pgfpathlineto{\pgfqpoint{2.920605in}{1.909729in}}%
\pgfpathlineto{\pgfqpoint{2.920605in}{1.912678in}}%
\pgfpathlineto{\pgfqpoint{2.925146in}{1.912678in}}%
\pgfpathlineto{\pgfqpoint{2.925146in}{1.909729in}}%
\pgfpathmoveto{\pgfqpoint{2.925146in}{1.909729in}}%
\pgfpathlineto{\pgfqpoint{2.925146in}{1.909729in}}%
\pgfpathlineto{\pgfqpoint{2.925146in}{1.912678in}}%
\pgfpathlineto{\pgfqpoint{2.929687in}{1.912678in}}%
\pgfpathlineto{\pgfqpoint{2.929687in}{1.909729in}}%
\pgfpathmoveto{\pgfqpoint{2.925146in}{1.912678in}}%
\pgfpathlineto{\pgfqpoint{2.925146in}{1.912678in}}%
\pgfpathlineto{\pgfqpoint{2.925146in}{1.915628in}}%
\pgfpathlineto{\pgfqpoint{2.929687in}{1.915628in}}%
\pgfpathlineto{\pgfqpoint{2.929687in}{1.912678in}}%
\pgfpathmoveto{\pgfqpoint{2.929687in}{1.912678in}}%
\pgfpathlineto{\pgfqpoint{2.929687in}{1.912678in}}%
\pgfpathlineto{\pgfqpoint{2.929687in}{1.915628in}}%
\pgfpathlineto{\pgfqpoint{2.934228in}{1.915628in}}%
\pgfpathlineto{\pgfqpoint{2.934228in}{1.912678in}}%
\pgfpathmoveto{\pgfqpoint{2.929687in}{1.915628in}}%
\pgfpathlineto{\pgfqpoint{2.929687in}{1.915628in}}%
\pgfpathlineto{\pgfqpoint{2.929687in}{1.918577in}}%
\pgfpathlineto{\pgfqpoint{2.934228in}{1.918577in}}%
\pgfpathlineto{\pgfqpoint{2.934228in}{1.915628in}}%
\pgfpathmoveto{\pgfqpoint{2.934228in}{1.915628in}}%
\pgfpathlineto{\pgfqpoint{2.934228in}{1.915628in}}%
\pgfpathlineto{\pgfqpoint{2.934228in}{1.918577in}}%
\pgfpathlineto{\pgfqpoint{2.938769in}{1.918577in}}%
\pgfpathlineto{\pgfqpoint{2.938769in}{1.915628in}}%
\pgfpathmoveto{\pgfqpoint{2.934228in}{1.918577in}}%
\pgfpathlineto{\pgfqpoint{2.934228in}{1.918577in}}%
\pgfpathlineto{\pgfqpoint{2.934228in}{1.921526in}}%
\pgfpathlineto{\pgfqpoint{2.938769in}{1.921526in}}%
\pgfpathlineto{\pgfqpoint{2.938769in}{1.918577in}}%
\pgfpathmoveto{\pgfqpoint{2.938769in}{1.918577in}}%
\pgfpathlineto{\pgfqpoint{2.938769in}{1.918577in}}%
\pgfpathlineto{\pgfqpoint{2.938769in}{1.921526in}}%
\pgfpathlineto{\pgfqpoint{2.943310in}{1.921526in}}%
\pgfpathlineto{\pgfqpoint{2.943310in}{1.918577in}}%
\pgfpathmoveto{\pgfqpoint{2.938769in}{1.921526in}}%
\pgfpathlineto{\pgfqpoint{2.938769in}{1.921526in}}%
\pgfpathlineto{\pgfqpoint{2.938769in}{1.924475in}}%
\pgfpathlineto{\pgfqpoint{2.943310in}{1.924475in}}%
\pgfpathlineto{\pgfqpoint{2.943310in}{1.921526in}}%
\pgfpathmoveto{\pgfqpoint{2.943310in}{1.921526in}}%
\pgfpathlineto{\pgfqpoint{2.943310in}{1.921526in}}%
\pgfpathlineto{\pgfqpoint{2.943310in}{1.924475in}}%
\pgfpathlineto{\pgfqpoint{2.947851in}{1.924475in}}%
\pgfpathlineto{\pgfqpoint{2.947851in}{1.921526in}}%
\pgfpathmoveto{\pgfqpoint{2.943310in}{1.924475in}}%
\pgfpathlineto{\pgfqpoint{2.943310in}{1.924475in}}%
\pgfpathlineto{\pgfqpoint{2.943310in}{1.927425in}}%
\pgfpathlineto{\pgfqpoint{2.947851in}{1.927425in}}%
\pgfpathlineto{\pgfqpoint{2.947851in}{1.924475in}}%
\pgfpathmoveto{\pgfqpoint{2.947851in}{1.924475in}}%
\pgfpathlineto{\pgfqpoint{2.947851in}{1.924475in}}%
\pgfpathlineto{\pgfqpoint{2.947851in}{1.927425in}}%
\pgfpathlineto{\pgfqpoint{2.952392in}{1.927425in}}%
\pgfpathlineto{\pgfqpoint{2.952392in}{1.924475in}}%
\pgfpathmoveto{\pgfqpoint{2.947851in}{1.927425in}}%
\pgfpathlineto{\pgfqpoint{2.947851in}{1.927425in}}%
\pgfpathlineto{\pgfqpoint{2.947851in}{1.930374in}}%
\pgfpathlineto{\pgfqpoint{2.952392in}{1.930374in}}%
\pgfpathlineto{\pgfqpoint{2.952392in}{1.927425in}}%
\pgfpathmoveto{\pgfqpoint{2.952392in}{1.927425in}}%
\pgfpathlineto{\pgfqpoint{2.952392in}{1.927425in}}%
\pgfpathlineto{\pgfqpoint{2.952392in}{1.930374in}}%
\pgfpathlineto{\pgfqpoint{2.956933in}{1.930374in}}%
\pgfpathlineto{\pgfqpoint{2.956933in}{1.927425in}}%
\pgfpathmoveto{\pgfqpoint{2.952392in}{1.930374in}}%
\pgfpathlineto{\pgfqpoint{2.952392in}{1.930374in}}%
\pgfpathlineto{\pgfqpoint{2.952392in}{1.933323in}}%
\pgfpathlineto{\pgfqpoint{2.956933in}{1.933323in}}%
\pgfpathlineto{\pgfqpoint{2.956933in}{1.930374in}}%
\pgfpathmoveto{\pgfqpoint{2.956933in}{1.930374in}}%
\pgfpathlineto{\pgfqpoint{2.956933in}{1.930374in}}%
\pgfpathlineto{\pgfqpoint{2.956933in}{1.933323in}}%
\pgfpathlineto{\pgfqpoint{2.961474in}{1.933323in}}%
\pgfpathlineto{\pgfqpoint{2.961474in}{1.930374in}}%
\pgfpathmoveto{\pgfqpoint{2.956933in}{1.933323in}}%
\pgfpathlineto{\pgfqpoint{2.956933in}{1.933323in}}%
\pgfpathlineto{\pgfqpoint{2.956933in}{1.936272in}}%
\pgfpathlineto{\pgfqpoint{2.961474in}{1.936272in}}%
\pgfpathlineto{\pgfqpoint{2.961474in}{1.933323in}}%
\pgfpathmoveto{\pgfqpoint{2.961474in}{1.933323in}}%
\pgfpathlineto{\pgfqpoint{2.961474in}{1.933323in}}%
\pgfpathlineto{\pgfqpoint{2.961474in}{1.936272in}}%
\pgfpathlineto{\pgfqpoint{2.966015in}{1.936272in}}%
\pgfpathlineto{\pgfqpoint{2.966015in}{1.933323in}}%
\pgfpathmoveto{\pgfqpoint{2.961474in}{1.936272in}}%
\pgfpathlineto{\pgfqpoint{2.961474in}{1.936272in}}%
\pgfpathlineto{\pgfqpoint{2.961474in}{1.939221in}}%
\pgfpathlineto{\pgfqpoint{2.966015in}{1.939221in}}%
\pgfpathlineto{\pgfqpoint{2.966015in}{1.936272in}}%
\pgfpathmoveto{\pgfqpoint{2.966015in}{1.936272in}}%
\pgfpathlineto{\pgfqpoint{2.966015in}{1.936272in}}%
\pgfpathlineto{\pgfqpoint{2.966015in}{1.939221in}}%
\pgfpathlineto{\pgfqpoint{2.970556in}{1.939221in}}%
\pgfpathlineto{\pgfqpoint{2.970556in}{1.936272in}}%
\pgfpathmoveto{\pgfqpoint{2.966015in}{1.939221in}}%
\pgfpathlineto{\pgfqpoint{2.966015in}{1.939221in}}%
\pgfpathlineto{\pgfqpoint{2.966015in}{1.942171in}}%
\pgfpathlineto{\pgfqpoint{2.970556in}{1.942171in}}%
\pgfpathlineto{\pgfqpoint{2.970556in}{1.939221in}}%
\pgfpathmoveto{\pgfqpoint{2.970556in}{1.939221in}}%
\pgfpathlineto{\pgfqpoint{2.970556in}{1.939221in}}%
\pgfpathlineto{\pgfqpoint{2.970556in}{1.942171in}}%
\pgfpathlineto{\pgfqpoint{2.975097in}{1.942171in}}%
\pgfpathlineto{\pgfqpoint{2.975097in}{1.939221in}}%
\pgfpathmoveto{\pgfqpoint{2.970556in}{1.942171in}}%
\pgfpathlineto{\pgfqpoint{2.970556in}{1.942171in}}%
\pgfpathlineto{\pgfqpoint{2.970556in}{1.945120in}}%
\pgfpathlineto{\pgfqpoint{2.975097in}{1.945120in}}%
\pgfpathlineto{\pgfqpoint{2.975097in}{1.942171in}}%
\pgfpathmoveto{\pgfqpoint{2.975097in}{1.942171in}}%
\pgfpathlineto{\pgfqpoint{2.975097in}{1.942171in}}%
\pgfpathlineto{\pgfqpoint{2.975097in}{1.945120in}}%
\pgfpathlineto{\pgfqpoint{2.979638in}{1.945120in}}%
\pgfpathlineto{\pgfqpoint{2.979638in}{1.942171in}}%
\pgfpathmoveto{\pgfqpoint{2.975097in}{1.945120in}}%
\pgfpathlineto{\pgfqpoint{2.975097in}{1.945120in}}%
\pgfpathlineto{\pgfqpoint{2.975097in}{1.948069in}}%
\pgfpathlineto{\pgfqpoint{2.979638in}{1.948069in}}%
\pgfpathlineto{\pgfqpoint{2.979638in}{1.945120in}}%
\pgfpathmoveto{\pgfqpoint{2.979638in}{1.945120in}}%
\pgfpathlineto{\pgfqpoint{2.979638in}{1.945120in}}%
\pgfpathlineto{\pgfqpoint{2.979638in}{1.948069in}}%
\pgfpathlineto{\pgfqpoint{2.984179in}{1.948069in}}%
\pgfpathlineto{\pgfqpoint{2.984179in}{1.945120in}}%
\pgfpathmoveto{\pgfqpoint{2.979638in}{1.948069in}}%
\pgfpathlineto{\pgfqpoint{2.979638in}{1.948069in}}%
\pgfpathlineto{\pgfqpoint{2.979638in}{1.951018in}}%
\pgfpathlineto{\pgfqpoint{2.984179in}{1.951018in}}%
\pgfpathlineto{\pgfqpoint{2.984179in}{1.948069in}}%
\pgfpathmoveto{\pgfqpoint{2.984179in}{1.948069in}}%
\pgfpathlineto{\pgfqpoint{2.984179in}{1.948069in}}%
\pgfpathlineto{\pgfqpoint{2.984179in}{1.951018in}}%
\pgfpathlineto{\pgfqpoint{2.988721in}{1.951018in}}%
\pgfpathlineto{\pgfqpoint{2.988721in}{1.948069in}}%
\pgfpathmoveto{\pgfqpoint{2.984179in}{1.951018in}}%
\pgfpathlineto{\pgfqpoint{2.984179in}{1.951018in}}%
\pgfpathlineto{\pgfqpoint{2.984179in}{1.953967in}}%
\pgfpathlineto{\pgfqpoint{2.988721in}{1.953967in}}%
\pgfpathlineto{\pgfqpoint{2.988721in}{1.951018in}}%
\pgfpathmoveto{\pgfqpoint{2.988721in}{1.951018in}}%
\pgfpathlineto{\pgfqpoint{2.988721in}{1.951018in}}%
\pgfpathlineto{\pgfqpoint{2.988721in}{1.953967in}}%
\pgfpathlineto{\pgfqpoint{2.993262in}{1.953967in}}%
\pgfpathlineto{\pgfqpoint{2.993262in}{1.951018in}}%
\pgfpathmoveto{\pgfqpoint{2.988721in}{1.953967in}}%
\pgfpathlineto{\pgfqpoint{2.988721in}{1.953967in}}%
\pgfpathlineto{\pgfqpoint{2.988721in}{1.956917in}}%
\pgfpathlineto{\pgfqpoint{2.993262in}{1.956917in}}%
\pgfpathlineto{\pgfqpoint{2.993262in}{1.953967in}}%
\pgfpathmoveto{\pgfqpoint{2.993262in}{1.953967in}}%
\pgfpathlineto{\pgfqpoint{2.993262in}{1.953967in}}%
\pgfpathlineto{\pgfqpoint{2.993262in}{1.956917in}}%
\pgfpathlineto{\pgfqpoint{2.997803in}{1.956917in}}%
\pgfpathlineto{\pgfqpoint{2.997803in}{1.953967in}}%
\pgfpathmoveto{\pgfqpoint{2.993262in}{1.956917in}}%
\pgfpathlineto{\pgfqpoint{2.993262in}{1.956917in}}%
\pgfpathlineto{\pgfqpoint{2.993262in}{1.959866in}}%
\pgfpathlineto{\pgfqpoint{2.997803in}{1.959866in}}%
\pgfpathlineto{\pgfqpoint{2.997803in}{1.956917in}}%
\pgfpathmoveto{\pgfqpoint{2.997803in}{1.956917in}}%
\pgfpathlineto{\pgfqpoint{2.997803in}{1.956917in}}%
\pgfpathlineto{\pgfqpoint{2.997803in}{1.959866in}}%
\pgfpathlineto{\pgfqpoint{3.002344in}{1.959866in}}%
\pgfpathlineto{\pgfqpoint{3.002344in}{1.956917in}}%
\pgfpathmoveto{\pgfqpoint{2.997803in}{1.959866in}}%
\pgfpathlineto{\pgfqpoint{2.997803in}{1.959866in}}%
\pgfpathlineto{\pgfqpoint{2.997803in}{1.962815in}}%
\pgfpathlineto{\pgfqpoint{3.002344in}{1.962815in}}%
\pgfpathlineto{\pgfqpoint{3.002344in}{1.959866in}}%
\pgfpathmoveto{\pgfqpoint{3.002344in}{1.959866in}}%
\pgfpathlineto{\pgfqpoint{3.002344in}{1.959866in}}%
\pgfpathlineto{\pgfqpoint{3.002344in}{1.962815in}}%
\pgfpathlineto{\pgfqpoint{3.006885in}{1.962815in}}%
\pgfpathlineto{\pgfqpoint{3.006885in}{1.959866in}}%
\pgfpathmoveto{\pgfqpoint{3.002344in}{1.962815in}}%
\pgfpathlineto{\pgfqpoint{3.002344in}{1.962815in}}%
\pgfpathlineto{\pgfqpoint{3.002344in}{1.965764in}}%
\pgfpathlineto{\pgfqpoint{3.006885in}{1.965764in}}%
\pgfpathlineto{\pgfqpoint{3.006885in}{1.962815in}}%
\pgfpathmoveto{\pgfqpoint{3.006885in}{1.962815in}}%
\pgfpathlineto{\pgfqpoint{3.006885in}{1.962815in}}%
\pgfpathlineto{\pgfqpoint{3.006885in}{1.965764in}}%
\pgfpathlineto{\pgfqpoint{3.011426in}{1.965764in}}%
\pgfpathlineto{\pgfqpoint{3.011426in}{1.962815in}}%
\pgfpathmoveto{\pgfqpoint{3.006885in}{1.965764in}}%
\pgfpathlineto{\pgfqpoint{3.006885in}{1.965764in}}%
\pgfpathlineto{\pgfqpoint{3.006885in}{1.968713in}}%
\pgfpathlineto{\pgfqpoint{3.011426in}{1.968713in}}%
\pgfpathlineto{\pgfqpoint{3.011426in}{1.965764in}}%
\pgfpathmoveto{\pgfqpoint{3.011426in}{1.965764in}}%
\pgfpathlineto{\pgfqpoint{3.011426in}{1.965764in}}%
\pgfpathlineto{\pgfqpoint{3.011426in}{1.968713in}}%
\pgfpathlineto{\pgfqpoint{3.015967in}{1.968713in}}%
\pgfpathlineto{\pgfqpoint{3.015967in}{1.965764in}}%
\pgfpathmoveto{\pgfqpoint{3.011426in}{1.968713in}}%
\pgfpathlineto{\pgfqpoint{3.011426in}{1.968713in}}%
\pgfpathlineto{\pgfqpoint{3.011426in}{1.971663in}}%
\pgfpathlineto{\pgfqpoint{3.015967in}{1.971663in}}%
\pgfpathlineto{\pgfqpoint{3.015967in}{1.968713in}}%
\pgfpathmoveto{\pgfqpoint{3.015967in}{1.968713in}}%
\pgfpathlineto{\pgfqpoint{3.015967in}{1.968713in}}%
\pgfpathlineto{\pgfqpoint{3.015967in}{1.971663in}}%
\pgfpathlineto{\pgfqpoint{3.020508in}{1.971663in}}%
\pgfpathlineto{\pgfqpoint{3.020508in}{1.968713in}}%
\pgfpathmoveto{\pgfqpoint{3.015967in}{1.971663in}}%
\pgfpathlineto{\pgfqpoint{3.015967in}{1.971663in}}%
\pgfpathlineto{\pgfqpoint{3.015967in}{1.974612in}}%
\pgfpathlineto{\pgfqpoint{3.020508in}{1.974612in}}%
\pgfpathlineto{\pgfqpoint{3.020508in}{1.971663in}}%
\pgfpathmoveto{\pgfqpoint{3.020508in}{1.971663in}}%
\pgfpathlineto{\pgfqpoint{3.020508in}{1.971663in}}%
\pgfpathlineto{\pgfqpoint{3.020508in}{1.974612in}}%
\pgfpathlineto{\pgfqpoint{3.025049in}{1.974612in}}%
\pgfpathlineto{\pgfqpoint{3.025049in}{1.971663in}}%
\pgfpathmoveto{\pgfqpoint{3.020508in}{1.974612in}}%
\pgfpathlineto{\pgfqpoint{3.020508in}{1.974612in}}%
\pgfpathlineto{\pgfqpoint{3.020508in}{1.977561in}}%
\pgfpathlineto{\pgfqpoint{3.025049in}{1.977561in}}%
\pgfpathlineto{\pgfqpoint{3.025049in}{1.974612in}}%
\pgfpathmoveto{\pgfqpoint{3.025049in}{1.974612in}}%
\pgfpathlineto{\pgfqpoint{3.025049in}{1.974612in}}%
\pgfpathlineto{\pgfqpoint{3.025049in}{1.977561in}}%
\pgfpathlineto{\pgfqpoint{3.029590in}{1.977561in}}%
\pgfpathlineto{\pgfqpoint{3.029590in}{1.974612in}}%
\pgfpathmoveto{\pgfqpoint{3.025049in}{1.977561in}}%
\pgfpathlineto{\pgfqpoint{3.025049in}{1.977561in}}%
\pgfpathlineto{\pgfqpoint{3.025049in}{1.980510in}}%
\pgfpathlineto{\pgfqpoint{3.029590in}{1.980510in}}%
\pgfpathlineto{\pgfqpoint{3.029590in}{1.977561in}}%
\pgfpathmoveto{\pgfqpoint{3.029590in}{1.977561in}}%
\pgfpathlineto{\pgfqpoint{3.029590in}{1.977561in}}%
\pgfpathlineto{\pgfqpoint{3.029590in}{1.980510in}}%
\pgfpathlineto{\pgfqpoint{3.034131in}{1.980510in}}%
\pgfpathlineto{\pgfqpoint{3.034131in}{1.977561in}}%
\pgfpathmoveto{\pgfqpoint{3.029590in}{1.980510in}}%
\pgfpathlineto{\pgfqpoint{3.029590in}{1.980510in}}%
\pgfpathlineto{\pgfqpoint{3.029590in}{1.983459in}}%
\pgfpathlineto{\pgfqpoint{3.034131in}{1.983459in}}%
\pgfpathlineto{\pgfqpoint{3.034131in}{1.980510in}}%
\pgfpathmoveto{\pgfqpoint{3.034131in}{1.980510in}}%
\pgfpathlineto{\pgfqpoint{3.034131in}{1.980510in}}%
\pgfpathlineto{\pgfqpoint{3.034131in}{1.983459in}}%
\pgfpathlineto{\pgfqpoint{3.038672in}{1.983459in}}%
\pgfpathlineto{\pgfqpoint{3.038672in}{1.980510in}}%
\pgfpathmoveto{\pgfqpoint{3.034131in}{1.983459in}}%
\pgfpathlineto{\pgfqpoint{3.034131in}{1.983459in}}%
\pgfpathlineto{\pgfqpoint{3.034131in}{1.986409in}}%
\pgfpathlineto{\pgfqpoint{3.038672in}{1.986409in}}%
\pgfpathlineto{\pgfqpoint{3.038672in}{1.983459in}}%
\pgfpathmoveto{\pgfqpoint{3.038672in}{1.983459in}}%
\pgfpathlineto{\pgfqpoint{3.038672in}{1.983459in}}%
\pgfpathlineto{\pgfqpoint{3.038672in}{1.986409in}}%
\pgfpathlineto{\pgfqpoint{3.043213in}{1.986409in}}%
\pgfpathlineto{\pgfqpoint{3.043213in}{1.983459in}}%
\pgfpathmoveto{\pgfqpoint{3.038672in}{1.986409in}}%
\pgfpathlineto{\pgfqpoint{3.038672in}{1.986409in}}%
\pgfpathlineto{\pgfqpoint{3.038672in}{1.989358in}}%
\pgfpathlineto{\pgfqpoint{3.043213in}{1.989358in}}%
\pgfpathlineto{\pgfqpoint{3.043213in}{1.986409in}}%
\pgfpathmoveto{\pgfqpoint{3.043213in}{1.986409in}}%
\pgfpathlineto{\pgfqpoint{3.043213in}{1.986409in}}%
\pgfpathlineto{\pgfqpoint{3.043213in}{1.989358in}}%
\pgfpathlineto{\pgfqpoint{3.047754in}{1.989358in}}%
\pgfpathlineto{\pgfqpoint{3.047754in}{1.986409in}}%
\pgfpathmoveto{\pgfqpoint{3.043213in}{1.989358in}}%
\pgfpathlineto{\pgfqpoint{3.043213in}{1.989358in}}%
\pgfpathlineto{\pgfqpoint{3.043213in}{1.992307in}}%
\pgfpathlineto{\pgfqpoint{3.047754in}{1.992307in}}%
\pgfpathlineto{\pgfqpoint{3.047754in}{1.989358in}}%
\pgfpathmoveto{\pgfqpoint{3.047754in}{1.989358in}}%
\pgfpathlineto{\pgfqpoint{3.047754in}{1.989358in}}%
\pgfpathlineto{\pgfqpoint{3.047754in}{1.992307in}}%
\pgfpathlineto{\pgfqpoint{3.052295in}{1.992307in}}%
\pgfpathlineto{\pgfqpoint{3.052295in}{1.989358in}}%
\pgfpathmoveto{\pgfqpoint{3.047754in}{1.992307in}}%
\pgfpathlineto{\pgfqpoint{3.047754in}{1.992307in}}%
\pgfpathlineto{\pgfqpoint{3.047754in}{1.995256in}}%
\pgfpathlineto{\pgfqpoint{3.052295in}{1.995256in}}%
\pgfpathlineto{\pgfqpoint{3.052295in}{1.992307in}}%
\pgfpathmoveto{\pgfqpoint{3.052295in}{1.992307in}}%
\pgfpathlineto{\pgfqpoint{3.052295in}{1.992307in}}%
\pgfpathlineto{\pgfqpoint{3.052295in}{1.995256in}}%
\pgfpathlineto{\pgfqpoint{3.056836in}{1.995256in}}%
\pgfpathlineto{\pgfqpoint{3.056836in}{1.992307in}}%
\pgfpathmoveto{\pgfqpoint{3.052295in}{1.995256in}}%
\pgfpathlineto{\pgfqpoint{3.052295in}{1.995256in}}%
\pgfpathlineto{\pgfqpoint{3.052295in}{1.998205in}}%
\pgfpathlineto{\pgfqpoint{3.056836in}{1.998205in}}%
\pgfpathlineto{\pgfqpoint{3.056836in}{1.995256in}}%
\pgfpathmoveto{\pgfqpoint{3.056836in}{1.995256in}}%
\pgfpathlineto{\pgfqpoint{3.056836in}{1.995256in}}%
\pgfpathlineto{\pgfqpoint{3.056836in}{1.998205in}}%
\pgfpathlineto{\pgfqpoint{3.061377in}{1.998205in}}%
\pgfpathlineto{\pgfqpoint{3.061377in}{1.995256in}}%
\pgfpathmoveto{\pgfqpoint{3.056836in}{1.998205in}}%
\pgfpathlineto{\pgfqpoint{3.056836in}{1.998205in}}%
\pgfpathlineto{\pgfqpoint{3.056836in}{2.001155in}}%
\pgfpathlineto{\pgfqpoint{3.061377in}{2.001155in}}%
\pgfpathlineto{\pgfqpoint{3.061377in}{1.998205in}}%
\pgfpathmoveto{\pgfqpoint{3.061377in}{1.998205in}}%
\pgfpathlineto{\pgfqpoint{3.061377in}{1.998205in}}%
\pgfpathlineto{\pgfqpoint{3.061377in}{2.001155in}}%
\pgfpathlineto{\pgfqpoint{3.065918in}{2.001155in}}%
\pgfpathlineto{\pgfqpoint{3.065918in}{1.998205in}}%
\pgfpathmoveto{\pgfqpoint{3.061377in}{2.001155in}}%
\pgfpathlineto{\pgfqpoint{3.061377in}{2.001155in}}%
\pgfpathlineto{\pgfqpoint{3.061377in}{2.004104in}}%
\pgfpathlineto{\pgfqpoint{3.065918in}{2.004104in}}%
\pgfpathlineto{\pgfqpoint{3.065918in}{2.001155in}}%
\pgfpathmoveto{\pgfqpoint{3.065918in}{2.001155in}}%
\pgfpathlineto{\pgfqpoint{3.065918in}{2.001155in}}%
\pgfpathlineto{\pgfqpoint{3.065918in}{2.004104in}}%
\pgfpathlineto{\pgfqpoint{3.070459in}{2.004104in}}%
\pgfpathlineto{\pgfqpoint{3.070459in}{2.001155in}}%
\pgfpathmoveto{\pgfqpoint{3.065918in}{2.004104in}}%
\pgfpathlineto{\pgfqpoint{3.065918in}{2.004104in}}%
\pgfpathlineto{\pgfqpoint{3.065918in}{2.007053in}}%
\pgfpathlineto{\pgfqpoint{3.070459in}{2.007053in}}%
\pgfpathlineto{\pgfqpoint{3.070459in}{2.004104in}}%
\pgfpathmoveto{\pgfqpoint{3.070459in}{2.004104in}}%
\pgfpathlineto{\pgfqpoint{3.070459in}{2.004104in}}%
\pgfpathlineto{\pgfqpoint{3.070459in}{2.007053in}}%
\pgfpathlineto{\pgfqpoint{3.075000in}{2.007053in}}%
\pgfpathlineto{\pgfqpoint{3.075000in}{2.004104in}}%
\pgfpathmoveto{\pgfqpoint{3.070459in}{2.007053in}}%
\pgfpathlineto{\pgfqpoint{3.070459in}{2.007053in}}%
\pgfpathlineto{\pgfqpoint{3.070459in}{2.010002in}}%
\pgfpathlineto{\pgfqpoint{3.075000in}{2.010002in}}%
\pgfpathlineto{\pgfqpoint{3.075000in}{2.007053in}}%
\pgfpathmoveto{\pgfqpoint{3.075000in}{2.007053in}}%
\pgfpathlineto{\pgfqpoint{3.075000in}{2.007053in}}%
\pgfpathlineto{\pgfqpoint{3.075000in}{2.010002in}}%
\pgfpathlineto{\pgfqpoint{3.079541in}{2.010002in}}%
\pgfpathlineto{\pgfqpoint{3.079541in}{2.007053in}}%
\pgfpathmoveto{\pgfqpoint{3.075000in}{2.010002in}}%
\pgfpathlineto{\pgfqpoint{3.075000in}{2.010002in}}%
\pgfpathlineto{\pgfqpoint{3.075000in}{2.012951in}}%
\pgfpathlineto{\pgfqpoint{3.079541in}{2.012951in}}%
\pgfpathlineto{\pgfqpoint{3.079541in}{2.010002in}}%
\pgfpathmoveto{\pgfqpoint{3.079541in}{2.010002in}}%
\pgfpathlineto{\pgfqpoint{3.079541in}{2.010002in}}%
\pgfpathlineto{\pgfqpoint{3.079541in}{2.012951in}}%
\pgfpathlineto{\pgfqpoint{3.084082in}{2.012951in}}%
\pgfpathlineto{\pgfqpoint{3.084082in}{2.010002in}}%
\pgfpathmoveto{\pgfqpoint{3.079541in}{2.012951in}}%
\pgfpathlineto{\pgfqpoint{3.079541in}{2.012951in}}%
\pgfpathlineto{\pgfqpoint{3.079541in}{2.015901in}}%
\pgfpathlineto{\pgfqpoint{3.084082in}{2.015901in}}%
\pgfpathlineto{\pgfqpoint{3.084082in}{2.012951in}}%
\pgfpathmoveto{\pgfqpoint{3.084082in}{2.012951in}}%
\pgfpathlineto{\pgfqpoint{3.084082in}{2.012951in}}%
\pgfpathlineto{\pgfqpoint{3.084082in}{2.015901in}}%
\pgfpathlineto{\pgfqpoint{3.088623in}{2.015901in}}%
\pgfpathlineto{\pgfqpoint{3.088623in}{2.012951in}}%
\pgfpathmoveto{\pgfqpoint{3.084082in}{2.015901in}}%
\pgfpathlineto{\pgfqpoint{3.084082in}{2.015901in}}%
\pgfpathlineto{\pgfqpoint{3.084082in}{2.018850in}}%
\pgfpathlineto{\pgfqpoint{3.088623in}{2.018850in}}%
\pgfpathlineto{\pgfqpoint{3.088623in}{2.015901in}}%
\pgfpathmoveto{\pgfqpoint{3.088623in}{2.015901in}}%
\pgfpathlineto{\pgfqpoint{3.088623in}{2.015901in}}%
\pgfpathlineto{\pgfqpoint{3.088623in}{2.018850in}}%
\pgfpathlineto{\pgfqpoint{3.093164in}{2.018850in}}%
\pgfpathlineto{\pgfqpoint{3.093164in}{2.015901in}}%
\pgfpathmoveto{\pgfqpoint{3.088623in}{2.018850in}}%
\pgfpathlineto{\pgfqpoint{3.088623in}{2.018850in}}%
\pgfpathlineto{\pgfqpoint{3.088623in}{2.021799in}}%
\pgfpathlineto{\pgfqpoint{3.093164in}{2.021799in}}%
\pgfpathlineto{\pgfqpoint{3.093164in}{2.018850in}}%
\pgfpathmoveto{\pgfqpoint{3.093164in}{2.018850in}}%
\pgfpathlineto{\pgfqpoint{3.093164in}{2.018850in}}%
\pgfpathlineto{\pgfqpoint{3.093164in}{2.021799in}}%
\pgfpathlineto{\pgfqpoint{3.097705in}{2.021799in}}%
\pgfpathlineto{\pgfqpoint{3.097705in}{2.018850in}}%
\pgfpathmoveto{\pgfqpoint{3.093164in}{2.021799in}}%
\pgfpathlineto{\pgfqpoint{3.093164in}{2.021799in}}%
\pgfpathlineto{\pgfqpoint{3.093164in}{2.024748in}}%
\pgfpathlineto{\pgfqpoint{3.097705in}{2.024748in}}%
\pgfpathlineto{\pgfqpoint{3.097705in}{2.021799in}}%
\pgfpathmoveto{\pgfqpoint{3.097705in}{2.021799in}}%
\pgfpathlineto{\pgfqpoint{3.097705in}{2.021799in}}%
\pgfpathlineto{\pgfqpoint{3.097705in}{2.024748in}}%
\pgfpathlineto{\pgfqpoint{3.102247in}{2.024748in}}%
\pgfpathlineto{\pgfqpoint{3.102247in}{2.021799in}}%
\pgfpathmoveto{\pgfqpoint{3.097705in}{2.024748in}}%
\pgfpathlineto{\pgfqpoint{3.097705in}{2.024748in}}%
\pgfpathlineto{\pgfqpoint{3.097705in}{2.027697in}}%
\pgfpathlineto{\pgfqpoint{3.102247in}{2.027697in}}%
\pgfpathlineto{\pgfqpoint{3.102247in}{2.024748in}}%
\pgfpathmoveto{\pgfqpoint{3.102247in}{2.024748in}}%
\pgfpathlineto{\pgfqpoint{3.102247in}{2.024748in}}%
\pgfpathlineto{\pgfqpoint{3.102247in}{2.027697in}}%
\pgfpathlineto{\pgfqpoint{3.106788in}{2.027697in}}%
\pgfpathlineto{\pgfqpoint{3.106788in}{2.024748in}}%
\pgfpathmoveto{\pgfqpoint{3.102247in}{2.027697in}}%
\pgfpathlineto{\pgfqpoint{3.102247in}{2.027697in}}%
\pgfpathlineto{\pgfqpoint{3.102247in}{2.030646in}}%
\pgfpathlineto{\pgfqpoint{3.106788in}{2.030646in}}%
\pgfpathlineto{\pgfqpoint{3.106788in}{2.027697in}}%
\pgfpathmoveto{\pgfqpoint{3.106788in}{2.027697in}}%
\pgfpathlineto{\pgfqpoint{3.106788in}{2.027697in}}%
\pgfpathlineto{\pgfqpoint{3.106788in}{2.030646in}}%
\pgfpathlineto{\pgfqpoint{3.111329in}{2.030646in}}%
\pgfpathlineto{\pgfqpoint{3.111329in}{2.027697in}}%
\pgfpathmoveto{\pgfqpoint{3.106788in}{2.030646in}}%
\pgfpathlineto{\pgfqpoint{3.106788in}{2.030646in}}%
\pgfpathlineto{\pgfqpoint{3.106788in}{2.033596in}}%
\pgfpathlineto{\pgfqpoint{3.111329in}{2.033596in}}%
\pgfpathlineto{\pgfqpoint{3.111329in}{2.030646in}}%
\pgfpathmoveto{\pgfqpoint{3.111329in}{2.030646in}}%
\pgfpathlineto{\pgfqpoint{3.111329in}{2.030646in}}%
\pgfpathlineto{\pgfqpoint{3.111329in}{2.033596in}}%
\pgfpathlineto{\pgfqpoint{3.115870in}{2.033596in}}%
\pgfpathlineto{\pgfqpoint{3.115870in}{2.030646in}}%
\pgfpathmoveto{\pgfqpoint{3.111329in}{2.033596in}}%
\pgfpathlineto{\pgfqpoint{3.111329in}{2.033596in}}%
\pgfpathlineto{\pgfqpoint{3.111329in}{2.036545in}}%
\pgfpathlineto{\pgfqpoint{3.115870in}{2.036545in}}%
\pgfpathlineto{\pgfqpoint{3.115870in}{2.033596in}}%
\pgfpathmoveto{\pgfqpoint{3.115870in}{2.033596in}}%
\pgfpathlineto{\pgfqpoint{3.115870in}{2.033596in}}%
\pgfpathlineto{\pgfqpoint{3.115870in}{2.036545in}}%
\pgfpathlineto{\pgfqpoint{3.120411in}{2.036545in}}%
\pgfpathlineto{\pgfqpoint{3.120411in}{2.033596in}}%
\pgfpathmoveto{\pgfqpoint{3.115870in}{2.036545in}}%
\pgfpathlineto{\pgfqpoint{3.115870in}{2.036545in}}%
\pgfpathlineto{\pgfqpoint{3.115870in}{2.039494in}}%
\pgfpathlineto{\pgfqpoint{3.120411in}{2.039494in}}%
\pgfpathlineto{\pgfqpoint{3.120411in}{2.036545in}}%
\pgfpathmoveto{\pgfqpoint{3.120411in}{2.036545in}}%
\pgfpathlineto{\pgfqpoint{3.120411in}{2.036545in}}%
\pgfpathlineto{\pgfqpoint{3.120411in}{2.039494in}}%
\pgfpathlineto{\pgfqpoint{3.124952in}{2.039494in}}%
\pgfpathlineto{\pgfqpoint{3.124952in}{2.036545in}}%
\pgfpathmoveto{\pgfqpoint{3.120411in}{2.039494in}}%
\pgfpathlineto{\pgfqpoint{3.120411in}{2.039494in}}%
\pgfpathlineto{\pgfqpoint{3.120411in}{2.042443in}}%
\pgfpathlineto{\pgfqpoint{3.124952in}{2.042443in}}%
\pgfpathlineto{\pgfqpoint{3.124952in}{2.039494in}}%
\pgfpathmoveto{\pgfqpoint{3.124952in}{2.039494in}}%
\pgfpathlineto{\pgfqpoint{3.124952in}{2.039494in}}%
\pgfpathlineto{\pgfqpoint{3.124952in}{2.042443in}}%
\pgfpathlineto{\pgfqpoint{3.129493in}{2.042443in}}%
\pgfpathlineto{\pgfqpoint{3.129493in}{2.039494in}}%
\pgfpathmoveto{\pgfqpoint{3.124952in}{2.042443in}}%
\pgfpathlineto{\pgfqpoint{3.124952in}{2.042443in}}%
\pgfpathlineto{\pgfqpoint{3.124952in}{2.045392in}}%
\pgfpathlineto{\pgfqpoint{3.129493in}{2.045392in}}%
\pgfpathlineto{\pgfqpoint{3.129493in}{2.042443in}}%
\pgfpathmoveto{\pgfqpoint{3.129493in}{2.042443in}}%
\pgfpathlineto{\pgfqpoint{3.129493in}{2.042443in}}%
\pgfpathlineto{\pgfqpoint{3.129493in}{2.045392in}}%
\pgfpathlineto{\pgfqpoint{3.134034in}{2.045392in}}%
\pgfpathlineto{\pgfqpoint{3.134034in}{2.042443in}}%
\pgfpathmoveto{\pgfqpoint{3.129493in}{2.045392in}}%
\pgfpathlineto{\pgfqpoint{3.129493in}{2.045392in}}%
\pgfpathlineto{\pgfqpoint{3.129493in}{2.048341in}}%
\pgfpathlineto{\pgfqpoint{3.134034in}{2.048341in}}%
\pgfpathlineto{\pgfqpoint{3.134034in}{2.045392in}}%
\pgfpathmoveto{\pgfqpoint{3.134034in}{2.045392in}}%
\pgfpathlineto{\pgfqpoint{3.134034in}{2.045392in}}%
\pgfpathlineto{\pgfqpoint{3.134034in}{2.048341in}}%
\pgfpathlineto{\pgfqpoint{3.138575in}{2.048341in}}%
\pgfpathlineto{\pgfqpoint{3.138575in}{2.045392in}}%
\pgfpathmoveto{\pgfqpoint{3.134034in}{2.048341in}}%
\pgfpathlineto{\pgfqpoint{3.134034in}{2.048341in}}%
\pgfpathlineto{\pgfqpoint{3.134034in}{2.051291in}}%
\pgfpathlineto{\pgfqpoint{3.138575in}{2.051291in}}%
\pgfpathlineto{\pgfqpoint{3.138575in}{2.048341in}}%
\pgfpathmoveto{\pgfqpoint{3.138575in}{2.048341in}}%
\pgfpathlineto{\pgfqpoint{3.138575in}{2.048341in}}%
\pgfpathlineto{\pgfqpoint{3.138575in}{2.051291in}}%
\pgfpathlineto{\pgfqpoint{3.143116in}{2.051291in}}%
\pgfpathlineto{\pgfqpoint{3.143116in}{2.048341in}}%
\pgfpathmoveto{\pgfqpoint{3.138575in}{2.051291in}}%
\pgfpathlineto{\pgfqpoint{3.138575in}{2.051291in}}%
\pgfpathlineto{\pgfqpoint{3.138575in}{2.054240in}}%
\pgfpathlineto{\pgfqpoint{3.143116in}{2.054240in}}%
\pgfpathlineto{\pgfqpoint{3.143116in}{2.051291in}}%
\pgfpathmoveto{\pgfqpoint{3.143116in}{2.051291in}}%
\pgfpathlineto{\pgfqpoint{3.143116in}{2.051291in}}%
\pgfpathlineto{\pgfqpoint{3.143116in}{2.054240in}}%
\pgfpathlineto{\pgfqpoint{3.147657in}{2.054240in}}%
\pgfpathlineto{\pgfqpoint{3.147657in}{2.051291in}}%
\pgfpathmoveto{\pgfqpoint{3.143116in}{2.054240in}}%
\pgfpathlineto{\pgfqpoint{3.143116in}{2.054240in}}%
\pgfpathlineto{\pgfqpoint{3.143116in}{2.057189in}}%
\pgfpathlineto{\pgfqpoint{3.147657in}{2.057189in}}%
\pgfpathlineto{\pgfqpoint{3.147657in}{2.054240in}}%
\pgfpathmoveto{\pgfqpoint{3.147657in}{2.054240in}}%
\pgfpathlineto{\pgfqpoint{3.147657in}{2.054240in}}%
\pgfpathlineto{\pgfqpoint{3.147657in}{2.057189in}}%
\pgfpathlineto{\pgfqpoint{3.152198in}{2.057189in}}%
\pgfpathlineto{\pgfqpoint{3.152198in}{2.054240in}}%
\pgfpathmoveto{\pgfqpoint{3.147657in}{2.057189in}}%
\pgfpathlineto{\pgfqpoint{3.147657in}{2.057189in}}%
\pgfpathlineto{\pgfqpoint{3.147657in}{2.060138in}}%
\pgfpathlineto{\pgfqpoint{3.152198in}{2.060138in}}%
\pgfpathlineto{\pgfqpoint{3.152198in}{2.057189in}}%
\pgfpathmoveto{\pgfqpoint{3.152198in}{2.057189in}}%
\pgfpathlineto{\pgfqpoint{3.152198in}{2.057189in}}%
\pgfpathlineto{\pgfqpoint{3.152198in}{2.060138in}}%
\pgfpathlineto{\pgfqpoint{3.156739in}{2.060138in}}%
\pgfpathlineto{\pgfqpoint{3.156739in}{2.057189in}}%
\pgfpathmoveto{\pgfqpoint{3.152198in}{2.060138in}}%
\pgfpathlineto{\pgfqpoint{3.152198in}{2.060138in}}%
\pgfpathlineto{\pgfqpoint{3.152198in}{2.063087in}}%
\pgfpathlineto{\pgfqpoint{3.156739in}{2.063087in}}%
\pgfpathlineto{\pgfqpoint{3.156739in}{2.060138in}}%
\pgfpathmoveto{\pgfqpoint{3.156739in}{2.060138in}}%
\pgfpathlineto{\pgfqpoint{3.156739in}{2.060138in}}%
\pgfpathlineto{\pgfqpoint{3.156739in}{2.063087in}}%
\pgfpathlineto{\pgfqpoint{3.161280in}{2.063087in}}%
\pgfpathlineto{\pgfqpoint{3.161280in}{2.060138in}}%
\pgfpathmoveto{\pgfqpoint{3.156739in}{2.063087in}}%
\pgfpathlineto{\pgfqpoint{3.156739in}{2.063087in}}%
\pgfpathlineto{\pgfqpoint{3.156739in}{2.066036in}}%
\pgfpathlineto{\pgfqpoint{3.161280in}{2.066036in}}%
\pgfpathlineto{\pgfqpoint{3.161280in}{2.063087in}}%
\pgfpathmoveto{\pgfqpoint{3.161280in}{2.063087in}}%
\pgfpathlineto{\pgfqpoint{3.161280in}{2.063087in}}%
\pgfpathlineto{\pgfqpoint{3.161280in}{2.066036in}}%
\pgfpathlineto{\pgfqpoint{3.165822in}{2.066036in}}%
\pgfpathlineto{\pgfqpoint{3.165822in}{2.063087in}}%
\pgfpathmoveto{\pgfqpoint{3.161280in}{2.066036in}}%
\pgfpathlineto{\pgfqpoint{3.161280in}{2.066036in}}%
\pgfpathlineto{\pgfqpoint{3.161280in}{2.068985in}}%
\pgfpathlineto{\pgfqpoint{3.165822in}{2.068985in}}%
\pgfpathlineto{\pgfqpoint{3.165822in}{2.066036in}}%
\pgfpathmoveto{\pgfqpoint{3.165822in}{2.066036in}}%
\pgfpathlineto{\pgfqpoint{3.165822in}{2.066036in}}%
\pgfpathlineto{\pgfqpoint{3.165822in}{2.068985in}}%
\pgfpathlineto{\pgfqpoint{3.170363in}{2.068985in}}%
\pgfpathlineto{\pgfqpoint{3.170363in}{2.066036in}}%
\pgfpathmoveto{\pgfqpoint{3.165822in}{2.068985in}}%
\pgfpathlineto{\pgfqpoint{3.165822in}{2.068985in}}%
\pgfpathlineto{\pgfqpoint{3.165822in}{2.071935in}}%
\pgfpathlineto{\pgfqpoint{3.170363in}{2.071935in}}%
\pgfpathlineto{\pgfqpoint{3.170363in}{2.068985in}}%
\pgfpathmoveto{\pgfqpoint{3.170363in}{2.068985in}}%
\pgfpathlineto{\pgfqpoint{3.170363in}{2.068985in}}%
\pgfpathlineto{\pgfqpoint{3.170363in}{2.071935in}}%
\pgfpathlineto{\pgfqpoint{3.174904in}{2.071935in}}%
\pgfpathlineto{\pgfqpoint{3.174904in}{2.068985in}}%
\pgfpathmoveto{\pgfqpoint{3.170363in}{2.071935in}}%
\pgfpathlineto{\pgfqpoint{3.170363in}{2.071935in}}%
\pgfpathlineto{\pgfqpoint{3.170363in}{2.074884in}}%
\pgfpathlineto{\pgfqpoint{3.174904in}{2.074884in}}%
\pgfpathlineto{\pgfqpoint{3.174904in}{2.071935in}}%
\pgfpathmoveto{\pgfqpoint{3.174904in}{2.071935in}}%
\pgfpathlineto{\pgfqpoint{3.174904in}{2.071935in}}%
\pgfpathlineto{\pgfqpoint{3.174904in}{2.074884in}}%
\pgfpathlineto{\pgfqpoint{3.179445in}{2.074884in}}%
\pgfpathlineto{\pgfqpoint{3.179445in}{2.071935in}}%
\pgfpathmoveto{\pgfqpoint{3.174904in}{2.074884in}}%
\pgfpathlineto{\pgfqpoint{3.174904in}{2.074884in}}%
\pgfpathlineto{\pgfqpoint{3.174904in}{2.077833in}}%
\pgfpathlineto{\pgfqpoint{3.179445in}{2.077833in}}%
\pgfpathlineto{\pgfqpoint{3.179445in}{2.074884in}}%
\pgfpathmoveto{\pgfqpoint{3.179445in}{2.074884in}}%
\pgfpathlineto{\pgfqpoint{3.179445in}{2.074884in}}%
\pgfpathlineto{\pgfqpoint{3.179445in}{2.077833in}}%
\pgfpathlineto{\pgfqpoint{3.183986in}{2.077833in}}%
\pgfpathlineto{\pgfqpoint{3.183986in}{2.074884in}}%
\pgfpathmoveto{\pgfqpoint{3.179445in}{2.077833in}}%
\pgfpathlineto{\pgfqpoint{3.179445in}{2.077833in}}%
\pgfpathlineto{\pgfqpoint{3.179445in}{2.080782in}}%
\pgfpathlineto{\pgfqpoint{3.183986in}{2.080782in}}%
\pgfpathlineto{\pgfqpoint{3.183986in}{2.077833in}}%
\pgfpathmoveto{\pgfqpoint{3.179445in}{2.080782in}}%
\pgfpathlineto{\pgfqpoint{3.179445in}{2.080782in}}%
\pgfpathlineto{\pgfqpoint{3.179445in}{2.083731in}}%
\pgfpathlineto{\pgfqpoint{3.183986in}{2.083731in}}%
\pgfpathlineto{\pgfqpoint{3.183986in}{2.080782in}}%
\pgfpathmoveto{\pgfqpoint{3.183986in}{2.080782in}}%
\pgfpathlineto{\pgfqpoint{3.183986in}{2.080782in}}%
\pgfpathlineto{\pgfqpoint{3.183986in}{2.083731in}}%
\pgfpathlineto{\pgfqpoint{3.188527in}{2.083731in}}%
\pgfpathlineto{\pgfqpoint{3.188527in}{2.080782in}}%
\pgfpathmoveto{\pgfqpoint{3.183986in}{2.083731in}}%
\pgfpathlineto{\pgfqpoint{3.183986in}{2.083731in}}%
\pgfpathlineto{\pgfqpoint{3.183986in}{2.086680in}}%
\pgfpathlineto{\pgfqpoint{3.188527in}{2.086680in}}%
\pgfpathlineto{\pgfqpoint{3.188527in}{2.083731in}}%
\pgfpathmoveto{\pgfqpoint{3.188527in}{2.083731in}}%
\pgfpathlineto{\pgfqpoint{3.188527in}{2.083731in}}%
\pgfpathlineto{\pgfqpoint{3.188527in}{2.086680in}}%
\pgfpathlineto{\pgfqpoint{3.193068in}{2.086680in}}%
\pgfpathlineto{\pgfqpoint{3.193068in}{2.083731in}}%
\pgfpathmoveto{\pgfqpoint{3.188527in}{2.086680in}}%
\pgfpathlineto{\pgfqpoint{3.188527in}{2.086680in}}%
\pgfpathlineto{\pgfqpoint{3.188527in}{2.089630in}}%
\pgfpathlineto{\pgfqpoint{3.193068in}{2.089630in}}%
\pgfpathlineto{\pgfqpoint{3.193068in}{2.086680in}}%
\pgfpathmoveto{\pgfqpoint{3.193068in}{2.086680in}}%
\pgfpathlineto{\pgfqpoint{3.193068in}{2.086680in}}%
\pgfpathlineto{\pgfqpoint{3.193068in}{2.089630in}}%
\pgfpathlineto{\pgfqpoint{3.197609in}{2.089630in}}%
\pgfpathlineto{\pgfqpoint{3.197609in}{2.086680in}}%
\pgfpathmoveto{\pgfqpoint{3.193068in}{2.089630in}}%
\pgfpathlineto{\pgfqpoint{3.193068in}{2.089630in}}%
\pgfpathlineto{\pgfqpoint{3.193068in}{2.092579in}}%
\pgfpathlineto{\pgfqpoint{3.197609in}{2.092579in}}%
\pgfpathlineto{\pgfqpoint{3.197609in}{2.089630in}}%
\pgfpathmoveto{\pgfqpoint{3.197609in}{2.089630in}}%
\pgfpathlineto{\pgfqpoint{3.197609in}{2.089630in}}%
\pgfpathlineto{\pgfqpoint{3.197609in}{2.092579in}}%
\pgfpathlineto{\pgfqpoint{3.202150in}{2.092579in}}%
\pgfpathlineto{\pgfqpoint{3.202150in}{2.089630in}}%
\pgfpathmoveto{\pgfqpoint{3.197609in}{2.092579in}}%
\pgfpathlineto{\pgfqpoint{3.197609in}{2.092579in}}%
\pgfpathlineto{\pgfqpoint{3.197609in}{2.095528in}}%
\pgfpathlineto{\pgfqpoint{3.202150in}{2.095528in}}%
\pgfpathlineto{\pgfqpoint{3.202150in}{2.092579in}}%
\pgfpathmoveto{\pgfqpoint{3.202150in}{2.092579in}}%
\pgfpathlineto{\pgfqpoint{3.202150in}{2.092579in}}%
\pgfpathlineto{\pgfqpoint{3.202150in}{2.095528in}}%
\pgfpathlineto{\pgfqpoint{3.206691in}{2.095528in}}%
\pgfpathlineto{\pgfqpoint{3.206691in}{2.092579in}}%
\pgfpathmoveto{\pgfqpoint{3.202150in}{2.095528in}}%
\pgfpathlineto{\pgfqpoint{3.202150in}{2.095528in}}%
\pgfpathlineto{\pgfqpoint{3.202150in}{2.098477in}}%
\pgfpathlineto{\pgfqpoint{3.206691in}{2.098477in}}%
\pgfpathlineto{\pgfqpoint{3.206691in}{2.095528in}}%
\pgfpathmoveto{\pgfqpoint{3.206691in}{2.095528in}}%
\pgfpathlineto{\pgfqpoint{3.206691in}{2.095528in}}%
\pgfpathlineto{\pgfqpoint{3.206691in}{2.098477in}}%
\pgfpathlineto{\pgfqpoint{3.211232in}{2.098477in}}%
\pgfpathlineto{\pgfqpoint{3.211232in}{2.095528in}}%
\pgfpathmoveto{\pgfqpoint{3.206691in}{2.098477in}}%
\pgfpathlineto{\pgfqpoint{3.206691in}{2.098477in}}%
\pgfpathlineto{\pgfqpoint{3.206691in}{2.101426in}}%
\pgfpathlineto{\pgfqpoint{3.211232in}{2.101426in}}%
\pgfpathlineto{\pgfqpoint{3.211232in}{2.098477in}}%
\pgfpathmoveto{\pgfqpoint{3.211232in}{2.098477in}}%
\pgfpathlineto{\pgfqpoint{3.211232in}{2.098477in}}%
\pgfpathlineto{\pgfqpoint{3.211232in}{2.101426in}}%
\pgfpathlineto{\pgfqpoint{3.215773in}{2.101426in}}%
\pgfpathlineto{\pgfqpoint{3.215773in}{2.098477in}}%
\pgfpathmoveto{\pgfqpoint{3.211232in}{2.101426in}}%
\pgfpathlineto{\pgfqpoint{3.211232in}{2.101426in}}%
\pgfpathlineto{\pgfqpoint{3.211232in}{2.104375in}}%
\pgfpathlineto{\pgfqpoint{3.215773in}{2.104375in}}%
\pgfpathlineto{\pgfqpoint{3.215773in}{2.101426in}}%
\pgfpathmoveto{\pgfqpoint{3.215773in}{2.101426in}}%
\pgfpathlineto{\pgfqpoint{3.215773in}{2.101426in}}%
\pgfpathlineto{\pgfqpoint{3.215773in}{2.104375in}}%
\pgfpathlineto{\pgfqpoint{3.220314in}{2.104375in}}%
\pgfpathlineto{\pgfqpoint{3.220314in}{2.101426in}}%
\pgfpathmoveto{\pgfqpoint{3.215773in}{2.104375in}}%
\pgfpathlineto{\pgfqpoint{3.215773in}{2.104375in}}%
\pgfpathlineto{\pgfqpoint{3.215773in}{2.107325in}}%
\pgfpathlineto{\pgfqpoint{3.220314in}{2.107325in}}%
\pgfpathlineto{\pgfqpoint{3.220314in}{2.104375in}}%
\pgfpathmoveto{\pgfqpoint{3.220314in}{2.104375in}}%
\pgfpathlineto{\pgfqpoint{3.220314in}{2.104375in}}%
\pgfpathlineto{\pgfqpoint{3.220314in}{2.107325in}}%
\pgfpathlineto{\pgfqpoint{3.224855in}{2.107325in}}%
\pgfpathlineto{\pgfqpoint{3.224855in}{2.104375in}}%
\pgfpathmoveto{\pgfqpoint{3.220314in}{2.107325in}}%
\pgfpathlineto{\pgfqpoint{3.220314in}{2.107325in}}%
\pgfpathlineto{\pgfqpoint{3.220314in}{2.110274in}}%
\pgfpathlineto{\pgfqpoint{3.224855in}{2.110274in}}%
\pgfpathlineto{\pgfqpoint{3.224855in}{2.107325in}}%
\pgfpathmoveto{\pgfqpoint{3.224855in}{2.107325in}}%
\pgfpathlineto{\pgfqpoint{3.224855in}{2.107325in}}%
\pgfpathlineto{\pgfqpoint{3.224855in}{2.110274in}}%
\pgfpathlineto{\pgfqpoint{3.229396in}{2.110274in}}%
\pgfpathlineto{\pgfqpoint{3.229396in}{2.107325in}}%
\pgfpathmoveto{\pgfqpoint{3.224855in}{2.110274in}}%
\pgfpathlineto{\pgfqpoint{3.224855in}{2.110274in}}%
\pgfpathlineto{\pgfqpoint{3.224855in}{2.113223in}}%
\pgfpathlineto{\pgfqpoint{3.229396in}{2.113223in}}%
\pgfpathlineto{\pgfqpoint{3.229396in}{2.110274in}}%
\pgfpathmoveto{\pgfqpoint{3.229396in}{2.110274in}}%
\pgfpathlineto{\pgfqpoint{3.229396in}{2.110274in}}%
\pgfpathlineto{\pgfqpoint{3.229396in}{2.113223in}}%
\pgfpathlineto{\pgfqpoint{3.233937in}{2.113223in}}%
\pgfpathlineto{\pgfqpoint{3.233937in}{2.110274in}}%
\pgfpathmoveto{\pgfqpoint{3.229396in}{2.113223in}}%
\pgfpathlineto{\pgfqpoint{3.229396in}{2.113223in}}%
\pgfpathlineto{\pgfqpoint{3.229396in}{2.116173in}}%
\pgfpathlineto{\pgfqpoint{3.233937in}{2.116173in}}%
\pgfpathlineto{\pgfqpoint{3.233937in}{2.113223in}}%
\pgfpathmoveto{\pgfqpoint{3.233937in}{2.113223in}}%
\pgfpathlineto{\pgfqpoint{3.233937in}{2.113223in}}%
\pgfpathlineto{\pgfqpoint{3.233937in}{2.116173in}}%
\pgfpathlineto{\pgfqpoint{3.238478in}{2.116173in}}%
\pgfpathlineto{\pgfqpoint{3.238478in}{2.113223in}}%
\pgfpathmoveto{\pgfqpoint{3.233937in}{2.116173in}}%
\pgfpathlineto{\pgfqpoint{3.233937in}{2.116173in}}%
\pgfpathlineto{\pgfqpoint{3.233937in}{2.119122in}}%
\pgfpathlineto{\pgfqpoint{3.238478in}{2.119122in}}%
\pgfpathlineto{\pgfqpoint{3.238478in}{2.116173in}}%
\pgfpathmoveto{\pgfqpoint{3.238478in}{2.116173in}}%
\pgfpathlineto{\pgfqpoint{3.238478in}{2.116173in}}%
\pgfpathlineto{\pgfqpoint{3.238478in}{2.119122in}}%
\pgfpathlineto{\pgfqpoint{3.243019in}{2.119122in}}%
\pgfpathlineto{\pgfqpoint{3.243019in}{2.116173in}}%
\pgfpathmoveto{\pgfqpoint{3.238478in}{2.119122in}}%
\pgfpathlineto{\pgfqpoint{3.238478in}{2.119122in}}%
\pgfpathlineto{\pgfqpoint{3.238478in}{2.122071in}}%
\pgfpathlineto{\pgfqpoint{3.243019in}{2.122071in}}%
\pgfpathlineto{\pgfqpoint{3.243019in}{2.119122in}}%
\pgfpathmoveto{\pgfqpoint{3.243019in}{2.119122in}}%
\pgfpathlineto{\pgfqpoint{3.243019in}{2.119122in}}%
\pgfpathlineto{\pgfqpoint{3.243019in}{2.122071in}}%
\pgfpathlineto{\pgfqpoint{3.247560in}{2.122071in}}%
\pgfpathlineto{\pgfqpoint{3.247560in}{2.119122in}}%
\pgfpathmoveto{\pgfqpoint{3.247560in}{2.119122in}}%
\pgfpathlineto{\pgfqpoint{3.247560in}{2.119122in}}%
\pgfpathlineto{\pgfqpoint{3.247560in}{2.122071in}}%
\pgfpathlineto{\pgfqpoint{3.252101in}{2.122071in}}%
\pgfpathlineto{\pgfqpoint{3.252101in}{2.119122in}}%
\pgfpathmoveto{\pgfqpoint{3.247560in}{2.122071in}}%
\pgfpathlineto{\pgfqpoint{3.247560in}{2.122071in}}%
\pgfpathlineto{\pgfqpoint{3.247560in}{2.125020in}}%
\pgfpathlineto{\pgfqpoint{3.252101in}{2.125020in}}%
\pgfpathlineto{\pgfqpoint{3.252101in}{2.122071in}}%
\pgfpathmoveto{\pgfqpoint{3.252101in}{2.122071in}}%
\pgfpathlineto{\pgfqpoint{3.252101in}{2.122071in}}%
\pgfpathlineto{\pgfqpoint{3.252101in}{2.125020in}}%
\pgfpathlineto{\pgfqpoint{3.256642in}{2.125020in}}%
\pgfpathlineto{\pgfqpoint{3.256642in}{2.122071in}}%
\pgfpathmoveto{\pgfqpoint{3.252101in}{2.125020in}}%
\pgfpathlineto{\pgfqpoint{3.252101in}{2.125020in}}%
\pgfpathlineto{\pgfqpoint{3.252101in}{2.127970in}}%
\pgfpathlineto{\pgfqpoint{3.256642in}{2.127970in}}%
\pgfpathlineto{\pgfqpoint{3.256642in}{2.125020in}}%
\pgfpathmoveto{\pgfqpoint{3.256642in}{2.125020in}}%
\pgfpathlineto{\pgfqpoint{3.256642in}{2.125020in}}%
\pgfpathlineto{\pgfqpoint{3.256642in}{2.127970in}}%
\pgfpathlineto{\pgfqpoint{3.261182in}{2.127970in}}%
\pgfpathlineto{\pgfqpoint{3.261182in}{2.125020in}}%
\pgfpathmoveto{\pgfqpoint{3.256642in}{2.127970in}}%
\pgfpathlineto{\pgfqpoint{3.256642in}{2.127970in}}%
\pgfpathlineto{\pgfqpoint{3.256642in}{2.130919in}}%
\pgfpathlineto{\pgfqpoint{3.261182in}{2.130919in}}%
\pgfpathlineto{\pgfqpoint{3.261182in}{2.127970in}}%
\pgfpathmoveto{\pgfqpoint{3.261182in}{2.127970in}}%
\pgfpathlineto{\pgfqpoint{3.261182in}{2.127970in}}%
\pgfpathlineto{\pgfqpoint{3.261182in}{2.130919in}}%
\pgfpathlineto{\pgfqpoint{3.265723in}{2.130919in}}%
\pgfpathlineto{\pgfqpoint{3.265723in}{2.127970in}}%
\pgfpathmoveto{\pgfqpoint{3.261182in}{2.130919in}}%
\pgfpathlineto{\pgfqpoint{3.261182in}{2.130919in}}%
\pgfpathlineto{\pgfqpoint{3.261182in}{2.133868in}}%
\pgfpathlineto{\pgfqpoint{3.265723in}{2.133868in}}%
\pgfpathlineto{\pgfqpoint{3.265723in}{2.130919in}}%
\pgfpathmoveto{\pgfqpoint{3.265723in}{2.130919in}}%
\pgfpathlineto{\pgfqpoint{3.265723in}{2.130919in}}%
\pgfpathlineto{\pgfqpoint{3.265723in}{2.133868in}}%
\pgfpathlineto{\pgfqpoint{3.270264in}{2.133868in}}%
\pgfpathlineto{\pgfqpoint{3.270264in}{2.130919in}}%
\pgfpathmoveto{\pgfqpoint{3.265723in}{2.133868in}}%
\pgfpathlineto{\pgfqpoint{3.265723in}{2.133868in}}%
\pgfpathlineto{\pgfqpoint{3.265723in}{2.136817in}}%
\pgfpathlineto{\pgfqpoint{3.270264in}{2.136817in}}%
\pgfpathlineto{\pgfqpoint{3.270264in}{2.133868in}}%
\pgfpathmoveto{\pgfqpoint{3.270264in}{2.133868in}}%
\pgfpathlineto{\pgfqpoint{3.270264in}{2.133868in}}%
\pgfpathlineto{\pgfqpoint{3.270264in}{2.136817in}}%
\pgfpathlineto{\pgfqpoint{3.274805in}{2.136817in}}%
\pgfpathlineto{\pgfqpoint{3.274805in}{2.133868in}}%
\pgfpathmoveto{\pgfqpoint{3.270264in}{2.136817in}}%
\pgfpathlineto{\pgfqpoint{3.270264in}{2.136817in}}%
\pgfpathlineto{\pgfqpoint{3.270264in}{2.139767in}}%
\pgfpathlineto{\pgfqpoint{3.274805in}{2.139767in}}%
\pgfpathlineto{\pgfqpoint{3.274805in}{2.136817in}}%
\pgfpathmoveto{\pgfqpoint{3.274805in}{2.136817in}}%
\pgfpathlineto{\pgfqpoint{3.274805in}{2.136817in}}%
\pgfpathlineto{\pgfqpoint{3.274805in}{2.139767in}}%
\pgfpathlineto{\pgfqpoint{3.279346in}{2.139767in}}%
\pgfpathlineto{\pgfqpoint{3.279346in}{2.136817in}}%
\pgfpathmoveto{\pgfqpoint{3.274805in}{2.139767in}}%
\pgfpathlineto{\pgfqpoint{3.274805in}{2.139767in}}%
\pgfpathlineto{\pgfqpoint{3.274805in}{2.142716in}}%
\pgfpathlineto{\pgfqpoint{3.279346in}{2.142716in}}%
\pgfpathlineto{\pgfqpoint{3.279346in}{2.139767in}}%
\pgfpathmoveto{\pgfqpoint{3.279346in}{2.139767in}}%
\pgfpathlineto{\pgfqpoint{3.279346in}{2.139767in}}%
\pgfpathlineto{\pgfqpoint{3.279346in}{2.142716in}}%
\pgfpathlineto{\pgfqpoint{3.283887in}{2.142716in}}%
\pgfpathlineto{\pgfqpoint{3.283887in}{2.139767in}}%
\pgfpathmoveto{\pgfqpoint{3.279346in}{2.142716in}}%
\pgfpathlineto{\pgfqpoint{3.279346in}{2.142716in}}%
\pgfpathlineto{\pgfqpoint{3.279346in}{2.145665in}}%
\pgfpathlineto{\pgfqpoint{3.283887in}{2.145665in}}%
\pgfpathlineto{\pgfqpoint{3.283887in}{2.142716in}}%
\pgfpathmoveto{\pgfqpoint{3.283887in}{2.142716in}}%
\pgfpathlineto{\pgfqpoint{3.283887in}{2.142716in}}%
\pgfpathlineto{\pgfqpoint{3.283887in}{2.145665in}}%
\pgfpathlineto{\pgfqpoint{3.288428in}{2.145665in}}%
\pgfpathlineto{\pgfqpoint{3.288428in}{2.142716in}}%
\pgfpathmoveto{\pgfqpoint{3.283887in}{2.145665in}}%
\pgfpathlineto{\pgfqpoint{3.283887in}{2.145665in}}%
\pgfpathlineto{\pgfqpoint{3.283887in}{2.148615in}}%
\pgfpathlineto{\pgfqpoint{3.288428in}{2.148615in}}%
\pgfpathlineto{\pgfqpoint{3.288428in}{2.145665in}}%
\pgfpathmoveto{\pgfqpoint{3.288428in}{2.145665in}}%
\pgfpathlineto{\pgfqpoint{3.288428in}{2.145665in}}%
\pgfpathlineto{\pgfqpoint{3.288428in}{2.148615in}}%
\pgfpathlineto{\pgfqpoint{3.292969in}{2.148615in}}%
\pgfpathlineto{\pgfqpoint{3.292969in}{2.145665in}}%
\pgfpathmoveto{\pgfqpoint{3.288428in}{2.148615in}}%
\pgfpathlineto{\pgfqpoint{3.288428in}{2.148615in}}%
\pgfpathlineto{\pgfqpoint{3.288428in}{2.151564in}}%
\pgfpathlineto{\pgfqpoint{3.292969in}{2.151564in}}%
\pgfpathlineto{\pgfqpoint{3.292969in}{2.148615in}}%
\pgfpathmoveto{\pgfqpoint{3.292969in}{2.148615in}}%
\pgfpathlineto{\pgfqpoint{3.292969in}{2.148615in}}%
\pgfpathlineto{\pgfqpoint{3.292969in}{2.151564in}}%
\pgfpathlineto{\pgfqpoint{3.297510in}{2.151564in}}%
\pgfpathlineto{\pgfqpoint{3.297510in}{2.148615in}}%
\pgfpathmoveto{\pgfqpoint{3.292969in}{2.151564in}}%
\pgfpathlineto{\pgfqpoint{3.292969in}{2.151564in}}%
\pgfpathlineto{\pgfqpoint{3.292969in}{2.154513in}}%
\pgfpathlineto{\pgfqpoint{3.297510in}{2.154513in}}%
\pgfpathlineto{\pgfqpoint{3.297510in}{2.151564in}}%
\pgfpathmoveto{\pgfqpoint{3.297510in}{2.151564in}}%
\pgfpathlineto{\pgfqpoint{3.297510in}{2.151564in}}%
\pgfpathlineto{\pgfqpoint{3.297510in}{2.154513in}}%
\pgfpathlineto{\pgfqpoint{3.302051in}{2.154513in}}%
\pgfpathlineto{\pgfqpoint{3.302051in}{2.151564in}}%
\pgfpathmoveto{\pgfqpoint{3.297510in}{2.154513in}}%
\pgfpathlineto{\pgfqpoint{3.297510in}{2.154513in}}%
\pgfpathlineto{\pgfqpoint{3.297510in}{2.157462in}}%
\pgfpathlineto{\pgfqpoint{3.302051in}{2.157462in}}%
\pgfpathlineto{\pgfqpoint{3.302051in}{2.154513in}}%
\pgfpathmoveto{\pgfqpoint{3.302051in}{2.154513in}}%
\pgfpathlineto{\pgfqpoint{3.302051in}{2.154513in}}%
\pgfpathlineto{\pgfqpoint{3.302051in}{2.157462in}}%
\pgfpathlineto{\pgfqpoint{3.306591in}{2.157462in}}%
\pgfpathlineto{\pgfqpoint{3.306591in}{2.154513in}}%
\pgfpathmoveto{\pgfqpoint{3.302051in}{2.157462in}}%
\pgfpathlineto{\pgfqpoint{3.302051in}{2.157462in}}%
\pgfpathlineto{\pgfqpoint{3.302051in}{2.160412in}}%
\pgfpathlineto{\pgfqpoint{3.306591in}{2.160412in}}%
\pgfpathlineto{\pgfqpoint{3.306591in}{2.157462in}}%
\pgfpathmoveto{\pgfqpoint{3.306591in}{2.157462in}}%
\pgfpathlineto{\pgfqpoint{3.306591in}{2.157462in}}%
\pgfpathlineto{\pgfqpoint{3.306591in}{2.160412in}}%
\pgfpathlineto{\pgfqpoint{3.311132in}{2.160412in}}%
\pgfpathlineto{\pgfqpoint{3.311132in}{2.157462in}}%
\pgfpathmoveto{\pgfqpoint{3.306591in}{2.160412in}}%
\pgfpathlineto{\pgfqpoint{3.306591in}{2.160412in}}%
\pgfpathlineto{\pgfqpoint{3.306591in}{2.163361in}}%
\pgfpathlineto{\pgfqpoint{3.311132in}{2.163361in}}%
\pgfpathlineto{\pgfqpoint{3.311132in}{2.160412in}}%
\pgfpathmoveto{\pgfqpoint{3.311132in}{2.160412in}}%
\pgfpathlineto{\pgfqpoint{3.311132in}{2.160412in}}%
\pgfpathlineto{\pgfqpoint{3.311132in}{2.163361in}}%
\pgfpathlineto{\pgfqpoint{3.315673in}{2.163361in}}%
\pgfpathlineto{\pgfqpoint{3.315673in}{2.160412in}}%
\pgfpathmoveto{\pgfqpoint{3.311132in}{2.163361in}}%
\pgfpathlineto{\pgfqpoint{3.311132in}{2.163361in}}%
\pgfpathlineto{\pgfqpoint{3.311132in}{2.166310in}}%
\pgfpathlineto{\pgfqpoint{3.315673in}{2.166310in}}%
\pgfpathlineto{\pgfqpoint{3.315673in}{2.163361in}}%
\pgfpathmoveto{\pgfqpoint{3.315673in}{2.163361in}}%
\pgfpathlineto{\pgfqpoint{3.315673in}{2.163361in}}%
\pgfpathlineto{\pgfqpoint{3.315673in}{2.166310in}}%
\pgfpathlineto{\pgfqpoint{3.320214in}{2.166310in}}%
\pgfpathlineto{\pgfqpoint{3.320214in}{2.163361in}}%
\pgfpathmoveto{\pgfqpoint{3.315673in}{2.166310in}}%
\pgfpathlineto{\pgfqpoint{3.315673in}{2.166310in}}%
\pgfpathlineto{\pgfqpoint{3.315673in}{2.169259in}}%
\pgfpathlineto{\pgfqpoint{3.320214in}{2.169259in}}%
\pgfpathlineto{\pgfqpoint{3.320214in}{2.166310in}}%
\pgfpathmoveto{\pgfqpoint{3.320214in}{2.166310in}}%
\pgfpathlineto{\pgfqpoint{3.320214in}{2.166310in}}%
\pgfpathlineto{\pgfqpoint{3.320214in}{2.169259in}}%
\pgfpathlineto{\pgfqpoint{3.324755in}{2.169259in}}%
\pgfpathlineto{\pgfqpoint{3.324755in}{2.166310in}}%
\pgfpathmoveto{\pgfqpoint{3.320214in}{2.169259in}}%
\pgfpathlineto{\pgfqpoint{3.320214in}{2.169259in}}%
\pgfpathlineto{\pgfqpoint{3.320214in}{2.172209in}}%
\pgfpathlineto{\pgfqpoint{3.324755in}{2.172209in}}%
\pgfpathlineto{\pgfqpoint{3.324755in}{2.169259in}}%
\pgfpathmoveto{\pgfqpoint{3.324755in}{2.169259in}}%
\pgfpathlineto{\pgfqpoint{3.324755in}{2.169259in}}%
\pgfpathlineto{\pgfqpoint{3.324755in}{2.172209in}}%
\pgfpathlineto{\pgfqpoint{3.329296in}{2.172209in}}%
\pgfpathlineto{\pgfqpoint{3.329296in}{2.169259in}}%
\pgfpathmoveto{\pgfqpoint{3.324755in}{2.172209in}}%
\pgfpathlineto{\pgfqpoint{3.324755in}{2.172209in}}%
\pgfpathlineto{\pgfqpoint{3.324755in}{2.175158in}}%
\pgfpathlineto{\pgfqpoint{3.329296in}{2.175158in}}%
\pgfpathlineto{\pgfqpoint{3.329296in}{2.172209in}}%
\pgfpathmoveto{\pgfqpoint{3.329296in}{2.172209in}}%
\pgfpathlineto{\pgfqpoint{3.329296in}{2.172209in}}%
\pgfpathlineto{\pgfqpoint{3.329296in}{2.175158in}}%
\pgfpathlineto{\pgfqpoint{3.333837in}{2.175158in}}%
\pgfpathlineto{\pgfqpoint{3.333837in}{2.172209in}}%
\pgfpathmoveto{\pgfqpoint{3.329296in}{2.175158in}}%
\pgfpathlineto{\pgfqpoint{3.329296in}{2.175158in}}%
\pgfpathlineto{\pgfqpoint{3.329296in}{2.178107in}}%
\pgfpathlineto{\pgfqpoint{3.333837in}{2.178107in}}%
\pgfpathlineto{\pgfqpoint{3.333837in}{2.175158in}}%
\pgfpathmoveto{\pgfqpoint{3.333837in}{2.175158in}}%
\pgfpathlineto{\pgfqpoint{3.333837in}{2.175158in}}%
\pgfpathlineto{\pgfqpoint{3.333837in}{2.178107in}}%
\pgfpathlineto{\pgfqpoint{3.338378in}{2.178107in}}%
\pgfpathlineto{\pgfqpoint{3.338378in}{2.175158in}}%
\pgfpathmoveto{\pgfqpoint{3.333837in}{2.178107in}}%
\pgfpathlineto{\pgfqpoint{3.333837in}{2.178107in}}%
\pgfpathlineto{\pgfqpoint{3.333837in}{2.181056in}}%
\pgfpathlineto{\pgfqpoint{3.338378in}{2.181056in}}%
\pgfpathlineto{\pgfqpoint{3.338378in}{2.178107in}}%
\pgfpathmoveto{\pgfqpoint{3.338378in}{2.178107in}}%
\pgfpathlineto{\pgfqpoint{3.338378in}{2.178107in}}%
\pgfpathlineto{\pgfqpoint{3.338378in}{2.181056in}}%
\pgfpathlineto{\pgfqpoint{3.342919in}{2.181056in}}%
\pgfpathlineto{\pgfqpoint{3.342919in}{2.178107in}}%
\pgfpathmoveto{\pgfqpoint{3.338378in}{2.181056in}}%
\pgfpathlineto{\pgfqpoint{3.338378in}{2.181056in}}%
\pgfpathlineto{\pgfqpoint{3.338378in}{2.184006in}}%
\pgfpathlineto{\pgfqpoint{3.342919in}{2.184006in}}%
\pgfpathlineto{\pgfqpoint{3.342919in}{2.181056in}}%
\pgfpathmoveto{\pgfqpoint{3.342919in}{2.181056in}}%
\pgfpathlineto{\pgfqpoint{3.342919in}{2.181056in}}%
\pgfpathlineto{\pgfqpoint{3.342919in}{2.184006in}}%
\pgfpathlineto{\pgfqpoint{3.347460in}{2.184006in}}%
\pgfpathlineto{\pgfqpoint{3.347460in}{2.181056in}}%
\pgfpathmoveto{\pgfqpoint{3.342919in}{2.184006in}}%
\pgfpathlineto{\pgfqpoint{3.342919in}{2.184006in}}%
\pgfpathlineto{\pgfqpoint{3.342919in}{2.186955in}}%
\pgfpathlineto{\pgfqpoint{3.347460in}{2.186955in}}%
\pgfpathlineto{\pgfqpoint{3.347460in}{2.184006in}}%
\pgfpathmoveto{\pgfqpoint{3.347460in}{2.184006in}}%
\pgfpathlineto{\pgfqpoint{3.347460in}{2.184006in}}%
\pgfpathlineto{\pgfqpoint{3.347460in}{2.186955in}}%
\pgfpathlineto{\pgfqpoint{3.352000in}{2.186955in}}%
\pgfpathlineto{\pgfqpoint{3.352000in}{2.184006in}}%
\pgfpathmoveto{\pgfqpoint{3.347460in}{2.186955in}}%
\pgfpathlineto{\pgfqpoint{3.347460in}{2.186955in}}%
\pgfpathlineto{\pgfqpoint{3.347460in}{2.189904in}}%
\pgfpathlineto{\pgfqpoint{3.352000in}{2.189904in}}%
\pgfpathlineto{\pgfqpoint{3.352000in}{2.186955in}}%
\pgfpathmoveto{\pgfqpoint{3.352000in}{2.186955in}}%
\pgfpathlineto{\pgfqpoint{3.352000in}{2.186955in}}%
\pgfpathlineto{\pgfqpoint{3.352000in}{2.189904in}}%
\pgfpathlineto{\pgfqpoint{3.356541in}{2.189904in}}%
\pgfpathlineto{\pgfqpoint{3.356541in}{2.186955in}}%
\pgfpathmoveto{\pgfqpoint{3.352000in}{2.189904in}}%
\pgfpathlineto{\pgfqpoint{3.352000in}{2.189904in}}%
\pgfpathlineto{\pgfqpoint{3.352000in}{2.192854in}}%
\pgfpathlineto{\pgfqpoint{3.356541in}{2.192854in}}%
\pgfpathlineto{\pgfqpoint{3.356541in}{2.189904in}}%
\pgfpathmoveto{\pgfqpoint{3.356541in}{2.189904in}}%
\pgfpathlineto{\pgfqpoint{3.356541in}{2.189904in}}%
\pgfpathlineto{\pgfqpoint{3.356541in}{2.192854in}}%
\pgfpathlineto{\pgfqpoint{3.361082in}{2.192854in}}%
\pgfpathlineto{\pgfqpoint{3.361082in}{2.189904in}}%
\pgfpathmoveto{\pgfqpoint{3.356541in}{2.192854in}}%
\pgfpathlineto{\pgfqpoint{3.356541in}{2.192854in}}%
\pgfpathlineto{\pgfqpoint{3.356541in}{2.195803in}}%
\pgfpathlineto{\pgfqpoint{3.361082in}{2.195803in}}%
\pgfpathlineto{\pgfqpoint{3.361082in}{2.192854in}}%
\pgfpathmoveto{\pgfqpoint{3.361082in}{2.192854in}}%
\pgfpathlineto{\pgfqpoint{3.361082in}{2.192854in}}%
\pgfpathlineto{\pgfqpoint{3.361082in}{2.195803in}}%
\pgfpathlineto{\pgfqpoint{3.365623in}{2.195803in}}%
\pgfpathlineto{\pgfqpoint{3.365623in}{2.192854in}}%
\pgfpathmoveto{\pgfqpoint{3.361082in}{2.195803in}}%
\pgfpathlineto{\pgfqpoint{3.361082in}{2.195803in}}%
\pgfpathlineto{\pgfqpoint{3.361082in}{2.198752in}}%
\pgfpathlineto{\pgfqpoint{3.365623in}{2.198752in}}%
\pgfpathlineto{\pgfqpoint{3.365623in}{2.195803in}}%
\pgfpathmoveto{\pgfqpoint{3.365623in}{2.195803in}}%
\pgfpathlineto{\pgfqpoint{3.365623in}{2.195803in}}%
\pgfpathlineto{\pgfqpoint{3.365623in}{2.198752in}}%
\pgfpathlineto{\pgfqpoint{3.370164in}{2.198752in}}%
\pgfpathlineto{\pgfqpoint{3.370164in}{2.195803in}}%
\pgfpathmoveto{\pgfqpoint{3.365623in}{2.198752in}}%
\pgfpathlineto{\pgfqpoint{3.365623in}{2.198752in}}%
\pgfpathlineto{\pgfqpoint{3.365623in}{2.201701in}}%
\pgfpathlineto{\pgfqpoint{3.370164in}{2.201701in}}%
\pgfpathlineto{\pgfqpoint{3.370164in}{2.198752in}}%
\pgfpathmoveto{\pgfqpoint{3.370164in}{2.198752in}}%
\pgfpathlineto{\pgfqpoint{3.370164in}{2.198752in}}%
\pgfpathlineto{\pgfqpoint{3.370164in}{2.201701in}}%
\pgfpathlineto{\pgfqpoint{3.374705in}{2.201701in}}%
\pgfpathlineto{\pgfqpoint{3.374705in}{2.198752in}}%
\pgfpathmoveto{\pgfqpoint{3.370164in}{2.201701in}}%
\pgfpathlineto{\pgfqpoint{3.370164in}{2.201701in}}%
\pgfpathlineto{\pgfqpoint{3.370164in}{2.204651in}}%
\pgfpathlineto{\pgfqpoint{3.374705in}{2.204651in}}%
\pgfpathlineto{\pgfqpoint{3.374705in}{2.201701in}}%
\pgfpathmoveto{\pgfqpoint{3.374705in}{2.201701in}}%
\pgfpathlineto{\pgfqpoint{3.374705in}{2.201701in}}%
\pgfpathlineto{\pgfqpoint{3.374705in}{2.204651in}}%
\pgfpathlineto{\pgfqpoint{3.379246in}{2.204651in}}%
\pgfpathlineto{\pgfqpoint{3.379246in}{2.201701in}}%
\pgfpathmoveto{\pgfqpoint{3.374705in}{2.204651in}}%
\pgfpathlineto{\pgfqpoint{3.374705in}{2.204651in}}%
\pgfpathlineto{\pgfqpoint{3.374705in}{2.207600in}}%
\pgfpathlineto{\pgfqpoint{3.379246in}{2.207600in}}%
\pgfpathlineto{\pgfqpoint{3.379246in}{2.204651in}}%
\pgfpathmoveto{\pgfqpoint{3.379246in}{2.204651in}}%
\pgfpathlineto{\pgfqpoint{3.379246in}{2.204651in}}%
\pgfpathlineto{\pgfqpoint{3.379246in}{2.207600in}}%
\pgfpathlineto{\pgfqpoint{3.383787in}{2.207600in}}%
\pgfpathlineto{\pgfqpoint{3.383787in}{2.204651in}}%
\pgfpathmoveto{\pgfqpoint{3.379246in}{2.207600in}}%
\pgfpathlineto{\pgfqpoint{3.379246in}{2.207600in}}%
\pgfpathlineto{\pgfqpoint{3.379246in}{2.210549in}}%
\pgfpathlineto{\pgfqpoint{3.383787in}{2.210549in}}%
\pgfpathlineto{\pgfqpoint{3.383787in}{2.207600in}}%
\pgfpathmoveto{\pgfqpoint{3.383787in}{2.207600in}}%
\pgfpathlineto{\pgfqpoint{3.383787in}{2.207600in}}%
\pgfpathlineto{\pgfqpoint{3.383787in}{2.210549in}}%
\pgfpathlineto{\pgfqpoint{3.388328in}{2.210549in}}%
\pgfpathlineto{\pgfqpoint{3.388328in}{2.207600in}}%
\pgfpathmoveto{\pgfqpoint{3.383787in}{2.210549in}}%
\pgfpathlineto{\pgfqpoint{3.383787in}{2.210549in}}%
\pgfpathlineto{\pgfqpoint{3.383787in}{2.213498in}}%
\pgfpathlineto{\pgfqpoint{3.388328in}{2.213498in}}%
\pgfpathlineto{\pgfqpoint{3.388328in}{2.210549in}}%
\pgfpathmoveto{\pgfqpoint{3.388328in}{2.210549in}}%
\pgfpathlineto{\pgfqpoint{3.388328in}{2.210549in}}%
\pgfpathlineto{\pgfqpoint{3.388328in}{2.213498in}}%
\pgfpathlineto{\pgfqpoint{3.392869in}{2.213498in}}%
\pgfpathlineto{\pgfqpoint{3.392869in}{2.210549in}}%
\pgfpathmoveto{\pgfqpoint{3.388328in}{2.213498in}}%
\pgfpathlineto{\pgfqpoint{3.388328in}{2.213498in}}%
\pgfpathlineto{\pgfqpoint{3.388328in}{2.216447in}}%
\pgfpathlineto{\pgfqpoint{3.392869in}{2.216447in}}%
\pgfpathlineto{\pgfqpoint{3.392869in}{2.213498in}}%
\pgfpathmoveto{\pgfqpoint{3.392869in}{2.213498in}}%
\pgfpathlineto{\pgfqpoint{3.392869in}{2.213498in}}%
\pgfpathlineto{\pgfqpoint{3.392869in}{2.216447in}}%
\pgfpathlineto{\pgfqpoint{3.397410in}{2.216447in}}%
\pgfpathlineto{\pgfqpoint{3.397410in}{2.213498in}}%
\pgfpathmoveto{\pgfqpoint{3.392869in}{2.216447in}}%
\pgfpathlineto{\pgfqpoint{3.392869in}{2.216447in}}%
\pgfpathlineto{\pgfqpoint{3.392869in}{2.219397in}}%
\pgfpathlineto{\pgfqpoint{3.397410in}{2.219397in}}%
\pgfpathlineto{\pgfqpoint{3.397410in}{2.216447in}}%
\pgfpathmoveto{\pgfqpoint{3.397410in}{2.216447in}}%
\pgfpathlineto{\pgfqpoint{3.397410in}{2.216447in}}%
\pgfpathlineto{\pgfqpoint{3.397410in}{2.219397in}}%
\pgfpathlineto{\pgfqpoint{3.401951in}{2.219397in}}%
\pgfpathlineto{\pgfqpoint{3.401951in}{2.216447in}}%
\pgfpathmoveto{\pgfqpoint{3.397410in}{2.219397in}}%
\pgfpathlineto{\pgfqpoint{3.397410in}{2.219397in}}%
\pgfpathlineto{\pgfqpoint{3.397410in}{2.222346in}}%
\pgfpathlineto{\pgfqpoint{3.401951in}{2.222346in}}%
\pgfpathlineto{\pgfqpoint{3.401951in}{2.219397in}}%
\pgfpathmoveto{\pgfqpoint{3.401951in}{2.219397in}}%
\pgfpathlineto{\pgfqpoint{3.401951in}{2.219397in}}%
\pgfpathlineto{\pgfqpoint{3.401951in}{2.222346in}}%
\pgfpathlineto{\pgfqpoint{3.406492in}{2.222346in}}%
\pgfpathlineto{\pgfqpoint{3.406492in}{2.219397in}}%
\pgfpathmoveto{\pgfqpoint{3.401951in}{2.222346in}}%
\pgfpathlineto{\pgfqpoint{3.401951in}{2.222346in}}%
\pgfpathlineto{\pgfqpoint{3.401951in}{2.225295in}}%
\pgfpathlineto{\pgfqpoint{3.406492in}{2.225295in}}%
\pgfpathlineto{\pgfqpoint{3.406492in}{2.222346in}}%
\pgfpathmoveto{\pgfqpoint{3.406492in}{2.222346in}}%
\pgfpathlineto{\pgfqpoint{3.406492in}{2.222346in}}%
\pgfpathlineto{\pgfqpoint{3.406492in}{2.225295in}}%
\pgfpathlineto{\pgfqpoint{3.411033in}{2.225295in}}%
\pgfpathlineto{\pgfqpoint{3.411033in}{2.222346in}}%
\pgfpathmoveto{\pgfqpoint{3.406492in}{2.225295in}}%
\pgfpathlineto{\pgfqpoint{3.406492in}{2.225295in}}%
\pgfpathlineto{\pgfqpoint{3.406492in}{2.228244in}}%
\pgfpathlineto{\pgfqpoint{3.411033in}{2.228244in}}%
\pgfpathlineto{\pgfqpoint{3.411033in}{2.225295in}}%
\pgfpathmoveto{\pgfqpoint{3.411033in}{2.225295in}}%
\pgfpathlineto{\pgfqpoint{3.411033in}{2.225295in}}%
\pgfpathlineto{\pgfqpoint{3.411033in}{2.228244in}}%
\pgfpathlineto{\pgfqpoint{3.415574in}{2.228244in}}%
\pgfpathlineto{\pgfqpoint{3.415574in}{2.225295in}}%
\pgfpathmoveto{\pgfqpoint{3.411033in}{2.228244in}}%
\pgfpathlineto{\pgfqpoint{3.411033in}{2.228244in}}%
\pgfpathlineto{\pgfqpoint{3.411033in}{2.231194in}}%
\pgfpathlineto{\pgfqpoint{3.415574in}{2.231194in}}%
\pgfpathlineto{\pgfqpoint{3.415574in}{2.228244in}}%
\pgfpathmoveto{\pgfqpoint{3.415574in}{2.228244in}}%
\pgfpathlineto{\pgfqpoint{3.415574in}{2.228244in}}%
\pgfpathlineto{\pgfqpoint{3.415574in}{2.231194in}}%
\pgfpathlineto{\pgfqpoint{3.420115in}{2.231194in}}%
\pgfpathlineto{\pgfqpoint{3.420115in}{2.228244in}}%
\pgfpathmoveto{\pgfqpoint{3.415574in}{2.231194in}}%
\pgfpathlineto{\pgfqpoint{3.415574in}{2.231194in}}%
\pgfpathlineto{\pgfqpoint{3.415574in}{2.234143in}}%
\pgfpathlineto{\pgfqpoint{3.420115in}{2.234143in}}%
\pgfpathlineto{\pgfqpoint{3.420115in}{2.231194in}}%
\pgfpathmoveto{\pgfqpoint{3.420115in}{2.231194in}}%
\pgfpathlineto{\pgfqpoint{3.420115in}{2.231194in}}%
\pgfpathlineto{\pgfqpoint{3.420115in}{2.234143in}}%
\pgfpathlineto{\pgfqpoint{3.424656in}{2.234143in}}%
\pgfpathlineto{\pgfqpoint{3.424656in}{2.231194in}}%
\pgfpathmoveto{\pgfqpoint{3.420115in}{2.234143in}}%
\pgfpathlineto{\pgfqpoint{3.420115in}{2.234143in}}%
\pgfpathlineto{\pgfqpoint{3.420115in}{2.237092in}}%
\pgfpathlineto{\pgfqpoint{3.424656in}{2.237092in}}%
\pgfpathlineto{\pgfqpoint{3.424656in}{2.234143in}}%
\pgfpathmoveto{\pgfqpoint{3.424656in}{2.234143in}}%
\pgfpathlineto{\pgfqpoint{3.424656in}{2.234143in}}%
\pgfpathlineto{\pgfqpoint{3.424656in}{2.237092in}}%
\pgfpathlineto{\pgfqpoint{3.429197in}{2.237092in}}%
\pgfpathlineto{\pgfqpoint{3.429197in}{2.234143in}}%
\pgfpathmoveto{\pgfqpoint{3.424656in}{2.237092in}}%
\pgfpathlineto{\pgfqpoint{3.424656in}{2.237092in}}%
\pgfpathlineto{\pgfqpoint{3.424656in}{2.240041in}}%
\pgfpathlineto{\pgfqpoint{3.429197in}{2.240041in}}%
\pgfpathlineto{\pgfqpoint{3.429197in}{2.237092in}}%
\pgfpathmoveto{\pgfqpoint{3.429197in}{2.237092in}}%
\pgfpathlineto{\pgfqpoint{3.429197in}{2.237092in}}%
\pgfpathlineto{\pgfqpoint{3.429197in}{2.240041in}}%
\pgfpathlineto{\pgfqpoint{3.433738in}{2.240041in}}%
\pgfpathlineto{\pgfqpoint{3.433738in}{2.237092in}}%
\pgfpathmoveto{\pgfqpoint{3.429197in}{2.240041in}}%
\pgfpathlineto{\pgfqpoint{3.429197in}{2.240041in}}%
\pgfpathlineto{\pgfqpoint{3.429197in}{2.242991in}}%
\pgfpathlineto{\pgfqpoint{3.433738in}{2.242991in}}%
\pgfpathlineto{\pgfqpoint{3.433738in}{2.240041in}}%
\pgfpathmoveto{\pgfqpoint{3.433738in}{2.240041in}}%
\pgfpathlineto{\pgfqpoint{3.433738in}{2.240041in}}%
\pgfpathlineto{\pgfqpoint{3.433738in}{2.242991in}}%
\pgfpathlineto{\pgfqpoint{3.438279in}{2.242991in}}%
\pgfpathlineto{\pgfqpoint{3.438279in}{2.240041in}}%
\pgfpathmoveto{\pgfqpoint{3.433738in}{2.242991in}}%
\pgfpathlineto{\pgfqpoint{3.433738in}{2.242991in}}%
\pgfpathlineto{\pgfqpoint{3.433738in}{2.245940in}}%
\pgfpathlineto{\pgfqpoint{3.438279in}{2.245940in}}%
\pgfpathlineto{\pgfqpoint{3.438279in}{2.242991in}}%
\pgfpathmoveto{\pgfqpoint{3.438279in}{2.242991in}}%
\pgfpathlineto{\pgfqpoint{3.438279in}{2.242991in}}%
\pgfpathlineto{\pgfqpoint{3.438279in}{2.245940in}}%
\pgfpathlineto{\pgfqpoint{3.442820in}{2.245940in}}%
\pgfpathlineto{\pgfqpoint{3.442820in}{2.242991in}}%
\pgfpathmoveto{\pgfqpoint{3.438279in}{2.245940in}}%
\pgfpathlineto{\pgfqpoint{3.438279in}{2.245940in}}%
\pgfpathlineto{\pgfqpoint{3.438279in}{2.248889in}}%
\pgfpathlineto{\pgfqpoint{3.442820in}{2.248889in}}%
\pgfpathlineto{\pgfqpoint{3.442820in}{2.245940in}}%
\pgfpathmoveto{\pgfqpoint{3.442820in}{2.245940in}}%
\pgfpathlineto{\pgfqpoint{3.442820in}{2.245940in}}%
\pgfpathlineto{\pgfqpoint{3.442820in}{2.248889in}}%
\pgfpathlineto{\pgfqpoint{3.447360in}{2.248889in}}%
\pgfpathlineto{\pgfqpoint{3.447360in}{2.245940in}}%
\pgfpathmoveto{\pgfqpoint{3.442820in}{2.248889in}}%
\pgfpathlineto{\pgfqpoint{3.442820in}{2.248889in}}%
\pgfpathlineto{\pgfqpoint{3.442820in}{2.251838in}}%
\pgfpathlineto{\pgfqpoint{3.447360in}{2.251838in}}%
\pgfpathlineto{\pgfqpoint{3.447360in}{2.248889in}}%
\pgfpathmoveto{\pgfqpoint{3.447360in}{2.248889in}}%
\pgfpathlineto{\pgfqpoint{3.447360in}{2.248889in}}%
\pgfpathlineto{\pgfqpoint{3.447360in}{2.251838in}}%
\pgfpathlineto{\pgfqpoint{3.451901in}{2.251838in}}%
\pgfpathlineto{\pgfqpoint{3.451901in}{2.248889in}}%
\pgfpathmoveto{\pgfqpoint{3.447360in}{2.251838in}}%
\pgfpathlineto{\pgfqpoint{3.447360in}{2.251838in}}%
\pgfpathlineto{\pgfqpoint{3.447360in}{2.254787in}}%
\pgfpathlineto{\pgfqpoint{3.451901in}{2.254787in}}%
\pgfpathlineto{\pgfqpoint{3.451901in}{2.251838in}}%
\pgfpathmoveto{\pgfqpoint{3.451901in}{2.251838in}}%
\pgfpathlineto{\pgfqpoint{3.451901in}{2.251838in}}%
\pgfpathlineto{\pgfqpoint{3.451901in}{2.254787in}}%
\pgfpathlineto{\pgfqpoint{3.456442in}{2.254787in}}%
\pgfpathlineto{\pgfqpoint{3.456442in}{2.251838in}}%
\pgfpathmoveto{\pgfqpoint{3.451901in}{2.254787in}}%
\pgfpathlineto{\pgfqpoint{3.451901in}{2.254787in}}%
\pgfpathlineto{\pgfqpoint{3.451901in}{2.257737in}}%
\pgfpathlineto{\pgfqpoint{3.456442in}{2.257737in}}%
\pgfpathlineto{\pgfqpoint{3.456442in}{2.254787in}}%
\pgfpathmoveto{\pgfqpoint{3.456442in}{2.254787in}}%
\pgfpathlineto{\pgfqpoint{3.456442in}{2.254787in}}%
\pgfpathlineto{\pgfqpoint{3.456442in}{2.257737in}}%
\pgfpathlineto{\pgfqpoint{3.460983in}{2.257737in}}%
\pgfpathlineto{\pgfqpoint{3.460983in}{2.254787in}}%
\pgfpathmoveto{\pgfqpoint{3.456442in}{2.257737in}}%
\pgfpathlineto{\pgfqpoint{3.456442in}{2.257737in}}%
\pgfpathlineto{\pgfqpoint{3.456442in}{2.260686in}}%
\pgfpathlineto{\pgfqpoint{3.460983in}{2.260686in}}%
\pgfpathlineto{\pgfqpoint{3.460983in}{2.257737in}}%
\pgfpathmoveto{\pgfqpoint{3.460983in}{2.257737in}}%
\pgfpathlineto{\pgfqpoint{3.460983in}{2.257737in}}%
\pgfpathlineto{\pgfqpoint{3.460983in}{2.260686in}}%
\pgfpathlineto{\pgfqpoint{3.465524in}{2.260686in}}%
\pgfpathlineto{\pgfqpoint{3.465524in}{2.257737in}}%
\pgfpathmoveto{\pgfqpoint{3.460983in}{2.260686in}}%
\pgfpathlineto{\pgfqpoint{3.460983in}{2.260686in}}%
\pgfpathlineto{\pgfqpoint{3.460983in}{2.263635in}}%
\pgfpathlineto{\pgfqpoint{3.465524in}{2.263635in}}%
\pgfpathlineto{\pgfqpoint{3.465524in}{2.260686in}}%
\pgfpathmoveto{\pgfqpoint{3.465524in}{2.260686in}}%
\pgfpathlineto{\pgfqpoint{3.465524in}{2.260686in}}%
\pgfpathlineto{\pgfqpoint{3.465524in}{2.263635in}}%
\pgfpathlineto{\pgfqpoint{3.470065in}{2.263635in}}%
\pgfpathlineto{\pgfqpoint{3.470065in}{2.260686in}}%
\pgfpathmoveto{\pgfqpoint{3.465524in}{2.263635in}}%
\pgfpathlineto{\pgfqpoint{3.465524in}{2.263635in}}%
\pgfpathlineto{\pgfqpoint{3.465524in}{2.266584in}}%
\pgfpathlineto{\pgfqpoint{3.470065in}{2.266584in}}%
\pgfpathlineto{\pgfqpoint{3.470065in}{2.263635in}}%
\pgfpathmoveto{\pgfqpoint{3.470065in}{2.263635in}}%
\pgfpathlineto{\pgfqpoint{3.470065in}{2.263635in}}%
\pgfpathlineto{\pgfqpoint{3.470065in}{2.266584in}}%
\pgfpathlineto{\pgfqpoint{3.474606in}{2.266584in}}%
\pgfpathlineto{\pgfqpoint{3.474606in}{2.263635in}}%
\pgfpathmoveto{\pgfqpoint{3.470065in}{2.266584in}}%
\pgfpathlineto{\pgfqpoint{3.470065in}{2.266584in}}%
\pgfpathlineto{\pgfqpoint{3.470065in}{2.269534in}}%
\pgfpathlineto{\pgfqpoint{3.474606in}{2.269534in}}%
\pgfpathlineto{\pgfqpoint{3.474606in}{2.266584in}}%
\pgfpathmoveto{\pgfqpoint{3.474606in}{2.266584in}}%
\pgfpathlineto{\pgfqpoint{3.474606in}{2.266584in}}%
\pgfpathlineto{\pgfqpoint{3.474606in}{2.269534in}}%
\pgfpathlineto{\pgfqpoint{3.479147in}{2.269534in}}%
\pgfpathlineto{\pgfqpoint{3.479147in}{2.266584in}}%
\pgfpathmoveto{\pgfqpoint{3.474606in}{2.269534in}}%
\pgfpathlineto{\pgfqpoint{3.474606in}{2.269534in}}%
\pgfpathlineto{\pgfqpoint{3.474606in}{2.272483in}}%
\pgfpathlineto{\pgfqpoint{3.479147in}{2.272483in}}%
\pgfpathlineto{\pgfqpoint{3.479147in}{2.269534in}}%
\pgfpathmoveto{\pgfqpoint{3.479147in}{2.269534in}}%
\pgfpathlineto{\pgfqpoint{3.479147in}{2.269534in}}%
\pgfpathlineto{\pgfqpoint{3.479147in}{2.272483in}}%
\pgfpathlineto{\pgfqpoint{3.483688in}{2.272483in}}%
\pgfpathlineto{\pgfqpoint{3.483688in}{2.269534in}}%
\pgfpathmoveto{\pgfqpoint{3.479147in}{2.272483in}}%
\pgfpathlineto{\pgfqpoint{3.479147in}{2.272483in}}%
\pgfpathlineto{\pgfqpoint{3.479147in}{2.275432in}}%
\pgfpathlineto{\pgfqpoint{3.483688in}{2.275432in}}%
\pgfpathlineto{\pgfqpoint{3.483688in}{2.272483in}}%
\pgfpathmoveto{\pgfqpoint{3.483688in}{2.272483in}}%
\pgfpathlineto{\pgfqpoint{3.483688in}{2.272483in}}%
\pgfpathlineto{\pgfqpoint{3.483688in}{2.275432in}}%
\pgfpathlineto{\pgfqpoint{3.488229in}{2.275432in}}%
\pgfpathlineto{\pgfqpoint{3.488229in}{2.272483in}}%
\pgfpathmoveto{\pgfqpoint{3.483688in}{2.275432in}}%
\pgfpathlineto{\pgfqpoint{3.483688in}{2.275432in}}%
\pgfpathlineto{\pgfqpoint{3.483688in}{2.278381in}}%
\pgfpathlineto{\pgfqpoint{3.488229in}{2.278381in}}%
\pgfpathlineto{\pgfqpoint{3.488229in}{2.275432in}}%
\pgfpathmoveto{\pgfqpoint{3.488229in}{2.275432in}}%
\pgfpathlineto{\pgfqpoint{3.488229in}{2.275432in}}%
\pgfpathlineto{\pgfqpoint{3.488229in}{2.278381in}}%
\pgfpathlineto{\pgfqpoint{3.492770in}{2.278381in}}%
\pgfpathlineto{\pgfqpoint{3.492770in}{2.275432in}}%
\pgfpathmoveto{\pgfqpoint{3.488229in}{2.278381in}}%
\pgfpathlineto{\pgfqpoint{3.488229in}{2.278381in}}%
\pgfpathlineto{\pgfqpoint{3.488229in}{2.281330in}}%
\pgfpathlineto{\pgfqpoint{3.492770in}{2.281330in}}%
\pgfpathlineto{\pgfqpoint{3.492770in}{2.278381in}}%
\pgfpathmoveto{\pgfqpoint{3.492770in}{2.278381in}}%
\pgfpathlineto{\pgfqpoint{3.492770in}{2.278381in}}%
\pgfpathlineto{\pgfqpoint{3.492770in}{2.281330in}}%
\pgfpathlineto{\pgfqpoint{3.497311in}{2.281330in}}%
\pgfpathlineto{\pgfqpoint{3.497311in}{2.278381in}}%
\pgfpathmoveto{\pgfqpoint{3.492770in}{2.281330in}}%
\pgfpathlineto{\pgfqpoint{3.492770in}{2.281330in}}%
\pgfpathlineto{\pgfqpoint{3.492770in}{2.284280in}}%
\pgfpathlineto{\pgfqpoint{3.497311in}{2.284280in}}%
\pgfpathlineto{\pgfqpoint{3.497311in}{2.281330in}}%
\pgfpathmoveto{\pgfqpoint{3.497311in}{2.281330in}}%
\pgfpathlineto{\pgfqpoint{3.497311in}{2.281330in}}%
\pgfpathlineto{\pgfqpoint{3.497311in}{2.284280in}}%
\pgfpathlineto{\pgfqpoint{3.501852in}{2.284280in}}%
\pgfpathlineto{\pgfqpoint{3.501852in}{2.281330in}}%
\pgfpathmoveto{\pgfqpoint{3.497311in}{2.284280in}}%
\pgfpathlineto{\pgfqpoint{3.497311in}{2.284280in}}%
\pgfpathlineto{\pgfqpoint{3.497311in}{2.287229in}}%
\pgfpathlineto{\pgfqpoint{3.501852in}{2.287229in}}%
\pgfpathlineto{\pgfqpoint{3.501852in}{2.284280in}}%
\pgfpathmoveto{\pgfqpoint{3.501852in}{2.284280in}}%
\pgfpathlineto{\pgfqpoint{3.501852in}{2.284280in}}%
\pgfpathlineto{\pgfqpoint{3.501852in}{2.287229in}}%
\pgfpathlineto{\pgfqpoint{3.506393in}{2.287229in}}%
\pgfpathlineto{\pgfqpoint{3.506393in}{2.284280in}}%
\pgfpathmoveto{\pgfqpoint{3.501852in}{2.287229in}}%
\pgfpathlineto{\pgfqpoint{3.501852in}{2.287229in}}%
\pgfpathlineto{\pgfqpoint{3.501852in}{2.290178in}}%
\pgfpathlineto{\pgfqpoint{3.506393in}{2.290178in}}%
\pgfpathlineto{\pgfqpoint{3.506393in}{2.287229in}}%
\pgfpathmoveto{\pgfqpoint{3.506393in}{2.287229in}}%
\pgfpathlineto{\pgfqpoint{3.506393in}{2.287229in}}%
\pgfpathlineto{\pgfqpoint{3.506393in}{2.290178in}}%
\pgfpathlineto{\pgfqpoint{3.510934in}{2.290178in}}%
\pgfpathlineto{\pgfqpoint{3.510934in}{2.287229in}}%
\pgfpathmoveto{\pgfqpoint{3.506393in}{2.290178in}}%
\pgfpathlineto{\pgfqpoint{3.506393in}{2.290178in}}%
\pgfpathlineto{\pgfqpoint{3.506393in}{2.293127in}}%
\pgfpathlineto{\pgfqpoint{3.510934in}{2.293127in}}%
\pgfpathlineto{\pgfqpoint{3.510934in}{2.290178in}}%
\pgfpathmoveto{\pgfqpoint{3.510934in}{2.290178in}}%
\pgfpathlineto{\pgfqpoint{3.510934in}{2.290178in}}%
\pgfpathlineto{\pgfqpoint{3.510934in}{2.293127in}}%
\pgfpathlineto{\pgfqpoint{3.515475in}{2.293127in}}%
\pgfpathlineto{\pgfqpoint{3.515475in}{2.290178in}}%
\pgfpathmoveto{\pgfqpoint{3.510934in}{2.293127in}}%
\pgfpathlineto{\pgfqpoint{3.510934in}{2.293127in}}%
\pgfpathlineto{\pgfqpoint{3.510934in}{2.296077in}}%
\pgfpathlineto{\pgfqpoint{3.515475in}{2.296077in}}%
\pgfpathlineto{\pgfqpoint{3.515475in}{2.293127in}}%
\pgfpathmoveto{\pgfqpoint{3.515475in}{2.293127in}}%
\pgfpathlineto{\pgfqpoint{3.515475in}{2.293127in}}%
\pgfpathlineto{\pgfqpoint{3.515475in}{2.296077in}}%
\pgfpathlineto{\pgfqpoint{3.520016in}{2.296077in}}%
\pgfpathlineto{\pgfqpoint{3.520016in}{2.293127in}}%
\pgfpathmoveto{\pgfqpoint{3.515475in}{2.296077in}}%
\pgfpathlineto{\pgfqpoint{3.515475in}{2.296077in}}%
\pgfpathlineto{\pgfqpoint{3.515475in}{2.299026in}}%
\pgfpathlineto{\pgfqpoint{3.520016in}{2.299026in}}%
\pgfpathlineto{\pgfqpoint{3.520016in}{2.296077in}}%
\pgfpathmoveto{\pgfqpoint{3.520016in}{2.296077in}}%
\pgfpathlineto{\pgfqpoint{3.520016in}{2.296077in}}%
\pgfpathlineto{\pgfqpoint{3.520016in}{2.299026in}}%
\pgfpathlineto{\pgfqpoint{3.524557in}{2.299026in}}%
\pgfpathlineto{\pgfqpoint{3.524557in}{2.296077in}}%
\pgfpathmoveto{\pgfqpoint{3.520016in}{2.299026in}}%
\pgfpathlineto{\pgfqpoint{3.520016in}{2.299026in}}%
\pgfpathlineto{\pgfqpoint{3.520016in}{2.301975in}}%
\pgfpathlineto{\pgfqpoint{3.524557in}{2.301975in}}%
\pgfpathlineto{\pgfqpoint{3.524557in}{2.299026in}}%
\pgfpathmoveto{\pgfqpoint{3.524557in}{2.299026in}}%
\pgfpathlineto{\pgfqpoint{3.524557in}{2.299026in}}%
\pgfpathlineto{\pgfqpoint{3.524557in}{2.301975in}}%
\pgfpathlineto{\pgfqpoint{3.529098in}{2.301975in}}%
\pgfpathlineto{\pgfqpoint{3.529098in}{2.299026in}}%
\pgfpathmoveto{\pgfqpoint{3.524557in}{2.301975in}}%
\pgfpathlineto{\pgfqpoint{3.524557in}{2.301975in}}%
\pgfpathlineto{\pgfqpoint{3.524557in}{2.304924in}}%
\pgfpathlineto{\pgfqpoint{3.529098in}{2.304924in}}%
\pgfpathlineto{\pgfqpoint{3.529098in}{2.301975in}}%
\pgfpathmoveto{\pgfqpoint{3.529098in}{2.301975in}}%
\pgfpathlineto{\pgfqpoint{3.529098in}{2.301975in}}%
\pgfpathlineto{\pgfqpoint{3.529098in}{2.304924in}}%
\pgfpathlineto{\pgfqpoint{3.533639in}{2.304924in}}%
\pgfpathlineto{\pgfqpoint{3.533639in}{2.301975in}}%
\pgfpathmoveto{\pgfqpoint{3.529098in}{2.304924in}}%
\pgfpathlineto{\pgfqpoint{3.529098in}{2.304924in}}%
\pgfpathlineto{\pgfqpoint{3.529098in}{2.307873in}}%
\pgfpathlineto{\pgfqpoint{3.533639in}{2.307873in}}%
\pgfpathlineto{\pgfqpoint{3.533639in}{2.304924in}}%
\pgfpathmoveto{\pgfqpoint{3.533639in}{2.304924in}}%
\pgfpathlineto{\pgfqpoint{3.533639in}{2.304924in}}%
\pgfpathlineto{\pgfqpoint{3.533639in}{2.307873in}}%
\pgfpathlineto{\pgfqpoint{3.538180in}{2.307873in}}%
\pgfpathlineto{\pgfqpoint{3.538180in}{2.304924in}}%
\pgfpathmoveto{\pgfqpoint{3.533639in}{2.307873in}}%
\pgfpathlineto{\pgfqpoint{3.533639in}{2.307873in}}%
\pgfpathlineto{\pgfqpoint{3.533639in}{2.310823in}}%
\pgfpathlineto{\pgfqpoint{3.538180in}{2.310823in}}%
\pgfpathlineto{\pgfqpoint{3.538180in}{2.307873in}}%
\pgfpathmoveto{\pgfqpoint{3.538180in}{2.307873in}}%
\pgfpathlineto{\pgfqpoint{3.538180in}{2.307873in}}%
\pgfpathlineto{\pgfqpoint{3.538180in}{2.310823in}}%
\pgfpathlineto{\pgfqpoint{3.542721in}{2.310823in}}%
\pgfpathlineto{\pgfqpoint{3.542721in}{2.307873in}}%
\pgfpathmoveto{\pgfqpoint{3.538180in}{2.310823in}}%
\pgfpathlineto{\pgfqpoint{3.538180in}{2.310823in}}%
\pgfpathlineto{\pgfqpoint{3.538180in}{2.313772in}}%
\pgfpathlineto{\pgfqpoint{3.542721in}{2.313772in}}%
\pgfpathlineto{\pgfqpoint{3.542721in}{2.310823in}}%
\pgfpathmoveto{\pgfqpoint{3.542721in}{2.310823in}}%
\pgfpathlineto{\pgfqpoint{3.542721in}{2.310823in}}%
\pgfpathlineto{\pgfqpoint{3.542721in}{2.313772in}}%
\pgfpathlineto{\pgfqpoint{3.547262in}{2.313772in}}%
\pgfpathlineto{\pgfqpoint{3.547262in}{2.310823in}}%
\pgfpathmoveto{\pgfqpoint{3.542721in}{2.313772in}}%
\pgfpathlineto{\pgfqpoint{3.542721in}{2.313772in}}%
\pgfpathlineto{\pgfqpoint{3.542721in}{2.316721in}}%
\pgfpathlineto{\pgfqpoint{3.547262in}{2.316721in}}%
\pgfpathlineto{\pgfqpoint{3.547262in}{2.313772in}}%
\pgfpathmoveto{\pgfqpoint{3.547262in}{2.313772in}}%
\pgfpathlineto{\pgfqpoint{3.547262in}{2.313772in}}%
\pgfpathlineto{\pgfqpoint{3.547262in}{2.316721in}}%
\pgfpathlineto{\pgfqpoint{3.551803in}{2.316721in}}%
\pgfpathlineto{\pgfqpoint{3.551803in}{2.313772in}}%
\pgfpathmoveto{\pgfqpoint{3.547262in}{2.316721in}}%
\pgfpathlineto{\pgfqpoint{3.547262in}{2.316721in}}%
\pgfpathlineto{\pgfqpoint{3.547262in}{2.319670in}}%
\pgfpathlineto{\pgfqpoint{3.551803in}{2.319670in}}%
\pgfpathlineto{\pgfqpoint{3.551803in}{2.316721in}}%
\pgfpathmoveto{\pgfqpoint{3.551803in}{2.316721in}}%
\pgfpathlineto{\pgfqpoint{3.551803in}{2.316721in}}%
\pgfpathlineto{\pgfqpoint{3.551803in}{2.319670in}}%
\pgfpathlineto{\pgfqpoint{3.556344in}{2.319670in}}%
\pgfpathlineto{\pgfqpoint{3.556344in}{2.316721in}}%
\pgfpathmoveto{\pgfqpoint{3.551803in}{2.319670in}}%
\pgfpathlineto{\pgfqpoint{3.551803in}{2.319670in}}%
\pgfpathlineto{\pgfqpoint{3.551803in}{2.322620in}}%
\pgfpathlineto{\pgfqpoint{3.556344in}{2.322620in}}%
\pgfpathlineto{\pgfqpoint{3.556344in}{2.319670in}}%
\pgfpathmoveto{\pgfqpoint{3.556344in}{2.319670in}}%
\pgfpathlineto{\pgfqpoint{3.556344in}{2.319670in}}%
\pgfpathlineto{\pgfqpoint{3.556344in}{2.322620in}}%
\pgfpathlineto{\pgfqpoint{3.560885in}{2.322620in}}%
\pgfpathlineto{\pgfqpoint{3.560885in}{2.319670in}}%
\pgfpathmoveto{\pgfqpoint{3.556344in}{2.322620in}}%
\pgfpathlineto{\pgfqpoint{3.556344in}{2.322620in}}%
\pgfpathlineto{\pgfqpoint{3.556344in}{2.325569in}}%
\pgfpathlineto{\pgfqpoint{3.560885in}{2.325569in}}%
\pgfpathlineto{\pgfqpoint{3.560885in}{2.322620in}}%
\pgfpathmoveto{\pgfqpoint{3.560885in}{2.322620in}}%
\pgfpathlineto{\pgfqpoint{3.560885in}{2.322620in}}%
\pgfpathlineto{\pgfqpoint{3.560885in}{2.325569in}}%
\pgfpathlineto{\pgfqpoint{3.565426in}{2.325569in}}%
\pgfpathlineto{\pgfqpoint{3.565426in}{2.322620in}}%
\pgfpathmoveto{\pgfqpoint{3.560885in}{2.325569in}}%
\pgfpathlineto{\pgfqpoint{3.560885in}{2.325569in}}%
\pgfpathlineto{\pgfqpoint{3.560885in}{2.328518in}}%
\pgfpathlineto{\pgfqpoint{3.565426in}{2.328518in}}%
\pgfpathlineto{\pgfqpoint{3.565426in}{2.325569in}}%
\pgfpathmoveto{\pgfqpoint{3.565426in}{2.325569in}}%
\pgfpathlineto{\pgfqpoint{3.565426in}{2.325569in}}%
\pgfpathlineto{\pgfqpoint{3.565426in}{2.328518in}}%
\pgfpathlineto{\pgfqpoint{3.569967in}{2.328518in}}%
\pgfpathlineto{\pgfqpoint{3.569967in}{2.325569in}}%
\pgfpathmoveto{\pgfqpoint{3.565426in}{2.328518in}}%
\pgfpathlineto{\pgfqpoint{3.565426in}{2.328518in}}%
\pgfpathlineto{\pgfqpoint{3.565426in}{2.331467in}}%
\pgfpathlineto{\pgfqpoint{3.569967in}{2.331467in}}%
\pgfpathlineto{\pgfqpoint{3.569967in}{2.328518in}}%
\pgfpathmoveto{\pgfqpoint{3.569967in}{2.328518in}}%
\pgfpathlineto{\pgfqpoint{3.569967in}{2.328518in}}%
\pgfpathlineto{\pgfqpoint{3.569967in}{2.331467in}}%
\pgfpathlineto{\pgfqpoint{3.574508in}{2.331467in}}%
\pgfpathlineto{\pgfqpoint{3.574508in}{2.328518in}}%
\pgfpathmoveto{\pgfqpoint{3.569967in}{2.331467in}}%
\pgfpathlineto{\pgfqpoint{3.569967in}{2.331467in}}%
\pgfpathlineto{\pgfqpoint{3.569967in}{2.334416in}}%
\pgfpathlineto{\pgfqpoint{3.574508in}{2.334416in}}%
\pgfpathlineto{\pgfqpoint{3.574508in}{2.331467in}}%
\pgfpathmoveto{\pgfqpoint{3.574508in}{2.331467in}}%
\pgfpathlineto{\pgfqpoint{3.574508in}{2.331467in}}%
\pgfpathlineto{\pgfqpoint{3.574508in}{2.334416in}}%
\pgfpathlineto{\pgfqpoint{3.579049in}{2.334416in}}%
\pgfpathlineto{\pgfqpoint{3.579049in}{2.331467in}}%
\pgfpathmoveto{\pgfqpoint{3.574508in}{2.334416in}}%
\pgfpathlineto{\pgfqpoint{3.574508in}{2.334416in}}%
\pgfpathlineto{\pgfqpoint{3.574508in}{2.337366in}}%
\pgfpathlineto{\pgfqpoint{3.579049in}{2.337366in}}%
\pgfpathlineto{\pgfqpoint{3.579049in}{2.334416in}}%
\pgfpathmoveto{\pgfqpoint{3.579049in}{2.334416in}}%
\pgfpathlineto{\pgfqpoint{3.579049in}{2.334416in}}%
\pgfpathlineto{\pgfqpoint{3.579049in}{2.337366in}}%
\pgfpathlineto{\pgfqpoint{3.583590in}{2.337366in}}%
\pgfpathlineto{\pgfqpoint{3.583590in}{2.334416in}}%
\pgfpathmoveto{\pgfqpoint{3.579049in}{2.337366in}}%
\pgfpathlineto{\pgfqpoint{3.579049in}{2.337366in}}%
\pgfpathlineto{\pgfqpoint{3.579049in}{2.340315in}}%
\pgfpathlineto{\pgfqpoint{3.583590in}{2.340315in}}%
\pgfpathlineto{\pgfqpoint{3.583590in}{2.337366in}}%
\pgfpathmoveto{\pgfqpoint{3.583590in}{2.337366in}}%
\pgfpathlineto{\pgfqpoint{3.583590in}{2.337366in}}%
\pgfpathlineto{\pgfqpoint{3.583590in}{2.340315in}}%
\pgfpathlineto{\pgfqpoint{3.588131in}{2.340315in}}%
\pgfpathlineto{\pgfqpoint{3.588131in}{2.337366in}}%
\pgfpathmoveto{\pgfqpoint{3.583590in}{2.340315in}}%
\pgfpathlineto{\pgfqpoint{3.583590in}{2.340315in}}%
\pgfpathlineto{\pgfqpoint{3.583590in}{2.343264in}}%
\pgfpathlineto{\pgfqpoint{3.588131in}{2.343264in}}%
\pgfpathlineto{\pgfqpoint{3.588131in}{2.340315in}}%
\pgfpathmoveto{\pgfqpoint{3.588131in}{2.340315in}}%
\pgfpathlineto{\pgfqpoint{3.588131in}{2.340315in}}%
\pgfpathlineto{\pgfqpoint{3.588131in}{2.343264in}}%
\pgfpathlineto{\pgfqpoint{3.592672in}{2.343264in}}%
\pgfpathlineto{\pgfqpoint{3.592672in}{2.340315in}}%
\pgfpathmoveto{\pgfqpoint{3.588131in}{2.343264in}}%
\pgfpathlineto{\pgfqpoint{3.588131in}{2.343264in}}%
\pgfpathlineto{\pgfqpoint{3.588131in}{2.346213in}}%
\pgfpathlineto{\pgfqpoint{3.592672in}{2.346213in}}%
\pgfpathlineto{\pgfqpoint{3.592672in}{2.343264in}}%
\pgfpathmoveto{\pgfqpoint{3.592672in}{2.343264in}}%
\pgfpathlineto{\pgfqpoint{3.592672in}{2.343264in}}%
\pgfpathlineto{\pgfqpoint{3.592672in}{2.346213in}}%
\pgfpathlineto{\pgfqpoint{3.597213in}{2.346213in}}%
\pgfpathlineto{\pgfqpoint{3.597213in}{2.343264in}}%
\pgfpathmoveto{\pgfqpoint{3.592672in}{2.346213in}}%
\pgfpathlineto{\pgfqpoint{3.592672in}{2.346213in}}%
\pgfpathlineto{\pgfqpoint{3.592672in}{2.349163in}}%
\pgfpathlineto{\pgfqpoint{3.597213in}{2.349163in}}%
\pgfpathlineto{\pgfqpoint{3.597213in}{2.346213in}}%
\pgfpathmoveto{\pgfqpoint{3.597213in}{2.346213in}}%
\pgfpathlineto{\pgfqpoint{3.597213in}{2.346213in}}%
\pgfpathlineto{\pgfqpoint{3.597213in}{2.349163in}}%
\pgfpathlineto{\pgfqpoint{3.601754in}{2.349163in}}%
\pgfpathlineto{\pgfqpoint{3.601754in}{2.346213in}}%
\pgfpathmoveto{\pgfqpoint{3.597213in}{2.349163in}}%
\pgfpathlineto{\pgfqpoint{3.597213in}{2.349163in}}%
\pgfpathlineto{\pgfqpoint{3.597213in}{2.352112in}}%
\pgfpathlineto{\pgfqpoint{3.601754in}{2.352112in}}%
\pgfpathlineto{\pgfqpoint{3.601754in}{2.349163in}}%
\pgfpathmoveto{\pgfqpoint{3.601754in}{2.349163in}}%
\pgfpathlineto{\pgfqpoint{3.601754in}{2.349163in}}%
\pgfpathlineto{\pgfqpoint{3.601754in}{2.352112in}}%
\pgfpathlineto{\pgfqpoint{3.606295in}{2.352112in}}%
\pgfpathlineto{\pgfqpoint{3.606295in}{2.349163in}}%
\pgfpathmoveto{\pgfqpoint{3.601754in}{2.352112in}}%
\pgfpathlineto{\pgfqpoint{3.601754in}{2.352112in}}%
\pgfpathlineto{\pgfqpoint{3.601754in}{2.355061in}}%
\pgfpathlineto{\pgfqpoint{3.606295in}{2.355061in}}%
\pgfpathlineto{\pgfqpoint{3.606295in}{2.352112in}}%
\pgfpathmoveto{\pgfqpoint{3.606295in}{2.352112in}}%
\pgfpathlineto{\pgfqpoint{3.606295in}{2.352112in}}%
\pgfpathlineto{\pgfqpoint{3.606295in}{2.355061in}}%
\pgfpathlineto{\pgfqpoint{3.610836in}{2.355061in}}%
\pgfpathlineto{\pgfqpoint{3.610836in}{2.352112in}}%
\pgfpathmoveto{\pgfqpoint{3.606295in}{2.355061in}}%
\pgfpathlineto{\pgfqpoint{3.606295in}{2.355061in}}%
\pgfpathlineto{\pgfqpoint{3.606295in}{2.358010in}}%
\pgfpathlineto{\pgfqpoint{3.610836in}{2.358010in}}%
\pgfpathlineto{\pgfqpoint{3.610836in}{2.355061in}}%
\pgfpathmoveto{\pgfqpoint{3.610836in}{2.355061in}}%
\pgfpathlineto{\pgfqpoint{3.610836in}{2.355061in}}%
\pgfpathlineto{\pgfqpoint{3.610836in}{2.358010in}}%
\pgfpathlineto{\pgfqpoint{3.615377in}{2.358010in}}%
\pgfpathlineto{\pgfqpoint{3.615377in}{2.355061in}}%
\pgfpathmoveto{\pgfqpoint{3.610836in}{2.358010in}}%
\pgfpathlineto{\pgfqpoint{3.610836in}{2.358010in}}%
\pgfpathlineto{\pgfqpoint{3.610836in}{2.360959in}}%
\pgfpathlineto{\pgfqpoint{3.615377in}{2.360959in}}%
\pgfpathlineto{\pgfqpoint{3.615377in}{2.358010in}}%
\pgfpathmoveto{\pgfqpoint{3.615377in}{2.358010in}}%
\pgfpathlineto{\pgfqpoint{3.615377in}{2.358010in}}%
\pgfpathlineto{\pgfqpoint{3.615377in}{2.360959in}}%
\pgfpathlineto{\pgfqpoint{3.619918in}{2.360959in}}%
\pgfpathlineto{\pgfqpoint{3.619918in}{2.358010in}}%
\pgfpathmoveto{\pgfqpoint{3.615377in}{2.360959in}}%
\pgfpathlineto{\pgfqpoint{3.615377in}{2.360959in}}%
\pgfpathlineto{\pgfqpoint{3.615377in}{2.363909in}}%
\pgfpathlineto{\pgfqpoint{3.619918in}{2.363909in}}%
\pgfpathlineto{\pgfqpoint{3.619918in}{2.360959in}}%
\pgfpathmoveto{\pgfqpoint{3.619918in}{2.360959in}}%
\pgfpathlineto{\pgfqpoint{3.619918in}{2.360959in}}%
\pgfpathlineto{\pgfqpoint{3.619918in}{2.363909in}}%
\pgfpathlineto{\pgfqpoint{3.624459in}{2.363909in}}%
\pgfpathlineto{\pgfqpoint{3.624459in}{2.360959in}}%
\pgfpathmoveto{\pgfqpoint{3.619918in}{2.363909in}}%
\pgfpathlineto{\pgfqpoint{3.619918in}{2.363909in}}%
\pgfpathlineto{\pgfqpoint{3.619918in}{2.366858in}}%
\pgfpathlineto{\pgfqpoint{3.624459in}{2.366858in}}%
\pgfpathlineto{\pgfqpoint{3.624459in}{2.363909in}}%
\pgfpathmoveto{\pgfqpoint{3.624459in}{2.363909in}}%
\pgfpathlineto{\pgfqpoint{3.624459in}{2.363909in}}%
\pgfpathlineto{\pgfqpoint{3.624459in}{2.366858in}}%
\pgfpathlineto{\pgfqpoint{3.629000in}{2.366858in}}%
\pgfpathlineto{\pgfqpoint{3.629000in}{2.363909in}}%
\pgfpathmoveto{\pgfqpoint{3.624459in}{2.366858in}}%
\pgfpathlineto{\pgfqpoint{3.624459in}{2.366858in}}%
\pgfpathlineto{\pgfqpoint{3.624459in}{2.369807in}}%
\pgfpathlineto{\pgfqpoint{3.629000in}{2.369807in}}%
\pgfpathlineto{\pgfqpoint{3.629000in}{2.366858in}}%
\pgfpathmoveto{\pgfqpoint{3.629000in}{2.366858in}}%
\pgfpathlineto{\pgfqpoint{3.629000in}{2.366858in}}%
\pgfpathlineto{\pgfqpoint{3.629000in}{2.369807in}}%
\pgfpathlineto{\pgfqpoint{3.633541in}{2.369807in}}%
\pgfpathlineto{\pgfqpoint{3.633541in}{2.366858in}}%
\pgfpathmoveto{\pgfqpoint{3.629000in}{2.369807in}}%
\pgfpathlineto{\pgfqpoint{3.629000in}{2.369807in}}%
\pgfpathlineto{\pgfqpoint{3.629000in}{2.372756in}}%
\pgfpathlineto{\pgfqpoint{3.633541in}{2.372756in}}%
\pgfpathlineto{\pgfqpoint{3.633541in}{2.369807in}}%
\pgfpathmoveto{\pgfqpoint{3.633541in}{2.369807in}}%
\pgfpathlineto{\pgfqpoint{3.633541in}{2.369807in}}%
\pgfpathlineto{\pgfqpoint{3.633541in}{2.372756in}}%
\pgfpathlineto{\pgfqpoint{3.638083in}{2.372756in}}%
\pgfpathlineto{\pgfqpoint{3.638083in}{2.369807in}}%
\pgfpathmoveto{\pgfqpoint{3.633541in}{2.372756in}}%
\pgfpathlineto{\pgfqpoint{3.633541in}{2.372756in}}%
\pgfpathlineto{\pgfqpoint{3.633541in}{2.375705in}}%
\pgfpathlineto{\pgfqpoint{3.638083in}{2.375705in}}%
\pgfpathlineto{\pgfqpoint{3.638083in}{2.372756in}}%
\pgfpathmoveto{\pgfqpoint{3.638083in}{2.372756in}}%
\pgfpathlineto{\pgfqpoint{3.638083in}{2.372756in}}%
\pgfpathlineto{\pgfqpoint{3.638083in}{2.375705in}}%
\pgfpathlineto{\pgfqpoint{3.642624in}{2.375705in}}%
\pgfpathlineto{\pgfqpoint{3.642624in}{2.372756in}}%
\pgfpathmoveto{\pgfqpoint{3.638083in}{2.375705in}}%
\pgfpathlineto{\pgfqpoint{3.638083in}{2.375705in}}%
\pgfpathlineto{\pgfqpoint{3.638083in}{2.378655in}}%
\pgfpathlineto{\pgfqpoint{3.642624in}{2.378655in}}%
\pgfpathlineto{\pgfqpoint{3.642624in}{2.375705in}}%
\pgfpathmoveto{\pgfqpoint{3.642624in}{2.375705in}}%
\pgfpathlineto{\pgfqpoint{3.642624in}{2.375705in}}%
\pgfpathlineto{\pgfqpoint{3.642624in}{2.378655in}}%
\pgfpathlineto{\pgfqpoint{3.647165in}{2.378655in}}%
\pgfpathlineto{\pgfqpoint{3.647165in}{2.375705in}}%
\pgfpathmoveto{\pgfqpoint{3.642624in}{2.378655in}}%
\pgfpathlineto{\pgfqpoint{3.642624in}{2.378655in}}%
\pgfpathlineto{\pgfqpoint{3.642624in}{2.381604in}}%
\pgfpathlineto{\pgfqpoint{3.647165in}{2.381604in}}%
\pgfpathlineto{\pgfqpoint{3.647165in}{2.378655in}}%
\pgfpathmoveto{\pgfqpoint{3.647165in}{2.378655in}}%
\pgfpathlineto{\pgfqpoint{3.647165in}{2.378655in}}%
\pgfpathlineto{\pgfqpoint{3.647165in}{2.381604in}}%
\pgfpathlineto{\pgfqpoint{3.651706in}{2.381604in}}%
\pgfpathlineto{\pgfqpoint{3.651706in}{2.378655in}}%
\pgfpathmoveto{\pgfqpoint{3.647165in}{2.381604in}}%
\pgfpathlineto{\pgfqpoint{3.647165in}{2.381604in}}%
\pgfpathlineto{\pgfqpoint{3.647165in}{2.384553in}}%
\pgfpathlineto{\pgfqpoint{3.651706in}{2.384553in}}%
\pgfpathlineto{\pgfqpoint{3.651706in}{2.381604in}}%
\pgfpathmoveto{\pgfqpoint{3.651706in}{2.381604in}}%
\pgfpathlineto{\pgfqpoint{3.651706in}{2.381604in}}%
\pgfpathlineto{\pgfqpoint{3.651706in}{2.384553in}}%
\pgfpathlineto{\pgfqpoint{3.656247in}{2.384553in}}%
\pgfpathlineto{\pgfqpoint{3.656247in}{2.381604in}}%
\pgfpathmoveto{\pgfqpoint{3.651706in}{2.384553in}}%
\pgfpathlineto{\pgfqpoint{3.651706in}{2.384553in}}%
\pgfpathlineto{\pgfqpoint{3.651706in}{2.387502in}}%
\pgfpathlineto{\pgfqpoint{3.656247in}{2.387502in}}%
\pgfpathlineto{\pgfqpoint{3.656247in}{2.384553in}}%
\pgfpathmoveto{\pgfqpoint{3.656247in}{2.384553in}}%
\pgfpathlineto{\pgfqpoint{3.656247in}{2.384553in}}%
\pgfpathlineto{\pgfqpoint{3.656247in}{2.387502in}}%
\pgfpathlineto{\pgfqpoint{3.660788in}{2.387502in}}%
\pgfpathlineto{\pgfqpoint{3.660788in}{2.384553in}}%
\pgfpathmoveto{\pgfqpoint{3.656247in}{2.387502in}}%
\pgfpathlineto{\pgfqpoint{3.656247in}{2.387502in}}%
\pgfpathlineto{\pgfqpoint{3.656247in}{2.390451in}}%
\pgfpathlineto{\pgfqpoint{3.660788in}{2.390451in}}%
\pgfpathlineto{\pgfqpoint{3.660788in}{2.387502in}}%
\pgfpathmoveto{\pgfqpoint{3.660788in}{2.387502in}}%
\pgfpathlineto{\pgfqpoint{3.660788in}{2.387502in}}%
\pgfpathlineto{\pgfqpoint{3.660788in}{2.390451in}}%
\pgfpathlineto{\pgfqpoint{3.665329in}{2.390451in}}%
\pgfpathlineto{\pgfqpoint{3.665329in}{2.387502in}}%
\pgfpathmoveto{\pgfqpoint{3.660788in}{2.390451in}}%
\pgfpathlineto{\pgfqpoint{3.660788in}{2.390451in}}%
\pgfpathlineto{\pgfqpoint{3.660788in}{2.393401in}}%
\pgfpathlineto{\pgfqpoint{3.665329in}{2.393401in}}%
\pgfpathlineto{\pgfqpoint{3.665329in}{2.390451in}}%
\pgfpathmoveto{\pgfqpoint{3.665329in}{2.390451in}}%
\pgfpathlineto{\pgfqpoint{3.665329in}{2.390451in}}%
\pgfpathlineto{\pgfqpoint{3.665329in}{2.393401in}}%
\pgfpathlineto{\pgfqpoint{3.669870in}{2.393401in}}%
\pgfpathlineto{\pgfqpoint{3.669870in}{2.390451in}}%
\pgfpathmoveto{\pgfqpoint{3.665329in}{2.393401in}}%
\pgfpathlineto{\pgfqpoint{3.665329in}{2.393401in}}%
\pgfpathlineto{\pgfqpoint{3.665329in}{2.396350in}}%
\pgfpathlineto{\pgfqpoint{3.669870in}{2.396350in}}%
\pgfpathlineto{\pgfqpoint{3.669870in}{2.393401in}}%
\pgfpathmoveto{\pgfqpoint{3.669870in}{2.393401in}}%
\pgfpathlineto{\pgfqpoint{3.669870in}{2.393401in}}%
\pgfpathlineto{\pgfqpoint{3.669870in}{2.396350in}}%
\pgfpathlineto{\pgfqpoint{3.674411in}{2.396350in}}%
\pgfpathlineto{\pgfqpoint{3.674411in}{2.393401in}}%
\pgfpathmoveto{\pgfqpoint{3.669870in}{2.396350in}}%
\pgfpathlineto{\pgfqpoint{3.669870in}{2.396350in}}%
\pgfpathlineto{\pgfqpoint{3.669870in}{2.399299in}}%
\pgfpathlineto{\pgfqpoint{3.674411in}{2.399299in}}%
\pgfpathlineto{\pgfqpoint{3.674411in}{2.396350in}}%
\pgfpathmoveto{\pgfqpoint{3.674411in}{2.396350in}}%
\pgfpathlineto{\pgfqpoint{3.674411in}{2.396350in}}%
\pgfpathlineto{\pgfqpoint{3.674411in}{2.399299in}}%
\pgfpathlineto{\pgfqpoint{3.678952in}{2.399299in}}%
\pgfpathlineto{\pgfqpoint{3.678952in}{2.396350in}}%
\pgfpathmoveto{\pgfqpoint{3.674411in}{2.399299in}}%
\pgfpathlineto{\pgfqpoint{3.674411in}{2.399299in}}%
\pgfpathlineto{\pgfqpoint{3.674411in}{2.402248in}}%
\pgfpathlineto{\pgfqpoint{3.678952in}{2.402248in}}%
\pgfpathlineto{\pgfqpoint{3.678952in}{2.399299in}}%
\pgfpathmoveto{\pgfqpoint{3.678952in}{2.399299in}}%
\pgfpathlineto{\pgfqpoint{3.678952in}{2.399299in}}%
\pgfpathlineto{\pgfqpoint{3.678952in}{2.402248in}}%
\pgfpathlineto{\pgfqpoint{3.683494in}{2.402248in}}%
\pgfpathlineto{\pgfqpoint{3.683494in}{2.399299in}}%
\pgfpathmoveto{\pgfqpoint{3.678952in}{2.402248in}}%
\pgfpathlineto{\pgfqpoint{3.678952in}{2.402248in}}%
\pgfpathlineto{\pgfqpoint{3.678952in}{2.405197in}}%
\pgfpathlineto{\pgfqpoint{3.683494in}{2.405197in}}%
\pgfpathlineto{\pgfqpoint{3.683494in}{2.402248in}}%
\pgfpathmoveto{\pgfqpoint{3.683494in}{2.402248in}}%
\pgfpathlineto{\pgfqpoint{3.683494in}{2.402248in}}%
\pgfpathlineto{\pgfqpoint{3.683494in}{2.405197in}}%
\pgfpathlineto{\pgfqpoint{3.688035in}{2.405197in}}%
\pgfpathlineto{\pgfqpoint{3.688035in}{2.402248in}}%
\pgfpathmoveto{\pgfqpoint{3.683494in}{2.405197in}}%
\pgfpathlineto{\pgfqpoint{3.683494in}{2.405197in}}%
\pgfpathlineto{\pgfqpoint{3.683494in}{2.408146in}}%
\pgfpathlineto{\pgfqpoint{3.688035in}{2.408146in}}%
\pgfpathlineto{\pgfqpoint{3.688035in}{2.405197in}}%
\pgfpathmoveto{\pgfqpoint{3.688035in}{2.405197in}}%
\pgfpathlineto{\pgfqpoint{3.688035in}{2.405197in}}%
\pgfpathlineto{\pgfqpoint{3.688035in}{2.408146in}}%
\pgfpathlineto{\pgfqpoint{3.692576in}{2.408146in}}%
\pgfpathlineto{\pgfqpoint{3.692576in}{2.405197in}}%
\pgfpathmoveto{\pgfqpoint{3.688035in}{2.408146in}}%
\pgfpathlineto{\pgfqpoint{3.688035in}{2.408146in}}%
\pgfpathlineto{\pgfqpoint{3.688035in}{2.411095in}}%
\pgfpathlineto{\pgfqpoint{3.692576in}{2.411095in}}%
\pgfpathlineto{\pgfqpoint{3.692576in}{2.408146in}}%
\pgfpathmoveto{\pgfqpoint{3.692576in}{2.408146in}}%
\pgfpathlineto{\pgfqpoint{3.692576in}{2.408146in}}%
\pgfpathlineto{\pgfqpoint{3.692576in}{2.411095in}}%
\pgfpathlineto{\pgfqpoint{3.697117in}{2.411095in}}%
\pgfpathlineto{\pgfqpoint{3.697117in}{2.408146in}}%
\pgfpathmoveto{\pgfqpoint{3.692576in}{2.411095in}}%
\pgfpathlineto{\pgfqpoint{3.692576in}{2.411095in}}%
\pgfpathlineto{\pgfqpoint{3.692576in}{2.414045in}}%
\pgfpathlineto{\pgfqpoint{3.697117in}{2.414045in}}%
\pgfpathlineto{\pgfqpoint{3.697117in}{2.411095in}}%
\pgfpathmoveto{\pgfqpoint{3.697117in}{2.411095in}}%
\pgfpathlineto{\pgfqpoint{3.697117in}{2.411095in}}%
\pgfpathlineto{\pgfqpoint{3.697117in}{2.414045in}}%
\pgfpathlineto{\pgfqpoint{3.701658in}{2.414045in}}%
\pgfpathlineto{\pgfqpoint{3.701658in}{2.411095in}}%
\pgfpathmoveto{\pgfqpoint{3.697117in}{2.414045in}}%
\pgfpathlineto{\pgfqpoint{3.697117in}{2.414045in}}%
\pgfpathlineto{\pgfqpoint{3.697117in}{2.416994in}}%
\pgfpathlineto{\pgfqpoint{3.701658in}{2.416994in}}%
\pgfpathlineto{\pgfqpoint{3.701658in}{2.414045in}}%
\pgfpathmoveto{\pgfqpoint{3.701658in}{2.414045in}}%
\pgfpathlineto{\pgfqpoint{3.701658in}{2.414045in}}%
\pgfpathlineto{\pgfqpoint{3.701658in}{2.416994in}}%
\pgfpathlineto{\pgfqpoint{3.706199in}{2.416994in}}%
\pgfpathlineto{\pgfqpoint{3.706199in}{2.414045in}}%
\pgfpathmoveto{\pgfqpoint{3.701658in}{2.416994in}}%
\pgfpathlineto{\pgfqpoint{3.701658in}{2.416994in}}%
\pgfpathlineto{\pgfqpoint{3.701658in}{2.419943in}}%
\pgfpathlineto{\pgfqpoint{3.706199in}{2.419943in}}%
\pgfpathlineto{\pgfqpoint{3.706199in}{2.416994in}}%
\pgfpathmoveto{\pgfqpoint{3.706199in}{2.416994in}}%
\pgfpathlineto{\pgfqpoint{3.706199in}{2.416994in}}%
\pgfpathlineto{\pgfqpoint{3.706199in}{2.419943in}}%
\pgfpathlineto{\pgfqpoint{3.710741in}{2.419943in}}%
\pgfpathlineto{\pgfqpoint{3.710741in}{2.416994in}}%
\pgfpathmoveto{\pgfqpoint{3.706199in}{2.419943in}}%
\pgfpathlineto{\pgfqpoint{3.706199in}{2.419943in}}%
\pgfpathlineto{\pgfqpoint{3.706199in}{2.422892in}}%
\pgfpathlineto{\pgfqpoint{3.710741in}{2.422892in}}%
\pgfpathlineto{\pgfqpoint{3.710741in}{2.419943in}}%
\pgfpathmoveto{\pgfqpoint{3.710741in}{2.419943in}}%
\pgfpathlineto{\pgfqpoint{3.710741in}{2.419943in}}%
\pgfpathlineto{\pgfqpoint{3.710741in}{2.422892in}}%
\pgfpathlineto{\pgfqpoint{3.715282in}{2.422892in}}%
\pgfpathlineto{\pgfqpoint{3.715282in}{2.419943in}}%
\pgfpathmoveto{\pgfqpoint{3.710741in}{2.422892in}}%
\pgfpathlineto{\pgfqpoint{3.710741in}{2.422892in}}%
\pgfpathlineto{\pgfqpoint{3.710741in}{2.425841in}}%
\pgfpathlineto{\pgfqpoint{3.715282in}{2.425841in}}%
\pgfpathlineto{\pgfqpoint{3.715282in}{2.422892in}}%
\pgfpathmoveto{\pgfqpoint{3.715282in}{2.422892in}}%
\pgfpathlineto{\pgfqpoint{3.715282in}{2.422892in}}%
\pgfpathlineto{\pgfqpoint{3.715282in}{2.425841in}}%
\pgfpathlineto{\pgfqpoint{3.719823in}{2.425841in}}%
\pgfpathlineto{\pgfqpoint{3.719823in}{2.422892in}}%
\pgfpathmoveto{\pgfqpoint{3.715282in}{2.425841in}}%
\pgfpathlineto{\pgfqpoint{3.715282in}{2.425841in}}%
\pgfpathlineto{\pgfqpoint{3.715282in}{2.428790in}}%
\pgfpathlineto{\pgfqpoint{3.719823in}{2.428790in}}%
\pgfpathlineto{\pgfqpoint{3.719823in}{2.425841in}}%
\pgfpathmoveto{\pgfqpoint{3.719823in}{2.425841in}}%
\pgfpathlineto{\pgfqpoint{3.719823in}{2.425841in}}%
\pgfpathlineto{\pgfqpoint{3.719823in}{2.428790in}}%
\pgfpathlineto{\pgfqpoint{3.724364in}{2.428790in}}%
\pgfpathlineto{\pgfqpoint{3.724364in}{2.425841in}}%
\pgfpathmoveto{\pgfqpoint{3.719823in}{2.428790in}}%
\pgfpathlineto{\pgfqpoint{3.719823in}{2.428790in}}%
\pgfpathlineto{\pgfqpoint{3.719823in}{2.431739in}}%
\pgfpathlineto{\pgfqpoint{3.724364in}{2.431739in}}%
\pgfpathlineto{\pgfqpoint{3.724364in}{2.428790in}}%
\pgfpathmoveto{\pgfqpoint{3.724364in}{2.428790in}}%
\pgfpathlineto{\pgfqpoint{3.724364in}{2.428790in}}%
\pgfpathlineto{\pgfqpoint{3.724364in}{2.431739in}}%
\pgfpathlineto{\pgfqpoint{3.728905in}{2.431739in}}%
\pgfpathlineto{\pgfqpoint{3.728905in}{2.428790in}}%
\pgfpathmoveto{\pgfqpoint{3.724364in}{2.431739in}}%
\pgfpathlineto{\pgfqpoint{3.724364in}{2.431739in}}%
\pgfpathlineto{\pgfqpoint{3.724364in}{2.434688in}}%
\pgfpathlineto{\pgfqpoint{3.728905in}{2.434688in}}%
\pgfpathlineto{\pgfqpoint{3.728905in}{2.431739in}}%
\pgfpathmoveto{\pgfqpoint{3.728905in}{2.431739in}}%
\pgfpathlineto{\pgfqpoint{3.728905in}{2.431739in}}%
\pgfpathlineto{\pgfqpoint{3.728905in}{2.434688in}}%
\pgfpathlineto{\pgfqpoint{3.733446in}{2.434688in}}%
\pgfpathlineto{\pgfqpoint{3.733446in}{2.431739in}}%
\pgfpathmoveto{\pgfqpoint{3.728905in}{2.434688in}}%
\pgfpathlineto{\pgfqpoint{3.728905in}{2.434688in}}%
\pgfpathlineto{\pgfqpoint{3.728905in}{2.437638in}}%
\pgfpathlineto{\pgfqpoint{3.733446in}{2.437638in}}%
\pgfpathlineto{\pgfqpoint{3.733446in}{2.434688in}}%
\pgfpathmoveto{\pgfqpoint{3.733446in}{2.434688in}}%
\pgfpathlineto{\pgfqpoint{3.733446in}{2.434688in}}%
\pgfpathlineto{\pgfqpoint{3.733446in}{2.437638in}}%
\pgfpathlineto{\pgfqpoint{3.737987in}{2.437638in}}%
\pgfpathlineto{\pgfqpoint{3.737987in}{2.434688in}}%
\pgfpathmoveto{\pgfqpoint{3.733446in}{2.437638in}}%
\pgfpathlineto{\pgfqpoint{3.733446in}{2.437638in}}%
\pgfpathlineto{\pgfqpoint{3.733446in}{2.440587in}}%
\pgfpathlineto{\pgfqpoint{3.737987in}{2.440587in}}%
\pgfpathlineto{\pgfqpoint{3.737987in}{2.437638in}}%
\pgfpathmoveto{\pgfqpoint{3.737987in}{2.437638in}}%
\pgfpathlineto{\pgfqpoint{3.737987in}{2.437638in}}%
\pgfpathlineto{\pgfqpoint{3.737987in}{2.440587in}}%
\pgfpathlineto{\pgfqpoint{3.742529in}{2.440587in}}%
\pgfpathlineto{\pgfqpoint{3.742529in}{2.437638in}}%
\pgfpathmoveto{\pgfqpoint{3.737987in}{2.440587in}}%
\pgfpathlineto{\pgfqpoint{3.737987in}{2.440587in}}%
\pgfpathlineto{\pgfqpoint{3.737987in}{2.443536in}}%
\pgfpathlineto{\pgfqpoint{3.742529in}{2.443536in}}%
\pgfpathlineto{\pgfqpoint{3.742529in}{2.440587in}}%
\pgfpathmoveto{\pgfqpoint{3.742529in}{2.440587in}}%
\pgfpathlineto{\pgfqpoint{3.742529in}{2.440587in}}%
\pgfpathlineto{\pgfqpoint{3.742529in}{2.443536in}}%
\pgfpathlineto{\pgfqpoint{3.747070in}{2.443536in}}%
\pgfpathlineto{\pgfqpoint{3.747070in}{2.440587in}}%
\pgfpathmoveto{\pgfqpoint{3.742529in}{2.443536in}}%
\pgfpathlineto{\pgfqpoint{3.742529in}{2.443536in}}%
\pgfpathlineto{\pgfqpoint{3.742529in}{2.446485in}}%
\pgfpathlineto{\pgfqpoint{3.747070in}{2.446485in}}%
\pgfpathlineto{\pgfqpoint{3.747070in}{2.443536in}}%
\pgfpathmoveto{\pgfqpoint{3.747070in}{2.443536in}}%
\pgfpathlineto{\pgfqpoint{3.747070in}{2.443536in}}%
\pgfpathlineto{\pgfqpoint{3.747070in}{2.446485in}}%
\pgfpathlineto{\pgfqpoint{3.751611in}{2.446485in}}%
\pgfpathlineto{\pgfqpoint{3.751611in}{2.443536in}}%
\pgfpathmoveto{\pgfqpoint{3.747070in}{2.446485in}}%
\pgfpathlineto{\pgfqpoint{3.747070in}{2.446485in}}%
\pgfpathlineto{\pgfqpoint{3.747070in}{2.449434in}}%
\pgfpathlineto{\pgfqpoint{3.751611in}{2.449434in}}%
\pgfpathlineto{\pgfqpoint{3.751611in}{2.446485in}}%
\pgfpathmoveto{\pgfqpoint{3.751611in}{2.446485in}}%
\pgfpathlineto{\pgfqpoint{3.751611in}{2.446485in}}%
\pgfpathlineto{\pgfqpoint{3.751611in}{2.449434in}}%
\pgfpathlineto{\pgfqpoint{3.756152in}{2.449434in}}%
\pgfpathlineto{\pgfqpoint{3.756152in}{2.446485in}}%
\pgfpathmoveto{\pgfqpoint{3.751611in}{2.449434in}}%
\pgfpathlineto{\pgfqpoint{3.751611in}{2.449434in}}%
\pgfpathlineto{\pgfqpoint{3.751611in}{2.452383in}}%
\pgfpathlineto{\pgfqpoint{3.756152in}{2.452383in}}%
\pgfpathlineto{\pgfqpoint{3.756152in}{2.449434in}}%
\pgfpathmoveto{\pgfqpoint{3.756152in}{2.449434in}}%
\pgfpathlineto{\pgfqpoint{3.756152in}{2.449434in}}%
\pgfpathlineto{\pgfqpoint{3.756152in}{2.452383in}}%
\pgfpathlineto{\pgfqpoint{3.760693in}{2.452383in}}%
\pgfpathlineto{\pgfqpoint{3.760693in}{2.449434in}}%
\pgfpathmoveto{\pgfqpoint{3.756152in}{2.452383in}}%
\pgfpathlineto{\pgfqpoint{3.756152in}{2.452383in}}%
\pgfpathlineto{\pgfqpoint{3.756152in}{2.455332in}}%
\pgfpathlineto{\pgfqpoint{3.760693in}{2.455332in}}%
\pgfpathlineto{\pgfqpoint{3.760693in}{2.452383in}}%
\pgfpathmoveto{\pgfqpoint{3.760693in}{2.452383in}}%
\pgfpathlineto{\pgfqpoint{3.760693in}{2.452383in}}%
\pgfpathlineto{\pgfqpoint{3.760693in}{2.455332in}}%
\pgfpathlineto{\pgfqpoint{3.765234in}{2.455332in}}%
\pgfpathlineto{\pgfqpoint{3.765234in}{2.452383in}}%
\pgfpathmoveto{\pgfqpoint{3.760693in}{2.455332in}}%
\pgfpathlineto{\pgfqpoint{3.760693in}{2.455332in}}%
\pgfpathlineto{\pgfqpoint{3.760693in}{2.458281in}}%
\pgfpathlineto{\pgfqpoint{3.765234in}{2.458281in}}%
\pgfpathlineto{\pgfqpoint{3.765234in}{2.455332in}}%
\pgfpathmoveto{\pgfqpoint{3.765234in}{2.455332in}}%
\pgfpathlineto{\pgfqpoint{3.765234in}{2.455332in}}%
\pgfpathlineto{\pgfqpoint{3.765234in}{2.458281in}}%
\pgfpathlineto{\pgfqpoint{3.769776in}{2.458281in}}%
\pgfpathlineto{\pgfqpoint{3.769776in}{2.455332in}}%
\pgfpathmoveto{\pgfqpoint{3.765234in}{2.458281in}}%
\pgfpathlineto{\pgfqpoint{3.765234in}{2.458281in}}%
\pgfpathlineto{\pgfqpoint{3.765234in}{2.461231in}}%
\pgfpathlineto{\pgfqpoint{3.769776in}{2.461231in}}%
\pgfpathlineto{\pgfqpoint{3.769776in}{2.458281in}}%
\pgfpathmoveto{\pgfqpoint{3.765234in}{2.461231in}}%
\pgfpathlineto{\pgfqpoint{3.765234in}{2.461231in}}%
\pgfpathlineto{\pgfqpoint{3.765234in}{2.464180in}}%
\pgfpathlineto{\pgfqpoint{3.769776in}{2.464180in}}%
\pgfpathlineto{\pgfqpoint{3.769776in}{2.461231in}}%
\pgfpathmoveto{\pgfqpoint{3.769776in}{2.461231in}}%
\pgfpathlineto{\pgfqpoint{3.769776in}{2.461231in}}%
\pgfpathlineto{\pgfqpoint{3.769776in}{2.464180in}}%
\pgfpathlineto{\pgfqpoint{3.774317in}{2.464180in}}%
\pgfpathlineto{\pgfqpoint{3.774317in}{2.461231in}}%
\pgfpathmoveto{\pgfqpoint{3.769776in}{2.464180in}}%
\pgfpathlineto{\pgfqpoint{3.769776in}{2.464180in}}%
\pgfpathlineto{\pgfqpoint{3.769776in}{2.467129in}}%
\pgfpathlineto{\pgfqpoint{3.774317in}{2.467129in}}%
\pgfpathlineto{\pgfqpoint{3.774317in}{2.464180in}}%
\pgfpathmoveto{\pgfqpoint{3.774317in}{2.464180in}}%
\pgfpathlineto{\pgfqpoint{3.774317in}{2.464180in}}%
\pgfpathlineto{\pgfqpoint{3.774317in}{2.467129in}}%
\pgfpathlineto{\pgfqpoint{3.778858in}{2.467129in}}%
\pgfpathlineto{\pgfqpoint{3.778858in}{2.464180in}}%
\pgfpathmoveto{\pgfqpoint{3.774317in}{2.467129in}}%
\pgfpathlineto{\pgfqpoint{3.774317in}{2.467129in}}%
\pgfpathlineto{\pgfqpoint{3.774317in}{2.470078in}}%
\pgfpathlineto{\pgfqpoint{3.778858in}{2.470078in}}%
\pgfpathlineto{\pgfqpoint{3.778858in}{2.467129in}}%
\pgfpathmoveto{\pgfqpoint{3.778858in}{2.467129in}}%
\pgfpathlineto{\pgfqpoint{3.778858in}{2.467129in}}%
\pgfpathlineto{\pgfqpoint{3.778858in}{2.470078in}}%
\pgfpathlineto{\pgfqpoint{3.783399in}{2.470078in}}%
\pgfpathlineto{\pgfqpoint{3.783399in}{2.467129in}}%
\pgfpathmoveto{\pgfqpoint{3.778858in}{2.470078in}}%
\pgfpathlineto{\pgfqpoint{3.778858in}{2.470078in}}%
\pgfpathlineto{\pgfqpoint{3.778858in}{2.473027in}}%
\pgfpathlineto{\pgfqpoint{3.783399in}{2.473027in}}%
\pgfpathlineto{\pgfqpoint{3.783399in}{2.470078in}}%
\pgfpathmoveto{\pgfqpoint{3.783399in}{2.470078in}}%
\pgfpathlineto{\pgfqpoint{3.783399in}{2.470078in}}%
\pgfpathlineto{\pgfqpoint{3.783399in}{2.473027in}}%
\pgfpathlineto{\pgfqpoint{3.787940in}{2.473027in}}%
\pgfpathlineto{\pgfqpoint{3.787940in}{2.470078in}}%
\pgfpathmoveto{\pgfqpoint{3.783399in}{2.473027in}}%
\pgfpathlineto{\pgfqpoint{3.783399in}{2.473027in}}%
\pgfpathlineto{\pgfqpoint{3.783399in}{2.475976in}}%
\pgfpathlineto{\pgfqpoint{3.787940in}{2.475976in}}%
\pgfpathlineto{\pgfqpoint{3.787940in}{2.473027in}}%
\pgfpathmoveto{\pgfqpoint{3.787940in}{2.473027in}}%
\pgfpathlineto{\pgfqpoint{3.787940in}{2.473027in}}%
\pgfpathlineto{\pgfqpoint{3.787940in}{2.475976in}}%
\pgfpathlineto{\pgfqpoint{3.792481in}{2.475976in}}%
\pgfpathlineto{\pgfqpoint{3.792481in}{2.473027in}}%
\pgfpathmoveto{\pgfqpoint{3.787940in}{2.475976in}}%
\pgfpathlineto{\pgfqpoint{3.787940in}{2.475976in}}%
\pgfpathlineto{\pgfqpoint{3.787940in}{2.478925in}}%
\pgfpathlineto{\pgfqpoint{3.792481in}{2.478925in}}%
\pgfpathlineto{\pgfqpoint{3.792481in}{2.475976in}}%
\pgfpathmoveto{\pgfqpoint{3.792481in}{2.475976in}}%
\pgfpathlineto{\pgfqpoint{3.792481in}{2.475976in}}%
\pgfpathlineto{\pgfqpoint{3.792481in}{2.478925in}}%
\pgfpathlineto{\pgfqpoint{3.797023in}{2.478925in}}%
\pgfpathlineto{\pgfqpoint{3.797023in}{2.475976in}}%
\pgfpathmoveto{\pgfqpoint{3.792481in}{2.478925in}}%
\pgfpathlineto{\pgfqpoint{3.792481in}{2.478925in}}%
\pgfpathlineto{\pgfqpoint{3.792481in}{2.481874in}}%
\pgfpathlineto{\pgfqpoint{3.797023in}{2.481874in}}%
\pgfpathlineto{\pgfqpoint{3.797023in}{2.478925in}}%
\pgfpathmoveto{\pgfqpoint{3.797023in}{2.478925in}}%
\pgfpathlineto{\pgfqpoint{3.797023in}{2.478925in}}%
\pgfpathlineto{\pgfqpoint{3.797023in}{2.481874in}}%
\pgfpathlineto{\pgfqpoint{3.801564in}{2.481874in}}%
\pgfpathlineto{\pgfqpoint{3.801564in}{2.478925in}}%
\pgfpathmoveto{\pgfqpoint{3.797023in}{2.481874in}}%
\pgfpathlineto{\pgfqpoint{3.797023in}{2.481874in}}%
\pgfpathlineto{\pgfqpoint{3.797023in}{2.484824in}}%
\pgfpathlineto{\pgfqpoint{3.801564in}{2.484824in}}%
\pgfpathlineto{\pgfqpoint{3.801564in}{2.481874in}}%
\pgfpathmoveto{\pgfqpoint{3.801564in}{2.481874in}}%
\pgfpathlineto{\pgfqpoint{3.801564in}{2.481874in}}%
\pgfpathlineto{\pgfqpoint{3.801564in}{2.484824in}}%
\pgfpathlineto{\pgfqpoint{3.806105in}{2.484824in}}%
\pgfpathlineto{\pgfqpoint{3.806105in}{2.481874in}}%
\pgfpathmoveto{\pgfqpoint{3.801564in}{2.484824in}}%
\pgfpathlineto{\pgfqpoint{3.801564in}{2.484824in}}%
\pgfpathlineto{\pgfqpoint{3.801564in}{2.487773in}}%
\pgfpathlineto{\pgfqpoint{3.806105in}{2.487773in}}%
\pgfpathlineto{\pgfqpoint{3.806105in}{2.484824in}}%
\pgfpathmoveto{\pgfqpoint{3.806105in}{2.484824in}}%
\pgfpathlineto{\pgfqpoint{3.806105in}{2.484824in}}%
\pgfpathlineto{\pgfqpoint{3.806105in}{2.487773in}}%
\pgfpathlineto{\pgfqpoint{3.810645in}{2.487773in}}%
\pgfpathlineto{\pgfqpoint{3.810645in}{2.484824in}}%
\pgfpathmoveto{\pgfqpoint{3.806105in}{2.487773in}}%
\pgfpathlineto{\pgfqpoint{3.806105in}{2.487773in}}%
\pgfpathlineto{\pgfqpoint{3.806105in}{2.490722in}}%
\pgfpathlineto{\pgfqpoint{3.810645in}{2.490722in}}%
\pgfpathlineto{\pgfqpoint{3.810645in}{2.487773in}}%
\pgfpathmoveto{\pgfqpoint{3.810645in}{2.487773in}}%
\pgfpathlineto{\pgfqpoint{3.810645in}{2.487773in}}%
\pgfpathlineto{\pgfqpoint{3.810645in}{2.490722in}}%
\pgfpathlineto{\pgfqpoint{3.815186in}{2.490722in}}%
\pgfpathlineto{\pgfqpoint{3.815186in}{2.487773in}}%
\pgfpathmoveto{\pgfqpoint{3.810645in}{2.490722in}}%
\pgfpathlineto{\pgfqpoint{3.810645in}{2.490722in}}%
\pgfpathlineto{\pgfqpoint{3.810645in}{2.493672in}}%
\pgfpathlineto{\pgfqpoint{3.815186in}{2.493672in}}%
\pgfpathlineto{\pgfqpoint{3.815186in}{2.490722in}}%
\pgfpathmoveto{\pgfqpoint{3.815186in}{2.490722in}}%
\pgfpathlineto{\pgfqpoint{3.815186in}{2.490722in}}%
\pgfpathlineto{\pgfqpoint{3.815186in}{2.493672in}}%
\pgfpathlineto{\pgfqpoint{3.819727in}{2.493672in}}%
\pgfpathlineto{\pgfqpoint{3.819727in}{2.490722in}}%
\pgfpathmoveto{\pgfqpoint{3.815186in}{2.493672in}}%
\pgfpathlineto{\pgfqpoint{3.815186in}{2.493672in}}%
\pgfpathlineto{\pgfqpoint{3.815186in}{2.496621in}}%
\pgfpathlineto{\pgfqpoint{3.819727in}{2.496621in}}%
\pgfpathlineto{\pgfqpoint{3.819727in}{2.493672in}}%
\pgfpathmoveto{\pgfqpoint{3.819727in}{2.493672in}}%
\pgfpathlineto{\pgfqpoint{3.819727in}{2.493672in}}%
\pgfpathlineto{\pgfqpoint{3.819727in}{2.496621in}}%
\pgfpathlineto{\pgfqpoint{3.824268in}{2.496621in}}%
\pgfpathlineto{\pgfqpoint{3.824268in}{2.493672in}}%
\pgfpathmoveto{\pgfqpoint{3.819727in}{2.496621in}}%
\pgfpathlineto{\pgfqpoint{3.819727in}{2.496621in}}%
\pgfpathlineto{\pgfqpoint{3.819727in}{2.499570in}}%
\pgfpathlineto{\pgfqpoint{3.824268in}{2.499570in}}%
\pgfpathlineto{\pgfqpoint{3.824268in}{2.496621in}}%
\pgfpathmoveto{\pgfqpoint{3.824268in}{2.496621in}}%
\pgfpathlineto{\pgfqpoint{3.824268in}{2.496621in}}%
\pgfpathlineto{\pgfqpoint{3.824268in}{2.499570in}}%
\pgfpathlineto{\pgfqpoint{3.828809in}{2.499570in}}%
\pgfpathlineto{\pgfqpoint{3.828809in}{2.496621in}}%
\pgfpathmoveto{\pgfqpoint{3.824268in}{2.499570in}}%
\pgfpathlineto{\pgfqpoint{3.824268in}{2.499570in}}%
\pgfpathlineto{\pgfqpoint{3.824268in}{2.502520in}}%
\pgfpathlineto{\pgfqpoint{3.828809in}{2.502520in}}%
\pgfpathlineto{\pgfqpoint{3.828809in}{2.499570in}}%
\pgfpathmoveto{\pgfqpoint{3.828809in}{2.499570in}}%
\pgfpathlineto{\pgfqpoint{3.828809in}{2.499570in}}%
\pgfpathlineto{\pgfqpoint{3.828809in}{2.502520in}}%
\pgfpathlineto{\pgfqpoint{3.833350in}{2.502520in}}%
\pgfpathlineto{\pgfqpoint{3.833350in}{2.499570in}}%
\pgfpathmoveto{\pgfqpoint{3.828809in}{2.502520in}}%
\pgfpathlineto{\pgfqpoint{3.828809in}{2.502520in}}%
\pgfpathlineto{\pgfqpoint{3.828809in}{2.505469in}}%
\pgfpathlineto{\pgfqpoint{3.833350in}{2.505469in}}%
\pgfpathlineto{\pgfqpoint{3.833350in}{2.502520in}}%
\pgfpathmoveto{\pgfqpoint{3.833350in}{2.502520in}}%
\pgfpathlineto{\pgfqpoint{3.833350in}{2.502520in}}%
\pgfpathlineto{\pgfqpoint{3.833350in}{2.505469in}}%
\pgfpathlineto{\pgfqpoint{3.837891in}{2.505469in}}%
\pgfpathlineto{\pgfqpoint{3.837891in}{2.502520in}}%
\pgfpathmoveto{\pgfqpoint{3.837891in}{2.502520in}}%
\pgfpathlineto{\pgfqpoint{3.837891in}{2.502520in}}%
\pgfpathlineto{\pgfqpoint{3.837891in}{2.505469in}}%
\pgfpathlineto{\pgfqpoint{3.842432in}{2.505469in}}%
\pgfpathlineto{\pgfqpoint{3.842432in}{2.502520in}}%
\pgfpathmoveto{\pgfqpoint{3.837891in}{2.505469in}}%
\pgfpathlineto{\pgfqpoint{3.837891in}{2.505469in}}%
\pgfpathlineto{\pgfqpoint{3.837891in}{2.508418in}}%
\pgfpathlineto{\pgfqpoint{3.842432in}{2.508418in}}%
\pgfpathlineto{\pgfqpoint{3.842432in}{2.505469in}}%
\pgfpathmoveto{\pgfqpoint{3.842432in}{2.505469in}}%
\pgfpathlineto{\pgfqpoint{3.842432in}{2.505469in}}%
\pgfpathlineto{\pgfqpoint{3.842432in}{2.508418in}}%
\pgfpathlineto{\pgfqpoint{3.846972in}{2.508418in}}%
\pgfpathlineto{\pgfqpoint{3.846972in}{2.505469in}}%
\pgfpathmoveto{\pgfqpoint{3.842432in}{2.508418in}}%
\pgfpathlineto{\pgfqpoint{3.842432in}{2.508418in}}%
\pgfpathlineto{\pgfqpoint{3.842432in}{2.511367in}}%
\pgfpathlineto{\pgfqpoint{3.846972in}{2.511367in}}%
\pgfpathlineto{\pgfqpoint{3.846972in}{2.508418in}}%
\pgfpathmoveto{\pgfqpoint{3.846972in}{2.508418in}}%
\pgfpathlineto{\pgfqpoint{3.846972in}{2.508418in}}%
\pgfpathlineto{\pgfqpoint{3.846972in}{2.511367in}}%
\pgfpathlineto{\pgfqpoint{3.851513in}{2.511367in}}%
\pgfpathlineto{\pgfqpoint{3.851513in}{2.508418in}}%
\pgfpathmoveto{\pgfqpoint{3.846972in}{2.511367in}}%
\pgfpathlineto{\pgfqpoint{3.846972in}{2.511367in}}%
\pgfpathlineto{\pgfqpoint{3.846972in}{2.514317in}}%
\pgfpathlineto{\pgfqpoint{3.851513in}{2.514317in}}%
\pgfpathlineto{\pgfqpoint{3.851513in}{2.511367in}}%
\pgfpathmoveto{\pgfqpoint{3.851513in}{2.511367in}}%
\pgfpathlineto{\pgfqpoint{3.851513in}{2.511367in}}%
\pgfpathlineto{\pgfqpoint{3.851513in}{2.514317in}}%
\pgfpathlineto{\pgfqpoint{3.856054in}{2.514317in}}%
\pgfpathlineto{\pgfqpoint{3.856054in}{2.511367in}}%
\pgfpathmoveto{\pgfqpoint{3.851513in}{2.514317in}}%
\pgfpathlineto{\pgfqpoint{3.851513in}{2.514317in}}%
\pgfpathlineto{\pgfqpoint{3.851513in}{2.517266in}}%
\pgfpathlineto{\pgfqpoint{3.856054in}{2.517266in}}%
\pgfpathlineto{\pgfqpoint{3.856054in}{2.514317in}}%
\pgfpathmoveto{\pgfqpoint{3.856054in}{2.514317in}}%
\pgfpathlineto{\pgfqpoint{3.856054in}{2.514317in}}%
\pgfpathlineto{\pgfqpoint{3.856054in}{2.517266in}}%
\pgfpathlineto{\pgfqpoint{3.860595in}{2.517266in}}%
\pgfpathlineto{\pgfqpoint{3.860595in}{2.514317in}}%
\pgfpathmoveto{\pgfqpoint{3.856054in}{2.517266in}}%
\pgfpathlineto{\pgfqpoint{3.856054in}{2.517266in}}%
\pgfpathlineto{\pgfqpoint{3.856054in}{2.520215in}}%
\pgfpathlineto{\pgfqpoint{3.860595in}{2.520215in}}%
\pgfpathlineto{\pgfqpoint{3.860595in}{2.517266in}}%
\pgfpathmoveto{\pgfqpoint{3.860595in}{2.517266in}}%
\pgfpathlineto{\pgfqpoint{3.860595in}{2.517266in}}%
\pgfpathlineto{\pgfqpoint{3.860595in}{2.520215in}}%
\pgfpathlineto{\pgfqpoint{3.865136in}{2.520215in}}%
\pgfpathlineto{\pgfqpoint{3.865136in}{2.517266in}}%
\pgfpathmoveto{\pgfqpoint{3.860595in}{2.520215in}}%
\pgfpathlineto{\pgfqpoint{3.860595in}{2.520215in}}%
\pgfpathlineto{\pgfqpoint{3.860595in}{2.523165in}}%
\pgfpathlineto{\pgfqpoint{3.865136in}{2.523165in}}%
\pgfpathlineto{\pgfqpoint{3.865136in}{2.520215in}}%
\pgfpathmoveto{\pgfqpoint{3.865136in}{2.520215in}}%
\pgfpathlineto{\pgfqpoint{3.865136in}{2.520215in}}%
\pgfpathlineto{\pgfqpoint{3.865136in}{2.523165in}}%
\pgfpathlineto{\pgfqpoint{3.869677in}{2.523165in}}%
\pgfpathlineto{\pgfqpoint{3.869677in}{2.520215in}}%
\pgfpathmoveto{\pgfqpoint{3.865136in}{2.523165in}}%
\pgfpathlineto{\pgfqpoint{3.865136in}{2.523165in}}%
\pgfpathlineto{\pgfqpoint{3.865136in}{2.526114in}}%
\pgfpathlineto{\pgfqpoint{3.869677in}{2.526114in}}%
\pgfpathlineto{\pgfqpoint{3.869677in}{2.523165in}}%
\pgfpathmoveto{\pgfqpoint{3.869677in}{2.523165in}}%
\pgfpathlineto{\pgfqpoint{3.869677in}{2.523165in}}%
\pgfpathlineto{\pgfqpoint{3.869677in}{2.526114in}}%
\pgfpathlineto{\pgfqpoint{3.874218in}{2.526114in}}%
\pgfpathlineto{\pgfqpoint{3.874218in}{2.523165in}}%
\pgfpathmoveto{\pgfqpoint{3.869677in}{2.526114in}}%
\pgfpathlineto{\pgfqpoint{3.869677in}{2.526114in}}%
\pgfpathlineto{\pgfqpoint{3.869677in}{2.529063in}}%
\pgfpathlineto{\pgfqpoint{3.874218in}{2.529063in}}%
\pgfpathlineto{\pgfqpoint{3.874218in}{2.526114in}}%
\pgfpathmoveto{\pgfqpoint{3.874218in}{2.526114in}}%
\pgfpathlineto{\pgfqpoint{3.874218in}{2.526114in}}%
\pgfpathlineto{\pgfqpoint{3.874218in}{2.529063in}}%
\pgfpathlineto{\pgfqpoint{3.878758in}{2.529063in}}%
\pgfpathlineto{\pgfqpoint{3.878758in}{2.526114in}}%
\pgfpathmoveto{\pgfqpoint{3.874218in}{2.529063in}}%
\pgfpathlineto{\pgfqpoint{3.874218in}{2.529063in}}%
\pgfpathlineto{\pgfqpoint{3.874218in}{2.532012in}}%
\pgfpathlineto{\pgfqpoint{3.878758in}{2.532012in}}%
\pgfpathlineto{\pgfqpoint{3.878758in}{2.529063in}}%
\pgfpathmoveto{\pgfqpoint{3.878758in}{2.529063in}}%
\pgfpathlineto{\pgfqpoint{3.878758in}{2.529063in}}%
\pgfpathlineto{\pgfqpoint{3.878758in}{2.532012in}}%
\pgfpathlineto{\pgfqpoint{3.883299in}{2.532012in}}%
\pgfpathlineto{\pgfqpoint{3.883299in}{2.529063in}}%
\pgfpathmoveto{\pgfqpoint{3.878758in}{2.532012in}}%
\pgfpathlineto{\pgfqpoint{3.878758in}{2.532012in}}%
\pgfpathlineto{\pgfqpoint{3.878758in}{2.534962in}}%
\pgfpathlineto{\pgfqpoint{3.883299in}{2.534962in}}%
\pgfpathlineto{\pgfqpoint{3.883299in}{2.532012in}}%
\pgfpathmoveto{\pgfqpoint{3.883299in}{2.532012in}}%
\pgfpathlineto{\pgfqpoint{3.883299in}{2.532012in}}%
\pgfpathlineto{\pgfqpoint{3.883299in}{2.534962in}}%
\pgfpathlineto{\pgfqpoint{3.887840in}{2.534962in}}%
\pgfpathlineto{\pgfqpoint{3.887840in}{2.532012in}}%
\pgfpathmoveto{\pgfqpoint{3.883299in}{2.534962in}}%
\pgfpathlineto{\pgfqpoint{3.883299in}{2.534962in}}%
\pgfpathlineto{\pgfqpoint{3.883299in}{2.537911in}}%
\pgfpathlineto{\pgfqpoint{3.887840in}{2.537911in}}%
\pgfpathlineto{\pgfqpoint{3.887840in}{2.534962in}}%
\pgfpathmoveto{\pgfqpoint{3.887840in}{2.534962in}}%
\pgfpathlineto{\pgfqpoint{3.887840in}{2.534962in}}%
\pgfpathlineto{\pgfqpoint{3.887840in}{2.537911in}}%
\pgfpathlineto{\pgfqpoint{3.892381in}{2.537911in}}%
\pgfpathlineto{\pgfqpoint{3.892381in}{2.534962in}}%
\pgfpathmoveto{\pgfqpoint{3.887840in}{2.537911in}}%
\pgfpathlineto{\pgfqpoint{3.887840in}{2.537911in}}%
\pgfpathlineto{\pgfqpoint{3.887840in}{2.540860in}}%
\pgfpathlineto{\pgfqpoint{3.892381in}{2.540860in}}%
\pgfpathlineto{\pgfqpoint{3.892381in}{2.537911in}}%
\pgfpathmoveto{\pgfqpoint{3.892381in}{2.537911in}}%
\pgfpathlineto{\pgfqpoint{3.892381in}{2.537911in}}%
\pgfpathlineto{\pgfqpoint{3.892381in}{2.540860in}}%
\pgfpathlineto{\pgfqpoint{3.896922in}{2.540860in}}%
\pgfpathlineto{\pgfqpoint{3.896922in}{2.537911in}}%
\pgfpathmoveto{\pgfqpoint{3.892381in}{2.540860in}}%
\pgfpathlineto{\pgfqpoint{3.892381in}{2.540860in}}%
\pgfpathlineto{\pgfqpoint{3.892381in}{2.543810in}}%
\pgfpathlineto{\pgfqpoint{3.896922in}{2.543810in}}%
\pgfpathlineto{\pgfqpoint{3.896922in}{2.540860in}}%
\pgfpathmoveto{\pgfqpoint{3.896922in}{2.540860in}}%
\pgfpathlineto{\pgfqpoint{3.896922in}{2.540860in}}%
\pgfpathlineto{\pgfqpoint{3.896922in}{2.543810in}}%
\pgfpathlineto{\pgfqpoint{3.901463in}{2.543810in}}%
\pgfpathlineto{\pgfqpoint{3.901463in}{2.540860in}}%
\pgfpathmoveto{\pgfqpoint{3.896922in}{2.543810in}}%
\pgfpathlineto{\pgfqpoint{3.896922in}{2.543810in}}%
\pgfpathlineto{\pgfqpoint{3.896922in}{2.546759in}}%
\pgfpathlineto{\pgfqpoint{3.901463in}{2.546759in}}%
\pgfpathlineto{\pgfqpoint{3.901463in}{2.543810in}}%
\pgfpathmoveto{\pgfqpoint{3.901463in}{2.543810in}}%
\pgfpathlineto{\pgfqpoint{3.901463in}{2.543810in}}%
\pgfpathlineto{\pgfqpoint{3.901463in}{2.546759in}}%
\pgfpathlineto{\pgfqpoint{3.906004in}{2.546759in}}%
\pgfpathlineto{\pgfqpoint{3.906004in}{2.543810in}}%
\pgfpathmoveto{\pgfqpoint{3.901463in}{2.546759in}}%
\pgfpathlineto{\pgfqpoint{3.901463in}{2.546759in}}%
\pgfpathlineto{\pgfqpoint{3.901463in}{2.549708in}}%
\pgfpathlineto{\pgfqpoint{3.906004in}{2.549708in}}%
\pgfpathlineto{\pgfqpoint{3.906004in}{2.546759in}}%
\pgfpathmoveto{\pgfqpoint{3.906004in}{2.546759in}}%
\pgfpathlineto{\pgfqpoint{3.906004in}{2.546759in}}%
\pgfpathlineto{\pgfqpoint{3.906004in}{2.549708in}}%
\pgfpathlineto{\pgfqpoint{3.910545in}{2.549708in}}%
\pgfpathlineto{\pgfqpoint{3.910545in}{2.546759in}}%
\pgfpathmoveto{\pgfqpoint{3.906004in}{2.549708in}}%
\pgfpathlineto{\pgfqpoint{3.906004in}{2.549708in}}%
\pgfpathlineto{\pgfqpoint{3.906004in}{2.552658in}}%
\pgfpathlineto{\pgfqpoint{3.910545in}{2.552658in}}%
\pgfpathlineto{\pgfqpoint{3.910545in}{2.549708in}}%
\pgfpathmoveto{\pgfqpoint{3.910545in}{2.549708in}}%
\pgfpathlineto{\pgfqpoint{3.910545in}{2.549708in}}%
\pgfpathlineto{\pgfqpoint{3.910545in}{2.552658in}}%
\pgfpathlineto{\pgfqpoint{3.915085in}{2.552658in}}%
\pgfpathlineto{\pgfqpoint{3.915085in}{2.549708in}}%
\pgfpathmoveto{\pgfqpoint{3.910545in}{2.552658in}}%
\pgfpathlineto{\pgfqpoint{3.910545in}{2.552658in}}%
\pgfpathlineto{\pgfqpoint{3.910545in}{2.555607in}}%
\pgfpathlineto{\pgfqpoint{3.915085in}{2.555607in}}%
\pgfpathlineto{\pgfqpoint{3.915085in}{2.552658in}}%
\pgfpathmoveto{\pgfqpoint{3.915085in}{2.552658in}}%
\pgfpathlineto{\pgfqpoint{3.915085in}{2.552658in}}%
\pgfpathlineto{\pgfqpoint{3.915085in}{2.555607in}}%
\pgfpathlineto{\pgfqpoint{3.919626in}{2.555607in}}%
\pgfpathlineto{\pgfqpoint{3.919626in}{2.552658in}}%
\pgfpathmoveto{\pgfqpoint{3.915085in}{2.555607in}}%
\pgfpathlineto{\pgfqpoint{3.915085in}{2.555607in}}%
\pgfpathlineto{\pgfqpoint{3.915085in}{2.558556in}}%
\pgfpathlineto{\pgfqpoint{3.919626in}{2.558556in}}%
\pgfpathlineto{\pgfqpoint{3.919626in}{2.555607in}}%
\pgfpathmoveto{\pgfqpoint{3.919626in}{2.555607in}}%
\pgfpathlineto{\pgfqpoint{3.919626in}{2.555607in}}%
\pgfpathlineto{\pgfqpoint{3.919626in}{2.558556in}}%
\pgfpathlineto{\pgfqpoint{3.924167in}{2.558556in}}%
\pgfpathlineto{\pgfqpoint{3.924167in}{2.555607in}}%
\pgfpathmoveto{\pgfqpoint{3.919626in}{2.558556in}}%
\pgfpathlineto{\pgfqpoint{3.919626in}{2.558556in}}%
\pgfpathlineto{\pgfqpoint{3.919626in}{2.561505in}}%
\pgfpathlineto{\pgfqpoint{3.924167in}{2.561505in}}%
\pgfpathlineto{\pgfqpoint{3.924167in}{2.558556in}}%
\pgfpathmoveto{\pgfqpoint{3.924167in}{2.558556in}}%
\pgfpathlineto{\pgfqpoint{3.924167in}{2.558556in}}%
\pgfpathlineto{\pgfqpoint{3.924167in}{2.561505in}}%
\pgfpathlineto{\pgfqpoint{3.928708in}{2.561505in}}%
\pgfpathlineto{\pgfqpoint{3.928708in}{2.558556in}}%
\pgfpathmoveto{\pgfqpoint{3.924167in}{2.561505in}}%
\pgfpathlineto{\pgfqpoint{3.924167in}{2.561505in}}%
\pgfpathlineto{\pgfqpoint{3.924167in}{2.564455in}}%
\pgfpathlineto{\pgfqpoint{3.928708in}{2.564455in}}%
\pgfpathlineto{\pgfqpoint{3.928708in}{2.561505in}}%
\pgfpathmoveto{\pgfqpoint{3.928708in}{2.561505in}}%
\pgfpathlineto{\pgfqpoint{3.928708in}{2.561505in}}%
\pgfpathlineto{\pgfqpoint{3.928708in}{2.564455in}}%
\pgfpathlineto{\pgfqpoint{3.933249in}{2.564455in}}%
\pgfpathlineto{\pgfqpoint{3.933249in}{2.561505in}}%
\pgfpathmoveto{\pgfqpoint{3.928708in}{2.564455in}}%
\pgfpathlineto{\pgfqpoint{3.928708in}{2.564455in}}%
\pgfpathlineto{\pgfqpoint{3.928708in}{2.567404in}}%
\pgfpathlineto{\pgfqpoint{3.933249in}{2.567404in}}%
\pgfpathlineto{\pgfqpoint{3.933249in}{2.564455in}}%
\pgfpathmoveto{\pgfqpoint{3.933249in}{2.564455in}}%
\pgfpathlineto{\pgfqpoint{3.933249in}{2.564455in}}%
\pgfpathlineto{\pgfqpoint{3.933249in}{2.567404in}}%
\pgfpathlineto{\pgfqpoint{3.937790in}{2.567404in}}%
\pgfpathlineto{\pgfqpoint{3.937790in}{2.564455in}}%
\pgfpathmoveto{\pgfqpoint{3.933249in}{2.567404in}}%
\pgfpathlineto{\pgfqpoint{3.933249in}{2.567404in}}%
\pgfpathlineto{\pgfqpoint{3.933249in}{2.570353in}}%
\pgfpathlineto{\pgfqpoint{3.937790in}{2.570353in}}%
\pgfpathlineto{\pgfqpoint{3.937790in}{2.567404in}}%
\pgfpathmoveto{\pgfqpoint{3.937790in}{2.567404in}}%
\pgfpathlineto{\pgfqpoint{3.937790in}{2.567404in}}%
\pgfpathlineto{\pgfqpoint{3.937790in}{2.570353in}}%
\pgfpathlineto{\pgfqpoint{3.942331in}{2.570353in}}%
\pgfpathlineto{\pgfqpoint{3.942331in}{2.567404in}}%
\pgfpathmoveto{\pgfqpoint{3.937790in}{2.570353in}}%
\pgfpathlineto{\pgfqpoint{3.937790in}{2.570353in}}%
\pgfpathlineto{\pgfqpoint{3.937790in}{2.573303in}}%
\pgfpathlineto{\pgfqpoint{3.942331in}{2.573303in}}%
\pgfpathlineto{\pgfqpoint{3.942331in}{2.570353in}}%
\pgfpathmoveto{\pgfqpoint{3.942331in}{2.570353in}}%
\pgfpathlineto{\pgfqpoint{3.942331in}{2.570353in}}%
\pgfpathlineto{\pgfqpoint{3.942331in}{2.573303in}}%
\pgfpathlineto{\pgfqpoint{3.946871in}{2.573303in}}%
\pgfpathlineto{\pgfqpoint{3.946871in}{2.570353in}}%
\pgfpathmoveto{\pgfqpoint{3.942331in}{2.573303in}}%
\pgfpathlineto{\pgfqpoint{3.942331in}{2.573303in}}%
\pgfpathlineto{\pgfqpoint{3.942331in}{2.576252in}}%
\pgfpathlineto{\pgfqpoint{3.946871in}{2.576252in}}%
\pgfpathlineto{\pgfqpoint{3.946871in}{2.573303in}}%
\pgfpathmoveto{\pgfqpoint{3.946871in}{2.573303in}}%
\pgfpathlineto{\pgfqpoint{3.946871in}{2.573303in}}%
\pgfpathlineto{\pgfqpoint{3.946871in}{2.576252in}}%
\pgfpathlineto{\pgfqpoint{3.951412in}{2.576252in}}%
\pgfpathlineto{\pgfqpoint{3.951412in}{2.573303in}}%
\pgfpathmoveto{\pgfqpoint{3.946871in}{2.576252in}}%
\pgfpathlineto{\pgfqpoint{3.946871in}{2.576252in}}%
\pgfpathlineto{\pgfqpoint{3.946871in}{2.579201in}}%
\pgfpathlineto{\pgfqpoint{3.951412in}{2.579201in}}%
\pgfpathlineto{\pgfqpoint{3.951412in}{2.576252in}}%
\pgfpathmoveto{\pgfqpoint{3.951412in}{2.576252in}}%
\pgfpathlineto{\pgfqpoint{3.951412in}{2.576252in}}%
\pgfpathlineto{\pgfqpoint{3.951412in}{2.579201in}}%
\pgfpathlineto{\pgfqpoint{3.955954in}{2.579201in}}%
\pgfpathlineto{\pgfqpoint{3.955954in}{2.576252in}}%
\pgfpathmoveto{\pgfqpoint{3.951412in}{2.579201in}}%
\pgfpathlineto{\pgfqpoint{3.951412in}{2.579201in}}%
\pgfpathlineto{\pgfqpoint{3.951412in}{2.582150in}}%
\pgfpathlineto{\pgfqpoint{3.955954in}{2.582150in}}%
\pgfpathlineto{\pgfqpoint{3.955954in}{2.579201in}}%
\pgfpathmoveto{\pgfqpoint{3.955954in}{2.579201in}}%
\pgfpathlineto{\pgfqpoint{3.955954in}{2.579201in}}%
\pgfpathlineto{\pgfqpoint{3.955954in}{2.582150in}}%
\pgfpathlineto{\pgfqpoint{3.960495in}{2.582150in}}%
\pgfpathlineto{\pgfqpoint{3.960495in}{2.579201in}}%
\pgfpathmoveto{\pgfqpoint{3.955954in}{2.582150in}}%
\pgfpathlineto{\pgfqpoint{3.955954in}{2.582150in}}%
\pgfpathlineto{\pgfqpoint{3.955954in}{2.585100in}}%
\pgfpathlineto{\pgfqpoint{3.960495in}{2.585100in}}%
\pgfpathlineto{\pgfqpoint{3.960495in}{2.582150in}}%
\pgfpathmoveto{\pgfqpoint{3.960495in}{2.582150in}}%
\pgfpathlineto{\pgfqpoint{3.960495in}{2.582150in}}%
\pgfpathlineto{\pgfqpoint{3.960495in}{2.585100in}}%
\pgfpathlineto{\pgfqpoint{3.965036in}{2.585100in}}%
\pgfpathlineto{\pgfqpoint{3.965036in}{2.582150in}}%
\pgfpathmoveto{\pgfqpoint{3.960495in}{2.585100in}}%
\pgfpathlineto{\pgfqpoint{3.960495in}{2.585100in}}%
\pgfpathlineto{\pgfqpoint{3.960495in}{2.588049in}}%
\pgfpathlineto{\pgfqpoint{3.965036in}{2.588049in}}%
\pgfpathlineto{\pgfqpoint{3.965036in}{2.585100in}}%
\pgfpathmoveto{\pgfqpoint{3.965036in}{2.585100in}}%
\pgfpathlineto{\pgfqpoint{3.965036in}{2.585100in}}%
\pgfpathlineto{\pgfqpoint{3.965036in}{2.588049in}}%
\pgfpathlineto{\pgfqpoint{3.969577in}{2.588049in}}%
\pgfpathlineto{\pgfqpoint{3.969577in}{2.585100in}}%
\pgfpathmoveto{\pgfqpoint{3.965036in}{2.588049in}}%
\pgfpathlineto{\pgfqpoint{3.965036in}{2.588049in}}%
\pgfpathlineto{\pgfqpoint{3.965036in}{2.590998in}}%
\pgfpathlineto{\pgfqpoint{3.969577in}{2.590998in}}%
\pgfpathlineto{\pgfqpoint{3.969577in}{2.588049in}}%
\pgfpathmoveto{\pgfqpoint{3.969577in}{2.588049in}}%
\pgfpathlineto{\pgfqpoint{3.969577in}{2.588049in}}%
\pgfpathlineto{\pgfqpoint{3.969577in}{2.590998in}}%
\pgfpathlineto{\pgfqpoint{3.974118in}{2.590998in}}%
\pgfpathlineto{\pgfqpoint{3.974118in}{2.588049in}}%
\pgfpathmoveto{\pgfqpoint{3.969577in}{2.590998in}}%
\pgfpathlineto{\pgfqpoint{3.969577in}{2.590998in}}%
\pgfpathlineto{\pgfqpoint{3.969577in}{2.593947in}}%
\pgfpathlineto{\pgfqpoint{3.974118in}{2.593947in}}%
\pgfpathlineto{\pgfqpoint{3.974118in}{2.590998in}}%
\pgfpathmoveto{\pgfqpoint{3.974118in}{2.590998in}}%
\pgfpathlineto{\pgfqpoint{3.974118in}{2.590998in}}%
\pgfpathlineto{\pgfqpoint{3.974118in}{2.593947in}}%
\pgfpathlineto{\pgfqpoint{3.978659in}{2.593947in}}%
\pgfpathlineto{\pgfqpoint{3.978659in}{2.590998in}}%
\pgfpathmoveto{\pgfqpoint{3.974118in}{2.593947in}}%
\pgfpathlineto{\pgfqpoint{3.974118in}{2.593947in}}%
\pgfpathlineto{\pgfqpoint{3.974118in}{2.596897in}}%
\pgfpathlineto{\pgfqpoint{3.978659in}{2.596897in}}%
\pgfpathlineto{\pgfqpoint{3.978659in}{2.593947in}}%
\pgfpathmoveto{\pgfqpoint{3.978659in}{2.593947in}}%
\pgfpathlineto{\pgfqpoint{3.978659in}{2.593947in}}%
\pgfpathlineto{\pgfqpoint{3.978659in}{2.596897in}}%
\pgfpathlineto{\pgfqpoint{3.983200in}{2.596897in}}%
\pgfpathlineto{\pgfqpoint{3.983200in}{2.593947in}}%
\pgfpathmoveto{\pgfqpoint{3.978659in}{2.596897in}}%
\pgfpathlineto{\pgfqpoint{3.978659in}{2.596897in}}%
\pgfpathlineto{\pgfqpoint{3.978659in}{2.599846in}}%
\pgfpathlineto{\pgfqpoint{3.983200in}{2.599846in}}%
\pgfpathlineto{\pgfqpoint{3.983200in}{2.596897in}}%
\pgfpathmoveto{\pgfqpoint{3.983200in}{2.596897in}}%
\pgfpathlineto{\pgfqpoint{3.983200in}{2.596897in}}%
\pgfpathlineto{\pgfqpoint{3.983200in}{2.599846in}}%
\pgfpathlineto{\pgfqpoint{3.987741in}{2.599846in}}%
\pgfpathlineto{\pgfqpoint{3.987741in}{2.596897in}}%
\pgfpathmoveto{\pgfqpoint{3.983200in}{2.599846in}}%
\pgfpathlineto{\pgfqpoint{3.983200in}{2.599846in}}%
\pgfpathlineto{\pgfqpoint{3.983200in}{2.602795in}}%
\pgfpathlineto{\pgfqpoint{3.987741in}{2.602795in}}%
\pgfpathlineto{\pgfqpoint{3.987741in}{2.599846in}}%
\pgfpathmoveto{\pgfqpoint{3.987741in}{2.599846in}}%
\pgfpathlineto{\pgfqpoint{3.987741in}{2.599846in}}%
\pgfpathlineto{\pgfqpoint{3.987741in}{2.602795in}}%
\pgfpathlineto{\pgfqpoint{3.992282in}{2.602795in}}%
\pgfpathlineto{\pgfqpoint{3.992282in}{2.599846in}}%
\pgfpathmoveto{\pgfqpoint{3.987741in}{2.602795in}}%
\pgfpathlineto{\pgfqpoint{3.987741in}{2.602795in}}%
\pgfpathlineto{\pgfqpoint{3.987741in}{2.605744in}}%
\pgfpathlineto{\pgfqpoint{3.992282in}{2.605744in}}%
\pgfpathlineto{\pgfqpoint{3.992282in}{2.602795in}}%
\pgfpathmoveto{\pgfqpoint{3.992282in}{2.602795in}}%
\pgfpathlineto{\pgfqpoint{3.992282in}{2.602795in}}%
\pgfpathlineto{\pgfqpoint{3.992282in}{2.605744in}}%
\pgfpathlineto{\pgfqpoint{3.996823in}{2.605744in}}%
\pgfpathlineto{\pgfqpoint{3.996823in}{2.602795in}}%
\pgfpathmoveto{\pgfqpoint{3.992282in}{2.605744in}}%
\pgfpathlineto{\pgfqpoint{3.992282in}{2.605744in}}%
\pgfpathlineto{\pgfqpoint{3.992282in}{2.608694in}}%
\pgfpathlineto{\pgfqpoint{3.996823in}{2.608694in}}%
\pgfpathlineto{\pgfqpoint{3.996823in}{2.605744in}}%
\pgfpathmoveto{\pgfqpoint{3.996823in}{2.605744in}}%
\pgfpathlineto{\pgfqpoint{3.996823in}{2.605744in}}%
\pgfpathlineto{\pgfqpoint{3.996823in}{2.608694in}}%
\pgfpathlineto{\pgfqpoint{4.001364in}{2.608694in}}%
\pgfpathlineto{\pgfqpoint{4.001364in}{2.605744in}}%
\pgfpathmoveto{\pgfqpoint{3.996823in}{2.608694in}}%
\pgfpathlineto{\pgfqpoint{3.996823in}{2.608694in}}%
\pgfpathlineto{\pgfqpoint{3.996823in}{2.611643in}}%
\pgfpathlineto{\pgfqpoint{4.001364in}{2.611643in}}%
\pgfpathlineto{\pgfqpoint{4.001364in}{2.608694in}}%
\pgfpathmoveto{\pgfqpoint{4.001364in}{2.608694in}}%
\pgfpathlineto{\pgfqpoint{4.001364in}{2.608694in}}%
\pgfpathlineto{\pgfqpoint{4.001364in}{2.611643in}}%
\pgfpathlineto{\pgfqpoint{4.005905in}{2.611643in}}%
\pgfpathlineto{\pgfqpoint{4.005905in}{2.608694in}}%
\pgfpathmoveto{\pgfqpoint{4.001364in}{2.611643in}}%
\pgfpathlineto{\pgfqpoint{4.001364in}{2.611643in}}%
\pgfpathlineto{\pgfqpoint{4.001364in}{2.614592in}}%
\pgfpathlineto{\pgfqpoint{4.005905in}{2.614592in}}%
\pgfpathlineto{\pgfqpoint{4.005905in}{2.611643in}}%
\pgfpathmoveto{\pgfqpoint{4.005905in}{2.611643in}}%
\pgfpathlineto{\pgfqpoint{4.005905in}{2.611643in}}%
\pgfpathlineto{\pgfqpoint{4.005905in}{2.614592in}}%
\pgfpathlineto{\pgfqpoint{4.010446in}{2.614592in}}%
\pgfpathlineto{\pgfqpoint{4.010446in}{2.611643in}}%
\pgfpathmoveto{\pgfqpoint{4.005905in}{2.614592in}}%
\pgfpathlineto{\pgfqpoint{4.005905in}{2.614592in}}%
\pgfpathlineto{\pgfqpoint{4.005905in}{2.617541in}}%
\pgfpathlineto{\pgfqpoint{4.010446in}{2.617541in}}%
\pgfpathlineto{\pgfqpoint{4.010446in}{2.614592in}}%
\pgfpathmoveto{\pgfqpoint{4.010446in}{2.614592in}}%
\pgfpathlineto{\pgfqpoint{4.010446in}{2.614592in}}%
\pgfpathlineto{\pgfqpoint{4.010446in}{2.617541in}}%
\pgfpathlineto{\pgfqpoint{4.014987in}{2.617541in}}%
\pgfpathlineto{\pgfqpoint{4.014987in}{2.614592in}}%
\pgfpathmoveto{\pgfqpoint{4.010446in}{2.617541in}}%
\pgfpathlineto{\pgfqpoint{4.010446in}{2.617541in}}%
\pgfpathlineto{\pgfqpoint{4.010446in}{2.620491in}}%
\pgfpathlineto{\pgfqpoint{4.014987in}{2.620491in}}%
\pgfpathlineto{\pgfqpoint{4.014987in}{2.617541in}}%
\pgfpathmoveto{\pgfqpoint{4.014987in}{2.617541in}}%
\pgfpathlineto{\pgfqpoint{4.014987in}{2.617541in}}%
\pgfpathlineto{\pgfqpoint{4.014987in}{2.620491in}}%
\pgfpathlineto{\pgfqpoint{4.019528in}{2.620491in}}%
\pgfpathlineto{\pgfqpoint{4.019528in}{2.617541in}}%
\pgfpathmoveto{\pgfqpoint{4.014987in}{2.620491in}}%
\pgfpathlineto{\pgfqpoint{4.014987in}{2.620491in}}%
\pgfpathlineto{\pgfqpoint{4.014987in}{2.623440in}}%
\pgfpathlineto{\pgfqpoint{4.019528in}{2.623440in}}%
\pgfpathlineto{\pgfqpoint{4.019528in}{2.620491in}}%
\pgfpathmoveto{\pgfqpoint{4.019528in}{2.620491in}}%
\pgfpathlineto{\pgfqpoint{4.019528in}{2.620491in}}%
\pgfpathlineto{\pgfqpoint{4.019528in}{2.623440in}}%
\pgfpathlineto{\pgfqpoint{4.024069in}{2.623440in}}%
\pgfpathlineto{\pgfqpoint{4.024069in}{2.620491in}}%
\pgfpathmoveto{\pgfqpoint{4.019528in}{2.623440in}}%
\pgfpathlineto{\pgfqpoint{4.019528in}{2.623440in}}%
\pgfpathlineto{\pgfqpoint{4.019528in}{2.626389in}}%
\pgfpathlineto{\pgfqpoint{4.024069in}{2.626389in}}%
\pgfpathlineto{\pgfqpoint{4.024069in}{2.623440in}}%
\pgfpathmoveto{\pgfqpoint{4.024069in}{2.623440in}}%
\pgfpathlineto{\pgfqpoint{4.024069in}{2.623440in}}%
\pgfpathlineto{\pgfqpoint{4.024069in}{2.626389in}}%
\pgfpathlineto{\pgfqpoint{4.028610in}{2.626389in}}%
\pgfpathlineto{\pgfqpoint{4.028610in}{2.623440in}}%
\pgfpathmoveto{\pgfqpoint{4.024069in}{2.626389in}}%
\pgfpathlineto{\pgfqpoint{4.024069in}{2.626389in}}%
\pgfpathlineto{\pgfqpoint{4.024069in}{2.629338in}}%
\pgfpathlineto{\pgfqpoint{4.028610in}{2.629338in}}%
\pgfpathlineto{\pgfqpoint{4.028610in}{2.626389in}}%
\pgfpathmoveto{\pgfqpoint{4.028610in}{2.626389in}}%
\pgfpathlineto{\pgfqpoint{4.028610in}{2.626389in}}%
\pgfpathlineto{\pgfqpoint{4.028610in}{2.629338in}}%
\pgfpathlineto{\pgfqpoint{4.033151in}{2.629338in}}%
\pgfpathlineto{\pgfqpoint{4.033151in}{2.626389in}}%
\pgfpathmoveto{\pgfqpoint{4.028610in}{2.629338in}}%
\pgfpathlineto{\pgfqpoint{4.028610in}{2.629338in}}%
\pgfpathlineto{\pgfqpoint{4.028610in}{2.632288in}}%
\pgfpathlineto{\pgfqpoint{4.033151in}{2.632288in}}%
\pgfpathlineto{\pgfqpoint{4.033151in}{2.629338in}}%
\pgfpathmoveto{\pgfqpoint{4.033151in}{2.629338in}}%
\pgfpathlineto{\pgfqpoint{4.033151in}{2.629338in}}%
\pgfpathlineto{\pgfqpoint{4.033151in}{2.632288in}}%
\pgfpathlineto{\pgfqpoint{4.037692in}{2.632288in}}%
\pgfpathlineto{\pgfqpoint{4.037692in}{2.629338in}}%
\pgfpathmoveto{\pgfqpoint{4.033151in}{2.632288in}}%
\pgfpathlineto{\pgfqpoint{4.033151in}{2.632288in}}%
\pgfpathlineto{\pgfqpoint{4.033151in}{2.635237in}}%
\pgfpathlineto{\pgfqpoint{4.037692in}{2.635237in}}%
\pgfpathlineto{\pgfqpoint{4.037692in}{2.632288in}}%
\pgfpathmoveto{\pgfqpoint{4.037692in}{2.632288in}}%
\pgfpathlineto{\pgfqpoint{4.037692in}{2.632288in}}%
\pgfpathlineto{\pgfqpoint{4.037692in}{2.635237in}}%
\pgfpathlineto{\pgfqpoint{4.042233in}{2.635237in}}%
\pgfpathlineto{\pgfqpoint{4.042233in}{2.632288in}}%
\pgfpathmoveto{\pgfqpoint{4.037692in}{2.635237in}}%
\pgfpathlineto{\pgfqpoint{4.037692in}{2.635237in}}%
\pgfpathlineto{\pgfqpoint{4.037692in}{2.638186in}}%
\pgfpathlineto{\pgfqpoint{4.042233in}{2.638186in}}%
\pgfpathlineto{\pgfqpoint{4.042233in}{2.635237in}}%
\pgfpathmoveto{\pgfqpoint{4.042233in}{2.635237in}}%
\pgfpathlineto{\pgfqpoint{4.042233in}{2.635237in}}%
\pgfpathlineto{\pgfqpoint{4.042233in}{2.638186in}}%
\pgfpathlineto{\pgfqpoint{4.046774in}{2.638186in}}%
\pgfpathlineto{\pgfqpoint{4.046774in}{2.635237in}}%
\pgfpathmoveto{\pgfqpoint{4.042233in}{2.638186in}}%
\pgfpathlineto{\pgfqpoint{4.042233in}{2.638186in}}%
\pgfpathlineto{\pgfqpoint{4.042233in}{2.641135in}}%
\pgfpathlineto{\pgfqpoint{4.046774in}{2.641135in}}%
\pgfpathlineto{\pgfqpoint{4.046774in}{2.638186in}}%
\pgfpathmoveto{\pgfqpoint{4.046774in}{2.638186in}}%
\pgfpathlineto{\pgfqpoint{4.046774in}{2.638186in}}%
\pgfpathlineto{\pgfqpoint{4.046774in}{2.641135in}}%
\pgfpathlineto{\pgfqpoint{4.051315in}{2.641135in}}%
\pgfpathlineto{\pgfqpoint{4.051315in}{2.638186in}}%
\pgfpathmoveto{\pgfqpoint{4.046774in}{2.641135in}}%
\pgfpathlineto{\pgfqpoint{4.046774in}{2.641135in}}%
\pgfpathlineto{\pgfqpoint{4.046774in}{2.644085in}}%
\pgfpathlineto{\pgfqpoint{4.051315in}{2.644085in}}%
\pgfpathlineto{\pgfqpoint{4.051315in}{2.641135in}}%
\pgfpathmoveto{\pgfqpoint{4.051315in}{2.641135in}}%
\pgfpathlineto{\pgfqpoint{4.051315in}{2.641135in}}%
\pgfpathlineto{\pgfqpoint{4.051315in}{2.644085in}}%
\pgfpathlineto{\pgfqpoint{4.055856in}{2.644085in}}%
\pgfpathlineto{\pgfqpoint{4.055856in}{2.641135in}}%
\pgfpathmoveto{\pgfqpoint{4.051315in}{2.644085in}}%
\pgfpathlineto{\pgfqpoint{4.051315in}{2.644085in}}%
\pgfpathlineto{\pgfqpoint{4.051315in}{2.647034in}}%
\pgfpathlineto{\pgfqpoint{4.055856in}{2.647034in}}%
\pgfpathlineto{\pgfqpoint{4.055856in}{2.644085in}}%
\pgfpathmoveto{\pgfqpoint{4.055856in}{2.644085in}}%
\pgfpathlineto{\pgfqpoint{4.055856in}{2.644085in}}%
\pgfpathlineto{\pgfqpoint{4.055856in}{2.647034in}}%
\pgfpathlineto{\pgfqpoint{4.060397in}{2.647034in}}%
\pgfpathlineto{\pgfqpoint{4.060397in}{2.644085in}}%
\pgfpathmoveto{\pgfqpoint{4.055856in}{2.647034in}}%
\pgfpathlineto{\pgfqpoint{4.055856in}{2.647034in}}%
\pgfpathlineto{\pgfqpoint{4.055856in}{2.649983in}}%
\pgfpathlineto{\pgfqpoint{4.060397in}{2.649983in}}%
\pgfpathlineto{\pgfqpoint{4.060397in}{2.647034in}}%
\pgfpathmoveto{\pgfqpoint{4.060397in}{2.647034in}}%
\pgfpathlineto{\pgfqpoint{4.060397in}{2.647034in}}%
\pgfpathlineto{\pgfqpoint{4.060397in}{2.649983in}}%
\pgfpathlineto{\pgfqpoint{4.064938in}{2.649983in}}%
\pgfpathlineto{\pgfqpoint{4.064938in}{2.647034in}}%
\pgfpathmoveto{\pgfqpoint{4.060397in}{2.649983in}}%
\pgfpathlineto{\pgfqpoint{4.060397in}{2.649983in}}%
\pgfpathlineto{\pgfqpoint{4.060397in}{2.652932in}}%
\pgfpathlineto{\pgfqpoint{4.064938in}{2.652932in}}%
\pgfpathlineto{\pgfqpoint{4.064938in}{2.649983in}}%
\pgfpathmoveto{\pgfqpoint{4.064938in}{2.649983in}}%
\pgfpathlineto{\pgfqpoint{4.064938in}{2.649983in}}%
\pgfpathlineto{\pgfqpoint{4.064938in}{2.652932in}}%
\pgfpathlineto{\pgfqpoint{4.069479in}{2.652932in}}%
\pgfpathlineto{\pgfqpoint{4.069479in}{2.649983in}}%
\pgfpathmoveto{\pgfqpoint{4.064938in}{2.652932in}}%
\pgfpathlineto{\pgfqpoint{4.064938in}{2.652932in}}%
\pgfpathlineto{\pgfqpoint{4.064938in}{2.655882in}}%
\pgfpathlineto{\pgfqpoint{4.069479in}{2.655882in}}%
\pgfpathlineto{\pgfqpoint{4.069479in}{2.652932in}}%
\pgfpathmoveto{\pgfqpoint{4.069479in}{2.652932in}}%
\pgfpathlineto{\pgfqpoint{4.069479in}{2.652932in}}%
\pgfpathlineto{\pgfqpoint{4.069479in}{2.655882in}}%
\pgfpathlineto{\pgfqpoint{4.074020in}{2.655882in}}%
\pgfpathlineto{\pgfqpoint{4.074020in}{2.652932in}}%
\pgfpathmoveto{\pgfqpoint{4.069479in}{2.655882in}}%
\pgfpathlineto{\pgfqpoint{4.069479in}{2.655882in}}%
\pgfpathlineto{\pgfqpoint{4.069479in}{2.658831in}}%
\pgfpathlineto{\pgfqpoint{4.074020in}{2.658831in}}%
\pgfpathlineto{\pgfqpoint{4.074020in}{2.655882in}}%
\pgfpathmoveto{\pgfqpoint{4.074020in}{2.655882in}}%
\pgfpathlineto{\pgfqpoint{4.074020in}{2.655882in}}%
\pgfpathlineto{\pgfqpoint{4.074020in}{2.658831in}}%
\pgfpathlineto{\pgfqpoint{4.078561in}{2.658831in}}%
\pgfpathlineto{\pgfqpoint{4.078561in}{2.655882in}}%
\pgfpathmoveto{\pgfqpoint{4.074020in}{2.658831in}}%
\pgfpathlineto{\pgfqpoint{4.074020in}{2.658831in}}%
\pgfpathlineto{\pgfqpoint{4.074020in}{2.661780in}}%
\pgfpathlineto{\pgfqpoint{4.078561in}{2.661780in}}%
\pgfpathlineto{\pgfqpoint{4.078561in}{2.658831in}}%
\pgfpathmoveto{\pgfqpoint{4.078561in}{2.658831in}}%
\pgfpathlineto{\pgfqpoint{4.078561in}{2.658831in}}%
\pgfpathlineto{\pgfqpoint{4.078561in}{2.661780in}}%
\pgfpathlineto{\pgfqpoint{4.083102in}{2.661780in}}%
\pgfpathlineto{\pgfqpoint{4.083102in}{2.658831in}}%
\pgfpathmoveto{\pgfqpoint{4.078561in}{2.661780in}}%
\pgfpathlineto{\pgfqpoint{4.078561in}{2.661780in}}%
\pgfpathlineto{\pgfqpoint{4.078561in}{2.664729in}}%
\pgfpathlineto{\pgfqpoint{4.083102in}{2.664729in}}%
\pgfpathlineto{\pgfqpoint{4.083102in}{2.661780in}}%
\pgfpathmoveto{\pgfqpoint{4.083102in}{2.661780in}}%
\pgfpathlineto{\pgfqpoint{4.083102in}{2.661780in}}%
\pgfpathlineto{\pgfqpoint{4.083102in}{2.664729in}}%
\pgfpathlineto{\pgfqpoint{4.087644in}{2.664729in}}%
\pgfpathlineto{\pgfqpoint{4.087644in}{2.661780in}}%
\pgfpathmoveto{\pgfqpoint{4.083102in}{2.664729in}}%
\pgfpathlineto{\pgfqpoint{4.083102in}{2.664729in}}%
\pgfpathlineto{\pgfqpoint{4.083102in}{2.667678in}}%
\pgfpathlineto{\pgfqpoint{4.087644in}{2.667678in}}%
\pgfpathlineto{\pgfqpoint{4.087644in}{2.664729in}}%
\pgfpathmoveto{\pgfqpoint{4.087644in}{2.664729in}}%
\pgfpathlineto{\pgfqpoint{4.087644in}{2.664729in}}%
\pgfpathlineto{\pgfqpoint{4.087644in}{2.667678in}}%
\pgfpathlineto{\pgfqpoint{4.092185in}{2.667678in}}%
\pgfpathlineto{\pgfqpoint{4.092185in}{2.664729in}}%
\pgfpathmoveto{\pgfqpoint{4.087644in}{2.667678in}}%
\pgfpathlineto{\pgfqpoint{4.087644in}{2.667678in}}%
\pgfpathlineto{\pgfqpoint{4.087644in}{2.670628in}}%
\pgfpathlineto{\pgfqpoint{4.092185in}{2.670628in}}%
\pgfpathlineto{\pgfqpoint{4.092185in}{2.667678in}}%
\pgfpathmoveto{\pgfqpoint{4.092185in}{2.667678in}}%
\pgfpathlineto{\pgfqpoint{4.092185in}{2.667678in}}%
\pgfpathlineto{\pgfqpoint{4.092185in}{2.670628in}}%
\pgfpathlineto{\pgfqpoint{4.096726in}{2.670628in}}%
\pgfpathlineto{\pgfqpoint{4.096726in}{2.667678in}}%
\pgfpathmoveto{\pgfqpoint{4.092185in}{2.670628in}}%
\pgfpathlineto{\pgfqpoint{4.092185in}{2.670628in}}%
\pgfpathlineto{\pgfqpoint{4.092185in}{2.673577in}}%
\pgfpathlineto{\pgfqpoint{4.096726in}{2.673577in}}%
\pgfpathlineto{\pgfqpoint{4.096726in}{2.670628in}}%
\pgfpathmoveto{\pgfqpoint{4.096726in}{2.670628in}}%
\pgfpathlineto{\pgfqpoint{4.096726in}{2.670628in}}%
\pgfpathlineto{\pgfqpoint{4.096726in}{2.673577in}}%
\pgfpathlineto{\pgfqpoint{4.101267in}{2.673577in}}%
\pgfpathlineto{\pgfqpoint{4.101267in}{2.670628in}}%
\pgfpathmoveto{\pgfqpoint{4.096726in}{2.673577in}}%
\pgfpathlineto{\pgfqpoint{4.096726in}{2.673577in}}%
\pgfpathlineto{\pgfqpoint{4.096726in}{2.676526in}}%
\pgfpathlineto{\pgfqpoint{4.101267in}{2.676526in}}%
\pgfpathlineto{\pgfqpoint{4.101267in}{2.673577in}}%
\pgfpathmoveto{\pgfqpoint{4.101267in}{2.673577in}}%
\pgfpathlineto{\pgfqpoint{4.101267in}{2.673577in}}%
\pgfpathlineto{\pgfqpoint{4.101267in}{2.676526in}}%
\pgfpathlineto{\pgfqpoint{4.105808in}{2.676526in}}%
\pgfpathlineto{\pgfqpoint{4.105808in}{2.673577in}}%
\pgfpathmoveto{\pgfqpoint{4.101267in}{2.676526in}}%
\pgfpathlineto{\pgfqpoint{4.101267in}{2.676526in}}%
\pgfpathlineto{\pgfqpoint{4.101267in}{2.679475in}}%
\pgfpathlineto{\pgfqpoint{4.105808in}{2.679475in}}%
\pgfpathlineto{\pgfqpoint{4.105808in}{2.676526in}}%
\pgfpathmoveto{\pgfqpoint{4.105808in}{2.676526in}}%
\pgfpathlineto{\pgfqpoint{4.105808in}{2.676526in}}%
\pgfpathlineto{\pgfqpoint{4.105808in}{2.679475in}}%
\pgfpathlineto{\pgfqpoint{4.110349in}{2.679475in}}%
\pgfpathlineto{\pgfqpoint{4.110349in}{2.676526in}}%
\pgfpathmoveto{\pgfqpoint{4.105808in}{2.679475in}}%
\pgfpathlineto{\pgfqpoint{4.105808in}{2.679475in}}%
\pgfpathlineto{\pgfqpoint{4.105808in}{2.682424in}}%
\pgfpathlineto{\pgfqpoint{4.110349in}{2.682424in}}%
\pgfpathlineto{\pgfqpoint{4.110349in}{2.679475in}}%
\pgfpathmoveto{\pgfqpoint{4.110349in}{2.679475in}}%
\pgfpathlineto{\pgfqpoint{4.110349in}{2.679475in}}%
\pgfpathlineto{\pgfqpoint{4.110349in}{2.682424in}}%
\pgfpathlineto{\pgfqpoint{4.114890in}{2.682424in}}%
\pgfpathlineto{\pgfqpoint{4.114890in}{2.679475in}}%
\pgfpathmoveto{\pgfqpoint{4.110349in}{2.682424in}}%
\pgfpathlineto{\pgfqpoint{4.110349in}{2.682424in}}%
\pgfpathlineto{\pgfqpoint{4.110349in}{2.685374in}}%
\pgfpathlineto{\pgfqpoint{4.114890in}{2.685374in}}%
\pgfpathlineto{\pgfqpoint{4.114890in}{2.682424in}}%
\pgfpathmoveto{\pgfqpoint{4.114890in}{2.682424in}}%
\pgfpathlineto{\pgfqpoint{4.114890in}{2.682424in}}%
\pgfpathlineto{\pgfqpoint{4.114890in}{2.685374in}}%
\pgfpathlineto{\pgfqpoint{4.119431in}{2.685374in}}%
\pgfpathlineto{\pgfqpoint{4.119431in}{2.682424in}}%
\pgfpathmoveto{\pgfqpoint{4.114890in}{2.685374in}}%
\pgfpathlineto{\pgfqpoint{4.114890in}{2.685374in}}%
\pgfpathlineto{\pgfqpoint{4.114890in}{2.688323in}}%
\pgfpathlineto{\pgfqpoint{4.119431in}{2.688323in}}%
\pgfpathlineto{\pgfqpoint{4.119431in}{2.685374in}}%
\pgfpathmoveto{\pgfqpoint{4.119431in}{2.685374in}}%
\pgfpathlineto{\pgfqpoint{4.119431in}{2.685374in}}%
\pgfpathlineto{\pgfqpoint{4.119431in}{2.688323in}}%
\pgfpathlineto{\pgfqpoint{4.123972in}{2.688323in}}%
\pgfpathlineto{\pgfqpoint{4.123972in}{2.685374in}}%
\pgfpathmoveto{\pgfqpoint{4.119431in}{2.688323in}}%
\pgfpathlineto{\pgfqpoint{4.119431in}{2.688323in}}%
\pgfpathlineto{\pgfqpoint{4.119431in}{2.691272in}}%
\pgfpathlineto{\pgfqpoint{4.123972in}{2.691272in}}%
\pgfpathlineto{\pgfqpoint{4.123972in}{2.688323in}}%
\pgfpathmoveto{\pgfqpoint{4.123972in}{2.688323in}}%
\pgfpathlineto{\pgfqpoint{4.123972in}{2.688323in}}%
\pgfpathlineto{\pgfqpoint{4.123972in}{2.691272in}}%
\pgfpathlineto{\pgfqpoint{4.128513in}{2.691272in}}%
\pgfpathlineto{\pgfqpoint{4.128513in}{2.688323in}}%
\pgfpathmoveto{\pgfqpoint{4.123972in}{2.691272in}}%
\pgfpathlineto{\pgfqpoint{4.123972in}{2.691272in}}%
\pgfpathlineto{\pgfqpoint{4.123972in}{2.694221in}}%
\pgfpathlineto{\pgfqpoint{4.128513in}{2.694221in}}%
\pgfpathlineto{\pgfqpoint{4.128513in}{2.691272in}}%
\pgfpathmoveto{\pgfqpoint{4.128513in}{2.691272in}}%
\pgfpathlineto{\pgfqpoint{4.128513in}{2.691272in}}%
\pgfpathlineto{\pgfqpoint{4.128513in}{2.694221in}}%
\pgfpathlineto{\pgfqpoint{4.133054in}{2.694221in}}%
\pgfpathlineto{\pgfqpoint{4.133054in}{2.691272in}}%
\pgfpathmoveto{\pgfqpoint{4.128513in}{2.694221in}}%
\pgfpathlineto{\pgfqpoint{4.128513in}{2.694221in}}%
\pgfpathlineto{\pgfqpoint{4.128513in}{2.697170in}}%
\pgfpathlineto{\pgfqpoint{4.133054in}{2.697170in}}%
\pgfpathlineto{\pgfqpoint{4.133054in}{2.694221in}}%
\pgfpathmoveto{\pgfqpoint{4.133054in}{2.694221in}}%
\pgfpathlineto{\pgfqpoint{4.133054in}{2.694221in}}%
\pgfpathlineto{\pgfqpoint{4.133054in}{2.697170in}}%
\pgfpathlineto{\pgfqpoint{4.137595in}{2.697170in}}%
\pgfpathlineto{\pgfqpoint{4.137595in}{2.694221in}}%
\pgfpathmoveto{\pgfqpoint{4.133054in}{2.697170in}}%
\pgfpathlineto{\pgfqpoint{4.133054in}{2.697170in}}%
\pgfpathlineto{\pgfqpoint{4.133054in}{2.700119in}}%
\pgfpathlineto{\pgfqpoint{4.137595in}{2.700119in}}%
\pgfpathlineto{\pgfqpoint{4.137595in}{2.697170in}}%
\pgfpathmoveto{\pgfqpoint{4.137595in}{2.697170in}}%
\pgfpathlineto{\pgfqpoint{4.137595in}{2.697170in}}%
\pgfpathlineto{\pgfqpoint{4.137595in}{2.700119in}}%
\pgfpathlineto{\pgfqpoint{4.142136in}{2.700119in}}%
\pgfpathlineto{\pgfqpoint{4.142136in}{2.697170in}}%
\pgfpathmoveto{\pgfqpoint{4.137595in}{2.700119in}}%
\pgfpathlineto{\pgfqpoint{4.137595in}{2.700119in}}%
\pgfpathlineto{\pgfqpoint{4.137595in}{2.703069in}}%
\pgfpathlineto{\pgfqpoint{4.142136in}{2.703069in}}%
\pgfpathlineto{\pgfqpoint{4.142136in}{2.700119in}}%
\pgfpathmoveto{\pgfqpoint{4.142136in}{2.700119in}}%
\pgfpathlineto{\pgfqpoint{4.142136in}{2.700119in}}%
\pgfpathlineto{\pgfqpoint{4.142136in}{2.703069in}}%
\pgfpathlineto{\pgfqpoint{4.146677in}{2.703069in}}%
\pgfpathlineto{\pgfqpoint{4.146677in}{2.700119in}}%
\pgfpathmoveto{\pgfqpoint{4.142136in}{2.703069in}}%
\pgfpathlineto{\pgfqpoint{4.142136in}{2.703069in}}%
\pgfpathlineto{\pgfqpoint{4.142136in}{2.706018in}}%
\pgfpathlineto{\pgfqpoint{4.146677in}{2.706018in}}%
\pgfpathlineto{\pgfqpoint{4.146677in}{2.703069in}}%
\pgfpathmoveto{\pgfqpoint{4.146677in}{2.703069in}}%
\pgfpathlineto{\pgfqpoint{4.146677in}{2.703069in}}%
\pgfpathlineto{\pgfqpoint{4.146677in}{2.706018in}}%
\pgfpathlineto{\pgfqpoint{4.151218in}{2.706018in}}%
\pgfpathlineto{\pgfqpoint{4.151218in}{2.703069in}}%
\pgfpathmoveto{\pgfqpoint{4.146677in}{2.706018in}}%
\pgfpathlineto{\pgfqpoint{4.146677in}{2.706018in}}%
\pgfpathlineto{\pgfqpoint{4.146677in}{2.708967in}}%
\pgfpathlineto{\pgfqpoint{4.151218in}{2.708967in}}%
\pgfpathlineto{\pgfqpoint{4.151218in}{2.706018in}}%
\pgfpathmoveto{\pgfqpoint{4.151218in}{2.706018in}}%
\pgfpathlineto{\pgfqpoint{4.151218in}{2.706018in}}%
\pgfpathlineto{\pgfqpoint{4.151218in}{2.708967in}}%
\pgfpathlineto{\pgfqpoint{4.155759in}{2.708967in}}%
\pgfpathlineto{\pgfqpoint{4.155759in}{2.706018in}}%
\pgfpathmoveto{\pgfqpoint{4.151218in}{2.708967in}}%
\pgfpathlineto{\pgfqpoint{4.151218in}{2.708967in}}%
\pgfpathlineto{\pgfqpoint{4.151218in}{2.711916in}}%
\pgfpathlineto{\pgfqpoint{4.155759in}{2.711916in}}%
\pgfpathlineto{\pgfqpoint{4.155759in}{2.708967in}}%
\pgfpathmoveto{\pgfqpoint{4.155759in}{2.708967in}}%
\pgfpathlineto{\pgfqpoint{4.155759in}{2.708967in}}%
\pgfpathlineto{\pgfqpoint{4.155759in}{2.711916in}}%
\pgfpathlineto{\pgfqpoint{4.160301in}{2.711916in}}%
\pgfpathlineto{\pgfqpoint{4.160301in}{2.708967in}}%
\pgfpathmoveto{\pgfqpoint{4.155759in}{2.711916in}}%
\pgfpathlineto{\pgfqpoint{4.155759in}{2.711916in}}%
\pgfpathlineto{\pgfqpoint{4.155759in}{2.714865in}}%
\pgfpathlineto{\pgfqpoint{4.160301in}{2.714865in}}%
\pgfpathlineto{\pgfqpoint{4.160301in}{2.711916in}}%
\pgfpathmoveto{\pgfqpoint{4.160301in}{2.711916in}}%
\pgfpathlineto{\pgfqpoint{4.160301in}{2.711916in}}%
\pgfpathlineto{\pgfqpoint{4.160301in}{2.714865in}}%
\pgfpathlineto{\pgfqpoint{4.164842in}{2.714865in}}%
\pgfpathlineto{\pgfqpoint{4.164842in}{2.711916in}}%
\pgfpathmoveto{\pgfqpoint{4.160301in}{2.714865in}}%
\pgfpathlineto{\pgfqpoint{4.160301in}{2.714865in}}%
\pgfpathlineto{\pgfqpoint{4.160301in}{2.717814in}}%
\pgfpathlineto{\pgfqpoint{4.164842in}{2.717814in}}%
\pgfpathlineto{\pgfqpoint{4.164842in}{2.714865in}}%
\pgfpathmoveto{\pgfqpoint{4.164842in}{2.714865in}}%
\pgfpathlineto{\pgfqpoint{4.164842in}{2.714865in}}%
\pgfpathlineto{\pgfqpoint{4.164842in}{2.717814in}}%
\pgfpathlineto{\pgfqpoint{4.169383in}{2.717814in}}%
\pgfpathlineto{\pgfqpoint{4.169383in}{2.714865in}}%
\pgfpathmoveto{\pgfqpoint{4.164842in}{2.717814in}}%
\pgfpathlineto{\pgfqpoint{4.164842in}{2.717814in}}%
\pgfpathlineto{\pgfqpoint{4.164842in}{2.720764in}}%
\pgfpathlineto{\pgfqpoint{4.169383in}{2.720764in}}%
\pgfpathlineto{\pgfqpoint{4.169383in}{2.717814in}}%
\pgfpathmoveto{\pgfqpoint{4.169383in}{2.717814in}}%
\pgfpathlineto{\pgfqpoint{4.169383in}{2.717814in}}%
\pgfpathlineto{\pgfqpoint{4.169383in}{2.720764in}}%
\pgfpathlineto{\pgfqpoint{4.173924in}{2.720764in}}%
\pgfpathlineto{\pgfqpoint{4.173924in}{2.717814in}}%
\pgfpathmoveto{\pgfqpoint{4.169383in}{2.720764in}}%
\pgfpathlineto{\pgfqpoint{4.169383in}{2.720764in}}%
\pgfpathlineto{\pgfqpoint{4.169383in}{2.723713in}}%
\pgfpathlineto{\pgfqpoint{4.173924in}{2.723713in}}%
\pgfpathlineto{\pgfqpoint{4.173924in}{2.720764in}}%
\pgfpathmoveto{\pgfqpoint{4.173924in}{2.720764in}}%
\pgfpathlineto{\pgfqpoint{4.173924in}{2.720764in}}%
\pgfpathlineto{\pgfqpoint{4.173924in}{2.723713in}}%
\pgfpathlineto{\pgfqpoint{4.178465in}{2.723713in}}%
\pgfpathlineto{\pgfqpoint{4.178465in}{2.720764in}}%
\pgfpathmoveto{\pgfqpoint{4.173924in}{2.723713in}}%
\pgfpathlineto{\pgfqpoint{4.173924in}{2.723713in}}%
\pgfpathlineto{\pgfqpoint{4.173924in}{2.726662in}}%
\pgfpathlineto{\pgfqpoint{4.178465in}{2.726662in}}%
\pgfpathlineto{\pgfqpoint{4.178465in}{2.723713in}}%
\pgfpathmoveto{\pgfqpoint{4.178465in}{2.723713in}}%
\pgfpathlineto{\pgfqpoint{4.178465in}{2.723713in}}%
\pgfpathlineto{\pgfqpoint{4.178465in}{2.726662in}}%
\pgfpathlineto{\pgfqpoint{4.183006in}{2.726662in}}%
\pgfpathlineto{\pgfqpoint{4.183006in}{2.723713in}}%
\pgfpathmoveto{\pgfqpoint{4.178465in}{2.726662in}}%
\pgfpathlineto{\pgfqpoint{4.178465in}{2.726662in}}%
\pgfpathlineto{\pgfqpoint{4.178465in}{2.729611in}}%
\pgfpathlineto{\pgfqpoint{4.183006in}{2.729611in}}%
\pgfpathlineto{\pgfqpoint{4.183006in}{2.726662in}}%
\pgfpathmoveto{\pgfqpoint{4.183006in}{2.726662in}}%
\pgfpathlineto{\pgfqpoint{4.183006in}{2.726662in}}%
\pgfpathlineto{\pgfqpoint{4.183006in}{2.729611in}}%
\pgfpathlineto{\pgfqpoint{4.187547in}{2.729611in}}%
\pgfpathlineto{\pgfqpoint{4.187547in}{2.726662in}}%
\pgfpathmoveto{\pgfqpoint{4.183006in}{2.729611in}}%
\pgfpathlineto{\pgfqpoint{4.183006in}{2.729611in}}%
\pgfpathlineto{\pgfqpoint{4.183006in}{2.732560in}}%
\pgfpathlineto{\pgfqpoint{4.187547in}{2.732560in}}%
\pgfpathlineto{\pgfqpoint{4.187547in}{2.729611in}}%
\pgfpathmoveto{\pgfqpoint{4.187547in}{2.729611in}}%
\pgfpathlineto{\pgfqpoint{4.187547in}{2.729611in}}%
\pgfpathlineto{\pgfqpoint{4.187547in}{2.732560in}}%
\pgfpathlineto{\pgfqpoint{4.192088in}{2.732560in}}%
\pgfpathlineto{\pgfqpoint{4.192088in}{2.729611in}}%
\pgfpathmoveto{\pgfqpoint{4.187547in}{2.732560in}}%
\pgfpathlineto{\pgfqpoint{4.187547in}{2.732560in}}%
\pgfpathlineto{\pgfqpoint{4.187547in}{2.735509in}}%
\pgfpathlineto{\pgfqpoint{4.192088in}{2.735509in}}%
\pgfpathlineto{\pgfqpoint{4.192088in}{2.732560in}}%
\pgfpathmoveto{\pgfqpoint{4.192088in}{2.732560in}}%
\pgfpathlineto{\pgfqpoint{4.192088in}{2.732560in}}%
\pgfpathlineto{\pgfqpoint{4.192088in}{2.735509in}}%
\pgfpathlineto{\pgfqpoint{4.196629in}{2.735509in}}%
\pgfpathlineto{\pgfqpoint{4.196629in}{2.732560in}}%
\pgfpathmoveto{\pgfqpoint{4.192088in}{2.735509in}}%
\pgfpathlineto{\pgfqpoint{4.192088in}{2.735509in}}%
\pgfpathlineto{\pgfqpoint{4.192088in}{2.738459in}}%
\pgfpathlineto{\pgfqpoint{4.196629in}{2.738459in}}%
\pgfpathlineto{\pgfqpoint{4.196629in}{2.735509in}}%
\pgfpathmoveto{\pgfqpoint{4.196629in}{2.735509in}}%
\pgfpathlineto{\pgfqpoint{4.196629in}{2.735509in}}%
\pgfpathlineto{\pgfqpoint{4.196629in}{2.738459in}}%
\pgfpathlineto{\pgfqpoint{4.201170in}{2.738459in}}%
\pgfpathlineto{\pgfqpoint{4.201170in}{2.735509in}}%
\pgfpathmoveto{\pgfqpoint{4.196629in}{2.738459in}}%
\pgfpathlineto{\pgfqpoint{4.196629in}{2.738459in}}%
\pgfpathlineto{\pgfqpoint{4.196629in}{2.741408in}}%
\pgfpathlineto{\pgfqpoint{4.201170in}{2.741408in}}%
\pgfpathlineto{\pgfqpoint{4.201170in}{2.738459in}}%
\pgfpathmoveto{\pgfqpoint{4.201170in}{2.738459in}}%
\pgfpathlineto{\pgfqpoint{4.201170in}{2.738459in}}%
\pgfpathlineto{\pgfqpoint{4.201170in}{2.741408in}}%
\pgfpathlineto{\pgfqpoint{4.205711in}{2.741408in}}%
\pgfpathlineto{\pgfqpoint{4.205711in}{2.738459in}}%
\pgfpathmoveto{\pgfqpoint{4.201170in}{2.741408in}}%
\pgfpathlineto{\pgfqpoint{4.201170in}{2.741408in}}%
\pgfpathlineto{\pgfqpoint{4.201170in}{2.744357in}}%
\pgfpathlineto{\pgfqpoint{4.205711in}{2.744357in}}%
\pgfpathlineto{\pgfqpoint{4.205711in}{2.741408in}}%
\pgfpathmoveto{\pgfqpoint{4.205711in}{2.741408in}}%
\pgfpathlineto{\pgfqpoint{4.205711in}{2.741408in}}%
\pgfpathlineto{\pgfqpoint{4.205711in}{2.744357in}}%
\pgfpathlineto{\pgfqpoint{4.210252in}{2.744357in}}%
\pgfpathlineto{\pgfqpoint{4.210252in}{2.741408in}}%
\pgfpathmoveto{\pgfqpoint{4.205711in}{2.744357in}}%
\pgfpathlineto{\pgfqpoint{4.205711in}{2.744357in}}%
\pgfpathlineto{\pgfqpoint{4.205711in}{2.747306in}}%
\pgfpathlineto{\pgfqpoint{4.210252in}{2.747306in}}%
\pgfpathlineto{\pgfqpoint{4.210252in}{2.744357in}}%
\pgfpathmoveto{\pgfqpoint{4.210252in}{2.744357in}}%
\pgfpathlineto{\pgfqpoint{4.210252in}{2.744357in}}%
\pgfpathlineto{\pgfqpoint{4.210252in}{2.747306in}}%
\pgfpathlineto{\pgfqpoint{4.214793in}{2.747306in}}%
\pgfpathlineto{\pgfqpoint{4.214793in}{2.744357in}}%
\pgfpathmoveto{\pgfqpoint{4.210252in}{2.747306in}}%
\pgfpathlineto{\pgfqpoint{4.210252in}{2.747306in}}%
\pgfpathlineto{\pgfqpoint{4.210252in}{2.750255in}}%
\pgfpathlineto{\pgfqpoint{4.214793in}{2.750255in}}%
\pgfpathlineto{\pgfqpoint{4.214793in}{2.747306in}}%
\pgfpathmoveto{\pgfqpoint{4.214793in}{2.747306in}}%
\pgfpathlineto{\pgfqpoint{4.214793in}{2.747306in}}%
\pgfpathlineto{\pgfqpoint{4.214793in}{2.750255in}}%
\pgfpathlineto{\pgfqpoint{4.219334in}{2.750255in}}%
\pgfpathlineto{\pgfqpoint{4.219334in}{2.747306in}}%
\pgfpathmoveto{\pgfqpoint{4.214793in}{2.750255in}}%
\pgfpathlineto{\pgfqpoint{4.214793in}{2.750255in}}%
\pgfpathlineto{\pgfqpoint{4.214793in}{2.753204in}}%
\pgfpathlineto{\pgfqpoint{4.219334in}{2.753204in}}%
\pgfpathlineto{\pgfqpoint{4.219334in}{2.750255in}}%
\pgfpathmoveto{\pgfqpoint{4.219334in}{2.750255in}}%
\pgfpathlineto{\pgfqpoint{4.219334in}{2.750255in}}%
\pgfpathlineto{\pgfqpoint{4.219334in}{2.753204in}}%
\pgfpathlineto{\pgfqpoint{4.223876in}{2.753204in}}%
\pgfpathlineto{\pgfqpoint{4.223876in}{2.750255in}}%
\pgfpathmoveto{\pgfqpoint{4.219334in}{2.753204in}}%
\pgfpathlineto{\pgfqpoint{4.219334in}{2.753204in}}%
\pgfpathlineto{\pgfqpoint{4.219334in}{2.756154in}}%
\pgfpathlineto{\pgfqpoint{4.223876in}{2.756154in}}%
\pgfpathlineto{\pgfqpoint{4.223876in}{2.753204in}}%
\pgfpathmoveto{\pgfqpoint{4.223876in}{2.753204in}}%
\pgfpathlineto{\pgfqpoint{4.223876in}{2.753204in}}%
\pgfpathlineto{\pgfqpoint{4.223876in}{2.756154in}}%
\pgfpathlineto{\pgfqpoint{4.228417in}{2.756154in}}%
\pgfpathlineto{\pgfqpoint{4.228417in}{2.753204in}}%
\pgfpathmoveto{\pgfqpoint{4.223876in}{2.756154in}}%
\pgfpathlineto{\pgfqpoint{4.223876in}{2.756154in}}%
\pgfpathlineto{\pgfqpoint{4.223876in}{2.759103in}}%
\pgfpathlineto{\pgfqpoint{4.228417in}{2.759103in}}%
\pgfpathlineto{\pgfqpoint{4.228417in}{2.756154in}}%
\pgfpathmoveto{\pgfqpoint{4.228417in}{2.756154in}}%
\pgfpathlineto{\pgfqpoint{4.228417in}{2.756154in}}%
\pgfpathlineto{\pgfqpoint{4.228417in}{2.759103in}}%
\pgfpathlineto{\pgfqpoint{4.232958in}{2.759103in}}%
\pgfpathlineto{\pgfqpoint{4.232958in}{2.756154in}}%
\pgfpathmoveto{\pgfqpoint{4.228417in}{2.759103in}}%
\pgfpathlineto{\pgfqpoint{4.228417in}{2.759103in}}%
\pgfpathlineto{\pgfqpoint{4.228417in}{2.762052in}}%
\pgfpathlineto{\pgfqpoint{4.232958in}{2.762052in}}%
\pgfpathlineto{\pgfqpoint{4.232958in}{2.759103in}}%
\pgfpathmoveto{\pgfqpoint{4.232958in}{2.759103in}}%
\pgfpathlineto{\pgfqpoint{4.232958in}{2.759103in}}%
\pgfpathlineto{\pgfqpoint{4.232958in}{2.762052in}}%
\pgfpathlineto{\pgfqpoint{4.237499in}{2.762052in}}%
\pgfpathlineto{\pgfqpoint{4.237499in}{2.759103in}}%
\pgfpathmoveto{\pgfqpoint{4.232958in}{2.762052in}}%
\pgfpathlineto{\pgfqpoint{4.232958in}{2.762052in}}%
\pgfpathlineto{\pgfqpoint{4.232958in}{2.765001in}}%
\pgfpathlineto{\pgfqpoint{4.237499in}{2.765001in}}%
\pgfpathlineto{\pgfqpoint{4.237499in}{2.762052in}}%
\pgfpathmoveto{\pgfqpoint{4.237499in}{2.762052in}}%
\pgfpathlineto{\pgfqpoint{4.237499in}{2.762052in}}%
\pgfpathlineto{\pgfqpoint{4.237499in}{2.765001in}}%
\pgfpathlineto{\pgfqpoint{4.242040in}{2.765001in}}%
\pgfpathlineto{\pgfqpoint{4.242040in}{2.762052in}}%
\pgfpathmoveto{\pgfqpoint{4.237499in}{2.765001in}}%
\pgfpathlineto{\pgfqpoint{4.237499in}{2.765001in}}%
\pgfpathlineto{\pgfqpoint{4.237499in}{2.767950in}}%
\pgfpathlineto{\pgfqpoint{4.242040in}{2.767950in}}%
\pgfpathlineto{\pgfqpoint{4.242040in}{2.765001in}}%
\pgfpathmoveto{\pgfqpoint{4.242040in}{2.765001in}}%
\pgfpathlineto{\pgfqpoint{4.242040in}{2.765001in}}%
\pgfpathlineto{\pgfqpoint{4.242040in}{2.767950in}}%
\pgfpathlineto{\pgfqpoint{4.246581in}{2.767950in}}%
\pgfpathlineto{\pgfqpoint{4.246581in}{2.765001in}}%
\pgfpathmoveto{\pgfqpoint{4.242040in}{2.767950in}}%
\pgfpathlineto{\pgfqpoint{4.242040in}{2.767950in}}%
\pgfpathlineto{\pgfqpoint{4.242040in}{2.770900in}}%
\pgfpathlineto{\pgfqpoint{4.246581in}{2.770900in}}%
\pgfpathlineto{\pgfqpoint{4.246581in}{2.767950in}}%
\pgfpathmoveto{\pgfqpoint{4.246581in}{2.767950in}}%
\pgfpathlineto{\pgfqpoint{4.246581in}{2.767950in}}%
\pgfpathlineto{\pgfqpoint{4.246581in}{2.770900in}}%
\pgfpathlineto{\pgfqpoint{4.251122in}{2.770900in}}%
\pgfpathlineto{\pgfqpoint{4.251122in}{2.767950in}}%
\pgfpathmoveto{\pgfqpoint{4.246581in}{2.770900in}}%
\pgfpathlineto{\pgfqpoint{4.246581in}{2.770900in}}%
\pgfpathlineto{\pgfqpoint{4.246581in}{2.773849in}}%
\pgfpathlineto{\pgfqpoint{4.251122in}{2.773849in}}%
\pgfpathlineto{\pgfqpoint{4.251122in}{2.770900in}}%
\pgfpathmoveto{\pgfqpoint{4.251122in}{2.770900in}}%
\pgfpathlineto{\pgfqpoint{4.251122in}{2.770900in}}%
\pgfpathlineto{\pgfqpoint{4.251122in}{2.773849in}}%
\pgfpathlineto{\pgfqpoint{4.255663in}{2.773849in}}%
\pgfpathlineto{\pgfqpoint{4.255663in}{2.770900in}}%
\pgfpathmoveto{\pgfqpoint{4.251122in}{2.773849in}}%
\pgfpathlineto{\pgfqpoint{4.251122in}{2.773849in}}%
\pgfpathlineto{\pgfqpoint{4.251122in}{2.776798in}}%
\pgfpathlineto{\pgfqpoint{4.255663in}{2.776798in}}%
\pgfpathlineto{\pgfqpoint{4.255663in}{2.773849in}}%
\pgfpathmoveto{\pgfqpoint{4.255663in}{2.773849in}}%
\pgfpathlineto{\pgfqpoint{4.255663in}{2.773849in}}%
\pgfpathlineto{\pgfqpoint{4.255663in}{2.776798in}}%
\pgfpathlineto{\pgfqpoint{4.260204in}{2.776798in}}%
\pgfpathlineto{\pgfqpoint{4.260204in}{2.773849in}}%
\pgfpathmoveto{\pgfqpoint{4.255663in}{2.776798in}}%
\pgfpathlineto{\pgfqpoint{4.255663in}{2.776798in}}%
\pgfpathlineto{\pgfqpoint{4.255663in}{2.779747in}}%
\pgfpathlineto{\pgfqpoint{4.260204in}{2.779747in}}%
\pgfpathlineto{\pgfqpoint{4.260204in}{2.776798in}}%
\pgfpathmoveto{\pgfqpoint{4.260204in}{2.776798in}}%
\pgfpathlineto{\pgfqpoint{4.260204in}{2.776798in}}%
\pgfpathlineto{\pgfqpoint{4.260204in}{2.779747in}}%
\pgfpathlineto{\pgfqpoint{4.264745in}{2.779747in}}%
\pgfpathlineto{\pgfqpoint{4.264745in}{2.776798in}}%
\pgfpathmoveto{\pgfqpoint{4.260204in}{2.779747in}}%
\pgfpathlineto{\pgfqpoint{4.260204in}{2.779747in}}%
\pgfpathlineto{\pgfqpoint{4.260204in}{2.782696in}}%
\pgfpathlineto{\pgfqpoint{4.264745in}{2.782696in}}%
\pgfpathlineto{\pgfqpoint{4.264745in}{2.779747in}}%
\pgfpathmoveto{\pgfqpoint{4.264745in}{2.779747in}}%
\pgfpathlineto{\pgfqpoint{4.264745in}{2.779747in}}%
\pgfpathlineto{\pgfqpoint{4.264745in}{2.782696in}}%
\pgfpathlineto{\pgfqpoint{4.269286in}{2.782696in}}%
\pgfpathlineto{\pgfqpoint{4.269286in}{2.779747in}}%
\pgfpathmoveto{\pgfqpoint{4.264745in}{2.782696in}}%
\pgfpathlineto{\pgfqpoint{4.264745in}{2.782696in}}%
\pgfpathlineto{\pgfqpoint{4.264745in}{2.785646in}}%
\pgfpathlineto{\pgfqpoint{4.269286in}{2.785646in}}%
\pgfpathlineto{\pgfqpoint{4.269286in}{2.782696in}}%
\pgfpathmoveto{\pgfqpoint{4.269286in}{2.782696in}}%
\pgfpathlineto{\pgfqpoint{4.269286in}{2.782696in}}%
\pgfpathlineto{\pgfqpoint{4.269286in}{2.785646in}}%
\pgfpathlineto{\pgfqpoint{4.273827in}{2.785646in}}%
\pgfpathlineto{\pgfqpoint{4.273827in}{2.782696in}}%
\pgfpathmoveto{\pgfqpoint{4.269286in}{2.785646in}}%
\pgfpathlineto{\pgfqpoint{4.269286in}{2.785646in}}%
\pgfpathlineto{\pgfqpoint{4.269286in}{2.788595in}}%
\pgfpathlineto{\pgfqpoint{4.273827in}{2.788595in}}%
\pgfpathlineto{\pgfqpoint{4.273827in}{2.785646in}}%
\pgfpathmoveto{\pgfqpoint{4.273827in}{2.785646in}}%
\pgfpathlineto{\pgfqpoint{4.273827in}{2.785646in}}%
\pgfpathlineto{\pgfqpoint{4.273827in}{2.788595in}}%
\pgfpathlineto{\pgfqpoint{4.278368in}{2.788595in}}%
\pgfpathlineto{\pgfqpoint{4.278368in}{2.785646in}}%
\pgfpathmoveto{\pgfqpoint{4.273827in}{2.788595in}}%
\pgfpathlineto{\pgfqpoint{4.273827in}{2.788595in}}%
\pgfpathlineto{\pgfqpoint{4.273827in}{2.791544in}}%
\pgfpathlineto{\pgfqpoint{4.278368in}{2.791544in}}%
\pgfpathlineto{\pgfqpoint{4.278368in}{2.788595in}}%
\pgfpathmoveto{\pgfqpoint{4.278368in}{2.788595in}}%
\pgfpathlineto{\pgfqpoint{4.278368in}{2.788595in}}%
\pgfpathlineto{\pgfqpoint{4.278368in}{2.791544in}}%
\pgfpathlineto{\pgfqpoint{4.282909in}{2.791544in}}%
\pgfpathlineto{\pgfqpoint{4.282909in}{2.788595in}}%
\pgfpathmoveto{\pgfqpoint{4.278368in}{2.791544in}}%
\pgfpathlineto{\pgfqpoint{4.278368in}{2.791544in}}%
\pgfpathlineto{\pgfqpoint{4.278368in}{2.794493in}}%
\pgfpathlineto{\pgfqpoint{4.282909in}{2.794493in}}%
\pgfpathlineto{\pgfqpoint{4.282909in}{2.791544in}}%
\pgfpathmoveto{\pgfqpoint{4.282909in}{2.791544in}}%
\pgfpathlineto{\pgfqpoint{4.282909in}{2.791544in}}%
\pgfpathlineto{\pgfqpoint{4.282909in}{2.794493in}}%
\pgfpathlineto{\pgfqpoint{4.287450in}{2.794493in}}%
\pgfpathlineto{\pgfqpoint{4.287450in}{2.791544in}}%
\pgfpathmoveto{\pgfqpoint{4.282909in}{2.794493in}}%
\pgfpathlineto{\pgfqpoint{4.282909in}{2.794493in}}%
\pgfpathlineto{\pgfqpoint{4.282909in}{2.797442in}}%
\pgfpathlineto{\pgfqpoint{4.287450in}{2.797442in}}%
\pgfpathlineto{\pgfqpoint{4.287450in}{2.794493in}}%
\pgfpathmoveto{\pgfqpoint{4.287450in}{2.794493in}}%
\pgfpathlineto{\pgfqpoint{4.287450in}{2.794493in}}%
\pgfpathlineto{\pgfqpoint{4.287450in}{2.797442in}}%
\pgfpathlineto{\pgfqpoint{4.291991in}{2.797442in}}%
\pgfpathlineto{\pgfqpoint{4.291991in}{2.794493in}}%
\pgfpathmoveto{\pgfqpoint{4.287450in}{2.797442in}}%
\pgfpathlineto{\pgfqpoint{4.287450in}{2.797442in}}%
\pgfpathlineto{\pgfqpoint{4.287450in}{2.800392in}}%
\pgfpathlineto{\pgfqpoint{4.291991in}{2.800392in}}%
\pgfpathlineto{\pgfqpoint{4.291991in}{2.797442in}}%
\pgfpathmoveto{\pgfqpoint{4.291991in}{2.797442in}}%
\pgfpathlineto{\pgfqpoint{4.291991in}{2.797442in}}%
\pgfpathlineto{\pgfqpoint{4.291991in}{2.800392in}}%
\pgfpathlineto{\pgfqpoint{4.296532in}{2.800392in}}%
\pgfpathlineto{\pgfqpoint{4.296532in}{2.797442in}}%
\pgfpathmoveto{\pgfqpoint{4.291991in}{2.800392in}}%
\pgfpathlineto{\pgfqpoint{4.291991in}{2.800392in}}%
\pgfpathlineto{\pgfqpoint{4.291991in}{2.803341in}}%
\pgfpathlineto{\pgfqpoint{4.296532in}{2.803341in}}%
\pgfpathlineto{\pgfqpoint{4.296532in}{2.800392in}}%
\pgfpathmoveto{\pgfqpoint{4.296532in}{2.800392in}}%
\pgfpathlineto{\pgfqpoint{4.296532in}{2.800392in}}%
\pgfpathlineto{\pgfqpoint{4.296532in}{2.803341in}}%
\pgfpathlineto{\pgfqpoint{4.301073in}{2.803341in}}%
\pgfpathlineto{\pgfqpoint{4.301073in}{2.800392in}}%
\pgfpathmoveto{\pgfqpoint{4.296532in}{2.803341in}}%
\pgfpathlineto{\pgfqpoint{4.296532in}{2.803341in}}%
\pgfpathlineto{\pgfqpoint{4.296532in}{2.806290in}}%
\pgfpathlineto{\pgfqpoint{4.301073in}{2.806290in}}%
\pgfpathlineto{\pgfqpoint{4.301073in}{2.803341in}}%
\pgfpathmoveto{\pgfqpoint{4.301073in}{2.803341in}}%
\pgfpathlineto{\pgfqpoint{4.301073in}{2.803341in}}%
\pgfpathlineto{\pgfqpoint{4.301073in}{2.806290in}}%
\pgfpathlineto{\pgfqpoint{4.305614in}{2.806290in}}%
\pgfpathlineto{\pgfqpoint{4.305614in}{2.803341in}}%
\pgfpathmoveto{\pgfqpoint{4.301073in}{2.806290in}}%
\pgfpathlineto{\pgfqpoint{4.301073in}{2.806290in}}%
\pgfpathlineto{\pgfqpoint{4.301073in}{2.809239in}}%
\pgfpathlineto{\pgfqpoint{4.305614in}{2.809239in}}%
\pgfpathlineto{\pgfqpoint{4.305614in}{2.806290in}}%
\pgfpathmoveto{\pgfqpoint{4.305614in}{2.806290in}}%
\pgfpathlineto{\pgfqpoint{4.305614in}{2.806290in}}%
\pgfpathlineto{\pgfqpoint{4.305614in}{2.809239in}}%
\pgfpathlineto{\pgfqpoint{4.310155in}{2.809239in}}%
\pgfpathlineto{\pgfqpoint{4.310155in}{2.806290in}}%
\pgfpathmoveto{\pgfqpoint{4.305614in}{2.809239in}}%
\pgfpathlineto{\pgfqpoint{4.305614in}{2.809239in}}%
\pgfpathlineto{\pgfqpoint{4.305614in}{2.812188in}}%
\pgfpathlineto{\pgfqpoint{4.310155in}{2.812188in}}%
\pgfpathlineto{\pgfqpoint{4.310155in}{2.809239in}}%
\pgfpathmoveto{\pgfqpoint{4.310155in}{2.809239in}}%
\pgfpathlineto{\pgfqpoint{4.310155in}{2.809239in}}%
\pgfpathlineto{\pgfqpoint{4.310155in}{2.812188in}}%
\pgfpathlineto{\pgfqpoint{4.314696in}{2.812188in}}%
\pgfpathlineto{\pgfqpoint{4.314696in}{2.809239in}}%
\pgfpathmoveto{\pgfqpoint{4.310155in}{2.812188in}}%
\pgfpathlineto{\pgfqpoint{4.310155in}{2.812188in}}%
\pgfpathlineto{\pgfqpoint{4.310155in}{2.815138in}}%
\pgfpathlineto{\pgfqpoint{4.314696in}{2.815138in}}%
\pgfpathlineto{\pgfqpoint{4.314696in}{2.812188in}}%
\pgfpathmoveto{\pgfqpoint{4.314696in}{2.812188in}}%
\pgfpathlineto{\pgfqpoint{4.314696in}{2.812188in}}%
\pgfpathlineto{\pgfqpoint{4.314696in}{2.815138in}}%
\pgfpathlineto{\pgfqpoint{4.319237in}{2.815138in}}%
\pgfpathlineto{\pgfqpoint{4.319237in}{2.812188in}}%
\pgfpathmoveto{\pgfqpoint{4.314696in}{2.815138in}}%
\pgfpathlineto{\pgfqpoint{4.314696in}{2.815138in}}%
\pgfpathlineto{\pgfqpoint{4.314696in}{2.818087in}}%
\pgfpathlineto{\pgfqpoint{4.319237in}{2.818087in}}%
\pgfpathlineto{\pgfqpoint{4.319237in}{2.815138in}}%
\pgfpathmoveto{\pgfqpoint{4.319237in}{2.815138in}}%
\pgfpathlineto{\pgfqpoint{4.319237in}{2.815138in}}%
\pgfpathlineto{\pgfqpoint{4.319237in}{2.818087in}}%
\pgfpathlineto{\pgfqpoint{4.323778in}{2.818087in}}%
\pgfpathlineto{\pgfqpoint{4.323778in}{2.815138in}}%
\pgfpathmoveto{\pgfqpoint{4.319237in}{2.818087in}}%
\pgfpathlineto{\pgfqpoint{4.319237in}{2.818087in}}%
\pgfpathlineto{\pgfqpoint{4.319237in}{2.821036in}}%
\pgfpathlineto{\pgfqpoint{4.323778in}{2.821036in}}%
\pgfpathlineto{\pgfqpoint{4.323778in}{2.818087in}}%
\pgfpathmoveto{\pgfqpoint{4.323778in}{2.818087in}}%
\pgfpathlineto{\pgfqpoint{4.323778in}{2.818087in}}%
\pgfpathlineto{\pgfqpoint{4.323778in}{2.821036in}}%
\pgfpathlineto{\pgfqpoint{4.328319in}{2.821036in}}%
\pgfpathlineto{\pgfqpoint{4.328319in}{2.818087in}}%
\pgfpathmoveto{\pgfqpoint{4.323778in}{2.821036in}}%
\pgfpathlineto{\pgfqpoint{4.323778in}{2.821036in}}%
\pgfpathlineto{\pgfqpoint{4.323778in}{2.823985in}}%
\pgfpathlineto{\pgfqpoint{4.328319in}{2.823985in}}%
\pgfpathlineto{\pgfqpoint{4.328319in}{2.821036in}}%
\pgfpathmoveto{\pgfqpoint{4.328319in}{2.821036in}}%
\pgfpathlineto{\pgfqpoint{4.328319in}{2.821036in}}%
\pgfpathlineto{\pgfqpoint{4.328319in}{2.823985in}}%
\pgfpathlineto{\pgfqpoint{4.332860in}{2.823985in}}%
\pgfpathlineto{\pgfqpoint{4.332860in}{2.821036in}}%
\pgfpathmoveto{\pgfqpoint{4.328319in}{2.823985in}}%
\pgfpathlineto{\pgfqpoint{4.328319in}{2.823985in}}%
\pgfpathlineto{\pgfqpoint{4.328319in}{2.826934in}}%
\pgfpathlineto{\pgfqpoint{4.332860in}{2.826934in}}%
\pgfpathlineto{\pgfqpoint{4.332860in}{2.823985in}}%
\pgfpathmoveto{\pgfqpoint{4.332860in}{2.823985in}}%
\pgfpathlineto{\pgfqpoint{4.332860in}{2.823985in}}%
\pgfpathlineto{\pgfqpoint{4.332860in}{2.826934in}}%
\pgfpathlineto{\pgfqpoint{4.337401in}{2.826934in}}%
\pgfpathlineto{\pgfqpoint{4.337401in}{2.823985in}}%
\pgfpathmoveto{\pgfqpoint{4.332860in}{2.826934in}}%
\pgfpathlineto{\pgfqpoint{4.332860in}{2.826934in}}%
\pgfpathlineto{\pgfqpoint{4.332860in}{2.829884in}}%
\pgfpathlineto{\pgfqpoint{4.337401in}{2.829884in}}%
\pgfpathlineto{\pgfqpoint{4.337401in}{2.826934in}}%
\pgfpathmoveto{\pgfqpoint{4.337401in}{2.826934in}}%
\pgfpathlineto{\pgfqpoint{4.337401in}{2.826934in}}%
\pgfpathlineto{\pgfqpoint{4.337401in}{2.829884in}}%
\pgfpathlineto{\pgfqpoint{4.341942in}{2.829884in}}%
\pgfpathlineto{\pgfqpoint{4.341942in}{2.826934in}}%
\pgfpathmoveto{\pgfqpoint{4.337401in}{2.829884in}}%
\pgfpathlineto{\pgfqpoint{4.337401in}{2.829884in}}%
\pgfpathlineto{\pgfqpoint{4.337401in}{2.832833in}}%
\pgfpathlineto{\pgfqpoint{4.341942in}{2.832833in}}%
\pgfpathlineto{\pgfqpoint{4.341942in}{2.829884in}}%
\pgfpathmoveto{\pgfqpoint{4.341942in}{2.829884in}}%
\pgfpathlineto{\pgfqpoint{4.341942in}{2.829884in}}%
\pgfpathlineto{\pgfqpoint{4.341942in}{2.832833in}}%
\pgfpathlineto{\pgfqpoint{4.346483in}{2.832833in}}%
\pgfpathlineto{\pgfqpoint{4.346483in}{2.829884in}}%
\pgfpathmoveto{\pgfqpoint{4.341942in}{2.832833in}}%
\pgfpathlineto{\pgfqpoint{4.341942in}{2.832833in}}%
\pgfpathlineto{\pgfqpoint{4.341942in}{2.835782in}}%
\pgfpathlineto{\pgfqpoint{4.346483in}{2.835782in}}%
\pgfpathlineto{\pgfqpoint{4.346483in}{2.832833in}}%
\pgfpathmoveto{\pgfqpoint{4.346483in}{2.832833in}}%
\pgfpathlineto{\pgfqpoint{4.346483in}{2.832833in}}%
\pgfpathlineto{\pgfqpoint{4.346483in}{2.835782in}}%
\pgfpathlineto{\pgfqpoint{4.351024in}{2.835782in}}%
\pgfpathlineto{\pgfqpoint{4.351024in}{2.832833in}}%
\pgfpathmoveto{\pgfqpoint{4.346483in}{2.835782in}}%
\pgfpathlineto{\pgfqpoint{4.346483in}{2.835782in}}%
\pgfpathlineto{\pgfqpoint{4.346483in}{2.838731in}}%
\pgfpathlineto{\pgfqpoint{4.351024in}{2.838731in}}%
\pgfpathlineto{\pgfqpoint{4.351024in}{2.835782in}}%
\pgfpathmoveto{\pgfqpoint{4.351024in}{2.835782in}}%
\pgfpathlineto{\pgfqpoint{4.351024in}{2.835782in}}%
\pgfpathlineto{\pgfqpoint{4.351024in}{2.838731in}}%
\pgfpathlineto{\pgfqpoint{4.355565in}{2.838731in}}%
\pgfpathlineto{\pgfqpoint{4.355565in}{2.835782in}}%
\pgfpathmoveto{\pgfqpoint{4.351024in}{2.838731in}}%
\pgfpathlineto{\pgfqpoint{4.351024in}{2.838731in}}%
\pgfpathlineto{\pgfqpoint{4.351024in}{2.841681in}}%
\pgfpathlineto{\pgfqpoint{4.355565in}{2.841681in}}%
\pgfpathlineto{\pgfqpoint{4.355565in}{2.838731in}}%
\pgfpathmoveto{\pgfqpoint{4.355565in}{2.838731in}}%
\pgfpathlineto{\pgfqpoint{4.355565in}{2.838731in}}%
\pgfpathlineto{\pgfqpoint{4.355565in}{2.841681in}}%
\pgfpathlineto{\pgfqpoint{4.360106in}{2.841681in}}%
\pgfpathlineto{\pgfqpoint{4.360106in}{2.838731in}}%
\pgfpathmoveto{\pgfqpoint{4.355565in}{2.841681in}}%
\pgfpathlineto{\pgfqpoint{4.355565in}{2.841681in}}%
\pgfpathlineto{\pgfqpoint{4.355565in}{2.844630in}}%
\pgfpathlineto{\pgfqpoint{4.360106in}{2.844630in}}%
\pgfpathlineto{\pgfqpoint{4.360106in}{2.841681in}}%
\pgfpathmoveto{\pgfqpoint{4.360106in}{2.841681in}}%
\pgfpathlineto{\pgfqpoint{4.360106in}{2.841681in}}%
\pgfpathlineto{\pgfqpoint{4.360106in}{2.844630in}}%
\pgfpathlineto{\pgfqpoint{4.364648in}{2.844630in}}%
\pgfpathlineto{\pgfqpoint{4.364648in}{2.841681in}}%
\pgfpathmoveto{\pgfqpoint{4.360106in}{2.844630in}}%
\pgfpathlineto{\pgfqpoint{4.360106in}{2.844630in}}%
\pgfpathlineto{\pgfqpoint{4.360106in}{2.847579in}}%
\pgfpathlineto{\pgfqpoint{4.364648in}{2.847579in}}%
\pgfpathlineto{\pgfqpoint{4.364648in}{2.844630in}}%
\pgfpathmoveto{\pgfqpoint{4.364648in}{2.844630in}}%
\pgfpathlineto{\pgfqpoint{4.364648in}{2.844630in}}%
\pgfpathlineto{\pgfqpoint{4.364648in}{2.847579in}}%
\pgfpathlineto{\pgfqpoint{4.369189in}{2.847579in}}%
\pgfpathlineto{\pgfqpoint{4.369189in}{2.844630in}}%
\pgfpathmoveto{\pgfqpoint{4.364648in}{2.847579in}}%
\pgfpathlineto{\pgfqpoint{4.364648in}{2.847579in}}%
\pgfpathlineto{\pgfqpoint{4.364648in}{2.850528in}}%
\pgfpathlineto{\pgfqpoint{4.369189in}{2.850528in}}%
\pgfpathlineto{\pgfqpoint{4.369189in}{2.847579in}}%
\pgfpathmoveto{\pgfqpoint{4.369189in}{2.847579in}}%
\pgfpathlineto{\pgfqpoint{4.369189in}{2.847579in}}%
\pgfpathlineto{\pgfqpoint{4.369189in}{2.850528in}}%
\pgfpathlineto{\pgfqpoint{4.373730in}{2.850528in}}%
\pgfpathlineto{\pgfqpoint{4.373730in}{2.847579in}}%
\pgfpathmoveto{\pgfqpoint{4.369189in}{2.850528in}}%
\pgfpathlineto{\pgfqpoint{4.369189in}{2.850528in}}%
\pgfpathlineto{\pgfqpoint{4.369189in}{2.853477in}}%
\pgfpathlineto{\pgfqpoint{4.373730in}{2.853477in}}%
\pgfpathlineto{\pgfqpoint{4.373730in}{2.850528in}}%
\pgfpathmoveto{\pgfqpoint{4.373730in}{2.850528in}}%
\pgfpathlineto{\pgfqpoint{4.373730in}{2.850528in}}%
\pgfpathlineto{\pgfqpoint{4.373730in}{2.853477in}}%
\pgfpathlineto{\pgfqpoint{4.378271in}{2.853477in}}%
\pgfpathlineto{\pgfqpoint{4.378271in}{2.850528in}}%
\pgfpathmoveto{\pgfqpoint{4.373730in}{2.853477in}}%
\pgfpathlineto{\pgfqpoint{4.373730in}{2.853477in}}%
\pgfpathlineto{\pgfqpoint{4.373730in}{2.856427in}}%
\pgfpathlineto{\pgfqpoint{4.378271in}{2.856427in}}%
\pgfpathlineto{\pgfqpoint{4.378271in}{2.853477in}}%
\pgfpathmoveto{\pgfqpoint{4.378271in}{2.853477in}}%
\pgfpathlineto{\pgfqpoint{4.378271in}{2.853477in}}%
\pgfpathlineto{\pgfqpoint{4.378271in}{2.856427in}}%
\pgfpathlineto{\pgfqpoint{4.382812in}{2.856427in}}%
\pgfpathlineto{\pgfqpoint{4.382812in}{2.853477in}}%
\pgfpathmoveto{\pgfqpoint{4.378271in}{2.856427in}}%
\pgfpathlineto{\pgfqpoint{4.378271in}{2.856427in}}%
\pgfpathlineto{\pgfqpoint{4.378271in}{2.859376in}}%
\pgfpathlineto{\pgfqpoint{4.382812in}{2.859376in}}%
\pgfpathlineto{\pgfqpoint{4.382812in}{2.856427in}}%
\pgfpathmoveto{\pgfqpoint{4.382812in}{2.856427in}}%
\pgfpathlineto{\pgfqpoint{4.382812in}{2.856427in}}%
\pgfpathlineto{\pgfqpoint{4.382812in}{2.859376in}}%
\pgfpathlineto{\pgfqpoint{4.387353in}{2.859376in}}%
\pgfpathlineto{\pgfqpoint{4.387353in}{2.856427in}}%
\pgfpathmoveto{\pgfqpoint{4.382812in}{2.859376in}}%
\pgfpathlineto{\pgfqpoint{4.382812in}{2.859376in}}%
\pgfpathlineto{\pgfqpoint{4.382812in}{2.862325in}}%
\pgfpathlineto{\pgfqpoint{4.387353in}{2.862325in}}%
\pgfpathlineto{\pgfqpoint{4.387353in}{2.859376in}}%
\pgfpathmoveto{\pgfqpoint{4.387353in}{2.859376in}}%
\pgfpathlineto{\pgfqpoint{4.387353in}{2.859376in}}%
\pgfpathlineto{\pgfqpoint{4.387353in}{2.862325in}}%
\pgfpathlineto{\pgfqpoint{4.391893in}{2.862325in}}%
\pgfpathlineto{\pgfqpoint{4.391893in}{2.859376in}}%
\pgfpathmoveto{\pgfqpoint{4.387353in}{2.862325in}}%
\pgfpathlineto{\pgfqpoint{4.387353in}{2.862325in}}%
\pgfpathlineto{\pgfqpoint{4.387353in}{2.865274in}}%
\pgfpathlineto{\pgfqpoint{4.391893in}{2.865274in}}%
\pgfpathlineto{\pgfqpoint{4.391893in}{2.862325in}}%
\pgfpathmoveto{\pgfqpoint{4.391893in}{2.862325in}}%
\pgfpathlineto{\pgfqpoint{4.391893in}{2.862325in}}%
\pgfpathlineto{\pgfqpoint{4.391893in}{2.865274in}}%
\pgfpathlineto{\pgfqpoint{4.396434in}{2.865274in}}%
\pgfpathlineto{\pgfqpoint{4.396434in}{2.862325in}}%
\pgfpathmoveto{\pgfqpoint{4.391893in}{2.865274in}}%
\pgfpathlineto{\pgfqpoint{4.391893in}{2.865274in}}%
\pgfpathlineto{\pgfqpoint{4.391893in}{2.868223in}}%
\pgfpathlineto{\pgfqpoint{4.396434in}{2.868223in}}%
\pgfpathlineto{\pgfqpoint{4.396434in}{2.865274in}}%
\pgfpathmoveto{\pgfqpoint{4.396434in}{2.865274in}}%
\pgfpathlineto{\pgfqpoint{4.396434in}{2.865274in}}%
\pgfpathlineto{\pgfqpoint{4.396434in}{2.868223in}}%
\pgfpathlineto{\pgfqpoint{4.400975in}{2.868223in}}%
\pgfpathlineto{\pgfqpoint{4.400975in}{2.865274in}}%
\pgfpathmoveto{\pgfqpoint{4.396434in}{2.868223in}}%
\pgfpathlineto{\pgfqpoint{4.396434in}{2.868223in}}%
\pgfpathlineto{\pgfqpoint{4.396434in}{2.871172in}}%
\pgfpathlineto{\pgfqpoint{4.400975in}{2.871172in}}%
\pgfpathlineto{\pgfqpoint{4.400975in}{2.868223in}}%
\pgfpathmoveto{\pgfqpoint{4.400975in}{2.868223in}}%
\pgfpathlineto{\pgfqpoint{4.400975in}{2.868223in}}%
\pgfpathlineto{\pgfqpoint{4.400975in}{2.871172in}}%
\pgfpathlineto{\pgfqpoint{4.405516in}{2.871172in}}%
\pgfpathlineto{\pgfqpoint{4.405516in}{2.868223in}}%
\pgfpathmoveto{\pgfqpoint{4.400975in}{2.871172in}}%
\pgfpathlineto{\pgfqpoint{4.400975in}{2.871172in}}%
\pgfpathlineto{\pgfqpoint{4.400975in}{2.874121in}}%
\pgfpathlineto{\pgfqpoint{4.405516in}{2.874121in}}%
\pgfpathlineto{\pgfqpoint{4.405516in}{2.871172in}}%
\pgfpathmoveto{\pgfqpoint{4.405516in}{2.871172in}}%
\pgfpathlineto{\pgfqpoint{4.405516in}{2.871172in}}%
\pgfpathlineto{\pgfqpoint{4.405516in}{2.874121in}}%
\pgfpathlineto{\pgfqpoint{4.410057in}{2.874121in}}%
\pgfpathlineto{\pgfqpoint{4.410057in}{2.871172in}}%
\pgfpathmoveto{\pgfqpoint{4.405516in}{2.874121in}}%
\pgfpathlineto{\pgfqpoint{4.405516in}{2.874121in}}%
\pgfpathlineto{\pgfqpoint{4.405516in}{2.877070in}}%
\pgfpathlineto{\pgfqpoint{4.410057in}{2.877070in}}%
\pgfpathlineto{\pgfqpoint{4.410057in}{2.874121in}}%
\pgfpathmoveto{\pgfqpoint{4.410057in}{2.874121in}}%
\pgfpathlineto{\pgfqpoint{4.410057in}{2.874121in}}%
\pgfpathlineto{\pgfqpoint{4.410057in}{2.877070in}}%
\pgfpathlineto{\pgfqpoint{4.414598in}{2.877070in}}%
\pgfpathlineto{\pgfqpoint{4.414598in}{2.874121in}}%
\pgfpathmoveto{\pgfqpoint{4.410057in}{2.877070in}}%
\pgfpathlineto{\pgfqpoint{4.410057in}{2.877070in}}%
\pgfpathlineto{\pgfqpoint{4.410057in}{2.880020in}}%
\pgfpathlineto{\pgfqpoint{4.414598in}{2.880020in}}%
\pgfpathlineto{\pgfqpoint{4.414598in}{2.877070in}}%
\pgfpathmoveto{\pgfqpoint{4.414598in}{2.877070in}}%
\pgfpathlineto{\pgfqpoint{4.414598in}{2.877070in}}%
\pgfpathlineto{\pgfqpoint{4.414598in}{2.880020in}}%
\pgfpathlineto{\pgfqpoint{4.419139in}{2.880020in}}%
\pgfpathlineto{\pgfqpoint{4.419139in}{2.877070in}}%
\pgfpathmoveto{\pgfqpoint{4.414598in}{2.880020in}}%
\pgfpathlineto{\pgfqpoint{4.414598in}{2.880020in}}%
\pgfpathlineto{\pgfqpoint{4.414598in}{2.882969in}}%
\pgfpathlineto{\pgfqpoint{4.419139in}{2.882969in}}%
\pgfpathlineto{\pgfqpoint{4.419139in}{2.880020in}}%
\pgfpathmoveto{\pgfqpoint{4.419139in}{2.880020in}}%
\pgfpathlineto{\pgfqpoint{4.419139in}{2.880020in}}%
\pgfpathlineto{\pgfqpoint{4.419139in}{2.882969in}}%
\pgfpathlineto{\pgfqpoint{4.423680in}{2.882969in}}%
\pgfpathlineto{\pgfqpoint{4.423680in}{2.880020in}}%
\pgfpathmoveto{\pgfqpoint{4.419139in}{2.882969in}}%
\pgfpathlineto{\pgfqpoint{4.419139in}{2.882969in}}%
\pgfpathlineto{\pgfqpoint{4.419139in}{2.885918in}}%
\pgfpathlineto{\pgfqpoint{4.423680in}{2.885918in}}%
\pgfpathlineto{\pgfqpoint{4.423680in}{2.882969in}}%
\pgfpathmoveto{\pgfqpoint{4.423680in}{2.882969in}}%
\pgfpathlineto{\pgfqpoint{4.423680in}{2.882969in}}%
\pgfpathlineto{\pgfqpoint{4.423680in}{2.885918in}}%
\pgfpathlineto{\pgfqpoint{4.428221in}{2.885918in}}%
\pgfpathlineto{\pgfqpoint{4.428221in}{2.882969in}}%
\pgfpathmoveto{\pgfqpoint{4.423680in}{2.885918in}}%
\pgfpathlineto{\pgfqpoint{4.423680in}{2.885918in}}%
\pgfpathlineto{\pgfqpoint{4.423680in}{2.888867in}}%
\pgfpathlineto{\pgfqpoint{4.428221in}{2.888867in}}%
\pgfpathlineto{\pgfqpoint{4.428221in}{2.885918in}}%
\pgfpathmoveto{\pgfqpoint{4.428221in}{2.885918in}}%
\pgfpathlineto{\pgfqpoint{4.428221in}{2.885918in}}%
\pgfpathlineto{\pgfqpoint{4.428221in}{2.888867in}}%
\pgfpathlineto{\pgfqpoint{4.432762in}{2.888867in}}%
\pgfpathlineto{\pgfqpoint{4.432762in}{2.885918in}}%
\pgfpathmoveto{\pgfqpoint{4.428221in}{2.888867in}}%
\pgfpathlineto{\pgfqpoint{4.428221in}{2.888867in}}%
\pgfpathlineto{\pgfqpoint{4.428221in}{2.891816in}}%
\pgfpathlineto{\pgfqpoint{4.432762in}{2.891816in}}%
\pgfpathlineto{\pgfqpoint{4.432762in}{2.888867in}}%
\pgfpathmoveto{\pgfqpoint{4.432762in}{2.888867in}}%
\pgfpathlineto{\pgfqpoint{4.432762in}{2.888867in}}%
\pgfpathlineto{\pgfqpoint{4.432762in}{2.891816in}}%
\pgfpathlineto{\pgfqpoint{4.437303in}{2.891816in}}%
\pgfpathlineto{\pgfqpoint{4.437303in}{2.888867in}}%
\pgfpathmoveto{\pgfqpoint{4.432762in}{2.891816in}}%
\pgfpathlineto{\pgfqpoint{4.432762in}{2.891816in}}%
\pgfpathlineto{\pgfqpoint{4.432762in}{2.894765in}}%
\pgfpathlineto{\pgfqpoint{4.437303in}{2.894765in}}%
\pgfpathlineto{\pgfqpoint{4.437303in}{2.891816in}}%
\pgfpathmoveto{\pgfqpoint{4.437303in}{2.891816in}}%
\pgfpathlineto{\pgfqpoint{4.437303in}{2.891816in}}%
\pgfpathlineto{\pgfqpoint{4.437303in}{2.894765in}}%
\pgfpathlineto{\pgfqpoint{4.441844in}{2.894765in}}%
\pgfpathlineto{\pgfqpoint{4.441844in}{2.891816in}}%
\pgfpathmoveto{\pgfqpoint{4.437303in}{2.894765in}}%
\pgfpathlineto{\pgfqpoint{4.437303in}{2.894765in}}%
\pgfpathlineto{\pgfqpoint{4.437303in}{2.897714in}}%
\pgfpathlineto{\pgfqpoint{4.441844in}{2.897714in}}%
\pgfpathlineto{\pgfqpoint{4.441844in}{2.894765in}}%
\pgfpathmoveto{\pgfqpoint{4.441844in}{2.894765in}}%
\pgfpathlineto{\pgfqpoint{4.441844in}{2.894765in}}%
\pgfpathlineto{\pgfqpoint{4.441844in}{2.897714in}}%
\pgfpathlineto{\pgfqpoint{4.446385in}{2.897714in}}%
\pgfpathlineto{\pgfqpoint{4.446385in}{2.894765in}}%
\pgfpathmoveto{\pgfqpoint{4.441844in}{2.897714in}}%
\pgfpathlineto{\pgfqpoint{4.441844in}{2.897714in}}%
\pgfpathlineto{\pgfqpoint{4.441844in}{2.900663in}}%
\pgfpathlineto{\pgfqpoint{4.446385in}{2.900663in}}%
\pgfpathlineto{\pgfqpoint{4.446385in}{2.897714in}}%
\pgfpathmoveto{\pgfqpoint{4.446385in}{2.897714in}}%
\pgfpathlineto{\pgfqpoint{4.446385in}{2.897714in}}%
\pgfpathlineto{\pgfqpoint{4.446385in}{2.900663in}}%
\pgfpathlineto{\pgfqpoint{4.450925in}{2.900663in}}%
\pgfpathlineto{\pgfqpoint{4.450925in}{2.897714in}}%
\pgfpathmoveto{\pgfqpoint{4.446385in}{2.900663in}}%
\pgfpathlineto{\pgfqpoint{4.446385in}{2.900663in}}%
\pgfpathlineto{\pgfqpoint{4.446385in}{2.903612in}}%
\pgfpathlineto{\pgfqpoint{4.450925in}{2.903612in}}%
\pgfpathlineto{\pgfqpoint{4.450925in}{2.900663in}}%
\pgfpathmoveto{\pgfqpoint{4.450925in}{2.900663in}}%
\pgfpathlineto{\pgfqpoint{4.450925in}{2.900663in}}%
\pgfpathlineto{\pgfqpoint{4.450925in}{2.903612in}}%
\pgfpathlineto{\pgfqpoint{4.455466in}{2.903612in}}%
\pgfpathlineto{\pgfqpoint{4.455466in}{2.900663in}}%
\pgfpathmoveto{\pgfqpoint{4.450925in}{2.903612in}}%
\pgfpathlineto{\pgfqpoint{4.450925in}{2.903612in}}%
\pgfpathlineto{\pgfqpoint{4.450925in}{2.906562in}}%
\pgfpathlineto{\pgfqpoint{4.455466in}{2.906562in}}%
\pgfpathlineto{\pgfqpoint{4.455466in}{2.903612in}}%
\pgfpathmoveto{\pgfqpoint{4.455466in}{2.903612in}}%
\pgfpathlineto{\pgfqpoint{4.455466in}{2.903612in}}%
\pgfpathlineto{\pgfqpoint{4.455466in}{2.906562in}}%
\pgfpathlineto{\pgfqpoint{4.460007in}{2.906562in}}%
\pgfpathlineto{\pgfqpoint{4.460007in}{2.903612in}}%
\pgfpathmoveto{\pgfqpoint{4.455466in}{2.906562in}}%
\pgfpathlineto{\pgfqpoint{4.455466in}{2.906562in}}%
\pgfpathlineto{\pgfqpoint{4.455466in}{2.909511in}}%
\pgfpathlineto{\pgfqpoint{4.460007in}{2.909511in}}%
\pgfpathlineto{\pgfqpoint{4.460007in}{2.906562in}}%
\pgfpathmoveto{\pgfqpoint{4.460007in}{2.906562in}}%
\pgfpathlineto{\pgfqpoint{4.460007in}{2.906562in}}%
\pgfpathlineto{\pgfqpoint{4.460007in}{2.909511in}}%
\pgfpathlineto{\pgfqpoint{4.464548in}{2.909511in}}%
\pgfpathlineto{\pgfqpoint{4.464548in}{2.906562in}}%
\pgfpathmoveto{\pgfqpoint{4.460007in}{2.909511in}}%
\pgfpathlineto{\pgfqpoint{4.460007in}{2.909511in}}%
\pgfpathlineto{\pgfqpoint{4.460007in}{2.912460in}}%
\pgfpathlineto{\pgfqpoint{4.464548in}{2.912460in}}%
\pgfpathlineto{\pgfqpoint{4.464548in}{2.909511in}}%
\pgfpathmoveto{\pgfqpoint{4.464548in}{2.909511in}}%
\pgfpathlineto{\pgfqpoint{4.464548in}{2.909511in}}%
\pgfpathlineto{\pgfqpoint{4.464548in}{2.912460in}}%
\pgfpathlineto{\pgfqpoint{4.469089in}{2.912460in}}%
\pgfpathlineto{\pgfqpoint{4.469089in}{2.909511in}}%
\pgfpathmoveto{\pgfqpoint{4.464548in}{2.912460in}}%
\pgfpathlineto{\pgfqpoint{4.464548in}{2.912460in}}%
\pgfpathlineto{\pgfqpoint{4.464548in}{2.915409in}}%
\pgfpathlineto{\pgfqpoint{4.469089in}{2.915409in}}%
\pgfpathlineto{\pgfqpoint{4.469089in}{2.912460in}}%
\pgfpathmoveto{\pgfqpoint{4.469089in}{2.912460in}}%
\pgfpathlineto{\pgfqpoint{4.469089in}{2.912460in}}%
\pgfpathlineto{\pgfqpoint{4.469089in}{2.915409in}}%
\pgfpathlineto{\pgfqpoint{4.473630in}{2.915409in}}%
\pgfpathlineto{\pgfqpoint{4.473630in}{2.912460in}}%
\pgfpathmoveto{\pgfqpoint{4.469089in}{2.915409in}}%
\pgfpathlineto{\pgfqpoint{4.469089in}{2.915409in}}%
\pgfpathlineto{\pgfqpoint{4.469089in}{2.918358in}}%
\pgfpathlineto{\pgfqpoint{4.473630in}{2.918358in}}%
\pgfpathlineto{\pgfqpoint{4.473630in}{2.915409in}}%
\pgfpathmoveto{\pgfqpoint{4.473630in}{2.915409in}}%
\pgfpathlineto{\pgfqpoint{4.473630in}{2.915409in}}%
\pgfpathlineto{\pgfqpoint{4.473630in}{2.918358in}}%
\pgfpathlineto{\pgfqpoint{4.478171in}{2.918358in}}%
\pgfpathlineto{\pgfqpoint{4.478171in}{2.915409in}}%
\pgfpathmoveto{\pgfqpoint{4.473630in}{2.918358in}}%
\pgfpathlineto{\pgfqpoint{4.473630in}{2.918358in}}%
\pgfpathlineto{\pgfqpoint{4.473630in}{2.921307in}}%
\pgfpathlineto{\pgfqpoint{4.478171in}{2.921307in}}%
\pgfpathlineto{\pgfqpoint{4.478171in}{2.918358in}}%
\pgfpathmoveto{\pgfqpoint{4.478171in}{2.918358in}}%
\pgfpathlineto{\pgfqpoint{4.478171in}{2.918358in}}%
\pgfpathlineto{\pgfqpoint{4.478171in}{2.921307in}}%
\pgfpathlineto{\pgfqpoint{4.482712in}{2.921307in}}%
\pgfpathlineto{\pgfqpoint{4.482712in}{2.918358in}}%
\pgfpathmoveto{\pgfqpoint{4.478171in}{2.921307in}}%
\pgfpathlineto{\pgfqpoint{4.478171in}{2.921307in}}%
\pgfpathlineto{\pgfqpoint{4.478171in}{2.924256in}}%
\pgfpathlineto{\pgfqpoint{4.482712in}{2.924256in}}%
\pgfpathlineto{\pgfqpoint{4.482712in}{2.921307in}}%
\pgfpathmoveto{\pgfqpoint{4.482712in}{2.921307in}}%
\pgfpathlineto{\pgfqpoint{4.482712in}{2.921307in}}%
\pgfpathlineto{\pgfqpoint{4.482712in}{2.924256in}}%
\pgfpathlineto{\pgfqpoint{4.487253in}{2.924256in}}%
\pgfpathlineto{\pgfqpoint{4.487253in}{2.921307in}}%
\pgfpathmoveto{\pgfqpoint{4.482712in}{2.924256in}}%
\pgfpathlineto{\pgfqpoint{4.482712in}{2.924256in}}%
\pgfpathlineto{\pgfqpoint{4.482712in}{2.927205in}}%
\pgfpathlineto{\pgfqpoint{4.487253in}{2.927205in}}%
\pgfpathlineto{\pgfqpoint{4.487253in}{2.924256in}}%
\pgfpathmoveto{\pgfqpoint{4.487253in}{2.924256in}}%
\pgfpathlineto{\pgfqpoint{4.487253in}{2.924256in}}%
\pgfpathlineto{\pgfqpoint{4.487253in}{2.927205in}}%
\pgfpathlineto{\pgfqpoint{4.491794in}{2.927205in}}%
\pgfpathlineto{\pgfqpoint{4.491794in}{2.924256in}}%
\pgfpathmoveto{\pgfqpoint{4.487253in}{2.927205in}}%
\pgfpathlineto{\pgfqpoint{4.487253in}{2.927205in}}%
\pgfpathlineto{\pgfqpoint{4.487253in}{2.930154in}}%
\pgfpathlineto{\pgfqpoint{4.491794in}{2.930154in}}%
\pgfpathlineto{\pgfqpoint{4.491794in}{2.927205in}}%
\pgfpathmoveto{\pgfqpoint{4.491794in}{2.927205in}}%
\pgfpathlineto{\pgfqpoint{4.491794in}{2.927205in}}%
\pgfpathlineto{\pgfqpoint{4.491794in}{2.930154in}}%
\pgfpathlineto{\pgfqpoint{4.496335in}{2.930154in}}%
\pgfpathlineto{\pgfqpoint{4.496335in}{2.927205in}}%
\pgfpathmoveto{\pgfqpoint{4.491794in}{2.930154in}}%
\pgfpathlineto{\pgfqpoint{4.491794in}{2.930154in}}%
\pgfpathlineto{\pgfqpoint{4.491794in}{2.933104in}}%
\pgfpathlineto{\pgfqpoint{4.496335in}{2.933104in}}%
\pgfpathlineto{\pgfqpoint{4.496335in}{2.930154in}}%
\pgfpathmoveto{\pgfqpoint{4.496335in}{2.930154in}}%
\pgfpathlineto{\pgfqpoint{4.496335in}{2.930154in}}%
\pgfpathlineto{\pgfqpoint{4.496335in}{2.933104in}}%
\pgfpathlineto{\pgfqpoint{4.500876in}{2.933104in}}%
\pgfpathlineto{\pgfqpoint{4.500876in}{2.930154in}}%
\pgfpathmoveto{\pgfqpoint{4.496335in}{2.933104in}}%
\pgfpathlineto{\pgfqpoint{4.496335in}{2.933104in}}%
\pgfpathlineto{\pgfqpoint{4.496335in}{2.936053in}}%
\pgfpathlineto{\pgfqpoint{4.500876in}{2.936053in}}%
\pgfpathlineto{\pgfqpoint{4.500876in}{2.933104in}}%
\pgfpathmoveto{\pgfqpoint{4.500876in}{2.933104in}}%
\pgfpathlineto{\pgfqpoint{4.500876in}{2.933104in}}%
\pgfpathlineto{\pgfqpoint{4.500876in}{2.936053in}}%
\pgfpathlineto{\pgfqpoint{4.505416in}{2.936053in}}%
\pgfpathlineto{\pgfqpoint{4.505416in}{2.933104in}}%
\pgfpathmoveto{\pgfqpoint{4.500876in}{2.936053in}}%
\pgfpathlineto{\pgfqpoint{4.500876in}{2.936053in}}%
\pgfpathlineto{\pgfqpoint{4.500876in}{2.939002in}}%
\pgfpathlineto{\pgfqpoint{4.505416in}{2.939002in}}%
\pgfpathlineto{\pgfqpoint{4.505416in}{2.936053in}}%
\pgfpathmoveto{\pgfqpoint{4.505416in}{2.936053in}}%
\pgfpathlineto{\pgfqpoint{4.505416in}{2.936053in}}%
\pgfpathlineto{\pgfqpoint{4.505416in}{2.939002in}}%
\pgfpathlineto{\pgfqpoint{4.509957in}{2.939002in}}%
\pgfpathlineto{\pgfqpoint{4.509957in}{2.936053in}}%
\pgfpathmoveto{\pgfqpoint{4.505416in}{2.939002in}}%
\pgfpathlineto{\pgfqpoint{4.505416in}{2.939002in}}%
\pgfpathlineto{\pgfqpoint{4.505416in}{2.941951in}}%
\pgfpathlineto{\pgfqpoint{4.509957in}{2.941951in}}%
\pgfpathlineto{\pgfqpoint{4.509957in}{2.939002in}}%
\pgfpathmoveto{\pgfqpoint{4.509957in}{2.939002in}}%
\pgfpathlineto{\pgfqpoint{4.509957in}{2.939002in}}%
\pgfpathlineto{\pgfqpoint{4.509957in}{2.941951in}}%
\pgfpathlineto{\pgfqpoint{4.514498in}{2.941951in}}%
\pgfpathlineto{\pgfqpoint{4.514498in}{2.939002in}}%
\pgfpathmoveto{\pgfqpoint{4.509957in}{2.941951in}}%
\pgfpathlineto{\pgfqpoint{4.509957in}{2.941951in}}%
\pgfpathlineto{\pgfqpoint{4.509957in}{2.944900in}}%
\pgfpathlineto{\pgfqpoint{4.514498in}{2.944900in}}%
\pgfpathlineto{\pgfqpoint{4.514498in}{2.941951in}}%
\pgfpathmoveto{\pgfqpoint{4.514498in}{2.941951in}}%
\pgfpathlineto{\pgfqpoint{4.514498in}{2.941951in}}%
\pgfpathlineto{\pgfqpoint{4.514498in}{2.944900in}}%
\pgfpathlineto{\pgfqpoint{4.519039in}{2.944900in}}%
\pgfpathlineto{\pgfqpoint{4.519039in}{2.941951in}}%
\pgfpathmoveto{\pgfqpoint{4.514498in}{2.944900in}}%
\pgfpathlineto{\pgfqpoint{4.514498in}{2.944900in}}%
\pgfpathlineto{\pgfqpoint{4.514498in}{2.947849in}}%
\pgfpathlineto{\pgfqpoint{4.519039in}{2.947849in}}%
\pgfpathlineto{\pgfqpoint{4.519039in}{2.944900in}}%
\pgfpathmoveto{\pgfqpoint{4.519039in}{2.944900in}}%
\pgfpathlineto{\pgfqpoint{4.519039in}{2.944900in}}%
\pgfpathlineto{\pgfqpoint{4.519039in}{2.947849in}}%
\pgfpathlineto{\pgfqpoint{4.523580in}{2.947849in}}%
\pgfpathlineto{\pgfqpoint{4.523580in}{2.944900in}}%
\pgfpathmoveto{\pgfqpoint{4.519039in}{2.947849in}}%
\pgfpathlineto{\pgfqpoint{4.519039in}{2.947849in}}%
\pgfpathlineto{\pgfqpoint{4.519039in}{2.950798in}}%
\pgfpathlineto{\pgfqpoint{4.523580in}{2.950798in}}%
\pgfpathlineto{\pgfqpoint{4.523580in}{2.947849in}}%
\pgfpathmoveto{\pgfqpoint{4.519039in}{2.950798in}}%
\pgfpathlineto{\pgfqpoint{4.519039in}{2.950798in}}%
\pgfpathlineto{\pgfqpoint{4.519039in}{2.953747in}}%
\pgfpathlineto{\pgfqpoint{4.523580in}{2.953747in}}%
\pgfpathlineto{\pgfqpoint{4.523580in}{2.950798in}}%
\pgfpathmoveto{\pgfqpoint{4.523580in}{2.950798in}}%
\pgfpathlineto{\pgfqpoint{4.523580in}{2.950798in}}%
\pgfpathlineto{\pgfqpoint{4.523580in}{2.953747in}}%
\pgfpathlineto{\pgfqpoint{4.528121in}{2.953747in}}%
\pgfpathlineto{\pgfqpoint{4.528121in}{2.950798in}}%
\pgfpathmoveto{\pgfqpoint{4.523580in}{2.953747in}}%
\pgfpathlineto{\pgfqpoint{4.523580in}{2.953747in}}%
\pgfpathlineto{\pgfqpoint{4.523580in}{2.956697in}}%
\pgfpathlineto{\pgfqpoint{4.528121in}{2.956697in}}%
\pgfpathlineto{\pgfqpoint{4.528121in}{2.953747in}}%
\pgfpathmoveto{\pgfqpoint{4.528121in}{2.953747in}}%
\pgfpathlineto{\pgfqpoint{4.528121in}{2.953747in}}%
\pgfpathlineto{\pgfqpoint{4.528121in}{2.956697in}}%
\pgfpathlineto{\pgfqpoint{4.532662in}{2.956697in}}%
\pgfpathlineto{\pgfqpoint{4.532662in}{2.953747in}}%
\pgfpathmoveto{\pgfqpoint{4.528121in}{2.956697in}}%
\pgfpathlineto{\pgfqpoint{4.528121in}{2.956697in}}%
\pgfpathlineto{\pgfqpoint{4.528121in}{2.959646in}}%
\pgfpathlineto{\pgfqpoint{4.532662in}{2.959646in}}%
\pgfpathlineto{\pgfqpoint{4.532662in}{2.956697in}}%
\pgfpathmoveto{\pgfqpoint{4.532662in}{2.956697in}}%
\pgfpathlineto{\pgfqpoint{4.532662in}{2.956697in}}%
\pgfpathlineto{\pgfqpoint{4.532662in}{2.959646in}}%
\pgfpathlineto{\pgfqpoint{4.537203in}{2.959646in}}%
\pgfpathlineto{\pgfqpoint{4.537203in}{2.956697in}}%
\pgfpathmoveto{\pgfqpoint{4.532662in}{2.959646in}}%
\pgfpathlineto{\pgfqpoint{4.532662in}{2.959646in}}%
\pgfpathlineto{\pgfqpoint{4.532662in}{2.962595in}}%
\pgfpathlineto{\pgfqpoint{4.537203in}{2.962595in}}%
\pgfpathlineto{\pgfqpoint{4.537203in}{2.959646in}}%
\pgfpathmoveto{\pgfqpoint{4.537203in}{2.959646in}}%
\pgfpathlineto{\pgfqpoint{4.537203in}{2.959646in}}%
\pgfpathlineto{\pgfqpoint{4.537203in}{2.962595in}}%
\pgfpathlineto{\pgfqpoint{4.541745in}{2.962595in}}%
\pgfpathlineto{\pgfqpoint{4.541745in}{2.959646in}}%
\pgfpathmoveto{\pgfqpoint{4.537203in}{2.962595in}}%
\pgfpathlineto{\pgfqpoint{4.537203in}{2.962595in}}%
\pgfpathlineto{\pgfqpoint{4.537203in}{2.965544in}}%
\pgfpathlineto{\pgfqpoint{4.541745in}{2.965544in}}%
\pgfpathlineto{\pgfqpoint{4.541745in}{2.962595in}}%
\pgfpathmoveto{\pgfqpoint{4.541745in}{2.962595in}}%
\pgfpathlineto{\pgfqpoint{4.541745in}{2.962595in}}%
\pgfpathlineto{\pgfqpoint{4.541745in}{2.965544in}}%
\pgfpathlineto{\pgfqpoint{4.546286in}{2.965544in}}%
\pgfpathlineto{\pgfqpoint{4.546286in}{2.962595in}}%
\pgfpathmoveto{\pgfqpoint{4.541745in}{2.965544in}}%
\pgfpathlineto{\pgfqpoint{4.541745in}{2.965544in}}%
\pgfpathlineto{\pgfqpoint{4.541745in}{2.968494in}}%
\pgfpathlineto{\pgfqpoint{4.546286in}{2.968494in}}%
\pgfpathlineto{\pgfqpoint{4.546286in}{2.965544in}}%
\pgfpathmoveto{\pgfqpoint{4.546286in}{2.965544in}}%
\pgfpathlineto{\pgfqpoint{4.546286in}{2.965544in}}%
\pgfpathlineto{\pgfqpoint{4.546286in}{2.968494in}}%
\pgfpathlineto{\pgfqpoint{4.550827in}{2.968494in}}%
\pgfpathlineto{\pgfqpoint{4.550827in}{2.965544in}}%
\pgfpathmoveto{\pgfqpoint{4.546286in}{2.968494in}}%
\pgfpathlineto{\pgfqpoint{4.546286in}{2.968494in}}%
\pgfpathlineto{\pgfqpoint{4.546286in}{2.971443in}}%
\pgfpathlineto{\pgfqpoint{4.550827in}{2.971443in}}%
\pgfpathlineto{\pgfqpoint{4.550827in}{2.968494in}}%
\pgfpathmoveto{\pgfqpoint{4.550827in}{2.968494in}}%
\pgfpathlineto{\pgfqpoint{4.550827in}{2.968494in}}%
\pgfpathlineto{\pgfqpoint{4.550827in}{2.971443in}}%
\pgfpathlineto{\pgfqpoint{4.555368in}{2.971443in}}%
\pgfpathlineto{\pgfqpoint{4.555368in}{2.968494in}}%
\pgfpathmoveto{\pgfqpoint{4.550827in}{2.971443in}}%
\pgfpathlineto{\pgfqpoint{4.550827in}{2.971443in}}%
\pgfpathlineto{\pgfqpoint{4.550827in}{2.974392in}}%
\pgfpathlineto{\pgfqpoint{4.555368in}{2.974392in}}%
\pgfpathlineto{\pgfqpoint{4.555368in}{2.971443in}}%
\pgfpathmoveto{\pgfqpoint{4.555368in}{2.971443in}}%
\pgfpathlineto{\pgfqpoint{4.555368in}{2.971443in}}%
\pgfpathlineto{\pgfqpoint{4.555368in}{2.974392in}}%
\pgfpathlineto{\pgfqpoint{4.559910in}{2.974392in}}%
\pgfpathlineto{\pgfqpoint{4.559910in}{2.971443in}}%
\pgfpathmoveto{\pgfqpoint{4.555368in}{2.974392in}}%
\pgfpathlineto{\pgfqpoint{4.555368in}{2.974392in}}%
\pgfpathlineto{\pgfqpoint{4.555368in}{2.977341in}}%
\pgfpathlineto{\pgfqpoint{4.559910in}{2.977341in}}%
\pgfpathlineto{\pgfqpoint{4.559910in}{2.974392in}}%
\pgfpathmoveto{\pgfqpoint{4.559910in}{2.974392in}}%
\pgfpathlineto{\pgfqpoint{4.559910in}{2.974392in}}%
\pgfpathlineto{\pgfqpoint{4.559910in}{2.977341in}}%
\pgfpathlineto{\pgfqpoint{4.564451in}{2.977341in}}%
\pgfpathlineto{\pgfqpoint{4.564451in}{2.974392in}}%
\pgfpathmoveto{\pgfqpoint{4.559910in}{2.977341in}}%
\pgfpathlineto{\pgfqpoint{4.559910in}{2.977341in}}%
\pgfpathlineto{\pgfqpoint{4.559910in}{2.980291in}}%
\pgfpathlineto{\pgfqpoint{4.564451in}{2.980291in}}%
\pgfpathlineto{\pgfqpoint{4.564451in}{2.977341in}}%
\pgfpathmoveto{\pgfqpoint{4.564451in}{2.977341in}}%
\pgfpathlineto{\pgfqpoint{4.564451in}{2.977341in}}%
\pgfpathlineto{\pgfqpoint{4.564451in}{2.980291in}}%
\pgfpathlineto{\pgfqpoint{4.568992in}{2.980291in}}%
\pgfpathlineto{\pgfqpoint{4.568992in}{2.977341in}}%
\pgfpathmoveto{\pgfqpoint{4.564451in}{2.980291in}}%
\pgfpathlineto{\pgfqpoint{4.564451in}{2.980291in}}%
\pgfpathlineto{\pgfqpoint{4.564451in}{2.983240in}}%
\pgfpathlineto{\pgfqpoint{4.568992in}{2.983240in}}%
\pgfpathlineto{\pgfqpoint{4.568992in}{2.980291in}}%
\pgfpathmoveto{\pgfqpoint{4.568992in}{2.980291in}}%
\pgfpathlineto{\pgfqpoint{4.568992in}{2.980291in}}%
\pgfpathlineto{\pgfqpoint{4.568992in}{2.983240in}}%
\pgfpathlineto{\pgfqpoint{4.573533in}{2.983240in}}%
\pgfpathlineto{\pgfqpoint{4.573533in}{2.980291in}}%
\pgfpathmoveto{\pgfqpoint{4.568992in}{2.983240in}}%
\pgfpathlineto{\pgfqpoint{4.568992in}{2.983240in}}%
\pgfpathlineto{\pgfqpoint{4.568992in}{2.986189in}}%
\pgfpathlineto{\pgfqpoint{4.573533in}{2.986189in}}%
\pgfpathlineto{\pgfqpoint{4.573533in}{2.983240in}}%
\pgfpathmoveto{\pgfqpoint{4.573533in}{2.983240in}}%
\pgfpathlineto{\pgfqpoint{4.573533in}{2.983240in}}%
\pgfpathlineto{\pgfqpoint{4.573533in}{2.986189in}}%
\pgfpathlineto{\pgfqpoint{4.578074in}{2.986189in}}%
\pgfpathlineto{\pgfqpoint{4.578074in}{2.983240in}}%
\pgfpathmoveto{\pgfqpoint{4.573533in}{2.986189in}}%
\pgfpathlineto{\pgfqpoint{4.573533in}{2.986189in}}%
\pgfpathlineto{\pgfqpoint{4.573533in}{2.989138in}}%
\pgfpathlineto{\pgfqpoint{4.578074in}{2.989138in}}%
\pgfpathlineto{\pgfqpoint{4.578074in}{2.986189in}}%
\pgfpathmoveto{\pgfqpoint{4.578074in}{2.986189in}}%
\pgfpathlineto{\pgfqpoint{4.578074in}{2.986189in}}%
\pgfpathlineto{\pgfqpoint{4.578074in}{2.989138in}}%
\pgfpathlineto{\pgfqpoint{4.582616in}{2.989138in}}%
\pgfpathlineto{\pgfqpoint{4.582616in}{2.986189in}}%
\pgfpathmoveto{\pgfqpoint{4.578074in}{2.989138in}}%
\pgfpathlineto{\pgfqpoint{4.578074in}{2.989138in}}%
\pgfpathlineto{\pgfqpoint{4.578074in}{2.992088in}}%
\pgfpathlineto{\pgfqpoint{4.582616in}{2.992088in}}%
\pgfpathlineto{\pgfqpoint{4.582616in}{2.989138in}}%
\pgfpathmoveto{\pgfqpoint{4.582616in}{2.989138in}}%
\pgfpathlineto{\pgfqpoint{4.582616in}{2.989138in}}%
\pgfpathlineto{\pgfqpoint{4.582616in}{2.992088in}}%
\pgfpathlineto{\pgfqpoint{4.587157in}{2.992088in}}%
\pgfpathlineto{\pgfqpoint{4.587157in}{2.989138in}}%
\pgfpathmoveto{\pgfqpoint{4.582616in}{2.992088in}}%
\pgfpathlineto{\pgfqpoint{4.582616in}{2.992088in}}%
\pgfpathlineto{\pgfqpoint{4.582616in}{2.995037in}}%
\pgfpathlineto{\pgfqpoint{4.587157in}{2.995037in}}%
\pgfpathlineto{\pgfqpoint{4.587157in}{2.992088in}}%
\pgfpathmoveto{\pgfqpoint{4.587157in}{2.992088in}}%
\pgfpathlineto{\pgfqpoint{4.587157in}{2.992088in}}%
\pgfpathlineto{\pgfqpoint{4.587157in}{2.995037in}}%
\pgfpathlineto{\pgfqpoint{4.591698in}{2.995037in}}%
\pgfpathlineto{\pgfqpoint{4.591698in}{2.992088in}}%
\pgfpathmoveto{\pgfqpoint{4.587157in}{2.995037in}}%
\pgfpathlineto{\pgfqpoint{4.587157in}{2.995037in}}%
\pgfpathlineto{\pgfqpoint{4.587157in}{2.997986in}}%
\pgfpathlineto{\pgfqpoint{4.591698in}{2.997986in}}%
\pgfpathlineto{\pgfqpoint{4.591698in}{2.995037in}}%
\pgfpathmoveto{\pgfqpoint{4.591698in}{2.995037in}}%
\pgfpathlineto{\pgfqpoint{4.591698in}{2.995037in}}%
\pgfpathlineto{\pgfqpoint{4.591698in}{2.997986in}}%
\pgfpathlineto{\pgfqpoint{4.596239in}{2.997986in}}%
\pgfpathlineto{\pgfqpoint{4.596239in}{2.995037in}}%
\pgfpathmoveto{\pgfqpoint{4.591698in}{2.997986in}}%
\pgfpathlineto{\pgfqpoint{4.591698in}{2.997986in}}%
\pgfpathlineto{\pgfqpoint{4.591698in}{3.000935in}}%
\pgfpathlineto{\pgfqpoint{4.596239in}{3.000935in}}%
\pgfpathlineto{\pgfqpoint{4.596239in}{2.997986in}}%
\pgfpathmoveto{\pgfqpoint{4.596239in}{2.997986in}}%
\pgfpathlineto{\pgfqpoint{4.596239in}{2.997986in}}%
\pgfpathlineto{\pgfqpoint{4.596239in}{3.000935in}}%
\pgfpathlineto{\pgfqpoint{4.600780in}{3.000935in}}%
\pgfpathlineto{\pgfqpoint{4.600780in}{2.997986in}}%
\pgfpathmoveto{\pgfqpoint{4.596239in}{3.000935in}}%
\pgfpathlineto{\pgfqpoint{4.596239in}{3.000935in}}%
\pgfpathlineto{\pgfqpoint{4.596239in}{3.003885in}}%
\pgfpathlineto{\pgfqpoint{4.600780in}{3.003885in}}%
\pgfpathlineto{\pgfqpoint{4.600780in}{3.000935in}}%
\pgfpathmoveto{\pgfqpoint{4.600780in}{3.000935in}}%
\pgfpathlineto{\pgfqpoint{4.600780in}{3.000935in}}%
\pgfpathlineto{\pgfqpoint{4.600780in}{3.003885in}}%
\pgfpathlineto{\pgfqpoint{4.605322in}{3.003885in}}%
\pgfpathlineto{\pgfqpoint{4.605322in}{3.000935in}}%
\pgfpathmoveto{\pgfqpoint{4.600780in}{3.003885in}}%
\pgfpathlineto{\pgfqpoint{4.600780in}{3.003885in}}%
\pgfpathlineto{\pgfqpoint{4.600780in}{3.006834in}}%
\pgfpathlineto{\pgfqpoint{4.605322in}{3.006834in}}%
\pgfpathlineto{\pgfqpoint{4.605322in}{3.003885in}}%
\pgfpathmoveto{\pgfqpoint{4.605322in}{3.003885in}}%
\pgfpathlineto{\pgfqpoint{4.605322in}{3.003885in}}%
\pgfpathlineto{\pgfqpoint{4.605322in}{3.006834in}}%
\pgfpathlineto{\pgfqpoint{4.609863in}{3.006834in}}%
\pgfpathlineto{\pgfqpoint{4.609863in}{3.003885in}}%
\pgfpathmoveto{\pgfqpoint{4.605322in}{3.006834in}}%
\pgfpathlineto{\pgfqpoint{4.605322in}{3.006834in}}%
\pgfpathlineto{\pgfqpoint{4.605322in}{3.009783in}}%
\pgfpathlineto{\pgfqpoint{4.609863in}{3.009783in}}%
\pgfpathlineto{\pgfqpoint{4.609863in}{3.006834in}}%
\pgfpathmoveto{\pgfqpoint{4.609863in}{3.006834in}}%
\pgfpathlineto{\pgfqpoint{4.609863in}{3.006834in}}%
\pgfpathlineto{\pgfqpoint{4.609863in}{3.009783in}}%
\pgfpathlineto{\pgfqpoint{4.614404in}{3.009783in}}%
\pgfpathlineto{\pgfqpoint{4.614404in}{3.006834in}}%
\pgfpathmoveto{\pgfqpoint{4.609863in}{3.009783in}}%
\pgfpathlineto{\pgfqpoint{4.609863in}{3.009783in}}%
\pgfpathlineto{\pgfqpoint{4.609863in}{3.012733in}}%
\pgfpathlineto{\pgfqpoint{4.614404in}{3.012733in}}%
\pgfpathlineto{\pgfqpoint{4.614404in}{3.009783in}}%
\pgfpathmoveto{\pgfqpoint{4.614404in}{3.009783in}}%
\pgfpathlineto{\pgfqpoint{4.614404in}{3.009783in}}%
\pgfpathlineto{\pgfqpoint{4.614404in}{3.012733in}}%
\pgfpathlineto{\pgfqpoint{4.618945in}{3.012733in}}%
\pgfpathlineto{\pgfqpoint{4.618945in}{3.009783in}}%
\pgfpathmoveto{\pgfqpoint{4.614404in}{3.012733in}}%
\pgfpathlineto{\pgfqpoint{4.614404in}{3.012733in}}%
\pgfpathlineto{\pgfqpoint{4.614404in}{3.015682in}}%
\pgfpathlineto{\pgfqpoint{4.618945in}{3.015682in}}%
\pgfpathlineto{\pgfqpoint{4.618945in}{3.012733in}}%
\pgfpathmoveto{\pgfqpoint{4.618945in}{3.012733in}}%
\pgfpathlineto{\pgfqpoint{4.618945in}{3.012733in}}%
\pgfpathlineto{\pgfqpoint{4.618945in}{3.015682in}}%
\pgfpathlineto{\pgfqpoint{4.623487in}{3.015682in}}%
\pgfpathlineto{\pgfqpoint{4.623487in}{3.012733in}}%
\pgfpathmoveto{\pgfqpoint{4.618945in}{3.015682in}}%
\pgfpathlineto{\pgfqpoint{4.618945in}{3.015682in}}%
\pgfpathlineto{\pgfqpoint{4.618945in}{3.018631in}}%
\pgfpathlineto{\pgfqpoint{4.623487in}{3.018631in}}%
\pgfpathlineto{\pgfqpoint{4.623487in}{3.015682in}}%
\pgfpathmoveto{\pgfqpoint{4.623487in}{3.015682in}}%
\pgfpathlineto{\pgfqpoint{4.623487in}{3.015682in}}%
\pgfpathlineto{\pgfqpoint{4.623487in}{3.018631in}}%
\pgfpathlineto{\pgfqpoint{4.628028in}{3.018631in}}%
\pgfpathlineto{\pgfqpoint{4.628028in}{3.015682in}}%
\pgfpathmoveto{\pgfqpoint{4.623487in}{3.018631in}}%
\pgfpathlineto{\pgfqpoint{4.623487in}{3.018631in}}%
\pgfpathlineto{\pgfqpoint{4.623487in}{3.021580in}}%
\pgfpathlineto{\pgfqpoint{4.628028in}{3.021580in}}%
\pgfpathlineto{\pgfqpoint{4.628028in}{3.018631in}}%
\pgfpathmoveto{\pgfqpoint{4.628028in}{3.018631in}}%
\pgfpathlineto{\pgfqpoint{4.628028in}{3.018631in}}%
\pgfpathlineto{\pgfqpoint{4.628028in}{3.021580in}}%
\pgfpathlineto{\pgfqpoint{4.632569in}{3.021580in}}%
\pgfpathlineto{\pgfqpoint{4.632569in}{3.018631in}}%
\pgfpathmoveto{\pgfqpoint{4.628028in}{3.021580in}}%
\pgfpathlineto{\pgfqpoint{4.628028in}{3.021580in}}%
\pgfpathlineto{\pgfqpoint{4.628028in}{3.024530in}}%
\pgfpathlineto{\pgfqpoint{4.632569in}{3.024530in}}%
\pgfpathlineto{\pgfqpoint{4.632569in}{3.021580in}}%
\pgfpathmoveto{\pgfqpoint{4.632569in}{3.021580in}}%
\pgfpathlineto{\pgfqpoint{4.632569in}{3.021580in}}%
\pgfpathlineto{\pgfqpoint{4.632569in}{3.024530in}}%
\pgfpathlineto{\pgfqpoint{4.637110in}{3.024530in}}%
\pgfpathlineto{\pgfqpoint{4.637110in}{3.021580in}}%
\pgfpathmoveto{\pgfqpoint{4.632569in}{3.024530in}}%
\pgfpathlineto{\pgfqpoint{4.632569in}{3.024530in}}%
\pgfpathlineto{\pgfqpoint{4.632569in}{3.027479in}}%
\pgfpathlineto{\pgfqpoint{4.637110in}{3.027479in}}%
\pgfpathlineto{\pgfqpoint{4.637110in}{3.024530in}}%
\pgfpathmoveto{\pgfqpoint{4.637110in}{3.024530in}}%
\pgfpathlineto{\pgfqpoint{4.637110in}{3.024530in}}%
\pgfpathlineto{\pgfqpoint{4.637110in}{3.027479in}}%
\pgfpathlineto{\pgfqpoint{4.641651in}{3.027479in}}%
\pgfpathlineto{\pgfqpoint{4.641651in}{3.024530in}}%
\pgfpathmoveto{\pgfqpoint{4.637110in}{3.027479in}}%
\pgfpathlineto{\pgfqpoint{4.637110in}{3.027479in}}%
\pgfpathlineto{\pgfqpoint{4.637110in}{3.030428in}}%
\pgfpathlineto{\pgfqpoint{4.641651in}{3.030428in}}%
\pgfpathlineto{\pgfqpoint{4.641651in}{3.027479in}}%
\pgfpathmoveto{\pgfqpoint{4.641651in}{3.027479in}}%
\pgfpathlineto{\pgfqpoint{4.641651in}{3.027479in}}%
\pgfpathlineto{\pgfqpoint{4.641651in}{3.030428in}}%
\pgfpathlineto{\pgfqpoint{4.646193in}{3.030428in}}%
\pgfpathlineto{\pgfqpoint{4.646193in}{3.027479in}}%
\pgfpathmoveto{\pgfqpoint{4.641651in}{3.030428in}}%
\pgfpathlineto{\pgfqpoint{4.641651in}{3.030428in}}%
\pgfpathlineto{\pgfqpoint{4.641651in}{3.033377in}}%
\pgfpathlineto{\pgfqpoint{4.646193in}{3.033377in}}%
\pgfpathlineto{\pgfqpoint{4.646193in}{3.030428in}}%
\pgfpathmoveto{\pgfqpoint{4.646193in}{3.030428in}}%
\pgfpathlineto{\pgfqpoint{4.646193in}{3.030428in}}%
\pgfpathlineto{\pgfqpoint{4.646193in}{3.033377in}}%
\pgfpathlineto{\pgfqpoint{4.650734in}{3.033377in}}%
\pgfpathlineto{\pgfqpoint{4.650734in}{3.030428in}}%
\pgfpathmoveto{\pgfqpoint{4.646193in}{3.033377in}}%
\pgfpathlineto{\pgfqpoint{4.646193in}{3.033377in}}%
\pgfpathlineto{\pgfqpoint{4.646193in}{3.036327in}}%
\pgfpathlineto{\pgfqpoint{4.650734in}{3.036327in}}%
\pgfpathlineto{\pgfqpoint{4.650734in}{3.033377in}}%
\pgfpathmoveto{\pgfqpoint{4.650734in}{3.033377in}}%
\pgfpathlineto{\pgfqpoint{4.650734in}{3.033377in}}%
\pgfpathlineto{\pgfqpoint{4.650734in}{3.036327in}}%
\pgfpathlineto{\pgfqpoint{4.655275in}{3.036327in}}%
\pgfpathlineto{\pgfqpoint{4.655275in}{3.033377in}}%
\pgfpathmoveto{\pgfqpoint{4.650734in}{3.036327in}}%
\pgfpathlineto{\pgfqpoint{4.650734in}{3.036327in}}%
\pgfpathlineto{\pgfqpoint{4.650734in}{3.039276in}}%
\pgfpathlineto{\pgfqpoint{4.655275in}{3.039276in}}%
\pgfpathlineto{\pgfqpoint{4.655275in}{3.036327in}}%
\pgfpathmoveto{\pgfqpoint{4.655275in}{3.036327in}}%
\pgfpathlineto{\pgfqpoint{4.655275in}{3.036327in}}%
\pgfpathlineto{\pgfqpoint{4.655275in}{3.039276in}}%
\pgfpathlineto{\pgfqpoint{4.659816in}{3.039276in}}%
\pgfpathlineto{\pgfqpoint{4.659816in}{3.036327in}}%
\pgfpathmoveto{\pgfqpoint{4.655275in}{3.039276in}}%
\pgfpathlineto{\pgfqpoint{4.655275in}{3.039276in}}%
\pgfpathlineto{\pgfqpoint{4.655275in}{3.042225in}}%
\pgfpathlineto{\pgfqpoint{4.659816in}{3.042225in}}%
\pgfpathlineto{\pgfqpoint{4.659816in}{3.039276in}}%
\pgfpathmoveto{\pgfqpoint{4.659816in}{3.039276in}}%
\pgfpathlineto{\pgfqpoint{4.659816in}{3.039276in}}%
\pgfpathlineto{\pgfqpoint{4.659816in}{3.042225in}}%
\pgfpathlineto{\pgfqpoint{4.664357in}{3.042225in}}%
\pgfpathlineto{\pgfqpoint{4.664357in}{3.039276in}}%
\pgfpathmoveto{\pgfqpoint{4.659816in}{3.042225in}}%
\pgfpathlineto{\pgfqpoint{4.659816in}{3.042225in}}%
\pgfpathlineto{\pgfqpoint{4.659816in}{3.045174in}}%
\pgfpathlineto{\pgfqpoint{4.664357in}{3.045174in}}%
\pgfpathlineto{\pgfqpoint{4.664357in}{3.042225in}}%
\pgfpathmoveto{\pgfqpoint{4.664357in}{3.042225in}}%
\pgfpathlineto{\pgfqpoint{4.664357in}{3.042225in}}%
\pgfpathlineto{\pgfqpoint{4.664357in}{3.045174in}}%
\pgfpathlineto{\pgfqpoint{4.668899in}{3.045174in}}%
\pgfpathlineto{\pgfqpoint{4.668899in}{3.042225in}}%
\pgfpathmoveto{\pgfqpoint{4.664357in}{3.045174in}}%
\pgfpathlineto{\pgfqpoint{4.664357in}{3.045174in}}%
\pgfpathlineto{\pgfqpoint{4.664357in}{3.048124in}}%
\pgfpathlineto{\pgfqpoint{4.668899in}{3.048124in}}%
\pgfpathlineto{\pgfqpoint{4.668899in}{3.045174in}}%
\pgfpathmoveto{\pgfqpoint{4.668899in}{3.045174in}}%
\pgfpathlineto{\pgfqpoint{4.668899in}{3.045174in}}%
\pgfpathlineto{\pgfqpoint{4.668899in}{3.048124in}}%
\pgfpathlineto{\pgfqpoint{4.673440in}{3.048124in}}%
\pgfpathlineto{\pgfqpoint{4.673440in}{3.045174in}}%
\pgfpathmoveto{\pgfqpoint{4.668899in}{3.048124in}}%
\pgfpathlineto{\pgfqpoint{4.668899in}{3.048124in}}%
\pgfpathlineto{\pgfqpoint{4.668899in}{3.051073in}}%
\pgfpathlineto{\pgfqpoint{4.673440in}{3.051073in}}%
\pgfpathlineto{\pgfqpoint{4.673440in}{3.048124in}}%
\pgfpathmoveto{\pgfqpoint{4.673440in}{3.048124in}}%
\pgfpathlineto{\pgfqpoint{4.673440in}{3.048124in}}%
\pgfpathlineto{\pgfqpoint{4.673440in}{3.051073in}}%
\pgfpathlineto{\pgfqpoint{4.677981in}{3.051073in}}%
\pgfpathlineto{\pgfqpoint{4.677981in}{3.048124in}}%
\pgfpathmoveto{\pgfqpoint{4.673440in}{3.051073in}}%
\pgfpathlineto{\pgfqpoint{4.673440in}{3.051073in}}%
\pgfpathlineto{\pgfqpoint{4.673440in}{3.054022in}}%
\pgfpathlineto{\pgfqpoint{4.677981in}{3.054022in}}%
\pgfpathlineto{\pgfqpoint{4.677981in}{3.051073in}}%
\pgfpathmoveto{\pgfqpoint{4.677981in}{3.051073in}}%
\pgfpathlineto{\pgfqpoint{4.677981in}{3.051073in}}%
\pgfpathlineto{\pgfqpoint{4.677981in}{3.054022in}}%
\pgfpathlineto{\pgfqpoint{4.682522in}{3.054022in}}%
\pgfpathlineto{\pgfqpoint{4.682522in}{3.051073in}}%
\pgfpathmoveto{\pgfqpoint{4.677981in}{3.054022in}}%
\pgfpathlineto{\pgfqpoint{4.677981in}{3.054022in}}%
\pgfpathlineto{\pgfqpoint{4.677981in}{3.056971in}}%
\pgfpathlineto{\pgfqpoint{4.682522in}{3.056971in}}%
\pgfpathlineto{\pgfqpoint{4.682522in}{3.054022in}}%
\pgfpathmoveto{\pgfqpoint{4.682522in}{3.054022in}}%
\pgfpathlineto{\pgfqpoint{4.682522in}{3.054022in}}%
\pgfpathlineto{\pgfqpoint{4.682522in}{3.056971in}}%
\pgfpathlineto{\pgfqpoint{4.687063in}{3.056971in}}%
\pgfpathlineto{\pgfqpoint{4.687063in}{3.054022in}}%
\pgfpathmoveto{\pgfqpoint{4.682522in}{3.056971in}}%
\pgfpathlineto{\pgfqpoint{4.682522in}{3.056971in}}%
\pgfpathlineto{\pgfqpoint{4.682522in}{3.059921in}}%
\pgfpathlineto{\pgfqpoint{4.687063in}{3.059921in}}%
\pgfpathlineto{\pgfqpoint{4.687063in}{3.056971in}}%
\pgfpathmoveto{\pgfqpoint{4.687063in}{3.056971in}}%
\pgfpathlineto{\pgfqpoint{4.687063in}{3.056971in}}%
\pgfpathlineto{\pgfqpoint{4.687063in}{3.059921in}}%
\pgfpathlineto{\pgfqpoint{4.691604in}{3.059921in}}%
\pgfpathlineto{\pgfqpoint{4.691604in}{3.056971in}}%
\pgfpathmoveto{\pgfqpoint{4.687063in}{3.059921in}}%
\pgfpathlineto{\pgfqpoint{4.687063in}{3.059921in}}%
\pgfpathlineto{\pgfqpoint{4.687063in}{3.062870in}}%
\pgfpathlineto{\pgfqpoint{4.691604in}{3.062870in}}%
\pgfpathlineto{\pgfqpoint{4.691604in}{3.059921in}}%
\pgfpathmoveto{\pgfqpoint{4.691604in}{3.059921in}}%
\pgfpathlineto{\pgfqpoint{4.691604in}{3.059921in}}%
\pgfpathlineto{\pgfqpoint{4.691604in}{3.062870in}}%
\pgfpathlineto{\pgfqpoint{4.696144in}{3.062870in}}%
\pgfpathlineto{\pgfqpoint{4.696144in}{3.059921in}}%
\pgfpathmoveto{\pgfqpoint{4.691604in}{3.062870in}}%
\pgfpathlineto{\pgfqpoint{4.691604in}{3.062870in}}%
\pgfpathlineto{\pgfqpoint{4.691604in}{3.065819in}}%
\pgfpathlineto{\pgfqpoint{4.696144in}{3.065819in}}%
\pgfpathlineto{\pgfqpoint{4.696144in}{3.062870in}}%
\pgfpathmoveto{\pgfqpoint{4.696144in}{3.062870in}}%
\pgfpathlineto{\pgfqpoint{4.696144in}{3.062870in}}%
\pgfpathlineto{\pgfqpoint{4.696144in}{3.065819in}}%
\pgfpathlineto{\pgfqpoint{4.700685in}{3.065819in}}%
\pgfpathlineto{\pgfqpoint{4.700685in}{3.062870in}}%
\pgfpathmoveto{\pgfqpoint{4.696144in}{3.065819in}}%
\pgfpathlineto{\pgfqpoint{4.696144in}{3.065819in}}%
\pgfpathlineto{\pgfqpoint{4.696144in}{3.068768in}}%
\pgfpathlineto{\pgfqpoint{4.700685in}{3.068768in}}%
\pgfpathlineto{\pgfqpoint{4.700685in}{3.065819in}}%
\pgfpathmoveto{\pgfqpoint{4.700685in}{3.065819in}}%
\pgfpathlineto{\pgfqpoint{4.700685in}{3.065819in}}%
\pgfpathlineto{\pgfqpoint{4.700685in}{3.068768in}}%
\pgfpathlineto{\pgfqpoint{4.705226in}{3.068768in}}%
\pgfpathlineto{\pgfqpoint{4.705226in}{3.065819in}}%
\pgfpathmoveto{\pgfqpoint{4.700685in}{3.068768in}}%
\pgfpathlineto{\pgfqpoint{4.700685in}{3.068768in}}%
\pgfpathlineto{\pgfqpoint{4.700685in}{3.071718in}}%
\pgfpathlineto{\pgfqpoint{4.705226in}{3.071718in}}%
\pgfpathlineto{\pgfqpoint{4.705226in}{3.068768in}}%
\pgfpathmoveto{\pgfqpoint{4.705226in}{3.068768in}}%
\pgfpathlineto{\pgfqpoint{4.705226in}{3.068768in}}%
\pgfpathlineto{\pgfqpoint{4.705226in}{3.071718in}}%
\pgfpathlineto{\pgfqpoint{4.709767in}{3.071718in}}%
\pgfpathlineto{\pgfqpoint{4.709767in}{3.068768in}}%
\pgfpathmoveto{\pgfqpoint{4.705226in}{3.071718in}}%
\pgfpathlineto{\pgfqpoint{4.705226in}{3.071718in}}%
\pgfpathlineto{\pgfqpoint{4.705226in}{3.074667in}}%
\pgfpathlineto{\pgfqpoint{4.709767in}{3.074667in}}%
\pgfpathlineto{\pgfqpoint{4.709767in}{3.071718in}}%
\pgfpathmoveto{\pgfqpoint{4.709767in}{3.071718in}}%
\pgfpathlineto{\pgfqpoint{4.709767in}{3.071718in}}%
\pgfpathlineto{\pgfqpoint{4.709767in}{3.074667in}}%
\pgfpathlineto{\pgfqpoint{4.714308in}{3.074667in}}%
\pgfpathlineto{\pgfqpoint{4.714308in}{3.071718in}}%
\pgfpathmoveto{\pgfqpoint{4.709767in}{3.074667in}}%
\pgfpathlineto{\pgfqpoint{4.709767in}{3.074667in}}%
\pgfpathlineto{\pgfqpoint{4.709767in}{3.077616in}}%
\pgfpathlineto{\pgfqpoint{4.714308in}{3.077616in}}%
\pgfpathlineto{\pgfqpoint{4.714308in}{3.074667in}}%
\pgfpathmoveto{\pgfqpoint{4.714308in}{3.074667in}}%
\pgfpathlineto{\pgfqpoint{4.714308in}{3.074667in}}%
\pgfpathlineto{\pgfqpoint{4.714308in}{3.077616in}}%
\pgfpathlineto{\pgfqpoint{4.718849in}{3.077616in}}%
\pgfpathlineto{\pgfqpoint{4.718849in}{3.074667in}}%
\pgfpathmoveto{\pgfqpoint{4.714308in}{3.077616in}}%
\pgfpathlineto{\pgfqpoint{4.714308in}{3.077616in}}%
\pgfpathlineto{\pgfqpoint{4.714308in}{3.080565in}}%
\pgfpathlineto{\pgfqpoint{4.718849in}{3.080565in}}%
\pgfpathlineto{\pgfqpoint{4.718849in}{3.077616in}}%
\pgfpathmoveto{\pgfqpoint{4.718849in}{3.077616in}}%
\pgfpathlineto{\pgfqpoint{4.718849in}{3.077616in}}%
\pgfpathlineto{\pgfqpoint{4.718849in}{3.080565in}}%
\pgfpathlineto{\pgfqpoint{4.723390in}{3.080565in}}%
\pgfpathlineto{\pgfqpoint{4.723390in}{3.077616in}}%
\pgfpathmoveto{\pgfqpoint{4.718849in}{3.080565in}}%
\pgfpathlineto{\pgfqpoint{4.718849in}{3.080565in}}%
\pgfpathlineto{\pgfqpoint{4.718849in}{3.083515in}}%
\pgfpathlineto{\pgfqpoint{4.723390in}{3.083515in}}%
\pgfpathlineto{\pgfqpoint{4.723390in}{3.080565in}}%
\pgfpathmoveto{\pgfqpoint{4.723390in}{3.080565in}}%
\pgfpathlineto{\pgfqpoint{4.723390in}{3.080565in}}%
\pgfpathlineto{\pgfqpoint{4.723390in}{3.083515in}}%
\pgfpathlineto{\pgfqpoint{4.727931in}{3.083515in}}%
\pgfpathlineto{\pgfqpoint{4.727931in}{3.080565in}}%
\pgfpathmoveto{\pgfqpoint{4.723390in}{3.083515in}}%
\pgfpathlineto{\pgfqpoint{4.723390in}{3.083515in}}%
\pgfpathlineto{\pgfqpoint{4.723390in}{3.086464in}}%
\pgfpathlineto{\pgfqpoint{4.727931in}{3.086464in}}%
\pgfpathlineto{\pgfqpoint{4.727931in}{3.083515in}}%
\pgfpathmoveto{\pgfqpoint{4.727931in}{3.083515in}}%
\pgfpathlineto{\pgfqpoint{4.727931in}{3.083515in}}%
\pgfpathlineto{\pgfqpoint{4.727931in}{3.086464in}}%
\pgfpathlineto{\pgfqpoint{4.732472in}{3.086464in}}%
\pgfpathlineto{\pgfqpoint{4.732472in}{3.083515in}}%
\pgfpathmoveto{\pgfqpoint{4.727931in}{3.086464in}}%
\pgfpathlineto{\pgfqpoint{4.727931in}{3.086464in}}%
\pgfpathlineto{\pgfqpoint{4.727931in}{3.089413in}}%
\pgfpathlineto{\pgfqpoint{4.732472in}{3.089413in}}%
\pgfpathlineto{\pgfqpoint{4.732472in}{3.086464in}}%
\pgfpathmoveto{\pgfqpoint{4.732472in}{3.086464in}}%
\pgfpathlineto{\pgfqpoint{4.732472in}{3.086464in}}%
\pgfpathlineto{\pgfqpoint{4.732472in}{3.089413in}}%
\pgfpathlineto{\pgfqpoint{4.737013in}{3.089413in}}%
\pgfpathlineto{\pgfqpoint{4.737013in}{3.086464in}}%
\pgfpathmoveto{\pgfqpoint{4.732472in}{3.089413in}}%
\pgfpathlineto{\pgfqpoint{4.732472in}{3.089413in}}%
\pgfpathlineto{\pgfqpoint{4.732472in}{3.092362in}}%
\pgfpathlineto{\pgfqpoint{4.737013in}{3.092362in}}%
\pgfpathlineto{\pgfqpoint{4.737013in}{3.089413in}}%
\pgfpathmoveto{\pgfqpoint{4.737013in}{3.089413in}}%
\pgfpathlineto{\pgfqpoint{4.737013in}{3.089413in}}%
\pgfpathlineto{\pgfqpoint{4.737013in}{3.092362in}}%
\pgfpathlineto{\pgfqpoint{4.741554in}{3.092362in}}%
\pgfpathlineto{\pgfqpoint{4.741554in}{3.089413in}}%
\pgfpathmoveto{\pgfqpoint{4.737013in}{3.092362in}}%
\pgfpathlineto{\pgfqpoint{4.737013in}{3.092362in}}%
\pgfpathlineto{\pgfqpoint{4.737013in}{3.095312in}}%
\pgfpathlineto{\pgfqpoint{4.741554in}{3.095312in}}%
\pgfpathlineto{\pgfqpoint{4.741554in}{3.092362in}}%
\pgfpathmoveto{\pgfqpoint{4.741554in}{3.092362in}}%
\pgfpathlineto{\pgfqpoint{4.741554in}{3.092362in}}%
\pgfpathlineto{\pgfqpoint{4.741554in}{3.095312in}}%
\pgfpathlineto{\pgfqpoint{4.746094in}{3.095312in}}%
\pgfpathlineto{\pgfqpoint{4.746094in}{3.092362in}}%
\pgfpathmoveto{\pgfqpoint{4.741554in}{3.095312in}}%
\pgfpathlineto{\pgfqpoint{4.741554in}{3.095312in}}%
\pgfpathlineto{\pgfqpoint{4.741554in}{3.098261in}}%
\pgfpathlineto{\pgfqpoint{4.746094in}{3.098261in}}%
\pgfpathlineto{\pgfqpoint{4.746094in}{3.095312in}}%
\pgfpathmoveto{\pgfqpoint{4.746094in}{3.095312in}}%
\pgfpathlineto{\pgfqpoint{4.746094in}{3.095312in}}%
\pgfpathlineto{\pgfqpoint{4.746094in}{3.098261in}}%
\pgfpathlineto{\pgfqpoint{4.750635in}{3.098261in}}%
\pgfpathlineto{\pgfqpoint{4.750635in}{3.095312in}}%
\pgfpathmoveto{\pgfqpoint{4.746094in}{3.098261in}}%
\pgfpathlineto{\pgfqpoint{4.746094in}{3.098261in}}%
\pgfpathlineto{\pgfqpoint{4.746094in}{3.101210in}}%
\pgfpathlineto{\pgfqpoint{4.750635in}{3.101210in}}%
\pgfpathlineto{\pgfqpoint{4.750635in}{3.098261in}}%
\pgfpathmoveto{\pgfqpoint{4.750635in}{3.098261in}}%
\pgfpathlineto{\pgfqpoint{4.750635in}{3.098261in}}%
\pgfpathlineto{\pgfqpoint{4.750635in}{3.101210in}}%
\pgfpathlineto{\pgfqpoint{4.755176in}{3.101210in}}%
\pgfpathlineto{\pgfqpoint{4.755176in}{3.098261in}}%
\pgfpathmoveto{\pgfqpoint{4.750635in}{3.101210in}}%
\pgfpathlineto{\pgfqpoint{4.750635in}{3.101210in}}%
\pgfpathlineto{\pgfqpoint{4.750635in}{3.104159in}}%
\pgfpathlineto{\pgfqpoint{4.755176in}{3.104159in}}%
\pgfpathlineto{\pgfqpoint{4.755176in}{3.101210in}}%
\pgfpathmoveto{\pgfqpoint{4.755176in}{3.101210in}}%
\pgfpathlineto{\pgfqpoint{4.755176in}{3.101210in}}%
\pgfpathlineto{\pgfqpoint{4.755176in}{3.104159in}}%
\pgfpathlineto{\pgfqpoint{4.759717in}{3.104159in}}%
\pgfpathlineto{\pgfqpoint{4.759717in}{3.101210in}}%
\pgfpathmoveto{\pgfqpoint{4.755176in}{3.104159in}}%
\pgfpathlineto{\pgfqpoint{4.755176in}{3.104159in}}%
\pgfpathlineto{\pgfqpoint{4.755176in}{3.107109in}}%
\pgfpathlineto{\pgfqpoint{4.759717in}{3.107109in}}%
\pgfpathlineto{\pgfqpoint{4.759717in}{3.104159in}}%
\pgfpathmoveto{\pgfqpoint{4.759717in}{3.104159in}}%
\pgfpathlineto{\pgfqpoint{4.759717in}{3.104159in}}%
\pgfpathlineto{\pgfqpoint{4.759717in}{3.107109in}}%
\pgfpathlineto{\pgfqpoint{4.764258in}{3.107109in}}%
\pgfpathlineto{\pgfqpoint{4.764258in}{3.104159in}}%
\pgfpathmoveto{\pgfqpoint{4.759717in}{3.107109in}}%
\pgfpathlineto{\pgfqpoint{4.759717in}{3.107109in}}%
\pgfpathlineto{\pgfqpoint{4.759717in}{3.110058in}}%
\pgfpathlineto{\pgfqpoint{4.764258in}{3.110058in}}%
\pgfpathlineto{\pgfqpoint{4.764258in}{3.107109in}}%
\pgfpathmoveto{\pgfqpoint{4.764258in}{3.107109in}}%
\pgfpathlineto{\pgfqpoint{4.764258in}{3.107109in}}%
\pgfpathlineto{\pgfqpoint{4.764258in}{3.110058in}}%
\pgfpathlineto{\pgfqpoint{4.768799in}{3.110058in}}%
\pgfpathlineto{\pgfqpoint{4.768799in}{3.107109in}}%
\pgfpathmoveto{\pgfqpoint{4.764258in}{3.110058in}}%
\pgfpathlineto{\pgfqpoint{4.764258in}{3.110058in}}%
\pgfpathlineto{\pgfqpoint{4.764258in}{3.113007in}}%
\pgfpathlineto{\pgfqpoint{4.768799in}{3.113007in}}%
\pgfpathlineto{\pgfqpoint{4.768799in}{3.110058in}}%
\pgfpathmoveto{\pgfqpoint{4.768799in}{3.110058in}}%
\pgfpathlineto{\pgfqpoint{4.768799in}{3.110058in}}%
\pgfpathlineto{\pgfqpoint{4.768799in}{3.113007in}}%
\pgfpathlineto{\pgfqpoint{4.773340in}{3.113007in}}%
\pgfpathlineto{\pgfqpoint{4.773340in}{3.110058in}}%
\pgfpathmoveto{\pgfqpoint{4.768799in}{3.113007in}}%
\pgfpathlineto{\pgfqpoint{4.768799in}{3.113007in}}%
\pgfpathlineto{\pgfqpoint{4.768799in}{3.115956in}}%
\pgfpathlineto{\pgfqpoint{4.773340in}{3.115956in}}%
\pgfpathlineto{\pgfqpoint{4.773340in}{3.113007in}}%
\pgfpathmoveto{\pgfqpoint{4.773340in}{3.113007in}}%
\pgfpathlineto{\pgfqpoint{4.773340in}{3.113007in}}%
\pgfpathlineto{\pgfqpoint{4.773340in}{3.115956in}}%
\pgfpathlineto{\pgfqpoint{4.777881in}{3.115956in}}%
\pgfpathlineto{\pgfqpoint{4.777881in}{3.113007in}}%
\pgfpathmoveto{\pgfqpoint{4.773340in}{3.115956in}}%
\pgfpathlineto{\pgfqpoint{4.773340in}{3.115956in}}%
\pgfpathlineto{\pgfqpoint{4.773340in}{3.118906in}}%
\pgfpathlineto{\pgfqpoint{4.777881in}{3.118906in}}%
\pgfpathlineto{\pgfqpoint{4.777881in}{3.115956in}}%
\pgfpathmoveto{\pgfqpoint{4.777881in}{3.115956in}}%
\pgfpathlineto{\pgfqpoint{4.777881in}{3.115956in}}%
\pgfpathlineto{\pgfqpoint{4.777881in}{3.118906in}}%
\pgfpathlineto{\pgfqpoint{4.782422in}{3.118906in}}%
\pgfpathlineto{\pgfqpoint{4.782422in}{3.115956in}}%
\pgfpathmoveto{\pgfqpoint{4.777881in}{3.118906in}}%
\pgfpathlineto{\pgfqpoint{4.777881in}{3.118906in}}%
\pgfpathlineto{\pgfqpoint{4.777881in}{3.121855in}}%
\pgfpathlineto{\pgfqpoint{4.782422in}{3.121855in}}%
\pgfpathlineto{\pgfqpoint{4.782422in}{3.118906in}}%
\pgfpathmoveto{\pgfqpoint{4.782422in}{3.118906in}}%
\pgfpathlineto{\pgfqpoint{4.782422in}{3.118906in}}%
\pgfpathlineto{\pgfqpoint{4.782422in}{3.121855in}}%
\pgfpathlineto{\pgfqpoint{4.786963in}{3.121855in}}%
\pgfpathlineto{\pgfqpoint{4.786963in}{3.118906in}}%
\pgfpathmoveto{\pgfqpoint{4.782422in}{3.121855in}}%
\pgfpathlineto{\pgfqpoint{4.782422in}{3.121855in}}%
\pgfpathlineto{\pgfqpoint{4.782422in}{3.124804in}}%
\pgfpathlineto{\pgfqpoint{4.786963in}{3.124804in}}%
\pgfpathlineto{\pgfqpoint{4.786963in}{3.121855in}}%
\pgfpathmoveto{\pgfqpoint{4.786963in}{3.121855in}}%
\pgfpathlineto{\pgfqpoint{4.786963in}{3.121855in}}%
\pgfpathlineto{\pgfqpoint{4.786963in}{3.124804in}}%
\pgfpathlineto{\pgfqpoint{4.791504in}{3.124804in}}%
\pgfpathlineto{\pgfqpoint{4.791504in}{3.121855in}}%
\pgfpathmoveto{\pgfqpoint{4.786963in}{3.124804in}}%
\pgfpathlineto{\pgfqpoint{4.786963in}{3.124804in}}%
\pgfpathlineto{\pgfqpoint{4.786963in}{3.127753in}}%
\pgfpathlineto{\pgfqpoint{4.791504in}{3.127753in}}%
\pgfpathlineto{\pgfqpoint{4.791504in}{3.124804in}}%
\pgfpathmoveto{\pgfqpoint{4.791504in}{3.124804in}}%
\pgfpathlineto{\pgfqpoint{4.791504in}{3.124804in}}%
\pgfpathlineto{\pgfqpoint{4.791504in}{3.127753in}}%
\pgfpathlineto{\pgfqpoint{4.796044in}{3.127753in}}%
\pgfpathlineto{\pgfqpoint{4.796044in}{3.124804in}}%
\pgfpathmoveto{\pgfqpoint{4.791504in}{3.127753in}}%
\pgfpathlineto{\pgfqpoint{4.791504in}{3.127753in}}%
\pgfpathlineto{\pgfqpoint{4.791504in}{3.130703in}}%
\pgfpathlineto{\pgfqpoint{4.796044in}{3.130703in}}%
\pgfpathlineto{\pgfqpoint{4.796044in}{3.127753in}}%
\pgfpathmoveto{\pgfqpoint{4.796044in}{3.127753in}}%
\pgfpathlineto{\pgfqpoint{4.796044in}{3.127753in}}%
\pgfpathlineto{\pgfqpoint{4.796044in}{3.130703in}}%
\pgfpathlineto{\pgfqpoint{4.800585in}{3.130703in}}%
\pgfpathlineto{\pgfqpoint{4.800585in}{3.127753in}}%
\pgfpathmoveto{\pgfqpoint{4.796044in}{3.130703in}}%
\pgfpathlineto{\pgfqpoint{4.796044in}{3.130703in}}%
\pgfpathlineto{\pgfqpoint{4.796044in}{3.133652in}}%
\pgfpathlineto{\pgfqpoint{4.800585in}{3.133652in}}%
\pgfpathlineto{\pgfqpoint{4.800585in}{3.130703in}}%
\pgfpathmoveto{\pgfqpoint{4.800585in}{3.130703in}}%
\pgfpathlineto{\pgfqpoint{4.800585in}{3.130703in}}%
\pgfpathlineto{\pgfqpoint{4.800585in}{3.133652in}}%
\pgfpathlineto{\pgfqpoint{4.805126in}{3.133652in}}%
\pgfpathlineto{\pgfqpoint{4.805126in}{3.130703in}}%
\pgfpathmoveto{\pgfqpoint{4.805126in}{3.130703in}}%
\pgfpathlineto{\pgfqpoint{4.805126in}{3.130703in}}%
\pgfpathlineto{\pgfqpoint{4.805126in}{3.133652in}}%
\pgfpathlineto{\pgfqpoint{4.809667in}{3.133652in}}%
\pgfpathlineto{\pgfqpoint{4.809667in}{3.130703in}}%
\pgfpathmoveto{\pgfqpoint{4.805126in}{3.133652in}}%
\pgfpathlineto{\pgfqpoint{4.805126in}{3.133652in}}%
\pgfpathlineto{\pgfqpoint{4.805126in}{3.136601in}}%
\pgfpathlineto{\pgfqpoint{4.809667in}{3.136601in}}%
\pgfpathlineto{\pgfqpoint{4.809667in}{3.133652in}}%
\pgfpathmoveto{\pgfqpoint{4.809667in}{3.133652in}}%
\pgfpathlineto{\pgfqpoint{4.809667in}{3.133652in}}%
\pgfpathlineto{\pgfqpoint{4.809667in}{3.136601in}}%
\pgfpathlineto{\pgfqpoint{4.814208in}{3.136601in}}%
\pgfpathlineto{\pgfqpoint{4.814208in}{3.133652in}}%
\pgfpathmoveto{\pgfqpoint{4.809667in}{3.136601in}}%
\pgfpathlineto{\pgfqpoint{4.809667in}{3.136601in}}%
\pgfpathlineto{\pgfqpoint{4.809667in}{3.139550in}}%
\pgfpathlineto{\pgfqpoint{4.814208in}{3.139550in}}%
\pgfpathlineto{\pgfqpoint{4.814208in}{3.136601in}}%
\pgfpathmoveto{\pgfqpoint{4.814208in}{3.136601in}}%
\pgfpathlineto{\pgfqpoint{4.814208in}{3.136601in}}%
\pgfpathlineto{\pgfqpoint{4.814208in}{3.139550in}}%
\pgfpathlineto{\pgfqpoint{4.818749in}{3.139550in}}%
\pgfpathlineto{\pgfqpoint{4.818749in}{3.136601in}}%
\pgfpathmoveto{\pgfqpoint{4.814208in}{3.139550in}}%
\pgfpathlineto{\pgfqpoint{4.814208in}{3.139550in}}%
\pgfpathlineto{\pgfqpoint{4.814208in}{3.142500in}}%
\pgfpathlineto{\pgfqpoint{4.818749in}{3.142500in}}%
\pgfpathlineto{\pgfqpoint{4.818749in}{3.139550in}}%
\pgfpathmoveto{\pgfqpoint{4.818749in}{3.139550in}}%
\pgfpathlineto{\pgfqpoint{4.818749in}{3.139550in}}%
\pgfpathlineto{\pgfqpoint{4.818749in}{3.142500in}}%
\pgfpathlineto{\pgfqpoint{4.823290in}{3.142500in}}%
\pgfpathlineto{\pgfqpoint{4.823290in}{3.139550in}}%
\pgfpathmoveto{\pgfqpoint{4.818749in}{3.142500in}}%
\pgfpathlineto{\pgfqpoint{4.818749in}{3.142500in}}%
\pgfpathlineto{\pgfqpoint{4.818749in}{3.145449in}}%
\pgfpathlineto{\pgfqpoint{4.823290in}{3.145449in}}%
\pgfpathlineto{\pgfqpoint{4.823290in}{3.142500in}}%
\pgfpathmoveto{\pgfqpoint{4.823290in}{3.142500in}}%
\pgfpathlineto{\pgfqpoint{4.823290in}{3.142500in}}%
\pgfpathlineto{\pgfqpoint{4.823290in}{3.145449in}}%
\pgfpathlineto{\pgfqpoint{4.827831in}{3.145449in}}%
\pgfpathlineto{\pgfqpoint{4.827831in}{3.142500in}}%
\pgfpathmoveto{\pgfqpoint{4.823290in}{3.145449in}}%
\pgfpathlineto{\pgfqpoint{4.823290in}{3.145449in}}%
\pgfpathlineto{\pgfqpoint{4.823290in}{3.148398in}}%
\pgfpathlineto{\pgfqpoint{4.827831in}{3.148398in}}%
\pgfpathlineto{\pgfqpoint{4.827831in}{3.145449in}}%
\pgfpathmoveto{\pgfqpoint{4.827831in}{3.145449in}}%
\pgfpathlineto{\pgfqpoint{4.827831in}{3.145449in}}%
\pgfpathlineto{\pgfqpoint{4.827831in}{3.148398in}}%
\pgfpathlineto{\pgfqpoint{4.832372in}{3.148398in}}%
\pgfpathlineto{\pgfqpoint{4.832372in}{3.145449in}}%
\pgfpathmoveto{\pgfqpoint{4.827831in}{3.148398in}}%
\pgfpathlineto{\pgfqpoint{4.827831in}{3.148398in}}%
\pgfpathlineto{\pgfqpoint{4.827831in}{3.151347in}}%
\pgfpathlineto{\pgfqpoint{4.832372in}{3.151347in}}%
\pgfpathlineto{\pgfqpoint{4.832372in}{3.148398in}}%
\pgfpathmoveto{\pgfqpoint{4.832372in}{3.148398in}}%
\pgfpathlineto{\pgfqpoint{4.832372in}{3.148398in}}%
\pgfpathlineto{\pgfqpoint{4.832372in}{3.151347in}}%
\pgfpathlineto{\pgfqpoint{4.836913in}{3.151347in}}%
\pgfpathlineto{\pgfqpoint{4.836913in}{3.148398in}}%
\pgfpathmoveto{\pgfqpoint{4.832372in}{3.151347in}}%
\pgfpathlineto{\pgfqpoint{4.832372in}{3.151347in}}%
\pgfpathlineto{\pgfqpoint{4.832372in}{3.154296in}}%
\pgfpathlineto{\pgfqpoint{4.836913in}{3.154296in}}%
\pgfpathlineto{\pgfqpoint{4.836913in}{3.151347in}}%
\pgfpathmoveto{\pgfqpoint{4.836913in}{3.151347in}}%
\pgfpathlineto{\pgfqpoint{4.836913in}{3.151347in}}%
\pgfpathlineto{\pgfqpoint{4.836913in}{3.154296in}}%
\pgfpathlineto{\pgfqpoint{4.841454in}{3.154296in}}%
\pgfpathlineto{\pgfqpoint{4.841454in}{3.151347in}}%
\pgfpathmoveto{\pgfqpoint{4.836913in}{3.154296in}}%
\pgfpathlineto{\pgfqpoint{4.836913in}{3.154296in}}%
\pgfpathlineto{\pgfqpoint{4.836913in}{3.157246in}}%
\pgfpathlineto{\pgfqpoint{4.841454in}{3.157246in}}%
\pgfpathlineto{\pgfqpoint{4.841454in}{3.154296in}}%
\pgfpathmoveto{\pgfqpoint{4.836913in}{3.157246in}}%
\pgfpathlineto{\pgfqpoint{4.836913in}{3.157246in}}%
\pgfpathlineto{\pgfqpoint{4.836913in}{3.160195in}}%
\pgfpathlineto{\pgfqpoint{4.841454in}{3.160195in}}%
\pgfpathlineto{\pgfqpoint{4.841454in}{3.157246in}}%
\pgfpathmoveto{\pgfqpoint{4.841454in}{3.157246in}}%
\pgfpathlineto{\pgfqpoint{4.841454in}{3.157246in}}%
\pgfpathlineto{\pgfqpoint{4.841454in}{3.160195in}}%
\pgfpathlineto{\pgfqpoint{4.845996in}{3.160195in}}%
\pgfpathlineto{\pgfqpoint{4.845996in}{3.157246in}}%
\pgfpathmoveto{\pgfqpoint{4.841454in}{3.160195in}}%
\pgfpathlineto{\pgfqpoint{4.841454in}{3.160195in}}%
\pgfpathlineto{\pgfqpoint{4.841454in}{3.163144in}}%
\pgfpathlineto{\pgfqpoint{4.845996in}{3.163144in}}%
\pgfpathlineto{\pgfqpoint{4.845996in}{3.160195in}}%
\pgfpathmoveto{\pgfqpoint{4.845996in}{3.160195in}}%
\pgfpathlineto{\pgfqpoint{4.845996in}{3.160195in}}%
\pgfpathlineto{\pgfqpoint{4.845996in}{3.163144in}}%
\pgfpathlineto{\pgfqpoint{4.850537in}{3.163144in}}%
\pgfpathlineto{\pgfqpoint{4.850537in}{3.160195in}}%
\pgfpathmoveto{\pgfqpoint{4.845996in}{3.163144in}}%
\pgfpathlineto{\pgfqpoint{4.845996in}{3.163144in}}%
\pgfpathlineto{\pgfqpoint{4.845996in}{3.166093in}}%
\pgfpathlineto{\pgfqpoint{4.850537in}{3.166093in}}%
\pgfpathlineto{\pgfqpoint{4.850537in}{3.163144in}}%
\pgfpathmoveto{\pgfqpoint{4.850537in}{3.163144in}}%
\pgfpathlineto{\pgfqpoint{4.850537in}{3.163144in}}%
\pgfpathlineto{\pgfqpoint{4.850537in}{3.166093in}}%
\pgfpathlineto{\pgfqpoint{4.855078in}{3.166093in}}%
\pgfpathlineto{\pgfqpoint{4.855078in}{3.163144in}}%
\pgfpathmoveto{\pgfqpoint{4.850537in}{3.166093in}}%
\pgfpathlineto{\pgfqpoint{4.850537in}{3.166093in}}%
\pgfpathlineto{\pgfqpoint{4.850537in}{3.169042in}}%
\pgfpathlineto{\pgfqpoint{4.855078in}{3.169042in}}%
\pgfpathlineto{\pgfqpoint{4.855078in}{3.166093in}}%
\pgfpathmoveto{\pgfqpoint{4.855078in}{3.166093in}}%
\pgfpathlineto{\pgfqpoint{4.855078in}{3.166093in}}%
\pgfpathlineto{\pgfqpoint{4.855078in}{3.169042in}}%
\pgfpathlineto{\pgfqpoint{4.859619in}{3.169042in}}%
\pgfpathlineto{\pgfqpoint{4.859619in}{3.166093in}}%
\pgfpathmoveto{\pgfqpoint{4.855078in}{3.169042in}}%
\pgfpathlineto{\pgfqpoint{4.855078in}{3.169042in}}%
\pgfpathlineto{\pgfqpoint{4.855078in}{3.171992in}}%
\pgfpathlineto{\pgfqpoint{4.859619in}{3.171992in}}%
\pgfpathlineto{\pgfqpoint{4.859619in}{3.169042in}}%
\pgfpathmoveto{\pgfqpoint{4.859619in}{3.169042in}}%
\pgfpathlineto{\pgfqpoint{4.859619in}{3.169042in}}%
\pgfpathlineto{\pgfqpoint{4.859619in}{3.171992in}}%
\pgfpathlineto{\pgfqpoint{4.864160in}{3.171992in}}%
\pgfpathlineto{\pgfqpoint{4.864160in}{3.169042in}}%
\pgfpathmoveto{\pgfqpoint{4.859619in}{3.171992in}}%
\pgfpathlineto{\pgfqpoint{4.859619in}{3.171992in}}%
\pgfpathlineto{\pgfqpoint{4.859619in}{3.174941in}}%
\pgfpathlineto{\pgfqpoint{4.864160in}{3.174941in}}%
\pgfpathlineto{\pgfqpoint{4.864160in}{3.171992in}}%
\pgfpathmoveto{\pgfqpoint{4.864160in}{3.171992in}}%
\pgfpathlineto{\pgfqpoint{4.864160in}{3.171992in}}%
\pgfpathlineto{\pgfqpoint{4.864160in}{3.174941in}}%
\pgfpathlineto{\pgfqpoint{4.868701in}{3.174941in}}%
\pgfpathlineto{\pgfqpoint{4.868701in}{3.171992in}}%
\pgfpathmoveto{\pgfqpoint{4.864160in}{3.174941in}}%
\pgfpathlineto{\pgfqpoint{4.864160in}{3.174941in}}%
\pgfpathlineto{\pgfqpoint{4.864160in}{3.177890in}}%
\pgfpathlineto{\pgfqpoint{4.868701in}{3.177890in}}%
\pgfpathlineto{\pgfqpoint{4.868701in}{3.174941in}}%
\pgfpathmoveto{\pgfqpoint{4.868701in}{3.174941in}}%
\pgfpathlineto{\pgfqpoint{4.868701in}{3.174941in}}%
\pgfpathlineto{\pgfqpoint{4.868701in}{3.177890in}}%
\pgfpathlineto{\pgfqpoint{4.873242in}{3.177890in}}%
\pgfpathlineto{\pgfqpoint{4.873242in}{3.174941in}}%
\pgfpathmoveto{\pgfqpoint{4.868701in}{3.177890in}}%
\pgfpathlineto{\pgfqpoint{4.868701in}{3.177890in}}%
\pgfpathlineto{\pgfqpoint{4.868701in}{3.180839in}}%
\pgfpathlineto{\pgfqpoint{4.873242in}{3.180839in}}%
\pgfpathlineto{\pgfqpoint{4.873242in}{3.177890in}}%
\pgfpathmoveto{\pgfqpoint{4.873242in}{3.177890in}}%
\pgfpathlineto{\pgfqpoint{4.873242in}{3.177890in}}%
\pgfpathlineto{\pgfqpoint{4.873242in}{3.180839in}}%
\pgfpathlineto{\pgfqpoint{4.877783in}{3.180839in}}%
\pgfpathlineto{\pgfqpoint{4.877783in}{3.177890in}}%
\pgfpathmoveto{\pgfqpoint{4.873242in}{3.180839in}}%
\pgfpathlineto{\pgfqpoint{4.873242in}{3.180839in}}%
\pgfpathlineto{\pgfqpoint{4.873242in}{3.183788in}}%
\pgfpathlineto{\pgfqpoint{4.877783in}{3.183788in}}%
\pgfpathlineto{\pgfqpoint{4.877783in}{3.180839in}}%
\pgfpathmoveto{\pgfqpoint{4.877783in}{3.180839in}}%
\pgfpathlineto{\pgfqpoint{4.877783in}{3.180839in}}%
\pgfpathlineto{\pgfqpoint{4.877783in}{3.183788in}}%
\pgfpathlineto{\pgfqpoint{4.882324in}{3.183788in}}%
\pgfpathlineto{\pgfqpoint{4.882324in}{3.180839in}}%
\pgfpathmoveto{\pgfqpoint{4.877783in}{3.183788in}}%
\pgfpathlineto{\pgfqpoint{4.877783in}{3.183788in}}%
\pgfpathlineto{\pgfqpoint{4.877783in}{3.186738in}}%
\pgfpathlineto{\pgfqpoint{4.882324in}{3.186738in}}%
\pgfpathlineto{\pgfqpoint{4.882324in}{3.183788in}}%
\pgfpathmoveto{\pgfqpoint{4.882324in}{3.183788in}}%
\pgfpathlineto{\pgfqpoint{4.882324in}{3.183788in}}%
\pgfpathlineto{\pgfqpoint{4.882324in}{3.186738in}}%
\pgfpathlineto{\pgfqpoint{4.886865in}{3.186738in}}%
\pgfpathlineto{\pgfqpoint{4.886865in}{3.183788in}}%
\pgfpathmoveto{\pgfqpoint{4.882324in}{3.186738in}}%
\pgfpathlineto{\pgfqpoint{4.882324in}{3.186738in}}%
\pgfpathlineto{\pgfqpoint{4.882324in}{3.189687in}}%
\pgfpathlineto{\pgfqpoint{4.886865in}{3.189687in}}%
\pgfpathlineto{\pgfqpoint{4.886865in}{3.186738in}}%
\pgfpathmoveto{\pgfqpoint{4.886865in}{3.186738in}}%
\pgfpathlineto{\pgfqpoint{4.886865in}{3.186738in}}%
\pgfpathlineto{\pgfqpoint{4.886865in}{3.189687in}}%
\pgfpathlineto{\pgfqpoint{4.891406in}{3.189687in}}%
\pgfpathlineto{\pgfqpoint{4.891406in}{3.186738in}}%
\pgfpathmoveto{\pgfqpoint{4.886865in}{3.189687in}}%
\pgfpathlineto{\pgfqpoint{4.886865in}{3.189687in}}%
\pgfpathlineto{\pgfqpoint{4.886865in}{3.192636in}}%
\pgfpathlineto{\pgfqpoint{4.891406in}{3.192636in}}%
\pgfpathlineto{\pgfqpoint{4.891406in}{3.189687in}}%
\pgfpathmoveto{\pgfqpoint{4.891406in}{3.189687in}}%
\pgfpathlineto{\pgfqpoint{4.891406in}{3.189687in}}%
\pgfpathlineto{\pgfqpoint{4.891406in}{3.192636in}}%
\pgfpathlineto{\pgfqpoint{4.895948in}{3.192636in}}%
\pgfpathlineto{\pgfqpoint{4.895948in}{3.189687in}}%
\pgfpathmoveto{\pgfqpoint{4.891406in}{3.192636in}}%
\pgfpathlineto{\pgfqpoint{4.891406in}{3.192636in}}%
\pgfpathlineto{\pgfqpoint{4.891406in}{3.195585in}}%
\pgfpathlineto{\pgfqpoint{4.895948in}{3.195585in}}%
\pgfpathlineto{\pgfqpoint{4.895948in}{3.192636in}}%
\pgfpathmoveto{\pgfqpoint{4.895948in}{3.192636in}}%
\pgfpathlineto{\pgfqpoint{4.895948in}{3.192636in}}%
\pgfpathlineto{\pgfqpoint{4.895948in}{3.195585in}}%
\pgfpathlineto{\pgfqpoint{4.900489in}{3.195585in}}%
\pgfpathlineto{\pgfqpoint{4.900489in}{3.192636in}}%
\pgfpathmoveto{\pgfqpoint{4.895948in}{3.195585in}}%
\pgfpathlineto{\pgfqpoint{4.895948in}{3.195585in}}%
\pgfpathlineto{\pgfqpoint{4.895948in}{3.198534in}}%
\pgfpathlineto{\pgfqpoint{4.900489in}{3.198534in}}%
\pgfpathlineto{\pgfqpoint{4.900489in}{3.195585in}}%
\pgfpathmoveto{\pgfqpoint{4.900489in}{3.195585in}}%
\pgfpathlineto{\pgfqpoint{4.900489in}{3.195585in}}%
\pgfpathlineto{\pgfqpoint{4.900489in}{3.198534in}}%
\pgfpathlineto{\pgfqpoint{4.905030in}{3.198534in}}%
\pgfpathlineto{\pgfqpoint{4.905030in}{3.195585in}}%
\pgfpathmoveto{\pgfqpoint{4.900489in}{3.198534in}}%
\pgfpathlineto{\pgfqpoint{4.900489in}{3.198534in}}%
\pgfpathlineto{\pgfqpoint{4.900489in}{3.201484in}}%
\pgfpathlineto{\pgfqpoint{4.905030in}{3.201484in}}%
\pgfpathlineto{\pgfqpoint{4.905030in}{3.198534in}}%
\pgfpathmoveto{\pgfqpoint{4.905030in}{3.198534in}}%
\pgfpathlineto{\pgfqpoint{4.905030in}{3.198534in}}%
\pgfpathlineto{\pgfqpoint{4.905030in}{3.201484in}}%
\pgfpathlineto{\pgfqpoint{4.909571in}{3.201484in}}%
\pgfpathlineto{\pgfqpoint{4.909571in}{3.198534in}}%
\pgfpathmoveto{\pgfqpoint{4.905030in}{3.201484in}}%
\pgfpathlineto{\pgfqpoint{4.905030in}{3.201484in}}%
\pgfpathlineto{\pgfqpoint{4.905030in}{3.204433in}}%
\pgfpathlineto{\pgfqpoint{4.909571in}{3.204433in}}%
\pgfpathlineto{\pgfqpoint{4.909571in}{3.201484in}}%
\pgfpathmoveto{\pgfqpoint{4.909571in}{3.201484in}}%
\pgfpathlineto{\pgfqpoint{4.909571in}{3.201484in}}%
\pgfpathlineto{\pgfqpoint{4.909571in}{3.204433in}}%
\pgfpathlineto{\pgfqpoint{4.914112in}{3.204433in}}%
\pgfpathlineto{\pgfqpoint{4.914112in}{3.201484in}}%
\pgfpathmoveto{\pgfqpoint{4.909571in}{3.204433in}}%
\pgfpathlineto{\pgfqpoint{4.909571in}{3.204433in}}%
\pgfpathlineto{\pgfqpoint{4.909571in}{3.207382in}}%
\pgfpathlineto{\pgfqpoint{4.914112in}{3.207382in}}%
\pgfpathlineto{\pgfqpoint{4.914112in}{3.204433in}}%
\pgfpathmoveto{\pgfqpoint{4.914112in}{3.204433in}}%
\pgfpathlineto{\pgfqpoint{4.914112in}{3.204433in}}%
\pgfpathlineto{\pgfqpoint{4.914112in}{3.207382in}}%
\pgfpathlineto{\pgfqpoint{4.918653in}{3.207382in}}%
\pgfpathlineto{\pgfqpoint{4.918653in}{3.204433in}}%
\pgfpathmoveto{\pgfqpoint{4.914112in}{3.207382in}}%
\pgfpathlineto{\pgfqpoint{4.914112in}{3.207382in}}%
\pgfpathlineto{\pgfqpoint{4.914112in}{3.210331in}}%
\pgfpathlineto{\pgfqpoint{4.918653in}{3.210331in}}%
\pgfpathlineto{\pgfqpoint{4.918653in}{3.207382in}}%
\pgfpathmoveto{\pgfqpoint{4.918653in}{3.207382in}}%
\pgfpathlineto{\pgfqpoint{4.918653in}{3.207382in}}%
\pgfpathlineto{\pgfqpoint{4.918653in}{3.210331in}}%
\pgfpathlineto{\pgfqpoint{4.923194in}{3.210331in}}%
\pgfpathlineto{\pgfqpoint{4.923194in}{3.207382in}}%
\pgfpathmoveto{\pgfqpoint{4.918653in}{3.210331in}}%
\pgfpathlineto{\pgfqpoint{4.918653in}{3.210331in}}%
\pgfpathlineto{\pgfqpoint{4.918653in}{3.213280in}}%
\pgfpathlineto{\pgfqpoint{4.923194in}{3.213280in}}%
\pgfpathlineto{\pgfqpoint{4.923194in}{3.210331in}}%
\pgfpathmoveto{\pgfqpoint{4.923194in}{3.210331in}}%
\pgfpathlineto{\pgfqpoint{4.923194in}{3.210331in}}%
\pgfpathlineto{\pgfqpoint{4.923194in}{3.213280in}}%
\pgfpathlineto{\pgfqpoint{4.927735in}{3.213280in}}%
\pgfpathlineto{\pgfqpoint{4.927735in}{3.210331in}}%
\pgfpathmoveto{\pgfqpoint{4.923194in}{3.213280in}}%
\pgfpathlineto{\pgfqpoint{4.923194in}{3.213280in}}%
\pgfpathlineto{\pgfqpoint{4.923194in}{3.216229in}}%
\pgfpathlineto{\pgfqpoint{4.927735in}{3.216229in}}%
\pgfpathlineto{\pgfqpoint{4.927735in}{3.213280in}}%
\pgfpathmoveto{\pgfqpoint{4.927735in}{3.213280in}}%
\pgfpathlineto{\pgfqpoint{4.927735in}{3.213280in}}%
\pgfpathlineto{\pgfqpoint{4.927735in}{3.216229in}}%
\pgfpathlineto{\pgfqpoint{4.932276in}{3.216229in}}%
\pgfpathlineto{\pgfqpoint{4.932276in}{3.213280in}}%
\pgfpathmoveto{\pgfqpoint{4.927735in}{3.216229in}}%
\pgfpathlineto{\pgfqpoint{4.927735in}{3.216229in}}%
\pgfpathlineto{\pgfqpoint{4.927735in}{3.219179in}}%
\pgfpathlineto{\pgfqpoint{4.932276in}{3.219179in}}%
\pgfpathlineto{\pgfqpoint{4.932276in}{3.216229in}}%
\pgfpathmoveto{\pgfqpoint{4.932276in}{3.216229in}}%
\pgfpathlineto{\pgfqpoint{4.932276in}{3.216229in}}%
\pgfpathlineto{\pgfqpoint{4.932276in}{3.219179in}}%
\pgfpathlineto{\pgfqpoint{4.936817in}{3.219179in}}%
\pgfpathlineto{\pgfqpoint{4.936817in}{3.216229in}}%
\pgfpathmoveto{\pgfqpoint{4.932276in}{3.219179in}}%
\pgfpathlineto{\pgfqpoint{4.932276in}{3.219179in}}%
\pgfpathlineto{\pgfqpoint{4.932276in}{3.222128in}}%
\pgfpathlineto{\pgfqpoint{4.936817in}{3.222128in}}%
\pgfpathlineto{\pgfqpoint{4.936817in}{3.219179in}}%
\pgfpathmoveto{\pgfqpoint{4.936817in}{3.219179in}}%
\pgfpathlineto{\pgfqpoint{4.936817in}{3.219179in}}%
\pgfpathlineto{\pgfqpoint{4.936817in}{3.222128in}}%
\pgfpathlineto{\pgfqpoint{4.941358in}{3.222128in}}%
\pgfpathlineto{\pgfqpoint{4.941358in}{3.219179in}}%
\pgfpathmoveto{\pgfqpoint{4.936817in}{3.222128in}}%
\pgfpathlineto{\pgfqpoint{4.936817in}{3.222128in}}%
\pgfpathlineto{\pgfqpoint{4.936817in}{3.225077in}}%
\pgfpathlineto{\pgfqpoint{4.941358in}{3.225077in}}%
\pgfpathlineto{\pgfqpoint{4.941358in}{3.222128in}}%
\pgfpathmoveto{\pgfqpoint{4.941358in}{3.222128in}}%
\pgfpathlineto{\pgfqpoint{4.941358in}{3.222128in}}%
\pgfpathlineto{\pgfqpoint{4.941358in}{3.225077in}}%
\pgfpathlineto{\pgfqpoint{4.945899in}{3.225077in}}%
\pgfpathlineto{\pgfqpoint{4.945899in}{3.222128in}}%
\pgfpathmoveto{\pgfqpoint{4.941358in}{3.225077in}}%
\pgfpathlineto{\pgfqpoint{4.941358in}{3.225077in}}%
\pgfpathlineto{\pgfqpoint{4.941358in}{3.228026in}}%
\pgfpathlineto{\pgfqpoint{4.945899in}{3.228026in}}%
\pgfpathlineto{\pgfqpoint{4.945899in}{3.225077in}}%
\pgfpathmoveto{\pgfqpoint{4.945899in}{3.225077in}}%
\pgfpathlineto{\pgfqpoint{4.945899in}{3.225077in}}%
\pgfpathlineto{\pgfqpoint{4.945899in}{3.228026in}}%
\pgfpathlineto{\pgfqpoint{4.950441in}{3.228026in}}%
\pgfpathlineto{\pgfqpoint{4.950441in}{3.225077in}}%
\pgfpathmoveto{\pgfqpoint{4.945899in}{3.228026in}}%
\pgfpathlineto{\pgfqpoint{4.945899in}{3.228026in}}%
\pgfpathlineto{\pgfqpoint{4.945899in}{3.230975in}}%
\pgfpathlineto{\pgfqpoint{4.950441in}{3.230975in}}%
\pgfpathlineto{\pgfqpoint{4.950441in}{3.228026in}}%
\pgfpathmoveto{\pgfqpoint{4.950441in}{3.228026in}}%
\pgfpathlineto{\pgfqpoint{4.950441in}{3.228026in}}%
\pgfpathlineto{\pgfqpoint{4.950441in}{3.230975in}}%
\pgfpathlineto{\pgfqpoint{4.954982in}{3.230975in}}%
\pgfpathlineto{\pgfqpoint{4.954982in}{3.228026in}}%
\pgfpathmoveto{\pgfqpoint{4.950441in}{3.230975in}}%
\pgfpathlineto{\pgfqpoint{4.950441in}{3.230975in}}%
\pgfpathlineto{\pgfqpoint{4.950441in}{3.233925in}}%
\pgfpathlineto{\pgfqpoint{4.954982in}{3.233925in}}%
\pgfpathlineto{\pgfqpoint{4.954982in}{3.230975in}}%
\pgfpathmoveto{\pgfqpoint{4.954982in}{3.230975in}}%
\pgfpathlineto{\pgfqpoint{4.954982in}{3.230975in}}%
\pgfpathlineto{\pgfqpoint{4.954982in}{3.233925in}}%
\pgfpathlineto{\pgfqpoint{4.959523in}{3.233925in}}%
\pgfpathlineto{\pgfqpoint{4.959523in}{3.230975in}}%
\pgfpathmoveto{\pgfqpoint{4.954982in}{3.233925in}}%
\pgfpathlineto{\pgfqpoint{4.954982in}{3.233925in}}%
\pgfpathlineto{\pgfqpoint{4.954982in}{3.236874in}}%
\pgfpathlineto{\pgfqpoint{4.959523in}{3.236874in}}%
\pgfpathlineto{\pgfqpoint{4.959523in}{3.233925in}}%
\pgfpathmoveto{\pgfqpoint{4.959523in}{3.233925in}}%
\pgfpathlineto{\pgfqpoint{4.959523in}{3.233925in}}%
\pgfpathlineto{\pgfqpoint{4.959523in}{3.236874in}}%
\pgfpathlineto{\pgfqpoint{4.964064in}{3.236874in}}%
\pgfpathlineto{\pgfqpoint{4.964064in}{3.233925in}}%
\pgfpathmoveto{\pgfqpoint{4.959523in}{3.236874in}}%
\pgfpathlineto{\pgfqpoint{4.959523in}{3.236874in}}%
\pgfpathlineto{\pgfqpoint{4.959523in}{3.239823in}}%
\pgfpathlineto{\pgfqpoint{4.964064in}{3.239823in}}%
\pgfpathlineto{\pgfqpoint{4.964064in}{3.236874in}}%
\pgfpathmoveto{\pgfqpoint{4.964064in}{3.236874in}}%
\pgfpathlineto{\pgfqpoint{4.964064in}{3.236874in}}%
\pgfpathlineto{\pgfqpoint{4.964064in}{3.239823in}}%
\pgfpathlineto{\pgfqpoint{4.968605in}{3.239823in}}%
\pgfpathlineto{\pgfqpoint{4.968605in}{3.236874in}}%
\pgfpathmoveto{\pgfqpoint{4.964064in}{3.239823in}}%
\pgfpathlineto{\pgfqpoint{4.964064in}{3.239823in}}%
\pgfpathlineto{\pgfqpoint{4.964064in}{3.242772in}}%
\pgfpathlineto{\pgfqpoint{4.968605in}{3.242772in}}%
\pgfpathlineto{\pgfqpoint{4.968605in}{3.239823in}}%
\pgfpathmoveto{\pgfqpoint{4.968605in}{3.239823in}}%
\pgfpathlineto{\pgfqpoint{4.968605in}{3.239823in}}%
\pgfpathlineto{\pgfqpoint{4.968605in}{3.242772in}}%
\pgfpathlineto{\pgfqpoint{4.973146in}{3.242772in}}%
\pgfpathlineto{\pgfqpoint{4.973146in}{3.239823in}}%
\pgfpathmoveto{\pgfqpoint{4.968605in}{3.242772in}}%
\pgfpathlineto{\pgfqpoint{4.968605in}{3.242772in}}%
\pgfpathlineto{\pgfqpoint{4.968605in}{3.245722in}}%
\pgfpathlineto{\pgfqpoint{4.973146in}{3.245722in}}%
\pgfpathlineto{\pgfqpoint{4.973146in}{3.242772in}}%
\pgfpathmoveto{\pgfqpoint{4.973146in}{3.242772in}}%
\pgfpathlineto{\pgfqpoint{4.973146in}{3.242772in}}%
\pgfpathlineto{\pgfqpoint{4.973146in}{3.245722in}}%
\pgfpathlineto{\pgfqpoint{4.977686in}{3.245722in}}%
\pgfpathlineto{\pgfqpoint{4.977686in}{3.242772in}}%
\pgfpathmoveto{\pgfqpoint{4.973146in}{3.245722in}}%
\pgfpathlineto{\pgfqpoint{4.973146in}{3.245722in}}%
\pgfpathlineto{\pgfqpoint{4.973146in}{3.248671in}}%
\pgfpathlineto{\pgfqpoint{4.977686in}{3.248671in}}%
\pgfpathlineto{\pgfqpoint{4.977686in}{3.245722in}}%
\pgfpathmoveto{\pgfqpoint{4.977686in}{3.245722in}}%
\pgfpathlineto{\pgfqpoint{4.977686in}{3.245722in}}%
\pgfpathlineto{\pgfqpoint{4.977686in}{3.248671in}}%
\pgfpathlineto{\pgfqpoint{4.982227in}{3.248671in}}%
\pgfpathlineto{\pgfqpoint{4.982227in}{3.245722in}}%
\pgfpathmoveto{\pgfqpoint{4.977686in}{3.248671in}}%
\pgfpathlineto{\pgfqpoint{4.977686in}{3.248671in}}%
\pgfpathlineto{\pgfqpoint{4.977686in}{3.251620in}}%
\pgfpathlineto{\pgfqpoint{4.982227in}{3.251620in}}%
\pgfpathlineto{\pgfqpoint{4.982227in}{3.248671in}}%
\pgfpathmoveto{\pgfqpoint{4.982227in}{3.248671in}}%
\pgfpathlineto{\pgfqpoint{4.982227in}{3.248671in}}%
\pgfpathlineto{\pgfqpoint{4.982227in}{3.251620in}}%
\pgfpathlineto{\pgfqpoint{4.986768in}{3.251620in}}%
\pgfpathlineto{\pgfqpoint{4.986768in}{3.248671in}}%
\pgfpathmoveto{\pgfqpoint{4.982227in}{3.251620in}}%
\pgfpathlineto{\pgfqpoint{4.982227in}{3.251620in}}%
\pgfpathlineto{\pgfqpoint{4.982227in}{3.254569in}}%
\pgfpathlineto{\pgfqpoint{4.986768in}{3.254569in}}%
\pgfpathlineto{\pgfqpoint{4.986768in}{3.251620in}}%
\pgfpathmoveto{\pgfqpoint{4.986768in}{3.251620in}}%
\pgfpathlineto{\pgfqpoint{4.986768in}{3.251620in}}%
\pgfpathlineto{\pgfqpoint{4.986768in}{3.254569in}}%
\pgfpathlineto{\pgfqpoint{4.991309in}{3.254569in}}%
\pgfpathlineto{\pgfqpoint{4.991309in}{3.251620in}}%
\pgfpathmoveto{\pgfqpoint{4.986768in}{3.254569in}}%
\pgfpathlineto{\pgfqpoint{4.986768in}{3.254569in}}%
\pgfpathlineto{\pgfqpoint{4.986768in}{3.257519in}}%
\pgfpathlineto{\pgfqpoint{4.991309in}{3.257519in}}%
\pgfpathlineto{\pgfqpoint{4.991309in}{3.254569in}}%
\pgfpathmoveto{\pgfqpoint{4.991309in}{3.254569in}}%
\pgfpathlineto{\pgfqpoint{4.991309in}{3.254569in}}%
\pgfpathlineto{\pgfqpoint{4.991309in}{3.257519in}}%
\pgfpathlineto{\pgfqpoint{4.995850in}{3.257519in}}%
\pgfpathlineto{\pgfqpoint{4.995850in}{3.254569in}}%
\pgfpathmoveto{\pgfqpoint{4.991309in}{3.257519in}}%
\pgfpathlineto{\pgfqpoint{4.991309in}{3.257519in}}%
\pgfpathlineto{\pgfqpoint{4.991309in}{3.260468in}}%
\pgfpathlineto{\pgfqpoint{4.995850in}{3.260468in}}%
\pgfpathlineto{\pgfqpoint{4.995850in}{3.257519in}}%
\pgfpathmoveto{\pgfqpoint{4.995850in}{3.257519in}}%
\pgfpathlineto{\pgfqpoint{4.995850in}{3.257519in}}%
\pgfpathlineto{\pgfqpoint{4.995850in}{3.260468in}}%
\pgfpathlineto{\pgfqpoint{5.000391in}{3.260468in}}%
\pgfpathlineto{\pgfqpoint{5.000391in}{3.257519in}}%
\pgfpathmoveto{\pgfqpoint{4.995850in}{3.260468in}}%
\pgfpathlineto{\pgfqpoint{4.995850in}{3.260468in}}%
\pgfpathlineto{\pgfqpoint{4.995850in}{3.263417in}}%
\pgfpathlineto{\pgfqpoint{5.000391in}{3.263417in}}%
\pgfpathlineto{\pgfqpoint{5.000391in}{3.260468in}}%
\pgfpathmoveto{\pgfqpoint{5.000391in}{3.260468in}}%
\pgfpathlineto{\pgfqpoint{5.000391in}{3.260468in}}%
\pgfpathlineto{\pgfqpoint{5.000391in}{3.263417in}}%
\pgfpathlineto{\pgfqpoint{5.004932in}{3.263417in}}%
\pgfpathlineto{\pgfqpoint{5.004932in}{3.260468in}}%
\pgfpathmoveto{\pgfqpoint{5.000391in}{3.263417in}}%
\pgfpathlineto{\pgfqpoint{5.000391in}{3.263417in}}%
\pgfpathlineto{\pgfqpoint{5.000391in}{3.266366in}}%
\pgfpathlineto{\pgfqpoint{5.004932in}{3.266366in}}%
\pgfpathlineto{\pgfqpoint{5.004932in}{3.263417in}}%
\pgfpathmoveto{\pgfqpoint{5.004932in}{3.263417in}}%
\pgfpathlineto{\pgfqpoint{5.004932in}{3.263417in}}%
\pgfpathlineto{\pgfqpoint{5.004932in}{3.266366in}}%
\pgfpathlineto{\pgfqpoint{5.009473in}{3.266366in}}%
\pgfpathlineto{\pgfqpoint{5.009473in}{3.263417in}}%
\pgfpathmoveto{\pgfqpoint{5.004932in}{3.266366in}}%
\pgfpathlineto{\pgfqpoint{5.004932in}{3.266366in}}%
\pgfpathlineto{\pgfqpoint{5.004932in}{3.269316in}}%
\pgfpathlineto{\pgfqpoint{5.009473in}{3.269316in}}%
\pgfpathlineto{\pgfqpoint{5.009473in}{3.266366in}}%
\pgfpathmoveto{\pgfqpoint{5.009473in}{3.266366in}}%
\pgfpathlineto{\pgfqpoint{5.009473in}{3.266366in}}%
\pgfpathlineto{\pgfqpoint{5.009473in}{3.269316in}}%
\pgfpathlineto{\pgfqpoint{5.014014in}{3.269316in}}%
\pgfpathlineto{\pgfqpoint{5.014014in}{3.266366in}}%
\pgfpathmoveto{\pgfqpoint{5.009473in}{3.269316in}}%
\pgfpathlineto{\pgfqpoint{5.009473in}{3.269316in}}%
\pgfpathlineto{\pgfqpoint{5.009473in}{3.272265in}}%
\pgfpathlineto{\pgfqpoint{5.014014in}{3.272265in}}%
\pgfpathlineto{\pgfqpoint{5.014014in}{3.269316in}}%
\pgfpathmoveto{\pgfqpoint{5.014014in}{3.269316in}}%
\pgfpathlineto{\pgfqpoint{5.014014in}{3.269316in}}%
\pgfpathlineto{\pgfqpoint{5.014014in}{3.272265in}}%
\pgfpathlineto{\pgfqpoint{5.018554in}{3.272265in}}%
\pgfpathlineto{\pgfqpoint{5.018554in}{3.269316in}}%
\pgfpathmoveto{\pgfqpoint{5.014014in}{3.272265in}}%
\pgfpathlineto{\pgfqpoint{5.014014in}{3.272265in}}%
\pgfpathlineto{\pgfqpoint{5.014014in}{3.275214in}}%
\pgfpathlineto{\pgfqpoint{5.018554in}{3.275214in}}%
\pgfpathlineto{\pgfqpoint{5.018554in}{3.272265in}}%
\pgfpathmoveto{\pgfqpoint{5.018554in}{3.272265in}}%
\pgfpathlineto{\pgfqpoint{5.018554in}{3.272265in}}%
\pgfpathlineto{\pgfqpoint{5.018554in}{3.275214in}}%
\pgfpathlineto{\pgfqpoint{5.023095in}{3.275214in}}%
\pgfpathlineto{\pgfqpoint{5.023095in}{3.272265in}}%
\pgfpathmoveto{\pgfqpoint{5.018554in}{3.275214in}}%
\pgfpathlineto{\pgfqpoint{5.018554in}{3.275214in}}%
\pgfpathlineto{\pgfqpoint{5.018554in}{3.278163in}}%
\pgfpathlineto{\pgfqpoint{5.023095in}{3.278163in}}%
\pgfpathlineto{\pgfqpoint{5.023095in}{3.275214in}}%
\pgfpathmoveto{\pgfqpoint{5.023095in}{3.275214in}}%
\pgfpathlineto{\pgfqpoint{5.023095in}{3.275214in}}%
\pgfpathlineto{\pgfqpoint{5.023095in}{3.278163in}}%
\pgfpathlineto{\pgfqpoint{5.027636in}{3.278163in}}%
\pgfpathlineto{\pgfqpoint{5.027636in}{3.275214in}}%
\pgfpathmoveto{\pgfqpoint{5.023095in}{3.278163in}}%
\pgfpathlineto{\pgfqpoint{5.023095in}{3.278163in}}%
\pgfpathlineto{\pgfqpoint{5.023095in}{3.281113in}}%
\pgfpathlineto{\pgfqpoint{5.027636in}{3.281113in}}%
\pgfpathlineto{\pgfqpoint{5.027636in}{3.278163in}}%
\pgfpathmoveto{\pgfqpoint{5.027636in}{3.278163in}}%
\pgfpathlineto{\pgfqpoint{5.027636in}{3.278163in}}%
\pgfpathlineto{\pgfqpoint{5.027636in}{3.281113in}}%
\pgfpathlineto{\pgfqpoint{5.032177in}{3.281113in}}%
\pgfpathlineto{\pgfqpoint{5.032177in}{3.278163in}}%
\pgfpathmoveto{\pgfqpoint{5.027636in}{3.281113in}}%
\pgfpathlineto{\pgfqpoint{5.027636in}{3.281113in}}%
\pgfpathlineto{\pgfqpoint{5.027636in}{3.284062in}}%
\pgfpathlineto{\pgfqpoint{5.032177in}{3.284062in}}%
\pgfpathlineto{\pgfqpoint{5.032177in}{3.281113in}}%
\pgfpathmoveto{\pgfqpoint{5.032177in}{3.281113in}}%
\pgfpathlineto{\pgfqpoint{5.032177in}{3.281113in}}%
\pgfpathlineto{\pgfqpoint{5.032177in}{3.284062in}}%
\pgfpathlineto{\pgfqpoint{5.036718in}{3.284062in}}%
\pgfpathlineto{\pgfqpoint{5.036718in}{3.281113in}}%
\pgfpathmoveto{\pgfqpoint{5.032177in}{3.284062in}}%
\pgfpathlineto{\pgfqpoint{5.032177in}{3.284062in}}%
\pgfpathlineto{\pgfqpoint{5.032177in}{3.287011in}}%
\pgfpathlineto{\pgfqpoint{5.036718in}{3.287011in}}%
\pgfpathlineto{\pgfqpoint{5.036718in}{3.284062in}}%
\pgfpathmoveto{\pgfqpoint{5.036718in}{3.284062in}}%
\pgfpathlineto{\pgfqpoint{5.036718in}{3.284062in}}%
\pgfpathlineto{\pgfqpoint{5.036718in}{3.287011in}}%
\pgfpathlineto{\pgfqpoint{5.041259in}{3.287011in}}%
\pgfpathlineto{\pgfqpoint{5.041259in}{3.284062in}}%
\pgfpathmoveto{\pgfqpoint{5.041259in}{3.284062in}}%
\pgfpathlineto{\pgfqpoint{5.041259in}{3.284062in}}%
\pgfpathlineto{\pgfqpoint{5.041259in}{3.287011in}}%
\pgfpathlineto{\pgfqpoint{5.045800in}{3.287011in}}%
\pgfpathlineto{\pgfqpoint{5.045800in}{3.284062in}}%
\pgfpathmoveto{\pgfqpoint{5.041259in}{3.287011in}}%
\pgfpathlineto{\pgfqpoint{5.041259in}{3.287011in}}%
\pgfpathlineto{\pgfqpoint{5.041259in}{3.289960in}}%
\pgfpathlineto{\pgfqpoint{5.045800in}{3.289960in}}%
\pgfpathlineto{\pgfqpoint{5.045800in}{3.287011in}}%
\pgfpathmoveto{\pgfqpoint{5.045800in}{3.287011in}}%
\pgfpathlineto{\pgfqpoint{5.045800in}{3.287011in}}%
\pgfpathlineto{\pgfqpoint{5.045800in}{3.289960in}}%
\pgfpathlineto{\pgfqpoint{5.050341in}{3.289960in}}%
\pgfpathlineto{\pgfqpoint{5.050341in}{3.287011in}}%
\pgfpathmoveto{\pgfqpoint{5.045800in}{3.289960in}}%
\pgfpathlineto{\pgfqpoint{5.045800in}{3.289960in}}%
\pgfpathlineto{\pgfqpoint{5.045800in}{3.292910in}}%
\pgfpathlineto{\pgfqpoint{5.050341in}{3.292910in}}%
\pgfpathlineto{\pgfqpoint{5.050341in}{3.289960in}}%
\pgfpathmoveto{\pgfqpoint{5.050341in}{3.289960in}}%
\pgfpathlineto{\pgfqpoint{5.050341in}{3.289960in}}%
\pgfpathlineto{\pgfqpoint{5.050341in}{3.292910in}}%
\pgfpathlineto{\pgfqpoint{5.054882in}{3.292910in}}%
\pgfpathlineto{\pgfqpoint{5.054882in}{3.289960in}}%
\pgfpathmoveto{\pgfqpoint{5.050341in}{3.292910in}}%
\pgfpathlineto{\pgfqpoint{5.050341in}{3.292910in}}%
\pgfpathlineto{\pgfqpoint{5.050341in}{3.295859in}}%
\pgfpathlineto{\pgfqpoint{5.054882in}{3.295859in}}%
\pgfpathlineto{\pgfqpoint{5.054882in}{3.292910in}}%
\pgfpathmoveto{\pgfqpoint{5.054882in}{3.292910in}}%
\pgfpathlineto{\pgfqpoint{5.054882in}{3.292910in}}%
\pgfpathlineto{\pgfqpoint{5.054882in}{3.295859in}}%
\pgfpathlineto{\pgfqpoint{5.059422in}{3.295859in}}%
\pgfpathlineto{\pgfqpoint{5.059422in}{3.292910in}}%
\pgfpathmoveto{\pgfqpoint{5.054882in}{3.295859in}}%
\pgfpathlineto{\pgfqpoint{5.054882in}{3.295859in}}%
\pgfpathlineto{\pgfqpoint{5.054882in}{3.298808in}}%
\pgfpathlineto{\pgfqpoint{5.059422in}{3.298808in}}%
\pgfpathlineto{\pgfqpoint{5.059422in}{3.295859in}}%
\pgfpathmoveto{\pgfqpoint{5.059422in}{3.295859in}}%
\pgfpathlineto{\pgfqpoint{5.059422in}{3.295859in}}%
\pgfpathlineto{\pgfqpoint{5.059422in}{3.298808in}}%
\pgfpathlineto{\pgfqpoint{5.063963in}{3.298808in}}%
\pgfpathlineto{\pgfqpoint{5.063963in}{3.295859in}}%
\pgfpathmoveto{\pgfqpoint{5.059422in}{3.298808in}}%
\pgfpathlineto{\pgfqpoint{5.059422in}{3.298808in}}%
\pgfpathlineto{\pgfqpoint{5.059422in}{3.301757in}}%
\pgfpathlineto{\pgfqpoint{5.063963in}{3.301757in}}%
\pgfpathlineto{\pgfqpoint{5.063963in}{3.298808in}}%
\pgfpathmoveto{\pgfqpoint{5.063963in}{3.298808in}}%
\pgfpathlineto{\pgfqpoint{5.063963in}{3.298808in}}%
\pgfpathlineto{\pgfqpoint{5.063963in}{3.301757in}}%
\pgfpathlineto{\pgfqpoint{5.068504in}{3.301757in}}%
\pgfpathlineto{\pgfqpoint{5.068504in}{3.298808in}}%
\pgfpathmoveto{\pgfqpoint{5.063963in}{3.301757in}}%
\pgfpathlineto{\pgfqpoint{5.063963in}{3.301757in}}%
\pgfpathlineto{\pgfqpoint{5.063963in}{3.304707in}}%
\pgfpathlineto{\pgfqpoint{5.068504in}{3.304707in}}%
\pgfpathlineto{\pgfqpoint{5.068504in}{3.301757in}}%
\pgfpathmoveto{\pgfqpoint{5.068504in}{3.301757in}}%
\pgfpathlineto{\pgfqpoint{5.068504in}{3.301757in}}%
\pgfpathlineto{\pgfqpoint{5.068504in}{3.304707in}}%
\pgfpathlineto{\pgfqpoint{5.073045in}{3.304707in}}%
\pgfpathlineto{\pgfqpoint{5.073045in}{3.301757in}}%
\pgfpathmoveto{\pgfqpoint{5.068504in}{3.304707in}}%
\pgfpathlineto{\pgfqpoint{5.068504in}{3.304707in}}%
\pgfpathlineto{\pgfqpoint{5.068504in}{3.307656in}}%
\pgfpathlineto{\pgfqpoint{5.073045in}{3.307656in}}%
\pgfpathlineto{\pgfqpoint{5.073045in}{3.304707in}}%
\pgfpathmoveto{\pgfqpoint{5.073045in}{3.304707in}}%
\pgfpathlineto{\pgfqpoint{5.073045in}{3.304707in}}%
\pgfpathlineto{\pgfqpoint{5.073045in}{3.307656in}}%
\pgfpathlineto{\pgfqpoint{5.077586in}{3.307656in}}%
\pgfpathlineto{\pgfqpoint{5.077586in}{3.304707in}}%
\pgfpathmoveto{\pgfqpoint{5.073045in}{3.307656in}}%
\pgfpathlineto{\pgfqpoint{5.073045in}{3.307656in}}%
\pgfpathlineto{\pgfqpoint{5.073045in}{3.310605in}}%
\pgfpathlineto{\pgfqpoint{5.077586in}{3.310605in}}%
\pgfpathlineto{\pgfqpoint{5.077586in}{3.307656in}}%
\pgfpathmoveto{\pgfqpoint{5.077586in}{3.307656in}}%
\pgfpathlineto{\pgfqpoint{5.077586in}{3.307656in}}%
\pgfpathlineto{\pgfqpoint{5.077586in}{3.310605in}}%
\pgfpathlineto{\pgfqpoint{5.082127in}{3.310605in}}%
\pgfpathlineto{\pgfqpoint{5.082127in}{3.307656in}}%
\pgfpathmoveto{\pgfqpoint{5.077586in}{3.310605in}}%
\pgfpathlineto{\pgfqpoint{5.077586in}{3.310605in}}%
\pgfpathlineto{\pgfqpoint{5.077586in}{3.313555in}}%
\pgfpathlineto{\pgfqpoint{5.082127in}{3.313555in}}%
\pgfpathlineto{\pgfqpoint{5.082127in}{3.310605in}}%
\pgfpathmoveto{\pgfqpoint{5.082127in}{3.310605in}}%
\pgfpathlineto{\pgfqpoint{5.082127in}{3.310605in}}%
\pgfpathlineto{\pgfqpoint{5.082127in}{3.313555in}}%
\pgfpathlineto{\pgfqpoint{5.086668in}{3.313555in}}%
\pgfpathlineto{\pgfqpoint{5.086668in}{3.310605in}}%
\pgfpathmoveto{\pgfqpoint{5.082127in}{3.313555in}}%
\pgfpathlineto{\pgfqpoint{5.082127in}{3.313555in}}%
\pgfpathlineto{\pgfqpoint{5.082127in}{3.316504in}}%
\pgfpathlineto{\pgfqpoint{5.086668in}{3.316504in}}%
\pgfpathlineto{\pgfqpoint{5.086668in}{3.313555in}}%
\pgfpathmoveto{\pgfqpoint{5.086668in}{3.313555in}}%
\pgfpathlineto{\pgfqpoint{5.086668in}{3.313555in}}%
\pgfpathlineto{\pgfqpoint{5.086668in}{3.316504in}}%
\pgfpathlineto{\pgfqpoint{5.091209in}{3.316504in}}%
\pgfpathlineto{\pgfqpoint{5.091209in}{3.313555in}}%
\pgfpathmoveto{\pgfqpoint{5.086668in}{3.316504in}}%
\pgfpathlineto{\pgfqpoint{5.086668in}{3.316504in}}%
\pgfpathlineto{\pgfqpoint{5.086668in}{3.319453in}}%
\pgfpathlineto{\pgfqpoint{5.091209in}{3.319453in}}%
\pgfpathlineto{\pgfqpoint{5.091209in}{3.316504in}}%
\pgfpathmoveto{\pgfqpoint{5.091209in}{3.316504in}}%
\pgfpathlineto{\pgfqpoint{5.091209in}{3.316504in}}%
\pgfpathlineto{\pgfqpoint{5.091209in}{3.319453in}}%
\pgfpathlineto{\pgfqpoint{5.095750in}{3.319453in}}%
\pgfpathlineto{\pgfqpoint{5.095750in}{3.316504in}}%
\pgfpathmoveto{\pgfqpoint{5.091209in}{3.319453in}}%
\pgfpathlineto{\pgfqpoint{5.091209in}{3.319453in}}%
\pgfpathlineto{\pgfqpoint{5.091209in}{3.322402in}}%
\pgfpathlineto{\pgfqpoint{5.095750in}{3.322402in}}%
\pgfpathlineto{\pgfqpoint{5.095750in}{3.319453in}}%
\pgfpathmoveto{\pgfqpoint{5.095750in}{3.319453in}}%
\pgfpathlineto{\pgfqpoint{5.095750in}{3.319453in}}%
\pgfpathlineto{\pgfqpoint{5.095750in}{3.322402in}}%
\pgfpathlineto{\pgfqpoint{5.100290in}{3.322402in}}%
\pgfpathlineto{\pgfqpoint{5.100290in}{3.319453in}}%
\pgfpathmoveto{\pgfqpoint{5.095750in}{3.322402in}}%
\pgfpathlineto{\pgfqpoint{5.095750in}{3.322402in}}%
\pgfpathlineto{\pgfqpoint{5.095750in}{3.325352in}}%
\pgfpathlineto{\pgfqpoint{5.100290in}{3.325352in}}%
\pgfpathlineto{\pgfqpoint{5.100290in}{3.322402in}}%
\pgfpathmoveto{\pgfqpoint{5.100290in}{3.322402in}}%
\pgfpathlineto{\pgfqpoint{5.100290in}{3.322402in}}%
\pgfpathlineto{\pgfqpoint{5.100290in}{3.325352in}}%
\pgfpathlineto{\pgfqpoint{5.104831in}{3.325352in}}%
\pgfpathlineto{\pgfqpoint{5.104831in}{3.322402in}}%
\pgfpathmoveto{\pgfqpoint{5.100290in}{3.325352in}}%
\pgfpathlineto{\pgfqpoint{5.100290in}{3.325352in}}%
\pgfpathlineto{\pgfqpoint{5.100290in}{3.328301in}}%
\pgfpathlineto{\pgfqpoint{5.104831in}{3.328301in}}%
\pgfpathlineto{\pgfqpoint{5.104831in}{3.325352in}}%
\pgfpathmoveto{\pgfqpoint{5.104831in}{3.325352in}}%
\pgfpathlineto{\pgfqpoint{5.104831in}{3.325352in}}%
\pgfpathlineto{\pgfqpoint{5.104831in}{3.328301in}}%
\pgfpathlineto{\pgfqpoint{5.109372in}{3.328301in}}%
\pgfpathlineto{\pgfqpoint{5.109372in}{3.325352in}}%
\pgfpathmoveto{\pgfqpoint{5.104831in}{3.328301in}}%
\pgfpathlineto{\pgfqpoint{5.104831in}{3.328301in}}%
\pgfpathlineto{\pgfqpoint{5.104831in}{3.331250in}}%
\pgfpathlineto{\pgfqpoint{5.109372in}{3.331250in}}%
\pgfpathlineto{\pgfqpoint{5.109372in}{3.328301in}}%
\pgfpathmoveto{\pgfqpoint{5.109372in}{3.328301in}}%
\pgfpathlineto{\pgfqpoint{5.109372in}{3.328301in}}%
\pgfpathlineto{\pgfqpoint{5.109372in}{3.331250in}}%
\pgfpathlineto{\pgfqpoint{5.113913in}{3.331250in}}%
\pgfpathlineto{\pgfqpoint{5.113913in}{3.328301in}}%
\pgfpathmoveto{\pgfqpoint{5.109372in}{3.331250in}}%
\pgfpathlineto{\pgfqpoint{5.109372in}{3.331250in}}%
\pgfpathlineto{\pgfqpoint{5.109372in}{3.334199in}}%
\pgfpathlineto{\pgfqpoint{5.113913in}{3.334199in}}%
\pgfpathlineto{\pgfqpoint{5.113913in}{3.331250in}}%
\pgfpathmoveto{\pgfqpoint{5.113913in}{3.331250in}}%
\pgfpathlineto{\pgfqpoint{5.113913in}{3.331250in}}%
\pgfpathlineto{\pgfqpoint{5.113913in}{3.334199in}}%
\pgfpathlineto{\pgfqpoint{5.118455in}{3.334199in}}%
\pgfpathlineto{\pgfqpoint{5.118455in}{3.331250in}}%
\pgfpathmoveto{\pgfqpoint{5.113913in}{3.334199in}}%
\pgfpathlineto{\pgfqpoint{5.113913in}{3.334199in}}%
\pgfpathlineto{\pgfqpoint{5.113913in}{3.337148in}}%
\pgfpathlineto{\pgfqpoint{5.118455in}{3.337148in}}%
\pgfpathlineto{\pgfqpoint{5.118455in}{3.334199in}}%
\pgfpathmoveto{\pgfqpoint{5.118455in}{3.334199in}}%
\pgfpathlineto{\pgfqpoint{5.118455in}{3.334199in}}%
\pgfpathlineto{\pgfqpoint{5.118455in}{3.337148in}}%
\pgfpathlineto{\pgfqpoint{5.122996in}{3.337148in}}%
\pgfpathlineto{\pgfqpoint{5.122996in}{3.334199in}}%
\pgfpathmoveto{\pgfqpoint{5.118455in}{3.337148in}}%
\pgfpathlineto{\pgfqpoint{5.118455in}{3.337148in}}%
\pgfpathlineto{\pgfqpoint{5.118455in}{3.340098in}}%
\pgfpathlineto{\pgfqpoint{5.122996in}{3.340098in}}%
\pgfpathlineto{\pgfqpoint{5.122996in}{3.337148in}}%
\pgfpathmoveto{\pgfqpoint{5.122996in}{3.337148in}}%
\pgfpathlineto{\pgfqpoint{5.122996in}{3.337148in}}%
\pgfpathlineto{\pgfqpoint{5.122996in}{3.340098in}}%
\pgfpathlineto{\pgfqpoint{5.127537in}{3.340098in}}%
\pgfpathlineto{\pgfqpoint{5.127537in}{3.337148in}}%
\pgfpathmoveto{\pgfqpoint{5.122996in}{3.340098in}}%
\pgfpathlineto{\pgfqpoint{5.122996in}{3.340098in}}%
\pgfpathlineto{\pgfqpoint{5.122996in}{3.343047in}}%
\pgfpathlineto{\pgfqpoint{5.127537in}{3.343047in}}%
\pgfpathlineto{\pgfqpoint{5.127537in}{3.340098in}}%
\pgfpathmoveto{\pgfqpoint{5.127537in}{3.340098in}}%
\pgfpathlineto{\pgfqpoint{5.127537in}{3.340098in}}%
\pgfpathlineto{\pgfqpoint{5.127537in}{3.343047in}}%
\pgfpathlineto{\pgfqpoint{5.132078in}{3.343047in}}%
\pgfpathlineto{\pgfqpoint{5.132078in}{3.340098in}}%
\pgfpathmoveto{\pgfqpoint{5.127537in}{3.343047in}}%
\pgfpathlineto{\pgfqpoint{5.127537in}{3.343047in}}%
\pgfpathlineto{\pgfqpoint{5.127537in}{3.345996in}}%
\pgfpathlineto{\pgfqpoint{5.132078in}{3.345996in}}%
\pgfpathlineto{\pgfqpoint{5.132078in}{3.343047in}}%
\pgfpathmoveto{\pgfqpoint{5.132078in}{3.343047in}}%
\pgfpathlineto{\pgfqpoint{5.132078in}{3.343047in}}%
\pgfpathlineto{\pgfqpoint{5.132078in}{3.345996in}}%
\pgfpathlineto{\pgfqpoint{5.136620in}{3.345996in}}%
\pgfpathlineto{\pgfqpoint{5.136620in}{3.343047in}}%
\pgfpathmoveto{\pgfqpoint{5.132078in}{3.345996in}}%
\pgfpathlineto{\pgfqpoint{5.132078in}{3.345996in}}%
\pgfpathlineto{\pgfqpoint{5.132078in}{3.348945in}}%
\pgfpathlineto{\pgfqpoint{5.136620in}{3.348945in}}%
\pgfpathlineto{\pgfqpoint{5.136620in}{3.345996in}}%
\pgfpathmoveto{\pgfqpoint{5.136620in}{3.345996in}}%
\pgfpathlineto{\pgfqpoint{5.136620in}{3.345996in}}%
\pgfpathlineto{\pgfqpoint{5.136620in}{3.348945in}}%
\pgfpathlineto{\pgfqpoint{5.141161in}{3.348945in}}%
\pgfpathlineto{\pgfqpoint{5.141161in}{3.345996in}}%
\pgfpathmoveto{\pgfqpoint{5.136620in}{3.348945in}}%
\pgfpathlineto{\pgfqpoint{5.136620in}{3.348945in}}%
\pgfpathlineto{\pgfqpoint{5.136620in}{3.351894in}}%
\pgfpathlineto{\pgfqpoint{5.141161in}{3.351894in}}%
\pgfpathlineto{\pgfqpoint{5.141161in}{3.348945in}}%
\pgfpathmoveto{\pgfqpoint{5.141161in}{3.348945in}}%
\pgfpathlineto{\pgfqpoint{5.141161in}{3.348945in}}%
\pgfpathlineto{\pgfqpoint{5.141161in}{3.351894in}}%
\pgfpathlineto{\pgfqpoint{5.145702in}{3.351894in}}%
\pgfpathlineto{\pgfqpoint{5.145702in}{3.348945in}}%
\pgfpathmoveto{\pgfqpoint{5.141161in}{3.351894in}}%
\pgfpathlineto{\pgfqpoint{5.141161in}{3.351894in}}%
\pgfpathlineto{\pgfqpoint{5.141161in}{3.354843in}}%
\pgfpathlineto{\pgfqpoint{5.145702in}{3.354843in}}%
\pgfpathlineto{\pgfqpoint{5.145702in}{3.351894in}}%
\pgfpathmoveto{\pgfqpoint{5.145702in}{3.351894in}}%
\pgfpathlineto{\pgfqpoint{5.145702in}{3.351894in}}%
\pgfpathlineto{\pgfqpoint{5.145702in}{3.354843in}}%
\pgfpathlineto{\pgfqpoint{5.150243in}{3.354843in}}%
\pgfpathlineto{\pgfqpoint{5.150243in}{3.351894in}}%
\pgfpathmoveto{\pgfqpoint{5.145702in}{3.354843in}}%
\pgfpathlineto{\pgfqpoint{5.145702in}{3.354843in}}%
\pgfpathlineto{\pgfqpoint{5.145702in}{3.357793in}}%
\pgfpathlineto{\pgfqpoint{5.150243in}{3.357793in}}%
\pgfpathlineto{\pgfqpoint{5.150243in}{3.354843in}}%
\pgfpathmoveto{\pgfqpoint{5.150243in}{3.354843in}}%
\pgfpathlineto{\pgfqpoint{5.150243in}{3.354843in}}%
\pgfpathlineto{\pgfqpoint{5.150243in}{3.357793in}}%
\pgfpathlineto{\pgfqpoint{5.154785in}{3.357793in}}%
\pgfpathlineto{\pgfqpoint{5.154785in}{3.354843in}}%
\pgfpathmoveto{\pgfqpoint{5.150243in}{3.357793in}}%
\pgfpathlineto{\pgfqpoint{5.150243in}{3.357793in}}%
\pgfpathlineto{\pgfqpoint{5.150243in}{3.360742in}}%
\pgfpathlineto{\pgfqpoint{5.154785in}{3.360742in}}%
\pgfpathlineto{\pgfqpoint{5.154785in}{3.357793in}}%
\pgfpathmoveto{\pgfqpoint{5.150243in}{3.360742in}}%
\pgfpathlineto{\pgfqpoint{5.150243in}{3.360742in}}%
\pgfpathlineto{\pgfqpoint{5.150243in}{3.363691in}}%
\pgfpathlineto{\pgfqpoint{5.154785in}{3.363691in}}%
\pgfpathlineto{\pgfqpoint{5.154785in}{3.360742in}}%
\pgfpathmoveto{\pgfqpoint{5.154785in}{3.360742in}}%
\pgfpathlineto{\pgfqpoint{5.154785in}{3.360742in}}%
\pgfpathlineto{\pgfqpoint{5.154785in}{3.363691in}}%
\pgfpathlineto{\pgfqpoint{5.159326in}{3.363691in}}%
\pgfpathlineto{\pgfqpoint{5.159326in}{3.360742in}}%
\pgfpathmoveto{\pgfqpoint{5.154785in}{3.363691in}}%
\pgfpathlineto{\pgfqpoint{5.154785in}{3.363691in}}%
\pgfpathlineto{\pgfqpoint{5.154785in}{3.366640in}}%
\pgfpathlineto{\pgfqpoint{5.159326in}{3.366640in}}%
\pgfpathlineto{\pgfqpoint{5.159326in}{3.363691in}}%
\pgfpathmoveto{\pgfqpoint{5.159326in}{3.363691in}}%
\pgfpathlineto{\pgfqpoint{5.159326in}{3.363691in}}%
\pgfpathlineto{\pgfqpoint{5.159326in}{3.366640in}}%
\pgfpathlineto{\pgfqpoint{5.163867in}{3.366640in}}%
\pgfpathlineto{\pgfqpoint{5.163867in}{3.363691in}}%
\pgfpathmoveto{\pgfqpoint{5.159326in}{3.366640in}}%
\pgfpathlineto{\pgfqpoint{5.159326in}{3.366640in}}%
\pgfpathlineto{\pgfqpoint{5.159326in}{3.369589in}}%
\pgfpathlineto{\pgfqpoint{5.163867in}{3.369589in}}%
\pgfpathlineto{\pgfqpoint{5.163867in}{3.366640in}}%
\pgfpathmoveto{\pgfqpoint{5.163867in}{3.366640in}}%
\pgfpathlineto{\pgfqpoint{5.163867in}{3.366640in}}%
\pgfpathlineto{\pgfqpoint{5.163867in}{3.369589in}}%
\pgfpathlineto{\pgfqpoint{5.168408in}{3.369589in}}%
\pgfpathlineto{\pgfqpoint{5.168408in}{3.366640in}}%
\pgfpathmoveto{\pgfqpoint{5.163867in}{3.369589in}}%
\pgfpathlineto{\pgfqpoint{5.163867in}{3.369589in}}%
\pgfpathlineto{\pgfqpoint{5.163867in}{3.372538in}}%
\pgfpathlineto{\pgfqpoint{5.168408in}{3.372538in}}%
\pgfpathlineto{\pgfqpoint{5.168408in}{3.369589in}}%
\pgfpathmoveto{\pgfqpoint{5.168408in}{3.369589in}}%
\pgfpathlineto{\pgfqpoint{5.168408in}{3.369589in}}%
\pgfpathlineto{\pgfqpoint{5.168408in}{3.372538in}}%
\pgfpathlineto{\pgfqpoint{5.172950in}{3.372538in}}%
\pgfpathlineto{\pgfqpoint{5.172950in}{3.369589in}}%
\pgfpathmoveto{\pgfqpoint{5.168408in}{3.372538in}}%
\pgfpathlineto{\pgfqpoint{5.168408in}{3.372538in}}%
\pgfpathlineto{\pgfqpoint{5.168408in}{3.375488in}}%
\pgfpathlineto{\pgfqpoint{5.172950in}{3.375488in}}%
\pgfpathlineto{\pgfqpoint{5.172950in}{3.372538in}}%
\pgfpathmoveto{\pgfqpoint{5.172950in}{3.372538in}}%
\pgfpathlineto{\pgfqpoint{5.172950in}{3.372538in}}%
\pgfpathlineto{\pgfqpoint{5.172950in}{3.375488in}}%
\pgfpathlineto{\pgfqpoint{5.177491in}{3.375488in}}%
\pgfpathlineto{\pgfqpoint{5.177491in}{3.372538in}}%
\pgfpathmoveto{\pgfqpoint{5.172950in}{3.375488in}}%
\pgfpathlineto{\pgfqpoint{5.172950in}{3.375488in}}%
\pgfpathlineto{\pgfqpoint{5.172950in}{3.378437in}}%
\pgfpathlineto{\pgfqpoint{5.177491in}{3.378437in}}%
\pgfpathlineto{\pgfqpoint{5.177491in}{3.375488in}}%
\pgfpathmoveto{\pgfqpoint{5.177491in}{3.375488in}}%
\pgfpathlineto{\pgfqpoint{5.177491in}{3.375488in}}%
\pgfpathlineto{\pgfqpoint{5.177491in}{3.378437in}}%
\pgfpathlineto{\pgfqpoint{5.182032in}{3.378437in}}%
\pgfpathlineto{\pgfqpoint{5.182032in}{3.375488in}}%
\pgfpathmoveto{\pgfqpoint{5.177491in}{3.378437in}}%
\pgfpathlineto{\pgfqpoint{5.177491in}{3.378437in}}%
\pgfpathlineto{\pgfqpoint{5.177491in}{3.381386in}}%
\pgfpathlineto{\pgfqpoint{5.182032in}{3.381386in}}%
\pgfpathlineto{\pgfqpoint{5.182032in}{3.378437in}}%
\pgfpathmoveto{\pgfqpoint{5.182032in}{3.378437in}}%
\pgfpathlineto{\pgfqpoint{5.182032in}{3.378437in}}%
\pgfpathlineto{\pgfqpoint{5.182032in}{3.381386in}}%
\pgfpathlineto{\pgfqpoint{5.186573in}{3.381386in}}%
\pgfpathlineto{\pgfqpoint{5.186573in}{3.378437in}}%
\pgfpathmoveto{\pgfqpoint{5.182032in}{3.381386in}}%
\pgfpathlineto{\pgfqpoint{5.182032in}{3.381386in}}%
\pgfpathlineto{\pgfqpoint{5.182032in}{3.384335in}}%
\pgfpathlineto{\pgfqpoint{5.186573in}{3.384335in}}%
\pgfpathlineto{\pgfqpoint{5.186573in}{3.381386in}}%
\pgfpathmoveto{\pgfqpoint{5.186573in}{3.381386in}}%
\pgfpathlineto{\pgfqpoint{5.186573in}{3.381386in}}%
\pgfpathlineto{\pgfqpoint{5.186573in}{3.384335in}}%
\pgfpathlineto{\pgfqpoint{5.191115in}{3.384335in}}%
\pgfpathlineto{\pgfqpoint{5.191115in}{3.381386in}}%
\pgfpathmoveto{\pgfqpoint{5.186573in}{3.384335in}}%
\pgfpathlineto{\pgfqpoint{5.186573in}{3.384335in}}%
\pgfpathlineto{\pgfqpoint{5.186573in}{3.387284in}}%
\pgfpathlineto{\pgfqpoint{5.191115in}{3.387284in}}%
\pgfpathlineto{\pgfqpoint{5.191115in}{3.384335in}}%
\pgfpathmoveto{\pgfqpoint{5.191115in}{3.384335in}}%
\pgfpathlineto{\pgfqpoint{5.191115in}{3.384335in}}%
\pgfpathlineto{\pgfqpoint{5.191115in}{3.387284in}}%
\pgfpathlineto{\pgfqpoint{5.195656in}{3.387284in}}%
\pgfpathlineto{\pgfqpoint{5.195656in}{3.384335in}}%
\pgfpathmoveto{\pgfqpoint{5.191115in}{3.387284in}}%
\pgfpathlineto{\pgfqpoint{5.191115in}{3.387284in}}%
\pgfpathlineto{\pgfqpoint{5.191115in}{3.390233in}}%
\pgfpathlineto{\pgfqpoint{5.195656in}{3.390233in}}%
\pgfpathlineto{\pgfqpoint{5.195656in}{3.387284in}}%
\pgfpathmoveto{\pgfqpoint{5.195656in}{3.387284in}}%
\pgfpathlineto{\pgfqpoint{5.195656in}{3.387284in}}%
\pgfpathlineto{\pgfqpoint{5.195656in}{3.390233in}}%
\pgfpathlineto{\pgfqpoint{5.200197in}{3.390233in}}%
\pgfpathlineto{\pgfqpoint{5.200197in}{3.387284in}}%
\pgfpathmoveto{\pgfqpoint{5.195656in}{3.390233in}}%
\pgfpathlineto{\pgfqpoint{5.195656in}{3.390233in}}%
\pgfpathlineto{\pgfqpoint{5.195656in}{3.393183in}}%
\pgfpathlineto{\pgfqpoint{5.200197in}{3.393183in}}%
\pgfpathlineto{\pgfqpoint{5.200197in}{3.390233in}}%
\pgfpathmoveto{\pgfqpoint{5.200197in}{3.390233in}}%
\pgfpathlineto{\pgfqpoint{5.200197in}{3.390233in}}%
\pgfpathlineto{\pgfqpoint{5.200197in}{3.393183in}}%
\pgfpathlineto{\pgfqpoint{5.204738in}{3.393183in}}%
\pgfpathlineto{\pgfqpoint{5.204738in}{3.390233in}}%
\pgfpathmoveto{\pgfqpoint{5.200197in}{3.393183in}}%
\pgfpathlineto{\pgfqpoint{5.200197in}{3.393183in}}%
\pgfpathlineto{\pgfqpoint{5.200197in}{3.396132in}}%
\pgfpathlineto{\pgfqpoint{5.204738in}{3.396132in}}%
\pgfpathlineto{\pgfqpoint{5.204738in}{3.393183in}}%
\pgfpathmoveto{\pgfqpoint{5.204738in}{3.393183in}}%
\pgfpathlineto{\pgfqpoint{5.204738in}{3.393183in}}%
\pgfpathlineto{\pgfqpoint{5.204738in}{3.396132in}}%
\pgfpathlineto{\pgfqpoint{5.209279in}{3.396132in}}%
\pgfpathlineto{\pgfqpoint{5.209279in}{3.393183in}}%
\pgfpathmoveto{\pgfqpoint{5.204738in}{3.396132in}}%
\pgfpathlineto{\pgfqpoint{5.204738in}{3.396132in}}%
\pgfpathlineto{\pgfqpoint{5.204738in}{3.399081in}}%
\pgfpathlineto{\pgfqpoint{5.209279in}{3.399081in}}%
\pgfpathlineto{\pgfqpoint{5.209279in}{3.396132in}}%
\pgfpathmoveto{\pgfqpoint{5.209279in}{3.396132in}}%
\pgfpathlineto{\pgfqpoint{5.209279in}{3.396132in}}%
\pgfpathlineto{\pgfqpoint{5.209279in}{3.399081in}}%
\pgfpathlineto{\pgfqpoint{5.213821in}{3.399081in}}%
\pgfpathlineto{\pgfqpoint{5.213821in}{3.396132in}}%
\pgfpathmoveto{\pgfqpoint{5.209279in}{3.399081in}}%
\pgfpathlineto{\pgfqpoint{5.209279in}{3.399081in}}%
\pgfpathlineto{\pgfqpoint{5.209279in}{3.402030in}}%
\pgfpathlineto{\pgfqpoint{5.213821in}{3.402030in}}%
\pgfpathlineto{\pgfqpoint{5.213821in}{3.399081in}}%
\pgfpathmoveto{\pgfqpoint{5.213821in}{3.399081in}}%
\pgfpathlineto{\pgfqpoint{5.213821in}{3.399081in}}%
\pgfpathlineto{\pgfqpoint{5.213821in}{3.402030in}}%
\pgfpathlineto{\pgfqpoint{5.218362in}{3.402030in}}%
\pgfpathlineto{\pgfqpoint{5.218362in}{3.399081in}}%
\pgfpathmoveto{\pgfqpoint{5.213821in}{3.402030in}}%
\pgfpathlineto{\pgfqpoint{5.213821in}{3.402030in}}%
\pgfpathlineto{\pgfqpoint{5.213821in}{3.404979in}}%
\pgfpathlineto{\pgfqpoint{5.218362in}{3.404979in}}%
\pgfpathlineto{\pgfqpoint{5.218362in}{3.402030in}}%
\pgfpathmoveto{\pgfqpoint{5.218362in}{3.402030in}}%
\pgfpathlineto{\pgfqpoint{5.218362in}{3.402030in}}%
\pgfpathlineto{\pgfqpoint{5.218362in}{3.404979in}}%
\pgfpathlineto{\pgfqpoint{5.222903in}{3.404979in}}%
\pgfpathlineto{\pgfqpoint{5.222903in}{3.402030in}}%
\pgfpathmoveto{\pgfqpoint{5.218362in}{3.404979in}}%
\pgfpathlineto{\pgfqpoint{5.218362in}{3.404979in}}%
\pgfpathlineto{\pgfqpoint{5.218362in}{3.407928in}}%
\pgfpathlineto{\pgfqpoint{5.222903in}{3.407928in}}%
\pgfpathlineto{\pgfqpoint{5.222903in}{3.404979in}}%
\pgfpathmoveto{\pgfqpoint{5.222903in}{3.404979in}}%
\pgfpathlineto{\pgfqpoint{5.222903in}{3.404979in}}%
\pgfpathlineto{\pgfqpoint{5.222903in}{3.407928in}}%
\pgfpathlineto{\pgfqpoint{5.227444in}{3.407928in}}%
\pgfpathlineto{\pgfqpoint{5.227444in}{3.404979in}}%
\pgfpathmoveto{\pgfqpoint{5.222903in}{3.407928in}}%
\pgfpathlineto{\pgfqpoint{5.222903in}{3.407928in}}%
\pgfpathlineto{\pgfqpoint{5.222903in}{3.410878in}}%
\pgfpathlineto{\pgfqpoint{5.227444in}{3.410878in}}%
\pgfpathlineto{\pgfqpoint{5.227444in}{3.407928in}}%
\pgfpathmoveto{\pgfqpoint{5.227444in}{3.407928in}}%
\pgfpathlineto{\pgfqpoint{5.227444in}{3.407928in}}%
\pgfpathlineto{\pgfqpoint{5.227444in}{3.410878in}}%
\pgfpathlineto{\pgfqpoint{5.231986in}{3.410878in}}%
\pgfpathlineto{\pgfqpoint{5.231986in}{3.407928in}}%
\pgfpathmoveto{\pgfqpoint{5.227444in}{3.410878in}}%
\pgfpathlineto{\pgfqpoint{5.227444in}{3.410878in}}%
\pgfpathlineto{\pgfqpoint{5.227444in}{3.413827in}}%
\pgfpathlineto{\pgfqpoint{5.231986in}{3.413827in}}%
\pgfpathlineto{\pgfqpoint{5.231986in}{3.410878in}}%
\pgfpathmoveto{\pgfqpoint{5.231986in}{3.410878in}}%
\pgfpathlineto{\pgfqpoint{5.231986in}{3.410878in}}%
\pgfpathlineto{\pgfqpoint{5.231986in}{3.413827in}}%
\pgfpathlineto{\pgfqpoint{5.236527in}{3.413827in}}%
\pgfpathlineto{\pgfqpoint{5.236527in}{3.410878in}}%
\pgfpathmoveto{\pgfqpoint{5.231986in}{3.413827in}}%
\pgfpathlineto{\pgfqpoint{5.231986in}{3.413827in}}%
\pgfpathlineto{\pgfqpoint{5.231986in}{3.416776in}}%
\pgfpathlineto{\pgfqpoint{5.236527in}{3.416776in}}%
\pgfpathlineto{\pgfqpoint{5.236527in}{3.413827in}}%
\pgfpathmoveto{\pgfqpoint{5.236527in}{3.413827in}}%
\pgfpathlineto{\pgfqpoint{5.236527in}{3.413827in}}%
\pgfpathlineto{\pgfqpoint{5.236527in}{3.416776in}}%
\pgfpathlineto{\pgfqpoint{5.241068in}{3.416776in}}%
\pgfpathlineto{\pgfqpoint{5.241068in}{3.413827in}}%
\pgfpathmoveto{\pgfqpoint{5.236527in}{3.416776in}}%
\pgfpathlineto{\pgfqpoint{5.236527in}{3.416776in}}%
\pgfpathlineto{\pgfqpoint{5.236527in}{3.419725in}}%
\pgfpathlineto{\pgfqpoint{5.241068in}{3.419725in}}%
\pgfpathlineto{\pgfqpoint{5.241068in}{3.416776in}}%
\pgfpathmoveto{\pgfqpoint{5.241068in}{3.416776in}}%
\pgfpathlineto{\pgfqpoint{5.241068in}{3.416776in}}%
\pgfpathlineto{\pgfqpoint{5.241068in}{3.419725in}}%
\pgfpathlineto{\pgfqpoint{5.245609in}{3.419725in}}%
\pgfpathlineto{\pgfqpoint{5.245609in}{3.416776in}}%
\pgfpathmoveto{\pgfqpoint{5.241068in}{3.419725in}}%
\pgfpathlineto{\pgfqpoint{5.241068in}{3.419725in}}%
\pgfpathlineto{\pgfqpoint{5.241068in}{3.422674in}}%
\pgfpathlineto{\pgfqpoint{5.245609in}{3.422674in}}%
\pgfpathlineto{\pgfqpoint{5.245609in}{3.419725in}}%
\pgfpathmoveto{\pgfqpoint{5.245609in}{3.419725in}}%
\pgfpathlineto{\pgfqpoint{5.245609in}{3.419725in}}%
\pgfpathlineto{\pgfqpoint{5.245609in}{3.422674in}}%
\pgfpathlineto{\pgfqpoint{5.250151in}{3.422674in}}%
\pgfpathlineto{\pgfqpoint{5.250151in}{3.419725in}}%
\pgfpathmoveto{\pgfqpoint{5.245609in}{3.422674in}}%
\pgfpathlineto{\pgfqpoint{5.245609in}{3.422674in}}%
\pgfpathlineto{\pgfqpoint{5.245609in}{3.425623in}}%
\pgfpathlineto{\pgfqpoint{5.250151in}{3.425623in}}%
\pgfpathlineto{\pgfqpoint{5.250151in}{3.422674in}}%
\pgfpathmoveto{\pgfqpoint{5.250151in}{3.422674in}}%
\pgfpathlineto{\pgfqpoint{5.250151in}{3.422674in}}%
\pgfpathlineto{\pgfqpoint{5.250151in}{3.425623in}}%
\pgfpathlineto{\pgfqpoint{5.254692in}{3.425623in}}%
\pgfpathlineto{\pgfqpoint{5.254692in}{3.422674in}}%
\pgfpathmoveto{\pgfqpoint{5.250151in}{3.425623in}}%
\pgfpathlineto{\pgfqpoint{5.250151in}{3.425623in}}%
\pgfpathlineto{\pgfqpoint{5.250151in}{3.428573in}}%
\pgfpathlineto{\pgfqpoint{5.254692in}{3.428573in}}%
\pgfpathlineto{\pgfqpoint{5.254692in}{3.425623in}}%
\pgfpathmoveto{\pgfqpoint{5.254692in}{3.425623in}}%
\pgfpathlineto{\pgfqpoint{5.254692in}{3.425623in}}%
\pgfpathlineto{\pgfqpoint{5.254692in}{3.428573in}}%
\pgfpathlineto{\pgfqpoint{5.259233in}{3.428573in}}%
\pgfpathlineto{\pgfqpoint{5.259233in}{3.425623in}}%
\pgfpathmoveto{\pgfqpoint{5.254692in}{3.428573in}}%
\pgfpathlineto{\pgfqpoint{5.254692in}{3.428573in}}%
\pgfpathlineto{\pgfqpoint{5.254692in}{3.431522in}}%
\pgfpathlineto{\pgfqpoint{5.259233in}{3.431522in}}%
\pgfpathlineto{\pgfqpoint{5.259233in}{3.428573in}}%
\pgfpathmoveto{\pgfqpoint{5.259233in}{3.428573in}}%
\pgfpathlineto{\pgfqpoint{5.259233in}{3.428573in}}%
\pgfpathlineto{\pgfqpoint{5.259233in}{3.431522in}}%
\pgfpathlineto{\pgfqpoint{5.263773in}{3.431522in}}%
\pgfpathlineto{\pgfqpoint{5.263773in}{3.428573in}}%
\pgfpathmoveto{\pgfqpoint{5.259233in}{3.431522in}}%
\pgfpathlineto{\pgfqpoint{5.259233in}{3.431522in}}%
\pgfpathlineto{\pgfqpoint{5.259233in}{3.434471in}}%
\pgfpathlineto{\pgfqpoint{5.263773in}{3.434471in}}%
\pgfpathlineto{\pgfqpoint{5.263773in}{3.431522in}}%
\pgfpathmoveto{\pgfqpoint{5.263773in}{3.431522in}}%
\pgfpathlineto{\pgfqpoint{5.263773in}{3.431522in}}%
\pgfpathlineto{\pgfqpoint{5.263773in}{3.434471in}}%
\pgfpathlineto{\pgfqpoint{5.268314in}{3.434471in}}%
\pgfpathlineto{\pgfqpoint{5.268314in}{3.431522in}}%
\pgfpathmoveto{\pgfqpoint{5.263773in}{3.434471in}}%
\pgfpathlineto{\pgfqpoint{5.263773in}{3.434471in}}%
\pgfpathlineto{\pgfqpoint{5.263773in}{3.437421in}}%
\pgfpathlineto{\pgfqpoint{5.268314in}{3.437421in}}%
\pgfpathlineto{\pgfqpoint{5.268314in}{3.434471in}}%
\pgfpathmoveto{\pgfqpoint{5.268314in}{3.434471in}}%
\pgfpathlineto{\pgfqpoint{5.268314in}{3.434471in}}%
\pgfpathlineto{\pgfqpoint{5.268314in}{3.437421in}}%
\pgfpathlineto{\pgfqpoint{5.272855in}{3.437421in}}%
\pgfpathlineto{\pgfqpoint{5.272855in}{3.434471in}}%
\pgfpathmoveto{\pgfqpoint{5.268314in}{3.437421in}}%
\pgfpathlineto{\pgfqpoint{5.268314in}{3.437421in}}%
\pgfpathlineto{\pgfqpoint{5.268314in}{3.440370in}}%
\pgfpathlineto{\pgfqpoint{5.272855in}{3.440370in}}%
\pgfpathlineto{\pgfqpoint{5.272855in}{3.437421in}}%
\pgfpathmoveto{\pgfqpoint{5.272855in}{3.437421in}}%
\pgfpathlineto{\pgfqpoint{5.272855in}{3.437421in}}%
\pgfpathlineto{\pgfqpoint{5.272855in}{3.440370in}}%
\pgfpathlineto{\pgfqpoint{5.277396in}{3.440370in}}%
\pgfpathlineto{\pgfqpoint{5.277396in}{3.437421in}}%
\pgfpathmoveto{\pgfqpoint{5.272855in}{3.440370in}}%
\pgfpathlineto{\pgfqpoint{5.272855in}{3.440370in}}%
\pgfpathlineto{\pgfqpoint{5.272855in}{3.443319in}}%
\pgfpathlineto{\pgfqpoint{5.277396in}{3.443319in}}%
\pgfpathlineto{\pgfqpoint{5.277396in}{3.440370in}}%
\pgfpathmoveto{\pgfqpoint{5.277396in}{3.440370in}}%
\pgfpathlineto{\pgfqpoint{5.277396in}{3.440370in}}%
\pgfpathlineto{\pgfqpoint{5.277396in}{3.443319in}}%
\pgfpathlineto{\pgfqpoint{5.281936in}{3.443319in}}%
\pgfpathlineto{\pgfqpoint{5.281936in}{3.440370in}}%
\pgfpathmoveto{\pgfqpoint{5.277396in}{3.443319in}}%
\pgfpathlineto{\pgfqpoint{5.277396in}{3.443319in}}%
\pgfpathlineto{\pgfqpoint{5.277396in}{3.446268in}}%
\pgfpathlineto{\pgfqpoint{5.281936in}{3.446268in}}%
\pgfpathlineto{\pgfqpoint{5.281936in}{3.443319in}}%
\pgfpathmoveto{\pgfqpoint{5.281936in}{3.443319in}}%
\pgfpathlineto{\pgfqpoint{5.281936in}{3.443319in}}%
\pgfpathlineto{\pgfqpoint{5.281936in}{3.446268in}}%
\pgfpathlineto{\pgfqpoint{5.286477in}{3.446268in}}%
\pgfpathlineto{\pgfqpoint{5.286477in}{3.443319in}}%
\pgfpathmoveto{\pgfqpoint{5.281936in}{3.446268in}}%
\pgfpathlineto{\pgfqpoint{5.281936in}{3.446268in}}%
\pgfpathlineto{\pgfqpoint{5.281936in}{3.449218in}}%
\pgfpathlineto{\pgfqpoint{5.286477in}{3.449218in}}%
\pgfpathlineto{\pgfqpoint{5.286477in}{3.446268in}}%
\pgfpathmoveto{\pgfqpoint{5.286477in}{3.446268in}}%
\pgfpathlineto{\pgfqpoint{5.286477in}{3.446268in}}%
\pgfpathlineto{\pgfqpoint{5.286477in}{3.449218in}}%
\pgfpathlineto{\pgfqpoint{5.291018in}{3.449218in}}%
\pgfpathlineto{\pgfqpoint{5.291018in}{3.446268in}}%
\pgfpathmoveto{\pgfqpoint{5.286477in}{3.449218in}}%
\pgfpathlineto{\pgfqpoint{5.286477in}{3.449218in}}%
\pgfpathlineto{\pgfqpoint{5.286477in}{3.452167in}}%
\pgfpathlineto{\pgfqpoint{5.291018in}{3.452167in}}%
\pgfpathlineto{\pgfqpoint{5.291018in}{3.449218in}}%
\pgfpathmoveto{\pgfqpoint{5.291018in}{3.449218in}}%
\pgfpathlineto{\pgfqpoint{5.291018in}{3.449218in}}%
\pgfpathlineto{\pgfqpoint{5.291018in}{3.452167in}}%
\pgfpathlineto{\pgfqpoint{5.295559in}{3.452167in}}%
\pgfpathlineto{\pgfqpoint{5.295559in}{3.449218in}}%
\pgfpathmoveto{\pgfqpoint{5.291018in}{3.452167in}}%
\pgfpathlineto{\pgfqpoint{5.291018in}{3.452167in}}%
\pgfpathlineto{\pgfqpoint{5.291018in}{3.455116in}}%
\pgfpathlineto{\pgfqpoint{5.295559in}{3.455116in}}%
\pgfpathlineto{\pgfqpoint{5.295559in}{3.452167in}}%
\pgfpathmoveto{\pgfqpoint{5.295559in}{3.452167in}}%
\pgfpathlineto{\pgfqpoint{5.295559in}{3.452167in}}%
\pgfpathlineto{\pgfqpoint{5.295559in}{3.455116in}}%
\pgfpathlineto{\pgfqpoint{5.300099in}{3.455116in}}%
\pgfpathlineto{\pgfqpoint{5.300099in}{3.452167in}}%
\pgfpathmoveto{\pgfqpoint{5.295559in}{3.455116in}}%
\pgfpathlineto{\pgfqpoint{5.295559in}{3.455116in}}%
\pgfpathlineto{\pgfqpoint{5.295559in}{3.458066in}}%
\pgfpathlineto{\pgfqpoint{5.300099in}{3.458066in}}%
\pgfpathlineto{\pgfqpoint{5.300099in}{3.455116in}}%
\pgfpathmoveto{\pgfqpoint{5.300099in}{3.455116in}}%
\pgfpathlineto{\pgfqpoint{5.300099in}{3.455116in}}%
\pgfpathlineto{\pgfqpoint{5.300099in}{3.458066in}}%
\pgfpathlineto{\pgfqpoint{5.304640in}{3.458066in}}%
\pgfpathlineto{\pgfqpoint{5.304640in}{3.455116in}}%
\pgfpathmoveto{\pgfqpoint{5.300099in}{3.458066in}}%
\pgfpathlineto{\pgfqpoint{5.300099in}{3.458066in}}%
\pgfpathlineto{\pgfqpoint{5.300099in}{3.461015in}}%
\pgfpathlineto{\pgfqpoint{5.304640in}{3.461015in}}%
\pgfpathlineto{\pgfqpoint{5.304640in}{3.458066in}}%
\pgfpathmoveto{\pgfqpoint{5.304640in}{3.458066in}}%
\pgfpathlineto{\pgfqpoint{5.304640in}{3.458066in}}%
\pgfpathlineto{\pgfqpoint{5.304640in}{3.461015in}}%
\pgfpathlineto{\pgfqpoint{5.309181in}{3.461015in}}%
\pgfpathlineto{\pgfqpoint{5.309181in}{3.458066in}}%
\pgfpathmoveto{\pgfqpoint{5.304640in}{3.461015in}}%
\pgfpathlineto{\pgfqpoint{5.304640in}{3.461015in}}%
\pgfpathlineto{\pgfqpoint{5.304640in}{3.463964in}}%
\pgfpathlineto{\pgfqpoint{5.309181in}{3.463964in}}%
\pgfpathlineto{\pgfqpoint{5.309181in}{3.461015in}}%
\pgfpathmoveto{\pgfqpoint{5.309181in}{3.461015in}}%
\pgfpathlineto{\pgfqpoint{5.309181in}{3.461015in}}%
\pgfpathlineto{\pgfqpoint{5.309181in}{3.463964in}}%
\pgfpathlineto{\pgfqpoint{5.313722in}{3.463964in}}%
\pgfpathlineto{\pgfqpoint{5.313722in}{3.461015in}}%
\pgfpathmoveto{\pgfqpoint{5.309181in}{3.463964in}}%
\pgfpathlineto{\pgfqpoint{5.309181in}{3.463964in}}%
\pgfpathlineto{\pgfqpoint{5.309181in}{3.466914in}}%
\pgfpathlineto{\pgfqpoint{5.313722in}{3.466914in}}%
\pgfpathlineto{\pgfqpoint{5.313722in}{3.463964in}}%
\pgfpathmoveto{\pgfqpoint{5.313722in}{3.463964in}}%
\pgfpathlineto{\pgfqpoint{5.313722in}{3.463964in}}%
\pgfpathlineto{\pgfqpoint{5.313722in}{3.466914in}}%
\pgfpathlineto{\pgfqpoint{5.318263in}{3.466914in}}%
\pgfpathlineto{\pgfqpoint{5.318263in}{3.463964in}}%
\pgfpathmoveto{\pgfqpoint{5.313722in}{3.466914in}}%
\pgfpathlineto{\pgfqpoint{5.313722in}{3.466914in}}%
\pgfpathlineto{\pgfqpoint{5.313722in}{3.469863in}}%
\pgfpathlineto{\pgfqpoint{5.318263in}{3.469863in}}%
\pgfpathlineto{\pgfqpoint{5.318263in}{3.466914in}}%
\pgfpathmoveto{\pgfqpoint{5.318263in}{3.466914in}}%
\pgfpathlineto{\pgfqpoint{5.318263in}{3.466914in}}%
\pgfpathlineto{\pgfqpoint{5.318263in}{3.469863in}}%
\pgfpathlineto{\pgfqpoint{5.322803in}{3.469863in}}%
\pgfpathlineto{\pgfqpoint{5.322803in}{3.466914in}}%
\pgfpathmoveto{\pgfqpoint{5.318263in}{3.469863in}}%
\pgfpathlineto{\pgfqpoint{5.318263in}{3.469863in}}%
\pgfpathlineto{\pgfqpoint{5.318263in}{3.472812in}}%
\pgfpathlineto{\pgfqpoint{5.322803in}{3.472812in}}%
\pgfpathlineto{\pgfqpoint{5.322803in}{3.469863in}}%
\pgfpathmoveto{\pgfqpoint{5.322803in}{3.469863in}}%
\pgfpathlineto{\pgfqpoint{5.322803in}{3.469863in}}%
\pgfpathlineto{\pgfqpoint{5.322803in}{3.472812in}}%
\pgfpathlineto{\pgfqpoint{5.327344in}{3.472812in}}%
\pgfpathlineto{\pgfqpoint{5.327344in}{3.469863in}}%
\pgfpathmoveto{\pgfqpoint{5.322803in}{3.472812in}}%
\pgfpathlineto{\pgfqpoint{5.322803in}{3.472812in}}%
\pgfpathlineto{\pgfqpoint{5.322803in}{3.475761in}}%
\pgfpathlineto{\pgfqpoint{5.327344in}{3.475761in}}%
\pgfpathlineto{\pgfqpoint{5.327344in}{3.472812in}}%
\pgfpathmoveto{\pgfqpoint{5.327344in}{3.472812in}}%
\pgfpathlineto{\pgfqpoint{5.327344in}{3.472812in}}%
\pgfpathlineto{\pgfqpoint{5.327344in}{3.475761in}}%
\pgfpathlineto{\pgfqpoint{5.331885in}{3.475761in}}%
\pgfpathlineto{\pgfqpoint{5.331885in}{3.472812in}}%
\pgfpathmoveto{\pgfqpoint{5.327344in}{3.475761in}}%
\pgfpathlineto{\pgfqpoint{5.327344in}{3.475761in}}%
\pgfpathlineto{\pgfqpoint{5.327344in}{3.478711in}}%
\pgfpathlineto{\pgfqpoint{5.331885in}{3.478711in}}%
\pgfpathlineto{\pgfqpoint{5.331885in}{3.475761in}}%
\pgfpathmoveto{\pgfqpoint{5.331885in}{3.475761in}}%
\pgfpathlineto{\pgfqpoint{5.331885in}{3.475761in}}%
\pgfpathlineto{\pgfqpoint{5.331885in}{3.478711in}}%
\pgfpathlineto{\pgfqpoint{5.336426in}{3.478711in}}%
\pgfpathlineto{\pgfqpoint{5.336426in}{3.475761in}}%
\pgfpathmoveto{\pgfqpoint{5.331885in}{3.478711in}}%
\pgfpathlineto{\pgfqpoint{5.331885in}{3.478711in}}%
\pgfpathlineto{\pgfqpoint{5.331885in}{3.481660in}}%
\pgfpathlineto{\pgfqpoint{5.336426in}{3.481660in}}%
\pgfpathlineto{\pgfqpoint{5.336426in}{3.478711in}}%
\pgfpathmoveto{\pgfqpoint{5.336426in}{3.478711in}}%
\pgfpathlineto{\pgfqpoint{5.336426in}{3.478711in}}%
\pgfpathlineto{\pgfqpoint{5.336426in}{3.481660in}}%
\pgfpathlineto{\pgfqpoint{5.340966in}{3.481660in}}%
\pgfpathlineto{\pgfqpoint{5.340966in}{3.478711in}}%
\pgfpathmoveto{\pgfqpoint{5.340966in}{3.478711in}}%
\pgfpathlineto{\pgfqpoint{5.340966in}{3.478711in}}%
\pgfpathlineto{\pgfqpoint{5.340966in}{3.481660in}}%
\pgfpathlineto{\pgfqpoint{5.345507in}{3.481660in}}%
\pgfpathlineto{\pgfqpoint{5.345507in}{3.478711in}}%
\pgfpathmoveto{\pgfqpoint{5.340966in}{3.481660in}}%
\pgfpathlineto{\pgfqpoint{5.340966in}{3.481660in}}%
\pgfpathlineto{\pgfqpoint{5.340966in}{3.484609in}}%
\pgfpathlineto{\pgfqpoint{5.345507in}{3.484609in}}%
\pgfpathlineto{\pgfqpoint{5.345507in}{3.481660in}}%
\pgfpathmoveto{\pgfqpoint{5.345507in}{3.481660in}}%
\pgfpathlineto{\pgfqpoint{5.345507in}{3.481660in}}%
\pgfpathlineto{\pgfqpoint{5.345507in}{3.484609in}}%
\pgfpathlineto{\pgfqpoint{5.350048in}{3.484609in}}%
\pgfpathlineto{\pgfqpoint{5.350048in}{3.481660in}}%
\pgfpathmoveto{\pgfqpoint{5.345507in}{3.484609in}}%
\pgfpathlineto{\pgfqpoint{5.345507in}{3.484609in}}%
\pgfpathlineto{\pgfqpoint{5.345507in}{3.487559in}}%
\pgfpathlineto{\pgfqpoint{5.350048in}{3.487559in}}%
\pgfpathlineto{\pgfqpoint{5.350048in}{3.484609in}}%
\pgfpathmoveto{\pgfqpoint{5.350048in}{3.484609in}}%
\pgfpathlineto{\pgfqpoint{5.350048in}{3.484609in}}%
\pgfpathlineto{\pgfqpoint{5.350048in}{3.487559in}}%
\pgfpathlineto{\pgfqpoint{5.354589in}{3.487559in}}%
\pgfpathlineto{\pgfqpoint{5.354589in}{3.484609in}}%
\pgfpathmoveto{\pgfqpoint{5.350048in}{3.487559in}}%
\pgfpathlineto{\pgfqpoint{5.350048in}{3.487559in}}%
\pgfpathlineto{\pgfqpoint{5.350048in}{3.490508in}}%
\pgfpathlineto{\pgfqpoint{5.354589in}{3.490508in}}%
\pgfpathlineto{\pgfqpoint{5.354589in}{3.487559in}}%
\pgfpathmoveto{\pgfqpoint{5.354589in}{3.487559in}}%
\pgfpathlineto{\pgfqpoint{5.354589in}{3.487559in}}%
\pgfpathlineto{\pgfqpoint{5.354589in}{3.490508in}}%
\pgfpathlineto{\pgfqpoint{5.359129in}{3.490508in}}%
\pgfpathlineto{\pgfqpoint{5.359129in}{3.487559in}}%
\pgfpathmoveto{\pgfqpoint{5.354589in}{3.490508in}}%
\pgfpathlineto{\pgfqpoint{5.354589in}{3.490508in}}%
\pgfpathlineto{\pgfqpoint{5.354589in}{3.493457in}}%
\pgfpathlineto{\pgfqpoint{5.359129in}{3.493457in}}%
\pgfpathlineto{\pgfqpoint{5.359129in}{3.490508in}}%
\pgfpathmoveto{\pgfqpoint{5.359129in}{3.490508in}}%
\pgfpathlineto{\pgfqpoint{5.359129in}{3.490508in}}%
\pgfpathlineto{\pgfqpoint{5.359129in}{3.493457in}}%
\pgfpathlineto{\pgfqpoint{5.363670in}{3.493457in}}%
\pgfpathlineto{\pgfqpoint{5.363670in}{3.490508in}}%
\pgfpathmoveto{\pgfqpoint{5.359129in}{3.493457in}}%
\pgfpathlineto{\pgfqpoint{5.359129in}{3.493457in}}%
\pgfpathlineto{\pgfqpoint{5.359129in}{3.496406in}}%
\pgfpathlineto{\pgfqpoint{5.363670in}{3.496406in}}%
\pgfpathlineto{\pgfqpoint{5.363670in}{3.493457in}}%
\pgfpathmoveto{\pgfqpoint{5.363670in}{3.493457in}}%
\pgfpathlineto{\pgfqpoint{5.363670in}{3.493457in}}%
\pgfpathlineto{\pgfqpoint{5.363670in}{3.496406in}}%
\pgfpathlineto{\pgfqpoint{5.368211in}{3.496406in}}%
\pgfpathlineto{\pgfqpoint{5.368211in}{3.493457in}}%
\pgfpathmoveto{\pgfqpoint{5.363670in}{3.496406in}}%
\pgfpathlineto{\pgfqpoint{5.363670in}{3.496406in}}%
\pgfpathlineto{\pgfqpoint{5.363670in}{3.499356in}}%
\pgfpathlineto{\pgfqpoint{5.368211in}{3.499356in}}%
\pgfpathlineto{\pgfqpoint{5.368211in}{3.496406in}}%
\pgfpathmoveto{\pgfqpoint{5.368211in}{3.496406in}}%
\pgfpathlineto{\pgfqpoint{5.368211in}{3.496406in}}%
\pgfpathlineto{\pgfqpoint{5.368211in}{3.499356in}}%
\pgfpathlineto{\pgfqpoint{5.372752in}{3.499356in}}%
\pgfpathlineto{\pgfqpoint{5.372752in}{3.496406in}}%
\pgfpathmoveto{\pgfqpoint{5.368211in}{3.499356in}}%
\pgfpathlineto{\pgfqpoint{5.368211in}{3.499356in}}%
\pgfpathlineto{\pgfqpoint{5.368211in}{3.502305in}}%
\pgfpathlineto{\pgfqpoint{5.372752in}{3.502305in}}%
\pgfpathlineto{\pgfqpoint{5.372752in}{3.499356in}}%
\pgfpathmoveto{\pgfqpoint{5.372752in}{3.499356in}}%
\pgfpathlineto{\pgfqpoint{5.372752in}{3.499356in}}%
\pgfpathlineto{\pgfqpoint{5.372752in}{3.502305in}}%
\pgfpathlineto{\pgfqpoint{5.377292in}{3.502305in}}%
\pgfpathlineto{\pgfqpoint{5.377292in}{3.499356in}}%
\pgfpathmoveto{\pgfqpoint{5.372752in}{3.502305in}}%
\pgfpathlineto{\pgfqpoint{5.372752in}{3.502305in}}%
\pgfpathlineto{\pgfqpoint{5.372752in}{3.505254in}}%
\pgfpathlineto{\pgfqpoint{5.377292in}{3.505254in}}%
\pgfpathlineto{\pgfqpoint{5.377292in}{3.502305in}}%
\pgfpathmoveto{\pgfqpoint{5.377292in}{3.502305in}}%
\pgfpathlineto{\pgfqpoint{5.377292in}{3.502305in}}%
\pgfpathlineto{\pgfqpoint{5.377292in}{3.505254in}}%
\pgfpathlineto{\pgfqpoint{5.381833in}{3.505254in}}%
\pgfpathlineto{\pgfqpoint{5.381833in}{3.502305in}}%
\pgfpathmoveto{\pgfqpoint{5.377292in}{3.505254in}}%
\pgfpathlineto{\pgfqpoint{5.377292in}{3.505254in}}%
\pgfpathlineto{\pgfqpoint{5.377292in}{3.508204in}}%
\pgfpathlineto{\pgfqpoint{5.381833in}{3.508204in}}%
\pgfpathlineto{\pgfqpoint{5.381833in}{3.505254in}}%
\pgfpathmoveto{\pgfqpoint{5.381833in}{3.505254in}}%
\pgfpathlineto{\pgfqpoint{5.381833in}{3.505254in}}%
\pgfpathlineto{\pgfqpoint{5.381833in}{3.508204in}}%
\pgfpathlineto{\pgfqpoint{5.386374in}{3.508204in}}%
\pgfpathlineto{\pgfqpoint{5.386374in}{3.505254in}}%
\pgfpathmoveto{\pgfqpoint{5.381833in}{3.508204in}}%
\pgfpathlineto{\pgfqpoint{5.381833in}{3.508204in}}%
\pgfpathlineto{\pgfqpoint{5.381833in}{3.511153in}}%
\pgfpathlineto{\pgfqpoint{5.386374in}{3.511153in}}%
\pgfpathlineto{\pgfqpoint{5.386374in}{3.508204in}}%
\pgfpathmoveto{\pgfqpoint{5.386374in}{3.508204in}}%
\pgfpathlineto{\pgfqpoint{5.386374in}{3.508204in}}%
\pgfpathlineto{\pgfqpoint{5.386374in}{3.511153in}}%
\pgfpathlineto{\pgfqpoint{5.390915in}{3.511153in}}%
\pgfpathlineto{\pgfqpoint{5.390915in}{3.508204in}}%
\pgfpathmoveto{\pgfqpoint{5.386374in}{3.511153in}}%
\pgfpathlineto{\pgfqpoint{5.386374in}{3.511153in}}%
\pgfpathlineto{\pgfqpoint{5.386374in}{3.514102in}}%
\pgfpathlineto{\pgfqpoint{5.390915in}{3.514102in}}%
\pgfpathlineto{\pgfqpoint{5.390915in}{3.511153in}}%
\pgfpathmoveto{\pgfqpoint{5.390915in}{3.511153in}}%
\pgfpathlineto{\pgfqpoint{5.390915in}{3.511153in}}%
\pgfpathlineto{\pgfqpoint{5.390915in}{3.514102in}}%
\pgfpathlineto{\pgfqpoint{5.395456in}{3.514102in}}%
\pgfpathlineto{\pgfqpoint{5.395456in}{3.511153in}}%
\pgfpathmoveto{\pgfqpoint{5.390915in}{3.514102in}}%
\pgfpathlineto{\pgfqpoint{5.390915in}{3.514102in}}%
\pgfpathlineto{\pgfqpoint{5.390915in}{3.517051in}}%
\pgfpathlineto{\pgfqpoint{5.395456in}{3.517051in}}%
\pgfpathlineto{\pgfqpoint{5.395456in}{3.514102in}}%
\pgfpathmoveto{\pgfqpoint{5.395456in}{3.514102in}}%
\pgfpathlineto{\pgfqpoint{5.395456in}{3.514102in}}%
\pgfpathlineto{\pgfqpoint{5.395456in}{3.517051in}}%
\pgfpathlineto{\pgfqpoint{5.399996in}{3.517051in}}%
\pgfpathlineto{\pgfqpoint{5.399996in}{3.514102in}}%
\pgfpathmoveto{\pgfqpoint{5.395456in}{3.517051in}}%
\pgfpathlineto{\pgfqpoint{5.395456in}{3.517051in}}%
\pgfpathlineto{\pgfqpoint{5.395456in}{3.520001in}}%
\pgfpathlineto{\pgfqpoint{5.399996in}{3.520001in}}%
\pgfpathlineto{\pgfqpoint{5.399996in}{3.517051in}}%
\pgfpathclose%
\pgfusepath{fill}%
\end{pgfscope}%
\begin{pgfscope}%
\pgfsetbuttcap%
\pgfsetroundjoin%
\definecolor{currentfill}{rgb}{0.000000,0.000000,0.000000}%
\pgfsetfillcolor{currentfill}%
\pgfsetlinewidth{0.803000pt}%
\definecolor{currentstroke}{rgb}{0.000000,0.000000,0.000000}%
\pgfsetstrokecolor{currentstroke}%
\pgfsetdash{}{0pt}%
\pgfsys@defobject{currentmarker}{\pgfqpoint{0.000000in}{-0.048611in}}{\pgfqpoint{0.000000in}{0.000000in}}{%
\pgfpathmoveto{\pgfqpoint{0.000000in}{0.000000in}}%
\pgfpathlineto{\pgfqpoint{0.000000in}{-0.048611in}}%
\pgfusepath{stroke,fill}%
}%
\begin{pgfscope}%
\pgfsys@transformshift{1.215000in}{2.010000in}%
\pgfsys@useobject{currentmarker}{}%
\end{pgfscope}%
\end{pgfscope}%
\begin{pgfscope}%
\definecolor{textcolor}{rgb}{0.000000,0.000000,0.000000}%
\pgfsetstrokecolor{textcolor}%
\pgfsetfillcolor{textcolor}%
\pgftext[x=1.215000in,y=1.912778in,,top]{\color{textcolor}\sffamily\fontsize{10.000000}{12.000000}\selectfont −4}%
\end{pgfscope}%
\begin{pgfscope}%
\pgfsetbuttcap%
\pgfsetroundjoin%
\definecolor{currentfill}{rgb}{0.000000,0.000000,0.000000}%
\pgfsetfillcolor{currentfill}%
\pgfsetlinewidth{0.803000pt}%
\definecolor{currentstroke}{rgb}{0.000000,0.000000,0.000000}%
\pgfsetstrokecolor{currentstroke}%
\pgfsetdash{}{0pt}%
\pgfsys@defobject{currentmarker}{\pgfqpoint{0.000000in}{-0.048611in}}{\pgfqpoint{0.000000in}{0.000000in}}{%
\pgfpathmoveto{\pgfqpoint{0.000000in}{0.000000in}}%
\pgfpathlineto{\pgfqpoint{0.000000in}{-0.048611in}}%
\pgfusepath{stroke,fill}%
}%
\begin{pgfscope}%
\pgfsys@transformshift{2.145000in}{2.010000in}%
\pgfsys@useobject{currentmarker}{}%
\end{pgfscope}%
\end{pgfscope}%
\begin{pgfscope}%
\definecolor{textcolor}{rgb}{0.000000,0.000000,0.000000}%
\pgfsetstrokecolor{textcolor}%
\pgfsetfillcolor{textcolor}%
\pgftext[x=2.145000in,y=1.912778in,,top]{\color{textcolor}\sffamily\fontsize{10.000000}{12.000000}\selectfont −2}%
\end{pgfscope}%
\begin{pgfscope}%
\pgfsetbuttcap%
\pgfsetroundjoin%
\definecolor{currentfill}{rgb}{0.000000,0.000000,0.000000}%
\pgfsetfillcolor{currentfill}%
\pgfsetlinewidth{0.803000pt}%
\definecolor{currentstroke}{rgb}{0.000000,0.000000,0.000000}%
\pgfsetstrokecolor{currentstroke}%
\pgfsetdash{}{0pt}%
\pgfsys@defobject{currentmarker}{\pgfqpoint{0.000000in}{-0.048611in}}{\pgfqpoint{0.000000in}{0.000000in}}{%
\pgfpathmoveto{\pgfqpoint{0.000000in}{0.000000in}}%
\pgfpathlineto{\pgfqpoint{0.000000in}{-0.048611in}}%
\pgfusepath{stroke,fill}%
}%
\begin{pgfscope}%
\pgfsys@transformshift{3.075000in}{2.010000in}%
\pgfsys@useobject{currentmarker}{}%
\end{pgfscope}%
\end{pgfscope}%
\begin{pgfscope}%
\definecolor{textcolor}{rgb}{0.000000,0.000000,0.000000}%
\pgfsetstrokecolor{textcolor}%
\pgfsetfillcolor{textcolor}%
\pgftext[x=3.075000in,y=1.912778in,,top]{\color{textcolor}\sffamily\fontsize{10.000000}{12.000000}\selectfont 0}%
\end{pgfscope}%
\begin{pgfscope}%
\pgfsetbuttcap%
\pgfsetroundjoin%
\definecolor{currentfill}{rgb}{0.000000,0.000000,0.000000}%
\pgfsetfillcolor{currentfill}%
\pgfsetlinewidth{0.803000pt}%
\definecolor{currentstroke}{rgb}{0.000000,0.000000,0.000000}%
\pgfsetstrokecolor{currentstroke}%
\pgfsetdash{}{0pt}%
\pgfsys@defobject{currentmarker}{\pgfqpoint{0.000000in}{-0.048611in}}{\pgfqpoint{0.000000in}{0.000000in}}{%
\pgfpathmoveto{\pgfqpoint{0.000000in}{0.000000in}}%
\pgfpathlineto{\pgfqpoint{0.000000in}{-0.048611in}}%
\pgfusepath{stroke,fill}%
}%
\begin{pgfscope}%
\pgfsys@transformshift{4.005000in}{2.010000in}%
\pgfsys@useobject{currentmarker}{}%
\end{pgfscope}%
\end{pgfscope}%
\begin{pgfscope}%
\definecolor{textcolor}{rgb}{0.000000,0.000000,0.000000}%
\pgfsetstrokecolor{textcolor}%
\pgfsetfillcolor{textcolor}%
\pgftext[x=4.005000in,y=1.912778in,,top]{\color{textcolor}\sffamily\fontsize{10.000000}{12.000000}\selectfont 2}%
\end{pgfscope}%
\begin{pgfscope}%
\pgfsetbuttcap%
\pgfsetroundjoin%
\definecolor{currentfill}{rgb}{0.000000,0.000000,0.000000}%
\pgfsetfillcolor{currentfill}%
\pgfsetlinewidth{0.803000pt}%
\definecolor{currentstroke}{rgb}{0.000000,0.000000,0.000000}%
\pgfsetstrokecolor{currentstroke}%
\pgfsetdash{}{0pt}%
\pgfsys@defobject{currentmarker}{\pgfqpoint{0.000000in}{-0.048611in}}{\pgfqpoint{0.000000in}{0.000000in}}{%
\pgfpathmoveto{\pgfqpoint{0.000000in}{0.000000in}}%
\pgfpathlineto{\pgfqpoint{0.000000in}{-0.048611in}}%
\pgfusepath{stroke,fill}%
}%
\begin{pgfscope}%
\pgfsys@transformshift{4.935000in}{2.010000in}%
\pgfsys@useobject{currentmarker}{}%
\end{pgfscope}%
\end{pgfscope}%
\begin{pgfscope}%
\definecolor{textcolor}{rgb}{0.000000,0.000000,0.000000}%
\pgfsetstrokecolor{textcolor}%
\pgfsetfillcolor{textcolor}%
\pgftext[x=4.935000in,y=1.912778in,,top]{\color{textcolor}\sffamily\fontsize{10.000000}{12.000000}\selectfont 4}%
\end{pgfscope}%
\begin{pgfscope}%
\definecolor{textcolor}{rgb}{0.000000,0.000000,0.000000}%
\pgfsetstrokecolor{textcolor}%
\pgfsetfillcolor{textcolor}%
\pgftext[x=5.400000in,y=1.722809in,,top]{\color{textcolor}\sffamily\fontsize{10.000000}{12.000000}\selectfont x}%
\end{pgfscope}%
\begin{pgfscope}%
\pgfsetbuttcap%
\pgfsetroundjoin%
\definecolor{currentfill}{rgb}{0.000000,0.000000,0.000000}%
\pgfsetfillcolor{currentfill}%
\pgfsetlinewidth{0.803000pt}%
\definecolor{currentstroke}{rgb}{0.000000,0.000000,0.000000}%
\pgfsetstrokecolor{currentstroke}%
\pgfsetdash{}{0pt}%
\pgfsys@defobject{currentmarker}{\pgfqpoint{-0.048611in}{0.000000in}}{\pgfqpoint{-0.000000in}{0.000000in}}{%
\pgfpathmoveto{\pgfqpoint{-0.000000in}{0.000000in}}%
\pgfpathlineto{\pgfqpoint{-0.048611in}{0.000000in}}%
\pgfusepath{stroke,fill}%
}%
\begin{pgfscope}%
\pgfsys@transformshift{3.075000in}{0.802000in}%
\pgfsys@useobject{currentmarker}{}%
\end{pgfscope}%
\end{pgfscope}%
\begin{pgfscope}%
\definecolor{textcolor}{rgb}{0.000000,0.000000,0.000000}%
\pgfsetstrokecolor{textcolor}%
\pgfsetfillcolor{textcolor}%
\pgftext[x=2.773039in, y=0.749238in, left, base]{\color{textcolor}\sffamily\fontsize{10.000000}{12.000000}\selectfont −4}%
\end{pgfscope}%
\begin{pgfscope}%
\pgfsetbuttcap%
\pgfsetroundjoin%
\definecolor{currentfill}{rgb}{0.000000,0.000000,0.000000}%
\pgfsetfillcolor{currentfill}%
\pgfsetlinewidth{0.803000pt}%
\definecolor{currentstroke}{rgb}{0.000000,0.000000,0.000000}%
\pgfsetstrokecolor{currentstroke}%
\pgfsetdash{}{0pt}%
\pgfsys@defobject{currentmarker}{\pgfqpoint{-0.048611in}{0.000000in}}{\pgfqpoint{-0.000000in}{0.000000in}}{%
\pgfpathmoveto{\pgfqpoint{-0.000000in}{0.000000in}}%
\pgfpathlineto{\pgfqpoint{-0.048611in}{0.000000in}}%
\pgfusepath{stroke,fill}%
}%
\begin{pgfscope}%
\pgfsys@transformshift{3.075000in}{1.406000in}%
\pgfsys@useobject{currentmarker}{}%
\end{pgfscope}%
\end{pgfscope}%
\begin{pgfscope}%
\definecolor{textcolor}{rgb}{0.000000,0.000000,0.000000}%
\pgfsetstrokecolor{textcolor}%
\pgfsetfillcolor{textcolor}%
\pgftext[x=2.773039in, y=1.353238in, left, base]{\color{textcolor}\sffamily\fontsize{10.000000}{12.000000}\selectfont −2}%
\end{pgfscope}%
\begin{pgfscope}%
\pgfsetbuttcap%
\pgfsetroundjoin%
\definecolor{currentfill}{rgb}{0.000000,0.000000,0.000000}%
\pgfsetfillcolor{currentfill}%
\pgfsetlinewidth{0.803000pt}%
\definecolor{currentstroke}{rgb}{0.000000,0.000000,0.000000}%
\pgfsetstrokecolor{currentstroke}%
\pgfsetdash{}{0pt}%
\pgfsys@defobject{currentmarker}{\pgfqpoint{-0.048611in}{0.000000in}}{\pgfqpoint{-0.000000in}{0.000000in}}{%
\pgfpathmoveto{\pgfqpoint{-0.000000in}{0.000000in}}%
\pgfpathlineto{\pgfqpoint{-0.048611in}{0.000000in}}%
\pgfusepath{stroke,fill}%
}%
\begin{pgfscope}%
\pgfsys@transformshift{3.075000in}{2.010000in}%
\pgfsys@useobject{currentmarker}{}%
\end{pgfscope}%
\end{pgfscope}%
\begin{pgfscope}%
\definecolor{textcolor}{rgb}{0.000000,0.000000,0.000000}%
\pgfsetstrokecolor{textcolor}%
\pgfsetfillcolor{textcolor}%
\pgftext[x=2.889413in, y=1.957238in, left, base]{\color{textcolor}\sffamily\fontsize{10.000000}{12.000000}\selectfont 0}%
\end{pgfscope}%
\begin{pgfscope}%
\pgfsetbuttcap%
\pgfsetroundjoin%
\definecolor{currentfill}{rgb}{0.000000,0.000000,0.000000}%
\pgfsetfillcolor{currentfill}%
\pgfsetlinewidth{0.803000pt}%
\definecolor{currentstroke}{rgb}{0.000000,0.000000,0.000000}%
\pgfsetstrokecolor{currentstroke}%
\pgfsetdash{}{0pt}%
\pgfsys@defobject{currentmarker}{\pgfqpoint{-0.048611in}{0.000000in}}{\pgfqpoint{-0.000000in}{0.000000in}}{%
\pgfpathmoveto{\pgfqpoint{-0.000000in}{0.000000in}}%
\pgfpathlineto{\pgfqpoint{-0.048611in}{0.000000in}}%
\pgfusepath{stroke,fill}%
}%
\begin{pgfscope}%
\pgfsys@transformshift{3.075000in}{2.614000in}%
\pgfsys@useobject{currentmarker}{}%
\end{pgfscope}%
\end{pgfscope}%
\begin{pgfscope}%
\definecolor{textcolor}{rgb}{0.000000,0.000000,0.000000}%
\pgfsetstrokecolor{textcolor}%
\pgfsetfillcolor{textcolor}%
\pgftext[x=2.889413in, y=2.561238in, left, base]{\color{textcolor}\sffamily\fontsize{10.000000}{12.000000}\selectfont 2}%
\end{pgfscope}%
\begin{pgfscope}%
\pgfsetbuttcap%
\pgfsetroundjoin%
\definecolor{currentfill}{rgb}{0.000000,0.000000,0.000000}%
\pgfsetfillcolor{currentfill}%
\pgfsetlinewidth{0.803000pt}%
\definecolor{currentstroke}{rgb}{0.000000,0.000000,0.000000}%
\pgfsetstrokecolor{currentstroke}%
\pgfsetdash{}{0pt}%
\pgfsys@defobject{currentmarker}{\pgfqpoint{-0.048611in}{0.000000in}}{\pgfqpoint{-0.000000in}{0.000000in}}{%
\pgfpathmoveto{\pgfqpoint{-0.000000in}{0.000000in}}%
\pgfpathlineto{\pgfqpoint{-0.048611in}{0.000000in}}%
\pgfusepath{stroke,fill}%
}%
\begin{pgfscope}%
\pgfsys@transformshift{3.075000in}{3.218000in}%
\pgfsys@useobject{currentmarker}{}%
\end{pgfscope}%
\end{pgfscope}%
\begin{pgfscope}%
\definecolor{textcolor}{rgb}{0.000000,0.000000,0.000000}%
\pgfsetstrokecolor{textcolor}%
\pgfsetfillcolor{textcolor}%
\pgftext[x=2.889413in, y=3.165238in, left, base]{\color{textcolor}\sffamily\fontsize{10.000000}{12.000000}\selectfont 4}%
\end{pgfscope}%
\begin{pgfscope}%
\definecolor{textcolor}{rgb}{0.000000,0.000000,0.000000}%
\pgfsetstrokecolor{textcolor}%
\pgfsetfillcolor{textcolor}%
\pgftext[x=2.717483in,y=3.520000in,,bottom,rotate=90.000000]{\color{textcolor}\sffamily\fontsize{10.000000}{12.000000}\selectfont y}%
\end{pgfscope}%
\begin{pgfscope}%
\pgfsetrectcap%
\pgfsetmiterjoin%
\pgfsetlinewidth{0.803000pt}%
\definecolor{currentstroke}{rgb}{0.000000,0.000000,0.000000}%
\pgfsetstrokecolor{currentstroke}%
\pgfsetdash{}{0pt}%
\pgfpathmoveto{\pgfqpoint{3.075000in}{0.500000in}}%
\pgfpathlineto{\pgfqpoint{3.075000in}{3.520000in}}%
\pgfusepath{stroke}%
\end{pgfscope}%
\begin{pgfscope}%
\pgfsetrectcap%
\pgfsetmiterjoin%
\pgfsetlinewidth{0.000000pt}%
\definecolor{currentstroke}{rgb}{0.000000,0.000000,0.000000}%
\pgfsetstrokecolor{currentstroke}%
\pgfsetstrokeopacity{0.000000}%
\pgfsetdash{}{0pt}%
\pgfpathmoveto{\pgfqpoint{5.400000in}{0.500000in}}%
\pgfpathlineto{\pgfqpoint{5.400000in}{3.520000in}}%
\pgfusepath{}%
\end{pgfscope}%
\begin{pgfscope}%
\pgfsetrectcap%
\pgfsetmiterjoin%
\pgfsetlinewidth{0.803000pt}%
\definecolor{currentstroke}{rgb}{0.000000,0.000000,0.000000}%
\pgfsetstrokecolor{currentstroke}%
\pgfsetdash{}{0pt}%
\pgfpathmoveto{\pgfqpoint{0.750000in}{2.010000in}}%
\pgfpathlineto{\pgfqpoint{5.400000in}{2.010000in}}%
\pgfusepath{stroke}%
\end{pgfscope}%
\begin{pgfscope}%
\pgfsetrectcap%
\pgfsetmiterjoin%
\pgfsetlinewidth{0.000000pt}%
\definecolor{currentstroke}{rgb}{0.000000,0.000000,0.000000}%
\pgfsetstrokecolor{currentstroke}%
\pgfsetstrokeopacity{0.000000}%
\pgfsetdash{}{0pt}%
\pgfpathmoveto{\pgfqpoint{0.750000in}{3.520000in}}%
\pgfpathlineto{\pgfqpoint{5.400000in}{3.520000in}}%
\pgfusepath{}%
\end{pgfscope}%
\end{pgfpicture}%
\makeatother%
\endgroup%
}
\end{solution}
\part[]  $\left\{ {\begin{matrix}
   {y \leqslant 2}  \\ 
   {x + y \leqslant 3}  \\ 
   {x \geqslant 0}  \\ 
   {y \geqslant 0}  \\ 

 \end{matrix} } \right.$
\begin{solution} \scalebox{.6}{%% Creator: Matplotlib, PGF backend
%%
%% To include the figure in your LaTeX document, write
%%   \input{<filename>.pgf}
%%
%% Make sure the required packages are loaded in your preamble
%%   \usepackage{pgf}
%%
%% and, on pdftex
%%   \usepackage[utf8]{inputenc}\DeclareUnicodeCharacter{2212}{-}
%%
%% or, on luatex and xetex
%%   \usepackage{unicode-math}
%%
%% Figures using additional raster images can only be included by \input if
%% they are in the same directory as the main LaTeX file. For loading figures
%% from other directories you can use the `import` package
%%   \usepackage{import}
%%
%% and then include the figures with
%%   \import{<path to file>}{<filename>.pgf}
%%
%% Matplotlib used the following preamble
%%   \usepackage{fontspec}
%%   \setmainfont{DejaVuSerif.ttf}[Path=/home/hp/Mis_aplicaciones/anaconda3/lib/python3.6/site-packages/matplotlib/mpl-data/fonts/ttf/]
%%   \setsansfont{DejaVuSans.ttf}[Path=/home/hp/Mis_aplicaciones/anaconda3/lib/python3.6/site-packages/matplotlib/mpl-data/fonts/ttf/]
%%   \setmonofont{DejaVuSansMono.ttf}[Path=/home/hp/Mis_aplicaciones/anaconda3/lib/python3.6/site-packages/matplotlib/mpl-data/fonts/ttf/]
%%
\begingroup%
\makeatletter%
\begin{pgfpicture}%
\pgfpathrectangle{\pgfpointorigin}{\pgfqpoint{6.000000in}{4.000000in}}%
\pgfusepath{use as bounding box, clip}%
\begin{pgfscope}%
\pgfsetbuttcap%
\pgfsetmiterjoin%
\pgfsetlinewidth{0.000000pt}%
\definecolor{currentstroke}{rgb}{1.000000,1.000000,1.000000}%
\pgfsetstrokecolor{currentstroke}%
\pgfsetstrokeopacity{0.000000}%
\pgfsetdash{}{0pt}%
\pgfpathmoveto{\pgfqpoint{0.000000in}{0.000000in}}%
\pgfpathlineto{\pgfqpoint{6.000000in}{0.000000in}}%
\pgfpathlineto{\pgfqpoint{6.000000in}{4.000000in}}%
\pgfpathlineto{\pgfqpoint{0.000000in}{4.000000in}}%
\pgfpathclose%
\pgfusepath{}%
\end{pgfscope}%
\begin{pgfscope}%
\pgfsetbuttcap%
\pgfsetmiterjoin%
\definecolor{currentfill}{rgb}{1.000000,1.000000,1.000000}%
\pgfsetfillcolor{currentfill}%
\pgfsetlinewidth{0.000000pt}%
\definecolor{currentstroke}{rgb}{0.000000,0.000000,0.000000}%
\pgfsetstrokecolor{currentstroke}%
\pgfsetstrokeopacity{0.000000}%
\pgfsetdash{}{0pt}%
\pgfpathmoveto{\pgfqpoint{0.750000in}{0.500000in}}%
\pgfpathlineto{\pgfqpoint{5.400000in}{0.500000in}}%
\pgfpathlineto{\pgfqpoint{5.400000in}{3.520000in}}%
\pgfpathlineto{\pgfqpoint{0.750000in}{3.520000in}}%
\pgfpathclose%
\pgfusepath{fill}%
\end{pgfscope}%
\begin{pgfscope}%
\pgfpathrectangle{\pgfqpoint{0.750000in}{0.500000in}}{\pgfqpoint{4.650000in}{3.020000in}}%
\pgfusepath{clip}%
\pgfsetbuttcap%
\pgfsetmiterjoin%
\definecolor{currentfill}{rgb}{0.000000,0.000000,1.000000}%
\pgfsetfillcolor{currentfill}%
\pgfsetlinewidth{0.000000pt}%
\definecolor{currentstroke}{rgb}{0.000000,0.000000,0.000000}%
\pgfsetstrokecolor{currentstroke}%
\pgfsetstrokeopacity{0.000000}%
\pgfsetdash{}{0pt}%
\pgfpathmoveto{\pgfqpoint{3.075004in}{2.104378in}}%
\pgfpathlineto{\pgfqpoint{3.075004in}{2.198748in}}%
\pgfpathlineto{\pgfqpoint{3.220311in}{2.198748in}}%
\pgfpathlineto{\pgfqpoint{3.220311in}{2.104378in}}%
\pgfpathmoveto{\pgfqpoint{3.075004in}{2.198748in}}%
\pgfpathlineto{\pgfqpoint{3.075004in}{2.198748in}}%
\pgfpathlineto{\pgfqpoint{3.075004in}{2.293125in}}%
\pgfpathlineto{\pgfqpoint{3.220311in}{2.293125in}}%
\pgfpathlineto{\pgfqpoint{3.220311in}{2.198748in}}%
\pgfpathmoveto{\pgfqpoint{3.075004in}{2.293125in}}%
\pgfpathlineto{\pgfqpoint{3.075004in}{2.293125in}}%
\pgfpathlineto{\pgfqpoint{3.075004in}{2.387502in}}%
\pgfpathlineto{\pgfqpoint{3.220311in}{2.387502in}}%
\pgfpathlineto{\pgfqpoint{3.220311in}{2.293125in}}%
\pgfpathmoveto{\pgfqpoint{3.075004in}{2.387502in}}%
\pgfpathlineto{\pgfqpoint{3.075004in}{2.387502in}}%
\pgfpathlineto{\pgfqpoint{3.075004in}{2.481873in}}%
\pgfpathlineto{\pgfqpoint{3.220311in}{2.481873in}}%
\pgfpathlineto{\pgfqpoint{3.220311in}{2.387502in}}%
\pgfpathmoveto{\pgfqpoint{3.075004in}{2.481873in}}%
\pgfpathlineto{\pgfqpoint{3.075004in}{2.481873in}}%
\pgfpathlineto{\pgfqpoint{3.075004in}{2.576249in}}%
\pgfpathlineto{\pgfqpoint{3.220311in}{2.576249in}}%
\pgfpathlineto{\pgfqpoint{3.220311in}{2.481873in}}%
\pgfpathmoveto{\pgfqpoint{3.220311in}{2.104378in}}%
\pgfpathlineto{\pgfqpoint{3.220311in}{2.104378in}}%
\pgfpathlineto{\pgfqpoint{3.220311in}{2.198748in}}%
\pgfpathlineto{\pgfqpoint{3.365625in}{2.198748in}}%
\pgfpathlineto{\pgfqpoint{3.365625in}{2.104378in}}%
\pgfpathmoveto{\pgfqpoint{3.220311in}{2.198748in}}%
\pgfpathlineto{\pgfqpoint{3.220311in}{2.198748in}}%
\pgfpathlineto{\pgfqpoint{3.220311in}{2.293125in}}%
\pgfpathlineto{\pgfqpoint{3.365625in}{2.293125in}}%
\pgfpathlineto{\pgfqpoint{3.365625in}{2.198748in}}%
\pgfpathmoveto{\pgfqpoint{3.220311in}{2.293125in}}%
\pgfpathlineto{\pgfqpoint{3.220311in}{2.293125in}}%
\pgfpathlineto{\pgfqpoint{3.220311in}{2.387502in}}%
\pgfpathlineto{\pgfqpoint{3.365625in}{2.387502in}}%
\pgfpathlineto{\pgfqpoint{3.365625in}{2.293125in}}%
\pgfpathmoveto{\pgfqpoint{3.220311in}{2.387502in}}%
\pgfpathlineto{\pgfqpoint{3.220311in}{2.387502in}}%
\pgfpathlineto{\pgfqpoint{3.220311in}{2.481873in}}%
\pgfpathlineto{\pgfqpoint{3.365625in}{2.481873in}}%
\pgfpathlineto{\pgfqpoint{3.365625in}{2.387502in}}%
\pgfpathmoveto{\pgfqpoint{3.220311in}{2.481873in}}%
\pgfpathlineto{\pgfqpoint{3.220311in}{2.481873in}}%
\pgfpathlineto{\pgfqpoint{3.220311in}{2.576249in}}%
\pgfpathlineto{\pgfqpoint{3.365625in}{2.576249in}}%
\pgfpathlineto{\pgfqpoint{3.365625in}{2.481873in}}%
\pgfpathmoveto{\pgfqpoint{3.365625in}{2.104378in}}%
\pgfpathlineto{\pgfqpoint{3.365625in}{2.104378in}}%
\pgfpathlineto{\pgfqpoint{3.365625in}{2.198748in}}%
\pgfpathlineto{\pgfqpoint{3.510939in}{2.198748in}}%
\pgfpathlineto{\pgfqpoint{3.510939in}{2.104378in}}%
\pgfpathmoveto{\pgfqpoint{3.365625in}{2.198748in}}%
\pgfpathlineto{\pgfqpoint{3.365625in}{2.198748in}}%
\pgfpathlineto{\pgfqpoint{3.365625in}{2.293125in}}%
\pgfpathlineto{\pgfqpoint{3.510939in}{2.293125in}}%
\pgfpathlineto{\pgfqpoint{3.510939in}{2.198748in}}%
\pgfpathmoveto{\pgfqpoint{3.365625in}{2.293125in}}%
\pgfpathlineto{\pgfqpoint{3.365625in}{2.293125in}}%
\pgfpathlineto{\pgfqpoint{3.365625in}{2.387502in}}%
\pgfpathlineto{\pgfqpoint{3.510939in}{2.387502in}}%
\pgfpathlineto{\pgfqpoint{3.510939in}{2.293125in}}%
\pgfpathmoveto{\pgfqpoint{3.365625in}{2.387502in}}%
\pgfpathlineto{\pgfqpoint{3.365625in}{2.387502in}}%
\pgfpathlineto{\pgfqpoint{3.365625in}{2.481873in}}%
\pgfpathlineto{\pgfqpoint{3.510939in}{2.481873in}}%
\pgfpathlineto{\pgfqpoint{3.510939in}{2.387502in}}%
\pgfpathmoveto{\pgfqpoint{3.365625in}{2.481873in}}%
\pgfpathlineto{\pgfqpoint{3.365625in}{2.481873in}}%
\pgfpathlineto{\pgfqpoint{3.365625in}{2.576249in}}%
\pgfpathlineto{\pgfqpoint{3.510939in}{2.576249in}}%
\pgfpathlineto{\pgfqpoint{3.510939in}{2.481873in}}%
\pgfpathmoveto{\pgfqpoint{3.510939in}{2.104378in}}%
\pgfpathlineto{\pgfqpoint{3.510939in}{2.104378in}}%
\pgfpathlineto{\pgfqpoint{3.510939in}{2.198748in}}%
\pgfpathlineto{\pgfqpoint{3.656253in}{2.198748in}}%
\pgfpathlineto{\pgfqpoint{3.656253in}{2.104378in}}%
\pgfpathmoveto{\pgfqpoint{3.510939in}{2.198748in}}%
\pgfpathlineto{\pgfqpoint{3.510939in}{2.198748in}}%
\pgfpathlineto{\pgfqpoint{3.510939in}{2.293125in}}%
\pgfpathlineto{\pgfqpoint{3.656253in}{2.293125in}}%
\pgfpathlineto{\pgfqpoint{3.656253in}{2.198748in}}%
\pgfpathmoveto{\pgfqpoint{3.510939in}{2.293125in}}%
\pgfpathlineto{\pgfqpoint{3.510939in}{2.293125in}}%
\pgfpathlineto{\pgfqpoint{3.510939in}{2.387502in}}%
\pgfpathlineto{\pgfqpoint{3.656253in}{2.387502in}}%
\pgfpathlineto{\pgfqpoint{3.656253in}{2.293125in}}%
\pgfpathmoveto{\pgfqpoint{3.510939in}{2.387502in}}%
\pgfpathlineto{\pgfqpoint{3.510939in}{2.387502in}}%
\pgfpathlineto{\pgfqpoint{3.510939in}{2.481873in}}%
\pgfpathlineto{\pgfqpoint{3.656253in}{2.481873in}}%
\pgfpathlineto{\pgfqpoint{3.656253in}{2.387502in}}%
\pgfpathmoveto{\pgfqpoint{3.656253in}{2.104378in}}%
\pgfpathlineto{\pgfqpoint{3.656253in}{2.104378in}}%
\pgfpathlineto{\pgfqpoint{3.656253in}{2.198748in}}%
\pgfpathlineto{\pgfqpoint{3.801561in}{2.198748in}}%
\pgfpathlineto{\pgfqpoint{3.801561in}{2.104378in}}%
\pgfpathmoveto{\pgfqpoint{3.656253in}{2.198748in}}%
\pgfpathlineto{\pgfqpoint{3.656253in}{2.198748in}}%
\pgfpathlineto{\pgfqpoint{3.656253in}{2.293125in}}%
\pgfpathlineto{\pgfqpoint{3.801561in}{2.293125in}}%
\pgfpathlineto{\pgfqpoint{3.801561in}{2.198748in}}%
\pgfpathmoveto{\pgfqpoint{3.656253in}{2.293125in}}%
\pgfpathlineto{\pgfqpoint{3.656253in}{2.293125in}}%
\pgfpathlineto{\pgfqpoint{3.656253in}{2.387502in}}%
\pgfpathlineto{\pgfqpoint{3.801561in}{2.387502in}}%
\pgfpathlineto{\pgfqpoint{3.801561in}{2.293125in}}%
\pgfpathmoveto{\pgfqpoint{3.801561in}{2.104378in}}%
\pgfpathlineto{\pgfqpoint{3.801561in}{2.104378in}}%
\pgfpathlineto{\pgfqpoint{3.801561in}{2.198748in}}%
\pgfpathlineto{\pgfqpoint{3.946874in}{2.198748in}}%
\pgfpathlineto{\pgfqpoint{3.946874in}{2.104378in}}%
\pgfpathmoveto{\pgfqpoint{3.801561in}{2.198748in}}%
\pgfpathlineto{\pgfqpoint{3.801561in}{2.198748in}}%
\pgfpathlineto{\pgfqpoint{3.801561in}{2.293125in}}%
\pgfpathlineto{\pgfqpoint{3.946874in}{2.293125in}}%
\pgfpathlineto{\pgfqpoint{3.946874in}{2.198748in}}%
\pgfpathmoveto{\pgfqpoint{3.946874in}{2.104378in}}%
\pgfpathlineto{\pgfqpoint{3.946874in}{2.104378in}}%
\pgfpathlineto{\pgfqpoint{3.946874in}{2.198748in}}%
\pgfpathlineto{\pgfqpoint{4.092185in}{2.198748in}}%
\pgfpathlineto{\pgfqpoint{4.092185in}{2.104378in}}%
\pgfpathmoveto{\pgfqpoint{3.075004in}{2.057187in}}%
\pgfpathlineto{\pgfqpoint{3.075004in}{2.057187in}}%
\pgfpathlineto{\pgfqpoint{3.075004in}{2.104378in}}%
\pgfpathlineto{\pgfqpoint{3.147658in}{2.104378in}}%
\pgfpathlineto{\pgfqpoint{3.147658in}{2.057187in}}%
\pgfpathmoveto{\pgfqpoint{3.147658in}{2.057187in}}%
\pgfpathlineto{\pgfqpoint{3.147658in}{2.057187in}}%
\pgfpathlineto{\pgfqpoint{3.147658in}{2.104378in}}%
\pgfpathlineto{\pgfqpoint{3.220311in}{2.104378in}}%
\pgfpathlineto{\pgfqpoint{3.220311in}{2.057187in}}%
\pgfpathmoveto{\pgfqpoint{3.220311in}{2.057187in}}%
\pgfpathlineto{\pgfqpoint{3.220311in}{2.057187in}}%
\pgfpathlineto{\pgfqpoint{3.220311in}{2.104378in}}%
\pgfpathlineto{\pgfqpoint{3.292968in}{2.104378in}}%
\pgfpathlineto{\pgfqpoint{3.292968in}{2.057187in}}%
\pgfpathmoveto{\pgfqpoint{3.292968in}{2.057187in}}%
\pgfpathlineto{\pgfqpoint{3.292968in}{2.057187in}}%
\pgfpathlineto{\pgfqpoint{3.292968in}{2.104378in}}%
\pgfpathlineto{\pgfqpoint{3.365625in}{2.104378in}}%
\pgfpathlineto{\pgfqpoint{3.365625in}{2.057187in}}%
\pgfpathmoveto{\pgfqpoint{3.365625in}{2.057187in}}%
\pgfpathlineto{\pgfqpoint{3.365625in}{2.057187in}}%
\pgfpathlineto{\pgfqpoint{3.365625in}{2.104378in}}%
\pgfpathlineto{\pgfqpoint{3.438282in}{2.104378in}}%
\pgfpathlineto{\pgfqpoint{3.438282in}{2.057187in}}%
\pgfpathmoveto{\pgfqpoint{3.438282in}{2.057187in}}%
\pgfpathlineto{\pgfqpoint{3.438282in}{2.057187in}}%
\pgfpathlineto{\pgfqpoint{3.438282in}{2.104378in}}%
\pgfpathlineto{\pgfqpoint{3.510939in}{2.104378in}}%
\pgfpathlineto{\pgfqpoint{3.510939in}{2.057187in}}%
\pgfpathmoveto{\pgfqpoint{3.510939in}{2.057187in}}%
\pgfpathlineto{\pgfqpoint{3.510939in}{2.057187in}}%
\pgfpathlineto{\pgfqpoint{3.510939in}{2.104378in}}%
\pgfpathlineto{\pgfqpoint{3.583596in}{2.104378in}}%
\pgfpathlineto{\pgfqpoint{3.583596in}{2.057187in}}%
\pgfpathmoveto{\pgfqpoint{3.583596in}{2.057187in}}%
\pgfpathlineto{\pgfqpoint{3.583596in}{2.057187in}}%
\pgfpathlineto{\pgfqpoint{3.583596in}{2.104378in}}%
\pgfpathlineto{\pgfqpoint{3.656253in}{2.104378in}}%
\pgfpathlineto{\pgfqpoint{3.656253in}{2.057187in}}%
\pgfpathmoveto{\pgfqpoint{3.510939in}{2.481873in}}%
\pgfpathlineto{\pgfqpoint{3.510939in}{2.481873in}}%
\pgfpathlineto{\pgfqpoint{3.510939in}{2.529061in}}%
\pgfpathlineto{\pgfqpoint{3.583596in}{2.529061in}}%
\pgfpathlineto{\pgfqpoint{3.583596in}{2.481873in}}%
\pgfpathmoveto{\pgfqpoint{3.510939in}{2.529061in}}%
\pgfpathlineto{\pgfqpoint{3.510939in}{2.529061in}}%
\pgfpathlineto{\pgfqpoint{3.510939in}{2.576249in}}%
\pgfpathlineto{\pgfqpoint{3.583596in}{2.576249in}}%
\pgfpathlineto{\pgfqpoint{3.583596in}{2.529061in}}%
\pgfpathmoveto{\pgfqpoint{3.583596in}{2.481873in}}%
\pgfpathlineto{\pgfqpoint{3.583596in}{2.481873in}}%
\pgfpathlineto{\pgfqpoint{3.583596in}{2.529061in}}%
\pgfpathlineto{\pgfqpoint{3.656253in}{2.529061in}}%
\pgfpathlineto{\pgfqpoint{3.656253in}{2.481873in}}%
\pgfpathmoveto{\pgfqpoint{3.656253in}{2.057187in}}%
\pgfpathlineto{\pgfqpoint{3.656253in}{2.057187in}}%
\pgfpathlineto{\pgfqpoint{3.656253in}{2.104378in}}%
\pgfpathlineto{\pgfqpoint{3.728907in}{2.104378in}}%
\pgfpathlineto{\pgfqpoint{3.728907in}{2.057187in}}%
\pgfpathmoveto{\pgfqpoint{3.728907in}{2.057187in}}%
\pgfpathlineto{\pgfqpoint{3.728907in}{2.057187in}}%
\pgfpathlineto{\pgfqpoint{3.728907in}{2.104378in}}%
\pgfpathlineto{\pgfqpoint{3.801561in}{2.104378in}}%
\pgfpathlineto{\pgfqpoint{3.801561in}{2.057187in}}%
\pgfpathmoveto{\pgfqpoint{3.656253in}{2.387502in}}%
\pgfpathlineto{\pgfqpoint{3.656253in}{2.387502in}}%
\pgfpathlineto{\pgfqpoint{3.656253in}{2.434687in}}%
\pgfpathlineto{\pgfqpoint{3.728907in}{2.434687in}}%
\pgfpathlineto{\pgfqpoint{3.728907in}{2.387502in}}%
\pgfpathmoveto{\pgfqpoint{3.656253in}{2.434687in}}%
\pgfpathlineto{\pgfqpoint{3.656253in}{2.434687in}}%
\pgfpathlineto{\pgfqpoint{3.656253in}{2.481873in}}%
\pgfpathlineto{\pgfqpoint{3.728907in}{2.481873in}}%
\pgfpathlineto{\pgfqpoint{3.728907in}{2.434687in}}%
\pgfpathmoveto{\pgfqpoint{3.728907in}{2.387502in}}%
\pgfpathlineto{\pgfqpoint{3.728907in}{2.387502in}}%
\pgfpathlineto{\pgfqpoint{3.728907in}{2.434687in}}%
\pgfpathlineto{\pgfqpoint{3.801561in}{2.434687in}}%
\pgfpathlineto{\pgfqpoint{3.801561in}{2.387502in}}%
\pgfpathmoveto{\pgfqpoint{3.801561in}{2.057187in}}%
\pgfpathlineto{\pgfqpoint{3.801561in}{2.057187in}}%
\pgfpathlineto{\pgfqpoint{3.801561in}{2.104378in}}%
\pgfpathlineto{\pgfqpoint{3.874218in}{2.104378in}}%
\pgfpathlineto{\pgfqpoint{3.874218in}{2.057187in}}%
\pgfpathmoveto{\pgfqpoint{3.874218in}{2.057187in}}%
\pgfpathlineto{\pgfqpoint{3.874218in}{2.057187in}}%
\pgfpathlineto{\pgfqpoint{3.874218in}{2.104378in}}%
\pgfpathlineto{\pgfqpoint{3.946874in}{2.104378in}}%
\pgfpathlineto{\pgfqpoint{3.946874in}{2.057187in}}%
\pgfpathmoveto{\pgfqpoint{3.801561in}{2.293125in}}%
\pgfpathlineto{\pgfqpoint{3.801561in}{2.293125in}}%
\pgfpathlineto{\pgfqpoint{3.801561in}{2.340313in}}%
\pgfpathlineto{\pgfqpoint{3.874218in}{2.340313in}}%
\pgfpathlineto{\pgfqpoint{3.874218in}{2.293125in}}%
\pgfpathmoveto{\pgfqpoint{3.801561in}{2.340313in}}%
\pgfpathlineto{\pgfqpoint{3.801561in}{2.340313in}}%
\pgfpathlineto{\pgfqpoint{3.801561in}{2.387502in}}%
\pgfpathlineto{\pgfqpoint{3.874218in}{2.387502in}}%
\pgfpathlineto{\pgfqpoint{3.874218in}{2.340313in}}%
\pgfpathmoveto{\pgfqpoint{3.874218in}{2.293125in}}%
\pgfpathlineto{\pgfqpoint{3.874218in}{2.293125in}}%
\pgfpathlineto{\pgfqpoint{3.874218in}{2.340313in}}%
\pgfpathlineto{\pgfqpoint{3.946874in}{2.340313in}}%
\pgfpathlineto{\pgfqpoint{3.946874in}{2.293125in}}%
\pgfpathmoveto{\pgfqpoint{3.946874in}{2.057187in}}%
\pgfpathlineto{\pgfqpoint{3.946874in}{2.057187in}}%
\pgfpathlineto{\pgfqpoint{3.946874in}{2.104378in}}%
\pgfpathlineto{\pgfqpoint{4.019529in}{2.104378in}}%
\pgfpathlineto{\pgfqpoint{4.019529in}{2.057187in}}%
\pgfpathmoveto{\pgfqpoint{4.019529in}{2.057187in}}%
\pgfpathlineto{\pgfqpoint{4.019529in}{2.057187in}}%
\pgfpathlineto{\pgfqpoint{4.019529in}{2.104378in}}%
\pgfpathlineto{\pgfqpoint{4.092185in}{2.104378in}}%
\pgfpathlineto{\pgfqpoint{4.092185in}{2.057187in}}%
\pgfpathmoveto{\pgfqpoint{3.946874in}{2.198748in}}%
\pgfpathlineto{\pgfqpoint{3.946874in}{2.198748in}}%
\pgfpathlineto{\pgfqpoint{3.946874in}{2.245936in}}%
\pgfpathlineto{\pgfqpoint{4.019529in}{2.245936in}}%
\pgfpathlineto{\pgfqpoint{4.019529in}{2.198748in}}%
\pgfpathmoveto{\pgfqpoint{3.946874in}{2.245936in}}%
\pgfpathlineto{\pgfqpoint{3.946874in}{2.245936in}}%
\pgfpathlineto{\pgfqpoint{3.946874in}{2.293125in}}%
\pgfpathlineto{\pgfqpoint{4.019529in}{2.293125in}}%
\pgfpathlineto{\pgfqpoint{4.019529in}{2.245936in}}%
\pgfpathmoveto{\pgfqpoint{4.019529in}{2.198748in}}%
\pgfpathlineto{\pgfqpoint{4.019529in}{2.198748in}}%
\pgfpathlineto{\pgfqpoint{4.019529in}{2.245936in}}%
\pgfpathlineto{\pgfqpoint{4.092185in}{2.245936in}}%
\pgfpathlineto{\pgfqpoint{4.092185in}{2.198748in}}%
\pgfpathmoveto{\pgfqpoint{4.092185in}{2.057187in}}%
\pgfpathlineto{\pgfqpoint{4.092185in}{2.057187in}}%
\pgfpathlineto{\pgfqpoint{4.092185in}{2.104378in}}%
\pgfpathlineto{\pgfqpoint{4.164844in}{2.104378in}}%
\pgfpathlineto{\pgfqpoint{4.164844in}{2.057187in}}%
\pgfpathmoveto{\pgfqpoint{4.164844in}{2.057187in}}%
\pgfpathlineto{\pgfqpoint{4.164844in}{2.057187in}}%
\pgfpathlineto{\pgfqpoint{4.164844in}{2.104378in}}%
\pgfpathlineto{\pgfqpoint{4.237504in}{2.104378in}}%
\pgfpathlineto{\pgfqpoint{4.237504in}{2.057187in}}%
\pgfpathmoveto{\pgfqpoint{4.092185in}{2.104378in}}%
\pgfpathlineto{\pgfqpoint{4.092185in}{2.104378in}}%
\pgfpathlineto{\pgfqpoint{4.092185in}{2.151563in}}%
\pgfpathlineto{\pgfqpoint{4.164844in}{2.151563in}}%
\pgfpathlineto{\pgfqpoint{4.164844in}{2.104378in}}%
\pgfpathmoveto{\pgfqpoint{4.092185in}{2.151563in}}%
\pgfpathlineto{\pgfqpoint{4.092185in}{2.151563in}}%
\pgfpathlineto{\pgfqpoint{4.092185in}{2.198748in}}%
\pgfpathlineto{\pgfqpoint{4.164844in}{2.198748in}}%
\pgfpathlineto{\pgfqpoint{4.164844in}{2.151563in}}%
\pgfpathmoveto{\pgfqpoint{4.164844in}{2.104378in}}%
\pgfpathlineto{\pgfqpoint{4.164844in}{2.104378in}}%
\pgfpathlineto{\pgfqpoint{4.164844in}{2.151563in}}%
\pgfpathlineto{\pgfqpoint{4.237504in}{2.151563in}}%
\pgfpathlineto{\pgfqpoint{4.237504in}{2.104378in}}%
\pgfpathmoveto{\pgfqpoint{4.237504in}{2.057187in}}%
\pgfpathlineto{\pgfqpoint{4.237504in}{2.057187in}}%
\pgfpathlineto{\pgfqpoint{4.237504in}{2.104378in}}%
\pgfpathlineto{\pgfqpoint{4.310157in}{2.104378in}}%
\pgfpathlineto{\pgfqpoint{4.310157in}{2.057187in}}%
\pgfpathmoveto{\pgfqpoint{3.075004in}{2.033592in}}%
\pgfpathlineto{\pgfqpoint{3.075004in}{2.033592in}}%
\pgfpathlineto{\pgfqpoint{3.075004in}{2.057187in}}%
\pgfpathlineto{\pgfqpoint{3.111331in}{2.057187in}}%
\pgfpathlineto{\pgfqpoint{3.111331in}{2.033592in}}%
\pgfpathmoveto{\pgfqpoint{3.111331in}{2.033592in}}%
\pgfpathlineto{\pgfqpoint{3.111331in}{2.033592in}}%
\pgfpathlineto{\pgfqpoint{3.111331in}{2.057187in}}%
\pgfpathlineto{\pgfqpoint{3.147658in}{2.057187in}}%
\pgfpathlineto{\pgfqpoint{3.147658in}{2.033592in}}%
\pgfpathmoveto{\pgfqpoint{3.147658in}{2.033592in}}%
\pgfpathlineto{\pgfqpoint{3.147658in}{2.033592in}}%
\pgfpathlineto{\pgfqpoint{3.147658in}{2.057187in}}%
\pgfpathlineto{\pgfqpoint{3.183984in}{2.057187in}}%
\pgfpathlineto{\pgfqpoint{3.183984in}{2.033592in}}%
\pgfpathmoveto{\pgfqpoint{3.183984in}{2.033592in}}%
\pgfpathlineto{\pgfqpoint{3.183984in}{2.033592in}}%
\pgfpathlineto{\pgfqpoint{3.183984in}{2.057187in}}%
\pgfpathlineto{\pgfqpoint{3.220311in}{2.057187in}}%
\pgfpathlineto{\pgfqpoint{3.220311in}{2.033592in}}%
\pgfpathmoveto{\pgfqpoint{3.075004in}{2.576249in}}%
\pgfpathlineto{\pgfqpoint{3.075004in}{2.576249in}}%
\pgfpathlineto{\pgfqpoint{3.075004in}{2.599843in}}%
\pgfpathlineto{\pgfqpoint{3.111331in}{2.599843in}}%
\pgfpathlineto{\pgfqpoint{3.111331in}{2.576249in}}%
\pgfpathmoveto{\pgfqpoint{3.111331in}{2.576249in}}%
\pgfpathlineto{\pgfqpoint{3.111331in}{2.576249in}}%
\pgfpathlineto{\pgfqpoint{3.111331in}{2.599843in}}%
\pgfpathlineto{\pgfqpoint{3.147658in}{2.599843in}}%
\pgfpathlineto{\pgfqpoint{3.147658in}{2.576249in}}%
\pgfpathmoveto{\pgfqpoint{3.147658in}{2.576249in}}%
\pgfpathlineto{\pgfqpoint{3.147658in}{2.576249in}}%
\pgfpathlineto{\pgfqpoint{3.147658in}{2.599843in}}%
\pgfpathlineto{\pgfqpoint{3.183984in}{2.599843in}}%
\pgfpathlineto{\pgfqpoint{3.183984in}{2.576249in}}%
\pgfpathmoveto{\pgfqpoint{3.183984in}{2.576249in}}%
\pgfpathlineto{\pgfqpoint{3.183984in}{2.576249in}}%
\pgfpathlineto{\pgfqpoint{3.183984in}{2.599843in}}%
\pgfpathlineto{\pgfqpoint{3.220311in}{2.599843in}}%
\pgfpathlineto{\pgfqpoint{3.220311in}{2.576249in}}%
\pgfpathmoveto{\pgfqpoint{3.220311in}{2.033592in}}%
\pgfpathlineto{\pgfqpoint{3.220311in}{2.033592in}}%
\pgfpathlineto{\pgfqpoint{3.220311in}{2.057187in}}%
\pgfpathlineto{\pgfqpoint{3.256640in}{2.057187in}}%
\pgfpathlineto{\pgfqpoint{3.256640in}{2.033592in}}%
\pgfpathmoveto{\pgfqpoint{3.256640in}{2.033592in}}%
\pgfpathlineto{\pgfqpoint{3.256640in}{2.033592in}}%
\pgfpathlineto{\pgfqpoint{3.256640in}{2.057187in}}%
\pgfpathlineto{\pgfqpoint{3.292968in}{2.057187in}}%
\pgfpathlineto{\pgfqpoint{3.292968in}{2.033592in}}%
\pgfpathmoveto{\pgfqpoint{3.292968in}{2.033592in}}%
\pgfpathlineto{\pgfqpoint{3.292968in}{2.033592in}}%
\pgfpathlineto{\pgfqpoint{3.292968in}{2.057187in}}%
\pgfpathlineto{\pgfqpoint{3.329297in}{2.057187in}}%
\pgfpathlineto{\pgfqpoint{3.329297in}{2.033592in}}%
\pgfpathmoveto{\pgfqpoint{3.329297in}{2.033592in}}%
\pgfpathlineto{\pgfqpoint{3.329297in}{2.033592in}}%
\pgfpathlineto{\pgfqpoint{3.329297in}{2.057187in}}%
\pgfpathlineto{\pgfqpoint{3.365625in}{2.057187in}}%
\pgfpathlineto{\pgfqpoint{3.365625in}{2.033592in}}%
\pgfpathmoveto{\pgfqpoint{3.220311in}{2.576249in}}%
\pgfpathlineto{\pgfqpoint{3.220311in}{2.576249in}}%
\pgfpathlineto{\pgfqpoint{3.220311in}{2.599843in}}%
\pgfpathlineto{\pgfqpoint{3.256640in}{2.599843in}}%
\pgfpathlineto{\pgfqpoint{3.256640in}{2.576249in}}%
\pgfpathmoveto{\pgfqpoint{3.256640in}{2.576249in}}%
\pgfpathlineto{\pgfqpoint{3.256640in}{2.576249in}}%
\pgfpathlineto{\pgfqpoint{3.256640in}{2.599843in}}%
\pgfpathlineto{\pgfqpoint{3.292968in}{2.599843in}}%
\pgfpathlineto{\pgfqpoint{3.292968in}{2.576249in}}%
\pgfpathmoveto{\pgfqpoint{3.292968in}{2.576249in}}%
\pgfpathlineto{\pgfqpoint{3.292968in}{2.576249in}}%
\pgfpathlineto{\pgfqpoint{3.292968in}{2.599843in}}%
\pgfpathlineto{\pgfqpoint{3.329297in}{2.599843in}}%
\pgfpathlineto{\pgfqpoint{3.329297in}{2.576249in}}%
\pgfpathmoveto{\pgfqpoint{3.329297in}{2.576249in}}%
\pgfpathlineto{\pgfqpoint{3.329297in}{2.576249in}}%
\pgfpathlineto{\pgfqpoint{3.329297in}{2.599843in}}%
\pgfpathlineto{\pgfqpoint{3.365625in}{2.599843in}}%
\pgfpathlineto{\pgfqpoint{3.365625in}{2.576249in}}%
\pgfpathmoveto{\pgfqpoint{3.365625in}{2.033592in}}%
\pgfpathlineto{\pgfqpoint{3.365625in}{2.033592in}}%
\pgfpathlineto{\pgfqpoint{3.365625in}{2.057187in}}%
\pgfpathlineto{\pgfqpoint{3.401954in}{2.057187in}}%
\pgfpathlineto{\pgfqpoint{3.401954in}{2.033592in}}%
\pgfpathmoveto{\pgfqpoint{3.401954in}{2.033592in}}%
\pgfpathlineto{\pgfqpoint{3.401954in}{2.033592in}}%
\pgfpathlineto{\pgfqpoint{3.401954in}{2.057187in}}%
\pgfpathlineto{\pgfqpoint{3.438282in}{2.057187in}}%
\pgfpathlineto{\pgfqpoint{3.438282in}{2.033592in}}%
\pgfpathmoveto{\pgfqpoint{3.438282in}{2.033592in}}%
\pgfpathlineto{\pgfqpoint{3.438282in}{2.033592in}}%
\pgfpathlineto{\pgfqpoint{3.438282in}{2.057187in}}%
\pgfpathlineto{\pgfqpoint{3.474610in}{2.057187in}}%
\pgfpathlineto{\pgfqpoint{3.474610in}{2.033592in}}%
\pgfpathmoveto{\pgfqpoint{3.474610in}{2.033592in}}%
\pgfpathlineto{\pgfqpoint{3.474610in}{2.033592in}}%
\pgfpathlineto{\pgfqpoint{3.474610in}{2.057187in}}%
\pgfpathlineto{\pgfqpoint{3.510939in}{2.057187in}}%
\pgfpathlineto{\pgfqpoint{3.510939in}{2.033592in}}%
\pgfpathmoveto{\pgfqpoint{3.365625in}{2.576249in}}%
\pgfpathlineto{\pgfqpoint{3.365625in}{2.576249in}}%
\pgfpathlineto{\pgfqpoint{3.365625in}{2.599843in}}%
\pgfpathlineto{\pgfqpoint{3.401954in}{2.599843in}}%
\pgfpathlineto{\pgfqpoint{3.401954in}{2.576249in}}%
\pgfpathmoveto{\pgfqpoint{3.401954in}{2.576249in}}%
\pgfpathlineto{\pgfqpoint{3.401954in}{2.576249in}}%
\pgfpathlineto{\pgfqpoint{3.401954in}{2.599843in}}%
\pgfpathlineto{\pgfqpoint{3.438282in}{2.599843in}}%
\pgfpathlineto{\pgfqpoint{3.438282in}{2.576249in}}%
\pgfpathmoveto{\pgfqpoint{3.438282in}{2.576249in}}%
\pgfpathlineto{\pgfqpoint{3.438282in}{2.576249in}}%
\pgfpathlineto{\pgfqpoint{3.438282in}{2.599843in}}%
\pgfpathlineto{\pgfqpoint{3.474610in}{2.599843in}}%
\pgfpathlineto{\pgfqpoint{3.474610in}{2.576249in}}%
\pgfpathmoveto{\pgfqpoint{3.474610in}{2.576249in}}%
\pgfpathlineto{\pgfqpoint{3.474610in}{2.576249in}}%
\pgfpathlineto{\pgfqpoint{3.474610in}{2.599843in}}%
\pgfpathlineto{\pgfqpoint{3.510939in}{2.599843in}}%
\pgfpathlineto{\pgfqpoint{3.510939in}{2.576249in}}%
\pgfpathmoveto{\pgfqpoint{3.510939in}{2.033592in}}%
\pgfpathlineto{\pgfqpoint{3.510939in}{2.033592in}}%
\pgfpathlineto{\pgfqpoint{3.510939in}{2.057187in}}%
\pgfpathlineto{\pgfqpoint{3.547267in}{2.057187in}}%
\pgfpathlineto{\pgfqpoint{3.547267in}{2.033592in}}%
\pgfpathmoveto{\pgfqpoint{3.547267in}{2.033592in}}%
\pgfpathlineto{\pgfqpoint{3.547267in}{2.033592in}}%
\pgfpathlineto{\pgfqpoint{3.547267in}{2.057187in}}%
\pgfpathlineto{\pgfqpoint{3.583596in}{2.057187in}}%
\pgfpathlineto{\pgfqpoint{3.583596in}{2.033592in}}%
\pgfpathmoveto{\pgfqpoint{3.583596in}{2.033592in}}%
\pgfpathlineto{\pgfqpoint{3.583596in}{2.033592in}}%
\pgfpathlineto{\pgfqpoint{3.583596in}{2.057187in}}%
\pgfpathlineto{\pgfqpoint{3.619924in}{2.057187in}}%
\pgfpathlineto{\pgfqpoint{3.619924in}{2.033592in}}%
\pgfpathmoveto{\pgfqpoint{3.619924in}{2.033592in}}%
\pgfpathlineto{\pgfqpoint{3.619924in}{2.033592in}}%
\pgfpathlineto{\pgfqpoint{3.619924in}{2.057187in}}%
\pgfpathlineto{\pgfqpoint{3.656253in}{2.057187in}}%
\pgfpathlineto{\pgfqpoint{3.656253in}{2.033592in}}%
\pgfpathmoveto{\pgfqpoint{3.583596in}{2.529061in}}%
\pgfpathlineto{\pgfqpoint{3.583596in}{2.529061in}}%
\pgfpathlineto{\pgfqpoint{3.583596in}{2.552655in}}%
\pgfpathlineto{\pgfqpoint{3.619924in}{2.552655in}}%
\pgfpathlineto{\pgfqpoint{3.619924in}{2.529061in}}%
\pgfpathmoveto{\pgfqpoint{3.510939in}{2.576249in}}%
\pgfpathlineto{\pgfqpoint{3.510939in}{2.576249in}}%
\pgfpathlineto{\pgfqpoint{3.510939in}{2.599843in}}%
\pgfpathlineto{\pgfqpoint{3.547267in}{2.599843in}}%
\pgfpathlineto{\pgfqpoint{3.547267in}{2.576249in}}%
\pgfpathmoveto{\pgfqpoint{3.656253in}{2.033592in}}%
\pgfpathlineto{\pgfqpoint{3.656253in}{2.033592in}}%
\pgfpathlineto{\pgfqpoint{3.656253in}{2.057187in}}%
\pgfpathlineto{\pgfqpoint{3.692580in}{2.057187in}}%
\pgfpathlineto{\pgfqpoint{3.692580in}{2.033592in}}%
\pgfpathmoveto{\pgfqpoint{3.692580in}{2.033592in}}%
\pgfpathlineto{\pgfqpoint{3.692580in}{2.033592in}}%
\pgfpathlineto{\pgfqpoint{3.692580in}{2.057187in}}%
\pgfpathlineto{\pgfqpoint{3.728907in}{2.057187in}}%
\pgfpathlineto{\pgfqpoint{3.728907in}{2.033592in}}%
\pgfpathmoveto{\pgfqpoint{3.728907in}{2.033592in}}%
\pgfpathlineto{\pgfqpoint{3.728907in}{2.033592in}}%
\pgfpathlineto{\pgfqpoint{3.728907in}{2.057187in}}%
\pgfpathlineto{\pgfqpoint{3.765234in}{2.057187in}}%
\pgfpathlineto{\pgfqpoint{3.765234in}{2.033592in}}%
\pgfpathmoveto{\pgfqpoint{3.765234in}{2.033592in}}%
\pgfpathlineto{\pgfqpoint{3.765234in}{2.033592in}}%
\pgfpathlineto{\pgfqpoint{3.765234in}{2.057187in}}%
\pgfpathlineto{\pgfqpoint{3.801561in}{2.057187in}}%
\pgfpathlineto{\pgfqpoint{3.801561in}{2.033592in}}%
\pgfpathmoveto{\pgfqpoint{3.728907in}{2.434687in}}%
\pgfpathlineto{\pgfqpoint{3.728907in}{2.434687in}}%
\pgfpathlineto{\pgfqpoint{3.728907in}{2.458280in}}%
\pgfpathlineto{\pgfqpoint{3.765234in}{2.458280in}}%
\pgfpathlineto{\pgfqpoint{3.765234in}{2.434687in}}%
\pgfpathmoveto{\pgfqpoint{3.656253in}{2.481873in}}%
\pgfpathlineto{\pgfqpoint{3.656253in}{2.481873in}}%
\pgfpathlineto{\pgfqpoint{3.656253in}{2.505467in}}%
\pgfpathlineto{\pgfqpoint{3.692580in}{2.505467in}}%
\pgfpathlineto{\pgfqpoint{3.692580in}{2.481873in}}%
\pgfpathmoveto{\pgfqpoint{3.801561in}{2.033592in}}%
\pgfpathlineto{\pgfqpoint{3.801561in}{2.033592in}}%
\pgfpathlineto{\pgfqpoint{3.801561in}{2.057187in}}%
\pgfpathlineto{\pgfqpoint{3.837890in}{2.057187in}}%
\pgfpathlineto{\pgfqpoint{3.837890in}{2.033592in}}%
\pgfpathmoveto{\pgfqpoint{3.837890in}{2.033592in}}%
\pgfpathlineto{\pgfqpoint{3.837890in}{2.033592in}}%
\pgfpathlineto{\pgfqpoint{3.837890in}{2.057187in}}%
\pgfpathlineto{\pgfqpoint{3.874218in}{2.057187in}}%
\pgfpathlineto{\pgfqpoint{3.874218in}{2.033592in}}%
\pgfpathmoveto{\pgfqpoint{3.874218in}{2.033592in}}%
\pgfpathlineto{\pgfqpoint{3.874218in}{2.033592in}}%
\pgfpathlineto{\pgfqpoint{3.874218in}{2.057187in}}%
\pgfpathlineto{\pgfqpoint{3.910546in}{2.057187in}}%
\pgfpathlineto{\pgfqpoint{3.910546in}{2.033592in}}%
\pgfpathmoveto{\pgfqpoint{3.910546in}{2.033592in}}%
\pgfpathlineto{\pgfqpoint{3.910546in}{2.033592in}}%
\pgfpathlineto{\pgfqpoint{3.910546in}{2.057187in}}%
\pgfpathlineto{\pgfqpoint{3.946874in}{2.057187in}}%
\pgfpathlineto{\pgfqpoint{3.946874in}{2.033592in}}%
\pgfpathmoveto{\pgfqpoint{3.874218in}{2.340313in}}%
\pgfpathlineto{\pgfqpoint{3.874218in}{2.340313in}}%
\pgfpathlineto{\pgfqpoint{3.874218in}{2.363907in}}%
\pgfpathlineto{\pgfqpoint{3.910546in}{2.363907in}}%
\pgfpathlineto{\pgfqpoint{3.910546in}{2.340313in}}%
\pgfpathmoveto{\pgfqpoint{3.801561in}{2.387502in}}%
\pgfpathlineto{\pgfqpoint{3.801561in}{2.387502in}}%
\pgfpathlineto{\pgfqpoint{3.801561in}{2.411094in}}%
\pgfpathlineto{\pgfqpoint{3.837890in}{2.411094in}}%
\pgfpathlineto{\pgfqpoint{3.837890in}{2.387502in}}%
\pgfpathmoveto{\pgfqpoint{3.946874in}{2.033592in}}%
\pgfpathlineto{\pgfqpoint{3.946874in}{2.033592in}}%
\pgfpathlineto{\pgfqpoint{3.946874in}{2.057187in}}%
\pgfpathlineto{\pgfqpoint{3.983202in}{2.057187in}}%
\pgfpathlineto{\pgfqpoint{3.983202in}{2.033592in}}%
\pgfpathmoveto{\pgfqpoint{3.983202in}{2.033592in}}%
\pgfpathlineto{\pgfqpoint{3.983202in}{2.033592in}}%
\pgfpathlineto{\pgfqpoint{3.983202in}{2.057187in}}%
\pgfpathlineto{\pgfqpoint{4.019529in}{2.057187in}}%
\pgfpathlineto{\pgfqpoint{4.019529in}{2.033592in}}%
\pgfpathmoveto{\pgfqpoint{4.019529in}{2.033592in}}%
\pgfpathlineto{\pgfqpoint{4.019529in}{2.033592in}}%
\pgfpathlineto{\pgfqpoint{4.019529in}{2.057187in}}%
\pgfpathlineto{\pgfqpoint{4.055857in}{2.057187in}}%
\pgfpathlineto{\pgfqpoint{4.055857in}{2.033592in}}%
\pgfpathmoveto{\pgfqpoint{4.055857in}{2.033592in}}%
\pgfpathlineto{\pgfqpoint{4.055857in}{2.033592in}}%
\pgfpathlineto{\pgfqpoint{4.055857in}{2.057187in}}%
\pgfpathlineto{\pgfqpoint{4.092185in}{2.057187in}}%
\pgfpathlineto{\pgfqpoint{4.092185in}{2.033592in}}%
\pgfpathmoveto{\pgfqpoint{4.019529in}{2.245936in}}%
\pgfpathlineto{\pgfqpoint{4.019529in}{2.245936in}}%
\pgfpathlineto{\pgfqpoint{4.019529in}{2.269531in}}%
\pgfpathlineto{\pgfqpoint{4.055857in}{2.269531in}}%
\pgfpathlineto{\pgfqpoint{4.055857in}{2.245936in}}%
\pgfpathmoveto{\pgfqpoint{3.946874in}{2.293125in}}%
\pgfpathlineto{\pgfqpoint{3.946874in}{2.293125in}}%
\pgfpathlineto{\pgfqpoint{3.946874in}{2.316719in}}%
\pgfpathlineto{\pgfqpoint{3.983202in}{2.316719in}}%
\pgfpathlineto{\pgfqpoint{3.983202in}{2.293125in}}%
\pgfpathmoveto{\pgfqpoint{4.092185in}{2.033592in}}%
\pgfpathlineto{\pgfqpoint{4.092185in}{2.033592in}}%
\pgfpathlineto{\pgfqpoint{4.092185in}{2.057187in}}%
\pgfpathlineto{\pgfqpoint{4.128515in}{2.057187in}}%
\pgfpathlineto{\pgfqpoint{4.128515in}{2.033592in}}%
\pgfpathmoveto{\pgfqpoint{4.128515in}{2.033592in}}%
\pgfpathlineto{\pgfqpoint{4.128515in}{2.033592in}}%
\pgfpathlineto{\pgfqpoint{4.128515in}{2.057187in}}%
\pgfpathlineto{\pgfqpoint{4.164844in}{2.057187in}}%
\pgfpathlineto{\pgfqpoint{4.164844in}{2.033592in}}%
\pgfpathmoveto{\pgfqpoint{4.164844in}{2.033592in}}%
\pgfpathlineto{\pgfqpoint{4.164844in}{2.033592in}}%
\pgfpathlineto{\pgfqpoint{4.164844in}{2.057187in}}%
\pgfpathlineto{\pgfqpoint{4.201174in}{2.057187in}}%
\pgfpathlineto{\pgfqpoint{4.201174in}{2.033592in}}%
\pgfpathmoveto{\pgfqpoint{4.201174in}{2.033592in}}%
\pgfpathlineto{\pgfqpoint{4.201174in}{2.033592in}}%
\pgfpathlineto{\pgfqpoint{4.201174in}{2.057187in}}%
\pgfpathlineto{\pgfqpoint{4.237504in}{2.057187in}}%
\pgfpathlineto{\pgfqpoint{4.237504in}{2.033592in}}%
\pgfpathmoveto{\pgfqpoint{4.164844in}{2.151563in}}%
\pgfpathlineto{\pgfqpoint{4.164844in}{2.151563in}}%
\pgfpathlineto{\pgfqpoint{4.164844in}{2.175155in}}%
\pgfpathlineto{\pgfqpoint{4.201174in}{2.175155in}}%
\pgfpathlineto{\pgfqpoint{4.201174in}{2.151563in}}%
\pgfpathmoveto{\pgfqpoint{4.092185in}{2.198748in}}%
\pgfpathlineto{\pgfqpoint{4.092185in}{2.198748in}}%
\pgfpathlineto{\pgfqpoint{4.092185in}{2.222342in}}%
\pgfpathlineto{\pgfqpoint{4.128515in}{2.222342in}}%
\pgfpathlineto{\pgfqpoint{4.128515in}{2.198748in}}%
\pgfpathmoveto{\pgfqpoint{4.237504in}{2.033592in}}%
\pgfpathlineto{\pgfqpoint{4.237504in}{2.033592in}}%
\pgfpathlineto{\pgfqpoint{4.237504in}{2.057187in}}%
\pgfpathlineto{\pgfqpoint{4.273831in}{2.057187in}}%
\pgfpathlineto{\pgfqpoint{4.273831in}{2.033592in}}%
\pgfpathmoveto{\pgfqpoint{4.273831in}{2.033592in}}%
\pgfpathlineto{\pgfqpoint{4.273831in}{2.033592in}}%
\pgfpathlineto{\pgfqpoint{4.273831in}{2.057187in}}%
\pgfpathlineto{\pgfqpoint{4.310157in}{2.057187in}}%
\pgfpathlineto{\pgfqpoint{4.310157in}{2.033592in}}%
\pgfpathmoveto{\pgfqpoint{4.310157in}{2.033592in}}%
\pgfpathlineto{\pgfqpoint{4.310157in}{2.033592in}}%
\pgfpathlineto{\pgfqpoint{4.310157in}{2.057187in}}%
\pgfpathlineto{\pgfqpoint{4.346484in}{2.057187in}}%
\pgfpathlineto{\pgfqpoint{4.346484in}{2.033592in}}%
\pgfpathmoveto{\pgfqpoint{4.346484in}{2.033592in}}%
\pgfpathlineto{\pgfqpoint{4.346484in}{2.033592in}}%
\pgfpathlineto{\pgfqpoint{4.346484in}{2.057187in}}%
\pgfpathlineto{\pgfqpoint{4.382811in}{2.057187in}}%
\pgfpathlineto{\pgfqpoint{4.382811in}{2.033592in}}%
\pgfpathmoveto{\pgfqpoint{4.310157in}{2.057187in}}%
\pgfpathlineto{\pgfqpoint{4.310157in}{2.057187in}}%
\pgfpathlineto{\pgfqpoint{4.310157in}{2.080783in}}%
\pgfpathlineto{\pgfqpoint{4.346484in}{2.080783in}}%
\pgfpathlineto{\pgfqpoint{4.346484in}{2.057187in}}%
\pgfpathmoveto{\pgfqpoint{4.237504in}{2.104378in}}%
\pgfpathlineto{\pgfqpoint{4.237504in}{2.104378in}}%
\pgfpathlineto{\pgfqpoint{4.237504in}{2.127970in}}%
\pgfpathlineto{\pgfqpoint{4.273831in}{2.127970in}}%
\pgfpathlineto{\pgfqpoint{4.273831in}{2.104378in}}%
\pgfpathmoveto{\pgfqpoint{3.075004in}{2.021795in}}%
\pgfpathlineto{\pgfqpoint{3.075004in}{2.021795in}}%
\pgfpathlineto{\pgfqpoint{3.075004in}{2.033592in}}%
\pgfpathlineto{\pgfqpoint{3.093168in}{2.033592in}}%
\pgfpathlineto{\pgfqpoint{3.093168in}{2.021795in}}%
\pgfpathmoveto{\pgfqpoint{3.093168in}{2.021795in}}%
\pgfpathlineto{\pgfqpoint{3.093168in}{2.021795in}}%
\pgfpathlineto{\pgfqpoint{3.093168in}{2.033592in}}%
\pgfpathlineto{\pgfqpoint{3.111331in}{2.033592in}}%
\pgfpathlineto{\pgfqpoint{3.111331in}{2.021795in}}%
\pgfpathmoveto{\pgfqpoint{3.111331in}{2.021795in}}%
\pgfpathlineto{\pgfqpoint{3.111331in}{2.021795in}}%
\pgfpathlineto{\pgfqpoint{3.111331in}{2.033592in}}%
\pgfpathlineto{\pgfqpoint{3.129494in}{2.033592in}}%
\pgfpathlineto{\pgfqpoint{3.129494in}{2.021795in}}%
\pgfpathmoveto{\pgfqpoint{3.129494in}{2.021795in}}%
\pgfpathlineto{\pgfqpoint{3.129494in}{2.021795in}}%
\pgfpathlineto{\pgfqpoint{3.129494in}{2.033592in}}%
\pgfpathlineto{\pgfqpoint{3.147658in}{2.033592in}}%
\pgfpathlineto{\pgfqpoint{3.147658in}{2.021795in}}%
\pgfpathmoveto{\pgfqpoint{3.147658in}{2.021795in}}%
\pgfpathlineto{\pgfqpoint{3.147658in}{2.021795in}}%
\pgfpathlineto{\pgfqpoint{3.147658in}{2.033592in}}%
\pgfpathlineto{\pgfqpoint{3.165821in}{2.033592in}}%
\pgfpathlineto{\pgfqpoint{3.165821in}{2.021795in}}%
\pgfpathmoveto{\pgfqpoint{3.165821in}{2.021795in}}%
\pgfpathlineto{\pgfqpoint{3.165821in}{2.021795in}}%
\pgfpathlineto{\pgfqpoint{3.165821in}{2.033592in}}%
\pgfpathlineto{\pgfqpoint{3.183984in}{2.033592in}}%
\pgfpathlineto{\pgfqpoint{3.183984in}{2.021795in}}%
\pgfpathmoveto{\pgfqpoint{3.183984in}{2.021795in}}%
\pgfpathlineto{\pgfqpoint{3.183984in}{2.021795in}}%
\pgfpathlineto{\pgfqpoint{3.183984in}{2.033592in}}%
\pgfpathlineto{\pgfqpoint{3.202148in}{2.033592in}}%
\pgfpathlineto{\pgfqpoint{3.202148in}{2.021795in}}%
\pgfpathmoveto{\pgfqpoint{3.202148in}{2.021795in}}%
\pgfpathlineto{\pgfqpoint{3.202148in}{2.021795in}}%
\pgfpathlineto{\pgfqpoint{3.202148in}{2.033592in}}%
\pgfpathlineto{\pgfqpoint{3.220311in}{2.033592in}}%
\pgfpathlineto{\pgfqpoint{3.220311in}{2.021795in}}%
\pgfpathmoveto{\pgfqpoint{3.075004in}{2.599843in}}%
\pgfpathlineto{\pgfqpoint{3.075004in}{2.599843in}}%
\pgfpathlineto{\pgfqpoint{3.075004in}{2.611639in}}%
\pgfpathlineto{\pgfqpoint{3.093168in}{2.611639in}}%
\pgfpathlineto{\pgfqpoint{3.093168in}{2.599843in}}%
\pgfpathmoveto{\pgfqpoint{3.093168in}{2.599843in}}%
\pgfpathlineto{\pgfqpoint{3.093168in}{2.599843in}}%
\pgfpathlineto{\pgfqpoint{3.093168in}{2.611639in}}%
\pgfpathlineto{\pgfqpoint{3.111331in}{2.611639in}}%
\pgfpathlineto{\pgfqpoint{3.111331in}{2.599843in}}%
\pgfpathmoveto{\pgfqpoint{3.111331in}{2.599843in}}%
\pgfpathlineto{\pgfqpoint{3.111331in}{2.599843in}}%
\pgfpathlineto{\pgfqpoint{3.111331in}{2.611639in}}%
\pgfpathlineto{\pgfqpoint{3.129494in}{2.611639in}}%
\pgfpathlineto{\pgfqpoint{3.129494in}{2.599843in}}%
\pgfpathmoveto{\pgfqpoint{3.129494in}{2.599843in}}%
\pgfpathlineto{\pgfqpoint{3.129494in}{2.599843in}}%
\pgfpathlineto{\pgfqpoint{3.129494in}{2.611639in}}%
\pgfpathlineto{\pgfqpoint{3.147658in}{2.611639in}}%
\pgfpathlineto{\pgfqpoint{3.147658in}{2.599843in}}%
\pgfpathmoveto{\pgfqpoint{3.147658in}{2.599843in}}%
\pgfpathlineto{\pgfqpoint{3.147658in}{2.599843in}}%
\pgfpathlineto{\pgfqpoint{3.147658in}{2.611639in}}%
\pgfpathlineto{\pgfqpoint{3.165821in}{2.611639in}}%
\pgfpathlineto{\pgfqpoint{3.165821in}{2.599843in}}%
\pgfpathmoveto{\pgfqpoint{3.165821in}{2.599843in}}%
\pgfpathlineto{\pgfqpoint{3.165821in}{2.599843in}}%
\pgfpathlineto{\pgfqpoint{3.165821in}{2.611639in}}%
\pgfpathlineto{\pgfqpoint{3.183984in}{2.611639in}}%
\pgfpathlineto{\pgfqpoint{3.183984in}{2.599843in}}%
\pgfpathmoveto{\pgfqpoint{3.183984in}{2.599843in}}%
\pgfpathlineto{\pgfqpoint{3.183984in}{2.599843in}}%
\pgfpathlineto{\pgfqpoint{3.183984in}{2.611639in}}%
\pgfpathlineto{\pgfqpoint{3.202148in}{2.611639in}}%
\pgfpathlineto{\pgfqpoint{3.202148in}{2.599843in}}%
\pgfpathmoveto{\pgfqpoint{3.202148in}{2.599843in}}%
\pgfpathlineto{\pgfqpoint{3.202148in}{2.599843in}}%
\pgfpathlineto{\pgfqpoint{3.202148in}{2.611639in}}%
\pgfpathlineto{\pgfqpoint{3.220311in}{2.611639in}}%
\pgfpathlineto{\pgfqpoint{3.220311in}{2.599843in}}%
\pgfpathmoveto{\pgfqpoint{3.220311in}{2.021795in}}%
\pgfpathlineto{\pgfqpoint{3.220311in}{2.021795in}}%
\pgfpathlineto{\pgfqpoint{3.220311in}{2.033592in}}%
\pgfpathlineto{\pgfqpoint{3.238475in}{2.033592in}}%
\pgfpathlineto{\pgfqpoint{3.238475in}{2.021795in}}%
\pgfpathmoveto{\pgfqpoint{3.238475in}{2.021795in}}%
\pgfpathlineto{\pgfqpoint{3.238475in}{2.021795in}}%
\pgfpathlineto{\pgfqpoint{3.238475in}{2.033592in}}%
\pgfpathlineto{\pgfqpoint{3.256640in}{2.033592in}}%
\pgfpathlineto{\pgfqpoint{3.256640in}{2.021795in}}%
\pgfpathmoveto{\pgfqpoint{3.256640in}{2.021795in}}%
\pgfpathlineto{\pgfqpoint{3.256640in}{2.021795in}}%
\pgfpathlineto{\pgfqpoint{3.256640in}{2.033592in}}%
\pgfpathlineto{\pgfqpoint{3.274804in}{2.033592in}}%
\pgfpathlineto{\pgfqpoint{3.274804in}{2.021795in}}%
\pgfpathmoveto{\pgfqpoint{3.274804in}{2.021795in}}%
\pgfpathlineto{\pgfqpoint{3.274804in}{2.021795in}}%
\pgfpathlineto{\pgfqpoint{3.274804in}{2.033592in}}%
\pgfpathlineto{\pgfqpoint{3.292968in}{2.033592in}}%
\pgfpathlineto{\pgfqpoint{3.292968in}{2.021795in}}%
\pgfpathmoveto{\pgfqpoint{3.292968in}{2.021795in}}%
\pgfpathlineto{\pgfqpoint{3.292968in}{2.021795in}}%
\pgfpathlineto{\pgfqpoint{3.292968in}{2.033592in}}%
\pgfpathlineto{\pgfqpoint{3.311132in}{2.033592in}}%
\pgfpathlineto{\pgfqpoint{3.311132in}{2.021795in}}%
\pgfpathmoveto{\pgfqpoint{3.311132in}{2.021795in}}%
\pgfpathlineto{\pgfqpoint{3.311132in}{2.021795in}}%
\pgfpathlineto{\pgfqpoint{3.311132in}{2.033592in}}%
\pgfpathlineto{\pgfqpoint{3.329297in}{2.033592in}}%
\pgfpathlineto{\pgfqpoint{3.329297in}{2.021795in}}%
\pgfpathmoveto{\pgfqpoint{3.329297in}{2.021795in}}%
\pgfpathlineto{\pgfqpoint{3.329297in}{2.021795in}}%
\pgfpathlineto{\pgfqpoint{3.329297in}{2.033592in}}%
\pgfpathlineto{\pgfqpoint{3.347461in}{2.033592in}}%
\pgfpathlineto{\pgfqpoint{3.347461in}{2.021795in}}%
\pgfpathmoveto{\pgfqpoint{3.347461in}{2.021795in}}%
\pgfpathlineto{\pgfqpoint{3.347461in}{2.021795in}}%
\pgfpathlineto{\pgfqpoint{3.347461in}{2.033592in}}%
\pgfpathlineto{\pgfqpoint{3.365625in}{2.033592in}}%
\pgfpathlineto{\pgfqpoint{3.365625in}{2.021795in}}%
\pgfpathmoveto{\pgfqpoint{3.220311in}{2.599843in}}%
\pgfpathlineto{\pgfqpoint{3.220311in}{2.599843in}}%
\pgfpathlineto{\pgfqpoint{3.220311in}{2.611639in}}%
\pgfpathlineto{\pgfqpoint{3.238475in}{2.611639in}}%
\pgfpathlineto{\pgfqpoint{3.238475in}{2.599843in}}%
\pgfpathmoveto{\pgfqpoint{3.238475in}{2.599843in}}%
\pgfpathlineto{\pgfqpoint{3.238475in}{2.599843in}}%
\pgfpathlineto{\pgfqpoint{3.238475in}{2.611639in}}%
\pgfpathlineto{\pgfqpoint{3.256640in}{2.611639in}}%
\pgfpathlineto{\pgfqpoint{3.256640in}{2.599843in}}%
\pgfpathmoveto{\pgfqpoint{3.256640in}{2.599843in}}%
\pgfpathlineto{\pgfqpoint{3.256640in}{2.599843in}}%
\pgfpathlineto{\pgfqpoint{3.256640in}{2.611639in}}%
\pgfpathlineto{\pgfqpoint{3.274804in}{2.611639in}}%
\pgfpathlineto{\pgfqpoint{3.274804in}{2.599843in}}%
\pgfpathmoveto{\pgfqpoint{3.274804in}{2.599843in}}%
\pgfpathlineto{\pgfqpoint{3.274804in}{2.599843in}}%
\pgfpathlineto{\pgfqpoint{3.274804in}{2.611639in}}%
\pgfpathlineto{\pgfqpoint{3.292968in}{2.611639in}}%
\pgfpathlineto{\pgfqpoint{3.292968in}{2.599843in}}%
\pgfpathmoveto{\pgfqpoint{3.292968in}{2.599843in}}%
\pgfpathlineto{\pgfqpoint{3.292968in}{2.599843in}}%
\pgfpathlineto{\pgfqpoint{3.292968in}{2.611639in}}%
\pgfpathlineto{\pgfqpoint{3.311132in}{2.611639in}}%
\pgfpathlineto{\pgfqpoint{3.311132in}{2.599843in}}%
\pgfpathmoveto{\pgfqpoint{3.311132in}{2.599843in}}%
\pgfpathlineto{\pgfqpoint{3.311132in}{2.599843in}}%
\pgfpathlineto{\pgfqpoint{3.311132in}{2.611639in}}%
\pgfpathlineto{\pgfqpoint{3.329297in}{2.611639in}}%
\pgfpathlineto{\pgfqpoint{3.329297in}{2.599843in}}%
\pgfpathmoveto{\pgfqpoint{3.329297in}{2.599843in}}%
\pgfpathlineto{\pgfqpoint{3.329297in}{2.599843in}}%
\pgfpathlineto{\pgfqpoint{3.329297in}{2.611639in}}%
\pgfpathlineto{\pgfqpoint{3.347461in}{2.611639in}}%
\pgfpathlineto{\pgfqpoint{3.347461in}{2.599843in}}%
\pgfpathmoveto{\pgfqpoint{3.347461in}{2.599843in}}%
\pgfpathlineto{\pgfqpoint{3.347461in}{2.599843in}}%
\pgfpathlineto{\pgfqpoint{3.347461in}{2.611639in}}%
\pgfpathlineto{\pgfqpoint{3.365625in}{2.611639in}}%
\pgfpathlineto{\pgfqpoint{3.365625in}{2.599843in}}%
\pgfpathmoveto{\pgfqpoint{3.365625in}{2.021795in}}%
\pgfpathlineto{\pgfqpoint{3.365625in}{2.021795in}}%
\pgfpathlineto{\pgfqpoint{3.365625in}{2.033592in}}%
\pgfpathlineto{\pgfqpoint{3.383789in}{2.033592in}}%
\pgfpathlineto{\pgfqpoint{3.383789in}{2.021795in}}%
\pgfpathmoveto{\pgfqpoint{3.383789in}{2.021795in}}%
\pgfpathlineto{\pgfqpoint{3.383789in}{2.021795in}}%
\pgfpathlineto{\pgfqpoint{3.383789in}{2.033592in}}%
\pgfpathlineto{\pgfqpoint{3.401954in}{2.033592in}}%
\pgfpathlineto{\pgfqpoint{3.401954in}{2.021795in}}%
\pgfpathmoveto{\pgfqpoint{3.401954in}{2.021795in}}%
\pgfpathlineto{\pgfqpoint{3.401954in}{2.021795in}}%
\pgfpathlineto{\pgfqpoint{3.401954in}{2.033592in}}%
\pgfpathlineto{\pgfqpoint{3.420118in}{2.033592in}}%
\pgfpathlineto{\pgfqpoint{3.420118in}{2.021795in}}%
\pgfpathmoveto{\pgfqpoint{3.420118in}{2.021795in}}%
\pgfpathlineto{\pgfqpoint{3.420118in}{2.021795in}}%
\pgfpathlineto{\pgfqpoint{3.420118in}{2.033592in}}%
\pgfpathlineto{\pgfqpoint{3.438282in}{2.033592in}}%
\pgfpathlineto{\pgfqpoint{3.438282in}{2.021795in}}%
\pgfpathmoveto{\pgfqpoint{3.438282in}{2.021795in}}%
\pgfpathlineto{\pgfqpoint{3.438282in}{2.021795in}}%
\pgfpathlineto{\pgfqpoint{3.438282in}{2.033592in}}%
\pgfpathlineto{\pgfqpoint{3.456446in}{2.033592in}}%
\pgfpathlineto{\pgfqpoint{3.456446in}{2.021795in}}%
\pgfpathmoveto{\pgfqpoint{3.456446in}{2.021795in}}%
\pgfpathlineto{\pgfqpoint{3.456446in}{2.021795in}}%
\pgfpathlineto{\pgfqpoint{3.456446in}{2.033592in}}%
\pgfpathlineto{\pgfqpoint{3.474610in}{2.033592in}}%
\pgfpathlineto{\pgfqpoint{3.474610in}{2.021795in}}%
\pgfpathmoveto{\pgfqpoint{3.474610in}{2.021795in}}%
\pgfpathlineto{\pgfqpoint{3.474610in}{2.021795in}}%
\pgfpathlineto{\pgfqpoint{3.474610in}{2.033592in}}%
\pgfpathlineto{\pgfqpoint{3.492775in}{2.033592in}}%
\pgfpathlineto{\pgfqpoint{3.492775in}{2.021795in}}%
\pgfpathmoveto{\pgfqpoint{3.492775in}{2.021795in}}%
\pgfpathlineto{\pgfqpoint{3.492775in}{2.021795in}}%
\pgfpathlineto{\pgfqpoint{3.492775in}{2.033592in}}%
\pgfpathlineto{\pgfqpoint{3.510939in}{2.033592in}}%
\pgfpathlineto{\pgfqpoint{3.510939in}{2.021795in}}%
\pgfpathmoveto{\pgfqpoint{3.365625in}{2.599843in}}%
\pgfpathlineto{\pgfqpoint{3.365625in}{2.599843in}}%
\pgfpathlineto{\pgfqpoint{3.365625in}{2.611639in}}%
\pgfpathlineto{\pgfqpoint{3.383789in}{2.611639in}}%
\pgfpathlineto{\pgfqpoint{3.383789in}{2.599843in}}%
\pgfpathmoveto{\pgfqpoint{3.383789in}{2.599843in}}%
\pgfpathlineto{\pgfqpoint{3.383789in}{2.599843in}}%
\pgfpathlineto{\pgfqpoint{3.383789in}{2.611639in}}%
\pgfpathlineto{\pgfqpoint{3.401954in}{2.611639in}}%
\pgfpathlineto{\pgfqpoint{3.401954in}{2.599843in}}%
\pgfpathmoveto{\pgfqpoint{3.401954in}{2.599843in}}%
\pgfpathlineto{\pgfqpoint{3.401954in}{2.599843in}}%
\pgfpathlineto{\pgfqpoint{3.401954in}{2.611639in}}%
\pgfpathlineto{\pgfqpoint{3.420118in}{2.611639in}}%
\pgfpathlineto{\pgfqpoint{3.420118in}{2.599843in}}%
\pgfpathmoveto{\pgfqpoint{3.420118in}{2.599843in}}%
\pgfpathlineto{\pgfqpoint{3.420118in}{2.599843in}}%
\pgfpathlineto{\pgfqpoint{3.420118in}{2.611639in}}%
\pgfpathlineto{\pgfqpoint{3.438282in}{2.611639in}}%
\pgfpathlineto{\pgfqpoint{3.438282in}{2.599843in}}%
\pgfpathmoveto{\pgfqpoint{3.438282in}{2.599843in}}%
\pgfpathlineto{\pgfqpoint{3.438282in}{2.599843in}}%
\pgfpathlineto{\pgfqpoint{3.438282in}{2.611639in}}%
\pgfpathlineto{\pgfqpoint{3.456446in}{2.611639in}}%
\pgfpathlineto{\pgfqpoint{3.456446in}{2.599843in}}%
\pgfpathmoveto{\pgfqpoint{3.456446in}{2.599843in}}%
\pgfpathlineto{\pgfqpoint{3.456446in}{2.599843in}}%
\pgfpathlineto{\pgfqpoint{3.456446in}{2.611639in}}%
\pgfpathlineto{\pgfqpoint{3.474610in}{2.611639in}}%
\pgfpathlineto{\pgfqpoint{3.474610in}{2.599843in}}%
\pgfpathmoveto{\pgfqpoint{3.474610in}{2.599843in}}%
\pgfpathlineto{\pgfqpoint{3.474610in}{2.599843in}}%
\pgfpathlineto{\pgfqpoint{3.474610in}{2.611639in}}%
\pgfpathlineto{\pgfqpoint{3.492775in}{2.611639in}}%
\pgfpathlineto{\pgfqpoint{3.492775in}{2.599843in}}%
\pgfpathmoveto{\pgfqpoint{3.492775in}{2.599843in}}%
\pgfpathlineto{\pgfqpoint{3.492775in}{2.599843in}}%
\pgfpathlineto{\pgfqpoint{3.492775in}{2.611639in}}%
\pgfpathlineto{\pgfqpoint{3.510939in}{2.611639in}}%
\pgfpathlineto{\pgfqpoint{3.510939in}{2.599843in}}%
\pgfpathmoveto{\pgfqpoint{3.510939in}{2.021795in}}%
\pgfpathlineto{\pgfqpoint{3.510939in}{2.021795in}}%
\pgfpathlineto{\pgfqpoint{3.510939in}{2.033592in}}%
\pgfpathlineto{\pgfqpoint{3.529103in}{2.033592in}}%
\pgfpathlineto{\pgfqpoint{3.529103in}{2.021795in}}%
\pgfpathmoveto{\pgfqpoint{3.529103in}{2.021795in}}%
\pgfpathlineto{\pgfqpoint{3.529103in}{2.021795in}}%
\pgfpathlineto{\pgfqpoint{3.529103in}{2.033592in}}%
\pgfpathlineto{\pgfqpoint{3.547267in}{2.033592in}}%
\pgfpathlineto{\pgfqpoint{3.547267in}{2.021795in}}%
\pgfpathmoveto{\pgfqpoint{3.547267in}{2.021795in}}%
\pgfpathlineto{\pgfqpoint{3.547267in}{2.021795in}}%
\pgfpathlineto{\pgfqpoint{3.547267in}{2.033592in}}%
\pgfpathlineto{\pgfqpoint{3.565432in}{2.033592in}}%
\pgfpathlineto{\pgfqpoint{3.565432in}{2.021795in}}%
\pgfpathmoveto{\pgfqpoint{3.565432in}{2.021795in}}%
\pgfpathlineto{\pgfqpoint{3.565432in}{2.021795in}}%
\pgfpathlineto{\pgfqpoint{3.565432in}{2.033592in}}%
\pgfpathlineto{\pgfqpoint{3.583596in}{2.033592in}}%
\pgfpathlineto{\pgfqpoint{3.583596in}{2.021795in}}%
\pgfpathmoveto{\pgfqpoint{3.583596in}{2.021795in}}%
\pgfpathlineto{\pgfqpoint{3.583596in}{2.021795in}}%
\pgfpathlineto{\pgfqpoint{3.583596in}{2.033592in}}%
\pgfpathlineto{\pgfqpoint{3.601760in}{2.033592in}}%
\pgfpathlineto{\pgfqpoint{3.601760in}{2.021795in}}%
\pgfpathmoveto{\pgfqpoint{3.601760in}{2.021795in}}%
\pgfpathlineto{\pgfqpoint{3.601760in}{2.021795in}}%
\pgfpathlineto{\pgfqpoint{3.601760in}{2.033592in}}%
\pgfpathlineto{\pgfqpoint{3.619924in}{2.033592in}}%
\pgfpathlineto{\pgfqpoint{3.619924in}{2.021795in}}%
\pgfpathmoveto{\pgfqpoint{3.619924in}{2.021795in}}%
\pgfpathlineto{\pgfqpoint{3.619924in}{2.021795in}}%
\pgfpathlineto{\pgfqpoint{3.619924in}{2.033592in}}%
\pgfpathlineto{\pgfqpoint{3.638089in}{2.033592in}}%
\pgfpathlineto{\pgfqpoint{3.638089in}{2.021795in}}%
\pgfpathmoveto{\pgfqpoint{3.638089in}{2.021795in}}%
\pgfpathlineto{\pgfqpoint{3.638089in}{2.021795in}}%
\pgfpathlineto{\pgfqpoint{3.638089in}{2.033592in}}%
\pgfpathlineto{\pgfqpoint{3.656253in}{2.033592in}}%
\pgfpathlineto{\pgfqpoint{3.656253in}{2.021795in}}%
\pgfpathmoveto{\pgfqpoint{3.583596in}{2.552655in}}%
\pgfpathlineto{\pgfqpoint{3.583596in}{2.552655in}}%
\pgfpathlineto{\pgfqpoint{3.583596in}{2.564452in}}%
\pgfpathlineto{\pgfqpoint{3.601760in}{2.564452in}}%
\pgfpathlineto{\pgfqpoint{3.601760in}{2.552655in}}%
\pgfpathmoveto{\pgfqpoint{3.619924in}{2.529061in}}%
\pgfpathlineto{\pgfqpoint{3.619924in}{2.529061in}}%
\pgfpathlineto{\pgfqpoint{3.619924in}{2.540858in}}%
\pgfpathlineto{\pgfqpoint{3.638089in}{2.540858in}}%
\pgfpathlineto{\pgfqpoint{3.638089in}{2.529061in}}%
\pgfpathmoveto{\pgfqpoint{3.510939in}{2.599843in}}%
\pgfpathlineto{\pgfqpoint{3.510939in}{2.599843in}}%
\pgfpathlineto{\pgfqpoint{3.510939in}{2.611639in}}%
\pgfpathlineto{\pgfqpoint{3.529103in}{2.611639in}}%
\pgfpathlineto{\pgfqpoint{3.529103in}{2.599843in}}%
\pgfpathmoveto{\pgfqpoint{3.547267in}{2.576249in}}%
\pgfpathlineto{\pgfqpoint{3.547267in}{2.576249in}}%
\pgfpathlineto{\pgfqpoint{3.547267in}{2.588046in}}%
\pgfpathlineto{\pgfqpoint{3.565432in}{2.588046in}}%
\pgfpathlineto{\pgfqpoint{3.565432in}{2.576249in}}%
\pgfpathmoveto{\pgfqpoint{3.656253in}{2.021795in}}%
\pgfpathlineto{\pgfqpoint{3.656253in}{2.021795in}}%
\pgfpathlineto{\pgfqpoint{3.656253in}{2.033592in}}%
\pgfpathlineto{\pgfqpoint{3.674417in}{2.033592in}}%
\pgfpathlineto{\pgfqpoint{3.674417in}{2.021795in}}%
\pgfpathmoveto{\pgfqpoint{3.674417in}{2.021795in}}%
\pgfpathlineto{\pgfqpoint{3.674417in}{2.021795in}}%
\pgfpathlineto{\pgfqpoint{3.674417in}{2.033592in}}%
\pgfpathlineto{\pgfqpoint{3.692580in}{2.033592in}}%
\pgfpathlineto{\pgfqpoint{3.692580in}{2.021795in}}%
\pgfpathmoveto{\pgfqpoint{3.692580in}{2.021795in}}%
\pgfpathlineto{\pgfqpoint{3.692580in}{2.021795in}}%
\pgfpathlineto{\pgfqpoint{3.692580in}{2.033592in}}%
\pgfpathlineto{\pgfqpoint{3.710744in}{2.033592in}}%
\pgfpathlineto{\pgfqpoint{3.710744in}{2.021795in}}%
\pgfpathmoveto{\pgfqpoint{3.710744in}{2.021795in}}%
\pgfpathlineto{\pgfqpoint{3.710744in}{2.021795in}}%
\pgfpathlineto{\pgfqpoint{3.710744in}{2.033592in}}%
\pgfpathlineto{\pgfqpoint{3.728907in}{2.033592in}}%
\pgfpathlineto{\pgfqpoint{3.728907in}{2.021795in}}%
\pgfpathmoveto{\pgfqpoint{3.728907in}{2.021795in}}%
\pgfpathlineto{\pgfqpoint{3.728907in}{2.021795in}}%
\pgfpathlineto{\pgfqpoint{3.728907in}{2.033592in}}%
\pgfpathlineto{\pgfqpoint{3.747071in}{2.033592in}}%
\pgfpathlineto{\pgfqpoint{3.747071in}{2.021795in}}%
\pgfpathmoveto{\pgfqpoint{3.747071in}{2.021795in}}%
\pgfpathlineto{\pgfqpoint{3.747071in}{2.021795in}}%
\pgfpathlineto{\pgfqpoint{3.747071in}{2.033592in}}%
\pgfpathlineto{\pgfqpoint{3.765234in}{2.033592in}}%
\pgfpathlineto{\pgfqpoint{3.765234in}{2.021795in}}%
\pgfpathmoveto{\pgfqpoint{3.765234in}{2.021795in}}%
\pgfpathlineto{\pgfqpoint{3.765234in}{2.021795in}}%
\pgfpathlineto{\pgfqpoint{3.765234in}{2.033592in}}%
\pgfpathlineto{\pgfqpoint{3.783398in}{2.033592in}}%
\pgfpathlineto{\pgfqpoint{3.783398in}{2.021795in}}%
\pgfpathmoveto{\pgfqpoint{3.783398in}{2.021795in}}%
\pgfpathlineto{\pgfqpoint{3.783398in}{2.021795in}}%
\pgfpathlineto{\pgfqpoint{3.783398in}{2.033592in}}%
\pgfpathlineto{\pgfqpoint{3.801561in}{2.033592in}}%
\pgfpathlineto{\pgfqpoint{3.801561in}{2.021795in}}%
\pgfpathmoveto{\pgfqpoint{3.728907in}{2.458280in}}%
\pgfpathlineto{\pgfqpoint{3.728907in}{2.458280in}}%
\pgfpathlineto{\pgfqpoint{3.728907in}{2.470076in}}%
\pgfpathlineto{\pgfqpoint{3.747071in}{2.470076in}}%
\pgfpathlineto{\pgfqpoint{3.747071in}{2.458280in}}%
\pgfpathmoveto{\pgfqpoint{3.765234in}{2.434687in}}%
\pgfpathlineto{\pgfqpoint{3.765234in}{2.434687in}}%
\pgfpathlineto{\pgfqpoint{3.765234in}{2.446483in}}%
\pgfpathlineto{\pgfqpoint{3.783398in}{2.446483in}}%
\pgfpathlineto{\pgfqpoint{3.783398in}{2.434687in}}%
\pgfpathmoveto{\pgfqpoint{3.656253in}{2.505467in}}%
\pgfpathlineto{\pgfqpoint{3.656253in}{2.505467in}}%
\pgfpathlineto{\pgfqpoint{3.656253in}{2.517264in}}%
\pgfpathlineto{\pgfqpoint{3.674417in}{2.517264in}}%
\pgfpathlineto{\pgfqpoint{3.674417in}{2.505467in}}%
\pgfpathmoveto{\pgfqpoint{3.692580in}{2.481873in}}%
\pgfpathlineto{\pgfqpoint{3.692580in}{2.481873in}}%
\pgfpathlineto{\pgfqpoint{3.692580in}{2.493670in}}%
\pgfpathlineto{\pgfqpoint{3.710744in}{2.493670in}}%
\pgfpathlineto{\pgfqpoint{3.710744in}{2.481873in}}%
\pgfpathmoveto{\pgfqpoint{3.801561in}{2.021795in}}%
\pgfpathlineto{\pgfqpoint{3.801561in}{2.021795in}}%
\pgfpathlineto{\pgfqpoint{3.801561in}{2.033592in}}%
\pgfpathlineto{\pgfqpoint{3.819725in}{2.033592in}}%
\pgfpathlineto{\pgfqpoint{3.819725in}{2.021795in}}%
\pgfpathmoveto{\pgfqpoint{3.819725in}{2.021795in}}%
\pgfpathlineto{\pgfqpoint{3.819725in}{2.021795in}}%
\pgfpathlineto{\pgfqpoint{3.819725in}{2.033592in}}%
\pgfpathlineto{\pgfqpoint{3.837890in}{2.033592in}}%
\pgfpathlineto{\pgfqpoint{3.837890in}{2.021795in}}%
\pgfpathmoveto{\pgfqpoint{3.837890in}{2.021795in}}%
\pgfpathlineto{\pgfqpoint{3.837890in}{2.021795in}}%
\pgfpathlineto{\pgfqpoint{3.837890in}{2.033592in}}%
\pgfpathlineto{\pgfqpoint{3.856054in}{2.033592in}}%
\pgfpathlineto{\pgfqpoint{3.856054in}{2.021795in}}%
\pgfpathmoveto{\pgfqpoint{3.856054in}{2.021795in}}%
\pgfpathlineto{\pgfqpoint{3.856054in}{2.021795in}}%
\pgfpathlineto{\pgfqpoint{3.856054in}{2.033592in}}%
\pgfpathlineto{\pgfqpoint{3.874218in}{2.033592in}}%
\pgfpathlineto{\pgfqpoint{3.874218in}{2.021795in}}%
\pgfpathmoveto{\pgfqpoint{3.874218in}{2.021795in}}%
\pgfpathlineto{\pgfqpoint{3.874218in}{2.021795in}}%
\pgfpathlineto{\pgfqpoint{3.874218in}{2.033592in}}%
\pgfpathlineto{\pgfqpoint{3.892382in}{2.033592in}}%
\pgfpathlineto{\pgfqpoint{3.892382in}{2.021795in}}%
\pgfpathmoveto{\pgfqpoint{3.892382in}{2.021795in}}%
\pgfpathlineto{\pgfqpoint{3.892382in}{2.021795in}}%
\pgfpathlineto{\pgfqpoint{3.892382in}{2.033592in}}%
\pgfpathlineto{\pgfqpoint{3.910546in}{2.033592in}}%
\pgfpathlineto{\pgfqpoint{3.910546in}{2.021795in}}%
\pgfpathmoveto{\pgfqpoint{3.910546in}{2.021795in}}%
\pgfpathlineto{\pgfqpoint{3.910546in}{2.021795in}}%
\pgfpathlineto{\pgfqpoint{3.910546in}{2.033592in}}%
\pgfpathlineto{\pgfqpoint{3.928710in}{2.033592in}}%
\pgfpathlineto{\pgfqpoint{3.928710in}{2.021795in}}%
\pgfpathmoveto{\pgfqpoint{3.928710in}{2.021795in}}%
\pgfpathlineto{\pgfqpoint{3.928710in}{2.021795in}}%
\pgfpathlineto{\pgfqpoint{3.928710in}{2.033592in}}%
\pgfpathlineto{\pgfqpoint{3.946874in}{2.033592in}}%
\pgfpathlineto{\pgfqpoint{3.946874in}{2.021795in}}%
\pgfpathmoveto{\pgfqpoint{3.874218in}{2.363907in}}%
\pgfpathlineto{\pgfqpoint{3.874218in}{2.363907in}}%
\pgfpathlineto{\pgfqpoint{3.874218in}{2.375705in}}%
\pgfpathlineto{\pgfqpoint{3.892382in}{2.375705in}}%
\pgfpathlineto{\pgfqpoint{3.892382in}{2.363907in}}%
\pgfpathmoveto{\pgfqpoint{3.910546in}{2.340313in}}%
\pgfpathlineto{\pgfqpoint{3.910546in}{2.340313in}}%
\pgfpathlineto{\pgfqpoint{3.910546in}{2.352110in}}%
\pgfpathlineto{\pgfqpoint{3.928710in}{2.352110in}}%
\pgfpathlineto{\pgfqpoint{3.928710in}{2.340313in}}%
\pgfpathmoveto{\pgfqpoint{3.801561in}{2.411094in}}%
\pgfpathlineto{\pgfqpoint{3.801561in}{2.411094in}}%
\pgfpathlineto{\pgfqpoint{3.801561in}{2.422891in}}%
\pgfpathlineto{\pgfqpoint{3.819725in}{2.422891in}}%
\pgfpathlineto{\pgfqpoint{3.819725in}{2.411094in}}%
\pgfpathmoveto{\pgfqpoint{3.837890in}{2.387502in}}%
\pgfpathlineto{\pgfqpoint{3.837890in}{2.387502in}}%
\pgfpathlineto{\pgfqpoint{3.837890in}{2.399298in}}%
\pgfpathlineto{\pgfqpoint{3.856054in}{2.399298in}}%
\pgfpathlineto{\pgfqpoint{3.856054in}{2.387502in}}%
\pgfpathmoveto{\pgfqpoint{3.946874in}{2.021795in}}%
\pgfpathlineto{\pgfqpoint{3.946874in}{2.021795in}}%
\pgfpathlineto{\pgfqpoint{3.946874in}{2.033592in}}%
\pgfpathlineto{\pgfqpoint{3.965038in}{2.033592in}}%
\pgfpathlineto{\pgfqpoint{3.965038in}{2.021795in}}%
\pgfpathmoveto{\pgfqpoint{3.965038in}{2.021795in}}%
\pgfpathlineto{\pgfqpoint{3.965038in}{2.021795in}}%
\pgfpathlineto{\pgfqpoint{3.965038in}{2.033592in}}%
\pgfpathlineto{\pgfqpoint{3.983202in}{2.033592in}}%
\pgfpathlineto{\pgfqpoint{3.983202in}{2.021795in}}%
\pgfpathmoveto{\pgfqpoint{3.983202in}{2.021795in}}%
\pgfpathlineto{\pgfqpoint{3.983202in}{2.021795in}}%
\pgfpathlineto{\pgfqpoint{3.983202in}{2.033592in}}%
\pgfpathlineto{\pgfqpoint{4.001365in}{2.033592in}}%
\pgfpathlineto{\pgfqpoint{4.001365in}{2.021795in}}%
\pgfpathmoveto{\pgfqpoint{4.001365in}{2.021795in}}%
\pgfpathlineto{\pgfqpoint{4.001365in}{2.021795in}}%
\pgfpathlineto{\pgfqpoint{4.001365in}{2.033592in}}%
\pgfpathlineto{\pgfqpoint{4.019529in}{2.033592in}}%
\pgfpathlineto{\pgfqpoint{4.019529in}{2.021795in}}%
\pgfpathmoveto{\pgfqpoint{4.019529in}{2.021795in}}%
\pgfpathlineto{\pgfqpoint{4.019529in}{2.021795in}}%
\pgfpathlineto{\pgfqpoint{4.019529in}{2.033592in}}%
\pgfpathlineto{\pgfqpoint{4.037693in}{2.033592in}}%
\pgfpathlineto{\pgfqpoint{4.037693in}{2.021795in}}%
\pgfpathmoveto{\pgfqpoint{4.037693in}{2.021795in}}%
\pgfpathlineto{\pgfqpoint{4.037693in}{2.021795in}}%
\pgfpathlineto{\pgfqpoint{4.037693in}{2.033592in}}%
\pgfpathlineto{\pgfqpoint{4.055857in}{2.033592in}}%
\pgfpathlineto{\pgfqpoint{4.055857in}{2.021795in}}%
\pgfpathmoveto{\pgfqpoint{4.055857in}{2.021795in}}%
\pgfpathlineto{\pgfqpoint{4.055857in}{2.021795in}}%
\pgfpathlineto{\pgfqpoint{4.055857in}{2.033592in}}%
\pgfpathlineto{\pgfqpoint{4.074021in}{2.033592in}}%
\pgfpathlineto{\pgfqpoint{4.074021in}{2.021795in}}%
\pgfpathmoveto{\pgfqpoint{4.074021in}{2.021795in}}%
\pgfpathlineto{\pgfqpoint{4.074021in}{2.021795in}}%
\pgfpathlineto{\pgfqpoint{4.074021in}{2.033592in}}%
\pgfpathlineto{\pgfqpoint{4.092185in}{2.033592in}}%
\pgfpathlineto{\pgfqpoint{4.092185in}{2.021795in}}%
\pgfpathmoveto{\pgfqpoint{4.019529in}{2.269531in}}%
\pgfpathlineto{\pgfqpoint{4.019529in}{2.269531in}}%
\pgfpathlineto{\pgfqpoint{4.019529in}{2.281328in}}%
\pgfpathlineto{\pgfqpoint{4.037693in}{2.281328in}}%
\pgfpathlineto{\pgfqpoint{4.037693in}{2.269531in}}%
\pgfpathmoveto{\pgfqpoint{4.055857in}{2.245936in}}%
\pgfpathlineto{\pgfqpoint{4.055857in}{2.245936in}}%
\pgfpathlineto{\pgfqpoint{4.055857in}{2.257733in}}%
\pgfpathlineto{\pgfqpoint{4.074021in}{2.257733in}}%
\pgfpathlineto{\pgfqpoint{4.074021in}{2.245936in}}%
\pgfpathmoveto{\pgfqpoint{3.946874in}{2.316719in}}%
\pgfpathlineto{\pgfqpoint{3.946874in}{2.316719in}}%
\pgfpathlineto{\pgfqpoint{3.946874in}{2.328516in}}%
\pgfpathlineto{\pgfqpoint{3.965038in}{2.328516in}}%
\pgfpathlineto{\pgfqpoint{3.965038in}{2.316719in}}%
\pgfpathmoveto{\pgfqpoint{3.983202in}{2.293125in}}%
\pgfpathlineto{\pgfqpoint{3.983202in}{2.293125in}}%
\pgfpathlineto{\pgfqpoint{3.983202in}{2.304922in}}%
\pgfpathlineto{\pgfqpoint{4.001365in}{2.304922in}}%
\pgfpathlineto{\pgfqpoint{4.001365in}{2.293125in}}%
\pgfpathmoveto{\pgfqpoint{4.092185in}{2.021795in}}%
\pgfpathlineto{\pgfqpoint{4.092185in}{2.021795in}}%
\pgfpathlineto{\pgfqpoint{4.092185in}{2.033592in}}%
\pgfpathlineto{\pgfqpoint{4.110350in}{2.033592in}}%
\pgfpathlineto{\pgfqpoint{4.110350in}{2.021795in}}%
\pgfpathmoveto{\pgfqpoint{4.110350in}{2.021795in}}%
\pgfpathlineto{\pgfqpoint{4.110350in}{2.021795in}}%
\pgfpathlineto{\pgfqpoint{4.110350in}{2.033592in}}%
\pgfpathlineto{\pgfqpoint{4.128515in}{2.033592in}}%
\pgfpathlineto{\pgfqpoint{4.128515in}{2.021795in}}%
\pgfpathmoveto{\pgfqpoint{4.128515in}{2.021795in}}%
\pgfpathlineto{\pgfqpoint{4.128515in}{2.021795in}}%
\pgfpathlineto{\pgfqpoint{4.128515in}{2.033592in}}%
\pgfpathlineto{\pgfqpoint{4.146680in}{2.033592in}}%
\pgfpathlineto{\pgfqpoint{4.146680in}{2.021795in}}%
\pgfpathmoveto{\pgfqpoint{4.146680in}{2.021795in}}%
\pgfpathlineto{\pgfqpoint{4.146680in}{2.021795in}}%
\pgfpathlineto{\pgfqpoint{4.146680in}{2.033592in}}%
\pgfpathlineto{\pgfqpoint{4.164844in}{2.033592in}}%
\pgfpathlineto{\pgfqpoint{4.164844in}{2.021795in}}%
\pgfpathmoveto{\pgfqpoint{4.164844in}{2.021795in}}%
\pgfpathlineto{\pgfqpoint{4.164844in}{2.021795in}}%
\pgfpathlineto{\pgfqpoint{4.164844in}{2.033592in}}%
\pgfpathlineto{\pgfqpoint{4.183009in}{2.033592in}}%
\pgfpathlineto{\pgfqpoint{4.183009in}{2.021795in}}%
\pgfpathmoveto{\pgfqpoint{4.183009in}{2.021795in}}%
\pgfpathlineto{\pgfqpoint{4.183009in}{2.021795in}}%
\pgfpathlineto{\pgfqpoint{4.183009in}{2.033592in}}%
\pgfpathlineto{\pgfqpoint{4.201174in}{2.033592in}}%
\pgfpathlineto{\pgfqpoint{4.201174in}{2.021795in}}%
\pgfpathmoveto{\pgfqpoint{4.201174in}{2.021795in}}%
\pgfpathlineto{\pgfqpoint{4.201174in}{2.021795in}}%
\pgfpathlineto{\pgfqpoint{4.201174in}{2.033592in}}%
\pgfpathlineto{\pgfqpoint{4.219339in}{2.033592in}}%
\pgfpathlineto{\pgfqpoint{4.219339in}{2.021795in}}%
\pgfpathmoveto{\pgfqpoint{4.219339in}{2.021795in}}%
\pgfpathlineto{\pgfqpoint{4.219339in}{2.021795in}}%
\pgfpathlineto{\pgfqpoint{4.219339in}{2.033592in}}%
\pgfpathlineto{\pgfqpoint{4.237504in}{2.033592in}}%
\pgfpathlineto{\pgfqpoint{4.237504in}{2.021795in}}%
\pgfpathmoveto{\pgfqpoint{4.164844in}{2.175155in}}%
\pgfpathlineto{\pgfqpoint{4.164844in}{2.175155in}}%
\pgfpathlineto{\pgfqpoint{4.164844in}{2.186952in}}%
\pgfpathlineto{\pgfqpoint{4.183009in}{2.186952in}}%
\pgfpathlineto{\pgfqpoint{4.183009in}{2.175155in}}%
\pgfpathmoveto{\pgfqpoint{4.201174in}{2.151563in}}%
\pgfpathlineto{\pgfqpoint{4.201174in}{2.151563in}}%
\pgfpathlineto{\pgfqpoint{4.201174in}{2.163359in}}%
\pgfpathlineto{\pgfqpoint{4.219339in}{2.163359in}}%
\pgfpathlineto{\pgfqpoint{4.219339in}{2.151563in}}%
\pgfpathmoveto{\pgfqpoint{4.092185in}{2.222342in}}%
\pgfpathlineto{\pgfqpoint{4.092185in}{2.222342in}}%
\pgfpathlineto{\pgfqpoint{4.092185in}{2.234139in}}%
\pgfpathlineto{\pgfqpoint{4.110350in}{2.234139in}}%
\pgfpathlineto{\pgfqpoint{4.110350in}{2.222342in}}%
\pgfpathmoveto{\pgfqpoint{4.128515in}{2.198748in}}%
\pgfpathlineto{\pgfqpoint{4.128515in}{2.198748in}}%
\pgfpathlineto{\pgfqpoint{4.128515in}{2.210545in}}%
\pgfpathlineto{\pgfqpoint{4.146680in}{2.210545in}}%
\pgfpathlineto{\pgfqpoint{4.146680in}{2.198748in}}%
\pgfpathmoveto{\pgfqpoint{4.237504in}{2.021795in}}%
\pgfpathlineto{\pgfqpoint{4.237504in}{2.021795in}}%
\pgfpathlineto{\pgfqpoint{4.237504in}{2.033592in}}%
\pgfpathlineto{\pgfqpoint{4.255667in}{2.033592in}}%
\pgfpathlineto{\pgfqpoint{4.255667in}{2.021795in}}%
\pgfpathmoveto{\pgfqpoint{4.255667in}{2.021795in}}%
\pgfpathlineto{\pgfqpoint{4.255667in}{2.021795in}}%
\pgfpathlineto{\pgfqpoint{4.255667in}{2.033592in}}%
\pgfpathlineto{\pgfqpoint{4.273831in}{2.033592in}}%
\pgfpathlineto{\pgfqpoint{4.273831in}{2.021795in}}%
\pgfpathmoveto{\pgfqpoint{4.273831in}{2.021795in}}%
\pgfpathlineto{\pgfqpoint{4.273831in}{2.021795in}}%
\pgfpathlineto{\pgfqpoint{4.273831in}{2.033592in}}%
\pgfpathlineto{\pgfqpoint{4.291994in}{2.033592in}}%
\pgfpathlineto{\pgfqpoint{4.291994in}{2.021795in}}%
\pgfpathmoveto{\pgfqpoint{4.291994in}{2.021795in}}%
\pgfpathlineto{\pgfqpoint{4.291994in}{2.021795in}}%
\pgfpathlineto{\pgfqpoint{4.291994in}{2.033592in}}%
\pgfpathlineto{\pgfqpoint{4.310157in}{2.033592in}}%
\pgfpathlineto{\pgfqpoint{4.310157in}{2.021795in}}%
\pgfpathmoveto{\pgfqpoint{4.310157in}{2.021795in}}%
\pgfpathlineto{\pgfqpoint{4.310157in}{2.021795in}}%
\pgfpathlineto{\pgfqpoint{4.310157in}{2.033592in}}%
\pgfpathlineto{\pgfqpoint{4.328321in}{2.033592in}}%
\pgfpathlineto{\pgfqpoint{4.328321in}{2.021795in}}%
\pgfpathmoveto{\pgfqpoint{4.328321in}{2.021795in}}%
\pgfpathlineto{\pgfqpoint{4.328321in}{2.021795in}}%
\pgfpathlineto{\pgfqpoint{4.328321in}{2.033592in}}%
\pgfpathlineto{\pgfqpoint{4.346484in}{2.033592in}}%
\pgfpathlineto{\pgfqpoint{4.346484in}{2.021795in}}%
\pgfpathmoveto{\pgfqpoint{4.346484in}{2.021795in}}%
\pgfpathlineto{\pgfqpoint{4.346484in}{2.021795in}}%
\pgfpathlineto{\pgfqpoint{4.346484in}{2.033592in}}%
\pgfpathlineto{\pgfqpoint{4.364647in}{2.033592in}}%
\pgfpathlineto{\pgfqpoint{4.364647in}{2.021795in}}%
\pgfpathmoveto{\pgfqpoint{4.364647in}{2.021795in}}%
\pgfpathlineto{\pgfqpoint{4.364647in}{2.021795in}}%
\pgfpathlineto{\pgfqpoint{4.364647in}{2.033592in}}%
\pgfpathlineto{\pgfqpoint{4.382811in}{2.033592in}}%
\pgfpathlineto{\pgfqpoint{4.382811in}{2.021795in}}%
\pgfpathmoveto{\pgfqpoint{4.310157in}{2.080783in}}%
\pgfpathlineto{\pgfqpoint{4.310157in}{2.080783in}}%
\pgfpathlineto{\pgfqpoint{4.310157in}{2.092580in}}%
\pgfpathlineto{\pgfqpoint{4.328321in}{2.092580in}}%
\pgfpathlineto{\pgfqpoint{4.328321in}{2.080783in}}%
\pgfpathmoveto{\pgfqpoint{4.346484in}{2.057187in}}%
\pgfpathlineto{\pgfqpoint{4.346484in}{2.057187in}}%
\pgfpathlineto{\pgfqpoint{4.346484in}{2.068985in}}%
\pgfpathlineto{\pgfqpoint{4.364647in}{2.068985in}}%
\pgfpathlineto{\pgfqpoint{4.364647in}{2.057187in}}%
\pgfpathmoveto{\pgfqpoint{4.237504in}{2.127970in}}%
\pgfpathlineto{\pgfqpoint{4.237504in}{2.127970in}}%
\pgfpathlineto{\pgfqpoint{4.237504in}{2.139766in}}%
\pgfpathlineto{\pgfqpoint{4.255667in}{2.139766in}}%
\pgfpathlineto{\pgfqpoint{4.255667in}{2.127970in}}%
\pgfpathmoveto{\pgfqpoint{4.273831in}{2.104378in}}%
\pgfpathlineto{\pgfqpoint{4.273831in}{2.104378in}}%
\pgfpathlineto{\pgfqpoint{4.273831in}{2.116174in}}%
\pgfpathlineto{\pgfqpoint{4.291994in}{2.116174in}}%
\pgfpathlineto{\pgfqpoint{4.291994in}{2.104378in}}%
\pgfpathmoveto{\pgfqpoint{4.382811in}{2.021795in}}%
\pgfpathlineto{\pgfqpoint{4.382811in}{2.021795in}}%
\pgfpathlineto{\pgfqpoint{4.382811in}{2.033592in}}%
\pgfpathlineto{\pgfqpoint{4.400975in}{2.033592in}}%
\pgfpathlineto{\pgfqpoint{4.400975in}{2.021795in}}%
\pgfpathmoveto{\pgfqpoint{4.400975in}{2.021795in}}%
\pgfpathlineto{\pgfqpoint{4.400975in}{2.021795in}}%
\pgfpathlineto{\pgfqpoint{4.400975in}{2.033592in}}%
\pgfpathlineto{\pgfqpoint{4.419140in}{2.033592in}}%
\pgfpathlineto{\pgfqpoint{4.419140in}{2.021795in}}%
\pgfpathmoveto{\pgfqpoint{4.382811in}{2.033592in}}%
\pgfpathlineto{\pgfqpoint{4.382811in}{2.033592in}}%
\pgfpathlineto{\pgfqpoint{4.382811in}{2.045390in}}%
\pgfpathlineto{\pgfqpoint{4.400975in}{2.045390in}}%
\pgfpathlineto{\pgfqpoint{4.400975in}{2.033592in}}%
\pgfpathmoveto{\pgfqpoint{3.075004in}{2.015896in}}%
\pgfpathlineto{\pgfqpoint{3.075004in}{2.015896in}}%
\pgfpathlineto{\pgfqpoint{3.075004in}{2.021795in}}%
\pgfpathlineto{\pgfqpoint{3.084086in}{2.021795in}}%
\pgfpathlineto{\pgfqpoint{3.084086in}{2.015896in}}%
\pgfpathmoveto{\pgfqpoint{3.084086in}{2.015896in}}%
\pgfpathlineto{\pgfqpoint{3.084086in}{2.015896in}}%
\pgfpathlineto{\pgfqpoint{3.084086in}{2.021795in}}%
\pgfpathlineto{\pgfqpoint{3.093168in}{2.021795in}}%
\pgfpathlineto{\pgfqpoint{3.093168in}{2.015896in}}%
\pgfpathmoveto{\pgfqpoint{3.093168in}{2.015896in}}%
\pgfpathlineto{\pgfqpoint{3.093168in}{2.015896in}}%
\pgfpathlineto{\pgfqpoint{3.093168in}{2.021795in}}%
\pgfpathlineto{\pgfqpoint{3.102249in}{2.021795in}}%
\pgfpathlineto{\pgfqpoint{3.102249in}{2.015896in}}%
\pgfpathmoveto{\pgfqpoint{3.102249in}{2.015896in}}%
\pgfpathlineto{\pgfqpoint{3.102249in}{2.015896in}}%
\pgfpathlineto{\pgfqpoint{3.102249in}{2.021795in}}%
\pgfpathlineto{\pgfqpoint{3.111331in}{2.021795in}}%
\pgfpathlineto{\pgfqpoint{3.111331in}{2.015896in}}%
\pgfpathmoveto{\pgfqpoint{3.111331in}{2.015896in}}%
\pgfpathlineto{\pgfqpoint{3.111331in}{2.015896in}}%
\pgfpathlineto{\pgfqpoint{3.111331in}{2.021795in}}%
\pgfpathlineto{\pgfqpoint{3.120413in}{2.021795in}}%
\pgfpathlineto{\pgfqpoint{3.120413in}{2.015896in}}%
\pgfpathmoveto{\pgfqpoint{3.120413in}{2.015896in}}%
\pgfpathlineto{\pgfqpoint{3.120413in}{2.015896in}}%
\pgfpathlineto{\pgfqpoint{3.120413in}{2.021795in}}%
\pgfpathlineto{\pgfqpoint{3.129494in}{2.021795in}}%
\pgfpathlineto{\pgfqpoint{3.129494in}{2.015896in}}%
\pgfpathmoveto{\pgfqpoint{3.129494in}{2.015896in}}%
\pgfpathlineto{\pgfqpoint{3.129494in}{2.015896in}}%
\pgfpathlineto{\pgfqpoint{3.129494in}{2.021795in}}%
\pgfpathlineto{\pgfqpoint{3.138576in}{2.021795in}}%
\pgfpathlineto{\pgfqpoint{3.138576in}{2.015896in}}%
\pgfpathmoveto{\pgfqpoint{3.138576in}{2.015896in}}%
\pgfpathlineto{\pgfqpoint{3.138576in}{2.015896in}}%
\pgfpathlineto{\pgfqpoint{3.138576in}{2.021795in}}%
\pgfpathlineto{\pgfqpoint{3.147658in}{2.021795in}}%
\pgfpathlineto{\pgfqpoint{3.147658in}{2.015896in}}%
\pgfpathmoveto{\pgfqpoint{3.147658in}{2.015896in}}%
\pgfpathlineto{\pgfqpoint{3.147658in}{2.015896in}}%
\pgfpathlineto{\pgfqpoint{3.147658in}{2.021795in}}%
\pgfpathlineto{\pgfqpoint{3.156739in}{2.021795in}}%
\pgfpathlineto{\pgfqpoint{3.156739in}{2.015896in}}%
\pgfpathmoveto{\pgfqpoint{3.156739in}{2.015896in}}%
\pgfpathlineto{\pgfqpoint{3.156739in}{2.015896in}}%
\pgfpathlineto{\pgfqpoint{3.156739in}{2.021795in}}%
\pgfpathlineto{\pgfqpoint{3.165821in}{2.021795in}}%
\pgfpathlineto{\pgfqpoint{3.165821in}{2.015896in}}%
\pgfpathmoveto{\pgfqpoint{3.165821in}{2.015896in}}%
\pgfpathlineto{\pgfqpoint{3.165821in}{2.015896in}}%
\pgfpathlineto{\pgfqpoint{3.165821in}{2.021795in}}%
\pgfpathlineto{\pgfqpoint{3.174903in}{2.021795in}}%
\pgfpathlineto{\pgfqpoint{3.174903in}{2.015896in}}%
\pgfpathmoveto{\pgfqpoint{3.174903in}{2.015896in}}%
\pgfpathlineto{\pgfqpoint{3.174903in}{2.015896in}}%
\pgfpathlineto{\pgfqpoint{3.174903in}{2.021795in}}%
\pgfpathlineto{\pgfqpoint{3.183984in}{2.021795in}}%
\pgfpathlineto{\pgfqpoint{3.183984in}{2.015896in}}%
\pgfpathmoveto{\pgfqpoint{3.183984in}{2.015896in}}%
\pgfpathlineto{\pgfqpoint{3.183984in}{2.015896in}}%
\pgfpathlineto{\pgfqpoint{3.183984in}{2.021795in}}%
\pgfpathlineto{\pgfqpoint{3.193066in}{2.021795in}}%
\pgfpathlineto{\pgfqpoint{3.193066in}{2.015896in}}%
\pgfpathmoveto{\pgfqpoint{3.193066in}{2.015896in}}%
\pgfpathlineto{\pgfqpoint{3.193066in}{2.015896in}}%
\pgfpathlineto{\pgfqpoint{3.193066in}{2.021795in}}%
\pgfpathlineto{\pgfqpoint{3.202148in}{2.021795in}}%
\pgfpathlineto{\pgfqpoint{3.202148in}{2.015896in}}%
\pgfpathmoveto{\pgfqpoint{3.202148in}{2.015896in}}%
\pgfpathlineto{\pgfqpoint{3.202148in}{2.015896in}}%
\pgfpathlineto{\pgfqpoint{3.202148in}{2.021795in}}%
\pgfpathlineto{\pgfqpoint{3.211229in}{2.021795in}}%
\pgfpathlineto{\pgfqpoint{3.211229in}{2.015896in}}%
\pgfpathmoveto{\pgfqpoint{3.211229in}{2.015896in}}%
\pgfpathlineto{\pgfqpoint{3.211229in}{2.015896in}}%
\pgfpathlineto{\pgfqpoint{3.211229in}{2.021795in}}%
\pgfpathlineto{\pgfqpoint{3.220311in}{2.021795in}}%
\pgfpathlineto{\pgfqpoint{3.220311in}{2.015896in}}%
\pgfpathmoveto{\pgfqpoint{3.220311in}{2.015896in}}%
\pgfpathlineto{\pgfqpoint{3.220311in}{2.015896in}}%
\pgfpathlineto{\pgfqpoint{3.220311in}{2.021795in}}%
\pgfpathlineto{\pgfqpoint{3.229393in}{2.021795in}}%
\pgfpathlineto{\pgfqpoint{3.229393in}{2.015896in}}%
\pgfpathmoveto{\pgfqpoint{3.229393in}{2.015896in}}%
\pgfpathlineto{\pgfqpoint{3.229393in}{2.015896in}}%
\pgfpathlineto{\pgfqpoint{3.229393in}{2.021795in}}%
\pgfpathlineto{\pgfqpoint{3.238475in}{2.021795in}}%
\pgfpathlineto{\pgfqpoint{3.238475in}{2.015896in}}%
\pgfpathmoveto{\pgfqpoint{3.238475in}{2.015896in}}%
\pgfpathlineto{\pgfqpoint{3.238475in}{2.015896in}}%
\pgfpathlineto{\pgfqpoint{3.238475in}{2.021795in}}%
\pgfpathlineto{\pgfqpoint{3.247557in}{2.021795in}}%
\pgfpathlineto{\pgfqpoint{3.247557in}{2.015896in}}%
\pgfpathmoveto{\pgfqpoint{3.247557in}{2.015896in}}%
\pgfpathlineto{\pgfqpoint{3.247557in}{2.015896in}}%
\pgfpathlineto{\pgfqpoint{3.247557in}{2.021795in}}%
\pgfpathlineto{\pgfqpoint{3.256640in}{2.021795in}}%
\pgfpathlineto{\pgfqpoint{3.256640in}{2.015896in}}%
\pgfpathmoveto{\pgfqpoint{3.256640in}{2.015896in}}%
\pgfpathlineto{\pgfqpoint{3.256640in}{2.015896in}}%
\pgfpathlineto{\pgfqpoint{3.256640in}{2.021795in}}%
\pgfpathlineto{\pgfqpoint{3.265722in}{2.021795in}}%
\pgfpathlineto{\pgfqpoint{3.265722in}{2.015896in}}%
\pgfpathmoveto{\pgfqpoint{3.265722in}{2.015896in}}%
\pgfpathlineto{\pgfqpoint{3.265722in}{2.015896in}}%
\pgfpathlineto{\pgfqpoint{3.265722in}{2.021795in}}%
\pgfpathlineto{\pgfqpoint{3.274804in}{2.021795in}}%
\pgfpathlineto{\pgfqpoint{3.274804in}{2.015896in}}%
\pgfpathmoveto{\pgfqpoint{3.274804in}{2.015896in}}%
\pgfpathlineto{\pgfqpoint{3.274804in}{2.015896in}}%
\pgfpathlineto{\pgfqpoint{3.274804in}{2.021795in}}%
\pgfpathlineto{\pgfqpoint{3.283886in}{2.021795in}}%
\pgfpathlineto{\pgfqpoint{3.283886in}{2.015896in}}%
\pgfpathmoveto{\pgfqpoint{3.283886in}{2.015896in}}%
\pgfpathlineto{\pgfqpoint{3.283886in}{2.015896in}}%
\pgfpathlineto{\pgfqpoint{3.283886in}{2.021795in}}%
\pgfpathlineto{\pgfqpoint{3.292968in}{2.021795in}}%
\pgfpathlineto{\pgfqpoint{3.292968in}{2.015896in}}%
\pgfpathmoveto{\pgfqpoint{3.292968in}{2.015896in}}%
\pgfpathlineto{\pgfqpoint{3.292968in}{2.015896in}}%
\pgfpathlineto{\pgfqpoint{3.292968in}{2.021795in}}%
\pgfpathlineto{\pgfqpoint{3.302050in}{2.021795in}}%
\pgfpathlineto{\pgfqpoint{3.302050in}{2.015896in}}%
\pgfpathmoveto{\pgfqpoint{3.302050in}{2.015896in}}%
\pgfpathlineto{\pgfqpoint{3.302050in}{2.015896in}}%
\pgfpathlineto{\pgfqpoint{3.302050in}{2.021795in}}%
\pgfpathlineto{\pgfqpoint{3.311132in}{2.021795in}}%
\pgfpathlineto{\pgfqpoint{3.311132in}{2.015896in}}%
\pgfpathmoveto{\pgfqpoint{3.311132in}{2.015896in}}%
\pgfpathlineto{\pgfqpoint{3.311132in}{2.015896in}}%
\pgfpathlineto{\pgfqpoint{3.311132in}{2.021795in}}%
\pgfpathlineto{\pgfqpoint{3.320214in}{2.021795in}}%
\pgfpathlineto{\pgfqpoint{3.320214in}{2.015896in}}%
\pgfpathmoveto{\pgfqpoint{3.320214in}{2.015896in}}%
\pgfpathlineto{\pgfqpoint{3.320214in}{2.015896in}}%
\pgfpathlineto{\pgfqpoint{3.320214in}{2.021795in}}%
\pgfpathlineto{\pgfqpoint{3.329297in}{2.021795in}}%
\pgfpathlineto{\pgfqpoint{3.329297in}{2.015896in}}%
\pgfpathmoveto{\pgfqpoint{3.329297in}{2.015896in}}%
\pgfpathlineto{\pgfqpoint{3.329297in}{2.015896in}}%
\pgfpathlineto{\pgfqpoint{3.329297in}{2.021795in}}%
\pgfpathlineto{\pgfqpoint{3.338379in}{2.021795in}}%
\pgfpathlineto{\pgfqpoint{3.338379in}{2.015896in}}%
\pgfpathmoveto{\pgfqpoint{3.338379in}{2.015896in}}%
\pgfpathlineto{\pgfqpoint{3.338379in}{2.015896in}}%
\pgfpathlineto{\pgfqpoint{3.338379in}{2.021795in}}%
\pgfpathlineto{\pgfqpoint{3.347461in}{2.021795in}}%
\pgfpathlineto{\pgfqpoint{3.347461in}{2.015896in}}%
\pgfpathmoveto{\pgfqpoint{3.347461in}{2.015896in}}%
\pgfpathlineto{\pgfqpoint{3.347461in}{2.015896in}}%
\pgfpathlineto{\pgfqpoint{3.347461in}{2.021795in}}%
\pgfpathlineto{\pgfqpoint{3.356543in}{2.021795in}}%
\pgfpathlineto{\pgfqpoint{3.356543in}{2.015896in}}%
\pgfpathmoveto{\pgfqpoint{3.356543in}{2.015896in}}%
\pgfpathlineto{\pgfqpoint{3.356543in}{2.015896in}}%
\pgfpathlineto{\pgfqpoint{3.356543in}{2.021795in}}%
\pgfpathlineto{\pgfqpoint{3.365625in}{2.021795in}}%
\pgfpathlineto{\pgfqpoint{3.365625in}{2.015896in}}%
\pgfpathmoveto{\pgfqpoint{3.365625in}{2.015896in}}%
\pgfpathlineto{\pgfqpoint{3.365625in}{2.015896in}}%
\pgfpathlineto{\pgfqpoint{3.365625in}{2.021795in}}%
\pgfpathlineto{\pgfqpoint{3.374707in}{2.021795in}}%
\pgfpathlineto{\pgfqpoint{3.374707in}{2.015896in}}%
\pgfpathmoveto{\pgfqpoint{3.374707in}{2.015896in}}%
\pgfpathlineto{\pgfqpoint{3.374707in}{2.015896in}}%
\pgfpathlineto{\pgfqpoint{3.374707in}{2.021795in}}%
\pgfpathlineto{\pgfqpoint{3.383789in}{2.021795in}}%
\pgfpathlineto{\pgfqpoint{3.383789in}{2.015896in}}%
\pgfpathmoveto{\pgfqpoint{3.383789in}{2.015896in}}%
\pgfpathlineto{\pgfqpoint{3.383789in}{2.015896in}}%
\pgfpathlineto{\pgfqpoint{3.383789in}{2.021795in}}%
\pgfpathlineto{\pgfqpoint{3.392871in}{2.021795in}}%
\pgfpathlineto{\pgfqpoint{3.392871in}{2.015896in}}%
\pgfpathmoveto{\pgfqpoint{3.392871in}{2.015896in}}%
\pgfpathlineto{\pgfqpoint{3.392871in}{2.015896in}}%
\pgfpathlineto{\pgfqpoint{3.392871in}{2.021795in}}%
\pgfpathlineto{\pgfqpoint{3.401954in}{2.021795in}}%
\pgfpathlineto{\pgfqpoint{3.401954in}{2.015896in}}%
\pgfpathmoveto{\pgfqpoint{3.401954in}{2.015896in}}%
\pgfpathlineto{\pgfqpoint{3.401954in}{2.015896in}}%
\pgfpathlineto{\pgfqpoint{3.401954in}{2.021795in}}%
\pgfpathlineto{\pgfqpoint{3.411036in}{2.021795in}}%
\pgfpathlineto{\pgfqpoint{3.411036in}{2.015896in}}%
\pgfpathmoveto{\pgfqpoint{3.411036in}{2.015896in}}%
\pgfpathlineto{\pgfqpoint{3.411036in}{2.015896in}}%
\pgfpathlineto{\pgfqpoint{3.411036in}{2.021795in}}%
\pgfpathlineto{\pgfqpoint{3.420118in}{2.021795in}}%
\pgfpathlineto{\pgfqpoint{3.420118in}{2.015896in}}%
\pgfpathmoveto{\pgfqpoint{3.420118in}{2.015896in}}%
\pgfpathlineto{\pgfqpoint{3.420118in}{2.015896in}}%
\pgfpathlineto{\pgfqpoint{3.420118in}{2.021795in}}%
\pgfpathlineto{\pgfqpoint{3.429200in}{2.021795in}}%
\pgfpathlineto{\pgfqpoint{3.429200in}{2.015896in}}%
\pgfpathmoveto{\pgfqpoint{3.429200in}{2.015896in}}%
\pgfpathlineto{\pgfqpoint{3.429200in}{2.015896in}}%
\pgfpathlineto{\pgfqpoint{3.429200in}{2.021795in}}%
\pgfpathlineto{\pgfqpoint{3.438282in}{2.021795in}}%
\pgfpathlineto{\pgfqpoint{3.438282in}{2.015896in}}%
\pgfpathmoveto{\pgfqpoint{3.438282in}{2.015896in}}%
\pgfpathlineto{\pgfqpoint{3.438282in}{2.015896in}}%
\pgfpathlineto{\pgfqpoint{3.438282in}{2.021795in}}%
\pgfpathlineto{\pgfqpoint{3.447364in}{2.021795in}}%
\pgfpathlineto{\pgfqpoint{3.447364in}{2.015896in}}%
\pgfpathmoveto{\pgfqpoint{3.447364in}{2.015896in}}%
\pgfpathlineto{\pgfqpoint{3.447364in}{2.015896in}}%
\pgfpathlineto{\pgfqpoint{3.447364in}{2.021795in}}%
\pgfpathlineto{\pgfqpoint{3.456446in}{2.021795in}}%
\pgfpathlineto{\pgfqpoint{3.456446in}{2.015896in}}%
\pgfpathmoveto{\pgfqpoint{3.456446in}{2.015896in}}%
\pgfpathlineto{\pgfqpoint{3.456446in}{2.015896in}}%
\pgfpathlineto{\pgfqpoint{3.456446in}{2.021795in}}%
\pgfpathlineto{\pgfqpoint{3.465528in}{2.021795in}}%
\pgfpathlineto{\pgfqpoint{3.465528in}{2.015896in}}%
\pgfpathmoveto{\pgfqpoint{3.465528in}{2.015896in}}%
\pgfpathlineto{\pgfqpoint{3.465528in}{2.015896in}}%
\pgfpathlineto{\pgfqpoint{3.465528in}{2.021795in}}%
\pgfpathlineto{\pgfqpoint{3.474610in}{2.021795in}}%
\pgfpathlineto{\pgfqpoint{3.474610in}{2.015896in}}%
\pgfpathmoveto{\pgfqpoint{3.474610in}{2.015896in}}%
\pgfpathlineto{\pgfqpoint{3.474610in}{2.015896in}}%
\pgfpathlineto{\pgfqpoint{3.474610in}{2.021795in}}%
\pgfpathlineto{\pgfqpoint{3.483692in}{2.021795in}}%
\pgfpathlineto{\pgfqpoint{3.483692in}{2.015896in}}%
\pgfpathmoveto{\pgfqpoint{3.483692in}{2.015896in}}%
\pgfpathlineto{\pgfqpoint{3.483692in}{2.015896in}}%
\pgfpathlineto{\pgfqpoint{3.483692in}{2.021795in}}%
\pgfpathlineto{\pgfqpoint{3.492775in}{2.021795in}}%
\pgfpathlineto{\pgfqpoint{3.492775in}{2.015896in}}%
\pgfpathmoveto{\pgfqpoint{3.492775in}{2.015896in}}%
\pgfpathlineto{\pgfqpoint{3.492775in}{2.015896in}}%
\pgfpathlineto{\pgfqpoint{3.492775in}{2.021795in}}%
\pgfpathlineto{\pgfqpoint{3.501857in}{2.021795in}}%
\pgfpathlineto{\pgfqpoint{3.501857in}{2.015896in}}%
\pgfpathmoveto{\pgfqpoint{3.501857in}{2.015896in}}%
\pgfpathlineto{\pgfqpoint{3.501857in}{2.015896in}}%
\pgfpathlineto{\pgfqpoint{3.501857in}{2.021795in}}%
\pgfpathlineto{\pgfqpoint{3.510939in}{2.021795in}}%
\pgfpathlineto{\pgfqpoint{3.510939in}{2.015896in}}%
\pgfpathmoveto{\pgfqpoint{3.510939in}{2.015896in}}%
\pgfpathlineto{\pgfqpoint{3.510939in}{2.015896in}}%
\pgfpathlineto{\pgfqpoint{3.510939in}{2.021795in}}%
\pgfpathlineto{\pgfqpoint{3.520021in}{2.021795in}}%
\pgfpathlineto{\pgfqpoint{3.520021in}{2.015896in}}%
\pgfpathmoveto{\pgfqpoint{3.520021in}{2.015896in}}%
\pgfpathlineto{\pgfqpoint{3.520021in}{2.015896in}}%
\pgfpathlineto{\pgfqpoint{3.520021in}{2.021795in}}%
\pgfpathlineto{\pgfqpoint{3.529103in}{2.021795in}}%
\pgfpathlineto{\pgfqpoint{3.529103in}{2.015896in}}%
\pgfpathmoveto{\pgfqpoint{3.529103in}{2.015896in}}%
\pgfpathlineto{\pgfqpoint{3.529103in}{2.015896in}}%
\pgfpathlineto{\pgfqpoint{3.529103in}{2.021795in}}%
\pgfpathlineto{\pgfqpoint{3.538185in}{2.021795in}}%
\pgfpathlineto{\pgfqpoint{3.538185in}{2.015896in}}%
\pgfpathmoveto{\pgfqpoint{3.538185in}{2.015896in}}%
\pgfpathlineto{\pgfqpoint{3.538185in}{2.015896in}}%
\pgfpathlineto{\pgfqpoint{3.538185in}{2.021795in}}%
\pgfpathlineto{\pgfqpoint{3.547267in}{2.021795in}}%
\pgfpathlineto{\pgfqpoint{3.547267in}{2.015896in}}%
\pgfpathmoveto{\pgfqpoint{3.547267in}{2.015896in}}%
\pgfpathlineto{\pgfqpoint{3.547267in}{2.015896in}}%
\pgfpathlineto{\pgfqpoint{3.547267in}{2.021795in}}%
\pgfpathlineto{\pgfqpoint{3.556349in}{2.021795in}}%
\pgfpathlineto{\pgfqpoint{3.556349in}{2.015896in}}%
\pgfpathmoveto{\pgfqpoint{3.556349in}{2.015896in}}%
\pgfpathlineto{\pgfqpoint{3.556349in}{2.015896in}}%
\pgfpathlineto{\pgfqpoint{3.556349in}{2.021795in}}%
\pgfpathlineto{\pgfqpoint{3.565432in}{2.021795in}}%
\pgfpathlineto{\pgfqpoint{3.565432in}{2.015896in}}%
\pgfpathmoveto{\pgfqpoint{3.565432in}{2.015896in}}%
\pgfpathlineto{\pgfqpoint{3.565432in}{2.015896in}}%
\pgfpathlineto{\pgfqpoint{3.565432in}{2.021795in}}%
\pgfpathlineto{\pgfqpoint{3.574514in}{2.021795in}}%
\pgfpathlineto{\pgfqpoint{3.574514in}{2.015896in}}%
\pgfpathmoveto{\pgfqpoint{3.574514in}{2.015896in}}%
\pgfpathlineto{\pgfqpoint{3.574514in}{2.015896in}}%
\pgfpathlineto{\pgfqpoint{3.574514in}{2.021795in}}%
\pgfpathlineto{\pgfqpoint{3.583596in}{2.021795in}}%
\pgfpathlineto{\pgfqpoint{3.583596in}{2.015896in}}%
\pgfpathmoveto{\pgfqpoint{3.583596in}{2.015896in}}%
\pgfpathlineto{\pgfqpoint{3.583596in}{2.015896in}}%
\pgfpathlineto{\pgfqpoint{3.583596in}{2.021795in}}%
\pgfpathlineto{\pgfqpoint{3.592678in}{2.021795in}}%
\pgfpathlineto{\pgfqpoint{3.592678in}{2.015896in}}%
\pgfpathmoveto{\pgfqpoint{3.592678in}{2.015896in}}%
\pgfpathlineto{\pgfqpoint{3.592678in}{2.015896in}}%
\pgfpathlineto{\pgfqpoint{3.592678in}{2.021795in}}%
\pgfpathlineto{\pgfqpoint{3.601760in}{2.021795in}}%
\pgfpathlineto{\pgfqpoint{3.601760in}{2.015896in}}%
\pgfpathmoveto{\pgfqpoint{3.601760in}{2.015896in}}%
\pgfpathlineto{\pgfqpoint{3.601760in}{2.015896in}}%
\pgfpathlineto{\pgfqpoint{3.601760in}{2.021795in}}%
\pgfpathlineto{\pgfqpoint{3.610842in}{2.021795in}}%
\pgfpathlineto{\pgfqpoint{3.610842in}{2.015896in}}%
\pgfpathmoveto{\pgfqpoint{3.610842in}{2.015896in}}%
\pgfpathlineto{\pgfqpoint{3.610842in}{2.015896in}}%
\pgfpathlineto{\pgfqpoint{3.610842in}{2.021795in}}%
\pgfpathlineto{\pgfqpoint{3.619924in}{2.021795in}}%
\pgfpathlineto{\pgfqpoint{3.619924in}{2.015896in}}%
\pgfpathmoveto{\pgfqpoint{3.619924in}{2.015896in}}%
\pgfpathlineto{\pgfqpoint{3.619924in}{2.015896in}}%
\pgfpathlineto{\pgfqpoint{3.619924in}{2.021795in}}%
\pgfpathlineto{\pgfqpoint{3.629007in}{2.021795in}}%
\pgfpathlineto{\pgfqpoint{3.629007in}{2.015896in}}%
\pgfpathmoveto{\pgfqpoint{3.629007in}{2.015896in}}%
\pgfpathlineto{\pgfqpoint{3.629007in}{2.015896in}}%
\pgfpathlineto{\pgfqpoint{3.629007in}{2.021795in}}%
\pgfpathlineto{\pgfqpoint{3.638089in}{2.021795in}}%
\pgfpathlineto{\pgfqpoint{3.638089in}{2.015896in}}%
\pgfpathmoveto{\pgfqpoint{3.638089in}{2.015896in}}%
\pgfpathlineto{\pgfqpoint{3.638089in}{2.015896in}}%
\pgfpathlineto{\pgfqpoint{3.638089in}{2.021795in}}%
\pgfpathlineto{\pgfqpoint{3.647171in}{2.021795in}}%
\pgfpathlineto{\pgfqpoint{3.647171in}{2.015896in}}%
\pgfpathmoveto{\pgfqpoint{3.647171in}{2.015896in}}%
\pgfpathlineto{\pgfqpoint{3.647171in}{2.015896in}}%
\pgfpathlineto{\pgfqpoint{3.647171in}{2.021795in}}%
\pgfpathlineto{\pgfqpoint{3.656253in}{2.021795in}}%
\pgfpathlineto{\pgfqpoint{3.656253in}{2.015896in}}%
\pgfpathmoveto{\pgfqpoint{3.583596in}{2.564452in}}%
\pgfpathlineto{\pgfqpoint{3.583596in}{2.564452in}}%
\pgfpathlineto{\pgfqpoint{3.583596in}{2.570351in}}%
\pgfpathlineto{\pgfqpoint{3.592678in}{2.570351in}}%
\pgfpathlineto{\pgfqpoint{3.592678in}{2.564452in}}%
\pgfpathmoveto{\pgfqpoint{3.583596in}{2.570351in}}%
\pgfpathlineto{\pgfqpoint{3.583596in}{2.570351in}}%
\pgfpathlineto{\pgfqpoint{3.583596in}{2.576249in}}%
\pgfpathlineto{\pgfqpoint{3.592678in}{2.576249in}}%
\pgfpathlineto{\pgfqpoint{3.592678in}{2.570351in}}%
\pgfpathmoveto{\pgfqpoint{3.592678in}{2.564452in}}%
\pgfpathlineto{\pgfqpoint{3.592678in}{2.564452in}}%
\pgfpathlineto{\pgfqpoint{3.592678in}{2.570351in}}%
\pgfpathlineto{\pgfqpoint{3.601760in}{2.570351in}}%
\pgfpathlineto{\pgfqpoint{3.601760in}{2.564452in}}%
\pgfpathmoveto{\pgfqpoint{3.601760in}{2.552655in}}%
\pgfpathlineto{\pgfqpoint{3.601760in}{2.552655in}}%
\pgfpathlineto{\pgfqpoint{3.601760in}{2.558554in}}%
\pgfpathlineto{\pgfqpoint{3.610842in}{2.558554in}}%
\pgfpathlineto{\pgfqpoint{3.610842in}{2.552655in}}%
\pgfpathmoveto{\pgfqpoint{3.601760in}{2.558554in}}%
\pgfpathlineto{\pgfqpoint{3.601760in}{2.558554in}}%
\pgfpathlineto{\pgfqpoint{3.601760in}{2.564452in}}%
\pgfpathlineto{\pgfqpoint{3.610842in}{2.564452in}}%
\pgfpathlineto{\pgfqpoint{3.610842in}{2.558554in}}%
\pgfpathmoveto{\pgfqpoint{3.610842in}{2.552655in}}%
\pgfpathlineto{\pgfqpoint{3.610842in}{2.552655in}}%
\pgfpathlineto{\pgfqpoint{3.610842in}{2.558554in}}%
\pgfpathlineto{\pgfqpoint{3.619924in}{2.558554in}}%
\pgfpathlineto{\pgfqpoint{3.619924in}{2.552655in}}%
\pgfpathmoveto{\pgfqpoint{3.619924in}{2.540858in}}%
\pgfpathlineto{\pgfqpoint{3.619924in}{2.540858in}}%
\pgfpathlineto{\pgfqpoint{3.619924in}{2.546757in}}%
\pgfpathlineto{\pgfqpoint{3.629007in}{2.546757in}}%
\pgfpathlineto{\pgfqpoint{3.629007in}{2.540858in}}%
\pgfpathmoveto{\pgfqpoint{3.619924in}{2.546757in}}%
\pgfpathlineto{\pgfqpoint{3.619924in}{2.546757in}}%
\pgfpathlineto{\pgfqpoint{3.619924in}{2.552655in}}%
\pgfpathlineto{\pgfqpoint{3.629007in}{2.552655in}}%
\pgfpathlineto{\pgfqpoint{3.629007in}{2.546757in}}%
\pgfpathmoveto{\pgfqpoint{3.629007in}{2.540858in}}%
\pgfpathlineto{\pgfqpoint{3.629007in}{2.540858in}}%
\pgfpathlineto{\pgfqpoint{3.629007in}{2.546757in}}%
\pgfpathlineto{\pgfqpoint{3.638089in}{2.546757in}}%
\pgfpathlineto{\pgfqpoint{3.638089in}{2.540858in}}%
\pgfpathmoveto{\pgfqpoint{3.638089in}{2.529061in}}%
\pgfpathlineto{\pgfqpoint{3.638089in}{2.529061in}}%
\pgfpathlineto{\pgfqpoint{3.638089in}{2.534959in}}%
\pgfpathlineto{\pgfqpoint{3.647171in}{2.534959in}}%
\pgfpathlineto{\pgfqpoint{3.647171in}{2.529061in}}%
\pgfpathmoveto{\pgfqpoint{3.638089in}{2.534959in}}%
\pgfpathlineto{\pgfqpoint{3.638089in}{2.534959in}}%
\pgfpathlineto{\pgfqpoint{3.638089in}{2.540858in}}%
\pgfpathlineto{\pgfqpoint{3.647171in}{2.540858in}}%
\pgfpathlineto{\pgfqpoint{3.647171in}{2.534959in}}%
\pgfpathmoveto{\pgfqpoint{3.647171in}{2.529061in}}%
\pgfpathlineto{\pgfqpoint{3.647171in}{2.529061in}}%
\pgfpathlineto{\pgfqpoint{3.647171in}{2.534959in}}%
\pgfpathlineto{\pgfqpoint{3.656253in}{2.534959in}}%
\pgfpathlineto{\pgfqpoint{3.656253in}{2.529061in}}%
\pgfpathmoveto{\pgfqpoint{3.529103in}{2.599843in}}%
\pgfpathlineto{\pgfqpoint{3.529103in}{2.599843in}}%
\pgfpathlineto{\pgfqpoint{3.529103in}{2.605741in}}%
\pgfpathlineto{\pgfqpoint{3.538185in}{2.605741in}}%
\pgfpathlineto{\pgfqpoint{3.538185in}{2.599843in}}%
\pgfpathmoveto{\pgfqpoint{3.529103in}{2.605741in}}%
\pgfpathlineto{\pgfqpoint{3.529103in}{2.605741in}}%
\pgfpathlineto{\pgfqpoint{3.529103in}{2.611639in}}%
\pgfpathlineto{\pgfqpoint{3.538185in}{2.611639in}}%
\pgfpathlineto{\pgfqpoint{3.538185in}{2.605741in}}%
\pgfpathmoveto{\pgfqpoint{3.538185in}{2.599843in}}%
\pgfpathlineto{\pgfqpoint{3.538185in}{2.599843in}}%
\pgfpathlineto{\pgfqpoint{3.538185in}{2.605741in}}%
\pgfpathlineto{\pgfqpoint{3.547267in}{2.605741in}}%
\pgfpathlineto{\pgfqpoint{3.547267in}{2.599843in}}%
\pgfpathmoveto{\pgfqpoint{3.547267in}{2.588046in}}%
\pgfpathlineto{\pgfqpoint{3.547267in}{2.588046in}}%
\pgfpathlineto{\pgfqpoint{3.547267in}{2.593944in}}%
\pgfpathlineto{\pgfqpoint{3.556349in}{2.593944in}}%
\pgfpathlineto{\pgfqpoint{3.556349in}{2.588046in}}%
\pgfpathmoveto{\pgfqpoint{3.547267in}{2.593944in}}%
\pgfpathlineto{\pgfqpoint{3.547267in}{2.593944in}}%
\pgfpathlineto{\pgfqpoint{3.547267in}{2.599843in}}%
\pgfpathlineto{\pgfqpoint{3.556349in}{2.599843in}}%
\pgfpathlineto{\pgfqpoint{3.556349in}{2.593944in}}%
\pgfpathmoveto{\pgfqpoint{3.556349in}{2.588046in}}%
\pgfpathlineto{\pgfqpoint{3.556349in}{2.588046in}}%
\pgfpathlineto{\pgfqpoint{3.556349in}{2.593944in}}%
\pgfpathlineto{\pgfqpoint{3.565432in}{2.593944in}}%
\pgfpathlineto{\pgfqpoint{3.565432in}{2.588046in}}%
\pgfpathmoveto{\pgfqpoint{3.565432in}{2.576249in}}%
\pgfpathlineto{\pgfqpoint{3.565432in}{2.576249in}}%
\pgfpathlineto{\pgfqpoint{3.565432in}{2.582148in}}%
\pgfpathlineto{\pgfqpoint{3.574514in}{2.582148in}}%
\pgfpathlineto{\pgfqpoint{3.574514in}{2.576249in}}%
\pgfpathmoveto{\pgfqpoint{3.565432in}{2.582148in}}%
\pgfpathlineto{\pgfqpoint{3.565432in}{2.582148in}}%
\pgfpathlineto{\pgfqpoint{3.565432in}{2.588046in}}%
\pgfpathlineto{\pgfqpoint{3.574514in}{2.588046in}}%
\pgfpathlineto{\pgfqpoint{3.574514in}{2.582148in}}%
\pgfpathmoveto{\pgfqpoint{3.574514in}{2.576249in}}%
\pgfpathlineto{\pgfqpoint{3.574514in}{2.576249in}}%
\pgfpathlineto{\pgfqpoint{3.574514in}{2.582148in}}%
\pgfpathlineto{\pgfqpoint{3.583596in}{2.582148in}}%
\pgfpathlineto{\pgfqpoint{3.583596in}{2.576249in}}%
\pgfpathmoveto{\pgfqpoint{3.656253in}{2.015896in}}%
\pgfpathlineto{\pgfqpoint{3.656253in}{2.015896in}}%
\pgfpathlineto{\pgfqpoint{3.656253in}{2.021795in}}%
\pgfpathlineto{\pgfqpoint{3.665335in}{2.021795in}}%
\pgfpathlineto{\pgfqpoint{3.665335in}{2.015896in}}%
\pgfpathmoveto{\pgfqpoint{3.665335in}{2.015896in}}%
\pgfpathlineto{\pgfqpoint{3.665335in}{2.015896in}}%
\pgfpathlineto{\pgfqpoint{3.665335in}{2.021795in}}%
\pgfpathlineto{\pgfqpoint{3.674417in}{2.021795in}}%
\pgfpathlineto{\pgfqpoint{3.674417in}{2.015896in}}%
\pgfpathmoveto{\pgfqpoint{3.674417in}{2.015896in}}%
\pgfpathlineto{\pgfqpoint{3.674417in}{2.015896in}}%
\pgfpathlineto{\pgfqpoint{3.674417in}{2.021795in}}%
\pgfpathlineto{\pgfqpoint{3.683498in}{2.021795in}}%
\pgfpathlineto{\pgfqpoint{3.683498in}{2.015896in}}%
\pgfpathmoveto{\pgfqpoint{3.683498in}{2.015896in}}%
\pgfpathlineto{\pgfqpoint{3.683498in}{2.015896in}}%
\pgfpathlineto{\pgfqpoint{3.683498in}{2.021795in}}%
\pgfpathlineto{\pgfqpoint{3.692580in}{2.021795in}}%
\pgfpathlineto{\pgfqpoint{3.692580in}{2.015896in}}%
\pgfpathmoveto{\pgfqpoint{3.692580in}{2.015896in}}%
\pgfpathlineto{\pgfqpoint{3.692580in}{2.015896in}}%
\pgfpathlineto{\pgfqpoint{3.692580in}{2.021795in}}%
\pgfpathlineto{\pgfqpoint{3.701662in}{2.021795in}}%
\pgfpathlineto{\pgfqpoint{3.701662in}{2.015896in}}%
\pgfpathmoveto{\pgfqpoint{3.701662in}{2.015896in}}%
\pgfpathlineto{\pgfqpoint{3.701662in}{2.015896in}}%
\pgfpathlineto{\pgfqpoint{3.701662in}{2.021795in}}%
\pgfpathlineto{\pgfqpoint{3.710744in}{2.021795in}}%
\pgfpathlineto{\pgfqpoint{3.710744in}{2.015896in}}%
\pgfpathmoveto{\pgfqpoint{3.710744in}{2.015896in}}%
\pgfpathlineto{\pgfqpoint{3.710744in}{2.015896in}}%
\pgfpathlineto{\pgfqpoint{3.710744in}{2.021795in}}%
\pgfpathlineto{\pgfqpoint{3.719825in}{2.021795in}}%
\pgfpathlineto{\pgfqpoint{3.719825in}{2.015896in}}%
\pgfpathmoveto{\pgfqpoint{3.719825in}{2.015896in}}%
\pgfpathlineto{\pgfqpoint{3.719825in}{2.015896in}}%
\pgfpathlineto{\pgfqpoint{3.719825in}{2.021795in}}%
\pgfpathlineto{\pgfqpoint{3.728907in}{2.021795in}}%
\pgfpathlineto{\pgfqpoint{3.728907in}{2.015896in}}%
\pgfpathmoveto{\pgfqpoint{3.728907in}{2.015896in}}%
\pgfpathlineto{\pgfqpoint{3.728907in}{2.015896in}}%
\pgfpathlineto{\pgfqpoint{3.728907in}{2.021795in}}%
\pgfpathlineto{\pgfqpoint{3.737989in}{2.021795in}}%
\pgfpathlineto{\pgfqpoint{3.737989in}{2.015896in}}%
\pgfpathmoveto{\pgfqpoint{3.737989in}{2.015896in}}%
\pgfpathlineto{\pgfqpoint{3.737989in}{2.015896in}}%
\pgfpathlineto{\pgfqpoint{3.737989in}{2.021795in}}%
\pgfpathlineto{\pgfqpoint{3.747071in}{2.021795in}}%
\pgfpathlineto{\pgfqpoint{3.747071in}{2.015896in}}%
\pgfpathmoveto{\pgfqpoint{3.747071in}{2.015896in}}%
\pgfpathlineto{\pgfqpoint{3.747071in}{2.015896in}}%
\pgfpathlineto{\pgfqpoint{3.747071in}{2.021795in}}%
\pgfpathlineto{\pgfqpoint{3.756153in}{2.021795in}}%
\pgfpathlineto{\pgfqpoint{3.756153in}{2.015896in}}%
\pgfpathmoveto{\pgfqpoint{3.756153in}{2.015896in}}%
\pgfpathlineto{\pgfqpoint{3.756153in}{2.015896in}}%
\pgfpathlineto{\pgfqpoint{3.756153in}{2.021795in}}%
\pgfpathlineto{\pgfqpoint{3.765234in}{2.021795in}}%
\pgfpathlineto{\pgfqpoint{3.765234in}{2.015896in}}%
\pgfpathmoveto{\pgfqpoint{3.765234in}{2.015896in}}%
\pgfpathlineto{\pgfqpoint{3.765234in}{2.015896in}}%
\pgfpathlineto{\pgfqpoint{3.765234in}{2.021795in}}%
\pgfpathlineto{\pgfqpoint{3.774316in}{2.021795in}}%
\pgfpathlineto{\pgfqpoint{3.774316in}{2.015896in}}%
\pgfpathmoveto{\pgfqpoint{3.774316in}{2.015896in}}%
\pgfpathlineto{\pgfqpoint{3.774316in}{2.015896in}}%
\pgfpathlineto{\pgfqpoint{3.774316in}{2.021795in}}%
\pgfpathlineto{\pgfqpoint{3.783398in}{2.021795in}}%
\pgfpathlineto{\pgfqpoint{3.783398in}{2.015896in}}%
\pgfpathmoveto{\pgfqpoint{3.783398in}{2.015896in}}%
\pgfpathlineto{\pgfqpoint{3.783398in}{2.015896in}}%
\pgfpathlineto{\pgfqpoint{3.783398in}{2.021795in}}%
\pgfpathlineto{\pgfqpoint{3.792480in}{2.021795in}}%
\pgfpathlineto{\pgfqpoint{3.792480in}{2.015896in}}%
\pgfpathmoveto{\pgfqpoint{3.792480in}{2.015896in}}%
\pgfpathlineto{\pgfqpoint{3.792480in}{2.015896in}}%
\pgfpathlineto{\pgfqpoint{3.792480in}{2.021795in}}%
\pgfpathlineto{\pgfqpoint{3.801561in}{2.021795in}}%
\pgfpathlineto{\pgfqpoint{3.801561in}{2.015896in}}%
\pgfpathmoveto{\pgfqpoint{3.728907in}{2.470076in}}%
\pgfpathlineto{\pgfqpoint{3.728907in}{2.470076in}}%
\pgfpathlineto{\pgfqpoint{3.728907in}{2.475974in}}%
\pgfpathlineto{\pgfqpoint{3.737989in}{2.475974in}}%
\pgfpathlineto{\pgfqpoint{3.737989in}{2.470076in}}%
\pgfpathmoveto{\pgfqpoint{3.728907in}{2.475974in}}%
\pgfpathlineto{\pgfqpoint{3.728907in}{2.475974in}}%
\pgfpathlineto{\pgfqpoint{3.728907in}{2.481873in}}%
\pgfpathlineto{\pgfqpoint{3.737989in}{2.481873in}}%
\pgfpathlineto{\pgfqpoint{3.737989in}{2.475974in}}%
\pgfpathmoveto{\pgfqpoint{3.737989in}{2.470076in}}%
\pgfpathlineto{\pgfqpoint{3.737989in}{2.470076in}}%
\pgfpathlineto{\pgfqpoint{3.737989in}{2.475974in}}%
\pgfpathlineto{\pgfqpoint{3.747071in}{2.475974in}}%
\pgfpathlineto{\pgfqpoint{3.747071in}{2.470076in}}%
\pgfpathmoveto{\pgfqpoint{3.747071in}{2.458280in}}%
\pgfpathlineto{\pgfqpoint{3.747071in}{2.458280in}}%
\pgfpathlineto{\pgfqpoint{3.747071in}{2.464178in}}%
\pgfpathlineto{\pgfqpoint{3.756153in}{2.464178in}}%
\pgfpathlineto{\pgfqpoint{3.756153in}{2.458280in}}%
\pgfpathmoveto{\pgfqpoint{3.747071in}{2.464178in}}%
\pgfpathlineto{\pgfqpoint{3.747071in}{2.464178in}}%
\pgfpathlineto{\pgfqpoint{3.747071in}{2.470076in}}%
\pgfpathlineto{\pgfqpoint{3.756153in}{2.470076in}}%
\pgfpathlineto{\pgfqpoint{3.756153in}{2.464178in}}%
\pgfpathmoveto{\pgfqpoint{3.756153in}{2.458280in}}%
\pgfpathlineto{\pgfqpoint{3.756153in}{2.458280in}}%
\pgfpathlineto{\pgfqpoint{3.756153in}{2.464178in}}%
\pgfpathlineto{\pgfqpoint{3.765234in}{2.464178in}}%
\pgfpathlineto{\pgfqpoint{3.765234in}{2.458280in}}%
\pgfpathmoveto{\pgfqpoint{3.765234in}{2.446483in}}%
\pgfpathlineto{\pgfqpoint{3.765234in}{2.446483in}}%
\pgfpathlineto{\pgfqpoint{3.765234in}{2.452382in}}%
\pgfpathlineto{\pgfqpoint{3.774316in}{2.452382in}}%
\pgfpathlineto{\pgfqpoint{3.774316in}{2.446483in}}%
\pgfpathmoveto{\pgfqpoint{3.765234in}{2.452382in}}%
\pgfpathlineto{\pgfqpoint{3.765234in}{2.452382in}}%
\pgfpathlineto{\pgfqpoint{3.765234in}{2.458280in}}%
\pgfpathlineto{\pgfqpoint{3.774316in}{2.458280in}}%
\pgfpathlineto{\pgfqpoint{3.774316in}{2.452382in}}%
\pgfpathmoveto{\pgfqpoint{3.774316in}{2.446483in}}%
\pgfpathlineto{\pgfqpoint{3.774316in}{2.446483in}}%
\pgfpathlineto{\pgfqpoint{3.774316in}{2.452382in}}%
\pgfpathlineto{\pgfqpoint{3.783398in}{2.452382in}}%
\pgfpathlineto{\pgfqpoint{3.783398in}{2.446483in}}%
\pgfpathmoveto{\pgfqpoint{3.783398in}{2.434687in}}%
\pgfpathlineto{\pgfqpoint{3.783398in}{2.434687in}}%
\pgfpathlineto{\pgfqpoint{3.783398in}{2.440585in}}%
\pgfpathlineto{\pgfqpoint{3.792480in}{2.440585in}}%
\pgfpathlineto{\pgfqpoint{3.792480in}{2.434687in}}%
\pgfpathmoveto{\pgfqpoint{3.783398in}{2.440585in}}%
\pgfpathlineto{\pgfqpoint{3.783398in}{2.440585in}}%
\pgfpathlineto{\pgfqpoint{3.783398in}{2.446483in}}%
\pgfpathlineto{\pgfqpoint{3.792480in}{2.446483in}}%
\pgfpathlineto{\pgfqpoint{3.792480in}{2.440585in}}%
\pgfpathmoveto{\pgfqpoint{3.792480in}{2.434687in}}%
\pgfpathlineto{\pgfqpoint{3.792480in}{2.434687in}}%
\pgfpathlineto{\pgfqpoint{3.792480in}{2.440585in}}%
\pgfpathlineto{\pgfqpoint{3.801561in}{2.440585in}}%
\pgfpathlineto{\pgfqpoint{3.801561in}{2.434687in}}%
\pgfpathmoveto{\pgfqpoint{3.656253in}{2.517264in}}%
\pgfpathlineto{\pgfqpoint{3.656253in}{2.517264in}}%
\pgfpathlineto{\pgfqpoint{3.656253in}{2.523162in}}%
\pgfpathlineto{\pgfqpoint{3.665335in}{2.523162in}}%
\pgfpathlineto{\pgfqpoint{3.665335in}{2.517264in}}%
\pgfpathmoveto{\pgfqpoint{3.656253in}{2.523162in}}%
\pgfpathlineto{\pgfqpoint{3.656253in}{2.523162in}}%
\pgfpathlineto{\pgfqpoint{3.656253in}{2.529061in}}%
\pgfpathlineto{\pgfqpoint{3.665335in}{2.529061in}}%
\pgfpathlineto{\pgfqpoint{3.665335in}{2.523162in}}%
\pgfpathmoveto{\pgfqpoint{3.665335in}{2.517264in}}%
\pgfpathlineto{\pgfqpoint{3.665335in}{2.517264in}}%
\pgfpathlineto{\pgfqpoint{3.665335in}{2.523162in}}%
\pgfpathlineto{\pgfqpoint{3.674417in}{2.523162in}}%
\pgfpathlineto{\pgfqpoint{3.674417in}{2.517264in}}%
\pgfpathmoveto{\pgfqpoint{3.674417in}{2.505467in}}%
\pgfpathlineto{\pgfqpoint{3.674417in}{2.505467in}}%
\pgfpathlineto{\pgfqpoint{3.674417in}{2.511365in}}%
\pgfpathlineto{\pgfqpoint{3.683498in}{2.511365in}}%
\pgfpathlineto{\pgfqpoint{3.683498in}{2.505467in}}%
\pgfpathmoveto{\pgfqpoint{3.674417in}{2.511365in}}%
\pgfpathlineto{\pgfqpoint{3.674417in}{2.511365in}}%
\pgfpathlineto{\pgfqpoint{3.674417in}{2.517264in}}%
\pgfpathlineto{\pgfqpoint{3.683498in}{2.517264in}}%
\pgfpathlineto{\pgfqpoint{3.683498in}{2.511365in}}%
\pgfpathmoveto{\pgfqpoint{3.683498in}{2.505467in}}%
\pgfpathlineto{\pgfqpoint{3.683498in}{2.505467in}}%
\pgfpathlineto{\pgfqpoint{3.683498in}{2.511365in}}%
\pgfpathlineto{\pgfqpoint{3.692580in}{2.511365in}}%
\pgfpathlineto{\pgfqpoint{3.692580in}{2.505467in}}%
\pgfpathmoveto{\pgfqpoint{3.692580in}{2.493670in}}%
\pgfpathlineto{\pgfqpoint{3.692580in}{2.493670in}}%
\pgfpathlineto{\pgfqpoint{3.692580in}{2.499568in}}%
\pgfpathlineto{\pgfqpoint{3.701662in}{2.499568in}}%
\pgfpathlineto{\pgfqpoint{3.701662in}{2.493670in}}%
\pgfpathmoveto{\pgfqpoint{3.692580in}{2.499568in}}%
\pgfpathlineto{\pgfqpoint{3.692580in}{2.499568in}}%
\pgfpathlineto{\pgfqpoint{3.692580in}{2.505467in}}%
\pgfpathlineto{\pgfqpoint{3.701662in}{2.505467in}}%
\pgfpathlineto{\pgfqpoint{3.701662in}{2.499568in}}%
\pgfpathmoveto{\pgfqpoint{3.701662in}{2.493670in}}%
\pgfpathlineto{\pgfqpoint{3.701662in}{2.493670in}}%
\pgfpathlineto{\pgfqpoint{3.701662in}{2.499568in}}%
\pgfpathlineto{\pgfqpoint{3.710744in}{2.499568in}}%
\pgfpathlineto{\pgfqpoint{3.710744in}{2.493670in}}%
\pgfpathmoveto{\pgfqpoint{3.710744in}{2.481873in}}%
\pgfpathlineto{\pgfqpoint{3.710744in}{2.481873in}}%
\pgfpathlineto{\pgfqpoint{3.710744in}{2.487771in}}%
\pgfpathlineto{\pgfqpoint{3.719825in}{2.487771in}}%
\pgfpathlineto{\pgfqpoint{3.719825in}{2.481873in}}%
\pgfpathmoveto{\pgfqpoint{3.710744in}{2.487771in}}%
\pgfpathlineto{\pgfqpoint{3.710744in}{2.487771in}}%
\pgfpathlineto{\pgfqpoint{3.710744in}{2.493670in}}%
\pgfpathlineto{\pgfqpoint{3.719825in}{2.493670in}}%
\pgfpathlineto{\pgfqpoint{3.719825in}{2.487771in}}%
\pgfpathmoveto{\pgfqpoint{3.719825in}{2.481873in}}%
\pgfpathlineto{\pgfqpoint{3.719825in}{2.481873in}}%
\pgfpathlineto{\pgfqpoint{3.719825in}{2.487771in}}%
\pgfpathlineto{\pgfqpoint{3.728907in}{2.487771in}}%
\pgfpathlineto{\pgfqpoint{3.728907in}{2.481873in}}%
\pgfpathmoveto{\pgfqpoint{3.801561in}{2.015896in}}%
\pgfpathlineto{\pgfqpoint{3.801561in}{2.015896in}}%
\pgfpathlineto{\pgfqpoint{3.801561in}{2.021795in}}%
\pgfpathlineto{\pgfqpoint{3.810643in}{2.021795in}}%
\pgfpathlineto{\pgfqpoint{3.810643in}{2.015896in}}%
\pgfpathmoveto{\pgfqpoint{3.810643in}{2.015896in}}%
\pgfpathlineto{\pgfqpoint{3.810643in}{2.015896in}}%
\pgfpathlineto{\pgfqpoint{3.810643in}{2.021795in}}%
\pgfpathlineto{\pgfqpoint{3.819725in}{2.021795in}}%
\pgfpathlineto{\pgfqpoint{3.819725in}{2.015896in}}%
\pgfpathmoveto{\pgfqpoint{3.819725in}{2.015896in}}%
\pgfpathlineto{\pgfqpoint{3.819725in}{2.015896in}}%
\pgfpathlineto{\pgfqpoint{3.819725in}{2.021795in}}%
\pgfpathlineto{\pgfqpoint{3.828808in}{2.021795in}}%
\pgfpathlineto{\pgfqpoint{3.828808in}{2.015896in}}%
\pgfpathmoveto{\pgfqpoint{3.828808in}{2.015896in}}%
\pgfpathlineto{\pgfqpoint{3.828808in}{2.015896in}}%
\pgfpathlineto{\pgfqpoint{3.828808in}{2.021795in}}%
\pgfpathlineto{\pgfqpoint{3.837890in}{2.021795in}}%
\pgfpathlineto{\pgfqpoint{3.837890in}{2.015896in}}%
\pgfpathmoveto{\pgfqpoint{3.837890in}{2.015896in}}%
\pgfpathlineto{\pgfqpoint{3.837890in}{2.015896in}}%
\pgfpathlineto{\pgfqpoint{3.837890in}{2.021795in}}%
\pgfpathlineto{\pgfqpoint{3.846972in}{2.021795in}}%
\pgfpathlineto{\pgfqpoint{3.846972in}{2.015896in}}%
\pgfpathmoveto{\pgfqpoint{3.846972in}{2.015896in}}%
\pgfpathlineto{\pgfqpoint{3.846972in}{2.015896in}}%
\pgfpathlineto{\pgfqpoint{3.846972in}{2.021795in}}%
\pgfpathlineto{\pgfqpoint{3.856054in}{2.021795in}}%
\pgfpathlineto{\pgfqpoint{3.856054in}{2.015896in}}%
\pgfpathmoveto{\pgfqpoint{3.856054in}{2.015896in}}%
\pgfpathlineto{\pgfqpoint{3.856054in}{2.015896in}}%
\pgfpathlineto{\pgfqpoint{3.856054in}{2.021795in}}%
\pgfpathlineto{\pgfqpoint{3.865136in}{2.021795in}}%
\pgfpathlineto{\pgfqpoint{3.865136in}{2.015896in}}%
\pgfpathmoveto{\pgfqpoint{3.865136in}{2.015896in}}%
\pgfpathlineto{\pgfqpoint{3.865136in}{2.015896in}}%
\pgfpathlineto{\pgfqpoint{3.865136in}{2.021795in}}%
\pgfpathlineto{\pgfqpoint{3.874218in}{2.021795in}}%
\pgfpathlineto{\pgfqpoint{3.874218in}{2.015896in}}%
\pgfpathmoveto{\pgfqpoint{3.874218in}{2.015896in}}%
\pgfpathlineto{\pgfqpoint{3.874218in}{2.015896in}}%
\pgfpathlineto{\pgfqpoint{3.874218in}{2.021795in}}%
\pgfpathlineto{\pgfqpoint{3.883300in}{2.021795in}}%
\pgfpathlineto{\pgfqpoint{3.883300in}{2.015896in}}%
\pgfpathmoveto{\pgfqpoint{3.883300in}{2.015896in}}%
\pgfpathlineto{\pgfqpoint{3.883300in}{2.015896in}}%
\pgfpathlineto{\pgfqpoint{3.883300in}{2.021795in}}%
\pgfpathlineto{\pgfqpoint{3.892382in}{2.021795in}}%
\pgfpathlineto{\pgfqpoint{3.892382in}{2.015896in}}%
\pgfpathmoveto{\pgfqpoint{3.892382in}{2.015896in}}%
\pgfpathlineto{\pgfqpoint{3.892382in}{2.015896in}}%
\pgfpathlineto{\pgfqpoint{3.892382in}{2.021795in}}%
\pgfpathlineto{\pgfqpoint{3.901464in}{2.021795in}}%
\pgfpathlineto{\pgfqpoint{3.901464in}{2.015896in}}%
\pgfpathmoveto{\pgfqpoint{3.901464in}{2.015896in}}%
\pgfpathlineto{\pgfqpoint{3.901464in}{2.015896in}}%
\pgfpathlineto{\pgfqpoint{3.901464in}{2.021795in}}%
\pgfpathlineto{\pgfqpoint{3.910546in}{2.021795in}}%
\pgfpathlineto{\pgfqpoint{3.910546in}{2.015896in}}%
\pgfpathmoveto{\pgfqpoint{3.910546in}{2.015896in}}%
\pgfpathlineto{\pgfqpoint{3.910546in}{2.015896in}}%
\pgfpathlineto{\pgfqpoint{3.910546in}{2.021795in}}%
\pgfpathlineto{\pgfqpoint{3.919628in}{2.021795in}}%
\pgfpathlineto{\pgfqpoint{3.919628in}{2.015896in}}%
\pgfpathmoveto{\pgfqpoint{3.919628in}{2.015896in}}%
\pgfpathlineto{\pgfqpoint{3.919628in}{2.015896in}}%
\pgfpathlineto{\pgfqpoint{3.919628in}{2.021795in}}%
\pgfpathlineto{\pgfqpoint{3.928710in}{2.021795in}}%
\pgfpathlineto{\pgfqpoint{3.928710in}{2.015896in}}%
\pgfpathmoveto{\pgfqpoint{3.928710in}{2.015896in}}%
\pgfpathlineto{\pgfqpoint{3.928710in}{2.015896in}}%
\pgfpathlineto{\pgfqpoint{3.928710in}{2.021795in}}%
\pgfpathlineto{\pgfqpoint{3.937792in}{2.021795in}}%
\pgfpathlineto{\pgfqpoint{3.937792in}{2.015896in}}%
\pgfpathmoveto{\pgfqpoint{3.937792in}{2.015896in}}%
\pgfpathlineto{\pgfqpoint{3.937792in}{2.015896in}}%
\pgfpathlineto{\pgfqpoint{3.937792in}{2.021795in}}%
\pgfpathlineto{\pgfqpoint{3.946874in}{2.021795in}}%
\pgfpathlineto{\pgfqpoint{3.946874in}{2.015896in}}%
\pgfpathmoveto{\pgfqpoint{3.874218in}{2.375705in}}%
\pgfpathlineto{\pgfqpoint{3.874218in}{2.375705in}}%
\pgfpathlineto{\pgfqpoint{3.874218in}{2.381603in}}%
\pgfpathlineto{\pgfqpoint{3.883300in}{2.381603in}}%
\pgfpathlineto{\pgfqpoint{3.883300in}{2.375705in}}%
\pgfpathmoveto{\pgfqpoint{3.874218in}{2.381603in}}%
\pgfpathlineto{\pgfqpoint{3.874218in}{2.381603in}}%
\pgfpathlineto{\pgfqpoint{3.874218in}{2.387502in}}%
\pgfpathlineto{\pgfqpoint{3.883300in}{2.387502in}}%
\pgfpathlineto{\pgfqpoint{3.883300in}{2.381603in}}%
\pgfpathmoveto{\pgfqpoint{3.883300in}{2.375705in}}%
\pgfpathlineto{\pgfqpoint{3.883300in}{2.375705in}}%
\pgfpathlineto{\pgfqpoint{3.883300in}{2.381603in}}%
\pgfpathlineto{\pgfqpoint{3.892382in}{2.381603in}}%
\pgfpathlineto{\pgfqpoint{3.892382in}{2.375705in}}%
\pgfpathmoveto{\pgfqpoint{3.892382in}{2.363907in}}%
\pgfpathlineto{\pgfqpoint{3.892382in}{2.363907in}}%
\pgfpathlineto{\pgfqpoint{3.892382in}{2.369806in}}%
\pgfpathlineto{\pgfqpoint{3.901464in}{2.369806in}}%
\pgfpathlineto{\pgfqpoint{3.901464in}{2.363907in}}%
\pgfpathmoveto{\pgfqpoint{3.892382in}{2.369806in}}%
\pgfpathlineto{\pgfqpoint{3.892382in}{2.369806in}}%
\pgfpathlineto{\pgfqpoint{3.892382in}{2.375705in}}%
\pgfpathlineto{\pgfqpoint{3.901464in}{2.375705in}}%
\pgfpathlineto{\pgfqpoint{3.901464in}{2.369806in}}%
\pgfpathmoveto{\pgfqpoint{3.901464in}{2.363907in}}%
\pgfpathlineto{\pgfqpoint{3.901464in}{2.363907in}}%
\pgfpathlineto{\pgfqpoint{3.901464in}{2.369806in}}%
\pgfpathlineto{\pgfqpoint{3.910546in}{2.369806in}}%
\pgfpathlineto{\pgfqpoint{3.910546in}{2.363907in}}%
\pgfpathmoveto{\pgfqpoint{3.910546in}{2.352110in}}%
\pgfpathlineto{\pgfqpoint{3.910546in}{2.352110in}}%
\pgfpathlineto{\pgfqpoint{3.910546in}{2.358009in}}%
\pgfpathlineto{\pgfqpoint{3.919628in}{2.358009in}}%
\pgfpathlineto{\pgfqpoint{3.919628in}{2.352110in}}%
\pgfpathmoveto{\pgfqpoint{3.910546in}{2.358009in}}%
\pgfpathlineto{\pgfqpoint{3.910546in}{2.358009in}}%
\pgfpathlineto{\pgfqpoint{3.910546in}{2.363907in}}%
\pgfpathlineto{\pgfqpoint{3.919628in}{2.363907in}}%
\pgfpathlineto{\pgfqpoint{3.919628in}{2.358009in}}%
\pgfpathmoveto{\pgfqpoint{3.919628in}{2.352110in}}%
\pgfpathlineto{\pgfqpoint{3.919628in}{2.352110in}}%
\pgfpathlineto{\pgfqpoint{3.919628in}{2.358009in}}%
\pgfpathlineto{\pgfqpoint{3.928710in}{2.358009in}}%
\pgfpathlineto{\pgfqpoint{3.928710in}{2.352110in}}%
\pgfpathmoveto{\pgfqpoint{3.928710in}{2.340313in}}%
\pgfpathlineto{\pgfqpoint{3.928710in}{2.340313in}}%
\pgfpathlineto{\pgfqpoint{3.928710in}{2.346212in}}%
\pgfpathlineto{\pgfqpoint{3.937792in}{2.346212in}}%
\pgfpathlineto{\pgfqpoint{3.937792in}{2.340313in}}%
\pgfpathmoveto{\pgfqpoint{3.928710in}{2.346212in}}%
\pgfpathlineto{\pgfqpoint{3.928710in}{2.346212in}}%
\pgfpathlineto{\pgfqpoint{3.928710in}{2.352110in}}%
\pgfpathlineto{\pgfqpoint{3.937792in}{2.352110in}}%
\pgfpathlineto{\pgfqpoint{3.937792in}{2.346212in}}%
\pgfpathmoveto{\pgfqpoint{3.937792in}{2.340313in}}%
\pgfpathlineto{\pgfqpoint{3.937792in}{2.340313in}}%
\pgfpathlineto{\pgfqpoint{3.937792in}{2.346212in}}%
\pgfpathlineto{\pgfqpoint{3.946874in}{2.346212in}}%
\pgfpathlineto{\pgfqpoint{3.946874in}{2.340313in}}%
\pgfpathmoveto{\pgfqpoint{3.801561in}{2.422891in}}%
\pgfpathlineto{\pgfqpoint{3.801561in}{2.422891in}}%
\pgfpathlineto{\pgfqpoint{3.801561in}{2.428789in}}%
\pgfpathlineto{\pgfqpoint{3.810643in}{2.428789in}}%
\pgfpathlineto{\pgfqpoint{3.810643in}{2.422891in}}%
\pgfpathmoveto{\pgfqpoint{3.801561in}{2.428789in}}%
\pgfpathlineto{\pgfqpoint{3.801561in}{2.428789in}}%
\pgfpathlineto{\pgfqpoint{3.801561in}{2.434687in}}%
\pgfpathlineto{\pgfqpoint{3.810643in}{2.434687in}}%
\pgfpathlineto{\pgfqpoint{3.810643in}{2.428789in}}%
\pgfpathmoveto{\pgfqpoint{3.810643in}{2.422891in}}%
\pgfpathlineto{\pgfqpoint{3.810643in}{2.422891in}}%
\pgfpathlineto{\pgfqpoint{3.810643in}{2.428789in}}%
\pgfpathlineto{\pgfqpoint{3.819725in}{2.428789in}}%
\pgfpathlineto{\pgfqpoint{3.819725in}{2.422891in}}%
\pgfpathmoveto{\pgfqpoint{3.819725in}{2.411094in}}%
\pgfpathlineto{\pgfqpoint{3.819725in}{2.411094in}}%
\pgfpathlineto{\pgfqpoint{3.819725in}{2.416993in}}%
\pgfpathlineto{\pgfqpoint{3.828808in}{2.416993in}}%
\pgfpathlineto{\pgfqpoint{3.828808in}{2.411094in}}%
\pgfpathmoveto{\pgfqpoint{3.819725in}{2.416993in}}%
\pgfpathlineto{\pgfqpoint{3.819725in}{2.416993in}}%
\pgfpathlineto{\pgfqpoint{3.819725in}{2.422891in}}%
\pgfpathlineto{\pgfqpoint{3.828808in}{2.422891in}}%
\pgfpathlineto{\pgfqpoint{3.828808in}{2.416993in}}%
\pgfpathmoveto{\pgfqpoint{3.828808in}{2.411094in}}%
\pgfpathlineto{\pgfqpoint{3.828808in}{2.411094in}}%
\pgfpathlineto{\pgfqpoint{3.828808in}{2.416993in}}%
\pgfpathlineto{\pgfqpoint{3.837890in}{2.416993in}}%
\pgfpathlineto{\pgfqpoint{3.837890in}{2.411094in}}%
\pgfpathmoveto{\pgfqpoint{3.837890in}{2.399298in}}%
\pgfpathlineto{\pgfqpoint{3.837890in}{2.399298in}}%
\pgfpathlineto{\pgfqpoint{3.837890in}{2.405196in}}%
\pgfpathlineto{\pgfqpoint{3.846972in}{2.405196in}}%
\pgfpathlineto{\pgfqpoint{3.846972in}{2.399298in}}%
\pgfpathmoveto{\pgfqpoint{3.837890in}{2.405196in}}%
\pgfpathlineto{\pgfqpoint{3.837890in}{2.405196in}}%
\pgfpathlineto{\pgfqpoint{3.837890in}{2.411094in}}%
\pgfpathlineto{\pgfqpoint{3.846972in}{2.411094in}}%
\pgfpathlineto{\pgfqpoint{3.846972in}{2.405196in}}%
\pgfpathmoveto{\pgfqpoint{3.846972in}{2.399298in}}%
\pgfpathlineto{\pgfqpoint{3.846972in}{2.399298in}}%
\pgfpathlineto{\pgfqpoint{3.846972in}{2.405196in}}%
\pgfpathlineto{\pgfqpoint{3.856054in}{2.405196in}}%
\pgfpathlineto{\pgfqpoint{3.856054in}{2.399298in}}%
\pgfpathmoveto{\pgfqpoint{3.856054in}{2.387502in}}%
\pgfpathlineto{\pgfqpoint{3.856054in}{2.387502in}}%
\pgfpathlineto{\pgfqpoint{3.856054in}{2.393400in}}%
\pgfpathlineto{\pgfqpoint{3.865136in}{2.393400in}}%
\pgfpathlineto{\pgfqpoint{3.865136in}{2.387502in}}%
\pgfpathmoveto{\pgfqpoint{3.856054in}{2.393400in}}%
\pgfpathlineto{\pgfqpoint{3.856054in}{2.393400in}}%
\pgfpathlineto{\pgfqpoint{3.856054in}{2.399298in}}%
\pgfpathlineto{\pgfqpoint{3.865136in}{2.399298in}}%
\pgfpathlineto{\pgfqpoint{3.865136in}{2.393400in}}%
\pgfpathmoveto{\pgfqpoint{3.865136in}{2.387502in}}%
\pgfpathlineto{\pgfqpoint{3.865136in}{2.387502in}}%
\pgfpathlineto{\pgfqpoint{3.865136in}{2.393400in}}%
\pgfpathlineto{\pgfqpoint{3.874218in}{2.393400in}}%
\pgfpathlineto{\pgfqpoint{3.874218in}{2.387502in}}%
\pgfpathmoveto{\pgfqpoint{3.946874in}{2.015896in}}%
\pgfpathlineto{\pgfqpoint{3.946874in}{2.015896in}}%
\pgfpathlineto{\pgfqpoint{3.946874in}{2.021795in}}%
\pgfpathlineto{\pgfqpoint{3.955956in}{2.021795in}}%
\pgfpathlineto{\pgfqpoint{3.955956in}{2.015896in}}%
\pgfpathmoveto{\pgfqpoint{3.955956in}{2.015896in}}%
\pgfpathlineto{\pgfqpoint{3.955956in}{2.015896in}}%
\pgfpathlineto{\pgfqpoint{3.955956in}{2.021795in}}%
\pgfpathlineto{\pgfqpoint{3.965038in}{2.021795in}}%
\pgfpathlineto{\pgfqpoint{3.965038in}{2.015896in}}%
\pgfpathmoveto{\pgfqpoint{3.965038in}{2.015896in}}%
\pgfpathlineto{\pgfqpoint{3.965038in}{2.015896in}}%
\pgfpathlineto{\pgfqpoint{3.965038in}{2.021795in}}%
\pgfpathlineto{\pgfqpoint{3.974120in}{2.021795in}}%
\pgfpathlineto{\pgfqpoint{3.974120in}{2.015896in}}%
\pgfpathmoveto{\pgfqpoint{3.974120in}{2.015896in}}%
\pgfpathlineto{\pgfqpoint{3.974120in}{2.015896in}}%
\pgfpathlineto{\pgfqpoint{3.974120in}{2.021795in}}%
\pgfpathlineto{\pgfqpoint{3.983202in}{2.021795in}}%
\pgfpathlineto{\pgfqpoint{3.983202in}{2.015896in}}%
\pgfpathmoveto{\pgfqpoint{3.983202in}{2.015896in}}%
\pgfpathlineto{\pgfqpoint{3.983202in}{2.015896in}}%
\pgfpathlineto{\pgfqpoint{3.983202in}{2.021795in}}%
\pgfpathlineto{\pgfqpoint{3.992284in}{2.021795in}}%
\pgfpathlineto{\pgfqpoint{3.992284in}{2.015896in}}%
\pgfpathmoveto{\pgfqpoint{3.992284in}{2.015896in}}%
\pgfpathlineto{\pgfqpoint{3.992284in}{2.015896in}}%
\pgfpathlineto{\pgfqpoint{3.992284in}{2.021795in}}%
\pgfpathlineto{\pgfqpoint{4.001365in}{2.021795in}}%
\pgfpathlineto{\pgfqpoint{4.001365in}{2.015896in}}%
\pgfpathmoveto{\pgfqpoint{4.001365in}{2.015896in}}%
\pgfpathlineto{\pgfqpoint{4.001365in}{2.015896in}}%
\pgfpathlineto{\pgfqpoint{4.001365in}{2.021795in}}%
\pgfpathlineto{\pgfqpoint{4.010447in}{2.021795in}}%
\pgfpathlineto{\pgfqpoint{4.010447in}{2.015896in}}%
\pgfpathmoveto{\pgfqpoint{4.010447in}{2.015896in}}%
\pgfpathlineto{\pgfqpoint{4.010447in}{2.015896in}}%
\pgfpathlineto{\pgfqpoint{4.010447in}{2.021795in}}%
\pgfpathlineto{\pgfqpoint{4.019529in}{2.021795in}}%
\pgfpathlineto{\pgfqpoint{4.019529in}{2.015896in}}%
\pgfpathmoveto{\pgfqpoint{4.019529in}{2.015896in}}%
\pgfpathlineto{\pgfqpoint{4.019529in}{2.015896in}}%
\pgfpathlineto{\pgfqpoint{4.019529in}{2.021795in}}%
\pgfpathlineto{\pgfqpoint{4.028611in}{2.021795in}}%
\pgfpathlineto{\pgfqpoint{4.028611in}{2.015896in}}%
\pgfpathmoveto{\pgfqpoint{4.028611in}{2.015896in}}%
\pgfpathlineto{\pgfqpoint{4.028611in}{2.015896in}}%
\pgfpathlineto{\pgfqpoint{4.028611in}{2.021795in}}%
\pgfpathlineto{\pgfqpoint{4.037693in}{2.021795in}}%
\pgfpathlineto{\pgfqpoint{4.037693in}{2.015896in}}%
\pgfpathmoveto{\pgfqpoint{4.037693in}{2.015896in}}%
\pgfpathlineto{\pgfqpoint{4.037693in}{2.015896in}}%
\pgfpathlineto{\pgfqpoint{4.037693in}{2.021795in}}%
\pgfpathlineto{\pgfqpoint{4.046775in}{2.021795in}}%
\pgfpathlineto{\pgfqpoint{4.046775in}{2.015896in}}%
\pgfpathmoveto{\pgfqpoint{4.046775in}{2.015896in}}%
\pgfpathlineto{\pgfqpoint{4.046775in}{2.015896in}}%
\pgfpathlineto{\pgfqpoint{4.046775in}{2.021795in}}%
\pgfpathlineto{\pgfqpoint{4.055857in}{2.021795in}}%
\pgfpathlineto{\pgfqpoint{4.055857in}{2.015896in}}%
\pgfpathmoveto{\pgfqpoint{4.055857in}{2.015896in}}%
\pgfpathlineto{\pgfqpoint{4.055857in}{2.015896in}}%
\pgfpathlineto{\pgfqpoint{4.055857in}{2.021795in}}%
\pgfpathlineto{\pgfqpoint{4.064939in}{2.021795in}}%
\pgfpathlineto{\pgfqpoint{4.064939in}{2.015896in}}%
\pgfpathmoveto{\pgfqpoint{4.064939in}{2.015896in}}%
\pgfpathlineto{\pgfqpoint{4.064939in}{2.015896in}}%
\pgfpathlineto{\pgfqpoint{4.064939in}{2.021795in}}%
\pgfpathlineto{\pgfqpoint{4.074021in}{2.021795in}}%
\pgfpathlineto{\pgfqpoint{4.074021in}{2.015896in}}%
\pgfpathmoveto{\pgfqpoint{4.074021in}{2.015896in}}%
\pgfpathlineto{\pgfqpoint{4.074021in}{2.015896in}}%
\pgfpathlineto{\pgfqpoint{4.074021in}{2.021795in}}%
\pgfpathlineto{\pgfqpoint{4.083103in}{2.021795in}}%
\pgfpathlineto{\pgfqpoint{4.083103in}{2.015896in}}%
\pgfpathmoveto{\pgfqpoint{4.083103in}{2.015896in}}%
\pgfpathlineto{\pgfqpoint{4.083103in}{2.015896in}}%
\pgfpathlineto{\pgfqpoint{4.083103in}{2.021795in}}%
\pgfpathlineto{\pgfqpoint{4.092185in}{2.021795in}}%
\pgfpathlineto{\pgfqpoint{4.092185in}{2.015896in}}%
\pgfpathmoveto{\pgfqpoint{4.019529in}{2.281328in}}%
\pgfpathlineto{\pgfqpoint{4.019529in}{2.281328in}}%
\pgfpathlineto{\pgfqpoint{4.019529in}{2.287226in}}%
\pgfpathlineto{\pgfqpoint{4.028611in}{2.287226in}}%
\pgfpathlineto{\pgfqpoint{4.028611in}{2.281328in}}%
\pgfpathmoveto{\pgfqpoint{4.019529in}{2.287226in}}%
\pgfpathlineto{\pgfqpoint{4.019529in}{2.287226in}}%
\pgfpathlineto{\pgfqpoint{4.019529in}{2.293125in}}%
\pgfpathlineto{\pgfqpoint{4.028611in}{2.293125in}}%
\pgfpathlineto{\pgfqpoint{4.028611in}{2.287226in}}%
\pgfpathmoveto{\pgfqpoint{4.028611in}{2.281328in}}%
\pgfpathlineto{\pgfqpoint{4.028611in}{2.281328in}}%
\pgfpathlineto{\pgfqpoint{4.028611in}{2.287226in}}%
\pgfpathlineto{\pgfqpoint{4.037693in}{2.287226in}}%
\pgfpathlineto{\pgfqpoint{4.037693in}{2.281328in}}%
\pgfpathmoveto{\pgfqpoint{4.037693in}{2.269531in}}%
\pgfpathlineto{\pgfqpoint{4.037693in}{2.269531in}}%
\pgfpathlineto{\pgfqpoint{4.037693in}{2.275429in}}%
\pgfpathlineto{\pgfqpoint{4.046775in}{2.275429in}}%
\pgfpathlineto{\pgfqpoint{4.046775in}{2.269531in}}%
\pgfpathmoveto{\pgfqpoint{4.037693in}{2.275429in}}%
\pgfpathlineto{\pgfqpoint{4.037693in}{2.275429in}}%
\pgfpathlineto{\pgfqpoint{4.037693in}{2.281328in}}%
\pgfpathlineto{\pgfqpoint{4.046775in}{2.281328in}}%
\pgfpathlineto{\pgfqpoint{4.046775in}{2.275429in}}%
\pgfpathmoveto{\pgfqpoint{4.046775in}{2.269531in}}%
\pgfpathlineto{\pgfqpoint{4.046775in}{2.269531in}}%
\pgfpathlineto{\pgfqpoint{4.046775in}{2.275429in}}%
\pgfpathlineto{\pgfqpoint{4.055857in}{2.275429in}}%
\pgfpathlineto{\pgfqpoint{4.055857in}{2.269531in}}%
\pgfpathmoveto{\pgfqpoint{4.055857in}{2.257733in}}%
\pgfpathlineto{\pgfqpoint{4.055857in}{2.257733in}}%
\pgfpathlineto{\pgfqpoint{4.055857in}{2.263632in}}%
\pgfpathlineto{\pgfqpoint{4.064939in}{2.263632in}}%
\pgfpathlineto{\pgfqpoint{4.064939in}{2.257733in}}%
\pgfpathmoveto{\pgfqpoint{4.055857in}{2.263632in}}%
\pgfpathlineto{\pgfqpoint{4.055857in}{2.263632in}}%
\pgfpathlineto{\pgfqpoint{4.055857in}{2.269531in}}%
\pgfpathlineto{\pgfqpoint{4.064939in}{2.269531in}}%
\pgfpathlineto{\pgfqpoint{4.064939in}{2.263632in}}%
\pgfpathmoveto{\pgfqpoint{4.064939in}{2.257733in}}%
\pgfpathlineto{\pgfqpoint{4.064939in}{2.257733in}}%
\pgfpathlineto{\pgfqpoint{4.064939in}{2.263632in}}%
\pgfpathlineto{\pgfqpoint{4.074021in}{2.263632in}}%
\pgfpathlineto{\pgfqpoint{4.074021in}{2.257733in}}%
\pgfpathmoveto{\pgfqpoint{4.074021in}{2.245936in}}%
\pgfpathlineto{\pgfqpoint{4.074021in}{2.245936in}}%
\pgfpathlineto{\pgfqpoint{4.074021in}{2.251835in}}%
\pgfpathlineto{\pgfqpoint{4.083103in}{2.251835in}}%
\pgfpathlineto{\pgfqpoint{4.083103in}{2.245936in}}%
\pgfpathmoveto{\pgfqpoint{4.074021in}{2.251835in}}%
\pgfpathlineto{\pgfqpoint{4.074021in}{2.251835in}}%
\pgfpathlineto{\pgfqpoint{4.074021in}{2.257733in}}%
\pgfpathlineto{\pgfqpoint{4.083103in}{2.257733in}}%
\pgfpathlineto{\pgfqpoint{4.083103in}{2.251835in}}%
\pgfpathmoveto{\pgfqpoint{4.083103in}{2.245936in}}%
\pgfpathlineto{\pgfqpoint{4.083103in}{2.245936in}}%
\pgfpathlineto{\pgfqpoint{4.083103in}{2.251835in}}%
\pgfpathlineto{\pgfqpoint{4.092185in}{2.251835in}}%
\pgfpathlineto{\pgfqpoint{4.092185in}{2.245936in}}%
\pgfpathmoveto{\pgfqpoint{3.946874in}{2.328516in}}%
\pgfpathlineto{\pgfqpoint{3.946874in}{2.328516in}}%
\pgfpathlineto{\pgfqpoint{3.946874in}{2.334415in}}%
\pgfpathlineto{\pgfqpoint{3.955956in}{2.334415in}}%
\pgfpathlineto{\pgfqpoint{3.955956in}{2.328516in}}%
\pgfpathmoveto{\pgfqpoint{3.946874in}{2.334415in}}%
\pgfpathlineto{\pgfqpoint{3.946874in}{2.334415in}}%
\pgfpathlineto{\pgfqpoint{3.946874in}{2.340313in}}%
\pgfpathlineto{\pgfqpoint{3.955956in}{2.340313in}}%
\pgfpathlineto{\pgfqpoint{3.955956in}{2.334415in}}%
\pgfpathmoveto{\pgfqpoint{3.955956in}{2.328516in}}%
\pgfpathlineto{\pgfqpoint{3.955956in}{2.328516in}}%
\pgfpathlineto{\pgfqpoint{3.955956in}{2.334415in}}%
\pgfpathlineto{\pgfqpoint{3.965038in}{2.334415in}}%
\pgfpathlineto{\pgfqpoint{3.965038in}{2.328516in}}%
\pgfpathmoveto{\pgfqpoint{3.965038in}{2.316719in}}%
\pgfpathlineto{\pgfqpoint{3.965038in}{2.316719in}}%
\pgfpathlineto{\pgfqpoint{3.965038in}{2.322618in}}%
\pgfpathlineto{\pgfqpoint{3.974120in}{2.322618in}}%
\pgfpathlineto{\pgfqpoint{3.974120in}{2.316719in}}%
\pgfpathmoveto{\pgfqpoint{3.965038in}{2.322618in}}%
\pgfpathlineto{\pgfqpoint{3.965038in}{2.322618in}}%
\pgfpathlineto{\pgfqpoint{3.965038in}{2.328516in}}%
\pgfpathlineto{\pgfqpoint{3.974120in}{2.328516in}}%
\pgfpathlineto{\pgfqpoint{3.974120in}{2.322618in}}%
\pgfpathmoveto{\pgfqpoint{3.974120in}{2.316719in}}%
\pgfpathlineto{\pgfqpoint{3.974120in}{2.316719in}}%
\pgfpathlineto{\pgfqpoint{3.974120in}{2.322618in}}%
\pgfpathlineto{\pgfqpoint{3.983202in}{2.322618in}}%
\pgfpathlineto{\pgfqpoint{3.983202in}{2.316719in}}%
\pgfpathmoveto{\pgfqpoint{3.983202in}{2.304922in}}%
\pgfpathlineto{\pgfqpoint{3.983202in}{2.304922in}}%
\pgfpathlineto{\pgfqpoint{3.983202in}{2.310820in}}%
\pgfpathlineto{\pgfqpoint{3.992284in}{2.310820in}}%
\pgfpathlineto{\pgfqpoint{3.992284in}{2.304922in}}%
\pgfpathmoveto{\pgfqpoint{3.983202in}{2.310820in}}%
\pgfpathlineto{\pgfqpoint{3.983202in}{2.310820in}}%
\pgfpathlineto{\pgfqpoint{3.983202in}{2.316719in}}%
\pgfpathlineto{\pgfqpoint{3.992284in}{2.316719in}}%
\pgfpathlineto{\pgfqpoint{3.992284in}{2.310820in}}%
\pgfpathmoveto{\pgfqpoint{3.992284in}{2.304922in}}%
\pgfpathlineto{\pgfqpoint{3.992284in}{2.304922in}}%
\pgfpathlineto{\pgfqpoint{3.992284in}{2.310820in}}%
\pgfpathlineto{\pgfqpoint{4.001365in}{2.310820in}}%
\pgfpathlineto{\pgfqpoint{4.001365in}{2.304922in}}%
\pgfpathmoveto{\pgfqpoint{4.001365in}{2.293125in}}%
\pgfpathlineto{\pgfqpoint{4.001365in}{2.293125in}}%
\pgfpathlineto{\pgfqpoint{4.001365in}{2.299023in}}%
\pgfpathlineto{\pgfqpoint{4.010447in}{2.299023in}}%
\pgfpathlineto{\pgfqpoint{4.010447in}{2.293125in}}%
\pgfpathmoveto{\pgfqpoint{4.001365in}{2.299023in}}%
\pgfpathlineto{\pgfqpoint{4.001365in}{2.299023in}}%
\pgfpathlineto{\pgfqpoint{4.001365in}{2.304922in}}%
\pgfpathlineto{\pgfqpoint{4.010447in}{2.304922in}}%
\pgfpathlineto{\pgfqpoint{4.010447in}{2.299023in}}%
\pgfpathmoveto{\pgfqpoint{4.010447in}{2.293125in}}%
\pgfpathlineto{\pgfqpoint{4.010447in}{2.293125in}}%
\pgfpathlineto{\pgfqpoint{4.010447in}{2.299023in}}%
\pgfpathlineto{\pgfqpoint{4.019529in}{2.299023in}}%
\pgfpathlineto{\pgfqpoint{4.019529in}{2.293125in}}%
\pgfpathmoveto{\pgfqpoint{4.092185in}{2.015896in}}%
\pgfpathlineto{\pgfqpoint{4.092185in}{2.015896in}}%
\pgfpathlineto{\pgfqpoint{4.092185in}{2.021795in}}%
\pgfpathlineto{\pgfqpoint{4.101267in}{2.021795in}}%
\pgfpathlineto{\pgfqpoint{4.101267in}{2.015896in}}%
\pgfpathmoveto{\pgfqpoint{4.101267in}{2.015896in}}%
\pgfpathlineto{\pgfqpoint{4.101267in}{2.015896in}}%
\pgfpathlineto{\pgfqpoint{4.101267in}{2.021795in}}%
\pgfpathlineto{\pgfqpoint{4.110350in}{2.021795in}}%
\pgfpathlineto{\pgfqpoint{4.110350in}{2.015896in}}%
\pgfpathmoveto{\pgfqpoint{4.110350in}{2.015896in}}%
\pgfpathlineto{\pgfqpoint{4.110350in}{2.015896in}}%
\pgfpathlineto{\pgfqpoint{4.110350in}{2.021795in}}%
\pgfpathlineto{\pgfqpoint{4.119432in}{2.021795in}}%
\pgfpathlineto{\pgfqpoint{4.119432in}{2.015896in}}%
\pgfpathmoveto{\pgfqpoint{4.119432in}{2.015896in}}%
\pgfpathlineto{\pgfqpoint{4.119432in}{2.015896in}}%
\pgfpathlineto{\pgfqpoint{4.119432in}{2.021795in}}%
\pgfpathlineto{\pgfqpoint{4.128515in}{2.021795in}}%
\pgfpathlineto{\pgfqpoint{4.128515in}{2.015896in}}%
\pgfpathmoveto{\pgfqpoint{4.128515in}{2.015896in}}%
\pgfpathlineto{\pgfqpoint{4.128515in}{2.015896in}}%
\pgfpathlineto{\pgfqpoint{4.128515in}{2.021795in}}%
\pgfpathlineto{\pgfqpoint{4.137597in}{2.021795in}}%
\pgfpathlineto{\pgfqpoint{4.137597in}{2.015896in}}%
\pgfpathmoveto{\pgfqpoint{4.137597in}{2.015896in}}%
\pgfpathlineto{\pgfqpoint{4.137597in}{2.015896in}}%
\pgfpathlineto{\pgfqpoint{4.137597in}{2.021795in}}%
\pgfpathlineto{\pgfqpoint{4.146680in}{2.021795in}}%
\pgfpathlineto{\pgfqpoint{4.146680in}{2.015896in}}%
\pgfpathmoveto{\pgfqpoint{4.146680in}{2.015896in}}%
\pgfpathlineto{\pgfqpoint{4.146680in}{2.015896in}}%
\pgfpathlineto{\pgfqpoint{4.146680in}{2.021795in}}%
\pgfpathlineto{\pgfqpoint{4.155762in}{2.021795in}}%
\pgfpathlineto{\pgfqpoint{4.155762in}{2.015896in}}%
\pgfpathmoveto{\pgfqpoint{4.155762in}{2.015896in}}%
\pgfpathlineto{\pgfqpoint{4.155762in}{2.015896in}}%
\pgfpathlineto{\pgfqpoint{4.155762in}{2.021795in}}%
\pgfpathlineto{\pgfqpoint{4.164844in}{2.021795in}}%
\pgfpathlineto{\pgfqpoint{4.164844in}{2.015896in}}%
\pgfpathmoveto{\pgfqpoint{4.164844in}{2.015896in}}%
\pgfpathlineto{\pgfqpoint{4.164844in}{2.015896in}}%
\pgfpathlineto{\pgfqpoint{4.164844in}{2.021795in}}%
\pgfpathlineto{\pgfqpoint{4.173927in}{2.021795in}}%
\pgfpathlineto{\pgfqpoint{4.173927in}{2.015896in}}%
\pgfpathmoveto{\pgfqpoint{4.173927in}{2.015896in}}%
\pgfpathlineto{\pgfqpoint{4.173927in}{2.015896in}}%
\pgfpathlineto{\pgfqpoint{4.173927in}{2.021795in}}%
\pgfpathlineto{\pgfqpoint{4.183009in}{2.021795in}}%
\pgfpathlineto{\pgfqpoint{4.183009in}{2.015896in}}%
\pgfpathmoveto{\pgfqpoint{4.183009in}{2.015896in}}%
\pgfpathlineto{\pgfqpoint{4.183009in}{2.015896in}}%
\pgfpathlineto{\pgfqpoint{4.183009in}{2.021795in}}%
\pgfpathlineto{\pgfqpoint{4.192092in}{2.021795in}}%
\pgfpathlineto{\pgfqpoint{4.192092in}{2.015896in}}%
\pgfpathmoveto{\pgfqpoint{4.192092in}{2.015896in}}%
\pgfpathlineto{\pgfqpoint{4.192092in}{2.015896in}}%
\pgfpathlineto{\pgfqpoint{4.192092in}{2.021795in}}%
\pgfpathlineto{\pgfqpoint{4.201174in}{2.021795in}}%
\pgfpathlineto{\pgfqpoint{4.201174in}{2.015896in}}%
\pgfpathmoveto{\pgfqpoint{4.201174in}{2.015896in}}%
\pgfpathlineto{\pgfqpoint{4.201174in}{2.015896in}}%
\pgfpathlineto{\pgfqpoint{4.201174in}{2.021795in}}%
\pgfpathlineto{\pgfqpoint{4.210257in}{2.021795in}}%
\pgfpathlineto{\pgfqpoint{4.210257in}{2.015896in}}%
\pgfpathmoveto{\pgfqpoint{4.210257in}{2.015896in}}%
\pgfpathlineto{\pgfqpoint{4.210257in}{2.015896in}}%
\pgfpathlineto{\pgfqpoint{4.210257in}{2.021795in}}%
\pgfpathlineto{\pgfqpoint{4.219339in}{2.021795in}}%
\pgfpathlineto{\pgfqpoint{4.219339in}{2.015896in}}%
\pgfpathmoveto{\pgfqpoint{4.219339in}{2.015896in}}%
\pgfpathlineto{\pgfqpoint{4.219339in}{2.015896in}}%
\pgfpathlineto{\pgfqpoint{4.219339in}{2.021795in}}%
\pgfpathlineto{\pgfqpoint{4.228422in}{2.021795in}}%
\pgfpathlineto{\pgfqpoint{4.228422in}{2.015896in}}%
\pgfpathmoveto{\pgfqpoint{4.228422in}{2.015896in}}%
\pgfpathlineto{\pgfqpoint{4.228422in}{2.015896in}}%
\pgfpathlineto{\pgfqpoint{4.228422in}{2.021795in}}%
\pgfpathlineto{\pgfqpoint{4.237504in}{2.021795in}}%
\pgfpathlineto{\pgfqpoint{4.237504in}{2.015896in}}%
\pgfpathmoveto{\pgfqpoint{4.164844in}{2.186952in}}%
\pgfpathlineto{\pgfqpoint{4.164844in}{2.186952in}}%
\pgfpathlineto{\pgfqpoint{4.164844in}{2.192850in}}%
\pgfpathlineto{\pgfqpoint{4.173927in}{2.192850in}}%
\pgfpathlineto{\pgfqpoint{4.173927in}{2.186952in}}%
\pgfpathmoveto{\pgfqpoint{4.164844in}{2.192850in}}%
\pgfpathlineto{\pgfqpoint{4.164844in}{2.192850in}}%
\pgfpathlineto{\pgfqpoint{4.164844in}{2.198748in}}%
\pgfpathlineto{\pgfqpoint{4.173927in}{2.198748in}}%
\pgfpathlineto{\pgfqpoint{4.173927in}{2.192850in}}%
\pgfpathmoveto{\pgfqpoint{4.173927in}{2.186952in}}%
\pgfpathlineto{\pgfqpoint{4.173927in}{2.186952in}}%
\pgfpathlineto{\pgfqpoint{4.173927in}{2.192850in}}%
\pgfpathlineto{\pgfqpoint{4.183009in}{2.192850in}}%
\pgfpathlineto{\pgfqpoint{4.183009in}{2.186952in}}%
\pgfpathmoveto{\pgfqpoint{4.183009in}{2.175155in}}%
\pgfpathlineto{\pgfqpoint{4.183009in}{2.175155in}}%
\pgfpathlineto{\pgfqpoint{4.183009in}{2.181053in}}%
\pgfpathlineto{\pgfqpoint{4.192092in}{2.181053in}}%
\pgfpathlineto{\pgfqpoint{4.192092in}{2.175155in}}%
\pgfpathmoveto{\pgfqpoint{4.183009in}{2.181053in}}%
\pgfpathlineto{\pgfqpoint{4.183009in}{2.181053in}}%
\pgfpathlineto{\pgfqpoint{4.183009in}{2.186952in}}%
\pgfpathlineto{\pgfqpoint{4.192092in}{2.186952in}}%
\pgfpathlineto{\pgfqpoint{4.192092in}{2.181053in}}%
\pgfpathmoveto{\pgfqpoint{4.192092in}{2.175155in}}%
\pgfpathlineto{\pgfqpoint{4.192092in}{2.175155in}}%
\pgfpathlineto{\pgfqpoint{4.192092in}{2.181053in}}%
\pgfpathlineto{\pgfqpoint{4.201174in}{2.181053in}}%
\pgfpathlineto{\pgfqpoint{4.201174in}{2.175155in}}%
\pgfpathmoveto{\pgfqpoint{4.201174in}{2.163359in}}%
\pgfpathlineto{\pgfqpoint{4.201174in}{2.163359in}}%
\pgfpathlineto{\pgfqpoint{4.201174in}{2.169257in}}%
\pgfpathlineto{\pgfqpoint{4.210257in}{2.169257in}}%
\pgfpathlineto{\pgfqpoint{4.210257in}{2.163359in}}%
\pgfpathmoveto{\pgfqpoint{4.201174in}{2.169257in}}%
\pgfpathlineto{\pgfqpoint{4.201174in}{2.169257in}}%
\pgfpathlineto{\pgfqpoint{4.201174in}{2.175155in}}%
\pgfpathlineto{\pgfqpoint{4.210257in}{2.175155in}}%
\pgfpathlineto{\pgfqpoint{4.210257in}{2.169257in}}%
\pgfpathmoveto{\pgfqpoint{4.210257in}{2.163359in}}%
\pgfpathlineto{\pgfqpoint{4.210257in}{2.163359in}}%
\pgfpathlineto{\pgfqpoint{4.210257in}{2.169257in}}%
\pgfpathlineto{\pgfqpoint{4.219339in}{2.169257in}}%
\pgfpathlineto{\pgfqpoint{4.219339in}{2.163359in}}%
\pgfpathmoveto{\pgfqpoint{4.219339in}{2.151563in}}%
\pgfpathlineto{\pgfqpoint{4.219339in}{2.151563in}}%
\pgfpathlineto{\pgfqpoint{4.219339in}{2.157461in}}%
\pgfpathlineto{\pgfqpoint{4.228422in}{2.157461in}}%
\pgfpathlineto{\pgfqpoint{4.228422in}{2.151563in}}%
\pgfpathmoveto{\pgfqpoint{4.219339in}{2.157461in}}%
\pgfpathlineto{\pgfqpoint{4.219339in}{2.157461in}}%
\pgfpathlineto{\pgfqpoint{4.219339in}{2.163359in}}%
\pgfpathlineto{\pgfqpoint{4.228422in}{2.163359in}}%
\pgfpathlineto{\pgfqpoint{4.228422in}{2.157461in}}%
\pgfpathmoveto{\pgfqpoint{4.228422in}{2.151563in}}%
\pgfpathlineto{\pgfqpoint{4.228422in}{2.151563in}}%
\pgfpathlineto{\pgfqpoint{4.228422in}{2.157461in}}%
\pgfpathlineto{\pgfqpoint{4.237504in}{2.157461in}}%
\pgfpathlineto{\pgfqpoint{4.237504in}{2.151563in}}%
\pgfpathmoveto{\pgfqpoint{4.092185in}{2.234139in}}%
\pgfpathlineto{\pgfqpoint{4.092185in}{2.234139in}}%
\pgfpathlineto{\pgfqpoint{4.092185in}{2.240038in}}%
\pgfpathlineto{\pgfqpoint{4.101267in}{2.240038in}}%
\pgfpathlineto{\pgfqpoint{4.101267in}{2.234139in}}%
\pgfpathmoveto{\pgfqpoint{4.092185in}{2.240038in}}%
\pgfpathlineto{\pgfqpoint{4.092185in}{2.240038in}}%
\pgfpathlineto{\pgfqpoint{4.092185in}{2.245936in}}%
\pgfpathlineto{\pgfqpoint{4.101267in}{2.245936in}}%
\pgfpathlineto{\pgfqpoint{4.101267in}{2.240038in}}%
\pgfpathmoveto{\pgfqpoint{4.101267in}{2.234139in}}%
\pgfpathlineto{\pgfqpoint{4.101267in}{2.234139in}}%
\pgfpathlineto{\pgfqpoint{4.101267in}{2.240038in}}%
\pgfpathlineto{\pgfqpoint{4.110350in}{2.240038in}}%
\pgfpathlineto{\pgfqpoint{4.110350in}{2.234139in}}%
\pgfpathmoveto{\pgfqpoint{4.110350in}{2.222342in}}%
\pgfpathlineto{\pgfqpoint{4.110350in}{2.222342in}}%
\pgfpathlineto{\pgfqpoint{4.110350in}{2.228241in}}%
\pgfpathlineto{\pgfqpoint{4.119432in}{2.228241in}}%
\pgfpathlineto{\pgfqpoint{4.119432in}{2.222342in}}%
\pgfpathmoveto{\pgfqpoint{4.110350in}{2.228241in}}%
\pgfpathlineto{\pgfqpoint{4.110350in}{2.228241in}}%
\pgfpathlineto{\pgfqpoint{4.110350in}{2.234139in}}%
\pgfpathlineto{\pgfqpoint{4.119432in}{2.234139in}}%
\pgfpathlineto{\pgfqpoint{4.119432in}{2.228241in}}%
\pgfpathmoveto{\pgfqpoint{4.119432in}{2.222342in}}%
\pgfpathlineto{\pgfqpoint{4.119432in}{2.222342in}}%
\pgfpathlineto{\pgfqpoint{4.119432in}{2.228241in}}%
\pgfpathlineto{\pgfqpoint{4.128515in}{2.228241in}}%
\pgfpathlineto{\pgfqpoint{4.128515in}{2.222342in}}%
\pgfpathmoveto{\pgfqpoint{4.128515in}{2.210545in}}%
\pgfpathlineto{\pgfqpoint{4.128515in}{2.210545in}}%
\pgfpathlineto{\pgfqpoint{4.128515in}{2.216444in}}%
\pgfpathlineto{\pgfqpoint{4.137597in}{2.216444in}}%
\pgfpathlineto{\pgfqpoint{4.137597in}{2.210545in}}%
\pgfpathmoveto{\pgfqpoint{4.128515in}{2.216444in}}%
\pgfpathlineto{\pgfqpoint{4.128515in}{2.216444in}}%
\pgfpathlineto{\pgfqpoint{4.128515in}{2.222342in}}%
\pgfpathlineto{\pgfqpoint{4.137597in}{2.222342in}}%
\pgfpathlineto{\pgfqpoint{4.137597in}{2.216444in}}%
\pgfpathmoveto{\pgfqpoint{4.137597in}{2.210545in}}%
\pgfpathlineto{\pgfqpoint{4.137597in}{2.210545in}}%
\pgfpathlineto{\pgfqpoint{4.137597in}{2.216444in}}%
\pgfpathlineto{\pgfqpoint{4.146680in}{2.216444in}}%
\pgfpathlineto{\pgfqpoint{4.146680in}{2.210545in}}%
\pgfpathmoveto{\pgfqpoint{4.146680in}{2.198748in}}%
\pgfpathlineto{\pgfqpoint{4.146680in}{2.198748in}}%
\pgfpathlineto{\pgfqpoint{4.146680in}{2.204646in}}%
\pgfpathlineto{\pgfqpoint{4.155762in}{2.204646in}}%
\pgfpathlineto{\pgfqpoint{4.155762in}{2.198748in}}%
\pgfpathmoveto{\pgfqpoint{4.146680in}{2.204646in}}%
\pgfpathlineto{\pgfqpoint{4.146680in}{2.204646in}}%
\pgfpathlineto{\pgfqpoint{4.146680in}{2.210545in}}%
\pgfpathlineto{\pgfqpoint{4.155762in}{2.210545in}}%
\pgfpathlineto{\pgfqpoint{4.155762in}{2.204646in}}%
\pgfpathmoveto{\pgfqpoint{4.155762in}{2.198748in}}%
\pgfpathlineto{\pgfqpoint{4.155762in}{2.198748in}}%
\pgfpathlineto{\pgfqpoint{4.155762in}{2.204646in}}%
\pgfpathlineto{\pgfqpoint{4.164844in}{2.204646in}}%
\pgfpathlineto{\pgfqpoint{4.164844in}{2.198748in}}%
\pgfpathmoveto{\pgfqpoint{4.237504in}{2.015896in}}%
\pgfpathlineto{\pgfqpoint{4.237504in}{2.015896in}}%
\pgfpathlineto{\pgfqpoint{4.237504in}{2.021795in}}%
\pgfpathlineto{\pgfqpoint{4.246586in}{2.021795in}}%
\pgfpathlineto{\pgfqpoint{4.246586in}{2.015896in}}%
\pgfpathmoveto{\pgfqpoint{4.246586in}{2.015896in}}%
\pgfpathlineto{\pgfqpoint{4.246586in}{2.015896in}}%
\pgfpathlineto{\pgfqpoint{4.246586in}{2.021795in}}%
\pgfpathlineto{\pgfqpoint{4.255667in}{2.021795in}}%
\pgfpathlineto{\pgfqpoint{4.255667in}{2.015896in}}%
\pgfpathmoveto{\pgfqpoint{4.255667in}{2.015896in}}%
\pgfpathlineto{\pgfqpoint{4.255667in}{2.015896in}}%
\pgfpathlineto{\pgfqpoint{4.255667in}{2.021795in}}%
\pgfpathlineto{\pgfqpoint{4.264749in}{2.021795in}}%
\pgfpathlineto{\pgfqpoint{4.264749in}{2.015896in}}%
\pgfpathmoveto{\pgfqpoint{4.264749in}{2.015896in}}%
\pgfpathlineto{\pgfqpoint{4.264749in}{2.015896in}}%
\pgfpathlineto{\pgfqpoint{4.264749in}{2.021795in}}%
\pgfpathlineto{\pgfqpoint{4.273831in}{2.021795in}}%
\pgfpathlineto{\pgfqpoint{4.273831in}{2.015896in}}%
\pgfpathmoveto{\pgfqpoint{4.273831in}{2.015896in}}%
\pgfpathlineto{\pgfqpoint{4.273831in}{2.015896in}}%
\pgfpathlineto{\pgfqpoint{4.273831in}{2.021795in}}%
\pgfpathlineto{\pgfqpoint{4.282912in}{2.021795in}}%
\pgfpathlineto{\pgfqpoint{4.282912in}{2.015896in}}%
\pgfpathmoveto{\pgfqpoint{4.282912in}{2.015896in}}%
\pgfpathlineto{\pgfqpoint{4.282912in}{2.015896in}}%
\pgfpathlineto{\pgfqpoint{4.282912in}{2.021795in}}%
\pgfpathlineto{\pgfqpoint{4.291994in}{2.021795in}}%
\pgfpathlineto{\pgfqpoint{4.291994in}{2.015896in}}%
\pgfpathmoveto{\pgfqpoint{4.291994in}{2.015896in}}%
\pgfpathlineto{\pgfqpoint{4.291994in}{2.015896in}}%
\pgfpathlineto{\pgfqpoint{4.291994in}{2.021795in}}%
\pgfpathlineto{\pgfqpoint{4.301076in}{2.021795in}}%
\pgfpathlineto{\pgfqpoint{4.301076in}{2.015896in}}%
\pgfpathmoveto{\pgfqpoint{4.301076in}{2.015896in}}%
\pgfpathlineto{\pgfqpoint{4.301076in}{2.015896in}}%
\pgfpathlineto{\pgfqpoint{4.301076in}{2.021795in}}%
\pgfpathlineto{\pgfqpoint{4.310157in}{2.021795in}}%
\pgfpathlineto{\pgfqpoint{4.310157in}{2.015896in}}%
\pgfpathmoveto{\pgfqpoint{4.310157in}{2.015896in}}%
\pgfpathlineto{\pgfqpoint{4.310157in}{2.015896in}}%
\pgfpathlineto{\pgfqpoint{4.310157in}{2.021795in}}%
\pgfpathlineto{\pgfqpoint{4.319239in}{2.021795in}}%
\pgfpathlineto{\pgfqpoint{4.319239in}{2.015896in}}%
\pgfpathmoveto{\pgfqpoint{4.319239in}{2.015896in}}%
\pgfpathlineto{\pgfqpoint{4.319239in}{2.015896in}}%
\pgfpathlineto{\pgfqpoint{4.319239in}{2.021795in}}%
\pgfpathlineto{\pgfqpoint{4.328321in}{2.021795in}}%
\pgfpathlineto{\pgfqpoint{4.328321in}{2.015896in}}%
\pgfpathmoveto{\pgfqpoint{4.328321in}{2.015896in}}%
\pgfpathlineto{\pgfqpoint{4.328321in}{2.015896in}}%
\pgfpathlineto{\pgfqpoint{4.328321in}{2.021795in}}%
\pgfpathlineto{\pgfqpoint{4.337402in}{2.021795in}}%
\pgfpathlineto{\pgfqpoint{4.337402in}{2.015896in}}%
\pgfpathmoveto{\pgfqpoint{4.337402in}{2.015896in}}%
\pgfpathlineto{\pgfqpoint{4.337402in}{2.015896in}}%
\pgfpathlineto{\pgfqpoint{4.337402in}{2.021795in}}%
\pgfpathlineto{\pgfqpoint{4.346484in}{2.021795in}}%
\pgfpathlineto{\pgfqpoint{4.346484in}{2.015896in}}%
\pgfpathmoveto{\pgfqpoint{4.346484in}{2.015896in}}%
\pgfpathlineto{\pgfqpoint{4.346484in}{2.015896in}}%
\pgfpathlineto{\pgfqpoint{4.346484in}{2.021795in}}%
\pgfpathlineto{\pgfqpoint{4.355566in}{2.021795in}}%
\pgfpathlineto{\pgfqpoint{4.355566in}{2.015896in}}%
\pgfpathmoveto{\pgfqpoint{4.355566in}{2.015896in}}%
\pgfpathlineto{\pgfqpoint{4.355566in}{2.015896in}}%
\pgfpathlineto{\pgfqpoint{4.355566in}{2.021795in}}%
\pgfpathlineto{\pgfqpoint{4.364647in}{2.021795in}}%
\pgfpathlineto{\pgfqpoint{4.364647in}{2.015896in}}%
\pgfpathmoveto{\pgfqpoint{4.364647in}{2.015896in}}%
\pgfpathlineto{\pgfqpoint{4.364647in}{2.015896in}}%
\pgfpathlineto{\pgfqpoint{4.364647in}{2.021795in}}%
\pgfpathlineto{\pgfqpoint{4.373729in}{2.021795in}}%
\pgfpathlineto{\pgfqpoint{4.373729in}{2.015896in}}%
\pgfpathmoveto{\pgfqpoint{4.373729in}{2.015896in}}%
\pgfpathlineto{\pgfqpoint{4.373729in}{2.015896in}}%
\pgfpathlineto{\pgfqpoint{4.373729in}{2.021795in}}%
\pgfpathlineto{\pgfqpoint{4.382811in}{2.021795in}}%
\pgfpathlineto{\pgfqpoint{4.382811in}{2.015896in}}%
\pgfpathmoveto{\pgfqpoint{4.310157in}{2.092580in}}%
\pgfpathlineto{\pgfqpoint{4.310157in}{2.092580in}}%
\pgfpathlineto{\pgfqpoint{4.310157in}{2.098479in}}%
\pgfpathlineto{\pgfqpoint{4.319239in}{2.098479in}}%
\pgfpathlineto{\pgfqpoint{4.319239in}{2.092580in}}%
\pgfpathmoveto{\pgfqpoint{4.310157in}{2.098479in}}%
\pgfpathlineto{\pgfqpoint{4.310157in}{2.098479in}}%
\pgfpathlineto{\pgfqpoint{4.310157in}{2.104378in}}%
\pgfpathlineto{\pgfqpoint{4.319239in}{2.104378in}}%
\pgfpathlineto{\pgfqpoint{4.319239in}{2.098479in}}%
\pgfpathmoveto{\pgfqpoint{4.319239in}{2.092580in}}%
\pgfpathlineto{\pgfqpoint{4.319239in}{2.092580in}}%
\pgfpathlineto{\pgfqpoint{4.319239in}{2.098479in}}%
\pgfpathlineto{\pgfqpoint{4.328321in}{2.098479in}}%
\pgfpathlineto{\pgfqpoint{4.328321in}{2.092580in}}%
\pgfpathmoveto{\pgfqpoint{4.328321in}{2.080783in}}%
\pgfpathlineto{\pgfqpoint{4.328321in}{2.080783in}}%
\pgfpathlineto{\pgfqpoint{4.328321in}{2.086681in}}%
\pgfpathlineto{\pgfqpoint{4.337402in}{2.086681in}}%
\pgfpathlineto{\pgfqpoint{4.337402in}{2.080783in}}%
\pgfpathmoveto{\pgfqpoint{4.328321in}{2.086681in}}%
\pgfpathlineto{\pgfqpoint{4.328321in}{2.086681in}}%
\pgfpathlineto{\pgfqpoint{4.328321in}{2.092580in}}%
\pgfpathlineto{\pgfqpoint{4.337402in}{2.092580in}}%
\pgfpathlineto{\pgfqpoint{4.337402in}{2.086681in}}%
\pgfpathmoveto{\pgfqpoint{4.337402in}{2.080783in}}%
\pgfpathlineto{\pgfqpoint{4.337402in}{2.080783in}}%
\pgfpathlineto{\pgfqpoint{4.337402in}{2.086681in}}%
\pgfpathlineto{\pgfqpoint{4.346484in}{2.086681in}}%
\pgfpathlineto{\pgfqpoint{4.346484in}{2.080783in}}%
\pgfpathmoveto{\pgfqpoint{4.346484in}{2.068985in}}%
\pgfpathlineto{\pgfqpoint{4.346484in}{2.068985in}}%
\pgfpathlineto{\pgfqpoint{4.346484in}{2.074884in}}%
\pgfpathlineto{\pgfqpoint{4.355566in}{2.074884in}}%
\pgfpathlineto{\pgfqpoint{4.355566in}{2.068985in}}%
\pgfpathmoveto{\pgfqpoint{4.346484in}{2.074884in}}%
\pgfpathlineto{\pgfqpoint{4.346484in}{2.074884in}}%
\pgfpathlineto{\pgfqpoint{4.346484in}{2.080783in}}%
\pgfpathlineto{\pgfqpoint{4.355566in}{2.080783in}}%
\pgfpathlineto{\pgfqpoint{4.355566in}{2.074884in}}%
\pgfpathmoveto{\pgfqpoint{4.355566in}{2.068985in}}%
\pgfpathlineto{\pgfqpoint{4.355566in}{2.068985in}}%
\pgfpathlineto{\pgfqpoint{4.355566in}{2.074884in}}%
\pgfpathlineto{\pgfqpoint{4.364647in}{2.074884in}}%
\pgfpathlineto{\pgfqpoint{4.364647in}{2.068985in}}%
\pgfpathmoveto{\pgfqpoint{4.364647in}{2.057187in}}%
\pgfpathlineto{\pgfqpoint{4.364647in}{2.057187in}}%
\pgfpathlineto{\pgfqpoint{4.364647in}{2.063086in}}%
\pgfpathlineto{\pgfqpoint{4.373729in}{2.063086in}}%
\pgfpathlineto{\pgfqpoint{4.373729in}{2.057187in}}%
\pgfpathmoveto{\pgfqpoint{4.364647in}{2.063086in}}%
\pgfpathlineto{\pgfqpoint{4.364647in}{2.063086in}}%
\pgfpathlineto{\pgfqpoint{4.364647in}{2.068985in}}%
\pgfpathlineto{\pgfqpoint{4.373729in}{2.068985in}}%
\pgfpathlineto{\pgfqpoint{4.373729in}{2.063086in}}%
\pgfpathmoveto{\pgfqpoint{4.373729in}{2.057187in}}%
\pgfpathlineto{\pgfqpoint{4.373729in}{2.057187in}}%
\pgfpathlineto{\pgfqpoint{4.373729in}{2.063086in}}%
\pgfpathlineto{\pgfqpoint{4.382811in}{2.063086in}}%
\pgfpathlineto{\pgfqpoint{4.382811in}{2.057187in}}%
\pgfpathmoveto{\pgfqpoint{4.237504in}{2.139766in}}%
\pgfpathlineto{\pgfqpoint{4.237504in}{2.139766in}}%
\pgfpathlineto{\pgfqpoint{4.237504in}{2.145665in}}%
\pgfpathlineto{\pgfqpoint{4.246586in}{2.145665in}}%
\pgfpathlineto{\pgfqpoint{4.246586in}{2.139766in}}%
\pgfpathmoveto{\pgfqpoint{4.237504in}{2.145665in}}%
\pgfpathlineto{\pgfqpoint{4.237504in}{2.145665in}}%
\pgfpathlineto{\pgfqpoint{4.237504in}{2.151563in}}%
\pgfpathlineto{\pgfqpoint{4.246586in}{2.151563in}}%
\pgfpathlineto{\pgfqpoint{4.246586in}{2.145665in}}%
\pgfpathmoveto{\pgfqpoint{4.246586in}{2.139766in}}%
\pgfpathlineto{\pgfqpoint{4.246586in}{2.139766in}}%
\pgfpathlineto{\pgfqpoint{4.246586in}{2.145665in}}%
\pgfpathlineto{\pgfqpoint{4.255667in}{2.145665in}}%
\pgfpathlineto{\pgfqpoint{4.255667in}{2.139766in}}%
\pgfpathmoveto{\pgfqpoint{4.255667in}{2.127970in}}%
\pgfpathlineto{\pgfqpoint{4.255667in}{2.127970in}}%
\pgfpathlineto{\pgfqpoint{4.255667in}{2.133868in}}%
\pgfpathlineto{\pgfqpoint{4.264749in}{2.133868in}}%
\pgfpathlineto{\pgfqpoint{4.264749in}{2.127970in}}%
\pgfpathmoveto{\pgfqpoint{4.255667in}{2.133868in}}%
\pgfpathlineto{\pgfqpoint{4.255667in}{2.133868in}}%
\pgfpathlineto{\pgfqpoint{4.255667in}{2.139766in}}%
\pgfpathlineto{\pgfqpoint{4.264749in}{2.139766in}}%
\pgfpathlineto{\pgfqpoint{4.264749in}{2.133868in}}%
\pgfpathmoveto{\pgfqpoint{4.264749in}{2.127970in}}%
\pgfpathlineto{\pgfqpoint{4.264749in}{2.127970in}}%
\pgfpathlineto{\pgfqpoint{4.264749in}{2.133868in}}%
\pgfpathlineto{\pgfqpoint{4.273831in}{2.133868in}}%
\pgfpathlineto{\pgfqpoint{4.273831in}{2.127970in}}%
\pgfpathmoveto{\pgfqpoint{4.273831in}{2.116174in}}%
\pgfpathlineto{\pgfqpoint{4.273831in}{2.116174in}}%
\pgfpathlineto{\pgfqpoint{4.273831in}{2.122072in}}%
\pgfpathlineto{\pgfqpoint{4.282912in}{2.122072in}}%
\pgfpathlineto{\pgfqpoint{4.282912in}{2.116174in}}%
\pgfpathmoveto{\pgfqpoint{4.273831in}{2.122072in}}%
\pgfpathlineto{\pgfqpoint{4.273831in}{2.122072in}}%
\pgfpathlineto{\pgfqpoint{4.273831in}{2.127970in}}%
\pgfpathlineto{\pgfqpoint{4.282912in}{2.127970in}}%
\pgfpathlineto{\pgfqpoint{4.282912in}{2.122072in}}%
\pgfpathmoveto{\pgfqpoint{4.282912in}{2.116174in}}%
\pgfpathlineto{\pgfqpoint{4.282912in}{2.116174in}}%
\pgfpathlineto{\pgfqpoint{4.282912in}{2.122072in}}%
\pgfpathlineto{\pgfqpoint{4.291994in}{2.122072in}}%
\pgfpathlineto{\pgfqpoint{4.291994in}{2.116174in}}%
\pgfpathmoveto{\pgfqpoint{4.291994in}{2.104378in}}%
\pgfpathlineto{\pgfqpoint{4.291994in}{2.104378in}}%
\pgfpathlineto{\pgfqpoint{4.291994in}{2.110276in}}%
\pgfpathlineto{\pgfqpoint{4.301076in}{2.110276in}}%
\pgfpathlineto{\pgfqpoint{4.301076in}{2.104378in}}%
\pgfpathmoveto{\pgfqpoint{4.291994in}{2.110276in}}%
\pgfpathlineto{\pgfqpoint{4.291994in}{2.110276in}}%
\pgfpathlineto{\pgfqpoint{4.291994in}{2.116174in}}%
\pgfpathlineto{\pgfqpoint{4.301076in}{2.116174in}}%
\pgfpathlineto{\pgfqpoint{4.301076in}{2.110276in}}%
\pgfpathmoveto{\pgfqpoint{4.301076in}{2.104378in}}%
\pgfpathlineto{\pgfqpoint{4.301076in}{2.104378in}}%
\pgfpathlineto{\pgfqpoint{4.301076in}{2.110276in}}%
\pgfpathlineto{\pgfqpoint{4.310157in}{2.110276in}}%
\pgfpathlineto{\pgfqpoint{4.310157in}{2.104378in}}%
\pgfpathmoveto{\pgfqpoint{4.382811in}{2.015896in}}%
\pgfpathlineto{\pgfqpoint{4.382811in}{2.015896in}}%
\pgfpathlineto{\pgfqpoint{4.382811in}{2.021795in}}%
\pgfpathlineto{\pgfqpoint{4.391893in}{2.021795in}}%
\pgfpathlineto{\pgfqpoint{4.391893in}{2.015896in}}%
\pgfpathmoveto{\pgfqpoint{4.391893in}{2.015896in}}%
\pgfpathlineto{\pgfqpoint{4.391893in}{2.015896in}}%
\pgfpathlineto{\pgfqpoint{4.391893in}{2.021795in}}%
\pgfpathlineto{\pgfqpoint{4.400975in}{2.021795in}}%
\pgfpathlineto{\pgfqpoint{4.400975in}{2.015896in}}%
\pgfpathmoveto{\pgfqpoint{4.400975in}{2.015896in}}%
\pgfpathlineto{\pgfqpoint{4.400975in}{2.015896in}}%
\pgfpathlineto{\pgfqpoint{4.400975in}{2.021795in}}%
\pgfpathlineto{\pgfqpoint{4.410058in}{2.021795in}}%
\pgfpathlineto{\pgfqpoint{4.410058in}{2.015896in}}%
\pgfpathmoveto{\pgfqpoint{4.410058in}{2.015896in}}%
\pgfpathlineto{\pgfqpoint{4.410058in}{2.015896in}}%
\pgfpathlineto{\pgfqpoint{4.410058in}{2.021795in}}%
\pgfpathlineto{\pgfqpoint{4.419140in}{2.021795in}}%
\pgfpathlineto{\pgfqpoint{4.419140in}{2.015896in}}%
\pgfpathmoveto{\pgfqpoint{4.382811in}{2.045390in}}%
\pgfpathlineto{\pgfqpoint{4.382811in}{2.045390in}}%
\pgfpathlineto{\pgfqpoint{4.382811in}{2.051289in}}%
\pgfpathlineto{\pgfqpoint{4.391893in}{2.051289in}}%
\pgfpathlineto{\pgfqpoint{4.391893in}{2.045390in}}%
\pgfpathmoveto{\pgfqpoint{4.382811in}{2.051289in}}%
\pgfpathlineto{\pgfqpoint{4.382811in}{2.051289in}}%
\pgfpathlineto{\pgfqpoint{4.382811in}{2.057187in}}%
\pgfpathlineto{\pgfqpoint{4.391893in}{2.057187in}}%
\pgfpathlineto{\pgfqpoint{4.391893in}{2.051289in}}%
\pgfpathmoveto{\pgfqpoint{4.391893in}{2.045390in}}%
\pgfpathlineto{\pgfqpoint{4.391893in}{2.045390in}}%
\pgfpathlineto{\pgfqpoint{4.391893in}{2.051289in}}%
\pgfpathlineto{\pgfqpoint{4.400975in}{2.051289in}}%
\pgfpathlineto{\pgfqpoint{4.400975in}{2.045390in}}%
\pgfpathmoveto{\pgfqpoint{4.400975in}{2.033592in}}%
\pgfpathlineto{\pgfqpoint{4.400975in}{2.033592in}}%
\pgfpathlineto{\pgfqpoint{4.400975in}{2.039491in}}%
\pgfpathlineto{\pgfqpoint{4.410058in}{2.039491in}}%
\pgfpathlineto{\pgfqpoint{4.410058in}{2.033592in}}%
\pgfpathmoveto{\pgfqpoint{4.400975in}{2.039491in}}%
\pgfpathlineto{\pgfqpoint{4.400975in}{2.039491in}}%
\pgfpathlineto{\pgfqpoint{4.400975in}{2.045390in}}%
\pgfpathlineto{\pgfqpoint{4.410058in}{2.045390in}}%
\pgfpathlineto{\pgfqpoint{4.410058in}{2.039491in}}%
\pgfpathmoveto{\pgfqpoint{4.410058in}{2.033592in}}%
\pgfpathlineto{\pgfqpoint{4.410058in}{2.033592in}}%
\pgfpathlineto{\pgfqpoint{4.410058in}{2.039491in}}%
\pgfpathlineto{\pgfqpoint{4.419140in}{2.039491in}}%
\pgfpathlineto{\pgfqpoint{4.419140in}{2.033592in}}%
\pgfpathmoveto{\pgfqpoint{4.419140in}{2.015896in}}%
\pgfpathlineto{\pgfqpoint{4.419140in}{2.015896in}}%
\pgfpathlineto{\pgfqpoint{4.419140in}{2.021795in}}%
\pgfpathlineto{\pgfqpoint{4.428222in}{2.021795in}}%
\pgfpathlineto{\pgfqpoint{4.428222in}{2.015896in}}%
\pgfpathmoveto{\pgfqpoint{4.428222in}{2.015896in}}%
\pgfpathlineto{\pgfqpoint{4.428222in}{2.015896in}}%
\pgfpathlineto{\pgfqpoint{4.428222in}{2.021795in}}%
\pgfpathlineto{\pgfqpoint{4.437305in}{2.021795in}}%
\pgfpathlineto{\pgfqpoint{4.437305in}{2.015896in}}%
\pgfpathmoveto{\pgfqpoint{4.419140in}{2.021795in}}%
\pgfpathlineto{\pgfqpoint{4.419140in}{2.021795in}}%
\pgfpathlineto{\pgfqpoint{4.419140in}{2.027694in}}%
\pgfpathlineto{\pgfqpoint{4.428222in}{2.027694in}}%
\pgfpathlineto{\pgfqpoint{4.428222in}{2.021795in}}%
\pgfpathmoveto{\pgfqpoint{4.419140in}{2.027694in}}%
\pgfpathlineto{\pgfqpoint{4.419140in}{2.027694in}}%
\pgfpathlineto{\pgfqpoint{4.419140in}{2.033592in}}%
\pgfpathlineto{\pgfqpoint{4.428222in}{2.033592in}}%
\pgfpathlineto{\pgfqpoint{4.428222in}{2.027694in}}%
\pgfpathmoveto{\pgfqpoint{4.428222in}{2.021795in}}%
\pgfpathlineto{\pgfqpoint{4.428222in}{2.021795in}}%
\pgfpathlineto{\pgfqpoint{4.428222in}{2.027694in}}%
\pgfpathlineto{\pgfqpoint{4.437305in}{2.027694in}}%
\pgfpathlineto{\pgfqpoint{4.437305in}{2.021795in}}%
\pgfpathmoveto{\pgfqpoint{4.437305in}{2.015896in}}%
\pgfpathlineto{\pgfqpoint{4.437305in}{2.015896in}}%
\pgfpathlineto{\pgfqpoint{4.437305in}{2.021795in}}%
\pgfpathlineto{\pgfqpoint{4.446387in}{2.021795in}}%
\pgfpathlineto{\pgfqpoint{4.446387in}{2.015896in}}%
\pgfpathmoveto{\pgfqpoint{3.070463in}{2.009997in}}%
\pgfpathlineto{\pgfqpoint{3.070463in}{2.009997in}}%
\pgfpathlineto{\pgfqpoint{3.070463in}{2.012947in}}%
\pgfpathlineto{\pgfqpoint{3.075004in}{2.012947in}}%
\pgfpathlineto{\pgfqpoint{3.075004in}{2.009997in}}%
\pgfpathmoveto{\pgfqpoint{3.070463in}{2.012947in}}%
\pgfpathlineto{\pgfqpoint{3.070463in}{2.012947in}}%
\pgfpathlineto{\pgfqpoint{3.070463in}{2.015896in}}%
\pgfpathlineto{\pgfqpoint{3.075004in}{2.015896in}}%
\pgfpathlineto{\pgfqpoint{3.075004in}{2.012947in}}%
\pgfpathmoveto{\pgfqpoint{3.070463in}{2.015896in}}%
\pgfpathlineto{\pgfqpoint{3.070463in}{2.015896in}}%
\pgfpathlineto{\pgfqpoint{3.070463in}{2.018846in}}%
\pgfpathlineto{\pgfqpoint{3.075004in}{2.018846in}}%
\pgfpathlineto{\pgfqpoint{3.075004in}{2.015896in}}%
\pgfpathmoveto{\pgfqpoint{3.070463in}{2.018846in}}%
\pgfpathlineto{\pgfqpoint{3.070463in}{2.018846in}}%
\pgfpathlineto{\pgfqpoint{3.070463in}{2.021795in}}%
\pgfpathlineto{\pgfqpoint{3.075004in}{2.021795in}}%
\pgfpathlineto{\pgfqpoint{3.075004in}{2.018846in}}%
\pgfpathmoveto{\pgfqpoint{3.070463in}{2.021795in}}%
\pgfpathlineto{\pgfqpoint{3.070463in}{2.021795in}}%
\pgfpathlineto{\pgfqpoint{3.070463in}{2.024744in}}%
\pgfpathlineto{\pgfqpoint{3.075004in}{2.024744in}}%
\pgfpathlineto{\pgfqpoint{3.075004in}{2.021795in}}%
\pgfpathmoveto{\pgfqpoint{3.070463in}{2.024744in}}%
\pgfpathlineto{\pgfqpoint{3.070463in}{2.024744in}}%
\pgfpathlineto{\pgfqpoint{3.070463in}{2.027694in}}%
\pgfpathlineto{\pgfqpoint{3.075004in}{2.027694in}}%
\pgfpathlineto{\pgfqpoint{3.075004in}{2.024744in}}%
\pgfpathmoveto{\pgfqpoint{3.070463in}{2.027694in}}%
\pgfpathlineto{\pgfqpoint{3.070463in}{2.027694in}}%
\pgfpathlineto{\pgfqpoint{3.070463in}{2.030643in}}%
\pgfpathlineto{\pgfqpoint{3.075004in}{2.030643in}}%
\pgfpathlineto{\pgfqpoint{3.075004in}{2.027694in}}%
\pgfpathmoveto{\pgfqpoint{3.070463in}{2.030643in}}%
\pgfpathlineto{\pgfqpoint{3.070463in}{2.030643in}}%
\pgfpathlineto{\pgfqpoint{3.070463in}{2.033592in}}%
\pgfpathlineto{\pgfqpoint{3.075004in}{2.033592in}}%
\pgfpathlineto{\pgfqpoint{3.075004in}{2.030643in}}%
\pgfpathmoveto{\pgfqpoint{3.070463in}{2.033592in}}%
\pgfpathlineto{\pgfqpoint{3.070463in}{2.033592in}}%
\pgfpathlineto{\pgfqpoint{3.070463in}{2.036542in}}%
\pgfpathlineto{\pgfqpoint{3.075004in}{2.036542in}}%
\pgfpathlineto{\pgfqpoint{3.075004in}{2.033592in}}%
\pgfpathmoveto{\pgfqpoint{3.070463in}{2.036542in}}%
\pgfpathlineto{\pgfqpoint{3.070463in}{2.036542in}}%
\pgfpathlineto{\pgfqpoint{3.070463in}{2.039491in}}%
\pgfpathlineto{\pgfqpoint{3.075004in}{2.039491in}}%
\pgfpathlineto{\pgfqpoint{3.075004in}{2.036542in}}%
\pgfpathmoveto{\pgfqpoint{3.070463in}{2.039491in}}%
\pgfpathlineto{\pgfqpoint{3.070463in}{2.039491in}}%
\pgfpathlineto{\pgfqpoint{3.070463in}{2.042441in}}%
\pgfpathlineto{\pgfqpoint{3.075004in}{2.042441in}}%
\pgfpathlineto{\pgfqpoint{3.075004in}{2.039491in}}%
\pgfpathmoveto{\pgfqpoint{3.070463in}{2.042441in}}%
\pgfpathlineto{\pgfqpoint{3.070463in}{2.042441in}}%
\pgfpathlineto{\pgfqpoint{3.070463in}{2.045390in}}%
\pgfpathlineto{\pgfqpoint{3.075004in}{2.045390in}}%
\pgfpathlineto{\pgfqpoint{3.075004in}{2.042441in}}%
\pgfpathmoveto{\pgfqpoint{3.070463in}{2.045390in}}%
\pgfpathlineto{\pgfqpoint{3.070463in}{2.045390in}}%
\pgfpathlineto{\pgfqpoint{3.070463in}{2.048339in}}%
\pgfpathlineto{\pgfqpoint{3.075004in}{2.048339in}}%
\pgfpathlineto{\pgfqpoint{3.075004in}{2.045390in}}%
\pgfpathmoveto{\pgfqpoint{3.070463in}{2.048339in}}%
\pgfpathlineto{\pgfqpoint{3.070463in}{2.048339in}}%
\pgfpathlineto{\pgfqpoint{3.070463in}{2.051289in}}%
\pgfpathlineto{\pgfqpoint{3.075004in}{2.051289in}}%
\pgfpathlineto{\pgfqpoint{3.075004in}{2.048339in}}%
\pgfpathmoveto{\pgfqpoint{3.070463in}{2.051289in}}%
\pgfpathlineto{\pgfqpoint{3.070463in}{2.051289in}}%
\pgfpathlineto{\pgfqpoint{3.070463in}{2.054238in}}%
\pgfpathlineto{\pgfqpoint{3.075004in}{2.054238in}}%
\pgfpathlineto{\pgfqpoint{3.075004in}{2.051289in}}%
\pgfpathmoveto{\pgfqpoint{3.070463in}{2.054238in}}%
\pgfpathlineto{\pgfqpoint{3.070463in}{2.054238in}}%
\pgfpathlineto{\pgfqpoint{3.070463in}{2.057187in}}%
\pgfpathlineto{\pgfqpoint{3.075004in}{2.057187in}}%
\pgfpathlineto{\pgfqpoint{3.075004in}{2.054238in}}%
\pgfpathmoveto{\pgfqpoint{3.070463in}{2.057187in}}%
\pgfpathlineto{\pgfqpoint{3.070463in}{2.057187in}}%
\pgfpathlineto{\pgfqpoint{3.070463in}{2.060137in}}%
\pgfpathlineto{\pgfqpoint{3.075004in}{2.060137in}}%
\pgfpathlineto{\pgfqpoint{3.075004in}{2.057187in}}%
\pgfpathmoveto{\pgfqpoint{3.070463in}{2.060137in}}%
\pgfpathlineto{\pgfqpoint{3.070463in}{2.060137in}}%
\pgfpathlineto{\pgfqpoint{3.070463in}{2.063086in}}%
\pgfpathlineto{\pgfqpoint{3.075004in}{2.063086in}}%
\pgfpathlineto{\pgfqpoint{3.075004in}{2.060137in}}%
\pgfpathmoveto{\pgfqpoint{3.070463in}{2.063086in}}%
\pgfpathlineto{\pgfqpoint{3.070463in}{2.063086in}}%
\pgfpathlineto{\pgfqpoint{3.070463in}{2.066036in}}%
\pgfpathlineto{\pgfqpoint{3.075004in}{2.066036in}}%
\pgfpathlineto{\pgfqpoint{3.075004in}{2.063086in}}%
\pgfpathmoveto{\pgfqpoint{3.070463in}{2.066036in}}%
\pgfpathlineto{\pgfqpoint{3.070463in}{2.066036in}}%
\pgfpathlineto{\pgfqpoint{3.070463in}{2.068985in}}%
\pgfpathlineto{\pgfqpoint{3.075004in}{2.068985in}}%
\pgfpathlineto{\pgfqpoint{3.075004in}{2.066036in}}%
\pgfpathmoveto{\pgfqpoint{3.070463in}{2.068985in}}%
\pgfpathlineto{\pgfqpoint{3.070463in}{2.068985in}}%
\pgfpathlineto{\pgfqpoint{3.070463in}{2.071934in}}%
\pgfpathlineto{\pgfqpoint{3.075004in}{2.071934in}}%
\pgfpathlineto{\pgfqpoint{3.075004in}{2.068985in}}%
\pgfpathmoveto{\pgfqpoint{3.070463in}{2.071934in}}%
\pgfpathlineto{\pgfqpoint{3.070463in}{2.071934in}}%
\pgfpathlineto{\pgfqpoint{3.070463in}{2.074884in}}%
\pgfpathlineto{\pgfqpoint{3.075004in}{2.074884in}}%
\pgfpathlineto{\pgfqpoint{3.075004in}{2.071934in}}%
\pgfpathmoveto{\pgfqpoint{3.070463in}{2.074884in}}%
\pgfpathlineto{\pgfqpoint{3.070463in}{2.074884in}}%
\pgfpathlineto{\pgfqpoint{3.070463in}{2.077833in}}%
\pgfpathlineto{\pgfqpoint{3.075004in}{2.077833in}}%
\pgfpathlineto{\pgfqpoint{3.075004in}{2.074884in}}%
\pgfpathmoveto{\pgfqpoint{3.070463in}{2.077833in}}%
\pgfpathlineto{\pgfqpoint{3.070463in}{2.077833in}}%
\pgfpathlineto{\pgfqpoint{3.070463in}{2.080783in}}%
\pgfpathlineto{\pgfqpoint{3.075004in}{2.080783in}}%
\pgfpathlineto{\pgfqpoint{3.075004in}{2.077833in}}%
\pgfpathmoveto{\pgfqpoint{3.070463in}{2.080783in}}%
\pgfpathlineto{\pgfqpoint{3.070463in}{2.080783in}}%
\pgfpathlineto{\pgfqpoint{3.070463in}{2.083732in}}%
\pgfpathlineto{\pgfqpoint{3.075004in}{2.083732in}}%
\pgfpathlineto{\pgfqpoint{3.075004in}{2.080783in}}%
\pgfpathmoveto{\pgfqpoint{3.070463in}{2.083732in}}%
\pgfpathlineto{\pgfqpoint{3.070463in}{2.083732in}}%
\pgfpathlineto{\pgfqpoint{3.070463in}{2.086681in}}%
\pgfpathlineto{\pgfqpoint{3.075004in}{2.086681in}}%
\pgfpathlineto{\pgfqpoint{3.075004in}{2.083732in}}%
\pgfpathmoveto{\pgfqpoint{3.070463in}{2.086681in}}%
\pgfpathlineto{\pgfqpoint{3.070463in}{2.086681in}}%
\pgfpathlineto{\pgfqpoint{3.070463in}{2.089631in}}%
\pgfpathlineto{\pgfqpoint{3.075004in}{2.089631in}}%
\pgfpathlineto{\pgfqpoint{3.075004in}{2.086681in}}%
\pgfpathmoveto{\pgfqpoint{3.070463in}{2.089631in}}%
\pgfpathlineto{\pgfqpoint{3.070463in}{2.089631in}}%
\pgfpathlineto{\pgfqpoint{3.070463in}{2.092580in}}%
\pgfpathlineto{\pgfqpoint{3.075004in}{2.092580in}}%
\pgfpathlineto{\pgfqpoint{3.075004in}{2.089631in}}%
\pgfpathmoveto{\pgfqpoint{3.070463in}{2.092580in}}%
\pgfpathlineto{\pgfqpoint{3.070463in}{2.092580in}}%
\pgfpathlineto{\pgfqpoint{3.070463in}{2.095529in}}%
\pgfpathlineto{\pgfqpoint{3.075004in}{2.095529in}}%
\pgfpathlineto{\pgfqpoint{3.075004in}{2.092580in}}%
\pgfpathmoveto{\pgfqpoint{3.070463in}{2.095529in}}%
\pgfpathlineto{\pgfqpoint{3.070463in}{2.095529in}}%
\pgfpathlineto{\pgfqpoint{3.070463in}{2.098479in}}%
\pgfpathlineto{\pgfqpoint{3.075004in}{2.098479in}}%
\pgfpathlineto{\pgfqpoint{3.075004in}{2.095529in}}%
\pgfpathmoveto{\pgfqpoint{3.070463in}{2.098479in}}%
\pgfpathlineto{\pgfqpoint{3.070463in}{2.098479in}}%
\pgfpathlineto{\pgfqpoint{3.070463in}{2.101428in}}%
\pgfpathlineto{\pgfqpoint{3.075004in}{2.101428in}}%
\pgfpathlineto{\pgfqpoint{3.075004in}{2.098479in}}%
\pgfpathmoveto{\pgfqpoint{3.070463in}{2.101428in}}%
\pgfpathlineto{\pgfqpoint{3.070463in}{2.101428in}}%
\pgfpathlineto{\pgfqpoint{3.070463in}{2.104378in}}%
\pgfpathlineto{\pgfqpoint{3.075004in}{2.104378in}}%
\pgfpathlineto{\pgfqpoint{3.075004in}{2.101428in}}%
\pgfpathmoveto{\pgfqpoint{3.070463in}{2.104378in}}%
\pgfpathlineto{\pgfqpoint{3.070463in}{2.104378in}}%
\pgfpathlineto{\pgfqpoint{3.070463in}{2.107327in}}%
\pgfpathlineto{\pgfqpoint{3.075004in}{2.107327in}}%
\pgfpathlineto{\pgfqpoint{3.075004in}{2.104378in}}%
\pgfpathmoveto{\pgfqpoint{3.070463in}{2.107327in}}%
\pgfpathlineto{\pgfqpoint{3.070463in}{2.107327in}}%
\pgfpathlineto{\pgfqpoint{3.070463in}{2.110276in}}%
\pgfpathlineto{\pgfqpoint{3.075004in}{2.110276in}}%
\pgfpathlineto{\pgfqpoint{3.075004in}{2.107327in}}%
\pgfpathmoveto{\pgfqpoint{3.070463in}{2.110276in}}%
\pgfpathlineto{\pgfqpoint{3.070463in}{2.110276in}}%
\pgfpathlineto{\pgfqpoint{3.070463in}{2.113225in}}%
\pgfpathlineto{\pgfqpoint{3.075004in}{2.113225in}}%
\pgfpathlineto{\pgfqpoint{3.075004in}{2.110276in}}%
\pgfpathmoveto{\pgfqpoint{3.070463in}{2.113225in}}%
\pgfpathlineto{\pgfqpoint{3.070463in}{2.113225in}}%
\pgfpathlineto{\pgfqpoint{3.070463in}{2.116174in}}%
\pgfpathlineto{\pgfqpoint{3.075004in}{2.116174in}}%
\pgfpathlineto{\pgfqpoint{3.075004in}{2.113225in}}%
\pgfpathmoveto{\pgfqpoint{3.070463in}{2.116174in}}%
\pgfpathlineto{\pgfqpoint{3.070463in}{2.116174in}}%
\pgfpathlineto{\pgfqpoint{3.070463in}{2.119123in}}%
\pgfpathlineto{\pgfqpoint{3.075004in}{2.119123in}}%
\pgfpathlineto{\pgfqpoint{3.075004in}{2.116174in}}%
\pgfpathmoveto{\pgfqpoint{3.070463in}{2.119123in}}%
\pgfpathlineto{\pgfqpoint{3.070463in}{2.119123in}}%
\pgfpathlineto{\pgfqpoint{3.070463in}{2.122072in}}%
\pgfpathlineto{\pgfqpoint{3.075004in}{2.122072in}}%
\pgfpathlineto{\pgfqpoint{3.075004in}{2.119123in}}%
\pgfpathmoveto{\pgfqpoint{3.070463in}{2.122072in}}%
\pgfpathlineto{\pgfqpoint{3.070463in}{2.122072in}}%
\pgfpathlineto{\pgfqpoint{3.070463in}{2.125021in}}%
\pgfpathlineto{\pgfqpoint{3.075004in}{2.125021in}}%
\pgfpathlineto{\pgfqpoint{3.075004in}{2.122072in}}%
\pgfpathmoveto{\pgfqpoint{3.070463in}{2.125021in}}%
\pgfpathlineto{\pgfqpoint{3.070463in}{2.125021in}}%
\pgfpathlineto{\pgfqpoint{3.070463in}{2.127970in}}%
\pgfpathlineto{\pgfqpoint{3.075004in}{2.127970in}}%
\pgfpathlineto{\pgfqpoint{3.075004in}{2.125021in}}%
\pgfpathmoveto{\pgfqpoint{3.070463in}{2.127970in}}%
\pgfpathlineto{\pgfqpoint{3.070463in}{2.127970in}}%
\pgfpathlineto{\pgfqpoint{3.070463in}{2.130919in}}%
\pgfpathlineto{\pgfqpoint{3.075004in}{2.130919in}}%
\pgfpathlineto{\pgfqpoint{3.075004in}{2.127970in}}%
\pgfpathmoveto{\pgfqpoint{3.070463in}{2.130919in}}%
\pgfpathlineto{\pgfqpoint{3.070463in}{2.130919in}}%
\pgfpathlineto{\pgfqpoint{3.070463in}{2.133868in}}%
\pgfpathlineto{\pgfqpoint{3.075004in}{2.133868in}}%
\pgfpathlineto{\pgfqpoint{3.075004in}{2.130919in}}%
\pgfpathmoveto{\pgfqpoint{3.070463in}{2.133868in}}%
\pgfpathlineto{\pgfqpoint{3.070463in}{2.133868in}}%
\pgfpathlineto{\pgfqpoint{3.070463in}{2.136817in}}%
\pgfpathlineto{\pgfqpoint{3.075004in}{2.136817in}}%
\pgfpathlineto{\pgfqpoint{3.075004in}{2.133868in}}%
\pgfpathmoveto{\pgfqpoint{3.070463in}{2.136817in}}%
\pgfpathlineto{\pgfqpoint{3.070463in}{2.136817in}}%
\pgfpathlineto{\pgfqpoint{3.070463in}{2.139766in}}%
\pgfpathlineto{\pgfqpoint{3.075004in}{2.139766in}}%
\pgfpathlineto{\pgfqpoint{3.075004in}{2.136817in}}%
\pgfpathmoveto{\pgfqpoint{3.070463in}{2.139766in}}%
\pgfpathlineto{\pgfqpoint{3.070463in}{2.139766in}}%
\pgfpathlineto{\pgfqpoint{3.070463in}{2.142715in}}%
\pgfpathlineto{\pgfqpoint{3.075004in}{2.142715in}}%
\pgfpathlineto{\pgfqpoint{3.075004in}{2.139766in}}%
\pgfpathmoveto{\pgfqpoint{3.070463in}{2.142715in}}%
\pgfpathlineto{\pgfqpoint{3.070463in}{2.142715in}}%
\pgfpathlineto{\pgfqpoint{3.070463in}{2.145665in}}%
\pgfpathlineto{\pgfqpoint{3.075004in}{2.145665in}}%
\pgfpathlineto{\pgfqpoint{3.075004in}{2.142715in}}%
\pgfpathmoveto{\pgfqpoint{3.070463in}{2.145665in}}%
\pgfpathlineto{\pgfqpoint{3.070463in}{2.145665in}}%
\pgfpathlineto{\pgfqpoint{3.070463in}{2.148614in}}%
\pgfpathlineto{\pgfqpoint{3.075004in}{2.148614in}}%
\pgfpathlineto{\pgfqpoint{3.075004in}{2.145665in}}%
\pgfpathmoveto{\pgfqpoint{3.070463in}{2.148614in}}%
\pgfpathlineto{\pgfqpoint{3.070463in}{2.148614in}}%
\pgfpathlineto{\pgfqpoint{3.070463in}{2.151563in}}%
\pgfpathlineto{\pgfqpoint{3.075004in}{2.151563in}}%
\pgfpathlineto{\pgfqpoint{3.075004in}{2.148614in}}%
\pgfpathmoveto{\pgfqpoint{3.070463in}{2.151563in}}%
\pgfpathlineto{\pgfqpoint{3.070463in}{2.151563in}}%
\pgfpathlineto{\pgfqpoint{3.070463in}{2.154512in}}%
\pgfpathlineto{\pgfqpoint{3.075004in}{2.154512in}}%
\pgfpathlineto{\pgfqpoint{3.075004in}{2.151563in}}%
\pgfpathmoveto{\pgfqpoint{3.070463in}{2.154512in}}%
\pgfpathlineto{\pgfqpoint{3.070463in}{2.154512in}}%
\pgfpathlineto{\pgfqpoint{3.070463in}{2.157461in}}%
\pgfpathlineto{\pgfqpoint{3.075004in}{2.157461in}}%
\pgfpathlineto{\pgfqpoint{3.075004in}{2.154512in}}%
\pgfpathmoveto{\pgfqpoint{3.070463in}{2.157461in}}%
\pgfpathlineto{\pgfqpoint{3.070463in}{2.157461in}}%
\pgfpathlineto{\pgfqpoint{3.070463in}{2.160410in}}%
\pgfpathlineto{\pgfqpoint{3.075004in}{2.160410in}}%
\pgfpathlineto{\pgfqpoint{3.075004in}{2.157461in}}%
\pgfpathmoveto{\pgfqpoint{3.070463in}{2.160410in}}%
\pgfpathlineto{\pgfqpoint{3.070463in}{2.160410in}}%
\pgfpathlineto{\pgfqpoint{3.070463in}{2.163359in}}%
\pgfpathlineto{\pgfqpoint{3.075004in}{2.163359in}}%
\pgfpathlineto{\pgfqpoint{3.075004in}{2.160410in}}%
\pgfpathmoveto{\pgfqpoint{3.070463in}{2.163359in}}%
\pgfpathlineto{\pgfqpoint{3.070463in}{2.163359in}}%
\pgfpathlineto{\pgfqpoint{3.070463in}{2.166308in}}%
\pgfpathlineto{\pgfqpoint{3.075004in}{2.166308in}}%
\pgfpathlineto{\pgfqpoint{3.075004in}{2.163359in}}%
\pgfpathmoveto{\pgfqpoint{3.070463in}{2.166308in}}%
\pgfpathlineto{\pgfqpoint{3.070463in}{2.166308in}}%
\pgfpathlineto{\pgfqpoint{3.070463in}{2.169257in}}%
\pgfpathlineto{\pgfqpoint{3.075004in}{2.169257in}}%
\pgfpathlineto{\pgfqpoint{3.075004in}{2.166308in}}%
\pgfpathmoveto{\pgfqpoint{3.070463in}{2.169257in}}%
\pgfpathlineto{\pgfqpoint{3.070463in}{2.169257in}}%
\pgfpathlineto{\pgfqpoint{3.070463in}{2.172206in}}%
\pgfpathlineto{\pgfqpoint{3.075004in}{2.172206in}}%
\pgfpathlineto{\pgfqpoint{3.075004in}{2.169257in}}%
\pgfpathmoveto{\pgfqpoint{3.070463in}{2.172206in}}%
\pgfpathlineto{\pgfqpoint{3.070463in}{2.172206in}}%
\pgfpathlineto{\pgfqpoint{3.070463in}{2.175155in}}%
\pgfpathlineto{\pgfqpoint{3.075004in}{2.175155in}}%
\pgfpathlineto{\pgfqpoint{3.075004in}{2.172206in}}%
\pgfpathmoveto{\pgfqpoint{3.070463in}{2.175155in}}%
\pgfpathlineto{\pgfqpoint{3.070463in}{2.175155in}}%
\pgfpathlineto{\pgfqpoint{3.070463in}{2.178104in}}%
\pgfpathlineto{\pgfqpoint{3.075004in}{2.178104in}}%
\pgfpathlineto{\pgfqpoint{3.075004in}{2.175155in}}%
\pgfpathmoveto{\pgfqpoint{3.070463in}{2.178104in}}%
\pgfpathlineto{\pgfqpoint{3.070463in}{2.178104in}}%
\pgfpathlineto{\pgfqpoint{3.070463in}{2.181053in}}%
\pgfpathlineto{\pgfqpoint{3.075004in}{2.181053in}}%
\pgfpathlineto{\pgfqpoint{3.075004in}{2.178104in}}%
\pgfpathmoveto{\pgfqpoint{3.070463in}{2.181053in}}%
\pgfpathlineto{\pgfqpoint{3.070463in}{2.181053in}}%
\pgfpathlineto{\pgfqpoint{3.070463in}{2.184002in}}%
\pgfpathlineto{\pgfqpoint{3.075004in}{2.184002in}}%
\pgfpathlineto{\pgfqpoint{3.075004in}{2.181053in}}%
\pgfpathmoveto{\pgfqpoint{3.070463in}{2.184002in}}%
\pgfpathlineto{\pgfqpoint{3.070463in}{2.184002in}}%
\pgfpathlineto{\pgfqpoint{3.070463in}{2.186952in}}%
\pgfpathlineto{\pgfqpoint{3.075004in}{2.186952in}}%
\pgfpathlineto{\pgfqpoint{3.075004in}{2.184002in}}%
\pgfpathmoveto{\pgfqpoint{3.070463in}{2.186952in}}%
\pgfpathlineto{\pgfqpoint{3.070463in}{2.186952in}}%
\pgfpathlineto{\pgfqpoint{3.070463in}{2.189901in}}%
\pgfpathlineto{\pgfqpoint{3.075004in}{2.189901in}}%
\pgfpathlineto{\pgfqpoint{3.075004in}{2.186952in}}%
\pgfpathmoveto{\pgfqpoint{3.070463in}{2.189901in}}%
\pgfpathlineto{\pgfqpoint{3.070463in}{2.189901in}}%
\pgfpathlineto{\pgfqpoint{3.070463in}{2.192850in}}%
\pgfpathlineto{\pgfqpoint{3.075004in}{2.192850in}}%
\pgfpathlineto{\pgfqpoint{3.075004in}{2.189901in}}%
\pgfpathmoveto{\pgfqpoint{3.070463in}{2.192850in}}%
\pgfpathlineto{\pgfqpoint{3.070463in}{2.192850in}}%
\pgfpathlineto{\pgfqpoint{3.070463in}{2.195799in}}%
\pgfpathlineto{\pgfqpoint{3.075004in}{2.195799in}}%
\pgfpathlineto{\pgfqpoint{3.075004in}{2.192850in}}%
\pgfpathmoveto{\pgfqpoint{3.070463in}{2.195799in}}%
\pgfpathlineto{\pgfqpoint{3.070463in}{2.195799in}}%
\pgfpathlineto{\pgfqpoint{3.070463in}{2.198748in}}%
\pgfpathlineto{\pgfqpoint{3.075004in}{2.198748in}}%
\pgfpathlineto{\pgfqpoint{3.075004in}{2.195799in}}%
\pgfpathmoveto{\pgfqpoint{3.070463in}{2.198748in}}%
\pgfpathlineto{\pgfqpoint{3.070463in}{2.198748in}}%
\pgfpathlineto{\pgfqpoint{3.070463in}{2.201697in}}%
\pgfpathlineto{\pgfqpoint{3.075004in}{2.201697in}}%
\pgfpathlineto{\pgfqpoint{3.075004in}{2.198748in}}%
\pgfpathmoveto{\pgfqpoint{3.070463in}{2.201697in}}%
\pgfpathlineto{\pgfqpoint{3.070463in}{2.201697in}}%
\pgfpathlineto{\pgfqpoint{3.070463in}{2.204646in}}%
\pgfpathlineto{\pgfqpoint{3.075004in}{2.204646in}}%
\pgfpathlineto{\pgfqpoint{3.075004in}{2.201697in}}%
\pgfpathmoveto{\pgfqpoint{3.070463in}{2.204646in}}%
\pgfpathlineto{\pgfqpoint{3.070463in}{2.204646in}}%
\pgfpathlineto{\pgfqpoint{3.070463in}{2.207596in}}%
\pgfpathlineto{\pgfqpoint{3.075004in}{2.207596in}}%
\pgfpathlineto{\pgfqpoint{3.075004in}{2.204646in}}%
\pgfpathmoveto{\pgfqpoint{3.070463in}{2.207596in}}%
\pgfpathlineto{\pgfqpoint{3.070463in}{2.207596in}}%
\pgfpathlineto{\pgfqpoint{3.070463in}{2.210545in}}%
\pgfpathlineto{\pgfqpoint{3.075004in}{2.210545in}}%
\pgfpathlineto{\pgfqpoint{3.075004in}{2.207596in}}%
\pgfpathmoveto{\pgfqpoint{3.070463in}{2.210545in}}%
\pgfpathlineto{\pgfqpoint{3.070463in}{2.210545in}}%
\pgfpathlineto{\pgfqpoint{3.070463in}{2.213494in}}%
\pgfpathlineto{\pgfqpoint{3.075004in}{2.213494in}}%
\pgfpathlineto{\pgfqpoint{3.075004in}{2.210545in}}%
\pgfpathmoveto{\pgfqpoint{3.070463in}{2.213494in}}%
\pgfpathlineto{\pgfqpoint{3.070463in}{2.213494in}}%
\pgfpathlineto{\pgfqpoint{3.070463in}{2.216444in}}%
\pgfpathlineto{\pgfqpoint{3.075004in}{2.216444in}}%
\pgfpathlineto{\pgfqpoint{3.075004in}{2.213494in}}%
\pgfpathmoveto{\pgfqpoint{3.070463in}{2.216444in}}%
\pgfpathlineto{\pgfqpoint{3.070463in}{2.216444in}}%
\pgfpathlineto{\pgfqpoint{3.070463in}{2.219393in}}%
\pgfpathlineto{\pgfqpoint{3.075004in}{2.219393in}}%
\pgfpathlineto{\pgfqpoint{3.075004in}{2.216444in}}%
\pgfpathmoveto{\pgfqpoint{3.070463in}{2.219393in}}%
\pgfpathlineto{\pgfqpoint{3.070463in}{2.219393in}}%
\pgfpathlineto{\pgfqpoint{3.070463in}{2.222342in}}%
\pgfpathlineto{\pgfqpoint{3.075004in}{2.222342in}}%
\pgfpathlineto{\pgfqpoint{3.075004in}{2.219393in}}%
\pgfpathmoveto{\pgfqpoint{3.070463in}{2.222342in}}%
\pgfpathlineto{\pgfqpoint{3.070463in}{2.222342in}}%
\pgfpathlineto{\pgfqpoint{3.070463in}{2.225291in}}%
\pgfpathlineto{\pgfqpoint{3.075004in}{2.225291in}}%
\pgfpathlineto{\pgfqpoint{3.075004in}{2.222342in}}%
\pgfpathmoveto{\pgfqpoint{3.070463in}{2.225291in}}%
\pgfpathlineto{\pgfqpoint{3.070463in}{2.225291in}}%
\pgfpathlineto{\pgfqpoint{3.070463in}{2.228241in}}%
\pgfpathlineto{\pgfqpoint{3.075004in}{2.228241in}}%
\pgfpathlineto{\pgfqpoint{3.075004in}{2.225291in}}%
\pgfpathmoveto{\pgfqpoint{3.070463in}{2.228241in}}%
\pgfpathlineto{\pgfqpoint{3.070463in}{2.228241in}}%
\pgfpathlineto{\pgfqpoint{3.070463in}{2.231190in}}%
\pgfpathlineto{\pgfqpoint{3.075004in}{2.231190in}}%
\pgfpathlineto{\pgfqpoint{3.075004in}{2.228241in}}%
\pgfpathmoveto{\pgfqpoint{3.070463in}{2.231190in}}%
\pgfpathlineto{\pgfqpoint{3.070463in}{2.231190in}}%
\pgfpathlineto{\pgfqpoint{3.070463in}{2.234139in}}%
\pgfpathlineto{\pgfqpoint{3.075004in}{2.234139in}}%
\pgfpathlineto{\pgfqpoint{3.075004in}{2.231190in}}%
\pgfpathmoveto{\pgfqpoint{3.070463in}{2.234139in}}%
\pgfpathlineto{\pgfqpoint{3.070463in}{2.234139in}}%
\pgfpathlineto{\pgfqpoint{3.070463in}{2.237088in}}%
\pgfpathlineto{\pgfqpoint{3.075004in}{2.237088in}}%
\pgfpathlineto{\pgfqpoint{3.075004in}{2.234139in}}%
\pgfpathmoveto{\pgfqpoint{3.070463in}{2.237088in}}%
\pgfpathlineto{\pgfqpoint{3.070463in}{2.237088in}}%
\pgfpathlineto{\pgfqpoint{3.070463in}{2.240038in}}%
\pgfpathlineto{\pgfqpoint{3.075004in}{2.240038in}}%
\pgfpathlineto{\pgfqpoint{3.075004in}{2.237088in}}%
\pgfpathmoveto{\pgfqpoint{3.070463in}{2.240038in}}%
\pgfpathlineto{\pgfqpoint{3.070463in}{2.240038in}}%
\pgfpathlineto{\pgfqpoint{3.070463in}{2.242987in}}%
\pgfpathlineto{\pgfqpoint{3.075004in}{2.242987in}}%
\pgfpathlineto{\pgfqpoint{3.075004in}{2.240038in}}%
\pgfpathmoveto{\pgfqpoint{3.070463in}{2.242987in}}%
\pgfpathlineto{\pgfqpoint{3.070463in}{2.242987in}}%
\pgfpathlineto{\pgfqpoint{3.070463in}{2.245936in}}%
\pgfpathlineto{\pgfqpoint{3.075004in}{2.245936in}}%
\pgfpathlineto{\pgfqpoint{3.075004in}{2.242987in}}%
\pgfpathmoveto{\pgfqpoint{3.070463in}{2.245936in}}%
\pgfpathlineto{\pgfqpoint{3.070463in}{2.245936in}}%
\pgfpathlineto{\pgfqpoint{3.070463in}{2.248886in}}%
\pgfpathlineto{\pgfqpoint{3.075004in}{2.248886in}}%
\pgfpathlineto{\pgfqpoint{3.075004in}{2.245936in}}%
\pgfpathmoveto{\pgfqpoint{3.070463in}{2.248886in}}%
\pgfpathlineto{\pgfqpoint{3.070463in}{2.248886in}}%
\pgfpathlineto{\pgfqpoint{3.070463in}{2.251835in}}%
\pgfpathlineto{\pgfqpoint{3.075004in}{2.251835in}}%
\pgfpathlineto{\pgfqpoint{3.075004in}{2.248886in}}%
\pgfpathmoveto{\pgfqpoint{3.070463in}{2.251835in}}%
\pgfpathlineto{\pgfqpoint{3.070463in}{2.251835in}}%
\pgfpathlineto{\pgfqpoint{3.070463in}{2.254784in}}%
\pgfpathlineto{\pgfqpoint{3.075004in}{2.254784in}}%
\pgfpathlineto{\pgfqpoint{3.075004in}{2.251835in}}%
\pgfpathmoveto{\pgfqpoint{3.070463in}{2.254784in}}%
\pgfpathlineto{\pgfqpoint{3.070463in}{2.254784in}}%
\pgfpathlineto{\pgfqpoint{3.070463in}{2.257733in}}%
\pgfpathlineto{\pgfqpoint{3.075004in}{2.257733in}}%
\pgfpathlineto{\pgfqpoint{3.075004in}{2.254784in}}%
\pgfpathmoveto{\pgfqpoint{3.070463in}{2.257733in}}%
\pgfpathlineto{\pgfqpoint{3.070463in}{2.257733in}}%
\pgfpathlineto{\pgfqpoint{3.070463in}{2.260683in}}%
\pgfpathlineto{\pgfqpoint{3.075004in}{2.260683in}}%
\pgfpathlineto{\pgfqpoint{3.075004in}{2.257733in}}%
\pgfpathmoveto{\pgfqpoint{3.070463in}{2.260683in}}%
\pgfpathlineto{\pgfqpoint{3.070463in}{2.260683in}}%
\pgfpathlineto{\pgfqpoint{3.070463in}{2.263632in}}%
\pgfpathlineto{\pgfqpoint{3.075004in}{2.263632in}}%
\pgfpathlineto{\pgfqpoint{3.075004in}{2.260683in}}%
\pgfpathmoveto{\pgfqpoint{3.070463in}{2.263632in}}%
\pgfpathlineto{\pgfqpoint{3.070463in}{2.263632in}}%
\pgfpathlineto{\pgfqpoint{3.070463in}{2.266581in}}%
\pgfpathlineto{\pgfqpoint{3.075004in}{2.266581in}}%
\pgfpathlineto{\pgfqpoint{3.075004in}{2.263632in}}%
\pgfpathmoveto{\pgfqpoint{3.070463in}{2.266581in}}%
\pgfpathlineto{\pgfqpoint{3.070463in}{2.266581in}}%
\pgfpathlineto{\pgfqpoint{3.070463in}{2.269531in}}%
\pgfpathlineto{\pgfqpoint{3.075004in}{2.269531in}}%
\pgfpathlineto{\pgfqpoint{3.075004in}{2.266581in}}%
\pgfpathmoveto{\pgfqpoint{3.070463in}{2.269531in}}%
\pgfpathlineto{\pgfqpoint{3.070463in}{2.269531in}}%
\pgfpathlineto{\pgfqpoint{3.070463in}{2.272480in}}%
\pgfpathlineto{\pgfqpoint{3.075004in}{2.272480in}}%
\pgfpathlineto{\pgfqpoint{3.075004in}{2.269531in}}%
\pgfpathmoveto{\pgfqpoint{3.070463in}{2.272480in}}%
\pgfpathlineto{\pgfqpoint{3.070463in}{2.272480in}}%
\pgfpathlineto{\pgfqpoint{3.070463in}{2.275429in}}%
\pgfpathlineto{\pgfqpoint{3.075004in}{2.275429in}}%
\pgfpathlineto{\pgfqpoint{3.075004in}{2.272480in}}%
\pgfpathmoveto{\pgfqpoint{3.070463in}{2.275429in}}%
\pgfpathlineto{\pgfqpoint{3.070463in}{2.275429in}}%
\pgfpathlineto{\pgfqpoint{3.070463in}{2.278378in}}%
\pgfpathlineto{\pgfqpoint{3.075004in}{2.278378in}}%
\pgfpathlineto{\pgfqpoint{3.075004in}{2.275429in}}%
\pgfpathmoveto{\pgfqpoint{3.070463in}{2.278378in}}%
\pgfpathlineto{\pgfqpoint{3.070463in}{2.278378in}}%
\pgfpathlineto{\pgfqpoint{3.070463in}{2.281328in}}%
\pgfpathlineto{\pgfqpoint{3.075004in}{2.281328in}}%
\pgfpathlineto{\pgfqpoint{3.075004in}{2.278378in}}%
\pgfpathmoveto{\pgfqpoint{3.070463in}{2.281328in}}%
\pgfpathlineto{\pgfqpoint{3.070463in}{2.281328in}}%
\pgfpathlineto{\pgfqpoint{3.070463in}{2.284277in}}%
\pgfpathlineto{\pgfqpoint{3.075004in}{2.284277in}}%
\pgfpathlineto{\pgfqpoint{3.075004in}{2.281328in}}%
\pgfpathmoveto{\pgfqpoint{3.070463in}{2.284277in}}%
\pgfpathlineto{\pgfqpoint{3.070463in}{2.284277in}}%
\pgfpathlineto{\pgfqpoint{3.070463in}{2.287226in}}%
\pgfpathlineto{\pgfqpoint{3.075004in}{2.287226in}}%
\pgfpathlineto{\pgfqpoint{3.075004in}{2.284277in}}%
\pgfpathmoveto{\pgfqpoint{3.070463in}{2.287226in}}%
\pgfpathlineto{\pgfqpoint{3.070463in}{2.287226in}}%
\pgfpathlineto{\pgfqpoint{3.070463in}{2.290175in}}%
\pgfpathlineto{\pgfqpoint{3.075004in}{2.290175in}}%
\pgfpathlineto{\pgfqpoint{3.075004in}{2.287226in}}%
\pgfpathmoveto{\pgfqpoint{3.070463in}{2.290175in}}%
\pgfpathlineto{\pgfqpoint{3.070463in}{2.290175in}}%
\pgfpathlineto{\pgfqpoint{3.070463in}{2.293125in}}%
\pgfpathlineto{\pgfqpoint{3.075004in}{2.293125in}}%
\pgfpathlineto{\pgfqpoint{3.075004in}{2.290175in}}%
\pgfpathmoveto{\pgfqpoint{3.070463in}{2.293125in}}%
\pgfpathlineto{\pgfqpoint{3.070463in}{2.293125in}}%
\pgfpathlineto{\pgfqpoint{3.070463in}{2.296074in}}%
\pgfpathlineto{\pgfqpoint{3.075004in}{2.296074in}}%
\pgfpathlineto{\pgfqpoint{3.075004in}{2.293125in}}%
\pgfpathmoveto{\pgfqpoint{3.070463in}{2.296074in}}%
\pgfpathlineto{\pgfqpoint{3.070463in}{2.296074in}}%
\pgfpathlineto{\pgfqpoint{3.070463in}{2.299023in}}%
\pgfpathlineto{\pgfqpoint{3.075004in}{2.299023in}}%
\pgfpathlineto{\pgfqpoint{3.075004in}{2.296074in}}%
\pgfpathmoveto{\pgfqpoint{3.070463in}{2.299023in}}%
\pgfpathlineto{\pgfqpoint{3.070463in}{2.299023in}}%
\pgfpathlineto{\pgfqpoint{3.070463in}{2.301973in}}%
\pgfpathlineto{\pgfqpoint{3.075004in}{2.301973in}}%
\pgfpathlineto{\pgfqpoint{3.075004in}{2.299023in}}%
\pgfpathmoveto{\pgfqpoint{3.070463in}{2.301973in}}%
\pgfpathlineto{\pgfqpoint{3.070463in}{2.301973in}}%
\pgfpathlineto{\pgfqpoint{3.070463in}{2.304922in}}%
\pgfpathlineto{\pgfqpoint{3.075004in}{2.304922in}}%
\pgfpathlineto{\pgfqpoint{3.075004in}{2.301973in}}%
\pgfpathmoveto{\pgfqpoint{3.070463in}{2.304922in}}%
\pgfpathlineto{\pgfqpoint{3.070463in}{2.304922in}}%
\pgfpathlineto{\pgfqpoint{3.070463in}{2.307871in}}%
\pgfpathlineto{\pgfqpoint{3.075004in}{2.307871in}}%
\pgfpathlineto{\pgfqpoint{3.075004in}{2.304922in}}%
\pgfpathmoveto{\pgfqpoint{3.070463in}{2.307871in}}%
\pgfpathlineto{\pgfqpoint{3.070463in}{2.307871in}}%
\pgfpathlineto{\pgfqpoint{3.070463in}{2.310820in}}%
\pgfpathlineto{\pgfqpoint{3.075004in}{2.310820in}}%
\pgfpathlineto{\pgfqpoint{3.075004in}{2.307871in}}%
\pgfpathmoveto{\pgfqpoint{3.070463in}{2.310820in}}%
\pgfpathlineto{\pgfqpoint{3.070463in}{2.310820in}}%
\pgfpathlineto{\pgfqpoint{3.070463in}{2.313770in}}%
\pgfpathlineto{\pgfqpoint{3.075004in}{2.313770in}}%
\pgfpathlineto{\pgfqpoint{3.075004in}{2.310820in}}%
\pgfpathmoveto{\pgfqpoint{3.070463in}{2.313770in}}%
\pgfpathlineto{\pgfqpoint{3.070463in}{2.313770in}}%
\pgfpathlineto{\pgfqpoint{3.070463in}{2.316719in}}%
\pgfpathlineto{\pgfqpoint{3.075004in}{2.316719in}}%
\pgfpathlineto{\pgfqpoint{3.075004in}{2.313770in}}%
\pgfpathmoveto{\pgfqpoint{3.070463in}{2.316719in}}%
\pgfpathlineto{\pgfqpoint{3.070463in}{2.316719in}}%
\pgfpathlineto{\pgfqpoint{3.070463in}{2.319668in}}%
\pgfpathlineto{\pgfqpoint{3.075004in}{2.319668in}}%
\pgfpathlineto{\pgfqpoint{3.075004in}{2.316719in}}%
\pgfpathmoveto{\pgfqpoint{3.070463in}{2.319668in}}%
\pgfpathlineto{\pgfqpoint{3.070463in}{2.319668in}}%
\pgfpathlineto{\pgfqpoint{3.070463in}{2.322618in}}%
\pgfpathlineto{\pgfqpoint{3.075004in}{2.322618in}}%
\pgfpathlineto{\pgfqpoint{3.075004in}{2.319668in}}%
\pgfpathmoveto{\pgfqpoint{3.070463in}{2.322618in}}%
\pgfpathlineto{\pgfqpoint{3.070463in}{2.322618in}}%
\pgfpathlineto{\pgfqpoint{3.070463in}{2.325567in}}%
\pgfpathlineto{\pgfqpoint{3.075004in}{2.325567in}}%
\pgfpathlineto{\pgfqpoint{3.075004in}{2.322618in}}%
\pgfpathmoveto{\pgfqpoint{3.070463in}{2.325567in}}%
\pgfpathlineto{\pgfqpoint{3.070463in}{2.325567in}}%
\pgfpathlineto{\pgfqpoint{3.070463in}{2.328516in}}%
\pgfpathlineto{\pgfqpoint{3.075004in}{2.328516in}}%
\pgfpathlineto{\pgfqpoint{3.075004in}{2.325567in}}%
\pgfpathmoveto{\pgfqpoint{3.070463in}{2.328516in}}%
\pgfpathlineto{\pgfqpoint{3.070463in}{2.328516in}}%
\pgfpathlineto{\pgfqpoint{3.070463in}{2.331465in}}%
\pgfpathlineto{\pgfqpoint{3.075004in}{2.331465in}}%
\pgfpathlineto{\pgfqpoint{3.075004in}{2.328516in}}%
\pgfpathmoveto{\pgfqpoint{3.070463in}{2.331465in}}%
\pgfpathlineto{\pgfqpoint{3.070463in}{2.331465in}}%
\pgfpathlineto{\pgfqpoint{3.070463in}{2.334415in}}%
\pgfpathlineto{\pgfqpoint{3.075004in}{2.334415in}}%
\pgfpathlineto{\pgfqpoint{3.075004in}{2.331465in}}%
\pgfpathmoveto{\pgfqpoint{3.070463in}{2.334415in}}%
\pgfpathlineto{\pgfqpoint{3.070463in}{2.334415in}}%
\pgfpathlineto{\pgfqpoint{3.070463in}{2.337364in}}%
\pgfpathlineto{\pgfqpoint{3.075004in}{2.337364in}}%
\pgfpathlineto{\pgfqpoint{3.075004in}{2.334415in}}%
\pgfpathmoveto{\pgfqpoint{3.070463in}{2.337364in}}%
\pgfpathlineto{\pgfqpoint{3.070463in}{2.337364in}}%
\pgfpathlineto{\pgfqpoint{3.070463in}{2.340313in}}%
\pgfpathlineto{\pgfqpoint{3.075004in}{2.340313in}}%
\pgfpathlineto{\pgfqpoint{3.075004in}{2.337364in}}%
\pgfpathmoveto{\pgfqpoint{3.070463in}{2.340313in}}%
\pgfpathlineto{\pgfqpoint{3.070463in}{2.340313in}}%
\pgfpathlineto{\pgfqpoint{3.070463in}{2.343263in}}%
\pgfpathlineto{\pgfqpoint{3.075004in}{2.343263in}}%
\pgfpathlineto{\pgfqpoint{3.075004in}{2.340313in}}%
\pgfpathmoveto{\pgfqpoint{3.070463in}{2.343263in}}%
\pgfpathlineto{\pgfqpoint{3.070463in}{2.343263in}}%
\pgfpathlineto{\pgfqpoint{3.070463in}{2.346212in}}%
\pgfpathlineto{\pgfqpoint{3.075004in}{2.346212in}}%
\pgfpathlineto{\pgfqpoint{3.075004in}{2.343263in}}%
\pgfpathmoveto{\pgfqpoint{3.070463in}{2.346212in}}%
\pgfpathlineto{\pgfqpoint{3.070463in}{2.346212in}}%
\pgfpathlineto{\pgfqpoint{3.070463in}{2.349161in}}%
\pgfpathlineto{\pgfqpoint{3.075004in}{2.349161in}}%
\pgfpathlineto{\pgfqpoint{3.075004in}{2.346212in}}%
\pgfpathmoveto{\pgfqpoint{3.070463in}{2.349161in}}%
\pgfpathlineto{\pgfqpoint{3.070463in}{2.349161in}}%
\pgfpathlineto{\pgfqpoint{3.070463in}{2.352110in}}%
\pgfpathlineto{\pgfqpoint{3.075004in}{2.352110in}}%
\pgfpathlineto{\pgfqpoint{3.075004in}{2.349161in}}%
\pgfpathmoveto{\pgfqpoint{3.070463in}{2.352110in}}%
\pgfpathlineto{\pgfqpoint{3.070463in}{2.352110in}}%
\pgfpathlineto{\pgfqpoint{3.070463in}{2.355060in}}%
\pgfpathlineto{\pgfqpoint{3.075004in}{2.355060in}}%
\pgfpathlineto{\pgfqpoint{3.075004in}{2.352110in}}%
\pgfpathmoveto{\pgfqpoint{3.070463in}{2.355060in}}%
\pgfpathlineto{\pgfqpoint{3.070463in}{2.355060in}}%
\pgfpathlineto{\pgfqpoint{3.070463in}{2.358009in}}%
\pgfpathlineto{\pgfqpoint{3.075004in}{2.358009in}}%
\pgfpathlineto{\pgfqpoint{3.075004in}{2.355060in}}%
\pgfpathmoveto{\pgfqpoint{3.070463in}{2.358009in}}%
\pgfpathlineto{\pgfqpoint{3.070463in}{2.358009in}}%
\pgfpathlineto{\pgfqpoint{3.070463in}{2.360958in}}%
\pgfpathlineto{\pgfqpoint{3.075004in}{2.360958in}}%
\pgfpathlineto{\pgfqpoint{3.075004in}{2.358009in}}%
\pgfpathmoveto{\pgfqpoint{3.070463in}{2.360958in}}%
\pgfpathlineto{\pgfqpoint{3.070463in}{2.360958in}}%
\pgfpathlineto{\pgfqpoint{3.070463in}{2.363907in}}%
\pgfpathlineto{\pgfqpoint{3.075004in}{2.363907in}}%
\pgfpathlineto{\pgfqpoint{3.075004in}{2.360958in}}%
\pgfpathmoveto{\pgfqpoint{3.070463in}{2.363907in}}%
\pgfpathlineto{\pgfqpoint{3.070463in}{2.363907in}}%
\pgfpathlineto{\pgfqpoint{3.070463in}{2.366857in}}%
\pgfpathlineto{\pgfqpoint{3.075004in}{2.366857in}}%
\pgfpathlineto{\pgfqpoint{3.075004in}{2.363907in}}%
\pgfpathmoveto{\pgfqpoint{3.070463in}{2.366857in}}%
\pgfpathlineto{\pgfqpoint{3.070463in}{2.366857in}}%
\pgfpathlineto{\pgfqpoint{3.070463in}{2.369806in}}%
\pgfpathlineto{\pgfqpoint{3.075004in}{2.369806in}}%
\pgfpathlineto{\pgfqpoint{3.075004in}{2.366857in}}%
\pgfpathmoveto{\pgfqpoint{3.070463in}{2.369806in}}%
\pgfpathlineto{\pgfqpoint{3.070463in}{2.369806in}}%
\pgfpathlineto{\pgfqpoint{3.070463in}{2.372755in}}%
\pgfpathlineto{\pgfqpoint{3.075004in}{2.372755in}}%
\pgfpathlineto{\pgfqpoint{3.075004in}{2.369806in}}%
\pgfpathmoveto{\pgfqpoint{3.070463in}{2.372755in}}%
\pgfpathlineto{\pgfqpoint{3.070463in}{2.372755in}}%
\pgfpathlineto{\pgfqpoint{3.070463in}{2.375705in}}%
\pgfpathlineto{\pgfqpoint{3.075004in}{2.375705in}}%
\pgfpathlineto{\pgfqpoint{3.075004in}{2.372755in}}%
\pgfpathmoveto{\pgfqpoint{3.070463in}{2.375705in}}%
\pgfpathlineto{\pgfqpoint{3.070463in}{2.375705in}}%
\pgfpathlineto{\pgfqpoint{3.070463in}{2.378654in}}%
\pgfpathlineto{\pgfqpoint{3.075004in}{2.378654in}}%
\pgfpathlineto{\pgfqpoint{3.075004in}{2.375705in}}%
\pgfpathmoveto{\pgfqpoint{3.070463in}{2.378654in}}%
\pgfpathlineto{\pgfqpoint{3.070463in}{2.378654in}}%
\pgfpathlineto{\pgfqpoint{3.070463in}{2.381603in}}%
\pgfpathlineto{\pgfqpoint{3.075004in}{2.381603in}}%
\pgfpathlineto{\pgfqpoint{3.075004in}{2.378654in}}%
\pgfpathmoveto{\pgfqpoint{3.070463in}{2.381603in}}%
\pgfpathlineto{\pgfqpoint{3.070463in}{2.381603in}}%
\pgfpathlineto{\pgfqpoint{3.070463in}{2.384552in}}%
\pgfpathlineto{\pgfqpoint{3.075004in}{2.384552in}}%
\pgfpathlineto{\pgfqpoint{3.075004in}{2.381603in}}%
\pgfpathmoveto{\pgfqpoint{3.070463in}{2.384552in}}%
\pgfpathlineto{\pgfqpoint{3.070463in}{2.384552in}}%
\pgfpathlineto{\pgfqpoint{3.070463in}{2.387502in}}%
\pgfpathlineto{\pgfqpoint{3.075004in}{2.387502in}}%
\pgfpathlineto{\pgfqpoint{3.075004in}{2.384552in}}%
\pgfpathmoveto{\pgfqpoint{3.070463in}{2.387502in}}%
\pgfpathlineto{\pgfqpoint{3.070463in}{2.387502in}}%
\pgfpathlineto{\pgfqpoint{3.070463in}{2.390451in}}%
\pgfpathlineto{\pgfqpoint{3.075004in}{2.390451in}}%
\pgfpathlineto{\pgfqpoint{3.075004in}{2.387502in}}%
\pgfpathmoveto{\pgfqpoint{3.070463in}{2.390451in}}%
\pgfpathlineto{\pgfqpoint{3.070463in}{2.390451in}}%
\pgfpathlineto{\pgfqpoint{3.070463in}{2.393400in}}%
\pgfpathlineto{\pgfqpoint{3.075004in}{2.393400in}}%
\pgfpathlineto{\pgfqpoint{3.075004in}{2.390451in}}%
\pgfpathmoveto{\pgfqpoint{3.070463in}{2.393400in}}%
\pgfpathlineto{\pgfqpoint{3.070463in}{2.393400in}}%
\pgfpathlineto{\pgfqpoint{3.070463in}{2.396349in}}%
\pgfpathlineto{\pgfqpoint{3.075004in}{2.396349in}}%
\pgfpathlineto{\pgfqpoint{3.075004in}{2.393400in}}%
\pgfpathmoveto{\pgfqpoint{3.070463in}{2.396349in}}%
\pgfpathlineto{\pgfqpoint{3.070463in}{2.396349in}}%
\pgfpathlineto{\pgfqpoint{3.070463in}{2.399298in}}%
\pgfpathlineto{\pgfqpoint{3.075004in}{2.399298in}}%
\pgfpathlineto{\pgfqpoint{3.075004in}{2.396349in}}%
\pgfpathmoveto{\pgfqpoint{3.070463in}{2.399298in}}%
\pgfpathlineto{\pgfqpoint{3.070463in}{2.399298in}}%
\pgfpathlineto{\pgfqpoint{3.070463in}{2.402247in}}%
\pgfpathlineto{\pgfqpoint{3.075004in}{2.402247in}}%
\pgfpathlineto{\pgfqpoint{3.075004in}{2.399298in}}%
\pgfpathmoveto{\pgfqpoint{3.070463in}{2.402247in}}%
\pgfpathlineto{\pgfqpoint{3.070463in}{2.402247in}}%
\pgfpathlineto{\pgfqpoint{3.070463in}{2.405196in}}%
\pgfpathlineto{\pgfqpoint{3.075004in}{2.405196in}}%
\pgfpathlineto{\pgfqpoint{3.075004in}{2.402247in}}%
\pgfpathmoveto{\pgfqpoint{3.070463in}{2.405196in}}%
\pgfpathlineto{\pgfqpoint{3.070463in}{2.405196in}}%
\pgfpathlineto{\pgfqpoint{3.070463in}{2.408145in}}%
\pgfpathlineto{\pgfqpoint{3.075004in}{2.408145in}}%
\pgfpathlineto{\pgfqpoint{3.075004in}{2.405196in}}%
\pgfpathmoveto{\pgfqpoint{3.070463in}{2.408145in}}%
\pgfpathlineto{\pgfqpoint{3.070463in}{2.408145in}}%
\pgfpathlineto{\pgfqpoint{3.070463in}{2.411094in}}%
\pgfpathlineto{\pgfqpoint{3.075004in}{2.411094in}}%
\pgfpathlineto{\pgfqpoint{3.075004in}{2.408145in}}%
\pgfpathmoveto{\pgfqpoint{3.070463in}{2.411094in}}%
\pgfpathlineto{\pgfqpoint{3.070463in}{2.411094in}}%
\pgfpathlineto{\pgfqpoint{3.070463in}{2.414044in}}%
\pgfpathlineto{\pgfqpoint{3.075004in}{2.414044in}}%
\pgfpathlineto{\pgfqpoint{3.075004in}{2.411094in}}%
\pgfpathmoveto{\pgfqpoint{3.070463in}{2.414044in}}%
\pgfpathlineto{\pgfqpoint{3.070463in}{2.414044in}}%
\pgfpathlineto{\pgfqpoint{3.070463in}{2.416993in}}%
\pgfpathlineto{\pgfqpoint{3.075004in}{2.416993in}}%
\pgfpathlineto{\pgfqpoint{3.075004in}{2.414044in}}%
\pgfpathmoveto{\pgfqpoint{3.070463in}{2.416993in}}%
\pgfpathlineto{\pgfqpoint{3.070463in}{2.416993in}}%
\pgfpathlineto{\pgfqpoint{3.070463in}{2.419942in}}%
\pgfpathlineto{\pgfqpoint{3.075004in}{2.419942in}}%
\pgfpathlineto{\pgfqpoint{3.075004in}{2.416993in}}%
\pgfpathmoveto{\pgfqpoint{3.070463in}{2.419942in}}%
\pgfpathlineto{\pgfqpoint{3.070463in}{2.419942in}}%
\pgfpathlineto{\pgfqpoint{3.070463in}{2.422891in}}%
\pgfpathlineto{\pgfqpoint{3.075004in}{2.422891in}}%
\pgfpathlineto{\pgfqpoint{3.075004in}{2.419942in}}%
\pgfpathmoveto{\pgfqpoint{3.070463in}{2.422891in}}%
\pgfpathlineto{\pgfqpoint{3.070463in}{2.422891in}}%
\pgfpathlineto{\pgfqpoint{3.070463in}{2.425840in}}%
\pgfpathlineto{\pgfqpoint{3.075004in}{2.425840in}}%
\pgfpathlineto{\pgfqpoint{3.075004in}{2.422891in}}%
\pgfpathmoveto{\pgfqpoint{3.070463in}{2.425840in}}%
\pgfpathlineto{\pgfqpoint{3.070463in}{2.425840in}}%
\pgfpathlineto{\pgfqpoint{3.070463in}{2.428789in}}%
\pgfpathlineto{\pgfqpoint{3.075004in}{2.428789in}}%
\pgfpathlineto{\pgfqpoint{3.075004in}{2.425840in}}%
\pgfpathmoveto{\pgfqpoint{3.070463in}{2.428789in}}%
\pgfpathlineto{\pgfqpoint{3.070463in}{2.428789in}}%
\pgfpathlineto{\pgfqpoint{3.070463in}{2.431738in}}%
\pgfpathlineto{\pgfqpoint{3.075004in}{2.431738in}}%
\pgfpathlineto{\pgfqpoint{3.075004in}{2.428789in}}%
\pgfpathmoveto{\pgfqpoint{3.070463in}{2.431738in}}%
\pgfpathlineto{\pgfqpoint{3.070463in}{2.431738in}}%
\pgfpathlineto{\pgfqpoint{3.070463in}{2.434687in}}%
\pgfpathlineto{\pgfqpoint{3.075004in}{2.434687in}}%
\pgfpathlineto{\pgfqpoint{3.075004in}{2.431738in}}%
\pgfpathmoveto{\pgfqpoint{3.070463in}{2.434687in}}%
\pgfpathlineto{\pgfqpoint{3.070463in}{2.434687in}}%
\pgfpathlineto{\pgfqpoint{3.070463in}{2.437636in}}%
\pgfpathlineto{\pgfqpoint{3.075004in}{2.437636in}}%
\pgfpathlineto{\pgfqpoint{3.075004in}{2.434687in}}%
\pgfpathmoveto{\pgfqpoint{3.070463in}{2.437636in}}%
\pgfpathlineto{\pgfqpoint{3.070463in}{2.437636in}}%
\pgfpathlineto{\pgfqpoint{3.070463in}{2.440585in}}%
\pgfpathlineto{\pgfqpoint{3.075004in}{2.440585in}}%
\pgfpathlineto{\pgfqpoint{3.075004in}{2.437636in}}%
\pgfpathmoveto{\pgfqpoint{3.070463in}{2.440585in}}%
\pgfpathlineto{\pgfqpoint{3.070463in}{2.440585in}}%
\pgfpathlineto{\pgfqpoint{3.070463in}{2.443534in}}%
\pgfpathlineto{\pgfqpoint{3.075004in}{2.443534in}}%
\pgfpathlineto{\pgfqpoint{3.075004in}{2.440585in}}%
\pgfpathmoveto{\pgfqpoint{3.070463in}{2.443534in}}%
\pgfpathlineto{\pgfqpoint{3.070463in}{2.443534in}}%
\pgfpathlineto{\pgfqpoint{3.070463in}{2.446483in}}%
\pgfpathlineto{\pgfqpoint{3.075004in}{2.446483in}}%
\pgfpathlineto{\pgfqpoint{3.075004in}{2.443534in}}%
\pgfpathmoveto{\pgfqpoint{3.070463in}{2.446483in}}%
\pgfpathlineto{\pgfqpoint{3.070463in}{2.446483in}}%
\pgfpathlineto{\pgfqpoint{3.070463in}{2.449433in}}%
\pgfpathlineto{\pgfqpoint{3.075004in}{2.449433in}}%
\pgfpathlineto{\pgfqpoint{3.075004in}{2.446483in}}%
\pgfpathmoveto{\pgfqpoint{3.070463in}{2.449433in}}%
\pgfpathlineto{\pgfqpoint{3.070463in}{2.449433in}}%
\pgfpathlineto{\pgfqpoint{3.070463in}{2.452382in}}%
\pgfpathlineto{\pgfqpoint{3.075004in}{2.452382in}}%
\pgfpathlineto{\pgfqpoint{3.075004in}{2.449433in}}%
\pgfpathmoveto{\pgfqpoint{3.070463in}{2.452382in}}%
\pgfpathlineto{\pgfqpoint{3.070463in}{2.452382in}}%
\pgfpathlineto{\pgfqpoint{3.070463in}{2.455331in}}%
\pgfpathlineto{\pgfqpoint{3.075004in}{2.455331in}}%
\pgfpathlineto{\pgfqpoint{3.075004in}{2.452382in}}%
\pgfpathmoveto{\pgfqpoint{3.070463in}{2.455331in}}%
\pgfpathlineto{\pgfqpoint{3.070463in}{2.455331in}}%
\pgfpathlineto{\pgfqpoint{3.070463in}{2.458280in}}%
\pgfpathlineto{\pgfqpoint{3.075004in}{2.458280in}}%
\pgfpathlineto{\pgfqpoint{3.075004in}{2.455331in}}%
\pgfpathmoveto{\pgfqpoint{3.070463in}{2.458280in}}%
\pgfpathlineto{\pgfqpoint{3.070463in}{2.458280in}}%
\pgfpathlineto{\pgfqpoint{3.070463in}{2.461229in}}%
\pgfpathlineto{\pgfqpoint{3.075004in}{2.461229in}}%
\pgfpathlineto{\pgfqpoint{3.075004in}{2.458280in}}%
\pgfpathmoveto{\pgfqpoint{3.070463in}{2.461229in}}%
\pgfpathlineto{\pgfqpoint{3.070463in}{2.461229in}}%
\pgfpathlineto{\pgfqpoint{3.070463in}{2.464178in}}%
\pgfpathlineto{\pgfqpoint{3.075004in}{2.464178in}}%
\pgfpathlineto{\pgfqpoint{3.075004in}{2.461229in}}%
\pgfpathmoveto{\pgfqpoint{3.070463in}{2.464178in}}%
\pgfpathlineto{\pgfqpoint{3.070463in}{2.464178in}}%
\pgfpathlineto{\pgfqpoint{3.070463in}{2.467127in}}%
\pgfpathlineto{\pgfqpoint{3.075004in}{2.467127in}}%
\pgfpathlineto{\pgfqpoint{3.075004in}{2.464178in}}%
\pgfpathmoveto{\pgfqpoint{3.070463in}{2.467127in}}%
\pgfpathlineto{\pgfqpoint{3.070463in}{2.467127in}}%
\pgfpathlineto{\pgfqpoint{3.070463in}{2.470076in}}%
\pgfpathlineto{\pgfqpoint{3.075004in}{2.470076in}}%
\pgfpathlineto{\pgfqpoint{3.075004in}{2.467127in}}%
\pgfpathmoveto{\pgfqpoint{3.070463in}{2.470076in}}%
\pgfpathlineto{\pgfqpoint{3.070463in}{2.470076in}}%
\pgfpathlineto{\pgfqpoint{3.070463in}{2.473025in}}%
\pgfpathlineto{\pgfqpoint{3.075004in}{2.473025in}}%
\pgfpathlineto{\pgfqpoint{3.075004in}{2.470076in}}%
\pgfpathmoveto{\pgfqpoint{3.070463in}{2.473025in}}%
\pgfpathlineto{\pgfqpoint{3.070463in}{2.473025in}}%
\pgfpathlineto{\pgfqpoint{3.070463in}{2.475974in}}%
\pgfpathlineto{\pgfqpoint{3.075004in}{2.475974in}}%
\pgfpathlineto{\pgfqpoint{3.075004in}{2.473025in}}%
\pgfpathmoveto{\pgfqpoint{3.070463in}{2.475974in}}%
\pgfpathlineto{\pgfqpoint{3.070463in}{2.475974in}}%
\pgfpathlineto{\pgfqpoint{3.070463in}{2.478923in}}%
\pgfpathlineto{\pgfqpoint{3.075004in}{2.478923in}}%
\pgfpathlineto{\pgfqpoint{3.075004in}{2.475974in}}%
\pgfpathmoveto{\pgfqpoint{3.070463in}{2.478923in}}%
\pgfpathlineto{\pgfqpoint{3.070463in}{2.478923in}}%
\pgfpathlineto{\pgfqpoint{3.070463in}{2.481873in}}%
\pgfpathlineto{\pgfqpoint{3.075004in}{2.481873in}}%
\pgfpathlineto{\pgfqpoint{3.075004in}{2.478923in}}%
\pgfpathmoveto{\pgfqpoint{3.070463in}{2.481873in}}%
\pgfpathlineto{\pgfqpoint{3.070463in}{2.481873in}}%
\pgfpathlineto{\pgfqpoint{3.070463in}{2.484822in}}%
\pgfpathlineto{\pgfqpoint{3.075004in}{2.484822in}}%
\pgfpathlineto{\pgfqpoint{3.075004in}{2.481873in}}%
\pgfpathmoveto{\pgfqpoint{3.070463in}{2.484822in}}%
\pgfpathlineto{\pgfqpoint{3.070463in}{2.484822in}}%
\pgfpathlineto{\pgfqpoint{3.070463in}{2.487771in}}%
\pgfpathlineto{\pgfqpoint{3.075004in}{2.487771in}}%
\pgfpathlineto{\pgfqpoint{3.075004in}{2.484822in}}%
\pgfpathmoveto{\pgfqpoint{3.070463in}{2.487771in}}%
\pgfpathlineto{\pgfqpoint{3.070463in}{2.487771in}}%
\pgfpathlineto{\pgfqpoint{3.070463in}{2.490720in}}%
\pgfpathlineto{\pgfqpoint{3.075004in}{2.490720in}}%
\pgfpathlineto{\pgfqpoint{3.075004in}{2.487771in}}%
\pgfpathmoveto{\pgfqpoint{3.070463in}{2.490720in}}%
\pgfpathlineto{\pgfqpoint{3.070463in}{2.490720in}}%
\pgfpathlineto{\pgfqpoint{3.070463in}{2.493670in}}%
\pgfpathlineto{\pgfqpoint{3.075004in}{2.493670in}}%
\pgfpathlineto{\pgfqpoint{3.075004in}{2.490720in}}%
\pgfpathmoveto{\pgfqpoint{3.070463in}{2.493670in}}%
\pgfpathlineto{\pgfqpoint{3.070463in}{2.493670in}}%
\pgfpathlineto{\pgfqpoint{3.070463in}{2.496619in}}%
\pgfpathlineto{\pgfqpoint{3.075004in}{2.496619in}}%
\pgfpathlineto{\pgfqpoint{3.075004in}{2.493670in}}%
\pgfpathmoveto{\pgfqpoint{3.070463in}{2.496619in}}%
\pgfpathlineto{\pgfqpoint{3.070463in}{2.496619in}}%
\pgfpathlineto{\pgfqpoint{3.070463in}{2.499568in}}%
\pgfpathlineto{\pgfqpoint{3.075004in}{2.499568in}}%
\pgfpathlineto{\pgfqpoint{3.075004in}{2.496619in}}%
\pgfpathmoveto{\pgfqpoint{3.070463in}{2.499568in}}%
\pgfpathlineto{\pgfqpoint{3.070463in}{2.499568in}}%
\pgfpathlineto{\pgfqpoint{3.070463in}{2.502517in}}%
\pgfpathlineto{\pgfqpoint{3.075004in}{2.502517in}}%
\pgfpathlineto{\pgfqpoint{3.075004in}{2.499568in}}%
\pgfpathmoveto{\pgfqpoint{3.070463in}{2.502517in}}%
\pgfpathlineto{\pgfqpoint{3.070463in}{2.502517in}}%
\pgfpathlineto{\pgfqpoint{3.070463in}{2.505467in}}%
\pgfpathlineto{\pgfqpoint{3.075004in}{2.505467in}}%
\pgfpathlineto{\pgfqpoint{3.075004in}{2.502517in}}%
\pgfpathmoveto{\pgfqpoint{3.070463in}{2.505467in}}%
\pgfpathlineto{\pgfqpoint{3.070463in}{2.505467in}}%
\pgfpathlineto{\pgfqpoint{3.070463in}{2.508416in}}%
\pgfpathlineto{\pgfqpoint{3.075004in}{2.508416in}}%
\pgfpathlineto{\pgfqpoint{3.075004in}{2.505467in}}%
\pgfpathmoveto{\pgfqpoint{3.070463in}{2.508416in}}%
\pgfpathlineto{\pgfqpoint{3.070463in}{2.508416in}}%
\pgfpathlineto{\pgfqpoint{3.070463in}{2.511365in}}%
\pgfpathlineto{\pgfqpoint{3.075004in}{2.511365in}}%
\pgfpathlineto{\pgfqpoint{3.075004in}{2.508416in}}%
\pgfpathmoveto{\pgfqpoint{3.070463in}{2.511365in}}%
\pgfpathlineto{\pgfqpoint{3.070463in}{2.511365in}}%
\pgfpathlineto{\pgfqpoint{3.070463in}{2.514315in}}%
\pgfpathlineto{\pgfqpoint{3.075004in}{2.514315in}}%
\pgfpathlineto{\pgfqpoint{3.075004in}{2.511365in}}%
\pgfpathmoveto{\pgfqpoint{3.070463in}{2.514315in}}%
\pgfpathlineto{\pgfqpoint{3.070463in}{2.514315in}}%
\pgfpathlineto{\pgfqpoint{3.070463in}{2.517264in}}%
\pgfpathlineto{\pgfqpoint{3.075004in}{2.517264in}}%
\pgfpathlineto{\pgfqpoint{3.075004in}{2.514315in}}%
\pgfpathmoveto{\pgfqpoint{3.070463in}{2.517264in}}%
\pgfpathlineto{\pgfqpoint{3.070463in}{2.517264in}}%
\pgfpathlineto{\pgfqpoint{3.070463in}{2.520213in}}%
\pgfpathlineto{\pgfqpoint{3.075004in}{2.520213in}}%
\pgfpathlineto{\pgfqpoint{3.075004in}{2.517264in}}%
\pgfpathmoveto{\pgfqpoint{3.070463in}{2.520213in}}%
\pgfpathlineto{\pgfqpoint{3.070463in}{2.520213in}}%
\pgfpathlineto{\pgfqpoint{3.070463in}{2.523162in}}%
\pgfpathlineto{\pgfqpoint{3.075004in}{2.523162in}}%
\pgfpathlineto{\pgfqpoint{3.075004in}{2.520213in}}%
\pgfpathmoveto{\pgfqpoint{3.070463in}{2.523162in}}%
\pgfpathlineto{\pgfqpoint{3.070463in}{2.523162in}}%
\pgfpathlineto{\pgfqpoint{3.070463in}{2.526112in}}%
\pgfpathlineto{\pgfqpoint{3.075004in}{2.526112in}}%
\pgfpathlineto{\pgfqpoint{3.075004in}{2.523162in}}%
\pgfpathmoveto{\pgfqpoint{3.070463in}{2.526112in}}%
\pgfpathlineto{\pgfqpoint{3.070463in}{2.526112in}}%
\pgfpathlineto{\pgfqpoint{3.070463in}{2.529061in}}%
\pgfpathlineto{\pgfqpoint{3.075004in}{2.529061in}}%
\pgfpathlineto{\pgfqpoint{3.075004in}{2.526112in}}%
\pgfpathmoveto{\pgfqpoint{3.070463in}{2.529061in}}%
\pgfpathlineto{\pgfqpoint{3.070463in}{2.529061in}}%
\pgfpathlineto{\pgfqpoint{3.070463in}{2.532010in}}%
\pgfpathlineto{\pgfqpoint{3.075004in}{2.532010in}}%
\pgfpathlineto{\pgfqpoint{3.075004in}{2.529061in}}%
\pgfpathmoveto{\pgfqpoint{3.070463in}{2.532010in}}%
\pgfpathlineto{\pgfqpoint{3.070463in}{2.532010in}}%
\pgfpathlineto{\pgfqpoint{3.070463in}{2.534959in}}%
\pgfpathlineto{\pgfqpoint{3.075004in}{2.534959in}}%
\pgfpathlineto{\pgfqpoint{3.075004in}{2.532010in}}%
\pgfpathmoveto{\pgfqpoint{3.070463in}{2.534959in}}%
\pgfpathlineto{\pgfqpoint{3.070463in}{2.534959in}}%
\pgfpathlineto{\pgfqpoint{3.070463in}{2.537909in}}%
\pgfpathlineto{\pgfqpoint{3.075004in}{2.537909in}}%
\pgfpathlineto{\pgfqpoint{3.075004in}{2.534959in}}%
\pgfpathmoveto{\pgfqpoint{3.070463in}{2.537909in}}%
\pgfpathlineto{\pgfqpoint{3.070463in}{2.537909in}}%
\pgfpathlineto{\pgfqpoint{3.070463in}{2.540858in}}%
\pgfpathlineto{\pgfqpoint{3.075004in}{2.540858in}}%
\pgfpathlineto{\pgfqpoint{3.075004in}{2.537909in}}%
\pgfpathmoveto{\pgfqpoint{3.070463in}{2.540858in}}%
\pgfpathlineto{\pgfqpoint{3.070463in}{2.540858in}}%
\pgfpathlineto{\pgfqpoint{3.070463in}{2.543807in}}%
\pgfpathlineto{\pgfqpoint{3.075004in}{2.543807in}}%
\pgfpathlineto{\pgfqpoint{3.075004in}{2.540858in}}%
\pgfpathmoveto{\pgfqpoint{3.070463in}{2.543807in}}%
\pgfpathlineto{\pgfqpoint{3.070463in}{2.543807in}}%
\pgfpathlineto{\pgfqpoint{3.070463in}{2.546757in}}%
\pgfpathlineto{\pgfqpoint{3.075004in}{2.546757in}}%
\pgfpathlineto{\pgfqpoint{3.075004in}{2.543807in}}%
\pgfpathmoveto{\pgfqpoint{3.070463in}{2.546757in}}%
\pgfpathlineto{\pgfqpoint{3.070463in}{2.546757in}}%
\pgfpathlineto{\pgfqpoint{3.070463in}{2.549706in}}%
\pgfpathlineto{\pgfqpoint{3.075004in}{2.549706in}}%
\pgfpathlineto{\pgfqpoint{3.075004in}{2.546757in}}%
\pgfpathmoveto{\pgfqpoint{3.070463in}{2.549706in}}%
\pgfpathlineto{\pgfqpoint{3.070463in}{2.549706in}}%
\pgfpathlineto{\pgfqpoint{3.070463in}{2.552655in}}%
\pgfpathlineto{\pgfqpoint{3.075004in}{2.552655in}}%
\pgfpathlineto{\pgfqpoint{3.075004in}{2.549706in}}%
\pgfpathmoveto{\pgfqpoint{3.070463in}{2.552655in}}%
\pgfpathlineto{\pgfqpoint{3.070463in}{2.552655in}}%
\pgfpathlineto{\pgfqpoint{3.070463in}{2.555604in}}%
\pgfpathlineto{\pgfqpoint{3.075004in}{2.555604in}}%
\pgfpathlineto{\pgfqpoint{3.075004in}{2.552655in}}%
\pgfpathmoveto{\pgfqpoint{3.070463in}{2.555604in}}%
\pgfpathlineto{\pgfqpoint{3.070463in}{2.555604in}}%
\pgfpathlineto{\pgfqpoint{3.070463in}{2.558554in}}%
\pgfpathlineto{\pgfqpoint{3.075004in}{2.558554in}}%
\pgfpathlineto{\pgfqpoint{3.075004in}{2.555604in}}%
\pgfpathmoveto{\pgfqpoint{3.070463in}{2.558554in}}%
\pgfpathlineto{\pgfqpoint{3.070463in}{2.558554in}}%
\pgfpathlineto{\pgfqpoint{3.070463in}{2.561503in}}%
\pgfpathlineto{\pgfqpoint{3.075004in}{2.561503in}}%
\pgfpathlineto{\pgfqpoint{3.075004in}{2.558554in}}%
\pgfpathmoveto{\pgfqpoint{3.070463in}{2.561503in}}%
\pgfpathlineto{\pgfqpoint{3.070463in}{2.561503in}}%
\pgfpathlineto{\pgfqpoint{3.070463in}{2.564452in}}%
\pgfpathlineto{\pgfqpoint{3.075004in}{2.564452in}}%
\pgfpathlineto{\pgfqpoint{3.075004in}{2.561503in}}%
\pgfpathmoveto{\pgfqpoint{3.070463in}{2.564452in}}%
\pgfpathlineto{\pgfqpoint{3.070463in}{2.564452in}}%
\pgfpathlineto{\pgfqpoint{3.070463in}{2.567402in}}%
\pgfpathlineto{\pgfqpoint{3.075004in}{2.567402in}}%
\pgfpathlineto{\pgfqpoint{3.075004in}{2.564452in}}%
\pgfpathmoveto{\pgfqpoint{3.070463in}{2.567402in}}%
\pgfpathlineto{\pgfqpoint{3.070463in}{2.567402in}}%
\pgfpathlineto{\pgfqpoint{3.070463in}{2.570351in}}%
\pgfpathlineto{\pgfqpoint{3.075004in}{2.570351in}}%
\pgfpathlineto{\pgfqpoint{3.075004in}{2.567402in}}%
\pgfpathmoveto{\pgfqpoint{3.070463in}{2.570351in}}%
\pgfpathlineto{\pgfqpoint{3.070463in}{2.570351in}}%
\pgfpathlineto{\pgfqpoint{3.070463in}{2.573300in}}%
\pgfpathlineto{\pgfqpoint{3.075004in}{2.573300in}}%
\pgfpathlineto{\pgfqpoint{3.075004in}{2.570351in}}%
\pgfpathmoveto{\pgfqpoint{3.070463in}{2.573300in}}%
\pgfpathlineto{\pgfqpoint{3.070463in}{2.573300in}}%
\pgfpathlineto{\pgfqpoint{3.070463in}{2.576249in}}%
\pgfpathlineto{\pgfqpoint{3.075004in}{2.576249in}}%
\pgfpathlineto{\pgfqpoint{3.075004in}{2.573300in}}%
\pgfpathmoveto{\pgfqpoint{3.070463in}{2.576249in}}%
\pgfpathlineto{\pgfqpoint{3.070463in}{2.576249in}}%
\pgfpathlineto{\pgfqpoint{3.070463in}{2.579199in}}%
\pgfpathlineto{\pgfqpoint{3.075004in}{2.579199in}}%
\pgfpathlineto{\pgfqpoint{3.075004in}{2.576249in}}%
\pgfpathmoveto{\pgfqpoint{3.070463in}{2.579199in}}%
\pgfpathlineto{\pgfqpoint{3.070463in}{2.579199in}}%
\pgfpathlineto{\pgfqpoint{3.070463in}{2.582148in}}%
\pgfpathlineto{\pgfqpoint{3.075004in}{2.582148in}}%
\pgfpathlineto{\pgfqpoint{3.075004in}{2.579199in}}%
\pgfpathmoveto{\pgfqpoint{3.070463in}{2.582148in}}%
\pgfpathlineto{\pgfqpoint{3.070463in}{2.582148in}}%
\pgfpathlineto{\pgfqpoint{3.070463in}{2.585097in}}%
\pgfpathlineto{\pgfqpoint{3.075004in}{2.585097in}}%
\pgfpathlineto{\pgfqpoint{3.075004in}{2.582148in}}%
\pgfpathmoveto{\pgfqpoint{3.070463in}{2.585097in}}%
\pgfpathlineto{\pgfqpoint{3.070463in}{2.585097in}}%
\pgfpathlineto{\pgfqpoint{3.070463in}{2.588046in}}%
\pgfpathlineto{\pgfqpoint{3.075004in}{2.588046in}}%
\pgfpathlineto{\pgfqpoint{3.075004in}{2.585097in}}%
\pgfpathmoveto{\pgfqpoint{3.070463in}{2.588046in}}%
\pgfpathlineto{\pgfqpoint{3.070463in}{2.588046in}}%
\pgfpathlineto{\pgfqpoint{3.070463in}{2.590995in}}%
\pgfpathlineto{\pgfqpoint{3.075004in}{2.590995in}}%
\pgfpathlineto{\pgfqpoint{3.075004in}{2.588046in}}%
\pgfpathmoveto{\pgfqpoint{3.070463in}{2.590995in}}%
\pgfpathlineto{\pgfqpoint{3.070463in}{2.590995in}}%
\pgfpathlineto{\pgfqpoint{3.070463in}{2.593944in}}%
\pgfpathlineto{\pgfqpoint{3.075004in}{2.593944in}}%
\pgfpathlineto{\pgfqpoint{3.075004in}{2.590995in}}%
\pgfpathmoveto{\pgfqpoint{3.070463in}{2.593944in}}%
\pgfpathlineto{\pgfqpoint{3.070463in}{2.593944in}}%
\pgfpathlineto{\pgfqpoint{3.070463in}{2.596894in}}%
\pgfpathlineto{\pgfqpoint{3.075004in}{2.596894in}}%
\pgfpathlineto{\pgfqpoint{3.075004in}{2.593944in}}%
\pgfpathmoveto{\pgfqpoint{3.070463in}{2.596894in}}%
\pgfpathlineto{\pgfqpoint{3.070463in}{2.596894in}}%
\pgfpathlineto{\pgfqpoint{3.070463in}{2.599843in}}%
\pgfpathlineto{\pgfqpoint{3.075004in}{2.599843in}}%
\pgfpathlineto{\pgfqpoint{3.075004in}{2.596894in}}%
\pgfpathmoveto{\pgfqpoint{3.070463in}{2.599843in}}%
\pgfpathlineto{\pgfqpoint{3.070463in}{2.599843in}}%
\pgfpathlineto{\pgfqpoint{3.070463in}{2.602792in}}%
\pgfpathlineto{\pgfqpoint{3.075004in}{2.602792in}}%
\pgfpathlineto{\pgfqpoint{3.075004in}{2.599843in}}%
\pgfpathmoveto{\pgfqpoint{3.070463in}{2.602792in}}%
\pgfpathlineto{\pgfqpoint{3.070463in}{2.602792in}}%
\pgfpathlineto{\pgfqpoint{3.070463in}{2.605741in}}%
\pgfpathlineto{\pgfqpoint{3.075004in}{2.605741in}}%
\pgfpathlineto{\pgfqpoint{3.075004in}{2.602792in}}%
\pgfpathmoveto{\pgfqpoint{3.070463in}{2.605741in}}%
\pgfpathlineto{\pgfqpoint{3.070463in}{2.605741in}}%
\pgfpathlineto{\pgfqpoint{3.070463in}{2.608690in}}%
\pgfpathlineto{\pgfqpoint{3.075004in}{2.608690in}}%
\pgfpathlineto{\pgfqpoint{3.075004in}{2.605741in}}%
\pgfpathmoveto{\pgfqpoint{3.070463in}{2.608690in}}%
\pgfpathlineto{\pgfqpoint{3.070463in}{2.608690in}}%
\pgfpathlineto{\pgfqpoint{3.070463in}{2.611639in}}%
\pgfpathlineto{\pgfqpoint{3.075004in}{2.611639in}}%
\pgfpathlineto{\pgfqpoint{3.075004in}{2.608690in}}%
\pgfpathmoveto{\pgfqpoint{3.070463in}{2.611639in}}%
\pgfpathlineto{\pgfqpoint{3.070463in}{2.611639in}}%
\pgfpathlineto{\pgfqpoint{3.070463in}{2.614589in}}%
\pgfpathlineto{\pgfqpoint{3.075004in}{2.614589in}}%
\pgfpathlineto{\pgfqpoint{3.075004in}{2.611639in}}%
\pgfpathmoveto{\pgfqpoint{3.075004in}{2.009997in}}%
\pgfpathlineto{\pgfqpoint{3.075004in}{2.009997in}}%
\pgfpathlineto{\pgfqpoint{3.075004in}{2.012947in}}%
\pgfpathlineto{\pgfqpoint{3.079545in}{2.012947in}}%
\pgfpathlineto{\pgfqpoint{3.079545in}{2.009997in}}%
\pgfpathmoveto{\pgfqpoint{3.075004in}{2.012947in}}%
\pgfpathlineto{\pgfqpoint{3.075004in}{2.012947in}}%
\pgfpathlineto{\pgfqpoint{3.075004in}{2.015896in}}%
\pgfpathlineto{\pgfqpoint{3.079545in}{2.015896in}}%
\pgfpathlineto{\pgfqpoint{3.079545in}{2.012947in}}%
\pgfpathmoveto{\pgfqpoint{3.079545in}{2.009997in}}%
\pgfpathlineto{\pgfqpoint{3.079545in}{2.009997in}}%
\pgfpathlineto{\pgfqpoint{3.079545in}{2.012947in}}%
\pgfpathlineto{\pgfqpoint{3.084086in}{2.012947in}}%
\pgfpathlineto{\pgfqpoint{3.084086in}{2.009997in}}%
\pgfpathmoveto{\pgfqpoint{3.079545in}{2.012947in}}%
\pgfpathlineto{\pgfqpoint{3.079545in}{2.012947in}}%
\pgfpathlineto{\pgfqpoint{3.079545in}{2.015896in}}%
\pgfpathlineto{\pgfqpoint{3.084086in}{2.015896in}}%
\pgfpathlineto{\pgfqpoint{3.084086in}{2.012947in}}%
\pgfpathmoveto{\pgfqpoint{3.084086in}{2.009997in}}%
\pgfpathlineto{\pgfqpoint{3.084086in}{2.009997in}}%
\pgfpathlineto{\pgfqpoint{3.084086in}{2.012947in}}%
\pgfpathlineto{\pgfqpoint{3.088627in}{2.012947in}}%
\pgfpathlineto{\pgfqpoint{3.088627in}{2.009997in}}%
\pgfpathmoveto{\pgfqpoint{3.084086in}{2.012947in}}%
\pgfpathlineto{\pgfqpoint{3.084086in}{2.012947in}}%
\pgfpathlineto{\pgfqpoint{3.084086in}{2.015896in}}%
\pgfpathlineto{\pgfqpoint{3.088627in}{2.015896in}}%
\pgfpathlineto{\pgfqpoint{3.088627in}{2.012947in}}%
\pgfpathmoveto{\pgfqpoint{3.088627in}{2.009997in}}%
\pgfpathlineto{\pgfqpoint{3.088627in}{2.009997in}}%
\pgfpathlineto{\pgfqpoint{3.088627in}{2.012947in}}%
\pgfpathlineto{\pgfqpoint{3.093168in}{2.012947in}}%
\pgfpathlineto{\pgfqpoint{3.093168in}{2.009997in}}%
\pgfpathmoveto{\pgfqpoint{3.088627in}{2.012947in}}%
\pgfpathlineto{\pgfqpoint{3.088627in}{2.012947in}}%
\pgfpathlineto{\pgfqpoint{3.088627in}{2.015896in}}%
\pgfpathlineto{\pgfqpoint{3.093168in}{2.015896in}}%
\pgfpathlineto{\pgfqpoint{3.093168in}{2.012947in}}%
\pgfpathmoveto{\pgfqpoint{3.093168in}{2.009997in}}%
\pgfpathlineto{\pgfqpoint{3.093168in}{2.009997in}}%
\pgfpathlineto{\pgfqpoint{3.093168in}{2.012947in}}%
\pgfpathlineto{\pgfqpoint{3.097708in}{2.012947in}}%
\pgfpathlineto{\pgfqpoint{3.097708in}{2.009997in}}%
\pgfpathmoveto{\pgfqpoint{3.093168in}{2.012947in}}%
\pgfpathlineto{\pgfqpoint{3.093168in}{2.012947in}}%
\pgfpathlineto{\pgfqpoint{3.093168in}{2.015896in}}%
\pgfpathlineto{\pgfqpoint{3.097708in}{2.015896in}}%
\pgfpathlineto{\pgfqpoint{3.097708in}{2.012947in}}%
\pgfpathmoveto{\pgfqpoint{3.097708in}{2.009997in}}%
\pgfpathlineto{\pgfqpoint{3.097708in}{2.009997in}}%
\pgfpathlineto{\pgfqpoint{3.097708in}{2.012947in}}%
\pgfpathlineto{\pgfqpoint{3.102249in}{2.012947in}}%
\pgfpathlineto{\pgfqpoint{3.102249in}{2.009997in}}%
\pgfpathmoveto{\pgfqpoint{3.097708in}{2.012947in}}%
\pgfpathlineto{\pgfqpoint{3.097708in}{2.012947in}}%
\pgfpathlineto{\pgfqpoint{3.097708in}{2.015896in}}%
\pgfpathlineto{\pgfqpoint{3.102249in}{2.015896in}}%
\pgfpathlineto{\pgfqpoint{3.102249in}{2.012947in}}%
\pgfpathmoveto{\pgfqpoint{3.102249in}{2.009997in}}%
\pgfpathlineto{\pgfqpoint{3.102249in}{2.009997in}}%
\pgfpathlineto{\pgfqpoint{3.102249in}{2.012947in}}%
\pgfpathlineto{\pgfqpoint{3.106790in}{2.012947in}}%
\pgfpathlineto{\pgfqpoint{3.106790in}{2.009997in}}%
\pgfpathmoveto{\pgfqpoint{3.102249in}{2.012947in}}%
\pgfpathlineto{\pgfqpoint{3.102249in}{2.012947in}}%
\pgfpathlineto{\pgfqpoint{3.102249in}{2.015896in}}%
\pgfpathlineto{\pgfqpoint{3.106790in}{2.015896in}}%
\pgfpathlineto{\pgfqpoint{3.106790in}{2.012947in}}%
\pgfpathmoveto{\pgfqpoint{3.106790in}{2.009997in}}%
\pgfpathlineto{\pgfqpoint{3.106790in}{2.009997in}}%
\pgfpathlineto{\pgfqpoint{3.106790in}{2.012947in}}%
\pgfpathlineto{\pgfqpoint{3.111331in}{2.012947in}}%
\pgfpathlineto{\pgfqpoint{3.111331in}{2.009997in}}%
\pgfpathmoveto{\pgfqpoint{3.106790in}{2.012947in}}%
\pgfpathlineto{\pgfqpoint{3.106790in}{2.012947in}}%
\pgfpathlineto{\pgfqpoint{3.106790in}{2.015896in}}%
\pgfpathlineto{\pgfqpoint{3.111331in}{2.015896in}}%
\pgfpathlineto{\pgfqpoint{3.111331in}{2.012947in}}%
\pgfpathmoveto{\pgfqpoint{3.111331in}{2.009997in}}%
\pgfpathlineto{\pgfqpoint{3.111331in}{2.009997in}}%
\pgfpathlineto{\pgfqpoint{3.111331in}{2.012947in}}%
\pgfpathlineto{\pgfqpoint{3.115872in}{2.012947in}}%
\pgfpathlineto{\pgfqpoint{3.115872in}{2.009997in}}%
\pgfpathmoveto{\pgfqpoint{3.111331in}{2.012947in}}%
\pgfpathlineto{\pgfqpoint{3.111331in}{2.012947in}}%
\pgfpathlineto{\pgfqpoint{3.111331in}{2.015896in}}%
\pgfpathlineto{\pgfqpoint{3.115872in}{2.015896in}}%
\pgfpathlineto{\pgfqpoint{3.115872in}{2.012947in}}%
\pgfpathmoveto{\pgfqpoint{3.115872in}{2.009997in}}%
\pgfpathlineto{\pgfqpoint{3.115872in}{2.009997in}}%
\pgfpathlineto{\pgfqpoint{3.115872in}{2.012947in}}%
\pgfpathlineto{\pgfqpoint{3.120413in}{2.012947in}}%
\pgfpathlineto{\pgfqpoint{3.120413in}{2.009997in}}%
\pgfpathmoveto{\pgfqpoint{3.115872in}{2.012947in}}%
\pgfpathlineto{\pgfqpoint{3.115872in}{2.012947in}}%
\pgfpathlineto{\pgfqpoint{3.115872in}{2.015896in}}%
\pgfpathlineto{\pgfqpoint{3.120413in}{2.015896in}}%
\pgfpathlineto{\pgfqpoint{3.120413in}{2.012947in}}%
\pgfpathmoveto{\pgfqpoint{3.120413in}{2.009997in}}%
\pgfpathlineto{\pgfqpoint{3.120413in}{2.009997in}}%
\pgfpathlineto{\pgfqpoint{3.120413in}{2.012947in}}%
\pgfpathlineto{\pgfqpoint{3.124953in}{2.012947in}}%
\pgfpathlineto{\pgfqpoint{3.124953in}{2.009997in}}%
\pgfpathmoveto{\pgfqpoint{3.120413in}{2.012947in}}%
\pgfpathlineto{\pgfqpoint{3.120413in}{2.012947in}}%
\pgfpathlineto{\pgfqpoint{3.120413in}{2.015896in}}%
\pgfpathlineto{\pgfqpoint{3.124953in}{2.015896in}}%
\pgfpathlineto{\pgfqpoint{3.124953in}{2.012947in}}%
\pgfpathmoveto{\pgfqpoint{3.124953in}{2.009997in}}%
\pgfpathlineto{\pgfqpoint{3.124953in}{2.009997in}}%
\pgfpathlineto{\pgfqpoint{3.124953in}{2.012947in}}%
\pgfpathlineto{\pgfqpoint{3.129494in}{2.012947in}}%
\pgfpathlineto{\pgfqpoint{3.129494in}{2.009997in}}%
\pgfpathmoveto{\pgfqpoint{3.124953in}{2.012947in}}%
\pgfpathlineto{\pgfqpoint{3.124953in}{2.012947in}}%
\pgfpathlineto{\pgfqpoint{3.124953in}{2.015896in}}%
\pgfpathlineto{\pgfqpoint{3.129494in}{2.015896in}}%
\pgfpathlineto{\pgfqpoint{3.129494in}{2.012947in}}%
\pgfpathmoveto{\pgfqpoint{3.129494in}{2.009997in}}%
\pgfpathlineto{\pgfqpoint{3.129494in}{2.009997in}}%
\pgfpathlineto{\pgfqpoint{3.129494in}{2.012947in}}%
\pgfpathlineto{\pgfqpoint{3.134035in}{2.012947in}}%
\pgfpathlineto{\pgfqpoint{3.134035in}{2.009997in}}%
\pgfpathmoveto{\pgfqpoint{3.129494in}{2.012947in}}%
\pgfpathlineto{\pgfqpoint{3.129494in}{2.012947in}}%
\pgfpathlineto{\pgfqpoint{3.129494in}{2.015896in}}%
\pgfpathlineto{\pgfqpoint{3.134035in}{2.015896in}}%
\pgfpathlineto{\pgfqpoint{3.134035in}{2.012947in}}%
\pgfpathmoveto{\pgfqpoint{3.134035in}{2.009997in}}%
\pgfpathlineto{\pgfqpoint{3.134035in}{2.009997in}}%
\pgfpathlineto{\pgfqpoint{3.134035in}{2.012947in}}%
\pgfpathlineto{\pgfqpoint{3.138576in}{2.012947in}}%
\pgfpathlineto{\pgfqpoint{3.138576in}{2.009997in}}%
\pgfpathmoveto{\pgfqpoint{3.134035in}{2.012947in}}%
\pgfpathlineto{\pgfqpoint{3.134035in}{2.012947in}}%
\pgfpathlineto{\pgfqpoint{3.134035in}{2.015896in}}%
\pgfpathlineto{\pgfqpoint{3.138576in}{2.015896in}}%
\pgfpathlineto{\pgfqpoint{3.138576in}{2.012947in}}%
\pgfpathmoveto{\pgfqpoint{3.138576in}{2.009997in}}%
\pgfpathlineto{\pgfqpoint{3.138576in}{2.009997in}}%
\pgfpathlineto{\pgfqpoint{3.138576in}{2.012947in}}%
\pgfpathlineto{\pgfqpoint{3.143117in}{2.012947in}}%
\pgfpathlineto{\pgfqpoint{3.143117in}{2.009997in}}%
\pgfpathmoveto{\pgfqpoint{3.138576in}{2.012947in}}%
\pgfpathlineto{\pgfqpoint{3.138576in}{2.012947in}}%
\pgfpathlineto{\pgfqpoint{3.138576in}{2.015896in}}%
\pgfpathlineto{\pgfqpoint{3.143117in}{2.015896in}}%
\pgfpathlineto{\pgfqpoint{3.143117in}{2.012947in}}%
\pgfpathmoveto{\pgfqpoint{3.143117in}{2.009997in}}%
\pgfpathlineto{\pgfqpoint{3.143117in}{2.009997in}}%
\pgfpathlineto{\pgfqpoint{3.143117in}{2.012947in}}%
\pgfpathlineto{\pgfqpoint{3.147658in}{2.012947in}}%
\pgfpathlineto{\pgfqpoint{3.147658in}{2.009997in}}%
\pgfpathmoveto{\pgfqpoint{3.143117in}{2.012947in}}%
\pgfpathlineto{\pgfqpoint{3.143117in}{2.012947in}}%
\pgfpathlineto{\pgfqpoint{3.143117in}{2.015896in}}%
\pgfpathlineto{\pgfqpoint{3.147658in}{2.015896in}}%
\pgfpathlineto{\pgfqpoint{3.147658in}{2.012947in}}%
\pgfpathmoveto{\pgfqpoint{3.147658in}{2.009997in}}%
\pgfpathlineto{\pgfqpoint{3.147658in}{2.009997in}}%
\pgfpathlineto{\pgfqpoint{3.147658in}{2.012947in}}%
\pgfpathlineto{\pgfqpoint{3.152198in}{2.012947in}}%
\pgfpathlineto{\pgfqpoint{3.152198in}{2.009997in}}%
\pgfpathmoveto{\pgfqpoint{3.147658in}{2.012947in}}%
\pgfpathlineto{\pgfqpoint{3.147658in}{2.012947in}}%
\pgfpathlineto{\pgfqpoint{3.147658in}{2.015896in}}%
\pgfpathlineto{\pgfqpoint{3.152198in}{2.015896in}}%
\pgfpathlineto{\pgfqpoint{3.152198in}{2.012947in}}%
\pgfpathmoveto{\pgfqpoint{3.152198in}{2.009997in}}%
\pgfpathlineto{\pgfqpoint{3.152198in}{2.009997in}}%
\pgfpathlineto{\pgfqpoint{3.152198in}{2.012947in}}%
\pgfpathlineto{\pgfqpoint{3.156739in}{2.012947in}}%
\pgfpathlineto{\pgfqpoint{3.156739in}{2.009997in}}%
\pgfpathmoveto{\pgfqpoint{3.152198in}{2.012947in}}%
\pgfpathlineto{\pgfqpoint{3.152198in}{2.012947in}}%
\pgfpathlineto{\pgfqpoint{3.152198in}{2.015896in}}%
\pgfpathlineto{\pgfqpoint{3.156739in}{2.015896in}}%
\pgfpathlineto{\pgfqpoint{3.156739in}{2.012947in}}%
\pgfpathmoveto{\pgfqpoint{3.156739in}{2.009997in}}%
\pgfpathlineto{\pgfqpoint{3.156739in}{2.009997in}}%
\pgfpathlineto{\pgfqpoint{3.156739in}{2.012947in}}%
\pgfpathlineto{\pgfqpoint{3.161280in}{2.012947in}}%
\pgfpathlineto{\pgfqpoint{3.161280in}{2.009997in}}%
\pgfpathmoveto{\pgfqpoint{3.156739in}{2.012947in}}%
\pgfpathlineto{\pgfqpoint{3.156739in}{2.012947in}}%
\pgfpathlineto{\pgfqpoint{3.156739in}{2.015896in}}%
\pgfpathlineto{\pgfqpoint{3.161280in}{2.015896in}}%
\pgfpathlineto{\pgfqpoint{3.161280in}{2.012947in}}%
\pgfpathmoveto{\pgfqpoint{3.161280in}{2.009997in}}%
\pgfpathlineto{\pgfqpoint{3.161280in}{2.009997in}}%
\pgfpathlineto{\pgfqpoint{3.161280in}{2.012947in}}%
\pgfpathlineto{\pgfqpoint{3.165821in}{2.012947in}}%
\pgfpathlineto{\pgfqpoint{3.165821in}{2.009997in}}%
\pgfpathmoveto{\pgfqpoint{3.161280in}{2.012947in}}%
\pgfpathlineto{\pgfqpoint{3.161280in}{2.012947in}}%
\pgfpathlineto{\pgfqpoint{3.161280in}{2.015896in}}%
\pgfpathlineto{\pgfqpoint{3.165821in}{2.015896in}}%
\pgfpathlineto{\pgfqpoint{3.165821in}{2.012947in}}%
\pgfpathmoveto{\pgfqpoint{3.165821in}{2.009997in}}%
\pgfpathlineto{\pgfqpoint{3.165821in}{2.009997in}}%
\pgfpathlineto{\pgfqpoint{3.165821in}{2.012947in}}%
\pgfpathlineto{\pgfqpoint{3.170362in}{2.012947in}}%
\pgfpathlineto{\pgfqpoint{3.170362in}{2.009997in}}%
\pgfpathmoveto{\pgfqpoint{3.165821in}{2.012947in}}%
\pgfpathlineto{\pgfqpoint{3.165821in}{2.012947in}}%
\pgfpathlineto{\pgfqpoint{3.165821in}{2.015896in}}%
\pgfpathlineto{\pgfqpoint{3.170362in}{2.015896in}}%
\pgfpathlineto{\pgfqpoint{3.170362in}{2.012947in}}%
\pgfpathmoveto{\pgfqpoint{3.170362in}{2.009997in}}%
\pgfpathlineto{\pgfqpoint{3.170362in}{2.009997in}}%
\pgfpathlineto{\pgfqpoint{3.170362in}{2.012947in}}%
\pgfpathlineto{\pgfqpoint{3.174903in}{2.012947in}}%
\pgfpathlineto{\pgfqpoint{3.174903in}{2.009997in}}%
\pgfpathmoveto{\pgfqpoint{3.170362in}{2.012947in}}%
\pgfpathlineto{\pgfqpoint{3.170362in}{2.012947in}}%
\pgfpathlineto{\pgfqpoint{3.170362in}{2.015896in}}%
\pgfpathlineto{\pgfqpoint{3.174903in}{2.015896in}}%
\pgfpathlineto{\pgfqpoint{3.174903in}{2.012947in}}%
\pgfpathmoveto{\pgfqpoint{3.174903in}{2.009997in}}%
\pgfpathlineto{\pgfqpoint{3.174903in}{2.009997in}}%
\pgfpathlineto{\pgfqpoint{3.174903in}{2.012947in}}%
\pgfpathlineto{\pgfqpoint{3.179443in}{2.012947in}}%
\pgfpathlineto{\pgfqpoint{3.179443in}{2.009997in}}%
\pgfpathmoveto{\pgfqpoint{3.174903in}{2.012947in}}%
\pgfpathlineto{\pgfqpoint{3.174903in}{2.012947in}}%
\pgfpathlineto{\pgfqpoint{3.174903in}{2.015896in}}%
\pgfpathlineto{\pgfqpoint{3.179443in}{2.015896in}}%
\pgfpathlineto{\pgfqpoint{3.179443in}{2.012947in}}%
\pgfpathmoveto{\pgfqpoint{3.179443in}{2.009997in}}%
\pgfpathlineto{\pgfqpoint{3.179443in}{2.009997in}}%
\pgfpathlineto{\pgfqpoint{3.179443in}{2.012947in}}%
\pgfpathlineto{\pgfqpoint{3.183984in}{2.012947in}}%
\pgfpathlineto{\pgfqpoint{3.183984in}{2.009997in}}%
\pgfpathmoveto{\pgfqpoint{3.179443in}{2.012947in}}%
\pgfpathlineto{\pgfqpoint{3.179443in}{2.012947in}}%
\pgfpathlineto{\pgfqpoint{3.179443in}{2.015896in}}%
\pgfpathlineto{\pgfqpoint{3.183984in}{2.015896in}}%
\pgfpathlineto{\pgfqpoint{3.183984in}{2.012947in}}%
\pgfpathmoveto{\pgfqpoint{3.183984in}{2.009997in}}%
\pgfpathlineto{\pgfqpoint{3.183984in}{2.009997in}}%
\pgfpathlineto{\pgfqpoint{3.183984in}{2.012947in}}%
\pgfpathlineto{\pgfqpoint{3.188525in}{2.012947in}}%
\pgfpathlineto{\pgfqpoint{3.188525in}{2.009997in}}%
\pgfpathmoveto{\pgfqpoint{3.183984in}{2.012947in}}%
\pgfpathlineto{\pgfqpoint{3.183984in}{2.012947in}}%
\pgfpathlineto{\pgfqpoint{3.183984in}{2.015896in}}%
\pgfpathlineto{\pgfqpoint{3.188525in}{2.015896in}}%
\pgfpathlineto{\pgfqpoint{3.188525in}{2.012947in}}%
\pgfpathmoveto{\pgfqpoint{3.188525in}{2.009997in}}%
\pgfpathlineto{\pgfqpoint{3.188525in}{2.009997in}}%
\pgfpathlineto{\pgfqpoint{3.188525in}{2.012947in}}%
\pgfpathlineto{\pgfqpoint{3.193066in}{2.012947in}}%
\pgfpathlineto{\pgfqpoint{3.193066in}{2.009997in}}%
\pgfpathmoveto{\pgfqpoint{3.188525in}{2.012947in}}%
\pgfpathlineto{\pgfqpoint{3.188525in}{2.012947in}}%
\pgfpathlineto{\pgfqpoint{3.188525in}{2.015896in}}%
\pgfpathlineto{\pgfqpoint{3.193066in}{2.015896in}}%
\pgfpathlineto{\pgfqpoint{3.193066in}{2.012947in}}%
\pgfpathmoveto{\pgfqpoint{3.193066in}{2.009997in}}%
\pgfpathlineto{\pgfqpoint{3.193066in}{2.009997in}}%
\pgfpathlineto{\pgfqpoint{3.193066in}{2.012947in}}%
\pgfpathlineto{\pgfqpoint{3.197607in}{2.012947in}}%
\pgfpathlineto{\pgfqpoint{3.197607in}{2.009997in}}%
\pgfpathmoveto{\pgfqpoint{3.193066in}{2.012947in}}%
\pgfpathlineto{\pgfqpoint{3.193066in}{2.012947in}}%
\pgfpathlineto{\pgfqpoint{3.193066in}{2.015896in}}%
\pgfpathlineto{\pgfqpoint{3.197607in}{2.015896in}}%
\pgfpathlineto{\pgfqpoint{3.197607in}{2.012947in}}%
\pgfpathmoveto{\pgfqpoint{3.197607in}{2.009997in}}%
\pgfpathlineto{\pgfqpoint{3.197607in}{2.009997in}}%
\pgfpathlineto{\pgfqpoint{3.197607in}{2.012947in}}%
\pgfpathlineto{\pgfqpoint{3.202148in}{2.012947in}}%
\pgfpathlineto{\pgfqpoint{3.202148in}{2.009997in}}%
\pgfpathmoveto{\pgfqpoint{3.197607in}{2.012947in}}%
\pgfpathlineto{\pgfqpoint{3.197607in}{2.012947in}}%
\pgfpathlineto{\pgfqpoint{3.197607in}{2.015896in}}%
\pgfpathlineto{\pgfqpoint{3.202148in}{2.015896in}}%
\pgfpathlineto{\pgfqpoint{3.202148in}{2.012947in}}%
\pgfpathmoveto{\pgfqpoint{3.202148in}{2.009997in}}%
\pgfpathlineto{\pgfqpoint{3.202148in}{2.009997in}}%
\pgfpathlineto{\pgfqpoint{3.202148in}{2.012947in}}%
\pgfpathlineto{\pgfqpoint{3.206689in}{2.012947in}}%
\pgfpathlineto{\pgfqpoint{3.206689in}{2.009997in}}%
\pgfpathmoveto{\pgfqpoint{3.202148in}{2.012947in}}%
\pgfpathlineto{\pgfqpoint{3.202148in}{2.012947in}}%
\pgfpathlineto{\pgfqpoint{3.202148in}{2.015896in}}%
\pgfpathlineto{\pgfqpoint{3.206689in}{2.015896in}}%
\pgfpathlineto{\pgfqpoint{3.206689in}{2.012947in}}%
\pgfpathmoveto{\pgfqpoint{3.206689in}{2.009997in}}%
\pgfpathlineto{\pgfqpoint{3.206689in}{2.009997in}}%
\pgfpathlineto{\pgfqpoint{3.206689in}{2.012947in}}%
\pgfpathlineto{\pgfqpoint{3.211229in}{2.012947in}}%
\pgfpathlineto{\pgfqpoint{3.211229in}{2.009997in}}%
\pgfpathmoveto{\pgfqpoint{3.206689in}{2.012947in}}%
\pgfpathlineto{\pgfqpoint{3.206689in}{2.012947in}}%
\pgfpathlineto{\pgfqpoint{3.206689in}{2.015896in}}%
\pgfpathlineto{\pgfqpoint{3.211229in}{2.015896in}}%
\pgfpathlineto{\pgfqpoint{3.211229in}{2.012947in}}%
\pgfpathmoveto{\pgfqpoint{3.211229in}{2.009997in}}%
\pgfpathlineto{\pgfqpoint{3.211229in}{2.009997in}}%
\pgfpathlineto{\pgfqpoint{3.211229in}{2.012947in}}%
\pgfpathlineto{\pgfqpoint{3.215770in}{2.012947in}}%
\pgfpathlineto{\pgfqpoint{3.215770in}{2.009997in}}%
\pgfpathmoveto{\pgfqpoint{3.211229in}{2.012947in}}%
\pgfpathlineto{\pgfqpoint{3.211229in}{2.012947in}}%
\pgfpathlineto{\pgfqpoint{3.211229in}{2.015896in}}%
\pgfpathlineto{\pgfqpoint{3.215770in}{2.015896in}}%
\pgfpathlineto{\pgfqpoint{3.215770in}{2.012947in}}%
\pgfpathmoveto{\pgfqpoint{3.215770in}{2.009997in}}%
\pgfpathlineto{\pgfqpoint{3.215770in}{2.009997in}}%
\pgfpathlineto{\pgfqpoint{3.215770in}{2.012947in}}%
\pgfpathlineto{\pgfqpoint{3.220311in}{2.012947in}}%
\pgfpathlineto{\pgfqpoint{3.220311in}{2.009997in}}%
\pgfpathmoveto{\pgfqpoint{3.215770in}{2.012947in}}%
\pgfpathlineto{\pgfqpoint{3.215770in}{2.012947in}}%
\pgfpathlineto{\pgfqpoint{3.215770in}{2.015896in}}%
\pgfpathlineto{\pgfqpoint{3.220311in}{2.015896in}}%
\pgfpathlineto{\pgfqpoint{3.220311in}{2.012947in}}%
\pgfpathmoveto{\pgfqpoint{3.075004in}{2.611639in}}%
\pgfpathlineto{\pgfqpoint{3.075004in}{2.611639in}}%
\pgfpathlineto{\pgfqpoint{3.075004in}{2.614589in}}%
\pgfpathlineto{\pgfqpoint{3.079545in}{2.614589in}}%
\pgfpathlineto{\pgfqpoint{3.079545in}{2.611639in}}%
\pgfpathmoveto{\pgfqpoint{3.079545in}{2.611639in}}%
\pgfpathlineto{\pgfqpoint{3.079545in}{2.611639in}}%
\pgfpathlineto{\pgfqpoint{3.079545in}{2.614589in}}%
\pgfpathlineto{\pgfqpoint{3.084086in}{2.614589in}}%
\pgfpathlineto{\pgfqpoint{3.084086in}{2.611639in}}%
\pgfpathmoveto{\pgfqpoint{3.084086in}{2.611639in}}%
\pgfpathlineto{\pgfqpoint{3.084086in}{2.611639in}}%
\pgfpathlineto{\pgfqpoint{3.084086in}{2.614589in}}%
\pgfpathlineto{\pgfqpoint{3.088627in}{2.614589in}}%
\pgfpathlineto{\pgfqpoint{3.088627in}{2.611639in}}%
\pgfpathmoveto{\pgfqpoint{3.088627in}{2.611639in}}%
\pgfpathlineto{\pgfqpoint{3.088627in}{2.611639in}}%
\pgfpathlineto{\pgfqpoint{3.088627in}{2.614589in}}%
\pgfpathlineto{\pgfqpoint{3.093168in}{2.614589in}}%
\pgfpathlineto{\pgfqpoint{3.093168in}{2.611639in}}%
\pgfpathmoveto{\pgfqpoint{3.093168in}{2.611639in}}%
\pgfpathlineto{\pgfqpoint{3.093168in}{2.611639in}}%
\pgfpathlineto{\pgfqpoint{3.093168in}{2.614589in}}%
\pgfpathlineto{\pgfqpoint{3.097708in}{2.614589in}}%
\pgfpathlineto{\pgfqpoint{3.097708in}{2.611639in}}%
\pgfpathmoveto{\pgfqpoint{3.097708in}{2.611639in}}%
\pgfpathlineto{\pgfqpoint{3.097708in}{2.611639in}}%
\pgfpathlineto{\pgfqpoint{3.097708in}{2.614589in}}%
\pgfpathlineto{\pgfqpoint{3.102249in}{2.614589in}}%
\pgfpathlineto{\pgfqpoint{3.102249in}{2.611639in}}%
\pgfpathmoveto{\pgfqpoint{3.102249in}{2.611639in}}%
\pgfpathlineto{\pgfqpoint{3.102249in}{2.611639in}}%
\pgfpathlineto{\pgfqpoint{3.102249in}{2.614589in}}%
\pgfpathlineto{\pgfqpoint{3.106790in}{2.614589in}}%
\pgfpathlineto{\pgfqpoint{3.106790in}{2.611639in}}%
\pgfpathmoveto{\pgfqpoint{3.106790in}{2.611639in}}%
\pgfpathlineto{\pgfqpoint{3.106790in}{2.611639in}}%
\pgfpathlineto{\pgfqpoint{3.106790in}{2.614589in}}%
\pgfpathlineto{\pgfqpoint{3.111331in}{2.614589in}}%
\pgfpathlineto{\pgfqpoint{3.111331in}{2.611639in}}%
\pgfpathmoveto{\pgfqpoint{3.111331in}{2.611639in}}%
\pgfpathlineto{\pgfqpoint{3.111331in}{2.611639in}}%
\pgfpathlineto{\pgfqpoint{3.111331in}{2.614589in}}%
\pgfpathlineto{\pgfqpoint{3.115872in}{2.614589in}}%
\pgfpathlineto{\pgfqpoint{3.115872in}{2.611639in}}%
\pgfpathmoveto{\pgfqpoint{3.115872in}{2.611639in}}%
\pgfpathlineto{\pgfqpoint{3.115872in}{2.611639in}}%
\pgfpathlineto{\pgfqpoint{3.115872in}{2.614589in}}%
\pgfpathlineto{\pgfqpoint{3.120413in}{2.614589in}}%
\pgfpathlineto{\pgfqpoint{3.120413in}{2.611639in}}%
\pgfpathmoveto{\pgfqpoint{3.120413in}{2.611639in}}%
\pgfpathlineto{\pgfqpoint{3.120413in}{2.611639in}}%
\pgfpathlineto{\pgfqpoint{3.120413in}{2.614589in}}%
\pgfpathlineto{\pgfqpoint{3.124953in}{2.614589in}}%
\pgfpathlineto{\pgfqpoint{3.124953in}{2.611639in}}%
\pgfpathmoveto{\pgfqpoint{3.124953in}{2.611639in}}%
\pgfpathlineto{\pgfqpoint{3.124953in}{2.611639in}}%
\pgfpathlineto{\pgfqpoint{3.124953in}{2.614589in}}%
\pgfpathlineto{\pgfqpoint{3.129494in}{2.614589in}}%
\pgfpathlineto{\pgfqpoint{3.129494in}{2.611639in}}%
\pgfpathmoveto{\pgfqpoint{3.129494in}{2.611639in}}%
\pgfpathlineto{\pgfqpoint{3.129494in}{2.611639in}}%
\pgfpathlineto{\pgfqpoint{3.129494in}{2.614589in}}%
\pgfpathlineto{\pgfqpoint{3.134035in}{2.614589in}}%
\pgfpathlineto{\pgfqpoint{3.134035in}{2.611639in}}%
\pgfpathmoveto{\pgfqpoint{3.134035in}{2.611639in}}%
\pgfpathlineto{\pgfqpoint{3.134035in}{2.611639in}}%
\pgfpathlineto{\pgfqpoint{3.134035in}{2.614589in}}%
\pgfpathlineto{\pgfqpoint{3.138576in}{2.614589in}}%
\pgfpathlineto{\pgfqpoint{3.138576in}{2.611639in}}%
\pgfpathmoveto{\pgfqpoint{3.138576in}{2.611639in}}%
\pgfpathlineto{\pgfqpoint{3.138576in}{2.611639in}}%
\pgfpathlineto{\pgfqpoint{3.138576in}{2.614589in}}%
\pgfpathlineto{\pgfqpoint{3.143117in}{2.614589in}}%
\pgfpathlineto{\pgfqpoint{3.143117in}{2.611639in}}%
\pgfpathmoveto{\pgfqpoint{3.143117in}{2.611639in}}%
\pgfpathlineto{\pgfqpoint{3.143117in}{2.611639in}}%
\pgfpathlineto{\pgfqpoint{3.143117in}{2.614589in}}%
\pgfpathlineto{\pgfqpoint{3.147658in}{2.614589in}}%
\pgfpathlineto{\pgfqpoint{3.147658in}{2.611639in}}%
\pgfpathmoveto{\pgfqpoint{3.147658in}{2.611639in}}%
\pgfpathlineto{\pgfqpoint{3.147658in}{2.611639in}}%
\pgfpathlineto{\pgfqpoint{3.147658in}{2.614589in}}%
\pgfpathlineto{\pgfqpoint{3.152198in}{2.614589in}}%
\pgfpathlineto{\pgfqpoint{3.152198in}{2.611639in}}%
\pgfpathmoveto{\pgfqpoint{3.152198in}{2.611639in}}%
\pgfpathlineto{\pgfqpoint{3.152198in}{2.611639in}}%
\pgfpathlineto{\pgfqpoint{3.152198in}{2.614589in}}%
\pgfpathlineto{\pgfqpoint{3.156739in}{2.614589in}}%
\pgfpathlineto{\pgfqpoint{3.156739in}{2.611639in}}%
\pgfpathmoveto{\pgfqpoint{3.156739in}{2.611639in}}%
\pgfpathlineto{\pgfqpoint{3.156739in}{2.611639in}}%
\pgfpathlineto{\pgfqpoint{3.156739in}{2.614589in}}%
\pgfpathlineto{\pgfqpoint{3.161280in}{2.614589in}}%
\pgfpathlineto{\pgfqpoint{3.161280in}{2.611639in}}%
\pgfpathmoveto{\pgfqpoint{3.161280in}{2.611639in}}%
\pgfpathlineto{\pgfqpoint{3.161280in}{2.611639in}}%
\pgfpathlineto{\pgfqpoint{3.161280in}{2.614589in}}%
\pgfpathlineto{\pgfqpoint{3.165821in}{2.614589in}}%
\pgfpathlineto{\pgfqpoint{3.165821in}{2.611639in}}%
\pgfpathmoveto{\pgfqpoint{3.165821in}{2.611639in}}%
\pgfpathlineto{\pgfqpoint{3.165821in}{2.611639in}}%
\pgfpathlineto{\pgfqpoint{3.165821in}{2.614589in}}%
\pgfpathlineto{\pgfqpoint{3.170362in}{2.614589in}}%
\pgfpathlineto{\pgfqpoint{3.170362in}{2.611639in}}%
\pgfpathmoveto{\pgfqpoint{3.170362in}{2.611639in}}%
\pgfpathlineto{\pgfqpoint{3.170362in}{2.611639in}}%
\pgfpathlineto{\pgfqpoint{3.170362in}{2.614589in}}%
\pgfpathlineto{\pgfqpoint{3.174903in}{2.614589in}}%
\pgfpathlineto{\pgfqpoint{3.174903in}{2.611639in}}%
\pgfpathmoveto{\pgfqpoint{3.174903in}{2.611639in}}%
\pgfpathlineto{\pgfqpoint{3.174903in}{2.611639in}}%
\pgfpathlineto{\pgfqpoint{3.174903in}{2.614589in}}%
\pgfpathlineto{\pgfqpoint{3.179443in}{2.614589in}}%
\pgfpathlineto{\pgfqpoint{3.179443in}{2.611639in}}%
\pgfpathmoveto{\pgfqpoint{3.179443in}{2.611639in}}%
\pgfpathlineto{\pgfqpoint{3.179443in}{2.611639in}}%
\pgfpathlineto{\pgfqpoint{3.179443in}{2.614589in}}%
\pgfpathlineto{\pgfqpoint{3.183984in}{2.614589in}}%
\pgfpathlineto{\pgfqpoint{3.183984in}{2.611639in}}%
\pgfpathmoveto{\pgfqpoint{3.183984in}{2.611639in}}%
\pgfpathlineto{\pgfqpoint{3.183984in}{2.611639in}}%
\pgfpathlineto{\pgfqpoint{3.183984in}{2.614589in}}%
\pgfpathlineto{\pgfqpoint{3.188525in}{2.614589in}}%
\pgfpathlineto{\pgfqpoint{3.188525in}{2.611639in}}%
\pgfpathmoveto{\pgfqpoint{3.188525in}{2.611639in}}%
\pgfpathlineto{\pgfqpoint{3.188525in}{2.611639in}}%
\pgfpathlineto{\pgfqpoint{3.188525in}{2.614589in}}%
\pgfpathlineto{\pgfqpoint{3.193066in}{2.614589in}}%
\pgfpathlineto{\pgfqpoint{3.193066in}{2.611639in}}%
\pgfpathmoveto{\pgfqpoint{3.193066in}{2.611639in}}%
\pgfpathlineto{\pgfqpoint{3.193066in}{2.611639in}}%
\pgfpathlineto{\pgfqpoint{3.193066in}{2.614589in}}%
\pgfpathlineto{\pgfqpoint{3.197607in}{2.614589in}}%
\pgfpathlineto{\pgfqpoint{3.197607in}{2.611639in}}%
\pgfpathmoveto{\pgfqpoint{3.197607in}{2.611639in}}%
\pgfpathlineto{\pgfqpoint{3.197607in}{2.611639in}}%
\pgfpathlineto{\pgfqpoint{3.197607in}{2.614589in}}%
\pgfpathlineto{\pgfqpoint{3.202148in}{2.614589in}}%
\pgfpathlineto{\pgfqpoint{3.202148in}{2.611639in}}%
\pgfpathmoveto{\pgfqpoint{3.202148in}{2.611639in}}%
\pgfpathlineto{\pgfqpoint{3.202148in}{2.611639in}}%
\pgfpathlineto{\pgfqpoint{3.202148in}{2.614589in}}%
\pgfpathlineto{\pgfqpoint{3.206689in}{2.614589in}}%
\pgfpathlineto{\pgfqpoint{3.206689in}{2.611639in}}%
\pgfpathmoveto{\pgfqpoint{3.206689in}{2.611639in}}%
\pgfpathlineto{\pgfqpoint{3.206689in}{2.611639in}}%
\pgfpathlineto{\pgfqpoint{3.206689in}{2.614589in}}%
\pgfpathlineto{\pgfqpoint{3.211229in}{2.614589in}}%
\pgfpathlineto{\pgfqpoint{3.211229in}{2.611639in}}%
\pgfpathmoveto{\pgfqpoint{3.211229in}{2.611639in}}%
\pgfpathlineto{\pgfqpoint{3.211229in}{2.611639in}}%
\pgfpathlineto{\pgfqpoint{3.211229in}{2.614589in}}%
\pgfpathlineto{\pgfqpoint{3.215770in}{2.614589in}}%
\pgfpathlineto{\pgfqpoint{3.215770in}{2.611639in}}%
\pgfpathmoveto{\pgfqpoint{3.215770in}{2.611639in}}%
\pgfpathlineto{\pgfqpoint{3.215770in}{2.611639in}}%
\pgfpathlineto{\pgfqpoint{3.215770in}{2.614589in}}%
\pgfpathlineto{\pgfqpoint{3.220311in}{2.614589in}}%
\pgfpathlineto{\pgfqpoint{3.220311in}{2.611639in}}%
\pgfpathmoveto{\pgfqpoint{3.220311in}{2.009997in}}%
\pgfpathlineto{\pgfqpoint{3.220311in}{2.009997in}}%
\pgfpathlineto{\pgfqpoint{3.220311in}{2.012947in}}%
\pgfpathlineto{\pgfqpoint{3.224852in}{2.012947in}}%
\pgfpathlineto{\pgfqpoint{3.224852in}{2.009997in}}%
\pgfpathmoveto{\pgfqpoint{3.220311in}{2.012947in}}%
\pgfpathlineto{\pgfqpoint{3.220311in}{2.012947in}}%
\pgfpathlineto{\pgfqpoint{3.220311in}{2.015896in}}%
\pgfpathlineto{\pgfqpoint{3.224852in}{2.015896in}}%
\pgfpathlineto{\pgfqpoint{3.224852in}{2.012947in}}%
\pgfpathmoveto{\pgfqpoint{3.224852in}{2.009997in}}%
\pgfpathlineto{\pgfqpoint{3.224852in}{2.009997in}}%
\pgfpathlineto{\pgfqpoint{3.224852in}{2.012947in}}%
\pgfpathlineto{\pgfqpoint{3.229393in}{2.012947in}}%
\pgfpathlineto{\pgfqpoint{3.229393in}{2.009997in}}%
\pgfpathmoveto{\pgfqpoint{3.224852in}{2.012947in}}%
\pgfpathlineto{\pgfqpoint{3.224852in}{2.012947in}}%
\pgfpathlineto{\pgfqpoint{3.224852in}{2.015896in}}%
\pgfpathlineto{\pgfqpoint{3.229393in}{2.015896in}}%
\pgfpathlineto{\pgfqpoint{3.229393in}{2.012947in}}%
\pgfpathmoveto{\pgfqpoint{3.229393in}{2.009997in}}%
\pgfpathlineto{\pgfqpoint{3.229393in}{2.009997in}}%
\pgfpathlineto{\pgfqpoint{3.229393in}{2.012947in}}%
\pgfpathlineto{\pgfqpoint{3.233934in}{2.012947in}}%
\pgfpathlineto{\pgfqpoint{3.233934in}{2.009997in}}%
\pgfpathmoveto{\pgfqpoint{3.229393in}{2.012947in}}%
\pgfpathlineto{\pgfqpoint{3.229393in}{2.012947in}}%
\pgfpathlineto{\pgfqpoint{3.229393in}{2.015896in}}%
\pgfpathlineto{\pgfqpoint{3.233934in}{2.015896in}}%
\pgfpathlineto{\pgfqpoint{3.233934in}{2.012947in}}%
\pgfpathmoveto{\pgfqpoint{3.233934in}{2.009997in}}%
\pgfpathlineto{\pgfqpoint{3.233934in}{2.009997in}}%
\pgfpathlineto{\pgfqpoint{3.233934in}{2.012947in}}%
\pgfpathlineto{\pgfqpoint{3.238475in}{2.012947in}}%
\pgfpathlineto{\pgfqpoint{3.238475in}{2.009997in}}%
\pgfpathmoveto{\pgfqpoint{3.233934in}{2.012947in}}%
\pgfpathlineto{\pgfqpoint{3.233934in}{2.012947in}}%
\pgfpathlineto{\pgfqpoint{3.233934in}{2.015896in}}%
\pgfpathlineto{\pgfqpoint{3.238475in}{2.015896in}}%
\pgfpathlineto{\pgfqpoint{3.238475in}{2.012947in}}%
\pgfpathmoveto{\pgfqpoint{3.238475in}{2.009997in}}%
\pgfpathlineto{\pgfqpoint{3.238475in}{2.009997in}}%
\pgfpathlineto{\pgfqpoint{3.238475in}{2.012947in}}%
\pgfpathlineto{\pgfqpoint{3.243016in}{2.012947in}}%
\pgfpathlineto{\pgfqpoint{3.243016in}{2.009997in}}%
\pgfpathmoveto{\pgfqpoint{3.238475in}{2.012947in}}%
\pgfpathlineto{\pgfqpoint{3.238475in}{2.012947in}}%
\pgfpathlineto{\pgfqpoint{3.238475in}{2.015896in}}%
\pgfpathlineto{\pgfqpoint{3.243016in}{2.015896in}}%
\pgfpathlineto{\pgfqpoint{3.243016in}{2.012947in}}%
\pgfpathmoveto{\pgfqpoint{3.243016in}{2.009997in}}%
\pgfpathlineto{\pgfqpoint{3.243016in}{2.009997in}}%
\pgfpathlineto{\pgfqpoint{3.243016in}{2.012947in}}%
\pgfpathlineto{\pgfqpoint{3.247557in}{2.012947in}}%
\pgfpathlineto{\pgfqpoint{3.247557in}{2.009997in}}%
\pgfpathmoveto{\pgfqpoint{3.243016in}{2.012947in}}%
\pgfpathlineto{\pgfqpoint{3.243016in}{2.012947in}}%
\pgfpathlineto{\pgfqpoint{3.243016in}{2.015896in}}%
\pgfpathlineto{\pgfqpoint{3.247557in}{2.015896in}}%
\pgfpathlineto{\pgfqpoint{3.247557in}{2.012947in}}%
\pgfpathmoveto{\pgfqpoint{3.247557in}{2.009997in}}%
\pgfpathlineto{\pgfqpoint{3.247557in}{2.009997in}}%
\pgfpathlineto{\pgfqpoint{3.247557in}{2.012947in}}%
\pgfpathlineto{\pgfqpoint{3.252098in}{2.012947in}}%
\pgfpathlineto{\pgfqpoint{3.252098in}{2.009997in}}%
\pgfpathmoveto{\pgfqpoint{3.247557in}{2.012947in}}%
\pgfpathlineto{\pgfqpoint{3.247557in}{2.012947in}}%
\pgfpathlineto{\pgfqpoint{3.247557in}{2.015896in}}%
\pgfpathlineto{\pgfqpoint{3.252098in}{2.015896in}}%
\pgfpathlineto{\pgfqpoint{3.252098in}{2.012947in}}%
\pgfpathmoveto{\pgfqpoint{3.252098in}{2.009997in}}%
\pgfpathlineto{\pgfqpoint{3.252098in}{2.009997in}}%
\pgfpathlineto{\pgfqpoint{3.252098in}{2.012947in}}%
\pgfpathlineto{\pgfqpoint{3.256640in}{2.012947in}}%
\pgfpathlineto{\pgfqpoint{3.256640in}{2.009997in}}%
\pgfpathmoveto{\pgfqpoint{3.252098in}{2.012947in}}%
\pgfpathlineto{\pgfqpoint{3.252098in}{2.012947in}}%
\pgfpathlineto{\pgfqpoint{3.252098in}{2.015896in}}%
\pgfpathlineto{\pgfqpoint{3.256640in}{2.015896in}}%
\pgfpathlineto{\pgfqpoint{3.256640in}{2.012947in}}%
\pgfpathmoveto{\pgfqpoint{3.256640in}{2.009997in}}%
\pgfpathlineto{\pgfqpoint{3.256640in}{2.009997in}}%
\pgfpathlineto{\pgfqpoint{3.256640in}{2.012947in}}%
\pgfpathlineto{\pgfqpoint{3.261181in}{2.012947in}}%
\pgfpathlineto{\pgfqpoint{3.261181in}{2.009997in}}%
\pgfpathmoveto{\pgfqpoint{3.256640in}{2.012947in}}%
\pgfpathlineto{\pgfqpoint{3.256640in}{2.012947in}}%
\pgfpathlineto{\pgfqpoint{3.256640in}{2.015896in}}%
\pgfpathlineto{\pgfqpoint{3.261181in}{2.015896in}}%
\pgfpathlineto{\pgfqpoint{3.261181in}{2.012947in}}%
\pgfpathmoveto{\pgfqpoint{3.261181in}{2.009997in}}%
\pgfpathlineto{\pgfqpoint{3.261181in}{2.009997in}}%
\pgfpathlineto{\pgfqpoint{3.261181in}{2.012947in}}%
\pgfpathlineto{\pgfqpoint{3.265722in}{2.012947in}}%
\pgfpathlineto{\pgfqpoint{3.265722in}{2.009997in}}%
\pgfpathmoveto{\pgfqpoint{3.261181in}{2.012947in}}%
\pgfpathlineto{\pgfqpoint{3.261181in}{2.012947in}}%
\pgfpathlineto{\pgfqpoint{3.261181in}{2.015896in}}%
\pgfpathlineto{\pgfqpoint{3.265722in}{2.015896in}}%
\pgfpathlineto{\pgfqpoint{3.265722in}{2.012947in}}%
\pgfpathmoveto{\pgfqpoint{3.265722in}{2.009997in}}%
\pgfpathlineto{\pgfqpoint{3.265722in}{2.009997in}}%
\pgfpathlineto{\pgfqpoint{3.265722in}{2.012947in}}%
\pgfpathlineto{\pgfqpoint{3.270263in}{2.012947in}}%
\pgfpathlineto{\pgfqpoint{3.270263in}{2.009997in}}%
\pgfpathmoveto{\pgfqpoint{3.265722in}{2.012947in}}%
\pgfpathlineto{\pgfqpoint{3.265722in}{2.012947in}}%
\pgfpathlineto{\pgfqpoint{3.265722in}{2.015896in}}%
\pgfpathlineto{\pgfqpoint{3.270263in}{2.015896in}}%
\pgfpathlineto{\pgfqpoint{3.270263in}{2.012947in}}%
\pgfpathmoveto{\pgfqpoint{3.270263in}{2.009997in}}%
\pgfpathlineto{\pgfqpoint{3.270263in}{2.009997in}}%
\pgfpathlineto{\pgfqpoint{3.270263in}{2.012947in}}%
\pgfpathlineto{\pgfqpoint{3.274804in}{2.012947in}}%
\pgfpathlineto{\pgfqpoint{3.274804in}{2.009997in}}%
\pgfpathmoveto{\pgfqpoint{3.270263in}{2.012947in}}%
\pgfpathlineto{\pgfqpoint{3.270263in}{2.012947in}}%
\pgfpathlineto{\pgfqpoint{3.270263in}{2.015896in}}%
\pgfpathlineto{\pgfqpoint{3.274804in}{2.015896in}}%
\pgfpathlineto{\pgfqpoint{3.274804in}{2.012947in}}%
\pgfpathmoveto{\pgfqpoint{3.274804in}{2.009997in}}%
\pgfpathlineto{\pgfqpoint{3.274804in}{2.009997in}}%
\pgfpathlineto{\pgfqpoint{3.274804in}{2.012947in}}%
\pgfpathlineto{\pgfqpoint{3.279345in}{2.012947in}}%
\pgfpathlineto{\pgfqpoint{3.279345in}{2.009997in}}%
\pgfpathmoveto{\pgfqpoint{3.274804in}{2.012947in}}%
\pgfpathlineto{\pgfqpoint{3.274804in}{2.012947in}}%
\pgfpathlineto{\pgfqpoint{3.274804in}{2.015896in}}%
\pgfpathlineto{\pgfqpoint{3.279345in}{2.015896in}}%
\pgfpathlineto{\pgfqpoint{3.279345in}{2.012947in}}%
\pgfpathmoveto{\pgfqpoint{3.279345in}{2.009997in}}%
\pgfpathlineto{\pgfqpoint{3.279345in}{2.009997in}}%
\pgfpathlineto{\pgfqpoint{3.279345in}{2.012947in}}%
\pgfpathlineto{\pgfqpoint{3.283886in}{2.012947in}}%
\pgfpathlineto{\pgfqpoint{3.283886in}{2.009997in}}%
\pgfpathmoveto{\pgfqpoint{3.279345in}{2.012947in}}%
\pgfpathlineto{\pgfqpoint{3.279345in}{2.012947in}}%
\pgfpathlineto{\pgfqpoint{3.279345in}{2.015896in}}%
\pgfpathlineto{\pgfqpoint{3.283886in}{2.015896in}}%
\pgfpathlineto{\pgfqpoint{3.283886in}{2.012947in}}%
\pgfpathmoveto{\pgfqpoint{3.283886in}{2.009997in}}%
\pgfpathlineto{\pgfqpoint{3.283886in}{2.009997in}}%
\pgfpathlineto{\pgfqpoint{3.283886in}{2.012947in}}%
\pgfpathlineto{\pgfqpoint{3.288427in}{2.012947in}}%
\pgfpathlineto{\pgfqpoint{3.288427in}{2.009997in}}%
\pgfpathmoveto{\pgfqpoint{3.283886in}{2.012947in}}%
\pgfpathlineto{\pgfqpoint{3.283886in}{2.012947in}}%
\pgfpathlineto{\pgfqpoint{3.283886in}{2.015896in}}%
\pgfpathlineto{\pgfqpoint{3.288427in}{2.015896in}}%
\pgfpathlineto{\pgfqpoint{3.288427in}{2.012947in}}%
\pgfpathmoveto{\pgfqpoint{3.288427in}{2.009997in}}%
\pgfpathlineto{\pgfqpoint{3.288427in}{2.009997in}}%
\pgfpathlineto{\pgfqpoint{3.288427in}{2.012947in}}%
\pgfpathlineto{\pgfqpoint{3.292968in}{2.012947in}}%
\pgfpathlineto{\pgfqpoint{3.292968in}{2.009997in}}%
\pgfpathmoveto{\pgfqpoint{3.288427in}{2.012947in}}%
\pgfpathlineto{\pgfqpoint{3.288427in}{2.012947in}}%
\pgfpathlineto{\pgfqpoint{3.288427in}{2.015896in}}%
\pgfpathlineto{\pgfqpoint{3.292968in}{2.015896in}}%
\pgfpathlineto{\pgfqpoint{3.292968in}{2.012947in}}%
\pgfpathmoveto{\pgfqpoint{3.292968in}{2.009997in}}%
\pgfpathlineto{\pgfqpoint{3.292968in}{2.009997in}}%
\pgfpathlineto{\pgfqpoint{3.292968in}{2.012947in}}%
\pgfpathlineto{\pgfqpoint{3.297509in}{2.012947in}}%
\pgfpathlineto{\pgfqpoint{3.297509in}{2.009997in}}%
\pgfpathmoveto{\pgfqpoint{3.292968in}{2.012947in}}%
\pgfpathlineto{\pgfqpoint{3.292968in}{2.012947in}}%
\pgfpathlineto{\pgfqpoint{3.292968in}{2.015896in}}%
\pgfpathlineto{\pgfqpoint{3.297509in}{2.015896in}}%
\pgfpathlineto{\pgfqpoint{3.297509in}{2.012947in}}%
\pgfpathmoveto{\pgfqpoint{3.297509in}{2.009997in}}%
\pgfpathlineto{\pgfqpoint{3.297509in}{2.009997in}}%
\pgfpathlineto{\pgfqpoint{3.297509in}{2.012947in}}%
\pgfpathlineto{\pgfqpoint{3.302050in}{2.012947in}}%
\pgfpathlineto{\pgfqpoint{3.302050in}{2.009997in}}%
\pgfpathmoveto{\pgfqpoint{3.297509in}{2.012947in}}%
\pgfpathlineto{\pgfqpoint{3.297509in}{2.012947in}}%
\pgfpathlineto{\pgfqpoint{3.297509in}{2.015896in}}%
\pgfpathlineto{\pgfqpoint{3.302050in}{2.015896in}}%
\pgfpathlineto{\pgfqpoint{3.302050in}{2.012947in}}%
\pgfpathmoveto{\pgfqpoint{3.302050in}{2.009997in}}%
\pgfpathlineto{\pgfqpoint{3.302050in}{2.009997in}}%
\pgfpathlineto{\pgfqpoint{3.302050in}{2.012947in}}%
\pgfpathlineto{\pgfqpoint{3.306591in}{2.012947in}}%
\pgfpathlineto{\pgfqpoint{3.306591in}{2.009997in}}%
\pgfpathmoveto{\pgfqpoint{3.302050in}{2.012947in}}%
\pgfpathlineto{\pgfqpoint{3.302050in}{2.012947in}}%
\pgfpathlineto{\pgfqpoint{3.302050in}{2.015896in}}%
\pgfpathlineto{\pgfqpoint{3.306591in}{2.015896in}}%
\pgfpathlineto{\pgfqpoint{3.306591in}{2.012947in}}%
\pgfpathmoveto{\pgfqpoint{3.306591in}{2.009997in}}%
\pgfpathlineto{\pgfqpoint{3.306591in}{2.009997in}}%
\pgfpathlineto{\pgfqpoint{3.306591in}{2.012947in}}%
\pgfpathlineto{\pgfqpoint{3.311132in}{2.012947in}}%
\pgfpathlineto{\pgfqpoint{3.311132in}{2.009997in}}%
\pgfpathmoveto{\pgfqpoint{3.306591in}{2.012947in}}%
\pgfpathlineto{\pgfqpoint{3.306591in}{2.012947in}}%
\pgfpathlineto{\pgfqpoint{3.306591in}{2.015896in}}%
\pgfpathlineto{\pgfqpoint{3.311132in}{2.015896in}}%
\pgfpathlineto{\pgfqpoint{3.311132in}{2.012947in}}%
\pgfpathmoveto{\pgfqpoint{3.311132in}{2.009997in}}%
\pgfpathlineto{\pgfqpoint{3.311132in}{2.009997in}}%
\pgfpathlineto{\pgfqpoint{3.311132in}{2.012947in}}%
\pgfpathlineto{\pgfqpoint{3.315673in}{2.012947in}}%
\pgfpathlineto{\pgfqpoint{3.315673in}{2.009997in}}%
\pgfpathmoveto{\pgfqpoint{3.311132in}{2.012947in}}%
\pgfpathlineto{\pgfqpoint{3.311132in}{2.012947in}}%
\pgfpathlineto{\pgfqpoint{3.311132in}{2.015896in}}%
\pgfpathlineto{\pgfqpoint{3.315673in}{2.015896in}}%
\pgfpathlineto{\pgfqpoint{3.315673in}{2.012947in}}%
\pgfpathmoveto{\pgfqpoint{3.315673in}{2.009997in}}%
\pgfpathlineto{\pgfqpoint{3.315673in}{2.009997in}}%
\pgfpathlineto{\pgfqpoint{3.315673in}{2.012947in}}%
\pgfpathlineto{\pgfqpoint{3.320214in}{2.012947in}}%
\pgfpathlineto{\pgfqpoint{3.320214in}{2.009997in}}%
\pgfpathmoveto{\pgfqpoint{3.315673in}{2.012947in}}%
\pgfpathlineto{\pgfqpoint{3.315673in}{2.012947in}}%
\pgfpathlineto{\pgfqpoint{3.315673in}{2.015896in}}%
\pgfpathlineto{\pgfqpoint{3.320214in}{2.015896in}}%
\pgfpathlineto{\pgfqpoint{3.320214in}{2.012947in}}%
\pgfpathmoveto{\pgfqpoint{3.320214in}{2.009997in}}%
\pgfpathlineto{\pgfqpoint{3.320214in}{2.009997in}}%
\pgfpathlineto{\pgfqpoint{3.320214in}{2.012947in}}%
\pgfpathlineto{\pgfqpoint{3.324756in}{2.012947in}}%
\pgfpathlineto{\pgfqpoint{3.324756in}{2.009997in}}%
\pgfpathmoveto{\pgfqpoint{3.320214in}{2.012947in}}%
\pgfpathlineto{\pgfqpoint{3.320214in}{2.012947in}}%
\pgfpathlineto{\pgfqpoint{3.320214in}{2.015896in}}%
\pgfpathlineto{\pgfqpoint{3.324756in}{2.015896in}}%
\pgfpathlineto{\pgfqpoint{3.324756in}{2.012947in}}%
\pgfpathmoveto{\pgfqpoint{3.324756in}{2.009997in}}%
\pgfpathlineto{\pgfqpoint{3.324756in}{2.009997in}}%
\pgfpathlineto{\pgfqpoint{3.324756in}{2.012947in}}%
\pgfpathlineto{\pgfqpoint{3.329297in}{2.012947in}}%
\pgfpathlineto{\pgfqpoint{3.329297in}{2.009997in}}%
\pgfpathmoveto{\pgfqpoint{3.324756in}{2.012947in}}%
\pgfpathlineto{\pgfqpoint{3.324756in}{2.012947in}}%
\pgfpathlineto{\pgfqpoint{3.324756in}{2.015896in}}%
\pgfpathlineto{\pgfqpoint{3.329297in}{2.015896in}}%
\pgfpathlineto{\pgfqpoint{3.329297in}{2.012947in}}%
\pgfpathmoveto{\pgfqpoint{3.329297in}{2.009997in}}%
\pgfpathlineto{\pgfqpoint{3.329297in}{2.009997in}}%
\pgfpathlineto{\pgfqpoint{3.329297in}{2.012947in}}%
\pgfpathlineto{\pgfqpoint{3.333838in}{2.012947in}}%
\pgfpathlineto{\pgfqpoint{3.333838in}{2.009997in}}%
\pgfpathmoveto{\pgfqpoint{3.329297in}{2.012947in}}%
\pgfpathlineto{\pgfqpoint{3.329297in}{2.012947in}}%
\pgfpathlineto{\pgfqpoint{3.329297in}{2.015896in}}%
\pgfpathlineto{\pgfqpoint{3.333838in}{2.015896in}}%
\pgfpathlineto{\pgfqpoint{3.333838in}{2.012947in}}%
\pgfpathmoveto{\pgfqpoint{3.333838in}{2.009997in}}%
\pgfpathlineto{\pgfqpoint{3.333838in}{2.009997in}}%
\pgfpathlineto{\pgfqpoint{3.333838in}{2.012947in}}%
\pgfpathlineto{\pgfqpoint{3.338379in}{2.012947in}}%
\pgfpathlineto{\pgfqpoint{3.338379in}{2.009997in}}%
\pgfpathmoveto{\pgfqpoint{3.333838in}{2.012947in}}%
\pgfpathlineto{\pgfqpoint{3.333838in}{2.012947in}}%
\pgfpathlineto{\pgfqpoint{3.333838in}{2.015896in}}%
\pgfpathlineto{\pgfqpoint{3.338379in}{2.015896in}}%
\pgfpathlineto{\pgfqpoint{3.338379in}{2.012947in}}%
\pgfpathmoveto{\pgfqpoint{3.338379in}{2.009997in}}%
\pgfpathlineto{\pgfqpoint{3.338379in}{2.009997in}}%
\pgfpathlineto{\pgfqpoint{3.338379in}{2.012947in}}%
\pgfpathlineto{\pgfqpoint{3.342920in}{2.012947in}}%
\pgfpathlineto{\pgfqpoint{3.342920in}{2.009997in}}%
\pgfpathmoveto{\pgfqpoint{3.338379in}{2.012947in}}%
\pgfpathlineto{\pgfqpoint{3.338379in}{2.012947in}}%
\pgfpathlineto{\pgfqpoint{3.338379in}{2.015896in}}%
\pgfpathlineto{\pgfqpoint{3.342920in}{2.015896in}}%
\pgfpathlineto{\pgfqpoint{3.342920in}{2.012947in}}%
\pgfpathmoveto{\pgfqpoint{3.342920in}{2.009997in}}%
\pgfpathlineto{\pgfqpoint{3.342920in}{2.009997in}}%
\pgfpathlineto{\pgfqpoint{3.342920in}{2.012947in}}%
\pgfpathlineto{\pgfqpoint{3.347461in}{2.012947in}}%
\pgfpathlineto{\pgfqpoint{3.347461in}{2.009997in}}%
\pgfpathmoveto{\pgfqpoint{3.342920in}{2.012947in}}%
\pgfpathlineto{\pgfqpoint{3.342920in}{2.012947in}}%
\pgfpathlineto{\pgfqpoint{3.342920in}{2.015896in}}%
\pgfpathlineto{\pgfqpoint{3.347461in}{2.015896in}}%
\pgfpathlineto{\pgfqpoint{3.347461in}{2.012947in}}%
\pgfpathmoveto{\pgfqpoint{3.347461in}{2.009997in}}%
\pgfpathlineto{\pgfqpoint{3.347461in}{2.009997in}}%
\pgfpathlineto{\pgfqpoint{3.347461in}{2.012947in}}%
\pgfpathlineto{\pgfqpoint{3.352002in}{2.012947in}}%
\pgfpathlineto{\pgfqpoint{3.352002in}{2.009997in}}%
\pgfpathmoveto{\pgfqpoint{3.347461in}{2.012947in}}%
\pgfpathlineto{\pgfqpoint{3.347461in}{2.012947in}}%
\pgfpathlineto{\pgfqpoint{3.347461in}{2.015896in}}%
\pgfpathlineto{\pgfqpoint{3.352002in}{2.015896in}}%
\pgfpathlineto{\pgfqpoint{3.352002in}{2.012947in}}%
\pgfpathmoveto{\pgfqpoint{3.352002in}{2.009997in}}%
\pgfpathlineto{\pgfqpoint{3.352002in}{2.009997in}}%
\pgfpathlineto{\pgfqpoint{3.352002in}{2.012947in}}%
\pgfpathlineto{\pgfqpoint{3.356543in}{2.012947in}}%
\pgfpathlineto{\pgfqpoint{3.356543in}{2.009997in}}%
\pgfpathmoveto{\pgfqpoint{3.352002in}{2.012947in}}%
\pgfpathlineto{\pgfqpoint{3.352002in}{2.012947in}}%
\pgfpathlineto{\pgfqpoint{3.352002in}{2.015896in}}%
\pgfpathlineto{\pgfqpoint{3.356543in}{2.015896in}}%
\pgfpathlineto{\pgfqpoint{3.356543in}{2.012947in}}%
\pgfpathmoveto{\pgfqpoint{3.356543in}{2.009997in}}%
\pgfpathlineto{\pgfqpoint{3.356543in}{2.009997in}}%
\pgfpathlineto{\pgfqpoint{3.356543in}{2.012947in}}%
\pgfpathlineto{\pgfqpoint{3.361084in}{2.012947in}}%
\pgfpathlineto{\pgfqpoint{3.361084in}{2.009997in}}%
\pgfpathmoveto{\pgfqpoint{3.356543in}{2.012947in}}%
\pgfpathlineto{\pgfqpoint{3.356543in}{2.012947in}}%
\pgfpathlineto{\pgfqpoint{3.356543in}{2.015896in}}%
\pgfpathlineto{\pgfqpoint{3.361084in}{2.015896in}}%
\pgfpathlineto{\pgfqpoint{3.361084in}{2.012947in}}%
\pgfpathmoveto{\pgfqpoint{3.361084in}{2.009997in}}%
\pgfpathlineto{\pgfqpoint{3.361084in}{2.009997in}}%
\pgfpathlineto{\pgfqpoint{3.361084in}{2.012947in}}%
\pgfpathlineto{\pgfqpoint{3.365625in}{2.012947in}}%
\pgfpathlineto{\pgfqpoint{3.365625in}{2.009997in}}%
\pgfpathmoveto{\pgfqpoint{3.361084in}{2.012947in}}%
\pgfpathlineto{\pgfqpoint{3.361084in}{2.012947in}}%
\pgfpathlineto{\pgfqpoint{3.361084in}{2.015896in}}%
\pgfpathlineto{\pgfqpoint{3.365625in}{2.015896in}}%
\pgfpathlineto{\pgfqpoint{3.365625in}{2.012947in}}%
\pgfpathmoveto{\pgfqpoint{3.220311in}{2.611639in}}%
\pgfpathlineto{\pgfqpoint{3.220311in}{2.611639in}}%
\pgfpathlineto{\pgfqpoint{3.220311in}{2.614589in}}%
\pgfpathlineto{\pgfqpoint{3.224852in}{2.614589in}}%
\pgfpathlineto{\pgfqpoint{3.224852in}{2.611639in}}%
\pgfpathmoveto{\pgfqpoint{3.224852in}{2.611639in}}%
\pgfpathlineto{\pgfqpoint{3.224852in}{2.611639in}}%
\pgfpathlineto{\pgfqpoint{3.224852in}{2.614589in}}%
\pgfpathlineto{\pgfqpoint{3.229393in}{2.614589in}}%
\pgfpathlineto{\pgfqpoint{3.229393in}{2.611639in}}%
\pgfpathmoveto{\pgfqpoint{3.229393in}{2.611639in}}%
\pgfpathlineto{\pgfqpoint{3.229393in}{2.611639in}}%
\pgfpathlineto{\pgfqpoint{3.229393in}{2.614589in}}%
\pgfpathlineto{\pgfqpoint{3.233934in}{2.614589in}}%
\pgfpathlineto{\pgfqpoint{3.233934in}{2.611639in}}%
\pgfpathmoveto{\pgfqpoint{3.233934in}{2.611639in}}%
\pgfpathlineto{\pgfqpoint{3.233934in}{2.611639in}}%
\pgfpathlineto{\pgfqpoint{3.233934in}{2.614589in}}%
\pgfpathlineto{\pgfqpoint{3.238475in}{2.614589in}}%
\pgfpathlineto{\pgfqpoint{3.238475in}{2.611639in}}%
\pgfpathmoveto{\pgfqpoint{3.238475in}{2.611639in}}%
\pgfpathlineto{\pgfqpoint{3.238475in}{2.611639in}}%
\pgfpathlineto{\pgfqpoint{3.238475in}{2.614589in}}%
\pgfpathlineto{\pgfqpoint{3.243016in}{2.614589in}}%
\pgfpathlineto{\pgfqpoint{3.243016in}{2.611639in}}%
\pgfpathmoveto{\pgfqpoint{3.243016in}{2.611639in}}%
\pgfpathlineto{\pgfqpoint{3.243016in}{2.611639in}}%
\pgfpathlineto{\pgfqpoint{3.243016in}{2.614589in}}%
\pgfpathlineto{\pgfqpoint{3.247557in}{2.614589in}}%
\pgfpathlineto{\pgfqpoint{3.247557in}{2.611639in}}%
\pgfpathmoveto{\pgfqpoint{3.247557in}{2.611639in}}%
\pgfpathlineto{\pgfqpoint{3.247557in}{2.611639in}}%
\pgfpathlineto{\pgfqpoint{3.247557in}{2.614589in}}%
\pgfpathlineto{\pgfqpoint{3.252098in}{2.614589in}}%
\pgfpathlineto{\pgfqpoint{3.252098in}{2.611639in}}%
\pgfpathmoveto{\pgfqpoint{3.252098in}{2.611639in}}%
\pgfpathlineto{\pgfqpoint{3.252098in}{2.611639in}}%
\pgfpathlineto{\pgfqpoint{3.252098in}{2.614589in}}%
\pgfpathlineto{\pgfqpoint{3.256640in}{2.614589in}}%
\pgfpathlineto{\pgfqpoint{3.256640in}{2.611639in}}%
\pgfpathmoveto{\pgfqpoint{3.256640in}{2.611639in}}%
\pgfpathlineto{\pgfqpoint{3.256640in}{2.611639in}}%
\pgfpathlineto{\pgfqpoint{3.256640in}{2.614589in}}%
\pgfpathlineto{\pgfqpoint{3.261181in}{2.614589in}}%
\pgfpathlineto{\pgfqpoint{3.261181in}{2.611639in}}%
\pgfpathmoveto{\pgfqpoint{3.261181in}{2.611639in}}%
\pgfpathlineto{\pgfqpoint{3.261181in}{2.611639in}}%
\pgfpathlineto{\pgfqpoint{3.261181in}{2.614589in}}%
\pgfpathlineto{\pgfqpoint{3.265722in}{2.614589in}}%
\pgfpathlineto{\pgfqpoint{3.265722in}{2.611639in}}%
\pgfpathmoveto{\pgfqpoint{3.265722in}{2.611639in}}%
\pgfpathlineto{\pgfqpoint{3.265722in}{2.611639in}}%
\pgfpathlineto{\pgfqpoint{3.265722in}{2.614589in}}%
\pgfpathlineto{\pgfqpoint{3.270263in}{2.614589in}}%
\pgfpathlineto{\pgfqpoint{3.270263in}{2.611639in}}%
\pgfpathmoveto{\pgfqpoint{3.270263in}{2.611639in}}%
\pgfpathlineto{\pgfqpoint{3.270263in}{2.611639in}}%
\pgfpathlineto{\pgfqpoint{3.270263in}{2.614589in}}%
\pgfpathlineto{\pgfqpoint{3.274804in}{2.614589in}}%
\pgfpathlineto{\pgfqpoint{3.274804in}{2.611639in}}%
\pgfpathmoveto{\pgfqpoint{3.274804in}{2.611639in}}%
\pgfpathlineto{\pgfqpoint{3.274804in}{2.611639in}}%
\pgfpathlineto{\pgfqpoint{3.274804in}{2.614589in}}%
\pgfpathlineto{\pgfqpoint{3.279345in}{2.614589in}}%
\pgfpathlineto{\pgfqpoint{3.279345in}{2.611639in}}%
\pgfpathmoveto{\pgfqpoint{3.279345in}{2.611639in}}%
\pgfpathlineto{\pgfqpoint{3.279345in}{2.611639in}}%
\pgfpathlineto{\pgfqpoint{3.279345in}{2.614589in}}%
\pgfpathlineto{\pgfqpoint{3.283886in}{2.614589in}}%
\pgfpathlineto{\pgfqpoint{3.283886in}{2.611639in}}%
\pgfpathmoveto{\pgfqpoint{3.283886in}{2.611639in}}%
\pgfpathlineto{\pgfqpoint{3.283886in}{2.611639in}}%
\pgfpathlineto{\pgfqpoint{3.283886in}{2.614589in}}%
\pgfpathlineto{\pgfqpoint{3.288427in}{2.614589in}}%
\pgfpathlineto{\pgfqpoint{3.288427in}{2.611639in}}%
\pgfpathmoveto{\pgfqpoint{3.288427in}{2.611639in}}%
\pgfpathlineto{\pgfqpoint{3.288427in}{2.611639in}}%
\pgfpathlineto{\pgfqpoint{3.288427in}{2.614589in}}%
\pgfpathlineto{\pgfqpoint{3.292968in}{2.614589in}}%
\pgfpathlineto{\pgfqpoint{3.292968in}{2.611639in}}%
\pgfpathmoveto{\pgfqpoint{3.292968in}{2.611639in}}%
\pgfpathlineto{\pgfqpoint{3.292968in}{2.611639in}}%
\pgfpathlineto{\pgfqpoint{3.292968in}{2.614589in}}%
\pgfpathlineto{\pgfqpoint{3.297509in}{2.614589in}}%
\pgfpathlineto{\pgfqpoint{3.297509in}{2.611639in}}%
\pgfpathmoveto{\pgfqpoint{3.297509in}{2.611639in}}%
\pgfpathlineto{\pgfqpoint{3.297509in}{2.611639in}}%
\pgfpathlineto{\pgfqpoint{3.297509in}{2.614589in}}%
\pgfpathlineto{\pgfqpoint{3.302050in}{2.614589in}}%
\pgfpathlineto{\pgfqpoint{3.302050in}{2.611639in}}%
\pgfpathmoveto{\pgfqpoint{3.302050in}{2.611639in}}%
\pgfpathlineto{\pgfqpoint{3.302050in}{2.611639in}}%
\pgfpathlineto{\pgfqpoint{3.302050in}{2.614589in}}%
\pgfpathlineto{\pgfqpoint{3.306591in}{2.614589in}}%
\pgfpathlineto{\pgfqpoint{3.306591in}{2.611639in}}%
\pgfpathmoveto{\pgfqpoint{3.306591in}{2.611639in}}%
\pgfpathlineto{\pgfqpoint{3.306591in}{2.611639in}}%
\pgfpathlineto{\pgfqpoint{3.306591in}{2.614589in}}%
\pgfpathlineto{\pgfqpoint{3.311132in}{2.614589in}}%
\pgfpathlineto{\pgfqpoint{3.311132in}{2.611639in}}%
\pgfpathmoveto{\pgfqpoint{3.311132in}{2.611639in}}%
\pgfpathlineto{\pgfqpoint{3.311132in}{2.611639in}}%
\pgfpathlineto{\pgfqpoint{3.311132in}{2.614589in}}%
\pgfpathlineto{\pgfqpoint{3.315673in}{2.614589in}}%
\pgfpathlineto{\pgfqpoint{3.315673in}{2.611639in}}%
\pgfpathmoveto{\pgfqpoint{3.315673in}{2.611639in}}%
\pgfpathlineto{\pgfqpoint{3.315673in}{2.611639in}}%
\pgfpathlineto{\pgfqpoint{3.315673in}{2.614589in}}%
\pgfpathlineto{\pgfqpoint{3.320214in}{2.614589in}}%
\pgfpathlineto{\pgfqpoint{3.320214in}{2.611639in}}%
\pgfpathmoveto{\pgfqpoint{3.320214in}{2.611639in}}%
\pgfpathlineto{\pgfqpoint{3.320214in}{2.611639in}}%
\pgfpathlineto{\pgfqpoint{3.320214in}{2.614589in}}%
\pgfpathlineto{\pgfqpoint{3.324756in}{2.614589in}}%
\pgfpathlineto{\pgfqpoint{3.324756in}{2.611639in}}%
\pgfpathmoveto{\pgfqpoint{3.324756in}{2.611639in}}%
\pgfpathlineto{\pgfqpoint{3.324756in}{2.611639in}}%
\pgfpathlineto{\pgfqpoint{3.324756in}{2.614589in}}%
\pgfpathlineto{\pgfqpoint{3.329297in}{2.614589in}}%
\pgfpathlineto{\pgfqpoint{3.329297in}{2.611639in}}%
\pgfpathmoveto{\pgfqpoint{3.329297in}{2.611639in}}%
\pgfpathlineto{\pgfqpoint{3.329297in}{2.611639in}}%
\pgfpathlineto{\pgfqpoint{3.329297in}{2.614589in}}%
\pgfpathlineto{\pgfqpoint{3.333838in}{2.614589in}}%
\pgfpathlineto{\pgfqpoint{3.333838in}{2.611639in}}%
\pgfpathmoveto{\pgfqpoint{3.333838in}{2.611639in}}%
\pgfpathlineto{\pgfqpoint{3.333838in}{2.611639in}}%
\pgfpathlineto{\pgfqpoint{3.333838in}{2.614589in}}%
\pgfpathlineto{\pgfqpoint{3.338379in}{2.614589in}}%
\pgfpathlineto{\pgfqpoint{3.338379in}{2.611639in}}%
\pgfpathmoveto{\pgfqpoint{3.338379in}{2.611639in}}%
\pgfpathlineto{\pgfqpoint{3.338379in}{2.611639in}}%
\pgfpathlineto{\pgfqpoint{3.338379in}{2.614589in}}%
\pgfpathlineto{\pgfqpoint{3.342920in}{2.614589in}}%
\pgfpathlineto{\pgfqpoint{3.342920in}{2.611639in}}%
\pgfpathmoveto{\pgfqpoint{3.342920in}{2.611639in}}%
\pgfpathlineto{\pgfqpoint{3.342920in}{2.611639in}}%
\pgfpathlineto{\pgfqpoint{3.342920in}{2.614589in}}%
\pgfpathlineto{\pgfqpoint{3.347461in}{2.614589in}}%
\pgfpathlineto{\pgfqpoint{3.347461in}{2.611639in}}%
\pgfpathmoveto{\pgfqpoint{3.347461in}{2.611639in}}%
\pgfpathlineto{\pgfqpoint{3.347461in}{2.611639in}}%
\pgfpathlineto{\pgfqpoint{3.347461in}{2.614589in}}%
\pgfpathlineto{\pgfqpoint{3.352002in}{2.614589in}}%
\pgfpathlineto{\pgfqpoint{3.352002in}{2.611639in}}%
\pgfpathmoveto{\pgfqpoint{3.352002in}{2.611639in}}%
\pgfpathlineto{\pgfqpoint{3.352002in}{2.611639in}}%
\pgfpathlineto{\pgfqpoint{3.352002in}{2.614589in}}%
\pgfpathlineto{\pgfqpoint{3.356543in}{2.614589in}}%
\pgfpathlineto{\pgfqpoint{3.356543in}{2.611639in}}%
\pgfpathmoveto{\pgfqpoint{3.356543in}{2.611639in}}%
\pgfpathlineto{\pgfqpoint{3.356543in}{2.611639in}}%
\pgfpathlineto{\pgfqpoint{3.356543in}{2.614589in}}%
\pgfpathlineto{\pgfqpoint{3.361084in}{2.614589in}}%
\pgfpathlineto{\pgfqpoint{3.361084in}{2.611639in}}%
\pgfpathmoveto{\pgfqpoint{3.361084in}{2.611639in}}%
\pgfpathlineto{\pgfqpoint{3.361084in}{2.611639in}}%
\pgfpathlineto{\pgfqpoint{3.361084in}{2.614589in}}%
\pgfpathlineto{\pgfqpoint{3.365625in}{2.614589in}}%
\pgfpathlineto{\pgfqpoint{3.365625in}{2.611639in}}%
\pgfpathmoveto{\pgfqpoint{3.365625in}{2.009997in}}%
\pgfpathlineto{\pgfqpoint{3.365625in}{2.009997in}}%
\pgfpathlineto{\pgfqpoint{3.365625in}{2.012947in}}%
\pgfpathlineto{\pgfqpoint{3.370166in}{2.012947in}}%
\pgfpathlineto{\pgfqpoint{3.370166in}{2.009997in}}%
\pgfpathmoveto{\pgfqpoint{3.365625in}{2.012947in}}%
\pgfpathlineto{\pgfqpoint{3.365625in}{2.012947in}}%
\pgfpathlineto{\pgfqpoint{3.365625in}{2.015896in}}%
\pgfpathlineto{\pgfqpoint{3.370166in}{2.015896in}}%
\pgfpathlineto{\pgfqpoint{3.370166in}{2.012947in}}%
\pgfpathmoveto{\pgfqpoint{3.370166in}{2.009997in}}%
\pgfpathlineto{\pgfqpoint{3.370166in}{2.009997in}}%
\pgfpathlineto{\pgfqpoint{3.370166in}{2.012947in}}%
\pgfpathlineto{\pgfqpoint{3.374707in}{2.012947in}}%
\pgfpathlineto{\pgfqpoint{3.374707in}{2.009997in}}%
\pgfpathmoveto{\pgfqpoint{3.370166in}{2.012947in}}%
\pgfpathlineto{\pgfqpoint{3.370166in}{2.012947in}}%
\pgfpathlineto{\pgfqpoint{3.370166in}{2.015896in}}%
\pgfpathlineto{\pgfqpoint{3.374707in}{2.015896in}}%
\pgfpathlineto{\pgfqpoint{3.374707in}{2.012947in}}%
\pgfpathmoveto{\pgfqpoint{3.374707in}{2.009997in}}%
\pgfpathlineto{\pgfqpoint{3.374707in}{2.009997in}}%
\pgfpathlineto{\pgfqpoint{3.374707in}{2.012947in}}%
\pgfpathlineto{\pgfqpoint{3.379248in}{2.012947in}}%
\pgfpathlineto{\pgfqpoint{3.379248in}{2.009997in}}%
\pgfpathmoveto{\pgfqpoint{3.374707in}{2.012947in}}%
\pgfpathlineto{\pgfqpoint{3.374707in}{2.012947in}}%
\pgfpathlineto{\pgfqpoint{3.374707in}{2.015896in}}%
\pgfpathlineto{\pgfqpoint{3.379248in}{2.015896in}}%
\pgfpathlineto{\pgfqpoint{3.379248in}{2.012947in}}%
\pgfpathmoveto{\pgfqpoint{3.379248in}{2.009997in}}%
\pgfpathlineto{\pgfqpoint{3.379248in}{2.009997in}}%
\pgfpathlineto{\pgfqpoint{3.379248in}{2.012947in}}%
\pgfpathlineto{\pgfqpoint{3.383789in}{2.012947in}}%
\pgfpathlineto{\pgfqpoint{3.383789in}{2.009997in}}%
\pgfpathmoveto{\pgfqpoint{3.379248in}{2.012947in}}%
\pgfpathlineto{\pgfqpoint{3.379248in}{2.012947in}}%
\pgfpathlineto{\pgfqpoint{3.379248in}{2.015896in}}%
\pgfpathlineto{\pgfqpoint{3.383789in}{2.015896in}}%
\pgfpathlineto{\pgfqpoint{3.383789in}{2.012947in}}%
\pgfpathmoveto{\pgfqpoint{3.383789in}{2.009997in}}%
\pgfpathlineto{\pgfqpoint{3.383789in}{2.009997in}}%
\pgfpathlineto{\pgfqpoint{3.383789in}{2.012947in}}%
\pgfpathlineto{\pgfqpoint{3.388330in}{2.012947in}}%
\pgfpathlineto{\pgfqpoint{3.388330in}{2.009997in}}%
\pgfpathmoveto{\pgfqpoint{3.383789in}{2.012947in}}%
\pgfpathlineto{\pgfqpoint{3.383789in}{2.012947in}}%
\pgfpathlineto{\pgfqpoint{3.383789in}{2.015896in}}%
\pgfpathlineto{\pgfqpoint{3.388330in}{2.015896in}}%
\pgfpathlineto{\pgfqpoint{3.388330in}{2.012947in}}%
\pgfpathmoveto{\pgfqpoint{3.388330in}{2.009997in}}%
\pgfpathlineto{\pgfqpoint{3.388330in}{2.009997in}}%
\pgfpathlineto{\pgfqpoint{3.388330in}{2.012947in}}%
\pgfpathlineto{\pgfqpoint{3.392871in}{2.012947in}}%
\pgfpathlineto{\pgfqpoint{3.392871in}{2.009997in}}%
\pgfpathmoveto{\pgfqpoint{3.388330in}{2.012947in}}%
\pgfpathlineto{\pgfqpoint{3.388330in}{2.012947in}}%
\pgfpathlineto{\pgfqpoint{3.388330in}{2.015896in}}%
\pgfpathlineto{\pgfqpoint{3.392871in}{2.015896in}}%
\pgfpathlineto{\pgfqpoint{3.392871in}{2.012947in}}%
\pgfpathmoveto{\pgfqpoint{3.392871in}{2.009997in}}%
\pgfpathlineto{\pgfqpoint{3.392871in}{2.009997in}}%
\pgfpathlineto{\pgfqpoint{3.392871in}{2.012947in}}%
\pgfpathlineto{\pgfqpoint{3.397412in}{2.012947in}}%
\pgfpathlineto{\pgfqpoint{3.397412in}{2.009997in}}%
\pgfpathmoveto{\pgfqpoint{3.392871in}{2.012947in}}%
\pgfpathlineto{\pgfqpoint{3.392871in}{2.012947in}}%
\pgfpathlineto{\pgfqpoint{3.392871in}{2.015896in}}%
\pgfpathlineto{\pgfqpoint{3.397412in}{2.015896in}}%
\pgfpathlineto{\pgfqpoint{3.397412in}{2.012947in}}%
\pgfpathmoveto{\pgfqpoint{3.397412in}{2.009997in}}%
\pgfpathlineto{\pgfqpoint{3.397412in}{2.009997in}}%
\pgfpathlineto{\pgfqpoint{3.397412in}{2.012947in}}%
\pgfpathlineto{\pgfqpoint{3.401954in}{2.012947in}}%
\pgfpathlineto{\pgfqpoint{3.401954in}{2.009997in}}%
\pgfpathmoveto{\pgfqpoint{3.397412in}{2.012947in}}%
\pgfpathlineto{\pgfqpoint{3.397412in}{2.012947in}}%
\pgfpathlineto{\pgfqpoint{3.397412in}{2.015896in}}%
\pgfpathlineto{\pgfqpoint{3.401954in}{2.015896in}}%
\pgfpathlineto{\pgfqpoint{3.401954in}{2.012947in}}%
\pgfpathmoveto{\pgfqpoint{3.401954in}{2.009997in}}%
\pgfpathlineto{\pgfqpoint{3.401954in}{2.009997in}}%
\pgfpathlineto{\pgfqpoint{3.401954in}{2.012947in}}%
\pgfpathlineto{\pgfqpoint{3.406495in}{2.012947in}}%
\pgfpathlineto{\pgfqpoint{3.406495in}{2.009997in}}%
\pgfpathmoveto{\pgfqpoint{3.401954in}{2.012947in}}%
\pgfpathlineto{\pgfqpoint{3.401954in}{2.012947in}}%
\pgfpathlineto{\pgfqpoint{3.401954in}{2.015896in}}%
\pgfpathlineto{\pgfqpoint{3.406495in}{2.015896in}}%
\pgfpathlineto{\pgfqpoint{3.406495in}{2.012947in}}%
\pgfpathmoveto{\pgfqpoint{3.406495in}{2.009997in}}%
\pgfpathlineto{\pgfqpoint{3.406495in}{2.009997in}}%
\pgfpathlineto{\pgfqpoint{3.406495in}{2.012947in}}%
\pgfpathlineto{\pgfqpoint{3.411036in}{2.012947in}}%
\pgfpathlineto{\pgfqpoint{3.411036in}{2.009997in}}%
\pgfpathmoveto{\pgfqpoint{3.406495in}{2.012947in}}%
\pgfpathlineto{\pgfqpoint{3.406495in}{2.012947in}}%
\pgfpathlineto{\pgfqpoint{3.406495in}{2.015896in}}%
\pgfpathlineto{\pgfqpoint{3.411036in}{2.015896in}}%
\pgfpathlineto{\pgfqpoint{3.411036in}{2.012947in}}%
\pgfpathmoveto{\pgfqpoint{3.411036in}{2.009997in}}%
\pgfpathlineto{\pgfqpoint{3.411036in}{2.009997in}}%
\pgfpathlineto{\pgfqpoint{3.411036in}{2.012947in}}%
\pgfpathlineto{\pgfqpoint{3.415577in}{2.012947in}}%
\pgfpathlineto{\pgfqpoint{3.415577in}{2.009997in}}%
\pgfpathmoveto{\pgfqpoint{3.411036in}{2.012947in}}%
\pgfpathlineto{\pgfqpoint{3.411036in}{2.012947in}}%
\pgfpathlineto{\pgfqpoint{3.411036in}{2.015896in}}%
\pgfpathlineto{\pgfqpoint{3.415577in}{2.015896in}}%
\pgfpathlineto{\pgfqpoint{3.415577in}{2.012947in}}%
\pgfpathmoveto{\pgfqpoint{3.415577in}{2.009997in}}%
\pgfpathlineto{\pgfqpoint{3.415577in}{2.009997in}}%
\pgfpathlineto{\pgfqpoint{3.415577in}{2.012947in}}%
\pgfpathlineto{\pgfqpoint{3.420118in}{2.012947in}}%
\pgfpathlineto{\pgfqpoint{3.420118in}{2.009997in}}%
\pgfpathmoveto{\pgfqpoint{3.415577in}{2.012947in}}%
\pgfpathlineto{\pgfqpoint{3.415577in}{2.012947in}}%
\pgfpathlineto{\pgfqpoint{3.415577in}{2.015896in}}%
\pgfpathlineto{\pgfqpoint{3.420118in}{2.015896in}}%
\pgfpathlineto{\pgfqpoint{3.420118in}{2.012947in}}%
\pgfpathmoveto{\pgfqpoint{3.420118in}{2.009997in}}%
\pgfpathlineto{\pgfqpoint{3.420118in}{2.009997in}}%
\pgfpathlineto{\pgfqpoint{3.420118in}{2.012947in}}%
\pgfpathlineto{\pgfqpoint{3.424659in}{2.012947in}}%
\pgfpathlineto{\pgfqpoint{3.424659in}{2.009997in}}%
\pgfpathmoveto{\pgfqpoint{3.420118in}{2.012947in}}%
\pgfpathlineto{\pgfqpoint{3.420118in}{2.012947in}}%
\pgfpathlineto{\pgfqpoint{3.420118in}{2.015896in}}%
\pgfpathlineto{\pgfqpoint{3.424659in}{2.015896in}}%
\pgfpathlineto{\pgfqpoint{3.424659in}{2.012947in}}%
\pgfpathmoveto{\pgfqpoint{3.424659in}{2.009997in}}%
\pgfpathlineto{\pgfqpoint{3.424659in}{2.009997in}}%
\pgfpathlineto{\pgfqpoint{3.424659in}{2.012947in}}%
\pgfpathlineto{\pgfqpoint{3.429200in}{2.012947in}}%
\pgfpathlineto{\pgfqpoint{3.429200in}{2.009997in}}%
\pgfpathmoveto{\pgfqpoint{3.424659in}{2.012947in}}%
\pgfpathlineto{\pgfqpoint{3.424659in}{2.012947in}}%
\pgfpathlineto{\pgfqpoint{3.424659in}{2.015896in}}%
\pgfpathlineto{\pgfqpoint{3.429200in}{2.015896in}}%
\pgfpathlineto{\pgfqpoint{3.429200in}{2.012947in}}%
\pgfpathmoveto{\pgfqpoint{3.429200in}{2.009997in}}%
\pgfpathlineto{\pgfqpoint{3.429200in}{2.009997in}}%
\pgfpathlineto{\pgfqpoint{3.429200in}{2.012947in}}%
\pgfpathlineto{\pgfqpoint{3.433741in}{2.012947in}}%
\pgfpathlineto{\pgfqpoint{3.433741in}{2.009997in}}%
\pgfpathmoveto{\pgfqpoint{3.429200in}{2.012947in}}%
\pgfpathlineto{\pgfqpoint{3.429200in}{2.012947in}}%
\pgfpathlineto{\pgfqpoint{3.429200in}{2.015896in}}%
\pgfpathlineto{\pgfqpoint{3.433741in}{2.015896in}}%
\pgfpathlineto{\pgfqpoint{3.433741in}{2.012947in}}%
\pgfpathmoveto{\pgfqpoint{3.433741in}{2.009997in}}%
\pgfpathlineto{\pgfqpoint{3.433741in}{2.009997in}}%
\pgfpathlineto{\pgfqpoint{3.433741in}{2.012947in}}%
\pgfpathlineto{\pgfqpoint{3.438282in}{2.012947in}}%
\pgfpathlineto{\pgfqpoint{3.438282in}{2.009997in}}%
\pgfpathmoveto{\pgfqpoint{3.433741in}{2.012947in}}%
\pgfpathlineto{\pgfqpoint{3.433741in}{2.012947in}}%
\pgfpathlineto{\pgfqpoint{3.433741in}{2.015896in}}%
\pgfpathlineto{\pgfqpoint{3.438282in}{2.015896in}}%
\pgfpathlineto{\pgfqpoint{3.438282in}{2.012947in}}%
\pgfpathmoveto{\pgfqpoint{3.438282in}{2.009997in}}%
\pgfpathlineto{\pgfqpoint{3.438282in}{2.009997in}}%
\pgfpathlineto{\pgfqpoint{3.438282in}{2.012947in}}%
\pgfpathlineto{\pgfqpoint{3.442823in}{2.012947in}}%
\pgfpathlineto{\pgfqpoint{3.442823in}{2.009997in}}%
\pgfpathmoveto{\pgfqpoint{3.438282in}{2.012947in}}%
\pgfpathlineto{\pgfqpoint{3.438282in}{2.012947in}}%
\pgfpathlineto{\pgfqpoint{3.438282in}{2.015896in}}%
\pgfpathlineto{\pgfqpoint{3.442823in}{2.015896in}}%
\pgfpathlineto{\pgfqpoint{3.442823in}{2.012947in}}%
\pgfpathmoveto{\pgfqpoint{3.442823in}{2.009997in}}%
\pgfpathlineto{\pgfqpoint{3.442823in}{2.009997in}}%
\pgfpathlineto{\pgfqpoint{3.442823in}{2.012947in}}%
\pgfpathlineto{\pgfqpoint{3.447364in}{2.012947in}}%
\pgfpathlineto{\pgfqpoint{3.447364in}{2.009997in}}%
\pgfpathmoveto{\pgfqpoint{3.442823in}{2.012947in}}%
\pgfpathlineto{\pgfqpoint{3.442823in}{2.012947in}}%
\pgfpathlineto{\pgfqpoint{3.442823in}{2.015896in}}%
\pgfpathlineto{\pgfqpoint{3.447364in}{2.015896in}}%
\pgfpathlineto{\pgfqpoint{3.447364in}{2.012947in}}%
\pgfpathmoveto{\pgfqpoint{3.447364in}{2.009997in}}%
\pgfpathlineto{\pgfqpoint{3.447364in}{2.009997in}}%
\pgfpathlineto{\pgfqpoint{3.447364in}{2.012947in}}%
\pgfpathlineto{\pgfqpoint{3.451905in}{2.012947in}}%
\pgfpathlineto{\pgfqpoint{3.451905in}{2.009997in}}%
\pgfpathmoveto{\pgfqpoint{3.447364in}{2.012947in}}%
\pgfpathlineto{\pgfqpoint{3.447364in}{2.012947in}}%
\pgfpathlineto{\pgfqpoint{3.447364in}{2.015896in}}%
\pgfpathlineto{\pgfqpoint{3.451905in}{2.015896in}}%
\pgfpathlineto{\pgfqpoint{3.451905in}{2.012947in}}%
\pgfpathmoveto{\pgfqpoint{3.451905in}{2.009997in}}%
\pgfpathlineto{\pgfqpoint{3.451905in}{2.009997in}}%
\pgfpathlineto{\pgfqpoint{3.451905in}{2.012947in}}%
\pgfpathlineto{\pgfqpoint{3.456446in}{2.012947in}}%
\pgfpathlineto{\pgfqpoint{3.456446in}{2.009997in}}%
\pgfpathmoveto{\pgfqpoint{3.451905in}{2.012947in}}%
\pgfpathlineto{\pgfqpoint{3.451905in}{2.012947in}}%
\pgfpathlineto{\pgfqpoint{3.451905in}{2.015896in}}%
\pgfpathlineto{\pgfqpoint{3.456446in}{2.015896in}}%
\pgfpathlineto{\pgfqpoint{3.456446in}{2.012947in}}%
\pgfpathmoveto{\pgfqpoint{3.456446in}{2.009997in}}%
\pgfpathlineto{\pgfqpoint{3.456446in}{2.009997in}}%
\pgfpathlineto{\pgfqpoint{3.456446in}{2.012947in}}%
\pgfpathlineto{\pgfqpoint{3.460987in}{2.012947in}}%
\pgfpathlineto{\pgfqpoint{3.460987in}{2.009997in}}%
\pgfpathmoveto{\pgfqpoint{3.456446in}{2.012947in}}%
\pgfpathlineto{\pgfqpoint{3.456446in}{2.012947in}}%
\pgfpathlineto{\pgfqpoint{3.456446in}{2.015896in}}%
\pgfpathlineto{\pgfqpoint{3.460987in}{2.015896in}}%
\pgfpathlineto{\pgfqpoint{3.460987in}{2.012947in}}%
\pgfpathmoveto{\pgfqpoint{3.460987in}{2.009997in}}%
\pgfpathlineto{\pgfqpoint{3.460987in}{2.009997in}}%
\pgfpathlineto{\pgfqpoint{3.460987in}{2.012947in}}%
\pgfpathlineto{\pgfqpoint{3.465528in}{2.012947in}}%
\pgfpathlineto{\pgfqpoint{3.465528in}{2.009997in}}%
\pgfpathmoveto{\pgfqpoint{3.460987in}{2.012947in}}%
\pgfpathlineto{\pgfqpoint{3.460987in}{2.012947in}}%
\pgfpathlineto{\pgfqpoint{3.460987in}{2.015896in}}%
\pgfpathlineto{\pgfqpoint{3.465528in}{2.015896in}}%
\pgfpathlineto{\pgfqpoint{3.465528in}{2.012947in}}%
\pgfpathmoveto{\pgfqpoint{3.465528in}{2.009997in}}%
\pgfpathlineto{\pgfqpoint{3.465528in}{2.009997in}}%
\pgfpathlineto{\pgfqpoint{3.465528in}{2.012947in}}%
\pgfpathlineto{\pgfqpoint{3.470069in}{2.012947in}}%
\pgfpathlineto{\pgfqpoint{3.470069in}{2.009997in}}%
\pgfpathmoveto{\pgfqpoint{3.465528in}{2.012947in}}%
\pgfpathlineto{\pgfqpoint{3.465528in}{2.012947in}}%
\pgfpathlineto{\pgfqpoint{3.465528in}{2.015896in}}%
\pgfpathlineto{\pgfqpoint{3.470069in}{2.015896in}}%
\pgfpathlineto{\pgfqpoint{3.470069in}{2.012947in}}%
\pgfpathmoveto{\pgfqpoint{3.470069in}{2.009997in}}%
\pgfpathlineto{\pgfqpoint{3.470069in}{2.009997in}}%
\pgfpathlineto{\pgfqpoint{3.470069in}{2.012947in}}%
\pgfpathlineto{\pgfqpoint{3.474610in}{2.012947in}}%
\pgfpathlineto{\pgfqpoint{3.474610in}{2.009997in}}%
\pgfpathmoveto{\pgfqpoint{3.470069in}{2.012947in}}%
\pgfpathlineto{\pgfqpoint{3.470069in}{2.012947in}}%
\pgfpathlineto{\pgfqpoint{3.470069in}{2.015896in}}%
\pgfpathlineto{\pgfqpoint{3.474610in}{2.015896in}}%
\pgfpathlineto{\pgfqpoint{3.474610in}{2.012947in}}%
\pgfpathmoveto{\pgfqpoint{3.474610in}{2.009997in}}%
\pgfpathlineto{\pgfqpoint{3.474610in}{2.009997in}}%
\pgfpathlineto{\pgfqpoint{3.474610in}{2.012947in}}%
\pgfpathlineto{\pgfqpoint{3.479151in}{2.012947in}}%
\pgfpathlineto{\pgfqpoint{3.479151in}{2.009997in}}%
\pgfpathmoveto{\pgfqpoint{3.474610in}{2.012947in}}%
\pgfpathlineto{\pgfqpoint{3.474610in}{2.012947in}}%
\pgfpathlineto{\pgfqpoint{3.474610in}{2.015896in}}%
\pgfpathlineto{\pgfqpoint{3.479151in}{2.015896in}}%
\pgfpathlineto{\pgfqpoint{3.479151in}{2.012947in}}%
\pgfpathmoveto{\pgfqpoint{3.479151in}{2.009997in}}%
\pgfpathlineto{\pgfqpoint{3.479151in}{2.009997in}}%
\pgfpathlineto{\pgfqpoint{3.479151in}{2.012947in}}%
\pgfpathlineto{\pgfqpoint{3.483692in}{2.012947in}}%
\pgfpathlineto{\pgfqpoint{3.483692in}{2.009997in}}%
\pgfpathmoveto{\pgfqpoint{3.479151in}{2.012947in}}%
\pgfpathlineto{\pgfqpoint{3.479151in}{2.012947in}}%
\pgfpathlineto{\pgfqpoint{3.479151in}{2.015896in}}%
\pgfpathlineto{\pgfqpoint{3.483692in}{2.015896in}}%
\pgfpathlineto{\pgfqpoint{3.483692in}{2.012947in}}%
\pgfpathmoveto{\pgfqpoint{3.483692in}{2.009997in}}%
\pgfpathlineto{\pgfqpoint{3.483692in}{2.009997in}}%
\pgfpathlineto{\pgfqpoint{3.483692in}{2.012947in}}%
\pgfpathlineto{\pgfqpoint{3.488234in}{2.012947in}}%
\pgfpathlineto{\pgfqpoint{3.488234in}{2.009997in}}%
\pgfpathmoveto{\pgfqpoint{3.483692in}{2.012947in}}%
\pgfpathlineto{\pgfqpoint{3.483692in}{2.012947in}}%
\pgfpathlineto{\pgfqpoint{3.483692in}{2.015896in}}%
\pgfpathlineto{\pgfqpoint{3.488234in}{2.015896in}}%
\pgfpathlineto{\pgfqpoint{3.488234in}{2.012947in}}%
\pgfpathmoveto{\pgfqpoint{3.488234in}{2.009997in}}%
\pgfpathlineto{\pgfqpoint{3.488234in}{2.009997in}}%
\pgfpathlineto{\pgfqpoint{3.488234in}{2.012947in}}%
\pgfpathlineto{\pgfqpoint{3.492775in}{2.012947in}}%
\pgfpathlineto{\pgfqpoint{3.492775in}{2.009997in}}%
\pgfpathmoveto{\pgfqpoint{3.488234in}{2.012947in}}%
\pgfpathlineto{\pgfqpoint{3.488234in}{2.012947in}}%
\pgfpathlineto{\pgfqpoint{3.488234in}{2.015896in}}%
\pgfpathlineto{\pgfqpoint{3.492775in}{2.015896in}}%
\pgfpathlineto{\pgfqpoint{3.492775in}{2.012947in}}%
\pgfpathmoveto{\pgfqpoint{3.492775in}{2.009997in}}%
\pgfpathlineto{\pgfqpoint{3.492775in}{2.009997in}}%
\pgfpathlineto{\pgfqpoint{3.492775in}{2.012947in}}%
\pgfpathlineto{\pgfqpoint{3.497316in}{2.012947in}}%
\pgfpathlineto{\pgfqpoint{3.497316in}{2.009997in}}%
\pgfpathmoveto{\pgfqpoint{3.492775in}{2.012947in}}%
\pgfpathlineto{\pgfqpoint{3.492775in}{2.012947in}}%
\pgfpathlineto{\pgfqpoint{3.492775in}{2.015896in}}%
\pgfpathlineto{\pgfqpoint{3.497316in}{2.015896in}}%
\pgfpathlineto{\pgfqpoint{3.497316in}{2.012947in}}%
\pgfpathmoveto{\pgfqpoint{3.497316in}{2.009997in}}%
\pgfpathlineto{\pgfqpoint{3.497316in}{2.009997in}}%
\pgfpathlineto{\pgfqpoint{3.497316in}{2.012947in}}%
\pgfpathlineto{\pgfqpoint{3.501857in}{2.012947in}}%
\pgfpathlineto{\pgfqpoint{3.501857in}{2.009997in}}%
\pgfpathmoveto{\pgfqpoint{3.497316in}{2.012947in}}%
\pgfpathlineto{\pgfqpoint{3.497316in}{2.012947in}}%
\pgfpathlineto{\pgfqpoint{3.497316in}{2.015896in}}%
\pgfpathlineto{\pgfqpoint{3.501857in}{2.015896in}}%
\pgfpathlineto{\pgfqpoint{3.501857in}{2.012947in}}%
\pgfpathmoveto{\pgfqpoint{3.501857in}{2.009997in}}%
\pgfpathlineto{\pgfqpoint{3.501857in}{2.009997in}}%
\pgfpathlineto{\pgfqpoint{3.501857in}{2.012947in}}%
\pgfpathlineto{\pgfqpoint{3.506398in}{2.012947in}}%
\pgfpathlineto{\pgfqpoint{3.506398in}{2.009997in}}%
\pgfpathmoveto{\pgfqpoint{3.501857in}{2.012947in}}%
\pgfpathlineto{\pgfqpoint{3.501857in}{2.012947in}}%
\pgfpathlineto{\pgfqpoint{3.501857in}{2.015896in}}%
\pgfpathlineto{\pgfqpoint{3.506398in}{2.015896in}}%
\pgfpathlineto{\pgfqpoint{3.506398in}{2.012947in}}%
\pgfpathmoveto{\pgfqpoint{3.506398in}{2.009997in}}%
\pgfpathlineto{\pgfqpoint{3.506398in}{2.009997in}}%
\pgfpathlineto{\pgfqpoint{3.506398in}{2.012947in}}%
\pgfpathlineto{\pgfqpoint{3.510939in}{2.012947in}}%
\pgfpathlineto{\pgfqpoint{3.510939in}{2.009997in}}%
\pgfpathmoveto{\pgfqpoint{3.506398in}{2.012947in}}%
\pgfpathlineto{\pgfqpoint{3.506398in}{2.012947in}}%
\pgfpathlineto{\pgfqpoint{3.506398in}{2.015896in}}%
\pgfpathlineto{\pgfqpoint{3.510939in}{2.015896in}}%
\pgfpathlineto{\pgfqpoint{3.510939in}{2.012947in}}%
\pgfpathmoveto{\pgfqpoint{3.365625in}{2.611639in}}%
\pgfpathlineto{\pgfqpoint{3.365625in}{2.611639in}}%
\pgfpathlineto{\pgfqpoint{3.365625in}{2.614589in}}%
\pgfpathlineto{\pgfqpoint{3.370166in}{2.614589in}}%
\pgfpathlineto{\pgfqpoint{3.370166in}{2.611639in}}%
\pgfpathmoveto{\pgfqpoint{3.370166in}{2.611639in}}%
\pgfpathlineto{\pgfqpoint{3.370166in}{2.611639in}}%
\pgfpathlineto{\pgfqpoint{3.370166in}{2.614589in}}%
\pgfpathlineto{\pgfqpoint{3.374707in}{2.614589in}}%
\pgfpathlineto{\pgfqpoint{3.374707in}{2.611639in}}%
\pgfpathmoveto{\pgfqpoint{3.374707in}{2.611639in}}%
\pgfpathlineto{\pgfqpoint{3.374707in}{2.611639in}}%
\pgfpathlineto{\pgfqpoint{3.374707in}{2.614589in}}%
\pgfpathlineto{\pgfqpoint{3.379248in}{2.614589in}}%
\pgfpathlineto{\pgfqpoint{3.379248in}{2.611639in}}%
\pgfpathmoveto{\pgfqpoint{3.379248in}{2.611639in}}%
\pgfpathlineto{\pgfqpoint{3.379248in}{2.611639in}}%
\pgfpathlineto{\pgfqpoint{3.379248in}{2.614589in}}%
\pgfpathlineto{\pgfqpoint{3.383789in}{2.614589in}}%
\pgfpathlineto{\pgfqpoint{3.383789in}{2.611639in}}%
\pgfpathmoveto{\pgfqpoint{3.383789in}{2.611639in}}%
\pgfpathlineto{\pgfqpoint{3.383789in}{2.611639in}}%
\pgfpathlineto{\pgfqpoint{3.383789in}{2.614589in}}%
\pgfpathlineto{\pgfqpoint{3.388330in}{2.614589in}}%
\pgfpathlineto{\pgfqpoint{3.388330in}{2.611639in}}%
\pgfpathmoveto{\pgfqpoint{3.388330in}{2.611639in}}%
\pgfpathlineto{\pgfqpoint{3.388330in}{2.611639in}}%
\pgfpathlineto{\pgfqpoint{3.388330in}{2.614589in}}%
\pgfpathlineto{\pgfqpoint{3.392871in}{2.614589in}}%
\pgfpathlineto{\pgfqpoint{3.392871in}{2.611639in}}%
\pgfpathmoveto{\pgfqpoint{3.392871in}{2.611639in}}%
\pgfpathlineto{\pgfqpoint{3.392871in}{2.611639in}}%
\pgfpathlineto{\pgfqpoint{3.392871in}{2.614589in}}%
\pgfpathlineto{\pgfqpoint{3.397412in}{2.614589in}}%
\pgfpathlineto{\pgfqpoint{3.397412in}{2.611639in}}%
\pgfpathmoveto{\pgfqpoint{3.397412in}{2.611639in}}%
\pgfpathlineto{\pgfqpoint{3.397412in}{2.611639in}}%
\pgfpathlineto{\pgfqpoint{3.397412in}{2.614589in}}%
\pgfpathlineto{\pgfqpoint{3.401954in}{2.614589in}}%
\pgfpathlineto{\pgfqpoint{3.401954in}{2.611639in}}%
\pgfpathmoveto{\pgfqpoint{3.401954in}{2.611639in}}%
\pgfpathlineto{\pgfqpoint{3.401954in}{2.611639in}}%
\pgfpathlineto{\pgfqpoint{3.401954in}{2.614589in}}%
\pgfpathlineto{\pgfqpoint{3.406495in}{2.614589in}}%
\pgfpathlineto{\pgfqpoint{3.406495in}{2.611639in}}%
\pgfpathmoveto{\pgfqpoint{3.406495in}{2.611639in}}%
\pgfpathlineto{\pgfqpoint{3.406495in}{2.611639in}}%
\pgfpathlineto{\pgfqpoint{3.406495in}{2.614589in}}%
\pgfpathlineto{\pgfqpoint{3.411036in}{2.614589in}}%
\pgfpathlineto{\pgfqpoint{3.411036in}{2.611639in}}%
\pgfpathmoveto{\pgfqpoint{3.411036in}{2.611639in}}%
\pgfpathlineto{\pgfqpoint{3.411036in}{2.611639in}}%
\pgfpathlineto{\pgfqpoint{3.411036in}{2.614589in}}%
\pgfpathlineto{\pgfqpoint{3.415577in}{2.614589in}}%
\pgfpathlineto{\pgfqpoint{3.415577in}{2.611639in}}%
\pgfpathmoveto{\pgfqpoint{3.415577in}{2.611639in}}%
\pgfpathlineto{\pgfqpoint{3.415577in}{2.611639in}}%
\pgfpathlineto{\pgfqpoint{3.415577in}{2.614589in}}%
\pgfpathlineto{\pgfqpoint{3.420118in}{2.614589in}}%
\pgfpathlineto{\pgfqpoint{3.420118in}{2.611639in}}%
\pgfpathmoveto{\pgfqpoint{3.420118in}{2.611639in}}%
\pgfpathlineto{\pgfqpoint{3.420118in}{2.611639in}}%
\pgfpathlineto{\pgfqpoint{3.420118in}{2.614589in}}%
\pgfpathlineto{\pgfqpoint{3.424659in}{2.614589in}}%
\pgfpathlineto{\pgfqpoint{3.424659in}{2.611639in}}%
\pgfpathmoveto{\pgfqpoint{3.424659in}{2.611639in}}%
\pgfpathlineto{\pgfqpoint{3.424659in}{2.611639in}}%
\pgfpathlineto{\pgfqpoint{3.424659in}{2.614589in}}%
\pgfpathlineto{\pgfqpoint{3.429200in}{2.614589in}}%
\pgfpathlineto{\pgfqpoint{3.429200in}{2.611639in}}%
\pgfpathmoveto{\pgfqpoint{3.429200in}{2.611639in}}%
\pgfpathlineto{\pgfqpoint{3.429200in}{2.611639in}}%
\pgfpathlineto{\pgfqpoint{3.429200in}{2.614589in}}%
\pgfpathlineto{\pgfqpoint{3.433741in}{2.614589in}}%
\pgfpathlineto{\pgfqpoint{3.433741in}{2.611639in}}%
\pgfpathmoveto{\pgfqpoint{3.433741in}{2.611639in}}%
\pgfpathlineto{\pgfqpoint{3.433741in}{2.611639in}}%
\pgfpathlineto{\pgfqpoint{3.433741in}{2.614589in}}%
\pgfpathlineto{\pgfqpoint{3.438282in}{2.614589in}}%
\pgfpathlineto{\pgfqpoint{3.438282in}{2.611639in}}%
\pgfpathmoveto{\pgfqpoint{3.438282in}{2.611639in}}%
\pgfpathlineto{\pgfqpoint{3.438282in}{2.611639in}}%
\pgfpathlineto{\pgfqpoint{3.438282in}{2.614589in}}%
\pgfpathlineto{\pgfqpoint{3.442823in}{2.614589in}}%
\pgfpathlineto{\pgfqpoint{3.442823in}{2.611639in}}%
\pgfpathmoveto{\pgfqpoint{3.442823in}{2.611639in}}%
\pgfpathlineto{\pgfqpoint{3.442823in}{2.611639in}}%
\pgfpathlineto{\pgfqpoint{3.442823in}{2.614589in}}%
\pgfpathlineto{\pgfqpoint{3.447364in}{2.614589in}}%
\pgfpathlineto{\pgfqpoint{3.447364in}{2.611639in}}%
\pgfpathmoveto{\pgfqpoint{3.447364in}{2.611639in}}%
\pgfpathlineto{\pgfqpoint{3.447364in}{2.611639in}}%
\pgfpathlineto{\pgfqpoint{3.447364in}{2.614589in}}%
\pgfpathlineto{\pgfqpoint{3.451905in}{2.614589in}}%
\pgfpathlineto{\pgfqpoint{3.451905in}{2.611639in}}%
\pgfpathmoveto{\pgfqpoint{3.451905in}{2.611639in}}%
\pgfpathlineto{\pgfqpoint{3.451905in}{2.611639in}}%
\pgfpathlineto{\pgfqpoint{3.451905in}{2.614589in}}%
\pgfpathlineto{\pgfqpoint{3.456446in}{2.614589in}}%
\pgfpathlineto{\pgfqpoint{3.456446in}{2.611639in}}%
\pgfpathmoveto{\pgfqpoint{3.456446in}{2.611639in}}%
\pgfpathlineto{\pgfqpoint{3.456446in}{2.611639in}}%
\pgfpathlineto{\pgfqpoint{3.456446in}{2.614589in}}%
\pgfpathlineto{\pgfqpoint{3.460987in}{2.614589in}}%
\pgfpathlineto{\pgfqpoint{3.460987in}{2.611639in}}%
\pgfpathmoveto{\pgfqpoint{3.460987in}{2.611639in}}%
\pgfpathlineto{\pgfqpoint{3.460987in}{2.611639in}}%
\pgfpathlineto{\pgfqpoint{3.460987in}{2.614589in}}%
\pgfpathlineto{\pgfqpoint{3.465528in}{2.614589in}}%
\pgfpathlineto{\pgfqpoint{3.465528in}{2.611639in}}%
\pgfpathmoveto{\pgfqpoint{3.465528in}{2.611639in}}%
\pgfpathlineto{\pgfqpoint{3.465528in}{2.611639in}}%
\pgfpathlineto{\pgfqpoint{3.465528in}{2.614589in}}%
\pgfpathlineto{\pgfqpoint{3.470069in}{2.614589in}}%
\pgfpathlineto{\pgfqpoint{3.470069in}{2.611639in}}%
\pgfpathmoveto{\pgfqpoint{3.470069in}{2.611639in}}%
\pgfpathlineto{\pgfqpoint{3.470069in}{2.611639in}}%
\pgfpathlineto{\pgfqpoint{3.470069in}{2.614589in}}%
\pgfpathlineto{\pgfqpoint{3.474610in}{2.614589in}}%
\pgfpathlineto{\pgfqpoint{3.474610in}{2.611639in}}%
\pgfpathmoveto{\pgfqpoint{3.474610in}{2.611639in}}%
\pgfpathlineto{\pgfqpoint{3.474610in}{2.611639in}}%
\pgfpathlineto{\pgfqpoint{3.474610in}{2.614589in}}%
\pgfpathlineto{\pgfqpoint{3.479151in}{2.614589in}}%
\pgfpathlineto{\pgfqpoint{3.479151in}{2.611639in}}%
\pgfpathmoveto{\pgfqpoint{3.479151in}{2.611639in}}%
\pgfpathlineto{\pgfqpoint{3.479151in}{2.611639in}}%
\pgfpathlineto{\pgfqpoint{3.479151in}{2.614589in}}%
\pgfpathlineto{\pgfqpoint{3.483692in}{2.614589in}}%
\pgfpathlineto{\pgfqpoint{3.483692in}{2.611639in}}%
\pgfpathmoveto{\pgfqpoint{3.483692in}{2.611639in}}%
\pgfpathlineto{\pgfqpoint{3.483692in}{2.611639in}}%
\pgfpathlineto{\pgfqpoint{3.483692in}{2.614589in}}%
\pgfpathlineto{\pgfqpoint{3.488234in}{2.614589in}}%
\pgfpathlineto{\pgfqpoint{3.488234in}{2.611639in}}%
\pgfpathmoveto{\pgfqpoint{3.488234in}{2.611639in}}%
\pgfpathlineto{\pgfqpoint{3.488234in}{2.611639in}}%
\pgfpathlineto{\pgfqpoint{3.488234in}{2.614589in}}%
\pgfpathlineto{\pgfqpoint{3.492775in}{2.614589in}}%
\pgfpathlineto{\pgfqpoint{3.492775in}{2.611639in}}%
\pgfpathmoveto{\pgfqpoint{3.492775in}{2.611639in}}%
\pgfpathlineto{\pgfqpoint{3.492775in}{2.611639in}}%
\pgfpathlineto{\pgfqpoint{3.492775in}{2.614589in}}%
\pgfpathlineto{\pgfqpoint{3.497316in}{2.614589in}}%
\pgfpathlineto{\pgfqpoint{3.497316in}{2.611639in}}%
\pgfpathmoveto{\pgfqpoint{3.497316in}{2.611639in}}%
\pgfpathlineto{\pgfqpoint{3.497316in}{2.611639in}}%
\pgfpathlineto{\pgfqpoint{3.497316in}{2.614589in}}%
\pgfpathlineto{\pgfqpoint{3.501857in}{2.614589in}}%
\pgfpathlineto{\pgfqpoint{3.501857in}{2.611639in}}%
\pgfpathmoveto{\pgfqpoint{3.501857in}{2.611639in}}%
\pgfpathlineto{\pgfqpoint{3.501857in}{2.611639in}}%
\pgfpathlineto{\pgfqpoint{3.501857in}{2.614589in}}%
\pgfpathlineto{\pgfqpoint{3.506398in}{2.614589in}}%
\pgfpathlineto{\pgfqpoint{3.506398in}{2.611639in}}%
\pgfpathmoveto{\pgfqpoint{3.506398in}{2.611639in}}%
\pgfpathlineto{\pgfqpoint{3.506398in}{2.611639in}}%
\pgfpathlineto{\pgfqpoint{3.506398in}{2.614589in}}%
\pgfpathlineto{\pgfqpoint{3.510939in}{2.614589in}}%
\pgfpathlineto{\pgfqpoint{3.510939in}{2.611639in}}%
\pgfpathmoveto{\pgfqpoint{3.510939in}{2.009997in}}%
\pgfpathlineto{\pgfqpoint{3.510939in}{2.009997in}}%
\pgfpathlineto{\pgfqpoint{3.510939in}{2.012947in}}%
\pgfpathlineto{\pgfqpoint{3.515480in}{2.012947in}}%
\pgfpathlineto{\pgfqpoint{3.515480in}{2.009997in}}%
\pgfpathmoveto{\pgfqpoint{3.510939in}{2.012947in}}%
\pgfpathlineto{\pgfqpoint{3.510939in}{2.012947in}}%
\pgfpathlineto{\pgfqpoint{3.510939in}{2.015896in}}%
\pgfpathlineto{\pgfqpoint{3.515480in}{2.015896in}}%
\pgfpathlineto{\pgfqpoint{3.515480in}{2.012947in}}%
\pgfpathmoveto{\pgfqpoint{3.515480in}{2.009997in}}%
\pgfpathlineto{\pgfqpoint{3.515480in}{2.009997in}}%
\pgfpathlineto{\pgfqpoint{3.515480in}{2.012947in}}%
\pgfpathlineto{\pgfqpoint{3.520021in}{2.012947in}}%
\pgfpathlineto{\pgfqpoint{3.520021in}{2.009997in}}%
\pgfpathmoveto{\pgfqpoint{3.515480in}{2.012947in}}%
\pgfpathlineto{\pgfqpoint{3.515480in}{2.012947in}}%
\pgfpathlineto{\pgfqpoint{3.515480in}{2.015896in}}%
\pgfpathlineto{\pgfqpoint{3.520021in}{2.015896in}}%
\pgfpathlineto{\pgfqpoint{3.520021in}{2.012947in}}%
\pgfpathmoveto{\pgfqpoint{3.520021in}{2.009997in}}%
\pgfpathlineto{\pgfqpoint{3.520021in}{2.009997in}}%
\pgfpathlineto{\pgfqpoint{3.520021in}{2.012947in}}%
\pgfpathlineto{\pgfqpoint{3.524562in}{2.012947in}}%
\pgfpathlineto{\pgfqpoint{3.524562in}{2.009997in}}%
\pgfpathmoveto{\pgfqpoint{3.520021in}{2.012947in}}%
\pgfpathlineto{\pgfqpoint{3.520021in}{2.012947in}}%
\pgfpathlineto{\pgfqpoint{3.520021in}{2.015896in}}%
\pgfpathlineto{\pgfqpoint{3.524562in}{2.015896in}}%
\pgfpathlineto{\pgfqpoint{3.524562in}{2.012947in}}%
\pgfpathmoveto{\pgfqpoint{3.524562in}{2.009997in}}%
\pgfpathlineto{\pgfqpoint{3.524562in}{2.009997in}}%
\pgfpathlineto{\pgfqpoint{3.524562in}{2.012947in}}%
\pgfpathlineto{\pgfqpoint{3.529103in}{2.012947in}}%
\pgfpathlineto{\pgfqpoint{3.529103in}{2.009997in}}%
\pgfpathmoveto{\pgfqpoint{3.524562in}{2.012947in}}%
\pgfpathlineto{\pgfqpoint{3.524562in}{2.012947in}}%
\pgfpathlineto{\pgfqpoint{3.524562in}{2.015896in}}%
\pgfpathlineto{\pgfqpoint{3.529103in}{2.015896in}}%
\pgfpathlineto{\pgfqpoint{3.529103in}{2.012947in}}%
\pgfpathmoveto{\pgfqpoint{3.529103in}{2.009997in}}%
\pgfpathlineto{\pgfqpoint{3.529103in}{2.009997in}}%
\pgfpathlineto{\pgfqpoint{3.529103in}{2.012947in}}%
\pgfpathlineto{\pgfqpoint{3.533644in}{2.012947in}}%
\pgfpathlineto{\pgfqpoint{3.533644in}{2.009997in}}%
\pgfpathmoveto{\pgfqpoint{3.529103in}{2.012947in}}%
\pgfpathlineto{\pgfqpoint{3.529103in}{2.012947in}}%
\pgfpathlineto{\pgfqpoint{3.529103in}{2.015896in}}%
\pgfpathlineto{\pgfqpoint{3.533644in}{2.015896in}}%
\pgfpathlineto{\pgfqpoint{3.533644in}{2.012947in}}%
\pgfpathmoveto{\pgfqpoint{3.533644in}{2.009997in}}%
\pgfpathlineto{\pgfqpoint{3.533644in}{2.009997in}}%
\pgfpathlineto{\pgfqpoint{3.533644in}{2.012947in}}%
\pgfpathlineto{\pgfqpoint{3.538185in}{2.012947in}}%
\pgfpathlineto{\pgfqpoint{3.538185in}{2.009997in}}%
\pgfpathmoveto{\pgfqpoint{3.533644in}{2.012947in}}%
\pgfpathlineto{\pgfqpoint{3.533644in}{2.012947in}}%
\pgfpathlineto{\pgfqpoint{3.533644in}{2.015896in}}%
\pgfpathlineto{\pgfqpoint{3.538185in}{2.015896in}}%
\pgfpathlineto{\pgfqpoint{3.538185in}{2.012947in}}%
\pgfpathmoveto{\pgfqpoint{3.538185in}{2.009997in}}%
\pgfpathlineto{\pgfqpoint{3.538185in}{2.009997in}}%
\pgfpathlineto{\pgfqpoint{3.538185in}{2.012947in}}%
\pgfpathlineto{\pgfqpoint{3.542726in}{2.012947in}}%
\pgfpathlineto{\pgfqpoint{3.542726in}{2.009997in}}%
\pgfpathmoveto{\pgfqpoint{3.538185in}{2.012947in}}%
\pgfpathlineto{\pgfqpoint{3.538185in}{2.012947in}}%
\pgfpathlineto{\pgfqpoint{3.538185in}{2.015896in}}%
\pgfpathlineto{\pgfqpoint{3.542726in}{2.015896in}}%
\pgfpathlineto{\pgfqpoint{3.542726in}{2.012947in}}%
\pgfpathmoveto{\pgfqpoint{3.542726in}{2.009997in}}%
\pgfpathlineto{\pgfqpoint{3.542726in}{2.009997in}}%
\pgfpathlineto{\pgfqpoint{3.542726in}{2.012947in}}%
\pgfpathlineto{\pgfqpoint{3.547267in}{2.012947in}}%
\pgfpathlineto{\pgfqpoint{3.547267in}{2.009997in}}%
\pgfpathmoveto{\pgfqpoint{3.542726in}{2.012947in}}%
\pgfpathlineto{\pgfqpoint{3.542726in}{2.012947in}}%
\pgfpathlineto{\pgfqpoint{3.542726in}{2.015896in}}%
\pgfpathlineto{\pgfqpoint{3.547267in}{2.015896in}}%
\pgfpathlineto{\pgfqpoint{3.547267in}{2.012947in}}%
\pgfpathmoveto{\pgfqpoint{3.547267in}{2.009997in}}%
\pgfpathlineto{\pgfqpoint{3.547267in}{2.009997in}}%
\pgfpathlineto{\pgfqpoint{3.547267in}{2.012947in}}%
\pgfpathlineto{\pgfqpoint{3.551808in}{2.012947in}}%
\pgfpathlineto{\pgfqpoint{3.551808in}{2.009997in}}%
\pgfpathmoveto{\pgfqpoint{3.547267in}{2.012947in}}%
\pgfpathlineto{\pgfqpoint{3.547267in}{2.012947in}}%
\pgfpathlineto{\pgfqpoint{3.547267in}{2.015896in}}%
\pgfpathlineto{\pgfqpoint{3.551808in}{2.015896in}}%
\pgfpathlineto{\pgfqpoint{3.551808in}{2.012947in}}%
\pgfpathmoveto{\pgfqpoint{3.551808in}{2.009997in}}%
\pgfpathlineto{\pgfqpoint{3.551808in}{2.009997in}}%
\pgfpathlineto{\pgfqpoint{3.551808in}{2.012947in}}%
\pgfpathlineto{\pgfqpoint{3.556349in}{2.012947in}}%
\pgfpathlineto{\pgfqpoint{3.556349in}{2.009997in}}%
\pgfpathmoveto{\pgfqpoint{3.551808in}{2.012947in}}%
\pgfpathlineto{\pgfqpoint{3.551808in}{2.012947in}}%
\pgfpathlineto{\pgfqpoint{3.551808in}{2.015896in}}%
\pgfpathlineto{\pgfqpoint{3.556349in}{2.015896in}}%
\pgfpathlineto{\pgfqpoint{3.556349in}{2.012947in}}%
\pgfpathmoveto{\pgfqpoint{3.556349in}{2.009997in}}%
\pgfpathlineto{\pgfqpoint{3.556349in}{2.009997in}}%
\pgfpathlineto{\pgfqpoint{3.556349in}{2.012947in}}%
\pgfpathlineto{\pgfqpoint{3.560891in}{2.012947in}}%
\pgfpathlineto{\pgfqpoint{3.560891in}{2.009997in}}%
\pgfpathmoveto{\pgfqpoint{3.556349in}{2.012947in}}%
\pgfpathlineto{\pgfqpoint{3.556349in}{2.012947in}}%
\pgfpathlineto{\pgfqpoint{3.556349in}{2.015896in}}%
\pgfpathlineto{\pgfqpoint{3.560891in}{2.015896in}}%
\pgfpathlineto{\pgfqpoint{3.560891in}{2.012947in}}%
\pgfpathmoveto{\pgfqpoint{3.560891in}{2.009997in}}%
\pgfpathlineto{\pgfqpoint{3.560891in}{2.009997in}}%
\pgfpathlineto{\pgfqpoint{3.560891in}{2.012947in}}%
\pgfpathlineto{\pgfqpoint{3.565432in}{2.012947in}}%
\pgfpathlineto{\pgfqpoint{3.565432in}{2.009997in}}%
\pgfpathmoveto{\pgfqpoint{3.560891in}{2.012947in}}%
\pgfpathlineto{\pgfqpoint{3.560891in}{2.012947in}}%
\pgfpathlineto{\pgfqpoint{3.560891in}{2.015896in}}%
\pgfpathlineto{\pgfqpoint{3.565432in}{2.015896in}}%
\pgfpathlineto{\pgfqpoint{3.565432in}{2.012947in}}%
\pgfpathmoveto{\pgfqpoint{3.565432in}{2.009997in}}%
\pgfpathlineto{\pgfqpoint{3.565432in}{2.009997in}}%
\pgfpathlineto{\pgfqpoint{3.565432in}{2.012947in}}%
\pgfpathlineto{\pgfqpoint{3.569973in}{2.012947in}}%
\pgfpathlineto{\pgfqpoint{3.569973in}{2.009997in}}%
\pgfpathmoveto{\pgfqpoint{3.565432in}{2.012947in}}%
\pgfpathlineto{\pgfqpoint{3.565432in}{2.012947in}}%
\pgfpathlineto{\pgfqpoint{3.565432in}{2.015896in}}%
\pgfpathlineto{\pgfqpoint{3.569973in}{2.015896in}}%
\pgfpathlineto{\pgfqpoint{3.569973in}{2.012947in}}%
\pgfpathmoveto{\pgfqpoint{3.569973in}{2.009997in}}%
\pgfpathlineto{\pgfqpoint{3.569973in}{2.009997in}}%
\pgfpathlineto{\pgfqpoint{3.569973in}{2.012947in}}%
\pgfpathlineto{\pgfqpoint{3.574514in}{2.012947in}}%
\pgfpathlineto{\pgfqpoint{3.574514in}{2.009997in}}%
\pgfpathmoveto{\pgfqpoint{3.569973in}{2.012947in}}%
\pgfpathlineto{\pgfqpoint{3.569973in}{2.012947in}}%
\pgfpathlineto{\pgfqpoint{3.569973in}{2.015896in}}%
\pgfpathlineto{\pgfqpoint{3.574514in}{2.015896in}}%
\pgfpathlineto{\pgfqpoint{3.574514in}{2.012947in}}%
\pgfpathmoveto{\pgfqpoint{3.574514in}{2.009997in}}%
\pgfpathlineto{\pgfqpoint{3.574514in}{2.009997in}}%
\pgfpathlineto{\pgfqpoint{3.574514in}{2.012947in}}%
\pgfpathlineto{\pgfqpoint{3.579055in}{2.012947in}}%
\pgfpathlineto{\pgfqpoint{3.579055in}{2.009997in}}%
\pgfpathmoveto{\pgfqpoint{3.574514in}{2.012947in}}%
\pgfpathlineto{\pgfqpoint{3.574514in}{2.012947in}}%
\pgfpathlineto{\pgfqpoint{3.574514in}{2.015896in}}%
\pgfpathlineto{\pgfqpoint{3.579055in}{2.015896in}}%
\pgfpathlineto{\pgfqpoint{3.579055in}{2.012947in}}%
\pgfpathmoveto{\pgfqpoint{3.579055in}{2.009997in}}%
\pgfpathlineto{\pgfqpoint{3.579055in}{2.009997in}}%
\pgfpathlineto{\pgfqpoint{3.579055in}{2.012947in}}%
\pgfpathlineto{\pgfqpoint{3.583596in}{2.012947in}}%
\pgfpathlineto{\pgfqpoint{3.583596in}{2.009997in}}%
\pgfpathmoveto{\pgfqpoint{3.579055in}{2.012947in}}%
\pgfpathlineto{\pgfqpoint{3.579055in}{2.012947in}}%
\pgfpathlineto{\pgfqpoint{3.579055in}{2.015896in}}%
\pgfpathlineto{\pgfqpoint{3.583596in}{2.015896in}}%
\pgfpathlineto{\pgfqpoint{3.583596in}{2.012947in}}%
\pgfpathmoveto{\pgfqpoint{3.583596in}{2.009997in}}%
\pgfpathlineto{\pgfqpoint{3.583596in}{2.009997in}}%
\pgfpathlineto{\pgfqpoint{3.583596in}{2.012947in}}%
\pgfpathlineto{\pgfqpoint{3.588137in}{2.012947in}}%
\pgfpathlineto{\pgfqpoint{3.588137in}{2.009997in}}%
\pgfpathmoveto{\pgfqpoint{3.583596in}{2.012947in}}%
\pgfpathlineto{\pgfqpoint{3.583596in}{2.012947in}}%
\pgfpathlineto{\pgfqpoint{3.583596in}{2.015896in}}%
\pgfpathlineto{\pgfqpoint{3.588137in}{2.015896in}}%
\pgfpathlineto{\pgfqpoint{3.588137in}{2.012947in}}%
\pgfpathmoveto{\pgfqpoint{3.588137in}{2.009997in}}%
\pgfpathlineto{\pgfqpoint{3.588137in}{2.009997in}}%
\pgfpathlineto{\pgfqpoint{3.588137in}{2.012947in}}%
\pgfpathlineto{\pgfqpoint{3.592678in}{2.012947in}}%
\pgfpathlineto{\pgfqpoint{3.592678in}{2.009997in}}%
\pgfpathmoveto{\pgfqpoint{3.588137in}{2.012947in}}%
\pgfpathlineto{\pgfqpoint{3.588137in}{2.012947in}}%
\pgfpathlineto{\pgfqpoint{3.588137in}{2.015896in}}%
\pgfpathlineto{\pgfqpoint{3.592678in}{2.015896in}}%
\pgfpathlineto{\pgfqpoint{3.592678in}{2.012947in}}%
\pgfpathmoveto{\pgfqpoint{3.592678in}{2.009997in}}%
\pgfpathlineto{\pgfqpoint{3.592678in}{2.009997in}}%
\pgfpathlineto{\pgfqpoint{3.592678in}{2.012947in}}%
\pgfpathlineto{\pgfqpoint{3.597219in}{2.012947in}}%
\pgfpathlineto{\pgfqpoint{3.597219in}{2.009997in}}%
\pgfpathmoveto{\pgfqpoint{3.592678in}{2.012947in}}%
\pgfpathlineto{\pgfqpoint{3.592678in}{2.012947in}}%
\pgfpathlineto{\pgfqpoint{3.592678in}{2.015896in}}%
\pgfpathlineto{\pgfqpoint{3.597219in}{2.015896in}}%
\pgfpathlineto{\pgfqpoint{3.597219in}{2.012947in}}%
\pgfpathmoveto{\pgfqpoint{3.597219in}{2.009997in}}%
\pgfpathlineto{\pgfqpoint{3.597219in}{2.009997in}}%
\pgfpathlineto{\pgfqpoint{3.597219in}{2.012947in}}%
\pgfpathlineto{\pgfqpoint{3.601760in}{2.012947in}}%
\pgfpathlineto{\pgfqpoint{3.601760in}{2.009997in}}%
\pgfpathmoveto{\pgfqpoint{3.597219in}{2.012947in}}%
\pgfpathlineto{\pgfqpoint{3.597219in}{2.012947in}}%
\pgfpathlineto{\pgfqpoint{3.597219in}{2.015896in}}%
\pgfpathlineto{\pgfqpoint{3.601760in}{2.015896in}}%
\pgfpathlineto{\pgfqpoint{3.601760in}{2.012947in}}%
\pgfpathmoveto{\pgfqpoint{3.601760in}{2.009997in}}%
\pgfpathlineto{\pgfqpoint{3.601760in}{2.009997in}}%
\pgfpathlineto{\pgfqpoint{3.601760in}{2.012947in}}%
\pgfpathlineto{\pgfqpoint{3.606301in}{2.012947in}}%
\pgfpathlineto{\pgfqpoint{3.606301in}{2.009997in}}%
\pgfpathmoveto{\pgfqpoint{3.601760in}{2.012947in}}%
\pgfpathlineto{\pgfqpoint{3.601760in}{2.012947in}}%
\pgfpathlineto{\pgfqpoint{3.601760in}{2.015896in}}%
\pgfpathlineto{\pgfqpoint{3.606301in}{2.015896in}}%
\pgfpathlineto{\pgfqpoint{3.606301in}{2.012947in}}%
\pgfpathmoveto{\pgfqpoint{3.606301in}{2.009997in}}%
\pgfpathlineto{\pgfqpoint{3.606301in}{2.009997in}}%
\pgfpathlineto{\pgfqpoint{3.606301in}{2.012947in}}%
\pgfpathlineto{\pgfqpoint{3.610842in}{2.012947in}}%
\pgfpathlineto{\pgfqpoint{3.610842in}{2.009997in}}%
\pgfpathmoveto{\pgfqpoint{3.606301in}{2.012947in}}%
\pgfpathlineto{\pgfqpoint{3.606301in}{2.012947in}}%
\pgfpathlineto{\pgfqpoint{3.606301in}{2.015896in}}%
\pgfpathlineto{\pgfqpoint{3.610842in}{2.015896in}}%
\pgfpathlineto{\pgfqpoint{3.610842in}{2.012947in}}%
\pgfpathmoveto{\pgfqpoint{3.610842in}{2.009997in}}%
\pgfpathlineto{\pgfqpoint{3.610842in}{2.009997in}}%
\pgfpathlineto{\pgfqpoint{3.610842in}{2.012947in}}%
\pgfpathlineto{\pgfqpoint{3.615383in}{2.012947in}}%
\pgfpathlineto{\pgfqpoint{3.615383in}{2.009997in}}%
\pgfpathmoveto{\pgfqpoint{3.610842in}{2.012947in}}%
\pgfpathlineto{\pgfqpoint{3.610842in}{2.012947in}}%
\pgfpathlineto{\pgfqpoint{3.610842in}{2.015896in}}%
\pgfpathlineto{\pgfqpoint{3.615383in}{2.015896in}}%
\pgfpathlineto{\pgfqpoint{3.615383in}{2.012947in}}%
\pgfpathmoveto{\pgfqpoint{3.615383in}{2.009997in}}%
\pgfpathlineto{\pgfqpoint{3.615383in}{2.009997in}}%
\pgfpathlineto{\pgfqpoint{3.615383in}{2.012947in}}%
\pgfpathlineto{\pgfqpoint{3.619924in}{2.012947in}}%
\pgfpathlineto{\pgfqpoint{3.619924in}{2.009997in}}%
\pgfpathmoveto{\pgfqpoint{3.615383in}{2.012947in}}%
\pgfpathlineto{\pgfqpoint{3.615383in}{2.012947in}}%
\pgfpathlineto{\pgfqpoint{3.615383in}{2.015896in}}%
\pgfpathlineto{\pgfqpoint{3.619924in}{2.015896in}}%
\pgfpathlineto{\pgfqpoint{3.619924in}{2.012947in}}%
\pgfpathmoveto{\pgfqpoint{3.619924in}{2.009997in}}%
\pgfpathlineto{\pgfqpoint{3.619924in}{2.009997in}}%
\pgfpathlineto{\pgfqpoint{3.619924in}{2.012947in}}%
\pgfpathlineto{\pgfqpoint{3.624465in}{2.012947in}}%
\pgfpathlineto{\pgfqpoint{3.624465in}{2.009997in}}%
\pgfpathmoveto{\pgfqpoint{3.619924in}{2.012947in}}%
\pgfpathlineto{\pgfqpoint{3.619924in}{2.012947in}}%
\pgfpathlineto{\pgfqpoint{3.619924in}{2.015896in}}%
\pgfpathlineto{\pgfqpoint{3.624465in}{2.015896in}}%
\pgfpathlineto{\pgfqpoint{3.624465in}{2.012947in}}%
\pgfpathmoveto{\pgfqpoint{3.624465in}{2.009997in}}%
\pgfpathlineto{\pgfqpoint{3.624465in}{2.009997in}}%
\pgfpathlineto{\pgfqpoint{3.624465in}{2.012947in}}%
\pgfpathlineto{\pgfqpoint{3.629007in}{2.012947in}}%
\pgfpathlineto{\pgfqpoint{3.629007in}{2.009997in}}%
\pgfpathmoveto{\pgfqpoint{3.624465in}{2.012947in}}%
\pgfpathlineto{\pgfqpoint{3.624465in}{2.012947in}}%
\pgfpathlineto{\pgfqpoint{3.624465in}{2.015896in}}%
\pgfpathlineto{\pgfqpoint{3.629007in}{2.015896in}}%
\pgfpathlineto{\pgfqpoint{3.629007in}{2.012947in}}%
\pgfpathmoveto{\pgfqpoint{3.629007in}{2.009997in}}%
\pgfpathlineto{\pgfqpoint{3.629007in}{2.009997in}}%
\pgfpathlineto{\pgfqpoint{3.629007in}{2.012947in}}%
\pgfpathlineto{\pgfqpoint{3.633548in}{2.012947in}}%
\pgfpathlineto{\pgfqpoint{3.633548in}{2.009997in}}%
\pgfpathmoveto{\pgfqpoint{3.629007in}{2.012947in}}%
\pgfpathlineto{\pgfqpoint{3.629007in}{2.012947in}}%
\pgfpathlineto{\pgfqpoint{3.629007in}{2.015896in}}%
\pgfpathlineto{\pgfqpoint{3.633548in}{2.015896in}}%
\pgfpathlineto{\pgfqpoint{3.633548in}{2.012947in}}%
\pgfpathmoveto{\pgfqpoint{3.633548in}{2.009997in}}%
\pgfpathlineto{\pgfqpoint{3.633548in}{2.009997in}}%
\pgfpathlineto{\pgfqpoint{3.633548in}{2.012947in}}%
\pgfpathlineto{\pgfqpoint{3.638089in}{2.012947in}}%
\pgfpathlineto{\pgfqpoint{3.638089in}{2.009997in}}%
\pgfpathmoveto{\pgfqpoint{3.633548in}{2.012947in}}%
\pgfpathlineto{\pgfqpoint{3.633548in}{2.012947in}}%
\pgfpathlineto{\pgfqpoint{3.633548in}{2.015896in}}%
\pgfpathlineto{\pgfqpoint{3.638089in}{2.015896in}}%
\pgfpathlineto{\pgfqpoint{3.638089in}{2.012947in}}%
\pgfpathmoveto{\pgfqpoint{3.638089in}{2.009997in}}%
\pgfpathlineto{\pgfqpoint{3.638089in}{2.009997in}}%
\pgfpathlineto{\pgfqpoint{3.638089in}{2.012947in}}%
\pgfpathlineto{\pgfqpoint{3.642630in}{2.012947in}}%
\pgfpathlineto{\pgfqpoint{3.642630in}{2.009997in}}%
\pgfpathmoveto{\pgfqpoint{3.638089in}{2.012947in}}%
\pgfpathlineto{\pgfqpoint{3.638089in}{2.012947in}}%
\pgfpathlineto{\pgfqpoint{3.638089in}{2.015896in}}%
\pgfpathlineto{\pgfqpoint{3.642630in}{2.015896in}}%
\pgfpathlineto{\pgfqpoint{3.642630in}{2.012947in}}%
\pgfpathmoveto{\pgfqpoint{3.642630in}{2.009997in}}%
\pgfpathlineto{\pgfqpoint{3.642630in}{2.009997in}}%
\pgfpathlineto{\pgfqpoint{3.642630in}{2.012947in}}%
\pgfpathlineto{\pgfqpoint{3.647171in}{2.012947in}}%
\pgfpathlineto{\pgfqpoint{3.647171in}{2.009997in}}%
\pgfpathmoveto{\pgfqpoint{3.642630in}{2.012947in}}%
\pgfpathlineto{\pgfqpoint{3.642630in}{2.012947in}}%
\pgfpathlineto{\pgfqpoint{3.642630in}{2.015896in}}%
\pgfpathlineto{\pgfqpoint{3.647171in}{2.015896in}}%
\pgfpathlineto{\pgfqpoint{3.647171in}{2.012947in}}%
\pgfpathmoveto{\pgfqpoint{3.647171in}{2.009997in}}%
\pgfpathlineto{\pgfqpoint{3.647171in}{2.009997in}}%
\pgfpathlineto{\pgfqpoint{3.647171in}{2.012947in}}%
\pgfpathlineto{\pgfqpoint{3.651712in}{2.012947in}}%
\pgfpathlineto{\pgfqpoint{3.651712in}{2.009997in}}%
\pgfpathmoveto{\pgfqpoint{3.647171in}{2.012947in}}%
\pgfpathlineto{\pgfqpoint{3.647171in}{2.012947in}}%
\pgfpathlineto{\pgfqpoint{3.647171in}{2.015896in}}%
\pgfpathlineto{\pgfqpoint{3.651712in}{2.015896in}}%
\pgfpathlineto{\pgfqpoint{3.651712in}{2.012947in}}%
\pgfpathmoveto{\pgfqpoint{3.651712in}{2.009997in}}%
\pgfpathlineto{\pgfqpoint{3.651712in}{2.009997in}}%
\pgfpathlineto{\pgfqpoint{3.651712in}{2.012947in}}%
\pgfpathlineto{\pgfqpoint{3.656253in}{2.012947in}}%
\pgfpathlineto{\pgfqpoint{3.656253in}{2.009997in}}%
\pgfpathmoveto{\pgfqpoint{3.651712in}{2.012947in}}%
\pgfpathlineto{\pgfqpoint{3.651712in}{2.012947in}}%
\pgfpathlineto{\pgfqpoint{3.651712in}{2.015896in}}%
\pgfpathlineto{\pgfqpoint{3.656253in}{2.015896in}}%
\pgfpathlineto{\pgfqpoint{3.656253in}{2.012947in}}%
\pgfpathmoveto{\pgfqpoint{3.592678in}{2.570351in}}%
\pgfpathlineto{\pgfqpoint{3.592678in}{2.570351in}}%
\pgfpathlineto{\pgfqpoint{3.592678in}{2.573300in}}%
\pgfpathlineto{\pgfqpoint{3.597219in}{2.573300in}}%
\pgfpathlineto{\pgfqpoint{3.597219in}{2.570351in}}%
\pgfpathmoveto{\pgfqpoint{3.592678in}{2.573300in}}%
\pgfpathlineto{\pgfqpoint{3.592678in}{2.573300in}}%
\pgfpathlineto{\pgfqpoint{3.592678in}{2.576249in}}%
\pgfpathlineto{\pgfqpoint{3.597219in}{2.576249in}}%
\pgfpathlineto{\pgfqpoint{3.597219in}{2.573300in}}%
\pgfpathmoveto{\pgfqpoint{3.597219in}{2.570351in}}%
\pgfpathlineto{\pgfqpoint{3.597219in}{2.570351in}}%
\pgfpathlineto{\pgfqpoint{3.597219in}{2.573300in}}%
\pgfpathlineto{\pgfqpoint{3.601760in}{2.573300in}}%
\pgfpathlineto{\pgfqpoint{3.601760in}{2.570351in}}%
\pgfpathmoveto{\pgfqpoint{3.597219in}{2.573300in}}%
\pgfpathlineto{\pgfqpoint{3.597219in}{2.573300in}}%
\pgfpathlineto{\pgfqpoint{3.597219in}{2.576249in}}%
\pgfpathlineto{\pgfqpoint{3.601760in}{2.576249in}}%
\pgfpathlineto{\pgfqpoint{3.601760in}{2.573300in}}%
\pgfpathmoveto{\pgfqpoint{3.610842in}{2.558554in}}%
\pgfpathlineto{\pgfqpoint{3.610842in}{2.558554in}}%
\pgfpathlineto{\pgfqpoint{3.610842in}{2.561503in}}%
\pgfpathlineto{\pgfqpoint{3.615383in}{2.561503in}}%
\pgfpathlineto{\pgfqpoint{3.615383in}{2.558554in}}%
\pgfpathmoveto{\pgfqpoint{3.610842in}{2.561503in}}%
\pgfpathlineto{\pgfqpoint{3.610842in}{2.561503in}}%
\pgfpathlineto{\pgfqpoint{3.610842in}{2.564452in}}%
\pgfpathlineto{\pgfqpoint{3.615383in}{2.564452in}}%
\pgfpathlineto{\pgfqpoint{3.615383in}{2.561503in}}%
\pgfpathmoveto{\pgfqpoint{3.615383in}{2.558554in}}%
\pgfpathlineto{\pgfqpoint{3.615383in}{2.558554in}}%
\pgfpathlineto{\pgfqpoint{3.615383in}{2.561503in}}%
\pgfpathlineto{\pgfqpoint{3.619924in}{2.561503in}}%
\pgfpathlineto{\pgfqpoint{3.619924in}{2.558554in}}%
\pgfpathmoveto{\pgfqpoint{3.615383in}{2.561503in}}%
\pgfpathlineto{\pgfqpoint{3.615383in}{2.561503in}}%
\pgfpathlineto{\pgfqpoint{3.615383in}{2.564452in}}%
\pgfpathlineto{\pgfqpoint{3.619924in}{2.564452in}}%
\pgfpathlineto{\pgfqpoint{3.619924in}{2.561503in}}%
\pgfpathmoveto{\pgfqpoint{3.601760in}{2.564452in}}%
\pgfpathlineto{\pgfqpoint{3.601760in}{2.564452in}}%
\pgfpathlineto{\pgfqpoint{3.601760in}{2.567402in}}%
\pgfpathlineto{\pgfqpoint{3.606301in}{2.567402in}}%
\pgfpathlineto{\pgfqpoint{3.606301in}{2.564452in}}%
\pgfpathmoveto{\pgfqpoint{3.601760in}{2.567402in}}%
\pgfpathlineto{\pgfqpoint{3.601760in}{2.567402in}}%
\pgfpathlineto{\pgfqpoint{3.601760in}{2.570351in}}%
\pgfpathlineto{\pgfqpoint{3.606301in}{2.570351in}}%
\pgfpathlineto{\pgfqpoint{3.606301in}{2.567402in}}%
\pgfpathmoveto{\pgfqpoint{3.606301in}{2.564452in}}%
\pgfpathlineto{\pgfqpoint{3.606301in}{2.564452in}}%
\pgfpathlineto{\pgfqpoint{3.606301in}{2.567402in}}%
\pgfpathlineto{\pgfqpoint{3.610842in}{2.567402in}}%
\pgfpathlineto{\pgfqpoint{3.610842in}{2.564452in}}%
\pgfpathmoveto{\pgfqpoint{3.606301in}{2.567402in}}%
\pgfpathlineto{\pgfqpoint{3.606301in}{2.567402in}}%
\pgfpathlineto{\pgfqpoint{3.606301in}{2.570351in}}%
\pgfpathlineto{\pgfqpoint{3.610842in}{2.570351in}}%
\pgfpathlineto{\pgfqpoint{3.610842in}{2.567402in}}%
\pgfpathmoveto{\pgfqpoint{3.601760in}{2.570351in}}%
\pgfpathlineto{\pgfqpoint{3.601760in}{2.570351in}}%
\pgfpathlineto{\pgfqpoint{3.601760in}{2.573300in}}%
\pgfpathlineto{\pgfqpoint{3.606301in}{2.573300in}}%
\pgfpathlineto{\pgfqpoint{3.606301in}{2.570351in}}%
\pgfpathmoveto{\pgfqpoint{3.601760in}{2.573300in}}%
\pgfpathlineto{\pgfqpoint{3.601760in}{2.573300in}}%
\pgfpathlineto{\pgfqpoint{3.601760in}{2.576249in}}%
\pgfpathlineto{\pgfqpoint{3.606301in}{2.576249in}}%
\pgfpathlineto{\pgfqpoint{3.606301in}{2.573300in}}%
\pgfpathmoveto{\pgfqpoint{3.606301in}{2.570351in}}%
\pgfpathlineto{\pgfqpoint{3.606301in}{2.570351in}}%
\pgfpathlineto{\pgfqpoint{3.606301in}{2.573300in}}%
\pgfpathlineto{\pgfqpoint{3.610842in}{2.573300in}}%
\pgfpathlineto{\pgfqpoint{3.610842in}{2.570351in}}%
\pgfpathmoveto{\pgfqpoint{3.610842in}{2.564452in}}%
\pgfpathlineto{\pgfqpoint{3.610842in}{2.564452in}}%
\pgfpathlineto{\pgfqpoint{3.610842in}{2.567402in}}%
\pgfpathlineto{\pgfqpoint{3.615383in}{2.567402in}}%
\pgfpathlineto{\pgfqpoint{3.615383in}{2.564452in}}%
\pgfpathmoveto{\pgfqpoint{3.610842in}{2.567402in}}%
\pgfpathlineto{\pgfqpoint{3.610842in}{2.567402in}}%
\pgfpathlineto{\pgfqpoint{3.610842in}{2.570351in}}%
\pgfpathlineto{\pgfqpoint{3.615383in}{2.570351in}}%
\pgfpathlineto{\pgfqpoint{3.615383in}{2.567402in}}%
\pgfpathmoveto{\pgfqpoint{3.615383in}{2.564452in}}%
\pgfpathlineto{\pgfqpoint{3.615383in}{2.564452in}}%
\pgfpathlineto{\pgfqpoint{3.615383in}{2.567402in}}%
\pgfpathlineto{\pgfqpoint{3.619924in}{2.567402in}}%
\pgfpathlineto{\pgfqpoint{3.619924in}{2.564452in}}%
\pgfpathmoveto{\pgfqpoint{3.629007in}{2.546757in}}%
\pgfpathlineto{\pgfqpoint{3.629007in}{2.546757in}}%
\pgfpathlineto{\pgfqpoint{3.629007in}{2.549706in}}%
\pgfpathlineto{\pgfqpoint{3.633548in}{2.549706in}}%
\pgfpathlineto{\pgfqpoint{3.633548in}{2.546757in}}%
\pgfpathmoveto{\pgfqpoint{3.629007in}{2.549706in}}%
\pgfpathlineto{\pgfqpoint{3.629007in}{2.549706in}}%
\pgfpathlineto{\pgfqpoint{3.629007in}{2.552655in}}%
\pgfpathlineto{\pgfqpoint{3.633548in}{2.552655in}}%
\pgfpathlineto{\pgfqpoint{3.633548in}{2.549706in}}%
\pgfpathmoveto{\pgfqpoint{3.633548in}{2.546757in}}%
\pgfpathlineto{\pgfqpoint{3.633548in}{2.546757in}}%
\pgfpathlineto{\pgfqpoint{3.633548in}{2.549706in}}%
\pgfpathlineto{\pgfqpoint{3.638089in}{2.549706in}}%
\pgfpathlineto{\pgfqpoint{3.638089in}{2.546757in}}%
\pgfpathmoveto{\pgfqpoint{3.633548in}{2.549706in}}%
\pgfpathlineto{\pgfqpoint{3.633548in}{2.549706in}}%
\pgfpathlineto{\pgfqpoint{3.633548in}{2.552655in}}%
\pgfpathlineto{\pgfqpoint{3.638089in}{2.552655in}}%
\pgfpathlineto{\pgfqpoint{3.638089in}{2.549706in}}%
\pgfpathmoveto{\pgfqpoint{3.647171in}{2.534959in}}%
\pgfpathlineto{\pgfqpoint{3.647171in}{2.534959in}}%
\pgfpathlineto{\pgfqpoint{3.647171in}{2.537909in}}%
\pgfpathlineto{\pgfqpoint{3.651712in}{2.537909in}}%
\pgfpathlineto{\pgfqpoint{3.651712in}{2.534959in}}%
\pgfpathmoveto{\pgfqpoint{3.647171in}{2.537909in}}%
\pgfpathlineto{\pgfqpoint{3.647171in}{2.537909in}}%
\pgfpathlineto{\pgfqpoint{3.647171in}{2.540858in}}%
\pgfpathlineto{\pgfqpoint{3.651712in}{2.540858in}}%
\pgfpathlineto{\pgfqpoint{3.651712in}{2.537909in}}%
\pgfpathmoveto{\pgfqpoint{3.651712in}{2.534959in}}%
\pgfpathlineto{\pgfqpoint{3.651712in}{2.534959in}}%
\pgfpathlineto{\pgfqpoint{3.651712in}{2.537909in}}%
\pgfpathlineto{\pgfqpoint{3.656253in}{2.537909in}}%
\pgfpathlineto{\pgfqpoint{3.656253in}{2.534959in}}%
\pgfpathmoveto{\pgfqpoint{3.651712in}{2.537909in}}%
\pgfpathlineto{\pgfqpoint{3.651712in}{2.537909in}}%
\pgfpathlineto{\pgfqpoint{3.651712in}{2.540858in}}%
\pgfpathlineto{\pgfqpoint{3.656253in}{2.540858in}}%
\pgfpathlineto{\pgfqpoint{3.656253in}{2.537909in}}%
\pgfpathmoveto{\pgfqpoint{3.638089in}{2.540858in}}%
\pgfpathlineto{\pgfqpoint{3.638089in}{2.540858in}}%
\pgfpathlineto{\pgfqpoint{3.638089in}{2.543807in}}%
\pgfpathlineto{\pgfqpoint{3.642630in}{2.543807in}}%
\pgfpathlineto{\pgfqpoint{3.642630in}{2.540858in}}%
\pgfpathmoveto{\pgfqpoint{3.638089in}{2.543807in}}%
\pgfpathlineto{\pgfqpoint{3.638089in}{2.543807in}}%
\pgfpathlineto{\pgfqpoint{3.638089in}{2.546757in}}%
\pgfpathlineto{\pgfqpoint{3.642630in}{2.546757in}}%
\pgfpathlineto{\pgfqpoint{3.642630in}{2.543807in}}%
\pgfpathmoveto{\pgfqpoint{3.642630in}{2.540858in}}%
\pgfpathlineto{\pgfqpoint{3.642630in}{2.540858in}}%
\pgfpathlineto{\pgfqpoint{3.642630in}{2.543807in}}%
\pgfpathlineto{\pgfqpoint{3.647171in}{2.543807in}}%
\pgfpathlineto{\pgfqpoint{3.647171in}{2.540858in}}%
\pgfpathmoveto{\pgfqpoint{3.642630in}{2.543807in}}%
\pgfpathlineto{\pgfqpoint{3.642630in}{2.543807in}}%
\pgfpathlineto{\pgfqpoint{3.642630in}{2.546757in}}%
\pgfpathlineto{\pgfqpoint{3.647171in}{2.546757in}}%
\pgfpathlineto{\pgfqpoint{3.647171in}{2.543807in}}%
\pgfpathmoveto{\pgfqpoint{3.638089in}{2.546757in}}%
\pgfpathlineto{\pgfqpoint{3.638089in}{2.546757in}}%
\pgfpathlineto{\pgfqpoint{3.638089in}{2.549706in}}%
\pgfpathlineto{\pgfqpoint{3.642630in}{2.549706in}}%
\pgfpathlineto{\pgfqpoint{3.642630in}{2.546757in}}%
\pgfpathmoveto{\pgfqpoint{3.638089in}{2.549706in}}%
\pgfpathlineto{\pgfqpoint{3.638089in}{2.549706in}}%
\pgfpathlineto{\pgfqpoint{3.638089in}{2.552655in}}%
\pgfpathlineto{\pgfqpoint{3.642630in}{2.552655in}}%
\pgfpathlineto{\pgfqpoint{3.642630in}{2.549706in}}%
\pgfpathmoveto{\pgfqpoint{3.642630in}{2.546757in}}%
\pgfpathlineto{\pgfqpoint{3.642630in}{2.546757in}}%
\pgfpathlineto{\pgfqpoint{3.642630in}{2.549706in}}%
\pgfpathlineto{\pgfqpoint{3.647171in}{2.549706in}}%
\pgfpathlineto{\pgfqpoint{3.647171in}{2.546757in}}%
\pgfpathmoveto{\pgfqpoint{3.647171in}{2.540858in}}%
\pgfpathlineto{\pgfqpoint{3.647171in}{2.540858in}}%
\pgfpathlineto{\pgfqpoint{3.647171in}{2.543807in}}%
\pgfpathlineto{\pgfqpoint{3.651712in}{2.543807in}}%
\pgfpathlineto{\pgfqpoint{3.651712in}{2.540858in}}%
\pgfpathmoveto{\pgfqpoint{3.647171in}{2.543807in}}%
\pgfpathlineto{\pgfqpoint{3.647171in}{2.543807in}}%
\pgfpathlineto{\pgfqpoint{3.647171in}{2.546757in}}%
\pgfpathlineto{\pgfqpoint{3.651712in}{2.546757in}}%
\pgfpathlineto{\pgfqpoint{3.651712in}{2.543807in}}%
\pgfpathmoveto{\pgfqpoint{3.651712in}{2.540858in}}%
\pgfpathlineto{\pgfqpoint{3.651712in}{2.540858in}}%
\pgfpathlineto{\pgfqpoint{3.651712in}{2.543807in}}%
\pgfpathlineto{\pgfqpoint{3.656253in}{2.543807in}}%
\pgfpathlineto{\pgfqpoint{3.656253in}{2.540858in}}%
\pgfpathmoveto{\pgfqpoint{3.619924in}{2.552655in}}%
\pgfpathlineto{\pgfqpoint{3.619924in}{2.552655in}}%
\pgfpathlineto{\pgfqpoint{3.619924in}{2.555604in}}%
\pgfpathlineto{\pgfqpoint{3.624465in}{2.555604in}}%
\pgfpathlineto{\pgfqpoint{3.624465in}{2.552655in}}%
\pgfpathmoveto{\pgfqpoint{3.619924in}{2.555604in}}%
\pgfpathlineto{\pgfqpoint{3.619924in}{2.555604in}}%
\pgfpathlineto{\pgfqpoint{3.619924in}{2.558554in}}%
\pgfpathlineto{\pgfqpoint{3.624465in}{2.558554in}}%
\pgfpathlineto{\pgfqpoint{3.624465in}{2.555604in}}%
\pgfpathmoveto{\pgfqpoint{3.624465in}{2.552655in}}%
\pgfpathlineto{\pgfqpoint{3.624465in}{2.552655in}}%
\pgfpathlineto{\pgfqpoint{3.624465in}{2.555604in}}%
\pgfpathlineto{\pgfqpoint{3.629007in}{2.555604in}}%
\pgfpathlineto{\pgfqpoint{3.629007in}{2.552655in}}%
\pgfpathmoveto{\pgfqpoint{3.624465in}{2.555604in}}%
\pgfpathlineto{\pgfqpoint{3.624465in}{2.555604in}}%
\pgfpathlineto{\pgfqpoint{3.624465in}{2.558554in}}%
\pgfpathlineto{\pgfqpoint{3.629007in}{2.558554in}}%
\pgfpathlineto{\pgfqpoint{3.629007in}{2.555604in}}%
\pgfpathmoveto{\pgfqpoint{3.619924in}{2.558554in}}%
\pgfpathlineto{\pgfqpoint{3.619924in}{2.558554in}}%
\pgfpathlineto{\pgfqpoint{3.619924in}{2.561503in}}%
\pgfpathlineto{\pgfqpoint{3.624465in}{2.561503in}}%
\pgfpathlineto{\pgfqpoint{3.624465in}{2.558554in}}%
\pgfpathmoveto{\pgfqpoint{3.619924in}{2.561503in}}%
\pgfpathlineto{\pgfqpoint{3.619924in}{2.561503in}}%
\pgfpathlineto{\pgfqpoint{3.619924in}{2.564452in}}%
\pgfpathlineto{\pgfqpoint{3.624465in}{2.564452in}}%
\pgfpathlineto{\pgfqpoint{3.624465in}{2.561503in}}%
\pgfpathmoveto{\pgfqpoint{3.624465in}{2.558554in}}%
\pgfpathlineto{\pgfqpoint{3.624465in}{2.558554in}}%
\pgfpathlineto{\pgfqpoint{3.624465in}{2.561503in}}%
\pgfpathlineto{\pgfqpoint{3.629007in}{2.561503in}}%
\pgfpathlineto{\pgfqpoint{3.629007in}{2.558554in}}%
\pgfpathmoveto{\pgfqpoint{3.629007in}{2.552655in}}%
\pgfpathlineto{\pgfqpoint{3.629007in}{2.552655in}}%
\pgfpathlineto{\pgfqpoint{3.629007in}{2.555604in}}%
\pgfpathlineto{\pgfqpoint{3.633548in}{2.555604in}}%
\pgfpathlineto{\pgfqpoint{3.633548in}{2.552655in}}%
\pgfpathmoveto{\pgfqpoint{3.629007in}{2.555604in}}%
\pgfpathlineto{\pgfqpoint{3.629007in}{2.555604in}}%
\pgfpathlineto{\pgfqpoint{3.629007in}{2.558554in}}%
\pgfpathlineto{\pgfqpoint{3.633548in}{2.558554in}}%
\pgfpathlineto{\pgfqpoint{3.633548in}{2.555604in}}%
\pgfpathmoveto{\pgfqpoint{3.633548in}{2.552655in}}%
\pgfpathlineto{\pgfqpoint{3.633548in}{2.552655in}}%
\pgfpathlineto{\pgfqpoint{3.633548in}{2.555604in}}%
\pgfpathlineto{\pgfqpoint{3.638089in}{2.555604in}}%
\pgfpathlineto{\pgfqpoint{3.638089in}{2.552655in}}%
\pgfpathmoveto{\pgfqpoint{3.510939in}{2.611639in}}%
\pgfpathlineto{\pgfqpoint{3.510939in}{2.611639in}}%
\pgfpathlineto{\pgfqpoint{3.510939in}{2.614589in}}%
\pgfpathlineto{\pgfqpoint{3.515480in}{2.614589in}}%
\pgfpathlineto{\pgfqpoint{3.515480in}{2.611639in}}%
\pgfpathmoveto{\pgfqpoint{3.515480in}{2.611639in}}%
\pgfpathlineto{\pgfqpoint{3.515480in}{2.611639in}}%
\pgfpathlineto{\pgfqpoint{3.515480in}{2.614589in}}%
\pgfpathlineto{\pgfqpoint{3.520021in}{2.614589in}}%
\pgfpathlineto{\pgfqpoint{3.520021in}{2.611639in}}%
\pgfpathmoveto{\pgfqpoint{3.520021in}{2.611639in}}%
\pgfpathlineto{\pgfqpoint{3.520021in}{2.611639in}}%
\pgfpathlineto{\pgfqpoint{3.520021in}{2.614589in}}%
\pgfpathlineto{\pgfqpoint{3.524562in}{2.614589in}}%
\pgfpathlineto{\pgfqpoint{3.524562in}{2.611639in}}%
\pgfpathmoveto{\pgfqpoint{3.524562in}{2.611639in}}%
\pgfpathlineto{\pgfqpoint{3.524562in}{2.611639in}}%
\pgfpathlineto{\pgfqpoint{3.524562in}{2.614589in}}%
\pgfpathlineto{\pgfqpoint{3.529103in}{2.614589in}}%
\pgfpathlineto{\pgfqpoint{3.529103in}{2.611639in}}%
\pgfpathmoveto{\pgfqpoint{3.538185in}{2.605741in}}%
\pgfpathlineto{\pgfqpoint{3.538185in}{2.605741in}}%
\pgfpathlineto{\pgfqpoint{3.538185in}{2.608690in}}%
\pgfpathlineto{\pgfqpoint{3.542726in}{2.608690in}}%
\pgfpathlineto{\pgfqpoint{3.542726in}{2.605741in}}%
\pgfpathmoveto{\pgfqpoint{3.538185in}{2.608690in}}%
\pgfpathlineto{\pgfqpoint{3.538185in}{2.608690in}}%
\pgfpathlineto{\pgfqpoint{3.538185in}{2.611639in}}%
\pgfpathlineto{\pgfqpoint{3.542726in}{2.611639in}}%
\pgfpathlineto{\pgfqpoint{3.542726in}{2.608690in}}%
\pgfpathmoveto{\pgfqpoint{3.542726in}{2.605741in}}%
\pgfpathlineto{\pgfqpoint{3.542726in}{2.605741in}}%
\pgfpathlineto{\pgfqpoint{3.542726in}{2.608690in}}%
\pgfpathlineto{\pgfqpoint{3.547267in}{2.608690in}}%
\pgfpathlineto{\pgfqpoint{3.547267in}{2.605741in}}%
\pgfpathmoveto{\pgfqpoint{3.542726in}{2.608690in}}%
\pgfpathlineto{\pgfqpoint{3.542726in}{2.608690in}}%
\pgfpathlineto{\pgfqpoint{3.542726in}{2.611639in}}%
\pgfpathlineto{\pgfqpoint{3.547267in}{2.611639in}}%
\pgfpathlineto{\pgfqpoint{3.547267in}{2.608690in}}%
\pgfpathmoveto{\pgfqpoint{3.529103in}{2.611639in}}%
\pgfpathlineto{\pgfqpoint{3.529103in}{2.611639in}}%
\pgfpathlineto{\pgfqpoint{3.529103in}{2.614589in}}%
\pgfpathlineto{\pgfqpoint{3.533644in}{2.614589in}}%
\pgfpathlineto{\pgfqpoint{3.533644in}{2.611639in}}%
\pgfpathmoveto{\pgfqpoint{3.533644in}{2.611639in}}%
\pgfpathlineto{\pgfqpoint{3.533644in}{2.611639in}}%
\pgfpathlineto{\pgfqpoint{3.533644in}{2.614589in}}%
\pgfpathlineto{\pgfqpoint{3.538185in}{2.614589in}}%
\pgfpathlineto{\pgfqpoint{3.538185in}{2.611639in}}%
\pgfpathmoveto{\pgfqpoint{3.538185in}{2.611639in}}%
\pgfpathlineto{\pgfqpoint{3.538185in}{2.611639in}}%
\pgfpathlineto{\pgfqpoint{3.538185in}{2.614589in}}%
\pgfpathlineto{\pgfqpoint{3.542726in}{2.614589in}}%
\pgfpathlineto{\pgfqpoint{3.542726in}{2.611639in}}%
\pgfpathmoveto{\pgfqpoint{3.542726in}{2.611639in}}%
\pgfpathlineto{\pgfqpoint{3.542726in}{2.611639in}}%
\pgfpathlineto{\pgfqpoint{3.542726in}{2.614589in}}%
\pgfpathlineto{\pgfqpoint{3.547267in}{2.614589in}}%
\pgfpathlineto{\pgfqpoint{3.547267in}{2.611639in}}%
\pgfpathmoveto{\pgfqpoint{3.556349in}{2.593944in}}%
\pgfpathlineto{\pgfqpoint{3.556349in}{2.593944in}}%
\pgfpathlineto{\pgfqpoint{3.556349in}{2.596894in}}%
\pgfpathlineto{\pgfqpoint{3.560891in}{2.596894in}}%
\pgfpathlineto{\pgfqpoint{3.560891in}{2.593944in}}%
\pgfpathmoveto{\pgfqpoint{3.556349in}{2.596894in}}%
\pgfpathlineto{\pgfqpoint{3.556349in}{2.596894in}}%
\pgfpathlineto{\pgfqpoint{3.556349in}{2.599843in}}%
\pgfpathlineto{\pgfqpoint{3.560891in}{2.599843in}}%
\pgfpathlineto{\pgfqpoint{3.560891in}{2.596894in}}%
\pgfpathmoveto{\pgfqpoint{3.560891in}{2.593944in}}%
\pgfpathlineto{\pgfqpoint{3.560891in}{2.593944in}}%
\pgfpathlineto{\pgfqpoint{3.560891in}{2.596894in}}%
\pgfpathlineto{\pgfqpoint{3.565432in}{2.596894in}}%
\pgfpathlineto{\pgfqpoint{3.565432in}{2.593944in}}%
\pgfpathmoveto{\pgfqpoint{3.560891in}{2.596894in}}%
\pgfpathlineto{\pgfqpoint{3.560891in}{2.596894in}}%
\pgfpathlineto{\pgfqpoint{3.560891in}{2.599843in}}%
\pgfpathlineto{\pgfqpoint{3.565432in}{2.599843in}}%
\pgfpathlineto{\pgfqpoint{3.565432in}{2.596894in}}%
\pgfpathmoveto{\pgfqpoint{3.574514in}{2.582148in}}%
\pgfpathlineto{\pgfqpoint{3.574514in}{2.582148in}}%
\pgfpathlineto{\pgfqpoint{3.574514in}{2.585097in}}%
\pgfpathlineto{\pgfqpoint{3.579055in}{2.585097in}}%
\pgfpathlineto{\pgfqpoint{3.579055in}{2.582148in}}%
\pgfpathmoveto{\pgfqpoint{3.574514in}{2.585097in}}%
\pgfpathlineto{\pgfqpoint{3.574514in}{2.585097in}}%
\pgfpathlineto{\pgfqpoint{3.574514in}{2.588046in}}%
\pgfpathlineto{\pgfqpoint{3.579055in}{2.588046in}}%
\pgfpathlineto{\pgfqpoint{3.579055in}{2.585097in}}%
\pgfpathmoveto{\pgfqpoint{3.579055in}{2.582148in}}%
\pgfpathlineto{\pgfqpoint{3.579055in}{2.582148in}}%
\pgfpathlineto{\pgfqpoint{3.579055in}{2.585097in}}%
\pgfpathlineto{\pgfqpoint{3.583596in}{2.585097in}}%
\pgfpathlineto{\pgfqpoint{3.583596in}{2.582148in}}%
\pgfpathmoveto{\pgfqpoint{3.579055in}{2.585097in}}%
\pgfpathlineto{\pgfqpoint{3.579055in}{2.585097in}}%
\pgfpathlineto{\pgfqpoint{3.579055in}{2.588046in}}%
\pgfpathlineto{\pgfqpoint{3.583596in}{2.588046in}}%
\pgfpathlineto{\pgfqpoint{3.583596in}{2.585097in}}%
\pgfpathmoveto{\pgfqpoint{3.565432in}{2.588046in}}%
\pgfpathlineto{\pgfqpoint{3.565432in}{2.588046in}}%
\pgfpathlineto{\pgfqpoint{3.565432in}{2.590995in}}%
\pgfpathlineto{\pgfqpoint{3.569973in}{2.590995in}}%
\pgfpathlineto{\pgfqpoint{3.569973in}{2.588046in}}%
\pgfpathmoveto{\pgfqpoint{3.565432in}{2.590995in}}%
\pgfpathlineto{\pgfqpoint{3.565432in}{2.590995in}}%
\pgfpathlineto{\pgfqpoint{3.565432in}{2.593944in}}%
\pgfpathlineto{\pgfqpoint{3.569973in}{2.593944in}}%
\pgfpathlineto{\pgfqpoint{3.569973in}{2.590995in}}%
\pgfpathmoveto{\pgfqpoint{3.569973in}{2.588046in}}%
\pgfpathlineto{\pgfqpoint{3.569973in}{2.588046in}}%
\pgfpathlineto{\pgfqpoint{3.569973in}{2.590995in}}%
\pgfpathlineto{\pgfqpoint{3.574514in}{2.590995in}}%
\pgfpathlineto{\pgfqpoint{3.574514in}{2.588046in}}%
\pgfpathmoveto{\pgfqpoint{3.569973in}{2.590995in}}%
\pgfpathlineto{\pgfqpoint{3.569973in}{2.590995in}}%
\pgfpathlineto{\pgfqpoint{3.569973in}{2.593944in}}%
\pgfpathlineto{\pgfqpoint{3.574514in}{2.593944in}}%
\pgfpathlineto{\pgfqpoint{3.574514in}{2.590995in}}%
\pgfpathmoveto{\pgfqpoint{3.565432in}{2.593944in}}%
\pgfpathlineto{\pgfqpoint{3.565432in}{2.593944in}}%
\pgfpathlineto{\pgfqpoint{3.565432in}{2.596894in}}%
\pgfpathlineto{\pgfqpoint{3.569973in}{2.596894in}}%
\pgfpathlineto{\pgfqpoint{3.569973in}{2.593944in}}%
\pgfpathmoveto{\pgfqpoint{3.565432in}{2.596894in}}%
\pgfpathlineto{\pgfqpoint{3.565432in}{2.596894in}}%
\pgfpathlineto{\pgfqpoint{3.565432in}{2.599843in}}%
\pgfpathlineto{\pgfqpoint{3.569973in}{2.599843in}}%
\pgfpathlineto{\pgfqpoint{3.569973in}{2.596894in}}%
\pgfpathmoveto{\pgfqpoint{3.569973in}{2.593944in}}%
\pgfpathlineto{\pgfqpoint{3.569973in}{2.593944in}}%
\pgfpathlineto{\pgfqpoint{3.569973in}{2.596894in}}%
\pgfpathlineto{\pgfqpoint{3.574514in}{2.596894in}}%
\pgfpathlineto{\pgfqpoint{3.574514in}{2.593944in}}%
\pgfpathmoveto{\pgfqpoint{3.574514in}{2.588046in}}%
\pgfpathlineto{\pgfqpoint{3.574514in}{2.588046in}}%
\pgfpathlineto{\pgfqpoint{3.574514in}{2.590995in}}%
\pgfpathlineto{\pgfqpoint{3.579055in}{2.590995in}}%
\pgfpathlineto{\pgfqpoint{3.579055in}{2.588046in}}%
\pgfpathmoveto{\pgfqpoint{3.574514in}{2.590995in}}%
\pgfpathlineto{\pgfqpoint{3.574514in}{2.590995in}}%
\pgfpathlineto{\pgfqpoint{3.574514in}{2.593944in}}%
\pgfpathlineto{\pgfqpoint{3.579055in}{2.593944in}}%
\pgfpathlineto{\pgfqpoint{3.579055in}{2.590995in}}%
\pgfpathmoveto{\pgfqpoint{3.579055in}{2.588046in}}%
\pgfpathlineto{\pgfqpoint{3.579055in}{2.588046in}}%
\pgfpathlineto{\pgfqpoint{3.579055in}{2.590995in}}%
\pgfpathlineto{\pgfqpoint{3.583596in}{2.590995in}}%
\pgfpathlineto{\pgfqpoint{3.583596in}{2.588046in}}%
\pgfpathmoveto{\pgfqpoint{3.547267in}{2.599843in}}%
\pgfpathlineto{\pgfqpoint{3.547267in}{2.599843in}}%
\pgfpathlineto{\pgfqpoint{3.547267in}{2.602792in}}%
\pgfpathlineto{\pgfqpoint{3.551808in}{2.602792in}}%
\pgfpathlineto{\pgfqpoint{3.551808in}{2.599843in}}%
\pgfpathmoveto{\pgfqpoint{3.547267in}{2.602792in}}%
\pgfpathlineto{\pgfqpoint{3.547267in}{2.602792in}}%
\pgfpathlineto{\pgfqpoint{3.547267in}{2.605741in}}%
\pgfpathlineto{\pgfqpoint{3.551808in}{2.605741in}}%
\pgfpathlineto{\pgfqpoint{3.551808in}{2.602792in}}%
\pgfpathmoveto{\pgfqpoint{3.551808in}{2.599843in}}%
\pgfpathlineto{\pgfqpoint{3.551808in}{2.599843in}}%
\pgfpathlineto{\pgfqpoint{3.551808in}{2.602792in}}%
\pgfpathlineto{\pgfqpoint{3.556349in}{2.602792in}}%
\pgfpathlineto{\pgfqpoint{3.556349in}{2.599843in}}%
\pgfpathmoveto{\pgfqpoint{3.551808in}{2.602792in}}%
\pgfpathlineto{\pgfqpoint{3.551808in}{2.602792in}}%
\pgfpathlineto{\pgfqpoint{3.551808in}{2.605741in}}%
\pgfpathlineto{\pgfqpoint{3.556349in}{2.605741in}}%
\pgfpathlineto{\pgfqpoint{3.556349in}{2.602792in}}%
\pgfpathmoveto{\pgfqpoint{3.547267in}{2.605741in}}%
\pgfpathlineto{\pgfqpoint{3.547267in}{2.605741in}}%
\pgfpathlineto{\pgfqpoint{3.547267in}{2.608690in}}%
\pgfpathlineto{\pgfqpoint{3.551808in}{2.608690in}}%
\pgfpathlineto{\pgfqpoint{3.551808in}{2.605741in}}%
\pgfpathmoveto{\pgfqpoint{3.547267in}{2.608690in}}%
\pgfpathlineto{\pgfqpoint{3.547267in}{2.608690in}}%
\pgfpathlineto{\pgfqpoint{3.547267in}{2.611639in}}%
\pgfpathlineto{\pgfqpoint{3.551808in}{2.611639in}}%
\pgfpathlineto{\pgfqpoint{3.551808in}{2.608690in}}%
\pgfpathmoveto{\pgfqpoint{3.551808in}{2.605741in}}%
\pgfpathlineto{\pgfqpoint{3.551808in}{2.605741in}}%
\pgfpathlineto{\pgfqpoint{3.551808in}{2.608690in}}%
\pgfpathlineto{\pgfqpoint{3.556349in}{2.608690in}}%
\pgfpathlineto{\pgfqpoint{3.556349in}{2.605741in}}%
\pgfpathmoveto{\pgfqpoint{3.556349in}{2.599843in}}%
\pgfpathlineto{\pgfqpoint{3.556349in}{2.599843in}}%
\pgfpathlineto{\pgfqpoint{3.556349in}{2.602792in}}%
\pgfpathlineto{\pgfqpoint{3.560891in}{2.602792in}}%
\pgfpathlineto{\pgfqpoint{3.560891in}{2.599843in}}%
\pgfpathmoveto{\pgfqpoint{3.556349in}{2.602792in}}%
\pgfpathlineto{\pgfqpoint{3.556349in}{2.602792in}}%
\pgfpathlineto{\pgfqpoint{3.556349in}{2.605741in}}%
\pgfpathlineto{\pgfqpoint{3.560891in}{2.605741in}}%
\pgfpathlineto{\pgfqpoint{3.560891in}{2.602792in}}%
\pgfpathmoveto{\pgfqpoint{3.560891in}{2.599843in}}%
\pgfpathlineto{\pgfqpoint{3.560891in}{2.599843in}}%
\pgfpathlineto{\pgfqpoint{3.560891in}{2.602792in}}%
\pgfpathlineto{\pgfqpoint{3.565432in}{2.602792in}}%
\pgfpathlineto{\pgfqpoint{3.565432in}{2.599843in}}%
\pgfpathmoveto{\pgfqpoint{3.583596in}{2.576249in}}%
\pgfpathlineto{\pgfqpoint{3.583596in}{2.576249in}}%
\pgfpathlineto{\pgfqpoint{3.583596in}{2.579199in}}%
\pgfpathlineto{\pgfqpoint{3.588137in}{2.579199in}}%
\pgfpathlineto{\pgfqpoint{3.588137in}{2.576249in}}%
\pgfpathmoveto{\pgfqpoint{3.583596in}{2.579199in}}%
\pgfpathlineto{\pgfqpoint{3.583596in}{2.579199in}}%
\pgfpathlineto{\pgfqpoint{3.583596in}{2.582148in}}%
\pgfpathlineto{\pgfqpoint{3.588137in}{2.582148in}}%
\pgfpathlineto{\pgfqpoint{3.588137in}{2.579199in}}%
\pgfpathmoveto{\pgfqpoint{3.588137in}{2.576249in}}%
\pgfpathlineto{\pgfqpoint{3.588137in}{2.576249in}}%
\pgfpathlineto{\pgfqpoint{3.588137in}{2.579199in}}%
\pgfpathlineto{\pgfqpoint{3.592678in}{2.579199in}}%
\pgfpathlineto{\pgfqpoint{3.592678in}{2.576249in}}%
\pgfpathmoveto{\pgfqpoint{3.588137in}{2.579199in}}%
\pgfpathlineto{\pgfqpoint{3.588137in}{2.579199in}}%
\pgfpathlineto{\pgfqpoint{3.588137in}{2.582148in}}%
\pgfpathlineto{\pgfqpoint{3.592678in}{2.582148in}}%
\pgfpathlineto{\pgfqpoint{3.592678in}{2.579199in}}%
\pgfpathmoveto{\pgfqpoint{3.583596in}{2.582148in}}%
\pgfpathlineto{\pgfqpoint{3.583596in}{2.582148in}}%
\pgfpathlineto{\pgfqpoint{3.583596in}{2.585097in}}%
\pgfpathlineto{\pgfqpoint{3.588137in}{2.585097in}}%
\pgfpathlineto{\pgfqpoint{3.588137in}{2.582148in}}%
\pgfpathmoveto{\pgfqpoint{3.583596in}{2.585097in}}%
\pgfpathlineto{\pgfqpoint{3.583596in}{2.585097in}}%
\pgfpathlineto{\pgfqpoint{3.583596in}{2.588046in}}%
\pgfpathlineto{\pgfqpoint{3.588137in}{2.588046in}}%
\pgfpathlineto{\pgfqpoint{3.588137in}{2.585097in}}%
\pgfpathmoveto{\pgfqpoint{3.588137in}{2.582148in}}%
\pgfpathlineto{\pgfqpoint{3.588137in}{2.582148in}}%
\pgfpathlineto{\pgfqpoint{3.588137in}{2.585097in}}%
\pgfpathlineto{\pgfqpoint{3.592678in}{2.585097in}}%
\pgfpathlineto{\pgfqpoint{3.592678in}{2.582148in}}%
\pgfpathmoveto{\pgfqpoint{3.592678in}{2.576249in}}%
\pgfpathlineto{\pgfqpoint{3.592678in}{2.576249in}}%
\pgfpathlineto{\pgfqpoint{3.592678in}{2.579199in}}%
\pgfpathlineto{\pgfqpoint{3.597219in}{2.579199in}}%
\pgfpathlineto{\pgfqpoint{3.597219in}{2.576249in}}%
\pgfpathmoveto{\pgfqpoint{3.592678in}{2.579199in}}%
\pgfpathlineto{\pgfqpoint{3.592678in}{2.579199in}}%
\pgfpathlineto{\pgfqpoint{3.592678in}{2.582148in}}%
\pgfpathlineto{\pgfqpoint{3.597219in}{2.582148in}}%
\pgfpathlineto{\pgfqpoint{3.597219in}{2.579199in}}%
\pgfpathmoveto{\pgfqpoint{3.597219in}{2.576249in}}%
\pgfpathlineto{\pgfqpoint{3.597219in}{2.576249in}}%
\pgfpathlineto{\pgfqpoint{3.597219in}{2.579199in}}%
\pgfpathlineto{\pgfqpoint{3.601760in}{2.579199in}}%
\pgfpathlineto{\pgfqpoint{3.601760in}{2.576249in}}%
\pgfpathmoveto{\pgfqpoint{3.656253in}{2.009997in}}%
\pgfpathlineto{\pgfqpoint{3.656253in}{2.009997in}}%
\pgfpathlineto{\pgfqpoint{3.656253in}{2.012947in}}%
\pgfpathlineto{\pgfqpoint{3.660794in}{2.012947in}}%
\pgfpathlineto{\pgfqpoint{3.660794in}{2.009997in}}%
\pgfpathmoveto{\pgfqpoint{3.656253in}{2.012947in}}%
\pgfpathlineto{\pgfqpoint{3.656253in}{2.012947in}}%
\pgfpathlineto{\pgfqpoint{3.656253in}{2.015896in}}%
\pgfpathlineto{\pgfqpoint{3.660794in}{2.015896in}}%
\pgfpathlineto{\pgfqpoint{3.660794in}{2.012947in}}%
\pgfpathmoveto{\pgfqpoint{3.660794in}{2.009997in}}%
\pgfpathlineto{\pgfqpoint{3.660794in}{2.009997in}}%
\pgfpathlineto{\pgfqpoint{3.660794in}{2.012947in}}%
\pgfpathlineto{\pgfqpoint{3.665335in}{2.012947in}}%
\pgfpathlineto{\pgfqpoint{3.665335in}{2.009997in}}%
\pgfpathmoveto{\pgfqpoint{3.660794in}{2.012947in}}%
\pgfpathlineto{\pgfqpoint{3.660794in}{2.012947in}}%
\pgfpathlineto{\pgfqpoint{3.660794in}{2.015896in}}%
\pgfpathlineto{\pgfqpoint{3.665335in}{2.015896in}}%
\pgfpathlineto{\pgfqpoint{3.665335in}{2.012947in}}%
\pgfpathmoveto{\pgfqpoint{3.665335in}{2.009997in}}%
\pgfpathlineto{\pgfqpoint{3.665335in}{2.009997in}}%
\pgfpathlineto{\pgfqpoint{3.665335in}{2.012947in}}%
\pgfpathlineto{\pgfqpoint{3.669876in}{2.012947in}}%
\pgfpathlineto{\pgfqpoint{3.669876in}{2.009997in}}%
\pgfpathmoveto{\pgfqpoint{3.665335in}{2.012947in}}%
\pgfpathlineto{\pgfqpoint{3.665335in}{2.012947in}}%
\pgfpathlineto{\pgfqpoint{3.665335in}{2.015896in}}%
\pgfpathlineto{\pgfqpoint{3.669876in}{2.015896in}}%
\pgfpathlineto{\pgfqpoint{3.669876in}{2.012947in}}%
\pgfpathmoveto{\pgfqpoint{3.669876in}{2.009997in}}%
\pgfpathlineto{\pgfqpoint{3.669876in}{2.009997in}}%
\pgfpathlineto{\pgfqpoint{3.669876in}{2.012947in}}%
\pgfpathlineto{\pgfqpoint{3.674417in}{2.012947in}}%
\pgfpathlineto{\pgfqpoint{3.674417in}{2.009997in}}%
\pgfpathmoveto{\pgfqpoint{3.669876in}{2.012947in}}%
\pgfpathlineto{\pgfqpoint{3.669876in}{2.012947in}}%
\pgfpathlineto{\pgfqpoint{3.669876in}{2.015896in}}%
\pgfpathlineto{\pgfqpoint{3.674417in}{2.015896in}}%
\pgfpathlineto{\pgfqpoint{3.674417in}{2.012947in}}%
\pgfpathmoveto{\pgfqpoint{3.674417in}{2.009997in}}%
\pgfpathlineto{\pgfqpoint{3.674417in}{2.009997in}}%
\pgfpathlineto{\pgfqpoint{3.674417in}{2.012947in}}%
\pgfpathlineto{\pgfqpoint{3.678957in}{2.012947in}}%
\pgfpathlineto{\pgfqpoint{3.678957in}{2.009997in}}%
\pgfpathmoveto{\pgfqpoint{3.674417in}{2.012947in}}%
\pgfpathlineto{\pgfqpoint{3.674417in}{2.012947in}}%
\pgfpathlineto{\pgfqpoint{3.674417in}{2.015896in}}%
\pgfpathlineto{\pgfqpoint{3.678957in}{2.015896in}}%
\pgfpathlineto{\pgfqpoint{3.678957in}{2.012947in}}%
\pgfpathmoveto{\pgfqpoint{3.678957in}{2.009997in}}%
\pgfpathlineto{\pgfqpoint{3.678957in}{2.009997in}}%
\pgfpathlineto{\pgfqpoint{3.678957in}{2.012947in}}%
\pgfpathlineto{\pgfqpoint{3.683498in}{2.012947in}}%
\pgfpathlineto{\pgfqpoint{3.683498in}{2.009997in}}%
\pgfpathmoveto{\pgfqpoint{3.678957in}{2.012947in}}%
\pgfpathlineto{\pgfqpoint{3.678957in}{2.012947in}}%
\pgfpathlineto{\pgfqpoint{3.678957in}{2.015896in}}%
\pgfpathlineto{\pgfqpoint{3.683498in}{2.015896in}}%
\pgfpathlineto{\pgfqpoint{3.683498in}{2.012947in}}%
\pgfpathmoveto{\pgfqpoint{3.683498in}{2.009997in}}%
\pgfpathlineto{\pgfqpoint{3.683498in}{2.009997in}}%
\pgfpathlineto{\pgfqpoint{3.683498in}{2.012947in}}%
\pgfpathlineto{\pgfqpoint{3.688039in}{2.012947in}}%
\pgfpathlineto{\pgfqpoint{3.688039in}{2.009997in}}%
\pgfpathmoveto{\pgfqpoint{3.683498in}{2.012947in}}%
\pgfpathlineto{\pgfqpoint{3.683498in}{2.012947in}}%
\pgfpathlineto{\pgfqpoint{3.683498in}{2.015896in}}%
\pgfpathlineto{\pgfqpoint{3.688039in}{2.015896in}}%
\pgfpathlineto{\pgfqpoint{3.688039in}{2.012947in}}%
\pgfpathmoveto{\pgfqpoint{3.688039in}{2.009997in}}%
\pgfpathlineto{\pgfqpoint{3.688039in}{2.009997in}}%
\pgfpathlineto{\pgfqpoint{3.688039in}{2.012947in}}%
\pgfpathlineto{\pgfqpoint{3.692580in}{2.012947in}}%
\pgfpathlineto{\pgfqpoint{3.692580in}{2.009997in}}%
\pgfpathmoveto{\pgfqpoint{3.688039in}{2.012947in}}%
\pgfpathlineto{\pgfqpoint{3.688039in}{2.012947in}}%
\pgfpathlineto{\pgfqpoint{3.688039in}{2.015896in}}%
\pgfpathlineto{\pgfqpoint{3.692580in}{2.015896in}}%
\pgfpathlineto{\pgfqpoint{3.692580in}{2.012947in}}%
\pgfpathmoveto{\pgfqpoint{3.692580in}{2.009997in}}%
\pgfpathlineto{\pgfqpoint{3.692580in}{2.009997in}}%
\pgfpathlineto{\pgfqpoint{3.692580in}{2.012947in}}%
\pgfpathlineto{\pgfqpoint{3.697121in}{2.012947in}}%
\pgfpathlineto{\pgfqpoint{3.697121in}{2.009997in}}%
\pgfpathmoveto{\pgfqpoint{3.692580in}{2.012947in}}%
\pgfpathlineto{\pgfqpoint{3.692580in}{2.012947in}}%
\pgfpathlineto{\pgfqpoint{3.692580in}{2.015896in}}%
\pgfpathlineto{\pgfqpoint{3.697121in}{2.015896in}}%
\pgfpathlineto{\pgfqpoint{3.697121in}{2.012947in}}%
\pgfpathmoveto{\pgfqpoint{3.697121in}{2.009997in}}%
\pgfpathlineto{\pgfqpoint{3.697121in}{2.009997in}}%
\pgfpathlineto{\pgfqpoint{3.697121in}{2.012947in}}%
\pgfpathlineto{\pgfqpoint{3.701662in}{2.012947in}}%
\pgfpathlineto{\pgfqpoint{3.701662in}{2.009997in}}%
\pgfpathmoveto{\pgfqpoint{3.697121in}{2.012947in}}%
\pgfpathlineto{\pgfqpoint{3.697121in}{2.012947in}}%
\pgfpathlineto{\pgfqpoint{3.697121in}{2.015896in}}%
\pgfpathlineto{\pgfqpoint{3.701662in}{2.015896in}}%
\pgfpathlineto{\pgfqpoint{3.701662in}{2.012947in}}%
\pgfpathmoveto{\pgfqpoint{3.701662in}{2.009997in}}%
\pgfpathlineto{\pgfqpoint{3.701662in}{2.009997in}}%
\pgfpathlineto{\pgfqpoint{3.701662in}{2.012947in}}%
\pgfpathlineto{\pgfqpoint{3.706203in}{2.012947in}}%
\pgfpathlineto{\pgfqpoint{3.706203in}{2.009997in}}%
\pgfpathmoveto{\pgfqpoint{3.701662in}{2.012947in}}%
\pgfpathlineto{\pgfqpoint{3.701662in}{2.012947in}}%
\pgfpathlineto{\pgfqpoint{3.701662in}{2.015896in}}%
\pgfpathlineto{\pgfqpoint{3.706203in}{2.015896in}}%
\pgfpathlineto{\pgfqpoint{3.706203in}{2.012947in}}%
\pgfpathmoveto{\pgfqpoint{3.706203in}{2.009997in}}%
\pgfpathlineto{\pgfqpoint{3.706203in}{2.009997in}}%
\pgfpathlineto{\pgfqpoint{3.706203in}{2.012947in}}%
\pgfpathlineto{\pgfqpoint{3.710744in}{2.012947in}}%
\pgfpathlineto{\pgfqpoint{3.710744in}{2.009997in}}%
\pgfpathmoveto{\pgfqpoint{3.706203in}{2.012947in}}%
\pgfpathlineto{\pgfqpoint{3.706203in}{2.012947in}}%
\pgfpathlineto{\pgfqpoint{3.706203in}{2.015896in}}%
\pgfpathlineto{\pgfqpoint{3.710744in}{2.015896in}}%
\pgfpathlineto{\pgfqpoint{3.710744in}{2.012947in}}%
\pgfpathmoveto{\pgfqpoint{3.710744in}{2.009997in}}%
\pgfpathlineto{\pgfqpoint{3.710744in}{2.009997in}}%
\pgfpathlineto{\pgfqpoint{3.710744in}{2.012947in}}%
\pgfpathlineto{\pgfqpoint{3.715285in}{2.012947in}}%
\pgfpathlineto{\pgfqpoint{3.715285in}{2.009997in}}%
\pgfpathmoveto{\pgfqpoint{3.710744in}{2.012947in}}%
\pgfpathlineto{\pgfqpoint{3.710744in}{2.012947in}}%
\pgfpathlineto{\pgfqpoint{3.710744in}{2.015896in}}%
\pgfpathlineto{\pgfqpoint{3.715285in}{2.015896in}}%
\pgfpathlineto{\pgfqpoint{3.715285in}{2.012947in}}%
\pgfpathmoveto{\pgfqpoint{3.715285in}{2.009997in}}%
\pgfpathlineto{\pgfqpoint{3.715285in}{2.009997in}}%
\pgfpathlineto{\pgfqpoint{3.715285in}{2.012947in}}%
\pgfpathlineto{\pgfqpoint{3.719825in}{2.012947in}}%
\pgfpathlineto{\pgfqpoint{3.719825in}{2.009997in}}%
\pgfpathmoveto{\pgfqpoint{3.715285in}{2.012947in}}%
\pgfpathlineto{\pgfqpoint{3.715285in}{2.012947in}}%
\pgfpathlineto{\pgfqpoint{3.715285in}{2.015896in}}%
\pgfpathlineto{\pgfqpoint{3.719825in}{2.015896in}}%
\pgfpathlineto{\pgfqpoint{3.719825in}{2.012947in}}%
\pgfpathmoveto{\pgfqpoint{3.719825in}{2.009997in}}%
\pgfpathlineto{\pgfqpoint{3.719825in}{2.009997in}}%
\pgfpathlineto{\pgfqpoint{3.719825in}{2.012947in}}%
\pgfpathlineto{\pgfqpoint{3.724366in}{2.012947in}}%
\pgfpathlineto{\pgfqpoint{3.724366in}{2.009997in}}%
\pgfpathmoveto{\pgfqpoint{3.719825in}{2.012947in}}%
\pgfpathlineto{\pgfqpoint{3.719825in}{2.012947in}}%
\pgfpathlineto{\pgfqpoint{3.719825in}{2.015896in}}%
\pgfpathlineto{\pgfqpoint{3.724366in}{2.015896in}}%
\pgfpathlineto{\pgfqpoint{3.724366in}{2.012947in}}%
\pgfpathmoveto{\pgfqpoint{3.724366in}{2.009997in}}%
\pgfpathlineto{\pgfqpoint{3.724366in}{2.009997in}}%
\pgfpathlineto{\pgfqpoint{3.724366in}{2.012947in}}%
\pgfpathlineto{\pgfqpoint{3.728907in}{2.012947in}}%
\pgfpathlineto{\pgfqpoint{3.728907in}{2.009997in}}%
\pgfpathmoveto{\pgfqpoint{3.724366in}{2.012947in}}%
\pgfpathlineto{\pgfqpoint{3.724366in}{2.012947in}}%
\pgfpathlineto{\pgfqpoint{3.724366in}{2.015896in}}%
\pgfpathlineto{\pgfqpoint{3.728907in}{2.015896in}}%
\pgfpathlineto{\pgfqpoint{3.728907in}{2.012947in}}%
\pgfpathmoveto{\pgfqpoint{3.728907in}{2.009997in}}%
\pgfpathlineto{\pgfqpoint{3.728907in}{2.009997in}}%
\pgfpathlineto{\pgfqpoint{3.728907in}{2.012947in}}%
\pgfpathlineto{\pgfqpoint{3.733448in}{2.012947in}}%
\pgfpathlineto{\pgfqpoint{3.733448in}{2.009997in}}%
\pgfpathmoveto{\pgfqpoint{3.728907in}{2.012947in}}%
\pgfpathlineto{\pgfqpoint{3.728907in}{2.012947in}}%
\pgfpathlineto{\pgfqpoint{3.728907in}{2.015896in}}%
\pgfpathlineto{\pgfqpoint{3.733448in}{2.015896in}}%
\pgfpathlineto{\pgfqpoint{3.733448in}{2.012947in}}%
\pgfpathmoveto{\pgfqpoint{3.733448in}{2.009997in}}%
\pgfpathlineto{\pgfqpoint{3.733448in}{2.009997in}}%
\pgfpathlineto{\pgfqpoint{3.733448in}{2.012947in}}%
\pgfpathlineto{\pgfqpoint{3.737989in}{2.012947in}}%
\pgfpathlineto{\pgfqpoint{3.737989in}{2.009997in}}%
\pgfpathmoveto{\pgfqpoint{3.733448in}{2.012947in}}%
\pgfpathlineto{\pgfqpoint{3.733448in}{2.012947in}}%
\pgfpathlineto{\pgfqpoint{3.733448in}{2.015896in}}%
\pgfpathlineto{\pgfqpoint{3.737989in}{2.015896in}}%
\pgfpathlineto{\pgfqpoint{3.737989in}{2.012947in}}%
\pgfpathmoveto{\pgfqpoint{3.737989in}{2.009997in}}%
\pgfpathlineto{\pgfqpoint{3.737989in}{2.009997in}}%
\pgfpathlineto{\pgfqpoint{3.737989in}{2.012947in}}%
\pgfpathlineto{\pgfqpoint{3.742530in}{2.012947in}}%
\pgfpathlineto{\pgfqpoint{3.742530in}{2.009997in}}%
\pgfpathmoveto{\pgfqpoint{3.737989in}{2.012947in}}%
\pgfpathlineto{\pgfqpoint{3.737989in}{2.012947in}}%
\pgfpathlineto{\pgfqpoint{3.737989in}{2.015896in}}%
\pgfpathlineto{\pgfqpoint{3.742530in}{2.015896in}}%
\pgfpathlineto{\pgfqpoint{3.742530in}{2.012947in}}%
\pgfpathmoveto{\pgfqpoint{3.742530in}{2.009997in}}%
\pgfpathlineto{\pgfqpoint{3.742530in}{2.009997in}}%
\pgfpathlineto{\pgfqpoint{3.742530in}{2.012947in}}%
\pgfpathlineto{\pgfqpoint{3.747071in}{2.012947in}}%
\pgfpathlineto{\pgfqpoint{3.747071in}{2.009997in}}%
\pgfpathmoveto{\pgfqpoint{3.742530in}{2.012947in}}%
\pgfpathlineto{\pgfqpoint{3.742530in}{2.012947in}}%
\pgfpathlineto{\pgfqpoint{3.742530in}{2.015896in}}%
\pgfpathlineto{\pgfqpoint{3.747071in}{2.015896in}}%
\pgfpathlineto{\pgfqpoint{3.747071in}{2.012947in}}%
\pgfpathmoveto{\pgfqpoint{3.747071in}{2.009997in}}%
\pgfpathlineto{\pgfqpoint{3.747071in}{2.009997in}}%
\pgfpathlineto{\pgfqpoint{3.747071in}{2.012947in}}%
\pgfpathlineto{\pgfqpoint{3.751612in}{2.012947in}}%
\pgfpathlineto{\pgfqpoint{3.751612in}{2.009997in}}%
\pgfpathmoveto{\pgfqpoint{3.747071in}{2.012947in}}%
\pgfpathlineto{\pgfqpoint{3.747071in}{2.012947in}}%
\pgfpathlineto{\pgfqpoint{3.747071in}{2.015896in}}%
\pgfpathlineto{\pgfqpoint{3.751612in}{2.015896in}}%
\pgfpathlineto{\pgfqpoint{3.751612in}{2.012947in}}%
\pgfpathmoveto{\pgfqpoint{3.751612in}{2.009997in}}%
\pgfpathlineto{\pgfqpoint{3.751612in}{2.009997in}}%
\pgfpathlineto{\pgfqpoint{3.751612in}{2.012947in}}%
\pgfpathlineto{\pgfqpoint{3.756153in}{2.012947in}}%
\pgfpathlineto{\pgfqpoint{3.756153in}{2.009997in}}%
\pgfpathmoveto{\pgfqpoint{3.751612in}{2.012947in}}%
\pgfpathlineto{\pgfqpoint{3.751612in}{2.012947in}}%
\pgfpathlineto{\pgfqpoint{3.751612in}{2.015896in}}%
\pgfpathlineto{\pgfqpoint{3.756153in}{2.015896in}}%
\pgfpathlineto{\pgfqpoint{3.756153in}{2.012947in}}%
\pgfpathmoveto{\pgfqpoint{3.756153in}{2.009997in}}%
\pgfpathlineto{\pgfqpoint{3.756153in}{2.009997in}}%
\pgfpathlineto{\pgfqpoint{3.756153in}{2.012947in}}%
\pgfpathlineto{\pgfqpoint{3.760693in}{2.012947in}}%
\pgfpathlineto{\pgfqpoint{3.760693in}{2.009997in}}%
\pgfpathmoveto{\pgfqpoint{3.756153in}{2.012947in}}%
\pgfpathlineto{\pgfqpoint{3.756153in}{2.012947in}}%
\pgfpathlineto{\pgfqpoint{3.756153in}{2.015896in}}%
\pgfpathlineto{\pgfqpoint{3.760693in}{2.015896in}}%
\pgfpathlineto{\pgfqpoint{3.760693in}{2.012947in}}%
\pgfpathmoveto{\pgfqpoint{3.760693in}{2.009997in}}%
\pgfpathlineto{\pgfqpoint{3.760693in}{2.009997in}}%
\pgfpathlineto{\pgfqpoint{3.760693in}{2.012947in}}%
\pgfpathlineto{\pgfqpoint{3.765234in}{2.012947in}}%
\pgfpathlineto{\pgfqpoint{3.765234in}{2.009997in}}%
\pgfpathmoveto{\pgfqpoint{3.760693in}{2.012947in}}%
\pgfpathlineto{\pgfqpoint{3.760693in}{2.012947in}}%
\pgfpathlineto{\pgfqpoint{3.760693in}{2.015896in}}%
\pgfpathlineto{\pgfqpoint{3.765234in}{2.015896in}}%
\pgfpathlineto{\pgfqpoint{3.765234in}{2.012947in}}%
\pgfpathmoveto{\pgfqpoint{3.765234in}{2.009997in}}%
\pgfpathlineto{\pgfqpoint{3.765234in}{2.009997in}}%
\pgfpathlineto{\pgfqpoint{3.765234in}{2.012947in}}%
\pgfpathlineto{\pgfqpoint{3.769775in}{2.012947in}}%
\pgfpathlineto{\pgfqpoint{3.769775in}{2.009997in}}%
\pgfpathmoveto{\pgfqpoint{3.765234in}{2.012947in}}%
\pgfpathlineto{\pgfqpoint{3.765234in}{2.012947in}}%
\pgfpathlineto{\pgfqpoint{3.765234in}{2.015896in}}%
\pgfpathlineto{\pgfqpoint{3.769775in}{2.015896in}}%
\pgfpathlineto{\pgfqpoint{3.769775in}{2.012947in}}%
\pgfpathmoveto{\pgfqpoint{3.769775in}{2.009997in}}%
\pgfpathlineto{\pgfqpoint{3.769775in}{2.009997in}}%
\pgfpathlineto{\pgfqpoint{3.769775in}{2.012947in}}%
\pgfpathlineto{\pgfqpoint{3.774316in}{2.012947in}}%
\pgfpathlineto{\pgfqpoint{3.774316in}{2.009997in}}%
\pgfpathmoveto{\pgfqpoint{3.769775in}{2.012947in}}%
\pgfpathlineto{\pgfqpoint{3.769775in}{2.012947in}}%
\pgfpathlineto{\pgfqpoint{3.769775in}{2.015896in}}%
\pgfpathlineto{\pgfqpoint{3.774316in}{2.015896in}}%
\pgfpathlineto{\pgfqpoint{3.774316in}{2.012947in}}%
\pgfpathmoveto{\pgfqpoint{3.774316in}{2.009997in}}%
\pgfpathlineto{\pgfqpoint{3.774316in}{2.009997in}}%
\pgfpathlineto{\pgfqpoint{3.774316in}{2.012947in}}%
\pgfpathlineto{\pgfqpoint{3.778857in}{2.012947in}}%
\pgfpathlineto{\pgfqpoint{3.778857in}{2.009997in}}%
\pgfpathmoveto{\pgfqpoint{3.774316in}{2.012947in}}%
\pgfpathlineto{\pgfqpoint{3.774316in}{2.012947in}}%
\pgfpathlineto{\pgfqpoint{3.774316in}{2.015896in}}%
\pgfpathlineto{\pgfqpoint{3.778857in}{2.015896in}}%
\pgfpathlineto{\pgfqpoint{3.778857in}{2.012947in}}%
\pgfpathmoveto{\pgfqpoint{3.778857in}{2.009997in}}%
\pgfpathlineto{\pgfqpoint{3.778857in}{2.009997in}}%
\pgfpathlineto{\pgfqpoint{3.778857in}{2.012947in}}%
\pgfpathlineto{\pgfqpoint{3.783398in}{2.012947in}}%
\pgfpathlineto{\pgfqpoint{3.783398in}{2.009997in}}%
\pgfpathmoveto{\pgfqpoint{3.778857in}{2.012947in}}%
\pgfpathlineto{\pgfqpoint{3.778857in}{2.012947in}}%
\pgfpathlineto{\pgfqpoint{3.778857in}{2.015896in}}%
\pgfpathlineto{\pgfqpoint{3.783398in}{2.015896in}}%
\pgfpathlineto{\pgfqpoint{3.783398in}{2.012947in}}%
\pgfpathmoveto{\pgfqpoint{3.783398in}{2.009997in}}%
\pgfpathlineto{\pgfqpoint{3.783398in}{2.009997in}}%
\pgfpathlineto{\pgfqpoint{3.783398in}{2.012947in}}%
\pgfpathlineto{\pgfqpoint{3.787939in}{2.012947in}}%
\pgfpathlineto{\pgfqpoint{3.787939in}{2.009997in}}%
\pgfpathmoveto{\pgfqpoint{3.783398in}{2.012947in}}%
\pgfpathlineto{\pgfqpoint{3.783398in}{2.012947in}}%
\pgfpathlineto{\pgfqpoint{3.783398in}{2.015896in}}%
\pgfpathlineto{\pgfqpoint{3.787939in}{2.015896in}}%
\pgfpathlineto{\pgfqpoint{3.787939in}{2.012947in}}%
\pgfpathmoveto{\pgfqpoint{3.787939in}{2.009997in}}%
\pgfpathlineto{\pgfqpoint{3.787939in}{2.009997in}}%
\pgfpathlineto{\pgfqpoint{3.787939in}{2.012947in}}%
\pgfpathlineto{\pgfqpoint{3.792480in}{2.012947in}}%
\pgfpathlineto{\pgfqpoint{3.792480in}{2.009997in}}%
\pgfpathmoveto{\pgfqpoint{3.787939in}{2.012947in}}%
\pgfpathlineto{\pgfqpoint{3.787939in}{2.012947in}}%
\pgfpathlineto{\pgfqpoint{3.787939in}{2.015896in}}%
\pgfpathlineto{\pgfqpoint{3.792480in}{2.015896in}}%
\pgfpathlineto{\pgfqpoint{3.792480in}{2.012947in}}%
\pgfpathmoveto{\pgfqpoint{3.792480in}{2.009997in}}%
\pgfpathlineto{\pgfqpoint{3.792480in}{2.009997in}}%
\pgfpathlineto{\pgfqpoint{3.792480in}{2.012947in}}%
\pgfpathlineto{\pgfqpoint{3.797021in}{2.012947in}}%
\pgfpathlineto{\pgfqpoint{3.797021in}{2.009997in}}%
\pgfpathmoveto{\pgfqpoint{3.792480in}{2.012947in}}%
\pgfpathlineto{\pgfqpoint{3.792480in}{2.012947in}}%
\pgfpathlineto{\pgfqpoint{3.792480in}{2.015896in}}%
\pgfpathlineto{\pgfqpoint{3.797021in}{2.015896in}}%
\pgfpathlineto{\pgfqpoint{3.797021in}{2.012947in}}%
\pgfpathmoveto{\pgfqpoint{3.797021in}{2.009997in}}%
\pgfpathlineto{\pgfqpoint{3.797021in}{2.009997in}}%
\pgfpathlineto{\pgfqpoint{3.797021in}{2.012947in}}%
\pgfpathlineto{\pgfqpoint{3.801561in}{2.012947in}}%
\pgfpathlineto{\pgfqpoint{3.801561in}{2.009997in}}%
\pgfpathmoveto{\pgfqpoint{3.797021in}{2.012947in}}%
\pgfpathlineto{\pgfqpoint{3.797021in}{2.012947in}}%
\pgfpathlineto{\pgfqpoint{3.797021in}{2.015896in}}%
\pgfpathlineto{\pgfqpoint{3.801561in}{2.015896in}}%
\pgfpathlineto{\pgfqpoint{3.801561in}{2.012947in}}%
\pgfpathmoveto{\pgfqpoint{3.737989in}{2.475974in}}%
\pgfpathlineto{\pgfqpoint{3.737989in}{2.475974in}}%
\pgfpathlineto{\pgfqpoint{3.737989in}{2.478923in}}%
\pgfpathlineto{\pgfqpoint{3.742530in}{2.478923in}}%
\pgfpathlineto{\pgfqpoint{3.742530in}{2.475974in}}%
\pgfpathmoveto{\pgfqpoint{3.737989in}{2.478923in}}%
\pgfpathlineto{\pgfqpoint{3.737989in}{2.478923in}}%
\pgfpathlineto{\pgfqpoint{3.737989in}{2.481873in}}%
\pgfpathlineto{\pgfqpoint{3.742530in}{2.481873in}}%
\pgfpathlineto{\pgfqpoint{3.742530in}{2.478923in}}%
\pgfpathmoveto{\pgfqpoint{3.742530in}{2.475974in}}%
\pgfpathlineto{\pgfqpoint{3.742530in}{2.475974in}}%
\pgfpathlineto{\pgfqpoint{3.742530in}{2.478923in}}%
\pgfpathlineto{\pgfqpoint{3.747071in}{2.478923in}}%
\pgfpathlineto{\pgfqpoint{3.747071in}{2.475974in}}%
\pgfpathmoveto{\pgfqpoint{3.742530in}{2.478923in}}%
\pgfpathlineto{\pgfqpoint{3.742530in}{2.478923in}}%
\pgfpathlineto{\pgfqpoint{3.742530in}{2.481873in}}%
\pgfpathlineto{\pgfqpoint{3.747071in}{2.481873in}}%
\pgfpathlineto{\pgfqpoint{3.747071in}{2.478923in}}%
\pgfpathmoveto{\pgfqpoint{3.756153in}{2.464178in}}%
\pgfpathlineto{\pgfqpoint{3.756153in}{2.464178in}}%
\pgfpathlineto{\pgfqpoint{3.756153in}{2.467127in}}%
\pgfpathlineto{\pgfqpoint{3.760693in}{2.467127in}}%
\pgfpathlineto{\pgfqpoint{3.760693in}{2.464178in}}%
\pgfpathmoveto{\pgfqpoint{3.756153in}{2.467127in}}%
\pgfpathlineto{\pgfqpoint{3.756153in}{2.467127in}}%
\pgfpathlineto{\pgfqpoint{3.756153in}{2.470076in}}%
\pgfpathlineto{\pgfqpoint{3.760693in}{2.470076in}}%
\pgfpathlineto{\pgfqpoint{3.760693in}{2.467127in}}%
\pgfpathmoveto{\pgfqpoint{3.760693in}{2.464178in}}%
\pgfpathlineto{\pgfqpoint{3.760693in}{2.464178in}}%
\pgfpathlineto{\pgfqpoint{3.760693in}{2.467127in}}%
\pgfpathlineto{\pgfqpoint{3.765234in}{2.467127in}}%
\pgfpathlineto{\pgfqpoint{3.765234in}{2.464178in}}%
\pgfpathmoveto{\pgfqpoint{3.760693in}{2.467127in}}%
\pgfpathlineto{\pgfqpoint{3.760693in}{2.467127in}}%
\pgfpathlineto{\pgfqpoint{3.760693in}{2.470076in}}%
\pgfpathlineto{\pgfqpoint{3.765234in}{2.470076in}}%
\pgfpathlineto{\pgfqpoint{3.765234in}{2.467127in}}%
\pgfpathmoveto{\pgfqpoint{3.747071in}{2.470076in}}%
\pgfpathlineto{\pgfqpoint{3.747071in}{2.470076in}}%
\pgfpathlineto{\pgfqpoint{3.747071in}{2.473025in}}%
\pgfpathlineto{\pgfqpoint{3.751612in}{2.473025in}}%
\pgfpathlineto{\pgfqpoint{3.751612in}{2.470076in}}%
\pgfpathmoveto{\pgfqpoint{3.747071in}{2.473025in}}%
\pgfpathlineto{\pgfqpoint{3.747071in}{2.473025in}}%
\pgfpathlineto{\pgfqpoint{3.747071in}{2.475974in}}%
\pgfpathlineto{\pgfqpoint{3.751612in}{2.475974in}}%
\pgfpathlineto{\pgfqpoint{3.751612in}{2.473025in}}%
\pgfpathmoveto{\pgfqpoint{3.751612in}{2.470076in}}%
\pgfpathlineto{\pgfqpoint{3.751612in}{2.470076in}}%
\pgfpathlineto{\pgfqpoint{3.751612in}{2.473025in}}%
\pgfpathlineto{\pgfqpoint{3.756153in}{2.473025in}}%
\pgfpathlineto{\pgfqpoint{3.756153in}{2.470076in}}%
\pgfpathmoveto{\pgfqpoint{3.751612in}{2.473025in}}%
\pgfpathlineto{\pgfqpoint{3.751612in}{2.473025in}}%
\pgfpathlineto{\pgfqpoint{3.751612in}{2.475974in}}%
\pgfpathlineto{\pgfqpoint{3.756153in}{2.475974in}}%
\pgfpathlineto{\pgfqpoint{3.756153in}{2.473025in}}%
\pgfpathmoveto{\pgfqpoint{3.747071in}{2.475974in}}%
\pgfpathlineto{\pgfqpoint{3.747071in}{2.475974in}}%
\pgfpathlineto{\pgfqpoint{3.747071in}{2.478923in}}%
\pgfpathlineto{\pgfqpoint{3.751612in}{2.478923in}}%
\pgfpathlineto{\pgfqpoint{3.751612in}{2.475974in}}%
\pgfpathmoveto{\pgfqpoint{3.747071in}{2.478923in}}%
\pgfpathlineto{\pgfqpoint{3.747071in}{2.478923in}}%
\pgfpathlineto{\pgfqpoint{3.747071in}{2.481873in}}%
\pgfpathlineto{\pgfqpoint{3.751612in}{2.481873in}}%
\pgfpathlineto{\pgfqpoint{3.751612in}{2.478923in}}%
\pgfpathmoveto{\pgfqpoint{3.751612in}{2.475974in}}%
\pgfpathlineto{\pgfqpoint{3.751612in}{2.475974in}}%
\pgfpathlineto{\pgfqpoint{3.751612in}{2.478923in}}%
\pgfpathlineto{\pgfqpoint{3.756153in}{2.478923in}}%
\pgfpathlineto{\pgfqpoint{3.756153in}{2.475974in}}%
\pgfpathmoveto{\pgfqpoint{3.756153in}{2.470076in}}%
\pgfpathlineto{\pgfqpoint{3.756153in}{2.470076in}}%
\pgfpathlineto{\pgfqpoint{3.756153in}{2.473025in}}%
\pgfpathlineto{\pgfqpoint{3.760693in}{2.473025in}}%
\pgfpathlineto{\pgfqpoint{3.760693in}{2.470076in}}%
\pgfpathmoveto{\pgfqpoint{3.756153in}{2.473025in}}%
\pgfpathlineto{\pgfqpoint{3.756153in}{2.473025in}}%
\pgfpathlineto{\pgfqpoint{3.756153in}{2.475974in}}%
\pgfpathlineto{\pgfqpoint{3.760693in}{2.475974in}}%
\pgfpathlineto{\pgfqpoint{3.760693in}{2.473025in}}%
\pgfpathmoveto{\pgfqpoint{3.760693in}{2.470076in}}%
\pgfpathlineto{\pgfqpoint{3.760693in}{2.470076in}}%
\pgfpathlineto{\pgfqpoint{3.760693in}{2.473025in}}%
\pgfpathlineto{\pgfqpoint{3.765234in}{2.473025in}}%
\pgfpathlineto{\pgfqpoint{3.765234in}{2.470076in}}%
\pgfpathmoveto{\pgfqpoint{3.774316in}{2.452382in}}%
\pgfpathlineto{\pgfqpoint{3.774316in}{2.452382in}}%
\pgfpathlineto{\pgfqpoint{3.774316in}{2.455331in}}%
\pgfpathlineto{\pgfqpoint{3.778857in}{2.455331in}}%
\pgfpathlineto{\pgfqpoint{3.778857in}{2.452382in}}%
\pgfpathmoveto{\pgfqpoint{3.774316in}{2.455331in}}%
\pgfpathlineto{\pgfqpoint{3.774316in}{2.455331in}}%
\pgfpathlineto{\pgfqpoint{3.774316in}{2.458280in}}%
\pgfpathlineto{\pgfqpoint{3.778857in}{2.458280in}}%
\pgfpathlineto{\pgfqpoint{3.778857in}{2.455331in}}%
\pgfpathmoveto{\pgfqpoint{3.778857in}{2.452382in}}%
\pgfpathlineto{\pgfqpoint{3.778857in}{2.452382in}}%
\pgfpathlineto{\pgfqpoint{3.778857in}{2.455331in}}%
\pgfpathlineto{\pgfqpoint{3.783398in}{2.455331in}}%
\pgfpathlineto{\pgfqpoint{3.783398in}{2.452382in}}%
\pgfpathmoveto{\pgfqpoint{3.778857in}{2.455331in}}%
\pgfpathlineto{\pgfqpoint{3.778857in}{2.455331in}}%
\pgfpathlineto{\pgfqpoint{3.778857in}{2.458280in}}%
\pgfpathlineto{\pgfqpoint{3.783398in}{2.458280in}}%
\pgfpathlineto{\pgfqpoint{3.783398in}{2.455331in}}%
\pgfpathmoveto{\pgfqpoint{3.792480in}{2.440585in}}%
\pgfpathlineto{\pgfqpoint{3.792480in}{2.440585in}}%
\pgfpathlineto{\pgfqpoint{3.792480in}{2.443534in}}%
\pgfpathlineto{\pgfqpoint{3.797021in}{2.443534in}}%
\pgfpathlineto{\pgfqpoint{3.797021in}{2.440585in}}%
\pgfpathmoveto{\pgfqpoint{3.792480in}{2.443534in}}%
\pgfpathlineto{\pgfqpoint{3.792480in}{2.443534in}}%
\pgfpathlineto{\pgfqpoint{3.792480in}{2.446483in}}%
\pgfpathlineto{\pgfqpoint{3.797021in}{2.446483in}}%
\pgfpathlineto{\pgfqpoint{3.797021in}{2.443534in}}%
\pgfpathmoveto{\pgfqpoint{3.797021in}{2.440585in}}%
\pgfpathlineto{\pgfqpoint{3.797021in}{2.440585in}}%
\pgfpathlineto{\pgfqpoint{3.797021in}{2.443534in}}%
\pgfpathlineto{\pgfqpoint{3.801561in}{2.443534in}}%
\pgfpathlineto{\pgfqpoint{3.801561in}{2.440585in}}%
\pgfpathmoveto{\pgfqpoint{3.797021in}{2.443534in}}%
\pgfpathlineto{\pgfqpoint{3.797021in}{2.443534in}}%
\pgfpathlineto{\pgfqpoint{3.797021in}{2.446483in}}%
\pgfpathlineto{\pgfqpoint{3.801561in}{2.446483in}}%
\pgfpathlineto{\pgfqpoint{3.801561in}{2.443534in}}%
\pgfpathmoveto{\pgfqpoint{3.783398in}{2.446483in}}%
\pgfpathlineto{\pgfqpoint{3.783398in}{2.446483in}}%
\pgfpathlineto{\pgfqpoint{3.783398in}{2.449433in}}%
\pgfpathlineto{\pgfqpoint{3.787939in}{2.449433in}}%
\pgfpathlineto{\pgfqpoint{3.787939in}{2.446483in}}%
\pgfpathmoveto{\pgfqpoint{3.783398in}{2.449433in}}%
\pgfpathlineto{\pgfqpoint{3.783398in}{2.449433in}}%
\pgfpathlineto{\pgfqpoint{3.783398in}{2.452382in}}%
\pgfpathlineto{\pgfqpoint{3.787939in}{2.452382in}}%
\pgfpathlineto{\pgfqpoint{3.787939in}{2.449433in}}%
\pgfpathmoveto{\pgfqpoint{3.787939in}{2.446483in}}%
\pgfpathlineto{\pgfqpoint{3.787939in}{2.446483in}}%
\pgfpathlineto{\pgfqpoint{3.787939in}{2.449433in}}%
\pgfpathlineto{\pgfqpoint{3.792480in}{2.449433in}}%
\pgfpathlineto{\pgfqpoint{3.792480in}{2.446483in}}%
\pgfpathmoveto{\pgfqpoint{3.787939in}{2.449433in}}%
\pgfpathlineto{\pgfqpoint{3.787939in}{2.449433in}}%
\pgfpathlineto{\pgfqpoint{3.787939in}{2.452382in}}%
\pgfpathlineto{\pgfqpoint{3.792480in}{2.452382in}}%
\pgfpathlineto{\pgfqpoint{3.792480in}{2.449433in}}%
\pgfpathmoveto{\pgfqpoint{3.783398in}{2.452382in}}%
\pgfpathlineto{\pgfqpoint{3.783398in}{2.452382in}}%
\pgfpathlineto{\pgfqpoint{3.783398in}{2.455331in}}%
\pgfpathlineto{\pgfqpoint{3.787939in}{2.455331in}}%
\pgfpathlineto{\pgfqpoint{3.787939in}{2.452382in}}%
\pgfpathmoveto{\pgfqpoint{3.783398in}{2.455331in}}%
\pgfpathlineto{\pgfqpoint{3.783398in}{2.455331in}}%
\pgfpathlineto{\pgfqpoint{3.783398in}{2.458280in}}%
\pgfpathlineto{\pgfqpoint{3.787939in}{2.458280in}}%
\pgfpathlineto{\pgfqpoint{3.787939in}{2.455331in}}%
\pgfpathmoveto{\pgfqpoint{3.787939in}{2.452382in}}%
\pgfpathlineto{\pgfqpoint{3.787939in}{2.452382in}}%
\pgfpathlineto{\pgfqpoint{3.787939in}{2.455331in}}%
\pgfpathlineto{\pgfqpoint{3.792480in}{2.455331in}}%
\pgfpathlineto{\pgfqpoint{3.792480in}{2.452382in}}%
\pgfpathmoveto{\pgfqpoint{3.792480in}{2.446483in}}%
\pgfpathlineto{\pgfqpoint{3.792480in}{2.446483in}}%
\pgfpathlineto{\pgfqpoint{3.792480in}{2.449433in}}%
\pgfpathlineto{\pgfqpoint{3.797021in}{2.449433in}}%
\pgfpathlineto{\pgfqpoint{3.797021in}{2.446483in}}%
\pgfpathmoveto{\pgfqpoint{3.792480in}{2.449433in}}%
\pgfpathlineto{\pgfqpoint{3.792480in}{2.449433in}}%
\pgfpathlineto{\pgfqpoint{3.792480in}{2.452382in}}%
\pgfpathlineto{\pgfqpoint{3.797021in}{2.452382in}}%
\pgfpathlineto{\pgfqpoint{3.797021in}{2.449433in}}%
\pgfpathmoveto{\pgfqpoint{3.797021in}{2.446483in}}%
\pgfpathlineto{\pgfqpoint{3.797021in}{2.446483in}}%
\pgfpathlineto{\pgfqpoint{3.797021in}{2.449433in}}%
\pgfpathlineto{\pgfqpoint{3.801561in}{2.449433in}}%
\pgfpathlineto{\pgfqpoint{3.801561in}{2.446483in}}%
\pgfpathmoveto{\pgfqpoint{3.765234in}{2.458280in}}%
\pgfpathlineto{\pgfqpoint{3.765234in}{2.458280in}}%
\pgfpathlineto{\pgfqpoint{3.765234in}{2.461229in}}%
\pgfpathlineto{\pgfqpoint{3.769775in}{2.461229in}}%
\pgfpathlineto{\pgfqpoint{3.769775in}{2.458280in}}%
\pgfpathmoveto{\pgfqpoint{3.765234in}{2.461229in}}%
\pgfpathlineto{\pgfqpoint{3.765234in}{2.461229in}}%
\pgfpathlineto{\pgfqpoint{3.765234in}{2.464178in}}%
\pgfpathlineto{\pgfqpoint{3.769775in}{2.464178in}}%
\pgfpathlineto{\pgfqpoint{3.769775in}{2.461229in}}%
\pgfpathmoveto{\pgfqpoint{3.769775in}{2.458280in}}%
\pgfpathlineto{\pgfqpoint{3.769775in}{2.458280in}}%
\pgfpathlineto{\pgfqpoint{3.769775in}{2.461229in}}%
\pgfpathlineto{\pgfqpoint{3.774316in}{2.461229in}}%
\pgfpathlineto{\pgfqpoint{3.774316in}{2.458280in}}%
\pgfpathmoveto{\pgfqpoint{3.769775in}{2.461229in}}%
\pgfpathlineto{\pgfqpoint{3.769775in}{2.461229in}}%
\pgfpathlineto{\pgfqpoint{3.769775in}{2.464178in}}%
\pgfpathlineto{\pgfqpoint{3.774316in}{2.464178in}}%
\pgfpathlineto{\pgfqpoint{3.774316in}{2.461229in}}%
\pgfpathmoveto{\pgfqpoint{3.765234in}{2.464178in}}%
\pgfpathlineto{\pgfqpoint{3.765234in}{2.464178in}}%
\pgfpathlineto{\pgfqpoint{3.765234in}{2.467127in}}%
\pgfpathlineto{\pgfqpoint{3.769775in}{2.467127in}}%
\pgfpathlineto{\pgfqpoint{3.769775in}{2.464178in}}%
\pgfpathmoveto{\pgfqpoint{3.765234in}{2.467127in}}%
\pgfpathlineto{\pgfqpoint{3.765234in}{2.467127in}}%
\pgfpathlineto{\pgfqpoint{3.765234in}{2.470076in}}%
\pgfpathlineto{\pgfqpoint{3.769775in}{2.470076in}}%
\pgfpathlineto{\pgfqpoint{3.769775in}{2.467127in}}%
\pgfpathmoveto{\pgfqpoint{3.769775in}{2.464178in}}%
\pgfpathlineto{\pgfqpoint{3.769775in}{2.464178in}}%
\pgfpathlineto{\pgfqpoint{3.769775in}{2.467127in}}%
\pgfpathlineto{\pgfqpoint{3.774316in}{2.467127in}}%
\pgfpathlineto{\pgfqpoint{3.774316in}{2.464178in}}%
\pgfpathmoveto{\pgfqpoint{3.774316in}{2.458280in}}%
\pgfpathlineto{\pgfqpoint{3.774316in}{2.458280in}}%
\pgfpathlineto{\pgfqpoint{3.774316in}{2.461229in}}%
\pgfpathlineto{\pgfqpoint{3.778857in}{2.461229in}}%
\pgfpathlineto{\pgfqpoint{3.778857in}{2.458280in}}%
\pgfpathmoveto{\pgfqpoint{3.774316in}{2.461229in}}%
\pgfpathlineto{\pgfqpoint{3.774316in}{2.461229in}}%
\pgfpathlineto{\pgfqpoint{3.774316in}{2.464178in}}%
\pgfpathlineto{\pgfqpoint{3.778857in}{2.464178in}}%
\pgfpathlineto{\pgfqpoint{3.778857in}{2.461229in}}%
\pgfpathmoveto{\pgfqpoint{3.778857in}{2.458280in}}%
\pgfpathlineto{\pgfqpoint{3.778857in}{2.458280in}}%
\pgfpathlineto{\pgfqpoint{3.778857in}{2.461229in}}%
\pgfpathlineto{\pgfqpoint{3.783398in}{2.461229in}}%
\pgfpathlineto{\pgfqpoint{3.783398in}{2.458280in}}%
\pgfpathmoveto{\pgfqpoint{3.665335in}{2.523162in}}%
\pgfpathlineto{\pgfqpoint{3.665335in}{2.523162in}}%
\pgfpathlineto{\pgfqpoint{3.665335in}{2.526112in}}%
\pgfpathlineto{\pgfqpoint{3.669876in}{2.526112in}}%
\pgfpathlineto{\pgfqpoint{3.669876in}{2.523162in}}%
\pgfpathmoveto{\pgfqpoint{3.665335in}{2.526112in}}%
\pgfpathlineto{\pgfqpoint{3.665335in}{2.526112in}}%
\pgfpathlineto{\pgfqpoint{3.665335in}{2.529061in}}%
\pgfpathlineto{\pgfqpoint{3.669876in}{2.529061in}}%
\pgfpathlineto{\pgfqpoint{3.669876in}{2.526112in}}%
\pgfpathmoveto{\pgfqpoint{3.669876in}{2.523162in}}%
\pgfpathlineto{\pgfqpoint{3.669876in}{2.523162in}}%
\pgfpathlineto{\pgfqpoint{3.669876in}{2.526112in}}%
\pgfpathlineto{\pgfqpoint{3.674417in}{2.526112in}}%
\pgfpathlineto{\pgfqpoint{3.674417in}{2.523162in}}%
\pgfpathmoveto{\pgfqpoint{3.669876in}{2.526112in}}%
\pgfpathlineto{\pgfqpoint{3.669876in}{2.526112in}}%
\pgfpathlineto{\pgfqpoint{3.669876in}{2.529061in}}%
\pgfpathlineto{\pgfqpoint{3.674417in}{2.529061in}}%
\pgfpathlineto{\pgfqpoint{3.674417in}{2.526112in}}%
\pgfpathmoveto{\pgfqpoint{3.683498in}{2.511365in}}%
\pgfpathlineto{\pgfqpoint{3.683498in}{2.511365in}}%
\pgfpathlineto{\pgfqpoint{3.683498in}{2.514315in}}%
\pgfpathlineto{\pgfqpoint{3.688039in}{2.514315in}}%
\pgfpathlineto{\pgfqpoint{3.688039in}{2.511365in}}%
\pgfpathmoveto{\pgfqpoint{3.683498in}{2.514315in}}%
\pgfpathlineto{\pgfqpoint{3.683498in}{2.514315in}}%
\pgfpathlineto{\pgfqpoint{3.683498in}{2.517264in}}%
\pgfpathlineto{\pgfqpoint{3.688039in}{2.517264in}}%
\pgfpathlineto{\pgfqpoint{3.688039in}{2.514315in}}%
\pgfpathmoveto{\pgfqpoint{3.688039in}{2.511365in}}%
\pgfpathlineto{\pgfqpoint{3.688039in}{2.511365in}}%
\pgfpathlineto{\pgfqpoint{3.688039in}{2.514315in}}%
\pgfpathlineto{\pgfqpoint{3.692580in}{2.514315in}}%
\pgfpathlineto{\pgfqpoint{3.692580in}{2.511365in}}%
\pgfpathmoveto{\pgfqpoint{3.688039in}{2.514315in}}%
\pgfpathlineto{\pgfqpoint{3.688039in}{2.514315in}}%
\pgfpathlineto{\pgfqpoint{3.688039in}{2.517264in}}%
\pgfpathlineto{\pgfqpoint{3.692580in}{2.517264in}}%
\pgfpathlineto{\pgfqpoint{3.692580in}{2.514315in}}%
\pgfpathmoveto{\pgfqpoint{3.674417in}{2.517264in}}%
\pgfpathlineto{\pgfqpoint{3.674417in}{2.517264in}}%
\pgfpathlineto{\pgfqpoint{3.674417in}{2.520213in}}%
\pgfpathlineto{\pgfqpoint{3.678957in}{2.520213in}}%
\pgfpathlineto{\pgfqpoint{3.678957in}{2.517264in}}%
\pgfpathmoveto{\pgfqpoint{3.674417in}{2.520213in}}%
\pgfpathlineto{\pgfqpoint{3.674417in}{2.520213in}}%
\pgfpathlineto{\pgfqpoint{3.674417in}{2.523162in}}%
\pgfpathlineto{\pgfqpoint{3.678957in}{2.523162in}}%
\pgfpathlineto{\pgfqpoint{3.678957in}{2.520213in}}%
\pgfpathmoveto{\pgfqpoint{3.678957in}{2.517264in}}%
\pgfpathlineto{\pgfqpoint{3.678957in}{2.517264in}}%
\pgfpathlineto{\pgfqpoint{3.678957in}{2.520213in}}%
\pgfpathlineto{\pgfqpoint{3.683498in}{2.520213in}}%
\pgfpathlineto{\pgfqpoint{3.683498in}{2.517264in}}%
\pgfpathmoveto{\pgfqpoint{3.678957in}{2.520213in}}%
\pgfpathlineto{\pgfqpoint{3.678957in}{2.520213in}}%
\pgfpathlineto{\pgfqpoint{3.678957in}{2.523162in}}%
\pgfpathlineto{\pgfqpoint{3.683498in}{2.523162in}}%
\pgfpathlineto{\pgfqpoint{3.683498in}{2.520213in}}%
\pgfpathmoveto{\pgfqpoint{3.674417in}{2.523162in}}%
\pgfpathlineto{\pgfqpoint{3.674417in}{2.523162in}}%
\pgfpathlineto{\pgfqpoint{3.674417in}{2.526112in}}%
\pgfpathlineto{\pgfqpoint{3.678957in}{2.526112in}}%
\pgfpathlineto{\pgfqpoint{3.678957in}{2.523162in}}%
\pgfpathmoveto{\pgfqpoint{3.674417in}{2.526112in}}%
\pgfpathlineto{\pgfqpoint{3.674417in}{2.526112in}}%
\pgfpathlineto{\pgfqpoint{3.674417in}{2.529061in}}%
\pgfpathlineto{\pgfqpoint{3.678957in}{2.529061in}}%
\pgfpathlineto{\pgfqpoint{3.678957in}{2.526112in}}%
\pgfpathmoveto{\pgfqpoint{3.678957in}{2.523162in}}%
\pgfpathlineto{\pgfqpoint{3.678957in}{2.523162in}}%
\pgfpathlineto{\pgfqpoint{3.678957in}{2.526112in}}%
\pgfpathlineto{\pgfqpoint{3.683498in}{2.526112in}}%
\pgfpathlineto{\pgfqpoint{3.683498in}{2.523162in}}%
\pgfpathmoveto{\pgfqpoint{3.683498in}{2.517264in}}%
\pgfpathlineto{\pgfqpoint{3.683498in}{2.517264in}}%
\pgfpathlineto{\pgfqpoint{3.683498in}{2.520213in}}%
\pgfpathlineto{\pgfqpoint{3.688039in}{2.520213in}}%
\pgfpathlineto{\pgfqpoint{3.688039in}{2.517264in}}%
\pgfpathmoveto{\pgfqpoint{3.683498in}{2.520213in}}%
\pgfpathlineto{\pgfqpoint{3.683498in}{2.520213in}}%
\pgfpathlineto{\pgfqpoint{3.683498in}{2.523162in}}%
\pgfpathlineto{\pgfqpoint{3.688039in}{2.523162in}}%
\pgfpathlineto{\pgfqpoint{3.688039in}{2.520213in}}%
\pgfpathmoveto{\pgfqpoint{3.688039in}{2.517264in}}%
\pgfpathlineto{\pgfqpoint{3.688039in}{2.517264in}}%
\pgfpathlineto{\pgfqpoint{3.688039in}{2.520213in}}%
\pgfpathlineto{\pgfqpoint{3.692580in}{2.520213in}}%
\pgfpathlineto{\pgfqpoint{3.692580in}{2.517264in}}%
\pgfpathmoveto{\pgfqpoint{3.701662in}{2.499568in}}%
\pgfpathlineto{\pgfqpoint{3.701662in}{2.499568in}}%
\pgfpathlineto{\pgfqpoint{3.701662in}{2.502517in}}%
\pgfpathlineto{\pgfqpoint{3.706203in}{2.502517in}}%
\pgfpathlineto{\pgfqpoint{3.706203in}{2.499568in}}%
\pgfpathmoveto{\pgfqpoint{3.701662in}{2.502517in}}%
\pgfpathlineto{\pgfqpoint{3.701662in}{2.502517in}}%
\pgfpathlineto{\pgfqpoint{3.701662in}{2.505467in}}%
\pgfpathlineto{\pgfqpoint{3.706203in}{2.505467in}}%
\pgfpathlineto{\pgfqpoint{3.706203in}{2.502517in}}%
\pgfpathmoveto{\pgfqpoint{3.706203in}{2.499568in}}%
\pgfpathlineto{\pgfqpoint{3.706203in}{2.499568in}}%
\pgfpathlineto{\pgfqpoint{3.706203in}{2.502517in}}%
\pgfpathlineto{\pgfqpoint{3.710744in}{2.502517in}}%
\pgfpathlineto{\pgfqpoint{3.710744in}{2.499568in}}%
\pgfpathmoveto{\pgfqpoint{3.706203in}{2.502517in}}%
\pgfpathlineto{\pgfqpoint{3.706203in}{2.502517in}}%
\pgfpathlineto{\pgfqpoint{3.706203in}{2.505467in}}%
\pgfpathlineto{\pgfqpoint{3.710744in}{2.505467in}}%
\pgfpathlineto{\pgfqpoint{3.710744in}{2.502517in}}%
\pgfpathmoveto{\pgfqpoint{3.719825in}{2.487771in}}%
\pgfpathlineto{\pgfqpoint{3.719825in}{2.487771in}}%
\pgfpathlineto{\pgfqpoint{3.719825in}{2.490720in}}%
\pgfpathlineto{\pgfqpoint{3.724366in}{2.490720in}}%
\pgfpathlineto{\pgfqpoint{3.724366in}{2.487771in}}%
\pgfpathmoveto{\pgfqpoint{3.719825in}{2.490720in}}%
\pgfpathlineto{\pgfqpoint{3.719825in}{2.490720in}}%
\pgfpathlineto{\pgfqpoint{3.719825in}{2.493670in}}%
\pgfpathlineto{\pgfqpoint{3.724366in}{2.493670in}}%
\pgfpathlineto{\pgfqpoint{3.724366in}{2.490720in}}%
\pgfpathmoveto{\pgfqpoint{3.724366in}{2.487771in}}%
\pgfpathlineto{\pgfqpoint{3.724366in}{2.487771in}}%
\pgfpathlineto{\pgfqpoint{3.724366in}{2.490720in}}%
\pgfpathlineto{\pgfqpoint{3.728907in}{2.490720in}}%
\pgfpathlineto{\pgfqpoint{3.728907in}{2.487771in}}%
\pgfpathmoveto{\pgfqpoint{3.724366in}{2.490720in}}%
\pgfpathlineto{\pgfqpoint{3.724366in}{2.490720in}}%
\pgfpathlineto{\pgfqpoint{3.724366in}{2.493670in}}%
\pgfpathlineto{\pgfqpoint{3.728907in}{2.493670in}}%
\pgfpathlineto{\pgfqpoint{3.728907in}{2.490720in}}%
\pgfpathmoveto{\pgfqpoint{3.710744in}{2.493670in}}%
\pgfpathlineto{\pgfqpoint{3.710744in}{2.493670in}}%
\pgfpathlineto{\pgfqpoint{3.710744in}{2.496619in}}%
\pgfpathlineto{\pgfqpoint{3.715285in}{2.496619in}}%
\pgfpathlineto{\pgfqpoint{3.715285in}{2.493670in}}%
\pgfpathmoveto{\pgfqpoint{3.710744in}{2.496619in}}%
\pgfpathlineto{\pgfqpoint{3.710744in}{2.496619in}}%
\pgfpathlineto{\pgfqpoint{3.710744in}{2.499568in}}%
\pgfpathlineto{\pgfqpoint{3.715285in}{2.499568in}}%
\pgfpathlineto{\pgfqpoint{3.715285in}{2.496619in}}%
\pgfpathmoveto{\pgfqpoint{3.715285in}{2.493670in}}%
\pgfpathlineto{\pgfqpoint{3.715285in}{2.493670in}}%
\pgfpathlineto{\pgfqpoint{3.715285in}{2.496619in}}%
\pgfpathlineto{\pgfqpoint{3.719825in}{2.496619in}}%
\pgfpathlineto{\pgfqpoint{3.719825in}{2.493670in}}%
\pgfpathmoveto{\pgfqpoint{3.715285in}{2.496619in}}%
\pgfpathlineto{\pgfqpoint{3.715285in}{2.496619in}}%
\pgfpathlineto{\pgfqpoint{3.715285in}{2.499568in}}%
\pgfpathlineto{\pgfqpoint{3.719825in}{2.499568in}}%
\pgfpathlineto{\pgfqpoint{3.719825in}{2.496619in}}%
\pgfpathmoveto{\pgfqpoint{3.710744in}{2.499568in}}%
\pgfpathlineto{\pgfqpoint{3.710744in}{2.499568in}}%
\pgfpathlineto{\pgfqpoint{3.710744in}{2.502517in}}%
\pgfpathlineto{\pgfqpoint{3.715285in}{2.502517in}}%
\pgfpathlineto{\pgfqpoint{3.715285in}{2.499568in}}%
\pgfpathmoveto{\pgfqpoint{3.710744in}{2.502517in}}%
\pgfpathlineto{\pgfqpoint{3.710744in}{2.502517in}}%
\pgfpathlineto{\pgfqpoint{3.710744in}{2.505467in}}%
\pgfpathlineto{\pgfqpoint{3.715285in}{2.505467in}}%
\pgfpathlineto{\pgfqpoint{3.715285in}{2.502517in}}%
\pgfpathmoveto{\pgfqpoint{3.715285in}{2.499568in}}%
\pgfpathlineto{\pgfqpoint{3.715285in}{2.499568in}}%
\pgfpathlineto{\pgfqpoint{3.715285in}{2.502517in}}%
\pgfpathlineto{\pgfqpoint{3.719825in}{2.502517in}}%
\pgfpathlineto{\pgfqpoint{3.719825in}{2.499568in}}%
\pgfpathmoveto{\pgfqpoint{3.719825in}{2.493670in}}%
\pgfpathlineto{\pgfqpoint{3.719825in}{2.493670in}}%
\pgfpathlineto{\pgfqpoint{3.719825in}{2.496619in}}%
\pgfpathlineto{\pgfqpoint{3.724366in}{2.496619in}}%
\pgfpathlineto{\pgfqpoint{3.724366in}{2.493670in}}%
\pgfpathmoveto{\pgfqpoint{3.719825in}{2.496619in}}%
\pgfpathlineto{\pgfqpoint{3.719825in}{2.496619in}}%
\pgfpathlineto{\pgfqpoint{3.719825in}{2.499568in}}%
\pgfpathlineto{\pgfqpoint{3.724366in}{2.499568in}}%
\pgfpathlineto{\pgfqpoint{3.724366in}{2.496619in}}%
\pgfpathmoveto{\pgfqpoint{3.724366in}{2.493670in}}%
\pgfpathlineto{\pgfqpoint{3.724366in}{2.493670in}}%
\pgfpathlineto{\pgfqpoint{3.724366in}{2.496619in}}%
\pgfpathlineto{\pgfqpoint{3.728907in}{2.496619in}}%
\pgfpathlineto{\pgfqpoint{3.728907in}{2.493670in}}%
\pgfpathmoveto{\pgfqpoint{3.692580in}{2.505467in}}%
\pgfpathlineto{\pgfqpoint{3.692580in}{2.505467in}}%
\pgfpathlineto{\pgfqpoint{3.692580in}{2.508416in}}%
\pgfpathlineto{\pgfqpoint{3.697121in}{2.508416in}}%
\pgfpathlineto{\pgfqpoint{3.697121in}{2.505467in}}%
\pgfpathmoveto{\pgfqpoint{3.692580in}{2.508416in}}%
\pgfpathlineto{\pgfqpoint{3.692580in}{2.508416in}}%
\pgfpathlineto{\pgfqpoint{3.692580in}{2.511365in}}%
\pgfpathlineto{\pgfqpoint{3.697121in}{2.511365in}}%
\pgfpathlineto{\pgfqpoint{3.697121in}{2.508416in}}%
\pgfpathmoveto{\pgfqpoint{3.697121in}{2.505467in}}%
\pgfpathlineto{\pgfqpoint{3.697121in}{2.505467in}}%
\pgfpathlineto{\pgfqpoint{3.697121in}{2.508416in}}%
\pgfpathlineto{\pgfqpoint{3.701662in}{2.508416in}}%
\pgfpathlineto{\pgfqpoint{3.701662in}{2.505467in}}%
\pgfpathmoveto{\pgfqpoint{3.697121in}{2.508416in}}%
\pgfpathlineto{\pgfqpoint{3.697121in}{2.508416in}}%
\pgfpathlineto{\pgfqpoint{3.697121in}{2.511365in}}%
\pgfpathlineto{\pgfqpoint{3.701662in}{2.511365in}}%
\pgfpathlineto{\pgfqpoint{3.701662in}{2.508416in}}%
\pgfpathmoveto{\pgfqpoint{3.692580in}{2.511365in}}%
\pgfpathlineto{\pgfqpoint{3.692580in}{2.511365in}}%
\pgfpathlineto{\pgfqpoint{3.692580in}{2.514315in}}%
\pgfpathlineto{\pgfqpoint{3.697121in}{2.514315in}}%
\pgfpathlineto{\pgfqpoint{3.697121in}{2.511365in}}%
\pgfpathmoveto{\pgfqpoint{3.692580in}{2.514315in}}%
\pgfpathlineto{\pgfqpoint{3.692580in}{2.514315in}}%
\pgfpathlineto{\pgfqpoint{3.692580in}{2.517264in}}%
\pgfpathlineto{\pgfqpoint{3.697121in}{2.517264in}}%
\pgfpathlineto{\pgfqpoint{3.697121in}{2.514315in}}%
\pgfpathmoveto{\pgfqpoint{3.697121in}{2.511365in}}%
\pgfpathlineto{\pgfqpoint{3.697121in}{2.511365in}}%
\pgfpathlineto{\pgfqpoint{3.697121in}{2.514315in}}%
\pgfpathlineto{\pgfqpoint{3.701662in}{2.514315in}}%
\pgfpathlineto{\pgfqpoint{3.701662in}{2.511365in}}%
\pgfpathmoveto{\pgfqpoint{3.701662in}{2.505467in}}%
\pgfpathlineto{\pgfqpoint{3.701662in}{2.505467in}}%
\pgfpathlineto{\pgfqpoint{3.701662in}{2.508416in}}%
\pgfpathlineto{\pgfqpoint{3.706203in}{2.508416in}}%
\pgfpathlineto{\pgfqpoint{3.706203in}{2.505467in}}%
\pgfpathmoveto{\pgfqpoint{3.701662in}{2.508416in}}%
\pgfpathlineto{\pgfqpoint{3.701662in}{2.508416in}}%
\pgfpathlineto{\pgfqpoint{3.701662in}{2.511365in}}%
\pgfpathlineto{\pgfqpoint{3.706203in}{2.511365in}}%
\pgfpathlineto{\pgfqpoint{3.706203in}{2.508416in}}%
\pgfpathmoveto{\pgfqpoint{3.706203in}{2.505467in}}%
\pgfpathlineto{\pgfqpoint{3.706203in}{2.505467in}}%
\pgfpathlineto{\pgfqpoint{3.706203in}{2.508416in}}%
\pgfpathlineto{\pgfqpoint{3.710744in}{2.508416in}}%
\pgfpathlineto{\pgfqpoint{3.710744in}{2.505467in}}%
\pgfpathmoveto{\pgfqpoint{3.656253in}{2.529061in}}%
\pgfpathlineto{\pgfqpoint{3.656253in}{2.529061in}}%
\pgfpathlineto{\pgfqpoint{3.656253in}{2.532010in}}%
\pgfpathlineto{\pgfqpoint{3.660794in}{2.532010in}}%
\pgfpathlineto{\pgfqpoint{3.660794in}{2.529061in}}%
\pgfpathmoveto{\pgfqpoint{3.656253in}{2.532010in}}%
\pgfpathlineto{\pgfqpoint{3.656253in}{2.532010in}}%
\pgfpathlineto{\pgfqpoint{3.656253in}{2.534959in}}%
\pgfpathlineto{\pgfqpoint{3.660794in}{2.534959in}}%
\pgfpathlineto{\pgfqpoint{3.660794in}{2.532010in}}%
\pgfpathmoveto{\pgfqpoint{3.660794in}{2.529061in}}%
\pgfpathlineto{\pgfqpoint{3.660794in}{2.529061in}}%
\pgfpathlineto{\pgfqpoint{3.660794in}{2.532010in}}%
\pgfpathlineto{\pgfqpoint{3.665335in}{2.532010in}}%
\pgfpathlineto{\pgfqpoint{3.665335in}{2.529061in}}%
\pgfpathmoveto{\pgfqpoint{3.660794in}{2.532010in}}%
\pgfpathlineto{\pgfqpoint{3.660794in}{2.532010in}}%
\pgfpathlineto{\pgfqpoint{3.660794in}{2.534959in}}%
\pgfpathlineto{\pgfqpoint{3.665335in}{2.534959in}}%
\pgfpathlineto{\pgfqpoint{3.665335in}{2.532010in}}%
\pgfpathmoveto{\pgfqpoint{3.656253in}{2.534959in}}%
\pgfpathlineto{\pgfqpoint{3.656253in}{2.534959in}}%
\pgfpathlineto{\pgfqpoint{3.656253in}{2.537909in}}%
\pgfpathlineto{\pgfqpoint{3.660794in}{2.537909in}}%
\pgfpathlineto{\pgfqpoint{3.660794in}{2.534959in}}%
\pgfpathmoveto{\pgfqpoint{3.656253in}{2.537909in}}%
\pgfpathlineto{\pgfqpoint{3.656253in}{2.537909in}}%
\pgfpathlineto{\pgfqpoint{3.656253in}{2.540858in}}%
\pgfpathlineto{\pgfqpoint{3.660794in}{2.540858in}}%
\pgfpathlineto{\pgfqpoint{3.660794in}{2.537909in}}%
\pgfpathmoveto{\pgfqpoint{3.660794in}{2.534959in}}%
\pgfpathlineto{\pgfqpoint{3.660794in}{2.534959in}}%
\pgfpathlineto{\pgfqpoint{3.660794in}{2.537909in}}%
\pgfpathlineto{\pgfqpoint{3.665335in}{2.537909in}}%
\pgfpathlineto{\pgfqpoint{3.665335in}{2.534959in}}%
\pgfpathmoveto{\pgfqpoint{3.665335in}{2.529061in}}%
\pgfpathlineto{\pgfqpoint{3.665335in}{2.529061in}}%
\pgfpathlineto{\pgfqpoint{3.665335in}{2.532010in}}%
\pgfpathlineto{\pgfqpoint{3.669876in}{2.532010in}}%
\pgfpathlineto{\pgfqpoint{3.669876in}{2.529061in}}%
\pgfpathmoveto{\pgfqpoint{3.665335in}{2.532010in}}%
\pgfpathlineto{\pgfqpoint{3.665335in}{2.532010in}}%
\pgfpathlineto{\pgfqpoint{3.665335in}{2.534959in}}%
\pgfpathlineto{\pgfqpoint{3.669876in}{2.534959in}}%
\pgfpathlineto{\pgfqpoint{3.669876in}{2.532010in}}%
\pgfpathmoveto{\pgfqpoint{3.669876in}{2.529061in}}%
\pgfpathlineto{\pgfqpoint{3.669876in}{2.529061in}}%
\pgfpathlineto{\pgfqpoint{3.669876in}{2.532010in}}%
\pgfpathlineto{\pgfqpoint{3.674417in}{2.532010in}}%
\pgfpathlineto{\pgfqpoint{3.674417in}{2.529061in}}%
\pgfpathmoveto{\pgfqpoint{3.728907in}{2.481873in}}%
\pgfpathlineto{\pgfqpoint{3.728907in}{2.481873in}}%
\pgfpathlineto{\pgfqpoint{3.728907in}{2.484822in}}%
\pgfpathlineto{\pgfqpoint{3.733448in}{2.484822in}}%
\pgfpathlineto{\pgfqpoint{3.733448in}{2.481873in}}%
\pgfpathmoveto{\pgfqpoint{3.728907in}{2.484822in}}%
\pgfpathlineto{\pgfqpoint{3.728907in}{2.484822in}}%
\pgfpathlineto{\pgfqpoint{3.728907in}{2.487771in}}%
\pgfpathlineto{\pgfqpoint{3.733448in}{2.487771in}}%
\pgfpathlineto{\pgfqpoint{3.733448in}{2.484822in}}%
\pgfpathmoveto{\pgfqpoint{3.733448in}{2.481873in}}%
\pgfpathlineto{\pgfqpoint{3.733448in}{2.481873in}}%
\pgfpathlineto{\pgfqpoint{3.733448in}{2.484822in}}%
\pgfpathlineto{\pgfqpoint{3.737989in}{2.484822in}}%
\pgfpathlineto{\pgfqpoint{3.737989in}{2.481873in}}%
\pgfpathmoveto{\pgfqpoint{3.733448in}{2.484822in}}%
\pgfpathlineto{\pgfqpoint{3.733448in}{2.484822in}}%
\pgfpathlineto{\pgfqpoint{3.733448in}{2.487771in}}%
\pgfpathlineto{\pgfqpoint{3.737989in}{2.487771in}}%
\pgfpathlineto{\pgfqpoint{3.737989in}{2.484822in}}%
\pgfpathmoveto{\pgfqpoint{3.728907in}{2.487771in}}%
\pgfpathlineto{\pgfqpoint{3.728907in}{2.487771in}}%
\pgfpathlineto{\pgfqpoint{3.728907in}{2.490720in}}%
\pgfpathlineto{\pgfqpoint{3.733448in}{2.490720in}}%
\pgfpathlineto{\pgfqpoint{3.733448in}{2.487771in}}%
\pgfpathmoveto{\pgfqpoint{3.728907in}{2.490720in}}%
\pgfpathlineto{\pgfqpoint{3.728907in}{2.490720in}}%
\pgfpathlineto{\pgfqpoint{3.728907in}{2.493670in}}%
\pgfpathlineto{\pgfqpoint{3.733448in}{2.493670in}}%
\pgfpathlineto{\pgfqpoint{3.733448in}{2.490720in}}%
\pgfpathmoveto{\pgfqpoint{3.733448in}{2.487771in}}%
\pgfpathlineto{\pgfqpoint{3.733448in}{2.487771in}}%
\pgfpathlineto{\pgfqpoint{3.733448in}{2.490720in}}%
\pgfpathlineto{\pgfqpoint{3.737989in}{2.490720in}}%
\pgfpathlineto{\pgfqpoint{3.737989in}{2.487771in}}%
\pgfpathmoveto{\pgfqpoint{3.737989in}{2.481873in}}%
\pgfpathlineto{\pgfqpoint{3.737989in}{2.481873in}}%
\pgfpathlineto{\pgfqpoint{3.737989in}{2.484822in}}%
\pgfpathlineto{\pgfqpoint{3.742530in}{2.484822in}}%
\pgfpathlineto{\pgfqpoint{3.742530in}{2.481873in}}%
\pgfpathmoveto{\pgfqpoint{3.737989in}{2.484822in}}%
\pgfpathlineto{\pgfqpoint{3.737989in}{2.484822in}}%
\pgfpathlineto{\pgfqpoint{3.737989in}{2.487771in}}%
\pgfpathlineto{\pgfqpoint{3.742530in}{2.487771in}}%
\pgfpathlineto{\pgfqpoint{3.742530in}{2.484822in}}%
\pgfpathmoveto{\pgfqpoint{3.742530in}{2.481873in}}%
\pgfpathlineto{\pgfqpoint{3.742530in}{2.481873in}}%
\pgfpathlineto{\pgfqpoint{3.742530in}{2.484822in}}%
\pgfpathlineto{\pgfqpoint{3.747071in}{2.484822in}}%
\pgfpathlineto{\pgfqpoint{3.747071in}{2.481873in}}%
\pgfpathmoveto{\pgfqpoint{3.801561in}{2.009997in}}%
\pgfpathlineto{\pgfqpoint{3.801561in}{2.009997in}}%
\pgfpathlineto{\pgfqpoint{3.801561in}{2.012947in}}%
\pgfpathlineto{\pgfqpoint{3.806102in}{2.012947in}}%
\pgfpathlineto{\pgfqpoint{3.806102in}{2.009997in}}%
\pgfpathmoveto{\pgfqpoint{3.801561in}{2.012947in}}%
\pgfpathlineto{\pgfqpoint{3.801561in}{2.012947in}}%
\pgfpathlineto{\pgfqpoint{3.801561in}{2.015896in}}%
\pgfpathlineto{\pgfqpoint{3.806102in}{2.015896in}}%
\pgfpathlineto{\pgfqpoint{3.806102in}{2.012947in}}%
\pgfpathmoveto{\pgfqpoint{3.806102in}{2.009997in}}%
\pgfpathlineto{\pgfqpoint{3.806102in}{2.009997in}}%
\pgfpathlineto{\pgfqpoint{3.806102in}{2.012947in}}%
\pgfpathlineto{\pgfqpoint{3.810643in}{2.012947in}}%
\pgfpathlineto{\pgfqpoint{3.810643in}{2.009997in}}%
\pgfpathmoveto{\pgfqpoint{3.806102in}{2.012947in}}%
\pgfpathlineto{\pgfqpoint{3.806102in}{2.012947in}}%
\pgfpathlineto{\pgfqpoint{3.806102in}{2.015896in}}%
\pgfpathlineto{\pgfqpoint{3.810643in}{2.015896in}}%
\pgfpathlineto{\pgfqpoint{3.810643in}{2.012947in}}%
\pgfpathmoveto{\pgfqpoint{3.810643in}{2.009997in}}%
\pgfpathlineto{\pgfqpoint{3.810643in}{2.009997in}}%
\pgfpathlineto{\pgfqpoint{3.810643in}{2.012947in}}%
\pgfpathlineto{\pgfqpoint{3.815184in}{2.012947in}}%
\pgfpathlineto{\pgfqpoint{3.815184in}{2.009997in}}%
\pgfpathmoveto{\pgfqpoint{3.810643in}{2.012947in}}%
\pgfpathlineto{\pgfqpoint{3.810643in}{2.012947in}}%
\pgfpathlineto{\pgfqpoint{3.810643in}{2.015896in}}%
\pgfpathlineto{\pgfqpoint{3.815184in}{2.015896in}}%
\pgfpathlineto{\pgfqpoint{3.815184in}{2.012947in}}%
\pgfpathmoveto{\pgfqpoint{3.815184in}{2.009997in}}%
\pgfpathlineto{\pgfqpoint{3.815184in}{2.009997in}}%
\pgfpathlineto{\pgfqpoint{3.815184in}{2.012947in}}%
\pgfpathlineto{\pgfqpoint{3.819725in}{2.012947in}}%
\pgfpathlineto{\pgfqpoint{3.819725in}{2.009997in}}%
\pgfpathmoveto{\pgfqpoint{3.815184in}{2.012947in}}%
\pgfpathlineto{\pgfqpoint{3.815184in}{2.012947in}}%
\pgfpathlineto{\pgfqpoint{3.815184in}{2.015896in}}%
\pgfpathlineto{\pgfqpoint{3.819725in}{2.015896in}}%
\pgfpathlineto{\pgfqpoint{3.819725in}{2.012947in}}%
\pgfpathmoveto{\pgfqpoint{3.819725in}{2.009997in}}%
\pgfpathlineto{\pgfqpoint{3.819725in}{2.009997in}}%
\pgfpathlineto{\pgfqpoint{3.819725in}{2.012947in}}%
\pgfpathlineto{\pgfqpoint{3.824266in}{2.012947in}}%
\pgfpathlineto{\pgfqpoint{3.824266in}{2.009997in}}%
\pgfpathmoveto{\pgfqpoint{3.819725in}{2.012947in}}%
\pgfpathlineto{\pgfqpoint{3.819725in}{2.012947in}}%
\pgfpathlineto{\pgfqpoint{3.819725in}{2.015896in}}%
\pgfpathlineto{\pgfqpoint{3.824266in}{2.015896in}}%
\pgfpathlineto{\pgfqpoint{3.824266in}{2.012947in}}%
\pgfpathmoveto{\pgfqpoint{3.824266in}{2.009997in}}%
\pgfpathlineto{\pgfqpoint{3.824266in}{2.009997in}}%
\pgfpathlineto{\pgfqpoint{3.824266in}{2.012947in}}%
\pgfpathlineto{\pgfqpoint{3.828808in}{2.012947in}}%
\pgfpathlineto{\pgfqpoint{3.828808in}{2.009997in}}%
\pgfpathmoveto{\pgfqpoint{3.824266in}{2.012947in}}%
\pgfpathlineto{\pgfqpoint{3.824266in}{2.012947in}}%
\pgfpathlineto{\pgfqpoint{3.824266in}{2.015896in}}%
\pgfpathlineto{\pgfqpoint{3.828808in}{2.015896in}}%
\pgfpathlineto{\pgfqpoint{3.828808in}{2.012947in}}%
\pgfpathmoveto{\pgfqpoint{3.828808in}{2.009997in}}%
\pgfpathlineto{\pgfqpoint{3.828808in}{2.009997in}}%
\pgfpathlineto{\pgfqpoint{3.828808in}{2.012947in}}%
\pgfpathlineto{\pgfqpoint{3.833349in}{2.012947in}}%
\pgfpathlineto{\pgfqpoint{3.833349in}{2.009997in}}%
\pgfpathmoveto{\pgfqpoint{3.828808in}{2.012947in}}%
\pgfpathlineto{\pgfqpoint{3.828808in}{2.012947in}}%
\pgfpathlineto{\pgfqpoint{3.828808in}{2.015896in}}%
\pgfpathlineto{\pgfqpoint{3.833349in}{2.015896in}}%
\pgfpathlineto{\pgfqpoint{3.833349in}{2.012947in}}%
\pgfpathmoveto{\pgfqpoint{3.833349in}{2.009997in}}%
\pgfpathlineto{\pgfqpoint{3.833349in}{2.009997in}}%
\pgfpathlineto{\pgfqpoint{3.833349in}{2.012947in}}%
\pgfpathlineto{\pgfqpoint{3.837890in}{2.012947in}}%
\pgfpathlineto{\pgfqpoint{3.837890in}{2.009997in}}%
\pgfpathmoveto{\pgfqpoint{3.833349in}{2.012947in}}%
\pgfpathlineto{\pgfqpoint{3.833349in}{2.012947in}}%
\pgfpathlineto{\pgfqpoint{3.833349in}{2.015896in}}%
\pgfpathlineto{\pgfqpoint{3.837890in}{2.015896in}}%
\pgfpathlineto{\pgfqpoint{3.837890in}{2.012947in}}%
\pgfpathmoveto{\pgfqpoint{3.837890in}{2.009997in}}%
\pgfpathlineto{\pgfqpoint{3.837890in}{2.009997in}}%
\pgfpathlineto{\pgfqpoint{3.837890in}{2.012947in}}%
\pgfpathlineto{\pgfqpoint{3.842431in}{2.012947in}}%
\pgfpathlineto{\pgfqpoint{3.842431in}{2.009997in}}%
\pgfpathmoveto{\pgfqpoint{3.837890in}{2.012947in}}%
\pgfpathlineto{\pgfqpoint{3.837890in}{2.012947in}}%
\pgfpathlineto{\pgfqpoint{3.837890in}{2.015896in}}%
\pgfpathlineto{\pgfqpoint{3.842431in}{2.015896in}}%
\pgfpathlineto{\pgfqpoint{3.842431in}{2.012947in}}%
\pgfpathmoveto{\pgfqpoint{3.842431in}{2.009997in}}%
\pgfpathlineto{\pgfqpoint{3.842431in}{2.009997in}}%
\pgfpathlineto{\pgfqpoint{3.842431in}{2.012947in}}%
\pgfpathlineto{\pgfqpoint{3.846972in}{2.012947in}}%
\pgfpathlineto{\pgfqpoint{3.846972in}{2.009997in}}%
\pgfpathmoveto{\pgfqpoint{3.842431in}{2.012947in}}%
\pgfpathlineto{\pgfqpoint{3.842431in}{2.012947in}}%
\pgfpathlineto{\pgfqpoint{3.842431in}{2.015896in}}%
\pgfpathlineto{\pgfqpoint{3.846972in}{2.015896in}}%
\pgfpathlineto{\pgfqpoint{3.846972in}{2.012947in}}%
\pgfpathmoveto{\pgfqpoint{3.846972in}{2.009997in}}%
\pgfpathlineto{\pgfqpoint{3.846972in}{2.009997in}}%
\pgfpathlineto{\pgfqpoint{3.846972in}{2.012947in}}%
\pgfpathlineto{\pgfqpoint{3.851513in}{2.012947in}}%
\pgfpathlineto{\pgfqpoint{3.851513in}{2.009997in}}%
\pgfpathmoveto{\pgfqpoint{3.846972in}{2.012947in}}%
\pgfpathlineto{\pgfqpoint{3.846972in}{2.012947in}}%
\pgfpathlineto{\pgfqpoint{3.846972in}{2.015896in}}%
\pgfpathlineto{\pgfqpoint{3.851513in}{2.015896in}}%
\pgfpathlineto{\pgfqpoint{3.851513in}{2.012947in}}%
\pgfpathmoveto{\pgfqpoint{3.851513in}{2.009997in}}%
\pgfpathlineto{\pgfqpoint{3.851513in}{2.009997in}}%
\pgfpathlineto{\pgfqpoint{3.851513in}{2.012947in}}%
\pgfpathlineto{\pgfqpoint{3.856054in}{2.012947in}}%
\pgfpathlineto{\pgfqpoint{3.856054in}{2.009997in}}%
\pgfpathmoveto{\pgfqpoint{3.851513in}{2.012947in}}%
\pgfpathlineto{\pgfqpoint{3.851513in}{2.012947in}}%
\pgfpathlineto{\pgfqpoint{3.851513in}{2.015896in}}%
\pgfpathlineto{\pgfqpoint{3.856054in}{2.015896in}}%
\pgfpathlineto{\pgfqpoint{3.856054in}{2.012947in}}%
\pgfpathmoveto{\pgfqpoint{3.856054in}{2.009997in}}%
\pgfpathlineto{\pgfqpoint{3.856054in}{2.009997in}}%
\pgfpathlineto{\pgfqpoint{3.856054in}{2.012947in}}%
\pgfpathlineto{\pgfqpoint{3.860595in}{2.012947in}}%
\pgfpathlineto{\pgfqpoint{3.860595in}{2.009997in}}%
\pgfpathmoveto{\pgfqpoint{3.856054in}{2.012947in}}%
\pgfpathlineto{\pgfqpoint{3.856054in}{2.012947in}}%
\pgfpathlineto{\pgfqpoint{3.856054in}{2.015896in}}%
\pgfpathlineto{\pgfqpoint{3.860595in}{2.015896in}}%
\pgfpathlineto{\pgfqpoint{3.860595in}{2.012947in}}%
\pgfpathmoveto{\pgfqpoint{3.860595in}{2.009997in}}%
\pgfpathlineto{\pgfqpoint{3.860595in}{2.009997in}}%
\pgfpathlineto{\pgfqpoint{3.860595in}{2.012947in}}%
\pgfpathlineto{\pgfqpoint{3.865136in}{2.012947in}}%
\pgfpathlineto{\pgfqpoint{3.865136in}{2.009997in}}%
\pgfpathmoveto{\pgfqpoint{3.860595in}{2.012947in}}%
\pgfpathlineto{\pgfqpoint{3.860595in}{2.012947in}}%
\pgfpathlineto{\pgfqpoint{3.860595in}{2.015896in}}%
\pgfpathlineto{\pgfqpoint{3.865136in}{2.015896in}}%
\pgfpathlineto{\pgfqpoint{3.865136in}{2.012947in}}%
\pgfpathmoveto{\pgfqpoint{3.865136in}{2.009997in}}%
\pgfpathlineto{\pgfqpoint{3.865136in}{2.009997in}}%
\pgfpathlineto{\pgfqpoint{3.865136in}{2.012947in}}%
\pgfpathlineto{\pgfqpoint{3.869677in}{2.012947in}}%
\pgfpathlineto{\pgfqpoint{3.869677in}{2.009997in}}%
\pgfpathmoveto{\pgfqpoint{3.865136in}{2.012947in}}%
\pgfpathlineto{\pgfqpoint{3.865136in}{2.012947in}}%
\pgfpathlineto{\pgfqpoint{3.865136in}{2.015896in}}%
\pgfpathlineto{\pgfqpoint{3.869677in}{2.015896in}}%
\pgfpathlineto{\pgfqpoint{3.869677in}{2.012947in}}%
\pgfpathmoveto{\pgfqpoint{3.869677in}{2.009997in}}%
\pgfpathlineto{\pgfqpoint{3.869677in}{2.009997in}}%
\pgfpathlineto{\pgfqpoint{3.869677in}{2.012947in}}%
\pgfpathlineto{\pgfqpoint{3.874218in}{2.012947in}}%
\pgfpathlineto{\pgfqpoint{3.874218in}{2.009997in}}%
\pgfpathmoveto{\pgfqpoint{3.869677in}{2.012947in}}%
\pgfpathlineto{\pgfqpoint{3.869677in}{2.012947in}}%
\pgfpathlineto{\pgfqpoint{3.869677in}{2.015896in}}%
\pgfpathlineto{\pgfqpoint{3.874218in}{2.015896in}}%
\pgfpathlineto{\pgfqpoint{3.874218in}{2.012947in}}%
\pgfpathmoveto{\pgfqpoint{3.874218in}{2.009997in}}%
\pgfpathlineto{\pgfqpoint{3.874218in}{2.009997in}}%
\pgfpathlineto{\pgfqpoint{3.874218in}{2.012947in}}%
\pgfpathlineto{\pgfqpoint{3.878759in}{2.012947in}}%
\pgfpathlineto{\pgfqpoint{3.878759in}{2.009997in}}%
\pgfpathmoveto{\pgfqpoint{3.874218in}{2.012947in}}%
\pgfpathlineto{\pgfqpoint{3.874218in}{2.012947in}}%
\pgfpathlineto{\pgfqpoint{3.874218in}{2.015896in}}%
\pgfpathlineto{\pgfqpoint{3.878759in}{2.015896in}}%
\pgfpathlineto{\pgfqpoint{3.878759in}{2.012947in}}%
\pgfpathmoveto{\pgfqpoint{3.878759in}{2.009997in}}%
\pgfpathlineto{\pgfqpoint{3.878759in}{2.009997in}}%
\pgfpathlineto{\pgfqpoint{3.878759in}{2.012947in}}%
\pgfpathlineto{\pgfqpoint{3.883300in}{2.012947in}}%
\pgfpathlineto{\pgfqpoint{3.883300in}{2.009997in}}%
\pgfpathmoveto{\pgfqpoint{3.878759in}{2.012947in}}%
\pgfpathlineto{\pgfqpoint{3.878759in}{2.012947in}}%
\pgfpathlineto{\pgfqpoint{3.878759in}{2.015896in}}%
\pgfpathlineto{\pgfqpoint{3.883300in}{2.015896in}}%
\pgfpathlineto{\pgfqpoint{3.883300in}{2.012947in}}%
\pgfpathmoveto{\pgfqpoint{3.883300in}{2.009997in}}%
\pgfpathlineto{\pgfqpoint{3.883300in}{2.009997in}}%
\pgfpathlineto{\pgfqpoint{3.883300in}{2.012947in}}%
\pgfpathlineto{\pgfqpoint{3.887841in}{2.012947in}}%
\pgfpathlineto{\pgfqpoint{3.887841in}{2.009997in}}%
\pgfpathmoveto{\pgfqpoint{3.883300in}{2.012947in}}%
\pgfpathlineto{\pgfqpoint{3.883300in}{2.012947in}}%
\pgfpathlineto{\pgfqpoint{3.883300in}{2.015896in}}%
\pgfpathlineto{\pgfqpoint{3.887841in}{2.015896in}}%
\pgfpathlineto{\pgfqpoint{3.887841in}{2.012947in}}%
\pgfpathmoveto{\pgfqpoint{3.887841in}{2.009997in}}%
\pgfpathlineto{\pgfqpoint{3.887841in}{2.009997in}}%
\pgfpathlineto{\pgfqpoint{3.887841in}{2.012947in}}%
\pgfpathlineto{\pgfqpoint{3.892382in}{2.012947in}}%
\pgfpathlineto{\pgfqpoint{3.892382in}{2.009997in}}%
\pgfpathmoveto{\pgfqpoint{3.887841in}{2.012947in}}%
\pgfpathlineto{\pgfqpoint{3.887841in}{2.012947in}}%
\pgfpathlineto{\pgfqpoint{3.887841in}{2.015896in}}%
\pgfpathlineto{\pgfqpoint{3.892382in}{2.015896in}}%
\pgfpathlineto{\pgfqpoint{3.892382in}{2.012947in}}%
\pgfpathmoveto{\pgfqpoint{3.892382in}{2.009997in}}%
\pgfpathlineto{\pgfqpoint{3.892382in}{2.009997in}}%
\pgfpathlineto{\pgfqpoint{3.892382in}{2.012947in}}%
\pgfpathlineto{\pgfqpoint{3.896923in}{2.012947in}}%
\pgfpathlineto{\pgfqpoint{3.896923in}{2.009997in}}%
\pgfpathmoveto{\pgfqpoint{3.892382in}{2.012947in}}%
\pgfpathlineto{\pgfqpoint{3.892382in}{2.012947in}}%
\pgfpathlineto{\pgfqpoint{3.892382in}{2.015896in}}%
\pgfpathlineto{\pgfqpoint{3.896923in}{2.015896in}}%
\pgfpathlineto{\pgfqpoint{3.896923in}{2.012947in}}%
\pgfpathmoveto{\pgfqpoint{3.896923in}{2.009997in}}%
\pgfpathlineto{\pgfqpoint{3.896923in}{2.009997in}}%
\pgfpathlineto{\pgfqpoint{3.896923in}{2.012947in}}%
\pgfpathlineto{\pgfqpoint{3.901464in}{2.012947in}}%
\pgfpathlineto{\pgfqpoint{3.901464in}{2.009997in}}%
\pgfpathmoveto{\pgfqpoint{3.896923in}{2.012947in}}%
\pgfpathlineto{\pgfqpoint{3.896923in}{2.012947in}}%
\pgfpathlineto{\pgfqpoint{3.896923in}{2.015896in}}%
\pgfpathlineto{\pgfqpoint{3.901464in}{2.015896in}}%
\pgfpathlineto{\pgfqpoint{3.901464in}{2.012947in}}%
\pgfpathmoveto{\pgfqpoint{3.901464in}{2.009997in}}%
\pgfpathlineto{\pgfqpoint{3.901464in}{2.009997in}}%
\pgfpathlineto{\pgfqpoint{3.901464in}{2.012947in}}%
\pgfpathlineto{\pgfqpoint{3.906005in}{2.012947in}}%
\pgfpathlineto{\pgfqpoint{3.906005in}{2.009997in}}%
\pgfpathmoveto{\pgfqpoint{3.901464in}{2.012947in}}%
\pgfpathlineto{\pgfqpoint{3.901464in}{2.012947in}}%
\pgfpathlineto{\pgfqpoint{3.901464in}{2.015896in}}%
\pgfpathlineto{\pgfqpoint{3.906005in}{2.015896in}}%
\pgfpathlineto{\pgfqpoint{3.906005in}{2.012947in}}%
\pgfpathmoveto{\pgfqpoint{3.906005in}{2.009997in}}%
\pgfpathlineto{\pgfqpoint{3.906005in}{2.009997in}}%
\pgfpathlineto{\pgfqpoint{3.906005in}{2.012947in}}%
\pgfpathlineto{\pgfqpoint{3.910546in}{2.012947in}}%
\pgfpathlineto{\pgfqpoint{3.910546in}{2.009997in}}%
\pgfpathmoveto{\pgfqpoint{3.906005in}{2.012947in}}%
\pgfpathlineto{\pgfqpoint{3.906005in}{2.012947in}}%
\pgfpathlineto{\pgfqpoint{3.906005in}{2.015896in}}%
\pgfpathlineto{\pgfqpoint{3.910546in}{2.015896in}}%
\pgfpathlineto{\pgfqpoint{3.910546in}{2.012947in}}%
\pgfpathmoveto{\pgfqpoint{3.910546in}{2.009997in}}%
\pgfpathlineto{\pgfqpoint{3.910546in}{2.009997in}}%
\pgfpathlineto{\pgfqpoint{3.910546in}{2.012947in}}%
\pgfpathlineto{\pgfqpoint{3.915087in}{2.012947in}}%
\pgfpathlineto{\pgfqpoint{3.915087in}{2.009997in}}%
\pgfpathmoveto{\pgfqpoint{3.910546in}{2.012947in}}%
\pgfpathlineto{\pgfqpoint{3.910546in}{2.012947in}}%
\pgfpathlineto{\pgfqpoint{3.910546in}{2.015896in}}%
\pgfpathlineto{\pgfqpoint{3.915087in}{2.015896in}}%
\pgfpathlineto{\pgfqpoint{3.915087in}{2.012947in}}%
\pgfpathmoveto{\pgfqpoint{3.915087in}{2.009997in}}%
\pgfpathlineto{\pgfqpoint{3.915087in}{2.009997in}}%
\pgfpathlineto{\pgfqpoint{3.915087in}{2.012947in}}%
\pgfpathlineto{\pgfqpoint{3.919628in}{2.012947in}}%
\pgfpathlineto{\pgfqpoint{3.919628in}{2.009997in}}%
\pgfpathmoveto{\pgfqpoint{3.915087in}{2.012947in}}%
\pgfpathlineto{\pgfqpoint{3.915087in}{2.012947in}}%
\pgfpathlineto{\pgfqpoint{3.915087in}{2.015896in}}%
\pgfpathlineto{\pgfqpoint{3.919628in}{2.015896in}}%
\pgfpathlineto{\pgfqpoint{3.919628in}{2.012947in}}%
\pgfpathmoveto{\pgfqpoint{3.919628in}{2.009997in}}%
\pgfpathlineto{\pgfqpoint{3.919628in}{2.009997in}}%
\pgfpathlineto{\pgfqpoint{3.919628in}{2.012947in}}%
\pgfpathlineto{\pgfqpoint{3.924169in}{2.012947in}}%
\pgfpathlineto{\pgfqpoint{3.924169in}{2.009997in}}%
\pgfpathmoveto{\pgfqpoint{3.919628in}{2.012947in}}%
\pgfpathlineto{\pgfqpoint{3.919628in}{2.012947in}}%
\pgfpathlineto{\pgfqpoint{3.919628in}{2.015896in}}%
\pgfpathlineto{\pgfqpoint{3.924169in}{2.015896in}}%
\pgfpathlineto{\pgfqpoint{3.924169in}{2.012947in}}%
\pgfpathmoveto{\pgfqpoint{3.924169in}{2.009997in}}%
\pgfpathlineto{\pgfqpoint{3.924169in}{2.009997in}}%
\pgfpathlineto{\pgfqpoint{3.924169in}{2.012947in}}%
\pgfpathlineto{\pgfqpoint{3.928710in}{2.012947in}}%
\pgfpathlineto{\pgfqpoint{3.928710in}{2.009997in}}%
\pgfpathmoveto{\pgfqpoint{3.924169in}{2.012947in}}%
\pgfpathlineto{\pgfqpoint{3.924169in}{2.012947in}}%
\pgfpathlineto{\pgfqpoint{3.924169in}{2.015896in}}%
\pgfpathlineto{\pgfqpoint{3.928710in}{2.015896in}}%
\pgfpathlineto{\pgfqpoint{3.928710in}{2.012947in}}%
\pgfpathmoveto{\pgfqpoint{3.928710in}{2.009997in}}%
\pgfpathlineto{\pgfqpoint{3.928710in}{2.009997in}}%
\pgfpathlineto{\pgfqpoint{3.928710in}{2.012947in}}%
\pgfpathlineto{\pgfqpoint{3.933251in}{2.012947in}}%
\pgfpathlineto{\pgfqpoint{3.933251in}{2.009997in}}%
\pgfpathmoveto{\pgfqpoint{3.928710in}{2.012947in}}%
\pgfpathlineto{\pgfqpoint{3.928710in}{2.012947in}}%
\pgfpathlineto{\pgfqpoint{3.928710in}{2.015896in}}%
\pgfpathlineto{\pgfqpoint{3.933251in}{2.015896in}}%
\pgfpathlineto{\pgfqpoint{3.933251in}{2.012947in}}%
\pgfpathmoveto{\pgfqpoint{3.933251in}{2.009997in}}%
\pgfpathlineto{\pgfqpoint{3.933251in}{2.009997in}}%
\pgfpathlineto{\pgfqpoint{3.933251in}{2.012947in}}%
\pgfpathlineto{\pgfqpoint{3.937792in}{2.012947in}}%
\pgfpathlineto{\pgfqpoint{3.937792in}{2.009997in}}%
\pgfpathmoveto{\pgfqpoint{3.933251in}{2.012947in}}%
\pgfpathlineto{\pgfqpoint{3.933251in}{2.012947in}}%
\pgfpathlineto{\pgfqpoint{3.933251in}{2.015896in}}%
\pgfpathlineto{\pgfqpoint{3.937792in}{2.015896in}}%
\pgfpathlineto{\pgfqpoint{3.937792in}{2.012947in}}%
\pgfpathmoveto{\pgfqpoint{3.937792in}{2.009997in}}%
\pgfpathlineto{\pgfqpoint{3.937792in}{2.009997in}}%
\pgfpathlineto{\pgfqpoint{3.937792in}{2.012947in}}%
\pgfpathlineto{\pgfqpoint{3.942333in}{2.012947in}}%
\pgfpathlineto{\pgfqpoint{3.942333in}{2.009997in}}%
\pgfpathmoveto{\pgfqpoint{3.937792in}{2.012947in}}%
\pgfpathlineto{\pgfqpoint{3.937792in}{2.012947in}}%
\pgfpathlineto{\pgfqpoint{3.937792in}{2.015896in}}%
\pgfpathlineto{\pgfqpoint{3.942333in}{2.015896in}}%
\pgfpathlineto{\pgfqpoint{3.942333in}{2.012947in}}%
\pgfpathmoveto{\pgfqpoint{3.942333in}{2.009997in}}%
\pgfpathlineto{\pgfqpoint{3.942333in}{2.009997in}}%
\pgfpathlineto{\pgfqpoint{3.942333in}{2.012947in}}%
\pgfpathlineto{\pgfqpoint{3.946874in}{2.012947in}}%
\pgfpathlineto{\pgfqpoint{3.946874in}{2.009997in}}%
\pgfpathmoveto{\pgfqpoint{3.942333in}{2.012947in}}%
\pgfpathlineto{\pgfqpoint{3.942333in}{2.012947in}}%
\pgfpathlineto{\pgfqpoint{3.942333in}{2.015896in}}%
\pgfpathlineto{\pgfqpoint{3.946874in}{2.015896in}}%
\pgfpathlineto{\pgfqpoint{3.946874in}{2.012947in}}%
\pgfpathmoveto{\pgfqpoint{3.883300in}{2.381603in}}%
\pgfpathlineto{\pgfqpoint{3.883300in}{2.381603in}}%
\pgfpathlineto{\pgfqpoint{3.883300in}{2.384552in}}%
\pgfpathlineto{\pgfqpoint{3.887841in}{2.384552in}}%
\pgfpathlineto{\pgfqpoint{3.887841in}{2.381603in}}%
\pgfpathmoveto{\pgfqpoint{3.883300in}{2.384552in}}%
\pgfpathlineto{\pgfqpoint{3.883300in}{2.384552in}}%
\pgfpathlineto{\pgfqpoint{3.883300in}{2.387502in}}%
\pgfpathlineto{\pgfqpoint{3.887841in}{2.387502in}}%
\pgfpathlineto{\pgfqpoint{3.887841in}{2.384552in}}%
\pgfpathmoveto{\pgfqpoint{3.887841in}{2.381603in}}%
\pgfpathlineto{\pgfqpoint{3.887841in}{2.381603in}}%
\pgfpathlineto{\pgfqpoint{3.887841in}{2.384552in}}%
\pgfpathlineto{\pgfqpoint{3.892382in}{2.384552in}}%
\pgfpathlineto{\pgfqpoint{3.892382in}{2.381603in}}%
\pgfpathmoveto{\pgfqpoint{3.887841in}{2.384552in}}%
\pgfpathlineto{\pgfqpoint{3.887841in}{2.384552in}}%
\pgfpathlineto{\pgfqpoint{3.887841in}{2.387502in}}%
\pgfpathlineto{\pgfqpoint{3.892382in}{2.387502in}}%
\pgfpathlineto{\pgfqpoint{3.892382in}{2.384552in}}%
\pgfpathmoveto{\pgfqpoint{3.901464in}{2.369806in}}%
\pgfpathlineto{\pgfqpoint{3.901464in}{2.369806in}}%
\pgfpathlineto{\pgfqpoint{3.901464in}{2.372755in}}%
\pgfpathlineto{\pgfqpoint{3.906005in}{2.372755in}}%
\pgfpathlineto{\pgfqpoint{3.906005in}{2.369806in}}%
\pgfpathmoveto{\pgfqpoint{3.901464in}{2.372755in}}%
\pgfpathlineto{\pgfqpoint{3.901464in}{2.372755in}}%
\pgfpathlineto{\pgfqpoint{3.901464in}{2.375705in}}%
\pgfpathlineto{\pgfqpoint{3.906005in}{2.375705in}}%
\pgfpathlineto{\pgfqpoint{3.906005in}{2.372755in}}%
\pgfpathmoveto{\pgfqpoint{3.906005in}{2.369806in}}%
\pgfpathlineto{\pgfqpoint{3.906005in}{2.369806in}}%
\pgfpathlineto{\pgfqpoint{3.906005in}{2.372755in}}%
\pgfpathlineto{\pgfqpoint{3.910546in}{2.372755in}}%
\pgfpathlineto{\pgfqpoint{3.910546in}{2.369806in}}%
\pgfpathmoveto{\pgfqpoint{3.906005in}{2.372755in}}%
\pgfpathlineto{\pgfqpoint{3.906005in}{2.372755in}}%
\pgfpathlineto{\pgfqpoint{3.906005in}{2.375705in}}%
\pgfpathlineto{\pgfqpoint{3.910546in}{2.375705in}}%
\pgfpathlineto{\pgfqpoint{3.910546in}{2.372755in}}%
\pgfpathmoveto{\pgfqpoint{3.892382in}{2.375705in}}%
\pgfpathlineto{\pgfqpoint{3.892382in}{2.375705in}}%
\pgfpathlineto{\pgfqpoint{3.892382in}{2.378654in}}%
\pgfpathlineto{\pgfqpoint{3.896923in}{2.378654in}}%
\pgfpathlineto{\pgfqpoint{3.896923in}{2.375705in}}%
\pgfpathmoveto{\pgfqpoint{3.892382in}{2.378654in}}%
\pgfpathlineto{\pgfqpoint{3.892382in}{2.378654in}}%
\pgfpathlineto{\pgfqpoint{3.892382in}{2.381603in}}%
\pgfpathlineto{\pgfqpoint{3.896923in}{2.381603in}}%
\pgfpathlineto{\pgfqpoint{3.896923in}{2.378654in}}%
\pgfpathmoveto{\pgfqpoint{3.896923in}{2.375705in}}%
\pgfpathlineto{\pgfqpoint{3.896923in}{2.375705in}}%
\pgfpathlineto{\pgfqpoint{3.896923in}{2.378654in}}%
\pgfpathlineto{\pgfqpoint{3.901464in}{2.378654in}}%
\pgfpathlineto{\pgfqpoint{3.901464in}{2.375705in}}%
\pgfpathmoveto{\pgfqpoint{3.896923in}{2.378654in}}%
\pgfpathlineto{\pgfqpoint{3.896923in}{2.378654in}}%
\pgfpathlineto{\pgfqpoint{3.896923in}{2.381603in}}%
\pgfpathlineto{\pgfqpoint{3.901464in}{2.381603in}}%
\pgfpathlineto{\pgfqpoint{3.901464in}{2.378654in}}%
\pgfpathmoveto{\pgfqpoint{3.892382in}{2.381603in}}%
\pgfpathlineto{\pgfqpoint{3.892382in}{2.381603in}}%
\pgfpathlineto{\pgfqpoint{3.892382in}{2.384552in}}%
\pgfpathlineto{\pgfqpoint{3.896923in}{2.384552in}}%
\pgfpathlineto{\pgfqpoint{3.896923in}{2.381603in}}%
\pgfpathmoveto{\pgfqpoint{3.892382in}{2.384552in}}%
\pgfpathlineto{\pgfqpoint{3.892382in}{2.384552in}}%
\pgfpathlineto{\pgfqpoint{3.892382in}{2.387502in}}%
\pgfpathlineto{\pgfqpoint{3.896923in}{2.387502in}}%
\pgfpathlineto{\pgfqpoint{3.896923in}{2.384552in}}%
\pgfpathmoveto{\pgfqpoint{3.896923in}{2.381603in}}%
\pgfpathlineto{\pgfqpoint{3.896923in}{2.381603in}}%
\pgfpathlineto{\pgfqpoint{3.896923in}{2.384552in}}%
\pgfpathlineto{\pgfqpoint{3.901464in}{2.384552in}}%
\pgfpathlineto{\pgfqpoint{3.901464in}{2.381603in}}%
\pgfpathmoveto{\pgfqpoint{3.901464in}{2.375705in}}%
\pgfpathlineto{\pgfqpoint{3.901464in}{2.375705in}}%
\pgfpathlineto{\pgfqpoint{3.901464in}{2.378654in}}%
\pgfpathlineto{\pgfqpoint{3.906005in}{2.378654in}}%
\pgfpathlineto{\pgfqpoint{3.906005in}{2.375705in}}%
\pgfpathmoveto{\pgfqpoint{3.901464in}{2.378654in}}%
\pgfpathlineto{\pgfqpoint{3.901464in}{2.378654in}}%
\pgfpathlineto{\pgfqpoint{3.901464in}{2.381603in}}%
\pgfpathlineto{\pgfqpoint{3.906005in}{2.381603in}}%
\pgfpathlineto{\pgfqpoint{3.906005in}{2.378654in}}%
\pgfpathmoveto{\pgfqpoint{3.906005in}{2.375705in}}%
\pgfpathlineto{\pgfqpoint{3.906005in}{2.375705in}}%
\pgfpathlineto{\pgfqpoint{3.906005in}{2.378654in}}%
\pgfpathlineto{\pgfqpoint{3.910546in}{2.378654in}}%
\pgfpathlineto{\pgfqpoint{3.910546in}{2.375705in}}%
\pgfpathmoveto{\pgfqpoint{3.919628in}{2.358009in}}%
\pgfpathlineto{\pgfqpoint{3.919628in}{2.358009in}}%
\pgfpathlineto{\pgfqpoint{3.919628in}{2.360958in}}%
\pgfpathlineto{\pgfqpoint{3.924169in}{2.360958in}}%
\pgfpathlineto{\pgfqpoint{3.924169in}{2.358009in}}%
\pgfpathmoveto{\pgfqpoint{3.919628in}{2.360958in}}%
\pgfpathlineto{\pgfqpoint{3.919628in}{2.360958in}}%
\pgfpathlineto{\pgfqpoint{3.919628in}{2.363907in}}%
\pgfpathlineto{\pgfqpoint{3.924169in}{2.363907in}}%
\pgfpathlineto{\pgfqpoint{3.924169in}{2.360958in}}%
\pgfpathmoveto{\pgfqpoint{3.924169in}{2.358009in}}%
\pgfpathlineto{\pgfqpoint{3.924169in}{2.358009in}}%
\pgfpathlineto{\pgfqpoint{3.924169in}{2.360958in}}%
\pgfpathlineto{\pgfqpoint{3.928710in}{2.360958in}}%
\pgfpathlineto{\pgfqpoint{3.928710in}{2.358009in}}%
\pgfpathmoveto{\pgfqpoint{3.924169in}{2.360958in}}%
\pgfpathlineto{\pgfqpoint{3.924169in}{2.360958in}}%
\pgfpathlineto{\pgfqpoint{3.924169in}{2.363907in}}%
\pgfpathlineto{\pgfqpoint{3.928710in}{2.363907in}}%
\pgfpathlineto{\pgfqpoint{3.928710in}{2.360958in}}%
\pgfpathmoveto{\pgfqpoint{3.937792in}{2.346212in}}%
\pgfpathlineto{\pgfqpoint{3.937792in}{2.346212in}}%
\pgfpathlineto{\pgfqpoint{3.937792in}{2.349161in}}%
\pgfpathlineto{\pgfqpoint{3.942333in}{2.349161in}}%
\pgfpathlineto{\pgfqpoint{3.942333in}{2.346212in}}%
\pgfpathmoveto{\pgfqpoint{3.937792in}{2.349161in}}%
\pgfpathlineto{\pgfqpoint{3.937792in}{2.349161in}}%
\pgfpathlineto{\pgfqpoint{3.937792in}{2.352110in}}%
\pgfpathlineto{\pgfqpoint{3.942333in}{2.352110in}}%
\pgfpathlineto{\pgfqpoint{3.942333in}{2.349161in}}%
\pgfpathmoveto{\pgfqpoint{3.942333in}{2.346212in}}%
\pgfpathlineto{\pgfqpoint{3.942333in}{2.346212in}}%
\pgfpathlineto{\pgfqpoint{3.942333in}{2.349161in}}%
\pgfpathlineto{\pgfqpoint{3.946874in}{2.349161in}}%
\pgfpathlineto{\pgfqpoint{3.946874in}{2.346212in}}%
\pgfpathmoveto{\pgfqpoint{3.942333in}{2.349161in}}%
\pgfpathlineto{\pgfqpoint{3.942333in}{2.349161in}}%
\pgfpathlineto{\pgfqpoint{3.942333in}{2.352110in}}%
\pgfpathlineto{\pgfqpoint{3.946874in}{2.352110in}}%
\pgfpathlineto{\pgfqpoint{3.946874in}{2.349161in}}%
\pgfpathmoveto{\pgfqpoint{3.928710in}{2.352110in}}%
\pgfpathlineto{\pgfqpoint{3.928710in}{2.352110in}}%
\pgfpathlineto{\pgfqpoint{3.928710in}{2.355060in}}%
\pgfpathlineto{\pgfqpoint{3.933251in}{2.355060in}}%
\pgfpathlineto{\pgfqpoint{3.933251in}{2.352110in}}%
\pgfpathmoveto{\pgfqpoint{3.928710in}{2.355060in}}%
\pgfpathlineto{\pgfqpoint{3.928710in}{2.355060in}}%
\pgfpathlineto{\pgfqpoint{3.928710in}{2.358009in}}%
\pgfpathlineto{\pgfqpoint{3.933251in}{2.358009in}}%
\pgfpathlineto{\pgfqpoint{3.933251in}{2.355060in}}%
\pgfpathmoveto{\pgfqpoint{3.933251in}{2.352110in}}%
\pgfpathlineto{\pgfqpoint{3.933251in}{2.352110in}}%
\pgfpathlineto{\pgfqpoint{3.933251in}{2.355060in}}%
\pgfpathlineto{\pgfqpoint{3.937792in}{2.355060in}}%
\pgfpathlineto{\pgfqpoint{3.937792in}{2.352110in}}%
\pgfpathmoveto{\pgfqpoint{3.933251in}{2.355060in}}%
\pgfpathlineto{\pgfqpoint{3.933251in}{2.355060in}}%
\pgfpathlineto{\pgfqpoint{3.933251in}{2.358009in}}%
\pgfpathlineto{\pgfqpoint{3.937792in}{2.358009in}}%
\pgfpathlineto{\pgfqpoint{3.937792in}{2.355060in}}%
\pgfpathmoveto{\pgfqpoint{3.928710in}{2.358009in}}%
\pgfpathlineto{\pgfqpoint{3.928710in}{2.358009in}}%
\pgfpathlineto{\pgfqpoint{3.928710in}{2.360958in}}%
\pgfpathlineto{\pgfqpoint{3.933251in}{2.360958in}}%
\pgfpathlineto{\pgfqpoint{3.933251in}{2.358009in}}%
\pgfpathmoveto{\pgfqpoint{3.928710in}{2.360958in}}%
\pgfpathlineto{\pgfqpoint{3.928710in}{2.360958in}}%
\pgfpathlineto{\pgfqpoint{3.928710in}{2.363907in}}%
\pgfpathlineto{\pgfqpoint{3.933251in}{2.363907in}}%
\pgfpathlineto{\pgfqpoint{3.933251in}{2.360958in}}%
\pgfpathmoveto{\pgfqpoint{3.933251in}{2.358009in}}%
\pgfpathlineto{\pgfqpoint{3.933251in}{2.358009in}}%
\pgfpathlineto{\pgfqpoint{3.933251in}{2.360958in}}%
\pgfpathlineto{\pgfqpoint{3.937792in}{2.360958in}}%
\pgfpathlineto{\pgfqpoint{3.937792in}{2.358009in}}%
\pgfpathmoveto{\pgfqpoint{3.937792in}{2.352110in}}%
\pgfpathlineto{\pgfqpoint{3.937792in}{2.352110in}}%
\pgfpathlineto{\pgfqpoint{3.937792in}{2.355060in}}%
\pgfpathlineto{\pgfqpoint{3.942333in}{2.355060in}}%
\pgfpathlineto{\pgfqpoint{3.942333in}{2.352110in}}%
\pgfpathmoveto{\pgfqpoint{3.937792in}{2.355060in}}%
\pgfpathlineto{\pgfqpoint{3.937792in}{2.355060in}}%
\pgfpathlineto{\pgfqpoint{3.937792in}{2.358009in}}%
\pgfpathlineto{\pgfqpoint{3.942333in}{2.358009in}}%
\pgfpathlineto{\pgfqpoint{3.942333in}{2.355060in}}%
\pgfpathmoveto{\pgfqpoint{3.942333in}{2.352110in}}%
\pgfpathlineto{\pgfqpoint{3.942333in}{2.352110in}}%
\pgfpathlineto{\pgfqpoint{3.942333in}{2.355060in}}%
\pgfpathlineto{\pgfqpoint{3.946874in}{2.355060in}}%
\pgfpathlineto{\pgfqpoint{3.946874in}{2.352110in}}%
\pgfpathmoveto{\pgfqpoint{3.910546in}{2.363907in}}%
\pgfpathlineto{\pgfqpoint{3.910546in}{2.363907in}}%
\pgfpathlineto{\pgfqpoint{3.910546in}{2.366857in}}%
\pgfpathlineto{\pgfqpoint{3.915087in}{2.366857in}}%
\pgfpathlineto{\pgfqpoint{3.915087in}{2.363907in}}%
\pgfpathmoveto{\pgfqpoint{3.910546in}{2.366857in}}%
\pgfpathlineto{\pgfqpoint{3.910546in}{2.366857in}}%
\pgfpathlineto{\pgfqpoint{3.910546in}{2.369806in}}%
\pgfpathlineto{\pgfqpoint{3.915087in}{2.369806in}}%
\pgfpathlineto{\pgfqpoint{3.915087in}{2.366857in}}%
\pgfpathmoveto{\pgfqpoint{3.915087in}{2.363907in}}%
\pgfpathlineto{\pgfqpoint{3.915087in}{2.363907in}}%
\pgfpathlineto{\pgfqpoint{3.915087in}{2.366857in}}%
\pgfpathlineto{\pgfqpoint{3.919628in}{2.366857in}}%
\pgfpathlineto{\pgfqpoint{3.919628in}{2.363907in}}%
\pgfpathmoveto{\pgfqpoint{3.915087in}{2.366857in}}%
\pgfpathlineto{\pgfqpoint{3.915087in}{2.366857in}}%
\pgfpathlineto{\pgfqpoint{3.915087in}{2.369806in}}%
\pgfpathlineto{\pgfqpoint{3.919628in}{2.369806in}}%
\pgfpathlineto{\pgfqpoint{3.919628in}{2.366857in}}%
\pgfpathmoveto{\pgfqpoint{3.910546in}{2.369806in}}%
\pgfpathlineto{\pgfqpoint{3.910546in}{2.369806in}}%
\pgfpathlineto{\pgfqpoint{3.910546in}{2.372755in}}%
\pgfpathlineto{\pgfqpoint{3.915087in}{2.372755in}}%
\pgfpathlineto{\pgfqpoint{3.915087in}{2.369806in}}%
\pgfpathmoveto{\pgfqpoint{3.910546in}{2.372755in}}%
\pgfpathlineto{\pgfqpoint{3.910546in}{2.372755in}}%
\pgfpathlineto{\pgfqpoint{3.910546in}{2.375705in}}%
\pgfpathlineto{\pgfqpoint{3.915087in}{2.375705in}}%
\pgfpathlineto{\pgfqpoint{3.915087in}{2.372755in}}%
\pgfpathmoveto{\pgfqpoint{3.915087in}{2.369806in}}%
\pgfpathlineto{\pgfqpoint{3.915087in}{2.369806in}}%
\pgfpathlineto{\pgfqpoint{3.915087in}{2.372755in}}%
\pgfpathlineto{\pgfqpoint{3.919628in}{2.372755in}}%
\pgfpathlineto{\pgfqpoint{3.919628in}{2.369806in}}%
\pgfpathmoveto{\pgfqpoint{3.919628in}{2.363907in}}%
\pgfpathlineto{\pgfqpoint{3.919628in}{2.363907in}}%
\pgfpathlineto{\pgfqpoint{3.919628in}{2.366857in}}%
\pgfpathlineto{\pgfqpoint{3.924169in}{2.366857in}}%
\pgfpathlineto{\pgfqpoint{3.924169in}{2.363907in}}%
\pgfpathmoveto{\pgfqpoint{3.919628in}{2.366857in}}%
\pgfpathlineto{\pgfqpoint{3.919628in}{2.366857in}}%
\pgfpathlineto{\pgfqpoint{3.919628in}{2.369806in}}%
\pgfpathlineto{\pgfqpoint{3.924169in}{2.369806in}}%
\pgfpathlineto{\pgfqpoint{3.924169in}{2.366857in}}%
\pgfpathmoveto{\pgfqpoint{3.924169in}{2.363907in}}%
\pgfpathlineto{\pgfqpoint{3.924169in}{2.363907in}}%
\pgfpathlineto{\pgfqpoint{3.924169in}{2.366857in}}%
\pgfpathlineto{\pgfqpoint{3.928710in}{2.366857in}}%
\pgfpathlineto{\pgfqpoint{3.928710in}{2.363907in}}%
\pgfpathmoveto{\pgfqpoint{3.810643in}{2.428789in}}%
\pgfpathlineto{\pgfqpoint{3.810643in}{2.428789in}}%
\pgfpathlineto{\pgfqpoint{3.810643in}{2.431738in}}%
\pgfpathlineto{\pgfqpoint{3.815184in}{2.431738in}}%
\pgfpathlineto{\pgfqpoint{3.815184in}{2.428789in}}%
\pgfpathmoveto{\pgfqpoint{3.810643in}{2.431738in}}%
\pgfpathlineto{\pgfqpoint{3.810643in}{2.431738in}}%
\pgfpathlineto{\pgfqpoint{3.810643in}{2.434687in}}%
\pgfpathlineto{\pgfqpoint{3.815184in}{2.434687in}}%
\pgfpathlineto{\pgfqpoint{3.815184in}{2.431738in}}%
\pgfpathmoveto{\pgfqpoint{3.815184in}{2.428789in}}%
\pgfpathlineto{\pgfqpoint{3.815184in}{2.428789in}}%
\pgfpathlineto{\pgfqpoint{3.815184in}{2.431738in}}%
\pgfpathlineto{\pgfqpoint{3.819725in}{2.431738in}}%
\pgfpathlineto{\pgfqpoint{3.819725in}{2.428789in}}%
\pgfpathmoveto{\pgfqpoint{3.815184in}{2.431738in}}%
\pgfpathlineto{\pgfqpoint{3.815184in}{2.431738in}}%
\pgfpathlineto{\pgfqpoint{3.815184in}{2.434687in}}%
\pgfpathlineto{\pgfqpoint{3.819725in}{2.434687in}}%
\pgfpathlineto{\pgfqpoint{3.819725in}{2.431738in}}%
\pgfpathmoveto{\pgfqpoint{3.828808in}{2.416993in}}%
\pgfpathlineto{\pgfqpoint{3.828808in}{2.416993in}}%
\pgfpathlineto{\pgfqpoint{3.828808in}{2.419942in}}%
\pgfpathlineto{\pgfqpoint{3.833349in}{2.419942in}}%
\pgfpathlineto{\pgfqpoint{3.833349in}{2.416993in}}%
\pgfpathmoveto{\pgfqpoint{3.828808in}{2.419942in}}%
\pgfpathlineto{\pgfqpoint{3.828808in}{2.419942in}}%
\pgfpathlineto{\pgfqpoint{3.828808in}{2.422891in}}%
\pgfpathlineto{\pgfqpoint{3.833349in}{2.422891in}}%
\pgfpathlineto{\pgfqpoint{3.833349in}{2.419942in}}%
\pgfpathmoveto{\pgfqpoint{3.833349in}{2.416993in}}%
\pgfpathlineto{\pgfqpoint{3.833349in}{2.416993in}}%
\pgfpathlineto{\pgfqpoint{3.833349in}{2.419942in}}%
\pgfpathlineto{\pgfqpoint{3.837890in}{2.419942in}}%
\pgfpathlineto{\pgfqpoint{3.837890in}{2.416993in}}%
\pgfpathmoveto{\pgfqpoint{3.833349in}{2.419942in}}%
\pgfpathlineto{\pgfqpoint{3.833349in}{2.419942in}}%
\pgfpathlineto{\pgfqpoint{3.833349in}{2.422891in}}%
\pgfpathlineto{\pgfqpoint{3.837890in}{2.422891in}}%
\pgfpathlineto{\pgfqpoint{3.837890in}{2.419942in}}%
\pgfpathmoveto{\pgfqpoint{3.819725in}{2.422891in}}%
\pgfpathlineto{\pgfqpoint{3.819725in}{2.422891in}}%
\pgfpathlineto{\pgfqpoint{3.819725in}{2.425840in}}%
\pgfpathlineto{\pgfqpoint{3.824266in}{2.425840in}}%
\pgfpathlineto{\pgfqpoint{3.824266in}{2.422891in}}%
\pgfpathmoveto{\pgfqpoint{3.819725in}{2.425840in}}%
\pgfpathlineto{\pgfqpoint{3.819725in}{2.425840in}}%
\pgfpathlineto{\pgfqpoint{3.819725in}{2.428789in}}%
\pgfpathlineto{\pgfqpoint{3.824266in}{2.428789in}}%
\pgfpathlineto{\pgfqpoint{3.824266in}{2.425840in}}%
\pgfpathmoveto{\pgfqpoint{3.824266in}{2.422891in}}%
\pgfpathlineto{\pgfqpoint{3.824266in}{2.422891in}}%
\pgfpathlineto{\pgfqpoint{3.824266in}{2.425840in}}%
\pgfpathlineto{\pgfqpoint{3.828808in}{2.425840in}}%
\pgfpathlineto{\pgfqpoint{3.828808in}{2.422891in}}%
\pgfpathmoveto{\pgfqpoint{3.824266in}{2.425840in}}%
\pgfpathlineto{\pgfqpoint{3.824266in}{2.425840in}}%
\pgfpathlineto{\pgfqpoint{3.824266in}{2.428789in}}%
\pgfpathlineto{\pgfqpoint{3.828808in}{2.428789in}}%
\pgfpathlineto{\pgfqpoint{3.828808in}{2.425840in}}%
\pgfpathmoveto{\pgfqpoint{3.819725in}{2.428789in}}%
\pgfpathlineto{\pgfqpoint{3.819725in}{2.428789in}}%
\pgfpathlineto{\pgfqpoint{3.819725in}{2.431738in}}%
\pgfpathlineto{\pgfqpoint{3.824266in}{2.431738in}}%
\pgfpathlineto{\pgfqpoint{3.824266in}{2.428789in}}%
\pgfpathmoveto{\pgfqpoint{3.819725in}{2.431738in}}%
\pgfpathlineto{\pgfqpoint{3.819725in}{2.431738in}}%
\pgfpathlineto{\pgfqpoint{3.819725in}{2.434687in}}%
\pgfpathlineto{\pgfqpoint{3.824266in}{2.434687in}}%
\pgfpathlineto{\pgfqpoint{3.824266in}{2.431738in}}%
\pgfpathmoveto{\pgfqpoint{3.824266in}{2.428789in}}%
\pgfpathlineto{\pgfqpoint{3.824266in}{2.428789in}}%
\pgfpathlineto{\pgfqpoint{3.824266in}{2.431738in}}%
\pgfpathlineto{\pgfqpoint{3.828808in}{2.431738in}}%
\pgfpathlineto{\pgfqpoint{3.828808in}{2.428789in}}%
\pgfpathmoveto{\pgfqpoint{3.828808in}{2.422891in}}%
\pgfpathlineto{\pgfqpoint{3.828808in}{2.422891in}}%
\pgfpathlineto{\pgfqpoint{3.828808in}{2.425840in}}%
\pgfpathlineto{\pgfqpoint{3.833349in}{2.425840in}}%
\pgfpathlineto{\pgfqpoint{3.833349in}{2.422891in}}%
\pgfpathmoveto{\pgfqpoint{3.828808in}{2.425840in}}%
\pgfpathlineto{\pgfqpoint{3.828808in}{2.425840in}}%
\pgfpathlineto{\pgfqpoint{3.828808in}{2.428789in}}%
\pgfpathlineto{\pgfqpoint{3.833349in}{2.428789in}}%
\pgfpathlineto{\pgfqpoint{3.833349in}{2.425840in}}%
\pgfpathmoveto{\pgfqpoint{3.833349in}{2.422891in}}%
\pgfpathlineto{\pgfqpoint{3.833349in}{2.422891in}}%
\pgfpathlineto{\pgfqpoint{3.833349in}{2.425840in}}%
\pgfpathlineto{\pgfqpoint{3.837890in}{2.425840in}}%
\pgfpathlineto{\pgfqpoint{3.837890in}{2.422891in}}%
\pgfpathmoveto{\pgfqpoint{3.846972in}{2.405196in}}%
\pgfpathlineto{\pgfqpoint{3.846972in}{2.405196in}}%
\pgfpathlineto{\pgfqpoint{3.846972in}{2.408145in}}%
\pgfpathlineto{\pgfqpoint{3.851513in}{2.408145in}}%
\pgfpathlineto{\pgfqpoint{3.851513in}{2.405196in}}%
\pgfpathmoveto{\pgfqpoint{3.846972in}{2.408145in}}%
\pgfpathlineto{\pgfqpoint{3.846972in}{2.408145in}}%
\pgfpathlineto{\pgfqpoint{3.846972in}{2.411094in}}%
\pgfpathlineto{\pgfqpoint{3.851513in}{2.411094in}}%
\pgfpathlineto{\pgfqpoint{3.851513in}{2.408145in}}%
\pgfpathmoveto{\pgfqpoint{3.851513in}{2.405196in}}%
\pgfpathlineto{\pgfqpoint{3.851513in}{2.405196in}}%
\pgfpathlineto{\pgfqpoint{3.851513in}{2.408145in}}%
\pgfpathlineto{\pgfqpoint{3.856054in}{2.408145in}}%
\pgfpathlineto{\pgfqpoint{3.856054in}{2.405196in}}%
\pgfpathmoveto{\pgfqpoint{3.851513in}{2.408145in}}%
\pgfpathlineto{\pgfqpoint{3.851513in}{2.408145in}}%
\pgfpathlineto{\pgfqpoint{3.851513in}{2.411094in}}%
\pgfpathlineto{\pgfqpoint{3.856054in}{2.411094in}}%
\pgfpathlineto{\pgfqpoint{3.856054in}{2.408145in}}%
\pgfpathmoveto{\pgfqpoint{3.865136in}{2.393400in}}%
\pgfpathlineto{\pgfqpoint{3.865136in}{2.393400in}}%
\pgfpathlineto{\pgfqpoint{3.865136in}{2.396349in}}%
\pgfpathlineto{\pgfqpoint{3.869677in}{2.396349in}}%
\pgfpathlineto{\pgfqpoint{3.869677in}{2.393400in}}%
\pgfpathmoveto{\pgfqpoint{3.865136in}{2.396349in}}%
\pgfpathlineto{\pgfqpoint{3.865136in}{2.396349in}}%
\pgfpathlineto{\pgfqpoint{3.865136in}{2.399298in}}%
\pgfpathlineto{\pgfqpoint{3.869677in}{2.399298in}}%
\pgfpathlineto{\pgfqpoint{3.869677in}{2.396349in}}%
\pgfpathmoveto{\pgfqpoint{3.869677in}{2.393400in}}%
\pgfpathlineto{\pgfqpoint{3.869677in}{2.393400in}}%
\pgfpathlineto{\pgfqpoint{3.869677in}{2.396349in}}%
\pgfpathlineto{\pgfqpoint{3.874218in}{2.396349in}}%
\pgfpathlineto{\pgfqpoint{3.874218in}{2.393400in}}%
\pgfpathmoveto{\pgfqpoint{3.869677in}{2.396349in}}%
\pgfpathlineto{\pgfqpoint{3.869677in}{2.396349in}}%
\pgfpathlineto{\pgfqpoint{3.869677in}{2.399298in}}%
\pgfpathlineto{\pgfqpoint{3.874218in}{2.399298in}}%
\pgfpathlineto{\pgfqpoint{3.874218in}{2.396349in}}%
\pgfpathmoveto{\pgfqpoint{3.856054in}{2.399298in}}%
\pgfpathlineto{\pgfqpoint{3.856054in}{2.399298in}}%
\pgfpathlineto{\pgfqpoint{3.856054in}{2.402247in}}%
\pgfpathlineto{\pgfqpoint{3.860595in}{2.402247in}}%
\pgfpathlineto{\pgfqpoint{3.860595in}{2.399298in}}%
\pgfpathmoveto{\pgfqpoint{3.856054in}{2.402247in}}%
\pgfpathlineto{\pgfqpoint{3.856054in}{2.402247in}}%
\pgfpathlineto{\pgfqpoint{3.856054in}{2.405196in}}%
\pgfpathlineto{\pgfqpoint{3.860595in}{2.405196in}}%
\pgfpathlineto{\pgfqpoint{3.860595in}{2.402247in}}%
\pgfpathmoveto{\pgfqpoint{3.860595in}{2.399298in}}%
\pgfpathlineto{\pgfqpoint{3.860595in}{2.399298in}}%
\pgfpathlineto{\pgfqpoint{3.860595in}{2.402247in}}%
\pgfpathlineto{\pgfqpoint{3.865136in}{2.402247in}}%
\pgfpathlineto{\pgfqpoint{3.865136in}{2.399298in}}%
\pgfpathmoveto{\pgfqpoint{3.860595in}{2.402247in}}%
\pgfpathlineto{\pgfqpoint{3.860595in}{2.402247in}}%
\pgfpathlineto{\pgfqpoint{3.860595in}{2.405196in}}%
\pgfpathlineto{\pgfqpoint{3.865136in}{2.405196in}}%
\pgfpathlineto{\pgfqpoint{3.865136in}{2.402247in}}%
\pgfpathmoveto{\pgfqpoint{3.856054in}{2.405196in}}%
\pgfpathlineto{\pgfqpoint{3.856054in}{2.405196in}}%
\pgfpathlineto{\pgfqpoint{3.856054in}{2.408145in}}%
\pgfpathlineto{\pgfqpoint{3.860595in}{2.408145in}}%
\pgfpathlineto{\pgfqpoint{3.860595in}{2.405196in}}%
\pgfpathmoveto{\pgfqpoint{3.856054in}{2.408145in}}%
\pgfpathlineto{\pgfqpoint{3.856054in}{2.408145in}}%
\pgfpathlineto{\pgfqpoint{3.856054in}{2.411094in}}%
\pgfpathlineto{\pgfqpoint{3.860595in}{2.411094in}}%
\pgfpathlineto{\pgfqpoint{3.860595in}{2.408145in}}%
\pgfpathmoveto{\pgfqpoint{3.860595in}{2.405196in}}%
\pgfpathlineto{\pgfqpoint{3.860595in}{2.405196in}}%
\pgfpathlineto{\pgfqpoint{3.860595in}{2.408145in}}%
\pgfpathlineto{\pgfqpoint{3.865136in}{2.408145in}}%
\pgfpathlineto{\pgfqpoint{3.865136in}{2.405196in}}%
\pgfpathmoveto{\pgfqpoint{3.865136in}{2.399298in}}%
\pgfpathlineto{\pgfqpoint{3.865136in}{2.399298in}}%
\pgfpathlineto{\pgfqpoint{3.865136in}{2.402247in}}%
\pgfpathlineto{\pgfqpoint{3.869677in}{2.402247in}}%
\pgfpathlineto{\pgfqpoint{3.869677in}{2.399298in}}%
\pgfpathmoveto{\pgfqpoint{3.865136in}{2.402247in}}%
\pgfpathlineto{\pgfqpoint{3.865136in}{2.402247in}}%
\pgfpathlineto{\pgfqpoint{3.865136in}{2.405196in}}%
\pgfpathlineto{\pgfqpoint{3.869677in}{2.405196in}}%
\pgfpathlineto{\pgfqpoint{3.869677in}{2.402247in}}%
\pgfpathmoveto{\pgfqpoint{3.869677in}{2.399298in}}%
\pgfpathlineto{\pgfqpoint{3.869677in}{2.399298in}}%
\pgfpathlineto{\pgfqpoint{3.869677in}{2.402247in}}%
\pgfpathlineto{\pgfqpoint{3.874218in}{2.402247in}}%
\pgfpathlineto{\pgfqpoint{3.874218in}{2.399298in}}%
\pgfpathmoveto{\pgfqpoint{3.837890in}{2.411094in}}%
\pgfpathlineto{\pgfqpoint{3.837890in}{2.411094in}}%
\pgfpathlineto{\pgfqpoint{3.837890in}{2.414044in}}%
\pgfpathlineto{\pgfqpoint{3.842431in}{2.414044in}}%
\pgfpathlineto{\pgfqpoint{3.842431in}{2.411094in}}%
\pgfpathmoveto{\pgfqpoint{3.837890in}{2.414044in}}%
\pgfpathlineto{\pgfqpoint{3.837890in}{2.414044in}}%
\pgfpathlineto{\pgfqpoint{3.837890in}{2.416993in}}%
\pgfpathlineto{\pgfqpoint{3.842431in}{2.416993in}}%
\pgfpathlineto{\pgfqpoint{3.842431in}{2.414044in}}%
\pgfpathmoveto{\pgfqpoint{3.842431in}{2.411094in}}%
\pgfpathlineto{\pgfqpoint{3.842431in}{2.411094in}}%
\pgfpathlineto{\pgfqpoint{3.842431in}{2.414044in}}%
\pgfpathlineto{\pgfqpoint{3.846972in}{2.414044in}}%
\pgfpathlineto{\pgfqpoint{3.846972in}{2.411094in}}%
\pgfpathmoveto{\pgfqpoint{3.842431in}{2.414044in}}%
\pgfpathlineto{\pgfqpoint{3.842431in}{2.414044in}}%
\pgfpathlineto{\pgfqpoint{3.842431in}{2.416993in}}%
\pgfpathlineto{\pgfqpoint{3.846972in}{2.416993in}}%
\pgfpathlineto{\pgfqpoint{3.846972in}{2.414044in}}%
\pgfpathmoveto{\pgfqpoint{3.837890in}{2.416993in}}%
\pgfpathlineto{\pgfqpoint{3.837890in}{2.416993in}}%
\pgfpathlineto{\pgfqpoint{3.837890in}{2.419942in}}%
\pgfpathlineto{\pgfqpoint{3.842431in}{2.419942in}}%
\pgfpathlineto{\pgfqpoint{3.842431in}{2.416993in}}%
\pgfpathmoveto{\pgfqpoint{3.837890in}{2.419942in}}%
\pgfpathlineto{\pgfqpoint{3.837890in}{2.419942in}}%
\pgfpathlineto{\pgfqpoint{3.837890in}{2.422891in}}%
\pgfpathlineto{\pgfqpoint{3.842431in}{2.422891in}}%
\pgfpathlineto{\pgfqpoint{3.842431in}{2.419942in}}%
\pgfpathmoveto{\pgfqpoint{3.842431in}{2.416993in}}%
\pgfpathlineto{\pgfqpoint{3.842431in}{2.416993in}}%
\pgfpathlineto{\pgfqpoint{3.842431in}{2.419942in}}%
\pgfpathlineto{\pgfqpoint{3.846972in}{2.419942in}}%
\pgfpathlineto{\pgfqpoint{3.846972in}{2.416993in}}%
\pgfpathmoveto{\pgfqpoint{3.846972in}{2.411094in}}%
\pgfpathlineto{\pgfqpoint{3.846972in}{2.411094in}}%
\pgfpathlineto{\pgfqpoint{3.846972in}{2.414044in}}%
\pgfpathlineto{\pgfqpoint{3.851513in}{2.414044in}}%
\pgfpathlineto{\pgfqpoint{3.851513in}{2.411094in}}%
\pgfpathmoveto{\pgfqpoint{3.846972in}{2.414044in}}%
\pgfpathlineto{\pgfqpoint{3.846972in}{2.414044in}}%
\pgfpathlineto{\pgfqpoint{3.846972in}{2.416993in}}%
\pgfpathlineto{\pgfqpoint{3.851513in}{2.416993in}}%
\pgfpathlineto{\pgfqpoint{3.851513in}{2.414044in}}%
\pgfpathmoveto{\pgfqpoint{3.851513in}{2.411094in}}%
\pgfpathlineto{\pgfqpoint{3.851513in}{2.411094in}}%
\pgfpathlineto{\pgfqpoint{3.851513in}{2.414044in}}%
\pgfpathlineto{\pgfqpoint{3.856054in}{2.414044in}}%
\pgfpathlineto{\pgfqpoint{3.856054in}{2.411094in}}%
\pgfpathmoveto{\pgfqpoint{3.801561in}{2.434687in}}%
\pgfpathlineto{\pgfqpoint{3.801561in}{2.434687in}}%
\pgfpathlineto{\pgfqpoint{3.801561in}{2.437636in}}%
\pgfpathlineto{\pgfqpoint{3.806102in}{2.437636in}}%
\pgfpathlineto{\pgfqpoint{3.806102in}{2.434687in}}%
\pgfpathmoveto{\pgfqpoint{3.801561in}{2.437636in}}%
\pgfpathlineto{\pgfqpoint{3.801561in}{2.437636in}}%
\pgfpathlineto{\pgfqpoint{3.801561in}{2.440585in}}%
\pgfpathlineto{\pgfqpoint{3.806102in}{2.440585in}}%
\pgfpathlineto{\pgfqpoint{3.806102in}{2.437636in}}%
\pgfpathmoveto{\pgfqpoint{3.806102in}{2.434687in}}%
\pgfpathlineto{\pgfqpoint{3.806102in}{2.434687in}}%
\pgfpathlineto{\pgfqpoint{3.806102in}{2.437636in}}%
\pgfpathlineto{\pgfqpoint{3.810643in}{2.437636in}}%
\pgfpathlineto{\pgfqpoint{3.810643in}{2.434687in}}%
\pgfpathmoveto{\pgfqpoint{3.806102in}{2.437636in}}%
\pgfpathlineto{\pgfqpoint{3.806102in}{2.437636in}}%
\pgfpathlineto{\pgfqpoint{3.806102in}{2.440585in}}%
\pgfpathlineto{\pgfqpoint{3.810643in}{2.440585in}}%
\pgfpathlineto{\pgfqpoint{3.810643in}{2.437636in}}%
\pgfpathmoveto{\pgfqpoint{3.801561in}{2.440585in}}%
\pgfpathlineto{\pgfqpoint{3.801561in}{2.440585in}}%
\pgfpathlineto{\pgfqpoint{3.801561in}{2.443534in}}%
\pgfpathlineto{\pgfqpoint{3.806102in}{2.443534in}}%
\pgfpathlineto{\pgfqpoint{3.806102in}{2.440585in}}%
\pgfpathmoveto{\pgfqpoint{3.801561in}{2.443534in}}%
\pgfpathlineto{\pgfqpoint{3.801561in}{2.443534in}}%
\pgfpathlineto{\pgfqpoint{3.801561in}{2.446483in}}%
\pgfpathlineto{\pgfqpoint{3.806102in}{2.446483in}}%
\pgfpathlineto{\pgfqpoint{3.806102in}{2.443534in}}%
\pgfpathmoveto{\pgfqpoint{3.806102in}{2.440585in}}%
\pgfpathlineto{\pgfqpoint{3.806102in}{2.440585in}}%
\pgfpathlineto{\pgfqpoint{3.806102in}{2.443534in}}%
\pgfpathlineto{\pgfqpoint{3.810643in}{2.443534in}}%
\pgfpathlineto{\pgfqpoint{3.810643in}{2.440585in}}%
\pgfpathmoveto{\pgfqpoint{3.810643in}{2.434687in}}%
\pgfpathlineto{\pgfqpoint{3.810643in}{2.434687in}}%
\pgfpathlineto{\pgfqpoint{3.810643in}{2.437636in}}%
\pgfpathlineto{\pgfqpoint{3.815184in}{2.437636in}}%
\pgfpathlineto{\pgfqpoint{3.815184in}{2.434687in}}%
\pgfpathmoveto{\pgfqpoint{3.810643in}{2.437636in}}%
\pgfpathlineto{\pgfqpoint{3.810643in}{2.437636in}}%
\pgfpathlineto{\pgfqpoint{3.810643in}{2.440585in}}%
\pgfpathlineto{\pgfqpoint{3.815184in}{2.440585in}}%
\pgfpathlineto{\pgfqpoint{3.815184in}{2.437636in}}%
\pgfpathmoveto{\pgfqpoint{3.815184in}{2.434687in}}%
\pgfpathlineto{\pgfqpoint{3.815184in}{2.434687in}}%
\pgfpathlineto{\pgfqpoint{3.815184in}{2.437636in}}%
\pgfpathlineto{\pgfqpoint{3.819725in}{2.437636in}}%
\pgfpathlineto{\pgfqpoint{3.819725in}{2.434687in}}%
\pgfpathmoveto{\pgfqpoint{3.874218in}{2.387502in}}%
\pgfpathlineto{\pgfqpoint{3.874218in}{2.387502in}}%
\pgfpathlineto{\pgfqpoint{3.874218in}{2.390451in}}%
\pgfpathlineto{\pgfqpoint{3.878759in}{2.390451in}}%
\pgfpathlineto{\pgfqpoint{3.878759in}{2.387502in}}%
\pgfpathmoveto{\pgfqpoint{3.874218in}{2.390451in}}%
\pgfpathlineto{\pgfqpoint{3.874218in}{2.390451in}}%
\pgfpathlineto{\pgfqpoint{3.874218in}{2.393400in}}%
\pgfpathlineto{\pgfqpoint{3.878759in}{2.393400in}}%
\pgfpathlineto{\pgfqpoint{3.878759in}{2.390451in}}%
\pgfpathmoveto{\pgfqpoint{3.878759in}{2.387502in}}%
\pgfpathlineto{\pgfqpoint{3.878759in}{2.387502in}}%
\pgfpathlineto{\pgfqpoint{3.878759in}{2.390451in}}%
\pgfpathlineto{\pgfqpoint{3.883300in}{2.390451in}}%
\pgfpathlineto{\pgfqpoint{3.883300in}{2.387502in}}%
\pgfpathmoveto{\pgfqpoint{3.878759in}{2.390451in}}%
\pgfpathlineto{\pgfqpoint{3.878759in}{2.390451in}}%
\pgfpathlineto{\pgfqpoint{3.878759in}{2.393400in}}%
\pgfpathlineto{\pgfqpoint{3.883300in}{2.393400in}}%
\pgfpathlineto{\pgfqpoint{3.883300in}{2.390451in}}%
\pgfpathmoveto{\pgfqpoint{3.874218in}{2.393400in}}%
\pgfpathlineto{\pgfqpoint{3.874218in}{2.393400in}}%
\pgfpathlineto{\pgfqpoint{3.874218in}{2.396349in}}%
\pgfpathlineto{\pgfqpoint{3.878759in}{2.396349in}}%
\pgfpathlineto{\pgfqpoint{3.878759in}{2.393400in}}%
\pgfpathmoveto{\pgfqpoint{3.874218in}{2.396349in}}%
\pgfpathlineto{\pgfqpoint{3.874218in}{2.396349in}}%
\pgfpathlineto{\pgfqpoint{3.874218in}{2.399298in}}%
\pgfpathlineto{\pgfqpoint{3.878759in}{2.399298in}}%
\pgfpathlineto{\pgfqpoint{3.878759in}{2.396349in}}%
\pgfpathmoveto{\pgfqpoint{3.878759in}{2.393400in}}%
\pgfpathlineto{\pgfqpoint{3.878759in}{2.393400in}}%
\pgfpathlineto{\pgfqpoint{3.878759in}{2.396349in}}%
\pgfpathlineto{\pgfqpoint{3.883300in}{2.396349in}}%
\pgfpathlineto{\pgfqpoint{3.883300in}{2.393400in}}%
\pgfpathmoveto{\pgfqpoint{3.883300in}{2.387502in}}%
\pgfpathlineto{\pgfqpoint{3.883300in}{2.387502in}}%
\pgfpathlineto{\pgfqpoint{3.883300in}{2.390451in}}%
\pgfpathlineto{\pgfqpoint{3.887841in}{2.390451in}}%
\pgfpathlineto{\pgfqpoint{3.887841in}{2.387502in}}%
\pgfpathmoveto{\pgfqpoint{3.883300in}{2.390451in}}%
\pgfpathlineto{\pgfqpoint{3.883300in}{2.390451in}}%
\pgfpathlineto{\pgfqpoint{3.883300in}{2.393400in}}%
\pgfpathlineto{\pgfqpoint{3.887841in}{2.393400in}}%
\pgfpathlineto{\pgfqpoint{3.887841in}{2.390451in}}%
\pgfpathmoveto{\pgfqpoint{3.887841in}{2.387502in}}%
\pgfpathlineto{\pgfqpoint{3.887841in}{2.387502in}}%
\pgfpathlineto{\pgfqpoint{3.887841in}{2.390451in}}%
\pgfpathlineto{\pgfqpoint{3.892382in}{2.390451in}}%
\pgfpathlineto{\pgfqpoint{3.892382in}{2.387502in}}%
\pgfpathmoveto{\pgfqpoint{3.946874in}{2.009997in}}%
\pgfpathlineto{\pgfqpoint{3.946874in}{2.009997in}}%
\pgfpathlineto{\pgfqpoint{3.946874in}{2.012947in}}%
\pgfpathlineto{\pgfqpoint{3.951415in}{2.012947in}}%
\pgfpathlineto{\pgfqpoint{3.951415in}{2.009997in}}%
\pgfpathmoveto{\pgfqpoint{3.946874in}{2.012947in}}%
\pgfpathlineto{\pgfqpoint{3.946874in}{2.012947in}}%
\pgfpathlineto{\pgfqpoint{3.946874in}{2.015896in}}%
\pgfpathlineto{\pgfqpoint{3.951415in}{2.015896in}}%
\pgfpathlineto{\pgfqpoint{3.951415in}{2.012947in}}%
\pgfpathmoveto{\pgfqpoint{3.951415in}{2.009997in}}%
\pgfpathlineto{\pgfqpoint{3.951415in}{2.009997in}}%
\pgfpathlineto{\pgfqpoint{3.951415in}{2.012947in}}%
\pgfpathlineto{\pgfqpoint{3.955956in}{2.012947in}}%
\pgfpathlineto{\pgfqpoint{3.955956in}{2.009997in}}%
\pgfpathmoveto{\pgfqpoint{3.951415in}{2.012947in}}%
\pgfpathlineto{\pgfqpoint{3.951415in}{2.012947in}}%
\pgfpathlineto{\pgfqpoint{3.951415in}{2.015896in}}%
\pgfpathlineto{\pgfqpoint{3.955956in}{2.015896in}}%
\pgfpathlineto{\pgfqpoint{3.955956in}{2.012947in}}%
\pgfpathmoveto{\pgfqpoint{3.955956in}{2.009997in}}%
\pgfpathlineto{\pgfqpoint{3.955956in}{2.009997in}}%
\pgfpathlineto{\pgfqpoint{3.955956in}{2.012947in}}%
\pgfpathlineto{\pgfqpoint{3.960497in}{2.012947in}}%
\pgfpathlineto{\pgfqpoint{3.960497in}{2.009997in}}%
\pgfpathmoveto{\pgfqpoint{3.955956in}{2.012947in}}%
\pgfpathlineto{\pgfqpoint{3.955956in}{2.012947in}}%
\pgfpathlineto{\pgfqpoint{3.955956in}{2.015896in}}%
\pgfpathlineto{\pgfqpoint{3.960497in}{2.015896in}}%
\pgfpathlineto{\pgfqpoint{3.960497in}{2.012947in}}%
\pgfpathmoveto{\pgfqpoint{3.960497in}{2.009997in}}%
\pgfpathlineto{\pgfqpoint{3.960497in}{2.009997in}}%
\pgfpathlineto{\pgfqpoint{3.960497in}{2.012947in}}%
\pgfpathlineto{\pgfqpoint{3.965038in}{2.012947in}}%
\pgfpathlineto{\pgfqpoint{3.965038in}{2.009997in}}%
\pgfpathmoveto{\pgfqpoint{3.960497in}{2.012947in}}%
\pgfpathlineto{\pgfqpoint{3.960497in}{2.012947in}}%
\pgfpathlineto{\pgfqpoint{3.960497in}{2.015896in}}%
\pgfpathlineto{\pgfqpoint{3.965038in}{2.015896in}}%
\pgfpathlineto{\pgfqpoint{3.965038in}{2.012947in}}%
\pgfpathmoveto{\pgfqpoint{3.965038in}{2.009997in}}%
\pgfpathlineto{\pgfqpoint{3.965038in}{2.009997in}}%
\pgfpathlineto{\pgfqpoint{3.965038in}{2.012947in}}%
\pgfpathlineto{\pgfqpoint{3.969579in}{2.012947in}}%
\pgfpathlineto{\pgfqpoint{3.969579in}{2.009997in}}%
\pgfpathmoveto{\pgfqpoint{3.965038in}{2.012947in}}%
\pgfpathlineto{\pgfqpoint{3.965038in}{2.012947in}}%
\pgfpathlineto{\pgfqpoint{3.965038in}{2.015896in}}%
\pgfpathlineto{\pgfqpoint{3.969579in}{2.015896in}}%
\pgfpathlineto{\pgfqpoint{3.969579in}{2.012947in}}%
\pgfpathmoveto{\pgfqpoint{3.969579in}{2.009997in}}%
\pgfpathlineto{\pgfqpoint{3.969579in}{2.009997in}}%
\pgfpathlineto{\pgfqpoint{3.969579in}{2.012947in}}%
\pgfpathlineto{\pgfqpoint{3.974120in}{2.012947in}}%
\pgfpathlineto{\pgfqpoint{3.974120in}{2.009997in}}%
\pgfpathmoveto{\pgfqpoint{3.969579in}{2.012947in}}%
\pgfpathlineto{\pgfqpoint{3.969579in}{2.012947in}}%
\pgfpathlineto{\pgfqpoint{3.969579in}{2.015896in}}%
\pgfpathlineto{\pgfqpoint{3.974120in}{2.015896in}}%
\pgfpathlineto{\pgfqpoint{3.974120in}{2.012947in}}%
\pgfpathmoveto{\pgfqpoint{3.974120in}{2.009997in}}%
\pgfpathlineto{\pgfqpoint{3.974120in}{2.009997in}}%
\pgfpathlineto{\pgfqpoint{3.974120in}{2.012947in}}%
\pgfpathlineto{\pgfqpoint{3.978661in}{2.012947in}}%
\pgfpathlineto{\pgfqpoint{3.978661in}{2.009997in}}%
\pgfpathmoveto{\pgfqpoint{3.974120in}{2.012947in}}%
\pgfpathlineto{\pgfqpoint{3.974120in}{2.012947in}}%
\pgfpathlineto{\pgfqpoint{3.974120in}{2.015896in}}%
\pgfpathlineto{\pgfqpoint{3.978661in}{2.015896in}}%
\pgfpathlineto{\pgfqpoint{3.978661in}{2.012947in}}%
\pgfpathmoveto{\pgfqpoint{3.978661in}{2.009997in}}%
\pgfpathlineto{\pgfqpoint{3.978661in}{2.009997in}}%
\pgfpathlineto{\pgfqpoint{3.978661in}{2.012947in}}%
\pgfpathlineto{\pgfqpoint{3.983202in}{2.012947in}}%
\pgfpathlineto{\pgfqpoint{3.983202in}{2.009997in}}%
\pgfpathmoveto{\pgfqpoint{3.978661in}{2.012947in}}%
\pgfpathlineto{\pgfqpoint{3.978661in}{2.012947in}}%
\pgfpathlineto{\pgfqpoint{3.978661in}{2.015896in}}%
\pgfpathlineto{\pgfqpoint{3.983202in}{2.015896in}}%
\pgfpathlineto{\pgfqpoint{3.983202in}{2.012947in}}%
\pgfpathmoveto{\pgfqpoint{3.983202in}{2.009997in}}%
\pgfpathlineto{\pgfqpoint{3.983202in}{2.009997in}}%
\pgfpathlineto{\pgfqpoint{3.983202in}{2.012947in}}%
\pgfpathlineto{\pgfqpoint{3.987743in}{2.012947in}}%
\pgfpathlineto{\pgfqpoint{3.987743in}{2.009997in}}%
\pgfpathmoveto{\pgfqpoint{3.983202in}{2.012947in}}%
\pgfpathlineto{\pgfqpoint{3.983202in}{2.012947in}}%
\pgfpathlineto{\pgfqpoint{3.983202in}{2.015896in}}%
\pgfpathlineto{\pgfqpoint{3.987743in}{2.015896in}}%
\pgfpathlineto{\pgfqpoint{3.987743in}{2.012947in}}%
\pgfpathmoveto{\pgfqpoint{3.987743in}{2.009997in}}%
\pgfpathlineto{\pgfqpoint{3.987743in}{2.009997in}}%
\pgfpathlineto{\pgfqpoint{3.987743in}{2.012947in}}%
\pgfpathlineto{\pgfqpoint{3.992284in}{2.012947in}}%
\pgfpathlineto{\pgfqpoint{3.992284in}{2.009997in}}%
\pgfpathmoveto{\pgfqpoint{3.987743in}{2.012947in}}%
\pgfpathlineto{\pgfqpoint{3.987743in}{2.012947in}}%
\pgfpathlineto{\pgfqpoint{3.987743in}{2.015896in}}%
\pgfpathlineto{\pgfqpoint{3.992284in}{2.015896in}}%
\pgfpathlineto{\pgfqpoint{3.992284in}{2.012947in}}%
\pgfpathmoveto{\pgfqpoint{3.992284in}{2.009997in}}%
\pgfpathlineto{\pgfqpoint{3.992284in}{2.009997in}}%
\pgfpathlineto{\pgfqpoint{3.992284in}{2.012947in}}%
\pgfpathlineto{\pgfqpoint{3.996825in}{2.012947in}}%
\pgfpathlineto{\pgfqpoint{3.996825in}{2.009997in}}%
\pgfpathmoveto{\pgfqpoint{3.992284in}{2.012947in}}%
\pgfpathlineto{\pgfqpoint{3.992284in}{2.012947in}}%
\pgfpathlineto{\pgfqpoint{3.992284in}{2.015896in}}%
\pgfpathlineto{\pgfqpoint{3.996825in}{2.015896in}}%
\pgfpathlineto{\pgfqpoint{3.996825in}{2.012947in}}%
\pgfpathmoveto{\pgfqpoint{3.996825in}{2.009997in}}%
\pgfpathlineto{\pgfqpoint{3.996825in}{2.009997in}}%
\pgfpathlineto{\pgfqpoint{3.996825in}{2.012947in}}%
\pgfpathlineto{\pgfqpoint{4.001365in}{2.012947in}}%
\pgfpathlineto{\pgfqpoint{4.001365in}{2.009997in}}%
\pgfpathmoveto{\pgfqpoint{3.996825in}{2.012947in}}%
\pgfpathlineto{\pgfqpoint{3.996825in}{2.012947in}}%
\pgfpathlineto{\pgfqpoint{3.996825in}{2.015896in}}%
\pgfpathlineto{\pgfqpoint{4.001365in}{2.015896in}}%
\pgfpathlineto{\pgfqpoint{4.001365in}{2.012947in}}%
\pgfpathmoveto{\pgfqpoint{4.001365in}{2.009997in}}%
\pgfpathlineto{\pgfqpoint{4.001365in}{2.009997in}}%
\pgfpathlineto{\pgfqpoint{4.001365in}{2.012947in}}%
\pgfpathlineto{\pgfqpoint{4.005906in}{2.012947in}}%
\pgfpathlineto{\pgfqpoint{4.005906in}{2.009997in}}%
\pgfpathmoveto{\pgfqpoint{4.001365in}{2.012947in}}%
\pgfpathlineto{\pgfqpoint{4.001365in}{2.012947in}}%
\pgfpathlineto{\pgfqpoint{4.001365in}{2.015896in}}%
\pgfpathlineto{\pgfqpoint{4.005906in}{2.015896in}}%
\pgfpathlineto{\pgfqpoint{4.005906in}{2.012947in}}%
\pgfpathmoveto{\pgfqpoint{4.005906in}{2.009997in}}%
\pgfpathlineto{\pgfqpoint{4.005906in}{2.009997in}}%
\pgfpathlineto{\pgfqpoint{4.005906in}{2.012947in}}%
\pgfpathlineto{\pgfqpoint{4.010447in}{2.012947in}}%
\pgfpathlineto{\pgfqpoint{4.010447in}{2.009997in}}%
\pgfpathmoveto{\pgfqpoint{4.005906in}{2.012947in}}%
\pgfpathlineto{\pgfqpoint{4.005906in}{2.012947in}}%
\pgfpathlineto{\pgfqpoint{4.005906in}{2.015896in}}%
\pgfpathlineto{\pgfqpoint{4.010447in}{2.015896in}}%
\pgfpathlineto{\pgfqpoint{4.010447in}{2.012947in}}%
\pgfpathmoveto{\pgfqpoint{4.010447in}{2.009997in}}%
\pgfpathlineto{\pgfqpoint{4.010447in}{2.009997in}}%
\pgfpathlineto{\pgfqpoint{4.010447in}{2.012947in}}%
\pgfpathlineto{\pgfqpoint{4.014988in}{2.012947in}}%
\pgfpathlineto{\pgfqpoint{4.014988in}{2.009997in}}%
\pgfpathmoveto{\pgfqpoint{4.010447in}{2.012947in}}%
\pgfpathlineto{\pgfqpoint{4.010447in}{2.012947in}}%
\pgfpathlineto{\pgfqpoint{4.010447in}{2.015896in}}%
\pgfpathlineto{\pgfqpoint{4.014988in}{2.015896in}}%
\pgfpathlineto{\pgfqpoint{4.014988in}{2.012947in}}%
\pgfpathmoveto{\pgfqpoint{4.014988in}{2.009997in}}%
\pgfpathlineto{\pgfqpoint{4.014988in}{2.009997in}}%
\pgfpathlineto{\pgfqpoint{4.014988in}{2.012947in}}%
\pgfpathlineto{\pgfqpoint{4.019529in}{2.012947in}}%
\pgfpathlineto{\pgfqpoint{4.019529in}{2.009997in}}%
\pgfpathmoveto{\pgfqpoint{4.014988in}{2.012947in}}%
\pgfpathlineto{\pgfqpoint{4.014988in}{2.012947in}}%
\pgfpathlineto{\pgfqpoint{4.014988in}{2.015896in}}%
\pgfpathlineto{\pgfqpoint{4.019529in}{2.015896in}}%
\pgfpathlineto{\pgfqpoint{4.019529in}{2.012947in}}%
\pgfpathmoveto{\pgfqpoint{4.019529in}{2.009997in}}%
\pgfpathlineto{\pgfqpoint{4.019529in}{2.009997in}}%
\pgfpathlineto{\pgfqpoint{4.019529in}{2.012947in}}%
\pgfpathlineto{\pgfqpoint{4.024070in}{2.012947in}}%
\pgfpathlineto{\pgfqpoint{4.024070in}{2.009997in}}%
\pgfpathmoveto{\pgfqpoint{4.019529in}{2.012947in}}%
\pgfpathlineto{\pgfqpoint{4.019529in}{2.012947in}}%
\pgfpathlineto{\pgfqpoint{4.019529in}{2.015896in}}%
\pgfpathlineto{\pgfqpoint{4.024070in}{2.015896in}}%
\pgfpathlineto{\pgfqpoint{4.024070in}{2.012947in}}%
\pgfpathmoveto{\pgfqpoint{4.024070in}{2.009997in}}%
\pgfpathlineto{\pgfqpoint{4.024070in}{2.009997in}}%
\pgfpathlineto{\pgfqpoint{4.024070in}{2.012947in}}%
\pgfpathlineto{\pgfqpoint{4.028611in}{2.012947in}}%
\pgfpathlineto{\pgfqpoint{4.028611in}{2.009997in}}%
\pgfpathmoveto{\pgfqpoint{4.024070in}{2.012947in}}%
\pgfpathlineto{\pgfqpoint{4.024070in}{2.012947in}}%
\pgfpathlineto{\pgfqpoint{4.024070in}{2.015896in}}%
\pgfpathlineto{\pgfqpoint{4.028611in}{2.015896in}}%
\pgfpathlineto{\pgfqpoint{4.028611in}{2.012947in}}%
\pgfpathmoveto{\pgfqpoint{4.028611in}{2.009997in}}%
\pgfpathlineto{\pgfqpoint{4.028611in}{2.009997in}}%
\pgfpathlineto{\pgfqpoint{4.028611in}{2.012947in}}%
\pgfpathlineto{\pgfqpoint{4.033152in}{2.012947in}}%
\pgfpathlineto{\pgfqpoint{4.033152in}{2.009997in}}%
\pgfpathmoveto{\pgfqpoint{4.028611in}{2.012947in}}%
\pgfpathlineto{\pgfqpoint{4.028611in}{2.012947in}}%
\pgfpathlineto{\pgfqpoint{4.028611in}{2.015896in}}%
\pgfpathlineto{\pgfqpoint{4.033152in}{2.015896in}}%
\pgfpathlineto{\pgfqpoint{4.033152in}{2.012947in}}%
\pgfpathmoveto{\pgfqpoint{4.033152in}{2.009997in}}%
\pgfpathlineto{\pgfqpoint{4.033152in}{2.009997in}}%
\pgfpathlineto{\pgfqpoint{4.033152in}{2.012947in}}%
\pgfpathlineto{\pgfqpoint{4.037693in}{2.012947in}}%
\pgfpathlineto{\pgfqpoint{4.037693in}{2.009997in}}%
\pgfpathmoveto{\pgfqpoint{4.033152in}{2.012947in}}%
\pgfpathlineto{\pgfqpoint{4.033152in}{2.012947in}}%
\pgfpathlineto{\pgfqpoint{4.033152in}{2.015896in}}%
\pgfpathlineto{\pgfqpoint{4.037693in}{2.015896in}}%
\pgfpathlineto{\pgfqpoint{4.037693in}{2.012947in}}%
\pgfpathmoveto{\pgfqpoint{4.037693in}{2.009997in}}%
\pgfpathlineto{\pgfqpoint{4.037693in}{2.009997in}}%
\pgfpathlineto{\pgfqpoint{4.037693in}{2.012947in}}%
\pgfpathlineto{\pgfqpoint{4.042234in}{2.012947in}}%
\pgfpathlineto{\pgfqpoint{4.042234in}{2.009997in}}%
\pgfpathmoveto{\pgfqpoint{4.037693in}{2.012947in}}%
\pgfpathlineto{\pgfqpoint{4.037693in}{2.012947in}}%
\pgfpathlineto{\pgfqpoint{4.037693in}{2.015896in}}%
\pgfpathlineto{\pgfqpoint{4.042234in}{2.015896in}}%
\pgfpathlineto{\pgfqpoint{4.042234in}{2.012947in}}%
\pgfpathmoveto{\pgfqpoint{4.042234in}{2.009997in}}%
\pgfpathlineto{\pgfqpoint{4.042234in}{2.009997in}}%
\pgfpathlineto{\pgfqpoint{4.042234in}{2.012947in}}%
\pgfpathlineto{\pgfqpoint{4.046775in}{2.012947in}}%
\pgfpathlineto{\pgfqpoint{4.046775in}{2.009997in}}%
\pgfpathmoveto{\pgfqpoint{4.042234in}{2.012947in}}%
\pgfpathlineto{\pgfqpoint{4.042234in}{2.012947in}}%
\pgfpathlineto{\pgfqpoint{4.042234in}{2.015896in}}%
\pgfpathlineto{\pgfqpoint{4.046775in}{2.015896in}}%
\pgfpathlineto{\pgfqpoint{4.046775in}{2.012947in}}%
\pgfpathmoveto{\pgfqpoint{4.046775in}{2.009997in}}%
\pgfpathlineto{\pgfqpoint{4.046775in}{2.009997in}}%
\pgfpathlineto{\pgfqpoint{4.046775in}{2.012947in}}%
\pgfpathlineto{\pgfqpoint{4.051316in}{2.012947in}}%
\pgfpathlineto{\pgfqpoint{4.051316in}{2.009997in}}%
\pgfpathmoveto{\pgfqpoint{4.046775in}{2.012947in}}%
\pgfpathlineto{\pgfqpoint{4.046775in}{2.012947in}}%
\pgfpathlineto{\pgfqpoint{4.046775in}{2.015896in}}%
\pgfpathlineto{\pgfqpoint{4.051316in}{2.015896in}}%
\pgfpathlineto{\pgfqpoint{4.051316in}{2.012947in}}%
\pgfpathmoveto{\pgfqpoint{4.051316in}{2.009997in}}%
\pgfpathlineto{\pgfqpoint{4.051316in}{2.009997in}}%
\pgfpathlineto{\pgfqpoint{4.051316in}{2.012947in}}%
\pgfpathlineto{\pgfqpoint{4.055857in}{2.012947in}}%
\pgfpathlineto{\pgfqpoint{4.055857in}{2.009997in}}%
\pgfpathmoveto{\pgfqpoint{4.051316in}{2.012947in}}%
\pgfpathlineto{\pgfqpoint{4.051316in}{2.012947in}}%
\pgfpathlineto{\pgfqpoint{4.051316in}{2.015896in}}%
\pgfpathlineto{\pgfqpoint{4.055857in}{2.015896in}}%
\pgfpathlineto{\pgfqpoint{4.055857in}{2.012947in}}%
\pgfpathmoveto{\pgfqpoint{4.055857in}{2.009997in}}%
\pgfpathlineto{\pgfqpoint{4.055857in}{2.009997in}}%
\pgfpathlineto{\pgfqpoint{4.055857in}{2.012947in}}%
\pgfpathlineto{\pgfqpoint{4.060398in}{2.012947in}}%
\pgfpathlineto{\pgfqpoint{4.060398in}{2.009997in}}%
\pgfpathmoveto{\pgfqpoint{4.055857in}{2.012947in}}%
\pgfpathlineto{\pgfqpoint{4.055857in}{2.012947in}}%
\pgfpathlineto{\pgfqpoint{4.055857in}{2.015896in}}%
\pgfpathlineto{\pgfqpoint{4.060398in}{2.015896in}}%
\pgfpathlineto{\pgfqpoint{4.060398in}{2.012947in}}%
\pgfpathmoveto{\pgfqpoint{4.060398in}{2.009997in}}%
\pgfpathlineto{\pgfqpoint{4.060398in}{2.009997in}}%
\pgfpathlineto{\pgfqpoint{4.060398in}{2.012947in}}%
\pgfpathlineto{\pgfqpoint{4.064939in}{2.012947in}}%
\pgfpathlineto{\pgfqpoint{4.064939in}{2.009997in}}%
\pgfpathmoveto{\pgfqpoint{4.060398in}{2.012947in}}%
\pgfpathlineto{\pgfqpoint{4.060398in}{2.012947in}}%
\pgfpathlineto{\pgfqpoint{4.060398in}{2.015896in}}%
\pgfpathlineto{\pgfqpoint{4.064939in}{2.015896in}}%
\pgfpathlineto{\pgfqpoint{4.064939in}{2.012947in}}%
\pgfpathmoveto{\pgfqpoint{4.064939in}{2.009997in}}%
\pgfpathlineto{\pgfqpoint{4.064939in}{2.009997in}}%
\pgfpathlineto{\pgfqpoint{4.064939in}{2.012947in}}%
\pgfpathlineto{\pgfqpoint{4.069480in}{2.012947in}}%
\pgfpathlineto{\pgfqpoint{4.069480in}{2.009997in}}%
\pgfpathmoveto{\pgfqpoint{4.064939in}{2.012947in}}%
\pgfpathlineto{\pgfqpoint{4.064939in}{2.012947in}}%
\pgfpathlineto{\pgfqpoint{4.064939in}{2.015896in}}%
\pgfpathlineto{\pgfqpoint{4.069480in}{2.015896in}}%
\pgfpathlineto{\pgfqpoint{4.069480in}{2.012947in}}%
\pgfpathmoveto{\pgfqpoint{4.069480in}{2.009997in}}%
\pgfpathlineto{\pgfqpoint{4.069480in}{2.009997in}}%
\pgfpathlineto{\pgfqpoint{4.069480in}{2.012947in}}%
\pgfpathlineto{\pgfqpoint{4.074021in}{2.012947in}}%
\pgfpathlineto{\pgfqpoint{4.074021in}{2.009997in}}%
\pgfpathmoveto{\pgfqpoint{4.069480in}{2.012947in}}%
\pgfpathlineto{\pgfqpoint{4.069480in}{2.012947in}}%
\pgfpathlineto{\pgfqpoint{4.069480in}{2.015896in}}%
\pgfpathlineto{\pgfqpoint{4.074021in}{2.015896in}}%
\pgfpathlineto{\pgfqpoint{4.074021in}{2.012947in}}%
\pgfpathmoveto{\pgfqpoint{4.074021in}{2.009997in}}%
\pgfpathlineto{\pgfqpoint{4.074021in}{2.009997in}}%
\pgfpathlineto{\pgfqpoint{4.074021in}{2.012947in}}%
\pgfpathlineto{\pgfqpoint{4.078562in}{2.012947in}}%
\pgfpathlineto{\pgfqpoint{4.078562in}{2.009997in}}%
\pgfpathmoveto{\pgfqpoint{4.074021in}{2.012947in}}%
\pgfpathlineto{\pgfqpoint{4.074021in}{2.012947in}}%
\pgfpathlineto{\pgfqpoint{4.074021in}{2.015896in}}%
\pgfpathlineto{\pgfqpoint{4.078562in}{2.015896in}}%
\pgfpathlineto{\pgfqpoint{4.078562in}{2.012947in}}%
\pgfpathmoveto{\pgfqpoint{4.078562in}{2.009997in}}%
\pgfpathlineto{\pgfqpoint{4.078562in}{2.009997in}}%
\pgfpathlineto{\pgfqpoint{4.078562in}{2.012947in}}%
\pgfpathlineto{\pgfqpoint{4.083103in}{2.012947in}}%
\pgfpathlineto{\pgfqpoint{4.083103in}{2.009997in}}%
\pgfpathmoveto{\pgfqpoint{4.078562in}{2.012947in}}%
\pgfpathlineto{\pgfqpoint{4.078562in}{2.012947in}}%
\pgfpathlineto{\pgfqpoint{4.078562in}{2.015896in}}%
\pgfpathlineto{\pgfqpoint{4.083103in}{2.015896in}}%
\pgfpathlineto{\pgfqpoint{4.083103in}{2.012947in}}%
\pgfpathmoveto{\pgfqpoint{4.083103in}{2.009997in}}%
\pgfpathlineto{\pgfqpoint{4.083103in}{2.009997in}}%
\pgfpathlineto{\pgfqpoint{4.083103in}{2.012947in}}%
\pgfpathlineto{\pgfqpoint{4.087644in}{2.012947in}}%
\pgfpathlineto{\pgfqpoint{4.087644in}{2.009997in}}%
\pgfpathmoveto{\pgfqpoint{4.083103in}{2.012947in}}%
\pgfpathlineto{\pgfqpoint{4.083103in}{2.012947in}}%
\pgfpathlineto{\pgfqpoint{4.083103in}{2.015896in}}%
\pgfpathlineto{\pgfqpoint{4.087644in}{2.015896in}}%
\pgfpathlineto{\pgfqpoint{4.087644in}{2.012947in}}%
\pgfpathmoveto{\pgfqpoint{4.087644in}{2.009997in}}%
\pgfpathlineto{\pgfqpoint{4.087644in}{2.009997in}}%
\pgfpathlineto{\pgfqpoint{4.087644in}{2.012947in}}%
\pgfpathlineto{\pgfqpoint{4.092185in}{2.012947in}}%
\pgfpathlineto{\pgfqpoint{4.092185in}{2.009997in}}%
\pgfpathmoveto{\pgfqpoint{4.087644in}{2.012947in}}%
\pgfpathlineto{\pgfqpoint{4.087644in}{2.012947in}}%
\pgfpathlineto{\pgfqpoint{4.087644in}{2.015896in}}%
\pgfpathlineto{\pgfqpoint{4.092185in}{2.015896in}}%
\pgfpathlineto{\pgfqpoint{4.092185in}{2.012947in}}%
\pgfpathmoveto{\pgfqpoint{4.028611in}{2.287226in}}%
\pgfpathlineto{\pgfqpoint{4.028611in}{2.287226in}}%
\pgfpathlineto{\pgfqpoint{4.028611in}{2.290175in}}%
\pgfpathlineto{\pgfqpoint{4.033152in}{2.290175in}}%
\pgfpathlineto{\pgfqpoint{4.033152in}{2.287226in}}%
\pgfpathmoveto{\pgfqpoint{4.028611in}{2.290175in}}%
\pgfpathlineto{\pgfqpoint{4.028611in}{2.290175in}}%
\pgfpathlineto{\pgfqpoint{4.028611in}{2.293125in}}%
\pgfpathlineto{\pgfqpoint{4.033152in}{2.293125in}}%
\pgfpathlineto{\pgfqpoint{4.033152in}{2.290175in}}%
\pgfpathmoveto{\pgfqpoint{4.033152in}{2.287226in}}%
\pgfpathlineto{\pgfqpoint{4.033152in}{2.287226in}}%
\pgfpathlineto{\pgfqpoint{4.033152in}{2.290175in}}%
\pgfpathlineto{\pgfqpoint{4.037693in}{2.290175in}}%
\pgfpathlineto{\pgfqpoint{4.037693in}{2.287226in}}%
\pgfpathmoveto{\pgfqpoint{4.033152in}{2.290175in}}%
\pgfpathlineto{\pgfqpoint{4.033152in}{2.290175in}}%
\pgfpathlineto{\pgfqpoint{4.033152in}{2.293125in}}%
\pgfpathlineto{\pgfqpoint{4.037693in}{2.293125in}}%
\pgfpathlineto{\pgfqpoint{4.037693in}{2.290175in}}%
\pgfpathmoveto{\pgfqpoint{4.046775in}{2.275429in}}%
\pgfpathlineto{\pgfqpoint{4.046775in}{2.275429in}}%
\pgfpathlineto{\pgfqpoint{4.046775in}{2.278378in}}%
\pgfpathlineto{\pgfqpoint{4.051316in}{2.278378in}}%
\pgfpathlineto{\pgfqpoint{4.051316in}{2.275429in}}%
\pgfpathmoveto{\pgfqpoint{4.046775in}{2.278378in}}%
\pgfpathlineto{\pgfqpoint{4.046775in}{2.278378in}}%
\pgfpathlineto{\pgfqpoint{4.046775in}{2.281328in}}%
\pgfpathlineto{\pgfqpoint{4.051316in}{2.281328in}}%
\pgfpathlineto{\pgfqpoint{4.051316in}{2.278378in}}%
\pgfpathmoveto{\pgfqpoint{4.051316in}{2.275429in}}%
\pgfpathlineto{\pgfqpoint{4.051316in}{2.275429in}}%
\pgfpathlineto{\pgfqpoint{4.051316in}{2.278378in}}%
\pgfpathlineto{\pgfqpoint{4.055857in}{2.278378in}}%
\pgfpathlineto{\pgfqpoint{4.055857in}{2.275429in}}%
\pgfpathmoveto{\pgfqpoint{4.051316in}{2.278378in}}%
\pgfpathlineto{\pgfqpoint{4.051316in}{2.278378in}}%
\pgfpathlineto{\pgfqpoint{4.051316in}{2.281328in}}%
\pgfpathlineto{\pgfqpoint{4.055857in}{2.281328in}}%
\pgfpathlineto{\pgfqpoint{4.055857in}{2.278378in}}%
\pgfpathmoveto{\pgfqpoint{4.037693in}{2.281328in}}%
\pgfpathlineto{\pgfqpoint{4.037693in}{2.281328in}}%
\pgfpathlineto{\pgfqpoint{4.037693in}{2.284277in}}%
\pgfpathlineto{\pgfqpoint{4.042234in}{2.284277in}}%
\pgfpathlineto{\pgfqpoint{4.042234in}{2.281328in}}%
\pgfpathmoveto{\pgfqpoint{4.037693in}{2.284277in}}%
\pgfpathlineto{\pgfqpoint{4.037693in}{2.284277in}}%
\pgfpathlineto{\pgfqpoint{4.037693in}{2.287226in}}%
\pgfpathlineto{\pgfqpoint{4.042234in}{2.287226in}}%
\pgfpathlineto{\pgfqpoint{4.042234in}{2.284277in}}%
\pgfpathmoveto{\pgfqpoint{4.042234in}{2.281328in}}%
\pgfpathlineto{\pgfqpoint{4.042234in}{2.281328in}}%
\pgfpathlineto{\pgfqpoint{4.042234in}{2.284277in}}%
\pgfpathlineto{\pgfqpoint{4.046775in}{2.284277in}}%
\pgfpathlineto{\pgfqpoint{4.046775in}{2.281328in}}%
\pgfpathmoveto{\pgfqpoint{4.042234in}{2.284277in}}%
\pgfpathlineto{\pgfqpoint{4.042234in}{2.284277in}}%
\pgfpathlineto{\pgfqpoint{4.042234in}{2.287226in}}%
\pgfpathlineto{\pgfqpoint{4.046775in}{2.287226in}}%
\pgfpathlineto{\pgfqpoint{4.046775in}{2.284277in}}%
\pgfpathmoveto{\pgfqpoint{4.037693in}{2.287226in}}%
\pgfpathlineto{\pgfqpoint{4.037693in}{2.287226in}}%
\pgfpathlineto{\pgfqpoint{4.037693in}{2.290175in}}%
\pgfpathlineto{\pgfqpoint{4.042234in}{2.290175in}}%
\pgfpathlineto{\pgfqpoint{4.042234in}{2.287226in}}%
\pgfpathmoveto{\pgfqpoint{4.037693in}{2.290175in}}%
\pgfpathlineto{\pgfqpoint{4.037693in}{2.290175in}}%
\pgfpathlineto{\pgfqpoint{4.037693in}{2.293125in}}%
\pgfpathlineto{\pgfqpoint{4.042234in}{2.293125in}}%
\pgfpathlineto{\pgfqpoint{4.042234in}{2.290175in}}%
\pgfpathmoveto{\pgfqpoint{4.042234in}{2.287226in}}%
\pgfpathlineto{\pgfqpoint{4.042234in}{2.287226in}}%
\pgfpathlineto{\pgfqpoint{4.042234in}{2.290175in}}%
\pgfpathlineto{\pgfqpoint{4.046775in}{2.290175in}}%
\pgfpathlineto{\pgfqpoint{4.046775in}{2.287226in}}%
\pgfpathmoveto{\pgfqpoint{4.046775in}{2.281328in}}%
\pgfpathlineto{\pgfqpoint{4.046775in}{2.281328in}}%
\pgfpathlineto{\pgfqpoint{4.046775in}{2.284277in}}%
\pgfpathlineto{\pgfqpoint{4.051316in}{2.284277in}}%
\pgfpathlineto{\pgfqpoint{4.051316in}{2.281328in}}%
\pgfpathmoveto{\pgfqpoint{4.046775in}{2.284277in}}%
\pgfpathlineto{\pgfqpoint{4.046775in}{2.284277in}}%
\pgfpathlineto{\pgfqpoint{4.046775in}{2.287226in}}%
\pgfpathlineto{\pgfqpoint{4.051316in}{2.287226in}}%
\pgfpathlineto{\pgfqpoint{4.051316in}{2.284277in}}%
\pgfpathmoveto{\pgfqpoint{4.051316in}{2.281328in}}%
\pgfpathlineto{\pgfqpoint{4.051316in}{2.281328in}}%
\pgfpathlineto{\pgfqpoint{4.051316in}{2.284277in}}%
\pgfpathlineto{\pgfqpoint{4.055857in}{2.284277in}}%
\pgfpathlineto{\pgfqpoint{4.055857in}{2.281328in}}%
\pgfpathmoveto{\pgfqpoint{4.064939in}{2.263632in}}%
\pgfpathlineto{\pgfqpoint{4.064939in}{2.263632in}}%
\pgfpathlineto{\pgfqpoint{4.064939in}{2.266581in}}%
\pgfpathlineto{\pgfqpoint{4.069480in}{2.266581in}}%
\pgfpathlineto{\pgfqpoint{4.069480in}{2.263632in}}%
\pgfpathmoveto{\pgfqpoint{4.064939in}{2.266581in}}%
\pgfpathlineto{\pgfqpoint{4.064939in}{2.266581in}}%
\pgfpathlineto{\pgfqpoint{4.064939in}{2.269531in}}%
\pgfpathlineto{\pgfqpoint{4.069480in}{2.269531in}}%
\pgfpathlineto{\pgfqpoint{4.069480in}{2.266581in}}%
\pgfpathmoveto{\pgfqpoint{4.069480in}{2.263632in}}%
\pgfpathlineto{\pgfqpoint{4.069480in}{2.263632in}}%
\pgfpathlineto{\pgfqpoint{4.069480in}{2.266581in}}%
\pgfpathlineto{\pgfqpoint{4.074021in}{2.266581in}}%
\pgfpathlineto{\pgfqpoint{4.074021in}{2.263632in}}%
\pgfpathmoveto{\pgfqpoint{4.069480in}{2.266581in}}%
\pgfpathlineto{\pgfqpoint{4.069480in}{2.266581in}}%
\pgfpathlineto{\pgfqpoint{4.069480in}{2.269531in}}%
\pgfpathlineto{\pgfqpoint{4.074021in}{2.269531in}}%
\pgfpathlineto{\pgfqpoint{4.074021in}{2.266581in}}%
\pgfpathmoveto{\pgfqpoint{4.083103in}{2.251835in}}%
\pgfpathlineto{\pgfqpoint{4.083103in}{2.251835in}}%
\pgfpathlineto{\pgfqpoint{4.083103in}{2.254784in}}%
\pgfpathlineto{\pgfqpoint{4.087644in}{2.254784in}}%
\pgfpathlineto{\pgfqpoint{4.087644in}{2.251835in}}%
\pgfpathmoveto{\pgfqpoint{4.083103in}{2.254784in}}%
\pgfpathlineto{\pgfqpoint{4.083103in}{2.254784in}}%
\pgfpathlineto{\pgfqpoint{4.083103in}{2.257733in}}%
\pgfpathlineto{\pgfqpoint{4.087644in}{2.257733in}}%
\pgfpathlineto{\pgfqpoint{4.087644in}{2.254784in}}%
\pgfpathmoveto{\pgfqpoint{4.087644in}{2.251835in}}%
\pgfpathlineto{\pgfqpoint{4.087644in}{2.251835in}}%
\pgfpathlineto{\pgfqpoint{4.087644in}{2.254784in}}%
\pgfpathlineto{\pgfqpoint{4.092185in}{2.254784in}}%
\pgfpathlineto{\pgfqpoint{4.092185in}{2.251835in}}%
\pgfpathmoveto{\pgfqpoint{4.087644in}{2.254784in}}%
\pgfpathlineto{\pgfqpoint{4.087644in}{2.254784in}}%
\pgfpathlineto{\pgfqpoint{4.087644in}{2.257733in}}%
\pgfpathlineto{\pgfqpoint{4.092185in}{2.257733in}}%
\pgfpathlineto{\pgfqpoint{4.092185in}{2.254784in}}%
\pgfpathmoveto{\pgfqpoint{4.074021in}{2.257733in}}%
\pgfpathlineto{\pgfqpoint{4.074021in}{2.257733in}}%
\pgfpathlineto{\pgfqpoint{4.074021in}{2.260683in}}%
\pgfpathlineto{\pgfqpoint{4.078562in}{2.260683in}}%
\pgfpathlineto{\pgfqpoint{4.078562in}{2.257733in}}%
\pgfpathmoveto{\pgfqpoint{4.074021in}{2.260683in}}%
\pgfpathlineto{\pgfqpoint{4.074021in}{2.260683in}}%
\pgfpathlineto{\pgfqpoint{4.074021in}{2.263632in}}%
\pgfpathlineto{\pgfqpoint{4.078562in}{2.263632in}}%
\pgfpathlineto{\pgfqpoint{4.078562in}{2.260683in}}%
\pgfpathmoveto{\pgfqpoint{4.078562in}{2.257733in}}%
\pgfpathlineto{\pgfqpoint{4.078562in}{2.257733in}}%
\pgfpathlineto{\pgfqpoint{4.078562in}{2.260683in}}%
\pgfpathlineto{\pgfqpoint{4.083103in}{2.260683in}}%
\pgfpathlineto{\pgfqpoint{4.083103in}{2.257733in}}%
\pgfpathmoveto{\pgfqpoint{4.078562in}{2.260683in}}%
\pgfpathlineto{\pgfqpoint{4.078562in}{2.260683in}}%
\pgfpathlineto{\pgfqpoint{4.078562in}{2.263632in}}%
\pgfpathlineto{\pgfqpoint{4.083103in}{2.263632in}}%
\pgfpathlineto{\pgfqpoint{4.083103in}{2.260683in}}%
\pgfpathmoveto{\pgfqpoint{4.074021in}{2.263632in}}%
\pgfpathlineto{\pgfqpoint{4.074021in}{2.263632in}}%
\pgfpathlineto{\pgfqpoint{4.074021in}{2.266581in}}%
\pgfpathlineto{\pgfqpoint{4.078562in}{2.266581in}}%
\pgfpathlineto{\pgfqpoint{4.078562in}{2.263632in}}%
\pgfpathmoveto{\pgfqpoint{4.074021in}{2.266581in}}%
\pgfpathlineto{\pgfqpoint{4.074021in}{2.266581in}}%
\pgfpathlineto{\pgfqpoint{4.074021in}{2.269531in}}%
\pgfpathlineto{\pgfqpoint{4.078562in}{2.269531in}}%
\pgfpathlineto{\pgfqpoint{4.078562in}{2.266581in}}%
\pgfpathmoveto{\pgfqpoint{4.078562in}{2.263632in}}%
\pgfpathlineto{\pgfqpoint{4.078562in}{2.263632in}}%
\pgfpathlineto{\pgfqpoint{4.078562in}{2.266581in}}%
\pgfpathlineto{\pgfqpoint{4.083103in}{2.266581in}}%
\pgfpathlineto{\pgfqpoint{4.083103in}{2.263632in}}%
\pgfpathmoveto{\pgfqpoint{4.083103in}{2.257733in}}%
\pgfpathlineto{\pgfqpoint{4.083103in}{2.257733in}}%
\pgfpathlineto{\pgfqpoint{4.083103in}{2.260683in}}%
\pgfpathlineto{\pgfqpoint{4.087644in}{2.260683in}}%
\pgfpathlineto{\pgfqpoint{4.087644in}{2.257733in}}%
\pgfpathmoveto{\pgfqpoint{4.083103in}{2.260683in}}%
\pgfpathlineto{\pgfqpoint{4.083103in}{2.260683in}}%
\pgfpathlineto{\pgfqpoint{4.083103in}{2.263632in}}%
\pgfpathlineto{\pgfqpoint{4.087644in}{2.263632in}}%
\pgfpathlineto{\pgfqpoint{4.087644in}{2.260683in}}%
\pgfpathmoveto{\pgfqpoint{4.087644in}{2.257733in}}%
\pgfpathlineto{\pgfqpoint{4.087644in}{2.257733in}}%
\pgfpathlineto{\pgfqpoint{4.087644in}{2.260683in}}%
\pgfpathlineto{\pgfqpoint{4.092185in}{2.260683in}}%
\pgfpathlineto{\pgfqpoint{4.092185in}{2.257733in}}%
\pgfpathmoveto{\pgfqpoint{4.055857in}{2.269531in}}%
\pgfpathlineto{\pgfqpoint{4.055857in}{2.269531in}}%
\pgfpathlineto{\pgfqpoint{4.055857in}{2.272480in}}%
\pgfpathlineto{\pgfqpoint{4.060398in}{2.272480in}}%
\pgfpathlineto{\pgfqpoint{4.060398in}{2.269531in}}%
\pgfpathmoveto{\pgfqpoint{4.055857in}{2.272480in}}%
\pgfpathlineto{\pgfqpoint{4.055857in}{2.272480in}}%
\pgfpathlineto{\pgfqpoint{4.055857in}{2.275429in}}%
\pgfpathlineto{\pgfqpoint{4.060398in}{2.275429in}}%
\pgfpathlineto{\pgfqpoint{4.060398in}{2.272480in}}%
\pgfpathmoveto{\pgfqpoint{4.060398in}{2.269531in}}%
\pgfpathlineto{\pgfqpoint{4.060398in}{2.269531in}}%
\pgfpathlineto{\pgfqpoint{4.060398in}{2.272480in}}%
\pgfpathlineto{\pgfqpoint{4.064939in}{2.272480in}}%
\pgfpathlineto{\pgfqpoint{4.064939in}{2.269531in}}%
\pgfpathmoveto{\pgfqpoint{4.060398in}{2.272480in}}%
\pgfpathlineto{\pgfqpoint{4.060398in}{2.272480in}}%
\pgfpathlineto{\pgfqpoint{4.060398in}{2.275429in}}%
\pgfpathlineto{\pgfqpoint{4.064939in}{2.275429in}}%
\pgfpathlineto{\pgfqpoint{4.064939in}{2.272480in}}%
\pgfpathmoveto{\pgfqpoint{4.055857in}{2.275429in}}%
\pgfpathlineto{\pgfqpoint{4.055857in}{2.275429in}}%
\pgfpathlineto{\pgfqpoint{4.055857in}{2.278378in}}%
\pgfpathlineto{\pgfqpoint{4.060398in}{2.278378in}}%
\pgfpathlineto{\pgfqpoint{4.060398in}{2.275429in}}%
\pgfpathmoveto{\pgfqpoint{4.055857in}{2.278378in}}%
\pgfpathlineto{\pgfqpoint{4.055857in}{2.278378in}}%
\pgfpathlineto{\pgfqpoint{4.055857in}{2.281328in}}%
\pgfpathlineto{\pgfqpoint{4.060398in}{2.281328in}}%
\pgfpathlineto{\pgfqpoint{4.060398in}{2.278378in}}%
\pgfpathmoveto{\pgfqpoint{4.060398in}{2.275429in}}%
\pgfpathlineto{\pgfqpoint{4.060398in}{2.275429in}}%
\pgfpathlineto{\pgfqpoint{4.060398in}{2.278378in}}%
\pgfpathlineto{\pgfqpoint{4.064939in}{2.278378in}}%
\pgfpathlineto{\pgfqpoint{4.064939in}{2.275429in}}%
\pgfpathmoveto{\pgfqpoint{4.064939in}{2.269531in}}%
\pgfpathlineto{\pgfqpoint{4.064939in}{2.269531in}}%
\pgfpathlineto{\pgfqpoint{4.064939in}{2.272480in}}%
\pgfpathlineto{\pgfqpoint{4.069480in}{2.272480in}}%
\pgfpathlineto{\pgfqpoint{4.069480in}{2.269531in}}%
\pgfpathmoveto{\pgfqpoint{4.064939in}{2.272480in}}%
\pgfpathlineto{\pgfqpoint{4.064939in}{2.272480in}}%
\pgfpathlineto{\pgfqpoint{4.064939in}{2.275429in}}%
\pgfpathlineto{\pgfqpoint{4.069480in}{2.275429in}}%
\pgfpathlineto{\pgfqpoint{4.069480in}{2.272480in}}%
\pgfpathmoveto{\pgfqpoint{4.069480in}{2.269531in}}%
\pgfpathlineto{\pgfqpoint{4.069480in}{2.269531in}}%
\pgfpathlineto{\pgfqpoint{4.069480in}{2.272480in}}%
\pgfpathlineto{\pgfqpoint{4.074021in}{2.272480in}}%
\pgfpathlineto{\pgfqpoint{4.074021in}{2.269531in}}%
\pgfpathmoveto{\pgfqpoint{3.955956in}{2.334415in}}%
\pgfpathlineto{\pgfqpoint{3.955956in}{2.334415in}}%
\pgfpathlineto{\pgfqpoint{3.955956in}{2.337364in}}%
\pgfpathlineto{\pgfqpoint{3.960497in}{2.337364in}}%
\pgfpathlineto{\pgfqpoint{3.960497in}{2.334415in}}%
\pgfpathmoveto{\pgfqpoint{3.955956in}{2.337364in}}%
\pgfpathlineto{\pgfqpoint{3.955956in}{2.337364in}}%
\pgfpathlineto{\pgfqpoint{3.955956in}{2.340313in}}%
\pgfpathlineto{\pgfqpoint{3.960497in}{2.340313in}}%
\pgfpathlineto{\pgfqpoint{3.960497in}{2.337364in}}%
\pgfpathmoveto{\pgfqpoint{3.960497in}{2.334415in}}%
\pgfpathlineto{\pgfqpoint{3.960497in}{2.334415in}}%
\pgfpathlineto{\pgfqpoint{3.960497in}{2.337364in}}%
\pgfpathlineto{\pgfqpoint{3.965038in}{2.337364in}}%
\pgfpathlineto{\pgfqpoint{3.965038in}{2.334415in}}%
\pgfpathmoveto{\pgfqpoint{3.960497in}{2.337364in}}%
\pgfpathlineto{\pgfqpoint{3.960497in}{2.337364in}}%
\pgfpathlineto{\pgfqpoint{3.960497in}{2.340313in}}%
\pgfpathlineto{\pgfqpoint{3.965038in}{2.340313in}}%
\pgfpathlineto{\pgfqpoint{3.965038in}{2.337364in}}%
\pgfpathmoveto{\pgfqpoint{3.974120in}{2.322618in}}%
\pgfpathlineto{\pgfqpoint{3.974120in}{2.322618in}}%
\pgfpathlineto{\pgfqpoint{3.974120in}{2.325567in}}%
\pgfpathlineto{\pgfqpoint{3.978661in}{2.325567in}}%
\pgfpathlineto{\pgfqpoint{3.978661in}{2.322618in}}%
\pgfpathmoveto{\pgfqpoint{3.974120in}{2.325567in}}%
\pgfpathlineto{\pgfqpoint{3.974120in}{2.325567in}}%
\pgfpathlineto{\pgfqpoint{3.974120in}{2.328516in}}%
\pgfpathlineto{\pgfqpoint{3.978661in}{2.328516in}}%
\pgfpathlineto{\pgfqpoint{3.978661in}{2.325567in}}%
\pgfpathmoveto{\pgfqpoint{3.978661in}{2.322618in}}%
\pgfpathlineto{\pgfqpoint{3.978661in}{2.322618in}}%
\pgfpathlineto{\pgfqpoint{3.978661in}{2.325567in}}%
\pgfpathlineto{\pgfqpoint{3.983202in}{2.325567in}}%
\pgfpathlineto{\pgfqpoint{3.983202in}{2.322618in}}%
\pgfpathmoveto{\pgfqpoint{3.978661in}{2.325567in}}%
\pgfpathlineto{\pgfqpoint{3.978661in}{2.325567in}}%
\pgfpathlineto{\pgfqpoint{3.978661in}{2.328516in}}%
\pgfpathlineto{\pgfqpoint{3.983202in}{2.328516in}}%
\pgfpathlineto{\pgfqpoint{3.983202in}{2.325567in}}%
\pgfpathmoveto{\pgfqpoint{3.965038in}{2.328516in}}%
\pgfpathlineto{\pgfqpoint{3.965038in}{2.328516in}}%
\pgfpathlineto{\pgfqpoint{3.965038in}{2.331465in}}%
\pgfpathlineto{\pgfqpoint{3.969579in}{2.331465in}}%
\pgfpathlineto{\pgfqpoint{3.969579in}{2.328516in}}%
\pgfpathmoveto{\pgfqpoint{3.965038in}{2.331465in}}%
\pgfpathlineto{\pgfqpoint{3.965038in}{2.331465in}}%
\pgfpathlineto{\pgfqpoint{3.965038in}{2.334415in}}%
\pgfpathlineto{\pgfqpoint{3.969579in}{2.334415in}}%
\pgfpathlineto{\pgfqpoint{3.969579in}{2.331465in}}%
\pgfpathmoveto{\pgfqpoint{3.969579in}{2.328516in}}%
\pgfpathlineto{\pgfqpoint{3.969579in}{2.328516in}}%
\pgfpathlineto{\pgfqpoint{3.969579in}{2.331465in}}%
\pgfpathlineto{\pgfqpoint{3.974120in}{2.331465in}}%
\pgfpathlineto{\pgfqpoint{3.974120in}{2.328516in}}%
\pgfpathmoveto{\pgfqpoint{3.969579in}{2.331465in}}%
\pgfpathlineto{\pgfqpoint{3.969579in}{2.331465in}}%
\pgfpathlineto{\pgfqpoint{3.969579in}{2.334415in}}%
\pgfpathlineto{\pgfqpoint{3.974120in}{2.334415in}}%
\pgfpathlineto{\pgfqpoint{3.974120in}{2.331465in}}%
\pgfpathmoveto{\pgfqpoint{3.965038in}{2.334415in}}%
\pgfpathlineto{\pgfqpoint{3.965038in}{2.334415in}}%
\pgfpathlineto{\pgfqpoint{3.965038in}{2.337364in}}%
\pgfpathlineto{\pgfqpoint{3.969579in}{2.337364in}}%
\pgfpathlineto{\pgfqpoint{3.969579in}{2.334415in}}%
\pgfpathmoveto{\pgfqpoint{3.965038in}{2.337364in}}%
\pgfpathlineto{\pgfqpoint{3.965038in}{2.337364in}}%
\pgfpathlineto{\pgfqpoint{3.965038in}{2.340313in}}%
\pgfpathlineto{\pgfqpoint{3.969579in}{2.340313in}}%
\pgfpathlineto{\pgfqpoint{3.969579in}{2.337364in}}%
\pgfpathmoveto{\pgfqpoint{3.969579in}{2.334415in}}%
\pgfpathlineto{\pgfqpoint{3.969579in}{2.334415in}}%
\pgfpathlineto{\pgfqpoint{3.969579in}{2.337364in}}%
\pgfpathlineto{\pgfqpoint{3.974120in}{2.337364in}}%
\pgfpathlineto{\pgfqpoint{3.974120in}{2.334415in}}%
\pgfpathmoveto{\pgfqpoint{3.974120in}{2.328516in}}%
\pgfpathlineto{\pgfqpoint{3.974120in}{2.328516in}}%
\pgfpathlineto{\pgfqpoint{3.974120in}{2.331465in}}%
\pgfpathlineto{\pgfqpoint{3.978661in}{2.331465in}}%
\pgfpathlineto{\pgfqpoint{3.978661in}{2.328516in}}%
\pgfpathmoveto{\pgfqpoint{3.974120in}{2.331465in}}%
\pgfpathlineto{\pgfqpoint{3.974120in}{2.331465in}}%
\pgfpathlineto{\pgfqpoint{3.974120in}{2.334415in}}%
\pgfpathlineto{\pgfqpoint{3.978661in}{2.334415in}}%
\pgfpathlineto{\pgfqpoint{3.978661in}{2.331465in}}%
\pgfpathmoveto{\pgfqpoint{3.978661in}{2.328516in}}%
\pgfpathlineto{\pgfqpoint{3.978661in}{2.328516in}}%
\pgfpathlineto{\pgfqpoint{3.978661in}{2.331465in}}%
\pgfpathlineto{\pgfqpoint{3.983202in}{2.331465in}}%
\pgfpathlineto{\pgfqpoint{3.983202in}{2.328516in}}%
\pgfpathmoveto{\pgfqpoint{3.992284in}{2.310820in}}%
\pgfpathlineto{\pgfqpoint{3.992284in}{2.310820in}}%
\pgfpathlineto{\pgfqpoint{3.992284in}{2.313770in}}%
\pgfpathlineto{\pgfqpoint{3.996825in}{2.313770in}}%
\pgfpathlineto{\pgfqpoint{3.996825in}{2.310820in}}%
\pgfpathmoveto{\pgfqpoint{3.992284in}{2.313770in}}%
\pgfpathlineto{\pgfqpoint{3.992284in}{2.313770in}}%
\pgfpathlineto{\pgfqpoint{3.992284in}{2.316719in}}%
\pgfpathlineto{\pgfqpoint{3.996825in}{2.316719in}}%
\pgfpathlineto{\pgfqpoint{3.996825in}{2.313770in}}%
\pgfpathmoveto{\pgfqpoint{3.996825in}{2.310820in}}%
\pgfpathlineto{\pgfqpoint{3.996825in}{2.310820in}}%
\pgfpathlineto{\pgfqpoint{3.996825in}{2.313770in}}%
\pgfpathlineto{\pgfqpoint{4.001365in}{2.313770in}}%
\pgfpathlineto{\pgfqpoint{4.001365in}{2.310820in}}%
\pgfpathmoveto{\pgfqpoint{3.996825in}{2.313770in}}%
\pgfpathlineto{\pgfqpoint{3.996825in}{2.313770in}}%
\pgfpathlineto{\pgfqpoint{3.996825in}{2.316719in}}%
\pgfpathlineto{\pgfqpoint{4.001365in}{2.316719in}}%
\pgfpathlineto{\pgfqpoint{4.001365in}{2.313770in}}%
\pgfpathmoveto{\pgfqpoint{4.010447in}{2.299023in}}%
\pgfpathlineto{\pgfqpoint{4.010447in}{2.299023in}}%
\pgfpathlineto{\pgfqpoint{4.010447in}{2.301973in}}%
\pgfpathlineto{\pgfqpoint{4.014988in}{2.301973in}}%
\pgfpathlineto{\pgfqpoint{4.014988in}{2.299023in}}%
\pgfpathmoveto{\pgfqpoint{4.010447in}{2.301973in}}%
\pgfpathlineto{\pgfqpoint{4.010447in}{2.301973in}}%
\pgfpathlineto{\pgfqpoint{4.010447in}{2.304922in}}%
\pgfpathlineto{\pgfqpoint{4.014988in}{2.304922in}}%
\pgfpathlineto{\pgfqpoint{4.014988in}{2.301973in}}%
\pgfpathmoveto{\pgfqpoint{4.014988in}{2.299023in}}%
\pgfpathlineto{\pgfqpoint{4.014988in}{2.299023in}}%
\pgfpathlineto{\pgfqpoint{4.014988in}{2.301973in}}%
\pgfpathlineto{\pgfqpoint{4.019529in}{2.301973in}}%
\pgfpathlineto{\pgfqpoint{4.019529in}{2.299023in}}%
\pgfpathmoveto{\pgfqpoint{4.014988in}{2.301973in}}%
\pgfpathlineto{\pgfqpoint{4.014988in}{2.301973in}}%
\pgfpathlineto{\pgfqpoint{4.014988in}{2.304922in}}%
\pgfpathlineto{\pgfqpoint{4.019529in}{2.304922in}}%
\pgfpathlineto{\pgfqpoint{4.019529in}{2.301973in}}%
\pgfpathmoveto{\pgfqpoint{4.001365in}{2.304922in}}%
\pgfpathlineto{\pgfqpoint{4.001365in}{2.304922in}}%
\pgfpathlineto{\pgfqpoint{4.001365in}{2.307871in}}%
\pgfpathlineto{\pgfqpoint{4.005906in}{2.307871in}}%
\pgfpathlineto{\pgfqpoint{4.005906in}{2.304922in}}%
\pgfpathmoveto{\pgfqpoint{4.001365in}{2.307871in}}%
\pgfpathlineto{\pgfqpoint{4.001365in}{2.307871in}}%
\pgfpathlineto{\pgfqpoint{4.001365in}{2.310820in}}%
\pgfpathlineto{\pgfqpoint{4.005906in}{2.310820in}}%
\pgfpathlineto{\pgfqpoint{4.005906in}{2.307871in}}%
\pgfpathmoveto{\pgfqpoint{4.005906in}{2.304922in}}%
\pgfpathlineto{\pgfqpoint{4.005906in}{2.304922in}}%
\pgfpathlineto{\pgfqpoint{4.005906in}{2.307871in}}%
\pgfpathlineto{\pgfqpoint{4.010447in}{2.307871in}}%
\pgfpathlineto{\pgfqpoint{4.010447in}{2.304922in}}%
\pgfpathmoveto{\pgfqpoint{4.005906in}{2.307871in}}%
\pgfpathlineto{\pgfqpoint{4.005906in}{2.307871in}}%
\pgfpathlineto{\pgfqpoint{4.005906in}{2.310820in}}%
\pgfpathlineto{\pgfqpoint{4.010447in}{2.310820in}}%
\pgfpathlineto{\pgfqpoint{4.010447in}{2.307871in}}%
\pgfpathmoveto{\pgfqpoint{4.001365in}{2.310820in}}%
\pgfpathlineto{\pgfqpoint{4.001365in}{2.310820in}}%
\pgfpathlineto{\pgfqpoint{4.001365in}{2.313770in}}%
\pgfpathlineto{\pgfqpoint{4.005906in}{2.313770in}}%
\pgfpathlineto{\pgfqpoint{4.005906in}{2.310820in}}%
\pgfpathmoveto{\pgfqpoint{4.001365in}{2.313770in}}%
\pgfpathlineto{\pgfqpoint{4.001365in}{2.313770in}}%
\pgfpathlineto{\pgfqpoint{4.001365in}{2.316719in}}%
\pgfpathlineto{\pgfqpoint{4.005906in}{2.316719in}}%
\pgfpathlineto{\pgfqpoint{4.005906in}{2.313770in}}%
\pgfpathmoveto{\pgfqpoint{4.005906in}{2.310820in}}%
\pgfpathlineto{\pgfqpoint{4.005906in}{2.310820in}}%
\pgfpathlineto{\pgfqpoint{4.005906in}{2.313770in}}%
\pgfpathlineto{\pgfqpoint{4.010447in}{2.313770in}}%
\pgfpathlineto{\pgfqpoint{4.010447in}{2.310820in}}%
\pgfpathmoveto{\pgfqpoint{4.010447in}{2.304922in}}%
\pgfpathlineto{\pgfqpoint{4.010447in}{2.304922in}}%
\pgfpathlineto{\pgfqpoint{4.010447in}{2.307871in}}%
\pgfpathlineto{\pgfqpoint{4.014988in}{2.307871in}}%
\pgfpathlineto{\pgfqpoint{4.014988in}{2.304922in}}%
\pgfpathmoveto{\pgfqpoint{4.010447in}{2.307871in}}%
\pgfpathlineto{\pgfqpoint{4.010447in}{2.307871in}}%
\pgfpathlineto{\pgfqpoint{4.010447in}{2.310820in}}%
\pgfpathlineto{\pgfqpoint{4.014988in}{2.310820in}}%
\pgfpathlineto{\pgfqpoint{4.014988in}{2.307871in}}%
\pgfpathmoveto{\pgfqpoint{4.014988in}{2.304922in}}%
\pgfpathlineto{\pgfqpoint{4.014988in}{2.304922in}}%
\pgfpathlineto{\pgfqpoint{4.014988in}{2.307871in}}%
\pgfpathlineto{\pgfqpoint{4.019529in}{2.307871in}}%
\pgfpathlineto{\pgfqpoint{4.019529in}{2.304922in}}%
\pgfpathmoveto{\pgfqpoint{3.983202in}{2.316719in}}%
\pgfpathlineto{\pgfqpoint{3.983202in}{2.316719in}}%
\pgfpathlineto{\pgfqpoint{3.983202in}{2.319668in}}%
\pgfpathlineto{\pgfqpoint{3.987743in}{2.319668in}}%
\pgfpathlineto{\pgfqpoint{3.987743in}{2.316719in}}%
\pgfpathmoveto{\pgfqpoint{3.983202in}{2.319668in}}%
\pgfpathlineto{\pgfqpoint{3.983202in}{2.319668in}}%
\pgfpathlineto{\pgfqpoint{3.983202in}{2.322618in}}%
\pgfpathlineto{\pgfqpoint{3.987743in}{2.322618in}}%
\pgfpathlineto{\pgfqpoint{3.987743in}{2.319668in}}%
\pgfpathmoveto{\pgfqpoint{3.987743in}{2.316719in}}%
\pgfpathlineto{\pgfqpoint{3.987743in}{2.316719in}}%
\pgfpathlineto{\pgfqpoint{3.987743in}{2.319668in}}%
\pgfpathlineto{\pgfqpoint{3.992284in}{2.319668in}}%
\pgfpathlineto{\pgfqpoint{3.992284in}{2.316719in}}%
\pgfpathmoveto{\pgfqpoint{3.987743in}{2.319668in}}%
\pgfpathlineto{\pgfqpoint{3.987743in}{2.319668in}}%
\pgfpathlineto{\pgfqpoint{3.987743in}{2.322618in}}%
\pgfpathlineto{\pgfqpoint{3.992284in}{2.322618in}}%
\pgfpathlineto{\pgfqpoint{3.992284in}{2.319668in}}%
\pgfpathmoveto{\pgfqpoint{3.983202in}{2.322618in}}%
\pgfpathlineto{\pgfqpoint{3.983202in}{2.322618in}}%
\pgfpathlineto{\pgfqpoint{3.983202in}{2.325567in}}%
\pgfpathlineto{\pgfqpoint{3.987743in}{2.325567in}}%
\pgfpathlineto{\pgfqpoint{3.987743in}{2.322618in}}%
\pgfpathmoveto{\pgfqpoint{3.983202in}{2.325567in}}%
\pgfpathlineto{\pgfqpoint{3.983202in}{2.325567in}}%
\pgfpathlineto{\pgfqpoint{3.983202in}{2.328516in}}%
\pgfpathlineto{\pgfqpoint{3.987743in}{2.328516in}}%
\pgfpathlineto{\pgfqpoint{3.987743in}{2.325567in}}%
\pgfpathmoveto{\pgfqpoint{3.987743in}{2.322618in}}%
\pgfpathlineto{\pgfqpoint{3.987743in}{2.322618in}}%
\pgfpathlineto{\pgfqpoint{3.987743in}{2.325567in}}%
\pgfpathlineto{\pgfqpoint{3.992284in}{2.325567in}}%
\pgfpathlineto{\pgfqpoint{3.992284in}{2.322618in}}%
\pgfpathmoveto{\pgfqpoint{3.992284in}{2.316719in}}%
\pgfpathlineto{\pgfqpoint{3.992284in}{2.316719in}}%
\pgfpathlineto{\pgfqpoint{3.992284in}{2.319668in}}%
\pgfpathlineto{\pgfqpoint{3.996825in}{2.319668in}}%
\pgfpathlineto{\pgfqpoint{3.996825in}{2.316719in}}%
\pgfpathmoveto{\pgfqpoint{3.992284in}{2.319668in}}%
\pgfpathlineto{\pgfqpoint{3.992284in}{2.319668in}}%
\pgfpathlineto{\pgfqpoint{3.992284in}{2.322618in}}%
\pgfpathlineto{\pgfqpoint{3.996825in}{2.322618in}}%
\pgfpathlineto{\pgfqpoint{3.996825in}{2.319668in}}%
\pgfpathmoveto{\pgfqpoint{3.996825in}{2.316719in}}%
\pgfpathlineto{\pgfqpoint{3.996825in}{2.316719in}}%
\pgfpathlineto{\pgfqpoint{3.996825in}{2.319668in}}%
\pgfpathlineto{\pgfqpoint{4.001365in}{2.319668in}}%
\pgfpathlineto{\pgfqpoint{4.001365in}{2.316719in}}%
\pgfpathmoveto{\pgfqpoint{3.946874in}{2.340313in}}%
\pgfpathlineto{\pgfqpoint{3.946874in}{2.340313in}}%
\pgfpathlineto{\pgfqpoint{3.946874in}{2.343263in}}%
\pgfpathlineto{\pgfqpoint{3.951415in}{2.343263in}}%
\pgfpathlineto{\pgfqpoint{3.951415in}{2.340313in}}%
\pgfpathmoveto{\pgfqpoint{3.946874in}{2.343263in}}%
\pgfpathlineto{\pgfqpoint{3.946874in}{2.343263in}}%
\pgfpathlineto{\pgfqpoint{3.946874in}{2.346212in}}%
\pgfpathlineto{\pgfqpoint{3.951415in}{2.346212in}}%
\pgfpathlineto{\pgfqpoint{3.951415in}{2.343263in}}%
\pgfpathmoveto{\pgfqpoint{3.951415in}{2.340313in}}%
\pgfpathlineto{\pgfqpoint{3.951415in}{2.340313in}}%
\pgfpathlineto{\pgfqpoint{3.951415in}{2.343263in}}%
\pgfpathlineto{\pgfqpoint{3.955956in}{2.343263in}}%
\pgfpathlineto{\pgfqpoint{3.955956in}{2.340313in}}%
\pgfpathmoveto{\pgfqpoint{3.951415in}{2.343263in}}%
\pgfpathlineto{\pgfqpoint{3.951415in}{2.343263in}}%
\pgfpathlineto{\pgfqpoint{3.951415in}{2.346212in}}%
\pgfpathlineto{\pgfqpoint{3.955956in}{2.346212in}}%
\pgfpathlineto{\pgfqpoint{3.955956in}{2.343263in}}%
\pgfpathmoveto{\pgfqpoint{3.946874in}{2.346212in}}%
\pgfpathlineto{\pgfqpoint{3.946874in}{2.346212in}}%
\pgfpathlineto{\pgfqpoint{3.946874in}{2.349161in}}%
\pgfpathlineto{\pgfqpoint{3.951415in}{2.349161in}}%
\pgfpathlineto{\pgfqpoint{3.951415in}{2.346212in}}%
\pgfpathmoveto{\pgfqpoint{3.946874in}{2.349161in}}%
\pgfpathlineto{\pgfqpoint{3.946874in}{2.349161in}}%
\pgfpathlineto{\pgfqpoint{3.946874in}{2.352110in}}%
\pgfpathlineto{\pgfqpoint{3.951415in}{2.352110in}}%
\pgfpathlineto{\pgfqpoint{3.951415in}{2.349161in}}%
\pgfpathmoveto{\pgfqpoint{3.951415in}{2.346212in}}%
\pgfpathlineto{\pgfqpoint{3.951415in}{2.346212in}}%
\pgfpathlineto{\pgfqpoint{3.951415in}{2.349161in}}%
\pgfpathlineto{\pgfqpoint{3.955956in}{2.349161in}}%
\pgfpathlineto{\pgfqpoint{3.955956in}{2.346212in}}%
\pgfpathmoveto{\pgfqpoint{3.955956in}{2.340313in}}%
\pgfpathlineto{\pgfqpoint{3.955956in}{2.340313in}}%
\pgfpathlineto{\pgfqpoint{3.955956in}{2.343263in}}%
\pgfpathlineto{\pgfqpoint{3.960497in}{2.343263in}}%
\pgfpathlineto{\pgfqpoint{3.960497in}{2.340313in}}%
\pgfpathmoveto{\pgfqpoint{3.955956in}{2.343263in}}%
\pgfpathlineto{\pgfqpoint{3.955956in}{2.343263in}}%
\pgfpathlineto{\pgfqpoint{3.955956in}{2.346212in}}%
\pgfpathlineto{\pgfqpoint{3.960497in}{2.346212in}}%
\pgfpathlineto{\pgfqpoint{3.960497in}{2.343263in}}%
\pgfpathmoveto{\pgfqpoint{3.960497in}{2.340313in}}%
\pgfpathlineto{\pgfqpoint{3.960497in}{2.340313in}}%
\pgfpathlineto{\pgfqpoint{3.960497in}{2.343263in}}%
\pgfpathlineto{\pgfqpoint{3.965038in}{2.343263in}}%
\pgfpathlineto{\pgfqpoint{3.965038in}{2.340313in}}%
\pgfpathmoveto{\pgfqpoint{4.019529in}{2.293125in}}%
\pgfpathlineto{\pgfqpoint{4.019529in}{2.293125in}}%
\pgfpathlineto{\pgfqpoint{4.019529in}{2.296074in}}%
\pgfpathlineto{\pgfqpoint{4.024070in}{2.296074in}}%
\pgfpathlineto{\pgfqpoint{4.024070in}{2.293125in}}%
\pgfpathmoveto{\pgfqpoint{4.019529in}{2.296074in}}%
\pgfpathlineto{\pgfqpoint{4.019529in}{2.296074in}}%
\pgfpathlineto{\pgfqpoint{4.019529in}{2.299023in}}%
\pgfpathlineto{\pgfqpoint{4.024070in}{2.299023in}}%
\pgfpathlineto{\pgfqpoint{4.024070in}{2.296074in}}%
\pgfpathmoveto{\pgfqpoint{4.024070in}{2.293125in}}%
\pgfpathlineto{\pgfqpoint{4.024070in}{2.293125in}}%
\pgfpathlineto{\pgfqpoint{4.024070in}{2.296074in}}%
\pgfpathlineto{\pgfqpoint{4.028611in}{2.296074in}}%
\pgfpathlineto{\pgfqpoint{4.028611in}{2.293125in}}%
\pgfpathmoveto{\pgfqpoint{4.024070in}{2.296074in}}%
\pgfpathlineto{\pgfqpoint{4.024070in}{2.296074in}}%
\pgfpathlineto{\pgfqpoint{4.024070in}{2.299023in}}%
\pgfpathlineto{\pgfqpoint{4.028611in}{2.299023in}}%
\pgfpathlineto{\pgfqpoint{4.028611in}{2.296074in}}%
\pgfpathmoveto{\pgfqpoint{4.019529in}{2.299023in}}%
\pgfpathlineto{\pgfqpoint{4.019529in}{2.299023in}}%
\pgfpathlineto{\pgfqpoint{4.019529in}{2.301973in}}%
\pgfpathlineto{\pgfqpoint{4.024070in}{2.301973in}}%
\pgfpathlineto{\pgfqpoint{4.024070in}{2.299023in}}%
\pgfpathmoveto{\pgfqpoint{4.019529in}{2.301973in}}%
\pgfpathlineto{\pgfqpoint{4.019529in}{2.301973in}}%
\pgfpathlineto{\pgfqpoint{4.019529in}{2.304922in}}%
\pgfpathlineto{\pgfqpoint{4.024070in}{2.304922in}}%
\pgfpathlineto{\pgfqpoint{4.024070in}{2.301973in}}%
\pgfpathmoveto{\pgfqpoint{4.024070in}{2.299023in}}%
\pgfpathlineto{\pgfqpoint{4.024070in}{2.299023in}}%
\pgfpathlineto{\pgfqpoint{4.024070in}{2.301973in}}%
\pgfpathlineto{\pgfqpoint{4.028611in}{2.301973in}}%
\pgfpathlineto{\pgfqpoint{4.028611in}{2.299023in}}%
\pgfpathmoveto{\pgfqpoint{4.028611in}{2.293125in}}%
\pgfpathlineto{\pgfqpoint{4.028611in}{2.293125in}}%
\pgfpathlineto{\pgfqpoint{4.028611in}{2.296074in}}%
\pgfpathlineto{\pgfqpoint{4.033152in}{2.296074in}}%
\pgfpathlineto{\pgfqpoint{4.033152in}{2.293125in}}%
\pgfpathmoveto{\pgfqpoint{4.028611in}{2.296074in}}%
\pgfpathlineto{\pgfqpoint{4.028611in}{2.296074in}}%
\pgfpathlineto{\pgfqpoint{4.028611in}{2.299023in}}%
\pgfpathlineto{\pgfqpoint{4.033152in}{2.299023in}}%
\pgfpathlineto{\pgfqpoint{4.033152in}{2.296074in}}%
\pgfpathmoveto{\pgfqpoint{4.033152in}{2.293125in}}%
\pgfpathlineto{\pgfqpoint{4.033152in}{2.293125in}}%
\pgfpathlineto{\pgfqpoint{4.033152in}{2.296074in}}%
\pgfpathlineto{\pgfqpoint{4.037693in}{2.296074in}}%
\pgfpathlineto{\pgfqpoint{4.037693in}{2.293125in}}%
\pgfpathmoveto{\pgfqpoint{4.092185in}{2.009997in}}%
\pgfpathlineto{\pgfqpoint{4.092185in}{2.009997in}}%
\pgfpathlineto{\pgfqpoint{4.092185in}{2.012947in}}%
\pgfpathlineto{\pgfqpoint{4.096726in}{2.012947in}}%
\pgfpathlineto{\pgfqpoint{4.096726in}{2.009997in}}%
\pgfpathmoveto{\pgfqpoint{4.092185in}{2.012947in}}%
\pgfpathlineto{\pgfqpoint{4.092185in}{2.012947in}}%
\pgfpathlineto{\pgfqpoint{4.092185in}{2.015896in}}%
\pgfpathlineto{\pgfqpoint{4.096726in}{2.015896in}}%
\pgfpathlineto{\pgfqpoint{4.096726in}{2.012947in}}%
\pgfpathmoveto{\pgfqpoint{4.096726in}{2.009997in}}%
\pgfpathlineto{\pgfqpoint{4.096726in}{2.009997in}}%
\pgfpathlineto{\pgfqpoint{4.096726in}{2.012947in}}%
\pgfpathlineto{\pgfqpoint{4.101267in}{2.012947in}}%
\pgfpathlineto{\pgfqpoint{4.101267in}{2.009997in}}%
\pgfpathmoveto{\pgfqpoint{4.096726in}{2.012947in}}%
\pgfpathlineto{\pgfqpoint{4.096726in}{2.012947in}}%
\pgfpathlineto{\pgfqpoint{4.096726in}{2.015896in}}%
\pgfpathlineto{\pgfqpoint{4.101267in}{2.015896in}}%
\pgfpathlineto{\pgfqpoint{4.101267in}{2.012947in}}%
\pgfpathmoveto{\pgfqpoint{4.101267in}{2.009997in}}%
\pgfpathlineto{\pgfqpoint{4.101267in}{2.009997in}}%
\pgfpathlineto{\pgfqpoint{4.101267in}{2.012947in}}%
\pgfpathlineto{\pgfqpoint{4.105808in}{2.012947in}}%
\pgfpathlineto{\pgfqpoint{4.105808in}{2.009997in}}%
\pgfpathmoveto{\pgfqpoint{4.101267in}{2.012947in}}%
\pgfpathlineto{\pgfqpoint{4.101267in}{2.012947in}}%
\pgfpathlineto{\pgfqpoint{4.101267in}{2.015896in}}%
\pgfpathlineto{\pgfqpoint{4.105808in}{2.015896in}}%
\pgfpathlineto{\pgfqpoint{4.105808in}{2.012947in}}%
\pgfpathmoveto{\pgfqpoint{4.105808in}{2.009997in}}%
\pgfpathlineto{\pgfqpoint{4.105808in}{2.009997in}}%
\pgfpathlineto{\pgfqpoint{4.105808in}{2.012947in}}%
\pgfpathlineto{\pgfqpoint{4.110350in}{2.012947in}}%
\pgfpathlineto{\pgfqpoint{4.110350in}{2.009997in}}%
\pgfpathmoveto{\pgfqpoint{4.105808in}{2.012947in}}%
\pgfpathlineto{\pgfqpoint{4.105808in}{2.012947in}}%
\pgfpathlineto{\pgfqpoint{4.105808in}{2.015896in}}%
\pgfpathlineto{\pgfqpoint{4.110350in}{2.015896in}}%
\pgfpathlineto{\pgfqpoint{4.110350in}{2.012947in}}%
\pgfpathmoveto{\pgfqpoint{4.110350in}{2.009997in}}%
\pgfpathlineto{\pgfqpoint{4.110350in}{2.009997in}}%
\pgfpathlineto{\pgfqpoint{4.110350in}{2.012947in}}%
\pgfpathlineto{\pgfqpoint{4.114891in}{2.012947in}}%
\pgfpathlineto{\pgfqpoint{4.114891in}{2.009997in}}%
\pgfpathmoveto{\pgfqpoint{4.110350in}{2.012947in}}%
\pgfpathlineto{\pgfqpoint{4.110350in}{2.012947in}}%
\pgfpathlineto{\pgfqpoint{4.110350in}{2.015896in}}%
\pgfpathlineto{\pgfqpoint{4.114891in}{2.015896in}}%
\pgfpathlineto{\pgfqpoint{4.114891in}{2.012947in}}%
\pgfpathmoveto{\pgfqpoint{4.114891in}{2.009997in}}%
\pgfpathlineto{\pgfqpoint{4.114891in}{2.009997in}}%
\pgfpathlineto{\pgfqpoint{4.114891in}{2.012947in}}%
\pgfpathlineto{\pgfqpoint{4.119432in}{2.012947in}}%
\pgfpathlineto{\pgfqpoint{4.119432in}{2.009997in}}%
\pgfpathmoveto{\pgfqpoint{4.114891in}{2.012947in}}%
\pgfpathlineto{\pgfqpoint{4.114891in}{2.012947in}}%
\pgfpathlineto{\pgfqpoint{4.114891in}{2.015896in}}%
\pgfpathlineto{\pgfqpoint{4.119432in}{2.015896in}}%
\pgfpathlineto{\pgfqpoint{4.119432in}{2.012947in}}%
\pgfpathmoveto{\pgfqpoint{4.119432in}{2.009997in}}%
\pgfpathlineto{\pgfqpoint{4.119432in}{2.009997in}}%
\pgfpathlineto{\pgfqpoint{4.119432in}{2.012947in}}%
\pgfpathlineto{\pgfqpoint{4.123973in}{2.012947in}}%
\pgfpathlineto{\pgfqpoint{4.123973in}{2.009997in}}%
\pgfpathmoveto{\pgfqpoint{4.119432in}{2.012947in}}%
\pgfpathlineto{\pgfqpoint{4.119432in}{2.012947in}}%
\pgfpathlineto{\pgfqpoint{4.119432in}{2.015896in}}%
\pgfpathlineto{\pgfqpoint{4.123973in}{2.015896in}}%
\pgfpathlineto{\pgfqpoint{4.123973in}{2.012947in}}%
\pgfpathmoveto{\pgfqpoint{4.123973in}{2.009997in}}%
\pgfpathlineto{\pgfqpoint{4.123973in}{2.009997in}}%
\pgfpathlineto{\pgfqpoint{4.123973in}{2.012947in}}%
\pgfpathlineto{\pgfqpoint{4.128515in}{2.012947in}}%
\pgfpathlineto{\pgfqpoint{4.128515in}{2.009997in}}%
\pgfpathmoveto{\pgfqpoint{4.123973in}{2.012947in}}%
\pgfpathlineto{\pgfqpoint{4.123973in}{2.012947in}}%
\pgfpathlineto{\pgfqpoint{4.123973in}{2.015896in}}%
\pgfpathlineto{\pgfqpoint{4.128515in}{2.015896in}}%
\pgfpathlineto{\pgfqpoint{4.128515in}{2.012947in}}%
\pgfpathmoveto{\pgfqpoint{4.128515in}{2.009997in}}%
\pgfpathlineto{\pgfqpoint{4.128515in}{2.009997in}}%
\pgfpathlineto{\pgfqpoint{4.128515in}{2.012947in}}%
\pgfpathlineto{\pgfqpoint{4.133056in}{2.012947in}}%
\pgfpathlineto{\pgfqpoint{4.133056in}{2.009997in}}%
\pgfpathmoveto{\pgfqpoint{4.128515in}{2.012947in}}%
\pgfpathlineto{\pgfqpoint{4.128515in}{2.012947in}}%
\pgfpathlineto{\pgfqpoint{4.128515in}{2.015896in}}%
\pgfpathlineto{\pgfqpoint{4.133056in}{2.015896in}}%
\pgfpathlineto{\pgfqpoint{4.133056in}{2.012947in}}%
\pgfpathmoveto{\pgfqpoint{4.133056in}{2.009997in}}%
\pgfpathlineto{\pgfqpoint{4.133056in}{2.009997in}}%
\pgfpathlineto{\pgfqpoint{4.133056in}{2.012947in}}%
\pgfpathlineto{\pgfqpoint{4.137597in}{2.012947in}}%
\pgfpathlineto{\pgfqpoint{4.137597in}{2.009997in}}%
\pgfpathmoveto{\pgfqpoint{4.133056in}{2.012947in}}%
\pgfpathlineto{\pgfqpoint{4.133056in}{2.012947in}}%
\pgfpathlineto{\pgfqpoint{4.133056in}{2.015896in}}%
\pgfpathlineto{\pgfqpoint{4.137597in}{2.015896in}}%
\pgfpathlineto{\pgfqpoint{4.137597in}{2.012947in}}%
\pgfpathmoveto{\pgfqpoint{4.137597in}{2.009997in}}%
\pgfpathlineto{\pgfqpoint{4.137597in}{2.009997in}}%
\pgfpathlineto{\pgfqpoint{4.137597in}{2.012947in}}%
\pgfpathlineto{\pgfqpoint{4.142138in}{2.012947in}}%
\pgfpathlineto{\pgfqpoint{4.142138in}{2.009997in}}%
\pgfpathmoveto{\pgfqpoint{4.137597in}{2.012947in}}%
\pgfpathlineto{\pgfqpoint{4.137597in}{2.012947in}}%
\pgfpathlineto{\pgfqpoint{4.137597in}{2.015896in}}%
\pgfpathlineto{\pgfqpoint{4.142138in}{2.015896in}}%
\pgfpathlineto{\pgfqpoint{4.142138in}{2.012947in}}%
\pgfpathmoveto{\pgfqpoint{4.142138in}{2.009997in}}%
\pgfpathlineto{\pgfqpoint{4.142138in}{2.009997in}}%
\pgfpathlineto{\pgfqpoint{4.142138in}{2.012947in}}%
\pgfpathlineto{\pgfqpoint{4.146680in}{2.012947in}}%
\pgfpathlineto{\pgfqpoint{4.146680in}{2.009997in}}%
\pgfpathmoveto{\pgfqpoint{4.142138in}{2.012947in}}%
\pgfpathlineto{\pgfqpoint{4.142138in}{2.012947in}}%
\pgfpathlineto{\pgfqpoint{4.142138in}{2.015896in}}%
\pgfpathlineto{\pgfqpoint{4.146680in}{2.015896in}}%
\pgfpathlineto{\pgfqpoint{4.146680in}{2.012947in}}%
\pgfpathmoveto{\pgfqpoint{4.146680in}{2.009997in}}%
\pgfpathlineto{\pgfqpoint{4.146680in}{2.009997in}}%
\pgfpathlineto{\pgfqpoint{4.146680in}{2.012947in}}%
\pgfpathlineto{\pgfqpoint{4.151221in}{2.012947in}}%
\pgfpathlineto{\pgfqpoint{4.151221in}{2.009997in}}%
\pgfpathmoveto{\pgfqpoint{4.146680in}{2.012947in}}%
\pgfpathlineto{\pgfqpoint{4.146680in}{2.012947in}}%
\pgfpathlineto{\pgfqpoint{4.146680in}{2.015896in}}%
\pgfpathlineto{\pgfqpoint{4.151221in}{2.015896in}}%
\pgfpathlineto{\pgfqpoint{4.151221in}{2.012947in}}%
\pgfpathmoveto{\pgfqpoint{4.151221in}{2.009997in}}%
\pgfpathlineto{\pgfqpoint{4.151221in}{2.009997in}}%
\pgfpathlineto{\pgfqpoint{4.151221in}{2.012947in}}%
\pgfpathlineto{\pgfqpoint{4.155762in}{2.012947in}}%
\pgfpathlineto{\pgfqpoint{4.155762in}{2.009997in}}%
\pgfpathmoveto{\pgfqpoint{4.151221in}{2.012947in}}%
\pgfpathlineto{\pgfqpoint{4.151221in}{2.012947in}}%
\pgfpathlineto{\pgfqpoint{4.151221in}{2.015896in}}%
\pgfpathlineto{\pgfqpoint{4.155762in}{2.015896in}}%
\pgfpathlineto{\pgfqpoint{4.155762in}{2.012947in}}%
\pgfpathmoveto{\pgfqpoint{4.155762in}{2.009997in}}%
\pgfpathlineto{\pgfqpoint{4.155762in}{2.009997in}}%
\pgfpathlineto{\pgfqpoint{4.155762in}{2.012947in}}%
\pgfpathlineto{\pgfqpoint{4.160303in}{2.012947in}}%
\pgfpathlineto{\pgfqpoint{4.160303in}{2.009997in}}%
\pgfpathmoveto{\pgfqpoint{4.155762in}{2.012947in}}%
\pgfpathlineto{\pgfqpoint{4.155762in}{2.012947in}}%
\pgfpathlineto{\pgfqpoint{4.155762in}{2.015896in}}%
\pgfpathlineto{\pgfqpoint{4.160303in}{2.015896in}}%
\pgfpathlineto{\pgfqpoint{4.160303in}{2.012947in}}%
\pgfpathmoveto{\pgfqpoint{4.160303in}{2.009997in}}%
\pgfpathlineto{\pgfqpoint{4.160303in}{2.009997in}}%
\pgfpathlineto{\pgfqpoint{4.160303in}{2.012947in}}%
\pgfpathlineto{\pgfqpoint{4.164844in}{2.012947in}}%
\pgfpathlineto{\pgfqpoint{4.164844in}{2.009997in}}%
\pgfpathmoveto{\pgfqpoint{4.160303in}{2.012947in}}%
\pgfpathlineto{\pgfqpoint{4.160303in}{2.012947in}}%
\pgfpathlineto{\pgfqpoint{4.160303in}{2.015896in}}%
\pgfpathlineto{\pgfqpoint{4.164844in}{2.015896in}}%
\pgfpathlineto{\pgfqpoint{4.164844in}{2.012947in}}%
\pgfpathmoveto{\pgfqpoint{4.164844in}{2.009997in}}%
\pgfpathlineto{\pgfqpoint{4.164844in}{2.009997in}}%
\pgfpathlineto{\pgfqpoint{4.164844in}{2.012947in}}%
\pgfpathlineto{\pgfqpoint{4.169386in}{2.012947in}}%
\pgfpathlineto{\pgfqpoint{4.169386in}{2.009997in}}%
\pgfpathmoveto{\pgfqpoint{4.164844in}{2.012947in}}%
\pgfpathlineto{\pgfqpoint{4.164844in}{2.012947in}}%
\pgfpathlineto{\pgfqpoint{4.164844in}{2.015896in}}%
\pgfpathlineto{\pgfqpoint{4.169386in}{2.015896in}}%
\pgfpathlineto{\pgfqpoint{4.169386in}{2.012947in}}%
\pgfpathmoveto{\pgfqpoint{4.169386in}{2.009997in}}%
\pgfpathlineto{\pgfqpoint{4.169386in}{2.009997in}}%
\pgfpathlineto{\pgfqpoint{4.169386in}{2.012947in}}%
\pgfpathlineto{\pgfqpoint{4.173927in}{2.012947in}}%
\pgfpathlineto{\pgfqpoint{4.173927in}{2.009997in}}%
\pgfpathmoveto{\pgfqpoint{4.169386in}{2.012947in}}%
\pgfpathlineto{\pgfqpoint{4.169386in}{2.012947in}}%
\pgfpathlineto{\pgfqpoint{4.169386in}{2.015896in}}%
\pgfpathlineto{\pgfqpoint{4.173927in}{2.015896in}}%
\pgfpathlineto{\pgfqpoint{4.173927in}{2.012947in}}%
\pgfpathmoveto{\pgfqpoint{4.173927in}{2.009997in}}%
\pgfpathlineto{\pgfqpoint{4.173927in}{2.009997in}}%
\pgfpathlineto{\pgfqpoint{4.173927in}{2.012947in}}%
\pgfpathlineto{\pgfqpoint{4.178468in}{2.012947in}}%
\pgfpathlineto{\pgfqpoint{4.178468in}{2.009997in}}%
\pgfpathmoveto{\pgfqpoint{4.173927in}{2.012947in}}%
\pgfpathlineto{\pgfqpoint{4.173927in}{2.012947in}}%
\pgfpathlineto{\pgfqpoint{4.173927in}{2.015896in}}%
\pgfpathlineto{\pgfqpoint{4.178468in}{2.015896in}}%
\pgfpathlineto{\pgfqpoint{4.178468in}{2.012947in}}%
\pgfpathmoveto{\pgfqpoint{4.178468in}{2.009997in}}%
\pgfpathlineto{\pgfqpoint{4.178468in}{2.009997in}}%
\pgfpathlineto{\pgfqpoint{4.178468in}{2.012947in}}%
\pgfpathlineto{\pgfqpoint{4.183009in}{2.012947in}}%
\pgfpathlineto{\pgfqpoint{4.183009in}{2.009997in}}%
\pgfpathmoveto{\pgfqpoint{4.178468in}{2.012947in}}%
\pgfpathlineto{\pgfqpoint{4.178468in}{2.012947in}}%
\pgfpathlineto{\pgfqpoint{4.178468in}{2.015896in}}%
\pgfpathlineto{\pgfqpoint{4.183009in}{2.015896in}}%
\pgfpathlineto{\pgfqpoint{4.183009in}{2.012947in}}%
\pgfpathmoveto{\pgfqpoint{4.183009in}{2.009997in}}%
\pgfpathlineto{\pgfqpoint{4.183009in}{2.009997in}}%
\pgfpathlineto{\pgfqpoint{4.183009in}{2.012947in}}%
\pgfpathlineto{\pgfqpoint{4.187551in}{2.012947in}}%
\pgfpathlineto{\pgfqpoint{4.187551in}{2.009997in}}%
\pgfpathmoveto{\pgfqpoint{4.183009in}{2.012947in}}%
\pgfpathlineto{\pgfqpoint{4.183009in}{2.012947in}}%
\pgfpathlineto{\pgfqpoint{4.183009in}{2.015896in}}%
\pgfpathlineto{\pgfqpoint{4.187551in}{2.015896in}}%
\pgfpathlineto{\pgfqpoint{4.187551in}{2.012947in}}%
\pgfpathmoveto{\pgfqpoint{4.187551in}{2.009997in}}%
\pgfpathlineto{\pgfqpoint{4.187551in}{2.009997in}}%
\pgfpathlineto{\pgfqpoint{4.187551in}{2.012947in}}%
\pgfpathlineto{\pgfqpoint{4.192092in}{2.012947in}}%
\pgfpathlineto{\pgfqpoint{4.192092in}{2.009997in}}%
\pgfpathmoveto{\pgfqpoint{4.187551in}{2.012947in}}%
\pgfpathlineto{\pgfqpoint{4.187551in}{2.012947in}}%
\pgfpathlineto{\pgfqpoint{4.187551in}{2.015896in}}%
\pgfpathlineto{\pgfqpoint{4.192092in}{2.015896in}}%
\pgfpathlineto{\pgfqpoint{4.192092in}{2.012947in}}%
\pgfpathmoveto{\pgfqpoint{4.192092in}{2.009997in}}%
\pgfpathlineto{\pgfqpoint{4.192092in}{2.009997in}}%
\pgfpathlineto{\pgfqpoint{4.192092in}{2.012947in}}%
\pgfpathlineto{\pgfqpoint{4.196633in}{2.012947in}}%
\pgfpathlineto{\pgfqpoint{4.196633in}{2.009997in}}%
\pgfpathmoveto{\pgfqpoint{4.192092in}{2.012947in}}%
\pgfpathlineto{\pgfqpoint{4.192092in}{2.012947in}}%
\pgfpathlineto{\pgfqpoint{4.192092in}{2.015896in}}%
\pgfpathlineto{\pgfqpoint{4.196633in}{2.015896in}}%
\pgfpathlineto{\pgfqpoint{4.196633in}{2.012947in}}%
\pgfpathmoveto{\pgfqpoint{4.196633in}{2.009997in}}%
\pgfpathlineto{\pgfqpoint{4.196633in}{2.009997in}}%
\pgfpathlineto{\pgfqpoint{4.196633in}{2.012947in}}%
\pgfpathlineto{\pgfqpoint{4.201174in}{2.012947in}}%
\pgfpathlineto{\pgfqpoint{4.201174in}{2.009997in}}%
\pgfpathmoveto{\pgfqpoint{4.196633in}{2.012947in}}%
\pgfpathlineto{\pgfqpoint{4.196633in}{2.012947in}}%
\pgfpathlineto{\pgfqpoint{4.196633in}{2.015896in}}%
\pgfpathlineto{\pgfqpoint{4.201174in}{2.015896in}}%
\pgfpathlineto{\pgfqpoint{4.201174in}{2.012947in}}%
\pgfpathmoveto{\pgfqpoint{4.201174in}{2.009997in}}%
\pgfpathlineto{\pgfqpoint{4.201174in}{2.009997in}}%
\pgfpathlineto{\pgfqpoint{4.201174in}{2.012947in}}%
\pgfpathlineto{\pgfqpoint{4.205716in}{2.012947in}}%
\pgfpathlineto{\pgfqpoint{4.205716in}{2.009997in}}%
\pgfpathmoveto{\pgfqpoint{4.201174in}{2.012947in}}%
\pgfpathlineto{\pgfqpoint{4.201174in}{2.012947in}}%
\pgfpathlineto{\pgfqpoint{4.201174in}{2.015896in}}%
\pgfpathlineto{\pgfqpoint{4.205716in}{2.015896in}}%
\pgfpathlineto{\pgfqpoint{4.205716in}{2.012947in}}%
\pgfpathmoveto{\pgfqpoint{4.205716in}{2.009997in}}%
\pgfpathlineto{\pgfqpoint{4.205716in}{2.009997in}}%
\pgfpathlineto{\pgfqpoint{4.205716in}{2.012947in}}%
\pgfpathlineto{\pgfqpoint{4.210257in}{2.012947in}}%
\pgfpathlineto{\pgfqpoint{4.210257in}{2.009997in}}%
\pgfpathmoveto{\pgfqpoint{4.205716in}{2.012947in}}%
\pgfpathlineto{\pgfqpoint{4.205716in}{2.012947in}}%
\pgfpathlineto{\pgfqpoint{4.205716in}{2.015896in}}%
\pgfpathlineto{\pgfqpoint{4.210257in}{2.015896in}}%
\pgfpathlineto{\pgfqpoint{4.210257in}{2.012947in}}%
\pgfpathmoveto{\pgfqpoint{4.210257in}{2.009997in}}%
\pgfpathlineto{\pgfqpoint{4.210257in}{2.009997in}}%
\pgfpathlineto{\pgfqpoint{4.210257in}{2.012947in}}%
\pgfpathlineto{\pgfqpoint{4.214798in}{2.012947in}}%
\pgfpathlineto{\pgfqpoint{4.214798in}{2.009997in}}%
\pgfpathmoveto{\pgfqpoint{4.210257in}{2.012947in}}%
\pgfpathlineto{\pgfqpoint{4.210257in}{2.012947in}}%
\pgfpathlineto{\pgfqpoint{4.210257in}{2.015896in}}%
\pgfpathlineto{\pgfqpoint{4.214798in}{2.015896in}}%
\pgfpathlineto{\pgfqpoint{4.214798in}{2.012947in}}%
\pgfpathmoveto{\pgfqpoint{4.214798in}{2.009997in}}%
\pgfpathlineto{\pgfqpoint{4.214798in}{2.009997in}}%
\pgfpathlineto{\pgfqpoint{4.214798in}{2.012947in}}%
\pgfpathlineto{\pgfqpoint{4.219339in}{2.012947in}}%
\pgfpathlineto{\pgfqpoint{4.219339in}{2.009997in}}%
\pgfpathmoveto{\pgfqpoint{4.214798in}{2.012947in}}%
\pgfpathlineto{\pgfqpoint{4.214798in}{2.012947in}}%
\pgfpathlineto{\pgfqpoint{4.214798in}{2.015896in}}%
\pgfpathlineto{\pgfqpoint{4.219339in}{2.015896in}}%
\pgfpathlineto{\pgfqpoint{4.219339in}{2.012947in}}%
\pgfpathmoveto{\pgfqpoint{4.219339in}{2.009997in}}%
\pgfpathlineto{\pgfqpoint{4.219339in}{2.009997in}}%
\pgfpathlineto{\pgfqpoint{4.219339in}{2.012947in}}%
\pgfpathlineto{\pgfqpoint{4.223880in}{2.012947in}}%
\pgfpathlineto{\pgfqpoint{4.223880in}{2.009997in}}%
\pgfpathmoveto{\pgfqpoint{4.219339in}{2.012947in}}%
\pgfpathlineto{\pgfqpoint{4.219339in}{2.012947in}}%
\pgfpathlineto{\pgfqpoint{4.219339in}{2.015896in}}%
\pgfpathlineto{\pgfqpoint{4.223880in}{2.015896in}}%
\pgfpathlineto{\pgfqpoint{4.223880in}{2.012947in}}%
\pgfpathmoveto{\pgfqpoint{4.223880in}{2.009997in}}%
\pgfpathlineto{\pgfqpoint{4.223880in}{2.009997in}}%
\pgfpathlineto{\pgfqpoint{4.223880in}{2.012947in}}%
\pgfpathlineto{\pgfqpoint{4.228422in}{2.012947in}}%
\pgfpathlineto{\pgfqpoint{4.228422in}{2.009997in}}%
\pgfpathmoveto{\pgfqpoint{4.223880in}{2.012947in}}%
\pgfpathlineto{\pgfqpoint{4.223880in}{2.012947in}}%
\pgfpathlineto{\pgfqpoint{4.223880in}{2.015896in}}%
\pgfpathlineto{\pgfqpoint{4.228422in}{2.015896in}}%
\pgfpathlineto{\pgfqpoint{4.228422in}{2.012947in}}%
\pgfpathmoveto{\pgfqpoint{4.228422in}{2.009997in}}%
\pgfpathlineto{\pgfqpoint{4.228422in}{2.009997in}}%
\pgfpathlineto{\pgfqpoint{4.228422in}{2.012947in}}%
\pgfpathlineto{\pgfqpoint{4.232963in}{2.012947in}}%
\pgfpathlineto{\pgfqpoint{4.232963in}{2.009997in}}%
\pgfpathmoveto{\pgfqpoint{4.228422in}{2.012947in}}%
\pgfpathlineto{\pgfqpoint{4.228422in}{2.012947in}}%
\pgfpathlineto{\pgfqpoint{4.228422in}{2.015896in}}%
\pgfpathlineto{\pgfqpoint{4.232963in}{2.015896in}}%
\pgfpathlineto{\pgfqpoint{4.232963in}{2.012947in}}%
\pgfpathmoveto{\pgfqpoint{4.232963in}{2.009997in}}%
\pgfpathlineto{\pgfqpoint{4.232963in}{2.009997in}}%
\pgfpathlineto{\pgfqpoint{4.232963in}{2.012947in}}%
\pgfpathlineto{\pgfqpoint{4.237504in}{2.012947in}}%
\pgfpathlineto{\pgfqpoint{4.237504in}{2.009997in}}%
\pgfpathmoveto{\pgfqpoint{4.232963in}{2.012947in}}%
\pgfpathlineto{\pgfqpoint{4.232963in}{2.012947in}}%
\pgfpathlineto{\pgfqpoint{4.232963in}{2.015896in}}%
\pgfpathlineto{\pgfqpoint{4.237504in}{2.015896in}}%
\pgfpathlineto{\pgfqpoint{4.237504in}{2.012947in}}%
\pgfpathmoveto{\pgfqpoint{4.173927in}{2.192850in}}%
\pgfpathlineto{\pgfqpoint{4.173927in}{2.192850in}}%
\pgfpathlineto{\pgfqpoint{4.173927in}{2.195799in}}%
\pgfpathlineto{\pgfqpoint{4.178468in}{2.195799in}}%
\pgfpathlineto{\pgfqpoint{4.178468in}{2.192850in}}%
\pgfpathmoveto{\pgfqpoint{4.173927in}{2.195799in}}%
\pgfpathlineto{\pgfqpoint{4.173927in}{2.195799in}}%
\pgfpathlineto{\pgfqpoint{4.173927in}{2.198748in}}%
\pgfpathlineto{\pgfqpoint{4.178468in}{2.198748in}}%
\pgfpathlineto{\pgfqpoint{4.178468in}{2.195799in}}%
\pgfpathmoveto{\pgfqpoint{4.178468in}{2.192850in}}%
\pgfpathlineto{\pgfqpoint{4.178468in}{2.192850in}}%
\pgfpathlineto{\pgfqpoint{4.178468in}{2.195799in}}%
\pgfpathlineto{\pgfqpoint{4.183009in}{2.195799in}}%
\pgfpathlineto{\pgfqpoint{4.183009in}{2.192850in}}%
\pgfpathmoveto{\pgfqpoint{4.178468in}{2.195799in}}%
\pgfpathlineto{\pgfqpoint{4.178468in}{2.195799in}}%
\pgfpathlineto{\pgfqpoint{4.178468in}{2.198748in}}%
\pgfpathlineto{\pgfqpoint{4.183009in}{2.198748in}}%
\pgfpathlineto{\pgfqpoint{4.183009in}{2.195799in}}%
\pgfpathmoveto{\pgfqpoint{4.192092in}{2.181053in}}%
\pgfpathlineto{\pgfqpoint{4.192092in}{2.181053in}}%
\pgfpathlineto{\pgfqpoint{4.192092in}{2.184002in}}%
\pgfpathlineto{\pgfqpoint{4.196633in}{2.184002in}}%
\pgfpathlineto{\pgfqpoint{4.196633in}{2.181053in}}%
\pgfpathmoveto{\pgfqpoint{4.192092in}{2.184002in}}%
\pgfpathlineto{\pgfqpoint{4.192092in}{2.184002in}}%
\pgfpathlineto{\pgfqpoint{4.192092in}{2.186952in}}%
\pgfpathlineto{\pgfqpoint{4.196633in}{2.186952in}}%
\pgfpathlineto{\pgfqpoint{4.196633in}{2.184002in}}%
\pgfpathmoveto{\pgfqpoint{4.196633in}{2.181053in}}%
\pgfpathlineto{\pgfqpoint{4.196633in}{2.181053in}}%
\pgfpathlineto{\pgfqpoint{4.196633in}{2.184002in}}%
\pgfpathlineto{\pgfqpoint{4.201174in}{2.184002in}}%
\pgfpathlineto{\pgfqpoint{4.201174in}{2.181053in}}%
\pgfpathmoveto{\pgfqpoint{4.196633in}{2.184002in}}%
\pgfpathlineto{\pgfqpoint{4.196633in}{2.184002in}}%
\pgfpathlineto{\pgfqpoint{4.196633in}{2.186952in}}%
\pgfpathlineto{\pgfqpoint{4.201174in}{2.186952in}}%
\pgfpathlineto{\pgfqpoint{4.201174in}{2.184002in}}%
\pgfpathmoveto{\pgfqpoint{4.183009in}{2.186952in}}%
\pgfpathlineto{\pgfqpoint{4.183009in}{2.186952in}}%
\pgfpathlineto{\pgfqpoint{4.183009in}{2.189901in}}%
\pgfpathlineto{\pgfqpoint{4.187551in}{2.189901in}}%
\pgfpathlineto{\pgfqpoint{4.187551in}{2.186952in}}%
\pgfpathmoveto{\pgfqpoint{4.183009in}{2.189901in}}%
\pgfpathlineto{\pgfqpoint{4.183009in}{2.189901in}}%
\pgfpathlineto{\pgfqpoint{4.183009in}{2.192850in}}%
\pgfpathlineto{\pgfqpoint{4.187551in}{2.192850in}}%
\pgfpathlineto{\pgfqpoint{4.187551in}{2.189901in}}%
\pgfpathmoveto{\pgfqpoint{4.187551in}{2.186952in}}%
\pgfpathlineto{\pgfqpoint{4.187551in}{2.186952in}}%
\pgfpathlineto{\pgfqpoint{4.187551in}{2.189901in}}%
\pgfpathlineto{\pgfqpoint{4.192092in}{2.189901in}}%
\pgfpathlineto{\pgfqpoint{4.192092in}{2.186952in}}%
\pgfpathmoveto{\pgfqpoint{4.187551in}{2.189901in}}%
\pgfpathlineto{\pgfqpoint{4.187551in}{2.189901in}}%
\pgfpathlineto{\pgfqpoint{4.187551in}{2.192850in}}%
\pgfpathlineto{\pgfqpoint{4.192092in}{2.192850in}}%
\pgfpathlineto{\pgfqpoint{4.192092in}{2.189901in}}%
\pgfpathmoveto{\pgfqpoint{4.183009in}{2.192850in}}%
\pgfpathlineto{\pgfqpoint{4.183009in}{2.192850in}}%
\pgfpathlineto{\pgfqpoint{4.183009in}{2.195799in}}%
\pgfpathlineto{\pgfqpoint{4.187551in}{2.195799in}}%
\pgfpathlineto{\pgfqpoint{4.187551in}{2.192850in}}%
\pgfpathmoveto{\pgfqpoint{4.183009in}{2.195799in}}%
\pgfpathlineto{\pgfqpoint{4.183009in}{2.195799in}}%
\pgfpathlineto{\pgfqpoint{4.183009in}{2.198748in}}%
\pgfpathlineto{\pgfqpoint{4.187551in}{2.198748in}}%
\pgfpathlineto{\pgfqpoint{4.187551in}{2.195799in}}%
\pgfpathmoveto{\pgfqpoint{4.187551in}{2.192850in}}%
\pgfpathlineto{\pgfqpoint{4.187551in}{2.192850in}}%
\pgfpathlineto{\pgfqpoint{4.187551in}{2.195799in}}%
\pgfpathlineto{\pgfqpoint{4.192092in}{2.195799in}}%
\pgfpathlineto{\pgfqpoint{4.192092in}{2.192850in}}%
\pgfpathmoveto{\pgfqpoint{4.192092in}{2.186952in}}%
\pgfpathlineto{\pgfqpoint{4.192092in}{2.186952in}}%
\pgfpathlineto{\pgfqpoint{4.192092in}{2.189901in}}%
\pgfpathlineto{\pgfqpoint{4.196633in}{2.189901in}}%
\pgfpathlineto{\pgfqpoint{4.196633in}{2.186952in}}%
\pgfpathmoveto{\pgfqpoint{4.192092in}{2.189901in}}%
\pgfpathlineto{\pgfqpoint{4.192092in}{2.189901in}}%
\pgfpathlineto{\pgfqpoint{4.192092in}{2.192850in}}%
\pgfpathlineto{\pgfqpoint{4.196633in}{2.192850in}}%
\pgfpathlineto{\pgfqpoint{4.196633in}{2.189901in}}%
\pgfpathmoveto{\pgfqpoint{4.196633in}{2.186952in}}%
\pgfpathlineto{\pgfqpoint{4.196633in}{2.186952in}}%
\pgfpathlineto{\pgfqpoint{4.196633in}{2.189901in}}%
\pgfpathlineto{\pgfqpoint{4.201174in}{2.189901in}}%
\pgfpathlineto{\pgfqpoint{4.201174in}{2.186952in}}%
\pgfpathmoveto{\pgfqpoint{4.210257in}{2.169257in}}%
\pgfpathlineto{\pgfqpoint{4.210257in}{2.169257in}}%
\pgfpathlineto{\pgfqpoint{4.210257in}{2.172206in}}%
\pgfpathlineto{\pgfqpoint{4.214798in}{2.172206in}}%
\pgfpathlineto{\pgfqpoint{4.214798in}{2.169257in}}%
\pgfpathmoveto{\pgfqpoint{4.210257in}{2.172206in}}%
\pgfpathlineto{\pgfqpoint{4.210257in}{2.172206in}}%
\pgfpathlineto{\pgfqpoint{4.210257in}{2.175155in}}%
\pgfpathlineto{\pgfqpoint{4.214798in}{2.175155in}}%
\pgfpathlineto{\pgfqpoint{4.214798in}{2.172206in}}%
\pgfpathmoveto{\pgfqpoint{4.214798in}{2.169257in}}%
\pgfpathlineto{\pgfqpoint{4.214798in}{2.169257in}}%
\pgfpathlineto{\pgfqpoint{4.214798in}{2.172206in}}%
\pgfpathlineto{\pgfqpoint{4.219339in}{2.172206in}}%
\pgfpathlineto{\pgfqpoint{4.219339in}{2.169257in}}%
\pgfpathmoveto{\pgfqpoint{4.214798in}{2.172206in}}%
\pgfpathlineto{\pgfqpoint{4.214798in}{2.172206in}}%
\pgfpathlineto{\pgfqpoint{4.214798in}{2.175155in}}%
\pgfpathlineto{\pgfqpoint{4.219339in}{2.175155in}}%
\pgfpathlineto{\pgfqpoint{4.219339in}{2.172206in}}%
\pgfpathmoveto{\pgfqpoint{4.228422in}{2.157461in}}%
\pgfpathlineto{\pgfqpoint{4.228422in}{2.157461in}}%
\pgfpathlineto{\pgfqpoint{4.228422in}{2.160410in}}%
\pgfpathlineto{\pgfqpoint{4.232963in}{2.160410in}}%
\pgfpathlineto{\pgfqpoint{4.232963in}{2.157461in}}%
\pgfpathmoveto{\pgfqpoint{4.228422in}{2.160410in}}%
\pgfpathlineto{\pgfqpoint{4.228422in}{2.160410in}}%
\pgfpathlineto{\pgfqpoint{4.228422in}{2.163359in}}%
\pgfpathlineto{\pgfqpoint{4.232963in}{2.163359in}}%
\pgfpathlineto{\pgfqpoint{4.232963in}{2.160410in}}%
\pgfpathmoveto{\pgfqpoint{4.232963in}{2.157461in}}%
\pgfpathlineto{\pgfqpoint{4.232963in}{2.157461in}}%
\pgfpathlineto{\pgfqpoint{4.232963in}{2.160410in}}%
\pgfpathlineto{\pgfqpoint{4.237504in}{2.160410in}}%
\pgfpathlineto{\pgfqpoint{4.237504in}{2.157461in}}%
\pgfpathmoveto{\pgfqpoint{4.232963in}{2.160410in}}%
\pgfpathlineto{\pgfqpoint{4.232963in}{2.160410in}}%
\pgfpathlineto{\pgfqpoint{4.232963in}{2.163359in}}%
\pgfpathlineto{\pgfqpoint{4.237504in}{2.163359in}}%
\pgfpathlineto{\pgfqpoint{4.237504in}{2.160410in}}%
\pgfpathmoveto{\pgfqpoint{4.219339in}{2.163359in}}%
\pgfpathlineto{\pgfqpoint{4.219339in}{2.163359in}}%
\pgfpathlineto{\pgfqpoint{4.219339in}{2.166308in}}%
\pgfpathlineto{\pgfqpoint{4.223880in}{2.166308in}}%
\pgfpathlineto{\pgfqpoint{4.223880in}{2.163359in}}%
\pgfpathmoveto{\pgfqpoint{4.219339in}{2.166308in}}%
\pgfpathlineto{\pgfqpoint{4.219339in}{2.166308in}}%
\pgfpathlineto{\pgfqpoint{4.219339in}{2.169257in}}%
\pgfpathlineto{\pgfqpoint{4.223880in}{2.169257in}}%
\pgfpathlineto{\pgfqpoint{4.223880in}{2.166308in}}%
\pgfpathmoveto{\pgfqpoint{4.223880in}{2.163359in}}%
\pgfpathlineto{\pgfqpoint{4.223880in}{2.163359in}}%
\pgfpathlineto{\pgfqpoint{4.223880in}{2.166308in}}%
\pgfpathlineto{\pgfqpoint{4.228422in}{2.166308in}}%
\pgfpathlineto{\pgfqpoint{4.228422in}{2.163359in}}%
\pgfpathmoveto{\pgfqpoint{4.223880in}{2.166308in}}%
\pgfpathlineto{\pgfqpoint{4.223880in}{2.166308in}}%
\pgfpathlineto{\pgfqpoint{4.223880in}{2.169257in}}%
\pgfpathlineto{\pgfqpoint{4.228422in}{2.169257in}}%
\pgfpathlineto{\pgfqpoint{4.228422in}{2.166308in}}%
\pgfpathmoveto{\pgfqpoint{4.219339in}{2.169257in}}%
\pgfpathlineto{\pgfqpoint{4.219339in}{2.169257in}}%
\pgfpathlineto{\pgfqpoint{4.219339in}{2.172206in}}%
\pgfpathlineto{\pgfqpoint{4.223880in}{2.172206in}}%
\pgfpathlineto{\pgfqpoint{4.223880in}{2.169257in}}%
\pgfpathmoveto{\pgfqpoint{4.219339in}{2.172206in}}%
\pgfpathlineto{\pgfqpoint{4.219339in}{2.172206in}}%
\pgfpathlineto{\pgfqpoint{4.219339in}{2.175155in}}%
\pgfpathlineto{\pgfqpoint{4.223880in}{2.175155in}}%
\pgfpathlineto{\pgfqpoint{4.223880in}{2.172206in}}%
\pgfpathmoveto{\pgfqpoint{4.223880in}{2.169257in}}%
\pgfpathlineto{\pgfqpoint{4.223880in}{2.169257in}}%
\pgfpathlineto{\pgfqpoint{4.223880in}{2.172206in}}%
\pgfpathlineto{\pgfqpoint{4.228422in}{2.172206in}}%
\pgfpathlineto{\pgfqpoint{4.228422in}{2.169257in}}%
\pgfpathmoveto{\pgfqpoint{4.228422in}{2.163359in}}%
\pgfpathlineto{\pgfqpoint{4.228422in}{2.163359in}}%
\pgfpathlineto{\pgfqpoint{4.228422in}{2.166308in}}%
\pgfpathlineto{\pgfqpoint{4.232963in}{2.166308in}}%
\pgfpathlineto{\pgfqpoint{4.232963in}{2.163359in}}%
\pgfpathmoveto{\pgfqpoint{4.228422in}{2.166308in}}%
\pgfpathlineto{\pgfqpoint{4.228422in}{2.166308in}}%
\pgfpathlineto{\pgfqpoint{4.228422in}{2.169257in}}%
\pgfpathlineto{\pgfqpoint{4.232963in}{2.169257in}}%
\pgfpathlineto{\pgfqpoint{4.232963in}{2.166308in}}%
\pgfpathmoveto{\pgfqpoint{4.232963in}{2.163359in}}%
\pgfpathlineto{\pgfqpoint{4.232963in}{2.163359in}}%
\pgfpathlineto{\pgfqpoint{4.232963in}{2.166308in}}%
\pgfpathlineto{\pgfqpoint{4.237504in}{2.166308in}}%
\pgfpathlineto{\pgfqpoint{4.237504in}{2.163359in}}%
\pgfpathmoveto{\pgfqpoint{4.201174in}{2.175155in}}%
\pgfpathlineto{\pgfqpoint{4.201174in}{2.175155in}}%
\pgfpathlineto{\pgfqpoint{4.201174in}{2.178104in}}%
\pgfpathlineto{\pgfqpoint{4.205716in}{2.178104in}}%
\pgfpathlineto{\pgfqpoint{4.205716in}{2.175155in}}%
\pgfpathmoveto{\pgfqpoint{4.201174in}{2.178104in}}%
\pgfpathlineto{\pgfqpoint{4.201174in}{2.178104in}}%
\pgfpathlineto{\pgfqpoint{4.201174in}{2.181053in}}%
\pgfpathlineto{\pgfqpoint{4.205716in}{2.181053in}}%
\pgfpathlineto{\pgfqpoint{4.205716in}{2.178104in}}%
\pgfpathmoveto{\pgfqpoint{4.205716in}{2.175155in}}%
\pgfpathlineto{\pgfqpoint{4.205716in}{2.175155in}}%
\pgfpathlineto{\pgfqpoint{4.205716in}{2.178104in}}%
\pgfpathlineto{\pgfqpoint{4.210257in}{2.178104in}}%
\pgfpathlineto{\pgfqpoint{4.210257in}{2.175155in}}%
\pgfpathmoveto{\pgfqpoint{4.205716in}{2.178104in}}%
\pgfpathlineto{\pgfqpoint{4.205716in}{2.178104in}}%
\pgfpathlineto{\pgfqpoint{4.205716in}{2.181053in}}%
\pgfpathlineto{\pgfqpoint{4.210257in}{2.181053in}}%
\pgfpathlineto{\pgfqpoint{4.210257in}{2.178104in}}%
\pgfpathmoveto{\pgfqpoint{4.201174in}{2.181053in}}%
\pgfpathlineto{\pgfqpoint{4.201174in}{2.181053in}}%
\pgfpathlineto{\pgfqpoint{4.201174in}{2.184002in}}%
\pgfpathlineto{\pgfqpoint{4.205716in}{2.184002in}}%
\pgfpathlineto{\pgfqpoint{4.205716in}{2.181053in}}%
\pgfpathmoveto{\pgfqpoint{4.201174in}{2.184002in}}%
\pgfpathlineto{\pgfqpoint{4.201174in}{2.184002in}}%
\pgfpathlineto{\pgfqpoint{4.201174in}{2.186952in}}%
\pgfpathlineto{\pgfqpoint{4.205716in}{2.186952in}}%
\pgfpathlineto{\pgfqpoint{4.205716in}{2.184002in}}%
\pgfpathmoveto{\pgfqpoint{4.205716in}{2.181053in}}%
\pgfpathlineto{\pgfqpoint{4.205716in}{2.181053in}}%
\pgfpathlineto{\pgfqpoint{4.205716in}{2.184002in}}%
\pgfpathlineto{\pgfqpoint{4.210257in}{2.184002in}}%
\pgfpathlineto{\pgfqpoint{4.210257in}{2.181053in}}%
\pgfpathmoveto{\pgfqpoint{4.210257in}{2.175155in}}%
\pgfpathlineto{\pgfqpoint{4.210257in}{2.175155in}}%
\pgfpathlineto{\pgfqpoint{4.210257in}{2.178104in}}%
\pgfpathlineto{\pgfqpoint{4.214798in}{2.178104in}}%
\pgfpathlineto{\pgfqpoint{4.214798in}{2.175155in}}%
\pgfpathmoveto{\pgfqpoint{4.210257in}{2.178104in}}%
\pgfpathlineto{\pgfqpoint{4.210257in}{2.178104in}}%
\pgfpathlineto{\pgfqpoint{4.210257in}{2.181053in}}%
\pgfpathlineto{\pgfqpoint{4.214798in}{2.181053in}}%
\pgfpathlineto{\pgfqpoint{4.214798in}{2.178104in}}%
\pgfpathmoveto{\pgfqpoint{4.214798in}{2.175155in}}%
\pgfpathlineto{\pgfqpoint{4.214798in}{2.175155in}}%
\pgfpathlineto{\pgfqpoint{4.214798in}{2.178104in}}%
\pgfpathlineto{\pgfqpoint{4.219339in}{2.178104in}}%
\pgfpathlineto{\pgfqpoint{4.219339in}{2.175155in}}%
\pgfpathmoveto{\pgfqpoint{4.101267in}{2.240038in}}%
\pgfpathlineto{\pgfqpoint{4.101267in}{2.240038in}}%
\pgfpathlineto{\pgfqpoint{4.101267in}{2.242987in}}%
\pgfpathlineto{\pgfqpoint{4.105808in}{2.242987in}}%
\pgfpathlineto{\pgfqpoint{4.105808in}{2.240038in}}%
\pgfpathmoveto{\pgfqpoint{4.101267in}{2.242987in}}%
\pgfpathlineto{\pgfqpoint{4.101267in}{2.242987in}}%
\pgfpathlineto{\pgfqpoint{4.101267in}{2.245936in}}%
\pgfpathlineto{\pgfqpoint{4.105808in}{2.245936in}}%
\pgfpathlineto{\pgfqpoint{4.105808in}{2.242987in}}%
\pgfpathmoveto{\pgfqpoint{4.105808in}{2.240038in}}%
\pgfpathlineto{\pgfqpoint{4.105808in}{2.240038in}}%
\pgfpathlineto{\pgfqpoint{4.105808in}{2.242987in}}%
\pgfpathlineto{\pgfqpoint{4.110350in}{2.242987in}}%
\pgfpathlineto{\pgfqpoint{4.110350in}{2.240038in}}%
\pgfpathmoveto{\pgfqpoint{4.105808in}{2.242987in}}%
\pgfpathlineto{\pgfqpoint{4.105808in}{2.242987in}}%
\pgfpathlineto{\pgfqpoint{4.105808in}{2.245936in}}%
\pgfpathlineto{\pgfqpoint{4.110350in}{2.245936in}}%
\pgfpathlineto{\pgfqpoint{4.110350in}{2.242987in}}%
\pgfpathmoveto{\pgfqpoint{4.119432in}{2.228241in}}%
\pgfpathlineto{\pgfqpoint{4.119432in}{2.228241in}}%
\pgfpathlineto{\pgfqpoint{4.119432in}{2.231190in}}%
\pgfpathlineto{\pgfqpoint{4.123973in}{2.231190in}}%
\pgfpathlineto{\pgfqpoint{4.123973in}{2.228241in}}%
\pgfpathmoveto{\pgfqpoint{4.119432in}{2.231190in}}%
\pgfpathlineto{\pgfqpoint{4.119432in}{2.231190in}}%
\pgfpathlineto{\pgfqpoint{4.119432in}{2.234139in}}%
\pgfpathlineto{\pgfqpoint{4.123973in}{2.234139in}}%
\pgfpathlineto{\pgfqpoint{4.123973in}{2.231190in}}%
\pgfpathmoveto{\pgfqpoint{4.123973in}{2.228241in}}%
\pgfpathlineto{\pgfqpoint{4.123973in}{2.228241in}}%
\pgfpathlineto{\pgfqpoint{4.123973in}{2.231190in}}%
\pgfpathlineto{\pgfqpoint{4.128515in}{2.231190in}}%
\pgfpathlineto{\pgfqpoint{4.128515in}{2.228241in}}%
\pgfpathmoveto{\pgfqpoint{4.123973in}{2.231190in}}%
\pgfpathlineto{\pgfqpoint{4.123973in}{2.231190in}}%
\pgfpathlineto{\pgfqpoint{4.123973in}{2.234139in}}%
\pgfpathlineto{\pgfqpoint{4.128515in}{2.234139in}}%
\pgfpathlineto{\pgfqpoint{4.128515in}{2.231190in}}%
\pgfpathmoveto{\pgfqpoint{4.110350in}{2.234139in}}%
\pgfpathlineto{\pgfqpoint{4.110350in}{2.234139in}}%
\pgfpathlineto{\pgfqpoint{4.110350in}{2.237088in}}%
\pgfpathlineto{\pgfqpoint{4.114891in}{2.237088in}}%
\pgfpathlineto{\pgfqpoint{4.114891in}{2.234139in}}%
\pgfpathmoveto{\pgfqpoint{4.110350in}{2.237088in}}%
\pgfpathlineto{\pgfqpoint{4.110350in}{2.237088in}}%
\pgfpathlineto{\pgfqpoint{4.110350in}{2.240038in}}%
\pgfpathlineto{\pgfqpoint{4.114891in}{2.240038in}}%
\pgfpathlineto{\pgfqpoint{4.114891in}{2.237088in}}%
\pgfpathmoveto{\pgfqpoint{4.114891in}{2.234139in}}%
\pgfpathlineto{\pgfqpoint{4.114891in}{2.234139in}}%
\pgfpathlineto{\pgfqpoint{4.114891in}{2.237088in}}%
\pgfpathlineto{\pgfqpoint{4.119432in}{2.237088in}}%
\pgfpathlineto{\pgfqpoint{4.119432in}{2.234139in}}%
\pgfpathmoveto{\pgfqpoint{4.114891in}{2.237088in}}%
\pgfpathlineto{\pgfqpoint{4.114891in}{2.237088in}}%
\pgfpathlineto{\pgfqpoint{4.114891in}{2.240038in}}%
\pgfpathlineto{\pgfqpoint{4.119432in}{2.240038in}}%
\pgfpathlineto{\pgfqpoint{4.119432in}{2.237088in}}%
\pgfpathmoveto{\pgfqpoint{4.110350in}{2.240038in}}%
\pgfpathlineto{\pgfqpoint{4.110350in}{2.240038in}}%
\pgfpathlineto{\pgfqpoint{4.110350in}{2.242987in}}%
\pgfpathlineto{\pgfqpoint{4.114891in}{2.242987in}}%
\pgfpathlineto{\pgfqpoint{4.114891in}{2.240038in}}%
\pgfpathmoveto{\pgfqpoint{4.110350in}{2.242987in}}%
\pgfpathlineto{\pgfqpoint{4.110350in}{2.242987in}}%
\pgfpathlineto{\pgfqpoint{4.110350in}{2.245936in}}%
\pgfpathlineto{\pgfqpoint{4.114891in}{2.245936in}}%
\pgfpathlineto{\pgfqpoint{4.114891in}{2.242987in}}%
\pgfpathmoveto{\pgfqpoint{4.114891in}{2.240038in}}%
\pgfpathlineto{\pgfqpoint{4.114891in}{2.240038in}}%
\pgfpathlineto{\pgfqpoint{4.114891in}{2.242987in}}%
\pgfpathlineto{\pgfqpoint{4.119432in}{2.242987in}}%
\pgfpathlineto{\pgfqpoint{4.119432in}{2.240038in}}%
\pgfpathmoveto{\pgfqpoint{4.119432in}{2.234139in}}%
\pgfpathlineto{\pgfqpoint{4.119432in}{2.234139in}}%
\pgfpathlineto{\pgfqpoint{4.119432in}{2.237088in}}%
\pgfpathlineto{\pgfqpoint{4.123973in}{2.237088in}}%
\pgfpathlineto{\pgfqpoint{4.123973in}{2.234139in}}%
\pgfpathmoveto{\pgfqpoint{4.119432in}{2.237088in}}%
\pgfpathlineto{\pgfqpoint{4.119432in}{2.237088in}}%
\pgfpathlineto{\pgfqpoint{4.119432in}{2.240038in}}%
\pgfpathlineto{\pgfqpoint{4.123973in}{2.240038in}}%
\pgfpathlineto{\pgfqpoint{4.123973in}{2.237088in}}%
\pgfpathmoveto{\pgfqpoint{4.123973in}{2.234139in}}%
\pgfpathlineto{\pgfqpoint{4.123973in}{2.234139in}}%
\pgfpathlineto{\pgfqpoint{4.123973in}{2.237088in}}%
\pgfpathlineto{\pgfqpoint{4.128515in}{2.237088in}}%
\pgfpathlineto{\pgfqpoint{4.128515in}{2.234139in}}%
\pgfpathmoveto{\pgfqpoint{4.137597in}{2.216444in}}%
\pgfpathlineto{\pgfqpoint{4.137597in}{2.216444in}}%
\pgfpathlineto{\pgfqpoint{4.137597in}{2.219393in}}%
\pgfpathlineto{\pgfqpoint{4.142138in}{2.219393in}}%
\pgfpathlineto{\pgfqpoint{4.142138in}{2.216444in}}%
\pgfpathmoveto{\pgfqpoint{4.137597in}{2.219393in}}%
\pgfpathlineto{\pgfqpoint{4.137597in}{2.219393in}}%
\pgfpathlineto{\pgfqpoint{4.137597in}{2.222342in}}%
\pgfpathlineto{\pgfqpoint{4.142138in}{2.222342in}}%
\pgfpathlineto{\pgfqpoint{4.142138in}{2.219393in}}%
\pgfpathmoveto{\pgfqpoint{4.142138in}{2.216444in}}%
\pgfpathlineto{\pgfqpoint{4.142138in}{2.216444in}}%
\pgfpathlineto{\pgfqpoint{4.142138in}{2.219393in}}%
\pgfpathlineto{\pgfqpoint{4.146680in}{2.219393in}}%
\pgfpathlineto{\pgfqpoint{4.146680in}{2.216444in}}%
\pgfpathmoveto{\pgfqpoint{4.142138in}{2.219393in}}%
\pgfpathlineto{\pgfqpoint{4.142138in}{2.219393in}}%
\pgfpathlineto{\pgfqpoint{4.142138in}{2.222342in}}%
\pgfpathlineto{\pgfqpoint{4.146680in}{2.222342in}}%
\pgfpathlineto{\pgfqpoint{4.146680in}{2.219393in}}%
\pgfpathmoveto{\pgfqpoint{4.155762in}{2.204646in}}%
\pgfpathlineto{\pgfqpoint{4.155762in}{2.204646in}}%
\pgfpathlineto{\pgfqpoint{4.155762in}{2.207596in}}%
\pgfpathlineto{\pgfqpoint{4.160303in}{2.207596in}}%
\pgfpathlineto{\pgfqpoint{4.160303in}{2.204646in}}%
\pgfpathmoveto{\pgfqpoint{4.155762in}{2.207596in}}%
\pgfpathlineto{\pgfqpoint{4.155762in}{2.207596in}}%
\pgfpathlineto{\pgfqpoint{4.155762in}{2.210545in}}%
\pgfpathlineto{\pgfqpoint{4.160303in}{2.210545in}}%
\pgfpathlineto{\pgfqpoint{4.160303in}{2.207596in}}%
\pgfpathmoveto{\pgfqpoint{4.160303in}{2.204646in}}%
\pgfpathlineto{\pgfqpoint{4.160303in}{2.204646in}}%
\pgfpathlineto{\pgfqpoint{4.160303in}{2.207596in}}%
\pgfpathlineto{\pgfqpoint{4.164844in}{2.207596in}}%
\pgfpathlineto{\pgfqpoint{4.164844in}{2.204646in}}%
\pgfpathmoveto{\pgfqpoint{4.160303in}{2.207596in}}%
\pgfpathlineto{\pgfqpoint{4.160303in}{2.207596in}}%
\pgfpathlineto{\pgfqpoint{4.160303in}{2.210545in}}%
\pgfpathlineto{\pgfqpoint{4.164844in}{2.210545in}}%
\pgfpathlineto{\pgfqpoint{4.164844in}{2.207596in}}%
\pgfpathmoveto{\pgfqpoint{4.146680in}{2.210545in}}%
\pgfpathlineto{\pgfqpoint{4.146680in}{2.210545in}}%
\pgfpathlineto{\pgfqpoint{4.146680in}{2.213494in}}%
\pgfpathlineto{\pgfqpoint{4.151221in}{2.213494in}}%
\pgfpathlineto{\pgfqpoint{4.151221in}{2.210545in}}%
\pgfpathmoveto{\pgfqpoint{4.146680in}{2.213494in}}%
\pgfpathlineto{\pgfqpoint{4.146680in}{2.213494in}}%
\pgfpathlineto{\pgfqpoint{4.146680in}{2.216444in}}%
\pgfpathlineto{\pgfqpoint{4.151221in}{2.216444in}}%
\pgfpathlineto{\pgfqpoint{4.151221in}{2.213494in}}%
\pgfpathmoveto{\pgfqpoint{4.151221in}{2.210545in}}%
\pgfpathlineto{\pgfqpoint{4.151221in}{2.210545in}}%
\pgfpathlineto{\pgfqpoint{4.151221in}{2.213494in}}%
\pgfpathlineto{\pgfqpoint{4.155762in}{2.213494in}}%
\pgfpathlineto{\pgfqpoint{4.155762in}{2.210545in}}%
\pgfpathmoveto{\pgfqpoint{4.151221in}{2.213494in}}%
\pgfpathlineto{\pgfqpoint{4.151221in}{2.213494in}}%
\pgfpathlineto{\pgfqpoint{4.151221in}{2.216444in}}%
\pgfpathlineto{\pgfqpoint{4.155762in}{2.216444in}}%
\pgfpathlineto{\pgfqpoint{4.155762in}{2.213494in}}%
\pgfpathmoveto{\pgfqpoint{4.146680in}{2.216444in}}%
\pgfpathlineto{\pgfqpoint{4.146680in}{2.216444in}}%
\pgfpathlineto{\pgfqpoint{4.146680in}{2.219393in}}%
\pgfpathlineto{\pgfqpoint{4.151221in}{2.219393in}}%
\pgfpathlineto{\pgfqpoint{4.151221in}{2.216444in}}%
\pgfpathmoveto{\pgfqpoint{4.146680in}{2.219393in}}%
\pgfpathlineto{\pgfqpoint{4.146680in}{2.219393in}}%
\pgfpathlineto{\pgfqpoint{4.146680in}{2.222342in}}%
\pgfpathlineto{\pgfqpoint{4.151221in}{2.222342in}}%
\pgfpathlineto{\pgfqpoint{4.151221in}{2.219393in}}%
\pgfpathmoveto{\pgfqpoint{4.151221in}{2.216444in}}%
\pgfpathlineto{\pgfqpoint{4.151221in}{2.216444in}}%
\pgfpathlineto{\pgfqpoint{4.151221in}{2.219393in}}%
\pgfpathlineto{\pgfqpoint{4.155762in}{2.219393in}}%
\pgfpathlineto{\pgfqpoint{4.155762in}{2.216444in}}%
\pgfpathmoveto{\pgfqpoint{4.155762in}{2.210545in}}%
\pgfpathlineto{\pgfqpoint{4.155762in}{2.210545in}}%
\pgfpathlineto{\pgfqpoint{4.155762in}{2.213494in}}%
\pgfpathlineto{\pgfqpoint{4.160303in}{2.213494in}}%
\pgfpathlineto{\pgfqpoint{4.160303in}{2.210545in}}%
\pgfpathmoveto{\pgfqpoint{4.155762in}{2.213494in}}%
\pgfpathlineto{\pgfqpoint{4.155762in}{2.213494in}}%
\pgfpathlineto{\pgfqpoint{4.155762in}{2.216444in}}%
\pgfpathlineto{\pgfqpoint{4.160303in}{2.216444in}}%
\pgfpathlineto{\pgfqpoint{4.160303in}{2.213494in}}%
\pgfpathmoveto{\pgfqpoint{4.160303in}{2.210545in}}%
\pgfpathlineto{\pgfqpoint{4.160303in}{2.210545in}}%
\pgfpathlineto{\pgfqpoint{4.160303in}{2.213494in}}%
\pgfpathlineto{\pgfqpoint{4.164844in}{2.213494in}}%
\pgfpathlineto{\pgfqpoint{4.164844in}{2.210545in}}%
\pgfpathmoveto{\pgfqpoint{4.128515in}{2.222342in}}%
\pgfpathlineto{\pgfqpoint{4.128515in}{2.222342in}}%
\pgfpathlineto{\pgfqpoint{4.128515in}{2.225291in}}%
\pgfpathlineto{\pgfqpoint{4.133056in}{2.225291in}}%
\pgfpathlineto{\pgfqpoint{4.133056in}{2.222342in}}%
\pgfpathmoveto{\pgfqpoint{4.128515in}{2.225291in}}%
\pgfpathlineto{\pgfqpoint{4.128515in}{2.225291in}}%
\pgfpathlineto{\pgfqpoint{4.128515in}{2.228241in}}%
\pgfpathlineto{\pgfqpoint{4.133056in}{2.228241in}}%
\pgfpathlineto{\pgfqpoint{4.133056in}{2.225291in}}%
\pgfpathmoveto{\pgfqpoint{4.133056in}{2.222342in}}%
\pgfpathlineto{\pgfqpoint{4.133056in}{2.222342in}}%
\pgfpathlineto{\pgfqpoint{4.133056in}{2.225291in}}%
\pgfpathlineto{\pgfqpoint{4.137597in}{2.225291in}}%
\pgfpathlineto{\pgfqpoint{4.137597in}{2.222342in}}%
\pgfpathmoveto{\pgfqpoint{4.133056in}{2.225291in}}%
\pgfpathlineto{\pgfqpoint{4.133056in}{2.225291in}}%
\pgfpathlineto{\pgfqpoint{4.133056in}{2.228241in}}%
\pgfpathlineto{\pgfqpoint{4.137597in}{2.228241in}}%
\pgfpathlineto{\pgfqpoint{4.137597in}{2.225291in}}%
\pgfpathmoveto{\pgfqpoint{4.128515in}{2.228241in}}%
\pgfpathlineto{\pgfqpoint{4.128515in}{2.228241in}}%
\pgfpathlineto{\pgfqpoint{4.128515in}{2.231190in}}%
\pgfpathlineto{\pgfqpoint{4.133056in}{2.231190in}}%
\pgfpathlineto{\pgfqpoint{4.133056in}{2.228241in}}%
\pgfpathmoveto{\pgfqpoint{4.128515in}{2.231190in}}%
\pgfpathlineto{\pgfqpoint{4.128515in}{2.231190in}}%
\pgfpathlineto{\pgfqpoint{4.128515in}{2.234139in}}%
\pgfpathlineto{\pgfqpoint{4.133056in}{2.234139in}}%
\pgfpathlineto{\pgfqpoint{4.133056in}{2.231190in}}%
\pgfpathmoveto{\pgfqpoint{4.133056in}{2.228241in}}%
\pgfpathlineto{\pgfqpoint{4.133056in}{2.228241in}}%
\pgfpathlineto{\pgfqpoint{4.133056in}{2.231190in}}%
\pgfpathlineto{\pgfqpoint{4.137597in}{2.231190in}}%
\pgfpathlineto{\pgfqpoint{4.137597in}{2.228241in}}%
\pgfpathmoveto{\pgfqpoint{4.137597in}{2.222342in}}%
\pgfpathlineto{\pgfqpoint{4.137597in}{2.222342in}}%
\pgfpathlineto{\pgfqpoint{4.137597in}{2.225291in}}%
\pgfpathlineto{\pgfqpoint{4.142138in}{2.225291in}}%
\pgfpathlineto{\pgfqpoint{4.142138in}{2.222342in}}%
\pgfpathmoveto{\pgfqpoint{4.137597in}{2.225291in}}%
\pgfpathlineto{\pgfqpoint{4.137597in}{2.225291in}}%
\pgfpathlineto{\pgfqpoint{4.137597in}{2.228241in}}%
\pgfpathlineto{\pgfqpoint{4.142138in}{2.228241in}}%
\pgfpathlineto{\pgfqpoint{4.142138in}{2.225291in}}%
\pgfpathmoveto{\pgfqpoint{4.142138in}{2.222342in}}%
\pgfpathlineto{\pgfqpoint{4.142138in}{2.222342in}}%
\pgfpathlineto{\pgfqpoint{4.142138in}{2.225291in}}%
\pgfpathlineto{\pgfqpoint{4.146680in}{2.225291in}}%
\pgfpathlineto{\pgfqpoint{4.146680in}{2.222342in}}%
\pgfpathmoveto{\pgfqpoint{4.092185in}{2.245936in}}%
\pgfpathlineto{\pgfqpoint{4.092185in}{2.245936in}}%
\pgfpathlineto{\pgfqpoint{4.092185in}{2.248886in}}%
\pgfpathlineto{\pgfqpoint{4.096726in}{2.248886in}}%
\pgfpathlineto{\pgfqpoint{4.096726in}{2.245936in}}%
\pgfpathmoveto{\pgfqpoint{4.092185in}{2.248886in}}%
\pgfpathlineto{\pgfqpoint{4.092185in}{2.248886in}}%
\pgfpathlineto{\pgfqpoint{4.092185in}{2.251835in}}%
\pgfpathlineto{\pgfqpoint{4.096726in}{2.251835in}}%
\pgfpathlineto{\pgfqpoint{4.096726in}{2.248886in}}%
\pgfpathmoveto{\pgfqpoint{4.096726in}{2.245936in}}%
\pgfpathlineto{\pgfqpoint{4.096726in}{2.245936in}}%
\pgfpathlineto{\pgfqpoint{4.096726in}{2.248886in}}%
\pgfpathlineto{\pgfqpoint{4.101267in}{2.248886in}}%
\pgfpathlineto{\pgfqpoint{4.101267in}{2.245936in}}%
\pgfpathmoveto{\pgfqpoint{4.096726in}{2.248886in}}%
\pgfpathlineto{\pgfqpoint{4.096726in}{2.248886in}}%
\pgfpathlineto{\pgfqpoint{4.096726in}{2.251835in}}%
\pgfpathlineto{\pgfqpoint{4.101267in}{2.251835in}}%
\pgfpathlineto{\pgfqpoint{4.101267in}{2.248886in}}%
\pgfpathmoveto{\pgfqpoint{4.092185in}{2.251835in}}%
\pgfpathlineto{\pgfqpoint{4.092185in}{2.251835in}}%
\pgfpathlineto{\pgfqpoint{4.092185in}{2.254784in}}%
\pgfpathlineto{\pgfqpoint{4.096726in}{2.254784in}}%
\pgfpathlineto{\pgfqpoint{4.096726in}{2.251835in}}%
\pgfpathmoveto{\pgfqpoint{4.092185in}{2.254784in}}%
\pgfpathlineto{\pgfqpoint{4.092185in}{2.254784in}}%
\pgfpathlineto{\pgfqpoint{4.092185in}{2.257733in}}%
\pgfpathlineto{\pgfqpoint{4.096726in}{2.257733in}}%
\pgfpathlineto{\pgfqpoint{4.096726in}{2.254784in}}%
\pgfpathmoveto{\pgfqpoint{4.096726in}{2.251835in}}%
\pgfpathlineto{\pgfqpoint{4.096726in}{2.251835in}}%
\pgfpathlineto{\pgfqpoint{4.096726in}{2.254784in}}%
\pgfpathlineto{\pgfqpoint{4.101267in}{2.254784in}}%
\pgfpathlineto{\pgfqpoint{4.101267in}{2.251835in}}%
\pgfpathmoveto{\pgfqpoint{4.101267in}{2.245936in}}%
\pgfpathlineto{\pgfqpoint{4.101267in}{2.245936in}}%
\pgfpathlineto{\pgfqpoint{4.101267in}{2.248886in}}%
\pgfpathlineto{\pgfqpoint{4.105808in}{2.248886in}}%
\pgfpathlineto{\pgfqpoint{4.105808in}{2.245936in}}%
\pgfpathmoveto{\pgfqpoint{4.101267in}{2.248886in}}%
\pgfpathlineto{\pgfqpoint{4.101267in}{2.248886in}}%
\pgfpathlineto{\pgfqpoint{4.101267in}{2.251835in}}%
\pgfpathlineto{\pgfqpoint{4.105808in}{2.251835in}}%
\pgfpathlineto{\pgfqpoint{4.105808in}{2.248886in}}%
\pgfpathmoveto{\pgfqpoint{4.105808in}{2.245936in}}%
\pgfpathlineto{\pgfqpoint{4.105808in}{2.245936in}}%
\pgfpathlineto{\pgfqpoint{4.105808in}{2.248886in}}%
\pgfpathlineto{\pgfqpoint{4.110350in}{2.248886in}}%
\pgfpathlineto{\pgfqpoint{4.110350in}{2.245936in}}%
\pgfpathmoveto{\pgfqpoint{4.164844in}{2.198748in}}%
\pgfpathlineto{\pgfqpoint{4.164844in}{2.198748in}}%
\pgfpathlineto{\pgfqpoint{4.164844in}{2.201697in}}%
\pgfpathlineto{\pgfqpoint{4.169386in}{2.201697in}}%
\pgfpathlineto{\pgfqpoint{4.169386in}{2.198748in}}%
\pgfpathmoveto{\pgfqpoint{4.164844in}{2.201697in}}%
\pgfpathlineto{\pgfqpoint{4.164844in}{2.201697in}}%
\pgfpathlineto{\pgfqpoint{4.164844in}{2.204646in}}%
\pgfpathlineto{\pgfqpoint{4.169386in}{2.204646in}}%
\pgfpathlineto{\pgfqpoint{4.169386in}{2.201697in}}%
\pgfpathmoveto{\pgfqpoint{4.169386in}{2.198748in}}%
\pgfpathlineto{\pgfqpoint{4.169386in}{2.198748in}}%
\pgfpathlineto{\pgfqpoint{4.169386in}{2.201697in}}%
\pgfpathlineto{\pgfqpoint{4.173927in}{2.201697in}}%
\pgfpathlineto{\pgfqpoint{4.173927in}{2.198748in}}%
\pgfpathmoveto{\pgfqpoint{4.169386in}{2.201697in}}%
\pgfpathlineto{\pgfqpoint{4.169386in}{2.201697in}}%
\pgfpathlineto{\pgfqpoint{4.169386in}{2.204646in}}%
\pgfpathlineto{\pgfqpoint{4.173927in}{2.204646in}}%
\pgfpathlineto{\pgfqpoint{4.173927in}{2.201697in}}%
\pgfpathmoveto{\pgfqpoint{4.164844in}{2.204646in}}%
\pgfpathlineto{\pgfqpoint{4.164844in}{2.204646in}}%
\pgfpathlineto{\pgfqpoint{4.164844in}{2.207596in}}%
\pgfpathlineto{\pgfqpoint{4.169386in}{2.207596in}}%
\pgfpathlineto{\pgfqpoint{4.169386in}{2.204646in}}%
\pgfpathmoveto{\pgfqpoint{4.164844in}{2.207596in}}%
\pgfpathlineto{\pgfqpoint{4.164844in}{2.207596in}}%
\pgfpathlineto{\pgfqpoint{4.164844in}{2.210545in}}%
\pgfpathlineto{\pgfqpoint{4.169386in}{2.210545in}}%
\pgfpathlineto{\pgfqpoint{4.169386in}{2.207596in}}%
\pgfpathmoveto{\pgfqpoint{4.169386in}{2.204646in}}%
\pgfpathlineto{\pgfqpoint{4.169386in}{2.204646in}}%
\pgfpathlineto{\pgfqpoint{4.169386in}{2.207596in}}%
\pgfpathlineto{\pgfqpoint{4.173927in}{2.207596in}}%
\pgfpathlineto{\pgfqpoint{4.173927in}{2.204646in}}%
\pgfpathmoveto{\pgfqpoint{4.173927in}{2.198748in}}%
\pgfpathlineto{\pgfqpoint{4.173927in}{2.198748in}}%
\pgfpathlineto{\pgfqpoint{4.173927in}{2.201697in}}%
\pgfpathlineto{\pgfqpoint{4.178468in}{2.201697in}}%
\pgfpathlineto{\pgfqpoint{4.178468in}{2.198748in}}%
\pgfpathmoveto{\pgfqpoint{4.173927in}{2.201697in}}%
\pgfpathlineto{\pgfqpoint{4.173927in}{2.201697in}}%
\pgfpathlineto{\pgfqpoint{4.173927in}{2.204646in}}%
\pgfpathlineto{\pgfqpoint{4.178468in}{2.204646in}}%
\pgfpathlineto{\pgfqpoint{4.178468in}{2.201697in}}%
\pgfpathmoveto{\pgfqpoint{4.178468in}{2.198748in}}%
\pgfpathlineto{\pgfqpoint{4.178468in}{2.198748in}}%
\pgfpathlineto{\pgfqpoint{4.178468in}{2.201697in}}%
\pgfpathlineto{\pgfqpoint{4.183009in}{2.201697in}}%
\pgfpathlineto{\pgfqpoint{4.183009in}{2.198748in}}%
\pgfpathmoveto{\pgfqpoint{4.237504in}{2.009997in}}%
\pgfpathlineto{\pgfqpoint{4.237504in}{2.009997in}}%
\pgfpathlineto{\pgfqpoint{4.237504in}{2.012947in}}%
\pgfpathlineto{\pgfqpoint{4.242045in}{2.012947in}}%
\pgfpathlineto{\pgfqpoint{4.242045in}{2.009997in}}%
\pgfpathmoveto{\pgfqpoint{4.237504in}{2.012947in}}%
\pgfpathlineto{\pgfqpoint{4.237504in}{2.012947in}}%
\pgfpathlineto{\pgfqpoint{4.237504in}{2.015896in}}%
\pgfpathlineto{\pgfqpoint{4.242045in}{2.015896in}}%
\pgfpathlineto{\pgfqpoint{4.242045in}{2.012947in}}%
\pgfpathmoveto{\pgfqpoint{4.242045in}{2.009997in}}%
\pgfpathlineto{\pgfqpoint{4.242045in}{2.009997in}}%
\pgfpathlineto{\pgfqpoint{4.242045in}{2.012947in}}%
\pgfpathlineto{\pgfqpoint{4.246586in}{2.012947in}}%
\pgfpathlineto{\pgfqpoint{4.246586in}{2.009997in}}%
\pgfpathmoveto{\pgfqpoint{4.242045in}{2.012947in}}%
\pgfpathlineto{\pgfqpoint{4.242045in}{2.012947in}}%
\pgfpathlineto{\pgfqpoint{4.242045in}{2.015896in}}%
\pgfpathlineto{\pgfqpoint{4.246586in}{2.015896in}}%
\pgfpathlineto{\pgfqpoint{4.246586in}{2.012947in}}%
\pgfpathmoveto{\pgfqpoint{4.246586in}{2.009997in}}%
\pgfpathlineto{\pgfqpoint{4.246586in}{2.009997in}}%
\pgfpathlineto{\pgfqpoint{4.246586in}{2.012947in}}%
\pgfpathlineto{\pgfqpoint{4.251127in}{2.012947in}}%
\pgfpathlineto{\pgfqpoint{4.251127in}{2.009997in}}%
\pgfpathmoveto{\pgfqpoint{4.246586in}{2.012947in}}%
\pgfpathlineto{\pgfqpoint{4.246586in}{2.012947in}}%
\pgfpathlineto{\pgfqpoint{4.246586in}{2.015896in}}%
\pgfpathlineto{\pgfqpoint{4.251127in}{2.015896in}}%
\pgfpathlineto{\pgfqpoint{4.251127in}{2.012947in}}%
\pgfpathmoveto{\pgfqpoint{4.251127in}{2.009997in}}%
\pgfpathlineto{\pgfqpoint{4.251127in}{2.009997in}}%
\pgfpathlineto{\pgfqpoint{4.251127in}{2.012947in}}%
\pgfpathlineto{\pgfqpoint{4.255667in}{2.012947in}}%
\pgfpathlineto{\pgfqpoint{4.255667in}{2.009997in}}%
\pgfpathmoveto{\pgfqpoint{4.251127in}{2.012947in}}%
\pgfpathlineto{\pgfqpoint{4.251127in}{2.012947in}}%
\pgfpathlineto{\pgfqpoint{4.251127in}{2.015896in}}%
\pgfpathlineto{\pgfqpoint{4.255667in}{2.015896in}}%
\pgfpathlineto{\pgfqpoint{4.255667in}{2.012947in}}%
\pgfpathmoveto{\pgfqpoint{4.255667in}{2.009997in}}%
\pgfpathlineto{\pgfqpoint{4.255667in}{2.009997in}}%
\pgfpathlineto{\pgfqpoint{4.255667in}{2.012947in}}%
\pgfpathlineto{\pgfqpoint{4.260208in}{2.012947in}}%
\pgfpathlineto{\pgfqpoint{4.260208in}{2.009997in}}%
\pgfpathmoveto{\pgfqpoint{4.255667in}{2.012947in}}%
\pgfpathlineto{\pgfqpoint{4.255667in}{2.012947in}}%
\pgfpathlineto{\pgfqpoint{4.255667in}{2.015896in}}%
\pgfpathlineto{\pgfqpoint{4.260208in}{2.015896in}}%
\pgfpathlineto{\pgfqpoint{4.260208in}{2.012947in}}%
\pgfpathmoveto{\pgfqpoint{4.260208in}{2.009997in}}%
\pgfpathlineto{\pgfqpoint{4.260208in}{2.009997in}}%
\pgfpathlineto{\pgfqpoint{4.260208in}{2.012947in}}%
\pgfpathlineto{\pgfqpoint{4.264749in}{2.012947in}}%
\pgfpathlineto{\pgfqpoint{4.264749in}{2.009997in}}%
\pgfpathmoveto{\pgfqpoint{4.260208in}{2.012947in}}%
\pgfpathlineto{\pgfqpoint{4.260208in}{2.012947in}}%
\pgfpathlineto{\pgfqpoint{4.260208in}{2.015896in}}%
\pgfpathlineto{\pgfqpoint{4.264749in}{2.015896in}}%
\pgfpathlineto{\pgfqpoint{4.264749in}{2.012947in}}%
\pgfpathmoveto{\pgfqpoint{4.264749in}{2.009997in}}%
\pgfpathlineto{\pgfqpoint{4.264749in}{2.009997in}}%
\pgfpathlineto{\pgfqpoint{4.264749in}{2.012947in}}%
\pgfpathlineto{\pgfqpoint{4.269290in}{2.012947in}}%
\pgfpathlineto{\pgfqpoint{4.269290in}{2.009997in}}%
\pgfpathmoveto{\pgfqpoint{4.264749in}{2.012947in}}%
\pgfpathlineto{\pgfqpoint{4.264749in}{2.012947in}}%
\pgfpathlineto{\pgfqpoint{4.264749in}{2.015896in}}%
\pgfpathlineto{\pgfqpoint{4.269290in}{2.015896in}}%
\pgfpathlineto{\pgfqpoint{4.269290in}{2.012947in}}%
\pgfpathmoveto{\pgfqpoint{4.269290in}{2.009997in}}%
\pgfpathlineto{\pgfqpoint{4.269290in}{2.009997in}}%
\pgfpathlineto{\pgfqpoint{4.269290in}{2.012947in}}%
\pgfpathlineto{\pgfqpoint{4.273831in}{2.012947in}}%
\pgfpathlineto{\pgfqpoint{4.273831in}{2.009997in}}%
\pgfpathmoveto{\pgfqpoint{4.269290in}{2.012947in}}%
\pgfpathlineto{\pgfqpoint{4.269290in}{2.012947in}}%
\pgfpathlineto{\pgfqpoint{4.269290in}{2.015896in}}%
\pgfpathlineto{\pgfqpoint{4.273831in}{2.015896in}}%
\pgfpathlineto{\pgfqpoint{4.273831in}{2.012947in}}%
\pgfpathmoveto{\pgfqpoint{4.273831in}{2.009997in}}%
\pgfpathlineto{\pgfqpoint{4.273831in}{2.009997in}}%
\pgfpathlineto{\pgfqpoint{4.273831in}{2.012947in}}%
\pgfpathlineto{\pgfqpoint{4.278372in}{2.012947in}}%
\pgfpathlineto{\pgfqpoint{4.278372in}{2.009997in}}%
\pgfpathmoveto{\pgfqpoint{4.273831in}{2.012947in}}%
\pgfpathlineto{\pgfqpoint{4.273831in}{2.012947in}}%
\pgfpathlineto{\pgfqpoint{4.273831in}{2.015896in}}%
\pgfpathlineto{\pgfqpoint{4.278372in}{2.015896in}}%
\pgfpathlineto{\pgfqpoint{4.278372in}{2.012947in}}%
\pgfpathmoveto{\pgfqpoint{4.278372in}{2.009997in}}%
\pgfpathlineto{\pgfqpoint{4.278372in}{2.009997in}}%
\pgfpathlineto{\pgfqpoint{4.278372in}{2.012947in}}%
\pgfpathlineto{\pgfqpoint{4.282912in}{2.012947in}}%
\pgfpathlineto{\pgfqpoint{4.282912in}{2.009997in}}%
\pgfpathmoveto{\pgfqpoint{4.278372in}{2.012947in}}%
\pgfpathlineto{\pgfqpoint{4.278372in}{2.012947in}}%
\pgfpathlineto{\pgfqpoint{4.278372in}{2.015896in}}%
\pgfpathlineto{\pgfqpoint{4.282912in}{2.015896in}}%
\pgfpathlineto{\pgfqpoint{4.282912in}{2.012947in}}%
\pgfpathmoveto{\pgfqpoint{4.282912in}{2.009997in}}%
\pgfpathlineto{\pgfqpoint{4.282912in}{2.009997in}}%
\pgfpathlineto{\pgfqpoint{4.282912in}{2.012947in}}%
\pgfpathlineto{\pgfqpoint{4.287453in}{2.012947in}}%
\pgfpathlineto{\pgfqpoint{4.287453in}{2.009997in}}%
\pgfpathmoveto{\pgfqpoint{4.282912in}{2.012947in}}%
\pgfpathlineto{\pgfqpoint{4.282912in}{2.012947in}}%
\pgfpathlineto{\pgfqpoint{4.282912in}{2.015896in}}%
\pgfpathlineto{\pgfqpoint{4.287453in}{2.015896in}}%
\pgfpathlineto{\pgfqpoint{4.287453in}{2.012947in}}%
\pgfpathmoveto{\pgfqpoint{4.287453in}{2.009997in}}%
\pgfpathlineto{\pgfqpoint{4.287453in}{2.009997in}}%
\pgfpathlineto{\pgfqpoint{4.287453in}{2.012947in}}%
\pgfpathlineto{\pgfqpoint{4.291994in}{2.012947in}}%
\pgfpathlineto{\pgfqpoint{4.291994in}{2.009997in}}%
\pgfpathmoveto{\pgfqpoint{4.287453in}{2.012947in}}%
\pgfpathlineto{\pgfqpoint{4.287453in}{2.012947in}}%
\pgfpathlineto{\pgfqpoint{4.287453in}{2.015896in}}%
\pgfpathlineto{\pgfqpoint{4.291994in}{2.015896in}}%
\pgfpathlineto{\pgfqpoint{4.291994in}{2.012947in}}%
\pgfpathmoveto{\pgfqpoint{4.291994in}{2.009997in}}%
\pgfpathlineto{\pgfqpoint{4.291994in}{2.009997in}}%
\pgfpathlineto{\pgfqpoint{4.291994in}{2.012947in}}%
\pgfpathlineto{\pgfqpoint{4.296535in}{2.012947in}}%
\pgfpathlineto{\pgfqpoint{4.296535in}{2.009997in}}%
\pgfpathmoveto{\pgfqpoint{4.291994in}{2.012947in}}%
\pgfpathlineto{\pgfqpoint{4.291994in}{2.012947in}}%
\pgfpathlineto{\pgfqpoint{4.291994in}{2.015896in}}%
\pgfpathlineto{\pgfqpoint{4.296535in}{2.015896in}}%
\pgfpathlineto{\pgfqpoint{4.296535in}{2.012947in}}%
\pgfpathmoveto{\pgfqpoint{4.296535in}{2.009997in}}%
\pgfpathlineto{\pgfqpoint{4.296535in}{2.009997in}}%
\pgfpathlineto{\pgfqpoint{4.296535in}{2.012947in}}%
\pgfpathlineto{\pgfqpoint{4.301076in}{2.012947in}}%
\pgfpathlineto{\pgfqpoint{4.301076in}{2.009997in}}%
\pgfpathmoveto{\pgfqpoint{4.296535in}{2.012947in}}%
\pgfpathlineto{\pgfqpoint{4.296535in}{2.012947in}}%
\pgfpathlineto{\pgfqpoint{4.296535in}{2.015896in}}%
\pgfpathlineto{\pgfqpoint{4.301076in}{2.015896in}}%
\pgfpathlineto{\pgfqpoint{4.301076in}{2.012947in}}%
\pgfpathmoveto{\pgfqpoint{4.301076in}{2.009997in}}%
\pgfpathlineto{\pgfqpoint{4.301076in}{2.009997in}}%
\pgfpathlineto{\pgfqpoint{4.301076in}{2.012947in}}%
\pgfpathlineto{\pgfqpoint{4.305617in}{2.012947in}}%
\pgfpathlineto{\pgfqpoint{4.305617in}{2.009997in}}%
\pgfpathmoveto{\pgfqpoint{4.301076in}{2.012947in}}%
\pgfpathlineto{\pgfqpoint{4.301076in}{2.012947in}}%
\pgfpathlineto{\pgfqpoint{4.301076in}{2.015896in}}%
\pgfpathlineto{\pgfqpoint{4.305617in}{2.015896in}}%
\pgfpathlineto{\pgfqpoint{4.305617in}{2.012947in}}%
\pgfpathmoveto{\pgfqpoint{4.305617in}{2.009997in}}%
\pgfpathlineto{\pgfqpoint{4.305617in}{2.009997in}}%
\pgfpathlineto{\pgfqpoint{4.305617in}{2.012947in}}%
\pgfpathlineto{\pgfqpoint{4.310157in}{2.012947in}}%
\pgfpathlineto{\pgfqpoint{4.310157in}{2.009997in}}%
\pgfpathmoveto{\pgfqpoint{4.305617in}{2.012947in}}%
\pgfpathlineto{\pgfqpoint{4.305617in}{2.012947in}}%
\pgfpathlineto{\pgfqpoint{4.305617in}{2.015896in}}%
\pgfpathlineto{\pgfqpoint{4.310157in}{2.015896in}}%
\pgfpathlineto{\pgfqpoint{4.310157in}{2.012947in}}%
\pgfpathmoveto{\pgfqpoint{4.310157in}{2.009997in}}%
\pgfpathlineto{\pgfqpoint{4.310157in}{2.009997in}}%
\pgfpathlineto{\pgfqpoint{4.310157in}{2.012947in}}%
\pgfpathlineto{\pgfqpoint{4.314698in}{2.012947in}}%
\pgfpathlineto{\pgfqpoint{4.314698in}{2.009997in}}%
\pgfpathmoveto{\pgfqpoint{4.310157in}{2.012947in}}%
\pgfpathlineto{\pgfqpoint{4.310157in}{2.012947in}}%
\pgfpathlineto{\pgfqpoint{4.310157in}{2.015896in}}%
\pgfpathlineto{\pgfqpoint{4.314698in}{2.015896in}}%
\pgfpathlineto{\pgfqpoint{4.314698in}{2.012947in}}%
\pgfpathmoveto{\pgfqpoint{4.314698in}{2.009997in}}%
\pgfpathlineto{\pgfqpoint{4.314698in}{2.009997in}}%
\pgfpathlineto{\pgfqpoint{4.314698in}{2.012947in}}%
\pgfpathlineto{\pgfqpoint{4.319239in}{2.012947in}}%
\pgfpathlineto{\pgfqpoint{4.319239in}{2.009997in}}%
\pgfpathmoveto{\pgfqpoint{4.314698in}{2.012947in}}%
\pgfpathlineto{\pgfqpoint{4.314698in}{2.012947in}}%
\pgfpathlineto{\pgfqpoint{4.314698in}{2.015896in}}%
\pgfpathlineto{\pgfqpoint{4.319239in}{2.015896in}}%
\pgfpathlineto{\pgfqpoint{4.319239in}{2.012947in}}%
\pgfpathmoveto{\pgfqpoint{4.319239in}{2.009997in}}%
\pgfpathlineto{\pgfqpoint{4.319239in}{2.009997in}}%
\pgfpathlineto{\pgfqpoint{4.319239in}{2.012947in}}%
\pgfpathlineto{\pgfqpoint{4.323780in}{2.012947in}}%
\pgfpathlineto{\pgfqpoint{4.323780in}{2.009997in}}%
\pgfpathmoveto{\pgfqpoint{4.319239in}{2.012947in}}%
\pgfpathlineto{\pgfqpoint{4.319239in}{2.012947in}}%
\pgfpathlineto{\pgfqpoint{4.319239in}{2.015896in}}%
\pgfpathlineto{\pgfqpoint{4.323780in}{2.015896in}}%
\pgfpathlineto{\pgfqpoint{4.323780in}{2.012947in}}%
\pgfpathmoveto{\pgfqpoint{4.323780in}{2.009997in}}%
\pgfpathlineto{\pgfqpoint{4.323780in}{2.009997in}}%
\pgfpathlineto{\pgfqpoint{4.323780in}{2.012947in}}%
\pgfpathlineto{\pgfqpoint{4.328321in}{2.012947in}}%
\pgfpathlineto{\pgfqpoint{4.328321in}{2.009997in}}%
\pgfpathmoveto{\pgfqpoint{4.323780in}{2.012947in}}%
\pgfpathlineto{\pgfqpoint{4.323780in}{2.012947in}}%
\pgfpathlineto{\pgfqpoint{4.323780in}{2.015896in}}%
\pgfpathlineto{\pgfqpoint{4.328321in}{2.015896in}}%
\pgfpathlineto{\pgfqpoint{4.328321in}{2.012947in}}%
\pgfpathmoveto{\pgfqpoint{4.328321in}{2.009997in}}%
\pgfpathlineto{\pgfqpoint{4.328321in}{2.009997in}}%
\pgfpathlineto{\pgfqpoint{4.328321in}{2.012947in}}%
\pgfpathlineto{\pgfqpoint{4.332862in}{2.012947in}}%
\pgfpathlineto{\pgfqpoint{4.332862in}{2.009997in}}%
\pgfpathmoveto{\pgfqpoint{4.328321in}{2.012947in}}%
\pgfpathlineto{\pgfqpoint{4.328321in}{2.012947in}}%
\pgfpathlineto{\pgfqpoint{4.328321in}{2.015896in}}%
\pgfpathlineto{\pgfqpoint{4.332862in}{2.015896in}}%
\pgfpathlineto{\pgfqpoint{4.332862in}{2.012947in}}%
\pgfpathmoveto{\pgfqpoint{4.332862in}{2.009997in}}%
\pgfpathlineto{\pgfqpoint{4.332862in}{2.009997in}}%
\pgfpathlineto{\pgfqpoint{4.332862in}{2.012947in}}%
\pgfpathlineto{\pgfqpoint{4.337402in}{2.012947in}}%
\pgfpathlineto{\pgfqpoint{4.337402in}{2.009997in}}%
\pgfpathmoveto{\pgfqpoint{4.332862in}{2.012947in}}%
\pgfpathlineto{\pgfqpoint{4.332862in}{2.012947in}}%
\pgfpathlineto{\pgfqpoint{4.332862in}{2.015896in}}%
\pgfpathlineto{\pgfqpoint{4.337402in}{2.015896in}}%
\pgfpathlineto{\pgfqpoint{4.337402in}{2.012947in}}%
\pgfpathmoveto{\pgfqpoint{4.337402in}{2.009997in}}%
\pgfpathlineto{\pgfqpoint{4.337402in}{2.009997in}}%
\pgfpathlineto{\pgfqpoint{4.337402in}{2.012947in}}%
\pgfpathlineto{\pgfqpoint{4.341943in}{2.012947in}}%
\pgfpathlineto{\pgfqpoint{4.341943in}{2.009997in}}%
\pgfpathmoveto{\pgfqpoint{4.337402in}{2.012947in}}%
\pgfpathlineto{\pgfqpoint{4.337402in}{2.012947in}}%
\pgfpathlineto{\pgfqpoint{4.337402in}{2.015896in}}%
\pgfpathlineto{\pgfqpoint{4.341943in}{2.015896in}}%
\pgfpathlineto{\pgfqpoint{4.341943in}{2.012947in}}%
\pgfpathmoveto{\pgfqpoint{4.341943in}{2.009997in}}%
\pgfpathlineto{\pgfqpoint{4.341943in}{2.009997in}}%
\pgfpathlineto{\pgfqpoint{4.341943in}{2.012947in}}%
\pgfpathlineto{\pgfqpoint{4.346484in}{2.012947in}}%
\pgfpathlineto{\pgfqpoint{4.346484in}{2.009997in}}%
\pgfpathmoveto{\pgfqpoint{4.341943in}{2.012947in}}%
\pgfpathlineto{\pgfqpoint{4.341943in}{2.012947in}}%
\pgfpathlineto{\pgfqpoint{4.341943in}{2.015896in}}%
\pgfpathlineto{\pgfqpoint{4.346484in}{2.015896in}}%
\pgfpathlineto{\pgfqpoint{4.346484in}{2.012947in}}%
\pgfpathmoveto{\pgfqpoint{4.346484in}{2.009997in}}%
\pgfpathlineto{\pgfqpoint{4.346484in}{2.009997in}}%
\pgfpathlineto{\pgfqpoint{4.346484in}{2.012947in}}%
\pgfpathlineto{\pgfqpoint{4.351025in}{2.012947in}}%
\pgfpathlineto{\pgfqpoint{4.351025in}{2.009997in}}%
\pgfpathmoveto{\pgfqpoint{4.346484in}{2.012947in}}%
\pgfpathlineto{\pgfqpoint{4.346484in}{2.012947in}}%
\pgfpathlineto{\pgfqpoint{4.346484in}{2.015896in}}%
\pgfpathlineto{\pgfqpoint{4.351025in}{2.015896in}}%
\pgfpathlineto{\pgfqpoint{4.351025in}{2.012947in}}%
\pgfpathmoveto{\pgfqpoint{4.351025in}{2.009997in}}%
\pgfpathlineto{\pgfqpoint{4.351025in}{2.009997in}}%
\pgfpathlineto{\pgfqpoint{4.351025in}{2.012947in}}%
\pgfpathlineto{\pgfqpoint{4.355566in}{2.012947in}}%
\pgfpathlineto{\pgfqpoint{4.355566in}{2.009997in}}%
\pgfpathmoveto{\pgfqpoint{4.351025in}{2.012947in}}%
\pgfpathlineto{\pgfqpoint{4.351025in}{2.012947in}}%
\pgfpathlineto{\pgfqpoint{4.351025in}{2.015896in}}%
\pgfpathlineto{\pgfqpoint{4.355566in}{2.015896in}}%
\pgfpathlineto{\pgfqpoint{4.355566in}{2.012947in}}%
\pgfpathmoveto{\pgfqpoint{4.355566in}{2.009997in}}%
\pgfpathlineto{\pgfqpoint{4.355566in}{2.009997in}}%
\pgfpathlineto{\pgfqpoint{4.355566in}{2.012947in}}%
\pgfpathlineto{\pgfqpoint{4.360107in}{2.012947in}}%
\pgfpathlineto{\pgfqpoint{4.360107in}{2.009997in}}%
\pgfpathmoveto{\pgfqpoint{4.355566in}{2.012947in}}%
\pgfpathlineto{\pgfqpoint{4.355566in}{2.012947in}}%
\pgfpathlineto{\pgfqpoint{4.355566in}{2.015896in}}%
\pgfpathlineto{\pgfqpoint{4.360107in}{2.015896in}}%
\pgfpathlineto{\pgfqpoint{4.360107in}{2.012947in}}%
\pgfpathmoveto{\pgfqpoint{4.360107in}{2.009997in}}%
\pgfpathlineto{\pgfqpoint{4.360107in}{2.009997in}}%
\pgfpathlineto{\pgfqpoint{4.360107in}{2.012947in}}%
\pgfpathlineto{\pgfqpoint{4.364647in}{2.012947in}}%
\pgfpathlineto{\pgfqpoint{4.364647in}{2.009997in}}%
\pgfpathmoveto{\pgfqpoint{4.360107in}{2.012947in}}%
\pgfpathlineto{\pgfqpoint{4.360107in}{2.012947in}}%
\pgfpathlineto{\pgfqpoint{4.360107in}{2.015896in}}%
\pgfpathlineto{\pgfqpoint{4.364647in}{2.015896in}}%
\pgfpathlineto{\pgfqpoint{4.364647in}{2.012947in}}%
\pgfpathmoveto{\pgfqpoint{4.364647in}{2.009997in}}%
\pgfpathlineto{\pgfqpoint{4.364647in}{2.009997in}}%
\pgfpathlineto{\pgfqpoint{4.364647in}{2.012947in}}%
\pgfpathlineto{\pgfqpoint{4.369188in}{2.012947in}}%
\pgfpathlineto{\pgfqpoint{4.369188in}{2.009997in}}%
\pgfpathmoveto{\pgfqpoint{4.364647in}{2.012947in}}%
\pgfpathlineto{\pgfqpoint{4.364647in}{2.012947in}}%
\pgfpathlineto{\pgfqpoint{4.364647in}{2.015896in}}%
\pgfpathlineto{\pgfqpoint{4.369188in}{2.015896in}}%
\pgfpathlineto{\pgfqpoint{4.369188in}{2.012947in}}%
\pgfpathmoveto{\pgfqpoint{4.369188in}{2.009997in}}%
\pgfpathlineto{\pgfqpoint{4.369188in}{2.009997in}}%
\pgfpathlineto{\pgfqpoint{4.369188in}{2.012947in}}%
\pgfpathlineto{\pgfqpoint{4.373729in}{2.012947in}}%
\pgfpathlineto{\pgfqpoint{4.373729in}{2.009997in}}%
\pgfpathmoveto{\pgfqpoint{4.369188in}{2.012947in}}%
\pgfpathlineto{\pgfqpoint{4.369188in}{2.012947in}}%
\pgfpathlineto{\pgfqpoint{4.369188in}{2.015896in}}%
\pgfpathlineto{\pgfqpoint{4.373729in}{2.015896in}}%
\pgfpathlineto{\pgfqpoint{4.373729in}{2.012947in}}%
\pgfpathmoveto{\pgfqpoint{4.373729in}{2.009997in}}%
\pgfpathlineto{\pgfqpoint{4.373729in}{2.009997in}}%
\pgfpathlineto{\pgfqpoint{4.373729in}{2.012947in}}%
\pgfpathlineto{\pgfqpoint{4.378270in}{2.012947in}}%
\pgfpathlineto{\pgfqpoint{4.378270in}{2.009997in}}%
\pgfpathmoveto{\pgfqpoint{4.373729in}{2.012947in}}%
\pgfpathlineto{\pgfqpoint{4.373729in}{2.012947in}}%
\pgfpathlineto{\pgfqpoint{4.373729in}{2.015896in}}%
\pgfpathlineto{\pgfqpoint{4.378270in}{2.015896in}}%
\pgfpathlineto{\pgfqpoint{4.378270in}{2.012947in}}%
\pgfpathmoveto{\pgfqpoint{4.378270in}{2.009997in}}%
\pgfpathlineto{\pgfqpoint{4.378270in}{2.009997in}}%
\pgfpathlineto{\pgfqpoint{4.378270in}{2.012947in}}%
\pgfpathlineto{\pgfqpoint{4.382811in}{2.012947in}}%
\pgfpathlineto{\pgfqpoint{4.382811in}{2.009997in}}%
\pgfpathmoveto{\pgfqpoint{4.378270in}{2.012947in}}%
\pgfpathlineto{\pgfqpoint{4.378270in}{2.012947in}}%
\pgfpathlineto{\pgfqpoint{4.378270in}{2.015896in}}%
\pgfpathlineto{\pgfqpoint{4.382811in}{2.015896in}}%
\pgfpathlineto{\pgfqpoint{4.382811in}{2.012947in}}%
\pgfpathmoveto{\pgfqpoint{4.319239in}{2.098479in}}%
\pgfpathlineto{\pgfqpoint{4.319239in}{2.098479in}}%
\pgfpathlineto{\pgfqpoint{4.319239in}{2.101428in}}%
\pgfpathlineto{\pgfqpoint{4.323780in}{2.101428in}}%
\pgfpathlineto{\pgfqpoint{4.323780in}{2.098479in}}%
\pgfpathmoveto{\pgfqpoint{4.319239in}{2.101428in}}%
\pgfpathlineto{\pgfqpoint{4.319239in}{2.101428in}}%
\pgfpathlineto{\pgfqpoint{4.319239in}{2.104378in}}%
\pgfpathlineto{\pgfqpoint{4.323780in}{2.104378in}}%
\pgfpathlineto{\pgfqpoint{4.323780in}{2.101428in}}%
\pgfpathmoveto{\pgfqpoint{4.323780in}{2.098479in}}%
\pgfpathlineto{\pgfqpoint{4.323780in}{2.098479in}}%
\pgfpathlineto{\pgfqpoint{4.323780in}{2.101428in}}%
\pgfpathlineto{\pgfqpoint{4.328321in}{2.101428in}}%
\pgfpathlineto{\pgfqpoint{4.328321in}{2.098479in}}%
\pgfpathmoveto{\pgfqpoint{4.323780in}{2.101428in}}%
\pgfpathlineto{\pgfqpoint{4.323780in}{2.101428in}}%
\pgfpathlineto{\pgfqpoint{4.323780in}{2.104378in}}%
\pgfpathlineto{\pgfqpoint{4.328321in}{2.104378in}}%
\pgfpathlineto{\pgfqpoint{4.328321in}{2.101428in}}%
\pgfpathmoveto{\pgfqpoint{4.337402in}{2.086681in}}%
\pgfpathlineto{\pgfqpoint{4.337402in}{2.086681in}}%
\pgfpathlineto{\pgfqpoint{4.337402in}{2.089631in}}%
\pgfpathlineto{\pgfqpoint{4.341943in}{2.089631in}}%
\pgfpathlineto{\pgfqpoint{4.341943in}{2.086681in}}%
\pgfpathmoveto{\pgfqpoint{4.337402in}{2.089631in}}%
\pgfpathlineto{\pgfqpoint{4.337402in}{2.089631in}}%
\pgfpathlineto{\pgfqpoint{4.337402in}{2.092580in}}%
\pgfpathlineto{\pgfqpoint{4.341943in}{2.092580in}}%
\pgfpathlineto{\pgfqpoint{4.341943in}{2.089631in}}%
\pgfpathmoveto{\pgfqpoint{4.341943in}{2.086681in}}%
\pgfpathlineto{\pgfqpoint{4.341943in}{2.086681in}}%
\pgfpathlineto{\pgfqpoint{4.341943in}{2.089631in}}%
\pgfpathlineto{\pgfqpoint{4.346484in}{2.089631in}}%
\pgfpathlineto{\pgfqpoint{4.346484in}{2.086681in}}%
\pgfpathmoveto{\pgfqpoint{4.341943in}{2.089631in}}%
\pgfpathlineto{\pgfqpoint{4.341943in}{2.089631in}}%
\pgfpathlineto{\pgfqpoint{4.341943in}{2.092580in}}%
\pgfpathlineto{\pgfqpoint{4.346484in}{2.092580in}}%
\pgfpathlineto{\pgfqpoint{4.346484in}{2.089631in}}%
\pgfpathmoveto{\pgfqpoint{4.328321in}{2.092580in}}%
\pgfpathlineto{\pgfqpoint{4.328321in}{2.092580in}}%
\pgfpathlineto{\pgfqpoint{4.328321in}{2.095529in}}%
\pgfpathlineto{\pgfqpoint{4.332862in}{2.095529in}}%
\pgfpathlineto{\pgfqpoint{4.332862in}{2.092580in}}%
\pgfpathmoveto{\pgfqpoint{4.328321in}{2.095529in}}%
\pgfpathlineto{\pgfqpoint{4.328321in}{2.095529in}}%
\pgfpathlineto{\pgfqpoint{4.328321in}{2.098479in}}%
\pgfpathlineto{\pgfqpoint{4.332862in}{2.098479in}}%
\pgfpathlineto{\pgfqpoint{4.332862in}{2.095529in}}%
\pgfpathmoveto{\pgfqpoint{4.332862in}{2.092580in}}%
\pgfpathlineto{\pgfqpoint{4.332862in}{2.092580in}}%
\pgfpathlineto{\pgfqpoint{4.332862in}{2.095529in}}%
\pgfpathlineto{\pgfqpoint{4.337402in}{2.095529in}}%
\pgfpathlineto{\pgfqpoint{4.337402in}{2.092580in}}%
\pgfpathmoveto{\pgfqpoint{4.332862in}{2.095529in}}%
\pgfpathlineto{\pgfqpoint{4.332862in}{2.095529in}}%
\pgfpathlineto{\pgfqpoint{4.332862in}{2.098479in}}%
\pgfpathlineto{\pgfqpoint{4.337402in}{2.098479in}}%
\pgfpathlineto{\pgfqpoint{4.337402in}{2.095529in}}%
\pgfpathmoveto{\pgfqpoint{4.328321in}{2.098479in}}%
\pgfpathlineto{\pgfqpoint{4.328321in}{2.098479in}}%
\pgfpathlineto{\pgfqpoint{4.328321in}{2.101428in}}%
\pgfpathlineto{\pgfqpoint{4.332862in}{2.101428in}}%
\pgfpathlineto{\pgfqpoint{4.332862in}{2.098479in}}%
\pgfpathmoveto{\pgfqpoint{4.328321in}{2.101428in}}%
\pgfpathlineto{\pgfqpoint{4.328321in}{2.101428in}}%
\pgfpathlineto{\pgfqpoint{4.328321in}{2.104378in}}%
\pgfpathlineto{\pgfqpoint{4.332862in}{2.104378in}}%
\pgfpathlineto{\pgfqpoint{4.332862in}{2.101428in}}%
\pgfpathmoveto{\pgfqpoint{4.332862in}{2.098479in}}%
\pgfpathlineto{\pgfqpoint{4.332862in}{2.098479in}}%
\pgfpathlineto{\pgfqpoint{4.332862in}{2.101428in}}%
\pgfpathlineto{\pgfqpoint{4.337402in}{2.101428in}}%
\pgfpathlineto{\pgfqpoint{4.337402in}{2.098479in}}%
\pgfpathmoveto{\pgfqpoint{4.337402in}{2.092580in}}%
\pgfpathlineto{\pgfqpoint{4.337402in}{2.092580in}}%
\pgfpathlineto{\pgfqpoint{4.337402in}{2.095529in}}%
\pgfpathlineto{\pgfqpoint{4.341943in}{2.095529in}}%
\pgfpathlineto{\pgfqpoint{4.341943in}{2.092580in}}%
\pgfpathmoveto{\pgfqpoint{4.337402in}{2.095529in}}%
\pgfpathlineto{\pgfqpoint{4.337402in}{2.095529in}}%
\pgfpathlineto{\pgfqpoint{4.337402in}{2.098479in}}%
\pgfpathlineto{\pgfqpoint{4.341943in}{2.098479in}}%
\pgfpathlineto{\pgfqpoint{4.341943in}{2.095529in}}%
\pgfpathmoveto{\pgfqpoint{4.341943in}{2.092580in}}%
\pgfpathlineto{\pgfqpoint{4.341943in}{2.092580in}}%
\pgfpathlineto{\pgfqpoint{4.341943in}{2.095529in}}%
\pgfpathlineto{\pgfqpoint{4.346484in}{2.095529in}}%
\pgfpathlineto{\pgfqpoint{4.346484in}{2.092580in}}%
\pgfpathmoveto{\pgfqpoint{4.355566in}{2.074884in}}%
\pgfpathlineto{\pgfqpoint{4.355566in}{2.074884in}}%
\pgfpathlineto{\pgfqpoint{4.355566in}{2.077833in}}%
\pgfpathlineto{\pgfqpoint{4.360107in}{2.077833in}}%
\pgfpathlineto{\pgfqpoint{4.360107in}{2.074884in}}%
\pgfpathmoveto{\pgfqpoint{4.355566in}{2.077833in}}%
\pgfpathlineto{\pgfqpoint{4.355566in}{2.077833in}}%
\pgfpathlineto{\pgfqpoint{4.355566in}{2.080783in}}%
\pgfpathlineto{\pgfqpoint{4.360107in}{2.080783in}}%
\pgfpathlineto{\pgfqpoint{4.360107in}{2.077833in}}%
\pgfpathmoveto{\pgfqpoint{4.360107in}{2.074884in}}%
\pgfpathlineto{\pgfqpoint{4.360107in}{2.074884in}}%
\pgfpathlineto{\pgfqpoint{4.360107in}{2.077833in}}%
\pgfpathlineto{\pgfqpoint{4.364647in}{2.077833in}}%
\pgfpathlineto{\pgfqpoint{4.364647in}{2.074884in}}%
\pgfpathmoveto{\pgfqpoint{4.360107in}{2.077833in}}%
\pgfpathlineto{\pgfqpoint{4.360107in}{2.077833in}}%
\pgfpathlineto{\pgfqpoint{4.360107in}{2.080783in}}%
\pgfpathlineto{\pgfqpoint{4.364647in}{2.080783in}}%
\pgfpathlineto{\pgfqpoint{4.364647in}{2.077833in}}%
\pgfpathmoveto{\pgfqpoint{4.373729in}{2.063086in}}%
\pgfpathlineto{\pgfqpoint{4.373729in}{2.063086in}}%
\pgfpathlineto{\pgfqpoint{4.373729in}{2.066036in}}%
\pgfpathlineto{\pgfqpoint{4.378270in}{2.066036in}}%
\pgfpathlineto{\pgfqpoint{4.378270in}{2.063086in}}%
\pgfpathmoveto{\pgfqpoint{4.373729in}{2.066036in}}%
\pgfpathlineto{\pgfqpoint{4.373729in}{2.066036in}}%
\pgfpathlineto{\pgfqpoint{4.373729in}{2.068985in}}%
\pgfpathlineto{\pgfqpoint{4.378270in}{2.068985in}}%
\pgfpathlineto{\pgfqpoint{4.378270in}{2.066036in}}%
\pgfpathmoveto{\pgfqpoint{4.378270in}{2.063086in}}%
\pgfpathlineto{\pgfqpoint{4.378270in}{2.063086in}}%
\pgfpathlineto{\pgfqpoint{4.378270in}{2.066036in}}%
\pgfpathlineto{\pgfqpoint{4.382811in}{2.066036in}}%
\pgfpathlineto{\pgfqpoint{4.382811in}{2.063086in}}%
\pgfpathmoveto{\pgfqpoint{4.378270in}{2.066036in}}%
\pgfpathlineto{\pgfqpoint{4.378270in}{2.066036in}}%
\pgfpathlineto{\pgfqpoint{4.378270in}{2.068985in}}%
\pgfpathlineto{\pgfqpoint{4.382811in}{2.068985in}}%
\pgfpathlineto{\pgfqpoint{4.382811in}{2.066036in}}%
\pgfpathmoveto{\pgfqpoint{4.364647in}{2.068985in}}%
\pgfpathlineto{\pgfqpoint{4.364647in}{2.068985in}}%
\pgfpathlineto{\pgfqpoint{4.364647in}{2.071934in}}%
\pgfpathlineto{\pgfqpoint{4.369188in}{2.071934in}}%
\pgfpathlineto{\pgfqpoint{4.369188in}{2.068985in}}%
\pgfpathmoveto{\pgfqpoint{4.364647in}{2.071934in}}%
\pgfpathlineto{\pgfqpoint{4.364647in}{2.071934in}}%
\pgfpathlineto{\pgfqpoint{4.364647in}{2.074884in}}%
\pgfpathlineto{\pgfqpoint{4.369188in}{2.074884in}}%
\pgfpathlineto{\pgfqpoint{4.369188in}{2.071934in}}%
\pgfpathmoveto{\pgfqpoint{4.369188in}{2.068985in}}%
\pgfpathlineto{\pgfqpoint{4.369188in}{2.068985in}}%
\pgfpathlineto{\pgfqpoint{4.369188in}{2.071934in}}%
\pgfpathlineto{\pgfqpoint{4.373729in}{2.071934in}}%
\pgfpathlineto{\pgfqpoint{4.373729in}{2.068985in}}%
\pgfpathmoveto{\pgfqpoint{4.369188in}{2.071934in}}%
\pgfpathlineto{\pgfqpoint{4.369188in}{2.071934in}}%
\pgfpathlineto{\pgfqpoint{4.369188in}{2.074884in}}%
\pgfpathlineto{\pgfqpoint{4.373729in}{2.074884in}}%
\pgfpathlineto{\pgfqpoint{4.373729in}{2.071934in}}%
\pgfpathmoveto{\pgfqpoint{4.364647in}{2.074884in}}%
\pgfpathlineto{\pgfqpoint{4.364647in}{2.074884in}}%
\pgfpathlineto{\pgfqpoint{4.364647in}{2.077833in}}%
\pgfpathlineto{\pgfqpoint{4.369188in}{2.077833in}}%
\pgfpathlineto{\pgfqpoint{4.369188in}{2.074884in}}%
\pgfpathmoveto{\pgfqpoint{4.364647in}{2.077833in}}%
\pgfpathlineto{\pgfqpoint{4.364647in}{2.077833in}}%
\pgfpathlineto{\pgfqpoint{4.364647in}{2.080783in}}%
\pgfpathlineto{\pgfqpoint{4.369188in}{2.080783in}}%
\pgfpathlineto{\pgfqpoint{4.369188in}{2.077833in}}%
\pgfpathmoveto{\pgfqpoint{4.369188in}{2.074884in}}%
\pgfpathlineto{\pgfqpoint{4.369188in}{2.074884in}}%
\pgfpathlineto{\pgfqpoint{4.369188in}{2.077833in}}%
\pgfpathlineto{\pgfqpoint{4.373729in}{2.077833in}}%
\pgfpathlineto{\pgfqpoint{4.373729in}{2.074884in}}%
\pgfpathmoveto{\pgfqpoint{4.373729in}{2.068985in}}%
\pgfpathlineto{\pgfqpoint{4.373729in}{2.068985in}}%
\pgfpathlineto{\pgfqpoint{4.373729in}{2.071934in}}%
\pgfpathlineto{\pgfqpoint{4.378270in}{2.071934in}}%
\pgfpathlineto{\pgfqpoint{4.378270in}{2.068985in}}%
\pgfpathmoveto{\pgfqpoint{4.373729in}{2.071934in}}%
\pgfpathlineto{\pgfqpoint{4.373729in}{2.071934in}}%
\pgfpathlineto{\pgfqpoint{4.373729in}{2.074884in}}%
\pgfpathlineto{\pgfqpoint{4.378270in}{2.074884in}}%
\pgfpathlineto{\pgfqpoint{4.378270in}{2.071934in}}%
\pgfpathmoveto{\pgfqpoint{4.378270in}{2.068985in}}%
\pgfpathlineto{\pgfqpoint{4.378270in}{2.068985in}}%
\pgfpathlineto{\pgfqpoint{4.378270in}{2.071934in}}%
\pgfpathlineto{\pgfqpoint{4.382811in}{2.071934in}}%
\pgfpathlineto{\pgfqpoint{4.382811in}{2.068985in}}%
\pgfpathmoveto{\pgfqpoint{4.346484in}{2.080783in}}%
\pgfpathlineto{\pgfqpoint{4.346484in}{2.080783in}}%
\pgfpathlineto{\pgfqpoint{4.346484in}{2.083732in}}%
\pgfpathlineto{\pgfqpoint{4.351025in}{2.083732in}}%
\pgfpathlineto{\pgfqpoint{4.351025in}{2.080783in}}%
\pgfpathmoveto{\pgfqpoint{4.346484in}{2.083732in}}%
\pgfpathlineto{\pgfqpoint{4.346484in}{2.083732in}}%
\pgfpathlineto{\pgfqpoint{4.346484in}{2.086681in}}%
\pgfpathlineto{\pgfqpoint{4.351025in}{2.086681in}}%
\pgfpathlineto{\pgfqpoint{4.351025in}{2.083732in}}%
\pgfpathmoveto{\pgfqpoint{4.351025in}{2.080783in}}%
\pgfpathlineto{\pgfqpoint{4.351025in}{2.080783in}}%
\pgfpathlineto{\pgfqpoint{4.351025in}{2.083732in}}%
\pgfpathlineto{\pgfqpoint{4.355566in}{2.083732in}}%
\pgfpathlineto{\pgfqpoint{4.355566in}{2.080783in}}%
\pgfpathmoveto{\pgfqpoint{4.351025in}{2.083732in}}%
\pgfpathlineto{\pgfqpoint{4.351025in}{2.083732in}}%
\pgfpathlineto{\pgfqpoint{4.351025in}{2.086681in}}%
\pgfpathlineto{\pgfqpoint{4.355566in}{2.086681in}}%
\pgfpathlineto{\pgfqpoint{4.355566in}{2.083732in}}%
\pgfpathmoveto{\pgfqpoint{4.346484in}{2.086681in}}%
\pgfpathlineto{\pgfqpoint{4.346484in}{2.086681in}}%
\pgfpathlineto{\pgfqpoint{4.346484in}{2.089631in}}%
\pgfpathlineto{\pgfqpoint{4.351025in}{2.089631in}}%
\pgfpathlineto{\pgfqpoint{4.351025in}{2.086681in}}%
\pgfpathmoveto{\pgfqpoint{4.346484in}{2.089631in}}%
\pgfpathlineto{\pgfqpoint{4.346484in}{2.089631in}}%
\pgfpathlineto{\pgfqpoint{4.346484in}{2.092580in}}%
\pgfpathlineto{\pgfqpoint{4.351025in}{2.092580in}}%
\pgfpathlineto{\pgfqpoint{4.351025in}{2.089631in}}%
\pgfpathmoveto{\pgfqpoint{4.351025in}{2.086681in}}%
\pgfpathlineto{\pgfqpoint{4.351025in}{2.086681in}}%
\pgfpathlineto{\pgfqpoint{4.351025in}{2.089631in}}%
\pgfpathlineto{\pgfqpoint{4.355566in}{2.089631in}}%
\pgfpathlineto{\pgfqpoint{4.355566in}{2.086681in}}%
\pgfpathmoveto{\pgfqpoint{4.355566in}{2.080783in}}%
\pgfpathlineto{\pgfqpoint{4.355566in}{2.080783in}}%
\pgfpathlineto{\pgfqpoint{4.355566in}{2.083732in}}%
\pgfpathlineto{\pgfqpoint{4.360107in}{2.083732in}}%
\pgfpathlineto{\pgfqpoint{4.360107in}{2.080783in}}%
\pgfpathmoveto{\pgfqpoint{4.355566in}{2.083732in}}%
\pgfpathlineto{\pgfqpoint{4.355566in}{2.083732in}}%
\pgfpathlineto{\pgfqpoint{4.355566in}{2.086681in}}%
\pgfpathlineto{\pgfqpoint{4.360107in}{2.086681in}}%
\pgfpathlineto{\pgfqpoint{4.360107in}{2.083732in}}%
\pgfpathmoveto{\pgfqpoint{4.360107in}{2.080783in}}%
\pgfpathlineto{\pgfqpoint{4.360107in}{2.080783in}}%
\pgfpathlineto{\pgfqpoint{4.360107in}{2.083732in}}%
\pgfpathlineto{\pgfqpoint{4.364647in}{2.083732in}}%
\pgfpathlineto{\pgfqpoint{4.364647in}{2.080783in}}%
\pgfpathmoveto{\pgfqpoint{4.246586in}{2.145665in}}%
\pgfpathlineto{\pgfqpoint{4.246586in}{2.145665in}}%
\pgfpathlineto{\pgfqpoint{4.246586in}{2.148614in}}%
\pgfpathlineto{\pgfqpoint{4.251127in}{2.148614in}}%
\pgfpathlineto{\pgfqpoint{4.251127in}{2.145665in}}%
\pgfpathmoveto{\pgfqpoint{4.246586in}{2.148614in}}%
\pgfpathlineto{\pgfqpoint{4.246586in}{2.148614in}}%
\pgfpathlineto{\pgfqpoint{4.246586in}{2.151563in}}%
\pgfpathlineto{\pgfqpoint{4.251127in}{2.151563in}}%
\pgfpathlineto{\pgfqpoint{4.251127in}{2.148614in}}%
\pgfpathmoveto{\pgfqpoint{4.251127in}{2.145665in}}%
\pgfpathlineto{\pgfqpoint{4.251127in}{2.145665in}}%
\pgfpathlineto{\pgfqpoint{4.251127in}{2.148614in}}%
\pgfpathlineto{\pgfqpoint{4.255667in}{2.148614in}}%
\pgfpathlineto{\pgfqpoint{4.255667in}{2.145665in}}%
\pgfpathmoveto{\pgfqpoint{4.251127in}{2.148614in}}%
\pgfpathlineto{\pgfqpoint{4.251127in}{2.148614in}}%
\pgfpathlineto{\pgfqpoint{4.251127in}{2.151563in}}%
\pgfpathlineto{\pgfqpoint{4.255667in}{2.151563in}}%
\pgfpathlineto{\pgfqpoint{4.255667in}{2.148614in}}%
\pgfpathmoveto{\pgfqpoint{4.264749in}{2.133868in}}%
\pgfpathlineto{\pgfqpoint{4.264749in}{2.133868in}}%
\pgfpathlineto{\pgfqpoint{4.264749in}{2.136817in}}%
\pgfpathlineto{\pgfqpoint{4.269290in}{2.136817in}}%
\pgfpathlineto{\pgfqpoint{4.269290in}{2.133868in}}%
\pgfpathmoveto{\pgfqpoint{4.264749in}{2.136817in}}%
\pgfpathlineto{\pgfqpoint{4.264749in}{2.136817in}}%
\pgfpathlineto{\pgfqpoint{4.264749in}{2.139766in}}%
\pgfpathlineto{\pgfqpoint{4.269290in}{2.139766in}}%
\pgfpathlineto{\pgfqpoint{4.269290in}{2.136817in}}%
\pgfpathmoveto{\pgfqpoint{4.269290in}{2.133868in}}%
\pgfpathlineto{\pgfqpoint{4.269290in}{2.133868in}}%
\pgfpathlineto{\pgfqpoint{4.269290in}{2.136817in}}%
\pgfpathlineto{\pgfqpoint{4.273831in}{2.136817in}}%
\pgfpathlineto{\pgfqpoint{4.273831in}{2.133868in}}%
\pgfpathmoveto{\pgfqpoint{4.269290in}{2.136817in}}%
\pgfpathlineto{\pgfqpoint{4.269290in}{2.136817in}}%
\pgfpathlineto{\pgfqpoint{4.269290in}{2.139766in}}%
\pgfpathlineto{\pgfqpoint{4.273831in}{2.139766in}}%
\pgfpathlineto{\pgfqpoint{4.273831in}{2.136817in}}%
\pgfpathmoveto{\pgfqpoint{4.255667in}{2.139766in}}%
\pgfpathlineto{\pgfqpoint{4.255667in}{2.139766in}}%
\pgfpathlineto{\pgfqpoint{4.255667in}{2.142715in}}%
\pgfpathlineto{\pgfqpoint{4.260208in}{2.142715in}}%
\pgfpathlineto{\pgfqpoint{4.260208in}{2.139766in}}%
\pgfpathmoveto{\pgfqpoint{4.255667in}{2.142715in}}%
\pgfpathlineto{\pgfqpoint{4.255667in}{2.142715in}}%
\pgfpathlineto{\pgfqpoint{4.255667in}{2.145665in}}%
\pgfpathlineto{\pgfqpoint{4.260208in}{2.145665in}}%
\pgfpathlineto{\pgfqpoint{4.260208in}{2.142715in}}%
\pgfpathmoveto{\pgfqpoint{4.260208in}{2.139766in}}%
\pgfpathlineto{\pgfqpoint{4.260208in}{2.139766in}}%
\pgfpathlineto{\pgfqpoint{4.260208in}{2.142715in}}%
\pgfpathlineto{\pgfqpoint{4.264749in}{2.142715in}}%
\pgfpathlineto{\pgfqpoint{4.264749in}{2.139766in}}%
\pgfpathmoveto{\pgfqpoint{4.260208in}{2.142715in}}%
\pgfpathlineto{\pgfqpoint{4.260208in}{2.142715in}}%
\pgfpathlineto{\pgfqpoint{4.260208in}{2.145665in}}%
\pgfpathlineto{\pgfqpoint{4.264749in}{2.145665in}}%
\pgfpathlineto{\pgfqpoint{4.264749in}{2.142715in}}%
\pgfpathmoveto{\pgfqpoint{4.255667in}{2.145665in}}%
\pgfpathlineto{\pgfqpoint{4.255667in}{2.145665in}}%
\pgfpathlineto{\pgfqpoint{4.255667in}{2.148614in}}%
\pgfpathlineto{\pgfqpoint{4.260208in}{2.148614in}}%
\pgfpathlineto{\pgfqpoint{4.260208in}{2.145665in}}%
\pgfpathmoveto{\pgfqpoint{4.255667in}{2.148614in}}%
\pgfpathlineto{\pgfqpoint{4.255667in}{2.148614in}}%
\pgfpathlineto{\pgfqpoint{4.255667in}{2.151563in}}%
\pgfpathlineto{\pgfqpoint{4.260208in}{2.151563in}}%
\pgfpathlineto{\pgfqpoint{4.260208in}{2.148614in}}%
\pgfpathmoveto{\pgfqpoint{4.260208in}{2.145665in}}%
\pgfpathlineto{\pgfqpoint{4.260208in}{2.145665in}}%
\pgfpathlineto{\pgfqpoint{4.260208in}{2.148614in}}%
\pgfpathlineto{\pgfqpoint{4.264749in}{2.148614in}}%
\pgfpathlineto{\pgfqpoint{4.264749in}{2.145665in}}%
\pgfpathmoveto{\pgfqpoint{4.264749in}{2.139766in}}%
\pgfpathlineto{\pgfqpoint{4.264749in}{2.139766in}}%
\pgfpathlineto{\pgfqpoint{4.264749in}{2.142715in}}%
\pgfpathlineto{\pgfqpoint{4.269290in}{2.142715in}}%
\pgfpathlineto{\pgfqpoint{4.269290in}{2.139766in}}%
\pgfpathmoveto{\pgfqpoint{4.264749in}{2.142715in}}%
\pgfpathlineto{\pgfqpoint{4.264749in}{2.142715in}}%
\pgfpathlineto{\pgfqpoint{4.264749in}{2.145665in}}%
\pgfpathlineto{\pgfqpoint{4.269290in}{2.145665in}}%
\pgfpathlineto{\pgfqpoint{4.269290in}{2.142715in}}%
\pgfpathmoveto{\pgfqpoint{4.269290in}{2.139766in}}%
\pgfpathlineto{\pgfqpoint{4.269290in}{2.139766in}}%
\pgfpathlineto{\pgfqpoint{4.269290in}{2.142715in}}%
\pgfpathlineto{\pgfqpoint{4.273831in}{2.142715in}}%
\pgfpathlineto{\pgfqpoint{4.273831in}{2.139766in}}%
\pgfpathmoveto{\pgfqpoint{4.282912in}{2.122072in}}%
\pgfpathlineto{\pgfqpoint{4.282912in}{2.122072in}}%
\pgfpathlineto{\pgfqpoint{4.282912in}{2.125021in}}%
\pgfpathlineto{\pgfqpoint{4.287453in}{2.125021in}}%
\pgfpathlineto{\pgfqpoint{4.287453in}{2.122072in}}%
\pgfpathmoveto{\pgfqpoint{4.282912in}{2.125021in}}%
\pgfpathlineto{\pgfqpoint{4.282912in}{2.125021in}}%
\pgfpathlineto{\pgfqpoint{4.282912in}{2.127970in}}%
\pgfpathlineto{\pgfqpoint{4.287453in}{2.127970in}}%
\pgfpathlineto{\pgfqpoint{4.287453in}{2.125021in}}%
\pgfpathmoveto{\pgfqpoint{4.287453in}{2.122072in}}%
\pgfpathlineto{\pgfqpoint{4.287453in}{2.122072in}}%
\pgfpathlineto{\pgfqpoint{4.287453in}{2.125021in}}%
\pgfpathlineto{\pgfqpoint{4.291994in}{2.125021in}}%
\pgfpathlineto{\pgfqpoint{4.291994in}{2.122072in}}%
\pgfpathmoveto{\pgfqpoint{4.287453in}{2.125021in}}%
\pgfpathlineto{\pgfqpoint{4.287453in}{2.125021in}}%
\pgfpathlineto{\pgfqpoint{4.287453in}{2.127970in}}%
\pgfpathlineto{\pgfqpoint{4.291994in}{2.127970in}}%
\pgfpathlineto{\pgfqpoint{4.291994in}{2.125021in}}%
\pgfpathmoveto{\pgfqpoint{4.301076in}{2.110276in}}%
\pgfpathlineto{\pgfqpoint{4.301076in}{2.110276in}}%
\pgfpathlineto{\pgfqpoint{4.301076in}{2.113225in}}%
\pgfpathlineto{\pgfqpoint{4.305617in}{2.113225in}}%
\pgfpathlineto{\pgfqpoint{4.305617in}{2.110276in}}%
\pgfpathmoveto{\pgfqpoint{4.301076in}{2.113225in}}%
\pgfpathlineto{\pgfqpoint{4.301076in}{2.113225in}}%
\pgfpathlineto{\pgfqpoint{4.301076in}{2.116174in}}%
\pgfpathlineto{\pgfqpoint{4.305617in}{2.116174in}}%
\pgfpathlineto{\pgfqpoint{4.305617in}{2.113225in}}%
\pgfpathmoveto{\pgfqpoint{4.305617in}{2.110276in}}%
\pgfpathlineto{\pgfqpoint{4.305617in}{2.110276in}}%
\pgfpathlineto{\pgfqpoint{4.305617in}{2.113225in}}%
\pgfpathlineto{\pgfqpoint{4.310157in}{2.113225in}}%
\pgfpathlineto{\pgfqpoint{4.310157in}{2.110276in}}%
\pgfpathmoveto{\pgfqpoint{4.305617in}{2.113225in}}%
\pgfpathlineto{\pgfqpoint{4.305617in}{2.113225in}}%
\pgfpathlineto{\pgfqpoint{4.305617in}{2.116174in}}%
\pgfpathlineto{\pgfqpoint{4.310157in}{2.116174in}}%
\pgfpathlineto{\pgfqpoint{4.310157in}{2.113225in}}%
\pgfpathmoveto{\pgfqpoint{4.291994in}{2.116174in}}%
\pgfpathlineto{\pgfqpoint{4.291994in}{2.116174in}}%
\pgfpathlineto{\pgfqpoint{4.291994in}{2.119123in}}%
\pgfpathlineto{\pgfqpoint{4.296535in}{2.119123in}}%
\pgfpathlineto{\pgfqpoint{4.296535in}{2.116174in}}%
\pgfpathmoveto{\pgfqpoint{4.291994in}{2.119123in}}%
\pgfpathlineto{\pgfqpoint{4.291994in}{2.119123in}}%
\pgfpathlineto{\pgfqpoint{4.291994in}{2.122072in}}%
\pgfpathlineto{\pgfqpoint{4.296535in}{2.122072in}}%
\pgfpathlineto{\pgfqpoint{4.296535in}{2.119123in}}%
\pgfpathmoveto{\pgfqpoint{4.296535in}{2.116174in}}%
\pgfpathlineto{\pgfqpoint{4.296535in}{2.116174in}}%
\pgfpathlineto{\pgfqpoint{4.296535in}{2.119123in}}%
\pgfpathlineto{\pgfqpoint{4.301076in}{2.119123in}}%
\pgfpathlineto{\pgfqpoint{4.301076in}{2.116174in}}%
\pgfpathmoveto{\pgfqpoint{4.296535in}{2.119123in}}%
\pgfpathlineto{\pgfqpoint{4.296535in}{2.119123in}}%
\pgfpathlineto{\pgfqpoint{4.296535in}{2.122072in}}%
\pgfpathlineto{\pgfqpoint{4.301076in}{2.122072in}}%
\pgfpathlineto{\pgfqpoint{4.301076in}{2.119123in}}%
\pgfpathmoveto{\pgfqpoint{4.291994in}{2.122072in}}%
\pgfpathlineto{\pgfqpoint{4.291994in}{2.122072in}}%
\pgfpathlineto{\pgfqpoint{4.291994in}{2.125021in}}%
\pgfpathlineto{\pgfqpoint{4.296535in}{2.125021in}}%
\pgfpathlineto{\pgfqpoint{4.296535in}{2.122072in}}%
\pgfpathmoveto{\pgfqpoint{4.291994in}{2.125021in}}%
\pgfpathlineto{\pgfqpoint{4.291994in}{2.125021in}}%
\pgfpathlineto{\pgfqpoint{4.291994in}{2.127970in}}%
\pgfpathlineto{\pgfqpoint{4.296535in}{2.127970in}}%
\pgfpathlineto{\pgfqpoint{4.296535in}{2.125021in}}%
\pgfpathmoveto{\pgfqpoint{4.296535in}{2.122072in}}%
\pgfpathlineto{\pgfqpoint{4.296535in}{2.122072in}}%
\pgfpathlineto{\pgfqpoint{4.296535in}{2.125021in}}%
\pgfpathlineto{\pgfqpoint{4.301076in}{2.125021in}}%
\pgfpathlineto{\pgfqpoint{4.301076in}{2.122072in}}%
\pgfpathmoveto{\pgfqpoint{4.301076in}{2.116174in}}%
\pgfpathlineto{\pgfqpoint{4.301076in}{2.116174in}}%
\pgfpathlineto{\pgfqpoint{4.301076in}{2.119123in}}%
\pgfpathlineto{\pgfqpoint{4.305617in}{2.119123in}}%
\pgfpathlineto{\pgfqpoint{4.305617in}{2.116174in}}%
\pgfpathmoveto{\pgfqpoint{4.301076in}{2.119123in}}%
\pgfpathlineto{\pgfqpoint{4.301076in}{2.119123in}}%
\pgfpathlineto{\pgfqpoint{4.301076in}{2.122072in}}%
\pgfpathlineto{\pgfqpoint{4.305617in}{2.122072in}}%
\pgfpathlineto{\pgfqpoint{4.305617in}{2.119123in}}%
\pgfpathmoveto{\pgfqpoint{4.305617in}{2.116174in}}%
\pgfpathlineto{\pgfqpoint{4.305617in}{2.116174in}}%
\pgfpathlineto{\pgfqpoint{4.305617in}{2.119123in}}%
\pgfpathlineto{\pgfqpoint{4.310157in}{2.119123in}}%
\pgfpathlineto{\pgfqpoint{4.310157in}{2.116174in}}%
\pgfpathmoveto{\pgfqpoint{4.273831in}{2.127970in}}%
\pgfpathlineto{\pgfqpoint{4.273831in}{2.127970in}}%
\pgfpathlineto{\pgfqpoint{4.273831in}{2.130919in}}%
\pgfpathlineto{\pgfqpoint{4.278372in}{2.130919in}}%
\pgfpathlineto{\pgfqpoint{4.278372in}{2.127970in}}%
\pgfpathmoveto{\pgfqpoint{4.273831in}{2.130919in}}%
\pgfpathlineto{\pgfqpoint{4.273831in}{2.130919in}}%
\pgfpathlineto{\pgfqpoint{4.273831in}{2.133868in}}%
\pgfpathlineto{\pgfqpoint{4.278372in}{2.133868in}}%
\pgfpathlineto{\pgfqpoint{4.278372in}{2.130919in}}%
\pgfpathmoveto{\pgfqpoint{4.278372in}{2.127970in}}%
\pgfpathlineto{\pgfqpoint{4.278372in}{2.127970in}}%
\pgfpathlineto{\pgfqpoint{4.278372in}{2.130919in}}%
\pgfpathlineto{\pgfqpoint{4.282912in}{2.130919in}}%
\pgfpathlineto{\pgfqpoint{4.282912in}{2.127970in}}%
\pgfpathmoveto{\pgfqpoint{4.278372in}{2.130919in}}%
\pgfpathlineto{\pgfqpoint{4.278372in}{2.130919in}}%
\pgfpathlineto{\pgfqpoint{4.278372in}{2.133868in}}%
\pgfpathlineto{\pgfqpoint{4.282912in}{2.133868in}}%
\pgfpathlineto{\pgfqpoint{4.282912in}{2.130919in}}%
\pgfpathmoveto{\pgfqpoint{4.273831in}{2.133868in}}%
\pgfpathlineto{\pgfqpoint{4.273831in}{2.133868in}}%
\pgfpathlineto{\pgfqpoint{4.273831in}{2.136817in}}%
\pgfpathlineto{\pgfqpoint{4.278372in}{2.136817in}}%
\pgfpathlineto{\pgfqpoint{4.278372in}{2.133868in}}%
\pgfpathmoveto{\pgfqpoint{4.273831in}{2.136817in}}%
\pgfpathlineto{\pgfqpoint{4.273831in}{2.136817in}}%
\pgfpathlineto{\pgfqpoint{4.273831in}{2.139766in}}%
\pgfpathlineto{\pgfqpoint{4.278372in}{2.139766in}}%
\pgfpathlineto{\pgfqpoint{4.278372in}{2.136817in}}%
\pgfpathmoveto{\pgfqpoint{4.278372in}{2.133868in}}%
\pgfpathlineto{\pgfqpoint{4.278372in}{2.133868in}}%
\pgfpathlineto{\pgfqpoint{4.278372in}{2.136817in}}%
\pgfpathlineto{\pgfqpoint{4.282912in}{2.136817in}}%
\pgfpathlineto{\pgfqpoint{4.282912in}{2.133868in}}%
\pgfpathmoveto{\pgfqpoint{4.282912in}{2.127970in}}%
\pgfpathlineto{\pgfqpoint{4.282912in}{2.127970in}}%
\pgfpathlineto{\pgfqpoint{4.282912in}{2.130919in}}%
\pgfpathlineto{\pgfqpoint{4.287453in}{2.130919in}}%
\pgfpathlineto{\pgfqpoint{4.287453in}{2.127970in}}%
\pgfpathmoveto{\pgfqpoint{4.282912in}{2.130919in}}%
\pgfpathlineto{\pgfqpoint{4.282912in}{2.130919in}}%
\pgfpathlineto{\pgfqpoint{4.282912in}{2.133868in}}%
\pgfpathlineto{\pgfqpoint{4.287453in}{2.133868in}}%
\pgfpathlineto{\pgfqpoint{4.287453in}{2.130919in}}%
\pgfpathmoveto{\pgfqpoint{4.287453in}{2.127970in}}%
\pgfpathlineto{\pgfqpoint{4.287453in}{2.127970in}}%
\pgfpathlineto{\pgfqpoint{4.287453in}{2.130919in}}%
\pgfpathlineto{\pgfqpoint{4.291994in}{2.130919in}}%
\pgfpathlineto{\pgfqpoint{4.291994in}{2.127970in}}%
\pgfpathmoveto{\pgfqpoint{4.237504in}{2.151563in}}%
\pgfpathlineto{\pgfqpoint{4.237504in}{2.151563in}}%
\pgfpathlineto{\pgfqpoint{4.237504in}{2.154512in}}%
\pgfpathlineto{\pgfqpoint{4.242045in}{2.154512in}}%
\pgfpathlineto{\pgfqpoint{4.242045in}{2.151563in}}%
\pgfpathmoveto{\pgfqpoint{4.237504in}{2.154512in}}%
\pgfpathlineto{\pgfqpoint{4.237504in}{2.154512in}}%
\pgfpathlineto{\pgfqpoint{4.237504in}{2.157461in}}%
\pgfpathlineto{\pgfqpoint{4.242045in}{2.157461in}}%
\pgfpathlineto{\pgfqpoint{4.242045in}{2.154512in}}%
\pgfpathmoveto{\pgfqpoint{4.242045in}{2.151563in}}%
\pgfpathlineto{\pgfqpoint{4.242045in}{2.151563in}}%
\pgfpathlineto{\pgfqpoint{4.242045in}{2.154512in}}%
\pgfpathlineto{\pgfqpoint{4.246586in}{2.154512in}}%
\pgfpathlineto{\pgfqpoint{4.246586in}{2.151563in}}%
\pgfpathmoveto{\pgfqpoint{4.242045in}{2.154512in}}%
\pgfpathlineto{\pgfqpoint{4.242045in}{2.154512in}}%
\pgfpathlineto{\pgfqpoint{4.242045in}{2.157461in}}%
\pgfpathlineto{\pgfqpoint{4.246586in}{2.157461in}}%
\pgfpathlineto{\pgfqpoint{4.246586in}{2.154512in}}%
\pgfpathmoveto{\pgfqpoint{4.237504in}{2.157461in}}%
\pgfpathlineto{\pgfqpoint{4.237504in}{2.157461in}}%
\pgfpathlineto{\pgfqpoint{4.237504in}{2.160410in}}%
\pgfpathlineto{\pgfqpoint{4.242045in}{2.160410in}}%
\pgfpathlineto{\pgfqpoint{4.242045in}{2.157461in}}%
\pgfpathmoveto{\pgfqpoint{4.237504in}{2.160410in}}%
\pgfpathlineto{\pgfqpoint{4.237504in}{2.160410in}}%
\pgfpathlineto{\pgfqpoint{4.237504in}{2.163359in}}%
\pgfpathlineto{\pgfqpoint{4.242045in}{2.163359in}}%
\pgfpathlineto{\pgfqpoint{4.242045in}{2.160410in}}%
\pgfpathmoveto{\pgfqpoint{4.242045in}{2.157461in}}%
\pgfpathlineto{\pgfqpoint{4.242045in}{2.157461in}}%
\pgfpathlineto{\pgfqpoint{4.242045in}{2.160410in}}%
\pgfpathlineto{\pgfqpoint{4.246586in}{2.160410in}}%
\pgfpathlineto{\pgfqpoint{4.246586in}{2.157461in}}%
\pgfpathmoveto{\pgfqpoint{4.246586in}{2.151563in}}%
\pgfpathlineto{\pgfqpoint{4.246586in}{2.151563in}}%
\pgfpathlineto{\pgfqpoint{4.246586in}{2.154512in}}%
\pgfpathlineto{\pgfqpoint{4.251127in}{2.154512in}}%
\pgfpathlineto{\pgfqpoint{4.251127in}{2.151563in}}%
\pgfpathmoveto{\pgfqpoint{4.246586in}{2.154512in}}%
\pgfpathlineto{\pgfqpoint{4.246586in}{2.154512in}}%
\pgfpathlineto{\pgfqpoint{4.246586in}{2.157461in}}%
\pgfpathlineto{\pgfqpoint{4.251127in}{2.157461in}}%
\pgfpathlineto{\pgfqpoint{4.251127in}{2.154512in}}%
\pgfpathmoveto{\pgfqpoint{4.251127in}{2.151563in}}%
\pgfpathlineto{\pgfqpoint{4.251127in}{2.151563in}}%
\pgfpathlineto{\pgfqpoint{4.251127in}{2.154512in}}%
\pgfpathlineto{\pgfqpoint{4.255667in}{2.154512in}}%
\pgfpathlineto{\pgfqpoint{4.255667in}{2.151563in}}%
\pgfpathmoveto{\pgfqpoint{4.310157in}{2.104378in}}%
\pgfpathlineto{\pgfqpoint{4.310157in}{2.104378in}}%
\pgfpathlineto{\pgfqpoint{4.310157in}{2.107327in}}%
\pgfpathlineto{\pgfqpoint{4.314698in}{2.107327in}}%
\pgfpathlineto{\pgfqpoint{4.314698in}{2.104378in}}%
\pgfpathmoveto{\pgfqpoint{4.310157in}{2.107327in}}%
\pgfpathlineto{\pgfqpoint{4.310157in}{2.107327in}}%
\pgfpathlineto{\pgfqpoint{4.310157in}{2.110276in}}%
\pgfpathlineto{\pgfqpoint{4.314698in}{2.110276in}}%
\pgfpathlineto{\pgfqpoint{4.314698in}{2.107327in}}%
\pgfpathmoveto{\pgfqpoint{4.314698in}{2.104378in}}%
\pgfpathlineto{\pgfqpoint{4.314698in}{2.104378in}}%
\pgfpathlineto{\pgfqpoint{4.314698in}{2.107327in}}%
\pgfpathlineto{\pgfqpoint{4.319239in}{2.107327in}}%
\pgfpathlineto{\pgfqpoint{4.319239in}{2.104378in}}%
\pgfpathmoveto{\pgfqpoint{4.314698in}{2.107327in}}%
\pgfpathlineto{\pgfqpoint{4.314698in}{2.107327in}}%
\pgfpathlineto{\pgfqpoint{4.314698in}{2.110276in}}%
\pgfpathlineto{\pgfqpoint{4.319239in}{2.110276in}}%
\pgfpathlineto{\pgfqpoint{4.319239in}{2.107327in}}%
\pgfpathmoveto{\pgfqpoint{4.310157in}{2.110276in}}%
\pgfpathlineto{\pgfqpoint{4.310157in}{2.110276in}}%
\pgfpathlineto{\pgfqpoint{4.310157in}{2.113225in}}%
\pgfpathlineto{\pgfqpoint{4.314698in}{2.113225in}}%
\pgfpathlineto{\pgfqpoint{4.314698in}{2.110276in}}%
\pgfpathmoveto{\pgfqpoint{4.310157in}{2.113225in}}%
\pgfpathlineto{\pgfqpoint{4.310157in}{2.113225in}}%
\pgfpathlineto{\pgfqpoint{4.310157in}{2.116174in}}%
\pgfpathlineto{\pgfqpoint{4.314698in}{2.116174in}}%
\pgfpathlineto{\pgfqpoint{4.314698in}{2.113225in}}%
\pgfpathmoveto{\pgfqpoint{4.314698in}{2.110276in}}%
\pgfpathlineto{\pgfqpoint{4.314698in}{2.110276in}}%
\pgfpathlineto{\pgfqpoint{4.314698in}{2.113225in}}%
\pgfpathlineto{\pgfqpoint{4.319239in}{2.113225in}}%
\pgfpathlineto{\pgfqpoint{4.319239in}{2.110276in}}%
\pgfpathmoveto{\pgfqpoint{4.319239in}{2.104378in}}%
\pgfpathlineto{\pgfqpoint{4.319239in}{2.104378in}}%
\pgfpathlineto{\pgfqpoint{4.319239in}{2.107327in}}%
\pgfpathlineto{\pgfqpoint{4.323780in}{2.107327in}}%
\pgfpathlineto{\pgfqpoint{4.323780in}{2.104378in}}%
\pgfpathmoveto{\pgfqpoint{4.319239in}{2.107327in}}%
\pgfpathlineto{\pgfqpoint{4.319239in}{2.107327in}}%
\pgfpathlineto{\pgfqpoint{4.319239in}{2.110276in}}%
\pgfpathlineto{\pgfqpoint{4.323780in}{2.110276in}}%
\pgfpathlineto{\pgfqpoint{4.323780in}{2.107327in}}%
\pgfpathmoveto{\pgfqpoint{4.323780in}{2.104378in}}%
\pgfpathlineto{\pgfqpoint{4.323780in}{2.104378in}}%
\pgfpathlineto{\pgfqpoint{4.323780in}{2.107327in}}%
\pgfpathlineto{\pgfqpoint{4.328321in}{2.107327in}}%
\pgfpathlineto{\pgfqpoint{4.328321in}{2.104378in}}%
\pgfpathmoveto{\pgfqpoint{4.382811in}{2.009997in}}%
\pgfpathlineto{\pgfqpoint{4.382811in}{2.009997in}}%
\pgfpathlineto{\pgfqpoint{4.382811in}{2.012947in}}%
\pgfpathlineto{\pgfqpoint{4.387352in}{2.012947in}}%
\pgfpathlineto{\pgfqpoint{4.387352in}{2.009997in}}%
\pgfpathmoveto{\pgfqpoint{4.382811in}{2.012947in}}%
\pgfpathlineto{\pgfqpoint{4.382811in}{2.012947in}}%
\pgfpathlineto{\pgfqpoint{4.382811in}{2.015896in}}%
\pgfpathlineto{\pgfqpoint{4.387352in}{2.015896in}}%
\pgfpathlineto{\pgfqpoint{4.387352in}{2.012947in}}%
\pgfpathmoveto{\pgfqpoint{4.387352in}{2.009997in}}%
\pgfpathlineto{\pgfqpoint{4.387352in}{2.009997in}}%
\pgfpathlineto{\pgfqpoint{4.387352in}{2.012947in}}%
\pgfpathlineto{\pgfqpoint{4.391893in}{2.012947in}}%
\pgfpathlineto{\pgfqpoint{4.391893in}{2.009997in}}%
\pgfpathmoveto{\pgfqpoint{4.387352in}{2.012947in}}%
\pgfpathlineto{\pgfqpoint{4.387352in}{2.012947in}}%
\pgfpathlineto{\pgfqpoint{4.387352in}{2.015896in}}%
\pgfpathlineto{\pgfqpoint{4.391893in}{2.015896in}}%
\pgfpathlineto{\pgfqpoint{4.391893in}{2.012947in}}%
\pgfpathmoveto{\pgfqpoint{4.391893in}{2.009997in}}%
\pgfpathlineto{\pgfqpoint{4.391893in}{2.009997in}}%
\pgfpathlineto{\pgfqpoint{4.391893in}{2.012947in}}%
\pgfpathlineto{\pgfqpoint{4.396434in}{2.012947in}}%
\pgfpathlineto{\pgfqpoint{4.396434in}{2.009997in}}%
\pgfpathmoveto{\pgfqpoint{4.391893in}{2.012947in}}%
\pgfpathlineto{\pgfqpoint{4.391893in}{2.012947in}}%
\pgfpathlineto{\pgfqpoint{4.391893in}{2.015896in}}%
\pgfpathlineto{\pgfqpoint{4.396434in}{2.015896in}}%
\pgfpathlineto{\pgfqpoint{4.396434in}{2.012947in}}%
\pgfpathmoveto{\pgfqpoint{4.396434in}{2.009997in}}%
\pgfpathlineto{\pgfqpoint{4.396434in}{2.009997in}}%
\pgfpathlineto{\pgfqpoint{4.396434in}{2.012947in}}%
\pgfpathlineto{\pgfqpoint{4.400975in}{2.012947in}}%
\pgfpathlineto{\pgfqpoint{4.400975in}{2.009997in}}%
\pgfpathmoveto{\pgfqpoint{4.396434in}{2.012947in}}%
\pgfpathlineto{\pgfqpoint{4.396434in}{2.012947in}}%
\pgfpathlineto{\pgfqpoint{4.396434in}{2.015896in}}%
\pgfpathlineto{\pgfqpoint{4.400975in}{2.015896in}}%
\pgfpathlineto{\pgfqpoint{4.400975in}{2.012947in}}%
\pgfpathmoveto{\pgfqpoint{4.400975in}{2.009997in}}%
\pgfpathlineto{\pgfqpoint{4.400975in}{2.009997in}}%
\pgfpathlineto{\pgfqpoint{4.400975in}{2.012947in}}%
\pgfpathlineto{\pgfqpoint{4.405517in}{2.012947in}}%
\pgfpathlineto{\pgfqpoint{4.405517in}{2.009997in}}%
\pgfpathmoveto{\pgfqpoint{4.400975in}{2.012947in}}%
\pgfpathlineto{\pgfqpoint{4.400975in}{2.012947in}}%
\pgfpathlineto{\pgfqpoint{4.400975in}{2.015896in}}%
\pgfpathlineto{\pgfqpoint{4.405517in}{2.015896in}}%
\pgfpathlineto{\pgfqpoint{4.405517in}{2.012947in}}%
\pgfpathmoveto{\pgfqpoint{4.405517in}{2.009997in}}%
\pgfpathlineto{\pgfqpoint{4.405517in}{2.009997in}}%
\pgfpathlineto{\pgfqpoint{4.405517in}{2.012947in}}%
\pgfpathlineto{\pgfqpoint{4.410058in}{2.012947in}}%
\pgfpathlineto{\pgfqpoint{4.410058in}{2.009997in}}%
\pgfpathmoveto{\pgfqpoint{4.405517in}{2.012947in}}%
\pgfpathlineto{\pgfqpoint{4.405517in}{2.012947in}}%
\pgfpathlineto{\pgfqpoint{4.405517in}{2.015896in}}%
\pgfpathlineto{\pgfqpoint{4.410058in}{2.015896in}}%
\pgfpathlineto{\pgfqpoint{4.410058in}{2.012947in}}%
\pgfpathmoveto{\pgfqpoint{4.410058in}{2.009997in}}%
\pgfpathlineto{\pgfqpoint{4.410058in}{2.009997in}}%
\pgfpathlineto{\pgfqpoint{4.410058in}{2.012947in}}%
\pgfpathlineto{\pgfqpoint{4.414599in}{2.012947in}}%
\pgfpathlineto{\pgfqpoint{4.414599in}{2.009997in}}%
\pgfpathmoveto{\pgfqpoint{4.410058in}{2.012947in}}%
\pgfpathlineto{\pgfqpoint{4.410058in}{2.012947in}}%
\pgfpathlineto{\pgfqpoint{4.410058in}{2.015896in}}%
\pgfpathlineto{\pgfqpoint{4.414599in}{2.015896in}}%
\pgfpathlineto{\pgfqpoint{4.414599in}{2.012947in}}%
\pgfpathmoveto{\pgfqpoint{4.414599in}{2.009997in}}%
\pgfpathlineto{\pgfqpoint{4.414599in}{2.009997in}}%
\pgfpathlineto{\pgfqpoint{4.414599in}{2.012947in}}%
\pgfpathlineto{\pgfqpoint{4.419140in}{2.012947in}}%
\pgfpathlineto{\pgfqpoint{4.419140in}{2.009997in}}%
\pgfpathmoveto{\pgfqpoint{4.414599in}{2.012947in}}%
\pgfpathlineto{\pgfqpoint{4.414599in}{2.012947in}}%
\pgfpathlineto{\pgfqpoint{4.414599in}{2.015896in}}%
\pgfpathlineto{\pgfqpoint{4.419140in}{2.015896in}}%
\pgfpathlineto{\pgfqpoint{4.419140in}{2.012947in}}%
\pgfpathmoveto{\pgfqpoint{4.391893in}{2.051289in}}%
\pgfpathlineto{\pgfqpoint{4.391893in}{2.051289in}}%
\pgfpathlineto{\pgfqpoint{4.391893in}{2.054238in}}%
\pgfpathlineto{\pgfqpoint{4.396434in}{2.054238in}}%
\pgfpathlineto{\pgfqpoint{4.396434in}{2.051289in}}%
\pgfpathmoveto{\pgfqpoint{4.391893in}{2.054238in}}%
\pgfpathlineto{\pgfqpoint{4.391893in}{2.054238in}}%
\pgfpathlineto{\pgfqpoint{4.391893in}{2.057187in}}%
\pgfpathlineto{\pgfqpoint{4.396434in}{2.057187in}}%
\pgfpathlineto{\pgfqpoint{4.396434in}{2.054238in}}%
\pgfpathmoveto{\pgfqpoint{4.396434in}{2.051289in}}%
\pgfpathlineto{\pgfqpoint{4.396434in}{2.051289in}}%
\pgfpathlineto{\pgfqpoint{4.396434in}{2.054238in}}%
\pgfpathlineto{\pgfqpoint{4.400975in}{2.054238in}}%
\pgfpathlineto{\pgfqpoint{4.400975in}{2.051289in}}%
\pgfpathmoveto{\pgfqpoint{4.396434in}{2.054238in}}%
\pgfpathlineto{\pgfqpoint{4.396434in}{2.054238in}}%
\pgfpathlineto{\pgfqpoint{4.396434in}{2.057187in}}%
\pgfpathlineto{\pgfqpoint{4.400975in}{2.057187in}}%
\pgfpathlineto{\pgfqpoint{4.400975in}{2.054238in}}%
\pgfpathmoveto{\pgfqpoint{4.410058in}{2.039491in}}%
\pgfpathlineto{\pgfqpoint{4.410058in}{2.039491in}}%
\pgfpathlineto{\pgfqpoint{4.410058in}{2.042441in}}%
\pgfpathlineto{\pgfqpoint{4.414599in}{2.042441in}}%
\pgfpathlineto{\pgfqpoint{4.414599in}{2.039491in}}%
\pgfpathmoveto{\pgfqpoint{4.410058in}{2.042441in}}%
\pgfpathlineto{\pgfqpoint{4.410058in}{2.042441in}}%
\pgfpathlineto{\pgfqpoint{4.410058in}{2.045390in}}%
\pgfpathlineto{\pgfqpoint{4.414599in}{2.045390in}}%
\pgfpathlineto{\pgfqpoint{4.414599in}{2.042441in}}%
\pgfpathmoveto{\pgfqpoint{4.414599in}{2.039491in}}%
\pgfpathlineto{\pgfqpoint{4.414599in}{2.039491in}}%
\pgfpathlineto{\pgfqpoint{4.414599in}{2.042441in}}%
\pgfpathlineto{\pgfqpoint{4.419140in}{2.042441in}}%
\pgfpathlineto{\pgfqpoint{4.419140in}{2.039491in}}%
\pgfpathmoveto{\pgfqpoint{4.414599in}{2.042441in}}%
\pgfpathlineto{\pgfqpoint{4.414599in}{2.042441in}}%
\pgfpathlineto{\pgfqpoint{4.414599in}{2.045390in}}%
\pgfpathlineto{\pgfqpoint{4.419140in}{2.045390in}}%
\pgfpathlineto{\pgfqpoint{4.419140in}{2.042441in}}%
\pgfpathmoveto{\pgfqpoint{4.400975in}{2.045390in}}%
\pgfpathlineto{\pgfqpoint{4.400975in}{2.045390in}}%
\pgfpathlineto{\pgfqpoint{4.400975in}{2.048339in}}%
\pgfpathlineto{\pgfqpoint{4.405517in}{2.048339in}}%
\pgfpathlineto{\pgfqpoint{4.405517in}{2.045390in}}%
\pgfpathmoveto{\pgfqpoint{4.400975in}{2.048339in}}%
\pgfpathlineto{\pgfqpoint{4.400975in}{2.048339in}}%
\pgfpathlineto{\pgfqpoint{4.400975in}{2.051289in}}%
\pgfpathlineto{\pgfqpoint{4.405517in}{2.051289in}}%
\pgfpathlineto{\pgfqpoint{4.405517in}{2.048339in}}%
\pgfpathmoveto{\pgfqpoint{4.405517in}{2.045390in}}%
\pgfpathlineto{\pgfqpoint{4.405517in}{2.045390in}}%
\pgfpathlineto{\pgfqpoint{4.405517in}{2.048339in}}%
\pgfpathlineto{\pgfqpoint{4.410058in}{2.048339in}}%
\pgfpathlineto{\pgfqpoint{4.410058in}{2.045390in}}%
\pgfpathmoveto{\pgfqpoint{4.405517in}{2.048339in}}%
\pgfpathlineto{\pgfqpoint{4.405517in}{2.048339in}}%
\pgfpathlineto{\pgfqpoint{4.405517in}{2.051289in}}%
\pgfpathlineto{\pgfqpoint{4.410058in}{2.051289in}}%
\pgfpathlineto{\pgfqpoint{4.410058in}{2.048339in}}%
\pgfpathmoveto{\pgfqpoint{4.400975in}{2.051289in}}%
\pgfpathlineto{\pgfqpoint{4.400975in}{2.051289in}}%
\pgfpathlineto{\pgfqpoint{4.400975in}{2.054238in}}%
\pgfpathlineto{\pgfqpoint{4.405517in}{2.054238in}}%
\pgfpathlineto{\pgfqpoint{4.405517in}{2.051289in}}%
\pgfpathmoveto{\pgfqpoint{4.400975in}{2.054238in}}%
\pgfpathlineto{\pgfqpoint{4.400975in}{2.054238in}}%
\pgfpathlineto{\pgfqpoint{4.400975in}{2.057187in}}%
\pgfpathlineto{\pgfqpoint{4.405517in}{2.057187in}}%
\pgfpathlineto{\pgfqpoint{4.405517in}{2.054238in}}%
\pgfpathmoveto{\pgfqpoint{4.405517in}{2.051289in}}%
\pgfpathlineto{\pgfqpoint{4.405517in}{2.051289in}}%
\pgfpathlineto{\pgfqpoint{4.405517in}{2.054238in}}%
\pgfpathlineto{\pgfqpoint{4.410058in}{2.054238in}}%
\pgfpathlineto{\pgfqpoint{4.410058in}{2.051289in}}%
\pgfpathmoveto{\pgfqpoint{4.410058in}{2.045390in}}%
\pgfpathlineto{\pgfqpoint{4.410058in}{2.045390in}}%
\pgfpathlineto{\pgfqpoint{4.410058in}{2.048339in}}%
\pgfpathlineto{\pgfqpoint{4.414599in}{2.048339in}}%
\pgfpathlineto{\pgfqpoint{4.414599in}{2.045390in}}%
\pgfpathmoveto{\pgfqpoint{4.410058in}{2.048339in}}%
\pgfpathlineto{\pgfqpoint{4.410058in}{2.048339in}}%
\pgfpathlineto{\pgfqpoint{4.410058in}{2.051289in}}%
\pgfpathlineto{\pgfqpoint{4.414599in}{2.051289in}}%
\pgfpathlineto{\pgfqpoint{4.414599in}{2.048339in}}%
\pgfpathmoveto{\pgfqpoint{4.414599in}{2.045390in}}%
\pgfpathlineto{\pgfqpoint{4.414599in}{2.045390in}}%
\pgfpathlineto{\pgfqpoint{4.414599in}{2.048339in}}%
\pgfpathlineto{\pgfqpoint{4.419140in}{2.048339in}}%
\pgfpathlineto{\pgfqpoint{4.419140in}{2.045390in}}%
\pgfpathmoveto{\pgfqpoint{4.419140in}{2.009997in}}%
\pgfpathlineto{\pgfqpoint{4.419140in}{2.009997in}}%
\pgfpathlineto{\pgfqpoint{4.419140in}{2.012947in}}%
\pgfpathlineto{\pgfqpoint{4.423681in}{2.012947in}}%
\pgfpathlineto{\pgfqpoint{4.423681in}{2.009997in}}%
\pgfpathmoveto{\pgfqpoint{4.419140in}{2.012947in}}%
\pgfpathlineto{\pgfqpoint{4.419140in}{2.012947in}}%
\pgfpathlineto{\pgfqpoint{4.419140in}{2.015896in}}%
\pgfpathlineto{\pgfqpoint{4.423681in}{2.015896in}}%
\pgfpathlineto{\pgfqpoint{4.423681in}{2.012947in}}%
\pgfpathmoveto{\pgfqpoint{4.423681in}{2.009997in}}%
\pgfpathlineto{\pgfqpoint{4.423681in}{2.009997in}}%
\pgfpathlineto{\pgfqpoint{4.423681in}{2.012947in}}%
\pgfpathlineto{\pgfqpoint{4.428222in}{2.012947in}}%
\pgfpathlineto{\pgfqpoint{4.428222in}{2.009997in}}%
\pgfpathmoveto{\pgfqpoint{4.423681in}{2.012947in}}%
\pgfpathlineto{\pgfqpoint{4.423681in}{2.012947in}}%
\pgfpathlineto{\pgfqpoint{4.423681in}{2.015896in}}%
\pgfpathlineto{\pgfqpoint{4.428222in}{2.015896in}}%
\pgfpathlineto{\pgfqpoint{4.428222in}{2.012947in}}%
\pgfpathmoveto{\pgfqpoint{4.428222in}{2.009997in}}%
\pgfpathlineto{\pgfqpoint{4.428222in}{2.009997in}}%
\pgfpathlineto{\pgfqpoint{4.428222in}{2.012947in}}%
\pgfpathlineto{\pgfqpoint{4.432764in}{2.012947in}}%
\pgfpathlineto{\pgfqpoint{4.432764in}{2.009997in}}%
\pgfpathmoveto{\pgfqpoint{4.428222in}{2.012947in}}%
\pgfpathlineto{\pgfqpoint{4.428222in}{2.012947in}}%
\pgfpathlineto{\pgfqpoint{4.428222in}{2.015896in}}%
\pgfpathlineto{\pgfqpoint{4.432764in}{2.015896in}}%
\pgfpathlineto{\pgfqpoint{4.432764in}{2.012947in}}%
\pgfpathmoveto{\pgfqpoint{4.432764in}{2.009997in}}%
\pgfpathlineto{\pgfqpoint{4.432764in}{2.009997in}}%
\pgfpathlineto{\pgfqpoint{4.432764in}{2.012947in}}%
\pgfpathlineto{\pgfqpoint{4.437305in}{2.012947in}}%
\pgfpathlineto{\pgfqpoint{4.437305in}{2.009997in}}%
\pgfpathmoveto{\pgfqpoint{4.432764in}{2.012947in}}%
\pgfpathlineto{\pgfqpoint{4.432764in}{2.012947in}}%
\pgfpathlineto{\pgfqpoint{4.432764in}{2.015896in}}%
\pgfpathlineto{\pgfqpoint{4.437305in}{2.015896in}}%
\pgfpathlineto{\pgfqpoint{4.437305in}{2.012947in}}%
\pgfpathmoveto{\pgfqpoint{4.428222in}{2.027694in}}%
\pgfpathlineto{\pgfqpoint{4.428222in}{2.027694in}}%
\pgfpathlineto{\pgfqpoint{4.428222in}{2.030643in}}%
\pgfpathlineto{\pgfqpoint{4.432764in}{2.030643in}}%
\pgfpathlineto{\pgfqpoint{4.432764in}{2.027694in}}%
\pgfpathmoveto{\pgfqpoint{4.428222in}{2.030643in}}%
\pgfpathlineto{\pgfqpoint{4.428222in}{2.030643in}}%
\pgfpathlineto{\pgfqpoint{4.428222in}{2.033592in}}%
\pgfpathlineto{\pgfqpoint{4.432764in}{2.033592in}}%
\pgfpathlineto{\pgfqpoint{4.432764in}{2.030643in}}%
\pgfpathmoveto{\pgfqpoint{4.432764in}{2.027694in}}%
\pgfpathlineto{\pgfqpoint{4.432764in}{2.027694in}}%
\pgfpathlineto{\pgfqpoint{4.432764in}{2.030643in}}%
\pgfpathlineto{\pgfqpoint{4.437305in}{2.030643in}}%
\pgfpathlineto{\pgfqpoint{4.437305in}{2.027694in}}%
\pgfpathmoveto{\pgfqpoint{4.432764in}{2.030643in}}%
\pgfpathlineto{\pgfqpoint{4.432764in}{2.030643in}}%
\pgfpathlineto{\pgfqpoint{4.432764in}{2.033592in}}%
\pgfpathlineto{\pgfqpoint{4.437305in}{2.033592in}}%
\pgfpathlineto{\pgfqpoint{4.437305in}{2.030643in}}%
\pgfpathmoveto{\pgfqpoint{4.437305in}{2.009997in}}%
\pgfpathlineto{\pgfqpoint{4.437305in}{2.009997in}}%
\pgfpathlineto{\pgfqpoint{4.437305in}{2.012947in}}%
\pgfpathlineto{\pgfqpoint{4.441846in}{2.012947in}}%
\pgfpathlineto{\pgfqpoint{4.441846in}{2.009997in}}%
\pgfpathmoveto{\pgfqpoint{4.437305in}{2.012947in}}%
\pgfpathlineto{\pgfqpoint{4.437305in}{2.012947in}}%
\pgfpathlineto{\pgfqpoint{4.437305in}{2.015896in}}%
\pgfpathlineto{\pgfqpoint{4.441846in}{2.015896in}}%
\pgfpathlineto{\pgfqpoint{4.441846in}{2.012947in}}%
\pgfpathmoveto{\pgfqpoint{4.441846in}{2.009997in}}%
\pgfpathlineto{\pgfqpoint{4.441846in}{2.009997in}}%
\pgfpathlineto{\pgfqpoint{4.441846in}{2.012947in}}%
\pgfpathlineto{\pgfqpoint{4.446387in}{2.012947in}}%
\pgfpathlineto{\pgfqpoint{4.446387in}{2.009997in}}%
\pgfpathmoveto{\pgfqpoint{4.441846in}{2.012947in}}%
\pgfpathlineto{\pgfqpoint{4.441846in}{2.012947in}}%
\pgfpathlineto{\pgfqpoint{4.441846in}{2.015896in}}%
\pgfpathlineto{\pgfqpoint{4.446387in}{2.015896in}}%
\pgfpathlineto{\pgfqpoint{4.446387in}{2.012947in}}%
\pgfpathmoveto{\pgfqpoint{4.446387in}{2.009997in}}%
\pgfpathlineto{\pgfqpoint{4.446387in}{2.009997in}}%
\pgfpathlineto{\pgfqpoint{4.446387in}{2.012947in}}%
\pgfpathlineto{\pgfqpoint{4.450928in}{2.012947in}}%
\pgfpathlineto{\pgfqpoint{4.450928in}{2.009997in}}%
\pgfpathmoveto{\pgfqpoint{4.446387in}{2.012947in}}%
\pgfpathlineto{\pgfqpoint{4.446387in}{2.012947in}}%
\pgfpathlineto{\pgfqpoint{4.446387in}{2.015896in}}%
\pgfpathlineto{\pgfqpoint{4.450928in}{2.015896in}}%
\pgfpathlineto{\pgfqpoint{4.450928in}{2.012947in}}%
\pgfpathmoveto{\pgfqpoint{4.450928in}{2.009997in}}%
\pgfpathlineto{\pgfqpoint{4.450928in}{2.009997in}}%
\pgfpathlineto{\pgfqpoint{4.450928in}{2.012947in}}%
\pgfpathlineto{\pgfqpoint{4.455469in}{2.012947in}}%
\pgfpathlineto{\pgfqpoint{4.455469in}{2.009997in}}%
\pgfpathmoveto{\pgfqpoint{4.450928in}{2.012947in}}%
\pgfpathlineto{\pgfqpoint{4.450928in}{2.012947in}}%
\pgfpathlineto{\pgfqpoint{4.450928in}{2.015896in}}%
\pgfpathlineto{\pgfqpoint{4.455469in}{2.015896in}}%
\pgfpathlineto{\pgfqpoint{4.455469in}{2.012947in}}%
\pgfpathmoveto{\pgfqpoint{4.446387in}{2.015896in}}%
\pgfpathlineto{\pgfqpoint{4.446387in}{2.015896in}}%
\pgfpathlineto{\pgfqpoint{4.446387in}{2.018846in}}%
\pgfpathlineto{\pgfqpoint{4.450928in}{2.018846in}}%
\pgfpathlineto{\pgfqpoint{4.450928in}{2.015896in}}%
\pgfpathmoveto{\pgfqpoint{4.446387in}{2.018846in}}%
\pgfpathlineto{\pgfqpoint{4.446387in}{2.018846in}}%
\pgfpathlineto{\pgfqpoint{4.446387in}{2.021795in}}%
\pgfpathlineto{\pgfqpoint{4.450928in}{2.021795in}}%
\pgfpathlineto{\pgfqpoint{4.450928in}{2.018846in}}%
\pgfpathmoveto{\pgfqpoint{4.450928in}{2.015896in}}%
\pgfpathlineto{\pgfqpoint{4.450928in}{2.015896in}}%
\pgfpathlineto{\pgfqpoint{4.450928in}{2.018846in}}%
\pgfpathlineto{\pgfqpoint{4.455469in}{2.018846in}}%
\pgfpathlineto{\pgfqpoint{4.455469in}{2.015896in}}%
\pgfpathmoveto{\pgfqpoint{4.450928in}{2.018846in}}%
\pgfpathlineto{\pgfqpoint{4.450928in}{2.018846in}}%
\pgfpathlineto{\pgfqpoint{4.450928in}{2.021795in}}%
\pgfpathlineto{\pgfqpoint{4.455469in}{2.021795in}}%
\pgfpathlineto{\pgfqpoint{4.455469in}{2.018846in}}%
\pgfpathmoveto{\pgfqpoint{4.437305in}{2.021795in}}%
\pgfpathlineto{\pgfqpoint{4.437305in}{2.021795in}}%
\pgfpathlineto{\pgfqpoint{4.437305in}{2.024744in}}%
\pgfpathlineto{\pgfqpoint{4.441846in}{2.024744in}}%
\pgfpathlineto{\pgfqpoint{4.441846in}{2.021795in}}%
\pgfpathmoveto{\pgfqpoint{4.437305in}{2.024744in}}%
\pgfpathlineto{\pgfqpoint{4.437305in}{2.024744in}}%
\pgfpathlineto{\pgfqpoint{4.437305in}{2.027694in}}%
\pgfpathlineto{\pgfqpoint{4.441846in}{2.027694in}}%
\pgfpathlineto{\pgfqpoint{4.441846in}{2.024744in}}%
\pgfpathmoveto{\pgfqpoint{4.441846in}{2.021795in}}%
\pgfpathlineto{\pgfqpoint{4.441846in}{2.021795in}}%
\pgfpathlineto{\pgfqpoint{4.441846in}{2.024744in}}%
\pgfpathlineto{\pgfqpoint{4.446387in}{2.024744in}}%
\pgfpathlineto{\pgfqpoint{4.446387in}{2.021795in}}%
\pgfpathmoveto{\pgfqpoint{4.441846in}{2.024744in}}%
\pgfpathlineto{\pgfqpoint{4.441846in}{2.024744in}}%
\pgfpathlineto{\pgfqpoint{4.441846in}{2.027694in}}%
\pgfpathlineto{\pgfqpoint{4.446387in}{2.027694in}}%
\pgfpathlineto{\pgfqpoint{4.446387in}{2.024744in}}%
\pgfpathmoveto{\pgfqpoint{4.437305in}{2.027694in}}%
\pgfpathlineto{\pgfqpoint{4.437305in}{2.027694in}}%
\pgfpathlineto{\pgfqpoint{4.437305in}{2.030643in}}%
\pgfpathlineto{\pgfqpoint{4.441846in}{2.030643in}}%
\pgfpathlineto{\pgfqpoint{4.441846in}{2.027694in}}%
\pgfpathmoveto{\pgfqpoint{4.437305in}{2.030643in}}%
\pgfpathlineto{\pgfqpoint{4.437305in}{2.030643in}}%
\pgfpathlineto{\pgfqpoint{4.437305in}{2.033592in}}%
\pgfpathlineto{\pgfqpoint{4.441846in}{2.033592in}}%
\pgfpathlineto{\pgfqpoint{4.441846in}{2.030643in}}%
\pgfpathmoveto{\pgfqpoint{4.441846in}{2.027694in}}%
\pgfpathlineto{\pgfqpoint{4.441846in}{2.027694in}}%
\pgfpathlineto{\pgfqpoint{4.441846in}{2.030643in}}%
\pgfpathlineto{\pgfqpoint{4.446387in}{2.030643in}}%
\pgfpathlineto{\pgfqpoint{4.446387in}{2.027694in}}%
\pgfpathmoveto{\pgfqpoint{4.446387in}{2.021795in}}%
\pgfpathlineto{\pgfqpoint{4.446387in}{2.021795in}}%
\pgfpathlineto{\pgfqpoint{4.446387in}{2.024744in}}%
\pgfpathlineto{\pgfqpoint{4.450928in}{2.024744in}}%
\pgfpathlineto{\pgfqpoint{4.450928in}{2.021795in}}%
\pgfpathmoveto{\pgfqpoint{4.446387in}{2.024744in}}%
\pgfpathlineto{\pgfqpoint{4.446387in}{2.024744in}}%
\pgfpathlineto{\pgfqpoint{4.446387in}{2.027694in}}%
\pgfpathlineto{\pgfqpoint{4.450928in}{2.027694in}}%
\pgfpathlineto{\pgfqpoint{4.450928in}{2.024744in}}%
\pgfpathmoveto{\pgfqpoint{4.450928in}{2.021795in}}%
\pgfpathlineto{\pgfqpoint{4.450928in}{2.021795in}}%
\pgfpathlineto{\pgfqpoint{4.450928in}{2.024744in}}%
\pgfpathlineto{\pgfqpoint{4.455469in}{2.024744in}}%
\pgfpathlineto{\pgfqpoint{4.455469in}{2.021795in}}%
\pgfpathmoveto{\pgfqpoint{4.419140in}{2.033592in}}%
\pgfpathlineto{\pgfqpoint{4.419140in}{2.033592in}}%
\pgfpathlineto{\pgfqpoint{4.419140in}{2.036542in}}%
\pgfpathlineto{\pgfqpoint{4.423681in}{2.036542in}}%
\pgfpathlineto{\pgfqpoint{4.423681in}{2.033592in}}%
\pgfpathmoveto{\pgfqpoint{4.419140in}{2.036542in}}%
\pgfpathlineto{\pgfqpoint{4.419140in}{2.036542in}}%
\pgfpathlineto{\pgfqpoint{4.419140in}{2.039491in}}%
\pgfpathlineto{\pgfqpoint{4.423681in}{2.039491in}}%
\pgfpathlineto{\pgfqpoint{4.423681in}{2.036542in}}%
\pgfpathmoveto{\pgfqpoint{4.423681in}{2.033592in}}%
\pgfpathlineto{\pgfqpoint{4.423681in}{2.033592in}}%
\pgfpathlineto{\pgfqpoint{4.423681in}{2.036542in}}%
\pgfpathlineto{\pgfqpoint{4.428222in}{2.036542in}}%
\pgfpathlineto{\pgfqpoint{4.428222in}{2.033592in}}%
\pgfpathmoveto{\pgfqpoint{4.423681in}{2.036542in}}%
\pgfpathlineto{\pgfqpoint{4.423681in}{2.036542in}}%
\pgfpathlineto{\pgfqpoint{4.423681in}{2.039491in}}%
\pgfpathlineto{\pgfqpoint{4.428222in}{2.039491in}}%
\pgfpathlineto{\pgfqpoint{4.428222in}{2.036542in}}%
\pgfpathmoveto{\pgfqpoint{4.419140in}{2.039491in}}%
\pgfpathlineto{\pgfqpoint{4.419140in}{2.039491in}}%
\pgfpathlineto{\pgfqpoint{4.419140in}{2.042441in}}%
\pgfpathlineto{\pgfqpoint{4.423681in}{2.042441in}}%
\pgfpathlineto{\pgfqpoint{4.423681in}{2.039491in}}%
\pgfpathmoveto{\pgfqpoint{4.419140in}{2.042441in}}%
\pgfpathlineto{\pgfqpoint{4.419140in}{2.042441in}}%
\pgfpathlineto{\pgfqpoint{4.419140in}{2.045390in}}%
\pgfpathlineto{\pgfqpoint{4.423681in}{2.045390in}}%
\pgfpathlineto{\pgfqpoint{4.423681in}{2.042441in}}%
\pgfpathmoveto{\pgfqpoint{4.423681in}{2.039491in}}%
\pgfpathlineto{\pgfqpoint{4.423681in}{2.039491in}}%
\pgfpathlineto{\pgfqpoint{4.423681in}{2.042441in}}%
\pgfpathlineto{\pgfqpoint{4.428222in}{2.042441in}}%
\pgfpathlineto{\pgfqpoint{4.428222in}{2.039491in}}%
\pgfpathmoveto{\pgfqpoint{4.428222in}{2.033592in}}%
\pgfpathlineto{\pgfqpoint{4.428222in}{2.033592in}}%
\pgfpathlineto{\pgfqpoint{4.428222in}{2.036542in}}%
\pgfpathlineto{\pgfqpoint{4.432764in}{2.036542in}}%
\pgfpathlineto{\pgfqpoint{4.432764in}{2.033592in}}%
\pgfpathmoveto{\pgfqpoint{4.428222in}{2.036542in}}%
\pgfpathlineto{\pgfqpoint{4.428222in}{2.036542in}}%
\pgfpathlineto{\pgfqpoint{4.428222in}{2.039491in}}%
\pgfpathlineto{\pgfqpoint{4.432764in}{2.039491in}}%
\pgfpathlineto{\pgfqpoint{4.432764in}{2.036542in}}%
\pgfpathmoveto{\pgfqpoint{4.432764in}{2.033592in}}%
\pgfpathlineto{\pgfqpoint{4.432764in}{2.033592in}}%
\pgfpathlineto{\pgfqpoint{4.432764in}{2.036542in}}%
\pgfpathlineto{\pgfqpoint{4.437305in}{2.036542in}}%
\pgfpathlineto{\pgfqpoint{4.437305in}{2.033592in}}%
\pgfpathmoveto{\pgfqpoint{4.382811in}{2.057187in}}%
\pgfpathlineto{\pgfqpoint{4.382811in}{2.057187in}}%
\pgfpathlineto{\pgfqpoint{4.382811in}{2.060137in}}%
\pgfpathlineto{\pgfqpoint{4.387352in}{2.060137in}}%
\pgfpathlineto{\pgfqpoint{4.387352in}{2.057187in}}%
\pgfpathmoveto{\pgfqpoint{4.382811in}{2.060137in}}%
\pgfpathlineto{\pgfqpoint{4.382811in}{2.060137in}}%
\pgfpathlineto{\pgfqpoint{4.382811in}{2.063086in}}%
\pgfpathlineto{\pgfqpoint{4.387352in}{2.063086in}}%
\pgfpathlineto{\pgfqpoint{4.387352in}{2.060137in}}%
\pgfpathmoveto{\pgfqpoint{4.387352in}{2.057187in}}%
\pgfpathlineto{\pgfqpoint{4.387352in}{2.057187in}}%
\pgfpathlineto{\pgfqpoint{4.387352in}{2.060137in}}%
\pgfpathlineto{\pgfqpoint{4.391893in}{2.060137in}}%
\pgfpathlineto{\pgfqpoint{4.391893in}{2.057187in}}%
\pgfpathmoveto{\pgfqpoint{4.387352in}{2.060137in}}%
\pgfpathlineto{\pgfqpoint{4.387352in}{2.060137in}}%
\pgfpathlineto{\pgfqpoint{4.387352in}{2.063086in}}%
\pgfpathlineto{\pgfqpoint{4.391893in}{2.063086in}}%
\pgfpathlineto{\pgfqpoint{4.391893in}{2.060137in}}%
\pgfpathmoveto{\pgfqpoint{4.382811in}{2.063086in}}%
\pgfpathlineto{\pgfqpoint{4.382811in}{2.063086in}}%
\pgfpathlineto{\pgfqpoint{4.382811in}{2.066036in}}%
\pgfpathlineto{\pgfqpoint{4.387352in}{2.066036in}}%
\pgfpathlineto{\pgfqpoint{4.387352in}{2.063086in}}%
\pgfpathmoveto{\pgfqpoint{4.382811in}{2.066036in}}%
\pgfpathlineto{\pgfqpoint{4.382811in}{2.066036in}}%
\pgfpathlineto{\pgfqpoint{4.382811in}{2.068985in}}%
\pgfpathlineto{\pgfqpoint{4.387352in}{2.068985in}}%
\pgfpathlineto{\pgfqpoint{4.387352in}{2.066036in}}%
\pgfpathmoveto{\pgfqpoint{4.387352in}{2.063086in}}%
\pgfpathlineto{\pgfqpoint{4.387352in}{2.063086in}}%
\pgfpathlineto{\pgfqpoint{4.387352in}{2.066036in}}%
\pgfpathlineto{\pgfqpoint{4.391893in}{2.066036in}}%
\pgfpathlineto{\pgfqpoint{4.391893in}{2.063086in}}%
\pgfpathmoveto{\pgfqpoint{4.391893in}{2.057187in}}%
\pgfpathlineto{\pgfqpoint{4.391893in}{2.057187in}}%
\pgfpathlineto{\pgfqpoint{4.391893in}{2.060137in}}%
\pgfpathlineto{\pgfqpoint{4.396434in}{2.060137in}}%
\pgfpathlineto{\pgfqpoint{4.396434in}{2.057187in}}%
\pgfpathmoveto{\pgfqpoint{4.391893in}{2.060137in}}%
\pgfpathlineto{\pgfqpoint{4.391893in}{2.060137in}}%
\pgfpathlineto{\pgfqpoint{4.391893in}{2.063086in}}%
\pgfpathlineto{\pgfqpoint{4.396434in}{2.063086in}}%
\pgfpathlineto{\pgfqpoint{4.396434in}{2.060137in}}%
\pgfpathmoveto{\pgfqpoint{4.396434in}{2.057187in}}%
\pgfpathlineto{\pgfqpoint{4.396434in}{2.057187in}}%
\pgfpathlineto{\pgfqpoint{4.396434in}{2.060137in}}%
\pgfpathlineto{\pgfqpoint{4.400975in}{2.060137in}}%
\pgfpathlineto{\pgfqpoint{4.400975in}{2.057187in}}%
\pgfpathmoveto{\pgfqpoint{4.455469in}{2.009997in}}%
\pgfpathlineto{\pgfqpoint{4.455469in}{2.009997in}}%
\pgfpathlineto{\pgfqpoint{4.455469in}{2.012947in}}%
\pgfpathlineto{\pgfqpoint{4.460011in}{2.012947in}}%
\pgfpathlineto{\pgfqpoint{4.460011in}{2.009997in}}%
\pgfpathmoveto{\pgfqpoint{4.455469in}{2.012947in}}%
\pgfpathlineto{\pgfqpoint{4.455469in}{2.012947in}}%
\pgfpathlineto{\pgfqpoint{4.455469in}{2.015896in}}%
\pgfpathlineto{\pgfqpoint{4.460011in}{2.015896in}}%
\pgfpathlineto{\pgfqpoint{4.460011in}{2.012947in}}%
\pgfpathmoveto{\pgfqpoint{4.460011in}{2.009997in}}%
\pgfpathlineto{\pgfqpoint{4.460011in}{2.009997in}}%
\pgfpathlineto{\pgfqpoint{4.460011in}{2.012947in}}%
\pgfpathlineto{\pgfqpoint{4.464552in}{2.012947in}}%
\pgfpathlineto{\pgfqpoint{4.464552in}{2.009997in}}%
\pgfpathmoveto{\pgfqpoint{4.460011in}{2.012947in}}%
\pgfpathlineto{\pgfqpoint{4.460011in}{2.012947in}}%
\pgfpathlineto{\pgfqpoint{4.460011in}{2.015896in}}%
\pgfpathlineto{\pgfqpoint{4.464552in}{2.015896in}}%
\pgfpathlineto{\pgfqpoint{4.464552in}{2.012947in}}%
\pgfpathmoveto{\pgfqpoint{4.455469in}{2.015896in}}%
\pgfpathlineto{\pgfqpoint{4.455469in}{2.015896in}}%
\pgfpathlineto{\pgfqpoint{4.455469in}{2.018846in}}%
\pgfpathlineto{\pgfqpoint{4.460011in}{2.018846in}}%
\pgfpathlineto{\pgfqpoint{4.460011in}{2.015896in}}%
\pgfpathmoveto{\pgfqpoint{4.455469in}{2.018846in}}%
\pgfpathlineto{\pgfqpoint{4.455469in}{2.018846in}}%
\pgfpathlineto{\pgfqpoint{4.455469in}{2.021795in}}%
\pgfpathlineto{\pgfqpoint{4.460011in}{2.021795in}}%
\pgfpathlineto{\pgfqpoint{4.460011in}{2.018846in}}%
\pgfpathmoveto{\pgfqpoint{4.460011in}{2.015896in}}%
\pgfpathlineto{\pgfqpoint{4.460011in}{2.015896in}}%
\pgfpathlineto{\pgfqpoint{4.460011in}{2.018846in}}%
\pgfpathlineto{\pgfqpoint{4.464552in}{2.018846in}}%
\pgfpathlineto{\pgfqpoint{4.464552in}{2.015896in}}%
\pgfpathmoveto{\pgfqpoint{4.464552in}{2.009997in}}%
\pgfpathlineto{\pgfqpoint{4.464552in}{2.009997in}}%
\pgfpathlineto{\pgfqpoint{4.464552in}{2.012947in}}%
\pgfpathlineto{\pgfqpoint{4.469093in}{2.012947in}}%
\pgfpathlineto{\pgfqpoint{4.469093in}{2.009997in}}%
\pgfpathmoveto{\pgfqpoint{4.464552in}{2.012947in}}%
\pgfpathlineto{\pgfqpoint{4.464552in}{2.012947in}}%
\pgfpathlineto{\pgfqpoint{4.464552in}{2.015896in}}%
\pgfpathlineto{\pgfqpoint{4.469093in}{2.015896in}}%
\pgfpathlineto{\pgfqpoint{4.469093in}{2.012947in}}%
\pgfpathmoveto{\pgfqpoint{4.469093in}{2.009997in}}%
\pgfpathlineto{\pgfqpoint{4.469093in}{2.009997in}}%
\pgfpathlineto{\pgfqpoint{4.469093in}{2.012947in}}%
\pgfpathlineto{\pgfqpoint{4.473634in}{2.012947in}}%
\pgfpathlineto{\pgfqpoint{4.473634in}{2.009997in}}%
\pgfpathclose%
\pgfusepath{fill}%
\end{pgfscope}%
\begin{pgfscope}%
\pgfpathrectangle{\pgfqpoint{0.750000in}{0.500000in}}{\pgfqpoint{4.650000in}{3.020000in}}%
\pgfusepath{clip}%
\pgfsetbuttcap%
\pgfsetmiterjoin%
\definecolor{currentfill}{rgb}{1.000000,0.000000,0.000000}%
\pgfsetfillcolor{currentfill}%
\pgfsetlinewidth{0.000000pt}%
\definecolor{currentstroke}{rgb}{0.000000,0.000000,0.000000}%
\pgfsetstrokecolor{currentstroke}%
\pgfsetstrokeopacity{0.000000}%
\pgfsetdash{}{0pt}%
\pgfpathmoveto{\pgfqpoint{0.749998in}{2.611640in}}%
\pgfpathlineto{\pgfqpoint{0.749998in}{2.614590in}}%
\pgfpathlineto{\pgfqpoint{0.754539in}{2.614590in}}%
\pgfpathlineto{\pgfqpoint{0.754539in}{2.611640in}}%
\pgfpathmoveto{\pgfqpoint{0.754539in}{2.611640in}}%
\pgfpathlineto{\pgfqpoint{0.754539in}{2.611640in}}%
\pgfpathlineto{\pgfqpoint{0.754539in}{2.614590in}}%
\pgfpathlineto{\pgfqpoint{0.759080in}{2.614590in}}%
\pgfpathlineto{\pgfqpoint{0.759080in}{2.611640in}}%
\pgfpathmoveto{\pgfqpoint{0.759080in}{2.611640in}}%
\pgfpathlineto{\pgfqpoint{0.759080in}{2.611640in}}%
\pgfpathlineto{\pgfqpoint{0.759080in}{2.614590in}}%
\pgfpathlineto{\pgfqpoint{0.763621in}{2.614590in}}%
\pgfpathlineto{\pgfqpoint{0.763621in}{2.611640in}}%
\pgfpathmoveto{\pgfqpoint{0.763621in}{2.611640in}}%
\pgfpathlineto{\pgfqpoint{0.763621in}{2.611640in}}%
\pgfpathlineto{\pgfqpoint{0.763621in}{2.614590in}}%
\pgfpathlineto{\pgfqpoint{0.768162in}{2.614590in}}%
\pgfpathlineto{\pgfqpoint{0.768162in}{2.611640in}}%
\pgfpathmoveto{\pgfqpoint{0.768162in}{2.611640in}}%
\pgfpathlineto{\pgfqpoint{0.768162in}{2.611640in}}%
\pgfpathlineto{\pgfqpoint{0.768162in}{2.614590in}}%
\pgfpathlineto{\pgfqpoint{0.772703in}{2.614590in}}%
\pgfpathlineto{\pgfqpoint{0.772703in}{2.611640in}}%
\pgfpathmoveto{\pgfqpoint{0.772703in}{2.611640in}}%
\pgfpathlineto{\pgfqpoint{0.772703in}{2.611640in}}%
\pgfpathlineto{\pgfqpoint{0.772703in}{2.614590in}}%
\pgfpathlineto{\pgfqpoint{0.777244in}{2.614590in}}%
\pgfpathlineto{\pgfqpoint{0.777244in}{2.611640in}}%
\pgfpathmoveto{\pgfqpoint{0.777244in}{2.611640in}}%
\pgfpathlineto{\pgfqpoint{0.777244in}{2.611640in}}%
\pgfpathlineto{\pgfqpoint{0.777244in}{2.614590in}}%
\pgfpathlineto{\pgfqpoint{0.781785in}{2.614590in}}%
\pgfpathlineto{\pgfqpoint{0.781785in}{2.611640in}}%
\pgfpathmoveto{\pgfqpoint{0.781785in}{2.611640in}}%
\pgfpathlineto{\pgfqpoint{0.781785in}{2.611640in}}%
\pgfpathlineto{\pgfqpoint{0.781785in}{2.614590in}}%
\pgfpathlineto{\pgfqpoint{0.786326in}{2.614590in}}%
\pgfpathlineto{\pgfqpoint{0.786326in}{2.611640in}}%
\pgfpathmoveto{\pgfqpoint{0.786326in}{2.611640in}}%
\pgfpathlineto{\pgfqpoint{0.786326in}{2.611640in}}%
\pgfpathlineto{\pgfqpoint{0.786326in}{2.614590in}}%
\pgfpathlineto{\pgfqpoint{0.790867in}{2.614590in}}%
\pgfpathlineto{\pgfqpoint{0.790867in}{2.611640in}}%
\pgfpathmoveto{\pgfqpoint{0.790867in}{2.611640in}}%
\pgfpathlineto{\pgfqpoint{0.790867in}{2.611640in}}%
\pgfpathlineto{\pgfqpoint{0.790867in}{2.614590in}}%
\pgfpathlineto{\pgfqpoint{0.795408in}{2.614590in}}%
\pgfpathlineto{\pgfqpoint{0.795408in}{2.611640in}}%
\pgfpathmoveto{\pgfqpoint{0.795408in}{2.611640in}}%
\pgfpathlineto{\pgfqpoint{0.795408in}{2.611640in}}%
\pgfpathlineto{\pgfqpoint{0.795408in}{2.614590in}}%
\pgfpathlineto{\pgfqpoint{0.799949in}{2.614590in}}%
\pgfpathlineto{\pgfqpoint{0.799949in}{2.611640in}}%
\pgfpathmoveto{\pgfqpoint{0.799949in}{2.611640in}}%
\pgfpathlineto{\pgfqpoint{0.799949in}{2.611640in}}%
\pgfpathlineto{\pgfqpoint{0.799949in}{2.614590in}}%
\pgfpathlineto{\pgfqpoint{0.804490in}{2.614590in}}%
\pgfpathlineto{\pgfqpoint{0.804490in}{2.611640in}}%
\pgfpathmoveto{\pgfqpoint{0.804490in}{2.611640in}}%
\pgfpathlineto{\pgfqpoint{0.804490in}{2.611640in}}%
\pgfpathlineto{\pgfqpoint{0.804490in}{2.614590in}}%
\pgfpathlineto{\pgfqpoint{0.809032in}{2.614590in}}%
\pgfpathlineto{\pgfqpoint{0.809032in}{2.611640in}}%
\pgfpathmoveto{\pgfqpoint{0.809032in}{2.611640in}}%
\pgfpathlineto{\pgfqpoint{0.809032in}{2.611640in}}%
\pgfpathlineto{\pgfqpoint{0.809032in}{2.614590in}}%
\pgfpathlineto{\pgfqpoint{0.813573in}{2.614590in}}%
\pgfpathlineto{\pgfqpoint{0.813573in}{2.611640in}}%
\pgfpathmoveto{\pgfqpoint{0.813573in}{2.611640in}}%
\pgfpathlineto{\pgfqpoint{0.813573in}{2.611640in}}%
\pgfpathlineto{\pgfqpoint{0.813573in}{2.614590in}}%
\pgfpathlineto{\pgfqpoint{0.818114in}{2.614590in}}%
\pgfpathlineto{\pgfqpoint{0.818114in}{2.611640in}}%
\pgfpathmoveto{\pgfqpoint{0.818114in}{2.611640in}}%
\pgfpathlineto{\pgfqpoint{0.818114in}{2.611640in}}%
\pgfpathlineto{\pgfqpoint{0.818114in}{2.614590in}}%
\pgfpathlineto{\pgfqpoint{0.822655in}{2.614590in}}%
\pgfpathlineto{\pgfqpoint{0.822655in}{2.611640in}}%
\pgfpathmoveto{\pgfqpoint{0.822655in}{2.611640in}}%
\pgfpathlineto{\pgfqpoint{0.822655in}{2.611640in}}%
\pgfpathlineto{\pgfqpoint{0.822655in}{2.614590in}}%
\pgfpathlineto{\pgfqpoint{0.827196in}{2.614590in}}%
\pgfpathlineto{\pgfqpoint{0.827196in}{2.611640in}}%
\pgfpathmoveto{\pgfqpoint{0.827196in}{2.611640in}}%
\pgfpathlineto{\pgfqpoint{0.827196in}{2.611640in}}%
\pgfpathlineto{\pgfqpoint{0.827196in}{2.614590in}}%
\pgfpathlineto{\pgfqpoint{0.831737in}{2.614590in}}%
\pgfpathlineto{\pgfqpoint{0.831737in}{2.611640in}}%
\pgfpathmoveto{\pgfqpoint{0.831737in}{2.611640in}}%
\pgfpathlineto{\pgfqpoint{0.831737in}{2.611640in}}%
\pgfpathlineto{\pgfqpoint{0.831737in}{2.614590in}}%
\pgfpathlineto{\pgfqpoint{0.836278in}{2.614590in}}%
\pgfpathlineto{\pgfqpoint{0.836278in}{2.611640in}}%
\pgfpathmoveto{\pgfqpoint{0.836278in}{2.611640in}}%
\pgfpathlineto{\pgfqpoint{0.836278in}{2.611640in}}%
\pgfpathlineto{\pgfqpoint{0.836278in}{2.614590in}}%
\pgfpathlineto{\pgfqpoint{0.840819in}{2.614590in}}%
\pgfpathlineto{\pgfqpoint{0.840819in}{2.611640in}}%
\pgfpathmoveto{\pgfqpoint{0.840819in}{2.611640in}}%
\pgfpathlineto{\pgfqpoint{0.840819in}{2.611640in}}%
\pgfpathlineto{\pgfqpoint{0.840819in}{2.614590in}}%
\pgfpathlineto{\pgfqpoint{0.845360in}{2.614590in}}%
\pgfpathlineto{\pgfqpoint{0.845360in}{2.611640in}}%
\pgfpathmoveto{\pgfqpoint{0.845360in}{2.611640in}}%
\pgfpathlineto{\pgfqpoint{0.845360in}{2.611640in}}%
\pgfpathlineto{\pgfqpoint{0.845360in}{2.614590in}}%
\pgfpathlineto{\pgfqpoint{0.849901in}{2.614590in}}%
\pgfpathlineto{\pgfqpoint{0.849901in}{2.611640in}}%
\pgfpathmoveto{\pgfqpoint{0.849901in}{2.611640in}}%
\pgfpathlineto{\pgfqpoint{0.849901in}{2.611640in}}%
\pgfpathlineto{\pgfqpoint{0.849901in}{2.614590in}}%
\pgfpathlineto{\pgfqpoint{0.854442in}{2.614590in}}%
\pgfpathlineto{\pgfqpoint{0.854442in}{2.611640in}}%
\pgfpathmoveto{\pgfqpoint{0.854442in}{2.611640in}}%
\pgfpathlineto{\pgfqpoint{0.854442in}{2.611640in}}%
\pgfpathlineto{\pgfqpoint{0.854442in}{2.614590in}}%
\pgfpathlineto{\pgfqpoint{0.858983in}{2.614590in}}%
\pgfpathlineto{\pgfqpoint{0.858983in}{2.611640in}}%
\pgfpathmoveto{\pgfqpoint{0.858983in}{2.611640in}}%
\pgfpathlineto{\pgfqpoint{0.858983in}{2.611640in}}%
\pgfpathlineto{\pgfqpoint{0.858983in}{2.614590in}}%
\pgfpathlineto{\pgfqpoint{0.863524in}{2.614590in}}%
\pgfpathlineto{\pgfqpoint{0.863524in}{2.611640in}}%
\pgfpathmoveto{\pgfqpoint{0.863524in}{2.611640in}}%
\pgfpathlineto{\pgfqpoint{0.863524in}{2.611640in}}%
\pgfpathlineto{\pgfqpoint{0.863524in}{2.614590in}}%
\pgfpathlineto{\pgfqpoint{0.868066in}{2.614590in}}%
\pgfpathlineto{\pgfqpoint{0.868066in}{2.611640in}}%
\pgfpathmoveto{\pgfqpoint{0.868066in}{2.611640in}}%
\pgfpathlineto{\pgfqpoint{0.868066in}{2.611640in}}%
\pgfpathlineto{\pgfqpoint{0.868066in}{2.614590in}}%
\pgfpathlineto{\pgfqpoint{0.872607in}{2.614590in}}%
\pgfpathlineto{\pgfqpoint{0.872607in}{2.611640in}}%
\pgfpathmoveto{\pgfqpoint{0.872607in}{2.611640in}}%
\pgfpathlineto{\pgfqpoint{0.872607in}{2.611640in}}%
\pgfpathlineto{\pgfqpoint{0.872607in}{2.614590in}}%
\pgfpathlineto{\pgfqpoint{0.877148in}{2.614590in}}%
\pgfpathlineto{\pgfqpoint{0.877148in}{2.611640in}}%
\pgfpathmoveto{\pgfqpoint{0.877148in}{2.611640in}}%
\pgfpathlineto{\pgfqpoint{0.877148in}{2.611640in}}%
\pgfpathlineto{\pgfqpoint{0.877148in}{2.614590in}}%
\pgfpathlineto{\pgfqpoint{0.881689in}{2.614590in}}%
\pgfpathlineto{\pgfqpoint{0.881689in}{2.611640in}}%
\pgfpathmoveto{\pgfqpoint{0.881689in}{2.611640in}}%
\pgfpathlineto{\pgfqpoint{0.881689in}{2.611640in}}%
\pgfpathlineto{\pgfqpoint{0.881689in}{2.614590in}}%
\pgfpathlineto{\pgfqpoint{0.886230in}{2.614590in}}%
\pgfpathlineto{\pgfqpoint{0.886230in}{2.611640in}}%
\pgfpathmoveto{\pgfqpoint{0.886230in}{2.611640in}}%
\pgfpathlineto{\pgfqpoint{0.886230in}{2.611640in}}%
\pgfpathlineto{\pgfqpoint{0.886230in}{2.614590in}}%
\pgfpathlineto{\pgfqpoint{0.890771in}{2.614590in}}%
\pgfpathlineto{\pgfqpoint{0.890771in}{2.611640in}}%
\pgfpathmoveto{\pgfqpoint{0.890771in}{2.611640in}}%
\pgfpathlineto{\pgfqpoint{0.890771in}{2.611640in}}%
\pgfpathlineto{\pgfqpoint{0.890771in}{2.614590in}}%
\pgfpathlineto{\pgfqpoint{0.895312in}{2.614590in}}%
\pgfpathlineto{\pgfqpoint{0.895312in}{2.611640in}}%
\pgfpathmoveto{\pgfqpoint{0.895312in}{2.611640in}}%
\pgfpathlineto{\pgfqpoint{0.895312in}{2.611640in}}%
\pgfpathlineto{\pgfqpoint{0.895312in}{2.614590in}}%
\pgfpathlineto{\pgfqpoint{0.899853in}{2.614590in}}%
\pgfpathlineto{\pgfqpoint{0.899853in}{2.611640in}}%
\pgfpathmoveto{\pgfqpoint{0.899853in}{2.611640in}}%
\pgfpathlineto{\pgfqpoint{0.899853in}{2.611640in}}%
\pgfpathlineto{\pgfqpoint{0.899853in}{2.614590in}}%
\pgfpathlineto{\pgfqpoint{0.904394in}{2.614590in}}%
\pgfpathlineto{\pgfqpoint{0.904394in}{2.611640in}}%
\pgfpathmoveto{\pgfqpoint{0.904394in}{2.611640in}}%
\pgfpathlineto{\pgfqpoint{0.904394in}{2.611640in}}%
\pgfpathlineto{\pgfqpoint{0.904394in}{2.614590in}}%
\pgfpathlineto{\pgfqpoint{0.908935in}{2.614590in}}%
\pgfpathlineto{\pgfqpoint{0.908935in}{2.611640in}}%
\pgfpathmoveto{\pgfqpoint{0.908935in}{2.611640in}}%
\pgfpathlineto{\pgfqpoint{0.908935in}{2.611640in}}%
\pgfpathlineto{\pgfqpoint{0.908935in}{2.614590in}}%
\pgfpathlineto{\pgfqpoint{0.913476in}{2.614590in}}%
\pgfpathlineto{\pgfqpoint{0.913476in}{2.611640in}}%
\pgfpathmoveto{\pgfqpoint{0.913476in}{2.611640in}}%
\pgfpathlineto{\pgfqpoint{0.913476in}{2.611640in}}%
\pgfpathlineto{\pgfqpoint{0.913476in}{2.614590in}}%
\pgfpathlineto{\pgfqpoint{0.918017in}{2.614590in}}%
\pgfpathlineto{\pgfqpoint{0.918017in}{2.611640in}}%
\pgfpathmoveto{\pgfqpoint{0.918017in}{2.611640in}}%
\pgfpathlineto{\pgfqpoint{0.918017in}{2.611640in}}%
\pgfpathlineto{\pgfqpoint{0.918017in}{2.614590in}}%
\pgfpathlineto{\pgfqpoint{0.922558in}{2.614590in}}%
\pgfpathlineto{\pgfqpoint{0.922558in}{2.611640in}}%
\pgfpathmoveto{\pgfqpoint{0.922558in}{2.611640in}}%
\pgfpathlineto{\pgfqpoint{0.922558in}{2.611640in}}%
\pgfpathlineto{\pgfqpoint{0.922558in}{2.614590in}}%
\pgfpathlineto{\pgfqpoint{0.927099in}{2.614590in}}%
\pgfpathlineto{\pgfqpoint{0.927099in}{2.611640in}}%
\pgfpathmoveto{\pgfqpoint{0.927099in}{2.611640in}}%
\pgfpathlineto{\pgfqpoint{0.927099in}{2.611640in}}%
\pgfpathlineto{\pgfqpoint{0.927099in}{2.614590in}}%
\pgfpathlineto{\pgfqpoint{0.931640in}{2.614590in}}%
\pgfpathlineto{\pgfqpoint{0.931640in}{2.611640in}}%
\pgfpathmoveto{\pgfqpoint{0.931640in}{2.611640in}}%
\pgfpathlineto{\pgfqpoint{0.931640in}{2.611640in}}%
\pgfpathlineto{\pgfqpoint{0.931640in}{2.614590in}}%
\pgfpathlineto{\pgfqpoint{0.936181in}{2.614590in}}%
\pgfpathlineto{\pgfqpoint{0.936181in}{2.611640in}}%
\pgfpathmoveto{\pgfqpoint{0.936181in}{2.611640in}}%
\pgfpathlineto{\pgfqpoint{0.936181in}{2.611640in}}%
\pgfpathlineto{\pgfqpoint{0.936181in}{2.614590in}}%
\pgfpathlineto{\pgfqpoint{0.940722in}{2.614590in}}%
\pgfpathlineto{\pgfqpoint{0.940722in}{2.611640in}}%
\pgfpathmoveto{\pgfqpoint{0.940722in}{2.611640in}}%
\pgfpathlineto{\pgfqpoint{0.940722in}{2.611640in}}%
\pgfpathlineto{\pgfqpoint{0.940722in}{2.614590in}}%
\pgfpathlineto{\pgfqpoint{0.945263in}{2.614590in}}%
\pgfpathlineto{\pgfqpoint{0.945263in}{2.611640in}}%
\pgfpathmoveto{\pgfqpoint{0.945263in}{2.611640in}}%
\pgfpathlineto{\pgfqpoint{0.945263in}{2.611640in}}%
\pgfpathlineto{\pgfqpoint{0.945263in}{2.614590in}}%
\pgfpathlineto{\pgfqpoint{0.949804in}{2.614590in}}%
\pgfpathlineto{\pgfqpoint{0.949804in}{2.611640in}}%
\pgfpathmoveto{\pgfqpoint{0.949804in}{2.611640in}}%
\pgfpathlineto{\pgfqpoint{0.949804in}{2.611640in}}%
\pgfpathlineto{\pgfqpoint{0.949804in}{2.614590in}}%
\pgfpathlineto{\pgfqpoint{0.954345in}{2.614590in}}%
\pgfpathlineto{\pgfqpoint{0.954345in}{2.611640in}}%
\pgfpathmoveto{\pgfqpoint{0.954345in}{2.611640in}}%
\pgfpathlineto{\pgfqpoint{0.954345in}{2.611640in}}%
\pgfpathlineto{\pgfqpoint{0.954345in}{2.614590in}}%
\pgfpathlineto{\pgfqpoint{0.958886in}{2.614590in}}%
\pgfpathlineto{\pgfqpoint{0.958886in}{2.611640in}}%
\pgfpathmoveto{\pgfqpoint{0.958886in}{2.611640in}}%
\pgfpathlineto{\pgfqpoint{0.958886in}{2.611640in}}%
\pgfpathlineto{\pgfqpoint{0.958886in}{2.614590in}}%
\pgfpathlineto{\pgfqpoint{0.963427in}{2.614590in}}%
\pgfpathlineto{\pgfqpoint{0.963427in}{2.611640in}}%
\pgfpathmoveto{\pgfqpoint{0.963427in}{2.611640in}}%
\pgfpathlineto{\pgfqpoint{0.963427in}{2.611640in}}%
\pgfpathlineto{\pgfqpoint{0.963427in}{2.614590in}}%
\pgfpathlineto{\pgfqpoint{0.967968in}{2.614590in}}%
\pgfpathlineto{\pgfqpoint{0.967968in}{2.611640in}}%
\pgfpathmoveto{\pgfqpoint{0.967968in}{2.611640in}}%
\pgfpathlineto{\pgfqpoint{0.967968in}{2.611640in}}%
\pgfpathlineto{\pgfqpoint{0.967968in}{2.614590in}}%
\pgfpathlineto{\pgfqpoint{0.972510in}{2.614590in}}%
\pgfpathlineto{\pgfqpoint{0.972510in}{2.611640in}}%
\pgfpathmoveto{\pgfqpoint{0.972510in}{2.611640in}}%
\pgfpathlineto{\pgfqpoint{0.972510in}{2.611640in}}%
\pgfpathlineto{\pgfqpoint{0.972510in}{2.614590in}}%
\pgfpathlineto{\pgfqpoint{0.977051in}{2.614590in}}%
\pgfpathlineto{\pgfqpoint{0.977051in}{2.611640in}}%
\pgfpathmoveto{\pgfqpoint{0.977051in}{2.611640in}}%
\pgfpathlineto{\pgfqpoint{0.977051in}{2.611640in}}%
\pgfpathlineto{\pgfqpoint{0.977051in}{2.614590in}}%
\pgfpathlineto{\pgfqpoint{0.981592in}{2.614590in}}%
\pgfpathlineto{\pgfqpoint{0.981592in}{2.611640in}}%
\pgfpathmoveto{\pgfqpoint{0.981592in}{2.611640in}}%
\pgfpathlineto{\pgfqpoint{0.981592in}{2.611640in}}%
\pgfpathlineto{\pgfqpoint{0.981592in}{2.614590in}}%
\pgfpathlineto{\pgfqpoint{0.986133in}{2.614590in}}%
\pgfpathlineto{\pgfqpoint{0.986133in}{2.611640in}}%
\pgfpathmoveto{\pgfqpoint{0.986133in}{2.611640in}}%
\pgfpathlineto{\pgfqpoint{0.986133in}{2.611640in}}%
\pgfpathlineto{\pgfqpoint{0.986133in}{2.614590in}}%
\pgfpathlineto{\pgfqpoint{0.990674in}{2.614590in}}%
\pgfpathlineto{\pgfqpoint{0.990674in}{2.611640in}}%
\pgfpathmoveto{\pgfqpoint{0.990674in}{2.611640in}}%
\pgfpathlineto{\pgfqpoint{0.990674in}{2.611640in}}%
\pgfpathlineto{\pgfqpoint{0.990674in}{2.614590in}}%
\pgfpathlineto{\pgfqpoint{0.995215in}{2.614590in}}%
\pgfpathlineto{\pgfqpoint{0.995215in}{2.611640in}}%
\pgfpathmoveto{\pgfqpoint{0.995215in}{2.611640in}}%
\pgfpathlineto{\pgfqpoint{0.995215in}{2.611640in}}%
\pgfpathlineto{\pgfqpoint{0.995215in}{2.614590in}}%
\pgfpathlineto{\pgfqpoint{0.999756in}{2.614590in}}%
\pgfpathlineto{\pgfqpoint{0.999756in}{2.611640in}}%
\pgfpathmoveto{\pgfqpoint{0.999756in}{2.611640in}}%
\pgfpathlineto{\pgfqpoint{0.999756in}{2.611640in}}%
\pgfpathlineto{\pgfqpoint{0.999756in}{2.614590in}}%
\pgfpathlineto{\pgfqpoint{1.004297in}{2.614590in}}%
\pgfpathlineto{\pgfqpoint{1.004297in}{2.611640in}}%
\pgfpathmoveto{\pgfqpoint{1.004297in}{2.611640in}}%
\pgfpathlineto{\pgfqpoint{1.004297in}{2.611640in}}%
\pgfpathlineto{\pgfqpoint{1.004297in}{2.614590in}}%
\pgfpathlineto{\pgfqpoint{1.008838in}{2.614590in}}%
\pgfpathlineto{\pgfqpoint{1.008838in}{2.611640in}}%
\pgfpathmoveto{\pgfqpoint{1.008838in}{2.611640in}}%
\pgfpathlineto{\pgfqpoint{1.008838in}{2.611640in}}%
\pgfpathlineto{\pgfqpoint{1.008838in}{2.614590in}}%
\pgfpathlineto{\pgfqpoint{1.013379in}{2.614590in}}%
\pgfpathlineto{\pgfqpoint{1.013379in}{2.611640in}}%
\pgfpathmoveto{\pgfqpoint{1.013379in}{2.611640in}}%
\pgfpathlineto{\pgfqpoint{1.013379in}{2.611640in}}%
\pgfpathlineto{\pgfqpoint{1.013379in}{2.614590in}}%
\pgfpathlineto{\pgfqpoint{1.017920in}{2.614590in}}%
\pgfpathlineto{\pgfqpoint{1.017920in}{2.611640in}}%
\pgfpathmoveto{\pgfqpoint{1.017920in}{2.611640in}}%
\pgfpathlineto{\pgfqpoint{1.017920in}{2.611640in}}%
\pgfpathlineto{\pgfqpoint{1.017920in}{2.614590in}}%
\pgfpathlineto{\pgfqpoint{1.022461in}{2.614590in}}%
\pgfpathlineto{\pgfqpoint{1.022461in}{2.611640in}}%
\pgfpathmoveto{\pgfqpoint{1.022461in}{2.611640in}}%
\pgfpathlineto{\pgfqpoint{1.022461in}{2.611640in}}%
\pgfpathlineto{\pgfqpoint{1.022461in}{2.614590in}}%
\pgfpathlineto{\pgfqpoint{1.027002in}{2.614590in}}%
\pgfpathlineto{\pgfqpoint{1.027002in}{2.611640in}}%
\pgfpathmoveto{\pgfqpoint{1.027002in}{2.611640in}}%
\pgfpathlineto{\pgfqpoint{1.027002in}{2.611640in}}%
\pgfpathlineto{\pgfqpoint{1.027002in}{2.614590in}}%
\pgfpathlineto{\pgfqpoint{1.031543in}{2.614590in}}%
\pgfpathlineto{\pgfqpoint{1.031543in}{2.611640in}}%
\pgfpathmoveto{\pgfqpoint{1.031543in}{2.611640in}}%
\pgfpathlineto{\pgfqpoint{1.031543in}{2.611640in}}%
\pgfpathlineto{\pgfqpoint{1.031543in}{2.614590in}}%
\pgfpathlineto{\pgfqpoint{1.036084in}{2.614590in}}%
\pgfpathlineto{\pgfqpoint{1.036084in}{2.611640in}}%
\pgfpathmoveto{\pgfqpoint{1.036084in}{2.611640in}}%
\pgfpathlineto{\pgfqpoint{1.036084in}{2.611640in}}%
\pgfpathlineto{\pgfqpoint{1.036084in}{2.614590in}}%
\pgfpathlineto{\pgfqpoint{1.040625in}{2.614590in}}%
\pgfpathlineto{\pgfqpoint{1.040625in}{2.611640in}}%
\pgfpathmoveto{\pgfqpoint{1.040625in}{2.611640in}}%
\pgfpathlineto{\pgfqpoint{1.040625in}{2.611640in}}%
\pgfpathlineto{\pgfqpoint{1.040625in}{2.614590in}}%
\pgfpathlineto{\pgfqpoint{1.045166in}{2.614590in}}%
\pgfpathlineto{\pgfqpoint{1.045166in}{2.611640in}}%
\pgfpathmoveto{\pgfqpoint{1.045166in}{2.611640in}}%
\pgfpathlineto{\pgfqpoint{1.045166in}{2.611640in}}%
\pgfpathlineto{\pgfqpoint{1.045166in}{2.614590in}}%
\pgfpathlineto{\pgfqpoint{1.049707in}{2.614590in}}%
\pgfpathlineto{\pgfqpoint{1.049707in}{2.611640in}}%
\pgfpathmoveto{\pgfqpoint{1.049707in}{2.611640in}}%
\pgfpathlineto{\pgfqpoint{1.049707in}{2.611640in}}%
\pgfpathlineto{\pgfqpoint{1.049707in}{2.614590in}}%
\pgfpathlineto{\pgfqpoint{1.054248in}{2.614590in}}%
\pgfpathlineto{\pgfqpoint{1.054248in}{2.611640in}}%
\pgfpathmoveto{\pgfqpoint{1.054248in}{2.611640in}}%
\pgfpathlineto{\pgfqpoint{1.054248in}{2.611640in}}%
\pgfpathlineto{\pgfqpoint{1.054248in}{2.614590in}}%
\pgfpathlineto{\pgfqpoint{1.058789in}{2.614590in}}%
\pgfpathlineto{\pgfqpoint{1.058789in}{2.611640in}}%
\pgfpathmoveto{\pgfqpoint{1.058789in}{2.611640in}}%
\pgfpathlineto{\pgfqpoint{1.058789in}{2.611640in}}%
\pgfpathlineto{\pgfqpoint{1.058789in}{2.614590in}}%
\pgfpathlineto{\pgfqpoint{1.063330in}{2.614590in}}%
\pgfpathlineto{\pgfqpoint{1.063330in}{2.611640in}}%
\pgfpathmoveto{\pgfqpoint{1.063330in}{2.611640in}}%
\pgfpathlineto{\pgfqpoint{1.063330in}{2.611640in}}%
\pgfpathlineto{\pgfqpoint{1.063330in}{2.614590in}}%
\pgfpathlineto{\pgfqpoint{1.067871in}{2.614590in}}%
\pgfpathlineto{\pgfqpoint{1.067871in}{2.611640in}}%
\pgfpathmoveto{\pgfqpoint{1.067871in}{2.611640in}}%
\pgfpathlineto{\pgfqpoint{1.067871in}{2.611640in}}%
\pgfpathlineto{\pgfqpoint{1.067871in}{2.614590in}}%
\pgfpathlineto{\pgfqpoint{1.072412in}{2.614590in}}%
\pgfpathlineto{\pgfqpoint{1.072412in}{2.611640in}}%
\pgfpathmoveto{\pgfqpoint{1.072412in}{2.611640in}}%
\pgfpathlineto{\pgfqpoint{1.072412in}{2.611640in}}%
\pgfpathlineto{\pgfqpoint{1.072412in}{2.614590in}}%
\pgfpathlineto{\pgfqpoint{1.076953in}{2.614590in}}%
\pgfpathlineto{\pgfqpoint{1.076953in}{2.611640in}}%
\pgfpathmoveto{\pgfqpoint{1.076953in}{2.611640in}}%
\pgfpathlineto{\pgfqpoint{1.076953in}{2.611640in}}%
\pgfpathlineto{\pgfqpoint{1.076953in}{2.614590in}}%
\pgfpathlineto{\pgfqpoint{1.081494in}{2.614590in}}%
\pgfpathlineto{\pgfqpoint{1.081494in}{2.611640in}}%
\pgfpathmoveto{\pgfqpoint{1.081494in}{2.611640in}}%
\pgfpathlineto{\pgfqpoint{1.081494in}{2.611640in}}%
\pgfpathlineto{\pgfqpoint{1.081494in}{2.614590in}}%
\pgfpathlineto{\pgfqpoint{1.086035in}{2.614590in}}%
\pgfpathlineto{\pgfqpoint{1.086035in}{2.611640in}}%
\pgfpathmoveto{\pgfqpoint{1.086035in}{2.611640in}}%
\pgfpathlineto{\pgfqpoint{1.086035in}{2.611640in}}%
\pgfpathlineto{\pgfqpoint{1.086035in}{2.614590in}}%
\pgfpathlineto{\pgfqpoint{1.090575in}{2.614590in}}%
\pgfpathlineto{\pgfqpoint{1.090575in}{2.611640in}}%
\pgfpathmoveto{\pgfqpoint{1.090575in}{2.611640in}}%
\pgfpathlineto{\pgfqpoint{1.090575in}{2.611640in}}%
\pgfpathlineto{\pgfqpoint{1.090575in}{2.614590in}}%
\pgfpathlineto{\pgfqpoint{1.095116in}{2.614590in}}%
\pgfpathlineto{\pgfqpoint{1.095116in}{2.611640in}}%
\pgfpathmoveto{\pgfqpoint{1.095116in}{2.611640in}}%
\pgfpathlineto{\pgfqpoint{1.095116in}{2.611640in}}%
\pgfpathlineto{\pgfqpoint{1.095116in}{2.614590in}}%
\pgfpathlineto{\pgfqpoint{1.099657in}{2.614590in}}%
\pgfpathlineto{\pgfqpoint{1.099657in}{2.611640in}}%
\pgfpathmoveto{\pgfqpoint{1.099657in}{2.611640in}}%
\pgfpathlineto{\pgfqpoint{1.099657in}{2.611640in}}%
\pgfpathlineto{\pgfqpoint{1.099657in}{2.614590in}}%
\pgfpathlineto{\pgfqpoint{1.104198in}{2.614590in}}%
\pgfpathlineto{\pgfqpoint{1.104198in}{2.611640in}}%
\pgfpathmoveto{\pgfqpoint{1.104198in}{2.611640in}}%
\pgfpathlineto{\pgfqpoint{1.104198in}{2.611640in}}%
\pgfpathlineto{\pgfqpoint{1.104198in}{2.614590in}}%
\pgfpathlineto{\pgfqpoint{1.108739in}{2.614590in}}%
\pgfpathlineto{\pgfqpoint{1.108739in}{2.611640in}}%
\pgfpathmoveto{\pgfqpoint{1.108739in}{2.611640in}}%
\pgfpathlineto{\pgfqpoint{1.108739in}{2.611640in}}%
\pgfpathlineto{\pgfqpoint{1.108739in}{2.614590in}}%
\pgfpathlineto{\pgfqpoint{1.113280in}{2.614590in}}%
\pgfpathlineto{\pgfqpoint{1.113280in}{2.611640in}}%
\pgfpathmoveto{\pgfqpoint{1.113280in}{2.611640in}}%
\pgfpathlineto{\pgfqpoint{1.113280in}{2.611640in}}%
\pgfpathlineto{\pgfqpoint{1.113280in}{2.614590in}}%
\pgfpathlineto{\pgfqpoint{1.117821in}{2.614590in}}%
\pgfpathlineto{\pgfqpoint{1.117821in}{2.611640in}}%
\pgfpathmoveto{\pgfqpoint{1.117821in}{2.611640in}}%
\pgfpathlineto{\pgfqpoint{1.117821in}{2.611640in}}%
\pgfpathlineto{\pgfqpoint{1.117821in}{2.614590in}}%
\pgfpathlineto{\pgfqpoint{1.122362in}{2.614590in}}%
\pgfpathlineto{\pgfqpoint{1.122362in}{2.611640in}}%
\pgfpathmoveto{\pgfqpoint{1.122362in}{2.611640in}}%
\pgfpathlineto{\pgfqpoint{1.122362in}{2.611640in}}%
\pgfpathlineto{\pgfqpoint{1.122362in}{2.614590in}}%
\pgfpathlineto{\pgfqpoint{1.126903in}{2.614590in}}%
\pgfpathlineto{\pgfqpoint{1.126903in}{2.611640in}}%
\pgfpathmoveto{\pgfqpoint{1.126903in}{2.611640in}}%
\pgfpathlineto{\pgfqpoint{1.126903in}{2.611640in}}%
\pgfpathlineto{\pgfqpoint{1.126903in}{2.614590in}}%
\pgfpathlineto{\pgfqpoint{1.131444in}{2.614590in}}%
\pgfpathlineto{\pgfqpoint{1.131444in}{2.611640in}}%
\pgfpathmoveto{\pgfqpoint{1.131444in}{2.611640in}}%
\pgfpathlineto{\pgfqpoint{1.131444in}{2.611640in}}%
\pgfpathlineto{\pgfqpoint{1.131444in}{2.614590in}}%
\pgfpathlineto{\pgfqpoint{1.135985in}{2.614590in}}%
\pgfpathlineto{\pgfqpoint{1.135985in}{2.611640in}}%
\pgfpathmoveto{\pgfqpoint{1.135985in}{2.611640in}}%
\pgfpathlineto{\pgfqpoint{1.135985in}{2.611640in}}%
\pgfpathlineto{\pgfqpoint{1.135985in}{2.614590in}}%
\pgfpathlineto{\pgfqpoint{1.140526in}{2.614590in}}%
\pgfpathlineto{\pgfqpoint{1.140526in}{2.611640in}}%
\pgfpathmoveto{\pgfqpoint{1.140526in}{2.611640in}}%
\pgfpathlineto{\pgfqpoint{1.140526in}{2.611640in}}%
\pgfpathlineto{\pgfqpoint{1.140526in}{2.614590in}}%
\pgfpathlineto{\pgfqpoint{1.145067in}{2.614590in}}%
\pgfpathlineto{\pgfqpoint{1.145067in}{2.611640in}}%
\pgfpathmoveto{\pgfqpoint{1.145067in}{2.611640in}}%
\pgfpathlineto{\pgfqpoint{1.145067in}{2.611640in}}%
\pgfpathlineto{\pgfqpoint{1.145067in}{2.614590in}}%
\pgfpathlineto{\pgfqpoint{1.149608in}{2.614590in}}%
\pgfpathlineto{\pgfqpoint{1.149608in}{2.611640in}}%
\pgfpathmoveto{\pgfqpoint{1.149608in}{2.611640in}}%
\pgfpathlineto{\pgfqpoint{1.149608in}{2.611640in}}%
\pgfpathlineto{\pgfqpoint{1.149608in}{2.614590in}}%
\pgfpathlineto{\pgfqpoint{1.154149in}{2.614590in}}%
\pgfpathlineto{\pgfqpoint{1.154149in}{2.611640in}}%
\pgfpathmoveto{\pgfqpoint{1.154149in}{2.611640in}}%
\pgfpathlineto{\pgfqpoint{1.154149in}{2.611640in}}%
\pgfpathlineto{\pgfqpoint{1.154149in}{2.614590in}}%
\pgfpathlineto{\pgfqpoint{1.158690in}{2.614590in}}%
\pgfpathlineto{\pgfqpoint{1.158690in}{2.611640in}}%
\pgfpathmoveto{\pgfqpoint{1.158690in}{2.611640in}}%
\pgfpathlineto{\pgfqpoint{1.158690in}{2.611640in}}%
\pgfpathlineto{\pgfqpoint{1.158690in}{2.614590in}}%
\pgfpathlineto{\pgfqpoint{1.163231in}{2.614590in}}%
\pgfpathlineto{\pgfqpoint{1.163231in}{2.611640in}}%
\pgfpathmoveto{\pgfqpoint{1.163231in}{2.611640in}}%
\pgfpathlineto{\pgfqpoint{1.163231in}{2.611640in}}%
\pgfpathlineto{\pgfqpoint{1.163231in}{2.614590in}}%
\pgfpathlineto{\pgfqpoint{1.167772in}{2.614590in}}%
\pgfpathlineto{\pgfqpoint{1.167772in}{2.611640in}}%
\pgfpathmoveto{\pgfqpoint{1.167772in}{2.611640in}}%
\pgfpathlineto{\pgfqpoint{1.167772in}{2.611640in}}%
\pgfpathlineto{\pgfqpoint{1.167772in}{2.614590in}}%
\pgfpathlineto{\pgfqpoint{1.172313in}{2.614590in}}%
\pgfpathlineto{\pgfqpoint{1.172313in}{2.611640in}}%
\pgfpathmoveto{\pgfqpoint{1.172313in}{2.611640in}}%
\pgfpathlineto{\pgfqpoint{1.172313in}{2.611640in}}%
\pgfpathlineto{\pgfqpoint{1.172313in}{2.614590in}}%
\pgfpathlineto{\pgfqpoint{1.176854in}{2.614590in}}%
\pgfpathlineto{\pgfqpoint{1.176854in}{2.611640in}}%
\pgfpathmoveto{\pgfqpoint{1.176854in}{2.611640in}}%
\pgfpathlineto{\pgfqpoint{1.176854in}{2.611640in}}%
\pgfpathlineto{\pgfqpoint{1.176854in}{2.614590in}}%
\pgfpathlineto{\pgfqpoint{1.181395in}{2.614590in}}%
\pgfpathlineto{\pgfqpoint{1.181395in}{2.611640in}}%
\pgfpathmoveto{\pgfqpoint{1.181395in}{2.611640in}}%
\pgfpathlineto{\pgfqpoint{1.181395in}{2.611640in}}%
\pgfpathlineto{\pgfqpoint{1.181395in}{2.614590in}}%
\pgfpathlineto{\pgfqpoint{1.185936in}{2.614590in}}%
\pgfpathlineto{\pgfqpoint{1.185936in}{2.611640in}}%
\pgfpathmoveto{\pgfqpoint{1.185936in}{2.611640in}}%
\pgfpathlineto{\pgfqpoint{1.185936in}{2.611640in}}%
\pgfpathlineto{\pgfqpoint{1.185936in}{2.614590in}}%
\pgfpathlineto{\pgfqpoint{1.190477in}{2.614590in}}%
\pgfpathlineto{\pgfqpoint{1.190477in}{2.611640in}}%
\pgfpathmoveto{\pgfqpoint{1.190477in}{2.611640in}}%
\pgfpathlineto{\pgfqpoint{1.190477in}{2.611640in}}%
\pgfpathlineto{\pgfqpoint{1.190477in}{2.614590in}}%
\pgfpathlineto{\pgfqpoint{1.195018in}{2.614590in}}%
\pgfpathlineto{\pgfqpoint{1.195018in}{2.611640in}}%
\pgfpathmoveto{\pgfqpoint{1.195018in}{2.611640in}}%
\pgfpathlineto{\pgfqpoint{1.195018in}{2.611640in}}%
\pgfpathlineto{\pgfqpoint{1.195018in}{2.614590in}}%
\pgfpathlineto{\pgfqpoint{1.199559in}{2.614590in}}%
\pgfpathlineto{\pgfqpoint{1.199559in}{2.611640in}}%
\pgfpathmoveto{\pgfqpoint{1.199559in}{2.611640in}}%
\pgfpathlineto{\pgfqpoint{1.199559in}{2.611640in}}%
\pgfpathlineto{\pgfqpoint{1.199559in}{2.614590in}}%
\pgfpathlineto{\pgfqpoint{1.204100in}{2.614590in}}%
\pgfpathlineto{\pgfqpoint{1.204100in}{2.611640in}}%
\pgfpathmoveto{\pgfqpoint{1.204100in}{2.611640in}}%
\pgfpathlineto{\pgfqpoint{1.204100in}{2.611640in}}%
\pgfpathlineto{\pgfqpoint{1.204100in}{2.614590in}}%
\pgfpathlineto{\pgfqpoint{1.208641in}{2.614590in}}%
\pgfpathlineto{\pgfqpoint{1.208641in}{2.611640in}}%
\pgfpathmoveto{\pgfqpoint{1.208641in}{2.611640in}}%
\pgfpathlineto{\pgfqpoint{1.208641in}{2.611640in}}%
\pgfpathlineto{\pgfqpoint{1.208641in}{2.614590in}}%
\pgfpathlineto{\pgfqpoint{1.213182in}{2.614590in}}%
\pgfpathlineto{\pgfqpoint{1.213182in}{2.611640in}}%
\pgfpathmoveto{\pgfqpoint{1.213182in}{2.611640in}}%
\pgfpathlineto{\pgfqpoint{1.213182in}{2.611640in}}%
\pgfpathlineto{\pgfqpoint{1.213182in}{2.614590in}}%
\pgfpathlineto{\pgfqpoint{1.217723in}{2.614590in}}%
\pgfpathlineto{\pgfqpoint{1.217723in}{2.611640in}}%
\pgfpathmoveto{\pgfqpoint{1.217723in}{2.611640in}}%
\pgfpathlineto{\pgfqpoint{1.217723in}{2.611640in}}%
\pgfpathlineto{\pgfqpoint{1.217723in}{2.614590in}}%
\pgfpathlineto{\pgfqpoint{1.222264in}{2.614590in}}%
\pgfpathlineto{\pgfqpoint{1.222264in}{2.611640in}}%
\pgfpathmoveto{\pgfqpoint{1.222264in}{2.611640in}}%
\pgfpathlineto{\pgfqpoint{1.222264in}{2.611640in}}%
\pgfpathlineto{\pgfqpoint{1.222264in}{2.614590in}}%
\pgfpathlineto{\pgfqpoint{1.226805in}{2.614590in}}%
\pgfpathlineto{\pgfqpoint{1.226805in}{2.611640in}}%
\pgfpathmoveto{\pgfqpoint{1.226805in}{2.611640in}}%
\pgfpathlineto{\pgfqpoint{1.226805in}{2.611640in}}%
\pgfpathlineto{\pgfqpoint{1.226805in}{2.614590in}}%
\pgfpathlineto{\pgfqpoint{1.231346in}{2.614590in}}%
\pgfpathlineto{\pgfqpoint{1.231346in}{2.611640in}}%
\pgfpathmoveto{\pgfqpoint{1.231346in}{2.611640in}}%
\pgfpathlineto{\pgfqpoint{1.231346in}{2.611640in}}%
\pgfpathlineto{\pgfqpoint{1.231346in}{2.614590in}}%
\pgfpathlineto{\pgfqpoint{1.235887in}{2.614590in}}%
\pgfpathlineto{\pgfqpoint{1.235887in}{2.611640in}}%
\pgfpathmoveto{\pgfqpoint{1.235887in}{2.611640in}}%
\pgfpathlineto{\pgfqpoint{1.235887in}{2.611640in}}%
\pgfpathlineto{\pgfqpoint{1.235887in}{2.614590in}}%
\pgfpathlineto{\pgfqpoint{1.240429in}{2.614590in}}%
\pgfpathlineto{\pgfqpoint{1.240429in}{2.611640in}}%
\pgfpathmoveto{\pgfqpoint{1.240429in}{2.611640in}}%
\pgfpathlineto{\pgfqpoint{1.240429in}{2.611640in}}%
\pgfpathlineto{\pgfqpoint{1.240429in}{2.614590in}}%
\pgfpathlineto{\pgfqpoint{1.244970in}{2.614590in}}%
\pgfpathlineto{\pgfqpoint{1.244970in}{2.611640in}}%
\pgfpathmoveto{\pgfqpoint{1.244970in}{2.611640in}}%
\pgfpathlineto{\pgfqpoint{1.244970in}{2.611640in}}%
\pgfpathlineto{\pgfqpoint{1.244970in}{2.614590in}}%
\pgfpathlineto{\pgfqpoint{1.249511in}{2.614590in}}%
\pgfpathlineto{\pgfqpoint{1.249511in}{2.611640in}}%
\pgfpathmoveto{\pgfqpoint{1.249511in}{2.611640in}}%
\pgfpathlineto{\pgfqpoint{1.249511in}{2.611640in}}%
\pgfpathlineto{\pgfqpoint{1.249511in}{2.614590in}}%
\pgfpathlineto{\pgfqpoint{1.254052in}{2.614590in}}%
\pgfpathlineto{\pgfqpoint{1.254052in}{2.611640in}}%
\pgfpathmoveto{\pgfqpoint{1.254052in}{2.611640in}}%
\pgfpathlineto{\pgfqpoint{1.254052in}{2.611640in}}%
\pgfpathlineto{\pgfqpoint{1.254052in}{2.614590in}}%
\pgfpathlineto{\pgfqpoint{1.258593in}{2.614590in}}%
\pgfpathlineto{\pgfqpoint{1.258593in}{2.611640in}}%
\pgfpathmoveto{\pgfqpoint{1.258593in}{2.611640in}}%
\pgfpathlineto{\pgfqpoint{1.258593in}{2.611640in}}%
\pgfpathlineto{\pgfqpoint{1.258593in}{2.614590in}}%
\pgfpathlineto{\pgfqpoint{1.263134in}{2.614590in}}%
\pgfpathlineto{\pgfqpoint{1.263134in}{2.611640in}}%
\pgfpathmoveto{\pgfqpoint{1.263134in}{2.611640in}}%
\pgfpathlineto{\pgfqpoint{1.263134in}{2.611640in}}%
\pgfpathlineto{\pgfqpoint{1.263134in}{2.614590in}}%
\pgfpathlineto{\pgfqpoint{1.267675in}{2.614590in}}%
\pgfpathlineto{\pgfqpoint{1.267675in}{2.611640in}}%
\pgfpathmoveto{\pgfqpoint{1.267675in}{2.611640in}}%
\pgfpathlineto{\pgfqpoint{1.267675in}{2.611640in}}%
\pgfpathlineto{\pgfqpoint{1.267675in}{2.614590in}}%
\pgfpathlineto{\pgfqpoint{1.272216in}{2.614590in}}%
\pgfpathlineto{\pgfqpoint{1.272216in}{2.611640in}}%
\pgfpathmoveto{\pgfqpoint{1.272216in}{2.611640in}}%
\pgfpathlineto{\pgfqpoint{1.272216in}{2.611640in}}%
\pgfpathlineto{\pgfqpoint{1.272216in}{2.614590in}}%
\pgfpathlineto{\pgfqpoint{1.276757in}{2.614590in}}%
\pgfpathlineto{\pgfqpoint{1.276757in}{2.611640in}}%
\pgfpathmoveto{\pgfqpoint{1.276757in}{2.611640in}}%
\pgfpathlineto{\pgfqpoint{1.276757in}{2.611640in}}%
\pgfpathlineto{\pgfqpoint{1.276757in}{2.614590in}}%
\pgfpathlineto{\pgfqpoint{1.281298in}{2.614590in}}%
\pgfpathlineto{\pgfqpoint{1.281298in}{2.611640in}}%
\pgfpathmoveto{\pgfqpoint{1.281298in}{2.611640in}}%
\pgfpathlineto{\pgfqpoint{1.281298in}{2.611640in}}%
\pgfpathlineto{\pgfqpoint{1.281298in}{2.614590in}}%
\pgfpathlineto{\pgfqpoint{1.285839in}{2.614590in}}%
\pgfpathlineto{\pgfqpoint{1.285839in}{2.611640in}}%
\pgfpathmoveto{\pgfqpoint{1.285839in}{2.611640in}}%
\pgfpathlineto{\pgfqpoint{1.285839in}{2.611640in}}%
\pgfpathlineto{\pgfqpoint{1.285839in}{2.614590in}}%
\pgfpathlineto{\pgfqpoint{1.290380in}{2.614590in}}%
\pgfpathlineto{\pgfqpoint{1.290380in}{2.611640in}}%
\pgfpathmoveto{\pgfqpoint{1.290380in}{2.611640in}}%
\pgfpathlineto{\pgfqpoint{1.290380in}{2.611640in}}%
\pgfpathlineto{\pgfqpoint{1.290380in}{2.614590in}}%
\pgfpathlineto{\pgfqpoint{1.294922in}{2.614590in}}%
\pgfpathlineto{\pgfqpoint{1.294922in}{2.611640in}}%
\pgfpathmoveto{\pgfqpoint{1.294922in}{2.611640in}}%
\pgfpathlineto{\pgfqpoint{1.294922in}{2.611640in}}%
\pgfpathlineto{\pgfqpoint{1.294922in}{2.614590in}}%
\pgfpathlineto{\pgfqpoint{1.299463in}{2.614590in}}%
\pgfpathlineto{\pgfqpoint{1.299463in}{2.611640in}}%
\pgfpathmoveto{\pgfqpoint{1.299463in}{2.611640in}}%
\pgfpathlineto{\pgfqpoint{1.299463in}{2.611640in}}%
\pgfpathlineto{\pgfqpoint{1.299463in}{2.614590in}}%
\pgfpathlineto{\pgfqpoint{1.304004in}{2.614590in}}%
\pgfpathlineto{\pgfqpoint{1.304004in}{2.611640in}}%
\pgfpathmoveto{\pgfqpoint{1.304004in}{2.611640in}}%
\pgfpathlineto{\pgfqpoint{1.304004in}{2.611640in}}%
\pgfpathlineto{\pgfqpoint{1.304004in}{2.614590in}}%
\pgfpathlineto{\pgfqpoint{1.308545in}{2.614590in}}%
\pgfpathlineto{\pgfqpoint{1.308545in}{2.611640in}}%
\pgfpathmoveto{\pgfqpoint{1.308545in}{2.611640in}}%
\pgfpathlineto{\pgfqpoint{1.308545in}{2.611640in}}%
\pgfpathlineto{\pgfqpoint{1.308545in}{2.614590in}}%
\pgfpathlineto{\pgfqpoint{1.313086in}{2.614590in}}%
\pgfpathlineto{\pgfqpoint{1.313086in}{2.611640in}}%
\pgfpathmoveto{\pgfqpoint{1.313086in}{2.611640in}}%
\pgfpathlineto{\pgfqpoint{1.313086in}{2.611640in}}%
\pgfpathlineto{\pgfqpoint{1.313086in}{2.614590in}}%
\pgfpathlineto{\pgfqpoint{1.317627in}{2.614590in}}%
\pgfpathlineto{\pgfqpoint{1.317627in}{2.611640in}}%
\pgfpathmoveto{\pgfqpoint{1.317627in}{2.611640in}}%
\pgfpathlineto{\pgfqpoint{1.317627in}{2.611640in}}%
\pgfpathlineto{\pgfqpoint{1.317627in}{2.614590in}}%
\pgfpathlineto{\pgfqpoint{1.322168in}{2.614590in}}%
\pgfpathlineto{\pgfqpoint{1.322168in}{2.611640in}}%
\pgfpathmoveto{\pgfqpoint{1.322168in}{2.611640in}}%
\pgfpathlineto{\pgfqpoint{1.322168in}{2.611640in}}%
\pgfpathlineto{\pgfqpoint{1.322168in}{2.614590in}}%
\pgfpathlineto{\pgfqpoint{1.326709in}{2.614590in}}%
\pgfpathlineto{\pgfqpoint{1.326709in}{2.611640in}}%
\pgfpathmoveto{\pgfqpoint{1.326709in}{2.611640in}}%
\pgfpathlineto{\pgfqpoint{1.326709in}{2.611640in}}%
\pgfpathlineto{\pgfqpoint{1.326709in}{2.614590in}}%
\pgfpathlineto{\pgfqpoint{1.331250in}{2.614590in}}%
\pgfpathlineto{\pgfqpoint{1.331250in}{2.611640in}}%
\pgfpathmoveto{\pgfqpoint{1.331250in}{2.611640in}}%
\pgfpathlineto{\pgfqpoint{1.331250in}{2.611640in}}%
\pgfpathlineto{\pgfqpoint{1.331250in}{2.614590in}}%
\pgfpathlineto{\pgfqpoint{1.335791in}{2.614590in}}%
\pgfpathlineto{\pgfqpoint{1.335791in}{2.611640in}}%
\pgfpathmoveto{\pgfqpoint{1.335791in}{2.611640in}}%
\pgfpathlineto{\pgfqpoint{1.335791in}{2.611640in}}%
\pgfpathlineto{\pgfqpoint{1.335791in}{2.614590in}}%
\pgfpathlineto{\pgfqpoint{1.340332in}{2.614590in}}%
\pgfpathlineto{\pgfqpoint{1.340332in}{2.611640in}}%
\pgfpathmoveto{\pgfqpoint{1.340332in}{2.611640in}}%
\pgfpathlineto{\pgfqpoint{1.340332in}{2.611640in}}%
\pgfpathlineto{\pgfqpoint{1.340332in}{2.614590in}}%
\pgfpathlineto{\pgfqpoint{1.344874in}{2.614590in}}%
\pgfpathlineto{\pgfqpoint{1.344874in}{2.611640in}}%
\pgfpathmoveto{\pgfqpoint{1.344874in}{2.611640in}}%
\pgfpathlineto{\pgfqpoint{1.344874in}{2.611640in}}%
\pgfpathlineto{\pgfqpoint{1.344874in}{2.614590in}}%
\pgfpathlineto{\pgfqpoint{1.349415in}{2.614590in}}%
\pgfpathlineto{\pgfqpoint{1.349415in}{2.611640in}}%
\pgfpathmoveto{\pgfqpoint{1.349415in}{2.611640in}}%
\pgfpathlineto{\pgfqpoint{1.349415in}{2.611640in}}%
\pgfpathlineto{\pgfqpoint{1.349415in}{2.614590in}}%
\pgfpathlineto{\pgfqpoint{1.353956in}{2.614590in}}%
\pgfpathlineto{\pgfqpoint{1.353956in}{2.611640in}}%
\pgfpathmoveto{\pgfqpoint{1.353956in}{2.611640in}}%
\pgfpathlineto{\pgfqpoint{1.353956in}{2.611640in}}%
\pgfpathlineto{\pgfqpoint{1.353956in}{2.614590in}}%
\pgfpathlineto{\pgfqpoint{1.358497in}{2.614590in}}%
\pgfpathlineto{\pgfqpoint{1.358497in}{2.611640in}}%
\pgfpathmoveto{\pgfqpoint{1.358497in}{2.611640in}}%
\pgfpathlineto{\pgfqpoint{1.358497in}{2.611640in}}%
\pgfpathlineto{\pgfqpoint{1.358497in}{2.614590in}}%
\pgfpathlineto{\pgfqpoint{1.363038in}{2.614590in}}%
\pgfpathlineto{\pgfqpoint{1.363038in}{2.611640in}}%
\pgfpathmoveto{\pgfqpoint{1.363038in}{2.611640in}}%
\pgfpathlineto{\pgfqpoint{1.363038in}{2.611640in}}%
\pgfpathlineto{\pgfqpoint{1.363038in}{2.614590in}}%
\pgfpathlineto{\pgfqpoint{1.367579in}{2.614590in}}%
\pgfpathlineto{\pgfqpoint{1.367579in}{2.611640in}}%
\pgfpathmoveto{\pgfqpoint{1.367579in}{2.611640in}}%
\pgfpathlineto{\pgfqpoint{1.367579in}{2.611640in}}%
\pgfpathlineto{\pgfqpoint{1.367579in}{2.614590in}}%
\pgfpathlineto{\pgfqpoint{1.372120in}{2.614590in}}%
\pgfpathlineto{\pgfqpoint{1.372120in}{2.611640in}}%
\pgfpathmoveto{\pgfqpoint{1.372120in}{2.611640in}}%
\pgfpathlineto{\pgfqpoint{1.372120in}{2.611640in}}%
\pgfpathlineto{\pgfqpoint{1.372120in}{2.614590in}}%
\pgfpathlineto{\pgfqpoint{1.376661in}{2.614590in}}%
\pgfpathlineto{\pgfqpoint{1.376661in}{2.611640in}}%
\pgfpathmoveto{\pgfqpoint{1.376661in}{2.611640in}}%
\pgfpathlineto{\pgfqpoint{1.376661in}{2.611640in}}%
\pgfpathlineto{\pgfqpoint{1.376661in}{2.614590in}}%
\pgfpathlineto{\pgfqpoint{1.381203in}{2.614590in}}%
\pgfpathlineto{\pgfqpoint{1.381203in}{2.611640in}}%
\pgfpathmoveto{\pgfqpoint{1.381203in}{2.611640in}}%
\pgfpathlineto{\pgfqpoint{1.381203in}{2.611640in}}%
\pgfpathlineto{\pgfqpoint{1.381203in}{2.614590in}}%
\pgfpathlineto{\pgfqpoint{1.385744in}{2.614590in}}%
\pgfpathlineto{\pgfqpoint{1.385744in}{2.611640in}}%
\pgfpathmoveto{\pgfqpoint{1.385744in}{2.611640in}}%
\pgfpathlineto{\pgfqpoint{1.385744in}{2.611640in}}%
\pgfpathlineto{\pgfqpoint{1.385744in}{2.614590in}}%
\pgfpathlineto{\pgfqpoint{1.390285in}{2.614590in}}%
\pgfpathlineto{\pgfqpoint{1.390285in}{2.611640in}}%
\pgfpathmoveto{\pgfqpoint{1.390285in}{2.611640in}}%
\pgfpathlineto{\pgfqpoint{1.390285in}{2.611640in}}%
\pgfpathlineto{\pgfqpoint{1.390285in}{2.614590in}}%
\pgfpathlineto{\pgfqpoint{1.394826in}{2.614590in}}%
\pgfpathlineto{\pgfqpoint{1.394826in}{2.611640in}}%
\pgfpathmoveto{\pgfqpoint{1.394826in}{2.611640in}}%
\pgfpathlineto{\pgfqpoint{1.394826in}{2.611640in}}%
\pgfpathlineto{\pgfqpoint{1.394826in}{2.614590in}}%
\pgfpathlineto{\pgfqpoint{1.399367in}{2.614590in}}%
\pgfpathlineto{\pgfqpoint{1.399367in}{2.611640in}}%
\pgfpathmoveto{\pgfqpoint{1.399367in}{2.611640in}}%
\pgfpathlineto{\pgfqpoint{1.399367in}{2.611640in}}%
\pgfpathlineto{\pgfqpoint{1.399367in}{2.614590in}}%
\pgfpathlineto{\pgfqpoint{1.403908in}{2.614590in}}%
\pgfpathlineto{\pgfqpoint{1.403908in}{2.611640in}}%
\pgfpathmoveto{\pgfqpoint{1.403908in}{2.611640in}}%
\pgfpathlineto{\pgfqpoint{1.403908in}{2.611640in}}%
\pgfpathlineto{\pgfqpoint{1.403908in}{2.614590in}}%
\pgfpathlineto{\pgfqpoint{1.408449in}{2.614590in}}%
\pgfpathlineto{\pgfqpoint{1.408449in}{2.611640in}}%
\pgfpathmoveto{\pgfqpoint{1.408449in}{2.611640in}}%
\pgfpathlineto{\pgfqpoint{1.408449in}{2.611640in}}%
\pgfpathlineto{\pgfqpoint{1.408449in}{2.614590in}}%
\pgfpathlineto{\pgfqpoint{1.412990in}{2.614590in}}%
\pgfpathlineto{\pgfqpoint{1.412990in}{2.611640in}}%
\pgfpathmoveto{\pgfqpoint{1.412990in}{2.611640in}}%
\pgfpathlineto{\pgfqpoint{1.412990in}{2.611640in}}%
\pgfpathlineto{\pgfqpoint{1.412990in}{2.614590in}}%
\pgfpathlineto{\pgfqpoint{1.417532in}{2.614590in}}%
\pgfpathlineto{\pgfqpoint{1.417532in}{2.611640in}}%
\pgfpathmoveto{\pgfqpoint{1.417532in}{2.611640in}}%
\pgfpathlineto{\pgfqpoint{1.417532in}{2.611640in}}%
\pgfpathlineto{\pgfqpoint{1.417532in}{2.614590in}}%
\pgfpathlineto{\pgfqpoint{1.422073in}{2.614590in}}%
\pgfpathlineto{\pgfqpoint{1.422073in}{2.611640in}}%
\pgfpathmoveto{\pgfqpoint{1.422073in}{2.611640in}}%
\pgfpathlineto{\pgfqpoint{1.422073in}{2.611640in}}%
\pgfpathlineto{\pgfqpoint{1.422073in}{2.614590in}}%
\pgfpathlineto{\pgfqpoint{1.426614in}{2.614590in}}%
\pgfpathlineto{\pgfqpoint{1.426614in}{2.611640in}}%
\pgfpathmoveto{\pgfqpoint{1.426614in}{2.611640in}}%
\pgfpathlineto{\pgfqpoint{1.426614in}{2.611640in}}%
\pgfpathlineto{\pgfqpoint{1.426614in}{2.614590in}}%
\pgfpathlineto{\pgfqpoint{1.431155in}{2.614590in}}%
\pgfpathlineto{\pgfqpoint{1.431155in}{2.611640in}}%
\pgfpathmoveto{\pgfqpoint{1.431155in}{2.611640in}}%
\pgfpathlineto{\pgfqpoint{1.431155in}{2.611640in}}%
\pgfpathlineto{\pgfqpoint{1.431155in}{2.614590in}}%
\pgfpathlineto{\pgfqpoint{1.435696in}{2.614590in}}%
\pgfpathlineto{\pgfqpoint{1.435696in}{2.611640in}}%
\pgfpathmoveto{\pgfqpoint{1.435696in}{2.611640in}}%
\pgfpathlineto{\pgfqpoint{1.435696in}{2.611640in}}%
\pgfpathlineto{\pgfqpoint{1.435696in}{2.614590in}}%
\pgfpathlineto{\pgfqpoint{1.440237in}{2.614590in}}%
\pgfpathlineto{\pgfqpoint{1.440237in}{2.611640in}}%
\pgfpathmoveto{\pgfqpoint{1.440237in}{2.611640in}}%
\pgfpathlineto{\pgfqpoint{1.440237in}{2.611640in}}%
\pgfpathlineto{\pgfqpoint{1.440237in}{2.614590in}}%
\pgfpathlineto{\pgfqpoint{1.444778in}{2.614590in}}%
\pgfpathlineto{\pgfqpoint{1.444778in}{2.611640in}}%
\pgfpathmoveto{\pgfqpoint{1.444778in}{2.611640in}}%
\pgfpathlineto{\pgfqpoint{1.444778in}{2.611640in}}%
\pgfpathlineto{\pgfqpoint{1.444778in}{2.614590in}}%
\pgfpathlineto{\pgfqpoint{1.449319in}{2.614590in}}%
\pgfpathlineto{\pgfqpoint{1.449319in}{2.611640in}}%
\pgfpathmoveto{\pgfqpoint{1.449319in}{2.611640in}}%
\pgfpathlineto{\pgfqpoint{1.449319in}{2.611640in}}%
\pgfpathlineto{\pgfqpoint{1.449319in}{2.614590in}}%
\pgfpathlineto{\pgfqpoint{1.453860in}{2.614590in}}%
\pgfpathlineto{\pgfqpoint{1.453860in}{2.611640in}}%
\pgfpathmoveto{\pgfqpoint{1.453860in}{2.611640in}}%
\pgfpathlineto{\pgfqpoint{1.453860in}{2.611640in}}%
\pgfpathlineto{\pgfqpoint{1.453860in}{2.614590in}}%
\pgfpathlineto{\pgfqpoint{1.458402in}{2.614590in}}%
\pgfpathlineto{\pgfqpoint{1.458402in}{2.611640in}}%
\pgfpathmoveto{\pgfqpoint{1.458402in}{2.611640in}}%
\pgfpathlineto{\pgfqpoint{1.458402in}{2.611640in}}%
\pgfpathlineto{\pgfqpoint{1.458402in}{2.614590in}}%
\pgfpathlineto{\pgfqpoint{1.462943in}{2.614590in}}%
\pgfpathlineto{\pgfqpoint{1.462943in}{2.611640in}}%
\pgfpathmoveto{\pgfqpoint{1.462943in}{2.611640in}}%
\pgfpathlineto{\pgfqpoint{1.462943in}{2.611640in}}%
\pgfpathlineto{\pgfqpoint{1.462943in}{2.614590in}}%
\pgfpathlineto{\pgfqpoint{1.467484in}{2.614590in}}%
\pgfpathlineto{\pgfqpoint{1.467484in}{2.611640in}}%
\pgfpathmoveto{\pgfqpoint{1.467484in}{2.611640in}}%
\pgfpathlineto{\pgfqpoint{1.467484in}{2.611640in}}%
\pgfpathlineto{\pgfqpoint{1.467484in}{2.614590in}}%
\pgfpathlineto{\pgfqpoint{1.472025in}{2.614590in}}%
\pgfpathlineto{\pgfqpoint{1.472025in}{2.611640in}}%
\pgfpathmoveto{\pgfqpoint{1.472025in}{2.611640in}}%
\pgfpathlineto{\pgfqpoint{1.472025in}{2.611640in}}%
\pgfpathlineto{\pgfqpoint{1.472025in}{2.614590in}}%
\pgfpathlineto{\pgfqpoint{1.476566in}{2.614590in}}%
\pgfpathlineto{\pgfqpoint{1.476566in}{2.611640in}}%
\pgfpathmoveto{\pgfqpoint{1.476566in}{2.611640in}}%
\pgfpathlineto{\pgfqpoint{1.476566in}{2.611640in}}%
\pgfpathlineto{\pgfqpoint{1.476566in}{2.614590in}}%
\pgfpathlineto{\pgfqpoint{1.481107in}{2.614590in}}%
\pgfpathlineto{\pgfqpoint{1.481107in}{2.611640in}}%
\pgfpathmoveto{\pgfqpoint{1.481107in}{2.611640in}}%
\pgfpathlineto{\pgfqpoint{1.481107in}{2.611640in}}%
\pgfpathlineto{\pgfqpoint{1.481107in}{2.614590in}}%
\pgfpathlineto{\pgfqpoint{1.485648in}{2.614590in}}%
\pgfpathlineto{\pgfqpoint{1.485648in}{2.611640in}}%
\pgfpathmoveto{\pgfqpoint{1.485648in}{2.611640in}}%
\pgfpathlineto{\pgfqpoint{1.485648in}{2.611640in}}%
\pgfpathlineto{\pgfqpoint{1.485648in}{2.614590in}}%
\pgfpathlineto{\pgfqpoint{1.490189in}{2.614590in}}%
\pgfpathlineto{\pgfqpoint{1.490189in}{2.611640in}}%
\pgfpathmoveto{\pgfqpoint{1.490189in}{2.611640in}}%
\pgfpathlineto{\pgfqpoint{1.490189in}{2.611640in}}%
\pgfpathlineto{\pgfqpoint{1.490189in}{2.614590in}}%
\pgfpathlineto{\pgfqpoint{1.494730in}{2.614590in}}%
\pgfpathlineto{\pgfqpoint{1.494730in}{2.611640in}}%
\pgfpathmoveto{\pgfqpoint{1.494730in}{2.611640in}}%
\pgfpathlineto{\pgfqpoint{1.494730in}{2.611640in}}%
\pgfpathlineto{\pgfqpoint{1.494730in}{2.614590in}}%
\pgfpathlineto{\pgfqpoint{1.499271in}{2.614590in}}%
\pgfpathlineto{\pgfqpoint{1.499271in}{2.611640in}}%
\pgfpathmoveto{\pgfqpoint{1.499271in}{2.611640in}}%
\pgfpathlineto{\pgfqpoint{1.499271in}{2.611640in}}%
\pgfpathlineto{\pgfqpoint{1.499271in}{2.614590in}}%
\pgfpathlineto{\pgfqpoint{1.503812in}{2.614590in}}%
\pgfpathlineto{\pgfqpoint{1.503812in}{2.611640in}}%
\pgfpathmoveto{\pgfqpoint{1.503812in}{2.611640in}}%
\pgfpathlineto{\pgfqpoint{1.503812in}{2.611640in}}%
\pgfpathlineto{\pgfqpoint{1.503812in}{2.614590in}}%
\pgfpathlineto{\pgfqpoint{1.508353in}{2.614590in}}%
\pgfpathlineto{\pgfqpoint{1.508353in}{2.611640in}}%
\pgfpathmoveto{\pgfqpoint{1.508353in}{2.611640in}}%
\pgfpathlineto{\pgfqpoint{1.508353in}{2.611640in}}%
\pgfpathlineto{\pgfqpoint{1.508353in}{2.614590in}}%
\pgfpathlineto{\pgfqpoint{1.512894in}{2.614590in}}%
\pgfpathlineto{\pgfqpoint{1.512894in}{2.611640in}}%
\pgfpathmoveto{\pgfqpoint{1.512894in}{2.611640in}}%
\pgfpathlineto{\pgfqpoint{1.512894in}{2.611640in}}%
\pgfpathlineto{\pgfqpoint{1.512894in}{2.614590in}}%
\pgfpathlineto{\pgfqpoint{1.517435in}{2.614590in}}%
\pgfpathlineto{\pgfqpoint{1.517435in}{2.611640in}}%
\pgfpathmoveto{\pgfqpoint{1.517435in}{2.611640in}}%
\pgfpathlineto{\pgfqpoint{1.517435in}{2.611640in}}%
\pgfpathlineto{\pgfqpoint{1.517435in}{2.614590in}}%
\pgfpathlineto{\pgfqpoint{1.521976in}{2.614590in}}%
\pgfpathlineto{\pgfqpoint{1.521976in}{2.611640in}}%
\pgfpathmoveto{\pgfqpoint{1.521976in}{2.611640in}}%
\pgfpathlineto{\pgfqpoint{1.521976in}{2.611640in}}%
\pgfpathlineto{\pgfqpoint{1.521976in}{2.614590in}}%
\pgfpathlineto{\pgfqpoint{1.526517in}{2.614590in}}%
\pgfpathlineto{\pgfqpoint{1.526517in}{2.611640in}}%
\pgfpathmoveto{\pgfqpoint{1.526517in}{2.611640in}}%
\pgfpathlineto{\pgfqpoint{1.526517in}{2.611640in}}%
\pgfpathlineto{\pgfqpoint{1.526517in}{2.614590in}}%
\pgfpathlineto{\pgfqpoint{1.531058in}{2.614590in}}%
\pgfpathlineto{\pgfqpoint{1.531058in}{2.611640in}}%
\pgfpathmoveto{\pgfqpoint{1.531058in}{2.611640in}}%
\pgfpathlineto{\pgfqpoint{1.531058in}{2.611640in}}%
\pgfpathlineto{\pgfqpoint{1.531058in}{2.614590in}}%
\pgfpathlineto{\pgfqpoint{1.535599in}{2.614590in}}%
\pgfpathlineto{\pgfqpoint{1.535599in}{2.611640in}}%
\pgfpathmoveto{\pgfqpoint{1.535599in}{2.611640in}}%
\pgfpathlineto{\pgfqpoint{1.535599in}{2.611640in}}%
\pgfpathlineto{\pgfqpoint{1.535599in}{2.614590in}}%
\pgfpathlineto{\pgfqpoint{1.540140in}{2.614590in}}%
\pgfpathlineto{\pgfqpoint{1.540140in}{2.611640in}}%
\pgfpathmoveto{\pgfqpoint{1.540140in}{2.611640in}}%
\pgfpathlineto{\pgfqpoint{1.540140in}{2.611640in}}%
\pgfpathlineto{\pgfqpoint{1.540140in}{2.614590in}}%
\pgfpathlineto{\pgfqpoint{1.544681in}{2.614590in}}%
\pgfpathlineto{\pgfqpoint{1.544681in}{2.611640in}}%
\pgfpathmoveto{\pgfqpoint{1.544681in}{2.611640in}}%
\pgfpathlineto{\pgfqpoint{1.544681in}{2.611640in}}%
\pgfpathlineto{\pgfqpoint{1.544681in}{2.614590in}}%
\pgfpathlineto{\pgfqpoint{1.549222in}{2.614590in}}%
\pgfpathlineto{\pgfqpoint{1.549222in}{2.611640in}}%
\pgfpathmoveto{\pgfqpoint{1.549222in}{2.611640in}}%
\pgfpathlineto{\pgfqpoint{1.549222in}{2.611640in}}%
\pgfpathlineto{\pgfqpoint{1.549222in}{2.614590in}}%
\pgfpathlineto{\pgfqpoint{1.553763in}{2.614590in}}%
\pgfpathlineto{\pgfqpoint{1.553763in}{2.611640in}}%
\pgfpathmoveto{\pgfqpoint{1.553763in}{2.611640in}}%
\pgfpathlineto{\pgfqpoint{1.553763in}{2.611640in}}%
\pgfpathlineto{\pgfqpoint{1.553763in}{2.614590in}}%
\pgfpathlineto{\pgfqpoint{1.558304in}{2.614590in}}%
\pgfpathlineto{\pgfqpoint{1.558304in}{2.611640in}}%
\pgfpathmoveto{\pgfqpoint{1.558304in}{2.611640in}}%
\pgfpathlineto{\pgfqpoint{1.558304in}{2.611640in}}%
\pgfpathlineto{\pgfqpoint{1.558304in}{2.614590in}}%
\pgfpathlineto{\pgfqpoint{1.562845in}{2.614590in}}%
\pgfpathlineto{\pgfqpoint{1.562845in}{2.611640in}}%
\pgfpathmoveto{\pgfqpoint{1.562845in}{2.611640in}}%
\pgfpathlineto{\pgfqpoint{1.562845in}{2.611640in}}%
\pgfpathlineto{\pgfqpoint{1.562845in}{2.614590in}}%
\pgfpathlineto{\pgfqpoint{1.567387in}{2.614590in}}%
\pgfpathlineto{\pgfqpoint{1.567387in}{2.611640in}}%
\pgfpathmoveto{\pgfqpoint{1.567387in}{2.611640in}}%
\pgfpathlineto{\pgfqpoint{1.567387in}{2.611640in}}%
\pgfpathlineto{\pgfqpoint{1.567387in}{2.614590in}}%
\pgfpathlineto{\pgfqpoint{1.571928in}{2.614590in}}%
\pgfpathlineto{\pgfqpoint{1.571928in}{2.611640in}}%
\pgfpathmoveto{\pgfqpoint{1.571928in}{2.611640in}}%
\pgfpathlineto{\pgfqpoint{1.571928in}{2.611640in}}%
\pgfpathlineto{\pgfqpoint{1.571928in}{2.614590in}}%
\pgfpathlineto{\pgfqpoint{1.576469in}{2.614590in}}%
\pgfpathlineto{\pgfqpoint{1.576469in}{2.611640in}}%
\pgfpathmoveto{\pgfqpoint{1.576469in}{2.611640in}}%
\pgfpathlineto{\pgfqpoint{1.576469in}{2.611640in}}%
\pgfpathlineto{\pgfqpoint{1.576469in}{2.614590in}}%
\pgfpathlineto{\pgfqpoint{1.581010in}{2.614590in}}%
\pgfpathlineto{\pgfqpoint{1.581010in}{2.611640in}}%
\pgfpathmoveto{\pgfqpoint{1.581010in}{2.611640in}}%
\pgfpathlineto{\pgfqpoint{1.581010in}{2.611640in}}%
\pgfpathlineto{\pgfqpoint{1.581010in}{2.614590in}}%
\pgfpathlineto{\pgfqpoint{1.585551in}{2.614590in}}%
\pgfpathlineto{\pgfqpoint{1.585551in}{2.611640in}}%
\pgfpathmoveto{\pgfqpoint{1.585551in}{2.611640in}}%
\pgfpathlineto{\pgfqpoint{1.585551in}{2.611640in}}%
\pgfpathlineto{\pgfqpoint{1.585551in}{2.614590in}}%
\pgfpathlineto{\pgfqpoint{1.590092in}{2.614590in}}%
\pgfpathlineto{\pgfqpoint{1.590092in}{2.611640in}}%
\pgfpathmoveto{\pgfqpoint{1.590092in}{2.611640in}}%
\pgfpathlineto{\pgfqpoint{1.590092in}{2.611640in}}%
\pgfpathlineto{\pgfqpoint{1.590092in}{2.614590in}}%
\pgfpathlineto{\pgfqpoint{1.594633in}{2.614590in}}%
\pgfpathlineto{\pgfqpoint{1.594633in}{2.611640in}}%
\pgfpathmoveto{\pgfqpoint{1.594633in}{2.611640in}}%
\pgfpathlineto{\pgfqpoint{1.594633in}{2.611640in}}%
\pgfpathlineto{\pgfqpoint{1.594633in}{2.614590in}}%
\pgfpathlineto{\pgfqpoint{1.599174in}{2.614590in}}%
\pgfpathlineto{\pgfqpoint{1.599174in}{2.611640in}}%
\pgfpathmoveto{\pgfqpoint{1.599174in}{2.611640in}}%
\pgfpathlineto{\pgfqpoint{1.599174in}{2.611640in}}%
\pgfpathlineto{\pgfqpoint{1.599174in}{2.614590in}}%
\pgfpathlineto{\pgfqpoint{1.603715in}{2.614590in}}%
\pgfpathlineto{\pgfqpoint{1.603715in}{2.611640in}}%
\pgfpathmoveto{\pgfqpoint{1.603715in}{2.611640in}}%
\pgfpathlineto{\pgfqpoint{1.603715in}{2.611640in}}%
\pgfpathlineto{\pgfqpoint{1.603715in}{2.614590in}}%
\pgfpathlineto{\pgfqpoint{1.608256in}{2.614590in}}%
\pgfpathlineto{\pgfqpoint{1.608256in}{2.611640in}}%
\pgfpathmoveto{\pgfqpoint{1.608256in}{2.611640in}}%
\pgfpathlineto{\pgfqpoint{1.608256in}{2.611640in}}%
\pgfpathlineto{\pgfqpoint{1.608256in}{2.614590in}}%
\pgfpathlineto{\pgfqpoint{1.612797in}{2.614590in}}%
\pgfpathlineto{\pgfqpoint{1.612797in}{2.611640in}}%
\pgfpathmoveto{\pgfqpoint{1.612797in}{2.611640in}}%
\pgfpathlineto{\pgfqpoint{1.612797in}{2.611640in}}%
\pgfpathlineto{\pgfqpoint{1.612797in}{2.614590in}}%
\pgfpathlineto{\pgfqpoint{1.617338in}{2.614590in}}%
\pgfpathlineto{\pgfqpoint{1.617338in}{2.611640in}}%
\pgfpathmoveto{\pgfqpoint{1.617338in}{2.611640in}}%
\pgfpathlineto{\pgfqpoint{1.617338in}{2.611640in}}%
\pgfpathlineto{\pgfqpoint{1.617338in}{2.614590in}}%
\pgfpathlineto{\pgfqpoint{1.621879in}{2.614590in}}%
\pgfpathlineto{\pgfqpoint{1.621879in}{2.611640in}}%
\pgfpathmoveto{\pgfqpoint{1.621879in}{2.611640in}}%
\pgfpathlineto{\pgfqpoint{1.621879in}{2.611640in}}%
\pgfpathlineto{\pgfqpoint{1.621879in}{2.614590in}}%
\pgfpathlineto{\pgfqpoint{1.626420in}{2.614590in}}%
\pgfpathlineto{\pgfqpoint{1.626420in}{2.611640in}}%
\pgfpathmoveto{\pgfqpoint{1.626420in}{2.611640in}}%
\pgfpathlineto{\pgfqpoint{1.626420in}{2.611640in}}%
\pgfpathlineto{\pgfqpoint{1.626420in}{2.614590in}}%
\pgfpathlineto{\pgfqpoint{1.630961in}{2.614590in}}%
\pgfpathlineto{\pgfqpoint{1.630961in}{2.611640in}}%
\pgfpathmoveto{\pgfqpoint{1.630961in}{2.611640in}}%
\pgfpathlineto{\pgfqpoint{1.630961in}{2.611640in}}%
\pgfpathlineto{\pgfqpoint{1.630961in}{2.614590in}}%
\pgfpathlineto{\pgfqpoint{1.635502in}{2.614590in}}%
\pgfpathlineto{\pgfqpoint{1.635502in}{2.611640in}}%
\pgfpathmoveto{\pgfqpoint{1.635502in}{2.611640in}}%
\pgfpathlineto{\pgfqpoint{1.635502in}{2.611640in}}%
\pgfpathlineto{\pgfqpoint{1.635502in}{2.614590in}}%
\pgfpathlineto{\pgfqpoint{1.640043in}{2.614590in}}%
\pgfpathlineto{\pgfqpoint{1.640043in}{2.611640in}}%
\pgfpathmoveto{\pgfqpoint{1.640043in}{2.611640in}}%
\pgfpathlineto{\pgfqpoint{1.640043in}{2.611640in}}%
\pgfpathlineto{\pgfqpoint{1.640043in}{2.614590in}}%
\pgfpathlineto{\pgfqpoint{1.644584in}{2.614590in}}%
\pgfpathlineto{\pgfqpoint{1.644584in}{2.611640in}}%
\pgfpathmoveto{\pgfqpoint{1.644584in}{2.611640in}}%
\pgfpathlineto{\pgfqpoint{1.644584in}{2.611640in}}%
\pgfpathlineto{\pgfqpoint{1.644584in}{2.614590in}}%
\pgfpathlineto{\pgfqpoint{1.649125in}{2.614590in}}%
\pgfpathlineto{\pgfqpoint{1.649125in}{2.611640in}}%
\pgfpathmoveto{\pgfqpoint{1.649125in}{2.611640in}}%
\pgfpathlineto{\pgfqpoint{1.649125in}{2.611640in}}%
\pgfpathlineto{\pgfqpoint{1.649125in}{2.614590in}}%
\pgfpathlineto{\pgfqpoint{1.653666in}{2.614590in}}%
\pgfpathlineto{\pgfqpoint{1.653666in}{2.611640in}}%
\pgfpathmoveto{\pgfqpoint{1.653666in}{2.611640in}}%
\pgfpathlineto{\pgfqpoint{1.653666in}{2.611640in}}%
\pgfpathlineto{\pgfqpoint{1.653666in}{2.614590in}}%
\pgfpathlineto{\pgfqpoint{1.658207in}{2.614590in}}%
\pgfpathlineto{\pgfqpoint{1.658207in}{2.611640in}}%
\pgfpathmoveto{\pgfqpoint{1.658207in}{2.611640in}}%
\pgfpathlineto{\pgfqpoint{1.658207in}{2.611640in}}%
\pgfpathlineto{\pgfqpoint{1.658207in}{2.614590in}}%
\pgfpathlineto{\pgfqpoint{1.662748in}{2.614590in}}%
\pgfpathlineto{\pgfqpoint{1.662748in}{2.611640in}}%
\pgfpathmoveto{\pgfqpoint{1.662748in}{2.611640in}}%
\pgfpathlineto{\pgfqpoint{1.662748in}{2.611640in}}%
\pgfpathlineto{\pgfqpoint{1.662748in}{2.614590in}}%
\pgfpathlineto{\pgfqpoint{1.667289in}{2.614590in}}%
\pgfpathlineto{\pgfqpoint{1.667289in}{2.611640in}}%
\pgfpathmoveto{\pgfqpoint{1.667289in}{2.611640in}}%
\pgfpathlineto{\pgfqpoint{1.667289in}{2.611640in}}%
\pgfpathlineto{\pgfqpoint{1.667289in}{2.614590in}}%
\pgfpathlineto{\pgfqpoint{1.671829in}{2.614590in}}%
\pgfpathlineto{\pgfqpoint{1.671829in}{2.611640in}}%
\pgfpathmoveto{\pgfqpoint{1.671829in}{2.611640in}}%
\pgfpathlineto{\pgfqpoint{1.671829in}{2.611640in}}%
\pgfpathlineto{\pgfqpoint{1.671829in}{2.614590in}}%
\pgfpathlineto{\pgfqpoint{1.676370in}{2.614590in}}%
\pgfpathlineto{\pgfqpoint{1.676370in}{2.611640in}}%
\pgfpathmoveto{\pgfqpoint{1.676370in}{2.611640in}}%
\pgfpathlineto{\pgfqpoint{1.676370in}{2.611640in}}%
\pgfpathlineto{\pgfqpoint{1.676370in}{2.614590in}}%
\pgfpathlineto{\pgfqpoint{1.680911in}{2.614590in}}%
\pgfpathlineto{\pgfqpoint{1.680911in}{2.611640in}}%
\pgfpathmoveto{\pgfqpoint{1.680911in}{2.611640in}}%
\pgfpathlineto{\pgfqpoint{1.680911in}{2.611640in}}%
\pgfpathlineto{\pgfqpoint{1.680911in}{2.614590in}}%
\pgfpathlineto{\pgfqpoint{1.685452in}{2.614590in}}%
\pgfpathlineto{\pgfqpoint{1.685452in}{2.611640in}}%
\pgfpathmoveto{\pgfqpoint{1.685452in}{2.611640in}}%
\pgfpathlineto{\pgfqpoint{1.685452in}{2.611640in}}%
\pgfpathlineto{\pgfqpoint{1.685452in}{2.614590in}}%
\pgfpathlineto{\pgfqpoint{1.689993in}{2.614590in}}%
\pgfpathlineto{\pgfqpoint{1.689993in}{2.611640in}}%
\pgfpathmoveto{\pgfqpoint{1.689993in}{2.611640in}}%
\pgfpathlineto{\pgfqpoint{1.689993in}{2.611640in}}%
\pgfpathlineto{\pgfqpoint{1.689993in}{2.614590in}}%
\pgfpathlineto{\pgfqpoint{1.694534in}{2.614590in}}%
\pgfpathlineto{\pgfqpoint{1.694534in}{2.611640in}}%
\pgfpathmoveto{\pgfqpoint{1.694534in}{2.611640in}}%
\pgfpathlineto{\pgfqpoint{1.694534in}{2.611640in}}%
\pgfpathlineto{\pgfqpoint{1.694534in}{2.614590in}}%
\pgfpathlineto{\pgfqpoint{1.699075in}{2.614590in}}%
\pgfpathlineto{\pgfqpoint{1.699075in}{2.611640in}}%
\pgfpathmoveto{\pgfqpoint{1.699075in}{2.611640in}}%
\pgfpathlineto{\pgfqpoint{1.699075in}{2.611640in}}%
\pgfpathlineto{\pgfqpoint{1.699075in}{2.614590in}}%
\pgfpathlineto{\pgfqpoint{1.703616in}{2.614590in}}%
\pgfpathlineto{\pgfqpoint{1.703616in}{2.611640in}}%
\pgfpathmoveto{\pgfqpoint{1.703616in}{2.611640in}}%
\pgfpathlineto{\pgfqpoint{1.703616in}{2.611640in}}%
\pgfpathlineto{\pgfqpoint{1.703616in}{2.614590in}}%
\pgfpathlineto{\pgfqpoint{1.708157in}{2.614590in}}%
\pgfpathlineto{\pgfqpoint{1.708157in}{2.611640in}}%
\pgfpathmoveto{\pgfqpoint{1.708157in}{2.611640in}}%
\pgfpathlineto{\pgfqpoint{1.708157in}{2.611640in}}%
\pgfpathlineto{\pgfqpoint{1.708157in}{2.614590in}}%
\pgfpathlineto{\pgfqpoint{1.712698in}{2.614590in}}%
\pgfpathlineto{\pgfqpoint{1.712698in}{2.611640in}}%
\pgfpathmoveto{\pgfqpoint{1.712698in}{2.611640in}}%
\pgfpathlineto{\pgfqpoint{1.712698in}{2.611640in}}%
\pgfpathlineto{\pgfqpoint{1.712698in}{2.614590in}}%
\pgfpathlineto{\pgfqpoint{1.717239in}{2.614590in}}%
\pgfpathlineto{\pgfqpoint{1.717239in}{2.611640in}}%
\pgfpathmoveto{\pgfqpoint{1.717239in}{2.611640in}}%
\pgfpathlineto{\pgfqpoint{1.717239in}{2.611640in}}%
\pgfpathlineto{\pgfqpoint{1.717239in}{2.614590in}}%
\pgfpathlineto{\pgfqpoint{1.721780in}{2.614590in}}%
\pgfpathlineto{\pgfqpoint{1.721780in}{2.611640in}}%
\pgfpathmoveto{\pgfqpoint{1.721780in}{2.611640in}}%
\pgfpathlineto{\pgfqpoint{1.721780in}{2.611640in}}%
\pgfpathlineto{\pgfqpoint{1.721780in}{2.614590in}}%
\pgfpathlineto{\pgfqpoint{1.726321in}{2.614590in}}%
\pgfpathlineto{\pgfqpoint{1.726321in}{2.611640in}}%
\pgfpathmoveto{\pgfqpoint{1.726321in}{2.611640in}}%
\pgfpathlineto{\pgfqpoint{1.726321in}{2.611640in}}%
\pgfpathlineto{\pgfqpoint{1.726321in}{2.614590in}}%
\pgfpathlineto{\pgfqpoint{1.730862in}{2.614590in}}%
\pgfpathlineto{\pgfqpoint{1.730862in}{2.611640in}}%
\pgfpathmoveto{\pgfqpoint{1.730862in}{2.611640in}}%
\pgfpathlineto{\pgfqpoint{1.730862in}{2.611640in}}%
\pgfpathlineto{\pgfqpoint{1.730862in}{2.614590in}}%
\pgfpathlineto{\pgfqpoint{1.735403in}{2.614590in}}%
\pgfpathlineto{\pgfqpoint{1.735403in}{2.611640in}}%
\pgfpathmoveto{\pgfqpoint{1.735403in}{2.611640in}}%
\pgfpathlineto{\pgfqpoint{1.735403in}{2.611640in}}%
\pgfpathlineto{\pgfqpoint{1.735403in}{2.614590in}}%
\pgfpathlineto{\pgfqpoint{1.739944in}{2.614590in}}%
\pgfpathlineto{\pgfqpoint{1.739944in}{2.611640in}}%
\pgfpathmoveto{\pgfqpoint{1.739944in}{2.611640in}}%
\pgfpathlineto{\pgfqpoint{1.739944in}{2.611640in}}%
\pgfpathlineto{\pgfqpoint{1.739944in}{2.614590in}}%
\pgfpathlineto{\pgfqpoint{1.744485in}{2.614590in}}%
\pgfpathlineto{\pgfqpoint{1.744485in}{2.611640in}}%
\pgfpathmoveto{\pgfqpoint{1.744485in}{2.611640in}}%
\pgfpathlineto{\pgfqpoint{1.744485in}{2.611640in}}%
\pgfpathlineto{\pgfqpoint{1.744485in}{2.614590in}}%
\pgfpathlineto{\pgfqpoint{1.749026in}{2.614590in}}%
\pgfpathlineto{\pgfqpoint{1.749026in}{2.611640in}}%
\pgfpathmoveto{\pgfqpoint{1.749026in}{2.611640in}}%
\pgfpathlineto{\pgfqpoint{1.749026in}{2.611640in}}%
\pgfpathlineto{\pgfqpoint{1.749026in}{2.614590in}}%
\pgfpathlineto{\pgfqpoint{1.753567in}{2.614590in}}%
\pgfpathlineto{\pgfqpoint{1.753567in}{2.611640in}}%
\pgfpathmoveto{\pgfqpoint{1.753567in}{2.611640in}}%
\pgfpathlineto{\pgfqpoint{1.753567in}{2.611640in}}%
\pgfpathlineto{\pgfqpoint{1.753567in}{2.614590in}}%
\pgfpathlineto{\pgfqpoint{1.758108in}{2.614590in}}%
\pgfpathlineto{\pgfqpoint{1.758108in}{2.611640in}}%
\pgfpathmoveto{\pgfqpoint{1.758108in}{2.611640in}}%
\pgfpathlineto{\pgfqpoint{1.758108in}{2.611640in}}%
\pgfpathlineto{\pgfqpoint{1.758108in}{2.614590in}}%
\pgfpathlineto{\pgfqpoint{1.762649in}{2.614590in}}%
\pgfpathlineto{\pgfqpoint{1.762649in}{2.611640in}}%
\pgfpathmoveto{\pgfqpoint{1.762649in}{2.611640in}}%
\pgfpathlineto{\pgfqpoint{1.762649in}{2.611640in}}%
\pgfpathlineto{\pgfqpoint{1.762649in}{2.614590in}}%
\pgfpathlineto{\pgfqpoint{1.767190in}{2.614590in}}%
\pgfpathlineto{\pgfqpoint{1.767190in}{2.611640in}}%
\pgfpathmoveto{\pgfqpoint{1.767190in}{2.611640in}}%
\pgfpathlineto{\pgfqpoint{1.767190in}{2.611640in}}%
\pgfpathlineto{\pgfqpoint{1.767190in}{2.614590in}}%
\pgfpathlineto{\pgfqpoint{1.771731in}{2.614590in}}%
\pgfpathlineto{\pgfqpoint{1.771731in}{2.611640in}}%
\pgfpathmoveto{\pgfqpoint{1.771731in}{2.611640in}}%
\pgfpathlineto{\pgfqpoint{1.771731in}{2.611640in}}%
\pgfpathlineto{\pgfqpoint{1.771731in}{2.614590in}}%
\pgfpathlineto{\pgfqpoint{1.776272in}{2.614590in}}%
\pgfpathlineto{\pgfqpoint{1.776272in}{2.611640in}}%
\pgfpathmoveto{\pgfqpoint{1.776272in}{2.611640in}}%
\pgfpathlineto{\pgfqpoint{1.776272in}{2.611640in}}%
\pgfpathlineto{\pgfqpoint{1.776272in}{2.614590in}}%
\pgfpathlineto{\pgfqpoint{1.780813in}{2.614590in}}%
\pgfpathlineto{\pgfqpoint{1.780813in}{2.611640in}}%
\pgfpathmoveto{\pgfqpoint{1.780813in}{2.611640in}}%
\pgfpathlineto{\pgfqpoint{1.780813in}{2.611640in}}%
\pgfpathlineto{\pgfqpoint{1.780813in}{2.614590in}}%
\pgfpathlineto{\pgfqpoint{1.785353in}{2.614590in}}%
\pgfpathlineto{\pgfqpoint{1.785353in}{2.611640in}}%
\pgfpathmoveto{\pgfqpoint{1.785353in}{2.611640in}}%
\pgfpathlineto{\pgfqpoint{1.785353in}{2.611640in}}%
\pgfpathlineto{\pgfqpoint{1.785353in}{2.614590in}}%
\pgfpathlineto{\pgfqpoint{1.789894in}{2.614590in}}%
\pgfpathlineto{\pgfqpoint{1.789894in}{2.611640in}}%
\pgfpathmoveto{\pgfqpoint{1.789894in}{2.611640in}}%
\pgfpathlineto{\pgfqpoint{1.789894in}{2.611640in}}%
\pgfpathlineto{\pgfqpoint{1.789894in}{2.614590in}}%
\pgfpathlineto{\pgfqpoint{1.794435in}{2.614590in}}%
\pgfpathlineto{\pgfqpoint{1.794435in}{2.611640in}}%
\pgfpathmoveto{\pgfqpoint{1.794435in}{2.611640in}}%
\pgfpathlineto{\pgfqpoint{1.794435in}{2.611640in}}%
\pgfpathlineto{\pgfqpoint{1.794435in}{2.614590in}}%
\pgfpathlineto{\pgfqpoint{1.798976in}{2.614590in}}%
\pgfpathlineto{\pgfqpoint{1.798976in}{2.611640in}}%
\pgfpathmoveto{\pgfqpoint{1.798976in}{2.611640in}}%
\pgfpathlineto{\pgfqpoint{1.798976in}{2.611640in}}%
\pgfpathlineto{\pgfqpoint{1.798976in}{2.614590in}}%
\pgfpathlineto{\pgfqpoint{1.803517in}{2.614590in}}%
\pgfpathlineto{\pgfqpoint{1.803517in}{2.611640in}}%
\pgfpathmoveto{\pgfqpoint{1.803517in}{2.611640in}}%
\pgfpathlineto{\pgfqpoint{1.803517in}{2.611640in}}%
\pgfpathlineto{\pgfqpoint{1.803517in}{2.614590in}}%
\pgfpathlineto{\pgfqpoint{1.808058in}{2.614590in}}%
\pgfpathlineto{\pgfqpoint{1.808058in}{2.611640in}}%
\pgfpathmoveto{\pgfqpoint{1.808058in}{2.611640in}}%
\pgfpathlineto{\pgfqpoint{1.808058in}{2.611640in}}%
\pgfpathlineto{\pgfqpoint{1.808058in}{2.614590in}}%
\pgfpathlineto{\pgfqpoint{1.812598in}{2.614590in}}%
\pgfpathlineto{\pgfqpoint{1.812598in}{2.611640in}}%
\pgfpathmoveto{\pgfqpoint{1.812598in}{2.611640in}}%
\pgfpathlineto{\pgfqpoint{1.812598in}{2.611640in}}%
\pgfpathlineto{\pgfqpoint{1.812598in}{2.614590in}}%
\pgfpathlineto{\pgfqpoint{1.817139in}{2.614590in}}%
\pgfpathlineto{\pgfqpoint{1.817139in}{2.611640in}}%
\pgfpathmoveto{\pgfqpoint{1.817139in}{2.611640in}}%
\pgfpathlineto{\pgfqpoint{1.817139in}{2.611640in}}%
\pgfpathlineto{\pgfqpoint{1.817139in}{2.614590in}}%
\pgfpathlineto{\pgfqpoint{1.821680in}{2.614590in}}%
\pgfpathlineto{\pgfqpoint{1.821680in}{2.611640in}}%
\pgfpathmoveto{\pgfqpoint{1.821680in}{2.611640in}}%
\pgfpathlineto{\pgfqpoint{1.821680in}{2.611640in}}%
\pgfpathlineto{\pgfqpoint{1.821680in}{2.614590in}}%
\pgfpathlineto{\pgfqpoint{1.826221in}{2.614590in}}%
\pgfpathlineto{\pgfqpoint{1.826221in}{2.611640in}}%
\pgfpathmoveto{\pgfqpoint{1.826221in}{2.611640in}}%
\pgfpathlineto{\pgfqpoint{1.826221in}{2.611640in}}%
\pgfpathlineto{\pgfqpoint{1.826221in}{2.614590in}}%
\pgfpathlineto{\pgfqpoint{1.830762in}{2.614590in}}%
\pgfpathlineto{\pgfqpoint{1.830762in}{2.611640in}}%
\pgfpathmoveto{\pgfqpoint{1.830762in}{2.611640in}}%
\pgfpathlineto{\pgfqpoint{1.830762in}{2.611640in}}%
\pgfpathlineto{\pgfqpoint{1.830762in}{2.614590in}}%
\pgfpathlineto{\pgfqpoint{1.835303in}{2.614590in}}%
\pgfpathlineto{\pgfqpoint{1.835303in}{2.611640in}}%
\pgfpathmoveto{\pgfqpoint{1.835303in}{2.611640in}}%
\pgfpathlineto{\pgfqpoint{1.835303in}{2.611640in}}%
\pgfpathlineto{\pgfqpoint{1.835303in}{2.614590in}}%
\pgfpathlineto{\pgfqpoint{1.839844in}{2.614590in}}%
\pgfpathlineto{\pgfqpoint{1.839844in}{2.611640in}}%
\pgfpathmoveto{\pgfqpoint{1.839844in}{2.611640in}}%
\pgfpathlineto{\pgfqpoint{1.839844in}{2.611640in}}%
\pgfpathlineto{\pgfqpoint{1.839844in}{2.614590in}}%
\pgfpathlineto{\pgfqpoint{1.844384in}{2.614590in}}%
\pgfpathlineto{\pgfqpoint{1.844384in}{2.611640in}}%
\pgfpathmoveto{\pgfqpoint{1.844384in}{2.611640in}}%
\pgfpathlineto{\pgfqpoint{1.844384in}{2.611640in}}%
\pgfpathlineto{\pgfqpoint{1.844384in}{2.614590in}}%
\pgfpathlineto{\pgfqpoint{1.848925in}{2.614590in}}%
\pgfpathlineto{\pgfqpoint{1.848925in}{2.611640in}}%
\pgfpathmoveto{\pgfqpoint{1.848925in}{2.611640in}}%
\pgfpathlineto{\pgfqpoint{1.848925in}{2.611640in}}%
\pgfpathlineto{\pgfqpoint{1.848925in}{2.614590in}}%
\pgfpathlineto{\pgfqpoint{1.853466in}{2.614590in}}%
\pgfpathlineto{\pgfqpoint{1.853466in}{2.611640in}}%
\pgfpathmoveto{\pgfqpoint{1.853466in}{2.611640in}}%
\pgfpathlineto{\pgfqpoint{1.853466in}{2.611640in}}%
\pgfpathlineto{\pgfqpoint{1.853466in}{2.614590in}}%
\pgfpathlineto{\pgfqpoint{1.858007in}{2.614590in}}%
\pgfpathlineto{\pgfqpoint{1.858007in}{2.611640in}}%
\pgfpathmoveto{\pgfqpoint{1.858007in}{2.611640in}}%
\pgfpathlineto{\pgfqpoint{1.858007in}{2.611640in}}%
\pgfpathlineto{\pgfqpoint{1.858007in}{2.614590in}}%
\pgfpathlineto{\pgfqpoint{1.862548in}{2.614590in}}%
\pgfpathlineto{\pgfqpoint{1.862548in}{2.611640in}}%
\pgfpathmoveto{\pgfqpoint{1.862548in}{2.611640in}}%
\pgfpathlineto{\pgfqpoint{1.862548in}{2.611640in}}%
\pgfpathlineto{\pgfqpoint{1.862548in}{2.614590in}}%
\pgfpathlineto{\pgfqpoint{1.867089in}{2.614590in}}%
\pgfpathlineto{\pgfqpoint{1.867089in}{2.611640in}}%
\pgfpathmoveto{\pgfqpoint{1.867089in}{2.611640in}}%
\pgfpathlineto{\pgfqpoint{1.867089in}{2.611640in}}%
\pgfpathlineto{\pgfqpoint{1.867089in}{2.614590in}}%
\pgfpathlineto{\pgfqpoint{1.871630in}{2.614590in}}%
\pgfpathlineto{\pgfqpoint{1.871630in}{2.611640in}}%
\pgfpathmoveto{\pgfqpoint{1.871630in}{2.611640in}}%
\pgfpathlineto{\pgfqpoint{1.871630in}{2.611640in}}%
\pgfpathlineto{\pgfqpoint{1.871630in}{2.614590in}}%
\pgfpathlineto{\pgfqpoint{1.876170in}{2.614590in}}%
\pgfpathlineto{\pgfqpoint{1.876170in}{2.611640in}}%
\pgfpathmoveto{\pgfqpoint{1.876170in}{2.611640in}}%
\pgfpathlineto{\pgfqpoint{1.876170in}{2.611640in}}%
\pgfpathlineto{\pgfqpoint{1.876170in}{2.614590in}}%
\pgfpathlineto{\pgfqpoint{1.880711in}{2.614590in}}%
\pgfpathlineto{\pgfqpoint{1.880711in}{2.611640in}}%
\pgfpathmoveto{\pgfqpoint{1.880711in}{2.611640in}}%
\pgfpathlineto{\pgfqpoint{1.880711in}{2.611640in}}%
\pgfpathlineto{\pgfqpoint{1.880711in}{2.614590in}}%
\pgfpathlineto{\pgfqpoint{1.885252in}{2.614590in}}%
\pgfpathlineto{\pgfqpoint{1.885252in}{2.611640in}}%
\pgfpathmoveto{\pgfqpoint{1.885252in}{2.611640in}}%
\pgfpathlineto{\pgfqpoint{1.885252in}{2.611640in}}%
\pgfpathlineto{\pgfqpoint{1.885252in}{2.614590in}}%
\pgfpathlineto{\pgfqpoint{1.889793in}{2.614590in}}%
\pgfpathlineto{\pgfqpoint{1.889793in}{2.611640in}}%
\pgfpathmoveto{\pgfqpoint{1.889793in}{2.611640in}}%
\pgfpathlineto{\pgfqpoint{1.889793in}{2.611640in}}%
\pgfpathlineto{\pgfqpoint{1.889793in}{2.614590in}}%
\pgfpathlineto{\pgfqpoint{1.894334in}{2.614590in}}%
\pgfpathlineto{\pgfqpoint{1.894334in}{2.611640in}}%
\pgfpathmoveto{\pgfqpoint{1.894334in}{2.611640in}}%
\pgfpathlineto{\pgfqpoint{1.894334in}{2.611640in}}%
\pgfpathlineto{\pgfqpoint{1.894334in}{2.614590in}}%
\pgfpathlineto{\pgfqpoint{1.898875in}{2.614590in}}%
\pgfpathlineto{\pgfqpoint{1.898875in}{2.611640in}}%
\pgfpathmoveto{\pgfqpoint{1.898875in}{2.611640in}}%
\pgfpathlineto{\pgfqpoint{1.898875in}{2.611640in}}%
\pgfpathlineto{\pgfqpoint{1.898875in}{2.614590in}}%
\pgfpathlineto{\pgfqpoint{1.903415in}{2.614590in}}%
\pgfpathlineto{\pgfqpoint{1.903415in}{2.611640in}}%
\pgfpathmoveto{\pgfqpoint{1.903415in}{2.611640in}}%
\pgfpathlineto{\pgfqpoint{1.903415in}{2.611640in}}%
\pgfpathlineto{\pgfqpoint{1.903415in}{2.614590in}}%
\pgfpathlineto{\pgfqpoint{1.907956in}{2.614590in}}%
\pgfpathlineto{\pgfqpoint{1.907956in}{2.611640in}}%
\pgfpathmoveto{\pgfqpoint{1.907956in}{2.611640in}}%
\pgfpathlineto{\pgfqpoint{1.907956in}{2.611640in}}%
\pgfpathlineto{\pgfqpoint{1.907956in}{2.614590in}}%
\pgfpathlineto{\pgfqpoint{1.912497in}{2.614590in}}%
\pgfpathlineto{\pgfqpoint{1.912497in}{2.611640in}}%
\pgfpathmoveto{\pgfqpoint{1.912497in}{2.611640in}}%
\pgfpathlineto{\pgfqpoint{1.912497in}{2.611640in}}%
\pgfpathlineto{\pgfqpoint{1.912497in}{2.614590in}}%
\pgfpathlineto{\pgfqpoint{1.917038in}{2.614590in}}%
\pgfpathlineto{\pgfqpoint{1.917038in}{2.611640in}}%
\pgfpathmoveto{\pgfqpoint{1.917038in}{2.611640in}}%
\pgfpathlineto{\pgfqpoint{1.917038in}{2.611640in}}%
\pgfpathlineto{\pgfqpoint{1.917038in}{2.614590in}}%
\pgfpathlineto{\pgfqpoint{1.921579in}{2.614590in}}%
\pgfpathlineto{\pgfqpoint{1.921579in}{2.611640in}}%
\pgfpathmoveto{\pgfqpoint{1.921579in}{2.611640in}}%
\pgfpathlineto{\pgfqpoint{1.921579in}{2.611640in}}%
\pgfpathlineto{\pgfqpoint{1.921579in}{2.614590in}}%
\pgfpathlineto{\pgfqpoint{1.926120in}{2.614590in}}%
\pgfpathlineto{\pgfqpoint{1.926120in}{2.611640in}}%
\pgfpathmoveto{\pgfqpoint{1.926120in}{2.611640in}}%
\pgfpathlineto{\pgfqpoint{1.926120in}{2.611640in}}%
\pgfpathlineto{\pgfqpoint{1.926120in}{2.614590in}}%
\pgfpathlineto{\pgfqpoint{1.930662in}{2.614590in}}%
\pgfpathlineto{\pgfqpoint{1.930662in}{2.611640in}}%
\pgfpathmoveto{\pgfqpoint{1.930662in}{2.611640in}}%
\pgfpathlineto{\pgfqpoint{1.930662in}{2.611640in}}%
\pgfpathlineto{\pgfqpoint{1.930662in}{2.614590in}}%
\pgfpathlineto{\pgfqpoint{1.935203in}{2.614590in}}%
\pgfpathlineto{\pgfqpoint{1.935203in}{2.611640in}}%
\pgfpathmoveto{\pgfqpoint{1.935203in}{2.611640in}}%
\pgfpathlineto{\pgfqpoint{1.935203in}{2.611640in}}%
\pgfpathlineto{\pgfqpoint{1.935203in}{2.614590in}}%
\pgfpathlineto{\pgfqpoint{1.939744in}{2.614590in}}%
\pgfpathlineto{\pgfqpoint{1.939744in}{2.611640in}}%
\pgfpathmoveto{\pgfqpoint{1.939744in}{2.611640in}}%
\pgfpathlineto{\pgfqpoint{1.939744in}{2.611640in}}%
\pgfpathlineto{\pgfqpoint{1.939744in}{2.614590in}}%
\pgfpathlineto{\pgfqpoint{1.944285in}{2.614590in}}%
\pgfpathlineto{\pgfqpoint{1.944285in}{2.611640in}}%
\pgfpathmoveto{\pgfqpoint{1.944285in}{2.611640in}}%
\pgfpathlineto{\pgfqpoint{1.944285in}{2.611640in}}%
\pgfpathlineto{\pgfqpoint{1.944285in}{2.614590in}}%
\pgfpathlineto{\pgfqpoint{1.948826in}{2.614590in}}%
\pgfpathlineto{\pgfqpoint{1.948826in}{2.611640in}}%
\pgfpathmoveto{\pgfqpoint{1.948826in}{2.611640in}}%
\pgfpathlineto{\pgfqpoint{1.948826in}{2.611640in}}%
\pgfpathlineto{\pgfqpoint{1.948826in}{2.614590in}}%
\pgfpathlineto{\pgfqpoint{1.953367in}{2.614590in}}%
\pgfpathlineto{\pgfqpoint{1.953367in}{2.611640in}}%
\pgfpathmoveto{\pgfqpoint{1.953367in}{2.611640in}}%
\pgfpathlineto{\pgfqpoint{1.953367in}{2.611640in}}%
\pgfpathlineto{\pgfqpoint{1.953367in}{2.614590in}}%
\pgfpathlineto{\pgfqpoint{1.957908in}{2.614590in}}%
\pgfpathlineto{\pgfqpoint{1.957908in}{2.611640in}}%
\pgfpathmoveto{\pgfqpoint{1.957908in}{2.611640in}}%
\pgfpathlineto{\pgfqpoint{1.957908in}{2.611640in}}%
\pgfpathlineto{\pgfqpoint{1.957908in}{2.614590in}}%
\pgfpathlineto{\pgfqpoint{1.962449in}{2.614590in}}%
\pgfpathlineto{\pgfqpoint{1.962449in}{2.611640in}}%
\pgfpathmoveto{\pgfqpoint{1.962449in}{2.611640in}}%
\pgfpathlineto{\pgfqpoint{1.962449in}{2.611640in}}%
\pgfpathlineto{\pgfqpoint{1.962449in}{2.614590in}}%
\pgfpathlineto{\pgfqpoint{1.966990in}{2.614590in}}%
\pgfpathlineto{\pgfqpoint{1.966990in}{2.611640in}}%
\pgfpathmoveto{\pgfqpoint{1.966990in}{2.611640in}}%
\pgfpathlineto{\pgfqpoint{1.966990in}{2.611640in}}%
\pgfpathlineto{\pgfqpoint{1.966990in}{2.614590in}}%
\pgfpathlineto{\pgfqpoint{1.971532in}{2.614590in}}%
\pgfpathlineto{\pgfqpoint{1.971532in}{2.611640in}}%
\pgfpathmoveto{\pgfqpoint{1.971532in}{2.611640in}}%
\pgfpathlineto{\pgfqpoint{1.971532in}{2.611640in}}%
\pgfpathlineto{\pgfqpoint{1.971532in}{2.614590in}}%
\pgfpathlineto{\pgfqpoint{1.976073in}{2.614590in}}%
\pgfpathlineto{\pgfqpoint{1.976073in}{2.611640in}}%
\pgfpathmoveto{\pgfqpoint{1.976073in}{2.611640in}}%
\pgfpathlineto{\pgfqpoint{1.976073in}{2.611640in}}%
\pgfpathlineto{\pgfqpoint{1.976073in}{2.614590in}}%
\pgfpathlineto{\pgfqpoint{1.980614in}{2.614590in}}%
\pgfpathlineto{\pgfqpoint{1.980614in}{2.611640in}}%
\pgfpathmoveto{\pgfqpoint{1.980614in}{2.611640in}}%
\pgfpathlineto{\pgfqpoint{1.980614in}{2.611640in}}%
\pgfpathlineto{\pgfqpoint{1.980614in}{2.614590in}}%
\pgfpathlineto{\pgfqpoint{1.985155in}{2.614590in}}%
\pgfpathlineto{\pgfqpoint{1.985155in}{2.611640in}}%
\pgfpathmoveto{\pgfqpoint{1.985155in}{2.611640in}}%
\pgfpathlineto{\pgfqpoint{1.985155in}{2.611640in}}%
\pgfpathlineto{\pgfqpoint{1.985155in}{2.614590in}}%
\pgfpathlineto{\pgfqpoint{1.989696in}{2.614590in}}%
\pgfpathlineto{\pgfqpoint{1.989696in}{2.611640in}}%
\pgfpathmoveto{\pgfqpoint{1.989696in}{2.611640in}}%
\pgfpathlineto{\pgfqpoint{1.989696in}{2.611640in}}%
\pgfpathlineto{\pgfqpoint{1.989696in}{2.614590in}}%
\pgfpathlineto{\pgfqpoint{1.994237in}{2.614590in}}%
\pgfpathlineto{\pgfqpoint{1.994237in}{2.611640in}}%
\pgfpathmoveto{\pgfqpoint{1.994237in}{2.611640in}}%
\pgfpathlineto{\pgfqpoint{1.994237in}{2.611640in}}%
\pgfpathlineto{\pgfqpoint{1.994237in}{2.614590in}}%
\pgfpathlineto{\pgfqpoint{1.998778in}{2.614590in}}%
\pgfpathlineto{\pgfqpoint{1.998778in}{2.611640in}}%
\pgfpathmoveto{\pgfqpoint{1.998778in}{2.611640in}}%
\pgfpathlineto{\pgfqpoint{1.998778in}{2.611640in}}%
\pgfpathlineto{\pgfqpoint{1.998778in}{2.614590in}}%
\pgfpathlineto{\pgfqpoint{2.003319in}{2.614590in}}%
\pgfpathlineto{\pgfqpoint{2.003319in}{2.611640in}}%
\pgfpathmoveto{\pgfqpoint{2.003319in}{2.611640in}}%
\pgfpathlineto{\pgfqpoint{2.003319in}{2.611640in}}%
\pgfpathlineto{\pgfqpoint{2.003319in}{2.614590in}}%
\pgfpathlineto{\pgfqpoint{2.007860in}{2.614590in}}%
\pgfpathlineto{\pgfqpoint{2.007860in}{2.611640in}}%
\pgfpathmoveto{\pgfqpoint{2.007860in}{2.611640in}}%
\pgfpathlineto{\pgfqpoint{2.007860in}{2.611640in}}%
\pgfpathlineto{\pgfqpoint{2.007860in}{2.614590in}}%
\pgfpathlineto{\pgfqpoint{2.012401in}{2.614590in}}%
\pgfpathlineto{\pgfqpoint{2.012401in}{2.611640in}}%
\pgfpathmoveto{\pgfqpoint{2.012401in}{2.611640in}}%
\pgfpathlineto{\pgfqpoint{2.012401in}{2.611640in}}%
\pgfpathlineto{\pgfqpoint{2.012401in}{2.614590in}}%
\pgfpathlineto{\pgfqpoint{2.016943in}{2.614590in}}%
\pgfpathlineto{\pgfqpoint{2.016943in}{2.611640in}}%
\pgfpathmoveto{\pgfqpoint{2.016943in}{2.611640in}}%
\pgfpathlineto{\pgfqpoint{2.016943in}{2.611640in}}%
\pgfpathlineto{\pgfqpoint{2.016943in}{2.614590in}}%
\pgfpathlineto{\pgfqpoint{2.021484in}{2.614590in}}%
\pgfpathlineto{\pgfqpoint{2.021484in}{2.611640in}}%
\pgfpathmoveto{\pgfqpoint{2.021484in}{2.611640in}}%
\pgfpathlineto{\pgfqpoint{2.021484in}{2.611640in}}%
\pgfpathlineto{\pgfqpoint{2.021484in}{2.614590in}}%
\pgfpathlineto{\pgfqpoint{2.026025in}{2.614590in}}%
\pgfpathlineto{\pgfqpoint{2.026025in}{2.611640in}}%
\pgfpathmoveto{\pgfqpoint{2.026025in}{2.611640in}}%
\pgfpathlineto{\pgfqpoint{2.026025in}{2.611640in}}%
\pgfpathlineto{\pgfqpoint{2.026025in}{2.614590in}}%
\pgfpathlineto{\pgfqpoint{2.030566in}{2.614590in}}%
\pgfpathlineto{\pgfqpoint{2.030566in}{2.611640in}}%
\pgfpathmoveto{\pgfqpoint{2.030566in}{2.611640in}}%
\pgfpathlineto{\pgfqpoint{2.030566in}{2.611640in}}%
\pgfpathlineto{\pgfqpoint{2.030566in}{2.614590in}}%
\pgfpathlineto{\pgfqpoint{2.035107in}{2.614590in}}%
\pgfpathlineto{\pgfqpoint{2.035107in}{2.611640in}}%
\pgfpathmoveto{\pgfqpoint{2.035107in}{2.611640in}}%
\pgfpathlineto{\pgfqpoint{2.035107in}{2.611640in}}%
\pgfpathlineto{\pgfqpoint{2.035107in}{2.614590in}}%
\pgfpathlineto{\pgfqpoint{2.039648in}{2.614590in}}%
\pgfpathlineto{\pgfqpoint{2.039648in}{2.611640in}}%
\pgfpathmoveto{\pgfqpoint{2.039648in}{2.611640in}}%
\pgfpathlineto{\pgfqpoint{2.039648in}{2.611640in}}%
\pgfpathlineto{\pgfqpoint{2.039648in}{2.614590in}}%
\pgfpathlineto{\pgfqpoint{2.044189in}{2.614590in}}%
\pgfpathlineto{\pgfqpoint{2.044189in}{2.611640in}}%
\pgfpathmoveto{\pgfqpoint{2.044189in}{2.611640in}}%
\pgfpathlineto{\pgfqpoint{2.044189in}{2.611640in}}%
\pgfpathlineto{\pgfqpoint{2.044189in}{2.614590in}}%
\pgfpathlineto{\pgfqpoint{2.048730in}{2.614590in}}%
\pgfpathlineto{\pgfqpoint{2.048730in}{2.611640in}}%
\pgfpathmoveto{\pgfqpoint{2.048730in}{2.611640in}}%
\pgfpathlineto{\pgfqpoint{2.048730in}{2.611640in}}%
\pgfpathlineto{\pgfqpoint{2.048730in}{2.614590in}}%
\pgfpathlineto{\pgfqpoint{2.053271in}{2.614590in}}%
\pgfpathlineto{\pgfqpoint{2.053271in}{2.611640in}}%
\pgfpathmoveto{\pgfqpoint{2.053271in}{2.611640in}}%
\pgfpathlineto{\pgfqpoint{2.053271in}{2.611640in}}%
\pgfpathlineto{\pgfqpoint{2.053271in}{2.614590in}}%
\pgfpathlineto{\pgfqpoint{2.057812in}{2.614590in}}%
\pgfpathlineto{\pgfqpoint{2.057812in}{2.611640in}}%
\pgfpathmoveto{\pgfqpoint{2.057812in}{2.611640in}}%
\pgfpathlineto{\pgfqpoint{2.057812in}{2.611640in}}%
\pgfpathlineto{\pgfqpoint{2.057812in}{2.614590in}}%
\pgfpathlineto{\pgfqpoint{2.062353in}{2.614590in}}%
\pgfpathlineto{\pgfqpoint{2.062353in}{2.611640in}}%
\pgfpathmoveto{\pgfqpoint{2.062353in}{2.611640in}}%
\pgfpathlineto{\pgfqpoint{2.062353in}{2.611640in}}%
\pgfpathlineto{\pgfqpoint{2.062353in}{2.614590in}}%
\pgfpathlineto{\pgfqpoint{2.066894in}{2.614590in}}%
\pgfpathlineto{\pgfqpoint{2.066894in}{2.611640in}}%
\pgfpathmoveto{\pgfqpoint{2.066894in}{2.611640in}}%
\pgfpathlineto{\pgfqpoint{2.066894in}{2.611640in}}%
\pgfpathlineto{\pgfqpoint{2.066894in}{2.614590in}}%
\pgfpathlineto{\pgfqpoint{2.071435in}{2.614590in}}%
\pgfpathlineto{\pgfqpoint{2.071435in}{2.611640in}}%
\pgfpathmoveto{\pgfqpoint{2.071435in}{2.611640in}}%
\pgfpathlineto{\pgfqpoint{2.071435in}{2.611640in}}%
\pgfpathlineto{\pgfqpoint{2.071435in}{2.614590in}}%
\pgfpathlineto{\pgfqpoint{2.075976in}{2.614590in}}%
\pgfpathlineto{\pgfqpoint{2.075976in}{2.611640in}}%
\pgfpathmoveto{\pgfqpoint{2.075976in}{2.611640in}}%
\pgfpathlineto{\pgfqpoint{2.075976in}{2.611640in}}%
\pgfpathlineto{\pgfqpoint{2.075976in}{2.614590in}}%
\pgfpathlineto{\pgfqpoint{2.080517in}{2.614590in}}%
\pgfpathlineto{\pgfqpoint{2.080517in}{2.611640in}}%
\pgfpathmoveto{\pgfqpoint{2.080517in}{2.611640in}}%
\pgfpathlineto{\pgfqpoint{2.080517in}{2.611640in}}%
\pgfpathlineto{\pgfqpoint{2.080517in}{2.614590in}}%
\pgfpathlineto{\pgfqpoint{2.085058in}{2.614590in}}%
\pgfpathlineto{\pgfqpoint{2.085058in}{2.611640in}}%
\pgfpathmoveto{\pgfqpoint{2.085058in}{2.611640in}}%
\pgfpathlineto{\pgfqpoint{2.085058in}{2.611640in}}%
\pgfpathlineto{\pgfqpoint{2.085058in}{2.614590in}}%
\pgfpathlineto{\pgfqpoint{2.089599in}{2.614590in}}%
\pgfpathlineto{\pgfqpoint{2.089599in}{2.611640in}}%
\pgfpathmoveto{\pgfqpoint{2.089599in}{2.611640in}}%
\pgfpathlineto{\pgfqpoint{2.089599in}{2.611640in}}%
\pgfpathlineto{\pgfqpoint{2.089599in}{2.614590in}}%
\pgfpathlineto{\pgfqpoint{2.094140in}{2.614590in}}%
\pgfpathlineto{\pgfqpoint{2.094140in}{2.611640in}}%
\pgfpathmoveto{\pgfqpoint{2.094140in}{2.611640in}}%
\pgfpathlineto{\pgfqpoint{2.094140in}{2.611640in}}%
\pgfpathlineto{\pgfqpoint{2.094140in}{2.614590in}}%
\pgfpathlineto{\pgfqpoint{2.098681in}{2.614590in}}%
\pgfpathlineto{\pgfqpoint{2.098681in}{2.611640in}}%
\pgfpathmoveto{\pgfqpoint{2.098681in}{2.611640in}}%
\pgfpathlineto{\pgfqpoint{2.098681in}{2.611640in}}%
\pgfpathlineto{\pgfqpoint{2.098681in}{2.614590in}}%
\pgfpathlineto{\pgfqpoint{2.103222in}{2.614590in}}%
\pgfpathlineto{\pgfqpoint{2.103222in}{2.611640in}}%
\pgfpathmoveto{\pgfqpoint{2.103222in}{2.611640in}}%
\pgfpathlineto{\pgfqpoint{2.103222in}{2.611640in}}%
\pgfpathlineto{\pgfqpoint{2.103222in}{2.614590in}}%
\pgfpathlineto{\pgfqpoint{2.107763in}{2.614590in}}%
\pgfpathlineto{\pgfqpoint{2.107763in}{2.611640in}}%
\pgfpathmoveto{\pgfqpoint{2.107763in}{2.611640in}}%
\pgfpathlineto{\pgfqpoint{2.107763in}{2.611640in}}%
\pgfpathlineto{\pgfqpoint{2.107763in}{2.614590in}}%
\pgfpathlineto{\pgfqpoint{2.112304in}{2.614590in}}%
\pgfpathlineto{\pgfqpoint{2.112304in}{2.611640in}}%
\pgfpathmoveto{\pgfqpoint{2.112304in}{2.611640in}}%
\pgfpathlineto{\pgfqpoint{2.112304in}{2.611640in}}%
\pgfpathlineto{\pgfqpoint{2.112304in}{2.614590in}}%
\pgfpathlineto{\pgfqpoint{2.116845in}{2.614590in}}%
\pgfpathlineto{\pgfqpoint{2.116845in}{2.611640in}}%
\pgfpathmoveto{\pgfqpoint{2.116845in}{2.611640in}}%
\pgfpathlineto{\pgfqpoint{2.116845in}{2.611640in}}%
\pgfpathlineto{\pgfqpoint{2.116845in}{2.614590in}}%
\pgfpathlineto{\pgfqpoint{2.121386in}{2.614590in}}%
\pgfpathlineto{\pgfqpoint{2.121386in}{2.611640in}}%
\pgfpathmoveto{\pgfqpoint{2.121386in}{2.611640in}}%
\pgfpathlineto{\pgfqpoint{2.121386in}{2.611640in}}%
\pgfpathlineto{\pgfqpoint{2.121386in}{2.614590in}}%
\pgfpathlineto{\pgfqpoint{2.125927in}{2.614590in}}%
\pgfpathlineto{\pgfqpoint{2.125927in}{2.611640in}}%
\pgfpathmoveto{\pgfqpoint{2.125927in}{2.611640in}}%
\pgfpathlineto{\pgfqpoint{2.125927in}{2.611640in}}%
\pgfpathlineto{\pgfqpoint{2.125927in}{2.614590in}}%
\pgfpathlineto{\pgfqpoint{2.130468in}{2.614590in}}%
\pgfpathlineto{\pgfqpoint{2.130468in}{2.611640in}}%
\pgfpathmoveto{\pgfqpoint{2.130468in}{2.611640in}}%
\pgfpathlineto{\pgfqpoint{2.130468in}{2.611640in}}%
\pgfpathlineto{\pgfqpoint{2.130468in}{2.614590in}}%
\pgfpathlineto{\pgfqpoint{2.135009in}{2.614590in}}%
\pgfpathlineto{\pgfqpoint{2.135009in}{2.611640in}}%
\pgfpathmoveto{\pgfqpoint{2.135009in}{2.611640in}}%
\pgfpathlineto{\pgfqpoint{2.135009in}{2.611640in}}%
\pgfpathlineto{\pgfqpoint{2.135009in}{2.614590in}}%
\pgfpathlineto{\pgfqpoint{2.139550in}{2.614590in}}%
\pgfpathlineto{\pgfqpoint{2.139550in}{2.611640in}}%
\pgfpathmoveto{\pgfqpoint{2.139550in}{2.611640in}}%
\pgfpathlineto{\pgfqpoint{2.139550in}{2.611640in}}%
\pgfpathlineto{\pgfqpoint{2.139550in}{2.614590in}}%
\pgfpathlineto{\pgfqpoint{2.144091in}{2.614590in}}%
\pgfpathlineto{\pgfqpoint{2.144091in}{2.611640in}}%
\pgfpathmoveto{\pgfqpoint{2.144091in}{2.611640in}}%
\pgfpathlineto{\pgfqpoint{2.144091in}{2.611640in}}%
\pgfpathlineto{\pgfqpoint{2.144091in}{2.614590in}}%
\pgfpathlineto{\pgfqpoint{2.148632in}{2.614590in}}%
\pgfpathlineto{\pgfqpoint{2.148632in}{2.611640in}}%
\pgfpathmoveto{\pgfqpoint{2.148632in}{2.611640in}}%
\pgfpathlineto{\pgfqpoint{2.148632in}{2.611640in}}%
\pgfpathlineto{\pgfqpoint{2.148632in}{2.614590in}}%
\pgfpathlineto{\pgfqpoint{2.153173in}{2.614590in}}%
\pgfpathlineto{\pgfqpoint{2.153173in}{2.611640in}}%
\pgfpathmoveto{\pgfqpoint{2.153173in}{2.611640in}}%
\pgfpathlineto{\pgfqpoint{2.153173in}{2.611640in}}%
\pgfpathlineto{\pgfqpoint{2.153173in}{2.614590in}}%
\pgfpathlineto{\pgfqpoint{2.157714in}{2.614590in}}%
\pgfpathlineto{\pgfqpoint{2.157714in}{2.611640in}}%
\pgfpathmoveto{\pgfqpoint{2.157714in}{2.611640in}}%
\pgfpathlineto{\pgfqpoint{2.157714in}{2.611640in}}%
\pgfpathlineto{\pgfqpoint{2.157714in}{2.614590in}}%
\pgfpathlineto{\pgfqpoint{2.162255in}{2.614590in}}%
\pgfpathlineto{\pgfqpoint{2.162255in}{2.611640in}}%
\pgfpathmoveto{\pgfqpoint{2.162255in}{2.611640in}}%
\pgfpathlineto{\pgfqpoint{2.162255in}{2.611640in}}%
\pgfpathlineto{\pgfqpoint{2.162255in}{2.614590in}}%
\pgfpathlineto{\pgfqpoint{2.166796in}{2.614590in}}%
\pgfpathlineto{\pgfqpoint{2.166796in}{2.611640in}}%
\pgfpathmoveto{\pgfqpoint{2.166796in}{2.611640in}}%
\pgfpathlineto{\pgfqpoint{2.166796in}{2.611640in}}%
\pgfpathlineto{\pgfqpoint{2.166796in}{2.614590in}}%
\pgfpathlineto{\pgfqpoint{2.171337in}{2.614590in}}%
\pgfpathlineto{\pgfqpoint{2.171337in}{2.611640in}}%
\pgfpathmoveto{\pgfqpoint{2.171337in}{2.611640in}}%
\pgfpathlineto{\pgfqpoint{2.171337in}{2.611640in}}%
\pgfpathlineto{\pgfqpoint{2.171337in}{2.614590in}}%
\pgfpathlineto{\pgfqpoint{2.175878in}{2.614590in}}%
\pgfpathlineto{\pgfqpoint{2.175878in}{2.611640in}}%
\pgfpathmoveto{\pgfqpoint{2.175878in}{2.611640in}}%
\pgfpathlineto{\pgfqpoint{2.175878in}{2.611640in}}%
\pgfpathlineto{\pgfqpoint{2.175878in}{2.614590in}}%
\pgfpathlineto{\pgfqpoint{2.180419in}{2.614590in}}%
\pgfpathlineto{\pgfqpoint{2.180419in}{2.611640in}}%
\pgfpathmoveto{\pgfqpoint{2.180419in}{2.611640in}}%
\pgfpathlineto{\pgfqpoint{2.180419in}{2.611640in}}%
\pgfpathlineto{\pgfqpoint{2.180419in}{2.614590in}}%
\pgfpathlineto{\pgfqpoint{2.184960in}{2.614590in}}%
\pgfpathlineto{\pgfqpoint{2.184960in}{2.611640in}}%
\pgfpathmoveto{\pgfqpoint{2.184960in}{2.611640in}}%
\pgfpathlineto{\pgfqpoint{2.184960in}{2.611640in}}%
\pgfpathlineto{\pgfqpoint{2.184960in}{2.614590in}}%
\pgfpathlineto{\pgfqpoint{2.189501in}{2.614590in}}%
\pgfpathlineto{\pgfqpoint{2.189501in}{2.611640in}}%
\pgfpathmoveto{\pgfqpoint{2.189501in}{2.611640in}}%
\pgfpathlineto{\pgfqpoint{2.189501in}{2.611640in}}%
\pgfpathlineto{\pgfqpoint{2.189501in}{2.614590in}}%
\pgfpathlineto{\pgfqpoint{2.194042in}{2.614590in}}%
\pgfpathlineto{\pgfqpoint{2.194042in}{2.611640in}}%
\pgfpathmoveto{\pgfqpoint{2.194042in}{2.611640in}}%
\pgfpathlineto{\pgfqpoint{2.194042in}{2.611640in}}%
\pgfpathlineto{\pgfqpoint{2.194042in}{2.614590in}}%
\pgfpathlineto{\pgfqpoint{2.198583in}{2.614590in}}%
\pgfpathlineto{\pgfqpoint{2.198583in}{2.611640in}}%
\pgfpathmoveto{\pgfqpoint{2.198583in}{2.611640in}}%
\pgfpathlineto{\pgfqpoint{2.198583in}{2.611640in}}%
\pgfpathlineto{\pgfqpoint{2.198583in}{2.614590in}}%
\pgfpathlineto{\pgfqpoint{2.203124in}{2.614590in}}%
\pgfpathlineto{\pgfqpoint{2.203124in}{2.611640in}}%
\pgfpathmoveto{\pgfqpoint{2.203124in}{2.611640in}}%
\pgfpathlineto{\pgfqpoint{2.203124in}{2.611640in}}%
\pgfpathlineto{\pgfqpoint{2.203124in}{2.614590in}}%
\pgfpathlineto{\pgfqpoint{2.207665in}{2.614590in}}%
\pgfpathlineto{\pgfqpoint{2.207665in}{2.611640in}}%
\pgfpathmoveto{\pgfqpoint{2.207665in}{2.611640in}}%
\pgfpathlineto{\pgfqpoint{2.207665in}{2.611640in}}%
\pgfpathlineto{\pgfqpoint{2.207665in}{2.614590in}}%
\pgfpathlineto{\pgfqpoint{2.212206in}{2.614590in}}%
\pgfpathlineto{\pgfqpoint{2.212206in}{2.611640in}}%
\pgfpathmoveto{\pgfqpoint{2.212206in}{2.611640in}}%
\pgfpathlineto{\pgfqpoint{2.212206in}{2.611640in}}%
\pgfpathlineto{\pgfqpoint{2.212206in}{2.614590in}}%
\pgfpathlineto{\pgfqpoint{2.216746in}{2.614590in}}%
\pgfpathlineto{\pgfqpoint{2.216746in}{2.611640in}}%
\pgfpathmoveto{\pgfqpoint{2.216746in}{2.611640in}}%
\pgfpathlineto{\pgfqpoint{2.216746in}{2.611640in}}%
\pgfpathlineto{\pgfqpoint{2.216746in}{2.614590in}}%
\pgfpathlineto{\pgfqpoint{2.221287in}{2.614590in}}%
\pgfpathlineto{\pgfqpoint{2.221287in}{2.611640in}}%
\pgfpathmoveto{\pgfqpoint{2.221287in}{2.611640in}}%
\pgfpathlineto{\pgfqpoint{2.221287in}{2.611640in}}%
\pgfpathlineto{\pgfqpoint{2.221287in}{2.614590in}}%
\pgfpathlineto{\pgfqpoint{2.225828in}{2.614590in}}%
\pgfpathlineto{\pgfqpoint{2.225828in}{2.611640in}}%
\pgfpathmoveto{\pgfqpoint{2.225828in}{2.611640in}}%
\pgfpathlineto{\pgfqpoint{2.225828in}{2.611640in}}%
\pgfpathlineto{\pgfqpoint{2.225828in}{2.614590in}}%
\pgfpathlineto{\pgfqpoint{2.230369in}{2.614590in}}%
\pgfpathlineto{\pgfqpoint{2.230369in}{2.611640in}}%
\pgfpathmoveto{\pgfqpoint{2.230369in}{2.611640in}}%
\pgfpathlineto{\pgfqpoint{2.230369in}{2.611640in}}%
\pgfpathlineto{\pgfqpoint{2.230369in}{2.614590in}}%
\pgfpathlineto{\pgfqpoint{2.234910in}{2.614590in}}%
\pgfpathlineto{\pgfqpoint{2.234910in}{2.611640in}}%
\pgfpathmoveto{\pgfqpoint{2.234910in}{2.611640in}}%
\pgfpathlineto{\pgfqpoint{2.234910in}{2.611640in}}%
\pgfpathlineto{\pgfqpoint{2.234910in}{2.614590in}}%
\pgfpathlineto{\pgfqpoint{2.239451in}{2.614590in}}%
\pgfpathlineto{\pgfqpoint{2.239451in}{2.611640in}}%
\pgfpathmoveto{\pgfqpoint{2.239451in}{2.611640in}}%
\pgfpathlineto{\pgfqpoint{2.239451in}{2.611640in}}%
\pgfpathlineto{\pgfqpoint{2.239451in}{2.614590in}}%
\pgfpathlineto{\pgfqpoint{2.243992in}{2.614590in}}%
\pgfpathlineto{\pgfqpoint{2.243992in}{2.611640in}}%
\pgfpathmoveto{\pgfqpoint{2.243992in}{2.611640in}}%
\pgfpathlineto{\pgfqpoint{2.243992in}{2.611640in}}%
\pgfpathlineto{\pgfqpoint{2.243992in}{2.614590in}}%
\pgfpathlineto{\pgfqpoint{2.248533in}{2.614590in}}%
\pgfpathlineto{\pgfqpoint{2.248533in}{2.611640in}}%
\pgfpathmoveto{\pgfqpoint{2.248533in}{2.611640in}}%
\pgfpathlineto{\pgfqpoint{2.248533in}{2.611640in}}%
\pgfpathlineto{\pgfqpoint{2.248533in}{2.614590in}}%
\pgfpathlineto{\pgfqpoint{2.253074in}{2.614590in}}%
\pgfpathlineto{\pgfqpoint{2.253074in}{2.611640in}}%
\pgfpathmoveto{\pgfqpoint{2.253074in}{2.611640in}}%
\pgfpathlineto{\pgfqpoint{2.253074in}{2.611640in}}%
\pgfpathlineto{\pgfqpoint{2.253074in}{2.614590in}}%
\pgfpathlineto{\pgfqpoint{2.257615in}{2.614590in}}%
\pgfpathlineto{\pgfqpoint{2.257615in}{2.611640in}}%
\pgfpathmoveto{\pgfqpoint{2.257615in}{2.611640in}}%
\pgfpathlineto{\pgfqpoint{2.257615in}{2.611640in}}%
\pgfpathlineto{\pgfqpoint{2.257615in}{2.614590in}}%
\pgfpathlineto{\pgfqpoint{2.262156in}{2.614590in}}%
\pgfpathlineto{\pgfqpoint{2.262156in}{2.611640in}}%
\pgfpathmoveto{\pgfqpoint{2.262156in}{2.611640in}}%
\pgfpathlineto{\pgfqpoint{2.262156in}{2.611640in}}%
\pgfpathlineto{\pgfqpoint{2.262156in}{2.614590in}}%
\pgfpathlineto{\pgfqpoint{2.266697in}{2.614590in}}%
\pgfpathlineto{\pgfqpoint{2.266697in}{2.611640in}}%
\pgfpathmoveto{\pgfqpoint{2.266697in}{2.611640in}}%
\pgfpathlineto{\pgfqpoint{2.266697in}{2.611640in}}%
\pgfpathlineto{\pgfqpoint{2.266697in}{2.614590in}}%
\pgfpathlineto{\pgfqpoint{2.271238in}{2.614590in}}%
\pgfpathlineto{\pgfqpoint{2.271238in}{2.611640in}}%
\pgfpathmoveto{\pgfqpoint{2.271238in}{2.611640in}}%
\pgfpathlineto{\pgfqpoint{2.271238in}{2.611640in}}%
\pgfpathlineto{\pgfqpoint{2.271238in}{2.614590in}}%
\pgfpathlineto{\pgfqpoint{2.275779in}{2.614590in}}%
\pgfpathlineto{\pgfqpoint{2.275779in}{2.611640in}}%
\pgfpathmoveto{\pgfqpoint{2.275779in}{2.611640in}}%
\pgfpathlineto{\pgfqpoint{2.275779in}{2.611640in}}%
\pgfpathlineto{\pgfqpoint{2.275779in}{2.614590in}}%
\pgfpathlineto{\pgfqpoint{2.280320in}{2.614590in}}%
\pgfpathlineto{\pgfqpoint{2.280320in}{2.611640in}}%
\pgfpathmoveto{\pgfqpoint{2.280320in}{2.611640in}}%
\pgfpathlineto{\pgfqpoint{2.280320in}{2.611640in}}%
\pgfpathlineto{\pgfqpoint{2.280320in}{2.614590in}}%
\pgfpathlineto{\pgfqpoint{2.284861in}{2.614590in}}%
\pgfpathlineto{\pgfqpoint{2.284861in}{2.611640in}}%
\pgfpathmoveto{\pgfqpoint{2.284861in}{2.611640in}}%
\pgfpathlineto{\pgfqpoint{2.284861in}{2.611640in}}%
\pgfpathlineto{\pgfqpoint{2.284861in}{2.614590in}}%
\pgfpathlineto{\pgfqpoint{2.289402in}{2.614590in}}%
\pgfpathlineto{\pgfqpoint{2.289402in}{2.611640in}}%
\pgfpathmoveto{\pgfqpoint{2.289402in}{2.611640in}}%
\pgfpathlineto{\pgfqpoint{2.289402in}{2.611640in}}%
\pgfpathlineto{\pgfqpoint{2.289402in}{2.614590in}}%
\pgfpathlineto{\pgfqpoint{2.293943in}{2.614590in}}%
\pgfpathlineto{\pgfqpoint{2.293943in}{2.611640in}}%
\pgfpathmoveto{\pgfqpoint{2.293943in}{2.611640in}}%
\pgfpathlineto{\pgfqpoint{2.293943in}{2.611640in}}%
\pgfpathlineto{\pgfqpoint{2.293943in}{2.614590in}}%
\pgfpathlineto{\pgfqpoint{2.298484in}{2.614590in}}%
\pgfpathlineto{\pgfqpoint{2.298484in}{2.611640in}}%
\pgfpathmoveto{\pgfqpoint{2.298484in}{2.611640in}}%
\pgfpathlineto{\pgfqpoint{2.298484in}{2.611640in}}%
\pgfpathlineto{\pgfqpoint{2.298484in}{2.614590in}}%
\pgfpathlineto{\pgfqpoint{2.303024in}{2.614590in}}%
\pgfpathlineto{\pgfqpoint{2.303024in}{2.611640in}}%
\pgfpathmoveto{\pgfqpoint{2.303024in}{2.611640in}}%
\pgfpathlineto{\pgfqpoint{2.303024in}{2.611640in}}%
\pgfpathlineto{\pgfqpoint{2.303024in}{2.614590in}}%
\pgfpathlineto{\pgfqpoint{2.307565in}{2.614590in}}%
\pgfpathlineto{\pgfqpoint{2.307565in}{2.611640in}}%
\pgfpathmoveto{\pgfqpoint{2.307565in}{2.611640in}}%
\pgfpathlineto{\pgfqpoint{2.307565in}{2.611640in}}%
\pgfpathlineto{\pgfqpoint{2.307565in}{2.614590in}}%
\pgfpathlineto{\pgfqpoint{2.312106in}{2.614590in}}%
\pgfpathlineto{\pgfqpoint{2.312106in}{2.611640in}}%
\pgfpathmoveto{\pgfqpoint{2.312106in}{2.611640in}}%
\pgfpathlineto{\pgfqpoint{2.312106in}{2.611640in}}%
\pgfpathlineto{\pgfqpoint{2.312106in}{2.614590in}}%
\pgfpathlineto{\pgfqpoint{2.316647in}{2.614590in}}%
\pgfpathlineto{\pgfqpoint{2.316647in}{2.611640in}}%
\pgfpathmoveto{\pgfqpoint{2.316647in}{2.611640in}}%
\pgfpathlineto{\pgfqpoint{2.316647in}{2.611640in}}%
\pgfpathlineto{\pgfqpoint{2.316647in}{2.614590in}}%
\pgfpathlineto{\pgfqpoint{2.321188in}{2.614590in}}%
\pgfpathlineto{\pgfqpoint{2.321188in}{2.611640in}}%
\pgfpathmoveto{\pgfqpoint{2.321188in}{2.611640in}}%
\pgfpathlineto{\pgfqpoint{2.321188in}{2.611640in}}%
\pgfpathlineto{\pgfqpoint{2.321188in}{2.614590in}}%
\pgfpathlineto{\pgfqpoint{2.325729in}{2.614590in}}%
\pgfpathlineto{\pgfqpoint{2.325729in}{2.611640in}}%
\pgfpathmoveto{\pgfqpoint{2.325729in}{2.611640in}}%
\pgfpathlineto{\pgfqpoint{2.325729in}{2.611640in}}%
\pgfpathlineto{\pgfqpoint{2.325729in}{2.614590in}}%
\pgfpathlineto{\pgfqpoint{2.330270in}{2.614590in}}%
\pgfpathlineto{\pgfqpoint{2.330270in}{2.611640in}}%
\pgfpathmoveto{\pgfqpoint{2.330270in}{2.611640in}}%
\pgfpathlineto{\pgfqpoint{2.330270in}{2.611640in}}%
\pgfpathlineto{\pgfqpoint{2.330270in}{2.614590in}}%
\pgfpathlineto{\pgfqpoint{2.334811in}{2.614590in}}%
\pgfpathlineto{\pgfqpoint{2.334811in}{2.611640in}}%
\pgfpathmoveto{\pgfqpoint{2.334811in}{2.611640in}}%
\pgfpathlineto{\pgfqpoint{2.334811in}{2.611640in}}%
\pgfpathlineto{\pgfqpoint{2.334811in}{2.614590in}}%
\pgfpathlineto{\pgfqpoint{2.339352in}{2.614590in}}%
\pgfpathlineto{\pgfqpoint{2.339352in}{2.611640in}}%
\pgfpathmoveto{\pgfqpoint{2.339352in}{2.611640in}}%
\pgfpathlineto{\pgfqpoint{2.339352in}{2.611640in}}%
\pgfpathlineto{\pgfqpoint{2.339352in}{2.614590in}}%
\pgfpathlineto{\pgfqpoint{2.343893in}{2.614590in}}%
\pgfpathlineto{\pgfqpoint{2.343893in}{2.611640in}}%
\pgfpathmoveto{\pgfqpoint{2.343893in}{2.611640in}}%
\pgfpathlineto{\pgfqpoint{2.343893in}{2.611640in}}%
\pgfpathlineto{\pgfqpoint{2.343893in}{2.614590in}}%
\pgfpathlineto{\pgfqpoint{2.348434in}{2.614590in}}%
\pgfpathlineto{\pgfqpoint{2.348434in}{2.611640in}}%
\pgfpathmoveto{\pgfqpoint{2.348434in}{2.611640in}}%
\pgfpathlineto{\pgfqpoint{2.348434in}{2.611640in}}%
\pgfpathlineto{\pgfqpoint{2.348434in}{2.614590in}}%
\pgfpathlineto{\pgfqpoint{2.352975in}{2.614590in}}%
\pgfpathlineto{\pgfqpoint{2.352975in}{2.611640in}}%
\pgfpathmoveto{\pgfqpoint{2.352975in}{2.611640in}}%
\pgfpathlineto{\pgfqpoint{2.352975in}{2.611640in}}%
\pgfpathlineto{\pgfqpoint{2.352975in}{2.614590in}}%
\pgfpathlineto{\pgfqpoint{2.357516in}{2.614590in}}%
\pgfpathlineto{\pgfqpoint{2.357516in}{2.611640in}}%
\pgfpathmoveto{\pgfqpoint{2.357516in}{2.611640in}}%
\pgfpathlineto{\pgfqpoint{2.357516in}{2.611640in}}%
\pgfpathlineto{\pgfqpoint{2.357516in}{2.614590in}}%
\pgfpathlineto{\pgfqpoint{2.362057in}{2.614590in}}%
\pgfpathlineto{\pgfqpoint{2.362057in}{2.611640in}}%
\pgfpathmoveto{\pgfqpoint{2.362057in}{2.611640in}}%
\pgfpathlineto{\pgfqpoint{2.362057in}{2.611640in}}%
\pgfpathlineto{\pgfqpoint{2.362057in}{2.614590in}}%
\pgfpathlineto{\pgfqpoint{2.366599in}{2.614590in}}%
\pgfpathlineto{\pgfqpoint{2.366599in}{2.611640in}}%
\pgfpathmoveto{\pgfqpoint{2.366599in}{2.611640in}}%
\pgfpathlineto{\pgfqpoint{2.366599in}{2.611640in}}%
\pgfpathlineto{\pgfqpoint{2.366599in}{2.614590in}}%
\pgfpathlineto{\pgfqpoint{2.371140in}{2.614590in}}%
\pgfpathlineto{\pgfqpoint{2.371140in}{2.611640in}}%
\pgfpathmoveto{\pgfqpoint{2.371140in}{2.611640in}}%
\pgfpathlineto{\pgfqpoint{2.371140in}{2.611640in}}%
\pgfpathlineto{\pgfqpoint{2.371140in}{2.614590in}}%
\pgfpathlineto{\pgfqpoint{2.375681in}{2.614590in}}%
\pgfpathlineto{\pgfqpoint{2.375681in}{2.611640in}}%
\pgfpathmoveto{\pgfqpoint{2.375681in}{2.611640in}}%
\pgfpathlineto{\pgfqpoint{2.375681in}{2.611640in}}%
\pgfpathlineto{\pgfqpoint{2.375681in}{2.614590in}}%
\pgfpathlineto{\pgfqpoint{2.380222in}{2.614590in}}%
\pgfpathlineto{\pgfqpoint{2.380222in}{2.611640in}}%
\pgfpathmoveto{\pgfqpoint{2.380222in}{2.611640in}}%
\pgfpathlineto{\pgfqpoint{2.380222in}{2.611640in}}%
\pgfpathlineto{\pgfqpoint{2.380222in}{2.614590in}}%
\pgfpathlineto{\pgfqpoint{2.384763in}{2.614590in}}%
\pgfpathlineto{\pgfqpoint{2.384763in}{2.611640in}}%
\pgfpathmoveto{\pgfqpoint{2.384763in}{2.611640in}}%
\pgfpathlineto{\pgfqpoint{2.384763in}{2.611640in}}%
\pgfpathlineto{\pgfqpoint{2.384763in}{2.614590in}}%
\pgfpathlineto{\pgfqpoint{2.389304in}{2.614590in}}%
\pgfpathlineto{\pgfqpoint{2.389304in}{2.611640in}}%
\pgfpathmoveto{\pgfqpoint{2.389304in}{2.611640in}}%
\pgfpathlineto{\pgfqpoint{2.389304in}{2.611640in}}%
\pgfpathlineto{\pgfqpoint{2.389304in}{2.614590in}}%
\pgfpathlineto{\pgfqpoint{2.393846in}{2.614590in}}%
\pgfpathlineto{\pgfqpoint{2.393846in}{2.611640in}}%
\pgfpathmoveto{\pgfqpoint{2.393846in}{2.611640in}}%
\pgfpathlineto{\pgfqpoint{2.393846in}{2.611640in}}%
\pgfpathlineto{\pgfqpoint{2.393846in}{2.614590in}}%
\pgfpathlineto{\pgfqpoint{2.398387in}{2.614590in}}%
\pgfpathlineto{\pgfqpoint{2.398387in}{2.611640in}}%
\pgfpathmoveto{\pgfqpoint{2.398387in}{2.611640in}}%
\pgfpathlineto{\pgfqpoint{2.398387in}{2.611640in}}%
\pgfpathlineto{\pgfqpoint{2.398387in}{2.614590in}}%
\pgfpathlineto{\pgfqpoint{2.402928in}{2.614590in}}%
\pgfpathlineto{\pgfqpoint{2.402928in}{2.611640in}}%
\pgfpathmoveto{\pgfqpoint{2.402928in}{2.611640in}}%
\pgfpathlineto{\pgfqpoint{2.402928in}{2.611640in}}%
\pgfpathlineto{\pgfqpoint{2.402928in}{2.614590in}}%
\pgfpathlineto{\pgfqpoint{2.407469in}{2.614590in}}%
\pgfpathlineto{\pgfqpoint{2.407469in}{2.611640in}}%
\pgfpathmoveto{\pgfqpoint{2.407469in}{2.611640in}}%
\pgfpathlineto{\pgfqpoint{2.407469in}{2.611640in}}%
\pgfpathlineto{\pgfqpoint{2.407469in}{2.614590in}}%
\pgfpathlineto{\pgfqpoint{2.412010in}{2.614590in}}%
\pgfpathlineto{\pgfqpoint{2.412010in}{2.611640in}}%
\pgfpathmoveto{\pgfqpoint{2.412010in}{2.611640in}}%
\pgfpathlineto{\pgfqpoint{2.412010in}{2.611640in}}%
\pgfpathlineto{\pgfqpoint{2.412010in}{2.614590in}}%
\pgfpathlineto{\pgfqpoint{2.416551in}{2.614590in}}%
\pgfpathlineto{\pgfqpoint{2.416551in}{2.611640in}}%
\pgfpathmoveto{\pgfqpoint{2.416551in}{2.611640in}}%
\pgfpathlineto{\pgfqpoint{2.416551in}{2.611640in}}%
\pgfpathlineto{\pgfqpoint{2.416551in}{2.614590in}}%
\pgfpathlineto{\pgfqpoint{2.421093in}{2.614590in}}%
\pgfpathlineto{\pgfqpoint{2.421093in}{2.611640in}}%
\pgfpathmoveto{\pgfqpoint{2.421093in}{2.611640in}}%
\pgfpathlineto{\pgfqpoint{2.421093in}{2.611640in}}%
\pgfpathlineto{\pgfqpoint{2.421093in}{2.614590in}}%
\pgfpathlineto{\pgfqpoint{2.425634in}{2.614590in}}%
\pgfpathlineto{\pgfqpoint{2.425634in}{2.611640in}}%
\pgfpathmoveto{\pgfqpoint{2.425634in}{2.611640in}}%
\pgfpathlineto{\pgfqpoint{2.425634in}{2.611640in}}%
\pgfpathlineto{\pgfqpoint{2.425634in}{2.614590in}}%
\pgfpathlineto{\pgfqpoint{2.430175in}{2.614590in}}%
\pgfpathlineto{\pgfqpoint{2.430175in}{2.611640in}}%
\pgfpathmoveto{\pgfqpoint{2.430175in}{2.611640in}}%
\pgfpathlineto{\pgfqpoint{2.430175in}{2.611640in}}%
\pgfpathlineto{\pgfqpoint{2.430175in}{2.614590in}}%
\pgfpathlineto{\pgfqpoint{2.434716in}{2.614590in}}%
\pgfpathlineto{\pgfqpoint{2.434716in}{2.611640in}}%
\pgfpathmoveto{\pgfqpoint{2.434716in}{2.611640in}}%
\pgfpathlineto{\pgfqpoint{2.434716in}{2.611640in}}%
\pgfpathlineto{\pgfqpoint{2.434716in}{2.614590in}}%
\pgfpathlineto{\pgfqpoint{2.439257in}{2.614590in}}%
\pgfpathlineto{\pgfqpoint{2.439257in}{2.611640in}}%
\pgfpathmoveto{\pgfqpoint{2.439257in}{2.611640in}}%
\pgfpathlineto{\pgfqpoint{2.439257in}{2.611640in}}%
\pgfpathlineto{\pgfqpoint{2.439257in}{2.614590in}}%
\pgfpathlineto{\pgfqpoint{2.443798in}{2.614590in}}%
\pgfpathlineto{\pgfqpoint{2.443798in}{2.611640in}}%
\pgfpathmoveto{\pgfqpoint{2.443798in}{2.611640in}}%
\pgfpathlineto{\pgfqpoint{2.443798in}{2.611640in}}%
\pgfpathlineto{\pgfqpoint{2.443798in}{2.614590in}}%
\pgfpathlineto{\pgfqpoint{2.448340in}{2.614590in}}%
\pgfpathlineto{\pgfqpoint{2.448340in}{2.611640in}}%
\pgfpathmoveto{\pgfqpoint{2.448340in}{2.611640in}}%
\pgfpathlineto{\pgfqpoint{2.448340in}{2.611640in}}%
\pgfpathlineto{\pgfqpoint{2.448340in}{2.614590in}}%
\pgfpathlineto{\pgfqpoint{2.452881in}{2.614590in}}%
\pgfpathlineto{\pgfqpoint{2.452881in}{2.611640in}}%
\pgfpathmoveto{\pgfqpoint{2.452881in}{2.611640in}}%
\pgfpathlineto{\pgfqpoint{2.452881in}{2.611640in}}%
\pgfpathlineto{\pgfqpoint{2.452881in}{2.614590in}}%
\pgfpathlineto{\pgfqpoint{2.457422in}{2.614590in}}%
\pgfpathlineto{\pgfqpoint{2.457422in}{2.611640in}}%
\pgfpathmoveto{\pgfqpoint{2.457422in}{2.611640in}}%
\pgfpathlineto{\pgfqpoint{2.457422in}{2.611640in}}%
\pgfpathlineto{\pgfqpoint{2.457422in}{2.614590in}}%
\pgfpathlineto{\pgfqpoint{2.461963in}{2.614590in}}%
\pgfpathlineto{\pgfqpoint{2.461963in}{2.611640in}}%
\pgfpathmoveto{\pgfqpoint{2.461963in}{2.611640in}}%
\pgfpathlineto{\pgfqpoint{2.461963in}{2.611640in}}%
\pgfpathlineto{\pgfqpoint{2.461963in}{2.614590in}}%
\pgfpathlineto{\pgfqpoint{2.466504in}{2.614590in}}%
\pgfpathlineto{\pgfqpoint{2.466504in}{2.611640in}}%
\pgfpathmoveto{\pgfqpoint{2.466504in}{2.611640in}}%
\pgfpathlineto{\pgfqpoint{2.466504in}{2.611640in}}%
\pgfpathlineto{\pgfqpoint{2.466504in}{2.614590in}}%
\pgfpathlineto{\pgfqpoint{2.471045in}{2.614590in}}%
\pgfpathlineto{\pgfqpoint{2.471045in}{2.611640in}}%
\pgfpathmoveto{\pgfqpoint{2.471045in}{2.611640in}}%
\pgfpathlineto{\pgfqpoint{2.471045in}{2.611640in}}%
\pgfpathlineto{\pgfqpoint{2.471045in}{2.614590in}}%
\pgfpathlineto{\pgfqpoint{2.475587in}{2.614590in}}%
\pgfpathlineto{\pgfqpoint{2.475587in}{2.611640in}}%
\pgfpathmoveto{\pgfqpoint{2.475587in}{2.611640in}}%
\pgfpathlineto{\pgfqpoint{2.475587in}{2.611640in}}%
\pgfpathlineto{\pgfqpoint{2.475587in}{2.614590in}}%
\pgfpathlineto{\pgfqpoint{2.480128in}{2.614590in}}%
\pgfpathlineto{\pgfqpoint{2.480128in}{2.611640in}}%
\pgfpathmoveto{\pgfqpoint{2.480128in}{2.611640in}}%
\pgfpathlineto{\pgfqpoint{2.480128in}{2.611640in}}%
\pgfpathlineto{\pgfqpoint{2.480128in}{2.614590in}}%
\pgfpathlineto{\pgfqpoint{2.484669in}{2.614590in}}%
\pgfpathlineto{\pgfqpoint{2.484669in}{2.611640in}}%
\pgfpathmoveto{\pgfqpoint{2.484669in}{2.611640in}}%
\pgfpathlineto{\pgfqpoint{2.484669in}{2.611640in}}%
\pgfpathlineto{\pgfqpoint{2.484669in}{2.614590in}}%
\pgfpathlineto{\pgfqpoint{2.489210in}{2.614590in}}%
\pgfpathlineto{\pgfqpoint{2.489210in}{2.611640in}}%
\pgfpathmoveto{\pgfqpoint{2.489210in}{2.611640in}}%
\pgfpathlineto{\pgfqpoint{2.489210in}{2.611640in}}%
\pgfpathlineto{\pgfqpoint{2.489210in}{2.614590in}}%
\pgfpathlineto{\pgfqpoint{2.493751in}{2.614590in}}%
\pgfpathlineto{\pgfqpoint{2.493751in}{2.611640in}}%
\pgfpathmoveto{\pgfqpoint{2.493751in}{2.611640in}}%
\pgfpathlineto{\pgfqpoint{2.493751in}{2.611640in}}%
\pgfpathlineto{\pgfqpoint{2.493751in}{2.614590in}}%
\pgfpathlineto{\pgfqpoint{2.498292in}{2.614590in}}%
\pgfpathlineto{\pgfqpoint{2.498292in}{2.611640in}}%
\pgfpathmoveto{\pgfqpoint{2.498292in}{2.611640in}}%
\pgfpathlineto{\pgfqpoint{2.498292in}{2.611640in}}%
\pgfpathlineto{\pgfqpoint{2.498292in}{2.614590in}}%
\pgfpathlineto{\pgfqpoint{2.502833in}{2.614590in}}%
\pgfpathlineto{\pgfqpoint{2.502833in}{2.611640in}}%
\pgfpathmoveto{\pgfqpoint{2.502833in}{2.611640in}}%
\pgfpathlineto{\pgfqpoint{2.502833in}{2.611640in}}%
\pgfpathlineto{\pgfqpoint{2.502833in}{2.614590in}}%
\pgfpathlineto{\pgfqpoint{2.507374in}{2.614590in}}%
\pgfpathlineto{\pgfqpoint{2.507374in}{2.611640in}}%
\pgfpathmoveto{\pgfqpoint{2.507374in}{2.611640in}}%
\pgfpathlineto{\pgfqpoint{2.507374in}{2.611640in}}%
\pgfpathlineto{\pgfqpoint{2.507374in}{2.614590in}}%
\pgfpathlineto{\pgfqpoint{2.511915in}{2.614590in}}%
\pgfpathlineto{\pgfqpoint{2.511915in}{2.611640in}}%
\pgfpathmoveto{\pgfqpoint{2.511915in}{2.611640in}}%
\pgfpathlineto{\pgfqpoint{2.511915in}{2.611640in}}%
\pgfpathlineto{\pgfqpoint{2.511915in}{2.614590in}}%
\pgfpathlineto{\pgfqpoint{2.516456in}{2.614590in}}%
\pgfpathlineto{\pgfqpoint{2.516456in}{2.611640in}}%
\pgfpathmoveto{\pgfqpoint{2.516456in}{2.611640in}}%
\pgfpathlineto{\pgfqpoint{2.516456in}{2.611640in}}%
\pgfpathlineto{\pgfqpoint{2.516456in}{2.614590in}}%
\pgfpathlineto{\pgfqpoint{2.520997in}{2.614590in}}%
\pgfpathlineto{\pgfqpoint{2.520997in}{2.611640in}}%
\pgfpathmoveto{\pgfqpoint{2.520997in}{2.611640in}}%
\pgfpathlineto{\pgfqpoint{2.520997in}{2.611640in}}%
\pgfpathlineto{\pgfqpoint{2.520997in}{2.614590in}}%
\pgfpathlineto{\pgfqpoint{2.525538in}{2.614590in}}%
\pgfpathlineto{\pgfqpoint{2.525538in}{2.611640in}}%
\pgfpathmoveto{\pgfqpoint{2.525538in}{2.611640in}}%
\pgfpathlineto{\pgfqpoint{2.525538in}{2.611640in}}%
\pgfpathlineto{\pgfqpoint{2.525538in}{2.614590in}}%
\pgfpathlineto{\pgfqpoint{2.530079in}{2.614590in}}%
\pgfpathlineto{\pgfqpoint{2.530079in}{2.611640in}}%
\pgfpathmoveto{\pgfqpoint{2.530079in}{2.611640in}}%
\pgfpathlineto{\pgfqpoint{2.530079in}{2.611640in}}%
\pgfpathlineto{\pgfqpoint{2.530079in}{2.614590in}}%
\pgfpathlineto{\pgfqpoint{2.534620in}{2.614590in}}%
\pgfpathlineto{\pgfqpoint{2.534620in}{2.611640in}}%
\pgfpathmoveto{\pgfqpoint{2.534620in}{2.611640in}}%
\pgfpathlineto{\pgfqpoint{2.534620in}{2.611640in}}%
\pgfpathlineto{\pgfqpoint{2.534620in}{2.614590in}}%
\pgfpathlineto{\pgfqpoint{2.539161in}{2.614590in}}%
\pgfpathlineto{\pgfqpoint{2.539161in}{2.611640in}}%
\pgfpathmoveto{\pgfqpoint{2.539161in}{2.611640in}}%
\pgfpathlineto{\pgfqpoint{2.539161in}{2.611640in}}%
\pgfpathlineto{\pgfqpoint{2.539161in}{2.614590in}}%
\pgfpathlineto{\pgfqpoint{2.543702in}{2.614590in}}%
\pgfpathlineto{\pgfqpoint{2.543702in}{2.611640in}}%
\pgfpathmoveto{\pgfqpoint{2.543702in}{2.611640in}}%
\pgfpathlineto{\pgfqpoint{2.543702in}{2.611640in}}%
\pgfpathlineto{\pgfqpoint{2.543702in}{2.614590in}}%
\pgfpathlineto{\pgfqpoint{2.548243in}{2.614590in}}%
\pgfpathlineto{\pgfqpoint{2.548243in}{2.611640in}}%
\pgfpathmoveto{\pgfqpoint{2.548243in}{2.611640in}}%
\pgfpathlineto{\pgfqpoint{2.548243in}{2.611640in}}%
\pgfpathlineto{\pgfqpoint{2.548243in}{2.614590in}}%
\pgfpathlineto{\pgfqpoint{2.552784in}{2.614590in}}%
\pgfpathlineto{\pgfqpoint{2.552784in}{2.611640in}}%
\pgfpathmoveto{\pgfqpoint{2.552784in}{2.611640in}}%
\pgfpathlineto{\pgfqpoint{2.552784in}{2.611640in}}%
\pgfpathlineto{\pgfqpoint{2.552784in}{2.614590in}}%
\pgfpathlineto{\pgfqpoint{2.557325in}{2.614590in}}%
\pgfpathlineto{\pgfqpoint{2.557325in}{2.611640in}}%
\pgfpathmoveto{\pgfqpoint{2.557325in}{2.611640in}}%
\pgfpathlineto{\pgfqpoint{2.557325in}{2.611640in}}%
\pgfpathlineto{\pgfqpoint{2.557325in}{2.614590in}}%
\pgfpathlineto{\pgfqpoint{2.561866in}{2.614590in}}%
\pgfpathlineto{\pgfqpoint{2.561866in}{2.611640in}}%
\pgfpathmoveto{\pgfqpoint{2.561866in}{2.611640in}}%
\pgfpathlineto{\pgfqpoint{2.561866in}{2.611640in}}%
\pgfpathlineto{\pgfqpoint{2.561866in}{2.614590in}}%
\pgfpathlineto{\pgfqpoint{2.566407in}{2.614590in}}%
\pgfpathlineto{\pgfqpoint{2.566407in}{2.611640in}}%
\pgfpathmoveto{\pgfqpoint{2.566407in}{2.611640in}}%
\pgfpathlineto{\pgfqpoint{2.566407in}{2.611640in}}%
\pgfpathlineto{\pgfqpoint{2.566407in}{2.614590in}}%
\pgfpathlineto{\pgfqpoint{2.570948in}{2.614590in}}%
\pgfpathlineto{\pgfqpoint{2.570948in}{2.611640in}}%
\pgfpathmoveto{\pgfqpoint{2.570948in}{2.611640in}}%
\pgfpathlineto{\pgfqpoint{2.570948in}{2.611640in}}%
\pgfpathlineto{\pgfqpoint{2.570948in}{2.614590in}}%
\pgfpathlineto{\pgfqpoint{2.575489in}{2.614590in}}%
\pgfpathlineto{\pgfqpoint{2.575489in}{2.611640in}}%
\pgfpathmoveto{\pgfqpoint{2.575489in}{2.611640in}}%
\pgfpathlineto{\pgfqpoint{2.575489in}{2.611640in}}%
\pgfpathlineto{\pgfqpoint{2.575489in}{2.614590in}}%
\pgfpathlineto{\pgfqpoint{2.580030in}{2.614590in}}%
\pgfpathlineto{\pgfqpoint{2.580030in}{2.611640in}}%
\pgfpathmoveto{\pgfqpoint{2.580030in}{2.611640in}}%
\pgfpathlineto{\pgfqpoint{2.580030in}{2.611640in}}%
\pgfpathlineto{\pgfqpoint{2.580030in}{2.614590in}}%
\pgfpathlineto{\pgfqpoint{2.584571in}{2.614590in}}%
\pgfpathlineto{\pgfqpoint{2.584571in}{2.611640in}}%
\pgfpathmoveto{\pgfqpoint{2.584571in}{2.611640in}}%
\pgfpathlineto{\pgfqpoint{2.584571in}{2.611640in}}%
\pgfpathlineto{\pgfqpoint{2.584571in}{2.614590in}}%
\pgfpathlineto{\pgfqpoint{2.589112in}{2.614590in}}%
\pgfpathlineto{\pgfqpoint{2.589112in}{2.611640in}}%
\pgfpathmoveto{\pgfqpoint{2.589112in}{2.611640in}}%
\pgfpathlineto{\pgfqpoint{2.589112in}{2.611640in}}%
\pgfpathlineto{\pgfqpoint{2.589112in}{2.614590in}}%
\pgfpathlineto{\pgfqpoint{2.593653in}{2.614590in}}%
\pgfpathlineto{\pgfqpoint{2.593653in}{2.611640in}}%
\pgfpathmoveto{\pgfqpoint{2.593653in}{2.611640in}}%
\pgfpathlineto{\pgfqpoint{2.593653in}{2.611640in}}%
\pgfpathlineto{\pgfqpoint{2.593653in}{2.614590in}}%
\pgfpathlineto{\pgfqpoint{2.598194in}{2.614590in}}%
\pgfpathlineto{\pgfqpoint{2.598194in}{2.611640in}}%
\pgfpathmoveto{\pgfqpoint{2.598194in}{2.611640in}}%
\pgfpathlineto{\pgfqpoint{2.598194in}{2.611640in}}%
\pgfpathlineto{\pgfqpoint{2.598194in}{2.614590in}}%
\pgfpathlineto{\pgfqpoint{2.602735in}{2.614590in}}%
\pgfpathlineto{\pgfqpoint{2.602735in}{2.611640in}}%
\pgfpathmoveto{\pgfqpoint{2.602735in}{2.611640in}}%
\pgfpathlineto{\pgfqpoint{2.602735in}{2.611640in}}%
\pgfpathlineto{\pgfqpoint{2.602735in}{2.614590in}}%
\pgfpathlineto{\pgfqpoint{2.607276in}{2.614590in}}%
\pgfpathlineto{\pgfqpoint{2.607276in}{2.611640in}}%
\pgfpathmoveto{\pgfqpoint{2.607276in}{2.611640in}}%
\pgfpathlineto{\pgfqpoint{2.607276in}{2.611640in}}%
\pgfpathlineto{\pgfqpoint{2.607276in}{2.614590in}}%
\pgfpathlineto{\pgfqpoint{2.611817in}{2.614590in}}%
\pgfpathlineto{\pgfqpoint{2.611817in}{2.611640in}}%
\pgfpathmoveto{\pgfqpoint{2.611817in}{2.611640in}}%
\pgfpathlineto{\pgfqpoint{2.611817in}{2.611640in}}%
\pgfpathlineto{\pgfqpoint{2.611817in}{2.614590in}}%
\pgfpathlineto{\pgfqpoint{2.616358in}{2.614590in}}%
\pgfpathlineto{\pgfqpoint{2.616358in}{2.611640in}}%
\pgfpathmoveto{\pgfqpoint{2.616358in}{2.611640in}}%
\pgfpathlineto{\pgfqpoint{2.616358in}{2.611640in}}%
\pgfpathlineto{\pgfqpoint{2.616358in}{2.614590in}}%
\pgfpathlineto{\pgfqpoint{2.620899in}{2.614590in}}%
\pgfpathlineto{\pgfqpoint{2.620899in}{2.611640in}}%
\pgfpathmoveto{\pgfqpoint{2.620899in}{2.611640in}}%
\pgfpathlineto{\pgfqpoint{2.620899in}{2.611640in}}%
\pgfpathlineto{\pgfqpoint{2.620899in}{2.614590in}}%
\pgfpathlineto{\pgfqpoint{2.625440in}{2.614590in}}%
\pgfpathlineto{\pgfqpoint{2.625440in}{2.611640in}}%
\pgfpathmoveto{\pgfqpoint{2.625440in}{2.611640in}}%
\pgfpathlineto{\pgfqpoint{2.625440in}{2.611640in}}%
\pgfpathlineto{\pgfqpoint{2.625440in}{2.614590in}}%
\pgfpathlineto{\pgfqpoint{2.629981in}{2.614590in}}%
\pgfpathlineto{\pgfqpoint{2.629981in}{2.611640in}}%
\pgfpathmoveto{\pgfqpoint{2.629981in}{2.611640in}}%
\pgfpathlineto{\pgfqpoint{2.629981in}{2.611640in}}%
\pgfpathlineto{\pgfqpoint{2.629981in}{2.614590in}}%
\pgfpathlineto{\pgfqpoint{2.634522in}{2.614590in}}%
\pgfpathlineto{\pgfqpoint{2.634522in}{2.611640in}}%
\pgfpathmoveto{\pgfqpoint{2.634522in}{2.611640in}}%
\pgfpathlineto{\pgfqpoint{2.634522in}{2.611640in}}%
\pgfpathlineto{\pgfqpoint{2.634522in}{2.614590in}}%
\pgfpathlineto{\pgfqpoint{2.639063in}{2.614590in}}%
\pgfpathlineto{\pgfqpoint{2.639063in}{2.611640in}}%
\pgfpathmoveto{\pgfqpoint{2.639063in}{2.611640in}}%
\pgfpathlineto{\pgfqpoint{2.639063in}{2.611640in}}%
\pgfpathlineto{\pgfqpoint{2.639063in}{2.614590in}}%
\pgfpathlineto{\pgfqpoint{2.643604in}{2.614590in}}%
\pgfpathlineto{\pgfqpoint{2.643604in}{2.611640in}}%
\pgfpathmoveto{\pgfqpoint{2.643604in}{2.611640in}}%
\pgfpathlineto{\pgfqpoint{2.643604in}{2.611640in}}%
\pgfpathlineto{\pgfqpoint{2.643604in}{2.614590in}}%
\pgfpathlineto{\pgfqpoint{2.648145in}{2.614590in}}%
\pgfpathlineto{\pgfqpoint{2.648145in}{2.611640in}}%
\pgfpathmoveto{\pgfqpoint{2.648145in}{2.611640in}}%
\pgfpathlineto{\pgfqpoint{2.648145in}{2.611640in}}%
\pgfpathlineto{\pgfqpoint{2.648145in}{2.614590in}}%
\pgfpathlineto{\pgfqpoint{2.652686in}{2.614590in}}%
\pgfpathlineto{\pgfqpoint{2.652686in}{2.611640in}}%
\pgfpathmoveto{\pgfqpoint{2.652686in}{2.611640in}}%
\pgfpathlineto{\pgfqpoint{2.652686in}{2.611640in}}%
\pgfpathlineto{\pgfqpoint{2.652686in}{2.614590in}}%
\pgfpathlineto{\pgfqpoint{2.657226in}{2.614590in}}%
\pgfpathlineto{\pgfqpoint{2.657226in}{2.611640in}}%
\pgfpathmoveto{\pgfqpoint{2.657226in}{2.611640in}}%
\pgfpathlineto{\pgfqpoint{2.657226in}{2.611640in}}%
\pgfpathlineto{\pgfqpoint{2.657226in}{2.614590in}}%
\pgfpathlineto{\pgfqpoint{2.661767in}{2.614590in}}%
\pgfpathlineto{\pgfqpoint{2.661767in}{2.611640in}}%
\pgfpathmoveto{\pgfqpoint{2.661767in}{2.611640in}}%
\pgfpathlineto{\pgfqpoint{2.661767in}{2.611640in}}%
\pgfpathlineto{\pgfqpoint{2.661767in}{2.614590in}}%
\pgfpathlineto{\pgfqpoint{2.666308in}{2.614590in}}%
\pgfpathlineto{\pgfqpoint{2.666308in}{2.611640in}}%
\pgfpathmoveto{\pgfqpoint{2.666308in}{2.611640in}}%
\pgfpathlineto{\pgfqpoint{2.666308in}{2.611640in}}%
\pgfpathlineto{\pgfqpoint{2.666308in}{2.614590in}}%
\pgfpathlineto{\pgfqpoint{2.670849in}{2.614590in}}%
\pgfpathlineto{\pgfqpoint{2.670849in}{2.611640in}}%
\pgfpathmoveto{\pgfqpoint{2.670849in}{2.611640in}}%
\pgfpathlineto{\pgfqpoint{2.670849in}{2.611640in}}%
\pgfpathlineto{\pgfqpoint{2.670849in}{2.614590in}}%
\pgfpathlineto{\pgfqpoint{2.675390in}{2.614590in}}%
\pgfpathlineto{\pgfqpoint{2.675390in}{2.611640in}}%
\pgfpathmoveto{\pgfqpoint{2.675390in}{2.611640in}}%
\pgfpathlineto{\pgfqpoint{2.675390in}{2.611640in}}%
\pgfpathlineto{\pgfqpoint{2.675390in}{2.614590in}}%
\pgfpathlineto{\pgfqpoint{2.679931in}{2.614590in}}%
\pgfpathlineto{\pgfqpoint{2.679931in}{2.611640in}}%
\pgfpathmoveto{\pgfqpoint{2.679931in}{2.611640in}}%
\pgfpathlineto{\pgfqpoint{2.679931in}{2.611640in}}%
\pgfpathlineto{\pgfqpoint{2.679931in}{2.614590in}}%
\pgfpathlineto{\pgfqpoint{2.684472in}{2.614590in}}%
\pgfpathlineto{\pgfqpoint{2.684472in}{2.611640in}}%
\pgfpathmoveto{\pgfqpoint{2.684472in}{2.611640in}}%
\pgfpathlineto{\pgfqpoint{2.684472in}{2.611640in}}%
\pgfpathlineto{\pgfqpoint{2.684472in}{2.614590in}}%
\pgfpathlineto{\pgfqpoint{2.689013in}{2.614590in}}%
\pgfpathlineto{\pgfqpoint{2.689013in}{2.611640in}}%
\pgfpathmoveto{\pgfqpoint{2.689013in}{2.611640in}}%
\pgfpathlineto{\pgfqpoint{2.689013in}{2.611640in}}%
\pgfpathlineto{\pgfqpoint{2.689013in}{2.614590in}}%
\pgfpathlineto{\pgfqpoint{2.693554in}{2.614590in}}%
\pgfpathlineto{\pgfqpoint{2.693554in}{2.611640in}}%
\pgfpathmoveto{\pgfqpoint{2.693554in}{2.611640in}}%
\pgfpathlineto{\pgfqpoint{2.693554in}{2.611640in}}%
\pgfpathlineto{\pgfqpoint{2.693554in}{2.614590in}}%
\pgfpathlineto{\pgfqpoint{2.698094in}{2.614590in}}%
\pgfpathlineto{\pgfqpoint{2.698094in}{2.611640in}}%
\pgfpathmoveto{\pgfqpoint{2.698094in}{2.611640in}}%
\pgfpathlineto{\pgfqpoint{2.698094in}{2.611640in}}%
\pgfpathlineto{\pgfqpoint{2.698094in}{2.614590in}}%
\pgfpathlineto{\pgfqpoint{2.702635in}{2.614590in}}%
\pgfpathlineto{\pgfqpoint{2.702635in}{2.611640in}}%
\pgfpathmoveto{\pgfqpoint{2.702635in}{2.611640in}}%
\pgfpathlineto{\pgfqpoint{2.702635in}{2.611640in}}%
\pgfpathlineto{\pgfqpoint{2.702635in}{2.614590in}}%
\pgfpathlineto{\pgfqpoint{2.707176in}{2.614590in}}%
\pgfpathlineto{\pgfqpoint{2.707176in}{2.611640in}}%
\pgfpathmoveto{\pgfqpoint{2.707176in}{2.611640in}}%
\pgfpathlineto{\pgfqpoint{2.707176in}{2.611640in}}%
\pgfpathlineto{\pgfqpoint{2.707176in}{2.614590in}}%
\pgfpathlineto{\pgfqpoint{2.711717in}{2.614590in}}%
\pgfpathlineto{\pgfqpoint{2.711717in}{2.611640in}}%
\pgfpathmoveto{\pgfqpoint{2.711717in}{2.611640in}}%
\pgfpathlineto{\pgfqpoint{2.711717in}{2.611640in}}%
\pgfpathlineto{\pgfqpoint{2.711717in}{2.614590in}}%
\pgfpathlineto{\pgfqpoint{2.716258in}{2.614590in}}%
\pgfpathlineto{\pgfqpoint{2.716258in}{2.611640in}}%
\pgfpathmoveto{\pgfqpoint{2.716258in}{2.611640in}}%
\pgfpathlineto{\pgfqpoint{2.716258in}{2.611640in}}%
\pgfpathlineto{\pgfqpoint{2.716258in}{2.614590in}}%
\pgfpathlineto{\pgfqpoint{2.720799in}{2.614590in}}%
\pgfpathlineto{\pgfqpoint{2.720799in}{2.611640in}}%
\pgfpathmoveto{\pgfqpoint{2.720799in}{2.611640in}}%
\pgfpathlineto{\pgfqpoint{2.720799in}{2.611640in}}%
\pgfpathlineto{\pgfqpoint{2.720799in}{2.614590in}}%
\pgfpathlineto{\pgfqpoint{2.725340in}{2.614590in}}%
\pgfpathlineto{\pgfqpoint{2.725340in}{2.611640in}}%
\pgfpathmoveto{\pgfqpoint{2.725340in}{2.611640in}}%
\pgfpathlineto{\pgfqpoint{2.725340in}{2.611640in}}%
\pgfpathlineto{\pgfqpoint{2.725340in}{2.614590in}}%
\pgfpathlineto{\pgfqpoint{2.729881in}{2.614590in}}%
\pgfpathlineto{\pgfqpoint{2.729881in}{2.611640in}}%
\pgfpathmoveto{\pgfqpoint{2.729881in}{2.611640in}}%
\pgfpathlineto{\pgfqpoint{2.729881in}{2.611640in}}%
\pgfpathlineto{\pgfqpoint{2.729881in}{2.614590in}}%
\pgfpathlineto{\pgfqpoint{2.734422in}{2.614590in}}%
\pgfpathlineto{\pgfqpoint{2.734422in}{2.611640in}}%
\pgfpathmoveto{\pgfqpoint{2.734422in}{2.611640in}}%
\pgfpathlineto{\pgfqpoint{2.734422in}{2.611640in}}%
\pgfpathlineto{\pgfqpoint{2.734422in}{2.614590in}}%
\pgfpathlineto{\pgfqpoint{2.738963in}{2.614590in}}%
\pgfpathlineto{\pgfqpoint{2.738963in}{2.611640in}}%
\pgfpathmoveto{\pgfqpoint{2.738963in}{2.611640in}}%
\pgfpathlineto{\pgfqpoint{2.738963in}{2.611640in}}%
\pgfpathlineto{\pgfqpoint{2.738963in}{2.614590in}}%
\pgfpathlineto{\pgfqpoint{2.743503in}{2.614590in}}%
\pgfpathlineto{\pgfqpoint{2.743503in}{2.611640in}}%
\pgfpathmoveto{\pgfqpoint{2.743503in}{2.611640in}}%
\pgfpathlineto{\pgfqpoint{2.743503in}{2.611640in}}%
\pgfpathlineto{\pgfqpoint{2.743503in}{2.614590in}}%
\pgfpathlineto{\pgfqpoint{2.748044in}{2.614590in}}%
\pgfpathlineto{\pgfqpoint{2.748044in}{2.611640in}}%
\pgfpathmoveto{\pgfqpoint{2.748044in}{2.611640in}}%
\pgfpathlineto{\pgfqpoint{2.748044in}{2.611640in}}%
\pgfpathlineto{\pgfqpoint{2.748044in}{2.614590in}}%
\pgfpathlineto{\pgfqpoint{2.752585in}{2.614590in}}%
\pgfpathlineto{\pgfqpoint{2.752585in}{2.611640in}}%
\pgfpathmoveto{\pgfqpoint{2.752585in}{2.611640in}}%
\pgfpathlineto{\pgfqpoint{2.752585in}{2.611640in}}%
\pgfpathlineto{\pgfqpoint{2.752585in}{2.614590in}}%
\pgfpathlineto{\pgfqpoint{2.757126in}{2.614590in}}%
\pgfpathlineto{\pgfqpoint{2.757126in}{2.611640in}}%
\pgfpathmoveto{\pgfqpoint{2.757126in}{2.611640in}}%
\pgfpathlineto{\pgfqpoint{2.757126in}{2.611640in}}%
\pgfpathlineto{\pgfqpoint{2.757126in}{2.614590in}}%
\pgfpathlineto{\pgfqpoint{2.761667in}{2.614590in}}%
\pgfpathlineto{\pgfqpoint{2.761667in}{2.611640in}}%
\pgfpathmoveto{\pgfqpoint{2.761667in}{2.611640in}}%
\pgfpathlineto{\pgfqpoint{2.761667in}{2.611640in}}%
\pgfpathlineto{\pgfqpoint{2.761667in}{2.614590in}}%
\pgfpathlineto{\pgfqpoint{2.766208in}{2.614590in}}%
\pgfpathlineto{\pgfqpoint{2.766208in}{2.611640in}}%
\pgfpathmoveto{\pgfqpoint{2.766208in}{2.611640in}}%
\pgfpathlineto{\pgfqpoint{2.766208in}{2.611640in}}%
\pgfpathlineto{\pgfqpoint{2.766208in}{2.614590in}}%
\pgfpathlineto{\pgfqpoint{2.770749in}{2.614590in}}%
\pgfpathlineto{\pgfqpoint{2.770749in}{2.611640in}}%
\pgfpathmoveto{\pgfqpoint{2.770749in}{2.611640in}}%
\pgfpathlineto{\pgfqpoint{2.770749in}{2.611640in}}%
\pgfpathlineto{\pgfqpoint{2.770749in}{2.614590in}}%
\pgfpathlineto{\pgfqpoint{2.775290in}{2.614590in}}%
\pgfpathlineto{\pgfqpoint{2.775290in}{2.611640in}}%
\pgfpathmoveto{\pgfqpoint{2.775290in}{2.611640in}}%
\pgfpathlineto{\pgfqpoint{2.775290in}{2.611640in}}%
\pgfpathlineto{\pgfqpoint{2.775290in}{2.614590in}}%
\pgfpathlineto{\pgfqpoint{2.779831in}{2.614590in}}%
\pgfpathlineto{\pgfqpoint{2.779831in}{2.611640in}}%
\pgfpathmoveto{\pgfqpoint{2.779831in}{2.611640in}}%
\pgfpathlineto{\pgfqpoint{2.779831in}{2.611640in}}%
\pgfpathlineto{\pgfqpoint{2.779831in}{2.614590in}}%
\pgfpathlineto{\pgfqpoint{2.784371in}{2.614590in}}%
\pgfpathlineto{\pgfqpoint{2.784371in}{2.611640in}}%
\pgfpathmoveto{\pgfqpoint{2.784371in}{2.611640in}}%
\pgfpathlineto{\pgfqpoint{2.784371in}{2.611640in}}%
\pgfpathlineto{\pgfqpoint{2.784371in}{2.614590in}}%
\pgfpathlineto{\pgfqpoint{2.788912in}{2.614590in}}%
\pgfpathlineto{\pgfqpoint{2.788912in}{2.611640in}}%
\pgfpathmoveto{\pgfqpoint{2.788912in}{2.611640in}}%
\pgfpathlineto{\pgfqpoint{2.788912in}{2.611640in}}%
\pgfpathlineto{\pgfqpoint{2.788912in}{2.614590in}}%
\pgfpathlineto{\pgfqpoint{2.793453in}{2.614590in}}%
\pgfpathlineto{\pgfqpoint{2.793453in}{2.611640in}}%
\pgfpathmoveto{\pgfqpoint{2.793453in}{2.611640in}}%
\pgfpathlineto{\pgfqpoint{2.793453in}{2.611640in}}%
\pgfpathlineto{\pgfqpoint{2.793453in}{2.614590in}}%
\pgfpathlineto{\pgfqpoint{2.797994in}{2.614590in}}%
\pgfpathlineto{\pgfqpoint{2.797994in}{2.611640in}}%
\pgfpathmoveto{\pgfqpoint{2.797994in}{2.611640in}}%
\pgfpathlineto{\pgfqpoint{2.797994in}{2.611640in}}%
\pgfpathlineto{\pgfqpoint{2.797994in}{2.614590in}}%
\pgfpathlineto{\pgfqpoint{2.802535in}{2.614590in}}%
\pgfpathlineto{\pgfqpoint{2.802535in}{2.611640in}}%
\pgfpathmoveto{\pgfqpoint{2.802535in}{2.611640in}}%
\pgfpathlineto{\pgfqpoint{2.802535in}{2.611640in}}%
\pgfpathlineto{\pgfqpoint{2.802535in}{2.614590in}}%
\pgfpathlineto{\pgfqpoint{2.807076in}{2.614590in}}%
\pgfpathlineto{\pgfqpoint{2.807076in}{2.611640in}}%
\pgfpathmoveto{\pgfqpoint{2.807076in}{2.611640in}}%
\pgfpathlineto{\pgfqpoint{2.807076in}{2.611640in}}%
\pgfpathlineto{\pgfqpoint{2.807076in}{2.614590in}}%
\pgfpathlineto{\pgfqpoint{2.811617in}{2.614590in}}%
\pgfpathlineto{\pgfqpoint{2.811617in}{2.611640in}}%
\pgfpathmoveto{\pgfqpoint{2.811617in}{2.611640in}}%
\pgfpathlineto{\pgfqpoint{2.811617in}{2.611640in}}%
\pgfpathlineto{\pgfqpoint{2.811617in}{2.614590in}}%
\pgfpathlineto{\pgfqpoint{2.816158in}{2.614590in}}%
\pgfpathlineto{\pgfqpoint{2.816158in}{2.611640in}}%
\pgfpathmoveto{\pgfqpoint{2.816158in}{2.611640in}}%
\pgfpathlineto{\pgfqpoint{2.816158in}{2.611640in}}%
\pgfpathlineto{\pgfqpoint{2.816158in}{2.614590in}}%
\pgfpathlineto{\pgfqpoint{2.820699in}{2.614590in}}%
\pgfpathlineto{\pgfqpoint{2.820699in}{2.611640in}}%
\pgfpathmoveto{\pgfqpoint{2.820699in}{2.611640in}}%
\pgfpathlineto{\pgfqpoint{2.820699in}{2.611640in}}%
\pgfpathlineto{\pgfqpoint{2.820699in}{2.614590in}}%
\pgfpathlineto{\pgfqpoint{2.825240in}{2.614590in}}%
\pgfpathlineto{\pgfqpoint{2.825240in}{2.611640in}}%
\pgfpathmoveto{\pgfqpoint{2.825240in}{2.611640in}}%
\pgfpathlineto{\pgfqpoint{2.825240in}{2.611640in}}%
\pgfpathlineto{\pgfqpoint{2.825240in}{2.614590in}}%
\pgfpathlineto{\pgfqpoint{2.829781in}{2.614590in}}%
\pgfpathlineto{\pgfqpoint{2.829781in}{2.611640in}}%
\pgfpathmoveto{\pgfqpoint{2.829781in}{2.611640in}}%
\pgfpathlineto{\pgfqpoint{2.829781in}{2.611640in}}%
\pgfpathlineto{\pgfqpoint{2.829781in}{2.614590in}}%
\pgfpathlineto{\pgfqpoint{2.834322in}{2.614590in}}%
\pgfpathlineto{\pgfqpoint{2.834322in}{2.611640in}}%
\pgfpathmoveto{\pgfqpoint{2.834322in}{2.611640in}}%
\pgfpathlineto{\pgfqpoint{2.834322in}{2.611640in}}%
\pgfpathlineto{\pgfqpoint{2.834322in}{2.614590in}}%
\pgfpathlineto{\pgfqpoint{2.838863in}{2.614590in}}%
\pgfpathlineto{\pgfqpoint{2.838863in}{2.611640in}}%
\pgfpathmoveto{\pgfqpoint{2.838863in}{2.611640in}}%
\pgfpathlineto{\pgfqpoint{2.838863in}{2.611640in}}%
\pgfpathlineto{\pgfqpoint{2.838863in}{2.614590in}}%
\pgfpathlineto{\pgfqpoint{2.843404in}{2.614590in}}%
\pgfpathlineto{\pgfqpoint{2.843404in}{2.611640in}}%
\pgfpathmoveto{\pgfqpoint{2.843404in}{2.611640in}}%
\pgfpathlineto{\pgfqpoint{2.843404in}{2.611640in}}%
\pgfpathlineto{\pgfqpoint{2.843404in}{2.614590in}}%
\pgfpathlineto{\pgfqpoint{2.847945in}{2.614590in}}%
\pgfpathlineto{\pgfqpoint{2.847945in}{2.611640in}}%
\pgfpathmoveto{\pgfqpoint{2.847945in}{2.611640in}}%
\pgfpathlineto{\pgfqpoint{2.847945in}{2.611640in}}%
\pgfpathlineto{\pgfqpoint{2.847945in}{2.614590in}}%
\pgfpathlineto{\pgfqpoint{2.852486in}{2.614590in}}%
\pgfpathlineto{\pgfqpoint{2.852486in}{2.611640in}}%
\pgfpathmoveto{\pgfqpoint{2.852486in}{2.611640in}}%
\pgfpathlineto{\pgfqpoint{2.852486in}{2.611640in}}%
\pgfpathlineto{\pgfqpoint{2.852486in}{2.614590in}}%
\pgfpathlineto{\pgfqpoint{2.857027in}{2.614590in}}%
\pgfpathlineto{\pgfqpoint{2.857027in}{2.611640in}}%
\pgfpathmoveto{\pgfqpoint{2.857027in}{2.611640in}}%
\pgfpathlineto{\pgfqpoint{2.857027in}{2.611640in}}%
\pgfpathlineto{\pgfqpoint{2.857027in}{2.614590in}}%
\pgfpathlineto{\pgfqpoint{2.861568in}{2.614590in}}%
\pgfpathlineto{\pgfqpoint{2.861568in}{2.611640in}}%
\pgfpathmoveto{\pgfqpoint{2.861568in}{2.611640in}}%
\pgfpathlineto{\pgfqpoint{2.861568in}{2.611640in}}%
\pgfpathlineto{\pgfqpoint{2.861568in}{2.614590in}}%
\pgfpathlineto{\pgfqpoint{2.866109in}{2.614590in}}%
\pgfpathlineto{\pgfqpoint{2.866109in}{2.611640in}}%
\pgfpathmoveto{\pgfqpoint{2.866109in}{2.611640in}}%
\pgfpathlineto{\pgfqpoint{2.866109in}{2.611640in}}%
\pgfpathlineto{\pgfqpoint{2.866109in}{2.614590in}}%
\pgfpathlineto{\pgfqpoint{2.870650in}{2.614590in}}%
\pgfpathlineto{\pgfqpoint{2.870650in}{2.611640in}}%
\pgfpathmoveto{\pgfqpoint{2.870650in}{2.611640in}}%
\pgfpathlineto{\pgfqpoint{2.870650in}{2.611640in}}%
\pgfpathlineto{\pgfqpoint{2.870650in}{2.614590in}}%
\pgfpathlineto{\pgfqpoint{2.875191in}{2.614590in}}%
\pgfpathlineto{\pgfqpoint{2.875191in}{2.611640in}}%
\pgfpathmoveto{\pgfqpoint{2.875191in}{2.611640in}}%
\pgfpathlineto{\pgfqpoint{2.875191in}{2.611640in}}%
\pgfpathlineto{\pgfqpoint{2.875191in}{2.614590in}}%
\pgfpathlineto{\pgfqpoint{2.879732in}{2.614590in}}%
\pgfpathlineto{\pgfqpoint{2.879732in}{2.611640in}}%
\pgfpathmoveto{\pgfqpoint{2.879732in}{2.611640in}}%
\pgfpathlineto{\pgfqpoint{2.879732in}{2.611640in}}%
\pgfpathlineto{\pgfqpoint{2.879732in}{2.614590in}}%
\pgfpathlineto{\pgfqpoint{2.884273in}{2.614590in}}%
\pgfpathlineto{\pgfqpoint{2.884273in}{2.611640in}}%
\pgfpathmoveto{\pgfqpoint{2.884273in}{2.611640in}}%
\pgfpathlineto{\pgfqpoint{2.884273in}{2.611640in}}%
\pgfpathlineto{\pgfqpoint{2.884273in}{2.614590in}}%
\pgfpathlineto{\pgfqpoint{2.888814in}{2.614590in}}%
\pgfpathlineto{\pgfqpoint{2.888814in}{2.611640in}}%
\pgfpathmoveto{\pgfqpoint{2.888814in}{2.611640in}}%
\pgfpathlineto{\pgfqpoint{2.888814in}{2.611640in}}%
\pgfpathlineto{\pgfqpoint{2.888814in}{2.614590in}}%
\pgfpathlineto{\pgfqpoint{2.893355in}{2.614590in}}%
\pgfpathlineto{\pgfqpoint{2.893355in}{2.611640in}}%
\pgfpathmoveto{\pgfqpoint{2.893355in}{2.611640in}}%
\pgfpathlineto{\pgfqpoint{2.893355in}{2.611640in}}%
\pgfpathlineto{\pgfqpoint{2.893355in}{2.614590in}}%
\pgfpathlineto{\pgfqpoint{2.897896in}{2.614590in}}%
\pgfpathlineto{\pgfqpoint{2.897896in}{2.611640in}}%
\pgfpathmoveto{\pgfqpoint{2.897896in}{2.611640in}}%
\pgfpathlineto{\pgfqpoint{2.897896in}{2.611640in}}%
\pgfpathlineto{\pgfqpoint{2.897896in}{2.614590in}}%
\pgfpathlineto{\pgfqpoint{2.902437in}{2.614590in}}%
\pgfpathlineto{\pgfqpoint{2.902437in}{2.611640in}}%
\pgfpathmoveto{\pgfqpoint{2.902437in}{2.611640in}}%
\pgfpathlineto{\pgfqpoint{2.902437in}{2.611640in}}%
\pgfpathlineto{\pgfqpoint{2.902437in}{2.614590in}}%
\pgfpathlineto{\pgfqpoint{2.906978in}{2.614590in}}%
\pgfpathlineto{\pgfqpoint{2.906978in}{2.611640in}}%
\pgfpathmoveto{\pgfqpoint{2.906978in}{2.611640in}}%
\pgfpathlineto{\pgfqpoint{2.906978in}{2.611640in}}%
\pgfpathlineto{\pgfqpoint{2.906978in}{2.614590in}}%
\pgfpathlineto{\pgfqpoint{2.911519in}{2.614590in}}%
\pgfpathlineto{\pgfqpoint{2.911519in}{2.611640in}}%
\pgfpathmoveto{\pgfqpoint{2.911519in}{2.611640in}}%
\pgfpathlineto{\pgfqpoint{2.911519in}{2.611640in}}%
\pgfpathlineto{\pgfqpoint{2.911519in}{2.614590in}}%
\pgfpathlineto{\pgfqpoint{2.916060in}{2.614590in}}%
\pgfpathlineto{\pgfqpoint{2.916060in}{2.611640in}}%
\pgfpathmoveto{\pgfqpoint{2.916060in}{2.611640in}}%
\pgfpathlineto{\pgfqpoint{2.916060in}{2.611640in}}%
\pgfpathlineto{\pgfqpoint{2.916060in}{2.614590in}}%
\pgfpathlineto{\pgfqpoint{2.920601in}{2.614590in}}%
\pgfpathlineto{\pgfqpoint{2.920601in}{2.611640in}}%
\pgfpathmoveto{\pgfqpoint{2.920601in}{2.611640in}}%
\pgfpathlineto{\pgfqpoint{2.920601in}{2.611640in}}%
\pgfpathlineto{\pgfqpoint{2.920601in}{2.614590in}}%
\pgfpathlineto{\pgfqpoint{2.925142in}{2.614590in}}%
\pgfpathlineto{\pgfqpoint{2.925142in}{2.611640in}}%
\pgfpathmoveto{\pgfqpoint{2.925142in}{2.611640in}}%
\pgfpathlineto{\pgfqpoint{2.925142in}{2.611640in}}%
\pgfpathlineto{\pgfqpoint{2.925142in}{2.614590in}}%
\pgfpathlineto{\pgfqpoint{2.929683in}{2.614590in}}%
\pgfpathlineto{\pgfqpoint{2.929683in}{2.611640in}}%
\pgfpathmoveto{\pgfqpoint{2.929683in}{2.611640in}}%
\pgfpathlineto{\pgfqpoint{2.929683in}{2.611640in}}%
\pgfpathlineto{\pgfqpoint{2.929683in}{2.614590in}}%
\pgfpathlineto{\pgfqpoint{2.934224in}{2.614590in}}%
\pgfpathlineto{\pgfqpoint{2.934224in}{2.611640in}}%
\pgfpathmoveto{\pgfqpoint{2.934224in}{2.611640in}}%
\pgfpathlineto{\pgfqpoint{2.934224in}{2.611640in}}%
\pgfpathlineto{\pgfqpoint{2.934224in}{2.614590in}}%
\pgfpathlineto{\pgfqpoint{2.938766in}{2.614590in}}%
\pgfpathlineto{\pgfqpoint{2.938766in}{2.611640in}}%
\pgfpathmoveto{\pgfqpoint{2.938766in}{2.611640in}}%
\pgfpathlineto{\pgfqpoint{2.938766in}{2.611640in}}%
\pgfpathlineto{\pgfqpoint{2.938766in}{2.614590in}}%
\pgfpathlineto{\pgfqpoint{2.943307in}{2.614590in}}%
\pgfpathlineto{\pgfqpoint{2.943307in}{2.611640in}}%
\pgfpathmoveto{\pgfqpoint{2.943307in}{2.611640in}}%
\pgfpathlineto{\pgfqpoint{2.943307in}{2.611640in}}%
\pgfpathlineto{\pgfqpoint{2.943307in}{2.614590in}}%
\pgfpathlineto{\pgfqpoint{2.947848in}{2.614590in}}%
\pgfpathlineto{\pgfqpoint{2.947848in}{2.611640in}}%
\pgfpathmoveto{\pgfqpoint{2.947848in}{2.611640in}}%
\pgfpathlineto{\pgfqpoint{2.947848in}{2.611640in}}%
\pgfpathlineto{\pgfqpoint{2.947848in}{2.614590in}}%
\pgfpathlineto{\pgfqpoint{2.952389in}{2.614590in}}%
\pgfpathlineto{\pgfqpoint{2.952389in}{2.611640in}}%
\pgfpathmoveto{\pgfqpoint{2.952389in}{2.611640in}}%
\pgfpathlineto{\pgfqpoint{2.952389in}{2.611640in}}%
\pgfpathlineto{\pgfqpoint{2.952389in}{2.614590in}}%
\pgfpathlineto{\pgfqpoint{2.956931in}{2.614590in}}%
\pgfpathlineto{\pgfqpoint{2.956931in}{2.611640in}}%
\pgfpathmoveto{\pgfqpoint{2.956931in}{2.611640in}}%
\pgfpathlineto{\pgfqpoint{2.956931in}{2.611640in}}%
\pgfpathlineto{\pgfqpoint{2.956931in}{2.614590in}}%
\pgfpathlineto{\pgfqpoint{2.961472in}{2.614590in}}%
\pgfpathlineto{\pgfqpoint{2.961472in}{2.611640in}}%
\pgfpathmoveto{\pgfqpoint{2.961472in}{2.611640in}}%
\pgfpathlineto{\pgfqpoint{2.961472in}{2.611640in}}%
\pgfpathlineto{\pgfqpoint{2.961472in}{2.614590in}}%
\pgfpathlineto{\pgfqpoint{2.966013in}{2.614590in}}%
\pgfpathlineto{\pgfqpoint{2.966013in}{2.611640in}}%
\pgfpathmoveto{\pgfqpoint{2.966013in}{2.611640in}}%
\pgfpathlineto{\pgfqpoint{2.966013in}{2.611640in}}%
\pgfpathlineto{\pgfqpoint{2.966013in}{2.614590in}}%
\pgfpathlineto{\pgfqpoint{2.970554in}{2.614590in}}%
\pgfpathlineto{\pgfqpoint{2.970554in}{2.611640in}}%
\pgfpathmoveto{\pgfqpoint{2.970554in}{2.611640in}}%
\pgfpathlineto{\pgfqpoint{2.970554in}{2.611640in}}%
\pgfpathlineto{\pgfqpoint{2.970554in}{2.614590in}}%
\pgfpathlineto{\pgfqpoint{2.975096in}{2.614590in}}%
\pgfpathlineto{\pgfqpoint{2.975096in}{2.611640in}}%
\pgfpathmoveto{\pgfqpoint{2.975096in}{2.611640in}}%
\pgfpathlineto{\pgfqpoint{2.975096in}{2.611640in}}%
\pgfpathlineto{\pgfqpoint{2.975096in}{2.614590in}}%
\pgfpathlineto{\pgfqpoint{2.979637in}{2.614590in}}%
\pgfpathlineto{\pgfqpoint{2.979637in}{2.611640in}}%
\pgfpathmoveto{\pgfqpoint{2.979637in}{2.611640in}}%
\pgfpathlineto{\pgfqpoint{2.979637in}{2.611640in}}%
\pgfpathlineto{\pgfqpoint{2.979637in}{2.614590in}}%
\pgfpathlineto{\pgfqpoint{2.984178in}{2.614590in}}%
\pgfpathlineto{\pgfqpoint{2.984178in}{2.611640in}}%
\pgfpathmoveto{\pgfqpoint{2.984178in}{2.611640in}}%
\pgfpathlineto{\pgfqpoint{2.984178in}{2.611640in}}%
\pgfpathlineto{\pgfqpoint{2.984178in}{2.614590in}}%
\pgfpathlineto{\pgfqpoint{2.988719in}{2.614590in}}%
\pgfpathlineto{\pgfqpoint{2.988719in}{2.611640in}}%
\pgfpathmoveto{\pgfqpoint{2.988719in}{2.611640in}}%
\pgfpathlineto{\pgfqpoint{2.988719in}{2.611640in}}%
\pgfpathlineto{\pgfqpoint{2.988719in}{2.614590in}}%
\pgfpathlineto{\pgfqpoint{2.993261in}{2.614590in}}%
\pgfpathlineto{\pgfqpoint{2.993261in}{2.611640in}}%
\pgfpathmoveto{\pgfqpoint{2.993261in}{2.611640in}}%
\pgfpathlineto{\pgfqpoint{2.993261in}{2.611640in}}%
\pgfpathlineto{\pgfqpoint{2.993261in}{2.614590in}}%
\pgfpathlineto{\pgfqpoint{2.997802in}{2.614590in}}%
\pgfpathlineto{\pgfqpoint{2.997802in}{2.611640in}}%
\pgfpathmoveto{\pgfqpoint{2.997802in}{2.611640in}}%
\pgfpathlineto{\pgfqpoint{2.997802in}{2.611640in}}%
\pgfpathlineto{\pgfqpoint{2.997802in}{2.614590in}}%
\pgfpathlineto{\pgfqpoint{3.002343in}{2.614590in}}%
\pgfpathlineto{\pgfqpoint{3.002343in}{2.611640in}}%
\pgfpathmoveto{\pgfqpoint{3.002343in}{2.611640in}}%
\pgfpathlineto{\pgfqpoint{3.002343in}{2.611640in}}%
\pgfpathlineto{\pgfqpoint{3.002343in}{2.614590in}}%
\pgfpathlineto{\pgfqpoint{3.006885in}{2.614590in}}%
\pgfpathlineto{\pgfqpoint{3.006885in}{2.611640in}}%
\pgfpathmoveto{\pgfqpoint{3.006885in}{2.611640in}}%
\pgfpathlineto{\pgfqpoint{3.006885in}{2.611640in}}%
\pgfpathlineto{\pgfqpoint{3.006885in}{2.614590in}}%
\pgfpathlineto{\pgfqpoint{3.011426in}{2.614590in}}%
\pgfpathlineto{\pgfqpoint{3.011426in}{2.611640in}}%
\pgfpathmoveto{\pgfqpoint{3.011426in}{2.611640in}}%
\pgfpathlineto{\pgfqpoint{3.011426in}{2.611640in}}%
\pgfpathlineto{\pgfqpoint{3.011426in}{2.614590in}}%
\pgfpathlineto{\pgfqpoint{3.015967in}{2.614590in}}%
\pgfpathlineto{\pgfqpoint{3.015967in}{2.611640in}}%
\pgfpathmoveto{\pgfqpoint{3.015967in}{2.611640in}}%
\pgfpathlineto{\pgfqpoint{3.015967in}{2.611640in}}%
\pgfpathlineto{\pgfqpoint{3.015967in}{2.614590in}}%
\pgfpathlineto{\pgfqpoint{3.020508in}{2.614590in}}%
\pgfpathlineto{\pgfqpoint{3.020508in}{2.611640in}}%
\pgfpathmoveto{\pgfqpoint{3.020508in}{2.611640in}}%
\pgfpathlineto{\pgfqpoint{3.020508in}{2.611640in}}%
\pgfpathlineto{\pgfqpoint{3.020508in}{2.614590in}}%
\pgfpathlineto{\pgfqpoint{3.025050in}{2.614590in}}%
\pgfpathlineto{\pgfqpoint{3.025050in}{2.611640in}}%
\pgfpathmoveto{\pgfqpoint{3.025050in}{2.611640in}}%
\pgfpathlineto{\pgfqpoint{3.025050in}{2.611640in}}%
\pgfpathlineto{\pgfqpoint{3.025050in}{2.614590in}}%
\pgfpathlineto{\pgfqpoint{3.029591in}{2.614590in}}%
\pgfpathlineto{\pgfqpoint{3.029591in}{2.611640in}}%
\pgfpathmoveto{\pgfqpoint{3.029591in}{2.611640in}}%
\pgfpathlineto{\pgfqpoint{3.029591in}{2.611640in}}%
\pgfpathlineto{\pgfqpoint{3.029591in}{2.614590in}}%
\pgfpathlineto{\pgfqpoint{3.034132in}{2.614590in}}%
\pgfpathlineto{\pgfqpoint{3.034132in}{2.611640in}}%
\pgfpathmoveto{\pgfqpoint{3.034132in}{2.611640in}}%
\pgfpathlineto{\pgfqpoint{3.034132in}{2.611640in}}%
\pgfpathlineto{\pgfqpoint{3.034132in}{2.614590in}}%
\pgfpathlineto{\pgfqpoint{3.038673in}{2.614590in}}%
\pgfpathlineto{\pgfqpoint{3.038673in}{2.611640in}}%
\pgfpathmoveto{\pgfqpoint{3.038673in}{2.611640in}}%
\pgfpathlineto{\pgfqpoint{3.038673in}{2.611640in}}%
\pgfpathlineto{\pgfqpoint{3.038673in}{2.614590in}}%
\pgfpathlineto{\pgfqpoint{3.043215in}{2.614590in}}%
\pgfpathlineto{\pgfqpoint{3.043215in}{2.611640in}}%
\pgfpathmoveto{\pgfqpoint{3.043215in}{2.611640in}}%
\pgfpathlineto{\pgfqpoint{3.043215in}{2.611640in}}%
\pgfpathlineto{\pgfqpoint{3.043215in}{2.614590in}}%
\pgfpathlineto{\pgfqpoint{3.047756in}{2.614590in}}%
\pgfpathlineto{\pgfqpoint{3.047756in}{2.611640in}}%
\pgfpathmoveto{\pgfqpoint{3.047756in}{2.611640in}}%
\pgfpathlineto{\pgfqpoint{3.047756in}{2.611640in}}%
\pgfpathlineto{\pgfqpoint{3.047756in}{2.614590in}}%
\pgfpathlineto{\pgfqpoint{3.052297in}{2.614590in}}%
\pgfpathlineto{\pgfqpoint{3.052297in}{2.611640in}}%
\pgfpathmoveto{\pgfqpoint{3.052297in}{2.611640in}}%
\pgfpathlineto{\pgfqpoint{3.052297in}{2.611640in}}%
\pgfpathlineto{\pgfqpoint{3.052297in}{2.614590in}}%
\pgfpathlineto{\pgfqpoint{3.056838in}{2.614590in}}%
\pgfpathlineto{\pgfqpoint{3.056838in}{2.611640in}}%
\pgfpathmoveto{\pgfqpoint{3.056838in}{2.611640in}}%
\pgfpathlineto{\pgfqpoint{3.056838in}{2.611640in}}%
\pgfpathlineto{\pgfqpoint{3.056838in}{2.614590in}}%
\pgfpathlineto{\pgfqpoint{3.061380in}{2.614590in}}%
\pgfpathlineto{\pgfqpoint{3.061380in}{2.611640in}}%
\pgfpathmoveto{\pgfqpoint{3.061380in}{2.611640in}}%
\pgfpathlineto{\pgfqpoint{3.061380in}{2.611640in}}%
\pgfpathlineto{\pgfqpoint{3.061380in}{2.614590in}}%
\pgfpathlineto{\pgfqpoint{3.065921in}{2.614590in}}%
\pgfpathlineto{\pgfqpoint{3.065921in}{2.611640in}}%
\pgfpathmoveto{\pgfqpoint{3.065921in}{2.611640in}}%
\pgfpathlineto{\pgfqpoint{3.065921in}{2.611640in}}%
\pgfpathlineto{\pgfqpoint{3.065921in}{2.614590in}}%
\pgfpathlineto{\pgfqpoint{3.070462in}{2.614590in}}%
\pgfpathlineto{\pgfqpoint{3.070462in}{2.611640in}}%
\pgfpathmoveto{\pgfqpoint{3.070462in}{2.611640in}}%
\pgfpathlineto{\pgfqpoint{3.070462in}{2.611640in}}%
\pgfpathlineto{\pgfqpoint{3.070462in}{2.614590in}}%
\pgfpathlineto{\pgfqpoint{3.075003in}{2.614590in}}%
\pgfpathlineto{\pgfqpoint{3.075003in}{2.611640in}}%
\pgfpathmoveto{\pgfqpoint{3.075003in}{2.611640in}}%
\pgfpathlineto{\pgfqpoint{3.075003in}{2.611640in}}%
\pgfpathlineto{\pgfqpoint{3.075003in}{2.614590in}}%
\pgfpathlineto{\pgfqpoint{3.079544in}{2.614590in}}%
\pgfpathlineto{\pgfqpoint{3.079544in}{2.611640in}}%
\pgfpathmoveto{\pgfqpoint{3.079544in}{2.611640in}}%
\pgfpathlineto{\pgfqpoint{3.079544in}{2.611640in}}%
\pgfpathlineto{\pgfqpoint{3.079544in}{2.614590in}}%
\pgfpathlineto{\pgfqpoint{3.084085in}{2.614590in}}%
\pgfpathlineto{\pgfqpoint{3.084085in}{2.611640in}}%
\pgfpathmoveto{\pgfqpoint{3.084085in}{2.611640in}}%
\pgfpathlineto{\pgfqpoint{3.084085in}{2.611640in}}%
\pgfpathlineto{\pgfqpoint{3.084085in}{2.614590in}}%
\pgfpathlineto{\pgfqpoint{3.088626in}{2.614590in}}%
\pgfpathlineto{\pgfqpoint{3.088626in}{2.611640in}}%
\pgfpathmoveto{\pgfqpoint{3.088626in}{2.611640in}}%
\pgfpathlineto{\pgfqpoint{3.088626in}{2.611640in}}%
\pgfpathlineto{\pgfqpoint{3.088626in}{2.614590in}}%
\pgfpathlineto{\pgfqpoint{3.093167in}{2.614590in}}%
\pgfpathlineto{\pgfqpoint{3.093167in}{2.611640in}}%
\pgfpathmoveto{\pgfqpoint{3.093167in}{2.611640in}}%
\pgfpathlineto{\pgfqpoint{3.093167in}{2.611640in}}%
\pgfpathlineto{\pgfqpoint{3.093167in}{2.614590in}}%
\pgfpathlineto{\pgfqpoint{3.097708in}{2.614590in}}%
\pgfpathlineto{\pgfqpoint{3.097708in}{2.611640in}}%
\pgfpathmoveto{\pgfqpoint{3.097708in}{2.611640in}}%
\pgfpathlineto{\pgfqpoint{3.097708in}{2.611640in}}%
\pgfpathlineto{\pgfqpoint{3.097708in}{2.614590in}}%
\pgfpathlineto{\pgfqpoint{3.102249in}{2.614590in}}%
\pgfpathlineto{\pgfqpoint{3.102249in}{2.611640in}}%
\pgfpathmoveto{\pgfqpoint{3.102249in}{2.611640in}}%
\pgfpathlineto{\pgfqpoint{3.102249in}{2.611640in}}%
\pgfpathlineto{\pgfqpoint{3.102249in}{2.614590in}}%
\pgfpathlineto{\pgfqpoint{3.106789in}{2.614590in}}%
\pgfpathlineto{\pgfqpoint{3.106789in}{2.611640in}}%
\pgfpathmoveto{\pgfqpoint{3.106789in}{2.611640in}}%
\pgfpathlineto{\pgfqpoint{3.106789in}{2.611640in}}%
\pgfpathlineto{\pgfqpoint{3.106789in}{2.614590in}}%
\pgfpathlineto{\pgfqpoint{3.111330in}{2.614590in}}%
\pgfpathlineto{\pgfqpoint{3.111330in}{2.611640in}}%
\pgfpathmoveto{\pgfqpoint{3.111330in}{2.611640in}}%
\pgfpathlineto{\pgfqpoint{3.111330in}{2.611640in}}%
\pgfpathlineto{\pgfqpoint{3.111330in}{2.614590in}}%
\pgfpathlineto{\pgfqpoint{3.115871in}{2.614590in}}%
\pgfpathlineto{\pgfqpoint{3.115871in}{2.611640in}}%
\pgfpathmoveto{\pgfqpoint{3.115871in}{2.611640in}}%
\pgfpathlineto{\pgfqpoint{3.115871in}{2.611640in}}%
\pgfpathlineto{\pgfqpoint{3.115871in}{2.614590in}}%
\pgfpathlineto{\pgfqpoint{3.120412in}{2.614590in}}%
\pgfpathlineto{\pgfqpoint{3.120412in}{2.611640in}}%
\pgfpathmoveto{\pgfqpoint{3.120412in}{2.611640in}}%
\pgfpathlineto{\pgfqpoint{3.120412in}{2.611640in}}%
\pgfpathlineto{\pgfqpoint{3.120412in}{2.614590in}}%
\pgfpathlineto{\pgfqpoint{3.124953in}{2.614590in}}%
\pgfpathlineto{\pgfqpoint{3.124953in}{2.611640in}}%
\pgfpathmoveto{\pgfqpoint{3.124953in}{2.611640in}}%
\pgfpathlineto{\pgfqpoint{3.124953in}{2.611640in}}%
\pgfpathlineto{\pgfqpoint{3.124953in}{2.614590in}}%
\pgfpathlineto{\pgfqpoint{3.129494in}{2.614590in}}%
\pgfpathlineto{\pgfqpoint{3.129494in}{2.611640in}}%
\pgfpathmoveto{\pgfqpoint{3.129494in}{2.611640in}}%
\pgfpathlineto{\pgfqpoint{3.129494in}{2.611640in}}%
\pgfpathlineto{\pgfqpoint{3.129494in}{2.614590in}}%
\pgfpathlineto{\pgfqpoint{3.134035in}{2.614590in}}%
\pgfpathlineto{\pgfqpoint{3.134035in}{2.611640in}}%
\pgfpathmoveto{\pgfqpoint{3.134035in}{2.611640in}}%
\pgfpathlineto{\pgfqpoint{3.134035in}{2.611640in}}%
\pgfpathlineto{\pgfqpoint{3.134035in}{2.614590in}}%
\pgfpathlineto{\pgfqpoint{3.138575in}{2.614590in}}%
\pgfpathlineto{\pgfqpoint{3.138575in}{2.611640in}}%
\pgfpathmoveto{\pgfqpoint{3.138575in}{2.611640in}}%
\pgfpathlineto{\pgfqpoint{3.138575in}{2.611640in}}%
\pgfpathlineto{\pgfqpoint{3.138575in}{2.614590in}}%
\pgfpathlineto{\pgfqpoint{3.143116in}{2.614590in}}%
\pgfpathlineto{\pgfqpoint{3.143116in}{2.611640in}}%
\pgfpathmoveto{\pgfqpoint{3.143116in}{2.611640in}}%
\pgfpathlineto{\pgfqpoint{3.143116in}{2.611640in}}%
\pgfpathlineto{\pgfqpoint{3.143116in}{2.614590in}}%
\pgfpathlineto{\pgfqpoint{3.147657in}{2.614590in}}%
\pgfpathlineto{\pgfqpoint{3.147657in}{2.611640in}}%
\pgfpathmoveto{\pgfqpoint{3.147657in}{2.611640in}}%
\pgfpathlineto{\pgfqpoint{3.147657in}{2.611640in}}%
\pgfpathlineto{\pgfqpoint{3.147657in}{2.614590in}}%
\pgfpathlineto{\pgfqpoint{3.152198in}{2.614590in}}%
\pgfpathlineto{\pgfqpoint{3.152198in}{2.611640in}}%
\pgfpathmoveto{\pgfqpoint{3.152198in}{2.611640in}}%
\pgfpathlineto{\pgfqpoint{3.152198in}{2.611640in}}%
\pgfpathlineto{\pgfqpoint{3.152198in}{2.614590in}}%
\pgfpathlineto{\pgfqpoint{3.156739in}{2.614590in}}%
\pgfpathlineto{\pgfqpoint{3.156739in}{2.611640in}}%
\pgfpathmoveto{\pgfqpoint{3.156739in}{2.611640in}}%
\pgfpathlineto{\pgfqpoint{3.156739in}{2.611640in}}%
\pgfpathlineto{\pgfqpoint{3.156739in}{2.614590in}}%
\pgfpathlineto{\pgfqpoint{3.161280in}{2.614590in}}%
\pgfpathlineto{\pgfqpoint{3.161280in}{2.611640in}}%
\pgfpathmoveto{\pgfqpoint{3.161280in}{2.611640in}}%
\pgfpathlineto{\pgfqpoint{3.161280in}{2.611640in}}%
\pgfpathlineto{\pgfqpoint{3.161280in}{2.614590in}}%
\pgfpathlineto{\pgfqpoint{3.165821in}{2.614590in}}%
\pgfpathlineto{\pgfqpoint{3.165821in}{2.611640in}}%
\pgfpathmoveto{\pgfqpoint{3.165821in}{2.611640in}}%
\pgfpathlineto{\pgfqpoint{3.165821in}{2.611640in}}%
\pgfpathlineto{\pgfqpoint{3.165821in}{2.614590in}}%
\pgfpathlineto{\pgfqpoint{3.170361in}{2.614590in}}%
\pgfpathlineto{\pgfqpoint{3.170361in}{2.611640in}}%
\pgfpathmoveto{\pgfqpoint{3.170361in}{2.611640in}}%
\pgfpathlineto{\pgfqpoint{3.170361in}{2.611640in}}%
\pgfpathlineto{\pgfqpoint{3.170361in}{2.614590in}}%
\pgfpathlineto{\pgfqpoint{3.174902in}{2.614590in}}%
\pgfpathlineto{\pgfqpoint{3.174902in}{2.611640in}}%
\pgfpathmoveto{\pgfqpoint{3.174902in}{2.611640in}}%
\pgfpathlineto{\pgfqpoint{3.174902in}{2.611640in}}%
\pgfpathlineto{\pgfqpoint{3.174902in}{2.614590in}}%
\pgfpathlineto{\pgfqpoint{3.179443in}{2.614590in}}%
\pgfpathlineto{\pgfqpoint{3.179443in}{2.611640in}}%
\pgfpathmoveto{\pgfqpoint{3.179443in}{2.611640in}}%
\pgfpathlineto{\pgfqpoint{3.179443in}{2.611640in}}%
\pgfpathlineto{\pgfqpoint{3.179443in}{2.614590in}}%
\pgfpathlineto{\pgfqpoint{3.183984in}{2.614590in}}%
\pgfpathlineto{\pgfqpoint{3.183984in}{2.611640in}}%
\pgfpathmoveto{\pgfqpoint{3.183984in}{2.611640in}}%
\pgfpathlineto{\pgfqpoint{3.183984in}{2.611640in}}%
\pgfpathlineto{\pgfqpoint{3.183984in}{2.614590in}}%
\pgfpathlineto{\pgfqpoint{3.188525in}{2.614590in}}%
\pgfpathlineto{\pgfqpoint{3.188525in}{2.611640in}}%
\pgfpathmoveto{\pgfqpoint{3.188525in}{2.611640in}}%
\pgfpathlineto{\pgfqpoint{3.188525in}{2.611640in}}%
\pgfpathlineto{\pgfqpoint{3.188525in}{2.614590in}}%
\pgfpathlineto{\pgfqpoint{3.193066in}{2.614590in}}%
\pgfpathlineto{\pgfqpoint{3.193066in}{2.611640in}}%
\pgfpathmoveto{\pgfqpoint{3.193066in}{2.611640in}}%
\pgfpathlineto{\pgfqpoint{3.193066in}{2.611640in}}%
\pgfpathlineto{\pgfqpoint{3.193066in}{2.614590in}}%
\pgfpathlineto{\pgfqpoint{3.197606in}{2.614590in}}%
\pgfpathlineto{\pgfqpoint{3.197606in}{2.611640in}}%
\pgfpathmoveto{\pgfqpoint{3.197606in}{2.611640in}}%
\pgfpathlineto{\pgfqpoint{3.197606in}{2.611640in}}%
\pgfpathlineto{\pgfqpoint{3.197606in}{2.614590in}}%
\pgfpathlineto{\pgfqpoint{3.202147in}{2.614590in}}%
\pgfpathlineto{\pgfqpoint{3.202147in}{2.611640in}}%
\pgfpathmoveto{\pgfqpoint{3.202147in}{2.611640in}}%
\pgfpathlineto{\pgfqpoint{3.202147in}{2.611640in}}%
\pgfpathlineto{\pgfqpoint{3.202147in}{2.614590in}}%
\pgfpathlineto{\pgfqpoint{3.206688in}{2.614590in}}%
\pgfpathlineto{\pgfqpoint{3.206688in}{2.611640in}}%
\pgfpathmoveto{\pgfqpoint{3.206688in}{2.611640in}}%
\pgfpathlineto{\pgfqpoint{3.206688in}{2.611640in}}%
\pgfpathlineto{\pgfqpoint{3.206688in}{2.614590in}}%
\pgfpathlineto{\pgfqpoint{3.211229in}{2.614590in}}%
\pgfpathlineto{\pgfqpoint{3.211229in}{2.611640in}}%
\pgfpathmoveto{\pgfqpoint{3.211229in}{2.611640in}}%
\pgfpathlineto{\pgfqpoint{3.211229in}{2.611640in}}%
\pgfpathlineto{\pgfqpoint{3.211229in}{2.614590in}}%
\pgfpathlineto{\pgfqpoint{3.215770in}{2.614590in}}%
\pgfpathlineto{\pgfqpoint{3.215770in}{2.611640in}}%
\pgfpathmoveto{\pgfqpoint{3.215770in}{2.611640in}}%
\pgfpathlineto{\pgfqpoint{3.215770in}{2.611640in}}%
\pgfpathlineto{\pgfqpoint{3.215770in}{2.614590in}}%
\pgfpathlineto{\pgfqpoint{3.220311in}{2.614590in}}%
\pgfpathlineto{\pgfqpoint{3.220311in}{2.611640in}}%
\pgfpathmoveto{\pgfqpoint{3.220311in}{2.611640in}}%
\pgfpathlineto{\pgfqpoint{3.220311in}{2.611640in}}%
\pgfpathlineto{\pgfqpoint{3.220311in}{2.614590in}}%
\pgfpathlineto{\pgfqpoint{3.224852in}{2.614590in}}%
\pgfpathlineto{\pgfqpoint{3.224852in}{2.611640in}}%
\pgfpathmoveto{\pgfqpoint{3.224852in}{2.611640in}}%
\pgfpathlineto{\pgfqpoint{3.224852in}{2.611640in}}%
\pgfpathlineto{\pgfqpoint{3.224852in}{2.614590in}}%
\pgfpathlineto{\pgfqpoint{3.229393in}{2.614590in}}%
\pgfpathlineto{\pgfqpoint{3.229393in}{2.611640in}}%
\pgfpathmoveto{\pgfqpoint{3.229393in}{2.611640in}}%
\pgfpathlineto{\pgfqpoint{3.229393in}{2.611640in}}%
\pgfpathlineto{\pgfqpoint{3.229393in}{2.614590in}}%
\pgfpathlineto{\pgfqpoint{3.233934in}{2.614590in}}%
\pgfpathlineto{\pgfqpoint{3.233934in}{2.611640in}}%
\pgfpathmoveto{\pgfqpoint{3.233934in}{2.611640in}}%
\pgfpathlineto{\pgfqpoint{3.233934in}{2.611640in}}%
\pgfpathlineto{\pgfqpoint{3.233934in}{2.614590in}}%
\pgfpathlineto{\pgfqpoint{3.238475in}{2.614590in}}%
\pgfpathlineto{\pgfqpoint{3.238475in}{2.611640in}}%
\pgfpathmoveto{\pgfqpoint{3.238475in}{2.611640in}}%
\pgfpathlineto{\pgfqpoint{3.238475in}{2.611640in}}%
\pgfpathlineto{\pgfqpoint{3.238475in}{2.614590in}}%
\pgfpathlineto{\pgfqpoint{3.243016in}{2.614590in}}%
\pgfpathlineto{\pgfqpoint{3.243016in}{2.611640in}}%
\pgfpathmoveto{\pgfqpoint{3.243016in}{2.611640in}}%
\pgfpathlineto{\pgfqpoint{3.243016in}{2.611640in}}%
\pgfpathlineto{\pgfqpoint{3.243016in}{2.614590in}}%
\pgfpathlineto{\pgfqpoint{3.247557in}{2.614590in}}%
\pgfpathlineto{\pgfqpoint{3.247557in}{2.611640in}}%
\pgfpathmoveto{\pgfqpoint{3.247557in}{2.611640in}}%
\pgfpathlineto{\pgfqpoint{3.247557in}{2.611640in}}%
\pgfpathlineto{\pgfqpoint{3.247557in}{2.614590in}}%
\pgfpathlineto{\pgfqpoint{3.252097in}{2.614590in}}%
\pgfpathlineto{\pgfqpoint{3.252097in}{2.611640in}}%
\pgfpathmoveto{\pgfqpoint{3.252097in}{2.611640in}}%
\pgfpathlineto{\pgfqpoint{3.252097in}{2.611640in}}%
\pgfpathlineto{\pgfqpoint{3.252097in}{2.614590in}}%
\pgfpathlineto{\pgfqpoint{3.256638in}{2.614590in}}%
\pgfpathlineto{\pgfqpoint{3.256638in}{2.611640in}}%
\pgfpathmoveto{\pgfqpoint{3.256638in}{2.611640in}}%
\pgfpathlineto{\pgfqpoint{3.256638in}{2.611640in}}%
\pgfpathlineto{\pgfqpoint{3.256638in}{2.614590in}}%
\pgfpathlineto{\pgfqpoint{3.261179in}{2.614590in}}%
\pgfpathlineto{\pgfqpoint{3.261179in}{2.611640in}}%
\pgfpathmoveto{\pgfqpoint{3.261179in}{2.611640in}}%
\pgfpathlineto{\pgfqpoint{3.261179in}{2.611640in}}%
\pgfpathlineto{\pgfqpoint{3.261179in}{2.614590in}}%
\pgfpathlineto{\pgfqpoint{3.265720in}{2.614590in}}%
\pgfpathlineto{\pgfqpoint{3.265720in}{2.611640in}}%
\pgfpathmoveto{\pgfqpoint{3.265720in}{2.611640in}}%
\pgfpathlineto{\pgfqpoint{3.265720in}{2.611640in}}%
\pgfpathlineto{\pgfqpoint{3.265720in}{2.614590in}}%
\pgfpathlineto{\pgfqpoint{3.270261in}{2.614590in}}%
\pgfpathlineto{\pgfqpoint{3.270261in}{2.611640in}}%
\pgfpathmoveto{\pgfqpoint{3.270261in}{2.611640in}}%
\pgfpathlineto{\pgfqpoint{3.270261in}{2.611640in}}%
\pgfpathlineto{\pgfqpoint{3.270261in}{2.614590in}}%
\pgfpathlineto{\pgfqpoint{3.274802in}{2.614590in}}%
\pgfpathlineto{\pgfqpoint{3.274802in}{2.611640in}}%
\pgfpathmoveto{\pgfqpoint{3.274802in}{2.611640in}}%
\pgfpathlineto{\pgfqpoint{3.274802in}{2.611640in}}%
\pgfpathlineto{\pgfqpoint{3.274802in}{2.614590in}}%
\pgfpathlineto{\pgfqpoint{3.279343in}{2.614590in}}%
\pgfpathlineto{\pgfqpoint{3.279343in}{2.611640in}}%
\pgfpathmoveto{\pgfqpoint{3.279343in}{2.611640in}}%
\pgfpathlineto{\pgfqpoint{3.279343in}{2.611640in}}%
\pgfpathlineto{\pgfqpoint{3.279343in}{2.614590in}}%
\pgfpathlineto{\pgfqpoint{3.283884in}{2.614590in}}%
\pgfpathlineto{\pgfqpoint{3.283884in}{2.611640in}}%
\pgfpathmoveto{\pgfqpoint{3.283884in}{2.611640in}}%
\pgfpathlineto{\pgfqpoint{3.283884in}{2.611640in}}%
\pgfpathlineto{\pgfqpoint{3.283884in}{2.614590in}}%
\pgfpathlineto{\pgfqpoint{3.288425in}{2.614590in}}%
\pgfpathlineto{\pgfqpoint{3.288425in}{2.611640in}}%
\pgfpathmoveto{\pgfqpoint{3.288425in}{2.611640in}}%
\pgfpathlineto{\pgfqpoint{3.288425in}{2.611640in}}%
\pgfpathlineto{\pgfqpoint{3.288425in}{2.614590in}}%
\pgfpathlineto{\pgfqpoint{3.292966in}{2.614590in}}%
\pgfpathlineto{\pgfqpoint{3.292966in}{2.611640in}}%
\pgfpathmoveto{\pgfqpoint{3.292966in}{2.611640in}}%
\pgfpathlineto{\pgfqpoint{3.292966in}{2.611640in}}%
\pgfpathlineto{\pgfqpoint{3.292966in}{2.614590in}}%
\pgfpathlineto{\pgfqpoint{3.297507in}{2.614590in}}%
\pgfpathlineto{\pgfqpoint{3.297507in}{2.611640in}}%
\pgfpathmoveto{\pgfqpoint{3.297507in}{2.611640in}}%
\pgfpathlineto{\pgfqpoint{3.297507in}{2.611640in}}%
\pgfpathlineto{\pgfqpoint{3.297507in}{2.614590in}}%
\pgfpathlineto{\pgfqpoint{3.302048in}{2.614590in}}%
\pgfpathlineto{\pgfqpoint{3.302048in}{2.611640in}}%
\pgfpathmoveto{\pgfqpoint{3.302048in}{2.611640in}}%
\pgfpathlineto{\pgfqpoint{3.302048in}{2.611640in}}%
\pgfpathlineto{\pgfqpoint{3.302048in}{2.614590in}}%
\pgfpathlineto{\pgfqpoint{3.306589in}{2.614590in}}%
\pgfpathlineto{\pgfqpoint{3.306589in}{2.611640in}}%
\pgfpathmoveto{\pgfqpoint{3.306589in}{2.611640in}}%
\pgfpathlineto{\pgfqpoint{3.306589in}{2.611640in}}%
\pgfpathlineto{\pgfqpoint{3.306589in}{2.614590in}}%
\pgfpathlineto{\pgfqpoint{3.311130in}{2.614590in}}%
\pgfpathlineto{\pgfqpoint{3.311130in}{2.611640in}}%
\pgfpathmoveto{\pgfqpoint{3.311130in}{2.611640in}}%
\pgfpathlineto{\pgfqpoint{3.311130in}{2.611640in}}%
\pgfpathlineto{\pgfqpoint{3.311130in}{2.614590in}}%
\pgfpathlineto{\pgfqpoint{3.315671in}{2.614590in}}%
\pgfpathlineto{\pgfqpoint{3.315671in}{2.611640in}}%
\pgfpathmoveto{\pgfqpoint{3.315671in}{2.611640in}}%
\pgfpathlineto{\pgfqpoint{3.315671in}{2.611640in}}%
\pgfpathlineto{\pgfqpoint{3.315671in}{2.614590in}}%
\pgfpathlineto{\pgfqpoint{3.320212in}{2.614590in}}%
\pgfpathlineto{\pgfqpoint{3.320212in}{2.611640in}}%
\pgfpathmoveto{\pgfqpoint{3.320212in}{2.611640in}}%
\pgfpathlineto{\pgfqpoint{3.320212in}{2.611640in}}%
\pgfpathlineto{\pgfqpoint{3.320212in}{2.614590in}}%
\pgfpathlineto{\pgfqpoint{3.324753in}{2.614590in}}%
\pgfpathlineto{\pgfqpoint{3.324753in}{2.611640in}}%
\pgfpathmoveto{\pgfqpoint{3.324753in}{2.611640in}}%
\pgfpathlineto{\pgfqpoint{3.324753in}{2.611640in}}%
\pgfpathlineto{\pgfqpoint{3.324753in}{2.614590in}}%
\pgfpathlineto{\pgfqpoint{3.329294in}{2.614590in}}%
\pgfpathlineto{\pgfqpoint{3.329294in}{2.611640in}}%
\pgfpathmoveto{\pgfqpoint{3.329294in}{2.611640in}}%
\pgfpathlineto{\pgfqpoint{3.329294in}{2.611640in}}%
\pgfpathlineto{\pgfqpoint{3.329294in}{2.614590in}}%
\pgfpathlineto{\pgfqpoint{3.333835in}{2.614590in}}%
\pgfpathlineto{\pgfqpoint{3.333835in}{2.611640in}}%
\pgfpathmoveto{\pgfqpoint{3.333835in}{2.611640in}}%
\pgfpathlineto{\pgfqpoint{3.333835in}{2.611640in}}%
\pgfpathlineto{\pgfqpoint{3.333835in}{2.614590in}}%
\pgfpathlineto{\pgfqpoint{3.338376in}{2.614590in}}%
\pgfpathlineto{\pgfqpoint{3.338376in}{2.611640in}}%
\pgfpathmoveto{\pgfqpoint{3.338376in}{2.611640in}}%
\pgfpathlineto{\pgfqpoint{3.338376in}{2.611640in}}%
\pgfpathlineto{\pgfqpoint{3.338376in}{2.614590in}}%
\pgfpathlineto{\pgfqpoint{3.342917in}{2.614590in}}%
\pgfpathlineto{\pgfqpoint{3.342917in}{2.611640in}}%
\pgfpathmoveto{\pgfqpoint{3.342917in}{2.611640in}}%
\pgfpathlineto{\pgfqpoint{3.342917in}{2.611640in}}%
\pgfpathlineto{\pgfqpoint{3.342917in}{2.614590in}}%
\pgfpathlineto{\pgfqpoint{3.347458in}{2.614590in}}%
\pgfpathlineto{\pgfqpoint{3.347458in}{2.611640in}}%
\pgfpathmoveto{\pgfqpoint{3.347458in}{2.611640in}}%
\pgfpathlineto{\pgfqpoint{3.347458in}{2.611640in}}%
\pgfpathlineto{\pgfqpoint{3.347458in}{2.614590in}}%
\pgfpathlineto{\pgfqpoint{3.351999in}{2.614590in}}%
\pgfpathlineto{\pgfqpoint{3.351999in}{2.611640in}}%
\pgfpathmoveto{\pgfqpoint{3.351999in}{2.611640in}}%
\pgfpathlineto{\pgfqpoint{3.351999in}{2.611640in}}%
\pgfpathlineto{\pgfqpoint{3.351999in}{2.614590in}}%
\pgfpathlineto{\pgfqpoint{3.356540in}{2.614590in}}%
\pgfpathlineto{\pgfqpoint{3.356540in}{2.611640in}}%
\pgfpathmoveto{\pgfqpoint{3.356540in}{2.611640in}}%
\pgfpathlineto{\pgfqpoint{3.356540in}{2.611640in}}%
\pgfpathlineto{\pgfqpoint{3.356540in}{2.614590in}}%
\pgfpathlineto{\pgfqpoint{3.361081in}{2.614590in}}%
\pgfpathlineto{\pgfqpoint{3.361081in}{2.611640in}}%
\pgfpathmoveto{\pgfqpoint{3.361081in}{2.611640in}}%
\pgfpathlineto{\pgfqpoint{3.361081in}{2.611640in}}%
\pgfpathlineto{\pgfqpoint{3.361081in}{2.614590in}}%
\pgfpathlineto{\pgfqpoint{3.365621in}{2.614590in}}%
\pgfpathlineto{\pgfqpoint{3.365621in}{2.611640in}}%
\pgfpathmoveto{\pgfqpoint{3.365621in}{2.611640in}}%
\pgfpathlineto{\pgfqpoint{3.365621in}{2.611640in}}%
\pgfpathlineto{\pgfqpoint{3.365621in}{2.614590in}}%
\pgfpathlineto{\pgfqpoint{3.370162in}{2.614590in}}%
\pgfpathlineto{\pgfqpoint{3.370162in}{2.611640in}}%
\pgfpathmoveto{\pgfqpoint{3.370162in}{2.611640in}}%
\pgfpathlineto{\pgfqpoint{3.370162in}{2.611640in}}%
\pgfpathlineto{\pgfqpoint{3.370162in}{2.614590in}}%
\pgfpathlineto{\pgfqpoint{3.374704in}{2.614590in}}%
\pgfpathlineto{\pgfqpoint{3.374704in}{2.611640in}}%
\pgfpathmoveto{\pgfqpoint{3.374704in}{2.611640in}}%
\pgfpathlineto{\pgfqpoint{3.374704in}{2.611640in}}%
\pgfpathlineto{\pgfqpoint{3.374704in}{2.614590in}}%
\pgfpathlineto{\pgfqpoint{3.379245in}{2.614590in}}%
\pgfpathlineto{\pgfqpoint{3.379245in}{2.611640in}}%
\pgfpathmoveto{\pgfqpoint{3.379245in}{2.611640in}}%
\pgfpathlineto{\pgfqpoint{3.379245in}{2.611640in}}%
\pgfpathlineto{\pgfqpoint{3.379245in}{2.614590in}}%
\pgfpathlineto{\pgfqpoint{3.383786in}{2.614590in}}%
\pgfpathlineto{\pgfqpoint{3.383786in}{2.611640in}}%
\pgfpathmoveto{\pgfqpoint{3.383786in}{2.611640in}}%
\pgfpathlineto{\pgfqpoint{3.383786in}{2.611640in}}%
\pgfpathlineto{\pgfqpoint{3.383786in}{2.614590in}}%
\pgfpathlineto{\pgfqpoint{3.388327in}{2.614590in}}%
\pgfpathlineto{\pgfqpoint{3.388327in}{2.611640in}}%
\pgfpathmoveto{\pgfqpoint{3.388327in}{2.611640in}}%
\pgfpathlineto{\pgfqpoint{3.388327in}{2.611640in}}%
\pgfpathlineto{\pgfqpoint{3.388327in}{2.614590in}}%
\pgfpathlineto{\pgfqpoint{3.392868in}{2.614590in}}%
\pgfpathlineto{\pgfqpoint{3.392868in}{2.611640in}}%
\pgfpathmoveto{\pgfqpoint{3.392868in}{2.611640in}}%
\pgfpathlineto{\pgfqpoint{3.392868in}{2.611640in}}%
\pgfpathlineto{\pgfqpoint{3.392868in}{2.614590in}}%
\pgfpathlineto{\pgfqpoint{3.397409in}{2.614590in}}%
\pgfpathlineto{\pgfqpoint{3.397409in}{2.611640in}}%
\pgfpathmoveto{\pgfqpoint{3.397409in}{2.611640in}}%
\pgfpathlineto{\pgfqpoint{3.397409in}{2.611640in}}%
\pgfpathlineto{\pgfqpoint{3.397409in}{2.614590in}}%
\pgfpathlineto{\pgfqpoint{3.401950in}{2.614590in}}%
\pgfpathlineto{\pgfqpoint{3.401950in}{2.611640in}}%
\pgfpathmoveto{\pgfqpoint{3.401950in}{2.611640in}}%
\pgfpathlineto{\pgfqpoint{3.401950in}{2.611640in}}%
\pgfpathlineto{\pgfqpoint{3.401950in}{2.614590in}}%
\pgfpathlineto{\pgfqpoint{3.406491in}{2.614590in}}%
\pgfpathlineto{\pgfqpoint{3.406491in}{2.611640in}}%
\pgfpathmoveto{\pgfqpoint{3.406491in}{2.611640in}}%
\pgfpathlineto{\pgfqpoint{3.406491in}{2.611640in}}%
\pgfpathlineto{\pgfqpoint{3.406491in}{2.614590in}}%
\pgfpathlineto{\pgfqpoint{3.411032in}{2.614590in}}%
\pgfpathlineto{\pgfqpoint{3.411032in}{2.611640in}}%
\pgfpathmoveto{\pgfqpoint{3.411032in}{2.611640in}}%
\pgfpathlineto{\pgfqpoint{3.411032in}{2.611640in}}%
\pgfpathlineto{\pgfqpoint{3.411032in}{2.614590in}}%
\pgfpathlineto{\pgfqpoint{3.415573in}{2.614590in}}%
\pgfpathlineto{\pgfqpoint{3.415573in}{2.611640in}}%
\pgfpathmoveto{\pgfqpoint{3.415573in}{2.611640in}}%
\pgfpathlineto{\pgfqpoint{3.415573in}{2.611640in}}%
\pgfpathlineto{\pgfqpoint{3.415573in}{2.614590in}}%
\pgfpathlineto{\pgfqpoint{3.420114in}{2.614590in}}%
\pgfpathlineto{\pgfqpoint{3.420114in}{2.611640in}}%
\pgfpathmoveto{\pgfqpoint{3.420114in}{2.611640in}}%
\pgfpathlineto{\pgfqpoint{3.420114in}{2.611640in}}%
\pgfpathlineto{\pgfqpoint{3.420114in}{2.614590in}}%
\pgfpathlineto{\pgfqpoint{3.424655in}{2.614590in}}%
\pgfpathlineto{\pgfqpoint{3.424655in}{2.611640in}}%
\pgfpathmoveto{\pgfqpoint{3.424655in}{2.611640in}}%
\pgfpathlineto{\pgfqpoint{3.424655in}{2.611640in}}%
\pgfpathlineto{\pgfqpoint{3.424655in}{2.614590in}}%
\pgfpathlineto{\pgfqpoint{3.429196in}{2.614590in}}%
\pgfpathlineto{\pgfqpoint{3.429196in}{2.611640in}}%
\pgfpathmoveto{\pgfqpoint{3.429196in}{2.611640in}}%
\pgfpathlineto{\pgfqpoint{3.429196in}{2.611640in}}%
\pgfpathlineto{\pgfqpoint{3.429196in}{2.614590in}}%
\pgfpathlineto{\pgfqpoint{3.433737in}{2.614590in}}%
\pgfpathlineto{\pgfqpoint{3.433737in}{2.611640in}}%
\pgfpathmoveto{\pgfqpoint{3.433737in}{2.611640in}}%
\pgfpathlineto{\pgfqpoint{3.433737in}{2.611640in}}%
\pgfpathlineto{\pgfqpoint{3.433737in}{2.614590in}}%
\pgfpathlineto{\pgfqpoint{3.438278in}{2.614590in}}%
\pgfpathlineto{\pgfqpoint{3.438278in}{2.611640in}}%
\pgfpathmoveto{\pgfqpoint{3.438278in}{2.611640in}}%
\pgfpathlineto{\pgfqpoint{3.438278in}{2.611640in}}%
\pgfpathlineto{\pgfqpoint{3.438278in}{2.614590in}}%
\pgfpathlineto{\pgfqpoint{3.442819in}{2.614590in}}%
\pgfpathlineto{\pgfqpoint{3.442819in}{2.611640in}}%
\pgfpathmoveto{\pgfqpoint{3.442819in}{2.611640in}}%
\pgfpathlineto{\pgfqpoint{3.442819in}{2.611640in}}%
\pgfpathlineto{\pgfqpoint{3.442819in}{2.614590in}}%
\pgfpathlineto{\pgfqpoint{3.447360in}{2.614590in}}%
\pgfpathlineto{\pgfqpoint{3.447360in}{2.611640in}}%
\pgfpathmoveto{\pgfqpoint{3.447360in}{2.611640in}}%
\pgfpathlineto{\pgfqpoint{3.447360in}{2.611640in}}%
\pgfpathlineto{\pgfqpoint{3.447360in}{2.614590in}}%
\pgfpathlineto{\pgfqpoint{3.451901in}{2.614590in}}%
\pgfpathlineto{\pgfqpoint{3.451901in}{2.611640in}}%
\pgfpathmoveto{\pgfqpoint{3.451901in}{2.611640in}}%
\pgfpathlineto{\pgfqpoint{3.451901in}{2.611640in}}%
\pgfpathlineto{\pgfqpoint{3.451901in}{2.614590in}}%
\pgfpathlineto{\pgfqpoint{3.456442in}{2.614590in}}%
\pgfpathlineto{\pgfqpoint{3.456442in}{2.611640in}}%
\pgfpathmoveto{\pgfqpoint{3.456442in}{2.611640in}}%
\pgfpathlineto{\pgfqpoint{3.456442in}{2.611640in}}%
\pgfpathlineto{\pgfqpoint{3.456442in}{2.614590in}}%
\pgfpathlineto{\pgfqpoint{3.460983in}{2.614590in}}%
\pgfpathlineto{\pgfqpoint{3.460983in}{2.611640in}}%
\pgfpathmoveto{\pgfqpoint{3.460983in}{2.611640in}}%
\pgfpathlineto{\pgfqpoint{3.460983in}{2.611640in}}%
\pgfpathlineto{\pgfqpoint{3.460983in}{2.614590in}}%
\pgfpathlineto{\pgfqpoint{3.465524in}{2.614590in}}%
\pgfpathlineto{\pgfqpoint{3.465524in}{2.611640in}}%
\pgfpathmoveto{\pgfqpoint{3.465524in}{2.611640in}}%
\pgfpathlineto{\pgfqpoint{3.465524in}{2.611640in}}%
\pgfpathlineto{\pgfqpoint{3.465524in}{2.614590in}}%
\pgfpathlineto{\pgfqpoint{3.470065in}{2.614590in}}%
\pgfpathlineto{\pgfqpoint{3.470065in}{2.611640in}}%
\pgfpathmoveto{\pgfqpoint{3.470065in}{2.611640in}}%
\pgfpathlineto{\pgfqpoint{3.470065in}{2.611640in}}%
\pgfpathlineto{\pgfqpoint{3.470065in}{2.614590in}}%
\pgfpathlineto{\pgfqpoint{3.474606in}{2.614590in}}%
\pgfpathlineto{\pgfqpoint{3.474606in}{2.611640in}}%
\pgfpathmoveto{\pgfqpoint{3.474606in}{2.611640in}}%
\pgfpathlineto{\pgfqpoint{3.474606in}{2.611640in}}%
\pgfpathlineto{\pgfqpoint{3.474606in}{2.614590in}}%
\pgfpathlineto{\pgfqpoint{3.479147in}{2.614590in}}%
\pgfpathlineto{\pgfqpoint{3.479147in}{2.611640in}}%
\pgfpathmoveto{\pgfqpoint{3.479147in}{2.611640in}}%
\pgfpathlineto{\pgfqpoint{3.479147in}{2.611640in}}%
\pgfpathlineto{\pgfqpoint{3.479147in}{2.614590in}}%
\pgfpathlineto{\pgfqpoint{3.483688in}{2.614590in}}%
\pgfpathlineto{\pgfqpoint{3.483688in}{2.611640in}}%
\pgfpathmoveto{\pgfqpoint{3.483688in}{2.611640in}}%
\pgfpathlineto{\pgfqpoint{3.483688in}{2.611640in}}%
\pgfpathlineto{\pgfqpoint{3.483688in}{2.614590in}}%
\pgfpathlineto{\pgfqpoint{3.488229in}{2.614590in}}%
\pgfpathlineto{\pgfqpoint{3.488229in}{2.611640in}}%
\pgfpathmoveto{\pgfqpoint{3.488229in}{2.611640in}}%
\pgfpathlineto{\pgfqpoint{3.488229in}{2.611640in}}%
\pgfpathlineto{\pgfqpoint{3.488229in}{2.614590in}}%
\pgfpathlineto{\pgfqpoint{3.492770in}{2.614590in}}%
\pgfpathlineto{\pgfqpoint{3.492770in}{2.611640in}}%
\pgfpathmoveto{\pgfqpoint{3.492770in}{2.611640in}}%
\pgfpathlineto{\pgfqpoint{3.492770in}{2.611640in}}%
\pgfpathlineto{\pgfqpoint{3.492770in}{2.614590in}}%
\pgfpathlineto{\pgfqpoint{3.497311in}{2.614590in}}%
\pgfpathlineto{\pgfqpoint{3.497311in}{2.611640in}}%
\pgfpathmoveto{\pgfqpoint{3.497311in}{2.611640in}}%
\pgfpathlineto{\pgfqpoint{3.497311in}{2.611640in}}%
\pgfpathlineto{\pgfqpoint{3.497311in}{2.614590in}}%
\pgfpathlineto{\pgfqpoint{3.501852in}{2.614590in}}%
\pgfpathlineto{\pgfqpoint{3.501852in}{2.611640in}}%
\pgfpathmoveto{\pgfqpoint{3.501852in}{2.611640in}}%
\pgfpathlineto{\pgfqpoint{3.501852in}{2.611640in}}%
\pgfpathlineto{\pgfqpoint{3.501852in}{2.614590in}}%
\pgfpathlineto{\pgfqpoint{3.506393in}{2.614590in}}%
\pgfpathlineto{\pgfqpoint{3.506393in}{2.611640in}}%
\pgfpathmoveto{\pgfqpoint{3.506393in}{2.611640in}}%
\pgfpathlineto{\pgfqpoint{3.506393in}{2.611640in}}%
\pgfpathlineto{\pgfqpoint{3.506393in}{2.614590in}}%
\pgfpathlineto{\pgfqpoint{3.510934in}{2.614590in}}%
\pgfpathlineto{\pgfqpoint{3.510934in}{2.611640in}}%
\pgfpathmoveto{\pgfqpoint{3.510934in}{2.611640in}}%
\pgfpathlineto{\pgfqpoint{3.510934in}{2.611640in}}%
\pgfpathlineto{\pgfqpoint{3.510934in}{2.614590in}}%
\pgfpathlineto{\pgfqpoint{3.515475in}{2.614590in}}%
\pgfpathlineto{\pgfqpoint{3.515475in}{2.611640in}}%
\pgfpathmoveto{\pgfqpoint{3.515475in}{2.611640in}}%
\pgfpathlineto{\pgfqpoint{3.515475in}{2.611640in}}%
\pgfpathlineto{\pgfqpoint{3.515475in}{2.614590in}}%
\pgfpathlineto{\pgfqpoint{3.520016in}{2.614590in}}%
\pgfpathlineto{\pgfqpoint{3.520016in}{2.611640in}}%
\pgfpathmoveto{\pgfqpoint{3.520016in}{2.611640in}}%
\pgfpathlineto{\pgfqpoint{3.520016in}{2.611640in}}%
\pgfpathlineto{\pgfqpoint{3.520016in}{2.614590in}}%
\pgfpathlineto{\pgfqpoint{3.524558in}{2.614590in}}%
\pgfpathlineto{\pgfqpoint{3.524558in}{2.611640in}}%
\pgfpathmoveto{\pgfqpoint{3.524558in}{2.611640in}}%
\pgfpathlineto{\pgfqpoint{3.524558in}{2.611640in}}%
\pgfpathlineto{\pgfqpoint{3.524558in}{2.614590in}}%
\pgfpathlineto{\pgfqpoint{3.529099in}{2.614590in}}%
\pgfpathlineto{\pgfqpoint{3.529099in}{2.611640in}}%
\pgfpathmoveto{\pgfqpoint{3.529099in}{2.611640in}}%
\pgfpathlineto{\pgfqpoint{3.529099in}{2.611640in}}%
\pgfpathlineto{\pgfqpoint{3.529099in}{2.614590in}}%
\pgfpathlineto{\pgfqpoint{3.533640in}{2.614590in}}%
\pgfpathlineto{\pgfqpoint{3.533640in}{2.611640in}}%
\pgfpathmoveto{\pgfqpoint{3.533640in}{2.611640in}}%
\pgfpathlineto{\pgfqpoint{3.533640in}{2.611640in}}%
\pgfpathlineto{\pgfqpoint{3.533640in}{2.614590in}}%
\pgfpathlineto{\pgfqpoint{3.538181in}{2.614590in}}%
\pgfpathlineto{\pgfqpoint{3.538181in}{2.611640in}}%
\pgfpathmoveto{\pgfqpoint{3.538181in}{2.611640in}}%
\pgfpathlineto{\pgfqpoint{3.538181in}{2.611640in}}%
\pgfpathlineto{\pgfqpoint{3.538181in}{2.614590in}}%
\pgfpathlineto{\pgfqpoint{3.542722in}{2.614590in}}%
\pgfpathlineto{\pgfqpoint{3.542722in}{2.611640in}}%
\pgfpathmoveto{\pgfqpoint{3.542722in}{2.611640in}}%
\pgfpathlineto{\pgfqpoint{3.542722in}{2.611640in}}%
\pgfpathlineto{\pgfqpoint{3.542722in}{2.614590in}}%
\pgfpathlineto{\pgfqpoint{3.547263in}{2.614590in}}%
\pgfpathlineto{\pgfqpoint{3.547263in}{2.611640in}}%
\pgfpathmoveto{\pgfqpoint{3.547263in}{2.611640in}}%
\pgfpathlineto{\pgfqpoint{3.547263in}{2.611640in}}%
\pgfpathlineto{\pgfqpoint{3.547263in}{2.614590in}}%
\pgfpathlineto{\pgfqpoint{3.551804in}{2.614590in}}%
\pgfpathlineto{\pgfqpoint{3.551804in}{2.611640in}}%
\pgfpathmoveto{\pgfqpoint{3.551804in}{2.611640in}}%
\pgfpathlineto{\pgfqpoint{3.551804in}{2.611640in}}%
\pgfpathlineto{\pgfqpoint{3.551804in}{2.614590in}}%
\pgfpathlineto{\pgfqpoint{3.556346in}{2.614590in}}%
\pgfpathlineto{\pgfqpoint{3.556346in}{2.611640in}}%
\pgfpathmoveto{\pgfqpoint{3.556346in}{2.611640in}}%
\pgfpathlineto{\pgfqpoint{3.556346in}{2.611640in}}%
\pgfpathlineto{\pgfqpoint{3.556346in}{2.614590in}}%
\pgfpathlineto{\pgfqpoint{3.560887in}{2.614590in}}%
\pgfpathlineto{\pgfqpoint{3.560887in}{2.611640in}}%
\pgfpathmoveto{\pgfqpoint{3.560887in}{2.611640in}}%
\pgfpathlineto{\pgfqpoint{3.560887in}{2.611640in}}%
\pgfpathlineto{\pgfqpoint{3.560887in}{2.614590in}}%
\pgfpathlineto{\pgfqpoint{3.565428in}{2.614590in}}%
\pgfpathlineto{\pgfqpoint{3.565428in}{2.611640in}}%
\pgfpathmoveto{\pgfqpoint{3.565428in}{2.611640in}}%
\pgfpathlineto{\pgfqpoint{3.565428in}{2.611640in}}%
\pgfpathlineto{\pgfqpoint{3.565428in}{2.614590in}}%
\pgfpathlineto{\pgfqpoint{3.569969in}{2.614590in}}%
\pgfpathlineto{\pgfqpoint{3.569969in}{2.611640in}}%
\pgfpathmoveto{\pgfqpoint{3.569969in}{2.611640in}}%
\pgfpathlineto{\pgfqpoint{3.569969in}{2.611640in}}%
\pgfpathlineto{\pgfqpoint{3.569969in}{2.614590in}}%
\pgfpathlineto{\pgfqpoint{3.574510in}{2.614590in}}%
\pgfpathlineto{\pgfqpoint{3.574510in}{2.611640in}}%
\pgfpathmoveto{\pgfqpoint{3.574510in}{2.611640in}}%
\pgfpathlineto{\pgfqpoint{3.574510in}{2.611640in}}%
\pgfpathlineto{\pgfqpoint{3.574510in}{2.614590in}}%
\pgfpathlineto{\pgfqpoint{3.579051in}{2.614590in}}%
\pgfpathlineto{\pgfqpoint{3.579051in}{2.611640in}}%
\pgfpathmoveto{\pgfqpoint{3.579051in}{2.611640in}}%
\pgfpathlineto{\pgfqpoint{3.579051in}{2.611640in}}%
\pgfpathlineto{\pgfqpoint{3.579051in}{2.614590in}}%
\pgfpathlineto{\pgfqpoint{3.583592in}{2.614590in}}%
\pgfpathlineto{\pgfqpoint{3.583592in}{2.611640in}}%
\pgfpathmoveto{\pgfqpoint{3.583592in}{2.611640in}}%
\pgfpathlineto{\pgfqpoint{3.583592in}{2.611640in}}%
\pgfpathlineto{\pgfqpoint{3.583592in}{2.614590in}}%
\pgfpathlineto{\pgfqpoint{3.588133in}{2.614590in}}%
\pgfpathlineto{\pgfqpoint{3.588133in}{2.611640in}}%
\pgfpathmoveto{\pgfqpoint{3.588133in}{2.611640in}}%
\pgfpathlineto{\pgfqpoint{3.588133in}{2.611640in}}%
\pgfpathlineto{\pgfqpoint{3.588133in}{2.614590in}}%
\pgfpathlineto{\pgfqpoint{3.592675in}{2.614590in}}%
\pgfpathlineto{\pgfqpoint{3.592675in}{2.611640in}}%
\pgfpathmoveto{\pgfqpoint{3.592675in}{2.611640in}}%
\pgfpathlineto{\pgfqpoint{3.592675in}{2.611640in}}%
\pgfpathlineto{\pgfqpoint{3.592675in}{2.614590in}}%
\pgfpathlineto{\pgfqpoint{3.597216in}{2.614590in}}%
\pgfpathlineto{\pgfqpoint{3.597216in}{2.611640in}}%
\pgfpathmoveto{\pgfqpoint{3.597216in}{2.611640in}}%
\pgfpathlineto{\pgfqpoint{3.597216in}{2.611640in}}%
\pgfpathlineto{\pgfqpoint{3.597216in}{2.614590in}}%
\pgfpathlineto{\pgfqpoint{3.601757in}{2.614590in}}%
\pgfpathlineto{\pgfqpoint{3.601757in}{2.611640in}}%
\pgfpathmoveto{\pgfqpoint{3.601757in}{2.611640in}}%
\pgfpathlineto{\pgfqpoint{3.601757in}{2.611640in}}%
\pgfpathlineto{\pgfqpoint{3.601757in}{2.614590in}}%
\pgfpathlineto{\pgfqpoint{3.606298in}{2.614590in}}%
\pgfpathlineto{\pgfqpoint{3.606298in}{2.611640in}}%
\pgfpathmoveto{\pgfqpoint{3.606298in}{2.611640in}}%
\pgfpathlineto{\pgfqpoint{3.606298in}{2.611640in}}%
\pgfpathlineto{\pgfqpoint{3.606298in}{2.614590in}}%
\pgfpathlineto{\pgfqpoint{3.610839in}{2.614590in}}%
\pgfpathlineto{\pgfqpoint{3.610839in}{2.611640in}}%
\pgfpathmoveto{\pgfqpoint{3.610839in}{2.611640in}}%
\pgfpathlineto{\pgfqpoint{3.610839in}{2.611640in}}%
\pgfpathlineto{\pgfqpoint{3.610839in}{2.614590in}}%
\pgfpathlineto{\pgfqpoint{3.615380in}{2.614590in}}%
\pgfpathlineto{\pgfqpoint{3.615380in}{2.611640in}}%
\pgfpathmoveto{\pgfqpoint{3.615380in}{2.611640in}}%
\pgfpathlineto{\pgfqpoint{3.615380in}{2.611640in}}%
\pgfpathlineto{\pgfqpoint{3.615380in}{2.614590in}}%
\pgfpathlineto{\pgfqpoint{3.619921in}{2.614590in}}%
\pgfpathlineto{\pgfqpoint{3.619921in}{2.611640in}}%
\pgfpathmoveto{\pgfqpoint{3.619921in}{2.611640in}}%
\pgfpathlineto{\pgfqpoint{3.619921in}{2.611640in}}%
\pgfpathlineto{\pgfqpoint{3.619921in}{2.614590in}}%
\pgfpathlineto{\pgfqpoint{3.624463in}{2.614590in}}%
\pgfpathlineto{\pgfqpoint{3.624463in}{2.611640in}}%
\pgfpathmoveto{\pgfqpoint{3.624463in}{2.611640in}}%
\pgfpathlineto{\pgfqpoint{3.624463in}{2.611640in}}%
\pgfpathlineto{\pgfqpoint{3.624463in}{2.614590in}}%
\pgfpathlineto{\pgfqpoint{3.629004in}{2.614590in}}%
\pgfpathlineto{\pgfqpoint{3.629004in}{2.611640in}}%
\pgfpathmoveto{\pgfqpoint{3.629004in}{2.611640in}}%
\pgfpathlineto{\pgfqpoint{3.629004in}{2.611640in}}%
\pgfpathlineto{\pgfqpoint{3.629004in}{2.614590in}}%
\pgfpathlineto{\pgfqpoint{3.633545in}{2.614590in}}%
\pgfpathlineto{\pgfqpoint{3.633545in}{2.611640in}}%
\pgfpathmoveto{\pgfqpoint{3.633545in}{2.611640in}}%
\pgfpathlineto{\pgfqpoint{3.633545in}{2.611640in}}%
\pgfpathlineto{\pgfqpoint{3.633545in}{2.614590in}}%
\pgfpathlineto{\pgfqpoint{3.638086in}{2.614590in}}%
\pgfpathlineto{\pgfqpoint{3.638086in}{2.611640in}}%
\pgfpathmoveto{\pgfqpoint{3.638086in}{2.611640in}}%
\pgfpathlineto{\pgfqpoint{3.638086in}{2.611640in}}%
\pgfpathlineto{\pgfqpoint{3.638086in}{2.614590in}}%
\pgfpathlineto{\pgfqpoint{3.642627in}{2.614590in}}%
\pgfpathlineto{\pgfqpoint{3.642627in}{2.611640in}}%
\pgfpathmoveto{\pgfqpoint{3.642627in}{2.611640in}}%
\pgfpathlineto{\pgfqpoint{3.642627in}{2.611640in}}%
\pgfpathlineto{\pgfqpoint{3.642627in}{2.614590in}}%
\pgfpathlineto{\pgfqpoint{3.647168in}{2.614590in}}%
\pgfpathlineto{\pgfqpoint{3.647168in}{2.611640in}}%
\pgfpathmoveto{\pgfqpoint{3.647168in}{2.611640in}}%
\pgfpathlineto{\pgfqpoint{3.647168in}{2.611640in}}%
\pgfpathlineto{\pgfqpoint{3.647168in}{2.614590in}}%
\pgfpathlineto{\pgfqpoint{3.651709in}{2.614590in}}%
\pgfpathlineto{\pgfqpoint{3.651709in}{2.611640in}}%
\pgfpathmoveto{\pgfqpoint{3.651709in}{2.611640in}}%
\pgfpathlineto{\pgfqpoint{3.651709in}{2.611640in}}%
\pgfpathlineto{\pgfqpoint{3.651709in}{2.614590in}}%
\pgfpathlineto{\pgfqpoint{3.656250in}{2.614590in}}%
\pgfpathlineto{\pgfqpoint{3.656250in}{2.611640in}}%
\pgfpathmoveto{\pgfqpoint{3.656250in}{2.611640in}}%
\pgfpathlineto{\pgfqpoint{3.656250in}{2.611640in}}%
\pgfpathlineto{\pgfqpoint{3.656250in}{2.614590in}}%
\pgfpathlineto{\pgfqpoint{3.660791in}{2.614590in}}%
\pgfpathlineto{\pgfqpoint{3.660791in}{2.611640in}}%
\pgfpathmoveto{\pgfqpoint{3.660791in}{2.611640in}}%
\pgfpathlineto{\pgfqpoint{3.660791in}{2.611640in}}%
\pgfpathlineto{\pgfqpoint{3.660791in}{2.614590in}}%
\pgfpathlineto{\pgfqpoint{3.665332in}{2.614590in}}%
\pgfpathlineto{\pgfqpoint{3.665332in}{2.611640in}}%
\pgfpathmoveto{\pgfqpoint{3.665332in}{2.611640in}}%
\pgfpathlineto{\pgfqpoint{3.665332in}{2.611640in}}%
\pgfpathlineto{\pgfqpoint{3.665332in}{2.614590in}}%
\pgfpathlineto{\pgfqpoint{3.669873in}{2.614590in}}%
\pgfpathlineto{\pgfqpoint{3.669873in}{2.611640in}}%
\pgfpathmoveto{\pgfqpoint{3.669873in}{2.611640in}}%
\pgfpathlineto{\pgfqpoint{3.669873in}{2.611640in}}%
\pgfpathlineto{\pgfqpoint{3.669873in}{2.614590in}}%
\pgfpathlineto{\pgfqpoint{3.674414in}{2.614590in}}%
\pgfpathlineto{\pgfqpoint{3.674414in}{2.611640in}}%
\pgfpathmoveto{\pgfqpoint{3.674414in}{2.611640in}}%
\pgfpathlineto{\pgfqpoint{3.674414in}{2.611640in}}%
\pgfpathlineto{\pgfqpoint{3.674414in}{2.614590in}}%
\pgfpathlineto{\pgfqpoint{3.678955in}{2.614590in}}%
\pgfpathlineto{\pgfqpoint{3.678955in}{2.611640in}}%
\pgfpathmoveto{\pgfqpoint{3.678955in}{2.611640in}}%
\pgfpathlineto{\pgfqpoint{3.678955in}{2.611640in}}%
\pgfpathlineto{\pgfqpoint{3.678955in}{2.614590in}}%
\pgfpathlineto{\pgfqpoint{3.683496in}{2.614590in}}%
\pgfpathlineto{\pgfqpoint{3.683496in}{2.611640in}}%
\pgfpathmoveto{\pgfqpoint{3.683496in}{2.611640in}}%
\pgfpathlineto{\pgfqpoint{3.683496in}{2.611640in}}%
\pgfpathlineto{\pgfqpoint{3.683496in}{2.614590in}}%
\pgfpathlineto{\pgfqpoint{3.688037in}{2.614590in}}%
\pgfpathlineto{\pgfqpoint{3.688037in}{2.611640in}}%
\pgfpathmoveto{\pgfqpoint{3.688037in}{2.611640in}}%
\pgfpathlineto{\pgfqpoint{3.688037in}{2.611640in}}%
\pgfpathlineto{\pgfqpoint{3.688037in}{2.614590in}}%
\pgfpathlineto{\pgfqpoint{3.692578in}{2.614590in}}%
\pgfpathlineto{\pgfqpoint{3.692578in}{2.611640in}}%
\pgfpathmoveto{\pgfqpoint{3.692578in}{2.611640in}}%
\pgfpathlineto{\pgfqpoint{3.692578in}{2.611640in}}%
\pgfpathlineto{\pgfqpoint{3.692578in}{2.614590in}}%
\pgfpathlineto{\pgfqpoint{3.697119in}{2.614590in}}%
\pgfpathlineto{\pgfqpoint{3.697119in}{2.611640in}}%
\pgfpathmoveto{\pgfqpoint{3.697119in}{2.611640in}}%
\pgfpathlineto{\pgfqpoint{3.697119in}{2.611640in}}%
\pgfpathlineto{\pgfqpoint{3.697119in}{2.614590in}}%
\pgfpathlineto{\pgfqpoint{3.701660in}{2.614590in}}%
\pgfpathlineto{\pgfqpoint{3.701660in}{2.611640in}}%
\pgfpathmoveto{\pgfqpoint{3.701660in}{2.611640in}}%
\pgfpathlineto{\pgfqpoint{3.701660in}{2.611640in}}%
\pgfpathlineto{\pgfqpoint{3.701660in}{2.614590in}}%
\pgfpathlineto{\pgfqpoint{3.706201in}{2.614590in}}%
\pgfpathlineto{\pgfqpoint{3.706201in}{2.611640in}}%
\pgfpathmoveto{\pgfqpoint{3.706201in}{2.611640in}}%
\pgfpathlineto{\pgfqpoint{3.706201in}{2.611640in}}%
\pgfpathlineto{\pgfqpoint{3.706201in}{2.614590in}}%
\pgfpathlineto{\pgfqpoint{3.710742in}{2.614590in}}%
\pgfpathlineto{\pgfqpoint{3.710742in}{2.611640in}}%
\pgfpathmoveto{\pgfqpoint{3.710742in}{2.611640in}}%
\pgfpathlineto{\pgfqpoint{3.710742in}{2.611640in}}%
\pgfpathlineto{\pgfqpoint{3.710742in}{2.614590in}}%
\pgfpathlineto{\pgfqpoint{3.715283in}{2.614590in}}%
\pgfpathlineto{\pgfqpoint{3.715283in}{2.611640in}}%
\pgfpathmoveto{\pgfqpoint{3.715283in}{2.611640in}}%
\pgfpathlineto{\pgfqpoint{3.715283in}{2.611640in}}%
\pgfpathlineto{\pgfqpoint{3.715283in}{2.614590in}}%
\pgfpathlineto{\pgfqpoint{3.719824in}{2.614590in}}%
\pgfpathlineto{\pgfqpoint{3.719824in}{2.611640in}}%
\pgfpathmoveto{\pgfqpoint{3.719824in}{2.611640in}}%
\pgfpathlineto{\pgfqpoint{3.719824in}{2.611640in}}%
\pgfpathlineto{\pgfqpoint{3.719824in}{2.614590in}}%
\pgfpathlineto{\pgfqpoint{3.724365in}{2.614590in}}%
\pgfpathlineto{\pgfqpoint{3.724365in}{2.611640in}}%
\pgfpathmoveto{\pgfqpoint{3.724365in}{2.611640in}}%
\pgfpathlineto{\pgfqpoint{3.724365in}{2.611640in}}%
\pgfpathlineto{\pgfqpoint{3.724365in}{2.614590in}}%
\pgfpathlineto{\pgfqpoint{3.728906in}{2.614590in}}%
\pgfpathlineto{\pgfqpoint{3.728906in}{2.611640in}}%
\pgfpathmoveto{\pgfqpoint{3.728906in}{2.611640in}}%
\pgfpathlineto{\pgfqpoint{3.728906in}{2.611640in}}%
\pgfpathlineto{\pgfqpoint{3.728906in}{2.614590in}}%
\pgfpathlineto{\pgfqpoint{3.733447in}{2.614590in}}%
\pgfpathlineto{\pgfqpoint{3.733447in}{2.611640in}}%
\pgfpathmoveto{\pgfqpoint{3.733447in}{2.611640in}}%
\pgfpathlineto{\pgfqpoint{3.733447in}{2.611640in}}%
\pgfpathlineto{\pgfqpoint{3.733447in}{2.614590in}}%
\pgfpathlineto{\pgfqpoint{3.737987in}{2.614590in}}%
\pgfpathlineto{\pgfqpoint{3.737987in}{2.611640in}}%
\pgfpathmoveto{\pgfqpoint{3.737987in}{2.611640in}}%
\pgfpathlineto{\pgfqpoint{3.737987in}{2.611640in}}%
\pgfpathlineto{\pgfqpoint{3.737987in}{2.614590in}}%
\pgfpathlineto{\pgfqpoint{3.742528in}{2.614590in}}%
\pgfpathlineto{\pgfqpoint{3.742528in}{2.611640in}}%
\pgfpathmoveto{\pgfqpoint{3.742528in}{2.611640in}}%
\pgfpathlineto{\pgfqpoint{3.742528in}{2.611640in}}%
\pgfpathlineto{\pgfqpoint{3.742528in}{2.614590in}}%
\pgfpathlineto{\pgfqpoint{3.747069in}{2.614590in}}%
\pgfpathlineto{\pgfqpoint{3.747069in}{2.611640in}}%
\pgfpathmoveto{\pgfqpoint{3.747069in}{2.611640in}}%
\pgfpathlineto{\pgfqpoint{3.747069in}{2.611640in}}%
\pgfpathlineto{\pgfqpoint{3.747069in}{2.614590in}}%
\pgfpathlineto{\pgfqpoint{3.751610in}{2.614590in}}%
\pgfpathlineto{\pgfqpoint{3.751610in}{2.611640in}}%
\pgfpathmoveto{\pgfqpoint{3.751610in}{2.611640in}}%
\pgfpathlineto{\pgfqpoint{3.751610in}{2.611640in}}%
\pgfpathlineto{\pgfqpoint{3.751610in}{2.614590in}}%
\pgfpathlineto{\pgfqpoint{3.756151in}{2.614590in}}%
\pgfpathlineto{\pgfqpoint{3.756151in}{2.611640in}}%
\pgfpathmoveto{\pgfqpoint{3.756151in}{2.611640in}}%
\pgfpathlineto{\pgfqpoint{3.756151in}{2.611640in}}%
\pgfpathlineto{\pgfqpoint{3.756151in}{2.614590in}}%
\pgfpathlineto{\pgfqpoint{3.760692in}{2.614590in}}%
\pgfpathlineto{\pgfqpoint{3.760692in}{2.611640in}}%
\pgfpathmoveto{\pgfqpoint{3.760692in}{2.611640in}}%
\pgfpathlineto{\pgfqpoint{3.760692in}{2.611640in}}%
\pgfpathlineto{\pgfqpoint{3.760692in}{2.614590in}}%
\pgfpathlineto{\pgfqpoint{3.765233in}{2.614590in}}%
\pgfpathlineto{\pgfqpoint{3.765233in}{2.611640in}}%
\pgfpathmoveto{\pgfqpoint{3.765233in}{2.611640in}}%
\pgfpathlineto{\pgfqpoint{3.765233in}{2.611640in}}%
\pgfpathlineto{\pgfqpoint{3.765233in}{2.614590in}}%
\pgfpathlineto{\pgfqpoint{3.769774in}{2.614590in}}%
\pgfpathlineto{\pgfqpoint{3.769774in}{2.611640in}}%
\pgfpathmoveto{\pgfqpoint{3.769774in}{2.611640in}}%
\pgfpathlineto{\pgfqpoint{3.769774in}{2.611640in}}%
\pgfpathlineto{\pgfqpoint{3.769774in}{2.614590in}}%
\pgfpathlineto{\pgfqpoint{3.774315in}{2.614590in}}%
\pgfpathlineto{\pgfqpoint{3.774315in}{2.611640in}}%
\pgfpathmoveto{\pgfqpoint{3.774315in}{2.611640in}}%
\pgfpathlineto{\pgfqpoint{3.774315in}{2.611640in}}%
\pgfpathlineto{\pgfqpoint{3.774315in}{2.614590in}}%
\pgfpathlineto{\pgfqpoint{3.778856in}{2.614590in}}%
\pgfpathlineto{\pgfqpoint{3.778856in}{2.611640in}}%
\pgfpathmoveto{\pgfqpoint{3.778856in}{2.611640in}}%
\pgfpathlineto{\pgfqpoint{3.778856in}{2.611640in}}%
\pgfpathlineto{\pgfqpoint{3.778856in}{2.614590in}}%
\pgfpathlineto{\pgfqpoint{3.783397in}{2.614590in}}%
\pgfpathlineto{\pgfqpoint{3.783397in}{2.611640in}}%
\pgfpathmoveto{\pgfqpoint{3.783397in}{2.611640in}}%
\pgfpathlineto{\pgfqpoint{3.783397in}{2.611640in}}%
\pgfpathlineto{\pgfqpoint{3.783397in}{2.614590in}}%
\pgfpathlineto{\pgfqpoint{3.787938in}{2.614590in}}%
\pgfpathlineto{\pgfqpoint{3.787938in}{2.611640in}}%
\pgfpathmoveto{\pgfqpoint{3.787938in}{2.611640in}}%
\pgfpathlineto{\pgfqpoint{3.787938in}{2.611640in}}%
\pgfpathlineto{\pgfqpoint{3.787938in}{2.614590in}}%
\pgfpathlineto{\pgfqpoint{3.792479in}{2.614590in}}%
\pgfpathlineto{\pgfqpoint{3.792479in}{2.611640in}}%
\pgfpathmoveto{\pgfqpoint{3.792479in}{2.611640in}}%
\pgfpathlineto{\pgfqpoint{3.792479in}{2.611640in}}%
\pgfpathlineto{\pgfqpoint{3.792479in}{2.614590in}}%
\pgfpathlineto{\pgfqpoint{3.797020in}{2.614590in}}%
\pgfpathlineto{\pgfqpoint{3.797020in}{2.611640in}}%
\pgfpathmoveto{\pgfqpoint{3.797020in}{2.611640in}}%
\pgfpathlineto{\pgfqpoint{3.797020in}{2.611640in}}%
\pgfpathlineto{\pgfqpoint{3.797020in}{2.614590in}}%
\pgfpathlineto{\pgfqpoint{3.801561in}{2.614590in}}%
\pgfpathlineto{\pgfqpoint{3.801561in}{2.611640in}}%
\pgfpathmoveto{\pgfqpoint{3.801561in}{2.611640in}}%
\pgfpathlineto{\pgfqpoint{3.801561in}{2.611640in}}%
\pgfpathlineto{\pgfqpoint{3.801561in}{2.614590in}}%
\pgfpathlineto{\pgfqpoint{3.806102in}{2.614590in}}%
\pgfpathlineto{\pgfqpoint{3.806102in}{2.611640in}}%
\pgfpathmoveto{\pgfqpoint{3.806102in}{2.611640in}}%
\pgfpathlineto{\pgfqpoint{3.806102in}{2.611640in}}%
\pgfpathlineto{\pgfqpoint{3.806102in}{2.614590in}}%
\pgfpathlineto{\pgfqpoint{3.810643in}{2.614590in}}%
\pgfpathlineto{\pgfqpoint{3.810643in}{2.611640in}}%
\pgfpathmoveto{\pgfqpoint{3.810643in}{2.611640in}}%
\pgfpathlineto{\pgfqpoint{3.810643in}{2.611640in}}%
\pgfpathlineto{\pgfqpoint{3.810643in}{2.614590in}}%
\pgfpathlineto{\pgfqpoint{3.815184in}{2.614590in}}%
\pgfpathlineto{\pgfqpoint{3.815184in}{2.611640in}}%
\pgfpathmoveto{\pgfqpoint{3.815184in}{2.611640in}}%
\pgfpathlineto{\pgfqpoint{3.815184in}{2.611640in}}%
\pgfpathlineto{\pgfqpoint{3.815184in}{2.614590in}}%
\pgfpathlineto{\pgfqpoint{3.819725in}{2.614590in}}%
\pgfpathlineto{\pgfqpoint{3.819725in}{2.611640in}}%
\pgfpathmoveto{\pgfqpoint{3.819725in}{2.611640in}}%
\pgfpathlineto{\pgfqpoint{3.819725in}{2.611640in}}%
\pgfpathlineto{\pgfqpoint{3.819725in}{2.614590in}}%
\pgfpathlineto{\pgfqpoint{3.824266in}{2.614590in}}%
\pgfpathlineto{\pgfqpoint{3.824266in}{2.611640in}}%
\pgfpathmoveto{\pgfqpoint{3.824266in}{2.611640in}}%
\pgfpathlineto{\pgfqpoint{3.824266in}{2.611640in}}%
\pgfpathlineto{\pgfqpoint{3.824266in}{2.614590in}}%
\pgfpathlineto{\pgfqpoint{3.828807in}{2.614590in}}%
\pgfpathlineto{\pgfqpoint{3.828807in}{2.611640in}}%
\pgfpathmoveto{\pgfqpoint{3.828807in}{2.611640in}}%
\pgfpathlineto{\pgfqpoint{3.828807in}{2.611640in}}%
\pgfpathlineto{\pgfqpoint{3.828807in}{2.614590in}}%
\pgfpathlineto{\pgfqpoint{3.833348in}{2.614590in}}%
\pgfpathlineto{\pgfqpoint{3.833348in}{2.611640in}}%
\pgfpathmoveto{\pgfqpoint{3.833348in}{2.611640in}}%
\pgfpathlineto{\pgfqpoint{3.833348in}{2.611640in}}%
\pgfpathlineto{\pgfqpoint{3.833348in}{2.614590in}}%
\pgfpathlineto{\pgfqpoint{3.837889in}{2.614590in}}%
\pgfpathlineto{\pgfqpoint{3.837889in}{2.611640in}}%
\pgfpathmoveto{\pgfqpoint{3.837889in}{2.611640in}}%
\pgfpathlineto{\pgfqpoint{3.837889in}{2.611640in}}%
\pgfpathlineto{\pgfqpoint{3.837889in}{2.614590in}}%
\pgfpathlineto{\pgfqpoint{3.842430in}{2.614590in}}%
\pgfpathlineto{\pgfqpoint{3.842430in}{2.611640in}}%
\pgfpathmoveto{\pgfqpoint{3.842430in}{2.611640in}}%
\pgfpathlineto{\pgfqpoint{3.842430in}{2.611640in}}%
\pgfpathlineto{\pgfqpoint{3.842430in}{2.614590in}}%
\pgfpathlineto{\pgfqpoint{3.846971in}{2.614590in}}%
\pgfpathlineto{\pgfqpoint{3.846971in}{2.611640in}}%
\pgfpathmoveto{\pgfqpoint{3.846971in}{2.611640in}}%
\pgfpathlineto{\pgfqpoint{3.846971in}{2.611640in}}%
\pgfpathlineto{\pgfqpoint{3.846971in}{2.614590in}}%
\pgfpathlineto{\pgfqpoint{3.851512in}{2.614590in}}%
\pgfpathlineto{\pgfqpoint{3.851512in}{2.611640in}}%
\pgfpathmoveto{\pgfqpoint{3.851512in}{2.611640in}}%
\pgfpathlineto{\pgfqpoint{3.851512in}{2.611640in}}%
\pgfpathlineto{\pgfqpoint{3.851512in}{2.614590in}}%
\pgfpathlineto{\pgfqpoint{3.856053in}{2.614590in}}%
\pgfpathlineto{\pgfqpoint{3.856053in}{2.611640in}}%
\pgfpathmoveto{\pgfqpoint{3.856053in}{2.611640in}}%
\pgfpathlineto{\pgfqpoint{3.856053in}{2.611640in}}%
\pgfpathlineto{\pgfqpoint{3.856053in}{2.614590in}}%
\pgfpathlineto{\pgfqpoint{3.860594in}{2.614590in}}%
\pgfpathlineto{\pgfqpoint{3.860594in}{2.611640in}}%
\pgfpathmoveto{\pgfqpoint{3.860594in}{2.611640in}}%
\pgfpathlineto{\pgfqpoint{3.860594in}{2.611640in}}%
\pgfpathlineto{\pgfqpoint{3.860594in}{2.614590in}}%
\pgfpathlineto{\pgfqpoint{3.865136in}{2.614590in}}%
\pgfpathlineto{\pgfqpoint{3.865136in}{2.611640in}}%
\pgfpathmoveto{\pgfqpoint{3.865136in}{2.611640in}}%
\pgfpathlineto{\pgfqpoint{3.865136in}{2.611640in}}%
\pgfpathlineto{\pgfqpoint{3.865136in}{2.614590in}}%
\pgfpathlineto{\pgfqpoint{3.869677in}{2.614590in}}%
\pgfpathlineto{\pgfqpoint{3.869677in}{2.611640in}}%
\pgfpathmoveto{\pgfqpoint{3.869677in}{2.611640in}}%
\pgfpathlineto{\pgfqpoint{3.869677in}{2.611640in}}%
\pgfpathlineto{\pgfqpoint{3.869677in}{2.614590in}}%
\pgfpathlineto{\pgfqpoint{3.874218in}{2.614590in}}%
\pgfpathlineto{\pgfqpoint{3.874218in}{2.611640in}}%
\pgfpathmoveto{\pgfqpoint{3.874218in}{2.611640in}}%
\pgfpathlineto{\pgfqpoint{3.874218in}{2.611640in}}%
\pgfpathlineto{\pgfqpoint{3.874218in}{2.614590in}}%
\pgfpathlineto{\pgfqpoint{3.878759in}{2.614590in}}%
\pgfpathlineto{\pgfqpoint{3.878759in}{2.611640in}}%
\pgfpathmoveto{\pgfqpoint{3.878759in}{2.611640in}}%
\pgfpathlineto{\pgfqpoint{3.878759in}{2.611640in}}%
\pgfpathlineto{\pgfqpoint{3.878759in}{2.614590in}}%
\pgfpathlineto{\pgfqpoint{3.883300in}{2.614590in}}%
\pgfpathlineto{\pgfqpoint{3.883300in}{2.611640in}}%
\pgfpathmoveto{\pgfqpoint{3.883300in}{2.611640in}}%
\pgfpathlineto{\pgfqpoint{3.883300in}{2.611640in}}%
\pgfpathlineto{\pgfqpoint{3.883300in}{2.614590in}}%
\pgfpathlineto{\pgfqpoint{3.887841in}{2.614590in}}%
\pgfpathlineto{\pgfqpoint{3.887841in}{2.611640in}}%
\pgfpathmoveto{\pgfqpoint{3.887841in}{2.611640in}}%
\pgfpathlineto{\pgfqpoint{3.887841in}{2.611640in}}%
\pgfpathlineto{\pgfqpoint{3.887841in}{2.614590in}}%
\pgfpathlineto{\pgfqpoint{3.892382in}{2.614590in}}%
\pgfpathlineto{\pgfqpoint{3.892382in}{2.611640in}}%
\pgfpathmoveto{\pgfqpoint{3.892382in}{2.611640in}}%
\pgfpathlineto{\pgfqpoint{3.892382in}{2.611640in}}%
\pgfpathlineto{\pgfqpoint{3.892382in}{2.614590in}}%
\pgfpathlineto{\pgfqpoint{3.896923in}{2.614590in}}%
\pgfpathlineto{\pgfqpoint{3.896923in}{2.611640in}}%
\pgfpathmoveto{\pgfqpoint{3.896923in}{2.611640in}}%
\pgfpathlineto{\pgfqpoint{3.896923in}{2.611640in}}%
\pgfpathlineto{\pgfqpoint{3.896923in}{2.614590in}}%
\pgfpathlineto{\pgfqpoint{3.901464in}{2.614590in}}%
\pgfpathlineto{\pgfqpoint{3.901464in}{2.611640in}}%
\pgfpathmoveto{\pgfqpoint{3.901464in}{2.611640in}}%
\pgfpathlineto{\pgfqpoint{3.901464in}{2.611640in}}%
\pgfpathlineto{\pgfqpoint{3.901464in}{2.614590in}}%
\pgfpathlineto{\pgfqpoint{3.906005in}{2.614590in}}%
\pgfpathlineto{\pgfqpoint{3.906005in}{2.611640in}}%
\pgfpathmoveto{\pgfqpoint{3.906005in}{2.611640in}}%
\pgfpathlineto{\pgfqpoint{3.906005in}{2.611640in}}%
\pgfpathlineto{\pgfqpoint{3.906005in}{2.614590in}}%
\pgfpathlineto{\pgfqpoint{3.910546in}{2.614590in}}%
\pgfpathlineto{\pgfqpoint{3.910546in}{2.611640in}}%
\pgfpathmoveto{\pgfqpoint{3.910546in}{2.611640in}}%
\pgfpathlineto{\pgfqpoint{3.910546in}{2.611640in}}%
\pgfpathlineto{\pgfqpoint{3.910546in}{2.614590in}}%
\pgfpathlineto{\pgfqpoint{3.915087in}{2.614590in}}%
\pgfpathlineto{\pgfqpoint{3.915087in}{2.611640in}}%
\pgfpathmoveto{\pgfqpoint{3.915087in}{2.611640in}}%
\pgfpathlineto{\pgfqpoint{3.915087in}{2.611640in}}%
\pgfpathlineto{\pgfqpoint{3.915087in}{2.614590in}}%
\pgfpathlineto{\pgfqpoint{3.919628in}{2.614590in}}%
\pgfpathlineto{\pgfqpoint{3.919628in}{2.611640in}}%
\pgfpathmoveto{\pgfqpoint{3.919628in}{2.611640in}}%
\pgfpathlineto{\pgfqpoint{3.919628in}{2.611640in}}%
\pgfpathlineto{\pgfqpoint{3.919628in}{2.614590in}}%
\pgfpathlineto{\pgfqpoint{3.924169in}{2.614590in}}%
\pgfpathlineto{\pgfqpoint{3.924169in}{2.611640in}}%
\pgfpathmoveto{\pgfqpoint{3.924169in}{2.611640in}}%
\pgfpathlineto{\pgfqpoint{3.924169in}{2.611640in}}%
\pgfpathlineto{\pgfqpoint{3.924169in}{2.614590in}}%
\pgfpathlineto{\pgfqpoint{3.928710in}{2.614590in}}%
\pgfpathlineto{\pgfqpoint{3.928710in}{2.611640in}}%
\pgfpathmoveto{\pgfqpoint{3.928710in}{2.611640in}}%
\pgfpathlineto{\pgfqpoint{3.928710in}{2.611640in}}%
\pgfpathlineto{\pgfqpoint{3.928710in}{2.614590in}}%
\pgfpathlineto{\pgfqpoint{3.933251in}{2.614590in}}%
\pgfpathlineto{\pgfqpoint{3.933251in}{2.611640in}}%
\pgfpathmoveto{\pgfqpoint{3.933251in}{2.611640in}}%
\pgfpathlineto{\pgfqpoint{3.933251in}{2.611640in}}%
\pgfpathlineto{\pgfqpoint{3.933251in}{2.614590in}}%
\pgfpathlineto{\pgfqpoint{3.937792in}{2.614590in}}%
\pgfpathlineto{\pgfqpoint{3.937792in}{2.611640in}}%
\pgfpathmoveto{\pgfqpoint{3.937792in}{2.611640in}}%
\pgfpathlineto{\pgfqpoint{3.937792in}{2.611640in}}%
\pgfpathlineto{\pgfqpoint{3.937792in}{2.614590in}}%
\pgfpathlineto{\pgfqpoint{3.942334in}{2.614590in}}%
\pgfpathlineto{\pgfqpoint{3.942334in}{2.611640in}}%
\pgfpathmoveto{\pgfqpoint{3.942334in}{2.611640in}}%
\pgfpathlineto{\pgfqpoint{3.942334in}{2.611640in}}%
\pgfpathlineto{\pgfqpoint{3.942334in}{2.614590in}}%
\pgfpathlineto{\pgfqpoint{3.946875in}{2.614590in}}%
\pgfpathlineto{\pgfqpoint{3.946875in}{2.611640in}}%
\pgfpathmoveto{\pgfqpoint{3.946875in}{2.611640in}}%
\pgfpathlineto{\pgfqpoint{3.946875in}{2.611640in}}%
\pgfpathlineto{\pgfqpoint{3.946875in}{2.614590in}}%
\pgfpathlineto{\pgfqpoint{3.951415in}{2.614590in}}%
\pgfpathlineto{\pgfqpoint{3.951415in}{2.611640in}}%
\pgfpathmoveto{\pgfqpoint{3.951415in}{2.611640in}}%
\pgfpathlineto{\pgfqpoint{3.951415in}{2.611640in}}%
\pgfpathlineto{\pgfqpoint{3.951415in}{2.614590in}}%
\pgfpathlineto{\pgfqpoint{3.955956in}{2.614590in}}%
\pgfpathlineto{\pgfqpoint{3.955956in}{2.611640in}}%
\pgfpathmoveto{\pgfqpoint{3.955956in}{2.611640in}}%
\pgfpathlineto{\pgfqpoint{3.955956in}{2.611640in}}%
\pgfpathlineto{\pgfqpoint{3.955956in}{2.614590in}}%
\pgfpathlineto{\pgfqpoint{3.960497in}{2.614590in}}%
\pgfpathlineto{\pgfqpoint{3.960497in}{2.611640in}}%
\pgfpathmoveto{\pgfqpoint{3.960497in}{2.611640in}}%
\pgfpathlineto{\pgfqpoint{3.960497in}{2.611640in}}%
\pgfpathlineto{\pgfqpoint{3.960497in}{2.614590in}}%
\pgfpathlineto{\pgfqpoint{3.965038in}{2.614590in}}%
\pgfpathlineto{\pgfqpoint{3.965038in}{2.611640in}}%
\pgfpathmoveto{\pgfqpoint{3.965038in}{2.611640in}}%
\pgfpathlineto{\pgfqpoint{3.965038in}{2.611640in}}%
\pgfpathlineto{\pgfqpoint{3.965038in}{2.614590in}}%
\pgfpathlineto{\pgfqpoint{3.969579in}{2.614590in}}%
\pgfpathlineto{\pgfqpoint{3.969579in}{2.611640in}}%
\pgfpathmoveto{\pgfqpoint{3.969579in}{2.611640in}}%
\pgfpathlineto{\pgfqpoint{3.969579in}{2.611640in}}%
\pgfpathlineto{\pgfqpoint{3.969579in}{2.614590in}}%
\pgfpathlineto{\pgfqpoint{3.974120in}{2.614590in}}%
\pgfpathlineto{\pgfqpoint{3.974120in}{2.611640in}}%
\pgfpathmoveto{\pgfqpoint{3.974120in}{2.611640in}}%
\pgfpathlineto{\pgfqpoint{3.974120in}{2.611640in}}%
\pgfpathlineto{\pgfqpoint{3.974120in}{2.614590in}}%
\pgfpathlineto{\pgfqpoint{3.978661in}{2.614590in}}%
\pgfpathlineto{\pgfqpoint{3.978661in}{2.611640in}}%
\pgfpathmoveto{\pgfqpoint{3.978661in}{2.611640in}}%
\pgfpathlineto{\pgfqpoint{3.978661in}{2.611640in}}%
\pgfpathlineto{\pgfqpoint{3.978661in}{2.614590in}}%
\pgfpathlineto{\pgfqpoint{3.983202in}{2.614590in}}%
\pgfpathlineto{\pgfqpoint{3.983202in}{2.611640in}}%
\pgfpathmoveto{\pgfqpoint{3.983202in}{2.611640in}}%
\pgfpathlineto{\pgfqpoint{3.983202in}{2.611640in}}%
\pgfpathlineto{\pgfqpoint{3.983202in}{2.614590in}}%
\pgfpathlineto{\pgfqpoint{3.987743in}{2.614590in}}%
\pgfpathlineto{\pgfqpoint{3.987743in}{2.611640in}}%
\pgfpathmoveto{\pgfqpoint{3.987743in}{2.611640in}}%
\pgfpathlineto{\pgfqpoint{3.987743in}{2.611640in}}%
\pgfpathlineto{\pgfqpoint{3.987743in}{2.614590in}}%
\pgfpathlineto{\pgfqpoint{3.992284in}{2.614590in}}%
\pgfpathlineto{\pgfqpoint{3.992284in}{2.611640in}}%
\pgfpathmoveto{\pgfqpoint{3.992284in}{2.611640in}}%
\pgfpathlineto{\pgfqpoint{3.992284in}{2.611640in}}%
\pgfpathlineto{\pgfqpoint{3.992284in}{2.614590in}}%
\pgfpathlineto{\pgfqpoint{3.996825in}{2.614590in}}%
\pgfpathlineto{\pgfqpoint{3.996825in}{2.611640in}}%
\pgfpathmoveto{\pgfqpoint{3.996825in}{2.611640in}}%
\pgfpathlineto{\pgfqpoint{3.996825in}{2.611640in}}%
\pgfpathlineto{\pgfqpoint{3.996825in}{2.614590in}}%
\pgfpathlineto{\pgfqpoint{4.001366in}{2.614590in}}%
\pgfpathlineto{\pgfqpoint{4.001366in}{2.611640in}}%
\pgfpathmoveto{\pgfqpoint{4.001366in}{2.611640in}}%
\pgfpathlineto{\pgfqpoint{4.001366in}{2.611640in}}%
\pgfpathlineto{\pgfqpoint{4.001366in}{2.614590in}}%
\pgfpathlineto{\pgfqpoint{4.005906in}{2.614590in}}%
\pgfpathlineto{\pgfqpoint{4.005906in}{2.611640in}}%
\pgfpathmoveto{\pgfqpoint{4.005906in}{2.611640in}}%
\pgfpathlineto{\pgfqpoint{4.005906in}{2.611640in}}%
\pgfpathlineto{\pgfqpoint{4.005906in}{2.614590in}}%
\pgfpathlineto{\pgfqpoint{4.010447in}{2.614590in}}%
\pgfpathlineto{\pgfqpoint{4.010447in}{2.611640in}}%
\pgfpathmoveto{\pgfqpoint{4.010447in}{2.611640in}}%
\pgfpathlineto{\pgfqpoint{4.010447in}{2.611640in}}%
\pgfpathlineto{\pgfqpoint{4.010447in}{2.614590in}}%
\pgfpathlineto{\pgfqpoint{4.014988in}{2.614590in}}%
\pgfpathlineto{\pgfqpoint{4.014988in}{2.611640in}}%
\pgfpathmoveto{\pgfqpoint{4.014988in}{2.611640in}}%
\pgfpathlineto{\pgfqpoint{4.014988in}{2.611640in}}%
\pgfpathlineto{\pgfqpoint{4.014988in}{2.614590in}}%
\pgfpathlineto{\pgfqpoint{4.019529in}{2.614590in}}%
\pgfpathlineto{\pgfqpoint{4.019529in}{2.611640in}}%
\pgfpathmoveto{\pgfqpoint{4.019529in}{2.611640in}}%
\pgfpathlineto{\pgfqpoint{4.019529in}{2.611640in}}%
\pgfpathlineto{\pgfqpoint{4.019529in}{2.614590in}}%
\pgfpathlineto{\pgfqpoint{4.024070in}{2.614590in}}%
\pgfpathlineto{\pgfqpoint{4.024070in}{2.611640in}}%
\pgfpathmoveto{\pgfqpoint{4.024070in}{2.611640in}}%
\pgfpathlineto{\pgfqpoint{4.024070in}{2.611640in}}%
\pgfpathlineto{\pgfqpoint{4.024070in}{2.614590in}}%
\pgfpathlineto{\pgfqpoint{4.028611in}{2.614590in}}%
\pgfpathlineto{\pgfqpoint{4.028611in}{2.611640in}}%
\pgfpathmoveto{\pgfqpoint{4.028611in}{2.611640in}}%
\pgfpathlineto{\pgfqpoint{4.028611in}{2.611640in}}%
\pgfpathlineto{\pgfqpoint{4.028611in}{2.614590in}}%
\pgfpathlineto{\pgfqpoint{4.033152in}{2.614590in}}%
\pgfpathlineto{\pgfqpoint{4.033152in}{2.611640in}}%
\pgfpathmoveto{\pgfqpoint{4.033152in}{2.611640in}}%
\pgfpathlineto{\pgfqpoint{4.033152in}{2.611640in}}%
\pgfpathlineto{\pgfqpoint{4.033152in}{2.614590in}}%
\pgfpathlineto{\pgfqpoint{4.037693in}{2.614590in}}%
\pgfpathlineto{\pgfqpoint{4.037693in}{2.611640in}}%
\pgfpathmoveto{\pgfqpoint{4.037693in}{2.611640in}}%
\pgfpathlineto{\pgfqpoint{4.037693in}{2.611640in}}%
\pgfpathlineto{\pgfqpoint{4.037693in}{2.614590in}}%
\pgfpathlineto{\pgfqpoint{4.042234in}{2.614590in}}%
\pgfpathlineto{\pgfqpoint{4.042234in}{2.611640in}}%
\pgfpathmoveto{\pgfqpoint{4.042234in}{2.611640in}}%
\pgfpathlineto{\pgfqpoint{4.042234in}{2.611640in}}%
\pgfpathlineto{\pgfqpoint{4.042234in}{2.614590in}}%
\pgfpathlineto{\pgfqpoint{4.046775in}{2.614590in}}%
\pgfpathlineto{\pgfqpoint{4.046775in}{2.611640in}}%
\pgfpathmoveto{\pgfqpoint{4.046775in}{2.611640in}}%
\pgfpathlineto{\pgfqpoint{4.046775in}{2.611640in}}%
\pgfpathlineto{\pgfqpoint{4.046775in}{2.614590in}}%
\pgfpathlineto{\pgfqpoint{4.051316in}{2.614590in}}%
\pgfpathlineto{\pgfqpoint{4.051316in}{2.611640in}}%
\pgfpathmoveto{\pgfqpoint{4.051316in}{2.611640in}}%
\pgfpathlineto{\pgfqpoint{4.051316in}{2.611640in}}%
\pgfpathlineto{\pgfqpoint{4.051316in}{2.614590in}}%
\pgfpathlineto{\pgfqpoint{4.055856in}{2.614590in}}%
\pgfpathlineto{\pgfqpoint{4.055856in}{2.611640in}}%
\pgfpathmoveto{\pgfqpoint{4.055856in}{2.611640in}}%
\pgfpathlineto{\pgfqpoint{4.055856in}{2.611640in}}%
\pgfpathlineto{\pgfqpoint{4.055856in}{2.614590in}}%
\pgfpathlineto{\pgfqpoint{4.060397in}{2.614590in}}%
\pgfpathlineto{\pgfqpoint{4.060397in}{2.611640in}}%
\pgfpathmoveto{\pgfqpoint{4.060397in}{2.611640in}}%
\pgfpathlineto{\pgfqpoint{4.060397in}{2.611640in}}%
\pgfpathlineto{\pgfqpoint{4.060397in}{2.614590in}}%
\pgfpathlineto{\pgfqpoint{4.064938in}{2.614590in}}%
\pgfpathlineto{\pgfqpoint{4.064938in}{2.611640in}}%
\pgfpathmoveto{\pgfqpoint{4.064938in}{2.611640in}}%
\pgfpathlineto{\pgfqpoint{4.064938in}{2.611640in}}%
\pgfpathlineto{\pgfqpoint{4.064938in}{2.614590in}}%
\pgfpathlineto{\pgfqpoint{4.069479in}{2.614590in}}%
\pgfpathlineto{\pgfqpoint{4.069479in}{2.611640in}}%
\pgfpathmoveto{\pgfqpoint{4.069479in}{2.611640in}}%
\pgfpathlineto{\pgfqpoint{4.069479in}{2.611640in}}%
\pgfpathlineto{\pgfqpoint{4.069479in}{2.614590in}}%
\pgfpathlineto{\pgfqpoint{4.074020in}{2.614590in}}%
\pgfpathlineto{\pgfqpoint{4.074020in}{2.611640in}}%
\pgfpathmoveto{\pgfqpoint{4.074020in}{2.611640in}}%
\pgfpathlineto{\pgfqpoint{4.074020in}{2.611640in}}%
\pgfpathlineto{\pgfqpoint{4.074020in}{2.614590in}}%
\pgfpathlineto{\pgfqpoint{4.078561in}{2.614590in}}%
\pgfpathlineto{\pgfqpoint{4.078561in}{2.611640in}}%
\pgfpathmoveto{\pgfqpoint{4.078561in}{2.611640in}}%
\pgfpathlineto{\pgfqpoint{4.078561in}{2.611640in}}%
\pgfpathlineto{\pgfqpoint{4.078561in}{2.614590in}}%
\pgfpathlineto{\pgfqpoint{4.083102in}{2.614590in}}%
\pgfpathlineto{\pgfqpoint{4.083102in}{2.611640in}}%
\pgfpathmoveto{\pgfqpoint{4.083102in}{2.611640in}}%
\pgfpathlineto{\pgfqpoint{4.083102in}{2.611640in}}%
\pgfpathlineto{\pgfqpoint{4.083102in}{2.614590in}}%
\pgfpathlineto{\pgfqpoint{4.087643in}{2.614590in}}%
\pgfpathlineto{\pgfqpoint{4.087643in}{2.611640in}}%
\pgfpathmoveto{\pgfqpoint{4.087643in}{2.611640in}}%
\pgfpathlineto{\pgfqpoint{4.087643in}{2.611640in}}%
\pgfpathlineto{\pgfqpoint{4.087643in}{2.614590in}}%
\pgfpathlineto{\pgfqpoint{4.092184in}{2.614590in}}%
\pgfpathlineto{\pgfqpoint{4.092184in}{2.611640in}}%
\pgfpathmoveto{\pgfqpoint{4.092184in}{2.611640in}}%
\pgfpathlineto{\pgfqpoint{4.092184in}{2.611640in}}%
\pgfpathlineto{\pgfqpoint{4.092184in}{2.614590in}}%
\pgfpathlineto{\pgfqpoint{4.096725in}{2.614590in}}%
\pgfpathlineto{\pgfqpoint{4.096725in}{2.611640in}}%
\pgfpathmoveto{\pgfqpoint{4.096725in}{2.611640in}}%
\pgfpathlineto{\pgfqpoint{4.096725in}{2.611640in}}%
\pgfpathlineto{\pgfqpoint{4.096725in}{2.614590in}}%
\pgfpathlineto{\pgfqpoint{4.101266in}{2.614590in}}%
\pgfpathlineto{\pgfqpoint{4.101266in}{2.611640in}}%
\pgfpathmoveto{\pgfqpoint{4.101266in}{2.611640in}}%
\pgfpathlineto{\pgfqpoint{4.101266in}{2.611640in}}%
\pgfpathlineto{\pgfqpoint{4.101266in}{2.614590in}}%
\pgfpathlineto{\pgfqpoint{4.105808in}{2.614590in}}%
\pgfpathlineto{\pgfqpoint{4.105808in}{2.611640in}}%
\pgfpathmoveto{\pgfqpoint{4.105808in}{2.611640in}}%
\pgfpathlineto{\pgfqpoint{4.105808in}{2.611640in}}%
\pgfpathlineto{\pgfqpoint{4.105808in}{2.614590in}}%
\pgfpathlineto{\pgfqpoint{4.110349in}{2.614590in}}%
\pgfpathlineto{\pgfqpoint{4.110349in}{2.611640in}}%
\pgfpathmoveto{\pgfqpoint{4.110349in}{2.611640in}}%
\pgfpathlineto{\pgfqpoint{4.110349in}{2.611640in}}%
\pgfpathlineto{\pgfqpoint{4.110349in}{2.614590in}}%
\pgfpathlineto{\pgfqpoint{4.114890in}{2.614590in}}%
\pgfpathlineto{\pgfqpoint{4.114890in}{2.611640in}}%
\pgfpathmoveto{\pgfqpoint{4.114890in}{2.611640in}}%
\pgfpathlineto{\pgfqpoint{4.114890in}{2.611640in}}%
\pgfpathlineto{\pgfqpoint{4.114890in}{2.614590in}}%
\pgfpathlineto{\pgfqpoint{4.119431in}{2.614590in}}%
\pgfpathlineto{\pgfqpoint{4.119431in}{2.611640in}}%
\pgfpathmoveto{\pgfqpoint{4.119431in}{2.611640in}}%
\pgfpathlineto{\pgfqpoint{4.119431in}{2.611640in}}%
\pgfpathlineto{\pgfqpoint{4.119431in}{2.614590in}}%
\pgfpathlineto{\pgfqpoint{4.123972in}{2.614590in}}%
\pgfpathlineto{\pgfqpoint{4.123972in}{2.611640in}}%
\pgfpathmoveto{\pgfqpoint{4.123972in}{2.611640in}}%
\pgfpathlineto{\pgfqpoint{4.123972in}{2.611640in}}%
\pgfpathlineto{\pgfqpoint{4.123972in}{2.614590in}}%
\pgfpathlineto{\pgfqpoint{4.128514in}{2.614590in}}%
\pgfpathlineto{\pgfqpoint{4.128514in}{2.611640in}}%
\pgfpathmoveto{\pgfqpoint{4.128514in}{2.611640in}}%
\pgfpathlineto{\pgfqpoint{4.128514in}{2.611640in}}%
\pgfpathlineto{\pgfqpoint{4.128514in}{2.614590in}}%
\pgfpathlineto{\pgfqpoint{4.133055in}{2.614590in}}%
\pgfpathlineto{\pgfqpoint{4.133055in}{2.611640in}}%
\pgfpathmoveto{\pgfqpoint{4.133055in}{2.611640in}}%
\pgfpathlineto{\pgfqpoint{4.133055in}{2.611640in}}%
\pgfpathlineto{\pgfqpoint{4.133055in}{2.614590in}}%
\pgfpathlineto{\pgfqpoint{4.137596in}{2.614590in}}%
\pgfpathlineto{\pgfqpoint{4.137596in}{2.611640in}}%
\pgfpathmoveto{\pgfqpoint{4.137596in}{2.611640in}}%
\pgfpathlineto{\pgfqpoint{4.137596in}{2.611640in}}%
\pgfpathlineto{\pgfqpoint{4.137596in}{2.614590in}}%
\pgfpathlineto{\pgfqpoint{4.142137in}{2.614590in}}%
\pgfpathlineto{\pgfqpoint{4.142137in}{2.611640in}}%
\pgfpathmoveto{\pgfqpoint{4.142137in}{2.611640in}}%
\pgfpathlineto{\pgfqpoint{4.142137in}{2.611640in}}%
\pgfpathlineto{\pgfqpoint{4.142137in}{2.614590in}}%
\pgfpathlineto{\pgfqpoint{4.146679in}{2.614590in}}%
\pgfpathlineto{\pgfqpoint{4.146679in}{2.611640in}}%
\pgfpathmoveto{\pgfqpoint{4.146679in}{2.611640in}}%
\pgfpathlineto{\pgfqpoint{4.146679in}{2.611640in}}%
\pgfpathlineto{\pgfqpoint{4.146679in}{2.614590in}}%
\pgfpathlineto{\pgfqpoint{4.151220in}{2.614590in}}%
\pgfpathlineto{\pgfqpoint{4.151220in}{2.611640in}}%
\pgfpathmoveto{\pgfqpoint{4.151220in}{2.611640in}}%
\pgfpathlineto{\pgfqpoint{4.151220in}{2.611640in}}%
\pgfpathlineto{\pgfqpoint{4.151220in}{2.614590in}}%
\pgfpathlineto{\pgfqpoint{4.155761in}{2.614590in}}%
\pgfpathlineto{\pgfqpoint{4.155761in}{2.611640in}}%
\pgfpathmoveto{\pgfqpoint{4.155761in}{2.611640in}}%
\pgfpathlineto{\pgfqpoint{4.155761in}{2.611640in}}%
\pgfpathlineto{\pgfqpoint{4.155761in}{2.614590in}}%
\pgfpathlineto{\pgfqpoint{4.160302in}{2.614590in}}%
\pgfpathlineto{\pgfqpoint{4.160302in}{2.611640in}}%
\pgfpathmoveto{\pgfqpoint{4.160302in}{2.611640in}}%
\pgfpathlineto{\pgfqpoint{4.160302in}{2.611640in}}%
\pgfpathlineto{\pgfqpoint{4.160302in}{2.614590in}}%
\pgfpathlineto{\pgfqpoint{4.164844in}{2.614590in}}%
\pgfpathlineto{\pgfqpoint{4.164844in}{2.611640in}}%
\pgfpathmoveto{\pgfqpoint{4.164844in}{2.611640in}}%
\pgfpathlineto{\pgfqpoint{4.164844in}{2.611640in}}%
\pgfpathlineto{\pgfqpoint{4.164844in}{2.614590in}}%
\pgfpathlineto{\pgfqpoint{4.169385in}{2.614590in}}%
\pgfpathlineto{\pgfqpoint{4.169385in}{2.611640in}}%
\pgfpathmoveto{\pgfqpoint{4.169385in}{2.611640in}}%
\pgfpathlineto{\pgfqpoint{4.169385in}{2.611640in}}%
\pgfpathlineto{\pgfqpoint{4.169385in}{2.614590in}}%
\pgfpathlineto{\pgfqpoint{4.173926in}{2.614590in}}%
\pgfpathlineto{\pgfqpoint{4.173926in}{2.611640in}}%
\pgfpathmoveto{\pgfqpoint{4.173926in}{2.611640in}}%
\pgfpathlineto{\pgfqpoint{4.173926in}{2.611640in}}%
\pgfpathlineto{\pgfqpoint{4.173926in}{2.614590in}}%
\pgfpathlineto{\pgfqpoint{4.178467in}{2.614590in}}%
\pgfpathlineto{\pgfqpoint{4.178467in}{2.611640in}}%
\pgfpathmoveto{\pgfqpoint{4.178467in}{2.611640in}}%
\pgfpathlineto{\pgfqpoint{4.178467in}{2.611640in}}%
\pgfpathlineto{\pgfqpoint{4.178467in}{2.614590in}}%
\pgfpathlineto{\pgfqpoint{4.183009in}{2.614590in}}%
\pgfpathlineto{\pgfqpoint{4.183009in}{2.611640in}}%
\pgfpathmoveto{\pgfqpoint{4.183009in}{2.611640in}}%
\pgfpathlineto{\pgfqpoint{4.183009in}{2.611640in}}%
\pgfpathlineto{\pgfqpoint{4.183009in}{2.614590in}}%
\pgfpathlineto{\pgfqpoint{4.187550in}{2.614590in}}%
\pgfpathlineto{\pgfqpoint{4.187550in}{2.611640in}}%
\pgfpathmoveto{\pgfqpoint{4.187550in}{2.611640in}}%
\pgfpathlineto{\pgfqpoint{4.187550in}{2.611640in}}%
\pgfpathlineto{\pgfqpoint{4.187550in}{2.614590in}}%
\pgfpathlineto{\pgfqpoint{4.192091in}{2.614590in}}%
\pgfpathlineto{\pgfqpoint{4.192091in}{2.611640in}}%
\pgfpathmoveto{\pgfqpoint{4.192091in}{2.611640in}}%
\pgfpathlineto{\pgfqpoint{4.192091in}{2.611640in}}%
\pgfpathlineto{\pgfqpoint{4.192091in}{2.614590in}}%
\pgfpathlineto{\pgfqpoint{4.196632in}{2.614590in}}%
\pgfpathlineto{\pgfqpoint{4.196632in}{2.611640in}}%
\pgfpathmoveto{\pgfqpoint{4.196632in}{2.611640in}}%
\pgfpathlineto{\pgfqpoint{4.196632in}{2.611640in}}%
\pgfpathlineto{\pgfqpoint{4.196632in}{2.614590in}}%
\pgfpathlineto{\pgfqpoint{4.201174in}{2.614590in}}%
\pgfpathlineto{\pgfqpoint{4.201174in}{2.611640in}}%
\pgfpathmoveto{\pgfqpoint{4.201174in}{2.611640in}}%
\pgfpathlineto{\pgfqpoint{4.201174in}{2.611640in}}%
\pgfpathlineto{\pgfqpoint{4.201174in}{2.614590in}}%
\pgfpathlineto{\pgfqpoint{4.205715in}{2.614590in}}%
\pgfpathlineto{\pgfqpoint{4.205715in}{2.611640in}}%
\pgfpathmoveto{\pgfqpoint{4.205715in}{2.611640in}}%
\pgfpathlineto{\pgfqpoint{4.205715in}{2.611640in}}%
\pgfpathlineto{\pgfqpoint{4.205715in}{2.614590in}}%
\pgfpathlineto{\pgfqpoint{4.210256in}{2.614590in}}%
\pgfpathlineto{\pgfqpoint{4.210256in}{2.611640in}}%
\pgfpathmoveto{\pgfqpoint{4.210256in}{2.611640in}}%
\pgfpathlineto{\pgfqpoint{4.210256in}{2.611640in}}%
\pgfpathlineto{\pgfqpoint{4.210256in}{2.614590in}}%
\pgfpathlineto{\pgfqpoint{4.214797in}{2.614590in}}%
\pgfpathlineto{\pgfqpoint{4.214797in}{2.611640in}}%
\pgfpathmoveto{\pgfqpoint{4.214797in}{2.611640in}}%
\pgfpathlineto{\pgfqpoint{4.214797in}{2.611640in}}%
\pgfpathlineto{\pgfqpoint{4.214797in}{2.614590in}}%
\pgfpathlineto{\pgfqpoint{4.219338in}{2.614590in}}%
\pgfpathlineto{\pgfqpoint{4.219338in}{2.611640in}}%
\pgfpathmoveto{\pgfqpoint{4.219338in}{2.611640in}}%
\pgfpathlineto{\pgfqpoint{4.219338in}{2.611640in}}%
\pgfpathlineto{\pgfqpoint{4.219338in}{2.614590in}}%
\pgfpathlineto{\pgfqpoint{4.223880in}{2.614590in}}%
\pgfpathlineto{\pgfqpoint{4.223880in}{2.611640in}}%
\pgfpathmoveto{\pgfqpoint{4.223880in}{2.611640in}}%
\pgfpathlineto{\pgfqpoint{4.223880in}{2.611640in}}%
\pgfpathlineto{\pgfqpoint{4.223880in}{2.614590in}}%
\pgfpathlineto{\pgfqpoint{4.228421in}{2.614590in}}%
\pgfpathlineto{\pgfqpoint{4.228421in}{2.611640in}}%
\pgfpathmoveto{\pgfqpoint{4.228421in}{2.611640in}}%
\pgfpathlineto{\pgfqpoint{4.228421in}{2.611640in}}%
\pgfpathlineto{\pgfqpoint{4.228421in}{2.614590in}}%
\pgfpathlineto{\pgfqpoint{4.232962in}{2.614590in}}%
\pgfpathlineto{\pgfqpoint{4.232962in}{2.611640in}}%
\pgfpathmoveto{\pgfqpoint{4.232962in}{2.611640in}}%
\pgfpathlineto{\pgfqpoint{4.232962in}{2.611640in}}%
\pgfpathlineto{\pgfqpoint{4.232962in}{2.614590in}}%
\pgfpathlineto{\pgfqpoint{4.237503in}{2.614590in}}%
\pgfpathlineto{\pgfqpoint{4.237503in}{2.611640in}}%
\pgfpathmoveto{\pgfqpoint{4.237503in}{2.611640in}}%
\pgfpathlineto{\pgfqpoint{4.237503in}{2.611640in}}%
\pgfpathlineto{\pgfqpoint{4.237503in}{2.614590in}}%
\pgfpathlineto{\pgfqpoint{4.242044in}{2.614590in}}%
\pgfpathlineto{\pgfqpoint{4.242044in}{2.611640in}}%
\pgfpathmoveto{\pgfqpoint{4.242044in}{2.611640in}}%
\pgfpathlineto{\pgfqpoint{4.242044in}{2.611640in}}%
\pgfpathlineto{\pgfqpoint{4.242044in}{2.614590in}}%
\pgfpathlineto{\pgfqpoint{4.246585in}{2.614590in}}%
\pgfpathlineto{\pgfqpoint{4.246585in}{2.611640in}}%
\pgfpathmoveto{\pgfqpoint{4.246585in}{2.611640in}}%
\pgfpathlineto{\pgfqpoint{4.246585in}{2.611640in}}%
\pgfpathlineto{\pgfqpoint{4.246585in}{2.614590in}}%
\pgfpathlineto{\pgfqpoint{4.251126in}{2.614590in}}%
\pgfpathlineto{\pgfqpoint{4.251126in}{2.611640in}}%
\pgfpathmoveto{\pgfqpoint{4.251126in}{2.611640in}}%
\pgfpathlineto{\pgfqpoint{4.251126in}{2.611640in}}%
\pgfpathlineto{\pgfqpoint{4.251126in}{2.614590in}}%
\pgfpathlineto{\pgfqpoint{4.255667in}{2.614590in}}%
\pgfpathlineto{\pgfqpoint{4.255667in}{2.611640in}}%
\pgfpathmoveto{\pgfqpoint{4.255667in}{2.611640in}}%
\pgfpathlineto{\pgfqpoint{4.255667in}{2.611640in}}%
\pgfpathlineto{\pgfqpoint{4.255667in}{2.614590in}}%
\pgfpathlineto{\pgfqpoint{4.260208in}{2.614590in}}%
\pgfpathlineto{\pgfqpoint{4.260208in}{2.611640in}}%
\pgfpathmoveto{\pgfqpoint{4.260208in}{2.611640in}}%
\pgfpathlineto{\pgfqpoint{4.260208in}{2.611640in}}%
\pgfpathlineto{\pgfqpoint{4.260208in}{2.614590in}}%
\pgfpathlineto{\pgfqpoint{4.264749in}{2.614590in}}%
\pgfpathlineto{\pgfqpoint{4.264749in}{2.611640in}}%
\pgfpathmoveto{\pgfqpoint{4.264749in}{2.611640in}}%
\pgfpathlineto{\pgfqpoint{4.264749in}{2.611640in}}%
\pgfpathlineto{\pgfqpoint{4.264749in}{2.614590in}}%
\pgfpathlineto{\pgfqpoint{4.269290in}{2.614590in}}%
\pgfpathlineto{\pgfqpoint{4.269290in}{2.611640in}}%
\pgfpathmoveto{\pgfqpoint{4.269290in}{2.611640in}}%
\pgfpathlineto{\pgfqpoint{4.269290in}{2.611640in}}%
\pgfpathlineto{\pgfqpoint{4.269290in}{2.614590in}}%
\pgfpathlineto{\pgfqpoint{4.273831in}{2.614590in}}%
\pgfpathlineto{\pgfqpoint{4.273831in}{2.611640in}}%
\pgfpathmoveto{\pgfqpoint{4.273831in}{2.611640in}}%
\pgfpathlineto{\pgfqpoint{4.273831in}{2.611640in}}%
\pgfpathlineto{\pgfqpoint{4.273831in}{2.614590in}}%
\pgfpathlineto{\pgfqpoint{4.278372in}{2.614590in}}%
\pgfpathlineto{\pgfqpoint{4.278372in}{2.611640in}}%
\pgfpathmoveto{\pgfqpoint{4.278372in}{2.611640in}}%
\pgfpathlineto{\pgfqpoint{4.278372in}{2.611640in}}%
\pgfpathlineto{\pgfqpoint{4.278372in}{2.614590in}}%
\pgfpathlineto{\pgfqpoint{4.282912in}{2.614590in}}%
\pgfpathlineto{\pgfqpoint{4.282912in}{2.611640in}}%
\pgfpathmoveto{\pgfqpoint{4.282912in}{2.611640in}}%
\pgfpathlineto{\pgfqpoint{4.282912in}{2.611640in}}%
\pgfpathlineto{\pgfqpoint{4.282912in}{2.614590in}}%
\pgfpathlineto{\pgfqpoint{4.287453in}{2.614590in}}%
\pgfpathlineto{\pgfqpoint{4.287453in}{2.611640in}}%
\pgfpathmoveto{\pgfqpoint{4.287453in}{2.611640in}}%
\pgfpathlineto{\pgfqpoint{4.287453in}{2.611640in}}%
\pgfpathlineto{\pgfqpoint{4.287453in}{2.614590in}}%
\pgfpathlineto{\pgfqpoint{4.291994in}{2.614590in}}%
\pgfpathlineto{\pgfqpoint{4.291994in}{2.611640in}}%
\pgfpathmoveto{\pgfqpoint{4.291994in}{2.611640in}}%
\pgfpathlineto{\pgfqpoint{4.291994in}{2.611640in}}%
\pgfpathlineto{\pgfqpoint{4.291994in}{2.614590in}}%
\pgfpathlineto{\pgfqpoint{4.296535in}{2.614590in}}%
\pgfpathlineto{\pgfqpoint{4.296535in}{2.611640in}}%
\pgfpathmoveto{\pgfqpoint{4.296535in}{2.611640in}}%
\pgfpathlineto{\pgfqpoint{4.296535in}{2.611640in}}%
\pgfpathlineto{\pgfqpoint{4.296535in}{2.614590in}}%
\pgfpathlineto{\pgfqpoint{4.301076in}{2.614590in}}%
\pgfpathlineto{\pgfqpoint{4.301076in}{2.611640in}}%
\pgfpathmoveto{\pgfqpoint{4.301076in}{2.611640in}}%
\pgfpathlineto{\pgfqpoint{4.301076in}{2.611640in}}%
\pgfpathlineto{\pgfqpoint{4.301076in}{2.614590in}}%
\pgfpathlineto{\pgfqpoint{4.305617in}{2.614590in}}%
\pgfpathlineto{\pgfqpoint{4.305617in}{2.611640in}}%
\pgfpathmoveto{\pgfqpoint{4.305617in}{2.611640in}}%
\pgfpathlineto{\pgfqpoint{4.305617in}{2.611640in}}%
\pgfpathlineto{\pgfqpoint{4.305617in}{2.614590in}}%
\pgfpathlineto{\pgfqpoint{4.310158in}{2.614590in}}%
\pgfpathlineto{\pgfqpoint{4.310158in}{2.611640in}}%
\pgfpathmoveto{\pgfqpoint{4.310158in}{2.611640in}}%
\pgfpathlineto{\pgfqpoint{4.310158in}{2.611640in}}%
\pgfpathlineto{\pgfqpoint{4.310158in}{2.614590in}}%
\pgfpathlineto{\pgfqpoint{4.314699in}{2.614590in}}%
\pgfpathlineto{\pgfqpoint{4.314699in}{2.611640in}}%
\pgfpathmoveto{\pgfqpoint{4.314699in}{2.611640in}}%
\pgfpathlineto{\pgfqpoint{4.314699in}{2.611640in}}%
\pgfpathlineto{\pgfqpoint{4.314699in}{2.614590in}}%
\pgfpathlineto{\pgfqpoint{4.319240in}{2.614590in}}%
\pgfpathlineto{\pgfqpoint{4.319240in}{2.611640in}}%
\pgfpathmoveto{\pgfqpoint{4.319240in}{2.611640in}}%
\pgfpathlineto{\pgfqpoint{4.319240in}{2.611640in}}%
\pgfpathlineto{\pgfqpoint{4.319240in}{2.614590in}}%
\pgfpathlineto{\pgfqpoint{4.323781in}{2.614590in}}%
\pgfpathlineto{\pgfqpoint{4.323781in}{2.611640in}}%
\pgfpathmoveto{\pgfqpoint{4.323781in}{2.611640in}}%
\pgfpathlineto{\pgfqpoint{4.323781in}{2.611640in}}%
\pgfpathlineto{\pgfqpoint{4.323781in}{2.614590in}}%
\pgfpathlineto{\pgfqpoint{4.328321in}{2.614590in}}%
\pgfpathlineto{\pgfqpoint{4.328321in}{2.611640in}}%
\pgfpathmoveto{\pgfqpoint{4.328321in}{2.611640in}}%
\pgfpathlineto{\pgfqpoint{4.328321in}{2.611640in}}%
\pgfpathlineto{\pgfqpoint{4.328321in}{2.614590in}}%
\pgfpathlineto{\pgfqpoint{4.332862in}{2.614590in}}%
\pgfpathlineto{\pgfqpoint{4.332862in}{2.611640in}}%
\pgfpathmoveto{\pgfqpoint{4.332862in}{2.611640in}}%
\pgfpathlineto{\pgfqpoint{4.332862in}{2.611640in}}%
\pgfpathlineto{\pgfqpoint{4.332862in}{2.614590in}}%
\pgfpathlineto{\pgfqpoint{4.337403in}{2.614590in}}%
\pgfpathlineto{\pgfqpoint{4.337403in}{2.611640in}}%
\pgfpathmoveto{\pgfqpoint{4.337403in}{2.611640in}}%
\pgfpathlineto{\pgfqpoint{4.337403in}{2.611640in}}%
\pgfpathlineto{\pgfqpoint{4.337403in}{2.614590in}}%
\pgfpathlineto{\pgfqpoint{4.341944in}{2.614590in}}%
\pgfpathlineto{\pgfqpoint{4.341944in}{2.611640in}}%
\pgfpathmoveto{\pgfqpoint{4.341944in}{2.611640in}}%
\pgfpathlineto{\pgfqpoint{4.341944in}{2.611640in}}%
\pgfpathlineto{\pgfqpoint{4.341944in}{2.614590in}}%
\pgfpathlineto{\pgfqpoint{4.346485in}{2.614590in}}%
\pgfpathlineto{\pgfqpoint{4.346485in}{2.611640in}}%
\pgfpathmoveto{\pgfqpoint{4.346485in}{2.611640in}}%
\pgfpathlineto{\pgfqpoint{4.346485in}{2.611640in}}%
\pgfpathlineto{\pgfqpoint{4.346485in}{2.614590in}}%
\pgfpathlineto{\pgfqpoint{4.351026in}{2.614590in}}%
\pgfpathlineto{\pgfqpoint{4.351026in}{2.611640in}}%
\pgfpathmoveto{\pgfqpoint{4.351026in}{2.611640in}}%
\pgfpathlineto{\pgfqpoint{4.351026in}{2.611640in}}%
\pgfpathlineto{\pgfqpoint{4.351026in}{2.614590in}}%
\pgfpathlineto{\pgfqpoint{4.355567in}{2.614590in}}%
\pgfpathlineto{\pgfqpoint{4.355567in}{2.611640in}}%
\pgfpathmoveto{\pgfqpoint{4.355567in}{2.611640in}}%
\pgfpathlineto{\pgfqpoint{4.355567in}{2.611640in}}%
\pgfpathlineto{\pgfqpoint{4.355567in}{2.614590in}}%
\pgfpathlineto{\pgfqpoint{4.360108in}{2.614590in}}%
\pgfpathlineto{\pgfqpoint{4.360108in}{2.611640in}}%
\pgfpathmoveto{\pgfqpoint{4.360108in}{2.611640in}}%
\pgfpathlineto{\pgfqpoint{4.360108in}{2.611640in}}%
\pgfpathlineto{\pgfqpoint{4.360108in}{2.614590in}}%
\pgfpathlineto{\pgfqpoint{4.364649in}{2.614590in}}%
\pgfpathlineto{\pgfqpoint{4.364649in}{2.611640in}}%
\pgfpathmoveto{\pgfqpoint{4.364649in}{2.611640in}}%
\pgfpathlineto{\pgfqpoint{4.364649in}{2.611640in}}%
\pgfpathlineto{\pgfqpoint{4.364649in}{2.614590in}}%
\pgfpathlineto{\pgfqpoint{4.369190in}{2.614590in}}%
\pgfpathlineto{\pgfqpoint{4.369190in}{2.611640in}}%
\pgfpathmoveto{\pgfqpoint{4.369190in}{2.611640in}}%
\pgfpathlineto{\pgfqpoint{4.369190in}{2.611640in}}%
\pgfpathlineto{\pgfqpoint{4.369190in}{2.614590in}}%
\pgfpathlineto{\pgfqpoint{4.373730in}{2.614590in}}%
\pgfpathlineto{\pgfqpoint{4.373730in}{2.611640in}}%
\pgfpathmoveto{\pgfqpoint{4.373730in}{2.611640in}}%
\pgfpathlineto{\pgfqpoint{4.373730in}{2.611640in}}%
\pgfpathlineto{\pgfqpoint{4.373730in}{2.614590in}}%
\pgfpathlineto{\pgfqpoint{4.378271in}{2.614590in}}%
\pgfpathlineto{\pgfqpoint{4.378271in}{2.611640in}}%
\pgfpathmoveto{\pgfqpoint{4.378271in}{2.611640in}}%
\pgfpathlineto{\pgfqpoint{4.378271in}{2.611640in}}%
\pgfpathlineto{\pgfqpoint{4.378271in}{2.614590in}}%
\pgfpathlineto{\pgfqpoint{4.382812in}{2.614590in}}%
\pgfpathlineto{\pgfqpoint{4.382812in}{2.611640in}}%
\pgfpathmoveto{\pgfqpoint{4.382812in}{2.611640in}}%
\pgfpathlineto{\pgfqpoint{4.382812in}{2.611640in}}%
\pgfpathlineto{\pgfqpoint{4.382812in}{2.614590in}}%
\pgfpathlineto{\pgfqpoint{4.387353in}{2.614590in}}%
\pgfpathlineto{\pgfqpoint{4.387353in}{2.611640in}}%
\pgfpathmoveto{\pgfqpoint{4.387353in}{2.611640in}}%
\pgfpathlineto{\pgfqpoint{4.387353in}{2.611640in}}%
\pgfpathlineto{\pgfqpoint{4.387353in}{2.614590in}}%
\pgfpathlineto{\pgfqpoint{4.391894in}{2.614590in}}%
\pgfpathlineto{\pgfqpoint{4.391894in}{2.611640in}}%
\pgfpathmoveto{\pgfqpoint{4.391894in}{2.611640in}}%
\pgfpathlineto{\pgfqpoint{4.391894in}{2.611640in}}%
\pgfpathlineto{\pgfqpoint{4.391894in}{2.614590in}}%
\pgfpathlineto{\pgfqpoint{4.396435in}{2.614590in}}%
\pgfpathlineto{\pgfqpoint{4.396435in}{2.611640in}}%
\pgfpathmoveto{\pgfqpoint{4.396435in}{2.611640in}}%
\pgfpathlineto{\pgfqpoint{4.396435in}{2.611640in}}%
\pgfpathlineto{\pgfqpoint{4.396435in}{2.614590in}}%
\pgfpathlineto{\pgfqpoint{4.400976in}{2.614590in}}%
\pgfpathlineto{\pgfqpoint{4.400976in}{2.611640in}}%
\pgfpathmoveto{\pgfqpoint{4.400976in}{2.611640in}}%
\pgfpathlineto{\pgfqpoint{4.400976in}{2.611640in}}%
\pgfpathlineto{\pgfqpoint{4.400976in}{2.614590in}}%
\pgfpathlineto{\pgfqpoint{4.405517in}{2.614590in}}%
\pgfpathlineto{\pgfqpoint{4.405517in}{2.611640in}}%
\pgfpathmoveto{\pgfqpoint{4.405517in}{2.611640in}}%
\pgfpathlineto{\pgfqpoint{4.405517in}{2.611640in}}%
\pgfpathlineto{\pgfqpoint{4.405517in}{2.614590in}}%
\pgfpathlineto{\pgfqpoint{4.410058in}{2.614590in}}%
\pgfpathlineto{\pgfqpoint{4.410058in}{2.611640in}}%
\pgfpathmoveto{\pgfqpoint{4.410058in}{2.611640in}}%
\pgfpathlineto{\pgfqpoint{4.410058in}{2.611640in}}%
\pgfpathlineto{\pgfqpoint{4.410058in}{2.614590in}}%
\pgfpathlineto{\pgfqpoint{4.414599in}{2.614590in}}%
\pgfpathlineto{\pgfqpoint{4.414599in}{2.611640in}}%
\pgfpathmoveto{\pgfqpoint{4.414599in}{2.611640in}}%
\pgfpathlineto{\pgfqpoint{4.414599in}{2.611640in}}%
\pgfpathlineto{\pgfqpoint{4.414599in}{2.614590in}}%
\pgfpathlineto{\pgfqpoint{4.419140in}{2.614590in}}%
\pgfpathlineto{\pgfqpoint{4.419140in}{2.611640in}}%
\pgfpathmoveto{\pgfqpoint{4.419140in}{2.611640in}}%
\pgfpathlineto{\pgfqpoint{4.419140in}{2.611640in}}%
\pgfpathlineto{\pgfqpoint{4.419140in}{2.614590in}}%
\pgfpathlineto{\pgfqpoint{4.423680in}{2.614590in}}%
\pgfpathlineto{\pgfqpoint{4.423680in}{2.611640in}}%
\pgfpathmoveto{\pgfqpoint{4.423680in}{2.611640in}}%
\pgfpathlineto{\pgfqpoint{4.423680in}{2.611640in}}%
\pgfpathlineto{\pgfqpoint{4.423680in}{2.614590in}}%
\pgfpathlineto{\pgfqpoint{4.428221in}{2.614590in}}%
\pgfpathlineto{\pgfqpoint{4.428221in}{2.611640in}}%
\pgfpathmoveto{\pgfqpoint{4.428221in}{2.611640in}}%
\pgfpathlineto{\pgfqpoint{4.428221in}{2.611640in}}%
\pgfpathlineto{\pgfqpoint{4.428221in}{2.614590in}}%
\pgfpathlineto{\pgfqpoint{4.432762in}{2.614590in}}%
\pgfpathlineto{\pgfqpoint{4.432762in}{2.611640in}}%
\pgfpathmoveto{\pgfqpoint{4.432762in}{2.611640in}}%
\pgfpathlineto{\pgfqpoint{4.432762in}{2.611640in}}%
\pgfpathlineto{\pgfqpoint{4.432762in}{2.614590in}}%
\pgfpathlineto{\pgfqpoint{4.437303in}{2.614590in}}%
\pgfpathlineto{\pgfqpoint{4.437303in}{2.611640in}}%
\pgfpathmoveto{\pgfqpoint{4.437303in}{2.611640in}}%
\pgfpathlineto{\pgfqpoint{4.437303in}{2.611640in}}%
\pgfpathlineto{\pgfqpoint{4.437303in}{2.614590in}}%
\pgfpathlineto{\pgfqpoint{4.441844in}{2.614590in}}%
\pgfpathlineto{\pgfqpoint{4.441844in}{2.611640in}}%
\pgfpathmoveto{\pgfqpoint{4.441844in}{2.611640in}}%
\pgfpathlineto{\pgfqpoint{4.441844in}{2.611640in}}%
\pgfpathlineto{\pgfqpoint{4.441844in}{2.614590in}}%
\pgfpathlineto{\pgfqpoint{4.446385in}{2.614590in}}%
\pgfpathlineto{\pgfqpoint{4.446385in}{2.611640in}}%
\pgfpathmoveto{\pgfqpoint{4.446385in}{2.611640in}}%
\pgfpathlineto{\pgfqpoint{4.446385in}{2.611640in}}%
\pgfpathlineto{\pgfqpoint{4.446385in}{2.614590in}}%
\pgfpathlineto{\pgfqpoint{4.450926in}{2.614590in}}%
\pgfpathlineto{\pgfqpoint{4.450926in}{2.611640in}}%
\pgfpathmoveto{\pgfqpoint{4.450926in}{2.611640in}}%
\pgfpathlineto{\pgfqpoint{4.450926in}{2.611640in}}%
\pgfpathlineto{\pgfqpoint{4.450926in}{2.614590in}}%
\pgfpathlineto{\pgfqpoint{4.455467in}{2.614590in}}%
\pgfpathlineto{\pgfqpoint{4.455467in}{2.611640in}}%
\pgfpathmoveto{\pgfqpoint{4.455467in}{2.611640in}}%
\pgfpathlineto{\pgfqpoint{4.455467in}{2.611640in}}%
\pgfpathlineto{\pgfqpoint{4.455467in}{2.614590in}}%
\pgfpathlineto{\pgfqpoint{4.460008in}{2.614590in}}%
\pgfpathlineto{\pgfqpoint{4.460008in}{2.611640in}}%
\pgfpathmoveto{\pgfqpoint{4.460008in}{2.611640in}}%
\pgfpathlineto{\pgfqpoint{4.460008in}{2.611640in}}%
\pgfpathlineto{\pgfqpoint{4.460008in}{2.614590in}}%
\pgfpathlineto{\pgfqpoint{4.464549in}{2.614590in}}%
\pgfpathlineto{\pgfqpoint{4.464549in}{2.611640in}}%
\pgfpathmoveto{\pgfqpoint{4.464549in}{2.611640in}}%
\pgfpathlineto{\pgfqpoint{4.464549in}{2.611640in}}%
\pgfpathlineto{\pgfqpoint{4.464549in}{2.614590in}}%
\pgfpathlineto{\pgfqpoint{4.469090in}{2.614590in}}%
\pgfpathlineto{\pgfqpoint{4.469090in}{2.611640in}}%
\pgfpathmoveto{\pgfqpoint{4.469090in}{2.611640in}}%
\pgfpathlineto{\pgfqpoint{4.469090in}{2.611640in}}%
\pgfpathlineto{\pgfqpoint{4.469090in}{2.614590in}}%
\pgfpathlineto{\pgfqpoint{4.473631in}{2.614590in}}%
\pgfpathlineto{\pgfqpoint{4.473631in}{2.611640in}}%
\pgfpathmoveto{\pgfqpoint{4.473631in}{2.611640in}}%
\pgfpathlineto{\pgfqpoint{4.473631in}{2.611640in}}%
\pgfpathlineto{\pgfqpoint{4.473631in}{2.614590in}}%
\pgfpathlineto{\pgfqpoint{4.478171in}{2.614590in}}%
\pgfpathlineto{\pgfqpoint{4.478171in}{2.611640in}}%
\pgfpathmoveto{\pgfqpoint{4.478171in}{2.611640in}}%
\pgfpathlineto{\pgfqpoint{4.478171in}{2.611640in}}%
\pgfpathlineto{\pgfqpoint{4.478171in}{2.614590in}}%
\pgfpathlineto{\pgfqpoint{4.482712in}{2.614590in}}%
\pgfpathlineto{\pgfqpoint{4.482712in}{2.611640in}}%
\pgfpathmoveto{\pgfqpoint{4.482712in}{2.611640in}}%
\pgfpathlineto{\pgfqpoint{4.482712in}{2.611640in}}%
\pgfpathlineto{\pgfqpoint{4.482712in}{2.614590in}}%
\pgfpathlineto{\pgfqpoint{4.487253in}{2.614590in}}%
\pgfpathlineto{\pgfqpoint{4.487253in}{2.611640in}}%
\pgfpathmoveto{\pgfqpoint{4.487253in}{2.611640in}}%
\pgfpathlineto{\pgfqpoint{4.487253in}{2.611640in}}%
\pgfpathlineto{\pgfqpoint{4.487253in}{2.614590in}}%
\pgfpathlineto{\pgfqpoint{4.491794in}{2.614590in}}%
\pgfpathlineto{\pgfqpoint{4.491794in}{2.611640in}}%
\pgfpathmoveto{\pgfqpoint{4.491794in}{2.611640in}}%
\pgfpathlineto{\pgfqpoint{4.491794in}{2.611640in}}%
\pgfpathlineto{\pgfqpoint{4.491794in}{2.614590in}}%
\pgfpathlineto{\pgfqpoint{4.496335in}{2.614590in}}%
\pgfpathlineto{\pgfqpoint{4.496335in}{2.611640in}}%
\pgfpathmoveto{\pgfqpoint{4.496335in}{2.611640in}}%
\pgfpathlineto{\pgfqpoint{4.496335in}{2.611640in}}%
\pgfpathlineto{\pgfqpoint{4.496335in}{2.614590in}}%
\pgfpathlineto{\pgfqpoint{4.500876in}{2.614590in}}%
\pgfpathlineto{\pgfqpoint{4.500876in}{2.611640in}}%
\pgfpathmoveto{\pgfqpoint{4.500876in}{2.611640in}}%
\pgfpathlineto{\pgfqpoint{4.500876in}{2.611640in}}%
\pgfpathlineto{\pgfqpoint{4.500876in}{2.614590in}}%
\pgfpathlineto{\pgfqpoint{4.505417in}{2.614590in}}%
\pgfpathlineto{\pgfqpoint{4.505417in}{2.611640in}}%
\pgfpathmoveto{\pgfqpoint{4.505417in}{2.611640in}}%
\pgfpathlineto{\pgfqpoint{4.505417in}{2.611640in}}%
\pgfpathlineto{\pgfqpoint{4.505417in}{2.614590in}}%
\pgfpathlineto{\pgfqpoint{4.509958in}{2.614590in}}%
\pgfpathlineto{\pgfqpoint{4.509958in}{2.611640in}}%
\pgfpathmoveto{\pgfqpoint{4.509958in}{2.611640in}}%
\pgfpathlineto{\pgfqpoint{4.509958in}{2.611640in}}%
\pgfpathlineto{\pgfqpoint{4.509958in}{2.614590in}}%
\pgfpathlineto{\pgfqpoint{4.514499in}{2.614590in}}%
\pgfpathlineto{\pgfqpoint{4.514499in}{2.611640in}}%
\pgfpathmoveto{\pgfqpoint{4.514499in}{2.611640in}}%
\pgfpathlineto{\pgfqpoint{4.514499in}{2.611640in}}%
\pgfpathlineto{\pgfqpoint{4.514499in}{2.614590in}}%
\pgfpathlineto{\pgfqpoint{4.519040in}{2.614590in}}%
\pgfpathlineto{\pgfqpoint{4.519040in}{2.611640in}}%
\pgfpathmoveto{\pgfqpoint{4.519040in}{2.611640in}}%
\pgfpathlineto{\pgfqpoint{4.519040in}{2.611640in}}%
\pgfpathlineto{\pgfqpoint{4.519040in}{2.614590in}}%
\pgfpathlineto{\pgfqpoint{4.523581in}{2.614590in}}%
\pgfpathlineto{\pgfqpoint{4.523581in}{2.611640in}}%
\pgfpathmoveto{\pgfqpoint{4.523581in}{2.611640in}}%
\pgfpathlineto{\pgfqpoint{4.523581in}{2.611640in}}%
\pgfpathlineto{\pgfqpoint{4.523581in}{2.614590in}}%
\pgfpathlineto{\pgfqpoint{4.528122in}{2.614590in}}%
\pgfpathlineto{\pgfqpoint{4.528122in}{2.611640in}}%
\pgfpathmoveto{\pgfqpoint{4.528122in}{2.611640in}}%
\pgfpathlineto{\pgfqpoint{4.528122in}{2.611640in}}%
\pgfpathlineto{\pgfqpoint{4.528122in}{2.614590in}}%
\pgfpathlineto{\pgfqpoint{4.532663in}{2.614590in}}%
\pgfpathlineto{\pgfqpoint{4.532663in}{2.611640in}}%
\pgfpathmoveto{\pgfqpoint{4.532663in}{2.611640in}}%
\pgfpathlineto{\pgfqpoint{4.532663in}{2.611640in}}%
\pgfpathlineto{\pgfqpoint{4.532663in}{2.614590in}}%
\pgfpathlineto{\pgfqpoint{4.537204in}{2.614590in}}%
\pgfpathlineto{\pgfqpoint{4.537204in}{2.611640in}}%
\pgfpathmoveto{\pgfqpoint{4.537204in}{2.611640in}}%
\pgfpathlineto{\pgfqpoint{4.537204in}{2.611640in}}%
\pgfpathlineto{\pgfqpoint{4.537204in}{2.614590in}}%
\pgfpathlineto{\pgfqpoint{4.541745in}{2.614590in}}%
\pgfpathlineto{\pgfqpoint{4.541745in}{2.611640in}}%
\pgfpathmoveto{\pgfqpoint{4.541745in}{2.611640in}}%
\pgfpathlineto{\pgfqpoint{4.541745in}{2.611640in}}%
\pgfpathlineto{\pgfqpoint{4.541745in}{2.614590in}}%
\pgfpathlineto{\pgfqpoint{4.546286in}{2.614590in}}%
\pgfpathlineto{\pgfqpoint{4.546286in}{2.611640in}}%
\pgfpathmoveto{\pgfqpoint{4.546286in}{2.611640in}}%
\pgfpathlineto{\pgfqpoint{4.546286in}{2.611640in}}%
\pgfpathlineto{\pgfqpoint{4.546286in}{2.614590in}}%
\pgfpathlineto{\pgfqpoint{4.550827in}{2.614590in}}%
\pgfpathlineto{\pgfqpoint{4.550827in}{2.611640in}}%
\pgfpathmoveto{\pgfqpoint{4.550827in}{2.611640in}}%
\pgfpathlineto{\pgfqpoint{4.550827in}{2.611640in}}%
\pgfpathlineto{\pgfqpoint{4.550827in}{2.614590in}}%
\pgfpathlineto{\pgfqpoint{4.555368in}{2.614590in}}%
\pgfpathlineto{\pgfqpoint{4.555368in}{2.611640in}}%
\pgfpathmoveto{\pgfqpoint{4.555368in}{2.611640in}}%
\pgfpathlineto{\pgfqpoint{4.555368in}{2.611640in}}%
\pgfpathlineto{\pgfqpoint{4.555368in}{2.614590in}}%
\pgfpathlineto{\pgfqpoint{4.559909in}{2.614590in}}%
\pgfpathlineto{\pgfqpoint{4.559909in}{2.611640in}}%
\pgfpathmoveto{\pgfqpoint{4.559909in}{2.611640in}}%
\pgfpathlineto{\pgfqpoint{4.559909in}{2.611640in}}%
\pgfpathlineto{\pgfqpoint{4.559909in}{2.614590in}}%
\pgfpathlineto{\pgfqpoint{4.564449in}{2.614590in}}%
\pgfpathlineto{\pgfqpoint{4.564449in}{2.611640in}}%
\pgfpathmoveto{\pgfqpoint{4.564449in}{2.611640in}}%
\pgfpathlineto{\pgfqpoint{4.564449in}{2.611640in}}%
\pgfpathlineto{\pgfqpoint{4.564449in}{2.614590in}}%
\pgfpathlineto{\pgfqpoint{4.568990in}{2.614590in}}%
\pgfpathlineto{\pgfqpoint{4.568990in}{2.611640in}}%
\pgfpathmoveto{\pgfqpoint{4.568990in}{2.611640in}}%
\pgfpathlineto{\pgfqpoint{4.568990in}{2.611640in}}%
\pgfpathlineto{\pgfqpoint{4.568990in}{2.614590in}}%
\pgfpathlineto{\pgfqpoint{4.573531in}{2.614590in}}%
\pgfpathlineto{\pgfqpoint{4.573531in}{2.611640in}}%
\pgfpathmoveto{\pgfqpoint{4.573531in}{2.611640in}}%
\pgfpathlineto{\pgfqpoint{4.573531in}{2.611640in}}%
\pgfpathlineto{\pgfqpoint{4.573531in}{2.614590in}}%
\pgfpathlineto{\pgfqpoint{4.578072in}{2.614590in}}%
\pgfpathlineto{\pgfqpoint{4.578072in}{2.611640in}}%
\pgfpathmoveto{\pgfqpoint{4.578072in}{2.611640in}}%
\pgfpathlineto{\pgfqpoint{4.578072in}{2.611640in}}%
\pgfpathlineto{\pgfqpoint{4.578072in}{2.614590in}}%
\pgfpathlineto{\pgfqpoint{4.582613in}{2.614590in}}%
\pgfpathlineto{\pgfqpoint{4.582613in}{2.611640in}}%
\pgfpathmoveto{\pgfqpoint{4.582613in}{2.611640in}}%
\pgfpathlineto{\pgfqpoint{4.582613in}{2.611640in}}%
\pgfpathlineto{\pgfqpoint{4.582613in}{2.614590in}}%
\pgfpathlineto{\pgfqpoint{4.587154in}{2.614590in}}%
\pgfpathlineto{\pgfqpoint{4.587154in}{2.611640in}}%
\pgfpathmoveto{\pgfqpoint{4.587154in}{2.611640in}}%
\pgfpathlineto{\pgfqpoint{4.587154in}{2.611640in}}%
\pgfpathlineto{\pgfqpoint{4.587154in}{2.614590in}}%
\pgfpathlineto{\pgfqpoint{4.591695in}{2.614590in}}%
\pgfpathlineto{\pgfqpoint{4.591695in}{2.611640in}}%
\pgfpathmoveto{\pgfqpoint{4.591695in}{2.611640in}}%
\pgfpathlineto{\pgfqpoint{4.591695in}{2.611640in}}%
\pgfpathlineto{\pgfqpoint{4.591695in}{2.614590in}}%
\pgfpathlineto{\pgfqpoint{4.596236in}{2.614590in}}%
\pgfpathlineto{\pgfqpoint{4.596236in}{2.611640in}}%
\pgfpathmoveto{\pgfqpoint{4.596236in}{2.611640in}}%
\pgfpathlineto{\pgfqpoint{4.596236in}{2.611640in}}%
\pgfpathlineto{\pgfqpoint{4.596236in}{2.614590in}}%
\pgfpathlineto{\pgfqpoint{4.600777in}{2.614590in}}%
\pgfpathlineto{\pgfqpoint{4.600777in}{2.611640in}}%
\pgfpathmoveto{\pgfqpoint{4.600777in}{2.611640in}}%
\pgfpathlineto{\pgfqpoint{4.600777in}{2.611640in}}%
\pgfpathlineto{\pgfqpoint{4.600777in}{2.614590in}}%
\pgfpathlineto{\pgfqpoint{4.605318in}{2.614590in}}%
\pgfpathlineto{\pgfqpoint{4.605318in}{2.611640in}}%
\pgfpathmoveto{\pgfqpoint{4.605318in}{2.611640in}}%
\pgfpathlineto{\pgfqpoint{4.605318in}{2.611640in}}%
\pgfpathlineto{\pgfqpoint{4.605318in}{2.614590in}}%
\pgfpathlineto{\pgfqpoint{4.609859in}{2.614590in}}%
\pgfpathlineto{\pgfqpoint{4.609859in}{2.611640in}}%
\pgfpathmoveto{\pgfqpoint{4.609859in}{2.611640in}}%
\pgfpathlineto{\pgfqpoint{4.609859in}{2.611640in}}%
\pgfpathlineto{\pgfqpoint{4.609859in}{2.614590in}}%
\pgfpathlineto{\pgfqpoint{4.614400in}{2.614590in}}%
\pgfpathlineto{\pgfqpoint{4.614400in}{2.611640in}}%
\pgfpathmoveto{\pgfqpoint{4.614400in}{2.611640in}}%
\pgfpathlineto{\pgfqpoint{4.614400in}{2.611640in}}%
\pgfpathlineto{\pgfqpoint{4.614400in}{2.614590in}}%
\pgfpathlineto{\pgfqpoint{4.618941in}{2.614590in}}%
\pgfpathlineto{\pgfqpoint{4.618941in}{2.611640in}}%
\pgfpathmoveto{\pgfqpoint{4.618941in}{2.611640in}}%
\pgfpathlineto{\pgfqpoint{4.618941in}{2.611640in}}%
\pgfpathlineto{\pgfqpoint{4.618941in}{2.614590in}}%
\pgfpathlineto{\pgfqpoint{4.623482in}{2.614590in}}%
\pgfpathlineto{\pgfqpoint{4.623482in}{2.611640in}}%
\pgfpathmoveto{\pgfqpoint{4.623482in}{2.611640in}}%
\pgfpathlineto{\pgfqpoint{4.623482in}{2.611640in}}%
\pgfpathlineto{\pgfqpoint{4.623482in}{2.614590in}}%
\pgfpathlineto{\pgfqpoint{4.628023in}{2.614590in}}%
\pgfpathlineto{\pgfqpoint{4.628023in}{2.611640in}}%
\pgfpathmoveto{\pgfqpoint{4.628023in}{2.611640in}}%
\pgfpathlineto{\pgfqpoint{4.628023in}{2.611640in}}%
\pgfpathlineto{\pgfqpoint{4.628023in}{2.614590in}}%
\pgfpathlineto{\pgfqpoint{4.632564in}{2.614590in}}%
\pgfpathlineto{\pgfqpoint{4.632564in}{2.611640in}}%
\pgfpathmoveto{\pgfqpoint{4.632564in}{2.611640in}}%
\pgfpathlineto{\pgfqpoint{4.632564in}{2.611640in}}%
\pgfpathlineto{\pgfqpoint{4.632564in}{2.614590in}}%
\pgfpathlineto{\pgfqpoint{4.637105in}{2.614590in}}%
\pgfpathlineto{\pgfqpoint{4.637105in}{2.611640in}}%
\pgfpathmoveto{\pgfqpoint{4.637105in}{2.611640in}}%
\pgfpathlineto{\pgfqpoint{4.637105in}{2.611640in}}%
\pgfpathlineto{\pgfqpoint{4.637105in}{2.614590in}}%
\pgfpathlineto{\pgfqpoint{4.641646in}{2.614590in}}%
\pgfpathlineto{\pgfqpoint{4.641646in}{2.611640in}}%
\pgfpathmoveto{\pgfqpoint{4.641646in}{2.611640in}}%
\pgfpathlineto{\pgfqpoint{4.641646in}{2.611640in}}%
\pgfpathlineto{\pgfqpoint{4.641646in}{2.614590in}}%
\pgfpathlineto{\pgfqpoint{4.646187in}{2.614590in}}%
\pgfpathlineto{\pgfqpoint{4.646187in}{2.611640in}}%
\pgfpathmoveto{\pgfqpoint{4.646187in}{2.611640in}}%
\pgfpathlineto{\pgfqpoint{4.646187in}{2.611640in}}%
\pgfpathlineto{\pgfqpoint{4.646187in}{2.614590in}}%
\pgfpathlineto{\pgfqpoint{4.650728in}{2.614590in}}%
\pgfpathlineto{\pgfqpoint{4.650728in}{2.611640in}}%
\pgfpathmoveto{\pgfqpoint{4.650728in}{2.611640in}}%
\pgfpathlineto{\pgfqpoint{4.650728in}{2.611640in}}%
\pgfpathlineto{\pgfqpoint{4.650728in}{2.614590in}}%
\pgfpathlineto{\pgfqpoint{4.655269in}{2.614590in}}%
\pgfpathlineto{\pgfqpoint{4.655269in}{2.611640in}}%
\pgfpathmoveto{\pgfqpoint{4.655269in}{2.611640in}}%
\pgfpathlineto{\pgfqpoint{4.655269in}{2.611640in}}%
\pgfpathlineto{\pgfqpoint{4.655269in}{2.614590in}}%
\pgfpathlineto{\pgfqpoint{4.659810in}{2.614590in}}%
\pgfpathlineto{\pgfqpoint{4.659810in}{2.611640in}}%
\pgfpathmoveto{\pgfqpoint{4.659810in}{2.611640in}}%
\pgfpathlineto{\pgfqpoint{4.659810in}{2.611640in}}%
\pgfpathlineto{\pgfqpoint{4.659810in}{2.614590in}}%
\pgfpathlineto{\pgfqpoint{4.664351in}{2.614590in}}%
\pgfpathlineto{\pgfqpoint{4.664351in}{2.611640in}}%
\pgfpathmoveto{\pgfqpoint{4.664351in}{2.611640in}}%
\pgfpathlineto{\pgfqpoint{4.664351in}{2.611640in}}%
\pgfpathlineto{\pgfqpoint{4.664351in}{2.614590in}}%
\pgfpathlineto{\pgfqpoint{4.668892in}{2.614590in}}%
\pgfpathlineto{\pgfqpoint{4.668892in}{2.611640in}}%
\pgfpathmoveto{\pgfqpoint{4.668892in}{2.611640in}}%
\pgfpathlineto{\pgfqpoint{4.668892in}{2.611640in}}%
\pgfpathlineto{\pgfqpoint{4.668892in}{2.614590in}}%
\pgfpathlineto{\pgfqpoint{4.673433in}{2.614590in}}%
\pgfpathlineto{\pgfqpoint{4.673433in}{2.611640in}}%
\pgfpathmoveto{\pgfqpoint{4.673433in}{2.611640in}}%
\pgfpathlineto{\pgfqpoint{4.673433in}{2.611640in}}%
\pgfpathlineto{\pgfqpoint{4.673433in}{2.614590in}}%
\pgfpathlineto{\pgfqpoint{4.677974in}{2.614590in}}%
\pgfpathlineto{\pgfqpoint{4.677974in}{2.611640in}}%
\pgfpathmoveto{\pgfqpoint{4.677974in}{2.611640in}}%
\pgfpathlineto{\pgfqpoint{4.677974in}{2.611640in}}%
\pgfpathlineto{\pgfqpoint{4.677974in}{2.614590in}}%
\pgfpathlineto{\pgfqpoint{4.682516in}{2.614590in}}%
\pgfpathlineto{\pgfqpoint{4.682516in}{2.611640in}}%
\pgfpathmoveto{\pgfqpoint{4.682516in}{2.611640in}}%
\pgfpathlineto{\pgfqpoint{4.682516in}{2.611640in}}%
\pgfpathlineto{\pgfqpoint{4.682516in}{2.614590in}}%
\pgfpathlineto{\pgfqpoint{4.687057in}{2.614590in}}%
\pgfpathlineto{\pgfqpoint{4.687057in}{2.611640in}}%
\pgfpathmoveto{\pgfqpoint{4.687057in}{2.611640in}}%
\pgfpathlineto{\pgfqpoint{4.687057in}{2.611640in}}%
\pgfpathlineto{\pgfqpoint{4.687057in}{2.614590in}}%
\pgfpathlineto{\pgfqpoint{4.691598in}{2.614590in}}%
\pgfpathlineto{\pgfqpoint{4.691598in}{2.611640in}}%
\pgfpathmoveto{\pgfqpoint{4.691598in}{2.611640in}}%
\pgfpathlineto{\pgfqpoint{4.691598in}{2.611640in}}%
\pgfpathlineto{\pgfqpoint{4.691598in}{2.614590in}}%
\pgfpathlineto{\pgfqpoint{4.696140in}{2.614590in}}%
\pgfpathlineto{\pgfqpoint{4.696140in}{2.611640in}}%
\pgfpathmoveto{\pgfqpoint{4.696140in}{2.611640in}}%
\pgfpathlineto{\pgfqpoint{4.696140in}{2.611640in}}%
\pgfpathlineto{\pgfqpoint{4.696140in}{2.614590in}}%
\pgfpathlineto{\pgfqpoint{4.700681in}{2.614590in}}%
\pgfpathlineto{\pgfqpoint{4.700681in}{2.611640in}}%
\pgfpathmoveto{\pgfqpoint{4.700681in}{2.611640in}}%
\pgfpathlineto{\pgfqpoint{4.700681in}{2.611640in}}%
\pgfpathlineto{\pgfqpoint{4.700681in}{2.614590in}}%
\pgfpathlineto{\pgfqpoint{4.705222in}{2.614590in}}%
\pgfpathlineto{\pgfqpoint{4.705222in}{2.611640in}}%
\pgfpathmoveto{\pgfqpoint{4.705222in}{2.611640in}}%
\pgfpathlineto{\pgfqpoint{4.705222in}{2.611640in}}%
\pgfpathlineto{\pgfqpoint{4.705222in}{2.614590in}}%
\pgfpathlineto{\pgfqpoint{4.709763in}{2.614590in}}%
\pgfpathlineto{\pgfqpoint{4.709763in}{2.611640in}}%
\pgfpathmoveto{\pgfqpoint{4.709763in}{2.611640in}}%
\pgfpathlineto{\pgfqpoint{4.709763in}{2.611640in}}%
\pgfpathlineto{\pgfqpoint{4.709763in}{2.614590in}}%
\pgfpathlineto{\pgfqpoint{4.714305in}{2.614590in}}%
\pgfpathlineto{\pgfqpoint{4.714305in}{2.611640in}}%
\pgfpathmoveto{\pgfqpoint{4.714305in}{2.611640in}}%
\pgfpathlineto{\pgfqpoint{4.714305in}{2.611640in}}%
\pgfpathlineto{\pgfqpoint{4.714305in}{2.614590in}}%
\pgfpathlineto{\pgfqpoint{4.718846in}{2.614590in}}%
\pgfpathlineto{\pgfqpoint{4.718846in}{2.611640in}}%
\pgfpathmoveto{\pgfqpoint{4.718846in}{2.611640in}}%
\pgfpathlineto{\pgfqpoint{4.718846in}{2.611640in}}%
\pgfpathlineto{\pgfqpoint{4.718846in}{2.614590in}}%
\pgfpathlineto{\pgfqpoint{4.723387in}{2.614590in}}%
\pgfpathlineto{\pgfqpoint{4.723387in}{2.611640in}}%
\pgfpathmoveto{\pgfqpoint{4.723387in}{2.611640in}}%
\pgfpathlineto{\pgfqpoint{4.723387in}{2.611640in}}%
\pgfpathlineto{\pgfqpoint{4.723387in}{2.614590in}}%
\pgfpathlineto{\pgfqpoint{4.727928in}{2.614590in}}%
\pgfpathlineto{\pgfqpoint{4.727928in}{2.611640in}}%
\pgfpathmoveto{\pgfqpoint{4.727928in}{2.611640in}}%
\pgfpathlineto{\pgfqpoint{4.727928in}{2.611640in}}%
\pgfpathlineto{\pgfqpoint{4.727928in}{2.614590in}}%
\pgfpathlineto{\pgfqpoint{4.732470in}{2.614590in}}%
\pgfpathlineto{\pgfqpoint{4.732470in}{2.611640in}}%
\pgfpathmoveto{\pgfqpoint{4.732470in}{2.611640in}}%
\pgfpathlineto{\pgfqpoint{4.732470in}{2.611640in}}%
\pgfpathlineto{\pgfqpoint{4.732470in}{2.614590in}}%
\pgfpathlineto{\pgfqpoint{4.737011in}{2.614590in}}%
\pgfpathlineto{\pgfqpoint{4.737011in}{2.611640in}}%
\pgfpathmoveto{\pgfqpoint{4.737011in}{2.611640in}}%
\pgfpathlineto{\pgfqpoint{4.737011in}{2.611640in}}%
\pgfpathlineto{\pgfqpoint{4.737011in}{2.614590in}}%
\pgfpathlineto{\pgfqpoint{4.741552in}{2.614590in}}%
\pgfpathlineto{\pgfqpoint{4.741552in}{2.611640in}}%
\pgfpathmoveto{\pgfqpoint{4.741552in}{2.611640in}}%
\pgfpathlineto{\pgfqpoint{4.741552in}{2.611640in}}%
\pgfpathlineto{\pgfqpoint{4.741552in}{2.614590in}}%
\pgfpathlineto{\pgfqpoint{4.746093in}{2.614590in}}%
\pgfpathlineto{\pgfqpoint{4.746093in}{2.611640in}}%
\pgfpathmoveto{\pgfqpoint{4.746093in}{2.611640in}}%
\pgfpathlineto{\pgfqpoint{4.746093in}{2.611640in}}%
\pgfpathlineto{\pgfqpoint{4.746093in}{2.614590in}}%
\pgfpathlineto{\pgfqpoint{4.750635in}{2.614590in}}%
\pgfpathlineto{\pgfqpoint{4.750635in}{2.611640in}}%
\pgfpathmoveto{\pgfqpoint{4.750635in}{2.611640in}}%
\pgfpathlineto{\pgfqpoint{4.750635in}{2.611640in}}%
\pgfpathlineto{\pgfqpoint{4.750635in}{2.614590in}}%
\pgfpathlineto{\pgfqpoint{4.755176in}{2.614590in}}%
\pgfpathlineto{\pgfqpoint{4.755176in}{2.611640in}}%
\pgfpathmoveto{\pgfqpoint{4.755176in}{2.611640in}}%
\pgfpathlineto{\pgfqpoint{4.755176in}{2.611640in}}%
\pgfpathlineto{\pgfqpoint{4.755176in}{2.614590in}}%
\pgfpathlineto{\pgfqpoint{4.759717in}{2.614590in}}%
\pgfpathlineto{\pgfqpoint{4.759717in}{2.611640in}}%
\pgfpathmoveto{\pgfqpoint{4.759717in}{2.611640in}}%
\pgfpathlineto{\pgfqpoint{4.759717in}{2.611640in}}%
\pgfpathlineto{\pgfqpoint{4.759717in}{2.614590in}}%
\pgfpathlineto{\pgfqpoint{4.764259in}{2.614590in}}%
\pgfpathlineto{\pgfqpoint{4.764259in}{2.611640in}}%
\pgfpathmoveto{\pgfqpoint{4.764259in}{2.611640in}}%
\pgfpathlineto{\pgfqpoint{4.764259in}{2.611640in}}%
\pgfpathlineto{\pgfqpoint{4.764259in}{2.614590in}}%
\pgfpathlineto{\pgfqpoint{4.768800in}{2.614590in}}%
\pgfpathlineto{\pgfqpoint{4.768800in}{2.611640in}}%
\pgfpathmoveto{\pgfqpoint{4.768800in}{2.611640in}}%
\pgfpathlineto{\pgfqpoint{4.768800in}{2.611640in}}%
\pgfpathlineto{\pgfqpoint{4.768800in}{2.614590in}}%
\pgfpathlineto{\pgfqpoint{4.773341in}{2.614590in}}%
\pgfpathlineto{\pgfqpoint{4.773341in}{2.611640in}}%
\pgfpathmoveto{\pgfqpoint{4.773341in}{2.611640in}}%
\pgfpathlineto{\pgfqpoint{4.773341in}{2.611640in}}%
\pgfpathlineto{\pgfqpoint{4.773341in}{2.614590in}}%
\pgfpathlineto{\pgfqpoint{4.777882in}{2.614590in}}%
\pgfpathlineto{\pgfqpoint{4.777882in}{2.611640in}}%
\pgfpathmoveto{\pgfqpoint{4.777882in}{2.611640in}}%
\pgfpathlineto{\pgfqpoint{4.777882in}{2.611640in}}%
\pgfpathlineto{\pgfqpoint{4.777882in}{2.614590in}}%
\pgfpathlineto{\pgfqpoint{4.782424in}{2.614590in}}%
\pgfpathlineto{\pgfqpoint{4.782424in}{2.611640in}}%
\pgfpathmoveto{\pgfqpoint{4.782424in}{2.611640in}}%
\pgfpathlineto{\pgfqpoint{4.782424in}{2.611640in}}%
\pgfpathlineto{\pgfqpoint{4.782424in}{2.614590in}}%
\pgfpathlineto{\pgfqpoint{4.786965in}{2.614590in}}%
\pgfpathlineto{\pgfqpoint{4.786965in}{2.611640in}}%
\pgfpathmoveto{\pgfqpoint{4.786965in}{2.611640in}}%
\pgfpathlineto{\pgfqpoint{4.786965in}{2.611640in}}%
\pgfpathlineto{\pgfqpoint{4.786965in}{2.614590in}}%
\pgfpathlineto{\pgfqpoint{4.791506in}{2.614590in}}%
\pgfpathlineto{\pgfqpoint{4.791506in}{2.611640in}}%
\pgfpathmoveto{\pgfqpoint{4.791506in}{2.611640in}}%
\pgfpathlineto{\pgfqpoint{4.791506in}{2.611640in}}%
\pgfpathlineto{\pgfqpoint{4.791506in}{2.614590in}}%
\pgfpathlineto{\pgfqpoint{4.796047in}{2.614590in}}%
\pgfpathlineto{\pgfqpoint{4.796047in}{2.611640in}}%
\pgfpathmoveto{\pgfqpoint{4.796047in}{2.611640in}}%
\pgfpathlineto{\pgfqpoint{4.796047in}{2.611640in}}%
\pgfpathlineto{\pgfqpoint{4.796047in}{2.614590in}}%
\pgfpathlineto{\pgfqpoint{4.800589in}{2.614590in}}%
\pgfpathlineto{\pgfqpoint{4.800589in}{2.611640in}}%
\pgfpathmoveto{\pgfqpoint{4.800589in}{2.611640in}}%
\pgfpathlineto{\pgfqpoint{4.800589in}{2.611640in}}%
\pgfpathlineto{\pgfqpoint{4.800589in}{2.614590in}}%
\pgfpathlineto{\pgfqpoint{4.805130in}{2.614590in}}%
\pgfpathlineto{\pgfqpoint{4.805130in}{2.611640in}}%
\pgfpathmoveto{\pgfqpoint{4.805130in}{2.611640in}}%
\pgfpathlineto{\pgfqpoint{4.805130in}{2.611640in}}%
\pgfpathlineto{\pgfqpoint{4.805130in}{2.614590in}}%
\pgfpathlineto{\pgfqpoint{4.809671in}{2.614590in}}%
\pgfpathlineto{\pgfqpoint{4.809671in}{2.611640in}}%
\pgfpathmoveto{\pgfqpoint{4.809671in}{2.611640in}}%
\pgfpathlineto{\pgfqpoint{4.809671in}{2.611640in}}%
\pgfpathlineto{\pgfqpoint{4.809671in}{2.614590in}}%
\pgfpathlineto{\pgfqpoint{4.814212in}{2.614590in}}%
\pgfpathlineto{\pgfqpoint{4.814212in}{2.611640in}}%
\pgfpathmoveto{\pgfqpoint{4.814212in}{2.611640in}}%
\pgfpathlineto{\pgfqpoint{4.814212in}{2.611640in}}%
\pgfpathlineto{\pgfqpoint{4.814212in}{2.614590in}}%
\pgfpathlineto{\pgfqpoint{4.818754in}{2.614590in}}%
\pgfpathlineto{\pgfqpoint{4.818754in}{2.611640in}}%
\pgfpathmoveto{\pgfqpoint{4.818754in}{2.611640in}}%
\pgfpathlineto{\pgfqpoint{4.818754in}{2.611640in}}%
\pgfpathlineto{\pgfqpoint{4.818754in}{2.614590in}}%
\pgfpathlineto{\pgfqpoint{4.823295in}{2.614590in}}%
\pgfpathlineto{\pgfqpoint{4.823295in}{2.611640in}}%
\pgfpathmoveto{\pgfqpoint{4.823295in}{2.611640in}}%
\pgfpathlineto{\pgfqpoint{4.823295in}{2.611640in}}%
\pgfpathlineto{\pgfqpoint{4.823295in}{2.614590in}}%
\pgfpathlineto{\pgfqpoint{4.827835in}{2.614590in}}%
\pgfpathlineto{\pgfqpoint{4.827835in}{2.611640in}}%
\pgfpathmoveto{\pgfqpoint{4.827835in}{2.611640in}}%
\pgfpathlineto{\pgfqpoint{4.827835in}{2.611640in}}%
\pgfpathlineto{\pgfqpoint{4.827835in}{2.614590in}}%
\pgfpathlineto{\pgfqpoint{4.832376in}{2.614590in}}%
\pgfpathlineto{\pgfqpoint{4.832376in}{2.611640in}}%
\pgfpathmoveto{\pgfqpoint{4.832376in}{2.611640in}}%
\pgfpathlineto{\pgfqpoint{4.832376in}{2.611640in}}%
\pgfpathlineto{\pgfqpoint{4.832376in}{2.614590in}}%
\pgfpathlineto{\pgfqpoint{4.836917in}{2.614590in}}%
\pgfpathlineto{\pgfqpoint{4.836917in}{2.611640in}}%
\pgfpathmoveto{\pgfqpoint{4.836917in}{2.611640in}}%
\pgfpathlineto{\pgfqpoint{4.836917in}{2.611640in}}%
\pgfpathlineto{\pgfqpoint{4.836917in}{2.614590in}}%
\pgfpathlineto{\pgfqpoint{4.841458in}{2.614590in}}%
\pgfpathlineto{\pgfqpoint{4.841458in}{2.611640in}}%
\pgfpathmoveto{\pgfqpoint{4.841458in}{2.611640in}}%
\pgfpathlineto{\pgfqpoint{4.841458in}{2.611640in}}%
\pgfpathlineto{\pgfqpoint{4.841458in}{2.614590in}}%
\pgfpathlineto{\pgfqpoint{4.845999in}{2.614590in}}%
\pgfpathlineto{\pgfqpoint{4.845999in}{2.611640in}}%
\pgfpathmoveto{\pgfqpoint{4.845999in}{2.611640in}}%
\pgfpathlineto{\pgfqpoint{4.845999in}{2.611640in}}%
\pgfpathlineto{\pgfqpoint{4.845999in}{2.614590in}}%
\pgfpathlineto{\pgfqpoint{4.850539in}{2.614590in}}%
\pgfpathlineto{\pgfqpoint{4.850539in}{2.611640in}}%
\pgfpathmoveto{\pgfqpoint{4.850539in}{2.611640in}}%
\pgfpathlineto{\pgfqpoint{4.850539in}{2.611640in}}%
\pgfpathlineto{\pgfqpoint{4.850539in}{2.614590in}}%
\pgfpathlineto{\pgfqpoint{4.855080in}{2.614590in}}%
\pgfpathlineto{\pgfqpoint{4.855080in}{2.611640in}}%
\pgfpathmoveto{\pgfqpoint{4.855080in}{2.611640in}}%
\pgfpathlineto{\pgfqpoint{4.855080in}{2.611640in}}%
\pgfpathlineto{\pgfqpoint{4.855080in}{2.614590in}}%
\pgfpathlineto{\pgfqpoint{4.859621in}{2.614590in}}%
\pgfpathlineto{\pgfqpoint{4.859621in}{2.611640in}}%
\pgfpathmoveto{\pgfqpoint{4.859621in}{2.611640in}}%
\pgfpathlineto{\pgfqpoint{4.859621in}{2.611640in}}%
\pgfpathlineto{\pgfqpoint{4.859621in}{2.614590in}}%
\pgfpathlineto{\pgfqpoint{4.864162in}{2.614590in}}%
\pgfpathlineto{\pgfqpoint{4.864162in}{2.611640in}}%
\pgfpathmoveto{\pgfqpoint{4.864162in}{2.611640in}}%
\pgfpathlineto{\pgfqpoint{4.864162in}{2.611640in}}%
\pgfpathlineto{\pgfqpoint{4.864162in}{2.614590in}}%
\pgfpathlineto{\pgfqpoint{4.868703in}{2.614590in}}%
\pgfpathlineto{\pgfqpoint{4.868703in}{2.611640in}}%
\pgfpathmoveto{\pgfqpoint{4.868703in}{2.611640in}}%
\pgfpathlineto{\pgfqpoint{4.868703in}{2.611640in}}%
\pgfpathlineto{\pgfqpoint{4.868703in}{2.614590in}}%
\pgfpathlineto{\pgfqpoint{4.873244in}{2.614590in}}%
\pgfpathlineto{\pgfqpoint{4.873244in}{2.611640in}}%
\pgfpathmoveto{\pgfqpoint{4.873244in}{2.611640in}}%
\pgfpathlineto{\pgfqpoint{4.873244in}{2.611640in}}%
\pgfpathlineto{\pgfqpoint{4.873244in}{2.614590in}}%
\pgfpathlineto{\pgfqpoint{4.877784in}{2.614590in}}%
\pgfpathlineto{\pgfqpoint{4.877784in}{2.611640in}}%
\pgfpathmoveto{\pgfqpoint{4.877784in}{2.611640in}}%
\pgfpathlineto{\pgfqpoint{4.877784in}{2.611640in}}%
\pgfpathlineto{\pgfqpoint{4.877784in}{2.614590in}}%
\pgfpathlineto{\pgfqpoint{4.882325in}{2.614590in}}%
\pgfpathlineto{\pgfqpoint{4.882325in}{2.611640in}}%
\pgfpathmoveto{\pgfqpoint{4.882325in}{2.611640in}}%
\pgfpathlineto{\pgfqpoint{4.882325in}{2.611640in}}%
\pgfpathlineto{\pgfqpoint{4.882325in}{2.614590in}}%
\pgfpathlineto{\pgfqpoint{4.886866in}{2.614590in}}%
\pgfpathlineto{\pgfqpoint{4.886866in}{2.611640in}}%
\pgfpathmoveto{\pgfqpoint{4.886866in}{2.611640in}}%
\pgfpathlineto{\pgfqpoint{4.886866in}{2.611640in}}%
\pgfpathlineto{\pgfqpoint{4.886866in}{2.614590in}}%
\pgfpathlineto{\pgfqpoint{4.891407in}{2.614590in}}%
\pgfpathlineto{\pgfqpoint{4.891407in}{2.611640in}}%
\pgfpathmoveto{\pgfqpoint{4.891407in}{2.611640in}}%
\pgfpathlineto{\pgfqpoint{4.891407in}{2.611640in}}%
\pgfpathlineto{\pgfqpoint{4.891407in}{2.614590in}}%
\pgfpathlineto{\pgfqpoint{4.895948in}{2.614590in}}%
\pgfpathlineto{\pgfqpoint{4.895948in}{2.611640in}}%
\pgfpathmoveto{\pgfqpoint{4.895948in}{2.611640in}}%
\pgfpathlineto{\pgfqpoint{4.895948in}{2.611640in}}%
\pgfpathlineto{\pgfqpoint{4.895948in}{2.614590in}}%
\pgfpathlineto{\pgfqpoint{4.900489in}{2.614590in}}%
\pgfpathlineto{\pgfqpoint{4.900489in}{2.611640in}}%
\pgfpathmoveto{\pgfqpoint{4.900489in}{2.611640in}}%
\pgfpathlineto{\pgfqpoint{4.900489in}{2.611640in}}%
\pgfpathlineto{\pgfqpoint{4.900489in}{2.614590in}}%
\pgfpathlineto{\pgfqpoint{4.905029in}{2.614590in}}%
\pgfpathlineto{\pgfqpoint{4.905029in}{2.611640in}}%
\pgfpathmoveto{\pgfqpoint{4.905029in}{2.611640in}}%
\pgfpathlineto{\pgfqpoint{4.905029in}{2.611640in}}%
\pgfpathlineto{\pgfqpoint{4.905029in}{2.614590in}}%
\pgfpathlineto{\pgfqpoint{4.909570in}{2.614590in}}%
\pgfpathlineto{\pgfqpoint{4.909570in}{2.611640in}}%
\pgfpathmoveto{\pgfqpoint{4.909570in}{2.611640in}}%
\pgfpathlineto{\pgfqpoint{4.909570in}{2.611640in}}%
\pgfpathlineto{\pgfqpoint{4.909570in}{2.614590in}}%
\pgfpathlineto{\pgfqpoint{4.914111in}{2.614590in}}%
\pgfpathlineto{\pgfqpoint{4.914111in}{2.611640in}}%
\pgfpathmoveto{\pgfqpoint{4.914111in}{2.611640in}}%
\pgfpathlineto{\pgfqpoint{4.914111in}{2.611640in}}%
\pgfpathlineto{\pgfqpoint{4.914111in}{2.614590in}}%
\pgfpathlineto{\pgfqpoint{4.918652in}{2.614590in}}%
\pgfpathlineto{\pgfqpoint{4.918652in}{2.611640in}}%
\pgfpathmoveto{\pgfqpoint{4.918652in}{2.611640in}}%
\pgfpathlineto{\pgfqpoint{4.918652in}{2.611640in}}%
\pgfpathlineto{\pgfqpoint{4.918652in}{2.614590in}}%
\pgfpathlineto{\pgfqpoint{4.923193in}{2.614590in}}%
\pgfpathlineto{\pgfqpoint{4.923193in}{2.611640in}}%
\pgfpathmoveto{\pgfqpoint{4.923193in}{2.611640in}}%
\pgfpathlineto{\pgfqpoint{4.923193in}{2.611640in}}%
\pgfpathlineto{\pgfqpoint{4.923193in}{2.614590in}}%
\pgfpathlineto{\pgfqpoint{4.927734in}{2.614590in}}%
\pgfpathlineto{\pgfqpoint{4.927734in}{2.611640in}}%
\pgfpathmoveto{\pgfqpoint{4.927734in}{2.611640in}}%
\pgfpathlineto{\pgfqpoint{4.927734in}{2.611640in}}%
\pgfpathlineto{\pgfqpoint{4.927734in}{2.614590in}}%
\pgfpathlineto{\pgfqpoint{4.932274in}{2.614590in}}%
\pgfpathlineto{\pgfqpoint{4.932274in}{2.611640in}}%
\pgfpathmoveto{\pgfqpoint{4.932274in}{2.611640in}}%
\pgfpathlineto{\pgfqpoint{4.932274in}{2.611640in}}%
\pgfpathlineto{\pgfqpoint{4.932274in}{2.614590in}}%
\pgfpathlineto{\pgfqpoint{4.936815in}{2.614590in}}%
\pgfpathlineto{\pgfqpoint{4.936815in}{2.611640in}}%
\pgfpathmoveto{\pgfqpoint{4.936815in}{2.611640in}}%
\pgfpathlineto{\pgfqpoint{4.936815in}{2.611640in}}%
\pgfpathlineto{\pgfqpoint{4.936815in}{2.614590in}}%
\pgfpathlineto{\pgfqpoint{4.941356in}{2.614590in}}%
\pgfpathlineto{\pgfqpoint{4.941356in}{2.611640in}}%
\pgfpathmoveto{\pgfqpoint{4.941356in}{2.611640in}}%
\pgfpathlineto{\pgfqpoint{4.941356in}{2.611640in}}%
\pgfpathlineto{\pgfqpoint{4.941356in}{2.614590in}}%
\pgfpathlineto{\pgfqpoint{4.945897in}{2.614590in}}%
\pgfpathlineto{\pgfqpoint{4.945897in}{2.611640in}}%
\pgfpathmoveto{\pgfqpoint{4.945897in}{2.611640in}}%
\pgfpathlineto{\pgfqpoint{4.945897in}{2.611640in}}%
\pgfpathlineto{\pgfqpoint{4.945897in}{2.614590in}}%
\pgfpathlineto{\pgfqpoint{4.950438in}{2.614590in}}%
\pgfpathlineto{\pgfqpoint{4.950438in}{2.611640in}}%
\pgfpathmoveto{\pgfqpoint{4.950438in}{2.611640in}}%
\pgfpathlineto{\pgfqpoint{4.950438in}{2.611640in}}%
\pgfpathlineto{\pgfqpoint{4.950438in}{2.614590in}}%
\pgfpathlineto{\pgfqpoint{4.954978in}{2.614590in}}%
\pgfpathlineto{\pgfqpoint{4.954978in}{2.611640in}}%
\pgfpathmoveto{\pgfqpoint{4.954978in}{2.611640in}}%
\pgfpathlineto{\pgfqpoint{4.954978in}{2.611640in}}%
\pgfpathlineto{\pgfqpoint{4.954978in}{2.614590in}}%
\pgfpathlineto{\pgfqpoint{4.959519in}{2.614590in}}%
\pgfpathlineto{\pgfqpoint{4.959519in}{2.611640in}}%
\pgfpathmoveto{\pgfqpoint{4.959519in}{2.611640in}}%
\pgfpathlineto{\pgfqpoint{4.959519in}{2.611640in}}%
\pgfpathlineto{\pgfqpoint{4.959519in}{2.614590in}}%
\pgfpathlineto{\pgfqpoint{4.964060in}{2.614590in}}%
\pgfpathlineto{\pgfqpoint{4.964060in}{2.611640in}}%
\pgfpathmoveto{\pgfqpoint{4.964060in}{2.611640in}}%
\pgfpathlineto{\pgfqpoint{4.964060in}{2.611640in}}%
\pgfpathlineto{\pgfqpoint{4.964060in}{2.614590in}}%
\pgfpathlineto{\pgfqpoint{4.968601in}{2.614590in}}%
\pgfpathlineto{\pgfqpoint{4.968601in}{2.611640in}}%
\pgfpathmoveto{\pgfqpoint{4.968601in}{2.611640in}}%
\pgfpathlineto{\pgfqpoint{4.968601in}{2.611640in}}%
\pgfpathlineto{\pgfqpoint{4.968601in}{2.614590in}}%
\pgfpathlineto{\pgfqpoint{4.973143in}{2.614590in}}%
\pgfpathlineto{\pgfqpoint{4.973143in}{2.611640in}}%
\pgfpathmoveto{\pgfqpoint{4.973143in}{2.611640in}}%
\pgfpathlineto{\pgfqpoint{4.973143in}{2.611640in}}%
\pgfpathlineto{\pgfqpoint{4.973143in}{2.614590in}}%
\pgfpathlineto{\pgfqpoint{4.977684in}{2.614590in}}%
\pgfpathlineto{\pgfqpoint{4.977684in}{2.611640in}}%
\pgfpathmoveto{\pgfqpoint{4.977684in}{2.611640in}}%
\pgfpathlineto{\pgfqpoint{4.977684in}{2.611640in}}%
\pgfpathlineto{\pgfqpoint{4.977684in}{2.614590in}}%
\pgfpathlineto{\pgfqpoint{4.982225in}{2.614590in}}%
\pgfpathlineto{\pgfqpoint{4.982225in}{2.611640in}}%
\pgfpathmoveto{\pgfqpoint{4.982225in}{2.611640in}}%
\pgfpathlineto{\pgfqpoint{4.982225in}{2.611640in}}%
\pgfpathlineto{\pgfqpoint{4.982225in}{2.614590in}}%
\pgfpathlineto{\pgfqpoint{4.986766in}{2.614590in}}%
\pgfpathlineto{\pgfqpoint{4.986766in}{2.611640in}}%
\pgfpathmoveto{\pgfqpoint{4.986766in}{2.611640in}}%
\pgfpathlineto{\pgfqpoint{4.986766in}{2.611640in}}%
\pgfpathlineto{\pgfqpoint{4.986766in}{2.614590in}}%
\pgfpathlineto{\pgfqpoint{4.991307in}{2.614590in}}%
\pgfpathlineto{\pgfqpoint{4.991307in}{2.611640in}}%
\pgfpathmoveto{\pgfqpoint{4.991307in}{2.611640in}}%
\pgfpathlineto{\pgfqpoint{4.991307in}{2.611640in}}%
\pgfpathlineto{\pgfqpoint{4.991307in}{2.614590in}}%
\pgfpathlineto{\pgfqpoint{4.995849in}{2.614590in}}%
\pgfpathlineto{\pgfqpoint{4.995849in}{2.611640in}}%
\pgfpathmoveto{\pgfqpoint{4.995849in}{2.611640in}}%
\pgfpathlineto{\pgfqpoint{4.995849in}{2.611640in}}%
\pgfpathlineto{\pgfqpoint{4.995849in}{2.614590in}}%
\pgfpathlineto{\pgfqpoint{5.000390in}{2.614590in}}%
\pgfpathlineto{\pgfqpoint{5.000390in}{2.611640in}}%
\pgfpathmoveto{\pgfqpoint{5.000390in}{2.611640in}}%
\pgfpathlineto{\pgfqpoint{5.000390in}{2.611640in}}%
\pgfpathlineto{\pgfqpoint{5.000390in}{2.614590in}}%
\pgfpathlineto{\pgfqpoint{5.004931in}{2.614590in}}%
\pgfpathlineto{\pgfqpoint{5.004931in}{2.611640in}}%
\pgfpathmoveto{\pgfqpoint{5.004931in}{2.611640in}}%
\pgfpathlineto{\pgfqpoint{5.004931in}{2.611640in}}%
\pgfpathlineto{\pgfqpoint{5.004931in}{2.614590in}}%
\pgfpathlineto{\pgfqpoint{5.009472in}{2.614590in}}%
\pgfpathlineto{\pgfqpoint{5.009472in}{2.611640in}}%
\pgfpathmoveto{\pgfqpoint{5.009472in}{2.611640in}}%
\pgfpathlineto{\pgfqpoint{5.009472in}{2.611640in}}%
\pgfpathlineto{\pgfqpoint{5.009472in}{2.614590in}}%
\pgfpathlineto{\pgfqpoint{5.014013in}{2.614590in}}%
\pgfpathlineto{\pgfqpoint{5.014013in}{2.611640in}}%
\pgfpathmoveto{\pgfqpoint{5.014013in}{2.611640in}}%
\pgfpathlineto{\pgfqpoint{5.014013in}{2.611640in}}%
\pgfpathlineto{\pgfqpoint{5.014013in}{2.614590in}}%
\pgfpathlineto{\pgfqpoint{5.018555in}{2.614590in}}%
\pgfpathlineto{\pgfqpoint{5.018555in}{2.611640in}}%
\pgfpathmoveto{\pgfqpoint{5.018555in}{2.611640in}}%
\pgfpathlineto{\pgfqpoint{5.018555in}{2.611640in}}%
\pgfpathlineto{\pgfqpoint{5.018555in}{2.614590in}}%
\pgfpathlineto{\pgfqpoint{5.023096in}{2.614590in}}%
\pgfpathlineto{\pgfqpoint{5.023096in}{2.611640in}}%
\pgfpathmoveto{\pgfqpoint{5.023096in}{2.611640in}}%
\pgfpathlineto{\pgfqpoint{5.023096in}{2.611640in}}%
\pgfpathlineto{\pgfqpoint{5.023096in}{2.614590in}}%
\pgfpathlineto{\pgfqpoint{5.027637in}{2.614590in}}%
\pgfpathlineto{\pgfqpoint{5.027637in}{2.611640in}}%
\pgfpathmoveto{\pgfqpoint{5.027637in}{2.611640in}}%
\pgfpathlineto{\pgfqpoint{5.027637in}{2.611640in}}%
\pgfpathlineto{\pgfqpoint{5.027637in}{2.614590in}}%
\pgfpathlineto{\pgfqpoint{5.032178in}{2.614590in}}%
\pgfpathlineto{\pgfqpoint{5.032178in}{2.611640in}}%
\pgfpathmoveto{\pgfqpoint{5.032178in}{2.611640in}}%
\pgfpathlineto{\pgfqpoint{5.032178in}{2.611640in}}%
\pgfpathlineto{\pgfqpoint{5.032178in}{2.614590in}}%
\pgfpathlineto{\pgfqpoint{5.036719in}{2.614590in}}%
\pgfpathlineto{\pgfqpoint{5.036719in}{2.611640in}}%
\pgfpathmoveto{\pgfqpoint{5.036719in}{2.611640in}}%
\pgfpathlineto{\pgfqpoint{5.036719in}{2.611640in}}%
\pgfpathlineto{\pgfqpoint{5.036719in}{2.614590in}}%
\pgfpathlineto{\pgfqpoint{5.041261in}{2.614590in}}%
\pgfpathlineto{\pgfqpoint{5.041261in}{2.611640in}}%
\pgfpathmoveto{\pgfqpoint{5.041261in}{2.611640in}}%
\pgfpathlineto{\pgfqpoint{5.041261in}{2.611640in}}%
\pgfpathlineto{\pgfqpoint{5.041261in}{2.614590in}}%
\pgfpathlineto{\pgfqpoint{5.045802in}{2.614590in}}%
\pgfpathlineto{\pgfqpoint{5.045802in}{2.611640in}}%
\pgfpathmoveto{\pgfqpoint{5.045802in}{2.611640in}}%
\pgfpathlineto{\pgfqpoint{5.045802in}{2.611640in}}%
\pgfpathlineto{\pgfqpoint{5.045802in}{2.614590in}}%
\pgfpathlineto{\pgfqpoint{5.050343in}{2.614590in}}%
\pgfpathlineto{\pgfqpoint{5.050343in}{2.611640in}}%
\pgfpathmoveto{\pgfqpoint{5.050343in}{2.611640in}}%
\pgfpathlineto{\pgfqpoint{5.050343in}{2.611640in}}%
\pgfpathlineto{\pgfqpoint{5.050343in}{2.614590in}}%
\pgfpathlineto{\pgfqpoint{5.054884in}{2.614590in}}%
\pgfpathlineto{\pgfqpoint{5.054884in}{2.611640in}}%
\pgfpathmoveto{\pgfqpoint{5.054884in}{2.611640in}}%
\pgfpathlineto{\pgfqpoint{5.054884in}{2.611640in}}%
\pgfpathlineto{\pgfqpoint{5.054884in}{2.614590in}}%
\pgfpathlineto{\pgfqpoint{5.059425in}{2.614590in}}%
\pgfpathlineto{\pgfqpoint{5.059425in}{2.611640in}}%
\pgfpathmoveto{\pgfqpoint{5.059425in}{2.611640in}}%
\pgfpathlineto{\pgfqpoint{5.059425in}{2.611640in}}%
\pgfpathlineto{\pgfqpoint{5.059425in}{2.614590in}}%
\pgfpathlineto{\pgfqpoint{5.063967in}{2.614590in}}%
\pgfpathlineto{\pgfqpoint{5.063967in}{2.611640in}}%
\pgfpathmoveto{\pgfqpoint{5.063967in}{2.611640in}}%
\pgfpathlineto{\pgfqpoint{5.063967in}{2.611640in}}%
\pgfpathlineto{\pgfqpoint{5.063967in}{2.614590in}}%
\pgfpathlineto{\pgfqpoint{5.068508in}{2.614590in}}%
\pgfpathlineto{\pgfqpoint{5.068508in}{2.611640in}}%
\pgfpathmoveto{\pgfqpoint{5.068508in}{2.611640in}}%
\pgfpathlineto{\pgfqpoint{5.068508in}{2.611640in}}%
\pgfpathlineto{\pgfqpoint{5.068508in}{2.614590in}}%
\pgfpathlineto{\pgfqpoint{5.073049in}{2.614590in}}%
\pgfpathlineto{\pgfqpoint{5.073049in}{2.611640in}}%
\pgfpathmoveto{\pgfqpoint{5.073049in}{2.611640in}}%
\pgfpathlineto{\pgfqpoint{5.073049in}{2.611640in}}%
\pgfpathlineto{\pgfqpoint{5.073049in}{2.614590in}}%
\pgfpathlineto{\pgfqpoint{5.077590in}{2.614590in}}%
\pgfpathlineto{\pgfqpoint{5.077590in}{2.611640in}}%
\pgfpathmoveto{\pgfqpoint{5.077590in}{2.611640in}}%
\pgfpathlineto{\pgfqpoint{5.077590in}{2.611640in}}%
\pgfpathlineto{\pgfqpoint{5.077590in}{2.614590in}}%
\pgfpathlineto{\pgfqpoint{5.082132in}{2.614590in}}%
\pgfpathlineto{\pgfqpoint{5.082132in}{2.611640in}}%
\pgfpathmoveto{\pgfqpoint{5.082132in}{2.611640in}}%
\pgfpathlineto{\pgfqpoint{5.082132in}{2.611640in}}%
\pgfpathlineto{\pgfqpoint{5.082132in}{2.614590in}}%
\pgfpathlineto{\pgfqpoint{5.086673in}{2.614590in}}%
\pgfpathlineto{\pgfqpoint{5.086673in}{2.611640in}}%
\pgfpathmoveto{\pgfqpoint{5.086673in}{2.611640in}}%
\pgfpathlineto{\pgfqpoint{5.086673in}{2.611640in}}%
\pgfpathlineto{\pgfqpoint{5.086673in}{2.614590in}}%
\pgfpathlineto{\pgfqpoint{5.091214in}{2.614590in}}%
\pgfpathlineto{\pgfqpoint{5.091214in}{2.611640in}}%
\pgfpathmoveto{\pgfqpoint{5.091214in}{2.611640in}}%
\pgfpathlineto{\pgfqpoint{5.091214in}{2.611640in}}%
\pgfpathlineto{\pgfqpoint{5.091214in}{2.614590in}}%
\pgfpathlineto{\pgfqpoint{5.095755in}{2.614590in}}%
\pgfpathlineto{\pgfqpoint{5.095755in}{2.611640in}}%
\pgfpathmoveto{\pgfqpoint{5.095755in}{2.611640in}}%
\pgfpathlineto{\pgfqpoint{5.095755in}{2.611640in}}%
\pgfpathlineto{\pgfqpoint{5.095755in}{2.614590in}}%
\pgfpathlineto{\pgfqpoint{5.100296in}{2.614590in}}%
\pgfpathlineto{\pgfqpoint{5.100296in}{2.611640in}}%
\pgfpathmoveto{\pgfqpoint{5.100296in}{2.611640in}}%
\pgfpathlineto{\pgfqpoint{5.100296in}{2.611640in}}%
\pgfpathlineto{\pgfqpoint{5.100296in}{2.614590in}}%
\pgfpathlineto{\pgfqpoint{5.104838in}{2.614590in}}%
\pgfpathlineto{\pgfqpoint{5.104838in}{2.611640in}}%
\pgfpathmoveto{\pgfqpoint{5.104838in}{2.611640in}}%
\pgfpathlineto{\pgfqpoint{5.104838in}{2.611640in}}%
\pgfpathlineto{\pgfqpoint{5.104838in}{2.614590in}}%
\pgfpathlineto{\pgfqpoint{5.109379in}{2.614590in}}%
\pgfpathlineto{\pgfqpoint{5.109379in}{2.611640in}}%
\pgfpathmoveto{\pgfqpoint{5.109379in}{2.611640in}}%
\pgfpathlineto{\pgfqpoint{5.109379in}{2.611640in}}%
\pgfpathlineto{\pgfqpoint{5.109379in}{2.614590in}}%
\pgfpathlineto{\pgfqpoint{5.113920in}{2.614590in}}%
\pgfpathlineto{\pgfqpoint{5.113920in}{2.611640in}}%
\pgfpathmoveto{\pgfqpoint{5.113920in}{2.611640in}}%
\pgfpathlineto{\pgfqpoint{5.113920in}{2.611640in}}%
\pgfpathlineto{\pgfqpoint{5.113920in}{2.614590in}}%
\pgfpathlineto{\pgfqpoint{5.118461in}{2.614590in}}%
\pgfpathlineto{\pgfqpoint{5.118461in}{2.611640in}}%
\pgfpathmoveto{\pgfqpoint{5.118461in}{2.611640in}}%
\pgfpathlineto{\pgfqpoint{5.118461in}{2.611640in}}%
\pgfpathlineto{\pgfqpoint{5.118461in}{2.614590in}}%
\pgfpathlineto{\pgfqpoint{5.123002in}{2.614590in}}%
\pgfpathlineto{\pgfqpoint{5.123002in}{2.611640in}}%
\pgfpathmoveto{\pgfqpoint{5.123002in}{2.611640in}}%
\pgfpathlineto{\pgfqpoint{5.123002in}{2.611640in}}%
\pgfpathlineto{\pgfqpoint{5.123002in}{2.614590in}}%
\pgfpathlineto{\pgfqpoint{5.127543in}{2.614590in}}%
\pgfpathlineto{\pgfqpoint{5.127543in}{2.611640in}}%
\pgfpathmoveto{\pgfqpoint{5.127543in}{2.611640in}}%
\pgfpathlineto{\pgfqpoint{5.127543in}{2.611640in}}%
\pgfpathlineto{\pgfqpoint{5.127543in}{2.614590in}}%
\pgfpathlineto{\pgfqpoint{5.132084in}{2.614590in}}%
\pgfpathlineto{\pgfqpoint{5.132084in}{2.611640in}}%
\pgfpathmoveto{\pgfqpoint{5.132084in}{2.611640in}}%
\pgfpathlineto{\pgfqpoint{5.132084in}{2.611640in}}%
\pgfpathlineto{\pgfqpoint{5.132084in}{2.614590in}}%
\pgfpathlineto{\pgfqpoint{5.136625in}{2.614590in}}%
\pgfpathlineto{\pgfqpoint{5.136625in}{2.611640in}}%
\pgfpathmoveto{\pgfqpoint{5.136625in}{2.611640in}}%
\pgfpathlineto{\pgfqpoint{5.136625in}{2.611640in}}%
\pgfpathlineto{\pgfqpoint{5.136625in}{2.614590in}}%
\pgfpathlineto{\pgfqpoint{5.141166in}{2.614590in}}%
\pgfpathlineto{\pgfqpoint{5.141166in}{2.611640in}}%
\pgfpathmoveto{\pgfqpoint{5.141166in}{2.611640in}}%
\pgfpathlineto{\pgfqpoint{5.141166in}{2.611640in}}%
\pgfpathlineto{\pgfqpoint{5.141166in}{2.614590in}}%
\pgfpathlineto{\pgfqpoint{5.145707in}{2.614590in}}%
\pgfpathlineto{\pgfqpoint{5.145707in}{2.611640in}}%
\pgfpathmoveto{\pgfqpoint{5.145707in}{2.611640in}}%
\pgfpathlineto{\pgfqpoint{5.145707in}{2.611640in}}%
\pgfpathlineto{\pgfqpoint{5.145707in}{2.614590in}}%
\pgfpathlineto{\pgfqpoint{5.150248in}{2.614590in}}%
\pgfpathlineto{\pgfqpoint{5.150248in}{2.611640in}}%
\pgfpathmoveto{\pgfqpoint{5.150248in}{2.611640in}}%
\pgfpathlineto{\pgfqpoint{5.150248in}{2.611640in}}%
\pgfpathlineto{\pgfqpoint{5.150248in}{2.614590in}}%
\pgfpathlineto{\pgfqpoint{5.154789in}{2.614590in}}%
\pgfpathlineto{\pgfqpoint{5.154789in}{2.611640in}}%
\pgfpathmoveto{\pgfqpoint{5.154789in}{2.611640in}}%
\pgfpathlineto{\pgfqpoint{5.154789in}{2.611640in}}%
\pgfpathlineto{\pgfqpoint{5.154789in}{2.614590in}}%
\pgfpathlineto{\pgfqpoint{5.159330in}{2.614590in}}%
\pgfpathlineto{\pgfqpoint{5.159330in}{2.611640in}}%
\pgfpathmoveto{\pgfqpoint{5.159330in}{2.611640in}}%
\pgfpathlineto{\pgfqpoint{5.159330in}{2.611640in}}%
\pgfpathlineto{\pgfqpoint{5.159330in}{2.614590in}}%
\pgfpathlineto{\pgfqpoint{5.163871in}{2.614590in}}%
\pgfpathlineto{\pgfqpoint{5.163871in}{2.611640in}}%
\pgfpathmoveto{\pgfqpoint{5.163871in}{2.611640in}}%
\pgfpathlineto{\pgfqpoint{5.163871in}{2.611640in}}%
\pgfpathlineto{\pgfqpoint{5.163871in}{2.614590in}}%
\pgfpathlineto{\pgfqpoint{5.168412in}{2.614590in}}%
\pgfpathlineto{\pgfqpoint{5.168412in}{2.611640in}}%
\pgfpathmoveto{\pgfqpoint{5.168412in}{2.611640in}}%
\pgfpathlineto{\pgfqpoint{5.168412in}{2.611640in}}%
\pgfpathlineto{\pgfqpoint{5.168412in}{2.614590in}}%
\pgfpathlineto{\pgfqpoint{5.172953in}{2.614590in}}%
\pgfpathlineto{\pgfqpoint{5.172953in}{2.611640in}}%
\pgfpathmoveto{\pgfqpoint{5.172953in}{2.611640in}}%
\pgfpathlineto{\pgfqpoint{5.172953in}{2.611640in}}%
\pgfpathlineto{\pgfqpoint{5.172953in}{2.614590in}}%
\pgfpathlineto{\pgfqpoint{5.177494in}{2.614590in}}%
\pgfpathlineto{\pgfqpoint{5.177494in}{2.611640in}}%
\pgfpathmoveto{\pgfqpoint{5.177494in}{2.611640in}}%
\pgfpathlineto{\pgfqpoint{5.177494in}{2.611640in}}%
\pgfpathlineto{\pgfqpoint{5.177494in}{2.614590in}}%
\pgfpathlineto{\pgfqpoint{5.182035in}{2.614590in}}%
\pgfpathlineto{\pgfqpoint{5.182035in}{2.611640in}}%
\pgfpathmoveto{\pgfqpoint{5.182035in}{2.611640in}}%
\pgfpathlineto{\pgfqpoint{5.182035in}{2.611640in}}%
\pgfpathlineto{\pgfqpoint{5.182035in}{2.614590in}}%
\pgfpathlineto{\pgfqpoint{5.186576in}{2.614590in}}%
\pgfpathlineto{\pgfqpoint{5.186576in}{2.611640in}}%
\pgfpathmoveto{\pgfqpoint{5.186576in}{2.611640in}}%
\pgfpathlineto{\pgfqpoint{5.186576in}{2.611640in}}%
\pgfpathlineto{\pgfqpoint{5.186576in}{2.614590in}}%
\pgfpathlineto{\pgfqpoint{5.191117in}{2.614590in}}%
\pgfpathlineto{\pgfqpoint{5.191117in}{2.611640in}}%
\pgfpathmoveto{\pgfqpoint{5.191117in}{2.611640in}}%
\pgfpathlineto{\pgfqpoint{5.191117in}{2.611640in}}%
\pgfpathlineto{\pgfqpoint{5.191117in}{2.614590in}}%
\pgfpathlineto{\pgfqpoint{5.195658in}{2.614590in}}%
\pgfpathlineto{\pgfqpoint{5.195658in}{2.611640in}}%
\pgfpathmoveto{\pgfqpoint{5.195658in}{2.611640in}}%
\pgfpathlineto{\pgfqpoint{5.195658in}{2.611640in}}%
\pgfpathlineto{\pgfqpoint{5.195658in}{2.614590in}}%
\pgfpathlineto{\pgfqpoint{5.200199in}{2.614590in}}%
\pgfpathlineto{\pgfqpoint{5.200199in}{2.611640in}}%
\pgfpathmoveto{\pgfqpoint{5.200199in}{2.611640in}}%
\pgfpathlineto{\pgfqpoint{5.200199in}{2.611640in}}%
\pgfpathlineto{\pgfqpoint{5.200199in}{2.614590in}}%
\pgfpathlineto{\pgfqpoint{5.204740in}{2.614590in}}%
\pgfpathlineto{\pgfqpoint{5.204740in}{2.611640in}}%
\pgfpathmoveto{\pgfqpoint{5.204740in}{2.611640in}}%
\pgfpathlineto{\pgfqpoint{5.204740in}{2.611640in}}%
\pgfpathlineto{\pgfqpoint{5.204740in}{2.614590in}}%
\pgfpathlineto{\pgfqpoint{5.209281in}{2.614590in}}%
\pgfpathlineto{\pgfqpoint{5.209281in}{2.611640in}}%
\pgfpathmoveto{\pgfqpoint{5.209281in}{2.611640in}}%
\pgfpathlineto{\pgfqpoint{5.209281in}{2.611640in}}%
\pgfpathlineto{\pgfqpoint{5.209281in}{2.614590in}}%
\pgfpathlineto{\pgfqpoint{5.213822in}{2.614590in}}%
\pgfpathlineto{\pgfqpoint{5.213822in}{2.611640in}}%
\pgfpathmoveto{\pgfqpoint{5.213822in}{2.611640in}}%
\pgfpathlineto{\pgfqpoint{5.213822in}{2.611640in}}%
\pgfpathlineto{\pgfqpoint{5.213822in}{2.614590in}}%
\pgfpathlineto{\pgfqpoint{5.218363in}{2.614590in}}%
\pgfpathlineto{\pgfqpoint{5.218363in}{2.611640in}}%
\pgfpathmoveto{\pgfqpoint{5.218363in}{2.611640in}}%
\pgfpathlineto{\pgfqpoint{5.218363in}{2.611640in}}%
\pgfpathlineto{\pgfqpoint{5.218363in}{2.614590in}}%
\pgfpathlineto{\pgfqpoint{5.222904in}{2.614590in}}%
\pgfpathlineto{\pgfqpoint{5.222904in}{2.611640in}}%
\pgfpathmoveto{\pgfqpoint{5.222904in}{2.611640in}}%
\pgfpathlineto{\pgfqpoint{5.222904in}{2.611640in}}%
\pgfpathlineto{\pgfqpoint{5.222904in}{2.614590in}}%
\pgfpathlineto{\pgfqpoint{5.227445in}{2.614590in}}%
\pgfpathlineto{\pgfqpoint{5.227445in}{2.611640in}}%
\pgfpathmoveto{\pgfqpoint{5.227445in}{2.611640in}}%
\pgfpathlineto{\pgfqpoint{5.227445in}{2.611640in}}%
\pgfpathlineto{\pgfqpoint{5.227445in}{2.614590in}}%
\pgfpathlineto{\pgfqpoint{5.231986in}{2.614590in}}%
\pgfpathlineto{\pgfqpoint{5.231986in}{2.611640in}}%
\pgfpathmoveto{\pgfqpoint{5.231986in}{2.611640in}}%
\pgfpathlineto{\pgfqpoint{5.231986in}{2.611640in}}%
\pgfpathlineto{\pgfqpoint{5.231986in}{2.614590in}}%
\pgfpathlineto{\pgfqpoint{5.236527in}{2.614590in}}%
\pgfpathlineto{\pgfqpoint{5.236527in}{2.611640in}}%
\pgfpathmoveto{\pgfqpoint{5.236527in}{2.611640in}}%
\pgfpathlineto{\pgfqpoint{5.236527in}{2.611640in}}%
\pgfpathlineto{\pgfqpoint{5.236527in}{2.614590in}}%
\pgfpathlineto{\pgfqpoint{5.241068in}{2.614590in}}%
\pgfpathlineto{\pgfqpoint{5.241068in}{2.611640in}}%
\pgfpathmoveto{\pgfqpoint{5.241068in}{2.611640in}}%
\pgfpathlineto{\pgfqpoint{5.241068in}{2.611640in}}%
\pgfpathlineto{\pgfqpoint{5.241068in}{2.614590in}}%
\pgfpathlineto{\pgfqpoint{5.245609in}{2.614590in}}%
\pgfpathlineto{\pgfqpoint{5.245609in}{2.611640in}}%
\pgfpathmoveto{\pgfqpoint{5.245609in}{2.611640in}}%
\pgfpathlineto{\pgfqpoint{5.245609in}{2.611640in}}%
\pgfpathlineto{\pgfqpoint{5.245609in}{2.614590in}}%
\pgfpathlineto{\pgfqpoint{5.250150in}{2.614590in}}%
\pgfpathlineto{\pgfqpoint{5.250150in}{2.611640in}}%
\pgfpathmoveto{\pgfqpoint{5.250150in}{2.611640in}}%
\pgfpathlineto{\pgfqpoint{5.250150in}{2.611640in}}%
\pgfpathlineto{\pgfqpoint{5.250150in}{2.614590in}}%
\pgfpathlineto{\pgfqpoint{5.254691in}{2.614590in}}%
\pgfpathlineto{\pgfqpoint{5.254691in}{2.611640in}}%
\pgfpathmoveto{\pgfqpoint{5.254691in}{2.611640in}}%
\pgfpathlineto{\pgfqpoint{5.254691in}{2.611640in}}%
\pgfpathlineto{\pgfqpoint{5.254691in}{2.614590in}}%
\pgfpathlineto{\pgfqpoint{5.259232in}{2.614590in}}%
\pgfpathlineto{\pgfqpoint{5.259232in}{2.611640in}}%
\pgfpathmoveto{\pgfqpoint{5.259232in}{2.611640in}}%
\pgfpathlineto{\pgfqpoint{5.259232in}{2.611640in}}%
\pgfpathlineto{\pgfqpoint{5.259232in}{2.614590in}}%
\pgfpathlineto{\pgfqpoint{5.263773in}{2.614590in}}%
\pgfpathlineto{\pgfqpoint{5.263773in}{2.611640in}}%
\pgfpathmoveto{\pgfqpoint{5.263773in}{2.611640in}}%
\pgfpathlineto{\pgfqpoint{5.263773in}{2.611640in}}%
\pgfpathlineto{\pgfqpoint{5.263773in}{2.614590in}}%
\pgfpathlineto{\pgfqpoint{5.268314in}{2.614590in}}%
\pgfpathlineto{\pgfqpoint{5.268314in}{2.611640in}}%
\pgfpathmoveto{\pgfqpoint{5.268314in}{2.611640in}}%
\pgfpathlineto{\pgfqpoint{5.268314in}{2.611640in}}%
\pgfpathlineto{\pgfqpoint{5.268314in}{2.614590in}}%
\pgfpathlineto{\pgfqpoint{5.272855in}{2.614590in}}%
\pgfpathlineto{\pgfqpoint{5.272855in}{2.611640in}}%
\pgfpathmoveto{\pgfqpoint{5.272855in}{2.611640in}}%
\pgfpathlineto{\pgfqpoint{5.272855in}{2.611640in}}%
\pgfpathlineto{\pgfqpoint{5.272855in}{2.614590in}}%
\pgfpathlineto{\pgfqpoint{5.277396in}{2.614590in}}%
\pgfpathlineto{\pgfqpoint{5.277396in}{2.611640in}}%
\pgfpathmoveto{\pgfqpoint{5.277396in}{2.611640in}}%
\pgfpathlineto{\pgfqpoint{5.277396in}{2.611640in}}%
\pgfpathlineto{\pgfqpoint{5.277396in}{2.614590in}}%
\pgfpathlineto{\pgfqpoint{5.281937in}{2.614590in}}%
\pgfpathlineto{\pgfqpoint{5.281937in}{2.611640in}}%
\pgfpathmoveto{\pgfqpoint{5.281937in}{2.611640in}}%
\pgfpathlineto{\pgfqpoint{5.281937in}{2.611640in}}%
\pgfpathlineto{\pgfqpoint{5.281937in}{2.614590in}}%
\pgfpathlineto{\pgfqpoint{5.286478in}{2.614590in}}%
\pgfpathlineto{\pgfqpoint{5.286478in}{2.611640in}}%
\pgfpathmoveto{\pgfqpoint{5.286478in}{2.611640in}}%
\pgfpathlineto{\pgfqpoint{5.286478in}{2.611640in}}%
\pgfpathlineto{\pgfqpoint{5.286478in}{2.614590in}}%
\pgfpathlineto{\pgfqpoint{5.291019in}{2.614590in}}%
\pgfpathlineto{\pgfqpoint{5.291019in}{2.611640in}}%
\pgfpathmoveto{\pgfqpoint{5.291019in}{2.611640in}}%
\pgfpathlineto{\pgfqpoint{5.291019in}{2.611640in}}%
\pgfpathlineto{\pgfqpoint{5.291019in}{2.614590in}}%
\pgfpathlineto{\pgfqpoint{5.295560in}{2.614590in}}%
\pgfpathlineto{\pgfqpoint{5.295560in}{2.611640in}}%
\pgfpathmoveto{\pgfqpoint{5.295560in}{2.611640in}}%
\pgfpathlineto{\pgfqpoint{5.295560in}{2.611640in}}%
\pgfpathlineto{\pgfqpoint{5.295560in}{2.614590in}}%
\pgfpathlineto{\pgfqpoint{5.300101in}{2.614590in}}%
\pgfpathlineto{\pgfqpoint{5.300101in}{2.611640in}}%
\pgfpathmoveto{\pgfqpoint{5.300101in}{2.611640in}}%
\pgfpathlineto{\pgfqpoint{5.300101in}{2.611640in}}%
\pgfpathlineto{\pgfqpoint{5.300101in}{2.614590in}}%
\pgfpathlineto{\pgfqpoint{5.304642in}{2.614590in}}%
\pgfpathlineto{\pgfqpoint{5.304642in}{2.611640in}}%
\pgfpathmoveto{\pgfqpoint{5.304642in}{2.611640in}}%
\pgfpathlineto{\pgfqpoint{5.304642in}{2.611640in}}%
\pgfpathlineto{\pgfqpoint{5.304642in}{2.614590in}}%
\pgfpathlineto{\pgfqpoint{5.309183in}{2.614590in}}%
\pgfpathlineto{\pgfqpoint{5.309183in}{2.611640in}}%
\pgfpathmoveto{\pgfqpoint{5.309183in}{2.611640in}}%
\pgfpathlineto{\pgfqpoint{5.309183in}{2.611640in}}%
\pgfpathlineto{\pgfqpoint{5.309183in}{2.614590in}}%
\pgfpathlineto{\pgfqpoint{5.313724in}{2.614590in}}%
\pgfpathlineto{\pgfqpoint{5.313724in}{2.611640in}}%
\pgfpathmoveto{\pgfqpoint{5.313724in}{2.611640in}}%
\pgfpathlineto{\pgfqpoint{5.313724in}{2.611640in}}%
\pgfpathlineto{\pgfqpoint{5.313724in}{2.614590in}}%
\pgfpathlineto{\pgfqpoint{5.318265in}{2.614590in}}%
\pgfpathlineto{\pgfqpoint{5.318265in}{2.611640in}}%
\pgfpathmoveto{\pgfqpoint{5.318265in}{2.611640in}}%
\pgfpathlineto{\pgfqpoint{5.318265in}{2.611640in}}%
\pgfpathlineto{\pgfqpoint{5.318265in}{2.614590in}}%
\pgfpathlineto{\pgfqpoint{5.322805in}{2.614590in}}%
\pgfpathlineto{\pgfqpoint{5.322805in}{2.611640in}}%
\pgfpathmoveto{\pgfqpoint{5.322805in}{2.611640in}}%
\pgfpathlineto{\pgfqpoint{5.322805in}{2.611640in}}%
\pgfpathlineto{\pgfqpoint{5.322805in}{2.614590in}}%
\pgfpathlineto{\pgfqpoint{5.327346in}{2.614590in}}%
\pgfpathlineto{\pgfqpoint{5.327346in}{2.611640in}}%
\pgfpathmoveto{\pgfqpoint{5.327346in}{2.611640in}}%
\pgfpathlineto{\pgfqpoint{5.327346in}{2.611640in}}%
\pgfpathlineto{\pgfqpoint{5.327346in}{2.614590in}}%
\pgfpathlineto{\pgfqpoint{5.331887in}{2.614590in}}%
\pgfpathlineto{\pgfqpoint{5.331887in}{2.611640in}}%
\pgfpathmoveto{\pgfqpoint{5.331887in}{2.611640in}}%
\pgfpathlineto{\pgfqpoint{5.331887in}{2.611640in}}%
\pgfpathlineto{\pgfqpoint{5.331887in}{2.614590in}}%
\pgfpathlineto{\pgfqpoint{5.336428in}{2.614590in}}%
\pgfpathlineto{\pgfqpoint{5.336428in}{2.611640in}}%
\pgfpathmoveto{\pgfqpoint{5.336428in}{2.611640in}}%
\pgfpathlineto{\pgfqpoint{5.336428in}{2.611640in}}%
\pgfpathlineto{\pgfqpoint{5.336428in}{2.614590in}}%
\pgfpathlineto{\pgfqpoint{5.340969in}{2.614590in}}%
\pgfpathlineto{\pgfqpoint{5.340969in}{2.611640in}}%
\pgfpathmoveto{\pgfqpoint{5.340969in}{2.611640in}}%
\pgfpathlineto{\pgfqpoint{5.340969in}{2.611640in}}%
\pgfpathlineto{\pgfqpoint{5.340969in}{2.614590in}}%
\pgfpathlineto{\pgfqpoint{5.345510in}{2.614590in}}%
\pgfpathlineto{\pgfqpoint{5.345510in}{2.611640in}}%
\pgfpathmoveto{\pgfqpoint{5.345510in}{2.611640in}}%
\pgfpathlineto{\pgfqpoint{5.345510in}{2.611640in}}%
\pgfpathlineto{\pgfqpoint{5.345510in}{2.614590in}}%
\pgfpathlineto{\pgfqpoint{5.350051in}{2.614590in}}%
\pgfpathlineto{\pgfqpoint{5.350051in}{2.611640in}}%
\pgfpathmoveto{\pgfqpoint{5.350051in}{2.611640in}}%
\pgfpathlineto{\pgfqpoint{5.350051in}{2.611640in}}%
\pgfpathlineto{\pgfqpoint{5.350051in}{2.614590in}}%
\pgfpathlineto{\pgfqpoint{5.354592in}{2.614590in}}%
\pgfpathlineto{\pgfqpoint{5.354592in}{2.611640in}}%
\pgfpathmoveto{\pgfqpoint{5.354592in}{2.611640in}}%
\pgfpathlineto{\pgfqpoint{5.354592in}{2.611640in}}%
\pgfpathlineto{\pgfqpoint{5.354592in}{2.614590in}}%
\pgfpathlineto{\pgfqpoint{5.359133in}{2.614590in}}%
\pgfpathlineto{\pgfqpoint{5.359133in}{2.611640in}}%
\pgfpathmoveto{\pgfqpoint{5.359133in}{2.611640in}}%
\pgfpathlineto{\pgfqpoint{5.359133in}{2.611640in}}%
\pgfpathlineto{\pgfqpoint{5.359133in}{2.614590in}}%
\pgfpathlineto{\pgfqpoint{5.363674in}{2.614590in}}%
\pgfpathlineto{\pgfqpoint{5.363674in}{2.611640in}}%
\pgfpathmoveto{\pgfqpoint{5.363674in}{2.611640in}}%
\pgfpathlineto{\pgfqpoint{5.363674in}{2.611640in}}%
\pgfpathlineto{\pgfqpoint{5.363674in}{2.614590in}}%
\pgfpathlineto{\pgfqpoint{5.368215in}{2.614590in}}%
\pgfpathlineto{\pgfqpoint{5.368215in}{2.611640in}}%
\pgfpathmoveto{\pgfqpoint{5.368215in}{2.611640in}}%
\pgfpathlineto{\pgfqpoint{5.368215in}{2.611640in}}%
\pgfpathlineto{\pgfqpoint{5.368215in}{2.614590in}}%
\pgfpathlineto{\pgfqpoint{5.372756in}{2.614590in}}%
\pgfpathlineto{\pgfqpoint{5.372756in}{2.611640in}}%
\pgfpathmoveto{\pgfqpoint{5.372756in}{2.611640in}}%
\pgfpathlineto{\pgfqpoint{5.372756in}{2.611640in}}%
\pgfpathlineto{\pgfqpoint{5.372756in}{2.614590in}}%
\pgfpathlineto{\pgfqpoint{5.377297in}{2.614590in}}%
\pgfpathlineto{\pgfqpoint{5.377297in}{2.611640in}}%
\pgfpathmoveto{\pgfqpoint{5.377297in}{2.611640in}}%
\pgfpathlineto{\pgfqpoint{5.377297in}{2.611640in}}%
\pgfpathlineto{\pgfqpoint{5.377297in}{2.614590in}}%
\pgfpathlineto{\pgfqpoint{5.381838in}{2.614590in}}%
\pgfpathlineto{\pgfqpoint{5.381838in}{2.611640in}}%
\pgfpathmoveto{\pgfqpoint{5.381838in}{2.611640in}}%
\pgfpathlineto{\pgfqpoint{5.381838in}{2.611640in}}%
\pgfpathlineto{\pgfqpoint{5.381838in}{2.614590in}}%
\pgfpathlineto{\pgfqpoint{5.386379in}{2.614590in}}%
\pgfpathlineto{\pgfqpoint{5.386379in}{2.611640in}}%
\pgfpathmoveto{\pgfqpoint{5.386379in}{2.611640in}}%
\pgfpathlineto{\pgfqpoint{5.386379in}{2.611640in}}%
\pgfpathlineto{\pgfqpoint{5.386379in}{2.614590in}}%
\pgfpathlineto{\pgfqpoint{5.390920in}{2.614590in}}%
\pgfpathlineto{\pgfqpoint{5.390920in}{2.611640in}}%
\pgfpathmoveto{\pgfqpoint{5.390920in}{2.611640in}}%
\pgfpathlineto{\pgfqpoint{5.390920in}{2.611640in}}%
\pgfpathlineto{\pgfqpoint{5.390920in}{2.614590in}}%
\pgfpathlineto{\pgfqpoint{5.395461in}{2.614590in}}%
\pgfpathlineto{\pgfqpoint{5.395461in}{2.611640in}}%
\pgfpathmoveto{\pgfqpoint{5.395461in}{2.611640in}}%
\pgfpathlineto{\pgfqpoint{5.395461in}{2.611640in}}%
\pgfpathlineto{\pgfqpoint{5.395461in}{2.614590in}}%
\pgfpathlineto{\pgfqpoint{5.400001in}{2.614590in}}%
\pgfpathlineto{\pgfqpoint{5.400001in}{2.611640in}}%
\pgfpathclose%
\pgfusepath{fill}%
\end{pgfscope}%
\begin{pgfscope}%
\pgfpathrectangle{\pgfqpoint{0.750000in}{0.500000in}}{\pgfqpoint{4.650000in}{3.020000in}}%
\pgfusepath{clip}%
\pgfsetbuttcap%
\pgfsetmiterjoin%
\definecolor{currentfill}{rgb}{1.000000,0.000000,0.000000}%
\pgfsetfillcolor{currentfill}%
\pgfsetlinewidth{0.000000pt}%
\definecolor{currentstroke}{rgb}{0.000000,0.000000,0.000000}%
\pgfsetstrokecolor{currentstroke}%
\pgfsetstrokeopacity{0.000000}%
\pgfsetdash{}{0pt}%
\pgfpathmoveto{\pgfqpoint{2.144095in}{3.517051in}}%
\pgfpathlineto{\pgfqpoint{2.144095in}{3.520000in}}%
\pgfpathlineto{\pgfqpoint{2.148636in}{3.520000in}}%
\pgfpathlineto{\pgfqpoint{2.148636in}{3.517051in}}%
\pgfpathmoveto{\pgfqpoint{2.162259in}{3.505255in}}%
\pgfpathlineto{\pgfqpoint{2.162259in}{3.505255in}}%
\pgfpathlineto{\pgfqpoint{2.162259in}{3.508204in}}%
\pgfpathlineto{\pgfqpoint{2.166800in}{3.508204in}}%
\pgfpathlineto{\pgfqpoint{2.166800in}{3.505255in}}%
\pgfpathmoveto{\pgfqpoint{2.153177in}{3.511153in}}%
\pgfpathlineto{\pgfqpoint{2.153177in}{3.511153in}}%
\pgfpathlineto{\pgfqpoint{2.153177in}{3.514102in}}%
\pgfpathlineto{\pgfqpoint{2.157718in}{3.514102in}}%
\pgfpathlineto{\pgfqpoint{2.157718in}{3.511153in}}%
\pgfpathmoveto{\pgfqpoint{2.148636in}{3.514102in}}%
\pgfpathlineto{\pgfqpoint{2.148636in}{3.514102in}}%
\pgfpathlineto{\pgfqpoint{2.148636in}{3.517051in}}%
\pgfpathlineto{\pgfqpoint{2.153177in}{3.517051in}}%
\pgfpathlineto{\pgfqpoint{2.153177in}{3.514102in}}%
\pgfpathmoveto{\pgfqpoint{2.148636in}{3.517051in}}%
\pgfpathlineto{\pgfqpoint{2.148636in}{3.517051in}}%
\pgfpathlineto{\pgfqpoint{2.148636in}{3.520000in}}%
\pgfpathlineto{\pgfqpoint{2.153177in}{3.520000in}}%
\pgfpathlineto{\pgfqpoint{2.153177in}{3.517051in}}%
\pgfpathmoveto{\pgfqpoint{2.153177in}{3.514102in}}%
\pgfpathlineto{\pgfqpoint{2.153177in}{3.514102in}}%
\pgfpathlineto{\pgfqpoint{2.153177in}{3.517051in}}%
\pgfpathlineto{\pgfqpoint{2.157718in}{3.517051in}}%
\pgfpathlineto{\pgfqpoint{2.157718in}{3.514102in}}%
\pgfpathmoveto{\pgfqpoint{2.157718in}{3.508204in}}%
\pgfpathlineto{\pgfqpoint{2.157718in}{3.508204in}}%
\pgfpathlineto{\pgfqpoint{2.157718in}{3.511153in}}%
\pgfpathlineto{\pgfqpoint{2.162259in}{3.511153in}}%
\pgfpathlineto{\pgfqpoint{2.162259in}{3.508204in}}%
\pgfpathmoveto{\pgfqpoint{2.157718in}{3.511153in}}%
\pgfpathlineto{\pgfqpoint{2.157718in}{3.511153in}}%
\pgfpathlineto{\pgfqpoint{2.157718in}{3.514102in}}%
\pgfpathlineto{\pgfqpoint{2.162259in}{3.514102in}}%
\pgfpathlineto{\pgfqpoint{2.162259in}{3.511153in}}%
\pgfpathmoveto{\pgfqpoint{2.162259in}{3.508204in}}%
\pgfpathlineto{\pgfqpoint{2.162259in}{3.508204in}}%
\pgfpathlineto{\pgfqpoint{2.162259in}{3.511153in}}%
\pgfpathlineto{\pgfqpoint{2.166800in}{3.511153in}}%
\pgfpathlineto{\pgfqpoint{2.166800in}{3.508204in}}%
\pgfpathmoveto{\pgfqpoint{2.180423in}{3.493458in}}%
\pgfpathlineto{\pgfqpoint{2.180423in}{3.493458in}}%
\pgfpathlineto{\pgfqpoint{2.180423in}{3.496407in}}%
\pgfpathlineto{\pgfqpoint{2.184964in}{3.496407in}}%
\pgfpathlineto{\pgfqpoint{2.184964in}{3.493458in}}%
\pgfpathmoveto{\pgfqpoint{2.198588in}{3.481661in}}%
\pgfpathlineto{\pgfqpoint{2.198588in}{3.481661in}}%
\pgfpathlineto{\pgfqpoint{2.198588in}{3.484610in}}%
\pgfpathlineto{\pgfqpoint{2.203129in}{3.484610in}}%
\pgfpathlineto{\pgfqpoint{2.203129in}{3.481661in}}%
\pgfpathmoveto{\pgfqpoint{2.189505in}{3.487560in}}%
\pgfpathlineto{\pgfqpoint{2.189505in}{3.487560in}}%
\pgfpathlineto{\pgfqpoint{2.189505in}{3.490509in}}%
\pgfpathlineto{\pgfqpoint{2.194046in}{3.490509in}}%
\pgfpathlineto{\pgfqpoint{2.194046in}{3.487560in}}%
\pgfpathmoveto{\pgfqpoint{2.184964in}{3.490509in}}%
\pgfpathlineto{\pgfqpoint{2.184964in}{3.490509in}}%
\pgfpathlineto{\pgfqpoint{2.184964in}{3.493458in}}%
\pgfpathlineto{\pgfqpoint{2.189505in}{3.493458in}}%
\pgfpathlineto{\pgfqpoint{2.189505in}{3.490509in}}%
\pgfpathmoveto{\pgfqpoint{2.184964in}{3.493458in}}%
\pgfpathlineto{\pgfqpoint{2.184964in}{3.493458in}}%
\pgfpathlineto{\pgfqpoint{2.184964in}{3.496407in}}%
\pgfpathlineto{\pgfqpoint{2.189505in}{3.496407in}}%
\pgfpathlineto{\pgfqpoint{2.189505in}{3.493458in}}%
\pgfpathmoveto{\pgfqpoint{2.189505in}{3.490509in}}%
\pgfpathlineto{\pgfqpoint{2.189505in}{3.490509in}}%
\pgfpathlineto{\pgfqpoint{2.189505in}{3.493458in}}%
\pgfpathlineto{\pgfqpoint{2.194046in}{3.493458in}}%
\pgfpathlineto{\pgfqpoint{2.194046in}{3.490509in}}%
\pgfpathmoveto{\pgfqpoint{2.194046in}{3.484610in}}%
\pgfpathlineto{\pgfqpoint{2.194046in}{3.484610in}}%
\pgfpathlineto{\pgfqpoint{2.194046in}{3.487560in}}%
\pgfpathlineto{\pgfqpoint{2.198588in}{3.487560in}}%
\pgfpathlineto{\pgfqpoint{2.198588in}{3.484610in}}%
\pgfpathmoveto{\pgfqpoint{2.194046in}{3.487560in}}%
\pgfpathlineto{\pgfqpoint{2.194046in}{3.487560in}}%
\pgfpathlineto{\pgfqpoint{2.194046in}{3.490509in}}%
\pgfpathlineto{\pgfqpoint{2.198588in}{3.490509in}}%
\pgfpathlineto{\pgfqpoint{2.198588in}{3.487560in}}%
\pgfpathmoveto{\pgfqpoint{2.198588in}{3.484610in}}%
\pgfpathlineto{\pgfqpoint{2.198588in}{3.484610in}}%
\pgfpathlineto{\pgfqpoint{2.198588in}{3.487560in}}%
\pgfpathlineto{\pgfqpoint{2.203129in}{3.487560in}}%
\pgfpathlineto{\pgfqpoint{2.203129in}{3.484610in}}%
\pgfpathmoveto{\pgfqpoint{2.171341in}{3.499356in}}%
\pgfpathlineto{\pgfqpoint{2.171341in}{3.499356in}}%
\pgfpathlineto{\pgfqpoint{2.171341in}{3.502305in}}%
\pgfpathlineto{\pgfqpoint{2.175882in}{3.502305in}}%
\pgfpathlineto{\pgfqpoint{2.175882in}{3.499356in}}%
\pgfpathmoveto{\pgfqpoint{2.166800in}{3.502305in}}%
\pgfpathlineto{\pgfqpoint{2.166800in}{3.502305in}}%
\pgfpathlineto{\pgfqpoint{2.166800in}{3.505255in}}%
\pgfpathlineto{\pgfqpoint{2.171341in}{3.505255in}}%
\pgfpathlineto{\pgfqpoint{2.171341in}{3.502305in}}%
\pgfpathmoveto{\pgfqpoint{2.166800in}{3.505255in}}%
\pgfpathlineto{\pgfqpoint{2.166800in}{3.505255in}}%
\pgfpathlineto{\pgfqpoint{2.166800in}{3.508204in}}%
\pgfpathlineto{\pgfqpoint{2.171341in}{3.508204in}}%
\pgfpathlineto{\pgfqpoint{2.171341in}{3.505255in}}%
\pgfpathmoveto{\pgfqpoint{2.171341in}{3.502305in}}%
\pgfpathlineto{\pgfqpoint{2.171341in}{3.502305in}}%
\pgfpathlineto{\pgfqpoint{2.171341in}{3.505255in}}%
\pgfpathlineto{\pgfqpoint{2.175882in}{3.505255in}}%
\pgfpathlineto{\pgfqpoint{2.175882in}{3.502305in}}%
\pgfpathmoveto{\pgfqpoint{2.175882in}{3.496407in}}%
\pgfpathlineto{\pgfqpoint{2.175882in}{3.496407in}}%
\pgfpathlineto{\pgfqpoint{2.175882in}{3.499356in}}%
\pgfpathlineto{\pgfqpoint{2.180423in}{3.499356in}}%
\pgfpathlineto{\pgfqpoint{2.180423in}{3.496407in}}%
\pgfpathmoveto{\pgfqpoint{2.175882in}{3.499356in}}%
\pgfpathlineto{\pgfqpoint{2.175882in}{3.499356in}}%
\pgfpathlineto{\pgfqpoint{2.175882in}{3.502305in}}%
\pgfpathlineto{\pgfqpoint{2.180423in}{3.502305in}}%
\pgfpathlineto{\pgfqpoint{2.180423in}{3.499356in}}%
\pgfpathmoveto{\pgfqpoint{2.180423in}{3.496407in}}%
\pgfpathlineto{\pgfqpoint{2.180423in}{3.496407in}}%
\pgfpathlineto{\pgfqpoint{2.180423in}{3.499356in}}%
\pgfpathlineto{\pgfqpoint{2.184964in}{3.499356in}}%
\pgfpathlineto{\pgfqpoint{2.184964in}{3.496407in}}%
\pgfpathmoveto{\pgfqpoint{2.289405in}{3.422678in}}%
\pgfpathlineto{\pgfqpoint{2.289405in}{3.422678in}}%
\pgfpathlineto{\pgfqpoint{2.289405in}{3.425627in}}%
\pgfpathlineto{\pgfqpoint{2.293945in}{3.425627in}}%
\pgfpathlineto{\pgfqpoint{2.293945in}{3.422678in}}%
\pgfpathmoveto{\pgfqpoint{2.307568in}{3.410880in}}%
\pgfpathlineto{\pgfqpoint{2.307568in}{3.410880in}}%
\pgfpathlineto{\pgfqpoint{2.307568in}{3.413830in}}%
\pgfpathlineto{\pgfqpoint{2.312109in}{3.413830in}}%
\pgfpathlineto{\pgfqpoint{2.312109in}{3.410880in}}%
\pgfpathmoveto{\pgfqpoint{2.298486in}{3.416779in}}%
\pgfpathlineto{\pgfqpoint{2.298486in}{3.416779in}}%
\pgfpathlineto{\pgfqpoint{2.298486in}{3.419728in}}%
\pgfpathlineto{\pgfqpoint{2.303027in}{3.419728in}}%
\pgfpathlineto{\pgfqpoint{2.303027in}{3.416779in}}%
\pgfpathmoveto{\pgfqpoint{2.293945in}{3.419728in}}%
\pgfpathlineto{\pgfqpoint{2.293945in}{3.419728in}}%
\pgfpathlineto{\pgfqpoint{2.293945in}{3.422678in}}%
\pgfpathlineto{\pgfqpoint{2.298486in}{3.422678in}}%
\pgfpathlineto{\pgfqpoint{2.298486in}{3.419728in}}%
\pgfpathmoveto{\pgfqpoint{2.293945in}{3.422678in}}%
\pgfpathlineto{\pgfqpoint{2.293945in}{3.422678in}}%
\pgfpathlineto{\pgfqpoint{2.293945in}{3.425627in}}%
\pgfpathlineto{\pgfqpoint{2.298486in}{3.425627in}}%
\pgfpathlineto{\pgfqpoint{2.298486in}{3.422678in}}%
\pgfpathmoveto{\pgfqpoint{2.298486in}{3.419728in}}%
\pgfpathlineto{\pgfqpoint{2.298486in}{3.419728in}}%
\pgfpathlineto{\pgfqpoint{2.298486in}{3.422678in}}%
\pgfpathlineto{\pgfqpoint{2.303027in}{3.422678in}}%
\pgfpathlineto{\pgfqpoint{2.303027in}{3.419728in}}%
\pgfpathmoveto{\pgfqpoint{2.303027in}{3.413830in}}%
\pgfpathlineto{\pgfqpoint{2.303027in}{3.413830in}}%
\pgfpathlineto{\pgfqpoint{2.303027in}{3.416779in}}%
\pgfpathlineto{\pgfqpoint{2.307568in}{3.416779in}}%
\pgfpathlineto{\pgfqpoint{2.307568in}{3.413830in}}%
\pgfpathmoveto{\pgfqpoint{2.303027in}{3.416779in}}%
\pgfpathlineto{\pgfqpoint{2.303027in}{3.416779in}}%
\pgfpathlineto{\pgfqpoint{2.303027in}{3.419728in}}%
\pgfpathlineto{\pgfqpoint{2.307568in}{3.419728in}}%
\pgfpathlineto{\pgfqpoint{2.307568in}{3.416779in}}%
\pgfpathmoveto{\pgfqpoint{2.307568in}{3.413830in}}%
\pgfpathlineto{\pgfqpoint{2.307568in}{3.413830in}}%
\pgfpathlineto{\pgfqpoint{2.307568in}{3.416779in}}%
\pgfpathlineto{\pgfqpoint{2.312109in}{3.416779in}}%
\pgfpathlineto{\pgfqpoint{2.312109in}{3.413830in}}%
\pgfpathmoveto{\pgfqpoint{2.325731in}{3.399083in}}%
\pgfpathlineto{\pgfqpoint{2.325731in}{3.399083in}}%
\pgfpathlineto{\pgfqpoint{2.325731in}{3.402033in}}%
\pgfpathlineto{\pgfqpoint{2.330272in}{3.402033in}}%
\pgfpathlineto{\pgfqpoint{2.330272in}{3.399083in}}%
\pgfpathmoveto{\pgfqpoint{2.343895in}{3.387286in}}%
\pgfpathlineto{\pgfqpoint{2.343895in}{3.387286in}}%
\pgfpathlineto{\pgfqpoint{2.343895in}{3.390235in}}%
\pgfpathlineto{\pgfqpoint{2.348436in}{3.390235in}}%
\pgfpathlineto{\pgfqpoint{2.348436in}{3.387286in}}%
\pgfpathmoveto{\pgfqpoint{2.334813in}{3.393185in}}%
\pgfpathlineto{\pgfqpoint{2.334813in}{3.393185in}}%
\pgfpathlineto{\pgfqpoint{2.334813in}{3.396134in}}%
\pgfpathlineto{\pgfqpoint{2.339354in}{3.396134in}}%
\pgfpathlineto{\pgfqpoint{2.339354in}{3.393185in}}%
\pgfpathmoveto{\pgfqpoint{2.330272in}{3.396134in}}%
\pgfpathlineto{\pgfqpoint{2.330272in}{3.396134in}}%
\pgfpathlineto{\pgfqpoint{2.330272in}{3.399083in}}%
\pgfpathlineto{\pgfqpoint{2.334813in}{3.399083in}}%
\pgfpathlineto{\pgfqpoint{2.334813in}{3.396134in}}%
\pgfpathmoveto{\pgfqpoint{2.330272in}{3.399083in}}%
\pgfpathlineto{\pgfqpoint{2.330272in}{3.399083in}}%
\pgfpathlineto{\pgfqpoint{2.330272in}{3.402033in}}%
\pgfpathlineto{\pgfqpoint{2.334813in}{3.402033in}}%
\pgfpathlineto{\pgfqpoint{2.334813in}{3.399083in}}%
\pgfpathmoveto{\pgfqpoint{2.334813in}{3.396134in}}%
\pgfpathlineto{\pgfqpoint{2.334813in}{3.396134in}}%
\pgfpathlineto{\pgfqpoint{2.334813in}{3.399083in}}%
\pgfpathlineto{\pgfqpoint{2.339354in}{3.399083in}}%
\pgfpathlineto{\pgfqpoint{2.339354in}{3.396134in}}%
\pgfpathmoveto{\pgfqpoint{2.339354in}{3.390235in}}%
\pgfpathlineto{\pgfqpoint{2.339354in}{3.390235in}}%
\pgfpathlineto{\pgfqpoint{2.339354in}{3.393185in}}%
\pgfpathlineto{\pgfqpoint{2.343895in}{3.393185in}}%
\pgfpathlineto{\pgfqpoint{2.343895in}{3.390235in}}%
\pgfpathmoveto{\pgfqpoint{2.339354in}{3.393185in}}%
\pgfpathlineto{\pgfqpoint{2.339354in}{3.393185in}}%
\pgfpathlineto{\pgfqpoint{2.339354in}{3.396134in}}%
\pgfpathlineto{\pgfqpoint{2.343895in}{3.396134in}}%
\pgfpathlineto{\pgfqpoint{2.343895in}{3.393185in}}%
\pgfpathmoveto{\pgfqpoint{2.343895in}{3.390235in}}%
\pgfpathlineto{\pgfqpoint{2.343895in}{3.390235in}}%
\pgfpathlineto{\pgfqpoint{2.343895in}{3.393185in}}%
\pgfpathlineto{\pgfqpoint{2.348436in}{3.393185in}}%
\pgfpathlineto{\pgfqpoint{2.348436in}{3.390235in}}%
\pgfpathmoveto{\pgfqpoint{2.316650in}{3.404982in}}%
\pgfpathlineto{\pgfqpoint{2.316650in}{3.404982in}}%
\pgfpathlineto{\pgfqpoint{2.316650in}{3.407931in}}%
\pgfpathlineto{\pgfqpoint{2.321191in}{3.407931in}}%
\pgfpathlineto{\pgfqpoint{2.321191in}{3.404982in}}%
\pgfpathmoveto{\pgfqpoint{2.312109in}{3.407931in}}%
\pgfpathlineto{\pgfqpoint{2.312109in}{3.407931in}}%
\pgfpathlineto{\pgfqpoint{2.312109in}{3.410880in}}%
\pgfpathlineto{\pgfqpoint{2.316650in}{3.410880in}}%
\pgfpathlineto{\pgfqpoint{2.316650in}{3.407931in}}%
\pgfpathmoveto{\pgfqpoint{2.312109in}{3.410880in}}%
\pgfpathlineto{\pgfqpoint{2.312109in}{3.410880in}}%
\pgfpathlineto{\pgfqpoint{2.312109in}{3.413830in}}%
\pgfpathlineto{\pgfqpoint{2.316650in}{3.413830in}}%
\pgfpathlineto{\pgfqpoint{2.316650in}{3.410880in}}%
\pgfpathmoveto{\pgfqpoint{2.316650in}{3.407931in}}%
\pgfpathlineto{\pgfqpoint{2.316650in}{3.407931in}}%
\pgfpathlineto{\pgfqpoint{2.316650in}{3.410880in}}%
\pgfpathlineto{\pgfqpoint{2.321191in}{3.410880in}}%
\pgfpathlineto{\pgfqpoint{2.321191in}{3.407931in}}%
\pgfpathmoveto{\pgfqpoint{2.321191in}{3.402033in}}%
\pgfpathlineto{\pgfqpoint{2.321191in}{3.402033in}}%
\pgfpathlineto{\pgfqpoint{2.321191in}{3.404982in}}%
\pgfpathlineto{\pgfqpoint{2.325731in}{3.404982in}}%
\pgfpathlineto{\pgfqpoint{2.325731in}{3.402033in}}%
\pgfpathmoveto{\pgfqpoint{2.321191in}{3.404982in}}%
\pgfpathlineto{\pgfqpoint{2.321191in}{3.404982in}}%
\pgfpathlineto{\pgfqpoint{2.321191in}{3.407931in}}%
\pgfpathlineto{\pgfqpoint{2.325731in}{3.407931in}}%
\pgfpathlineto{\pgfqpoint{2.325731in}{3.404982in}}%
\pgfpathmoveto{\pgfqpoint{2.325731in}{3.402033in}}%
\pgfpathlineto{\pgfqpoint{2.325731in}{3.402033in}}%
\pgfpathlineto{\pgfqpoint{2.325731in}{3.404982in}}%
\pgfpathlineto{\pgfqpoint{2.330272in}{3.404982in}}%
\pgfpathlineto{\pgfqpoint{2.330272in}{3.402033in}}%
\pgfpathmoveto{\pgfqpoint{2.216751in}{3.469865in}}%
\pgfpathlineto{\pgfqpoint{2.216751in}{3.469865in}}%
\pgfpathlineto{\pgfqpoint{2.216751in}{3.472814in}}%
\pgfpathlineto{\pgfqpoint{2.221292in}{3.472814in}}%
\pgfpathlineto{\pgfqpoint{2.221292in}{3.469865in}}%
\pgfpathmoveto{\pgfqpoint{2.234914in}{3.458068in}}%
\pgfpathlineto{\pgfqpoint{2.234914in}{3.458068in}}%
\pgfpathlineto{\pgfqpoint{2.234914in}{3.461017in}}%
\pgfpathlineto{\pgfqpoint{2.239455in}{3.461017in}}%
\pgfpathlineto{\pgfqpoint{2.239455in}{3.458068in}}%
\pgfpathmoveto{\pgfqpoint{2.225833in}{3.463966in}}%
\pgfpathlineto{\pgfqpoint{2.225833in}{3.463966in}}%
\pgfpathlineto{\pgfqpoint{2.225833in}{3.466915in}}%
\pgfpathlineto{\pgfqpoint{2.230374in}{3.466915in}}%
\pgfpathlineto{\pgfqpoint{2.230374in}{3.463966in}}%
\pgfpathmoveto{\pgfqpoint{2.221292in}{3.466915in}}%
\pgfpathlineto{\pgfqpoint{2.221292in}{3.466915in}}%
\pgfpathlineto{\pgfqpoint{2.221292in}{3.469865in}}%
\pgfpathlineto{\pgfqpoint{2.225833in}{3.469865in}}%
\pgfpathlineto{\pgfqpoint{2.225833in}{3.466915in}}%
\pgfpathmoveto{\pgfqpoint{2.221292in}{3.469865in}}%
\pgfpathlineto{\pgfqpoint{2.221292in}{3.469865in}}%
\pgfpathlineto{\pgfqpoint{2.221292in}{3.472814in}}%
\pgfpathlineto{\pgfqpoint{2.225833in}{3.472814in}}%
\pgfpathlineto{\pgfqpoint{2.225833in}{3.469865in}}%
\pgfpathmoveto{\pgfqpoint{2.225833in}{3.466915in}}%
\pgfpathlineto{\pgfqpoint{2.225833in}{3.466915in}}%
\pgfpathlineto{\pgfqpoint{2.225833in}{3.469865in}}%
\pgfpathlineto{\pgfqpoint{2.230374in}{3.469865in}}%
\pgfpathlineto{\pgfqpoint{2.230374in}{3.466915in}}%
\pgfpathmoveto{\pgfqpoint{2.230374in}{3.461017in}}%
\pgfpathlineto{\pgfqpoint{2.230374in}{3.461017in}}%
\pgfpathlineto{\pgfqpoint{2.230374in}{3.463966in}}%
\pgfpathlineto{\pgfqpoint{2.234914in}{3.463966in}}%
\pgfpathlineto{\pgfqpoint{2.234914in}{3.461017in}}%
\pgfpathmoveto{\pgfqpoint{2.230374in}{3.463966in}}%
\pgfpathlineto{\pgfqpoint{2.230374in}{3.463966in}}%
\pgfpathlineto{\pgfqpoint{2.230374in}{3.466915in}}%
\pgfpathlineto{\pgfqpoint{2.234914in}{3.466915in}}%
\pgfpathlineto{\pgfqpoint{2.234914in}{3.463966in}}%
\pgfpathmoveto{\pgfqpoint{2.234914in}{3.461017in}}%
\pgfpathlineto{\pgfqpoint{2.234914in}{3.461017in}}%
\pgfpathlineto{\pgfqpoint{2.234914in}{3.463966in}}%
\pgfpathlineto{\pgfqpoint{2.239455in}{3.463966in}}%
\pgfpathlineto{\pgfqpoint{2.239455in}{3.461017in}}%
\pgfpathmoveto{\pgfqpoint{2.253078in}{3.446271in}}%
\pgfpathlineto{\pgfqpoint{2.253078in}{3.446271in}}%
\pgfpathlineto{\pgfqpoint{2.253078in}{3.449220in}}%
\pgfpathlineto{\pgfqpoint{2.257619in}{3.449220in}}%
\pgfpathlineto{\pgfqpoint{2.257619in}{3.446271in}}%
\pgfpathmoveto{\pgfqpoint{2.271241in}{3.434474in}}%
\pgfpathlineto{\pgfqpoint{2.271241in}{3.434474in}}%
\pgfpathlineto{\pgfqpoint{2.271241in}{3.437424in}}%
\pgfpathlineto{\pgfqpoint{2.275782in}{3.437424in}}%
\pgfpathlineto{\pgfqpoint{2.275782in}{3.434474in}}%
\pgfpathmoveto{\pgfqpoint{2.262160in}{3.440373in}}%
\pgfpathlineto{\pgfqpoint{2.262160in}{3.440373in}}%
\pgfpathlineto{\pgfqpoint{2.262160in}{3.443322in}}%
\pgfpathlineto{\pgfqpoint{2.266700in}{3.443322in}}%
\pgfpathlineto{\pgfqpoint{2.266700in}{3.440373in}}%
\pgfpathmoveto{\pgfqpoint{2.257619in}{3.443322in}}%
\pgfpathlineto{\pgfqpoint{2.257619in}{3.443322in}}%
\pgfpathlineto{\pgfqpoint{2.257619in}{3.446271in}}%
\pgfpathlineto{\pgfqpoint{2.262160in}{3.446271in}}%
\pgfpathlineto{\pgfqpoint{2.262160in}{3.443322in}}%
\pgfpathmoveto{\pgfqpoint{2.257619in}{3.446271in}}%
\pgfpathlineto{\pgfqpoint{2.257619in}{3.446271in}}%
\pgfpathlineto{\pgfqpoint{2.257619in}{3.449220in}}%
\pgfpathlineto{\pgfqpoint{2.262160in}{3.449220in}}%
\pgfpathlineto{\pgfqpoint{2.262160in}{3.446271in}}%
\pgfpathmoveto{\pgfqpoint{2.262160in}{3.443322in}}%
\pgfpathlineto{\pgfqpoint{2.262160in}{3.443322in}}%
\pgfpathlineto{\pgfqpoint{2.262160in}{3.446271in}}%
\pgfpathlineto{\pgfqpoint{2.266700in}{3.446271in}}%
\pgfpathlineto{\pgfqpoint{2.266700in}{3.443322in}}%
\pgfpathmoveto{\pgfqpoint{2.266700in}{3.437424in}}%
\pgfpathlineto{\pgfqpoint{2.266700in}{3.437424in}}%
\pgfpathlineto{\pgfqpoint{2.266700in}{3.440373in}}%
\pgfpathlineto{\pgfqpoint{2.271241in}{3.440373in}}%
\pgfpathlineto{\pgfqpoint{2.271241in}{3.437424in}}%
\pgfpathmoveto{\pgfqpoint{2.266700in}{3.440373in}}%
\pgfpathlineto{\pgfqpoint{2.266700in}{3.440373in}}%
\pgfpathlineto{\pgfqpoint{2.266700in}{3.443322in}}%
\pgfpathlineto{\pgfqpoint{2.271241in}{3.443322in}}%
\pgfpathlineto{\pgfqpoint{2.271241in}{3.440373in}}%
\pgfpathmoveto{\pgfqpoint{2.271241in}{3.437424in}}%
\pgfpathlineto{\pgfqpoint{2.271241in}{3.437424in}}%
\pgfpathlineto{\pgfqpoint{2.271241in}{3.440373in}}%
\pgfpathlineto{\pgfqpoint{2.275782in}{3.440373in}}%
\pgfpathlineto{\pgfqpoint{2.275782in}{3.437424in}}%
\pgfpathmoveto{\pgfqpoint{2.243996in}{3.452169in}}%
\pgfpathlineto{\pgfqpoint{2.243996in}{3.452169in}}%
\pgfpathlineto{\pgfqpoint{2.243996in}{3.455119in}}%
\pgfpathlineto{\pgfqpoint{2.248537in}{3.455119in}}%
\pgfpathlineto{\pgfqpoint{2.248537in}{3.452169in}}%
\pgfpathmoveto{\pgfqpoint{2.239455in}{3.455119in}}%
\pgfpathlineto{\pgfqpoint{2.239455in}{3.455119in}}%
\pgfpathlineto{\pgfqpoint{2.239455in}{3.458068in}}%
\pgfpathlineto{\pgfqpoint{2.243996in}{3.458068in}}%
\pgfpathlineto{\pgfqpoint{2.243996in}{3.455119in}}%
\pgfpathmoveto{\pgfqpoint{2.239455in}{3.458068in}}%
\pgfpathlineto{\pgfqpoint{2.239455in}{3.458068in}}%
\pgfpathlineto{\pgfqpoint{2.239455in}{3.461017in}}%
\pgfpathlineto{\pgfqpoint{2.243996in}{3.461017in}}%
\pgfpathlineto{\pgfqpoint{2.243996in}{3.458068in}}%
\pgfpathmoveto{\pgfqpoint{2.243996in}{3.455119in}}%
\pgfpathlineto{\pgfqpoint{2.243996in}{3.455119in}}%
\pgfpathlineto{\pgfqpoint{2.243996in}{3.458068in}}%
\pgfpathlineto{\pgfqpoint{2.248537in}{3.458068in}}%
\pgfpathlineto{\pgfqpoint{2.248537in}{3.455119in}}%
\pgfpathmoveto{\pgfqpoint{2.248537in}{3.449220in}}%
\pgfpathlineto{\pgfqpoint{2.248537in}{3.449220in}}%
\pgfpathlineto{\pgfqpoint{2.248537in}{3.452169in}}%
\pgfpathlineto{\pgfqpoint{2.253078in}{3.452169in}}%
\pgfpathlineto{\pgfqpoint{2.253078in}{3.449220in}}%
\pgfpathmoveto{\pgfqpoint{2.248537in}{3.452169in}}%
\pgfpathlineto{\pgfqpoint{2.248537in}{3.452169in}}%
\pgfpathlineto{\pgfqpoint{2.248537in}{3.455119in}}%
\pgfpathlineto{\pgfqpoint{2.253078in}{3.455119in}}%
\pgfpathlineto{\pgfqpoint{2.253078in}{3.452169in}}%
\pgfpathmoveto{\pgfqpoint{2.253078in}{3.449220in}}%
\pgfpathlineto{\pgfqpoint{2.253078in}{3.449220in}}%
\pgfpathlineto{\pgfqpoint{2.253078in}{3.452169in}}%
\pgfpathlineto{\pgfqpoint{2.257619in}{3.452169in}}%
\pgfpathlineto{\pgfqpoint{2.257619in}{3.449220in}}%
\pgfpathmoveto{\pgfqpoint{2.207669in}{3.475763in}}%
\pgfpathlineto{\pgfqpoint{2.207669in}{3.475763in}}%
\pgfpathlineto{\pgfqpoint{2.207669in}{3.478712in}}%
\pgfpathlineto{\pgfqpoint{2.212210in}{3.478712in}}%
\pgfpathlineto{\pgfqpoint{2.212210in}{3.475763in}}%
\pgfpathmoveto{\pgfqpoint{2.203129in}{3.478712in}}%
\pgfpathlineto{\pgfqpoint{2.203129in}{3.478712in}}%
\pgfpathlineto{\pgfqpoint{2.203129in}{3.481661in}}%
\pgfpathlineto{\pgfqpoint{2.207669in}{3.481661in}}%
\pgfpathlineto{\pgfqpoint{2.207669in}{3.478712in}}%
\pgfpathmoveto{\pgfqpoint{2.203129in}{3.481661in}}%
\pgfpathlineto{\pgfqpoint{2.203129in}{3.481661in}}%
\pgfpathlineto{\pgfqpoint{2.203129in}{3.484610in}}%
\pgfpathlineto{\pgfqpoint{2.207669in}{3.484610in}}%
\pgfpathlineto{\pgfqpoint{2.207669in}{3.481661in}}%
\pgfpathmoveto{\pgfqpoint{2.207669in}{3.478712in}}%
\pgfpathlineto{\pgfqpoint{2.207669in}{3.478712in}}%
\pgfpathlineto{\pgfqpoint{2.207669in}{3.481661in}}%
\pgfpathlineto{\pgfqpoint{2.212210in}{3.481661in}}%
\pgfpathlineto{\pgfqpoint{2.212210in}{3.478712in}}%
\pgfpathmoveto{\pgfqpoint{2.212210in}{3.472814in}}%
\pgfpathlineto{\pgfqpoint{2.212210in}{3.472814in}}%
\pgfpathlineto{\pgfqpoint{2.212210in}{3.475763in}}%
\pgfpathlineto{\pgfqpoint{2.216751in}{3.475763in}}%
\pgfpathlineto{\pgfqpoint{2.216751in}{3.472814in}}%
\pgfpathmoveto{\pgfqpoint{2.212210in}{3.475763in}}%
\pgfpathlineto{\pgfqpoint{2.212210in}{3.475763in}}%
\pgfpathlineto{\pgfqpoint{2.212210in}{3.478712in}}%
\pgfpathlineto{\pgfqpoint{2.216751in}{3.478712in}}%
\pgfpathlineto{\pgfqpoint{2.216751in}{3.475763in}}%
\pgfpathmoveto{\pgfqpoint{2.216751in}{3.472814in}}%
\pgfpathlineto{\pgfqpoint{2.216751in}{3.472814in}}%
\pgfpathlineto{\pgfqpoint{2.216751in}{3.475763in}}%
\pgfpathlineto{\pgfqpoint{2.221292in}{3.475763in}}%
\pgfpathlineto{\pgfqpoint{2.221292in}{3.472814in}}%
\pgfpathmoveto{\pgfqpoint{2.280323in}{3.428576in}}%
\pgfpathlineto{\pgfqpoint{2.280323in}{3.428576in}}%
\pgfpathlineto{\pgfqpoint{2.280323in}{3.431525in}}%
\pgfpathlineto{\pgfqpoint{2.284864in}{3.431525in}}%
\pgfpathlineto{\pgfqpoint{2.284864in}{3.428576in}}%
\pgfpathmoveto{\pgfqpoint{2.275782in}{3.431525in}}%
\pgfpathlineto{\pgfqpoint{2.275782in}{3.431525in}}%
\pgfpathlineto{\pgfqpoint{2.275782in}{3.434474in}}%
\pgfpathlineto{\pgfqpoint{2.280323in}{3.434474in}}%
\pgfpathlineto{\pgfqpoint{2.280323in}{3.431525in}}%
\pgfpathmoveto{\pgfqpoint{2.275782in}{3.434474in}}%
\pgfpathlineto{\pgfqpoint{2.275782in}{3.434474in}}%
\pgfpathlineto{\pgfqpoint{2.275782in}{3.437424in}}%
\pgfpathlineto{\pgfqpoint{2.280323in}{3.437424in}}%
\pgfpathlineto{\pgfqpoint{2.280323in}{3.434474in}}%
\pgfpathmoveto{\pgfqpoint{2.280323in}{3.431525in}}%
\pgfpathlineto{\pgfqpoint{2.280323in}{3.431525in}}%
\pgfpathlineto{\pgfqpoint{2.280323in}{3.434474in}}%
\pgfpathlineto{\pgfqpoint{2.284864in}{3.434474in}}%
\pgfpathlineto{\pgfqpoint{2.284864in}{3.431525in}}%
\pgfpathmoveto{\pgfqpoint{2.284864in}{3.425627in}}%
\pgfpathlineto{\pgfqpoint{2.284864in}{3.425627in}}%
\pgfpathlineto{\pgfqpoint{2.284864in}{3.428576in}}%
\pgfpathlineto{\pgfqpoint{2.289405in}{3.428576in}}%
\pgfpathlineto{\pgfqpoint{2.289405in}{3.425627in}}%
\pgfpathmoveto{\pgfqpoint{2.284864in}{3.428576in}}%
\pgfpathlineto{\pgfqpoint{2.284864in}{3.428576in}}%
\pgfpathlineto{\pgfqpoint{2.284864in}{3.431525in}}%
\pgfpathlineto{\pgfqpoint{2.289405in}{3.431525in}}%
\pgfpathlineto{\pgfqpoint{2.289405in}{3.428576in}}%
\pgfpathmoveto{\pgfqpoint{2.289405in}{3.425627in}}%
\pgfpathlineto{\pgfqpoint{2.289405in}{3.425627in}}%
\pgfpathlineto{\pgfqpoint{2.289405in}{3.428576in}}%
\pgfpathlineto{\pgfqpoint{2.293945in}{3.428576in}}%
\pgfpathlineto{\pgfqpoint{2.293945in}{3.425627in}}%
\pgfpathmoveto{\pgfqpoint{2.434716in}{3.328300in}}%
\pgfpathlineto{\pgfqpoint{2.434716in}{3.328300in}}%
\pgfpathlineto{\pgfqpoint{2.434716in}{3.331250in}}%
\pgfpathlineto{\pgfqpoint{2.439257in}{3.331250in}}%
\pgfpathlineto{\pgfqpoint{2.439257in}{3.328300in}}%
\pgfpathmoveto{\pgfqpoint{2.452881in}{3.316504in}}%
\pgfpathlineto{\pgfqpoint{2.452881in}{3.316504in}}%
\pgfpathlineto{\pgfqpoint{2.452881in}{3.319453in}}%
\pgfpathlineto{\pgfqpoint{2.457422in}{3.319453in}}%
\pgfpathlineto{\pgfqpoint{2.457422in}{3.316504in}}%
\pgfpathmoveto{\pgfqpoint{2.443798in}{3.322402in}}%
\pgfpathlineto{\pgfqpoint{2.443798in}{3.322402in}}%
\pgfpathlineto{\pgfqpoint{2.443798in}{3.325351in}}%
\pgfpathlineto{\pgfqpoint{2.448340in}{3.325351in}}%
\pgfpathlineto{\pgfqpoint{2.448340in}{3.322402in}}%
\pgfpathmoveto{\pgfqpoint{2.439257in}{3.325351in}}%
\pgfpathlineto{\pgfqpoint{2.439257in}{3.325351in}}%
\pgfpathlineto{\pgfqpoint{2.439257in}{3.328300in}}%
\pgfpathlineto{\pgfqpoint{2.443798in}{3.328300in}}%
\pgfpathlineto{\pgfqpoint{2.443798in}{3.325351in}}%
\pgfpathmoveto{\pgfqpoint{2.439257in}{3.328300in}}%
\pgfpathlineto{\pgfqpoint{2.439257in}{3.328300in}}%
\pgfpathlineto{\pgfqpoint{2.439257in}{3.331250in}}%
\pgfpathlineto{\pgfqpoint{2.443798in}{3.331250in}}%
\pgfpathlineto{\pgfqpoint{2.443798in}{3.328300in}}%
\pgfpathmoveto{\pgfqpoint{2.443798in}{3.325351in}}%
\pgfpathlineto{\pgfqpoint{2.443798in}{3.325351in}}%
\pgfpathlineto{\pgfqpoint{2.443798in}{3.328300in}}%
\pgfpathlineto{\pgfqpoint{2.448340in}{3.328300in}}%
\pgfpathlineto{\pgfqpoint{2.448340in}{3.325351in}}%
\pgfpathmoveto{\pgfqpoint{2.448340in}{3.319453in}}%
\pgfpathlineto{\pgfqpoint{2.448340in}{3.319453in}}%
\pgfpathlineto{\pgfqpoint{2.448340in}{3.322402in}}%
\pgfpathlineto{\pgfqpoint{2.452881in}{3.322402in}}%
\pgfpathlineto{\pgfqpoint{2.452881in}{3.319453in}}%
\pgfpathmoveto{\pgfqpoint{2.448340in}{3.322402in}}%
\pgfpathlineto{\pgfqpoint{2.448340in}{3.322402in}}%
\pgfpathlineto{\pgfqpoint{2.448340in}{3.325351in}}%
\pgfpathlineto{\pgfqpoint{2.452881in}{3.325351in}}%
\pgfpathlineto{\pgfqpoint{2.452881in}{3.322402in}}%
\pgfpathmoveto{\pgfqpoint{2.452881in}{3.319453in}}%
\pgfpathlineto{\pgfqpoint{2.452881in}{3.319453in}}%
\pgfpathlineto{\pgfqpoint{2.452881in}{3.322402in}}%
\pgfpathlineto{\pgfqpoint{2.457422in}{3.322402in}}%
\pgfpathlineto{\pgfqpoint{2.457422in}{3.319453in}}%
\pgfpathmoveto{\pgfqpoint{2.471045in}{3.304707in}}%
\pgfpathlineto{\pgfqpoint{2.471045in}{3.304707in}}%
\pgfpathlineto{\pgfqpoint{2.471045in}{3.307657in}}%
\pgfpathlineto{\pgfqpoint{2.475586in}{3.307657in}}%
\pgfpathlineto{\pgfqpoint{2.475586in}{3.304707in}}%
\pgfpathmoveto{\pgfqpoint{2.489209in}{3.292911in}}%
\pgfpathlineto{\pgfqpoint{2.489209in}{3.292911in}}%
\pgfpathlineto{\pgfqpoint{2.489209in}{3.295860in}}%
\pgfpathlineto{\pgfqpoint{2.493750in}{3.295860in}}%
\pgfpathlineto{\pgfqpoint{2.493750in}{3.292911in}}%
\pgfpathmoveto{\pgfqpoint{2.480127in}{3.298809in}}%
\pgfpathlineto{\pgfqpoint{2.480127in}{3.298809in}}%
\pgfpathlineto{\pgfqpoint{2.480127in}{3.301758in}}%
\pgfpathlineto{\pgfqpoint{2.484668in}{3.301758in}}%
\pgfpathlineto{\pgfqpoint{2.484668in}{3.298809in}}%
\pgfpathmoveto{\pgfqpoint{2.475586in}{3.301758in}}%
\pgfpathlineto{\pgfqpoint{2.475586in}{3.301758in}}%
\pgfpathlineto{\pgfqpoint{2.475586in}{3.304707in}}%
\pgfpathlineto{\pgfqpoint{2.480127in}{3.304707in}}%
\pgfpathlineto{\pgfqpoint{2.480127in}{3.301758in}}%
\pgfpathmoveto{\pgfqpoint{2.475586in}{3.304707in}}%
\pgfpathlineto{\pgfqpoint{2.475586in}{3.304707in}}%
\pgfpathlineto{\pgfqpoint{2.475586in}{3.307657in}}%
\pgfpathlineto{\pgfqpoint{2.480127in}{3.307657in}}%
\pgfpathlineto{\pgfqpoint{2.480127in}{3.304707in}}%
\pgfpathmoveto{\pgfqpoint{2.480127in}{3.301758in}}%
\pgfpathlineto{\pgfqpoint{2.480127in}{3.301758in}}%
\pgfpathlineto{\pgfqpoint{2.480127in}{3.304707in}}%
\pgfpathlineto{\pgfqpoint{2.484668in}{3.304707in}}%
\pgfpathlineto{\pgfqpoint{2.484668in}{3.301758in}}%
\pgfpathmoveto{\pgfqpoint{2.484668in}{3.295860in}}%
\pgfpathlineto{\pgfqpoint{2.484668in}{3.295860in}}%
\pgfpathlineto{\pgfqpoint{2.484668in}{3.298809in}}%
\pgfpathlineto{\pgfqpoint{2.489209in}{3.298809in}}%
\pgfpathlineto{\pgfqpoint{2.489209in}{3.295860in}}%
\pgfpathmoveto{\pgfqpoint{2.484668in}{3.298809in}}%
\pgfpathlineto{\pgfqpoint{2.484668in}{3.298809in}}%
\pgfpathlineto{\pgfqpoint{2.484668in}{3.301758in}}%
\pgfpathlineto{\pgfqpoint{2.489209in}{3.301758in}}%
\pgfpathlineto{\pgfqpoint{2.489209in}{3.298809in}}%
\pgfpathmoveto{\pgfqpoint{2.489209in}{3.295860in}}%
\pgfpathlineto{\pgfqpoint{2.489209in}{3.295860in}}%
\pgfpathlineto{\pgfqpoint{2.489209in}{3.298809in}}%
\pgfpathlineto{\pgfqpoint{2.493750in}{3.298809in}}%
\pgfpathlineto{\pgfqpoint{2.493750in}{3.295860in}}%
\pgfpathmoveto{\pgfqpoint{2.461963in}{3.310606in}}%
\pgfpathlineto{\pgfqpoint{2.461963in}{3.310606in}}%
\pgfpathlineto{\pgfqpoint{2.461963in}{3.313555in}}%
\pgfpathlineto{\pgfqpoint{2.466504in}{3.313555in}}%
\pgfpathlineto{\pgfqpoint{2.466504in}{3.310606in}}%
\pgfpathmoveto{\pgfqpoint{2.457422in}{3.313555in}}%
\pgfpathlineto{\pgfqpoint{2.457422in}{3.313555in}}%
\pgfpathlineto{\pgfqpoint{2.457422in}{3.316504in}}%
\pgfpathlineto{\pgfqpoint{2.461963in}{3.316504in}}%
\pgfpathlineto{\pgfqpoint{2.461963in}{3.313555in}}%
\pgfpathmoveto{\pgfqpoint{2.457422in}{3.316504in}}%
\pgfpathlineto{\pgfqpoint{2.457422in}{3.316504in}}%
\pgfpathlineto{\pgfqpoint{2.457422in}{3.319453in}}%
\pgfpathlineto{\pgfqpoint{2.461963in}{3.319453in}}%
\pgfpathlineto{\pgfqpoint{2.461963in}{3.316504in}}%
\pgfpathmoveto{\pgfqpoint{2.461963in}{3.313555in}}%
\pgfpathlineto{\pgfqpoint{2.461963in}{3.313555in}}%
\pgfpathlineto{\pgfqpoint{2.461963in}{3.316504in}}%
\pgfpathlineto{\pgfqpoint{2.466504in}{3.316504in}}%
\pgfpathlineto{\pgfqpoint{2.466504in}{3.313555in}}%
\pgfpathmoveto{\pgfqpoint{2.466504in}{3.307657in}}%
\pgfpathlineto{\pgfqpoint{2.466504in}{3.307657in}}%
\pgfpathlineto{\pgfqpoint{2.466504in}{3.310606in}}%
\pgfpathlineto{\pgfqpoint{2.471045in}{3.310606in}}%
\pgfpathlineto{\pgfqpoint{2.471045in}{3.307657in}}%
\pgfpathmoveto{\pgfqpoint{2.466504in}{3.310606in}}%
\pgfpathlineto{\pgfqpoint{2.466504in}{3.310606in}}%
\pgfpathlineto{\pgfqpoint{2.466504in}{3.313555in}}%
\pgfpathlineto{\pgfqpoint{2.471045in}{3.313555in}}%
\pgfpathlineto{\pgfqpoint{2.471045in}{3.310606in}}%
\pgfpathmoveto{\pgfqpoint{2.471045in}{3.307657in}}%
\pgfpathlineto{\pgfqpoint{2.471045in}{3.307657in}}%
\pgfpathlineto{\pgfqpoint{2.471045in}{3.310606in}}%
\pgfpathlineto{\pgfqpoint{2.475586in}{3.310606in}}%
\pgfpathlineto{\pgfqpoint{2.475586in}{3.307657in}}%
\pgfpathmoveto{\pgfqpoint{2.362059in}{3.375489in}}%
\pgfpathlineto{\pgfqpoint{2.362059in}{3.375489in}}%
\pgfpathlineto{\pgfqpoint{2.362059in}{3.378438in}}%
\pgfpathlineto{\pgfqpoint{2.366600in}{3.378438in}}%
\pgfpathlineto{\pgfqpoint{2.366600in}{3.375489in}}%
\pgfpathmoveto{\pgfqpoint{2.380223in}{3.363692in}}%
\pgfpathlineto{\pgfqpoint{2.380223in}{3.363692in}}%
\pgfpathlineto{\pgfqpoint{2.380223in}{3.366641in}}%
\pgfpathlineto{\pgfqpoint{2.384764in}{3.366641in}}%
\pgfpathlineto{\pgfqpoint{2.384764in}{3.363692in}}%
\pgfpathmoveto{\pgfqpoint{2.371141in}{3.369590in}}%
\pgfpathlineto{\pgfqpoint{2.371141in}{3.369590in}}%
\pgfpathlineto{\pgfqpoint{2.371141in}{3.372540in}}%
\pgfpathlineto{\pgfqpoint{2.375682in}{3.372540in}}%
\pgfpathlineto{\pgfqpoint{2.375682in}{3.369590in}}%
\pgfpathmoveto{\pgfqpoint{2.366600in}{3.372540in}}%
\pgfpathlineto{\pgfqpoint{2.366600in}{3.372540in}}%
\pgfpathlineto{\pgfqpoint{2.366600in}{3.375489in}}%
\pgfpathlineto{\pgfqpoint{2.371141in}{3.375489in}}%
\pgfpathlineto{\pgfqpoint{2.371141in}{3.372540in}}%
\pgfpathmoveto{\pgfqpoint{2.366600in}{3.375489in}}%
\pgfpathlineto{\pgfqpoint{2.366600in}{3.375489in}}%
\pgfpathlineto{\pgfqpoint{2.366600in}{3.378438in}}%
\pgfpathlineto{\pgfqpoint{2.371141in}{3.378438in}}%
\pgfpathlineto{\pgfqpoint{2.371141in}{3.375489in}}%
\pgfpathmoveto{\pgfqpoint{2.371141in}{3.372540in}}%
\pgfpathlineto{\pgfqpoint{2.371141in}{3.372540in}}%
\pgfpathlineto{\pgfqpoint{2.371141in}{3.375489in}}%
\pgfpathlineto{\pgfqpoint{2.375682in}{3.375489in}}%
\pgfpathlineto{\pgfqpoint{2.375682in}{3.372540in}}%
\pgfpathmoveto{\pgfqpoint{2.375682in}{3.366641in}}%
\pgfpathlineto{\pgfqpoint{2.375682in}{3.366641in}}%
\pgfpathlineto{\pgfqpoint{2.375682in}{3.369590in}}%
\pgfpathlineto{\pgfqpoint{2.380223in}{3.369590in}}%
\pgfpathlineto{\pgfqpoint{2.380223in}{3.366641in}}%
\pgfpathmoveto{\pgfqpoint{2.375682in}{3.369590in}}%
\pgfpathlineto{\pgfqpoint{2.375682in}{3.369590in}}%
\pgfpathlineto{\pgfqpoint{2.375682in}{3.372540in}}%
\pgfpathlineto{\pgfqpoint{2.380223in}{3.372540in}}%
\pgfpathlineto{\pgfqpoint{2.380223in}{3.369590in}}%
\pgfpathmoveto{\pgfqpoint{2.380223in}{3.366641in}}%
\pgfpathlineto{\pgfqpoint{2.380223in}{3.366641in}}%
\pgfpathlineto{\pgfqpoint{2.380223in}{3.369590in}}%
\pgfpathlineto{\pgfqpoint{2.384764in}{3.369590in}}%
\pgfpathlineto{\pgfqpoint{2.384764in}{3.366641in}}%
\pgfpathmoveto{\pgfqpoint{2.398388in}{3.351895in}}%
\pgfpathlineto{\pgfqpoint{2.398388in}{3.351895in}}%
\pgfpathlineto{\pgfqpoint{2.398388in}{3.354844in}}%
\pgfpathlineto{\pgfqpoint{2.402929in}{3.354844in}}%
\pgfpathlineto{\pgfqpoint{2.402929in}{3.351895in}}%
\pgfpathmoveto{\pgfqpoint{2.416552in}{3.340097in}}%
\pgfpathlineto{\pgfqpoint{2.416552in}{3.340097in}}%
\pgfpathlineto{\pgfqpoint{2.416552in}{3.343047in}}%
\pgfpathlineto{\pgfqpoint{2.421093in}{3.343047in}}%
\pgfpathlineto{\pgfqpoint{2.421093in}{3.340097in}}%
\pgfpathmoveto{\pgfqpoint{2.407470in}{3.345996in}}%
\pgfpathlineto{\pgfqpoint{2.407470in}{3.345996in}}%
\pgfpathlineto{\pgfqpoint{2.407470in}{3.348945in}}%
\pgfpathlineto{\pgfqpoint{2.412011in}{3.348945in}}%
\pgfpathlineto{\pgfqpoint{2.412011in}{3.345996in}}%
\pgfpathmoveto{\pgfqpoint{2.402929in}{3.348945in}}%
\pgfpathlineto{\pgfqpoint{2.402929in}{3.348945in}}%
\pgfpathlineto{\pgfqpoint{2.402929in}{3.351895in}}%
\pgfpathlineto{\pgfqpoint{2.407470in}{3.351895in}}%
\pgfpathlineto{\pgfqpoint{2.407470in}{3.348945in}}%
\pgfpathmoveto{\pgfqpoint{2.402929in}{3.351895in}}%
\pgfpathlineto{\pgfqpoint{2.402929in}{3.351895in}}%
\pgfpathlineto{\pgfqpoint{2.402929in}{3.354844in}}%
\pgfpathlineto{\pgfqpoint{2.407470in}{3.354844in}}%
\pgfpathlineto{\pgfqpoint{2.407470in}{3.351895in}}%
\pgfpathmoveto{\pgfqpoint{2.407470in}{3.348945in}}%
\pgfpathlineto{\pgfqpoint{2.407470in}{3.348945in}}%
\pgfpathlineto{\pgfqpoint{2.407470in}{3.351895in}}%
\pgfpathlineto{\pgfqpoint{2.412011in}{3.351895in}}%
\pgfpathlineto{\pgfqpoint{2.412011in}{3.348945in}}%
\pgfpathmoveto{\pgfqpoint{2.412011in}{3.343047in}}%
\pgfpathlineto{\pgfqpoint{2.412011in}{3.343047in}}%
\pgfpathlineto{\pgfqpoint{2.412011in}{3.345996in}}%
\pgfpathlineto{\pgfqpoint{2.416552in}{3.345996in}}%
\pgfpathlineto{\pgfqpoint{2.416552in}{3.343047in}}%
\pgfpathmoveto{\pgfqpoint{2.412011in}{3.345996in}}%
\pgfpathlineto{\pgfqpoint{2.412011in}{3.345996in}}%
\pgfpathlineto{\pgfqpoint{2.412011in}{3.348945in}}%
\pgfpathlineto{\pgfqpoint{2.416552in}{3.348945in}}%
\pgfpathlineto{\pgfqpoint{2.416552in}{3.345996in}}%
\pgfpathmoveto{\pgfqpoint{2.416552in}{3.343047in}}%
\pgfpathlineto{\pgfqpoint{2.416552in}{3.343047in}}%
\pgfpathlineto{\pgfqpoint{2.416552in}{3.345996in}}%
\pgfpathlineto{\pgfqpoint{2.421093in}{3.345996in}}%
\pgfpathlineto{\pgfqpoint{2.421093in}{3.343047in}}%
\pgfpathmoveto{\pgfqpoint{2.389305in}{3.357793in}}%
\pgfpathlineto{\pgfqpoint{2.389305in}{3.357793in}}%
\pgfpathlineto{\pgfqpoint{2.389305in}{3.360742in}}%
\pgfpathlineto{\pgfqpoint{2.393846in}{3.360742in}}%
\pgfpathlineto{\pgfqpoint{2.393846in}{3.357793in}}%
\pgfpathmoveto{\pgfqpoint{2.384764in}{3.360742in}}%
\pgfpathlineto{\pgfqpoint{2.384764in}{3.360742in}}%
\pgfpathlineto{\pgfqpoint{2.384764in}{3.363692in}}%
\pgfpathlineto{\pgfqpoint{2.389305in}{3.363692in}}%
\pgfpathlineto{\pgfqpoint{2.389305in}{3.360742in}}%
\pgfpathmoveto{\pgfqpoint{2.384764in}{3.363692in}}%
\pgfpathlineto{\pgfqpoint{2.384764in}{3.363692in}}%
\pgfpathlineto{\pgfqpoint{2.384764in}{3.366641in}}%
\pgfpathlineto{\pgfqpoint{2.389305in}{3.366641in}}%
\pgfpathlineto{\pgfqpoint{2.389305in}{3.363692in}}%
\pgfpathmoveto{\pgfqpoint{2.389305in}{3.360742in}}%
\pgfpathlineto{\pgfqpoint{2.389305in}{3.360742in}}%
\pgfpathlineto{\pgfqpoint{2.389305in}{3.363692in}}%
\pgfpathlineto{\pgfqpoint{2.393846in}{3.363692in}}%
\pgfpathlineto{\pgfqpoint{2.393846in}{3.360742in}}%
\pgfpathmoveto{\pgfqpoint{2.393846in}{3.354844in}}%
\pgfpathlineto{\pgfqpoint{2.393846in}{3.354844in}}%
\pgfpathlineto{\pgfqpoint{2.393846in}{3.357793in}}%
\pgfpathlineto{\pgfqpoint{2.398388in}{3.357793in}}%
\pgfpathlineto{\pgfqpoint{2.398388in}{3.354844in}}%
\pgfpathmoveto{\pgfqpoint{2.393846in}{3.357793in}}%
\pgfpathlineto{\pgfqpoint{2.393846in}{3.357793in}}%
\pgfpathlineto{\pgfqpoint{2.393846in}{3.360742in}}%
\pgfpathlineto{\pgfqpoint{2.398388in}{3.360742in}}%
\pgfpathlineto{\pgfqpoint{2.398388in}{3.357793in}}%
\pgfpathmoveto{\pgfqpoint{2.398388in}{3.354844in}}%
\pgfpathlineto{\pgfqpoint{2.398388in}{3.354844in}}%
\pgfpathlineto{\pgfqpoint{2.398388in}{3.357793in}}%
\pgfpathlineto{\pgfqpoint{2.402929in}{3.357793in}}%
\pgfpathlineto{\pgfqpoint{2.402929in}{3.354844in}}%
\pgfpathmoveto{\pgfqpoint{2.352977in}{3.381388in}}%
\pgfpathlineto{\pgfqpoint{2.352977in}{3.381388in}}%
\pgfpathlineto{\pgfqpoint{2.352977in}{3.384337in}}%
\pgfpathlineto{\pgfqpoint{2.357518in}{3.384337in}}%
\pgfpathlineto{\pgfqpoint{2.357518in}{3.381388in}}%
\pgfpathmoveto{\pgfqpoint{2.348436in}{3.384337in}}%
\pgfpathlineto{\pgfqpoint{2.348436in}{3.384337in}}%
\pgfpathlineto{\pgfqpoint{2.348436in}{3.387286in}}%
\pgfpathlineto{\pgfqpoint{2.352977in}{3.387286in}}%
\pgfpathlineto{\pgfqpoint{2.352977in}{3.384337in}}%
\pgfpathmoveto{\pgfqpoint{2.348436in}{3.387286in}}%
\pgfpathlineto{\pgfqpoint{2.348436in}{3.387286in}}%
\pgfpathlineto{\pgfqpoint{2.348436in}{3.390235in}}%
\pgfpathlineto{\pgfqpoint{2.352977in}{3.390235in}}%
\pgfpathlineto{\pgfqpoint{2.352977in}{3.387286in}}%
\pgfpathmoveto{\pgfqpoint{2.352977in}{3.384337in}}%
\pgfpathlineto{\pgfqpoint{2.352977in}{3.384337in}}%
\pgfpathlineto{\pgfqpoint{2.352977in}{3.387286in}}%
\pgfpathlineto{\pgfqpoint{2.357518in}{3.387286in}}%
\pgfpathlineto{\pgfqpoint{2.357518in}{3.384337in}}%
\pgfpathmoveto{\pgfqpoint{2.357518in}{3.378438in}}%
\pgfpathlineto{\pgfqpoint{2.357518in}{3.378438in}}%
\pgfpathlineto{\pgfqpoint{2.357518in}{3.381388in}}%
\pgfpathlineto{\pgfqpoint{2.362059in}{3.381388in}}%
\pgfpathlineto{\pgfqpoint{2.362059in}{3.378438in}}%
\pgfpathmoveto{\pgfqpoint{2.357518in}{3.381388in}}%
\pgfpathlineto{\pgfqpoint{2.357518in}{3.381388in}}%
\pgfpathlineto{\pgfqpoint{2.357518in}{3.384337in}}%
\pgfpathlineto{\pgfqpoint{2.362059in}{3.384337in}}%
\pgfpathlineto{\pgfqpoint{2.362059in}{3.381388in}}%
\pgfpathmoveto{\pgfqpoint{2.362059in}{3.378438in}}%
\pgfpathlineto{\pgfqpoint{2.362059in}{3.378438in}}%
\pgfpathlineto{\pgfqpoint{2.362059in}{3.381388in}}%
\pgfpathlineto{\pgfqpoint{2.366600in}{3.381388in}}%
\pgfpathlineto{\pgfqpoint{2.366600in}{3.378438in}}%
\pgfpathmoveto{\pgfqpoint{2.425634in}{3.334199in}}%
\pgfpathlineto{\pgfqpoint{2.425634in}{3.334199in}}%
\pgfpathlineto{\pgfqpoint{2.425634in}{3.337148in}}%
\pgfpathlineto{\pgfqpoint{2.430175in}{3.337148in}}%
\pgfpathlineto{\pgfqpoint{2.430175in}{3.334199in}}%
\pgfpathmoveto{\pgfqpoint{2.421093in}{3.337148in}}%
\pgfpathlineto{\pgfqpoint{2.421093in}{3.337148in}}%
\pgfpathlineto{\pgfqpoint{2.421093in}{3.340097in}}%
\pgfpathlineto{\pgfqpoint{2.425634in}{3.340097in}}%
\pgfpathlineto{\pgfqpoint{2.425634in}{3.337148in}}%
\pgfpathmoveto{\pgfqpoint{2.421093in}{3.340097in}}%
\pgfpathlineto{\pgfqpoint{2.421093in}{3.340097in}}%
\pgfpathlineto{\pgfqpoint{2.421093in}{3.343047in}}%
\pgfpathlineto{\pgfqpoint{2.425634in}{3.343047in}}%
\pgfpathlineto{\pgfqpoint{2.425634in}{3.340097in}}%
\pgfpathmoveto{\pgfqpoint{2.425634in}{3.337148in}}%
\pgfpathlineto{\pgfqpoint{2.425634in}{3.337148in}}%
\pgfpathlineto{\pgfqpoint{2.425634in}{3.340097in}}%
\pgfpathlineto{\pgfqpoint{2.430175in}{3.340097in}}%
\pgfpathlineto{\pgfqpoint{2.430175in}{3.337148in}}%
\pgfpathmoveto{\pgfqpoint{2.430175in}{3.331250in}}%
\pgfpathlineto{\pgfqpoint{2.430175in}{3.331250in}}%
\pgfpathlineto{\pgfqpoint{2.430175in}{3.334199in}}%
\pgfpathlineto{\pgfqpoint{2.434716in}{3.334199in}}%
\pgfpathlineto{\pgfqpoint{2.434716in}{3.331250in}}%
\pgfpathmoveto{\pgfqpoint{2.430175in}{3.334199in}}%
\pgfpathlineto{\pgfqpoint{2.430175in}{3.334199in}}%
\pgfpathlineto{\pgfqpoint{2.430175in}{3.337148in}}%
\pgfpathlineto{\pgfqpoint{2.434716in}{3.337148in}}%
\pgfpathlineto{\pgfqpoint{2.434716in}{3.334199in}}%
\pgfpathmoveto{\pgfqpoint{2.434716in}{3.331250in}}%
\pgfpathlineto{\pgfqpoint{2.434716in}{3.331250in}}%
\pgfpathlineto{\pgfqpoint{2.434716in}{3.334199in}}%
\pgfpathlineto{\pgfqpoint{2.439257in}{3.334199in}}%
\pgfpathlineto{\pgfqpoint{2.439257in}{3.331250in}}%
\pgfpathmoveto{\pgfqpoint{2.580028in}{3.233928in}}%
\pgfpathlineto{\pgfqpoint{2.580028in}{3.233928in}}%
\pgfpathlineto{\pgfqpoint{2.580028in}{3.236878in}}%
\pgfpathlineto{\pgfqpoint{2.584569in}{3.236878in}}%
\pgfpathlineto{\pgfqpoint{2.584569in}{3.233928in}}%
\pgfpathmoveto{\pgfqpoint{2.598191in}{3.222131in}}%
\pgfpathlineto{\pgfqpoint{2.598191in}{3.222131in}}%
\pgfpathlineto{\pgfqpoint{2.598191in}{3.225081in}}%
\pgfpathlineto{\pgfqpoint{2.602732in}{3.225081in}}%
\pgfpathlineto{\pgfqpoint{2.602732in}{3.222131in}}%
\pgfpathmoveto{\pgfqpoint{2.589110in}{3.228030in}}%
\pgfpathlineto{\pgfqpoint{2.589110in}{3.228030in}}%
\pgfpathlineto{\pgfqpoint{2.589110in}{3.230979in}}%
\pgfpathlineto{\pgfqpoint{2.593651in}{3.230979in}}%
\pgfpathlineto{\pgfqpoint{2.593651in}{3.228030in}}%
\pgfpathmoveto{\pgfqpoint{2.584569in}{3.230979in}}%
\pgfpathlineto{\pgfqpoint{2.584569in}{3.230979in}}%
\pgfpathlineto{\pgfqpoint{2.584569in}{3.233928in}}%
\pgfpathlineto{\pgfqpoint{2.589110in}{3.233928in}}%
\pgfpathlineto{\pgfqpoint{2.589110in}{3.230979in}}%
\pgfpathmoveto{\pgfqpoint{2.584569in}{3.233928in}}%
\pgfpathlineto{\pgfqpoint{2.584569in}{3.233928in}}%
\pgfpathlineto{\pgfqpoint{2.584569in}{3.236878in}}%
\pgfpathlineto{\pgfqpoint{2.589110in}{3.236878in}}%
\pgfpathlineto{\pgfqpoint{2.589110in}{3.233928in}}%
\pgfpathmoveto{\pgfqpoint{2.589110in}{3.230979in}}%
\pgfpathlineto{\pgfqpoint{2.589110in}{3.230979in}}%
\pgfpathlineto{\pgfqpoint{2.589110in}{3.233928in}}%
\pgfpathlineto{\pgfqpoint{2.593651in}{3.233928in}}%
\pgfpathlineto{\pgfqpoint{2.593651in}{3.230979in}}%
\pgfpathmoveto{\pgfqpoint{2.593651in}{3.225081in}}%
\pgfpathlineto{\pgfqpoint{2.593651in}{3.225081in}}%
\pgfpathlineto{\pgfqpoint{2.593651in}{3.228030in}}%
\pgfpathlineto{\pgfqpoint{2.598191in}{3.228030in}}%
\pgfpathlineto{\pgfqpoint{2.598191in}{3.225081in}}%
\pgfpathmoveto{\pgfqpoint{2.593651in}{3.228030in}}%
\pgfpathlineto{\pgfqpoint{2.593651in}{3.228030in}}%
\pgfpathlineto{\pgfqpoint{2.593651in}{3.230979in}}%
\pgfpathlineto{\pgfqpoint{2.598191in}{3.230979in}}%
\pgfpathlineto{\pgfqpoint{2.598191in}{3.228030in}}%
\pgfpathmoveto{\pgfqpoint{2.598191in}{3.225081in}}%
\pgfpathlineto{\pgfqpoint{2.598191in}{3.225081in}}%
\pgfpathlineto{\pgfqpoint{2.598191in}{3.228030in}}%
\pgfpathlineto{\pgfqpoint{2.602732in}{3.228030in}}%
\pgfpathlineto{\pgfqpoint{2.602732in}{3.225081in}}%
\pgfpathmoveto{\pgfqpoint{2.616355in}{3.210334in}}%
\pgfpathlineto{\pgfqpoint{2.616355in}{3.210334in}}%
\pgfpathlineto{\pgfqpoint{2.616355in}{3.213284in}}%
\pgfpathlineto{\pgfqpoint{2.620896in}{3.213284in}}%
\pgfpathlineto{\pgfqpoint{2.620896in}{3.210334in}}%
\pgfpathmoveto{\pgfqpoint{2.634519in}{3.198538in}}%
\pgfpathlineto{\pgfqpoint{2.634519in}{3.198538in}}%
\pgfpathlineto{\pgfqpoint{2.634519in}{3.201487in}}%
\pgfpathlineto{\pgfqpoint{2.639060in}{3.201487in}}%
\pgfpathlineto{\pgfqpoint{2.639060in}{3.198538in}}%
\pgfpathmoveto{\pgfqpoint{2.625437in}{3.204436in}}%
\pgfpathlineto{\pgfqpoint{2.625437in}{3.204436in}}%
\pgfpathlineto{\pgfqpoint{2.625437in}{3.207385in}}%
\pgfpathlineto{\pgfqpoint{2.629978in}{3.207385in}}%
\pgfpathlineto{\pgfqpoint{2.629978in}{3.204436in}}%
\pgfpathmoveto{\pgfqpoint{2.620896in}{3.207385in}}%
\pgfpathlineto{\pgfqpoint{2.620896in}{3.207385in}}%
\pgfpathlineto{\pgfqpoint{2.620896in}{3.210334in}}%
\pgfpathlineto{\pgfqpoint{2.625437in}{3.210334in}}%
\pgfpathlineto{\pgfqpoint{2.625437in}{3.207385in}}%
\pgfpathmoveto{\pgfqpoint{2.620896in}{3.210334in}}%
\pgfpathlineto{\pgfqpoint{2.620896in}{3.210334in}}%
\pgfpathlineto{\pgfqpoint{2.620896in}{3.213284in}}%
\pgfpathlineto{\pgfqpoint{2.625437in}{3.213284in}}%
\pgfpathlineto{\pgfqpoint{2.625437in}{3.210334in}}%
\pgfpathmoveto{\pgfqpoint{2.625437in}{3.207385in}}%
\pgfpathlineto{\pgfqpoint{2.625437in}{3.207385in}}%
\pgfpathlineto{\pgfqpoint{2.625437in}{3.210334in}}%
\pgfpathlineto{\pgfqpoint{2.629978in}{3.210334in}}%
\pgfpathlineto{\pgfqpoint{2.629978in}{3.207385in}}%
\pgfpathmoveto{\pgfqpoint{2.629978in}{3.201487in}}%
\pgfpathlineto{\pgfqpoint{2.629978in}{3.201487in}}%
\pgfpathlineto{\pgfqpoint{2.629978in}{3.204436in}}%
\pgfpathlineto{\pgfqpoint{2.634519in}{3.204436in}}%
\pgfpathlineto{\pgfqpoint{2.634519in}{3.201487in}}%
\pgfpathmoveto{\pgfqpoint{2.629978in}{3.204436in}}%
\pgfpathlineto{\pgfqpoint{2.629978in}{3.204436in}}%
\pgfpathlineto{\pgfqpoint{2.629978in}{3.207385in}}%
\pgfpathlineto{\pgfqpoint{2.634519in}{3.207385in}}%
\pgfpathlineto{\pgfqpoint{2.634519in}{3.204436in}}%
\pgfpathmoveto{\pgfqpoint{2.634519in}{3.201487in}}%
\pgfpathlineto{\pgfqpoint{2.634519in}{3.201487in}}%
\pgfpathlineto{\pgfqpoint{2.634519in}{3.204436in}}%
\pgfpathlineto{\pgfqpoint{2.639060in}{3.204436in}}%
\pgfpathlineto{\pgfqpoint{2.639060in}{3.201487in}}%
\pgfpathmoveto{\pgfqpoint{2.607273in}{3.216233in}}%
\pgfpathlineto{\pgfqpoint{2.607273in}{3.216233in}}%
\pgfpathlineto{\pgfqpoint{2.607273in}{3.219182in}}%
\pgfpathlineto{\pgfqpoint{2.611814in}{3.219182in}}%
\pgfpathlineto{\pgfqpoint{2.611814in}{3.216233in}}%
\pgfpathmoveto{\pgfqpoint{2.602732in}{3.219182in}}%
\pgfpathlineto{\pgfqpoint{2.602732in}{3.219182in}}%
\pgfpathlineto{\pgfqpoint{2.602732in}{3.222131in}}%
\pgfpathlineto{\pgfqpoint{2.607273in}{3.222131in}}%
\pgfpathlineto{\pgfqpoint{2.607273in}{3.219182in}}%
\pgfpathmoveto{\pgfqpoint{2.602732in}{3.222131in}}%
\pgfpathlineto{\pgfqpoint{2.602732in}{3.222131in}}%
\pgfpathlineto{\pgfqpoint{2.602732in}{3.225081in}}%
\pgfpathlineto{\pgfqpoint{2.607273in}{3.225081in}}%
\pgfpathlineto{\pgfqpoint{2.607273in}{3.222131in}}%
\pgfpathmoveto{\pgfqpoint{2.607273in}{3.219182in}}%
\pgfpathlineto{\pgfqpoint{2.607273in}{3.219182in}}%
\pgfpathlineto{\pgfqpoint{2.607273in}{3.222131in}}%
\pgfpathlineto{\pgfqpoint{2.611814in}{3.222131in}}%
\pgfpathlineto{\pgfqpoint{2.611814in}{3.219182in}}%
\pgfpathmoveto{\pgfqpoint{2.611814in}{3.213284in}}%
\pgfpathlineto{\pgfqpoint{2.611814in}{3.213284in}}%
\pgfpathlineto{\pgfqpoint{2.611814in}{3.216233in}}%
\pgfpathlineto{\pgfqpoint{2.616355in}{3.216233in}}%
\pgfpathlineto{\pgfqpoint{2.616355in}{3.213284in}}%
\pgfpathmoveto{\pgfqpoint{2.611814in}{3.216233in}}%
\pgfpathlineto{\pgfqpoint{2.611814in}{3.216233in}}%
\pgfpathlineto{\pgfqpoint{2.611814in}{3.219182in}}%
\pgfpathlineto{\pgfqpoint{2.616355in}{3.219182in}}%
\pgfpathlineto{\pgfqpoint{2.616355in}{3.216233in}}%
\pgfpathmoveto{\pgfqpoint{2.616355in}{3.213284in}}%
\pgfpathlineto{\pgfqpoint{2.616355in}{3.213284in}}%
\pgfpathlineto{\pgfqpoint{2.616355in}{3.216233in}}%
\pgfpathlineto{\pgfqpoint{2.620896in}{3.216233in}}%
\pgfpathlineto{\pgfqpoint{2.620896in}{3.213284in}}%
\pgfpathmoveto{\pgfqpoint{2.507373in}{3.281114in}}%
\pgfpathlineto{\pgfqpoint{2.507373in}{3.281114in}}%
\pgfpathlineto{\pgfqpoint{2.507373in}{3.284064in}}%
\pgfpathlineto{\pgfqpoint{2.511914in}{3.284064in}}%
\pgfpathlineto{\pgfqpoint{2.511914in}{3.281114in}}%
\pgfpathmoveto{\pgfqpoint{2.525537in}{3.269318in}}%
\pgfpathlineto{\pgfqpoint{2.525537in}{3.269318in}}%
\pgfpathlineto{\pgfqpoint{2.525537in}{3.272267in}}%
\pgfpathlineto{\pgfqpoint{2.530078in}{3.272267in}}%
\pgfpathlineto{\pgfqpoint{2.530078in}{3.269318in}}%
\pgfpathmoveto{\pgfqpoint{2.516455in}{3.275216in}}%
\pgfpathlineto{\pgfqpoint{2.516455in}{3.275216in}}%
\pgfpathlineto{\pgfqpoint{2.516455in}{3.278165in}}%
\pgfpathlineto{\pgfqpoint{2.520996in}{3.278165in}}%
\pgfpathlineto{\pgfqpoint{2.520996in}{3.275216in}}%
\pgfpathmoveto{\pgfqpoint{2.511914in}{3.278165in}}%
\pgfpathlineto{\pgfqpoint{2.511914in}{3.278165in}}%
\pgfpathlineto{\pgfqpoint{2.511914in}{3.281114in}}%
\pgfpathlineto{\pgfqpoint{2.516455in}{3.281114in}}%
\pgfpathlineto{\pgfqpoint{2.516455in}{3.278165in}}%
\pgfpathmoveto{\pgfqpoint{2.511914in}{3.281114in}}%
\pgfpathlineto{\pgfqpoint{2.511914in}{3.281114in}}%
\pgfpathlineto{\pgfqpoint{2.511914in}{3.284064in}}%
\pgfpathlineto{\pgfqpoint{2.516455in}{3.284064in}}%
\pgfpathlineto{\pgfqpoint{2.516455in}{3.281114in}}%
\pgfpathmoveto{\pgfqpoint{2.516455in}{3.278165in}}%
\pgfpathlineto{\pgfqpoint{2.516455in}{3.278165in}}%
\pgfpathlineto{\pgfqpoint{2.516455in}{3.281114in}}%
\pgfpathlineto{\pgfqpoint{2.520996in}{3.281114in}}%
\pgfpathlineto{\pgfqpoint{2.520996in}{3.278165in}}%
\pgfpathmoveto{\pgfqpoint{2.520996in}{3.272267in}}%
\pgfpathlineto{\pgfqpoint{2.520996in}{3.272267in}}%
\pgfpathlineto{\pgfqpoint{2.520996in}{3.275216in}}%
\pgfpathlineto{\pgfqpoint{2.525537in}{3.275216in}}%
\pgfpathlineto{\pgfqpoint{2.525537in}{3.272267in}}%
\pgfpathmoveto{\pgfqpoint{2.520996in}{3.275216in}}%
\pgfpathlineto{\pgfqpoint{2.520996in}{3.275216in}}%
\pgfpathlineto{\pgfqpoint{2.520996in}{3.278165in}}%
\pgfpathlineto{\pgfqpoint{2.525537in}{3.278165in}}%
\pgfpathlineto{\pgfqpoint{2.525537in}{3.275216in}}%
\pgfpathmoveto{\pgfqpoint{2.525537in}{3.272267in}}%
\pgfpathlineto{\pgfqpoint{2.525537in}{3.272267in}}%
\pgfpathlineto{\pgfqpoint{2.525537in}{3.275216in}}%
\pgfpathlineto{\pgfqpoint{2.530078in}{3.275216in}}%
\pgfpathlineto{\pgfqpoint{2.530078in}{3.272267in}}%
\pgfpathmoveto{\pgfqpoint{2.543700in}{3.257521in}}%
\pgfpathlineto{\pgfqpoint{2.543700in}{3.257521in}}%
\pgfpathlineto{\pgfqpoint{2.543700in}{3.260471in}}%
\pgfpathlineto{\pgfqpoint{2.548241in}{3.260471in}}%
\pgfpathlineto{\pgfqpoint{2.548241in}{3.257521in}}%
\pgfpathmoveto{\pgfqpoint{2.561864in}{3.245725in}}%
\pgfpathlineto{\pgfqpoint{2.561864in}{3.245725in}}%
\pgfpathlineto{\pgfqpoint{2.561864in}{3.248674in}}%
\pgfpathlineto{\pgfqpoint{2.566405in}{3.248674in}}%
\pgfpathlineto{\pgfqpoint{2.566405in}{3.245725in}}%
\pgfpathmoveto{\pgfqpoint{2.552782in}{3.251623in}}%
\pgfpathlineto{\pgfqpoint{2.552782in}{3.251623in}}%
\pgfpathlineto{\pgfqpoint{2.552782in}{3.254572in}}%
\pgfpathlineto{\pgfqpoint{2.557323in}{3.254572in}}%
\pgfpathlineto{\pgfqpoint{2.557323in}{3.251623in}}%
\pgfpathmoveto{\pgfqpoint{2.548241in}{3.254572in}}%
\pgfpathlineto{\pgfqpoint{2.548241in}{3.254572in}}%
\pgfpathlineto{\pgfqpoint{2.548241in}{3.257521in}}%
\pgfpathlineto{\pgfqpoint{2.552782in}{3.257521in}}%
\pgfpathlineto{\pgfqpoint{2.552782in}{3.254572in}}%
\pgfpathmoveto{\pgfqpoint{2.548241in}{3.257521in}}%
\pgfpathlineto{\pgfqpoint{2.548241in}{3.257521in}}%
\pgfpathlineto{\pgfqpoint{2.548241in}{3.260471in}}%
\pgfpathlineto{\pgfqpoint{2.552782in}{3.260471in}}%
\pgfpathlineto{\pgfqpoint{2.552782in}{3.257521in}}%
\pgfpathmoveto{\pgfqpoint{2.552782in}{3.254572in}}%
\pgfpathlineto{\pgfqpoint{2.552782in}{3.254572in}}%
\pgfpathlineto{\pgfqpoint{2.552782in}{3.257521in}}%
\pgfpathlineto{\pgfqpoint{2.557323in}{3.257521in}}%
\pgfpathlineto{\pgfqpoint{2.557323in}{3.254572in}}%
\pgfpathmoveto{\pgfqpoint{2.557323in}{3.248674in}}%
\pgfpathlineto{\pgfqpoint{2.557323in}{3.248674in}}%
\pgfpathlineto{\pgfqpoint{2.557323in}{3.251623in}}%
\pgfpathlineto{\pgfqpoint{2.561864in}{3.251623in}}%
\pgfpathlineto{\pgfqpoint{2.561864in}{3.248674in}}%
\pgfpathmoveto{\pgfqpoint{2.557323in}{3.251623in}}%
\pgfpathlineto{\pgfqpoint{2.557323in}{3.251623in}}%
\pgfpathlineto{\pgfqpoint{2.557323in}{3.254572in}}%
\pgfpathlineto{\pgfqpoint{2.561864in}{3.254572in}}%
\pgfpathlineto{\pgfqpoint{2.561864in}{3.251623in}}%
\pgfpathmoveto{\pgfqpoint{2.561864in}{3.248674in}}%
\pgfpathlineto{\pgfqpoint{2.561864in}{3.248674in}}%
\pgfpathlineto{\pgfqpoint{2.561864in}{3.251623in}}%
\pgfpathlineto{\pgfqpoint{2.566405in}{3.251623in}}%
\pgfpathlineto{\pgfqpoint{2.566405in}{3.248674in}}%
\pgfpathmoveto{\pgfqpoint{2.534619in}{3.263420in}}%
\pgfpathlineto{\pgfqpoint{2.534619in}{3.263420in}}%
\pgfpathlineto{\pgfqpoint{2.534619in}{3.266369in}}%
\pgfpathlineto{\pgfqpoint{2.539160in}{3.266369in}}%
\pgfpathlineto{\pgfqpoint{2.539160in}{3.263420in}}%
\pgfpathmoveto{\pgfqpoint{2.530078in}{3.266369in}}%
\pgfpathlineto{\pgfqpoint{2.530078in}{3.266369in}}%
\pgfpathlineto{\pgfqpoint{2.530078in}{3.269318in}}%
\pgfpathlineto{\pgfqpoint{2.534619in}{3.269318in}}%
\pgfpathlineto{\pgfqpoint{2.534619in}{3.266369in}}%
\pgfpathmoveto{\pgfqpoint{2.530078in}{3.269318in}}%
\pgfpathlineto{\pgfqpoint{2.530078in}{3.269318in}}%
\pgfpathlineto{\pgfqpoint{2.530078in}{3.272267in}}%
\pgfpathlineto{\pgfqpoint{2.534619in}{3.272267in}}%
\pgfpathlineto{\pgfqpoint{2.534619in}{3.269318in}}%
\pgfpathmoveto{\pgfqpoint{2.534619in}{3.266369in}}%
\pgfpathlineto{\pgfqpoint{2.534619in}{3.266369in}}%
\pgfpathlineto{\pgfqpoint{2.534619in}{3.269318in}}%
\pgfpathlineto{\pgfqpoint{2.539160in}{3.269318in}}%
\pgfpathlineto{\pgfqpoint{2.539160in}{3.266369in}}%
\pgfpathmoveto{\pgfqpoint{2.539160in}{3.260471in}}%
\pgfpathlineto{\pgfqpoint{2.539160in}{3.260471in}}%
\pgfpathlineto{\pgfqpoint{2.539160in}{3.263420in}}%
\pgfpathlineto{\pgfqpoint{2.543700in}{3.263420in}}%
\pgfpathlineto{\pgfqpoint{2.543700in}{3.260471in}}%
\pgfpathmoveto{\pgfqpoint{2.539160in}{3.263420in}}%
\pgfpathlineto{\pgfqpoint{2.539160in}{3.263420in}}%
\pgfpathlineto{\pgfqpoint{2.539160in}{3.266369in}}%
\pgfpathlineto{\pgfqpoint{2.543700in}{3.266369in}}%
\pgfpathlineto{\pgfqpoint{2.543700in}{3.263420in}}%
\pgfpathmoveto{\pgfqpoint{2.543700in}{3.260471in}}%
\pgfpathlineto{\pgfqpoint{2.543700in}{3.260471in}}%
\pgfpathlineto{\pgfqpoint{2.543700in}{3.263420in}}%
\pgfpathlineto{\pgfqpoint{2.548241in}{3.263420in}}%
\pgfpathlineto{\pgfqpoint{2.548241in}{3.260471in}}%
\pgfpathmoveto{\pgfqpoint{2.498291in}{3.287013in}}%
\pgfpathlineto{\pgfqpoint{2.498291in}{3.287013in}}%
\pgfpathlineto{\pgfqpoint{2.498291in}{3.289962in}}%
\pgfpathlineto{\pgfqpoint{2.502832in}{3.289962in}}%
\pgfpathlineto{\pgfqpoint{2.502832in}{3.287013in}}%
\pgfpathmoveto{\pgfqpoint{2.493750in}{3.289962in}}%
\pgfpathlineto{\pgfqpoint{2.493750in}{3.289962in}}%
\pgfpathlineto{\pgfqpoint{2.493750in}{3.292911in}}%
\pgfpathlineto{\pgfqpoint{2.498291in}{3.292911in}}%
\pgfpathlineto{\pgfqpoint{2.498291in}{3.289962in}}%
\pgfpathmoveto{\pgfqpoint{2.493750in}{3.292911in}}%
\pgfpathlineto{\pgfqpoint{2.493750in}{3.292911in}}%
\pgfpathlineto{\pgfqpoint{2.493750in}{3.295860in}}%
\pgfpathlineto{\pgfqpoint{2.498291in}{3.295860in}}%
\pgfpathlineto{\pgfqpoint{2.498291in}{3.292911in}}%
\pgfpathmoveto{\pgfqpoint{2.498291in}{3.289962in}}%
\pgfpathlineto{\pgfqpoint{2.498291in}{3.289962in}}%
\pgfpathlineto{\pgfqpoint{2.498291in}{3.292911in}}%
\pgfpathlineto{\pgfqpoint{2.502832in}{3.292911in}}%
\pgfpathlineto{\pgfqpoint{2.502832in}{3.289962in}}%
\pgfpathmoveto{\pgfqpoint{2.502832in}{3.284064in}}%
\pgfpathlineto{\pgfqpoint{2.502832in}{3.284064in}}%
\pgfpathlineto{\pgfqpoint{2.502832in}{3.287013in}}%
\pgfpathlineto{\pgfqpoint{2.507373in}{3.287013in}}%
\pgfpathlineto{\pgfqpoint{2.507373in}{3.284064in}}%
\pgfpathmoveto{\pgfqpoint{2.502832in}{3.287013in}}%
\pgfpathlineto{\pgfqpoint{2.502832in}{3.287013in}}%
\pgfpathlineto{\pgfqpoint{2.502832in}{3.289962in}}%
\pgfpathlineto{\pgfqpoint{2.507373in}{3.289962in}}%
\pgfpathlineto{\pgfqpoint{2.507373in}{3.287013in}}%
\pgfpathmoveto{\pgfqpoint{2.507373in}{3.284064in}}%
\pgfpathlineto{\pgfqpoint{2.507373in}{3.284064in}}%
\pgfpathlineto{\pgfqpoint{2.507373in}{3.287013in}}%
\pgfpathlineto{\pgfqpoint{2.511914in}{3.287013in}}%
\pgfpathlineto{\pgfqpoint{2.511914in}{3.284064in}}%
\pgfpathmoveto{\pgfqpoint{2.570946in}{3.239827in}}%
\pgfpathlineto{\pgfqpoint{2.570946in}{3.239827in}}%
\pgfpathlineto{\pgfqpoint{2.570946in}{3.242776in}}%
\pgfpathlineto{\pgfqpoint{2.575487in}{3.242776in}}%
\pgfpathlineto{\pgfqpoint{2.575487in}{3.239827in}}%
\pgfpathmoveto{\pgfqpoint{2.566405in}{3.242776in}}%
\pgfpathlineto{\pgfqpoint{2.566405in}{3.242776in}}%
\pgfpathlineto{\pgfqpoint{2.566405in}{3.245725in}}%
\pgfpathlineto{\pgfqpoint{2.570946in}{3.245725in}}%
\pgfpathlineto{\pgfqpoint{2.570946in}{3.242776in}}%
\pgfpathmoveto{\pgfqpoint{2.566405in}{3.245725in}}%
\pgfpathlineto{\pgfqpoint{2.566405in}{3.245725in}}%
\pgfpathlineto{\pgfqpoint{2.566405in}{3.248674in}}%
\pgfpathlineto{\pgfqpoint{2.570946in}{3.248674in}}%
\pgfpathlineto{\pgfqpoint{2.570946in}{3.245725in}}%
\pgfpathmoveto{\pgfqpoint{2.570946in}{3.242776in}}%
\pgfpathlineto{\pgfqpoint{2.570946in}{3.242776in}}%
\pgfpathlineto{\pgfqpoint{2.570946in}{3.245725in}}%
\pgfpathlineto{\pgfqpoint{2.575487in}{3.245725in}}%
\pgfpathlineto{\pgfqpoint{2.575487in}{3.242776in}}%
\pgfpathmoveto{\pgfqpoint{2.575487in}{3.236878in}}%
\pgfpathlineto{\pgfqpoint{2.575487in}{3.236878in}}%
\pgfpathlineto{\pgfqpoint{2.575487in}{3.239827in}}%
\pgfpathlineto{\pgfqpoint{2.580028in}{3.239827in}}%
\pgfpathlineto{\pgfqpoint{2.580028in}{3.236878in}}%
\pgfpathmoveto{\pgfqpoint{2.575487in}{3.239827in}}%
\pgfpathlineto{\pgfqpoint{2.575487in}{3.239827in}}%
\pgfpathlineto{\pgfqpoint{2.575487in}{3.242776in}}%
\pgfpathlineto{\pgfqpoint{2.580028in}{3.242776in}}%
\pgfpathlineto{\pgfqpoint{2.580028in}{3.239827in}}%
\pgfpathmoveto{\pgfqpoint{2.580028in}{3.236878in}}%
\pgfpathlineto{\pgfqpoint{2.580028in}{3.236878in}}%
\pgfpathlineto{\pgfqpoint{2.580028in}{3.239827in}}%
\pgfpathlineto{\pgfqpoint{2.584569in}{3.239827in}}%
\pgfpathlineto{\pgfqpoint{2.584569in}{3.236878in}}%
\pgfpathmoveto{\pgfqpoint{2.725342in}{3.139553in}}%
\pgfpathlineto{\pgfqpoint{2.725342in}{3.139553in}}%
\pgfpathlineto{\pgfqpoint{2.725342in}{3.142502in}}%
\pgfpathlineto{\pgfqpoint{2.729883in}{3.142502in}}%
\pgfpathlineto{\pgfqpoint{2.729883in}{3.139553in}}%
\pgfpathmoveto{\pgfqpoint{2.743507in}{3.127756in}}%
\pgfpathlineto{\pgfqpoint{2.743507in}{3.127756in}}%
\pgfpathlineto{\pgfqpoint{2.743507in}{3.130705in}}%
\pgfpathlineto{\pgfqpoint{2.748048in}{3.130705in}}%
\pgfpathlineto{\pgfqpoint{2.748048in}{3.127756in}}%
\pgfpathmoveto{\pgfqpoint{2.734425in}{3.133654in}}%
\pgfpathlineto{\pgfqpoint{2.734425in}{3.133654in}}%
\pgfpathlineto{\pgfqpoint{2.734425in}{3.136604in}}%
\pgfpathlineto{\pgfqpoint{2.738966in}{3.136604in}}%
\pgfpathlineto{\pgfqpoint{2.738966in}{3.133654in}}%
\pgfpathmoveto{\pgfqpoint{2.729883in}{3.136604in}}%
\pgfpathlineto{\pgfqpoint{2.729883in}{3.136604in}}%
\pgfpathlineto{\pgfqpoint{2.729883in}{3.139553in}}%
\pgfpathlineto{\pgfqpoint{2.734425in}{3.139553in}}%
\pgfpathlineto{\pgfqpoint{2.734425in}{3.136604in}}%
\pgfpathmoveto{\pgfqpoint{2.729883in}{3.139553in}}%
\pgfpathlineto{\pgfqpoint{2.729883in}{3.139553in}}%
\pgfpathlineto{\pgfqpoint{2.729883in}{3.142502in}}%
\pgfpathlineto{\pgfqpoint{2.734425in}{3.142502in}}%
\pgfpathlineto{\pgfqpoint{2.734425in}{3.139553in}}%
\pgfpathmoveto{\pgfqpoint{2.734425in}{3.136604in}}%
\pgfpathlineto{\pgfqpoint{2.734425in}{3.136604in}}%
\pgfpathlineto{\pgfqpoint{2.734425in}{3.139553in}}%
\pgfpathlineto{\pgfqpoint{2.738966in}{3.139553in}}%
\pgfpathlineto{\pgfqpoint{2.738966in}{3.136604in}}%
\pgfpathmoveto{\pgfqpoint{2.738966in}{3.130705in}}%
\pgfpathlineto{\pgfqpoint{2.738966in}{3.130705in}}%
\pgfpathlineto{\pgfqpoint{2.738966in}{3.133654in}}%
\pgfpathlineto{\pgfqpoint{2.743507in}{3.133654in}}%
\pgfpathlineto{\pgfqpoint{2.743507in}{3.130705in}}%
\pgfpathmoveto{\pgfqpoint{2.738966in}{3.133654in}}%
\pgfpathlineto{\pgfqpoint{2.738966in}{3.133654in}}%
\pgfpathlineto{\pgfqpoint{2.738966in}{3.136604in}}%
\pgfpathlineto{\pgfqpoint{2.743507in}{3.136604in}}%
\pgfpathlineto{\pgfqpoint{2.743507in}{3.133654in}}%
\pgfpathmoveto{\pgfqpoint{2.743507in}{3.130705in}}%
\pgfpathlineto{\pgfqpoint{2.743507in}{3.130705in}}%
\pgfpathlineto{\pgfqpoint{2.743507in}{3.133654in}}%
\pgfpathlineto{\pgfqpoint{2.748048in}{3.133654in}}%
\pgfpathlineto{\pgfqpoint{2.748048in}{3.130705in}}%
\pgfpathmoveto{\pgfqpoint{2.761672in}{3.115959in}}%
\pgfpathlineto{\pgfqpoint{2.761672in}{3.115959in}}%
\pgfpathlineto{\pgfqpoint{2.761672in}{3.118908in}}%
\pgfpathlineto{\pgfqpoint{2.766213in}{3.118908in}}%
\pgfpathlineto{\pgfqpoint{2.766213in}{3.115959in}}%
\pgfpathmoveto{\pgfqpoint{2.779836in}{3.104162in}}%
\pgfpathlineto{\pgfqpoint{2.779836in}{3.104162in}}%
\pgfpathlineto{\pgfqpoint{2.779836in}{3.107112in}}%
\pgfpathlineto{\pgfqpoint{2.784378in}{3.107112in}}%
\pgfpathlineto{\pgfqpoint{2.784378in}{3.104162in}}%
\pgfpathmoveto{\pgfqpoint{2.770754in}{3.110061in}}%
\pgfpathlineto{\pgfqpoint{2.770754in}{3.110061in}}%
\pgfpathlineto{\pgfqpoint{2.770754in}{3.113010in}}%
\pgfpathlineto{\pgfqpoint{2.775295in}{3.113010in}}%
\pgfpathlineto{\pgfqpoint{2.775295in}{3.110061in}}%
\pgfpathmoveto{\pgfqpoint{2.766213in}{3.113010in}}%
\pgfpathlineto{\pgfqpoint{2.766213in}{3.113010in}}%
\pgfpathlineto{\pgfqpoint{2.766213in}{3.115959in}}%
\pgfpathlineto{\pgfqpoint{2.770754in}{3.115959in}}%
\pgfpathlineto{\pgfqpoint{2.770754in}{3.113010in}}%
\pgfpathmoveto{\pgfqpoint{2.766213in}{3.115959in}}%
\pgfpathlineto{\pgfqpoint{2.766213in}{3.115959in}}%
\pgfpathlineto{\pgfqpoint{2.766213in}{3.118908in}}%
\pgfpathlineto{\pgfqpoint{2.770754in}{3.118908in}}%
\pgfpathlineto{\pgfqpoint{2.770754in}{3.115959in}}%
\pgfpathmoveto{\pgfqpoint{2.770754in}{3.113010in}}%
\pgfpathlineto{\pgfqpoint{2.770754in}{3.113010in}}%
\pgfpathlineto{\pgfqpoint{2.770754in}{3.115959in}}%
\pgfpathlineto{\pgfqpoint{2.775295in}{3.115959in}}%
\pgfpathlineto{\pgfqpoint{2.775295in}{3.113010in}}%
\pgfpathmoveto{\pgfqpoint{2.775295in}{3.107112in}}%
\pgfpathlineto{\pgfqpoint{2.775295in}{3.107112in}}%
\pgfpathlineto{\pgfqpoint{2.775295in}{3.110061in}}%
\pgfpathlineto{\pgfqpoint{2.779836in}{3.110061in}}%
\pgfpathlineto{\pgfqpoint{2.779836in}{3.107112in}}%
\pgfpathmoveto{\pgfqpoint{2.775295in}{3.110061in}}%
\pgfpathlineto{\pgfqpoint{2.775295in}{3.110061in}}%
\pgfpathlineto{\pgfqpoint{2.775295in}{3.113010in}}%
\pgfpathlineto{\pgfqpoint{2.779836in}{3.113010in}}%
\pgfpathlineto{\pgfqpoint{2.779836in}{3.110061in}}%
\pgfpathmoveto{\pgfqpoint{2.779836in}{3.107112in}}%
\pgfpathlineto{\pgfqpoint{2.779836in}{3.107112in}}%
\pgfpathlineto{\pgfqpoint{2.779836in}{3.110061in}}%
\pgfpathlineto{\pgfqpoint{2.784378in}{3.110061in}}%
\pgfpathlineto{\pgfqpoint{2.784378in}{3.107112in}}%
\pgfpathmoveto{\pgfqpoint{2.752589in}{3.121858in}}%
\pgfpathlineto{\pgfqpoint{2.752589in}{3.121858in}}%
\pgfpathlineto{\pgfqpoint{2.752589in}{3.124807in}}%
\pgfpathlineto{\pgfqpoint{2.757131in}{3.124807in}}%
\pgfpathlineto{\pgfqpoint{2.757131in}{3.121858in}}%
\pgfpathmoveto{\pgfqpoint{2.748048in}{3.124807in}}%
\pgfpathlineto{\pgfqpoint{2.748048in}{3.124807in}}%
\pgfpathlineto{\pgfqpoint{2.748048in}{3.127756in}}%
\pgfpathlineto{\pgfqpoint{2.752589in}{3.127756in}}%
\pgfpathlineto{\pgfqpoint{2.752589in}{3.124807in}}%
\pgfpathmoveto{\pgfqpoint{2.748048in}{3.127756in}}%
\pgfpathlineto{\pgfqpoint{2.748048in}{3.127756in}}%
\pgfpathlineto{\pgfqpoint{2.748048in}{3.130705in}}%
\pgfpathlineto{\pgfqpoint{2.752589in}{3.130705in}}%
\pgfpathlineto{\pgfqpoint{2.752589in}{3.127756in}}%
\pgfpathmoveto{\pgfqpoint{2.752589in}{3.124807in}}%
\pgfpathlineto{\pgfqpoint{2.752589in}{3.124807in}}%
\pgfpathlineto{\pgfqpoint{2.752589in}{3.127756in}}%
\pgfpathlineto{\pgfqpoint{2.757131in}{3.127756in}}%
\pgfpathlineto{\pgfqpoint{2.757131in}{3.124807in}}%
\pgfpathmoveto{\pgfqpoint{2.757131in}{3.118908in}}%
\pgfpathlineto{\pgfqpoint{2.757131in}{3.118908in}}%
\pgfpathlineto{\pgfqpoint{2.757131in}{3.121858in}}%
\pgfpathlineto{\pgfqpoint{2.761672in}{3.121858in}}%
\pgfpathlineto{\pgfqpoint{2.761672in}{3.118908in}}%
\pgfpathmoveto{\pgfqpoint{2.757131in}{3.121858in}}%
\pgfpathlineto{\pgfqpoint{2.757131in}{3.121858in}}%
\pgfpathlineto{\pgfqpoint{2.757131in}{3.124807in}}%
\pgfpathlineto{\pgfqpoint{2.761672in}{3.124807in}}%
\pgfpathlineto{\pgfqpoint{2.761672in}{3.121858in}}%
\pgfpathmoveto{\pgfqpoint{2.761672in}{3.118908in}}%
\pgfpathlineto{\pgfqpoint{2.761672in}{3.118908in}}%
\pgfpathlineto{\pgfqpoint{2.761672in}{3.121858in}}%
\pgfpathlineto{\pgfqpoint{2.766213in}{3.121858in}}%
\pgfpathlineto{\pgfqpoint{2.766213in}{3.118908in}}%
\pgfpathmoveto{\pgfqpoint{2.652683in}{3.186741in}}%
\pgfpathlineto{\pgfqpoint{2.652683in}{3.186741in}}%
\pgfpathlineto{\pgfqpoint{2.652683in}{3.189690in}}%
\pgfpathlineto{\pgfqpoint{2.657224in}{3.189690in}}%
\pgfpathlineto{\pgfqpoint{2.657224in}{3.186741in}}%
\pgfpathmoveto{\pgfqpoint{2.670848in}{3.174944in}}%
\pgfpathlineto{\pgfqpoint{2.670848in}{3.174944in}}%
\pgfpathlineto{\pgfqpoint{2.670848in}{3.177893in}}%
\pgfpathlineto{\pgfqpoint{2.675389in}{3.177893in}}%
\pgfpathlineto{\pgfqpoint{2.675389in}{3.174944in}}%
\pgfpathmoveto{\pgfqpoint{2.661766in}{3.180842in}}%
\pgfpathlineto{\pgfqpoint{2.661766in}{3.180842in}}%
\pgfpathlineto{\pgfqpoint{2.661766in}{3.183791in}}%
\pgfpathlineto{\pgfqpoint{2.666307in}{3.183791in}}%
\pgfpathlineto{\pgfqpoint{2.666307in}{3.180842in}}%
\pgfpathmoveto{\pgfqpoint{2.657224in}{3.183791in}}%
\pgfpathlineto{\pgfqpoint{2.657224in}{3.183791in}}%
\pgfpathlineto{\pgfqpoint{2.657224in}{3.186741in}}%
\pgfpathlineto{\pgfqpoint{2.661766in}{3.186741in}}%
\pgfpathlineto{\pgfqpoint{2.661766in}{3.183791in}}%
\pgfpathmoveto{\pgfqpoint{2.657224in}{3.186741in}}%
\pgfpathlineto{\pgfqpoint{2.657224in}{3.186741in}}%
\pgfpathlineto{\pgfqpoint{2.657224in}{3.189690in}}%
\pgfpathlineto{\pgfqpoint{2.661766in}{3.189690in}}%
\pgfpathlineto{\pgfqpoint{2.661766in}{3.186741in}}%
\pgfpathmoveto{\pgfqpoint{2.661766in}{3.183791in}}%
\pgfpathlineto{\pgfqpoint{2.661766in}{3.183791in}}%
\pgfpathlineto{\pgfqpoint{2.661766in}{3.186741in}}%
\pgfpathlineto{\pgfqpoint{2.666307in}{3.186741in}}%
\pgfpathlineto{\pgfqpoint{2.666307in}{3.183791in}}%
\pgfpathmoveto{\pgfqpoint{2.666307in}{3.177893in}}%
\pgfpathlineto{\pgfqpoint{2.666307in}{3.177893in}}%
\pgfpathlineto{\pgfqpoint{2.666307in}{3.180842in}}%
\pgfpathlineto{\pgfqpoint{2.670848in}{3.180842in}}%
\pgfpathlineto{\pgfqpoint{2.670848in}{3.177893in}}%
\pgfpathmoveto{\pgfqpoint{2.666307in}{3.180842in}}%
\pgfpathlineto{\pgfqpoint{2.666307in}{3.180842in}}%
\pgfpathlineto{\pgfqpoint{2.666307in}{3.183791in}}%
\pgfpathlineto{\pgfqpoint{2.670848in}{3.183791in}}%
\pgfpathlineto{\pgfqpoint{2.670848in}{3.180842in}}%
\pgfpathmoveto{\pgfqpoint{2.670848in}{3.177893in}}%
\pgfpathlineto{\pgfqpoint{2.670848in}{3.177893in}}%
\pgfpathlineto{\pgfqpoint{2.670848in}{3.180842in}}%
\pgfpathlineto{\pgfqpoint{2.675389in}{3.180842in}}%
\pgfpathlineto{\pgfqpoint{2.675389in}{3.177893in}}%
\pgfpathmoveto{\pgfqpoint{2.689013in}{3.163147in}}%
\pgfpathlineto{\pgfqpoint{2.689013in}{3.163147in}}%
\pgfpathlineto{\pgfqpoint{2.689013in}{3.166096in}}%
\pgfpathlineto{\pgfqpoint{2.693554in}{3.166096in}}%
\pgfpathlineto{\pgfqpoint{2.693554in}{3.163147in}}%
\pgfpathmoveto{\pgfqpoint{2.707177in}{3.151350in}}%
\pgfpathlineto{\pgfqpoint{2.707177in}{3.151350in}}%
\pgfpathlineto{\pgfqpoint{2.707177in}{3.154299in}}%
\pgfpathlineto{\pgfqpoint{2.711719in}{3.154299in}}%
\pgfpathlineto{\pgfqpoint{2.711719in}{3.151350in}}%
\pgfpathmoveto{\pgfqpoint{2.698095in}{3.157248in}}%
\pgfpathlineto{\pgfqpoint{2.698095in}{3.157248in}}%
\pgfpathlineto{\pgfqpoint{2.698095in}{3.160197in}}%
\pgfpathlineto{\pgfqpoint{2.702636in}{3.160197in}}%
\pgfpathlineto{\pgfqpoint{2.702636in}{3.157248in}}%
\pgfpathmoveto{\pgfqpoint{2.693554in}{3.160197in}}%
\pgfpathlineto{\pgfqpoint{2.693554in}{3.160197in}}%
\pgfpathlineto{\pgfqpoint{2.693554in}{3.163147in}}%
\pgfpathlineto{\pgfqpoint{2.698095in}{3.163147in}}%
\pgfpathlineto{\pgfqpoint{2.698095in}{3.160197in}}%
\pgfpathmoveto{\pgfqpoint{2.693554in}{3.163147in}}%
\pgfpathlineto{\pgfqpoint{2.693554in}{3.163147in}}%
\pgfpathlineto{\pgfqpoint{2.693554in}{3.166096in}}%
\pgfpathlineto{\pgfqpoint{2.698095in}{3.166096in}}%
\pgfpathlineto{\pgfqpoint{2.698095in}{3.163147in}}%
\pgfpathmoveto{\pgfqpoint{2.698095in}{3.160197in}}%
\pgfpathlineto{\pgfqpoint{2.698095in}{3.160197in}}%
\pgfpathlineto{\pgfqpoint{2.698095in}{3.163147in}}%
\pgfpathlineto{\pgfqpoint{2.702636in}{3.163147in}}%
\pgfpathlineto{\pgfqpoint{2.702636in}{3.160197in}}%
\pgfpathmoveto{\pgfqpoint{2.702636in}{3.154299in}}%
\pgfpathlineto{\pgfqpoint{2.702636in}{3.154299in}}%
\pgfpathlineto{\pgfqpoint{2.702636in}{3.157248in}}%
\pgfpathlineto{\pgfqpoint{2.707177in}{3.157248in}}%
\pgfpathlineto{\pgfqpoint{2.707177in}{3.154299in}}%
\pgfpathmoveto{\pgfqpoint{2.702636in}{3.157248in}}%
\pgfpathlineto{\pgfqpoint{2.702636in}{3.157248in}}%
\pgfpathlineto{\pgfqpoint{2.702636in}{3.160197in}}%
\pgfpathlineto{\pgfqpoint{2.707177in}{3.160197in}}%
\pgfpathlineto{\pgfqpoint{2.707177in}{3.157248in}}%
\pgfpathmoveto{\pgfqpoint{2.707177in}{3.154299in}}%
\pgfpathlineto{\pgfqpoint{2.707177in}{3.154299in}}%
\pgfpathlineto{\pgfqpoint{2.707177in}{3.157248in}}%
\pgfpathlineto{\pgfqpoint{2.711719in}{3.157248in}}%
\pgfpathlineto{\pgfqpoint{2.711719in}{3.154299in}}%
\pgfpathmoveto{\pgfqpoint{2.679930in}{3.169045in}}%
\pgfpathlineto{\pgfqpoint{2.679930in}{3.169045in}}%
\pgfpathlineto{\pgfqpoint{2.679930in}{3.171994in}}%
\pgfpathlineto{\pgfqpoint{2.684472in}{3.171994in}}%
\pgfpathlineto{\pgfqpoint{2.684472in}{3.169045in}}%
\pgfpathmoveto{\pgfqpoint{2.675389in}{3.171994in}}%
\pgfpathlineto{\pgfqpoint{2.675389in}{3.171994in}}%
\pgfpathlineto{\pgfqpoint{2.675389in}{3.174944in}}%
\pgfpathlineto{\pgfqpoint{2.679930in}{3.174944in}}%
\pgfpathlineto{\pgfqpoint{2.679930in}{3.171994in}}%
\pgfpathmoveto{\pgfqpoint{2.675389in}{3.174944in}}%
\pgfpathlineto{\pgfqpoint{2.675389in}{3.174944in}}%
\pgfpathlineto{\pgfqpoint{2.675389in}{3.177893in}}%
\pgfpathlineto{\pgfqpoint{2.679930in}{3.177893in}}%
\pgfpathlineto{\pgfqpoint{2.679930in}{3.174944in}}%
\pgfpathmoveto{\pgfqpoint{2.679930in}{3.171994in}}%
\pgfpathlineto{\pgfqpoint{2.679930in}{3.171994in}}%
\pgfpathlineto{\pgfqpoint{2.679930in}{3.174944in}}%
\pgfpathlineto{\pgfqpoint{2.684472in}{3.174944in}}%
\pgfpathlineto{\pgfqpoint{2.684472in}{3.171994in}}%
\pgfpathmoveto{\pgfqpoint{2.684472in}{3.166096in}}%
\pgfpathlineto{\pgfqpoint{2.684472in}{3.166096in}}%
\pgfpathlineto{\pgfqpoint{2.684472in}{3.169045in}}%
\pgfpathlineto{\pgfqpoint{2.689013in}{3.169045in}}%
\pgfpathlineto{\pgfqpoint{2.689013in}{3.166096in}}%
\pgfpathmoveto{\pgfqpoint{2.684472in}{3.169045in}}%
\pgfpathlineto{\pgfqpoint{2.684472in}{3.169045in}}%
\pgfpathlineto{\pgfqpoint{2.684472in}{3.171994in}}%
\pgfpathlineto{\pgfqpoint{2.689013in}{3.171994in}}%
\pgfpathlineto{\pgfqpoint{2.689013in}{3.169045in}}%
\pgfpathmoveto{\pgfqpoint{2.689013in}{3.166096in}}%
\pgfpathlineto{\pgfqpoint{2.689013in}{3.166096in}}%
\pgfpathlineto{\pgfqpoint{2.689013in}{3.169045in}}%
\pgfpathlineto{\pgfqpoint{2.693554in}{3.169045in}}%
\pgfpathlineto{\pgfqpoint{2.693554in}{3.166096in}}%
\pgfpathmoveto{\pgfqpoint{2.643601in}{3.192639in}}%
\pgfpathlineto{\pgfqpoint{2.643601in}{3.192639in}}%
\pgfpathlineto{\pgfqpoint{2.643601in}{3.195588in}}%
\pgfpathlineto{\pgfqpoint{2.648142in}{3.195588in}}%
\pgfpathlineto{\pgfqpoint{2.648142in}{3.192639in}}%
\pgfpathmoveto{\pgfqpoint{2.639060in}{3.195588in}}%
\pgfpathlineto{\pgfqpoint{2.639060in}{3.195588in}}%
\pgfpathlineto{\pgfqpoint{2.639060in}{3.198538in}}%
\pgfpathlineto{\pgfqpoint{2.643601in}{3.198538in}}%
\pgfpathlineto{\pgfqpoint{2.643601in}{3.195588in}}%
\pgfpathmoveto{\pgfqpoint{2.639060in}{3.198538in}}%
\pgfpathlineto{\pgfqpoint{2.639060in}{3.198538in}}%
\pgfpathlineto{\pgfqpoint{2.639060in}{3.201487in}}%
\pgfpathlineto{\pgfqpoint{2.643601in}{3.201487in}}%
\pgfpathlineto{\pgfqpoint{2.643601in}{3.198538in}}%
\pgfpathmoveto{\pgfqpoint{2.643601in}{3.195588in}}%
\pgfpathlineto{\pgfqpoint{2.643601in}{3.195588in}}%
\pgfpathlineto{\pgfqpoint{2.643601in}{3.198538in}}%
\pgfpathlineto{\pgfqpoint{2.648142in}{3.198538in}}%
\pgfpathlineto{\pgfqpoint{2.648142in}{3.195588in}}%
\pgfpathmoveto{\pgfqpoint{2.648142in}{3.189690in}}%
\pgfpathlineto{\pgfqpoint{2.648142in}{3.189690in}}%
\pgfpathlineto{\pgfqpoint{2.648142in}{3.192639in}}%
\pgfpathlineto{\pgfqpoint{2.652683in}{3.192639in}}%
\pgfpathlineto{\pgfqpoint{2.652683in}{3.189690in}}%
\pgfpathmoveto{\pgfqpoint{2.648142in}{3.192639in}}%
\pgfpathlineto{\pgfqpoint{2.648142in}{3.192639in}}%
\pgfpathlineto{\pgfqpoint{2.648142in}{3.195588in}}%
\pgfpathlineto{\pgfqpoint{2.652683in}{3.195588in}}%
\pgfpathlineto{\pgfqpoint{2.652683in}{3.192639in}}%
\pgfpathmoveto{\pgfqpoint{2.652683in}{3.189690in}}%
\pgfpathlineto{\pgfqpoint{2.652683in}{3.189690in}}%
\pgfpathlineto{\pgfqpoint{2.652683in}{3.192639in}}%
\pgfpathlineto{\pgfqpoint{2.657224in}{3.192639in}}%
\pgfpathlineto{\pgfqpoint{2.657224in}{3.189690in}}%
\pgfpathmoveto{\pgfqpoint{2.716260in}{3.145451in}}%
\pgfpathlineto{\pgfqpoint{2.716260in}{3.145451in}}%
\pgfpathlineto{\pgfqpoint{2.716260in}{3.148400in}}%
\pgfpathlineto{\pgfqpoint{2.720801in}{3.148400in}}%
\pgfpathlineto{\pgfqpoint{2.720801in}{3.145451in}}%
\pgfpathmoveto{\pgfqpoint{2.711719in}{3.148400in}}%
\pgfpathlineto{\pgfqpoint{2.711719in}{3.148400in}}%
\pgfpathlineto{\pgfqpoint{2.711719in}{3.151350in}}%
\pgfpathlineto{\pgfqpoint{2.716260in}{3.151350in}}%
\pgfpathlineto{\pgfqpoint{2.716260in}{3.148400in}}%
\pgfpathmoveto{\pgfqpoint{2.711719in}{3.151350in}}%
\pgfpathlineto{\pgfqpoint{2.711719in}{3.151350in}}%
\pgfpathlineto{\pgfqpoint{2.711719in}{3.154299in}}%
\pgfpathlineto{\pgfqpoint{2.716260in}{3.154299in}}%
\pgfpathlineto{\pgfqpoint{2.716260in}{3.151350in}}%
\pgfpathmoveto{\pgfqpoint{2.716260in}{3.148400in}}%
\pgfpathlineto{\pgfqpoint{2.716260in}{3.148400in}}%
\pgfpathlineto{\pgfqpoint{2.716260in}{3.151350in}}%
\pgfpathlineto{\pgfqpoint{2.720801in}{3.151350in}}%
\pgfpathlineto{\pgfqpoint{2.720801in}{3.148400in}}%
\pgfpathmoveto{\pgfqpoint{2.720801in}{3.142502in}}%
\pgfpathlineto{\pgfqpoint{2.720801in}{3.142502in}}%
\pgfpathlineto{\pgfqpoint{2.720801in}{3.145451in}}%
\pgfpathlineto{\pgfqpoint{2.725342in}{3.145451in}}%
\pgfpathlineto{\pgfqpoint{2.725342in}{3.142502in}}%
\pgfpathmoveto{\pgfqpoint{2.720801in}{3.145451in}}%
\pgfpathlineto{\pgfqpoint{2.720801in}{3.145451in}}%
\pgfpathlineto{\pgfqpoint{2.720801in}{3.148400in}}%
\pgfpathlineto{\pgfqpoint{2.725342in}{3.148400in}}%
\pgfpathlineto{\pgfqpoint{2.725342in}{3.145451in}}%
\pgfpathmoveto{\pgfqpoint{2.725342in}{3.142502in}}%
\pgfpathlineto{\pgfqpoint{2.725342in}{3.142502in}}%
\pgfpathlineto{\pgfqpoint{2.725342in}{3.145451in}}%
\pgfpathlineto{\pgfqpoint{2.729883in}{3.145451in}}%
\pgfpathlineto{\pgfqpoint{2.729883in}{3.142502in}}%
\pgfpathmoveto{\pgfqpoint{2.870654in}{3.045178in}}%
\pgfpathlineto{\pgfqpoint{2.870654in}{3.045178in}}%
\pgfpathlineto{\pgfqpoint{2.870654in}{3.048128in}}%
\pgfpathlineto{\pgfqpoint{2.875194in}{3.048128in}}%
\pgfpathlineto{\pgfqpoint{2.875194in}{3.045178in}}%
\pgfpathmoveto{\pgfqpoint{2.888817in}{3.033381in}}%
\pgfpathlineto{\pgfqpoint{2.888817in}{3.033381in}}%
\pgfpathlineto{\pgfqpoint{2.888817in}{3.036330in}}%
\pgfpathlineto{\pgfqpoint{2.893358in}{3.036330in}}%
\pgfpathlineto{\pgfqpoint{2.893358in}{3.033381in}}%
\pgfpathmoveto{\pgfqpoint{2.879735in}{3.039280in}}%
\pgfpathlineto{\pgfqpoint{2.879735in}{3.039280in}}%
\pgfpathlineto{\pgfqpoint{2.879735in}{3.042229in}}%
\pgfpathlineto{\pgfqpoint{2.884276in}{3.042229in}}%
\pgfpathlineto{\pgfqpoint{2.884276in}{3.039280in}}%
\pgfpathmoveto{\pgfqpoint{2.875194in}{3.042229in}}%
\pgfpathlineto{\pgfqpoint{2.875194in}{3.042229in}}%
\pgfpathlineto{\pgfqpoint{2.875194in}{3.045178in}}%
\pgfpathlineto{\pgfqpoint{2.879735in}{3.045178in}}%
\pgfpathlineto{\pgfqpoint{2.879735in}{3.042229in}}%
\pgfpathmoveto{\pgfqpoint{2.875194in}{3.045178in}}%
\pgfpathlineto{\pgfqpoint{2.875194in}{3.045178in}}%
\pgfpathlineto{\pgfqpoint{2.875194in}{3.048128in}}%
\pgfpathlineto{\pgfqpoint{2.879735in}{3.048128in}}%
\pgfpathlineto{\pgfqpoint{2.879735in}{3.045178in}}%
\pgfpathmoveto{\pgfqpoint{2.879735in}{3.042229in}}%
\pgfpathlineto{\pgfqpoint{2.879735in}{3.042229in}}%
\pgfpathlineto{\pgfqpoint{2.879735in}{3.045178in}}%
\pgfpathlineto{\pgfqpoint{2.884276in}{3.045178in}}%
\pgfpathlineto{\pgfqpoint{2.884276in}{3.042229in}}%
\pgfpathmoveto{\pgfqpoint{2.884276in}{3.036330in}}%
\pgfpathlineto{\pgfqpoint{2.884276in}{3.036330in}}%
\pgfpathlineto{\pgfqpoint{2.884276in}{3.039280in}}%
\pgfpathlineto{\pgfqpoint{2.888817in}{3.039280in}}%
\pgfpathlineto{\pgfqpoint{2.888817in}{3.036330in}}%
\pgfpathmoveto{\pgfqpoint{2.884276in}{3.039280in}}%
\pgfpathlineto{\pgfqpoint{2.884276in}{3.039280in}}%
\pgfpathlineto{\pgfqpoint{2.884276in}{3.042229in}}%
\pgfpathlineto{\pgfqpoint{2.888817in}{3.042229in}}%
\pgfpathlineto{\pgfqpoint{2.888817in}{3.039280in}}%
\pgfpathmoveto{\pgfqpoint{2.888817in}{3.036330in}}%
\pgfpathlineto{\pgfqpoint{2.888817in}{3.036330in}}%
\pgfpathlineto{\pgfqpoint{2.888817in}{3.039280in}}%
\pgfpathlineto{\pgfqpoint{2.893358in}{3.039280in}}%
\pgfpathlineto{\pgfqpoint{2.893358in}{3.036330in}}%
\pgfpathmoveto{\pgfqpoint{2.906980in}{3.021584in}}%
\pgfpathlineto{\pgfqpoint{2.906980in}{3.021584in}}%
\pgfpathlineto{\pgfqpoint{2.906980in}{3.024533in}}%
\pgfpathlineto{\pgfqpoint{2.911521in}{3.024533in}}%
\pgfpathlineto{\pgfqpoint{2.911521in}{3.021584in}}%
\pgfpathmoveto{\pgfqpoint{2.925144in}{3.009786in}}%
\pgfpathlineto{\pgfqpoint{2.925144in}{3.009786in}}%
\pgfpathlineto{\pgfqpoint{2.925144in}{3.012736in}}%
\pgfpathlineto{\pgfqpoint{2.929684in}{3.012736in}}%
\pgfpathlineto{\pgfqpoint{2.929684in}{3.009786in}}%
\pgfpathmoveto{\pgfqpoint{2.916062in}{3.015685in}}%
\pgfpathlineto{\pgfqpoint{2.916062in}{3.015685in}}%
\pgfpathlineto{\pgfqpoint{2.916062in}{3.018634in}}%
\pgfpathlineto{\pgfqpoint{2.920603in}{3.018634in}}%
\pgfpathlineto{\pgfqpoint{2.920603in}{3.015685in}}%
\pgfpathmoveto{\pgfqpoint{2.911521in}{3.018634in}}%
\pgfpathlineto{\pgfqpoint{2.911521in}{3.018634in}}%
\pgfpathlineto{\pgfqpoint{2.911521in}{3.021584in}}%
\pgfpathlineto{\pgfqpoint{2.916062in}{3.021584in}}%
\pgfpathlineto{\pgfqpoint{2.916062in}{3.018634in}}%
\pgfpathmoveto{\pgfqpoint{2.911521in}{3.021584in}}%
\pgfpathlineto{\pgfqpoint{2.911521in}{3.021584in}}%
\pgfpathlineto{\pgfqpoint{2.911521in}{3.024533in}}%
\pgfpathlineto{\pgfqpoint{2.916062in}{3.024533in}}%
\pgfpathlineto{\pgfqpoint{2.916062in}{3.021584in}}%
\pgfpathmoveto{\pgfqpoint{2.916062in}{3.018634in}}%
\pgfpathlineto{\pgfqpoint{2.916062in}{3.018634in}}%
\pgfpathlineto{\pgfqpoint{2.916062in}{3.021584in}}%
\pgfpathlineto{\pgfqpoint{2.920603in}{3.021584in}}%
\pgfpathlineto{\pgfqpoint{2.920603in}{3.018634in}}%
\pgfpathmoveto{\pgfqpoint{2.920603in}{3.012736in}}%
\pgfpathlineto{\pgfqpoint{2.920603in}{3.012736in}}%
\pgfpathlineto{\pgfqpoint{2.920603in}{3.015685in}}%
\pgfpathlineto{\pgfqpoint{2.925144in}{3.015685in}}%
\pgfpathlineto{\pgfqpoint{2.925144in}{3.012736in}}%
\pgfpathmoveto{\pgfqpoint{2.920603in}{3.015685in}}%
\pgfpathlineto{\pgfqpoint{2.920603in}{3.015685in}}%
\pgfpathlineto{\pgfqpoint{2.920603in}{3.018634in}}%
\pgfpathlineto{\pgfqpoint{2.925144in}{3.018634in}}%
\pgfpathlineto{\pgfqpoint{2.925144in}{3.015685in}}%
\pgfpathmoveto{\pgfqpoint{2.925144in}{3.012736in}}%
\pgfpathlineto{\pgfqpoint{2.925144in}{3.012736in}}%
\pgfpathlineto{\pgfqpoint{2.925144in}{3.015685in}}%
\pgfpathlineto{\pgfqpoint{2.929684in}{3.015685in}}%
\pgfpathlineto{\pgfqpoint{2.929684in}{3.012736in}}%
\pgfpathmoveto{\pgfqpoint{2.897899in}{3.027482in}}%
\pgfpathlineto{\pgfqpoint{2.897899in}{3.027482in}}%
\pgfpathlineto{\pgfqpoint{2.897899in}{3.030432in}}%
\pgfpathlineto{\pgfqpoint{2.902439in}{3.030432in}}%
\pgfpathlineto{\pgfqpoint{2.902439in}{3.027482in}}%
\pgfpathmoveto{\pgfqpoint{2.893358in}{3.030432in}}%
\pgfpathlineto{\pgfqpoint{2.893358in}{3.030432in}}%
\pgfpathlineto{\pgfqpoint{2.893358in}{3.033381in}}%
\pgfpathlineto{\pgfqpoint{2.897899in}{3.033381in}}%
\pgfpathlineto{\pgfqpoint{2.897899in}{3.030432in}}%
\pgfpathmoveto{\pgfqpoint{2.893358in}{3.033381in}}%
\pgfpathlineto{\pgfqpoint{2.893358in}{3.033381in}}%
\pgfpathlineto{\pgfqpoint{2.893358in}{3.036330in}}%
\pgfpathlineto{\pgfqpoint{2.897899in}{3.036330in}}%
\pgfpathlineto{\pgfqpoint{2.897899in}{3.033381in}}%
\pgfpathmoveto{\pgfqpoint{2.897899in}{3.030432in}}%
\pgfpathlineto{\pgfqpoint{2.897899in}{3.030432in}}%
\pgfpathlineto{\pgfqpoint{2.897899in}{3.033381in}}%
\pgfpathlineto{\pgfqpoint{2.902439in}{3.033381in}}%
\pgfpathlineto{\pgfqpoint{2.902439in}{3.030432in}}%
\pgfpathmoveto{\pgfqpoint{2.902439in}{3.024533in}}%
\pgfpathlineto{\pgfqpoint{2.902439in}{3.024533in}}%
\pgfpathlineto{\pgfqpoint{2.902439in}{3.027482in}}%
\pgfpathlineto{\pgfqpoint{2.906980in}{3.027482in}}%
\pgfpathlineto{\pgfqpoint{2.906980in}{3.024533in}}%
\pgfpathmoveto{\pgfqpoint{2.902439in}{3.027482in}}%
\pgfpathlineto{\pgfqpoint{2.902439in}{3.027482in}}%
\pgfpathlineto{\pgfqpoint{2.902439in}{3.030432in}}%
\pgfpathlineto{\pgfqpoint{2.906980in}{3.030432in}}%
\pgfpathlineto{\pgfqpoint{2.906980in}{3.027482in}}%
\pgfpathmoveto{\pgfqpoint{2.906980in}{3.024533in}}%
\pgfpathlineto{\pgfqpoint{2.906980in}{3.024533in}}%
\pgfpathlineto{\pgfqpoint{2.906980in}{3.027482in}}%
\pgfpathlineto{\pgfqpoint{2.911521in}{3.027482in}}%
\pgfpathlineto{\pgfqpoint{2.911521in}{3.024533in}}%
\pgfpathmoveto{\pgfqpoint{2.798000in}{3.092366in}}%
\pgfpathlineto{\pgfqpoint{2.798000in}{3.092366in}}%
\pgfpathlineto{\pgfqpoint{2.798000in}{3.095315in}}%
\pgfpathlineto{\pgfqpoint{2.802541in}{3.095315in}}%
\pgfpathlineto{\pgfqpoint{2.802541in}{3.092366in}}%
\pgfpathmoveto{\pgfqpoint{2.816164in}{3.080569in}}%
\pgfpathlineto{\pgfqpoint{2.816164in}{3.080569in}}%
\pgfpathlineto{\pgfqpoint{2.816164in}{3.083518in}}%
\pgfpathlineto{\pgfqpoint{2.820704in}{3.083518in}}%
\pgfpathlineto{\pgfqpoint{2.820704in}{3.080569in}}%
\pgfpathmoveto{\pgfqpoint{2.807082in}{3.086467in}}%
\pgfpathlineto{\pgfqpoint{2.807082in}{3.086467in}}%
\pgfpathlineto{\pgfqpoint{2.807082in}{3.089416in}}%
\pgfpathlineto{\pgfqpoint{2.811623in}{3.089416in}}%
\pgfpathlineto{\pgfqpoint{2.811623in}{3.086467in}}%
\pgfpathmoveto{\pgfqpoint{2.802541in}{3.089416in}}%
\pgfpathlineto{\pgfqpoint{2.802541in}{3.089416in}}%
\pgfpathlineto{\pgfqpoint{2.802541in}{3.092366in}}%
\pgfpathlineto{\pgfqpoint{2.807082in}{3.092366in}}%
\pgfpathlineto{\pgfqpoint{2.807082in}{3.089416in}}%
\pgfpathmoveto{\pgfqpoint{2.802541in}{3.092366in}}%
\pgfpathlineto{\pgfqpoint{2.802541in}{3.092366in}}%
\pgfpathlineto{\pgfqpoint{2.802541in}{3.095315in}}%
\pgfpathlineto{\pgfqpoint{2.807082in}{3.095315in}}%
\pgfpathlineto{\pgfqpoint{2.807082in}{3.092366in}}%
\pgfpathmoveto{\pgfqpoint{2.807082in}{3.089416in}}%
\pgfpathlineto{\pgfqpoint{2.807082in}{3.089416in}}%
\pgfpathlineto{\pgfqpoint{2.807082in}{3.092366in}}%
\pgfpathlineto{\pgfqpoint{2.811623in}{3.092366in}}%
\pgfpathlineto{\pgfqpoint{2.811623in}{3.089416in}}%
\pgfpathmoveto{\pgfqpoint{2.811623in}{3.083518in}}%
\pgfpathlineto{\pgfqpoint{2.811623in}{3.083518in}}%
\pgfpathlineto{\pgfqpoint{2.811623in}{3.086467in}}%
\pgfpathlineto{\pgfqpoint{2.816164in}{3.086467in}}%
\pgfpathlineto{\pgfqpoint{2.816164in}{3.083518in}}%
\pgfpathmoveto{\pgfqpoint{2.811623in}{3.086467in}}%
\pgfpathlineto{\pgfqpoint{2.811623in}{3.086467in}}%
\pgfpathlineto{\pgfqpoint{2.811623in}{3.089416in}}%
\pgfpathlineto{\pgfqpoint{2.816164in}{3.089416in}}%
\pgfpathlineto{\pgfqpoint{2.816164in}{3.086467in}}%
\pgfpathmoveto{\pgfqpoint{2.816164in}{3.083518in}}%
\pgfpathlineto{\pgfqpoint{2.816164in}{3.083518in}}%
\pgfpathlineto{\pgfqpoint{2.816164in}{3.086467in}}%
\pgfpathlineto{\pgfqpoint{2.820704in}{3.086467in}}%
\pgfpathlineto{\pgfqpoint{2.820704in}{3.083518in}}%
\pgfpathmoveto{\pgfqpoint{2.834327in}{3.068772in}}%
\pgfpathlineto{\pgfqpoint{2.834327in}{3.068772in}}%
\pgfpathlineto{\pgfqpoint{2.834327in}{3.071721in}}%
\pgfpathlineto{\pgfqpoint{2.838868in}{3.071721in}}%
\pgfpathlineto{\pgfqpoint{2.838868in}{3.068772in}}%
\pgfpathmoveto{\pgfqpoint{2.852490in}{3.056975in}}%
\pgfpathlineto{\pgfqpoint{2.852490in}{3.056975in}}%
\pgfpathlineto{\pgfqpoint{2.852490in}{3.059924in}}%
\pgfpathlineto{\pgfqpoint{2.857031in}{3.059924in}}%
\pgfpathlineto{\pgfqpoint{2.857031in}{3.056975in}}%
\pgfpathmoveto{\pgfqpoint{2.843409in}{3.062874in}}%
\pgfpathlineto{\pgfqpoint{2.843409in}{3.062874in}}%
\pgfpathlineto{\pgfqpoint{2.843409in}{3.065823in}}%
\pgfpathlineto{\pgfqpoint{2.847949in}{3.065823in}}%
\pgfpathlineto{\pgfqpoint{2.847949in}{3.062874in}}%
\pgfpathmoveto{\pgfqpoint{2.838868in}{3.065823in}}%
\pgfpathlineto{\pgfqpoint{2.838868in}{3.065823in}}%
\pgfpathlineto{\pgfqpoint{2.838868in}{3.068772in}}%
\pgfpathlineto{\pgfqpoint{2.843409in}{3.068772in}}%
\pgfpathlineto{\pgfqpoint{2.843409in}{3.065823in}}%
\pgfpathmoveto{\pgfqpoint{2.838868in}{3.068772in}}%
\pgfpathlineto{\pgfqpoint{2.838868in}{3.068772in}}%
\pgfpathlineto{\pgfqpoint{2.838868in}{3.071721in}}%
\pgfpathlineto{\pgfqpoint{2.843409in}{3.071721in}}%
\pgfpathlineto{\pgfqpoint{2.843409in}{3.068772in}}%
\pgfpathmoveto{\pgfqpoint{2.843409in}{3.065823in}}%
\pgfpathlineto{\pgfqpoint{2.843409in}{3.065823in}}%
\pgfpathlineto{\pgfqpoint{2.843409in}{3.068772in}}%
\pgfpathlineto{\pgfqpoint{2.847949in}{3.068772in}}%
\pgfpathlineto{\pgfqpoint{2.847949in}{3.065823in}}%
\pgfpathmoveto{\pgfqpoint{2.847949in}{3.059924in}}%
\pgfpathlineto{\pgfqpoint{2.847949in}{3.059924in}}%
\pgfpathlineto{\pgfqpoint{2.847949in}{3.062874in}}%
\pgfpathlineto{\pgfqpoint{2.852490in}{3.062874in}}%
\pgfpathlineto{\pgfqpoint{2.852490in}{3.059924in}}%
\pgfpathmoveto{\pgfqpoint{2.847949in}{3.062874in}}%
\pgfpathlineto{\pgfqpoint{2.847949in}{3.062874in}}%
\pgfpathlineto{\pgfqpoint{2.847949in}{3.065823in}}%
\pgfpathlineto{\pgfqpoint{2.852490in}{3.065823in}}%
\pgfpathlineto{\pgfqpoint{2.852490in}{3.062874in}}%
\pgfpathmoveto{\pgfqpoint{2.852490in}{3.059924in}}%
\pgfpathlineto{\pgfqpoint{2.852490in}{3.059924in}}%
\pgfpathlineto{\pgfqpoint{2.852490in}{3.062874in}}%
\pgfpathlineto{\pgfqpoint{2.857031in}{3.062874in}}%
\pgfpathlineto{\pgfqpoint{2.857031in}{3.059924in}}%
\pgfpathmoveto{\pgfqpoint{2.825245in}{3.074670in}}%
\pgfpathlineto{\pgfqpoint{2.825245in}{3.074670in}}%
\pgfpathlineto{\pgfqpoint{2.825245in}{3.077620in}}%
\pgfpathlineto{\pgfqpoint{2.829786in}{3.077620in}}%
\pgfpathlineto{\pgfqpoint{2.829786in}{3.074670in}}%
\pgfpathmoveto{\pgfqpoint{2.820704in}{3.077620in}}%
\pgfpathlineto{\pgfqpoint{2.820704in}{3.077620in}}%
\pgfpathlineto{\pgfqpoint{2.820704in}{3.080569in}}%
\pgfpathlineto{\pgfqpoint{2.825245in}{3.080569in}}%
\pgfpathlineto{\pgfqpoint{2.825245in}{3.077620in}}%
\pgfpathmoveto{\pgfqpoint{2.820704in}{3.080569in}}%
\pgfpathlineto{\pgfqpoint{2.820704in}{3.080569in}}%
\pgfpathlineto{\pgfqpoint{2.820704in}{3.083518in}}%
\pgfpathlineto{\pgfqpoint{2.825245in}{3.083518in}}%
\pgfpathlineto{\pgfqpoint{2.825245in}{3.080569in}}%
\pgfpathmoveto{\pgfqpoint{2.825245in}{3.077620in}}%
\pgfpathlineto{\pgfqpoint{2.825245in}{3.077620in}}%
\pgfpathlineto{\pgfqpoint{2.825245in}{3.080569in}}%
\pgfpathlineto{\pgfqpoint{2.829786in}{3.080569in}}%
\pgfpathlineto{\pgfqpoint{2.829786in}{3.077620in}}%
\pgfpathmoveto{\pgfqpoint{2.829786in}{3.071721in}}%
\pgfpathlineto{\pgfqpoint{2.829786in}{3.071721in}}%
\pgfpathlineto{\pgfqpoint{2.829786in}{3.074670in}}%
\pgfpathlineto{\pgfqpoint{2.834327in}{3.074670in}}%
\pgfpathlineto{\pgfqpoint{2.834327in}{3.071721in}}%
\pgfpathmoveto{\pgfqpoint{2.829786in}{3.074670in}}%
\pgfpathlineto{\pgfqpoint{2.829786in}{3.074670in}}%
\pgfpathlineto{\pgfqpoint{2.829786in}{3.077620in}}%
\pgfpathlineto{\pgfqpoint{2.834327in}{3.077620in}}%
\pgfpathlineto{\pgfqpoint{2.834327in}{3.074670in}}%
\pgfpathmoveto{\pgfqpoint{2.834327in}{3.071721in}}%
\pgfpathlineto{\pgfqpoint{2.834327in}{3.071721in}}%
\pgfpathlineto{\pgfqpoint{2.834327in}{3.074670in}}%
\pgfpathlineto{\pgfqpoint{2.838868in}{3.074670in}}%
\pgfpathlineto{\pgfqpoint{2.838868in}{3.071721in}}%
\pgfpathmoveto{\pgfqpoint{2.788919in}{3.098264in}}%
\pgfpathlineto{\pgfqpoint{2.788919in}{3.098264in}}%
\pgfpathlineto{\pgfqpoint{2.788919in}{3.101213in}}%
\pgfpathlineto{\pgfqpoint{2.793459in}{3.101213in}}%
\pgfpathlineto{\pgfqpoint{2.793459in}{3.098264in}}%
\pgfpathmoveto{\pgfqpoint{2.784378in}{3.101213in}}%
\pgfpathlineto{\pgfqpoint{2.784378in}{3.101213in}}%
\pgfpathlineto{\pgfqpoint{2.784378in}{3.104162in}}%
\pgfpathlineto{\pgfqpoint{2.788919in}{3.104162in}}%
\pgfpathlineto{\pgfqpoint{2.788919in}{3.101213in}}%
\pgfpathmoveto{\pgfqpoint{2.784378in}{3.104162in}}%
\pgfpathlineto{\pgfqpoint{2.784378in}{3.104162in}}%
\pgfpathlineto{\pgfqpoint{2.784378in}{3.107112in}}%
\pgfpathlineto{\pgfqpoint{2.788919in}{3.107112in}}%
\pgfpathlineto{\pgfqpoint{2.788919in}{3.104162in}}%
\pgfpathmoveto{\pgfqpoint{2.788919in}{3.101213in}}%
\pgfpathlineto{\pgfqpoint{2.788919in}{3.101213in}}%
\pgfpathlineto{\pgfqpoint{2.788919in}{3.104162in}}%
\pgfpathlineto{\pgfqpoint{2.793459in}{3.104162in}}%
\pgfpathlineto{\pgfqpoint{2.793459in}{3.101213in}}%
\pgfpathmoveto{\pgfqpoint{2.793459in}{3.095315in}}%
\pgfpathlineto{\pgfqpoint{2.793459in}{3.095315in}}%
\pgfpathlineto{\pgfqpoint{2.793459in}{3.098264in}}%
\pgfpathlineto{\pgfqpoint{2.798000in}{3.098264in}}%
\pgfpathlineto{\pgfqpoint{2.798000in}{3.095315in}}%
\pgfpathmoveto{\pgfqpoint{2.793459in}{3.098264in}}%
\pgfpathlineto{\pgfqpoint{2.793459in}{3.098264in}}%
\pgfpathlineto{\pgfqpoint{2.793459in}{3.101213in}}%
\pgfpathlineto{\pgfqpoint{2.798000in}{3.101213in}}%
\pgfpathlineto{\pgfqpoint{2.798000in}{3.098264in}}%
\pgfpathmoveto{\pgfqpoint{2.798000in}{3.095315in}}%
\pgfpathlineto{\pgfqpoint{2.798000in}{3.095315in}}%
\pgfpathlineto{\pgfqpoint{2.798000in}{3.098264in}}%
\pgfpathlineto{\pgfqpoint{2.802541in}{3.098264in}}%
\pgfpathlineto{\pgfqpoint{2.802541in}{3.095315in}}%
\pgfpathmoveto{\pgfqpoint{2.861572in}{3.051077in}}%
\pgfpathlineto{\pgfqpoint{2.861572in}{3.051077in}}%
\pgfpathlineto{\pgfqpoint{2.861572in}{3.054026in}}%
\pgfpathlineto{\pgfqpoint{2.866113in}{3.054026in}}%
\pgfpathlineto{\pgfqpoint{2.866113in}{3.051077in}}%
\pgfpathmoveto{\pgfqpoint{2.857031in}{3.054026in}}%
\pgfpathlineto{\pgfqpoint{2.857031in}{3.054026in}}%
\pgfpathlineto{\pgfqpoint{2.857031in}{3.056975in}}%
\pgfpathlineto{\pgfqpoint{2.861572in}{3.056975in}}%
\pgfpathlineto{\pgfqpoint{2.861572in}{3.054026in}}%
\pgfpathmoveto{\pgfqpoint{2.857031in}{3.056975in}}%
\pgfpathlineto{\pgfqpoint{2.857031in}{3.056975in}}%
\pgfpathlineto{\pgfqpoint{2.857031in}{3.059924in}}%
\pgfpathlineto{\pgfqpoint{2.861572in}{3.059924in}}%
\pgfpathlineto{\pgfqpoint{2.861572in}{3.056975in}}%
\pgfpathmoveto{\pgfqpoint{2.861572in}{3.054026in}}%
\pgfpathlineto{\pgfqpoint{2.861572in}{3.054026in}}%
\pgfpathlineto{\pgfqpoint{2.861572in}{3.056975in}}%
\pgfpathlineto{\pgfqpoint{2.866113in}{3.056975in}}%
\pgfpathlineto{\pgfqpoint{2.866113in}{3.054026in}}%
\pgfpathmoveto{\pgfqpoint{2.866113in}{3.048128in}}%
\pgfpathlineto{\pgfqpoint{2.866113in}{3.048128in}}%
\pgfpathlineto{\pgfqpoint{2.866113in}{3.051077in}}%
\pgfpathlineto{\pgfqpoint{2.870654in}{3.051077in}}%
\pgfpathlineto{\pgfqpoint{2.870654in}{3.048128in}}%
\pgfpathmoveto{\pgfqpoint{2.866113in}{3.051077in}}%
\pgfpathlineto{\pgfqpoint{2.866113in}{3.051077in}}%
\pgfpathlineto{\pgfqpoint{2.866113in}{3.054026in}}%
\pgfpathlineto{\pgfqpoint{2.870654in}{3.054026in}}%
\pgfpathlineto{\pgfqpoint{2.870654in}{3.051077in}}%
\pgfpathmoveto{\pgfqpoint{2.870654in}{3.048128in}}%
\pgfpathlineto{\pgfqpoint{2.870654in}{3.048128in}}%
\pgfpathlineto{\pgfqpoint{2.870654in}{3.051077in}}%
\pgfpathlineto{\pgfqpoint{2.875194in}{3.051077in}}%
\pgfpathlineto{\pgfqpoint{2.875194in}{3.048128in}}%
\pgfpathmoveto{\pgfqpoint{3.015964in}{2.950800in}}%
\pgfpathlineto{\pgfqpoint{3.015964in}{2.950800in}}%
\pgfpathlineto{\pgfqpoint{3.015964in}{2.953749in}}%
\pgfpathlineto{\pgfqpoint{3.020505in}{2.953749in}}%
\pgfpathlineto{\pgfqpoint{3.020505in}{2.950800in}}%
\pgfpathmoveto{\pgfqpoint{3.034128in}{2.939003in}}%
\pgfpathlineto{\pgfqpoint{3.034128in}{2.939003in}}%
\pgfpathlineto{\pgfqpoint{3.034128in}{2.941952in}}%
\pgfpathlineto{\pgfqpoint{3.038669in}{2.941952in}}%
\pgfpathlineto{\pgfqpoint{3.038669in}{2.939003in}}%
\pgfpathmoveto{\pgfqpoint{3.025046in}{2.944902in}}%
\pgfpathlineto{\pgfqpoint{3.025046in}{2.944902in}}%
\pgfpathlineto{\pgfqpoint{3.025046in}{2.947851in}}%
\pgfpathlineto{\pgfqpoint{3.029587in}{2.947851in}}%
\pgfpathlineto{\pgfqpoint{3.029587in}{2.944902in}}%
\pgfpathmoveto{\pgfqpoint{3.020505in}{2.947851in}}%
\pgfpathlineto{\pgfqpoint{3.020505in}{2.947851in}}%
\pgfpathlineto{\pgfqpoint{3.020505in}{2.950800in}}%
\pgfpathlineto{\pgfqpoint{3.025046in}{2.950800in}}%
\pgfpathlineto{\pgfqpoint{3.025046in}{2.947851in}}%
\pgfpathmoveto{\pgfqpoint{3.020505in}{2.950800in}}%
\pgfpathlineto{\pgfqpoint{3.020505in}{2.950800in}}%
\pgfpathlineto{\pgfqpoint{3.020505in}{2.953749in}}%
\pgfpathlineto{\pgfqpoint{3.025046in}{2.953749in}}%
\pgfpathlineto{\pgfqpoint{3.025046in}{2.950800in}}%
\pgfpathmoveto{\pgfqpoint{3.025046in}{2.947851in}}%
\pgfpathlineto{\pgfqpoint{3.025046in}{2.947851in}}%
\pgfpathlineto{\pgfqpoint{3.025046in}{2.950800in}}%
\pgfpathlineto{\pgfqpoint{3.029587in}{2.950800in}}%
\pgfpathlineto{\pgfqpoint{3.029587in}{2.947851in}}%
\pgfpathmoveto{\pgfqpoint{3.029587in}{2.941952in}}%
\pgfpathlineto{\pgfqpoint{3.029587in}{2.941952in}}%
\pgfpathlineto{\pgfqpoint{3.029587in}{2.944902in}}%
\pgfpathlineto{\pgfqpoint{3.034128in}{2.944902in}}%
\pgfpathlineto{\pgfqpoint{3.034128in}{2.941952in}}%
\pgfpathmoveto{\pgfqpoint{3.029587in}{2.944902in}}%
\pgfpathlineto{\pgfqpoint{3.029587in}{2.944902in}}%
\pgfpathlineto{\pgfqpoint{3.029587in}{2.947851in}}%
\pgfpathlineto{\pgfqpoint{3.034128in}{2.947851in}}%
\pgfpathlineto{\pgfqpoint{3.034128in}{2.944902in}}%
\pgfpathmoveto{\pgfqpoint{3.034128in}{2.941952in}}%
\pgfpathlineto{\pgfqpoint{3.034128in}{2.941952in}}%
\pgfpathlineto{\pgfqpoint{3.034128in}{2.944902in}}%
\pgfpathlineto{\pgfqpoint{3.038669in}{2.944902in}}%
\pgfpathlineto{\pgfqpoint{3.038669in}{2.941952in}}%
\pgfpathmoveto{\pgfqpoint{3.052292in}{2.927206in}}%
\pgfpathlineto{\pgfqpoint{3.052292in}{2.927206in}}%
\pgfpathlineto{\pgfqpoint{3.052292in}{2.930155in}}%
\pgfpathlineto{\pgfqpoint{3.056833in}{2.930155in}}%
\pgfpathlineto{\pgfqpoint{3.056833in}{2.927206in}}%
\pgfpathmoveto{\pgfqpoint{3.070457in}{2.915409in}}%
\pgfpathlineto{\pgfqpoint{3.070457in}{2.915409in}}%
\pgfpathlineto{\pgfqpoint{3.070457in}{2.918359in}}%
\pgfpathlineto{\pgfqpoint{3.074998in}{2.918359in}}%
\pgfpathlineto{\pgfqpoint{3.074998in}{2.915409in}}%
\pgfpathmoveto{\pgfqpoint{3.061374in}{2.921308in}}%
\pgfpathlineto{\pgfqpoint{3.061374in}{2.921308in}}%
\pgfpathlineto{\pgfqpoint{3.061374in}{2.924257in}}%
\pgfpathlineto{\pgfqpoint{3.065915in}{2.924257in}}%
\pgfpathlineto{\pgfqpoint{3.065915in}{2.921308in}}%
\pgfpathmoveto{\pgfqpoint{3.056833in}{2.924257in}}%
\pgfpathlineto{\pgfqpoint{3.056833in}{2.924257in}}%
\pgfpathlineto{\pgfqpoint{3.056833in}{2.927206in}}%
\pgfpathlineto{\pgfqpoint{3.061374in}{2.927206in}}%
\pgfpathlineto{\pgfqpoint{3.061374in}{2.924257in}}%
\pgfpathmoveto{\pgfqpoint{3.056833in}{2.927206in}}%
\pgfpathlineto{\pgfqpoint{3.056833in}{2.927206in}}%
\pgfpathlineto{\pgfqpoint{3.056833in}{2.930155in}}%
\pgfpathlineto{\pgfqpoint{3.061374in}{2.930155in}}%
\pgfpathlineto{\pgfqpoint{3.061374in}{2.927206in}}%
\pgfpathmoveto{\pgfqpoint{3.061374in}{2.924257in}}%
\pgfpathlineto{\pgfqpoint{3.061374in}{2.924257in}}%
\pgfpathlineto{\pgfqpoint{3.061374in}{2.927206in}}%
\pgfpathlineto{\pgfqpoint{3.065915in}{2.927206in}}%
\pgfpathlineto{\pgfqpoint{3.065915in}{2.924257in}}%
\pgfpathmoveto{\pgfqpoint{3.065915in}{2.918359in}}%
\pgfpathlineto{\pgfqpoint{3.065915in}{2.918359in}}%
\pgfpathlineto{\pgfqpoint{3.065915in}{2.921308in}}%
\pgfpathlineto{\pgfqpoint{3.070457in}{2.921308in}}%
\pgfpathlineto{\pgfqpoint{3.070457in}{2.918359in}}%
\pgfpathmoveto{\pgfqpoint{3.065915in}{2.921308in}}%
\pgfpathlineto{\pgfqpoint{3.065915in}{2.921308in}}%
\pgfpathlineto{\pgfqpoint{3.065915in}{2.924257in}}%
\pgfpathlineto{\pgfqpoint{3.070457in}{2.924257in}}%
\pgfpathlineto{\pgfqpoint{3.070457in}{2.921308in}}%
\pgfpathmoveto{\pgfqpoint{3.070457in}{2.918359in}}%
\pgfpathlineto{\pgfqpoint{3.070457in}{2.918359in}}%
\pgfpathlineto{\pgfqpoint{3.070457in}{2.921308in}}%
\pgfpathlineto{\pgfqpoint{3.074998in}{2.921308in}}%
\pgfpathlineto{\pgfqpoint{3.074998in}{2.918359in}}%
\pgfpathmoveto{\pgfqpoint{3.043210in}{2.933105in}}%
\pgfpathlineto{\pgfqpoint{3.043210in}{2.933105in}}%
\pgfpathlineto{\pgfqpoint{3.043210in}{2.936054in}}%
\pgfpathlineto{\pgfqpoint{3.047751in}{2.936054in}}%
\pgfpathlineto{\pgfqpoint{3.047751in}{2.933105in}}%
\pgfpathmoveto{\pgfqpoint{3.038669in}{2.936054in}}%
\pgfpathlineto{\pgfqpoint{3.038669in}{2.936054in}}%
\pgfpathlineto{\pgfqpoint{3.038669in}{2.939003in}}%
\pgfpathlineto{\pgfqpoint{3.043210in}{2.939003in}}%
\pgfpathlineto{\pgfqpoint{3.043210in}{2.936054in}}%
\pgfpathmoveto{\pgfqpoint{3.038669in}{2.939003in}}%
\pgfpathlineto{\pgfqpoint{3.038669in}{2.939003in}}%
\pgfpathlineto{\pgfqpoint{3.038669in}{2.941952in}}%
\pgfpathlineto{\pgfqpoint{3.043210in}{2.941952in}}%
\pgfpathlineto{\pgfqpoint{3.043210in}{2.939003in}}%
\pgfpathmoveto{\pgfqpoint{3.043210in}{2.936054in}}%
\pgfpathlineto{\pgfqpoint{3.043210in}{2.936054in}}%
\pgfpathlineto{\pgfqpoint{3.043210in}{2.939003in}}%
\pgfpathlineto{\pgfqpoint{3.047751in}{2.939003in}}%
\pgfpathlineto{\pgfqpoint{3.047751in}{2.936054in}}%
\pgfpathmoveto{\pgfqpoint{3.047751in}{2.930155in}}%
\pgfpathlineto{\pgfqpoint{3.047751in}{2.930155in}}%
\pgfpathlineto{\pgfqpoint{3.047751in}{2.933105in}}%
\pgfpathlineto{\pgfqpoint{3.052292in}{2.933105in}}%
\pgfpathlineto{\pgfqpoint{3.052292in}{2.930155in}}%
\pgfpathmoveto{\pgfqpoint{3.047751in}{2.933105in}}%
\pgfpathlineto{\pgfqpoint{3.047751in}{2.933105in}}%
\pgfpathlineto{\pgfqpoint{3.047751in}{2.936054in}}%
\pgfpathlineto{\pgfqpoint{3.052292in}{2.936054in}}%
\pgfpathlineto{\pgfqpoint{3.052292in}{2.933105in}}%
\pgfpathmoveto{\pgfqpoint{3.052292in}{2.930155in}}%
\pgfpathlineto{\pgfqpoint{3.052292in}{2.930155in}}%
\pgfpathlineto{\pgfqpoint{3.052292in}{2.933105in}}%
\pgfpathlineto{\pgfqpoint{3.056833in}{2.933105in}}%
\pgfpathlineto{\pgfqpoint{3.056833in}{2.930155in}}%
\pgfpathmoveto{\pgfqpoint{2.943308in}{2.997989in}}%
\pgfpathlineto{\pgfqpoint{2.943308in}{2.997989in}}%
\pgfpathlineto{\pgfqpoint{2.943308in}{3.000938in}}%
\pgfpathlineto{\pgfqpoint{2.947849in}{3.000938in}}%
\pgfpathlineto{\pgfqpoint{2.947849in}{2.997989in}}%
\pgfpathmoveto{\pgfqpoint{2.961472in}{2.986192in}}%
\pgfpathlineto{\pgfqpoint{2.961472in}{2.986192in}}%
\pgfpathlineto{\pgfqpoint{2.961472in}{2.989141in}}%
\pgfpathlineto{\pgfqpoint{2.966013in}{2.989141in}}%
\pgfpathlineto{\pgfqpoint{2.966013in}{2.986192in}}%
\pgfpathmoveto{\pgfqpoint{2.952390in}{2.992090in}}%
\pgfpathlineto{\pgfqpoint{2.952390in}{2.992090in}}%
\pgfpathlineto{\pgfqpoint{2.952390in}{2.995040in}}%
\pgfpathlineto{\pgfqpoint{2.956931in}{2.995040in}}%
\pgfpathlineto{\pgfqpoint{2.956931in}{2.992090in}}%
\pgfpathmoveto{\pgfqpoint{2.947849in}{2.995040in}}%
\pgfpathlineto{\pgfqpoint{2.947849in}{2.995040in}}%
\pgfpathlineto{\pgfqpoint{2.947849in}{2.997989in}}%
\pgfpathlineto{\pgfqpoint{2.952390in}{2.997989in}}%
\pgfpathlineto{\pgfqpoint{2.952390in}{2.995040in}}%
\pgfpathmoveto{\pgfqpoint{2.947849in}{2.997989in}}%
\pgfpathlineto{\pgfqpoint{2.947849in}{2.997989in}}%
\pgfpathlineto{\pgfqpoint{2.947849in}{3.000938in}}%
\pgfpathlineto{\pgfqpoint{2.952390in}{3.000938in}}%
\pgfpathlineto{\pgfqpoint{2.952390in}{2.997989in}}%
\pgfpathmoveto{\pgfqpoint{2.952390in}{2.995040in}}%
\pgfpathlineto{\pgfqpoint{2.952390in}{2.995040in}}%
\pgfpathlineto{\pgfqpoint{2.952390in}{2.997989in}}%
\pgfpathlineto{\pgfqpoint{2.956931in}{2.997989in}}%
\pgfpathlineto{\pgfqpoint{2.956931in}{2.995040in}}%
\pgfpathmoveto{\pgfqpoint{2.956931in}{2.989141in}}%
\pgfpathlineto{\pgfqpoint{2.956931in}{2.989141in}}%
\pgfpathlineto{\pgfqpoint{2.956931in}{2.992090in}}%
\pgfpathlineto{\pgfqpoint{2.961472in}{2.992090in}}%
\pgfpathlineto{\pgfqpoint{2.961472in}{2.989141in}}%
\pgfpathmoveto{\pgfqpoint{2.956931in}{2.992090in}}%
\pgfpathlineto{\pgfqpoint{2.956931in}{2.992090in}}%
\pgfpathlineto{\pgfqpoint{2.956931in}{2.995040in}}%
\pgfpathlineto{\pgfqpoint{2.961472in}{2.995040in}}%
\pgfpathlineto{\pgfqpoint{2.961472in}{2.992090in}}%
\pgfpathmoveto{\pgfqpoint{2.961472in}{2.989141in}}%
\pgfpathlineto{\pgfqpoint{2.961472in}{2.989141in}}%
\pgfpathlineto{\pgfqpoint{2.961472in}{2.992090in}}%
\pgfpathlineto{\pgfqpoint{2.966013in}{2.992090in}}%
\pgfpathlineto{\pgfqpoint{2.966013in}{2.989141in}}%
\pgfpathmoveto{\pgfqpoint{2.979636in}{2.974395in}}%
\pgfpathlineto{\pgfqpoint{2.979636in}{2.974395in}}%
\pgfpathlineto{\pgfqpoint{2.979636in}{2.977344in}}%
\pgfpathlineto{\pgfqpoint{2.984177in}{2.977344in}}%
\pgfpathlineto{\pgfqpoint{2.984177in}{2.974395in}}%
\pgfpathmoveto{\pgfqpoint{2.997800in}{2.962597in}}%
\pgfpathlineto{\pgfqpoint{2.997800in}{2.962597in}}%
\pgfpathlineto{\pgfqpoint{2.997800in}{2.965547in}}%
\pgfpathlineto{\pgfqpoint{3.002341in}{2.965547in}}%
\pgfpathlineto{\pgfqpoint{3.002341in}{2.962597in}}%
\pgfpathmoveto{\pgfqpoint{2.988718in}{2.968496in}}%
\pgfpathlineto{\pgfqpoint{2.988718in}{2.968496in}}%
\pgfpathlineto{\pgfqpoint{2.988718in}{2.971445in}}%
\pgfpathlineto{\pgfqpoint{2.993259in}{2.971445in}}%
\pgfpathlineto{\pgfqpoint{2.993259in}{2.968496in}}%
\pgfpathmoveto{\pgfqpoint{2.984177in}{2.971445in}}%
\pgfpathlineto{\pgfqpoint{2.984177in}{2.971445in}}%
\pgfpathlineto{\pgfqpoint{2.984177in}{2.974395in}}%
\pgfpathlineto{\pgfqpoint{2.988718in}{2.974395in}}%
\pgfpathlineto{\pgfqpoint{2.988718in}{2.971445in}}%
\pgfpathmoveto{\pgfqpoint{2.984177in}{2.974395in}}%
\pgfpathlineto{\pgfqpoint{2.984177in}{2.974395in}}%
\pgfpathlineto{\pgfqpoint{2.984177in}{2.977344in}}%
\pgfpathlineto{\pgfqpoint{2.988718in}{2.977344in}}%
\pgfpathlineto{\pgfqpoint{2.988718in}{2.974395in}}%
\pgfpathmoveto{\pgfqpoint{2.988718in}{2.971445in}}%
\pgfpathlineto{\pgfqpoint{2.988718in}{2.971445in}}%
\pgfpathlineto{\pgfqpoint{2.988718in}{2.974395in}}%
\pgfpathlineto{\pgfqpoint{2.993259in}{2.974395in}}%
\pgfpathlineto{\pgfqpoint{2.993259in}{2.971445in}}%
\pgfpathmoveto{\pgfqpoint{2.993259in}{2.965547in}}%
\pgfpathlineto{\pgfqpoint{2.993259in}{2.965547in}}%
\pgfpathlineto{\pgfqpoint{2.993259in}{2.968496in}}%
\pgfpathlineto{\pgfqpoint{2.997800in}{2.968496in}}%
\pgfpathlineto{\pgfqpoint{2.997800in}{2.965547in}}%
\pgfpathmoveto{\pgfqpoint{2.993259in}{2.968496in}}%
\pgfpathlineto{\pgfqpoint{2.993259in}{2.968496in}}%
\pgfpathlineto{\pgfqpoint{2.993259in}{2.971445in}}%
\pgfpathlineto{\pgfqpoint{2.997800in}{2.971445in}}%
\pgfpathlineto{\pgfqpoint{2.997800in}{2.968496in}}%
\pgfpathmoveto{\pgfqpoint{2.997800in}{2.965547in}}%
\pgfpathlineto{\pgfqpoint{2.997800in}{2.965547in}}%
\pgfpathlineto{\pgfqpoint{2.997800in}{2.968496in}}%
\pgfpathlineto{\pgfqpoint{3.002341in}{2.968496in}}%
\pgfpathlineto{\pgfqpoint{3.002341in}{2.965547in}}%
\pgfpathmoveto{\pgfqpoint{2.970554in}{2.980293in}}%
\pgfpathlineto{\pgfqpoint{2.970554in}{2.980293in}}%
\pgfpathlineto{\pgfqpoint{2.970554in}{2.983243in}}%
\pgfpathlineto{\pgfqpoint{2.975095in}{2.983243in}}%
\pgfpathlineto{\pgfqpoint{2.975095in}{2.980293in}}%
\pgfpathmoveto{\pgfqpoint{2.966013in}{2.983243in}}%
\pgfpathlineto{\pgfqpoint{2.966013in}{2.983243in}}%
\pgfpathlineto{\pgfqpoint{2.966013in}{2.986192in}}%
\pgfpathlineto{\pgfqpoint{2.970554in}{2.986192in}}%
\pgfpathlineto{\pgfqpoint{2.970554in}{2.983243in}}%
\pgfpathmoveto{\pgfqpoint{2.966013in}{2.986192in}}%
\pgfpathlineto{\pgfqpoint{2.966013in}{2.986192in}}%
\pgfpathlineto{\pgfqpoint{2.966013in}{2.989141in}}%
\pgfpathlineto{\pgfqpoint{2.970554in}{2.989141in}}%
\pgfpathlineto{\pgfqpoint{2.970554in}{2.986192in}}%
\pgfpathmoveto{\pgfqpoint{2.970554in}{2.983243in}}%
\pgfpathlineto{\pgfqpoint{2.970554in}{2.983243in}}%
\pgfpathlineto{\pgfqpoint{2.970554in}{2.986192in}}%
\pgfpathlineto{\pgfqpoint{2.975095in}{2.986192in}}%
\pgfpathlineto{\pgfqpoint{2.975095in}{2.983243in}}%
\pgfpathmoveto{\pgfqpoint{2.975095in}{2.977344in}}%
\pgfpathlineto{\pgfqpoint{2.975095in}{2.977344in}}%
\pgfpathlineto{\pgfqpoint{2.975095in}{2.980293in}}%
\pgfpathlineto{\pgfqpoint{2.979636in}{2.980293in}}%
\pgfpathlineto{\pgfqpoint{2.979636in}{2.977344in}}%
\pgfpathmoveto{\pgfqpoint{2.975095in}{2.980293in}}%
\pgfpathlineto{\pgfqpoint{2.975095in}{2.980293in}}%
\pgfpathlineto{\pgfqpoint{2.975095in}{2.983243in}}%
\pgfpathlineto{\pgfqpoint{2.979636in}{2.983243in}}%
\pgfpathlineto{\pgfqpoint{2.979636in}{2.980293in}}%
\pgfpathmoveto{\pgfqpoint{2.979636in}{2.977344in}}%
\pgfpathlineto{\pgfqpoint{2.979636in}{2.977344in}}%
\pgfpathlineto{\pgfqpoint{2.979636in}{2.980293in}}%
\pgfpathlineto{\pgfqpoint{2.984177in}{2.980293in}}%
\pgfpathlineto{\pgfqpoint{2.984177in}{2.977344in}}%
\pgfpathmoveto{\pgfqpoint{2.934225in}{3.003888in}}%
\pgfpathlineto{\pgfqpoint{2.934225in}{3.003888in}}%
\pgfpathlineto{\pgfqpoint{2.934225in}{3.006837in}}%
\pgfpathlineto{\pgfqpoint{2.938766in}{3.006837in}}%
\pgfpathlineto{\pgfqpoint{2.938766in}{3.003888in}}%
\pgfpathmoveto{\pgfqpoint{2.929684in}{3.006837in}}%
\pgfpathlineto{\pgfqpoint{2.929684in}{3.006837in}}%
\pgfpathlineto{\pgfqpoint{2.929684in}{3.009786in}}%
\pgfpathlineto{\pgfqpoint{2.934225in}{3.009786in}}%
\pgfpathlineto{\pgfqpoint{2.934225in}{3.006837in}}%
\pgfpathmoveto{\pgfqpoint{2.929684in}{3.009786in}}%
\pgfpathlineto{\pgfqpoint{2.929684in}{3.009786in}}%
\pgfpathlineto{\pgfqpoint{2.929684in}{3.012736in}}%
\pgfpathlineto{\pgfqpoint{2.934225in}{3.012736in}}%
\pgfpathlineto{\pgfqpoint{2.934225in}{3.009786in}}%
\pgfpathmoveto{\pgfqpoint{2.934225in}{3.006837in}}%
\pgfpathlineto{\pgfqpoint{2.934225in}{3.006837in}}%
\pgfpathlineto{\pgfqpoint{2.934225in}{3.009786in}}%
\pgfpathlineto{\pgfqpoint{2.938766in}{3.009786in}}%
\pgfpathlineto{\pgfqpoint{2.938766in}{3.006837in}}%
\pgfpathmoveto{\pgfqpoint{2.938766in}{3.000938in}}%
\pgfpathlineto{\pgfqpoint{2.938766in}{3.000938in}}%
\pgfpathlineto{\pgfqpoint{2.938766in}{3.003888in}}%
\pgfpathlineto{\pgfqpoint{2.943308in}{3.003888in}}%
\pgfpathlineto{\pgfqpoint{2.943308in}{3.000938in}}%
\pgfpathmoveto{\pgfqpoint{2.938766in}{3.003888in}}%
\pgfpathlineto{\pgfqpoint{2.938766in}{3.003888in}}%
\pgfpathlineto{\pgfqpoint{2.938766in}{3.006837in}}%
\pgfpathlineto{\pgfqpoint{2.943308in}{3.006837in}}%
\pgfpathlineto{\pgfqpoint{2.943308in}{3.003888in}}%
\pgfpathmoveto{\pgfqpoint{2.943308in}{3.000938in}}%
\pgfpathlineto{\pgfqpoint{2.943308in}{3.000938in}}%
\pgfpathlineto{\pgfqpoint{2.943308in}{3.003888in}}%
\pgfpathlineto{\pgfqpoint{2.947849in}{3.003888in}}%
\pgfpathlineto{\pgfqpoint{2.947849in}{3.000938in}}%
\pgfpathmoveto{\pgfqpoint{3.006882in}{2.956699in}}%
\pgfpathlineto{\pgfqpoint{3.006882in}{2.956699in}}%
\pgfpathlineto{\pgfqpoint{3.006882in}{2.959648in}}%
\pgfpathlineto{\pgfqpoint{3.011423in}{2.959648in}}%
\pgfpathlineto{\pgfqpoint{3.011423in}{2.956699in}}%
\pgfpathmoveto{\pgfqpoint{3.002341in}{2.959648in}}%
\pgfpathlineto{\pgfqpoint{3.002341in}{2.959648in}}%
\pgfpathlineto{\pgfqpoint{3.002341in}{2.962597in}}%
\pgfpathlineto{\pgfqpoint{3.006882in}{2.962597in}}%
\pgfpathlineto{\pgfqpoint{3.006882in}{2.959648in}}%
\pgfpathmoveto{\pgfqpoint{3.002341in}{2.962597in}}%
\pgfpathlineto{\pgfqpoint{3.002341in}{2.962597in}}%
\pgfpathlineto{\pgfqpoint{3.002341in}{2.965547in}}%
\pgfpathlineto{\pgfqpoint{3.006882in}{2.965547in}}%
\pgfpathlineto{\pgfqpoint{3.006882in}{2.962597in}}%
\pgfpathmoveto{\pgfqpoint{3.006882in}{2.959648in}}%
\pgfpathlineto{\pgfqpoint{3.006882in}{2.959648in}}%
\pgfpathlineto{\pgfqpoint{3.006882in}{2.962597in}}%
\pgfpathlineto{\pgfqpoint{3.011423in}{2.962597in}}%
\pgfpathlineto{\pgfqpoint{3.011423in}{2.959648in}}%
\pgfpathmoveto{\pgfqpoint{3.011423in}{2.953749in}}%
\pgfpathlineto{\pgfqpoint{3.011423in}{2.953749in}}%
\pgfpathlineto{\pgfqpoint{3.011423in}{2.956699in}}%
\pgfpathlineto{\pgfqpoint{3.015964in}{2.956699in}}%
\pgfpathlineto{\pgfqpoint{3.015964in}{2.953749in}}%
\pgfpathmoveto{\pgfqpoint{3.011423in}{2.956699in}}%
\pgfpathlineto{\pgfqpoint{3.011423in}{2.956699in}}%
\pgfpathlineto{\pgfqpoint{3.011423in}{2.959648in}}%
\pgfpathlineto{\pgfqpoint{3.015964in}{2.959648in}}%
\pgfpathlineto{\pgfqpoint{3.015964in}{2.956699in}}%
\pgfpathmoveto{\pgfqpoint{3.015964in}{2.953749in}}%
\pgfpathlineto{\pgfqpoint{3.015964in}{2.953749in}}%
\pgfpathlineto{\pgfqpoint{3.015964in}{2.956699in}}%
\pgfpathlineto{\pgfqpoint{3.020505in}{2.956699in}}%
\pgfpathlineto{\pgfqpoint{3.020505in}{2.953749in}}%
\pgfpathmoveto{\pgfqpoint{3.161280in}{2.856425in}}%
\pgfpathlineto{\pgfqpoint{3.161280in}{2.856425in}}%
\pgfpathlineto{\pgfqpoint{3.161280in}{2.859374in}}%
\pgfpathlineto{\pgfqpoint{3.165821in}{2.859374in}}%
\pgfpathlineto{\pgfqpoint{3.165821in}{2.856425in}}%
\pgfpathmoveto{\pgfqpoint{3.179444in}{2.844629in}}%
\pgfpathlineto{\pgfqpoint{3.179444in}{2.844629in}}%
\pgfpathlineto{\pgfqpoint{3.179444in}{2.847578in}}%
\pgfpathlineto{\pgfqpoint{3.183985in}{2.847578in}}%
\pgfpathlineto{\pgfqpoint{3.183985in}{2.844629in}}%
\pgfpathmoveto{\pgfqpoint{3.170362in}{2.850527in}}%
\pgfpathlineto{\pgfqpoint{3.170362in}{2.850527in}}%
\pgfpathlineto{\pgfqpoint{3.170362in}{2.853476in}}%
\pgfpathlineto{\pgfqpoint{3.174903in}{2.853476in}}%
\pgfpathlineto{\pgfqpoint{3.174903in}{2.850527in}}%
\pgfpathmoveto{\pgfqpoint{3.165821in}{2.853476in}}%
\pgfpathlineto{\pgfqpoint{3.165821in}{2.853476in}}%
\pgfpathlineto{\pgfqpoint{3.165821in}{2.856425in}}%
\pgfpathlineto{\pgfqpoint{3.170362in}{2.856425in}}%
\pgfpathlineto{\pgfqpoint{3.170362in}{2.853476in}}%
\pgfpathmoveto{\pgfqpoint{3.165821in}{2.856425in}}%
\pgfpathlineto{\pgfqpoint{3.165821in}{2.856425in}}%
\pgfpathlineto{\pgfqpoint{3.165821in}{2.859374in}}%
\pgfpathlineto{\pgfqpoint{3.170362in}{2.859374in}}%
\pgfpathlineto{\pgfqpoint{3.170362in}{2.856425in}}%
\pgfpathmoveto{\pgfqpoint{3.170362in}{2.853476in}}%
\pgfpathlineto{\pgfqpoint{3.170362in}{2.853476in}}%
\pgfpathlineto{\pgfqpoint{3.170362in}{2.856425in}}%
\pgfpathlineto{\pgfqpoint{3.174903in}{2.856425in}}%
\pgfpathlineto{\pgfqpoint{3.174903in}{2.853476in}}%
\pgfpathmoveto{\pgfqpoint{3.174903in}{2.847578in}}%
\pgfpathlineto{\pgfqpoint{3.174903in}{2.847578in}}%
\pgfpathlineto{\pgfqpoint{3.174903in}{2.850527in}}%
\pgfpathlineto{\pgfqpoint{3.179444in}{2.850527in}}%
\pgfpathlineto{\pgfqpoint{3.179444in}{2.847578in}}%
\pgfpathmoveto{\pgfqpoint{3.174903in}{2.850527in}}%
\pgfpathlineto{\pgfqpoint{3.174903in}{2.850527in}}%
\pgfpathlineto{\pgfqpoint{3.174903in}{2.853476in}}%
\pgfpathlineto{\pgfqpoint{3.179444in}{2.853476in}}%
\pgfpathlineto{\pgfqpoint{3.179444in}{2.850527in}}%
\pgfpathmoveto{\pgfqpoint{3.179444in}{2.847578in}}%
\pgfpathlineto{\pgfqpoint{3.179444in}{2.847578in}}%
\pgfpathlineto{\pgfqpoint{3.179444in}{2.850527in}}%
\pgfpathlineto{\pgfqpoint{3.183985in}{2.850527in}}%
\pgfpathlineto{\pgfqpoint{3.183985in}{2.847578in}}%
\pgfpathmoveto{\pgfqpoint{3.197609in}{2.832832in}}%
\pgfpathlineto{\pgfqpoint{3.197609in}{2.832832in}}%
\pgfpathlineto{\pgfqpoint{3.197609in}{2.835781in}}%
\pgfpathlineto{\pgfqpoint{3.202150in}{2.835781in}}%
\pgfpathlineto{\pgfqpoint{3.202150in}{2.832832in}}%
\pgfpathmoveto{\pgfqpoint{3.215773in}{2.821036in}}%
\pgfpathlineto{\pgfqpoint{3.215773in}{2.821036in}}%
\pgfpathlineto{\pgfqpoint{3.215773in}{2.823985in}}%
\pgfpathlineto{\pgfqpoint{3.220315in}{2.823985in}}%
\pgfpathlineto{\pgfqpoint{3.220315in}{2.821036in}}%
\pgfpathmoveto{\pgfqpoint{3.206691in}{2.826934in}}%
\pgfpathlineto{\pgfqpoint{3.206691in}{2.826934in}}%
\pgfpathlineto{\pgfqpoint{3.206691in}{2.829883in}}%
\pgfpathlineto{\pgfqpoint{3.211232in}{2.829883in}}%
\pgfpathlineto{\pgfqpoint{3.211232in}{2.826934in}}%
\pgfpathmoveto{\pgfqpoint{3.202150in}{2.829883in}}%
\pgfpathlineto{\pgfqpoint{3.202150in}{2.829883in}}%
\pgfpathlineto{\pgfqpoint{3.202150in}{2.832832in}}%
\pgfpathlineto{\pgfqpoint{3.206691in}{2.832832in}}%
\pgfpathlineto{\pgfqpoint{3.206691in}{2.829883in}}%
\pgfpathmoveto{\pgfqpoint{3.202150in}{2.832832in}}%
\pgfpathlineto{\pgfqpoint{3.202150in}{2.832832in}}%
\pgfpathlineto{\pgfqpoint{3.202150in}{2.835781in}}%
\pgfpathlineto{\pgfqpoint{3.206691in}{2.835781in}}%
\pgfpathlineto{\pgfqpoint{3.206691in}{2.832832in}}%
\pgfpathmoveto{\pgfqpoint{3.206691in}{2.829883in}}%
\pgfpathlineto{\pgfqpoint{3.206691in}{2.829883in}}%
\pgfpathlineto{\pgfqpoint{3.206691in}{2.832832in}}%
\pgfpathlineto{\pgfqpoint{3.211232in}{2.832832in}}%
\pgfpathlineto{\pgfqpoint{3.211232in}{2.829883in}}%
\pgfpathmoveto{\pgfqpoint{3.211232in}{2.823985in}}%
\pgfpathlineto{\pgfqpoint{3.211232in}{2.823985in}}%
\pgfpathlineto{\pgfqpoint{3.211232in}{2.826934in}}%
\pgfpathlineto{\pgfqpoint{3.215773in}{2.826934in}}%
\pgfpathlineto{\pgfqpoint{3.215773in}{2.823985in}}%
\pgfpathmoveto{\pgfqpoint{3.211232in}{2.826934in}}%
\pgfpathlineto{\pgfqpoint{3.211232in}{2.826934in}}%
\pgfpathlineto{\pgfqpoint{3.211232in}{2.829883in}}%
\pgfpathlineto{\pgfqpoint{3.215773in}{2.829883in}}%
\pgfpathlineto{\pgfqpoint{3.215773in}{2.826934in}}%
\pgfpathmoveto{\pgfqpoint{3.215773in}{2.823985in}}%
\pgfpathlineto{\pgfqpoint{3.215773in}{2.823985in}}%
\pgfpathlineto{\pgfqpoint{3.215773in}{2.826934in}}%
\pgfpathlineto{\pgfqpoint{3.220315in}{2.826934in}}%
\pgfpathlineto{\pgfqpoint{3.220315in}{2.823985in}}%
\pgfpathmoveto{\pgfqpoint{3.188526in}{2.838730in}}%
\pgfpathlineto{\pgfqpoint{3.188526in}{2.838730in}}%
\pgfpathlineto{\pgfqpoint{3.188526in}{2.841679in}}%
\pgfpathlineto{\pgfqpoint{3.193068in}{2.841679in}}%
\pgfpathlineto{\pgfqpoint{3.193068in}{2.838730in}}%
\pgfpathmoveto{\pgfqpoint{3.183985in}{2.841679in}}%
\pgfpathlineto{\pgfqpoint{3.183985in}{2.841679in}}%
\pgfpathlineto{\pgfqpoint{3.183985in}{2.844629in}}%
\pgfpathlineto{\pgfqpoint{3.188526in}{2.844629in}}%
\pgfpathlineto{\pgfqpoint{3.188526in}{2.841679in}}%
\pgfpathmoveto{\pgfqpoint{3.183985in}{2.844629in}}%
\pgfpathlineto{\pgfqpoint{3.183985in}{2.844629in}}%
\pgfpathlineto{\pgfqpoint{3.183985in}{2.847578in}}%
\pgfpathlineto{\pgfqpoint{3.188526in}{2.847578in}}%
\pgfpathlineto{\pgfqpoint{3.188526in}{2.844629in}}%
\pgfpathmoveto{\pgfqpoint{3.188526in}{2.841679in}}%
\pgfpathlineto{\pgfqpoint{3.188526in}{2.841679in}}%
\pgfpathlineto{\pgfqpoint{3.188526in}{2.844629in}}%
\pgfpathlineto{\pgfqpoint{3.193068in}{2.844629in}}%
\pgfpathlineto{\pgfqpoint{3.193068in}{2.841679in}}%
\pgfpathmoveto{\pgfqpoint{3.193068in}{2.835781in}}%
\pgfpathlineto{\pgfqpoint{3.193068in}{2.835781in}}%
\pgfpathlineto{\pgfqpoint{3.193068in}{2.838730in}}%
\pgfpathlineto{\pgfqpoint{3.197609in}{2.838730in}}%
\pgfpathlineto{\pgfqpoint{3.197609in}{2.835781in}}%
\pgfpathmoveto{\pgfqpoint{3.193068in}{2.838730in}}%
\pgfpathlineto{\pgfqpoint{3.193068in}{2.838730in}}%
\pgfpathlineto{\pgfqpoint{3.193068in}{2.841679in}}%
\pgfpathlineto{\pgfqpoint{3.197609in}{2.841679in}}%
\pgfpathlineto{\pgfqpoint{3.197609in}{2.838730in}}%
\pgfpathmoveto{\pgfqpoint{3.197609in}{2.835781in}}%
\pgfpathlineto{\pgfqpoint{3.197609in}{2.835781in}}%
\pgfpathlineto{\pgfqpoint{3.197609in}{2.838730in}}%
\pgfpathlineto{\pgfqpoint{3.202150in}{2.838730in}}%
\pgfpathlineto{\pgfqpoint{3.202150in}{2.835781in}}%
\pgfpathmoveto{\pgfqpoint{3.088621in}{2.903612in}}%
\pgfpathlineto{\pgfqpoint{3.088621in}{2.903612in}}%
\pgfpathlineto{\pgfqpoint{3.088621in}{2.906562in}}%
\pgfpathlineto{\pgfqpoint{3.093162in}{2.906562in}}%
\pgfpathlineto{\pgfqpoint{3.093162in}{2.903612in}}%
\pgfpathmoveto{\pgfqpoint{3.106786in}{2.891816in}}%
\pgfpathlineto{\pgfqpoint{3.106786in}{2.891816in}}%
\pgfpathlineto{\pgfqpoint{3.106786in}{2.894765in}}%
\pgfpathlineto{\pgfqpoint{3.111327in}{2.894765in}}%
\pgfpathlineto{\pgfqpoint{3.111327in}{2.891816in}}%
\pgfpathmoveto{\pgfqpoint{3.097703in}{2.897714in}}%
\pgfpathlineto{\pgfqpoint{3.097703in}{2.897714in}}%
\pgfpathlineto{\pgfqpoint{3.097703in}{2.900663in}}%
\pgfpathlineto{\pgfqpoint{3.102244in}{2.900663in}}%
\pgfpathlineto{\pgfqpoint{3.102244in}{2.897714in}}%
\pgfpathmoveto{\pgfqpoint{3.093162in}{2.900663in}}%
\pgfpathlineto{\pgfqpoint{3.093162in}{2.900663in}}%
\pgfpathlineto{\pgfqpoint{3.093162in}{2.903612in}}%
\pgfpathlineto{\pgfqpoint{3.097703in}{2.903612in}}%
\pgfpathlineto{\pgfqpoint{3.097703in}{2.900663in}}%
\pgfpathmoveto{\pgfqpoint{3.093162in}{2.903612in}}%
\pgfpathlineto{\pgfqpoint{3.093162in}{2.903612in}}%
\pgfpathlineto{\pgfqpoint{3.093162in}{2.906562in}}%
\pgfpathlineto{\pgfqpoint{3.097703in}{2.906562in}}%
\pgfpathlineto{\pgfqpoint{3.097703in}{2.903612in}}%
\pgfpathmoveto{\pgfqpoint{3.097703in}{2.900663in}}%
\pgfpathlineto{\pgfqpoint{3.097703in}{2.900663in}}%
\pgfpathlineto{\pgfqpoint{3.097703in}{2.903612in}}%
\pgfpathlineto{\pgfqpoint{3.102244in}{2.903612in}}%
\pgfpathlineto{\pgfqpoint{3.102244in}{2.900663in}}%
\pgfpathmoveto{\pgfqpoint{3.102244in}{2.894765in}}%
\pgfpathlineto{\pgfqpoint{3.102244in}{2.894765in}}%
\pgfpathlineto{\pgfqpoint{3.102244in}{2.897714in}}%
\pgfpathlineto{\pgfqpoint{3.106786in}{2.897714in}}%
\pgfpathlineto{\pgfqpoint{3.106786in}{2.894765in}}%
\pgfpathmoveto{\pgfqpoint{3.102244in}{2.897714in}}%
\pgfpathlineto{\pgfqpoint{3.102244in}{2.897714in}}%
\pgfpathlineto{\pgfqpoint{3.102244in}{2.900663in}}%
\pgfpathlineto{\pgfqpoint{3.106786in}{2.900663in}}%
\pgfpathlineto{\pgfqpoint{3.106786in}{2.897714in}}%
\pgfpathmoveto{\pgfqpoint{3.106786in}{2.894765in}}%
\pgfpathlineto{\pgfqpoint{3.106786in}{2.894765in}}%
\pgfpathlineto{\pgfqpoint{3.106786in}{2.897714in}}%
\pgfpathlineto{\pgfqpoint{3.111327in}{2.897714in}}%
\pgfpathlineto{\pgfqpoint{3.111327in}{2.894765in}}%
\pgfpathmoveto{\pgfqpoint{3.124950in}{2.880019in}}%
\pgfpathlineto{\pgfqpoint{3.124950in}{2.880019in}}%
\pgfpathlineto{\pgfqpoint{3.124950in}{2.882968in}}%
\pgfpathlineto{\pgfqpoint{3.129491in}{2.882968in}}%
\pgfpathlineto{\pgfqpoint{3.129491in}{2.880019in}}%
\pgfpathmoveto{\pgfqpoint{3.143115in}{2.868222in}}%
\pgfpathlineto{\pgfqpoint{3.143115in}{2.868222in}}%
\pgfpathlineto{\pgfqpoint{3.143115in}{2.871171in}}%
\pgfpathlineto{\pgfqpoint{3.147656in}{2.871171in}}%
\pgfpathlineto{\pgfqpoint{3.147656in}{2.868222in}}%
\pgfpathmoveto{\pgfqpoint{3.134033in}{2.874120in}}%
\pgfpathlineto{\pgfqpoint{3.134033in}{2.874120in}}%
\pgfpathlineto{\pgfqpoint{3.134033in}{2.877069in}}%
\pgfpathlineto{\pgfqpoint{3.138574in}{2.877069in}}%
\pgfpathlineto{\pgfqpoint{3.138574in}{2.874120in}}%
\pgfpathmoveto{\pgfqpoint{3.129491in}{2.877069in}}%
\pgfpathlineto{\pgfqpoint{3.129491in}{2.877069in}}%
\pgfpathlineto{\pgfqpoint{3.129491in}{2.880019in}}%
\pgfpathlineto{\pgfqpoint{3.134033in}{2.880019in}}%
\pgfpathlineto{\pgfqpoint{3.134033in}{2.877069in}}%
\pgfpathmoveto{\pgfqpoint{3.129491in}{2.880019in}}%
\pgfpathlineto{\pgfqpoint{3.129491in}{2.880019in}}%
\pgfpathlineto{\pgfqpoint{3.129491in}{2.882968in}}%
\pgfpathlineto{\pgfqpoint{3.134033in}{2.882968in}}%
\pgfpathlineto{\pgfqpoint{3.134033in}{2.880019in}}%
\pgfpathmoveto{\pgfqpoint{3.134033in}{2.877069in}}%
\pgfpathlineto{\pgfqpoint{3.134033in}{2.877069in}}%
\pgfpathlineto{\pgfqpoint{3.134033in}{2.880019in}}%
\pgfpathlineto{\pgfqpoint{3.138574in}{2.880019in}}%
\pgfpathlineto{\pgfqpoint{3.138574in}{2.877069in}}%
\pgfpathmoveto{\pgfqpoint{3.138574in}{2.871171in}}%
\pgfpathlineto{\pgfqpoint{3.138574in}{2.871171in}}%
\pgfpathlineto{\pgfqpoint{3.138574in}{2.874120in}}%
\pgfpathlineto{\pgfqpoint{3.143115in}{2.874120in}}%
\pgfpathlineto{\pgfqpoint{3.143115in}{2.871171in}}%
\pgfpathmoveto{\pgfqpoint{3.138574in}{2.874120in}}%
\pgfpathlineto{\pgfqpoint{3.138574in}{2.874120in}}%
\pgfpathlineto{\pgfqpoint{3.138574in}{2.877069in}}%
\pgfpathlineto{\pgfqpoint{3.143115in}{2.877069in}}%
\pgfpathlineto{\pgfqpoint{3.143115in}{2.874120in}}%
\pgfpathmoveto{\pgfqpoint{3.143115in}{2.871171in}}%
\pgfpathlineto{\pgfqpoint{3.143115in}{2.871171in}}%
\pgfpathlineto{\pgfqpoint{3.143115in}{2.874120in}}%
\pgfpathlineto{\pgfqpoint{3.147656in}{2.874120in}}%
\pgfpathlineto{\pgfqpoint{3.147656in}{2.871171in}}%
\pgfpathmoveto{\pgfqpoint{3.115868in}{2.885917in}}%
\pgfpathlineto{\pgfqpoint{3.115868in}{2.885917in}}%
\pgfpathlineto{\pgfqpoint{3.115868in}{2.888866in}}%
\pgfpathlineto{\pgfqpoint{3.120409in}{2.888866in}}%
\pgfpathlineto{\pgfqpoint{3.120409in}{2.885917in}}%
\pgfpathmoveto{\pgfqpoint{3.111327in}{2.888866in}}%
\pgfpathlineto{\pgfqpoint{3.111327in}{2.888866in}}%
\pgfpathlineto{\pgfqpoint{3.111327in}{2.891816in}}%
\pgfpathlineto{\pgfqpoint{3.115868in}{2.891816in}}%
\pgfpathlineto{\pgfqpoint{3.115868in}{2.888866in}}%
\pgfpathmoveto{\pgfqpoint{3.111327in}{2.891816in}}%
\pgfpathlineto{\pgfqpoint{3.111327in}{2.891816in}}%
\pgfpathlineto{\pgfqpoint{3.111327in}{2.894765in}}%
\pgfpathlineto{\pgfqpoint{3.115868in}{2.894765in}}%
\pgfpathlineto{\pgfqpoint{3.115868in}{2.891816in}}%
\pgfpathmoveto{\pgfqpoint{3.115868in}{2.888866in}}%
\pgfpathlineto{\pgfqpoint{3.115868in}{2.888866in}}%
\pgfpathlineto{\pgfqpoint{3.115868in}{2.891816in}}%
\pgfpathlineto{\pgfqpoint{3.120409in}{2.891816in}}%
\pgfpathlineto{\pgfqpoint{3.120409in}{2.888866in}}%
\pgfpathmoveto{\pgfqpoint{3.120409in}{2.882968in}}%
\pgfpathlineto{\pgfqpoint{3.120409in}{2.882968in}}%
\pgfpathlineto{\pgfqpoint{3.120409in}{2.885917in}}%
\pgfpathlineto{\pgfqpoint{3.124950in}{2.885917in}}%
\pgfpathlineto{\pgfqpoint{3.124950in}{2.882968in}}%
\pgfpathmoveto{\pgfqpoint{3.120409in}{2.885917in}}%
\pgfpathlineto{\pgfqpoint{3.120409in}{2.885917in}}%
\pgfpathlineto{\pgfqpoint{3.120409in}{2.888866in}}%
\pgfpathlineto{\pgfqpoint{3.124950in}{2.888866in}}%
\pgfpathlineto{\pgfqpoint{3.124950in}{2.885917in}}%
\pgfpathmoveto{\pgfqpoint{3.124950in}{2.882968in}}%
\pgfpathlineto{\pgfqpoint{3.124950in}{2.882968in}}%
\pgfpathlineto{\pgfqpoint{3.124950in}{2.885917in}}%
\pgfpathlineto{\pgfqpoint{3.129491in}{2.885917in}}%
\pgfpathlineto{\pgfqpoint{3.129491in}{2.882968in}}%
\pgfpathmoveto{\pgfqpoint{3.079539in}{2.909511in}}%
\pgfpathlineto{\pgfqpoint{3.079539in}{2.909511in}}%
\pgfpathlineto{\pgfqpoint{3.079539in}{2.912460in}}%
\pgfpathlineto{\pgfqpoint{3.084080in}{2.912460in}}%
\pgfpathlineto{\pgfqpoint{3.084080in}{2.909511in}}%
\pgfpathmoveto{\pgfqpoint{3.074998in}{2.912460in}}%
\pgfpathlineto{\pgfqpoint{3.074998in}{2.912460in}}%
\pgfpathlineto{\pgfqpoint{3.074998in}{2.915409in}}%
\pgfpathlineto{\pgfqpoint{3.079539in}{2.915409in}}%
\pgfpathlineto{\pgfqpoint{3.079539in}{2.912460in}}%
\pgfpathmoveto{\pgfqpoint{3.074998in}{2.915409in}}%
\pgfpathlineto{\pgfqpoint{3.074998in}{2.915409in}}%
\pgfpathlineto{\pgfqpoint{3.074998in}{2.918359in}}%
\pgfpathlineto{\pgfqpoint{3.079539in}{2.918359in}}%
\pgfpathlineto{\pgfqpoint{3.079539in}{2.915409in}}%
\pgfpathmoveto{\pgfqpoint{3.079539in}{2.912460in}}%
\pgfpathlineto{\pgfqpoint{3.079539in}{2.912460in}}%
\pgfpathlineto{\pgfqpoint{3.079539in}{2.915409in}}%
\pgfpathlineto{\pgfqpoint{3.084080in}{2.915409in}}%
\pgfpathlineto{\pgfqpoint{3.084080in}{2.912460in}}%
\pgfpathmoveto{\pgfqpoint{3.084080in}{2.906562in}}%
\pgfpathlineto{\pgfqpoint{3.084080in}{2.906562in}}%
\pgfpathlineto{\pgfqpoint{3.084080in}{2.909511in}}%
\pgfpathlineto{\pgfqpoint{3.088621in}{2.909511in}}%
\pgfpathlineto{\pgfqpoint{3.088621in}{2.906562in}}%
\pgfpathmoveto{\pgfqpoint{3.084080in}{2.909511in}}%
\pgfpathlineto{\pgfqpoint{3.084080in}{2.909511in}}%
\pgfpathlineto{\pgfqpoint{3.084080in}{2.912460in}}%
\pgfpathlineto{\pgfqpoint{3.088621in}{2.912460in}}%
\pgfpathlineto{\pgfqpoint{3.088621in}{2.909511in}}%
\pgfpathmoveto{\pgfqpoint{3.088621in}{2.906562in}}%
\pgfpathlineto{\pgfqpoint{3.088621in}{2.906562in}}%
\pgfpathlineto{\pgfqpoint{3.088621in}{2.909511in}}%
\pgfpathlineto{\pgfqpoint{3.093162in}{2.909511in}}%
\pgfpathlineto{\pgfqpoint{3.093162in}{2.906562in}}%
\pgfpathmoveto{\pgfqpoint{3.152197in}{2.862323in}}%
\pgfpathlineto{\pgfqpoint{3.152197in}{2.862323in}}%
\pgfpathlineto{\pgfqpoint{3.152197in}{2.865273in}}%
\pgfpathlineto{\pgfqpoint{3.156738in}{2.865273in}}%
\pgfpathlineto{\pgfqpoint{3.156738in}{2.862323in}}%
\pgfpathmoveto{\pgfqpoint{3.147656in}{2.865273in}}%
\pgfpathlineto{\pgfqpoint{3.147656in}{2.865273in}}%
\pgfpathlineto{\pgfqpoint{3.147656in}{2.868222in}}%
\pgfpathlineto{\pgfqpoint{3.152197in}{2.868222in}}%
\pgfpathlineto{\pgfqpoint{3.152197in}{2.865273in}}%
\pgfpathmoveto{\pgfqpoint{3.147656in}{2.868222in}}%
\pgfpathlineto{\pgfqpoint{3.147656in}{2.868222in}}%
\pgfpathlineto{\pgfqpoint{3.147656in}{2.871171in}}%
\pgfpathlineto{\pgfqpoint{3.152197in}{2.871171in}}%
\pgfpathlineto{\pgfqpoint{3.152197in}{2.868222in}}%
\pgfpathmoveto{\pgfqpoint{3.152197in}{2.865273in}}%
\pgfpathlineto{\pgfqpoint{3.152197in}{2.865273in}}%
\pgfpathlineto{\pgfqpoint{3.152197in}{2.868222in}}%
\pgfpathlineto{\pgfqpoint{3.156738in}{2.868222in}}%
\pgfpathlineto{\pgfqpoint{3.156738in}{2.865273in}}%
\pgfpathmoveto{\pgfqpoint{3.156738in}{2.859374in}}%
\pgfpathlineto{\pgfqpoint{3.156738in}{2.859374in}}%
\pgfpathlineto{\pgfqpoint{3.156738in}{2.862323in}}%
\pgfpathlineto{\pgfqpoint{3.161280in}{2.862323in}}%
\pgfpathlineto{\pgfqpoint{3.161280in}{2.859374in}}%
\pgfpathmoveto{\pgfqpoint{3.156738in}{2.862323in}}%
\pgfpathlineto{\pgfqpoint{3.156738in}{2.862323in}}%
\pgfpathlineto{\pgfqpoint{3.156738in}{2.865273in}}%
\pgfpathlineto{\pgfqpoint{3.161280in}{2.865273in}}%
\pgfpathlineto{\pgfqpoint{3.161280in}{2.862323in}}%
\pgfpathmoveto{\pgfqpoint{3.161280in}{2.859374in}}%
\pgfpathlineto{\pgfqpoint{3.161280in}{2.859374in}}%
\pgfpathlineto{\pgfqpoint{3.161280in}{2.862323in}}%
\pgfpathlineto{\pgfqpoint{3.165821in}{2.862323in}}%
\pgfpathlineto{\pgfqpoint{3.165821in}{2.859374in}}%
\pgfpathmoveto{\pgfqpoint{3.306590in}{2.762053in}}%
\pgfpathlineto{\pgfqpoint{3.306590in}{2.762053in}}%
\pgfpathlineto{\pgfqpoint{3.306590in}{2.765003in}}%
\pgfpathlineto{\pgfqpoint{3.311131in}{2.765003in}}%
\pgfpathlineto{\pgfqpoint{3.311131in}{2.762053in}}%
\pgfpathmoveto{\pgfqpoint{3.324753in}{2.750256in}}%
\pgfpathlineto{\pgfqpoint{3.324753in}{2.750256in}}%
\pgfpathlineto{\pgfqpoint{3.324753in}{2.753205in}}%
\pgfpathlineto{\pgfqpoint{3.329294in}{2.753205in}}%
\pgfpathlineto{\pgfqpoint{3.329294in}{2.750256in}}%
\pgfpathmoveto{\pgfqpoint{3.315672in}{2.756155in}}%
\pgfpathlineto{\pgfqpoint{3.315672in}{2.756155in}}%
\pgfpathlineto{\pgfqpoint{3.315672in}{2.759104in}}%
\pgfpathlineto{\pgfqpoint{3.320213in}{2.759104in}}%
\pgfpathlineto{\pgfqpoint{3.320213in}{2.756155in}}%
\pgfpathmoveto{\pgfqpoint{3.311131in}{2.759104in}}%
\pgfpathlineto{\pgfqpoint{3.311131in}{2.759104in}}%
\pgfpathlineto{\pgfqpoint{3.311131in}{2.762053in}}%
\pgfpathlineto{\pgfqpoint{3.315672in}{2.762053in}}%
\pgfpathlineto{\pgfqpoint{3.315672in}{2.759104in}}%
\pgfpathmoveto{\pgfqpoint{3.311131in}{2.762053in}}%
\pgfpathlineto{\pgfqpoint{3.311131in}{2.762053in}}%
\pgfpathlineto{\pgfqpoint{3.311131in}{2.765003in}}%
\pgfpathlineto{\pgfqpoint{3.315672in}{2.765003in}}%
\pgfpathlineto{\pgfqpoint{3.315672in}{2.762053in}}%
\pgfpathmoveto{\pgfqpoint{3.315672in}{2.759104in}}%
\pgfpathlineto{\pgfqpoint{3.315672in}{2.759104in}}%
\pgfpathlineto{\pgfqpoint{3.315672in}{2.762053in}}%
\pgfpathlineto{\pgfqpoint{3.320213in}{2.762053in}}%
\pgfpathlineto{\pgfqpoint{3.320213in}{2.759104in}}%
\pgfpathmoveto{\pgfqpoint{3.320213in}{2.753205in}}%
\pgfpathlineto{\pgfqpoint{3.320213in}{2.753205in}}%
\pgfpathlineto{\pgfqpoint{3.320213in}{2.756155in}}%
\pgfpathlineto{\pgfqpoint{3.324753in}{2.756155in}}%
\pgfpathlineto{\pgfqpoint{3.324753in}{2.753205in}}%
\pgfpathmoveto{\pgfqpoint{3.320213in}{2.756155in}}%
\pgfpathlineto{\pgfqpoint{3.320213in}{2.756155in}}%
\pgfpathlineto{\pgfqpoint{3.320213in}{2.759104in}}%
\pgfpathlineto{\pgfqpoint{3.324753in}{2.759104in}}%
\pgfpathlineto{\pgfqpoint{3.324753in}{2.756155in}}%
\pgfpathmoveto{\pgfqpoint{3.324753in}{2.753205in}}%
\pgfpathlineto{\pgfqpoint{3.324753in}{2.753205in}}%
\pgfpathlineto{\pgfqpoint{3.324753in}{2.756155in}}%
\pgfpathlineto{\pgfqpoint{3.329294in}{2.756155in}}%
\pgfpathlineto{\pgfqpoint{3.329294in}{2.753205in}}%
\pgfpathmoveto{\pgfqpoint{3.342917in}{2.738459in}}%
\pgfpathlineto{\pgfqpoint{3.342917in}{2.738459in}}%
\pgfpathlineto{\pgfqpoint{3.342917in}{2.741408in}}%
\pgfpathlineto{\pgfqpoint{3.347458in}{2.741408in}}%
\pgfpathlineto{\pgfqpoint{3.347458in}{2.738459in}}%
\pgfpathmoveto{\pgfqpoint{3.361080in}{2.726662in}}%
\pgfpathlineto{\pgfqpoint{3.361080in}{2.726662in}}%
\pgfpathlineto{\pgfqpoint{3.361080in}{2.729611in}}%
\pgfpathlineto{\pgfqpoint{3.365621in}{2.729611in}}%
\pgfpathlineto{\pgfqpoint{3.365621in}{2.726662in}}%
\pgfpathmoveto{\pgfqpoint{3.351998in}{2.732560in}}%
\pgfpathlineto{\pgfqpoint{3.351998in}{2.732560in}}%
\pgfpathlineto{\pgfqpoint{3.351998in}{2.735510in}}%
\pgfpathlineto{\pgfqpoint{3.356539in}{2.735510in}}%
\pgfpathlineto{\pgfqpoint{3.356539in}{2.732560in}}%
\pgfpathmoveto{\pgfqpoint{3.347458in}{2.735510in}}%
\pgfpathlineto{\pgfqpoint{3.347458in}{2.735510in}}%
\pgfpathlineto{\pgfqpoint{3.347458in}{2.738459in}}%
\pgfpathlineto{\pgfqpoint{3.351998in}{2.738459in}}%
\pgfpathlineto{\pgfqpoint{3.351998in}{2.735510in}}%
\pgfpathmoveto{\pgfqpoint{3.347458in}{2.738459in}}%
\pgfpathlineto{\pgfqpoint{3.347458in}{2.738459in}}%
\pgfpathlineto{\pgfqpoint{3.347458in}{2.741408in}}%
\pgfpathlineto{\pgfqpoint{3.351998in}{2.741408in}}%
\pgfpathlineto{\pgfqpoint{3.351998in}{2.738459in}}%
\pgfpathmoveto{\pgfqpoint{3.351998in}{2.735510in}}%
\pgfpathlineto{\pgfqpoint{3.351998in}{2.735510in}}%
\pgfpathlineto{\pgfqpoint{3.351998in}{2.738459in}}%
\pgfpathlineto{\pgfqpoint{3.356539in}{2.738459in}}%
\pgfpathlineto{\pgfqpoint{3.356539in}{2.735510in}}%
\pgfpathmoveto{\pgfqpoint{3.356539in}{2.729611in}}%
\pgfpathlineto{\pgfqpoint{3.356539in}{2.729611in}}%
\pgfpathlineto{\pgfqpoint{3.356539in}{2.732560in}}%
\pgfpathlineto{\pgfqpoint{3.361080in}{2.732560in}}%
\pgfpathlineto{\pgfqpoint{3.361080in}{2.729611in}}%
\pgfpathmoveto{\pgfqpoint{3.356539in}{2.732560in}}%
\pgfpathlineto{\pgfqpoint{3.356539in}{2.732560in}}%
\pgfpathlineto{\pgfqpoint{3.356539in}{2.735510in}}%
\pgfpathlineto{\pgfqpoint{3.361080in}{2.735510in}}%
\pgfpathlineto{\pgfqpoint{3.361080in}{2.732560in}}%
\pgfpathmoveto{\pgfqpoint{3.361080in}{2.729611in}}%
\pgfpathlineto{\pgfqpoint{3.361080in}{2.729611in}}%
\pgfpathlineto{\pgfqpoint{3.361080in}{2.732560in}}%
\pgfpathlineto{\pgfqpoint{3.365621in}{2.732560in}}%
\pgfpathlineto{\pgfqpoint{3.365621in}{2.729611in}}%
\pgfpathmoveto{\pgfqpoint{3.333835in}{2.744358in}}%
\pgfpathlineto{\pgfqpoint{3.333835in}{2.744358in}}%
\pgfpathlineto{\pgfqpoint{3.333835in}{2.747307in}}%
\pgfpathlineto{\pgfqpoint{3.338376in}{2.747307in}}%
\pgfpathlineto{\pgfqpoint{3.338376in}{2.744358in}}%
\pgfpathmoveto{\pgfqpoint{3.329294in}{2.747307in}}%
\pgfpathlineto{\pgfqpoint{3.329294in}{2.747307in}}%
\pgfpathlineto{\pgfqpoint{3.329294in}{2.750256in}}%
\pgfpathlineto{\pgfqpoint{3.333835in}{2.750256in}}%
\pgfpathlineto{\pgfqpoint{3.333835in}{2.747307in}}%
\pgfpathmoveto{\pgfqpoint{3.329294in}{2.750256in}}%
\pgfpathlineto{\pgfqpoint{3.329294in}{2.750256in}}%
\pgfpathlineto{\pgfqpoint{3.329294in}{2.753205in}}%
\pgfpathlineto{\pgfqpoint{3.333835in}{2.753205in}}%
\pgfpathlineto{\pgfqpoint{3.333835in}{2.750256in}}%
\pgfpathmoveto{\pgfqpoint{3.333835in}{2.747307in}}%
\pgfpathlineto{\pgfqpoint{3.333835in}{2.747307in}}%
\pgfpathlineto{\pgfqpoint{3.333835in}{2.750256in}}%
\pgfpathlineto{\pgfqpoint{3.338376in}{2.750256in}}%
\pgfpathlineto{\pgfqpoint{3.338376in}{2.747307in}}%
\pgfpathmoveto{\pgfqpoint{3.338376in}{2.741408in}}%
\pgfpathlineto{\pgfqpoint{3.338376in}{2.741408in}}%
\pgfpathlineto{\pgfqpoint{3.338376in}{2.744358in}}%
\pgfpathlineto{\pgfqpoint{3.342917in}{2.744358in}}%
\pgfpathlineto{\pgfqpoint{3.342917in}{2.741408in}}%
\pgfpathmoveto{\pgfqpoint{3.338376in}{2.744358in}}%
\pgfpathlineto{\pgfqpoint{3.338376in}{2.744358in}}%
\pgfpathlineto{\pgfqpoint{3.338376in}{2.747307in}}%
\pgfpathlineto{\pgfqpoint{3.342917in}{2.747307in}}%
\pgfpathlineto{\pgfqpoint{3.342917in}{2.744358in}}%
\pgfpathmoveto{\pgfqpoint{3.342917in}{2.741408in}}%
\pgfpathlineto{\pgfqpoint{3.342917in}{2.741408in}}%
\pgfpathlineto{\pgfqpoint{3.342917in}{2.744358in}}%
\pgfpathlineto{\pgfqpoint{3.347458in}{2.744358in}}%
\pgfpathlineto{\pgfqpoint{3.347458in}{2.741408in}}%
\pgfpathmoveto{\pgfqpoint{3.233937in}{2.809239in}}%
\pgfpathlineto{\pgfqpoint{3.233937in}{2.809239in}}%
\pgfpathlineto{\pgfqpoint{3.233937in}{2.812188in}}%
\pgfpathlineto{\pgfqpoint{3.238478in}{2.812188in}}%
\pgfpathlineto{\pgfqpoint{3.238478in}{2.809239in}}%
\pgfpathmoveto{\pgfqpoint{3.252100in}{2.797443in}}%
\pgfpathlineto{\pgfqpoint{3.252100in}{2.797443in}}%
\pgfpathlineto{\pgfqpoint{3.252100in}{2.800392in}}%
\pgfpathlineto{\pgfqpoint{3.256641in}{2.800392in}}%
\pgfpathlineto{\pgfqpoint{3.256641in}{2.797443in}}%
\pgfpathmoveto{\pgfqpoint{3.243019in}{2.803341in}}%
\pgfpathlineto{\pgfqpoint{3.243019in}{2.803341in}}%
\pgfpathlineto{\pgfqpoint{3.243019in}{2.806290in}}%
\pgfpathlineto{\pgfqpoint{3.247559in}{2.806290in}}%
\pgfpathlineto{\pgfqpoint{3.247559in}{2.803341in}}%
\pgfpathmoveto{\pgfqpoint{3.238478in}{2.806290in}}%
\pgfpathlineto{\pgfqpoint{3.238478in}{2.806290in}}%
\pgfpathlineto{\pgfqpoint{3.238478in}{2.809239in}}%
\pgfpathlineto{\pgfqpoint{3.243019in}{2.809239in}}%
\pgfpathlineto{\pgfqpoint{3.243019in}{2.806290in}}%
\pgfpathmoveto{\pgfqpoint{3.238478in}{2.809239in}}%
\pgfpathlineto{\pgfqpoint{3.238478in}{2.809239in}}%
\pgfpathlineto{\pgfqpoint{3.238478in}{2.812188in}}%
\pgfpathlineto{\pgfqpoint{3.243019in}{2.812188in}}%
\pgfpathlineto{\pgfqpoint{3.243019in}{2.809239in}}%
\pgfpathmoveto{\pgfqpoint{3.243019in}{2.806290in}}%
\pgfpathlineto{\pgfqpoint{3.243019in}{2.806290in}}%
\pgfpathlineto{\pgfqpoint{3.243019in}{2.809239in}}%
\pgfpathlineto{\pgfqpoint{3.247559in}{2.809239in}}%
\pgfpathlineto{\pgfqpoint{3.247559in}{2.806290in}}%
\pgfpathmoveto{\pgfqpoint{3.247559in}{2.800392in}}%
\pgfpathlineto{\pgfqpoint{3.247559in}{2.800392in}}%
\pgfpathlineto{\pgfqpoint{3.247559in}{2.803341in}}%
\pgfpathlineto{\pgfqpoint{3.252100in}{2.803341in}}%
\pgfpathlineto{\pgfqpoint{3.252100in}{2.800392in}}%
\pgfpathmoveto{\pgfqpoint{3.247559in}{2.803341in}}%
\pgfpathlineto{\pgfqpoint{3.247559in}{2.803341in}}%
\pgfpathlineto{\pgfqpoint{3.247559in}{2.806290in}}%
\pgfpathlineto{\pgfqpoint{3.252100in}{2.806290in}}%
\pgfpathlineto{\pgfqpoint{3.252100in}{2.803341in}}%
\pgfpathmoveto{\pgfqpoint{3.252100in}{2.800392in}}%
\pgfpathlineto{\pgfqpoint{3.252100in}{2.800392in}}%
\pgfpathlineto{\pgfqpoint{3.252100in}{2.803341in}}%
\pgfpathlineto{\pgfqpoint{3.256641in}{2.803341in}}%
\pgfpathlineto{\pgfqpoint{3.256641in}{2.800392in}}%
\pgfpathmoveto{\pgfqpoint{3.270264in}{2.785646in}}%
\pgfpathlineto{\pgfqpoint{3.270264in}{2.785646in}}%
\pgfpathlineto{\pgfqpoint{3.270264in}{2.788595in}}%
\pgfpathlineto{\pgfqpoint{3.274804in}{2.788595in}}%
\pgfpathlineto{\pgfqpoint{3.274804in}{2.785646in}}%
\pgfpathmoveto{\pgfqpoint{3.288427in}{2.773850in}}%
\pgfpathlineto{\pgfqpoint{3.288427in}{2.773850in}}%
\pgfpathlineto{\pgfqpoint{3.288427in}{2.776799in}}%
\pgfpathlineto{\pgfqpoint{3.292968in}{2.776799in}}%
\pgfpathlineto{\pgfqpoint{3.292968in}{2.773850in}}%
\pgfpathmoveto{\pgfqpoint{3.279345in}{2.779748in}}%
\pgfpathlineto{\pgfqpoint{3.279345in}{2.779748in}}%
\pgfpathlineto{\pgfqpoint{3.279345in}{2.782697in}}%
\pgfpathlineto{\pgfqpoint{3.283886in}{2.782697in}}%
\pgfpathlineto{\pgfqpoint{3.283886in}{2.779748in}}%
\pgfpathmoveto{\pgfqpoint{3.274804in}{2.782697in}}%
\pgfpathlineto{\pgfqpoint{3.274804in}{2.782697in}}%
\pgfpathlineto{\pgfqpoint{3.274804in}{2.785646in}}%
\pgfpathlineto{\pgfqpoint{3.279345in}{2.785646in}}%
\pgfpathlineto{\pgfqpoint{3.279345in}{2.782697in}}%
\pgfpathmoveto{\pgfqpoint{3.274804in}{2.785646in}}%
\pgfpathlineto{\pgfqpoint{3.274804in}{2.785646in}}%
\pgfpathlineto{\pgfqpoint{3.274804in}{2.788595in}}%
\pgfpathlineto{\pgfqpoint{3.279345in}{2.788595in}}%
\pgfpathlineto{\pgfqpoint{3.279345in}{2.785646in}}%
\pgfpathmoveto{\pgfqpoint{3.279345in}{2.782697in}}%
\pgfpathlineto{\pgfqpoint{3.279345in}{2.782697in}}%
\pgfpathlineto{\pgfqpoint{3.279345in}{2.785646in}}%
\pgfpathlineto{\pgfqpoint{3.283886in}{2.785646in}}%
\pgfpathlineto{\pgfqpoint{3.283886in}{2.782697in}}%
\pgfpathmoveto{\pgfqpoint{3.283886in}{2.776799in}}%
\pgfpathlineto{\pgfqpoint{3.283886in}{2.776799in}}%
\pgfpathlineto{\pgfqpoint{3.283886in}{2.779748in}}%
\pgfpathlineto{\pgfqpoint{3.288427in}{2.779748in}}%
\pgfpathlineto{\pgfqpoint{3.288427in}{2.776799in}}%
\pgfpathmoveto{\pgfqpoint{3.283886in}{2.779748in}}%
\pgfpathlineto{\pgfqpoint{3.283886in}{2.779748in}}%
\pgfpathlineto{\pgfqpoint{3.283886in}{2.782697in}}%
\pgfpathlineto{\pgfqpoint{3.288427in}{2.782697in}}%
\pgfpathlineto{\pgfqpoint{3.288427in}{2.779748in}}%
\pgfpathmoveto{\pgfqpoint{3.288427in}{2.776799in}}%
\pgfpathlineto{\pgfqpoint{3.288427in}{2.776799in}}%
\pgfpathlineto{\pgfqpoint{3.288427in}{2.779748in}}%
\pgfpathlineto{\pgfqpoint{3.292968in}{2.779748in}}%
\pgfpathlineto{\pgfqpoint{3.292968in}{2.776799in}}%
\pgfpathmoveto{\pgfqpoint{3.261182in}{2.791545in}}%
\pgfpathlineto{\pgfqpoint{3.261182in}{2.791545in}}%
\pgfpathlineto{\pgfqpoint{3.261182in}{2.794494in}}%
\pgfpathlineto{\pgfqpoint{3.265723in}{2.794494in}}%
\pgfpathlineto{\pgfqpoint{3.265723in}{2.791545in}}%
\pgfpathmoveto{\pgfqpoint{3.256641in}{2.794494in}}%
\pgfpathlineto{\pgfqpoint{3.256641in}{2.794494in}}%
\pgfpathlineto{\pgfqpoint{3.256641in}{2.797443in}}%
\pgfpathlineto{\pgfqpoint{3.261182in}{2.797443in}}%
\pgfpathlineto{\pgfqpoint{3.261182in}{2.794494in}}%
\pgfpathmoveto{\pgfqpoint{3.256641in}{2.797443in}}%
\pgfpathlineto{\pgfqpoint{3.256641in}{2.797443in}}%
\pgfpathlineto{\pgfqpoint{3.256641in}{2.800392in}}%
\pgfpathlineto{\pgfqpoint{3.261182in}{2.800392in}}%
\pgfpathlineto{\pgfqpoint{3.261182in}{2.797443in}}%
\pgfpathmoveto{\pgfqpoint{3.261182in}{2.794494in}}%
\pgfpathlineto{\pgfqpoint{3.261182in}{2.794494in}}%
\pgfpathlineto{\pgfqpoint{3.261182in}{2.797443in}}%
\pgfpathlineto{\pgfqpoint{3.265723in}{2.797443in}}%
\pgfpathlineto{\pgfqpoint{3.265723in}{2.794494in}}%
\pgfpathmoveto{\pgfqpoint{3.265723in}{2.788595in}}%
\pgfpathlineto{\pgfqpoint{3.265723in}{2.788595in}}%
\pgfpathlineto{\pgfqpoint{3.265723in}{2.791545in}}%
\pgfpathlineto{\pgfqpoint{3.270264in}{2.791545in}}%
\pgfpathlineto{\pgfqpoint{3.270264in}{2.788595in}}%
\pgfpathmoveto{\pgfqpoint{3.265723in}{2.791545in}}%
\pgfpathlineto{\pgfqpoint{3.265723in}{2.791545in}}%
\pgfpathlineto{\pgfqpoint{3.265723in}{2.794494in}}%
\pgfpathlineto{\pgfqpoint{3.270264in}{2.794494in}}%
\pgfpathlineto{\pgfqpoint{3.270264in}{2.791545in}}%
\pgfpathmoveto{\pgfqpoint{3.270264in}{2.788595in}}%
\pgfpathlineto{\pgfqpoint{3.270264in}{2.788595in}}%
\pgfpathlineto{\pgfqpoint{3.270264in}{2.791545in}}%
\pgfpathlineto{\pgfqpoint{3.274804in}{2.791545in}}%
\pgfpathlineto{\pgfqpoint{3.274804in}{2.788595in}}%
\pgfpathmoveto{\pgfqpoint{3.224855in}{2.815137in}}%
\pgfpathlineto{\pgfqpoint{3.224855in}{2.815137in}}%
\pgfpathlineto{\pgfqpoint{3.224855in}{2.818087in}}%
\pgfpathlineto{\pgfqpoint{3.229396in}{2.818087in}}%
\pgfpathlineto{\pgfqpoint{3.229396in}{2.815137in}}%
\pgfpathmoveto{\pgfqpoint{3.220315in}{2.818087in}}%
\pgfpathlineto{\pgfqpoint{3.220315in}{2.818087in}}%
\pgfpathlineto{\pgfqpoint{3.220315in}{2.821036in}}%
\pgfpathlineto{\pgfqpoint{3.224855in}{2.821036in}}%
\pgfpathlineto{\pgfqpoint{3.224855in}{2.818087in}}%
\pgfpathmoveto{\pgfqpoint{3.220315in}{2.821036in}}%
\pgfpathlineto{\pgfqpoint{3.220315in}{2.821036in}}%
\pgfpathlineto{\pgfqpoint{3.220315in}{2.823985in}}%
\pgfpathlineto{\pgfqpoint{3.224855in}{2.823985in}}%
\pgfpathlineto{\pgfqpoint{3.224855in}{2.821036in}}%
\pgfpathmoveto{\pgfqpoint{3.224855in}{2.818087in}}%
\pgfpathlineto{\pgfqpoint{3.224855in}{2.818087in}}%
\pgfpathlineto{\pgfqpoint{3.224855in}{2.821036in}}%
\pgfpathlineto{\pgfqpoint{3.229396in}{2.821036in}}%
\pgfpathlineto{\pgfqpoint{3.229396in}{2.818087in}}%
\pgfpathmoveto{\pgfqpoint{3.229396in}{2.812188in}}%
\pgfpathlineto{\pgfqpoint{3.229396in}{2.812188in}}%
\pgfpathlineto{\pgfqpoint{3.229396in}{2.815137in}}%
\pgfpathlineto{\pgfqpoint{3.233937in}{2.815137in}}%
\pgfpathlineto{\pgfqpoint{3.233937in}{2.812188in}}%
\pgfpathmoveto{\pgfqpoint{3.229396in}{2.815137in}}%
\pgfpathlineto{\pgfqpoint{3.229396in}{2.815137in}}%
\pgfpathlineto{\pgfqpoint{3.229396in}{2.818087in}}%
\pgfpathlineto{\pgfqpoint{3.233937in}{2.818087in}}%
\pgfpathlineto{\pgfqpoint{3.233937in}{2.815137in}}%
\pgfpathmoveto{\pgfqpoint{3.233937in}{2.812188in}}%
\pgfpathlineto{\pgfqpoint{3.233937in}{2.812188in}}%
\pgfpathlineto{\pgfqpoint{3.233937in}{2.815137in}}%
\pgfpathlineto{\pgfqpoint{3.238478in}{2.815137in}}%
\pgfpathlineto{\pgfqpoint{3.238478in}{2.812188in}}%
\pgfpathmoveto{\pgfqpoint{3.297509in}{2.767952in}}%
\pgfpathlineto{\pgfqpoint{3.297509in}{2.767952in}}%
\pgfpathlineto{\pgfqpoint{3.297509in}{2.770901in}}%
\pgfpathlineto{\pgfqpoint{3.302049in}{2.770901in}}%
\pgfpathlineto{\pgfqpoint{3.302049in}{2.767952in}}%
\pgfpathmoveto{\pgfqpoint{3.292968in}{2.770901in}}%
\pgfpathlineto{\pgfqpoint{3.292968in}{2.770901in}}%
\pgfpathlineto{\pgfqpoint{3.292968in}{2.773850in}}%
\pgfpathlineto{\pgfqpoint{3.297509in}{2.773850in}}%
\pgfpathlineto{\pgfqpoint{3.297509in}{2.770901in}}%
\pgfpathmoveto{\pgfqpoint{3.292968in}{2.773850in}}%
\pgfpathlineto{\pgfqpoint{3.292968in}{2.773850in}}%
\pgfpathlineto{\pgfqpoint{3.292968in}{2.776799in}}%
\pgfpathlineto{\pgfqpoint{3.297509in}{2.776799in}}%
\pgfpathlineto{\pgfqpoint{3.297509in}{2.773850in}}%
\pgfpathmoveto{\pgfqpoint{3.297509in}{2.770901in}}%
\pgfpathlineto{\pgfqpoint{3.297509in}{2.770901in}}%
\pgfpathlineto{\pgfqpoint{3.297509in}{2.773850in}}%
\pgfpathlineto{\pgfqpoint{3.302049in}{2.773850in}}%
\pgfpathlineto{\pgfqpoint{3.302049in}{2.770901in}}%
\pgfpathmoveto{\pgfqpoint{3.302049in}{2.765003in}}%
\pgfpathlineto{\pgfqpoint{3.302049in}{2.765003in}}%
\pgfpathlineto{\pgfqpoint{3.302049in}{2.767952in}}%
\pgfpathlineto{\pgfqpoint{3.306590in}{2.767952in}}%
\pgfpathlineto{\pgfqpoint{3.306590in}{2.765003in}}%
\pgfpathmoveto{\pgfqpoint{3.302049in}{2.767952in}}%
\pgfpathlineto{\pgfqpoint{3.302049in}{2.767952in}}%
\pgfpathlineto{\pgfqpoint{3.302049in}{2.770901in}}%
\pgfpathlineto{\pgfqpoint{3.306590in}{2.770901in}}%
\pgfpathlineto{\pgfqpoint{3.306590in}{2.767952in}}%
\pgfpathmoveto{\pgfqpoint{3.306590in}{2.765003in}}%
\pgfpathlineto{\pgfqpoint{3.306590in}{2.765003in}}%
\pgfpathlineto{\pgfqpoint{3.306590in}{2.767952in}}%
\pgfpathlineto{\pgfqpoint{3.311131in}{2.767952in}}%
\pgfpathlineto{\pgfqpoint{3.311131in}{2.765003in}}%
\pgfpathmoveto{\pgfqpoint{3.451901in}{2.667676in}}%
\pgfpathlineto{\pgfqpoint{3.451901in}{2.667676in}}%
\pgfpathlineto{\pgfqpoint{3.451901in}{2.670626in}}%
\pgfpathlineto{\pgfqpoint{3.456442in}{2.670626in}}%
\pgfpathlineto{\pgfqpoint{3.456442in}{2.667676in}}%
\pgfpathmoveto{\pgfqpoint{3.470066in}{2.655879in}}%
\pgfpathlineto{\pgfqpoint{3.470066in}{2.655879in}}%
\pgfpathlineto{\pgfqpoint{3.470066in}{2.658829in}}%
\pgfpathlineto{\pgfqpoint{3.474607in}{2.658829in}}%
\pgfpathlineto{\pgfqpoint{3.474607in}{2.655879in}}%
\pgfpathmoveto{\pgfqpoint{3.460983in}{2.661778in}}%
\pgfpathlineto{\pgfqpoint{3.460983in}{2.661778in}}%
\pgfpathlineto{\pgfqpoint{3.460983in}{2.664727in}}%
\pgfpathlineto{\pgfqpoint{3.465524in}{2.664727in}}%
\pgfpathlineto{\pgfqpoint{3.465524in}{2.661778in}}%
\pgfpathmoveto{\pgfqpoint{3.456442in}{2.664727in}}%
\pgfpathlineto{\pgfqpoint{3.456442in}{2.664727in}}%
\pgfpathlineto{\pgfqpoint{3.456442in}{2.667676in}}%
\pgfpathlineto{\pgfqpoint{3.460983in}{2.667676in}}%
\pgfpathlineto{\pgfqpoint{3.460983in}{2.664727in}}%
\pgfpathmoveto{\pgfqpoint{3.456442in}{2.667676in}}%
\pgfpathlineto{\pgfqpoint{3.456442in}{2.667676in}}%
\pgfpathlineto{\pgfqpoint{3.456442in}{2.670626in}}%
\pgfpathlineto{\pgfqpoint{3.460983in}{2.670626in}}%
\pgfpathlineto{\pgfqpoint{3.460983in}{2.667676in}}%
\pgfpathmoveto{\pgfqpoint{3.460983in}{2.664727in}}%
\pgfpathlineto{\pgfqpoint{3.460983in}{2.664727in}}%
\pgfpathlineto{\pgfqpoint{3.460983in}{2.667676in}}%
\pgfpathlineto{\pgfqpoint{3.465524in}{2.667676in}}%
\pgfpathlineto{\pgfqpoint{3.465524in}{2.664727in}}%
\pgfpathmoveto{\pgfqpoint{3.465524in}{2.658829in}}%
\pgfpathlineto{\pgfqpoint{3.465524in}{2.658829in}}%
\pgfpathlineto{\pgfqpoint{3.465524in}{2.661778in}}%
\pgfpathlineto{\pgfqpoint{3.470066in}{2.661778in}}%
\pgfpathlineto{\pgfqpoint{3.470066in}{2.658829in}}%
\pgfpathmoveto{\pgfqpoint{3.465524in}{2.661778in}}%
\pgfpathlineto{\pgfqpoint{3.465524in}{2.661778in}}%
\pgfpathlineto{\pgfqpoint{3.465524in}{2.664727in}}%
\pgfpathlineto{\pgfqpoint{3.470066in}{2.664727in}}%
\pgfpathlineto{\pgfqpoint{3.470066in}{2.661778in}}%
\pgfpathmoveto{\pgfqpoint{3.470066in}{2.658829in}}%
\pgfpathlineto{\pgfqpoint{3.470066in}{2.658829in}}%
\pgfpathlineto{\pgfqpoint{3.470066in}{2.661778in}}%
\pgfpathlineto{\pgfqpoint{3.474607in}{2.661778in}}%
\pgfpathlineto{\pgfqpoint{3.474607in}{2.658829in}}%
\pgfpathmoveto{\pgfqpoint{3.488230in}{2.644083in}}%
\pgfpathlineto{\pgfqpoint{3.488230in}{2.644083in}}%
\pgfpathlineto{\pgfqpoint{3.488230in}{2.647032in}}%
\pgfpathlineto{\pgfqpoint{3.492771in}{2.647032in}}%
\pgfpathlineto{\pgfqpoint{3.492771in}{2.644083in}}%
\pgfpathmoveto{\pgfqpoint{3.506394in}{2.632286in}}%
\pgfpathlineto{\pgfqpoint{3.506394in}{2.632286in}}%
\pgfpathlineto{\pgfqpoint{3.506394in}{2.635235in}}%
\pgfpathlineto{\pgfqpoint{3.510935in}{2.635235in}}%
\pgfpathlineto{\pgfqpoint{3.510935in}{2.632286in}}%
\pgfpathmoveto{\pgfqpoint{3.497312in}{2.638184in}}%
\pgfpathlineto{\pgfqpoint{3.497312in}{2.638184in}}%
\pgfpathlineto{\pgfqpoint{3.497312in}{2.641133in}}%
\pgfpathlineto{\pgfqpoint{3.501853in}{2.641133in}}%
\pgfpathlineto{\pgfqpoint{3.501853in}{2.638184in}}%
\pgfpathmoveto{\pgfqpoint{3.492771in}{2.641133in}}%
\pgfpathlineto{\pgfqpoint{3.492771in}{2.641133in}}%
\pgfpathlineto{\pgfqpoint{3.492771in}{2.644083in}}%
\pgfpathlineto{\pgfqpoint{3.497312in}{2.644083in}}%
\pgfpathlineto{\pgfqpoint{3.497312in}{2.641133in}}%
\pgfpathmoveto{\pgfqpoint{3.492771in}{2.644083in}}%
\pgfpathlineto{\pgfqpoint{3.492771in}{2.644083in}}%
\pgfpathlineto{\pgfqpoint{3.492771in}{2.647032in}}%
\pgfpathlineto{\pgfqpoint{3.497312in}{2.647032in}}%
\pgfpathlineto{\pgfqpoint{3.497312in}{2.644083in}}%
\pgfpathmoveto{\pgfqpoint{3.497312in}{2.641133in}}%
\pgfpathlineto{\pgfqpoint{3.497312in}{2.641133in}}%
\pgfpathlineto{\pgfqpoint{3.497312in}{2.644083in}}%
\pgfpathlineto{\pgfqpoint{3.501853in}{2.644083in}}%
\pgfpathlineto{\pgfqpoint{3.501853in}{2.641133in}}%
\pgfpathmoveto{\pgfqpoint{3.501853in}{2.635235in}}%
\pgfpathlineto{\pgfqpoint{3.501853in}{2.635235in}}%
\pgfpathlineto{\pgfqpoint{3.501853in}{2.638184in}}%
\pgfpathlineto{\pgfqpoint{3.506394in}{2.638184in}}%
\pgfpathlineto{\pgfqpoint{3.506394in}{2.635235in}}%
\pgfpathmoveto{\pgfqpoint{3.501853in}{2.638184in}}%
\pgfpathlineto{\pgfqpoint{3.501853in}{2.638184in}}%
\pgfpathlineto{\pgfqpoint{3.501853in}{2.641133in}}%
\pgfpathlineto{\pgfqpoint{3.506394in}{2.641133in}}%
\pgfpathlineto{\pgfqpoint{3.506394in}{2.638184in}}%
\pgfpathmoveto{\pgfqpoint{3.506394in}{2.635235in}}%
\pgfpathlineto{\pgfqpoint{3.506394in}{2.635235in}}%
\pgfpathlineto{\pgfqpoint{3.506394in}{2.638184in}}%
\pgfpathlineto{\pgfqpoint{3.510935in}{2.638184in}}%
\pgfpathlineto{\pgfqpoint{3.510935in}{2.635235in}}%
\pgfpathmoveto{\pgfqpoint{3.479148in}{2.649981in}}%
\pgfpathlineto{\pgfqpoint{3.479148in}{2.649981in}}%
\pgfpathlineto{\pgfqpoint{3.479148in}{2.652930in}}%
\pgfpathlineto{\pgfqpoint{3.483689in}{2.652930in}}%
\pgfpathlineto{\pgfqpoint{3.483689in}{2.649981in}}%
\pgfpathmoveto{\pgfqpoint{3.474607in}{2.652930in}}%
\pgfpathlineto{\pgfqpoint{3.474607in}{2.652930in}}%
\pgfpathlineto{\pgfqpoint{3.474607in}{2.655879in}}%
\pgfpathlineto{\pgfqpoint{3.479148in}{2.655879in}}%
\pgfpathlineto{\pgfqpoint{3.479148in}{2.652930in}}%
\pgfpathmoveto{\pgfqpoint{3.474607in}{2.655879in}}%
\pgfpathlineto{\pgfqpoint{3.474607in}{2.655879in}}%
\pgfpathlineto{\pgfqpoint{3.474607in}{2.658829in}}%
\pgfpathlineto{\pgfqpoint{3.479148in}{2.658829in}}%
\pgfpathlineto{\pgfqpoint{3.479148in}{2.655879in}}%
\pgfpathmoveto{\pgfqpoint{3.479148in}{2.652930in}}%
\pgfpathlineto{\pgfqpoint{3.479148in}{2.652930in}}%
\pgfpathlineto{\pgfqpoint{3.479148in}{2.655879in}}%
\pgfpathlineto{\pgfqpoint{3.483689in}{2.655879in}}%
\pgfpathlineto{\pgfqpoint{3.483689in}{2.652930in}}%
\pgfpathmoveto{\pgfqpoint{3.483689in}{2.647032in}}%
\pgfpathlineto{\pgfqpoint{3.483689in}{2.647032in}}%
\pgfpathlineto{\pgfqpoint{3.483689in}{2.649981in}}%
\pgfpathlineto{\pgfqpoint{3.488230in}{2.649981in}}%
\pgfpathlineto{\pgfqpoint{3.488230in}{2.647032in}}%
\pgfpathmoveto{\pgfqpoint{3.483689in}{2.649981in}}%
\pgfpathlineto{\pgfqpoint{3.483689in}{2.649981in}}%
\pgfpathlineto{\pgfqpoint{3.483689in}{2.652930in}}%
\pgfpathlineto{\pgfqpoint{3.488230in}{2.652930in}}%
\pgfpathlineto{\pgfqpoint{3.488230in}{2.649981in}}%
\pgfpathmoveto{\pgfqpoint{3.488230in}{2.647032in}}%
\pgfpathlineto{\pgfqpoint{3.488230in}{2.647032in}}%
\pgfpathlineto{\pgfqpoint{3.488230in}{2.649981in}}%
\pgfpathlineto{\pgfqpoint{3.492771in}{2.649981in}}%
\pgfpathlineto{\pgfqpoint{3.492771in}{2.647032in}}%
\pgfpathmoveto{\pgfqpoint{3.379244in}{2.714865in}}%
\pgfpathlineto{\pgfqpoint{3.379244in}{2.714865in}}%
\pgfpathlineto{\pgfqpoint{3.379244in}{2.717814in}}%
\pgfpathlineto{\pgfqpoint{3.383785in}{2.717814in}}%
\pgfpathlineto{\pgfqpoint{3.383785in}{2.714865in}}%
\pgfpathmoveto{\pgfqpoint{3.397408in}{2.703068in}}%
\pgfpathlineto{\pgfqpoint{3.397408in}{2.703068in}}%
\pgfpathlineto{\pgfqpoint{3.397408in}{2.706017in}}%
\pgfpathlineto{\pgfqpoint{3.401949in}{2.706017in}}%
\pgfpathlineto{\pgfqpoint{3.401949in}{2.703068in}}%
\pgfpathmoveto{\pgfqpoint{3.388326in}{2.708966in}}%
\pgfpathlineto{\pgfqpoint{3.388326in}{2.708966in}}%
\pgfpathlineto{\pgfqpoint{3.388326in}{2.711915in}}%
\pgfpathlineto{\pgfqpoint{3.392867in}{2.711915in}}%
\pgfpathlineto{\pgfqpoint{3.392867in}{2.708966in}}%
\pgfpathmoveto{\pgfqpoint{3.383785in}{2.711915in}}%
\pgfpathlineto{\pgfqpoint{3.383785in}{2.711915in}}%
\pgfpathlineto{\pgfqpoint{3.383785in}{2.714865in}}%
\pgfpathlineto{\pgfqpoint{3.388326in}{2.714865in}}%
\pgfpathlineto{\pgfqpoint{3.388326in}{2.711915in}}%
\pgfpathmoveto{\pgfqpoint{3.383785in}{2.714865in}}%
\pgfpathlineto{\pgfqpoint{3.383785in}{2.714865in}}%
\pgfpathlineto{\pgfqpoint{3.383785in}{2.717814in}}%
\pgfpathlineto{\pgfqpoint{3.388326in}{2.717814in}}%
\pgfpathlineto{\pgfqpoint{3.388326in}{2.714865in}}%
\pgfpathmoveto{\pgfqpoint{3.388326in}{2.711915in}}%
\pgfpathlineto{\pgfqpoint{3.388326in}{2.711915in}}%
\pgfpathlineto{\pgfqpoint{3.388326in}{2.714865in}}%
\pgfpathlineto{\pgfqpoint{3.392867in}{2.714865in}}%
\pgfpathlineto{\pgfqpoint{3.392867in}{2.711915in}}%
\pgfpathmoveto{\pgfqpoint{3.392867in}{2.706017in}}%
\pgfpathlineto{\pgfqpoint{3.392867in}{2.706017in}}%
\pgfpathlineto{\pgfqpoint{3.392867in}{2.708966in}}%
\pgfpathlineto{\pgfqpoint{3.397408in}{2.708966in}}%
\pgfpathlineto{\pgfqpoint{3.397408in}{2.706017in}}%
\pgfpathmoveto{\pgfqpoint{3.392867in}{2.708966in}}%
\pgfpathlineto{\pgfqpoint{3.392867in}{2.708966in}}%
\pgfpathlineto{\pgfqpoint{3.392867in}{2.711915in}}%
\pgfpathlineto{\pgfqpoint{3.397408in}{2.711915in}}%
\pgfpathlineto{\pgfqpoint{3.397408in}{2.708966in}}%
\pgfpathmoveto{\pgfqpoint{3.397408in}{2.706017in}}%
\pgfpathlineto{\pgfqpoint{3.397408in}{2.706017in}}%
\pgfpathlineto{\pgfqpoint{3.397408in}{2.708966in}}%
\pgfpathlineto{\pgfqpoint{3.401949in}{2.708966in}}%
\pgfpathlineto{\pgfqpoint{3.401949in}{2.706017in}}%
\pgfpathmoveto{\pgfqpoint{3.415573in}{2.691271in}}%
\pgfpathlineto{\pgfqpoint{3.415573in}{2.691271in}}%
\pgfpathlineto{\pgfqpoint{3.415573in}{2.694220in}}%
\pgfpathlineto{\pgfqpoint{3.420114in}{2.694220in}}%
\pgfpathlineto{\pgfqpoint{3.420114in}{2.691271in}}%
\pgfpathmoveto{\pgfqpoint{3.433737in}{2.679473in}}%
\pgfpathlineto{\pgfqpoint{3.433737in}{2.679473in}}%
\pgfpathlineto{\pgfqpoint{3.433737in}{2.682423in}}%
\pgfpathlineto{\pgfqpoint{3.438278in}{2.682423in}}%
\pgfpathlineto{\pgfqpoint{3.438278in}{2.679473in}}%
\pgfpathmoveto{\pgfqpoint{3.424655in}{2.685372in}}%
\pgfpathlineto{\pgfqpoint{3.424655in}{2.685372in}}%
\pgfpathlineto{\pgfqpoint{3.424655in}{2.688321in}}%
\pgfpathlineto{\pgfqpoint{3.429196in}{2.688321in}}%
\pgfpathlineto{\pgfqpoint{3.429196in}{2.685372in}}%
\pgfpathmoveto{\pgfqpoint{3.420114in}{2.688321in}}%
\pgfpathlineto{\pgfqpoint{3.420114in}{2.688321in}}%
\pgfpathlineto{\pgfqpoint{3.420114in}{2.691271in}}%
\pgfpathlineto{\pgfqpoint{3.424655in}{2.691271in}}%
\pgfpathlineto{\pgfqpoint{3.424655in}{2.688321in}}%
\pgfpathmoveto{\pgfqpoint{3.420114in}{2.691271in}}%
\pgfpathlineto{\pgfqpoint{3.420114in}{2.691271in}}%
\pgfpathlineto{\pgfqpoint{3.420114in}{2.694220in}}%
\pgfpathlineto{\pgfqpoint{3.424655in}{2.694220in}}%
\pgfpathlineto{\pgfqpoint{3.424655in}{2.691271in}}%
\pgfpathmoveto{\pgfqpoint{3.424655in}{2.688321in}}%
\pgfpathlineto{\pgfqpoint{3.424655in}{2.688321in}}%
\pgfpathlineto{\pgfqpoint{3.424655in}{2.691271in}}%
\pgfpathlineto{\pgfqpoint{3.429196in}{2.691271in}}%
\pgfpathlineto{\pgfqpoint{3.429196in}{2.688321in}}%
\pgfpathmoveto{\pgfqpoint{3.429196in}{2.682423in}}%
\pgfpathlineto{\pgfqpoint{3.429196in}{2.682423in}}%
\pgfpathlineto{\pgfqpoint{3.429196in}{2.685372in}}%
\pgfpathlineto{\pgfqpoint{3.433737in}{2.685372in}}%
\pgfpathlineto{\pgfqpoint{3.433737in}{2.682423in}}%
\pgfpathmoveto{\pgfqpoint{3.429196in}{2.685372in}}%
\pgfpathlineto{\pgfqpoint{3.429196in}{2.685372in}}%
\pgfpathlineto{\pgfqpoint{3.429196in}{2.688321in}}%
\pgfpathlineto{\pgfqpoint{3.433737in}{2.688321in}}%
\pgfpathlineto{\pgfqpoint{3.433737in}{2.685372in}}%
\pgfpathmoveto{\pgfqpoint{3.433737in}{2.682423in}}%
\pgfpathlineto{\pgfqpoint{3.433737in}{2.682423in}}%
\pgfpathlineto{\pgfqpoint{3.433737in}{2.685372in}}%
\pgfpathlineto{\pgfqpoint{3.438278in}{2.685372in}}%
\pgfpathlineto{\pgfqpoint{3.438278in}{2.682423in}}%
\pgfpathmoveto{\pgfqpoint{3.406491in}{2.697169in}}%
\pgfpathlineto{\pgfqpoint{3.406491in}{2.697169in}}%
\pgfpathlineto{\pgfqpoint{3.406491in}{2.700118in}}%
\pgfpathlineto{\pgfqpoint{3.411032in}{2.700118in}}%
\pgfpathlineto{\pgfqpoint{3.411032in}{2.697169in}}%
\pgfpathmoveto{\pgfqpoint{3.401949in}{2.700118in}}%
\pgfpathlineto{\pgfqpoint{3.401949in}{2.700118in}}%
\pgfpathlineto{\pgfqpoint{3.401949in}{2.703068in}}%
\pgfpathlineto{\pgfqpoint{3.406491in}{2.703068in}}%
\pgfpathlineto{\pgfqpoint{3.406491in}{2.700118in}}%
\pgfpathmoveto{\pgfqpoint{3.401949in}{2.703068in}}%
\pgfpathlineto{\pgfqpoint{3.401949in}{2.703068in}}%
\pgfpathlineto{\pgfqpoint{3.401949in}{2.706017in}}%
\pgfpathlineto{\pgfqpoint{3.406491in}{2.706017in}}%
\pgfpathlineto{\pgfqpoint{3.406491in}{2.703068in}}%
\pgfpathmoveto{\pgfqpoint{3.406491in}{2.700118in}}%
\pgfpathlineto{\pgfqpoint{3.406491in}{2.700118in}}%
\pgfpathlineto{\pgfqpoint{3.406491in}{2.703068in}}%
\pgfpathlineto{\pgfqpoint{3.411032in}{2.703068in}}%
\pgfpathlineto{\pgfqpoint{3.411032in}{2.700118in}}%
\pgfpathmoveto{\pgfqpoint{3.411032in}{2.694220in}}%
\pgfpathlineto{\pgfqpoint{3.411032in}{2.694220in}}%
\pgfpathlineto{\pgfqpoint{3.411032in}{2.697169in}}%
\pgfpathlineto{\pgfqpoint{3.415573in}{2.697169in}}%
\pgfpathlineto{\pgfqpoint{3.415573in}{2.694220in}}%
\pgfpathmoveto{\pgfqpoint{3.411032in}{2.697169in}}%
\pgfpathlineto{\pgfqpoint{3.411032in}{2.697169in}}%
\pgfpathlineto{\pgfqpoint{3.411032in}{2.700118in}}%
\pgfpathlineto{\pgfqpoint{3.415573in}{2.700118in}}%
\pgfpathlineto{\pgfqpoint{3.415573in}{2.697169in}}%
\pgfpathmoveto{\pgfqpoint{3.415573in}{2.694220in}}%
\pgfpathlineto{\pgfqpoint{3.415573in}{2.694220in}}%
\pgfpathlineto{\pgfqpoint{3.415573in}{2.697169in}}%
\pgfpathlineto{\pgfqpoint{3.420114in}{2.697169in}}%
\pgfpathlineto{\pgfqpoint{3.420114in}{2.694220in}}%
\pgfpathmoveto{\pgfqpoint{3.370162in}{2.720763in}}%
\pgfpathlineto{\pgfqpoint{3.370162in}{2.720763in}}%
\pgfpathlineto{\pgfqpoint{3.370162in}{2.723713in}}%
\pgfpathlineto{\pgfqpoint{3.374703in}{2.723713in}}%
\pgfpathlineto{\pgfqpoint{3.374703in}{2.720763in}}%
\pgfpathmoveto{\pgfqpoint{3.365621in}{2.723713in}}%
\pgfpathlineto{\pgfqpoint{3.365621in}{2.723713in}}%
\pgfpathlineto{\pgfqpoint{3.365621in}{2.726662in}}%
\pgfpathlineto{\pgfqpoint{3.370162in}{2.726662in}}%
\pgfpathlineto{\pgfqpoint{3.370162in}{2.723713in}}%
\pgfpathmoveto{\pgfqpoint{3.365621in}{2.726662in}}%
\pgfpathlineto{\pgfqpoint{3.365621in}{2.726662in}}%
\pgfpathlineto{\pgfqpoint{3.365621in}{2.729611in}}%
\pgfpathlineto{\pgfqpoint{3.370162in}{2.729611in}}%
\pgfpathlineto{\pgfqpoint{3.370162in}{2.726662in}}%
\pgfpathmoveto{\pgfqpoint{3.370162in}{2.723713in}}%
\pgfpathlineto{\pgfqpoint{3.370162in}{2.723713in}}%
\pgfpathlineto{\pgfqpoint{3.370162in}{2.726662in}}%
\pgfpathlineto{\pgfqpoint{3.374703in}{2.726662in}}%
\pgfpathlineto{\pgfqpoint{3.374703in}{2.723713in}}%
\pgfpathmoveto{\pgfqpoint{3.374703in}{2.717814in}}%
\pgfpathlineto{\pgfqpoint{3.374703in}{2.717814in}}%
\pgfpathlineto{\pgfqpoint{3.374703in}{2.720763in}}%
\pgfpathlineto{\pgfqpoint{3.379244in}{2.720763in}}%
\pgfpathlineto{\pgfqpoint{3.379244in}{2.717814in}}%
\pgfpathmoveto{\pgfqpoint{3.374703in}{2.720763in}}%
\pgfpathlineto{\pgfqpoint{3.374703in}{2.720763in}}%
\pgfpathlineto{\pgfqpoint{3.374703in}{2.723713in}}%
\pgfpathlineto{\pgfqpoint{3.379244in}{2.723713in}}%
\pgfpathlineto{\pgfqpoint{3.379244in}{2.720763in}}%
\pgfpathmoveto{\pgfqpoint{3.379244in}{2.717814in}}%
\pgfpathlineto{\pgfqpoint{3.379244in}{2.717814in}}%
\pgfpathlineto{\pgfqpoint{3.379244in}{2.720763in}}%
\pgfpathlineto{\pgfqpoint{3.383785in}{2.720763in}}%
\pgfpathlineto{\pgfqpoint{3.383785in}{2.717814in}}%
\pgfpathmoveto{\pgfqpoint{3.442819in}{2.673575in}}%
\pgfpathlineto{\pgfqpoint{3.442819in}{2.673575in}}%
\pgfpathlineto{\pgfqpoint{3.442819in}{2.676524in}}%
\pgfpathlineto{\pgfqpoint{3.447360in}{2.676524in}}%
\pgfpathlineto{\pgfqpoint{3.447360in}{2.673575in}}%
\pgfpathmoveto{\pgfqpoint{3.438278in}{2.676524in}}%
\pgfpathlineto{\pgfqpoint{3.438278in}{2.676524in}}%
\pgfpathlineto{\pgfqpoint{3.438278in}{2.679473in}}%
\pgfpathlineto{\pgfqpoint{3.442819in}{2.679473in}}%
\pgfpathlineto{\pgfqpoint{3.442819in}{2.676524in}}%
\pgfpathmoveto{\pgfqpoint{3.438278in}{2.679473in}}%
\pgfpathlineto{\pgfqpoint{3.438278in}{2.679473in}}%
\pgfpathlineto{\pgfqpoint{3.438278in}{2.682423in}}%
\pgfpathlineto{\pgfqpoint{3.442819in}{2.682423in}}%
\pgfpathlineto{\pgfqpoint{3.442819in}{2.679473in}}%
\pgfpathmoveto{\pgfqpoint{3.442819in}{2.676524in}}%
\pgfpathlineto{\pgfqpoint{3.442819in}{2.676524in}}%
\pgfpathlineto{\pgfqpoint{3.442819in}{2.679473in}}%
\pgfpathlineto{\pgfqpoint{3.447360in}{2.679473in}}%
\pgfpathlineto{\pgfqpoint{3.447360in}{2.676524in}}%
\pgfpathmoveto{\pgfqpoint{3.447360in}{2.670626in}}%
\pgfpathlineto{\pgfqpoint{3.447360in}{2.670626in}}%
\pgfpathlineto{\pgfqpoint{3.447360in}{2.673575in}}%
\pgfpathlineto{\pgfqpoint{3.451901in}{2.673575in}}%
\pgfpathlineto{\pgfqpoint{3.451901in}{2.670626in}}%
\pgfpathmoveto{\pgfqpoint{3.447360in}{2.673575in}}%
\pgfpathlineto{\pgfqpoint{3.447360in}{2.673575in}}%
\pgfpathlineto{\pgfqpoint{3.447360in}{2.676524in}}%
\pgfpathlineto{\pgfqpoint{3.451901in}{2.676524in}}%
\pgfpathlineto{\pgfqpoint{3.451901in}{2.673575in}}%
\pgfpathmoveto{\pgfqpoint{3.451901in}{2.670626in}}%
\pgfpathlineto{\pgfqpoint{3.451901in}{2.670626in}}%
\pgfpathlineto{\pgfqpoint{3.451901in}{2.673575in}}%
\pgfpathlineto{\pgfqpoint{3.456442in}{2.673575in}}%
\pgfpathlineto{\pgfqpoint{3.456442in}{2.670626in}}%
\pgfpathmoveto{\pgfqpoint{3.597215in}{2.573301in}}%
\pgfpathlineto{\pgfqpoint{3.597215in}{2.573301in}}%
\pgfpathlineto{\pgfqpoint{3.597215in}{2.576251in}}%
\pgfpathlineto{\pgfqpoint{3.601756in}{2.576251in}}%
\pgfpathlineto{\pgfqpoint{3.601756in}{2.573301in}}%
\pgfpathmoveto{\pgfqpoint{3.615379in}{2.561504in}}%
\pgfpathlineto{\pgfqpoint{3.615379in}{2.561504in}}%
\pgfpathlineto{\pgfqpoint{3.615379in}{2.564454in}}%
\pgfpathlineto{\pgfqpoint{3.619920in}{2.564454in}}%
\pgfpathlineto{\pgfqpoint{3.619920in}{2.561504in}}%
\pgfpathmoveto{\pgfqpoint{3.606297in}{2.567403in}}%
\pgfpathlineto{\pgfqpoint{3.606297in}{2.567403in}}%
\pgfpathlineto{\pgfqpoint{3.606297in}{2.570352in}}%
\pgfpathlineto{\pgfqpoint{3.610838in}{2.570352in}}%
\pgfpathlineto{\pgfqpoint{3.610838in}{2.567403in}}%
\pgfpathmoveto{\pgfqpoint{3.601756in}{2.570352in}}%
\pgfpathlineto{\pgfqpoint{3.601756in}{2.570352in}}%
\pgfpathlineto{\pgfqpoint{3.601756in}{2.573301in}}%
\pgfpathlineto{\pgfqpoint{3.606297in}{2.573301in}}%
\pgfpathlineto{\pgfqpoint{3.606297in}{2.570352in}}%
\pgfpathmoveto{\pgfqpoint{3.601756in}{2.573301in}}%
\pgfpathlineto{\pgfqpoint{3.601756in}{2.573301in}}%
\pgfpathlineto{\pgfqpoint{3.601756in}{2.576251in}}%
\pgfpathlineto{\pgfqpoint{3.606297in}{2.576251in}}%
\pgfpathlineto{\pgfqpoint{3.606297in}{2.573301in}}%
\pgfpathmoveto{\pgfqpoint{3.606297in}{2.570352in}}%
\pgfpathlineto{\pgfqpoint{3.606297in}{2.570352in}}%
\pgfpathlineto{\pgfqpoint{3.606297in}{2.573301in}}%
\pgfpathlineto{\pgfqpoint{3.610838in}{2.573301in}}%
\pgfpathlineto{\pgfqpoint{3.610838in}{2.570352in}}%
\pgfpathmoveto{\pgfqpoint{3.610838in}{2.564454in}}%
\pgfpathlineto{\pgfqpoint{3.610838in}{2.564454in}}%
\pgfpathlineto{\pgfqpoint{3.610838in}{2.567403in}}%
\pgfpathlineto{\pgfqpoint{3.615379in}{2.567403in}}%
\pgfpathlineto{\pgfqpoint{3.615379in}{2.564454in}}%
\pgfpathmoveto{\pgfqpoint{3.610838in}{2.567403in}}%
\pgfpathlineto{\pgfqpoint{3.610838in}{2.567403in}}%
\pgfpathlineto{\pgfqpoint{3.610838in}{2.570352in}}%
\pgfpathlineto{\pgfqpoint{3.615379in}{2.570352in}}%
\pgfpathlineto{\pgfqpoint{3.615379in}{2.567403in}}%
\pgfpathmoveto{\pgfqpoint{3.615379in}{2.564454in}}%
\pgfpathlineto{\pgfqpoint{3.615379in}{2.564454in}}%
\pgfpathlineto{\pgfqpoint{3.615379in}{2.567403in}}%
\pgfpathlineto{\pgfqpoint{3.619920in}{2.567403in}}%
\pgfpathlineto{\pgfqpoint{3.619920in}{2.564454in}}%
\pgfpathmoveto{\pgfqpoint{3.633543in}{2.549707in}}%
\pgfpathlineto{\pgfqpoint{3.633543in}{2.549707in}}%
\pgfpathlineto{\pgfqpoint{3.633543in}{2.552657in}}%
\pgfpathlineto{\pgfqpoint{3.638084in}{2.552657in}}%
\pgfpathlineto{\pgfqpoint{3.638084in}{2.549707in}}%
\pgfpathmoveto{\pgfqpoint{3.651707in}{2.537910in}}%
\pgfpathlineto{\pgfqpoint{3.651707in}{2.537910in}}%
\pgfpathlineto{\pgfqpoint{3.651707in}{2.540860in}}%
\pgfpathlineto{\pgfqpoint{3.656248in}{2.540860in}}%
\pgfpathlineto{\pgfqpoint{3.656248in}{2.537910in}}%
\pgfpathmoveto{\pgfqpoint{3.642625in}{2.543809in}}%
\pgfpathlineto{\pgfqpoint{3.642625in}{2.543809in}}%
\pgfpathlineto{\pgfqpoint{3.642625in}{2.546758in}}%
\pgfpathlineto{\pgfqpoint{3.647166in}{2.546758in}}%
\pgfpathlineto{\pgfqpoint{3.647166in}{2.543809in}}%
\pgfpathmoveto{\pgfqpoint{3.638084in}{2.546758in}}%
\pgfpathlineto{\pgfqpoint{3.638084in}{2.546758in}}%
\pgfpathlineto{\pgfqpoint{3.638084in}{2.549707in}}%
\pgfpathlineto{\pgfqpoint{3.642625in}{2.549707in}}%
\pgfpathlineto{\pgfqpoint{3.642625in}{2.546758in}}%
\pgfpathmoveto{\pgfqpoint{3.638084in}{2.549707in}}%
\pgfpathlineto{\pgfqpoint{3.638084in}{2.549707in}}%
\pgfpathlineto{\pgfqpoint{3.638084in}{2.552657in}}%
\pgfpathlineto{\pgfqpoint{3.642625in}{2.552657in}}%
\pgfpathlineto{\pgfqpoint{3.642625in}{2.549707in}}%
\pgfpathmoveto{\pgfqpoint{3.642625in}{2.546758in}}%
\pgfpathlineto{\pgfqpoint{3.642625in}{2.546758in}}%
\pgfpathlineto{\pgfqpoint{3.642625in}{2.549707in}}%
\pgfpathlineto{\pgfqpoint{3.647166in}{2.549707in}}%
\pgfpathlineto{\pgfqpoint{3.647166in}{2.546758in}}%
\pgfpathmoveto{\pgfqpoint{3.647166in}{2.540860in}}%
\pgfpathlineto{\pgfqpoint{3.647166in}{2.540860in}}%
\pgfpathlineto{\pgfqpoint{3.647166in}{2.543809in}}%
\pgfpathlineto{\pgfqpoint{3.651707in}{2.543809in}}%
\pgfpathlineto{\pgfqpoint{3.651707in}{2.540860in}}%
\pgfpathmoveto{\pgfqpoint{3.647166in}{2.543809in}}%
\pgfpathlineto{\pgfqpoint{3.647166in}{2.543809in}}%
\pgfpathlineto{\pgfqpoint{3.647166in}{2.546758in}}%
\pgfpathlineto{\pgfqpoint{3.651707in}{2.546758in}}%
\pgfpathlineto{\pgfqpoint{3.651707in}{2.543809in}}%
\pgfpathmoveto{\pgfqpoint{3.651707in}{2.540860in}}%
\pgfpathlineto{\pgfqpoint{3.651707in}{2.540860in}}%
\pgfpathlineto{\pgfqpoint{3.651707in}{2.543809in}}%
\pgfpathlineto{\pgfqpoint{3.656248in}{2.543809in}}%
\pgfpathlineto{\pgfqpoint{3.656248in}{2.540860in}}%
\pgfpathmoveto{\pgfqpoint{3.624461in}{2.555606in}}%
\pgfpathlineto{\pgfqpoint{3.624461in}{2.555606in}}%
\pgfpathlineto{\pgfqpoint{3.624461in}{2.558555in}}%
\pgfpathlineto{\pgfqpoint{3.629002in}{2.558555in}}%
\pgfpathlineto{\pgfqpoint{3.629002in}{2.555606in}}%
\pgfpathmoveto{\pgfqpoint{3.619920in}{2.558555in}}%
\pgfpathlineto{\pgfqpoint{3.619920in}{2.558555in}}%
\pgfpathlineto{\pgfqpoint{3.619920in}{2.561504in}}%
\pgfpathlineto{\pgfqpoint{3.624461in}{2.561504in}}%
\pgfpathlineto{\pgfqpoint{3.624461in}{2.558555in}}%
\pgfpathmoveto{\pgfqpoint{3.619920in}{2.561504in}}%
\pgfpathlineto{\pgfqpoint{3.619920in}{2.561504in}}%
\pgfpathlineto{\pgfqpoint{3.619920in}{2.564454in}}%
\pgfpathlineto{\pgfqpoint{3.624461in}{2.564454in}}%
\pgfpathlineto{\pgfqpoint{3.624461in}{2.561504in}}%
\pgfpathmoveto{\pgfqpoint{3.624461in}{2.558555in}}%
\pgfpathlineto{\pgfqpoint{3.624461in}{2.558555in}}%
\pgfpathlineto{\pgfqpoint{3.624461in}{2.561504in}}%
\pgfpathlineto{\pgfqpoint{3.629002in}{2.561504in}}%
\pgfpathlineto{\pgfqpoint{3.629002in}{2.558555in}}%
\pgfpathmoveto{\pgfqpoint{3.629002in}{2.552657in}}%
\pgfpathlineto{\pgfqpoint{3.629002in}{2.552657in}}%
\pgfpathlineto{\pgfqpoint{3.629002in}{2.555606in}}%
\pgfpathlineto{\pgfqpoint{3.633543in}{2.555606in}}%
\pgfpathlineto{\pgfqpoint{3.633543in}{2.552657in}}%
\pgfpathmoveto{\pgfqpoint{3.629002in}{2.555606in}}%
\pgfpathlineto{\pgfqpoint{3.629002in}{2.555606in}}%
\pgfpathlineto{\pgfqpoint{3.629002in}{2.558555in}}%
\pgfpathlineto{\pgfqpoint{3.633543in}{2.558555in}}%
\pgfpathlineto{\pgfqpoint{3.633543in}{2.555606in}}%
\pgfpathmoveto{\pgfqpoint{3.633543in}{2.552657in}}%
\pgfpathlineto{\pgfqpoint{3.633543in}{2.552657in}}%
\pgfpathlineto{\pgfqpoint{3.633543in}{2.555606in}}%
\pgfpathlineto{\pgfqpoint{3.638084in}{2.555606in}}%
\pgfpathlineto{\pgfqpoint{3.638084in}{2.552657in}}%
\pgfpathmoveto{\pgfqpoint{3.524558in}{2.620489in}}%
\pgfpathlineto{\pgfqpoint{3.524558in}{2.620489in}}%
\pgfpathlineto{\pgfqpoint{3.524558in}{2.623438in}}%
\pgfpathlineto{\pgfqpoint{3.529099in}{2.623438in}}%
\pgfpathlineto{\pgfqpoint{3.529099in}{2.620489in}}%
\pgfpathmoveto{\pgfqpoint{3.542722in}{2.608692in}}%
\pgfpathlineto{\pgfqpoint{3.542722in}{2.608692in}}%
\pgfpathlineto{\pgfqpoint{3.542722in}{2.611641in}}%
\pgfpathlineto{\pgfqpoint{3.547263in}{2.611641in}}%
\pgfpathlineto{\pgfqpoint{3.547263in}{2.608692in}}%
\pgfpathmoveto{\pgfqpoint{3.533640in}{2.614590in}}%
\pgfpathlineto{\pgfqpoint{3.533640in}{2.614590in}}%
\pgfpathlineto{\pgfqpoint{3.533640in}{2.617540in}}%
\pgfpathlineto{\pgfqpoint{3.538181in}{2.617540in}}%
\pgfpathlineto{\pgfqpoint{3.538181in}{2.614590in}}%
\pgfpathmoveto{\pgfqpoint{3.529099in}{2.617540in}}%
\pgfpathlineto{\pgfqpoint{3.529099in}{2.617540in}}%
\pgfpathlineto{\pgfqpoint{3.529099in}{2.620489in}}%
\pgfpathlineto{\pgfqpoint{3.533640in}{2.620489in}}%
\pgfpathlineto{\pgfqpoint{3.533640in}{2.617540in}}%
\pgfpathmoveto{\pgfqpoint{3.529099in}{2.620489in}}%
\pgfpathlineto{\pgfqpoint{3.529099in}{2.620489in}}%
\pgfpathlineto{\pgfqpoint{3.529099in}{2.623438in}}%
\pgfpathlineto{\pgfqpoint{3.533640in}{2.623438in}}%
\pgfpathlineto{\pgfqpoint{3.533640in}{2.620489in}}%
\pgfpathmoveto{\pgfqpoint{3.533640in}{2.617540in}}%
\pgfpathlineto{\pgfqpoint{3.533640in}{2.617540in}}%
\pgfpathlineto{\pgfqpoint{3.533640in}{2.620489in}}%
\pgfpathlineto{\pgfqpoint{3.538181in}{2.620489in}}%
\pgfpathlineto{\pgfqpoint{3.538181in}{2.617540in}}%
\pgfpathmoveto{\pgfqpoint{3.538181in}{2.611641in}}%
\pgfpathlineto{\pgfqpoint{3.538181in}{2.611641in}}%
\pgfpathlineto{\pgfqpoint{3.538181in}{2.614590in}}%
\pgfpathlineto{\pgfqpoint{3.542722in}{2.614590in}}%
\pgfpathlineto{\pgfqpoint{3.542722in}{2.611641in}}%
\pgfpathmoveto{\pgfqpoint{3.538181in}{2.614590in}}%
\pgfpathlineto{\pgfqpoint{3.538181in}{2.614590in}}%
\pgfpathlineto{\pgfqpoint{3.538181in}{2.617540in}}%
\pgfpathlineto{\pgfqpoint{3.542722in}{2.617540in}}%
\pgfpathlineto{\pgfqpoint{3.542722in}{2.614590in}}%
\pgfpathmoveto{\pgfqpoint{3.542722in}{2.611641in}}%
\pgfpathlineto{\pgfqpoint{3.542722in}{2.611641in}}%
\pgfpathlineto{\pgfqpoint{3.542722in}{2.614590in}}%
\pgfpathlineto{\pgfqpoint{3.547263in}{2.614590in}}%
\pgfpathlineto{\pgfqpoint{3.547263in}{2.611641in}}%
\pgfpathmoveto{\pgfqpoint{3.560887in}{2.596895in}}%
\pgfpathlineto{\pgfqpoint{3.560887in}{2.596895in}}%
\pgfpathlineto{\pgfqpoint{3.560887in}{2.599844in}}%
\pgfpathlineto{\pgfqpoint{3.565428in}{2.599844in}}%
\pgfpathlineto{\pgfqpoint{3.565428in}{2.596895in}}%
\pgfpathmoveto{\pgfqpoint{3.579051in}{2.585098in}}%
\pgfpathlineto{\pgfqpoint{3.579051in}{2.585098in}}%
\pgfpathlineto{\pgfqpoint{3.579051in}{2.588048in}}%
\pgfpathlineto{\pgfqpoint{3.583592in}{2.588048in}}%
\pgfpathlineto{\pgfqpoint{3.583592in}{2.585098in}}%
\pgfpathmoveto{\pgfqpoint{3.569969in}{2.590997in}}%
\pgfpathlineto{\pgfqpoint{3.569969in}{2.590997in}}%
\pgfpathlineto{\pgfqpoint{3.569969in}{2.593946in}}%
\pgfpathlineto{\pgfqpoint{3.574510in}{2.593946in}}%
\pgfpathlineto{\pgfqpoint{3.574510in}{2.590997in}}%
\pgfpathmoveto{\pgfqpoint{3.565428in}{2.593946in}}%
\pgfpathlineto{\pgfqpoint{3.565428in}{2.593946in}}%
\pgfpathlineto{\pgfqpoint{3.565428in}{2.596895in}}%
\pgfpathlineto{\pgfqpoint{3.569969in}{2.596895in}}%
\pgfpathlineto{\pgfqpoint{3.569969in}{2.593946in}}%
\pgfpathmoveto{\pgfqpoint{3.565428in}{2.596895in}}%
\pgfpathlineto{\pgfqpoint{3.565428in}{2.596895in}}%
\pgfpathlineto{\pgfqpoint{3.565428in}{2.599844in}}%
\pgfpathlineto{\pgfqpoint{3.569969in}{2.599844in}}%
\pgfpathlineto{\pgfqpoint{3.569969in}{2.596895in}}%
\pgfpathmoveto{\pgfqpoint{3.569969in}{2.593946in}}%
\pgfpathlineto{\pgfqpoint{3.569969in}{2.593946in}}%
\pgfpathlineto{\pgfqpoint{3.569969in}{2.596895in}}%
\pgfpathlineto{\pgfqpoint{3.574510in}{2.596895in}}%
\pgfpathlineto{\pgfqpoint{3.574510in}{2.593946in}}%
\pgfpathmoveto{\pgfqpoint{3.574510in}{2.588048in}}%
\pgfpathlineto{\pgfqpoint{3.574510in}{2.588048in}}%
\pgfpathlineto{\pgfqpoint{3.574510in}{2.590997in}}%
\pgfpathlineto{\pgfqpoint{3.579051in}{2.590997in}}%
\pgfpathlineto{\pgfqpoint{3.579051in}{2.588048in}}%
\pgfpathmoveto{\pgfqpoint{3.574510in}{2.590997in}}%
\pgfpathlineto{\pgfqpoint{3.574510in}{2.590997in}}%
\pgfpathlineto{\pgfqpoint{3.574510in}{2.593946in}}%
\pgfpathlineto{\pgfqpoint{3.579051in}{2.593946in}}%
\pgfpathlineto{\pgfqpoint{3.579051in}{2.590997in}}%
\pgfpathmoveto{\pgfqpoint{3.579051in}{2.588048in}}%
\pgfpathlineto{\pgfqpoint{3.579051in}{2.588048in}}%
\pgfpathlineto{\pgfqpoint{3.579051in}{2.590997in}}%
\pgfpathlineto{\pgfqpoint{3.583592in}{2.590997in}}%
\pgfpathlineto{\pgfqpoint{3.583592in}{2.588048in}}%
\pgfpathmoveto{\pgfqpoint{3.551804in}{2.602794in}}%
\pgfpathlineto{\pgfqpoint{3.551804in}{2.602794in}}%
\pgfpathlineto{\pgfqpoint{3.551804in}{2.605743in}}%
\pgfpathlineto{\pgfqpoint{3.556345in}{2.605743in}}%
\pgfpathlineto{\pgfqpoint{3.556345in}{2.602794in}}%
\pgfpathmoveto{\pgfqpoint{3.547263in}{2.605743in}}%
\pgfpathlineto{\pgfqpoint{3.547263in}{2.605743in}}%
\pgfpathlineto{\pgfqpoint{3.547263in}{2.608692in}}%
\pgfpathlineto{\pgfqpoint{3.551804in}{2.608692in}}%
\pgfpathlineto{\pgfqpoint{3.551804in}{2.605743in}}%
\pgfpathmoveto{\pgfqpoint{3.547263in}{2.608692in}}%
\pgfpathlineto{\pgfqpoint{3.547263in}{2.608692in}}%
\pgfpathlineto{\pgfqpoint{3.547263in}{2.611641in}}%
\pgfpathlineto{\pgfqpoint{3.551804in}{2.611641in}}%
\pgfpathlineto{\pgfqpoint{3.551804in}{2.608692in}}%
\pgfpathmoveto{\pgfqpoint{3.551804in}{2.605743in}}%
\pgfpathlineto{\pgfqpoint{3.551804in}{2.605743in}}%
\pgfpathlineto{\pgfqpoint{3.551804in}{2.608692in}}%
\pgfpathlineto{\pgfqpoint{3.556345in}{2.608692in}}%
\pgfpathlineto{\pgfqpoint{3.556345in}{2.605743in}}%
\pgfpathmoveto{\pgfqpoint{3.556345in}{2.599844in}}%
\pgfpathlineto{\pgfqpoint{3.556345in}{2.599844in}}%
\pgfpathlineto{\pgfqpoint{3.556345in}{2.602794in}}%
\pgfpathlineto{\pgfqpoint{3.560887in}{2.602794in}}%
\pgfpathlineto{\pgfqpoint{3.560887in}{2.599844in}}%
\pgfpathmoveto{\pgfqpoint{3.556345in}{2.602794in}}%
\pgfpathlineto{\pgfqpoint{3.556345in}{2.602794in}}%
\pgfpathlineto{\pgfqpoint{3.556345in}{2.605743in}}%
\pgfpathlineto{\pgfqpoint{3.560887in}{2.605743in}}%
\pgfpathlineto{\pgfqpoint{3.560887in}{2.602794in}}%
\pgfpathmoveto{\pgfqpoint{3.560887in}{2.599844in}}%
\pgfpathlineto{\pgfqpoint{3.560887in}{2.599844in}}%
\pgfpathlineto{\pgfqpoint{3.560887in}{2.602794in}}%
\pgfpathlineto{\pgfqpoint{3.565428in}{2.602794in}}%
\pgfpathlineto{\pgfqpoint{3.565428in}{2.599844in}}%
\pgfpathmoveto{\pgfqpoint{3.515476in}{2.626387in}}%
\pgfpathlineto{\pgfqpoint{3.515476in}{2.626387in}}%
\pgfpathlineto{\pgfqpoint{3.515476in}{2.629337in}}%
\pgfpathlineto{\pgfqpoint{3.520017in}{2.629337in}}%
\pgfpathlineto{\pgfqpoint{3.520017in}{2.626387in}}%
\pgfpathmoveto{\pgfqpoint{3.510935in}{2.629337in}}%
\pgfpathlineto{\pgfqpoint{3.510935in}{2.629337in}}%
\pgfpathlineto{\pgfqpoint{3.510935in}{2.632286in}}%
\pgfpathlineto{\pgfqpoint{3.515476in}{2.632286in}}%
\pgfpathlineto{\pgfqpoint{3.515476in}{2.629337in}}%
\pgfpathmoveto{\pgfqpoint{3.510935in}{2.632286in}}%
\pgfpathlineto{\pgfqpoint{3.510935in}{2.632286in}}%
\pgfpathlineto{\pgfqpoint{3.510935in}{2.635235in}}%
\pgfpathlineto{\pgfqpoint{3.515476in}{2.635235in}}%
\pgfpathlineto{\pgfqpoint{3.515476in}{2.632286in}}%
\pgfpathmoveto{\pgfqpoint{3.515476in}{2.629337in}}%
\pgfpathlineto{\pgfqpoint{3.515476in}{2.629337in}}%
\pgfpathlineto{\pgfqpoint{3.515476in}{2.632286in}}%
\pgfpathlineto{\pgfqpoint{3.520017in}{2.632286in}}%
\pgfpathlineto{\pgfqpoint{3.520017in}{2.629337in}}%
\pgfpathmoveto{\pgfqpoint{3.520017in}{2.623438in}}%
\pgfpathlineto{\pgfqpoint{3.520017in}{2.623438in}}%
\pgfpathlineto{\pgfqpoint{3.520017in}{2.626387in}}%
\pgfpathlineto{\pgfqpoint{3.524558in}{2.626387in}}%
\pgfpathlineto{\pgfqpoint{3.524558in}{2.623438in}}%
\pgfpathmoveto{\pgfqpoint{3.520017in}{2.626387in}}%
\pgfpathlineto{\pgfqpoint{3.520017in}{2.626387in}}%
\pgfpathlineto{\pgfqpoint{3.520017in}{2.629337in}}%
\pgfpathlineto{\pgfqpoint{3.524558in}{2.629337in}}%
\pgfpathlineto{\pgfqpoint{3.524558in}{2.626387in}}%
\pgfpathmoveto{\pgfqpoint{3.524558in}{2.623438in}}%
\pgfpathlineto{\pgfqpoint{3.524558in}{2.623438in}}%
\pgfpathlineto{\pgfqpoint{3.524558in}{2.626387in}}%
\pgfpathlineto{\pgfqpoint{3.529099in}{2.626387in}}%
\pgfpathlineto{\pgfqpoint{3.529099in}{2.623438in}}%
\pgfpathmoveto{\pgfqpoint{3.588133in}{2.579200in}}%
\pgfpathlineto{\pgfqpoint{3.588133in}{2.579200in}}%
\pgfpathlineto{\pgfqpoint{3.588133in}{2.582149in}}%
\pgfpathlineto{\pgfqpoint{3.592674in}{2.582149in}}%
\pgfpathlineto{\pgfqpoint{3.592674in}{2.579200in}}%
\pgfpathmoveto{\pgfqpoint{3.583592in}{2.582149in}}%
\pgfpathlineto{\pgfqpoint{3.583592in}{2.582149in}}%
\pgfpathlineto{\pgfqpoint{3.583592in}{2.585098in}}%
\pgfpathlineto{\pgfqpoint{3.588133in}{2.585098in}}%
\pgfpathlineto{\pgfqpoint{3.588133in}{2.582149in}}%
\pgfpathmoveto{\pgfqpoint{3.583592in}{2.585098in}}%
\pgfpathlineto{\pgfqpoint{3.583592in}{2.585098in}}%
\pgfpathlineto{\pgfqpoint{3.583592in}{2.588048in}}%
\pgfpathlineto{\pgfqpoint{3.588133in}{2.588048in}}%
\pgfpathlineto{\pgfqpoint{3.588133in}{2.585098in}}%
\pgfpathmoveto{\pgfqpoint{3.588133in}{2.582149in}}%
\pgfpathlineto{\pgfqpoint{3.588133in}{2.582149in}}%
\pgfpathlineto{\pgfqpoint{3.588133in}{2.585098in}}%
\pgfpathlineto{\pgfqpoint{3.592674in}{2.585098in}}%
\pgfpathlineto{\pgfqpoint{3.592674in}{2.582149in}}%
\pgfpathmoveto{\pgfqpoint{3.592674in}{2.576251in}}%
\pgfpathlineto{\pgfqpoint{3.592674in}{2.576251in}}%
\pgfpathlineto{\pgfqpoint{3.592674in}{2.579200in}}%
\pgfpathlineto{\pgfqpoint{3.597215in}{2.579200in}}%
\pgfpathlineto{\pgfqpoint{3.597215in}{2.576251in}}%
\pgfpathmoveto{\pgfqpoint{3.592674in}{2.579200in}}%
\pgfpathlineto{\pgfqpoint{3.592674in}{2.579200in}}%
\pgfpathlineto{\pgfqpoint{3.592674in}{2.582149in}}%
\pgfpathlineto{\pgfqpoint{3.597215in}{2.582149in}}%
\pgfpathlineto{\pgfqpoint{3.597215in}{2.579200in}}%
\pgfpathmoveto{\pgfqpoint{3.597215in}{2.576251in}}%
\pgfpathlineto{\pgfqpoint{3.597215in}{2.576251in}}%
\pgfpathlineto{\pgfqpoint{3.597215in}{2.579200in}}%
\pgfpathlineto{\pgfqpoint{3.601756in}{2.579200in}}%
\pgfpathlineto{\pgfqpoint{3.601756in}{2.576251in}}%
\pgfpathmoveto{\pgfqpoint{3.742530in}{2.478925in}}%
\pgfpathlineto{\pgfqpoint{3.742530in}{2.478925in}}%
\pgfpathlineto{\pgfqpoint{3.742530in}{2.481874in}}%
\pgfpathlineto{\pgfqpoint{3.747071in}{2.481874in}}%
\pgfpathlineto{\pgfqpoint{3.747071in}{2.478925in}}%
\pgfpathmoveto{\pgfqpoint{3.760694in}{2.467128in}}%
\pgfpathlineto{\pgfqpoint{3.760694in}{2.467128in}}%
\pgfpathlineto{\pgfqpoint{3.760694in}{2.470078in}}%
\pgfpathlineto{\pgfqpoint{3.765236in}{2.470078in}}%
\pgfpathlineto{\pgfqpoint{3.765236in}{2.467128in}}%
\pgfpathmoveto{\pgfqpoint{3.751612in}{2.473027in}}%
\pgfpathlineto{\pgfqpoint{3.751612in}{2.473027in}}%
\pgfpathlineto{\pgfqpoint{3.751612in}{2.475976in}}%
\pgfpathlineto{\pgfqpoint{3.756153in}{2.475976in}}%
\pgfpathlineto{\pgfqpoint{3.756153in}{2.473027in}}%
\pgfpathmoveto{\pgfqpoint{3.747071in}{2.475976in}}%
\pgfpathlineto{\pgfqpoint{3.747071in}{2.475976in}}%
\pgfpathlineto{\pgfqpoint{3.747071in}{2.478925in}}%
\pgfpathlineto{\pgfqpoint{3.751612in}{2.478925in}}%
\pgfpathlineto{\pgfqpoint{3.751612in}{2.475976in}}%
\pgfpathmoveto{\pgfqpoint{3.747071in}{2.478925in}}%
\pgfpathlineto{\pgfqpoint{3.747071in}{2.478925in}}%
\pgfpathlineto{\pgfqpoint{3.747071in}{2.481874in}}%
\pgfpathlineto{\pgfqpoint{3.751612in}{2.481874in}}%
\pgfpathlineto{\pgfqpoint{3.751612in}{2.478925in}}%
\pgfpathmoveto{\pgfqpoint{3.751612in}{2.475976in}}%
\pgfpathlineto{\pgfqpoint{3.751612in}{2.475976in}}%
\pgfpathlineto{\pgfqpoint{3.751612in}{2.478925in}}%
\pgfpathlineto{\pgfqpoint{3.756153in}{2.478925in}}%
\pgfpathlineto{\pgfqpoint{3.756153in}{2.475976in}}%
\pgfpathmoveto{\pgfqpoint{3.756153in}{2.470078in}}%
\pgfpathlineto{\pgfqpoint{3.756153in}{2.470078in}}%
\pgfpathlineto{\pgfqpoint{3.756153in}{2.473027in}}%
\pgfpathlineto{\pgfqpoint{3.760694in}{2.473027in}}%
\pgfpathlineto{\pgfqpoint{3.760694in}{2.470078in}}%
\pgfpathmoveto{\pgfqpoint{3.756153in}{2.473027in}}%
\pgfpathlineto{\pgfqpoint{3.756153in}{2.473027in}}%
\pgfpathlineto{\pgfqpoint{3.756153in}{2.475976in}}%
\pgfpathlineto{\pgfqpoint{3.760694in}{2.475976in}}%
\pgfpathlineto{\pgfqpoint{3.760694in}{2.473027in}}%
\pgfpathmoveto{\pgfqpoint{3.760694in}{2.470078in}}%
\pgfpathlineto{\pgfqpoint{3.760694in}{2.470078in}}%
\pgfpathlineto{\pgfqpoint{3.760694in}{2.473027in}}%
\pgfpathlineto{\pgfqpoint{3.765236in}{2.473027in}}%
\pgfpathlineto{\pgfqpoint{3.765236in}{2.470078in}}%
\pgfpathmoveto{\pgfqpoint{3.778859in}{2.455332in}}%
\pgfpathlineto{\pgfqpoint{3.778859in}{2.455332in}}%
\pgfpathlineto{\pgfqpoint{3.778859in}{2.458281in}}%
\pgfpathlineto{\pgfqpoint{3.783400in}{2.458281in}}%
\pgfpathlineto{\pgfqpoint{3.783400in}{2.455332in}}%
\pgfpathmoveto{\pgfqpoint{3.797024in}{2.443535in}}%
\pgfpathlineto{\pgfqpoint{3.797024in}{2.443535in}}%
\pgfpathlineto{\pgfqpoint{3.797024in}{2.446484in}}%
\pgfpathlineto{\pgfqpoint{3.801565in}{2.446484in}}%
\pgfpathlineto{\pgfqpoint{3.801565in}{2.443535in}}%
\pgfpathmoveto{\pgfqpoint{3.787941in}{2.449433in}}%
\pgfpathlineto{\pgfqpoint{3.787941in}{2.449433in}}%
\pgfpathlineto{\pgfqpoint{3.787941in}{2.452382in}}%
\pgfpathlineto{\pgfqpoint{3.792482in}{2.452382in}}%
\pgfpathlineto{\pgfqpoint{3.792482in}{2.449433in}}%
\pgfpathmoveto{\pgfqpoint{3.783400in}{2.452382in}}%
\pgfpathlineto{\pgfqpoint{3.783400in}{2.452382in}}%
\pgfpathlineto{\pgfqpoint{3.783400in}{2.455332in}}%
\pgfpathlineto{\pgfqpoint{3.787941in}{2.455332in}}%
\pgfpathlineto{\pgfqpoint{3.787941in}{2.452382in}}%
\pgfpathmoveto{\pgfqpoint{3.783400in}{2.455332in}}%
\pgfpathlineto{\pgfqpoint{3.783400in}{2.455332in}}%
\pgfpathlineto{\pgfqpoint{3.783400in}{2.458281in}}%
\pgfpathlineto{\pgfqpoint{3.787941in}{2.458281in}}%
\pgfpathlineto{\pgfqpoint{3.787941in}{2.455332in}}%
\pgfpathmoveto{\pgfqpoint{3.787941in}{2.452382in}}%
\pgfpathlineto{\pgfqpoint{3.787941in}{2.452382in}}%
\pgfpathlineto{\pgfqpoint{3.787941in}{2.455332in}}%
\pgfpathlineto{\pgfqpoint{3.792482in}{2.455332in}}%
\pgfpathlineto{\pgfqpoint{3.792482in}{2.452382in}}%
\pgfpathmoveto{\pgfqpoint{3.792482in}{2.446484in}}%
\pgfpathlineto{\pgfqpoint{3.792482in}{2.446484in}}%
\pgfpathlineto{\pgfqpoint{3.792482in}{2.449433in}}%
\pgfpathlineto{\pgfqpoint{3.797024in}{2.449433in}}%
\pgfpathlineto{\pgfqpoint{3.797024in}{2.446484in}}%
\pgfpathmoveto{\pgfqpoint{3.792482in}{2.449433in}}%
\pgfpathlineto{\pgfqpoint{3.792482in}{2.449433in}}%
\pgfpathlineto{\pgfqpoint{3.792482in}{2.452382in}}%
\pgfpathlineto{\pgfqpoint{3.797024in}{2.452382in}}%
\pgfpathlineto{\pgfqpoint{3.797024in}{2.449433in}}%
\pgfpathmoveto{\pgfqpoint{3.797024in}{2.446484in}}%
\pgfpathlineto{\pgfqpoint{3.797024in}{2.446484in}}%
\pgfpathlineto{\pgfqpoint{3.797024in}{2.449433in}}%
\pgfpathlineto{\pgfqpoint{3.801565in}{2.449433in}}%
\pgfpathlineto{\pgfqpoint{3.801565in}{2.446484in}}%
\pgfpathmoveto{\pgfqpoint{3.769777in}{2.461230in}}%
\pgfpathlineto{\pgfqpoint{3.769777in}{2.461230in}}%
\pgfpathlineto{\pgfqpoint{3.769777in}{2.464179in}}%
\pgfpathlineto{\pgfqpoint{3.774318in}{2.464179in}}%
\pgfpathlineto{\pgfqpoint{3.774318in}{2.461230in}}%
\pgfpathmoveto{\pgfqpoint{3.765236in}{2.464179in}}%
\pgfpathlineto{\pgfqpoint{3.765236in}{2.464179in}}%
\pgfpathlineto{\pgfqpoint{3.765236in}{2.467128in}}%
\pgfpathlineto{\pgfqpoint{3.769777in}{2.467128in}}%
\pgfpathlineto{\pgfqpoint{3.769777in}{2.464179in}}%
\pgfpathmoveto{\pgfqpoint{3.765236in}{2.467128in}}%
\pgfpathlineto{\pgfqpoint{3.765236in}{2.467128in}}%
\pgfpathlineto{\pgfqpoint{3.765236in}{2.470078in}}%
\pgfpathlineto{\pgfqpoint{3.769777in}{2.470078in}}%
\pgfpathlineto{\pgfqpoint{3.769777in}{2.467128in}}%
\pgfpathmoveto{\pgfqpoint{3.769777in}{2.464179in}}%
\pgfpathlineto{\pgfqpoint{3.769777in}{2.464179in}}%
\pgfpathlineto{\pgfqpoint{3.769777in}{2.467128in}}%
\pgfpathlineto{\pgfqpoint{3.774318in}{2.467128in}}%
\pgfpathlineto{\pgfqpoint{3.774318in}{2.464179in}}%
\pgfpathmoveto{\pgfqpoint{3.774318in}{2.458281in}}%
\pgfpathlineto{\pgfqpoint{3.774318in}{2.458281in}}%
\pgfpathlineto{\pgfqpoint{3.774318in}{2.461230in}}%
\pgfpathlineto{\pgfqpoint{3.778859in}{2.461230in}}%
\pgfpathlineto{\pgfqpoint{3.778859in}{2.458281in}}%
\pgfpathmoveto{\pgfqpoint{3.774318in}{2.461230in}}%
\pgfpathlineto{\pgfqpoint{3.774318in}{2.461230in}}%
\pgfpathlineto{\pgfqpoint{3.774318in}{2.464179in}}%
\pgfpathlineto{\pgfqpoint{3.778859in}{2.464179in}}%
\pgfpathlineto{\pgfqpoint{3.778859in}{2.461230in}}%
\pgfpathmoveto{\pgfqpoint{3.778859in}{2.458281in}}%
\pgfpathlineto{\pgfqpoint{3.778859in}{2.458281in}}%
\pgfpathlineto{\pgfqpoint{3.778859in}{2.461230in}}%
\pgfpathlineto{\pgfqpoint{3.783400in}{2.461230in}}%
\pgfpathlineto{\pgfqpoint{3.783400in}{2.458281in}}%
\pgfpathmoveto{\pgfqpoint{3.669872in}{2.526113in}}%
\pgfpathlineto{\pgfqpoint{3.669872in}{2.526113in}}%
\pgfpathlineto{\pgfqpoint{3.669872in}{2.529063in}}%
\pgfpathlineto{\pgfqpoint{3.674413in}{2.529063in}}%
\pgfpathlineto{\pgfqpoint{3.674413in}{2.526113in}}%
\pgfpathmoveto{\pgfqpoint{3.688036in}{2.514316in}}%
\pgfpathlineto{\pgfqpoint{3.688036in}{2.514316in}}%
\pgfpathlineto{\pgfqpoint{3.688036in}{2.517266in}}%
\pgfpathlineto{\pgfqpoint{3.692577in}{2.517266in}}%
\pgfpathlineto{\pgfqpoint{3.692577in}{2.514316in}}%
\pgfpathmoveto{\pgfqpoint{3.678954in}{2.520215in}}%
\pgfpathlineto{\pgfqpoint{3.678954in}{2.520215in}}%
\pgfpathlineto{\pgfqpoint{3.678954in}{2.523164in}}%
\pgfpathlineto{\pgfqpoint{3.683495in}{2.523164in}}%
\pgfpathlineto{\pgfqpoint{3.683495in}{2.520215in}}%
\pgfpathmoveto{\pgfqpoint{3.674413in}{2.523164in}}%
\pgfpathlineto{\pgfqpoint{3.674413in}{2.523164in}}%
\pgfpathlineto{\pgfqpoint{3.674413in}{2.526113in}}%
\pgfpathlineto{\pgfqpoint{3.678954in}{2.526113in}}%
\pgfpathlineto{\pgfqpoint{3.678954in}{2.523164in}}%
\pgfpathmoveto{\pgfqpoint{3.674413in}{2.526113in}}%
\pgfpathlineto{\pgfqpoint{3.674413in}{2.526113in}}%
\pgfpathlineto{\pgfqpoint{3.674413in}{2.529063in}}%
\pgfpathlineto{\pgfqpoint{3.678954in}{2.529063in}}%
\pgfpathlineto{\pgfqpoint{3.678954in}{2.526113in}}%
\pgfpathmoveto{\pgfqpoint{3.678954in}{2.523164in}}%
\pgfpathlineto{\pgfqpoint{3.678954in}{2.523164in}}%
\pgfpathlineto{\pgfqpoint{3.678954in}{2.526113in}}%
\pgfpathlineto{\pgfqpoint{3.683495in}{2.526113in}}%
\pgfpathlineto{\pgfqpoint{3.683495in}{2.523164in}}%
\pgfpathmoveto{\pgfqpoint{3.683495in}{2.517266in}}%
\pgfpathlineto{\pgfqpoint{3.683495in}{2.517266in}}%
\pgfpathlineto{\pgfqpoint{3.683495in}{2.520215in}}%
\pgfpathlineto{\pgfqpoint{3.688036in}{2.520215in}}%
\pgfpathlineto{\pgfqpoint{3.688036in}{2.517266in}}%
\pgfpathmoveto{\pgfqpoint{3.683495in}{2.520215in}}%
\pgfpathlineto{\pgfqpoint{3.683495in}{2.520215in}}%
\pgfpathlineto{\pgfqpoint{3.683495in}{2.523164in}}%
\pgfpathlineto{\pgfqpoint{3.688036in}{2.523164in}}%
\pgfpathlineto{\pgfqpoint{3.688036in}{2.520215in}}%
\pgfpathmoveto{\pgfqpoint{3.688036in}{2.517266in}}%
\pgfpathlineto{\pgfqpoint{3.688036in}{2.517266in}}%
\pgfpathlineto{\pgfqpoint{3.688036in}{2.520215in}}%
\pgfpathlineto{\pgfqpoint{3.692577in}{2.520215in}}%
\pgfpathlineto{\pgfqpoint{3.692577in}{2.517266in}}%
\pgfpathmoveto{\pgfqpoint{3.706201in}{2.502519in}}%
\pgfpathlineto{\pgfqpoint{3.706201in}{2.502519in}}%
\pgfpathlineto{\pgfqpoint{3.706201in}{2.505469in}}%
\pgfpathlineto{\pgfqpoint{3.710742in}{2.505469in}}%
\pgfpathlineto{\pgfqpoint{3.710742in}{2.502519in}}%
\pgfpathmoveto{\pgfqpoint{3.724365in}{2.490722in}}%
\pgfpathlineto{\pgfqpoint{3.724365in}{2.490722in}}%
\pgfpathlineto{\pgfqpoint{3.724365in}{2.493672in}}%
\pgfpathlineto{\pgfqpoint{3.728906in}{2.493672in}}%
\pgfpathlineto{\pgfqpoint{3.728906in}{2.490722in}}%
\pgfpathmoveto{\pgfqpoint{3.715283in}{2.496621in}}%
\pgfpathlineto{\pgfqpoint{3.715283in}{2.496621in}}%
\pgfpathlineto{\pgfqpoint{3.715283in}{2.499570in}}%
\pgfpathlineto{\pgfqpoint{3.719824in}{2.499570in}}%
\pgfpathlineto{\pgfqpoint{3.719824in}{2.496621in}}%
\pgfpathmoveto{\pgfqpoint{3.710742in}{2.499570in}}%
\pgfpathlineto{\pgfqpoint{3.710742in}{2.499570in}}%
\pgfpathlineto{\pgfqpoint{3.710742in}{2.502519in}}%
\pgfpathlineto{\pgfqpoint{3.715283in}{2.502519in}}%
\pgfpathlineto{\pgfqpoint{3.715283in}{2.499570in}}%
\pgfpathmoveto{\pgfqpoint{3.710742in}{2.502519in}}%
\pgfpathlineto{\pgfqpoint{3.710742in}{2.502519in}}%
\pgfpathlineto{\pgfqpoint{3.710742in}{2.505469in}}%
\pgfpathlineto{\pgfqpoint{3.715283in}{2.505469in}}%
\pgfpathlineto{\pgfqpoint{3.715283in}{2.502519in}}%
\pgfpathmoveto{\pgfqpoint{3.715283in}{2.499570in}}%
\pgfpathlineto{\pgfqpoint{3.715283in}{2.499570in}}%
\pgfpathlineto{\pgfqpoint{3.715283in}{2.502519in}}%
\pgfpathlineto{\pgfqpoint{3.719824in}{2.502519in}}%
\pgfpathlineto{\pgfqpoint{3.719824in}{2.499570in}}%
\pgfpathmoveto{\pgfqpoint{3.719824in}{2.493672in}}%
\pgfpathlineto{\pgfqpoint{3.719824in}{2.493672in}}%
\pgfpathlineto{\pgfqpoint{3.719824in}{2.496621in}}%
\pgfpathlineto{\pgfqpoint{3.724365in}{2.496621in}}%
\pgfpathlineto{\pgfqpoint{3.724365in}{2.493672in}}%
\pgfpathmoveto{\pgfqpoint{3.719824in}{2.496621in}}%
\pgfpathlineto{\pgfqpoint{3.719824in}{2.496621in}}%
\pgfpathlineto{\pgfqpoint{3.719824in}{2.499570in}}%
\pgfpathlineto{\pgfqpoint{3.724365in}{2.499570in}}%
\pgfpathlineto{\pgfqpoint{3.724365in}{2.496621in}}%
\pgfpathmoveto{\pgfqpoint{3.724365in}{2.493672in}}%
\pgfpathlineto{\pgfqpoint{3.724365in}{2.493672in}}%
\pgfpathlineto{\pgfqpoint{3.724365in}{2.496621in}}%
\pgfpathlineto{\pgfqpoint{3.728906in}{2.496621in}}%
\pgfpathlineto{\pgfqpoint{3.728906in}{2.493672in}}%
\pgfpathmoveto{\pgfqpoint{3.697118in}{2.508418in}}%
\pgfpathlineto{\pgfqpoint{3.697118in}{2.508418in}}%
\pgfpathlineto{\pgfqpoint{3.697118in}{2.511367in}}%
\pgfpathlineto{\pgfqpoint{3.701660in}{2.511367in}}%
\pgfpathlineto{\pgfqpoint{3.701660in}{2.508418in}}%
\pgfpathmoveto{\pgfqpoint{3.692577in}{2.511367in}}%
\pgfpathlineto{\pgfqpoint{3.692577in}{2.511367in}}%
\pgfpathlineto{\pgfqpoint{3.692577in}{2.514316in}}%
\pgfpathlineto{\pgfqpoint{3.697118in}{2.514316in}}%
\pgfpathlineto{\pgfqpoint{3.697118in}{2.511367in}}%
\pgfpathmoveto{\pgfqpoint{3.692577in}{2.514316in}}%
\pgfpathlineto{\pgfqpoint{3.692577in}{2.514316in}}%
\pgfpathlineto{\pgfqpoint{3.692577in}{2.517266in}}%
\pgfpathlineto{\pgfqpoint{3.697118in}{2.517266in}}%
\pgfpathlineto{\pgfqpoint{3.697118in}{2.514316in}}%
\pgfpathmoveto{\pgfqpoint{3.697118in}{2.511367in}}%
\pgfpathlineto{\pgfqpoint{3.697118in}{2.511367in}}%
\pgfpathlineto{\pgfqpoint{3.697118in}{2.514316in}}%
\pgfpathlineto{\pgfqpoint{3.701660in}{2.514316in}}%
\pgfpathlineto{\pgfqpoint{3.701660in}{2.511367in}}%
\pgfpathmoveto{\pgfqpoint{3.701660in}{2.505469in}}%
\pgfpathlineto{\pgfqpoint{3.701660in}{2.505469in}}%
\pgfpathlineto{\pgfqpoint{3.701660in}{2.508418in}}%
\pgfpathlineto{\pgfqpoint{3.706201in}{2.508418in}}%
\pgfpathlineto{\pgfqpoint{3.706201in}{2.505469in}}%
\pgfpathmoveto{\pgfqpoint{3.701660in}{2.508418in}}%
\pgfpathlineto{\pgfqpoint{3.701660in}{2.508418in}}%
\pgfpathlineto{\pgfqpoint{3.701660in}{2.511367in}}%
\pgfpathlineto{\pgfqpoint{3.706201in}{2.511367in}}%
\pgfpathlineto{\pgfqpoint{3.706201in}{2.508418in}}%
\pgfpathmoveto{\pgfqpoint{3.706201in}{2.505469in}}%
\pgfpathlineto{\pgfqpoint{3.706201in}{2.505469in}}%
\pgfpathlineto{\pgfqpoint{3.706201in}{2.508418in}}%
\pgfpathlineto{\pgfqpoint{3.710742in}{2.508418in}}%
\pgfpathlineto{\pgfqpoint{3.710742in}{2.505469in}}%
\pgfpathmoveto{\pgfqpoint{3.660789in}{2.532012in}}%
\pgfpathlineto{\pgfqpoint{3.660789in}{2.532012in}}%
\pgfpathlineto{\pgfqpoint{3.660789in}{2.534961in}}%
\pgfpathlineto{\pgfqpoint{3.665330in}{2.534961in}}%
\pgfpathlineto{\pgfqpoint{3.665330in}{2.532012in}}%
\pgfpathmoveto{\pgfqpoint{3.656248in}{2.534961in}}%
\pgfpathlineto{\pgfqpoint{3.656248in}{2.534961in}}%
\pgfpathlineto{\pgfqpoint{3.656248in}{2.537910in}}%
\pgfpathlineto{\pgfqpoint{3.660789in}{2.537910in}}%
\pgfpathlineto{\pgfqpoint{3.660789in}{2.534961in}}%
\pgfpathmoveto{\pgfqpoint{3.656248in}{2.537910in}}%
\pgfpathlineto{\pgfqpoint{3.656248in}{2.537910in}}%
\pgfpathlineto{\pgfqpoint{3.656248in}{2.540860in}}%
\pgfpathlineto{\pgfqpoint{3.660789in}{2.540860in}}%
\pgfpathlineto{\pgfqpoint{3.660789in}{2.537910in}}%
\pgfpathmoveto{\pgfqpoint{3.660789in}{2.534961in}}%
\pgfpathlineto{\pgfqpoint{3.660789in}{2.534961in}}%
\pgfpathlineto{\pgfqpoint{3.660789in}{2.537910in}}%
\pgfpathlineto{\pgfqpoint{3.665330in}{2.537910in}}%
\pgfpathlineto{\pgfqpoint{3.665330in}{2.534961in}}%
\pgfpathmoveto{\pgfqpoint{3.665330in}{2.529063in}}%
\pgfpathlineto{\pgfqpoint{3.665330in}{2.529063in}}%
\pgfpathlineto{\pgfqpoint{3.665330in}{2.532012in}}%
\pgfpathlineto{\pgfqpoint{3.669872in}{2.532012in}}%
\pgfpathlineto{\pgfqpoint{3.669872in}{2.529063in}}%
\pgfpathmoveto{\pgfqpoint{3.665330in}{2.532012in}}%
\pgfpathlineto{\pgfqpoint{3.665330in}{2.532012in}}%
\pgfpathlineto{\pgfqpoint{3.665330in}{2.534961in}}%
\pgfpathlineto{\pgfqpoint{3.669872in}{2.534961in}}%
\pgfpathlineto{\pgfqpoint{3.669872in}{2.532012in}}%
\pgfpathmoveto{\pgfqpoint{3.669872in}{2.529063in}}%
\pgfpathlineto{\pgfqpoint{3.669872in}{2.529063in}}%
\pgfpathlineto{\pgfqpoint{3.669872in}{2.532012in}}%
\pgfpathlineto{\pgfqpoint{3.674413in}{2.532012in}}%
\pgfpathlineto{\pgfqpoint{3.674413in}{2.529063in}}%
\pgfpathmoveto{\pgfqpoint{3.733448in}{2.484824in}}%
\pgfpathlineto{\pgfqpoint{3.733448in}{2.484824in}}%
\pgfpathlineto{\pgfqpoint{3.733448in}{2.487773in}}%
\pgfpathlineto{\pgfqpoint{3.737989in}{2.487773in}}%
\pgfpathlineto{\pgfqpoint{3.737989in}{2.484824in}}%
\pgfpathmoveto{\pgfqpoint{3.728906in}{2.487773in}}%
\pgfpathlineto{\pgfqpoint{3.728906in}{2.487773in}}%
\pgfpathlineto{\pgfqpoint{3.728906in}{2.490722in}}%
\pgfpathlineto{\pgfqpoint{3.733448in}{2.490722in}}%
\pgfpathlineto{\pgfqpoint{3.733448in}{2.487773in}}%
\pgfpathmoveto{\pgfqpoint{3.728906in}{2.490722in}}%
\pgfpathlineto{\pgfqpoint{3.728906in}{2.490722in}}%
\pgfpathlineto{\pgfqpoint{3.728906in}{2.493672in}}%
\pgfpathlineto{\pgfqpoint{3.733448in}{2.493672in}}%
\pgfpathlineto{\pgfqpoint{3.733448in}{2.490722in}}%
\pgfpathmoveto{\pgfqpoint{3.733448in}{2.487773in}}%
\pgfpathlineto{\pgfqpoint{3.733448in}{2.487773in}}%
\pgfpathlineto{\pgfqpoint{3.733448in}{2.490722in}}%
\pgfpathlineto{\pgfqpoint{3.737989in}{2.490722in}}%
\pgfpathlineto{\pgfqpoint{3.737989in}{2.487773in}}%
\pgfpathmoveto{\pgfqpoint{3.737989in}{2.481874in}}%
\pgfpathlineto{\pgfqpoint{3.737989in}{2.481874in}}%
\pgfpathlineto{\pgfqpoint{3.737989in}{2.484824in}}%
\pgfpathlineto{\pgfqpoint{3.742530in}{2.484824in}}%
\pgfpathlineto{\pgfqpoint{3.742530in}{2.481874in}}%
\pgfpathmoveto{\pgfqpoint{3.737989in}{2.484824in}}%
\pgfpathlineto{\pgfqpoint{3.737989in}{2.484824in}}%
\pgfpathlineto{\pgfqpoint{3.737989in}{2.487773in}}%
\pgfpathlineto{\pgfqpoint{3.742530in}{2.487773in}}%
\pgfpathlineto{\pgfqpoint{3.742530in}{2.484824in}}%
\pgfpathmoveto{\pgfqpoint{3.742530in}{2.481874in}}%
\pgfpathlineto{\pgfqpoint{3.742530in}{2.481874in}}%
\pgfpathlineto{\pgfqpoint{3.742530in}{2.484824in}}%
\pgfpathlineto{\pgfqpoint{3.747071in}{2.484824in}}%
\pgfpathlineto{\pgfqpoint{3.747071in}{2.481874in}}%
\pgfpathmoveto{\pgfqpoint{3.887845in}{2.384550in}}%
\pgfpathlineto{\pgfqpoint{3.887845in}{2.384550in}}%
\pgfpathlineto{\pgfqpoint{3.887845in}{2.387500in}}%
\pgfpathlineto{\pgfqpoint{3.892386in}{2.387500in}}%
\pgfpathlineto{\pgfqpoint{3.892386in}{2.384550in}}%
\pgfpathmoveto{\pgfqpoint{3.906009in}{2.372754in}}%
\pgfpathlineto{\pgfqpoint{3.906009in}{2.372754in}}%
\pgfpathlineto{\pgfqpoint{3.906009in}{2.375703in}}%
\pgfpathlineto{\pgfqpoint{3.910551in}{2.375703in}}%
\pgfpathlineto{\pgfqpoint{3.910551in}{2.372754in}}%
\pgfpathmoveto{\pgfqpoint{3.896927in}{2.378652in}}%
\pgfpathlineto{\pgfqpoint{3.896927in}{2.378652in}}%
\pgfpathlineto{\pgfqpoint{3.896927in}{2.381601in}}%
\pgfpathlineto{\pgfqpoint{3.901468in}{2.381601in}}%
\pgfpathlineto{\pgfqpoint{3.901468in}{2.378652in}}%
\pgfpathmoveto{\pgfqpoint{3.892386in}{2.381601in}}%
\pgfpathlineto{\pgfqpoint{3.892386in}{2.381601in}}%
\pgfpathlineto{\pgfqpoint{3.892386in}{2.384550in}}%
\pgfpathlineto{\pgfqpoint{3.896927in}{2.384550in}}%
\pgfpathlineto{\pgfqpoint{3.896927in}{2.381601in}}%
\pgfpathmoveto{\pgfqpoint{3.892386in}{2.384550in}}%
\pgfpathlineto{\pgfqpoint{3.892386in}{2.384550in}}%
\pgfpathlineto{\pgfqpoint{3.892386in}{2.387500in}}%
\pgfpathlineto{\pgfqpoint{3.896927in}{2.387500in}}%
\pgfpathlineto{\pgfqpoint{3.896927in}{2.384550in}}%
\pgfpathmoveto{\pgfqpoint{3.896927in}{2.381601in}}%
\pgfpathlineto{\pgfqpoint{3.896927in}{2.381601in}}%
\pgfpathlineto{\pgfqpoint{3.896927in}{2.384550in}}%
\pgfpathlineto{\pgfqpoint{3.901468in}{2.384550in}}%
\pgfpathlineto{\pgfqpoint{3.901468in}{2.381601in}}%
\pgfpathmoveto{\pgfqpoint{3.901468in}{2.375703in}}%
\pgfpathlineto{\pgfqpoint{3.901468in}{2.375703in}}%
\pgfpathlineto{\pgfqpoint{3.901468in}{2.378652in}}%
\pgfpathlineto{\pgfqpoint{3.906009in}{2.378652in}}%
\pgfpathlineto{\pgfqpoint{3.906009in}{2.375703in}}%
\pgfpathmoveto{\pgfqpoint{3.901468in}{2.378652in}}%
\pgfpathlineto{\pgfqpoint{3.901468in}{2.378652in}}%
\pgfpathlineto{\pgfqpoint{3.901468in}{2.381601in}}%
\pgfpathlineto{\pgfqpoint{3.906009in}{2.381601in}}%
\pgfpathlineto{\pgfqpoint{3.906009in}{2.378652in}}%
\pgfpathmoveto{\pgfqpoint{3.906009in}{2.375703in}}%
\pgfpathlineto{\pgfqpoint{3.906009in}{2.375703in}}%
\pgfpathlineto{\pgfqpoint{3.906009in}{2.378652in}}%
\pgfpathlineto{\pgfqpoint{3.910551in}{2.378652in}}%
\pgfpathlineto{\pgfqpoint{3.910551in}{2.375703in}}%
\pgfpathmoveto{\pgfqpoint{3.924174in}{2.360957in}}%
\pgfpathlineto{\pgfqpoint{3.924174in}{2.360957in}}%
\pgfpathlineto{\pgfqpoint{3.924174in}{2.363906in}}%
\pgfpathlineto{\pgfqpoint{3.928715in}{2.363906in}}%
\pgfpathlineto{\pgfqpoint{3.928715in}{2.360957in}}%
\pgfpathmoveto{\pgfqpoint{3.942338in}{2.349160in}}%
\pgfpathlineto{\pgfqpoint{3.942338in}{2.349160in}}%
\pgfpathlineto{\pgfqpoint{3.942338in}{2.352109in}}%
\pgfpathlineto{\pgfqpoint{3.946879in}{2.352109in}}%
\pgfpathlineto{\pgfqpoint{3.946879in}{2.349160in}}%
\pgfpathmoveto{\pgfqpoint{3.933256in}{2.355058in}}%
\pgfpathlineto{\pgfqpoint{3.933256in}{2.355058in}}%
\pgfpathlineto{\pgfqpoint{3.933256in}{2.358008in}}%
\pgfpathlineto{\pgfqpoint{3.937797in}{2.358008in}}%
\pgfpathlineto{\pgfqpoint{3.937797in}{2.355058in}}%
\pgfpathmoveto{\pgfqpoint{3.928715in}{2.358008in}}%
\pgfpathlineto{\pgfqpoint{3.928715in}{2.358008in}}%
\pgfpathlineto{\pgfqpoint{3.928715in}{2.360957in}}%
\pgfpathlineto{\pgfqpoint{3.933256in}{2.360957in}}%
\pgfpathlineto{\pgfqpoint{3.933256in}{2.358008in}}%
\pgfpathmoveto{\pgfqpoint{3.928715in}{2.360957in}}%
\pgfpathlineto{\pgfqpoint{3.928715in}{2.360957in}}%
\pgfpathlineto{\pgfqpoint{3.928715in}{2.363906in}}%
\pgfpathlineto{\pgfqpoint{3.933256in}{2.363906in}}%
\pgfpathlineto{\pgfqpoint{3.933256in}{2.360957in}}%
\pgfpathmoveto{\pgfqpoint{3.933256in}{2.358008in}}%
\pgfpathlineto{\pgfqpoint{3.933256in}{2.358008in}}%
\pgfpathlineto{\pgfqpoint{3.933256in}{2.360957in}}%
\pgfpathlineto{\pgfqpoint{3.937797in}{2.360957in}}%
\pgfpathlineto{\pgfqpoint{3.937797in}{2.358008in}}%
\pgfpathmoveto{\pgfqpoint{3.937797in}{2.352109in}}%
\pgfpathlineto{\pgfqpoint{3.937797in}{2.352109in}}%
\pgfpathlineto{\pgfqpoint{3.937797in}{2.355058in}}%
\pgfpathlineto{\pgfqpoint{3.942338in}{2.355058in}}%
\pgfpathlineto{\pgfqpoint{3.942338in}{2.352109in}}%
\pgfpathmoveto{\pgfqpoint{3.937797in}{2.355058in}}%
\pgfpathlineto{\pgfqpoint{3.937797in}{2.355058in}}%
\pgfpathlineto{\pgfqpoint{3.937797in}{2.358008in}}%
\pgfpathlineto{\pgfqpoint{3.942338in}{2.358008in}}%
\pgfpathlineto{\pgfqpoint{3.942338in}{2.355058in}}%
\pgfpathmoveto{\pgfqpoint{3.942338in}{2.352109in}}%
\pgfpathlineto{\pgfqpoint{3.942338in}{2.352109in}}%
\pgfpathlineto{\pgfqpoint{3.942338in}{2.355058in}}%
\pgfpathlineto{\pgfqpoint{3.946879in}{2.355058in}}%
\pgfpathlineto{\pgfqpoint{3.946879in}{2.352109in}}%
\pgfpathmoveto{\pgfqpoint{3.915092in}{2.366855in}}%
\pgfpathlineto{\pgfqpoint{3.915092in}{2.366855in}}%
\pgfpathlineto{\pgfqpoint{3.915092in}{2.369804in}}%
\pgfpathlineto{\pgfqpoint{3.919633in}{2.369804in}}%
\pgfpathlineto{\pgfqpoint{3.919633in}{2.366855in}}%
\pgfpathmoveto{\pgfqpoint{3.910551in}{2.369804in}}%
\pgfpathlineto{\pgfqpoint{3.910551in}{2.369804in}}%
\pgfpathlineto{\pgfqpoint{3.910551in}{2.372754in}}%
\pgfpathlineto{\pgfqpoint{3.915092in}{2.372754in}}%
\pgfpathlineto{\pgfqpoint{3.915092in}{2.369804in}}%
\pgfpathmoveto{\pgfqpoint{3.910551in}{2.372754in}}%
\pgfpathlineto{\pgfqpoint{3.910551in}{2.372754in}}%
\pgfpathlineto{\pgfqpoint{3.910551in}{2.375703in}}%
\pgfpathlineto{\pgfqpoint{3.915092in}{2.375703in}}%
\pgfpathlineto{\pgfqpoint{3.915092in}{2.372754in}}%
\pgfpathmoveto{\pgfqpoint{3.915092in}{2.369804in}}%
\pgfpathlineto{\pgfqpoint{3.915092in}{2.369804in}}%
\pgfpathlineto{\pgfqpoint{3.915092in}{2.372754in}}%
\pgfpathlineto{\pgfqpoint{3.919633in}{2.372754in}}%
\pgfpathlineto{\pgfqpoint{3.919633in}{2.369804in}}%
\pgfpathmoveto{\pgfqpoint{3.919633in}{2.363906in}}%
\pgfpathlineto{\pgfqpoint{3.919633in}{2.363906in}}%
\pgfpathlineto{\pgfqpoint{3.919633in}{2.366855in}}%
\pgfpathlineto{\pgfqpoint{3.924174in}{2.366855in}}%
\pgfpathlineto{\pgfqpoint{3.924174in}{2.363906in}}%
\pgfpathmoveto{\pgfqpoint{3.919633in}{2.366855in}}%
\pgfpathlineto{\pgfqpoint{3.919633in}{2.366855in}}%
\pgfpathlineto{\pgfqpoint{3.919633in}{2.369804in}}%
\pgfpathlineto{\pgfqpoint{3.924174in}{2.369804in}}%
\pgfpathlineto{\pgfqpoint{3.924174in}{2.366855in}}%
\pgfpathmoveto{\pgfqpoint{3.924174in}{2.363906in}}%
\pgfpathlineto{\pgfqpoint{3.924174in}{2.363906in}}%
\pgfpathlineto{\pgfqpoint{3.924174in}{2.366855in}}%
\pgfpathlineto{\pgfqpoint{3.928715in}{2.366855in}}%
\pgfpathlineto{\pgfqpoint{3.928715in}{2.363906in}}%
\pgfpathmoveto{\pgfqpoint{3.815188in}{2.431738in}}%
\pgfpathlineto{\pgfqpoint{3.815188in}{2.431738in}}%
\pgfpathlineto{\pgfqpoint{3.815188in}{2.434687in}}%
\pgfpathlineto{\pgfqpoint{3.819729in}{2.434687in}}%
\pgfpathlineto{\pgfqpoint{3.819729in}{2.431738in}}%
\pgfpathmoveto{\pgfqpoint{3.833352in}{2.419941in}}%
\pgfpathlineto{\pgfqpoint{3.833352in}{2.419941in}}%
\pgfpathlineto{\pgfqpoint{3.833352in}{2.422890in}}%
\pgfpathlineto{\pgfqpoint{3.837893in}{2.422890in}}%
\pgfpathlineto{\pgfqpoint{3.837893in}{2.419941in}}%
\pgfpathmoveto{\pgfqpoint{3.824270in}{2.425839in}}%
\pgfpathlineto{\pgfqpoint{3.824270in}{2.425839in}}%
\pgfpathlineto{\pgfqpoint{3.824270in}{2.428789in}}%
\pgfpathlineto{\pgfqpoint{3.828811in}{2.428789in}}%
\pgfpathlineto{\pgfqpoint{3.828811in}{2.425839in}}%
\pgfpathmoveto{\pgfqpoint{3.819729in}{2.428789in}}%
\pgfpathlineto{\pgfqpoint{3.819729in}{2.428789in}}%
\pgfpathlineto{\pgfqpoint{3.819729in}{2.431738in}}%
\pgfpathlineto{\pgfqpoint{3.824270in}{2.431738in}}%
\pgfpathlineto{\pgfqpoint{3.824270in}{2.428789in}}%
\pgfpathmoveto{\pgfqpoint{3.819729in}{2.431738in}}%
\pgfpathlineto{\pgfqpoint{3.819729in}{2.431738in}}%
\pgfpathlineto{\pgfqpoint{3.819729in}{2.434687in}}%
\pgfpathlineto{\pgfqpoint{3.824270in}{2.434687in}}%
\pgfpathlineto{\pgfqpoint{3.824270in}{2.431738in}}%
\pgfpathmoveto{\pgfqpoint{3.824270in}{2.428789in}}%
\pgfpathlineto{\pgfqpoint{3.824270in}{2.428789in}}%
\pgfpathlineto{\pgfqpoint{3.824270in}{2.431738in}}%
\pgfpathlineto{\pgfqpoint{3.828811in}{2.431738in}}%
\pgfpathlineto{\pgfqpoint{3.828811in}{2.428789in}}%
\pgfpathmoveto{\pgfqpoint{3.828811in}{2.422890in}}%
\pgfpathlineto{\pgfqpoint{3.828811in}{2.422890in}}%
\pgfpathlineto{\pgfqpoint{3.828811in}{2.425839in}}%
\pgfpathlineto{\pgfqpoint{3.833352in}{2.425839in}}%
\pgfpathlineto{\pgfqpoint{3.833352in}{2.422890in}}%
\pgfpathmoveto{\pgfqpoint{3.828811in}{2.425839in}}%
\pgfpathlineto{\pgfqpoint{3.828811in}{2.425839in}}%
\pgfpathlineto{\pgfqpoint{3.828811in}{2.428789in}}%
\pgfpathlineto{\pgfqpoint{3.833352in}{2.428789in}}%
\pgfpathlineto{\pgfqpoint{3.833352in}{2.425839in}}%
\pgfpathmoveto{\pgfqpoint{3.833352in}{2.422890in}}%
\pgfpathlineto{\pgfqpoint{3.833352in}{2.422890in}}%
\pgfpathlineto{\pgfqpoint{3.833352in}{2.425839in}}%
\pgfpathlineto{\pgfqpoint{3.837893in}{2.425839in}}%
\pgfpathlineto{\pgfqpoint{3.837893in}{2.422890in}}%
\pgfpathmoveto{\pgfqpoint{3.851517in}{2.408144in}}%
\pgfpathlineto{\pgfqpoint{3.851517in}{2.408144in}}%
\pgfpathlineto{\pgfqpoint{3.851517in}{2.411093in}}%
\pgfpathlineto{\pgfqpoint{3.856058in}{2.411093in}}%
\pgfpathlineto{\pgfqpoint{3.856058in}{2.408144in}}%
\pgfpathmoveto{\pgfqpoint{3.869681in}{2.396347in}}%
\pgfpathlineto{\pgfqpoint{3.869681in}{2.396347in}}%
\pgfpathlineto{\pgfqpoint{3.869681in}{2.399296in}}%
\pgfpathlineto{\pgfqpoint{3.874222in}{2.399296in}}%
\pgfpathlineto{\pgfqpoint{3.874222in}{2.396347in}}%
\pgfpathmoveto{\pgfqpoint{3.860599in}{2.402246in}}%
\pgfpathlineto{\pgfqpoint{3.860599in}{2.402246in}}%
\pgfpathlineto{\pgfqpoint{3.860599in}{2.405195in}}%
\pgfpathlineto{\pgfqpoint{3.865140in}{2.405195in}}%
\pgfpathlineto{\pgfqpoint{3.865140in}{2.402246in}}%
\pgfpathmoveto{\pgfqpoint{3.856058in}{2.405195in}}%
\pgfpathlineto{\pgfqpoint{3.856058in}{2.405195in}}%
\pgfpathlineto{\pgfqpoint{3.856058in}{2.408144in}}%
\pgfpathlineto{\pgfqpoint{3.860599in}{2.408144in}}%
\pgfpathlineto{\pgfqpoint{3.860599in}{2.405195in}}%
\pgfpathmoveto{\pgfqpoint{3.856058in}{2.408144in}}%
\pgfpathlineto{\pgfqpoint{3.856058in}{2.408144in}}%
\pgfpathlineto{\pgfqpoint{3.856058in}{2.411093in}}%
\pgfpathlineto{\pgfqpoint{3.860599in}{2.411093in}}%
\pgfpathlineto{\pgfqpoint{3.860599in}{2.408144in}}%
\pgfpathmoveto{\pgfqpoint{3.860599in}{2.405195in}}%
\pgfpathlineto{\pgfqpoint{3.860599in}{2.405195in}}%
\pgfpathlineto{\pgfqpoint{3.860599in}{2.408144in}}%
\pgfpathlineto{\pgfqpoint{3.865140in}{2.408144in}}%
\pgfpathlineto{\pgfqpoint{3.865140in}{2.405195in}}%
\pgfpathmoveto{\pgfqpoint{3.865140in}{2.399296in}}%
\pgfpathlineto{\pgfqpoint{3.865140in}{2.399296in}}%
\pgfpathlineto{\pgfqpoint{3.865140in}{2.402246in}}%
\pgfpathlineto{\pgfqpoint{3.869681in}{2.402246in}}%
\pgfpathlineto{\pgfqpoint{3.869681in}{2.399296in}}%
\pgfpathmoveto{\pgfqpoint{3.865140in}{2.402246in}}%
\pgfpathlineto{\pgfqpoint{3.865140in}{2.402246in}}%
\pgfpathlineto{\pgfqpoint{3.865140in}{2.405195in}}%
\pgfpathlineto{\pgfqpoint{3.869681in}{2.405195in}}%
\pgfpathlineto{\pgfqpoint{3.869681in}{2.402246in}}%
\pgfpathmoveto{\pgfqpoint{3.869681in}{2.399296in}}%
\pgfpathlineto{\pgfqpoint{3.869681in}{2.399296in}}%
\pgfpathlineto{\pgfqpoint{3.869681in}{2.402246in}}%
\pgfpathlineto{\pgfqpoint{3.874222in}{2.402246in}}%
\pgfpathlineto{\pgfqpoint{3.874222in}{2.399296in}}%
\pgfpathmoveto{\pgfqpoint{3.842434in}{2.414042in}}%
\pgfpathlineto{\pgfqpoint{3.842434in}{2.414042in}}%
\pgfpathlineto{\pgfqpoint{3.842434in}{2.416992in}}%
\pgfpathlineto{\pgfqpoint{3.846975in}{2.416992in}}%
\pgfpathlineto{\pgfqpoint{3.846975in}{2.414042in}}%
\pgfpathmoveto{\pgfqpoint{3.837893in}{2.416992in}}%
\pgfpathlineto{\pgfqpoint{3.837893in}{2.416992in}}%
\pgfpathlineto{\pgfqpoint{3.837893in}{2.419941in}}%
\pgfpathlineto{\pgfqpoint{3.842434in}{2.419941in}}%
\pgfpathlineto{\pgfqpoint{3.842434in}{2.416992in}}%
\pgfpathmoveto{\pgfqpoint{3.837893in}{2.419941in}}%
\pgfpathlineto{\pgfqpoint{3.837893in}{2.419941in}}%
\pgfpathlineto{\pgfqpoint{3.837893in}{2.422890in}}%
\pgfpathlineto{\pgfqpoint{3.842434in}{2.422890in}}%
\pgfpathlineto{\pgfqpoint{3.842434in}{2.419941in}}%
\pgfpathmoveto{\pgfqpoint{3.842434in}{2.416992in}}%
\pgfpathlineto{\pgfqpoint{3.842434in}{2.416992in}}%
\pgfpathlineto{\pgfqpoint{3.842434in}{2.419941in}}%
\pgfpathlineto{\pgfqpoint{3.846975in}{2.419941in}}%
\pgfpathlineto{\pgfqpoint{3.846975in}{2.416992in}}%
\pgfpathmoveto{\pgfqpoint{3.846975in}{2.411093in}}%
\pgfpathlineto{\pgfqpoint{3.846975in}{2.411093in}}%
\pgfpathlineto{\pgfqpoint{3.846975in}{2.414042in}}%
\pgfpathlineto{\pgfqpoint{3.851517in}{2.414042in}}%
\pgfpathlineto{\pgfqpoint{3.851517in}{2.411093in}}%
\pgfpathmoveto{\pgfqpoint{3.846975in}{2.414042in}}%
\pgfpathlineto{\pgfqpoint{3.846975in}{2.414042in}}%
\pgfpathlineto{\pgfqpoint{3.846975in}{2.416992in}}%
\pgfpathlineto{\pgfqpoint{3.851517in}{2.416992in}}%
\pgfpathlineto{\pgfqpoint{3.851517in}{2.414042in}}%
\pgfpathmoveto{\pgfqpoint{3.851517in}{2.411093in}}%
\pgfpathlineto{\pgfqpoint{3.851517in}{2.411093in}}%
\pgfpathlineto{\pgfqpoint{3.851517in}{2.414042in}}%
\pgfpathlineto{\pgfqpoint{3.856058in}{2.414042in}}%
\pgfpathlineto{\pgfqpoint{3.856058in}{2.411093in}}%
\pgfpathmoveto{\pgfqpoint{3.806106in}{2.437636in}}%
\pgfpathlineto{\pgfqpoint{3.806106in}{2.437636in}}%
\pgfpathlineto{\pgfqpoint{3.806106in}{2.440585in}}%
\pgfpathlineto{\pgfqpoint{3.810647in}{2.440585in}}%
\pgfpathlineto{\pgfqpoint{3.810647in}{2.437636in}}%
\pgfpathmoveto{\pgfqpoint{3.801565in}{2.440585in}}%
\pgfpathlineto{\pgfqpoint{3.801565in}{2.440585in}}%
\pgfpathlineto{\pgfqpoint{3.801565in}{2.443535in}}%
\pgfpathlineto{\pgfqpoint{3.806106in}{2.443535in}}%
\pgfpathlineto{\pgfqpoint{3.806106in}{2.440585in}}%
\pgfpathmoveto{\pgfqpoint{3.801565in}{2.443535in}}%
\pgfpathlineto{\pgfqpoint{3.801565in}{2.443535in}}%
\pgfpathlineto{\pgfqpoint{3.801565in}{2.446484in}}%
\pgfpathlineto{\pgfqpoint{3.806106in}{2.446484in}}%
\pgfpathlineto{\pgfqpoint{3.806106in}{2.443535in}}%
\pgfpathmoveto{\pgfqpoint{3.806106in}{2.440585in}}%
\pgfpathlineto{\pgfqpoint{3.806106in}{2.440585in}}%
\pgfpathlineto{\pgfqpoint{3.806106in}{2.443535in}}%
\pgfpathlineto{\pgfqpoint{3.810647in}{2.443535in}}%
\pgfpathlineto{\pgfqpoint{3.810647in}{2.440585in}}%
\pgfpathmoveto{\pgfqpoint{3.810647in}{2.434687in}}%
\pgfpathlineto{\pgfqpoint{3.810647in}{2.434687in}}%
\pgfpathlineto{\pgfqpoint{3.810647in}{2.437636in}}%
\pgfpathlineto{\pgfqpoint{3.815188in}{2.437636in}}%
\pgfpathlineto{\pgfqpoint{3.815188in}{2.434687in}}%
\pgfpathmoveto{\pgfqpoint{3.810647in}{2.437636in}}%
\pgfpathlineto{\pgfqpoint{3.810647in}{2.437636in}}%
\pgfpathlineto{\pgfqpoint{3.810647in}{2.440585in}}%
\pgfpathlineto{\pgfqpoint{3.815188in}{2.440585in}}%
\pgfpathlineto{\pgfqpoint{3.815188in}{2.437636in}}%
\pgfpathmoveto{\pgfqpoint{3.815188in}{2.434687in}}%
\pgfpathlineto{\pgfqpoint{3.815188in}{2.434687in}}%
\pgfpathlineto{\pgfqpoint{3.815188in}{2.437636in}}%
\pgfpathlineto{\pgfqpoint{3.819729in}{2.437636in}}%
\pgfpathlineto{\pgfqpoint{3.819729in}{2.434687in}}%
\pgfpathmoveto{\pgfqpoint{3.878763in}{2.390449in}}%
\pgfpathlineto{\pgfqpoint{3.878763in}{2.390449in}}%
\pgfpathlineto{\pgfqpoint{3.878763in}{2.393398in}}%
\pgfpathlineto{\pgfqpoint{3.883304in}{2.393398in}}%
\pgfpathlineto{\pgfqpoint{3.883304in}{2.390449in}}%
\pgfpathmoveto{\pgfqpoint{3.874222in}{2.393398in}}%
\pgfpathlineto{\pgfqpoint{3.874222in}{2.393398in}}%
\pgfpathlineto{\pgfqpoint{3.874222in}{2.396347in}}%
\pgfpathlineto{\pgfqpoint{3.878763in}{2.396347in}}%
\pgfpathlineto{\pgfqpoint{3.878763in}{2.393398in}}%
\pgfpathmoveto{\pgfqpoint{3.874222in}{2.396347in}}%
\pgfpathlineto{\pgfqpoint{3.874222in}{2.396347in}}%
\pgfpathlineto{\pgfqpoint{3.874222in}{2.399296in}}%
\pgfpathlineto{\pgfqpoint{3.878763in}{2.399296in}}%
\pgfpathlineto{\pgfqpoint{3.878763in}{2.396347in}}%
\pgfpathmoveto{\pgfqpoint{3.878763in}{2.393398in}}%
\pgfpathlineto{\pgfqpoint{3.878763in}{2.393398in}}%
\pgfpathlineto{\pgfqpoint{3.878763in}{2.396347in}}%
\pgfpathlineto{\pgfqpoint{3.883304in}{2.396347in}}%
\pgfpathlineto{\pgfqpoint{3.883304in}{2.393398in}}%
\pgfpathmoveto{\pgfqpoint{3.883304in}{2.387500in}}%
\pgfpathlineto{\pgfqpoint{3.883304in}{2.387500in}}%
\pgfpathlineto{\pgfqpoint{3.883304in}{2.390449in}}%
\pgfpathlineto{\pgfqpoint{3.887845in}{2.390449in}}%
\pgfpathlineto{\pgfqpoint{3.887845in}{2.387500in}}%
\pgfpathmoveto{\pgfqpoint{3.883304in}{2.390449in}}%
\pgfpathlineto{\pgfqpoint{3.883304in}{2.390449in}}%
\pgfpathlineto{\pgfqpoint{3.883304in}{2.393398in}}%
\pgfpathlineto{\pgfqpoint{3.887845in}{2.393398in}}%
\pgfpathlineto{\pgfqpoint{3.887845in}{2.390449in}}%
\pgfpathmoveto{\pgfqpoint{3.887845in}{2.387500in}}%
\pgfpathlineto{\pgfqpoint{3.887845in}{2.387500in}}%
\pgfpathlineto{\pgfqpoint{3.887845in}{2.390449in}}%
\pgfpathlineto{\pgfqpoint{3.892386in}{2.390449in}}%
\pgfpathlineto{\pgfqpoint{3.892386in}{2.387500in}}%
\pgfpathmoveto{\pgfqpoint{4.033156in}{2.290176in}}%
\pgfpathlineto{\pgfqpoint{4.033156in}{2.290176in}}%
\pgfpathlineto{\pgfqpoint{4.033156in}{2.293125in}}%
\pgfpathlineto{\pgfqpoint{4.037697in}{2.293125in}}%
\pgfpathlineto{\pgfqpoint{4.037697in}{2.290176in}}%
\pgfpathmoveto{\pgfqpoint{4.051320in}{2.278379in}}%
\pgfpathlineto{\pgfqpoint{4.051320in}{2.278379in}}%
\pgfpathlineto{\pgfqpoint{4.051320in}{2.281328in}}%
\pgfpathlineto{\pgfqpoint{4.055861in}{2.281328in}}%
\pgfpathlineto{\pgfqpoint{4.055861in}{2.278379in}}%
\pgfpathmoveto{\pgfqpoint{4.042238in}{2.284278in}}%
\pgfpathlineto{\pgfqpoint{4.042238in}{2.284278in}}%
\pgfpathlineto{\pgfqpoint{4.042238in}{2.287227in}}%
\pgfpathlineto{\pgfqpoint{4.046779in}{2.287227in}}%
\pgfpathlineto{\pgfqpoint{4.046779in}{2.284278in}}%
\pgfpathmoveto{\pgfqpoint{4.037697in}{2.287227in}}%
\pgfpathlineto{\pgfqpoint{4.037697in}{2.287227in}}%
\pgfpathlineto{\pgfqpoint{4.037697in}{2.290176in}}%
\pgfpathlineto{\pgfqpoint{4.042238in}{2.290176in}}%
\pgfpathlineto{\pgfqpoint{4.042238in}{2.287227in}}%
\pgfpathmoveto{\pgfqpoint{4.037697in}{2.290176in}}%
\pgfpathlineto{\pgfqpoint{4.037697in}{2.290176in}}%
\pgfpathlineto{\pgfqpoint{4.037697in}{2.293125in}}%
\pgfpathlineto{\pgfqpoint{4.042238in}{2.293125in}}%
\pgfpathlineto{\pgfqpoint{4.042238in}{2.290176in}}%
\pgfpathmoveto{\pgfqpoint{4.042238in}{2.287227in}}%
\pgfpathlineto{\pgfqpoint{4.042238in}{2.287227in}}%
\pgfpathlineto{\pgfqpoint{4.042238in}{2.290176in}}%
\pgfpathlineto{\pgfqpoint{4.046779in}{2.290176in}}%
\pgfpathlineto{\pgfqpoint{4.046779in}{2.287227in}}%
\pgfpathmoveto{\pgfqpoint{4.046779in}{2.281328in}}%
\pgfpathlineto{\pgfqpoint{4.046779in}{2.281328in}}%
\pgfpathlineto{\pgfqpoint{4.046779in}{2.284278in}}%
\pgfpathlineto{\pgfqpoint{4.051320in}{2.284278in}}%
\pgfpathlineto{\pgfqpoint{4.051320in}{2.281328in}}%
\pgfpathmoveto{\pgfqpoint{4.046779in}{2.284278in}}%
\pgfpathlineto{\pgfqpoint{4.046779in}{2.284278in}}%
\pgfpathlineto{\pgfqpoint{4.046779in}{2.287227in}}%
\pgfpathlineto{\pgfqpoint{4.051320in}{2.287227in}}%
\pgfpathlineto{\pgfqpoint{4.051320in}{2.284278in}}%
\pgfpathmoveto{\pgfqpoint{4.051320in}{2.281328in}}%
\pgfpathlineto{\pgfqpoint{4.051320in}{2.281328in}}%
\pgfpathlineto{\pgfqpoint{4.051320in}{2.284278in}}%
\pgfpathlineto{\pgfqpoint{4.055861in}{2.284278in}}%
\pgfpathlineto{\pgfqpoint{4.055861in}{2.281328in}}%
\pgfpathmoveto{\pgfqpoint{4.069483in}{2.266582in}}%
\pgfpathlineto{\pgfqpoint{4.069483in}{2.266582in}}%
\pgfpathlineto{\pgfqpoint{4.069483in}{2.269531in}}%
\pgfpathlineto{\pgfqpoint{4.074024in}{2.269531in}}%
\pgfpathlineto{\pgfqpoint{4.074024in}{2.266582in}}%
\pgfpathmoveto{\pgfqpoint{4.087647in}{2.254785in}}%
\pgfpathlineto{\pgfqpoint{4.087647in}{2.254785in}}%
\pgfpathlineto{\pgfqpoint{4.087647in}{2.257734in}}%
\pgfpathlineto{\pgfqpoint{4.092188in}{2.257734in}}%
\pgfpathlineto{\pgfqpoint{4.092188in}{2.254785in}}%
\pgfpathmoveto{\pgfqpoint{4.078565in}{2.260683in}}%
\pgfpathlineto{\pgfqpoint{4.078565in}{2.260683in}}%
\pgfpathlineto{\pgfqpoint{4.078565in}{2.263633in}}%
\pgfpathlineto{\pgfqpoint{4.083106in}{2.263633in}}%
\pgfpathlineto{\pgfqpoint{4.083106in}{2.260683in}}%
\pgfpathmoveto{\pgfqpoint{4.074024in}{2.263633in}}%
\pgfpathlineto{\pgfqpoint{4.074024in}{2.263633in}}%
\pgfpathlineto{\pgfqpoint{4.074024in}{2.266582in}}%
\pgfpathlineto{\pgfqpoint{4.078565in}{2.266582in}}%
\pgfpathlineto{\pgfqpoint{4.078565in}{2.263633in}}%
\pgfpathmoveto{\pgfqpoint{4.074024in}{2.266582in}}%
\pgfpathlineto{\pgfqpoint{4.074024in}{2.266582in}}%
\pgfpathlineto{\pgfqpoint{4.074024in}{2.269531in}}%
\pgfpathlineto{\pgfqpoint{4.078565in}{2.269531in}}%
\pgfpathlineto{\pgfqpoint{4.078565in}{2.266582in}}%
\pgfpathmoveto{\pgfqpoint{4.078565in}{2.263633in}}%
\pgfpathlineto{\pgfqpoint{4.078565in}{2.263633in}}%
\pgfpathlineto{\pgfqpoint{4.078565in}{2.266582in}}%
\pgfpathlineto{\pgfqpoint{4.083106in}{2.266582in}}%
\pgfpathlineto{\pgfqpoint{4.083106in}{2.263633in}}%
\pgfpathmoveto{\pgfqpoint{4.083106in}{2.257734in}}%
\pgfpathlineto{\pgfqpoint{4.083106in}{2.257734in}}%
\pgfpathlineto{\pgfqpoint{4.083106in}{2.260683in}}%
\pgfpathlineto{\pgfqpoint{4.087647in}{2.260683in}}%
\pgfpathlineto{\pgfqpoint{4.087647in}{2.257734in}}%
\pgfpathmoveto{\pgfqpoint{4.083106in}{2.260683in}}%
\pgfpathlineto{\pgfqpoint{4.083106in}{2.260683in}}%
\pgfpathlineto{\pgfqpoint{4.083106in}{2.263633in}}%
\pgfpathlineto{\pgfqpoint{4.087647in}{2.263633in}}%
\pgfpathlineto{\pgfqpoint{4.087647in}{2.260683in}}%
\pgfpathmoveto{\pgfqpoint{4.087647in}{2.257734in}}%
\pgfpathlineto{\pgfqpoint{4.087647in}{2.257734in}}%
\pgfpathlineto{\pgfqpoint{4.087647in}{2.260683in}}%
\pgfpathlineto{\pgfqpoint{4.092188in}{2.260683in}}%
\pgfpathlineto{\pgfqpoint{4.092188in}{2.257734in}}%
\pgfpathmoveto{\pgfqpoint{4.060402in}{2.272480in}}%
\pgfpathlineto{\pgfqpoint{4.060402in}{2.272480in}}%
\pgfpathlineto{\pgfqpoint{4.060402in}{2.275430in}}%
\pgfpathlineto{\pgfqpoint{4.064942in}{2.275430in}}%
\pgfpathlineto{\pgfqpoint{4.064942in}{2.272480in}}%
\pgfpathmoveto{\pgfqpoint{4.055861in}{2.275430in}}%
\pgfpathlineto{\pgfqpoint{4.055861in}{2.275430in}}%
\pgfpathlineto{\pgfqpoint{4.055861in}{2.278379in}}%
\pgfpathlineto{\pgfqpoint{4.060402in}{2.278379in}}%
\pgfpathlineto{\pgfqpoint{4.060402in}{2.275430in}}%
\pgfpathmoveto{\pgfqpoint{4.055861in}{2.278379in}}%
\pgfpathlineto{\pgfqpoint{4.055861in}{2.278379in}}%
\pgfpathlineto{\pgfqpoint{4.055861in}{2.281328in}}%
\pgfpathlineto{\pgfqpoint{4.060402in}{2.281328in}}%
\pgfpathlineto{\pgfqpoint{4.060402in}{2.278379in}}%
\pgfpathmoveto{\pgfqpoint{4.060402in}{2.275430in}}%
\pgfpathlineto{\pgfqpoint{4.060402in}{2.275430in}}%
\pgfpathlineto{\pgfqpoint{4.060402in}{2.278379in}}%
\pgfpathlineto{\pgfqpoint{4.064942in}{2.278379in}}%
\pgfpathlineto{\pgfqpoint{4.064942in}{2.275430in}}%
\pgfpathmoveto{\pgfqpoint{4.064942in}{2.269531in}}%
\pgfpathlineto{\pgfqpoint{4.064942in}{2.269531in}}%
\pgfpathlineto{\pgfqpoint{4.064942in}{2.272480in}}%
\pgfpathlineto{\pgfqpoint{4.069483in}{2.272480in}}%
\pgfpathlineto{\pgfqpoint{4.069483in}{2.269531in}}%
\pgfpathmoveto{\pgfqpoint{4.064942in}{2.272480in}}%
\pgfpathlineto{\pgfqpoint{4.064942in}{2.272480in}}%
\pgfpathlineto{\pgfqpoint{4.064942in}{2.275430in}}%
\pgfpathlineto{\pgfqpoint{4.069483in}{2.275430in}}%
\pgfpathlineto{\pgfqpoint{4.069483in}{2.272480in}}%
\pgfpathmoveto{\pgfqpoint{4.069483in}{2.269531in}}%
\pgfpathlineto{\pgfqpoint{4.069483in}{2.269531in}}%
\pgfpathlineto{\pgfqpoint{4.069483in}{2.272480in}}%
\pgfpathlineto{\pgfqpoint{4.074024in}{2.272480in}}%
\pgfpathlineto{\pgfqpoint{4.074024in}{2.269531in}}%
\pgfpathmoveto{\pgfqpoint{3.960502in}{2.337363in}}%
\pgfpathlineto{\pgfqpoint{3.960502in}{2.337363in}}%
\pgfpathlineto{\pgfqpoint{3.960502in}{2.340312in}}%
\pgfpathlineto{\pgfqpoint{3.965043in}{2.340312in}}%
\pgfpathlineto{\pgfqpoint{3.965043in}{2.337363in}}%
\pgfpathmoveto{\pgfqpoint{3.978665in}{2.325566in}}%
\pgfpathlineto{\pgfqpoint{3.978665in}{2.325566in}}%
\pgfpathlineto{\pgfqpoint{3.978665in}{2.328516in}}%
\pgfpathlineto{\pgfqpoint{3.983206in}{2.328516in}}%
\pgfpathlineto{\pgfqpoint{3.983206in}{2.325566in}}%
\pgfpathmoveto{\pgfqpoint{3.969584in}{2.331465in}}%
\pgfpathlineto{\pgfqpoint{3.969584in}{2.331465in}}%
\pgfpathlineto{\pgfqpoint{3.969584in}{2.334414in}}%
\pgfpathlineto{\pgfqpoint{3.974124in}{2.334414in}}%
\pgfpathlineto{\pgfqpoint{3.974124in}{2.331465in}}%
\pgfpathmoveto{\pgfqpoint{3.965043in}{2.334414in}}%
\pgfpathlineto{\pgfqpoint{3.965043in}{2.334414in}}%
\pgfpathlineto{\pgfqpoint{3.965043in}{2.337363in}}%
\pgfpathlineto{\pgfqpoint{3.969584in}{2.337363in}}%
\pgfpathlineto{\pgfqpoint{3.969584in}{2.334414in}}%
\pgfpathmoveto{\pgfqpoint{3.965043in}{2.337363in}}%
\pgfpathlineto{\pgfqpoint{3.965043in}{2.337363in}}%
\pgfpathlineto{\pgfqpoint{3.965043in}{2.340312in}}%
\pgfpathlineto{\pgfqpoint{3.969584in}{2.340312in}}%
\pgfpathlineto{\pgfqpoint{3.969584in}{2.337363in}}%
\pgfpathmoveto{\pgfqpoint{3.969584in}{2.334414in}}%
\pgfpathlineto{\pgfqpoint{3.969584in}{2.334414in}}%
\pgfpathlineto{\pgfqpoint{3.969584in}{2.337363in}}%
\pgfpathlineto{\pgfqpoint{3.974124in}{2.337363in}}%
\pgfpathlineto{\pgfqpoint{3.974124in}{2.334414in}}%
\pgfpathmoveto{\pgfqpoint{3.974124in}{2.328516in}}%
\pgfpathlineto{\pgfqpoint{3.974124in}{2.328516in}}%
\pgfpathlineto{\pgfqpoint{3.974124in}{2.331465in}}%
\pgfpathlineto{\pgfqpoint{3.978665in}{2.331465in}}%
\pgfpathlineto{\pgfqpoint{3.978665in}{2.328516in}}%
\pgfpathmoveto{\pgfqpoint{3.974124in}{2.331465in}}%
\pgfpathlineto{\pgfqpoint{3.974124in}{2.331465in}}%
\pgfpathlineto{\pgfqpoint{3.974124in}{2.334414in}}%
\pgfpathlineto{\pgfqpoint{3.978665in}{2.334414in}}%
\pgfpathlineto{\pgfqpoint{3.978665in}{2.331465in}}%
\pgfpathmoveto{\pgfqpoint{3.978665in}{2.328516in}}%
\pgfpathlineto{\pgfqpoint{3.978665in}{2.328516in}}%
\pgfpathlineto{\pgfqpoint{3.978665in}{2.331465in}}%
\pgfpathlineto{\pgfqpoint{3.983206in}{2.331465in}}%
\pgfpathlineto{\pgfqpoint{3.983206in}{2.328516in}}%
\pgfpathmoveto{\pgfqpoint{3.996829in}{2.313770in}}%
\pgfpathlineto{\pgfqpoint{3.996829in}{2.313770in}}%
\pgfpathlineto{\pgfqpoint{3.996829in}{2.316719in}}%
\pgfpathlineto{\pgfqpoint{4.001370in}{2.316719in}}%
\pgfpathlineto{\pgfqpoint{4.001370in}{2.313770in}}%
\pgfpathmoveto{\pgfqpoint{4.014993in}{2.301973in}}%
\pgfpathlineto{\pgfqpoint{4.014993in}{2.301973in}}%
\pgfpathlineto{\pgfqpoint{4.014993in}{2.304922in}}%
\pgfpathlineto{\pgfqpoint{4.019533in}{2.304922in}}%
\pgfpathlineto{\pgfqpoint{4.019533in}{2.301973in}}%
\pgfpathmoveto{\pgfqpoint{4.005911in}{2.307871in}}%
\pgfpathlineto{\pgfqpoint{4.005911in}{2.307871in}}%
\pgfpathlineto{\pgfqpoint{4.005911in}{2.310821in}}%
\pgfpathlineto{\pgfqpoint{4.010452in}{2.310821in}}%
\pgfpathlineto{\pgfqpoint{4.010452in}{2.307871in}}%
\pgfpathmoveto{\pgfqpoint{4.001370in}{2.310821in}}%
\pgfpathlineto{\pgfqpoint{4.001370in}{2.310821in}}%
\pgfpathlineto{\pgfqpoint{4.001370in}{2.313770in}}%
\pgfpathlineto{\pgfqpoint{4.005911in}{2.313770in}}%
\pgfpathlineto{\pgfqpoint{4.005911in}{2.310821in}}%
\pgfpathmoveto{\pgfqpoint{4.001370in}{2.313770in}}%
\pgfpathlineto{\pgfqpoint{4.001370in}{2.313770in}}%
\pgfpathlineto{\pgfqpoint{4.001370in}{2.316719in}}%
\pgfpathlineto{\pgfqpoint{4.005911in}{2.316719in}}%
\pgfpathlineto{\pgfqpoint{4.005911in}{2.313770in}}%
\pgfpathmoveto{\pgfqpoint{4.005911in}{2.310821in}}%
\pgfpathlineto{\pgfqpoint{4.005911in}{2.310821in}}%
\pgfpathlineto{\pgfqpoint{4.005911in}{2.313770in}}%
\pgfpathlineto{\pgfqpoint{4.010452in}{2.313770in}}%
\pgfpathlineto{\pgfqpoint{4.010452in}{2.310821in}}%
\pgfpathmoveto{\pgfqpoint{4.010452in}{2.304922in}}%
\pgfpathlineto{\pgfqpoint{4.010452in}{2.304922in}}%
\pgfpathlineto{\pgfqpoint{4.010452in}{2.307871in}}%
\pgfpathlineto{\pgfqpoint{4.014993in}{2.307871in}}%
\pgfpathlineto{\pgfqpoint{4.014993in}{2.304922in}}%
\pgfpathmoveto{\pgfqpoint{4.010452in}{2.307871in}}%
\pgfpathlineto{\pgfqpoint{4.010452in}{2.307871in}}%
\pgfpathlineto{\pgfqpoint{4.010452in}{2.310821in}}%
\pgfpathlineto{\pgfqpoint{4.014993in}{2.310821in}}%
\pgfpathlineto{\pgfqpoint{4.014993in}{2.307871in}}%
\pgfpathmoveto{\pgfqpoint{4.014993in}{2.304922in}}%
\pgfpathlineto{\pgfqpoint{4.014993in}{2.304922in}}%
\pgfpathlineto{\pgfqpoint{4.014993in}{2.307871in}}%
\pgfpathlineto{\pgfqpoint{4.019533in}{2.307871in}}%
\pgfpathlineto{\pgfqpoint{4.019533in}{2.304922in}}%
\pgfpathmoveto{\pgfqpoint{3.987747in}{2.319668in}}%
\pgfpathlineto{\pgfqpoint{3.987747in}{2.319668in}}%
\pgfpathlineto{\pgfqpoint{3.987747in}{2.322617in}}%
\pgfpathlineto{\pgfqpoint{3.992288in}{2.322617in}}%
\pgfpathlineto{\pgfqpoint{3.992288in}{2.319668in}}%
\pgfpathmoveto{\pgfqpoint{3.983206in}{2.322617in}}%
\pgfpathlineto{\pgfqpoint{3.983206in}{2.322617in}}%
\pgfpathlineto{\pgfqpoint{3.983206in}{2.325566in}}%
\pgfpathlineto{\pgfqpoint{3.987747in}{2.325566in}}%
\pgfpathlineto{\pgfqpoint{3.987747in}{2.322617in}}%
\pgfpathmoveto{\pgfqpoint{3.983206in}{2.325566in}}%
\pgfpathlineto{\pgfqpoint{3.983206in}{2.325566in}}%
\pgfpathlineto{\pgfqpoint{3.983206in}{2.328516in}}%
\pgfpathlineto{\pgfqpoint{3.987747in}{2.328516in}}%
\pgfpathlineto{\pgfqpoint{3.987747in}{2.325566in}}%
\pgfpathmoveto{\pgfqpoint{3.987747in}{2.322617in}}%
\pgfpathlineto{\pgfqpoint{3.987747in}{2.322617in}}%
\pgfpathlineto{\pgfqpoint{3.987747in}{2.325566in}}%
\pgfpathlineto{\pgfqpoint{3.992288in}{2.325566in}}%
\pgfpathlineto{\pgfqpoint{3.992288in}{2.322617in}}%
\pgfpathmoveto{\pgfqpoint{3.992288in}{2.316719in}}%
\pgfpathlineto{\pgfqpoint{3.992288in}{2.316719in}}%
\pgfpathlineto{\pgfqpoint{3.992288in}{2.319668in}}%
\pgfpathlineto{\pgfqpoint{3.996829in}{2.319668in}}%
\pgfpathlineto{\pgfqpoint{3.996829in}{2.316719in}}%
\pgfpathmoveto{\pgfqpoint{3.992288in}{2.319668in}}%
\pgfpathlineto{\pgfqpoint{3.992288in}{2.319668in}}%
\pgfpathlineto{\pgfqpoint{3.992288in}{2.322617in}}%
\pgfpathlineto{\pgfqpoint{3.996829in}{2.322617in}}%
\pgfpathlineto{\pgfqpoint{3.996829in}{2.319668in}}%
\pgfpathmoveto{\pgfqpoint{3.996829in}{2.316719in}}%
\pgfpathlineto{\pgfqpoint{3.996829in}{2.316719in}}%
\pgfpathlineto{\pgfqpoint{3.996829in}{2.319668in}}%
\pgfpathlineto{\pgfqpoint{4.001370in}{2.319668in}}%
\pgfpathlineto{\pgfqpoint{4.001370in}{2.316719in}}%
\pgfpathmoveto{\pgfqpoint{3.951420in}{2.343262in}}%
\pgfpathlineto{\pgfqpoint{3.951420in}{2.343262in}}%
\pgfpathlineto{\pgfqpoint{3.951420in}{2.346211in}}%
\pgfpathlineto{\pgfqpoint{3.955961in}{2.346211in}}%
\pgfpathlineto{\pgfqpoint{3.955961in}{2.343262in}}%
\pgfpathmoveto{\pgfqpoint{3.946879in}{2.346211in}}%
\pgfpathlineto{\pgfqpoint{3.946879in}{2.346211in}}%
\pgfpathlineto{\pgfqpoint{3.946879in}{2.349160in}}%
\pgfpathlineto{\pgfqpoint{3.951420in}{2.349160in}}%
\pgfpathlineto{\pgfqpoint{3.951420in}{2.346211in}}%
\pgfpathmoveto{\pgfqpoint{3.946879in}{2.349160in}}%
\pgfpathlineto{\pgfqpoint{3.946879in}{2.349160in}}%
\pgfpathlineto{\pgfqpoint{3.946879in}{2.352109in}}%
\pgfpathlineto{\pgfqpoint{3.951420in}{2.352109in}}%
\pgfpathlineto{\pgfqpoint{3.951420in}{2.349160in}}%
\pgfpathmoveto{\pgfqpoint{3.951420in}{2.346211in}}%
\pgfpathlineto{\pgfqpoint{3.951420in}{2.346211in}}%
\pgfpathlineto{\pgfqpoint{3.951420in}{2.349160in}}%
\pgfpathlineto{\pgfqpoint{3.955961in}{2.349160in}}%
\pgfpathlineto{\pgfqpoint{3.955961in}{2.346211in}}%
\pgfpathmoveto{\pgfqpoint{3.955961in}{2.340312in}}%
\pgfpathlineto{\pgfqpoint{3.955961in}{2.340312in}}%
\pgfpathlineto{\pgfqpoint{3.955961in}{2.343262in}}%
\pgfpathlineto{\pgfqpoint{3.960502in}{2.343262in}}%
\pgfpathlineto{\pgfqpoint{3.960502in}{2.340312in}}%
\pgfpathmoveto{\pgfqpoint{3.955961in}{2.343262in}}%
\pgfpathlineto{\pgfqpoint{3.955961in}{2.343262in}}%
\pgfpathlineto{\pgfqpoint{3.955961in}{2.346211in}}%
\pgfpathlineto{\pgfqpoint{3.960502in}{2.346211in}}%
\pgfpathlineto{\pgfqpoint{3.960502in}{2.343262in}}%
\pgfpathmoveto{\pgfqpoint{3.960502in}{2.340312in}}%
\pgfpathlineto{\pgfqpoint{3.960502in}{2.340312in}}%
\pgfpathlineto{\pgfqpoint{3.960502in}{2.343262in}}%
\pgfpathlineto{\pgfqpoint{3.965043in}{2.343262in}}%
\pgfpathlineto{\pgfqpoint{3.965043in}{2.340312in}}%
\pgfpathmoveto{\pgfqpoint{4.024074in}{2.296075in}}%
\pgfpathlineto{\pgfqpoint{4.024074in}{2.296075in}}%
\pgfpathlineto{\pgfqpoint{4.024074in}{2.299024in}}%
\pgfpathlineto{\pgfqpoint{4.028615in}{2.299024in}}%
\pgfpathlineto{\pgfqpoint{4.028615in}{2.296075in}}%
\pgfpathmoveto{\pgfqpoint{4.019533in}{2.299024in}}%
\pgfpathlineto{\pgfqpoint{4.019533in}{2.299024in}}%
\pgfpathlineto{\pgfqpoint{4.019533in}{2.301973in}}%
\pgfpathlineto{\pgfqpoint{4.024074in}{2.301973in}}%
\pgfpathlineto{\pgfqpoint{4.024074in}{2.299024in}}%
\pgfpathmoveto{\pgfqpoint{4.019533in}{2.301973in}}%
\pgfpathlineto{\pgfqpoint{4.019533in}{2.301973in}}%
\pgfpathlineto{\pgfqpoint{4.019533in}{2.304922in}}%
\pgfpathlineto{\pgfqpoint{4.024074in}{2.304922in}}%
\pgfpathlineto{\pgfqpoint{4.024074in}{2.301973in}}%
\pgfpathmoveto{\pgfqpoint{4.024074in}{2.299024in}}%
\pgfpathlineto{\pgfqpoint{4.024074in}{2.299024in}}%
\pgfpathlineto{\pgfqpoint{4.024074in}{2.301973in}}%
\pgfpathlineto{\pgfqpoint{4.028615in}{2.301973in}}%
\pgfpathlineto{\pgfqpoint{4.028615in}{2.299024in}}%
\pgfpathmoveto{\pgfqpoint{4.028615in}{2.293125in}}%
\pgfpathlineto{\pgfqpoint{4.028615in}{2.293125in}}%
\pgfpathlineto{\pgfqpoint{4.028615in}{2.296075in}}%
\pgfpathlineto{\pgfqpoint{4.033156in}{2.296075in}}%
\pgfpathlineto{\pgfqpoint{4.033156in}{2.293125in}}%
\pgfpathmoveto{\pgfqpoint{4.028615in}{2.296075in}}%
\pgfpathlineto{\pgfqpoint{4.028615in}{2.296075in}}%
\pgfpathlineto{\pgfqpoint{4.028615in}{2.299024in}}%
\pgfpathlineto{\pgfqpoint{4.033156in}{2.299024in}}%
\pgfpathlineto{\pgfqpoint{4.033156in}{2.296075in}}%
\pgfpathmoveto{\pgfqpoint{4.033156in}{2.293125in}}%
\pgfpathlineto{\pgfqpoint{4.033156in}{2.293125in}}%
\pgfpathlineto{\pgfqpoint{4.033156in}{2.296075in}}%
\pgfpathlineto{\pgfqpoint{4.037697in}{2.296075in}}%
\pgfpathlineto{\pgfqpoint{4.037697in}{2.293125in}}%
\pgfpathmoveto{\pgfqpoint{4.178465in}{2.195800in}}%
\pgfpathlineto{\pgfqpoint{4.178465in}{2.195800in}}%
\pgfpathlineto{\pgfqpoint{4.178465in}{2.198749in}}%
\pgfpathlineto{\pgfqpoint{4.183006in}{2.198749in}}%
\pgfpathlineto{\pgfqpoint{4.183006in}{2.195800in}}%
\pgfpathmoveto{\pgfqpoint{4.196628in}{2.184003in}}%
\pgfpathlineto{\pgfqpoint{4.196628in}{2.184003in}}%
\pgfpathlineto{\pgfqpoint{4.196628in}{2.186952in}}%
\pgfpathlineto{\pgfqpoint{4.201169in}{2.186952in}}%
\pgfpathlineto{\pgfqpoint{4.201169in}{2.184003in}}%
\pgfpathmoveto{\pgfqpoint{4.187546in}{2.189901in}}%
\pgfpathlineto{\pgfqpoint{4.187546in}{2.189901in}}%
\pgfpathlineto{\pgfqpoint{4.187546in}{2.192850in}}%
\pgfpathlineto{\pgfqpoint{4.192087in}{2.192850in}}%
\pgfpathlineto{\pgfqpoint{4.192087in}{2.189901in}}%
\pgfpathmoveto{\pgfqpoint{4.183006in}{2.192850in}}%
\pgfpathlineto{\pgfqpoint{4.183006in}{2.192850in}}%
\pgfpathlineto{\pgfqpoint{4.183006in}{2.195800in}}%
\pgfpathlineto{\pgfqpoint{4.187546in}{2.195800in}}%
\pgfpathlineto{\pgfqpoint{4.187546in}{2.192850in}}%
\pgfpathmoveto{\pgfqpoint{4.183006in}{2.195800in}}%
\pgfpathlineto{\pgfqpoint{4.183006in}{2.195800in}}%
\pgfpathlineto{\pgfqpoint{4.183006in}{2.198749in}}%
\pgfpathlineto{\pgfqpoint{4.187546in}{2.198749in}}%
\pgfpathlineto{\pgfqpoint{4.187546in}{2.195800in}}%
\pgfpathmoveto{\pgfqpoint{4.187546in}{2.192850in}}%
\pgfpathlineto{\pgfqpoint{4.187546in}{2.192850in}}%
\pgfpathlineto{\pgfqpoint{4.187546in}{2.195800in}}%
\pgfpathlineto{\pgfqpoint{4.192087in}{2.195800in}}%
\pgfpathlineto{\pgfqpoint{4.192087in}{2.192850in}}%
\pgfpathmoveto{\pgfqpoint{4.192087in}{2.186952in}}%
\pgfpathlineto{\pgfqpoint{4.192087in}{2.186952in}}%
\pgfpathlineto{\pgfqpoint{4.192087in}{2.189901in}}%
\pgfpathlineto{\pgfqpoint{4.196628in}{2.189901in}}%
\pgfpathlineto{\pgfqpoint{4.196628in}{2.186952in}}%
\pgfpathmoveto{\pgfqpoint{4.192087in}{2.189901in}}%
\pgfpathlineto{\pgfqpoint{4.192087in}{2.189901in}}%
\pgfpathlineto{\pgfqpoint{4.192087in}{2.192850in}}%
\pgfpathlineto{\pgfqpoint{4.196628in}{2.192850in}}%
\pgfpathlineto{\pgfqpoint{4.196628in}{2.189901in}}%
\pgfpathmoveto{\pgfqpoint{4.196628in}{2.186952in}}%
\pgfpathlineto{\pgfqpoint{4.196628in}{2.186952in}}%
\pgfpathlineto{\pgfqpoint{4.196628in}{2.189901in}}%
\pgfpathlineto{\pgfqpoint{4.201169in}{2.189901in}}%
\pgfpathlineto{\pgfqpoint{4.201169in}{2.186952in}}%
\pgfpathmoveto{\pgfqpoint{4.214792in}{2.172206in}}%
\pgfpathlineto{\pgfqpoint{4.214792in}{2.172206in}}%
\pgfpathlineto{\pgfqpoint{4.214792in}{2.175155in}}%
\pgfpathlineto{\pgfqpoint{4.219333in}{2.175155in}}%
\pgfpathlineto{\pgfqpoint{4.219333in}{2.172206in}}%
\pgfpathmoveto{\pgfqpoint{4.232955in}{2.160409in}}%
\pgfpathlineto{\pgfqpoint{4.232955in}{2.160409in}}%
\pgfpathlineto{\pgfqpoint{4.232955in}{2.163358in}}%
\pgfpathlineto{\pgfqpoint{4.237496in}{2.163358in}}%
\pgfpathlineto{\pgfqpoint{4.237496in}{2.160409in}}%
\pgfpathmoveto{\pgfqpoint{4.223874in}{2.166307in}}%
\pgfpathlineto{\pgfqpoint{4.223874in}{2.166307in}}%
\pgfpathlineto{\pgfqpoint{4.223874in}{2.169257in}}%
\pgfpathlineto{\pgfqpoint{4.228414in}{2.169257in}}%
\pgfpathlineto{\pgfqpoint{4.228414in}{2.166307in}}%
\pgfpathmoveto{\pgfqpoint{4.219333in}{2.169257in}}%
\pgfpathlineto{\pgfqpoint{4.219333in}{2.169257in}}%
\pgfpathlineto{\pgfqpoint{4.219333in}{2.172206in}}%
\pgfpathlineto{\pgfqpoint{4.223874in}{2.172206in}}%
\pgfpathlineto{\pgfqpoint{4.223874in}{2.169257in}}%
\pgfpathmoveto{\pgfqpoint{4.219333in}{2.172206in}}%
\pgfpathlineto{\pgfqpoint{4.219333in}{2.172206in}}%
\pgfpathlineto{\pgfqpoint{4.219333in}{2.175155in}}%
\pgfpathlineto{\pgfqpoint{4.223874in}{2.175155in}}%
\pgfpathlineto{\pgfqpoint{4.223874in}{2.172206in}}%
\pgfpathmoveto{\pgfqpoint{4.223874in}{2.169257in}}%
\pgfpathlineto{\pgfqpoint{4.223874in}{2.169257in}}%
\pgfpathlineto{\pgfqpoint{4.223874in}{2.172206in}}%
\pgfpathlineto{\pgfqpoint{4.228414in}{2.172206in}}%
\pgfpathlineto{\pgfqpoint{4.228414in}{2.169257in}}%
\pgfpathmoveto{\pgfqpoint{4.228414in}{2.163358in}}%
\pgfpathlineto{\pgfqpoint{4.228414in}{2.163358in}}%
\pgfpathlineto{\pgfqpoint{4.228414in}{2.166307in}}%
\pgfpathlineto{\pgfqpoint{4.232955in}{2.166307in}}%
\pgfpathlineto{\pgfqpoint{4.232955in}{2.163358in}}%
\pgfpathmoveto{\pgfqpoint{4.228414in}{2.166307in}}%
\pgfpathlineto{\pgfqpoint{4.228414in}{2.166307in}}%
\pgfpathlineto{\pgfqpoint{4.228414in}{2.169257in}}%
\pgfpathlineto{\pgfqpoint{4.232955in}{2.169257in}}%
\pgfpathlineto{\pgfqpoint{4.232955in}{2.166307in}}%
\pgfpathmoveto{\pgfqpoint{4.232955in}{2.163358in}}%
\pgfpathlineto{\pgfqpoint{4.232955in}{2.163358in}}%
\pgfpathlineto{\pgfqpoint{4.232955in}{2.166307in}}%
\pgfpathlineto{\pgfqpoint{4.237496in}{2.166307in}}%
\pgfpathlineto{\pgfqpoint{4.237496in}{2.163358in}}%
\pgfpathmoveto{\pgfqpoint{4.205710in}{2.178104in}}%
\pgfpathlineto{\pgfqpoint{4.205710in}{2.178104in}}%
\pgfpathlineto{\pgfqpoint{4.205710in}{2.181053in}}%
\pgfpathlineto{\pgfqpoint{4.210251in}{2.181053in}}%
\pgfpathlineto{\pgfqpoint{4.210251in}{2.178104in}}%
\pgfpathmoveto{\pgfqpoint{4.201169in}{2.181053in}}%
\pgfpathlineto{\pgfqpoint{4.201169in}{2.181053in}}%
\pgfpathlineto{\pgfqpoint{4.201169in}{2.184003in}}%
\pgfpathlineto{\pgfqpoint{4.205710in}{2.184003in}}%
\pgfpathlineto{\pgfqpoint{4.205710in}{2.181053in}}%
\pgfpathmoveto{\pgfqpoint{4.201169in}{2.184003in}}%
\pgfpathlineto{\pgfqpoint{4.201169in}{2.184003in}}%
\pgfpathlineto{\pgfqpoint{4.201169in}{2.186952in}}%
\pgfpathlineto{\pgfqpoint{4.205710in}{2.186952in}}%
\pgfpathlineto{\pgfqpoint{4.205710in}{2.184003in}}%
\pgfpathmoveto{\pgfqpoint{4.205710in}{2.181053in}}%
\pgfpathlineto{\pgfqpoint{4.205710in}{2.181053in}}%
\pgfpathlineto{\pgfqpoint{4.205710in}{2.184003in}}%
\pgfpathlineto{\pgfqpoint{4.210251in}{2.184003in}}%
\pgfpathlineto{\pgfqpoint{4.210251in}{2.181053in}}%
\pgfpathmoveto{\pgfqpoint{4.210251in}{2.175155in}}%
\pgfpathlineto{\pgfqpoint{4.210251in}{2.175155in}}%
\pgfpathlineto{\pgfqpoint{4.210251in}{2.178104in}}%
\pgfpathlineto{\pgfqpoint{4.214792in}{2.178104in}}%
\pgfpathlineto{\pgfqpoint{4.214792in}{2.175155in}}%
\pgfpathmoveto{\pgfqpoint{4.210251in}{2.178104in}}%
\pgfpathlineto{\pgfqpoint{4.210251in}{2.178104in}}%
\pgfpathlineto{\pgfqpoint{4.210251in}{2.181053in}}%
\pgfpathlineto{\pgfqpoint{4.214792in}{2.181053in}}%
\pgfpathlineto{\pgfqpoint{4.214792in}{2.178104in}}%
\pgfpathmoveto{\pgfqpoint{4.214792in}{2.175155in}}%
\pgfpathlineto{\pgfqpoint{4.214792in}{2.175155in}}%
\pgfpathlineto{\pgfqpoint{4.214792in}{2.178104in}}%
\pgfpathlineto{\pgfqpoint{4.219333in}{2.178104in}}%
\pgfpathlineto{\pgfqpoint{4.219333in}{2.175155in}}%
\pgfpathmoveto{\pgfqpoint{4.105811in}{2.242988in}}%
\pgfpathlineto{\pgfqpoint{4.105811in}{2.242988in}}%
\pgfpathlineto{\pgfqpoint{4.105811in}{2.245937in}}%
\pgfpathlineto{\pgfqpoint{4.110351in}{2.245937in}}%
\pgfpathlineto{\pgfqpoint{4.110351in}{2.242988in}}%
\pgfpathmoveto{\pgfqpoint{4.123974in}{2.231191in}}%
\pgfpathlineto{\pgfqpoint{4.123974in}{2.231191in}}%
\pgfpathlineto{\pgfqpoint{4.123974in}{2.234140in}}%
\pgfpathlineto{\pgfqpoint{4.128515in}{2.234140in}}%
\pgfpathlineto{\pgfqpoint{4.128515in}{2.231191in}}%
\pgfpathmoveto{\pgfqpoint{4.114892in}{2.237089in}}%
\pgfpathlineto{\pgfqpoint{4.114892in}{2.237089in}}%
\pgfpathlineto{\pgfqpoint{4.114892in}{2.240039in}}%
\pgfpathlineto{\pgfqpoint{4.119433in}{2.240039in}}%
\pgfpathlineto{\pgfqpoint{4.119433in}{2.237089in}}%
\pgfpathmoveto{\pgfqpoint{4.110351in}{2.240039in}}%
\pgfpathlineto{\pgfqpoint{4.110351in}{2.240039in}}%
\pgfpathlineto{\pgfqpoint{4.110351in}{2.242988in}}%
\pgfpathlineto{\pgfqpoint{4.114892in}{2.242988in}}%
\pgfpathlineto{\pgfqpoint{4.114892in}{2.240039in}}%
\pgfpathmoveto{\pgfqpoint{4.110351in}{2.242988in}}%
\pgfpathlineto{\pgfqpoint{4.110351in}{2.242988in}}%
\pgfpathlineto{\pgfqpoint{4.110351in}{2.245937in}}%
\pgfpathlineto{\pgfqpoint{4.114892in}{2.245937in}}%
\pgfpathlineto{\pgfqpoint{4.114892in}{2.242988in}}%
\pgfpathmoveto{\pgfqpoint{4.114892in}{2.240039in}}%
\pgfpathlineto{\pgfqpoint{4.114892in}{2.240039in}}%
\pgfpathlineto{\pgfqpoint{4.114892in}{2.242988in}}%
\pgfpathlineto{\pgfqpoint{4.119433in}{2.242988in}}%
\pgfpathlineto{\pgfqpoint{4.119433in}{2.240039in}}%
\pgfpathmoveto{\pgfqpoint{4.119433in}{2.234140in}}%
\pgfpathlineto{\pgfqpoint{4.119433in}{2.234140in}}%
\pgfpathlineto{\pgfqpoint{4.119433in}{2.237089in}}%
\pgfpathlineto{\pgfqpoint{4.123974in}{2.237089in}}%
\pgfpathlineto{\pgfqpoint{4.123974in}{2.234140in}}%
\pgfpathmoveto{\pgfqpoint{4.119433in}{2.237089in}}%
\pgfpathlineto{\pgfqpoint{4.119433in}{2.237089in}}%
\pgfpathlineto{\pgfqpoint{4.119433in}{2.240039in}}%
\pgfpathlineto{\pgfqpoint{4.123974in}{2.240039in}}%
\pgfpathlineto{\pgfqpoint{4.123974in}{2.237089in}}%
\pgfpathmoveto{\pgfqpoint{4.123974in}{2.234140in}}%
\pgfpathlineto{\pgfqpoint{4.123974in}{2.234140in}}%
\pgfpathlineto{\pgfqpoint{4.123974in}{2.237089in}}%
\pgfpathlineto{\pgfqpoint{4.128515in}{2.237089in}}%
\pgfpathlineto{\pgfqpoint{4.128515in}{2.234140in}}%
\pgfpathmoveto{\pgfqpoint{4.142138in}{2.219394in}}%
\pgfpathlineto{\pgfqpoint{4.142138in}{2.219394in}}%
\pgfpathlineto{\pgfqpoint{4.142138in}{2.222343in}}%
\pgfpathlineto{\pgfqpoint{4.146678in}{2.222343in}}%
\pgfpathlineto{\pgfqpoint{4.146678in}{2.219394in}}%
\pgfpathmoveto{\pgfqpoint{4.160301in}{2.207597in}}%
\pgfpathlineto{\pgfqpoint{4.160301in}{2.207597in}}%
\pgfpathlineto{\pgfqpoint{4.160301in}{2.210546in}}%
\pgfpathlineto{\pgfqpoint{4.164842in}{2.210546in}}%
\pgfpathlineto{\pgfqpoint{4.164842in}{2.207597in}}%
\pgfpathmoveto{\pgfqpoint{4.151219in}{2.213495in}}%
\pgfpathlineto{\pgfqpoint{4.151219in}{2.213495in}}%
\pgfpathlineto{\pgfqpoint{4.151219in}{2.216444in}}%
\pgfpathlineto{\pgfqpoint{4.155760in}{2.216444in}}%
\pgfpathlineto{\pgfqpoint{4.155760in}{2.213495in}}%
\pgfpathmoveto{\pgfqpoint{4.146678in}{2.216444in}}%
\pgfpathlineto{\pgfqpoint{4.146678in}{2.216444in}}%
\pgfpathlineto{\pgfqpoint{4.146678in}{2.219394in}}%
\pgfpathlineto{\pgfqpoint{4.151219in}{2.219394in}}%
\pgfpathlineto{\pgfqpoint{4.151219in}{2.216444in}}%
\pgfpathmoveto{\pgfqpoint{4.146678in}{2.219394in}}%
\pgfpathlineto{\pgfqpoint{4.146678in}{2.219394in}}%
\pgfpathlineto{\pgfqpoint{4.146678in}{2.222343in}}%
\pgfpathlineto{\pgfqpoint{4.151219in}{2.222343in}}%
\pgfpathlineto{\pgfqpoint{4.151219in}{2.219394in}}%
\pgfpathmoveto{\pgfqpoint{4.151219in}{2.216444in}}%
\pgfpathlineto{\pgfqpoint{4.151219in}{2.216444in}}%
\pgfpathlineto{\pgfqpoint{4.151219in}{2.219394in}}%
\pgfpathlineto{\pgfqpoint{4.155760in}{2.219394in}}%
\pgfpathlineto{\pgfqpoint{4.155760in}{2.216444in}}%
\pgfpathmoveto{\pgfqpoint{4.155760in}{2.210546in}}%
\pgfpathlineto{\pgfqpoint{4.155760in}{2.210546in}}%
\pgfpathlineto{\pgfqpoint{4.155760in}{2.213495in}}%
\pgfpathlineto{\pgfqpoint{4.160301in}{2.213495in}}%
\pgfpathlineto{\pgfqpoint{4.160301in}{2.210546in}}%
\pgfpathmoveto{\pgfqpoint{4.155760in}{2.213495in}}%
\pgfpathlineto{\pgfqpoint{4.155760in}{2.213495in}}%
\pgfpathlineto{\pgfqpoint{4.155760in}{2.216444in}}%
\pgfpathlineto{\pgfqpoint{4.160301in}{2.216444in}}%
\pgfpathlineto{\pgfqpoint{4.160301in}{2.213495in}}%
\pgfpathmoveto{\pgfqpoint{4.160301in}{2.210546in}}%
\pgfpathlineto{\pgfqpoint{4.160301in}{2.210546in}}%
\pgfpathlineto{\pgfqpoint{4.160301in}{2.213495in}}%
\pgfpathlineto{\pgfqpoint{4.164842in}{2.213495in}}%
\pgfpathlineto{\pgfqpoint{4.164842in}{2.210546in}}%
\pgfpathmoveto{\pgfqpoint{4.133056in}{2.225292in}}%
\pgfpathlineto{\pgfqpoint{4.133056in}{2.225292in}}%
\pgfpathlineto{\pgfqpoint{4.133056in}{2.228241in}}%
\pgfpathlineto{\pgfqpoint{4.137597in}{2.228241in}}%
\pgfpathlineto{\pgfqpoint{4.137597in}{2.225292in}}%
\pgfpathmoveto{\pgfqpoint{4.128515in}{2.228241in}}%
\pgfpathlineto{\pgfqpoint{4.128515in}{2.228241in}}%
\pgfpathlineto{\pgfqpoint{4.128515in}{2.231191in}}%
\pgfpathlineto{\pgfqpoint{4.133056in}{2.231191in}}%
\pgfpathlineto{\pgfqpoint{4.133056in}{2.228241in}}%
\pgfpathmoveto{\pgfqpoint{4.128515in}{2.231191in}}%
\pgfpathlineto{\pgfqpoint{4.128515in}{2.231191in}}%
\pgfpathlineto{\pgfqpoint{4.128515in}{2.234140in}}%
\pgfpathlineto{\pgfqpoint{4.133056in}{2.234140in}}%
\pgfpathlineto{\pgfqpoint{4.133056in}{2.231191in}}%
\pgfpathmoveto{\pgfqpoint{4.133056in}{2.228241in}}%
\pgfpathlineto{\pgfqpoint{4.133056in}{2.228241in}}%
\pgfpathlineto{\pgfqpoint{4.133056in}{2.231191in}}%
\pgfpathlineto{\pgfqpoint{4.137597in}{2.231191in}}%
\pgfpathlineto{\pgfqpoint{4.137597in}{2.228241in}}%
\pgfpathmoveto{\pgfqpoint{4.137597in}{2.222343in}}%
\pgfpathlineto{\pgfqpoint{4.137597in}{2.222343in}}%
\pgfpathlineto{\pgfqpoint{4.137597in}{2.225292in}}%
\pgfpathlineto{\pgfqpoint{4.142138in}{2.225292in}}%
\pgfpathlineto{\pgfqpoint{4.142138in}{2.222343in}}%
\pgfpathmoveto{\pgfqpoint{4.137597in}{2.225292in}}%
\pgfpathlineto{\pgfqpoint{4.137597in}{2.225292in}}%
\pgfpathlineto{\pgfqpoint{4.137597in}{2.228241in}}%
\pgfpathlineto{\pgfqpoint{4.142138in}{2.228241in}}%
\pgfpathlineto{\pgfqpoint{4.142138in}{2.225292in}}%
\pgfpathmoveto{\pgfqpoint{4.142138in}{2.222343in}}%
\pgfpathlineto{\pgfqpoint{4.142138in}{2.222343in}}%
\pgfpathlineto{\pgfqpoint{4.142138in}{2.225292in}}%
\pgfpathlineto{\pgfqpoint{4.146678in}{2.225292in}}%
\pgfpathlineto{\pgfqpoint{4.146678in}{2.222343in}}%
\pgfpathmoveto{\pgfqpoint{4.096729in}{2.248886in}}%
\pgfpathlineto{\pgfqpoint{4.096729in}{2.248886in}}%
\pgfpathlineto{\pgfqpoint{4.096729in}{2.251836in}}%
\pgfpathlineto{\pgfqpoint{4.101270in}{2.251836in}}%
\pgfpathlineto{\pgfqpoint{4.101270in}{2.248886in}}%
\pgfpathmoveto{\pgfqpoint{4.092188in}{2.251836in}}%
\pgfpathlineto{\pgfqpoint{4.092188in}{2.251836in}}%
\pgfpathlineto{\pgfqpoint{4.092188in}{2.254785in}}%
\pgfpathlineto{\pgfqpoint{4.096729in}{2.254785in}}%
\pgfpathlineto{\pgfqpoint{4.096729in}{2.251836in}}%
\pgfpathmoveto{\pgfqpoint{4.092188in}{2.254785in}}%
\pgfpathlineto{\pgfqpoint{4.092188in}{2.254785in}}%
\pgfpathlineto{\pgfqpoint{4.092188in}{2.257734in}}%
\pgfpathlineto{\pgfqpoint{4.096729in}{2.257734in}}%
\pgfpathlineto{\pgfqpoint{4.096729in}{2.254785in}}%
\pgfpathmoveto{\pgfqpoint{4.096729in}{2.251836in}}%
\pgfpathlineto{\pgfqpoint{4.096729in}{2.251836in}}%
\pgfpathlineto{\pgfqpoint{4.096729in}{2.254785in}}%
\pgfpathlineto{\pgfqpoint{4.101270in}{2.254785in}}%
\pgfpathlineto{\pgfqpoint{4.101270in}{2.251836in}}%
\pgfpathmoveto{\pgfqpoint{4.101270in}{2.245937in}}%
\pgfpathlineto{\pgfqpoint{4.101270in}{2.245937in}}%
\pgfpathlineto{\pgfqpoint{4.101270in}{2.248886in}}%
\pgfpathlineto{\pgfqpoint{4.105811in}{2.248886in}}%
\pgfpathlineto{\pgfqpoint{4.105811in}{2.245937in}}%
\pgfpathmoveto{\pgfqpoint{4.101270in}{2.248886in}}%
\pgfpathlineto{\pgfqpoint{4.101270in}{2.248886in}}%
\pgfpathlineto{\pgfqpoint{4.101270in}{2.251836in}}%
\pgfpathlineto{\pgfqpoint{4.105811in}{2.251836in}}%
\pgfpathlineto{\pgfqpoint{4.105811in}{2.248886in}}%
\pgfpathmoveto{\pgfqpoint{4.105811in}{2.245937in}}%
\pgfpathlineto{\pgfqpoint{4.105811in}{2.245937in}}%
\pgfpathlineto{\pgfqpoint{4.105811in}{2.248886in}}%
\pgfpathlineto{\pgfqpoint{4.110351in}{2.248886in}}%
\pgfpathlineto{\pgfqpoint{4.110351in}{2.245937in}}%
\pgfpathmoveto{\pgfqpoint{4.169383in}{2.201698in}}%
\pgfpathlineto{\pgfqpoint{4.169383in}{2.201698in}}%
\pgfpathlineto{\pgfqpoint{4.169383in}{2.204647in}}%
\pgfpathlineto{\pgfqpoint{4.173924in}{2.204647in}}%
\pgfpathlineto{\pgfqpoint{4.173924in}{2.201698in}}%
\pgfpathmoveto{\pgfqpoint{4.164842in}{2.204647in}}%
\pgfpathlineto{\pgfqpoint{4.164842in}{2.204647in}}%
\pgfpathlineto{\pgfqpoint{4.164842in}{2.207597in}}%
\pgfpathlineto{\pgfqpoint{4.169383in}{2.207597in}}%
\pgfpathlineto{\pgfqpoint{4.169383in}{2.204647in}}%
\pgfpathmoveto{\pgfqpoint{4.164842in}{2.207597in}}%
\pgfpathlineto{\pgfqpoint{4.164842in}{2.207597in}}%
\pgfpathlineto{\pgfqpoint{4.164842in}{2.210546in}}%
\pgfpathlineto{\pgfqpoint{4.169383in}{2.210546in}}%
\pgfpathlineto{\pgfqpoint{4.169383in}{2.207597in}}%
\pgfpathmoveto{\pgfqpoint{4.169383in}{2.204647in}}%
\pgfpathlineto{\pgfqpoint{4.169383in}{2.204647in}}%
\pgfpathlineto{\pgfqpoint{4.169383in}{2.207597in}}%
\pgfpathlineto{\pgfqpoint{4.173924in}{2.207597in}}%
\pgfpathlineto{\pgfqpoint{4.173924in}{2.204647in}}%
\pgfpathmoveto{\pgfqpoint{4.173924in}{2.198749in}}%
\pgfpathlineto{\pgfqpoint{4.173924in}{2.198749in}}%
\pgfpathlineto{\pgfqpoint{4.173924in}{2.201698in}}%
\pgfpathlineto{\pgfqpoint{4.178465in}{2.201698in}}%
\pgfpathlineto{\pgfqpoint{4.178465in}{2.198749in}}%
\pgfpathmoveto{\pgfqpoint{4.173924in}{2.201698in}}%
\pgfpathlineto{\pgfqpoint{4.173924in}{2.201698in}}%
\pgfpathlineto{\pgfqpoint{4.173924in}{2.204647in}}%
\pgfpathlineto{\pgfqpoint{4.178465in}{2.204647in}}%
\pgfpathlineto{\pgfqpoint{4.178465in}{2.201698in}}%
\pgfpathmoveto{\pgfqpoint{4.178465in}{2.198749in}}%
\pgfpathlineto{\pgfqpoint{4.178465in}{2.198749in}}%
\pgfpathlineto{\pgfqpoint{4.178465in}{2.201698in}}%
\pgfpathlineto{\pgfqpoint{4.183006in}{2.201698in}}%
\pgfpathlineto{\pgfqpoint{4.183006in}{2.198749in}}%
\pgfpathmoveto{\pgfqpoint{4.323779in}{2.101424in}}%
\pgfpathlineto{\pgfqpoint{4.323779in}{2.101424in}}%
\pgfpathlineto{\pgfqpoint{4.323779in}{2.104374in}}%
\pgfpathlineto{\pgfqpoint{4.328320in}{2.104374in}}%
\pgfpathlineto{\pgfqpoint{4.328320in}{2.101424in}}%
\pgfpathmoveto{\pgfqpoint{4.341943in}{2.089628in}}%
\pgfpathlineto{\pgfqpoint{4.341943in}{2.089628in}}%
\pgfpathlineto{\pgfqpoint{4.341943in}{2.092577in}}%
\pgfpathlineto{\pgfqpoint{4.346485in}{2.092577in}}%
\pgfpathlineto{\pgfqpoint{4.346485in}{2.089628in}}%
\pgfpathmoveto{\pgfqpoint{4.332861in}{2.095526in}}%
\pgfpathlineto{\pgfqpoint{4.332861in}{2.095526in}}%
\pgfpathlineto{\pgfqpoint{4.332861in}{2.098475in}}%
\pgfpathlineto{\pgfqpoint{4.337402in}{2.098475in}}%
\pgfpathlineto{\pgfqpoint{4.337402in}{2.095526in}}%
\pgfpathmoveto{\pgfqpoint{4.328320in}{2.098475in}}%
\pgfpathlineto{\pgfqpoint{4.328320in}{2.098475in}}%
\pgfpathlineto{\pgfqpoint{4.328320in}{2.101424in}}%
\pgfpathlineto{\pgfqpoint{4.332861in}{2.101424in}}%
\pgfpathlineto{\pgfqpoint{4.332861in}{2.098475in}}%
\pgfpathmoveto{\pgfqpoint{4.328320in}{2.101424in}}%
\pgfpathlineto{\pgfqpoint{4.328320in}{2.101424in}}%
\pgfpathlineto{\pgfqpoint{4.328320in}{2.104374in}}%
\pgfpathlineto{\pgfqpoint{4.332861in}{2.104374in}}%
\pgfpathlineto{\pgfqpoint{4.332861in}{2.101424in}}%
\pgfpathmoveto{\pgfqpoint{4.332861in}{2.098475in}}%
\pgfpathlineto{\pgfqpoint{4.332861in}{2.098475in}}%
\pgfpathlineto{\pgfqpoint{4.332861in}{2.101424in}}%
\pgfpathlineto{\pgfqpoint{4.337402in}{2.101424in}}%
\pgfpathlineto{\pgfqpoint{4.337402in}{2.098475in}}%
\pgfpathmoveto{\pgfqpoint{4.337402in}{2.092577in}}%
\pgfpathlineto{\pgfqpoint{4.337402in}{2.092577in}}%
\pgfpathlineto{\pgfqpoint{4.337402in}{2.095526in}}%
\pgfpathlineto{\pgfqpoint{4.341943in}{2.095526in}}%
\pgfpathlineto{\pgfqpoint{4.341943in}{2.092577in}}%
\pgfpathmoveto{\pgfqpoint{4.337402in}{2.095526in}}%
\pgfpathlineto{\pgfqpoint{4.337402in}{2.095526in}}%
\pgfpathlineto{\pgfqpoint{4.337402in}{2.098475in}}%
\pgfpathlineto{\pgfqpoint{4.341943in}{2.098475in}}%
\pgfpathlineto{\pgfqpoint{4.341943in}{2.095526in}}%
\pgfpathmoveto{\pgfqpoint{4.341943in}{2.092577in}}%
\pgfpathlineto{\pgfqpoint{4.341943in}{2.092577in}}%
\pgfpathlineto{\pgfqpoint{4.341943in}{2.095526in}}%
\pgfpathlineto{\pgfqpoint{4.346485in}{2.095526in}}%
\pgfpathlineto{\pgfqpoint{4.346485in}{2.092577in}}%
\pgfpathmoveto{\pgfqpoint{4.360108in}{2.077832in}}%
\pgfpathlineto{\pgfqpoint{4.360108in}{2.077832in}}%
\pgfpathlineto{\pgfqpoint{4.360108in}{2.080781in}}%
\pgfpathlineto{\pgfqpoint{4.364649in}{2.080781in}}%
\pgfpathlineto{\pgfqpoint{4.364649in}{2.077832in}}%
\pgfpathmoveto{\pgfqpoint{4.378273in}{2.066035in}}%
\pgfpathlineto{\pgfqpoint{4.378273in}{2.066035in}}%
\pgfpathlineto{\pgfqpoint{4.378273in}{2.068984in}}%
\pgfpathlineto{\pgfqpoint{4.382814in}{2.068984in}}%
\pgfpathlineto{\pgfqpoint{4.382814in}{2.066035in}}%
\pgfpathmoveto{\pgfqpoint{4.369190in}{2.071933in}}%
\pgfpathlineto{\pgfqpoint{4.369190in}{2.071933in}}%
\pgfpathlineto{\pgfqpoint{4.369190in}{2.074882in}}%
\pgfpathlineto{\pgfqpoint{4.373732in}{2.074882in}}%
\pgfpathlineto{\pgfqpoint{4.373732in}{2.071933in}}%
\pgfpathmoveto{\pgfqpoint{4.364649in}{2.074882in}}%
\pgfpathlineto{\pgfqpoint{4.364649in}{2.074882in}}%
\pgfpathlineto{\pgfqpoint{4.364649in}{2.077832in}}%
\pgfpathlineto{\pgfqpoint{4.369190in}{2.077832in}}%
\pgfpathlineto{\pgfqpoint{4.369190in}{2.074882in}}%
\pgfpathmoveto{\pgfqpoint{4.364649in}{2.077832in}}%
\pgfpathlineto{\pgfqpoint{4.364649in}{2.077832in}}%
\pgfpathlineto{\pgfqpoint{4.364649in}{2.080781in}}%
\pgfpathlineto{\pgfqpoint{4.369190in}{2.080781in}}%
\pgfpathlineto{\pgfqpoint{4.369190in}{2.077832in}}%
\pgfpathmoveto{\pgfqpoint{4.369190in}{2.074882in}}%
\pgfpathlineto{\pgfqpoint{4.369190in}{2.074882in}}%
\pgfpathlineto{\pgfqpoint{4.369190in}{2.077832in}}%
\pgfpathlineto{\pgfqpoint{4.373732in}{2.077832in}}%
\pgfpathlineto{\pgfqpoint{4.373732in}{2.074882in}}%
\pgfpathmoveto{\pgfqpoint{4.373732in}{2.068984in}}%
\pgfpathlineto{\pgfqpoint{4.373732in}{2.068984in}}%
\pgfpathlineto{\pgfqpoint{4.373732in}{2.071933in}}%
\pgfpathlineto{\pgfqpoint{4.378273in}{2.071933in}}%
\pgfpathlineto{\pgfqpoint{4.378273in}{2.068984in}}%
\pgfpathmoveto{\pgfqpoint{4.373732in}{2.071933in}}%
\pgfpathlineto{\pgfqpoint{4.373732in}{2.071933in}}%
\pgfpathlineto{\pgfqpoint{4.373732in}{2.074882in}}%
\pgfpathlineto{\pgfqpoint{4.378273in}{2.074882in}}%
\pgfpathlineto{\pgfqpoint{4.378273in}{2.071933in}}%
\pgfpathmoveto{\pgfqpoint{4.378273in}{2.068984in}}%
\pgfpathlineto{\pgfqpoint{4.378273in}{2.068984in}}%
\pgfpathlineto{\pgfqpoint{4.378273in}{2.071933in}}%
\pgfpathlineto{\pgfqpoint{4.382814in}{2.071933in}}%
\pgfpathlineto{\pgfqpoint{4.382814in}{2.068984in}}%
\pgfpathmoveto{\pgfqpoint{4.351026in}{2.083730in}}%
\pgfpathlineto{\pgfqpoint{4.351026in}{2.083730in}}%
\pgfpathlineto{\pgfqpoint{4.351026in}{2.086679in}}%
\pgfpathlineto{\pgfqpoint{4.355567in}{2.086679in}}%
\pgfpathlineto{\pgfqpoint{4.355567in}{2.083730in}}%
\pgfpathmoveto{\pgfqpoint{4.346485in}{2.086679in}}%
\pgfpathlineto{\pgfqpoint{4.346485in}{2.086679in}}%
\pgfpathlineto{\pgfqpoint{4.346485in}{2.089628in}}%
\pgfpathlineto{\pgfqpoint{4.351026in}{2.089628in}}%
\pgfpathlineto{\pgfqpoint{4.351026in}{2.086679in}}%
\pgfpathmoveto{\pgfqpoint{4.346485in}{2.089628in}}%
\pgfpathlineto{\pgfqpoint{4.346485in}{2.089628in}}%
\pgfpathlineto{\pgfqpoint{4.346485in}{2.092577in}}%
\pgfpathlineto{\pgfqpoint{4.351026in}{2.092577in}}%
\pgfpathlineto{\pgfqpoint{4.351026in}{2.089628in}}%
\pgfpathmoveto{\pgfqpoint{4.351026in}{2.086679in}}%
\pgfpathlineto{\pgfqpoint{4.351026in}{2.086679in}}%
\pgfpathlineto{\pgfqpoint{4.351026in}{2.089628in}}%
\pgfpathlineto{\pgfqpoint{4.355567in}{2.089628in}}%
\pgfpathlineto{\pgfqpoint{4.355567in}{2.086679in}}%
\pgfpathmoveto{\pgfqpoint{4.355567in}{2.080781in}}%
\pgfpathlineto{\pgfqpoint{4.355567in}{2.080781in}}%
\pgfpathlineto{\pgfqpoint{4.355567in}{2.083730in}}%
\pgfpathlineto{\pgfqpoint{4.360108in}{2.083730in}}%
\pgfpathlineto{\pgfqpoint{4.360108in}{2.080781in}}%
\pgfpathmoveto{\pgfqpoint{4.355567in}{2.083730in}}%
\pgfpathlineto{\pgfqpoint{4.355567in}{2.083730in}}%
\pgfpathlineto{\pgfqpoint{4.355567in}{2.086679in}}%
\pgfpathlineto{\pgfqpoint{4.360108in}{2.086679in}}%
\pgfpathlineto{\pgfqpoint{4.360108in}{2.083730in}}%
\pgfpathmoveto{\pgfqpoint{4.360108in}{2.080781in}}%
\pgfpathlineto{\pgfqpoint{4.360108in}{2.080781in}}%
\pgfpathlineto{\pgfqpoint{4.360108in}{2.083730in}}%
\pgfpathlineto{\pgfqpoint{4.364649in}{2.083730in}}%
\pgfpathlineto{\pgfqpoint{4.364649in}{2.080781in}}%
\pgfpathmoveto{\pgfqpoint{4.251120in}{2.148612in}}%
\pgfpathlineto{\pgfqpoint{4.251120in}{2.148612in}}%
\pgfpathlineto{\pgfqpoint{4.251120in}{2.151561in}}%
\pgfpathlineto{\pgfqpoint{4.255661in}{2.151561in}}%
\pgfpathlineto{\pgfqpoint{4.255661in}{2.148612in}}%
\pgfpathmoveto{\pgfqpoint{4.269284in}{2.136815in}}%
\pgfpathlineto{\pgfqpoint{4.269284in}{2.136815in}}%
\pgfpathlineto{\pgfqpoint{4.269284in}{2.139764in}}%
\pgfpathlineto{\pgfqpoint{4.273826in}{2.139764in}}%
\pgfpathlineto{\pgfqpoint{4.273826in}{2.136815in}}%
\pgfpathmoveto{\pgfqpoint{4.260202in}{2.142713in}}%
\pgfpathlineto{\pgfqpoint{4.260202in}{2.142713in}}%
\pgfpathlineto{\pgfqpoint{4.260202in}{2.145663in}}%
\pgfpathlineto{\pgfqpoint{4.264743in}{2.145663in}}%
\pgfpathlineto{\pgfqpoint{4.264743in}{2.142713in}}%
\pgfpathmoveto{\pgfqpoint{4.255661in}{2.145663in}}%
\pgfpathlineto{\pgfqpoint{4.255661in}{2.145663in}}%
\pgfpathlineto{\pgfqpoint{4.255661in}{2.148612in}}%
\pgfpathlineto{\pgfqpoint{4.260202in}{2.148612in}}%
\pgfpathlineto{\pgfqpoint{4.260202in}{2.145663in}}%
\pgfpathmoveto{\pgfqpoint{4.255661in}{2.148612in}}%
\pgfpathlineto{\pgfqpoint{4.255661in}{2.148612in}}%
\pgfpathlineto{\pgfqpoint{4.255661in}{2.151561in}}%
\pgfpathlineto{\pgfqpoint{4.260202in}{2.151561in}}%
\pgfpathlineto{\pgfqpoint{4.260202in}{2.148612in}}%
\pgfpathmoveto{\pgfqpoint{4.260202in}{2.145663in}}%
\pgfpathlineto{\pgfqpoint{4.260202in}{2.145663in}}%
\pgfpathlineto{\pgfqpoint{4.260202in}{2.148612in}}%
\pgfpathlineto{\pgfqpoint{4.264743in}{2.148612in}}%
\pgfpathlineto{\pgfqpoint{4.264743in}{2.145663in}}%
\pgfpathmoveto{\pgfqpoint{4.264743in}{2.139764in}}%
\pgfpathlineto{\pgfqpoint{4.264743in}{2.139764in}}%
\pgfpathlineto{\pgfqpoint{4.264743in}{2.142713in}}%
\pgfpathlineto{\pgfqpoint{4.269284in}{2.142713in}}%
\pgfpathlineto{\pgfqpoint{4.269284in}{2.139764in}}%
\pgfpathmoveto{\pgfqpoint{4.264743in}{2.142713in}}%
\pgfpathlineto{\pgfqpoint{4.264743in}{2.142713in}}%
\pgfpathlineto{\pgfqpoint{4.264743in}{2.145663in}}%
\pgfpathlineto{\pgfqpoint{4.269284in}{2.145663in}}%
\pgfpathlineto{\pgfqpoint{4.269284in}{2.142713in}}%
\pgfpathmoveto{\pgfqpoint{4.269284in}{2.139764in}}%
\pgfpathlineto{\pgfqpoint{4.269284in}{2.139764in}}%
\pgfpathlineto{\pgfqpoint{4.269284in}{2.142713in}}%
\pgfpathlineto{\pgfqpoint{4.273826in}{2.142713in}}%
\pgfpathlineto{\pgfqpoint{4.273826in}{2.139764in}}%
\pgfpathmoveto{\pgfqpoint{4.287449in}{2.125018in}}%
\pgfpathlineto{\pgfqpoint{4.287449in}{2.125018in}}%
\pgfpathlineto{\pgfqpoint{4.287449in}{2.127967in}}%
\pgfpathlineto{\pgfqpoint{4.291990in}{2.127967in}}%
\pgfpathlineto{\pgfqpoint{4.291990in}{2.125018in}}%
\pgfpathmoveto{\pgfqpoint{4.305614in}{2.113221in}}%
\pgfpathlineto{\pgfqpoint{4.305614in}{2.113221in}}%
\pgfpathlineto{\pgfqpoint{4.305614in}{2.116170in}}%
\pgfpathlineto{\pgfqpoint{4.310155in}{2.116170in}}%
\pgfpathlineto{\pgfqpoint{4.310155in}{2.113221in}}%
\pgfpathmoveto{\pgfqpoint{4.296532in}{2.119120in}}%
\pgfpathlineto{\pgfqpoint{4.296532in}{2.119120in}}%
\pgfpathlineto{\pgfqpoint{4.296532in}{2.122069in}}%
\pgfpathlineto{\pgfqpoint{4.301073in}{2.122069in}}%
\pgfpathlineto{\pgfqpoint{4.301073in}{2.119120in}}%
\pgfpathmoveto{\pgfqpoint{4.291990in}{2.122069in}}%
\pgfpathlineto{\pgfqpoint{4.291990in}{2.122069in}}%
\pgfpathlineto{\pgfqpoint{4.291990in}{2.125018in}}%
\pgfpathlineto{\pgfqpoint{4.296532in}{2.125018in}}%
\pgfpathlineto{\pgfqpoint{4.296532in}{2.122069in}}%
\pgfpathmoveto{\pgfqpoint{4.291990in}{2.125018in}}%
\pgfpathlineto{\pgfqpoint{4.291990in}{2.125018in}}%
\pgfpathlineto{\pgfqpoint{4.291990in}{2.127967in}}%
\pgfpathlineto{\pgfqpoint{4.296532in}{2.127967in}}%
\pgfpathlineto{\pgfqpoint{4.296532in}{2.125018in}}%
\pgfpathmoveto{\pgfqpoint{4.296532in}{2.122069in}}%
\pgfpathlineto{\pgfqpoint{4.296532in}{2.122069in}}%
\pgfpathlineto{\pgfqpoint{4.296532in}{2.125018in}}%
\pgfpathlineto{\pgfqpoint{4.301073in}{2.125018in}}%
\pgfpathlineto{\pgfqpoint{4.301073in}{2.122069in}}%
\pgfpathmoveto{\pgfqpoint{4.301073in}{2.116170in}}%
\pgfpathlineto{\pgfqpoint{4.301073in}{2.116170in}}%
\pgfpathlineto{\pgfqpoint{4.301073in}{2.119120in}}%
\pgfpathlineto{\pgfqpoint{4.305614in}{2.119120in}}%
\pgfpathlineto{\pgfqpoint{4.305614in}{2.116170in}}%
\pgfpathmoveto{\pgfqpoint{4.301073in}{2.119120in}}%
\pgfpathlineto{\pgfqpoint{4.301073in}{2.119120in}}%
\pgfpathlineto{\pgfqpoint{4.301073in}{2.122069in}}%
\pgfpathlineto{\pgfqpoint{4.305614in}{2.122069in}}%
\pgfpathlineto{\pgfqpoint{4.305614in}{2.119120in}}%
\pgfpathmoveto{\pgfqpoint{4.305614in}{2.116170in}}%
\pgfpathlineto{\pgfqpoint{4.305614in}{2.116170in}}%
\pgfpathlineto{\pgfqpoint{4.305614in}{2.119120in}}%
\pgfpathlineto{\pgfqpoint{4.310155in}{2.119120in}}%
\pgfpathlineto{\pgfqpoint{4.310155in}{2.116170in}}%
\pgfpathmoveto{\pgfqpoint{4.278367in}{2.130917in}}%
\pgfpathlineto{\pgfqpoint{4.278367in}{2.130917in}}%
\pgfpathlineto{\pgfqpoint{4.278367in}{2.133866in}}%
\pgfpathlineto{\pgfqpoint{4.282908in}{2.133866in}}%
\pgfpathlineto{\pgfqpoint{4.282908in}{2.130917in}}%
\pgfpathmoveto{\pgfqpoint{4.273826in}{2.133866in}}%
\pgfpathlineto{\pgfqpoint{4.273826in}{2.133866in}}%
\pgfpathlineto{\pgfqpoint{4.273826in}{2.136815in}}%
\pgfpathlineto{\pgfqpoint{4.278367in}{2.136815in}}%
\pgfpathlineto{\pgfqpoint{4.278367in}{2.133866in}}%
\pgfpathmoveto{\pgfqpoint{4.273826in}{2.136815in}}%
\pgfpathlineto{\pgfqpoint{4.273826in}{2.136815in}}%
\pgfpathlineto{\pgfqpoint{4.273826in}{2.139764in}}%
\pgfpathlineto{\pgfqpoint{4.278367in}{2.139764in}}%
\pgfpathlineto{\pgfqpoint{4.278367in}{2.136815in}}%
\pgfpathmoveto{\pgfqpoint{4.278367in}{2.133866in}}%
\pgfpathlineto{\pgfqpoint{4.278367in}{2.133866in}}%
\pgfpathlineto{\pgfqpoint{4.278367in}{2.136815in}}%
\pgfpathlineto{\pgfqpoint{4.282908in}{2.136815in}}%
\pgfpathlineto{\pgfqpoint{4.282908in}{2.133866in}}%
\pgfpathmoveto{\pgfqpoint{4.282908in}{2.127967in}}%
\pgfpathlineto{\pgfqpoint{4.282908in}{2.127967in}}%
\pgfpathlineto{\pgfqpoint{4.282908in}{2.130917in}}%
\pgfpathlineto{\pgfqpoint{4.287449in}{2.130917in}}%
\pgfpathlineto{\pgfqpoint{4.287449in}{2.127967in}}%
\pgfpathmoveto{\pgfqpoint{4.282908in}{2.130917in}}%
\pgfpathlineto{\pgfqpoint{4.282908in}{2.130917in}}%
\pgfpathlineto{\pgfqpoint{4.282908in}{2.133866in}}%
\pgfpathlineto{\pgfqpoint{4.287449in}{2.133866in}}%
\pgfpathlineto{\pgfqpoint{4.287449in}{2.130917in}}%
\pgfpathmoveto{\pgfqpoint{4.287449in}{2.127967in}}%
\pgfpathlineto{\pgfqpoint{4.287449in}{2.127967in}}%
\pgfpathlineto{\pgfqpoint{4.287449in}{2.130917in}}%
\pgfpathlineto{\pgfqpoint{4.291990in}{2.130917in}}%
\pgfpathlineto{\pgfqpoint{4.291990in}{2.127967in}}%
\pgfpathmoveto{\pgfqpoint{4.242037in}{2.154510in}}%
\pgfpathlineto{\pgfqpoint{4.242037in}{2.154510in}}%
\pgfpathlineto{\pgfqpoint{4.242037in}{2.157460in}}%
\pgfpathlineto{\pgfqpoint{4.246579in}{2.157460in}}%
\pgfpathlineto{\pgfqpoint{4.246579in}{2.154510in}}%
\pgfpathmoveto{\pgfqpoint{4.237496in}{2.157460in}}%
\pgfpathlineto{\pgfqpoint{4.237496in}{2.157460in}}%
\pgfpathlineto{\pgfqpoint{4.237496in}{2.160409in}}%
\pgfpathlineto{\pgfqpoint{4.242037in}{2.160409in}}%
\pgfpathlineto{\pgfqpoint{4.242037in}{2.157460in}}%
\pgfpathmoveto{\pgfqpoint{4.237496in}{2.160409in}}%
\pgfpathlineto{\pgfqpoint{4.237496in}{2.160409in}}%
\pgfpathlineto{\pgfqpoint{4.237496in}{2.163358in}}%
\pgfpathlineto{\pgfqpoint{4.242037in}{2.163358in}}%
\pgfpathlineto{\pgfqpoint{4.242037in}{2.160409in}}%
\pgfpathmoveto{\pgfqpoint{4.242037in}{2.157460in}}%
\pgfpathlineto{\pgfqpoint{4.242037in}{2.157460in}}%
\pgfpathlineto{\pgfqpoint{4.242037in}{2.160409in}}%
\pgfpathlineto{\pgfqpoint{4.246579in}{2.160409in}}%
\pgfpathlineto{\pgfqpoint{4.246579in}{2.157460in}}%
\pgfpathmoveto{\pgfqpoint{4.246579in}{2.151561in}}%
\pgfpathlineto{\pgfqpoint{4.246579in}{2.151561in}}%
\pgfpathlineto{\pgfqpoint{4.246579in}{2.154510in}}%
\pgfpathlineto{\pgfqpoint{4.251120in}{2.154510in}}%
\pgfpathlineto{\pgfqpoint{4.251120in}{2.151561in}}%
\pgfpathmoveto{\pgfqpoint{4.246579in}{2.154510in}}%
\pgfpathlineto{\pgfqpoint{4.246579in}{2.154510in}}%
\pgfpathlineto{\pgfqpoint{4.246579in}{2.157460in}}%
\pgfpathlineto{\pgfqpoint{4.251120in}{2.157460in}}%
\pgfpathlineto{\pgfqpoint{4.251120in}{2.154510in}}%
\pgfpathmoveto{\pgfqpoint{4.251120in}{2.151561in}}%
\pgfpathlineto{\pgfqpoint{4.251120in}{2.151561in}}%
\pgfpathlineto{\pgfqpoint{4.251120in}{2.154510in}}%
\pgfpathlineto{\pgfqpoint{4.255661in}{2.154510in}}%
\pgfpathlineto{\pgfqpoint{4.255661in}{2.151561in}}%
\pgfpathmoveto{\pgfqpoint{4.314696in}{2.107323in}}%
\pgfpathlineto{\pgfqpoint{4.314696in}{2.107323in}}%
\pgfpathlineto{\pgfqpoint{4.314696in}{2.110272in}}%
\pgfpathlineto{\pgfqpoint{4.319237in}{2.110272in}}%
\pgfpathlineto{\pgfqpoint{4.319237in}{2.107323in}}%
\pgfpathmoveto{\pgfqpoint{4.310155in}{2.110272in}}%
\pgfpathlineto{\pgfqpoint{4.310155in}{2.110272in}}%
\pgfpathlineto{\pgfqpoint{4.310155in}{2.113221in}}%
\pgfpathlineto{\pgfqpoint{4.314696in}{2.113221in}}%
\pgfpathlineto{\pgfqpoint{4.314696in}{2.110272in}}%
\pgfpathmoveto{\pgfqpoint{4.310155in}{2.113221in}}%
\pgfpathlineto{\pgfqpoint{4.310155in}{2.113221in}}%
\pgfpathlineto{\pgfqpoint{4.310155in}{2.116170in}}%
\pgfpathlineto{\pgfqpoint{4.314696in}{2.116170in}}%
\pgfpathlineto{\pgfqpoint{4.314696in}{2.113221in}}%
\pgfpathmoveto{\pgfqpoint{4.314696in}{2.110272in}}%
\pgfpathlineto{\pgfqpoint{4.314696in}{2.110272in}}%
\pgfpathlineto{\pgfqpoint{4.314696in}{2.113221in}}%
\pgfpathlineto{\pgfqpoint{4.319237in}{2.113221in}}%
\pgfpathlineto{\pgfqpoint{4.319237in}{2.110272in}}%
\pgfpathmoveto{\pgfqpoint{4.319237in}{2.104374in}}%
\pgfpathlineto{\pgfqpoint{4.319237in}{2.104374in}}%
\pgfpathlineto{\pgfqpoint{4.319237in}{2.107323in}}%
\pgfpathlineto{\pgfqpoint{4.323779in}{2.107323in}}%
\pgfpathlineto{\pgfqpoint{4.323779in}{2.104374in}}%
\pgfpathmoveto{\pgfqpoint{4.319237in}{2.107323in}}%
\pgfpathlineto{\pgfqpoint{4.319237in}{2.107323in}}%
\pgfpathlineto{\pgfqpoint{4.319237in}{2.110272in}}%
\pgfpathlineto{\pgfqpoint{4.323779in}{2.110272in}}%
\pgfpathlineto{\pgfqpoint{4.323779in}{2.107323in}}%
\pgfpathmoveto{\pgfqpoint{4.323779in}{2.104374in}}%
\pgfpathlineto{\pgfqpoint{4.323779in}{2.104374in}}%
\pgfpathlineto{\pgfqpoint{4.323779in}{2.107323in}}%
\pgfpathlineto{\pgfqpoint{4.328320in}{2.107323in}}%
\pgfpathlineto{\pgfqpoint{4.328320in}{2.104374in}}%
\pgfpathmoveto{\pgfqpoint{4.469094in}{2.007053in}}%
\pgfpathlineto{\pgfqpoint{4.469094in}{2.007053in}}%
\pgfpathlineto{\pgfqpoint{4.469094in}{2.010002in}}%
\pgfpathlineto{\pgfqpoint{4.473635in}{2.010002in}}%
\pgfpathlineto{\pgfqpoint{4.473635in}{2.007053in}}%
\pgfpathmoveto{\pgfqpoint{4.487258in}{1.995255in}}%
\pgfpathlineto{\pgfqpoint{4.487258in}{1.995255in}}%
\pgfpathlineto{\pgfqpoint{4.487258in}{1.998205in}}%
\pgfpathlineto{\pgfqpoint{4.491799in}{1.998205in}}%
\pgfpathlineto{\pgfqpoint{4.491799in}{1.995255in}}%
\pgfpathmoveto{\pgfqpoint{4.478176in}{2.001154in}}%
\pgfpathlineto{\pgfqpoint{4.478176in}{2.001154in}}%
\pgfpathlineto{\pgfqpoint{4.478176in}{2.004103in}}%
\pgfpathlineto{\pgfqpoint{4.482717in}{2.004103in}}%
\pgfpathlineto{\pgfqpoint{4.482717in}{2.001154in}}%
\pgfpathmoveto{\pgfqpoint{4.473635in}{2.004103in}}%
\pgfpathlineto{\pgfqpoint{4.473635in}{2.004103in}}%
\pgfpathlineto{\pgfqpoint{4.473635in}{2.007053in}}%
\pgfpathlineto{\pgfqpoint{4.478176in}{2.007053in}}%
\pgfpathlineto{\pgfqpoint{4.478176in}{2.004103in}}%
\pgfpathmoveto{\pgfqpoint{4.473635in}{2.007053in}}%
\pgfpathlineto{\pgfqpoint{4.473635in}{2.007053in}}%
\pgfpathlineto{\pgfqpoint{4.473635in}{2.010002in}}%
\pgfpathlineto{\pgfqpoint{4.478176in}{2.010002in}}%
\pgfpathlineto{\pgfqpoint{4.478176in}{2.007053in}}%
\pgfpathmoveto{\pgfqpoint{4.478176in}{2.004103in}}%
\pgfpathlineto{\pgfqpoint{4.478176in}{2.004103in}}%
\pgfpathlineto{\pgfqpoint{4.478176in}{2.007053in}}%
\pgfpathlineto{\pgfqpoint{4.482717in}{2.007053in}}%
\pgfpathlineto{\pgfqpoint{4.482717in}{2.004103in}}%
\pgfpathmoveto{\pgfqpoint{4.482717in}{1.998205in}}%
\pgfpathlineto{\pgfqpoint{4.482717in}{1.998205in}}%
\pgfpathlineto{\pgfqpoint{4.482717in}{2.001154in}}%
\pgfpathlineto{\pgfqpoint{4.487258in}{2.001154in}}%
\pgfpathlineto{\pgfqpoint{4.487258in}{1.998205in}}%
\pgfpathmoveto{\pgfqpoint{4.482717in}{2.001154in}}%
\pgfpathlineto{\pgfqpoint{4.482717in}{2.001154in}}%
\pgfpathlineto{\pgfqpoint{4.482717in}{2.004103in}}%
\pgfpathlineto{\pgfqpoint{4.487258in}{2.004103in}}%
\pgfpathlineto{\pgfqpoint{4.487258in}{2.001154in}}%
\pgfpathmoveto{\pgfqpoint{4.487258in}{1.998205in}}%
\pgfpathlineto{\pgfqpoint{4.487258in}{1.998205in}}%
\pgfpathlineto{\pgfqpoint{4.487258in}{2.001154in}}%
\pgfpathlineto{\pgfqpoint{4.491799in}{2.001154in}}%
\pgfpathlineto{\pgfqpoint{4.491799in}{1.998205in}}%
\pgfpathmoveto{\pgfqpoint{4.505422in}{1.983458in}}%
\pgfpathlineto{\pgfqpoint{4.505422in}{1.983458in}}%
\pgfpathlineto{\pgfqpoint{4.505422in}{1.986407in}}%
\pgfpathlineto{\pgfqpoint{4.509963in}{1.986407in}}%
\pgfpathlineto{\pgfqpoint{4.509963in}{1.983458in}}%
\pgfpathmoveto{\pgfqpoint{4.523587in}{1.971661in}}%
\pgfpathlineto{\pgfqpoint{4.523587in}{1.971661in}}%
\pgfpathlineto{\pgfqpoint{4.523587in}{1.974610in}}%
\pgfpathlineto{\pgfqpoint{4.528128in}{1.974610in}}%
\pgfpathlineto{\pgfqpoint{4.528128in}{1.971661in}}%
\pgfpathmoveto{\pgfqpoint{4.514505in}{1.977559in}}%
\pgfpathlineto{\pgfqpoint{4.514505in}{1.977559in}}%
\pgfpathlineto{\pgfqpoint{4.514505in}{1.980509in}}%
\pgfpathlineto{\pgfqpoint{4.519046in}{1.980509in}}%
\pgfpathlineto{\pgfqpoint{4.519046in}{1.977559in}}%
\pgfpathmoveto{\pgfqpoint{4.509963in}{1.980509in}}%
\pgfpathlineto{\pgfqpoint{4.509963in}{1.980509in}}%
\pgfpathlineto{\pgfqpoint{4.509963in}{1.983458in}}%
\pgfpathlineto{\pgfqpoint{4.514505in}{1.983458in}}%
\pgfpathlineto{\pgfqpoint{4.514505in}{1.980509in}}%
\pgfpathmoveto{\pgfqpoint{4.509963in}{1.983458in}}%
\pgfpathlineto{\pgfqpoint{4.509963in}{1.983458in}}%
\pgfpathlineto{\pgfqpoint{4.509963in}{1.986407in}}%
\pgfpathlineto{\pgfqpoint{4.514505in}{1.986407in}}%
\pgfpathlineto{\pgfqpoint{4.514505in}{1.983458in}}%
\pgfpathmoveto{\pgfqpoint{4.514505in}{1.980509in}}%
\pgfpathlineto{\pgfqpoint{4.514505in}{1.980509in}}%
\pgfpathlineto{\pgfqpoint{4.514505in}{1.983458in}}%
\pgfpathlineto{\pgfqpoint{4.519046in}{1.983458in}}%
\pgfpathlineto{\pgfqpoint{4.519046in}{1.980509in}}%
\pgfpathmoveto{\pgfqpoint{4.519046in}{1.974610in}}%
\pgfpathlineto{\pgfqpoint{4.519046in}{1.974610in}}%
\pgfpathlineto{\pgfqpoint{4.519046in}{1.977559in}}%
\pgfpathlineto{\pgfqpoint{4.523587in}{1.977559in}}%
\pgfpathlineto{\pgfqpoint{4.523587in}{1.974610in}}%
\pgfpathmoveto{\pgfqpoint{4.519046in}{1.977559in}}%
\pgfpathlineto{\pgfqpoint{4.519046in}{1.977559in}}%
\pgfpathlineto{\pgfqpoint{4.519046in}{1.980509in}}%
\pgfpathlineto{\pgfqpoint{4.523587in}{1.980509in}}%
\pgfpathlineto{\pgfqpoint{4.523587in}{1.977559in}}%
\pgfpathmoveto{\pgfqpoint{4.523587in}{1.974610in}}%
\pgfpathlineto{\pgfqpoint{4.523587in}{1.974610in}}%
\pgfpathlineto{\pgfqpoint{4.523587in}{1.977559in}}%
\pgfpathlineto{\pgfqpoint{4.528128in}{1.977559in}}%
\pgfpathlineto{\pgfqpoint{4.528128in}{1.974610in}}%
\pgfpathmoveto{\pgfqpoint{4.496340in}{1.989357in}}%
\pgfpathlineto{\pgfqpoint{4.496340in}{1.989357in}}%
\pgfpathlineto{\pgfqpoint{4.496340in}{1.992306in}}%
\pgfpathlineto{\pgfqpoint{4.500881in}{1.992306in}}%
\pgfpathlineto{\pgfqpoint{4.500881in}{1.989357in}}%
\pgfpathmoveto{\pgfqpoint{4.491799in}{1.992306in}}%
\pgfpathlineto{\pgfqpoint{4.491799in}{1.992306in}}%
\pgfpathlineto{\pgfqpoint{4.491799in}{1.995255in}}%
\pgfpathlineto{\pgfqpoint{4.496340in}{1.995255in}}%
\pgfpathlineto{\pgfqpoint{4.496340in}{1.992306in}}%
\pgfpathmoveto{\pgfqpoint{4.491799in}{1.995255in}}%
\pgfpathlineto{\pgfqpoint{4.491799in}{1.995255in}}%
\pgfpathlineto{\pgfqpoint{4.491799in}{1.998205in}}%
\pgfpathlineto{\pgfqpoint{4.496340in}{1.998205in}}%
\pgfpathlineto{\pgfqpoint{4.496340in}{1.995255in}}%
\pgfpathmoveto{\pgfqpoint{4.496340in}{1.992306in}}%
\pgfpathlineto{\pgfqpoint{4.496340in}{1.992306in}}%
\pgfpathlineto{\pgfqpoint{4.496340in}{1.995255in}}%
\pgfpathlineto{\pgfqpoint{4.500881in}{1.995255in}}%
\pgfpathlineto{\pgfqpoint{4.500881in}{1.992306in}}%
\pgfpathmoveto{\pgfqpoint{4.500881in}{1.986407in}}%
\pgfpathlineto{\pgfqpoint{4.500881in}{1.986407in}}%
\pgfpathlineto{\pgfqpoint{4.500881in}{1.989357in}}%
\pgfpathlineto{\pgfqpoint{4.505422in}{1.989357in}}%
\pgfpathlineto{\pgfqpoint{4.505422in}{1.986407in}}%
\pgfpathmoveto{\pgfqpoint{4.500881in}{1.989357in}}%
\pgfpathlineto{\pgfqpoint{4.500881in}{1.989357in}}%
\pgfpathlineto{\pgfqpoint{4.500881in}{1.992306in}}%
\pgfpathlineto{\pgfqpoint{4.505422in}{1.992306in}}%
\pgfpathlineto{\pgfqpoint{4.505422in}{1.989357in}}%
\pgfpathmoveto{\pgfqpoint{4.505422in}{1.986407in}}%
\pgfpathlineto{\pgfqpoint{4.505422in}{1.986407in}}%
\pgfpathlineto{\pgfqpoint{4.505422in}{1.989357in}}%
\pgfpathlineto{\pgfqpoint{4.509963in}{1.989357in}}%
\pgfpathlineto{\pgfqpoint{4.509963in}{1.986407in}}%
\pgfpathmoveto{\pgfqpoint{4.396437in}{2.054239in}}%
\pgfpathlineto{\pgfqpoint{4.396437in}{2.054239in}}%
\pgfpathlineto{\pgfqpoint{4.396437in}{2.057188in}}%
\pgfpathlineto{\pgfqpoint{4.400978in}{2.057188in}}%
\pgfpathlineto{\pgfqpoint{4.400978in}{2.054239in}}%
\pgfpathmoveto{\pgfqpoint{4.414601in}{2.042442in}}%
\pgfpathlineto{\pgfqpoint{4.414601in}{2.042442in}}%
\pgfpathlineto{\pgfqpoint{4.414601in}{2.045391in}}%
\pgfpathlineto{\pgfqpoint{4.419142in}{2.045391in}}%
\pgfpathlineto{\pgfqpoint{4.419142in}{2.042442in}}%
\pgfpathmoveto{\pgfqpoint{4.405519in}{2.048340in}}%
\pgfpathlineto{\pgfqpoint{4.405519in}{2.048340in}}%
\pgfpathlineto{\pgfqpoint{4.405519in}{2.051290in}}%
\pgfpathlineto{\pgfqpoint{4.410060in}{2.051290in}}%
\pgfpathlineto{\pgfqpoint{4.410060in}{2.048340in}}%
\pgfpathmoveto{\pgfqpoint{4.400978in}{2.051290in}}%
\pgfpathlineto{\pgfqpoint{4.400978in}{2.051290in}}%
\pgfpathlineto{\pgfqpoint{4.400978in}{2.054239in}}%
\pgfpathlineto{\pgfqpoint{4.405519in}{2.054239in}}%
\pgfpathlineto{\pgfqpoint{4.405519in}{2.051290in}}%
\pgfpathmoveto{\pgfqpoint{4.400978in}{2.054239in}}%
\pgfpathlineto{\pgfqpoint{4.400978in}{2.054239in}}%
\pgfpathlineto{\pgfqpoint{4.400978in}{2.057188in}}%
\pgfpathlineto{\pgfqpoint{4.405519in}{2.057188in}}%
\pgfpathlineto{\pgfqpoint{4.405519in}{2.054239in}}%
\pgfpathmoveto{\pgfqpoint{4.405519in}{2.051290in}}%
\pgfpathlineto{\pgfqpoint{4.405519in}{2.051290in}}%
\pgfpathlineto{\pgfqpoint{4.405519in}{2.054239in}}%
\pgfpathlineto{\pgfqpoint{4.410060in}{2.054239in}}%
\pgfpathlineto{\pgfqpoint{4.410060in}{2.051290in}}%
\pgfpathmoveto{\pgfqpoint{4.410060in}{2.045391in}}%
\pgfpathlineto{\pgfqpoint{4.410060in}{2.045391in}}%
\pgfpathlineto{\pgfqpoint{4.410060in}{2.048340in}}%
\pgfpathlineto{\pgfqpoint{4.414601in}{2.048340in}}%
\pgfpathlineto{\pgfqpoint{4.414601in}{2.045391in}}%
\pgfpathmoveto{\pgfqpoint{4.410060in}{2.048340in}}%
\pgfpathlineto{\pgfqpoint{4.410060in}{2.048340in}}%
\pgfpathlineto{\pgfqpoint{4.410060in}{2.051290in}}%
\pgfpathlineto{\pgfqpoint{4.414601in}{2.051290in}}%
\pgfpathlineto{\pgfqpoint{4.414601in}{2.048340in}}%
\pgfpathmoveto{\pgfqpoint{4.414601in}{2.045391in}}%
\pgfpathlineto{\pgfqpoint{4.414601in}{2.045391in}}%
\pgfpathlineto{\pgfqpoint{4.414601in}{2.048340in}}%
\pgfpathlineto{\pgfqpoint{4.419142in}{2.048340in}}%
\pgfpathlineto{\pgfqpoint{4.419142in}{2.045391in}}%
\pgfpathmoveto{\pgfqpoint{4.432766in}{2.030646in}}%
\pgfpathlineto{\pgfqpoint{4.432766in}{2.030646in}}%
\pgfpathlineto{\pgfqpoint{4.432766in}{2.033595in}}%
\pgfpathlineto{\pgfqpoint{4.437307in}{2.033595in}}%
\pgfpathlineto{\pgfqpoint{4.437307in}{2.030646in}}%
\pgfpathmoveto{\pgfqpoint{4.450930in}{2.018849in}}%
\pgfpathlineto{\pgfqpoint{4.450930in}{2.018849in}}%
\pgfpathlineto{\pgfqpoint{4.450930in}{2.021798in}}%
\pgfpathlineto{\pgfqpoint{4.455471in}{2.021798in}}%
\pgfpathlineto{\pgfqpoint{4.455471in}{2.018849in}}%
\pgfpathmoveto{\pgfqpoint{4.441848in}{2.024747in}}%
\pgfpathlineto{\pgfqpoint{4.441848in}{2.024747in}}%
\pgfpathlineto{\pgfqpoint{4.441848in}{2.027697in}}%
\pgfpathlineto{\pgfqpoint{4.446389in}{2.027697in}}%
\pgfpathlineto{\pgfqpoint{4.446389in}{2.024747in}}%
\pgfpathmoveto{\pgfqpoint{4.437307in}{2.027697in}}%
\pgfpathlineto{\pgfqpoint{4.437307in}{2.027697in}}%
\pgfpathlineto{\pgfqpoint{4.437307in}{2.030646in}}%
\pgfpathlineto{\pgfqpoint{4.441848in}{2.030646in}}%
\pgfpathlineto{\pgfqpoint{4.441848in}{2.027697in}}%
\pgfpathmoveto{\pgfqpoint{4.437307in}{2.030646in}}%
\pgfpathlineto{\pgfqpoint{4.437307in}{2.030646in}}%
\pgfpathlineto{\pgfqpoint{4.437307in}{2.033595in}}%
\pgfpathlineto{\pgfqpoint{4.441848in}{2.033595in}}%
\pgfpathlineto{\pgfqpoint{4.441848in}{2.030646in}}%
\pgfpathmoveto{\pgfqpoint{4.441848in}{2.027697in}}%
\pgfpathlineto{\pgfqpoint{4.441848in}{2.027697in}}%
\pgfpathlineto{\pgfqpoint{4.441848in}{2.030646in}}%
\pgfpathlineto{\pgfqpoint{4.446389in}{2.030646in}}%
\pgfpathlineto{\pgfqpoint{4.446389in}{2.027697in}}%
\pgfpathmoveto{\pgfqpoint{4.446389in}{2.021798in}}%
\pgfpathlineto{\pgfqpoint{4.446389in}{2.021798in}}%
\pgfpathlineto{\pgfqpoint{4.446389in}{2.024747in}}%
\pgfpathlineto{\pgfqpoint{4.450930in}{2.024747in}}%
\pgfpathlineto{\pgfqpoint{4.450930in}{2.021798in}}%
\pgfpathmoveto{\pgfqpoint{4.446389in}{2.024747in}}%
\pgfpathlineto{\pgfqpoint{4.446389in}{2.024747in}}%
\pgfpathlineto{\pgfqpoint{4.446389in}{2.027697in}}%
\pgfpathlineto{\pgfqpoint{4.450930in}{2.027697in}}%
\pgfpathlineto{\pgfqpoint{4.450930in}{2.024747in}}%
\pgfpathmoveto{\pgfqpoint{4.450930in}{2.021798in}}%
\pgfpathlineto{\pgfqpoint{4.450930in}{2.021798in}}%
\pgfpathlineto{\pgfqpoint{4.450930in}{2.024747in}}%
\pgfpathlineto{\pgfqpoint{4.455471in}{2.024747in}}%
\pgfpathlineto{\pgfqpoint{4.455471in}{2.021798in}}%
\pgfpathmoveto{\pgfqpoint{4.423683in}{2.036544in}}%
\pgfpathlineto{\pgfqpoint{4.423683in}{2.036544in}}%
\pgfpathlineto{\pgfqpoint{4.423683in}{2.039493in}}%
\pgfpathlineto{\pgfqpoint{4.428224in}{2.039493in}}%
\pgfpathlineto{\pgfqpoint{4.428224in}{2.036544in}}%
\pgfpathmoveto{\pgfqpoint{4.419142in}{2.039493in}}%
\pgfpathlineto{\pgfqpoint{4.419142in}{2.039493in}}%
\pgfpathlineto{\pgfqpoint{4.419142in}{2.042442in}}%
\pgfpathlineto{\pgfqpoint{4.423683in}{2.042442in}}%
\pgfpathlineto{\pgfqpoint{4.423683in}{2.039493in}}%
\pgfpathmoveto{\pgfqpoint{4.419142in}{2.042442in}}%
\pgfpathlineto{\pgfqpoint{4.419142in}{2.042442in}}%
\pgfpathlineto{\pgfqpoint{4.419142in}{2.045391in}}%
\pgfpathlineto{\pgfqpoint{4.423683in}{2.045391in}}%
\pgfpathlineto{\pgfqpoint{4.423683in}{2.042442in}}%
\pgfpathmoveto{\pgfqpoint{4.423683in}{2.039493in}}%
\pgfpathlineto{\pgfqpoint{4.423683in}{2.039493in}}%
\pgfpathlineto{\pgfqpoint{4.423683in}{2.042442in}}%
\pgfpathlineto{\pgfqpoint{4.428224in}{2.042442in}}%
\pgfpathlineto{\pgfqpoint{4.428224in}{2.039493in}}%
\pgfpathmoveto{\pgfqpoint{4.428224in}{2.033595in}}%
\pgfpathlineto{\pgfqpoint{4.428224in}{2.033595in}}%
\pgfpathlineto{\pgfqpoint{4.428224in}{2.036544in}}%
\pgfpathlineto{\pgfqpoint{4.432766in}{2.036544in}}%
\pgfpathlineto{\pgfqpoint{4.432766in}{2.033595in}}%
\pgfpathmoveto{\pgfqpoint{4.428224in}{2.036544in}}%
\pgfpathlineto{\pgfqpoint{4.428224in}{2.036544in}}%
\pgfpathlineto{\pgfqpoint{4.428224in}{2.039493in}}%
\pgfpathlineto{\pgfqpoint{4.432766in}{2.039493in}}%
\pgfpathlineto{\pgfqpoint{4.432766in}{2.036544in}}%
\pgfpathmoveto{\pgfqpoint{4.432766in}{2.033595in}}%
\pgfpathlineto{\pgfqpoint{4.432766in}{2.033595in}}%
\pgfpathlineto{\pgfqpoint{4.432766in}{2.036544in}}%
\pgfpathlineto{\pgfqpoint{4.437307in}{2.036544in}}%
\pgfpathlineto{\pgfqpoint{4.437307in}{2.033595in}}%
\pgfpathmoveto{\pgfqpoint{4.387355in}{2.060137in}}%
\pgfpathlineto{\pgfqpoint{4.387355in}{2.060137in}}%
\pgfpathlineto{\pgfqpoint{4.387355in}{2.063086in}}%
\pgfpathlineto{\pgfqpoint{4.391896in}{2.063086in}}%
\pgfpathlineto{\pgfqpoint{4.391896in}{2.060137in}}%
\pgfpathmoveto{\pgfqpoint{4.382814in}{2.063086in}}%
\pgfpathlineto{\pgfqpoint{4.382814in}{2.063086in}}%
\pgfpathlineto{\pgfqpoint{4.382814in}{2.066035in}}%
\pgfpathlineto{\pgfqpoint{4.387355in}{2.066035in}}%
\pgfpathlineto{\pgfqpoint{4.387355in}{2.063086in}}%
\pgfpathmoveto{\pgfqpoint{4.382814in}{2.066035in}}%
\pgfpathlineto{\pgfqpoint{4.382814in}{2.066035in}}%
\pgfpathlineto{\pgfqpoint{4.382814in}{2.068984in}}%
\pgfpathlineto{\pgfqpoint{4.387355in}{2.068984in}}%
\pgfpathlineto{\pgfqpoint{4.387355in}{2.066035in}}%
\pgfpathmoveto{\pgfqpoint{4.387355in}{2.063086in}}%
\pgfpathlineto{\pgfqpoint{4.387355in}{2.063086in}}%
\pgfpathlineto{\pgfqpoint{4.387355in}{2.066035in}}%
\pgfpathlineto{\pgfqpoint{4.391896in}{2.066035in}}%
\pgfpathlineto{\pgfqpoint{4.391896in}{2.063086in}}%
\pgfpathmoveto{\pgfqpoint{4.391896in}{2.057188in}}%
\pgfpathlineto{\pgfqpoint{4.391896in}{2.057188in}}%
\pgfpathlineto{\pgfqpoint{4.391896in}{2.060137in}}%
\pgfpathlineto{\pgfqpoint{4.396437in}{2.060137in}}%
\pgfpathlineto{\pgfqpoint{4.396437in}{2.057188in}}%
\pgfpathmoveto{\pgfqpoint{4.391896in}{2.060137in}}%
\pgfpathlineto{\pgfqpoint{4.391896in}{2.060137in}}%
\pgfpathlineto{\pgfqpoint{4.391896in}{2.063086in}}%
\pgfpathlineto{\pgfqpoint{4.396437in}{2.063086in}}%
\pgfpathlineto{\pgfqpoint{4.396437in}{2.060137in}}%
\pgfpathmoveto{\pgfqpoint{4.396437in}{2.057188in}}%
\pgfpathlineto{\pgfqpoint{4.396437in}{2.057188in}}%
\pgfpathlineto{\pgfqpoint{4.396437in}{2.060137in}}%
\pgfpathlineto{\pgfqpoint{4.400978in}{2.060137in}}%
\pgfpathlineto{\pgfqpoint{4.400978in}{2.057188in}}%
\pgfpathmoveto{\pgfqpoint{4.460012in}{2.012951in}}%
\pgfpathlineto{\pgfqpoint{4.460012in}{2.012951in}}%
\pgfpathlineto{\pgfqpoint{4.460012in}{2.015900in}}%
\pgfpathlineto{\pgfqpoint{4.464553in}{2.015900in}}%
\pgfpathlineto{\pgfqpoint{4.464553in}{2.012951in}}%
\pgfpathmoveto{\pgfqpoint{4.455471in}{2.015900in}}%
\pgfpathlineto{\pgfqpoint{4.455471in}{2.015900in}}%
\pgfpathlineto{\pgfqpoint{4.455471in}{2.018849in}}%
\pgfpathlineto{\pgfqpoint{4.460012in}{2.018849in}}%
\pgfpathlineto{\pgfqpoint{4.460012in}{2.015900in}}%
\pgfpathmoveto{\pgfqpoint{4.455471in}{2.018849in}}%
\pgfpathlineto{\pgfqpoint{4.455471in}{2.018849in}}%
\pgfpathlineto{\pgfqpoint{4.455471in}{2.021798in}}%
\pgfpathlineto{\pgfqpoint{4.460012in}{2.021798in}}%
\pgfpathlineto{\pgfqpoint{4.460012in}{2.018849in}}%
\pgfpathmoveto{\pgfqpoint{4.460012in}{2.015900in}}%
\pgfpathlineto{\pgfqpoint{4.460012in}{2.015900in}}%
\pgfpathlineto{\pgfqpoint{4.460012in}{2.018849in}}%
\pgfpathlineto{\pgfqpoint{4.464553in}{2.018849in}}%
\pgfpathlineto{\pgfqpoint{4.464553in}{2.015900in}}%
\pgfpathmoveto{\pgfqpoint{4.464553in}{2.010002in}}%
\pgfpathlineto{\pgfqpoint{4.464553in}{2.010002in}}%
\pgfpathlineto{\pgfqpoint{4.464553in}{2.012951in}}%
\pgfpathlineto{\pgfqpoint{4.469094in}{2.012951in}}%
\pgfpathlineto{\pgfqpoint{4.469094in}{2.010002in}}%
\pgfpathmoveto{\pgfqpoint{4.464553in}{2.012951in}}%
\pgfpathlineto{\pgfqpoint{4.464553in}{2.012951in}}%
\pgfpathlineto{\pgfqpoint{4.464553in}{2.015900in}}%
\pgfpathlineto{\pgfqpoint{4.469094in}{2.015900in}}%
\pgfpathlineto{\pgfqpoint{4.469094in}{2.012951in}}%
\pgfpathmoveto{\pgfqpoint{4.469094in}{2.010002in}}%
\pgfpathlineto{\pgfqpoint{4.469094in}{2.010002in}}%
\pgfpathlineto{\pgfqpoint{4.469094in}{2.012951in}}%
\pgfpathlineto{\pgfqpoint{4.473635in}{2.012951in}}%
\pgfpathlineto{\pgfqpoint{4.473635in}{2.010002in}}%
\pgfpathmoveto{\pgfqpoint{4.614408in}{1.912675in}}%
\pgfpathlineto{\pgfqpoint{4.614408in}{1.912675in}}%
\pgfpathlineto{\pgfqpoint{4.614408in}{1.915624in}}%
\pgfpathlineto{\pgfqpoint{4.618949in}{1.915624in}}%
\pgfpathlineto{\pgfqpoint{4.618949in}{1.912675in}}%
\pgfpathmoveto{\pgfqpoint{4.632572in}{1.900878in}}%
\pgfpathlineto{\pgfqpoint{4.632572in}{1.900878in}}%
\pgfpathlineto{\pgfqpoint{4.632572in}{1.903827in}}%
\pgfpathlineto{\pgfqpoint{4.637113in}{1.903827in}}%
\pgfpathlineto{\pgfqpoint{4.637113in}{1.900878in}}%
\pgfpathmoveto{\pgfqpoint{4.623490in}{1.906776in}}%
\pgfpathlineto{\pgfqpoint{4.623490in}{1.906776in}}%
\pgfpathlineto{\pgfqpoint{4.623490in}{1.909725in}}%
\pgfpathlineto{\pgfqpoint{4.628031in}{1.909725in}}%
\pgfpathlineto{\pgfqpoint{4.628031in}{1.906776in}}%
\pgfpathmoveto{\pgfqpoint{4.618949in}{1.909725in}}%
\pgfpathlineto{\pgfqpoint{4.618949in}{1.909725in}}%
\pgfpathlineto{\pgfqpoint{4.618949in}{1.912675in}}%
\pgfpathlineto{\pgfqpoint{4.623490in}{1.912675in}}%
\pgfpathlineto{\pgfqpoint{4.623490in}{1.909725in}}%
\pgfpathmoveto{\pgfqpoint{4.618949in}{1.912675in}}%
\pgfpathlineto{\pgfqpoint{4.618949in}{1.912675in}}%
\pgfpathlineto{\pgfqpoint{4.618949in}{1.915624in}}%
\pgfpathlineto{\pgfqpoint{4.623490in}{1.915624in}}%
\pgfpathlineto{\pgfqpoint{4.623490in}{1.912675in}}%
\pgfpathmoveto{\pgfqpoint{4.623490in}{1.909725in}}%
\pgfpathlineto{\pgfqpoint{4.623490in}{1.909725in}}%
\pgfpathlineto{\pgfqpoint{4.623490in}{1.912675in}}%
\pgfpathlineto{\pgfqpoint{4.628031in}{1.912675in}}%
\pgfpathlineto{\pgfqpoint{4.628031in}{1.909725in}}%
\pgfpathmoveto{\pgfqpoint{4.628031in}{1.903827in}}%
\pgfpathlineto{\pgfqpoint{4.628031in}{1.903827in}}%
\pgfpathlineto{\pgfqpoint{4.628031in}{1.906776in}}%
\pgfpathlineto{\pgfqpoint{4.632572in}{1.906776in}}%
\pgfpathlineto{\pgfqpoint{4.632572in}{1.903827in}}%
\pgfpathmoveto{\pgfqpoint{4.628031in}{1.906776in}}%
\pgfpathlineto{\pgfqpoint{4.628031in}{1.906776in}}%
\pgfpathlineto{\pgfqpoint{4.628031in}{1.909725in}}%
\pgfpathlineto{\pgfqpoint{4.632572in}{1.909725in}}%
\pgfpathlineto{\pgfqpoint{4.632572in}{1.906776in}}%
\pgfpathmoveto{\pgfqpoint{4.632572in}{1.903827in}}%
\pgfpathlineto{\pgfqpoint{4.632572in}{1.903827in}}%
\pgfpathlineto{\pgfqpoint{4.632572in}{1.906776in}}%
\pgfpathlineto{\pgfqpoint{4.637113in}{1.906776in}}%
\pgfpathlineto{\pgfqpoint{4.637113in}{1.903827in}}%
\pgfpathmoveto{\pgfqpoint{4.650736in}{1.889081in}}%
\pgfpathlineto{\pgfqpoint{4.650736in}{1.889081in}}%
\pgfpathlineto{\pgfqpoint{4.650736in}{1.892030in}}%
\pgfpathlineto{\pgfqpoint{4.655277in}{1.892030in}}%
\pgfpathlineto{\pgfqpoint{4.655277in}{1.889081in}}%
\pgfpathmoveto{\pgfqpoint{4.668900in}{1.877285in}}%
\pgfpathlineto{\pgfqpoint{4.668900in}{1.877285in}}%
\pgfpathlineto{\pgfqpoint{4.668900in}{1.880234in}}%
\pgfpathlineto{\pgfqpoint{4.673441in}{1.880234in}}%
\pgfpathlineto{\pgfqpoint{4.673441in}{1.877285in}}%
\pgfpathmoveto{\pgfqpoint{4.659818in}{1.883183in}}%
\pgfpathlineto{\pgfqpoint{4.659818in}{1.883183in}}%
\pgfpathlineto{\pgfqpoint{4.659818in}{1.886132in}}%
\pgfpathlineto{\pgfqpoint{4.664359in}{1.886132in}}%
\pgfpathlineto{\pgfqpoint{4.664359in}{1.883183in}}%
\pgfpathmoveto{\pgfqpoint{4.655277in}{1.886132in}}%
\pgfpathlineto{\pgfqpoint{4.655277in}{1.886132in}}%
\pgfpathlineto{\pgfqpoint{4.655277in}{1.889081in}}%
\pgfpathlineto{\pgfqpoint{4.659818in}{1.889081in}}%
\pgfpathlineto{\pgfqpoint{4.659818in}{1.886132in}}%
\pgfpathmoveto{\pgfqpoint{4.655277in}{1.889081in}}%
\pgfpathlineto{\pgfqpoint{4.655277in}{1.889081in}}%
\pgfpathlineto{\pgfqpoint{4.655277in}{1.892030in}}%
\pgfpathlineto{\pgfqpoint{4.659818in}{1.892030in}}%
\pgfpathlineto{\pgfqpoint{4.659818in}{1.889081in}}%
\pgfpathmoveto{\pgfqpoint{4.659818in}{1.886132in}}%
\pgfpathlineto{\pgfqpoint{4.659818in}{1.886132in}}%
\pgfpathlineto{\pgfqpoint{4.659818in}{1.889081in}}%
\pgfpathlineto{\pgfqpoint{4.664359in}{1.889081in}}%
\pgfpathlineto{\pgfqpoint{4.664359in}{1.886132in}}%
\pgfpathmoveto{\pgfqpoint{4.664359in}{1.880234in}}%
\pgfpathlineto{\pgfqpoint{4.664359in}{1.880234in}}%
\pgfpathlineto{\pgfqpoint{4.664359in}{1.883183in}}%
\pgfpathlineto{\pgfqpoint{4.668900in}{1.883183in}}%
\pgfpathlineto{\pgfqpoint{4.668900in}{1.880234in}}%
\pgfpathmoveto{\pgfqpoint{4.664359in}{1.883183in}}%
\pgfpathlineto{\pgfqpoint{4.664359in}{1.883183in}}%
\pgfpathlineto{\pgfqpoint{4.664359in}{1.886132in}}%
\pgfpathlineto{\pgfqpoint{4.668900in}{1.886132in}}%
\pgfpathlineto{\pgfqpoint{4.668900in}{1.883183in}}%
\pgfpathmoveto{\pgfqpoint{4.668900in}{1.880234in}}%
\pgfpathlineto{\pgfqpoint{4.668900in}{1.880234in}}%
\pgfpathlineto{\pgfqpoint{4.668900in}{1.883183in}}%
\pgfpathlineto{\pgfqpoint{4.673441in}{1.883183in}}%
\pgfpathlineto{\pgfqpoint{4.673441in}{1.880234in}}%
\pgfpathmoveto{\pgfqpoint{4.641654in}{1.894980in}}%
\pgfpathlineto{\pgfqpoint{4.641654in}{1.894980in}}%
\pgfpathlineto{\pgfqpoint{4.641654in}{1.897929in}}%
\pgfpathlineto{\pgfqpoint{4.646195in}{1.897929in}}%
\pgfpathlineto{\pgfqpoint{4.646195in}{1.894980in}}%
\pgfpathmoveto{\pgfqpoint{4.637113in}{1.897929in}}%
\pgfpathlineto{\pgfqpoint{4.637113in}{1.897929in}}%
\pgfpathlineto{\pgfqpoint{4.637113in}{1.900878in}}%
\pgfpathlineto{\pgfqpoint{4.641654in}{1.900878in}}%
\pgfpathlineto{\pgfqpoint{4.641654in}{1.897929in}}%
\pgfpathmoveto{\pgfqpoint{4.637113in}{1.900878in}}%
\pgfpathlineto{\pgfqpoint{4.637113in}{1.900878in}}%
\pgfpathlineto{\pgfqpoint{4.637113in}{1.903827in}}%
\pgfpathlineto{\pgfqpoint{4.641654in}{1.903827in}}%
\pgfpathlineto{\pgfqpoint{4.641654in}{1.900878in}}%
\pgfpathmoveto{\pgfqpoint{4.641654in}{1.897929in}}%
\pgfpathlineto{\pgfqpoint{4.641654in}{1.897929in}}%
\pgfpathlineto{\pgfqpoint{4.641654in}{1.900878in}}%
\pgfpathlineto{\pgfqpoint{4.646195in}{1.900878in}}%
\pgfpathlineto{\pgfqpoint{4.646195in}{1.897929in}}%
\pgfpathmoveto{\pgfqpoint{4.646195in}{1.892030in}}%
\pgfpathlineto{\pgfqpoint{4.646195in}{1.892030in}}%
\pgfpathlineto{\pgfqpoint{4.646195in}{1.894980in}}%
\pgfpathlineto{\pgfqpoint{4.650736in}{1.894980in}}%
\pgfpathlineto{\pgfqpoint{4.650736in}{1.892030in}}%
\pgfpathmoveto{\pgfqpoint{4.646195in}{1.894980in}}%
\pgfpathlineto{\pgfqpoint{4.646195in}{1.894980in}}%
\pgfpathlineto{\pgfqpoint{4.646195in}{1.897929in}}%
\pgfpathlineto{\pgfqpoint{4.650736in}{1.897929in}}%
\pgfpathlineto{\pgfqpoint{4.650736in}{1.894980in}}%
\pgfpathmoveto{\pgfqpoint{4.650736in}{1.892030in}}%
\pgfpathlineto{\pgfqpoint{4.650736in}{1.892030in}}%
\pgfpathlineto{\pgfqpoint{4.650736in}{1.894980in}}%
\pgfpathlineto{\pgfqpoint{4.655277in}{1.894980in}}%
\pgfpathlineto{\pgfqpoint{4.655277in}{1.892030in}}%
\pgfpathmoveto{\pgfqpoint{4.541751in}{1.959863in}}%
\pgfpathlineto{\pgfqpoint{4.541751in}{1.959863in}}%
\pgfpathlineto{\pgfqpoint{4.541751in}{1.962813in}}%
\pgfpathlineto{\pgfqpoint{4.546292in}{1.962813in}}%
\pgfpathlineto{\pgfqpoint{4.546292in}{1.959863in}}%
\pgfpathmoveto{\pgfqpoint{4.559915in}{1.948066in}}%
\pgfpathlineto{\pgfqpoint{4.559915in}{1.948066in}}%
\pgfpathlineto{\pgfqpoint{4.559915in}{1.951016in}}%
\pgfpathlineto{\pgfqpoint{4.564456in}{1.951016in}}%
\pgfpathlineto{\pgfqpoint{4.564456in}{1.948066in}}%
\pgfpathmoveto{\pgfqpoint{4.550833in}{1.953965in}}%
\pgfpathlineto{\pgfqpoint{4.550833in}{1.953965in}}%
\pgfpathlineto{\pgfqpoint{4.550833in}{1.956914in}}%
\pgfpathlineto{\pgfqpoint{4.555374in}{1.956914in}}%
\pgfpathlineto{\pgfqpoint{4.555374in}{1.953965in}}%
\pgfpathmoveto{\pgfqpoint{4.546292in}{1.956914in}}%
\pgfpathlineto{\pgfqpoint{4.546292in}{1.956914in}}%
\pgfpathlineto{\pgfqpoint{4.546292in}{1.959863in}}%
\pgfpathlineto{\pgfqpoint{4.550833in}{1.959863in}}%
\pgfpathlineto{\pgfqpoint{4.550833in}{1.956914in}}%
\pgfpathmoveto{\pgfqpoint{4.546292in}{1.959863in}}%
\pgfpathlineto{\pgfqpoint{4.546292in}{1.959863in}}%
\pgfpathlineto{\pgfqpoint{4.546292in}{1.962813in}}%
\pgfpathlineto{\pgfqpoint{4.550833in}{1.962813in}}%
\pgfpathlineto{\pgfqpoint{4.550833in}{1.959863in}}%
\pgfpathmoveto{\pgfqpoint{4.550833in}{1.956914in}}%
\pgfpathlineto{\pgfqpoint{4.550833in}{1.956914in}}%
\pgfpathlineto{\pgfqpoint{4.550833in}{1.959863in}}%
\pgfpathlineto{\pgfqpoint{4.555374in}{1.959863in}}%
\pgfpathlineto{\pgfqpoint{4.555374in}{1.956914in}}%
\pgfpathmoveto{\pgfqpoint{4.555374in}{1.951016in}}%
\pgfpathlineto{\pgfqpoint{4.555374in}{1.951016in}}%
\pgfpathlineto{\pgfqpoint{4.555374in}{1.953965in}}%
\pgfpathlineto{\pgfqpoint{4.559915in}{1.953965in}}%
\pgfpathlineto{\pgfqpoint{4.559915in}{1.951016in}}%
\pgfpathmoveto{\pgfqpoint{4.555374in}{1.953965in}}%
\pgfpathlineto{\pgfqpoint{4.555374in}{1.953965in}}%
\pgfpathlineto{\pgfqpoint{4.555374in}{1.956914in}}%
\pgfpathlineto{\pgfqpoint{4.559915in}{1.956914in}}%
\pgfpathlineto{\pgfqpoint{4.559915in}{1.953965in}}%
\pgfpathmoveto{\pgfqpoint{4.559915in}{1.951016in}}%
\pgfpathlineto{\pgfqpoint{4.559915in}{1.951016in}}%
\pgfpathlineto{\pgfqpoint{4.559915in}{1.953965in}}%
\pgfpathlineto{\pgfqpoint{4.564456in}{1.953965in}}%
\pgfpathlineto{\pgfqpoint{4.564456in}{1.951016in}}%
\pgfpathmoveto{\pgfqpoint{4.578079in}{1.936269in}}%
\pgfpathlineto{\pgfqpoint{4.578079in}{1.936269in}}%
\pgfpathlineto{\pgfqpoint{4.578079in}{1.939218in}}%
\pgfpathlineto{\pgfqpoint{4.582620in}{1.939218in}}%
\pgfpathlineto{\pgfqpoint{4.582620in}{1.936269in}}%
\pgfpathmoveto{\pgfqpoint{4.596243in}{1.924472in}}%
\pgfpathlineto{\pgfqpoint{4.596243in}{1.924472in}}%
\pgfpathlineto{\pgfqpoint{4.596243in}{1.927421in}}%
\pgfpathlineto{\pgfqpoint{4.600785in}{1.927421in}}%
\pgfpathlineto{\pgfqpoint{4.600785in}{1.924472in}}%
\pgfpathmoveto{\pgfqpoint{4.587161in}{1.930370in}}%
\pgfpathlineto{\pgfqpoint{4.587161in}{1.930370in}}%
\pgfpathlineto{\pgfqpoint{4.587161in}{1.933320in}}%
\pgfpathlineto{\pgfqpoint{4.591702in}{1.933320in}}%
\pgfpathlineto{\pgfqpoint{4.591702in}{1.930370in}}%
\pgfpathmoveto{\pgfqpoint{4.582620in}{1.933320in}}%
\pgfpathlineto{\pgfqpoint{4.582620in}{1.933320in}}%
\pgfpathlineto{\pgfqpoint{4.582620in}{1.936269in}}%
\pgfpathlineto{\pgfqpoint{4.587161in}{1.936269in}}%
\pgfpathlineto{\pgfqpoint{4.587161in}{1.933320in}}%
\pgfpathmoveto{\pgfqpoint{4.582620in}{1.936269in}}%
\pgfpathlineto{\pgfqpoint{4.582620in}{1.936269in}}%
\pgfpathlineto{\pgfqpoint{4.582620in}{1.939218in}}%
\pgfpathlineto{\pgfqpoint{4.587161in}{1.939218in}}%
\pgfpathlineto{\pgfqpoint{4.587161in}{1.936269in}}%
\pgfpathmoveto{\pgfqpoint{4.587161in}{1.933320in}}%
\pgfpathlineto{\pgfqpoint{4.587161in}{1.933320in}}%
\pgfpathlineto{\pgfqpoint{4.587161in}{1.936269in}}%
\pgfpathlineto{\pgfqpoint{4.591702in}{1.936269in}}%
\pgfpathlineto{\pgfqpoint{4.591702in}{1.933320in}}%
\pgfpathmoveto{\pgfqpoint{4.591702in}{1.927421in}}%
\pgfpathlineto{\pgfqpoint{4.591702in}{1.927421in}}%
\pgfpathlineto{\pgfqpoint{4.591702in}{1.930370in}}%
\pgfpathlineto{\pgfqpoint{4.596243in}{1.930370in}}%
\pgfpathlineto{\pgfqpoint{4.596243in}{1.927421in}}%
\pgfpathmoveto{\pgfqpoint{4.591702in}{1.930370in}}%
\pgfpathlineto{\pgfqpoint{4.591702in}{1.930370in}}%
\pgfpathlineto{\pgfqpoint{4.591702in}{1.933320in}}%
\pgfpathlineto{\pgfqpoint{4.596243in}{1.933320in}}%
\pgfpathlineto{\pgfqpoint{4.596243in}{1.930370in}}%
\pgfpathmoveto{\pgfqpoint{4.596243in}{1.927421in}}%
\pgfpathlineto{\pgfqpoint{4.596243in}{1.927421in}}%
\pgfpathlineto{\pgfqpoint{4.596243in}{1.930370in}}%
\pgfpathlineto{\pgfqpoint{4.600785in}{1.930370in}}%
\pgfpathlineto{\pgfqpoint{4.600785in}{1.927421in}}%
\pgfpathmoveto{\pgfqpoint{4.568997in}{1.942168in}}%
\pgfpathlineto{\pgfqpoint{4.568997in}{1.942168in}}%
\pgfpathlineto{\pgfqpoint{4.568997in}{1.945117in}}%
\pgfpathlineto{\pgfqpoint{4.573538in}{1.945117in}}%
\pgfpathlineto{\pgfqpoint{4.573538in}{1.942168in}}%
\pgfpathmoveto{\pgfqpoint{4.564456in}{1.945117in}}%
\pgfpathlineto{\pgfqpoint{4.564456in}{1.945117in}}%
\pgfpathlineto{\pgfqpoint{4.564456in}{1.948066in}}%
\pgfpathlineto{\pgfqpoint{4.568997in}{1.948066in}}%
\pgfpathlineto{\pgfqpoint{4.568997in}{1.945117in}}%
\pgfpathmoveto{\pgfqpoint{4.564456in}{1.948066in}}%
\pgfpathlineto{\pgfqpoint{4.564456in}{1.948066in}}%
\pgfpathlineto{\pgfqpoint{4.564456in}{1.951016in}}%
\pgfpathlineto{\pgfqpoint{4.568997in}{1.951016in}}%
\pgfpathlineto{\pgfqpoint{4.568997in}{1.948066in}}%
\pgfpathmoveto{\pgfqpoint{4.568997in}{1.945117in}}%
\pgfpathlineto{\pgfqpoint{4.568997in}{1.945117in}}%
\pgfpathlineto{\pgfqpoint{4.568997in}{1.948066in}}%
\pgfpathlineto{\pgfqpoint{4.573538in}{1.948066in}}%
\pgfpathlineto{\pgfqpoint{4.573538in}{1.945117in}}%
\pgfpathmoveto{\pgfqpoint{4.573538in}{1.939218in}}%
\pgfpathlineto{\pgfqpoint{4.573538in}{1.939218in}}%
\pgfpathlineto{\pgfqpoint{4.573538in}{1.942168in}}%
\pgfpathlineto{\pgfqpoint{4.578079in}{1.942168in}}%
\pgfpathlineto{\pgfqpoint{4.578079in}{1.939218in}}%
\pgfpathmoveto{\pgfqpoint{4.573538in}{1.942168in}}%
\pgfpathlineto{\pgfqpoint{4.573538in}{1.942168in}}%
\pgfpathlineto{\pgfqpoint{4.573538in}{1.945117in}}%
\pgfpathlineto{\pgfqpoint{4.578079in}{1.945117in}}%
\pgfpathlineto{\pgfqpoint{4.578079in}{1.942168in}}%
\pgfpathmoveto{\pgfqpoint{4.578079in}{1.939218in}}%
\pgfpathlineto{\pgfqpoint{4.578079in}{1.939218in}}%
\pgfpathlineto{\pgfqpoint{4.578079in}{1.942168in}}%
\pgfpathlineto{\pgfqpoint{4.582620in}{1.942168in}}%
\pgfpathlineto{\pgfqpoint{4.582620in}{1.939218in}}%
\pgfpathmoveto{\pgfqpoint{4.532669in}{1.965762in}}%
\pgfpathlineto{\pgfqpoint{4.532669in}{1.965762in}}%
\pgfpathlineto{\pgfqpoint{4.532669in}{1.968711in}}%
\pgfpathlineto{\pgfqpoint{4.537210in}{1.968711in}}%
\pgfpathlineto{\pgfqpoint{4.537210in}{1.965762in}}%
\pgfpathmoveto{\pgfqpoint{4.528128in}{1.968711in}}%
\pgfpathlineto{\pgfqpoint{4.528128in}{1.968711in}}%
\pgfpathlineto{\pgfqpoint{4.528128in}{1.971661in}}%
\pgfpathlineto{\pgfqpoint{4.532669in}{1.971661in}}%
\pgfpathlineto{\pgfqpoint{4.532669in}{1.968711in}}%
\pgfpathmoveto{\pgfqpoint{4.528128in}{1.971661in}}%
\pgfpathlineto{\pgfqpoint{4.528128in}{1.971661in}}%
\pgfpathlineto{\pgfqpoint{4.528128in}{1.974610in}}%
\pgfpathlineto{\pgfqpoint{4.532669in}{1.974610in}}%
\pgfpathlineto{\pgfqpoint{4.532669in}{1.971661in}}%
\pgfpathmoveto{\pgfqpoint{4.532669in}{1.968711in}}%
\pgfpathlineto{\pgfqpoint{4.532669in}{1.968711in}}%
\pgfpathlineto{\pgfqpoint{4.532669in}{1.971661in}}%
\pgfpathlineto{\pgfqpoint{4.537210in}{1.971661in}}%
\pgfpathlineto{\pgfqpoint{4.537210in}{1.968711in}}%
\pgfpathmoveto{\pgfqpoint{4.537210in}{1.962813in}}%
\pgfpathlineto{\pgfqpoint{4.537210in}{1.962813in}}%
\pgfpathlineto{\pgfqpoint{4.537210in}{1.965762in}}%
\pgfpathlineto{\pgfqpoint{4.541751in}{1.965762in}}%
\pgfpathlineto{\pgfqpoint{4.541751in}{1.962813in}}%
\pgfpathmoveto{\pgfqpoint{4.537210in}{1.965762in}}%
\pgfpathlineto{\pgfqpoint{4.537210in}{1.965762in}}%
\pgfpathlineto{\pgfqpoint{4.537210in}{1.968711in}}%
\pgfpathlineto{\pgfqpoint{4.541751in}{1.968711in}}%
\pgfpathlineto{\pgfqpoint{4.541751in}{1.965762in}}%
\pgfpathmoveto{\pgfqpoint{4.541751in}{1.962813in}}%
\pgfpathlineto{\pgfqpoint{4.541751in}{1.962813in}}%
\pgfpathlineto{\pgfqpoint{4.541751in}{1.965762in}}%
\pgfpathlineto{\pgfqpoint{4.546292in}{1.965762in}}%
\pgfpathlineto{\pgfqpoint{4.546292in}{1.962813in}}%
\pgfpathmoveto{\pgfqpoint{4.605326in}{1.918573in}}%
\pgfpathlineto{\pgfqpoint{4.605326in}{1.918573in}}%
\pgfpathlineto{\pgfqpoint{4.605326in}{1.921522in}}%
\pgfpathlineto{\pgfqpoint{4.609867in}{1.921522in}}%
\pgfpathlineto{\pgfqpoint{4.609867in}{1.918573in}}%
\pgfpathmoveto{\pgfqpoint{4.600785in}{1.921522in}}%
\pgfpathlineto{\pgfqpoint{4.600785in}{1.921522in}}%
\pgfpathlineto{\pgfqpoint{4.600785in}{1.924472in}}%
\pgfpathlineto{\pgfqpoint{4.605326in}{1.924472in}}%
\pgfpathlineto{\pgfqpoint{4.605326in}{1.921522in}}%
\pgfpathmoveto{\pgfqpoint{4.600785in}{1.924472in}}%
\pgfpathlineto{\pgfqpoint{4.600785in}{1.924472in}}%
\pgfpathlineto{\pgfqpoint{4.600785in}{1.927421in}}%
\pgfpathlineto{\pgfqpoint{4.605326in}{1.927421in}}%
\pgfpathlineto{\pgfqpoint{4.605326in}{1.924472in}}%
\pgfpathmoveto{\pgfqpoint{4.605326in}{1.921522in}}%
\pgfpathlineto{\pgfqpoint{4.605326in}{1.921522in}}%
\pgfpathlineto{\pgfqpoint{4.605326in}{1.924472in}}%
\pgfpathlineto{\pgfqpoint{4.609867in}{1.924472in}}%
\pgfpathlineto{\pgfqpoint{4.609867in}{1.921522in}}%
\pgfpathmoveto{\pgfqpoint{4.609867in}{1.915624in}}%
\pgfpathlineto{\pgfqpoint{4.609867in}{1.915624in}}%
\pgfpathlineto{\pgfqpoint{4.609867in}{1.918573in}}%
\pgfpathlineto{\pgfqpoint{4.614408in}{1.918573in}}%
\pgfpathlineto{\pgfqpoint{4.614408in}{1.915624in}}%
\pgfpathmoveto{\pgfqpoint{4.609867in}{1.918573in}}%
\pgfpathlineto{\pgfqpoint{4.609867in}{1.918573in}}%
\pgfpathlineto{\pgfqpoint{4.609867in}{1.921522in}}%
\pgfpathlineto{\pgfqpoint{4.614408in}{1.921522in}}%
\pgfpathlineto{\pgfqpoint{4.614408in}{1.918573in}}%
\pgfpathmoveto{\pgfqpoint{4.614408in}{1.915624in}}%
\pgfpathlineto{\pgfqpoint{4.614408in}{1.915624in}}%
\pgfpathlineto{\pgfqpoint{4.614408in}{1.918573in}}%
\pgfpathlineto{\pgfqpoint{4.618949in}{1.918573in}}%
\pgfpathlineto{\pgfqpoint{4.618949in}{1.915624in}}%
\pgfpathmoveto{\pgfqpoint{4.759717in}{1.818301in}}%
\pgfpathlineto{\pgfqpoint{4.759717in}{1.818301in}}%
\pgfpathlineto{\pgfqpoint{4.759717in}{1.821251in}}%
\pgfpathlineto{\pgfqpoint{4.764258in}{1.821251in}}%
\pgfpathlineto{\pgfqpoint{4.764258in}{1.818301in}}%
\pgfpathmoveto{\pgfqpoint{4.777880in}{1.806505in}}%
\pgfpathlineto{\pgfqpoint{4.777880in}{1.806505in}}%
\pgfpathlineto{\pgfqpoint{4.777880in}{1.809454in}}%
\pgfpathlineto{\pgfqpoint{4.782421in}{1.809454in}}%
\pgfpathlineto{\pgfqpoint{4.782421in}{1.806505in}}%
\pgfpathmoveto{\pgfqpoint{4.768798in}{1.812403in}}%
\pgfpathlineto{\pgfqpoint{4.768798in}{1.812403in}}%
\pgfpathlineto{\pgfqpoint{4.768798in}{1.815352in}}%
\pgfpathlineto{\pgfqpoint{4.773339in}{1.815352in}}%
\pgfpathlineto{\pgfqpoint{4.773339in}{1.812403in}}%
\pgfpathmoveto{\pgfqpoint{4.764258in}{1.815352in}}%
\pgfpathlineto{\pgfqpoint{4.764258in}{1.815352in}}%
\pgfpathlineto{\pgfqpoint{4.764258in}{1.818301in}}%
\pgfpathlineto{\pgfqpoint{4.768798in}{1.818301in}}%
\pgfpathlineto{\pgfqpoint{4.768798in}{1.815352in}}%
\pgfpathmoveto{\pgfqpoint{4.764258in}{1.818301in}}%
\pgfpathlineto{\pgfqpoint{4.764258in}{1.818301in}}%
\pgfpathlineto{\pgfqpoint{4.764258in}{1.821251in}}%
\pgfpathlineto{\pgfqpoint{4.768798in}{1.821251in}}%
\pgfpathlineto{\pgfqpoint{4.768798in}{1.818301in}}%
\pgfpathmoveto{\pgfqpoint{4.768798in}{1.815352in}}%
\pgfpathlineto{\pgfqpoint{4.768798in}{1.815352in}}%
\pgfpathlineto{\pgfqpoint{4.768798in}{1.818301in}}%
\pgfpathlineto{\pgfqpoint{4.773339in}{1.818301in}}%
\pgfpathlineto{\pgfqpoint{4.773339in}{1.815352in}}%
\pgfpathmoveto{\pgfqpoint{4.773339in}{1.809454in}}%
\pgfpathlineto{\pgfqpoint{4.773339in}{1.809454in}}%
\pgfpathlineto{\pgfqpoint{4.773339in}{1.812403in}}%
\pgfpathlineto{\pgfqpoint{4.777880in}{1.812403in}}%
\pgfpathlineto{\pgfqpoint{4.777880in}{1.809454in}}%
\pgfpathmoveto{\pgfqpoint{4.773339in}{1.812403in}}%
\pgfpathlineto{\pgfqpoint{4.773339in}{1.812403in}}%
\pgfpathlineto{\pgfqpoint{4.773339in}{1.815352in}}%
\pgfpathlineto{\pgfqpoint{4.777880in}{1.815352in}}%
\pgfpathlineto{\pgfqpoint{4.777880in}{1.812403in}}%
\pgfpathmoveto{\pgfqpoint{4.777880in}{1.809454in}}%
\pgfpathlineto{\pgfqpoint{4.777880in}{1.809454in}}%
\pgfpathlineto{\pgfqpoint{4.777880in}{1.812403in}}%
\pgfpathlineto{\pgfqpoint{4.782421in}{1.812403in}}%
\pgfpathlineto{\pgfqpoint{4.782421in}{1.809454in}}%
\pgfpathmoveto{\pgfqpoint{4.796043in}{1.794708in}}%
\pgfpathlineto{\pgfqpoint{4.796043in}{1.794708in}}%
\pgfpathlineto{\pgfqpoint{4.796043in}{1.797657in}}%
\pgfpathlineto{\pgfqpoint{4.800584in}{1.797657in}}%
\pgfpathlineto{\pgfqpoint{4.800584in}{1.794708in}}%
\pgfpathmoveto{\pgfqpoint{4.814207in}{1.782911in}}%
\pgfpathlineto{\pgfqpoint{4.814207in}{1.782911in}}%
\pgfpathlineto{\pgfqpoint{4.814207in}{1.785860in}}%
\pgfpathlineto{\pgfqpoint{4.818747in}{1.785860in}}%
\pgfpathlineto{\pgfqpoint{4.818747in}{1.782911in}}%
\pgfpathmoveto{\pgfqpoint{4.805125in}{1.788810in}}%
\pgfpathlineto{\pgfqpoint{4.805125in}{1.788810in}}%
\pgfpathlineto{\pgfqpoint{4.805125in}{1.791759in}}%
\pgfpathlineto{\pgfqpoint{4.809666in}{1.791759in}}%
\pgfpathlineto{\pgfqpoint{4.809666in}{1.788810in}}%
\pgfpathmoveto{\pgfqpoint{4.800584in}{1.791759in}}%
\pgfpathlineto{\pgfqpoint{4.800584in}{1.791759in}}%
\pgfpathlineto{\pgfqpoint{4.800584in}{1.794708in}}%
\pgfpathlineto{\pgfqpoint{4.805125in}{1.794708in}}%
\pgfpathlineto{\pgfqpoint{4.805125in}{1.791759in}}%
\pgfpathmoveto{\pgfqpoint{4.800584in}{1.794708in}}%
\pgfpathlineto{\pgfqpoint{4.800584in}{1.794708in}}%
\pgfpathlineto{\pgfqpoint{4.800584in}{1.797657in}}%
\pgfpathlineto{\pgfqpoint{4.805125in}{1.797657in}}%
\pgfpathlineto{\pgfqpoint{4.805125in}{1.794708in}}%
\pgfpathmoveto{\pgfqpoint{4.805125in}{1.791759in}}%
\pgfpathlineto{\pgfqpoint{4.805125in}{1.791759in}}%
\pgfpathlineto{\pgfqpoint{4.805125in}{1.794708in}}%
\pgfpathlineto{\pgfqpoint{4.809666in}{1.794708in}}%
\pgfpathlineto{\pgfqpoint{4.809666in}{1.791759in}}%
\pgfpathmoveto{\pgfqpoint{4.809666in}{1.785860in}}%
\pgfpathlineto{\pgfqpoint{4.809666in}{1.785860in}}%
\pgfpathlineto{\pgfqpoint{4.809666in}{1.788810in}}%
\pgfpathlineto{\pgfqpoint{4.814207in}{1.788810in}}%
\pgfpathlineto{\pgfqpoint{4.814207in}{1.785860in}}%
\pgfpathmoveto{\pgfqpoint{4.809666in}{1.788810in}}%
\pgfpathlineto{\pgfqpoint{4.809666in}{1.788810in}}%
\pgfpathlineto{\pgfqpoint{4.809666in}{1.791759in}}%
\pgfpathlineto{\pgfqpoint{4.814207in}{1.791759in}}%
\pgfpathlineto{\pgfqpoint{4.814207in}{1.788810in}}%
\pgfpathmoveto{\pgfqpoint{4.814207in}{1.785860in}}%
\pgfpathlineto{\pgfqpoint{4.814207in}{1.785860in}}%
\pgfpathlineto{\pgfqpoint{4.814207in}{1.788810in}}%
\pgfpathlineto{\pgfqpoint{4.818747in}{1.788810in}}%
\pgfpathlineto{\pgfqpoint{4.818747in}{1.785860in}}%
\pgfpathmoveto{\pgfqpoint{4.786962in}{1.800606in}}%
\pgfpathlineto{\pgfqpoint{4.786962in}{1.800606in}}%
\pgfpathlineto{\pgfqpoint{4.786962in}{1.803555in}}%
\pgfpathlineto{\pgfqpoint{4.791503in}{1.803555in}}%
\pgfpathlineto{\pgfqpoint{4.791503in}{1.800606in}}%
\pgfpathmoveto{\pgfqpoint{4.782421in}{1.803555in}}%
\pgfpathlineto{\pgfqpoint{4.782421in}{1.803555in}}%
\pgfpathlineto{\pgfqpoint{4.782421in}{1.806505in}}%
\pgfpathlineto{\pgfqpoint{4.786962in}{1.806505in}}%
\pgfpathlineto{\pgfqpoint{4.786962in}{1.803555in}}%
\pgfpathmoveto{\pgfqpoint{4.782421in}{1.806505in}}%
\pgfpathlineto{\pgfqpoint{4.782421in}{1.806505in}}%
\pgfpathlineto{\pgfqpoint{4.782421in}{1.809454in}}%
\pgfpathlineto{\pgfqpoint{4.786962in}{1.809454in}}%
\pgfpathlineto{\pgfqpoint{4.786962in}{1.806505in}}%
\pgfpathmoveto{\pgfqpoint{4.786962in}{1.803555in}}%
\pgfpathlineto{\pgfqpoint{4.786962in}{1.803555in}}%
\pgfpathlineto{\pgfqpoint{4.786962in}{1.806505in}}%
\pgfpathlineto{\pgfqpoint{4.791503in}{1.806505in}}%
\pgfpathlineto{\pgfqpoint{4.791503in}{1.803555in}}%
\pgfpathmoveto{\pgfqpoint{4.791503in}{1.797657in}}%
\pgfpathlineto{\pgfqpoint{4.791503in}{1.797657in}}%
\pgfpathlineto{\pgfqpoint{4.791503in}{1.800606in}}%
\pgfpathlineto{\pgfqpoint{4.796043in}{1.800606in}}%
\pgfpathlineto{\pgfqpoint{4.796043in}{1.797657in}}%
\pgfpathmoveto{\pgfqpoint{4.791503in}{1.800606in}}%
\pgfpathlineto{\pgfqpoint{4.791503in}{1.800606in}}%
\pgfpathlineto{\pgfqpoint{4.791503in}{1.803555in}}%
\pgfpathlineto{\pgfqpoint{4.796043in}{1.803555in}}%
\pgfpathlineto{\pgfqpoint{4.796043in}{1.800606in}}%
\pgfpathmoveto{\pgfqpoint{4.796043in}{1.797657in}}%
\pgfpathlineto{\pgfqpoint{4.796043in}{1.797657in}}%
\pgfpathlineto{\pgfqpoint{4.796043in}{1.800606in}}%
\pgfpathlineto{\pgfqpoint{4.800584in}{1.800606in}}%
\pgfpathlineto{\pgfqpoint{4.800584in}{1.797657in}}%
\pgfpathmoveto{\pgfqpoint{4.687064in}{1.865488in}}%
\pgfpathlineto{\pgfqpoint{4.687064in}{1.865488in}}%
\pgfpathlineto{\pgfqpoint{4.687064in}{1.868437in}}%
\pgfpathlineto{\pgfqpoint{4.691605in}{1.868437in}}%
\pgfpathlineto{\pgfqpoint{4.691605in}{1.865488in}}%
\pgfpathmoveto{\pgfqpoint{4.705227in}{1.853691in}}%
\pgfpathlineto{\pgfqpoint{4.705227in}{1.853691in}}%
\pgfpathlineto{\pgfqpoint{4.705227in}{1.856640in}}%
\pgfpathlineto{\pgfqpoint{4.709768in}{1.856640in}}%
\pgfpathlineto{\pgfqpoint{4.709768in}{1.853691in}}%
\pgfpathmoveto{\pgfqpoint{4.696145in}{1.859590in}}%
\pgfpathlineto{\pgfqpoint{4.696145in}{1.859590in}}%
\pgfpathlineto{\pgfqpoint{4.696145in}{1.862539in}}%
\pgfpathlineto{\pgfqpoint{4.700686in}{1.862539in}}%
\pgfpathlineto{\pgfqpoint{4.700686in}{1.859590in}}%
\pgfpathmoveto{\pgfqpoint{4.691605in}{1.862539in}}%
\pgfpathlineto{\pgfqpoint{4.691605in}{1.862539in}}%
\pgfpathlineto{\pgfqpoint{4.691605in}{1.865488in}}%
\pgfpathlineto{\pgfqpoint{4.696145in}{1.865488in}}%
\pgfpathlineto{\pgfqpoint{4.696145in}{1.862539in}}%
\pgfpathmoveto{\pgfqpoint{4.691605in}{1.865488in}}%
\pgfpathlineto{\pgfqpoint{4.691605in}{1.865488in}}%
\pgfpathlineto{\pgfqpoint{4.691605in}{1.868437in}}%
\pgfpathlineto{\pgfqpoint{4.696145in}{1.868437in}}%
\pgfpathlineto{\pgfqpoint{4.696145in}{1.865488in}}%
\pgfpathmoveto{\pgfqpoint{4.696145in}{1.862539in}}%
\pgfpathlineto{\pgfqpoint{4.696145in}{1.862539in}}%
\pgfpathlineto{\pgfqpoint{4.696145in}{1.865488in}}%
\pgfpathlineto{\pgfqpoint{4.700686in}{1.865488in}}%
\pgfpathlineto{\pgfqpoint{4.700686in}{1.862539in}}%
\pgfpathmoveto{\pgfqpoint{4.700686in}{1.856640in}}%
\pgfpathlineto{\pgfqpoint{4.700686in}{1.856640in}}%
\pgfpathlineto{\pgfqpoint{4.700686in}{1.859590in}}%
\pgfpathlineto{\pgfqpoint{4.705227in}{1.859590in}}%
\pgfpathlineto{\pgfqpoint{4.705227in}{1.856640in}}%
\pgfpathmoveto{\pgfqpoint{4.700686in}{1.859590in}}%
\pgfpathlineto{\pgfqpoint{4.700686in}{1.859590in}}%
\pgfpathlineto{\pgfqpoint{4.700686in}{1.862539in}}%
\pgfpathlineto{\pgfqpoint{4.705227in}{1.862539in}}%
\pgfpathlineto{\pgfqpoint{4.705227in}{1.859590in}}%
\pgfpathmoveto{\pgfqpoint{4.705227in}{1.856640in}}%
\pgfpathlineto{\pgfqpoint{4.705227in}{1.856640in}}%
\pgfpathlineto{\pgfqpoint{4.705227in}{1.859590in}}%
\pgfpathlineto{\pgfqpoint{4.709768in}{1.859590in}}%
\pgfpathlineto{\pgfqpoint{4.709768in}{1.856640in}}%
\pgfpathmoveto{\pgfqpoint{4.723390in}{1.841895in}}%
\pgfpathlineto{\pgfqpoint{4.723390in}{1.841895in}}%
\pgfpathlineto{\pgfqpoint{4.723390in}{1.844844in}}%
\pgfpathlineto{\pgfqpoint{4.727931in}{1.844844in}}%
\pgfpathlineto{\pgfqpoint{4.727931in}{1.841895in}}%
\pgfpathmoveto{\pgfqpoint{4.741554in}{1.830098in}}%
\pgfpathlineto{\pgfqpoint{4.741554in}{1.830098in}}%
\pgfpathlineto{\pgfqpoint{4.741554in}{1.833047in}}%
\pgfpathlineto{\pgfqpoint{4.746094in}{1.833047in}}%
\pgfpathlineto{\pgfqpoint{4.746094in}{1.830098in}}%
\pgfpathmoveto{\pgfqpoint{4.732472in}{1.835996in}}%
\pgfpathlineto{\pgfqpoint{4.732472in}{1.835996in}}%
\pgfpathlineto{\pgfqpoint{4.732472in}{1.838946in}}%
\pgfpathlineto{\pgfqpoint{4.737013in}{1.838946in}}%
\pgfpathlineto{\pgfqpoint{4.737013in}{1.835996in}}%
\pgfpathmoveto{\pgfqpoint{4.727931in}{1.838946in}}%
\pgfpathlineto{\pgfqpoint{4.727931in}{1.838946in}}%
\pgfpathlineto{\pgfqpoint{4.727931in}{1.841895in}}%
\pgfpathlineto{\pgfqpoint{4.732472in}{1.841895in}}%
\pgfpathlineto{\pgfqpoint{4.732472in}{1.838946in}}%
\pgfpathmoveto{\pgfqpoint{4.727931in}{1.841895in}}%
\pgfpathlineto{\pgfqpoint{4.727931in}{1.841895in}}%
\pgfpathlineto{\pgfqpoint{4.727931in}{1.844844in}}%
\pgfpathlineto{\pgfqpoint{4.732472in}{1.844844in}}%
\pgfpathlineto{\pgfqpoint{4.732472in}{1.841895in}}%
\pgfpathmoveto{\pgfqpoint{4.732472in}{1.838946in}}%
\pgfpathlineto{\pgfqpoint{4.732472in}{1.838946in}}%
\pgfpathlineto{\pgfqpoint{4.732472in}{1.841895in}}%
\pgfpathlineto{\pgfqpoint{4.737013in}{1.841895in}}%
\pgfpathlineto{\pgfqpoint{4.737013in}{1.838946in}}%
\pgfpathmoveto{\pgfqpoint{4.737013in}{1.833047in}}%
\pgfpathlineto{\pgfqpoint{4.737013in}{1.833047in}}%
\pgfpathlineto{\pgfqpoint{4.737013in}{1.835996in}}%
\pgfpathlineto{\pgfqpoint{4.741554in}{1.835996in}}%
\pgfpathlineto{\pgfqpoint{4.741554in}{1.833047in}}%
\pgfpathmoveto{\pgfqpoint{4.737013in}{1.835996in}}%
\pgfpathlineto{\pgfqpoint{4.737013in}{1.835996in}}%
\pgfpathlineto{\pgfqpoint{4.737013in}{1.838946in}}%
\pgfpathlineto{\pgfqpoint{4.741554in}{1.838946in}}%
\pgfpathlineto{\pgfqpoint{4.741554in}{1.835996in}}%
\pgfpathmoveto{\pgfqpoint{4.741554in}{1.833047in}}%
\pgfpathlineto{\pgfqpoint{4.741554in}{1.833047in}}%
\pgfpathlineto{\pgfqpoint{4.741554in}{1.835996in}}%
\pgfpathlineto{\pgfqpoint{4.746094in}{1.835996in}}%
\pgfpathlineto{\pgfqpoint{4.746094in}{1.833047in}}%
\pgfpathmoveto{\pgfqpoint{4.714309in}{1.847793in}}%
\pgfpathlineto{\pgfqpoint{4.714309in}{1.847793in}}%
\pgfpathlineto{\pgfqpoint{4.714309in}{1.850742in}}%
\pgfpathlineto{\pgfqpoint{4.718850in}{1.850742in}}%
\pgfpathlineto{\pgfqpoint{4.718850in}{1.847793in}}%
\pgfpathmoveto{\pgfqpoint{4.709768in}{1.850742in}}%
\pgfpathlineto{\pgfqpoint{4.709768in}{1.850742in}}%
\pgfpathlineto{\pgfqpoint{4.709768in}{1.853691in}}%
\pgfpathlineto{\pgfqpoint{4.714309in}{1.853691in}}%
\pgfpathlineto{\pgfqpoint{4.714309in}{1.850742in}}%
\pgfpathmoveto{\pgfqpoint{4.709768in}{1.853691in}}%
\pgfpathlineto{\pgfqpoint{4.709768in}{1.853691in}}%
\pgfpathlineto{\pgfqpoint{4.709768in}{1.856640in}}%
\pgfpathlineto{\pgfqpoint{4.714309in}{1.856640in}}%
\pgfpathlineto{\pgfqpoint{4.714309in}{1.853691in}}%
\pgfpathmoveto{\pgfqpoint{4.714309in}{1.850742in}}%
\pgfpathlineto{\pgfqpoint{4.714309in}{1.850742in}}%
\pgfpathlineto{\pgfqpoint{4.714309in}{1.853691in}}%
\pgfpathlineto{\pgfqpoint{4.718850in}{1.853691in}}%
\pgfpathlineto{\pgfqpoint{4.718850in}{1.850742in}}%
\pgfpathmoveto{\pgfqpoint{4.718850in}{1.844844in}}%
\pgfpathlineto{\pgfqpoint{4.718850in}{1.844844in}}%
\pgfpathlineto{\pgfqpoint{4.718850in}{1.847793in}}%
\pgfpathlineto{\pgfqpoint{4.723390in}{1.847793in}}%
\pgfpathlineto{\pgfqpoint{4.723390in}{1.844844in}}%
\pgfpathmoveto{\pgfqpoint{4.718850in}{1.847793in}}%
\pgfpathlineto{\pgfqpoint{4.718850in}{1.847793in}}%
\pgfpathlineto{\pgfqpoint{4.718850in}{1.850742in}}%
\pgfpathlineto{\pgfqpoint{4.723390in}{1.850742in}}%
\pgfpathlineto{\pgfqpoint{4.723390in}{1.847793in}}%
\pgfpathmoveto{\pgfqpoint{4.723390in}{1.844844in}}%
\pgfpathlineto{\pgfqpoint{4.723390in}{1.844844in}}%
\pgfpathlineto{\pgfqpoint{4.723390in}{1.847793in}}%
\pgfpathlineto{\pgfqpoint{4.727931in}{1.847793in}}%
\pgfpathlineto{\pgfqpoint{4.727931in}{1.844844in}}%
\pgfpathmoveto{\pgfqpoint{4.677982in}{1.871386in}}%
\pgfpathlineto{\pgfqpoint{4.677982in}{1.871386in}}%
\pgfpathlineto{\pgfqpoint{4.677982in}{1.874335in}}%
\pgfpathlineto{\pgfqpoint{4.682523in}{1.874335in}}%
\pgfpathlineto{\pgfqpoint{4.682523in}{1.871386in}}%
\pgfpathmoveto{\pgfqpoint{4.673441in}{1.874335in}}%
\pgfpathlineto{\pgfqpoint{4.673441in}{1.874335in}}%
\pgfpathlineto{\pgfqpoint{4.673441in}{1.877285in}}%
\pgfpathlineto{\pgfqpoint{4.677982in}{1.877285in}}%
\pgfpathlineto{\pgfqpoint{4.677982in}{1.874335in}}%
\pgfpathmoveto{\pgfqpoint{4.673441in}{1.877285in}}%
\pgfpathlineto{\pgfqpoint{4.673441in}{1.877285in}}%
\pgfpathlineto{\pgfqpoint{4.673441in}{1.880234in}}%
\pgfpathlineto{\pgfqpoint{4.677982in}{1.880234in}}%
\pgfpathlineto{\pgfqpoint{4.677982in}{1.877285in}}%
\pgfpathmoveto{\pgfqpoint{4.677982in}{1.874335in}}%
\pgfpathlineto{\pgfqpoint{4.677982in}{1.874335in}}%
\pgfpathlineto{\pgfqpoint{4.677982in}{1.877285in}}%
\pgfpathlineto{\pgfqpoint{4.682523in}{1.877285in}}%
\pgfpathlineto{\pgfqpoint{4.682523in}{1.874335in}}%
\pgfpathmoveto{\pgfqpoint{4.682523in}{1.868437in}}%
\pgfpathlineto{\pgfqpoint{4.682523in}{1.868437in}}%
\pgfpathlineto{\pgfqpoint{4.682523in}{1.871386in}}%
\pgfpathlineto{\pgfqpoint{4.687064in}{1.871386in}}%
\pgfpathlineto{\pgfqpoint{4.687064in}{1.868437in}}%
\pgfpathmoveto{\pgfqpoint{4.682523in}{1.871386in}}%
\pgfpathlineto{\pgfqpoint{4.682523in}{1.871386in}}%
\pgfpathlineto{\pgfqpoint{4.682523in}{1.874335in}}%
\pgfpathlineto{\pgfqpoint{4.687064in}{1.874335in}}%
\pgfpathlineto{\pgfqpoint{4.687064in}{1.871386in}}%
\pgfpathmoveto{\pgfqpoint{4.687064in}{1.868437in}}%
\pgfpathlineto{\pgfqpoint{4.687064in}{1.868437in}}%
\pgfpathlineto{\pgfqpoint{4.687064in}{1.871386in}}%
\pgfpathlineto{\pgfqpoint{4.691605in}{1.871386in}}%
\pgfpathlineto{\pgfqpoint{4.691605in}{1.868437in}}%
\pgfpathmoveto{\pgfqpoint{4.750635in}{1.824200in}}%
\pgfpathlineto{\pgfqpoint{4.750635in}{1.824200in}}%
\pgfpathlineto{\pgfqpoint{4.750635in}{1.827149in}}%
\pgfpathlineto{\pgfqpoint{4.755176in}{1.827149in}}%
\pgfpathlineto{\pgfqpoint{4.755176in}{1.824200in}}%
\pgfpathmoveto{\pgfqpoint{4.746094in}{1.827149in}}%
\pgfpathlineto{\pgfqpoint{4.746094in}{1.827149in}}%
\pgfpathlineto{\pgfqpoint{4.746094in}{1.830098in}}%
\pgfpathlineto{\pgfqpoint{4.750635in}{1.830098in}}%
\pgfpathlineto{\pgfqpoint{4.750635in}{1.827149in}}%
\pgfpathmoveto{\pgfqpoint{4.746094in}{1.830098in}}%
\pgfpathlineto{\pgfqpoint{4.746094in}{1.830098in}}%
\pgfpathlineto{\pgfqpoint{4.746094in}{1.833047in}}%
\pgfpathlineto{\pgfqpoint{4.750635in}{1.833047in}}%
\pgfpathlineto{\pgfqpoint{4.750635in}{1.830098in}}%
\pgfpathmoveto{\pgfqpoint{4.750635in}{1.827149in}}%
\pgfpathlineto{\pgfqpoint{4.750635in}{1.827149in}}%
\pgfpathlineto{\pgfqpoint{4.750635in}{1.830098in}}%
\pgfpathlineto{\pgfqpoint{4.755176in}{1.830098in}}%
\pgfpathlineto{\pgfqpoint{4.755176in}{1.827149in}}%
\pgfpathmoveto{\pgfqpoint{4.755176in}{1.821251in}}%
\pgfpathlineto{\pgfqpoint{4.755176in}{1.821251in}}%
\pgfpathlineto{\pgfqpoint{4.755176in}{1.824200in}}%
\pgfpathlineto{\pgfqpoint{4.759717in}{1.824200in}}%
\pgfpathlineto{\pgfqpoint{4.759717in}{1.821251in}}%
\pgfpathmoveto{\pgfqpoint{4.755176in}{1.824200in}}%
\pgfpathlineto{\pgfqpoint{4.755176in}{1.824200in}}%
\pgfpathlineto{\pgfqpoint{4.755176in}{1.827149in}}%
\pgfpathlineto{\pgfqpoint{4.759717in}{1.827149in}}%
\pgfpathlineto{\pgfqpoint{4.759717in}{1.824200in}}%
\pgfpathmoveto{\pgfqpoint{4.759717in}{1.821251in}}%
\pgfpathlineto{\pgfqpoint{4.759717in}{1.821251in}}%
\pgfpathlineto{\pgfqpoint{4.759717in}{1.824200in}}%
\pgfpathlineto{\pgfqpoint{4.764258in}{1.824200in}}%
\pgfpathlineto{\pgfqpoint{4.764258in}{1.821251in}}%
\pgfpathmoveto{\pgfqpoint{4.905028in}{1.723927in}}%
\pgfpathlineto{\pgfqpoint{4.905028in}{1.723927in}}%
\pgfpathlineto{\pgfqpoint{4.905028in}{1.726877in}}%
\pgfpathlineto{\pgfqpoint{4.909569in}{1.726877in}}%
\pgfpathlineto{\pgfqpoint{4.909569in}{1.723927in}}%
\pgfpathmoveto{\pgfqpoint{4.923192in}{1.712130in}}%
\pgfpathlineto{\pgfqpoint{4.923192in}{1.712130in}}%
\pgfpathlineto{\pgfqpoint{4.923192in}{1.715080in}}%
\pgfpathlineto{\pgfqpoint{4.927733in}{1.715080in}}%
\pgfpathlineto{\pgfqpoint{4.927733in}{1.712130in}}%
\pgfpathmoveto{\pgfqpoint{4.914110in}{1.718029in}}%
\pgfpathlineto{\pgfqpoint{4.914110in}{1.718029in}}%
\pgfpathlineto{\pgfqpoint{4.914110in}{1.720978in}}%
\pgfpathlineto{\pgfqpoint{4.918651in}{1.720978in}}%
\pgfpathlineto{\pgfqpoint{4.918651in}{1.718029in}}%
\pgfpathmoveto{\pgfqpoint{4.909569in}{1.720978in}}%
\pgfpathlineto{\pgfqpoint{4.909569in}{1.720978in}}%
\pgfpathlineto{\pgfqpoint{4.909569in}{1.723927in}}%
\pgfpathlineto{\pgfqpoint{4.914110in}{1.723927in}}%
\pgfpathlineto{\pgfqpoint{4.914110in}{1.720978in}}%
\pgfpathmoveto{\pgfqpoint{4.909569in}{1.723927in}}%
\pgfpathlineto{\pgfqpoint{4.909569in}{1.723927in}}%
\pgfpathlineto{\pgfqpoint{4.909569in}{1.726877in}}%
\pgfpathlineto{\pgfqpoint{4.914110in}{1.726877in}}%
\pgfpathlineto{\pgfqpoint{4.914110in}{1.723927in}}%
\pgfpathmoveto{\pgfqpoint{4.914110in}{1.720978in}}%
\pgfpathlineto{\pgfqpoint{4.914110in}{1.720978in}}%
\pgfpathlineto{\pgfqpoint{4.914110in}{1.723927in}}%
\pgfpathlineto{\pgfqpoint{4.918651in}{1.723927in}}%
\pgfpathlineto{\pgfqpoint{4.918651in}{1.720978in}}%
\pgfpathmoveto{\pgfqpoint{4.918651in}{1.715080in}}%
\pgfpathlineto{\pgfqpoint{4.918651in}{1.715080in}}%
\pgfpathlineto{\pgfqpoint{4.918651in}{1.718029in}}%
\pgfpathlineto{\pgfqpoint{4.923192in}{1.718029in}}%
\pgfpathlineto{\pgfqpoint{4.923192in}{1.715080in}}%
\pgfpathmoveto{\pgfqpoint{4.918651in}{1.718029in}}%
\pgfpathlineto{\pgfqpoint{4.918651in}{1.718029in}}%
\pgfpathlineto{\pgfqpoint{4.918651in}{1.720978in}}%
\pgfpathlineto{\pgfqpoint{4.923192in}{1.720978in}}%
\pgfpathlineto{\pgfqpoint{4.923192in}{1.718029in}}%
\pgfpathmoveto{\pgfqpoint{4.923192in}{1.715080in}}%
\pgfpathlineto{\pgfqpoint{4.923192in}{1.715080in}}%
\pgfpathlineto{\pgfqpoint{4.923192in}{1.718029in}}%
\pgfpathlineto{\pgfqpoint{4.927733in}{1.718029in}}%
\pgfpathlineto{\pgfqpoint{4.927733in}{1.715080in}}%
\pgfpathmoveto{\pgfqpoint{4.941356in}{1.700333in}}%
\pgfpathlineto{\pgfqpoint{4.941356in}{1.700333in}}%
\pgfpathlineto{\pgfqpoint{4.941356in}{1.703282in}}%
\pgfpathlineto{\pgfqpoint{4.945898in}{1.703282in}}%
\pgfpathlineto{\pgfqpoint{4.945898in}{1.700333in}}%
\pgfpathmoveto{\pgfqpoint{4.959521in}{1.688536in}}%
\pgfpathlineto{\pgfqpoint{4.959521in}{1.688536in}}%
\pgfpathlineto{\pgfqpoint{4.959521in}{1.691485in}}%
\pgfpathlineto{\pgfqpoint{4.964062in}{1.691485in}}%
\pgfpathlineto{\pgfqpoint{4.964062in}{1.688536in}}%
\pgfpathmoveto{\pgfqpoint{4.950439in}{1.694434in}}%
\pgfpathlineto{\pgfqpoint{4.950439in}{1.694434in}}%
\pgfpathlineto{\pgfqpoint{4.950439in}{1.697384in}}%
\pgfpathlineto{\pgfqpoint{4.954980in}{1.697384in}}%
\pgfpathlineto{\pgfqpoint{4.954980in}{1.694434in}}%
\pgfpathmoveto{\pgfqpoint{4.945898in}{1.697384in}}%
\pgfpathlineto{\pgfqpoint{4.945898in}{1.697384in}}%
\pgfpathlineto{\pgfqpoint{4.945898in}{1.700333in}}%
\pgfpathlineto{\pgfqpoint{4.950439in}{1.700333in}}%
\pgfpathlineto{\pgfqpoint{4.950439in}{1.697384in}}%
\pgfpathmoveto{\pgfqpoint{4.945898in}{1.700333in}}%
\pgfpathlineto{\pgfqpoint{4.945898in}{1.700333in}}%
\pgfpathlineto{\pgfqpoint{4.945898in}{1.703282in}}%
\pgfpathlineto{\pgfqpoint{4.950439in}{1.703282in}}%
\pgfpathlineto{\pgfqpoint{4.950439in}{1.700333in}}%
\pgfpathmoveto{\pgfqpoint{4.950439in}{1.697384in}}%
\pgfpathlineto{\pgfqpoint{4.950439in}{1.697384in}}%
\pgfpathlineto{\pgfqpoint{4.950439in}{1.700333in}}%
\pgfpathlineto{\pgfqpoint{4.954980in}{1.700333in}}%
\pgfpathlineto{\pgfqpoint{4.954980in}{1.697384in}}%
\pgfpathmoveto{\pgfqpoint{4.954980in}{1.691485in}}%
\pgfpathlineto{\pgfqpoint{4.954980in}{1.691485in}}%
\pgfpathlineto{\pgfqpoint{4.954980in}{1.694434in}}%
\pgfpathlineto{\pgfqpoint{4.959521in}{1.694434in}}%
\pgfpathlineto{\pgfqpoint{4.959521in}{1.691485in}}%
\pgfpathmoveto{\pgfqpoint{4.954980in}{1.694434in}}%
\pgfpathlineto{\pgfqpoint{4.954980in}{1.694434in}}%
\pgfpathlineto{\pgfqpoint{4.954980in}{1.697384in}}%
\pgfpathlineto{\pgfqpoint{4.959521in}{1.697384in}}%
\pgfpathlineto{\pgfqpoint{4.959521in}{1.694434in}}%
\pgfpathmoveto{\pgfqpoint{4.959521in}{1.691485in}}%
\pgfpathlineto{\pgfqpoint{4.959521in}{1.691485in}}%
\pgfpathlineto{\pgfqpoint{4.959521in}{1.694434in}}%
\pgfpathlineto{\pgfqpoint{4.964062in}{1.694434in}}%
\pgfpathlineto{\pgfqpoint{4.964062in}{1.691485in}}%
\pgfpathmoveto{\pgfqpoint{4.932274in}{1.706232in}}%
\pgfpathlineto{\pgfqpoint{4.932274in}{1.706232in}}%
\pgfpathlineto{\pgfqpoint{4.932274in}{1.709181in}}%
\pgfpathlineto{\pgfqpoint{4.936815in}{1.709181in}}%
\pgfpathlineto{\pgfqpoint{4.936815in}{1.706232in}}%
\pgfpathmoveto{\pgfqpoint{4.927733in}{1.709181in}}%
\pgfpathlineto{\pgfqpoint{4.927733in}{1.709181in}}%
\pgfpathlineto{\pgfqpoint{4.927733in}{1.712130in}}%
\pgfpathlineto{\pgfqpoint{4.932274in}{1.712130in}}%
\pgfpathlineto{\pgfqpoint{4.932274in}{1.709181in}}%
\pgfpathmoveto{\pgfqpoint{4.927733in}{1.712130in}}%
\pgfpathlineto{\pgfqpoint{4.927733in}{1.712130in}}%
\pgfpathlineto{\pgfqpoint{4.927733in}{1.715080in}}%
\pgfpathlineto{\pgfqpoint{4.932274in}{1.715080in}}%
\pgfpathlineto{\pgfqpoint{4.932274in}{1.712130in}}%
\pgfpathmoveto{\pgfqpoint{4.932274in}{1.709181in}}%
\pgfpathlineto{\pgfqpoint{4.932274in}{1.709181in}}%
\pgfpathlineto{\pgfqpoint{4.932274in}{1.712130in}}%
\pgfpathlineto{\pgfqpoint{4.936815in}{1.712130in}}%
\pgfpathlineto{\pgfqpoint{4.936815in}{1.709181in}}%
\pgfpathmoveto{\pgfqpoint{4.936815in}{1.703282in}}%
\pgfpathlineto{\pgfqpoint{4.936815in}{1.703282in}}%
\pgfpathlineto{\pgfqpoint{4.936815in}{1.706232in}}%
\pgfpathlineto{\pgfqpoint{4.941356in}{1.706232in}}%
\pgfpathlineto{\pgfqpoint{4.941356in}{1.703282in}}%
\pgfpathmoveto{\pgfqpoint{4.936815in}{1.706232in}}%
\pgfpathlineto{\pgfqpoint{4.936815in}{1.706232in}}%
\pgfpathlineto{\pgfqpoint{4.936815in}{1.709181in}}%
\pgfpathlineto{\pgfqpoint{4.941356in}{1.709181in}}%
\pgfpathlineto{\pgfqpoint{4.941356in}{1.706232in}}%
\pgfpathmoveto{\pgfqpoint{4.941356in}{1.703282in}}%
\pgfpathlineto{\pgfqpoint{4.941356in}{1.703282in}}%
\pgfpathlineto{\pgfqpoint{4.941356in}{1.706232in}}%
\pgfpathlineto{\pgfqpoint{4.945898in}{1.706232in}}%
\pgfpathlineto{\pgfqpoint{4.945898in}{1.703282in}}%
\pgfpathmoveto{\pgfqpoint{4.832371in}{1.771114in}}%
\pgfpathlineto{\pgfqpoint{4.832371in}{1.771114in}}%
\pgfpathlineto{\pgfqpoint{4.832371in}{1.774064in}}%
\pgfpathlineto{\pgfqpoint{4.836912in}{1.774064in}}%
\pgfpathlineto{\pgfqpoint{4.836912in}{1.771114in}}%
\pgfpathmoveto{\pgfqpoint{4.850535in}{1.759318in}}%
\pgfpathlineto{\pgfqpoint{4.850535in}{1.759318in}}%
\pgfpathlineto{\pgfqpoint{4.850535in}{1.762267in}}%
\pgfpathlineto{\pgfqpoint{4.855076in}{1.762267in}}%
\pgfpathlineto{\pgfqpoint{4.855076in}{1.759318in}}%
\pgfpathmoveto{\pgfqpoint{4.841453in}{1.765216in}}%
\pgfpathlineto{\pgfqpoint{4.841453in}{1.765216in}}%
\pgfpathlineto{\pgfqpoint{4.841453in}{1.768165in}}%
\pgfpathlineto{\pgfqpoint{4.845994in}{1.768165in}}%
\pgfpathlineto{\pgfqpoint{4.845994in}{1.765216in}}%
\pgfpathmoveto{\pgfqpoint{4.836912in}{1.768165in}}%
\pgfpathlineto{\pgfqpoint{4.836912in}{1.768165in}}%
\pgfpathlineto{\pgfqpoint{4.836912in}{1.771114in}}%
\pgfpathlineto{\pgfqpoint{4.841453in}{1.771114in}}%
\pgfpathlineto{\pgfqpoint{4.841453in}{1.768165in}}%
\pgfpathmoveto{\pgfqpoint{4.836912in}{1.771114in}}%
\pgfpathlineto{\pgfqpoint{4.836912in}{1.771114in}}%
\pgfpathlineto{\pgfqpoint{4.836912in}{1.774064in}}%
\pgfpathlineto{\pgfqpoint{4.841453in}{1.774064in}}%
\pgfpathlineto{\pgfqpoint{4.841453in}{1.771114in}}%
\pgfpathmoveto{\pgfqpoint{4.841453in}{1.768165in}}%
\pgfpathlineto{\pgfqpoint{4.841453in}{1.768165in}}%
\pgfpathlineto{\pgfqpoint{4.841453in}{1.771114in}}%
\pgfpathlineto{\pgfqpoint{4.845994in}{1.771114in}}%
\pgfpathlineto{\pgfqpoint{4.845994in}{1.768165in}}%
\pgfpathmoveto{\pgfqpoint{4.845994in}{1.762267in}}%
\pgfpathlineto{\pgfqpoint{4.845994in}{1.762267in}}%
\pgfpathlineto{\pgfqpoint{4.845994in}{1.765216in}}%
\pgfpathlineto{\pgfqpoint{4.850535in}{1.765216in}}%
\pgfpathlineto{\pgfqpoint{4.850535in}{1.762267in}}%
\pgfpathmoveto{\pgfqpoint{4.845994in}{1.765216in}}%
\pgfpathlineto{\pgfqpoint{4.845994in}{1.765216in}}%
\pgfpathlineto{\pgfqpoint{4.845994in}{1.768165in}}%
\pgfpathlineto{\pgfqpoint{4.850535in}{1.768165in}}%
\pgfpathlineto{\pgfqpoint{4.850535in}{1.765216in}}%
\pgfpathmoveto{\pgfqpoint{4.850535in}{1.762267in}}%
\pgfpathlineto{\pgfqpoint{4.850535in}{1.762267in}}%
\pgfpathlineto{\pgfqpoint{4.850535in}{1.765216in}}%
\pgfpathlineto{\pgfqpoint{4.855076in}{1.765216in}}%
\pgfpathlineto{\pgfqpoint{4.855076in}{1.762267in}}%
\pgfpathmoveto{\pgfqpoint{4.868699in}{1.747521in}}%
\pgfpathlineto{\pgfqpoint{4.868699in}{1.747521in}}%
\pgfpathlineto{\pgfqpoint{4.868699in}{1.750470in}}%
\pgfpathlineto{\pgfqpoint{4.873240in}{1.750470in}}%
\pgfpathlineto{\pgfqpoint{4.873240in}{1.747521in}}%
\pgfpathmoveto{\pgfqpoint{4.886864in}{1.735724in}}%
\pgfpathlineto{\pgfqpoint{4.886864in}{1.735724in}}%
\pgfpathlineto{\pgfqpoint{4.886864in}{1.738673in}}%
\pgfpathlineto{\pgfqpoint{4.891405in}{1.738673in}}%
\pgfpathlineto{\pgfqpoint{4.891405in}{1.735724in}}%
\pgfpathmoveto{\pgfqpoint{4.877781in}{1.741623in}}%
\pgfpathlineto{\pgfqpoint{4.877781in}{1.741623in}}%
\pgfpathlineto{\pgfqpoint{4.877781in}{1.744572in}}%
\pgfpathlineto{\pgfqpoint{4.882322in}{1.744572in}}%
\pgfpathlineto{\pgfqpoint{4.882322in}{1.741623in}}%
\pgfpathmoveto{\pgfqpoint{4.873240in}{1.744572in}}%
\pgfpathlineto{\pgfqpoint{4.873240in}{1.744572in}}%
\pgfpathlineto{\pgfqpoint{4.873240in}{1.747521in}}%
\pgfpathlineto{\pgfqpoint{4.877781in}{1.747521in}}%
\pgfpathlineto{\pgfqpoint{4.877781in}{1.744572in}}%
\pgfpathmoveto{\pgfqpoint{4.873240in}{1.747521in}}%
\pgfpathlineto{\pgfqpoint{4.873240in}{1.747521in}}%
\pgfpathlineto{\pgfqpoint{4.873240in}{1.750470in}}%
\pgfpathlineto{\pgfqpoint{4.877781in}{1.750470in}}%
\pgfpathlineto{\pgfqpoint{4.877781in}{1.747521in}}%
\pgfpathmoveto{\pgfqpoint{4.877781in}{1.744572in}}%
\pgfpathlineto{\pgfqpoint{4.877781in}{1.744572in}}%
\pgfpathlineto{\pgfqpoint{4.877781in}{1.747521in}}%
\pgfpathlineto{\pgfqpoint{4.882322in}{1.747521in}}%
\pgfpathlineto{\pgfqpoint{4.882322in}{1.744572in}}%
\pgfpathmoveto{\pgfqpoint{4.882322in}{1.738673in}}%
\pgfpathlineto{\pgfqpoint{4.882322in}{1.738673in}}%
\pgfpathlineto{\pgfqpoint{4.882322in}{1.741623in}}%
\pgfpathlineto{\pgfqpoint{4.886864in}{1.741623in}}%
\pgfpathlineto{\pgfqpoint{4.886864in}{1.738673in}}%
\pgfpathmoveto{\pgfqpoint{4.882322in}{1.741623in}}%
\pgfpathlineto{\pgfqpoint{4.882322in}{1.741623in}}%
\pgfpathlineto{\pgfqpoint{4.882322in}{1.744572in}}%
\pgfpathlineto{\pgfqpoint{4.886864in}{1.744572in}}%
\pgfpathlineto{\pgfqpoint{4.886864in}{1.741623in}}%
\pgfpathmoveto{\pgfqpoint{4.886864in}{1.738673in}}%
\pgfpathlineto{\pgfqpoint{4.886864in}{1.738673in}}%
\pgfpathlineto{\pgfqpoint{4.886864in}{1.741623in}}%
\pgfpathlineto{\pgfqpoint{4.891405in}{1.741623in}}%
\pgfpathlineto{\pgfqpoint{4.891405in}{1.738673in}}%
\pgfpathmoveto{\pgfqpoint{4.859617in}{1.753419in}}%
\pgfpathlineto{\pgfqpoint{4.859617in}{1.753419in}}%
\pgfpathlineto{\pgfqpoint{4.859617in}{1.756369in}}%
\pgfpathlineto{\pgfqpoint{4.864158in}{1.756369in}}%
\pgfpathlineto{\pgfqpoint{4.864158in}{1.753419in}}%
\pgfpathmoveto{\pgfqpoint{4.855076in}{1.756369in}}%
\pgfpathlineto{\pgfqpoint{4.855076in}{1.756369in}}%
\pgfpathlineto{\pgfqpoint{4.855076in}{1.759318in}}%
\pgfpathlineto{\pgfqpoint{4.859617in}{1.759318in}}%
\pgfpathlineto{\pgfqpoint{4.859617in}{1.756369in}}%
\pgfpathmoveto{\pgfqpoint{4.855076in}{1.759318in}}%
\pgfpathlineto{\pgfqpoint{4.855076in}{1.759318in}}%
\pgfpathlineto{\pgfqpoint{4.855076in}{1.762267in}}%
\pgfpathlineto{\pgfqpoint{4.859617in}{1.762267in}}%
\pgfpathlineto{\pgfqpoint{4.859617in}{1.759318in}}%
\pgfpathmoveto{\pgfqpoint{4.859617in}{1.756369in}}%
\pgfpathlineto{\pgfqpoint{4.859617in}{1.756369in}}%
\pgfpathlineto{\pgfqpoint{4.859617in}{1.759318in}}%
\pgfpathlineto{\pgfqpoint{4.864158in}{1.759318in}}%
\pgfpathlineto{\pgfqpoint{4.864158in}{1.756369in}}%
\pgfpathmoveto{\pgfqpoint{4.864158in}{1.750470in}}%
\pgfpathlineto{\pgfqpoint{4.864158in}{1.750470in}}%
\pgfpathlineto{\pgfqpoint{4.864158in}{1.753419in}}%
\pgfpathlineto{\pgfqpoint{4.868699in}{1.753419in}}%
\pgfpathlineto{\pgfqpoint{4.868699in}{1.750470in}}%
\pgfpathmoveto{\pgfqpoint{4.864158in}{1.753419in}}%
\pgfpathlineto{\pgfqpoint{4.864158in}{1.753419in}}%
\pgfpathlineto{\pgfqpoint{4.864158in}{1.756369in}}%
\pgfpathlineto{\pgfqpoint{4.868699in}{1.756369in}}%
\pgfpathlineto{\pgfqpoint{4.868699in}{1.753419in}}%
\pgfpathmoveto{\pgfqpoint{4.868699in}{1.750470in}}%
\pgfpathlineto{\pgfqpoint{4.868699in}{1.750470in}}%
\pgfpathlineto{\pgfqpoint{4.868699in}{1.753419in}}%
\pgfpathlineto{\pgfqpoint{4.873240in}{1.753419in}}%
\pgfpathlineto{\pgfqpoint{4.873240in}{1.750470in}}%
\pgfpathmoveto{\pgfqpoint{4.823289in}{1.777013in}}%
\pgfpathlineto{\pgfqpoint{4.823289in}{1.777013in}}%
\pgfpathlineto{\pgfqpoint{4.823289in}{1.779962in}}%
\pgfpathlineto{\pgfqpoint{4.827830in}{1.779962in}}%
\pgfpathlineto{\pgfqpoint{4.827830in}{1.777013in}}%
\pgfpathmoveto{\pgfqpoint{4.818747in}{1.779962in}}%
\pgfpathlineto{\pgfqpoint{4.818747in}{1.779962in}}%
\pgfpathlineto{\pgfqpoint{4.818747in}{1.782911in}}%
\pgfpathlineto{\pgfqpoint{4.823289in}{1.782911in}}%
\pgfpathlineto{\pgfqpoint{4.823289in}{1.779962in}}%
\pgfpathmoveto{\pgfqpoint{4.818747in}{1.782911in}}%
\pgfpathlineto{\pgfqpoint{4.818747in}{1.782911in}}%
\pgfpathlineto{\pgfqpoint{4.818747in}{1.785860in}}%
\pgfpathlineto{\pgfqpoint{4.823289in}{1.785860in}}%
\pgfpathlineto{\pgfqpoint{4.823289in}{1.782911in}}%
\pgfpathmoveto{\pgfqpoint{4.823289in}{1.779962in}}%
\pgfpathlineto{\pgfqpoint{4.823289in}{1.779962in}}%
\pgfpathlineto{\pgfqpoint{4.823289in}{1.782911in}}%
\pgfpathlineto{\pgfqpoint{4.827830in}{1.782911in}}%
\pgfpathlineto{\pgfqpoint{4.827830in}{1.779962in}}%
\pgfpathmoveto{\pgfqpoint{4.827830in}{1.774064in}}%
\pgfpathlineto{\pgfqpoint{4.827830in}{1.774064in}}%
\pgfpathlineto{\pgfqpoint{4.827830in}{1.777013in}}%
\pgfpathlineto{\pgfqpoint{4.832371in}{1.777013in}}%
\pgfpathlineto{\pgfqpoint{4.832371in}{1.774064in}}%
\pgfpathmoveto{\pgfqpoint{4.827830in}{1.777013in}}%
\pgfpathlineto{\pgfqpoint{4.827830in}{1.777013in}}%
\pgfpathlineto{\pgfqpoint{4.827830in}{1.779962in}}%
\pgfpathlineto{\pgfqpoint{4.832371in}{1.779962in}}%
\pgfpathlineto{\pgfqpoint{4.832371in}{1.777013in}}%
\pgfpathmoveto{\pgfqpoint{4.832371in}{1.774064in}}%
\pgfpathlineto{\pgfqpoint{4.832371in}{1.774064in}}%
\pgfpathlineto{\pgfqpoint{4.832371in}{1.777013in}}%
\pgfpathlineto{\pgfqpoint{4.836912in}{1.777013in}}%
\pgfpathlineto{\pgfqpoint{4.836912in}{1.774064in}}%
\pgfpathmoveto{\pgfqpoint{4.895946in}{1.729826in}}%
\pgfpathlineto{\pgfqpoint{4.895946in}{1.729826in}}%
\pgfpathlineto{\pgfqpoint{4.895946in}{1.732775in}}%
\pgfpathlineto{\pgfqpoint{4.900487in}{1.732775in}}%
\pgfpathlineto{\pgfqpoint{4.900487in}{1.729826in}}%
\pgfpathmoveto{\pgfqpoint{4.891405in}{1.732775in}}%
\pgfpathlineto{\pgfqpoint{4.891405in}{1.732775in}}%
\pgfpathlineto{\pgfqpoint{4.891405in}{1.735724in}}%
\pgfpathlineto{\pgfqpoint{4.895946in}{1.735724in}}%
\pgfpathlineto{\pgfqpoint{4.895946in}{1.732775in}}%
\pgfpathmoveto{\pgfqpoint{4.891405in}{1.735724in}}%
\pgfpathlineto{\pgfqpoint{4.891405in}{1.735724in}}%
\pgfpathlineto{\pgfqpoint{4.891405in}{1.738673in}}%
\pgfpathlineto{\pgfqpoint{4.895946in}{1.738673in}}%
\pgfpathlineto{\pgfqpoint{4.895946in}{1.735724in}}%
\pgfpathmoveto{\pgfqpoint{4.895946in}{1.732775in}}%
\pgfpathlineto{\pgfqpoint{4.895946in}{1.732775in}}%
\pgfpathlineto{\pgfqpoint{4.895946in}{1.735724in}}%
\pgfpathlineto{\pgfqpoint{4.900487in}{1.735724in}}%
\pgfpathlineto{\pgfqpoint{4.900487in}{1.732775in}}%
\pgfpathmoveto{\pgfqpoint{4.900487in}{1.726877in}}%
\pgfpathlineto{\pgfqpoint{4.900487in}{1.726877in}}%
\pgfpathlineto{\pgfqpoint{4.900487in}{1.729826in}}%
\pgfpathlineto{\pgfqpoint{4.905028in}{1.729826in}}%
\pgfpathlineto{\pgfqpoint{4.905028in}{1.726877in}}%
\pgfpathmoveto{\pgfqpoint{4.900487in}{1.729826in}}%
\pgfpathlineto{\pgfqpoint{4.900487in}{1.729826in}}%
\pgfpathlineto{\pgfqpoint{4.900487in}{1.732775in}}%
\pgfpathlineto{\pgfqpoint{4.905028in}{1.732775in}}%
\pgfpathlineto{\pgfqpoint{4.905028in}{1.729826in}}%
\pgfpathmoveto{\pgfqpoint{4.905028in}{1.726877in}}%
\pgfpathlineto{\pgfqpoint{4.905028in}{1.726877in}}%
\pgfpathlineto{\pgfqpoint{4.905028in}{1.729826in}}%
\pgfpathlineto{\pgfqpoint{4.909569in}{1.729826in}}%
\pgfpathlineto{\pgfqpoint{4.909569in}{1.726877in}}%
\pgfpathmoveto{\pgfqpoint{5.050340in}{1.629550in}}%
\pgfpathlineto{\pgfqpoint{5.050340in}{1.629550in}}%
\pgfpathlineto{\pgfqpoint{5.050340in}{1.632499in}}%
\pgfpathlineto{\pgfqpoint{5.054881in}{1.632499in}}%
\pgfpathlineto{\pgfqpoint{5.054881in}{1.629550in}}%
\pgfpathmoveto{\pgfqpoint{5.068504in}{1.617753in}}%
\pgfpathlineto{\pgfqpoint{5.068504in}{1.617753in}}%
\pgfpathlineto{\pgfqpoint{5.068504in}{1.620703in}}%
\pgfpathlineto{\pgfqpoint{5.073045in}{1.620703in}}%
\pgfpathlineto{\pgfqpoint{5.073045in}{1.617753in}}%
\pgfpathmoveto{\pgfqpoint{5.059422in}{1.623652in}}%
\pgfpathlineto{\pgfqpoint{5.059422in}{1.623652in}}%
\pgfpathlineto{\pgfqpoint{5.059422in}{1.626601in}}%
\pgfpathlineto{\pgfqpoint{5.063963in}{1.626601in}}%
\pgfpathlineto{\pgfqpoint{5.063963in}{1.623652in}}%
\pgfpathmoveto{\pgfqpoint{5.054881in}{1.626601in}}%
\pgfpathlineto{\pgfqpoint{5.054881in}{1.626601in}}%
\pgfpathlineto{\pgfqpoint{5.054881in}{1.629550in}}%
\pgfpathlineto{\pgfqpoint{5.059422in}{1.629550in}}%
\pgfpathlineto{\pgfqpoint{5.059422in}{1.626601in}}%
\pgfpathmoveto{\pgfqpoint{5.054881in}{1.629550in}}%
\pgfpathlineto{\pgfqpoint{5.054881in}{1.629550in}}%
\pgfpathlineto{\pgfqpoint{5.054881in}{1.632499in}}%
\pgfpathlineto{\pgfqpoint{5.059422in}{1.632499in}}%
\pgfpathlineto{\pgfqpoint{5.059422in}{1.629550in}}%
\pgfpathmoveto{\pgfqpoint{5.059422in}{1.626601in}}%
\pgfpathlineto{\pgfqpoint{5.059422in}{1.626601in}}%
\pgfpathlineto{\pgfqpoint{5.059422in}{1.629550in}}%
\pgfpathlineto{\pgfqpoint{5.063963in}{1.629550in}}%
\pgfpathlineto{\pgfqpoint{5.063963in}{1.626601in}}%
\pgfpathmoveto{\pgfqpoint{5.063963in}{1.620703in}}%
\pgfpathlineto{\pgfqpoint{5.063963in}{1.620703in}}%
\pgfpathlineto{\pgfqpoint{5.063963in}{1.623652in}}%
\pgfpathlineto{\pgfqpoint{5.068504in}{1.623652in}}%
\pgfpathlineto{\pgfqpoint{5.068504in}{1.620703in}}%
\pgfpathmoveto{\pgfqpoint{5.063963in}{1.623652in}}%
\pgfpathlineto{\pgfqpoint{5.063963in}{1.623652in}}%
\pgfpathlineto{\pgfqpoint{5.063963in}{1.626601in}}%
\pgfpathlineto{\pgfqpoint{5.068504in}{1.626601in}}%
\pgfpathlineto{\pgfqpoint{5.068504in}{1.623652in}}%
\pgfpathmoveto{\pgfqpoint{5.068504in}{1.620703in}}%
\pgfpathlineto{\pgfqpoint{5.068504in}{1.620703in}}%
\pgfpathlineto{\pgfqpoint{5.068504in}{1.623652in}}%
\pgfpathlineto{\pgfqpoint{5.073045in}{1.623652in}}%
\pgfpathlineto{\pgfqpoint{5.073045in}{1.620703in}}%
\pgfpathmoveto{\pgfqpoint{5.086668in}{1.605957in}}%
\pgfpathlineto{\pgfqpoint{5.086668in}{1.605957in}}%
\pgfpathlineto{\pgfqpoint{5.086668in}{1.608906in}}%
\pgfpathlineto{\pgfqpoint{5.091209in}{1.608906in}}%
\pgfpathlineto{\pgfqpoint{5.091209in}{1.605957in}}%
\pgfpathmoveto{\pgfqpoint{5.104832in}{1.594160in}}%
\pgfpathlineto{\pgfqpoint{5.104832in}{1.594160in}}%
\pgfpathlineto{\pgfqpoint{5.104832in}{1.597109in}}%
\pgfpathlineto{\pgfqpoint{5.109373in}{1.597109in}}%
\pgfpathlineto{\pgfqpoint{5.109373in}{1.594160in}}%
\pgfpathmoveto{\pgfqpoint{5.095750in}{1.600059in}}%
\pgfpathlineto{\pgfqpoint{5.095750in}{1.600059in}}%
\pgfpathlineto{\pgfqpoint{5.095750in}{1.603008in}}%
\pgfpathlineto{\pgfqpoint{5.100291in}{1.603008in}}%
\pgfpathlineto{\pgfqpoint{5.100291in}{1.600059in}}%
\pgfpathmoveto{\pgfqpoint{5.091209in}{1.603008in}}%
\pgfpathlineto{\pgfqpoint{5.091209in}{1.603008in}}%
\pgfpathlineto{\pgfqpoint{5.091209in}{1.605957in}}%
\pgfpathlineto{\pgfqpoint{5.095750in}{1.605957in}}%
\pgfpathlineto{\pgfqpoint{5.095750in}{1.603008in}}%
\pgfpathmoveto{\pgfqpoint{5.091209in}{1.605957in}}%
\pgfpathlineto{\pgfqpoint{5.091209in}{1.605957in}}%
\pgfpathlineto{\pgfqpoint{5.091209in}{1.608906in}}%
\pgfpathlineto{\pgfqpoint{5.095750in}{1.608906in}}%
\pgfpathlineto{\pgfqpoint{5.095750in}{1.605957in}}%
\pgfpathmoveto{\pgfqpoint{5.095750in}{1.603008in}}%
\pgfpathlineto{\pgfqpoint{5.095750in}{1.603008in}}%
\pgfpathlineto{\pgfqpoint{5.095750in}{1.605957in}}%
\pgfpathlineto{\pgfqpoint{5.100291in}{1.605957in}}%
\pgfpathlineto{\pgfqpoint{5.100291in}{1.603008in}}%
\pgfpathmoveto{\pgfqpoint{5.100291in}{1.597109in}}%
\pgfpathlineto{\pgfqpoint{5.100291in}{1.597109in}}%
\pgfpathlineto{\pgfqpoint{5.100291in}{1.600059in}}%
\pgfpathlineto{\pgfqpoint{5.104832in}{1.600059in}}%
\pgfpathlineto{\pgfqpoint{5.104832in}{1.597109in}}%
\pgfpathmoveto{\pgfqpoint{5.100291in}{1.600059in}}%
\pgfpathlineto{\pgfqpoint{5.100291in}{1.600059in}}%
\pgfpathlineto{\pgfqpoint{5.100291in}{1.603008in}}%
\pgfpathlineto{\pgfqpoint{5.104832in}{1.603008in}}%
\pgfpathlineto{\pgfqpoint{5.104832in}{1.600059in}}%
\pgfpathmoveto{\pgfqpoint{5.104832in}{1.597109in}}%
\pgfpathlineto{\pgfqpoint{5.104832in}{1.597109in}}%
\pgfpathlineto{\pgfqpoint{5.104832in}{1.600059in}}%
\pgfpathlineto{\pgfqpoint{5.109373in}{1.600059in}}%
\pgfpathlineto{\pgfqpoint{5.109373in}{1.597109in}}%
\pgfpathmoveto{\pgfqpoint{5.077586in}{1.611855in}}%
\pgfpathlineto{\pgfqpoint{5.077586in}{1.611855in}}%
\pgfpathlineto{\pgfqpoint{5.077586in}{1.614804in}}%
\pgfpathlineto{\pgfqpoint{5.082127in}{1.614804in}}%
\pgfpathlineto{\pgfqpoint{5.082127in}{1.611855in}}%
\pgfpathmoveto{\pgfqpoint{5.073045in}{1.614804in}}%
\pgfpathlineto{\pgfqpoint{5.073045in}{1.614804in}}%
\pgfpathlineto{\pgfqpoint{5.073045in}{1.617753in}}%
\pgfpathlineto{\pgfqpoint{5.077586in}{1.617753in}}%
\pgfpathlineto{\pgfqpoint{5.077586in}{1.614804in}}%
\pgfpathmoveto{\pgfqpoint{5.073045in}{1.617753in}}%
\pgfpathlineto{\pgfqpoint{5.073045in}{1.617753in}}%
\pgfpathlineto{\pgfqpoint{5.073045in}{1.620703in}}%
\pgfpathlineto{\pgfqpoint{5.077586in}{1.620703in}}%
\pgfpathlineto{\pgfqpoint{5.077586in}{1.617753in}}%
\pgfpathmoveto{\pgfqpoint{5.077586in}{1.614804in}}%
\pgfpathlineto{\pgfqpoint{5.077586in}{1.614804in}}%
\pgfpathlineto{\pgfqpoint{5.077586in}{1.617753in}}%
\pgfpathlineto{\pgfqpoint{5.082127in}{1.617753in}}%
\pgfpathlineto{\pgfqpoint{5.082127in}{1.614804in}}%
\pgfpathmoveto{\pgfqpoint{5.082127in}{1.608906in}}%
\pgfpathlineto{\pgfqpoint{5.082127in}{1.608906in}}%
\pgfpathlineto{\pgfqpoint{5.082127in}{1.611855in}}%
\pgfpathlineto{\pgfqpoint{5.086668in}{1.611855in}}%
\pgfpathlineto{\pgfqpoint{5.086668in}{1.608906in}}%
\pgfpathmoveto{\pgfqpoint{5.082127in}{1.611855in}}%
\pgfpathlineto{\pgfqpoint{5.082127in}{1.611855in}}%
\pgfpathlineto{\pgfqpoint{5.082127in}{1.614804in}}%
\pgfpathlineto{\pgfqpoint{5.086668in}{1.614804in}}%
\pgfpathlineto{\pgfqpoint{5.086668in}{1.611855in}}%
\pgfpathmoveto{\pgfqpoint{5.086668in}{1.608906in}}%
\pgfpathlineto{\pgfqpoint{5.086668in}{1.608906in}}%
\pgfpathlineto{\pgfqpoint{5.086668in}{1.611855in}}%
\pgfpathlineto{\pgfqpoint{5.091209in}{1.611855in}}%
\pgfpathlineto{\pgfqpoint{5.091209in}{1.608906in}}%
\pgfpathmoveto{\pgfqpoint{4.977685in}{1.676739in}}%
\pgfpathlineto{\pgfqpoint{4.977685in}{1.676739in}}%
\pgfpathlineto{\pgfqpoint{4.977685in}{1.679688in}}%
\pgfpathlineto{\pgfqpoint{4.982226in}{1.679688in}}%
\pgfpathlineto{\pgfqpoint{4.982226in}{1.676739in}}%
\pgfpathmoveto{\pgfqpoint{4.995849in}{1.664941in}}%
\pgfpathlineto{\pgfqpoint{4.995849in}{1.664941in}}%
\pgfpathlineto{\pgfqpoint{4.995849in}{1.667891in}}%
\pgfpathlineto{\pgfqpoint{5.000390in}{1.667891in}}%
\pgfpathlineto{\pgfqpoint{5.000390in}{1.664941in}}%
\pgfpathmoveto{\pgfqpoint{4.986767in}{1.670840in}}%
\pgfpathlineto{\pgfqpoint{4.986767in}{1.670840in}}%
\pgfpathlineto{\pgfqpoint{4.986767in}{1.673789in}}%
\pgfpathlineto{\pgfqpoint{4.991308in}{1.673789in}}%
\pgfpathlineto{\pgfqpoint{4.991308in}{1.670840in}}%
\pgfpathmoveto{\pgfqpoint{4.982226in}{1.673789in}}%
\pgfpathlineto{\pgfqpoint{4.982226in}{1.673789in}}%
\pgfpathlineto{\pgfqpoint{4.982226in}{1.676739in}}%
\pgfpathlineto{\pgfqpoint{4.986767in}{1.676739in}}%
\pgfpathlineto{\pgfqpoint{4.986767in}{1.673789in}}%
\pgfpathmoveto{\pgfqpoint{4.982226in}{1.676739in}}%
\pgfpathlineto{\pgfqpoint{4.982226in}{1.676739in}}%
\pgfpathlineto{\pgfqpoint{4.982226in}{1.679688in}}%
\pgfpathlineto{\pgfqpoint{4.986767in}{1.679688in}}%
\pgfpathlineto{\pgfqpoint{4.986767in}{1.676739in}}%
\pgfpathmoveto{\pgfqpoint{4.986767in}{1.673789in}}%
\pgfpathlineto{\pgfqpoint{4.986767in}{1.673789in}}%
\pgfpathlineto{\pgfqpoint{4.986767in}{1.676739in}}%
\pgfpathlineto{\pgfqpoint{4.991308in}{1.676739in}}%
\pgfpathlineto{\pgfqpoint{4.991308in}{1.673789in}}%
\pgfpathmoveto{\pgfqpoint{4.991308in}{1.667891in}}%
\pgfpathlineto{\pgfqpoint{4.991308in}{1.667891in}}%
\pgfpathlineto{\pgfqpoint{4.991308in}{1.670840in}}%
\pgfpathlineto{\pgfqpoint{4.995849in}{1.670840in}}%
\pgfpathlineto{\pgfqpoint{4.995849in}{1.667891in}}%
\pgfpathmoveto{\pgfqpoint{4.991308in}{1.670840in}}%
\pgfpathlineto{\pgfqpoint{4.991308in}{1.670840in}}%
\pgfpathlineto{\pgfqpoint{4.991308in}{1.673789in}}%
\pgfpathlineto{\pgfqpoint{4.995849in}{1.673789in}}%
\pgfpathlineto{\pgfqpoint{4.995849in}{1.670840in}}%
\pgfpathmoveto{\pgfqpoint{4.995849in}{1.667891in}}%
\pgfpathlineto{\pgfqpoint{4.995849in}{1.667891in}}%
\pgfpathlineto{\pgfqpoint{4.995849in}{1.670840in}}%
\pgfpathlineto{\pgfqpoint{5.000390in}{1.670840in}}%
\pgfpathlineto{\pgfqpoint{5.000390in}{1.667891in}}%
\pgfpathmoveto{\pgfqpoint{5.014012in}{1.653144in}}%
\pgfpathlineto{\pgfqpoint{5.014012in}{1.653144in}}%
\pgfpathlineto{\pgfqpoint{5.014012in}{1.656093in}}%
\pgfpathlineto{\pgfqpoint{5.018553in}{1.656093in}}%
\pgfpathlineto{\pgfqpoint{5.018553in}{1.653144in}}%
\pgfpathmoveto{\pgfqpoint{5.032176in}{1.641347in}}%
\pgfpathlineto{\pgfqpoint{5.032176in}{1.641347in}}%
\pgfpathlineto{\pgfqpoint{5.032176in}{1.644296in}}%
\pgfpathlineto{\pgfqpoint{5.036717in}{1.644296in}}%
\pgfpathlineto{\pgfqpoint{5.036717in}{1.641347in}}%
\pgfpathmoveto{\pgfqpoint{5.023094in}{1.647246in}}%
\pgfpathlineto{\pgfqpoint{5.023094in}{1.647246in}}%
\pgfpathlineto{\pgfqpoint{5.023094in}{1.650195in}}%
\pgfpathlineto{\pgfqpoint{5.027635in}{1.650195in}}%
\pgfpathlineto{\pgfqpoint{5.027635in}{1.647246in}}%
\pgfpathmoveto{\pgfqpoint{5.018553in}{1.650195in}}%
\pgfpathlineto{\pgfqpoint{5.018553in}{1.650195in}}%
\pgfpathlineto{\pgfqpoint{5.018553in}{1.653144in}}%
\pgfpathlineto{\pgfqpoint{5.023094in}{1.653144in}}%
\pgfpathlineto{\pgfqpoint{5.023094in}{1.650195in}}%
\pgfpathmoveto{\pgfqpoint{5.018553in}{1.653144in}}%
\pgfpathlineto{\pgfqpoint{5.018553in}{1.653144in}}%
\pgfpathlineto{\pgfqpoint{5.018553in}{1.656093in}}%
\pgfpathlineto{\pgfqpoint{5.023094in}{1.656093in}}%
\pgfpathlineto{\pgfqpoint{5.023094in}{1.653144in}}%
\pgfpathmoveto{\pgfqpoint{5.023094in}{1.650195in}}%
\pgfpathlineto{\pgfqpoint{5.023094in}{1.650195in}}%
\pgfpathlineto{\pgfqpoint{5.023094in}{1.653144in}}%
\pgfpathlineto{\pgfqpoint{5.027635in}{1.653144in}}%
\pgfpathlineto{\pgfqpoint{5.027635in}{1.650195in}}%
\pgfpathmoveto{\pgfqpoint{5.027635in}{1.644296in}}%
\pgfpathlineto{\pgfqpoint{5.027635in}{1.644296in}}%
\pgfpathlineto{\pgfqpoint{5.027635in}{1.647246in}}%
\pgfpathlineto{\pgfqpoint{5.032176in}{1.647246in}}%
\pgfpathlineto{\pgfqpoint{5.032176in}{1.644296in}}%
\pgfpathmoveto{\pgfqpoint{5.027635in}{1.647246in}}%
\pgfpathlineto{\pgfqpoint{5.027635in}{1.647246in}}%
\pgfpathlineto{\pgfqpoint{5.027635in}{1.650195in}}%
\pgfpathlineto{\pgfqpoint{5.032176in}{1.650195in}}%
\pgfpathlineto{\pgfqpoint{5.032176in}{1.647246in}}%
\pgfpathmoveto{\pgfqpoint{5.032176in}{1.644296in}}%
\pgfpathlineto{\pgfqpoint{5.032176in}{1.644296in}}%
\pgfpathlineto{\pgfqpoint{5.032176in}{1.647246in}}%
\pgfpathlineto{\pgfqpoint{5.036717in}{1.647246in}}%
\pgfpathlineto{\pgfqpoint{5.036717in}{1.644296in}}%
\pgfpathmoveto{\pgfqpoint{5.004930in}{1.659043in}}%
\pgfpathlineto{\pgfqpoint{5.004930in}{1.659043in}}%
\pgfpathlineto{\pgfqpoint{5.004930in}{1.661992in}}%
\pgfpathlineto{\pgfqpoint{5.009471in}{1.661992in}}%
\pgfpathlineto{\pgfqpoint{5.009471in}{1.659043in}}%
\pgfpathmoveto{\pgfqpoint{5.000390in}{1.661992in}}%
\pgfpathlineto{\pgfqpoint{5.000390in}{1.661992in}}%
\pgfpathlineto{\pgfqpoint{5.000390in}{1.664941in}}%
\pgfpathlineto{\pgfqpoint{5.004930in}{1.664941in}}%
\pgfpathlineto{\pgfqpoint{5.004930in}{1.661992in}}%
\pgfpathmoveto{\pgfqpoint{5.000390in}{1.664941in}}%
\pgfpathlineto{\pgfqpoint{5.000390in}{1.664941in}}%
\pgfpathlineto{\pgfqpoint{5.000390in}{1.667891in}}%
\pgfpathlineto{\pgfqpoint{5.004930in}{1.667891in}}%
\pgfpathlineto{\pgfqpoint{5.004930in}{1.664941in}}%
\pgfpathmoveto{\pgfqpoint{5.004930in}{1.661992in}}%
\pgfpathlineto{\pgfqpoint{5.004930in}{1.661992in}}%
\pgfpathlineto{\pgfqpoint{5.004930in}{1.664941in}}%
\pgfpathlineto{\pgfqpoint{5.009471in}{1.664941in}}%
\pgfpathlineto{\pgfqpoint{5.009471in}{1.661992in}}%
\pgfpathmoveto{\pgfqpoint{5.009471in}{1.656093in}}%
\pgfpathlineto{\pgfqpoint{5.009471in}{1.656093in}}%
\pgfpathlineto{\pgfqpoint{5.009471in}{1.659043in}}%
\pgfpathlineto{\pgfqpoint{5.014012in}{1.659043in}}%
\pgfpathlineto{\pgfqpoint{5.014012in}{1.656093in}}%
\pgfpathmoveto{\pgfqpoint{5.009471in}{1.659043in}}%
\pgfpathlineto{\pgfqpoint{5.009471in}{1.659043in}}%
\pgfpathlineto{\pgfqpoint{5.009471in}{1.661992in}}%
\pgfpathlineto{\pgfqpoint{5.014012in}{1.661992in}}%
\pgfpathlineto{\pgfqpoint{5.014012in}{1.659043in}}%
\pgfpathmoveto{\pgfqpoint{5.014012in}{1.656093in}}%
\pgfpathlineto{\pgfqpoint{5.014012in}{1.656093in}}%
\pgfpathlineto{\pgfqpoint{5.014012in}{1.659043in}}%
\pgfpathlineto{\pgfqpoint{5.018553in}{1.659043in}}%
\pgfpathlineto{\pgfqpoint{5.018553in}{1.656093in}}%
\pgfpathmoveto{\pgfqpoint{4.968603in}{1.682637in}}%
\pgfpathlineto{\pgfqpoint{4.968603in}{1.682637in}}%
\pgfpathlineto{\pgfqpoint{4.968603in}{1.685587in}}%
\pgfpathlineto{\pgfqpoint{4.973144in}{1.685587in}}%
\pgfpathlineto{\pgfqpoint{4.973144in}{1.682637in}}%
\pgfpathmoveto{\pgfqpoint{4.964062in}{1.685587in}}%
\pgfpathlineto{\pgfqpoint{4.964062in}{1.685587in}}%
\pgfpathlineto{\pgfqpoint{4.964062in}{1.688536in}}%
\pgfpathlineto{\pgfqpoint{4.968603in}{1.688536in}}%
\pgfpathlineto{\pgfqpoint{4.968603in}{1.685587in}}%
\pgfpathmoveto{\pgfqpoint{4.964062in}{1.688536in}}%
\pgfpathlineto{\pgfqpoint{4.964062in}{1.688536in}}%
\pgfpathlineto{\pgfqpoint{4.964062in}{1.691485in}}%
\pgfpathlineto{\pgfqpoint{4.968603in}{1.691485in}}%
\pgfpathlineto{\pgfqpoint{4.968603in}{1.688536in}}%
\pgfpathmoveto{\pgfqpoint{4.968603in}{1.685587in}}%
\pgfpathlineto{\pgfqpoint{4.968603in}{1.685587in}}%
\pgfpathlineto{\pgfqpoint{4.968603in}{1.688536in}}%
\pgfpathlineto{\pgfqpoint{4.973144in}{1.688536in}}%
\pgfpathlineto{\pgfqpoint{4.973144in}{1.685587in}}%
\pgfpathmoveto{\pgfqpoint{4.973144in}{1.679688in}}%
\pgfpathlineto{\pgfqpoint{4.973144in}{1.679688in}}%
\pgfpathlineto{\pgfqpoint{4.973144in}{1.682637in}}%
\pgfpathlineto{\pgfqpoint{4.977685in}{1.682637in}}%
\pgfpathlineto{\pgfqpoint{4.977685in}{1.679688in}}%
\pgfpathmoveto{\pgfqpoint{4.973144in}{1.682637in}}%
\pgfpathlineto{\pgfqpoint{4.973144in}{1.682637in}}%
\pgfpathlineto{\pgfqpoint{4.973144in}{1.685587in}}%
\pgfpathlineto{\pgfqpoint{4.977685in}{1.685587in}}%
\pgfpathlineto{\pgfqpoint{4.977685in}{1.682637in}}%
\pgfpathmoveto{\pgfqpoint{4.977685in}{1.679688in}}%
\pgfpathlineto{\pgfqpoint{4.977685in}{1.679688in}}%
\pgfpathlineto{\pgfqpoint{4.977685in}{1.682637in}}%
\pgfpathlineto{\pgfqpoint{4.982226in}{1.682637in}}%
\pgfpathlineto{\pgfqpoint{4.982226in}{1.679688in}}%
\pgfpathmoveto{\pgfqpoint{5.041258in}{1.635448in}}%
\pgfpathlineto{\pgfqpoint{5.041258in}{1.635448in}}%
\pgfpathlineto{\pgfqpoint{5.041258in}{1.638398in}}%
\pgfpathlineto{\pgfqpoint{5.045799in}{1.638398in}}%
\pgfpathlineto{\pgfqpoint{5.045799in}{1.635448in}}%
\pgfpathmoveto{\pgfqpoint{5.036717in}{1.638398in}}%
\pgfpathlineto{\pgfqpoint{5.036717in}{1.638398in}}%
\pgfpathlineto{\pgfqpoint{5.036717in}{1.641347in}}%
\pgfpathlineto{\pgfqpoint{5.041258in}{1.641347in}}%
\pgfpathlineto{\pgfqpoint{5.041258in}{1.638398in}}%
\pgfpathmoveto{\pgfqpoint{5.036717in}{1.641347in}}%
\pgfpathlineto{\pgfqpoint{5.036717in}{1.641347in}}%
\pgfpathlineto{\pgfqpoint{5.036717in}{1.644296in}}%
\pgfpathlineto{\pgfqpoint{5.041258in}{1.644296in}}%
\pgfpathlineto{\pgfqpoint{5.041258in}{1.641347in}}%
\pgfpathmoveto{\pgfqpoint{5.041258in}{1.638398in}}%
\pgfpathlineto{\pgfqpoint{5.041258in}{1.638398in}}%
\pgfpathlineto{\pgfqpoint{5.041258in}{1.641347in}}%
\pgfpathlineto{\pgfqpoint{5.045799in}{1.641347in}}%
\pgfpathlineto{\pgfqpoint{5.045799in}{1.638398in}}%
\pgfpathmoveto{\pgfqpoint{5.045799in}{1.632499in}}%
\pgfpathlineto{\pgfqpoint{5.045799in}{1.632499in}}%
\pgfpathlineto{\pgfqpoint{5.045799in}{1.635448in}}%
\pgfpathlineto{\pgfqpoint{5.050340in}{1.635448in}}%
\pgfpathlineto{\pgfqpoint{5.050340in}{1.632499in}}%
\pgfpathmoveto{\pgfqpoint{5.045799in}{1.635448in}}%
\pgfpathlineto{\pgfqpoint{5.045799in}{1.635448in}}%
\pgfpathlineto{\pgfqpoint{5.045799in}{1.638398in}}%
\pgfpathlineto{\pgfqpoint{5.050340in}{1.638398in}}%
\pgfpathlineto{\pgfqpoint{5.050340in}{1.635448in}}%
\pgfpathmoveto{\pgfqpoint{5.050340in}{1.632499in}}%
\pgfpathlineto{\pgfqpoint{5.050340in}{1.632499in}}%
\pgfpathlineto{\pgfqpoint{5.050340in}{1.635448in}}%
\pgfpathlineto{\pgfqpoint{5.054881in}{1.635448in}}%
\pgfpathlineto{\pgfqpoint{5.054881in}{1.632499in}}%
\pgfpathmoveto{\pgfqpoint{5.195652in}{1.535177in}}%
\pgfpathlineto{\pgfqpoint{5.195652in}{1.535177in}}%
\pgfpathlineto{\pgfqpoint{5.195652in}{1.538127in}}%
\pgfpathlineto{\pgfqpoint{5.200193in}{1.538127in}}%
\pgfpathlineto{\pgfqpoint{5.200193in}{1.535177in}}%
\pgfpathmoveto{\pgfqpoint{5.213816in}{1.523380in}}%
\pgfpathlineto{\pgfqpoint{5.213816in}{1.523380in}}%
\pgfpathlineto{\pgfqpoint{5.213816in}{1.526330in}}%
\pgfpathlineto{\pgfqpoint{5.218357in}{1.526330in}}%
\pgfpathlineto{\pgfqpoint{5.218357in}{1.523380in}}%
\pgfpathmoveto{\pgfqpoint{5.204734in}{1.529279in}}%
\pgfpathlineto{\pgfqpoint{5.204734in}{1.529279in}}%
\pgfpathlineto{\pgfqpoint{5.204734in}{1.532228in}}%
\pgfpathlineto{\pgfqpoint{5.209275in}{1.532228in}}%
\pgfpathlineto{\pgfqpoint{5.209275in}{1.529279in}}%
\pgfpathmoveto{\pgfqpoint{5.200193in}{1.532228in}}%
\pgfpathlineto{\pgfqpoint{5.200193in}{1.532228in}}%
\pgfpathlineto{\pgfqpoint{5.200193in}{1.535177in}}%
\pgfpathlineto{\pgfqpoint{5.204734in}{1.535177in}}%
\pgfpathlineto{\pgfqpoint{5.204734in}{1.532228in}}%
\pgfpathmoveto{\pgfqpoint{5.200193in}{1.535177in}}%
\pgfpathlineto{\pgfqpoint{5.200193in}{1.535177in}}%
\pgfpathlineto{\pgfqpoint{5.200193in}{1.538127in}}%
\pgfpathlineto{\pgfqpoint{5.204734in}{1.538127in}}%
\pgfpathlineto{\pgfqpoint{5.204734in}{1.535177in}}%
\pgfpathmoveto{\pgfqpoint{5.204734in}{1.532228in}}%
\pgfpathlineto{\pgfqpoint{5.204734in}{1.532228in}}%
\pgfpathlineto{\pgfqpoint{5.204734in}{1.535177in}}%
\pgfpathlineto{\pgfqpoint{5.209275in}{1.535177in}}%
\pgfpathlineto{\pgfqpoint{5.209275in}{1.532228in}}%
\pgfpathmoveto{\pgfqpoint{5.209275in}{1.526330in}}%
\pgfpathlineto{\pgfqpoint{5.209275in}{1.526330in}}%
\pgfpathlineto{\pgfqpoint{5.209275in}{1.529279in}}%
\pgfpathlineto{\pgfqpoint{5.213816in}{1.529279in}}%
\pgfpathlineto{\pgfqpoint{5.213816in}{1.526330in}}%
\pgfpathmoveto{\pgfqpoint{5.209275in}{1.529279in}}%
\pgfpathlineto{\pgfqpoint{5.209275in}{1.529279in}}%
\pgfpathlineto{\pgfqpoint{5.209275in}{1.532228in}}%
\pgfpathlineto{\pgfqpoint{5.213816in}{1.532228in}}%
\pgfpathlineto{\pgfqpoint{5.213816in}{1.529279in}}%
\pgfpathmoveto{\pgfqpoint{5.213816in}{1.526330in}}%
\pgfpathlineto{\pgfqpoint{5.213816in}{1.526330in}}%
\pgfpathlineto{\pgfqpoint{5.213816in}{1.529279in}}%
\pgfpathlineto{\pgfqpoint{5.218357in}{1.529279in}}%
\pgfpathlineto{\pgfqpoint{5.218357in}{1.526330in}}%
\pgfpathmoveto{\pgfqpoint{5.231980in}{1.511583in}}%
\pgfpathlineto{\pgfqpoint{5.231980in}{1.511583in}}%
\pgfpathlineto{\pgfqpoint{5.231980in}{1.514533in}}%
\pgfpathlineto{\pgfqpoint{5.236521in}{1.514533in}}%
\pgfpathlineto{\pgfqpoint{5.236521in}{1.511583in}}%
\pgfpathmoveto{\pgfqpoint{5.250144in}{1.499786in}}%
\pgfpathlineto{\pgfqpoint{5.250144in}{1.499786in}}%
\pgfpathlineto{\pgfqpoint{5.250144in}{1.502736in}}%
\pgfpathlineto{\pgfqpoint{5.254685in}{1.502736in}}%
\pgfpathlineto{\pgfqpoint{5.254685in}{1.499786in}}%
\pgfpathmoveto{\pgfqpoint{5.241062in}{1.505685in}}%
\pgfpathlineto{\pgfqpoint{5.241062in}{1.505685in}}%
\pgfpathlineto{\pgfqpoint{5.241062in}{1.508634in}}%
\pgfpathlineto{\pgfqpoint{5.245603in}{1.508634in}}%
\pgfpathlineto{\pgfqpoint{5.245603in}{1.505685in}}%
\pgfpathmoveto{\pgfqpoint{5.236521in}{1.508634in}}%
\pgfpathlineto{\pgfqpoint{5.236521in}{1.508634in}}%
\pgfpathlineto{\pgfqpoint{5.236521in}{1.511583in}}%
\pgfpathlineto{\pgfqpoint{5.241062in}{1.511583in}}%
\pgfpathlineto{\pgfqpoint{5.241062in}{1.508634in}}%
\pgfpathmoveto{\pgfqpoint{5.236521in}{1.511583in}}%
\pgfpathlineto{\pgfqpoint{5.236521in}{1.511583in}}%
\pgfpathlineto{\pgfqpoint{5.236521in}{1.514533in}}%
\pgfpathlineto{\pgfqpoint{5.241062in}{1.514533in}}%
\pgfpathlineto{\pgfqpoint{5.241062in}{1.511583in}}%
\pgfpathmoveto{\pgfqpoint{5.241062in}{1.508634in}}%
\pgfpathlineto{\pgfqpoint{5.241062in}{1.508634in}}%
\pgfpathlineto{\pgfqpoint{5.241062in}{1.511583in}}%
\pgfpathlineto{\pgfqpoint{5.245603in}{1.511583in}}%
\pgfpathlineto{\pgfqpoint{5.245603in}{1.508634in}}%
\pgfpathmoveto{\pgfqpoint{5.245603in}{1.502736in}}%
\pgfpathlineto{\pgfqpoint{5.245603in}{1.502736in}}%
\pgfpathlineto{\pgfqpoint{5.245603in}{1.505685in}}%
\pgfpathlineto{\pgfqpoint{5.250144in}{1.505685in}}%
\pgfpathlineto{\pgfqpoint{5.250144in}{1.502736in}}%
\pgfpathmoveto{\pgfqpoint{5.245603in}{1.505685in}}%
\pgfpathlineto{\pgfqpoint{5.245603in}{1.505685in}}%
\pgfpathlineto{\pgfqpoint{5.245603in}{1.508634in}}%
\pgfpathlineto{\pgfqpoint{5.250144in}{1.508634in}}%
\pgfpathlineto{\pgfqpoint{5.250144in}{1.505685in}}%
\pgfpathmoveto{\pgfqpoint{5.250144in}{1.502736in}}%
\pgfpathlineto{\pgfqpoint{5.250144in}{1.502736in}}%
\pgfpathlineto{\pgfqpoint{5.250144in}{1.505685in}}%
\pgfpathlineto{\pgfqpoint{5.254685in}{1.505685in}}%
\pgfpathlineto{\pgfqpoint{5.254685in}{1.502736in}}%
\pgfpathmoveto{\pgfqpoint{5.222898in}{1.517482in}}%
\pgfpathlineto{\pgfqpoint{5.222898in}{1.517482in}}%
\pgfpathlineto{\pgfqpoint{5.222898in}{1.520431in}}%
\pgfpathlineto{\pgfqpoint{5.227439in}{1.520431in}}%
\pgfpathlineto{\pgfqpoint{5.227439in}{1.517482in}}%
\pgfpathmoveto{\pgfqpoint{5.218357in}{1.520431in}}%
\pgfpathlineto{\pgfqpoint{5.218357in}{1.520431in}}%
\pgfpathlineto{\pgfqpoint{5.218357in}{1.523380in}}%
\pgfpathlineto{\pgfqpoint{5.222898in}{1.523380in}}%
\pgfpathlineto{\pgfqpoint{5.222898in}{1.520431in}}%
\pgfpathmoveto{\pgfqpoint{5.218357in}{1.523380in}}%
\pgfpathlineto{\pgfqpoint{5.218357in}{1.523380in}}%
\pgfpathlineto{\pgfqpoint{5.218357in}{1.526330in}}%
\pgfpathlineto{\pgfqpoint{5.222898in}{1.526330in}}%
\pgfpathlineto{\pgfqpoint{5.222898in}{1.523380in}}%
\pgfpathmoveto{\pgfqpoint{5.222898in}{1.520431in}}%
\pgfpathlineto{\pgfqpoint{5.222898in}{1.520431in}}%
\pgfpathlineto{\pgfqpoint{5.222898in}{1.523380in}}%
\pgfpathlineto{\pgfqpoint{5.227439in}{1.523380in}}%
\pgfpathlineto{\pgfqpoint{5.227439in}{1.520431in}}%
\pgfpathmoveto{\pgfqpoint{5.227439in}{1.514533in}}%
\pgfpathlineto{\pgfqpoint{5.227439in}{1.514533in}}%
\pgfpathlineto{\pgfqpoint{5.227439in}{1.517482in}}%
\pgfpathlineto{\pgfqpoint{5.231980in}{1.517482in}}%
\pgfpathlineto{\pgfqpoint{5.231980in}{1.514533in}}%
\pgfpathmoveto{\pgfqpoint{5.227439in}{1.517482in}}%
\pgfpathlineto{\pgfqpoint{5.227439in}{1.517482in}}%
\pgfpathlineto{\pgfqpoint{5.227439in}{1.520431in}}%
\pgfpathlineto{\pgfqpoint{5.231980in}{1.520431in}}%
\pgfpathlineto{\pgfqpoint{5.231980in}{1.517482in}}%
\pgfpathmoveto{\pgfqpoint{5.231980in}{1.514533in}}%
\pgfpathlineto{\pgfqpoint{5.231980in}{1.514533in}}%
\pgfpathlineto{\pgfqpoint{5.231980in}{1.517482in}}%
\pgfpathlineto{\pgfqpoint{5.236521in}{1.517482in}}%
\pgfpathlineto{\pgfqpoint{5.236521in}{1.514533in}}%
\pgfpathmoveto{\pgfqpoint{5.122996in}{1.582364in}}%
\pgfpathlineto{\pgfqpoint{5.122996in}{1.582364in}}%
\pgfpathlineto{\pgfqpoint{5.122996in}{1.585313in}}%
\pgfpathlineto{\pgfqpoint{5.127537in}{1.585313in}}%
\pgfpathlineto{\pgfqpoint{5.127537in}{1.582364in}}%
\pgfpathmoveto{\pgfqpoint{5.141160in}{1.570567in}}%
\pgfpathlineto{\pgfqpoint{5.141160in}{1.570567in}}%
\pgfpathlineto{\pgfqpoint{5.141160in}{1.573516in}}%
\pgfpathlineto{\pgfqpoint{5.145701in}{1.573516in}}%
\pgfpathlineto{\pgfqpoint{5.145701in}{1.570567in}}%
\pgfpathmoveto{\pgfqpoint{5.132078in}{1.576465in}}%
\pgfpathlineto{\pgfqpoint{5.132078in}{1.576465in}}%
\pgfpathlineto{\pgfqpoint{5.132078in}{1.579415in}}%
\pgfpathlineto{\pgfqpoint{5.136619in}{1.579415in}}%
\pgfpathlineto{\pgfqpoint{5.136619in}{1.576465in}}%
\pgfpathmoveto{\pgfqpoint{5.127537in}{1.579415in}}%
\pgfpathlineto{\pgfqpoint{5.127537in}{1.579415in}}%
\pgfpathlineto{\pgfqpoint{5.127537in}{1.582364in}}%
\pgfpathlineto{\pgfqpoint{5.132078in}{1.582364in}}%
\pgfpathlineto{\pgfqpoint{5.132078in}{1.579415in}}%
\pgfpathmoveto{\pgfqpoint{5.127537in}{1.582364in}}%
\pgfpathlineto{\pgfqpoint{5.127537in}{1.582364in}}%
\pgfpathlineto{\pgfqpoint{5.127537in}{1.585313in}}%
\pgfpathlineto{\pgfqpoint{5.132078in}{1.585313in}}%
\pgfpathlineto{\pgfqpoint{5.132078in}{1.582364in}}%
\pgfpathmoveto{\pgfqpoint{5.132078in}{1.579415in}}%
\pgfpathlineto{\pgfqpoint{5.132078in}{1.579415in}}%
\pgfpathlineto{\pgfqpoint{5.132078in}{1.582364in}}%
\pgfpathlineto{\pgfqpoint{5.136619in}{1.582364in}}%
\pgfpathlineto{\pgfqpoint{5.136619in}{1.579415in}}%
\pgfpathmoveto{\pgfqpoint{5.136619in}{1.573516in}}%
\pgfpathlineto{\pgfqpoint{5.136619in}{1.573516in}}%
\pgfpathlineto{\pgfqpoint{5.136619in}{1.576465in}}%
\pgfpathlineto{\pgfqpoint{5.141160in}{1.576465in}}%
\pgfpathlineto{\pgfqpoint{5.141160in}{1.573516in}}%
\pgfpathmoveto{\pgfqpoint{5.136619in}{1.576465in}}%
\pgfpathlineto{\pgfqpoint{5.136619in}{1.576465in}}%
\pgfpathlineto{\pgfqpoint{5.136619in}{1.579415in}}%
\pgfpathlineto{\pgfqpoint{5.141160in}{1.579415in}}%
\pgfpathlineto{\pgfqpoint{5.141160in}{1.576465in}}%
\pgfpathmoveto{\pgfqpoint{5.141160in}{1.573516in}}%
\pgfpathlineto{\pgfqpoint{5.141160in}{1.573516in}}%
\pgfpathlineto{\pgfqpoint{5.141160in}{1.576465in}}%
\pgfpathlineto{\pgfqpoint{5.145701in}{1.576465in}}%
\pgfpathlineto{\pgfqpoint{5.145701in}{1.573516in}}%
\pgfpathmoveto{\pgfqpoint{5.159324in}{1.558771in}}%
\pgfpathlineto{\pgfqpoint{5.159324in}{1.558771in}}%
\pgfpathlineto{\pgfqpoint{5.159324in}{1.561720in}}%
\pgfpathlineto{\pgfqpoint{5.163865in}{1.561720in}}%
\pgfpathlineto{\pgfqpoint{5.163865in}{1.558771in}}%
\pgfpathmoveto{\pgfqpoint{5.177488in}{1.546974in}}%
\pgfpathlineto{\pgfqpoint{5.177488in}{1.546974in}}%
\pgfpathlineto{\pgfqpoint{5.177488in}{1.549923in}}%
\pgfpathlineto{\pgfqpoint{5.182029in}{1.549923in}}%
\pgfpathlineto{\pgfqpoint{5.182029in}{1.546974in}}%
\pgfpathmoveto{\pgfqpoint{5.168406in}{1.552872in}}%
\pgfpathlineto{\pgfqpoint{5.168406in}{1.552872in}}%
\pgfpathlineto{\pgfqpoint{5.168406in}{1.555822in}}%
\pgfpathlineto{\pgfqpoint{5.172947in}{1.555822in}}%
\pgfpathlineto{\pgfqpoint{5.172947in}{1.552872in}}%
\pgfpathmoveto{\pgfqpoint{5.163865in}{1.555822in}}%
\pgfpathlineto{\pgfqpoint{5.163865in}{1.555822in}}%
\pgfpathlineto{\pgfqpoint{5.163865in}{1.558771in}}%
\pgfpathlineto{\pgfqpoint{5.168406in}{1.558771in}}%
\pgfpathlineto{\pgfqpoint{5.168406in}{1.555822in}}%
\pgfpathmoveto{\pgfqpoint{5.163865in}{1.558771in}}%
\pgfpathlineto{\pgfqpoint{5.163865in}{1.558771in}}%
\pgfpathlineto{\pgfqpoint{5.163865in}{1.561720in}}%
\pgfpathlineto{\pgfqpoint{5.168406in}{1.561720in}}%
\pgfpathlineto{\pgfqpoint{5.168406in}{1.558771in}}%
\pgfpathmoveto{\pgfqpoint{5.168406in}{1.555822in}}%
\pgfpathlineto{\pgfqpoint{5.168406in}{1.555822in}}%
\pgfpathlineto{\pgfqpoint{5.168406in}{1.558771in}}%
\pgfpathlineto{\pgfqpoint{5.172947in}{1.558771in}}%
\pgfpathlineto{\pgfqpoint{5.172947in}{1.555822in}}%
\pgfpathmoveto{\pgfqpoint{5.172947in}{1.549923in}}%
\pgfpathlineto{\pgfqpoint{5.172947in}{1.549923in}}%
\pgfpathlineto{\pgfqpoint{5.172947in}{1.552872in}}%
\pgfpathlineto{\pgfqpoint{5.177488in}{1.552872in}}%
\pgfpathlineto{\pgfqpoint{5.177488in}{1.549923in}}%
\pgfpathmoveto{\pgfqpoint{5.172947in}{1.552872in}}%
\pgfpathlineto{\pgfqpoint{5.172947in}{1.552872in}}%
\pgfpathlineto{\pgfqpoint{5.172947in}{1.555822in}}%
\pgfpathlineto{\pgfqpoint{5.177488in}{1.555822in}}%
\pgfpathlineto{\pgfqpoint{5.177488in}{1.552872in}}%
\pgfpathmoveto{\pgfqpoint{5.177488in}{1.549923in}}%
\pgfpathlineto{\pgfqpoint{5.177488in}{1.549923in}}%
\pgfpathlineto{\pgfqpoint{5.177488in}{1.552872in}}%
\pgfpathlineto{\pgfqpoint{5.182029in}{1.552872in}}%
\pgfpathlineto{\pgfqpoint{5.182029in}{1.549923in}}%
\pgfpathmoveto{\pgfqpoint{5.150242in}{1.564669in}}%
\pgfpathlineto{\pgfqpoint{5.150242in}{1.564669in}}%
\pgfpathlineto{\pgfqpoint{5.150242in}{1.567618in}}%
\pgfpathlineto{\pgfqpoint{5.154783in}{1.567618in}}%
\pgfpathlineto{\pgfqpoint{5.154783in}{1.564669in}}%
\pgfpathmoveto{\pgfqpoint{5.145701in}{1.567618in}}%
\pgfpathlineto{\pgfqpoint{5.145701in}{1.567618in}}%
\pgfpathlineto{\pgfqpoint{5.145701in}{1.570567in}}%
\pgfpathlineto{\pgfqpoint{5.150242in}{1.570567in}}%
\pgfpathlineto{\pgfqpoint{5.150242in}{1.567618in}}%
\pgfpathmoveto{\pgfqpoint{5.145701in}{1.570567in}}%
\pgfpathlineto{\pgfqpoint{5.145701in}{1.570567in}}%
\pgfpathlineto{\pgfqpoint{5.145701in}{1.573516in}}%
\pgfpathlineto{\pgfqpoint{5.150242in}{1.573516in}}%
\pgfpathlineto{\pgfqpoint{5.150242in}{1.570567in}}%
\pgfpathmoveto{\pgfqpoint{5.150242in}{1.567618in}}%
\pgfpathlineto{\pgfqpoint{5.150242in}{1.567618in}}%
\pgfpathlineto{\pgfqpoint{5.150242in}{1.570567in}}%
\pgfpathlineto{\pgfqpoint{5.154783in}{1.570567in}}%
\pgfpathlineto{\pgfqpoint{5.154783in}{1.567618in}}%
\pgfpathmoveto{\pgfqpoint{5.154783in}{1.561720in}}%
\pgfpathlineto{\pgfqpoint{5.154783in}{1.561720in}}%
\pgfpathlineto{\pgfqpoint{5.154783in}{1.564669in}}%
\pgfpathlineto{\pgfqpoint{5.159324in}{1.564669in}}%
\pgfpathlineto{\pgfqpoint{5.159324in}{1.561720in}}%
\pgfpathmoveto{\pgfqpoint{5.154783in}{1.564669in}}%
\pgfpathlineto{\pgfqpoint{5.154783in}{1.564669in}}%
\pgfpathlineto{\pgfqpoint{5.154783in}{1.567618in}}%
\pgfpathlineto{\pgfqpoint{5.159324in}{1.567618in}}%
\pgfpathlineto{\pgfqpoint{5.159324in}{1.564669in}}%
\pgfpathmoveto{\pgfqpoint{5.159324in}{1.561720in}}%
\pgfpathlineto{\pgfqpoint{5.159324in}{1.561720in}}%
\pgfpathlineto{\pgfqpoint{5.159324in}{1.564669in}}%
\pgfpathlineto{\pgfqpoint{5.163865in}{1.564669in}}%
\pgfpathlineto{\pgfqpoint{5.163865in}{1.561720in}}%
\pgfpathmoveto{\pgfqpoint{5.113914in}{1.588262in}}%
\pgfpathlineto{\pgfqpoint{5.113914in}{1.588262in}}%
\pgfpathlineto{\pgfqpoint{5.113914in}{1.591211in}}%
\pgfpathlineto{\pgfqpoint{5.118455in}{1.591211in}}%
\pgfpathlineto{\pgfqpoint{5.118455in}{1.588262in}}%
\pgfpathmoveto{\pgfqpoint{5.109373in}{1.591211in}}%
\pgfpathlineto{\pgfqpoint{5.109373in}{1.591211in}}%
\pgfpathlineto{\pgfqpoint{5.109373in}{1.594160in}}%
\pgfpathlineto{\pgfqpoint{5.113914in}{1.594160in}}%
\pgfpathlineto{\pgfqpoint{5.113914in}{1.591211in}}%
\pgfpathmoveto{\pgfqpoint{5.109373in}{1.594160in}}%
\pgfpathlineto{\pgfqpoint{5.109373in}{1.594160in}}%
\pgfpathlineto{\pgfqpoint{5.109373in}{1.597109in}}%
\pgfpathlineto{\pgfqpoint{5.113914in}{1.597109in}}%
\pgfpathlineto{\pgfqpoint{5.113914in}{1.594160in}}%
\pgfpathmoveto{\pgfqpoint{5.113914in}{1.591211in}}%
\pgfpathlineto{\pgfqpoint{5.113914in}{1.591211in}}%
\pgfpathlineto{\pgfqpoint{5.113914in}{1.594160in}}%
\pgfpathlineto{\pgfqpoint{5.118455in}{1.594160in}}%
\pgfpathlineto{\pgfqpoint{5.118455in}{1.591211in}}%
\pgfpathmoveto{\pgfqpoint{5.118455in}{1.585313in}}%
\pgfpathlineto{\pgfqpoint{5.118455in}{1.585313in}}%
\pgfpathlineto{\pgfqpoint{5.118455in}{1.588262in}}%
\pgfpathlineto{\pgfqpoint{5.122996in}{1.588262in}}%
\pgfpathlineto{\pgfqpoint{5.122996in}{1.585313in}}%
\pgfpathmoveto{\pgfqpoint{5.118455in}{1.588262in}}%
\pgfpathlineto{\pgfqpoint{5.118455in}{1.588262in}}%
\pgfpathlineto{\pgfqpoint{5.118455in}{1.591211in}}%
\pgfpathlineto{\pgfqpoint{5.122996in}{1.591211in}}%
\pgfpathlineto{\pgfqpoint{5.122996in}{1.588262in}}%
\pgfpathmoveto{\pgfqpoint{5.122996in}{1.585313in}}%
\pgfpathlineto{\pgfqpoint{5.122996in}{1.585313in}}%
\pgfpathlineto{\pgfqpoint{5.122996in}{1.588262in}}%
\pgfpathlineto{\pgfqpoint{5.127537in}{1.588262in}}%
\pgfpathlineto{\pgfqpoint{5.127537in}{1.585313in}}%
\pgfpathmoveto{\pgfqpoint{5.186570in}{1.541076in}}%
\pgfpathlineto{\pgfqpoint{5.186570in}{1.541076in}}%
\pgfpathlineto{\pgfqpoint{5.186570in}{1.544025in}}%
\pgfpathlineto{\pgfqpoint{5.191111in}{1.544025in}}%
\pgfpathlineto{\pgfqpoint{5.191111in}{1.541076in}}%
\pgfpathmoveto{\pgfqpoint{5.182029in}{1.544025in}}%
\pgfpathlineto{\pgfqpoint{5.182029in}{1.544025in}}%
\pgfpathlineto{\pgfqpoint{5.182029in}{1.546974in}}%
\pgfpathlineto{\pgfqpoint{5.186570in}{1.546974in}}%
\pgfpathlineto{\pgfqpoint{5.186570in}{1.544025in}}%
\pgfpathmoveto{\pgfqpoint{5.182029in}{1.546974in}}%
\pgfpathlineto{\pgfqpoint{5.182029in}{1.546974in}}%
\pgfpathlineto{\pgfqpoint{5.182029in}{1.549923in}}%
\pgfpathlineto{\pgfqpoint{5.186570in}{1.549923in}}%
\pgfpathlineto{\pgfqpoint{5.186570in}{1.546974in}}%
\pgfpathmoveto{\pgfqpoint{5.186570in}{1.544025in}}%
\pgfpathlineto{\pgfqpoint{5.186570in}{1.544025in}}%
\pgfpathlineto{\pgfqpoint{5.186570in}{1.546974in}}%
\pgfpathlineto{\pgfqpoint{5.191111in}{1.546974in}}%
\pgfpathlineto{\pgfqpoint{5.191111in}{1.544025in}}%
\pgfpathmoveto{\pgfqpoint{5.191111in}{1.538127in}}%
\pgfpathlineto{\pgfqpoint{5.191111in}{1.538127in}}%
\pgfpathlineto{\pgfqpoint{5.191111in}{1.541076in}}%
\pgfpathlineto{\pgfqpoint{5.195652in}{1.541076in}}%
\pgfpathlineto{\pgfqpoint{5.195652in}{1.538127in}}%
\pgfpathmoveto{\pgfqpoint{5.191111in}{1.541076in}}%
\pgfpathlineto{\pgfqpoint{5.191111in}{1.541076in}}%
\pgfpathlineto{\pgfqpoint{5.191111in}{1.544025in}}%
\pgfpathlineto{\pgfqpoint{5.195652in}{1.544025in}}%
\pgfpathlineto{\pgfqpoint{5.195652in}{1.541076in}}%
\pgfpathmoveto{\pgfqpoint{5.195652in}{1.538127in}}%
\pgfpathlineto{\pgfqpoint{5.195652in}{1.538127in}}%
\pgfpathlineto{\pgfqpoint{5.195652in}{1.541076in}}%
\pgfpathlineto{\pgfqpoint{5.200193in}{1.541076in}}%
\pgfpathlineto{\pgfqpoint{5.200193in}{1.538127in}}%
\pgfpathmoveto{\pgfqpoint{5.340968in}{1.440801in}}%
\pgfpathlineto{\pgfqpoint{5.340968in}{1.440801in}}%
\pgfpathlineto{\pgfqpoint{5.340968in}{1.443750in}}%
\pgfpathlineto{\pgfqpoint{5.345509in}{1.443750in}}%
\pgfpathlineto{\pgfqpoint{5.345509in}{1.440801in}}%
\pgfpathmoveto{\pgfqpoint{5.359133in}{1.429004in}}%
\pgfpathlineto{\pgfqpoint{5.359133in}{1.429004in}}%
\pgfpathlineto{\pgfqpoint{5.359133in}{1.431953in}}%
\pgfpathlineto{\pgfqpoint{5.363674in}{1.431953in}}%
\pgfpathlineto{\pgfqpoint{5.363674in}{1.429004in}}%
\pgfpathmoveto{\pgfqpoint{5.350050in}{1.434903in}}%
\pgfpathlineto{\pgfqpoint{5.350050in}{1.434903in}}%
\pgfpathlineto{\pgfqpoint{5.350050in}{1.437852in}}%
\pgfpathlineto{\pgfqpoint{5.354592in}{1.437852in}}%
\pgfpathlineto{\pgfqpoint{5.354592in}{1.434903in}}%
\pgfpathmoveto{\pgfqpoint{5.345509in}{1.437852in}}%
\pgfpathlineto{\pgfqpoint{5.345509in}{1.437852in}}%
\pgfpathlineto{\pgfqpoint{5.345509in}{1.440801in}}%
\pgfpathlineto{\pgfqpoint{5.350050in}{1.440801in}}%
\pgfpathlineto{\pgfqpoint{5.350050in}{1.437852in}}%
\pgfpathmoveto{\pgfqpoint{5.345509in}{1.440801in}}%
\pgfpathlineto{\pgfqpoint{5.345509in}{1.440801in}}%
\pgfpathlineto{\pgfqpoint{5.345509in}{1.443750in}}%
\pgfpathlineto{\pgfqpoint{5.350050in}{1.443750in}}%
\pgfpathlineto{\pgfqpoint{5.350050in}{1.440801in}}%
\pgfpathmoveto{\pgfqpoint{5.350050in}{1.437852in}}%
\pgfpathlineto{\pgfqpoint{5.350050in}{1.437852in}}%
\pgfpathlineto{\pgfqpoint{5.350050in}{1.440801in}}%
\pgfpathlineto{\pgfqpoint{5.354592in}{1.440801in}}%
\pgfpathlineto{\pgfqpoint{5.354592in}{1.437852in}}%
\pgfpathmoveto{\pgfqpoint{5.354592in}{1.431953in}}%
\pgfpathlineto{\pgfqpoint{5.354592in}{1.431953in}}%
\pgfpathlineto{\pgfqpoint{5.354592in}{1.434903in}}%
\pgfpathlineto{\pgfqpoint{5.359133in}{1.434903in}}%
\pgfpathlineto{\pgfqpoint{5.359133in}{1.431953in}}%
\pgfpathmoveto{\pgfqpoint{5.354592in}{1.434903in}}%
\pgfpathlineto{\pgfqpoint{5.354592in}{1.434903in}}%
\pgfpathlineto{\pgfqpoint{5.354592in}{1.437852in}}%
\pgfpathlineto{\pgfqpoint{5.359133in}{1.437852in}}%
\pgfpathlineto{\pgfqpoint{5.359133in}{1.434903in}}%
\pgfpathmoveto{\pgfqpoint{5.359133in}{1.431953in}}%
\pgfpathlineto{\pgfqpoint{5.359133in}{1.431953in}}%
\pgfpathlineto{\pgfqpoint{5.359133in}{1.434903in}}%
\pgfpathlineto{\pgfqpoint{5.363674in}{1.434903in}}%
\pgfpathlineto{\pgfqpoint{5.363674in}{1.431953in}}%
\pgfpathmoveto{\pgfqpoint{5.377298in}{1.417207in}}%
\pgfpathlineto{\pgfqpoint{5.377298in}{1.417207in}}%
\pgfpathlineto{\pgfqpoint{5.377298in}{1.420157in}}%
\pgfpathlineto{\pgfqpoint{5.381839in}{1.420157in}}%
\pgfpathlineto{\pgfqpoint{5.381839in}{1.417207in}}%
\pgfpathmoveto{\pgfqpoint{5.395462in}{1.405410in}}%
\pgfpathlineto{\pgfqpoint{5.395462in}{1.405410in}}%
\pgfpathlineto{\pgfqpoint{5.395462in}{1.408360in}}%
\pgfpathlineto{\pgfqpoint{5.400004in}{1.408360in}}%
\pgfpathlineto{\pgfqpoint{5.400004in}{1.405410in}}%
\pgfpathmoveto{\pgfqpoint{5.386380in}{1.411309in}}%
\pgfpathlineto{\pgfqpoint{5.386380in}{1.411309in}}%
\pgfpathlineto{\pgfqpoint{5.386380in}{1.414258in}}%
\pgfpathlineto{\pgfqpoint{5.390921in}{1.414258in}}%
\pgfpathlineto{\pgfqpoint{5.390921in}{1.411309in}}%
\pgfpathmoveto{\pgfqpoint{5.381839in}{1.414258in}}%
\pgfpathlineto{\pgfqpoint{5.381839in}{1.414258in}}%
\pgfpathlineto{\pgfqpoint{5.381839in}{1.417207in}}%
\pgfpathlineto{\pgfqpoint{5.386380in}{1.417207in}}%
\pgfpathlineto{\pgfqpoint{5.386380in}{1.414258in}}%
\pgfpathmoveto{\pgfqpoint{5.381839in}{1.417207in}}%
\pgfpathlineto{\pgfqpoint{5.381839in}{1.417207in}}%
\pgfpathlineto{\pgfqpoint{5.381839in}{1.420157in}}%
\pgfpathlineto{\pgfqpoint{5.386380in}{1.420157in}}%
\pgfpathlineto{\pgfqpoint{5.386380in}{1.417207in}}%
\pgfpathmoveto{\pgfqpoint{5.386380in}{1.414258in}}%
\pgfpathlineto{\pgfqpoint{5.386380in}{1.414258in}}%
\pgfpathlineto{\pgfqpoint{5.386380in}{1.417207in}}%
\pgfpathlineto{\pgfqpoint{5.390921in}{1.417207in}}%
\pgfpathlineto{\pgfqpoint{5.390921in}{1.414258in}}%
\pgfpathmoveto{\pgfqpoint{5.390921in}{1.408360in}}%
\pgfpathlineto{\pgfqpoint{5.390921in}{1.408360in}}%
\pgfpathlineto{\pgfqpoint{5.390921in}{1.411309in}}%
\pgfpathlineto{\pgfqpoint{5.395462in}{1.411309in}}%
\pgfpathlineto{\pgfqpoint{5.395462in}{1.408360in}}%
\pgfpathmoveto{\pgfqpoint{5.390921in}{1.411309in}}%
\pgfpathlineto{\pgfqpoint{5.390921in}{1.411309in}}%
\pgfpathlineto{\pgfqpoint{5.390921in}{1.414258in}}%
\pgfpathlineto{\pgfqpoint{5.395462in}{1.414258in}}%
\pgfpathlineto{\pgfqpoint{5.395462in}{1.411309in}}%
\pgfpathmoveto{\pgfqpoint{5.395462in}{1.408360in}}%
\pgfpathlineto{\pgfqpoint{5.395462in}{1.408360in}}%
\pgfpathlineto{\pgfqpoint{5.395462in}{1.411309in}}%
\pgfpathlineto{\pgfqpoint{5.400004in}{1.411309in}}%
\pgfpathlineto{\pgfqpoint{5.400004in}{1.408360in}}%
\pgfpathmoveto{\pgfqpoint{5.368215in}{1.423106in}}%
\pgfpathlineto{\pgfqpoint{5.368215in}{1.423106in}}%
\pgfpathlineto{\pgfqpoint{5.368215in}{1.426055in}}%
\pgfpathlineto{\pgfqpoint{5.372756in}{1.426055in}}%
\pgfpathlineto{\pgfqpoint{5.372756in}{1.423106in}}%
\pgfpathmoveto{\pgfqpoint{5.363674in}{1.426055in}}%
\pgfpathlineto{\pgfqpoint{5.363674in}{1.426055in}}%
\pgfpathlineto{\pgfqpoint{5.363674in}{1.429004in}}%
\pgfpathlineto{\pgfqpoint{5.368215in}{1.429004in}}%
\pgfpathlineto{\pgfqpoint{5.368215in}{1.426055in}}%
\pgfpathmoveto{\pgfqpoint{5.363674in}{1.429004in}}%
\pgfpathlineto{\pgfqpoint{5.363674in}{1.429004in}}%
\pgfpathlineto{\pgfqpoint{5.363674in}{1.431953in}}%
\pgfpathlineto{\pgfqpoint{5.368215in}{1.431953in}}%
\pgfpathlineto{\pgfqpoint{5.368215in}{1.429004in}}%
\pgfpathmoveto{\pgfqpoint{5.368215in}{1.426055in}}%
\pgfpathlineto{\pgfqpoint{5.368215in}{1.426055in}}%
\pgfpathlineto{\pgfqpoint{5.368215in}{1.429004in}}%
\pgfpathlineto{\pgfqpoint{5.372756in}{1.429004in}}%
\pgfpathlineto{\pgfqpoint{5.372756in}{1.426055in}}%
\pgfpathmoveto{\pgfqpoint{5.372756in}{1.420157in}}%
\pgfpathlineto{\pgfqpoint{5.372756in}{1.420157in}}%
\pgfpathlineto{\pgfqpoint{5.372756in}{1.423106in}}%
\pgfpathlineto{\pgfqpoint{5.377298in}{1.423106in}}%
\pgfpathlineto{\pgfqpoint{5.377298in}{1.420157in}}%
\pgfpathmoveto{\pgfqpoint{5.372756in}{1.423106in}}%
\pgfpathlineto{\pgfqpoint{5.372756in}{1.423106in}}%
\pgfpathlineto{\pgfqpoint{5.372756in}{1.426055in}}%
\pgfpathlineto{\pgfqpoint{5.377298in}{1.426055in}}%
\pgfpathlineto{\pgfqpoint{5.377298in}{1.423106in}}%
\pgfpathmoveto{\pgfqpoint{5.377298in}{1.420157in}}%
\pgfpathlineto{\pgfqpoint{5.377298in}{1.420157in}}%
\pgfpathlineto{\pgfqpoint{5.377298in}{1.423106in}}%
\pgfpathlineto{\pgfqpoint{5.381839in}{1.423106in}}%
\pgfpathlineto{\pgfqpoint{5.381839in}{1.420157in}}%
\pgfpathmoveto{\pgfqpoint{5.268309in}{1.487989in}}%
\pgfpathlineto{\pgfqpoint{5.268309in}{1.487989in}}%
\pgfpathlineto{\pgfqpoint{5.268309in}{1.490938in}}%
\pgfpathlineto{\pgfqpoint{5.272850in}{1.490938in}}%
\pgfpathlineto{\pgfqpoint{5.272850in}{1.487989in}}%
\pgfpathmoveto{\pgfqpoint{5.286474in}{1.476192in}}%
\pgfpathlineto{\pgfqpoint{5.286474in}{1.476192in}}%
\pgfpathlineto{\pgfqpoint{5.286474in}{1.479141in}}%
\pgfpathlineto{\pgfqpoint{5.291015in}{1.479141in}}%
\pgfpathlineto{\pgfqpoint{5.291015in}{1.476192in}}%
\pgfpathmoveto{\pgfqpoint{5.277391in}{1.482091in}}%
\pgfpathlineto{\pgfqpoint{5.277391in}{1.482091in}}%
\pgfpathlineto{\pgfqpoint{5.277391in}{1.485040in}}%
\pgfpathlineto{\pgfqpoint{5.281932in}{1.485040in}}%
\pgfpathlineto{\pgfqpoint{5.281932in}{1.482091in}}%
\pgfpathmoveto{\pgfqpoint{5.272850in}{1.485040in}}%
\pgfpathlineto{\pgfqpoint{5.272850in}{1.485040in}}%
\pgfpathlineto{\pgfqpoint{5.272850in}{1.487989in}}%
\pgfpathlineto{\pgfqpoint{5.277391in}{1.487989in}}%
\pgfpathlineto{\pgfqpoint{5.277391in}{1.485040in}}%
\pgfpathmoveto{\pgfqpoint{5.272850in}{1.487989in}}%
\pgfpathlineto{\pgfqpoint{5.272850in}{1.487989in}}%
\pgfpathlineto{\pgfqpoint{5.272850in}{1.490938in}}%
\pgfpathlineto{\pgfqpoint{5.277391in}{1.490938in}}%
\pgfpathlineto{\pgfqpoint{5.277391in}{1.487989in}}%
\pgfpathmoveto{\pgfqpoint{5.277391in}{1.485040in}}%
\pgfpathlineto{\pgfqpoint{5.277391in}{1.485040in}}%
\pgfpathlineto{\pgfqpoint{5.277391in}{1.487989in}}%
\pgfpathlineto{\pgfqpoint{5.281932in}{1.487989in}}%
\pgfpathlineto{\pgfqpoint{5.281932in}{1.485040in}}%
\pgfpathmoveto{\pgfqpoint{5.281932in}{1.479141in}}%
\pgfpathlineto{\pgfqpoint{5.281932in}{1.479141in}}%
\pgfpathlineto{\pgfqpoint{5.281932in}{1.482091in}}%
\pgfpathlineto{\pgfqpoint{5.286474in}{1.482091in}}%
\pgfpathlineto{\pgfqpoint{5.286474in}{1.479141in}}%
\pgfpathmoveto{\pgfqpoint{5.281932in}{1.482091in}}%
\pgfpathlineto{\pgfqpoint{5.281932in}{1.482091in}}%
\pgfpathlineto{\pgfqpoint{5.281932in}{1.485040in}}%
\pgfpathlineto{\pgfqpoint{5.286474in}{1.485040in}}%
\pgfpathlineto{\pgfqpoint{5.286474in}{1.482091in}}%
\pgfpathmoveto{\pgfqpoint{5.286474in}{1.479141in}}%
\pgfpathlineto{\pgfqpoint{5.286474in}{1.479141in}}%
\pgfpathlineto{\pgfqpoint{5.286474in}{1.482091in}}%
\pgfpathlineto{\pgfqpoint{5.291015in}{1.482091in}}%
\pgfpathlineto{\pgfqpoint{5.291015in}{1.479141in}}%
\pgfpathmoveto{\pgfqpoint{5.304638in}{1.464395in}}%
\pgfpathlineto{\pgfqpoint{5.304638in}{1.464395in}}%
\pgfpathlineto{\pgfqpoint{5.304638in}{1.467344in}}%
\pgfpathlineto{\pgfqpoint{5.309180in}{1.467344in}}%
\pgfpathlineto{\pgfqpoint{5.309180in}{1.464395in}}%
\pgfpathmoveto{\pgfqpoint{5.322803in}{1.452598in}}%
\pgfpathlineto{\pgfqpoint{5.322803in}{1.452598in}}%
\pgfpathlineto{\pgfqpoint{5.322803in}{1.455547in}}%
\pgfpathlineto{\pgfqpoint{5.327344in}{1.455547in}}%
\pgfpathlineto{\pgfqpoint{5.327344in}{1.452598in}}%
\pgfpathmoveto{\pgfqpoint{5.313721in}{1.458497in}}%
\pgfpathlineto{\pgfqpoint{5.313721in}{1.458497in}}%
\pgfpathlineto{\pgfqpoint{5.313721in}{1.461446in}}%
\pgfpathlineto{\pgfqpoint{5.318262in}{1.461446in}}%
\pgfpathlineto{\pgfqpoint{5.318262in}{1.458497in}}%
\pgfpathmoveto{\pgfqpoint{5.309180in}{1.461446in}}%
\pgfpathlineto{\pgfqpoint{5.309180in}{1.461446in}}%
\pgfpathlineto{\pgfqpoint{5.309180in}{1.464395in}}%
\pgfpathlineto{\pgfqpoint{5.313721in}{1.464395in}}%
\pgfpathlineto{\pgfqpoint{5.313721in}{1.461446in}}%
\pgfpathmoveto{\pgfqpoint{5.309180in}{1.464395in}}%
\pgfpathlineto{\pgfqpoint{5.309180in}{1.464395in}}%
\pgfpathlineto{\pgfqpoint{5.309180in}{1.467344in}}%
\pgfpathlineto{\pgfqpoint{5.313721in}{1.467344in}}%
\pgfpathlineto{\pgfqpoint{5.313721in}{1.464395in}}%
\pgfpathmoveto{\pgfqpoint{5.313721in}{1.461446in}}%
\pgfpathlineto{\pgfqpoint{5.313721in}{1.461446in}}%
\pgfpathlineto{\pgfqpoint{5.313721in}{1.464395in}}%
\pgfpathlineto{\pgfqpoint{5.318262in}{1.464395in}}%
\pgfpathlineto{\pgfqpoint{5.318262in}{1.461446in}}%
\pgfpathmoveto{\pgfqpoint{5.318262in}{1.455547in}}%
\pgfpathlineto{\pgfqpoint{5.318262in}{1.455547in}}%
\pgfpathlineto{\pgfqpoint{5.318262in}{1.458497in}}%
\pgfpathlineto{\pgfqpoint{5.322803in}{1.458497in}}%
\pgfpathlineto{\pgfqpoint{5.322803in}{1.455547in}}%
\pgfpathmoveto{\pgfqpoint{5.318262in}{1.458497in}}%
\pgfpathlineto{\pgfqpoint{5.318262in}{1.458497in}}%
\pgfpathlineto{\pgfqpoint{5.318262in}{1.461446in}}%
\pgfpathlineto{\pgfqpoint{5.322803in}{1.461446in}}%
\pgfpathlineto{\pgfqpoint{5.322803in}{1.458497in}}%
\pgfpathmoveto{\pgfqpoint{5.322803in}{1.455547in}}%
\pgfpathlineto{\pgfqpoint{5.322803in}{1.455547in}}%
\pgfpathlineto{\pgfqpoint{5.322803in}{1.458497in}}%
\pgfpathlineto{\pgfqpoint{5.327344in}{1.458497in}}%
\pgfpathlineto{\pgfqpoint{5.327344in}{1.455547in}}%
\pgfpathmoveto{\pgfqpoint{5.295556in}{1.470294in}}%
\pgfpathlineto{\pgfqpoint{5.295556in}{1.470294in}}%
\pgfpathlineto{\pgfqpoint{5.295556in}{1.473243in}}%
\pgfpathlineto{\pgfqpoint{5.300097in}{1.473243in}}%
\pgfpathlineto{\pgfqpoint{5.300097in}{1.470294in}}%
\pgfpathmoveto{\pgfqpoint{5.291015in}{1.473243in}}%
\pgfpathlineto{\pgfqpoint{5.291015in}{1.473243in}}%
\pgfpathlineto{\pgfqpoint{5.291015in}{1.476192in}}%
\pgfpathlineto{\pgfqpoint{5.295556in}{1.476192in}}%
\pgfpathlineto{\pgfqpoint{5.295556in}{1.473243in}}%
\pgfpathmoveto{\pgfqpoint{5.291015in}{1.476192in}}%
\pgfpathlineto{\pgfqpoint{5.291015in}{1.476192in}}%
\pgfpathlineto{\pgfqpoint{5.291015in}{1.479141in}}%
\pgfpathlineto{\pgfqpoint{5.295556in}{1.479141in}}%
\pgfpathlineto{\pgfqpoint{5.295556in}{1.476192in}}%
\pgfpathmoveto{\pgfqpoint{5.295556in}{1.473243in}}%
\pgfpathlineto{\pgfqpoint{5.295556in}{1.473243in}}%
\pgfpathlineto{\pgfqpoint{5.295556in}{1.476192in}}%
\pgfpathlineto{\pgfqpoint{5.300097in}{1.476192in}}%
\pgfpathlineto{\pgfqpoint{5.300097in}{1.473243in}}%
\pgfpathmoveto{\pgfqpoint{5.300097in}{1.467344in}}%
\pgfpathlineto{\pgfqpoint{5.300097in}{1.467344in}}%
\pgfpathlineto{\pgfqpoint{5.300097in}{1.470294in}}%
\pgfpathlineto{\pgfqpoint{5.304638in}{1.470294in}}%
\pgfpathlineto{\pgfqpoint{5.304638in}{1.467344in}}%
\pgfpathmoveto{\pgfqpoint{5.300097in}{1.470294in}}%
\pgfpathlineto{\pgfqpoint{5.300097in}{1.470294in}}%
\pgfpathlineto{\pgfqpoint{5.300097in}{1.473243in}}%
\pgfpathlineto{\pgfqpoint{5.304638in}{1.473243in}}%
\pgfpathlineto{\pgfqpoint{5.304638in}{1.470294in}}%
\pgfpathmoveto{\pgfqpoint{5.304638in}{1.467344in}}%
\pgfpathlineto{\pgfqpoint{5.304638in}{1.467344in}}%
\pgfpathlineto{\pgfqpoint{5.304638in}{1.470294in}}%
\pgfpathlineto{\pgfqpoint{5.309180in}{1.470294in}}%
\pgfpathlineto{\pgfqpoint{5.309180in}{1.467344in}}%
\pgfpathmoveto{\pgfqpoint{5.259226in}{1.493888in}}%
\pgfpathlineto{\pgfqpoint{5.259226in}{1.493888in}}%
\pgfpathlineto{\pgfqpoint{5.259226in}{1.496837in}}%
\pgfpathlineto{\pgfqpoint{5.263768in}{1.496837in}}%
\pgfpathlineto{\pgfqpoint{5.263768in}{1.493888in}}%
\pgfpathmoveto{\pgfqpoint{5.254685in}{1.496837in}}%
\pgfpathlineto{\pgfqpoint{5.254685in}{1.496837in}}%
\pgfpathlineto{\pgfqpoint{5.254685in}{1.499786in}}%
\pgfpathlineto{\pgfqpoint{5.259226in}{1.499786in}}%
\pgfpathlineto{\pgfqpoint{5.259226in}{1.496837in}}%
\pgfpathmoveto{\pgfqpoint{5.254685in}{1.499786in}}%
\pgfpathlineto{\pgfqpoint{5.254685in}{1.499786in}}%
\pgfpathlineto{\pgfqpoint{5.254685in}{1.502736in}}%
\pgfpathlineto{\pgfqpoint{5.259226in}{1.502736in}}%
\pgfpathlineto{\pgfqpoint{5.259226in}{1.499786in}}%
\pgfpathmoveto{\pgfqpoint{5.259226in}{1.496837in}}%
\pgfpathlineto{\pgfqpoint{5.259226in}{1.496837in}}%
\pgfpathlineto{\pgfqpoint{5.259226in}{1.499786in}}%
\pgfpathlineto{\pgfqpoint{5.263768in}{1.499786in}}%
\pgfpathlineto{\pgfqpoint{5.263768in}{1.496837in}}%
\pgfpathmoveto{\pgfqpoint{5.263768in}{1.490938in}}%
\pgfpathlineto{\pgfqpoint{5.263768in}{1.490938in}}%
\pgfpathlineto{\pgfqpoint{5.263768in}{1.493888in}}%
\pgfpathlineto{\pgfqpoint{5.268309in}{1.493888in}}%
\pgfpathlineto{\pgfqpoint{5.268309in}{1.490938in}}%
\pgfpathmoveto{\pgfqpoint{5.263768in}{1.493888in}}%
\pgfpathlineto{\pgfqpoint{5.263768in}{1.493888in}}%
\pgfpathlineto{\pgfqpoint{5.263768in}{1.496837in}}%
\pgfpathlineto{\pgfqpoint{5.268309in}{1.496837in}}%
\pgfpathlineto{\pgfqpoint{5.268309in}{1.493888in}}%
\pgfpathmoveto{\pgfqpoint{5.268309in}{1.490938in}}%
\pgfpathlineto{\pgfqpoint{5.268309in}{1.490938in}}%
\pgfpathlineto{\pgfqpoint{5.268309in}{1.493888in}}%
\pgfpathlineto{\pgfqpoint{5.272850in}{1.493888in}}%
\pgfpathlineto{\pgfqpoint{5.272850in}{1.490938in}}%
\pgfpathmoveto{\pgfqpoint{5.331886in}{1.446700in}}%
\pgfpathlineto{\pgfqpoint{5.331886in}{1.446700in}}%
\pgfpathlineto{\pgfqpoint{5.331886in}{1.449649in}}%
\pgfpathlineto{\pgfqpoint{5.336427in}{1.449649in}}%
\pgfpathlineto{\pgfqpoint{5.336427in}{1.446700in}}%
\pgfpathmoveto{\pgfqpoint{5.327344in}{1.449649in}}%
\pgfpathlineto{\pgfqpoint{5.327344in}{1.449649in}}%
\pgfpathlineto{\pgfqpoint{5.327344in}{1.452598in}}%
\pgfpathlineto{\pgfqpoint{5.331886in}{1.452598in}}%
\pgfpathlineto{\pgfqpoint{5.331886in}{1.449649in}}%
\pgfpathmoveto{\pgfqpoint{5.327344in}{1.452598in}}%
\pgfpathlineto{\pgfqpoint{5.327344in}{1.452598in}}%
\pgfpathlineto{\pgfqpoint{5.327344in}{1.455547in}}%
\pgfpathlineto{\pgfqpoint{5.331886in}{1.455547in}}%
\pgfpathlineto{\pgfqpoint{5.331886in}{1.452598in}}%
\pgfpathmoveto{\pgfqpoint{5.331886in}{1.449649in}}%
\pgfpathlineto{\pgfqpoint{5.331886in}{1.449649in}}%
\pgfpathlineto{\pgfqpoint{5.331886in}{1.452598in}}%
\pgfpathlineto{\pgfqpoint{5.336427in}{1.452598in}}%
\pgfpathlineto{\pgfqpoint{5.336427in}{1.449649in}}%
\pgfpathmoveto{\pgfqpoint{5.336427in}{1.443750in}}%
\pgfpathlineto{\pgfqpoint{5.336427in}{1.443750in}}%
\pgfpathlineto{\pgfqpoint{5.336427in}{1.446700in}}%
\pgfpathlineto{\pgfqpoint{5.340968in}{1.446700in}}%
\pgfpathlineto{\pgfqpoint{5.340968in}{1.443750in}}%
\pgfpathmoveto{\pgfqpoint{5.336427in}{1.446700in}}%
\pgfpathlineto{\pgfqpoint{5.336427in}{1.446700in}}%
\pgfpathlineto{\pgfqpoint{5.336427in}{1.449649in}}%
\pgfpathlineto{\pgfqpoint{5.340968in}{1.449649in}}%
\pgfpathlineto{\pgfqpoint{5.340968in}{1.446700in}}%
\pgfpathmoveto{\pgfqpoint{5.340968in}{1.443750in}}%
\pgfpathlineto{\pgfqpoint{5.340968in}{1.443750in}}%
\pgfpathlineto{\pgfqpoint{5.340968in}{1.446700in}}%
\pgfpathlineto{\pgfqpoint{5.345509in}{1.446700in}}%
\pgfpathlineto{\pgfqpoint{5.345509in}{1.443750in}}%
\pgfpathclose%
\pgfusepath{fill}%
\end{pgfscope}%
\begin{pgfscope}%
\pgfpathrectangle{\pgfqpoint{0.750000in}{0.500000in}}{\pgfqpoint{4.650000in}{3.020000in}}%
\pgfusepath{clip}%
\pgfsetbuttcap%
\pgfsetmiterjoin%
\definecolor{currentfill}{rgb}{1.000000,0.000000,0.000000}%
\pgfsetfillcolor{currentfill}%
\pgfsetlinewidth{0.000000pt}%
\definecolor{currentstroke}{rgb}{0.000000,0.000000,0.000000}%
\pgfsetstrokecolor{currentstroke}%
\pgfsetstrokeopacity{0.000000}%
\pgfsetdash{}{0pt}%
\pgfpathmoveto{\pgfqpoint{3.070461in}{0.499998in}}%
\pgfpathlineto{\pgfqpoint{3.070461in}{0.502947in}}%
\pgfpathlineto{\pgfqpoint{3.075002in}{0.502947in}}%
\pgfpathlineto{\pgfqpoint{3.075002in}{0.499998in}}%
\pgfpathmoveto{\pgfqpoint{3.070461in}{0.502947in}}%
\pgfpathlineto{\pgfqpoint{3.070461in}{0.502947in}}%
\pgfpathlineto{\pgfqpoint{3.070461in}{0.505896in}}%
\pgfpathlineto{\pgfqpoint{3.075002in}{0.505896in}}%
\pgfpathlineto{\pgfqpoint{3.075002in}{0.502947in}}%
\pgfpathmoveto{\pgfqpoint{3.070461in}{0.505896in}}%
\pgfpathlineto{\pgfqpoint{3.070461in}{0.505896in}}%
\pgfpathlineto{\pgfqpoint{3.070461in}{0.508845in}}%
\pgfpathlineto{\pgfqpoint{3.075002in}{0.508845in}}%
\pgfpathlineto{\pgfqpoint{3.075002in}{0.505896in}}%
\pgfpathmoveto{\pgfqpoint{3.070461in}{0.508845in}}%
\pgfpathlineto{\pgfqpoint{3.070461in}{0.508845in}}%
\pgfpathlineto{\pgfqpoint{3.070461in}{0.511795in}}%
\pgfpathlineto{\pgfqpoint{3.075002in}{0.511795in}}%
\pgfpathlineto{\pgfqpoint{3.075002in}{0.508845in}}%
\pgfpathmoveto{\pgfqpoint{3.070461in}{0.511795in}}%
\pgfpathlineto{\pgfqpoint{3.070461in}{0.511795in}}%
\pgfpathlineto{\pgfqpoint{3.070461in}{0.514744in}}%
\pgfpathlineto{\pgfqpoint{3.075002in}{0.514744in}}%
\pgfpathlineto{\pgfqpoint{3.075002in}{0.511795in}}%
\pgfpathmoveto{\pgfqpoint{3.070461in}{0.514744in}}%
\pgfpathlineto{\pgfqpoint{3.070461in}{0.514744in}}%
\pgfpathlineto{\pgfqpoint{3.070461in}{0.517693in}}%
\pgfpathlineto{\pgfqpoint{3.075002in}{0.517693in}}%
\pgfpathlineto{\pgfqpoint{3.075002in}{0.514744in}}%
\pgfpathmoveto{\pgfqpoint{3.070461in}{0.517693in}}%
\pgfpathlineto{\pgfqpoint{3.070461in}{0.517693in}}%
\pgfpathlineto{\pgfqpoint{3.070461in}{0.520642in}}%
\pgfpathlineto{\pgfqpoint{3.075002in}{0.520642in}}%
\pgfpathlineto{\pgfqpoint{3.075002in}{0.517693in}}%
\pgfpathmoveto{\pgfqpoint{3.070461in}{0.520642in}}%
\pgfpathlineto{\pgfqpoint{3.070461in}{0.520642in}}%
\pgfpathlineto{\pgfqpoint{3.070461in}{0.523592in}}%
\pgfpathlineto{\pgfqpoint{3.075002in}{0.523592in}}%
\pgfpathlineto{\pgfqpoint{3.075002in}{0.520642in}}%
\pgfpathmoveto{\pgfqpoint{3.070461in}{0.523592in}}%
\pgfpathlineto{\pgfqpoint{3.070461in}{0.523592in}}%
\pgfpathlineto{\pgfqpoint{3.070461in}{0.526541in}}%
\pgfpathlineto{\pgfqpoint{3.075002in}{0.526541in}}%
\pgfpathlineto{\pgfqpoint{3.075002in}{0.523592in}}%
\pgfpathmoveto{\pgfqpoint{3.070461in}{0.526541in}}%
\pgfpathlineto{\pgfqpoint{3.070461in}{0.526541in}}%
\pgfpathlineto{\pgfqpoint{3.070461in}{0.529490in}}%
\pgfpathlineto{\pgfqpoint{3.075002in}{0.529490in}}%
\pgfpathlineto{\pgfqpoint{3.075002in}{0.526541in}}%
\pgfpathmoveto{\pgfqpoint{3.070461in}{0.529490in}}%
\pgfpathlineto{\pgfqpoint{3.070461in}{0.529490in}}%
\pgfpathlineto{\pgfqpoint{3.070461in}{0.532439in}}%
\pgfpathlineto{\pgfqpoint{3.075002in}{0.532439in}}%
\pgfpathlineto{\pgfqpoint{3.075002in}{0.529490in}}%
\pgfpathmoveto{\pgfqpoint{3.070461in}{0.532439in}}%
\pgfpathlineto{\pgfqpoint{3.070461in}{0.532439in}}%
\pgfpathlineto{\pgfqpoint{3.070461in}{0.535389in}}%
\pgfpathlineto{\pgfqpoint{3.075002in}{0.535389in}}%
\pgfpathlineto{\pgfqpoint{3.075002in}{0.532439in}}%
\pgfpathmoveto{\pgfqpoint{3.070461in}{0.535389in}}%
\pgfpathlineto{\pgfqpoint{3.070461in}{0.535389in}}%
\pgfpathlineto{\pgfqpoint{3.070461in}{0.538338in}}%
\pgfpathlineto{\pgfqpoint{3.075002in}{0.538338in}}%
\pgfpathlineto{\pgfqpoint{3.075002in}{0.535389in}}%
\pgfpathmoveto{\pgfqpoint{3.070461in}{0.538338in}}%
\pgfpathlineto{\pgfqpoint{3.070461in}{0.538338in}}%
\pgfpathlineto{\pgfqpoint{3.070461in}{0.541287in}}%
\pgfpathlineto{\pgfqpoint{3.075002in}{0.541287in}}%
\pgfpathlineto{\pgfqpoint{3.075002in}{0.538338in}}%
\pgfpathmoveto{\pgfqpoint{3.070461in}{0.541287in}}%
\pgfpathlineto{\pgfqpoint{3.070461in}{0.541287in}}%
\pgfpathlineto{\pgfqpoint{3.070461in}{0.544236in}}%
\pgfpathlineto{\pgfqpoint{3.075002in}{0.544236in}}%
\pgfpathlineto{\pgfqpoint{3.075002in}{0.541287in}}%
\pgfpathmoveto{\pgfqpoint{3.070461in}{0.544236in}}%
\pgfpathlineto{\pgfqpoint{3.070461in}{0.544236in}}%
\pgfpathlineto{\pgfqpoint{3.070461in}{0.547186in}}%
\pgfpathlineto{\pgfqpoint{3.075002in}{0.547186in}}%
\pgfpathlineto{\pgfqpoint{3.075002in}{0.544236in}}%
\pgfpathmoveto{\pgfqpoint{3.070461in}{0.547186in}}%
\pgfpathlineto{\pgfqpoint{3.070461in}{0.547186in}}%
\pgfpathlineto{\pgfqpoint{3.070461in}{0.550135in}}%
\pgfpathlineto{\pgfqpoint{3.075002in}{0.550135in}}%
\pgfpathlineto{\pgfqpoint{3.075002in}{0.547186in}}%
\pgfpathmoveto{\pgfqpoint{3.070461in}{0.550135in}}%
\pgfpathlineto{\pgfqpoint{3.070461in}{0.550135in}}%
\pgfpathlineto{\pgfqpoint{3.070461in}{0.553084in}}%
\pgfpathlineto{\pgfqpoint{3.075002in}{0.553084in}}%
\pgfpathlineto{\pgfqpoint{3.075002in}{0.550135in}}%
\pgfpathmoveto{\pgfqpoint{3.070461in}{0.553084in}}%
\pgfpathlineto{\pgfqpoint{3.070461in}{0.553084in}}%
\pgfpathlineto{\pgfqpoint{3.070461in}{0.556033in}}%
\pgfpathlineto{\pgfqpoint{3.075002in}{0.556033in}}%
\pgfpathlineto{\pgfqpoint{3.075002in}{0.553084in}}%
\pgfpathmoveto{\pgfqpoint{3.070461in}{0.556033in}}%
\pgfpathlineto{\pgfqpoint{3.070461in}{0.556033in}}%
\pgfpathlineto{\pgfqpoint{3.070461in}{0.558982in}}%
\pgfpathlineto{\pgfqpoint{3.075002in}{0.558982in}}%
\pgfpathlineto{\pgfqpoint{3.075002in}{0.556033in}}%
\pgfpathmoveto{\pgfqpoint{3.070461in}{0.558982in}}%
\pgfpathlineto{\pgfqpoint{3.070461in}{0.558982in}}%
\pgfpathlineto{\pgfqpoint{3.070461in}{0.561932in}}%
\pgfpathlineto{\pgfqpoint{3.075002in}{0.561932in}}%
\pgfpathlineto{\pgfqpoint{3.075002in}{0.558982in}}%
\pgfpathmoveto{\pgfqpoint{3.070461in}{0.561932in}}%
\pgfpathlineto{\pgfqpoint{3.070461in}{0.561932in}}%
\pgfpathlineto{\pgfqpoint{3.070461in}{0.564881in}}%
\pgfpathlineto{\pgfqpoint{3.075002in}{0.564881in}}%
\pgfpathlineto{\pgfqpoint{3.075002in}{0.561932in}}%
\pgfpathmoveto{\pgfqpoint{3.070461in}{0.564881in}}%
\pgfpathlineto{\pgfqpoint{3.070461in}{0.564881in}}%
\pgfpathlineto{\pgfqpoint{3.070461in}{0.567830in}}%
\pgfpathlineto{\pgfqpoint{3.075002in}{0.567830in}}%
\pgfpathlineto{\pgfqpoint{3.075002in}{0.564881in}}%
\pgfpathmoveto{\pgfqpoint{3.070461in}{0.567830in}}%
\pgfpathlineto{\pgfqpoint{3.070461in}{0.567830in}}%
\pgfpathlineto{\pgfqpoint{3.070461in}{0.570779in}}%
\pgfpathlineto{\pgfqpoint{3.075002in}{0.570779in}}%
\pgfpathlineto{\pgfqpoint{3.075002in}{0.567830in}}%
\pgfpathmoveto{\pgfqpoint{3.070461in}{0.570779in}}%
\pgfpathlineto{\pgfqpoint{3.070461in}{0.570779in}}%
\pgfpathlineto{\pgfqpoint{3.070461in}{0.573729in}}%
\pgfpathlineto{\pgfqpoint{3.075002in}{0.573729in}}%
\pgfpathlineto{\pgfqpoint{3.075002in}{0.570779in}}%
\pgfpathmoveto{\pgfqpoint{3.070461in}{0.573729in}}%
\pgfpathlineto{\pgfqpoint{3.070461in}{0.573729in}}%
\pgfpathlineto{\pgfqpoint{3.070461in}{0.576678in}}%
\pgfpathlineto{\pgfqpoint{3.075002in}{0.576678in}}%
\pgfpathlineto{\pgfqpoint{3.075002in}{0.573729in}}%
\pgfpathmoveto{\pgfqpoint{3.070461in}{0.576678in}}%
\pgfpathlineto{\pgfqpoint{3.070461in}{0.576678in}}%
\pgfpathlineto{\pgfqpoint{3.070461in}{0.579627in}}%
\pgfpathlineto{\pgfqpoint{3.075002in}{0.579627in}}%
\pgfpathlineto{\pgfqpoint{3.075002in}{0.576678in}}%
\pgfpathmoveto{\pgfqpoint{3.070461in}{0.579627in}}%
\pgfpathlineto{\pgfqpoint{3.070461in}{0.579627in}}%
\pgfpathlineto{\pgfqpoint{3.070461in}{0.582576in}}%
\pgfpathlineto{\pgfqpoint{3.075002in}{0.582576in}}%
\pgfpathlineto{\pgfqpoint{3.075002in}{0.579627in}}%
\pgfpathmoveto{\pgfqpoint{3.070461in}{0.582576in}}%
\pgfpathlineto{\pgfqpoint{3.070461in}{0.582576in}}%
\pgfpathlineto{\pgfqpoint{3.070461in}{0.585526in}}%
\pgfpathlineto{\pgfqpoint{3.075002in}{0.585526in}}%
\pgfpathlineto{\pgfqpoint{3.075002in}{0.582576in}}%
\pgfpathmoveto{\pgfqpoint{3.070461in}{0.585526in}}%
\pgfpathlineto{\pgfqpoint{3.070461in}{0.585526in}}%
\pgfpathlineto{\pgfqpoint{3.070461in}{0.588475in}}%
\pgfpathlineto{\pgfqpoint{3.075002in}{0.588475in}}%
\pgfpathlineto{\pgfqpoint{3.075002in}{0.585526in}}%
\pgfpathmoveto{\pgfqpoint{3.070461in}{0.588475in}}%
\pgfpathlineto{\pgfqpoint{3.070461in}{0.588475in}}%
\pgfpathlineto{\pgfqpoint{3.070461in}{0.591424in}}%
\pgfpathlineto{\pgfqpoint{3.075002in}{0.591424in}}%
\pgfpathlineto{\pgfqpoint{3.075002in}{0.588475in}}%
\pgfpathmoveto{\pgfqpoint{3.070461in}{0.591424in}}%
\pgfpathlineto{\pgfqpoint{3.070461in}{0.591424in}}%
\pgfpathlineto{\pgfqpoint{3.070461in}{0.594373in}}%
\pgfpathlineto{\pgfqpoint{3.075002in}{0.594373in}}%
\pgfpathlineto{\pgfqpoint{3.075002in}{0.591424in}}%
\pgfpathmoveto{\pgfqpoint{3.070461in}{0.594373in}}%
\pgfpathlineto{\pgfqpoint{3.070461in}{0.594373in}}%
\pgfpathlineto{\pgfqpoint{3.070461in}{0.597323in}}%
\pgfpathlineto{\pgfqpoint{3.075002in}{0.597323in}}%
\pgfpathlineto{\pgfqpoint{3.075002in}{0.594373in}}%
\pgfpathmoveto{\pgfqpoint{3.070461in}{0.597323in}}%
\pgfpathlineto{\pgfqpoint{3.070461in}{0.597323in}}%
\pgfpathlineto{\pgfqpoint{3.070461in}{0.600272in}}%
\pgfpathlineto{\pgfqpoint{3.075002in}{0.600272in}}%
\pgfpathlineto{\pgfqpoint{3.075002in}{0.597323in}}%
\pgfpathmoveto{\pgfqpoint{3.070461in}{0.600272in}}%
\pgfpathlineto{\pgfqpoint{3.070461in}{0.600272in}}%
\pgfpathlineto{\pgfqpoint{3.070461in}{0.603221in}}%
\pgfpathlineto{\pgfqpoint{3.075002in}{0.603221in}}%
\pgfpathlineto{\pgfqpoint{3.075002in}{0.600272in}}%
\pgfpathmoveto{\pgfqpoint{3.070461in}{0.603221in}}%
\pgfpathlineto{\pgfqpoint{3.070461in}{0.603221in}}%
\pgfpathlineto{\pgfqpoint{3.070461in}{0.606170in}}%
\pgfpathlineto{\pgfqpoint{3.075002in}{0.606170in}}%
\pgfpathlineto{\pgfqpoint{3.075002in}{0.603221in}}%
\pgfpathmoveto{\pgfqpoint{3.070461in}{0.606170in}}%
\pgfpathlineto{\pgfqpoint{3.070461in}{0.606170in}}%
\pgfpathlineto{\pgfqpoint{3.070461in}{0.609119in}}%
\pgfpathlineto{\pgfqpoint{3.075002in}{0.609119in}}%
\pgfpathlineto{\pgfqpoint{3.075002in}{0.606170in}}%
\pgfpathmoveto{\pgfqpoint{3.070461in}{0.609119in}}%
\pgfpathlineto{\pgfqpoint{3.070461in}{0.609119in}}%
\pgfpathlineto{\pgfqpoint{3.070461in}{0.612069in}}%
\pgfpathlineto{\pgfqpoint{3.075002in}{0.612069in}}%
\pgfpathlineto{\pgfqpoint{3.075002in}{0.609119in}}%
\pgfpathmoveto{\pgfqpoint{3.070461in}{0.612069in}}%
\pgfpathlineto{\pgfqpoint{3.070461in}{0.612069in}}%
\pgfpathlineto{\pgfqpoint{3.070461in}{0.615018in}}%
\pgfpathlineto{\pgfqpoint{3.075002in}{0.615018in}}%
\pgfpathlineto{\pgfqpoint{3.075002in}{0.612069in}}%
\pgfpathmoveto{\pgfqpoint{3.070461in}{0.615018in}}%
\pgfpathlineto{\pgfqpoint{3.070461in}{0.615018in}}%
\pgfpathlineto{\pgfqpoint{3.070461in}{0.617967in}}%
\pgfpathlineto{\pgfqpoint{3.075002in}{0.617967in}}%
\pgfpathlineto{\pgfqpoint{3.075002in}{0.615018in}}%
\pgfpathmoveto{\pgfqpoint{3.070461in}{0.617967in}}%
\pgfpathlineto{\pgfqpoint{3.070461in}{0.617967in}}%
\pgfpathlineto{\pgfqpoint{3.070461in}{0.620916in}}%
\pgfpathlineto{\pgfqpoint{3.075002in}{0.620916in}}%
\pgfpathlineto{\pgfqpoint{3.075002in}{0.617967in}}%
\pgfpathmoveto{\pgfqpoint{3.070461in}{0.620916in}}%
\pgfpathlineto{\pgfqpoint{3.070461in}{0.620916in}}%
\pgfpathlineto{\pgfqpoint{3.070461in}{0.623865in}}%
\pgfpathlineto{\pgfqpoint{3.075002in}{0.623865in}}%
\pgfpathlineto{\pgfqpoint{3.075002in}{0.620916in}}%
\pgfpathmoveto{\pgfqpoint{3.070461in}{0.623865in}}%
\pgfpathlineto{\pgfqpoint{3.070461in}{0.623865in}}%
\pgfpathlineto{\pgfqpoint{3.070461in}{0.626814in}}%
\pgfpathlineto{\pgfqpoint{3.075002in}{0.626814in}}%
\pgfpathlineto{\pgfqpoint{3.075002in}{0.623865in}}%
\pgfpathmoveto{\pgfqpoint{3.070461in}{0.626814in}}%
\pgfpathlineto{\pgfqpoint{3.070461in}{0.626814in}}%
\pgfpathlineto{\pgfqpoint{3.070461in}{0.629764in}}%
\pgfpathlineto{\pgfqpoint{3.075002in}{0.629764in}}%
\pgfpathlineto{\pgfqpoint{3.075002in}{0.626814in}}%
\pgfpathmoveto{\pgfqpoint{3.070461in}{0.629764in}}%
\pgfpathlineto{\pgfqpoint{3.070461in}{0.629764in}}%
\pgfpathlineto{\pgfqpoint{3.070461in}{0.632713in}}%
\pgfpathlineto{\pgfqpoint{3.075002in}{0.632713in}}%
\pgfpathlineto{\pgfqpoint{3.075002in}{0.629764in}}%
\pgfpathmoveto{\pgfqpoint{3.070461in}{0.632713in}}%
\pgfpathlineto{\pgfqpoint{3.070461in}{0.632713in}}%
\pgfpathlineto{\pgfqpoint{3.070461in}{0.635662in}}%
\pgfpathlineto{\pgfqpoint{3.075002in}{0.635662in}}%
\pgfpathlineto{\pgfqpoint{3.075002in}{0.632713in}}%
\pgfpathmoveto{\pgfqpoint{3.070461in}{0.635662in}}%
\pgfpathlineto{\pgfqpoint{3.070461in}{0.635662in}}%
\pgfpathlineto{\pgfqpoint{3.070461in}{0.638611in}}%
\pgfpathlineto{\pgfqpoint{3.075002in}{0.638611in}}%
\pgfpathlineto{\pgfqpoint{3.075002in}{0.635662in}}%
\pgfpathmoveto{\pgfqpoint{3.070461in}{0.638611in}}%
\pgfpathlineto{\pgfqpoint{3.070461in}{0.638611in}}%
\pgfpathlineto{\pgfqpoint{3.070461in}{0.641560in}}%
\pgfpathlineto{\pgfqpoint{3.075002in}{0.641560in}}%
\pgfpathlineto{\pgfqpoint{3.075002in}{0.638611in}}%
\pgfpathmoveto{\pgfqpoint{3.070461in}{0.641560in}}%
\pgfpathlineto{\pgfqpoint{3.070461in}{0.641560in}}%
\pgfpathlineto{\pgfqpoint{3.070461in}{0.644510in}}%
\pgfpathlineto{\pgfqpoint{3.075002in}{0.644510in}}%
\pgfpathlineto{\pgfqpoint{3.075002in}{0.641560in}}%
\pgfpathmoveto{\pgfqpoint{3.070461in}{0.644510in}}%
\pgfpathlineto{\pgfqpoint{3.070461in}{0.644510in}}%
\pgfpathlineto{\pgfqpoint{3.070461in}{0.647459in}}%
\pgfpathlineto{\pgfqpoint{3.075002in}{0.647459in}}%
\pgfpathlineto{\pgfqpoint{3.075002in}{0.644510in}}%
\pgfpathmoveto{\pgfqpoint{3.070461in}{0.647459in}}%
\pgfpathlineto{\pgfqpoint{3.070461in}{0.647459in}}%
\pgfpathlineto{\pgfqpoint{3.070461in}{0.650408in}}%
\pgfpathlineto{\pgfqpoint{3.075002in}{0.650408in}}%
\pgfpathlineto{\pgfqpoint{3.075002in}{0.647459in}}%
\pgfpathmoveto{\pgfqpoint{3.070461in}{0.650408in}}%
\pgfpathlineto{\pgfqpoint{3.070461in}{0.650408in}}%
\pgfpathlineto{\pgfqpoint{3.070461in}{0.653357in}}%
\pgfpathlineto{\pgfqpoint{3.075002in}{0.653357in}}%
\pgfpathlineto{\pgfqpoint{3.075002in}{0.650408in}}%
\pgfpathmoveto{\pgfqpoint{3.070461in}{0.653357in}}%
\pgfpathlineto{\pgfqpoint{3.070461in}{0.653357in}}%
\pgfpathlineto{\pgfqpoint{3.070461in}{0.656306in}}%
\pgfpathlineto{\pgfqpoint{3.075002in}{0.656306in}}%
\pgfpathlineto{\pgfqpoint{3.075002in}{0.653357in}}%
\pgfpathmoveto{\pgfqpoint{3.070461in}{0.656306in}}%
\pgfpathlineto{\pgfqpoint{3.070461in}{0.656306in}}%
\pgfpathlineto{\pgfqpoint{3.070461in}{0.659256in}}%
\pgfpathlineto{\pgfqpoint{3.075002in}{0.659256in}}%
\pgfpathlineto{\pgfqpoint{3.075002in}{0.656306in}}%
\pgfpathmoveto{\pgfqpoint{3.070461in}{0.659256in}}%
\pgfpathlineto{\pgfqpoint{3.070461in}{0.659256in}}%
\pgfpathlineto{\pgfqpoint{3.070461in}{0.662205in}}%
\pgfpathlineto{\pgfqpoint{3.075002in}{0.662205in}}%
\pgfpathlineto{\pgfqpoint{3.075002in}{0.659256in}}%
\pgfpathmoveto{\pgfqpoint{3.070461in}{0.662205in}}%
\pgfpathlineto{\pgfqpoint{3.070461in}{0.662205in}}%
\pgfpathlineto{\pgfqpoint{3.070461in}{0.665154in}}%
\pgfpathlineto{\pgfqpoint{3.075002in}{0.665154in}}%
\pgfpathlineto{\pgfqpoint{3.075002in}{0.662205in}}%
\pgfpathmoveto{\pgfqpoint{3.070461in}{0.665154in}}%
\pgfpathlineto{\pgfqpoint{3.070461in}{0.665154in}}%
\pgfpathlineto{\pgfqpoint{3.070461in}{0.668103in}}%
\pgfpathlineto{\pgfqpoint{3.075002in}{0.668103in}}%
\pgfpathlineto{\pgfqpoint{3.075002in}{0.665154in}}%
\pgfpathmoveto{\pgfqpoint{3.070461in}{0.668103in}}%
\pgfpathlineto{\pgfqpoint{3.070461in}{0.668103in}}%
\pgfpathlineto{\pgfqpoint{3.070461in}{0.671052in}}%
\pgfpathlineto{\pgfqpoint{3.075002in}{0.671052in}}%
\pgfpathlineto{\pgfqpoint{3.075002in}{0.668103in}}%
\pgfpathmoveto{\pgfqpoint{3.070461in}{0.671052in}}%
\pgfpathlineto{\pgfqpoint{3.070461in}{0.671052in}}%
\pgfpathlineto{\pgfqpoint{3.070461in}{0.674001in}}%
\pgfpathlineto{\pgfqpoint{3.075002in}{0.674001in}}%
\pgfpathlineto{\pgfqpoint{3.075002in}{0.671052in}}%
\pgfpathmoveto{\pgfqpoint{3.070461in}{0.674001in}}%
\pgfpathlineto{\pgfqpoint{3.070461in}{0.674001in}}%
\pgfpathlineto{\pgfqpoint{3.070461in}{0.676951in}}%
\pgfpathlineto{\pgfqpoint{3.075002in}{0.676951in}}%
\pgfpathlineto{\pgfqpoint{3.075002in}{0.674001in}}%
\pgfpathmoveto{\pgfqpoint{3.070461in}{0.676951in}}%
\pgfpathlineto{\pgfqpoint{3.070461in}{0.676951in}}%
\pgfpathlineto{\pgfqpoint{3.070461in}{0.679900in}}%
\pgfpathlineto{\pgfqpoint{3.075002in}{0.679900in}}%
\pgfpathlineto{\pgfqpoint{3.075002in}{0.676951in}}%
\pgfpathmoveto{\pgfqpoint{3.070461in}{0.679900in}}%
\pgfpathlineto{\pgfqpoint{3.070461in}{0.679900in}}%
\pgfpathlineto{\pgfqpoint{3.070461in}{0.682849in}}%
\pgfpathlineto{\pgfqpoint{3.075002in}{0.682849in}}%
\pgfpathlineto{\pgfqpoint{3.075002in}{0.679900in}}%
\pgfpathmoveto{\pgfqpoint{3.070461in}{0.682849in}}%
\pgfpathlineto{\pgfqpoint{3.070461in}{0.682849in}}%
\pgfpathlineto{\pgfqpoint{3.070461in}{0.685798in}}%
\pgfpathlineto{\pgfqpoint{3.075002in}{0.685798in}}%
\pgfpathlineto{\pgfqpoint{3.075002in}{0.682849in}}%
\pgfpathmoveto{\pgfqpoint{3.070461in}{0.685798in}}%
\pgfpathlineto{\pgfqpoint{3.070461in}{0.685798in}}%
\pgfpathlineto{\pgfqpoint{3.070461in}{0.688747in}}%
\pgfpathlineto{\pgfqpoint{3.075002in}{0.688747in}}%
\pgfpathlineto{\pgfqpoint{3.075002in}{0.685798in}}%
\pgfpathmoveto{\pgfqpoint{3.070461in}{0.688747in}}%
\pgfpathlineto{\pgfqpoint{3.070461in}{0.688747in}}%
\pgfpathlineto{\pgfqpoint{3.070461in}{0.691697in}}%
\pgfpathlineto{\pgfqpoint{3.075002in}{0.691697in}}%
\pgfpathlineto{\pgfqpoint{3.075002in}{0.688747in}}%
\pgfpathmoveto{\pgfqpoint{3.070461in}{0.691697in}}%
\pgfpathlineto{\pgfqpoint{3.070461in}{0.691697in}}%
\pgfpathlineto{\pgfqpoint{3.070461in}{0.694646in}}%
\pgfpathlineto{\pgfqpoint{3.075002in}{0.694646in}}%
\pgfpathlineto{\pgfqpoint{3.075002in}{0.691697in}}%
\pgfpathmoveto{\pgfqpoint{3.070461in}{0.694646in}}%
\pgfpathlineto{\pgfqpoint{3.070461in}{0.694646in}}%
\pgfpathlineto{\pgfqpoint{3.070461in}{0.697595in}}%
\pgfpathlineto{\pgfqpoint{3.075002in}{0.697595in}}%
\pgfpathlineto{\pgfqpoint{3.075002in}{0.694646in}}%
\pgfpathmoveto{\pgfqpoint{3.070461in}{0.697595in}}%
\pgfpathlineto{\pgfqpoint{3.070461in}{0.697595in}}%
\pgfpathlineto{\pgfqpoint{3.070461in}{0.700545in}}%
\pgfpathlineto{\pgfqpoint{3.075002in}{0.700545in}}%
\pgfpathlineto{\pgfqpoint{3.075002in}{0.697595in}}%
\pgfpathmoveto{\pgfqpoint{3.070461in}{0.700545in}}%
\pgfpathlineto{\pgfqpoint{3.070461in}{0.700545in}}%
\pgfpathlineto{\pgfqpoint{3.070461in}{0.703494in}}%
\pgfpathlineto{\pgfqpoint{3.075002in}{0.703494in}}%
\pgfpathlineto{\pgfqpoint{3.075002in}{0.700545in}}%
\pgfpathmoveto{\pgfqpoint{3.070461in}{0.703494in}}%
\pgfpathlineto{\pgfqpoint{3.070461in}{0.703494in}}%
\pgfpathlineto{\pgfqpoint{3.070461in}{0.706443in}}%
\pgfpathlineto{\pgfqpoint{3.075002in}{0.706443in}}%
\pgfpathlineto{\pgfqpoint{3.075002in}{0.703494in}}%
\pgfpathmoveto{\pgfqpoint{3.070461in}{0.706443in}}%
\pgfpathlineto{\pgfqpoint{3.070461in}{0.706443in}}%
\pgfpathlineto{\pgfqpoint{3.070461in}{0.709393in}}%
\pgfpathlineto{\pgfqpoint{3.075002in}{0.709393in}}%
\pgfpathlineto{\pgfqpoint{3.075002in}{0.706443in}}%
\pgfpathmoveto{\pgfqpoint{3.070461in}{0.709393in}}%
\pgfpathlineto{\pgfqpoint{3.070461in}{0.709393in}}%
\pgfpathlineto{\pgfqpoint{3.070461in}{0.712342in}}%
\pgfpathlineto{\pgfqpoint{3.075002in}{0.712342in}}%
\pgfpathlineto{\pgfqpoint{3.075002in}{0.709393in}}%
\pgfpathmoveto{\pgfqpoint{3.070461in}{0.712342in}}%
\pgfpathlineto{\pgfqpoint{3.070461in}{0.712342in}}%
\pgfpathlineto{\pgfqpoint{3.070461in}{0.715291in}}%
\pgfpathlineto{\pgfqpoint{3.075002in}{0.715291in}}%
\pgfpathlineto{\pgfqpoint{3.075002in}{0.712342in}}%
\pgfpathmoveto{\pgfqpoint{3.070461in}{0.715291in}}%
\pgfpathlineto{\pgfqpoint{3.070461in}{0.715291in}}%
\pgfpathlineto{\pgfqpoint{3.070461in}{0.718241in}}%
\pgfpathlineto{\pgfqpoint{3.075002in}{0.718241in}}%
\pgfpathlineto{\pgfqpoint{3.075002in}{0.715291in}}%
\pgfpathmoveto{\pgfqpoint{3.070461in}{0.718241in}}%
\pgfpathlineto{\pgfqpoint{3.070461in}{0.718241in}}%
\pgfpathlineto{\pgfqpoint{3.070461in}{0.721190in}}%
\pgfpathlineto{\pgfqpoint{3.075002in}{0.721190in}}%
\pgfpathlineto{\pgfqpoint{3.075002in}{0.718241in}}%
\pgfpathmoveto{\pgfqpoint{3.070461in}{0.721190in}}%
\pgfpathlineto{\pgfqpoint{3.070461in}{0.721190in}}%
\pgfpathlineto{\pgfqpoint{3.070461in}{0.724139in}}%
\pgfpathlineto{\pgfqpoint{3.075002in}{0.724139in}}%
\pgfpathlineto{\pgfqpoint{3.075002in}{0.721190in}}%
\pgfpathmoveto{\pgfqpoint{3.070461in}{0.724139in}}%
\pgfpathlineto{\pgfqpoint{3.070461in}{0.724139in}}%
\pgfpathlineto{\pgfqpoint{3.070461in}{0.727088in}}%
\pgfpathlineto{\pgfqpoint{3.075002in}{0.727088in}}%
\pgfpathlineto{\pgfqpoint{3.075002in}{0.724139in}}%
\pgfpathmoveto{\pgfqpoint{3.070461in}{0.727088in}}%
\pgfpathlineto{\pgfqpoint{3.070461in}{0.727088in}}%
\pgfpathlineto{\pgfqpoint{3.070461in}{0.730038in}}%
\pgfpathlineto{\pgfqpoint{3.075002in}{0.730038in}}%
\pgfpathlineto{\pgfqpoint{3.075002in}{0.727088in}}%
\pgfpathmoveto{\pgfqpoint{3.070461in}{0.730038in}}%
\pgfpathlineto{\pgfqpoint{3.070461in}{0.730038in}}%
\pgfpathlineto{\pgfqpoint{3.070461in}{0.732987in}}%
\pgfpathlineto{\pgfqpoint{3.075002in}{0.732987in}}%
\pgfpathlineto{\pgfqpoint{3.075002in}{0.730038in}}%
\pgfpathmoveto{\pgfqpoint{3.070461in}{0.732987in}}%
\pgfpathlineto{\pgfqpoint{3.070461in}{0.732987in}}%
\pgfpathlineto{\pgfqpoint{3.070461in}{0.735936in}}%
\pgfpathlineto{\pgfqpoint{3.075002in}{0.735936in}}%
\pgfpathlineto{\pgfqpoint{3.075002in}{0.732987in}}%
\pgfpathmoveto{\pgfqpoint{3.070461in}{0.735936in}}%
\pgfpathlineto{\pgfqpoint{3.070461in}{0.735936in}}%
\pgfpathlineto{\pgfqpoint{3.070461in}{0.738886in}}%
\pgfpathlineto{\pgfqpoint{3.075002in}{0.738886in}}%
\pgfpathlineto{\pgfqpoint{3.075002in}{0.735936in}}%
\pgfpathmoveto{\pgfqpoint{3.070461in}{0.738886in}}%
\pgfpathlineto{\pgfqpoint{3.070461in}{0.738886in}}%
\pgfpathlineto{\pgfqpoint{3.070461in}{0.741835in}}%
\pgfpathlineto{\pgfqpoint{3.075002in}{0.741835in}}%
\pgfpathlineto{\pgfqpoint{3.075002in}{0.738886in}}%
\pgfpathmoveto{\pgfqpoint{3.070461in}{0.741835in}}%
\pgfpathlineto{\pgfqpoint{3.070461in}{0.741835in}}%
\pgfpathlineto{\pgfqpoint{3.070461in}{0.744784in}}%
\pgfpathlineto{\pgfqpoint{3.075002in}{0.744784in}}%
\pgfpathlineto{\pgfqpoint{3.075002in}{0.741835in}}%
\pgfpathmoveto{\pgfqpoint{3.070461in}{0.744784in}}%
\pgfpathlineto{\pgfqpoint{3.070461in}{0.744784in}}%
\pgfpathlineto{\pgfqpoint{3.070461in}{0.747734in}}%
\pgfpathlineto{\pgfqpoint{3.075002in}{0.747734in}}%
\pgfpathlineto{\pgfqpoint{3.075002in}{0.744784in}}%
\pgfpathmoveto{\pgfqpoint{3.070461in}{0.747734in}}%
\pgfpathlineto{\pgfqpoint{3.070461in}{0.747734in}}%
\pgfpathlineto{\pgfqpoint{3.070461in}{0.750683in}}%
\pgfpathlineto{\pgfqpoint{3.075002in}{0.750683in}}%
\pgfpathlineto{\pgfqpoint{3.075002in}{0.747734in}}%
\pgfpathmoveto{\pgfqpoint{3.070461in}{0.750683in}}%
\pgfpathlineto{\pgfqpoint{3.070461in}{0.750683in}}%
\pgfpathlineto{\pgfqpoint{3.070461in}{0.753632in}}%
\pgfpathlineto{\pgfqpoint{3.075002in}{0.753632in}}%
\pgfpathlineto{\pgfqpoint{3.075002in}{0.750683in}}%
\pgfpathmoveto{\pgfqpoint{3.070461in}{0.753632in}}%
\pgfpathlineto{\pgfqpoint{3.070461in}{0.753632in}}%
\pgfpathlineto{\pgfqpoint{3.070461in}{0.756582in}}%
\pgfpathlineto{\pgfqpoint{3.075002in}{0.756582in}}%
\pgfpathlineto{\pgfqpoint{3.075002in}{0.753632in}}%
\pgfpathmoveto{\pgfqpoint{3.070461in}{0.756582in}}%
\pgfpathlineto{\pgfqpoint{3.070461in}{0.756582in}}%
\pgfpathlineto{\pgfqpoint{3.070461in}{0.759531in}}%
\pgfpathlineto{\pgfqpoint{3.075002in}{0.759531in}}%
\pgfpathlineto{\pgfqpoint{3.075002in}{0.756582in}}%
\pgfpathmoveto{\pgfqpoint{3.070461in}{0.759531in}}%
\pgfpathlineto{\pgfqpoint{3.070461in}{0.759531in}}%
\pgfpathlineto{\pgfqpoint{3.070461in}{0.762480in}}%
\pgfpathlineto{\pgfqpoint{3.075002in}{0.762480in}}%
\pgfpathlineto{\pgfqpoint{3.075002in}{0.759531in}}%
\pgfpathmoveto{\pgfqpoint{3.070461in}{0.762480in}}%
\pgfpathlineto{\pgfqpoint{3.070461in}{0.762480in}}%
\pgfpathlineto{\pgfqpoint{3.070461in}{0.765430in}}%
\pgfpathlineto{\pgfqpoint{3.075002in}{0.765430in}}%
\pgfpathlineto{\pgfqpoint{3.075002in}{0.762480in}}%
\pgfpathmoveto{\pgfqpoint{3.070461in}{0.765430in}}%
\pgfpathlineto{\pgfqpoint{3.070461in}{0.765430in}}%
\pgfpathlineto{\pgfqpoint{3.070461in}{0.768379in}}%
\pgfpathlineto{\pgfqpoint{3.075002in}{0.768379in}}%
\pgfpathlineto{\pgfqpoint{3.075002in}{0.765430in}}%
\pgfpathmoveto{\pgfqpoint{3.070461in}{0.768379in}}%
\pgfpathlineto{\pgfqpoint{3.070461in}{0.768379in}}%
\pgfpathlineto{\pgfqpoint{3.070461in}{0.771328in}}%
\pgfpathlineto{\pgfqpoint{3.075002in}{0.771328in}}%
\pgfpathlineto{\pgfqpoint{3.075002in}{0.768379in}}%
\pgfpathmoveto{\pgfqpoint{3.070461in}{0.771328in}}%
\pgfpathlineto{\pgfqpoint{3.070461in}{0.771328in}}%
\pgfpathlineto{\pgfqpoint{3.070461in}{0.774277in}}%
\pgfpathlineto{\pgfqpoint{3.075002in}{0.774277in}}%
\pgfpathlineto{\pgfqpoint{3.075002in}{0.771328in}}%
\pgfpathmoveto{\pgfqpoint{3.070461in}{0.774277in}}%
\pgfpathlineto{\pgfqpoint{3.070461in}{0.774277in}}%
\pgfpathlineto{\pgfqpoint{3.070461in}{0.777227in}}%
\pgfpathlineto{\pgfqpoint{3.075002in}{0.777227in}}%
\pgfpathlineto{\pgfqpoint{3.075002in}{0.774277in}}%
\pgfpathmoveto{\pgfqpoint{3.070461in}{0.777227in}}%
\pgfpathlineto{\pgfqpoint{3.070461in}{0.777227in}}%
\pgfpathlineto{\pgfqpoint{3.070461in}{0.780176in}}%
\pgfpathlineto{\pgfqpoint{3.075002in}{0.780176in}}%
\pgfpathlineto{\pgfqpoint{3.075002in}{0.777227in}}%
\pgfpathmoveto{\pgfqpoint{3.070461in}{0.780176in}}%
\pgfpathlineto{\pgfqpoint{3.070461in}{0.780176in}}%
\pgfpathlineto{\pgfqpoint{3.070461in}{0.783125in}}%
\pgfpathlineto{\pgfqpoint{3.075002in}{0.783125in}}%
\pgfpathlineto{\pgfqpoint{3.075002in}{0.780176in}}%
\pgfpathmoveto{\pgfqpoint{3.070461in}{0.783125in}}%
\pgfpathlineto{\pgfqpoint{3.070461in}{0.783125in}}%
\pgfpathlineto{\pgfqpoint{3.070461in}{0.786075in}}%
\pgfpathlineto{\pgfqpoint{3.075002in}{0.786075in}}%
\pgfpathlineto{\pgfqpoint{3.075002in}{0.783125in}}%
\pgfpathmoveto{\pgfqpoint{3.070461in}{0.786075in}}%
\pgfpathlineto{\pgfqpoint{3.070461in}{0.786075in}}%
\pgfpathlineto{\pgfqpoint{3.070461in}{0.789024in}}%
\pgfpathlineto{\pgfqpoint{3.075002in}{0.789024in}}%
\pgfpathlineto{\pgfqpoint{3.075002in}{0.786075in}}%
\pgfpathmoveto{\pgfqpoint{3.070461in}{0.789024in}}%
\pgfpathlineto{\pgfqpoint{3.070461in}{0.789024in}}%
\pgfpathlineto{\pgfqpoint{3.070461in}{0.791973in}}%
\pgfpathlineto{\pgfqpoint{3.075002in}{0.791973in}}%
\pgfpathlineto{\pgfqpoint{3.075002in}{0.789024in}}%
\pgfpathmoveto{\pgfqpoint{3.070461in}{0.791973in}}%
\pgfpathlineto{\pgfqpoint{3.070461in}{0.791973in}}%
\pgfpathlineto{\pgfqpoint{3.070461in}{0.794922in}}%
\pgfpathlineto{\pgfqpoint{3.075002in}{0.794922in}}%
\pgfpathlineto{\pgfqpoint{3.075002in}{0.791973in}}%
\pgfpathmoveto{\pgfqpoint{3.070461in}{0.794922in}}%
\pgfpathlineto{\pgfqpoint{3.070461in}{0.794922in}}%
\pgfpathlineto{\pgfqpoint{3.070461in}{0.797871in}}%
\pgfpathlineto{\pgfqpoint{3.075002in}{0.797871in}}%
\pgfpathlineto{\pgfqpoint{3.075002in}{0.794922in}}%
\pgfpathmoveto{\pgfqpoint{3.070461in}{0.797871in}}%
\pgfpathlineto{\pgfqpoint{3.070461in}{0.797871in}}%
\pgfpathlineto{\pgfqpoint{3.070461in}{0.800820in}}%
\pgfpathlineto{\pgfqpoint{3.075002in}{0.800820in}}%
\pgfpathlineto{\pgfqpoint{3.075002in}{0.797871in}}%
\pgfpathmoveto{\pgfqpoint{3.070461in}{0.800820in}}%
\pgfpathlineto{\pgfqpoint{3.070461in}{0.800820in}}%
\pgfpathlineto{\pgfqpoint{3.070461in}{0.803770in}}%
\pgfpathlineto{\pgfqpoint{3.075002in}{0.803770in}}%
\pgfpathlineto{\pgfqpoint{3.075002in}{0.800820in}}%
\pgfpathmoveto{\pgfqpoint{3.070461in}{0.803770in}}%
\pgfpathlineto{\pgfqpoint{3.070461in}{0.803770in}}%
\pgfpathlineto{\pgfqpoint{3.070461in}{0.806719in}}%
\pgfpathlineto{\pgfqpoint{3.075002in}{0.806719in}}%
\pgfpathlineto{\pgfqpoint{3.075002in}{0.803770in}}%
\pgfpathmoveto{\pgfqpoint{3.070461in}{0.806719in}}%
\pgfpathlineto{\pgfqpoint{3.070461in}{0.806719in}}%
\pgfpathlineto{\pgfqpoint{3.070461in}{0.809668in}}%
\pgfpathlineto{\pgfqpoint{3.075002in}{0.809668in}}%
\pgfpathlineto{\pgfqpoint{3.075002in}{0.806719in}}%
\pgfpathmoveto{\pgfqpoint{3.070461in}{0.809668in}}%
\pgfpathlineto{\pgfqpoint{3.070461in}{0.809668in}}%
\pgfpathlineto{\pgfqpoint{3.070461in}{0.812617in}}%
\pgfpathlineto{\pgfqpoint{3.075002in}{0.812617in}}%
\pgfpathlineto{\pgfqpoint{3.075002in}{0.809668in}}%
\pgfpathmoveto{\pgfqpoint{3.070461in}{0.812617in}}%
\pgfpathlineto{\pgfqpoint{3.070461in}{0.812617in}}%
\pgfpathlineto{\pgfqpoint{3.070461in}{0.815566in}}%
\pgfpathlineto{\pgfqpoint{3.075002in}{0.815566in}}%
\pgfpathlineto{\pgfqpoint{3.075002in}{0.812617in}}%
\pgfpathmoveto{\pgfqpoint{3.070461in}{0.815566in}}%
\pgfpathlineto{\pgfqpoint{3.070461in}{0.815566in}}%
\pgfpathlineto{\pgfqpoint{3.070461in}{0.818515in}}%
\pgfpathlineto{\pgfqpoint{3.075002in}{0.818515in}}%
\pgfpathlineto{\pgfqpoint{3.075002in}{0.815566in}}%
\pgfpathmoveto{\pgfqpoint{3.070461in}{0.818515in}}%
\pgfpathlineto{\pgfqpoint{3.070461in}{0.818515in}}%
\pgfpathlineto{\pgfqpoint{3.070461in}{0.821465in}}%
\pgfpathlineto{\pgfqpoint{3.075002in}{0.821465in}}%
\pgfpathlineto{\pgfqpoint{3.075002in}{0.818515in}}%
\pgfpathmoveto{\pgfqpoint{3.070461in}{0.821465in}}%
\pgfpathlineto{\pgfqpoint{3.070461in}{0.821465in}}%
\pgfpathlineto{\pgfqpoint{3.070461in}{0.824414in}}%
\pgfpathlineto{\pgfqpoint{3.075002in}{0.824414in}}%
\pgfpathlineto{\pgfqpoint{3.075002in}{0.821465in}}%
\pgfpathmoveto{\pgfqpoint{3.070461in}{0.824414in}}%
\pgfpathlineto{\pgfqpoint{3.070461in}{0.824414in}}%
\pgfpathlineto{\pgfqpoint{3.070461in}{0.827363in}}%
\pgfpathlineto{\pgfqpoint{3.075002in}{0.827363in}}%
\pgfpathlineto{\pgfqpoint{3.075002in}{0.824414in}}%
\pgfpathmoveto{\pgfqpoint{3.070461in}{0.827363in}}%
\pgfpathlineto{\pgfqpoint{3.070461in}{0.827363in}}%
\pgfpathlineto{\pgfqpoint{3.070461in}{0.830312in}}%
\pgfpathlineto{\pgfqpoint{3.075002in}{0.830312in}}%
\pgfpathlineto{\pgfqpoint{3.075002in}{0.827363in}}%
\pgfpathmoveto{\pgfqpoint{3.070461in}{0.830312in}}%
\pgfpathlineto{\pgfqpoint{3.070461in}{0.830312in}}%
\pgfpathlineto{\pgfqpoint{3.070461in}{0.833261in}}%
\pgfpathlineto{\pgfqpoint{3.075002in}{0.833261in}}%
\pgfpathlineto{\pgfqpoint{3.075002in}{0.830312in}}%
\pgfpathmoveto{\pgfqpoint{3.070461in}{0.833261in}}%
\pgfpathlineto{\pgfqpoint{3.070461in}{0.833261in}}%
\pgfpathlineto{\pgfqpoint{3.070461in}{0.836210in}}%
\pgfpathlineto{\pgfqpoint{3.075002in}{0.836210in}}%
\pgfpathlineto{\pgfqpoint{3.075002in}{0.833261in}}%
\pgfpathmoveto{\pgfqpoint{3.070461in}{0.836210in}}%
\pgfpathlineto{\pgfqpoint{3.070461in}{0.836210in}}%
\pgfpathlineto{\pgfqpoint{3.070461in}{0.839160in}}%
\pgfpathlineto{\pgfqpoint{3.075002in}{0.839160in}}%
\pgfpathlineto{\pgfqpoint{3.075002in}{0.836210in}}%
\pgfpathmoveto{\pgfqpoint{3.070461in}{0.839160in}}%
\pgfpathlineto{\pgfqpoint{3.070461in}{0.839160in}}%
\pgfpathlineto{\pgfqpoint{3.070461in}{0.842109in}}%
\pgfpathlineto{\pgfqpoint{3.075002in}{0.842109in}}%
\pgfpathlineto{\pgfqpoint{3.075002in}{0.839160in}}%
\pgfpathmoveto{\pgfqpoint{3.070461in}{0.842109in}}%
\pgfpathlineto{\pgfqpoint{3.070461in}{0.842109in}}%
\pgfpathlineto{\pgfqpoint{3.070461in}{0.845058in}}%
\pgfpathlineto{\pgfqpoint{3.075002in}{0.845058in}}%
\pgfpathlineto{\pgfqpoint{3.075002in}{0.842109in}}%
\pgfpathmoveto{\pgfqpoint{3.070461in}{0.845058in}}%
\pgfpathlineto{\pgfqpoint{3.070461in}{0.845058in}}%
\pgfpathlineto{\pgfqpoint{3.070461in}{0.848007in}}%
\pgfpathlineto{\pgfqpoint{3.075002in}{0.848007in}}%
\pgfpathlineto{\pgfqpoint{3.075002in}{0.845058in}}%
\pgfpathmoveto{\pgfqpoint{3.070461in}{0.848007in}}%
\pgfpathlineto{\pgfqpoint{3.070461in}{0.848007in}}%
\pgfpathlineto{\pgfqpoint{3.070461in}{0.850956in}}%
\pgfpathlineto{\pgfqpoint{3.075002in}{0.850956in}}%
\pgfpathlineto{\pgfqpoint{3.075002in}{0.848007in}}%
\pgfpathmoveto{\pgfqpoint{3.070461in}{0.850956in}}%
\pgfpathlineto{\pgfqpoint{3.070461in}{0.850956in}}%
\pgfpathlineto{\pgfqpoint{3.070461in}{0.853905in}}%
\pgfpathlineto{\pgfqpoint{3.075002in}{0.853905in}}%
\pgfpathlineto{\pgfqpoint{3.075002in}{0.850956in}}%
\pgfpathmoveto{\pgfqpoint{3.070461in}{0.853905in}}%
\pgfpathlineto{\pgfqpoint{3.070461in}{0.853905in}}%
\pgfpathlineto{\pgfqpoint{3.070461in}{0.856855in}}%
\pgfpathlineto{\pgfqpoint{3.075002in}{0.856855in}}%
\pgfpathlineto{\pgfqpoint{3.075002in}{0.853905in}}%
\pgfpathmoveto{\pgfqpoint{3.070461in}{0.856855in}}%
\pgfpathlineto{\pgfqpoint{3.070461in}{0.856855in}}%
\pgfpathlineto{\pgfqpoint{3.070461in}{0.859804in}}%
\pgfpathlineto{\pgfqpoint{3.075002in}{0.859804in}}%
\pgfpathlineto{\pgfqpoint{3.075002in}{0.856855in}}%
\pgfpathmoveto{\pgfqpoint{3.070461in}{0.859804in}}%
\pgfpathlineto{\pgfqpoint{3.070461in}{0.859804in}}%
\pgfpathlineto{\pgfqpoint{3.070461in}{0.862753in}}%
\pgfpathlineto{\pgfqpoint{3.075002in}{0.862753in}}%
\pgfpathlineto{\pgfqpoint{3.075002in}{0.859804in}}%
\pgfpathmoveto{\pgfqpoint{3.070461in}{0.862753in}}%
\pgfpathlineto{\pgfqpoint{3.070461in}{0.862753in}}%
\pgfpathlineto{\pgfqpoint{3.070461in}{0.865702in}}%
\pgfpathlineto{\pgfqpoint{3.075002in}{0.865702in}}%
\pgfpathlineto{\pgfqpoint{3.075002in}{0.862753in}}%
\pgfpathmoveto{\pgfqpoint{3.070461in}{0.865702in}}%
\pgfpathlineto{\pgfqpoint{3.070461in}{0.865702in}}%
\pgfpathlineto{\pgfqpoint{3.070461in}{0.868651in}}%
\pgfpathlineto{\pgfqpoint{3.075002in}{0.868651in}}%
\pgfpathlineto{\pgfqpoint{3.075002in}{0.865702in}}%
\pgfpathmoveto{\pgfqpoint{3.070461in}{0.868651in}}%
\pgfpathlineto{\pgfqpoint{3.070461in}{0.868651in}}%
\pgfpathlineto{\pgfqpoint{3.070461in}{0.871600in}}%
\pgfpathlineto{\pgfqpoint{3.075002in}{0.871600in}}%
\pgfpathlineto{\pgfqpoint{3.075002in}{0.868651in}}%
\pgfpathmoveto{\pgfqpoint{3.070461in}{0.871600in}}%
\pgfpathlineto{\pgfqpoint{3.070461in}{0.871600in}}%
\pgfpathlineto{\pgfqpoint{3.070461in}{0.874550in}}%
\pgfpathlineto{\pgfqpoint{3.075002in}{0.874550in}}%
\pgfpathlineto{\pgfqpoint{3.075002in}{0.871600in}}%
\pgfpathmoveto{\pgfqpoint{3.070461in}{0.874550in}}%
\pgfpathlineto{\pgfqpoint{3.070461in}{0.874550in}}%
\pgfpathlineto{\pgfqpoint{3.070461in}{0.877499in}}%
\pgfpathlineto{\pgfqpoint{3.075002in}{0.877499in}}%
\pgfpathlineto{\pgfqpoint{3.075002in}{0.874550in}}%
\pgfpathmoveto{\pgfqpoint{3.070461in}{0.877499in}}%
\pgfpathlineto{\pgfqpoint{3.070461in}{0.877499in}}%
\pgfpathlineto{\pgfqpoint{3.070461in}{0.880448in}}%
\pgfpathlineto{\pgfqpoint{3.075002in}{0.880448in}}%
\pgfpathlineto{\pgfqpoint{3.075002in}{0.877499in}}%
\pgfpathmoveto{\pgfqpoint{3.070461in}{0.880448in}}%
\pgfpathlineto{\pgfqpoint{3.070461in}{0.880448in}}%
\pgfpathlineto{\pgfqpoint{3.070461in}{0.883397in}}%
\pgfpathlineto{\pgfqpoint{3.075002in}{0.883397in}}%
\pgfpathlineto{\pgfqpoint{3.075002in}{0.880448in}}%
\pgfpathmoveto{\pgfqpoint{3.070461in}{0.883397in}}%
\pgfpathlineto{\pgfqpoint{3.070461in}{0.883397in}}%
\pgfpathlineto{\pgfqpoint{3.070461in}{0.886346in}}%
\pgfpathlineto{\pgfqpoint{3.075002in}{0.886346in}}%
\pgfpathlineto{\pgfqpoint{3.075002in}{0.883397in}}%
\pgfpathmoveto{\pgfqpoint{3.070461in}{0.886346in}}%
\pgfpathlineto{\pgfqpoint{3.070461in}{0.886346in}}%
\pgfpathlineto{\pgfqpoint{3.070461in}{0.889295in}}%
\pgfpathlineto{\pgfqpoint{3.075002in}{0.889295in}}%
\pgfpathlineto{\pgfqpoint{3.075002in}{0.886346in}}%
\pgfpathmoveto{\pgfqpoint{3.070461in}{0.889295in}}%
\pgfpathlineto{\pgfqpoint{3.070461in}{0.889295in}}%
\pgfpathlineto{\pgfqpoint{3.070461in}{0.892245in}}%
\pgfpathlineto{\pgfqpoint{3.075002in}{0.892245in}}%
\pgfpathlineto{\pgfqpoint{3.075002in}{0.889295in}}%
\pgfpathmoveto{\pgfqpoint{3.070461in}{0.892245in}}%
\pgfpathlineto{\pgfqpoint{3.070461in}{0.892245in}}%
\pgfpathlineto{\pgfqpoint{3.070461in}{0.895194in}}%
\pgfpathlineto{\pgfqpoint{3.075002in}{0.895194in}}%
\pgfpathlineto{\pgfqpoint{3.075002in}{0.892245in}}%
\pgfpathmoveto{\pgfqpoint{3.070461in}{0.895194in}}%
\pgfpathlineto{\pgfqpoint{3.070461in}{0.895194in}}%
\pgfpathlineto{\pgfqpoint{3.070461in}{0.898143in}}%
\pgfpathlineto{\pgfqpoint{3.075002in}{0.898143in}}%
\pgfpathlineto{\pgfqpoint{3.075002in}{0.895194in}}%
\pgfpathmoveto{\pgfqpoint{3.070461in}{0.898143in}}%
\pgfpathlineto{\pgfqpoint{3.070461in}{0.898143in}}%
\pgfpathlineto{\pgfqpoint{3.070461in}{0.901092in}}%
\pgfpathlineto{\pgfqpoint{3.075002in}{0.901092in}}%
\pgfpathlineto{\pgfqpoint{3.075002in}{0.898143in}}%
\pgfpathmoveto{\pgfqpoint{3.070461in}{0.901092in}}%
\pgfpathlineto{\pgfqpoint{3.070461in}{0.901092in}}%
\pgfpathlineto{\pgfqpoint{3.070461in}{0.904041in}}%
\pgfpathlineto{\pgfqpoint{3.075002in}{0.904041in}}%
\pgfpathlineto{\pgfqpoint{3.075002in}{0.901092in}}%
\pgfpathmoveto{\pgfqpoint{3.070461in}{0.904041in}}%
\pgfpathlineto{\pgfqpoint{3.070461in}{0.904041in}}%
\pgfpathlineto{\pgfqpoint{3.070461in}{0.906990in}}%
\pgfpathlineto{\pgfqpoint{3.075002in}{0.906990in}}%
\pgfpathlineto{\pgfqpoint{3.075002in}{0.904041in}}%
\pgfpathmoveto{\pgfqpoint{3.070461in}{0.906990in}}%
\pgfpathlineto{\pgfqpoint{3.070461in}{0.906990in}}%
\pgfpathlineto{\pgfqpoint{3.070461in}{0.909940in}}%
\pgfpathlineto{\pgfqpoint{3.075002in}{0.909940in}}%
\pgfpathlineto{\pgfqpoint{3.075002in}{0.906990in}}%
\pgfpathmoveto{\pgfqpoint{3.070461in}{0.909940in}}%
\pgfpathlineto{\pgfqpoint{3.070461in}{0.909940in}}%
\pgfpathlineto{\pgfqpoint{3.070461in}{0.912889in}}%
\pgfpathlineto{\pgfqpoint{3.075002in}{0.912889in}}%
\pgfpathlineto{\pgfqpoint{3.075002in}{0.909940in}}%
\pgfpathmoveto{\pgfqpoint{3.070461in}{0.912889in}}%
\pgfpathlineto{\pgfqpoint{3.070461in}{0.912889in}}%
\pgfpathlineto{\pgfqpoint{3.070461in}{0.915838in}}%
\pgfpathlineto{\pgfqpoint{3.075002in}{0.915838in}}%
\pgfpathlineto{\pgfqpoint{3.075002in}{0.912889in}}%
\pgfpathmoveto{\pgfqpoint{3.070461in}{0.915838in}}%
\pgfpathlineto{\pgfqpoint{3.070461in}{0.915838in}}%
\pgfpathlineto{\pgfqpoint{3.070461in}{0.918787in}}%
\pgfpathlineto{\pgfqpoint{3.075002in}{0.918787in}}%
\pgfpathlineto{\pgfqpoint{3.075002in}{0.915838in}}%
\pgfpathmoveto{\pgfqpoint{3.070461in}{0.918787in}}%
\pgfpathlineto{\pgfqpoint{3.070461in}{0.918787in}}%
\pgfpathlineto{\pgfqpoint{3.070461in}{0.921736in}}%
\pgfpathlineto{\pgfqpoint{3.075002in}{0.921736in}}%
\pgfpathlineto{\pgfqpoint{3.075002in}{0.918787in}}%
\pgfpathmoveto{\pgfqpoint{3.070461in}{0.921736in}}%
\pgfpathlineto{\pgfqpoint{3.070461in}{0.921736in}}%
\pgfpathlineto{\pgfqpoint{3.070461in}{0.924685in}}%
\pgfpathlineto{\pgfqpoint{3.075002in}{0.924685in}}%
\pgfpathlineto{\pgfqpoint{3.075002in}{0.921736in}}%
\pgfpathmoveto{\pgfqpoint{3.070461in}{0.924685in}}%
\pgfpathlineto{\pgfqpoint{3.070461in}{0.924685in}}%
\pgfpathlineto{\pgfqpoint{3.070461in}{0.927635in}}%
\pgfpathlineto{\pgfqpoint{3.075002in}{0.927635in}}%
\pgfpathlineto{\pgfqpoint{3.075002in}{0.924685in}}%
\pgfpathmoveto{\pgfqpoint{3.070461in}{0.927635in}}%
\pgfpathlineto{\pgfqpoint{3.070461in}{0.927635in}}%
\pgfpathlineto{\pgfqpoint{3.070461in}{0.930584in}}%
\pgfpathlineto{\pgfqpoint{3.075002in}{0.930584in}}%
\pgfpathlineto{\pgfqpoint{3.075002in}{0.927635in}}%
\pgfpathmoveto{\pgfqpoint{3.070461in}{0.930584in}}%
\pgfpathlineto{\pgfqpoint{3.070461in}{0.930584in}}%
\pgfpathlineto{\pgfqpoint{3.070461in}{0.933533in}}%
\pgfpathlineto{\pgfqpoint{3.075002in}{0.933533in}}%
\pgfpathlineto{\pgfqpoint{3.075002in}{0.930584in}}%
\pgfpathmoveto{\pgfqpoint{3.070461in}{0.933533in}}%
\pgfpathlineto{\pgfqpoint{3.070461in}{0.933533in}}%
\pgfpathlineto{\pgfqpoint{3.070461in}{0.936482in}}%
\pgfpathlineto{\pgfqpoint{3.075002in}{0.936482in}}%
\pgfpathlineto{\pgfqpoint{3.075002in}{0.933533in}}%
\pgfpathmoveto{\pgfqpoint{3.070461in}{0.936482in}}%
\pgfpathlineto{\pgfqpoint{3.070461in}{0.936482in}}%
\pgfpathlineto{\pgfqpoint{3.070461in}{0.939431in}}%
\pgfpathlineto{\pgfqpoint{3.075002in}{0.939431in}}%
\pgfpathlineto{\pgfqpoint{3.075002in}{0.936482in}}%
\pgfpathmoveto{\pgfqpoint{3.070461in}{0.939431in}}%
\pgfpathlineto{\pgfqpoint{3.070461in}{0.939431in}}%
\pgfpathlineto{\pgfqpoint{3.070461in}{0.942380in}}%
\pgfpathlineto{\pgfqpoint{3.075002in}{0.942380in}}%
\pgfpathlineto{\pgfqpoint{3.075002in}{0.939431in}}%
\pgfpathmoveto{\pgfqpoint{3.070461in}{0.942380in}}%
\pgfpathlineto{\pgfqpoint{3.070461in}{0.942380in}}%
\pgfpathlineto{\pgfqpoint{3.070461in}{0.945330in}}%
\pgfpathlineto{\pgfqpoint{3.075002in}{0.945330in}}%
\pgfpathlineto{\pgfqpoint{3.075002in}{0.942380in}}%
\pgfpathmoveto{\pgfqpoint{3.070461in}{0.945330in}}%
\pgfpathlineto{\pgfqpoint{3.070461in}{0.945330in}}%
\pgfpathlineto{\pgfqpoint{3.070461in}{0.948279in}}%
\pgfpathlineto{\pgfqpoint{3.075002in}{0.948279in}}%
\pgfpathlineto{\pgfqpoint{3.075002in}{0.945330in}}%
\pgfpathmoveto{\pgfqpoint{3.070461in}{0.948279in}}%
\pgfpathlineto{\pgfqpoint{3.070461in}{0.948279in}}%
\pgfpathlineto{\pgfqpoint{3.070461in}{0.951228in}}%
\pgfpathlineto{\pgfqpoint{3.075002in}{0.951228in}}%
\pgfpathlineto{\pgfqpoint{3.075002in}{0.948279in}}%
\pgfpathmoveto{\pgfqpoint{3.070461in}{0.951228in}}%
\pgfpathlineto{\pgfqpoint{3.070461in}{0.951228in}}%
\pgfpathlineto{\pgfqpoint{3.070461in}{0.954177in}}%
\pgfpathlineto{\pgfqpoint{3.075002in}{0.954177in}}%
\pgfpathlineto{\pgfqpoint{3.075002in}{0.951228in}}%
\pgfpathmoveto{\pgfqpoint{3.070461in}{0.954177in}}%
\pgfpathlineto{\pgfqpoint{3.070461in}{0.954177in}}%
\pgfpathlineto{\pgfqpoint{3.070461in}{0.957126in}}%
\pgfpathlineto{\pgfqpoint{3.075002in}{0.957126in}}%
\pgfpathlineto{\pgfqpoint{3.075002in}{0.954177in}}%
\pgfpathmoveto{\pgfqpoint{3.070461in}{0.957126in}}%
\pgfpathlineto{\pgfqpoint{3.070461in}{0.957126in}}%
\pgfpathlineto{\pgfqpoint{3.070461in}{0.960075in}}%
\pgfpathlineto{\pgfqpoint{3.075002in}{0.960075in}}%
\pgfpathlineto{\pgfqpoint{3.075002in}{0.957126in}}%
\pgfpathmoveto{\pgfqpoint{3.070461in}{0.960075in}}%
\pgfpathlineto{\pgfqpoint{3.070461in}{0.960075in}}%
\pgfpathlineto{\pgfqpoint{3.070461in}{0.963025in}}%
\pgfpathlineto{\pgfqpoint{3.075002in}{0.963025in}}%
\pgfpathlineto{\pgfqpoint{3.075002in}{0.960075in}}%
\pgfpathmoveto{\pgfqpoint{3.070461in}{0.963025in}}%
\pgfpathlineto{\pgfqpoint{3.070461in}{0.963025in}}%
\pgfpathlineto{\pgfqpoint{3.070461in}{0.965974in}}%
\pgfpathlineto{\pgfqpoint{3.075002in}{0.965974in}}%
\pgfpathlineto{\pgfqpoint{3.075002in}{0.963025in}}%
\pgfpathmoveto{\pgfqpoint{3.070461in}{0.965974in}}%
\pgfpathlineto{\pgfqpoint{3.070461in}{0.965974in}}%
\pgfpathlineto{\pgfqpoint{3.070461in}{0.968923in}}%
\pgfpathlineto{\pgfqpoint{3.075002in}{0.968923in}}%
\pgfpathlineto{\pgfqpoint{3.075002in}{0.965974in}}%
\pgfpathmoveto{\pgfqpoint{3.070461in}{0.968923in}}%
\pgfpathlineto{\pgfqpoint{3.070461in}{0.968923in}}%
\pgfpathlineto{\pgfqpoint{3.070461in}{0.971872in}}%
\pgfpathlineto{\pgfqpoint{3.075002in}{0.971872in}}%
\pgfpathlineto{\pgfqpoint{3.075002in}{0.968923in}}%
\pgfpathmoveto{\pgfqpoint{3.070461in}{0.971872in}}%
\pgfpathlineto{\pgfqpoint{3.070461in}{0.971872in}}%
\pgfpathlineto{\pgfqpoint{3.070461in}{0.974822in}}%
\pgfpathlineto{\pgfqpoint{3.075002in}{0.974822in}}%
\pgfpathlineto{\pgfqpoint{3.075002in}{0.971872in}}%
\pgfpathmoveto{\pgfqpoint{3.070461in}{0.974822in}}%
\pgfpathlineto{\pgfqpoint{3.070461in}{0.974822in}}%
\pgfpathlineto{\pgfqpoint{3.070461in}{0.977771in}}%
\pgfpathlineto{\pgfqpoint{3.075002in}{0.977771in}}%
\pgfpathlineto{\pgfqpoint{3.075002in}{0.974822in}}%
\pgfpathmoveto{\pgfqpoint{3.070461in}{0.977771in}}%
\pgfpathlineto{\pgfqpoint{3.070461in}{0.977771in}}%
\pgfpathlineto{\pgfqpoint{3.070461in}{0.980720in}}%
\pgfpathlineto{\pgfqpoint{3.075002in}{0.980720in}}%
\pgfpathlineto{\pgfqpoint{3.075002in}{0.977771in}}%
\pgfpathmoveto{\pgfqpoint{3.070461in}{0.980720in}}%
\pgfpathlineto{\pgfqpoint{3.070461in}{0.980720in}}%
\pgfpathlineto{\pgfqpoint{3.070461in}{0.983670in}}%
\pgfpathlineto{\pgfqpoint{3.075002in}{0.983670in}}%
\pgfpathlineto{\pgfqpoint{3.075002in}{0.980720in}}%
\pgfpathmoveto{\pgfqpoint{3.070461in}{0.983670in}}%
\pgfpathlineto{\pgfqpoint{3.070461in}{0.983670in}}%
\pgfpathlineto{\pgfqpoint{3.070461in}{0.986619in}}%
\pgfpathlineto{\pgfqpoint{3.075002in}{0.986619in}}%
\pgfpathlineto{\pgfqpoint{3.075002in}{0.983670in}}%
\pgfpathmoveto{\pgfqpoint{3.070461in}{0.986619in}}%
\pgfpathlineto{\pgfqpoint{3.070461in}{0.986619in}}%
\pgfpathlineto{\pgfqpoint{3.070461in}{0.989568in}}%
\pgfpathlineto{\pgfqpoint{3.075002in}{0.989568in}}%
\pgfpathlineto{\pgfqpoint{3.075002in}{0.986619in}}%
\pgfpathmoveto{\pgfqpoint{3.070461in}{0.989568in}}%
\pgfpathlineto{\pgfqpoint{3.070461in}{0.989568in}}%
\pgfpathlineto{\pgfqpoint{3.070461in}{0.992518in}}%
\pgfpathlineto{\pgfqpoint{3.075002in}{0.992518in}}%
\pgfpathlineto{\pgfqpoint{3.075002in}{0.989568in}}%
\pgfpathmoveto{\pgfqpoint{3.070461in}{0.992518in}}%
\pgfpathlineto{\pgfqpoint{3.070461in}{0.992518in}}%
\pgfpathlineto{\pgfqpoint{3.070461in}{0.995467in}}%
\pgfpathlineto{\pgfqpoint{3.075002in}{0.995467in}}%
\pgfpathlineto{\pgfqpoint{3.075002in}{0.992518in}}%
\pgfpathmoveto{\pgfqpoint{3.070461in}{0.995467in}}%
\pgfpathlineto{\pgfqpoint{3.070461in}{0.995467in}}%
\pgfpathlineto{\pgfqpoint{3.070461in}{0.998416in}}%
\pgfpathlineto{\pgfqpoint{3.075002in}{0.998416in}}%
\pgfpathlineto{\pgfqpoint{3.075002in}{0.995467in}}%
\pgfpathmoveto{\pgfqpoint{3.070461in}{0.998416in}}%
\pgfpathlineto{\pgfqpoint{3.070461in}{0.998416in}}%
\pgfpathlineto{\pgfqpoint{3.070461in}{1.001366in}}%
\pgfpathlineto{\pgfqpoint{3.075002in}{1.001366in}}%
\pgfpathlineto{\pgfqpoint{3.075002in}{0.998416in}}%
\pgfpathmoveto{\pgfqpoint{3.070461in}{1.001366in}}%
\pgfpathlineto{\pgfqpoint{3.070461in}{1.001366in}}%
\pgfpathlineto{\pgfqpoint{3.070461in}{1.004315in}}%
\pgfpathlineto{\pgfqpoint{3.075002in}{1.004315in}}%
\pgfpathlineto{\pgfqpoint{3.075002in}{1.001366in}}%
\pgfpathmoveto{\pgfqpoint{3.070461in}{1.004315in}}%
\pgfpathlineto{\pgfqpoint{3.070461in}{1.004315in}}%
\pgfpathlineto{\pgfqpoint{3.070461in}{1.007265in}}%
\pgfpathlineto{\pgfqpoint{3.075002in}{1.007265in}}%
\pgfpathlineto{\pgfqpoint{3.075002in}{1.004315in}}%
\pgfpathmoveto{\pgfqpoint{3.070461in}{1.007265in}}%
\pgfpathlineto{\pgfqpoint{3.070461in}{1.007265in}}%
\pgfpathlineto{\pgfqpoint{3.070461in}{1.010214in}}%
\pgfpathlineto{\pgfqpoint{3.075002in}{1.010214in}}%
\pgfpathlineto{\pgfqpoint{3.075002in}{1.007265in}}%
\pgfpathmoveto{\pgfqpoint{3.070461in}{1.010214in}}%
\pgfpathlineto{\pgfqpoint{3.070461in}{1.010214in}}%
\pgfpathlineto{\pgfqpoint{3.070461in}{1.013163in}}%
\pgfpathlineto{\pgfqpoint{3.075002in}{1.013163in}}%
\pgfpathlineto{\pgfqpoint{3.075002in}{1.010214in}}%
\pgfpathmoveto{\pgfqpoint{3.070461in}{1.013163in}}%
\pgfpathlineto{\pgfqpoint{3.070461in}{1.013163in}}%
\pgfpathlineto{\pgfqpoint{3.070461in}{1.016113in}}%
\pgfpathlineto{\pgfqpoint{3.075002in}{1.016113in}}%
\pgfpathlineto{\pgfqpoint{3.075002in}{1.013163in}}%
\pgfpathmoveto{\pgfqpoint{3.070461in}{1.016113in}}%
\pgfpathlineto{\pgfqpoint{3.070461in}{1.016113in}}%
\pgfpathlineto{\pgfqpoint{3.070461in}{1.019062in}}%
\pgfpathlineto{\pgfqpoint{3.075002in}{1.019062in}}%
\pgfpathlineto{\pgfqpoint{3.075002in}{1.016113in}}%
\pgfpathmoveto{\pgfqpoint{3.070461in}{1.019062in}}%
\pgfpathlineto{\pgfqpoint{3.070461in}{1.019062in}}%
\pgfpathlineto{\pgfqpoint{3.070461in}{1.022011in}}%
\pgfpathlineto{\pgfqpoint{3.075002in}{1.022011in}}%
\pgfpathlineto{\pgfqpoint{3.075002in}{1.019062in}}%
\pgfpathmoveto{\pgfqpoint{3.070461in}{1.022011in}}%
\pgfpathlineto{\pgfqpoint{3.070461in}{1.022011in}}%
\pgfpathlineto{\pgfqpoint{3.070461in}{1.024961in}}%
\pgfpathlineto{\pgfqpoint{3.075002in}{1.024961in}}%
\pgfpathlineto{\pgfqpoint{3.075002in}{1.022011in}}%
\pgfpathmoveto{\pgfqpoint{3.070461in}{1.024961in}}%
\pgfpathlineto{\pgfqpoint{3.070461in}{1.024961in}}%
\pgfpathlineto{\pgfqpoint{3.070461in}{1.027910in}}%
\pgfpathlineto{\pgfqpoint{3.075002in}{1.027910in}}%
\pgfpathlineto{\pgfqpoint{3.075002in}{1.024961in}}%
\pgfpathmoveto{\pgfqpoint{3.070461in}{1.027910in}}%
\pgfpathlineto{\pgfqpoint{3.070461in}{1.027910in}}%
\pgfpathlineto{\pgfqpoint{3.070461in}{1.030859in}}%
\pgfpathlineto{\pgfqpoint{3.075002in}{1.030859in}}%
\pgfpathlineto{\pgfqpoint{3.075002in}{1.027910in}}%
\pgfpathmoveto{\pgfqpoint{3.070461in}{1.030859in}}%
\pgfpathlineto{\pgfqpoint{3.070461in}{1.030859in}}%
\pgfpathlineto{\pgfqpoint{3.070461in}{1.033809in}}%
\pgfpathlineto{\pgfqpoint{3.075002in}{1.033809in}}%
\pgfpathlineto{\pgfqpoint{3.075002in}{1.030859in}}%
\pgfpathmoveto{\pgfqpoint{3.070461in}{1.033809in}}%
\pgfpathlineto{\pgfqpoint{3.070461in}{1.033809in}}%
\pgfpathlineto{\pgfqpoint{3.070461in}{1.036758in}}%
\pgfpathlineto{\pgfqpoint{3.075002in}{1.036758in}}%
\pgfpathlineto{\pgfqpoint{3.075002in}{1.033809in}}%
\pgfpathmoveto{\pgfqpoint{3.070461in}{1.036758in}}%
\pgfpathlineto{\pgfqpoint{3.070461in}{1.036758in}}%
\pgfpathlineto{\pgfqpoint{3.070461in}{1.039707in}}%
\pgfpathlineto{\pgfqpoint{3.075002in}{1.039707in}}%
\pgfpathlineto{\pgfqpoint{3.075002in}{1.036758in}}%
\pgfpathmoveto{\pgfqpoint{3.070461in}{1.039707in}}%
\pgfpathlineto{\pgfqpoint{3.070461in}{1.039707in}}%
\pgfpathlineto{\pgfqpoint{3.070461in}{1.042657in}}%
\pgfpathlineto{\pgfqpoint{3.075002in}{1.042657in}}%
\pgfpathlineto{\pgfqpoint{3.075002in}{1.039707in}}%
\pgfpathmoveto{\pgfqpoint{3.070461in}{1.042657in}}%
\pgfpathlineto{\pgfqpoint{3.070461in}{1.042657in}}%
\pgfpathlineto{\pgfqpoint{3.070461in}{1.045606in}}%
\pgfpathlineto{\pgfqpoint{3.075002in}{1.045606in}}%
\pgfpathlineto{\pgfqpoint{3.075002in}{1.042657in}}%
\pgfpathmoveto{\pgfqpoint{3.070461in}{1.045606in}}%
\pgfpathlineto{\pgfqpoint{3.070461in}{1.045606in}}%
\pgfpathlineto{\pgfqpoint{3.070461in}{1.048556in}}%
\pgfpathlineto{\pgfqpoint{3.075002in}{1.048556in}}%
\pgfpathlineto{\pgfqpoint{3.075002in}{1.045606in}}%
\pgfpathmoveto{\pgfqpoint{3.070461in}{1.048556in}}%
\pgfpathlineto{\pgfqpoint{3.070461in}{1.048556in}}%
\pgfpathlineto{\pgfqpoint{3.070461in}{1.051505in}}%
\pgfpathlineto{\pgfqpoint{3.075002in}{1.051505in}}%
\pgfpathlineto{\pgfqpoint{3.075002in}{1.048556in}}%
\pgfpathmoveto{\pgfqpoint{3.070461in}{1.051505in}}%
\pgfpathlineto{\pgfqpoint{3.070461in}{1.051505in}}%
\pgfpathlineto{\pgfqpoint{3.070461in}{1.054454in}}%
\pgfpathlineto{\pgfqpoint{3.075002in}{1.054454in}}%
\pgfpathlineto{\pgfqpoint{3.075002in}{1.051505in}}%
\pgfpathmoveto{\pgfqpoint{3.070461in}{1.054454in}}%
\pgfpathlineto{\pgfqpoint{3.070461in}{1.054454in}}%
\pgfpathlineto{\pgfqpoint{3.070461in}{1.057404in}}%
\pgfpathlineto{\pgfqpoint{3.075002in}{1.057404in}}%
\pgfpathlineto{\pgfqpoint{3.075002in}{1.054454in}}%
\pgfpathmoveto{\pgfqpoint{3.070461in}{1.057404in}}%
\pgfpathlineto{\pgfqpoint{3.070461in}{1.057404in}}%
\pgfpathlineto{\pgfqpoint{3.070461in}{1.060353in}}%
\pgfpathlineto{\pgfqpoint{3.075002in}{1.060353in}}%
\pgfpathlineto{\pgfqpoint{3.075002in}{1.057404in}}%
\pgfpathmoveto{\pgfqpoint{3.070461in}{1.060353in}}%
\pgfpathlineto{\pgfqpoint{3.070461in}{1.060353in}}%
\pgfpathlineto{\pgfqpoint{3.070461in}{1.063302in}}%
\pgfpathlineto{\pgfqpoint{3.075002in}{1.063302in}}%
\pgfpathlineto{\pgfqpoint{3.075002in}{1.060353in}}%
\pgfpathmoveto{\pgfqpoint{3.070461in}{1.063302in}}%
\pgfpathlineto{\pgfqpoint{3.070461in}{1.063302in}}%
\pgfpathlineto{\pgfqpoint{3.070461in}{1.066252in}}%
\pgfpathlineto{\pgfqpoint{3.075002in}{1.066252in}}%
\pgfpathlineto{\pgfqpoint{3.075002in}{1.063302in}}%
\pgfpathmoveto{\pgfqpoint{3.070461in}{1.066252in}}%
\pgfpathlineto{\pgfqpoint{3.070461in}{1.066252in}}%
\pgfpathlineto{\pgfqpoint{3.070461in}{1.069201in}}%
\pgfpathlineto{\pgfqpoint{3.075002in}{1.069201in}}%
\pgfpathlineto{\pgfqpoint{3.075002in}{1.066252in}}%
\pgfpathmoveto{\pgfqpoint{3.070461in}{1.069201in}}%
\pgfpathlineto{\pgfqpoint{3.070461in}{1.069201in}}%
\pgfpathlineto{\pgfqpoint{3.070461in}{1.072150in}}%
\pgfpathlineto{\pgfqpoint{3.075002in}{1.072150in}}%
\pgfpathlineto{\pgfqpoint{3.075002in}{1.069201in}}%
\pgfpathmoveto{\pgfqpoint{3.070461in}{1.072150in}}%
\pgfpathlineto{\pgfqpoint{3.070461in}{1.072150in}}%
\pgfpathlineto{\pgfqpoint{3.070461in}{1.075099in}}%
\pgfpathlineto{\pgfqpoint{3.075002in}{1.075099in}}%
\pgfpathlineto{\pgfqpoint{3.075002in}{1.072150in}}%
\pgfpathmoveto{\pgfqpoint{3.070461in}{1.075099in}}%
\pgfpathlineto{\pgfqpoint{3.070461in}{1.075099in}}%
\pgfpathlineto{\pgfqpoint{3.070461in}{1.078048in}}%
\pgfpathlineto{\pgfqpoint{3.075002in}{1.078048in}}%
\pgfpathlineto{\pgfqpoint{3.075002in}{1.075099in}}%
\pgfpathmoveto{\pgfqpoint{3.070461in}{1.078048in}}%
\pgfpathlineto{\pgfqpoint{3.070461in}{1.078048in}}%
\pgfpathlineto{\pgfqpoint{3.070461in}{1.080998in}}%
\pgfpathlineto{\pgfqpoint{3.075002in}{1.080998in}}%
\pgfpathlineto{\pgfqpoint{3.075002in}{1.078048in}}%
\pgfpathmoveto{\pgfqpoint{3.070461in}{1.080998in}}%
\pgfpathlineto{\pgfqpoint{3.070461in}{1.080998in}}%
\pgfpathlineto{\pgfqpoint{3.070461in}{1.083947in}}%
\pgfpathlineto{\pgfqpoint{3.075002in}{1.083947in}}%
\pgfpathlineto{\pgfqpoint{3.075002in}{1.080998in}}%
\pgfpathmoveto{\pgfqpoint{3.070461in}{1.083947in}}%
\pgfpathlineto{\pgfqpoint{3.070461in}{1.083947in}}%
\pgfpathlineto{\pgfqpoint{3.070461in}{1.086896in}}%
\pgfpathlineto{\pgfqpoint{3.075002in}{1.086896in}}%
\pgfpathlineto{\pgfqpoint{3.075002in}{1.083947in}}%
\pgfpathmoveto{\pgfqpoint{3.070461in}{1.086896in}}%
\pgfpathlineto{\pgfqpoint{3.070461in}{1.086896in}}%
\pgfpathlineto{\pgfqpoint{3.070461in}{1.089845in}}%
\pgfpathlineto{\pgfqpoint{3.075002in}{1.089845in}}%
\pgfpathlineto{\pgfqpoint{3.075002in}{1.086896in}}%
\pgfpathmoveto{\pgfqpoint{3.070461in}{1.089845in}}%
\pgfpathlineto{\pgfqpoint{3.070461in}{1.089845in}}%
\pgfpathlineto{\pgfqpoint{3.070461in}{1.092794in}}%
\pgfpathlineto{\pgfqpoint{3.075002in}{1.092794in}}%
\pgfpathlineto{\pgfqpoint{3.075002in}{1.089845in}}%
\pgfpathmoveto{\pgfqpoint{3.070461in}{1.092794in}}%
\pgfpathlineto{\pgfqpoint{3.070461in}{1.092794in}}%
\pgfpathlineto{\pgfqpoint{3.070461in}{1.095744in}}%
\pgfpathlineto{\pgfqpoint{3.075002in}{1.095744in}}%
\pgfpathlineto{\pgfqpoint{3.075002in}{1.092794in}}%
\pgfpathmoveto{\pgfqpoint{3.070461in}{1.095744in}}%
\pgfpathlineto{\pgfqpoint{3.070461in}{1.095744in}}%
\pgfpathlineto{\pgfqpoint{3.070461in}{1.098693in}}%
\pgfpathlineto{\pgfqpoint{3.075002in}{1.098693in}}%
\pgfpathlineto{\pgfqpoint{3.075002in}{1.095744in}}%
\pgfpathmoveto{\pgfqpoint{3.070461in}{1.098693in}}%
\pgfpathlineto{\pgfqpoint{3.070461in}{1.098693in}}%
\pgfpathlineto{\pgfqpoint{3.070461in}{1.101642in}}%
\pgfpathlineto{\pgfqpoint{3.075002in}{1.101642in}}%
\pgfpathlineto{\pgfqpoint{3.075002in}{1.098693in}}%
\pgfpathmoveto{\pgfqpoint{3.070461in}{1.101642in}}%
\pgfpathlineto{\pgfqpoint{3.070461in}{1.101642in}}%
\pgfpathlineto{\pgfqpoint{3.070461in}{1.104591in}}%
\pgfpathlineto{\pgfqpoint{3.075002in}{1.104591in}}%
\pgfpathlineto{\pgfqpoint{3.075002in}{1.101642in}}%
\pgfpathmoveto{\pgfqpoint{3.070461in}{1.104591in}}%
\pgfpathlineto{\pgfqpoint{3.070461in}{1.104591in}}%
\pgfpathlineto{\pgfqpoint{3.070461in}{1.107540in}}%
\pgfpathlineto{\pgfqpoint{3.075002in}{1.107540in}}%
\pgfpathlineto{\pgfqpoint{3.075002in}{1.104591in}}%
\pgfpathmoveto{\pgfqpoint{3.070461in}{1.107540in}}%
\pgfpathlineto{\pgfqpoint{3.070461in}{1.107540in}}%
\pgfpathlineto{\pgfqpoint{3.070461in}{1.110490in}}%
\pgfpathlineto{\pgfqpoint{3.075002in}{1.110490in}}%
\pgfpathlineto{\pgfqpoint{3.075002in}{1.107540in}}%
\pgfpathmoveto{\pgfqpoint{3.070461in}{1.110490in}}%
\pgfpathlineto{\pgfqpoint{3.070461in}{1.110490in}}%
\pgfpathlineto{\pgfqpoint{3.070461in}{1.113439in}}%
\pgfpathlineto{\pgfqpoint{3.075002in}{1.113439in}}%
\pgfpathlineto{\pgfqpoint{3.075002in}{1.110490in}}%
\pgfpathmoveto{\pgfqpoint{3.070461in}{1.113439in}}%
\pgfpathlineto{\pgfqpoint{3.070461in}{1.113439in}}%
\pgfpathlineto{\pgfqpoint{3.070461in}{1.116388in}}%
\pgfpathlineto{\pgfqpoint{3.075002in}{1.116388in}}%
\pgfpathlineto{\pgfqpoint{3.075002in}{1.113439in}}%
\pgfpathmoveto{\pgfqpoint{3.070461in}{1.116388in}}%
\pgfpathlineto{\pgfqpoint{3.070461in}{1.116388in}}%
\pgfpathlineto{\pgfqpoint{3.070461in}{1.119337in}}%
\pgfpathlineto{\pgfqpoint{3.075002in}{1.119337in}}%
\pgfpathlineto{\pgfqpoint{3.075002in}{1.116388in}}%
\pgfpathmoveto{\pgfqpoint{3.070461in}{1.119337in}}%
\pgfpathlineto{\pgfqpoint{3.070461in}{1.119337in}}%
\pgfpathlineto{\pgfqpoint{3.070461in}{1.122286in}}%
\pgfpathlineto{\pgfqpoint{3.075002in}{1.122286in}}%
\pgfpathlineto{\pgfqpoint{3.075002in}{1.119337in}}%
\pgfpathmoveto{\pgfqpoint{3.070461in}{1.122286in}}%
\pgfpathlineto{\pgfqpoint{3.070461in}{1.122286in}}%
\pgfpathlineto{\pgfqpoint{3.070461in}{1.125235in}}%
\pgfpathlineto{\pgfqpoint{3.075002in}{1.125235in}}%
\pgfpathlineto{\pgfqpoint{3.075002in}{1.122286in}}%
\pgfpathmoveto{\pgfqpoint{3.070461in}{1.125235in}}%
\pgfpathlineto{\pgfqpoint{3.070461in}{1.125235in}}%
\pgfpathlineto{\pgfqpoint{3.070461in}{1.128185in}}%
\pgfpathlineto{\pgfqpoint{3.075002in}{1.128185in}}%
\pgfpathlineto{\pgfqpoint{3.075002in}{1.125235in}}%
\pgfpathmoveto{\pgfqpoint{3.070461in}{1.128185in}}%
\pgfpathlineto{\pgfqpoint{3.070461in}{1.128185in}}%
\pgfpathlineto{\pgfqpoint{3.070461in}{1.131134in}}%
\pgfpathlineto{\pgfqpoint{3.075002in}{1.131134in}}%
\pgfpathlineto{\pgfqpoint{3.075002in}{1.128185in}}%
\pgfpathmoveto{\pgfqpoint{3.070461in}{1.131134in}}%
\pgfpathlineto{\pgfqpoint{3.070461in}{1.131134in}}%
\pgfpathlineto{\pgfqpoint{3.070461in}{1.134083in}}%
\pgfpathlineto{\pgfqpoint{3.075002in}{1.134083in}}%
\pgfpathlineto{\pgfqpoint{3.075002in}{1.131134in}}%
\pgfpathmoveto{\pgfqpoint{3.070461in}{1.134083in}}%
\pgfpathlineto{\pgfqpoint{3.070461in}{1.134083in}}%
\pgfpathlineto{\pgfqpoint{3.070461in}{1.137032in}}%
\pgfpathlineto{\pgfqpoint{3.075002in}{1.137032in}}%
\pgfpathlineto{\pgfqpoint{3.075002in}{1.134083in}}%
\pgfpathmoveto{\pgfqpoint{3.070461in}{1.137032in}}%
\pgfpathlineto{\pgfqpoint{3.070461in}{1.137032in}}%
\pgfpathlineto{\pgfqpoint{3.070461in}{1.139981in}}%
\pgfpathlineto{\pgfqpoint{3.075002in}{1.139981in}}%
\pgfpathlineto{\pgfqpoint{3.075002in}{1.137032in}}%
\pgfpathmoveto{\pgfqpoint{3.070461in}{1.139981in}}%
\pgfpathlineto{\pgfqpoint{3.070461in}{1.139981in}}%
\pgfpathlineto{\pgfqpoint{3.070461in}{1.142931in}}%
\pgfpathlineto{\pgfqpoint{3.075002in}{1.142931in}}%
\pgfpathlineto{\pgfqpoint{3.075002in}{1.139981in}}%
\pgfpathmoveto{\pgfqpoint{3.070461in}{1.142931in}}%
\pgfpathlineto{\pgfqpoint{3.070461in}{1.142931in}}%
\pgfpathlineto{\pgfqpoint{3.070461in}{1.145880in}}%
\pgfpathlineto{\pgfqpoint{3.075002in}{1.145880in}}%
\pgfpathlineto{\pgfqpoint{3.075002in}{1.142931in}}%
\pgfpathmoveto{\pgfqpoint{3.070461in}{1.145880in}}%
\pgfpathlineto{\pgfqpoint{3.070461in}{1.145880in}}%
\pgfpathlineto{\pgfqpoint{3.070461in}{1.148829in}}%
\pgfpathlineto{\pgfqpoint{3.075002in}{1.148829in}}%
\pgfpathlineto{\pgfqpoint{3.075002in}{1.145880in}}%
\pgfpathmoveto{\pgfqpoint{3.070461in}{1.148829in}}%
\pgfpathlineto{\pgfqpoint{3.070461in}{1.148829in}}%
\pgfpathlineto{\pgfqpoint{3.070461in}{1.151778in}}%
\pgfpathlineto{\pgfqpoint{3.075002in}{1.151778in}}%
\pgfpathlineto{\pgfqpoint{3.075002in}{1.148829in}}%
\pgfpathmoveto{\pgfqpoint{3.070461in}{1.151778in}}%
\pgfpathlineto{\pgfqpoint{3.070461in}{1.151778in}}%
\pgfpathlineto{\pgfqpoint{3.070461in}{1.154727in}}%
\pgfpathlineto{\pgfqpoint{3.075002in}{1.154727in}}%
\pgfpathlineto{\pgfqpoint{3.075002in}{1.151778in}}%
\pgfpathmoveto{\pgfqpoint{3.070461in}{1.154727in}}%
\pgfpathlineto{\pgfqpoint{3.070461in}{1.154727in}}%
\pgfpathlineto{\pgfqpoint{3.070461in}{1.157677in}}%
\pgfpathlineto{\pgfqpoint{3.075002in}{1.157677in}}%
\pgfpathlineto{\pgfqpoint{3.075002in}{1.154727in}}%
\pgfpathmoveto{\pgfqpoint{3.070461in}{1.157677in}}%
\pgfpathlineto{\pgfqpoint{3.070461in}{1.157677in}}%
\pgfpathlineto{\pgfqpoint{3.070461in}{1.160626in}}%
\pgfpathlineto{\pgfqpoint{3.075002in}{1.160626in}}%
\pgfpathlineto{\pgfqpoint{3.075002in}{1.157677in}}%
\pgfpathmoveto{\pgfqpoint{3.070461in}{1.160626in}}%
\pgfpathlineto{\pgfqpoint{3.070461in}{1.160626in}}%
\pgfpathlineto{\pgfqpoint{3.070461in}{1.163575in}}%
\pgfpathlineto{\pgfqpoint{3.075002in}{1.163575in}}%
\pgfpathlineto{\pgfqpoint{3.075002in}{1.160626in}}%
\pgfpathmoveto{\pgfqpoint{3.070461in}{1.163575in}}%
\pgfpathlineto{\pgfqpoint{3.070461in}{1.163575in}}%
\pgfpathlineto{\pgfqpoint{3.070461in}{1.166524in}}%
\pgfpathlineto{\pgfqpoint{3.075002in}{1.166524in}}%
\pgfpathlineto{\pgfqpoint{3.075002in}{1.163575in}}%
\pgfpathmoveto{\pgfqpoint{3.070461in}{1.166524in}}%
\pgfpathlineto{\pgfqpoint{3.070461in}{1.166524in}}%
\pgfpathlineto{\pgfqpoint{3.070461in}{1.169473in}}%
\pgfpathlineto{\pgfqpoint{3.075002in}{1.169473in}}%
\pgfpathlineto{\pgfqpoint{3.075002in}{1.166524in}}%
\pgfpathmoveto{\pgfqpoint{3.070461in}{1.169473in}}%
\pgfpathlineto{\pgfqpoint{3.070461in}{1.169473in}}%
\pgfpathlineto{\pgfqpoint{3.070461in}{1.172422in}}%
\pgfpathlineto{\pgfqpoint{3.075002in}{1.172422in}}%
\pgfpathlineto{\pgfqpoint{3.075002in}{1.169473in}}%
\pgfpathmoveto{\pgfqpoint{3.070461in}{1.172422in}}%
\pgfpathlineto{\pgfqpoint{3.070461in}{1.172422in}}%
\pgfpathlineto{\pgfqpoint{3.070461in}{1.175371in}}%
\pgfpathlineto{\pgfqpoint{3.075002in}{1.175371in}}%
\pgfpathlineto{\pgfqpoint{3.075002in}{1.172422in}}%
\pgfpathmoveto{\pgfqpoint{3.070461in}{1.175371in}}%
\pgfpathlineto{\pgfqpoint{3.070461in}{1.175371in}}%
\pgfpathlineto{\pgfqpoint{3.070461in}{1.178320in}}%
\pgfpathlineto{\pgfqpoint{3.075002in}{1.178320in}}%
\pgfpathlineto{\pgfqpoint{3.075002in}{1.175371in}}%
\pgfpathmoveto{\pgfqpoint{3.070461in}{1.178320in}}%
\pgfpathlineto{\pgfqpoint{3.070461in}{1.178320in}}%
\pgfpathlineto{\pgfqpoint{3.070461in}{1.181270in}}%
\pgfpathlineto{\pgfqpoint{3.075002in}{1.181270in}}%
\pgfpathlineto{\pgfqpoint{3.075002in}{1.178320in}}%
\pgfpathmoveto{\pgfqpoint{3.070461in}{1.181270in}}%
\pgfpathlineto{\pgfqpoint{3.070461in}{1.181270in}}%
\pgfpathlineto{\pgfqpoint{3.070461in}{1.184219in}}%
\pgfpathlineto{\pgfqpoint{3.075002in}{1.184219in}}%
\pgfpathlineto{\pgfqpoint{3.075002in}{1.181270in}}%
\pgfpathmoveto{\pgfqpoint{3.070461in}{1.184219in}}%
\pgfpathlineto{\pgfqpoint{3.070461in}{1.184219in}}%
\pgfpathlineto{\pgfqpoint{3.070461in}{1.187168in}}%
\pgfpathlineto{\pgfqpoint{3.075002in}{1.187168in}}%
\pgfpathlineto{\pgfqpoint{3.075002in}{1.184219in}}%
\pgfpathmoveto{\pgfqpoint{3.070461in}{1.187168in}}%
\pgfpathlineto{\pgfqpoint{3.070461in}{1.187168in}}%
\pgfpathlineto{\pgfqpoint{3.070461in}{1.190117in}}%
\pgfpathlineto{\pgfqpoint{3.075002in}{1.190117in}}%
\pgfpathlineto{\pgfqpoint{3.075002in}{1.187168in}}%
\pgfpathmoveto{\pgfqpoint{3.070461in}{1.190117in}}%
\pgfpathlineto{\pgfqpoint{3.070461in}{1.190117in}}%
\pgfpathlineto{\pgfqpoint{3.070461in}{1.193066in}}%
\pgfpathlineto{\pgfqpoint{3.075002in}{1.193066in}}%
\pgfpathlineto{\pgfqpoint{3.075002in}{1.190117in}}%
\pgfpathmoveto{\pgfqpoint{3.070461in}{1.193066in}}%
\pgfpathlineto{\pgfqpoint{3.070461in}{1.193066in}}%
\pgfpathlineto{\pgfqpoint{3.070461in}{1.196015in}}%
\pgfpathlineto{\pgfqpoint{3.075002in}{1.196015in}}%
\pgfpathlineto{\pgfqpoint{3.075002in}{1.193066in}}%
\pgfpathmoveto{\pgfqpoint{3.070461in}{1.196015in}}%
\pgfpathlineto{\pgfqpoint{3.070461in}{1.196015in}}%
\pgfpathlineto{\pgfqpoint{3.070461in}{1.198964in}}%
\pgfpathlineto{\pgfqpoint{3.075002in}{1.198964in}}%
\pgfpathlineto{\pgfqpoint{3.075002in}{1.196015in}}%
\pgfpathmoveto{\pgfqpoint{3.070461in}{1.198964in}}%
\pgfpathlineto{\pgfqpoint{3.070461in}{1.198964in}}%
\pgfpathlineto{\pgfqpoint{3.070461in}{1.201913in}}%
\pgfpathlineto{\pgfqpoint{3.075002in}{1.201913in}}%
\pgfpathlineto{\pgfqpoint{3.075002in}{1.198964in}}%
\pgfpathmoveto{\pgfqpoint{3.070461in}{1.201913in}}%
\pgfpathlineto{\pgfqpoint{3.070461in}{1.201913in}}%
\pgfpathlineto{\pgfqpoint{3.070461in}{1.204863in}}%
\pgfpathlineto{\pgfqpoint{3.075002in}{1.204863in}}%
\pgfpathlineto{\pgfqpoint{3.075002in}{1.201913in}}%
\pgfpathmoveto{\pgfqpoint{3.070461in}{1.204863in}}%
\pgfpathlineto{\pgfqpoint{3.070461in}{1.204863in}}%
\pgfpathlineto{\pgfqpoint{3.070461in}{1.207812in}}%
\pgfpathlineto{\pgfqpoint{3.075002in}{1.207812in}}%
\pgfpathlineto{\pgfqpoint{3.075002in}{1.204863in}}%
\pgfpathmoveto{\pgfqpoint{3.070461in}{1.207812in}}%
\pgfpathlineto{\pgfqpoint{3.070461in}{1.207812in}}%
\pgfpathlineto{\pgfqpoint{3.070461in}{1.210761in}}%
\pgfpathlineto{\pgfqpoint{3.075002in}{1.210761in}}%
\pgfpathlineto{\pgfqpoint{3.075002in}{1.207812in}}%
\pgfpathmoveto{\pgfqpoint{3.070461in}{1.210761in}}%
\pgfpathlineto{\pgfqpoint{3.070461in}{1.210761in}}%
\pgfpathlineto{\pgfqpoint{3.070461in}{1.213710in}}%
\pgfpathlineto{\pgfqpoint{3.075002in}{1.213710in}}%
\pgfpathlineto{\pgfqpoint{3.075002in}{1.210761in}}%
\pgfpathmoveto{\pgfqpoint{3.070461in}{1.213710in}}%
\pgfpathlineto{\pgfqpoint{3.070461in}{1.213710in}}%
\pgfpathlineto{\pgfqpoint{3.070461in}{1.216659in}}%
\pgfpathlineto{\pgfqpoint{3.075002in}{1.216659in}}%
\pgfpathlineto{\pgfqpoint{3.075002in}{1.213710in}}%
\pgfpathmoveto{\pgfqpoint{3.070461in}{1.216659in}}%
\pgfpathlineto{\pgfqpoint{3.070461in}{1.216659in}}%
\pgfpathlineto{\pgfqpoint{3.070461in}{1.219608in}}%
\pgfpathlineto{\pgfqpoint{3.075002in}{1.219608in}}%
\pgfpathlineto{\pgfqpoint{3.075002in}{1.216659in}}%
\pgfpathmoveto{\pgfqpoint{3.070461in}{1.219608in}}%
\pgfpathlineto{\pgfqpoint{3.070461in}{1.219608in}}%
\pgfpathlineto{\pgfqpoint{3.070461in}{1.222557in}}%
\pgfpathlineto{\pgfqpoint{3.075002in}{1.222557in}}%
\pgfpathlineto{\pgfqpoint{3.075002in}{1.219608in}}%
\pgfpathmoveto{\pgfqpoint{3.070461in}{1.222557in}}%
\pgfpathlineto{\pgfqpoint{3.070461in}{1.222557in}}%
\pgfpathlineto{\pgfqpoint{3.070461in}{1.225506in}}%
\pgfpathlineto{\pgfqpoint{3.075002in}{1.225506in}}%
\pgfpathlineto{\pgfqpoint{3.075002in}{1.222557in}}%
\pgfpathmoveto{\pgfqpoint{3.070461in}{1.225506in}}%
\pgfpathlineto{\pgfqpoint{3.070461in}{1.225506in}}%
\pgfpathlineto{\pgfqpoint{3.070461in}{1.228456in}}%
\pgfpathlineto{\pgfqpoint{3.075002in}{1.228456in}}%
\pgfpathlineto{\pgfqpoint{3.075002in}{1.225506in}}%
\pgfpathmoveto{\pgfqpoint{3.070461in}{1.228456in}}%
\pgfpathlineto{\pgfqpoint{3.070461in}{1.228456in}}%
\pgfpathlineto{\pgfqpoint{3.070461in}{1.231405in}}%
\pgfpathlineto{\pgfqpoint{3.075002in}{1.231405in}}%
\pgfpathlineto{\pgfqpoint{3.075002in}{1.228456in}}%
\pgfpathmoveto{\pgfqpoint{3.070461in}{1.231405in}}%
\pgfpathlineto{\pgfqpoint{3.070461in}{1.231405in}}%
\pgfpathlineto{\pgfqpoint{3.070461in}{1.234354in}}%
\pgfpathlineto{\pgfqpoint{3.075002in}{1.234354in}}%
\pgfpathlineto{\pgfqpoint{3.075002in}{1.231405in}}%
\pgfpathmoveto{\pgfqpoint{3.070461in}{1.234354in}}%
\pgfpathlineto{\pgfqpoint{3.070461in}{1.234354in}}%
\pgfpathlineto{\pgfqpoint{3.070461in}{1.237303in}}%
\pgfpathlineto{\pgfqpoint{3.075002in}{1.237303in}}%
\pgfpathlineto{\pgfqpoint{3.075002in}{1.234354in}}%
\pgfpathmoveto{\pgfqpoint{3.070461in}{1.237303in}}%
\pgfpathlineto{\pgfqpoint{3.070461in}{1.237303in}}%
\pgfpathlineto{\pgfqpoint{3.070461in}{1.240252in}}%
\pgfpathlineto{\pgfqpoint{3.075002in}{1.240252in}}%
\pgfpathlineto{\pgfqpoint{3.075002in}{1.237303in}}%
\pgfpathmoveto{\pgfqpoint{3.070461in}{1.240252in}}%
\pgfpathlineto{\pgfqpoint{3.070461in}{1.240252in}}%
\pgfpathlineto{\pgfqpoint{3.070461in}{1.243201in}}%
\pgfpathlineto{\pgfqpoint{3.075002in}{1.243201in}}%
\pgfpathlineto{\pgfqpoint{3.075002in}{1.240252in}}%
\pgfpathmoveto{\pgfqpoint{3.070461in}{1.243201in}}%
\pgfpathlineto{\pgfqpoint{3.070461in}{1.243201in}}%
\pgfpathlineto{\pgfqpoint{3.070461in}{1.246150in}}%
\pgfpathlineto{\pgfqpoint{3.075002in}{1.246150in}}%
\pgfpathlineto{\pgfqpoint{3.075002in}{1.243201in}}%
\pgfpathmoveto{\pgfqpoint{3.070461in}{1.246150in}}%
\pgfpathlineto{\pgfqpoint{3.070461in}{1.246150in}}%
\pgfpathlineto{\pgfqpoint{3.070461in}{1.249099in}}%
\pgfpathlineto{\pgfqpoint{3.075002in}{1.249099in}}%
\pgfpathlineto{\pgfqpoint{3.075002in}{1.246150in}}%
\pgfpathmoveto{\pgfqpoint{3.070461in}{1.249099in}}%
\pgfpathlineto{\pgfqpoint{3.070461in}{1.249099in}}%
\pgfpathlineto{\pgfqpoint{3.070461in}{1.252049in}}%
\pgfpathlineto{\pgfqpoint{3.075002in}{1.252049in}}%
\pgfpathlineto{\pgfqpoint{3.075002in}{1.249099in}}%
\pgfpathmoveto{\pgfqpoint{3.070461in}{1.252049in}}%
\pgfpathlineto{\pgfqpoint{3.070461in}{1.252049in}}%
\pgfpathlineto{\pgfqpoint{3.070461in}{1.254998in}}%
\pgfpathlineto{\pgfqpoint{3.075002in}{1.254998in}}%
\pgfpathlineto{\pgfqpoint{3.075002in}{1.252049in}}%
\pgfpathmoveto{\pgfqpoint{3.070461in}{1.254998in}}%
\pgfpathlineto{\pgfqpoint{3.070461in}{1.254998in}}%
\pgfpathlineto{\pgfqpoint{3.070461in}{1.257947in}}%
\pgfpathlineto{\pgfqpoint{3.075002in}{1.257947in}}%
\pgfpathlineto{\pgfqpoint{3.075002in}{1.254998in}}%
\pgfpathmoveto{\pgfqpoint{3.070461in}{1.257947in}}%
\pgfpathlineto{\pgfqpoint{3.070461in}{1.257947in}}%
\pgfpathlineto{\pgfqpoint{3.070461in}{1.260896in}}%
\pgfpathlineto{\pgfqpoint{3.075002in}{1.260896in}}%
\pgfpathlineto{\pgfqpoint{3.075002in}{1.257947in}}%
\pgfpathmoveto{\pgfqpoint{3.070461in}{1.260896in}}%
\pgfpathlineto{\pgfqpoint{3.070461in}{1.260896in}}%
\pgfpathlineto{\pgfqpoint{3.070461in}{1.263846in}}%
\pgfpathlineto{\pgfqpoint{3.075002in}{1.263846in}}%
\pgfpathlineto{\pgfqpoint{3.075002in}{1.260896in}}%
\pgfpathmoveto{\pgfqpoint{3.070461in}{1.263846in}}%
\pgfpathlineto{\pgfqpoint{3.070461in}{1.263846in}}%
\pgfpathlineto{\pgfqpoint{3.070461in}{1.266795in}}%
\pgfpathlineto{\pgfqpoint{3.075002in}{1.266795in}}%
\pgfpathlineto{\pgfqpoint{3.075002in}{1.263846in}}%
\pgfpathmoveto{\pgfqpoint{3.070461in}{1.266795in}}%
\pgfpathlineto{\pgfqpoint{3.070461in}{1.266795in}}%
\pgfpathlineto{\pgfqpoint{3.070461in}{1.269744in}}%
\pgfpathlineto{\pgfqpoint{3.075002in}{1.269744in}}%
\pgfpathlineto{\pgfqpoint{3.075002in}{1.266795in}}%
\pgfpathmoveto{\pgfqpoint{3.070461in}{1.269744in}}%
\pgfpathlineto{\pgfqpoint{3.070461in}{1.269744in}}%
\pgfpathlineto{\pgfqpoint{3.070461in}{1.272694in}}%
\pgfpathlineto{\pgfqpoint{3.075002in}{1.272694in}}%
\pgfpathlineto{\pgfqpoint{3.075002in}{1.269744in}}%
\pgfpathmoveto{\pgfqpoint{3.070461in}{1.272694in}}%
\pgfpathlineto{\pgfqpoint{3.070461in}{1.272694in}}%
\pgfpathlineto{\pgfqpoint{3.070461in}{1.275643in}}%
\pgfpathlineto{\pgfqpoint{3.075002in}{1.275643in}}%
\pgfpathlineto{\pgfqpoint{3.075002in}{1.272694in}}%
\pgfpathmoveto{\pgfqpoint{3.070461in}{1.275643in}}%
\pgfpathlineto{\pgfqpoint{3.070461in}{1.275643in}}%
\pgfpathlineto{\pgfqpoint{3.070461in}{1.278592in}}%
\pgfpathlineto{\pgfqpoint{3.075002in}{1.278592in}}%
\pgfpathlineto{\pgfqpoint{3.075002in}{1.275643in}}%
\pgfpathmoveto{\pgfqpoint{3.070461in}{1.278592in}}%
\pgfpathlineto{\pgfqpoint{3.070461in}{1.278592in}}%
\pgfpathlineto{\pgfqpoint{3.070461in}{1.281541in}}%
\pgfpathlineto{\pgfqpoint{3.075002in}{1.281541in}}%
\pgfpathlineto{\pgfqpoint{3.075002in}{1.278592in}}%
\pgfpathmoveto{\pgfqpoint{3.070461in}{1.281541in}}%
\pgfpathlineto{\pgfqpoint{3.070461in}{1.281541in}}%
\pgfpathlineto{\pgfqpoint{3.070461in}{1.284491in}}%
\pgfpathlineto{\pgfqpoint{3.075002in}{1.284491in}}%
\pgfpathlineto{\pgfqpoint{3.075002in}{1.281541in}}%
\pgfpathmoveto{\pgfqpoint{3.070461in}{1.284491in}}%
\pgfpathlineto{\pgfqpoint{3.070461in}{1.284491in}}%
\pgfpathlineto{\pgfqpoint{3.070461in}{1.287440in}}%
\pgfpathlineto{\pgfqpoint{3.075002in}{1.287440in}}%
\pgfpathlineto{\pgfqpoint{3.075002in}{1.284491in}}%
\pgfpathmoveto{\pgfqpoint{3.070461in}{1.287440in}}%
\pgfpathlineto{\pgfqpoint{3.070461in}{1.287440in}}%
\pgfpathlineto{\pgfqpoint{3.070461in}{1.290389in}}%
\pgfpathlineto{\pgfqpoint{3.075002in}{1.290389in}}%
\pgfpathlineto{\pgfqpoint{3.075002in}{1.287440in}}%
\pgfpathmoveto{\pgfqpoint{3.070461in}{1.290389in}}%
\pgfpathlineto{\pgfqpoint{3.070461in}{1.290389in}}%
\pgfpathlineto{\pgfqpoint{3.070461in}{1.293339in}}%
\pgfpathlineto{\pgfqpoint{3.075002in}{1.293339in}}%
\pgfpathlineto{\pgfqpoint{3.075002in}{1.290389in}}%
\pgfpathmoveto{\pgfqpoint{3.070461in}{1.293339in}}%
\pgfpathlineto{\pgfqpoint{3.070461in}{1.293339in}}%
\pgfpathlineto{\pgfqpoint{3.070461in}{1.296288in}}%
\pgfpathlineto{\pgfqpoint{3.075002in}{1.296288in}}%
\pgfpathlineto{\pgfqpoint{3.075002in}{1.293339in}}%
\pgfpathmoveto{\pgfqpoint{3.070461in}{1.296288in}}%
\pgfpathlineto{\pgfqpoint{3.070461in}{1.296288in}}%
\pgfpathlineto{\pgfqpoint{3.070461in}{1.299237in}}%
\pgfpathlineto{\pgfqpoint{3.075002in}{1.299237in}}%
\pgfpathlineto{\pgfqpoint{3.075002in}{1.296288in}}%
\pgfpathmoveto{\pgfqpoint{3.070461in}{1.299237in}}%
\pgfpathlineto{\pgfqpoint{3.070461in}{1.299237in}}%
\pgfpathlineto{\pgfqpoint{3.070461in}{1.302187in}}%
\pgfpathlineto{\pgfqpoint{3.075002in}{1.302187in}}%
\pgfpathlineto{\pgfqpoint{3.075002in}{1.299237in}}%
\pgfpathmoveto{\pgfqpoint{3.070461in}{1.302187in}}%
\pgfpathlineto{\pgfqpoint{3.070461in}{1.302187in}}%
\pgfpathlineto{\pgfqpoint{3.070461in}{1.305136in}}%
\pgfpathlineto{\pgfqpoint{3.075002in}{1.305136in}}%
\pgfpathlineto{\pgfqpoint{3.075002in}{1.302187in}}%
\pgfpathmoveto{\pgfqpoint{3.070461in}{1.305136in}}%
\pgfpathlineto{\pgfqpoint{3.070461in}{1.305136in}}%
\pgfpathlineto{\pgfqpoint{3.070461in}{1.308085in}}%
\pgfpathlineto{\pgfqpoint{3.075002in}{1.308085in}}%
\pgfpathlineto{\pgfqpoint{3.075002in}{1.305136in}}%
\pgfpathmoveto{\pgfqpoint{3.070461in}{1.308085in}}%
\pgfpathlineto{\pgfqpoint{3.070461in}{1.308085in}}%
\pgfpathlineto{\pgfqpoint{3.070461in}{1.311035in}}%
\pgfpathlineto{\pgfqpoint{3.075002in}{1.311035in}}%
\pgfpathlineto{\pgfqpoint{3.075002in}{1.308085in}}%
\pgfpathmoveto{\pgfqpoint{3.070461in}{1.311035in}}%
\pgfpathlineto{\pgfqpoint{3.070461in}{1.311035in}}%
\pgfpathlineto{\pgfqpoint{3.070461in}{1.313984in}}%
\pgfpathlineto{\pgfqpoint{3.075002in}{1.313984in}}%
\pgfpathlineto{\pgfqpoint{3.075002in}{1.311035in}}%
\pgfpathmoveto{\pgfqpoint{3.070461in}{1.313984in}}%
\pgfpathlineto{\pgfqpoint{3.070461in}{1.313984in}}%
\pgfpathlineto{\pgfqpoint{3.070461in}{1.316933in}}%
\pgfpathlineto{\pgfqpoint{3.075002in}{1.316933in}}%
\pgfpathlineto{\pgfqpoint{3.075002in}{1.313984in}}%
\pgfpathmoveto{\pgfqpoint{3.070461in}{1.316933in}}%
\pgfpathlineto{\pgfqpoint{3.070461in}{1.316933in}}%
\pgfpathlineto{\pgfqpoint{3.070461in}{1.319882in}}%
\pgfpathlineto{\pgfqpoint{3.075002in}{1.319882in}}%
\pgfpathlineto{\pgfqpoint{3.075002in}{1.316933in}}%
\pgfpathmoveto{\pgfqpoint{3.070461in}{1.319882in}}%
\pgfpathlineto{\pgfqpoint{3.070461in}{1.319882in}}%
\pgfpathlineto{\pgfqpoint{3.070461in}{1.322832in}}%
\pgfpathlineto{\pgfqpoint{3.075002in}{1.322832in}}%
\pgfpathlineto{\pgfqpoint{3.075002in}{1.319882in}}%
\pgfpathmoveto{\pgfqpoint{3.070461in}{1.322832in}}%
\pgfpathlineto{\pgfqpoint{3.070461in}{1.322832in}}%
\pgfpathlineto{\pgfqpoint{3.070461in}{1.325781in}}%
\pgfpathlineto{\pgfqpoint{3.075002in}{1.325781in}}%
\pgfpathlineto{\pgfqpoint{3.075002in}{1.322832in}}%
\pgfpathmoveto{\pgfqpoint{3.070461in}{1.325781in}}%
\pgfpathlineto{\pgfqpoint{3.070461in}{1.325781in}}%
\pgfpathlineto{\pgfqpoint{3.070461in}{1.328730in}}%
\pgfpathlineto{\pgfqpoint{3.075002in}{1.328730in}}%
\pgfpathlineto{\pgfqpoint{3.075002in}{1.325781in}}%
\pgfpathmoveto{\pgfqpoint{3.070461in}{1.328730in}}%
\pgfpathlineto{\pgfqpoint{3.070461in}{1.328730in}}%
\pgfpathlineto{\pgfqpoint{3.070461in}{1.331680in}}%
\pgfpathlineto{\pgfqpoint{3.075002in}{1.331680in}}%
\pgfpathlineto{\pgfqpoint{3.075002in}{1.328730in}}%
\pgfpathmoveto{\pgfqpoint{3.070461in}{1.331680in}}%
\pgfpathlineto{\pgfqpoint{3.070461in}{1.331680in}}%
\pgfpathlineto{\pgfqpoint{3.070461in}{1.334629in}}%
\pgfpathlineto{\pgfqpoint{3.075002in}{1.334629in}}%
\pgfpathlineto{\pgfqpoint{3.075002in}{1.331680in}}%
\pgfpathmoveto{\pgfqpoint{3.070461in}{1.334629in}}%
\pgfpathlineto{\pgfqpoint{3.070461in}{1.334629in}}%
\pgfpathlineto{\pgfqpoint{3.070461in}{1.337578in}}%
\pgfpathlineto{\pgfqpoint{3.075002in}{1.337578in}}%
\pgfpathlineto{\pgfqpoint{3.075002in}{1.334629in}}%
\pgfpathmoveto{\pgfqpoint{3.070461in}{1.337578in}}%
\pgfpathlineto{\pgfqpoint{3.070461in}{1.337578in}}%
\pgfpathlineto{\pgfqpoint{3.070461in}{1.340528in}}%
\pgfpathlineto{\pgfqpoint{3.075002in}{1.340528in}}%
\pgfpathlineto{\pgfqpoint{3.075002in}{1.337578in}}%
\pgfpathmoveto{\pgfqpoint{3.070461in}{1.340528in}}%
\pgfpathlineto{\pgfqpoint{3.070461in}{1.340528in}}%
\pgfpathlineto{\pgfqpoint{3.070461in}{1.343477in}}%
\pgfpathlineto{\pgfqpoint{3.075002in}{1.343477in}}%
\pgfpathlineto{\pgfqpoint{3.075002in}{1.340528in}}%
\pgfpathmoveto{\pgfqpoint{3.070461in}{1.343477in}}%
\pgfpathlineto{\pgfqpoint{3.070461in}{1.343477in}}%
\pgfpathlineto{\pgfqpoint{3.070461in}{1.346426in}}%
\pgfpathlineto{\pgfqpoint{3.075002in}{1.346426in}}%
\pgfpathlineto{\pgfqpoint{3.075002in}{1.343477in}}%
\pgfpathmoveto{\pgfqpoint{3.070461in}{1.346426in}}%
\pgfpathlineto{\pgfqpoint{3.070461in}{1.346426in}}%
\pgfpathlineto{\pgfqpoint{3.070461in}{1.349376in}}%
\pgfpathlineto{\pgfqpoint{3.075002in}{1.349376in}}%
\pgfpathlineto{\pgfqpoint{3.075002in}{1.346426in}}%
\pgfpathmoveto{\pgfqpoint{3.070461in}{1.349376in}}%
\pgfpathlineto{\pgfqpoint{3.070461in}{1.349376in}}%
\pgfpathlineto{\pgfqpoint{3.070461in}{1.352325in}}%
\pgfpathlineto{\pgfqpoint{3.075002in}{1.352325in}}%
\pgfpathlineto{\pgfqpoint{3.075002in}{1.349376in}}%
\pgfpathmoveto{\pgfqpoint{3.070461in}{1.352325in}}%
\pgfpathlineto{\pgfqpoint{3.070461in}{1.352325in}}%
\pgfpathlineto{\pgfqpoint{3.070461in}{1.355274in}}%
\pgfpathlineto{\pgfqpoint{3.075002in}{1.355274in}}%
\pgfpathlineto{\pgfqpoint{3.075002in}{1.352325in}}%
\pgfpathmoveto{\pgfqpoint{3.070461in}{1.355274in}}%
\pgfpathlineto{\pgfqpoint{3.070461in}{1.355274in}}%
\pgfpathlineto{\pgfqpoint{3.070461in}{1.358223in}}%
\pgfpathlineto{\pgfqpoint{3.075002in}{1.358223in}}%
\pgfpathlineto{\pgfqpoint{3.075002in}{1.355274in}}%
\pgfpathmoveto{\pgfqpoint{3.070461in}{1.358223in}}%
\pgfpathlineto{\pgfqpoint{3.070461in}{1.358223in}}%
\pgfpathlineto{\pgfqpoint{3.070461in}{1.361173in}}%
\pgfpathlineto{\pgfqpoint{3.075002in}{1.361173in}}%
\pgfpathlineto{\pgfqpoint{3.075002in}{1.358223in}}%
\pgfpathmoveto{\pgfqpoint{3.070461in}{1.361173in}}%
\pgfpathlineto{\pgfqpoint{3.070461in}{1.361173in}}%
\pgfpathlineto{\pgfqpoint{3.070461in}{1.364122in}}%
\pgfpathlineto{\pgfqpoint{3.075002in}{1.364122in}}%
\pgfpathlineto{\pgfqpoint{3.075002in}{1.361173in}}%
\pgfpathmoveto{\pgfqpoint{3.070461in}{1.364122in}}%
\pgfpathlineto{\pgfqpoint{3.070461in}{1.364122in}}%
\pgfpathlineto{\pgfqpoint{3.070461in}{1.367071in}}%
\pgfpathlineto{\pgfqpoint{3.075002in}{1.367071in}}%
\pgfpathlineto{\pgfqpoint{3.075002in}{1.364122in}}%
\pgfpathmoveto{\pgfqpoint{3.070461in}{1.367071in}}%
\pgfpathlineto{\pgfqpoint{3.070461in}{1.367071in}}%
\pgfpathlineto{\pgfqpoint{3.070461in}{1.370020in}}%
\pgfpathlineto{\pgfqpoint{3.075002in}{1.370020in}}%
\pgfpathlineto{\pgfqpoint{3.075002in}{1.367071in}}%
\pgfpathmoveto{\pgfqpoint{3.070461in}{1.370020in}}%
\pgfpathlineto{\pgfqpoint{3.070461in}{1.370020in}}%
\pgfpathlineto{\pgfqpoint{3.070461in}{1.372970in}}%
\pgfpathlineto{\pgfqpoint{3.075002in}{1.372970in}}%
\pgfpathlineto{\pgfqpoint{3.075002in}{1.370020in}}%
\pgfpathmoveto{\pgfqpoint{3.070461in}{1.372970in}}%
\pgfpathlineto{\pgfqpoint{3.070461in}{1.372970in}}%
\pgfpathlineto{\pgfqpoint{3.070461in}{1.375919in}}%
\pgfpathlineto{\pgfqpoint{3.075002in}{1.375919in}}%
\pgfpathlineto{\pgfqpoint{3.075002in}{1.372970in}}%
\pgfpathmoveto{\pgfqpoint{3.070461in}{1.375919in}}%
\pgfpathlineto{\pgfqpoint{3.070461in}{1.375919in}}%
\pgfpathlineto{\pgfqpoint{3.070461in}{1.378868in}}%
\pgfpathlineto{\pgfqpoint{3.075002in}{1.378868in}}%
\pgfpathlineto{\pgfqpoint{3.075002in}{1.375919in}}%
\pgfpathmoveto{\pgfqpoint{3.070461in}{1.378868in}}%
\pgfpathlineto{\pgfqpoint{3.070461in}{1.378868in}}%
\pgfpathlineto{\pgfqpoint{3.070461in}{1.381817in}}%
\pgfpathlineto{\pgfqpoint{3.075002in}{1.381817in}}%
\pgfpathlineto{\pgfqpoint{3.075002in}{1.378868in}}%
\pgfpathmoveto{\pgfqpoint{3.070461in}{1.381817in}}%
\pgfpathlineto{\pgfqpoint{3.070461in}{1.381817in}}%
\pgfpathlineto{\pgfqpoint{3.070461in}{1.384767in}}%
\pgfpathlineto{\pgfqpoint{3.075002in}{1.384767in}}%
\pgfpathlineto{\pgfqpoint{3.075002in}{1.381817in}}%
\pgfpathmoveto{\pgfqpoint{3.070461in}{1.384767in}}%
\pgfpathlineto{\pgfqpoint{3.070461in}{1.384767in}}%
\pgfpathlineto{\pgfqpoint{3.070461in}{1.387716in}}%
\pgfpathlineto{\pgfqpoint{3.075002in}{1.387716in}}%
\pgfpathlineto{\pgfqpoint{3.075002in}{1.384767in}}%
\pgfpathmoveto{\pgfqpoint{3.070461in}{1.387716in}}%
\pgfpathlineto{\pgfqpoint{3.070461in}{1.387716in}}%
\pgfpathlineto{\pgfqpoint{3.070461in}{1.390665in}}%
\pgfpathlineto{\pgfqpoint{3.075002in}{1.390665in}}%
\pgfpathlineto{\pgfqpoint{3.075002in}{1.387716in}}%
\pgfpathmoveto{\pgfqpoint{3.070461in}{1.390665in}}%
\pgfpathlineto{\pgfqpoint{3.070461in}{1.390665in}}%
\pgfpathlineto{\pgfqpoint{3.070461in}{1.393614in}}%
\pgfpathlineto{\pgfqpoint{3.075002in}{1.393614in}}%
\pgfpathlineto{\pgfqpoint{3.075002in}{1.390665in}}%
\pgfpathmoveto{\pgfqpoint{3.070461in}{1.393614in}}%
\pgfpathlineto{\pgfqpoint{3.070461in}{1.393614in}}%
\pgfpathlineto{\pgfqpoint{3.070461in}{1.396564in}}%
\pgfpathlineto{\pgfqpoint{3.075002in}{1.396564in}}%
\pgfpathlineto{\pgfqpoint{3.075002in}{1.393614in}}%
\pgfpathmoveto{\pgfqpoint{3.070461in}{1.396564in}}%
\pgfpathlineto{\pgfqpoint{3.070461in}{1.396564in}}%
\pgfpathlineto{\pgfqpoint{3.070461in}{1.399513in}}%
\pgfpathlineto{\pgfqpoint{3.075002in}{1.399513in}}%
\pgfpathlineto{\pgfqpoint{3.075002in}{1.396564in}}%
\pgfpathmoveto{\pgfqpoint{3.070461in}{1.399513in}}%
\pgfpathlineto{\pgfqpoint{3.070461in}{1.399513in}}%
\pgfpathlineto{\pgfqpoint{3.070461in}{1.402462in}}%
\pgfpathlineto{\pgfqpoint{3.075002in}{1.402462in}}%
\pgfpathlineto{\pgfqpoint{3.075002in}{1.399513in}}%
\pgfpathmoveto{\pgfqpoint{3.070461in}{1.402462in}}%
\pgfpathlineto{\pgfqpoint{3.070461in}{1.402462in}}%
\pgfpathlineto{\pgfqpoint{3.070461in}{1.405411in}}%
\pgfpathlineto{\pgfqpoint{3.075002in}{1.405411in}}%
\pgfpathlineto{\pgfqpoint{3.075002in}{1.402462in}}%
\pgfpathmoveto{\pgfqpoint{3.070461in}{1.405411in}}%
\pgfpathlineto{\pgfqpoint{3.070461in}{1.405411in}}%
\pgfpathlineto{\pgfqpoint{3.070461in}{1.408361in}}%
\pgfpathlineto{\pgfqpoint{3.075002in}{1.408361in}}%
\pgfpathlineto{\pgfqpoint{3.075002in}{1.405411in}}%
\pgfpathmoveto{\pgfqpoint{3.070461in}{1.408361in}}%
\pgfpathlineto{\pgfqpoint{3.070461in}{1.408361in}}%
\pgfpathlineto{\pgfqpoint{3.070461in}{1.411310in}}%
\pgfpathlineto{\pgfqpoint{3.075002in}{1.411310in}}%
\pgfpathlineto{\pgfqpoint{3.075002in}{1.408361in}}%
\pgfpathmoveto{\pgfqpoint{3.070461in}{1.411310in}}%
\pgfpathlineto{\pgfqpoint{3.070461in}{1.411310in}}%
\pgfpathlineto{\pgfqpoint{3.070461in}{1.414259in}}%
\pgfpathlineto{\pgfqpoint{3.075002in}{1.414259in}}%
\pgfpathlineto{\pgfqpoint{3.075002in}{1.411310in}}%
\pgfpathmoveto{\pgfqpoint{3.070461in}{1.414259in}}%
\pgfpathlineto{\pgfqpoint{3.070461in}{1.414259in}}%
\pgfpathlineto{\pgfqpoint{3.070461in}{1.417209in}}%
\pgfpathlineto{\pgfqpoint{3.075002in}{1.417209in}}%
\pgfpathlineto{\pgfqpoint{3.075002in}{1.414259in}}%
\pgfpathmoveto{\pgfqpoint{3.070461in}{1.417209in}}%
\pgfpathlineto{\pgfqpoint{3.070461in}{1.417209in}}%
\pgfpathlineto{\pgfqpoint{3.070461in}{1.420158in}}%
\pgfpathlineto{\pgfqpoint{3.075002in}{1.420158in}}%
\pgfpathlineto{\pgfqpoint{3.075002in}{1.417209in}}%
\pgfpathmoveto{\pgfqpoint{3.070461in}{1.420158in}}%
\pgfpathlineto{\pgfqpoint{3.070461in}{1.420158in}}%
\pgfpathlineto{\pgfqpoint{3.070461in}{1.423107in}}%
\pgfpathlineto{\pgfqpoint{3.075002in}{1.423107in}}%
\pgfpathlineto{\pgfqpoint{3.075002in}{1.420158in}}%
\pgfpathmoveto{\pgfqpoint{3.070461in}{1.423107in}}%
\pgfpathlineto{\pgfqpoint{3.070461in}{1.423107in}}%
\pgfpathlineto{\pgfqpoint{3.070461in}{1.426056in}}%
\pgfpathlineto{\pgfqpoint{3.075002in}{1.426056in}}%
\pgfpathlineto{\pgfqpoint{3.075002in}{1.423107in}}%
\pgfpathmoveto{\pgfqpoint{3.070461in}{1.426056in}}%
\pgfpathlineto{\pgfqpoint{3.070461in}{1.426056in}}%
\pgfpathlineto{\pgfqpoint{3.070461in}{1.429006in}}%
\pgfpathlineto{\pgfqpoint{3.075002in}{1.429006in}}%
\pgfpathlineto{\pgfqpoint{3.075002in}{1.426056in}}%
\pgfpathmoveto{\pgfqpoint{3.070461in}{1.429006in}}%
\pgfpathlineto{\pgfqpoint{3.070461in}{1.429006in}}%
\pgfpathlineto{\pgfqpoint{3.070461in}{1.431955in}}%
\pgfpathlineto{\pgfqpoint{3.075002in}{1.431955in}}%
\pgfpathlineto{\pgfqpoint{3.075002in}{1.429006in}}%
\pgfpathmoveto{\pgfqpoint{3.070461in}{1.431955in}}%
\pgfpathlineto{\pgfqpoint{3.070461in}{1.431955in}}%
\pgfpathlineto{\pgfqpoint{3.070461in}{1.434904in}}%
\pgfpathlineto{\pgfqpoint{3.075002in}{1.434904in}}%
\pgfpathlineto{\pgfqpoint{3.075002in}{1.431955in}}%
\pgfpathmoveto{\pgfqpoint{3.070461in}{1.434904in}}%
\pgfpathlineto{\pgfqpoint{3.070461in}{1.434904in}}%
\pgfpathlineto{\pgfqpoint{3.070461in}{1.437853in}}%
\pgfpathlineto{\pgfqpoint{3.075002in}{1.437853in}}%
\pgfpathlineto{\pgfqpoint{3.075002in}{1.434904in}}%
\pgfpathmoveto{\pgfqpoint{3.070461in}{1.437853in}}%
\pgfpathlineto{\pgfqpoint{3.070461in}{1.437853in}}%
\pgfpathlineto{\pgfqpoint{3.070461in}{1.440803in}}%
\pgfpathlineto{\pgfqpoint{3.075002in}{1.440803in}}%
\pgfpathlineto{\pgfqpoint{3.075002in}{1.437853in}}%
\pgfpathmoveto{\pgfqpoint{3.070461in}{1.440803in}}%
\pgfpathlineto{\pgfqpoint{3.070461in}{1.440803in}}%
\pgfpathlineto{\pgfqpoint{3.070461in}{1.443752in}}%
\pgfpathlineto{\pgfqpoint{3.075002in}{1.443752in}}%
\pgfpathlineto{\pgfqpoint{3.075002in}{1.440803in}}%
\pgfpathmoveto{\pgfqpoint{3.070461in}{1.443752in}}%
\pgfpathlineto{\pgfqpoint{3.070461in}{1.443752in}}%
\pgfpathlineto{\pgfqpoint{3.070461in}{1.446701in}}%
\pgfpathlineto{\pgfqpoint{3.075002in}{1.446701in}}%
\pgfpathlineto{\pgfqpoint{3.075002in}{1.443752in}}%
\pgfpathmoveto{\pgfqpoint{3.070461in}{1.446701in}}%
\pgfpathlineto{\pgfqpoint{3.070461in}{1.446701in}}%
\pgfpathlineto{\pgfqpoint{3.070461in}{1.449650in}}%
\pgfpathlineto{\pgfqpoint{3.075002in}{1.449650in}}%
\pgfpathlineto{\pgfqpoint{3.075002in}{1.446701in}}%
\pgfpathmoveto{\pgfqpoint{3.070461in}{1.449650in}}%
\pgfpathlineto{\pgfqpoint{3.070461in}{1.449650in}}%
\pgfpathlineto{\pgfqpoint{3.070461in}{1.452599in}}%
\pgfpathlineto{\pgfqpoint{3.075002in}{1.452599in}}%
\pgfpathlineto{\pgfqpoint{3.075002in}{1.449650in}}%
\pgfpathmoveto{\pgfqpoint{3.070461in}{1.452599in}}%
\pgfpathlineto{\pgfqpoint{3.070461in}{1.452599in}}%
\pgfpathlineto{\pgfqpoint{3.070461in}{1.455548in}}%
\pgfpathlineto{\pgfqpoint{3.075002in}{1.455548in}}%
\pgfpathlineto{\pgfqpoint{3.075002in}{1.452599in}}%
\pgfpathmoveto{\pgfqpoint{3.070461in}{1.455548in}}%
\pgfpathlineto{\pgfqpoint{3.070461in}{1.455548in}}%
\pgfpathlineto{\pgfqpoint{3.070461in}{1.458497in}}%
\pgfpathlineto{\pgfqpoint{3.075002in}{1.458497in}}%
\pgfpathlineto{\pgfqpoint{3.075002in}{1.455548in}}%
\pgfpathmoveto{\pgfqpoint{3.070461in}{1.458497in}}%
\pgfpathlineto{\pgfqpoint{3.070461in}{1.458497in}}%
\pgfpathlineto{\pgfqpoint{3.070461in}{1.461447in}}%
\pgfpathlineto{\pgfqpoint{3.075002in}{1.461447in}}%
\pgfpathlineto{\pgfqpoint{3.075002in}{1.458497in}}%
\pgfpathmoveto{\pgfqpoint{3.070461in}{1.461447in}}%
\pgfpathlineto{\pgfqpoint{3.070461in}{1.461447in}}%
\pgfpathlineto{\pgfqpoint{3.070461in}{1.464396in}}%
\pgfpathlineto{\pgfqpoint{3.075002in}{1.464396in}}%
\pgfpathlineto{\pgfqpoint{3.075002in}{1.461447in}}%
\pgfpathmoveto{\pgfqpoint{3.070461in}{1.464396in}}%
\pgfpathlineto{\pgfqpoint{3.070461in}{1.464396in}}%
\pgfpathlineto{\pgfqpoint{3.070461in}{1.467345in}}%
\pgfpathlineto{\pgfqpoint{3.075002in}{1.467345in}}%
\pgfpathlineto{\pgfqpoint{3.075002in}{1.464396in}}%
\pgfpathmoveto{\pgfqpoint{3.070461in}{1.467345in}}%
\pgfpathlineto{\pgfqpoint{3.070461in}{1.467345in}}%
\pgfpathlineto{\pgfqpoint{3.070461in}{1.470294in}}%
\pgfpathlineto{\pgfqpoint{3.075002in}{1.470294in}}%
\pgfpathlineto{\pgfqpoint{3.075002in}{1.467345in}}%
\pgfpathmoveto{\pgfqpoint{3.070461in}{1.470294in}}%
\pgfpathlineto{\pgfqpoint{3.070461in}{1.470294in}}%
\pgfpathlineto{\pgfqpoint{3.070461in}{1.473243in}}%
\pgfpathlineto{\pgfqpoint{3.075002in}{1.473243in}}%
\pgfpathlineto{\pgfqpoint{3.075002in}{1.470294in}}%
\pgfpathmoveto{\pgfqpoint{3.070461in}{1.473243in}}%
\pgfpathlineto{\pgfqpoint{3.070461in}{1.473243in}}%
\pgfpathlineto{\pgfqpoint{3.070461in}{1.476192in}}%
\pgfpathlineto{\pgfqpoint{3.075002in}{1.476192in}}%
\pgfpathlineto{\pgfqpoint{3.075002in}{1.473243in}}%
\pgfpathmoveto{\pgfqpoint{3.070461in}{1.476192in}}%
\pgfpathlineto{\pgfqpoint{3.070461in}{1.476192in}}%
\pgfpathlineto{\pgfqpoint{3.070461in}{1.479141in}}%
\pgfpathlineto{\pgfqpoint{3.075002in}{1.479141in}}%
\pgfpathlineto{\pgfqpoint{3.075002in}{1.476192in}}%
\pgfpathmoveto{\pgfqpoint{3.070461in}{1.479141in}}%
\pgfpathlineto{\pgfqpoint{3.070461in}{1.479141in}}%
\pgfpathlineto{\pgfqpoint{3.070461in}{1.482090in}}%
\pgfpathlineto{\pgfqpoint{3.075002in}{1.482090in}}%
\pgfpathlineto{\pgfqpoint{3.075002in}{1.479141in}}%
\pgfpathmoveto{\pgfqpoint{3.070461in}{1.482090in}}%
\pgfpathlineto{\pgfqpoint{3.070461in}{1.482090in}}%
\pgfpathlineto{\pgfqpoint{3.070461in}{1.485039in}}%
\pgfpathlineto{\pgfqpoint{3.075002in}{1.485039in}}%
\pgfpathlineto{\pgfqpoint{3.075002in}{1.482090in}}%
\pgfpathmoveto{\pgfqpoint{3.070461in}{1.485039in}}%
\pgfpathlineto{\pgfqpoint{3.070461in}{1.485039in}}%
\pgfpathlineto{\pgfqpoint{3.070461in}{1.487989in}}%
\pgfpathlineto{\pgfqpoint{3.075002in}{1.487989in}}%
\pgfpathlineto{\pgfqpoint{3.075002in}{1.485039in}}%
\pgfpathmoveto{\pgfqpoint{3.070461in}{1.487989in}}%
\pgfpathlineto{\pgfqpoint{3.070461in}{1.487989in}}%
\pgfpathlineto{\pgfqpoint{3.070461in}{1.490938in}}%
\pgfpathlineto{\pgfqpoint{3.075002in}{1.490938in}}%
\pgfpathlineto{\pgfqpoint{3.075002in}{1.487989in}}%
\pgfpathmoveto{\pgfqpoint{3.070461in}{1.490938in}}%
\pgfpathlineto{\pgfqpoint{3.070461in}{1.490938in}}%
\pgfpathlineto{\pgfqpoint{3.070461in}{1.493887in}}%
\pgfpathlineto{\pgfqpoint{3.075002in}{1.493887in}}%
\pgfpathlineto{\pgfqpoint{3.075002in}{1.490938in}}%
\pgfpathmoveto{\pgfqpoint{3.070461in}{1.493887in}}%
\pgfpathlineto{\pgfqpoint{3.070461in}{1.493887in}}%
\pgfpathlineto{\pgfqpoint{3.070461in}{1.496836in}}%
\pgfpathlineto{\pgfqpoint{3.075002in}{1.496836in}}%
\pgfpathlineto{\pgfqpoint{3.075002in}{1.493887in}}%
\pgfpathmoveto{\pgfqpoint{3.070461in}{1.496836in}}%
\pgfpathlineto{\pgfqpoint{3.070461in}{1.496836in}}%
\pgfpathlineto{\pgfqpoint{3.070461in}{1.499785in}}%
\pgfpathlineto{\pgfqpoint{3.075002in}{1.499785in}}%
\pgfpathlineto{\pgfqpoint{3.075002in}{1.496836in}}%
\pgfpathmoveto{\pgfqpoint{3.070461in}{1.499785in}}%
\pgfpathlineto{\pgfqpoint{3.070461in}{1.499785in}}%
\pgfpathlineto{\pgfqpoint{3.070461in}{1.502734in}}%
\pgfpathlineto{\pgfqpoint{3.075002in}{1.502734in}}%
\pgfpathlineto{\pgfqpoint{3.075002in}{1.499785in}}%
\pgfpathmoveto{\pgfqpoint{3.070461in}{1.502734in}}%
\pgfpathlineto{\pgfqpoint{3.070461in}{1.502734in}}%
\pgfpathlineto{\pgfqpoint{3.070461in}{1.505683in}}%
\pgfpathlineto{\pgfqpoint{3.075002in}{1.505683in}}%
\pgfpathlineto{\pgfqpoint{3.075002in}{1.502734in}}%
\pgfpathmoveto{\pgfqpoint{3.070461in}{1.505683in}}%
\pgfpathlineto{\pgfqpoint{3.070461in}{1.505683in}}%
\pgfpathlineto{\pgfqpoint{3.070461in}{1.508632in}}%
\pgfpathlineto{\pgfqpoint{3.075002in}{1.508632in}}%
\pgfpathlineto{\pgfqpoint{3.075002in}{1.505683in}}%
\pgfpathmoveto{\pgfqpoint{3.070461in}{1.508632in}}%
\pgfpathlineto{\pgfqpoint{3.070461in}{1.508632in}}%
\pgfpathlineto{\pgfqpoint{3.070461in}{1.511581in}}%
\pgfpathlineto{\pgfqpoint{3.075002in}{1.511581in}}%
\pgfpathlineto{\pgfqpoint{3.075002in}{1.508632in}}%
\pgfpathmoveto{\pgfqpoint{3.070461in}{1.511581in}}%
\pgfpathlineto{\pgfqpoint{3.070461in}{1.511581in}}%
\pgfpathlineto{\pgfqpoint{3.070461in}{1.514531in}}%
\pgfpathlineto{\pgfqpoint{3.075002in}{1.514531in}}%
\pgfpathlineto{\pgfqpoint{3.075002in}{1.511581in}}%
\pgfpathmoveto{\pgfqpoint{3.070461in}{1.514531in}}%
\pgfpathlineto{\pgfqpoint{3.070461in}{1.514531in}}%
\pgfpathlineto{\pgfqpoint{3.070461in}{1.517480in}}%
\pgfpathlineto{\pgfqpoint{3.075002in}{1.517480in}}%
\pgfpathlineto{\pgfqpoint{3.075002in}{1.514531in}}%
\pgfpathmoveto{\pgfqpoint{3.070461in}{1.517480in}}%
\pgfpathlineto{\pgfqpoint{3.070461in}{1.517480in}}%
\pgfpathlineto{\pgfqpoint{3.070461in}{1.520429in}}%
\pgfpathlineto{\pgfqpoint{3.075002in}{1.520429in}}%
\pgfpathlineto{\pgfqpoint{3.075002in}{1.517480in}}%
\pgfpathmoveto{\pgfqpoint{3.070461in}{1.520429in}}%
\pgfpathlineto{\pgfqpoint{3.070461in}{1.520429in}}%
\pgfpathlineto{\pgfqpoint{3.070461in}{1.523378in}}%
\pgfpathlineto{\pgfqpoint{3.075002in}{1.523378in}}%
\pgfpathlineto{\pgfqpoint{3.075002in}{1.520429in}}%
\pgfpathmoveto{\pgfqpoint{3.070461in}{1.523378in}}%
\pgfpathlineto{\pgfqpoint{3.070461in}{1.523378in}}%
\pgfpathlineto{\pgfqpoint{3.070461in}{1.526327in}}%
\pgfpathlineto{\pgfqpoint{3.075002in}{1.526327in}}%
\pgfpathlineto{\pgfqpoint{3.075002in}{1.523378in}}%
\pgfpathmoveto{\pgfqpoint{3.070461in}{1.526327in}}%
\pgfpathlineto{\pgfqpoint{3.070461in}{1.526327in}}%
\pgfpathlineto{\pgfqpoint{3.070461in}{1.529276in}}%
\pgfpathlineto{\pgfqpoint{3.075002in}{1.529276in}}%
\pgfpathlineto{\pgfqpoint{3.075002in}{1.526327in}}%
\pgfpathmoveto{\pgfqpoint{3.070461in}{1.529276in}}%
\pgfpathlineto{\pgfqpoint{3.070461in}{1.529276in}}%
\pgfpathlineto{\pgfqpoint{3.070461in}{1.532225in}}%
\pgfpathlineto{\pgfqpoint{3.075002in}{1.532225in}}%
\pgfpathlineto{\pgfqpoint{3.075002in}{1.529276in}}%
\pgfpathmoveto{\pgfqpoint{3.070461in}{1.532225in}}%
\pgfpathlineto{\pgfqpoint{3.070461in}{1.532225in}}%
\pgfpathlineto{\pgfqpoint{3.070461in}{1.535174in}}%
\pgfpathlineto{\pgfqpoint{3.075002in}{1.535174in}}%
\pgfpathlineto{\pgfqpoint{3.075002in}{1.532225in}}%
\pgfpathmoveto{\pgfqpoint{3.070461in}{1.535174in}}%
\pgfpathlineto{\pgfqpoint{3.070461in}{1.535174in}}%
\pgfpathlineto{\pgfqpoint{3.070461in}{1.538123in}}%
\pgfpathlineto{\pgfqpoint{3.075002in}{1.538123in}}%
\pgfpathlineto{\pgfqpoint{3.075002in}{1.535174in}}%
\pgfpathmoveto{\pgfqpoint{3.070461in}{1.538123in}}%
\pgfpathlineto{\pgfqpoint{3.070461in}{1.538123in}}%
\pgfpathlineto{\pgfqpoint{3.070461in}{1.541073in}}%
\pgfpathlineto{\pgfqpoint{3.075002in}{1.541073in}}%
\pgfpathlineto{\pgfqpoint{3.075002in}{1.538123in}}%
\pgfpathmoveto{\pgfqpoint{3.070461in}{1.541073in}}%
\pgfpathlineto{\pgfqpoint{3.070461in}{1.541073in}}%
\pgfpathlineto{\pgfqpoint{3.070461in}{1.544022in}}%
\pgfpathlineto{\pgfqpoint{3.075002in}{1.544022in}}%
\pgfpathlineto{\pgfqpoint{3.075002in}{1.541073in}}%
\pgfpathmoveto{\pgfqpoint{3.070461in}{1.544022in}}%
\pgfpathlineto{\pgfqpoint{3.070461in}{1.544022in}}%
\pgfpathlineto{\pgfqpoint{3.070461in}{1.546971in}}%
\pgfpathlineto{\pgfqpoint{3.075002in}{1.546971in}}%
\pgfpathlineto{\pgfqpoint{3.075002in}{1.544022in}}%
\pgfpathmoveto{\pgfqpoint{3.070461in}{1.546971in}}%
\pgfpathlineto{\pgfqpoint{3.070461in}{1.546971in}}%
\pgfpathlineto{\pgfqpoint{3.070461in}{1.549920in}}%
\pgfpathlineto{\pgfqpoint{3.075002in}{1.549920in}}%
\pgfpathlineto{\pgfqpoint{3.075002in}{1.546971in}}%
\pgfpathmoveto{\pgfqpoint{3.070461in}{1.549920in}}%
\pgfpathlineto{\pgfqpoint{3.070461in}{1.549920in}}%
\pgfpathlineto{\pgfqpoint{3.070461in}{1.552870in}}%
\pgfpathlineto{\pgfqpoint{3.075002in}{1.552870in}}%
\pgfpathlineto{\pgfqpoint{3.075002in}{1.549920in}}%
\pgfpathmoveto{\pgfqpoint{3.070461in}{1.552870in}}%
\pgfpathlineto{\pgfqpoint{3.070461in}{1.552870in}}%
\pgfpathlineto{\pgfqpoint{3.070461in}{1.555819in}}%
\pgfpathlineto{\pgfqpoint{3.075002in}{1.555819in}}%
\pgfpathlineto{\pgfqpoint{3.075002in}{1.552870in}}%
\pgfpathmoveto{\pgfqpoint{3.070461in}{1.555819in}}%
\pgfpathlineto{\pgfqpoint{3.070461in}{1.555819in}}%
\pgfpathlineto{\pgfqpoint{3.070461in}{1.558768in}}%
\pgfpathlineto{\pgfqpoint{3.075002in}{1.558768in}}%
\pgfpathlineto{\pgfqpoint{3.075002in}{1.555819in}}%
\pgfpathmoveto{\pgfqpoint{3.070461in}{1.558768in}}%
\pgfpathlineto{\pgfqpoint{3.070461in}{1.558768in}}%
\pgfpathlineto{\pgfqpoint{3.070461in}{1.561718in}}%
\pgfpathlineto{\pgfqpoint{3.075002in}{1.561718in}}%
\pgfpathlineto{\pgfqpoint{3.075002in}{1.558768in}}%
\pgfpathmoveto{\pgfqpoint{3.070461in}{1.561718in}}%
\pgfpathlineto{\pgfqpoint{3.070461in}{1.561718in}}%
\pgfpathlineto{\pgfqpoint{3.070461in}{1.564667in}}%
\pgfpathlineto{\pgfqpoint{3.075002in}{1.564667in}}%
\pgfpathlineto{\pgfqpoint{3.075002in}{1.561718in}}%
\pgfpathmoveto{\pgfqpoint{3.070461in}{1.564667in}}%
\pgfpathlineto{\pgfqpoint{3.070461in}{1.564667in}}%
\pgfpathlineto{\pgfqpoint{3.070461in}{1.567616in}}%
\pgfpathlineto{\pgfqpoint{3.075002in}{1.567616in}}%
\pgfpathlineto{\pgfqpoint{3.075002in}{1.564667in}}%
\pgfpathmoveto{\pgfqpoint{3.070461in}{1.567616in}}%
\pgfpathlineto{\pgfqpoint{3.070461in}{1.567616in}}%
\pgfpathlineto{\pgfqpoint{3.070461in}{1.570565in}}%
\pgfpathlineto{\pgfqpoint{3.075002in}{1.570565in}}%
\pgfpathlineto{\pgfqpoint{3.075002in}{1.567616in}}%
\pgfpathmoveto{\pgfqpoint{3.070461in}{1.570565in}}%
\pgfpathlineto{\pgfqpoint{3.070461in}{1.570565in}}%
\pgfpathlineto{\pgfqpoint{3.070461in}{1.573515in}}%
\pgfpathlineto{\pgfqpoint{3.075002in}{1.573515in}}%
\pgfpathlineto{\pgfqpoint{3.075002in}{1.570565in}}%
\pgfpathmoveto{\pgfqpoint{3.070461in}{1.573515in}}%
\pgfpathlineto{\pgfqpoint{3.070461in}{1.573515in}}%
\pgfpathlineto{\pgfqpoint{3.070461in}{1.576464in}}%
\pgfpathlineto{\pgfqpoint{3.075002in}{1.576464in}}%
\pgfpathlineto{\pgfqpoint{3.075002in}{1.573515in}}%
\pgfpathmoveto{\pgfqpoint{3.070461in}{1.576464in}}%
\pgfpathlineto{\pgfqpoint{3.070461in}{1.576464in}}%
\pgfpathlineto{\pgfqpoint{3.070461in}{1.579413in}}%
\pgfpathlineto{\pgfqpoint{3.075002in}{1.579413in}}%
\pgfpathlineto{\pgfqpoint{3.075002in}{1.576464in}}%
\pgfpathmoveto{\pgfqpoint{3.070461in}{1.579413in}}%
\pgfpathlineto{\pgfqpoint{3.070461in}{1.579413in}}%
\pgfpathlineto{\pgfqpoint{3.070461in}{1.582363in}}%
\pgfpathlineto{\pgfqpoint{3.075002in}{1.582363in}}%
\pgfpathlineto{\pgfqpoint{3.075002in}{1.579413in}}%
\pgfpathmoveto{\pgfqpoint{3.070461in}{1.582363in}}%
\pgfpathlineto{\pgfqpoint{3.070461in}{1.582363in}}%
\pgfpathlineto{\pgfqpoint{3.070461in}{1.585312in}}%
\pgfpathlineto{\pgfqpoint{3.075002in}{1.585312in}}%
\pgfpathlineto{\pgfqpoint{3.075002in}{1.582363in}}%
\pgfpathmoveto{\pgfqpoint{3.070461in}{1.585312in}}%
\pgfpathlineto{\pgfqpoint{3.070461in}{1.585312in}}%
\pgfpathlineto{\pgfqpoint{3.070461in}{1.588261in}}%
\pgfpathlineto{\pgfqpoint{3.075002in}{1.588261in}}%
\pgfpathlineto{\pgfqpoint{3.075002in}{1.585312in}}%
\pgfpathmoveto{\pgfqpoint{3.070461in}{1.588261in}}%
\pgfpathlineto{\pgfqpoint{3.070461in}{1.588261in}}%
\pgfpathlineto{\pgfqpoint{3.070461in}{1.591210in}}%
\pgfpathlineto{\pgfqpoint{3.075002in}{1.591210in}}%
\pgfpathlineto{\pgfqpoint{3.075002in}{1.588261in}}%
\pgfpathmoveto{\pgfqpoint{3.070461in}{1.591210in}}%
\pgfpathlineto{\pgfqpoint{3.070461in}{1.591210in}}%
\pgfpathlineto{\pgfqpoint{3.070461in}{1.594160in}}%
\pgfpathlineto{\pgfqpoint{3.075002in}{1.594160in}}%
\pgfpathlineto{\pgfqpoint{3.075002in}{1.591210in}}%
\pgfpathmoveto{\pgfqpoint{3.070461in}{1.594160in}}%
\pgfpathlineto{\pgfqpoint{3.070461in}{1.594160in}}%
\pgfpathlineto{\pgfqpoint{3.070461in}{1.597109in}}%
\pgfpathlineto{\pgfqpoint{3.075002in}{1.597109in}}%
\pgfpathlineto{\pgfqpoint{3.075002in}{1.594160in}}%
\pgfpathmoveto{\pgfqpoint{3.070461in}{1.597109in}}%
\pgfpathlineto{\pgfqpoint{3.070461in}{1.597109in}}%
\pgfpathlineto{\pgfqpoint{3.070461in}{1.600058in}}%
\pgfpathlineto{\pgfqpoint{3.075002in}{1.600058in}}%
\pgfpathlineto{\pgfqpoint{3.075002in}{1.597109in}}%
\pgfpathmoveto{\pgfqpoint{3.070461in}{1.600058in}}%
\pgfpathlineto{\pgfqpoint{3.070461in}{1.600058in}}%
\pgfpathlineto{\pgfqpoint{3.070461in}{1.603007in}}%
\pgfpathlineto{\pgfqpoint{3.075002in}{1.603007in}}%
\pgfpathlineto{\pgfqpoint{3.075002in}{1.600058in}}%
\pgfpathmoveto{\pgfqpoint{3.070461in}{1.603007in}}%
\pgfpathlineto{\pgfqpoint{3.070461in}{1.603007in}}%
\pgfpathlineto{\pgfqpoint{3.070461in}{1.605957in}}%
\pgfpathlineto{\pgfqpoint{3.075002in}{1.605957in}}%
\pgfpathlineto{\pgfqpoint{3.075002in}{1.603007in}}%
\pgfpathmoveto{\pgfqpoint{3.070461in}{1.605957in}}%
\pgfpathlineto{\pgfqpoint{3.070461in}{1.605957in}}%
\pgfpathlineto{\pgfqpoint{3.070461in}{1.608906in}}%
\pgfpathlineto{\pgfqpoint{3.075002in}{1.608906in}}%
\pgfpathlineto{\pgfqpoint{3.075002in}{1.605957in}}%
\pgfpathmoveto{\pgfqpoint{3.070461in}{1.608906in}}%
\pgfpathlineto{\pgfqpoint{3.070461in}{1.608906in}}%
\pgfpathlineto{\pgfqpoint{3.070461in}{1.611855in}}%
\pgfpathlineto{\pgfqpoint{3.075002in}{1.611855in}}%
\pgfpathlineto{\pgfqpoint{3.075002in}{1.608906in}}%
\pgfpathmoveto{\pgfqpoint{3.070461in}{1.611855in}}%
\pgfpathlineto{\pgfqpoint{3.070461in}{1.611855in}}%
\pgfpathlineto{\pgfqpoint{3.070461in}{1.614805in}}%
\pgfpathlineto{\pgfqpoint{3.075002in}{1.614805in}}%
\pgfpathlineto{\pgfqpoint{3.075002in}{1.611855in}}%
\pgfpathmoveto{\pgfqpoint{3.070461in}{1.614805in}}%
\pgfpathlineto{\pgfqpoint{3.070461in}{1.614805in}}%
\pgfpathlineto{\pgfqpoint{3.070461in}{1.617754in}}%
\pgfpathlineto{\pgfqpoint{3.075002in}{1.617754in}}%
\pgfpathlineto{\pgfqpoint{3.075002in}{1.614805in}}%
\pgfpathmoveto{\pgfqpoint{3.070461in}{1.617754in}}%
\pgfpathlineto{\pgfqpoint{3.070461in}{1.617754in}}%
\pgfpathlineto{\pgfqpoint{3.070461in}{1.620703in}}%
\pgfpathlineto{\pgfqpoint{3.075002in}{1.620703in}}%
\pgfpathlineto{\pgfqpoint{3.075002in}{1.617754in}}%
\pgfpathmoveto{\pgfqpoint{3.070461in}{1.620703in}}%
\pgfpathlineto{\pgfqpoint{3.070461in}{1.620703in}}%
\pgfpathlineto{\pgfqpoint{3.070461in}{1.623652in}}%
\pgfpathlineto{\pgfqpoint{3.075002in}{1.623652in}}%
\pgfpathlineto{\pgfqpoint{3.075002in}{1.620703in}}%
\pgfpathmoveto{\pgfqpoint{3.070461in}{1.623652in}}%
\pgfpathlineto{\pgfqpoint{3.070461in}{1.623652in}}%
\pgfpathlineto{\pgfqpoint{3.070461in}{1.626602in}}%
\pgfpathlineto{\pgfqpoint{3.075002in}{1.626602in}}%
\pgfpathlineto{\pgfqpoint{3.075002in}{1.623652in}}%
\pgfpathmoveto{\pgfqpoint{3.070461in}{1.626602in}}%
\pgfpathlineto{\pgfqpoint{3.070461in}{1.626602in}}%
\pgfpathlineto{\pgfqpoint{3.070461in}{1.629551in}}%
\pgfpathlineto{\pgfqpoint{3.075002in}{1.629551in}}%
\pgfpathlineto{\pgfqpoint{3.075002in}{1.626602in}}%
\pgfpathmoveto{\pgfqpoint{3.070461in}{1.629551in}}%
\pgfpathlineto{\pgfqpoint{3.070461in}{1.629551in}}%
\pgfpathlineto{\pgfqpoint{3.070461in}{1.632500in}}%
\pgfpathlineto{\pgfqpoint{3.075002in}{1.632500in}}%
\pgfpathlineto{\pgfqpoint{3.075002in}{1.629551in}}%
\pgfpathmoveto{\pgfqpoint{3.070461in}{1.632500in}}%
\pgfpathlineto{\pgfqpoint{3.070461in}{1.632500in}}%
\pgfpathlineto{\pgfqpoint{3.070461in}{1.635449in}}%
\pgfpathlineto{\pgfqpoint{3.075002in}{1.635449in}}%
\pgfpathlineto{\pgfqpoint{3.075002in}{1.632500in}}%
\pgfpathmoveto{\pgfqpoint{3.070461in}{1.635449in}}%
\pgfpathlineto{\pgfqpoint{3.070461in}{1.635449in}}%
\pgfpathlineto{\pgfqpoint{3.070461in}{1.638399in}}%
\pgfpathlineto{\pgfqpoint{3.075002in}{1.638399in}}%
\pgfpathlineto{\pgfqpoint{3.075002in}{1.635449in}}%
\pgfpathmoveto{\pgfqpoint{3.070461in}{1.638399in}}%
\pgfpathlineto{\pgfqpoint{3.070461in}{1.638399in}}%
\pgfpathlineto{\pgfqpoint{3.070461in}{1.641348in}}%
\pgfpathlineto{\pgfqpoint{3.075002in}{1.641348in}}%
\pgfpathlineto{\pgfqpoint{3.075002in}{1.638399in}}%
\pgfpathmoveto{\pgfqpoint{3.070461in}{1.641348in}}%
\pgfpathlineto{\pgfqpoint{3.070461in}{1.641348in}}%
\pgfpathlineto{\pgfqpoint{3.070461in}{1.644297in}}%
\pgfpathlineto{\pgfqpoint{3.075002in}{1.644297in}}%
\pgfpathlineto{\pgfqpoint{3.075002in}{1.641348in}}%
\pgfpathmoveto{\pgfqpoint{3.070461in}{1.644297in}}%
\pgfpathlineto{\pgfqpoint{3.070461in}{1.644297in}}%
\pgfpathlineto{\pgfqpoint{3.070461in}{1.647246in}}%
\pgfpathlineto{\pgfqpoint{3.075002in}{1.647246in}}%
\pgfpathlineto{\pgfqpoint{3.075002in}{1.644297in}}%
\pgfpathmoveto{\pgfqpoint{3.070461in}{1.647246in}}%
\pgfpathlineto{\pgfqpoint{3.070461in}{1.647246in}}%
\pgfpathlineto{\pgfqpoint{3.070461in}{1.650195in}}%
\pgfpathlineto{\pgfqpoint{3.075002in}{1.650195in}}%
\pgfpathlineto{\pgfqpoint{3.075002in}{1.647246in}}%
\pgfpathmoveto{\pgfqpoint{3.070461in}{1.650195in}}%
\pgfpathlineto{\pgfqpoint{3.070461in}{1.650195in}}%
\pgfpathlineto{\pgfqpoint{3.070461in}{1.653144in}}%
\pgfpathlineto{\pgfqpoint{3.075002in}{1.653144in}}%
\pgfpathlineto{\pgfqpoint{3.075002in}{1.650195in}}%
\pgfpathmoveto{\pgfqpoint{3.070461in}{1.653144in}}%
\pgfpathlineto{\pgfqpoint{3.070461in}{1.653144in}}%
\pgfpathlineto{\pgfqpoint{3.070461in}{1.656094in}}%
\pgfpathlineto{\pgfqpoint{3.075002in}{1.656094in}}%
\pgfpathlineto{\pgfqpoint{3.075002in}{1.653144in}}%
\pgfpathmoveto{\pgfqpoint{3.070461in}{1.656094in}}%
\pgfpathlineto{\pgfqpoint{3.070461in}{1.656094in}}%
\pgfpathlineto{\pgfqpoint{3.070461in}{1.659043in}}%
\pgfpathlineto{\pgfqpoint{3.075002in}{1.659043in}}%
\pgfpathlineto{\pgfqpoint{3.075002in}{1.656094in}}%
\pgfpathmoveto{\pgfqpoint{3.070461in}{1.659043in}}%
\pgfpathlineto{\pgfqpoint{3.070461in}{1.659043in}}%
\pgfpathlineto{\pgfqpoint{3.070461in}{1.661992in}}%
\pgfpathlineto{\pgfqpoint{3.075002in}{1.661992in}}%
\pgfpathlineto{\pgfqpoint{3.075002in}{1.659043in}}%
\pgfpathmoveto{\pgfqpoint{3.070461in}{1.661992in}}%
\pgfpathlineto{\pgfqpoint{3.070461in}{1.661992in}}%
\pgfpathlineto{\pgfqpoint{3.070461in}{1.664941in}}%
\pgfpathlineto{\pgfqpoint{3.075002in}{1.664941in}}%
\pgfpathlineto{\pgfqpoint{3.075002in}{1.661992in}}%
\pgfpathmoveto{\pgfqpoint{3.070461in}{1.664941in}}%
\pgfpathlineto{\pgfqpoint{3.070461in}{1.664941in}}%
\pgfpathlineto{\pgfqpoint{3.070461in}{1.667890in}}%
\pgfpathlineto{\pgfqpoint{3.075002in}{1.667890in}}%
\pgfpathlineto{\pgfqpoint{3.075002in}{1.664941in}}%
\pgfpathmoveto{\pgfqpoint{3.070461in}{1.667890in}}%
\pgfpathlineto{\pgfqpoint{3.070461in}{1.667890in}}%
\pgfpathlineto{\pgfqpoint{3.070461in}{1.670839in}}%
\pgfpathlineto{\pgfqpoint{3.075002in}{1.670839in}}%
\pgfpathlineto{\pgfqpoint{3.075002in}{1.667890in}}%
\pgfpathmoveto{\pgfqpoint{3.070461in}{1.670839in}}%
\pgfpathlineto{\pgfqpoint{3.070461in}{1.670839in}}%
\pgfpathlineto{\pgfqpoint{3.070461in}{1.673789in}}%
\pgfpathlineto{\pgfqpoint{3.075002in}{1.673789in}}%
\pgfpathlineto{\pgfqpoint{3.075002in}{1.670839in}}%
\pgfpathmoveto{\pgfqpoint{3.070461in}{1.673789in}}%
\pgfpathlineto{\pgfqpoint{3.070461in}{1.673789in}}%
\pgfpathlineto{\pgfqpoint{3.070461in}{1.676738in}}%
\pgfpathlineto{\pgfqpoint{3.075002in}{1.676738in}}%
\pgfpathlineto{\pgfqpoint{3.075002in}{1.673789in}}%
\pgfpathmoveto{\pgfqpoint{3.070461in}{1.676738in}}%
\pgfpathlineto{\pgfqpoint{3.070461in}{1.676738in}}%
\pgfpathlineto{\pgfqpoint{3.070461in}{1.679687in}}%
\pgfpathlineto{\pgfqpoint{3.075002in}{1.679687in}}%
\pgfpathlineto{\pgfqpoint{3.075002in}{1.676738in}}%
\pgfpathmoveto{\pgfqpoint{3.070461in}{1.679687in}}%
\pgfpathlineto{\pgfqpoint{3.070461in}{1.679687in}}%
\pgfpathlineto{\pgfqpoint{3.070461in}{1.682636in}}%
\pgfpathlineto{\pgfqpoint{3.075002in}{1.682636in}}%
\pgfpathlineto{\pgfqpoint{3.075002in}{1.679687in}}%
\pgfpathmoveto{\pgfqpoint{3.070461in}{1.682636in}}%
\pgfpathlineto{\pgfqpoint{3.070461in}{1.682636in}}%
\pgfpathlineto{\pgfqpoint{3.070461in}{1.685585in}}%
\pgfpathlineto{\pgfqpoint{3.075002in}{1.685585in}}%
\pgfpathlineto{\pgfqpoint{3.075002in}{1.682636in}}%
\pgfpathmoveto{\pgfqpoint{3.070461in}{1.685585in}}%
\pgfpathlineto{\pgfqpoint{3.070461in}{1.685585in}}%
\pgfpathlineto{\pgfqpoint{3.070461in}{1.688534in}}%
\pgfpathlineto{\pgfqpoint{3.075002in}{1.688534in}}%
\pgfpathlineto{\pgfqpoint{3.075002in}{1.685585in}}%
\pgfpathmoveto{\pgfqpoint{3.070461in}{1.688534in}}%
\pgfpathlineto{\pgfqpoint{3.070461in}{1.688534in}}%
\pgfpathlineto{\pgfqpoint{3.070461in}{1.691484in}}%
\pgfpathlineto{\pgfqpoint{3.075002in}{1.691484in}}%
\pgfpathlineto{\pgfqpoint{3.075002in}{1.688534in}}%
\pgfpathmoveto{\pgfqpoint{3.070461in}{1.691484in}}%
\pgfpathlineto{\pgfqpoint{3.070461in}{1.691484in}}%
\pgfpathlineto{\pgfqpoint{3.070461in}{1.694433in}}%
\pgfpathlineto{\pgfqpoint{3.075002in}{1.694433in}}%
\pgfpathlineto{\pgfqpoint{3.075002in}{1.691484in}}%
\pgfpathmoveto{\pgfqpoint{3.070461in}{1.694433in}}%
\pgfpathlineto{\pgfqpoint{3.070461in}{1.694433in}}%
\pgfpathlineto{\pgfqpoint{3.070461in}{1.697382in}}%
\pgfpathlineto{\pgfqpoint{3.075002in}{1.697382in}}%
\pgfpathlineto{\pgfqpoint{3.075002in}{1.694433in}}%
\pgfpathmoveto{\pgfqpoint{3.070461in}{1.697382in}}%
\pgfpathlineto{\pgfqpoint{3.070461in}{1.697382in}}%
\pgfpathlineto{\pgfqpoint{3.070461in}{1.700331in}}%
\pgfpathlineto{\pgfqpoint{3.075002in}{1.700331in}}%
\pgfpathlineto{\pgfqpoint{3.075002in}{1.697382in}}%
\pgfpathmoveto{\pgfqpoint{3.070461in}{1.700331in}}%
\pgfpathlineto{\pgfqpoint{3.070461in}{1.700331in}}%
\pgfpathlineto{\pgfqpoint{3.070461in}{1.703280in}}%
\pgfpathlineto{\pgfqpoint{3.075002in}{1.703280in}}%
\pgfpathlineto{\pgfqpoint{3.075002in}{1.700331in}}%
\pgfpathmoveto{\pgfqpoint{3.070461in}{1.703280in}}%
\pgfpathlineto{\pgfqpoint{3.070461in}{1.703280in}}%
\pgfpathlineto{\pgfqpoint{3.070461in}{1.706229in}}%
\pgfpathlineto{\pgfqpoint{3.075002in}{1.706229in}}%
\pgfpathlineto{\pgfqpoint{3.075002in}{1.703280in}}%
\pgfpathmoveto{\pgfqpoint{3.070461in}{1.706229in}}%
\pgfpathlineto{\pgfqpoint{3.070461in}{1.706229in}}%
\pgfpathlineto{\pgfqpoint{3.070461in}{1.709179in}}%
\pgfpathlineto{\pgfqpoint{3.075002in}{1.709179in}}%
\pgfpathlineto{\pgfqpoint{3.075002in}{1.706229in}}%
\pgfpathmoveto{\pgfqpoint{3.070461in}{1.709179in}}%
\pgfpathlineto{\pgfqpoint{3.070461in}{1.709179in}}%
\pgfpathlineto{\pgfqpoint{3.070461in}{1.712128in}}%
\pgfpathlineto{\pgfqpoint{3.075002in}{1.712128in}}%
\pgfpathlineto{\pgfqpoint{3.075002in}{1.709179in}}%
\pgfpathmoveto{\pgfqpoint{3.070461in}{1.712128in}}%
\pgfpathlineto{\pgfqpoint{3.070461in}{1.712128in}}%
\pgfpathlineto{\pgfqpoint{3.070461in}{1.715077in}}%
\pgfpathlineto{\pgfqpoint{3.075002in}{1.715077in}}%
\pgfpathlineto{\pgfqpoint{3.075002in}{1.712128in}}%
\pgfpathmoveto{\pgfqpoint{3.070461in}{1.715077in}}%
\pgfpathlineto{\pgfqpoint{3.070461in}{1.715077in}}%
\pgfpathlineto{\pgfqpoint{3.070461in}{1.718026in}}%
\pgfpathlineto{\pgfqpoint{3.075002in}{1.718026in}}%
\pgfpathlineto{\pgfqpoint{3.075002in}{1.715077in}}%
\pgfpathmoveto{\pgfqpoint{3.070461in}{1.718026in}}%
\pgfpathlineto{\pgfqpoint{3.070461in}{1.718026in}}%
\pgfpathlineto{\pgfqpoint{3.070461in}{1.720975in}}%
\pgfpathlineto{\pgfqpoint{3.075002in}{1.720975in}}%
\pgfpathlineto{\pgfqpoint{3.075002in}{1.718026in}}%
\pgfpathmoveto{\pgfqpoint{3.070461in}{1.720975in}}%
\pgfpathlineto{\pgfqpoint{3.070461in}{1.720975in}}%
\pgfpathlineto{\pgfqpoint{3.070461in}{1.723924in}}%
\pgfpathlineto{\pgfqpoint{3.075002in}{1.723924in}}%
\pgfpathlineto{\pgfqpoint{3.075002in}{1.720975in}}%
\pgfpathmoveto{\pgfqpoint{3.070461in}{1.723924in}}%
\pgfpathlineto{\pgfqpoint{3.070461in}{1.723924in}}%
\pgfpathlineto{\pgfqpoint{3.070461in}{1.726874in}}%
\pgfpathlineto{\pgfqpoint{3.075002in}{1.726874in}}%
\pgfpathlineto{\pgfqpoint{3.075002in}{1.723924in}}%
\pgfpathmoveto{\pgfqpoint{3.070461in}{1.726874in}}%
\pgfpathlineto{\pgfqpoint{3.070461in}{1.726874in}}%
\pgfpathlineto{\pgfqpoint{3.070461in}{1.729823in}}%
\pgfpathlineto{\pgfqpoint{3.075002in}{1.729823in}}%
\pgfpathlineto{\pgfqpoint{3.075002in}{1.726874in}}%
\pgfpathmoveto{\pgfqpoint{3.070461in}{1.729823in}}%
\pgfpathlineto{\pgfqpoint{3.070461in}{1.729823in}}%
\pgfpathlineto{\pgfqpoint{3.070461in}{1.732772in}}%
\pgfpathlineto{\pgfqpoint{3.075002in}{1.732772in}}%
\pgfpathlineto{\pgfqpoint{3.075002in}{1.729823in}}%
\pgfpathmoveto{\pgfqpoint{3.070461in}{1.732772in}}%
\pgfpathlineto{\pgfqpoint{3.070461in}{1.732772in}}%
\pgfpathlineto{\pgfqpoint{3.070461in}{1.735721in}}%
\pgfpathlineto{\pgfqpoint{3.075002in}{1.735721in}}%
\pgfpathlineto{\pgfqpoint{3.075002in}{1.732772in}}%
\pgfpathmoveto{\pgfqpoint{3.070461in}{1.735721in}}%
\pgfpathlineto{\pgfqpoint{3.070461in}{1.735721in}}%
\pgfpathlineto{\pgfqpoint{3.070461in}{1.738671in}}%
\pgfpathlineto{\pgfqpoint{3.075002in}{1.738671in}}%
\pgfpathlineto{\pgfqpoint{3.075002in}{1.735721in}}%
\pgfpathmoveto{\pgfqpoint{3.070461in}{1.738671in}}%
\pgfpathlineto{\pgfqpoint{3.070461in}{1.738671in}}%
\pgfpathlineto{\pgfqpoint{3.070461in}{1.741620in}}%
\pgfpathlineto{\pgfqpoint{3.075002in}{1.741620in}}%
\pgfpathlineto{\pgfqpoint{3.075002in}{1.738671in}}%
\pgfpathmoveto{\pgfqpoint{3.070461in}{1.741620in}}%
\pgfpathlineto{\pgfqpoint{3.070461in}{1.741620in}}%
\pgfpathlineto{\pgfqpoint{3.070461in}{1.744569in}}%
\pgfpathlineto{\pgfqpoint{3.075002in}{1.744569in}}%
\pgfpathlineto{\pgfqpoint{3.075002in}{1.741620in}}%
\pgfpathmoveto{\pgfqpoint{3.070461in}{1.744569in}}%
\pgfpathlineto{\pgfqpoint{3.070461in}{1.744569in}}%
\pgfpathlineto{\pgfqpoint{3.070461in}{1.747518in}}%
\pgfpathlineto{\pgfqpoint{3.075002in}{1.747518in}}%
\pgfpathlineto{\pgfqpoint{3.075002in}{1.744569in}}%
\pgfpathmoveto{\pgfqpoint{3.070461in}{1.747518in}}%
\pgfpathlineto{\pgfqpoint{3.070461in}{1.747518in}}%
\pgfpathlineto{\pgfqpoint{3.070461in}{1.750468in}}%
\pgfpathlineto{\pgfqpoint{3.075002in}{1.750468in}}%
\pgfpathlineto{\pgfqpoint{3.075002in}{1.747518in}}%
\pgfpathmoveto{\pgfqpoint{3.070461in}{1.750468in}}%
\pgfpathlineto{\pgfqpoint{3.070461in}{1.750468in}}%
\pgfpathlineto{\pgfqpoint{3.070461in}{1.753417in}}%
\pgfpathlineto{\pgfqpoint{3.075002in}{1.753417in}}%
\pgfpathlineto{\pgfqpoint{3.075002in}{1.750468in}}%
\pgfpathmoveto{\pgfqpoint{3.070461in}{1.753417in}}%
\pgfpathlineto{\pgfqpoint{3.070461in}{1.753417in}}%
\pgfpathlineto{\pgfqpoint{3.070461in}{1.756366in}}%
\pgfpathlineto{\pgfqpoint{3.075002in}{1.756366in}}%
\pgfpathlineto{\pgfqpoint{3.075002in}{1.753417in}}%
\pgfpathmoveto{\pgfqpoint{3.070461in}{1.756366in}}%
\pgfpathlineto{\pgfqpoint{3.070461in}{1.756366in}}%
\pgfpathlineto{\pgfqpoint{3.070461in}{1.759315in}}%
\pgfpathlineto{\pgfqpoint{3.075002in}{1.759315in}}%
\pgfpathlineto{\pgfqpoint{3.075002in}{1.756366in}}%
\pgfpathmoveto{\pgfqpoint{3.070461in}{1.759315in}}%
\pgfpathlineto{\pgfqpoint{3.070461in}{1.759315in}}%
\pgfpathlineto{\pgfqpoint{3.070461in}{1.762265in}}%
\pgfpathlineto{\pgfqpoint{3.075002in}{1.762265in}}%
\pgfpathlineto{\pgfqpoint{3.075002in}{1.759315in}}%
\pgfpathmoveto{\pgfqpoint{3.070461in}{1.762265in}}%
\pgfpathlineto{\pgfqpoint{3.070461in}{1.762265in}}%
\pgfpathlineto{\pgfqpoint{3.070461in}{1.765214in}}%
\pgfpathlineto{\pgfqpoint{3.075002in}{1.765214in}}%
\pgfpathlineto{\pgfqpoint{3.075002in}{1.762265in}}%
\pgfpathmoveto{\pgfqpoint{3.070461in}{1.765214in}}%
\pgfpathlineto{\pgfqpoint{3.070461in}{1.765214in}}%
\pgfpathlineto{\pgfqpoint{3.070461in}{1.768163in}}%
\pgfpathlineto{\pgfqpoint{3.075002in}{1.768163in}}%
\pgfpathlineto{\pgfqpoint{3.075002in}{1.765214in}}%
\pgfpathmoveto{\pgfqpoint{3.070461in}{1.768163in}}%
\pgfpathlineto{\pgfqpoint{3.070461in}{1.768163in}}%
\pgfpathlineto{\pgfqpoint{3.070461in}{1.771112in}}%
\pgfpathlineto{\pgfqpoint{3.075002in}{1.771112in}}%
\pgfpathlineto{\pgfqpoint{3.075002in}{1.768163in}}%
\pgfpathmoveto{\pgfqpoint{3.070461in}{1.771112in}}%
\pgfpathlineto{\pgfqpoint{3.070461in}{1.771112in}}%
\pgfpathlineto{\pgfqpoint{3.070461in}{1.774061in}}%
\pgfpathlineto{\pgfqpoint{3.075002in}{1.774061in}}%
\pgfpathlineto{\pgfqpoint{3.075002in}{1.771112in}}%
\pgfpathmoveto{\pgfqpoint{3.070461in}{1.774061in}}%
\pgfpathlineto{\pgfqpoint{3.070461in}{1.774061in}}%
\pgfpathlineto{\pgfqpoint{3.070461in}{1.777011in}}%
\pgfpathlineto{\pgfqpoint{3.075002in}{1.777011in}}%
\pgfpathlineto{\pgfqpoint{3.075002in}{1.774061in}}%
\pgfpathmoveto{\pgfqpoint{3.070461in}{1.777011in}}%
\pgfpathlineto{\pgfqpoint{3.070461in}{1.777011in}}%
\pgfpathlineto{\pgfqpoint{3.070461in}{1.779960in}}%
\pgfpathlineto{\pgfqpoint{3.075002in}{1.779960in}}%
\pgfpathlineto{\pgfqpoint{3.075002in}{1.777011in}}%
\pgfpathmoveto{\pgfqpoint{3.070461in}{1.779960in}}%
\pgfpathlineto{\pgfqpoint{3.070461in}{1.779960in}}%
\pgfpathlineto{\pgfqpoint{3.070461in}{1.782909in}}%
\pgfpathlineto{\pgfqpoint{3.075002in}{1.782909in}}%
\pgfpathlineto{\pgfqpoint{3.075002in}{1.779960in}}%
\pgfpathmoveto{\pgfqpoint{3.070461in}{1.782909in}}%
\pgfpathlineto{\pgfqpoint{3.070461in}{1.782909in}}%
\pgfpathlineto{\pgfqpoint{3.070461in}{1.785858in}}%
\pgfpathlineto{\pgfqpoint{3.075002in}{1.785858in}}%
\pgfpathlineto{\pgfqpoint{3.075002in}{1.782909in}}%
\pgfpathmoveto{\pgfqpoint{3.070461in}{1.785858in}}%
\pgfpathlineto{\pgfqpoint{3.070461in}{1.785858in}}%
\pgfpathlineto{\pgfqpoint{3.070461in}{1.788808in}}%
\pgfpathlineto{\pgfqpoint{3.075002in}{1.788808in}}%
\pgfpathlineto{\pgfqpoint{3.075002in}{1.785858in}}%
\pgfpathmoveto{\pgfqpoint{3.070461in}{1.788808in}}%
\pgfpathlineto{\pgfqpoint{3.070461in}{1.788808in}}%
\pgfpathlineto{\pgfqpoint{3.070461in}{1.791757in}}%
\pgfpathlineto{\pgfqpoint{3.075002in}{1.791757in}}%
\pgfpathlineto{\pgfqpoint{3.075002in}{1.788808in}}%
\pgfpathmoveto{\pgfqpoint{3.070461in}{1.791757in}}%
\pgfpathlineto{\pgfqpoint{3.070461in}{1.791757in}}%
\pgfpathlineto{\pgfqpoint{3.070461in}{1.794706in}}%
\pgfpathlineto{\pgfqpoint{3.075002in}{1.794706in}}%
\pgfpathlineto{\pgfqpoint{3.075002in}{1.791757in}}%
\pgfpathmoveto{\pgfqpoint{3.070461in}{1.794706in}}%
\pgfpathlineto{\pgfqpoint{3.070461in}{1.794706in}}%
\pgfpathlineto{\pgfqpoint{3.070461in}{1.797655in}}%
\pgfpathlineto{\pgfqpoint{3.075002in}{1.797655in}}%
\pgfpathlineto{\pgfqpoint{3.075002in}{1.794706in}}%
\pgfpathmoveto{\pgfqpoint{3.070461in}{1.797655in}}%
\pgfpathlineto{\pgfqpoint{3.070461in}{1.797655in}}%
\pgfpathlineto{\pgfqpoint{3.070461in}{1.800605in}}%
\pgfpathlineto{\pgfqpoint{3.075002in}{1.800605in}}%
\pgfpathlineto{\pgfqpoint{3.075002in}{1.797655in}}%
\pgfpathmoveto{\pgfqpoint{3.070461in}{1.800605in}}%
\pgfpathlineto{\pgfqpoint{3.070461in}{1.800605in}}%
\pgfpathlineto{\pgfqpoint{3.070461in}{1.803554in}}%
\pgfpathlineto{\pgfqpoint{3.075002in}{1.803554in}}%
\pgfpathlineto{\pgfqpoint{3.075002in}{1.800605in}}%
\pgfpathmoveto{\pgfqpoint{3.070461in}{1.803554in}}%
\pgfpathlineto{\pgfqpoint{3.070461in}{1.803554in}}%
\pgfpathlineto{\pgfqpoint{3.070461in}{1.806503in}}%
\pgfpathlineto{\pgfqpoint{3.075002in}{1.806503in}}%
\pgfpathlineto{\pgfqpoint{3.075002in}{1.803554in}}%
\pgfpathmoveto{\pgfqpoint{3.070461in}{1.806503in}}%
\pgfpathlineto{\pgfqpoint{3.070461in}{1.806503in}}%
\pgfpathlineto{\pgfqpoint{3.070461in}{1.809452in}}%
\pgfpathlineto{\pgfqpoint{3.075002in}{1.809452in}}%
\pgfpathlineto{\pgfqpoint{3.075002in}{1.806503in}}%
\pgfpathmoveto{\pgfqpoint{3.070461in}{1.809452in}}%
\pgfpathlineto{\pgfqpoint{3.070461in}{1.809452in}}%
\pgfpathlineto{\pgfqpoint{3.070461in}{1.812402in}}%
\pgfpathlineto{\pgfqpoint{3.075002in}{1.812402in}}%
\pgfpathlineto{\pgfqpoint{3.075002in}{1.809452in}}%
\pgfpathmoveto{\pgfqpoint{3.070461in}{1.812402in}}%
\pgfpathlineto{\pgfqpoint{3.070461in}{1.812402in}}%
\pgfpathlineto{\pgfqpoint{3.070461in}{1.815351in}}%
\pgfpathlineto{\pgfqpoint{3.075002in}{1.815351in}}%
\pgfpathlineto{\pgfqpoint{3.075002in}{1.812402in}}%
\pgfpathmoveto{\pgfqpoint{3.070461in}{1.815351in}}%
\pgfpathlineto{\pgfqpoint{3.070461in}{1.815351in}}%
\pgfpathlineto{\pgfqpoint{3.070461in}{1.818300in}}%
\pgfpathlineto{\pgfqpoint{3.075002in}{1.818300in}}%
\pgfpathlineto{\pgfqpoint{3.075002in}{1.815351in}}%
\pgfpathmoveto{\pgfqpoint{3.070461in}{1.818300in}}%
\pgfpathlineto{\pgfqpoint{3.070461in}{1.818300in}}%
\pgfpathlineto{\pgfqpoint{3.070461in}{1.821249in}}%
\pgfpathlineto{\pgfqpoint{3.075002in}{1.821249in}}%
\pgfpathlineto{\pgfqpoint{3.075002in}{1.818300in}}%
\pgfpathmoveto{\pgfqpoint{3.070461in}{1.821249in}}%
\pgfpathlineto{\pgfqpoint{3.070461in}{1.821249in}}%
\pgfpathlineto{\pgfqpoint{3.070461in}{1.824199in}}%
\pgfpathlineto{\pgfqpoint{3.075002in}{1.824199in}}%
\pgfpathlineto{\pgfqpoint{3.075002in}{1.821249in}}%
\pgfpathmoveto{\pgfqpoint{3.070461in}{1.824199in}}%
\pgfpathlineto{\pgfqpoint{3.070461in}{1.824199in}}%
\pgfpathlineto{\pgfqpoint{3.070461in}{1.827148in}}%
\pgfpathlineto{\pgfqpoint{3.075002in}{1.827148in}}%
\pgfpathlineto{\pgfqpoint{3.075002in}{1.824199in}}%
\pgfpathmoveto{\pgfqpoint{3.070461in}{1.827148in}}%
\pgfpathlineto{\pgfqpoint{3.070461in}{1.827148in}}%
\pgfpathlineto{\pgfqpoint{3.070461in}{1.830097in}}%
\pgfpathlineto{\pgfqpoint{3.075002in}{1.830097in}}%
\pgfpathlineto{\pgfqpoint{3.075002in}{1.827148in}}%
\pgfpathmoveto{\pgfqpoint{3.070461in}{1.830097in}}%
\pgfpathlineto{\pgfqpoint{3.070461in}{1.830097in}}%
\pgfpathlineto{\pgfqpoint{3.070461in}{1.833046in}}%
\pgfpathlineto{\pgfqpoint{3.075002in}{1.833046in}}%
\pgfpathlineto{\pgfqpoint{3.075002in}{1.830097in}}%
\pgfpathmoveto{\pgfqpoint{3.070461in}{1.833046in}}%
\pgfpathlineto{\pgfqpoint{3.070461in}{1.833046in}}%
\pgfpathlineto{\pgfqpoint{3.070461in}{1.835996in}}%
\pgfpathlineto{\pgfqpoint{3.075002in}{1.835996in}}%
\pgfpathlineto{\pgfqpoint{3.075002in}{1.833046in}}%
\pgfpathmoveto{\pgfqpoint{3.070461in}{1.835996in}}%
\pgfpathlineto{\pgfqpoint{3.070461in}{1.835996in}}%
\pgfpathlineto{\pgfqpoint{3.070461in}{1.838945in}}%
\pgfpathlineto{\pgfqpoint{3.075002in}{1.838945in}}%
\pgfpathlineto{\pgfqpoint{3.075002in}{1.835996in}}%
\pgfpathmoveto{\pgfqpoint{3.070461in}{1.838945in}}%
\pgfpathlineto{\pgfqpoint{3.070461in}{1.838945in}}%
\pgfpathlineto{\pgfqpoint{3.070461in}{1.841894in}}%
\pgfpathlineto{\pgfqpoint{3.075002in}{1.841894in}}%
\pgfpathlineto{\pgfqpoint{3.075002in}{1.838945in}}%
\pgfpathmoveto{\pgfqpoint{3.070461in}{1.841894in}}%
\pgfpathlineto{\pgfqpoint{3.070461in}{1.841894in}}%
\pgfpathlineto{\pgfqpoint{3.070461in}{1.844843in}}%
\pgfpathlineto{\pgfqpoint{3.075002in}{1.844843in}}%
\pgfpathlineto{\pgfqpoint{3.075002in}{1.841894in}}%
\pgfpathmoveto{\pgfqpoint{3.070461in}{1.844843in}}%
\pgfpathlineto{\pgfqpoint{3.070461in}{1.844843in}}%
\pgfpathlineto{\pgfqpoint{3.070461in}{1.847793in}}%
\pgfpathlineto{\pgfqpoint{3.075002in}{1.847793in}}%
\pgfpathlineto{\pgfqpoint{3.075002in}{1.844843in}}%
\pgfpathmoveto{\pgfqpoint{3.070461in}{1.847793in}}%
\pgfpathlineto{\pgfqpoint{3.070461in}{1.847793in}}%
\pgfpathlineto{\pgfqpoint{3.070461in}{1.850742in}}%
\pgfpathlineto{\pgfqpoint{3.075002in}{1.850742in}}%
\pgfpathlineto{\pgfqpoint{3.075002in}{1.847793in}}%
\pgfpathmoveto{\pgfqpoint{3.070461in}{1.850742in}}%
\pgfpathlineto{\pgfqpoint{3.070461in}{1.850742in}}%
\pgfpathlineto{\pgfqpoint{3.070461in}{1.853691in}}%
\pgfpathlineto{\pgfqpoint{3.075002in}{1.853691in}}%
\pgfpathlineto{\pgfqpoint{3.075002in}{1.850742in}}%
\pgfpathmoveto{\pgfqpoint{3.070461in}{1.853691in}}%
\pgfpathlineto{\pgfqpoint{3.070461in}{1.853691in}}%
\pgfpathlineto{\pgfqpoint{3.070461in}{1.856641in}}%
\pgfpathlineto{\pgfqpoint{3.075002in}{1.856641in}}%
\pgfpathlineto{\pgfqpoint{3.075002in}{1.853691in}}%
\pgfpathmoveto{\pgfqpoint{3.070461in}{1.856641in}}%
\pgfpathlineto{\pgfqpoint{3.070461in}{1.856641in}}%
\pgfpathlineto{\pgfqpoint{3.070461in}{1.859590in}}%
\pgfpathlineto{\pgfqpoint{3.075002in}{1.859590in}}%
\pgfpathlineto{\pgfqpoint{3.075002in}{1.856641in}}%
\pgfpathmoveto{\pgfqpoint{3.070461in}{1.859590in}}%
\pgfpathlineto{\pgfqpoint{3.070461in}{1.859590in}}%
\pgfpathlineto{\pgfqpoint{3.070461in}{1.862539in}}%
\pgfpathlineto{\pgfqpoint{3.075002in}{1.862539in}}%
\pgfpathlineto{\pgfqpoint{3.075002in}{1.859590in}}%
\pgfpathmoveto{\pgfqpoint{3.070461in}{1.862539in}}%
\pgfpathlineto{\pgfqpoint{3.070461in}{1.862539in}}%
\pgfpathlineto{\pgfqpoint{3.070461in}{1.865488in}}%
\pgfpathlineto{\pgfqpoint{3.075002in}{1.865488in}}%
\pgfpathlineto{\pgfqpoint{3.075002in}{1.862539in}}%
\pgfpathmoveto{\pgfqpoint{3.070461in}{1.865488in}}%
\pgfpathlineto{\pgfqpoint{3.070461in}{1.865488in}}%
\pgfpathlineto{\pgfqpoint{3.070461in}{1.868438in}}%
\pgfpathlineto{\pgfqpoint{3.075002in}{1.868438in}}%
\pgfpathlineto{\pgfqpoint{3.075002in}{1.865488in}}%
\pgfpathmoveto{\pgfqpoint{3.070461in}{1.868438in}}%
\pgfpathlineto{\pgfqpoint{3.070461in}{1.868438in}}%
\pgfpathlineto{\pgfqpoint{3.070461in}{1.871387in}}%
\pgfpathlineto{\pgfqpoint{3.075002in}{1.871387in}}%
\pgfpathlineto{\pgfqpoint{3.075002in}{1.868438in}}%
\pgfpathmoveto{\pgfqpoint{3.070461in}{1.871387in}}%
\pgfpathlineto{\pgfqpoint{3.070461in}{1.871387in}}%
\pgfpathlineto{\pgfqpoint{3.070461in}{1.874336in}}%
\pgfpathlineto{\pgfqpoint{3.075002in}{1.874336in}}%
\pgfpathlineto{\pgfqpoint{3.075002in}{1.871387in}}%
\pgfpathmoveto{\pgfqpoint{3.070461in}{1.874336in}}%
\pgfpathlineto{\pgfqpoint{3.070461in}{1.874336in}}%
\pgfpathlineto{\pgfqpoint{3.070461in}{1.877285in}}%
\pgfpathlineto{\pgfqpoint{3.075002in}{1.877285in}}%
\pgfpathlineto{\pgfqpoint{3.075002in}{1.874336in}}%
\pgfpathmoveto{\pgfqpoint{3.070461in}{1.877285in}}%
\pgfpathlineto{\pgfqpoint{3.070461in}{1.877285in}}%
\pgfpathlineto{\pgfqpoint{3.070461in}{1.880235in}}%
\pgfpathlineto{\pgfqpoint{3.075002in}{1.880235in}}%
\pgfpathlineto{\pgfqpoint{3.075002in}{1.877285in}}%
\pgfpathmoveto{\pgfqpoint{3.070461in}{1.880235in}}%
\pgfpathlineto{\pgfqpoint{3.070461in}{1.880235in}}%
\pgfpathlineto{\pgfqpoint{3.070461in}{1.883184in}}%
\pgfpathlineto{\pgfqpoint{3.075002in}{1.883184in}}%
\pgfpathlineto{\pgfqpoint{3.075002in}{1.880235in}}%
\pgfpathmoveto{\pgfqpoint{3.070461in}{1.883184in}}%
\pgfpathlineto{\pgfqpoint{3.070461in}{1.883184in}}%
\pgfpathlineto{\pgfqpoint{3.070461in}{1.886133in}}%
\pgfpathlineto{\pgfqpoint{3.075002in}{1.886133in}}%
\pgfpathlineto{\pgfqpoint{3.075002in}{1.883184in}}%
\pgfpathmoveto{\pgfqpoint{3.070461in}{1.886133in}}%
\pgfpathlineto{\pgfqpoint{3.070461in}{1.886133in}}%
\pgfpathlineto{\pgfqpoint{3.070461in}{1.889083in}}%
\pgfpathlineto{\pgfqpoint{3.075002in}{1.889083in}}%
\pgfpathlineto{\pgfqpoint{3.075002in}{1.886133in}}%
\pgfpathmoveto{\pgfqpoint{3.070461in}{1.889083in}}%
\pgfpathlineto{\pgfqpoint{3.070461in}{1.889083in}}%
\pgfpathlineto{\pgfqpoint{3.070461in}{1.892032in}}%
\pgfpathlineto{\pgfqpoint{3.075002in}{1.892032in}}%
\pgfpathlineto{\pgfqpoint{3.075002in}{1.889083in}}%
\pgfpathmoveto{\pgfqpoint{3.070461in}{1.892032in}}%
\pgfpathlineto{\pgfqpoint{3.070461in}{1.892032in}}%
\pgfpathlineto{\pgfqpoint{3.070461in}{1.894981in}}%
\pgfpathlineto{\pgfqpoint{3.075002in}{1.894981in}}%
\pgfpathlineto{\pgfqpoint{3.075002in}{1.892032in}}%
\pgfpathmoveto{\pgfqpoint{3.070461in}{1.894981in}}%
\pgfpathlineto{\pgfqpoint{3.070461in}{1.894981in}}%
\pgfpathlineto{\pgfqpoint{3.070461in}{1.897930in}}%
\pgfpathlineto{\pgfqpoint{3.075002in}{1.897930in}}%
\pgfpathlineto{\pgfqpoint{3.075002in}{1.894981in}}%
\pgfpathmoveto{\pgfqpoint{3.070461in}{1.897930in}}%
\pgfpathlineto{\pgfqpoint{3.070461in}{1.897930in}}%
\pgfpathlineto{\pgfqpoint{3.070461in}{1.900880in}}%
\pgfpathlineto{\pgfqpoint{3.075002in}{1.900880in}}%
\pgfpathlineto{\pgfqpoint{3.075002in}{1.897930in}}%
\pgfpathmoveto{\pgfqpoint{3.070461in}{1.900880in}}%
\pgfpathlineto{\pgfqpoint{3.070461in}{1.900880in}}%
\pgfpathlineto{\pgfqpoint{3.070461in}{1.903829in}}%
\pgfpathlineto{\pgfqpoint{3.075002in}{1.903829in}}%
\pgfpathlineto{\pgfqpoint{3.075002in}{1.900880in}}%
\pgfpathmoveto{\pgfqpoint{3.070461in}{1.903829in}}%
\pgfpathlineto{\pgfqpoint{3.070461in}{1.903829in}}%
\pgfpathlineto{\pgfqpoint{3.070461in}{1.906778in}}%
\pgfpathlineto{\pgfqpoint{3.075002in}{1.906778in}}%
\pgfpathlineto{\pgfqpoint{3.075002in}{1.903829in}}%
\pgfpathmoveto{\pgfqpoint{3.070461in}{1.906778in}}%
\pgfpathlineto{\pgfqpoint{3.070461in}{1.906778in}}%
\pgfpathlineto{\pgfqpoint{3.070461in}{1.909727in}}%
\pgfpathlineto{\pgfqpoint{3.075002in}{1.909727in}}%
\pgfpathlineto{\pgfqpoint{3.075002in}{1.906778in}}%
\pgfpathmoveto{\pgfqpoint{3.070461in}{1.909727in}}%
\pgfpathlineto{\pgfqpoint{3.070461in}{1.909727in}}%
\pgfpathlineto{\pgfqpoint{3.070461in}{1.912677in}}%
\pgfpathlineto{\pgfqpoint{3.075002in}{1.912677in}}%
\pgfpathlineto{\pgfqpoint{3.075002in}{1.909727in}}%
\pgfpathmoveto{\pgfqpoint{3.070461in}{1.912677in}}%
\pgfpathlineto{\pgfqpoint{3.070461in}{1.912677in}}%
\pgfpathlineto{\pgfqpoint{3.070461in}{1.915626in}}%
\pgfpathlineto{\pgfqpoint{3.075002in}{1.915626in}}%
\pgfpathlineto{\pgfqpoint{3.075002in}{1.912677in}}%
\pgfpathmoveto{\pgfqpoint{3.070461in}{1.915626in}}%
\pgfpathlineto{\pgfqpoint{3.070461in}{1.915626in}}%
\pgfpathlineto{\pgfqpoint{3.070461in}{1.918575in}}%
\pgfpathlineto{\pgfqpoint{3.075002in}{1.918575in}}%
\pgfpathlineto{\pgfqpoint{3.075002in}{1.915626in}}%
\pgfpathmoveto{\pgfqpoint{3.070461in}{1.918575in}}%
\pgfpathlineto{\pgfqpoint{3.070461in}{1.918575in}}%
\pgfpathlineto{\pgfqpoint{3.070461in}{1.921524in}}%
\pgfpathlineto{\pgfqpoint{3.075002in}{1.921524in}}%
\pgfpathlineto{\pgfqpoint{3.075002in}{1.918575in}}%
\pgfpathmoveto{\pgfqpoint{3.070461in}{1.921524in}}%
\pgfpathlineto{\pgfqpoint{3.070461in}{1.921524in}}%
\pgfpathlineto{\pgfqpoint{3.070461in}{1.924474in}}%
\pgfpathlineto{\pgfqpoint{3.075002in}{1.924474in}}%
\pgfpathlineto{\pgfqpoint{3.075002in}{1.921524in}}%
\pgfpathmoveto{\pgfqpoint{3.070461in}{1.924474in}}%
\pgfpathlineto{\pgfqpoint{3.070461in}{1.924474in}}%
\pgfpathlineto{\pgfqpoint{3.070461in}{1.927423in}}%
\pgfpathlineto{\pgfqpoint{3.075002in}{1.927423in}}%
\pgfpathlineto{\pgfqpoint{3.075002in}{1.924474in}}%
\pgfpathmoveto{\pgfqpoint{3.070461in}{1.927423in}}%
\pgfpathlineto{\pgfqpoint{3.070461in}{1.927423in}}%
\pgfpathlineto{\pgfqpoint{3.070461in}{1.930372in}}%
\pgfpathlineto{\pgfqpoint{3.075002in}{1.930372in}}%
\pgfpathlineto{\pgfqpoint{3.075002in}{1.927423in}}%
\pgfpathmoveto{\pgfqpoint{3.070461in}{1.930372in}}%
\pgfpathlineto{\pgfqpoint{3.070461in}{1.930372in}}%
\pgfpathlineto{\pgfqpoint{3.070461in}{1.933321in}}%
\pgfpathlineto{\pgfqpoint{3.075002in}{1.933321in}}%
\pgfpathlineto{\pgfqpoint{3.075002in}{1.930372in}}%
\pgfpathmoveto{\pgfqpoint{3.070461in}{1.933321in}}%
\pgfpathlineto{\pgfqpoint{3.070461in}{1.933321in}}%
\pgfpathlineto{\pgfqpoint{3.070461in}{1.936270in}}%
\pgfpathlineto{\pgfqpoint{3.075002in}{1.936270in}}%
\pgfpathlineto{\pgfqpoint{3.075002in}{1.933321in}}%
\pgfpathmoveto{\pgfqpoint{3.070461in}{1.936270in}}%
\pgfpathlineto{\pgfqpoint{3.070461in}{1.936270in}}%
\pgfpathlineto{\pgfqpoint{3.070461in}{1.939219in}}%
\pgfpathlineto{\pgfqpoint{3.075002in}{1.939219in}}%
\pgfpathlineto{\pgfqpoint{3.075002in}{1.936270in}}%
\pgfpathmoveto{\pgfqpoint{3.070461in}{1.939219in}}%
\pgfpathlineto{\pgfqpoint{3.070461in}{1.939219in}}%
\pgfpathlineto{\pgfqpoint{3.070461in}{1.942169in}}%
\pgfpathlineto{\pgfqpoint{3.075002in}{1.942169in}}%
\pgfpathlineto{\pgfqpoint{3.075002in}{1.939219in}}%
\pgfpathmoveto{\pgfqpoint{3.070461in}{1.942169in}}%
\pgfpathlineto{\pgfqpoint{3.070461in}{1.942169in}}%
\pgfpathlineto{\pgfqpoint{3.070461in}{1.945118in}}%
\pgfpathlineto{\pgfqpoint{3.075002in}{1.945118in}}%
\pgfpathlineto{\pgfqpoint{3.075002in}{1.942169in}}%
\pgfpathmoveto{\pgfqpoint{3.070461in}{1.945118in}}%
\pgfpathlineto{\pgfqpoint{3.070461in}{1.945118in}}%
\pgfpathlineto{\pgfqpoint{3.070461in}{1.948067in}}%
\pgfpathlineto{\pgfqpoint{3.075002in}{1.948067in}}%
\pgfpathlineto{\pgfqpoint{3.075002in}{1.945118in}}%
\pgfpathmoveto{\pgfqpoint{3.070461in}{1.948067in}}%
\pgfpathlineto{\pgfqpoint{3.070461in}{1.948067in}}%
\pgfpathlineto{\pgfqpoint{3.070461in}{1.951016in}}%
\pgfpathlineto{\pgfqpoint{3.075002in}{1.951016in}}%
\pgfpathlineto{\pgfqpoint{3.075002in}{1.948067in}}%
\pgfpathmoveto{\pgfqpoint{3.070461in}{1.951016in}}%
\pgfpathlineto{\pgfqpoint{3.070461in}{1.951016in}}%
\pgfpathlineto{\pgfqpoint{3.070461in}{1.953965in}}%
\pgfpathlineto{\pgfqpoint{3.075002in}{1.953965in}}%
\pgfpathlineto{\pgfqpoint{3.075002in}{1.951016in}}%
\pgfpathmoveto{\pgfqpoint{3.070461in}{1.953965in}}%
\pgfpathlineto{\pgfqpoint{3.070461in}{1.953965in}}%
\pgfpathlineto{\pgfqpoint{3.070461in}{1.956914in}}%
\pgfpathlineto{\pgfqpoint{3.075002in}{1.956914in}}%
\pgfpathlineto{\pgfqpoint{3.075002in}{1.953965in}}%
\pgfpathmoveto{\pgfqpoint{3.070461in}{1.956914in}}%
\pgfpathlineto{\pgfqpoint{3.070461in}{1.956914in}}%
\pgfpathlineto{\pgfqpoint{3.070461in}{1.959864in}}%
\pgfpathlineto{\pgfqpoint{3.075002in}{1.959864in}}%
\pgfpathlineto{\pgfqpoint{3.075002in}{1.956914in}}%
\pgfpathmoveto{\pgfqpoint{3.070461in}{1.959864in}}%
\pgfpathlineto{\pgfqpoint{3.070461in}{1.959864in}}%
\pgfpathlineto{\pgfqpoint{3.070461in}{1.962813in}}%
\pgfpathlineto{\pgfqpoint{3.075002in}{1.962813in}}%
\pgfpathlineto{\pgfqpoint{3.075002in}{1.959864in}}%
\pgfpathmoveto{\pgfqpoint{3.070461in}{1.962813in}}%
\pgfpathlineto{\pgfqpoint{3.070461in}{1.962813in}}%
\pgfpathlineto{\pgfqpoint{3.070461in}{1.965762in}}%
\pgfpathlineto{\pgfqpoint{3.075002in}{1.965762in}}%
\pgfpathlineto{\pgfqpoint{3.075002in}{1.962813in}}%
\pgfpathmoveto{\pgfqpoint{3.070461in}{1.965762in}}%
\pgfpathlineto{\pgfqpoint{3.070461in}{1.965762in}}%
\pgfpathlineto{\pgfqpoint{3.070461in}{1.968711in}}%
\pgfpathlineto{\pgfqpoint{3.075002in}{1.968711in}}%
\pgfpathlineto{\pgfqpoint{3.075002in}{1.965762in}}%
\pgfpathmoveto{\pgfqpoint{3.070461in}{1.968711in}}%
\pgfpathlineto{\pgfqpoint{3.070461in}{1.968711in}}%
\pgfpathlineto{\pgfqpoint{3.070461in}{1.971660in}}%
\pgfpathlineto{\pgfqpoint{3.075002in}{1.971660in}}%
\pgfpathlineto{\pgfqpoint{3.075002in}{1.968711in}}%
\pgfpathmoveto{\pgfqpoint{3.070461in}{1.971660in}}%
\pgfpathlineto{\pgfqpoint{3.070461in}{1.971660in}}%
\pgfpathlineto{\pgfqpoint{3.070461in}{1.974610in}}%
\pgfpathlineto{\pgfqpoint{3.075002in}{1.974610in}}%
\pgfpathlineto{\pgfqpoint{3.075002in}{1.971660in}}%
\pgfpathmoveto{\pgfqpoint{3.070461in}{1.974610in}}%
\pgfpathlineto{\pgfqpoint{3.070461in}{1.974610in}}%
\pgfpathlineto{\pgfqpoint{3.070461in}{1.977559in}}%
\pgfpathlineto{\pgfqpoint{3.075002in}{1.977559in}}%
\pgfpathlineto{\pgfqpoint{3.075002in}{1.974610in}}%
\pgfpathmoveto{\pgfqpoint{3.070461in}{1.977559in}}%
\pgfpathlineto{\pgfqpoint{3.070461in}{1.977559in}}%
\pgfpathlineto{\pgfqpoint{3.070461in}{1.980508in}}%
\pgfpathlineto{\pgfqpoint{3.075002in}{1.980508in}}%
\pgfpathlineto{\pgfqpoint{3.075002in}{1.977559in}}%
\pgfpathmoveto{\pgfqpoint{3.070461in}{1.980508in}}%
\pgfpathlineto{\pgfqpoint{3.070461in}{1.980508in}}%
\pgfpathlineto{\pgfqpoint{3.070461in}{1.983457in}}%
\pgfpathlineto{\pgfqpoint{3.075002in}{1.983457in}}%
\pgfpathlineto{\pgfqpoint{3.075002in}{1.980508in}}%
\pgfpathmoveto{\pgfqpoint{3.070461in}{1.983457in}}%
\pgfpathlineto{\pgfqpoint{3.070461in}{1.983457in}}%
\pgfpathlineto{\pgfqpoint{3.070461in}{1.986406in}}%
\pgfpathlineto{\pgfqpoint{3.075002in}{1.986406in}}%
\pgfpathlineto{\pgfqpoint{3.075002in}{1.983457in}}%
\pgfpathmoveto{\pgfqpoint{3.070461in}{1.986406in}}%
\pgfpathlineto{\pgfqpoint{3.070461in}{1.986406in}}%
\pgfpathlineto{\pgfqpoint{3.070461in}{1.989355in}}%
\pgfpathlineto{\pgfqpoint{3.075002in}{1.989355in}}%
\pgfpathlineto{\pgfqpoint{3.075002in}{1.986406in}}%
\pgfpathmoveto{\pgfqpoint{3.070461in}{1.989355in}}%
\pgfpathlineto{\pgfqpoint{3.070461in}{1.989355in}}%
\pgfpathlineto{\pgfqpoint{3.070461in}{1.992305in}}%
\pgfpathlineto{\pgfqpoint{3.075002in}{1.992305in}}%
\pgfpathlineto{\pgfqpoint{3.075002in}{1.989355in}}%
\pgfpathmoveto{\pgfqpoint{3.070461in}{1.992305in}}%
\pgfpathlineto{\pgfqpoint{3.070461in}{1.992305in}}%
\pgfpathlineto{\pgfqpoint{3.070461in}{1.995254in}}%
\pgfpathlineto{\pgfqpoint{3.075002in}{1.995254in}}%
\pgfpathlineto{\pgfqpoint{3.075002in}{1.992305in}}%
\pgfpathmoveto{\pgfqpoint{3.070461in}{1.995254in}}%
\pgfpathlineto{\pgfqpoint{3.070461in}{1.995254in}}%
\pgfpathlineto{\pgfqpoint{3.070461in}{1.998203in}}%
\pgfpathlineto{\pgfqpoint{3.075002in}{1.998203in}}%
\pgfpathlineto{\pgfqpoint{3.075002in}{1.995254in}}%
\pgfpathmoveto{\pgfqpoint{3.070461in}{1.998203in}}%
\pgfpathlineto{\pgfqpoint{3.070461in}{1.998203in}}%
\pgfpathlineto{\pgfqpoint{3.070461in}{2.001152in}}%
\pgfpathlineto{\pgfqpoint{3.075002in}{2.001152in}}%
\pgfpathlineto{\pgfqpoint{3.075002in}{1.998203in}}%
\pgfpathmoveto{\pgfqpoint{3.070461in}{2.001152in}}%
\pgfpathlineto{\pgfqpoint{3.070461in}{2.001152in}}%
\pgfpathlineto{\pgfqpoint{3.070461in}{2.004101in}}%
\pgfpathlineto{\pgfqpoint{3.075002in}{2.004101in}}%
\pgfpathlineto{\pgfqpoint{3.075002in}{2.001152in}}%
\pgfpathmoveto{\pgfqpoint{3.070461in}{2.004101in}}%
\pgfpathlineto{\pgfqpoint{3.070461in}{2.004101in}}%
\pgfpathlineto{\pgfqpoint{3.070461in}{2.007051in}}%
\pgfpathlineto{\pgfqpoint{3.075002in}{2.007051in}}%
\pgfpathlineto{\pgfqpoint{3.075002in}{2.004101in}}%
\pgfpathmoveto{\pgfqpoint{3.070461in}{2.007051in}}%
\pgfpathlineto{\pgfqpoint{3.070461in}{2.007051in}}%
\pgfpathlineto{\pgfqpoint{3.070461in}{2.010000in}}%
\pgfpathlineto{\pgfqpoint{3.075002in}{2.010000in}}%
\pgfpathlineto{\pgfqpoint{3.075002in}{2.007051in}}%
\pgfpathmoveto{\pgfqpoint{3.070461in}{2.010000in}}%
\pgfpathlineto{\pgfqpoint{3.070461in}{2.010000in}}%
\pgfpathlineto{\pgfqpoint{3.070461in}{2.012949in}}%
\pgfpathlineto{\pgfqpoint{3.075002in}{2.012949in}}%
\pgfpathlineto{\pgfqpoint{3.075002in}{2.010000in}}%
\pgfpathmoveto{\pgfqpoint{3.070461in}{2.012949in}}%
\pgfpathlineto{\pgfqpoint{3.070461in}{2.012949in}}%
\pgfpathlineto{\pgfqpoint{3.070461in}{2.015898in}}%
\pgfpathlineto{\pgfqpoint{3.075002in}{2.015898in}}%
\pgfpathlineto{\pgfqpoint{3.075002in}{2.012949in}}%
\pgfpathmoveto{\pgfqpoint{3.070461in}{2.015898in}}%
\pgfpathlineto{\pgfqpoint{3.070461in}{2.015898in}}%
\pgfpathlineto{\pgfqpoint{3.070461in}{2.018847in}}%
\pgfpathlineto{\pgfqpoint{3.075002in}{2.018847in}}%
\pgfpathlineto{\pgfqpoint{3.075002in}{2.015898in}}%
\pgfpathmoveto{\pgfqpoint{3.070461in}{2.018847in}}%
\pgfpathlineto{\pgfqpoint{3.070461in}{2.018847in}}%
\pgfpathlineto{\pgfqpoint{3.070461in}{2.021796in}}%
\pgfpathlineto{\pgfqpoint{3.075002in}{2.021796in}}%
\pgfpathlineto{\pgfqpoint{3.075002in}{2.018847in}}%
\pgfpathmoveto{\pgfqpoint{3.070461in}{2.021796in}}%
\pgfpathlineto{\pgfqpoint{3.070461in}{2.021796in}}%
\pgfpathlineto{\pgfqpoint{3.070461in}{2.024746in}}%
\pgfpathlineto{\pgfqpoint{3.075002in}{2.024746in}}%
\pgfpathlineto{\pgfqpoint{3.075002in}{2.021796in}}%
\pgfpathmoveto{\pgfqpoint{3.070461in}{2.024746in}}%
\pgfpathlineto{\pgfqpoint{3.070461in}{2.024746in}}%
\pgfpathlineto{\pgfqpoint{3.070461in}{2.027695in}}%
\pgfpathlineto{\pgfqpoint{3.075002in}{2.027695in}}%
\pgfpathlineto{\pgfqpoint{3.075002in}{2.024746in}}%
\pgfpathmoveto{\pgfqpoint{3.070461in}{2.027695in}}%
\pgfpathlineto{\pgfqpoint{3.070461in}{2.027695in}}%
\pgfpathlineto{\pgfqpoint{3.070461in}{2.030644in}}%
\pgfpathlineto{\pgfqpoint{3.075002in}{2.030644in}}%
\pgfpathlineto{\pgfqpoint{3.075002in}{2.027695in}}%
\pgfpathmoveto{\pgfqpoint{3.070461in}{2.030644in}}%
\pgfpathlineto{\pgfqpoint{3.070461in}{2.030644in}}%
\pgfpathlineto{\pgfqpoint{3.070461in}{2.033593in}}%
\pgfpathlineto{\pgfqpoint{3.075002in}{2.033593in}}%
\pgfpathlineto{\pgfqpoint{3.075002in}{2.030644in}}%
\pgfpathmoveto{\pgfqpoint{3.070461in}{2.033593in}}%
\pgfpathlineto{\pgfqpoint{3.070461in}{2.033593in}}%
\pgfpathlineto{\pgfqpoint{3.070461in}{2.036542in}}%
\pgfpathlineto{\pgfqpoint{3.075002in}{2.036542in}}%
\pgfpathlineto{\pgfqpoint{3.075002in}{2.033593in}}%
\pgfpathmoveto{\pgfqpoint{3.070461in}{2.036542in}}%
\pgfpathlineto{\pgfqpoint{3.070461in}{2.036542in}}%
\pgfpathlineto{\pgfqpoint{3.070461in}{2.039491in}}%
\pgfpathlineto{\pgfqpoint{3.075002in}{2.039491in}}%
\pgfpathlineto{\pgfqpoint{3.075002in}{2.036542in}}%
\pgfpathmoveto{\pgfqpoint{3.070461in}{2.039491in}}%
\pgfpathlineto{\pgfqpoint{3.070461in}{2.039491in}}%
\pgfpathlineto{\pgfqpoint{3.070461in}{2.042441in}}%
\pgfpathlineto{\pgfqpoint{3.075002in}{2.042441in}}%
\pgfpathlineto{\pgfqpoint{3.075002in}{2.039491in}}%
\pgfpathmoveto{\pgfqpoint{3.070461in}{2.042441in}}%
\pgfpathlineto{\pgfqpoint{3.070461in}{2.042441in}}%
\pgfpathlineto{\pgfqpoint{3.070461in}{2.045390in}}%
\pgfpathlineto{\pgfqpoint{3.075002in}{2.045390in}}%
\pgfpathlineto{\pgfqpoint{3.075002in}{2.042441in}}%
\pgfpathmoveto{\pgfqpoint{3.070461in}{2.045390in}}%
\pgfpathlineto{\pgfqpoint{3.070461in}{2.045390in}}%
\pgfpathlineto{\pgfqpoint{3.070461in}{2.048339in}}%
\pgfpathlineto{\pgfqpoint{3.075002in}{2.048339in}}%
\pgfpathlineto{\pgfqpoint{3.075002in}{2.045390in}}%
\pgfpathmoveto{\pgfqpoint{3.070461in}{2.048339in}}%
\pgfpathlineto{\pgfqpoint{3.070461in}{2.048339in}}%
\pgfpathlineto{\pgfqpoint{3.070461in}{2.051288in}}%
\pgfpathlineto{\pgfqpoint{3.075002in}{2.051288in}}%
\pgfpathlineto{\pgfqpoint{3.075002in}{2.048339in}}%
\pgfpathmoveto{\pgfqpoint{3.070461in}{2.051288in}}%
\pgfpathlineto{\pgfqpoint{3.070461in}{2.051288in}}%
\pgfpathlineto{\pgfqpoint{3.070461in}{2.054237in}}%
\pgfpathlineto{\pgfqpoint{3.075002in}{2.054237in}}%
\pgfpathlineto{\pgfqpoint{3.075002in}{2.051288in}}%
\pgfpathmoveto{\pgfqpoint{3.070461in}{2.054237in}}%
\pgfpathlineto{\pgfqpoint{3.070461in}{2.054237in}}%
\pgfpathlineto{\pgfqpoint{3.070461in}{2.057187in}}%
\pgfpathlineto{\pgfqpoint{3.075002in}{2.057187in}}%
\pgfpathlineto{\pgfqpoint{3.075002in}{2.054237in}}%
\pgfpathmoveto{\pgfqpoint{3.070461in}{2.057187in}}%
\pgfpathlineto{\pgfqpoint{3.070461in}{2.057187in}}%
\pgfpathlineto{\pgfqpoint{3.070461in}{2.060136in}}%
\pgfpathlineto{\pgfqpoint{3.075002in}{2.060136in}}%
\pgfpathlineto{\pgfqpoint{3.075002in}{2.057187in}}%
\pgfpathmoveto{\pgfqpoint{3.070461in}{2.060136in}}%
\pgfpathlineto{\pgfqpoint{3.070461in}{2.060136in}}%
\pgfpathlineto{\pgfqpoint{3.070461in}{2.063085in}}%
\pgfpathlineto{\pgfqpoint{3.075002in}{2.063085in}}%
\pgfpathlineto{\pgfqpoint{3.075002in}{2.060136in}}%
\pgfpathmoveto{\pgfqpoint{3.070461in}{2.063085in}}%
\pgfpathlineto{\pgfqpoint{3.070461in}{2.063085in}}%
\pgfpathlineto{\pgfqpoint{3.070461in}{2.066034in}}%
\pgfpathlineto{\pgfqpoint{3.075002in}{2.066034in}}%
\pgfpathlineto{\pgfqpoint{3.075002in}{2.063085in}}%
\pgfpathmoveto{\pgfqpoint{3.070461in}{2.066034in}}%
\pgfpathlineto{\pgfqpoint{3.070461in}{2.066034in}}%
\pgfpathlineto{\pgfqpoint{3.070461in}{2.068983in}}%
\pgfpathlineto{\pgfqpoint{3.075002in}{2.068983in}}%
\pgfpathlineto{\pgfqpoint{3.075002in}{2.066034in}}%
\pgfpathmoveto{\pgfqpoint{3.070461in}{2.068983in}}%
\pgfpathlineto{\pgfqpoint{3.070461in}{2.068983in}}%
\pgfpathlineto{\pgfqpoint{3.070461in}{2.071932in}}%
\pgfpathlineto{\pgfqpoint{3.075002in}{2.071932in}}%
\pgfpathlineto{\pgfqpoint{3.075002in}{2.068983in}}%
\pgfpathmoveto{\pgfqpoint{3.070461in}{2.071932in}}%
\pgfpathlineto{\pgfqpoint{3.070461in}{2.071932in}}%
\pgfpathlineto{\pgfqpoint{3.070461in}{2.074882in}}%
\pgfpathlineto{\pgfqpoint{3.075002in}{2.074882in}}%
\pgfpathlineto{\pgfqpoint{3.075002in}{2.071932in}}%
\pgfpathmoveto{\pgfqpoint{3.070461in}{2.074882in}}%
\pgfpathlineto{\pgfqpoint{3.070461in}{2.074882in}}%
\pgfpathlineto{\pgfqpoint{3.070461in}{2.077831in}}%
\pgfpathlineto{\pgfqpoint{3.075002in}{2.077831in}}%
\pgfpathlineto{\pgfqpoint{3.075002in}{2.074882in}}%
\pgfpathmoveto{\pgfqpoint{3.070461in}{2.077831in}}%
\pgfpathlineto{\pgfqpoint{3.070461in}{2.077831in}}%
\pgfpathlineto{\pgfqpoint{3.070461in}{2.080780in}}%
\pgfpathlineto{\pgfqpoint{3.075002in}{2.080780in}}%
\pgfpathlineto{\pgfqpoint{3.075002in}{2.077831in}}%
\pgfpathmoveto{\pgfqpoint{3.070461in}{2.080780in}}%
\pgfpathlineto{\pgfqpoint{3.070461in}{2.080780in}}%
\pgfpathlineto{\pgfqpoint{3.070461in}{2.083729in}}%
\pgfpathlineto{\pgfqpoint{3.075002in}{2.083729in}}%
\pgfpathlineto{\pgfqpoint{3.075002in}{2.080780in}}%
\pgfpathmoveto{\pgfqpoint{3.070461in}{2.083729in}}%
\pgfpathlineto{\pgfqpoint{3.070461in}{2.083729in}}%
\pgfpathlineto{\pgfqpoint{3.070461in}{2.086678in}}%
\pgfpathlineto{\pgfqpoint{3.075002in}{2.086678in}}%
\pgfpathlineto{\pgfqpoint{3.075002in}{2.083729in}}%
\pgfpathmoveto{\pgfqpoint{3.070461in}{2.086678in}}%
\pgfpathlineto{\pgfqpoint{3.070461in}{2.086678in}}%
\pgfpathlineto{\pgfqpoint{3.070461in}{2.089628in}}%
\pgfpathlineto{\pgfqpoint{3.075002in}{2.089628in}}%
\pgfpathlineto{\pgfqpoint{3.075002in}{2.086678in}}%
\pgfpathmoveto{\pgfqpoint{3.070461in}{2.089628in}}%
\pgfpathlineto{\pgfqpoint{3.070461in}{2.089628in}}%
\pgfpathlineto{\pgfqpoint{3.070461in}{2.092577in}}%
\pgfpathlineto{\pgfqpoint{3.075002in}{2.092577in}}%
\pgfpathlineto{\pgfqpoint{3.075002in}{2.089628in}}%
\pgfpathmoveto{\pgfqpoint{3.070461in}{2.092577in}}%
\pgfpathlineto{\pgfqpoint{3.070461in}{2.092577in}}%
\pgfpathlineto{\pgfqpoint{3.070461in}{2.095526in}}%
\pgfpathlineto{\pgfqpoint{3.075002in}{2.095526in}}%
\pgfpathlineto{\pgfqpoint{3.075002in}{2.092577in}}%
\pgfpathmoveto{\pgfqpoint{3.070461in}{2.095526in}}%
\pgfpathlineto{\pgfqpoint{3.070461in}{2.095526in}}%
\pgfpathlineto{\pgfqpoint{3.070461in}{2.098475in}}%
\pgfpathlineto{\pgfqpoint{3.075002in}{2.098475in}}%
\pgfpathlineto{\pgfqpoint{3.075002in}{2.095526in}}%
\pgfpathmoveto{\pgfqpoint{3.070461in}{2.098475in}}%
\pgfpathlineto{\pgfqpoint{3.070461in}{2.098475in}}%
\pgfpathlineto{\pgfqpoint{3.070461in}{2.101424in}}%
\pgfpathlineto{\pgfqpoint{3.075002in}{2.101424in}}%
\pgfpathlineto{\pgfqpoint{3.075002in}{2.098475in}}%
\pgfpathmoveto{\pgfqpoint{3.070461in}{2.101424in}}%
\pgfpathlineto{\pgfqpoint{3.070461in}{2.101424in}}%
\pgfpathlineto{\pgfqpoint{3.070461in}{2.104373in}}%
\pgfpathlineto{\pgfqpoint{3.075002in}{2.104373in}}%
\pgfpathlineto{\pgfqpoint{3.075002in}{2.101424in}}%
\pgfpathmoveto{\pgfqpoint{3.070461in}{2.104373in}}%
\pgfpathlineto{\pgfqpoint{3.070461in}{2.104373in}}%
\pgfpathlineto{\pgfqpoint{3.070461in}{2.107323in}}%
\pgfpathlineto{\pgfqpoint{3.075002in}{2.107323in}}%
\pgfpathlineto{\pgfqpoint{3.075002in}{2.104373in}}%
\pgfpathmoveto{\pgfqpoint{3.070461in}{2.107323in}}%
\pgfpathlineto{\pgfqpoint{3.070461in}{2.107323in}}%
\pgfpathlineto{\pgfqpoint{3.070461in}{2.110272in}}%
\pgfpathlineto{\pgfqpoint{3.075002in}{2.110272in}}%
\pgfpathlineto{\pgfqpoint{3.075002in}{2.107323in}}%
\pgfpathmoveto{\pgfqpoint{3.070461in}{2.110272in}}%
\pgfpathlineto{\pgfqpoint{3.070461in}{2.110272in}}%
\pgfpathlineto{\pgfqpoint{3.070461in}{2.113222in}}%
\pgfpathlineto{\pgfqpoint{3.075002in}{2.113222in}}%
\pgfpathlineto{\pgfqpoint{3.075002in}{2.110272in}}%
\pgfpathmoveto{\pgfqpoint{3.070461in}{2.113222in}}%
\pgfpathlineto{\pgfqpoint{3.070461in}{2.113222in}}%
\pgfpathlineto{\pgfqpoint{3.070461in}{2.116171in}}%
\pgfpathlineto{\pgfqpoint{3.075002in}{2.116171in}}%
\pgfpathlineto{\pgfqpoint{3.075002in}{2.113222in}}%
\pgfpathmoveto{\pgfqpoint{3.070461in}{2.116171in}}%
\pgfpathlineto{\pgfqpoint{3.070461in}{2.116171in}}%
\pgfpathlineto{\pgfqpoint{3.070461in}{2.119120in}}%
\pgfpathlineto{\pgfqpoint{3.075002in}{2.119120in}}%
\pgfpathlineto{\pgfqpoint{3.075002in}{2.116171in}}%
\pgfpathmoveto{\pgfqpoint{3.070461in}{2.119120in}}%
\pgfpathlineto{\pgfqpoint{3.070461in}{2.119120in}}%
\pgfpathlineto{\pgfqpoint{3.070461in}{2.122070in}}%
\pgfpathlineto{\pgfqpoint{3.075002in}{2.122070in}}%
\pgfpathlineto{\pgfqpoint{3.075002in}{2.119120in}}%
\pgfpathmoveto{\pgfqpoint{3.070461in}{2.122070in}}%
\pgfpathlineto{\pgfqpoint{3.070461in}{2.122070in}}%
\pgfpathlineto{\pgfqpoint{3.070461in}{2.125019in}}%
\pgfpathlineto{\pgfqpoint{3.075002in}{2.125019in}}%
\pgfpathlineto{\pgfqpoint{3.075002in}{2.122070in}}%
\pgfpathmoveto{\pgfqpoint{3.070461in}{2.125019in}}%
\pgfpathlineto{\pgfqpoint{3.070461in}{2.125019in}}%
\pgfpathlineto{\pgfqpoint{3.070461in}{2.127968in}}%
\pgfpathlineto{\pgfqpoint{3.075002in}{2.127968in}}%
\pgfpathlineto{\pgfqpoint{3.075002in}{2.125019in}}%
\pgfpathmoveto{\pgfqpoint{3.070461in}{2.127968in}}%
\pgfpathlineto{\pgfqpoint{3.070461in}{2.127968in}}%
\pgfpathlineto{\pgfqpoint{3.070461in}{2.130918in}}%
\pgfpathlineto{\pgfqpoint{3.075002in}{2.130918in}}%
\pgfpathlineto{\pgfqpoint{3.075002in}{2.127968in}}%
\pgfpathmoveto{\pgfqpoint{3.070461in}{2.130918in}}%
\pgfpathlineto{\pgfqpoint{3.070461in}{2.130918in}}%
\pgfpathlineto{\pgfqpoint{3.070461in}{2.133867in}}%
\pgfpathlineto{\pgfqpoint{3.075002in}{2.133867in}}%
\pgfpathlineto{\pgfqpoint{3.075002in}{2.130918in}}%
\pgfpathmoveto{\pgfqpoint{3.070461in}{2.133867in}}%
\pgfpathlineto{\pgfqpoint{3.070461in}{2.133867in}}%
\pgfpathlineto{\pgfqpoint{3.070461in}{2.136816in}}%
\pgfpathlineto{\pgfqpoint{3.075002in}{2.136816in}}%
\pgfpathlineto{\pgfqpoint{3.075002in}{2.133867in}}%
\pgfpathmoveto{\pgfqpoint{3.070461in}{2.136816in}}%
\pgfpathlineto{\pgfqpoint{3.070461in}{2.136816in}}%
\pgfpathlineto{\pgfqpoint{3.070461in}{2.139766in}}%
\pgfpathlineto{\pgfqpoint{3.075002in}{2.139766in}}%
\pgfpathlineto{\pgfqpoint{3.075002in}{2.136816in}}%
\pgfpathmoveto{\pgfqpoint{3.070461in}{2.139766in}}%
\pgfpathlineto{\pgfqpoint{3.070461in}{2.139766in}}%
\pgfpathlineto{\pgfqpoint{3.070461in}{2.142715in}}%
\pgfpathlineto{\pgfqpoint{3.075002in}{2.142715in}}%
\pgfpathlineto{\pgfqpoint{3.075002in}{2.139766in}}%
\pgfpathmoveto{\pgfqpoint{3.070461in}{2.142715in}}%
\pgfpathlineto{\pgfqpoint{3.070461in}{2.142715in}}%
\pgfpathlineto{\pgfqpoint{3.070461in}{2.145664in}}%
\pgfpathlineto{\pgfqpoint{3.075002in}{2.145664in}}%
\pgfpathlineto{\pgfqpoint{3.075002in}{2.142715in}}%
\pgfpathmoveto{\pgfqpoint{3.070461in}{2.145664in}}%
\pgfpathlineto{\pgfqpoint{3.070461in}{2.145664in}}%
\pgfpathlineto{\pgfqpoint{3.070461in}{2.148614in}}%
\pgfpathlineto{\pgfqpoint{3.075002in}{2.148614in}}%
\pgfpathlineto{\pgfqpoint{3.075002in}{2.145664in}}%
\pgfpathmoveto{\pgfqpoint{3.070461in}{2.148614in}}%
\pgfpathlineto{\pgfqpoint{3.070461in}{2.148614in}}%
\pgfpathlineto{\pgfqpoint{3.070461in}{2.151563in}}%
\pgfpathlineto{\pgfqpoint{3.075002in}{2.151563in}}%
\pgfpathlineto{\pgfqpoint{3.075002in}{2.148614in}}%
\pgfpathmoveto{\pgfqpoint{3.070461in}{2.151563in}}%
\pgfpathlineto{\pgfqpoint{3.070461in}{2.151563in}}%
\pgfpathlineto{\pgfqpoint{3.070461in}{2.154512in}}%
\pgfpathlineto{\pgfqpoint{3.075002in}{2.154512in}}%
\pgfpathlineto{\pgfqpoint{3.075002in}{2.151563in}}%
\pgfpathmoveto{\pgfqpoint{3.070461in}{2.154512in}}%
\pgfpathlineto{\pgfqpoint{3.070461in}{2.154512in}}%
\pgfpathlineto{\pgfqpoint{3.070461in}{2.157462in}}%
\pgfpathlineto{\pgfqpoint{3.075002in}{2.157462in}}%
\pgfpathlineto{\pgfqpoint{3.075002in}{2.154512in}}%
\pgfpathmoveto{\pgfqpoint{3.070461in}{2.157462in}}%
\pgfpathlineto{\pgfqpoint{3.070461in}{2.157462in}}%
\pgfpathlineto{\pgfqpoint{3.070461in}{2.160411in}}%
\pgfpathlineto{\pgfqpoint{3.075002in}{2.160411in}}%
\pgfpathlineto{\pgfqpoint{3.075002in}{2.157462in}}%
\pgfpathmoveto{\pgfqpoint{3.070461in}{2.160411in}}%
\pgfpathlineto{\pgfqpoint{3.070461in}{2.160411in}}%
\pgfpathlineto{\pgfqpoint{3.070461in}{2.163360in}}%
\pgfpathlineto{\pgfqpoint{3.075002in}{2.163360in}}%
\pgfpathlineto{\pgfqpoint{3.075002in}{2.160411in}}%
\pgfpathmoveto{\pgfqpoint{3.070461in}{2.163360in}}%
\pgfpathlineto{\pgfqpoint{3.070461in}{2.163360in}}%
\pgfpathlineto{\pgfqpoint{3.070461in}{2.166310in}}%
\pgfpathlineto{\pgfqpoint{3.075002in}{2.166310in}}%
\pgfpathlineto{\pgfqpoint{3.075002in}{2.163360in}}%
\pgfpathmoveto{\pgfqpoint{3.070461in}{2.166310in}}%
\pgfpathlineto{\pgfqpoint{3.070461in}{2.166310in}}%
\pgfpathlineto{\pgfqpoint{3.070461in}{2.169259in}}%
\pgfpathlineto{\pgfqpoint{3.075002in}{2.169259in}}%
\pgfpathlineto{\pgfqpoint{3.075002in}{2.166310in}}%
\pgfpathmoveto{\pgfqpoint{3.070461in}{2.169259in}}%
\pgfpathlineto{\pgfqpoint{3.070461in}{2.169259in}}%
\pgfpathlineto{\pgfqpoint{3.070461in}{2.172209in}}%
\pgfpathlineto{\pgfqpoint{3.075002in}{2.172209in}}%
\pgfpathlineto{\pgfqpoint{3.075002in}{2.169259in}}%
\pgfpathmoveto{\pgfqpoint{3.070461in}{2.172209in}}%
\pgfpathlineto{\pgfqpoint{3.070461in}{2.172209in}}%
\pgfpathlineto{\pgfqpoint{3.070461in}{2.175158in}}%
\pgfpathlineto{\pgfqpoint{3.075002in}{2.175158in}}%
\pgfpathlineto{\pgfqpoint{3.075002in}{2.172209in}}%
\pgfpathmoveto{\pgfqpoint{3.070461in}{2.175158in}}%
\pgfpathlineto{\pgfqpoint{3.070461in}{2.175158in}}%
\pgfpathlineto{\pgfqpoint{3.070461in}{2.178107in}}%
\pgfpathlineto{\pgfqpoint{3.075002in}{2.178107in}}%
\pgfpathlineto{\pgfqpoint{3.075002in}{2.175158in}}%
\pgfpathmoveto{\pgfqpoint{3.070461in}{2.178107in}}%
\pgfpathlineto{\pgfqpoint{3.070461in}{2.178107in}}%
\pgfpathlineto{\pgfqpoint{3.070461in}{2.181057in}}%
\pgfpathlineto{\pgfqpoint{3.075002in}{2.181057in}}%
\pgfpathlineto{\pgfqpoint{3.075002in}{2.178107in}}%
\pgfpathmoveto{\pgfqpoint{3.070461in}{2.181057in}}%
\pgfpathlineto{\pgfqpoint{3.070461in}{2.181057in}}%
\pgfpathlineto{\pgfqpoint{3.070461in}{2.184006in}}%
\pgfpathlineto{\pgfqpoint{3.075002in}{2.184006in}}%
\pgfpathlineto{\pgfqpoint{3.075002in}{2.181057in}}%
\pgfpathmoveto{\pgfqpoint{3.070461in}{2.184006in}}%
\pgfpathlineto{\pgfqpoint{3.070461in}{2.184006in}}%
\pgfpathlineto{\pgfqpoint{3.070461in}{2.186955in}}%
\pgfpathlineto{\pgfqpoint{3.075002in}{2.186955in}}%
\pgfpathlineto{\pgfqpoint{3.075002in}{2.184006in}}%
\pgfpathmoveto{\pgfqpoint{3.070461in}{2.186955in}}%
\pgfpathlineto{\pgfqpoint{3.070461in}{2.186955in}}%
\pgfpathlineto{\pgfqpoint{3.070461in}{2.189905in}}%
\pgfpathlineto{\pgfqpoint{3.075002in}{2.189905in}}%
\pgfpathlineto{\pgfqpoint{3.075002in}{2.186955in}}%
\pgfpathmoveto{\pgfqpoint{3.070461in}{2.189905in}}%
\pgfpathlineto{\pgfqpoint{3.070461in}{2.189905in}}%
\pgfpathlineto{\pgfqpoint{3.070461in}{2.192854in}}%
\pgfpathlineto{\pgfqpoint{3.075002in}{2.192854in}}%
\pgfpathlineto{\pgfqpoint{3.075002in}{2.189905in}}%
\pgfpathmoveto{\pgfqpoint{3.070461in}{2.192854in}}%
\pgfpathlineto{\pgfqpoint{3.070461in}{2.192854in}}%
\pgfpathlineto{\pgfqpoint{3.070461in}{2.195803in}}%
\pgfpathlineto{\pgfqpoint{3.075002in}{2.195803in}}%
\pgfpathlineto{\pgfqpoint{3.075002in}{2.192854in}}%
\pgfpathmoveto{\pgfqpoint{3.070461in}{2.195803in}}%
\pgfpathlineto{\pgfqpoint{3.070461in}{2.195803in}}%
\pgfpathlineto{\pgfqpoint{3.070461in}{2.198753in}}%
\pgfpathlineto{\pgfqpoint{3.075002in}{2.198753in}}%
\pgfpathlineto{\pgfqpoint{3.075002in}{2.195803in}}%
\pgfpathmoveto{\pgfqpoint{3.070461in}{2.198753in}}%
\pgfpathlineto{\pgfqpoint{3.070461in}{2.198753in}}%
\pgfpathlineto{\pgfqpoint{3.070461in}{2.201702in}}%
\pgfpathlineto{\pgfqpoint{3.075002in}{2.201702in}}%
\pgfpathlineto{\pgfqpoint{3.075002in}{2.198753in}}%
\pgfpathmoveto{\pgfqpoint{3.070461in}{2.201702in}}%
\pgfpathlineto{\pgfqpoint{3.070461in}{2.201702in}}%
\pgfpathlineto{\pgfqpoint{3.070461in}{2.204651in}}%
\pgfpathlineto{\pgfqpoint{3.075002in}{2.204651in}}%
\pgfpathlineto{\pgfqpoint{3.075002in}{2.201702in}}%
\pgfpathmoveto{\pgfqpoint{3.070461in}{2.204651in}}%
\pgfpathlineto{\pgfqpoint{3.070461in}{2.204651in}}%
\pgfpathlineto{\pgfqpoint{3.070461in}{2.207600in}}%
\pgfpathlineto{\pgfqpoint{3.075002in}{2.207600in}}%
\pgfpathlineto{\pgfqpoint{3.075002in}{2.204651in}}%
\pgfpathmoveto{\pgfqpoint{3.070461in}{2.207600in}}%
\pgfpathlineto{\pgfqpoint{3.070461in}{2.207600in}}%
\pgfpathlineto{\pgfqpoint{3.070461in}{2.210549in}}%
\pgfpathlineto{\pgfqpoint{3.075002in}{2.210549in}}%
\pgfpathlineto{\pgfqpoint{3.075002in}{2.207600in}}%
\pgfpathmoveto{\pgfqpoint{3.070461in}{2.210549in}}%
\pgfpathlineto{\pgfqpoint{3.070461in}{2.210549in}}%
\pgfpathlineto{\pgfqpoint{3.070461in}{2.213498in}}%
\pgfpathlineto{\pgfqpoint{3.075002in}{2.213498in}}%
\pgfpathlineto{\pgfqpoint{3.075002in}{2.210549in}}%
\pgfpathmoveto{\pgfqpoint{3.070461in}{2.213498in}}%
\pgfpathlineto{\pgfqpoint{3.070461in}{2.213498in}}%
\pgfpathlineto{\pgfqpoint{3.070461in}{2.216447in}}%
\pgfpathlineto{\pgfqpoint{3.075002in}{2.216447in}}%
\pgfpathlineto{\pgfqpoint{3.075002in}{2.213498in}}%
\pgfpathmoveto{\pgfqpoint{3.070461in}{2.216447in}}%
\pgfpathlineto{\pgfqpoint{3.070461in}{2.216447in}}%
\pgfpathlineto{\pgfqpoint{3.070461in}{2.219396in}}%
\pgfpathlineto{\pgfqpoint{3.075002in}{2.219396in}}%
\pgfpathlineto{\pgfqpoint{3.075002in}{2.216447in}}%
\pgfpathmoveto{\pgfqpoint{3.070461in}{2.219396in}}%
\pgfpathlineto{\pgfqpoint{3.070461in}{2.219396in}}%
\pgfpathlineto{\pgfqpoint{3.070461in}{2.222345in}}%
\pgfpathlineto{\pgfqpoint{3.075002in}{2.222345in}}%
\pgfpathlineto{\pgfqpoint{3.075002in}{2.219396in}}%
\pgfpathmoveto{\pgfqpoint{3.070461in}{2.222345in}}%
\pgfpathlineto{\pgfqpoint{3.070461in}{2.222345in}}%
\pgfpathlineto{\pgfqpoint{3.070461in}{2.225294in}}%
\pgfpathlineto{\pgfqpoint{3.075002in}{2.225294in}}%
\pgfpathlineto{\pgfqpoint{3.075002in}{2.222345in}}%
\pgfpathmoveto{\pgfqpoint{3.070461in}{2.225294in}}%
\pgfpathlineto{\pgfqpoint{3.070461in}{2.225294in}}%
\pgfpathlineto{\pgfqpoint{3.070461in}{2.228243in}}%
\pgfpathlineto{\pgfqpoint{3.075002in}{2.228243in}}%
\pgfpathlineto{\pgfqpoint{3.075002in}{2.225294in}}%
\pgfpathmoveto{\pgfqpoint{3.070461in}{2.228243in}}%
\pgfpathlineto{\pgfqpoint{3.070461in}{2.228243in}}%
\pgfpathlineto{\pgfqpoint{3.070461in}{2.231192in}}%
\pgfpathlineto{\pgfqpoint{3.075002in}{2.231192in}}%
\pgfpathlineto{\pgfqpoint{3.075002in}{2.228243in}}%
\pgfpathmoveto{\pgfqpoint{3.070461in}{2.231192in}}%
\pgfpathlineto{\pgfqpoint{3.070461in}{2.231192in}}%
\pgfpathlineto{\pgfqpoint{3.070461in}{2.234141in}}%
\pgfpathlineto{\pgfqpoint{3.075002in}{2.234141in}}%
\pgfpathlineto{\pgfqpoint{3.075002in}{2.231192in}}%
\pgfpathmoveto{\pgfqpoint{3.070461in}{2.234141in}}%
\pgfpathlineto{\pgfqpoint{3.070461in}{2.234141in}}%
\pgfpathlineto{\pgfqpoint{3.070461in}{2.237090in}}%
\pgfpathlineto{\pgfqpoint{3.075002in}{2.237090in}}%
\pgfpathlineto{\pgfqpoint{3.075002in}{2.234141in}}%
\pgfpathmoveto{\pgfqpoint{3.070461in}{2.237090in}}%
\pgfpathlineto{\pgfqpoint{3.070461in}{2.237090in}}%
\pgfpathlineto{\pgfqpoint{3.070461in}{2.240039in}}%
\pgfpathlineto{\pgfqpoint{3.075002in}{2.240039in}}%
\pgfpathlineto{\pgfqpoint{3.075002in}{2.237090in}}%
\pgfpathmoveto{\pgfqpoint{3.070461in}{2.240039in}}%
\pgfpathlineto{\pgfqpoint{3.070461in}{2.240039in}}%
\pgfpathlineto{\pgfqpoint{3.070461in}{2.242989in}}%
\pgfpathlineto{\pgfqpoint{3.075002in}{2.242989in}}%
\pgfpathlineto{\pgfqpoint{3.075002in}{2.240039in}}%
\pgfpathmoveto{\pgfqpoint{3.070461in}{2.242989in}}%
\pgfpathlineto{\pgfqpoint{3.070461in}{2.242989in}}%
\pgfpathlineto{\pgfqpoint{3.070461in}{2.245938in}}%
\pgfpathlineto{\pgfqpoint{3.075002in}{2.245938in}}%
\pgfpathlineto{\pgfqpoint{3.075002in}{2.242989in}}%
\pgfpathmoveto{\pgfqpoint{3.070461in}{2.245938in}}%
\pgfpathlineto{\pgfqpoint{3.070461in}{2.245938in}}%
\pgfpathlineto{\pgfqpoint{3.070461in}{2.248887in}}%
\pgfpathlineto{\pgfqpoint{3.075002in}{2.248887in}}%
\pgfpathlineto{\pgfqpoint{3.075002in}{2.245938in}}%
\pgfpathmoveto{\pgfqpoint{3.070461in}{2.248887in}}%
\pgfpathlineto{\pgfqpoint{3.070461in}{2.248887in}}%
\pgfpathlineto{\pgfqpoint{3.070461in}{2.251836in}}%
\pgfpathlineto{\pgfqpoint{3.075002in}{2.251836in}}%
\pgfpathlineto{\pgfqpoint{3.075002in}{2.248887in}}%
\pgfpathmoveto{\pgfqpoint{3.070461in}{2.251836in}}%
\pgfpathlineto{\pgfqpoint{3.070461in}{2.251836in}}%
\pgfpathlineto{\pgfqpoint{3.070461in}{2.254785in}}%
\pgfpathlineto{\pgfqpoint{3.075002in}{2.254785in}}%
\pgfpathlineto{\pgfqpoint{3.075002in}{2.251836in}}%
\pgfpathmoveto{\pgfqpoint{3.070461in}{2.254785in}}%
\pgfpathlineto{\pgfqpoint{3.070461in}{2.254785in}}%
\pgfpathlineto{\pgfqpoint{3.070461in}{2.257734in}}%
\pgfpathlineto{\pgfqpoint{3.075002in}{2.257734in}}%
\pgfpathlineto{\pgfqpoint{3.075002in}{2.254785in}}%
\pgfpathmoveto{\pgfqpoint{3.070461in}{2.257734in}}%
\pgfpathlineto{\pgfqpoint{3.070461in}{2.257734in}}%
\pgfpathlineto{\pgfqpoint{3.070461in}{2.260683in}}%
\pgfpathlineto{\pgfqpoint{3.075002in}{2.260683in}}%
\pgfpathlineto{\pgfqpoint{3.075002in}{2.257734in}}%
\pgfpathmoveto{\pgfqpoint{3.070461in}{2.260683in}}%
\pgfpathlineto{\pgfqpoint{3.070461in}{2.260683in}}%
\pgfpathlineto{\pgfqpoint{3.070461in}{2.263632in}}%
\pgfpathlineto{\pgfqpoint{3.075002in}{2.263632in}}%
\pgfpathlineto{\pgfqpoint{3.075002in}{2.260683in}}%
\pgfpathmoveto{\pgfqpoint{3.070461in}{2.263632in}}%
\pgfpathlineto{\pgfqpoint{3.070461in}{2.263632in}}%
\pgfpathlineto{\pgfqpoint{3.070461in}{2.266581in}}%
\pgfpathlineto{\pgfqpoint{3.075002in}{2.266581in}}%
\pgfpathlineto{\pgfqpoint{3.075002in}{2.263632in}}%
\pgfpathmoveto{\pgfqpoint{3.070461in}{2.266581in}}%
\pgfpathlineto{\pgfqpoint{3.070461in}{2.266581in}}%
\pgfpathlineto{\pgfqpoint{3.070461in}{2.269530in}}%
\pgfpathlineto{\pgfqpoint{3.075002in}{2.269530in}}%
\pgfpathlineto{\pgfqpoint{3.075002in}{2.266581in}}%
\pgfpathmoveto{\pgfqpoint{3.070461in}{2.269530in}}%
\pgfpathlineto{\pgfqpoint{3.070461in}{2.269530in}}%
\pgfpathlineto{\pgfqpoint{3.070461in}{2.272479in}}%
\pgfpathlineto{\pgfqpoint{3.075002in}{2.272479in}}%
\pgfpathlineto{\pgfqpoint{3.075002in}{2.269530in}}%
\pgfpathmoveto{\pgfqpoint{3.070461in}{2.272479in}}%
\pgfpathlineto{\pgfqpoint{3.070461in}{2.272479in}}%
\pgfpathlineto{\pgfqpoint{3.070461in}{2.275428in}}%
\pgfpathlineto{\pgfqpoint{3.075002in}{2.275428in}}%
\pgfpathlineto{\pgfqpoint{3.075002in}{2.272479in}}%
\pgfpathmoveto{\pgfqpoint{3.070461in}{2.275428in}}%
\pgfpathlineto{\pgfqpoint{3.070461in}{2.275428in}}%
\pgfpathlineto{\pgfqpoint{3.070461in}{2.278377in}}%
\pgfpathlineto{\pgfqpoint{3.075002in}{2.278377in}}%
\pgfpathlineto{\pgfqpoint{3.075002in}{2.275428in}}%
\pgfpathmoveto{\pgfqpoint{3.070461in}{2.278377in}}%
\pgfpathlineto{\pgfqpoint{3.070461in}{2.278377in}}%
\pgfpathlineto{\pgfqpoint{3.070461in}{2.281326in}}%
\pgfpathlineto{\pgfqpoint{3.075002in}{2.281326in}}%
\pgfpathlineto{\pgfqpoint{3.075002in}{2.278377in}}%
\pgfpathmoveto{\pgfqpoint{3.070461in}{2.281326in}}%
\pgfpathlineto{\pgfqpoint{3.070461in}{2.281326in}}%
\pgfpathlineto{\pgfqpoint{3.070461in}{2.284275in}}%
\pgfpathlineto{\pgfqpoint{3.075002in}{2.284275in}}%
\pgfpathlineto{\pgfqpoint{3.075002in}{2.281326in}}%
\pgfpathmoveto{\pgfqpoint{3.070461in}{2.284275in}}%
\pgfpathlineto{\pgfqpoint{3.070461in}{2.284275in}}%
\pgfpathlineto{\pgfqpoint{3.070461in}{2.287224in}}%
\pgfpathlineto{\pgfqpoint{3.075002in}{2.287224in}}%
\pgfpathlineto{\pgfqpoint{3.075002in}{2.284275in}}%
\pgfpathmoveto{\pgfqpoint{3.070461in}{2.287224in}}%
\pgfpathlineto{\pgfqpoint{3.070461in}{2.287224in}}%
\pgfpathlineto{\pgfqpoint{3.070461in}{2.290173in}}%
\pgfpathlineto{\pgfqpoint{3.075002in}{2.290173in}}%
\pgfpathlineto{\pgfqpoint{3.075002in}{2.287224in}}%
\pgfpathmoveto{\pgfqpoint{3.070461in}{2.290173in}}%
\pgfpathlineto{\pgfqpoint{3.070461in}{2.290173in}}%
\pgfpathlineto{\pgfqpoint{3.070461in}{2.293122in}}%
\pgfpathlineto{\pgfqpoint{3.075002in}{2.293122in}}%
\pgfpathlineto{\pgfqpoint{3.075002in}{2.290173in}}%
\pgfpathmoveto{\pgfqpoint{3.070461in}{2.293122in}}%
\pgfpathlineto{\pgfqpoint{3.070461in}{2.293122in}}%
\pgfpathlineto{\pgfqpoint{3.070461in}{2.296072in}}%
\pgfpathlineto{\pgfqpoint{3.075002in}{2.296072in}}%
\pgfpathlineto{\pgfqpoint{3.075002in}{2.293122in}}%
\pgfpathmoveto{\pgfqpoint{3.070461in}{2.296072in}}%
\pgfpathlineto{\pgfqpoint{3.070461in}{2.296072in}}%
\pgfpathlineto{\pgfqpoint{3.070461in}{2.299021in}}%
\pgfpathlineto{\pgfqpoint{3.075002in}{2.299021in}}%
\pgfpathlineto{\pgfqpoint{3.075002in}{2.296072in}}%
\pgfpathmoveto{\pgfqpoint{3.070461in}{2.299021in}}%
\pgfpathlineto{\pgfqpoint{3.070461in}{2.299021in}}%
\pgfpathlineto{\pgfqpoint{3.070461in}{2.301970in}}%
\pgfpathlineto{\pgfqpoint{3.075002in}{2.301970in}}%
\pgfpathlineto{\pgfqpoint{3.075002in}{2.299021in}}%
\pgfpathmoveto{\pgfqpoint{3.070461in}{2.301970in}}%
\pgfpathlineto{\pgfqpoint{3.070461in}{2.301970in}}%
\pgfpathlineto{\pgfqpoint{3.070461in}{2.304920in}}%
\pgfpathlineto{\pgfqpoint{3.075002in}{2.304920in}}%
\pgfpathlineto{\pgfqpoint{3.075002in}{2.301970in}}%
\pgfpathmoveto{\pgfqpoint{3.070461in}{2.304920in}}%
\pgfpathlineto{\pgfqpoint{3.070461in}{2.304920in}}%
\pgfpathlineto{\pgfqpoint{3.070461in}{2.307869in}}%
\pgfpathlineto{\pgfqpoint{3.075002in}{2.307869in}}%
\pgfpathlineto{\pgfqpoint{3.075002in}{2.304920in}}%
\pgfpathmoveto{\pgfqpoint{3.070461in}{2.307869in}}%
\pgfpathlineto{\pgfqpoint{3.070461in}{2.307869in}}%
\pgfpathlineto{\pgfqpoint{3.070461in}{2.310818in}}%
\pgfpathlineto{\pgfqpoint{3.075002in}{2.310818in}}%
\pgfpathlineto{\pgfqpoint{3.075002in}{2.307869in}}%
\pgfpathmoveto{\pgfqpoint{3.070461in}{2.310818in}}%
\pgfpathlineto{\pgfqpoint{3.070461in}{2.310818in}}%
\pgfpathlineto{\pgfqpoint{3.070461in}{2.313768in}}%
\pgfpathlineto{\pgfqpoint{3.075002in}{2.313768in}}%
\pgfpathlineto{\pgfqpoint{3.075002in}{2.310818in}}%
\pgfpathmoveto{\pgfqpoint{3.070461in}{2.313768in}}%
\pgfpathlineto{\pgfqpoint{3.070461in}{2.313768in}}%
\pgfpathlineto{\pgfqpoint{3.070461in}{2.316717in}}%
\pgfpathlineto{\pgfqpoint{3.075002in}{2.316717in}}%
\pgfpathlineto{\pgfqpoint{3.075002in}{2.313768in}}%
\pgfpathmoveto{\pgfqpoint{3.070461in}{2.316717in}}%
\pgfpathlineto{\pgfqpoint{3.070461in}{2.316717in}}%
\pgfpathlineto{\pgfqpoint{3.070461in}{2.319666in}}%
\pgfpathlineto{\pgfqpoint{3.075002in}{2.319666in}}%
\pgfpathlineto{\pgfqpoint{3.075002in}{2.316717in}}%
\pgfpathmoveto{\pgfqpoint{3.070461in}{2.319666in}}%
\pgfpathlineto{\pgfqpoint{3.070461in}{2.319666in}}%
\pgfpathlineto{\pgfqpoint{3.070461in}{2.322616in}}%
\pgfpathlineto{\pgfqpoint{3.075002in}{2.322616in}}%
\pgfpathlineto{\pgfqpoint{3.075002in}{2.319666in}}%
\pgfpathmoveto{\pgfqpoint{3.070461in}{2.322616in}}%
\pgfpathlineto{\pgfqpoint{3.070461in}{2.322616in}}%
\pgfpathlineto{\pgfqpoint{3.070461in}{2.325565in}}%
\pgfpathlineto{\pgfqpoint{3.075002in}{2.325565in}}%
\pgfpathlineto{\pgfqpoint{3.075002in}{2.322616in}}%
\pgfpathmoveto{\pgfqpoint{3.070461in}{2.325565in}}%
\pgfpathlineto{\pgfqpoint{3.070461in}{2.325565in}}%
\pgfpathlineto{\pgfqpoint{3.070461in}{2.328514in}}%
\pgfpathlineto{\pgfqpoint{3.075002in}{2.328514in}}%
\pgfpathlineto{\pgfqpoint{3.075002in}{2.325565in}}%
\pgfpathmoveto{\pgfqpoint{3.070461in}{2.328514in}}%
\pgfpathlineto{\pgfqpoint{3.070461in}{2.328514in}}%
\pgfpathlineto{\pgfqpoint{3.070461in}{2.331464in}}%
\pgfpathlineto{\pgfqpoint{3.075002in}{2.331464in}}%
\pgfpathlineto{\pgfqpoint{3.075002in}{2.328514in}}%
\pgfpathmoveto{\pgfqpoint{3.070461in}{2.331464in}}%
\pgfpathlineto{\pgfqpoint{3.070461in}{2.331464in}}%
\pgfpathlineto{\pgfqpoint{3.070461in}{2.334413in}}%
\pgfpathlineto{\pgfqpoint{3.075002in}{2.334413in}}%
\pgfpathlineto{\pgfqpoint{3.075002in}{2.331464in}}%
\pgfpathmoveto{\pgfqpoint{3.070461in}{2.334413in}}%
\pgfpathlineto{\pgfqpoint{3.070461in}{2.334413in}}%
\pgfpathlineto{\pgfqpoint{3.070461in}{2.337362in}}%
\pgfpathlineto{\pgfqpoint{3.075002in}{2.337362in}}%
\pgfpathlineto{\pgfqpoint{3.075002in}{2.334413in}}%
\pgfpathmoveto{\pgfqpoint{3.070461in}{2.337362in}}%
\pgfpathlineto{\pgfqpoint{3.070461in}{2.337362in}}%
\pgfpathlineto{\pgfqpoint{3.070461in}{2.340311in}}%
\pgfpathlineto{\pgfqpoint{3.075002in}{2.340311in}}%
\pgfpathlineto{\pgfqpoint{3.075002in}{2.337362in}}%
\pgfpathmoveto{\pgfqpoint{3.070461in}{2.340311in}}%
\pgfpathlineto{\pgfqpoint{3.070461in}{2.340311in}}%
\pgfpathlineto{\pgfqpoint{3.070461in}{2.343261in}}%
\pgfpathlineto{\pgfqpoint{3.075002in}{2.343261in}}%
\pgfpathlineto{\pgfqpoint{3.075002in}{2.340311in}}%
\pgfpathmoveto{\pgfqpoint{3.070461in}{2.343261in}}%
\pgfpathlineto{\pgfqpoint{3.070461in}{2.343261in}}%
\pgfpathlineto{\pgfqpoint{3.070461in}{2.346210in}}%
\pgfpathlineto{\pgfqpoint{3.075002in}{2.346210in}}%
\pgfpathlineto{\pgfqpoint{3.075002in}{2.343261in}}%
\pgfpathmoveto{\pgfqpoint{3.070461in}{2.346210in}}%
\pgfpathlineto{\pgfqpoint{3.070461in}{2.346210in}}%
\pgfpathlineto{\pgfqpoint{3.070461in}{2.349159in}}%
\pgfpathlineto{\pgfqpoint{3.075002in}{2.349159in}}%
\pgfpathlineto{\pgfqpoint{3.075002in}{2.346210in}}%
\pgfpathmoveto{\pgfqpoint{3.070461in}{2.349159in}}%
\pgfpathlineto{\pgfqpoint{3.070461in}{2.349159in}}%
\pgfpathlineto{\pgfqpoint{3.070461in}{2.352109in}}%
\pgfpathlineto{\pgfqpoint{3.075002in}{2.352109in}}%
\pgfpathlineto{\pgfqpoint{3.075002in}{2.349159in}}%
\pgfpathmoveto{\pgfqpoint{3.070461in}{2.352109in}}%
\pgfpathlineto{\pgfqpoint{3.070461in}{2.352109in}}%
\pgfpathlineto{\pgfqpoint{3.070461in}{2.355058in}}%
\pgfpathlineto{\pgfqpoint{3.075002in}{2.355058in}}%
\pgfpathlineto{\pgfqpoint{3.075002in}{2.352109in}}%
\pgfpathmoveto{\pgfqpoint{3.070461in}{2.355058in}}%
\pgfpathlineto{\pgfqpoint{3.070461in}{2.355058in}}%
\pgfpathlineto{\pgfqpoint{3.070461in}{2.358007in}}%
\pgfpathlineto{\pgfqpoint{3.075002in}{2.358007in}}%
\pgfpathlineto{\pgfqpoint{3.075002in}{2.355058in}}%
\pgfpathmoveto{\pgfqpoint{3.070461in}{2.358007in}}%
\pgfpathlineto{\pgfqpoint{3.070461in}{2.358007in}}%
\pgfpathlineto{\pgfqpoint{3.070461in}{2.360957in}}%
\pgfpathlineto{\pgfqpoint{3.075002in}{2.360957in}}%
\pgfpathlineto{\pgfqpoint{3.075002in}{2.358007in}}%
\pgfpathmoveto{\pgfqpoint{3.070461in}{2.360957in}}%
\pgfpathlineto{\pgfqpoint{3.070461in}{2.360957in}}%
\pgfpathlineto{\pgfqpoint{3.070461in}{2.363906in}}%
\pgfpathlineto{\pgfqpoint{3.075002in}{2.363906in}}%
\pgfpathlineto{\pgfqpoint{3.075002in}{2.360957in}}%
\pgfpathmoveto{\pgfqpoint{3.070461in}{2.363906in}}%
\pgfpathlineto{\pgfqpoint{3.070461in}{2.363906in}}%
\pgfpathlineto{\pgfqpoint{3.070461in}{2.366855in}}%
\pgfpathlineto{\pgfqpoint{3.075002in}{2.366855in}}%
\pgfpathlineto{\pgfqpoint{3.075002in}{2.363906in}}%
\pgfpathmoveto{\pgfqpoint{3.070461in}{2.366855in}}%
\pgfpathlineto{\pgfqpoint{3.070461in}{2.366855in}}%
\pgfpathlineto{\pgfqpoint{3.070461in}{2.369805in}}%
\pgfpathlineto{\pgfqpoint{3.075002in}{2.369805in}}%
\pgfpathlineto{\pgfqpoint{3.075002in}{2.366855in}}%
\pgfpathmoveto{\pgfqpoint{3.070461in}{2.369805in}}%
\pgfpathlineto{\pgfqpoint{3.070461in}{2.369805in}}%
\pgfpathlineto{\pgfqpoint{3.070461in}{2.372754in}}%
\pgfpathlineto{\pgfqpoint{3.075002in}{2.372754in}}%
\pgfpathlineto{\pgfqpoint{3.075002in}{2.369805in}}%
\pgfpathmoveto{\pgfqpoint{3.070461in}{2.372754in}}%
\pgfpathlineto{\pgfqpoint{3.070461in}{2.372754in}}%
\pgfpathlineto{\pgfqpoint{3.070461in}{2.375703in}}%
\pgfpathlineto{\pgfqpoint{3.075002in}{2.375703in}}%
\pgfpathlineto{\pgfqpoint{3.075002in}{2.372754in}}%
\pgfpathmoveto{\pgfqpoint{3.070461in}{2.375703in}}%
\pgfpathlineto{\pgfqpoint{3.070461in}{2.375703in}}%
\pgfpathlineto{\pgfqpoint{3.070461in}{2.378653in}}%
\pgfpathlineto{\pgfqpoint{3.075002in}{2.378653in}}%
\pgfpathlineto{\pgfqpoint{3.075002in}{2.375703in}}%
\pgfpathmoveto{\pgfqpoint{3.070461in}{2.378653in}}%
\pgfpathlineto{\pgfqpoint{3.070461in}{2.378653in}}%
\pgfpathlineto{\pgfqpoint{3.070461in}{2.381602in}}%
\pgfpathlineto{\pgfqpoint{3.075002in}{2.381602in}}%
\pgfpathlineto{\pgfqpoint{3.075002in}{2.378653in}}%
\pgfpathmoveto{\pgfqpoint{3.070461in}{2.381602in}}%
\pgfpathlineto{\pgfqpoint{3.070461in}{2.381602in}}%
\pgfpathlineto{\pgfqpoint{3.070461in}{2.384551in}}%
\pgfpathlineto{\pgfqpoint{3.075002in}{2.384551in}}%
\pgfpathlineto{\pgfqpoint{3.075002in}{2.381602in}}%
\pgfpathmoveto{\pgfqpoint{3.070461in}{2.384551in}}%
\pgfpathlineto{\pgfqpoint{3.070461in}{2.384551in}}%
\pgfpathlineto{\pgfqpoint{3.070461in}{2.387500in}}%
\pgfpathlineto{\pgfqpoint{3.075002in}{2.387500in}}%
\pgfpathlineto{\pgfqpoint{3.075002in}{2.384551in}}%
\pgfpathmoveto{\pgfqpoint{3.070461in}{2.387500in}}%
\pgfpathlineto{\pgfqpoint{3.070461in}{2.387500in}}%
\pgfpathlineto{\pgfqpoint{3.070461in}{2.390450in}}%
\pgfpathlineto{\pgfqpoint{3.075002in}{2.390450in}}%
\pgfpathlineto{\pgfqpoint{3.075002in}{2.387500in}}%
\pgfpathmoveto{\pgfqpoint{3.070461in}{2.390450in}}%
\pgfpathlineto{\pgfqpoint{3.070461in}{2.390450in}}%
\pgfpathlineto{\pgfqpoint{3.070461in}{2.393399in}}%
\pgfpathlineto{\pgfqpoint{3.075002in}{2.393399in}}%
\pgfpathlineto{\pgfqpoint{3.075002in}{2.390450in}}%
\pgfpathmoveto{\pgfqpoint{3.070461in}{2.393399in}}%
\pgfpathlineto{\pgfqpoint{3.070461in}{2.393399in}}%
\pgfpathlineto{\pgfqpoint{3.070461in}{2.396348in}}%
\pgfpathlineto{\pgfqpoint{3.075002in}{2.396348in}}%
\pgfpathlineto{\pgfqpoint{3.075002in}{2.393399in}}%
\pgfpathmoveto{\pgfqpoint{3.070461in}{2.396348in}}%
\pgfpathlineto{\pgfqpoint{3.070461in}{2.396348in}}%
\pgfpathlineto{\pgfqpoint{3.070461in}{2.399298in}}%
\pgfpathlineto{\pgfqpoint{3.075002in}{2.399298in}}%
\pgfpathlineto{\pgfqpoint{3.075002in}{2.396348in}}%
\pgfpathmoveto{\pgfqpoint{3.070461in}{2.399298in}}%
\pgfpathlineto{\pgfqpoint{3.070461in}{2.399298in}}%
\pgfpathlineto{\pgfqpoint{3.070461in}{2.402247in}}%
\pgfpathlineto{\pgfqpoint{3.075002in}{2.402247in}}%
\pgfpathlineto{\pgfqpoint{3.075002in}{2.399298in}}%
\pgfpathmoveto{\pgfqpoint{3.070461in}{2.402247in}}%
\pgfpathlineto{\pgfqpoint{3.070461in}{2.402247in}}%
\pgfpathlineto{\pgfqpoint{3.070461in}{2.405196in}}%
\pgfpathlineto{\pgfqpoint{3.075002in}{2.405196in}}%
\pgfpathlineto{\pgfqpoint{3.075002in}{2.402247in}}%
\pgfpathmoveto{\pgfqpoint{3.070461in}{2.405196in}}%
\pgfpathlineto{\pgfqpoint{3.070461in}{2.405196in}}%
\pgfpathlineto{\pgfqpoint{3.070461in}{2.408145in}}%
\pgfpathlineto{\pgfqpoint{3.075002in}{2.408145in}}%
\pgfpathlineto{\pgfqpoint{3.075002in}{2.405196in}}%
\pgfpathmoveto{\pgfqpoint{3.070461in}{2.408145in}}%
\pgfpathlineto{\pgfqpoint{3.070461in}{2.408145in}}%
\pgfpathlineto{\pgfqpoint{3.070461in}{2.411095in}}%
\pgfpathlineto{\pgfqpoint{3.075002in}{2.411095in}}%
\pgfpathlineto{\pgfqpoint{3.075002in}{2.408145in}}%
\pgfpathmoveto{\pgfqpoint{3.070461in}{2.411095in}}%
\pgfpathlineto{\pgfqpoint{3.070461in}{2.411095in}}%
\pgfpathlineto{\pgfqpoint{3.070461in}{2.414044in}}%
\pgfpathlineto{\pgfqpoint{3.075002in}{2.414044in}}%
\pgfpathlineto{\pgfqpoint{3.075002in}{2.411095in}}%
\pgfpathmoveto{\pgfqpoint{3.070461in}{2.414044in}}%
\pgfpathlineto{\pgfqpoint{3.070461in}{2.414044in}}%
\pgfpathlineto{\pgfqpoint{3.070461in}{2.416993in}}%
\pgfpathlineto{\pgfqpoint{3.075002in}{2.416993in}}%
\pgfpathlineto{\pgfqpoint{3.075002in}{2.414044in}}%
\pgfpathmoveto{\pgfqpoint{3.070461in}{2.416993in}}%
\pgfpathlineto{\pgfqpoint{3.070461in}{2.416993in}}%
\pgfpathlineto{\pgfqpoint{3.070461in}{2.419943in}}%
\pgfpathlineto{\pgfqpoint{3.075002in}{2.419943in}}%
\pgfpathlineto{\pgfqpoint{3.075002in}{2.416993in}}%
\pgfpathmoveto{\pgfqpoint{3.070461in}{2.419943in}}%
\pgfpathlineto{\pgfqpoint{3.070461in}{2.419943in}}%
\pgfpathlineto{\pgfqpoint{3.070461in}{2.422892in}}%
\pgfpathlineto{\pgfqpoint{3.075002in}{2.422892in}}%
\pgfpathlineto{\pgfqpoint{3.075002in}{2.419943in}}%
\pgfpathmoveto{\pgfqpoint{3.070461in}{2.422892in}}%
\pgfpathlineto{\pgfqpoint{3.070461in}{2.422892in}}%
\pgfpathlineto{\pgfqpoint{3.070461in}{2.425841in}}%
\pgfpathlineto{\pgfqpoint{3.075002in}{2.425841in}}%
\pgfpathlineto{\pgfqpoint{3.075002in}{2.422892in}}%
\pgfpathmoveto{\pgfqpoint{3.070461in}{2.425841in}}%
\pgfpathlineto{\pgfqpoint{3.070461in}{2.425841in}}%
\pgfpathlineto{\pgfqpoint{3.070461in}{2.428790in}}%
\pgfpathlineto{\pgfqpoint{3.075002in}{2.428790in}}%
\pgfpathlineto{\pgfqpoint{3.075002in}{2.425841in}}%
\pgfpathmoveto{\pgfqpoint{3.070461in}{2.428790in}}%
\pgfpathlineto{\pgfqpoint{3.070461in}{2.428790in}}%
\pgfpathlineto{\pgfqpoint{3.070461in}{2.431740in}}%
\pgfpathlineto{\pgfqpoint{3.075002in}{2.431740in}}%
\pgfpathlineto{\pgfqpoint{3.075002in}{2.428790in}}%
\pgfpathmoveto{\pgfqpoint{3.070461in}{2.431740in}}%
\pgfpathlineto{\pgfqpoint{3.070461in}{2.431740in}}%
\pgfpathlineto{\pgfqpoint{3.070461in}{2.434689in}}%
\pgfpathlineto{\pgfqpoint{3.075002in}{2.434689in}}%
\pgfpathlineto{\pgfqpoint{3.075002in}{2.431740in}}%
\pgfpathmoveto{\pgfqpoint{3.070461in}{2.434689in}}%
\pgfpathlineto{\pgfqpoint{3.070461in}{2.434689in}}%
\pgfpathlineto{\pgfqpoint{3.070461in}{2.437638in}}%
\pgfpathlineto{\pgfqpoint{3.075002in}{2.437638in}}%
\pgfpathlineto{\pgfqpoint{3.075002in}{2.434689in}}%
\pgfpathmoveto{\pgfqpoint{3.070461in}{2.437638in}}%
\pgfpathlineto{\pgfqpoint{3.070461in}{2.437638in}}%
\pgfpathlineto{\pgfqpoint{3.070461in}{2.440588in}}%
\pgfpathlineto{\pgfqpoint{3.075002in}{2.440588in}}%
\pgfpathlineto{\pgfqpoint{3.075002in}{2.437638in}}%
\pgfpathmoveto{\pgfqpoint{3.070461in}{2.440588in}}%
\pgfpathlineto{\pgfqpoint{3.070461in}{2.440588in}}%
\pgfpathlineto{\pgfqpoint{3.070461in}{2.443537in}}%
\pgfpathlineto{\pgfqpoint{3.075002in}{2.443537in}}%
\pgfpathlineto{\pgfqpoint{3.075002in}{2.440588in}}%
\pgfpathmoveto{\pgfqpoint{3.070461in}{2.443537in}}%
\pgfpathlineto{\pgfqpoint{3.070461in}{2.443537in}}%
\pgfpathlineto{\pgfqpoint{3.070461in}{2.446486in}}%
\pgfpathlineto{\pgfqpoint{3.075002in}{2.446486in}}%
\pgfpathlineto{\pgfqpoint{3.075002in}{2.443537in}}%
\pgfpathmoveto{\pgfqpoint{3.070461in}{2.446486in}}%
\pgfpathlineto{\pgfqpoint{3.070461in}{2.446486in}}%
\pgfpathlineto{\pgfqpoint{3.070461in}{2.449435in}}%
\pgfpathlineto{\pgfqpoint{3.075002in}{2.449435in}}%
\pgfpathlineto{\pgfqpoint{3.075002in}{2.446486in}}%
\pgfpathmoveto{\pgfqpoint{3.070461in}{2.449435in}}%
\pgfpathlineto{\pgfqpoint{3.070461in}{2.449435in}}%
\pgfpathlineto{\pgfqpoint{3.070461in}{2.452385in}}%
\pgfpathlineto{\pgfqpoint{3.075002in}{2.452385in}}%
\pgfpathlineto{\pgfqpoint{3.075002in}{2.449435in}}%
\pgfpathmoveto{\pgfqpoint{3.070461in}{2.452385in}}%
\pgfpathlineto{\pgfqpoint{3.070461in}{2.452385in}}%
\pgfpathlineto{\pgfqpoint{3.070461in}{2.455334in}}%
\pgfpathlineto{\pgfqpoint{3.075002in}{2.455334in}}%
\pgfpathlineto{\pgfqpoint{3.075002in}{2.452385in}}%
\pgfpathmoveto{\pgfqpoint{3.070461in}{2.455334in}}%
\pgfpathlineto{\pgfqpoint{3.070461in}{2.455334in}}%
\pgfpathlineto{\pgfqpoint{3.070461in}{2.458283in}}%
\pgfpathlineto{\pgfqpoint{3.075002in}{2.458283in}}%
\pgfpathlineto{\pgfqpoint{3.075002in}{2.455334in}}%
\pgfpathmoveto{\pgfqpoint{3.070461in}{2.458283in}}%
\pgfpathlineto{\pgfqpoint{3.070461in}{2.458283in}}%
\pgfpathlineto{\pgfqpoint{3.070461in}{2.461233in}}%
\pgfpathlineto{\pgfqpoint{3.075002in}{2.461233in}}%
\pgfpathlineto{\pgfqpoint{3.075002in}{2.458283in}}%
\pgfpathmoveto{\pgfqpoint{3.070461in}{2.461233in}}%
\pgfpathlineto{\pgfqpoint{3.070461in}{2.461233in}}%
\pgfpathlineto{\pgfqpoint{3.070461in}{2.464182in}}%
\pgfpathlineto{\pgfqpoint{3.075002in}{2.464182in}}%
\pgfpathlineto{\pgfqpoint{3.075002in}{2.461233in}}%
\pgfpathmoveto{\pgfqpoint{3.070461in}{2.464182in}}%
\pgfpathlineto{\pgfqpoint{3.070461in}{2.464182in}}%
\pgfpathlineto{\pgfqpoint{3.070461in}{2.467131in}}%
\pgfpathlineto{\pgfqpoint{3.075002in}{2.467131in}}%
\pgfpathlineto{\pgfqpoint{3.075002in}{2.464182in}}%
\pgfpathmoveto{\pgfqpoint{3.070461in}{2.467131in}}%
\pgfpathlineto{\pgfqpoint{3.070461in}{2.467131in}}%
\pgfpathlineto{\pgfqpoint{3.070461in}{2.470080in}}%
\pgfpathlineto{\pgfqpoint{3.075002in}{2.470080in}}%
\pgfpathlineto{\pgfqpoint{3.075002in}{2.467131in}}%
\pgfpathmoveto{\pgfqpoint{3.070461in}{2.470080in}}%
\pgfpathlineto{\pgfqpoint{3.070461in}{2.470080in}}%
\pgfpathlineto{\pgfqpoint{3.070461in}{2.473030in}}%
\pgfpathlineto{\pgfqpoint{3.075002in}{2.473030in}}%
\pgfpathlineto{\pgfqpoint{3.075002in}{2.470080in}}%
\pgfpathmoveto{\pgfqpoint{3.070461in}{2.473030in}}%
\pgfpathlineto{\pgfqpoint{3.070461in}{2.473030in}}%
\pgfpathlineto{\pgfqpoint{3.070461in}{2.475979in}}%
\pgfpathlineto{\pgfqpoint{3.075002in}{2.475979in}}%
\pgfpathlineto{\pgfqpoint{3.075002in}{2.473030in}}%
\pgfpathmoveto{\pgfqpoint{3.070461in}{2.475979in}}%
\pgfpathlineto{\pgfqpoint{3.070461in}{2.475979in}}%
\pgfpathlineto{\pgfqpoint{3.070461in}{2.478928in}}%
\pgfpathlineto{\pgfqpoint{3.075002in}{2.478928in}}%
\pgfpathlineto{\pgfqpoint{3.075002in}{2.475979in}}%
\pgfpathmoveto{\pgfqpoint{3.070461in}{2.478928in}}%
\pgfpathlineto{\pgfqpoint{3.070461in}{2.478928in}}%
\pgfpathlineto{\pgfqpoint{3.070461in}{2.481877in}}%
\pgfpathlineto{\pgfqpoint{3.075002in}{2.481877in}}%
\pgfpathlineto{\pgfqpoint{3.075002in}{2.478928in}}%
\pgfpathmoveto{\pgfqpoint{3.070461in}{2.481877in}}%
\pgfpathlineto{\pgfqpoint{3.070461in}{2.481877in}}%
\pgfpathlineto{\pgfqpoint{3.070461in}{2.484827in}}%
\pgfpathlineto{\pgfqpoint{3.075002in}{2.484827in}}%
\pgfpathlineto{\pgfqpoint{3.075002in}{2.481877in}}%
\pgfpathmoveto{\pgfqpoint{3.070461in}{2.484827in}}%
\pgfpathlineto{\pgfqpoint{3.070461in}{2.484827in}}%
\pgfpathlineto{\pgfqpoint{3.070461in}{2.487776in}}%
\pgfpathlineto{\pgfqpoint{3.075002in}{2.487776in}}%
\pgfpathlineto{\pgfqpoint{3.075002in}{2.484827in}}%
\pgfpathmoveto{\pgfqpoint{3.070461in}{2.487776in}}%
\pgfpathlineto{\pgfqpoint{3.070461in}{2.487776in}}%
\pgfpathlineto{\pgfqpoint{3.070461in}{2.490725in}}%
\pgfpathlineto{\pgfqpoint{3.075002in}{2.490725in}}%
\pgfpathlineto{\pgfqpoint{3.075002in}{2.487776in}}%
\pgfpathmoveto{\pgfqpoint{3.070461in}{2.490725in}}%
\pgfpathlineto{\pgfqpoint{3.070461in}{2.490725in}}%
\pgfpathlineto{\pgfqpoint{3.070461in}{2.493674in}}%
\pgfpathlineto{\pgfqpoint{3.075002in}{2.493674in}}%
\pgfpathlineto{\pgfqpoint{3.075002in}{2.490725in}}%
\pgfpathmoveto{\pgfqpoint{3.070461in}{2.493674in}}%
\pgfpathlineto{\pgfqpoint{3.070461in}{2.493674in}}%
\pgfpathlineto{\pgfqpoint{3.070461in}{2.496623in}}%
\pgfpathlineto{\pgfqpoint{3.075002in}{2.496623in}}%
\pgfpathlineto{\pgfqpoint{3.075002in}{2.493674in}}%
\pgfpathmoveto{\pgfqpoint{3.070461in}{2.496623in}}%
\pgfpathlineto{\pgfqpoint{3.070461in}{2.496623in}}%
\pgfpathlineto{\pgfqpoint{3.070461in}{2.499572in}}%
\pgfpathlineto{\pgfqpoint{3.075002in}{2.499572in}}%
\pgfpathlineto{\pgfqpoint{3.075002in}{2.496623in}}%
\pgfpathmoveto{\pgfqpoint{3.070461in}{2.499572in}}%
\pgfpathlineto{\pgfqpoint{3.070461in}{2.499572in}}%
\pgfpathlineto{\pgfqpoint{3.070461in}{2.502521in}}%
\pgfpathlineto{\pgfqpoint{3.075002in}{2.502521in}}%
\pgfpathlineto{\pgfqpoint{3.075002in}{2.499572in}}%
\pgfpathmoveto{\pgfqpoint{3.070461in}{2.502521in}}%
\pgfpathlineto{\pgfqpoint{3.070461in}{2.502521in}}%
\pgfpathlineto{\pgfqpoint{3.070461in}{2.505471in}}%
\pgfpathlineto{\pgfqpoint{3.075002in}{2.505471in}}%
\pgfpathlineto{\pgfqpoint{3.075002in}{2.502521in}}%
\pgfpathmoveto{\pgfqpoint{3.070461in}{2.505471in}}%
\pgfpathlineto{\pgfqpoint{3.070461in}{2.505471in}}%
\pgfpathlineto{\pgfqpoint{3.070461in}{2.508420in}}%
\pgfpathlineto{\pgfqpoint{3.075002in}{2.508420in}}%
\pgfpathlineto{\pgfqpoint{3.075002in}{2.505471in}}%
\pgfpathmoveto{\pgfqpoint{3.070461in}{2.508420in}}%
\pgfpathlineto{\pgfqpoint{3.070461in}{2.508420in}}%
\pgfpathlineto{\pgfqpoint{3.070461in}{2.511369in}}%
\pgfpathlineto{\pgfqpoint{3.075002in}{2.511369in}}%
\pgfpathlineto{\pgfqpoint{3.075002in}{2.508420in}}%
\pgfpathmoveto{\pgfqpoint{3.070461in}{2.511369in}}%
\pgfpathlineto{\pgfqpoint{3.070461in}{2.511369in}}%
\pgfpathlineto{\pgfqpoint{3.070461in}{2.514318in}}%
\pgfpathlineto{\pgfqpoint{3.075002in}{2.514318in}}%
\pgfpathlineto{\pgfqpoint{3.075002in}{2.511369in}}%
\pgfpathmoveto{\pgfqpoint{3.070461in}{2.514318in}}%
\pgfpathlineto{\pgfqpoint{3.070461in}{2.514318in}}%
\pgfpathlineto{\pgfqpoint{3.070461in}{2.517267in}}%
\pgfpathlineto{\pgfqpoint{3.075002in}{2.517267in}}%
\pgfpathlineto{\pgfqpoint{3.075002in}{2.514318in}}%
\pgfpathmoveto{\pgfqpoint{3.070461in}{2.517267in}}%
\pgfpathlineto{\pgfqpoint{3.070461in}{2.517267in}}%
\pgfpathlineto{\pgfqpoint{3.070461in}{2.520216in}}%
\pgfpathlineto{\pgfqpoint{3.075002in}{2.520216in}}%
\pgfpathlineto{\pgfqpoint{3.075002in}{2.517267in}}%
\pgfpathmoveto{\pgfqpoint{3.070461in}{2.520216in}}%
\pgfpathlineto{\pgfqpoint{3.070461in}{2.520216in}}%
\pgfpathlineto{\pgfqpoint{3.070461in}{2.523166in}}%
\pgfpathlineto{\pgfqpoint{3.075002in}{2.523166in}}%
\pgfpathlineto{\pgfqpoint{3.075002in}{2.520216in}}%
\pgfpathmoveto{\pgfqpoint{3.070461in}{2.523166in}}%
\pgfpathlineto{\pgfqpoint{3.070461in}{2.523166in}}%
\pgfpathlineto{\pgfqpoint{3.070461in}{2.526115in}}%
\pgfpathlineto{\pgfqpoint{3.075002in}{2.526115in}}%
\pgfpathlineto{\pgfqpoint{3.075002in}{2.523166in}}%
\pgfpathmoveto{\pgfqpoint{3.070461in}{2.526115in}}%
\pgfpathlineto{\pgfqpoint{3.070461in}{2.526115in}}%
\pgfpathlineto{\pgfqpoint{3.070461in}{2.529064in}}%
\pgfpathlineto{\pgfqpoint{3.075002in}{2.529064in}}%
\pgfpathlineto{\pgfqpoint{3.075002in}{2.526115in}}%
\pgfpathmoveto{\pgfqpoint{3.070461in}{2.529064in}}%
\pgfpathlineto{\pgfqpoint{3.070461in}{2.529064in}}%
\pgfpathlineto{\pgfqpoint{3.070461in}{2.532013in}}%
\pgfpathlineto{\pgfqpoint{3.075002in}{2.532013in}}%
\pgfpathlineto{\pgfqpoint{3.075002in}{2.529064in}}%
\pgfpathmoveto{\pgfqpoint{3.070461in}{2.532013in}}%
\pgfpathlineto{\pgfqpoint{3.070461in}{2.532013in}}%
\pgfpathlineto{\pgfqpoint{3.070461in}{2.534962in}}%
\pgfpathlineto{\pgfqpoint{3.075002in}{2.534962in}}%
\pgfpathlineto{\pgfqpoint{3.075002in}{2.532013in}}%
\pgfpathmoveto{\pgfqpoint{3.070461in}{2.534962in}}%
\pgfpathlineto{\pgfqpoint{3.070461in}{2.534962in}}%
\pgfpathlineto{\pgfqpoint{3.070461in}{2.537911in}}%
\pgfpathlineto{\pgfqpoint{3.075002in}{2.537911in}}%
\pgfpathlineto{\pgfqpoint{3.075002in}{2.534962in}}%
\pgfpathmoveto{\pgfqpoint{3.070461in}{2.537911in}}%
\pgfpathlineto{\pgfqpoint{3.070461in}{2.537911in}}%
\pgfpathlineto{\pgfqpoint{3.070461in}{2.540860in}}%
\pgfpathlineto{\pgfqpoint{3.075002in}{2.540860in}}%
\pgfpathlineto{\pgfqpoint{3.075002in}{2.537911in}}%
\pgfpathmoveto{\pgfqpoint{3.070461in}{2.540860in}}%
\pgfpathlineto{\pgfqpoint{3.070461in}{2.540860in}}%
\pgfpathlineto{\pgfqpoint{3.070461in}{2.543810in}}%
\pgfpathlineto{\pgfqpoint{3.075002in}{2.543810in}}%
\pgfpathlineto{\pgfqpoint{3.075002in}{2.540860in}}%
\pgfpathmoveto{\pgfqpoint{3.070461in}{2.543810in}}%
\pgfpathlineto{\pgfqpoint{3.070461in}{2.543810in}}%
\pgfpathlineto{\pgfqpoint{3.070461in}{2.546759in}}%
\pgfpathlineto{\pgfqpoint{3.075002in}{2.546759in}}%
\pgfpathlineto{\pgfqpoint{3.075002in}{2.543810in}}%
\pgfpathmoveto{\pgfqpoint{3.070461in}{2.546759in}}%
\pgfpathlineto{\pgfqpoint{3.070461in}{2.546759in}}%
\pgfpathlineto{\pgfqpoint{3.070461in}{2.549708in}}%
\pgfpathlineto{\pgfqpoint{3.075002in}{2.549708in}}%
\pgfpathlineto{\pgfqpoint{3.075002in}{2.546759in}}%
\pgfpathmoveto{\pgfqpoint{3.070461in}{2.549708in}}%
\pgfpathlineto{\pgfqpoint{3.070461in}{2.549708in}}%
\pgfpathlineto{\pgfqpoint{3.070461in}{2.552657in}}%
\pgfpathlineto{\pgfqpoint{3.075002in}{2.552657in}}%
\pgfpathlineto{\pgfqpoint{3.075002in}{2.549708in}}%
\pgfpathmoveto{\pgfqpoint{3.070461in}{2.552657in}}%
\pgfpathlineto{\pgfqpoint{3.070461in}{2.552657in}}%
\pgfpathlineto{\pgfqpoint{3.070461in}{2.555606in}}%
\pgfpathlineto{\pgfqpoint{3.075002in}{2.555606in}}%
\pgfpathlineto{\pgfqpoint{3.075002in}{2.552657in}}%
\pgfpathmoveto{\pgfqpoint{3.070461in}{2.555606in}}%
\pgfpathlineto{\pgfqpoint{3.070461in}{2.555606in}}%
\pgfpathlineto{\pgfqpoint{3.070461in}{2.558555in}}%
\pgfpathlineto{\pgfqpoint{3.075002in}{2.558555in}}%
\pgfpathlineto{\pgfqpoint{3.075002in}{2.555606in}}%
\pgfpathmoveto{\pgfqpoint{3.070461in}{2.558555in}}%
\pgfpathlineto{\pgfqpoint{3.070461in}{2.558555in}}%
\pgfpathlineto{\pgfqpoint{3.070461in}{2.561504in}}%
\pgfpathlineto{\pgfqpoint{3.075002in}{2.561504in}}%
\pgfpathlineto{\pgfqpoint{3.075002in}{2.558555in}}%
\pgfpathmoveto{\pgfqpoint{3.070461in}{2.561504in}}%
\pgfpathlineto{\pgfqpoint{3.070461in}{2.561504in}}%
\pgfpathlineto{\pgfqpoint{3.070461in}{2.564454in}}%
\pgfpathlineto{\pgfqpoint{3.075002in}{2.564454in}}%
\pgfpathlineto{\pgfqpoint{3.075002in}{2.561504in}}%
\pgfpathmoveto{\pgfqpoint{3.070461in}{2.564454in}}%
\pgfpathlineto{\pgfqpoint{3.070461in}{2.564454in}}%
\pgfpathlineto{\pgfqpoint{3.070461in}{2.567403in}}%
\pgfpathlineto{\pgfqpoint{3.075002in}{2.567403in}}%
\pgfpathlineto{\pgfqpoint{3.075002in}{2.564454in}}%
\pgfpathmoveto{\pgfqpoint{3.070461in}{2.567403in}}%
\pgfpathlineto{\pgfqpoint{3.070461in}{2.567403in}}%
\pgfpathlineto{\pgfqpoint{3.070461in}{2.570352in}}%
\pgfpathlineto{\pgfqpoint{3.075002in}{2.570352in}}%
\pgfpathlineto{\pgfqpoint{3.075002in}{2.567403in}}%
\pgfpathmoveto{\pgfqpoint{3.070461in}{2.570352in}}%
\pgfpathlineto{\pgfqpoint{3.070461in}{2.570352in}}%
\pgfpathlineto{\pgfqpoint{3.070461in}{2.573301in}}%
\pgfpathlineto{\pgfqpoint{3.075002in}{2.573301in}}%
\pgfpathlineto{\pgfqpoint{3.075002in}{2.570352in}}%
\pgfpathmoveto{\pgfqpoint{3.070461in}{2.573301in}}%
\pgfpathlineto{\pgfqpoint{3.070461in}{2.573301in}}%
\pgfpathlineto{\pgfqpoint{3.070461in}{2.576250in}}%
\pgfpathlineto{\pgfqpoint{3.075002in}{2.576250in}}%
\pgfpathlineto{\pgfqpoint{3.075002in}{2.573301in}}%
\pgfpathmoveto{\pgfqpoint{3.070461in}{2.576250in}}%
\pgfpathlineto{\pgfqpoint{3.070461in}{2.576250in}}%
\pgfpathlineto{\pgfqpoint{3.070461in}{2.579199in}}%
\pgfpathlineto{\pgfqpoint{3.075002in}{2.579199in}}%
\pgfpathlineto{\pgfqpoint{3.075002in}{2.576250in}}%
\pgfpathmoveto{\pgfqpoint{3.070461in}{2.579199in}}%
\pgfpathlineto{\pgfqpoint{3.070461in}{2.579199in}}%
\pgfpathlineto{\pgfqpoint{3.070461in}{2.582149in}}%
\pgfpathlineto{\pgfqpoint{3.075002in}{2.582149in}}%
\pgfpathlineto{\pgfqpoint{3.075002in}{2.579199in}}%
\pgfpathmoveto{\pgfqpoint{3.070461in}{2.582149in}}%
\pgfpathlineto{\pgfqpoint{3.070461in}{2.582149in}}%
\pgfpathlineto{\pgfqpoint{3.070461in}{2.585098in}}%
\pgfpathlineto{\pgfqpoint{3.075002in}{2.585098in}}%
\pgfpathlineto{\pgfqpoint{3.075002in}{2.582149in}}%
\pgfpathmoveto{\pgfqpoint{3.070461in}{2.585098in}}%
\pgfpathlineto{\pgfqpoint{3.070461in}{2.585098in}}%
\pgfpathlineto{\pgfqpoint{3.070461in}{2.588047in}}%
\pgfpathlineto{\pgfqpoint{3.075002in}{2.588047in}}%
\pgfpathlineto{\pgfqpoint{3.075002in}{2.585098in}}%
\pgfpathmoveto{\pgfqpoint{3.070461in}{2.588047in}}%
\pgfpathlineto{\pgfqpoint{3.070461in}{2.588047in}}%
\pgfpathlineto{\pgfqpoint{3.070461in}{2.590996in}}%
\pgfpathlineto{\pgfqpoint{3.075002in}{2.590996in}}%
\pgfpathlineto{\pgfqpoint{3.075002in}{2.588047in}}%
\pgfpathmoveto{\pgfqpoint{3.070461in}{2.590996in}}%
\pgfpathlineto{\pgfqpoint{3.070461in}{2.590996in}}%
\pgfpathlineto{\pgfqpoint{3.070461in}{2.593946in}}%
\pgfpathlineto{\pgfqpoint{3.075002in}{2.593946in}}%
\pgfpathlineto{\pgfqpoint{3.075002in}{2.590996in}}%
\pgfpathmoveto{\pgfqpoint{3.070461in}{2.593946in}}%
\pgfpathlineto{\pgfqpoint{3.070461in}{2.593946in}}%
\pgfpathlineto{\pgfqpoint{3.070461in}{2.596895in}}%
\pgfpathlineto{\pgfqpoint{3.075002in}{2.596895in}}%
\pgfpathlineto{\pgfqpoint{3.075002in}{2.593946in}}%
\pgfpathmoveto{\pgfqpoint{3.070461in}{2.596895in}}%
\pgfpathlineto{\pgfqpoint{3.070461in}{2.596895in}}%
\pgfpathlineto{\pgfqpoint{3.070461in}{2.599844in}}%
\pgfpathlineto{\pgfqpoint{3.075002in}{2.599844in}}%
\pgfpathlineto{\pgfqpoint{3.075002in}{2.596895in}}%
\pgfpathmoveto{\pgfqpoint{3.070461in}{2.599844in}}%
\pgfpathlineto{\pgfqpoint{3.070461in}{2.599844in}}%
\pgfpathlineto{\pgfqpoint{3.070461in}{2.602794in}}%
\pgfpathlineto{\pgfqpoint{3.075002in}{2.602794in}}%
\pgfpathlineto{\pgfqpoint{3.075002in}{2.599844in}}%
\pgfpathmoveto{\pgfqpoint{3.070461in}{2.602794in}}%
\pgfpathlineto{\pgfqpoint{3.070461in}{2.602794in}}%
\pgfpathlineto{\pgfqpoint{3.070461in}{2.605743in}}%
\pgfpathlineto{\pgfqpoint{3.075002in}{2.605743in}}%
\pgfpathlineto{\pgfqpoint{3.075002in}{2.602794in}}%
\pgfpathmoveto{\pgfqpoint{3.070461in}{2.605743in}}%
\pgfpathlineto{\pgfqpoint{3.070461in}{2.605743in}}%
\pgfpathlineto{\pgfqpoint{3.070461in}{2.608692in}}%
\pgfpathlineto{\pgfqpoint{3.075002in}{2.608692in}}%
\pgfpathlineto{\pgfqpoint{3.075002in}{2.605743in}}%
\pgfpathmoveto{\pgfqpoint{3.070461in}{2.608692in}}%
\pgfpathlineto{\pgfqpoint{3.070461in}{2.608692in}}%
\pgfpathlineto{\pgfqpoint{3.070461in}{2.611641in}}%
\pgfpathlineto{\pgfqpoint{3.075002in}{2.611641in}}%
\pgfpathlineto{\pgfqpoint{3.075002in}{2.608692in}}%
\pgfpathmoveto{\pgfqpoint{3.070461in}{2.611641in}}%
\pgfpathlineto{\pgfqpoint{3.070461in}{2.611641in}}%
\pgfpathlineto{\pgfqpoint{3.070461in}{2.614591in}}%
\pgfpathlineto{\pgfqpoint{3.075002in}{2.614591in}}%
\pgfpathlineto{\pgfqpoint{3.075002in}{2.611641in}}%
\pgfpathmoveto{\pgfqpoint{3.070461in}{2.614591in}}%
\pgfpathlineto{\pgfqpoint{3.070461in}{2.614591in}}%
\pgfpathlineto{\pgfqpoint{3.070461in}{2.617540in}}%
\pgfpathlineto{\pgfqpoint{3.075002in}{2.617540in}}%
\pgfpathlineto{\pgfqpoint{3.075002in}{2.614591in}}%
\pgfpathmoveto{\pgfqpoint{3.070461in}{2.617540in}}%
\pgfpathlineto{\pgfqpoint{3.070461in}{2.617540in}}%
\pgfpathlineto{\pgfqpoint{3.070461in}{2.620489in}}%
\pgfpathlineto{\pgfqpoint{3.075002in}{2.620489in}}%
\pgfpathlineto{\pgfqpoint{3.075002in}{2.617540in}}%
\pgfpathmoveto{\pgfqpoint{3.070461in}{2.620489in}}%
\pgfpathlineto{\pgfqpoint{3.070461in}{2.620489in}}%
\pgfpathlineto{\pgfqpoint{3.070461in}{2.623438in}}%
\pgfpathlineto{\pgfqpoint{3.075002in}{2.623438in}}%
\pgfpathlineto{\pgfqpoint{3.075002in}{2.620489in}}%
\pgfpathmoveto{\pgfqpoint{3.070461in}{2.623438in}}%
\pgfpathlineto{\pgfqpoint{3.070461in}{2.623438in}}%
\pgfpathlineto{\pgfqpoint{3.070461in}{2.626388in}}%
\pgfpathlineto{\pgfqpoint{3.075002in}{2.626388in}}%
\pgfpathlineto{\pgfqpoint{3.075002in}{2.623438in}}%
\pgfpathmoveto{\pgfqpoint{3.070461in}{2.626388in}}%
\pgfpathlineto{\pgfqpoint{3.070461in}{2.626388in}}%
\pgfpathlineto{\pgfqpoint{3.070461in}{2.629337in}}%
\pgfpathlineto{\pgfqpoint{3.075002in}{2.629337in}}%
\pgfpathlineto{\pgfqpoint{3.075002in}{2.626388in}}%
\pgfpathmoveto{\pgfqpoint{3.070461in}{2.629337in}}%
\pgfpathlineto{\pgfqpoint{3.070461in}{2.629337in}}%
\pgfpathlineto{\pgfqpoint{3.070461in}{2.632286in}}%
\pgfpathlineto{\pgfqpoint{3.075002in}{2.632286in}}%
\pgfpathlineto{\pgfqpoint{3.075002in}{2.629337in}}%
\pgfpathmoveto{\pgfqpoint{3.070461in}{2.632286in}}%
\pgfpathlineto{\pgfqpoint{3.070461in}{2.632286in}}%
\pgfpathlineto{\pgfqpoint{3.070461in}{2.635236in}}%
\pgfpathlineto{\pgfqpoint{3.075002in}{2.635236in}}%
\pgfpathlineto{\pgfqpoint{3.075002in}{2.632286in}}%
\pgfpathmoveto{\pgfqpoint{3.070461in}{2.635236in}}%
\pgfpathlineto{\pgfqpoint{3.070461in}{2.635236in}}%
\pgfpathlineto{\pgfqpoint{3.070461in}{2.638185in}}%
\pgfpathlineto{\pgfqpoint{3.075002in}{2.638185in}}%
\pgfpathlineto{\pgfqpoint{3.075002in}{2.635236in}}%
\pgfpathmoveto{\pgfqpoint{3.070461in}{2.638185in}}%
\pgfpathlineto{\pgfqpoint{3.070461in}{2.638185in}}%
\pgfpathlineto{\pgfqpoint{3.070461in}{2.641134in}}%
\pgfpathlineto{\pgfqpoint{3.075002in}{2.641134in}}%
\pgfpathlineto{\pgfqpoint{3.075002in}{2.638185in}}%
\pgfpathmoveto{\pgfqpoint{3.070461in}{2.641134in}}%
\pgfpathlineto{\pgfqpoint{3.070461in}{2.641134in}}%
\pgfpathlineto{\pgfqpoint{3.070461in}{2.644083in}}%
\pgfpathlineto{\pgfqpoint{3.075002in}{2.644083in}}%
\pgfpathlineto{\pgfqpoint{3.075002in}{2.641134in}}%
\pgfpathmoveto{\pgfqpoint{3.070461in}{2.644083in}}%
\pgfpathlineto{\pgfqpoint{3.070461in}{2.644083in}}%
\pgfpathlineto{\pgfqpoint{3.070461in}{2.647033in}}%
\pgfpathlineto{\pgfqpoint{3.075002in}{2.647033in}}%
\pgfpathlineto{\pgfqpoint{3.075002in}{2.644083in}}%
\pgfpathmoveto{\pgfqpoint{3.070461in}{2.647033in}}%
\pgfpathlineto{\pgfqpoint{3.070461in}{2.647033in}}%
\pgfpathlineto{\pgfqpoint{3.070461in}{2.649982in}}%
\pgfpathlineto{\pgfqpoint{3.075002in}{2.649982in}}%
\pgfpathlineto{\pgfqpoint{3.075002in}{2.647033in}}%
\pgfpathmoveto{\pgfqpoint{3.070461in}{2.649982in}}%
\pgfpathlineto{\pgfqpoint{3.070461in}{2.649982in}}%
\pgfpathlineto{\pgfqpoint{3.070461in}{2.652931in}}%
\pgfpathlineto{\pgfqpoint{3.075002in}{2.652931in}}%
\pgfpathlineto{\pgfqpoint{3.075002in}{2.649982in}}%
\pgfpathmoveto{\pgfqpoint{3.070461in}{2.652931in}}%
\pgfpathlineto{\pgfqpoint{3.070461in}{2.652931in}}%
\pgfpathlineto{\pgfqpoint{3.070461in}{2.655880in}}%
\pgfpathlineto{\pgfqpoint{3.075002in}{2.655880in}}%
\pgfpathlineto{\pgfqpoint{3.075002in}{2.652931in}}%
\pgfpathmoveto{\pgfqpoint{3.070461in}{2.655880in}}%
\pgfpathlineto{\pgfqpoint{3.070461in}{2.655880in}}%
\pgfpathlineto{\pgfqpoint{3.070461in}{2.658830in}}%
\pgfpathlineto{\pgfqpoint{3.075002in}{2.658830in}}%
\pgfpathlineto{\pgfqpoint{3.075002in}{2.655880in}}%
\pgfpathmoveto{\pgfqpoint{3.070461in}{2.658830in}}%
\pgfpathlineto{\pgfqpoint{3.070461in}{2.658830in}}%
\pgfpathlineto{\pgfqpoint{3.070461in}{2.661779in}}%
\pgfpathlineto{\pgfqpoint{3.075002in}{2.661779in}}%
\pgfpathlineto{\pgfqpoint{3.075002in}{2.658830in}}%
\pgfpathmoveto{\pgfqpoint{3.070461in}{2.661779in}}%
\pgfpathlineto{\pgfqpoint{3.070461in}{2.661779in}}%
\pgfpathlineto{\pgfqpoint{3.070461in}{2.664728in}}%
\pgfpathlineto{\pgfqpoint{3.075002in}{2.664728in}}%
\pgfpathlineto{\pgfqpoint{3.075002in}{2.661779in}}%
\pgfpathmoveto{\pgfqpoint{3.070461in}{2.664728in}}%
\pgfpathlineto{\pgfqpoint{3.070461in}{2.664728in}}%
\pgfpathlineto{\pgfqpoint{3.070461in}{2.667678in}}%
\pgfpathlineto{\pgfqpoint{3.075002in}{2.667678in}}%
\pgfpathlineto{\pgfqpoint{3.075002in}{2.664728in}}%
\pgfpathmoveto{\pgfqpoint{3.070461in}{2.667678in}}%
\pgfpathlineto{\pgfqpoint{3.070461in}{2.667678in}}%
\pgfpathlineto{\pgfqpoint{3.070461in}{2.670627in}}%
\pgfpathlineto{\pgfqpoint{3.075002in}{2.670627in}}%
\pgfpathlineto{\pgfqpoint{3.075002in}{2.667678in}}%
\pgfpathmoveto{\pgfqpoint{3.070461in}{2.670627in}}%
\pgfpathlineto{\pgfqpoint{3.070461in}{2.670627in}}%
\pgfpathlineto{\pgfqpoint{3.070461in}{2.673576in}}%
\pgfpathlineto{\pgfqpoint{3.075002in}{2.673576in}}%
\pgfpathlineto{\pgfqpoint{3.075002in}{2.670627in}}%
\pgfpathmoveto{\pgfqpoint{3.070461in}{2.673576in}}%
\pgfpathlineto{\pgfqpoint{3.070461in}{2.673576in}}%
\pgfpathlineto{\pgfqpoint{3.070461in}{2.676525in}}%
\pgfpathlineto{\pgfqpoint{3.075002in}{2.676525in}}%
\pgfpathlineto{\pgfqpoint{3.075002in}{2.673576in}}%
\pgfpathmoveto{\pgfqpoint{3.070461in}{2.676525in}}%
\pgfpathlineto{\pgfqpoint{3.070461in}{2.676525in}}%
\pgfpathlineto{\pgfqpoint{3.070461in}{2.679474in}}%
\pgfpathlineto{\pgfqpoint{3.075002in}{2.679474in}}%
\pgfpathlineto{\pgfqpoint{3.075002in}{2.676525in}}%
\pgfpathmoveto{\pgfqpoint{3.070461in}{2.679474in}}%
\pgfpathlineto{\pgfqpoint{3.070461in}{2.679474in}}%
\pgfpathlineto{\pgfqpoint{3.070461in}{2.682423in}}%
\pgfpathlineto{\pgfqpoint{3.075002in}{2.682423in}}%
\pgfpathlineto{\pgfqpoint{3.075002in}{2.679474in}}%
\pgfpathmoveto{\pgfqpoint{3.070461in}{2.682423in}}%
\pgfpathlineto{\pgfqpoint{3.070461in}{2.682423in}}%
\pgfpathlineto{\pgfqpoint{3.070461in}{2.685372in}}%
\pgfpathlineto{\pgfqpoint{3.075002in}{2.685372in}}%
\pgfpathlineto{\pgfqpoint{3.075002in}{2.682423in}}%
\pgfpathmoveto{\pgfqpoint{3.070461in}{2.685372in}}%
\pgfpathlineto{\pgfqpoint{3.070461in}{2.685372in}}%
\pgfpathlineto{\pgfqpoint{3.070461in}{2.688321in}}%
\pgfpathlineto{\pgfqpoint{3.075002in}{2.688321in}}%
\pgfpathlineto{\pgfqpoint{3.075002in}{2.685372in}}%
\pgfpathmoveto{\pgfqpoint{3.070461in}{2.688321in}}%
\pgfpathlineto{\pgfqpoint{3.070461in}{2.688321in}}%
\pgfpathlineto{\pgfqpoint{3.070461in}{2.691270in}}%
\pgfpathlineto{\pgfqpoint{3.075002in}{2.691270in}}%
\pgfpathlineto{\pgfqpoint{3.075002in}{2.688321in}}%
\pgfpathmoveto{\pgfqpoint{3.070461in}{2.691270in}}%
\pgfpathlineto{\pgfqpoint{3.070461in}{2.691270in}}%
\pgfpathlineto{\pgfqpoint{3.070461in}{2.694220in}}%
\pgfpathlineto{\pgfqpoint{3.075002in}{2.694220in}}%
\pgfpathlineto{\pgfqpoint{3.075002in}{2.691270in}}%
\pgfpathmoveto{\pgfqpoint{3.070461in}{2.694220in}}%
\pgfpathlineto{\pgfqpoint{3.070461in}{2.694220in}}%
\pgfpathlineto{\pgfqpoint{3.070461in}{2.697169in}}%
\pgfpathlineto{\pgfqpoint{3.075002in}{2.697169in}}%
\pgfpathlineto{\pgfqpoint{3.075002in}{2.694220in}}%
\pgfpathmoveto{\pgfqpoint{3.070461in}{2.697169in}}%
\pgfpathlineto{\pgfqpoint{3.070461in}{2.697169in}}%
\pgfpathlineto{\pgfqpoint{3.070461in}{2.700118in}}%
\pgfpathlineto{\pgfqpoint{3.075002in}{2.700118in}}%
\pgfpathlineto{\pgfqpoint{3.075002in}{2.697169in}}%
\pgfpathmoveto{\pgfqpoint{3.070461in}{2.700118in}}%
\pgfpathlineto{\pgfqpoint{3.070461in}{2.700118in}}%
\pgfpathlineto{\pgfqpoint{3.070461in}{2.703067in}}%
\pgfpathlineto{\pgfqpoint{3.075002in}{2.703067in}}%
\pgfpathlineto{\pgfqpoint{3.075002in}{2.700118in}}%
\pgfpathmoveto{\pgfqpoint{3.070461in}{2.703067in}}%
\pgfpathlineto{\pgfqpoint{3.070461in}{2.703067in}}%
\pgfpathlineto{\pgfqpoint{3.070461in}{2.706016in}}%
\pgfpathlineto{\pgfqpoint{3.075002in}{2.706016in}}%
\pgfpathlineto{\pgfqpoint{3.075002in}{2.703067in}}%
\pgfpathmoveto{\pgfqpoint{3.070461in}{2.706016in}}%
\pgfpathlineto{\pgfqpoint{3.070461in}{2.706016in}}%
\pgfpathlineto{\pgfqpoint{3.070461in}{2.708965in}}%
\pgfpathlineto{\pgfqpoint{3.075002in}{2.708965in}}%
\pgfpathlineto{\pgfqpoint{3.075002in}{2.706016in}}%
\pgfpathmoveto{\pgfqpoint{3.070461in}{2.708965in}}%
\pgfpathlineto{\pgfqpoint{3.070461in}{2.708965in}}%
\pgfpathlineto{\pgfqpoint{3.070461in}{2.711914in}}%
\pgfpathlineto{\pgfqpoint{3.075002in}{2.711914in}}%
\pgfpathlineto{\pgfqpoint{3.075002in}{2.708965in}}%
\pgfpathmoveto{\pgfqpoint{3.070461in}{2.711914in}}%
\pgfpathlineto{\pgfqpoint{3.070461in}{2.711914in}}%
\pgfpathlineto{\pgfqpoint{3.070461in}{2.714863in}}%
\pgfpathlineto{\pgfqpoint{3.075002in}{2.714863in}}%
\pgfpathlineto{\pgfqpoint{3.075002in}{2.711914in}}%
\pgfpathmoveto{\pgfqpoint{3.070461in}{2.714863in}}%
\pgfpathlineto{\pgfqpoint{3.070461in}{2.714863in}}%
\pgfpathlineto{\pgfqpoint{3.070461in}{2.717812in}}%
\pgfpathlineto{\pgfqpoint{3.075002in}{2.717812in}}%
\pgfpathlineto{\pgfqpoint{3.075002in}{2.714863in}}%
\pgfpathmoveto{\pgfqpoint{3.070461in}{2.717812in}}%
\pgfpathlineto{\pgfqpoint{3.070461in}{2.717812in}}%
\pgfpathlineto{\pgfqpoint{3.070461in}{2.720761in}}%
\pgfpathlineto{\pgfqpoint{3.075002in}{2.720761in}}%
\pgfpathlineto{\pgfqpoint{3.075002in}{2.717812in}}%
\pgfpathmoveto{\pgfqpoint{3.070461in}{2.720761in}}%
\pgfpathlineto{\pgfqpoint{3.070461in}{2.720761in}}%
\pgfpathlineto{\pgfqpoint{3.070461in}{2.723711in}}%
\pgfpathlineto{\pgfqpoint{3.075002in}{2.723711in}}%
\pgfpathlineto{\pgfqpoint{3.075002in}{2.720761in}}%
\pgfpathmoveto{\pgfqpoint{3.070461in}{2.723711in}}%
\pgfpathlineto{\pgfqpoint{3.070461in}{2.723711in}}%
\pgfpathlineto{\pgfqpoint{3.070461in}{2.726660in}}%
\pgfpathlineto{\pgfqpoint{3.075002in}{2.726660in}}%
\pgfpathlineto{\pgfqpoint{3.075002in}{2.723711in}}%
\pgfpathmoveto{\pgfqpoint{3.070461in}{2.726660in}}%
\pgfpathlineto{\pgfqpoint{3.070461in}{2.726660in}}%
\pgfpathlineto{\pgfqpoint{3.070461in}{2.729609in}}%
\pgfpathlineto{\pgfqpoint{3.075002in}{2.729609in}}%
\pgfpathlineto{\pgfqpoint{3.075002in}{2.726660in}}%
\pgfpathmoveto{\pgfqpoint{3.070461in}{2.729609in}}%
\pgfpathlineto{\pgfqpoint{3.070461in}{2.729609in}}%
\pgfpathlineto{\pgfqpoint{3.070461in}{2.732558in}}%
\pgfpathlineto{\pgfqpoint{3.075002in}{2.732558in}}%
\pgfpathlineto{\pgfqpoint{3.075002in}{2.729609in}}%
\pgfpathmoveto{\pgfqpoint{3.070461in}{2.732558in}}%
\pgfpathlineto{\pgfqpoint{3.070461in}{2.732558in}}%
\pgfpathlineto{\pgfqpoint{3.070461in}{2.735507in}}%
\pgfpathlineto{\pgfqpoint{3.075002in}{2.735507in}}%
\pgfpathlineto{\pgfqpoint{3.075002in}{2.732558in}}%
\pgfpathmoveto{\pgfqpoint{3.070461in}{2.735507in}}%
\pgfpathlineto{\pgfqpoint{3.070461in}{2.735507in}}%
\pgfpathlineto{\pgfqpoint{3.070461in}{2.738456in}}%
\pgfpathlineto{\pgfqpoint{3.075002in}{2.738456in}}%
\pgfpathlineto{\pgfqpoint{3.075002in}{2.735507in}}%
\pgfpathmoveto{\pgfqpoint{3.070461in}{2.738456in}}%
\pgfpathlineto{\pgfqpoint{3.070461in}{2.738456in}}%
\pgfpathlineto{\pgfqpoint{3.070461in}{2.741405in}}%
\pgfpathlineto{\pgfqpoint{3.075002in}{2.741405in}}%
\pgfpathlineto{\pgfqpoint{3.075002in}{2.738456in}}%
\pgfpathmoveto{\pgfqpoint{3.070461in}{2.741405in}}%
\pgfpathlineto{\pgfqpoint{3.070461in}{2.741405in}}%
\pgfpathlineto{\pgfqpoint{3.070461in}{2.744354in}}%
\pgfpathlineto{\pgfqpoint{3.075002in}{2.744354in}}%
\pgfpathlineto{\pgfqpoint{3.075002in}{2.741405in}}%
\pgfpathmoveto{\pgfqpoint{3.070461in}{2.744354in}}%
\pgfpathlineto{\pgfqpoint{3.070461in}{2.744354in}}%
\pgfpathlineto{\pgfqpoint{3.070461in}{2.747303in}}%
\pgfpathlineto{\pgfqpoint{3.075002in}{2.747303in}}%
\pgfpathlineto{\pgfqpoint{3.075002in}{2.744354in}}%
\pgfpathmoveto{\pgfqpoint{3.070461in}{2.747303in}}%
\pgfpathlineto{\pgfqpoint{3.070461in}{2.747303in}}%
\pgfpathlineto{\pgfqpoint{3.070461in}{2.750252in}}%
\pgfpathlineto{\pgfqpoint{3.075002in}{2.750252in}}%
\pgfpathlineto{\pgfqpoint{3.075002in}{2.747303in}}%
\pgfpathmoveto{\pgfqpoint{3.070461in}{2.750252in}}%
\pgfpathlineto{\pgfqpoint{3.070461in}{2.750252in}}%
\pgfpathlineto{\pgfqpoint{3.070461in}{2.753202in}}%
\pgfpathlineto{\pgfqpoint{3.075002in}{2.753202in}}%
\pgfpathlineto{\pgfqpoint{3.075002in}{2.750252in}}%
\pgfpathmoveto{\pgfqpoint{3.070461in}{2.753202in}}%
\pgfpathlineto{\pgfqpoint{3.070461in}{2.753202in}}%
\pgfpathlineto{\pgfqpoint{3.070461in}{2.756151in}}%
\pgfpathlineto{\pgfqpoint{3.075002in}{2.756151in}}%
\pgfpathlineto{\pgfqpoint{3.075002in}{2.753202in}}%
\pgfpathmoveto{\pgfqpoint{3.070461in}{2.756151in}}%
\pgfpathlineto{\pgfqpoint{3.070461in}{2.756151in}}%
\pgfpathlineto{\pgfqpoint{3.070461in}{2.759100in}}%
\pgfpathlineto{\pgfqpoint{3.075002in}{2.759100in}}%
\pgfpathlineto{\pgfqpoint{3.075002in}{2.756151in}}%
\pgfpathmoveto{\pgfqpoint{3.070461in}{2.759100in}}%
\pgfpathlineto{\pgfqpoint{3.070461in}{2.759100in}}%
\pgfpathlineto{\pgfqpoint{3.070461in}{2.762049in}}%
\pgfpathlineto{\pgfqpoint{3.075002in}{2.762049in}}%
\pgfpathlineto{\pgfqpoint{3.075002in}{2.759100in}}%
\pgfpathmoveto{\pgfqpoint{3.070461in}{2.762049in}}%
\pgfpathlineto{\pgfqpoint{3.070461in}{2.762049in}}%
\pgfpathlineto{\pgfqpoint{3.070461in}{2.764998in}}%
\pgfpathlineto{\pgfqpoint{3.075002in}{2.764998in}}%
\pgfpathlineto{\pgfqpoint{3.075002in}{2.762049in}}%
\pgfpathmoveto{\pgfqpoint{3.070461in}{2.764998in}}%
\pgfpathlineto{\pgfqpoint{3.070461in}{2.764998in}}%
\pgfpathlineto{\pgfqpoint{3.070461in}{2.767947in}}%
\pgfpathlineto{\pgfqpoint{3.075002in}{2.767947in}}%
\pgfpathlineto{\pgfqpoint{3.075002in}{2.764998in}}%
\pgfpathmoveto{\pgfqpoint{3.070461in}{2.767947in}}%
\pgfpathlineto{\pgfqpoint{3.070461in}{2.767947in}}%
\pgfpathlineto{\pgfqpoint{3.070461in}{2.770897in}}%
\pgfpathlineto{\pgfqpoint{3.075002in}{2.770897in}}%
\pgfpathlineto{\pgfqpoint{3.075002in}{2.767947in}}%
\pgfpathmoveto{\pgfqpoint{3.070461in}{2.770897in}}%
\pgfpathlineto{\pgfqpoint{3.070461in}{2.770897in}}%
\pgfpathlineto{\pgfqpoint{3.070461in}{2.773846in}}%
\pgfpathlineto{\pgfqpoint{3.075002in}{2.773846in}}%
\pgfpathlineto{\pgfqpoint{3.075002in}{2.770897in}}%
\pgfpathmoveto{\pgfqpoint{3.070461in}{2.773846in}}%
\pgfpathlineto{\pgfqpoint{3.070461in}{2.773846in}}%
\pgfpathlineto{\pgfqpoint{3.070461in}{2.776795in}}%
\pgfpathlineto{\pgfqpoint{3.075002in}{2.776795in}}%
\pgfpathlineto{\pgfqpoint{3.075002in}{2.773846in}}%
\pgfpathmoveto{\pgfqpoint{3.070461in}{2.776795in}}%
\pgfpathlineto{\pgfqpoint{3.070461in}{2.776795in}}%
\pgfpathlineto{\pgfqpoint{3.070461in}{2.779744in}}%
\pgfpathlineto{\pgfqpoint{3.075002in}{2.779744in}}%
\pgfpathlineto{\pgfqpoint{3.075002in}{2.776795in}}%
\pgfpathmoveto{\pgfqpoint{3.070461in}{2.779744in}}%
\pgfpathlineto{\pgfqpoint{3.070461in}{2.779744in}}%
\pgfpathlineto{\pgfqpoint{3.070461in}{2.782694in}}%
\pgfpathlineto{\pgfqpoint{3.075002in}{2.782694in}}%
\pgfpathlineto{\pgfqpoint{3.075002in}{2.779744in}}%
\pgfpathmoveto{\pgfqpoint{3.070461in}{2.782694in}}%
\pgfpathlineto{\pgfqpoint{3.070461in}{2.782694in}}%
\pgfpathlineto{\pgfqpoint{3.070461in}{2.785643in}}%
\pgfpathlineto{\pgfqpoint{3.075002in}{2.785643in}}%
\pgfpathlineto{\pgfqpoint{3.075002in}{2.782694in}}%
\pgfpathmoveto{\pgfqpoint{3.070461in}{2.785643in}}%
\pgfpathlineto{\pgfqpoint{3.070461in}{2.785643in}}%
\pgfpathlineto{\pgfqpoint{3.070461in}{2.788592in}}%
\pgfpathlineto{\pgfqpoint{3.075002in}{2.788592in}}%
\pgfpathlineto{\pgfqpoint{3.075002in}{2.785643in}}%
\pgfpathmoveto{\pgfqpoint{3.070461in}{2.788592in}}%
\pgfpathlineto{\pgfqpoint{3.070461in}{2.788592in}}%
\pgfpathlineto{\pgfqpoint{3.070461in}{2.791542in}}%
\pgfpathlineto{\pgfqpoint{3.075002in}{2.791542in}}%
\pgfpathlineto{\pgfqpoint{3.075002in}{2.788592in}}%
\pgfpathmoveto{\pgfqpoint{3.070461in}{2.791542in}}%
\pgfpathlineto{\pgfqpoint{3.070461in}{2.791542in}}%
\pgfpathlineto{\pgfqpoint{3.070461in}{2.794491in}}%
\pgfpathlineto{\pgfqpoint{3.075002in}{2.794491in}}%
\pgfpathlineto{\pgfqpoint{3.075002in}{2.791542in}}%
\pgfpathmoveto{\pgfqpoint{3.070461in}{2.794491in}}%
\pgfpathlineto{\pgfqpoint{3.070461in}{2.794491in}}%
\pgfpathlineto{\pgfqpoint{3.070461in}{2.797440in}}%
\pgfpathlineto{\pgfqpoint{3.075002in}{2.797440in}}%
\pgfpathlineto{\pgfqpoint{3.075002in}{2.794491in}}%
\pgfpathmoveto{\pgfqpoint{3.070461in}{2.797440in}}%
\pgfpathlineto{\pgfqpoint{3.070461in}{2.797440in}}%
\pgfpathlineto{\pgfqpoint{3.070461in}{2.800389in}}%
\pgfpathlineto{\pgfqpoint{3.075002in}{2.800389in}}%
\pgfpathlineto{\pgfqpoint{3.075002in}{2.797440in}}%
\pgfpathmoveto{\pgfqpoint{3.070461in}{2.800389in}}%
\pgfpathlineto{\pgfqpoint{3.070461in}{2.800389in}}%
\pgfpathlineto{\pgfqpoint{3.070461in}{2.803339in}}%
\pgfpathlineto{\pgfqpoint{3.075002in}{2.803339in}}%
\pgfpathlineto{\pgfqpoint{3.075002in}{2.800389in}}%
\pgfpathmoveto{\pgfqpoint{3.070461in}{2.803339in}}%
\pgfpathlineto{\pgfqpoint{3.070461in}{2.803339in}}%
\pgfpathlineto{\pgfqpoint{3.070461in}{2.806288in}}%
\pgfpathlineto{\pgfqpoint{3.075002in}{2.806288in}}%
\pgfpathlineto{\pgfqpoint{3.075002in}{2.803339in}}%
\pgfpathmoveto{\pgfqpoint{3.070461in}{2.806288in}}%
\pgfpathlineto{\pgfqpoint{3.070461in}{2.806288in}}%
\pgfpathlineto{\pgfqpoint{3.070461in}{2.809237in}}%
\pgfpathlineto{\pgfqpoint{3.075002in}{2.809237in}}%
\pgfpathlineto{\pgfqpoint{3.075002in}{2.806288in}}%
\pgfpathmoveto{\pgfqpoint{3.070461in}{2.809237in}}%
\pgfpathlineto{\pgfqpoint{3.070461in}{2.809237in}}%
\pgfpathlineto{\pgfqpoint{3.070461in}{2.812187in}}%
\pgfpathlineto{\pgfqpoint{3.075002in}{2.812187in}}%
\pgfpathlineto{\pgfqpoint{3.075002in}{2.809237in}}%
\pgfpathmoveto{\pgfqpoint{3.070461in}{2.812187in}}%
\pgfpathlineto{\pgfqpoint{3.070461in}{2.812187in}}%
\pgfpathlineto{\pgfqpoint{3.070461in}{2.815136in}}%
\pgfpathlineto{\pgfqpoint{3.075002in}{2.815136in}}%
\pgfpathlineto{\pgfqpoint{3.075002in}{2.812187in}}%
\pgfpathmoveto{\pgfqpoint{3.070461in}{2.815136in}}%
\pgfpathlineto{\pgfqpoint{3.070461in}{2.815136in}}%
\pgfpathlineto{\pgfqpoint{3.070461in}{2.818085in}}%
\pgfpathlineto{\pgfqpoint{3.075002in}{2.818085in}}%
\pgfpathlineto{\pgfqpoint{3.075002in}{2.815136in}}%
\pgfpathmoveto{\pgfqpoint{3.070461in}{2.818085in}}%
\pgfpathlineto{\pgfqpoint{3.070461in}{2.818085in}}%
\pgfpathlineto{\pgfqpoint{3.070461in}{2.821034in}}%
\pgfpathlineto{\pgfqpoint{3.075002in}{2.821034in}}%
\pgfpathlineto{\pgfqpoint{3.075002in}{2.818085in}}%
\pgfpathmoveto{\pgfqpoint{3.070461in}{2.821034in}}%
\pgfpathlineto{\pgfqpoint{3.070461in}{2.821034in}}%
\pgfpathlineto{\pgfqpoint{3.070461in}{2.823984in}}%
\pgfpathlineto{\pgfqpoint{3.075002in}{2.823984in}}%
\pgfpathlineto{\pgfqpoint{3.075002in}{2.821034in}}%
\pgfpathmoveto{\pgfqpoint{3.070461in}{2.823984in}}%
\pgfpathlineto{\pgfqpoint{3.070461in}{2.823984in}}%
\pgfpathlineto{\pgfqpoint{3.070461in}{2.826933in}}%
\pgfpathlineto{\pgfqpoint{3.075002in}{2.826933in}}%
\pgfpathlineto{\pgfqpoint{3.075002in}{2.823984in}}%
\pgfpathmoveto{\pgfqpoint{3.070461in}{2.826933in}}%
\pgfpathlineto{\pgfqpoint{3.070461in}{2.826933in}}%
\pgfpathlineto{\pgfqpoint{3.070461in}{2.829882in}}%
\pgfpathlineto{\pgfqpoint{3.075002in}{2.829882in}}%
\pgfpathlineto{\pgfqpoint{3.075002in}{2.826933in}}%
\pgfpathmoveto{\pgfqpoint{3.070461in}{2.829882in}}%
\pgfpathlineto{\pgfqpoint{3.070461in}{2.829882in}}%
\pgfpathlineto{\pgfqpoint{3.070461in}{2.832832in}}%
\pgfpathlineto{\pgfqpoint{3.075002in}{2.832832in}}%
\pgfpathlineto{\pgfqpoint{3.075002in}{2.829882in}}%
\pgfpathmoveto{\pgfqpoint{3.070461in}{2.832832in}}%
\pgfpathlineto{\pgfqpoint{3.070461in}{2.832832in}}%
\pgfpathlineto{\pgfqpoint{3.070461in}{2.835781in}}%
\pgfpathlineto{\pgfqpoint{3.075002in}{2.835781in}}%
\pgfpathlineto{\pgfqpoint{3.075002in}{2.832832in}}%
\pgfpathmoveto{\pgfqpoint{3.070461in}{2.835781in}}%
\pgfpathlineto{\pgfqpoint{3.070461in}{2.835781in}}%
\pgfpathlineto{\pgfqpoint{3.070461in}{2.838730in}}%
\pgfpathlineto{\pgfqpoint{3.075002in}{2.838730in}}%
\pgfpathlineto{\pgfqpoint{3.075002in}{2.835781in}}%
\pgfpathmoveto{\pgfqpoint{3.070461in}{2.838730in}}%
\pgfpathlineto{\pgfqpoint{3.070461in}{2.838730in}}%
\pgfpathlineto{\pgfqpoint{3.070461in}{2.841679in}}%
\pgfpathlineto{\pgfqpoint{3.075002in}{2.841679in}}%
\pgfpathlineto{\pgfqpoint{3.075002in}{2.838730in}}%
\pgfpathmoveto{\pgfqpoint{3.070461in}{2.841679in}}%
\pgfpathlineto{\pgfqpoint{3.070461in}{2.841679in}}%
\pgfpathlineto{\pgfqpoint{3.070461in}{2.844629in}}%
\pgfpathlineto{\pgfqpoint{3.075002in}{2.844629in}}%
\pgfpathlineto{\pgfqpoint{3.075002in}{2.841679in}}%
\pgfpathmoveto{\pgfqpoint{3.070461in}{2.844629in}}%
\pgfpathlineto{\pgfqpoint{3.070461in}{2.844629in}}%
\pgfpathlineto{\pgfqpoint{3.070461in}{2.847578in}}%
\pgfpathlineto{\pgfqpoint{3.075002in}{2.847578in}}%
\pgfpathlineto{\pgfqpoint{3.075002in}{2.844629in}}%
\pgfpathmoveto{\pgfqpoint{3.070461in}{2.847578in}}%
\pgfpathlineto{\pgfqpoint{3.070461in}{2.847578in}}%
\pgfpathlineto{\pgfqpoint{3.070461in}{2.850527in}}%
\pgfpathlineto{\pgfqpoint{3.075002in}{2.850527in}}%
\pgfpathlineto{\pgfqpoint{3.075002in}{2.847578in}}%
\pgfpathmoveto{\pgfqpoint{3.070461in}{2.850527in}}%
\pgfpathlineto{\pgfqpoint{3.070461in}{2.850527in}}%
\pgfpathlineto{\pgfqpoint{3.070461in}{2.853477in}}%
\pgfpathlineto{\pgfqpoint{3.075002in}{2.853477in}}%
\pgfpathlineto{\pgfqpoint{3.075002in}{2.850527in}}%
\pgfpathmoveto{\pgfqpoint{3.070461in}{2.853477in}}%
\pgfpathlineto{\pgfqpoint{3.070461in}{2.853477in}}%
\pgfpathlineto{\pgfqpoint{3.070461in}{2.856426in}}%
\pgfpathlineto{\pgfqpoint{3.075002in}{2.856426in}}%
\pgfpathlineto{\pgfqpoint{3.075002in}{2.853477in}}%
\pgfpathmoveto{\pgfqpoint{3.070461in}{2.856426in}}%
\pgfpathlineto{\pgfqpoint{3.070461in}{2.856426in}}%
\pgfpathlineto{\pgfqpoint{3.070461in}{2.859375in}}%
\pgfpathlineto{\pgfqpoint{3.075002in}{2.859375in}}%
\pgfpathlineto{\pgfqpoint{3.075002in}{2.856426in}}%
\pgfpathmoveto{\pgfqpoint{3.070461in}{2.859375in}}%
\pgfpathlineto{\pgfqpoint{3.070461in}{2.859375in}}%
\pgfpathlineto{\pgfqpoint{3.070461in}{2.862324in}}%
\pgfpathlineto{\pgfqpoint{3.075002in}{2.862324in}}%
\pgfpathlineto{\pgfqpoint{3.075002in}{2.859375in}}%
\pgfpathmoveto{\pgfqpoint{3.070461in}{2.862324in}}%
\pgfpathlineto{\pgfqpoint{3.070461in}{2.862324in}}%
\pgfpathlineto{\pgfqpoint{3.070461in}{2.865274in}}%
\pgfpathlineto{\pgfqpoint{3.075002in}{2.865274in}}%
\pgfpathlineto{\pgfqpoint{3.075002in}{2.862324in}}%
\pgfpathmoveto{\pgfqpoint{3.070461in}{2.865274in}}%
\pgfpathlineto{\pgfqpoint{3.070461in}{2.865274in}}%
\pgfpathlineto{\pgfqpoint{3.070461in}{2.868223in}}%
\pgfpathlineto{\pgfqpoint{3.075002in}{2.868223in}}%
\pgfpathlineto{\pgfqpoint{3.075002in}{2.865274in}}%
\pgfpathmoveto{\pgfqpoint{3.070461in}{2.868223in}}%
\pgfpathlineto{\pgfqpoint{3.070461in}{2.868223in}}%
\pgfpathlineto{\pgfqpoint{3.070461in}{2.871172in}}%
\pgfpathlineto{\pgfqpoint{3.075002in}{2.871172in}}%
\pgfpathlineto{\pgfqpoint{3.075002in}{2.868223in}}%
\pgfpathmoveto{\pgfqpoint{3.070461in}{2.871172in}}%
\pgfpathlineto{\pgfqpoint{3.070461in}{2.871172in}}%
\pgfpathlineto{\pgfqpoint{3.070461in}{2.874121in}}%
\pgfpathlineto{\pgfqpoint{3.075002in}{2.874121in}}%
\pgfpathlineto{\pgfqpoint{3.075002in}{2.871172in}}%
\pgfpathmoveto{\pgfqpoint{3.070461in}{2.874121in}}%
\pgfpathlineto{\pgfqpoint{3.070461in}{2.874121in}}%
\pgfpathlineto{\pgfqpoint{3.070461in}{2.877071in}}%
\pgfpathlineto{\pgfqpoint{3.075002in}{2.877071in}}%
\pgfpathlineto{\pgfqpoint{3.075002in}{2.874121in}}%
\pgfpathmoveto{\pgfqpoint{3.070461in}{2.877071in}}%
\pgfpathlineto{\pgfqpoint{3.070461in}{2.877071in}}%
\pgfpathlineto{\pgfqpoint{3.070461in}{2.880020in}}%
\pgfpathlineto{\pgfqpoint{3.075002in}{2.880020in}}%
\pgfpathlineto{\pgfqpoint{3.075002in}{2.877071in}}%
\pgfpathmoveto{\pgfqpoint{3.070461in}{2.880020in}}%
\pgfpathlineto{\pgfqpoint{3.070461in}{2.880020in}}%
\pgfpathlineto{\pgfqpoint{3.070461in}{2.882969in}}%
\pgfpathlineto{\pgfqpoint{3.075002in}{2.882969in}}%
\pgfpathlineto{\pgfqpoint{3.075002in}{2.880020in}}%
\pgfpathmoveto{\pgfqpoint{3.070461in}{2.882969in}}%
\pgfpathlineto{\pgfqpoint{3.070461in}{2.882969in}}%
\pgfpathlineto{\pgfqpoint{3.070461in}{2.885918in}}%
\pgfpathlineto{\pgfqpoint{3.075002in}{2.885918in}}%
\pgfpathlineto{\pgfqpoint{3.075002in}{2.882969in}}%
\pgfpathmoveto{\pgfqpoint{3.070461in}{2.885918in}}%
\pgfpathlineto{\pgfqpoint{3.070461in}{2.885918in}}%
\pgfpathlineto{\pgfqpoint{3.070461in}{2.888868in}}%
\pgfpathlineto{\pgfqpoint{3.075002in}{2.888868in}}%
\pgfpathlineto{\pgfqpoint{3.075002in}{2.885918in}}%
\pgfpathmoveto{\pgfqpoint{3.070461in}{2.888868in}}%
\pgfpathlineto{\pgfqpoint{3.070461in}{2.888868in}}%
\pgfpathlineto{\pgfqpoint{3.070461in}{2.891817in}}%
\pgfpathlineto{\pgfqpoint{3.075002in}{2.891817in}}%
\pgfpathlineto{\pgfqpoint{3.075002in}{2.888868in}}%
\pgfpathmoveto{\pgfqpoint{3.070461in}{2.891817in}}%
\pgfpathlineto{\pgfqpoint{3.070461in}{2.891817in}}%
\pgfpathlineto{\pgfqpoint{3.070461in}{2.894766in}}%
\pgfpathlineto{\pgfqpoint{3.075002in}{2.894766in}}%
\pgfpathlineto{\pgfqpoint{3.075002in}{2.891817in}}%
\pgfpathmoveto{\pgfqpoint{3.070461in}{2.894766in}}%
\pgfpathlineto{\pgfqpoint{3.070461in}{2.894766in}}%
\pgfpathlineto{\pgfqpoint{3.070461in}{2.897716in}}%
\pgfpathlineto{\pgfqpoint{3.075002in}{2.897716in}}%
\pgfpathlineto{\pgfqpoint{3.075002in}{2.894766in}}%
\pgfpathmoveto{\pgfqpoint{3.070461in}{2.897716in}}%
\pgfpathlineto{\pgfqpoint{3.070461in}{2.897716in}}%
\pgfpathlineto{\pgfqpoint{3.070461in}{2.900665in}}%
\pgfpathlineto{\pgfqpoint{3.075002in}{2.900665in}}%
\pgfpathlineto{\pgfqpoint{3.075002in}{2.897716in}}%
\pgfpathmoveto{\pgfqpoint{3.070461in}{2.900665in}}%
\pgfpathlineto{\pgfqpoint{3.070461in}{2.900665in}}%
\pgfpathlineto{\pgfqpoint{3.070461in}{2.903614in}}%
\pgfpathlineto{\pgfqpoint{3.075002in}{2.903614in}}%
\pgfpathlineto{\pgfqpoint{3.075002in}{2.900665in}}%
\pgfpathmoveto{\pgfqpoint{3.070461in}{2.903614in}}%
\pgfpathlineto{\pgfqpoint{3.070461in}{2.903614in}}%
\pgfpathlineto{\pgfqpoint{3.070461in}{2.906563in}}%
\pgfpathlineto{\pgfqpoint{3.075002in}{2.906563in}}%
\pgfpathlineto{\pgfqpoint{3.075002in}{2.903614in}}%
\pgfpathmoveto{\pgfqpoint{3.070461in}{2.906563in}}%
\pgfpathlineto{\pgfqpoint{3.070461in}{2.906563in}}%
\pgfpathlineto{\pgfqpoint{3.070461in}{2.909513in}}%
\pgfpathlineto{\pgfqpoint{3.075002in}{2.909513in}}%
\pgfpathlineto{\pgfqpoint{3.075002in}{2.906563in}}%
\pgfpathmoveto{\pgfqpoint{3.070461in}{2.909513in}}%
\pgfpathlineto{\pgfqpoint{3.070461in}{2.909513in}}%
\pgfpathlineto{\pgfqpoint{3.070461in}{2.912462in}}%
\pgfpathlineto{\pgfqpoint{3.075002in}{2.912462in}}%
\pgfpathlineto{\pgfqpoint{3.075002in}{2.909513in}}%
\pgfpathmoveto{\pgfqpoint{3.070461in}{2.912462in}}%
\pgfpathlineto{\pgfqpoint{3.070461in}{2.912462in}}%
\pgfpathlineto{\pgfqpoint{3.070461in}{2.915411in}}%
\pgfpathlineto{\pgfqpoint{3.075002in}{2.915411in}}%
\pgfpathlineto{\pgfqpoint{3.075002in}{2.912462in}}%
\pgfpathmoveto{\pgfqpoint{3.070461in}{2.915411in}}%
\pgfpathlineto{\pgfqpoint{3.070461in}{2.915411in}}%
\pgfpathlineto{\pgfqpoint{3.070461in}{2.918360in}}%
\pgfpathlineto{\pgfqpoint{3.075002in}{2.918360in}}%
\pgfpathlineto{\pgfqpoint{3.075002in}{2.915411in}}%
\pgfpathmoveto{\pgfqpoint{3.070461in}{2.918360in}}%
\pgfpathlineto{\pgfqpoint{3.070461in}{2.918360in}}%
\pgfpathlineto{\pgfqpoint{3.070461in}{2.921310in}}%
\pgfpathlineto{\pgfqpoint{3.075002in}{2.921310in}}%
\pgfpathlineto{\pgfqpoint{3.075002in}{2.918360in}}%
\pgfpathmoveto{\pgfqpoint{3.070461in}{2.921310in}}%
\pgfpathlineto{\pgfqpoint{3.070461in}{2.921310in}}%
\pgfpathlineto{\pgfqpoint{3.070461in}{2.924259in}}%
\pgfpathlineto{\pgfqpoint{3.075002in}{2.924259in}}%
\pgfpathlineto{\pgfqpoint{3.075002in}{2.921310in}}%
\pgfpathmoveto{\pgfqpoint{3.070461in}{2.924259in}}%
\pgfpathlineto{\pgfqpoint{3.070461in}{2.924259in}}%
\pgfpathlineto{\pgfqpoint{3.070461in}{2.927208in}}%
\pgfpathlineto{\pgfqpoint{3.075002in}{2.927208in}}%
\pgfpathlineto{\pgfqpoint{3.075002in}{2.924259in}}%
\pgfpathmoveto{\pgfqpoint{3.070461in}{2.927208in}}%
\pgfpathlineto{\pgfqpoint{3.070461in}{2.927208in}}%
\pgfpathlineto{\pgfqpoint{3.070461in}{2.930157in}}%
\pgfpathlineto{\pgfqpoint{3.075002in}{2.930157in}}%
\pgfpathlineto{\pgfqpoint{3.075002in}{2.927208in}}%
\pgfpathmoveto{\pgfqpoint{3.070461in}{2.930157in}}%
\pgfpathlineto{\pgfqpoint{3.070461in}{2.930157in}}%
\pgfpathlineto{\pgfqpoint{3.070461in}{2.933107in}}%
\pgfpathlineto{\pgfqpoint{3.075002in}{2.933107in}}%
\pgfpathlineto{\pgfqpoint{3.075002in}{2.930157in}}%
\pgfpathmoveto{\pgfqpoint{3.070461in}{2.933107in}}%
\pgfpathlineto{\pgfqpoint{3.070461in}{2.933107in}}%
\pgfpathlineto{\pgfqpoint{3.070461in}{2.936056in}}%
\pgfpathlineto{\pgfqpoint{3.075002in}{2.936056in}}%
\pgfpathlineto{\pgfqpoint{3.075002in}{2.933107in}}%
\pgfpathmoveto{\pgfqpoint{3.070461in}{2.936056in}}%
\pgfpathlineto{\pgfqpoint{3.070461in}{2.936056in}}%
\pgfpathlineto{\pgfqpoint{3.070461in}{2.939005in}}%
\pgfpathlineto{\pgfqpoint{3.075002in}{2.939005in}}%
\pgfpathlineto{\pgfqpoint{3.075002in}{2.936056in}}%
\pgfpathmoveto{\pgfqpoint{3.070461in}{2.939005in}}%
\pgfpathlineto{\pgfqpoint{3.070461in}{2.939005in}}%
\pgfpathlineto{\pgfqpoint{3.070461in}{2.941954in}}%
\pgfpathlineto{\pgfqpoint{3.075002in}{2.941954in}}%
\pgfpathlineto{\pgfqpoint{3.075002in}{2.939005in}}%
\pgfpathmoveto{\pgfqpoint{3.070461in}{2.941954in}}%
\pgfpathlineto{\pgfqpoint{3.070461in}{2.941954in}}%
\pgfpathlineto{\pgfqpoint{3.070461in}{2.944904in}}%
\pgfpathlineto{\pgfqpoint{3.075002in}{2.944904in}}%
\pgfpathlineto{\pgfqpoint{3.075002in}{2.941954in}}%
\pgfpathmoveto{\pgfqpoint{3.070461in}{2.944904in}}%
\pgfpathlineto{\pgfqpoint{3.070461in}{2.944904in}}%
\pgfpathlineto{\pgfqpoint{3.070461in}{2.947853in}}%
\pgfpathlineto{\pgfqpoint{3.075002in}{2.947853in}}%
\pgfpathlineto{\pgfqpoint{3.075002in}{2.944904in}}%
\pgfpathmoveto{\pgfqpoint{3.070461in}{2.947853in}}%
\pgfpathlineto{\pgfqpoint{3.070461in}{2.947853in}}%
\pgfpathlineto{\pgfqpoint{3.070461in}{2.950802in}}%
\pgfpathlineto{\pgfqpoint{3.075002in}{2.950802in}}%
\pgfpathlineto{\pgfqpoint{3.075002in}{2.947853in}}%
\pgfpathmoveto{\pgfqpoint{3.070461in}{2.950802in}}%
\pgfpathlineto{\pgfqpoint{3.070461in}{2.950802in}}%
\pgfpathlineto{\pgfqpoint{3.070461in}{2.953751in}}%
\pgfpathlineto{\pgfqpoint{3.075002in}{2.953751in}}%
\pgfpathlineto{\pgfqpoint{3.075002in}{2.950802in}}%
\pgfpathmoveto{\pgfqpoint{3.070461in}{2.953751in}}%
\pgfpathlineto{\pgfqpoint{3.070461in}{2.953751in}}%
\pgfpathlineto{\pgfqpoint{3.070461in}{2.956701in}}%
\pgfpathlineto{\pgfqpoint{3.075002in}{2.956701in}}%
\pgfpathlineto{\pgfqpoint{3.075002in}{2.953751in}}%
\pgfpathmoveto{\pgfqpoint{3.070461in}{2.956701in}}%
\pgfpathlineto{\pgfqpoint{3.070461in}{2.956701in}}%
\pgfpathlineto{\pgfqpoint{3.070461in}{2.959650in}}%
\pgfpathlineto{\pgfqpoint{3.075002in}{2.959650in}}%
\pgfpathlineto{\pgfqpoint{3.075002in}{2.956701in}}%
\pgfpathmoveto{\pgfqpoint{3.070461in}{2.959650in}}%
\pgfpathlineto{\pgfqpoint{3.070461in}{2.959650in}}%
\pgfpathlineto{\pgfqpoint{3.070461in}{2.962599in}}%
\pgfpathlineto{\pgfqpoint{3.075002in}{2.962599in}}%
\pgfpathlineto{\pgfqpoint{3.075002in}{2.959650in}}%
\pgfpathmoveto{\pgfqpoint{3.070461in}{2.962599in}}%
\pgfpathlineto{\pgfqpoint{3.070461in}{2.962599in}}%
\pgfpathlineto{\pgfqpoint{3.070461in}{2.965548in}}%
\pgfpathlineto{\pgfqpoint{3.075002in}{2.965548in}}%
\pgfpathlineto{\pgfqpoint{3.075002in}{2.962599in}}%
\pgfpathmoveto{\pgfqpoint{3.070461in}{2.965548in}}%
\pgfpathlineto{\pgfqpoint{3.070461in}{2.965548in}}%
\pgfpathlineto{\pgfqpoint{3.070461in}{2.968497in}}%
\pgfpathlineto{\pgfqpoint{3.075002in}{2.968497in}}%
\pgfpathlineto{\pgfqpoint{3.075002in}{2.965548in}}%
\pgfpathmoveto{\pgfqpoint{3.070461in}{2.968497in}}%
\pgfpathlineto{\pgfqpoint{3.070461in}{2.968497in}}%
\pgfpathlineto{\pgfqpoint{3.070461in}{2.971447in}}%
\pgfpathlineto{\pgfqpoint{3.075002in}{2.971447in}}%
\pgfpathlineto{\pgfqpoint{3.075002in}{2.968497in}}%
\pgfpathmoveto{\pgfqpoint{3.070461in}{2.971447in}}%
\pgfpathlineto{\pgfqpoint{3.070461in}{2.971447in}}%
\pgfpathlineto{\pgfqpoint{3.070461in}{2.974396in}}%
\pgfpathlineto{\pgfqpoint{3.075002in}{2.974396in}}%
\pgfpathlineto{\pgfqpoint{3.075002in}{2.971447in}}%
\pgfpathmoveto{\pgfqpoint{3.070461in}{2.974396in}}%
\pgfpathlineto{\pgfqpoint{3.070461in}{2.974396in}}%
\pgfpathlineto{\pgfqpoint{3.070461in}{2.977345in}}%
\pgfpathlineto{\pgfqpoint{3.075002in}{2.977345in}}%
\pgfpathlineto{\pgfqpoint{3.075002in}{2.974396in}}%
\pgfpathmoveto{\pgfqpoint{3.070461in}{2.977345in}}%
\pgfpathlineto{\pgfqpoint{3.070461in}{2.977345in}}%
\pgfpathlineto{\pgfqpoint{3.070461in}{2.980294in}}%
\pgfpathlineto{\pgfqpoint{3.075002in}{2.980294in}}%
\pgfpathlineto{\pgfqpoint{3.075002in}{2.977345in}}%
\pgfpathmoveto{\pgfqpoint{3.070461in}{2.980294in}}%
\pgfpathlineto{\pgfqpoint{3.070461in}{2.980294in}}%
\pgfpathlineto{\pgfqpoint{3.070461in}{2.983243in}}%
\pgfpathlineto{\pgfqpoint{3.075002in}{2.983243in}}%
\pgfpathlineto{\pgfqpoint{3.075002in}{2.980294in}}%
\pgfpathmoveto{\pgfqpoint{3.070461in}{2.983243in}}%
\pgfpathlineto{\pgfqpoint{3.070461in}{2.983243in}}%
\pgfpathlineto{\pgfqpoint{3.070461in}{2.986193in}}%
\pgfpathlineto{\pgfqpoint{3.075002in}{2.986193in}}%
\pgfpathlineto{\pgfqpoint{3.075002in}{2.983243in}}%
\pgfpathmoveto{\pgfqpoint{3.070461in}{2.986193in}}%
\pgfpathlineto{\pgfqpoint{3.070461in}{2.986193in}}%
\pgfpathlineto{\pgfqpoint{3.070461in}{2.989142in}}%
\pgfpathlineto{\pgfqpoint{3.075002in}{2.989142in}}%
\pgfpathlineto{\pgfqpoint{3.075002in}{2.986193in}}%
\pgfpathmoveto{\pgfqpoint{3.070461in}{2.989142in}}%
\pgfpathlineto{\pgfqpoint{3.070461in}{2.989142in}}%
\pgfpathlineto{\pgfqpoint{3.070461in}{2.992091in}}%
\pgfpathlineto{\pgfqpoint{3.075002in}{2.992091in}}%
\pgfpathlineto{\pgfqpoint{3.075002in}{2.989142in}}%
\pgfpathmoveto{\pgfqpoint{3.070461in}{2.992091in}}%
\pgfpathlineto{\pgfqpoint{3.070461in}{2.992091in}}%
\pgfpathlineto{\pgfqpoint{3.070461in}{2.995040in}}%
\pgfpathlineto{\pgfqpoint{3.075002in}{2.995040in}}%
\pgfpathlineto{\pgfqpoint{3.075002in}{2.992091in}}%
\pgfpathmoveto{\pgfqpoint{3.070461in}{2.995040in}}%
\pgfpathlineto{\pgfqpoint{3.070461in}{2.995040in}}%
\pgfpathlineto{\pgfqpoint{3.070461in}{2.997989in}}%
\pgfpathlineto{\pgfqpoint{3.075002in}{2.997989in}}%
\pgfpathlineto{\pgfqpoint{3.075002in}{2.995040in}}%
\pgfpathmoveto{\pgfqpoint{3.070461in}{2.997989in}}%
\pgfpathlineto{\pgfqpoint{3.070461in}{2.997989in}}%
\pgfpathlineto{\pgfqpoint{3.070461in}{3.000938in}}%
\pgfpathlineto{\pgfqpoint{3.075002in}{3.000938in}}%
\pgfpathlineto{\pgfqpoint{3.075002in}{2.997989in}}%
\pgfpathmoveto{\pgfqpoint{3.070461in}{3.000938in}}%
\pgfpathlineto{\pgfqpoint{3.070461in}{3.000938in}}%
\pgfpathlineto{\pgfqpoint{3.070461in}{3.003888in}}%
\pgfpathlineto{\pgfqpoint{3.075002in}{3.003888in}}%
\pgfpathlineto{\pgfqpoint{3.075002in}{3.000938in}}%
\pgfpathmoveto{\pgfqpoint{3.070461in}{3.003888in}}%
\pgfpathlineto{\pgfqpoint{3.070461in}{3.003888in}}%
\pgfpathlineto{\pgfqpoint{3.070461in}{3.006837in}}%
\pgfpathlineto{\pgfqpoint{3.075002in}{3.006837in}}%
\pgfpathlineto{\pgfqpoint{3.075002in}{3.003888in}}%
\pgfpathmoveto{\pgfqpoint{3.070461in}{3.006837in}}%
\pgfpathlineto{\pgfqpoint{3.070461in}{3.006837in}}%
\pgfpathlineto{\pgfqpoint{3.070461in}{3.009786in}}%
\pgfpathlineto{\pgfqpoint{3.075002in}{3.009786in}}%
\pgfpathlineto{\pgfqpoint{3.075002in}{3.006837in}}%
\pgfpathmoveto{\pgfqpoint{3.070461in}{3.009786in}}%
\pgfpathlineto{\pgfqpoint{3.070461in}{3.009786in}}%
\pgfpathlineto{\pgfqpoint{3.070461in}{3.012735in}}%
\pgfpathlineto{\pgfqpoint{3.075002in}{3.012735in}}%
\pgfpathlineto{\pgfqpoint{3.075002in}{3.009786in}}%
\pgfpathmoveto{\pgfqpoint{3.070461in}{3.012735in}}%
\pgfpathlineto{\pgfqpoint{3.070461in}{3.012735in}}%
\pgfpathlineto{\pgfqpoint{3.070461in}{3.015684in}}%
\pgfpathlineto{\pgfqpoint{3.075002in}{3.015684in}}%
\pgfpathlineto{\pgfqpoint{3.075002in}{3.012735in}}%
\pgfpathmoveto{\pgfqpoint{3.070461in}{3.015684in}}%
\pgfpathlineto{\pgfqpoint{3.070461in}{3.015684in}}%
\pgfpathlineto{\pgfqpoint{3.070461in}{3.018634in}}%
\pgfpathlineto{\pgfqpoint{3.075002in}{3.018634in}}%
\pgfpathlineto{\pgfqpoint{3.075002in}{3.015684in}}%
\pgfpathmoveto{\pgfqpoint{3.070461in}{3.018634in}}%
\pgfpathlineto{\pgfqpoint{3.070461in}{3.018634in}}%
\pgfpathlineto{\pgfqpoint{3.070461in}{3.021583in}}%
\pgfpathlineto{\pgfqpoint{3.075002in}{3.021583in}}%
\pgfpathlineto{\pgfqpoint{3.075002in}{3.018634in}}%
\pgfpathmoveto{\pgfqpoint{3.070461in}{3.021583in}}%
\pgfpathlineto{\pgfqpoint{3.070461in}{3.021583in}}%
\pgfpathlineto{\pgfqpoint{3.070461in}{3.024532in}}%
\pgfpathlineto{\pgfqpoint{3.075002in}{3.024532in}}%
\pgfpathlineto{\pgfqpoint{3.075002in}{3.021583in}}%
\pgfpathmoveto{\pgfqpoint{3.070461in}{3.024532in}}%
\pgfpathlineto{\pgfqpoint{3.070461in}{3.024532in}}%
\pgfpathlineto{\pgfqpoint{3.070461in}{3.027481in}}%
\pgfpathlineto{\pgfqpoint{3.075002in}{3.027481in}}%
\pgfpathlineto{\pgfqpoint{3.075002in}{3.024532in}}%
\pgfpathmoveto{\pgfqpoint{3.070461in}{3.027481in}}%
\pgfpathlineto{\pgfqpoint{3.070461in}{3.027481in}}%
\pgfpathlineto{\pgfqpoint{3.070461in}{3.030430in}}%
\pgfpathlineto{\pgfqpoint{3.075002in}{3.030430in}}%
\pgfpathlineto{\pgfqpoint{3.075002in}{3.027481in}}%
\pgfpathmoveto{\pgfqpoint{3.070461in}{3.030430in}}%
\pgfpathlineto{\pgfqpoint{3.070461in}{3.030430in}}%
\pgfpathlineto{\pgfqpoint{3.070461in}{3.033379in}}%
\pgfpathlineto{\pgfqpoint{3.075002in}{3.033379in}}%
\pgfpathlineto{\pgfqpoint{3.075002in}{3.030430in}}%
\pgfpathmoveto{\pgfqpoint{3.070461in}{3.033379in}}%
\pgfpathlineto{\pgfqpoint{3.070461in}{3.033379in}}%
\pgfpathlineto{\pgfqpoint{3.070461in}{3.036329in}}%
\pgfpathlineto{\pgfqpoint{3.075002in}{3.036329in}}%
\pgfpathlineto{\pgfqpoint{3.075002in}{3.033379in}}%
\pgfpathmoveto{\pgfqpoint{3.070461in}{3.036329in}}%
\pgfpathlineto{\pgfqpoint{3.070461in}{3.036329in}}%
\pgfpathlineto{\pgfqpoint{3.070461in}{3.039278in}}%
\pgfpathlineto{\pgfqpoint{3.075002in}{3.039278in}}%
\pgfpathlineto{\pgfqpoint{3.075002in}{3.036329in}}%
\pgfpathmoveto{\pgfqpoint{3.070461in}{3.039278in}}%
\pgfpathlineto{\pgfqpoint{3.070461in}{3.039278in}}%
\pgfpathlineto{\pgfqpoint{3.070461in}{3.042227in}}%
\pgfpathlineto{\pgfqpoint{3.075002in}{3.042227in}}%
\pgfpathlineto{\pgfqpoint{3.075002in}{3.039278in}}%
\pgfpathmoveto{\pgfqpoint{3.070461in}{3.042227in}}%
\pgfpathlineto{\pgfqpoint{3.070461in}{3.042227in}}%
\pgfpathlineto{\pgfqpoint{3.070461in}{3.045176in}}%
\pgfpathlineto{\pgfqpoint{3.075002in}{3.045176in}}%
\pgfpathlineto{\pgfqpoint{3.075002in}{3.042227in}}%
\pgfpathmoveto{\pgfqpoint{3.070461in}{3.045176in}}%
\pgfpathlineto{\pgfqpoint{3.070461in}{3.045176in}}%
\pgfpathlineto{\pgfqpoint{3.070461in}{3.048125in}}%
\pgfpathlineto{\pgfqpoint{3.075002in}{3.048125in}}%
\pgfpathlineto{\pgfqpoint{3.075002in}{3.045176in}}%
\pgfpathmoveto{\pgfqpoint{3.070461in}{3.048125in}}%
\pgfpathlineto{\pgfqpoint{3.070461in}{3.048125in}}%
\pgfpathlineto{\pgfqpoint{3.070461in}{3.051075in}}%
\pgfpathlineto{\pgfqpoint{3.075002in}{3.051075in}}%
\pgfpathlineto{\pgfqpoint{3.075002in}{3.048125in}}%
\pgfpathmoveto{\pgfqpoint{3.070461in}{3.051075in}}%
\pgfpathlineto{\pgfqpoint{3.070461in}{3.051075in}}%
\pgfpathlineto{\pgfqpoint{3.070461in}{3.054024in}}%
\pgfpathlineto{\pgfqpoint{3.075002in}{3.054024in}}%
\pgfpathlineto{\pgfqpoint{3.075002in}{3.051075in}}%
\pgfpathmoveto{\pgfqpoint{3.070461in}{3.054024in}}%
\pgfpathlineto{\pgfqpoint{3.070461in}{3.054024in}}%
\pgfpathlineto{\pgfqpoint{3.070461in}{3.056973in}}%
\pgfpathlineto{\pgfqpoint{3.075002in}{3.056973in}}%
\pgfpathlineto{\pgfqpoint{3.075002in}{3.054024in}}%
\pgfpathmoveto{\pgfqpoint{3.070461in}{3.056973in}}%
\pgfpathlineto{\pgfqpoint{3.070461in}{3.056973in}}%
\pgfpathlineto{\pgfqpoint{3.070461in}{3.059922in}}%
\pgfpathlineto{\pgfqpoint{3.075002in}{3.059922in}}%
\pgfpathlineto{\pgfqpoint{3.075002in}{3.056973in}}%
\pgfpathmoveto{\pgfqpoint{3.070461in}{3.059922in}}%
\pgfpathlineto{\pgfqpoint{3.070461in}{3.059922in}}%
\pgfpathlineto{\pgfqpoint{3.070461in}{3.062871in}}%
\pgfpathlineto{\pgfqpoint{3.075002in}{3.062871in}}%
\pgfpathlineto{\pgfqpoint{3.075002in}{3.059922in}}%
\pgfpathmoveto{\pgfqpoint{3.070461in}{3.062871in}}%
\pgfpathlineto{\pgfqpoint{3.070461in}{3.062871in}}%
\pgfpathlineto{\pgfqpoint{3.070461in}{3.065820in}}%
\pgfpathlineto{\pgfqpoint{3.075002in}{3.065820in}}%
\pgfpathlineto{\pgfqpoint{3.075002in}{3.062871in}}%
\pgfpathmoveto{\pgfqpoint{3.070461in}{3.065820in}}%
\pgfpathlineto{\pgfqpoint{3.070461in}{3.065820in}}%
\pgfpathlineto{\pgfqpoint{3.070461in}{3.068769in}}%
\pgfpathlineto{\pgfqpoint{3.075002in}{3.068769in}}%
\pgfpathlineto{\pgfqpoint{3.075002in}{3.065820in}}%
\pgfpathmoveto{\pgfqpoint{3.070461in}{3.068769in}}%
\pgfpathlineto{\pgfqpoint{3.070461in}{3.068769in}}%
\pgfpathlineto{\pgfqpoint{3.070461in}{3.071719in}}%
\pgfpathlineto{\pgfqpoint{3.075002in}{3.071719in}}%
\pgfpathlineto{\pgfqpoint{3.075002in}{3.068769in}}%
\pgfpathmoveto{\pgfqpoint{3.070461in}{3.071719in}}%
\pgfpathlineto{\pgfqpoint{3.070461in}{3.071719in}}%
\pgfpathlineto{\pgfqpoint{3.070461in}{3.074668in}}%
\pgfpathlineto{\pgfqpoint{3.075002in}{3.074668in}}%
\pgfpathlineto{\pgfqpoint{3.075002in}{3.071719in}}%
\pgfpathmoveto{\pgfqpoint{3.070461in}{3.074668in}}%
\pgfpathlineto{\pgfqpoint{3.070461in}{3.074668in}}%
\pgfpathlineto{\pgfqpoint{3.070461in}{3.077617in}}%
\pgfpathlineto{\pgfqpoint{3.075002in}{3.077617in}}%
\pgfpathlineto{\pgfqpoint{3.075002in}{3.074668in}}%
\pgfpathmoveto{\pgfqpoint{3.070461in}{3.077617in}}%
\pgfpathlineto{\pgfqpoint{3.070461in}{3.077617in}}%
\pgfpathlineto{\pgfqpoint{3.070461in}{3.080566in}}%
\pgfpathlineto{\pgfqpoint{3.075002in}{3.080566in}}%
\pgfpathlineto{\pgfqpoint{3.075002in}{3.077617in}}%
\pgfpathmoveto{\pgfqpoint{3.070461in}{3.080566in}}%
\pgfpathlineto{\pgfqpoint{3.070461in}{3.080566in}}%
\pgfpathlineto{\pgfqpoint{3.070461in}{3.083515in}}%
\pgfpathlineto{\pgfqpoint{3.075002in}{3.083515in}}%
\pgfpathlineto{\pgfqpoint{3.075002in}{3.080566in}}%
\pgfpathmoveto{\pgfqpoint{3.070461in}{3.083515in}}%
\pgfpathlineto{\pgfqpoint{3.070461in}{3.083515in}}%
\pgfpathlineto{\pgfqpoint{3.070461in}{3.086464in}}%
\pgfpathlineto{\pgfqpoint{3.075002in}{3.086464in}}%
\pgfpathlineto{\pgfqpoint{3.075002in}{3.083515in}}%
\pgfpathmoveto{\pgfqpoint{3.070461in}{3.086464in}}%
\pgfpathlineto{\pgfqpoint{3.070461in}{3.086464in}}%
\pgfpathlineto{\pgfqpoint{3.070461in}{3.089413in}}%
\pgfpathlineto{\pgfqpoint{3.075002in}{3.089413in}}%
\pgfpathlineto{\pgfqpoint{3.075002in}{3.086464in}}%
\pgfpathmoveto{\pgfqpoint{3.070461in}{3.089413in}}%
\pgfpathlineto{\pgfqpoint{3.070461in}{3.089413in}}%
\pgfpathlineto{\pgfqpoint{3.070461in}{3.092363in}}%
\pgfpathlineto{\pgfqpoint{3.075002in}{3.092363in}}%
\pgfpathlineto{\pgfqpoint{3.075002in}{3.089413in}}%
\pgfpathmoveto{\pgfqpoint{3.070461in}{3.092363in}}%
\pgfpathlineto{\pgfqpoint{3.070461in}{3.092363in}}%
\pgfpathlineto{\pgfqpoint{3.070461in}{3.095312in}}%
\pgfpathlineto{\pgfqpoint{3.075002in}{3.095312in}}%
\pgfpathlineto{\pgfqpoint{3.075002in}{3.092363in}}%
\pgfpathmoveto{\pgfqpoint{3.070461in}{3.095312in}}%
\pgfpathlineto{\pgfqpoint{3.070461in}{3.095312in}}%
\pgfpathlineto{\pgfqpoint{3.070461in}{3.098261in}}%
\pgfpathlineto{\pgfqpoint{3.075002in}{3.098261in}}%
\pgfpathlineto{\pgfqpoint{3.075002in}{3.095312in}}%
\pgfpathmoveto{\pgfqpoint{3.070461in}{3.098261in}}%
\pgfpathlineto{\pgfqpoint{3.070461in}{3.098261in}}%
\pgfpathlineto{\pgfqpoint{3.070461in}{3.101210in}}%
\pgfpathlineto{\pgfqpoint{3.075002in}{3.101210in}}%
\pgfpathlineto{\pgfqpoint{3.075002in}{3.098261in}}%
\pgfpathmoveto{\pgfqpoint{3.070461in}{3.101210in}}%
\pgfpathlineto{\pgfqpoint{3.070461in}{3.101210in}}%
\pgfpathlineto{\pgfqpoint{3.070461in}{3.104159in}}%
\pgfpathlineto{\pgfqpoint{3.075002in}{3.104159in}}%
\pgfpathlineto{\pgfqpoint{3.075002in}{3.101210in}}%
\pgfpathmoveto{\pgfqpoint{3.070461in}{3.104159in}}%
\pgfpathlineto{\pgfqpoint{3.070461in}{3.104159in}}%
\pgfpathlineto{\pgfqpoint{3.070461in}{3.107108in}}%
\pgfpathlineto{\pgfqpoint{3.075002in}{3.107108in}}%
\pgfpathlineto{\pgfqpoint{3.075002in}{3.104159in}}%
\pgfpathmoveto{\pgfqpoint{3.070461in}{3.107108in}}%
\pgfpathlineto{\pgfqpoint{3.070461in}{3.107108in}}%
\pgfpathlineto{\pgfqpoint{3.070461in}{3.110058in}}%
\pgfpathlineto{\pgfqpoint{3.075002in}{3.110058in}}%
\pgfpathlineto{\pgfqpoint{3.075002in}{3.107108in}}%
\pgfpathmoveto{\pgfqpoint{3.070461in}{3.110058in}}%
\pgfpathlineto{\pgfqpoint{3.070461in}{3.110058in}}%
\pgfpathlineto{\pgfqpoint{3.070461in}{3.113007in}}%
\pgfpathlineto{\pgfqpoint{3.075002in}{3.113007in}}%
\pgfpathlineto{\pgfqpoint{3.075002in}{3.110058in}}%
\pgfpathmoveto{\pgfqpoint{3.070461in}{3.113007in}}%
\pgfpathlineto{\pgfqpoint{3.070461in}{3.113007in}}%
\pgfpathlineto{\pgfqpoint{3.070461in}{3.115956in}}%
\pgfpathlineto{\pgfqpoint{3.075002in}{3.115956in}}%
\pgfpathlineto{\pgfqpoint{3.075002in}{3.113007in}}%
\pgfpathmoveto{\pgfqpoint{3.070461in}{3.115956in}}%
\pgfpathlineto{\pgfqpoint{3.070461in}{3.115956in}}%
\pgfpathlineto{\pgfqpoint{3.070461in}{3.118905in}}%
\pgfpathlineto{\pgfqpoint{3.075002in}{3.118905in}}%
\pgfpathlineto{\pgfqpoint{3.075002in}{3.115956in}}%
\pgfpathmoveto{\pgfqpoint{3.070461in}{3.118905in}}%
\pgfpathlineto{\pgfqpoint{3.070461in}{3.118905in}}%
\pgfpathlineto{\pgfqpoint{3.070461in}{3.121854in}}%
\pgfpathlineto{\pgfqpoint{3.075002in}{3.121854in}}%
\pgfpathlineto{\pgfqpoint{3.075002in}{3.118905in}}%
\pgfpathmoveto{\pgfqpoint{3.070461in}{3.121854in}}%
\pgfpathlineto{\pgfqpoint{3.070461in}{3.121854in}}%
\pgfpathlineto{\pgfqpoint{3.070461in}{3.124803in}}%
\pgfpathlineto{\pgfqpoint{3.075002in}{3.124803in}}%
\pgfpathlineto{\pgfqpoint{3.075002in}{3.121854in}}%
\pgfpathmoveto{\pgfqpoint{3.070461in}{3.124803in}}%
\pgfpathlineto{\pgfqpoint{3.070461in}{3.124803in}}%
\pgfpathlineto{\pgfqpoint{3.070461in}{3.127752in}}%
\pgfpathlineto{\pgfqpoint{3.075002in}{3.127752in}}%
\pgfpathlineto{\pgfqpoint{3.075002in}{3.124803in}}%
\pgfpathmoveto{\pgfqpoint{3.070461in}{3.127752in}}%
\pgfpathlineto{\pgfqpoint{3.070461in}{3.127752in}}%
\pgfpathlineto{\pgfqpoint{3.070461in}{3.130702in}}%
\pgfpathlineto{\pgfqpoint{3.075002in}{3.130702in}}%
\pgfpathlineto{\pgfqpoint{3.075002in}{3.127752in}}%
\pgfpathmoveto{\pgfqpoint{3.070461in}{3.130702in}}%
\pgfpathlineto{\pgfqpoint{3.070461in}{3.130702in}}%
\pgfpathlineto{\pgfqpoint{3.070461in}{3.133651in}}%
\pgfpathlineto{\pgfqpoint{3.075002in}{3.133651in}}%
\pgfpathlineto{\pgfqpoint{3.075002in}{3.130702in}}%
\pgfpathmoveto{\pgfqpoint{3.070461in}{3.133651in}}%
\pgfpathlineto{\pgfqpoint{3.070461in}{3.133651in}}%
\pgfpathlineto{\pgfqpoint{3.070461in}{3.136600in}}%
\pgfpathlineto{\pgfqpoint{3.075002in}{3.136600in}}%
\pgfpathlineto{\pgfqpoint{3.075002in}{3.133651in}}%
\pgfpathmoveto{\pgfqpoint{3.070461in}{3.136600in}}%
\pgfpathlineto{\pgfqpoint{3.070461in}{3.136600in}}%
\pgfpathlineto{\pgfqpoint{3.070461in}{3.139549in}}%
\pgfpathlineto{\pgfqpoint{3.075002in}{3.139549in}}%
\pgfpathlineto{\pgfqpoint{3.075002in}{3.136600in}}%
\pgfpathmoveto{\pgfqpoint{3.070461in}{3.139549in}}%
\pgfpathlineto{\pgfqpoint{3.070461in}{3.139549in}}%
\pgfpathlineto{\pgfqpoint{3.070461in}{3.142498in}}%
\pgfpathlineto{\pgfqpoint{3.075002in}{3.142498in}}%
\pgfpathlineto{\pgfqpoint{3.075002in}{3.139549in}}%
\pgfpathmoveto{\pgfqpoint{3.070461in}{3.142498in}}%
\pgfpathlineto{\pgfqpoint{3.070461in}{3.142498in}}%
\pgfpathlineto{\pgfqpoint{3.070461in}{3.145447in}}%
\pgfpathlineto{\pgfqpoint{3.075002in}{3.145447in}}%
\pgfpathlineto{\pgfqpoint{3.075002in}{3.142498in}}%
\pgfpathmoveto{\pgfqpoint{3.070461in}{3.145447in}}%
\pgfpathlineto{\pgfqpoint{3.070461in}{3.145447in}}%
\pgfpathlineto{\pgfqpoint{3.070461in}{3.148397in}}%
\pgfpathlineto{\pgfqpoint{3.075002in}{3.148397in}}%
\pgfpathlineto{\pgfqpoint{3.075002in}{3.145447in}}%
\pgfpathmoveto{\pgfqpoint{3.070461in}{3.148397in}}%
\pgfpathlineto{\pgfqpoint{3.070461in}{3.148397in}}%
\pgfpathlineto{\pgfqpoint{3.070461in}{3.151346in}}%
\pgfpathlineto{\pgfqpoint{3.075002in}{3.151346in}}%
\pgfpathlineto{\pgfqpoint{3.075002in}{3.148397in}}%
\pgfpathmoveto{\pgfqpoint{3.070461in}{3.151346in}}%
\pgfpathlineto{\pgfqpoint{3.070461in}{3.151346in}}%
\pgfpathlineto{\pgfqpoint{3.070461in}{3.154295in}}%
\pgfpathlineto{\pgfqpoint{3.075002in}{3.154295in}}%
\pgfpathlineto{\pgfqpoint{3.075002in}{3.151346in}}%
\pgfpathmoveto{\pgfqpoint{3.070461in}{3.154295in}}%
\pgfpathlineto{\pgfqpoint{3.070461in}{3.154295in}}%
\pgfpathlineto{\pgfqpoint{3.070461in}{3.157245in}}%
\pgfpathlineto{\pgfqpoint{3.075002in}{3.157245in}}%
\pgfpathlineto{\pgfqpoint{3.075002in}{3.154295in}}%
\pgfpathmoveto{\pgfqpoint{3.070461in}{3.157245in}}%
\pgfpathlineto{\pgfqpoint{3.070461in}{3.157245in}}%
\pgfpathlineto{\pgfqpoint{3.070461in}{3.160194in}}%
\pgfpathlineto{\pgfqpoint{3.075002in}{3.160194in}}%
\pgfpathlineto{\pgfqpoint{3.075002in}{3.157245in}}%
\pgfpathmoveto{\pgfqpoint{3.070461in}{3.160194in}}%
\pgfpathlineto{\pgfqpoint{3.070461in}{3.160194in}}%
\pgfpathlineto{\pgfqpoint{3.070461in}{3.163143in}}%
\pgfpathlineto{\pgfqpoint{3.075002in}{3.163143in}}%
\pgfpathlineto{\pgfqpoint{3.075002in}{3.160194in}}%
\pgfpathmoveto{\pgfqpoint{3.070461in}{3.163143in}}%
\pgfpathlineto{\pgfqpoint{3.070461in}{3.163143in}}%
\pgfpathlineto{\pgfqpoint{3.070461in}{3.166092in}}%
\pgfpathlineto{\pgfqpoint{3.075002in}{3.166092in}}%
\pgfpathlineto{\pgfqpoint{3.075002in}{3.163143in}}%
\pgfpathmoveto{\pgfqpoint{3.070461in}{3.166092in}}%
\pgfpathlineto{\pgfqpoint{3.070461in}{3.166092in}}%
\pgfpathlineto{\pgfqpoint{3.070461in}{3.169042in}}%
\pgfpathlineto{\pgfqpoint{3.075002in}{3.169042in}}%
\pgfpathlineto{\pgfqpoint{3.075002in}{3.166092in}}%
\pgfpathmoveto{\pgfqpoint{3.070461in}{3.169042in}}%
\pgfpathlineto{\pgfqpoint{3.070461in}{3.169042in}}%
\pgfpathlineto{\pgfqpoint{3.070461in}{3.171991in}}%
\pgfpathlineto{\pgfqpoint{3.075002in}{3.171991in}}%
\pgfpathlineto{\pgfqpoint{3.075002in}{3.169042in}}%
\pgfpathmoveto{\pgfqpoint{3.070461in}{3.171991in}}%
\pgfpathlineto{\pgfqpoint{3.070461in}{3.171991in}}%
\pgfpathlineto{\pgfqpoint{3.070461in}{3.174940in}}%
\pgfpathlineto{\pgfqpoint{3.075002in}{3.174940in}}%
\pgfpathlineto{\pgfqpoint{3.075002in}{3.171991in}}%
\pgfpathmoveto{\pgfqpoint{3.070461in}{3.174940in}}%
\pgfpathlineto{\pgfqpoint{3.070461in}{3.174940in}}%
\pgfpathlineto{\pgfqpoint{3.070461in}{3.177890in}}%
\pgfpathlineto{\pgfqpoint{3.075002in}{3.177890in}}%
\pgfpathlineto{\pgfqpoint{3.075002in}{3.174940in}}%
\pgfpathmoveto{\pgfqpoint{3.070461in}{3.177890in}}%
\pgfpathlineto{\pgfqpoint{3.070461in}{3.177890in}}%
\pgfpathlineto{\pgfqpoint{3.070461in}{3.180839in}}%
\pgfpathlineto{\pgfqpoint{3.075002in}{3.180839in}}%
\pgfpathlineto{\pgfqpoint{3.075002in}{3.177890in}}%
\pgfpathmoveto{\pgfqpoint{3.070461in}{3.180839in}}%
\pgfpathlineto{\pgfqpoint{3.070461in}{3.180839in}}%
\pgfpathlineto{\pgfqpoint{3.070461in}{3.183788in}}%
\pgfpathlineto{\pgfqpoint{3.075002in}{3.183788in}}%
\pgfpathlineto{\pgfqpoint{3.075002in}{3.180839in}}%
\pgfpathmoveto{\pgfqpoint{3.070461in}{3.183788in}}%
\pgfpathlineto{\pgfqpoint{3.070461in}{3.183788in}}%
\pgfpathlineto{\pgfqpoint{3.070461in}{3.186737in}}%
\pgfpathlineto{\pgfqpoint{3.075002in}{3.186737in}}%
\pgfpathlineto{\pgfqpoint{3.075002in}{3.183788in}}%
\pgfpathmoveto{\pgfqpoint{3.070461in}{3.186737in}}%
\pgfpathlineto{\pgfqpoint{3.070461in}{3.186737in}}%
\pgfpathlineto{\pgfqpoint{3.070461in}{3.189687in}}%
\pgfpathlineto{\pgfqpoint{3.075002in}{3.189687in}}%
\pgfpathlineto{\pgfqpoint{3.075002in}{3.186737in}}%
\pgfpathmoveto{\pgfqpoint{3.070461in}{3.189687in}}%
\pgfpathlineto{\pgfqpoint{3.070461in}{3.189687in}}%
\pgfpathlineto{\pgfqpoint{3.070461in}{3.192636in}}%
\pgfpathlineto{\pgfqpoint{3.075002in}{3.192636in}}%
\pgfpathlineto{\pgfqpoint{3.075002in}{3.189687in}}%
\pgfpathmoveto{\pgfqpoint{3.070461in}{3.192636in}}%
\pgfpathlineto{\pgfqpoint{3.070461in}{3.192636in}}%
\pgfpathlineto{\pgfqpoint{3.070461in}{3.195585in}}%
\pgfpathlineto{\pgfqpoint{3.075002in}{3.195585in}}%
\pgfpathlineto{\pgfqpoint{3.075002in}{3.192636in}}%
\pgfpathmoveto{\pgfqpoint{3.070461in}{3.195585in}}%
\pgfpathlineto{\pgfqpoint{3.070461in}{3.195585in}}%
\pgfpathlineto{\pgfqpoint{3.070461in}{3.198535in}}%
\pgfpathlineto{\pgfqpoint{3.075002in}{3.198535in}}%
\pgfpathlineto{\pgfqpoint{3.075002in}{3.195585in}}%
\pgfpathmoveto{\pgfqpoint{3.070461in}{3.198535in}}%
\pgfpathlineto{\pgfqpoint{3.070461in}{3.198535in}}%
\pgfpathlineto{\pgfqpoint{3.070461in}{3.201484in}}%
\pgfpathlineto{\pgfqpoint{3.075002in}{3.201484in}}%
\pgfpathlineto{\pgfqpoint{3.075002in}{3.198535in}}%
\pgfpathmoveto{\pgfqpoint{3.070461in}{3.201484in}}%
\pgfpathlineto{\pgfqpoint{3.070461in}{3.201484in}}%
\pgfpathlineto{\pgfqpoint{3.070461in}{3.204433in}}%
\pgfpathlineto{\pgfqpoint{3.075002in}{3.204433in}}%
\pgfpathlineto{\pgfqpoint{3.075002in}{3.201484in}}%
\pgfpathmoveto{\pgfqpoint{3.070461in}{3.204433in}}%
\pgfpathlineto{\pgfqpoint{3.070461in}{3.204433in}}%
\pgfpathlineto{\pgfqpoint{3.070461in}{3.207383in}}%
\pgfpathlineto{\pgfqpoint{3.075002in}{3.207383in}}%
\pgfpathlineto{\pgfqpoint{3.075002in}{3.204433in}}%
\pgfpathmoveto{\pgfqpoint{3.070461in}{3.207383in}}%
\pgfpathlineto{\pgfqpoint{3.070461in}{3.207383in}}%
\pgfpathlineto{\pgfqpoint{3.070461in}{3.210332in}}%
\pgfpathlineto{\pgfqpoint{3.075002in}{3.210332in}}%
\pgfpathlineto{\pgfqpoint{3.075002in}{3.207383in}}%
\pgfpathmoveto{\pgfqpoint{3.070461in}{3.210332in}}%
\pgfpathlineto{\pgfqpoint{3.070461in}{3.210332in}}%
\pgfpathlineto{\pgfqpoint{3.070461in}{3.213281in}}%
\pgfpathlineto{\pgfqpoint{3.075002in}{3.213281in}}%
\pgfpathlineto{\pgfqpoint{3.075002in}{3.210332in}}%
\pgfpathmoveto{\pgfqpoint{3.070461in}{3.213281in}}%
\pgfpathlineto{\pgfqpoint{3.070461in}{3.213281in}}%
\pgfpathlineto{\pgfqpoint{3.070461in}{3.216230in}}%
\pgfpathlineto{\pgfqpoint{3.075002in}{3.216230in}}%
\pgfpathlineto{\pgfqpoint{3.075002in}{3.213281in}}%
\pgfpathmoveto{\pgfqpoint{3.070461in}{3.216230in}}%
\pgfpathlineto{\pgfqpoint{3.070461in}{3.216230in}}%
\pgfpathlineto{\pgfqpoint{3.070461in}{3.219180in}}%
\pgfpathlineto{\pgfqpoint{3.075002in}{3.219180in}}%
\pgfpathlineto{\pgfqpoint{3.075002in}{3.216230in}}%
\pgfpathmoveto{\pgfqpoint{3.070461in}{3.219180in}}%
\pgfpathlineto{\pgfqpoint{3.070461in}{3.219180in}}%
\pgfpathlineto{\pgfqpoint{3.070461in}{3.222129in}}%
\pgfpathlineto{\pgfqpoint{3.075002in}{3.222129in}}%
\pgfpathlineto{\pgfqpoint{3.075002in}{3.219180in}}%
\pgfpathmoveto{\pgfqpoint{3.070461in}{3.222129in}}%
\pgfpathlineto{\pgfqpoint{3.070461in}{3.222129in}}%
\pgfpathlineto{\pgfqpoint{3.070461in}{3.225078in}}%
\pgfpathlineto{\pgfqpoint{3.075002in}{3.225078in}}%
\pgfpathlineto{\pgfqpoint{3.075002in}{3.222129in}}%
\pgfpathmoveto{\pgfqpoint{3.070461in}{3.225078in}}%
\pgfpathlineto{\pgfqpoint{3.070461in}{3.225078in}}%
\pgfpathlineto{\pgfqpoint{3.070461in}{3.228028in}}%
\pgfpathlineto{\pgfqpoint{3.075002in}{3.228028in}}%
\pgfpathlineto{\pgfqpoint{3.075002in}{3.225078in}}%
\pgfpathmoveto{\pgfqpoint{3.070461in}{3.228028in}}%
\pgfpathlineto{\pgfqpoint{3.070461in}{3.228028in}}%
\pgfpathlineto{\pgfqpoint{3.070461in}{3.230977in}}%
\pgfpathlineto{\pgfqpoint{3.075002in}{3.230977in}}%
\pgfpathlineto{\pgfqpoint{3.075002in}{3.228028in}}%
\pgfpathmoveto{\pgfqpoint{3.070461in}{3.230977in}}%
\pgfpathlineto{\pgfqpoint{3.070461in}{3.230977in}}%
\pgfpathlineto{\pgfqpoint{3.070461in}{3.233926in}}%
\pgfpathlineto{\pgfqpoint{3.075002in}{3.233926in}}%
\pgfpathlineto{\pgfqpoint{3.075002in}{3.230977in}}%
\pgfpathmoveto{\pgfqpoint{3.070461in}{3.233926in}}%
\pgfpathlineto{\pgfqpoint{3.070461in}{3.233926in}}%
\pgfpathlineto{\pgfqpoint{3.070461in}{3.236875in}}%
\pgfpathlineto{\pgfqpoint{3.075002in}{3.236875in}}%
\pgfpathlineto{\pgfqpoint{3.075002in}{3.233926in}}%
\pgfpathmoveto{\pgfqpoint{3.070461in}{3.236875in}}%
\pgfpathlineto{\pgfqpoint{3.070461in}{3.236875in}}%
\pgfpathlineto{\pgfqpoint{3.070461in}{3.239825in}}%
\pgfpathlineto{\pgfqpoint{3.075002in}{3.239825in}}%
\pgfpathlineto{\pgfqpoint{3.075002in}{3.236875in}}%
\pgfpathmoveto{\pgfqpoint{3.070461in}{3.239825in}}%
\pgfpathlineto{\pgfqpoint{3.070461in}{3.239825in}}%
\pgfpathlineto{\pgfqpoint{3.070461in}{3.242774in}}%
\pgfpathlineto{\pgfqpoint{3.075002in}{3.242774in}}%
\pgfpathlineto{\pgfqpoint{3.075002in}{3.239825in}}%
\pgfpathmoveto{\pgfqpoint{3.070461in}{3.242774in}}%
\pgfpathlineto{\pgfqpoint{3.070461in}{3.242774in}}%
\pgfpathlineto{\pgfqpoint{3.070461in}{3.245723in}}%
\pgfpathlineto{\pgfqpoint{3.075002in}{3.245723in}}%
\pgfpathlineto{\pgfqpoint{3.075002in}{3.242774in}}%
\pgfpathmoveto{\pgfqpoint{3.070461in}{3.245723in}}%
\pgfpathlineto{\pgfqpoint{3.070461in}{3.245723in}}%
\pgfpathlineto{\pgfqpoint{3.070461in}{3.248672in}}%
\pgfpathlineto{\pgfqpoint{3.075002in}{3.248672in}}%
\pgfpathlineto{\pgfqpoint{3.075002in}{3.245723in}}%
\pgfpathmoveto{\pgfqpoint{3.070461in}{3.248672in}}%
\pgfpathlineto{\pgfqpoint{3.070461in}{3.248672in}}%
\pgfpathlineto{\pgfqpoint{3.070461in}{3.251622in}}%
\pgfpathlineto{\pgfqpoint{3.075002in}{3.251622in}}%
\pgfpathlineto{\pgfqpoint{3.075002in}{3.248672in}}%
\pgfpathmoveto{\pgfqpoint{3.070461in}{3.251622in}}%
\pgfpathlineto{\pgfqpoint{3.070461in}{3.251622in}}%
\pgfpathlineto{\pgfqpoint{3.070461in}{3.254571in}}%
\pgfpathlineto{\pgfqpoint{3.075002in}{3.254571in}}%
\pgfpathlineto{\pgfqpoint{3.075002in}{3.251622in}}%
\pgfpathmoveto{\pgfqpoint{3.070461in}{3.254571in}}%
\pgfpathlineto{\pgfqpoint{3.070461in}{3.254571in}}%
\pgfpathlineto{\pgfqpoint{3.070461in}{3.257520in}}%
\pgfpathlineto{\pgfqpoint{3.075002in}{3.257520in}}%
\pgfpathlineto{\pgfqpoint{3.075002in}{3.254571in}}%
\pgfpathmoveto{\pgfqpoint{3.070461in}{3.257520in}}%
\pgfpathlineto{\pgfqpoint{3.070461in}{3.257520in}}%
\pgfpathlineto{\pgfqpoint{3.070461in}{3.260470in}}%
\pgfpathlineto{\pgfqpoint{3.075002in}{3.260470in}}%
\pgfpathlineto{\pgfqpoint{3.075002in}{3.257520in}}%
\pgfpathmoveto{\pgfqpoint{3.070461in}{3.260470in}}%
\pgfpathlineto{\pgfqpoint{3.070461in}{3.260470in}}%
\pgfpathlineto{\pgfqpoint{3.070461in}{3.263419in}}%
\pgfpathlineto{\pgfqpoint{3.075002in}{3.263419in}}%
\pgfpathlineto{\pgfqpoint{3.075002in}{3.260470in}}%
\pgfpathmoveto{\pgfqpoint{3.070461in}{3.263419in}}%
\pgfpathlineto{\pgfqpoint{3.070461in}{3.263419in}}%
\pgfpathlineto{\pgfqpoint{3.070461in}{3.266368in}}%
\pgfpathlineto{\pgfqpoint{3.075002in}{3.266368in}}%
\pgfpathlineto{\pgfqpoint{3.075002in}{3.263419in}}%
\pgfpathmoveto{\pgfqpoint{3.070461in}{3.266368in}}%
\pgfpathlineto{\pgfqpoint{3.070461in}{3.266368in}}%
\pgfpathlineto{\pgfqpoint{3.070461in}{3.269317in}}%
\pgfpathlineto{\pgfqpoint{3.075002in}{3.269317in}}%
\pgfpathlineto{\pgfqpoint{3.075002in}{3.266368in}}%
\pgfpathmoveto{\pgfqpoint{3.070461in}{3.269317in}}%
\pgfpathlineto{\pgfqpoint{3.070461in}{3.269317in}}%
\pgfpathlineto{\pgfqpoint{3.070461in}{3.272267in}}%
\pgfpathlineto{\pgfqpoint{3.075002in}{3.272267in}}%
\pgfpathlineto{\pgfqpoint{3.075002in}{3.269317in}}%
\pgfpathmoveto{\pgfqpoint{3.070461in}{3.272267in}}%
\pgfpathlineto{\pgfqpoint{3.070461in}{3.272267in}}%
\pgfpathlineto{\pgfqpoint{3.070461in}{3.275216in}}%
\pgfpathlineto{\pgfqpoint{3.075002in}{3.275216in}}%
\pgfpathlineto{\pgfqpoint{3.075002in}{3.272267in}}%
\pgfpathmoveto{\pgfqpoint{3.070461in}{3.275216in}}%
\pgfpathlineto{\pgfqpoint{3.070461in}{3.275216in}}%
\pgfpathlineto{\pgfqpoint{3.070461in}{3.278165in}}%
\pgfpathlineto{\pgfqpoint{3.075002in}{3.278165in}}%
\pgfpathlineto{\pgfqpoint{3.075002in}{3.275216in}}%
\pgfpathmoveto{\pgfqpoint{3.070461in}{3.278165in}}%
\pgfpathlineto{\pgfqpoint{3.070461in}{3.278165in}}%
\pgfpathlineto{\pgfqpoint{3.070461in}{3.281114in}}%
\pgfpathlineto{\pgfqpoint{3.075002in}{3.281114in}}%
\pgfpathlineto{\pgfqpoint{3.075002in}{3.278165in}}%
\pgfpathmoveto{\pgfqpoint{3.070461in}{3.281114in}}%
\pgfpathlineto{\pgfqpoint{3.070461in}{3.281114in}}%
\pgfpathlineto{\pgfqpoint{3.070461in}{3.284064in}}%
\pgfpathlineto{\pgfqpoint{3.075002in}{3.284064in}}%
\pgfpathlineto{\pgfqpoint{3.075002in}{3.281114in}}%
\pgfpathmoveto{\pgfqpoint{3.070461in}{3.284064in}}%
\pgfpathlineto{\pgfqpoint{3.070461in}{3.284064in}}%
\pgfpathlineto{\pgfqpoint{3.070461in}{3.287013in}}%
\pgfpathlineto{\pgfqpoint{3.075002in}{3.287013in}}%
\pgfpathlineto{\pgfqpoint{3.075002in}{3.284064in}}%
\pgfpathmoveto{\pgfqpoint{3.070461in}{3.287013in}}%
\pgfpathlineto{\pgfqpoint{3.070461in}{3.287013in}}%
\pgfpathlineto{\pgfqpoint{3.070461in}{3.289962in}}%
\pgfpathlineto{\pgfqpoint{3.075002in}{3.289962in}}%
\pgfpathlineto{\pgfqpoint{3.075002in}{3.287013in}}%
\pgfpathmoveto{\pgfqpoint{3.070461in}{3.289962in}}%
\pgfpathlineto{\pgfqpoint{3.070461in}{3.289962in}}%
\pgfpathlineto{\pgfqpoint{3.070461in}{3.292911in}}%
\pgfpathlineto{\pgfqpoint{3.075002in}{3.292911in}}%
\pgfpathlineto{\pgfqpoint{3.075002in}{3.289962in}}%
\pgfpathmoveto{\pgfqpoint{3.070461in}{3.292911in}}%
\pgfpathlineto{\pgfqpoint{3.070461in}{3.292911in}}%
\pgfpathlineto{\pgfqpoint{3.070461in}{3.295861in}}%
\pgfpathlineto{\pgfqpoint{3.075002in}{3.295861in}}%
\pgfpathlineto{\pgfqpoint{3.075002in}{3.292911in}}%
\pgfpathmoveto{\pgfqpoint{3.070461in}{3.295861in}}%
\pgfpathlineto{\pgfqpoint{3.070461in}{3.295861in}}%
\pgfpathlineto{\pgfqpoint{3.070461in}{3.298810in}}%
\pgfpathlineto{\pgfqpoint{3.075002in}{3.298810in}}%
\pgfpathlineto{\pgfqpoint{3.075002in}{3.295861in}}%
\pgfpathmoveto{\pgfqpoint{3.070461in}{3.298810in}}%
\pgfpathlineto{\pgfqpoint{3.070461in}{3.298810in}}%
\pgfpathlineto{\pgfqpoint{3.070461in}{3.301759in}}%
\pgfpathlineto{\pgfqpoint{3.075002in}{3.301759in}}%
\pgfpathlineto{\pgfqpoint{3.075002in}{3.298810in}}%
\pgfpathmoveto{\pgfqpoint{3.070461in}{3.301759in}}%
\pgfpathlineto{\pgfqpoint{3.070461in}{3.301759in}}%
\pgfpathlineto{\pgfqpoint{3.070461in}{3.304709in}}%
\pgfpathlineto{\pgfqpoint{3.075002in}{3.304709in}}%
\pgfpathlineto{\pgfqpoint{3.075002in}{3.301759in}}%
\pgfpathmoveto{\pgfqpoint{3.070461in}{3.304709in}}%
\pgfpathlineto{\pgfqpoint{3.070461in}{3.304709in}}%
\pgfpathlineto{\pgfqpoint{3.070461in}{3.307658in}}%
\pgfpathlineto{\pgfqpoint{3.075002in}{3.307658in}}%
\pgfpathlineto{\pgfqpoint{3.075002in}{3.304709in}}%
\pgfpathmoveto{\pgfqpoint{3.070461in}{3.307658in}}%
\pgfpathlineto{\pgfqpoint{3.070461in}{3.307658in}}%
\pgfpathlineto{\pgfqpoint{3.070461in}{3.310607in}}%
\pgfpathlineto{\pgfqpoint{3.075002in}{3.310607in}}%
\pgfpathlineto{\pgfqpoint{3.075002in}{3.307658in}}%
\pgfpathmoveto{\pgfqpoint{3.070461in}{3.310607in}}%
\pgfpathlineto{\pgfqpoint{3.070461in}{3.310607in}}%
\pgfpathlineto{\pgfqpoint{3.070461in}{3.313556in}}%
\pgfpathlineto{\pgfqpoint{3.075002in}{3.313556in}}%
\pgfpathlineto{\pgfqpoint{3.075002in}{3.310607in}}%
\pgfpathmoveto{\pgfqpoint{3.070461in}{3.313556in}}%
\pgfpathlineto{\pgfqpoint{3.070461in}{3.313556in}}%
\pgfpathlineto{\pgfqpoint{3.070461in}{3.316506in}}%
\pgfpathlineto{\pgfqpoint{3.075002in}{3.316506in}}%
\pgfpathlineto{\pgfqpoint{3.075002in}{3.313556in}}%
\pgfpathmoveto{\pgfqpoint{3.070461in}{3.316506in}}%
\pgfpathlineto{\pgfqpoint{3.070461in}{3.316506in}}%
\pgfpathlineto{\pgfqpoint{3.070461in}{3.319455in}}%
\pgfpathlineto{\pgfqpoint{3.075002in}{3.319455in}}%
\pgfpathlineto{\pgfqpoint{3.075002in}{3.316506in}}%
\pgfpathmoveto{\pgfqpoint{3.070461in}{3.319455in}}%
\pgfpathlineto{\pgfqpoint{3.070461in}{3.319455in}}%
\pgfpathlineto{\pgfqpoint{3.070461in}{3.322404in}}%
\pgfpathlineto{\pgfqpoint{3.075002in}{3.322404in}}%
\pgfpathlineto{\pgfqpoint{3.075002in}{3.319455in}}%
\pgfpathmoveto{\pgfqpoint{3.070461in}{3.322404in}}%
\pgfpathlineto{\pgfqpoint{3.070461in}{3.322404in}}%
\pgfpathlineto{\pgfqpoint{3.070461in}{3.325353in}}%
\pgfpathlineto{\pgfqpoint{3.075002in}{3.325353in}}%
\pgfpathlineto{\pgfqpoint{3.075002in}{3.322404in}}%
\pgfpathmoveto{\pgfqpoint{3.070461in}{3.325353in}}%
\pgfpathlineto{\pgfqpoint{3.070461in}{3.325353in}}%
\pgfpathlineto{\pgfqpoint{3.070461in}{3.328303in}}%
\pgfpathlineto{\pgfqpoint{3.075002in}{3.328303in}}%
\pgfpathlineto{\pgfqpoint{3.075002in}{3.325353in}}%
\pgfpathmoveto{\pgfqpoint{3.070461in}{3.328303in}}%
\pgfpathlineto{\pgfqpoint{3.070461in}{3.328303in}}%
\pgfpathlineto{\pgfqpoint{3.070461in}{3.331252in}}%
\pgfpathlineto{\pgfqpoint{3.075002in}{3.331252in}}%
\pgfpathlineto{\pgfqpoint{3.075002in}{3.328303in}}%
\pgfpathmoveto{\pgfqpoint{3.070461in}{3.331252in}}%
\pgfpathlineto{\pgfqpoint{3.070461in}{3.331252in}}%
\pgfpathlineto{\pgfqpoint{3.070461in}{3.334201in}}%
\pgfpathlineto{\pgfqpoint{3.075002in}{3.334201in}}%
\pgfpathlineto{\pgfqpoint{3.075002in}{3.331252in}}%
\pgfpathmoveto{\pgfqpoint{3.070461in}{3.334201in}}%
\pgfpathlineto{\pgfqpoint{3.070461in}{3.334201in}}%
\pgfpathlineto{\pgfqpoint{3.070461in}{3.337150in}}%
\pgfpathlineto{\pgfqpoint{3.075002in}{3.337150in}}%
\pgfpathlineto{\pgfqpoint{3.075002in}{3.334201in}}%
\pgfpathmoveto{\pgfqpoint{3.070461in}{3.337150in}}%
\pgfpathlineto{\pgfqpoint{3.070461in}{3.337150in}}%
\pgfpathlineto{\pgfqpoint{3.070461in}{3.340099in}}%
\pgfpathlineto{\pgfqpoint{3.075002in}{3.340099in}}%
\pgfpathlineto{\pgfqpoint{3.075002in}{3.337150in}}%
\pgfpathmoveto{\pgfqpoint{3.070461in}{3.340099in}}%
\pgfpathlineto{\pgfqpoint{3.070461in}{3.340099in}}%
\pgfpathlineto{\pgfqpoint{3.070461in}{3.343048in}}%
\pgfpathlineto{\pgfqpoint{3.075002in}{3.343048in}}%
\pgfpathlineto{\pgfqpoint{3.075002in}{3.340099in}}%
\pgfpathmoveto{\pgfqpoint{3.070461in}{3.343048in}}%
\pgfpathlineto{\pgfqpoint{3.070461in}{3.343048in}}%
\pgfpathlineto{\pgfqpoint{3.070461in}{3.345998in}}%
\pgfpathlineto{\pgfqpoint{3.075002in}{3.345998in}}%
\pgfpathlineto{\pgfqpoint{3.075002in}{3.343048in}}%
\pgfpathmoveto{\pgfqpoint{3.070461in}{3.345998in}}%
\pgfpathlineto{\pgfqpoint{3.070461in}{3.345998in}}%
\pgfpathlineto{\pgfqpoint{3.070461in}{3.348947in}}%
\pgfpathlineto{\pgfqpoint{3.075002in}{3.348947in}}%
\pgfpathlineto{\pgfqpoint{3.075002in}{3.345998in}}%
\pgfpathmoveto{\pgfqpoint{3.070461in}{3.348947in}}%
\pgfpathlineto{\pgfqpoint{3.070461in}{3.348947in}}%
\pgfpathlineto{\pgfqpoint{3.070461in}{3.351896in}}%
\pgfpathlineto{\pgfqpoint{3.075002in}{3.351896in}}%
\pgfpathlineto{\pgfqpoint{3.075002in}{3.348947in}}%
\pgfpathmoveto{\pgfqpoint{3.070461in}{3.351896in}}%
\pgfpathlineto{\pgfqpoint{3.070461in}{3.351896in}}%
\pgfpathlineto{\pgfqpoint{3.070461in}{3.354845in}}%
\pgfpathlineto{\pgfqpoint{3.075002in}{3.354845in}}%
\pgfpathlineto{\pgfqpoint{3.075002in}{3.351896in}}%
\pgfpathmoveto{\pgfqpoint{3.070461in}{3.354845in}}%
\pgfpathlineto{\pgfqpoint{3.070461in}{3.354845in}}%
\pgfpathlineto{\pgfqpoint{3.070461in}{3.357794in}}%
\pgfpathlineto{\pgfqpoint{3.075002in}{3.357794in}}%
\pgfpathlineto{\pgfqpoint{3.075002in}{3.354845in}}%
\pgfpathmoveto{\pgfqpoint{3.070461in}{3.357794in}}%
\pgfpathlineto{\pgfqpoint{3.070461in}{3.357794in}}%
\pgfpathlineto{\pgfqpoint{3.070461in}{3.360743in}}%
\pgfpathlineto{\pgfqpoint{3.075002in}{3.360743in}}%
\pgfpathlineto{\pgfqpoint{3.075002in}{3.357794in}}%
\pgfpathmoveto{\pgfqpoint{3.070461in}{3.360743in}}%
\pgfpathlineto{\pgfqpoint{3.070461in}{3.360743in}}%
\pgfpathlineto{\pgfqpoint{3.070461in}{3.363692in}}%
\pgfpathlineto{\pgfqpoint{3.075002in}{3.363692in}}%
\pgfpathlineto{\pgfqpoint{3.075002in}{3.360743in}}%
\pgfpathmoveto{\pgfqpoint{3.070461in}{3.363692in}}%
\pgfpathlineto{\pgfqpoint{3.070461in}{3.363692in}}%
\pgfpathlineto{\pgfqpoint{3.070461in}{3.366641in}}%
\pgfpathlineto{\pgfqpoint{3.075002in}{3.366641in}}%
\pgfpathlineto{\pgfqpoint{3.075002in}{3.363692in}}%
\pgfpathmoveto{\pgfqpoint{3.070461in}{3.366641in}}%
\pgfpathlineto{\pgfqpoint{3.070461in}{3.366641in}}%
\pgfpathlineto{\pgfqpoint{3.070461in}{3.369591in}}%
\pgfpathlineto{\pgfqpoint{3.075002in}{3.369591in}}%
\pgfpathlineto{\pgfqpoint{3.075002in}{3.366641in}}%
\pgfpathmoveto{\pgfqpoint{3.070461in}{3.369591in}}%
\pgfpathlineto{\pgfqpoint{3.070461in}{3.369591in}}%
\pgfpathlineto{\pgfqpoint{3.070461in}{3.372540in}}%
\pgfpathlineto{\pgfqpoint{3.075002in}{3.372540in}}%
\pgfpathlineto{\pgfqpoint{3.075002in}{3.369591in}}%
\pgfpathmoveto{\pgfqpoint{3.070461in}{3.372540in}}%
\pgfpathlineto{\pgfqpoint{3.070461in}{3.372540in}}%
\pgfpathlineto{\pgfqpoint{3.070461in}{3.375489in}}%
\pgfpathlineto{\pgfqpoint{3.075002in}{3.375489in}}%
\pgfpathlineto{\pgfqpoint{3.075002in}{3.372540in}}%
\pgfpathmoveto{\pgfqpoint{3.070461in}{3.375489in}}%
\pgfpathlineto{\pgfqpoint{3.070461in}{3.375489in}}%
\pgfpathlineto{\pgfqpoint{3.070461in}{3.378438in}}%
\pgfpathlineto{\pgfqpoint{3.075002in}{3.378438in}}%
\pgfpathlineto{\pgfqpoint{3.075002in}{3.375489in}}%
\pgfpathmoveto{\pgfqpoint{3.070461in}{3.378438in}}%
\pgfpathlineto{\pgfqpoint{3.070461in}{3.378438in}}%
\pgfpathlineto{\pgfqpoint{3.070461in}{3.381387in}}%
\pgfpathlineto{\pgfqpoint{3.075002in}{3.381387in}}%
\pgfpathlineto{\pgfqpoint{3.075002in}{3.378438in}}%
\pgfpathmoveto{\pgfqpoint{3.070461in}{3.381387in}}%
\pgfpathlineto{\pgfqpoint{3.070461in}{3.381387in}}%
\pgfpathlineto{\pgfqpoint{3.070461in}{3.384336in}}%
\pgfpathlineto{\pgfqpoint{3.075002in}{3.384336in}}%
\pgfpathlineto{\pgfqpoint{3.075002in}{3.381387in}}%
\pgfpathmoveto{\pgfqpoint{3.070461in}{3.384336in}}%
\pgfpathlineto{\pgfqpoint{3.070461in}{3.384336in}}%
\pgfpathlineto{\pgfqpoint{3.070461in}{3.387285in}}%
\pgfpathlineto{\pgfqpoint{3.075002in}{3.387285in}}%
\pgfpathlineto{\pgfqpoint{3.075002in}{3.384336in}}%
\pgfpathmoveto{\pgfqpoint{3.070461in}{3.387285in}}%
\pgfpathlineto{\pgfqpoint{3.070461in}{3.387285in}}%
\pgfpathlineto{\pgfqpoint{3.070461in}{3.390234in}}%
\pgfpathlineto{\pgfqpoint{3.075002in}{3.390234in}}%
\pgfpathlineto{\pgfqpoint{3.075002in}{3.387285in}}%
\pgfpathmoveto{\pgfqpoint{3.070461in}{3.390234in}}%
\pgfpathlineto{\pgfqpoint{3.070461in}{3.390234in}}%
\pgfpathlineto{\pgfqpoint{3.070461in}{3.393184in}}%
\pgfpathlineto{\pgfqpoint{3.075002in}{3.393184in}}%
\pgfpathlineto{\pgfqpoint{3.075002in}{3.390234in}}%
\pgfpathmoveto{\pgfqpoint{3.070461in}{3.393184in}}%
\pgfpathlineto{\pgfqpoint{3.070461in}{3.393184in}}%
\pgfpathlineto{\pgfqpoint{3.070461in}{3.396133in}}%
\pgfpathlineto{\pgfqpoint{3.075002in}{3.396133in}}%
\pgfpathlineto{\pgfqpoint{3.075002in}{3.393184in}}%
\pgfpathmoveto{\pgfqpoint{3.070461in}{3.396133in}}%
\pgfpathlineto{\pgfqpoint{3.070461in}{3.396133in}}%
\pgfpathlineto{\pgfqpoint{3.070461in}{3.399082in}}%
\pgfpathlineto{\pgfqpoint{3.075002in}{3.399082in}}%
\pgfpathlineto{\pgfqpoint{3.075002in}{3.396133in}}%
\pgfpathmoveto{\pgfqpoint{3.070461in}{3.399082in}}%
\pgfpathlineto{\pgfqpoint{3.070461in}{3.399082in}}%
\pgfpathlineto{\pgfqpoint{3.070461in}{3.402031in}}%
\pgfpathlineto{\pgfqpoint{3.075002in}{3.402031in}}%
\pgfpathlineto{\pgfqpoint{3.075002in}{3.399082in}}%
\pgfpathmoveto{\pgfqpoint{3.070461in}{3.402031in}}%
\pgfpathlineto{\pgfqpoint{3.070461in}{3.402031in}}%
\pgfpathlineto{\pgfqpoint{3.070461in}{3.404980in}}%
\pgfpathlineto{\pgfqpoint{3.075002in}{3.404980in}}%
\pgfpathlineto{\pgfqpoint{3.075002in}{3.402031in}}%
\pgfpathmoveto{\pgfqpoint{3.070461in}{3.404980in}}%
\pgfpathlineto{\pgfqpoint{3.070461in}{3.404980in}}%
\pgfpathlineto{\pgfqpoint{3.070461in}{3.407929in}}%
\pgfpathlineto{\pgfqpoint{3.075002in}{3.407929in}}%
\pgfpathlineto{\pgfqpoint{3.075002in}{3.404980in}}%
\pgfpathmoveto{\pgfqpoint{3.070461in}{3.407929in}}%
\pgfpathlineto{\pgfqpoint{3.070461in}{3.407929in}}%
\pgfpathlineto{\pgfqpoint{3.070461in}{3.410878in}}%
\pgfpathlineto{\pgfqpoint{3.075002in}{3.410878in}}%
\pgfpathlineto{\pgfqpoint{3.075002in}{3.407929in}}%
\pgfpathmoveto{\pgfqpoint{3.070461in}{3.410878in}}%
\pgfpathlineto{\pgfqpoint{3.070461in}{3.410878in}}%
\pgfpathlineto{\pgfqpoint{3.070461in}{3.413828in}}%
\pgfpathlineto{\pgfqpoint{3.075002in}{3.413828in}}%
\pgfpathlineto{\pgfqpoint{3.075002in}{3.410878in}}%
\pgfpathmoveto{\pgfqpoint{3.070461in}{3.413828in}}%
\pgfpathlineto{\pgfqpoint{3.070461in}{3.413828in}}%
\pgfpathlineto{\pgfqpoint{3.070461in}{3.416777in}}%
\pgfpathlineto{\pgfqpoint{3.075002in}{3.416777in}}%
\pgfpathlineto{\pgfqpoint{3.075002in}{3.413828in}}%
\pgfpathmoveto{\pgfqpoint{3.070461in}{3.416777in}}%
\pgfpathlineto{\pgfqpoint{3.070461in}{3.416777in}}%
\pgfpathlineto{\pgfqpoint{3.070461in}{3.419726in}}%
\pgfpathlineto{\pgfqpoint{3.075002in}{3.419726in}}%
\pgfpathlineto{\pgfqpoint{3.075002in}{3.416777in}}%
\pgfpathmoveto{\pgfqpoint{3.070461in}{3.419726in}}%
\pgfpathlineto{\pgfqpoint{3.070461in}{3.419726in}}%
\pgfpathlineto{\pgfqpoint{3.070461in}{3.422675in}}%
\pgfpathlineto{\pgfqpoint{3.075002in}{3.422675in}}%
\pgfpathlineto{\pgfqpoint{3.075002in}{3.419726in}}%
\pgfpathmoveto{\pgfqpoint{3.070461in}{3.422675in}}%
\pgfpathlineto{\pgfqpoint{3.070461in}{3.422675in}}%
\pgfpathlineto{\pgfqpoint{3.070461in}{3.425624in}}%
\pgfpathlineto{\pgfqpoint{3.075002in}{3.425624in}}%
\pgfpathlineto{\pgfqpoint{3.075002in}{3.422675in}}%
\pgfpathmoveto{\pgfqpoint{3.070461in}{3.425624in}}%
\pgfpathlineto{\pgfqpoint{3.070461in}{3.425624in}}%
\pgfpathlineto{\pgfqpoint{3.070461in}{3.428573in}}%
\pgfpathlineto{\pgfqpoint{3.075002in}{3.428573in}}%
\pgfpathlineto{\pgfqpoint{3.075002in}{3.425624in}}%
\pgfpathmoveto{\pgfqpoint{3.070461in}{3.428573in}}%
\pgfpathlineto{\pgfqpoint{3.070461in}{3.428573in}}%
\pgfpathlineto{\pgfqpoint{3.070461in}{3.431523in}}%
\pgfpathlineto{\pgfqpoint{3.075002in}{3.431523in}}%
\pgfpathlineto{\pgfqpoint{3.075002in}{3.428573in}}%
\pgfpathmoveto{\pgfqpoint{3.070461in}{3.431523in}}%
\pgfpathlineto{\pgfqpoint{3.070461in}{3.431523in}}%
\pgfpathlineto{\pgfqpoint{3.070461in}{3.434472in}}%
\pgfpathlineto{\pgfqpoint{3.075002in}{3.434472in}}%
\pgfpathlineto{\pgfqpoint{3.075002in}{3.431523in}}%
\pgfpathmoveto{\pgfqpoint{3.070461in}{3.434472in}}%
\pgfpathlineto{\pgfqpoint{3.070461in}{3.434472in}}%
\pgfpathlineto{\pgfqpoint{3.070461in}{3.437421in}}%
\pgfpathlineto{\pgfqpoint{3.075002in}{3.437421in}}%
\pgfpathlineto{\pgfqpoint{3.075002in}{3.434472in}}%
\pgfpathmoveto{\pgfqpoint{3.070461in}{3.437421in}}%
\pgfpathlineto{\pgfqpoint{3.070461in}{3.437421in}}%
\pgfpathlineto{\pgfqpoint{3.070461in}{3.440370in}}%
\pgfpathlineto{\pgfqpoint{3.075002in}{3.440370in}}%
\pgfpathlineto{\pgfqpoint{3.075002in}{3.437421in}}%
\pgfpathmoveto{\pgfqpoint{3.070461in}{3.440370in}}%
\pgfpathlineto{\pgfqpoint{3.070461in}{3.440370in}}%
\pgfpathlineto{\pgfqpoint{3.070461in}{3.443319in}}%
\pgfpathlineto{\pgfqpoint{3.075002in}{3.443319in}}%
\pgfpathlineto{\pgfqpoint{3.075002in}{3.440370in}}%
\pgfpathmoveto{\pgfqpoint{3.070461in}{3.443319in}}%
\pgfpathlineto{\pgfqpoint{3.070461in}{3.443319in}}%
\pgfpathlineto{\pgfqpoint{3.070461in}{3.446269in}}%
\pgfpathlineto{\pgfqpoint{3.075002in}{3.446269in}}%
\pgfpathlineto{\pgfqpoint{3.075002in}{3.443319in}}%
\pgfpathmoveto{\pgfqpoint{3.070461in}{3.446269in}}%
\pgfpathlineto{\pgfqpoint{3.070461in}{3.446269in}}%
\pgfpathlineto{\pgfqpoint{3.070461in}{3.449218in}}%
\pgfpathlineto{\pgfqpoint{3.075002in}{3.449218in}}%
\pgfpathlineto{\pgfqpoint{3.075002in}{3.446269in}}%
\pgfpathmoveto{\pgfqpoint{3.070461in}{3.449218in}}%
\pgfpathlineto{\pgfqpoint{3.070461in}{3.449218in}}%
\pgfpathlineto{\pgfqpoint{3.070461in}{3.452167in}}%
\pgfpathlineto{\pgfqpoint{3.075002in}{3.452167in}}%
\pgfpathlineto{\pgfqpoint{3.075002in}{3.449218in}}%
\pgfpathmoveto{\pgfqpoint{3.070461in}{3.452167in}}%
\pgfpathlineto{\pgfqpoint{3.070461in}{3.452167in}}%
\pgfpathlineto{\pgfqpoint{3.070461in}{3.455116in}}%
\pgfpathlineto{\pgfqpoint{3.075002in}{3.455116in}}%
\pgfpathlineto{\pgfqpoint{3.075002in}{3.452167in}}%
\pgfpathmoveto{\pgfqpoint{3.070461in}{3.455116in}}%
\pgfpathlineto{\pgfqpoint{3.070461in}{3.455116in}}%
\pgfpathlineto{\pgfqpoint{3.070461in}{3.458066in}}%
\pgfpathlineto{\pgfqpoint{3.075002in}{3.458066in}}%
\pgfpathlineto{\pgfqpoint{3.075002in}{3.455116in}}%
\pgfpathmoveto{\pgfqpoint{3.070461in}{3.458066in}}%
\pgfpathlineto{\pgfqpoint{3.070461in}{3.458066in}}%
\pgfpathlineto{\pgfqpoint{3.070461in}{3.461015in}}%
\pgfpathlineto{\pgfqpoint{3.075002in}{3.461015in}}%
\pgfpathlineto{\pgfqpoint{3.075002in}{3.458066in}}%
\pgfpathmoveto{\pgfqpoint{3.070461in}{3.461015in}}%
\pgfpathlineto{\pgfqpoint{3.070461in}{3.461015in}}%
\pgfpathlineto{\pgfqpoint{3.070461in}{3.463964in}}%
\pgfpathlineto{\pgfqpoint{3.075002in}{3.463964in}}%
\pgfpathlineto{\pgfqpoint{3.075002in}{3.461015in}}%
\pgfpathmoveto{\pgfqpoint{3.070461in}{3.463964in}}%
\pgfpathlineto{\pgfqpoint{3.070461in}{3.463964in}}%
\pgfpathlineto{\pgfqpoint{3.070461in}{3.466913in}}%
\pgfpathlineto{\pgfqpoint{3.075002in}{3.466913in}}%
\pgfpathlineto{\pgfqpoint{3.075002in}{3.463964in}}%
\pgfpathmoveto{\pgfqpoint{3.070461in}{3.466913in}}%
\pgfpathlineto{\pgfqpoint{3.070461in}{3.466913in}}%
\pgfpathlineto{\pgfqpoint{3.070461in}{3.469863in}}%
\pgfpathlineto{\pgfqpoint{3.075002in}{3.469863in}}%
\pgfpathlineto{\pgfqpoint{3.075002in}{3.466913in}}%
\pgfpathmoveto{\pgfqpoint{3.070461in}{3.469863in}}%
\pgfpathlineto{\pgfqpoint{3.070461in}{3.469863in}}%
\pgfpathlineto{\pgfqpoint{3.070461in}{3.472812in}}%
\pgfpathlineto{\pgfqpoint{3.075002in}{3.472812in}}%
\pgfpathlineto{\pgfqpoint{3.075002in}{3.469863in}}%
\pgfpathmoveto{\pgfqpoint{3.070461in}{3.472812in}}%
\pgfpathlineto{\pgfqpoint{3.070461in}{3.472812in}}%
\pgfpathlineto{\pgfqpoint{3.070461in}{3.475761in}}%
\pgfpathlineto{\pgfqpoint{3.075002in}{3.475761in}}%
\pgfpathlineto{\pgfqpoint{3.075002in}{3.472812in}}%
\pgfpathmoveto{\pgfqpoint{3.070461in}{3.475761in}}%
\pgfpathlineto{\pgfqpoint{3.070461in}{3.475761in}}%
\pgfpathlineto{\pgfqpoint{3.070461in}{3.478710in}}%
\pgfpathlineto{\pgfqpoint{3.075002in}{3.478710in}}%
\pgfpathlineto{\pgfqpoint{3.075002in}{3.475761in}}%
\pgfpathmoveto{\pgfqpoint{3.070461in}{3.478710in}}%
\pgfpathlineto{\pgfqpoint{3.070461in}{3.478710in}}%
\pgfpathlineto{\pgfqpoint{3.070461in}{3.481660in}}%
\pgfpathlineto{\pgfqpoint{3.075002in}{3.481660in}}%
\pgfpathlineto{\pgfqpoint{3.075002in}{3.478710in}}%
\pgfpathmoveto{\pgfqpoint{3.070461in}{3.481660in}}%
\pgfpathlineto{\pgfqpoint{3.070461in}{3.481660in}}%
\pgfpathlineto{\pgfqpoint{3.070461in}{3.484609in}}%
\pgfpathlineto{\pgfqpoint{3.075002in}{3.484609in}}%
\pgfpathlineto{\pgfqpoint{3.075002in}{3.481660in}}%
\pgfpathmoveto{\pgfqpoint{3.070461in}{3.484609in}}%
\pgfpathlineto{\pgfqpoint{3.070461in}{3.484609in}}%
\pgfpathlineto{\pgfqpoint{3.070461in}{3.487558in}}%
\pgfpathlineto{\pgfqpoint{3.075002in}{3.487558in}}%
\pgfpathlineto{\pgfqpoint{3.075002in}{3.484609in}}%
\pgfpathmoveto{\pgfqpoint{3.070461in}{3.487558in}}%
\pgfpathlineto{\pgfqpoint{3.070461in}{3.487558in}}%
\pgfpathlineto{\pgfqpoint{3.070461in}{3.490507in}}%
\pgfpathlineto{\pgfqpoint{3.075002in}{3.490507in}}%
\pgfpathlineto{\pgfqpoint{3.075002in}{3.487558in}}%
\pgfpathmoveto{\pgfqpoint{3.070461in}{3.490507in}}%
\pgfpathlineto{\pgfqpoint{3.070461in}{3.490507in}}%
\pgfpathlineto{\pgfqpoint{3.070461in}{3.493457in}}%
\pgfpathlineto{\pgfqpoint{3.075002in}{3.493457in}}%
\pgfpathlineto{\pgfqpoint{3.075002in}{3.490507in}}%
\pgfpathmoveto{\pgfqpoint{3.070461in}{3.493457in}}%
\pgfpathlineto{\pgfqpoint{3.070461in}{3.493457in}}%
\pgfpathlineto{\pgfqpoint{3.070461in}{3.496406in}}%
\pgfpathlineto{\pgfqpoint{3.075002in}{3.496406in}}%
\pgfpathlineto{\pgfqpoint{3.075002in}{3.493457in}}%
\pgfpathmoveto{\pgfqpoint{3.070461in}{3.496406in}}%
\pgfpathlineto{\pgfqpoint{3.070461in}{3.496406in}}%
\pgfpathlineto{\pgfqpoint{3.070461in}{3.499355in}}%
\pgfpathlineto{\pgfqpoint{3.075002in}{3.499355in}}%
\pgfpathlineto{\pgfqpoint{3.075002in}{3.496406in}}%
\pgfpathmoveto{\pgfqpoint{3.070461in}{3.499355in}}%
\pgfpathlineto{\pgfqpoint{3.070461in}{3.499355in}}%
\pgfpathlineto{\pgfqpoint{3.070461in}{3.502304in}}%
\pgfpathlineto{\pgfqpoint{3.075002in}{3.502304in}}%
\pgfpathlineto{\pgfqpoint{3.075002in}{3.499355in}}%
\pgfpathmoveto{\pgfqpoint{3.070461in}{3.502304in}}%
\pgfpathlineto{\pgfqpoint{3.070461in}{3.502304in}}%
\pgfpathlineto{\pgfqpoint{3.070461in}{3.505254in}}%
\pgfpathlineto{\pgfqpoint{3.075002in}{3.505254in}}%
\pgfpathlineto{\pgfqpoint{3.075002in}{3.502304in}}%
\pgfpathmoveto{\pgfqpoint{3.070461in}{3.505254in}}%
\pgfpathlineto{\pgfqpoint{3.070461in}{3.505254in}}%
\pgfpathlineto{\pgfqpoint{3.070461in}{3.508203in}}%
\pgfpathlineto{\pgfqpoint{3.075002in}{3.508203in}}%
\pgfpathlineto{\pgfqpoint{3.075002in}{3.505254in}}%
\pgfpathmoveto{\pgfqpoint{3.070461in}{3.508203in}}%
\pgfpathlineto{\pgfqpoint{3.070461in}{3.508203in}}%
\pgfpathlineto{\pgfqpoint{3.070461in}{3.511152in}}%
\pgfpathlineto{\pgfqpoint{3.075002in}{3.511152in}}%
\pgfpathlineto{\pgfqpoint{3.075002in}{3.508203in}}%
\pgfpathmoveto{\pgfqpoint{3.070461in}{3.511152in}}%
\pgfpathlineto{\pgfqpoint{3.070461in}{3.511152in}}%
\pgfpathlineto{\pgfqpoint{3.070461in}{3.514101in}}%
\pgfpathlineto{\pgfqpoint{3.075002in}{3.514101in}}%
\pgfpathlineto{\pgfqpoint{3.075002in}{3.511152in}}%
\pgfpathmoveto{\pgfqpoint{3.070461in}{3.514101in}}%
\pgfpathlineto{\pgfqpoint{3.070461in}{3.514101in}}%
\pgfpathlineto{\pgfqpoint{3.070461in}{3.517050in}}%
\pgfpathlineto{\pgfqpoint{3.075002in}{3.517050in}}%
\pgfpathlineto{\pgfqpoint{3.075002in}{3.514101in}}%
\pgfpathmoveto{\pgfqpoint{3.070461in}{3.517050in}}%
\pgfpathlineto{\pgfqpoint{3.070461in}{3.517050in}}%
\pgfpathlineto{\pgfqpoint{3.070461in}{3.520000in}}%
\pgfpathlineto{\pgfqpoint{3.075002in}{3.520000in}}%
\pgfpathlineto{\pgfqpoint{3.075002in}{3.517050in}}%
\pgfpathclose%
\pgfusepath{fill}%
\end{pgfscope}%
\begin{pgfscope}%
\pgfpathrectangle{\pgfqpoint{0.750000in}{0.500000in}}{\pgfqpoint{4.650000in}{3.020000in}}%
\pgfusepath{clip}%
\pgfsetbuttcap%
\pgfsetmiterjoin%
\definecolor{currentfill}{rgb}{1.000000,0.000000,0.000000}%
\pgfsetfillcolor{currentfill}%
\pgfsetlinewidth{0.000000pt}%
\definecolor{currentstroke}{rgb}{0.000000,0.000000,0.000000}%
\pgfsetstrokecolor{currentstroke}%
\pgfsetstrokeopacity{0.000000}%
\pgfsetdash{}{0pt}%
\pgfpathmoveto{\pgfqpoint{0.750004in}{2.007051in}}%
\pgfpathlineto{\pgfqpoint{0.750004in}{2.010001in}}%
\pgfpathlineto{\pgfqpoint{0.754545in}{2.010001in}}%
\pgfpathlineto{\pgfqpoint{0.754545in}{2.007051in}}%
\pgfpathmoveto{\pgfqpoint{0.754545in}{2.007051in}}%
\pgfpathlineto{\pgfqpoint{0.754545in}{2.007051in}}%
\pgfpathlineto{\pgfqpoint{0.754545in}{2.010001in}}%
\pgfpathlineto{\pgfqpoint{0.759086in}{2.010001in}}%
\pgfpathlineto{\pgfqpoint{0.759086in}{2.007051in}}%
\pgfpathmoveto{\pgfqpoint{0.759086in}{2.007051in}}%
\pgfpathlineto{\pgfqpoint{0.759086in}{2.007051in}}%
\pgfpathlineto{\pgfqpoint{0.759086in}{2.010001in}}%
\pgfpathlineto{\pgfqpoint{0.763627in}{2.010001in}}%
\pgfpathlineto{\pgfqpoint{0.763627in}{2.007051in}}%
\pgfpathmoveto{\pgfqpoint{0.763627in}{2.007051in}}%
\pgfpathlineto{\pgfqpoint{0.763627in}{2.007051in}}%
\pgfpathlineto{\pgfqpoint{0.763627in}{2.010001in}}%
\pgfpathlineto{\pgfqpoint{0.768168in}{2.010001in}}%
\pgfpathlineto{\pgfqpoint{0.768168in}{2.007051in}}%
\pgfpathmoveto{\pgfqpoint{0.768168in}{2.007051in}}%
\pgfpathlineto{\pgfqpoint{0.768168in}{2.007051in}}%
\pgfpathlineto{\pgfqpoint{0.768168in}{2.010001in}}%
\pgfpathlineto{\pgfqpoint{0.772709in}{2.010001in}}%
\pgfpathlineto{\pgfqpoint{0.772709in}{2.007051in}}%
\pgfpathmoveto{\pgfqpoint{0.772709in}{2.007051in}}%
\pgfpathlineto{\pgfqpoint{0.772709in}{2.007051in}}%
\pgfpathlineto{\pgfqpoint{0.772709in}{2.010001in}}%
\pgfpathlineto{\pgfqpoint{0.777250in}{2.010001in}}%
\pgfpathlineto{\pgfqpoint{0.777250in}{2.007051in}}%
\pgfpathmoveto{\pgfqpoint{0.777250in}{2.007051in}}%
\pgfpathlineto{\pgfqpoint{0.777250in}{2.007051in}}%
\pgfpathlineto{\pgfqpoint{0.777250in}{2.010001in}}%
\pgfpathlineto{\pgfqpoint{0.781791in}{2.010001in}}%
\pgfpathlineto{\pgfqpoint{0.781791in}{2.007051in}}%
\pgfpathmoveto{\pgfqpoint{0.781791in}{2.007051in}}%
\pgfpathlineto{\pgfqpoint{0.781791in}{2.007051in}}%
\pgfpathlineto{\pgfqpoint{0.781791in}{2.010001in}}%
\pgfpathlineto{\pgfqpoint{0.786332in}{2.010001in}}%
\pgfpathlineto{\pgfqpoint{0.786332in}{2.007051in}}%
\pgfpathmoveto{\pgfqpoint{0.786332in}{2.007051in}}%
\pgfpathlineto{\pgfqpoint{0.786332in}{2.007051in}}%
\pgfpathlineto{\pgfqpoint{0.786332in}{2.010001in}}%
\pgfpathlineto{\pgfqpoint{0.790873in}{2.010001in}}%
\pgfpathlineto{\pgfqpoint{0.790873in}{2.007051in}}%
\pgfpathmoveto{\pgfqpoint{0.790873in}{2.007051in}}%
\pgfpathlineto{\pgfqpoint{0.790873in}{2.007051in}}%
\pgfpathlineto{\pgfqpoint{0.790873in}{2.010001in}}%
\pgfpathlineto{\pgfqpoint{0.795414in}{2.010001in}}%
\pgfpathlineto{\pgfqpoint{0.795414in}{2.007051in}}%
\pgfpathmoveto{\pgfqpoint{0.795414in}{2.007051in}}%
\pgfpathlineto{\pgfqpoint{0.795414in}{2.007051in}}%
\pgfpathlineto{\pgfqpoint{0.795414in}{2.010001in}}%
\pgfpathlineto{\pgfqpoint{0.799954in}{2.010001in}}%
\pgfpathlineto{\pgfqpoint{0.799954in}{2.007051in}}%
\pgfpathmoveto{\pgfqpoint{0.799954in}{2.007051in}}%
\pgfpathlineto{\pgfqpoint{0.799954in}{2.007051in}}%
\pgfpathlineto{\pgfqpoint{0.799954in}{2.010001in}}%
\pgfpathlineto{\pgfqpoint{0.804495in}{2.010001in}}%
\pgfpathlineto{\pgfqpoint{0.804495in}{2.007051in}}%
\pgfpathmoveto{\pgfqpoint{0.804495in}{2.007051in}}%
\pgfpathlineto{\pgfqpoint{0.804495in}{2.007051in}}%
\pgfpathlineto{\pgfqpoint{0.804495in}{2.010001in}}%
\pgfpathlineto{\pgfqpoint{0.809036in}{2.010001in}}%
\pgfpathlineto{\pgfqpoint{0.809036in}{2.007051in}}%
\pgfpathmoveto{\pgfqpoint{0.809036in}{2.007051in}}%
\pgfpathlineto{\pgfqpoint{0.809036in}{2.007051in}}%
\pgfpathlineto{\pgfqpoint{0.809036in}{2.010001in}}%
\pgfpathlineto{\pgfqpoint{0.813577in}{2.010001in}}%
\pgfpathlineto{\pgfqpoint{0.813577in}{2.007051in}}%
\pgfpathmoveto{\pgfqpoint{0.813577in}{2.007051in}}%
\pgfpathlineto{\pgfqpoint{0.813577in}{2.007051in}}%
\pgfpathlineto{\pgfqpoint{0.813577in}{2.010001in}}%
\pgfpathlineto{\pgfqpoint{0.818118in}{2.010001in}}%
\pgfpathlineto{\pgfqpoint{0.818118in}{2.007051in}}%
\pgfpathmoveto{\pgfqpoint{0.818118in}{2.007051in}}%
\pgfpathlineto{\pgfqpoint{0.818118in}{2.007051in}}%
\pgfpathlineto{\pgfqpoint{0.818118in}{2.010001in}}%
\pgfpathlineto{\pgfqpoint{0.822659in}{2.010001in}}%
\pgfpathlineto{\pgfqpoint{0.822659in}{2.007051in}}%
\pgfpathmoveto{\pgfqpoint{0.822659in}{2.007051in}}%
\pgfpathlineto{\pgfqpoint{0.822659in}{2.007051in}}%
\pgfpathlineto{\pgfqpoint{0.822659in}{2.010001in}}%
\pgfpathlineto{\pgfqpoint{0.827200in}{2.010001in}}%
\pgfpathlineto{\pgfqpoint{0.827200in}{2.007051in}}%
\pgfpathmoveto{\pgfqpoint{0.827200in}{2.007051in}}%
\pgfpathlineto{\pgfqpoint{0.827200in}{2.007051in}}%
\pgfpathlineto{\pgfqpoint{0.827200in}{2.010001in}}%
\pgfpathlineto{\pgfqpoint{0.831741in}{2.010001in}}%
\pgfpathlineto{\pgfqpoint{0.831741in}{2.007051in}}%
\pgfpathmoveto{\pgfqpoint{0.831741in}{2.007051in}}%
\pgfpathlineto{\pgfqpoint{0.831741in}{2.007051in}}%
\pgfpathlineto{\pgfqpoint{0.831741in}{2.010001in}}%
\pgfpathlineto{\pgfqpoint{0.836282in}{2.010001in}}%
\pgfpathlineto{\pgfqpoint{0.836282in}{2.007051in}}%
\pgfpathmoveto{\pgfqpoint{0.836282in}{2.007051in}}%
\pgfpathlineto{\pgfqpoint{0.836282in}{2.007051in}}%
\pgfpathlineto{\pgfqpoint{0.836282in}{2.010001in}}%
\pgfpathlineto{\pgfqpoint{0.840823in}{2.010001in}}%
\pgfpathlineto{\pgfqpoint{0.840823in}{2.007051in}}%
\pgfpathmoveto{\pgfqpoint{0.840823in}{2.007051in}}%
\pgfpathlineto{\pgfqpoint{0.840823in}{2.007051in}}%
\pgfpathlineto{\pgfqpoint{0.840823in}{2.010001in}}%
\pgfpathlineto{\pgfqpoint{0.845364in}{2.010001in}}%
\pgfpathlineto{\pgfqpoint{0.845364in}{2.007051in}}%
\pgfpathmoveto{\pgfqpoint{0.845364in}{2.007051in}}%
\pgfpathlineto{\pgfqpoint{0.845364in}{2.007051in}}%
\pgfpathlineto{\pgfqpoint{0.845364in}{2.010001in}}%
\pgfpathlineto{\pgfqpoint{0.849905in}{2.010001in}}%
\pgfpathlineto{\pgfqpoint{0.849905in}{2.007051in}}%
\pgfpathmoveto{\pgfqpoint{0.849905in}{2.007051in}}%
\pgfpathlineto{\pgfqpoint{0.849905in}{2.007051in}}%
\pgfpathlineto{\pgfqpoint{0.849905in}{2.010001in}}%
\pgfpathlineto{\pgfqpoint{0.854446in}{2.010001in}}%
\pgfpathlineto{\pgfqpoint{0.854446in}{2.007051in}}%
\pgfpathmoveto{\pgfqpoint{0.854446in}{2.007051in}}%
\pgfpathlineto{\pgfqpoint{0.854446in}{2.007051in}}%
\pgfpathlineto{\pgfqpoint{0.854446in}{2.010001in}}%
\pgfpathlineto{\pgfqpoint{0.858987in}{2.010001in}}%
\pgfpathlineto{\pgfqpoint{0.858987in}{2.007051in}}%
\pgfpathmoveto{\pgfqpoint{0.858987in}{2.007051in}}%
\pgfpathlineto{\pgfqpoint{0.858987in}{2.007051in}}%
\pgfpathlineto{\pgfqpoint{0.858987in}{2.010001in}}%
\pgfpathlineto{\pgfqpoint{0.863528in}{2.010001in}}%
\pgfpathlineto{\pgfqpoint{0.863528in}{2.007051in}}%
\pgfpathmoveto{\pgfqpoint{0.863528in}{2.007051in}}%
\pgfpathlineto{\pgfqpoint{0.863528in}{2.007051in}}%
\pgfpathlineto{\pgfqpoint{0.863528in}{2.010001in}}%
\pgfpathlineto{\pgfqpoint{0.868069in}{2.010001in}}%
\pgfpathlineto{\pgfqpoint{0.868069in}{2.007051in}}%
\pgfpathmoveto{\pgfqpoint{0.868069in}{2.007051in}}%
\pgfpathlineto{\pgfqpoint{0.868069in}{2.007051in}}%
\pgfpathlineto{\pgfqpoint{0.868069in}{2.010001in}}%
\pgfpathlineto{\pgfqpoint{0.872610in}{2.010001in}}%
\pgfpathlineto{\pgfqpoint{0.872610in}{2.007051in}}%
\pgfpathmoveto{\pgfqpoint{0.872610in}{2.007051in}}%
\pgfpathlineto{\pgfqpoint{0.872610in}{2.007051in}}%
\pgfpathlineto{\pgfqpoint{0.872610in}{2.010001in}}%
\pgfpathlineto{\pgfqpoint{0.877151in}{2.010001in}}%
\pgfpathlineto{\pgfqpoint{0.877151in}{2.007051in}}%
\pgfpathmoveto{\pgfqpoint{0.877151in}{2.007051in}}%
\pgfpathlineto{\pgfqpoint{0.877151in}{2.007051in}}%
\pgfpathlineto{\pgfqpoint{0.877151in}{2.010001in}}%
\pgfpathlineto{\pgfqpoint{0.881692in}{2.010001in}}%
\pgfpathlineto{\pgfqpoint{0.881692in}{2.007051in}}%
\pgfpathmoveto{\pgfqpoint{0.881692in}{2.007051in}}%
\pgfpathlineto{\pgfqpoint{0.881692in}{2.007051in}}%
\pgfpathlineto{\pgfqpoint{0.881692in}{2.010001in}}%
\pgfpathlineto{\pgfqpoint{0.886233in}{2.010001in}}%
\pgfpathlineto{\pgfqpoint{0.886233in}{2.007051in}}%
\pgfpathmoveto{\pgfqpoint{0.886233in}{2.007051in}}%
\pgfpathlineto{\pgfqpoint{0.886233in}{2.007051in}}%
\pgfpathlineto{\pgfqpoint{0.886233in}{2.010001in}}%
\pgfpathlineto{\pgfqpoint{0.890774in}{2.010001in}}%
\pgfpathlineto{\pgfqpoint{0.890774in}{2.007051in}}%
\pgfpathmoveto{\pgfqpoint{0.890774in}{2.007051in}}%
\pgfpathlineto{\pgfqpoint{0.890774in}{2.007051in}}%
\pgfpathlineto{\pgfqpoint{0.890774in}{2.010001in}}%
\pgfpathlineto{\pgfqpoint{0.895315in}{2.010001in}}%
\pgfpathlineto{\pgfqpoint{0.895315in}{2.007051in}}%
\pgfpathmoveto{\pgfqpoint{0.895315in}{2.007051in}}%
\pgfpathlineto{\pgfqpoint{0.895315in}{2.007051in}}%
\pgfpathlineto{\pgfqpoint{0.895315in}{2.010001in}}%
\pgfpathlineto{\pgfqpoint{0.899856in}{2.010001in}}%
\pgfpathlineto{\pgfqpoint{0.899856in}{2.007051in}}%
\pgfpathmoveto{\pgfqpoint{0.899856in}{2.007051in}}%
\pgfpathlineto{\pgfqpoint{0.899856in}{2.007051in}}%
\pgfpathlineto{\pgfqpoint{0.899856in}{2.010001in}}%
\pgfpathlineto{\pgfqpoint{0.904397in}{2.010001in}}%
\pgfpathlineto{\pgfqpoint{0.904397in}{2.007051in}}%
\pgfpathmoveto{\pgfqpoint{0.904397in}{2.007051in}}%
\pgfpathlineto{\pgfqpoint{0.904397in}{2.007051in}}%
\pgfpathlineto{\pgfqpoint{0.904397in}{2.010001in}}%
\pgfpathlineto{\pgfqpoint{0.908938in}{2.010001in}}%
\pgfpathlineto{\pgfqpoint{0.908938in}{2.007051in}}%
\pgfpathmoveto{\pgfqpoint{0.908938in}{2.007051in}}%
\pgfpathlineto{\pgfqpoint{0.908938in}{2.007051in}}%
\pgfpathlineto{\pgfqpoint{0.908938in}{2.010001in}}%
\pgfpathlineto{\pgfqpoint{0.913478in}{2.010001in}}%
\pgfpathlineto{\pgfqpoint{0.913478in}{2.007051in}}%
\pgfpathmoveto{\pgfqpoint{0.913478in}{2.007051in}}%
\pgfpathlineto{\pgfqpoint{0.913478in}{2.007051in}}%
\pgfpathlineto{\pgfqpoint{0.913478in}{2.010001in}}%
\pgfpathlineto{\pgfqpoint{0.918019in}{2.010001in}}%
\pgfpathlineto{\pgfqpoint{0.918019in}{2.007051in}}%
\pgfpathmoveto{\pgfqpoint{0.918019in}{2.007051in}}%
\pgfpathlineto{\pgfqpoint{0.918019in}{2.007051in}}%
\pgfpathlineto{\pgfqpoint{0.918019in}{2.010001in}}%
\pgfpathlineto{\pgfqpoint{0.922560in}{2.010001in}}%
\pgfpathlineto{\pgfqpoint{0.922560in}{2.007051in}}%
\pgfpathmoveto{\pgfqpoint{0.922560in}{2.007051in}}%
\pgfpathlineto{\pgfqpoint{0.922560in}{2.007051in}}%
\pgfpathlineto{\pgfqpoint{0.922560in}{2.010001in}}%
\pgfpathlineto{\pgfqpoint{0.927101in}{2.010001in}}%
\pgfpathlineto{\pgfqpoint{0.927101in}{2.007051in}}%
\pgfpathmoveto{\pgfqpoint{0.927101in}{2.007051in}}%
\pgfpathlineto{\pgfqpoint{0.927101in}{2.007051in}}%
\pgfpathlineto{\pgfqpoint{0.927101in}{2.010001in}}%
\pgfpathlineto{\pgfqpoint{0.931642in}{2.010001in}}%
\pgfpathlineto{\pgfqpoint{0.931642in}{2.007051in}}%
\pgfpathmoveto{\pgfqpoint{0.931642in}{2.007051in}}%
\pgfpathlineto{\pgfqpoint{0.931642in}{2.007051in}}%
\pgfpathlineto{\pgfqpoint{0.931642in}{2.010001in}}%
\pgfpathlineto{\pgfqpoint{0.936183in}{2.010001in}}%
\pgfpathlineto{\pgfqpoint{0.936183in}{2.007051in}}%
\pgfpathmoveto{\pgfqpoint{0.936183in}{2.007051in}}%
\pgfpathlineto{\pgfqpoint{0.936183in}{2.007051in}}%
\pgfpathlineto{\pgfqpoint{0.936183in}{2.010001in}}%
\pgfpathlineto{\pgfqpoint{0.940724in}{2.010001in}}%
\pgfpathlineto{\pgfqpoint{0.940724in}{2.007051in}}%
\pgfpathmoveto{\pgfqpoint{0.940724in}{2.007051in}}%
\pgfpathlineto{\pgfqpoint{0.940724in}{2.007051in}}%
\pgfpathlineto{\pgfqpoint{0.940724in}{2.010001in}}%
\pgfpathlineto{\pgfqpoint{0.945265in}{2.010001in}}%
\pgfpathlineto{\pgfqpoint{0.945265in}{2.007051in}}%
\pgfpathmoveto{\pgfqpoint{0.945265in}{2.007051in}}%
\pgfpathlineto{\pgfqpoint{0.945265in}{2.007051in}}%
\pgfpathlineto{\pgfqpoint{0.945265in}{2.010001in}}%
\pgfpathlineto{\pgfqpoint{0.949806in}{2.010001in}}%
\pgfpathlineto{\pgfqpoint{0.949806in}{2.007051in}}%
\pgfpathmoveto{\pgfqpoint{0.949806in}{2.007051in}}%
\pgfpathlineto{\pgfqpoint{0.949806in}{2.007051in}}%
\pgfpathlineto{\pgfqpoint{0.949806in}{2.010001in}}%
\pgfpathlineto{\pgfqpoint{0.954347in}{2.010001in}}%
\pgfpathlineto{\pgfqpoint{0.954347in}{2.007051in}}%
\pgfpathmoveto{\pgfqpoint{0.954347in}{2.007051in}}%
\pgfpathlineto{\pgfqpoint{0.954347in}{2.007051in}}%
\pgfpathlineto{\pgfqpoint{0.954347in}{2.010001in}}%
\pgfpathlineto{\pgfqpoint{0.958888in}{2.010001in}}%
\pgfpathlineto{\pgfqpoint{0.958888in}{2.007051in}}%
\pgfpathmoveto{\pgfqpoint{0.958888in}{2.007051in}}%
\pgfpathlineto{\pgfqpoint{0.958888in}{2.007051in}}%
\pgfpathlineto{\pgfqpoint{0.958888in}{2.010001in}}%
\pgfpathlineto{\pgfqpoint{0.963428in}{2.010001in}}%
\pgfpathlineto{\pgfqpoint{0.963428in}{2.007051in}}%
\pgfpathmoveto{\pgfqpoint{0.963428in}{2.007051in}}%
\pgfpathlineto{\pgfqpoint{0.963428in}{2.007051in}}%
\pgfpathlineto{\pgfqpoint{0.963428in}{2.010001in}}%
\pgfpathlineto{\pgfqpoint{0.967969in}{2.010001in}}%
\pgfpathlineto{\pgfqpoint{0.967969in}{2.007051in}}%
\pgfpathmoveto{\pgfqpoint{0.967969in}{2.007051in}}%
\pgfpathlineto{\pgfqpoint{0.967969in}{2.007051in}}%
\pgfpathlineto{\pgfqpoint{0.967969in}{2.010001in}}%
\pgfpathlineto{\pgfqpoint{0.972510in}{2.010001in}}%
\pgfpathlineto{\pgfqpoint{0.972510in}{2.007051in}}%
\pgfpathmoveto{\pgfqpoint{0.972510in}{2.007051in}}%
\pgfpathlineto{\pgfqpoint{0.972510in}{2.007051in}}%
\pgfpathlineto{\pgfqpoint{0.972510in}{2.010001in}}%
\pgfpathlineto{\pgfqpoint{0.977051in}{2.010001in}}%
\pgfpathlineto{\pgfqpoint{0.977051in}{2.007051in}}%
\pgfpathmoveto{\pgfqpoint{0.977051in}{2.007051in}}%
\pgfpathlineto{\pgfqpoint{0.977051in}{2.007051in}}%
\pgfpathlineto{\pgfqpoint{0.977051in}{2.010001in}}%
\pgfpathlineto{\pgfqpoint{0.981592in}{2.010001in}}%
\pgfpathlineto{\pgfqpoint{0.981592in}{2.007051in}}%
\pgfpathmoveto{\pgfqpoint{0.981592in}{2.007051in}}%
\pgfpathlineto{\pgfqpoint{0.981592in}{2.007051in}}%
\pgfpathlineto{\pgfqpoint{0.981592in}{2.010001in}}%
\pgfpathlineto{\pgfqpoint{0.986133in}{2.010001in}}%
\pgfpathlineto{\pgfqpoint{0.986133in}{2.007051in}}%
\pgfpathmoveto{\pgfqpoint{0.986133in}{2.007051in}}%
\pgfpathlineto{\pgfqpoint{0.986133in}{2.007051in}}%
\pgfpathlineto{\pgfqpoint{0.986133in}{2.010001in}}%
\pgfpathlineto{\pgfqpoint{0.990674in}{2.010001in}}%
\pgfpathlineto{\pgfqpoint{0.990674in}{2.007051in}}%
\pgfpathmoveto{\pgfqpoint{0.990674in}{2.007051in}}%
\pgfpathlineto{\pgfqpoint{0.990674in}{2.007051in}}%
\pgfpathlineto{\pgfqpoint{0.990674in}{2.010001in}}%
\pgfpathlineto{\pgfqpoint{0.995215in}{2.010001in}}%
\pgfpathlineto{\pgfqpoint{0.995215in}{2.007051in}}%
\pgfpathmoveto{\pgfqpoint{0.995215in}{2.007051in}}%
\pgfpathlineto{\pgfqpoint{0.995215in}{2.007051in}}%
\pgfpathlineto{\pgfqpoint{0.995215in}{2.010001in}}%
\pgfpathlineto{\pgfqpoint{0.999756in}{2.010001in}}%
\pgfpathlineto{\pgfqpoint{0.999756in}{2.007051in}}%
\pgfpathmoveto{\pgfqpoint{0.999756in}{2.007051in}}%
\pgfpathlineto{\pgfqpoint{0.999756in}{2.007051in}}%
\pgfpathlineto{\pgfqpoint{0.999756in}{2.010001in}}%
\pgfpathlineto{\pgfqpoint{1.004297in}{2.010001in}}%
\pgfpathlineto{\pgfqpoint{1.004297in}{2.007051in}}%
\pgfpathmoveto{\pgfqpoint{1.004297in}{2.007051in}}%
\pgfpathlineto{\pgfqpoint{1.004297in}{2.007051in}}%
\pgfpathlineto{\pgfqpoint{1.004297in}{2.010001in}}%
\pgfpathlineto{\pgfqpoint{1.008837in}{2.010001in}}%
\pgfpathlineto{\pgfqpoint{1.008837in}{2.007051in}}%
\pgfpathmoveto{\pgfqpoint{1.008837in}{2.007051in}}%
\pgfpathlineto{\pgfqpoint{1.008837in}{2.007051in}}%
\pgfpathlineto{\pgfqpoint{1.008837in}{2.010001in}}%
\pgfpathlineto{\pgfqpoint{1.013378in}{2.010001in}}%
\pgfpathlineto{\pgfqpoint{1.013378in}{2.007051in}}%
\pgfpathmoveto{\pgfqpoint{1.013378in}{2.007051in}}%
\pgfpathlineto{\pgfqpoint{1.013378in}{2.007051in}}%
\pgfpathlineto{\pgfqpoint{1.013378in}{2.010001in}}%
\pgfpathlineto{\pgfqpoint{1.017919in}{2.010001in}}%
\pgfpathlineto{\pgfqpoint{1.017919in}{2.007051in}}%
\pgfpathmoveto{\pgfqpoint{1.017919in}{2.007051in}}%
\pgfpathlineto{\pgfqpoint{1.017919in}{2.007051in}}%
\pgfpathlineto{\pgfqpoint{1.017919in}{2.010001in}}%
\pgfpathlineto{\pgfqpoint{1.022460in}{2.010001in}}%
\pgfpathlineto{\pgfqpoint{1.022460in}{2.007051in}}%
\pgfpathmoveto{\pgfqpoint{1.022460in}{2.007051in}}%
\pgfpathlineto{\pgfqpoint{1.022460in}{2.007051in}}%
\pgfpathlineto{\pgfqpoint{1.022460in}{2.010001in}}%
\pgfpathlineto{\pgfqpoint{1.027001in}{2.010001in}}%
\pgfpathlineto{\pgfqpoint{1.027001in}{2.007051in}}%
\pgfpathmoveto{\pgfqpoint{1.027001in}{2.007051in}}%
\pgfpathlineto{\pgfqpoint{1.027001in}{2.007051in}}%
\pgfpathlineto{\pgfqpoint{1.027001in}{2.010001in}}%
\pgfpathlineto{\pgfqpoint{1.031542in}{2.010001in}}%
\pgfpathlineto{\pgfqpoint{1.031542in}{2.007051in}}%
\pgfpathmoveto{\pgfqpoint{1.031542in}{2.007051in}}%
\pgfpathlineto{\pgfqpoint{1.031542in}{2.007051in}}%
\pgfpathlineto{\pgfqpoint{1.031542in}{2.010001in}}%
\pgfpathlineto{\pgfqpoint{1.036083in}{2.010001in}}%
\pgfpathlineto{\pgfqpoint{1.036083in}{2.007051in}}%
\pgfpathmoveto{\pgfqpoint{1.036083in}{2.007051in}}%
\pgfpathlineto{\pgfqpoint{1.036083in}{2.007051in}}%
\pgfpathlineto{\pgfqpoint{1.036083in}{2.010001in}}%
\pgfpathlineto{\pgfqpoint{1.040624in}{2.010001in}}%
\pgfpathlineto{\pgfqpoint{1.040624in}{2.007051in}}%
\pgfpathmoveto{\pgfqpoint{1.040624in}{2.007051in}}%
\pgfpathlineto{\pgfqpoint{1.040624in}{2.007051in}}%
\pgfpathlineto{\pgfqpoint{1.040624in}{2.010001in}}%
\pgfpathlineto{\pgfqpoint{1.045165in}{2.010001in}}%
\pgfpathlineto{\pgfqpoint{1.045165in}{2.007051in}}%
\pgfpathmoveto{\pgfqpoint{1.045165in}{2.007051in}}%
\pgfpathlineto{\pgfqpoint{1.045165in}{2.007051in}}%
\pgfpathlineto{\pgfqpoint{1.045165in}{2.010001in}}%
\pgfpathlineto{\pgfqpoint{1.049706in}{2.010001in}}%
\pgfpathlineto{\pgfqpoint{1.049706in}{2.007051in}}%
\pgfpathmoveto{\pgfqpoint{1.049706in}{2.007051in}}%
\pgfpathlineto{\pgfqpoint{1.049706in}{2.007051in}}%
\pgfpathlineto{\pgfqpoint{1.049706in}{2.010001in}}%
\pgfpathlineto{\pgfqpoint{1.054247in}{2.010001in}}%
\pgfpathlineto{\pgfqpoint{1.054247in}{2.007051in}}%
\pgfpathmoveto{\pgfqpoint{1.054247in}{2.007051in}}%
\pgfpathlineto{\pgfqpoint{1.054247in}{2.007051in}}%
\pgfpathlineto{\pgfqpoint{1.054247in}{2.010001in}}%
\pgfpathlineto{\pgfqpoint{1.058788in}{2.010001in}}%
\pgfpathlineto{\pgfqpoint{1.058788in}{2.007051in}}%
\pgfpathmoveto{\pgfqpoint{1.058788in}{2.007051in}}%
\pgfpathlineto{\pgfqpoint{1.058788in}{2.007051in}}%
\pgfpathlineto{\pgfqpoint{1.058788in}{2.010001in}}%
\pgfpathlineto{\pgfqpoint{1.063329in}{2.010001in}}%
\pgfpathlineto{\pgfqpoint{1.063329in}{2.007051in}}%
\pgfpathmoveto{\pgfqpoint{1.063329in}{2.007051in}}%
\pgfpathlineto{\pgfqpoint{1.063329in}{2.007051in}}%
\pgfpathlineto{\pgfqpoint{1.063329in}{2.010001in}}%
\pgfpathlineto{\pgfqpoint{1.067870in}{2.010001in}}%
\pgfpathlineto{\pgfqpoint{1.067870in}{2.007051in}}%
\pgfpathmoveto{\pgfqpoint{1.067870in}{2.007051in}}%
\pgfpathlineto{\pgfqpoint{1.067870in}{2.007051in}}%
\pgfpathlineto{\pgfqpoint{1.067870in}{2.010001in}}%
\pgfpathlineto{\pgfqpoint{1.072411in}{2.010001in}}%
\pgfpathlineto{\pgfqpoint{1.072411in}{2.007051in}}%
\pgfpathmoveto{\pgfqpoint{1.072411in}{2.007051in}}%
\pgfpathlineto{\pgfqpoint{1.072411in}{2.007051in}}%
\pgfpathlineto{\pgfqpoint{1.072411in}{2.010001in}}%
\pgfpathlineto{\pgfqpoint{1.076952in}{2.010001in}}%
\pgfpathlineto{\pgfqpoint{1.076952in}{2.007051in}}%
\pgfpathmoveto{\pgfqpoint{1.076952in}{2.007051in}}%
\pgfpathlineto{\pgfqpoint{1.076952in}{2.007051in}}%
\pgfpathlineto{\pgfqpoint{1.076952in}{2.010001in}}%
\pgfpathlineto{\pgfqpoint{1.081493in}{2.010001in}}%
\pgfpathlineto{\pgfqpoint{1.081493in}{2.007051in}}%
\pgfpathmoveto{\pgfqpoint{1.081493in}{2.007051in}}%
\pgfpathlineto{\pgfqpoint{1.081493in}{2.007051in}}%
\pgfpathlineto{\pgfqpoint{1.081493in}{2.010001in}}%
\pgfpathlineto{\pgfqpoint{1.086034in}{2.010001in}}%
\pgfpathlineto{\pgfqpoint{1.086034in}{2.007051in}}%
\pgfpathmoveto{\pgfqpoint{1.086034in}{2.007051in}}%
\pgfpathlineto{\pgfqpoint{1.086034in}{2.007051in}}%
\pgfpathlineto{\pgfqpoint{1.086034in}{2.010001in}}%
\pgfpathlineto{\pgfqpoint{1.090575in}{2.010001in}}%
\pgfpathlineto{\pgfqpoint{1.090575in}{2.007051in}}%
\pgfpathmoveto{\pgfqpoint{1.090575in}{2.007051in}}%
\pgfpathlineto{\pgfqpoint{1.090575in}{2.007051in}}%
\pgfpathlineto{\pgfqpoint{1.090575in}{2.010001in}}%
\pgfpathlineto{\pgfqpoint{1.095116in}{2.010001in}}%
\pgfpathlineto{\pgfqpoint{1.095116in}{2.007051in}}%
\pgfpathmoveto{\pgfqpoint{1.095116in}{2.007051in}}%
\pgfpathlineto{\pgfqpoint{1.095116in}{2.007051in}}%
\pgfpathlineto{\pgfqpoint{1.095116in}{2.010001in}}%
\pgfpathlineto{\pgfqpoint{1.099657in}{2.010001in}}%
\pgfpathlineto{\pgfqpoint{1.099657in}{2.007051in}}%
\pgfpathmoveto{\pgfqpoint{1.099657in}{2.007051in}}%
\pgfpathlineto{\pgfqpoint{1.099657in}{2.007051in}}%
\pgfpathlineto{\pgfqpoint{1.099657in}{2.010001in}}%
\pgfpathlineto{\pgfqpoint{1.104199in}{2.010001in}}%
\pgfpathlineto{\pgfqpoint{1.104199in}{2.007051in}}%
\pgfpathmoveto{\pgfqpoint{1.104199in}{2.007051in}}%
\pgfpathlineto{\pgfqpoint{1.104199in}{2.007051in}}%
\pgfpathlineto{\pgfqpoint{1.104199in}{2.010001in}}%
\pgfpathlineto{\pgfqpoint{1.108740in}{2.010001in}}%
\pgfpathlineto{\pgfqpoint{1.108740in}{2.007051in}}%
\pgfpathmoveto{\pgfqpoint{1.108740in}{2.007051in}}%
\pgfpathlineto{\pgfqpoint{1.108740in}{2.007051in}}%
\pgfpathlineto{\pgfqpoint{1.108740in}{2.010001in}}%
\pgfpathlineto{\pgfqpoint{1.113281in}{2.010001in}}%
\pgfpathlineto{\pgfqpoint{1.113281in}{2.007051in}}%
\pgfpathmoveto{\pgfqpoint{1.113281in}{2.007051in}}%
\pgfpathlineto{\pgfqpoint{1.113281in}{2.007051in}}%
\pgfpathlineto{\pgfqpoint{1.113281in}{2.010001in}}%
\pgfpathlineto{\pgfqpoint{1.117822in}{2.010001in}}%
\pgfpathlineto{\pgfqpoint{1.117822in}{2.007051in}}%
\pgfpathmoveto{\pgfqpoint{1.117822in}{2.007051in}}%
\pgfpathlineto{\pgfqpoint{1.117822in}{2.007051in}}%
\pgfpathlineto{\pgfqpoint{1.117822in}{2.010001in}}%
\pgfpathlineto{\pgfqpoint{1.122363in}{2.010001in}}%
\pgfpathlineto{\pgfqpoint{1.122363in}{2.007051in}}%
\pgfpathmoveto{\pgfqpoint{1.122363in}{2.007051in}}%
\pgfpathlineto{\pgfqpoint{1.122363in}{2.007051in}}%
\pgfpathlineto{\pgfqpoint{1.122363in}{2.010001in}}%
\pgfpathlineto{\pgfqpoint{1.126904in}{2.010001in}}%
\pgfpathlineto{\pgfqpoint{1.126904in}{2.007051in}}%
\pgfpathmoveto{\pgfqpoint{1.126904in}{2.007051in}}%
\pgfpathlineto{\pgfqpoint{1.126904in}{2.007051in}}%
\pgfpathlineto{\pgfqpoint{1.126904in}{2.010001in}}%
\pgfpathlineto{\pgfqpoint{1.131445in}{2.010001in}}%
\pgfpathlineto{\pgfqpoint{1.131445in}{2.007051in}}%
\pgfpathmoveto{\pgfqpoint{1.131445in}{2.007051in}}%
\pgfpathlineto{\pgfqpoint{1.131445in}{2.007051in}}%
\pgfpathlineto{\pgfqpoint{1.131445in}{2.010001in}}%
\pgfpathlineto{\pgfqpoint{1.135986in}{2.010001in}}%
\pgfpathlineto{\pgfqpoint{1.135986in}{2.007051in}}%
\pgfpathmoveto{\pgfqpoint{1.135986in}{2.007051in}}%
\pgfpathlineto{\pgfqpoint{1.135986in}{2.007051in}}%
\pgfpathlineto{\pgfqpoint{1.135986in}{2.010001in}}%
\pgfpathlineto{\pgfqpoint{1.140527in}{2.010001in}}%
\pgfpathlineto{\pgfqpoint{1.140527in}{2.007051in}}%
\pgfpathmoveto{\pgfqpoint{1.140527in}{2.007051in}}%
\pgfpathlineto{\pgfqpoint{1.140527in}{2.007051in}}%
\pgfpathlineto{\pgfqpoint{1.140527in}{2.010001in}}%
\pgfpathlineto{\pgfqpoint{1.145068in}{2.010001in}}%
\pgfpathlineto{\pgfqpoint{1.145068in}{2.007051in}}%
\pgfpathmoveto{\pgfqpoint{1.145068in}{2.007051in}}%
\pgfpathlineto{\pgfqpoint{1.145068in}{2.007051in}}%
\pgfpathlineto{\pgfqpoint{1.145068in}{2.010001in}}%
\pgfpathlineto{\pgfqpoint{1.149609in}{2.010001in}}%
\pgfpathlineto{\pgfqpoint{1.149609in}{2.007051in}}%
\pgfpathmoveto{\pgfqpoint{1.149609in}{2.007051in}}%
\pgfpathlineto{\pgfqpoint{1.149609in}{2.007051in}}%
\pgfpathlineto{\pgfqpoint{1.149609in}{2.010001in}}%
\pgfpathlineto{\pgfqpoint{1.154150in}{2.010001in}}%
\pgfpathlineto{\pgfqpoint{1.154150in}{2.007051in}}%
\pgfpathmoveto{\pgfqpoint{1.154150in}{2.007051in}}%
\pgfpathlineto{\pgfqpoint{1.154150in}{2.007051in}}%
\pgfpathlineto{\pgfqpoint{1.154150in}{2.010001in}}%
\pgfpathlineto{\pgfqpoint{1.158691in}{2.010001in}}%
\pgfpathlineto{\pgfqpoint{1.158691in}{2.007051in}}%
\pgfpathmoveto{\pgfqpoint{1.158691in}{2.007051in}}%
\pgfpathlineto{\pgfqpoint{1.158691in}{2.007051in}}%
\pgfpathlineto{\pgfqpoint{1.158691in}{2.010001in}}%
\pgfpathlineto{\pgfqpoint{1.163232in}{2.010001in}}%
\pgfpathlineto{\pgfqpoint{1.163232in}{2.007051in}}%
\pgfpathmoveto{\pgfqpoint{1.163232in}{2.007051in}}%
\pgfpathlineto{\pgfqpoint{1.163232in}{2.007051in}}%
\pgfpathlineto{\pgfqpoint{1.163232in}{2.010001in}}%
\pgfpathlineto{\pgfqpoint{1.167773in}{2.010001in}}%
\pgfpathlineto{\pgfqpoint{1.167773in}{2.007051in}}%
\pgfpathmoveto{\pgfqpoint{1.167773in}{2.007051in}}%
\pgfpathlineto{\pgfqpoint{1.167773in}{2.007051in}}%
\pgfpathlineto{\pgfqpoint{1.167773in}{2.010001in}}%
\pgfpathlineto{\pgfqpoint{1.172314in}{2.010001in}}%
\pgfpathlineto{\pgfqpoint{1.172314in}{2.007051in}}%
\pgfpathmoveto{\pgfqpoint{1.172314in}{2.007051in}}%
\pgfpathlineto{\pgfqpoint{1.172314in}{2.007051in}}%
\pgfpathlineto{\pgfqpoint{1.172314in}{2.010001in}}%
\pgfpathlineto{\pgfqpoint{1.176855in}{2.010001in}}%
\pgfpathlineto{\pgfqpoint{1.176855in}{2.007051in}}%
\pgfpathmoveto{\pgfqpoint{1.176855in}{2.007051in}}%
\pgfpathlineto{\pgfqpoint{1.176855in}{2.007051in}}%
\pgfpathlineto{\pgfqpoint{1.176855in}{2.010001in}}%
\pgfpathlineto{\pgfqpoint{1.181397in}{2.010001in}}%
\pgfpathlineto{\pgfqpoint{1.181397in}{2.007051in}}%
\pgfpathmoveto{\pgfqpoint{1.181397in}{2.007051in}}%
\pgfpathlineto{\pgfqpoint{1.181397in}{2.007051in}}%
\pgfpathlineto{\pgfqpoint{1.181397in}{2.010001in}}%
\pgfpathlineto{\pgfqpoint{1.185938in}{2.010001in}}%
\pgfpathlineto{\pgfqpoint{1.185938in}{2.007051in}}%
\pgfpathmoveto{\pgfqpoint{1.185938in}{2.007051in}}%
\pgfpathlineto{\pgfqpoint{1.185938in}{2.007051in}}%
\pgfpathlineto{\pgfqpoint{1.185938in}{2.010001in}}%
\pgfpathlineto{\pgfqpoint{1.190479in}{2.010001in}}%
\pgfpathlineto{\pgfqpoint{1.190479in}{2.007051in}}%
\pgfpathmoveto{\pgfqpoint{1.190479in}{2.007051in}}%
\pgfpathlineto{\pgfqpoint{1.190479in}{2.007051in}}%
\pgfpathlineto{\pgfqpoint{1.190479in}{2.010001in}}%
\pgfpathlineto{\pgfqpoint{1.195020in}{2.010001in}}%
\pgfpathlineto{\pgfqpoint{1.195020in}{2.007051in}}%
\pgfpathmoveto{\pgfqpoint{1.195020in}{2.007051in}}%
\pgfpathlineto{\pgfqpoint{1.195020in}{2.007051in}}%
\pgfpathlineto{\pgfqpoint{1.195020in}{2.010001in}}%
\pgfpathlineto{\pgfqpoint{1.199561in}{2.010001in}}%
\pgfpathlineto{\pgfqpoint{1.199561in}{2.007051in}}%
\pgfpathmoveto{\pgfqpoint{1.199561in}{2.007051in}}%
\pgfpathlineto{\pgfqpoint{1.199561in}{2.007051in}}%
\pgfpathlineto{\pgfqpoint{1.199561in}{2.010001in}}%
\pgfpathlineto{\pgfqpoint{1.204102in}{2.010001in}}%
\pgfpathlineto{\pgfqpoint{1.204102in}{2.007051in}}%
\pgfpathmoveto{\pgfqpoint{1.204102in}{2.007051in}}%
\pgfpathlineto{\pgfqpoint{1.204102in}{2.007051in}}%
\pgfpathlineto{\pgfqpoint{1.204102in}{2.010001in}}%
\pgfpathlineto{\pgfqpoint{1.208643in}{2.010001in}}%
\pgfpathlineto{\pgfqpoint{1.208643in}{2.007051in}}%
\pgfpathmoveto{\pgfqpoint{1.208643in}{2.007051in}}%
\pgfpathlineto{\pgfqpoint{1.208643in}{2.007051in}}%
\pgfpathlineto{\pgfqpoint{1.208643in}{2.010001in}}%
\pgfpathlineto{\pgfqpoint{1.213184in}{2.010001in}}%
\pgfpathlineto{\pgfqpoint{1.213184in}{2.007051in}}%
\pgfpathmoveto{\pgfqpoint{1.213184in}{2.007051in}}%
\pgfpathlineto{\pgfqpoint{1.213184in}{2.007051in}}%
\pgfpathlineto{\pgfqpoint{1.213184in}{2.010001in}}%
\pgfpathlineto{\pgfqpoint{1.217725in}{2.010001in}}%
\pgfpathlineto{\pgfqpoint{1.217725in}{2.007051in}}%
\pgfpathmoveto{\pgfqpoint{1.217725in}{2.007051in}}%
\pgfpathlineto{\pgfqpoint{1.217725in}{2.007051in}}%
\pgfpathlineto{\pgfqpoint{1.217725in}{2.010001in}}%
\pgfpathlineto{\pgfqpoint{1.222266in}{2.010001in}}%
\pgfpathlineto{\pgfqpoint{1.222266in}{2.007051in}}%
\pgfpathmoveto{\pgfqpoint{1.222266in}{2.007051in}}%
\pgfpathlineto{\pgfqpoint{1.222266in}{2.007051in}}%
\pgfpathlineto{\pgfqpoint{1.222266in}{2.010001in}}%
\pgfpathlineto{\pgfqpoint{1.226807in}{2.010001in}}%
\pgfpathlineto{\pgfqpoint{1.226807in}{2.007051in}}%
\pgfpathmoveto{\pgfqpoint{1.226807in}{2.007051in}}%
\pgfpathlineto{\pgfqpoint{1.226807in}{2.007051in}}%
\pgfpathlineto{\pgfqpoint{1.226807in}{2.010001in}}%
\pgfpathlineto{\pgfqpoint{1.231348in}{2.010001in}}%
\pgfpathlineto{\pgfqpoint{1.231348in}{2.007051in}}%
\pgfpathmoveto{\pgfqpoint{1.231348in}{2.007051in}}%
\pgfpathlineto{\pgfqpoint{1.231348in}{2.007051in}}%
\pgfpathlineto{\pgfqpoint{1.231348in}{2.010001in}}%
\pgfpathlineto{\pgfqpoint{1.235889in}{2.010001in}}%
\pgfpathlineto{\pgfqpoint{1.235889in}{2.007051in}}%
\pgfpathmoveto{\pgfqpoint{1.235889in}{2.007051in}}%
\pgfpathlineto{\pgfqpoint{1.235889in}{2.007051in}}%
\pgfpathlineto{\pgfqpoint{1.235889in}{2.010001in}}%
\pgfpathlineto{\pgfqpoint{1.240430in}{2.010001in}}%
\pgfpathlineto{\pgfqpoint{1.240430in}{2.007051in}}%
\pgfpathmoveto{\pgfqpoint{1.240430in}{2.007051in}}%
\pgfpathlineto{\pgfqpoint{1.240430in}{2.007051in}}%
\pgfpathlineto{\pgfqpoint{1.240430in}{2.010001in}}%
\pgfpathlineto{\pgfqpoint{1.244971in}{2.010001in}}%
\pgfpathlineto{\pgfqpoint{1.244971in}{2.007051in}}%
\pgfpathmoveto{\pgfqpoint{1.244971in}{2.007051in}}%
\pgfpathlineto{\pgfqpoint{1.244971in}{2.007051in}}%
\pgfpathlineto{\pgfqpoint{1.244971in}{2.010001in}}%
\pgfpathlineto{\pgfqpoint{1.249512in}{2.010001in}}%
\pgfpathlineto{\pgfqpoint{1.249512in}{2.007051in}}%
\pgfpathmoveto{\pgfqpoint{1.249512in}{2.007051in}}%
\pgfpathlineto{\pgfqpoint{1.249512in}{2.007051in}}%
\pgfpathlineto{\pgfqpoint{1.249512in}{2.010001in}}%
\pgfpathlineto{\pgfqpoint{1.254053in}{2.010001in}}%
\pgfpathlineto{\pgfqpoint{1.254053in}{2.007051in}}%
\pgfpathmoveto{\pgfqpoint{1.254053in}{2.007051in}}%
\pgfpathlineto{\pgfqpoint{1.254053in}{2.007051in}}%
\pgfpathlineto{\pgfqpoint{1.254053in}{2.010001in}}%
\pgfpathlineto{\pgfqpoint{1.258594in}{2.010001in}}%
\pgfpathlineto{\pgfqpoint{1.258594in}{2.007051in}}%
\pgfpathmoveto{\pgfqpoint{1.258594in}{2.007051in}}%
\pgfpathlineto{\pgfqpoint{1.258594in}{2.007051in}}%
\pgfpathlineto{\pgfqpoint{1.258594in}{2.010001in}}%
\pgfpathlineto{\pgfqpoint{1.263136in}{2.010001in}}%
\pgfpathlineto{\pgfqpoint{1.263136in}{2.007051in}}%
\pgfpathmoveto{\pgfqpoint{1.263136in}{2.007051in}}%
\pgfpathlineto{\pgfqpoint{1.263136in}{2.007051in}}%
\pgfpathlineto{\pgfqpoint{1.263136in}{2.010001in}}%
\pgfpathlineto{\pgfqpoint{1.267677in}{2.010001in}}%
\pgfpathlineto{\pgfqpoint{1.267677in}{2.007051in}}%
\pgfpathmoveto{\pgfqpoint{1.267677in}{2.007051in}}%
\pgfpathlineto{\pgfqpoint{1.267677in}{2.007051in}}%
\pgfpathlineto{\pgfqpoint{1.267677in}{2.010001in}}%
\pgfpathlineto{\pgfqpoint{1.272218in}{2.010001in}}%
\pgfpathlineto{\pgfqpoint{1.272218in}{2.007051in}}%
\pgfpathmoveto{\pgfqpoint{1.272218in}{2.007051in}}%
\pgfpathlineto{\pgfqpoint{1.272218in}{2.007051in}}%
\pgfpathlineto{\pgfqpoint{1.272218in}{2.010001in}}%
\pgfpathlineto{\pgfqpoint{1.276759in}{2.010001in}}%
\pgfpathlineto{\pgfqpoint{1.276759in}{2.007051in}}%
\pgfpathmoveto{\pgfqpoint{1.276759in}{2.007051in}}%
\pgfpathlineto{\pgfqpoint{1.276759in}{2.007051in}}%
\pgfpathlineto{\pgfqpoint{1.276759in}{2.010001in}}%
\pgfpathlineto{\pgfqpoint{1.281300in}{2.010001in}}%
\pgfpathlineto{\pgfqpoint{1.281300in}{2.007051in}}%
\pgfpathmoveto{\pgfqpoint{1.281300in}{2.007051in}}%
\pgfpathlineto{\pgfqpoint{1.281300in}{2.007051in}}%
\pgfpathlineto{\pgfqpoint{1.281300in}{2.010001in}}%
\pgfpathlineto{\pgfqpoint{1.285841in}{2.010001in}}%
\pgfpathlineto{\pgfqpoint{1.285841in}{2.007051in}}%
\pgfpathmoveto{\pgfqpoint{1.285841in}{2.007051in}}%
\pgfpathlineto{\pgfqpoint{1.285841in}{2.007051in}}%
\pgfpathlineto{\pgfqpoint{1.285841in}{2.010001in}}%
\pgfpathlineto{\pgfqpoint{1.290382in}{2.010001in}}%
\pgfpathlineto{\pgfqpoint{1.290382in}{2.007051in}}%
\pgfpathmoveto{\pgfqpoint{1.290382in}{2.007051in}}%
\pgfpathlineto{\pgfqpoint{1.290382in}{2.007051in}}%
\pgfpathlineto{\pgfqpoint{1.290382in}{2.010001in}}%
\pgfpathlineto{\pgfqpoint{1.294923in}{2.010001in}}%
\pgfpathlineto{\pgfqpoint{1.294923in}{2.007051in}}%
\pgfpathmoveto{\pgfqpoint{1.294923in}{2.007051in}}%
\pgfpathlineto{\pgfqpoint{1.294923in}{2.007051in}}%
\pgfpathlineto{\pgfqpoint{1.294923in}{2.010001in}}%
\pgfpathlineto{\pgfqpoint{1.299464in}{2.010001in}}%
\pgfpathlineto{\pgfqpoint{1.299464in}{2.007051in}}%
\pgfpathmoveto{\pgfqpoint{1.299464in}{2.007051in}}%
\pgfpathlineto{\pgfqpoint{1.299464in}{2.007051in}}%
\pgfpathlineto{\pgfqpoint{1.299464in}{2.010001in}}%
\pgfpathlineto{\pgfqpoint{1.304005in}{2.010001in}}%
\pgfpathlineto{\pgfqpoint{1.304005in}{2.007051in}}%
\pgfpathmoveto{\pgfqpoint{1.304005in}{2.007051in}}%
\pgfpathlineto{\pgfqpoint{1.304005in}{2.007051in}}%
\pgfpathlineto{\pgfqpoint{1.304005in}{2.010001in}}%
\pgfpathlineto{\pgfqpoint{1.308546in}{2.010001in}}%
\pgfpathlineto{\pgfqpoint{1.308546in}{2.007051in}}%
\pgfpathmoveto{\pgfqpoint{1.308546in}{2.007051in}}%
\pgfpathlineto{\pgfqpoint{1.308546in}{2.007051in}}%
\pgfpathlineto{\pgfqpoint{1.308546in}{2.010001in}}%
\pgfpathlineto{\pgfqpoint{1.313087in}{2.010001in}}%
\pgfpathlineto{\pgfqpoint{1.313087in}{2.007051in}}%
\pgfpathmoveto{\pgfqpoint{1.313087in}{2.007051in}}%
\pgfpathlineto{\pgfqpoint{1.313087in}{2.007051in}}%
\pgfpathlineto{\pgfqpoint{1.313087in}{2.010001in}}%
\pgfpathlineto{\pgfqpoint{1.317628in}{2.010001in}}%
\pgfpathlineto{\pgfqpoint{1.317628in}{2.007051in}}%
\pgfpathmoveto{\pgfqpoint{1.317628in}{2.007051in}}%
\pgfpathlineto{\pgfqpoint{1.317628in}{2.007051in}}%
\pgfpathlineto{\pgfqpoint{1.317628in}{2.010001in}}%
\pgfpathlineto{\pgfqpoint{1.322169in}{2.010001in}}%
\pgfpathlineto{\pgfqpoint{1.322169in}{2.007051in}}%
\pgfpathmoveto{\pgfqpoint{1.322169in}{2.007051in}}%
\pgfpathlineto{\pgfqpoint{1.322169in}{2.007051in}}%
\pgfpathlineto{\pgfqpoint{1.322169in}{2.010001in}}%
\pgfpathlineto{\pgfqpoint{1.326710in}{2.010001in}}%
\pgfpathlineto{\pgfqpoint{1.326710in}{2.007051in}}%
\pgfpathmoveto{\pgfqpoint{1.326710in}{2.007051in}}%
\pgfpathlineto{\pgfqpoint{1.326710in}{2.007051in}}%
\pgfpathlineto{\pgfqpoint{1.326710in}{2.010001in}}%
\pgfpathlineto{\pgfqpoint{1.331251in}{2.010001in}}%
\pgfpathlineto{\pgfqpoint{1.331251in}{2.007051in}}%
\pgfpathmoveto{\pgfqpoint{1.331251in}{2.007051in}}%
\pgfpathlineto{\pgfqpoint{1.331251in}{2.007051in}}%
\pgfpathlineto{\pgfqpoint{1.331251in}{2.010001in}}%
\pgfpathlineto{\pgfqpoint{1.335792in}{2.010001in}}%
\pgfpathlineto{\pgfqpoint{1.335792in}{2.007051in}}%
\pgfpathmoveto{\pgfqpoint{1.335792in}{2.007051in}}%
\pgfpathlineto{\pgfqpoint{1.335792in}{2.007051in}}%
\pgfpathlineto{\pgfqpoint{1.335792in}{2.010001in}}%
\pgfpathlineto{\pgfqpoint{1.340333in}{2.010001in}}%
\pgfpathlineto{\pgfqpoint{1.340333in}{2.007051in}}%
\pgfpathmoveto{\pgfqpoint{1.340333in}{2.007051in}}%
\pgfpathlineto{\pgfqpoint{1.340333in}{2.007051in}}%
\pgfpathlineto{\pgfqpoint{1.340333in}{2.010001in}}%
\pgfpathlineto{\pgfqpoint{1.344874in}{2.010001in}}%
\pgfpathlineto{\pgfqpoint{1.344874in}{2.007051in}}%
\pgfpathmoveto{\pgfqpoint{1.344874in}{2.007051in}}%
\pgfpathlineto{\pgfqpoint{1.344874in}{2.007051in}}%
\pgfpathlineto{\pgfqpoint{1.344874in}{2.010001in}}%
\pgfpathlineto{\pgfqpoint{1.349415in}{2.010001in}}%
\pgfpathlineto{\pgfqpoint{1.349415in}{2.007051in}}%
\pgfpathmoveto{\pgfqpoint{1.349415in}{2.007051in}}%
\pgfpathlineto{\pgfqpoint{1.349415in}{2.007051in}}%
\pgfpathlineto{\pgfqpoint{1.349415in}{2.010001in}}%
\pgfpathlineto{\pgfqpoint{1.353956in}{2.010001in}}%
\pgfpathlineto{\pgfqpoint{1.353956in}{2.007051in}}%
\pgfpathmoveto{\pgfqpoint{1.353956in}{2.007051in}}%
\pgfpathlineto{\pgfqpoint{1.353956in}{2.007051in}}%
\pgfpathlineto{\pgfqpoint{1.353956in}{2.010001in}}%
\pgfpathlineto{\pgfqpoint{1.358497in}{2.010001in}}%
\pgfpathlineto{\pgfqpoint{1.358497in}{2.007051in}}%
\pgfpathmoveto{\pgfqpoint{1.358497in}{2.007051in}}%
\pgfpathlineto{\pgfqpoint{1.358497in}{2.007051in}}%
\pgfpathlineto{\pgfqpoint{1.358497in}{2.010001in}}%
\pgfpathlineto{\pgfqpoint{1.363038in}{2.010001in}}%
\pgfpathlineto{\pgfqpoint{1.363038in}{2.007051in}}%
\pgfpathmoveto{\pgfqpoint{1.363038in}{2.007051in}}%
\pgfpathlineto{\pgfqpoint{1.363038in}{2.007051in}}%
\pgfpathlineto{\pgfqpoint{1.363038in}{2.010001in}}%
\pgfpathlineto{\pgfqpoint{1.367579in}{2.010001in}}%
\pgfpathlineto{\pgfqpoint{1.367579in}{2.007051in}}%
\pgfpathmoveto{\pgfqpoint{1.367579in}{2.007051in}}%
\pgfpathlineto{\pgfqpoint{1.367579in}{2.007051in}}%
\pgfpathlineto{\pgfqpoint{1.367579in}{2.010001in}}%
\pgfpathlineto{\pgfqpoint{1.372120in}{2.010001in}}%
\pgfpathlineto{\pgfqpoint{1.372120in}{2.007051in}}%
\pgfpathmoveto{\pgfqpoint{1.372120in}{2.007051in}}%
\pgfpathlineto{\pgfqpoint{1.372120in}{2.007051in}}%
\pgfpathlineto{\pgfqpoint{1.372120in}{2.010001in}}%
\pgfpathlineto{\pgfqpoint{1.376660in}{2.010001in}}%
\pgfpathlineto{\pgfqpoint{1.376660in}{2.007051in}}%
\pgfpathmoveto{\pgfqpoint{1.376660in}{2.007051in}}%
\pgfpathlineto{\pgfqpoint{1.376660in}{2.007051in}}%
\pgfpathlineto{\pgfqpoint{1.376660in}{2.010001in}}%
\pgfpathlineto{\pgfqpoint{1.381201in}{2.010001in}}%
\pgfpathlineto{\pgfqpoint{1.381201in}{2.007051in}}%
\pgfpathmoveto{\pgfqpoint{1.381201in}{2.007051in}}%
\pgfpathlineto{\pgfqpoint{1.381201in}{2.007051in}}%
\pgfpathlineto{\pgfqpoint{1.381201in}{2.010001in}}%
\pgfpathlineto{\pgfqpoint{1.385742in}{2.010001in}}%
\pgfpathlineto{\pgfqpoint{1.385742in}{2.007051in}}%
\pgfpathmoveto{\pgfqpoint{1.385742in}{2.007051in}}%
\pgfpathlineto{\pgfqpoint{1.385742in}{2.007051in}}%
\pgfpathlineto{\pgfqpoint{1.385742in}{2.010001in}}%
\pgfpathlineto{\pgfqpoint{1.390283in}{2.010001in}}%
\pgfpathlineto{\pgfqpoint{1.390283in}{2.007051in}}%
\pgfpathmoveto{\pgfqpoint{1.390283in}{2.007051in}}%
\pgfpathlineto{\pgfqpoint{1.390283in}{2.007051in}}%
\pgfpathlineto{\pgfqpoint{1.390283in}{2.010001in}}%
\pgfpathlineto{\pgfqpoint{1.394824in}{2.010001in}}%
\pgfpathlineto{\pgfqpoint{1.394824in}{2.007051in}}%
\pgfpathmoveto{\pgfqpoint{1.394824in}{2.007051in}}%
\pgfpathlineto{\pgfqpoint{1.394824in}{2.007051in}}%
\pgfpathlineto{\pgfqpoint{1.394824in}{2.010001in}}%
\pgfpathlineto{\pgfqpoint{1.399365in}{2.010001in}}%
\pgfpathlineto{\pgfqpoint{1.399365in}{2.007051in}}%
\pgfpathmoveto{\pgfqpoint{1.399365in}{2.007051in}}%
\pgfpathlineto{\pgfqpoint{1.399365in}{2.007051in}}%
\pgfpathlineto{\pgfqpoint{1.399365in}{2.010001in}}%
\pgfpathlineto{\pgfqpoint{1.403906in}{2.010001in}}%
\pgfpathlineto{\pgfqpoint{1.403906in}{2.007051in}}%
\pgfpathmoveto{\pgfqpoint{1.403906in}{2.007051in}}%
\pgfpathlineto{\pgfqpoint{1.403906in}{2.007051in}}%
\pgfpathlineto{\pgfqpoint{1.403906in}{2.010001in}}%
\pgfpathlineto{\pgfqpoint{1.408447in}{2.010001in}}%
\pgfpathlineto{\pgfqpoint{1.408447in}{2.007051in}}%
\pgfpathmoveto{\pgfqpoint{1.408447in}{2.007051in}}%
\pgfpathlineto{\pgfqpoint{1.408447in}{2.007051in}}%
\pgfpathlineto{\pgfqpoint{1.408447in}{2.010001in}}%
\pgfpathlineto{\pgfqpoint{1.412988in}{2.010001in}}%
\pgfpathlineto{\pgfqpoint{1.412988in}{2.007051in}}%
\pgfpathmoveto{\pgfqpoint{1.412988in}{2.007051in}}%
\pgfpathlineto{\pgfqpoint{1.412988in}{2.007051in}}%
\pgfpathlineto{\pgfqpoint{1.412988in}{2.010001in}}%
\pgfpathlineto{\pgfqpoint{1.417529in}{2.010001in}}%
\pgfpathlineto{\pgfqpoint{1.417529in}{2.007051in}}%
\pgfpathmoveto{\pgfqpoint{1.417529in}{2.007051in}}%
\pgfpathlineto{\pgfqpoint{1.417529in}{2.007051in}}%
\pgfpathlineto{\pgfqpoint{1.417529in}{2.010001in}}%
\pgfpathlineto{\pgfqpoint{1.422069in}{2.010001in}}%
\pgfpathlineto{\pgfqpoint{1.422069in}{2.007051in}}%
\pgfpathmoveto{\pgfqpoint{1.422069in}{2.007051in}}%
\pgfpathlineto{\pgfqpoint{1.422069in}{2.007051in}}%
\pgfpathlineto{\pgfqpoint{1.422069in}{2.010001in}}%
\pgfpathlineto{\pgfqpoint{1.426610in}{2.010001in}}%
\pgfpathlineto{\pgfqpoint{1.426610in}{2.007051in}}%
\pgfpathmoveto{\pgfqpoint{1.426610in}{2.007051in}}%
\pgfpathlineto{\pgfqpoint{1.426610in}{2.007051in}}%
\pgfpathlineto{\pgfqpoint{1.426610in}{2.010001in}}%
\pgfpathlineto{\pgfqpoint{1.431151in}{2.010001in}}%
\pgfpathlineto{\pgfqpoint{1.431151in}{2.007051in}}%
\pgfpathmoveto{\pgfqpoint{1.431151in}{2.007051in}}%
\pgfpathlineto{\pgfqpoint{1.431151in}{2.007051in}}%
\pgfpathlineto{\pgfqpoint{1.431151in}{2.010001in}}%
\pgfpathlineto{\pgfqpoint{1.435692in}{2.010001in}}%
\pgfpathlineto{\pgfqpoint{1.435692in}{2.007051in}}%
\pgfpathmoveto{\pgfqpoint{1.435692in}{2.007051in}}%
\pgfpathlineto{\pgfqpoint{1.435692in}{2.007051in}}%
\pgfpathlineto{\pgfqpoint{1.435692in}{2.010001in}}%
\pgfpathlineto{\pgfqpoint{1.440233in}{2.010001in}}%
\pgfpathlineto{\pgfqpoint{1.440233in}{2.007051in}}%
\pgfpathmoveto{\pgfqpoint{1.440233in}{2.007051in}}%
\pgfpathlineto{\pgfqpoint{1.440233in}{2.007051in}}%
\pgfpathlineto{\pgfqpoint{1.440233in}{2.010001in}}%
\pgfpathlineto{\pgfqpoint{1.444774in}{2.010001in}}%
\pgfpathlineto{\pgfqpoint{1.444774in}{2.007051in}}%
\pgfpathmoveto{\pgfqpoint{1.444774in}{2.007051in}}%
\pgfpathlineto{\pgfqpoint{1.444774in}{2.007051in}}%
\pgfpathlineto{\pgfqpoint{1.444774in}{2.010001in}}%
\pgfpathlineto{\pgfqpoint{1.449315in}{2.010001in}}%
\pgfpathlineto{\pgfqpoint{1.449315in}{2.007051in}}%
\pgfpathmoveto{\pgfqpoint{1.449315in}{2.007051in}}%
\pgfpathlineto{\pgfqpoint{1.449315in}{2.007051in}}%
\pgfpathlineto{\pgfqpoint{1.449315in}{2.010001in}}%
\pgfpathlineto{\pgfqpoint{1.453856in}{2.010001in}}%
\pgfpathlineto{\pgfqpoint{1.453856in}{2.007051in}}%
\pgfpathmoveto{\pgfqpoint{1.453856in}{2.007051in}}%
\pgfpathlineto{\pgfqpoint{1.453856in}{2.007051in}}%
\pgfpathlineto{\pgfqpoint{1.453856in}{2.010001in}}%
\pgfpathlineto{\pgfqpoint{1.458397in}{2.010001in}}%
\pgfpathlineto{\pgfqpoint{1.458397in}{2.007051in}}%
\pgfpathmoveto{\pgfqpoint{1.458397in}{2.007051in}}%
\pgfpathlineto{\pgfqpoint{1.458397in}{2.007051in}}%
\pgfpathlineto{\pgfqpoint{1.458397in}{2.010001in}}%
\pgfpathlineto{\pgfqpoint{1.462938in}{2.010001in}}%
\pgfpathlineto{\pgfqpoint{1.462938in}{2.007051in}}%
\pgfpathmoveto{\pgfqpoint{1.462938in}{2.007051in}}%
\pgfpathlineto{\pgfqpoint{1.462938in}{2.007051in}}%
\pgfpathlineto{\pgfqpoint{1.462938in}{2.010001in}}%
\pgfpathlineto{\pgfqpoint{1.467479in}{2.010001in}}%
\pgfpathlineto{\pgfqpoint{1.467479in}{2.007051in}}%
\pgfpathmoveto{\pgfqpoint{1.467479in}{2.007051in}}%
\pgfpathlineto{\pgfqpoint{1.467479in}{2.007051in}}%
\pgfpathlineto{\pgfqpoint{1.467479in}{2.010001in}}%
\pgfpathlineto{\pgfqpoint{1.472019in}{2.010001in}}%
\pgfpathlineto{\pgfqpoint{1.472019in}{2.007051in}}%
\pgfpathmoveto{\pgfqpoint{1.472019in}{2.007051in}}%
\pgfpathlineto{\pgfqpoint{1.472019in}{2.007051in}}%
\pgfpathlineto{\pgfqpoint{1.472019in}{2.010001in}}%
\pgfpathlineto{\pgfqpoint{1.476560in}{2.010001in}}%
\pgfpathlineto{\pgfqpoint{1.476560in}{2.007051in}}%
\pgfpathmoveto{\pgfqpoint{1.476560in}{2.007051in}}%
\pgfpathlineto{\pgfqpoint{1.476560in}{2.007051in}}%
\pgfpathlineto{\pgfqpoint{1.476560in}{2.010001in}}%
\pgfpathlineto{\pgfqpoint{1.481102in}{2.010001in}}%
\pgfpathlineto{\pgfqpoint{1.481102in}{2.007051in}}%
\pgfpathmoveto{\pgfqpoint{1.481102in}{2.007051in}}%
\pgfpathlineto{\pgfqpoint{1.481102in}{2.007051in}}%
\pgfpathlineto{\pgfqpoint{1.481102in}{2.010001in}}%
\pgfpathlineto{\pgfqpoint{1.485643in}{2.010001in}}%
\pgfpathlineto{\pgfqpoint{1.485643in}{2.007051in}}%
\pgfpathmoveto{\pgfqpoint{1.485643in}{2.007051in}}%
\pgfpathlineto{\pgfqpoint{1.485643in}{2.007051in}}%
\pgfpathlineto{\pgfqpoint{1.485643in}{2.010001in}}%
\pgfpathlineto{\pgfqpoint{1.490184in}{2.010001in}}%
\pgfpathlineto{\pgfqpoint{1.490184in}{2.007051in}}%
\pgfpathmoveto{\pgfqpoint{1.490184in}{2.007051in}}%
\pgfpathlineto{\pgfqpoint{1.490184in}{2.007051in}}%
\pgfpathlineto{\pgfqpoint{1.490184in}{2.010001in}}%
\pgfpathlineto{\pgfqpoint{1.494725in}{2.010001in}}%
\pgfpathlineto{\pgfqpoint{1.494725in}{2.007051in}}%
\pgfpathmoveto{\pgfqpoint{1.494725in}{2.007051in}}%
\pgfpathlineto{\pgfqpoint{1.494725in}{2.007051in}}%
\pgfpathlineto{\pgfqpoint{1.494725in}{2.010001in}}%
\pgfpathlineto{\pgfqpoint{1.499266in}{2.010001in}}%
\pgfpathlineto{\pgfqpoint{1.499266in}{2.007051in}}%
\pgfpathmoveto{\pgfqpoint{1.499266in}{2.007051in}}%
\pgfpathlineto{\pgfqpoint{1.499266in}{2.007051in}}%
\pgfpathlineto{\pgfqpoint{1.499266in}{2.010001in}}%
\pgfpathlineto{\pgfqpoint{1.503808in}{2.010001in}}%
\pgfpathlineto{\pgfqpoint{1.503808in}{2.007051in}}%
\pgfpathmoveto{\pgfqpoint{1.503808in}{2.007051in}}%
\pgfpathlineto{\pgfqpoint{1.503808in}{2.007051in}}%
\pgfpathlineto{\pgfqpoint{1.503808in}{2.010001in}}%
\pgfpathlineto{\pgfqpoint{1.508349in}{2.010001in}}%
\pgfpathlineto{\pgfqpoint{1.508349in}{2.007051in}}%
\pgfpathmoveto{\pgfqpoint{1.508349in}{2.007051in}}%
\pgfpathlineto{\pgfqpoint{1.508349in}{2.007051in}}%
\pgfpathlineto{\pgfqpoint{1.508349in}{2.010001in}}%
\pgfpathlineto{\pgfqpoint{1.512890in}{2.010001in}}%
\pgfpathlineto{\pgfqpoint{1.512890in}{2.007051in}}%
\pgfpathmoveto{\pgfqpoint{1.512890in}{2.007051in}}%
\pgfpathlineto{\pgfqpoint{1.512890in}{2.007051in}}%
\pgfpathlineto{\pgfqpoint{1.512890in}{2.010001in}}%
\pgfpathlineto{\pgfqpoint{1.517431in}{2.010001in}}%
\pgfpathlineto{\pgfqpoint{1.517431in}{2.007051in}}%
\pgfpathmoveto{\pgfqpoint{1.517431in}{2.007051in}}%
\pgfpathlineto{\pgfqpoint{1.517431in}{2.007051in}}%
\pgfpathlineto{\pgfqpoint{1.517431in}{2.010001in}}%
\pgfpathlineto{\pgfqpoint{1.521972in}{2.010001in}}%
\pgfpathlineto{\pgfqpoint{1.521972in}{2.007051in}}%
\pgfpathmoveto{\pgfqpoint{1.521972in}{2.007051in}}%
\pgfpathlineto{\pgfqpoint{1.521972in}{2.007051in}}%
\pgfpathlineto{\pgfqpoint{1.521972in}{2.010001in}}%
\pgfpathlineto{\pgfqpoint{1.526514in}{2.010001in}}%
\pgfpathlineto{\pgfqpoint{1.526514in}{2.007051in}}%
\pgfpathmoveto{\pgfqpoint{1.526514in}{2.007051in}}%
\pgfpathlineto{\pgfqpoint{1.526514in}{2.007051in}}%
\pgfpathlineto{\pgfqpoint{1.526514in}{2.010001in}}%
\pgfpathlineto{\pgfqpoint{1.531055in}{2.010001in}}%
\pgfpathlineto{\pgfqpoint{1.531055in}{2.007051in}}%
\pgfpathmoveto{\pgfqpoint{1.531055in}{2.007051in}}%
\pgfpathlineto{\pgfqpoint{1.531055in}{2.007051in}}%
\pgfpathlineto{\pgfqpoint{1.531055in}{2.010001in}}%
\pgfpathlineto{\pgfqpoint{1.535596in}{2.010001in}}%
\pgfpathlineto{\pgfqpoint{1.535596in}{2.007051in}}%
\pgfpathmoveto{\pgfqpoint{1.535596in}{2.007051in}}%
\pgfpathlineto{\pgfqpoint{1.535596in}{2.007051in}}%
\pgfpathlineto{\pgfqpoint{1.535596in}{2.010001in}}%
\pgfpathlineto{\pgfqpoint{1.540137in}{2.010001in}}%
\pgfpathlineto{\pgfqpoint{1.540137in}{2.007051in}}%
\pgfpathmoveto{\pgfqpoint{1.540137in}{2.007051in}}%
\pgfpathlineto{\pgfqpoint{1.540137in}{2.007051in}}%
\pgfpathlineto{\pgfqpoint{1.540137in}{2.010001in}}%
\pgfpathlineto{\pgfqpoint{1.544678in}{2.010001in}}%
\pgfpathlineto{\pgfqpoint{1.544678in}{2.007051in}}%
\pgfpathmoveto{\pgfqpoint{1.544678in}{2.007051in}}%
\pgfpathlineto{\pgfqpoint{1.544678in}{2.007051in}}%
\pgfpathlineto{\pgfqpoint{1.544678in}{2.010001in}}%
\pgfpathlineto{\pgfqpoint{1.549220in}{2.010001in}}%
\pgfpathlineto{\pgfqpoint{1.549220in}{2.007051in}}%
\pgfpathmoveto{\pgfqpoint{1.549220in}{2.007051in}}%
\pgfpathlineto{\pgfqpoint{1.549220in}{2.007051in}}%
\pgfpathlineto{\pgfqpoint{1.549220in}{2.010001in}}%
\pgfpathlineto{\pgfqpoint{1.553761in}{2.010001in}}%
\pgfpathlineto{\pgfqpoint{1.553761in}{2.007051in}}%
\pgfpathmoveto{\pgfqpoint{1.553761in}{2.007051in}}%
\pgfpathlineto{\pgfqpoint{1.553761in}{2.007051in}}%
\pgfpathlineto{\pgfqpoint{1.553761in}{2.010001in}}%
\pgfpathlineto{\pgfqpoint{1.558302in}{2.010001in}}%
\pgfpathlineto{\pgfqpoint{1.558302in}{2.007051in}}%
\pgfpathmoveto{\pgfqpoint{1.558302in}{2.007051in}}%
\pgfpathlineto{\pgfqpoint{1.558302in}{2.007051in}}%
\pgfpathlineto{\pgfqpoint{1.558302in}{2.010001in}}%
\pgfpathlineto{\pgfqpoint{1.562843in}{2.010001in}}%
\pgfpathlineto{\pgfqpoint{1.562843in}{2.007051in}}%
\pgfpathmoveto{\pgfqpoint{1.562843in}{2.007051in}}%
\pgfpathlineto{\pgfqpoint{1.562843in}{2.007051in}}%
\pgfpathlineto{\pgfqpoint{1.562843in}{2.010001in}}%
\pgfpathlineto{\pgfqpoint{1.567384in}{2.010001in}}%
\pgfpathlineto{\pgfqpoint{1.567384in}{2.007051in}}%
\pgfpathmoveto{\pgfqpoint{1.567384in}{2.007051in}}%
\pgfpathlineto{\pgfqpoint{1.567384in}{2.007051in}}%
\pgfpathlineto{\pgfqpoint{1.567384in}{2.010001in}}%
\pgfpathlineto{\pgfqpoint{1.571926in}{2.010001in}}%
\pgfpathlineto{\pgfqpoint{1.571926in}{2.007051in}}%
\pgfpathmoveto{\pgfqpoint{1.571926in}{2.007051in}}%
\pgfpathlineto{\pgfqpoint{1.571926in}{2.007051in}}%
\pgfpathlineto{\pgfqpoint{1.571926in}{2.010001in}}%
\pgfpathlineto{\pgfqpoint{1.576467in}{2.010001in}}%
\pgfpathlineto{\pgfqpoint{1.576467in}{2.007051in}}%
\pgfpathmoveto{\pgfqpoint{1.576467in}{2.007051in}}%
\pgfpathlineto{\pgfqpoint{1.576467in}{2.007051in}}%
\pgfpathlineto{\pgfqpoint{1.576467in}{2.010001in}}%
\pgfpathlineto{\pgfqpoint{1.581008in}{2.010001in}}%
\pgfpathlineto{\pgfqpoint{1.581008in}{2.007051in}}%
\pgfpathmoveto{\pgfqpoint{1.581008in}{2.007051in}}%
\pgfpathlineto{\pgfqpoint{1.581008in}{2.007051in}}%
\pgfpathlineto{\pgfqpoint{1.581008in}{2.010001in}}%
\pgfpathlineto{\pgfqpoint{1.585549in}{2.010001in}}%
\pgfpathlineto{\pgfqpoint{1.585549in}{2.007051in}}%
\pgfpathmoveto{\pgfqpoint{1.585549in}{2.007051in}}%
\pgfpathlineto{\pgfqpoint{1.585549in}{2.007051in}}%
\pgfpathlineto{\pgfqpoint{1.585549in}{2.010001in}}%
\pgfpathlineto{\pgfqpoint{1.590090in}{2.010001in}}%
\pgfpathlineto{\pgfqpoint{1.590090in}{2.007051in}}%
\pgfpathmoveto{\pgfqpoint{1.590090in}{2.007051in}}%
\pgfpathlineto{\pgfqpoint{1.590090in}{2.007051in}}%
\pgfpathlineto{\pgfqpoint{1.590090in}{2.010001in}}%
\pgfpathlineto{\pgfqpoint{1.594632in}{2.010001in}}%
\pgfpathlineto{\pgfqpoint{1.594632in}{2.007051in}}%
\pgfpathmoveto{\pgfqpoint{1.594632in}{2.007051in}}%
\pgfpathlineto{\pgfqpoint{1.594632in}{2.007051in}}%
\pgfpathlineto{\pgfqpoint{1.594632in}{2.010001in}}%
\pgfpathlineto{\pgfqpoint{1.599173in}{2.010001in}}%
\pgfpathlineto{\pgfqpoint{1.599173in}{2.007051in}}%
\pgfpathmoveto{\pgfqpoint{1.599173in}{2.007051in}}%
\pgfpathlineto{\pgfqpoint{1.599173in}{2.007051in}}%
\pgfpathlineto{\pgfqpoint{1.599173in}{2.010001in}}%
\pgfpathlineto{\pgfqpoint{1.603714in}{2.010001in}}%
\pgfpathlineto{\pgfqpoint{1.603714in}{2.007051in}}%
\pgfpathmoveto{\pgfqpoint{1.603714in}{2.007051in}}%
\pgfpathlineto{\pgfqpoint{1.603714in}{2.007051in}}%
\pgfpathlineto{\pgfqpoint{1.603714in}{2.010001in}}%
\pgfpathlineto{\pgfqpoint{1.608255in}{2.010001in}}%
\pgfpathlineto{\pgfqpoint{1.608255in}{2.007051in}}%
\pgfpathmoveto{\pgfqpoint{1.608255in}{2.007051in}}%
\pgfpathlineto{\pgfqpoint{1.608255in}{2.007051in}}%
\pgfpathlineto{\pgfqpoint{1.608255in}{2.010001in}}%
\pgfpathlineto{\pgfqpoint{1.612796in}{2.010001in}}%
\pgfpathlineto{\pgfqpoint{1.612796in}{2.007051in}}%
\pgfpathmoveto{\pgfqpoint{1.612796in}{2.007051in}}%
\pgfpathlineto{\pgfqpoint{1.612796in}{2.007051in}}%
\pgfpathlineto{\pgfqpoint{1.612796in}{2.010001in}}%
\pgfpathlineto{\pgfqpoint{1.617338in}{2.010001in}}%
\pgfpathlineto{\pgfqpoint{1.617338in}{2.007051in}}%
\pgfpathmoveto{\pgfqpoint{1.617338in}{2.007051in}}%
\pgfpathlineto{\pgfqpoint{1.617338in}{2.007051in}}%
\pgfpathlineto{\pgfqpoint{1.617338in}{2.010001in}}%
\pgfpathlineto{\pgfqpoint{1.621879in}{2.010001in}}%
\pgfpathlineto{\pgfqpoint{1.621879in}{2.007051in}}%
\pgfpathmoveto{\pgfqpoint{1.621879in}{2.007051in}}%
\pgfpathlineto{\pgfqpoint{1.621879in}{2.007051in}}%
\pgfpathlineto{\pgfqpoint{1.621879in}{2.010001in}}%
\pgfpathlineto{\pgfqpoint{1.626420in}{2.010001in}}%
\pgfpathlineto{\pgfqpoint{1.626420in}{2.007051in}}%
\pgfpathmoveto{\pgfqpoint{1.626420in}{2.007051in}}%
\pgfpathlineto{\pgfqpoint{1.626420in}{2.007051in}}%
\pgfpathlineto{\pgfqpoint{1.626420in}{2.010001in}}%
\pgfpathlineto{\pgfqpoint{1.630961in}{2.010001in}}%
\pgfpathlineto{\pgfqpoint{1.630961in}{2.007051in}}%
\pgfpathmoveto{\pgfqpoint{1.630961in}{2.007051in}}%
\pgfpathlineto{\pgfqpoint{1.630961in}{2.007051in}}%
\pgfpathlineto{\pgfqpoint{1.630961in}{2.010001in}}%
\pgfpathlineto{\pgfqpoint{1.635501in}{2.010001in}}%
\pgfpathlineto{\pgfqpoint{1.635501in}{2.007051in}}%
\pgfpathmoveto{\pgfqpoint{1.635501in}{2.007051in}}%
\pgfpathlineto{\pgfqpoint{1.635501in}{2.007051in}}%
\pgfpathlineto{\pgfqpoint{1.635501in}{2.010001in}}%
\pgfpathlineto{\pgfqpoint{1.640042in}{2.010001in}}%
\pgfpathlineto{\pgfqpoint{1.640042in}{2.007051in}}%
\pgfpathmoveto{\pgfqpoint{1.640042in}{2.007051in}}%
\pgfpathlineto{\pgfqpoint{1.640042in}{2.007051in}}%
\pgfpathlineto{\pgfqpoint{1.640042in}{2.010001in}}%
\pgfpathlineto{\pgfqpoint{1.644583in}{2.010001in}}%
\pgfpathlineto{\pgfqpoint{1.644583in}{2.007051in}}%
\pgfpathmoveto{\pgfqpoint{1.644583in}{2.007051in}}%
\pgfpathlineto{\pgfqpoint{1.644583in}{2.007051in}}%
\pgfpathlineto{\pgfqpoint{1.644583in}{2.010001in}}%
\pgfpathlineto{\pgfqpoint{1.649124in}{2.010001in}}%
\pgfpathlineto{\pgfqpoint{1.649124in}{2.007051in}}%
\pgfpathmoveto{\pgfqpoint{1.649124in}{2.007051in}}%
\pgfpathlineto{\pgfqpoint{1.649124in}{2.007051in}}%
\pgfpathlineto{\pgfqpoint{1.649124in}{2.010001in}}%
\pgfpathlineto{\pgfqpoint{1.653665in}{2.010001in}}%
\pgfpathlineto{\pgfqpoint{1.653665in}{2.007051in}}%
\pgfpathmoveto{\pgfqpoint{1.653665in}{2.007051in}}%
\pgfpathlineto{\pgfqpoint{1.653665in}{2.007051in}}%
\pgfpathlineto{\pgfqpoint{1.653665in}{2.010001in}}%
\pgfpathlineto{\pgfqpoint{1.658206in}{2.010001in}}%
\pgfpathlineto{\pgfqpoint{1.658206in}{2.007051in}}%
\pgfpathmoveto{\pgfqpoint{1.658206in}{2.007051in}}%
\pgfpathlineto{\pgfqpoint{1.658206in}{2.007051in}}%
\pgfpathlineto{\pgfqpoint{1.658206in}{2.010001in}}%
\pgfpathlineto{\pgfqpoint{1.662747in}{2.010001in}}%
\pgfpathlineto{\pgfqpoint{1.662747in}{2.007051in}}%
\pgfpathmoveto{\pgfqpoint{1.662747in}{2.007051in}}%
\pgfpathlineto{\pgfqpoint{1.662747in}{2.007051in}}%
\pgfpathlineto{\pgfqpoint{1.662747in}{2.010001in}}%
\pgfpathlineto{\pgfqpoint{1.667287in}{2.010001in}}%
\pgfpathlineto{\pgfqpoint{1.667287in}{2.007051in}}%
\pgfpathmoveto{\pgfqpoint{1.667287in}{2.007051in}}%
\pgfpathlineto{\pgfqpoint{1.667287in}{2.007051in}}%
\pgfpathlineto{\pgfqpoint{1.667287in}{2.010001in}}%
\pgfpathlineto{\pgfqpoint{1.671828in}{2.010001in}}%
\pgfpathlineto{\pgfqpoint{1.671828in}{2.007051in}}%
\pgfpathmoveto{\pgfqpoint{1.671828in}{2.007051in}}%
\pgfpathlineto{\pgfqpoint{1.671828in}{2.007051in}}%
\pgfpathlineto{\pgfqpoint{1.671828in}{2.010001in}}%
\pgfpathlineto{\pgfqpoint{1.676369in}{2.010001in}}%
\pgfpathlineto{\pgfqpoint{1.676369in}{2.007051in}}%
\pgfpathmoveto{\pgfqpoint{1.676369in}{2.007051in}}%
\pgfpathlineto{\pgfqpoint{1.676369in}{2.007051in}}%
\pgfpathlineto{\pgfqpoint{1.676369in}{2.010001in}}%
\pgfpathlineto{\pgfqpoint{1.680910in}{2.010001in}}%
\pgfpathlineto{\pgfqpoint{1.680910in}{2.007051in}}%
\pgfpathmoveto{\pgfqpoint{1.680910in}{2.007051in}}%
\pgfpathlineto{\pgfqpoint{1.680910in}{2.007051in}}%
\pgfpathlineto{\pgfqpoint{1.680910in}{2.010001in}}%
\pgfpathlineto{\pgfqpoint{1.685451in}{2.010001in}}%
\pgfpathlineto{\pgfqpoint{1.685451in}{2.007051in}}%
\pgfpathmoveto{\pgfqpoint{1.685451in}{2.007051in}}%
\pgfpathlineto{\pgfqpoint{1.685451in}{2.007051in}}%
\pgfpathlineto{\pgfqpoint{1.685451in}{2.010001in}}%
\pgfpathlineto{\pgfqpoint{1.689992in}{2.010001in}}%
\pgfpathlineto{\pgfqpoint{1.689992in}{2.007051in}}%
\pgfpathmoveto{\pgfqpoint{1.689992in}{2.007051in}}%
\pgfpathlineto{\pgfqpoint{1.689992in}{2.007051in}}%
\pgfpathlineto{\pgfqpoint{1.689992in}{2.010001in}}%
\pgfpathlineto{\pgfqpoint{1.694533in}{2.010001in}}%
\pgfpathlineto{\pgfqpoint{1.694533in}{2.007051in}}%
\pgfpathmoveto{\pgfqpoint{1.694533in}{2.007051in}}%
\pgfpathlineto{\pgfqpoint{1.694533in}{2.007051in}}%
\pgfpathlineto{\pgfqpoint{1.694533in}{2.010001in}}%
\pgfpathlineto{\pgfqpoint{1.699074in}{2.010001in}}%
\pgfpathlineto{\pgfqpoint{1.699074in}{2.007051in}}%
\pgfpathmoveto{\pgfqpoint{1.699074in}{2.007051in}}%
\pgfpathlineto{\pgfqpoint{1.699074in}{2.007051in}}%
\pgfpathlineto{\pgfqpoint{1.699074in}{2.010001in}}%
\pgfpathlineto{\pgfqpoint{1.703614in}{2.010001in}}%
\pgfpathlineto{\pgfqpoint{1.703614in}{2.007051in}}%
\pgfpathmoveto{\pgfqpoint{1.703614in}{2.007051in}}%
\pgfpathlineto{\pgfqpoint{1.703614in}{2.007051in}}%
\pgfpathlineto{\pgfqpoint{1.703614in}{2.010001in}}%
\pgfpathlineto{\pgfqpoint{1.708155in}{2.010001in}}%
\pgfpathlineto{\pgfqpoint{1.708155in}{2.007051in}}%
\pgfpathmoveto{\pgfqpoint{1.708155in}{2.007051in}}%
\pgfpathlineto{\pgfqpoint{1.708155in}{2.007051in}}%
\pgfpathlineto{\pgfqpoint{1.708155in}{2.010001in}}%
\pgfpathlineto{\pgfqpoint{1.712696in}{2.010001in}}%
\pgfpathlineto{\pgfqpoint{1.712696in}{2.007051in}}%
\pgfpathmoveto{\pgfqpoint{1.712696in}{2.007051in}}%
\pgfpathlineto{\pgfqpoint{1.712696in}{2.007051in}}%
\pgfpathlineto{\pgfqpoint{1.712696in}{2.010001in}}%
\pgfpathlineto{\pgfqpoint{1.717237in}{2.010001in}}%
\pgfpathlineto{\pgfqpoint{1.717237in}{2.007051in}}%
\pgfpathmoveto{\pgfqpoint{1.717237in}{2.007051in}}%
\pgfpathlineto{\pgfqpoint{1.717237in}{2.007051in}}%
\pgfpathlineto{\pgfqpoint{1.717237in}{2.010001in}}%
\pgfpathlineto{\pgfqpoint{1.721778in}{2.010001in}}%
\pgfpathlineto{\pgfqpoint{1.721778in}{2.007051in}}%
\pgfpathmoveto{\pgfqpoint{1.721778in}{2.007051in}}%
\pgfpathlineto{\pgfqpoint{1.721778in}{2.007051in}}%
\pgfpathlineto{\pgfqpoint{1.721778in}{2.010001in}}%
\pgfpathlineto{\pgfqpoint{1.726319in}{2.010001in}}%
\pgfpathlineto{\pgfqpoint{1.726319in}{2.007051in}}%
\pgfpathmoveto{\pgfqpoint{1.726319in}{2.007051in}}%
\pgfpathlineto{\pgfqpoint{1.726319in}{2.007051in}}%
\pgfpathlineto{\pgfqpoint{1.726319in}{2.010001in}}%
\pgfpathlineto{\pgfqpoint{1.730860in}{2.010001in}}%
\pgfpathlineto{\pgfqpoint{1.730860in}{2.007051in}}%
\pgfpathmoveto{\pgfqpoint{1.730860in}{2.007051in}}%
\pgfpathlineto{\pgfqpoint{1.730860in}{2.007051in}}%
\pgfpathlineto{\pgfqpoint{1.730860in}{2.010001in}}%
\pgfpathlineto{\pgfqpoint{1.735400in}{2.010001in}}%
\pgfpathlineto{\pgfqpoint{1.735400in}{2.007051in}}%
\pgfpathmoveto{\pgfqpoint{1.735400in}{2.007051in}}%
\pgfpathlineto{\pgfqpoint{1.735400in}{2.007051in}}%
\pgfpathlineto{\pgfqpoint{1.735400in}{2.010001in}}%
\pgfpathlineto{\pgfqpoint{1.739941in}{2.010001in}}%
\pgfpathlineto{\pgfqpoint{1.739941in}{2.007051in}}%
\pgfpathmoveto{\pgfqpoint{1.739941in}{2.007051in}}%
\pgfpathlineto{\pgfqpoint{1.739941in}{2.007051in}}%
\pgfpathlineto{\pgfqpoint{1.739941in}{2.010001in}}%
\pgfpathlineto{\pgfqpoint{1.744482in}{2.010001in}}%
\pgfpathlineto{\pgfqpoint{1.744482in}{2.007051in}}%
\pgfpathmoveto{\pgfqpoint{1.744482in}{2.007051in}}%
\pgfpathlineto{\pgfqpoint{1.744482in}{2.007051in}}%
\pgfpathlineto{\pgfqpoint{1.744482in}{2.010001in}}%
\pgfpathlineto{\pgfqpoint{1.749023in}{2.010001in}}%
\pgfpathlineto{\pgfqpoint{1.749023in}{2.007051in}}%
\pgfpathmoveto{\pgfqpoint{1.749023in}{2.007051in}}%
\pgfpathlineto{\pgfqpoint{1.749023in}{2.007051in}}%
\pgfpathlineto{\pgfqpoint{1.749023in}{2.010001in}}%
\pgfpathlineto{\pgfqpoint{1.753564in}{2.010001in}}%
\pgfpathlineto{\pgfqpoint{1.753564in}{2.007051in}}%
\pgfpathmoveto{\pgfqpoint{1.753564in}{2.007051in}}%
\pgfpathlineto{\pgfqpoint{1.753564in}{2.007051in}}%
\pgfpathlineto{\pgfqpoint{1.753564in}{2.010001in}}%
\pgfpathlineto{\pgfqpoint{1.758105in}{2.010001in}}%
\pgfpathlineto{\pgfqpoint{1.758105in}{2.007051in}}%
\pgfpathmoveto{\pgfqpoint{1.758105in}{2.007051in}}%
\pgfpathlineto{\pgfqpoint{1.758105in}{2.007051in}}%
\pgfpathlineto{\pgfqpoint{1.758105in}{2.010001in}}%
\pgfpathlineto{\pgfqpoint{1.762646in}{2.010001in}}%
\pgfpathlineto{\pgfqpoint{1.762646in}{2.007051in}}%
\pgfpathmoveto{\pgfqpoint{1.762646in}{2.007051in}}%
\pgfpathlineto{\pgfqpoint{1.762646in}{2.007051in}}%
\pgfpathlineto{\pgfqpoint{1.762646in}{2.010001in}}%
\pgfpathlineto{\pgfqpoint{1.767186in}{2.010001in}}%
\pgfpathlineto{\pgfqpoint{1.767186in}{2.007051in}}%
\pgfpathmoveto{\pgfqpoint{1.767186in}{2.007051in}}%
\pgfpathlineto{\pgfqpoint{1.767186in}{2.007051in}}%
\pgfpathlineto{\pgfqpoint{1.767186in}{2.010001in}}%
\pgfpathlineto{\pgfqpoint{1.771727in}{2.010001in}}%
\pgfpathlineto{\pgfqpoint{1.771727in}{2.007051in}}%
\pgfpathmoveto{\pgfqpoint{1.771727in}{2.007051in}}%
\pgfpathlineto{\pgfqpoint{1.771727in}{2.007051in}}%
\pgfpathlineto{\pgfqpoint{1.771727in}{2.010001in}}%
\pgfpathlineto{\pgfqpoint{1.776268in}{2.010001in}}%
\pgfpathlineto{\pgfqpoint{1.776268in}{2.007051in}}%
\pgfpathmoveto{\pgfqpoint{1.776268in}{2.007051in}}%
\pgfpathlineto{\pgfqpoint{1.776268in}{2.007051in}}%
\pgfpathlineto{\pgfqpoint{1.776268in}{2.010001in}}%
\pgfpathlineto{\pgfqpoint{1.780809in}{2.010001in}}%
\pgfpathlineto{\pgfqpoint{1.780809in}{2.007051in}}%
\pgfpathmoveto{\pgfqpoint{1.780809in}{2.007051in}}%
\pgfpathlineto{\pgfqpoint{1.780809in}{2.007051in}}%
\pgfpathlineto{\pgfqpoint{1.780809in}{2.010001in}}%
\pgfpathlineto{\pgfqpoint{1.785350in}{2.010001in}}%
\pgfpathlineto{\pgfqpoint{1.785350in}{2.007051in}}%
\pgfpathmoveto{\pgfqpoint{1.785350in}{2.007051in}}%
\pgfpathlineto{\pgfqpoint{1.785350in}{2.007051in}}%
\pgfpathlineto{\pgfqpoint{1.785350in}{2.010001in}}%
\pgfpathlineto{\pgfqpoint{1.789891in}{2.010001in}}%
\pgfpathlineto{\pgfqpoint{1.789891in}{2.007051in}}%
\pgfpathmoveto{\pgfqpoint{1.789891in}{2.007051in}}%
\pgfpathlineto{\pgfqpoint{1.789891in}{2.007051in}}%
\pgfpathlineto{\pgfqpoint{1.789891in}{2.010001in}}%
\pgfpathlineto{\pgfqpoint{1.794432in}{2.010001in}}%
\pgfpathlineto{\pgfqpoint{1.794432in}{2.007051in}}%
\pgfpathmoveto{\pgfqpoint{1.794432in}{2.007051in}}%
\pgfpathlineto{\pgfqpoint{1.794432in}{2.007051in}}%
\pgfpathlineto{\pgfqpoint{1.794432in}{2.010001in}}%
\pgfpathlineto{\pgfqpoint{1.798973in}{2.010001in}}%
\pgfpathlineto{\pgfqpoint{1.798973in}{2.007051in}}%
\pgfpathmoveto{\pgfqpoint{1.798973in}{2.007051in}}%
\pgfpathlineto{\pgfqpoint{1.798973in}{2.007051in}}%
\pgfpathlineto{\pgfqpoint{1.798973in}{2.010001in}}%
\pgfpathlineto{\pgfqpoint{1.803514in}{2.010001in}}%
\pgfpathlineto{\pgfqpoint{1.803514in}{2.007051in}}%
\pgfpathmoveto{\pgfqpoint{1.803514in}{2.007051in}}%
\pgfpathlineto{\pgfqpoint{1.803514in}{2.007051in}}%
\pgfpathlineto{\pgfqpoint{1.803514in}{2.010001in}}%
\pgfpathlineto{\pgfqpoint{1.808055in}{2.010001in}}%
\pgfpathlineto{\pgfqpoint{1.808055in}{2.007051in}}%
\pgfpathmoveto{\pgfqpoint{1.808055in}{2.007051in}}%
\pgfpathlineto{\pgfqpoint{1.808055in}{2.007051in}}%
\pgfpathlineto{\pgfqpoint{1.808055in}{2.010001in}}%
\pgfpathlineto{\pgfqpoint{1.812596in}{2.010001in}}%
\pgfpathlineto{\pgfqpoint{1.812596in}{2.007051in}}%
\pgfpathmoveto{\pgfqpoint{1.812596in}{2.007051in}}%
\pgfpathlineto{\pgfqpoint{1.812596in}{2.007051in}}%
\pgfpathlineto{\pgfqpoint{1.812596in}{2.010001in}}%
\pgfpathlineto{\pgfqpoint{1.817136in}{2.010001in}}%
\pgfpathlineto{\pgfqpoint{1.817136in}{2.007051in}}%
\pgfpathmoveto{\pgfqpoint{1.817136in}{2.007051in}}%
\pgfpathlineto{\pgfqpoint{1.817136in}{2.007051in}}%
\pgfpathlineto{\pgfqpoint{1.817136in}{2.010001in}}%
\pgfpathlineto{\pgfqpoint{1.821677in}{2.010001in}}%
\pgfpathlineto{\pgfqpoint{1.821677in}{2.007051in}}%
\pgfpathmoveto{\pgfqpoint{1.821677in}{2.007051in}}%
\pgfpathlineto{\pgfqpoint{1.821677in}{2.007051in}}%
\pgfpathlineto{\pgfqpoint{1.821677in}{2.010001in}}%
\pgfpathlineto{\pgfqpoint{1.826218in}{2.010001in}}%
\pgfpathlineto{\pgfqpoint{1.826218in}{2.007051in}}%
\pgfpathmoveto{\pgfqpoint{1.826218in}{2.007051in}}%
\pgfpathlineto{\pgfqpoint{1.826218in}{2.007051in}}%
\pgfpathlineto{\pgfqpoint{1.826218in}{2.010001in}}%
\pgfpathlineto{\pgfqpoint{1.830759in}{2.010001in}}%
\pgfpathlineto{\pgfqpoint{1.830759in}{2.007051in}}%
\pgfpathmoveto{\pgfqpoint{1.830759in}{2.007051in}}%
\pgfpathlineto{\pgfqpoint{1.830759in}{2.007051in}}%
\pgfpathlineto{\pgfqpoint{1.830759in}{2.010001in}}%
\pgfpathlineto{\pgfqpoint{1.835300in}{2.010001in}}%
\pgfpathlineto{\pgfqpoint{1.835300in}{2.007051in}}%
\pgfpathmoveto{\pgfqpoint{1.835300in}{2.007051in}}%
\pgfpathlineto{\pgfqpoint{1.835300in}{2.007051in}}%
\pgfpathlineto{\pgfqpoint{1.835300in}{2.010001in}}%
\pgfpathlineto{\pgfqpoint{1.839841in}{2.010001in}}%
\pgfpathlineto{\pgfqpoint{1.839841in}{2.007051in}}%
\pgfpathmoveto{\pgfqpoint{1.839841in}{2.007051in}}%
\pgfpathlineto{\pgfqpoint{1.839841in}{2.007051in}}%
\pgfpathlineto{\pgfqpoint{1.839841in}{2.010001in}}%
\pgfpathlineto{\pgfqpoint{1.844382in}{2.010001in}}%
\pgfpathlineto{\pgfqpoint{1.844382in}{2.007051in}}%
\pgfpathmoveto{\pgfqpoint{1.844382in}{2.007051in}}%
\pgfpathlineto{\pgfqpoint{1.844382in}{2.007051in}}%
\pgfpathlineto{\pgfqpoint{1.844382in}{2.010001in}}%
\pgfpathlineto{\pgfqpoint{1.848923in}{2.010001in}}%
\pgfpathlineto{\pgfqpoint{1.848923in}{2.007051in}}%
\pgfpathmoveto{\pgfqpoint{1.848923in}{2.007051in}}%
\pgfpathlineto{\pgfqpoint{1.848923in}{2.007051in}}%
\pgfpathlineto{\pgfqpoint{1.848923in}{2.010001in}}%
\pgfpathlineto{\pgfqpoint{1.853464in}{2.010001in}}%
\pgfpathlineto{\pgfqpoint{1.853464in}{2.007051in}}%
\pgfpathmoveto{\pgfqpoint{1.853464in}{2.007051in}}%
\pgfpathlineto{\pgfqpoint{1.853464in}{2.007051in}}%
\pgfpathlineto{\pgfqpoint{1.853464in}{2.010001in}}%
\pgfpathlineto{\pgfqpoint{1.858005in}{2.010001in}}%
\pgfpathlineto{\pgfqpoint{1.858005in}{2.007051in}}%
\pgfpathmoveto{\pgfqpoint{1.858005in}{2.007051in}}%
\pgfpathlineto{\pgfqpoint{1.858005in}{2.007051in}}%
\pgfpathlineto{\pgfqpoint{1.858005in}{2.010001in}}%
\pgfpathlineto{\pgfqpoint{1.862546in}{2.010001in}}%
\pgfpathlineto{\pgfqpoint{1.862546in}{2.007051in}}%
\pgfpathmoveto{\pgfqpoint{1.862546in}{2.007051in}}%
\pgfpathlineto{\pgfqpoint{1.862546in}{2.007051in}}%
\pgfpathlineto{\pgfqpoint{1.862546in}{2.010001in}}%
\pgfpathlineto{\pgfqpoint{1.867086in}{2.010001in}}%
\pgfpathlineto{\pgfqpoint{1.867086in}{2.007051in}}%
\pgfpathmoveto{\pgfqpoint{1.867086in}{2.007051in}}%
\pgfpathlineto{\pgfqpoint{1.867086in}{2.007051in}}%
\pgfpathlineto{\pgfqpoint{1.867086in}{2.010001in}}%
\pgfpathlineto{\pgfqpoint{1.871627in}{2.010001in}}%
\pgfpathlineto{\pgfqpoint{1.871627in}{2.007051in}}%
\pgfpathmoveto{\pgfqpoint{1.871627in}{2.007051in}}%
\pgfpathlineto{\pgfqpoint{1.871627in}{2.007051in}}%
\pgfpathlineto{\pgfqpoint{1.871627in}{2.010001in}}%
\pgfpathlineto{\pgfqpoint{1.876168in}{2.010001in}}%
\pgfpathlineto{\pgfqpoint{1.876168in}{2.007051in}}%
\pgfpathmoveto{\pgfqpoint{1.876168in}{2.007051in}}%
\pgfpathlineto{\pgfqpoint{1.876168in}{2.007051in}}%
\pgfpathlineto{\pgfqpoint{1.876168in}{2.010001in}}%
\pgfpathlineto{\pgfqpoint{1.880709in}{2.010001in}}%
\pgfpathlineto{\pgfqpoint{1.880709in}{2.007051in}}%
\pgfpathmoveto{\pgfqpoint{1.880709in}{2.007051in}}%
\pgfpathlineto{\pgfqpoint{1.880709in}{2.007051in}}%
\pgfpathlineto{\pgfqpoint{1.880709in}{2.010001in}}%
\pgfpathlineto{\pgfqpoint{1.885250in}{2.010001in}}%
\pgfpathlineto{\pgfqpoint{1.885250in}{2.007051in}}%
\pgfpathmoveto{\pgfqpoint{1.885250in}{2.007051in}}%
\pgfpathlineto{\pgfqpoint{1.885250in}{2.007051in}}%
\pgfpathlineto{\pgfqpoint{1.885250in}{2.010001in}}%
\pgfpathlineto{\pgfqpoint{1.889791in}{2.010001in}}%
\pgfpathlineto{\pgfqpoint{1.889791in}{2.007051in}}%
\pgfpathmoveto{\pgfqpoint{1.889791in}{2.007051in}}%
\pgfpathlineto{\pgfqpoint{1.889791in}{2.007051in}}%
\pgfpathlineto{\pgfqpoint{1.889791in}{2.010001in}}%
\pgfpathlineto{\pgfqpoint{1.894332in}{2.010001in}}%
\pgfpathlineto{\pgfqpoint{1.894332in}{2.007051in}}%
\pgfpathmoveto{\pgfqpoint{1.894332in}{2.007051in}}%
\pgfpathlineto{\pgfqpoint{1.894332in}{2.007051in}}%
\pgfpathlineto{\pgfqpoint{1.894332in}{2.010001in}}%
\pgfpathlineto{\pgfqpoint{1.898873in}{2.010001in}}%
\pgfpathlineto{\pgfqpoint{1.898873in}{2.007051in}}%
\pgfpathmoveto{\pgfqpoint{1.898873in}{2.007051in}}%
\pgfpathlineto{\pgfqpoint{1.898873in}{2.007051in}}%
\pgfpathlineto{\pgfqpoint{1.898873in}{2.010001in}}%
\pgfpathlineto{\pgfqpoint{1.903414in}{2.010001in}}%
\pgfpathlineto{\pgfqpoint{1.903414in}{2.007051in}}%
\pgfpathmoveto{\pgfqpoint{1.903414in}{2.007051in}}%
\pgfpathlineto{\pgfqpoint{1.903414in}{2.007051in}}%
\pgfpathlineto{\pgfqpoint{1.903414in}{2.010001in}}%
\pgfpathlineto{\pgfqpoint{1.907955in}{2.010001in}}%
\pgfpathlineto{\pgfqpoint{1.907955in}{2.007051in}}%
\pgfpathmoveto{\pgfqpoint{1.907955in}{2.007051in}}%
\pgfpathlineto{\pgfqpoint{1.907955in}{2.007051in}}%
\pgfpathlineto{\pgfqpoint{1.907955in}{2.010001in}}%
\pgfpathlineto{\pgfqpoint{1.912496in}{2.010001in}}%
\pgfpathlineto{\pgfqpoint{1.912496in}{2.007051in}}%
\pgfpathmoveto{\pgfqpoint{1.912496in}{2.007051in}}%
\pgfpathlineto{\pgfqpoint{1.912496in}{2.007051in}}%
\pgfpathlineto{\pgfqpoint{1.912496in}{2.010001in}}%
\pgfpathlineto{\pgfqpoint{1.917037in}{2.010001in}}%
\pgfpathlineto{\pgfqpoint{1.917037in}{2.007051in}}%
\pgfpathmoveto{\pgfqpoint{1.917037in}{2.007051in}}%
\pgfpathlineto{\pgfqpoint{1.917037in}{2.007051in}}%
\pgfpathlineto{\pgfqpoint{1.917037in}{2.010001in}}%
\pgfpathlineto{\pgfqpoint{1.921578in}{2.010001in}}%
\pgfpathlineto{\pgfqpoint{1.921578in}{2.007051in}}%
\pgfpathmoveto{\pgfqpoint{1.921578in}{2.007051in}}%
\pgfpathlineto{\pgfqpoint{1.921578in}{2.007051in}}%
\pgfpathlineto{\pgfqpoint{1.921578in}{2.010001in}}%
\pgfpathlineto{\pgfqpoint{1.926119in}{2.010001in}}%
\pgfpathlineto{\pgfqpoint{1.926119in}{2.007051in}}%
\pgfpathmoveto{\pgfqpoint{1.926119in}{2.007051in}}%
\pgfpathlineto{\pgfqpoint{1.926119in}{2.007051in}}%
\pgfpathlineto{\pgfqpoint{1.926119in}{2.010001in}}%
\pgfpathlineto{\pgfqpoint{1.930660in}{2.010001in}}%
\pgfpathlineto{\pgfqpoint{1.930660in}{2.007051in}}%
\pgfpathmoveto{\pgfqpoint{1.930660in}{2.007051in}}%
\pgfpathlineto{\pgfqpoint{1.930660in}{2.007051in}}%
\pgfpathlineto{\pgfqpoint{1.930660in}{2.010001in}}%
\pgfpathlineto{\pgfqpoint{1.935202in}{2.010001in}}%
\pgfpathlineto{\pgfqpoint{1.935202in}{2.007051in}}%
\pgfpathmoveto{\pgfqpoint{1.935202in}{2.007051in}}%
\pgfpathlineto{\pgfqpoint{1.935202in}{2.007051in}}%
\pgfpathlineto{\pgfqpoint{1.935202in}{2.010001in}}%
\pgfpathlineto{\pgfqpoint{1.939743in}{2.010001in}}%
\pgfpathlineto{\pgfqpoint{1.939743in}{2.007051in}}%
\pgfpathmoveto{\pgfqpoint{1.939743in}{2.007051in}}%
\pgfpathlineto{\pgfqpoint{1.939743in}{2.007051in}}%
\pgfpathlineto{\pgfqpoint{1.939743in}{2.010001in}}%
\pgfpathlineto{\pgfqpoint{1.944284in}{2.010001in}}%
\pgfpathlineto{\pgfqpoint{1.944284in}{2.007051in}}%
\pgfpathmoveto{\pgfqpoint{1.944284in}{2.007051in}}%
\pgfpathlineto{\pgfqpoint{1.944284in}{2.007051in}}%
\pgfpathlineto{\pgfqpoint{1.944284in}{2.010001in}}%
\pgfpathlineto{\pgfqpoint{1.948825in}{2.010001in}}%
\pgfpathlineto{\pgfqpoint{1.948825in}{2.007051in}}%
\pgfpathmoveto{\pgfqpoint{1.948825in}{2.007051in}}%
\pgfpathlineto{\pgfqpoint{1.948825in}{2.007051in}}%
\pgfpathlineto{\pgfqpoint{1.948825in}{2.010001in}}%
\pgfpathlineto{\pgfqpoint{1.953366in}{2.010001in}}%
\pgfpathlineto{\pgfqpoint{1.953366in}{2.007051in}}%
\pgfpathmoveto{\pgfqpoint{1.953366in}{2.007051in}}%
\pgfpathlineto{\pgfqpoint{1.953366in}{2.007051in}}%
\pgfpathlineto{\pgfqpoint{1.953366in}{2.010001in}}%
\pgfpathlineto{\pgfqpoint{1.957908in}{2.010001in}}%
\pgfpathlineto{\pgfqpoint{1.957908in}{2.007051in}}%
\pgfpathmoveto{\pgfqpoint{1.957908in}{2.007051in}}%
\pgfpathlineto{\pgfqpoint{1.957908in}{2.007051in}}%
\pgfpathlineto{\pgfqpoint{1.957908in}{2.010001in}}%
\pgfpathlineto{\pgfqpoint{1.962449in}{2.010001in}}%
\pgfpathlineto{\pgfqpoint{1.962449in}{2.007051in}}%
\pgfpathmoveto{\pgfqpoint{1.962449in}{2.007051in}}%
\pgfpathlineto{\pgfqpoint{1.962449in}{2.007051in}}%
\pgfpathlineto{\pgfqpoint{1.962449in}{2.010001in}}%
\pgfpathlineto{\pgfqpoint{1.966990in}{2.010001in}}%
\pgfpathlineto{\pgfqpoint{1.966990in}{2.007051in}}%
\pgfpathmoveto{\pgfqpoint{1.966990in}{2.007051in}}%
\pgfpathlineto{\pgfqpoint{1.966990in}{2.007051in}}%
\pgfpathlineto{\pgfqpoint{1.966990in}{2.010001in}}%
\pgfpathlineto{\pgfqpoint{1.971531in}{2.010001in}}%
\pgfpathlineto{\pgfqpoint{1.971531in}{2.007051in}}%
\pgfpathmoveto{\pgfqpoint{1.971531in}{2.007051in}}%
\pgfpathlineto{\pgfqpoint{1.971531in}{2.007051in}}%
\pgfpathlineto{\pgfqpoint{1.971531in}{2.010001in}}%
\pgfpathlineto{\pgfqpoint{1.976072in}{2.010001in}}%
\pgfpathlineto{\pgfqpoint{1.976072in}{2.007051in}}%
\pgfpathmoveto{\pgfqpoint{1.976072in}{2.007051in}}%
\pgfpathlineto{\pgfqpoint{1.976072in}{2.007051in}}%
\pgfpathlineto{\pgfqpoint{1.976072in}{2.010001in}}%
\pgfpathlineto{\pgfqpoint{1.980614in}{2.010001in}}%
\pgfpathlineto{\pgfqpoint{1.980614in}{2.007051in}}%
\pgfpathmoveto{\pgfqpoint{1.980614in}{2.007051in}}%
\pgfpathlineto{\pgfqpoint{1.980614in}{2.007051in}}%
\pgfpathlineto{\pgfqpoint{1.980614in}{2.010001in}}%
\pgfpathlineto{\pgfqpoint{1.985155in}{2.010001in}}%
\pgfpathlineto{\pgfqpoint{1.985155in}{2.007051in}}%
\pgfpathmoveto{\pgfqpoint{1.985155in}{2.007051in}}%
\pgfpathlineto{\pgfqpoint{1.985155in}{2.007051in}}%
\pgfpathlineto{\pgfqpoint{1.985155in}{2.010001in}}%
\pgfpathlineto{\pgfqpoint{1.989696in}{2.010001in}}%
\pgfpathlineto{\pgfqpoint{1.989696in}{2.007051in}}%
\pgfpathmoveto{\pgfqpoint{1.989696in}{2.007051in}}%
\pgfpathlineto{\pgfqpoint{1.989696in}{2.007051in}}%
\pgfpathlineto{\pgfqpoint{1.989696in}{2.010001in}}%
\pgfpathlineto{\pgfqpoint{1.994237in}{2.010001in}}%
\pgfpathlineto{\pgfqpoint{1.994237in}{2.007051in}}%
\pgfpathmoveto{\pgfqpoint{1.994237in}{2.007051in}}%
\pgfpathlineto{\pgfqpoint{1.994237in}{2.007051in}}%
\pgfpathlineto{\pgfqpoint{1.994237in}{2.010001in}}%
\pgfpathlineto{\pgfqpoint{1.998778in}{2.010001in}}%
\pgfpathlineto{\pgfqpoint{1.998778in}{2.007051in}}%
\pgfpathmoveto{\pgfqpoint{1.998778in}{2.007051in}}%
\pgfpathlineto{\pgfqpoint{1.998778in}{2.007051in}}%
\pgfpathlineto{\pgfqpoint{1.998778in}{2.010001in}}%
\pgfpathlineto{\pgfqpoint{2.003320in}{2.010001in}}%
\pgfpathlineto{\pgfqpoint{2.003320in}{2.007051in}}%
\pgfpathmoveto{\pgfqpoint{2.003320in}{2.007051in}}%
\pgfpathlineto{\pgfqpoint{2.003320in}{2.007051in}}%
\pgfpathlineto{\pgfqpoint{2.003320in}{2.010001in}}%
\pgfpathlineto{\pgfqpoint{2.007861in}{2.010001in}}%
\pgfpathlineto{\pgfqpoint{2.007861in}{2.007051in}}%
\pgfpathmoveto{\pgfqpoint{2.007861in}{2.007051in}}%
\pgfpathlineto{\pgfqpoint{2.007861in}{2.007051in}}%
\pgfpathlineto{\pgfqpoint{2.007861in}{2.010001in}}%
\pgfpathlineto{\pgfqpoint{2.012402in}{2.010001in}}%
\pgfpathlineto{\pgfqpoint{2.012402in}{2.007051in}}%
\pgfpathmoveto{\pgfqpoint{2.012402in}{2.007051in}}%
\pgfpathlineto{\pgfqpoint{2.012402in}{2.007051in}}%
\pgfpathlineto{\pgfqpoint{2.012402in}{2.010001in}}%
\pgfpathlineto{\pgfqpoint{2.016943in}{2.010001in}}%
\pgfpathlineto{\pgfqpoint{2.016943in}{2.007051in}}%
\pgfpathmoveto{\pgfqpoint{2.016943in}{2.007051in}}%
\pgfpathlineto{\pgfqpoint{2.016943in}{2.007051in}}%
\pgfpathlineto{\pgfqpoint{2.016943in}{2.010001in}}%
\pgfpathlineto{\pgfqpoint{2.021484in}{2.010001in}}%
\pgfpathlineto{\pgfqpoint{2.021484in}{2.007051in}}%
\pgfpathmoveto{\pgfqpoint{2.021484in}{2.007051in}}%
\pgfpathlineto{\pgfqpoint{2.021484in}{2.007051in}}%
\pgfpathlineto{\pgfqpoint{2.021484in}{2.010001in}}%
\pgfpathlineto{\pgfqpoint{2.026026in}{2.010001in}}%
\pgfpathlineto{\pgfqpoint{2.026026in}{2.007051in}}%
\pgfpathmoveto{\pgfqpoint{2.026026in}{2.007051in}}%
\pgfpathlineto{\pgfqpoint{2.026026in}{2.007051in}}%
\pgfpathlineto{\pgfqpoint{2.026026in}{2.010001in}}%
\pgfpathlineto{\pgfqpoint{2.030567in}{2.010001in}}%
\pgfpathlineto{\pgfqpoint{2.030567in}{2.007051in}}%
\pgfpathmoveto{\pgfqpoint{2.030567in}{2.007051in}}%
\pgfpathlineto{\pgfqpoint{2.030567in}{2.007051in}}%
\pgfpathlineto{\pgfqpoint{2.030567in}{2.010001in}}%
\pgfpathlineto{\pgfqpoint{2.035108in}{2.010001in}}%
\pgfpathlineto{\pgfqpoint{2.035108in}{2.007051in}}%
\pgfpathmoveto{\pgfqpoint{2.035108in}{2.007051in}}%
\pgfpathlineto{\pgfqpoint{2.035108in}{2.007051in}}%
\pgfpathlineto{\pgfqpoint{2.035108in}{2.010001in}}%
\pgfpathlineto{\pgfqpoint{2.039649in}{2.010001in}}%
\pgfpathlineto{\pgfqpoint{2.039649in}{2.007051in}}%
\pgfpathmoveto{\pgfqpoint{2.039649in}{2.007051in}}%
\pgfpathlineto{\pgfqpoint{2.039649in}{2.007051in}}%
\pgfpathlineto{\pgfqpoint{2.039649in}{2.010001in}}%
\pgfpathlineto{\pgfqpoint{2.044190in}{2.010001in}}%
\pgfpathlineto{\pgfqpoint{2.044190in}{2.007051in}}%
\pgfpathmoveto{\pgfqpoint{2.044190in}{2.007051in}}%
\pgfpathlineto{\pgfqpoint{2.044190in}{2.007051in}}%
\pgfpathlineto{\pgfqpoint{2.044190in}{2.010001in}}%
\pgfpathlineto{\pgfqpoint{2.048732in}{2.010001in}}%
\pgfpathlineto{\pgfqpoint{2.048732in}{2.007051in}}%
\pgfpathmoveto{\pgfqpoint{2.048732in}{2.007051in}}%
\pgfpathlineto{\pgfqpoint{2.048732in}{2.007051in}}%
\pgfpathlineto{\pgfqpoint{2.048732in}{2.010001in}}%
\pgfpathlineto{\pgfqpoint{2.053273in}{2.010001in}}%
\pgfpathlineto{\pgfqpoint{2.053273in}{2.007051in}}%
\pgfpathmoveto{\pgfqpoint{2.053273in}{2.007051in}}%
\pgfpathlineto{\pgfqpoint{2.053273in}{2.007051in}}%
\pgfpathlineto{\pgfqpoint{2.053273in}{2.010001in}}%
\pgfpathlineto{\pgfqpoint{2.057814in}{2.010001in}}%
\pgfpathlineto{\pgfqpoint{2.057814in}{2.007051in}}%
\pgfpathmoveto{\pgfqpoint{2.057814in}{2.007051in}}%
\pgfpathlineto{\pgfqpoint{2.057814in}{2.007051in}}%
\pgfpathlineto{\pgfqpoint{2.057814in}{2.010001in}}%
\pgfpathlineto{\pgfqpoint{2.062355in}{2.010001in}}%
\pgfpathlineto{\pgfqpoint{2.062355in}{2.007051in}}%
\pgfpathmoveto{\pgfqpoint{2.062355in}{2.007051in}}%
\pgfpathlineto{\pgfqpoint{2.062355in}{2.007051in}}%
\pgfpathlineto{\pgfqpoint{2.062355in}{2.010001in}}%
\pgfpathlineto{\pgfqpoint{2.066896in}{2.010001in}}%
\pgfpathlineto{\pgfqpoint{2.066896in}{2.007051in}}%
\pgfpathmoveto{\pgfqpoint{2.066896in}{2.007051in}}%
\pgfpathlineto{\pgfqpoint{2.066896in}{2.007051in}}%
\pgfpathlineto{\pgfqpoint{2.066896in}{2.010001in}}%
\pgfpathlineto{\pgfqpoint{2.071437in}{2.010001in}}%
\pgfpathlineto{\pgfqpoint{2.071437in}{2.007051in}}%
\pgfpathmoveto{\pgfqpoint{2.071437in}{2.007051in}}%
\pgfpathlineto{\pgfqpoint{2.071437in}{2.007051in}}%
\pgfpathlineto{\pgfqpoint{2.071437in}{2.010001in}}%
\pgfpathlineto{\pgfqpoint{2.075977in}{2.010001in}}%
\pgfpathlineto{\pgfqpoint{2.075977in}{2.007051in}}%
\pgfpathmoveto{\pgfqpoint{2.075977in}{2.007051in}}%
\pgfpathlineto{\pgfqpoint{2.075977in}{2.007051in}}%
\pgfpathlineto{\pgfqpoint{2.075977in}{2.010001in}}%
\pgfpathlineto{\pgfqpoint{2.080518in}{2.010001in}}%
\pgfpathlineto{\pgfqpoint{2.080518in}{2.007051in}}%
\pgfpathmoveto{\pgfqpoint{2.080518in}{2.007051in}}%
\pgfpathlineto{\pgfqpoint{2.080518in}{2.007051in}}%
\pgfpathlineto{\pgfqpoint{2.080518in}{2.010001in}}%
\pgfpathlineto{\pgfqpoint{2.085059in}{2.010001in}}%
\pgfpathlineto{\pgfqpoint{2.085059in}{2.007051in}}%
\pgfpathmoveto{\pgfqpoint{2.085059in}{2.007051in}}%
\pgfpathlineto{\pgfqpoint{2.085059in}{2.007051in}}%
\pgfpathlineto{\pgfqpoint{2.085059in}{2.010001in}}%
\pgfpathlineto{\pgfqpoint{2.089600in}{2.010001in}}%
\pgfpathlineto{\pgfqpoint{2.089600in}{2.007051in}}%
\pgfpathmoveto{\pgfqpoint{2.089600in}{2.007051in}}%
\pgfpathlineto{\pgfqpoint{2.089600in}{2.007051in}}%
\pgfpathlineto{\pgfqpoint{2.089600in}{2.010001in}}%
\pgfpathlineto{\pgfqpoint{2.094141in}{2.010001in}}%
\pgfpathlineto{\pgfqpoint{2.094141in}{2.007051in}}%
\pgfpathmoveto{\pgfqpoint{2.094141in}{2.007051in}}%
\pgfpathlineto{\pgfqpoint{2.094141in}{2.007051in}}%
\pgfpathlineto{\pgfqpoint{2.094141in}{2.010001in}}%
\pgfpathlineto{\pgfqpoint{2.098682in}{2.010001in}}%
\pgfpathlineto{\pgfqpoint{2.098682in}{2.007051in}}%
\pgfpathmoveto{\pgfqpoint{2.098682in}{2.007051in}}%
\pgfpathlineto{\pgfqpoint{2.098682in}{2.007051in}}%
\pgfpathlineto{\pgfqpoint{2.098682in}{2.010001in}}%
\pgfpathlineto{\pgfqpoint{2.103222in}{2.010001in}}%
\pgfpathlineto{\pgfqpoint{2.103222in}{2.007051in}}%
\pgfpathmoveto{\pgfqpoint{2.103222in}{2.007051in}}%
\pgfpathlineto{\pgfqpoint{2.103222in}{2.007051in}}%
\pgfpathlineto{\pgfqpoint{2.103222in}{2.010001in}}%
\pgfpathlineto{\pgfqpoint{2.107763in}{2.010001in}}%
\pgfpathlineto{\pgfqpoint{2.107763in}{2.007051in}}%
\pgfpathmoveto{\pgfqpoint{2.107763in}{2.007051in}}%
\pgfpathlineto{\pgfqpoint{2.107763in}{2.007051in}}%
\pgfpathlineto{\pgfqpoint{2.107763in}{2.010001in}}%
\pgfpathlineto{\pgfqpoint{2.112304in}{2.010001in}}%
\pgfpathlineto{\pgfqpoint{2.112304in}{2.007051in}}%
\pgfpathmoveto{\pgfqpoint{2.112304in}{2.007051in}}%
\pgfpathlineto{\pgfqpoint{2.112304in}{2.007051in}}%
\pgfpathlineto{\pgfqpoint{2.112304in}{2.010001in}}%
\pgfpathlineto{\pgfqpoint{2.116845in}{2.010001in}}%
\pgfpathlineto{\pgfqpoint{2.116845in}{2.007051in}}%
\pgfpathmoveto{\pgfqpoint{2.116845in}{2.007051in}}%
\pgfpathlineto{\pgfqpoint{2.116845in}{2.007051in}}%
\pgfpathlineto{\pgfqpoint{2.116845in}{2.010001in}}%
\pgfpathlineto{\pgfqpoint{2.121386in}{2.010001in}}%
\pgfpathlineto{\pgfqpoint{2.121386in}{2.007051in}}%
\pgfpathmoveto{\pgfqpoint{2.121386in}{2.007051in}}%
\pgfpathlineto{\pgfqpoint{2.121386in}{2.007051in}}%
\pgfpathlineto{\pgfqpoint{2.121386in}{2.010001in}}%
\pgfpathlineto{\pgfqpoint{2.125927in}{2.010001in}}%
\pgfpathlineto{\pgfqpoint{2.125927in}{2.007051in}}%
\pgfpathmoveto{\pgfqpoint{2.125927in}{2.007051in}}%
\pgfpathlineto{\pgfqpoint{2.125927in}{2.007051in}}%
\pgfpathlineto{\pgfqpoint{2.125927in}{2.010001in}}%
\pgfpathlineto{\pgfqpoint{2.130467in}{2.010001in}}%
\pgfpathlineto{\pgfqpoint{2.130467in}{2.007051in}}%
\pgfpathmoveto{\pgfqpoint{2.130467in}{2.007051in}}%
\pgfpathlineto{\pgfqpoint{2.130467in}{2.007051in}}%
\pgfpathlineto{\pgfqpoint{2.130467in}{2.010001in}}%
\pgfpathlineto{\pgfqpoint{2.135008in}{2.010001in}}%
\pgfpathlineto{\pgfqpoint{2.135008in}{2.007051in}}%
\pgfpathmoveto{\pgfqpoint{2.135008in}{2.007051in}}%
\pgfpathlineto{\pgfqpoint{2.135008in}{2.007051in}}%
\pgfpathlineto{\pgfqpoint{2.135008in}{2.010001in}}%
\pgfpathlineto{\pgfqpoint{2.139549in}{2.010001in}}%
\pgfpathlineto{\pgfqpoint{2.139549in}{2.007051in}}%
\pgfpathmoveto{\pgfqpoint{2.139549in}{2.007051in}}%
\pgfpathlineto{\pgfqpoint{2.139549in}{2.007051in}}%
\pgfpathlineto{\pgfqpoint{2.139549in}{2.010001in}}%
\pgfpathlineto{\pgfqpoint{2.144090in}{2.010001in}}%
\pgfpathlineto{\pgfqpoint{2.144090in}{2.007051in}}%
\pgfpathmoveto{\pgfqpoint{2.144090in}{2.007051in}}%
\pgfpathlineto{\pgfqpoint{2.144090in}{2.007051in}}%
\pgfpathlineto{\pgfqpoint{2.144090in}{2.010001in}}%
\pgfpathlineto{\pgfqpoint{2.148631in}{2.010001in}}%
\pgfpathlineto{\pgfqpoint{2.148631in}{2.007051in}}%
\pgfpathmoveto{\pgfqpoint{2.148631in}{2.007051in}}%
\pgfpathlineto{\pgfqpoint{2.148631in}{2.007051in}}%
\pgfpathlineto{\pgfqpoint{2.148631in}{2.010001in}}%
\pgfpathlineto{\pgfqpoint{2.153172in}{2.010001in}}%
\pgfpathlineto{\pgfqpoint{2.153172in}{2.007051in}}%
\pgfpathmoveto{\pgfqpoint{2.153172in}{2.007051in}}%
\pgfpathlineto{\pgfqpoint{2.153172in}{2.007051in}}%
\pgfpathlineto{\pgfqpoint{2.153172in}{2.010001in}}%
\pgfpathlineto{\pgfqpoint{2.157712in}{2.010001in}}%
\pgfpathlineto{\pgfqpoint{2.157712in}{2.007051in}}%
\pgfpathmoveto{\pgfqpoint{2.157712in}{2.007051in}}%
\pgfpathlineto{\pgfqpoint{2.157712in}{2.007051in}}%
\pgfpathlineto{\pgfqpoint{2.157712in}{2.010001in}}%
\pgfpathlineto{\pgfqpoint{2.162253in}{2.010001in}}%
\pgfpathlineto{\pgfqpoint{2.162253in}{2.007051in}}%
\pgfpathmoveto{\pgfqpoint{2.162253in}{2.007051in}}%
\pgfpathlineto{\pgfqpoint{2.162253in}{2.007051in}}%
\pgfpathlineto{\pgfqpoint{2.162253in}{2.010001in}}%
\pgfpathlineto{\pgfqpoint{2.166794in}{2.010001in}}%
\pgfpathlineto{\pgfqpoint{2.166794in}{2.007051in}}%
\pgfpathmoveto{\pgfqpoint{2.166794in}{2.007051in}}%
\pgfpathlineto{\pgfqpoint{2.166794in}{2.007051in}}%
\pgfpathlineto{\pgfqpoint{2.166794in}{2.010001in}}%
\pgfpathlineto{\pgfqpoint{2.171335in}{2.010001in}}%
\pgfpathlineto{\pgfqpoint{2.171335in}{2.007051in}}%
\pgfpathmoveto{\pgfqpoint{2.171335in}{2.007051in}}%
\pgfpathlineto{\pgfqpoint{2.171335in}{2.007051in}}%
\pgfpathlineto{\pgfqpoint{2.171335in}{2.010001in}}%
\pgfpathlineto{\pgfqpoint{2.175876in}{2.010001in}}%
\pgfpathlineto{\pgfqpoint{2.175876in}{2.007051in}}%
\pgfpathmoveto{\pgfqpoint{2.175876in}{2.007051in}}%
\pgfpathlineto{\pgfqpoint{2.175876in}{2.007051in}}%
\pgfpathlineto{\pgfqpoint{2.175876in}{2.010001in}}%
\pgfpathlineto{\pgfqpoint{2.180417in}{2.010001in}}%
\pgfpathlineto{\pgfqpoint{2.180417in}{2.007051in}}%
\pgfpathmoveto{\pgfqpoint{2.180417in}{2.007051in}}%
\pgfpathlineto{\pgfqpoint{2.180417in}{2.007051in}}%
\pgfpathlineto{\pgfqpoint{2.180417in}{2.010001in}}%
\pgfpathlineto{\pgfqpoint{2.184957in}{2.010001in}}%
\pgfpathlineto{\pgfqpoint{2.184957in}{2.007051in}}%
\pgfpathmoveto{\pgfqpoint{2.184957in}{2.007051in}}%
\pgfpathlineto{\pgfqpoint{2.184957in}{2.007051in}}%
\pgfpathlineto{\pgfqpoint{2.184957in}{2.010001in}}%
\pgfpathlineto{\pgfqpoint{2.189498in}{2.010001in}}%
\pgfpathlineto{\pgfqpoint{2.189498in}{2.007051in}}%
\pgfpathmoveto{\pgfqpoint{2.189498in}{2.007051in}}%
\pgfpathlineto{\pgfqpoint{2.189498in}{2.007051in}}%
\pgfpathlineto{\pgfqpoint{2.189498in}{2.010001in}}%
\pgfpathlineto{\pgfqpoint{2.194039in}{2.010001in}}%
\pgfpathlineto{\pgfqpoint{2.194039in}{2.007051in}}%
\pgfpathmoveto{\pgfqpoint{2.194039in}{2.007051in}}%
\pgfpathlineto{\pgfqpoint{2.194039in}{2.007051in}}%
\pgfpathlineto{\pgfqpoint{2.194039in}{2.010001in}}%
\pgfpathlineto{\pgfqpoint{2.198580in}{2.010001in}}%
\pgfpathlineto{\pgfqpoint{2.198580in}{2.007051in}}%
\pgfpathmoveto{\pgfqpoint{2.198580in}{2.007051in}}%
\pgfpathlineto{\pgfqpoint{2.198580in}{2.007051in}}%
\pgfpathlineto{\pgfqpoint{2.198580in}{2.010001in}}%
\pgfpathlineto{\pgfqpoint{2.203121in}{2.010001in}}%
\pgfpathlineto{\pgfqpoint{2.203121in}{2.007051in}}%
\pgfpathmoveto{\pgfqpoint{2.203121in}{2.007051in}}%
\pgfpathlineto{\pgfqpoint{2.203121in}{2.007051in}}%
\pgfpathlineto{\pgfqpoint{2.203121in}{2.010001in}}%
\pgfpathlineto{\pgfqpoint{2.207662in}{2.010001in}}%
\pgfpathlineto{\pgfqpoint{2.207662in}{2.007051in}}%
\pgfpathmoveto{\pgfqpoint{2.207662in}{2.007051in}}%
\pgfpathlineto{\pgfqpoint{2.207662in}{2.007051in}}%
\pgfpathlineto{\pgfqpoint{2.207662in}{2.010001in}}%
\pgfpathlineto{\pgfqpoint{2.212203in}{2.010001in}}%
\pgfpathlineto{\pgfqpoint{2.212203in}{2.007051in}}%
\pgfpathmoveto{\pgfqpoint{2.212203in}{2.007051in}}%
\pgfpathlineto{\pgfqpoint{2.212203in}{2.007051in}}%
\pgfpathlineto{\pgfqpoint{2.212203in}{2.010001in}}%
\pgfpathlineto{\pgfqpoint{2.216744in}{2.010001in}}%
\pgfpathlineto{\pgfqpoint{2.216744in}{2.007051in}}%
\pgfpathmoveto{\pgfqpoint{2.216744in}{2.007051in}}%
\pgfpathlineto{\pgfqpoint{2.216744in}{2.007051in}}%
\pgfpathlineto{\pgfqpoint{2.216744in}{2.010001in}}%
\pgfpathlineto{\pgfqpoint{2.221285in}{2.010001in}}%
\pgfpathlineto{\pgfqpoint{2.221285in}{2.007051in}}%
\pgfpathmoveto{\pgfqpoint{2.221285in}{2.007051in}}%
\pgfpathlineto{\pgfqpoint{2.221285in}{2.007051in}}%
\pgfpathlineto{\pgfqpoint{2.221285in}{2.010001in}}%
\pgfpathlineto{\pgfqpoint{2.225826in}{2.010001in}}%
\pgfpathlineto{\pgfqpoint{2.225826in}{2.007051in}}%
\pgfpathmoveto{\pgfqpoint{2.225826in}{2.007051in}}%
\pgfpathlineto{\pgfqpoint{2.225826in}{2.007051in}}%
\pgfpathlineto{\pgfqpoint{2.225826in}{2.010001in}}%
\pgfpathlineto{\pgfqpoint{2.230367in}{2.010001in}}%
\pgfpathlineto{\pgfqpoint{2.230367in}{2.007051in}}%
\pgfpathmoveto{\pgfqpoint{2.230367in}{2.007051in}}%
\pgfpathlineto{\pgfqpoint{2.230367in}{2.007051in}}%
\pgfpathlineto{\pgfqpoint{2.230367in}{2.010001in}}%
\pgfpathlineto{\pgfqpoint{2.234908in}{2.010001in}}%
\pgfpathlineto{\pgfqpoint{2.234908in}{2.007051in}}%
\pgfpathmoveto{\pgfqpoint{2.234908in}{2.007051in}}%
\pgfpathlineto{\pgfqpoint{2.234908in}{2.007051in}}%
\pgfpathlineto{\pgfqpoint{2.234908in}{2.010001in}}%
\pgfpathlineto{\pgfqpoint{2.239449in}{2.010001in}}%
\pgfpathlineto{\pgfqpoint{2.239449in}{2.007051in}}%
\pgfpathmoveto{\pgfqpoint{2.239449in}{2.007051in}}%
\pgfpathlineto{\pgfqpoint{2.239449in}{2.007051in}}%
\pgfpathlineto{\pgfqpoint{2.239449in}{2.010001in}}%
\pgfpathlineto{\pgfqpoint{2.243990in}{2.010001in}}%
\pgfpathlineto{\pgfqpoint{2.243990in}{2.007051in}}%
\pgfpathmoveto{\pgfqpoint{2.243990in}{2.007051in}}%
\pgfpathlineto{\pgfqpoint{2.243990in}{2.007051in}}%
\pgfpathlineto{\pgfqpoint{2.243990in}{2.010001in}}%
\pgfpathlineto{\pgfqpoint{2.248531in}{2.010001in}}%
\pgfpathlineto{\pgfqpoint{2.248531in}{2.007051in}}%
\pgfpathmoveto{\pgfqpoint{2.248531in}{2.007051in}}%
\pgfpathlineto{\pgfqpoint{2.248531in}{2.007051in}}%
\pgfpathlineto{\pgfqpoint{2.248531in}{2.010001in}}%
\pgfpathlineto{\pgfqpoint{2.253072in}{2.010001in}}%
\pgfpathlineto{\pgfqpoint{2.253072in}{2.007051in}}%
\pgfpathmoveto{\pgfqpoint{2.253072in}{2.007051in}}%
\pgfpathlineto{\pgfqpoint{2.253072in}{2.007051in}}%
\pgfpathlineto{\pgfqpoint{2.253072in}{2.010001in}}%
\pgfpathlineto{\pgfqpoint{2.257613in}{2.010001in}}%
\pgfpathlineto{\pgfqpoint{2.257613in}{2.007051in}}%
\pgfpathmoveto{\pgfqpoint{2.257613in}{2.007051in}}%
\pgfpathlineto{\pgfqpoint{2.257613in}{2.007051in}}%
\pgfpathlineto{\pgfqpoint{2.257613in}{2.010001in}}%
\pgfpathlineto{\pgfqpoint{2.262154in}{2.010001in}}%
\pgfpathlineto{\pgfqpoint{2.262154in}{2.007051in}}%
\pgfpathmoveto{\pgfqpoint{2.262154in}{2.007051in}}%
\pgfpathlineto{\pgfqpoint{2.262154in}{2.007051in}}%
\pgfpathlineto{\pgfqpoint{2.262154in}{2.010001in}}%
\pgfpathlineto{\pgfqpoint{2.266695in}{2.010001in}}%
\pgfpathlineto{\pgfqpoint{2.266695in}{2.007051in}}%
\pgfpathmoveto{\pgfqpoint{2.266695in}{2.007051in}}%
\pgfpathlineto{\pgfqpoint{2.266695in}{2.007051in}}%
\pgfpathlineto{\pgfqpoint{2.266695in}{2.010001in}}%
\pgfpathlineto{\pgfqpoint{2.271236in}{2.010001in}}%
\pgfpathlineto{\pgfqpoint{2.271236in}{2.007051in}}%
\pgfpathmoveto{\pgfqpoint{2.271236in}{2.007051in}}%
\pgfpathlineto{\pgfqpoint{2.271236in}{2.007051in}}%
\pgfpathlineto{\pgfqpoint{2.271236in}{2.010001in}}%
\pgfpathlineto{\pgfqpoint{2.275777in}{2.010001in}}%
\pgfpathlineto{\pgfqpoint{2.275777in}{2.007051in}}%
\pgfpathmoveto{\pgfqpoint{2.275777in}{2.007051in}}%
\pgfpathlineto{\pgfqpoint{2.275777in}{2.007051in}}%
\pgfpathlineto{\pgfqpoint{2.275777in}{2.010001in}}%
\pgfpathlineto{\pgfqpoint{2.280318in}{2.010001in}}%
\pgfpathlineto{\pgfqpoint{2.280318in}{2.007051in}}%
\pgfpathmoveto{\pgfqpoint{2.280318in}{2.007051in}}%
\pgfpathlineto{\pgfqpoint{2.280318in}{2.007051in}}%
\pgfpathlineto{\pgfqpoint{2.280318in}{2.010001in}}%
\pgfpathlineto{\pgfqpoint{2.284859in}{2.010001in}}%
\pgfpathlineto{\pgfqpoint{2.284859in}{2.007051in}}%
\pgfpathmoveto{\pgfqpoint{2.284859in}{2.007051in}}%
\pgfpathlineto{\pgfqpoint{2.284859in}{2.007051in}}%
\pgfpathlineto{\pgfqpoint{2.284859in}{2.010001in}}%
\pgfpathlineto{\pgfqpoint{2.289400in}{2.010001in}}%
\pgfpathlineto{\pgfqpoint{2.289400in}{2.007051in}}%
\pgfpathmoveto{\pgfqpoint{2.289400in}{2.007051in}}%
\pgfpathlineto{\pgfqpoint{2.289400in}{2.007051in}}%
\pgfpathlineto{\pgfqpoint{2.289400in}{2.010001in}}%
\pgfpathlineto{\pgfqpoint{2.293941in}{2.010001in}}%
\pgfpathlineto{\pgfqpoint{2.293941in}{2.007051in}}%
\pgfpathmoveto{\pgfqpoint{2.293941in}{2.007051in}}%
\pgfpathlineto{\pgfqpoint{2.293941in}{2.007051in}}%
\pgfpathlineto{\pgfqpoint{2.293941in}{2.010001in}}%
\pgfpathlineto{\pgfqpoint{2.298482in}{2.010001in}}%
\pgfpathlineto{\pgfqpoint{2.298482in}{2.007051in}}%
\pgfpathmoveto{\pgfqpoint{2.298482in}{2.007051in}}%
\pgfpathlineto{\pgfqpoint{2.298482in}{2.007051in}}%
\pgfpathlineto{\pgfqpoint{2.298482in}{2.010001in}}%
\pgfpathlineto{\pgfqpoint{2.303023in}{2.010001in}}%
\pgfpathlineto{\pgfqpoint{2.303023in}{2.007051in}}%
\pgfpathmoveto{\pgfqpoint{2.303023in}{2.007051in}}%
\pgfpathlineto{\pgfqpoint{2.303023in}{2.007051in}}%
\pgfpathlineto{\pgfqpoint{2.303023in}{2.010001in}}%
\pgfpathlineto{\pgfqpoint{2.307564in}{2.010001in}}%
\pgfpathlineto{\pgfqpoint{2.307564in}{2.007051in}}%
\pgfpathmoveto{\pgfqpoint{2.307564in}{2.007051in}}%
\pgfpathlineto{\pgfqpoint{2.307564in}{2.007051in}}%
\pgfpathlineto{\pgfqpoint{2.307564in}{2.010001in}}%
\pgfpathlineto{\pgfqpoint{2.312105in}{2.010001in}}%
\pgfpathlineto{\pgfqpoint{2.312105in}{2.007051in}}%
\pgfpathmoveto{\pgfqpoint{2.312105in}{2.007051in}}%
\pgfpathlineto{\pgfqpoint{2.312105in}{2.007051in}}%
\pgfpathlineto{\pgfqpoint{2.312105in}{2.010001in}}%
\pgfpathlineto{\pgfqpoint{2.316646in}{2.010001in}}%
\pgfpathlineto{\pgfqpoint{2.316646in}{2.007051in}}%
\pgfpathmoveto{\pgfqpoint{2.316646in}{2.007051in}}%
\pgfpathlineto{\pgfqpoint{2.316646in}{2.007051in}}%
\pgfpathlineto{\pgfqpoint{2.316646in}{2.010001in}}%
\pgfpathlineto{\pgfqpoint{2.321187in}{2.010001in}}%
\pgfpathlineto{\pgfqpoint{2.321187in}{2.007051in}}%
\pgfpathmoveto{\pgfqpoint{2.321187in}{2.007051in}}%
\pgfpathlineto{\pgfqpoint{2.321187in}{2.007051in}}%
\pgfpathlineto{\pgfqpoint{2.321187in}{2.010001in}}%
\pgfpathlineto{\pgfqpoint{2.325728in}{2.010001in}}%
\pgfpathlineto{\pgfqpoint{2.325728in}{2.007051in}}%
\pgfpathmoveto{\pgfqpoint{2.325728in}{2.007051in}}%
\pgfpathlineto{\pgfqpoint{2.325728in}{2.007051in}}%
\pgfpathlineto{\pgfqpoint{2.325728in}{2.010001in}}%
\pgfpathlineto{\pgfqpoint{2.330269in}{2.010001in}}%
\pgfpathlineto{\pgfqpoint{2.330269in}{2.007051in}}%
\pgfpathmoveto{\pgfqpoint{2.330269in}{2.007051in}}%
\pgfpathlineto{\pgfqpoint{2.330269in}{2.007051in}}%
\pgfpathlineto{\pgfqpoint{2.330269in}{2.010001in}}%
\pgfpathlineto{\pgfqpoint{2.334810in}{2.010001in}}%
\pgfpathlineto{\pgfqpoint{2.334810in}{2.007051in}}%
\pgfpathmoveto{\pgfqpoint{2.334810in}{2.007051in}}%
\pgfpathlineto{\pgfqpoint{2.334810in}{2.007051in}}%
\pgfpathlineto{\pgfqpoint{2.334810in}{2.010001in}}%
\pgfpathlineto{\pgfqpoint{2.339351in}{2.010001in}}%
\pgfpathlineto{\pgfqpoint{2.339351in}{2.007051in}}%
\pgfpathmoveto{\pgfqpoint{2.339351in}{2.007051in}}%
\pgfpathlineto{\pgfqpoint{2.339351in}{2.007051in}}%
\pgfpathlineto{\pgfqpoint{2.339351in}{2.010001in}}%
\pgfpathlineto{\pgfqpoint{2.343892in}{2.010001in}}%
\pgfpathlineto{\pgfqpoint{2.343892in}{2.007051in}}%
\pgfpathmoveto{\pgfqpoint{2.343892in}{2.007051in}}%
\pgfpathlineto{\pgfqpoint{2.343892in}{2.007051in}}%
\pgfpathlineto{\pgfqpoint{2.343892in}{2.010001in}}%
\pgfpathlineto{\pgfqpoint{2.348433in}{2.010001in}}%
\pgfpathlineto{\pgfqpoint{2.348433in}{2.007051in}}%
\pgfpathmoveto{\pgfqpoint{2.348433in}{2.007051in}}%
\pgfpathlineto{\pgfqpoint{2.348433in}{2.007051in}}%
\pgfpathlineto{\pgfqpoint{2.348433in}{2.010001in}}%
\pgfpathlineto{\pgfqpoint{2.352974in}{2.010001in}}%
\pgfpathlineto{\pgfqpoint{2.352974in}{2.007051in}}%
\pgfpathmoveto{\pgfqpoint{2.352974in}{2.007051in}}%
\pgfpathlineto{\pgfqpoint{2.352974in}{2.007051in}}%
\pgfpathlineto{\pgfqpoint{2.352974in}{2.010001in}}%
\pgfpathlineto{\pgfqpoint{2.357515in}{2.010001in}}%
\pgfpathlineto{\pgfqpoint{2.357515in}{2.007051in}}%
\pgfpathmoveto{\pgfqpoint{2.357515in}{2.007051in}}%
\pgfpathlineto{\pgfqpoint{2.357515in}{2.007051in}}%
\pgfpathlineto{\pgfqpoint{2.357515in}{2.010001in}}%
\pgfpathlineto{\pgfqpoint{2.362056in}{2.010001in}}%
\pgfpathlineto{\pgfqpoint{2.362056in}{2.007051in}}%
\pgfpathmoveto{\pgfqpoint{2.362056in}{2.007051in}}%
\pgfpathlineto{\pgfqpoint{2.362056in}{2.007051in}}%
\pgfpathlineto{\pgfqpoint{2.362056in}{2.010001in}}%
\pgfpathlineto{\pgfqpoint{2.366597in}{2.010001in}}%
\pgfpathlineto{\pgfqpoint{2.366597in}{2.007051in}}%
\pgfpathmoveto{\pgfqpoint{2.366597in}{2.007051in}}%
\pgfpathlineto{\pgfqpoint{2.366597in}{2.007051in}}%
\pgfpathlineto{\pgfqpoint{2.366597in}{2.010001in}}%
\pgfpathlineto{\pgfqpoint{2.371138in}{2.010001in}}%
\pgfpathlineto{\pgfqpoint{2.371138in}{2.007051in}}%
\pgfpathmoveto{\pgfqpoint{2.371138in}{2.007051in}}%
\pgfpathlineto{\pgfqpoint{2.371138in}{2.007051in}}%
\pgfpathlineto{\pgfqpoint{2.371138in}{2.010001in}}%
\pgfpathlineto{\pgfqpoint{2.375679in}{2.010001in}}%
\pgfpathlineto{\pgfqpoint{2.375679in}{2.007051in}}%
\pgfpathmoveto{\pgfqpoint{2.375679in}{2.007051in}}%
\pgfpathlineto{\pgfqpoint{2.375679in}{2.007051in}}%
\pgfpathlineto{\pgfqpoint{2.375679in}{2.010001in}}%
\pgfpathlineto{\pgfqpoint{2.380220in}{2.010001in}}%
\pgfpathlineto{\pgfqpoint{2.380220in}{2.007051in}}%
\pgfpathmoveto{\pgfqpoint{2.380220in}{2.007051in}}%
\pgfpathlineto{\pgfqpoint{2.380220in}{2.007051in}}%
\pgfpathlineto{\pgfqpoint{2.380220in}{2.010001in}}%
\pgfpathlineto{\pgfqpoint{2.384761in}{2.010001in}}%
\pgfpathlineto{\pgfqpoint{2.384761in}{2.007051in}}%
\pgfpathmoveto{\pgfqpoint{2.384761in}{2.007051in}}%
\pgfpathlineto{\pgfqpoint{2.384761in}{2.007051in}}%
\pgfpathlineto{\pgfqpoint{2.384761in}{2.010001in}}%
\pgfpathlineto{\pgfqpoint{2.389302in}{2.010001in}}%
\pgfpathlineto{\pgfqpoint{2.389302in}{2.007051in}}%
\pgfpathmoveto{\pgfqpoint{2.389302in}{2.007051in}}%
\pgfpathlineto{\pgfqpoint{2.389302in}{2.007051in}}%
\pgfpathlineto{\pgfqpoint{2.389302in}{2.010001in}}%
\pgfpathlineto{\pgfqpoint{2.393843in}{2.010001in}}%
\pgfpathlineto{\pgfqpoint{2.393843in}{2.007051in}}%
\pgfpathmoveto{\pgfqpoint{2.393843in}{2.007051in}}%
\pgfpathlineto{\pgfqpoint{2.393843in}{2.007051in}}%
\pgfpathlineto{\pgfqpoint{2.393843in}{2.010001in}}%
\pgfpathlineto{\pgfqpoint{2.398384in}{2.010001in}}%
\pgfpathlineto{\pgfqpoint{2.398384in}{2.007051in}}%
\pgfpathmoveto{\pgfqpoint{2.398384in}{2.007051in}}%
\pgfpathlineto{\pgfqpoint{2.398384in}{2.007051in}}%
\pgfpathlineto{\pgfqpoint{2.398384in}{2.010001in}}%
\pgfpathlineto{\pgfqpoint{2.402926in}{2.010001in}}%
\pgfpathlineto{\pgfqpoint{2.402926in}{2.007051in}}%
\pgfpathmoveto{\pgfqpoint{2.402926in}{2.007051in}}%
\pgfpathlineto{\pgfqpoint{2.402926in}{2.007051in}}%
\pgfpathlineto{\pgfqpoint{2.402926in}{2.010001in}}%
\pgfpathlineto{\pgfqpoint{2.407467in}{2.010001in}}%
\pgfpathlineto{\pgfqpoint{2.407467in}{2.007051in}}%
\pgfpathmoveto{\pgfqpoint{2.407467in}{2.007051in}}%
\pgfpathlineto{\pgfqpoint{2.407467in}{2.007051in}}%
\pgfpathlineto{\pgfqpoint{2.407467in}{2.010001in}}%
\pgfpathlineto{\pgfqpoint{2.412008in}{2.010001in}}%
\pgfpathlineto{\pgfqpoint{2.412008in}{2.007051in}}%
\pgfpathmoveto{\pgfqpoint{2.412008in}{2.007051in}}%
\pgfpathlineto{\pgfqpoint{2.412008in}{2.007051in}}%
\pgfpathlineto{\pgfqpoint{2.412008in}{2.010001in}}%
\pgfpathlineto{\pgfqpoint{2.416549in}{2.010001in}}%
\pgfpathlineto{\pgfqpoint{2.416549in}{2.007051in}}%
\pgfpathmoveto{\pgfqpoint{2.416549in}{2.007051in}}%
\pgfpathlineto{\pgfqpoint{2.416549in}{2.007051in}}%
\pgfpathlineto{\pgfqpoint{2.416549in}{2.010001in}}%
\pgfpathlineto{\pgfqpoint{2.421090in}{2.010001in}}%
\pgfpathlineto{\pgfqpoint{2.421090in}{2.007051in}}%
\pgfpathmoveto{\pgfqpoint{2.421090in}{2.007051in}}%
\pgfpathlineto{\pgfqpoint{2.421090in}{2.007051in}}%
\pgfpathlineto{\pgfqpoint{2.421090in}{2.010001in}}%
\pgfpathlineto{\pgfqpoint{2.425631in}{2.010001in}}%
\pgfpathlineto{\pgfqpoint{2.425631in}{2.007051in}}%
\pgfpathmoveto{\pgfqpoint{2.425631in}{2.007051in}}%
\pgfpathlineto{\pgfqpoint{2.425631in}{2.007051in}}%
\pgfpathlineto{\pgfqpoint{2.425631in}{2.010001in}}%
\pgfpathlineto{\pgfqpoint{2.430172in}{2.010001in}}%
\pgfpathlineto{\pgfqpoint{2.430172in}{2.007051in}}%
\pgfpathmoveto{\pgfqpoint{2.430172in}{2.007051in}}%
\pgfpathlineto{\pgfqpoint{2.430172in}{2.007051in}}%
\pgfpathlineto{\pgfqpoint{2.430172in}{2.010001in}}%
\pgfpathlineto{\pgfqpoint{2.434713in}{2.010001in}}%
\pgfpathlineto{\pgfqpoint{2.434713in}{2.007051in}}%
\pgfpathmoveto{\pgfqpoint{2.434713in}{2.007051in}}%
\pgfpathlineto{\pgfqpoint{2.434713in}{2.007051in}}%
\pgfpathlineto{\pgfqpoint{2.434713in}{2.010001in}}%
\pgfpathlineto{\pgfqpoint{2.439254in}{2.010001in}}%
\pgfpathlineto{\pgfqpoint{2.439254in}{2.007051in}}%
\pgfpathmoveto{\pgfqpoint{2.439254in}{2.007051in}}%
\pgfpathlineto{\pgfqpoint{2.439254in}{2.007051in}}%
\pgfpathlineto{\pgfqpoint{2.439254in}{2.010001in}}%
\pgfpathlineto{\pgfqpoint{2.443795in}{2.010001in}}%
\pgfpathlineto{\pgfqpoint{2.443795in}{2.007051in}}%
\pgfpathmoveto{\pgfqpoint{2.443795in}{2.007051in}}%
\pgfpathlineto{\pgfqpoint{2.443795in}{2.007051in}}%
\pgfpathlineto{\pgfqpoint{2.443795in}{2.010001in}}%
\pgfpathlineto{\pgfqpoint{2.448336in}{2.010001in}}%
\pgfpathlineto{\pgfqpoint{2.448336in}{2.007051in}}%
\pgfpathmoveto{\pgfqpoint{2.448336in}{2.007051in}}%
\pgfpathlineto{\pgfqpoint{2.448336in}{2.007051in}}%
\pgfpathlineto{\pgfqpoint{2.448336in}{2.010001in}}%
\pgfpathlineto{\pgfqpoint{2.452877in}{2.010001in}}%
\pgfpathlineto{\pgfqpoint{2.452877in}{2.007051in}}%
\pgfpathmoveto{\pgfqpoint{2.452877in}{2.007051in}}%
\pgfpathlineto{\pgfqpoint{2.452877in}{2.007051in}}%
\pgfpathlineto{\pgfqpoint{2.452877in}{2.010001in}}%
\pgfpathlineto{\pgfqpoint{2.457418in}{2.010001in}}%
\pgfpathlineto{\pgfqpoint{2.457418in}{2.007051in}}%
\pgfpathmoveto{\pgfqpoint{2.457418in}{2.007051in}}%
\pgfpathlineto{\pgfqpoint{2.457418in}{2.007051in}}%
\pgfpathlineto{\pgfqpoint{2.457418in}{2.010001in}}%
\pgfpathlineto{\pgfqpoint{2.461959in}{2.010001in}}%
\pgfpathlineto{\pgfqpoint{2.461959in}{2.007051in}}%
\pgfpathmoveto{\pgfqpoint{2.461959in}{2.007051in}}%
\pgfpathlineto{\pgfqpoint{2.461959in}{2.007051in}}%
\pgfpathlineto{\pgfqpoint{2.461959in}{2.010001in}}%
\pgfpathlineto{\pgfqpoint{2.466500in}{2.010001in}}%
\pgfpathlineto{\pgfqpoint{2.466500in}{2.007051in}}%
\pgfpathmoveto{\pgfqpoint{2.466500in}{2.007051in}}%
\pgfpathlineto{\pgfqpoint{2.466500in}{2.007051in}}%
\pgfpathlineto{\pgfqpoint{2.466500in}{2.010001in}}%
\pgfpathlineto{\pgfqpoint{2.471041in}{2.010001in}}%
\pgfpathlineto{\pgfqpoint{2.471041in}{2.007051in}}%
\pgfpathmoveto{\pgfqpoint{2.471041in}{2.007051in}}%
\pgfpathlineto{\pgfqpoint{2.471041in}{2.007051in}}%
\pgfpathlineto{\pgfqpoint{2.471041in}{2.010001in}}%
\pgfpathlineto{\pgfqpoint{2.475582in}{2.010001in}}%
\pgfpathlineto{\pgfqpoint{2.475582in}{2.007051in}}%
\pgfpathmoveto{\pgfqpoint{2.475582in}{2.007051in}}%
\pgfpathlineto{\pgfqpoint{2.475582in}{2.007051in}}%
\pgfpathlineto{\pgfqpoint{2.475582in}{2.010001in}}%
\pgfpathlineto{\pgfqpoint{2.480123in}{2.010001in}}%
\pgfpathlineto{\pgfqpoint{2.480123in}{2.007051in}}%
\pgfpathmoveto{\pgfqpoint{2.480123in}{2.007051in}}%
\pgfpathlineto{\pgfqpoint{2.480123in}{2.007051in}}%
\pgfpathlineto{\pgfqpoint{2.480123in}{2.010001in}}%
\pgfpathlineto{\pgfqpoint{2.484664in}{2.010001in}}%
\pgfpathlineto{\pgfqpoint{2.484664in}{2.007051in}}%
\pgfpathmoveto{\pgfqpoint{2.484664in}{2.007051in}}%
\pgfpathlineto{\pgfqpoint{2.484664in}{2.007051in}}%
\pgfpathlineto{\pgfqpoint{2.484664in}{2.010001in}}%
\pgfpathlineto{\pgfqpoint{2.489205in}{2.010001in}}%
\pgfpathlineto{\pgfqpoint{2.489205in}{2.007051in}}%
\pgfpathmoveto{\pgfqpoint{2.489205in}{2.007051in}}%
\pgfpathlineto{\pgfqpoint{2.489205in}{2.007051in}}%
\pgfpathlineto{\pgfqpoint{2.489205in}{2.010001in}}%
\pgfpathlineto{\pgfqpoint{2.493746in}{2.010001in}}%
\pgfpathlineto{\pgfqpoint{2.493746in}{2.007051in}}%
\pgfpathmoveto{\pgfqpoint{2.493746in}{2.007051in}}%
\pgfpathlineto{\pgfqpoint{2.493746in}{2.007051in}}%
\pgfpathlineto{\pgfqpoint{2.493746in}{2.010001in}}%
\pgfpathlineto{\pgfqpoint{2.498287in}{2.010001in}}%
\pgfpathlineto{\pgfqpoint{2.498287in}{2.007051in}}%
\pgfpathmoveto{\pgfqpoint{2.498287in}{2.007051in}}%
\pgfpathlineto{\pgfqpoint{2.498287in}{2.007051in}}%
\pgfpathlineto{\pgfqpoint{2.498287in}{2.010001in}}%
\pgfpathlineto{\pgfqpoint{2.502828in}{2.010001in}}%
\pgfpathlineto{\pgfqpoint{2.502828in}{2.007051in}}%
\pgfpathmoveto{\pgfqpoint{2.502828in}{2.007051in}}%
\pgfpathlineto{\pgfqpoint{2.502828in}{2.007051in}}%
\pgfpathlineto{\pgfqpoint{2.502828in}{2.010001in}}%
\pgfpathlineto{\pgfqpoint{2.507370in}{2.010001in}}%
\pgfpathlineto{\pgfqpoint{2.507370in}{2.007051in}}%
\pgfpathmoveto{\pgfqpoint{2.507370in}{2.007051in}}%
\pgfpathlineto{\pgfqpoint{2.507370in}{2.007051in}}%
\pgfpathlineto{\pgfqpoint{2.507370in}{2.010001in}}%
\pgfpathlineto{\pgfqpoint{2.511911in}{2.010001in}}%
\pgfpathlineto{\pgfqpoint{2.511911in}{2.007051in}}%
\pgfpathmoveto{\pgfqpoint{2.511911in}{2.007051in}}%
\pgfpathlineto{\pgfqpoint{2.511911in}{2.007051in}}%
\pgfpathlineto{\pgfqpoint{2.511911in}{2.010001in}}%
\pgfpathlineto{\pgfqpoint{2.516452in}{2.010001in}}%
\pgfpathlineto{\pgfqpoint{2.516452in}{2.007051in}}%
\pgfpathmoveto{\pgfqpoint{2.516452in}{2.007051in}}%
\pgfpathlineto{\pgfqpoint{2.516452in}{2.007051in}}%
\pgfpathlineto{\pgfqpoint{2.516452in}{2.010001in}}%
\pgfpathlineto{\pgfqpoint{2.520993in}{2.010001in}}%
\pgfpathlineto{\pgfqpoint{2.520993in}{2.007051in}}%
\pgfpathmoveto{\pgfqpoint{2.520993in}{2.007051in}}%
\pgfpathlineto{\pgfqpoint{2.520993in}{2.007051in}}%
\pgfpathlineto{\pgfqpoint{2.520993in}{2.010001in}}%
\pgfpathlineto{\pgfqpoint{2.525534in}{2.010001in}}%
\pgfpathlineto{\pgfqpoint{2.525534in}{2.007051in}}%
\pgfpathmoveto{\pgfqpoint{2.525534in}{2.007051in}}%
\pgfpathlineto{\pgfqpoint{2.525534in}{2.007051in}}%
\pgfpathlineto{\pgfqpoint{2.525534in}{2.010001in}}%
\pgfpathlineto{\pgfqpoint{2.530076in}{2.010001in}}%
\pgfpathlineto{\pgfqpoint{2.530076in}{2.007051in}}%
\pgfpathmoveto{\pgfqpoint{2.530076in}{2.007051in}}%
\pgfpathlineto{\pgfqpoint{2.530076in}{2.007051in}}%
\pgfpathlineto{\pgfqpoint{2.530076in}{2.010001in}}%
\pgfpathlineto{\pgfqpoint{2.534617in}{2.010001in}}%
\pgfpathlineto{\pgfqpoint{2.534617in}{2.007051in}}%
\pgfpathmoveto{\pgfqpoint{2.534617in}{2.007051in}}%
\pgfpathlineto{\pgfqpoint{2.534617in}{2.007051in}}%
\pgfpathlineto{\pgfqpoint{2.534617in}{2.010001in}}%
\pgfpathlineto{\pgfqpoint{2.539158in}{2.010001in}}%
\pgfpathlineto{\pgfqpoint{2.539158in}{2.007051in}}%
\pgfpathmoveto{\pgfqpoint{2.539158in}{2.007051in}}%
\pgfpathlineto{\pgfqpoint{2.539158in}{2.007051in}}%
\pgfpathlineto{\pgfqpoint{2.539158in}{2.010001in}}%
\pgfpathlineto{\pgfqpoint{2.543699in}{2.010001in}}%
\pgfpathlineto{\pgfqpoint{2.543699in}{2.007051in}}%
\pgfpathmoveto{\pgfqpoint{2.543699in}{2.007051in}}%
\pgfpathlineto{\pgfqpoint{2.543699in}{2.007051in}}%
\pgfpathlineto{\pgfqpoint{2.543699in}{2.010001in}}%
\pgfpathlineto{\pgfqpoint{2.548240in}{2.010001in}}%
\pgfpathlineto{\pgfqpoint{2.548240in}{2.007051in}}%
\pgfpathmoveto{\pgfqpoint{2.548240in}{2.007051in}}%
\pgfpathlineto{\pgfqpoint{2.548240in}{2.007051in}}%
\pgfpathlineto{\pgfqpoint{2.548240in}{2.010001in}}%
\pgfpathlineto{\pgfqpoint{2.552781in}{2.010001in}}%
\pgfpathlineto{\pgfqpoint{2.552781in}{2.007051in}}%
\pgfpathmoveto{\pgfqpoint{2.552781in}{2.007051in}}%
\pgfpathlineto{\pgfqpoint{2.552781in}{2.007051in}}%
\pgfpathlineto{\pgfqpoint{2.552781in}{2.010001in}}%
\pgfpathlineto{\pgfqpoint{2.557323in}{2.010001in}}%
\pgfpathlineto{\pgfqpoint{2.557323in}{2.007051in}}%
\pgfpathmoveto{\pgfqpoint{2.557323in}{2.007051in}}%
\pgfpathlineto{\pgfqpoint{2.557323in}{2.007051in}}%
\pgfpathlineto{\pgfqpoint{2.557323in}{2.010001in}}%
\pgfpathlineto{\pgfqpoint{2.561864in}{2.010001in}}%
\pgfpathlineto{\pgfqpoint{2.561864in}{2.007051in}}%
\pgfpathmoveto{\pgfqpoint{2.561864in}{2.007051in}}%
\pgfpathlineto{\pgfqpoint{2.561864in}{2.007051in}}%
\pgfpathlineto{\pgfqpoint{2.561864in}{2.010001in}}%
\pgfpathlineto{\pgfqpoint{2.566405in}{2.010001in}}%
\pgfpathlineto{\pgfqpoint{2.566405in}{2.007051in}}%
\pgfpathmoveto{\pgfqpoint{2.566405in}{2.007051in}}%
\pgfpathlineto{\pgfqpoint{2.566405in}{2.007051in}}%
\pgfpathlineto{\pgfqpoint{2.566405in}{2.010001in}}%
\pgfpathlineto{\pgfqpoint{2.570946in}{2.010001in}}%
\pgfpathlineto{\pgfqpoint{2.570946in}{2.007051in}}%
\pgfpathmoveto{\pgfqpoint{2.570946in}{2.007051in}}%
\pgfpathlineto{\pgfqpoint{2.570946in}{2.007051in}}%
\pgfpathlineto{\pgfqpoint{2.570946in}{2.010001in}}%
\pgfpathlineto{\pgfqpoint{2.575487in}{2.010001in}}%
\pgfpathlineto{\pgfqpoint{2.575487in}{2.007051in}}%
\pgfpathmoveto{\pgfqpoint{2.575487in}{2.007051in}}%
\pgfpathlineto{\pgfqpoint{2.575487in}{2.007051in}}%
\pgfpathlineto{\pgfqpoint{2.575487in}{2.010001in}}%
\pgfpathlineto{\pgfqpoint{2.580029in}{2.010001in}}%
\pgfpathlineto{\pgfqpoint{2.580029in}{2.007051in}}%
\pgfpathmoveto{\pgfqpoint{2.580029in}{2.007051in}}%
\pgfpathlineto{\pgfqpoint{2.580029in}{2.007051in}}%
\pgfpathlineto{\pgfqpoint{2.580029in}{2.010001in}}%
\pgfpathlineto{\pgfqpoint{2.584570in}{2.010001in}}%
\pgfpathlineto{\pgfqpoint{2.584570in}{2.007051in}}%
\pgfpathmoveto{\pgfqpoint{2.584570in}{2.007051in}}%
\pgfpathlineto{\pgfqpoint{2.584570in}{2.007051in}}%
\pgfpathlineto{\pgfqpoint{2.584570in}{2.010001in}}%
\pgfpathlineto{\pgfqpoint{2.589111in}{2.010001in}}%
\pgfpathlineto{\pgfqpoint{2.589111in}{2.007051in}}%
\pgfpathmoveto{\pgfqpoint{2.589111in}{2.007051in}}%
\pgfpathlineto{\pgfqpoint{2.589111in}{2.007051in}}%
\pgfpathlineto{\pgfqpoint{2.589111in}{2.010001in}}%
\pgfpathlineto{\pgfqpoint{2.593652in}{2.010001in}}%
\pgfpathlineto{\pgfqpoint{2.593652in}{2.007051in}}%
\pgfpathmoveto{\pgfqpoint{2.593652in}{2.007051in}}%
\pgfpathlineto{\pgfqpoint{2.593652in}{2.007051in}}%
\pgfpathlineto{\pgfqpoint{2.593652in}{2.010001in}}%
\pgfpathlineto{\pgfqpoint{2.598193in}{2.010001in}}%
\pgfpathlineto{\pgfqpoint{2.598193in}{2.007051in}}%
\pgfpathmoveto{\pgfqpoint{2.598193in}{2.007051in}}%
\pgfpathlineto{\pgfqpoint{2.598193in}{2.007051in}}%
\pgfpathlineto{\pgfqpoint{2.598193in}{2.010001in}}%
\pgfpathlineto{\pgfqpoint{2.602734in}{2.010001in}}%
\pgfpathlineto{\pgfqpoint{2.602734in}{2.007051in}}%
\pgfpathmoveto{\pgfqpoint{2.602734in}{2.007051in}}%
\pgfpathlineto{\pgfqpoint{2.602734in}{2.007051in}}%
\pgfpathlineto{\pgfqpoint{2.602734in}{2.010001in}}%
\pgfpathlineto{\pgfqpoint{2.607276in}{2.010001in}}%
\pgfpathlineto{\pgfqpoint{2.607276in}{2.007051in}}%
\pgfpathmoveto{\pgfqpoint{2.607276in}{2.007051in}}%
\pgfpathlineto{\pgfqpoint{2.607276in}{2.007051in}}%
\pgfpathlineto{\pgfqpoint{2.607276in}{2.010001in}}%
\pgfpathlineto{\pgfqpoint{2.611817in}{2.010001in}}%
\pgfpathlineto{\pgfqpoint{2.611817in}{2.007051in}}%
\pgfpathmoveto{\pgfqpoint{2.611817in}{2.007051in}}%
\pgfpathlineto{\pgfqpoint{2.611817in}{2.007051in}}%
\pgfpathlineto{\pgfqpoint{2.611817in}{2.010001in}}%
\pgfpathlineto{\pgfqpoint{2.616358in}{2.010001in}}%
\pgfpathlineto{\pgfqpoint{2.616358in}{2.007051in}}%
\pgfpathmoveto{\pgfqpoint{2.616358in}{2.007051in}}%
\pgfpathlineto{\pgfqpoint{2.616358in}{2.007051in}}%
\pgfpathlineto{\pgfqpoint{2.616358in}{2.010001in}}%
\pgfpathlineto{\pgfqpoint{2.620899in}{2.010001in}}%
\pgfpathlineto{\pgfqpoint{2.620899in}{2.007051in}}%
\pgfpathmoveto{\pgfqpoint{2.620899in}{2.007051in}}%
\pgfpathlineto{\pgfqpoint{2.620899in}{2.007051in}}%
\pgfpathlineto{\pgfqpoint{2.620899in}{2.010001in}}%
\pgfpathlineto{\pgfqpoint{2.625440in}{2.010001in}}%
\pgfpathlineto{\pgfqpoint{2.625440in}{2.007051in}}%
\pgfpathmoveto{\pgfqpoint{2.625440in}{2.007051in}}%
\pgfpathlineto{\pgfqpoint{2.625440in}{2.007051in}}%
\pgfpathlineto{\pgfqpoint{2.625440in}{2.010001in}}%
\pgfpathlineto{\pgfqpoint{2.629982in}{2.010001in}}%
\pgfpathlineto{\pgfqpoint{2.629982in}{2.007051in}}%
\pgfpathmoveto{\pgfqpoint{2.629982in}{2.007051in}}%
\pgfpathlineto{\pgfqpoint{2.629982in}{2.007051in}}%
\pgfpathlineto{\pgfqpoint{2.629982in}{2.010001in}}%
\pgfpathlineto{\pgfqpoint{2.634523in}{2.010001in}}%
\pgfpathlineto{\pgfqpoint{2.634523in}{2.007051in}}%
\pgfpathmoveto{\pgfqpoint{2.634523in}{2.007051in}}%
\pgfpathlineto{\pgfqpoint{2.634523in}{2.007051in}}%
\pgfpathlineto{\pgfqpoint{2.634523in}{2.010001in}}%
\pgfpathlineto{\pgfqpoint{2.639064in}{2.010001in}}%
\pgfpathlineto{\pgfqpoint{2.639064in}{2.007051in}}%
\pgfpathmoveto{\pgfqpoint{2.639064in}{2.007051in}}%
\pgfpathlineto{\pgfqpoint{2.639064in}{2.007051in}}%
\pgfpathlineto{\pgfqpoint{2.639064in}{2.010001in}}%
\pgfpathlineto{\pgfqpoint{2.643605in}{2.010001in}}%
\pgfpathlineto{\pgfqpoint{2.643605in}{2.007051in}}%
\pgfpathmoveto{\pgfqpoint{2.643605in}{2.007051in}}%
\pgfpathlineto{\pgfqpoint{2.643605in}{2.007051in}}%
\pgfpathlineto{\pgfqpoint{2.643605in}{2.010001in}}%
\pgfpathlineto{\pgfqpoint{2.648146in}{2.010001in}}%
\pgfpathlineto{\pgfqpoint{2.648146in}{2.007051in}}%
\pgfpathmoveto{\pgfqpoint{2.648146in}{2.007051in}}%
\pgfpathlineto{\pgfqpoint{2.648146in}{2.007051in}}%
\pgfpathlineto{\pgfqpoint{2.648146in}{2.010001in}}%
\pgfpathlineto{\pgfqpoint{2.652686in}{2.010001in}}%
\pgfpathlineto{\pgfqpoint{2.652686in}{2.007051in}}%
\pgfpathmoveto{\pgfqpoint{2.652686in}{2.007051in}}%
\pgfpathlineto{\pgfqpoint{2.652686in}{2.007051in}}%
\pgfpathlineto{\pgfqpoint{2.652686in}{2.010001in}}%
\pgfpathlineto{\pgfqpoint{2.657227in}{2.010001in}}%
\pgfpathlineto{\pgfqpoint{2.657227in}{2.007051in}}%
\pgfpathmoveto{\pgfqpoint{2.657227in}{2.007051in}}%
\pgfpathlineto{\pgfqpoint{2.657227in}{2.007051in}}%
\pgfpathlineto{\pgfqpoint{2.657227in}{2.010001in}}%
\pgfpathlineto{\pgfqpoint{2.661768in}{2.010001in}}%
\pgfpathlineto{\pgfqpoint{2.661768in}{2.007051in}}%
\pgfpathmoveto{\pgfqpoint{2.661768in}{2.007051in}}%
\pgfpathlineto{\pgfqpoint{2.661768in}{2.007051in}}%
\pgfpathlineto{\pgfqpoint{2.661768in}{2.010001in}}%
\pgfpathlineto{\pgfqpoint{2.666309in}{2.010001in}}%
\pgfpathlineto{\pgfqpoint{2.666309in}{2.007051in}}%
\pgfpathmoveto{\pgfqpoint{2.666309in}{2.007051in}}%
\pgfpathlineto{\pgfqpoint{2.666309in}{2.007051in}}%
\pgfpathlineto{\pgfqpoint{2.666309in}{2.010001in}}%
\pgfpathlineto{\pgfqpoint{2.670850in}{2.010001in}}%
\pgfpathlineto{\pgfqpoint{2.670850in}{2.007051in}}%
\pgfpathmoveto{\pgfqpoint{2.670850in}{2.007051in}}%
\pgfpathlineto{\pgfqpoint{2.670850in}{2.007051in}}%
\pgfpathlineto{\pgfqpoint{2.670850in}{2.010001in}}%
\pgfpathlineto{\pgfqpoint{2.675391in}{2.010001in}}%
\pgfpathlineto{\pgfqpoint{2.675391in}{2.007051in}}%
\pgfpathmoveto{\pgfqpoint{2.675391in}{2.007051in}}%
\pgfpathlineto{\pgfqpoint{2.675391in}{2.007051in}}%
\pgfpathlineto{\pgfqpoint{2.675391in}{2.010001in}}%
\pgfpathlineto{\pgfqpoint{2.679932in}{2.010001in}}%
\pgfpathlineto{\pgfqpoint{2.679932in}{2.007051in}}%
\pgfpathmoveto{\pgfqpoint{2.679932in}{2.007051in}}%
\pgfpathlineto{\pgfqpoint{2.679932in}{2.007051in}}%
\pgfpathlineto{\pgfqpoint{2.679932in}{2.010001in}}%
\pgfpathlineto{\pgfqpoint{2.684472in}{2.010001in}}%
\pgfpathlineto{\pgfqpoint{2.684472in}{2.007051in}}%
\pgfpathmoveto{\pgfqpoint{2.684472in}{2.007051in}}%
\pgfpathlineto{\pgfqpoint{2.684472in}{2.007051in}}%
\pgfpathlineto{\pgfqpoint{2.684472in}{2.010001in}}%
\pgfpathlineto{\pgfqpoint{2.689013in}{2.010001in}}%
\pgfpathlineto{\pgfqpoint{2.689013in}{2.007051in}}%
\pgfpathmoveto{\pgfqpoint{2.689013in}{2.007051in}}%
\pgfpathlineto{\pgfqpoint{2.689013in}{2.007051in}}%
\pgfpathlineto{\pgfqpoint{2.689013in}{2.010001in}}%
\pgfpathlineto{\pgfqpoint{2.693554in}{2.010001in}}%
\pgfpathlineto{\pgfqpoint{2.693554in}{2.007051in}}%
\pgfpathmoveto{\pgfqpoint{2.693554in}{2.007051in}}%
\pgfpathlineto{\pgfqpoint{2.693554in}{2.007051in}}%
\pgfpathlineto{\pgfqpoint{2.693554in}{2.010001in}}%
\pgfpathlineto{\pgfqpoint{2.698095in}{2.010001in}}%
\pgfpathlineto{\pgfqpoint{2.698095in}{2.007051in}}%
\pgfpathmoveto{\pgfqpoint{2.698095in}{2.007051in}}%
\pgfpathlineto{\pgfqpoint{2.698095in}{2.007051in}}%
\pgfpathlineto{\pgfqpoint{2.698095in}{2.010001in}}%
\pgfpathlineto{\pgfqpoint{2.702636in}{2.010001in}}%
\pgfpathlineto{\pgfqpoint{2.702636in}{2.007051in}}%
\pgfpathmoveto{\pgfqpoint{2.702636in}{2.007051in}}%
\pgfpathlineto{\pgfqpoint{2.702636in}{2.007051in}}%
\pgfpathlineto{\pgfqpoint{2.702636in}{2.010001in}}%
\pgfpathlineto{\pgfqpoint{2.707177in}{2.010001in}}%
\pgfpathlineto{\pgfqpoint{2.707177in}{2.007051in}}%
\pgfpathmoveto{\pgfqpoint{2.707177in}{2.007051in}}%
\pgfpathlineto{\pgfqpoint{2.707177in}{2.007051in}}%
\pgfpathlineto{\pgfqpoint{2.707177in}{2.010001in}}%
\pgfpathlineto{\pgfqpoint{2.711717in}{2.010001in}}%
\pgfpathlineto{\pgfqpoint{2.711717in}{2.007051in}}%
\pgfpathmoveto{\pgfqpoint{2.711717in}{2.007051in}}%
\pgfpathlineto{\pgfqpoint{2.711717in}{2.007051in}}%
\pgfpathlineto{\pgfqpoint{2.711717in}{2.010001in}}%
\pgfpathlineto{\pgfqpoint{2.716258in}{2.010001in}}%
\pgfpathlineto{\pgfqpoint{2.716258in}{2.007051in}}%
\pgfpathmoveto{\pgfqpoint{2.716258in}{2.007051in}}%
\pgfpathlineto{\pgfqpoint{2.716258in}{2.007051in}}%
\pgfpathlineto{\pgfqpoint{2.716258in}{2.010001in}}%
\pgfpathlineto{\pgfqpoint{2.720799in}{2.010001in}}%
\pgfpathlineto{\pgfqpoint{2.720799in}{2.007051in}}%
\pgfpathmoveto{\pgfqpoint{2.720799in}{2.007051in}}%
\pgfpathlineto{\pgfqpoint{2.720799in}{2.007051in}}%
\pgfpathlineto{\pgfqpoint{2.720799in}{2.010001in}}%
\pgfpathlineto{\pgfqpoint{2.725340in}{2.010001in}}%
\pgfpathlineto{\pgfqpoint{2.725340in}{2.007051in}}%
\pgfpathmoveto{\pgfqpoint{2.725340in}{2.007051in}}%
\pgfpathlineto{\pgfqpoint{2.725340in}{2.007051in}}%
\pgfpathlineto{\pgfqpoint{2.725340in}{2.010001in}}%
\pgfpathlineto{\pgfqpoint{2.729881in}{2.010001in}}%
\pgfpathlineto{\pgfqpoint{2.729881in}{2.007051in}}%
\pgfpathmoveto{\pgfqpoint{2.729881in}{2.007051in}}%
\pgfpathlineto{\pgfqpoint{2.729881in}{2.007051in}}%
\pgfpathlineto{\pgfqpoint{2.729881in}{2.010001in}}%
\pgfpathlineto{\pgfqpoint{2.734422in}{2.010001in}}%
\pgfpathlineto{\pgfqpoint{2.734422in}{2.007051in}}%
\pgfpathmoveto{\pgfqpoint{2.734422in}{2.007051in}}%
\pgfpathlineto{\pgfqpoint{2.734422in}{2.007051in}}%
\pgfpathlineto{\pgfqpoint{2.734422in}{2.010001in}}%
\pgfpathlineto{\pgfqpoint{2.738962in}{2.010001in}}%
\pgfpathlineto{\pgfqpoint{2.738962in}{2.007051in}}%
\pgfpathmoveto{\pgfqpoint{2.738962in}{2.007051in}}%
\pgfpathlineto{\pgfqpoint{2.738962in}{2.007051in}}%
\pgfpathlineto{\pgfqpoint{2.738962in}{2.010001in}}%
\pgfpathlineto{\pgfqpoint{2.743503in}{2.010001in}}%
\pgfpathlineto{\pgfqpoint{2.743503in}{2.007051in}}%
\pgfpathmoveto{\pgfqpoint{2.743503in}{2.007051in}}%
\pgfpathlineto{\pgfqpoint{2.743503in}{2.007051in}}%
\pgfpathlineto{\pgfqpoint{2.743503in}{2.010001in}}%
\pgfpathlineto{\pgfqpoint{2.748044in}{2.010001in}}%
\pgfpathlineto{\pgfqpoint{2.748044in}{2.007051in}}%
\pgfpathmoveto{\pgfqpoint{2.748044in}{2.007051in}}%
\pgfpathlineto{\pgfqpoint{2.748044in}{2.007051in}}%
\pgfpathlineto{\pgfqpoint{2.748044in}{2.010001in}}%
\pgfpathlineto{\pgfqpoint{2.752585in}{2.010001in}}%
\pgfpathlineto{\pgfqpoint{2.752585in}{2.007051in}}%
\pgfpathmoveto{\pgfqpoint{2.752585in}{2.007051in}}%
\pgfpathlineto{\pgfqpoint{2.752585in}{2.007051in}}%
\pgfpathlineto{\pgfqpoint{2.752585in}{2.010001in}}%
\pgfpathlineto{\pgfqpoint{2.757126in}{2.010001in}}%
\pgfpathlineto{\pgfqpoint{2.757126in}{2.007051in}}%
\pgfpathmoveto{\pgfqpoint{2.757126in}{2.007051in}}%
\pgfpathlineto{\pgfqpoint{2.757126in}{2.007051in}}%
\pgfpathlineto{\pgfqpoint{2.757126in}{2.010001in}}%
\pgfpathlineto{\pgfqpoint{2.761667in}{2.010001in}}%
\pgfpathlineto{\pgfqpoint{2.761667in}{2.007051in}}%
\pgfpathmoveto{\pgfqpoint{2.761667in}{2.007051in}}%
\pgfpathlineto{\pgfqpoint{2.761667in}{2.007051in}}%
\pgfpathlineto{\pgfqpoint{2.761667in}{2.010001in}}%
\pgfpathlineto{\pgfqpoint{2.766208in}{2.010001in}}%
\pgfpathlineto{\pgfqpoint{2.766208in}{2.007051in}}%
\pgfpathmoveto{\pgfqpoint{2.766208in}{2.007051in}}%
\pgfpathlineto{\pgfqpoint{2.766208in}{2.007051in}}%
\pgfpathlineto{\pgfqpoint{2.766208in}{2.010001in}}%
\pgfpathlineto{\pgfqpoint{2.770748in}{2.010001in}}%
\pgfpathlineto{\pgfqpoint{2.770748in}{2.007051in}}%
\pgfpathmoveto{\pgfqpoint{2.770748in}{2.007051in}}%
\pgfpathlineto{\pgfqpoint{2.770748in}{2.007051in}}%
\pgfpathlineto{\pgfqpoint{2.770748in}{2.010001in}}%
\pgfpathlineto{\pgfqpoint{2.775289in}{2.010001in}}%
\pgfpathlineto{\pgfqpoint{2.775289in}{2.007051in}}%
\pgfpathmoveto{\pgfqpoint{2.775289in}{2.007051in}}%
\pgfpathlineto{\pgfqpoint{2.775289in}{2.007051in}}%
\pgfpathlineto{\pgfqpoint{2.775289in}{2.010001in}}%
\pgfpathlineto{\pgfqpoint{2.779830in}{2.010001in}}%
\pgfpathlineto{\pgfqpoint{2.779830in}{2.007051in}}%
\pgfpathmoveto{\pgfqpoint{2.779830in}{2.007051in}}%
\pgfpathlineto{\pgfqpoint{2.779830in}{2.007051in}}%
\pgfpathlineto{\pgfqpoint{2.779830in}{2.010001in}}%
\pgfpathlineto{\pgfqpoint{2.784371in}{2.010001in}}%
\pgfpathlineto{\pgfqpoint{2.784371in}{2.007051in}}%
\pgfpathmoveto{\pgfqpoint{2.784371in}{2.007051in}}%
\pgfpathlineto{\pgfqpoint{2.784371in}{2.007051in}}%
\pgfpathlineto{\pgfqpoint{2.784371in}{2.010001in}}%
\pgfpathlineto{\pgfqpoint{2.788912in}{2.010001in}}%
\pgfpathlineto{\pgfqpoint{2.788912in}{2.007051in}}%
\pgfpathmoveto{\pgfqpoint{2.788912in}{2.007051in}}%
\pgfpathlineto{\pgfqpoint{2.788912in}{2.007051in}}%
\pgfpathlineto{\pgfqpoint{2.788912in}{2.010001in}}%
\pgfpathlineto{\pgfqpoint{2.793453in}{2.010001in}}%
\pgfpathlineto{\pgfqpoint{2.793453in}{2.007051in}}%
\pgfpathmoveto{\pgfqpoint{2.793453in}{2.007051in}}%
\pgfpathlineto{\pgfqpoint{2.793453in}{2.007051in}}%
\pgfpathlineto{\pgfqpoint{2.793453in}{2.010001in}}%
\pgfpathlineto{\pgfqpoint{2.797995in}{2.010001in}}%
\pgfpathlineto{\pgfqpoint{2.797995in}{2.007051in}}%
\pgfpathmoveto{\pgfqpoint{2.797995in}{2.007051in}}%
\pgfpathlineto{\pgfqpoint{2.797995in}{2.007051in}}%
\pgfpathlineto{\pgfqpoint{2.797995in}{2.010001in}}%
\pgfpathlineto{\pgfqpoint{2.802536in}{2.010001in}}%
\pgfpathlineto{\pgfqpoint{2.802536in}{2.007051in}}%
\pgfpathmoveto{\pgfqpoint{2.802536in}{2.007051in}}%
\pgfpathlineto{\pgfqpoint{2.802536in}{2.007051in}}%
\pgfpathlineto{\pgfqpoint{2.802536in}{2.010001in}}%
\pgfpathlineto{\pgfqpoint{2.807077in}{2.010001in}}%
\pgfpathlineto{\pgfqpoint{2.807077in}{2.007051in}}%
\pgfpathmoveto{\pgfqpoint{2.807077in}{2.007051in}}%
\pgfpathlineto{\pgfqpoint{2.807077in}{2.007051in}}%
\pgfpathlineto{\pgfqpoint{2.807077in}{2.010001in}}%
\pgfpathlineto{\pgfqpoint{2.811618in}{2.010001in}}%
\pgfpathlineto{\pgfqpoint{2.811618in}{2.007051in}}%
\pgfpathmoveto{\pgfqpoint{2.811618in}{2.007051in}}%
\pgfpathlineto{\pgfqpoint{2.811618in}{2.007051in}}%
\pgfpathlineto{\pgfqpoint{2.811618in}{2.010001in}}%
\pgfpathlineto{\pgfqpoint{2.816160in}{2.010001in}}%
\pgfpathlineto{\pgfqpoint{2.816160in}{2.007051in}}%
\pgfpathmoveto{\pgfqpoint{2.816160in}{2.007051in}}%
\pgfpathlineto{\pgfqpoint{2.816160in}{2.007051in}}%
\pgfpathlineto{\pgfqpoint{2.816160in}{2.010001in}}%
\pgfpathlineto{\pgfqpoint{2.820701in}{2.010001in}}%
\pgfpathlineto{\pgfqpoint{2.820701in}{2.007051in}}%
\pgfpathmoveto{\pgfqpoint{2.820701in}{2.007051in}}%
\pgfpathlineto{\pgfqpoint{2.820701in}{2.007051in}}%
\pgfpathlineto{\pgfqpoint{2.820701in}{2.010001in}}%
\pgfpathlineto{\pgfqpoint{2.825242in}{2.010001in}}%
\pgfpathlineto{\pgfqpoint{2.825242in}{2.007051in}}%
\pgfpathmoveto{\pgfqpoint{2.825242in}{2.007051in}}%
\pgfpathlineto{\pgfqpoint{2.825242in}{2.007051in}}%
\pgfpathlineto{\pgfqpoint{2.825242in}{2.010001in}}%
\pgfpathlineto{\pgfqpoint{2.829784in}{2.010001in}}%
\pgfpathlineto{\pgfqpoint{2.829784in}{2.007051in}}%
\pgfpathmoveto{\pgfqpoint{2.829784in}{2.007051in}}%
\pgfpathlineto{\pgfqpoint{2.829784in}{2.007051in}}%
\pgfpathlineto{\pgfqpoint{2.829784in}{2.010001in}}%
\pgfpathlineto{\pgfqpoint{2.834325in}{2.010001in}}%
\pgfpathlineto{\pgfqpoint{2.834325in}{2.007051in}}%
\pgfpathmoveto{\pgfqpoint{2.834325in}{2.007051in}}%
\pgfpathlineto{\pgfqpoint{2.834325in}{2.007051in}}%
\pgfpathlineto{\pgfqpoint{2.834325in}{2.010001in}}%
\pgfpathlineto{\pgfqpoint{2.838866in}{2.010001in}}%
\pgfpathlineto{\pgfqpoint{2.838866in}{2.007051in}}%
\pgfpathmoveto{\pgfqpoint{2.838866in}{2.007051in}}%
\pgfpathlineto{\pgfqpoint{2.838866in}{2.007051in}}%
\pgfpathlineto{\pgfqpoint{2.838866in}{2.010001in}}%
\pgfpathlineto{\pgfqpoint{2.843407in}{2.010001in}}%
\pgfpathlineto{\pgfqpoint{2.843407in}{2.007051in}}%
\pgfpathmoveto{\pgfqpoint{2.843407in}{2.007051in}}%
\pgfpathlineto{\pgfqpoint{2.843407in}{2.007051in}}%
\pgfpathlineto{\pgfqpoint{2.843407in}{2.010001in}}%
\pgfpathlineto{\pgfqpoint{2.847949in}{2.010001in}}%
\pgfpathlineto{\pgfqpoint{2.847949in}{2.007051in}}%
\pgfpathmoveto{\pgfqpoint{2.847949in}{2.007051in}}%
\pgfpathlineto{\pgfqpoint{2.847949in}{2.007051in}}%
\pgfpathlineto{\pgfqpoint{2.847949in}{2.010001in}}%
\pgfpathlineto{\pgfqpoint{2.852490in}{2.010001in}}%
\pgfpathlineto{\pgfqpoint{2.852490in}{2.007051in}}%
\pgfpathmoveto{\pgfqpoint{2.852490in}{2.007051in}}%
\pgfpathlineto{\pgfqpoint{2.852490in}{2.007051in}}%
\pgfpathlineto{\pgfqpoint{2.852490in}{2.010001in}}%
\pgfpathlineto{\pgfqpoint{2.857031in}{2.010001in}}%
\pgfpathlineto{\pgfqpoint{2.857031in}{2.007051in}}%
\pgfpathmoveto{\pgfqpoint{2.857031in}{2.007051in}}%
\pgfpathlineto{\pgfqpoint{2.857031in}{2.007051in}}%
\pgfpathlineto{\pgfqpoint{2.857031in}{2.010001in}}%
\pgfpathlineto{\pgfqpoint{2.861572in}{2.010001in}}%
\pgfpathlineto{\pgfqpoint{2.861572in}{2.007051in}}%
\pgfpathmoveto{\pgfqpoint{2.861572in}{2.007051in}}%
\pgfpathlineto{\pgfqpoint{2.861572in}{2.007051in}}%
\pgfpathlineto{\pgfqpoint{2.861572in}{2.010001in}}%
\pgfpathlineto{\pgfqpoint{2.866114in}{2.010001in}}%
\pgfpathlineto{\pgfqpoint{2.866114in}{2.007051in}}%
\pgfpathmoveto{\pgfqpoint{2.866114in}{2.007051in}}%
\pgfpathlineto{\pgfqpoint{2.866114in}{2.007051in}}%
\pgfpathlineto{\pgfqpoint{2.866114in}{2.010001in}}%
\pgfpathlineto{\pgfqpoint{2.870655in}{2.010001in}}%
\pgfpathlineto{\pgfqpoint{2.870655in}{2.007051in}}%
\pgfpathmoveto{\pgfqpoint{2.870655in}{2.007051in}}%
\pgfpathlineto{\pgfqpoint{2.870655in}{2.007051in}}%
\pgfpathlineto{\pgfqpoint{2.870655in}{2.010001in}}%
\pgfpathlineto{\pgfqpoint{2.875196in}{2.010001in}}%
\pgfpathlineto{\pgfqpoint{2.875196in}{2.007051in}}%
\pgfpathmoveto{\pgfqpoint{2.875196in}{2.007051in}}%
\pgfpathlineto{\pgfqpoint{2.875196in}{2.007051in}}%
\pgfpathlineto{\pgfqpoint{2.875196in}{2.010001in}}%
\pgfpathlineto{\pgfqpoint{2.879737in}{2.010001in}}%
\pgfpathlineto{\pgfqpoint{2.879737in}{2.007051in}}%
\pgfpathmoveto{\pgfqpoint{2.879737in}{2.007051in}}%
\pgfpathlineto{\pgfqpoint{2.879737in}{2.007051in}}%
\pgfpathlineto{\pgfqpoint{2.879737in}{2.010001in}}%
\pgfpathlineto{\pgfqpoint{2.884279in}{2.010001in}}%
\pgfpathlineto{\pgfqpoint{2.884279in}{2.007051in}}%
\pgfpathmoveto{\pgfqpoint{2.884279in}{2.007051in}}%
\pgfpathlineto{\pgfqpoint{2.884279in}{2.007051in}}%
\pgfpathlineto{\pgfqpoint{2.884279in}{2.010001in}}%
\pgfpathlineto{\pgfqpoint{2.888820in}{2.010001in}}%
\pgfpathlineto{\pgfqpoint{2.888820in}{2.007051in}}%
\pgfpathmoveto{\pgfqpoint{2.888820in}{2.007051in}}%
\pgfpathlineto{\pgfqpoint{2.888820in}{2.007051in}}%
\pgfpathlineto{\pgfqpoint{2.888820in}{2.010001in}}%
\pgfpathlineto{\pgfqpoint{2.893361in}{2.010001in}}%
\pgfpathlineto{\pgfqpoint{2.893361in}{2.007051in}}%
\pgfpathmoveto{\pgfqpoint{2.893361in}{2.007051in}}%
\pgfpathlineto{\pgfqpoint{2.893361in}{2.007051in}}%
\pgfpathlineto{\pgfqpoint{2.893361in}{2.010001in}}%
\pgfpathlineto{\pgfqpoint{2.897903in}{2.010001in}}%
\pgfpathlineto{\pgfqpoint{2.897903in}{2.007051in}}%
\pgfpathmoveto{\pgfqpoint{2.897903in}{2.007051in}}%
\pgfpathlineto{\pgfqpoint{2.897903in}{2.007051in}}%
\pgfpathlineto{\pgfqpoint{2.897903in}{2.010001in}}%
\pgfpathlineto{\pgfqpoint{2.902444in}{2.010001in}}%
\pgfpathlineto{\pgfqpoint{2.902444in}{2.007051in}}%
\pgfpathmoveto{\pgfqpoint{2.902444in}{2.007051in}}%
\pgfpathlineto{\pgfqpoint{2.902444in}{2.007051in}}%
\pgfpathlineto{\pgfqpoint{2.902444in}{2.010001in}}%
\pgfpathlineto{\pgfqpoint{2.906985in}{2.010001in}}%
\pgfpathlineto{\pgfqpoint{2.906985in}{2.007051in}}%
\pgfpathmoveto{\pgfqpoint{2.906985in}{2.007051in}}%
\pgfpathlineto{\pgfqpoint{2.906985in}{2.007051in}}%
\pgfpathlineto{\pgfqpoint{2.906985in}{2.010001in}}%
\pgfpathlineto{\pgfqpoint{2.911526in}{2.010001in}}%
\pgfpathlineto{\pgfqpoint{2.911526in}{2.007051in}}%
\pgfpathmoveto{\pgfqpoint{2.911526in}{2.007051in}}%
\pgfpathlineto{\pgfqpoint{2.911526in}{2.007051in}}%
\pgfpathlineto{\pgfqpoint{2.911526in}{2.010001in}}%
\pgfpathlineto{\pgfqpoint{2.916068in}{2.010001in}}%
\pgfpathlineto{\pgfqpoint{2.916068in}{2.007051in}}%
\pgfpathmoveto{\pgfqpoint{2.916068in}{2.007051in}}%
\pgfpathlineto{\pgfqpoint{2.916068in}{2.007051in}}%
\pgfpathlineto{\pgfqpoint{2.916068in}{2.010001in}}%
\pgfpathlineto{\pgfqpoint{2.920609in}{2.010001in}}%
\pgfpathlineto{\pgfqpoint{2.920609in}{2.007051in}}%
\pgfpathmoveto{\pgfqpoint{2.920609in}{2.007051in}}%
\pgfpathlineto{\pgfqpoint{2.920609in}{2.007051in}}%
\pgfpathlineto{\pgfqpoint{2.920609in}{2.010001in}}%
\pgfpathlineto{\pgfqpoint{2.925150in}{2.010001in}}%
\pgfpathlineto{\pgfqpoint{2.925150in}{2.007051in}}%
\pgfpathmoveto{\pgfqpoint{2.925150in}{2.007051in}}%
\pgfpathlineto{\pgfqpoint{2.925150in}{2.007051in}}%
\pgfpathlineto{\pgfqpoint{2.925150in}{2.010001in}}%
\pgfpathlineto{\pgfqpoint{2.929691in}{2.010001in}}%
\pgfpathlineto{\pgfqpoint{2.929691in}{2.007051in}}%
\pgfpathmoveto{\pgfqpoint{2.929691in}{2.007051in}}%
\pgfpathlineto{\pgfqpoint{2.929691in}{2.007051in}}%
\pgfpathlineto{\pgfqpoint{2.929691in}{2.010001in}}%
\pgfpathlineto{\pgfqpoint{2.934232in}{2.010001in}}%
\pgfpathlineto{\pgfqpoint{2.934232in}{2.007051in}}%
\pgfpathmoveto{\pgfqpoint{2.934232in}{2.007051in}}%
\pgfpathlineto{\pgfqpoint{2.934232in}{2.007051in}}%
\pgfpathlineto{\pgfqpoint{2.934232in}{2.010001in}}%
\pgfpathlineto{\pgfqpoint{2.938773in}{2.010001in}}%
\pgfpathlineto{\pgfqpoint{2.938773in}{2.007051in}}%
\pgfpathmoveto{\pgfqpoint{2.938773in}{2.007051in}}%
\pgfpathlineto{\pgfqpoint{2.938773in}{2.007051in}}%
\pgfpathlineto{\pgfqpoint{2.938773in}{2.010001in}}%
\pgfpathlineto{\pgfqpoint{2.943314in}{2.010001in}}%
\pgfpathlineto{\pgfqpoint{2.943314in}{2.007051in}}%
\pgfpathmoveto{\pgfqpoint{2.943314in}{2.007051in}}%
\pgfpathlineto{\pgfqpoint{2.943314in}{2.007051in}}%
\pgfpathlineto{\pgfqpoint{2.943314in}{2.010001in}}%
\pgfpathlineto{\pgfqpoint{2.947855in}{2.010001in}}%
\pgfpathlineto{\pgfqpoint{2.947855in}{2.007051in}}%
\pgfpathmoveto{\pgfqpoint{2.947855in}{2.007051in}}%
\pgfpathlineto{\pgfqpoint{2.947855in}{2.007051in}}%
\pgfpathlineto{\pgfqpoint{2.947855in}{2.010001in}}%
\pgfpathlineto{\pgfqpoint{2.952396in}{2.010001in}}%
\pgfpathlineto{\pgfqpoint{2.952396in}{2.007051in}}%
\pgfpathmoveto{\pgfqpoint{2.952396in}{2.007051in}}%
\pgfpathlineto{\pgfqpoint{2.952396in}{2.007051in}}%
\pgfpathlineto{\pgfqpoint{2.952396in}{2.010001in}}%
\pgfpathlineto{\pgfqpoint{2.956937in}{2.010001in}}%
\pgfpathlineto{\pgfqpoint{2.956937in}{2.007051in}}%
\pgfpathmoveto{\pgfqpoint{2.956937in}{2.007051in}}%
\pgfpathlineto{\pgfqpoint{2.956937in}{2.007051in}}%
\pgfpathlineto{\pgfqpoint{2.956937in}{2.010001in}}%
\pgfpathlineto{\pgfqpoint{2.961478in}{2.010001in}}%
\pgfpathlineto{\pgfqpoint{2.961478in}{2.007051in}}%
\pgfpathmoveto{\pgfqpoint{2.961478in}{2.007051in}}%
\pgfpathlineto{\pgfqpoint{2.961478in}{2.007051in}}%
\pgfpathlineto{\pgfqpoint{2.961478in}{2.010001in}}%
\pgfpathlineto{\pgfqpoint{2.966019in}{2.010001in}}%
\pgfpathlineto{\pgfqpoint{2.966019in}{2.007051in}}%
\pgfpathmoveto{\pgfqpoint{2.966019in}{2.007051in}}%
\pgfpathlineto{\pgfqpoint{2.966019in}{2.007051in}}%
\pgfpathlineto{\pgfqpoint{2.966019in}{2.010001in}}%
\pgfpathlineto{\pgfqpoint{2.970559in}{2.010001in}}%
\pgfpathlineto{\pgfqpoint{2.970559in}{2.007051in}}%
\pgfpathmoveto{\pgfqpoint{2.970559in}{2.007051in}}%
\pgfpathlineto{\pgfqpoint{2.970559in}{2.007051in}}%
\pgfpathlineto{\pgfqpoint{2.970559in}{2.010001in}}%
\pgfpathlineto{\pgfqpoint{2.975100in}{2.010001in}}%
\pgfpathlineto{\pgfqpoint{2.975100in}{2.007051in}}%
\pgfpathmoveto{\pgfqpoint{2.975100in}{2.007051in}}%
\pgfpathlineto{\pgfqpoint{2.975100in}{2.007051in}}%
\pgfpathlineto{\pgfqpoint{2.975100in}{2.010001in}}%
\pgfpathlineto{\pgfqpoint{2.979641in}{2.010001in}}%
\pgfpathlineto{\pgfqpoint{2.979641in}{2.007051in}}%
\pgfpathmoveto{\pgfqpoint{2.979641in}{2.007051in}}%
\pgfpathlineto{\pgfqpoint{2.979641in}{2.007051in}}%
\pgfpathlineto{\pgfqpoint{2.979641in}{2.010001in}}%
\pgfpathlineto{\pgfqpoint{2.984182in}{2.010001in}}%
\pgfpathlineto{\pgfqpoint{2.984182in}{2.007051in}}%
\pgfpathmoveto{\pgfqpoint{2.984182in}{2.007051in}}%
\pgfpathlineto{\pgfqpoint{2.984182in}{2.007051in}}%
\pgfpathlineto{\pgfqpoint{2.984182in}{2.010001in}}%
\pgfpathlineto{\pgfqpoint{2.988723in}{2.010001in}}%
\pgfpathlineto{\pgfqpoint{2.988723in}{2.007051in}}%
\pgfpathmoveto{\pgfqpoint{2.988723in}{2.007051in}}%
\pgfpathlineto{\pgfqpoint{2.988723in}{2.007051in}}%
\pgfpathlineto{\pgfqpoint{2.988723in}{2.010001in}}%
\pgfpathlineto{\pgfqpoint{2.993264in}{2.010001in}}%
\pgfpathlineto{\pgfqpoint{2.993264in}{2.007051in}}%
\pgfpathmoveto{\pgfqpoint{2.993264in}{2.007051in}}%
\pgfpathlineto{\pgfqpoint{2.993264in}{2.007051in}}%
\pgfpathlineto{\pgfqpoint{2.993264in}{2.010001in}}%
\pgfpathlineto{\pgfqpoint{2.997805in}{2.010001in}}%
\pgfpathlineto{\pgfqpoint{2.997805in}{2.007051in}}%
\pgfpathmoveto{\pgfqpoint{2.997805in}{2.007051in}}%
\pgfpathlineto{\pgfqpoint{2.997805in}{2.007051in}}%
\pgfpathlineto{\pgfqpoint{2.997805in}{2.010001in}}%
\pgfpathlineto{\pgfqpoint{3.002346in}{2.010001in}}%
\pgfpathlineto{\pgfqpoint{3.002346in}{2.007051in}}%
\pgfpathmoveto{\pgfqpoint{3.002346in}{2.007051in}}%
\pgfpathlineto{\pgfqpoint{3.002346in}{2.007051in}}%
\pgfpathlineto{\pgfqpoint{3.002346in}{2.010001in}}%
\pgfpathlineto{\pgfqpoint{3.006887in}{2.010001in}}%
\pgfpathlineto{\pgfqpoint{3.006887in}{2.007051in}}%
\pgfpathmoveto{\pgfqpoint{3.006887in}{2.007051in}}%
\pgfpathlineto{\pgfqpoint{3.006887in}{2.007051in}}%
\pgfpathlineto{\pgfqpoint{3.006887in}{2.010001in}}%
\pgfpathlineto{\pgfqpoint{3.011427in}{2.010001in}}%
\pgfpathlineto{\pgfqpoint{3.011427in}{2.007051in}}%
\pgfpathmoveto{\pgfqpoint{3.011427in}{2.007051in}}%
\pgfpathlineto{\pgfqpoint{3.011427in}{2.007051in}}%
\pgfpathlineto{\pgfqpoint{3.011427in}{2.010001in}}%
\pgfpathlineto{\pgfqpoint{3.015968in}{2.010001in}}%
\pgfpathlineto{\pgfqpoint{3.015968in}{2.007051in}}%
\pgfpathmoveto{\pgfqpoint{3.015968in}{2.007051in}}%
\pgfpathlineto{\pgfqpoint{3.015968in}{2.007051in}}%
\pgfpathlineto{\pgfqpoint{3.015968in}{2.010001in}}%
\pgfpathlineto{\pgfqpoint{3.020509in}{2.010001in}}%
\pgfpathlineto{\pgfqpoint{3.020509in}{2.007051in}}%
\pgfpathmoveto{\pgfqpoint{3.020509in}{2.007051in}}%
\pgfpathlineto{\pgfqpoint{3.020509in}{2.007051in}}%
\pgfpathlineto{\pgfqpoint{3.020509in}{2.010001in}}%
\pgfpathlineto{\pgfqpoint{3.025050in}{2.010001in}}%
\pgfpathlineto{\pgfqpoint{3.025050in}{2.007051in}}%
\pgfpathmoveto{\pgfqpoint{3.025050in}{2.007051in}}%
\pgfpathlineto{\pgfqpoint{3.025050in}{2.007051in}}%
\pgfpathlineto{\pgfqpoint{3.025050in}{2.010001in}}%
\pgfpathlineto{\pgfqpoint{3.029591in}{2.010001in}}%
\pgfpathlineto{\pgfqpoint{3.029591in}{2.007051in}}%
\pgfpathmoveto{\pgfqpoint{3.029591in}{2.007051in}}%
\pgfpathlineto{\pgfqpoint{3.029591in}{2.007051in}}%
\pgfpathlineto{\pgfqpoint{3.029591in}{2.010001in}}%
\pgfpathlineto{\pgfqpoint{3.034132in}{2.010001in}}%
\pgfpathlineto{\pgfqpoint{3.034132in}{2.007051in}}%
\pgfpathmoveto{\pgfqpoint{3.034132in}{2.007051in}}%
\pgfpathlineto{\pgfqpoint{3.034132in}{2.007051in}}%
\pgfpathlineto{\pgfqpoint{3.034132in}{2.010001in}}%
\pgfpathlineto{\pgfqpoint{3.038673in}{2.010001in}}%
\pgfpathlineto{\pgfqpoint{3.038673in}{2.007051in}}%
\pgfpathmoveto{\pgfqpoint{3.038673in}{2.007051in}}%
\pgfpathlineto{\pgfqpoint{3.038673in}{2.007051in}}%
\pgfpathlineto{\pgfqpoint{3.038673in}{2.010001in}}%
\pgfpathlineto{\pgfqpoint{3.043214in}{2.010001in}}%
\pgfpathlineto{\pgfqpoint{3.043214in}{2.007051in}}%
\pgfpathmoveto{\pgfqpoint{3.043214in}{2.007051in}}%
\pgfpathlineto{\pgfqpoint{3.043214in}{2.007051in}}%
\pgfpathlineto{\pgfqpoint{3.043214in}{2.010001in}}%
\pgfpathlineto{\pgfqpoint{3.047755in}{2.010001in}}%
\pgfpathlineto{\pgfqpoint{3.047755in}{2.007051in}}%
\pgfpathmoveto{\pgfqpoint{3.047755in}{2.007051in}}%
\pgfpathlineto{\pgfqpoint{3.047755in}{2.007051in}}%
\pgfpathlineto{\pgfqpoint{3.047755in}{2.010001in}}%
\pgfpathlineto{\pgfqpoint{3.052295in}{2.010001in}}%
\pgfpathlineto{\pgfqpoint{3.052295in}{2.007051in}}%
\pgfpathmoveto{\pgfqpoint{3.052295in}{2.007051in}}%
\pgfpathlineto{\pgfqpoint{3.052295in}{2.007051in}}%
\pgfpathlineto{\pgfqpoint{3.052295in}{2.010001in}}%
\pgfpathlineto{\pgfqpoint{3.056836in}{2.010001in}}%
\pgfpathlineto{\pgfqpoint{3.056836in}{2.007051in}}%
\pgfpathmoveto{\pgfqpoint{3.056836in}{2.007051in}}%
\pgfpathlineto{\pgfqpoint{3.056836in}{2.007051in}}%
\pgfpathlineto{\pgfqpoint{3.056836in}{2.010001in}}%
\pgfpathlineto{\pgfqpoint{3.061377in}{2.010001in}}%
\pgfpathlineto{\pgfqpoint{3.061377in}{2.007051in}}%
\pgfpathmoveto{\pgfqpoint{3.061377in}{2.007051in}}%
\pgfpathlineto{\pgfqpoint{3.061377in}{2.007051in}}%
\pgfpathlineto{\pgfqpoint{3.061377in}{2.010001in}}%
\pgfpathlineto{\pgfqpoint{3.065918in}{2.010001in}}%
\pgfpathlineto{\pgfqpoint{3.065918in}{2.007051in}}%
\pgfpathmoveto{\pgfqpoint{3.065918in}{2.007051in}}%
\pgfpathlineto{\pgfqpoint{3.065918in}{2.007051in}}%
\pgfpathlineto{\pgfqpoint{3.065918in}{2.010001in}}%
\pgfpathlineto{\pgfqpoint{3.070459in}{2.010001in}}%
\pgfpathlineto{\pgfqpoint{3.070459in}{2.007051in}}%
\pgfpathmoveto{\pgfqpoint{3.070459in}{2.007051in}}%
\pgfpathlineto{\pgfqpoint{3.070459in}{2.007051in}}%
\pgfpathlineto{\pgfqpoint{3.070459in}{2.010001in}}%
\pgfpathlineto{\pgfqpoint{3.075000in}{2.010001in}}%
\pgfpathlineto{\pgfqpoint{3.075000in}{2.007051in}}%
\pgfpathmoveto{\pgfqpoint{3.075000in}{2.007051in}}%
\pgfpathlineto{\pgfqpoint{3.075000in}{2.007051in}}%
\pgfpathlineto{\pgfqpoint{3.075000in}{2.010001in}}%
\pgfpathlineto{\pgfqpoint{3.079541in}{2.010001in}}%
\pgfpathlineto{\pgfqpoint{3.079541in}{2.007051in}}%
\pgfpathmoveto{\pgfqpoint{3.079541in}{2.007051in}}%
\pgfpathlineto{\pgfqpoint{3.079541in}{2.007051in}}%
\pgfpathlineto{\pgfqpoint{3.079541in}{2.010001in}}%
\pgfpathlineto{\pgfqpoint{3.084082in}{2.010001in}}%
\pgfpathlineto{\pgfqpoint{3.084082in}{2.007051in}}%
\pgfpathmoveto{\pgfqpoint{3.084082in}{2.007051in}}%
\pgfpathlineto{\pgfqpoint{3.084082in}{2.007051in}}%
\pgfpathlineto{\pgfqpoint{3.084082in}{2.010001in}}%
\pgfpathlineto{\pgfqpoint{3.088623in}{2.010001in}}%
\pgfpathlineto{\pgfqpoint{3.088623in}{2.007051in}}%
\pgfpathmoveto{\pgfqpoint{3.088623in}{2.007051in}}%
\pgfpathlineto{\pgfqpoint{3.088623in}{2.007051in}}%
\pgfpathlineto{\pgfqpoint{3.088623in}{2.010001in}}%
\pgfpathlineto{\pgfqpoint{3.093164in}{2.010001in}}%
\pgfpathlineto{\pgfqpoint{3.093164in}{2.007051in}}%
\pgfpathmoveto{\pgfqpoint{3.093164in}{2.007051in}}%
\pgfpathlineto{\pgfqpoint{3.093164in}{2.007051in}}%
\pgfpathlineto{\pgfqpoint{3.093164in}{2.010001in}}%
\pgfpathlineto{\pgfqpoint{3.097706in}{2.010001in}}%
\pgfpathlineto{\pgfqpoint{3.097706in}{2.007051in}}%
\pgfpathmoveto{\pgfqpoint{3.097706in}{2.007051in}}%
\pgfpathlineto{\pgfqpoint{3.097706in}{2.007051in}}%
\pgfpathlineto{\pgfqpoint{3.097706in}{2.010001in}}%
\pgfpathlineto{\pgfqpoint{3.102247in}{2.010001in}}%
\pgfpathlineto{\pgfqpoint{3.102247in}{2.007051in}}%
\pgfpathmoveto{\pgfqpoint{3.102247in}{2.007051in}}%
\pgfpathlineto{\pgfqpoint{3.102247in}{2.007051in}}%
\pgfpathlineto{\pgfqpoint{3.102247in}{2.010001in}}%
\pgfpathlineto{\pgfqpoint{3.106788in}{2.010001in}}%
\pgfpathlineto{\pgfqpoint{3.106788in}{2.007051in}}%
\pgfpathmoveto{\pgfqpoint{3.106788in}{2.007051in}}%
\pgfpathlineto{\pgfqpoint{3.106788in}{2.007051in}}%
\pgfpathlineto{\pgfqpoint{3.106788in}{2.010001in}}%
\pgfpathlineto{\pgfqpoint{3.111329in}{2.010001in}}%
\pgfpathlineto{\pgfqpoint{3.111329in}{2.007051in}}%
\pgfpathmoveto{\pgfqpoint{3.111329in}{2.007051in}}%
\pgfpathlineto{\pgfqpoint{3.111329in}{2.007051in}}%
\pgfpathlineto{\pgfqpoint{3.111329in}{2.010001in}}%
\pgfpathlineto{\pgfqpoint{3.115870in}{2.010001in}}%
\pgfpathlineto{\pgfqpoint{3.115870in}{2.007051in}}%
\pgfpathmoveto{\pgfqpoint{3.115870in}{2.007051in}}%
\pgfpathlineto{\pgfqpoint{3.115870in}{2.007051in}}%
\pgfpathlineto{\pgfqpoint{3.115870in}{2.010001in}}%
\pgfpathlineto{\pgfqpoint{3.120411in}{2.010001in}}%
\pgfpathlineto{\pgfqpoint{3.120411in}{2.007051in}}%
\pgfpathmoveto{\pgfqpoint{3.120411in}{2.007051in}}%
\pgfpathlineto{\pgfqpoint{3.120411in}{2.007051in}}%
\pgfpathlineto{\pgfqpoint{3.120411in}{2.010001in}}%
\pgfpathlineto{\pgfqpoint{3.124952in}{2.010001in}}%
\pgfpathlineto{\pgfqpoint{3.124952in}{2.007051in}}%
\pgfpathmoveto{\pgfqpoint{3.124952in}{2.007051in}}%
\pgfpathlineto{\pgfqpoint{3.124952in}{2.007051in}}%
\pgfpathlineto{\pgfqpoint{3.124952in}{2.010001in}}%
\pgfpathlineto{\pgfqpoint{3.129494in}{2.010001in}}%
\pgfpathlineto{\pgfqpoint{3.129494in}{2.007051in}}%
\pgfpathmoveto{\pgfqpoint{3.129494in}{2.007051in}}%
\pgfpathlineto{\pgfqpoint{3.129494in}{2.007051in}}%
\pgfpathlineto{\pgfqpoint{3.129494in}{2.010001in}}%
\pgfpathlineto{\pgfqpoint{3.134035in}{2.010001in}}%
\pgfpathlineto{\pgfqpoint{3.134035in}{2.007051in}}%
\pgfpathmoveto{\pgfqpoint{3.134035in}{2.007051in}}%
\pgfpathlineto{\pgfqpoint{3.134035in}{2.007051in}}%
\pgfpathlineto{\pgfqpoint{3.134035in}{2.010001in}}%
\pgfpathlineto{\pgfqpoint{3.138576in}{2.010001in}}%
\pgfpathlineto{\pgfqpoint{3.138576in}{2.007051in}}%
\pgfpathmoveto{\pgfqpoint{3.138576in}{2.007051in}}%
\pgfpathlineto{\pgfqpoint{3.138576in}{2.007051in}}%
\pgfpathlineto{\pgfqpoint{3.138576in}{2.010001in}}%
\pgfpathlineto{\pgfqpoint{3.143117in}{2.010001in}}%
\pgfpathlineto{\pgfqpoint{3.143117in}{2.007051in}}%
\pgfpathmoveto{\pgfqpoint{3.143117in}{2.007051in}}%
\pgfpathlineto{\pgfqpoint{3.143117in}{2.007051in}}%
\pgfpathlineto{\pgfqpoint{3.143117in}{2.010001in}}%
\pgfpathlineto{\pgfqpoint{3.147658in}{2.010001in}}%
\pgfpathlineto{\pgfqpoint{3.147658in}{2.007051in}}%
\pgfpathmoveto{\pgfqpoint{3.147658in}{2.007051in}}%
\pgfpathlineto{\pgfqpoint{3.147658in}{2.007051in}}%
\pgfpathlineto{\pgfqpoint{3.147658in}{2.010001in}}%
\pgfpathlineto{\pgfqpoint{3.152199in}{2.010001in}}%
\pgfpathlineto{\pgfqpoint{3.152199in}{2.007051in}}%
\pgfpathmoveto{\pgfqpoint{3.152199in}{2.007051in}}%
\pgfpathlineto{\pgfqpoint{3.152199in}{2.007051in}}%
\pgfpathlineto{\pgfqpoint{3.152199in}{2.010001in}}%
\pgfpathlineto{\pgfqpoint{3.156740in}{2.010001in}}%
\pgfpathlineto{\pgfqpoint{3.156740in}{2.007051in}}%
\pgfpathmoveto{\pgfqpoint{3.156740in}{2.007051in}}%
\pgfpathlineto{\pgfqpoint{3.156740in}{2.007051in}}%
\pgfpathlineto{\pgfqpoint{3.156740in}{2.010001in}}%
\pgfpathlineto{\pgfqpoint{3.161281in}{2.010001in}}%
\pgfpathlineto{\pgfqpoint{3.161281in}{2.007051in}}%
\pgfpathmoveto{\pgfqpoint{3.161281in}{2.007051in}}%
\pgfpathlineto{\pgfqpoint{3.161281in}{2.007051in}}%
\pgfpathlineto{\pgfqpoint{3.161281in}{2.010001in}}%
\pgfpathlineto{\pgfqpoint{3.165823in}{2.010001in}}%
\pgfpathlineto{\pgfqpoint{3.165823in}{2.007051in}}%
\pgfpathmoveto{\pgfqpoint{3.165823in}{2.007051in}}%
\pgfpathlineto{\pgfqpoint{3.165823in}{2.007051in}}%
\pgfpathlineto{\pgfqpoint{3.165823in}{2.010001in}}%
\pgfpathlineto{\pgfqpoint{3.170364in}{2.010001in}}%
\pgfpathlineto{\pgfqpoint{3.170364in}{2.007051in}}%
\pgfpathmoveto{\pgfqpoint{3.170364in}{2.007051in}}%
\pgfpathlineto{\pgfqpoint{3.170364in}{2.007051in}}%
\pgfpathlineto{\pgfqpoint{3.170364in}{2.010001in}}%
\pgfpathlineto{\pgfqpoint{3.174905in}{2.010001in}}%
\pgfpathlineto{\pgfqpoint{3.174905in}{2.007051in}}%
\pgfpathmoveto{\pgfqpoint{3.174905in}{2.007051in}}%
\pgfpathlineto{\pgfqpoint{3.174905in}{2.007051in}}%
\pgfpathlineto{\pgfqpoint{3.174905in}{2.010001in}}%
\pgfpathlineto{\pgfqpoint{3.179446in}{2.010001in}}%
\pgfpathlineto{\pgfqpoint{3.179446in}{2.007051in}}%
\pgfpathmoveto{\pgfqpoint{3.179446in}{2.007051in}}%
\pgfpathlineto{\pgfqpoint{3.179446in}{2.007051in}}%
\pgfpathlineto{\pgfqpoint{3.179446in}{2.010001in}}%
\pgfpathlineto{\pgfqpoint{3.183987in}{2.010001in}}%
\pgfpathlineto{\pgfqpoint{3.183987in}{2.007051in}}%
\pgfpathmoveto{\pgfqpoint{3.183987in}{2.007051in}}%
\pgfpathlineto{\pgfqpoint{3.183987in}{2.007051in}}%
\pgfpathlineto{\pgfqpoint{3.183987in}{2.010001in}}%
\pgfpathlineto{\pgfqpoint{3.188528in}{2.010001in}}%
\pgfpathlineto{\pgfqpoint{3.188528in}{2.007051in}}%
\pgfpathmoveto{\pgfqpoint{3.188528in}{2.007051in}}%
\pgfpathlineto{\pgfqpoint{3.188528in}{2.007051in}}%
\pgfpathlineto{\pgfqpoint{3.188528in}{2.010001in}}%
\pgfpathlineto{\pgfqpoint{3.193069in}{2.010001in}}%
\pgfpathlineto{\pgfqpoint{3.193069in}{2.007051in}}%
\pgfpathmoveto{\pgfqpoint{3.193069in}{2.007051in}}%
\pgfpathlineto{\pgfqpoint{3.193069in}{2.007051in}}%
\pgfpathlineto{\pgfqpoint{3.193069in}{2.010001in}}%
\pgfpathlineto{\pgfqpoint{3.197611in}{2.010001in}}%
\pgfpathlineto{\pgfqpoint{3.197611in}{2.007051in}}%
\pgfpathmoveto{\pgfqpoint{3.197611in}{2.007051in}}%
\pgfpathlineto{\pgfqpoint{3.197611in}{2.007051in}}%
\pgfpathlineto{\pgfqpoint{3.197611in}{2.010001in}}%
\pgfpathlineto{\pgfqpoint{3.202152in}{2.010001in}}%
\pgfpathlineto{\pgfqpoint{3.202152in}{2.007051in}}%
\pgfpathmoveto{\pgfqpoint{3.202152in}{2.007051in}}%
\pgfpathlineto{\pgfqpoint{3.202152in}{2.007051in}}%
\pgfpathlineto{\pgfqpoint{3.202152in}{2.010001in}}%
\pgfpathlineto{\pgfqpoint{3.206693in}{2.010001in}}%
\pgfpathlineto{\pgfqpoint{3.206693in}{2.007051in}}%
\pgfpathmoveto{\pgfqpoint{3.206693in}{2.007051in}}%
\pgfpathlineto{\pgfqpoint{3.206693in}{2.007051in}}%
\pgfpathlineto{\pgfqpoint{3.206693in}{2.010001in}}%
\pgfpathlineto{\pgfqpoint{3.211234in}{2.010001in}}%
\pgfpathlineto{\pgfqpoint{3.211234in}{2.007051in}}%
\pgfpathmoveto{\pgfqpoint{3.211234in}{2.007051in}}%
\pgfpathlineto{\pgfqpoint{3.211234in}{2.007051in}}%
\pgfpathlineto{\pgfqpoint{3.211234in}{2.010001in}}%
\pgfpathlineto{\pgfqpoint{3.215775in}{2.010001in}}%
\pgfpathlineto{\pgfqpoint{3.215775in}{2.007051in}}%
\pgfpathmoveto{\pgfqpoint{3.215775in}{2.007051in}}%
\pgfpathlineto{\pgfqpoint{3.215775in}{2.007051in}}%
\pgfpathlineto{\pgfqpoint{3.215775in}{2.010001in}}%
\pgfpathlineto{\pgfqpoint{3.220316in}{2.010001in}}%
\pgfpathlineto{\pgfqpoint{3.220316in}{2.007051in}}%
\pgfpathmoveto{\pgfqpoint{3.220316in}{2.007051in}}%
\pgfpathlineto{\pgfqpoint{3.220316in}{2.007051in}}%
\pgfpathlineto{\pgfqpoint{3.220316in}{2.010001in}}%
\pgfpathlineto{\pgfqpoint{3.224857in}{2.010001in}}%
\pgfpathlineto{\pgfqpoint{3.224857in}{2.007051in}}%
\pgfpathmoveto{\pgfqpoint{3.224857in}{2.007051in}}%
\pgfpathlineto{\pgfqpoint{3.224857in}{2.007051in}}%
\pgfpathlineto{\pgfqpoint{3.224857in}{2.010001in}}%
\pgfpathlineto{\pgfqpoint{3.229398in}{2.010001in}}%
\pgfpathlineto{\pgfqpoint{3.229398in}{2.007051in}}%
\pgfpathmoveto{\pgfqpoint{3.229398in}{2.007051in}}%
\pgfpathlineto{\pgfqpoint{3.229398in}{2.007051in}}%
\pgfpathlineto{\pgfqpoint{3.229398in}{2.010001in}}%
\pgfpathlineto{\pgfqpoint{3.233939in}{2.010001in}}%
\pgfpathlineto{\pgfqpoint{3.233939in}{2.007051in}}%
\pgfpathmoveto{\pgfqpoint{3.233939in}{2.007051in}}%
\pgfpathlineto{\pgfqpoint{3.233939in}{2.007051in}}%
\pgfpathlineto{\pgfqpoint{3.233939in}{2.010001in}}%
\pgfpathlineto{\pgfqpoint{3.238480in}{2.010001in}}%
\pgfpathlineto{\pgfqpoint{3.238480in}{2.007051in}}%
\pgfpathmoveto{\pgfqpoint{3.238480in}{2.007051in}}%
\pgfpathlineto{\pgfqpoint{3.238480in}{2.007051in}}%
\pgfpathlineto{\pgfqpoint{3.238480in}{2.010001in}}%
\pgfpathlineto{\pgfqpoint{3.243021in}{2.010001in}}%
\pgfpathlineto{\pgfqpoint{3.243021in}{2.007051in}}%
\pgfpathmoveto{\pgfqpoint{3.243021in}{2.007051in}}%
\pgfpathlineto{\pgfqpoint{3.243021in}{2.007051in}}%
\pgfpathlineto{\pgfqpoint{3.243021in}{2.010001in}}%
\pgfpathlineto{\pgfqpoint{3.247562in}{2.010001in}}%
\pgfpathlineto{\pgfqpoint{3.247562in}{2.007051in}}%
\pgfpathmoveto{\pgfqpoint{3.247562in}{2.007051in}}%
\pgfpathlineto{\pgfqpoint{3.247562in}{2.007051in}}%
\pgfpathlineto{\pgfqpoint{3.247562in}{2.010001in}}%
\pgfpathlineto{\pgfqpoint{3.252103in}{2.010001in}}%
\pgfpathlineto{\pgfqpoint{3.252103in}{2.007051in}}%
\pgfpathmoveto{\pgfqpoint{3.252103in}{2.007051in}}%
\pgfpathlineto{\pgfqpoint{3.252103in}{2.007051in}}%
\pgfpathlineto{\pgfqpoint{3.252103in}{2.010001in}}%
\pgfpathlineto{\pgfqpoint{3.256644in}{2.010001in}}%
\pgfpathlineto{\pgfqpoint{3.256644in}{2.007051in}}%
\pgfpathmoveto{\pgfqpoint{3.256644in}{2.007051in}}%
\pgfpathlineto{\pgfqpoint{3.256644in}{2.007051in}}%
\pgfpathlineto{\pgfqpoint{3.256644in}{2.010001in}}%
\pgfpathlineto{\pgfqpoint{3.261185in}{2.010001in}}%
\pgfpathlineto{\pgfqpoint{3.261185in}{2.007051in}}%
\pgfpathmoveto{\pgfqpoint{3.261185in}{2.007051in}}%
\pgfpathlineto{\pgfqpoint{3.261185in}{2.007051in}}%
\pgfpathlineto{\pgfqpoint{3.261185in}{2.010001in}}%
\pgfpathlineto{\pgfqpoint{3.265726in}{2.010001in}}%
\pgfpathlineto{\pgfqpoint{3.265726in}{2.007051in}}%
\pgfpathmoveto{\pgfqpoint{3.265726in}{2.007051in}}%
\pgfpathlineto{\pgfqpoint{3.265726in}{2.007051in}}%
\pgfpathlineto{\pgfqpoint{3.265726in}{2.010001in}}%
\pgfpathlineto{\pgfqpoint{3.270266in}{2.010001in}}%
\pgfpathlineto{\pgfqpoint{3.270266in}{2.007051in}}%
\pgfpathmoveto{\pgfqpoint{3.270266in}{2.007051in}}%
\pgfpathlineto{\pgfqpoint{3.270266in}{2.007051in}}%
\pgfpathlineto{\pgfqpoint{3.270266in}{2.010001in}}%
\pgfpathlineto{\pgfqpoint{3.274807in}{2.010001in}}%
\pgfpathlineto{\pgfqpoint{3.274807in}{2.007051in}}%
\pgfpathmoveto{\pgfqpoint{3.274807in}{2.007051in}}%
\pgfpathlineto{\pgfqpoint{3.274807in}{2.007051in}}%
\pgfpathlineto{\pgfqpoint{3.274807in}{2.010001in}}%
\pgfpathlineto{\pgfqpoint{3.279348in}{2.010001in}}%
\pgfpathlineto{\pgfqpoint{3.279348in}{2.007051in}}%
\pgfpathmoveto{\pgfqpoint{3.279348in}{2.007051in}}%
\pgfpathlineto{\pgfqpoint{3.279348in}{2.007051in}}%
\pgfpathlineto{\pgfqpoint{3.279348in}{2.010001in}}%
\pgfpathlineto{\pgfqpoint{3.283889in}{2.010001in}}%
\pgfpathlineto{\pgfqpoint{3.283889in}{2.007051in}}%
\pgfpathmoveto{\pgfqpoint{3.283889in}{2.007051in}}%
\pgfpathlineto{\pgfqpoint{3.283889in}{2.007051in}}%
\pgfpathlineto{\pgfqpoint{3.283889in}{2.010001in}}%
\pgfpathlineto{\pgfqpoint{3.288430in}{2.010001in}}%
\pgfpathlineto{\pgfqpoint{3.288430in}{2.007051in}}%
\pgfpathmoveto{\pgfqpoint{3.288430in}{2.007051in}}%
\pgfpathlineto{\pgfqpoint{3.288430in}{2.007051in}}%
\pgfpathlineto{\pgfqpoint{3.288430in}{2.010001in}}%
\pgfpathlineto{\pgfqpoint{3.292971in}{2.010001in}}%
\pgfpathlineto{\pgfqpoint{3.292971in}{2.007051in}}%
\pgfpathmoveto{\pgfqpoint{3.292971in}{2.007051in}}%
\pgfpathlineto{\pgfqpoint{3.292971in}{2.007051in}}%
\pgfpathlineto{\pgfqpoint{3.292971in}{2.010001in}}%
\pgfpathlineto{\pgfqpoint{3.297512in}{2.010001in}}%
\pgfpathlineto{\pgfqpoint{3.297512in}{2.007051in}}%
\pgfpathmoveto{\pgfqpoint{3.297512in}{2.007051in}}%
\pgfpathlineto{\pgfqpoint{3.297512in}{2.007051in}}%
\pgfpathlineto{\pgfqpoint{3.297512in}{2.010001in}}%
\pgfpathlineto{\pgfqpoint{3.302053in}{2.010001in}}%
\pgfpathlineto{\pgfqpoint{3.302053in}{2.007051in}}%
\pgfpathmoveto{\pgfqpoint{3.302053in}{2.007051in}}%
\pgfpathlineto{\pgfqpoint{3.302053in}{2.007051in}}%
\pgfpathlineto{\pgfqpoint{3.302053in}{2.010001in}}%
\pgfpathlineto{\pgfqpoint{3.306594in}{2.010001in}}%
\pgfpathlineto{\pgfqpoint{3.306594in}{2.007051in}}%
\pgfpathmoveto{\pgfqpoint{3.306594in}{2.007051in}}%
\pgfpathlineto{\pgfqpoint{3.306594in}{2.007051in}}%
\pgfpathlineto{\pgfqpoint{3.306594in}{2.010001in}}%
\pgfpathlineto{\pgfqpoint{3.311135in}{2.010001in}}%
\pgfpathlineto{\pgfqpoint{3.311135in}{2.007051in}}%
\pgfpathmoveto{\pgfqpoint{3.311135in}{2.007051in}}%
\pgfpathlineto{\pgfqpoint{3.311135in}{2.007051in}}%
\pgfpathlineto{\pgfqpoint{3.311135in}{2.010001in}}%
\pgfpathlineto{\pgfqpoint{3.315676in}{2.010001in}}%
\pgfpathlineto{\pgfqpoint{3.315676in}{2.007051in}}%
\pgfpathmoveto{\pgfqpoint{3.315676in}{2.007051in}}%
\pgfpathlineto{\pgfqpoint{3.315676in}{2.007051in}}%
\pgfpathlineto{\pgfqpoint{3.315676in}{2.010001in}}%
\pgfpathlineto{\pgfqpoint{3.320217in}{2.010001in}}%
\pgfpathlineto{\pgfqpoint{3.320217in}{2.007051in}}%
\pgfpathmoveto{\pgfqpoint{3.320217in}{2.007051in}}%
\pgfpathlineto{\pgfqpoint{3.320217in}{2.007051in}}%
\pgfpathlineto{\pgfqpoint{3.320217in}{2.010001in}}%
\pgfpathlineto{\pgfqpoint{3.324758in}{2.010001in}}%
\pgfpathlineto{\pgfqpoint{3.324758in}{2.007051in}}%
\pgfpathmoveto{\pgfqpoint{3.324758in}{2.007051in}}%
\pgfpathlineto{\pgfqpoint{3.324758in}{2.007051in}}%
\pgfpathlineto{\pgfqpoint{3.324758in}{2.010001in}}%
\pgfpathlineto{\pgfqpoint{3.329298in}{2.010001in}}%
\pgfpathlineto{\pgfqpoint{3.329298in}{2.007051in}}%
\pgfpathmoveto{\pgfqpoint{3.329298in}{2.007051in}}%
\pgfpathlineto{\pgfqpoint{3.329298in}{2.007051in}}%
\pgfpathlineto{\pgfqpoint{3.329298in}{2.010001in}}%
\pgfpathlineto{\pgfqpoint{3.333839in}{2.010001in}}%
\pgfpathlineto{\pgfqpoint{3.333839in}{2.007051in}}%
\pgfpathmoveto{\pgfqpoint{3.333839in}{2.007051in}}%
\pgfpathlineto{\pgfqpoint{3.333839in}{2.007051in}}%
\pgfpathlineto{\pgfqpoint{3.333839in}{2.010001in}}%
\pgfpathlineto{\pgfqpoint{3.338380in}{2.010001in}}%
\pgfpathlineto{\pgfqpoint{3.338380in}{2.007051in}}%
\pgfpathmoveto{\pgfqpoint{3.338380in}{2.007051in}}%
\pgfpathlineto{\pgfqpoint{3.338380in}{2.007051in}}%
\pgfpathlineto{\pgfqpoint{3.338380in}{2.010001in}}%
\pgfpathlineto{\pgfqpoint{3.342921in}{2.010001in}}%
\pgfpathlineto{\pgfqpoint{3.342921in}{2.007051in}}%
\pgfpathmoveto{\pgfqpoint{3.342921in}{2.007051in}}%
\pgfpathlineto{\pgfqpoint{3.342921in}{2.007051in}}%
\pgfpathlineto{\pgfqpoint{3.342921in}{2.010001in}}%
\pgfpathlineto{\pgfqpoint{3.347462in}{2.010001in}}%
\pgfpathlineto{\pgfqpoint{3.347462in}{2.007051in}}%
\pgfpathmoveto{\pgfqpoint{3.347462in}{2.007051in}}%
\pgfpathlineto{\pgfqpoint{3.347462in}{2.007051in}}%
\pgfpathlineto{\pgfqpoint{3.347462in}{2.010001in}}%
\pgfpathlineto{\pgfqpoint{3.352003in}{2.010001in}}%
\pgfpathlineto{\pgfqpoint{3.352003in}{2.007051in}}%
\pgfpathmoveto{\pgfqpoint{3.352003in}{2.007051in}}%
\pgfpathlineto{\pgfqpoint{3.352003in}{2.007051in}}%
\pgfpathlineto{\pgfqpoint{3.352003in}{2.010001in}}%
\pgfpathlineto{\pgfqpoint{3.356544in}{2.010001in}}%
\pgfpathlineto{\pgfqpoint{3.356544in}{2.007051in}}%
\pgfpathmoveto{\pgfqpoint{3.356544in}{2.007051in}}%
\pgfpathlineto{\pgfqpoint{3.356544in}{2.007051in}}%
\pgfpathlineto{\pgfqpoint{3.356544in}{2.010001in}}%
\pgfpathlineto{\pgfqpoint{3.361085in}{2.010001in}}%
\pgfpathlineto{\pgfqpoint{3.361085in}{2.007051in}}%
\pgfpathmoveto{\pgfqpoint{3.361085in}{2.007051in}}%
\pgfpathlineto{\pgfqpoint{3.361085in}{2.007051in}}%
\pgfpathlineto{\pgfqpoint{3.361085in}{2.010001in}}%
\pgfpathlineto{\pgfqpoint{3.365626in}{2.010001in}}%
\pgfpathlineto{\pgfqpoint{3.365626in}{2.007051in}}%
\pgfpathmoveto{\pgfqpoint{3.365626in}{2.007051in}}%
\pgfpathlineto{\pgfqpoint{3.365626in}{2.007051in}}%
\pgfpathlineto{\pgfqpoint{3.365626in}{2.010001in}}%
\pgfpathlineto{\pgfqpoint{3.370167in}{2.010001in}}%
\pgfpathlineto{\pgfqpoint{3.370167in}{2.007051in}}%
\pgfpathmoveto{\pgfqpoint{3.370167in}{2.007051in}}%
\pgfpathlineto{\pgfqpoint{3.370167in}{2.007051in}}%
\pgfpathlineto{\pgfqpoint{3.370167in}{2.010001in}}%
\pgfpathlineto{\pgfqpoint{3.374708in}{2.010001in}}%
\pgfpathlineto{\pgfqpoint{3.374708in}{2.007051in}}%
\pgfpathmoveto{\pgfqpoint{3.374708in}{2.007051in}}%
\pgfpathlineto{\pgfqpoint{3.374708in}{2.007051in}}%
\pgfpathlineto{\pgfqpoint{3.374708in}{2.010001in}}%
\pgfpathlineto{\pgfqpoint{3.379249in}{2.010001in}}%
\pgfpathlineto{\pgfqpoint{3.379249in}{2.007051in}}%
\pgfpathmoveto{\pgfqpoint{3.379249in}{2.007051in}}%
\pgfpathlineto{\pgfqpoint{3.379249in}{2.007051in}}%
\pgfpathlineto{\pgfqpoint{3.379249in}{2.010001in}}%
\pgfpathlineto{\pgfqpoint{3.383790in}{2.010001in}}%
\pgfpathlineto{\pgfqpoint{3.383790in}{2.007051in}}%
\pgfpathmoveto{\pgfqpoint{3.383790in}{2.007051in}}%
\pgfpathlineto{\pgfqpoint{3.383790in}{2.007051in}}%
\pgfpathlineto{\pgfqpoint{3.383790in}{2.010001in}}%
\pgfpathlineto{\pgfqpoint{3.388331in}{2.010001in}}%
\pgfpathlineto{\pgfqpoint{3.388331in}{2.007051in}}%
\pgfpathmoveto{\pgfqpoint{3.388331in}{2.007051in}}%
\pgfpathlineto{\pgfqpoint{3.388331in}{2.007051in}}%
\pgfpathlineto{\pgfqpoint{3.388331in}{2.010001in}}%
\pgfpathlineto{\pgfqpoint{3.392872in}{2.010001in}}%
\pgfpathlineto{\pgfqpoint{3.392872in}{2.007051in}}%
\pgfpathmoveto{\pgfqpoint{3.392872in}{2.007051in}}%
\pgfpathlineto{\pgfqpoint{3.392872in}{2.007051in}}%
\pgfpathlineto{\pgfqpoint{3.392872in}{2.010001in}}%
\pgfpathlineto{\pgfqpoint{3.397413in}{2.010001in}}%
\pgfpathlineto{\pgfqpoint{3.397413in}{2.007051in}}%
\pgfpathmoveto{\pgfqpoint{3.397413in}{2.007051in}}%
\pgfpathlineto{\pgfqpoint{3.397413in}{2.007051in}}%
\pgfpathlineto{\pgfqpoint{3.397413in}{2.010001in}}%
\pgfpathlineto{\pgfqpoint{3.401954in}{2.010001in}}%
\pgfpathlineto{\pgfqpoint{3.401954in}{2.007051in}}%
\pgfpathmoveto{\pgfqpoint{3.401954in}{2.007051in}}%
\pgfpathlineto{\pgfqpoint{3.401954in}{2.007051in}}%
\pgfpathlineto{\pgfqpoint{3.401954in}{2.010001in}}%
\pgfpathlineto{\pgfqpoint{3.406495in}{2.010001in}}%
\pgfpathlineto{\pgfqpoint{3.406495in}{2.007051in}}%
\pgfpathmoveto{\pgfqpoint{3.406495in}{2.007051in}}%
\pgfpathlineto{\pgfqpoint{3.406495in}{2.007051in}}%
\pgfpathlineto{\pgfqpoint{3.406495in}{2.010001in}}%
\pgfpathlineto{\pgfqpoint{3.411036in}{2.010001in}}%
\pgfpathlineto{\pgfqpoint{3.411036in}{2.007051in}}%
\pgfpathmoveto{\pgfqpoint{3.411036in}{2.007051in}}%
\pgfpathlineto{\pgfqpoint{3.411036in}{2.007051in}}%
\pgfpathlineto{\pgfqpoint{3.411036in}{2.010001in}}%
\pgfpathlineto{\pgfqpoint{3.415578in}{2.010001in}}%
\pgfpathlineto{\pgfqpoint{3.415578in}{2.007051in}}%
\pgfpathmoveto{\pgfqpoint{3.415578in}{2.007051in}}%
\pgfpathlineto{\pgfqpoint{3.415578in}{2.007051in}}%
\pgfpathlineto{\pgfqpoint{3.415578in}{2.010001in}}%
\pgfpathlineto{\pgfqpoint{3.420119in}{2.010001in}}%
\pgfpathlineto{\pgfqpoint{3.420119in}{2.007051in}}%
\pgfpathmoveto{\pgfqpoint{3.420119in}{2.007051in}}%
\pgfpathlineto{\pgfqpoint{3.420119in}{2.007051in}}%
\pgfpathlineto{\pgfqpoint{3.420119in}{2.010001in}}%
\pgfpathlineto{\pgfqpoint{3.424660in}{2.010001in}}%
\pgfpathlineto{\pgfqpoint{3.424660in}{2.007051in}}%
\pgfpathmoveto{\pgfqpoint{3.424660in}{2.007051in}}%
\pgfpathlineto{\pgfqpoint{3.424660in}{2.007051in}}%
\pgfpathlineto{\pgfqpoint{3.424660in}{2.010001in}}%
\pgfpathlineto{\pgfqpoint{3.429201in}{2.010001in}}%
\pgfpathlineto{\pgfqpoint{3.429201in}{2.007051in}}%
\pgfpathmoveto{\pgfqpoint{3.429201in}{2.007051in}}%
\pgfpathlineto{\pgfqpoint{3.429201in}{2.007051in}}%
\pgfpathlineto{\pgfqpoint{3.429201in}{2.010001in}}%
\pgfpathlineto{\pgfqpoint{3.433742in}{2.010001in}}%
\pgfpathlineto{\pgfqpoint{3.433742in}{2.007051in}}%
\pgfpathmoveto{\pgfqpoint{3.433742in}{2.007051in}}%
\pgfpathlineto{\pgfqpoint{3.433742in}{2.007051in}}%
\pgfpathlineto{\pgfqpoint{3.433742in}{2.010001in}}%
\pgfpathlineto{\pgfqpoint{3.438283in}{2.010001in}}%
\pgfpathlineto{\pgfqpoint{3.438283in}{2.007051in}}%
\pgfpathmoveto{\pgfqpoint{3.438283in}{2.007051in}}%
\pgfpathlineto{\pgfqpoint{3.438283in}{2.007051in}}%
\pgfpathlineto{\pgfqpoint{3.438283in}{2.010001in}}%
\pgfpathlineto{\pgfqpoint{3.442824in}{2.010001in}}%
\pgfpathlineto{\pgfqpoint{3.442824in}{2.007051in}}%
\pgfpathmoveto{\pgfqpoint{3.442824in}{2.007051in}}%
\pgfpathlineto{\pgfqpoint{3.442824in}{2.007051in}}%
\pgfpathlineto{\pgfqpoint{3.442824in}{2.010001in}}%
\pgfpathlineto{\pgfqpoint{3.447365in}{2.010001in}}%
\pgfpathlineto{\pgfqpoint{3.447365in}{2.007051in}}%
\pgfpathmoveto{\pgfqpoint{3.447365in}{2.007051in}}%
\pgfpathlineto{\pgfqpoint{3.447365in}{2.007051in}}%
\pgfpathlineto{\pgfqpoint{3.447365in}{2.010001in}}%
\pgfpathlineto{\pgfqpoint{3.451906in}{2.010001in}}%
\pgfpathlineto{\pgfqpoint{3.451906in}{2.007051in}}%
\pgfpathmoveto{\pgfqpoint{3.451906in}{2.007051in}}%
\pgfpathlineto{\pgfqpoint{3.451906in}{2.007051in}}%
\pgfpathlineto{\pgfqpoint{3.451906in}{2.010001in}}%
\pgfpathlineto{\pgfqpoint{3.456447in}{2.010001in}}%
\pgfpathlineto{\pgfqpoint{3.456447in}{2.007051in}}%
\pgfpathmoveto{\pgfqpoint{3.456447in}{2.007051in}}%
\pgfpathlineto{\pgfqpoint{3.456447in}{2.007051in}}%
\pgfpathlineto{\pgfqpoint{3.456447in}{2.010001in}}%
\pgfpathlineto{\pgfqpoint{3.460988in}{2.010001in}}%
\pgfpathlineto{\pgfqpoint{3.460988in}{2.007051in}}%
\pgfpathmoveto{\pgfqpoint{3.460988in}{2.007051in}}%
\pgfpathlineto{\pgfqpoint{3.460988in}{2.007051in}}%
\pgfpathlineto{\pgfqpoint{3.460988in}{2.010001in}}%
\pgfpathlineto{\pgfqpoint{3.465529in}{2.010001in}}%
\pgfpathlineto{\pgfqpoint{3.465529in}{2.007051in}}%
\pgfpathmoveto{\pgfqpoint{3.465529in}{2.007051in}}%
\pgfpathlineto{\pgfqpoint{3.465529in}{2.007051in}}%
\pgfpathlineto{\pgfqpoint{3.465529in}{2.010001in}}%
\pgfpathlineto{\pgfqpoint{3.470070in}{2.010001in}}%
\pgfpathlineto{\pgfqpoint{3.470070in}{2.007051in}}%
\pgfpathmoveto{\pgfqpoint{3.470070in}{2.007051in}}%
\pgfpathlineto{\pgfqpoint{3.470070in}{2.007051in}}%
\pgfpathlineto{\pgfqpoint{3.470070in}{2.010001in}}%
\pgfpathlineto{\pgfqpoint{3.474611in}{2.010001in}}%
\pgfpathlineto{\pgfqpoint{3.474611in}{2.007051in}}%
\pgfpathmoveto{\pgfqpoint{3.474611in}{2.007051in}}%
\pgfpathlineto{\pgfqpoint{3.474611in}{2.007051in}}%
\pgfpathlineto{\pgfqpoint{3.474611in}{2.010001in}}%
\pgfpathlineto{\pgfqpoint{3.479152in}{2.010001in}}%
\pgfpathlineto{\pgfqpoint{3.479152in}{2.007051in}}%
\pgfpathmoveto{\pgfqpoint{3.479152in}{2.007051in}}%
\pgfpathlineto{\pgfqpoint{3.479152in}{2.007051in}}%
\pgfpathlineto{\pgfqpoint{3.479152in}{2.010001in}}%
\pgfpathlineto{\pgfqpoint{3.483693in}{2.010001in}}%
\pgfpathlineto{\pgfqpoint{3.483693in}{2.007051in}}%
\pgfpathmoveto{\pgfqpoint{3.483693in}{2.007051in}}%
\pgfpathlineto{\pgfqpoint{3.483693in}{2.007051in}}%
\pgfpathlineto{\pgfqpoint{3.483693in}{2.010001in}}%
\pgfpathlineto{\pgfqpoint{3.488234in}{2.010001in}}%
\pgfpathlineto{\pgfqpoint{3.488234in}{2.007051in}}%
\pgfpathmoveto{\pgfqpoint{3.488234in}{2.007051in}}%
\pgfpathlineto{\pgfqpoint{3.488234in}{2.007051in}}%
\pgfpathlineto{\pgfqpoint{3.488234in}{2.010001in}}%
\pgfpathlineto{\pgfqpoint{3.492776in}{2.010001in}}%
\pgfpathlineto{\pgfqpoint{3.492776in}{2.007051in}}%
\pgfpathmoveto{\pgfqpoint{3.492776in}{2.007051in}}%
\pgfpathlineto{\pgfqpoint{3.492776in}{2.007051in}}%
\pgfpathlineto{\pgfqpoint{3.492776in}{2.010001in}}%
\pgfpathlineto{\pgfqpoint{3.497317in}{2.010001in}}%
\pgfpathlineto{\pgfqpoint{3.497317in}{2.007051in}}%
\pgfpathmoveto{\pgfqpoint{3.497317in}{2.007051in}}%
\pgfpathlineto{\pgfqpoint{3.497317in}{2.007051in}}%
\pgfpathlineto{\pgfqpoint{3.497317in}{2.010001in}}%
\pgfpathlineto{\pgfqpoint{3.501858in}{2.010001in}}%
\pgfpathlineto{\pgfqpoint{3.501858in}{2.007051in}}%
\pgfpathmoveto{\pgfqpoint{3.501858in}{2.007051in}}%
\pgfpathlineto{\pgfqpoint{3.501858in}{2.007051in}}%
\pgfpathlineto{\pgfqpoint{3.501858in}{2.010001in}}%
\pgfpathlineto{\pgfqpoint{3.506399in}{2.010001in}}%
\pgfpathlineto{\pgfqpoint{3.506399in}{2.007051in}}%
\pgfpathmoveto{\pgfqpoint{3.506399in}{2.007051in}}%
\pgfpathlineto{\pgfqpoint{3.506399in}{2.007051in}}%
\pgfpathlineto{\pgfqpoint{3.506399in}{2.010001in}}%
\pgfpathlineto{\pgfqpoint{3.510940in}{2.010001in}}%
\pgfpathlineto{\pgfqpoint{3.510940in}{2.007051in}}%
\pgfpathmoveto{\pgfqpoint{3.510940in}{2.007051in}}%
\pgfpathlineto{\pgfqpoint{3.510940in}{2.007051in}}%
\pgfpathlineto{\pgfqpoint{3.510940in}{2.010001in}}%
\pgfpathlineto{\pgfqpoint{3.515481in}{2.010001in}}%
\pgfpathlineto{\pgfqpoint{3.515481in}{2.007051in}}%
\pgfpathmoveto{\pgfqpoint{3.515481in}{2.007051in}}%
\pgfpathlineto{\pgfqpoint{3.515481in}{2.007051in}}%
\pgfpathlineto{\pgfqpoint{3.515481in}{2.010001in}}%
\pgfpathlineto{\pgfqpoint{3.520022in}{2.010001in}}%
\pgfpathlineto{\pgfqpoint{3.520022in}{2.007051in}}%
\pgfpathmoveto{\pgfqpoint{3.520022in}{2.007051in}}%
\pgfpathlineto{\pgfqpoint{3.520022in}{2.007051in}}%
\pgfpathlineto{\pgfqpoint{3.520022in}{2.010001in}}%
\pgfpathlineto{\pgfqpoint{3.524563in}{2.010001in}}%
\pgfpathlineto{\pgfqpoint{3.524563in}{2.007051in}}%
\pgfpathmoveto{\pgfqpoint{3.524563in}{2.007051in}}%
\pgfpathlineto{\pgfqpoint{3.524563in}{2.007051in}}%
\pgfpathlineto{\pgfqpoint{3.524563in}{2.010001in}}%
\pgfpathlineto{\pgfqpoint{3.529104in}{2.010001in}}%
\pgfpathlineto{\pgfqpoint{3.529104in}{2.007051in}}%
\pgfpathmoveto{\pgfqpoint{3.529104in}{2.007051in}}%
\pgfpathlineto{\pgfqpoint{3.529104in}{2.007051in}}%
\pgfpathlineto{\pgfqpoint{3.529104in}{2.010001in}}%
\pgfpathlineto{\pgfqpoint{3.533645in}{2.010001in}}%
\pgfpathlineto{\pgfqpoint{3.533645in}{2.007051in}}%
\pgfpathmoveto{\pgfqpoint{3.533645in}{2.007051in}}%
\pgfpathlineto{\pgfqpoint{3.533645in}{2.007051in}}%
\pgfpathlineto{\pgfqpoint{3.533645in}{2.010001in}}%
\pgfpathlineto{\pgfqpoint{3.538186in}{2.010001in}}%
\pgfpathlineto{\pgfqpoint{3.538186in}{2.007051in}}%
\pgfpathmoveto{\pgfqpoint{3.538186in}{2.007051in}}%
\pgfpathlineto{\pgfqpoint{3.538186in}{2.007051in}}%
\pgfpathlineto{\pgfqpoint{3.538186in}{2.010001in}}%
\pgfpathlineto{\pgfqpoint{3.542727in}{2.010001in}}%
\pgfpathlineto{\pgfqpoint{3.542727in}{2.007051in}}%
\pgfpathmoveto{\pgfqpoint{3.542727in}{2.007051in}}%
\pgfpathlineto{\pgfqpoint{3.542727in}{2.007051in}}%
\pgfpathlineto{\pgfqpoint{3.542727in}{2.010001in}}%
\pgfpathlineto{\pgfqpoint{3.547268in}{2.010001in}}%
\pgfpathlineto{\pgfqpoint{3.547268in}{2.007051in}}%
\pgfpathmoveto{\pgfqpoint{3.547268in}{2.007051in}}%
\pgfpathlineto{\pgfqpoint{3.547268in}{2.007051in}}%
\pgfpathlineto{\pgfqpoint{3.547268in}{2.010001in}}%
\pgfpathlineto{\pgfqpoint{3.551809in}{2.010001in}}%
\pgfpathlineto{\pgfqpoint{3.551809in}{2.007051in}}%
\pgfpathmoveto{\pgfqpoint{3.551809in}{2.007051in}}%
\pgfpathlineto{\pgfqpoint{3.551809in}{2.007051in}}%
\pgfpathlineto{\pgfqpoint{3.551809in}{2.010001in}}%
\pgfpathlineto{\pgfqpoint{3.556350in}{2.010001in}}%
\pgfpathlineto{\pgfqpoint{3.556350in}{2.007051in}}%
\pgfpathmoveto{\pgfqpoint{3.556350in}{2.007051in}}%
\pgfpathlineto{\pgfqpoint{3.556350in}{2.007051in}}%
\pgfpathlineto{\pgfqpoint{3.556350in}{2.010001in}}%
\pgfpathlineto{\pgfqpoint{3.560891in}{2.010001in}}%
\pgfpathlineto{\pgfqpoint{3.560891in}{2.007051in}}%
\pgfpathmoveto{\pgfqpoint{3.560891in}{2.007051in}}%
\pgfpathlineto{\pgfqpoint{3.560891in}{2.007051in}}%
\pgfpathlineto{\pgfqpoint{3.560891in}{2.010001in}}%
\pgfpathlineto{\pgfqpoint{3.565432in}{2.010001in}}%
\pgfpathlineto{\pgfqpoint{3.565432in}{2.007051in}}%
\pgfpathmoveto{\pgfqpoint{3.565432in}{2.007051in}}%
\pgfpathlineto{\pgfqpoint{3.565432in}{2.007051in}}%
\pgfpathlineto{\pgfqpoint{3.565432in}{2.010001in}}%
\pgfpathlineto{\pgfqpoint{3.569973in}{2.010001in}}%
\pgfpathlineto{\pgfqpoint{3.569973in}{2.007051in}}%
\pgfpathmoveto{\pgfqpoint{3.569973in}{2.007051in}}%
\pgfpathlineto{\pgfqpoint{3.569973in}{2.007051in}}%
\pgfpathlineto{\pgfqpoint{3.569973in}{2.010001in}}%
\pgfpathlineto{\pgfqpoint{3.574513in}{2.010001in}}%
\pgfpathlineto{\pgfqpoint{3.574513in}{2.007051in}}%
\pgfpathmoveto{\pgfqpoint{3.574513in}{2.007051in}}%
\pgfpathlineto{\pgfqpoint{3.574513in}{2.007051in}}%
\pgfpathlineto{\pgfqpoint{3.574513in}{2.010001in}}%
\pgfpathlineto{\pgfqpoint{3.579054in}{2.010001in}}%
\pgfpathlineto{\pgfqpoint{3.579054in}{2.007051in}}%
\pgfpathmoveto{\pgfqpoint{3.579054in}{2.007051in}}%
\pgfpathlineto{\pgfqpoint{3.579054in}{2.007051in}}%
\pgfpathlineto{\pgfqpoint{3.579054in}{2.010001in}}%
\pgfpathlineto{\pgfqpoint{3.583595in}{2.010001in}}%
\pgfpathlineto{\pgfqpoint{3.583595in}{2.007051in}}%
\pgfpathmoveto{\pgfqpoint{3.583595in}{2.007051in}}%
\pgfpathlineto{\pgfqpoint{3.583595in}{2.007051in}}%
\pgfpathlineto{\pgfqpoint{3.583595in}{2.010001in}}%
\pgfpathlineto{\pgfqpoint{3.588136in}{2.010001in}}%
\pgfpathlineto{\pgfqpoint{3.588136in}{2.007051in}}%
\pgfpathmoveto{\pgfqpoint{3.588136in}{2.007051in}}%
\pgfpathlineto{\pgfqpoint{3.588136in}{2.007051in}}%
\pgfpathlineto{\pgfqpoint{3.588136in}{2.010001in}}%
\pgfpathlineto{\pgfqpoint{3.592677in}{2.010001in}}%
\pgfpathlineto{\pgfqpoint{3.592677in}{2.007051in}}%
\pgfpathmoveto{\pgfqpoint{3.592677in}{2.007051in}}%
\pgfpathlineto{\pgfqpoint{3.592677in}{2.007051in}}%
\pgfpathlineto{\pgfqpoint{3.592677in}{2.010001in}}%
\pgfpathlineto{\pgfqpoint{3.597218in}{2.010001in}}%
\pgfpathlineto{\pgfqpoint{3.597218in}{2.007051in}}%
\pgfpathmoveto{\pgfqpoint{3.597218in}{2.007051in}}%
\pgfpathlineto{\pgfqpoint{3.597218in}{2.007051in}}%
\pgfpathlineto{\pgfqpoint{3.597218in}{2.010001in}}%
\pgfpathlineto{\pgfqpoint{3.601759in}{2.010001in}}%
\pgfpathlineto{\pgfqpoint{3.601759in}{2.007051in}}%
\pgfpathmoveto{\pgfqpoint{3.601759in}{2.007051in}}%
\pgfpathlineto{\pgfqpoint{3.601759in}{2.007051in}}%
\pgfpathlineto{\pgfqpoint{3.601759in}{2.010001in}}%
\pgfpathlineto{\pgfqpoint{3.606300in}{2.010001in}}%
\pgfpathlineto{\pgfqpoint{3.606300in}{2.007051in}}%
\pgfpathmoveto{\pgfqpoint{3.606300in}{2.007051in}}%
\pgfpathlineto{\pgfqpoint{3.606300in}{2.007051in}}%
\pgfpathlineto{\pgfqpoint{3.606300in}{2.010001in}}%
\pgfpathlineto{\pgfqpoint{3.610841in}{2.010001in}}%
\pgfpathlineto{\pgfqpoint{3.610841in}{2.007051in}}%
\pgfpathmoveto{\pgfqpoint{3.610841in}{2.007051in}}%
\pgfpathlineto{\pgfqpoint{3.610841in}{2.007051in}}%
\pgfpathlineto{\pgfqpoint{3.610841in}{2.010001in}}%
\pgfpathlineto{\pgfqpoint{3.615382in}{2.010001in}}%
\pgfpathlineto{\pgfqpoint{3.615382in}{2.007051in}}%
\pgfpathmoveto{\pgfqpoint{3.615382in}{2.007051in}}%
\pgfpathlineto{\pgfqpoint{3.615382in}{2.007051in}}%
\pgfpathlineto{\pgfqpoint{3.615382in}{2.010001in}}%
\pgfpathlineto{\pgfqpoint{3.619923in}{2.010001in}}%
\pgfpathlineto{\pgfqpoint{3.619923in}{2.007051in}}%
\pgfpathmoveto{\pgfqpoint{3.619923in}{2.007051in}}%
\pgfpathlineto{\pgfqpoint{3.619923in}{2.007051in}}%
\pgfpathlineto{\pgfqpoint{3.619923in}{2.010001in}}%
\pgfpathlineto{\pgfqpoint{3.624464in}{2.010001in}}%
\pgfpathlineto{\pgfqpoint{3.624464in}{2.007051in}}%
\pgfpathmoveto{\pgfqpoint{3.624464in}{2.007051in}}%
\pgfpathlineto{\pgfqpoint{3.624464in}{2.007051in}}%
\pgfpathlineto{\pgfqpoint{3.624464in}{2.010001in}}%
\pgfpathlineto{\pgfqpoint{3.629005in}{2.010001in}}%
\pgfpathlineto{\pgfqpoint{3.629005in}{2.007051in}}%
\pgfpathmoveto{\pgfqpoint{3.629005in}{2.007051in}}%
\pgfpathlineto{\pgfqpoint{3.629005in}{2.007051in}}%
\pgfpathlineto{\pgfqpoint{3.629005in}{2.010001in}}%
\pgfpathlineto{\pgfqpoint{3.633546in}{2.010001in}}%
\pgfpathlineto{\pgfqpoint{3.633546in}{2.007051in}}%
\pgfpathmoveto{\pgfqpoint{3.633546in}{2.007051in}}%
\pgfpathlineto{\pgfqpoint{3.633546in}{2.007051in}}%
\pgfpathlineto{\pgfqpoint{3.633546in}{2.010001in}}%
\pgfpathlineto{\pgfqpoint{3.638087in}{2.010001in}}%
\pgfpathlineto{\pgfqpoint{3.638087in}{2.007051in}}%
\pgfpathmoveto{\pgfqpoint{3.638087in}{2.007051in}}%
\pgfpathlineto{\pgfqpoint{3.638087in}{2.007051in}}%
\pgfpathlineto{\pgfqpoint{3.638087in}{2.010001in}}%
\pgfpathlineto{\pgfqpoint{3.642628in}{2.010001in}}%
\pgfpathlineto{\pgfqpoint{3.642628in}{2.007051in}}%
\pgfpathmoveto{\pgfqpoint{3.642628in}{2.007051in}}%
\pgfpathlineto{\pgfqpoint{3.642628in}{2.007051in}}%
\pgfpathlineto{\pgfqpoint{3.642628in}{2.010001in}}%
\pgfpathlineto{\pgfqpoint{3.647169in}{2.010001in}}%
\pgfpathlineto{\pgfqpoint{3.647169in}{2.007051in}}%
\pgfpathmoveto{\pgfqpoint{3.647169in}{2.007051in}}%
\pgfpathlineto{\pgfqpoint{3.647169in}{2.007051in}}%
\pgfpathlineto{\pgfqpoint{3.647169in}{2.010001in}}%
\pgfpathlineto{\pgfqpoint{3.651710in}{2.010001in}}%
\pgfpathlineto{\pgfqpoint{3.651710in}{2.007051in}}%
\pgfpathmoveto{\pgfqpoint{3.651710in}{2.007051in}}%
\pgfpathlineto{\pgfqpoint{3.651710in}{2.007051in}}%
\pgfpathlineto{\pgfqpoint{3.651710in}{2.010001in}}%
\pgfpathlineto{\pgfqpoint{3.656251in}{2.010001in}}%
\pgfpathlineto{\pgfqpoint{3.656251in}{2.007051in}}%
\pgfpathmoveto{\pgfqpoint{3.656251in}{2.007051in}}%
\pgfpathlineto{\pgfqpoint{3.656251in}{2.007051in}}%
\pgfpathlineto{\pgfqpoint{3.656251in}{2.010001in}}%
\pgfpathlineto{\pgfqpoint{3.660792in}{2.010001in}}%
\pgfpathlineto{\pgfqpoint{3.660792in}{2.007051in}}%
\pgfpathmoveto{\pgfqpoint{3.660792in}{2.007051in}}%
\pgfpathlineto{\pgfqpoint{3.660792in}{2.007051in}}%
\pgfpathlineto{\pgfqpoint{3.660792in}{2.010001in}}%
\pgfpathlineto{\pgfqpoint{3.665333in}{2.010001in}}%
\pgfpathlineto{\pgfqpoint{3.665333in}{2.007051in}}%
\pgfpathmoveto{\pgfqpoint{3.665333in}{2.007051in}}%
\pgfpathlineto{\pgfqpoint{3.665333in}{2.007051in}}%
\pgfpathlineto{\pgfqpoint{3.665333in}{2.010001in}}%
\pgfpathlineto{\pgfqpoint{3.669874in}{2.010001in}}%
\pgfpathlineto{\pgfqpoint{3.669874in}{2.007051in}}%
\pgfpathmoveto{\pgfqpoint{3.669874in}{2.007051in}}%
\pgfpathlineto{\pgfqpoint{3.669874in}{2.007051in}}%
\pgfpathlineto{\pgfqpoint{3.669874in}{2.010001in}}%
\pgfpathlineto{\pgfqpoint{3.674415in}{2.010001in}}%
\pgfpathlineto{\pgfqpoint{3.674415in}{2.007051in}}%
\pgfpathmoveto{\pgfqpoint{3.674415in}{2.007051in}}%
\pgfpathlineto{\pgfqpoint{3.674415in}{2.007051in}}%
\pgfpathlineto{\pgfqpoint{3.674415in}{2.010001in}}%
\pgfpathlineto{\pgfqpoint{3.678956in}{2.010001in}}%
\pgfpathlineto{\pgfqpoint{3.678956in}{2.007051in}}%
\pgfpathmoveto{\pgfqpoint{3.678956in}{2.007051in}}%
\pgfpathlineto{\pgfqpoint{3.678956in}{2.007051in}}%
\pgfpathlineto{\pgfqpoint{3.678956in}{2.010001in}}%
\pgfpathlineto{\pgfqpoint{3.683497in}{2.010001in}}%
\pgfpathlineto{\pgfqpoint{3.683497in}{2.007051in}}%
\pgfpathmoveto{\pgfqpoint{3.683497in}{2.007051in}}%
\pgfpathlineto{\pgfqpoint{3.683497in}{2.007051in}}%
\pgfpathlineto{\pgfqpoint{3.683497in}{2.010001in}}%
\pgfpathlineto{\pgfqpoint{3.688038in}{2.010001in}}%
\pgfpathlineto{\pgfqpoint{3.688038in}{2.007051in}}%
\pgfpathmoveto{\pgfqpoint{3.688038in}{2.007051in}}%
\pgfpathlineto{\pgfqpoint{3.688038in}{2.007051in}}%
\pgfpathlineto{\pgfqpoint{3.688038in}{2.010001in}}%
\pgfpathlineto{\pgfqpoint{3.692579in}{2.010001in}}%
\pgfpathlineto{\pgfqpoint{3.692579in}{2.007051in}}%
\pgfpathmoveto{\pgfqpoint{3.692579in}{2.007051in}}%
\pgfpathlineto{\pgfqpoint{3.692579in}{2.007051in}}%
\pgfpathlineto{\pgfqpoint{3.692579in}{2.010001in}}%
\pgfpathlineto{\pgfqpoint{3.697120in}{2.010001in}}%
\pgfpathlineto{\pgfqpoint{3.697120in}{2.007051in}}%
\pgfpathmoveto{\pgfqpoint{3.697120in}{2.007051in}}%
\pgfpathlineto{\pgfqpoint{3.697120in}{2.007051in}}%
\pgfpathlineto{\pgfqpoint{3.697120in}{2.010001in}}%
\pgfpathlineto{\pgfqpoint{3.701661in}{2.010001in}}%
\pgfpathlineto{\pgfqpoint{3.701661in}{2.007051in}}%
\pgfpathmoveto{\pgfqpoint{3.701661in}{2.007051in}}%
\pgfpathlineto{\pgfqpoint{3.701661in}{2.007051in}}%
\pgfpathlineto{\pgfqpoint{3.701661in}{2.010001in}}%
\pgfpathlineto{\pgfqpoint{3.706202in}{2.010001in}}%
\pgfpathlineto{\pgfqpoint{3.706202in}{2.007051in}}%
\pgfpathmoveto{\pgfqpoint{3.706202in}{2.007051in}}%
\pgfpathlineto{\pgfqpoint{3.706202in}{2.007051in}}%
\pgfpathlineto{\pgfqpoint{3.706202in}{2.010001in}}%
\pgfpathlineto{\pgfqpoint{3.710743in}{2.010001in}}%
\pgfpathlineto{\pgfqpoint{3.710743in}{2.007051in}}%
\pgfpathmoveto{\pgfqpoint{3.710743in}{2.007051in}}%
\pgfpathlineto{\pgfqpoint{3.710743in}{2.007051in}}%
\pgfpathlineto{\pgfqpoint{3.710743in}{2.010001in}}%
\pgfpathlineto{\pgfqpoint{3.715284in}{2.010001in}}%
\pgfpathlineto{\pgfqpoint{3.715284in}{2.007051in}}%
\pgfpathmoveto{\pgfqpoint{3.715284in}{2.007051in}}%
\pgfpathlineto{\pgfqpoint{3.715284in}{2.007051in}}%
\pgfpathlineto{\pgfqpoint{3.715284in}{2.010001in}}%
\pgfpathlineto{\pgfqpoint{3.719825in}{2.010001in}}%
\pgfpathlineto{\pgfqpoint{3.719825in}{2.007051in}}%
\pgfpathmoveto{\pgfqpoint{3.719825in}{2.007051in}}%
\pgfpathlineto{\pgfqpoint{3.719825in}{2.007051in}}%
\pgfpathlineto{\pgfqpoint{3.719825in}{2.010001in}}%
\pgfpathlineto{\pgfqpoint{3.724366in}{2.010001in}}%
\pgfpathlineto{\pgfqpoint{3.724366in}{2.007051in}}%
\pgfpathmoveto{\pgfqpoint{3.724366in}{2.007051in}}%
\pgfpathlineto{\pgfqpoint{3.724366in}{2.007051in}}%
\pgfpathlineto{\pgfqpoint{3.724366in}{2.010001in}}%
\pgfpathlineto{\pgfqpoint{3.728907in}{2.010001in}}%
\pgfpathlineto{\pgfqpoint{3.728907in}{2.007051in}}%
\pgfpathmoveto{\pgfqpoint{3.728907in}{2.007051in}}%
\pgfpathlineto{\pgfqpoint{3.728907in}{2.007051in}}%
\pgfpathlineto{\pgfqpoint{3.728907in}{2.010001in}}%
\pgfpathlineto{\pgfqpoint{3.733448in}{2.010001in}}%
\pgfpathlineto{\pgfqpoint{3.733448in}{2.007051in}}%
\pgfpathmoveto{\pgfqpoint{3.733448in}{2.007051in}}%
\pgfpathlineto{\pgfqpoint{3.733448in}{2.007051in}}%
\pgfpathlineto{\pgfqpoint{3.733448in}{2.010001in}}%
\pgfpathlineto{\pgfqpoint{3.737989in}{2.010001in}}%
\pgfpathlineto{\pgfqpoint{3.737989in}{2.007051in}}%
\pgfpathmoveto{\pgfqpoint{3.737989in}{2.007051in}}%
\pgfpathlineto{\pgfqpoint{3.737989in}{2.007051in}}%
\pgfpathlineto{\pgfqpoint{3.737989in}{2.010001in}}%
\pgfpathlineto{\pgfqpoint{3.742530in}{2.010001in}}%
\pgfpathlineto{\pgfqpoint{3.742530in}{2.007051in}}%
\pgfpathmoveto{\pgfqpoint{3.742530in}{2.007051in}}%
\pgfpathlineto{\pgfqpoint{3.742530in}{2.007051in}}%
\pgfpathlineto{\pgfqpoint{3.742530in}{2.010001in}}%
\pgfpathlineto{\pgfqpoint{3.747071in}{2.010001in}}%
\pgfpathlineto{\pgfqpoint{3.747071in}{2.007051in}}%
\pgfpathmoveto{\pgfqpoint{3.747071in}{2.007051in}}%
\pgfpathlineto{\pgfqpoint{3.747071in}{2.007051in}}%
\pgfpathlineto{\pgfqpoint{3.747071in}{2.010001in}}%
\pgfpathlineto{\pgfqpoint{3.751612in}{2.010001in}}%
\pgfpathlineto{\pgfqpoint{3.751612in}{2.007051in}}%
\pgfpathmoveto{\pgfqpoint{3.751612in}{2.007051in}}%
\pgfpathlineto{\pgfqpoint{3.751612in}{2.007051in}}%
\pgfpathlineto{\pgfqpoint{3.751612in}{2.010001in}}%
\pgfpathlineto{\pgfqpoint{3.756153in}{2.010001in}}%
\pgfpathlineto{\pgfqpoint{3.756153in}{2.007051in}}%
\pgfpathmoveto{\pgfqpoint{3.756153in}{2.007051in}}%
\pgfpathlineto{\pgfqpoint{3.756153in}{2.007051in}}%
\pgfpathlineto{\pgfqpoint{3.756153in}{2.010001in}}%
\pgfpathlineto{\pgfqpoint{3.760694in}{2.010001in}}%
\pgfpathlineto{\pgfqpoint{3.760694in}{2.007051in}}%
\pgfpathmoveto{\pgfqpoint{3.760694in}{2.007051in}}%
\pgfpathlineto{\pgfqpoint{3.760694in}{2.007051in}}%
\pgfpathlineto{\pgfqpoint{3.760694in}{2.010001in}}%
\pgfpathlineto{\pgfqpoint{3.765235in}{2.010001in}}%
\pgfpathlineto{\pgfqpoint{3.765235in}{2.007051in}}%
\pgfpathmoveto{\pgfqpoint{3.765235in}{2.007051in}}%
\pgfpathlineto{\pgfqpoint{3.765235in}{2.007051in}}%
\pgfpathlineto{\pgfqpoint{3.765235in}{2.010001in}}%
\pgfpathlineto{\pgfqpoint{3.769776in}{2.010001in}}%
\pgfpathlineto{\pgfqpoint{3.769776in}{2.007051in}}%
\pgfpathmoveto{\pgfqpoint{3.769776in}{2.007051in}}%
\pgfpathlineto{\pgfqpoint{3.769776in}{2.007051in}}%
\pgfpathlineto{\pgfqpoint{3.769776in}{2.010001in}}%
\pgfpathlineto{\pgfqpoint{3.774317in}{2.010001in}}%
\pgfpathlineto{\pgfqpoint{3.774317in}{2.007051in}}%
\pgfpathmoveto{\pgfqpoint{3.774317in}{2.007051in}}%
\pgfpathlineto{\pgfqpoint{3.774317in}{2.007051in}}%
\pgfpathlineto{\pgfqpoint{3.774317in}{2.010001in}}%
\pgfpathlineto{\pgfqpoint{3.778858in}{2.010001in}}%
\pgfpathlineto{\pgfqpoint{3.778858in}{2.007051in}}%
\pgfpathmoveto{\pgfqpoint{3.778858in}{2.007051in}}%
\pgfpathlineto{\pgfqpoint{3.778858in}{2.007051in}}%
\pgfpathlineto{\pgfqpoint{3.778858in}{2.010001in}}%
\pgfpathlineto{\pgfqpoint{3.783399in}{2.010001in}}%
\pgfpathlineto{\pgfqpoint{3.783399in}{2.007051in}}%
\pgfpathmoveto{\pgfqpoint{3.783399in}{2.007051in}}%
\pgfpathlineto{\pgfqpoint{3.783399in}{2.007051in}}%
\pgfpathlineto{\pgfqpoint{3.783399in}{2.010001in}}%
\pgfpathlineto{\pgfqpoint{3.787940in}{2.010001in}}%
\pgfpathlineto{\pgfqpoint{3.787940in}{2.007051in}}%
\pgfpathmoveto{\pgfqpoint{3.787940in}{2.007051in}}%
\pgfpathlineto{\pgfqpoint{3.787940in}{2.007051in}}%
\pgfpathlineto{\pgfqpoint{3.787940in}{2.010001in}}%
\pgfpathlineto{\pgfqpoint{3.792481in}{2.010001in}}%
\pgfpathlineto{\pgfqpoint{3.792481in}{2.007051in}}%
\pgfpathmoveto{\pgfqpoint{3.792481in}{2.007051in}}%
\pgfpathlineto{\pgfqpoint{3.792481in}{2.007051in}}%
\pgfpathlineto{\pgfqpoint{3.792481in}{2.010001in}}%
\pgfpathlineto{\pgfqpoint{3.797022in}{2.010001in}}%
\pgfpathlineto{\pgfqpoint{3.797022in}{2.007051in}}%
\pgfpathmoveto{\pgfqpoint{3.797022in}{2.007051in}}%
\pgfpathlineto{\pgfqpoint{3.797022in}{2.007051in}}%
\pgfpathlineto{\pgfqpoint{3.797022in}{2.010001in}}%
\pgfpathlineto{\pgfqpoint{3.801563in}{2.010001in}}%
\pgfpathlineto{\pgfqpoint{3.801563in}{2.007051in}}%
\pgfpathmoveto{\pgfqpoint{3.801563in}{2.007051in}}%
\pgfpathlineto{\pgfqpoint{3.801563in}{2.007051in}}%
\pgfpathlineto{\pgfqpoint{3.801563in}{2.010001in}}%
\pgfpathlineto{\pgfqpoint{3.806103in}{2.010001in}}%
\pgfpathlineto{\pgfqpoint{3.806103in}{2.007051in}}%
\pgfpathmoveto{\pgfqpoint{3.806103in}{2.007051in}}%
\pgfpathlineto{\pgfqpoint{3.806103in}{2.007051in}}%
\pgfpathlineto{\pgfqpoint{3.806103in}{2.010001in}}%
\pgfpathlineto{\pgfqpoint{3.810644in}{2.010001in}}%
\pgfpathlineto{\pgfqpoint{3.810644in}{2.007051in}}%
\pgfpathmoveto{\pgfqpoint{3.810644in}{2.007051in}}%
\pgfpathlineto{\pgfqpoint{3.810644in}{2.007051in}}%
\pgfpathlineto{\pgfqpoint{3.810644in}{2.010001in}}%
\pgfpathlineto{\pgfqpoint{3.815185in}{2.010001in}}%
\pgfpathlineto{\pgfqpoint{3.815185in}{2.007051in}}%
\pgfpathmoveto{\pgfqpoint{3.815185in}{2.007051in}}%
\pgfpathlineto{\pgfqpoint{3.815185in}{2.007051in}}%
\pgfpathlineto{\pgfqpoint{3.815185in}{2.010001in}}%
\pgfpathlineto{\pgfqpoint{3.819726in}{2.010001in}}%
\pgfpathlineto{\pgfqpoint{3.819726in}{2.007051in}}%
\pgfpathmoveto{\pgfqpoint{3.819726in}{2.007051in}}%
\pgfpathlineto{\pgfqpoint{3.819726in}{2.007051in}}%
\pgfpathlineto{\pgfqpoint{3.819726in}{2.010001in}}%
\pgfpathlineto{\pgfqpoint{3.824267in}{2.010001in}}%
\pgfpathlineto{\pgfqpoint{3.824267in}{2.007051in}}%
\pgfpathmoveto{\pgfqpoint{3.824267in}{2.007051in}}%
\pgfpathlineto{\pgfqpoint{3.824267in}{2.007051in}}%
\pgfpathlineto{\pgfqpoint{3.824267in}{2.010001in}}%
\pgfpathlineto{\pgfqpoint{3.828808in}{2.010001in}}%
\pgfpathlineto{\pgfqpoint{3.828808in}{2.007051in}}%
\pgfpathmoveto{\pgfqpoint{3.828808in}{2.007051in}}%
\pgfpathlineto{\pgfqpoint{3.828808in}{2.007051in}}%
\pgfpathlineto{\pgfqpoint{3.828808in}{2.010001in}}%
\pgfpathlineto{\pgfqpoint{3.833349in}{2.010001in}}%
\pgfpathlineto{\pgfqpoint{3.833349in}{2.007051in}}%
\pgfpathmoveto{\pgfqpoint{3.833349in}{2.007051in}}%
\pgfpathlineto{\pgfqpoint{3.833349in}{2.007051in}}%
\pgfpathlineto{\pgfqpoint{3.833349in}{2.010001in}}%
\pgfpathlineto{\pgfqpoint{3.837890in}{2.010001in}}%
\pgfpathlineto{\pgfqpoint{3.837890in}{2.007051in}}%
\pgfpathmoveto{\pgfqpoint{3.837890in}{2.007051in}}%
\pgfpathlineto{\pgfqpoint{3.837890in}{2.007051in}}%
\pgfpathlineto{\pgfqpoint{3.837890in}{2.010001in}}%
\pgfpathlineto{\pgfqpoint{3.842431in}{2.010001in}}%
\pgfpathlineto{\pgfqpoint{3.842431in}{2.007051in}}%
\pgfpathmoveto{\pgfqpoint{3.842431in}{2.007051in}}%
\pgfpathlineto{\pgfqpoint{3.842431in}{2.007051in}}%
\pgfpathlineto{\pgfqpoint{3.842431in}{2.010001in}}%
\pgfpathlineto{\pgfqpoint{3.846971in}{2.010001in}}%
\pgfpathlineto{\pgfqpoint{3.846971in}{2.007051in}}%
\pgfpathmoveto{\pgfqpoint{3.846971in}{2.007051in}}%
\pgfpathlineto{\pgfqpoint{3.846971in}{2.007051in}}%
\pgfpathlineto{\pgfqpoint{3.846971in}{2.010001in}}%
\pgfpathlineto{\pgfqpoint{3.851512in}{2.010001in}}%
\pgfpathlineto{\pgfqpoint{3.851512in}{2.007051in}}%
\pgfpathmoveto{\pgfqpoint{3.851512in}{2.007051in}}%
\pgfpathlineto{\pgfqpoint{3.851512in}{2.007051in}}%
\pgfpathlineto{\pgfqpoint{3.851512in}{2.010001in}}%
\pgfpathlineto{\pgfqpoint{3.856053in}{2.010001in}}%
\pgfpathlineto{\pgfqpoint{3.856053in}{2.007051in}}%
\pgfpathmoveto{\pgfqpoint{3.856053in}{2.007051in}}%
\pgfpathlineto{\pgfqpoint{3.856053in}{2.007051in}}%
\pgfpathlineto{\pgfqpoint{3.856053in}{2.010001in}}%
\pgfpathlineto{\pgfqpoint{3.860594in}{2.010001in}}%
\pgfpathlineto{\pgfqpoint{3.860594in}{2.007051in}}%
\pgfpathmoveto{\pgfqpoint{3.860594in}{2.007051in}}%
\pgfpathlineto{\pgfqpoint{3.860594in}{2.007051in}}%
\pgfpathlineto{\pgfqpoint{3.860594in}{2.010001in}}%
\pgfpathlineto{\pgfqpoint{3.865135in}{2.010001in}}%
\pgfpathlineto{\pgfqpoint{3.865135in}{2.007051in}}%
\pgfpathmoveto{\pgfqpoint{3.865135in}{2.007051in}}%
\pgfpathlineto{\pgfqpoint{3.865135in}{2.007051in}}%
\pgfpathlineto{\pgfqpoint{3.865135in}{2.010001in}}%
\pgfpathlineto{\pgfqpoint{3.869676in}{2.010001in}}%
\pgfpathlineto{\pgfqpoint{3.869676in}{2.007051in}}%
\pgfpathmoveto{\pgfqpoint{3.869676in}{2.007051in}}%
\pgfpathlineto{\pgfqpoint{3.869676in}{2.007051in}}%
\pgfpathlineto{\pgfqpoint{3.869676in}{2.010001in}}%
\pgfpathlineto{\pgfqpoint{3.874217in}{2.010001in}}%
\pgfpathlineto{\pgfqpoint{3.874217in}{2.007051in}}%
\pgfpathmoveto{\pgfqpoint{3.874217in}{2.007051in}}%
\pgfpathlineto{\pgfqpoint{3.874217in}{2.007051in}}%
\pgfpathlineto{\pgfqpoint{3.874217in}{2.010001in}}%
\pgfpathlineto{\pgfqpoint{3.878758in}{2.010001in}}%
\pgfpathlineto{\pgfqpoint{3.878758in}{2.007051in}}%
\pgfpathmoveto{\pgfqpoint{3.878758in}{2.007051in}}%
\pgfpathlineto{\pgfqpoint{3.878758in}{2.007051in}}%
\pgfpathlineto{\pgfqpoint{3.878758in}{2.010001in}}%
\pgfpathlineto{\pgfqpoint{3.883299in}{2.010001in}}%
\pgfpathlineto{\pgfqpoint{3.883299in}{2.007051in}}%
\pgfpathmoveto{\pgfqpoint{3.883299in}{2.007051in}}%
\pgfpathlineto{\pgfqpoint{3.883299in}{2.007051in}}%
\pgfpathlineto{\pgfqpoint{3.883299in}{2.010001in}}%
\pgfpathlineto{\pgfqpoint{3.887840in}{2.010001in}}%
\pgfpathlineto{\pgfqpoint{3.887840in}{2.007051in}}%
\pgfpathmoveto{\pgfqpoint{3.887840in}{2.007051in}}%
\pgfpathlineto{\pgfqpoint{3.887840in}{2.007051in}}%
\pgfpathlineto{\pgfqpoint{3.887840in}{2.010001in}}%
\pgfpathlineto{\pgfqpoint{3.892380in}{2.010001in}}%
\pgfpathlineto{\pgfqpoint{3.892380in}{2.007051in}}%
\pgfpathmoveto{\pgfqpoint{3.892380in}{2.007051in}}%
\pgfpathlineto{\pgfqpoint{3.892380in}{2.007051in}}%
\pgfpathlineto{\pgfqpoint{3.892380in}{2.010001in}}%
\pgfpathlineto{\pgfqpoint{3.896921in}{2.010001in}}%
\pgfpathlineto{\pgfqpoint{3.896921in}{2.007051in}}%
\pgfpathmoveto{\pgfqpoint{3.896921in}{2.007051in}}%
\pgfpathlineto{\pgfqpoint{3.896921in}{2.007051in}}%
\pgfpathlineto{\pgfqpoint{3.896921in}{2.010001in}}%
\pgfpathlineto{\pgfqpoint{3.901462in}{2.010001in}}%
\pgfpathlineto{\pgfqpoint{3.901462in}{2.007051in}}%
\pgfpathmoveto{\pgfqpoint{3.901462in}{2.007051in}}%
\pgfpathlineto{\pgfqpoint{3.901462in}{2.007051in}}%
\pgfpathlineto{\pgfqpoint{3.901462in}{2.010001in}}%
\pgfpathlineto{\pgfqpoint{3.906003in}{2.010001in}}%
\pgfpathlineto{\pgfqpoint{3.906003in}{2.007051in}}%
\pgfpathmoveto{\pgfqpoint{3.906003in}{2.007051in}}%
\pgfpathlineto{\pgfqpoint{3.906003in}{2.007051in}}%
\pgfpathlineto{\pgfqpoint{3.906003in}{2.010001in}}%
\pgfpathlineto{\pgfqpoint{3.910544in}{2.010001in}}%
\pgfpathlineto{\pgfqpoint{3.910544in}{2.007051in}}%
\pgfpathmoveto{\pgfqpoint{3.910544in}{2.007051in}}%
\pgfpathlineto{\pgfqpoint{3.910544in}{2.007051in}}%
\pgfpathlineto{\pgfqpoint{3.910544in}{2.010001in}}%
\pgfpathlineto{\pgfqpoint{3.915085in}{2.010001in}}%
\pgfpathlineto{\pgfqpoint{3.915085in}{2.007051in}}%
\pgfpathmoveto{\pgfqpoint{3.915085in}{2.007051in}}%
\pgfpathlineto{\pgfqpoint{3.915085in}{2.007051in}}%
\pgfpathlineto{\pgfqpoint{3.915085in}{2.010001in}}%
\pgfpathlineto{\pgfqpoint{3.919626in}{2.010001in}}%
\pgfpathlineto{\pgfqpoint{3.919626in}{2.007051in}}%
\pgfpathmoveto{\pgfqpoint{3.919626in}{2.007051in}}%
\pgfpathlineto{\pgfqpoint{3.919626in}{2.007051in}}%
\pgfpathlineto{\pgfqpoint{3.919626in}{2.010001in}}%
\pgfpathlineto{\pgfqpoint{3.924167in}{2.010001in}}%
\pgfpathlineto{\pgfqpoint{3.924167in}{2.007051in}}%
\pgfpathmoveto{\pgfqpoint{3.924167in}{2.007051in}}%
\pgfpathlineto{\pgfqpoint{3.924167in}{2.007051in}}%
\pgfpathlineto{\pgfqpoint{3.924167in}{2.010001in}}%
\pgfpathlineto{\pgfqpoint{3.928708in}{2.010001in}}%
\pgfpathlineto{\pgfqpoint{3.928708in}{2.007051in}}%
\pgfpathmoveto{\pgfqpoint{3.928708in}{2.007051in}}%
\pgfpathlineto{\pgfqpoint{3.928708in}{2.007051in}}%
\pgfpathlineto{\pgfqpoint{3.928708in}{2.010001in}}%
\pgfpathlineto{\pgfqpoint{3.933249in}{2.010001in}}%
\pgfpathlineto{\pgfqpoint{3.933249in}{2.007051in}}%
\pgfpathmoveto{\pgfqpoint{3.933249in}{2.007051in}}%
\pgfpathlineto{\pgfqpoint{3.933249in}{2.007051in}}%
\pgfpathlineto{\pgfqpoint{3.933249in}{2.010001in}}%
\pgfpathlineto{\pgfqpoint{3.937789in}{2.010001in}}%
\pgfpathlineto{\pgfqpoint{3.937789in}{2.007051in}}%
\pgfpathmoveto{\pgfqpoint{3.937789in}{2.007051in}}%
\pgfpathlineto{\pgfqpoint{3.937789in}{2.007051in}}%
\pgfpathlineto{\pgfqpoint{3.937789in}{2.010001in}}%
\pgfpathlineto{\pgfqpoint{3.942330in}{2.010001in}}%
\pgfpathlineto{\pgfqpoint{3.942330in}{2.007051in}}%
\pgfpathmoveto{\pgfqpoint{3.942330in}{2.007051in}}%
\pgfpathlineto{\pgfqpoint{3.942330in}{2.007051in}}%
\pgfpathlineto{\pgfqpoint{3.942330in}{2.010001in}}%
\pgfpathlineto{\pgfqpoint{3.946871in}{2.010001in}}%
\pgfpathlineto{\pgfqpoint{3.946871in}{2.007051in}}%
\pgfpathmoveto{\pgfqpoint{3.946871in}{2.007051in}}%
\pgfpathlineto{\pgfqpoint{3.946871in}{2.007051in}}%
\pgfpathlineto{\pgfqpoint{3.946871in}{2.010001in}}%
\pgfpathlineto{\pgfqpoint{3.951412in}{2.010001in}}%
\pgfpathlineto{\pgfqpoint{3.951412in}{2.007051in}}%
\pgfpathmoveto{\pgfqpoint{3.951412in}{2.007051in}}%
\pgfpathlineto{\pgfqpoint{3.951412in}{2.007051in}}%
\pgfpathlineto{\pgfqpoint{3.951412in}{2.010001in}}%
\pgfpathlineto{\pgfqpoint{3.955953in}{2.010001in}}%
\pgfpathlineto{\pgfqpoint{3.955953in}{2.007051in}}%
\pgfpathmoveto{\pgfqpoint{3.955953in}{2.007051in}}%
\pgfpathlineto{\pgfqpoint{3.955953in}{2.007051in}}%
\pgfpathlineto{\pgfqpoint{3.955953in}{2.010001in}}%
\pgfpathlineto{\pgfqpoint{3.960494in}{2.010001in}}%
\pgfpathlineto{\pgfqpoint{3.960494in}{2.007051in}}%
\pgfpathmoveto{\pgfqpoint{3.960494in}{2.007051in}}%
\pgfpathlineto{\pgfqpoint{3.960494in}{2.007051in}}%
\pgfpathlineto{\pgfqpoint{3.960494in}{2.010001in}}%
\pgfpathlineto{\pgfqpoint{3.965035in}{2.010001in}}%
\pgfpathlineto{\pgfqpoint{3.965035in}{2.007051in}}%
\pgfpathmoveto{\pgfqpoint{3.965035in}{2.007051in}}%
\pgfpathlineto{\pgfqpoint{3.965035in}{2.007051in}}%
\pgfpathlineto{\pgfqpoint{3.965035in}{2.010001in}}%
\pgfpathlineto{\pgfqpoint{3.969576in}{2.010001in}}%
\pgfpathlineto{\pgfqpoint{3.969576in}{2.007051in}}%
\pgfpathmoveto{\pgfqpoint{3.969576in}{2.007051in}}%
\pgfpathlineto{\pgfqpoint{3.969576in}{2.007051in}}%
\pgfpathlineto{\pgfqpoint{3.969576in}{2.010001in}}%
\pgfpathlineto{\pgfqpoint{3.974117in}{2.010001in}}%
\pgfpathlineto{\pgfqpoint{3.974117in}{2.007051in}}%
\pgfpathmoveto{\pgfqpoint{3.974117in}{2.007051in}}%
\pgfpathlineto{\pgfqpoint{3.974117in}{2.007051in}}%
\pgfpathlineto{\pgfqpoint{3.974117in}{2.010001in}}%
\pgfpathlineto{\pgfqpoint{3.978658in}{2.010001in}}%
\pgfpathlineto{\pgfqpoint{3.978658in}{2.007051in}}%
\pgfpathmoveto{\pgfqpoint{3.978658in}{2.007051in}}%
\pgfpathlineto{\pgfqpoint{3.978658in}{2.007051in}}%
\pgfpathlineto{\pgfqpoint{3.978658in}{2.010001in}}%
\pgfpathlineto{\pgfqpoint{3.983199in}{2.010001in}}%
\pgfpathlineto{\pgfqpoint{3.983199in}{2.007051in}}%
\pgfpathmoveto{\pgfqpoint{3.983199in}{2.007051in}}%
\pgfpathlineto{\pgfqpoint{3.983199in}{2.007051in}}%
\pgfpathlineto{\pgfqpoint{3.983199in}{2.010001in}}%
\pgfpathlineto{\pgfqpoint{3.987740in}{2.010001in}}%
\pgfpathlineto{\pgfqpoint{3.987740in}{2.007051in}}%
\pgfpathmoveto{\pgfqpoint{3.987740in}{2.007051in}}%
\pgfpathlineto{\pgfqpoint{3.987740in}{2.007051in}}%
\pgfpathlineto{\pgfqpoint{3.987740in}{2.010001in}}%
\pgfpathlineto{\pgfqpoint{3.992281in}{2.010001in}}%
\pgfpathlineto{\pgfqpoint{3.992281in}{2.007051in}}%
\pgfpathmoveto{\pgfqpoint{3.992281in}{2.007051in}}%
\pgfpathlineto{\pgfqpoint{3.992281in}{2.007051in}}%
\pgfpathlineto{\pgfqpoint{3.992281in}{2.010001in}}%
\pgfpathlineto{\pgfqpoint{3.996822in}{2.010001in}}%
\pgfpathlineto{\pgfqpoint{3.996822in}{2.007051in}}%
\pgfpathmoveto{\pgfqpoint{3.996822in}{2.007051in}}%
\pgfpathlineto{\pgfqpoint{3.996822in}{2.007051in}}%
\pgfpathlineto{\pgfqpoint{3.996822in}{2.010001in}}%
\pgfpathlineto{\pgfqpoint{4.001363in}{2.010001in}}%
\pgfpathlineto{\pgfqpoint{4.001363in}{2.007051in}}%
\pgfpathmoveto{\pgfqpoint{4.001363in}{2.007051in}}%
\pgfpathlineto{\pgfqpoint{4.001363in}{2.007051in}}%
\pgfpathlineto{\pgfqpoint{4.001363in}{2.010001in}}%
\pgfpathlineto{\pgfqpoint{4.005904in}{2.010001in}}%
\pgfpathlineto{\pgfqpoint{4.005904in}{2.007051in}}%
\pgfpathmoveto{\pgfqpoint{4.005904in}{2.007051in}}%
\pgfpathlineto{\pgfqpoint{4.005904in}{2.007051in}}%
\pgfpathlineto{\pgfqpoint{4.005904in}{2.010001in}}%
\pgfpathlineto{\pgfqpoint{4.010445in}{2.010001in}}%
\pgfpathlineto{\pgfqpoint{4.010445in}{2.007051in}}%
\pgfpathmoveto{\pgfqpoint{4.010445in}{2.007051in}}%
\pgfpathlineto{\pgfqpoint{4.010445in}{2.007051in}}%
\pgfpathlineto{\pgfqpoint{4.010445in}{2.010001in}}%
\pgfpathlineto{\pgfqpoint{4.014986in}{2.010001in}}%
\pgfpathlineto{\pgfqpoint{4.014986in}{2.007051in}}%
\pgfpathmoveto{\pgfqpoint{4.014986in}{2.007051in}}%
\pgfpathlineto{\pgfqpoint{4.014986in}{2.007051in}}%
\pgfpathlineto{\pgfqpoint{4.014986in}{2.010001in}}%
\pgfpathlineto{\pgfqpoint{4.019527in}{2.010001in}}%
\pgfpathlineto{\pgfqpoint{4.019527in}{2.007051in}}%
\pgfpathmoveto{\pgfqpoint{4.019527in}{2.007051in}}%
\pgfpathlineto{\pgfqpoint{4.019527in}{2.007051in}}%
\pgfpathlineto{\pgfqpoint{4.019527in}{2.010001in}}%
\pgfpathlineto{\pgfqpoint{4.024068in}{2.010001in}}%
\pgfpathlineto{\pgfqpoint{4.024068in}{2.007051in}}%
\pgfpathmoveto{\pgfqpoint{4.024068in}{2.007051in}}%
\pgfpathlineto{\pgfqpoint{4.024068in}{2.007051in}}%
\pgfpathlineto{\pgfqpoint{4.024068in}{2.010001in}}%
\pgfpathlineto{\pgfqpoint{4.028609in}{2.010001in}}%
\pgfpathlineto{\pgfqpoint{4.028609in}{2.007051in}}%
\pgfpathmoveto{\pgfqpoint{4.028609in}{2.007051in}}%
\pgfpathlineto{\pgfqpoint{4.028609in}{2.007051in}}%
\pgfpathlineto{\pgfqpoint{4.028609in}{2.010001in}}%
\pgfpathlineto{\pgfqpoint{4.033150in}{2.010001in}}%
\pgfpathlineto{\pgfqpoint{4.033150in}{2.007051in}}%
\pgfpathmoveto{\pgfqpoint{4.033150in}{2.007051in}}%
\pgfpathlineto{\pgfqpoint{4.033150in}{2.007051in}}%
\pgfpathlineto{\pgfqpoint{4.033150in}{2.010001in}}%
\pgfpathlineto{\pgfqpoint{4.037691in}{2.010001in}}%
\pgfpathlineto{\pgfqpoint{4.037691in}{2.007051in}}%
\pgfpathmoveto{\pgfqpoint{4.037691in}{2.007051in}}%
\pgfpathlineto{\pgfqpoint{4.037691in}{2.007051in}}%
\pgfpathlineto{\pgfqpoint{4.037691in}{2.010001in}}%
\pgfpathlineto{\pgfqpoint{4.042232in}{2.010001in}}%
\pgfpathlineto{\pgfqpoint{4.042232in}{2.007051in}}%
\pgfpathmoveto{\pgfqpoint{4.042232in}{2.007051in}}%
\pgfpathlineto{\pgfqpoint{4.042232in}{2.007051in}}%
\pgfpathlineto{\pgfqpoint{4.042232in}{2.010001in}}%
\pgfpathlineto{\pgfqpoint{4.046773in}{2.010001in}}%
\pgfpathlineto{\pgfqpoint{4.046773in}{2.007051in}}%
\pgfpathmoveto{\pgfqpoint{4.046773in}{2.007051in}}%
\pgfpathlineto{\pgfqpoint{4.046773in}{2.007051in}}%
\pgfpathlineto{\pgfqpoint{4.046773in}{2.010001in}}%
\pgfpathlineto{\pgfqpoint{4.051314in}{2.010001in}}%
\pgfpathlineto{\pgfqpoint{4.051314in}{2.007051in}}%
\pgfpathmoveto{\pgfqpoint{4.051314in}{2.007051in}}%
\pgfpathlineto{\pgfqpoint{4.051314in}{2.007051in}}%
\pgfpathlineto{\pgfqpoint{4.051314in}{2.010001in}}%
\pgfpathlineto{\pgfqpoint{4.055855in}{2.010001in}}%
\pgfpathlineto{\pgfqpoint{4.055855in}{2.007051in}}%
\pgfpathmoveto{\pgfqpoint{4.055855in}{2.007051in}}%
\pgfpathlineto{\pgfqpoint{4.055855in}{2.007051in}}%
\pgfpathlineto{\pgfqpoint{4.055855in}{2.010001in}}%
\pgfpathlineto{\pgfqpoint{4.060396in}{2.010001in}}%
\pgfpathlineto{\pgfqpoint{4.060396in}{2.007051in}}%
\pgfpathmoveto{\pgfqpoint{4.060396in}{2.007051in}}%
\pgfpathlineto{\pgfqpoint{4.060396in}{2.007051in}}%
\pgfpathlineto{\pgfqpoint{4.060396in}{2.010001in}}%
\pgfpathlineto{\pgfqpoint{4.064937in}{2.010001in}}%
\pgfpathlineto{\pgfqpoint{4.064937in}{2.007051in}}%
\pgfpathmoveto{\pgfqpoint{4.064937in}{2.007051in}}%
\pgfpathlineto{\pgfqpoint{4.064937in}{2.007051in}}%
\pgfpathlineto{\pgfqpoint{4.064937in}{2.010001in}}%
\pgfpathlineto{\pgfqpoint{4.069478in}{2.010001in}}%
\pgfpathlineto{\pgfqpoint{4.069478in}{2.007051in}}%
\pgfpathmoveto{\pgfqpoint{4.069478in}{2.007051in}}%
\pgfpathlineto{\pgfqpoint{4.069478in}{2.007051in}}%
\pgfpathlineto{\pgfqpoint{4.069478in}{2.010001in}}%
\pgfpathlineto{\pgfqpoint{4.074019in}{2.010001in}}%
\pgfpathlineto{\pgfqpoint{4.074019in}{2.007051in}}%
\pgfpathmoveto{\pgfqpoint{4.074019in}{2.007051in}}%
\pgfpathlineto{\pgfqpoint{4.074019in}{2.007051in}}%
\pgfpathlineto{\pgfqpoint{4.074019in}{2.010001in}}%
\pgfpathlineto{\pgfqpoint{4.078560in}{2.010001in}}%
\pgfpathlineto{\pgfqpoint{4.078560in}{2.007051in}}%
\pgfpathmoveto{\pgfqpoint{4.078560in}{2.007051in}}%
\pgfpathlineto{\pgfqpoint{4.078560in}{2.007051in}}%
\pgfpathlineto{\pgfqpoint{4.078560in}{2.010001in}}%
\pgfpathlineto{\pgfqpoint{4.083101in}{2.010001in}}%
\pgfpathlineto{\pgfqpoint{4.083101in}{2.007051in}}%
\pgfpathmoveto{\pgfqpoint{4.083101in}{2.007051in}}%
\pgfpathlineto{\pgfqpoint{4.083101in}{2.007051in}}%
\pgfpathlineto{\pgfqpoint{4.083101in}{2.010001in}}%
\pgfpathlineto{\pgfqpoint{4.087642in}{2.010001in}}%
\pgfpathlineto{\pgfqpoint{4.087642in}{2.007051in}}%
\pgfpathmoveto{\pgfqpoint{4.087642in}{2.007051in}}%
\pgfpathlineto{\pgfqpoint{4.087642in}{2.007051in}}%
\pgfpathlineto{\pgfqpoint{4.087642in}{2.010001in}}%
\pgfpathlineto{\pgfqpoint{4.092183in}{2.010001in}}%
\pgfpathlineto{\pgfqpoint{4.092183in}{2.007051in}}%
\pgfpathmoveto{\pgfqpoint{4.092183in}{2.007051in}}%
\pgfpathlineto{\pgfqpoint{4.092183in}{2.007051in}}%
\pgfpathlineto{\pgfqpoint{4.092183in}{2.010001in}}%
\pgfpathlineto{\pgfqpoint{4.096725in}{2.010001in}}%
\pgfpathlineto{\pgfqpoint{4.096725in}{2.007051in}}%
\pgfpathmoveto{\pgfqpoint{4.096725in}{2.007051in}}%
\pgfpathlineto{\pgfqpoint{4.096725in}{2.007051in}}%
\pgfpathlineto{\pgfqpoint{4.096725in}{2.010001in}}%
\pgfpathlineto{\pgfqpoint{4.101266in}{2.010001in}}%
\pgfpathlineto{\pgfqpoint{4.101266in}{2.007051in}}%
\pgfpathmoveto{\pgfqpoint{4.101266in}{2.007051in}}%
\pgfpathlineto{\pgfqpoint{4.101266in}{2.007051in}}%
\pgfpathlineto{\pgfqpoint{4.101266in}{2.010001in}}%
\pgfpathlineto{\pgfqpoint{4.105807in}{2.010001in}}%
\pgfpathlineto{\pgfqpoint{4.105807in}{2.007051in}}%
\pgfpathmoveto{\pgfqpoint{4.105807in}{2.007051in}}%
\pgfpathlineto{\pgfqpoint{4.105807in}{2.007051in}}%
\pgfpathlineto{\pgfqpoint{4.105807in}{2.010001in}}%
\pgfpathlineto{\pgfqpoint{4.110348in}{2.010001in}}%
\pgfpathlineto{\pgfqpoint{4.110348in}{2.007051in}}%
\pgfpathmoveto{\pgfqpoint{4.110348in}{2.007051in}}%
\pgfpathlineto{\pgfqpoint{4.110348in}{2.007051in}}%
\pgfpathlineto{\pgfqpoint{4.110348in}{2.010001in}}%
\pgfpathlineto{\pgfqpoint{4.114889in}{2.010001in}}%
\pgfpathlineto{\pgfqpoint{4.114889in}{2.007051in}}%
\pgfpathmoveto{\pgfqpoint{4.114889in}{2.007051in}}%
\pgfpathlineto{\pgfqpoint{4.114889in}{2.007051in}}%
\pgfpathlineto{\pgfqpoint{4.114889in}{2.010001in}}%
\pgfpathlineto{\pgfqpoint{4.119431in}{2.010001in}}%
\pgfpathlineto{\pgfqpoint{4.119431in}{2.007051in}}%
\pgfpathmoveto{\pgfqpoint{4.119431in}{2.007051in}}%
\pgfpathlineto{\pgfqpoint{4.119431in}{2.007051in}}%
\pgfpathlineto{\pgfqpoint{4.119431in}{2.010001in}}%
\pgfpathlineto{\pgfqpoint{4.123972in}{2.010001in}}%
\pgfpathlineto{\pgfqpoint{4.123972in}{2.007051in}}%
\pgfpathmoveto{\pgfqpoint{4.123972in}{2.007051in}}%
\pgfpathlineto{\pgfqpoint{4.123972in}{2.007051in}}%
\pgfpathlineto{\pgfqpoint{4.123972in}{2.010001in}}%
\pgfpathlineto{\pgfqpoint{4.128513in}{2.010001in}}%
\pgfpathlineto{\pgfqpoint{4.128513in}{2.007051in}}%
\pgfpathmoveto{\pgfqpoint{4.128513in}{2.007051in}}%
\pgfpathlineto{\pgfqpoint{4.128513in}{2.007051in}}%
\pgfpathlineto{\pgfqpoint{4.128513in}{2.010001in}}%
\pgfpathlineto{\pgfqpoint{4.133054in}{2.010001in}}%
\pgfpathlineto{\pgfqpoint{4.133054in}{2.007051in}}%
\pgfpathmoveto{\pgfqpoint{4.133054in}{2.007051in}}%
\pgfpathlineto{\pgfqpoint{4.133054in}{2.007051in}}%
\pgfpathlineto{\pgfqpoint{4.133054in}{2.010001in}}%
\pgfpathlineto{\pgfqpoint{4.137595in}{2.010001in}}%
\pgfpathlineto{\pgfqpoint{4.137595in}{2.007051in}}%
\pgfpathmoveto{\pgfqpoint{4.137595in}{2.007051in}}%
\pgfpathlineto{\pgfqpoint{4.137595in}{2.007051in}}%
\pgfpathlineto{\pgfqpoint{4.137595in}{2.010001in}}%
\pgfpathlineto{\pgfqpoint{4.142137in}{2.010001in}}%
\pgfpathlineto{\pgfqpoint{4.142137in}{2.007051in}}%
\pgfpathmoveto{\pgfqpoint{4.142137in}{2.007051in}}%
\pgfpathlineto{\pgfqpoint{4.142137in}{2.007051in}}%
\pgfpathlineto{\pgfqpoint{4.142137in}{2.010001in}}%
\pgfpathlineto{\pgfqpoint{4.146678in}{2.010001in}}%
\pgfpathlineto{\pgfqpoint{4.146678in}{2.007051in}}%
\pgfpathmoveto{\pgfqpoint{4.146678in}{2.007051in}}%
\pgfpathlineto{\pgfqpoint{4.146678in}{2.007051in}}%
\pgfpathlineto{\pgfqpoint{4.146678in}{2.010001in}}%
\pgfpathlineto{\pgfqpoint{4.151219in}{2.010001in}}%
\pgfpathlineto{\pgfqpoint{4.151219in}{2.007051in}}%
\pgfpathmoveto{\pgfqpoint{4.151219in}{2.007051in}}%
\pgfpathlineto{\pgfqpoint{4.151219in}{2.007051in}}%
\pgfpathlineto{\pgfqpoint{4.151219in}{2.010001in}}%
\pgfpathlineto{\pgfqpoint{4.155760in}{2.010001in}}%
\pgfpathlineto{\pgfqpoint{4.155760in}{2.007051in}}%
\pgfpathmoveto{\pgfqpoint{4.155760in}{2.007051in}}%
\pgfpathlineto{\pgfqpoint{4.155760in}{2.007051in}}%
\pgfpathlineto{\pgfqpoint{4.155760in}{2.010001in}}%
\pgfpathlineto{\pgfqpoint{4.160302in}{2.010001in}}%
\pgfpathlineto{\pgfqpoint{4.160302in}{2.007051in}}%
\pgfpathmoveto{\pgfqpoint{4.160302in}{2.007051in}}%
\pgfpathlineto{\pgfqpoint{4.160302in}{2.007051in}}%
\pgfpathlineto{\pgfqpoint{4.160302in}{2.010001in}}%
\pgfpathlineto{\pgfqpoint{4.164843in}{2.010001in}}%
\pgfpathlineto{\pgfqpoint{4.164843in}{2.007051in}}%
\pgfpathmoveto{\pgfqpoint{4.164843in}{2.007051in}}%
\pgfpathlineto{\pgfqpoint{4.164843in}{2.007051in}}%
\pgfpathlineto{\pgfqpoint{4.164843in}{2.010001in}}%
\pgfpathlineto{\pgfqpoint{4.169384in}{2.010001in}}%
\pgfpathlineto{\pgfqpoint{4.169384in}{2.007051in}}%
\pgfpathmoveto{\pgfqpoint{4.169384in}{2.007051in}}%
\pgfpathlineto{\pgfqpoint{4.169384in}{2.007051in}}%
\pgfpathlineto{\pgfqpoint{4.169384in}{2.010001in}}%
\pgfpathlineto{\pgfqpoint{4.173925in}{2.010001in}}%
\pgfpathlineto{\pgfqpoint{4.173925in}{2.007051in}}%
\pgfpathmoveto{\pgfqpoint{4.173925in}{2.007051in}}%
\pgfpathlineto{\pgfqpoint{4.173925in}{2.007051in}}%
\pgfpathlineto{\pgfqpoint{4.173925in}{2.010001in}}%
\pgfpathlineto{\pgfqpoint{4.178466in}{2.010001in}}%
\pgfpathlineto{\pgfqpoint{4.178466in}{2.007051in}}%
\pgfpathmoveto{\pgfqpoint{4.178466in}{2.007051in}}%
\pgfpathlineto{\pgfqpoint{4.178466in}{2.007051in}}%
\pgfpathlineto{\pgfqpoint{4.178466in}{2.010001in}}%
\pgfpathlineto{\pgfqpoint{4.183008in}{2.010001in}}%
\pgfpathlineto{\pgfqpoint{4.183008in}{2.007051in}}%
\pgfpathmoveto{\pgfqpoint{4.183008in}{2.007051in}}%
\pgfpathlineto{\pgfqpoint{4.183008in}{2.007051in}}%
\pgfpathlineto{\pgfqpoint{4.183008in}{2.010001in}}%
\pgfpathlineto{\pgfqpoint{4.187549in}{2.010001in}}%
\pgfpathlineto{\pgfqpoint{4.187549in}{2.007051in}}%
\pgfpathmoveto{\pgfqpoint{4.187549in}{2.007051in}}%
\pgfpathlineto{\pgfqpoint{4.187549in}{2.007051in}}%
\pgfpathlineto{\pgfqpoint{4.187549in}{2.010001in}}%
\pgfpathlineto{\pgfqpoint{4.192090in}{2.010001in}}%
\pgfpathlineto{\pgfqpoint{4.192090in}{2.007051in}}%
\pgfpathmoveto{\pgfqpoint{4.192090in}{2.007051in}}%
\pgfpathlineto{\pgfqpoint{4.192090in}{2.007051in}}%
\pgfpathlineto{\pgfqpoint{4.192090in}{2.010001in}}%
\pgfpathlineto{\pgfqpoint{4.196631in}{2.010001in}}%
\pgfpathlineto{\pgfqpoint{4.196631in}{2.007051in}}%
\pgfpathmoveto{\pgfqpoint{4.196631in}{2.007051in}}%
\pgfpathlineto{\pgfqpoint{4.196631in}{2.007051in}}%
\pgfpathlineto{\pgfqpoint{4.196631in}{2.010001in}}%
\pgfpathlineto{\pgfqpoint{4.201172in}{2.010001in}}%
\pgfpathlineto{\pgfqpoint{4.201172in}{2.007051in}}%
\pgfpathmoveto{\pgfqpoint{4.201172in}{2.007051in}}%
\pgfpathlineto{\pgfqpoint{4.201172in}{2.007051in}}%
\pgfpathlineto{\pgfqpoint{4.201172in}{2.010001in}}%
\pgfpathlineto{\pgfqpoint{4.205714in}{2.010001in}}%
\pgfpathlineto{\pgfqpoint{4.205714in}{2.007051in}}%
\pgfpathmoveto{\pgfqpoint{4.205714in}{2.007051in}}%
\pgfpathlineto{\pgfqpoint{4.205714in}{2.007051in}}%
\pgfpathlineto{\pgfqpoint{4.205714in}{2.010001in}}%
\pgfpathlineto{\pgfqpoint{4.210255in}{2.010001in}}%
\pgfpathlineto{\pgfqpoint{4.210255in}{2.007051in}}%
\pgfpathmoveto{\pgfqpoint{4.210255in}{2.007051in}}%
\pgfpathlineto{\pgfqpoint{4.210255in}{2.007051in}}%
\pgfpathlineto{\pgfqpoint{4.210255in}{2.010001in}}%
\pgfpathlineto{\pgfqpoint{4.214796in}{2.010001in}}%
\pgfpathlineto{\pgfqpoint{4.214796in}{2.007051in}}%
\pgfpathmoveto{\pgfqpoint{4.214796in}{2.007051in}}%
\pgfpathlineto{\pgfqpoint{4.214796in}{2.007051in}}%
\pgfpathlineto{\pgfqpoint{4.214796in}{2.010001in}}%
\pgfpathlineto{\pgfqpoint{4.219337in}{2.010001in}}%
\pgfpathlineto{\pgfqpoint{4.219337in}{2.007051in}}%
\pgfpathmoveto{\pgfqpoint{4.219337in}{2.007051in}}%
\pgfpathlineto{\pgfqpoint{4.219337in}{2.007051in}}%
\pgfpathlineto{\pgfqpoint{4.219337in}{2.010001in}}%
\pgfpathlineto{\pgfqpoint{4.223878in}{2.010001in}}%
\pgfpathlineto{\pgfqpoint{4.223878in}{2.007051in}}%
\pgfpathmoveto{\pgfqpoint{4.223878in}{2.007051in}}%
\pgfpathlineto{\pgfqpoint{4.223878in}{2.007051in}}%
\pgfpathlineto{\pgfqpoint{4.223878in}{2.010001in}}%
\pgfpathlineto{\pgfqpoint{4.228420in}{2.010001in}}%
\pgfpathlineto{\pgfqpoint{4.228420in}{2.007051in}}%
\pgfpathmoveto{\pgfqpoint{4.228420in}{2.007051in}}%
\pgfpathlineto{\pgfqpoint{4.228420in}{2.007051in}}%
\pgfpathlineto{\pgfqpoint{4.228420in}{2.010001in}}%
\pgfpathlineto{\pgfqpoint{4.232961in}{2.010001in}}%
\pgfpathlineto{\pgfqpoint{4.232961in}{2.007051in}}%
\pgfpathmoveto{\pgfqpoint{4.232961in}{2.007051in}}%
\pgfpathlineto{\pgfqpoint{4.232961in}{2.007051in}}%
\pgfpathlineto{\pgfqpoint{4.232961in}{2.010001in}}%
\pgfpathlineto{\pgfqpoint{4.237502in}{2.010001in}}%
\pgfpathlineto{\pgfqpoint{4.237502in}{2.007051in}}%
\pgfpathmoveto{\pgfqpoint{4.237502in}{2.007051in}}%
\pgfpathlineto{\pgfqpoint{4.237502in}{2.007051in}}%
\pgfpathlineto{\pgfqpoint{4.237502in}{2.010001in}}%
\pgfpathlineto{\pgfqpoint{4.242043in}{2.010001in}}%
\pgfpathlineto{\pgfqpoint{4.242043in}{2.007051in}}%
\pgfpathmoveto{\pgfqpoint{4.242043in}{2.007051in}}%
\pgfpathlineto{\pgfqpoint{4.242043in}{2.007051in}}%
\pgfpathlineto{\pgfqpoint{4.242043in}{2.010001in}}%
\pgfpathlineto{\pgfqpoint{4.246584in}{2.010001in}}%
\pgfpathlineto{\pgfqpoint{4.246584in}{2.007051in}}%
\pgfpathmoveto{\pgfqpoint{4.246584in}{2.007051in}}%
\pgfpathlineto{\pgfqpoint{4.246584in}{2.007051in}}%
\pgfpathlineto{\pgfqpoint{4.246584in}{2.010001in}}%
\pgfpathlineto{\pgfqpoint{4.251125in}{2.010001in}}%
\pgfpathlineto{\pgfqpoint{4.251125in}{2.007051in}}%
\pgfpathmoveto{\pgfqpoint{4.251125in}{2.007051in}}%
\pgfpathlineto{\pgfqpoint{4.251125in}{2.007051in}}%
\pgfpathlineto{\pgfqpoint{4.251125in}{2.010001in}}%
\pgfpathlineto{\pgfqpoint{4.255666in}{2.010001in}}%
\pgfpathlineto{\pgfqpoint{4.255666in}{2.007051in}}%
\pgfpathmoveto{\pgfqpoint{4.255666in}{2.007051in}}%
\pgfpathlineto{\pgfqpoint{4.255666in}{2.007051in}}%
\pgfpathlineto{\pgfqpoint{4.255666in}{2.010001in}}%
\pgfpathlineto{\pgfqpoint{4.260207in}{2.010001in}}%
\pgfpathlineto{\pgfqpoint{4.260207in}{2.007051in}}%
\pgfpathmoveto{\pgfqpoint{4.260207in}{2.007051in}}%
\pgfpathlineto{\pgfqpoint{4.260207in}{2.007051in}}%
\pgfpathlineto{\pgfqpoint{4.260207in}{2.010001in}}%
\pgfpathlineto{\pgfqpoint{4.264749in}{2.010001in}}%
\pgfpathlineto{\pgfqpoint{4.264749in}{2.007051in}}%
\pgfpathmoveto{\pgfqpoint{4.264749in}{2.007051in}}%
\pgfpathlineto{\pgfqpoint{4.264749in}{2.007051in}}%
\pgfpathlineto{\pgfqpoint{4.264749in}{2.010001in}}%
\pgfpathlineto{\pgfqpoint{4.269290in}{2.010001in}}%
\pgfpathlineto{\pgfqpoint{4.269290in}{2.007051in}}%
\pgfpathmoveto{\pgfqpoint{4.269290in}{2.007051in}}%
\pgfpathlineto{\pgfqpoint{4.269290in}{2.007051in}}%
\pgfpathlineto{\pgfqpoint{4.269290in}{2.010001in}}%
\pgfpathlineto{\pgfqpoint{4.273831in}{2.010001in}}%
\pgfpathlineto{\pgfqpoint{4.273831in}{2.007051in}}%
\pgfpathmoveto{\pgfqpoint{4.273831in}{2.007051in}}%
\pgfpathlineto{\pgfqpoint{4.273831in}{2.007051in}}%
\pgfpathlineto{\pgfqpoint{4.273831in}{2.010001in}}%
\pgfpathlineto{\pgfqpoint{4.278372in}{2.010001in}}%
\pgfpathlineto{\pgfqpoint{4.278372in}{2.007051in}}%
\pgfpathmoveto{\pgfqpoint{4.278372in}{2.007051in}}%
\pgfpathlineto{\pgfqpoint{4.278372in}{2.007051in}}%
\pgfpathlineto{\pgfqpoint{4.278372in}{2.010001in}}%
\pgfpathlineto{\pgfqpoint{4.282913in}{2.010001in}}%
\pgfpathlineto{\pgfqpoint{4.282913in}{2.007051in}}%
\pgfpathmoveto{\pgfqpoint{4.282913in}{2.007051in}}%
\pgfpathlineto{\pgfqpoint{4.282913in}{2.007051in}}%
\pgfpathlineto{\pgfqpoint{4.282913in}{2.010001in}}%
\pgfpathlineto{\pgfqpoint{4.287454in}{2.010001in}}%
\pgfpathlineto{\pgfqpoint{4.287454in}{2.007051in}}%
\pgfpathmoveto{\pgfqpoint{4.287454in}{2.007051in}}%
\pgfpathlineto{\pgfqpoint{4.287454in}{2.007051in}}%
\pgfpathlineto{\pgfqpoint{4.287454in}{2.010001in}}%
\pgfpathlineto{\pgfqpoint{4.291995in}{2.010001in}}%
\pgfpathlineto{\pgfqpoint{4.291995in}{2.007051in}}%
\pgfpathmoveto{\pgfqpoint{4.291995in}{2.007051in}}%
\pgfpathlineto{\pgfqpoint{4.291995in}{2.007051in}}%
\pgfpathlineto{\pgfqpoint{4.291995in}{2.010001in}}%
\pgfpathlineto{\pgfqpoint{4.296536in}{2.010001in}}%
\pgfpathlineto{\pgfqpoint{4.296536in}{2.007051in}}%
\pgfpathmoveto{\pgfqpoint{4.296536in}{2.007051in}}%
\pgfpathlineto{\pgfqpoint{4.296536in}{2.007051in}}%
\pgfpathlineto{\pgfqpoint{4.296536in}{2.010001in}}%
\pgfpathlineto{\pgfqpoint{4.301077in}{2.010001in}}%
\pgfpathlineto{\pgfqpoint{4.301077in}{2.007051in}}%
\pgfpathmoveto{\pgfqpoint{4.301077in}{2.007051in}}%
\pgfpathlineto{\pgfqpoint{4.301077in}{2.007051in}}%
\pgfpathlineto{\pgfqpoint{4.301077in}{2.010001in}}%
\pgfpathlineto{\pgfqpoint{4.305618in}{2.010001in}}%
\pgfpathlineto{\pgfqpoint{4.305618in}{2.007051in}}%
\pgfpathmoveto{\pgfqpoint{4.305618in}{2.007051in}}%
\pgfpathlineto{\pgfqpoint{4.305618in}{2.007051in}}%
\pgfpathlineto{\pgfqpoint{4.305618in}{2.010001in}}%
\pgfpathlineto{\pgfqpoint{4.310159in}{2.010001in}}%
\pgfpathlineto{\pgfqpoint{4.310159in}{2.007051in}}%
\pgfpathmoveto{\pgfqpoint{4.310159in}{2.007051in}}%
\pgfpathlineto{\pgfqpoint{4.310159in}{2.007051in}}%
\pgfpathlineto{\pgfqpoint{4.310159in}{2.010001in}}%
\pgfpathlineto{\pgfqpoint{4.314700in}{2.010001in}}%
\pgfpathlineto{\pgfqpoint{4.314700in}{2.007051in}}%
\pgfpathmoveto{\pgfqpoint{4.314700in}{2.007051in}}%
\pgfpathlineto{\pgfqpoint{4.314700in}{2.007051in}}%
\pgfpathlineto{\pgfqpoint{4.314700in}{2.010001in}}%
\pgfpathlineto{\pgfqpoint{4.319241in}{2.010001in}}%
\pgfpathlineto{\pgfqpoint{4.319241in}{2.007051in}}%
\pgfpathmoveto{\pgfqpoint{4.319241in}{2.007051in}}%
\pgfpathlineto{\pgfqpoint{4.319241in}{2.007051in}}%
\pgfpathlineto{\pgfqpoint{4.319241in}{2.010001in}}%
\pgfpathlineto{\pgfqpoint{4.323782in}{2.010001in}}%
\pgfpathlineto{\pgfqpoint{4.323782in}{2.007051in}}%
\pgfpathmoveto{\pgfqpoint{4.323782in}{2.007051in}}%
\pgfpathlineto{\pgfqpoint{4.323782in}{2.007051in}}%
\pgfpathlineto{\pgfqpoint{4.323782in}{2.010001in}}%
\pgfpathlineto{\pgfqpoint{4.328323in}{2.010001in}}%
\pgfpathlineto{\pgfqpoint{4.328323in}{2.007051in}}%
\pgfpathmoveto{\pgfqpoint{4.328323in}{2.007051in}}%
\pgfpathlineto{\pgfqpoint{4.328323in}{2.007051in}}%
\pgfpathlineto{\pgfqpoint{4.328323in}{2.010001in}}%
\pgfpathlineto{\pgfqpoint{4.332865in}{2.010001in}}%
\pgfpathlineto{\pgfqpoint{4.332865in}{2.007051in}}%
\pgfpathmoveto{\pgfqpoint{4.332865in}{2.007051in}}%
\pgfpathlineto{\pgfqpoint{4.332865in}{2.007051in}}%
\pgfpathlineto{\pgfqpoint{4.332865in}{2.010001in}}%
\pgfpathlineto{\pgfqpoint{4.337406in}{2.010001in}}%
\pgfpathlineto{\pgfqpoint{4.337406in}{2.007051in}}%
\pgfpathmoveto{\pgfqpoint{4.337406in}{2.007051in}}%
\pgfpathlineto{\pgfqpoint{4.337406in}{2.007051in}}%
\pgfpathlineto{\pgfqpoint{4.337406in}{2.010001in}}%
\pgfpathlineto{\pgfqpoint{4.341947in}{2.010001in}}%
\pgfpathlineto{\pgfqpoint{4.341947in}{2.007051in}}%
\pgfpathmoveto{\pgfqpoint{4.341947in}{2.007051in}}%
\pgfpathlineto{\pgfqpoint{4.341947in}{2.007051in}}%
\pgfpathlineto{\pgfqpoint{4.341947in}{2.010001in}}%
\pgfpathlineto{\pgfqpoint{4.346488in}{2.010001in}}%
\pgfpathlineto{\pgfqpoint{4.346488in}{2.007051in}}%
\pgfpathmoveto{\pgfqpoint{4.346488in}{2.007051in}}%
\pgfpathlineto{\pgfqpoint{4.346488in}{2.007051in}}%
\pgfpathlineto{\pgfqpoint{4.346488in}{2.010001in}}%
\pgfpathlineto{\pgfqpoint{4.351029in}{2.010001in}}%
\pgfpathlineto{\pgfqpoint{4.351029in}{2.007051in}}%
\pgfpathmoveto{\pgfqpoint{4.351029in}{2.007051in}}%
\pgfpathlineto{\pgfqpoint{4.351029in}{2.007051in}}%
\pgfpathlineto{\pgfqpoint{4.351029in}{2.010001in}}%
\pgfpathlineto{\pgfqpoint{4.355570in}{2.010001in}}%
\pgfpathlineto{\pgfqpoint{4.355570in}{2.007051in}}%
\pgfpathmoveto{\pgfqpoint{4.355570in}{2.007051in}}%
\pgfpathlineto{\pgfqpoint{4.355570in}{2.007051in}}%
\pgfpathlineto{\pgfqpoint{4.355570in}{2.010001in}}%
\pgfpathlineto{\pgfqpoint{4.360111in}{2.010001in}}%
\pgfpathlineto{\pgfqpoint{4.360111in}{2.007051in}}%
\pgfpathmoveto{\pgfqpoint{4.360111in}{2.007051in}}%
\pgfpathlineto{\pgfqpoint{4.360111in}{2.007051in}}%
\pgfpathlineto{\pgfqpoint{4.360111in}{2.010001in}}%
\pgfpathlineto{\pgfqpoint{4.364652in}{2.010001in}}%
\pgfpathlineto{\pgfqpoint{4.364652in}{2.007051in}}%
\pgfpathmoveto{\pgfqpoint{4.364652in}{2.007051in}}%
\pgfpathlineto{\pgfqpoint{4.364652in}{2.007051in}}%
\pgfpathlineto{\pgfqpoint{4.364652in}{2.010001in}}%
\pgfpathlineto{\pgfqpoint{4.369193in}{2.010001in}}%
\pgfpathlineto{\pgfqpoint{4.369193in}{2.007051in}}%
\pgfpathmoveto{\pgfqpoint{4.369193in}{2.007051in}}%
\pgfpathlineto{\pgfqpoint{4.369193in}{2.007051in}}%
\pgfpathlineto{\pgfqpoint{4.369193in}{2.010001in}}%
\pgfpathlineto{\pgfqpoint{4.373734in}{2.010001in}}%
\pgfpathlineto{\pgfqpoint{4.373734in}{2.007051in}}%
\pgfpathmoveto{\pgfqpoint{4.373734in}{2.007051in}}%
\pgfpathlineto{\pgfqpoint{4.373734in}{2.007051in}}%
\pgfpathlineto{\pgfqpoint{4.373734in}{2.010001in}}%
\pgfpathlineto{\pgfqpoint{4.378275in}{2.010001in}}%
\pgfpathlineto{\pgfqpoint{4.378275in}{2.007051in}}%
\pgfpathmoveto{\pgfqpoint{4.378275in}{2.007051in}}%
\pgfpathlineto{\pgfqpoint{4.378275in}{2.007051in}}%
\pgfpathlineto{\pgfqpoint{4.378275in}{2.010001in}}%
\pgfpathlineto{\pgfqpoint{4.382816in}{2.010001in}}%
\pgfpathlineto{\pgfqpoint{4.382816in}{2.007051in}}%
\pgfpathmoveto{\pgfqpoint{4.382816in}{2.007051in}}%
\pgfpathlineto{\pgfqpoint{4.382816in}{2.007051in}}%
\pgfpathlineto{\pgfqpoint{4.382816in}{2.010001in}}%
\pgfpathlineto{\pgfqpoint{4.387357in}{2.010001in}}%
\pgfpathlineto{\pgfqpoint{4.387357in}{2.007051in}}%
\pgfpathmoveto{\pgfqpoint{4.387357in}{2.007051in}}%
\pgfpathlineto{\pgfqpoint{4.387357in}{2.007051in}}%
\pgfpathlineto{\pgfqpoint{4.387357in}{2.010001in}}%
\pgfpathlineto{\pgfqpoint{4.391898in}{2.010001in}}%
\pgfpathlineto{\pgfqpoint{4.391898in}{2.007051in}}%
\pgfpathmoveto{\pgfqpoint{4.391898in}{2.007051in}}%
\pgfpathlineto{\pgfqpoint{4.391898in}{2.007051in}}%
\pgfpathlineto{\pgfqpoint{4.391898in}{2.010001in}}%
\pgfpathlineto{\pgfqpoint{4.396439in}{2.010001in}}%
\pgfpathlineto{\pgfqpoint{4.396439in}{2.007051in}}%
\pgfpathmoveto{\pgfqpoint{4.396439in}{2.007051in}}%
\pgfpathlineto{\pgfqpoint{4.396439in}{2.007051in}}%
\pgfpathlineto{\pgfqpoint{4.396439in}{2.010001in}}%
\pgfpathlineto{\pgfqpoint{4.400980in}{2.010001in}}%
\pgfpathlineto{\pgfqpoint{4.400980in}{2.007051in}}%
\pgfpathmoveto{\pgfqpoint{4.400980in}{2.007051in}}%
\pgfpathlineto{\pgfqpoint{4.400980in}{2.007051in}}%
\pgfpathlineto{\pgfqpoint{4.400980in}{2.010001in}}%
\pgfpathlineto{\pgfqpoint{4.405521in}{2.010001in}}%
\pgfpathlineto{\pgfqpoint{4.405521in}{2.007051in}}%
\pgfpathmoveto{\pgfqpoint{4.405521in}{2.007051in}}%
\pgfpathlineto{\pgfqpoint{4.405521in}{2.007051in}}%
\pgfpathlineto{\pgfqpoint{4.405521in}{2.010001in}}%
\pgfpathlineto{\pgfqpoint{4.410061in}{2.010001in}}%
\pgfpathlineto{\pgfqpoint{4.410061in}{2.007051in}}%
\pgfpathmoveto{\pgfqpoint{4.410061in}{2.007051in}}%
\pgfpathlineto{\pgfqpoint{4.410061in}{2.007051in}}%
\pgfpathlineto{\pgfqpoint{4.410061in}{2.010001in}}%
\pgfpathlineto{\pgfqpoint{4.414602in}{2.010001in}}%
\pgfpathlineto{\pgfqpoint{4.414602in}{2.007051in}}%
\pgfpathmoveto{\pgfqpoint{4.414602in}{2.007051in}}%
\pgfpathlineto{\pgfqpoint{4.414602in}{2.007051in}}%
\pgfpathlineto{\pgfqpoint{4.414602in}{2.010001in}}%
\pgfpathlineto{\pgfqpoint{4.419143in}{2.010001in}}%
\pgfpathlineto{\pgfqpoint{4.419143in}{2.007051in}}%
\pgfpathmoveto{\pgfqpoint{4.419143in}{2.007051in}}%
\pgfpathlineto{\pgfqpoint{4.419143in}{2.007051in}}%
\pgfpathlineto{\pgfqpoint{4.419143in}{2.010001in}}%
\pgfpathlineto{\pgfqpoint{4.423684in}{2.010001in}}%
\pgfpathlineto{\pgfqpoint{4.423684in}{2.007051in}}%
\pgfpathmoveto{\pgfqpoint{4.423684in}{2.007051in}}%
\pgfpathlineto{\pgfqpoint{4.423684in}{2.007051in}}%
\pgfpathlineto{\pgfqpoint{4.423684in}{2.010001in}}%
\pgfpathlineto{\pgfqpoint{4.428225in}{2.010001in}}%
\pgfpathlineto{\pgfqpoint{4.428225in}{2.007051in}}%
\pgfpathmoveto{\pgfqpoint{4.428225in}{2.007051in}}%
\pgfpathlineto{\pgfqpoint{4.428225in}{2.007051in}}%
\pgfpathlineto{\pgfqpoint{4.428225in}{2.010001in}}%
\pgfpathlineto{\pgfqpoint{4.432766in}{2.010001in}}%
\pgfpathlineto{\pgfqpoint{4.432766in}{2.007051in}}%
\pgfpathmoveto{\pgfqpoint{4.432766in}{2.007051in}}%
\pgfpathlineto{\pgfqpoint{4.432766in}{2.007051in}}%
\pgfpathlineto{\pgfqpoint{4.432766in}{2.010001in}}%
\pgfpathlineto{\pgfqpoint{4.437306in}{2.010001in}}%
\pgfpathlineto{\pgfqpoint{4.437306in}{2.007051in}}%
\pgfpathmoveto{\pgfqpoint{4.437306in}{2.007051in}}%
\pgfpathlineto{\pgfqpoint{4.437306in}{2.007051in}}%
\pgfpathlineto{\pgfqpoint{4.437306in}{2.010001in}}%
\pgfpathlineto{\pgfqpoint{4.441847in}{2.010001in}}%
\pgfpathlineto{\pgfqpoint{4.441847in}{2.007051in}}%
\pgfpathmoveto{\pgfqpoint{4.441847in}{2.007051in}}%
\pgfpathlineto{\pgfqpoint{4.441847in}{2.007051in}}%
\pgfpathlineto{\pgfqpoint{4.441847in}{2.010001in}}%
\pgfpathlineto{\pgfqpoint{4.446388in}{2.010001in}}%
\pgfpathlineto{\pgfqpoint{4.446388in}{2.007051in}}%
\pgfpathmoveto{\pgfqpoint{4.446388in}{2.007051in}}%
\pgfpathlineto{\pgfqpoint{4.446388in}{2.007051in}}%
\pgfpathlineto{\pgfqpoint{4.446388in}{2.010001in}}%
\pgfpathlineto{\pgfqpoint{4.450929in}{2.010001in}}%
\pgfpathlineto{\pgfqpoint{4.450929in}{2.007051in}}%
\pgfpathmoveto{\pgfqpoint{4.450929in}{2.007051in}}%
\pgfpathlineto{\pgfqpoint{4.450929in}{2.007051in}}%
\pgfpathlineto{\pgfqpoint{4.450929in}{2.010001in}}%
\pgfpathlineto{\pgfqpoint{4.455470in}{2.010001in}}%
\pgfpathlineto{\pgfqpoint{4.455470in}{2.007051in}}%
\pgfpathmoveto{\pgfqpoint{4.455470in}{2.007051in}}%
\pgfpathlineto{\pgfqpoint{4.455470in}{2.007051in}}%
\pgfpathlineto{\pgfqpoint{4.455470in}{2.010001in}}%
\pgfpathlineto{\pgfqpoint{4.460011in}{2.010001in}}%
\pgfpathlineto{\pgfqpoint{4.460011in}{2.007051in}}%
\pgfpathmoveto{\pgfqpoint{4.460011in}{2.007051in}}%
\pgfpathlineto{\pgfqpoint{4.460011in}{2.007051in}}%
\pgfpathlineto{\pgfqpoint{4.460011in}{2.010001in}}%
\pgfpathlineto{\pgfqpoint{4.464552in}{2.010001in}}%
\pgfpathlineto{\pgfqpoint{4.464552in}{2.007051in}}%
\pgfpathmoveto{\pgfqpoint{4.464552in}{2.007051in}}%
\pgfpathlineto{\pgfqpoint{4.464552in}{2.007051in}}%
\pgfpathlineto{\pgfqpoint{4.464552in}{2.010001in}}%
\pgfpathlineto{\pgfqpoint{4.469092in}{2.010001in}}%
\pgfpathlineto{\pgfqpoint{4.469092in}{2.007051in}}%
\pgfpathmoveto{\pgfqpoint{4.469092in}{2.007051in}}%
\pgfpathlineto{\pgfqpoint{4.469092in}{2.007051in}}%
\pgfpathlineto{\pgfqpoint{4.469092in}{2.010001in}}%
\pgfpathlineto{\pgfqpoint{4.473633in}{2.010001in}}%
\pgfpathlineto{\pgfqpoint{4.473633in}{2.007051in}}%
\pgfpathmoveto{\pgfqpoint{4.473633in}{2.007051in}}%
\pgfpathlineto{\pgfqpoint{4.473633in}{2.007051in}}%
\pgfpathlineto{\pgfqpoint{4.473633in}{2.010001in}}%
\pgfpathlineto{\pgfqpoint{4.478174in}{2.010001in}}%
\pgfpathlineto{\pgfqpoint{4.478174in}{2.007051in}}%
\pgfpathmoveto{\pgfqpoint{4.478174in}{2.007051in}}%
\pgfpathlineto{\pgfqpoint{4.478174in}{2.007051in}}%
\pgfpathlineto{\pgfqpoint{4.478174in}{2.010001in}}%
\pgfpathlineto{\pgfqpoint{4.482715in}{2.010001in}}%
\pgfpathlineto{\pgfqpoint{4.482715in}{2.007051in}}%
\pgfpathmoveto{\pgfqpoint{4.482715in}{2.007051in}}%
\pgfpathlineto{\pgfqpoint{4.482715in}{2.007051in}}%
\pgfpathlineto{\pgfqpoint{4.482715in}{2.010001in}}%
\pgfpathlineto{\pgfqpoint{4.487256in}{2.010001in}}%
\pgfpathlineto{\pgfqpoint{4.487256in}{2.007051in}}%
\pgfpathmoveto{\pgfqpoint{4.487256in}{2.007051in}}%
\pgfpathlineto{\pgfqpoint{4.487256in}{2.007051in}}%
\pgfpathlineto{\pgfqpoint{4.487256in}{2.010001in}}%
\pgfpathlineto{\pgfqpoint{4.491797in}{2.010001in}}%
\pgfpathlineto{\pgfqpoint{4.491797in}{2.007051in}}%
\pgfpathmoveto{\pgfqpoint{4.491797in}{2.007051in}}%
\pgfpathlineto{\pgfqpoint{4.491797in}{2.007051in}}%
\pgfpathlineto{\pgfqpoint{4.491797in}{2.010001in}}%
\pgfpathlineto{\pgfqpoint{4.496337in}{2.010001in}}%
\pgfpathlineto{\pgfqpoint{4.496337in}{2.007051in}}%
\pgfpathmoveto{\pgfqpoint{4.496337in}{2.007051in}}%
\pgfpathlineto{\pgfqpoint{4.496337in}{2.007051in}}%
\pgfpathlineto{\pgfqpoint{4.496337in}{2.010001in}}%
\pgfpathlineto{\pgfqpoint{4.500878in}{2.010001in}}%
\pgfpathlineto{\pgfqpoint{4.500878in}{2.007051in}}%
\pgfpathmoveto{\pgfqpoint{4.500878in}{2.007051in}}%
\pgfpathlineto{\pgfqpoint{4.500878in}{2.007051in}}%
\pgfpathlineto{\pgfqpoint{4.500878in}{2.010001in}}%
\pgfpathlineto{\pgfqpoint{4.505419in}{2.010001in}}%
\pgfpathlineto{\pgfqpoint{4.505419in}{2.007051in}}%
\pgfpathmoveto{\pgfqpoint{4.505419in}{2.007051in}}%
\pgfpathlineto{\pgfqpoint{4.505419in}{2.007051in}}%
\pgfpathlineto{\pgfqpoint{4.505419in}{2.010001in}}%
\pgfpathlineto{\pgfqpoint{4.509960in}{2.010001in}}%
\pgfpathlineto{\pgfqpoint{4.509960in}{2.007051in}}%
\pgfpathmoveto{\pgfqpoint{4.509960in}{2.007051in}}%
\pgfpathlineto{\pgfqpoint{4.509960in}{2.007051in}}%
\pgfpathlineto{\pgfqpoint{4.509960in}{2.010001in}}%
\pgfpathlineto{\pgfqpoint{4.514501in}{2.010001in}}%
\pgfpathlineto{\pgfqpoint{4.514501in}{2.007051in}}%
\pgfpathmoveto{\pgfqpoint{4.514501in}{2.007051in}}%
\pgfpathlineto{\pgfqpoint{4.514501in}{2.007051in}}%
\pgfpathlineto{\pgfqpoint{4.514501in}{2.010001in}}%
\pgfpathlineto{\pgfqpoint{4.519042in}{2.010001in}}%
\pgfpathlineto{\pgfqpoint{4.519042in}{2.007051in}}%
\pgfpathmoveto{\pgfqpoint{4.519042in}{2.007051in}}%
\pgfpathlineto{\pgfqpoint{4.519042in}{2.007051in}}%
\pgfpathlineto{\pgfqpoint{4.519042in}{2.010001in}}%
\pgfpathlineto{\pgfqpoint{4.523583in}{2.010001in}}%
\pgfpathlineto{\pgfqpoint{4.523583in}{2.007051in}}%
\pgfpathmoveto{\pgfqpoint{4.523583in}{2.007051in}}%
\pgfpathlineto{\pgfqpoint{4.523583in}{2.007051in}}%
\pgfpathlineto{\pgfqpoint{4.523583in}{2.010001in}}%
\pgfpathlineto{\pgfqpoint{4.528123in}{2.010001in}}%
\pgfpathlineto{\pgfqpoint{4.528123in}{2.007051in}}%
\pgfpathmoveto{\pgfqpoint{4.528123in}{2.007051in}}%
\pgfpathlineto{\pgfqpoint{4.528123in}{2.007051in}}%
\pgfpathlineto{\pgfqpoint{4.528123in}{2.010001in}}%
\pgfpathlineto{\pgfqpoint{4.532665in}{2.010001in}}%
\pgfpathlineto{\pgfqpoint{4.532665in}{2.007051in}}%
\pgfpathmoveto{\pgfqpoint{4.532665in}{2.007051in}}%
\pgfpathlineto{\pgfqpoint{4.532665in}{2.007051in}}%
\pgfpathlineto{\pgfqpoint{4.532665in}{2.010001in}}%
\pgfpathlineto{\pgfqpoint{4.537206in}{2.010001in}}%
\pgfpathlineto{\pgfqpoint{4.537206in}{2.007051in}}%
\pgfpathmoveto{\pgfqpoint{4.537206in}{2.007051in}}%
\pgfpathlineto{\pgfqpoint{4.537206in}{2.007051in}}%
\pgfpathlineto{\pgfqpoint{4.537206in}{2.010001in}}%
\pgfpathlineto{\pgfqpoint{4.541747in}{2.010001in}}%
\pgfpathlineto{\pgfqpoint{4.541747in}{2.007051in}}%
\pgfpathmoveto{\pgfqpoint{4.541747in}{2.007051in}}%
\pgfpathlineto{\pgfqpoint{4.541747in}{2.007051in}}%
\pgfpathlineto{\pgfqpoint{4.541747in}{2.010001in}}%
\pgfpathlineto{\pgfqpoint{4.546288in}{2.010001in}}%
\pgfpathlineto{\pgfqpoint{4.546288in}{2.007051in}}%
\pgfpathmoveto{\pgfqpoint{4.546288in}{2.007051in}}%
\pgfpathlineto{\pgfqpoint{4.546288in}{2.007051in}}%
\pgfpathlineto{\pgfqpoint{4.546288in}{2.010001in}}%
\pgfpathlineto{\pgfqpoint{4.550829in}{2.010001in}}%
\pgfpathlineto{\pgfqpoint{4.550829in}{2.007051in}}%
\pgfpathmoveto{\pgfqpoint{4.550829in}{2.007051in}}%
\pgfpathlineto{\pgfqpoint{4.550829in}{2.007051in}}%
\pgfpathlineto{\pgfqpoint{4.550829in}{2.010001in}}%
\pgfpathlineto{\pgfqpoint{4.555370in}{2.010001in}}%
\pgfpathlineto{\pgfqpoint{4.555370in}{2.007051in}}%
\pgfpathmoveto{\pgfqpoint{4.555370in}{2.007051in}}%
\pgfpathlineto{\pgfqpoint{4.555370in}{2.007051in}}%
\pgfpathlineto{\pgfqpoint{4.555370in}{2.010001in}}%
\pgfpathlineto{\pgfqpoint{4.559911in}{2.010001in}}%
\pgfpathlineto{\pgfqpoint{4.559911in}{2.007051in}}%
\pgfpathmoveto{\pgfqpoint{4.559911in}{2.007051in}}%
\pgfpathlineto{\pgfqpoint{4.559911in}{2.007051in}}%
\pgfpathlineto{\pgfqpoint{4.559911in}{2.010001in}}%
\pgfpathlineto{\pgfqpoint{4.564452in}{2.010001in}}%
\pgfpathlineto{\pgfqpoint{4.564452in}{2.007051in}}%
\pgfpathmoveto{\pgfqpoint{4.564452in}{2.007051in}}%
\pgfpathlineto{\pgfqpoint{4.564452in}{2.007051in}}%
\pgfpathlineto{\pgfqpoint{4.564452in}{2.010001in}}%
\pgfpathlineto{\pgfqpoint{4.568993in}{2.010001in}}%
\pgfpathlineto{\pgfqpoint{4.568993in}{2.007051in}}%
\pgfpathmoveto{\pgfqpoint{4.568993in}{2.007051in}}%
\pgfpathlineto{\pgfqpoint{4.568993in}{2.007051in}}%
\pgfpathlineto{\pgfqpoint{4.568993in}{2.010001in}}%
\pgfpathlineto{\pgfqpoint{4.573535in}{2.010001in}}%
\pgfpathlineto{\pgfqpoint{4.573535in}{2.007051in}}%
\pgfpathmoveto{\pgfqpoint{4.573535in}{2.007051in}}%
\pgfpathlineto{\pgfqpoint{4.573535in}{2.007051in}}%
\pgfpathlineto{\pgfqpoint{4.573535in}{2.010001in}}%
\pgfpathlineto{\pgfqpoint{4.578076in}{2.010001in}}%
\pgfpathlineto{\pgfqpoint{4.578076in}{2.007051in}}%
\pgfpathmoveto{\pgfqpoint{4.578076in}{2.007051in}}%
\pgfpathlineto{\pgfqpoint{4.578076in}{2.007051in}}%
\pgfpathlineto{\pgfqpoint{4.578076in}{2.010001in}}%
\pgfpathlineto{\pgfqpoint{4.582617in}{2.010001in}}%
\pgfpathlineto{\pgfqpoint{4.582617in}{2.007051in}}%
\pgfpathmoveto{\pgfqpoint{4.582617in}{2.007051in}}%
\pgfpathlineto{\pgfqpoint{4.582617in}{2.007051in}}%
\pgfpathlineto{\pgfqpoint{4.582617in}{2.010001in}}%
\pgfpathlineto{\pgfqpoint{4.587158in}{2.010001in}}%
\pgfpathlineto{\pgfqpoint{4.587158in}{2.007051in}}%
\pgfpathmoveto{\pgfqpoint{4.587158in}{2.007051in}}%
\pgfpathlineto{\pgfqpoint{4.587158in}{2.007051in}}%
\pgfpathlineto{\pgfqpoint{4.587158in}{2.010001in}}%
\pgfpathlineto{\pgfqpoint{4.591699in}{2.010001in}}%
\pgfpathlineto{\pgfqpoint{4.591699in}{2.007051in}}%
\pgfpathmoveto{\pgfqpoint{4.591699in}{2.007051in}}%
\pgfpathlineto{\pgfqpoint{4.591699in}{2.007051in}}%
\pgfpathlineto{\pgfqpoint{4.591699in}{2.010001in}}%
\pgfpathlineto{\pgfqpoint{4.596240in}{2.010001in}}%
\pgfpathlineto{\pgfqpoint{4.596240in}{2.007051in}}%
\pgfpathmoveto{\pgfqpoint{4.596240in}{2.007051in}}%
\pgfpathlineto{\pgfqpoint{4.596240in}{2.007051in}}%
\pgfpathlineto{\pgfqpoint{4.596240in}{2.010001in}}%
\pgfpathlineto{\pgfqpoint{4.600781in}{2.010001in}}%
\pgfpathlineto{\pgfqpoint{4.600781in}{2.007051in}}%
\pgfpathmoveto{\pgfqpoint{4.600781in}{2.007051in}}%
\pgfpathlineto{\pgfqpoint{4.600781in}{2.007051in}}%
\pgfpathlineto{\pgfqpoint{4.600781in}{2.010001in}}%
\pgfpathlineto{\pgfqpoint{4.605322in}{2.010001in}}%
\pgfpathlineto{\pgfqpoint{4.605322in}{2.007051in}}%
\pgfpathmoveto{\pgfqpoint{4.605322in}{2.007051in}}%
\pgfpathlineto{\pgfqpoint{4.605322in}{2.007051in}}%
\pgfpathlineto{\pgfqpoint{4.605322in}{2.010001in}}%
\pgfpathlineto{\pgfqpoint{4.609863in}{2.010001in}}%
\pgfpathlineto{\pgfqpoint{4.609863in}{2.007051in}}%
\pgfpathmoveto{\pgfqpoint{4.609863in}{2.007051in}}%
\pgfpathlineto{\pgfqpoint{4.609863in}{2.007051in}}%
\pgfpathlineto{\pgfqpoint{4.609863in}{2.010001in}}%
\pgfpathlineto{\pgfqpoint{4.614405in}{2.010001in}}%
\pgfpathlineto{\pgfqpoint{4.614405in}{2.007051in}}%
\pgfpathmoveto{\pgfqpoint{4.614405in}{2.007051in}}%
\pgfpathlineto{\pgfqpoint{4.614405in}{2.007051in}}%
\pgfpathlineto{\pgfqpoint{4.614405in}{2.010001in}}%
\pgfpathlineto{\pgfqpoint{4.618946in}{2.010001in}}%
\pgfpathlineto{\pgfqpoint{4.618946in}{2.007051in}}%
\pgfpathmoveto{\pgfqpoint{4.618946in}{2.007051in}}%
\pgfpathlineto{\pgfqpoint{4.618946in}{2.007051in}}%
\pgfpathlineto{\pgfqpoint{4.618946in}{2.010001in}}%
\pgfpathlineto{\pgfqpoint{4.623487in}{2.010001in}}%
\pgfpathlineto{\pgfqpoint{4.623487in}{2.007051in}}%
\pgfpathmoveto{\pgfqpoint{4.623487in}{2.007051in}}%
\pgfpathlineto{\pgfqpoint{4.623487in}{2.007051in}}%
\pgfpathlineto{\pgfqpoint{4.623487in}{2.010001in}}%
\pgfpathlineto{\pgfqpoint{4.628028in}{2.010001in}}%
\pgfpathlineto{\pgfqpoint{4.628028in}{2.007051in}}%
\pgfpathmoveto{\pgfqpoint{4.628028in}{2.007051in}}%
\pgfpathlineto{\pgfqpoint{4.628028in}{2.007051in}}%
\pgfpathlineto{\pgfqpoint{4.628028in}{2.010001in}}%
\pgfpathlineto{\pgfqpoint{4.632569in}{2.010001in}}%
\pgfpathlineto{\pgfqpoint{4.632569in}{2.007051in}}%
\pgfpathmoveto{\pgfqpoint{4.632569in}{2.007051in}}%
\pgfpathlineto{\pgfqpoint{4.632569in}{2.007051in}}%
\pgfpathlineto{\pgfqpoint{4.632569in}{2.010001in}}%
\pgfpathlineto{\pgfqpoint{4.637110in}{2.010001in}}%
\pgfpathlineto{\pgfqpoint{4.637110in}{2.007051in}}%
\pgfpathmoveto{\pgfqpoint{4.637110in}{2.007051in}}%
\pgfpathlineto{\pgfqpoint{4.637110in}{2.007051in}}%
\pgfpathlineto{\pgfqpoint{4.637110in}{2.010001in}}%
\pgfpathlineto{\pgfqpoint{4.641651in}{2.010001in}}%
\pgfpathlineto{\pgfqpoint{4.641651in}{2.007051in}}%
\pgfpathmoveto{\pgfqpoint{4.641651in}{2.007051in}}%
\pgfpathlineto{\pgfqpoint{4.641651in}{2.007051in}}%
\pgfpathlineto{\pgfqpoint{4.641651in}{2.010001in}}%
\pgfpathlineto{\pgfqpoint{4.646192in}{2.010001in}}%
\pgfpathlineto{\pgfqpoint{4.646192in}{2.007051in}}%
\pgfpathmoveto{\pgfqpoint{4.646192in}{2.007051in}}%
\pgfpathlineto{\pgfqpoint{4.646192in}{2.007051in}}%
\pgfpathlineto{\pgfqpoint{4.646192in}{2.010001in}}%
\pgfpathlineto{\pgfqpoint{4.650733in}{2.010001in}}%
\pgfpathlineto{\pgfqpoint{4.650733in}{2.007051in}}%
\pgfpathmoveto{\pgfqpoint{4.650733in}{2.007051in}}%
\pgfpathlineto{\pgfqpoint{4.650733in}{2.007051in}}%
\pgfpathlineto{\pgfqpoint{4.650733in}{2.010001in}}%
\pgfpathlineto{\pgfqpoint{4.655275in}{2.010001in}}%
\pgfpathlineto{\pgfqpoint{4.655275in}{2.007051in}}%
\pgfpathmoveto{\pgfqpoint{4.655275in}{2.007051in}}%
\pgfpathlineto{\pgfqpoint{4.655275in}{2.007051in}}%
\pgfpathlineto{\pgfqpoint{4.655275in}{2.010001in}}%
\pgfpathlineto{\pgfqpoint{4.659816in}{2.010001in}}%
\pgfpathlineto{\pgfqpoint{4.659816in}{2.007051in}}%
\pgfpathmoveto{\pgfqpoint{4.659816in}{2.007051in}}%
\pgfpathlineto{\pgfqpoint{4.659816in}{2.007051in}}%
\pgfpathlineto{\pgfqpoint{4.659816in}{2.010001in}}%
\pgfpathlineto{\pgfqpoint{4.664357in}{2.010001in}}%
\pgfpathlineto{\pgfqpoint{4.664357in}{2.007051in}}%
\pgfpathmoveto{\pgfqpoint{4.664357in}{2.007051in}}%
\pgfpathlineto{\pgfqpoint{4.664357in}{2.007051in}}%
\pgfpathlineto{\pgfqpoint{4.664357in}{2.010001in}}%
\pgfpathlineto{\pgfqpoint{4.668898in}{2.010001in}}%
\pgfpathlineto{\pgfqpoint{4.668898in}{2.007051in}}%
\pgfpathmoveto{\pgfqpoint{4.668898in}{2.007051in}}%
\pgfpathlineto{\pgfqpoint{4.668898in}{2.007051in}}%
\pgfpathlineto{\pgfqpoint{4.668898in}{2.010001in}}%
\pgfpathlineto{\pgfqpoint{4.673439in}{2.010001in}}%
\pgfpathlineto{\pgfqpoint{4.673439in}{2.007051in}}%
\pgfpathmoveto{\pgfqpoint{4.673439in}{2.007051in}}%
\pgfpathlineto{\pgfqpoint{4.673439in}{2.007051in}}%
\pgfpathlineto{\pgfqpoint{4.673439in}{2.010001in}}%
\pgfpathlineto{\pgfqpoint{4.677980in}{2.010001in}}%
\pgfpathlineto{\pgfqpoint{4.677980in}{2.007051in}}%
\pgfpathmoveto{\pgfqpoint{4.677980in}{2.007051in}}%
\pgfpathlineto{\pgfqpoint{4.677980in}{2.007051in}}%
\pgfpathlineto{\pgfqpoint{4.677980in}{2.010001in}}%
\pgfpathlineto{\pgfqpoint{4.682521in}{2.010001in}}%
\pgfpathlineto{\pgfqpoint{4.682521in}{2.007051in}}%
\pgfpathmoveto{\pgfqpoint{4.682521in}{2.007051in}}%
\pgfpathlineto{\pgfqpoint{4.682521in}{2.007051in}}%
\pgfpathlineto{\pgfqpoint{4.682521in}{2.010001in}}%
\pgfpathlineto{\pgfqpoint{4.687062in}{2.010001in}}%
\pgfpathlineto{\pgfqpoint{4.687062in}{2.007051in}}%
\pgfpathmoveto{\pgfqpoint{4.687062in}{2.007051in}}%
\pgfpathlineto{\pgfqpoint{4.687062in}{2.007051in}}%
\pgfpathlineto{\pgfqpoint{4.687062in}{2.010001in}}%
\pgfpathlineto{\pgfqpoint{4.691603in}{2.010001in}}%
\pgfpathlineto{\pgfqpoint{4.691603in}{2.007051in}}%
\pgfpathmoveto{\pgfqpoint{4.691603in}{2.007051in}}%
\pgfpathlineto{\pgfqpoint{4.691603in}{2.007051in}}%
\pgfpathlineto{\pgfqpoint{4.691603in}{2.010001in}}%
\pgfpathlineto{\pgfqpoint{4.696144in}{2.010001in}}%
\pgfpathlineto{\pgfqpoint{4.696144in}{2.007051in}}%
\pgfpathmoveto{\pgfqpoint{4.696144in}{2.007051in}}%
\pgfpathlineto{\pgfqpoint{4.696144in}{2.007051in}}%
\pgfpathlineto{\pgfqpoint{4.696144in}{2.010001in}}%
\pgfpathlineto{\pgfqpoint{4.700685in}{2.010001in}}%
\pgfpathlineto{\pgfqpoint{4.700685in}{2.007051in}}%
\pgfpathmoveto{\pgfqpoint{4.700685in}{2.007051in}}%
\pgfpathlineto{\pgfqpoint{4.700685in}{2.007051in}}%
\pgfpathlineto{\pgfqpoint{4.700685in}{2.010001in}}%
\pgfpathlineto{\pgfqpoint{4.705226in}{2.010001in}}%
\pgfpathlineto{\pgfqpoint{4.705226in}{2.007051in}}%
\pgfpathmoveto{\pgfqpoint{4.705226in}{2.007051in}}%
\pgfpathlineto{\pgfqpoint{4.705226in}{2.007051in}}%
\pgfpathlineto{\pgfqpoint{4.705226in}{2.010001in}}%
\pgfpathlineto{\pgfqpoint{4.709767in}{2.010001in}}%
\pgfpathlineto{\pgfqpoint{4.709767in}{2.007051in}}%
\pgfpathmoveto{\pgfqpoint{4.709767in}{2.007051in}}%
\pgfpathlineto{\pgfqpoint{4.709767in}{2.007051in}}%
\pgfpathlineto{\pgfqpoint{4.709767in}{2.010001in}}%
\pgfpathlineto{\pgfqpoint{4.714308in}{2.010001in}}%
\pgfpathlineto{\pgfqpoint{4.714308in}{2.007051in}}%
\pgfpathmoveto{\pgfqpoint{4.714308in}{2.007051in}}%
\pgfpathlineto{\pgfqpoint{4.714308in}{2.007051in}}%
\pgfpathlineto{\pgfqpoint{4.714308in}{2.010001in}}%
\pgfpathlineto{\pgfqpoint{4.718849in}{2.010001in}}%
\pgfpathlineto{\pgfqpoint{4.718849in}{2.007051in}}%
\pgfpathmoveto{\pgfqpoint{4.718849in}{2.007051in}}%
\pgfpathlineto{\pgfqpoint{4.718849in}{2.007051in}}%
\pgfpathlineto{\pgfqpoint{4.718849in}{2.010001in}}%
\pgfpathlineto{\pgfqpoint{4.723390in}{2.010001in}}%
\pgfpathlineto{\pgfqpoint{4.723390in}{2.007051in}}%
\pgfpathmoveto{\pgfqpoint{4.723390in}{2.007051in}}%
\pgfpathlineto{\pgfqpoint{4.723390in}{2.007051in}}%
\pgfpathlineto{\pgfqpoint{4.723390in}{2.010001in}}%
\pgfpathlineto{\pgfqpoint{4.727931in}{2.010001in}}%
\pgfpathlineto{\pgfqpoint{4.727931in}{2.007051in}}%
\pgfpathmoveto{\pgfqpoint{4.727931in}{2.007051in}}%
\pgfpathlineto{\pgfqpoint{4.727931in}{2.007051in}}%
\pgfpathlineto{\pgfqpoint{4.727931in}{2.010001in}}%
\pgfpathlineto{\pgfqpoint{4.732472in}{2.010001in}}%
\pgfpathlineto{\pgfqpoint{4.732472in}{2.007051in}}%
\pgfpathmoveto{\pgfqpoint{4.732472in}{2.007051in}}%
\pgfpathlineto{\pgfqpoint{4.732472in}{2.007051in}}%
\pgfpathlineto{\pgfqpoint{4.732472in}{2.010001in}}%
\pgfpathlineto{\pgfqpoint{4.737013in}{2.010001in}}%
\pgfpathlineto{\pgfqpoint{4.737013in}{2.007051in}}%
\pgfpathmoveto{\pgfqpoint{4.737013in}{2.007051in}}%
\pgfpathlineto{\pgfqpoint{4.737013in}{2.007051in}}%
\pgfpathlineto{\pgfqpoint{4.737013in}{2.010001in}}%
\pgfpathlineto{\pgfqpoint{4.741554in}{2.010001in}}%
\pgfpathlineto{\pgfqpoint{4.741554in}{2.007051in}}%
\pgfpathmoveto{\pgfqpoint{4.741554in}{2.007051in}}%
\pgfpathlineto{\pgfqpoint{4.741554in}{2.007051in}}%
\pgfpathlineto{\pgfqpoint{4.741554in}{2.010001in}}%
\pgfpathlineto{\pgfqpoint{4.746095in}{2.010001in}}%
\pgfpathlineto{\pgfqpoint{4.746095in}{2.007051in}}%
\pgfpathmoveto{\pgfqpoint{4.746095in}{2.007051in}}%
\pgfpathlineto{\pgfqpoint{4.746095in}{2.007051in}}%
\pgfpathlineto{\pgfqpoint{4.746095in}{2.010001in}}%
\pgfpathlineto{\pgfqpoint{4.750636in}{2.010001in}}%
\pgfpathlineto{\pgfqpoint{4.750636in}{2.007051in}}%
\pgfpathmoveto{\pgfqpoint{4.750636in}{2.007051in}}%
\pgfpathlineto{\pgfqpoint{4.750636in}{2.007051in}}%
\pgfpathlineto{\pgfqpoint{4.750636in}{2.010001in}}%
\pgfpathlineto{\pgfqpoint{4.755177in}{2.010001in}}%
\pgfpathlineto{\pgfqpoint{4.755177in}{2.007051in}}%
\pgfpathmoveto{\pgfqpoint{4.755177in}{2.007051in}}%
\pgfpathlineto{\pgfqpoint{4.755177in}{2.007051in}}%
\pgfpathlineto{\pgfqpoint{4.755177in}{2.010001in}}%
\pgfpathlineto{\pgfqpoint{4.759718in}{2.010001in}}%
\pgfpathlineto{\pgfqpoint{4.759718in}{2.007051in}}%
\pgfpathmoveto{\pgfqpoint{4.759718in}{2.007051in}}%
\pgfpathlineto{\pgfqpoint{4.759718in}{2.007051in}}%
\pgfpathlineto{\pgfqpoint{4.759718in}{2.010001in}}%
\pgfpathlineto{\pgfqpoint{4.764259in}{2.010001in}}%
\pgfpathlineto{\pgfqpoint{4.764259in}{2.007051in}}%
\pgfpathmoveto{\pgfqpoint{4.764259in}{2.007051in}}%
\pgfpathlineto{\pgfqpoint{4.764259in}{2.007051in}}%
\pgfpathlineto{\pgfqpoint{4.764259in}{2.010001in}}%
\pgfpathlineto{\pgfqpoint{4.768800in}{2.010001in}}%
\pgfpathlineto{\pgfqpoint{4.768800in}{2.007051in}}%
\pgfpathmoveto{\pgfqpoint{4.768800in}{2.007051in}}%
\pgfpathlineto{\pgfqpoint{4.768800in}{2.007051in}}%
\pgfpathlineto{\pgfqpoint{4.768800in}{2.010001in}}%
\pgfpathlineto{\pgfqpoint{4.773341in}{2.010001in}}%
\pgfpathlineto{\pgfqpoint{4.773341in}{2.007051in}}%
\pgfpathmoveto{\pgfqpoint{4.773341in}{2.007051in}}%
\pgfpathlineto{\pgfqpoint{4.773341in}{2.007051in}}%
\pgfpathlineto{\pgfqpoint{4.773341in}{2.010001in}}%
\pgfpathlineto{\pgfqpoint{4.777882in}{2.010001in}}%
\pgfpathlineto{\pgfqpoint{4.777882in}{2.007051in}}%
\pgfpathmoveto{\pgfqpoint{4.777882in}{2.007051in}}%
\pgfpathlineto{\pgfqpoint{4.777882in}{2.007051in}}%
\pgfpathlineto{\pgfqpoint{4.777882in}{2.010001in}}%
\pgfpathlineto{\pgfqpoint{4.782423in}{2.010001in}}%
\pgfpathlineto{\pgfqpoint{4.782423in}{2.007051in}}%
\pgfpathmoveto{\pgfqpoint{4.782423in}{2.007051in}}%
\pgfpathlineto{\pgfqpoint{4.782423in}{2.007051in}}%
\pgfpathlineto{\pgfqpoint{4.782423in}{2.010001in}}%
\pgfpathlineto{\pgfqpoint{4.786964in}{2.010001in}}%
\pgfpathlineto{\pgfqpoint{4.786964in}{2.007051in}}%
\pgfpathmoveto{\pgfqpoint{4.786964in}{2.007051in}}%
\pgfpathlineto{\pgfqpoint{4.786964in}{2.007051in}}%
\pgfpathlineto{\pgfqpoint{4.786964in}{2.010001in}}%
\pgfpathlineto{\pgfqpoint{4.791505in}{2.010001in}}%
\pgfpathlineto{\pgfqpoint{4.791505in}{2.007051in}}%
\pgfpathmoveto{\pgfqpoint{4.791505in}{2.007051in}}%
\pgfpathlineto{\pgfqpoint{4.791505in}{2.007051in}}%
\pgfpathlineto{\pgfqpoint{4.791505in}{2.010001in}}%
\pgfpathlineto{\pgfqpoint{4.796046in}{2.010001in}}%
\pgfpathlineto{\pgfqpoint{4.796046in}{2.007051in}}%
\pgfpathmoveto{\pgfqpoint{4.796046in}{2.007051in}}%
\pgfpathlineto{\pgfqpoint{4.796046in}{2.007051in}}%
\pgfpathlineto{\pgfqpoint{4.796046in}{2.010001in}}%
\pgfpathlineto{\pgfqpoint{4.800587in}{2.010001in}}%
\pgfpathlineto{\pgfqpoint{4.800587in}{2.007051in}}%
\pgfpathmoveto{\pgfqpoint{4.800587in}{2.007051in}}%
\pgfpathlineto{\pgfqpoint{4.800587in}{2.007051in}}%
\pgfpathlineto{\pgfqpoint{4.800587in}{2.010001in}}%
\pgfpathlineto{\pgfqpoint{4.805128in}{2.010001in}}%
\pgfpathlineto{\pgfqpoint{4.805128in}{2.007051in}}%
\pgfpathmoveto{\pgfqpoint{4.805128in}{2.007051in}}%
\pgfpathlineto{\pgfqpoint{4.805128in}{2.007051in}}%
\pgfpathlineto{\pgfqpoint{4.805128in}{2.010001in}}%
\pgfpathlineto{\pgfqpoint{4.809669in}{2.010001in}}%
\pgfpathlineto{\pgfqpoint{4.809669in}{2.007051in}}%
\pgfpathmoveto{\pgfqpoint{4.809669in}{2.007051in}}%
\pgfpathlineto{\pgfqpoint{4.809669in}{2.007051in}}%
\pgfpathlineto{\pgfqpoint{4.809669in}{2.010001in}}%
\pgfpathlineto{\pgfqpoint{4.814210in}{2.010001in}}%
\pgfpathlineto{\pgfqpoint{4.814210in}{2.007051in}}%
\pgfpathmoveto{\pgfqpoint{4.814210in}{2.007051in}}%
\pgfpathlineto{\pgfqpoint{4.814210in}{2.007051in}}%
\pgfpathlineto{\pgfqpoint{4.814210in}{2.010001in}}%
\pgfpathlineto{\pgfqpoint{4.818751in}{2.010001in}}%
\pgfpathlineto{\pgfqpoint{4.818751in}{2.007051in}}%
\pgfpathmoveto{\pgfqpoint{4.818751in}{2.007051in}}%
\pgfpathlineto{\pgfqpoint{4.818751in}{2.007051in}}%
\pgfpathlineto{\pgfqpoint{4.818751in}{2.010001in}}%
\pgfpathlineto{\pgfqpoint{4.823292in}{2.010001in}}%
\pgfpathlineto{\pgfqpoint{4.823292in}{2.007051in}}%
\pgfpathmoveto{\pgfqpoint{4.823292in}{2.007051in}}%
\pgfpathlineto{\pgfqpoint{4.823292in}{2.007051in}}%
\pgfpathlineto{\pgfqpoint{4.823292in}{2.010001in}}%
\pgfpathlineto{\pgfqpoint{4.827833in}{2.010001in}}%
\pgfpathlineto{\pgfqpoint{4.827833in}{2.007051in}}%
\pgfpathmoveto{\pgfqpoint{4.827833in}{2.007051in}}%
\pgfpathlineto{\pgfqpoint{4.827833in}{2.007051in}}%
\pgfpathlineto{\pgfqpoint{4.827833in}{2.010001in}}%
\pgfpathlineto{\pgfqpoint{4.832374in}{2.010001in}}%
\pgfpathlineto{\pgfqpoint{4.832374in}{2.007051in}}%
\pgfpathmoveto{\pgfqpoint{4.832374in}{2.007051in}}%
\pgfpathlineto{\pgfqpoint{4.832374in}{2.007051in}}%
\pgfpathlineto{\pgfqpoint{4.832374in}{2.010001in}}%
\pgfpathlineto{\pgfqpoint{4.836915in}{2.010001in}}%
\pgfpathlineto{\pgfqpoint{4.836915in}{2.007051in}}%
\pgfpathmoveto{\pgfqpoint{4.836915in}{2.007051in}}%
\pgfpathlineto{\pgfqpoint{4.836915in}{2.007051in}}%
\pgfpathlineto{\pgfqpoint{4.836915in}{2.010001in}}%
\pgfpathlineto{\pgfqpoint{4.841456in}{2.010001in}}%
\pgfpathlineto{\pgfqpoint{4.841456in}{2.007051in}}%
\pgfpathmoveto{\pgfqpoint{4.841456in}{2.007051in}}%
\pgfpathlineto{\pgfqpoint{4.841456in}{2.007051in}}%
\pgfpathlineto{\pgfqpoint{4.841456in}{2.010001in}}%
\pgfpathlineto{\pgfqpoint{4.845997in}{2.010001in}}%
\pgfpathlineto{\pgfqpoint{4.845997in}{2.007051in}}%
\pgfpathmoveto{\pgfqpoint{4.845997in}{2.007051in}}%
\pgfpathlineto{\pgfqpoint{4.845997in}{2.007051in}}%
\pgfpathlineto{\pgfqpoint{4.845997in}{2.010001in}}%
\pgfpathlineto{\pgfqpoint{4.850538in}{2.010001in}}%
\pgfpathlineto{\pgfqpoint{4.850538in}{2.007051in}}%
\pgfpathmoveto{\pgfqpoint{4.850538in}{2.007051in}}%
\pgfpathlineto{\pgfqpoint{4.850538in}{2.007051in}}%
\pgfpathlineto{\pgfqpoint{4.850538in}{2.010001in}}%
\pgfpathlineto{\pgfqpoint{4.855079in}{2.010001in}}%
\pgfpathlineto{\pgfqpoint{4.855079in}{2.007051in}}%
\pgfpathmoveto{\pgfqpoint{4.855079in}{2.007051in}}%
\pgfpathlineto{\pgfqpoint{4.855079in}{2.007051in}}%
\pgfpathlineto{\pgfqpoint{4.855079in}{2.010001in}}%
\pgfpathlineto{\pgfqpoint{4.859620in}{2.010001in}}%
\pgfpathlineto{\pgfqpoint{4.859620in}{2.007051in}}%
\pgfpathmoveto{\pgfqpoint{4.859620in}{2.007051in}}%
\pgfpathlineto{\pgfqpoint{4.859620in}{2.007051in}}%
\pgfpathlineto{\pgfqpoint{4.859620in}{2.010001in}}%
\pgfpathlineto{\pgfqpoint{4.864161in}{2.010001in}}%
\pgfpathlineto{\pgfqpoint{4.864161in}{2.007051in}}%
\pgfpathmoveto{\pgfqpoint{4.864161in}{2.007051in}}%
\pgfpathlineto{\pgfqpoint{4.864161in}{2.007051in}}%
\pgfpathlineto{\pgfqpoint{4.864161in}{2.010001in}}%
\pgfpathlineto{\pgfqpoint{4.868702in}{2.010001in}}%
\pgfpathlineto{\pgfqpoint{4.868702in}{2.007051in}}%
\pgfpathmoveto{\pgfqpoint{4.868702in}{2.007051in}}%
\pgfpathlineto{\pgfqpoint{4.868702in}{2.007051in}}%
\pgfpathlineto{\pgfqpoint{4.868702in}{2.010001in}}%
\pgfpathlineto{\pgfqpoint{4.873243in}{2.010001in}}%
\pgfpathlineto{\pgfqpoint{4.873243in}{2.007051in}}%
\pgfpathmoveto{\pgfqpoint{4.873243in}{2.007051in}}%
\pgfpathlineto{\pgfqpoint{4.873243in}{2.007051in}}%
\pgfpathlineto{\pgfqpoint{4.873243in}{2.010001in}}%
\pgfpathlineto{\pgfqpoint{4.877784in}{2.010001in}}%
\pgfpathlineto{\pgfqpoint{4.877784in}{2.007051in}}%
\pgfpathmoveto{\pgfqpoint{4.877784in}{2.007051in}}%
\pgfpathlineto{\pgfqpoint{4.877784in}{2.007051in}}%
\pgfpathlineto{\pgfqpoint{4.877784in}{2.010001in}}%
\pgfpathlineto{\pgfqpoint{4.882325in}{2.010001in}}%
\pgfpathlineto{\pgfqpoint{4.882325in}{2.007051in}}%
\pgfpathmoveto{\pgfqpoint{4.882325in}{2.007051in}}%
\pgfpathlineto{\pgfqpoint{4.882325in}{2.007051in}}%
\pgfpathlineto{\pgfqpoint{4.882325in}{2.010001in}}%
\pgfpathlineto{\pgfqpoint{4.886866in}{2.010001in}}%
\pgfpathlineto{\pgfqpoint{4.886866in}{2.007051in}}%
\pgfpathmoveto{\pgfqpoint{4.886866in}{2.007051in}}%
\pgfpathlineto{\pgfqpoint{4.886866in}{2.007051in}}%
\pgfpathlineto{\pgfqpoint{4.886866in}{2.010001in}}%
\pgfpathlineto{\pgfqpoint{4.891407in}{2.010001in}}%
\pgfpathlineto{\pgfqpoint{4.891407in}{2.007051in}}%
\pgfpathmoveto{\pgfqpoint{4.891407in}{2.007051in}}%
\pgfpathlineto{\pgfqpoint{4.891407in}{2.007051in}}%
\pgfpathlineto{\pgfqpoint{4.891407in}{2.010001in}}%
\pgfpathlineto{\pgfqpoint{4.895948in}{2.010001in}}%
\pgfpathlineto{\pgfqpoint{4.895948in}{2.007051in}}%
\pgfpathmoveto{\pgfqpoint{4.895948in}{2.007051in}}%
\pgfpathlineto{\pgfqpoint{4.895948in}{2.007051in}}%
\pgfpathlineto{\pgfqpoint{4.895948in}{2.010001in}}%
\pgfpathlineto{\pgfqpoint{4.900489in}{2.010001in}}%
\pgfpathlineto{\pgfqpoint{4.900489in}{2.007051in}}%
\pgfpathmoveto{\pgfqpoint{4.900489in}{2.007051in}}%
\pgfpathlineto{\pgfqpoint{4.900489in}{2.007051in}}%
\pgfpathlineto{\pgfqpoint{4.900489in}{2.010001in}}%
\pgfpathlineto{\pgfqpoint{4.905031in}{2.010001in}}%
\pgfpathlineto{\pgfqpoint{4.905031in}{2.007051in}}%
\pgfpathmoveto{\pgfqpoint{4.905031in}{2.007051in}}%
\pgfpathlineto{\pgfqpoint{4.905031in}{2.007051in}}%
\pgfpathlineto{\pgfqpoint{4.905031in}{2.010001in}}%
\pgfpathlineto{\pgfqpoint{4.909572in}{2.010001in}}%
\pgfpathlineto{\pgfqpoint{4.909572in}{2.007051in}}%
\pgfpathmoveto{\pgfqpoint{4.909572in}{2.007051in}}%
\pgfpathlineto{\pgfqpoint{4.909572in}{2.007051in}}%
\pgfpathlineto{\pgfqpoint{4.909572in}{2.010001in}}%
\pgfpathlineto{\pgfqpoint{4.914113in}{2.010001in}}%
\pgfpathlineto{\pgfqpoint{4.914113in}{2.007051in}}%
\pgfpathmoveto{\pgfqpoint{4.914113in}{2.007051in}}%
\pgfpathlineto{\pgfqpoint{4.914113in}{2.007051in}}%
\pgfpathlineto{\pgfqpoint{4.914113in}{2.010001in}}%
\pgfpathlineto{\pgfqpoint{4.918654in}{2.010001in}}%
\pgfpathlineto{\pgfqpoint{4.918654in}{2.007051in}}%
\pgfpathmoveto{\pgfqpoint{4.918654in}{2.007051in}}%
\pgfpathlineto{\pgfqpoint{4.918654in}{2.007051in}}%
\pgfpathlineto{\pgfqpoint{4.918654in}{2.010001in}}%
\pgfpathlineto{\pgfqpoint{4.923195in}{2.010001in}}%
\pgfpathlineto{\pgfqpoint{4.923195in}{2.007051in}}%
\pgfpathmoveto{\pgfqpoint{4.923195in}{2.007051in}}%
\pgfpathlineto{\pgfqpoint{4.923195in}{2.007051in}}%
\pgfpathlineto{\pgfqpoint{4.923195in}{2.010001in}}%
\pgfpathlineto{\pgfqpoint{4.927736in}{2.010001in}}%
\pgfpathlineto{\pgfqpoint{4.927736in}{2.007051in}}%
\pgfpathmoveto{\pgfqpoint{4.927736in}{2.007051in}}%
\pgfpathlineto{\pgfqpoint{4.927736in}{2.007051in}}%
\pgfpathlineto{\pgfqpoint{4.927736in}{2.010001in}}%
\pgfpathlineto{\pgfqpoint{4.932277in}{2.010001in}}%
\pgfpathlineto{\pgfqpoint{4.932277in}{2.007051in}}%
\pgfpathmoveto{\pgfqpoint{4.932277in}{2.007051in}}%
\pgfpathlineto{\pgfqpoint{4.932277in}{2.007051in}}%
\pgfpathlineto{\pgfqpoint{4.932277in}{2.010001in}}%
\pgfpathlineto{\pgfqpoint{4.936818in}{2.010001in}}%
\pgfpathlineto{\pgfqpoint{4.936818in}{2.007051in}}%
\pgfpathmoveto{\pgfqpoint{4.936818in}{2.007051in}}%
\pgfpathlineto{\pgfqpoint{4.936818in}{2.007051in}}%
\pgfpathlineto{\pgfqpoint{4.936818in}{2.010001in}}%
\pgfpathlineto{\pgfqpoint{4.941359in}{2.010001in}}%
\pgfpathlineto{\pgfqpoint{4.941359in}{2.007051in}}%
\pgfpathmoveto{\pgfqpoint{4.941359in}{2.007051in}}%
\pgfpathlineto{\pgfqpoint{4.941359in}{2.007051in}}%
\pgfpathlineto{\pgfqpoint{4.941359in}{2.010001in}}%
\pgfpathlineto{\pgfqpoint{4.945900in}{2.010001in}}%
\pgfpathlineto{\pgfqpoint{4.945900in}{2.007051in}}%
\pgfpathmoveto{\pgfqpoint{4.945900in}{2.007051in}}%
\pgfpathlineto{\pgfqpoint{4.945900in}{2.007051in}}%
\pgfpathlineto{\pgfqpoint{4.945900in}{2.010001in}}%
\pgfpathlineto{\pgfqpoint{4.950441in}{2.010001in}}%
\pgfpathlineto{\pgfqpoint{4.950441in}{2.007051in}}%
\pgfpathmoveto{\pgfqpoint{4.950441in}{2.007051in}}%
\pgfpathlineto{\pgfqpoint{4.950441in}{2.007051in}}%
\pgfpathlineto{\pgfqpoint{4.950441in}{2.010001in}}%
\pgfpathlineto{\pgfqpoint{4.954982in}{2.010001in}}%
\pgfpathlineto{\pgfqpoint{4.954982in}{2.007051in}}%
\pgfpathmoveto{\pgfqpoint{4.954982in}{2.007051in}}%
\pgfpathlineto{\pgfqpoint{4.954982in}{2.007051in}}%
\pgfpathlineto{\pgfqpoint{4.954982in}{2.010001in}}%
\pgfpathlineto{\pgfqpoint{4.959523in}{2.010001in}}%
\pgfpathlineto{\pgfqpoint{4.959523in}{2.007051in}}%
\pgfpathmoveto{\pgfqpoint{4.959523in}{2.007051in}}%
\pgfpathlineto{\pgfqpoint{4.959523in}{2.007051in}}%
\pgfpathlineto{\pgfqpoint{4.959523in}{2.010001in}}%
\pgfpathlineto{\pgfqpoint{4.964064in}{2.010001in}}%
\pgfpathlineto{\pgfqpoint{4.964064in}{2.007051in}}%
\pgfpathmoveto{\pgfqpoint{4.964064in}{2.007051in}}%
\pgfpathlineto{\pgfqpoint{4.964064in}{2.007051in}}%
\pgfpathlineto{\pgfqpoint{4.964064in}{2.010001in}}%
\pgfpathlineto{\pgfqpoint{4.968605in}{2.010001in}}%
\pgfpathlineto{\pgfqpoint{4.968605in}{2.007051in}}%
\pgfpathmoveto{\pgfqpoint{4.968605in}{2.007051in}}%
\pgfpathlineto{\pgfqpoint{4.968605in}{2.007051in}}%
\pgfpathlineto{\pgfqpoint{4.968605in}{2.010001in}}%
\pgfpathlineto{\pgfqpoint{4.973146in}{2.010001in}}%
\pgfpathlineto{\pgfqpoint{4.973146in}{2.007051in}}%
\pgfpathmoveto{\pgfqpoint{4.973146in}{2.007051in}}%
\pgfpathlineto{\pgfqpoint{4.973146in}{2.007051in}}%
\pgfpathlineto{\pgfqpoint{4.973146in}{2.010001in}}%
\pgfpathlineto{\pgfqpoint{4.977687in}{2.010001in}}%
\pgfpathlineto{\pgfqpoint{4.977687in}{2.007051in}}%
\pgfpathmoveto{\pgfqpoint{4.977687in}{2.007051in}}%
\pgfpathlineto{\pgfqpoint{4.977687in}{2.007051in}}%
\pgfpathlineto{\pgfqpoint{4.977687in}{2.010001in}}%
\pgfpathlineto{\pgfqpoint{4.982228in}{2.010001in}}%
\pgfpathlineto{\pgfqpoint{4.982228in}{2.007051in}}%
\pgfpathmoveto{\pgfqpoint{4.982228in}{2.007051in}}%
\pgfpathlineto{\pgfqpoint{4.982228in}{2.007051in}}%
\pgfpathlineto{\pgfqpoint{4.982228in}{2.010001in}}%
\pgfpathlineto{\pgfqpoint{4.986768in}{2.010001in}}%
\pgfpathlineto{\pgfqpoint{4.986768in}{2.007051in}}%
\pgfpathmoveto{\pgfqpoint{4.986768in}{2.007051in}}%
\pgfpathlineto{\pgfqpoint{4.986768in}{2.007051in}}%
\pgfpathlineto{\pgfqpoint{4.986768in}{2.010001in}}%
\pgfpathlineto{\pgfqpoint{4.991309in}{2.010001in}}%
\pgfpathlineto{\pgfqpoint{4.991309in}{2.007051in}}%
\pgfpathmoveto{\pgfqpoint{4.991309in}{2.007051in}}%
\pgfpathlineto{\pgfqpoint{4.991309in}{2.007051in}}%
\pgfpathlineto{\pgfqpoint{4.991309in}{2.010001in}}%
\pgfpathlineto{\pgfqpoint{4.995850in}{2.010001in}}%
\pgfpathlineto{\pgfqpoint{4.995850in}{2.007051in}}%
\pgfpathmoveto{\pgfqpoint{4.995850in}{2.007051in}}%
\pgfpathlineto{\pgfqpoint{4.995850in}{2.007051in}}%
\pgfpathlineto{\pgfqpoint{4.995850in}{2.010001in}}%
\pgfpathlineto{\pgfqpoint{5.000391in}{2.010001in}}%
\pgfpathlineto{\pgfqpoint{5.000391in}{2.007051in}}%
\pgfpathmoveto{\pgfqpoint{5.000391in}{2.007051in}}%
\pgfpathlineto{\pgfqpoint{5.000391in}{2.007051in}}%
\pgfpathlineto{\pgfqpoint{5.000391in}{2.010001in}}%
\pgfpathlineto{\pgfqpoint{5.004932in}{2.010001in}}%
\pgfpathlineto{\pgfqpoint{5.004932in}{2.007051in}}%
\pgfpathmoveto{\pgfqpoint{5.004932in}{2.007051in}}%
\pgfpathlineto{\pgfqpoint{5.004932in}{2.007051in}}%
\pgfpathlineto{\pgfqpoint{5.004932in}{2.010001in}}%
\pgfpathlineto{\pgfqpoint{5.009473in}{2.010001in}}%
\pgfpathlineto{\pgfqpoint{5.009473in}{2.007051in}}%
\pgfpathmoveto{\pgfqpoint{5.009473in}{2.007051in}}%
\pgfpathlineto{\pgfqpoint{5.009473in}{2.007051in}}%
\pgfpathlineto{\pgfqpoint{5.009473in}{2.010001in}}%
\pgfpathlineto{\pgfqpoint{5.014014in}{2.010001in}}%
\pgfpathlineto{\pgfqpoint{5.014014in}{2.007051in}}%
\pgfpathmoveto{\pgfqpoint{5.014014in}{2.007051in}}%
\pgfpathlineto{\pgfqpoint{5.014014in}{2.007051in}}%
\pgfpathlineto{\pgfqpoint{5.014014in}{2.010001in}}%
\pgfpathlineto{\pgfqpoint{5.018554in}{2.010001in}}%
\pgfpathlineto{\pgfqpoint{5.018554in}{2.007051in}}%
\pgfpathmoveto{\pgfqpoint{5.018554in}{2.007051in}}%
\pgfpathlineto{\pgfqpoint{5.018554in}{2.007051in}}%
\pgfpathlineto{\pgfqpoint{5.018554in}{2.010001in}}%
\pgfpathlineto{\pgfqpoint{5.023095in}{2.010001in}}%
\pgfpathlineto{\pgfqpoint{5.023095in}{2.007051in}}%
\pgfpathmoveto{\pgfqpoint{5.023095in}{2.007051in}}%
\pgfpathlineto{\pgfqpoint{5.023095in}{2.007051in}}%
\pgfpathlineto{\pgfqpoint{5.023095in}{2.010001in}}%
\pgfpathlineto{\pgfqpoint{5.027636in}{2.010001in}}%
\pgfpathlineto{\pgfqpoint{5.027636in}{2.007051in}}%
\pgfpathmoveto{\pgfqpoint{5.027636in}{2.007051in}}%
\pgfpathlineto{\pgfqpoint{5.027636in}{2.007051in}}%
\pgfpathlineto{\pgfqpoint{5.027636in}{2.010001in}}%
\pgfpathlineto{\pgfqpoint{5.032177in}{2.010001in}}%
\pgfpathlineto{\pgfqpoint{5.032177in}{2.007051in}}%
\pgfpathmoveto{\pgfqpoint{5.032177in}{2.007051in}}%
\pgfpathlineto{\pgfqpoint{5.032177in}{2.007051in}}%
\pgfpathlineto{\pgfqpoint{5.032177in}{2.010001in}}%
\pgfpathlineto{\pgfqpoint{5.036718in}{2.010001in}}%
\pgfpathlineto{\pgfqpoint{5.036718in}{2.007051in}}%
\pgfpathmoveto{\pgfqpoint{5.036718in}{2.007051in}}%
\pgfpathlineto{\pgfqpoint{5.036718in}{2.007051in}}%
\pgfpathlineto{\pgfqpoint{5.036718in}{2.010001in}}%
\pgfpathlineto{\pgfqpoint{5.041259in}{2.010001in}}%
\pgfpathlineto{\pgfqpoint{5.041259in}{2.007051in}}%
\pgfpathmoveto{\pgfqpoint{5.041259in}{2.007051in}}%
\pgfpathlineto{\pgfqpoint{5.041259in}{2.007051in}}%
\pgfpathlineto{\pgfqpoint{5.041259in}{2.010001in}}%
\pgfpathlineto{\pgfqpoint{5.045800in}{2.010001in}}%
\pgfpathlineto{\pgfqpoint{5.045800in}{2.007051in}}%
\pgfpathmoveto{\pgfqpoint{5.045800in}{2.007051in}}%
\pgfpathlineto{\pgfqpoint{5.045800in}{2.007051in}}%
\pgfpathlineto{\pgfqpoint{5.045800in}{2.010001in}}%
\pgfpathlineto{\pgfqpoint{5.050340in}{2.010001in}}%
\pgfpathlineto{\pgfqpoint{5.050340in}{2.007051in}}%
\pgfpathmoveto{\pgfqpoint{5.050340in}{2.007051in}}%
\pgfpathlineto{\pgfqpoint{5.050340in}{2.007051in}}%
\pgfpathlineto{\pgfqpoint{5.050340in}{2.010001in}}%
\pgfpathlineto{\pgfqpoint{5.054881in}{2.010001in}}%
\pgfpathlineto{\pgfqpoint{5.054881in}{2.007051in}}%
\pgfpathmoveto{\pgfqpoint{5.054881in}{2.007051in}}%
\pgfpathlineto{\pgfqpoint{5.054881in}{2.007051in}}%
\pgfpathlineto{\pgfqpoint{5.054881in}{2.010001in}}%
\pgfpathlineto{\pgfqpoint{5.059422in}{2.010001in}}%
\pgfpathlineto{\pgfqpoint{5.059422in}{2.007051in}}%
\pgfpathmoveto{\pgfqpoint{5.059422in}{2.007051in}}%
\pgfpathlineto{\pgfqpoint{5.059422in}{2.007051in}}%
\pgfpathlineto{\pgfqpoint{5.059422in}{2.010001in}}%
\pgfpathlineto{\pgfqpoint{5.063963in}{2.010001in}}%
\pgfpathlineto{\pgfqpoint{5.063963in}{2.007051in}}%
\pgfpathmoveto{\pgfqpoint{5.063963in}{2.007051in}}%
\pgfpathlineto{\pgfqpoint{5.063963in}{2.007051in}}%
\pgfpathlineto{\pgfqpoint{5.063963in}{2.010001in}}%
\pgfpathlineto{\pgfqpoint{5.068504in}{2.010001in}}%
\pgfpathlineto{\pgfqpoint{5.068504in}{2.007051in}}%
\pgfpathmoveto{\pgfqpoint{5.068504in}{2.007051in}}%
\pgfpathlineto{\pgfqpoint{5.068504in}{2.007051in}}%
\pgfpathlineto{\pgfqpoint{5.068504in}{2.010001in}}%
\pgfpathlineto{\pgfqpoint{5.073045in}{2.010001in}}%
\pgfpathlineto{\pgfqpoint{5.073045in}{2.007051in}}%
\pgfpathmoveto{\pgfqpoint{5.073045in}{2.007051in}}%
\pgfpathlineto{\pgfqpoint{5.073045in}{2.007051in}}%
\pgfpathlineto{\pgfqpoint{5.073045in}{2.010001in}}%
\pgfpathlineto{\pgfqpoint{5.077586in}{2.010001in}}%
\pgfpathlineto{\pgfqpoint{5.077586in}{2.007051in}}%
\pgfpathmoveto{\pgfqpoint{5.077586in}{2.007051in}}%
\pgfpathlineto{\pgfqpoint{5.077586in}{2.007051in}}%
\pgfpathlineto{\pgfqpoint{5.077586in}{2.010001in}}%
\pgfpathlineto{\pgfqpoint{5.082126in}{2.010001in}}%
\pgfpathlineto{\pgfqpoint{5.082126in}{2.007051in}}%
\pgfpathmoveto{\pgfqpoint{5.082126in}{2.007051in}}%
\pgfpathlineto{\pgfqpoint{5.082126in}{2.007051in}}%
\pgfpathlineto{\pgfqpoint{5.082126in}{2.010001in}}%
\pgfpathlineto{\pgfqpoint{5.086667in}{2.010001in}}%
\pgfpathlineto{\pgfqpoint{5.086667in}{2.007051in}}%
\pgfpathmoveto{\pgfqpoint{5.086667in}{2.007051in}}%
\pgfpathlineto{\pgfqpoint{5.086667in}{2.007051in}}%
\pgfpathlineto{\pgfqpoint{5.086667in}{2.010001in}}%
\pgfpathlineto{\pgfqpoint{5.091208in}{2.010001in}}%
\pgfpathlineto{\pgfqpoint{5.091208in}{2.007051in}}%
\pgfpathmoveto{\pgfqpoint{5.091208in}{2.007051in}}%
\pgfpathlineto{\pgfqpoint{5.091208in}{2.007051in}}%
\pgfpathlineto{\pgfqpoint{5.091208in}{2.010001in}}%
\pgfpathlineto{\pgfqpoint{5.095749in}{2.010001in}}%
\pgfpathlineto{\pgfqpoint{5.095749in}{2.007051in}}%
\pgfpathmoveto{\pgfqpoint{5.095749in}{2.007051in}}%
\pgfpathlineto{\pgfqpoint{5.095749in}{2.007051in}}%
\pgfpathlineto{\pgfqpoint{5.095749in}{2.010001in}}%
\pgfpathlineto{\pgfqpoint{5.100290in}{2.010001in}}%
\pgfpathlineto{\pgfqpoint{5.100290in}{2.007051in}}%
\pgfpathmoveto{\pgfqpoint{5.100290in}{2.007051in}}%
\pgfpathlineto{\pgfqpoint{5.100290in}{2.007051in}}%
\pgfpathlineto{\pgfqpoint{5.100290in}{2.010001in}}%
\pgfpathlineto{\pgfqpoint{5.104831in}{2.010001in}}%
\pgfpathlineto{\pgfqpoint{5.104831in}{2.007051in}}%
\pgfpathmoveto{\pgfqpoint{5.104831in}{2.007051in}}%
\pgfpathlineto{\pgfqpoint{5.104831in}{2.007051in}}%
\pgfpathlineto{\pgfqpoint{5.104831in}{2.010001in}}%
\pgfpathlineto{\pgfqpoint{5.109372in}{2.010001in}}%
\pgfpathlineto{\pgfqpoint{5.109372in}{2.007051in}}%
\pgfpathmoveto{\pgfqpoint{5.109372in}{2.007051in}}%
\pgfpathlineto{\pgfqpoint{5.109372in}{2.007051in}}%
\pgfpathlineto{\pgfqpoint{5.109372in}{2.010001in}}%
\pgfpathlineto{\pgfqpoint{5.113913in}{2.010001in}}%
\pgfpathlineto{\pgfqpoint{5.113913in}{2.007051in}}%
\pgfpathmoveto{\pgfqpoint{5.113913in}{2.007051in}}%
\pgfpathlineto{\pgfqpoint{5.113913in}{2.007051in}}%
\pgfpathlineto{\pgfqpoint{5.113913in}{2.010001in}}%
\pgfpathlineto{\pgfqpoint{5.118454in}{2.010001in}}%
\pgfpathlineto{\pgfqpoint{5.118454in}{2.007051in}}%
\pgfpathmoveto{\pgfqpoint{5.118454in}{2.007051in}}%
\pgfpathlineto{\pgfqpoint{5.118454in}{2.007051in}}%
\pgfpathlineto{\pgfqpoint{5.118454in}{2.010001in}}%
\pgfpathlineto{\pgfqpoint{5.122995in}{2.010001in}}%
\pgfpathlineto{\pgfqpoint{5.122995in}{2.007051in}}%
\pgfpathmoveto{\pgfqpoint{5.122995in}{2.007051in}}%
\pgfpathlineto{\pgfqpoint{5.122995in}{2.007051in}}%
\pgfpathlineto{\pgfqpoint{5.122995in}{2.010001in}}%
\pgfpathlineto{\pgfqpoint{5.127536in}{2.010001in}}%
\pgfpathlineto{\pgfqpoint{5.127536in}{2.007051in}}%
\pgfpathmoveto{\pgfqpoint{5.127536in}{2.007051in}}%
\pgfpathlineto{\pgfqpoint{5.127536in}{2.007051in}}%
\pgfpathlineto{\pgfqpoint{5.127536in}{2.010001in}}%
\pgfpathlineto{\pgfqpoint{5.132077in}{2.010001in}}%
\pgfpathlineto{\pgfqpoint{5.132077in}{2.007051in}}%
\pgfpathmoveto{\pgfqpoint{5.132077in}{2.007051in}}%
\pgfpathlineto{\pgfqpoint{5.132077in}{2.007051in}}%
\pgfpathlineto{\pgfqpoint{5.132077in}{2.010001in}}%
\pgfpathlineto{\pgfqpoint{5.136618in}{2.010001in}}%
\pgfpathlineto{\pgfqpoint{5.136618in}{2.007051in}}%
\pgfpathmoveto{\pgfqpoint{5.136618in}{2.007051in}}%
\pgfpathlineto{\pgfqpoint{5.136618in}{2.007051in}}%
\pgfpathlineto{\pgfqpoint{5.136618in}{2.010001in}}%
\pgfpathlineto{\pgfqpoint{5.141159in}{2.010001in}}%
\pgfpathlineto{\pgfqpoint{5.141159in}{2.007051in}}%
\pgfpathmoveto{\pgfqpoint{5.141159in}{2.007051in}}%
\pgfpathlineto{\pgfqpoint{5.141159in}{2.007051in}}%
\pgfpathlineto{\pgfqpoint{5.141159in}{2.010001in}}%
\pgfpathlineto{\pgfqpoint{5.145700in}{2.010001in}}%
\pgfpathlineto{\pgfqpoint{5.145700in}{2.007051in}}%
\pgfpathmoveto{\pgfqpoint{5.145700in}{2.007051in}}%
\pgfpathlineto{\pgfqpoint{5.145700in}{2.007051in}}%
\pgfpathlineto{\pgfqpoint{5.145700in}{2.010001in}}%
\pgfpathlineto{\pgfqpoint{5.150241in}{2.010001in}}%
\pgfpathlineto{\pgfqpoint{5.150241in}{2.007051in}}%
\pgfpathmoveto{\pgfqpoint{5.150241in}{2.007051in}}%
\pgfpathlineto{\pgfqpoint{5.150241in}{2.007051in}}%
\pgfpathlineto{\pgfqpoint{5.150241in}{2.010001in}}%
\pgfpathlineto{\pgfqpoint{5.154782in}{2.010001in}}%
\pgfpathlineto{\pgfqpoint{5.154782in}{2.007051in}}%
\pgfpathmoveto{\pgfqpoint{5.154782in}{2.007051in}}%
\pgfpathlineto{\pgfqpoint{5.154782in}{2.007051in}}%
\pgfpathlineto{\pgfqpoint{5.154782in}{2.010001in}}%
\pgfpathlineto{\pgfqpoint{5.159323in}{2.010001in}}%
\pgfpathlineto{\pgfqpoint{5.159323in}{2.007051in}}%
\pgfpathmoveto{\pgfqpoint{5.159323in}{2.007051in}}%
\pgfpathlineto{\pgfqpoint{5.159323in}{2.007051in}}%
\pgfpathlineto{\pgfqpoint{5.159323in}{2.010001in}}%
\pgfpathlineto{\pgfqpoint{5.163864in}{2.010001in}}%
\pgfpathlineto{\pgfqpoint{5.163864in}{2.007051in}}%
\pgfpathmoveto{\pgfqpoint{5.163864in}{2.007051in}}%
\pgfpathlineto{\pgfqpoint{5.163864in}{2.007051in}}%
\pgfpathlineto{\pgfqpoint{5.163864in}{2.010001in}}%
\pgfpathlineto{\pgfqpoint{5.168405in}{2.010001in}}%
\pgfpathlineto{\pgfqpoint{5.168405in}{2.007051in}}%
\pgfpathmoveto{\pgfqpoint{5.168405in}{2.007051in}}%
\pgfpathlineto{\pgfqpoint{5.168405in}{2.007051in}}%
\pgfpathlineto{\pgfqpoint{5.168405in}{2.010001in}}%
\pgfpathlineto{\pgfqpoint{5.172946in}{2.010001in}}%
\pgfpathlineto{\pgfqpoint{5.172946in}{2.007051in}}%
\pgfpathmoveto{\pgfqpoint{5.172946in}{2.007051in}}%
\pgfpathlineto{\pgfqpoint{5.172946in}{2.007051in}}%
\pgfpathlineto{\pgfqpoint{5.172946in}{2.010001in}}%
\pgfpathlineto{\pgfqpoint{5.177487in}{2.010001in}}%
\pgfpathlineto{\pgfqpoint{5.177487in}{2.007051in}}%
\pgfpathmoveto{\pgfqpoint{5.177487in}{2.007051in}}%
\pgfpathlineto{\pgfqpoint{5.177487in}{2.007051in}}%
\pgfpathlineto{\pgfqpoint{5.177487in}{2.010001in}}%
\pgfpathlineto{\pgfqpoint{5.182028in}{2.010001in}}%
\pgfpathlineto{\pgfqpoint{5.182028in}{2.007051in}}%
\pgfpathmoveto{\pgfqpoint{5.182028in}{2.007051in}}%
\pgfpathlineto{\pgfqpoint{5.182028in}{2.007051in}}%
\pgfpathlineto{\pgfqpoint{5.182028in}{2.010001in}}%
\pgfpathlineto{\pgfqpoint{5.186569in}{2.010001in}}%
\pgfpathlineto{\pgfqpoint{5.186569in}{2.007051in}}%
\pgfpathmoveto{\pgfqpoint{5.186569in}{2.007051in}}%
\pgfpathlineto{\pgfqpoint{5.186569in}{2.007051in}}%
\pgfpathlineto{\pgfqpoint{5.186569in}{2.010001in}}%
\pgfpathlineto{\pgfqpoint{5.191110in}{2.010001in}}%
\pgfpathlineto{\pgfqpoint{5.191110in}{2.007051in}}%
\pgfpathmoveto{\pgfqpoint{5.191110in}{2.007051in}}%
\pgfpathlineto{\pgfqpoint{5.191110in}{2.007051in}}%
\pgfpathlineto{\pgfqpoint{5.191110in}{2.010001in}}%
\pgfpathlineto{\pgfqpoint{5.195651in}{2.010001in}}%
\pgfpathlineto{\pgfqpoint{5.195651in}{2.007051in}}%
\pgfpathmoveto{\pgfqpoint{5.195651in}{2.007051in}}%
\pgfpathlineto{\pgfqpoint{5.195651in}{2.007051in}}%
\pgfpathlineto{\pgfqpoint{5.195651in}{2.010001in}}%
\pgfpathlineto{\pgfqpoint{5.200192in}{2.010001in}}%
\pgfpathlineto{\pgfqpoint{5.200192in}{2.007051in}}%
\pgfpathmoveto{\pgfqpoint{5.200192in}{2.007051in}}%
\pgfpathlineto{\pgfqpoint{5.200192in}{2.007051in}}%
\pgfpathlineto{\pgfqpoint{5.200192in}{2.010001in}}%
\pgfpathlineto{\pgfqpoint{5.204733in}{2.010001in}}%
\pgfpathlineto{\pgfqpoint{5.204733in}{2.007051in}}%
\pgfpathmoveto{\pgfqpoint{5.204733in}{2.007051in}}%
\pgfpathlineto{\pgfqpoint{5.204733in}{2.007051in}}%
\pgfpathlineto{\pgfqpoint{5.204733in}{2.010001in}}%
\pgfpathlineto{\pgfqpoint{5.209274in}{2.010001in}}%
\pgfpathlineto{\pgfqpoint{5.209274in}{2.007051in}}%
\pgfpathmoveto{\pgfqpoint{5.209274in}{2.007051in}}%
\pgfpathlineto{\pgfqpoint{5.209274in}{2.007051in}}%
\pgfpathlineto{\pgfqpoint{5.209274in}{2.010001in}}%
\pgfpathlineto{\pgfqpoint{5.213815in}{2.010001in}}%
\pgfpathlineto{\pgfqpoint{5.213815in}{2.007051in}}%
\pgfpathmoveto{\pgfqpoint{5.213815in}{2.007051in}}%
\pgfpathlineto{\pgfqpoint{5.213815in}{2.007051in}}%
\pgfpathlineto{\pgfqpoint{5.213815in}{2.010001in}}%
\pgfpathlineto{\pgfqpoint{5.218356in}{2.010001in}}%
\pgfpathlineto{\pgfqpoint{5.218356in}{2.007051in}}%
\pgfpathmoveto{\pgfqpoint{5.218356in}{2.007051in}}%
\pgfpathlineto{\pgfqpoint{5.218356in}{2.007051in}}%
\pgfpathlineto{\pgfqpoint{5.218356in}{2.010001in}}%
\pgfpathlineto{\pgfqpoint{5.222897in}{2.010001in}}%
\pgfpathlineto{\pgfqpoint{5.222897in}{2.007051in}}%
\pgfpathmoveto{\pgfqpoint{5.222897in}{2.007051in}}%
\pgfpathlineto{\pgfqpoint{5.222897in}{2.007051in}}%
\pgfpathlineto{\pgfqpoint{5.222897in}{2.010001in}}%
\pgfpathlineto{\pgfqpoint{5.227438in}{2.010001in}}%
\pgfpathlineto{\pgfqpoint{5.227438in}{2.007051in}}%
\pgfpathmoveto{\pgfqpoint{5.227438in}{2.007051in}}%
\pgfpathlineto{\pgfqpoint{5.227438in}{2.007051in}}%
\pgfpathlineto{\pgfqpoint{5.227438in}{2.010001in}}%
\pgfpathlineto{\pgfqpoint{5.231979in}{2.010001in}}%
\pgfpathlineto{\pgfqpoint{5.231979in}{2.007051in}}%
\pgfpathmoveto{\pgfqpoint{5.231979in}{2.007051in}}%
\pgfpathlineto{\pgfqpoint{5.231979in}{2.007051in}}%
\pgfpathlineto{\pgfqpoint{5.231979in}{2.010001in}}%
\pgfpathlineto{\pgfqpoint{5.236520in}{2.010001in}}%
\pgfpathlineto{\pgfqpoint{5.236520in}{2.007051in}}%
\pgfpathmoveto{\pgfqpoint{5.236520in}{2.007051in}}%
\pgfpathlineto{\pgfqpoint{5.236520in}{2.007051in}}%
\pgfpathlineto{\pgfqpoint{5.236520in}{2.010001in}}%
\pgfpathlineto{\pgfqpoint{5.241061in}{2.010001in}}%
\pgfpathlineto{\pgfqpoint{5.241061in}{2.007051in}}%
\pgfpathmoveto{\pgfqpoint{5.241061in}{2.007051in}}%
\pgfpathlineto{\pgfqpoint{5.241061in}{2.007051in}}%
\pgfpathlineto{\pgfqpoint{5.241061in}{2.010001in}}%
\pgfpathlineto{\pgfqpoint{5.245602in}{2.010001in}}%
\pgfpathlineto{\pgfqpoint{5.245602in}{2.007051in}}%
\pgfpathmoveto{\pgfqpoint{5.245602in}{2.007051in}}%
\pgfpathlineto{\pgfqpoint{5.245602in}{2.007051in}}%
\pgfpathlineto{\pgfqpoint{5.245602in}{2.010001in}}%
\pgfpathlineto{\pgfqpoint{5.250143in}{2.010001in}}%
\pgfpathlineto{\pgfqpoint{5.250143in}{2.007051in}}%
\pgfpathmoveto{\pgfqpoint{5.250143in}{2.007051in}}%
\pgfpathlineto{\pgfqpoint{5.250143in}{2.007051in}}%
\pgfpathlineto{\pgfqpoint{5.250143in}{2.010001in}}%
\pgfpathlineto{\pgfqpoint{5.254684in}{2.010001in}}%
\pgfpathlineto{\pgfqpoint{5.254684in}{2.007051in}}%
\pgfpathmoveto{\pgfqpoint{5.254684in}{2.007051in}}%
\pgfpathlineto{\pgfqpoint{5.254684in}{2.007051in}}%
\pgfpathlineto{\pgfqpoint{5.254684in}{2.010001in}}%
\pgfpathlineto{\pgfqpoint{5.259225in}{2.010001in}}%
\pgfpathlineto{\pgfqpoint{5.259225in}{2.007051in}}%
\pgfpathmoveto{\pgfqpoint{5.259225in}{2.007051in}}%
\pgfpathlineto{\pgfqpoint{5.259225in}{2.007051in}}%
\pgfpathlineto{\pgfqpoint{5.259225in}{2.010001in}}%
\pgfpathlineto{\pgfqpoint{5.263766in}{2.010001in}}%
\pgfpathlineto{\pgfqpoint{5.263766in}{2.007051in}}%
\pgfpathmoveto{\pgfqpoint{5.263766in}{2.007051in}}%
\pgfpathlineto{\pgfqpoint{5.263766in}{2.007051in}}%
\pgfpathlineto{\pgfqpoint{5.263766in}{2.010001in}}%
\pgfpathlineto{\pgfqpoint{5.268307in}{2.010001in}}%
\pgfpathlineto{\pgfqpoint{5.268307in}{2.007051in}}%
\pgfpathmoveto{\pgfqpoint{5.268307in}{2.007051in}}%
\pgfpathlineto{\pgfqpoint{5.268307in}{2.007051in}}%
\pgfpathlineto{\pgfqpoint{5.268307in}{2.010001in}}%
\pgfpathlineto{\pgfqpoint{5.272848in}{2.010001in}}%
\pgfpathlineto{\pgfqpoint{5.272848in}{2.007051in}}%
\pgfpathmoveto{\pgfqpoint{5.272848in}{2.007051in}}%
\pgfpathlineto{\pgfqpoint{5.272848in}{2.007051in}}%
\pgfpathlineto{\pgfqpoint{5.272848in}{2.010001in}}%
\pgfpathlineto{\pgfqpoint{5.277389in}{2.010001in}}%
\pgfpathlineto{\pgfqpoint{5.277389in}{2.007051in}}%
\pgfpathmoveto{\pgfqpoint{5.277389in}{2.007051in}}%
\pgfpathlineto{\pgfqpoint{5.277389in}{2.007051in}}%
\pgfpathlineto{\pgfqpoint{5.277389in}{2.010001in}}%
\pgfpathlineto{\pgfqpoint{5.281930in}{2.010001in}}%
\pgfpathlineto{\pgfqpoint{5.281930in}{2.007051in}}%
\pgfpathmoveto{\pgfqpoint{5.281930in}{2.007051in}}%
\pgfpathlineto{\pgfqpoint{5.281930in}{2.007051in}}%
\pgfpathlineto{\pgfqpoint{5.281930in}{2.010001in}}%
\pgfpathlineto{\pgfqpoint{5.286471in}{2.010001in}}%
\pgfpathlineto{\pgfqpoint{5.286471in}{2.007051in}}%
\pgfpathmoveto{\pgfqpoint{5.286471in}{2.007051in}}%
\pgfpathlineto{\pgfqpoint{5.286471in}{2.007051in}}%
\pgfpathlineto{\pgfqpoint{5.286471in}{2.010001in}}%
\pgfpathlineto{\pgfqpoint{5.291012in}{2.010001in}}%
\pgfpathlineto{\pgfqpoint{5.291012in}{2.007051in}}%
\pgfpathmoveto{\pgfqpoint{5.291012in}{2.007051in}}%
\pgfpathlineto{\pgfqpoint{5.291012in}{2.007051in}}%
\pgfpathlineto{\pgfqpoint{5.291012in}{2.010001in}}%
\pgfpathlineto{\pgfqpoint{5.295553in}{2.010001in}}%
\pgfpathlineto{\pgfqpoint{5.295553in}{2.007051in}}%
\pgfpathmoveto{\pgfqpoint{5.295553in}{2.007051in}}%
\pgfpathlineto{\pgfqpoint{5.295553in}{2.007051in}}%
\pgfpathlineto{\pgfqpoint{5.295553in}{2.010001in}}%
\pgfpathlineto{\pgfqpoint{5.300094in}{2.010001in}}%
\pgfpathlineto{\pgfqpoint{5.300094in}{2.007051in}}%
\pgfpathmoveto{\pgfqpoint{5.300094in}{2.007051in}}%
\pgfpathlineto{\pgfqpoint{5.300094in}{2.007051in}}%
\pgfpathlineto{\pgfqpoint{5.300094in}{2.010001in}}%
\pgfpathlineto{\pgfqpoint{5.304635in}{2.010001in}}%
\pgfpathlineto{\pgfqpoint{5.304635in}{2.007051in}}%
\pgfpathmoveto{\pgfqpoint{5.304635in}{2.007051in}}%
\pgfpathlineto{\pgfqpoint{5.304635in}{2.007051in}}%
\pgfpathlineto{\pgfqpoint{5.304635in}{2.010001in}}%
\pgfpathlineto{\pgfqpoint{5.309176in}{2.010001in}}%
\pgfpathlineto{\pgfqpoint{5.309176in}{2.007051in}}%
\pgfpathmoveto{\pgfqpoint{5.309176in}{2.007051in}}%
\pgfpathlineto{\pgfqpoint{5.309176in}{2.007051in}}%
\pgfpathlineto{\pgfqpoint{5.309176in}{2.010001in}}%
\pgfpathlineto{\pgfqpoint{5.313717in}{2.010001in}}%
\pgfpathlineto{\pgfqpoint{5.313717in}{2.007051in}}%
\pgfpathmoveto{\pgfqpoint{5.313717in}{2.007051in}}%
\pgfpathlineto{\pgfqpoint{5.313717in}{2.007051in}}%
\pgfpathlineto{\pgfqpoint{5.313717in}{2.010001in}}%
\pgfpathlineto{\pgfqpoint{5.318258in}{2.010001in}}%
\pgfpathlineto{\pgfqpoint{5.318258in}{2.007051in}}%
\pgfpathmoveto{\pgfqpoint{5.318258in}{2.007051in}}%
\pgfpathlineto{\pgfqpoint{5.318258in}{2.007051in}}%
\pgfpathlineto{\pgfqpoint{5.318258in}{2.010001in}}%
\pgfpathlineto{\pgfqpoint{5.322799in}{2.010001in}}%
\pgfpathlineto{\pgfqpoint{5.322799in}{2.007051in}}%
\pgfpathmoveto{\pgfqpoint{5.322799in}{2.007051in}}%
\pgfpathlineto{\pgfqpoint{5.322799in}{2.007051in}}%
\pgfpathlineto{\pgfqpoint{5.322799in}{2.010001in}}%
\pgfpathlineto{\pgfqpoint{5.327340in}{2.010001in}}%
\pgfpathlineto{\pgfqpoint{5.327340in}{2.007051in}}%
\pgfpathmoveto{\pgfqpoint{5.327340in}{2.007051in}}%
\pgfpathlineto{\pgfqpoint{5.327340in}{2.007051in}}%
\pgfpathlineto{\pgfqpoint{5.327340in}{2.010001in}}%
\pgfpathlineto{\pgfqpoint{5.331881in}{2.010001in}}%
\pgfpathlineto{\pgfqpoint{5.331881in}{2.007051in}}%
\pgfpathmoveto{\pgfqpoint{5.331881in}{2.007051in}}%
\pgfpathlineto{\pgfqpoint{5.331881in}{2.007051in}}%
\pgfpathlineto{\pgfqpoint{5.331881in}{2.010001in}}%
\pgfpathlineto{\pgfqpoint{5.336422in}{2.010001in}}%
\pgfpathlineto{\pgfqpoint{5.336422in}{2.007051in}}%
\pgfpathmoveto{\pgfqpoint{5.336422in}{2.007051in}}%
\pgfpathlineto{\pgfqpoint{5.336422in}{2.007051in}}%
\pgfpathlineto{\pgfqpoint{5.336422in}{2.010001in}}%
\pgfpathlineto{\pgfqpoint{5.340963in}{2.010001in}}%
\pgfpathlineto{\pgfqpoint{5.340963in}{2.007051in}}%
\pgfpathmoveto{\pgfqpoint{5.340963in}{2.007051in}}%
\pgfpathlineto{\pgfqpoint{5.340963in}{2.007051in}}%
\pgfpathlineto{\pgfqpoint{5.340963in}{2.010001in}}%
\pgfpathlineto{\pgfqpoint{5.345504in}{2.010001in}}%
\pgfpathlineto{\pgfqpoint{5.345504in}{2.007051in}}%
\pgfpathmoveto{\pgfqpoint{5.345504in}{2.007051in}}%
\pgfpathlineto{\pgfqpoint{5.345504in}{2.007051in}}%
\pgfpathlineto{\pgfqpoint{5.345504in}{2.010001in}}%
\pgfpathlineto{\pgfqpoint{5.350045in}{2.010001in}}%
\pgfpathlineto{\pgfqpoint{5.350045in}{2.007051in}}%
\pgfpathmoveto{\pgfqpoint{5.350045in}{2.007051in}}%
\pgfpathlineto{\pgfqpoint{5.350045in}{2.007051in}}%
\pgfpathlineto{\pgfqpoint{5.350045in}{2.010001in}}%
\pgfpathlineto{\pgfqpoint{5.354586in}{2.010001in}}%
\pgfpathlineto{\pgfqpoint{5.354586in}{2.007051in}}%
\pgfpathmoveto{\pgfqpoint{5.354586in}{2.007051in}}%
\pgfpathlineto{\pgfqpoint{5.354586in}{2.007051in}}%
\pgfpathlineto{\pgfqpoint{5.354586in}{2.010001in}}%
\pgfpathlineto{\pgfqpoint{5.359127in}{2.010001in}}%
\pgfpathlineto{\pgfqpoint{5.359127in}{2.007051in}}%
\pgfpathmoveto{\pgfqpoint{5.359127in}{2.007051in}}%
\pgfpathlineto{\pgfqpoint{5.359127in}{2.007051in}}%
\pgfpathlineto{\pgfqpoint{5.359127in}{2.010001in}}%
\pgfpathlineto{\pgfqpoint{5.363668in}{2.010001in}}%
\pgfpathlineto{\pgfqpoint{5.363668in}{2.007051in}}%
\pgfpathmoveto{\pgfqpoint{5.363668in}{2.007051in}}%
\pgfpathlineto{\pgfqpoint{5.363668in}{2.007051in}}%
\pgfpathlineto{\pgfqpoint{5.363668in}{2.010001in}}%
\pgfpathlineto{\pgfqpoint{5.368209in}{2.010001in}}%
\pgfpathlineto{\pgfqpoint{5.368209in}{2.007051in}}%
\pgfpathmoveto{\pgfqpoint{5.368209in}{2.007051in}}%
\pgfpathlineto{\pgfqpoint{5.368209in}{2.007051in}}%
\pgfpathlineto{\pgfqpoint{5.368209in}{2.010001in}}%
\pgfpathlineto{\pgfqpoint{5.372750in}{2.010001in}}%
\pgfpathlineto{\pgfqpoint{5.372750in}{2.007051in}}%
\pgfpathmoveto{\pgfqpoint{5.372750in}{2.007051in}}%
\pgfpathlineto{\pgfqpoint{5.372750in}{2.007051in}}%
\pgfpathlineto{\pgfqpoint{5.372750in}{2.010001in}}%
\pgfpathlineto{\pgfqpoint{5.377291in}{2.010001in}}%
\pgfpathlineto{\pgfqpoint{5.377291in}{2.007051in}}%
\pgfpathmoveto{\pgfqpoint{5.377291in}{2.007051in}}%
\pgfpathlineto{\pgfqpoint{5.377291in}{2.007051in}}%
\pgfpathlineto{\pgfqpoint{5.377291in}{2.010001in}}%
\pgfpathlineto{\pgfqpoint{5.381832in}{2.010001in}}%
\pgfpathlineto{\pgfqpoint{5.381832in}{2.007051in}}%
\pgfpathmoveto{\pgfqpoint{5.381832in}{2.007051in}}%
\pgfpathlineto{\pgfqpoint{5.381832in}{2.007051in}}%
\pgfpathlineto{\pgfqpoint{5.381832in}{2.010001in}}%
\pgfpathlineto{\pgfqpoint{5.386373in}{2.010001in}}%
\pgfpathlineto{\pgfqpoint{5.386373in}{2.007051in}}%
\pgfpathmoveto{\pgfqpoint{5.386373in}{2.007051in}}%
\pgfpathlineto{\pgfqpoint{5.386373in}{2.007051in}}%
\pgfpathlineto{\pgfqpoint{5.386373in}{2.010001in}}%
\pgfpathlineto{\pgfqpoint{5.390914in}{2.010001in}}%
\pgfpathlineto{\pgfqpoint{5.390914in}{2.007051in}}%
\pgfpathmoveto{\pgfqpoint{5.390914in}{2.007051in}}%
\pgfpathlineto{\pgfqpoint{5.390914in}{2.007051in}}%
\pgfpathlineto{\pgfqpoint{5.390914in}{2.010001in}}%
\pgfpathlineto{\pgfqpoint{5.395455in}{2.010001in}}%
\pgfpathlineto{\pgfqpoint{5.395455in}{2.007051in}}%
\pgfpathmoveto{\pgfqpoint{5.395455in}{2.007051in}}%
\pgfpathlineto{\pgfqpoint{5.395455in}{2.007051in}}%
\pgfpathlineto{\pgfqpoint{5.395455in}{2.010001in}}%
\pgfpathlineto{\pgfqpoint{5.399996in}{2.010001in}}%
\pgfpathlineto{\pgfqpoint{5.399996in}{2.007051in}}%
\pgfpathclose%
\pgfusepath{fill}%
\end{pgfscope}%
\begin{pgfscope}%
\pgfsetbuttcap%
\pgfsetroundjoin%
\definecolor{currentfill}{rgb}{0.000000,0.000000,0.000000}%
\pgfsetfillcolor{currentfill}%
\pgfsetlinewidth{0.803000pt}%
\definecolor{currentstroke}{rgb}{0.000000,0.000000,0.000000}%
\pgfsetstrokecolor{currentstroke}%
\pgfsetdash{}{0pt}%
\pgfsys@defobject{currentmarker}{\pgfqpoint{0.000000in}{-0.048611in}}{\pgfqpoint{0.000000in}{0.000000in}}{%
\pgfpathmoveto{\pgfqpoint{0.000000in}{0.000000in}}%
\pgfpathlineto{\pgfqpoint{0.000000in}{-0.048611in}}%
\pgfusepath{stroke,fill}%
}%
\begin{pgfscope}%
\pgfsys@transformshift{1.215000in}{2.010000in}%
\pgfsys@useobject{currentmarker}{}%
\end{pgfscope}%
\end{pgfscope}%
\begin{pgfscope}%
\definecolor{textcolor}{rgb}{0.000000,0.000000,0.000000}%
\pgfsetstrokecolor{textcolor}%
\pgfsetfillcolor{textcolor}%
\pgftext[x=1.215000in,y=1.912778in,,top]{\color{textcolor}\sffamily\fontsize{10.000000}{12.000000}\selectfont −4}%
\end{pgfscope}%
\begin{pgfscope}%
\pgfsetbuttcap%
\pgfsetroundjoin%
\definecolor{currentfill}{rgb}{0.000000,0.000000,0.000000}%
\pgfsetfillcolor{currentfill}%
\pgfsetlinewidth{0.803000pt}%
\definecolor{currentstroke}{rgb}{0.000000,0.000000,0.000000}%
\pgfsetstrokecolor{currentstroke}%
\pgfsetdash{}{0pt}%
\pgfsys@defobject{currentmarker}{\pgfqpoint{0.000000in}{-0.048611in}}{\pgfqpoint{0.000000in}{0.000000in}}{%
\pgfpathmoveto{\pgfqpoint{0.000000in}{0.000000in}}%
\pgfpathlineto{\pgfqpoint{0.000000in}{-0.048611in}}%
\pgfusepath{stroke,fill}%
}%
\begin{pgfscope}%
\pgfsys@transformshift{2.145000in}{2.010000in}%
\pgfsys@useobject{currentmarker}{}%
\end{pgfscope}%
\end{pgfscope}%
\begin{pgfscope}%
\definecolor{textcolor}{rgb}{0.000000,0.000000,0.000000}%
\pgfsetstrokecolor{textcolor}%
\pgfsetfillcolor{textcolor}%
\pgftext[x=2.145000in,y=1.912778in,,top]{\color{textcolor}\sffamily\fontsize{10.000000}{12.000000}\selectfont −2}%
\end{pgfscope}%
\begin{pgfscope}%
\pgfsetbuttcap%
\pgfsetroundjoin%
\definecolor{currentfill}{rgb}{0.000000,0.000000,0.000000}%
\pgfsetfillcolor{currentfill}%
\pgfsetlinewidth{0.803000pt}%
\definecolor{currentstroke}{rgb}{0.000000,0.000000,0.000000}%
\pgfsetstrokecolor{currentstroke}%
\pgfsetdash{}{0pt}%
\pgfsys@defobject{currentmarker}{\pgfqpoint{0.000000in}{-0.048611in}}{\pgfqpoint{0.000000in}{0.000000in}}{%
\pgfpathmoveto{\pgfqpoint{0.000000in}{0.000000in}}%
\pgfpathlineto{\pgfqpoint{0.000000in}{-0.048611in}}%
\pgfusepath{stroke,fill}%
}%
\begin{pgfscope}%
\pgfsys@transformshift{3.075000in}{2.010000in}%
\pgfsys@useobject{currentmarker}{}%
\end{pgfscope}%
\end{pgfscope}%
\begin{pgfscope}%
\definecolor{textcolor}{rgb}{0.000000,0.000000,0.000000}%
\pgfsetstrokecolor{textcolor}%
\pgfsetfillcolor{textcolor}%
\pgftext[x=3.075000in,y=1.912778in,,top]{\color{textcolor}\sffamily\fontsize{10.000000}{12.000000}\selectfont 0}%
\end{pgfscope}%
\begin{pgfscope}%
\pgfsetbuttcap%
\pgfsetroundjoin%
\definecolor{currentfill}{rgb}{0.000000,0.000000,0.000000}%
\pgfsetfillcolor{currentfill}%
\pgfsetlinewidth{0.803000pt}%
\definecolor{currentstroke}{rgb}{0.000000,0.000000,0.000000}%
\pgfsetstrokecolor{currentstroke}%
\pgfsetdash{}{0pt}%
\pgfsys@defobject{currentmarker}{\pgfqpoint{0.000000in}{-0.048611in}}{\pgfqpoint{0.000000in}{0.000000in}}{%
\pgfpathmoveto{\pgfqpoint{0.000000in}{0.000000in}}%
\pgfpathlineto{\pgfqpoint{0.000000in}{-0.048611in}}%
\pgfusepath{stroke,fill}%
}%
\begin{pgfscope}%
\pgfsys@transformshift{4.005000in}{2.010000in}%
\pgfsys@useobject{currentmarker}{}%
\end{pgfscope}%
\end{pgfscope}%
\begin{pgfscope}%
\definecolor{textcolor}{rgb}{0.000000,0.000000,0.000000}%
\pgfsetstrokecolor{textcolor}%
\pgfsetfillcolor{textcolor}%
\pgftext[x=4.005000in,y=1.912778in,,top]{\color{textcolor}\sffamily\fontsize{10.000000}{12.000000}\selectfont 2}%
\end{pgfscope}%
\begin{pgfscope}%
\pgfsetbuttcap%
\pgfsetroundjoin%
\definecolor{currentfill}{rgb}{0.000000,0.000000,0.000000}%
\pgfsetfillcolor{currentfill}%
\pgfsetlinewidth{0.803000pt}%
\definecolor{currentstroke}{rgb}{0.000000,0.000000,0.000000}%
\pgfsetstrokecolor{currentstroke}%
\pgfsetdash{}{0pt}%
\pgfsys@defobject{currentmarker}{\pgfqpoint{0.000000in}{-0.048611in}}{\pgfqpoint{0.000000in}{0.000000in}}{%
\pgfpathmoveto{\pgfqpoint{0.000000in}{0.000000in}}%
\pgfpathlineto{\pgfqpoint{0.000000in}{-0.048611in}}%
\pgfusepath{stroke,fill}%
}%
\begin{pgfscope}%
\pgfsys@transformshift{4.935000in}{2.010000in}%
\pgfsys@useobject{currentmarker}{}%
\end{pgfscope}%
\end{pgfscope}%
\begin{pgfscope}%
\definecolor{textcolor}{rgb}{0.000000,0.000000,0.000000}%
\pgfsetstrokecolor{textcolor}%
\pgfsetfillcolor{textcolor}%
\pgftext[x=4.935000in,y=1.912778in,,top]{\color{textcolor}\sffamily\fontsize{10.000000}{12.000000}\selectfont 4}%
\end{pgfscope}%
\begin{pgfscope}%
\definecolor{textcolor}{rgb}{0.000000,0.000000,0.000000}%
\pgfsetstrokecolor{textcolor}%
\pgfsetfillcolor{textcolor}%
\pgftext[x=5.400000in,y=1.722809in,,top]{\color{textcolor}\sffamily\fontsize{10.000000}{12.000000}\selectfont x}%
\end{pgfscope}%
\begin{pgfscope}%
\pgfsetbuttcap%
\pgfsetroundjoin%
\definecolor{currentfill}{rgb}{0.000000,0.000000,0.000000}%
\pgfsetfillcolor{currentfill}%
\pgfsetlinewidth{0.803000pt}%
\definecolor{currentstroke}{rgb}{0.000000,0.000000,0.000000}%
\pgfsetstrokecolor{currentstroke}%
\pgfsetdash{}{0pt}%
\pgfsys@defobject{currentmarker}{\pgfqpoint{-0.048611in}{0.000000in}}{\pgfqpoint{-0.000000in}{0.000000in}}{%
\pgfpathmoveto{\pgfqpoint{-0.000000in}{0.000000in}}%
\pgfpathlineto{\pgfqpoint{-0.048611in}{0.000000in}}%
\pgfusepath{stroke,fill}%
}%
\begin{pgfscope}%
\pgfsys@transformshift{3.075000in}{0.802000in}%
\pgfsys@useobject{currentmarker}{}%
\end{pgfscope}%
\end{pgfscope}%
\begin{pgfscope}%
\definecolor{textcolor}{rgb}{0.000000,0.000000,0.000000}%
\pgfsetstrokecolor{textcolor}%
\pgfsetfillcolor{textcolor}%
\pgftext[x=2.773039in, y=0.749238in, left, base]{\color{textcolor}\sffamily\fontsize{10.000000}{12.000000}\selectfont −4}%
\end{pgfscope}%
\begin{pgfscope}%
\pgfsetbuttcap%
\pgfsetroundjoin%
\definecolor{currentfill}{rgb}{0.000000,0.000000,0.000000}%
\pgfsetfillcolor{currentfill}%
\pgfsetlinewidth{0.803000pt}%
\definecolor{currentstroke}{rgb}{0.000000,0.000000,0.000000}%
\pgfsetstrokecolor{currentstroke}%
\pgfsetdash{}{0pt}%
\pgfsys@defobject{currentmarker}{\pgfqpoint{-0.048611in}{0.000000in}}{\pgfqpoint{-0.000000in}{0.000000in}}{%
\pgfpathmoveto{\pgfqpoint{-0.000000in}{0.000000in}}%
\pgfpathlineto{\pgfqpoint{-0.048611in}{0.000000in}}%
\pgfusepath{stroke,fill}%
}%
\begin{pgfscope}%
\pgfsys@transformshift{3.075000in}{1.406000in}%
\pgfsys@useobject{currentmarker}{}%
\end{pgfscope}%
\end{pgfscope}%
\begin{pgfscope}%
\definecolor{textcolor}{rgb}{0.000000,0.000000,0.000000}%
\pgfsetstrokecolor{textcolor}%
\pgfsetfillcolor{textcolor}%
\pgftext[x=2.773039in, y=1.353238in, left, base]{\color{textcolor}\sffamily\fontsize{10.000000}{12.000000}\selectfont −2}%
\end{pgfscope}%
\begin{pgfscope}%
\pgfsetbuttcap%
\pgfsetroundjoin%
\definecolor{currentfill}{rgb}{0.000000,0.000000,0.000000}%
\pgfsetfillcolor{currentfill}%
\pgfsetlinewidth{0.803000pt}%
\definecolor{currentstroke}{rgb}{0.000000,0.000000,0.000000}%
\pgfsetstrokecolor{currentstroke}%
\pgfsetdash{}{0pt}%
\pgfsys@defobject{currentmarker}{\pgfqpoint{-0.048611in}{0.000000in}}{\pgfqpoint{-0.000000in}{0.000000in}}{%
\pgfpathmoveto{\pgfqpoint{-0.000000in}{0.000000in}}%
\pgfpathlineto{\pgfqpoint{-0.048611in}{0.000000in}}%
\pgfusepath{stroke,fill}%
}%
\begin{pgfscope}%
\pgfsys@transformshift{3.075000in}{2.010000in}%
\pgfsys@useobject{currentmarker}{}%
\end{pgfscope}%
\end{pgfscope}%
\begin{pgfscope}%
\definecolor{textcolor}{rgb}{0.000000,0.000000,0.000000}%
\pgfsetstrokecolor{textcolor}%
\pgfsetfillcolor{textcolor}%
\pgftext[x=2.889413in, y=1.957238in, left, base]{\color{textcolor}\sffamily\fontsize{10.000000}{12.000000}\selectfont 0}%
\end{pgfscope}%
\begin{pgfscope}%
\pgfsetbuttcap%
\pgfsetroundjoin%
\definecolor{currentfill}{rgb}{0.000000,0.000000,0.000000}%
\pgfsetfillcolor{currentfill}%
\pgfsetlinewidth{0.803000pt}%
\definecolor{currentstroke}{rgb}{0.000000,0.000000,0.000000}%
\pgfsetstrokecolor{currentstroke}%
\pgfsetdash{}{0pt}%
\pgfsys@defobject{currentmarker}{\pgfqpoint{-0.048611in}{0.000000in}}{\pgfqpoint{-0.000000in}{0.000000in}}{%
\pgfpathmoveto{\pgfqpoint{-0.000000in}{0.000000in}}%
\pgfpathlineto{\pgfqpoint{-0.048611in}{0.000000in}}%
\pgfusepath{stroke,fill}%
}%
\begin{pgfscope}%
\pgfsys@transformshift{3.075000in}{2.614000in}%
\pgfsys@useobject{currentmarker}{}%
\end{pgfscope}%
\end{pgfscope}%
\begin{pgfscope}%
\definecolor{textcolor}{rgb}{0.000000,0.000000,0.000000}%
\pgfsetstrokecolor{textcolor}%
\pgfsetfillcolor{textcolor}%
\pgftext[x=2.889413in, y=2.561238in, left, base]{\color{textcolor}\sffamily\fontsize{10.000000}{12.000000}\selectfont 2}%
\end{pgfscope}%
\begin{pgfscope}%
\pgfsetbuttcap%
\pgfsetroundjoin%
\definecolor{currentfill}{rgb}{0.000000,0.000000,0.000000}%
\pgfsetfillcolor{currentfill}%
\pgfsetlinewidth{0.803000pt}%
\definecolor{currentstroke}{rgb}{0.000000,0.000000,0.000000}%
\pgfsetstrokecolor{currentstroke}%
\pgfsetdash{}{0pt}%
\pgfsys@defobject{currentmarker}{\pgfqpoint{-0.048611in}{0.000000in}}{\pgfqpoint{-0.000000in}{0.000000in}}{%
\pgfpathmoveto{\pgfqpoint{-0.000000in}{0.000000in}}%
\pgfpathlineto{\pgfqpoint{-0.048611in}{0.000000in}}%
\pgfusepath{stroke,fill}%
}%
\begin{pgfscope}%
\pgfsys@transformshift{3.075000in}{3.218000in}%
\pgfsys@useobject{currentmarker}{}%
\end{pgfscope}%
\end{pgfscope}%
\begin{pgfscope}%
\definecolor{textcolor}{rgb}{0.000000,0.000000,0.000000}%
\pgfsetstrokecolor{textcolor}%
\pgfsetfillcolor{textcolor}%
\pgftext[x=2.889413in, y=3.165238in, left, base]{\color{textcolor}\sffamily\fontsize{10.000000}{12.000000}\selectfont 4}%
\end{pgfscope}%
\begin{pgfscope}%
\definecolor{textcolor}{rgb}{0.000000,0.000000,0.000000}%
\pgfsetstrokecolor{textcolor}%
\pgfsetfillcolor{textcolor}%
\pgftext[x=2.717483in,y=3.520000in,,bottom,rotate=90.000000]{\color{textcolor}\sffamily\fontsize{10.000000}{12.000000}\selectfont y}%
\end{pgfscope}%
\begin{pgfscope}%
\pgfsetrectcap%
\pgfsetmiterjoin%
\pgfsetlinewidth{0.803000pt}%
\definecolor{currentstroke}{rgb}{0.000000,0.000000,0.000000}%
\pgfsetstrokecolor{currentstroke}%
\pgfsetdash{}{0pt}%
\pgfpathmoveto{\pgfqpoint{3.075000in}{0.500000in}}%
\pgfpathlineto{\pgfqpoint{3.075000in}{3.520000in}}%
\pgfusepath{stroke}%
\end{pgfscope}%
\begin{pgfscope}%
\pgfsetrectcap%
\pgfsetmiterjoin%
\pgfsetlinewidth{0.000000pt}%
\definecolor{currentstroke}{rgb}{0.000000,0.000000,0.000000}%
\pgfsetstrokecolor{currentstroke}%
\pgfsetstrokeopacity{0.000000}%
\pgfsetdash{}{0pt}%
\pgfpathmoveto{\pgfqpoint{5.400000in}{0.500000in}}%
\pgfpathlineto{\pgfqpoint{5.400000in}{3.520000in}}%
\pgfusepath{}%
\end{pgfscope}%
\begin{pgfscope}%
\pgfsetrectcap%
\pgfsetmiterjoin%
\pgfsetlinewidth{0.803000pt}%
\definecolor{currentstroke}{rgb}{0.000000,0.000000,0.000000}%
\pgfsetstrokecolor{currentstroke}%
\pgfsetdash{}{0pt}%
\pgfpathmoveto{\pgfqpoint{0.750000in}{2.010000in}}%
\pgfpathlineto{\pgfqpoint{5.400000in}{2.010000in}}%
\pgfusepath{stroke}%
\end{pgfscope}%
\begin{pgfscope}%
\pgfsetrectcap%
\pgfsetmiterjoin%
\pgfsetlinewidth{0.000000pt}%
\definecolor{currentstroke}{rgb}{0.000000,0.000000,0.000000}%
\pgfsetstrokecolor{currentstroke}%
\pgfsetstrokeopacity{0.000000}%
\pgfsetdash{}{0pt}%
\pgfpathmoveto{\pgfqpoint{0.750000in}{3.520000in}}%
\pgfpathlineto{\pgfqpoint{5.400000in}{3.520000in}}%
\pgfusepath{}%
\end{pgfscope}%
\end{pgfpicture}%
\makeatother%
\endgroup%
} \end{solution} 
 
\part[] $\left\{ {\begin{matrix}
   {6x - 5y \leqslant 30}  \\ 
   {4x + 3y \leqslant 0}  \\ 

 \end{matrix} } \right.$
\begin{solution} \scalebox{.6}{%% Creator: Matplotlib, PGF backend
%%
%% To include the figure in your LaTeX document, write
%%   \input{<filename>.pgf}
%%
%% Make sure the required packages are loaded in your preamble
%%   \usepackage{pgf}
%%
%% and, on pdftex
%%   \usepackage[utf8]{inputenc}\DeclareUnicodeCharacter{2212}{-}
%%
%% or, on luatex and xetex
%%   \usepackage{unicode-math}
%%
%% Figures using additional raster images can only be included by \input if
%% they are in the same directory as the main LaTeX file. For loading figures
%% from other directories you can use the `import` package
%%   \usepackage{import}
%%
%% and then include the figures with
%%   \import{<path to file>}{<filename>.pgf}
%%
%% Matplotlib used the following preamble
%%   \usepackage{fontspec}
%%   \setmainfont{DejaVuSerif.ttf}[Path=/home/hp/Mis_aplicaciones/anaconda3/lib/python3.6/site-packages/matplotlib/mpl-data/fonts/ttf/]
%%   \setsansfont{DejaVuSans.ttf}[Path=/home/hp/Mis_aplicaciones/anaconda3/lib/python3.6/site-packages/matplotlib/mpl-data/fonts/ttf/]
%%   \setmonofont{DejaVuSansMono.ttf}[Path=/home/hp/Mis_aplicaciones/anaconda3/lib/python3.6/site-packages/matplotlib/mpl-data/fonts/ttf/]
%%
\begingroup%
\makeatletter%
\begin{pgfpicture}%
\pgfpathrectangle{\pgfpointorigin}{\pgfqpoint{6.000000in}{4.000000in}}%
\pgfusepath{use as bounding box, clip}%
\begin{pgfscope}%
\pgfsetbuttcap%
\pgfsetmiterjoin%
\pgfsetlinewidth{0.000000pt}%
\definecolor{currentstroke}{rgb}{1.000000,1.000000,1.000000}%
\pgfsetstrokecolor{currentstroke}%
\pgfsetstrokeopacity{0.000000}%
\pgfsetdash{}{0pt}%
\pgfpathmoveto{\pgfqpoint{0.000000in}{0.000000in}}%
\pgfpathlineto{\pgfqpoint{6.000000in}{0.000000in}}%
\pgfpathlineto{\pgfqpoint{6.000000in}{4.000000in}}%
\pgfpathlineto{\pgfqpoint{0.000000in}{4.000000in}}%
\pgfpathclose%
\pgfusepath{}%
\end{pgfscope}%
\begin{pgfscope}%
\pgfsetbuttcap%
\pgfsetmiterjoin%
\definecolor{currentfill}{rgb}{1.000000,1.000000,1.000000}%
\pgfsetfillcolor{currentfill}%
\pgfsetlinewidth{0.000000pt}%
\definecolor{currentstroke}{rgb}{0.000000,0.000000,0.000000}%
\pgfsetstrokecolor{currentstroke}%
\pgfsetstrokeopacity{0.000000}%
\pgfsetdash{}{0pt}%
\pgfpathmoveto{\pgfqpoint{0.750000in}{0.500000in}}%
\pgfpathlineto{\pgfqpoint{5.400000in}{0.500000in}}%
\pgfpathlineto{\pgfqpoint{5.400000in}{3.520000in}}%
\pgfpathlineto{\pgfqpoint{0.750000in}{3.520000in}}%
\pgfpathclose%
\pgfusepath{fill}%
\end{pgfscope}%
\begin{pgfscope}%
\pgfpathrectangle{\pgfqpoint{0.750000in}{0.500000in}}{\pgfqpoint{4.650000in}{3.020000in}}%
\pgfusepath{clip}%
\pgfsetbuttcap%
\pgfsetmiterjoin%
\definecolor{currentfill}{rgb}{0.000000,0.000000,1.000000}%
\pgfsetfillcolor{currentfill}%
\pgfsetlinewidth{0.000000pt}%
\definecolor{currentstroke}{rgb}{0.000000,0.000000,0.000000}%
\pgfsetstrokecolor{currentstroke}%
\pgfsetstrokeopacity{0.000000}%
\pgfsetdash{}{0pt}%
\pgfpathmoveto{\pgfqpoint{0.750003in}{0.499998in}}%
\pgfpathlineto{\pgfqpoint{0.750003in}{0.594373in}}%
\pgfpathlineto{\pgfqpoint{0.895317in}{0.594373in}}%
\pgfpathlineto{\pgfqpoint{0.895317in}{0.499998in}}%
\pgfpathmoveto{\pgfqpoint{0.750003in}{0.594373in}}%
\pgfpathlineto{\pgfqpoint{0.750003in}{0.594373in}}%
\pgfpathlineto{\pgfqpoint{0.750003in}{0.688753in}}%
\pgfpathlineto{\pgfqpoint{0.895317in}{0.688753in}}%
\pgfpathlineto{\pgfqpoint{0.895317in}{0.594373in}}%
\pgfpathmoveto{\pgfqpoint{0.750003in}{0.688753in}}%
\pgfpathlineto{\pgfqpoint{0.750003in}{0.688753in}}%
\pgfpathlineto{\pgfqpoint{0.750003in}{0.783125in}}%
\pgfpathlineto{\pgfqpoint{0.895317in}{0.783125in}}%
\pgfpathlineto{\pgfqpoint{0.895317in}{0.688753in}}%
\pgfpathmoveto{\pgfqpoint{0.750003in}{0.783125in}}%
\pgfpathlineto{\pgfqpoint{0.750003in}{0.783125in}}%
\pgfpathlineto{\pgfqpoint{0.750003in}{0.877501in}}%
\pgfpathlineto{\pgfqpoint{0.895317in}{0.877501in}}%
\pgfpathlineto{\pgfqpoint{0.895317in}{0.783125in}}%
\pgfpathmoveto{\pgfqpoint{0.750003in}{0.877501in}}%
\pgfpathlineto{\pgfqpoint{0.750003in}{0.877501in}}%
\pgfpathlineto{\pgfqpoint{0.750003in}{0.971874in}}%
\pgfpathlineto{\pgfqpoint{0.895317in}{0.971874in}}%
\pgfpathlineto{\pgfqpoint{0.895317in}{0.877501in}}%
\pgfpathmoveto{\pgfqpoint{0.750003in}{0.971874in}}%
\pgfpathlineto{\pgfqpoint{0.750003in}{0.971874in}}%
\pgfpathlineto{\pgfqpoint{0.750003in}{1.066247in}}%
\pgfpathlineto{\pgfqpoint{0.895317in}{1.066247in}}%
\pgfpathlineto{\pgfqpoint{0.895317in}{0.971874in}}%
\pgfpathmoveto{\pgfqpoint{0.750003in}{1.066247in}}%
\pgfpathlineto{\pgfqpoint{0.750003in}{1.066247in}}%
\pgfpathlineto{\pgfqpoint{0.750003in}{1.160624in}}%
\pgfpathlineto{\pgfqpoint{0.895317in}{1.160624in}}%
\pgfpathlineto{\pgfqpoint{0.895317in}{1.066247in}}%
\pgfpathmoveto{\pgfqpoint{0.750003in}{1.160624in}}%
\pgfpathlineto{\pgfqpoint{0.750003in}{1.160624in}}%
\pgfpathlineto{\pgfqpoint{0.750003in}{1.254999in}}%
\pgfpathlineto{\pgfqpoint{0.895317in}{1.254999in}}%
\pgfpathlineto{\pgfqpoint{0.895317in}{1.160624in}}%
\pgfpathmoveto{\pgfqpoint{0.750003in}{1.254999in}}%
\pgfpathlineto{\pgfqpoint{0.750003in}{1.254999in}}%
\pgfpathlineto{\pgfqpoint{0.750003in}{1.349373in}}%
\pgfpathlineto{\pgfqpoint{0.895317in}{1.349373in}}%
\pgfpathlineto{\pgfqpoint{0.895317in}{1.254999in}}%
\pgfpathmoveto{\pgfqpoint{0.750003in}{1.349373in}}%
\pgfpathlineto{\pgfqpoint{0.750003in}{1.349373in}}%
\pgfpathlineto{\pgfqpoint{0.750003in}{1.443752in}}%
\pgfpathlineto{\pgfqpoint{0.895317in}{1.443752in}}%
\pgfpathlineto{\pgfqpoint{0.895317in}{1.349373in}}%
\pgfpathmoveto{\pgfqpoint{0.750003in}{1.443752in}}%
\pgfpathlineto{\pgfqpoint{0.750003in}{1.443752in}}%
\pgfpathlineto{\pgfqpoint{0.750003in}{1.538128in}}%
\pgfpathlineto{\pgfqpoint{0.895317in}{1.538128in}}%
\pgfpathlineto{\pgfqpoint{0.895317in}{1.443752in}}%
\pgfpathmoveto{\pgfqpoint{0.750003in}{1.538128in}}%
\pgfpathlineto{\pgfqpoint{0.750003in}{1.538128in}}%
\pgfpathlineto{\pgfqpoint{0.750003in}{1.632499in}}%
\pgfpathlineto{\pgfqpoint{0.895317in}{1.632499in}}%
\pgfpathlineto{\pgfqpoint{0.895317in}{1.538128in}}%
\pgfpathmoveto{\pgfqpoint{0.750003in}{1.632499in}}%
\pgfpathlineto{\pgfqpoint{0.750003in}{1.632499in}}%
\pgfpathlineto{\pgfqpoint{0.750003in}{1.726877in}}%
\pgfpathlineto{\pgfqpoint{0.895317in}{1.726877in}}%
\pgfpathlineto{\pgfqpoint{0.895317in}{1.632499in}}%
\pgfpathmoveto{\pgfqpoint{0.750003in}{1.726877in}}%
\pgfpathlineto{\pgfqpoint{0.750003in}{1.726877in}}%
\pgfpathlineto{\pgfqpoint{0.750003in}{1.821249in}}%
\pgfpathlineto{\pgfqpoint{0.895317in}{1.821249in}}%
\pgfpathlineto{\pgfqpoint{0.895317in}{1.726877in}}%
\pgfpathmoveto{\pgfqpoint{0.750003in}{1.821249in}}%
\pgfpathlineto{\pgfqpoint{0.750003in}{1.821249in}}%
\pgfpathlineto{\pgfqpoint{0.750003in}{1.915628in}}%
\pgfpathlineto{\pgfqpoint{0.895317in}{1.915628in}}%
\pgfpathlineto{\pgfqpoint{0.895317in}{1.821249in}}%
\pgfpathmoveto{\pgfqpoint{0.750003in}{1.915628in}}%
\pgfpathlineto{\pgfqpoint{0.750003in}{1.915628in}}%
\pgfpathlineto{\pgfqpoint{0.750003in}{2.009998in}}%
\pgfpathlineto{\pgfqpoint{0.895317in}{2.009998in}}%
\pgfpathlineto{\pgfqpoint{0.895317in}{1.915628in}}%
\pgfpathmoveto{\pgfqpoint{0.750003in}{2.009998in}}%
\pgfpathlineto{\pgfqpoint{0.750003in}{2.009998in}}%
\pgfpathlineto{\pgfqpoint{0.750003in}{2.104374in}}%
\pgfpathlineto{\pgfqpoint{0.895317in}{2.104374in}}%
\pgfpathlineto{\pgfqpoint{0.895317in}{2.009998in}}%
\pgfpathmoveto{\pgfqpoint{0.750003in}{2.104374in}}%
\pgfpathlineto{\pgfqpoint{0.750003in}{2.104374in}}%
\pgfpathlineto{\pgfqpoint{0.750003in}{2.198749in}}%
\pgfpathlineto{\pgfqpoint{0.895317in}{2.198749in}}%
\pgfpathlineto{\pgfqpoint{0.895317in}{2.104374in}}%
\pgfpathmoveto{\pgfqpoint{0.750003in}{2.198749in}}%
\pgfpathlineto{\pgfqpoint{0.750003in}{2.198749in}}%
\pgfpathlineto{\pgfqpoint{0.750003in}{2.293123in}}%
\pgfpathlineto{\pgfqpoint{0.895317in}{2.293123in}}%
\pgfpathlineto{\pgfqpoint{0.895317in}{2.198749in}}%
\pgfpathmoveto{\pgfqpoint{0.750003in}{2.293123in}}%
\pgfpathlineto{\pgfqpoint{0.750003in}{2.293123in}}%
\pgfpathlineto{\pgfqpoint{0.750003in}{2.387498in}}%
\pgfpathlineto{\pgfqpoint{0.895317in}{2.387498in}}%
\pgfpathlineto{\pgfqpoint{0.895317in}{2.293123in}}%
\pgfpathmoveto{\pgfqpoint{0.750003in}{2.387498in}}%
\pgfpathlineto{\pgfqpoint{0.750003in}{2.387498in}}%
\pgfpathlineto{\pgfqpoint{0.750003in}{2.481875in}}%
\pgfpathlineto{\pgfqpoint{0.895317in}{2.481875in}}%
\pgfpathlineto{\pgfqpoint{0.895317in}{2.387498in}}%
\pgfpathmoveto{\pgfqpoint{0.750003in}{2.481875in}}%
\pgfpathlineto{\pgfqpoint{0.750003in}{2.481875in}}%
\pgfpathlineto{\pgfqpoint{0.750003in}{2.576249in}}%
\pgfpathlineto{\pgfqpoint{0.895317in}{2.576249in}}%
\pgfpathlineto{\pgfqpoint{0.895317in}{2.481875in}}%
\pgfpathmoveto{\pgfqpoint{0.750003in}{2.576249in}}%
\pgfpathlineto{\pgfqpoint{0.750003in}{2.576249in}}%
\pgfpathlineto{\pgfqpoint{0.750003in}{2.670624in}}%
\pgfpathlineto{\pgfqpoint{0.895317in}{2.670624in}}%
\pgfpathlineto{\pgfqpoint{0.895317in}{2.576249in}}%
\pgfpathmoveto{\pgfqpoint{0.750003in}{2.670624in}}%
\pgfpathlineto{\pgfqpoint{0.750003in}{2.670624in}}%
\pgfpathlineto{\pgfqpoint{0.750003in}{2.765003in}}%
\pgfpathlineto{\pgfqpoint{0.895317in}{2.765003in}}%
\pgfpathlineto{\pgfqpoint{0.895317in}{2.670624in}}%
\pgfpathmoveto{\pgfqpoint{0.750003in}{2.765003in}}%
\pgfpathlineto{\pgfqpoint{0.750003in}{2.765003in}}%
\pgfpathlineto{\pgfqpoint{0.750003in}{2.859372in}}%
\pgfpathlineto{\pgfqpoint{0.895317in}{2.859372in}}%
\pgfpathlineto{\pgfqpoint{0.895317in}{2.765003in}}%
\pgfpathmoveto{\pgfqpoint{0.750003in}{2.859372in}}%
\pgfpathlineto{\pgfqpoint{0.750003in}{2.859372in}}%
\pgfpathlineto{\pgfqpoint{0.750003in}{2.953749in}}%
\pgfpathlineto{\pgfqpoint{0.895317in}{2.953749in}}%
\pgfpathlineto{\pgfqpoint{0.895317in}{2.859372in}}%
\pgfpathmoveto{\pgfqpoint{0.750003in}{2.953749in}}%
\pgfpathlineto{\pgfqpoint{0.750003in}{2.953749in}}%
\pgfpathlineto{\pgfqpoint{0.750003in}{3.048124in}}%
\pgfpathlineto{\pgfqpoint{0.895317in}{3.048124in}}%
\pgfpathlineto{\pgfqpoint{0.895317in}{2.953749in}}%
\pgfpathmoveto{\pgfqpoint{0.750003in}{3.048124in}}%
\pgfpathlineto{\pgfqpoint{0.750003in}{3.048124in}}%
\pgfpathlineto{\pgfqpoint{0.750003in}{3.142498in}}%
\pgfpathlineto{\pgfqpoint{0.895317in}{3.142498in}}%
\pgfpathlineto{\pgfqpoint{0.895317in}{3.048124in}}%
\pgfpathmoveto{\pgfqpoint{0.750003in}{3.142498in}}%
\pgfpathlineto{\pgfqpoint{0.750003in}{3.142498in}}%
\pgfpathlineto{\pgfqpoint{0.750003in}{3.236876in}}%
\pgfpathlineto{\pgfqpoint{0.895317in}{3.236876in}}%
\pgfpathlineto{\pgfqpoint{0.895317in}{3.142498in}}%
\pgfpathmoveto{\pgfqpoint{0.750003in}{3.236876in}}%
\pgfpathlineto{\pgfqpoint{0.750003in}{3.236876in}}%
\pgfpathlineto{\pgfqpoint{0.750003in}{3.331248in}}%
\pgfpathlineto{\pgfqpoint{0.895317in}{3.331248in}}%
\pgfpathlineto{\pgfqpoint{0.895317in}{3.236876in}}%
\pgfpathmoveto{\pgfqpoint{0.750003in}{3.331248in}}%
\pgfpathlineto{\pgfqpoint{0.750003in}{3.331248in}}%
\pgfpathlineto{\pgfqpoint{0.750003in}{3.425623in}}%
\pgfpathlineto{\pgfqpoint{0.895317in}{3.425623in}}%
\pgfpathlineto{\pgfqpoint{0.895317in}{3.331248in}}%
\pgfpathmoveto{\pgfqpoint{0.750003in}{3.425623in}}%
\pgfpathlineto{\pgfqpoint{0.750003in}{3.425623in}}%
\pgfpathlineto{\pgfqpoint{0.750003in}{3.519998in}}%
\pgfpathlineto{\pgfqpoint{0.895317in}{3.519998in}}%
\pgfpathlineto{\pgfqpoint{0.895317in}{3.425623in}}%
\pgfpathmoveto{\pgfqpoint{0.895317in}{0.499998in}}%
\pgfpathlineto{\pgfqpoint{0.895317in}{0.499998in}}%
\pgfpathlineto{\pgfqpoint{0.895317in}{0.594373in}}%
\pgfpathlineto{\pgfqpoint{1.040622in}{0.594373in}}%
\pgfpathlineto{\pgfqpoint{1.040622in}{0.499998in}}%
\pgfpathmoveto{\pgfqpoint{0.895317in}{0.594373in}}%
\pgfpathlineto{\pgfqpoint{0.895317in}{0.594373in}}%
\pgfpathlineto{\pgfqpoint{0.895317in}{0.688753in}}%
\pgfpathlineto{\pgfqpoint{1.040622in}{0.688753in}}%
\pgfpathlineto{\pgfqpoint{1.040622in}{0.594373in}}%
\pgfpathmoveto{\pgfqpoint{0.895317in}{0.688753in}}%
\pgfpathlineto{\pgfqpoint{0.895317in}{0.688753in}}%
\pgfpathlineto{\pgfqpoint{0.895317in}{0.783125in}}%
\pgfpathlineto{\pgfqpoint{1.040622in}{0.783125in}}%
\pgfpathlineto{\pgfqpoint{1.040622in}{0.688753in}}%
\pgfpathmoveto{\pgfqpoint{0.895317in}{0.783125in}}%
\pgfpathlineto{\pgfqpoint{0.895317in}{0.783125in}}%
\pgfpathlineto{\pgfqpoint{0.895317in}{0.877501in}}%
\pgfpathlineto{\pgfqpoint{1.040622in}{0.877501in}}%
\pgfpathlineto{\pgfqpoint{1.040622in}{0.783125in}}%
\pgfpathmoveto{\pgfqpoint{0.895317in}{0.877501in}}%
\pgfpathlineto{\pgfqpoint{0.895317in}{0.877501in}}%
\pgfpathlineto{\pgfqpoint{0.895317in}{0.971874in}}%
\pgfpathlineto{\pgfqpoint{1.040622in}{0.971874in}}%
\pgfpathlineto{\pgfqpoint{1.040622in}{0.877501in}}%
\pgfpathmoveto{\pgfqpoint{0.895317in}{0.971874in}}%
\pgfpathlineto{\pgfqpoint{0.895317in}{0.971874in}}%
\pgfpathlineto{\pgfqpoint{0.895317in}{1.066247in}}%
\pgfpathlineto{\pgfqpoint{1.040622in}{1.066247in}}%
\pgfpathlineto{\pgfqpoint{1.040622in}{0.971874in}}%
\pgfpathmoveto{\pgfqpoint{0.895317in}{1.066247in}}%
\pgfpathlineto{\pgfqpoint{0.895317in}{1.066247in}}%
\pgfpathlineto{\pgfqpoint{0.895317in}{1.160624in}}%
\pgfpathlineto{\pgfqpoint{1.040622in}{1.160624in}}%
\pgfpathlineto{\pgfqpoint{1.040622in}{1.066247in}}%
\pgfpathmoveto{\pgfqpoint{0.895317in}{1.160624in}}%
\pgfpathlineto{\pgfqpoint{0.895317in}{1.160624in}}%
\pgfpathlineto{\pgfqpoint{0.895317in}{1.254999in}}%
\pgfpathlineto{\pgfqpoint{1.040622in}{1.254999in}}%
\pgfpathlineto{\pgfqpoint{1.040622in}{1.160624in}}%
\pgfpathmoveto{\pgfqpoint{0.895317in}{1.254999in}}%
\pgfpathlineto{\pgfqpoint{0.895317in}{1.254999in}}%
\pgfpathlineto{\pgfqpoint{0.895317in}{1.349373in}}%
\pgfpathlineto{\pgfqpoint{1.040622in}{1.349373in}}%
\pgfpathlineto{\pgfqpoint{1.040622in}{1.254999in}}%
\pgfpathmoveto{\pgfqpoint{0.895317in}{1.349373in}}%
\pgfpathlineto{\pgfqpoint{0.895317in}{1.349373in}}%
\pgfpathlineto{\pgfqpoint{0.895317in}{1.443752in}}%
\pgfpathlineto{\pgfqpoint{1.040622in}{1.443752in}}%
\pgfpathlineto{\pgfqpoint{1.040622in}{1.349373in}}%
\pgfpathmoveto{\pgfqpoint{0.895317in}{1.443752in}}%
\pgfpathlineto{\pgfqpoint{0.895317in}{1.443752in}}%
\pgfpathlineto{\pgfqpoint{0.895317in}{1.538128in}}%
\pgfpathlineto{\pgfqpoint{1.040622in}{1.538128in}}%
\pgfpathlineto{\pgfqpoint{1.040622in}{1.443752in}}%
\pgfpathmoveto{\pgfqpoint{0.895317in}{1.538128in}}%
\pgfpathlineto{\pgfqpoint{0.895317in}{1.538128in}}%
\pgfpathlineto{\pgfqpoint{0.895317in}{1.632499in}}%
\pgfpathlineto{\pgfqpoint{1.040622in}{1.632499in}}%
\pgfpathlineto{\pgfqpoint{1.040622in}{1.538128in}}%
\pgfpathmoveto{\pgfqpoint{0.895317in}{1.632499in}}%
\pgfpathlineto{\pgfqpoint{0.895317in}{1.632499in}}%
\pgfpathlineto{\pgfqpoint{0.895317in}{1.726877in}}%
\pgfpathlineto{\pgfqpoint{1.040622in}{1.726877in}}%
\pgfpathlineto{\pgfqpoint{1.040622in}{1.632499in}}%
\pgfpathmoveto{\pgfqpoint{0.895317in}{1.726877in}}%
\pgfpathlineto{\pgfqpoint{0.895317in}{1.726877in}}%
\pgfpathlineto{\pgfqpoint{0.895317in}{1.821249in}}%
\pgfpathlineto{\pgfqpoint{1.040622in}{1.821249in}}%
\pgfpathlineto{\pgfqpoint{1.040622in}{1.726877in}}%
\pgfpathmoveto{\pgfqpoint{0.895317in}{1.821249in}}%
\pgfpathlineto{\pgfqpoint{0.895317in}{1.821249in}}%
\pgfpathlineto{\pgfqpoint{0.895317in}{1.915628in}}%
\pgfpathlineto{\pgfqpoint{1.040622in}{1.915628in}}%
\pgfpathlineto{\pgfqpoint{1.040622in}{1.821249in}}%
\pgfpathmoveto{\pgfqpoint{0.895317in}{1.915628in}}%
\pgfpathlineto{\pgfqpoint{0.895317in}{1.915628in}}%
\pgfpathlineto{\pgfqpoint{0.895317in}{2.009998in}}%
\pgfpathlineto{\pgfqpoint{1.040622in}{2.009998in}}%
\pgfpathlineto{\pgfqpoint{1.040622in}{1.915628in}}%
\pgfpathmoveto{\pgfqpoint{0.895317in}{2.009998in}}%
\pgfpathlineto{\pgfqpoint{0.895317in}{2.009998in}}%
\pgfpathlineto{\pgfqpoint{0.895317in}{2.104374in}}%
\pgfpathlineto{\pgfqpoint{1.040622in}{2.104374in}}%
\pgfpathlineto{\pgfqpoint{1.040622in}{2.009998in}}%
\pgfpathmoveto{\pgfqpoint{0.895317in}{2.104374in}}%
\pgfpathlineto{\pgfqpoint{0.895317in}{2.104374in}}%
\pgfpathlineto{\pgfqpoint{0.895317in}{2.198749in}}%
\pgfpathlineto{\pgfqpoint{1.040622in}{2.198749in}}%
\pgfpathlineto{\pgfqpoint{1.040622in}{2.104374in}}%
\pgfpathmoveto{\pgfqpoint{0.895317in}{2.198749in}}%
\pgfpathlineto{\pgfqpoint{0.895317in}{2.198749in}}%
\pgfpathlineto{\pgfqpoint{0.895317in}{2.293123in}}%
\pgfpathlineto{\pgfqpoint{1.040622in}{2.293123in}}%
\pgfpathlineto{\pgfqpoint{1.040622in}{2.198749in}}%
\pgfpathmoveto{\pgfqpoint{0.895317in}{2.293123in}}%
\pgfpathlineto{\pgfqpoint{0.895317in}{2.293123in}}%
\pgfpathlineto{\pgfqpoint{0.895317in}{2.387498in}}%
\pgfpathlineto{\pgfqpoint{1.040622in}{2.387498in}}%
\pgfpathlineto{\pgfqpoint{1.040622in}{2.293123in}}%
\pgfpathmoveto{\pgfqpoint{0.895317in}{2.387498in}}%
\pgfpathlineto{\pgfqpoint{0.895317in}{2.387498in}}%
\pgfpathlineto{\pgfqpoint{0.895317in}{2.481875in}}%
\pgfpathlineto{\pgfqpoint{1.040622in}{2.481875in}}%
\pgfpathlineto{\pgfqpoint{1.040622in}{2.387498in}}%
\pgfpathmoveto{\pgfqpoint{0.895317in}{2.481875in}}%
\pgfpathlineto{\pgfqpoint{0.895317in}{2.481875in}}%
\pgfpathlineto{\pgfqpoint{0.895317in}{2.576249in}}%
\pgfpathlineto{\pgfqpoint{1.040622in}{2.576249in}}%
\pgfpathlineto{\pgfqpoint{1.040622in}{2.481875in}}%
\pgfpathmoveto{\pgfqpoint{0.895317in}{2.576249in}}%
\pgfpathlineto{\pgfqpoint{0.895317in}{2.576249in}}%
\pgfpathlineto{\pgfqpoint{0.895317in}{2.670624in}}%
\pgfpathlineto{\pgfqpoint{1.040622in}{2.670624in}}%
\pgfpathlineto{\pgfqpoint{1.040622in}{2.576249in}}%
\pgfpathmoveto{\pgfqpoint{0.895317in}{2.670624in}}%
\pgfpathlineto{\pgfqpoint{0.895317in}{2.670624in}}%
\pgfpathlineto{\pgfqpoint{0.895317in}{2.765003in}}%
\pgfpathlineto{\pgfqpoint{1.040622in}{2.765003in}}%
\pgfpathlineto{\pgfqpoint{1.040622in}{2.670624in}}%
\pgfpathmoveto{\pgfqpoint{0.895317in}{2.765003in}}%
\pgfpathlineto{\pgfqpoint{0.895317in}{2.765003in}}%
\pgfpathlineto{\pgfqpoint{0.895317in}{2.859372in}}%
\pgfpathlineto{\pgfqpoint{1.040622in}{2.859372in}}%
\pgfpathlineto{\pgfqpoint{1.040622in}{2.765003in}}%
\pgfpathmoveto{\pgfqpoint{0.895317in}{2.859372in}}%
\pgfpathlineto{\pgfqpoint{0.895317in}{2.859372in}}%
\pgfpathlineto{\pgfqpoint{0.895317in}{2.953749in}}%
\pgfpathlineto{\pgfqpoint{1.040622in}{2.953749in}}%
\pgfpathlineto{\pgfqpoint{1.040622in}{2.859372in}}%
\pgfpathmoveto{\pgfqpoint{0.895317in}{2.953749in}}%
\pgfpathlineto{\pgfqpoint{0.895317in}{2.953749in}}%
\pgfpathlineto{\pgfqpoint{0.895317in}{3.048124in}}%
\pgfpathlineto{\pgfqpoint{1.040622in}{3.048124in}}%
\pgfpathlineto{\pgfqpoint{1.040622in}{2.953749in}}%
\pgfpathmoveto{\pgfqpoint{0.895317in}{3.048124in}}%
\pgfpathlineto{\pgfqpoint{0.895317in}{3.048124in}}%
\pgfpathlineto{\pgfqpoint{0.895317in}{3.142498in}}%
\pgfpathlineto{\pgfqpoint{1.040622in}{3.142498in}}%
\pgfpathlineto{\pgfqpoint{1.040622in}{3.048124in}}%
\pgfpathmoveto{\pgfqpoint{0.895317in}{3.142498in}}%
\pgfpathlineto{\pgfqpoint{0.895317in}{3.142498in}}%
\pgfpathlineto{\pgfqpoint{0.895317in}{3.236876in}}%
\pgfpathlineto{\pgfqpoint{1.040622in}{3.236876in}}%
\pgfpathlineto{\pgfqpoint{1.040622in}{3.142498in}}%
\pgfpathmoveto{\pgfqpoint{0.895317in}{3.236876in}}%
\pgfpathlineto{\pgfqpoint{0.895317in}{3.236876in}}%
\pgfpathlineto{\pgfqpoint{0.895317in}{3.331248in}}%
\pgfpathlineto{\pgfqpoint{1.040622in}{3.331248in}}%
\pgfpathlineto{\pgfqpoint{1.040622in}{3.236876in}}%
\pgfpathmoveto{\pgfqpoint{0.895317in}{3.331248in}}%
\pgfpathlineto{\pgfqpoint{0.895317in}{3.331248in}}%
\pgfpathlineto{\pgfqpoint{0.895317in}{3.425623in}}%
\pgfpathlineto{\pgfqpoint{1.040622in}{3.425623in}}%
\pgfpathlineto{\pgfqpoint{1.040622in}{3.331248in}}%
\pgfpathmoveto{\pgfqpoint{0.895317in}{3.425623in}}%
\pgfpathlineto{\pgfqpoint{0.895317in}{3.425623in}}%
\pgfpathlineto{\pgfqpoint{0.895317in}{3.519998in}}%
\pgfpathlineto{\pgfqpoint{1.040622in}{3.519998in}}%
\pgfpathlineto{\pgfqpoint{1.040622in}{3.425623in}}%
\pgfpathmoveto{\pgfqpoint{1.040622in}{0.499998in}}%
\pgfpathlineto{\pgfqpoint{1.040622in}{0.499998in}}%
\pgfpathlineto{\pgfqpoint{1.040622in}{0.594373in}}%
\pgfpathlineto{\pgfqpoint{1.185941in}{0.594373in}}%
\pgfpathlineto{\pgfqpoint{1.185941in}{0.499998in}}%
\pgfpathmoveto{\pgfqpoint{1.040622in}{0.594373in}}%
\pgfpathlineto{\pgfqpoint{1.040622in}{0.594373in}}%
\pgfpathlineto{\pgfqpoint{1.040622in}{0.688753in}}%
\pgfpathlineto{\pgfqpoint{1.185941in}{0.688753in}}%
\pgfpathlineto{\pgfqpoint{1.185941in}{0.594373in}}%
\pgfpathmoveto{\pgfqpoint{1.040622in}{0.688753in}}%
\pgfpathlineto{\pgfqpoint{1.040622in}{0.688753in}}%
\pgfpathlineto{\pgfqpoint{1.040622in}{0.783125in}}%
\pgfpathlineto{\pgfqpoint{1.185941in}{0.783125in}}%
\pgfpathlineto{\pgfqpoint{1.185941in}{0.688753in}}%
\pgfpathmoveto{\pgfqpoint{1.040622in}{0.783125in}}%
\pgfpathlineto{\pgfqpoint{1.040622in}{0.783125in}}%
\pgfpathlineto{\pgfqpoint{1.040622in}{0.877501in}}%
\pgfpathlineto{\pgfqpoint{1.185941in}{0.877501in}}%
\pgfpathlineto{\pgfqpoint{1.185941in}{0.783125in}}%
\pgfpathmoveto{\pgfqpoint{1.040622in}{0.877501in}}%
\pgfpathlineto{\pgfqpoint{1.040622in}{0.877501in}}%
\pgfpathlineto{\pgfqpoint{1.040622in}{0.971874in}}%
\pgfpathlineto{\pgfqpoint{1.185941in}{0.971874in}}%
\pgfpathlineto{\pgfqpoint{1.185941in}{0.877501in}}%
\pgfpathmoveto{\pgfqpoint{1.040622in}{0.971874in}}%
\pgfpathlineto{\pgfqpoint{1.040622in}{0.971874in}}%
\pgfpathlineto{\pgfqpoint{1.040622in}{1.066247in}}%
\pgfpathlineto{\pgfqpoint{1.185941in}{1.066247in}}%
\pgfpathlineto{\pgfqpoint{1.185941in}{0.971874in}}%
\pgfpathmoveto{\pgfqpoint{1.040622in}{1.066247in}}%
\pgfpathlineto{\pgfqpoint{1.040622in}{1.066247in}}%
\pgfpathlineto{\pgfqpoint{1.040622in}{1.160624in}}%
\pgfpathlineto{\pgfqpoint{1.185941in}{1.160624in}}%
\pgfpathlineto{\pgfqpoint{1.185941in}{1.066247in}}%
\pgfpathmoveto{\pgfqpoint{1.040622in}{1.160624in}}%
\pgfpathlineto{\pgfqpoint{1.040622in}{1.160624in}}%
\pgfpathlineto{\pgfqpoint{1.040622in}{1.254999in}}%
\pgfpathlineto{\pgfqpoint{1.185941in}{1.254999in}}%
\pgfpathlineto{\pgfqpoint{1.185941in}{1.160624in}}%
\pgfpathmoveto{\pgfqpoint{1.040622in}{1.254999in}}%
\pgfpathlineto{\pgfqpoint{1.040622in}{1.254999in}}%
\pgfpathlineto{\pgfqpoint{1.040622in}{1.349373in}}%
\pgfpathlineto{\pgfqpoint{1.185941in}{1.349373in}}%
\pgfpathlineto{\pgfqpoint{1.185941in}{1.254999in}}%
\pgfpathmoveto{\pgfqpoint{1.040622in}{1.349373in}}%
\pgfpathlineto{\pgfqpoint{1.040622in}{1.349373in}}%
\pgfpathlineto{\pgfqpoint{1.040622in}{1.443752in}}%
\pgfpathlineto{\pgfqpoint{1.185941in}{1.443752in}}%
\pgfpathlineto{\pgfqpoint{1.185941in}{1.349373in}}%
\pgfpathmoveto{\pgfqpoint{1.040622in}{1.443752in}}%
\pgfpathlineto{\pgfqpoint{1.040622in}{1.443752in}}%
\pgfpathlineto{\pgfqpoint{1.040622in}{1.538128in}}%
\pgfpathlineto{\pgfqpoint{1.185941in}{1.538128in}}%
\pgfpathlineto{\pgfqpoint{1.185941in}{1.443752in}}%
\pgfpathmoveto{\pgfqpoint{1.040622in}{1.538128in}}%
\pgfpathlineto{\pgfqpoint{1.040622in}{1.538128in}}%
\pgfpathlineto{\pgfqpoint{1.040622in}{1.632499in}}%
\pgfpathlineto{\pgfqpoint{1.185941in}{1.632499in}}%
\pgfpathlineto{\pgfqpoint{1.185941in}{1.538128in}}%
\pgfpathmoveto{\pgfqpoint{1.040622in}{1.632499in}}%
\pgfpathlineto{\pgfqpoint{1.040622in}{1.632499in}}%
\pgfpathlineto{\pgfqpoint{1.040622in}{1.726877in}}%
\pgfpathlineto{\pgfqpoint{1.185941in}{1.726877in}}%
\pgfpathlineto{\pgfqpoint{1.185941in}{1.632499in}}%
\pgfpathmoveto{\pgfqpoint{1.040622in}{1.726877in}}%
\pgfpathlineto{\pgfqpoint{1.040622in}{1.726877in}}%
\pgfpathlineto{\pgfqpoint{1.040622in}{1.821249in}}%
\pgfpathlineto{\pgfqpoint{1.185941in}{1.821249in}}%
\pgfpathlineto{\pgfqpoint{1.185941in}{1.726877in}}%
\pgfpathmoveto{\pgfqpoint{1.040622in}{1.821249in}}%
\pgfpathlineto{\pgfqpoint{1.040622in}{1.821249in}}%
\pgfpathlineto{\pgfqpoint{1.040622in}{1.915628in}}%
\pgfpathlineto{\pgfqpoint{1.185941in}{1.915628in}}%
\pgfpathlineto{\pgfqpoint{1.185941in}{1.821249in}}%
\pgfpathmoveto{\pgfqpoint{1.040622in}{1.915628in}}%
\pgfpathlineto{\pgfqpoint{1.040622in}{1.915628in}}%
\pgfpathlineto{\pgfqpoint{1.040622in}{2.009998in}}%
\pgfpathlineto{\pgfqpoint{1.185941in}{2.009998in}}%
\pgfpathlineto{\pgfqpoint{1.185941in}{1.915628in}}%
\pgfpathmoveto{\pgfqpoint{1.040622in}{2.009998in}}%
\pgfpathlineto{\pgfqpoint{1.040622in}{2.009998in}}%
\pgfpathlineto{\pgfqpoint{1.040622in}{2.104374in}}%
\pgfpathlineto{\pgfqpoint{1.185941in}{2.104374in}}%
\pgfpathlineto{\pgfqpoint{1.185941in}{2.009998in}}%
\pgfpathmoveto{\pgfqpoint{1.040622in}{2.104374in}}%
\pgfpathlineto{\pgfqpoint{1.040622in}{2.104374in}}%
\pgfpathlineto{\pgfqpoint{1.040622in}{2.198749in}}%
\pgfpathlineto{\pgfqpoint{1.185941in}{2.198749in}}%
\pgfpathlineto{\pgfqpoint{1.185941in}{2.104374in}}%
\pgfpathmoveto{\pgfqpoint{1.040622in}{2.198749in}}%
\pgfpathlineto{\pgfqpoint{1.040622in}{2.198749in}}%
\pgfpathlineto{\pgfqpoint{1.040622in}{2.293123in}}%
\pgfpathlineto{\pgfqpoint{1.185941in}{2.293123in}}%
\pgfpathlineto{\pgfqpoint{1.185941in}{2.198749in}}%
\pgfpathmoveto{\pgfqpoint{1.040622in}{2.293123in}}%
\pgfpathlineto{\pgfqpoint{1.040622in}{2.293123in}}%
\pgfpathlineto{\pgfqpoint{1.040622in}{2.387498in}}%
\pgfpathlineto{\pgfqpoint{1.185941in}{2.387498in}}%
\pgfpathlineto{\pgfqpoint{1.185941in}{2.293123in}}%
\pgfpathmoveto{\pgfqpoint{1.040622in}{2.387498in}}%
\pgfpathlineto{\pgfqpoint{1.040622in}{2.387498in}}%
\pgfpathlineto{\pgfqpoint{1.040622in}{2.481875in}}%
\pgfpathlineto{\pgfqpoint{1.185941in}{2.481875in}}%
\pgfpathlineto{\pgfqpoint{1.185941in}{2.387498in}}%
\pgfpathmoveto{\pgfqpoint{1.040622in}{2.481875in}}%
\pgfpathlineto{\pgfqpoint{1.040622in}{2.481875in}}%
\pgfpathlineto{\pgfqpoint{1.040622in}{2.576249in}}%
\pgfpathlineto{\pgfqpoint{1.185941in}{2.576249in}}%
\pgfpathlineto{\pgfqpoint{1.185941in}{2.481875in}}%
\pgfpathmoveto{\pgfqpoint{1.040622in}{2.576249in}}%
\pgfpathlineto{\pgfqpoint{1.040622in}{2.576249in}}%
\pgfpathlineto{\pgfqpoint{1.040622in}{2.670624in}}%
\pgfpathlineto{\pgfqpoint{1.185941in}{2.670624in}}%
\pgfpathlineto{\pgfqpoint{1.185941in}{2.576249in}}%
\pgfpathmoveto{\pgfqpoint{1.040622in}{2.670624in}}%
\pgfpathlineto{\pgfqpoint{1.040622in}{2.670624in}}%
\pgfpathlineto{\pgfqpoint{1.040622in}{2.765003in}}%
\pgfpathlineto{\pgfqpoint{1.185941in}{2.765003in}}%
\pgfpathlineto{\pgfqpoint{1.185941in}{2.670624in}}%
\pgfpathmoveto{\pgfqpoint{1.040622in}{2.765003in}}%
\pgfpathlineto{\pgfqpoint{1.040622in}{2.765003in}}%
\pgfpathlineto{\pgfqpoint{1.040622in}{2.859372in}}%
\pgfpathlineto{\pgfqpoint{1.185941in}{2.859372in}}%
\pgfpathlineto{\pgfqpoint{1.185941in}{2.765003in}}%
\pgfpathmoveto{\pgfqpoint{1.040622in}{2.859372in}}%
\pgfpathlineto{\pgfqpoint{1.040622in}{2.859372in}}%
\pgfpathlineto{\pgfqpoint{1.040622in}{2.953749in}}%
\pgfpathlineto{\pgfqpoint{1.185941in}{2.953749in}}%
\pgfpathlineto{\pgfqpoint{1.185941in}{2.859372in}}%
\pgfpathmoveto{\pgfqpoint{1.040622in}{2.953749in}}%
\pgfpathlineto{\pgfqpoint{1.040622in}{2.953749in}}%
\pgfpathlineto{\pgfqpoint{1.040622in}{3.048124in}}%
\pgfpathlineto{\pgfqpoint{1.185941in}{3.048124in}}%
\pgfpathlineto{\pgfqpoint{1.185941in}{2.953749in}}%
\pgfpathmoveto{\pgfqpoint{1.040622in}{3.048124in}}%
\pgfpathlineto{\pgfqpoint{1.040622in}{3.048124in}}%
\pgfpathlineto{\pgfqpoint{1.040622in}{3.142498in}}%
\pgfpathlineto{\pgfqpoint{1.185941in}{3.142498in}}%
\pgfpathlineto{\pgfqpoint{1.185941in}{3.048124in}}%
\pgfpathmoveto{\pgfqpoint{1.040622in}{3.142498in}}%
\pgfpathlineto{\pgfqpoint{1.040622in}{3.142498in}}%
\pgfpathlineto{\pgfqpoint{1.040622in}{3.236876in}}%
\pgfpathlineto{\pgfqpoint{1.185941in}{3.236876in}}%
\pgfpathlineto{\pgfqpoint{1.185941in}{3.142498in}}%
\pgfpathmoveto{\pgfqpoint{1.040622in}{3.236876in}}%
\pgfpathlineto{\pgfqpoint{1.040622in}{3.236876in}}%
\pgfpathlineto{\pgfqpoint{1.040622in}{3.331248in}}%
\pgfpathlineto{\pgfqpoint{1.185941in}{3.331248in}}%
\pgfpathlineto{\pgfqpoint{1.185941in}{3.236876in}}%
\pgfpathmoveto{\pgfqpoint{1.040622in}{3.331248in}}%
\pgfpathlineto{\pgfqpoint{1.040622in}{3.331248in}}%
\pgfpathlineto{\pgfqpoint{1.040622in}{3.425623in}}%
\pgfpathlineto{\pgfqpoint{1.185941in}{3.425623in}}%
\pgfpathlineto{\pgfqpoint{1.185941in}{3.331248in}}%
\pgfpathmoveto{\pgfqpoint{1.040622in}{3.425623in}}%
\pgfpathlineto{\pgfqpoint{1.040622in}{3.425623in}}%
\pgfpathlineto{\pgfqpoint{1.040622in}{3.519998in}}%
\pgfpathlineto{\pgfqpoint{1.185941in}{3.519998in}}%
\pgfpathlineto{\pgfqpoint{1.185941in}{3.425623in}}%
\pgfpathmoveto{\pgfqpoint{1.185941in}{0.499998in}}%
\pgfpathlineto{\pgfqpoint{1.185941in}{0.499998in}}%
\pgfpathlineto{\pgfqpoint{1.185941in}{0.594373in}}%
\pgfpathlineto{\pgfqpoint{1.331252in}{0.594373in}}%
\pgfpathlineto{\pgfqpoint{1.331252in}{0.499998in}}%
\pgfpathmoveto{\pgfqpoint{1.185941in}{0.594373in}}%
\pgfpathlineto{\pgfqpoint{1.185941in}{0.594373in}}%
\pgfpathlineto{\pgfqpoint{1.185941in}{0.688753in}}%
\pgfpathlineto{\pgfqpoint{1.331252in}{0.688753in}}%
\pgfpathlineto{\pgfqpoint{1.331252in}{0.594373in}}%
\pgfpathmoveto{\pgfqpoint{1.185941in}{0.688753in}}%
\pgfpathlineto{\pgfqpoint{1.185941in}{0.688753in}}%
\pgfpathlineto{\pgfqpoint{1.185941in}{0.783125in}}%
\pgfpathlineto{\pgfqpoint{1.331252in}{0.783125in}}%
\pgfpathlineto{\pgfqpoint{1.331252in}{0.688753in}}%
\pgfpathmoveto{\pgfqpoint{1.185941in}{0.783125in}}%
\pgfpathlineto{\pgfqpoint{1.185941in}{0.783125in}}%
\pgfpathlineto{\pgfqpoint{1.185941in}{0.877501in}}%
\pgfpathlineto{\pgfqpoint{1.331252in}{0.877501in}}%
\pgfpathlineto{\pgfqpoint{1.331252in}{0.783125in}}%
\pgfpathmoveto{\pgfqpoint{1.185941in}{0.877501in}}%
\pgfpathlineto{\pgfqpoint{1.185941in}{0.877501in}}%
\pgfpathlineto{\pgfqpoint{1.185941in}{0.971874in}}%
\pgfpathlineto{\pgfqpoint{1.331252in}{0.971874in}}%
\pgfpathlineto{\pgfqpoint{1.331252in}{0.877501in}}%
\pgfpathmoveto{\pgfqpoint{1.185941in}{0.971874in}}%
\pgfpathlineto{\pgfqpoint{1.185941in}{0.971874in}}%
\pgfpathlineto{\pgfqpoint{1.185941in}{1.066247in}}%
\pgfpathlineto{\pgfqpoint{1.331252in}{1.066247in}}%
\pgfpathlineto{\pgfqpoint{1.331252in}{0.971874in}}%
\pgfpathmoveto{\pgfqpoint{1.185941in}{1.066247in}}%
\pgfpathlineto{\pgfqpoint{1.185941in}{1.066247in}}%
\pgfpathlineto{\pgfqpoint{1.185941in}{1.160624in}}%
\pgfpathlineto{\pgfqpoint{1.331252in}{1.160624in}}%
\pgfpathlineto{\pgfqpoint{1.331252in}{1.066247in}}%
\pgfpathmoveto{\pgfqpoint{1.185941in}{1.160624in}}%
\pgfpathlineto{\pgfqpoint{1.185941in}{1.160624in}}%
\pgfpathlineto{\pgfqpoint{1.185941in}{1.254999in}}%
\pgfpathlineto{\pgfqpoint{1.331252in}{1.254999in}}%
\pgfpathlineto{\pgfqpoint{1.331252in}{1.160624in}}%
\pgfpathmoveto{\pgfqpoint{1.185941in}{1.254999in}}%
\pgfpathlineto{\pgfqpoint{1.185941in}{1.254999in}}%
\pgfpathlineto{\pgfqpoint{1.185941in}{1.349373in}}%
\pgfpathlineto{\pgfqpoint{1.331252in}{1.349373in}}%
\pgfpathlineto{\pgfqpoint{1.331252in}{1.254999in}}%
\pgfpathmoveto{\pgfqpoint{1.185941in}{1.349373in}}%
\pgfpathlineto{\pgfqpoint{1.185941in}{1.349373in}}%
\pgfpathlineto{\pgfqpoint{1.185941in}{1.443752in}}%
\pgfpathlineto{\pgfqpoint{1.331252in}{1.443752in}}%
\pgfpathlineto{\pgfqpoint{1.331252in}{1.349373in}}%
\pgfpathmoveto{\pgfqpoint{1.185941in}{1.443752in}}%
\pgfpathlineto{\pgfqpoint{1.185941in}{1.443752in}}%
\pgfpathlineto{\pgfqpoint{1.185941in}{1.538128in}}%
\pgfpathlineto{\pgfqpoint{1.331252in}{1.538128in}}%
\pgfpathlineto{\pgfqpoint{1.331252in}{1.443752in}}%
\pgfpathmoveto{\pgfqpoint{1.185941in}{1.538128in}}%
\pgfpathlineto{\pgfqpoint{1.185941in}{1.538128in}}%
\pgfpathlineto{\pgfqpoint{1.185941in}{1.632499in}}%
\pgfpathlineto{\pgfqpoint{1.331252in}{1.632499in}}%
\pgfpathlineto{\pgfqpoint{1.331252in}{1.538128in}}%
\pgfpathmoveto{\pgfqpoint{1.185941in}{1.632499in}}%
\pgfpathlineto{\pgfqpoint{1.185941in}{1.632499in}}%
\pgfpathlineto{\pgfqpoint{1.185941in}{1.726877in}}%
\pgfpathlineto{\pgfqpoint{1.331252in}{1.726877in}}%
\pgfpathlineto{\pgfqpoint{1.331252in}{1.632499in}}%
\pgfpathmoveto{\pgfqpoint{1.185941in}{1.726877in}}%
\pgfpathlineto{\pgfqpoint{1.185941in}{1.726877in}}%
\pgfpathlineto{\pgfqpoint{1.185941in}{1.821249in}}%
\pgfpathlineto{\pgfqpoint{1.331252in}{1.821249in}}%
\pgfpathlineto{\pgfqpoint{1.331252in}{1.726877in}}%
\pgfpathmoveto{\pgfqpoint{1.185941in}{1.821249in}}%
\pgfpathlineto{\pgfqpoint{1.185941in}{1.821249in}}%
\pgfpathlineto{\pgfqpoint{1.185941in}{1.915628in}}%
\pgfpathlineto{\pgfqpoint{1.331252in}{1.915628in}}%
\pgfpathlineto{\pgfqpoint{1.331252in}{1.821249in}}%
\pgfpathmoveto{\pgfqpoint{1.185941in}{1.915628in}}%
\pgfpathlineto{\pgfqpoint{1.185941in}{1.915628in}}%
\pgfpathlineto{\pgfqpoint{1.185941in}{2.009998in}}%
\pgfpathlineto{\pgfqpoint{1.331252in}{2.009998in}}%
\pgfpathlineto{\pgfqpoint{1.331252in}{1.915628in}}%
\pgfpathmoveto{\pgfqpoint{1.185941in}{2.009998in}}%
\pgfpathlineto{\pgfqpoint{1.185941in}{2.009998in}}%
\pgfpathlineto{\pgfqpoint{1.185941in}{2.104374in}}%
\pgfpathlineto{\pgfqpoint{1.331252in}{2.104374in}}%
\pgfpathlineto{\pgfqpoint{1.331252in}{2.009998in}}%
\pgfpathmoveto{\pgfqpoint{1.185941in}{2.104374in}}%
\pgfpathlineto{\pgfqpoint{1.185941in}{2.104374in}}%
\pgfpathlineto{\pgfqpoint{1.185941in}{2.198749in}}%
\pgfpathlineto{\pgfqpoint{1.331252in}{2.198749in}}%
\pgfpathlineto{\pgfqpoint{1.331252in}{2.104374in}}%
\pgfpathmoveto{\pgfqpoint{1.185941in}{2.198749in}}%
\pgfpathlineto{\pgfqpoint{1.185941in}{2.198749in}}%
\pgfpathlineto{\pgfqpoint{1.185941in}{2.293123in}}%
\pgfpathlineto{\pgfqpoint{1.331252in}{2.293123in}}%
\pgfpathlineto{\pgfqpoint{1.331252in}{2.198749in}}%
\pgfpathmoveto{\pgfqpoint{1.185941in}{2.293123in}}%
\pgfpathlineto{\pgfqpoint{1.185941in}{2.293123in}}%
\pgfpathlineto{\pgfqpoint{1.185941in}{2.387498in}}%
\pgfpathlineto{\pgfqpoint{1.331252in}{2.387498in}}%
\pgfpathlineto{\pgfqpoint{1.331252in}{2.293123in}}%
\pgfpathmoveto{\pgfqpoint{1.185941in}{2.387498in}}%
\pgfpathlineto{\pgfqpoint{1.185941in}{2.387498in}}%
\pgfpathlineto{\pgfqpoint{1.185941in}{2.481875in}}%
\pgfpathlineto{\pgfqpoint{1.331252in}{2.481875in}}%
\pgfpathlineto{\pgfqpoint{1.331252in}{2.387498in}}%
\pgfpathmoveto{\pgfqpoint{1.185941in}{2.481875in}}%
\pgfpathlineto{\pgfqpoint{1.185941in}{2.481875in}}%
\pgfpathlineto{\pgfqpoint{1.185941in}{2.576249in}}%
\pgfpathlineto{\pgfqpoint{1.331252in}{2.576249in}}%
\pgfpathlineto{\pgfqpoint{1.331252in}{2.481875in}}%
\pgfpathmoveto{\pgfqpoint{1.185941in}{2.576249in}}%
\pgfpathlineto{\pgfqpoint{1.185941in}{2.576249in}}%
\pgfpathlineto{\pgfqpoint{1.185941in}{2.670624in}}%
\pgfpathlineto{\pgfqpoint{1.331252in}{2.670624in}}%
\pgfpathlineto{\pgfqpoint{1.331252in}{2.576249in}}%
\pgfpathmoveto{\pgfqpoint{1.185941in}{2.670624in}}%
\pgfpathlineto{\pgfqpoint{1.185941in}{2.670624in}}%
\pgfpathlineto{\pgfqpoint{1.185941in}{2.765003in}}%
\pgfpathlineto{\pgfqpoint{1.331252in}{2.765003in}}%
\pgfpathlineto{\pgfqpoint{1.331252in}{2.670624in}}%
\pgfpathmoveto{\pgfqpoint{1.185941in}{2.765003in}}%
\pgfpathlineto{\pgfqpoint{1.185941in}{2.765003in}}%
\pgfpathlineto{\pgfqpoint{1.185941in}{2.859372in}}%
\pgfpathlineto{\pgfqpoint{1.331252in}{2.859372in}}%
\pgfpathlineto{\pgfqpoint{1.331252in}{2.765003in}}%
\pgfpathmoveto{\pgfqpoint{1.185941in}{2.859372in}}%
\pgfpathlineto{\pgfqpoint{1.185941in}{2.859372in}}%
\pgfpathlineto{\pgfqpoint{1.185941in}{2.953749in}}%
\pgfpathlineto{\pgfqpoint{1.331252in}{2.953749in}}%
\pgfpathlineto{\pgfqpoint{1.331252in}{2.859372in}}%
\pgfpathmoveto{\pgfqpoint{1.185941in}{2.953749in}}%
\pgfpathlineto{\pgfqpoint{1.185941in}{2.953749in}}%
\pgfpathlineto{\pgfqpoint{1.185941in}{3.048124in}}%
\pgfpathlineto{\pgfqpoint{1.331252in}{3.048124in}}%
\pgfpathlineto{\pgfqpoint{1.331252in}{2.953749in}}%
\pgfpathmoveto{\pgfqpoint{1.185941in}{3.048124in}}%
\pgfpathlineto{\pgfqpoint{1.185941in}{3.048124in}}%
\pgfpathlineto{\pgfqpoint{1.185941in}{3.142498in}}%
\pgfpathlineto{\pgfqpoint{1.331252in}{3.142498in}}%
\pgfpathlineto{\pgfqpoint{1.331252in}{3.048124in}}%
\pgfpathmoveto{\pgfqpoint{1.185941in}{3.142498in}}%
\pgfpathlineto{\pgfqpoint{1.185941in}{3.142498in}}%
\pgfpathlineto{\pgfqpoint{1.185941in}{3.236876in}}%
\pgfpathlineto{\pgfqpoint{1.331252in}{3.236876in}}%
\pgfpathlineto{\pgfqpoint{1.331252in}{3.142498in}}%
\pgfpathmoveto{\pgfqpoint{1.185941in}{3.236876in}}%
\pgfpathlineto{\pgfqpoint{1.185941in}{3.236876in}}%
\pgfpathlineto{\pgfqpoint{1.185941in}{3.331248in}}%
\pgfpathlineto{\pgfqpoint{1.331252in}{3.331248in}}%
\pgfpathlineto{\pgfqpoint{1.331252in}{3.236876in}}%
\pgfpathmoveto{\pgfqpoint{1.185941in}{3.331248in}}%
\pgfpathlineto{\pgfqpoint{1.185941in}{3.331248in}}%
\pgfpathlineto{\pgfqpoint{1.185941in}{3.425623in}}%
\pgfpathlineto{\pgfqpoint{1.331252in}{3.425623in}}%
\pgfpathlineto{\pgfqpoint{1.331252in}{3.331248in}}%
\pgfpathmoveto{\pgfqpoint{1.185941in}{3.425623in}}%
\pgfpathlineto{\pgfqpoint{1.185941in}{3.425623in}}%
\pgfpathlineto{\pgfqpoint{1.185941in}{3.519998in}}%
\pgfpathlineto{\pgfqpoint{1.331252in}{3.519998in}}%
\pgfpathlineto{\pgfqpoint{1.331252in}{3.425623in}}%
\pgfpathmoveto{\pgfqpoint{1.331252in}{0.499998in}}%
\pgfpathlineto{\pgfqpoint{1.331252in}{0.499998in}}%
\pgfpathlineto{\pgfqpoint{1.331252in}{0.594373in}}%
\pgfpathlineto{\pgfqpoint{1.476560in}{0.594373in}}%
\pgfpathlineto{\pgfqpoint{1.476560in}{0.499998in}}%
\pgfpathmoveto{\pgfqpoint{1.331252in}{0.594373in}}%
\pgfpathlineto{\pgfqpoint{1.331252in}{0.594373in}}%
\pgfpathlineto{\pgfqpoint{1.331252in}{0.688753in}}%
\pgfpathlineto{\pgfqpoint{1.476560in}{0.688753in}}%
\pgfpathlineto{\pgfqpoint{1.476560in}{0.594373in}}%
\pgfpathmoveto{\pgfqpoint{1.331252in}{0.688753in}}%
\pgfpathlineto{\pgfqpoint{1.331252in}{0.688753in}}%
\pgfpathlineto{\pgfqpoint{1.331252in}{0.783125in}}%
\pgfpathlineto{\pgfqpoint{1.476560in}{0.783125in}}%
\pgfpathlineto{\pgfqpoint{1.476560in}{0.688753in}}%
\pgfpathmoveto{\pgfqpoint{1.331252in}{0.783125in}}%
\pgfpathlineto{\pgfqpoint{1.331252in}{0.783125in}}%
\pgfpathlineto{\pgfqpoint{1.331252in}{0.877501in}}%
\pgfpathlineto{\pgfqpoint{1.476560in}{0.877501in}}%
\pgfpathlineto{\pgfqpoint{1.476560in}{0.783125in}}%
\pgfpathmoveto{\pgfqpoint{1.331252in}{0.877501in}}%
\pgfpathlineto{\pgfqpoint{1.331252in}{0.877501in}}%
\pgfpathlineto{\pgfqpoint{1.331252in}{0.971874in}}%
\pgfpathlineto{\pgfqpoint{1.476560in}{0.971874in}}%
\pgfpathlineto{\pgfqpoint{1.476560in}{0.877501in}}%
\pgfpathmoveto{\pgfqpoint{1.331252in}{0.971874in}}%
\pgfpathlineto{\pgfqpoint{1.331252in}{0.971874in}}%
\pgfpathlineto{\pgfqpoint{1.331252in}{1.066247in}}%
\pgfpathlineto{\pgfqpoint{1.476560in}{1.066247in}}%
\pgfpathlineto{\pgfqpoint{1.476560in}{0.971874in}}%
\pgfpathmoveto{\pgfqpoint{1.331252in}{1.066247in}}%
\pgfpathlineto{\pgfqpoint{1.331252in}{1.066247in}}%
\pgfpathlineto{\pgfqpoint{1.331252in}{1.160624in}}%
\pgfpathlineto{\pgfqpoint{1.476560in}{1.160624in}}%
\pgfpathlineto{\pgfqpoint{1.476560in}{1.066247in}}%
\pgfpathmoveto{\pgfqpoint{1.331252in}{1.160624in}}%
\pgfpathlineto{\pgfqpoint{1.331252in}{1.160624in}}%
\pgfpathlineto{\pgfqpoint{1.331252in}{1.254999in}}%
\pgfpathlineto{\pgfqpoint{1.476560in}{1.254999in}}%
\pgfpathlineto{\pgfqpoint{1.476560in}{1.160624in}}%
\pgfpathmoveto{\pgfqpoint{1.331252in}{1.254999in}}%
\pgfpathlineto{\pgfqpoint{1.331252in}{1.254999in}}%
\pgfpathlineto{\pgfqpoint{1.331252in}{1.349373in}}%
\pgfpathlineto{\pgfqpoint{1.476560in}{1.349373in}}%
\pgfpathlineto{\pgfqpoint{1.476560in}{1.254999in}}%
\pgfpathmoveto{\pgfqpoint{1.331252in}{1.349373in}}%
\pgfpathlineto{\pgfqpoint{1.331252in}{1.349373in}}%
\pgfpathlineto{\pgfqpoint{1.331252in}{1.443752in}}%
\pgfpathlineto{\pgfqpoint{1.476560in}{1.443752in}}%
\pgfpathlineto{\pgfqpoint{1.476560in}{1.349373in}}%
\pgfpathmoveto{\pgfqpoint{1.331252in}{1.443752in}}%
\pgfpathlineto{\pgfqpoint{1.331252in}{1.443752in}}%
\pgfpathlineto{\pgfqpoint{1.331252in}{1.538128in}}%
\pgfpathlineto{\pgfqpoint{1.476560in}{1.538128in}}%
\pgfpathlineto{\pgfqpoint{1.476560in}{1.443752in}}%
\pgfpathmoveto{\pgfqpoint{1.331252in}{1.538128in}}%
\pgfpathlineto{\pgfqpoint{1.331252in}{1.538128in}}%
\pgfpathlineto{\pgfqpoint{1.331252in}{1.632499in}}%
\pgfpathlineto{\pgfqpoint{1.476560in}{1.632499in}}%
\pgfpathlineto{\pgfqpoint{1.476560in}{1.538128in}}%
\pgfpathmoveto{\pgfqpoint{1.331252in}{1.632499in}}%
\pgfpathlineto{\pgfqpoint{1.331252in}{1.632499in}}%
\pgfpathlineto{\pgfqpoint{1.331252in}{1.726877in}}%
\pgfpathlineto{\pgfqpoint{1.476560in}{1.726877in}}%
\pgfpathlineto{\pgfqpoint{1.476560in}{1.632499in}}%
\pgfpathmoveto{\pgfqpoint{1.331252in}{1.726877in}}%
\pgfpathlineto{\pgfqpoint{1.331252in}{1.726877in}}%
\pgfpathlineto{\pgfqpoint{1.331252in}{1.821249in}}%
\pgfpathlineto{\pgfqpoint{1.476560in}{1.821249in}}%
\pgfpathlineto{\pgfqpoint{1.476560in}{1.726877in}}%
\pgfpathmoveto{\pgfqpoint{1.331252in}{1.821249in}}%
\pgfpathlineto{\pgfqpoint{1.331252in}{1.821249in}}%
\pgfpathlineto{\pgfqpoint{1.331252in}{1.915628in}}%
\pgfpathlineto{\pgfqpoint{1.476560in}{1.915628in}}%
\pgfpathlineto{\pgfqpoint{1.476560in}{1.821249in}}%
\pgfpathmoveto{\pgfqpoint{1.331252in}{1.915628in}}%
\pgfpathlineto{\pgfqpoint{1.331252in}{1.915628in}}%
\pgfpathlineto{\pgfqpoint{1.331252in}{2.009998in}}%
\pgfpathlineto{\pgfqpoint{1.476560in}{2.009998in}}%
\pgfpathlineto{\pgfqpoint{1.476560in}{1.915628in}}%
\pgfpathmoveto{\pgfqpoint{1.331252in}{2.009998in}}%
\pgfpathlineto{\pgfqpoint{1.331252in}{2.009998in}}%
\pgfpathlineto{\pgfqpoint{1.331252in}{2.104374in}}%
\pgfpathlineto{\pgfqpoint{1.476560in}{2.104374in}}%
\pgfpathlineto{\pgfqpoint{1.476560in}{2.009998in}}%
\pgfpathmoveto{\pgfqpoint{1.331252in}{2.104374in}}%
\pgfpathlineto{\pgfqpoint{1.331252in}{2.104374in}}%
\pgfpathlineto{\pgfqpoint{1.331252in}{2.198749in}}%
\pgfpathlineto{\pgfqpoint{1.476560in}{2.198749in}}%
\pgfpathlineto{\pgfqpoint{1.476560in}{2.104374in}}%
\pgfpathmoveto{\pgfqpoint{1.331252in}{2.198749in}}%
\pgfpathlineto{\pgfqpoint{1.331252in}{2.198749in}}%
\pgfpathlineto{\pgfqpoint{1.331252in}{2.293123in}}%
\pgfpathlineto{\pgfqpoint{1.476560in}{2.293123in}}%
\pgfpathlineto{\pgfqpoint{1.476560in}{2.198749in}}%
\pgfpathmoveto{\pgfqpoint{1.331252in}{2.293123in}}%
\pgfpathlineto{\pgfqpoint{1.331252in}{2.293123in}}%
\pgfpathlineto{\pgfqpoint{1.331252in}{2.387498in}}%
\pgfpathlineto{\pgfqpoint{1.476560in}{2.387498in}}%
\pgfpathlineto{\pgfqpoint{1.476560in}{2.293123in}}%
\pgfpathmoveto{\pgfqpoint{1.331252in}{2.387498in}}%
\pgfpathlineto{\pgfqpoint{1.331252in}{2.387498in}}%
\pgfpathlineto{\pgfqpoint{1.331252in}{2.481875in}}%
\pgfpathlineto{\pgfqpoint{1.476560in}{2.481875in}}%
\pgfpathlineto{\pgfqpoint{1.476560in}{2.387498in}}%
\pgfpathmoveto{\pgfqpoint{1.331252in}{2.481875in}}%
\pgfpathlineto{\pgfqpoint{1.331252in}{2.481875in}}%
\pgfpathlineto{\pgfqpoint{1.331252in}{2.576249in}}%
\pgfpathlineto{\pgfqpoint{1.476560in}{2.576249in}}%
\pgfpathlineto{\pgfqpoint{1.476560in}{2.481875in}}%
\pgfpathmoveto{\pgfqpoint{1.331252in}{2.576249in}}%
\pgfpathlineto{\pgfqpoint{1.331252in}{2.576249in}}%
\pgfpathlineto{\pgfqpoint{1.331252in}{2.670624in}}%
\pgfpathlineto{\pgfqpoint{1.476560in}{2.670624in}}%
\pgfpathlineto{\pgfqpoint{1.476560in}{2.576249in}}%
\pgfpathmoveto{\pgfqpoint{1.331252in}{2.670624in}}%
\pgfpathlineto{\pgfqpoint{1.331252in}{2.670624in}}%
\pgfpathlineto{\pgfqpoint{1.331252in}{2.765003in}}%
\pgfpathlineto{\pgfqpoint{1.476560in}{2.765003in}}%
\pgfpathlineto{\pgfqpoint{1.476560in}{2.670624in}}%
\pgfpathmoveto{\pgfqpoint{1.331252in}{2.765003in}}%
\pgfpathlineto{\pgfqpoint{1.331252in}{2.765003in}}%
\pgfpathlineto{\pgfqpoint{1.331252in}{2.859372in}}%
\pgfpathlineto{\pgfqpoint{1.476560in}{2.859372in}}%
\pgfpathlineto{\pgfqpoint{1.476560in}{2.765003in}}%
\pgfpathmoveto{\pgfqpoint{1.331252in}{2.859372in}}%
\pgfpathlineto{\pgfqpoint{1.331252in}{2.859372in}}%
\pgfpathlineto{\pgfqpoint{1.331252in}{2.953749in}}%
\pgfpathlineto{\pgfqpoint{1.476560in}{2.953749in}}%
\pgfpathlineto{\pgfqpoint{1.476560in}{2.859372in}}%
\pgfpathmoveto{\pgfqpoint{1.331252in}{2.953749in}}%
\pgfpathlineto{\pgfqpoint{1.331252in}{2.953749in}}%
\pgfpathlineto{\pgfqpoint{1.331252in}{3.048124in}}%
\pgfpathlineto{\pgfqpoint{1.476560in}{3.048124in}}%
\pgfpathlineto{\pgfqpoint{1.476560in}{2.953749in}}%
\pgfpathmoveto{\pgfqpoint{1.331252in}{3.048124in}}%
\pgfpathlineto{\pgfqpoint{1.331252in}{3.048124in}}%
\pgfpathlineto{\pgfqpoint{1.331252in}{3.142498in}}%
\pgfpathlineto{\pgfqpoint{1.476560in}{3.142498in}}%
\pgfpathlineto{\pgfqpoint{1.476560in}{3.048124in}}%
\pgfpathmoveto{\pgfqpoint{1.331252in}{3.142498in}}%
\pgfpathlineto{\pgfqpoint{1.331252in}{3.142498in}}%
\pgfpathlineto{\pgfqpoint{1.331252in}{3.236876in}}%
\pgfpathlineto{\pgfqpoint{1.476560in}{3.236876in}}%
\pgfpathlineto{\pgfqpoint{1.476560in}{3.142498in}}%
\pgfpathmoveto{\pgfqpoint{1.331252in}{3.236876in}}%
\pgfpathlineto{\pgfqpoint{1.331252in}{3.236876in}}%
\pgfpathlineto{\pgfqpoint{1.331252in}{3.331248in}}%
\pgfpathlineto{\pgfqpoint{1.476560in}{3.331248in}}%
\pgfpathlineto{\pgfqpoint{1.476560in}{3.236876in}}%
\pgfpathmoveto{\pgfqpoint{1.476560in}{0.499998in}}%
\pgfpathlineto{\pgfqpoint{1.476560in}{0.499998in}}%
\pgfpathlineto{\pgfqpoint{1.476560in}{0.594373in}}%
\pgfpathlineto{\pgfqpoint{1.621879in}{0.594373in}}%
\pgfpathlineto{\pgfqpoint{1.621879in}{0.499998in}}%
\pgfpathmoveto{\pgfqpoint{1.476560in}{0.594373in}}%
\pgfpathlineto{\pgfqpoint{1.476560in}{0.594373in}}%
\pgfpathlineto{\pgfqpoint{1.476560in}{0.688753in}}%
\pgfpathlineto{\pgfqpoint{1.621879in}{0.688753in}}%
\pgfpathlineto{\pgfqpoint{1.621879in}{0.594373in}}%
\pgfpathmoveto{\pgfqpoint{1.476560in}{0.688753in}}%
\pgfpathlineto{\pgfqpoint{1.476560in}{0.688753in}}%
\pgfpathlineto{\pgfqpoint{1.476560in}{0.783125in}}%
\pgfpathlineto{\pgfqpoint{1.621879in}{0.783125in}}%
\pgfpathlineto{\pgfqpoint{1.621879in}{0.688753in}}%
\pgfpathmoveto{\pgfqpoint{1.476560in}{0.783125in}}%
\pgfpathlineto{\pgfqpoint{1.476560in}{0.783125in}}%
\pgfpathlineto{\pgfqpoint{1.476560in}{0.877501in}}%
\pgfpathlineto{\pgfqpoint{1.621879in}{0.877501in}}%
\pgfpathlineto{\pgfqpoint{1.621879in}{0.783125in}}%
\pgfpathmoveto{\pgfqpoint{1.476560in}{0.877501in}}%
\pgfpathlineto{\pgfqpoint{1.476560in}{0.877501in}}%
\pgfpathlineto{\pgfqpoint{1.476560in}{0.971874in}}%
\pgfpathlineto{\pgfqpoint{1.621879in}{0.971874in}}%
\pgfpathlineto{\pgfqpoint{1.621879in}{0.877501in}}%
\pgfpathmoveto{\pgfqpoint{1.476560in}{0.971874in}}%
\pgfpathlineto{\pgfqpoint{1.476560in}{0.971874in}}%
\pgfpathlineto{\pgfqpoint{1.476560in}{1.066247in}}%
\pgfpathlineto{\pgfqpoint{1.621879in}{1.066247in}}%
\pgfpathlineto{\pgfqpoint{1.621879in}{0.971874in}}%
\pgfpathmoveto{\pgfqpoint{1.476560in}{1.066247in}}%
\pgfpathlineto{\pgfqpoint{1.476560in}{1.066247in}}%
\pgfpathlineto{\pgfqpoint{1.476560in}{1.160624in}}%
\pgfpathlineto{\pgfqpoint{1.621879in}{1.160624in}}%
\pgfpathlineto{\pgfqpoint{1.621879in}{1.066247in}}%
\pgfpathmoveto{\pgfqpoint{1.476560in}{1.160624in}}%
\pgfpathlineto{\pgfqpoint{1.476560in}{1.160624in}}%
\pgfpathlineto{\pgfqpoint{1.476560in}{1.254999in}}%
\pgfpathlineto{\pgfqpoint{1.621879in}{1.254999in}}%
\pgfpathlineto{\pgfqpoint{1.621879in}{1.160624in}}%
\pgfpathmoveto{\pgfqpoint{1.476560in}{1.254999in}}%
\pgfpathlineto{\pgfqpoint{1.476560in}{1.254999in}}%
\pgfpathlineto{\pgfqpoint{1.476560in}{1.349373in}}%
\pgfpathlineto{\pgfqpoint{1.621879in}{1.349373in}}%
\pgfpathlineto{\pgfqpoint{1.621879in}{1.254999in}}%
\pgfpathmoveto{\pgfqpoint{1.476560in}{1.349373in}}%
\pgfpathlineto{\pgfqpoint{1.476560in}{1.349373in}}%
\pgfpathlineto{\pgfqpoint{1.476560in}{1.443752in}}%
\pgfpathlineto{\pgfqpoint{1.621879in}{1.443752in}}%
\pgfpathlineto{\pgfqpoint{1.621879in}{1.349373in}}%
\pgfpathmoveto{\pgfqpoint{1.476560in}{1.443752in}}%
\pgfpathlineto{\pgfqpoint{1.476560in}{1.443752in}}%
\pgfpathlineto{\pgfqpoint{1.476560in}{1.538128in}}%
\pgfpathlineto{\pgfqpoint{1.621879in}{1.538128in}}%
\pgfpathlineto{\pgfqpoint{1.621879in}{1.443752in}}%
\pgfpathmoveto{\pgfqpoint{1.476560in}{1.538128in}}%
\pgfpathlineto{\pgfqpoint{1.476560in}{1.538128in}}%
\pgfpathlineto{\pgfqpoint{1.476560in}{1.632499in}}%
\pgfpathlineto{\pgfqpoint{1.621879in}{1.632499in}}%
\pgfpathlineto{\pgfqpoint{1.621879in}{1.538128in}}%
\pgfpathmoveto{\pgfqpoint{1.476560in}{1.632499in}}%
\pgfpathlineto{\pgfqpoint{1.476560in}{1.632499in}}%
\pgfpathlineto{\pgfqpoint{1.476560in}{1.726877in}}%
\pgfpathlineto{\pgfqpoint{1.621879in}{1.726877in}}%
\pgfpathlineto{\pgfqpoint{1.621879in}{1.632499in}}%
\pgfpathmoveto{\pgfqpoint{1.476560in}{1.726877in}}%
\pgfpathlineto{\pgfqpoint{1.476560in}{1.726877in}}%
\pgfpathlineto{\pgfqpoint{1.476560in}{1.821249in}}%
\pgfpathlineto{\pgfqpoint{1.621879in}{1.821249in}}%
\pgfpathlineto{\pgfqpoint{1.621879in}{1.726877in}}%
\pgfpathmoveto{\pgfqpoint{1.476560in}{1.821249in}}%
\pgfpathlineto{\pgfqpoint{1.476560in}{1.821249in}}%
\pgfpathlineto{\pgfqpoint{1.476560in}{1.915628in}}%
\pgfpathlineto{\pgfqpoint{1.621879in}{1.915628in}}%
\pgfpathlineto{\pgfqpoint{1.621879in}{1.821249in}}%
\pgfpathmoveto{\pgfqpoint{1.476560in}{1.915628in}}%
\pgfpathlineto{\pgfqpoint{1.476560in}{1.915628in}}%
\pgfpathlineto{\pgfqpoint{1.476560in}{2.009998in}}%
\pgfpathlineto{\pgfqpoint{1.621879in}{2.009998in}}%
\pgfpathlineto{\pgfqpoint{1.621879in}{1.915628in}}%
\pgfpathmoveto{\pgfqpoint{1.476560in}{2.009998in}}%
\pgfpathlineto{\pgfqpoint{1.476560in}{2.009998in}}%
\pgfpathlineto{\pgfqpoint{1.476560in}{2.104374in}}%
\pgfpathlineto{\pgfqpoint{1.621879in}{2.104374in}}%
\pgfpathlineto{\pgfqpoint{1.621879in}{2.009998in}}%
\pgfpathmoveto{\pgfqpoint{1.476560in}{2.104374in}}%
\pgfpathlineto{\pgfqpoint{1.476560in}{2.104374in}}%
\pgfpathlineto{\pgfqpoint{1.476560in}{2.198749in}}%
\pgfpathlineto{\pgfqpoint{1.621879in}{2.198749in}}%
\pgfpathlineto{\pgfqpoint{1.621879in}{2.104374in}}%
\pgfpathmoveto{\pgfqpoint{1.476560in}{2.198749in}}%
\pgfpathlineto{\pgfqpoint{1.476560in}{2.198749in}}%
\pgfpathlineto{\pgfqpoint{1.476560in}{2.293123in}}%
\pgfpathlineto{\pgfqpoint{1.621879in}{2.293123in}}%
\pgfpathlineto{\pgfqpoint{1.621879in}{2.198749in}}%
\pgfpathmoveto{\pgfqpoint{1.476560in}{2.293123in}}%
\pgfpathlineto{\pgfqpoint{1.476560in}{2.293123in}}%
\pgfpathlineto{\pgfqpoint{1.476560in}{2.387498in}}%
\pgfpathlineto{\pgfqpoint{1.621879in}{2.387498in}}%
\pgfpathlineto{\pgfqpoint{1.621879in}{2.293123in}}%
\pgfpathmoveto{\pgfqpoint{1.476560in}{2.387498in}}%
\pgfpathlineto{\pgfqpoint{1.476560in}{2.387498in}}%
\pgfpathlineto{\pgfqpoint{1.476560in}{2.481875in}}%
\pgfpathlineto{\pgfqpoint{1.621879in}{2.481875in}}%
\pgfpathlineto{\pgfqpoint{1.621879in}{2.387498in}}%
\pgfpathmoveto{\pgfqpoint{1.476560in}{2.481875in}}%
\pgfpathlineto{\pgfqpoint{1.476560in}{2.481875in}}%
\pgfpathlineto{\pgfqpoint{1.476560in}{2.576249in}}%
\pgfpathlineto{\pgfqpoint{1.621879in}{2.576249in}}%
\pgfpathlineto{\pgfqpoint{1.621879in}{2.481875in}}%
\pgfpathmoveto{\pgfqpoint{1.476560in}{2.576249in}}%
\pgfpathlineto{\pgfqpoint{1.476560in}{2.576249in}}%
\pgfpathlineto{\pgfqpoint{1.476560in}{2.670624in}}%
\pgfpathlineto{\pgfqpoint{1.621879in}{2.670624in}}%
\pgfpathlineto{\pgfqpoint{1.621879in}{2.576249in}}%
\pgfpathmoveto{\pgfqpoint{1.476560in}{2.670624in}}%
\pgfpathlineto{\pgfqpoint{1.476560in}{2.670624in}}%
\pgfpathlineto{\pgfqpoint{1.476560in}{2.765003in}}%
\pgfpathlineto{\pgfqpoint{1.621879in}{2.765003in}}%
\pgfpathlineto{\pgfqpoint{1.621879in}{2.670624in}}%
\pgfpathmoveto{\pgfqpoint{1.476560in}{2.765003in}}%
\pgfpathlineto{\pgfqpoint{1.476560in}{2.765003in}}%
\pgfpathlineto{\pgfqpoint{1.476560in}{2.859372in}}%
\pgfpathlineto{\pgfqpoint{1.621879in}{2.859372in}}%
\pgfpathlineto{\pgfqpoint{1.621879in}{2.765003in}}%
\pgfpathmoveto{\pgfqpoint{1.476560in}{2.859372in}}%
\pgfpathlineto{\pgfqpoint{1.476560in}{2.859372in}}%
\pgfpathlineto{\pgfqpoint{1.476560in}{2.953749in}}%
\pgfpathlineto{\pgfqpoint{1.621879in}{2.953749in}}%
\pgfpathlineto{\pgfqpoint{1.621879in}{2.859372in}}%
\pgfpathmoveto{\pgfqpoint{1.476560in}{2.953749in}}%
\pgfpathlineto{\pgfqpoint{1.476560in}{2.953749in}}%
\pgfpathlineto{\pgfqpoint{1.476560in}{3.048124in}}%
\pgfpathlineto{\pgfqpoint{1.621879in}{3.048124in}}%
\pgfpathlineto{\pgfqpoint{1.621879in}{2.953749in}}%
\pgfpathmoveto{\pgfqpoint{1.476560in}{3.048124in}}%
\pgfpathlineto{\pgfqpoint{1.476560in}{3.048124in}}%
\pgfpathlineto{\pgfqpoint{1.476560in}{3.142498in}}%
\pgfpathlineto{\pgfqpoint{1.621879in}{3.142498in}}%
\pgfpathlineto{\pgfqpoint{1.621879in}{3.048124in}}%
\pgfpathmoveto{\pgfqpoint{1.476560in}{3.142498in}}%
\pgfpathlineto{\pgfqpoint{1.476560in}{3.142498in}}%
\pgfpathlineto{\pgfqpoint{1.476560in}{3.236876in}}%
\pgfpathlineto{\pgfqpoint{1.621879in}{3.236876in}}%
\pgfpathlineto{\pgfqpoint{1.621879in}{3.142498in}}%
\pgfpathmoveto{\pgfqpoint{1.621879in}{0.499998in}}%
\pgfpathlineto{\pgfqpoint{1.621879in}{0.499998in}}%
\pgfpathlineto{\pgfqpoint{1.621879in}{0.594373in}}%
\pgfpathlineto{\pgfqpoint{1.767189in}{0.594373in}}%
\pgfpathlineto{\pgfqpoint{1.767189in}{0.499998in}}%
\pgfpathmoveto{\pgfqpoint{1.621879in}{0.594373in}}%
\pgfpathlineto{\pgfqpoint{1.621879in}{0.594373in}}%
\pgfpathlineto{\pgfqpoint{1.621879in}{0.688753in}}%
\pgfpathlineto{\pgfqpoint{1.767189in}{0.688753in}}%
\pgfpathlineto{\pgfqpoint{1.767189in}{0.594373in}}%
\pgfpathmoveto{\pgfqpoint{1.621879in}{0.688753in}}%
\pgfpathlineto{\pgfqpoint{1.621879in}{0.688753in}}%
\pgfpathlineto{\pgfqpoint{1.621879in}{0.783125in}}%
\pgfpathlineto{\pgfqpoint{1.767189in}{0.783125in}}%
\pgfpathlineto{\pgfqpoint{1.767189in}{0.688753in}}%
\pgfpathmoveto{\pgfqpoint{1.621879in}{0.783125in}}%
\pgfpathlineto{\pgfqpoint{1.621879in}{0.783125in}}%
\pgfpathlineto{\pgfqpoint{1.621879in}{0.877501in}}%
\pgfpathlineto{\pgfqpoint{1.767189in}{0.877501in}}%
\pgfpathlineto{\pgfqpoint{1.767189in}{0.783125in}}%
\pgfpathmoveto{\pgfqpoint{1.621879in}{0.877501in}}%
\pgfpathlineto{\pgfqpoint{1.621879in}{0.877501in}}%
\pgfpathlineto{\pgfqpoint{1.621879in}{0.971874in}}%
\pgfpathlineto{\pgfqpoint{1.767189in}{0.971874in}}%
\pgfpathlineto{\pgfqpoint{1.767189in}{0.877501in}}%
\pgfpathmoveto{\pgfqpoint{1.621879in}{0.971874in}}%
\pgfpathlineto{\pgfqpoint{1.621879in}{0.971874in}}%
\pgfpathlineto{\pgfqpoint{1.621879in}{1.066247in}}%
\pgfpathlineto{\pgfqpoint{1.767189in}{1.066247in}}%
\pgfpathlineto{\pgfqpoint{1.767189in}{0.971874in}}%
\pgfpathmoveto{\pgfqpoint{1.621879in}{1.066247in}}%
\pgfpathlineto{\pgfqpoint{1.621879in}{1.066247in}}%
\pgfpathlineto{\pgfqpoint{1.621879in}{1.160624in}}%
\pgfpathlineto{\pgfqpoint{1.767189in}{1.160624in}}%
\pgfpathlineto{\pgfqpoint{1.767189in}{1.066247in}}%
\pgfpathmoveto{\pgfqpoint{1.621879in}{1.160624in}}%
\pgfpathlineto{\pgfqpoint{1.621879in}{1.160624in}}%
\pgfpathlineto{\pgfqpoint{1.621879in}{1.254999in}}%
\pgfpathlineto{\pgfqpoint{1.767189in}{1.254999in}}%
\pgfpathlineto{\pgfqpoint{1.767189in}{1.160624in}}%
\pgfpathmoveto{\pgfqpoint{1.621879in}{1.254999in}}%
\pgfpathlineto{\pgfqpoint{1.621879in}{1.254999in}}%
\pgfpathlineto{\pgfqpoint{1.621879in}{1.349373in}}%
\pgfpathlineto{\pgfqpoint{1.767189in}{1.349373in}}%
\pgfpathlineto{\pgfqpoint{1.767189in}{1.254999in}}%
\pgfpathmoveto{\pgfqpoint{1.621879in}{1.349373in}}%
\pgfpathlineto{\pgfqpoint{1.621879in}{1.349373in}}%
\pgfpathlineto{\pgfqpoint{1.621879in}{1.443752in}}%
\pgfpathlineto{\pgfqpoint{1.767189in}{1.443752in}}%
\pgfpathlineto{\pgfqpoint{1.767189in}{1.349373in}}%
\pgfpathmoveto{\pgfqpoint{1.621879in}{1.443752in}}%
\pgfpathlineto{\pgfqpoint{1.621879in}{1.443752in}}%
\pgfpathlineto{\pgfqpoint{1.621879in}{1.538128in}}%
\pgfpathlineto{\pgfqpoint{1.767189in}{1.538128in}}%
\pgfpathlineto{\pgfqpoint{1.767189in}{1.443752in}}%
\pgfpathmoveto{\pgfqpoint{1.621879in}{1.538128in}}%
\pgfpathlineto{\pgfqpoint{1.621879in}{1.538128in}}%
\pgfpathlineto{\pgfqpoint{1.621879in}{1.632499in}}%
\pgfpathlineto{\pgfqpoint{1.767189in}{1.632499in}}%
\pgfpathlineto{\pgfqpoint{1.767189in}{1.538128in}}%
\pgfpathmoveto{\pgfqpoint{1.621879in}{1.632499in}}%
\pgfpathlineto{\pgfqpoint{1.621879in}{1.632499in}}%
\pgfpathlineto{\pgfqpoint{1.621879in}{1.726877in}}%
\pgfpathlineto{\pgfqpoint{1.767189in}{1.726877in}}%
\pgfpathlineto{\pgfqpoint{1.767189in}{1.632499in}}%
\pgfpathmoveto{\pgfqpoint{1.621879in}{1.726877in}}%
\pgfpathlineto{\pgfqpoint{1.621879in}{1.726877in}}%
\pgfpathlineto{\pgfqpoint{1.621879in}{1.821249in}}%
\pgfpathlineto{\pgfqpoint{1.767189in}{1.821249in}}%
\pgfpathlineto{\pgfqpoint{1.767189in}{1.726877in}}%
\pgfpathmoveto{\pgfqpoint{1.621879in}{1.821249in}}%
\pgfpathlineto{\pgfqpoint{1.621879in}{1.821249in}}%
\pgfpathlineto{\pgfqpoint{1.621879in}{1.915628in}}%
\pgfpathlineto{\pgfqpoint{1.767189in}{1.915628in}}%
\pgfpathlineto{\pgfqpoint{1.767189in}{1.821249in}}%
\pgfpathmoveto{\pgfqpoint{1.621879in}{1.915628in}}%
\pgfpathlineto{\pgfqpoint{1.621879in}{1.915628in}}%
\pgfpathlineto{\pgfqpoint{1.621879in}{2.009998in}}%
\pgfpathlineto{\pgfqpoint{1.767189in}{2.009998in}}%
\pgfpathlineto{\pgfqpoint{1.767189in}{1.915628in}}%
\pgfpathmoveto{\pgfqpoint{1.621879in}{2.009998in}}%
\pgfpathlineto{\pgfqpoint{1.621879in}{2.009998in}}%
\pgfpathlineto{\pgfqpoint{1.621879in}{2.104374in}}%
\pgfpathlineto{\pgfqpoint{1.767189in}{2.104374in}}%
\pgfpathlineto{\pgfqpoint{1.767189in}{2.009998in}}%
\pgfpathmoveto{\pgfqpoint{1.621879in}{2.104374in}}%
\pgfpathlineto{\pgfqpoint{1.621879in}{2.104374in}}%
\pgfpathlineto{\pgfqpoint{1.621879in}{2.198749in}}%
\pgfpathlineto{\pgfqpoint{1.767189in}{2.198749in}}%
\pgfpathlineto{\pgfqpoint{1.767189in}{2.104374in}}%
\pgfpathmoveto{\pgfqpoint{1.621879in}{2.198749in}}%
\pgfpathlineto{\pgfqpoint{1.621879in}{2.198749in}}%
\pgfpathlineto{\pgfqpoint{1.621879in}{2.293123in}}%
\pgfpathlineto{\pgfqpoint{1.767189in}{2.293123in}}%
\pgfpathlineto{\pgfqpoint{1.767189in}{2.198749in}}%
\pgfpathmoveto{\pgfqpoint{1.621879in}{2.293123in}}%
\pgfpathlineto{\pgfqpoint{1.621879in}{2.293123in}}%
\pgfpathlineto{\pgfqpoint{1.621879in}{2.387498in}}%
\pgfpathlineto{\pgfqpoint{1.767189in}{2.387498in}}%
\pgfpathlineto{\pgfqpoint{1.767189in}{2.293123in}}%
\pgfpathmoveto{\pgfqpoint{1.621879in}{2.387498in}}%
\pgfpathlineto{\pgfqpoint{1.621879in}{2.387498in}}%
\pgfpathlineto{\pgfqpoint{1.621879in}{2.481875in}}%
\pgfpathlineto{\pgfqpoint{1.767189in}{2.481875in}}%
\pgfpathlineto{\pgfqpoint{1.767189in}{2.387498in}}%
\pgfpathmoveto{\pgfqpoint{1.621879in}{2.481875in}}%
\pgfpathlineto{\pgfqpoint{1.621879in}{2.481875in}}%
\pgfpathlineto{\pgfqpoint{1.621879in}{2.576249in}}%
\pgfpathlineto{\pgfqpoint{1.767189in}{2.576249in}}%
\pgfpathlineto{\pgfqpoint{1.767189in}{2.481875in}}%
\pgfpathmoveto{\pgfqpoint{1.621879in}{2.576249in}}%
\pgfpathlineto{\pgfqpoint{1.621879in}{2.576249in}}%
\pgfpathlineto{\pgfqpoint{1.621879in}{2.670624in}}%
\pgfpathlineto{\pgfqpoint{1.767189in}{2.670624in}}%
\pgfpathlineto{\pgfqpoint{1.767189in}{2.576249in}}%
\pgfpathmoveto{\pgfqpoint{1.621879in}{2.670624in}}%
\pgfpathlineto{\pgfqpoint{1.621879in}{2.670624in}}%
\pgfpathlineto{\pgfqpoint{1.621879in}{2.765003in}}%
\pgfpathlineto{\pgfqpoint{1.767189in}{2.765003in}}%
\pgfpathlineto{\pgfqpoint{1.767189in}{2.670624in}}%
\pgfpathmoveto{\pgfqpoint{1.621879in}{2.765003in}}%
\pgfpathlineto{\pgfqpoint{1.621879in}{2.765003in}}%
\pgfpathlineto{\pgfqpoint{1.621879in}{2.859372in}}%
\pgfpathlineto{\pgfqpoint{1.767189in}{2.859372in}}%
\pgfpathlineto{\pgfqpoint{1.767189in}{2.765003in}}%
\pgfpathmoveto{\pgfqpoint{1.621879in}{2.859372in}}%
\pgfpathlineto{\pgfqpoint{1.621879in}{2.859372in}}%
\pgfpathlineto{\pgfqpoint{1.621879in}{2.953749in}}%
\pgfpathlineto{\pgfqpoint{1.767189in}{2.953749in}}%
\pgfpathlineto{\pgfqpoint{1.767189in}{2.859372in}}%
\pgfpathmoveto{\pgfqpoint{1.621879in}{2.953749in}}%
\pgfpathlineto{\pgfqpoint{1.621879in}{2.953749in}}%
\pgfpathlineto{\pgfqpoint{1.621879in}{3.048124in}}%
\pgfpathlineto{\pgfqpoint{1.767189in}{3.048124in}}%
\pgfpathlineto{\pgfqpoint{1.767189in}{2.953749in}}%
\pgfpathmoveto{\pgfqpoint{1.621879in}{3.048124in}}%
\pgfpathlineto{\pgfqpoint{1.621879in}{3.048124in}}%
\pgfpathlineto{\pgfqpoint{1.621879in}{3.142498in}}%
\pgfpathlineto{\pgfqpoint{1.767189in}{3.142498in}}%
\pgfpathlineto{\pgfqpoint{1.767189in}{3.048124in}}%
\pgfpathmoveto{\pgfqpoint{1.767189in}{0.499998in}}%
\pgfpathlineto{\pgfqpoint{1.767189in}{0.499998in}}%
\pgfpathlineto{\pgfqpoint{1.767189in}{0.594373in}}%
\pgfpathlineto{\pgfqpoint{1.912499in}{0.594373in}}%
\pgfpathlineto{\pgfqpoint{1.912499in}{0.499998in}}%
\pgfpathmoveto{\pgfqpoint{1.767189in}{0.594373in}}%
\pgfpathlineto{\pgfqpoint{1.767189in}{0.594373in}}%
\pgfpathlineto{\pgfqpoint{1.767189in}{0.688753in}}%
\pgfpathlineto{\pgfqpoint{1.912499in}{0.688753in}}%
\pgfpathlineto{\pgfqpoint{1.912499in}{0.594373in}}%
\pgfpathmoveto{\pgfqpoint{1.767189in}{0.688753in}}%
\pgfpathlineto{\pgfqpoint{1.767189in}{0.688753in}}%
\pgfpathlineto{\pgfqpoint{1.767189in}{0.783125in}}%
\pgfpathlineto{\pgfqpoint{1.912499in}{0.783125in}}%
\pgfpathlineto{\pgfqpoint{1.912499in}{0.688753in}}%
\pgfpathmoveto{\pgfqpoint{1.767189in}{0.783125in}}%
\pgfpathlineto{\pgfqpoint{1.767189in}{0.783125in}}%
\pgfpathlineto{\pgfqpoint{1.767189in}{0.877501in}}%
\pgfpathlineto{\pgfqpoint{1.912499in}{0.877501in}}%
\pgfpathlineto{\pgfqpoint{1.912499in}{0.783125in}}%
\pgfpathmoveto{\pgfqpoint{1.767189in}{0.877501in}}%
\pgfpathlineto{\pgfqpoint{1.767189in}{0.877501in}}%
\pgfpathlineto{\pgfqpoint{1.767189in}{0.971874in}}%
\pgfpathlineto{\pgfqpoint{1.912499in}{0.971874in}}%
\pgfpathlineto{\pgfqpoint{1.912499in}{0.877501in}}%
\pgfpathmoveto{\pgfqpoint{1.767189in}{0.971874in}}%
\pgfpathlineto{\pgfqpoint{1.767189in}{0.971874in}}%
\pgfpathlineto{\pgfqpoint{1.767189in}{1.066247in}}%
\pgfpathlineto{\pgfqpoint{1.912499in}{1.066247in}}%
\pgfpathlineto{\pgfqpoint{1.912499in}{0.971874in}}%
\pgfpathmoveto{\pgfqpoint{1.767189in}{1.066247in}}%
\pgfpathlineto{\pgfqpoint{1.767189in}{1.066247in}}%
\pgfpathlineto{\pgfqpoint{1.767189in}{1.160624in}}%
\pgfpathlineto{\pgfqpoint{1.912499in}{1.160624in}}%
\pgfpathlineto{\pgfqpoint{1.912499in}{1.066247in}}%
\pgfpathmoveto{\pgfqpoint{1.767189in}{1.160624in}}%
\pgfpathlineto{\pgfqpoint{1.767189in}{1.160624in}}%
\pgfpathlineto{\pgfqpoint{1.767189in}{1.254999in}}%
\pgfpathlineto{\pgfqpoint{1.912499in}{1.254999in}}%
\pgfpathlineto{\pgfqpoint{1.912499in}{1.160624in}}%
\pgfpathmoveto{\pgfqpoint{1.767189in}{1.254999in}}%
\pgfpathlineto{\pgfqpoint{1.767189in}{1.254999in}}%
\pgfpathlineto{\pgfqpoint{1.767189in}{1.349373in}}%
\pgfpathlineto{\pgfqpoint{1.912499in}{1.349373in}}%
\pgfpathlineto{\pgfqpoint{1.912499in}{1.254999in}}%
\pgfpathmoveto{\pgfqpoint{1.767189in}{1.349373in}}%
\pgfpathlineto{\pgfqpoint{1.767189in}{1.349373in}}%
\pgfpathlineto{\pgfqpoint{1.767189in}{1.443752in}}%
\pgfpathlineto{\pgfqpoint{1.912499in}{1.443752in}}%
\pgfpathlineto{\pgfqpoint{1.912499in}{1.349373in}}%
\pgfpathmoveto{\pgfqpoint{1.767189in}{1.443752in}}%
\pgfpathlineto{\pgfqpoint{1.767189in}{1.443752in}}%
\pgfpathlineto{\pgfqpoint{1.767189in}{1.538128in}}%
\pgfpathlineto{\pgfqpoint{1.912499in}{1.538128in}}%
\pgfpathlineto{\pgfqpoint{1.912499in}{1.443752in}}%
\pgfpathmoveto{\pgfqpoint{1.767189in}{1.538128in}}%
\pgfpathlineto{\pgfqpoint{1.767189in}{1.538128in}}%
\pgfpathlineto{\pgfqpoint{1.767189in}{1.632499in}}%
\pgfpathlineto{\pgfqpoint{1.912499in}{1.632499in}}%
\pgfpathlineto{\pgfqpoint{1.912499in}{1.538128in}}%
\pgfpathmoveto{\pgfqpoint{1.767189in}{1.632499in}}%
\pgfpathlineto{\pgfqpoint{1.767189in}{1.632499in}}%
\pgfpathlineto{\pgfqpoint{1.767189in}{1.726877in}}%
\pgfpathlineto{\pgfqpoint{1.912499in}{1.726877in}}%
\pgfpathlineto{\pgfqpoint{1.912499in}{1.632499in}}%
\pgfpathmoveto{\pgfqpoint{1.767189in}{1.726877in}}%
\pgfpathlineto{\pgfqpoint{1.767189in}{1.726877in}}%
\pgfpathlineto{\pgfqpoint{1.767189in}{1.821249in}}%
\pgfpathlineto{\pgfqpoint{1.912499in}{1.821249in}}%
\pgfpathlineto{\pgfqpoint{1.912499in}{1.726877in}}%
\pgfpathmoveto{\pgfqpoint{1.767189in}{1.821249in}}%
\pgfpathlineto{\pgfqpoint{1.767189in}{1.821249in}}%
\pgfpathlineto{\pgfqpoint{1.767189in}{1.915628in}}%
\pgfpathlineto{\pgfqpoint{1.912499in}{1.915628in}}%
\pgfpathlineto{\pgfqpoint{1.912499in}{1.821249in}}%
\pgfpathmoveto{\pgfqpoint{1.767189in}{1.915628in}}%
\pgfpathlineto{\pgfqpoint{1.767189in}{1.915628in}}%
\pgfpathlineto{\pgfqpoint{1.767189in}{2.009998in}}%
\pgfpathlineto{\pgfqpoint{1.912499in}{2.009998in}}%
\pgfpathlineto{\pgfqpoint{1.912499in}{1.915628in}}%
\pgfpathmoveto{\pgfqpoint{1.767189in}{2.009998in}}%
\pgfpathlineto{\pgfqpoint{1.767189in}{2.009998in}}%
\pgfpathlineto{\pgfqpoint{1.767189in}{2.104374in}}%
\pgfpathlineto{\pgfqpoint{1.912499in}{2.104374in}}%
\pgfpathlineto{\pgfqpoint{1.912499in}{2.009998in}}%
\pgfpathmoveto{\pgfqpoint{1.767189in}{2.104374in}}%
\pgfpathlineto{\pgfqpoint{1.767189in}{2.104374in}}%
\pgfpathlineto{\pgfqpoint{1.767189in}{2.198749in}}%
\pgfpathlineto{\pgfqpoint{1.912499in}{2.198749in}}%
\pgfpathlineto{\pgfqpoint{1.912499in}{2.104374in}}%
\pgfpathmoveto{\pgfqpoint{1.767189in}{2.198749in}}%
\pgfpathlineto{\pgfqpoint{1.767189in}{2.198749in}}%
\pgfpathlineto{\pgfqpoint{1.767189in}{2.293123in}}%
\pgfpathlineto{\pgfqpoint{1.912499in}{2.293123in}}%
\pgfpathlineto{\pgfqpoint{1.912499in}{2.198749in}}%
\pgfpathmoveto{\pgfqpoint{1.767189in}{2.293123in}}%
\pgfpathlineto{\pgfqpoint{1.767189in}{2.293123in}}%
\pgfpathlineto{\pgfqpoint{1.767189in}{2.387498in}}%
\pgfpathlineto{\pgfqpoint{1.912499in}{2.387498in}}%
\pgfpathlineto{\pgfqpoint{1.912499in}{2.293123in}}%
\pgfpathmoveto{\pgfqpoint{1.767189in}{2.387498in}}%
\pgfpathlineto{\pgfqpoint{1.767189in}{2.387498in}}%
\pgfpathlineto{\pgfqpoint{1.767189in}{2.481875in}}%
\pgfpathlineto{\pgfqpoint{1.912499in}{2.481875in}}%
\pgfpathlineto{\pgfqpoint{1.912499in}{2.387498in}}%
\pgfpathmoveto{\pgfqpoint{1.767189in}{2.481875in}}%
\pgfpathlineto{\pgfqpoint{1.767189in}{2.481875in}}%
\pgfpathlineto{\pgfqpoint{1.767189in}{2.576249in}}%
\pgfpathlineto{\pgfqpoint{1.912499in}{2.576249in}}%
\pgfpathlineto{\pgfqpoint{1.912499in}{2.481875in}}%
\pgfpathmoveto{\pgfqpoint{1.767189in}{2.576249in}}%
\pgfpathlineto{\pgfqpoint{1.767189in}{2.576249in}}%
\pgfpathlineto{\pgfqpoint{1.767189in}{2.670624in}}%
\pgfpathlineto{\pgfqpoint{1.912499in}{2.670624in}}%
\pgfpathlineto{\pgfqpoint{1.912499in}{2.576249in}}%
\pgfpathmoveto{\pgfqpoint{1.767189in}{2.670624in}}%
\pgfpathlineto{\pgfqpoint{1.767189in}{2.670624in}}%
\pgfpathlineto{\pgfqpoint{1.767189in}{2.765003in}}%
\pgfpathlineto{\pgfqpoint{1.912499in}{2.765003in}}%
\pgfpathlineto{\pgfqpoint{1.912499in}{2.670624in}}%
\pgfpathmoveto{\pgfqpoint{1.767189in}{2.765003in}}%
\pgfpathlineto{\pgfqpoint{1.767189in}{2.765003in}}%
\pgfpathlineto{\pgfqpoint{1.767189in}{2.859372in}}%
\pgfpathlineto{\pgfqpoint{1.912499in}{2.859372in}}%
\pgfpathlineto{\pgfqpoint{1.912499in}{2.765003in}}%
\pgfpathmoveto{\pgfqpoint{1.767189in}{2.859372in}}%
\pgfpathlineto{\pgfqpoint{1.767189in}{2.859372in}}%
\pgfpathlineto{\pgfqpoint{1.767189in}{2.953749in}}%
\pgfpathlineto{\pgfqpoint{1.912499in}{2.953749in}}%
\pgfpathlineto{\pgfqpoint{1.912499in}{2.859372in}}%
\pgfpathmoveto{\pgfqpoint{1.912499in}{0.499998in}}%
\pgfpathlineto{\pgfqpoint{1.912499in}{0.499998in}}%
\pgfpathlineto{\pgfqpoint{1.912499in}{0.594373in}}%
\pgfpathlineto{\pgfqpoint{2.057816in}{0.594373in}}%
\pgfpathlineto{\pgfqpoint{2.057816in}{0.499998in}}%
\pgfpathmoveto{\pgfqpoint{1.912499in}{0.594373in}}%
\pgfpathlineto{\pgfqpoint{1.912499in}{0.594373in}}%
\pgfpathlineto{\pgfqpoint{1.912499in}{0.688753in}}%
\pgfpathlineto{\pgfqpoint{2.057816in}{0.688753in}}%
\pgfpathlineto{\pgfqpoint{2.057816in}{0.594373in}}%
\pgfpathmoveto{\pgfqpoint{1.912499in}{0.688753in}}%
\pgfpathlineto{\pgfqpoint{1.912499in}{0.688753in}}%
\pgfpathlineto{\pgfqpoint{1.912499in}{0.783125in}}%
\pgfpathlineto{\pgfqpoint{2.057816in}{0.783125in}}%
\pgfpathlineto{\pgfqpoint{2.057816in}{0.688753in}}%
\pgfpathmoveto{\pgfqpoint{1.912499in}{0.783125in}}%
\pgfpathlineto{\pgfqpoint{1.912499in}{0.783125in}}%
\pgfpathlineto{\pgfqpoint{1.912499in}{0.877501in}}%
\pgfpathlineto{\pgfqpoint{2.057816in}{0.877501in}}%
\pgfpathlineto{\pgfqpoint{2.057816in}{0.783125in}}%
\pgfpathmoveto{\pgfqpoint{1.912499in}{0.877501in}}%
\pgfpathlineto{\pgfqpoint{1.912499in}{0.877501in}}%
\pgfpathlineto{\pgfqpoint{1.912499in}{0.971874in}}%
\pgfpathlineto{\pgfqpoint{2.057816in}{0.971874in}}%
\pgfpathlineto{\pgfqpoint{2.057816in}{0.877501in}}%
\pgfpathmoveto{\pgfqpoint{1.912499in}{0.971874in}}%
\pgfpathlineto{\pgfqpoint{1.912499in}{0.971874in}}%
\pgfpathlineto{\pgfqpoint{1.912499in}{1.066247in}}%
\pgfpathlineto{\pgfqpoint{2.057816in}{1.066247in}}%
\pgfpathlineto{\pgfqpoint{2.057816in}{0.971874in}}%
\pgfpathmoveto{\pgfqpoint{1.912499in}{1.066247in}}%
\pgfpathlineto{\pgfqpoint{1.912499in}{1.066247in}}%
\pgfpathlineto{\pgfqpoint{1.912499in}{1.160624in}}%
\pgfpathlineto{\pgfqpoint{2.057816in}{1.160624in}}%
\pgfpathlineto{\pgfqpoint{2.057816in}{1.066247in}}%
\pgfpathmoveto{\pgfqpoint{1.912499in}{1.160624in}}%
\pgfpathlineto{\pgfqpoint{1.912499in}{1.160624in}}%
\pgfpathlineto{\pgfqpoint{1.912499in}{1.254999in}}%
\pgfpathlineto{\pgfqpoint{2.057816in}{1.254999in}}%
\pgfpathlineto{\pgfqpoint{2.057816in}{1.160624in}}%
\pgfpathmoveto{\pgfqpoint{1.912499in}{1.254999in}}%
\pgfpathlineto{\pgfqpoint{1.912499in}{1.254999in}}%
\pgfpathlineto{\pgfqpoint{1.912499in}{1.349373in}}%
\pgfpathlineto{\pgfqpoint{2.057816in}{1.349373in}}%
\pgfpathlineto{\pgfqpoint{2.057816in}{1.254999in}}%
\pgfpathmoveto{\pgfqpoint{1.912499in}{1.349373in}}%
\pgfpathlineto{\pgfqpoint{1.912499in}{1.349373in}}%
\pgfpathlineto{\pgfqpoint{1.912499in}{1.443752in}}%
\pgfpathlineto{\pgfqpoint{2.057816in}{1.443752in}}%
\pgfpathlineto{\pgfqpoint{2.057816in}{1.349373in}}%
\pgfpathmoveto{\pgfqpoint{1.912499in}{1.443752in}}%
\pgfpathlineto{\pgfqpoint{1.912499in}{1.443752in}}%
\pgfpathlineto{\pgfqpoint{1.912499in}{1.538128in}}%
\pgfpathlineto{\pgfqpoint{2.057816in}{1.538128in}}%
\pgfpathlineto{\pgfqpoint{2.057816in}{1.443752in}}%
\pgfpathmoveto{\pgfqpoint{1.912499in}{1.538128in}}%
\pgfpathlineto{\pgfqpoint{1.912499in}{1.538128in}}%
\pgfpathlineto{\pgfqpoint{1.912499in}{1.632499in}}%
\pgfpathlineto{\pgfqpoint{2.057816in}{1.632499in}}%
\pgfpathlineto{\pgfqpoint{2.057816in}{1.538128in}}%
\pgfpathmoveto{\pgfqpoint{1.912499in}{1.632499in}}%
\pgfpathlineto{\pgfqpoint{1.912499in}{1.632499in}}%
\pgfpathlineto{\pgfqpoint{1.912499in}{1.726877in}}%
\pgfpathlineto{\pgfqpoint{2.057816in}{1.726877in}}%
\pgfpathlineto{\pgfqpoint{2.057816in}{1.632499in}}%
\pgfpathmoveto{\pgfqpoint{1.912499in}{1.726877in}}%
\pgfpathlineto{\pgfqpoint{1.912499in}{1.726877in}}%
\pgfpathlineto{\pgfqpoint{1.912499in}{1.821249in}}%
\pgfpathlineto{\pgfqpoint{2.057816in}{1.821249in}}%
\pgfpathlineto{\pgfqpoint{2.057816in}{1.726877in}}%
\pgfpathmoveto{\pgfqpoint{1.912499in}{1.821249in}}%
\pgfpathlineto{\pgfqpoint{1.912499in}{1.821249in}}%
\pgfpathlineto{\pgfqpoint{1.912499in}{1.915628in}}%
\pgfpathlineto{\pgfqpoint{2.057816in}{1.915628in}}%
\pgfpathlineto{\pgfqpoint{2.057816in}{1.821249in}}%
\pgfpathmoveto{\pgfqpoint{1.912499in}{1.915628in}}%
\pgfpathlineto{\pgfqpoint{1.912499in}{1.915628in}}%
\pgfpathlineto{\pgfqpoint{1.912499in}{2.009998in}}%
\pgfpathlineto{\pgfqpoint{2.057816in}{2.009998in}}%
\pgfpathlineto{\pgfqpoint{2.057816in}{1.915628in}}%
\pgfpathmoveto{\pgfqpoint{1.912499in}{2.009998in}}%
\pgfpathlineto{\pgfqpoint{1.912499in}{2.009998in}}%
\pgfpathlineto{\pgfqpoint{1.912499in}{2.104374in}}%
\pgfpathlineto{\pgfqpoint{2.057816in}{2.104374in}}%
\pgfpathlineto{\pgfqpoint{2.057816in}{2.009998in}}%
\pgfpathmoveto{\pgfqpoint{1.912499in}{2.104374in}}%
\pgfpathlineto{\pgfqpoint{1.912499in}{2.104374in}}%
\pgfpathlineto{\pgfqpoint{1.912499in}{2.198749in}}%
\pgfpathlineto{\pgfqpoint{2.057816in}{2.198749in}}%
\pgfpathlineto{\pgfqpoint{2.057816in}{2.104374in}}%
\pgfpathmoveto{\pgfqpoint{1.912499in}{2.198749in}}%
\pgfpathlineto{\pgfqpoint{1.912499in}{2.198749in}}%
\pgfpathlineto{\pgfqpoint{1.912499in}{2.293123in}}%
\pgfpathlineto{\pgfqpoint{2.057816in}{2.293123in}}%
\pgfpathlineto{\pgfqpoint{2.057816in}{2.198749in}}%
\pgfpathmoveto{\pgfqpoint{1.912499in}{2.293123in}}%
\pgfpathlineto{\pgfqpoint{1.912499in}{2.293123in}}%
\pgfpathlineto{\pgfqpoint{1.912499in}{2.387498in}}%
\pgfpathlineto{\pgfqpoint{2.057816in}{2.387498in}}%
\pgfpathlineto{\pgfqpoint{2.057816in}{2.293123in}}%
\pgfpathmoveto{\pgfqpoint{1.912499in}{2.387498in}}%
\pgfpathlineto{\pgfqpoint{1.912499in}{2.387498in}}%
\pgfpathlineto{\pgfqpoint{1.912499in}{2.481875in}}%
\pgfpathlineto{\pgfqpoint{2.057816in}{2.481875in}}%
\pgfpathlineto{\pgfqpoint{2.057816in}{2.387498in}}%
\pgfpathmoveto{\pgfqpoint{1.912499in}{2.481875in}}%
\pgfpathlineto{\pgfqpoint{1.912499in}{2.481875in}}%
\pgfpathlineto{\pgfqpoint{1.912499in}{2.576249in}}%
\pgfpathlineto{\pgfqpoint{2.057816in}{2.576249in}}%
\pgfpathlineto{\pgfqpoint{2.057816in}{2.481875in}}%
\pgfpathmoveto{\pgfqpoint{1.912499in}{2.576249in}}%
\pgfpathlineto{\pgfqpoint{1.912499in}{2.576249in}}%
\pgfpathlineto{\pgfqpoint{1.912499in}{2.670624in}}%
\pgfpathlineto{\pgfqpoint{2.057816in}{2.670624in}}%
\pgfpathlineto{\pgfqpoint{2.057816in}{2.576249in}}%
\pgfpathmoveto{\pgfqpoint{1.912499in}{2.670624in}}%
\pgfpathlineto{\pgfqpoint{1.912499in}{2.670624in}}%
\pgfpathlineto{\pgfqpoint{1.912499in}{2.765003in}}%
\pgfpathlineto{\pgfqpoint{2.057816in}{2.765003in}}%
\pgfpathlineto{\pgfqpoint{2.057816in}{2.670624in}}%
\pgfpathmoveto{\pgfqpoint{1.912499in}{2.765003in}}%
\pgfpathlineto{\pgfqpoint{1.912499in}{2.765003in}}%
\pgfpathlineto{\pgfqpoint{1.912499in}{2.859372in}}%
\pgfpathlineto{\pgfqpoint{2.057816in}{2.859372in}}%
\pgfpathlineto{\pgfqpoint{2.057816in}{2.765003in}}%
\pgfpathmoveto{\pgfqpoint{2.057816in}{0.499998in}}%
\pgfpathlineto{\pgfqpoint{2.057816in}{0.499998in}}%
\pgfpathlineto{\pgfqpoint{2.057816in}{0.594373in}}%
\pgfpathlineto{\pgfqpoint{2.203122in}{0.594373in}}%
\pgfpathlineto{\pgfqpoint{2.203122in}{0.499998in}}%
\pgfpathmoveto{\pgfqpoint{2.057816in}{0.594373in}}%
\pgfpathlineto{\pgfqpoint{2.057816in}{0.594373in}}%
\pgfpathlineto{\pgfqpoint{2.057816in}{0.688753in}}%
\pgfpathlineto{\pgfqpoint{2.203122in}{0.688753in}}%
\pgfpathlineto{\pgfqpoint{2.203122in}{0.594373in}}%
\pgfpathmoveto{\pgfqpoint{2.057816in}{0.688753in}}%
\pgfpathlineto{\pgfqpoint{2.057816in}{0.688753in}}%
\pgfpathlineto{\pgfqpoint{2.057816in}{0.783125in}}%
\pgfpathlineto{\pgfqpoint{2.203122in}{0.783125in}}%
\pgfpathlineto{\pgfqpoint{2.203122in}{0.688753in}}%
\pgfpathmoveto{\pgfqpoint{2.057816in}{0.783125in}}%
\pgfpathlineto{\pgfqpoint{2.057816in}{0.783125in}}%
\pgfpathlineto{\pgfqpoint{2.057816in}{0.877501in}}%
\pgfpathlineto{\pgfqpoint{2.203122in}{0.877501in}}%
\pgfpathlineto{\pgfqpoint{2.203122in}{0.783125in}}%
\pgfpathmoveto{\pgfqpoint{2.057816in}{0.877501in}}%
\pgfpathlineto{\pgfqpoint{2.057816in}{0.877501in}}%
\pgfpathlineto{\pgfqpoint{2.057816in}{0.971874in}}%
\pgfpathlineto{\pgfqpoint{2.203122in}{0.971874in}}%
\pgfpathlineto{\pgfqpoint{2.203122in}{0.877501in}}%
\pgfpathmoveto{\pgfqpoint{2.057816in}{0.971874in}}%
\pgfpathlineto{\pgfqpoint{2.057816in}{0.971874in}}%
\pgfpathlineto{\pgfqpoint{2.057816in}{1.066247in}}%
\pgfpathlineto{\pgfqpoint{2.203122in}{1.066247in}}%
\pgfpathlineto{\pgfqpoint{2.203122in}{0.971874in}}%
\pgfpathmoveto{\pgfqpoint{2.057816in}{1.066247in}}%
\pgfpathlineto{\pgfqpoint{2.057816in}{1.066247in}}%
\pgfpathlineto{\pgfqpoint{2.057816in}{1.160624in}}%
\pgfpathlineto{\pgfqpoint{2.203122in}{1.160624in}}%
\pgfpathlineto{\pgfqpoint{2.203122in}{1.066247in}}%
\pgfpathmoveto{\pgfqpoint{2.057816in}{1.160624in}}%
\pgfpathlineto{\pgfqpoint{2.057816in}{1.160624in}}%
\pgfpathlineto{\pgfqpoint{2.057816in}{1.254999in}}%
\pgfpathlineto{\pgfqpoint{2.203122in}{1.254999in}}%
\pgfpathlineto{\pgfqpoint{2.203122in}{1.160624in}}%
\pgfpathmoveto{\pgfqpoint{2.057816in}{1.254999in}}%
\pgfpathlineto{\pgfqpoint{2.057816in}{1.254999in}}%
\pgfpathlineto{\pgfqpoint{2.057816in}{1.349373in}}%
\pgfpathlineto{\pgfqpoint{2.203122in}{1.349373in}}%
\pgfpathlineto{\pgfqpoint{2.203122in}{1.254999in}}%
\pgfpathmoveto{\pgfqpoint{2.057816in}{1.349373in}}%
\pgfpathlineto{\pgfqpoint{2.057816in}{1.349373in}}%
\pgfpathlineto{\pgfqpoint{2.057816in}{1.443752in}}%
\pgfpathlineto{\pgfqpoint{2.203122in}{1.443752in}}%
\pgfpathlineto{\pgfqpoint{2.203122in}{1.349373in}}%
\pgfpathmoveto{\pgfqpoint{2.057816in}{1.443752in}}%
\pgfpathlineto{\pgfqpoint{2.057816in}{1.443752in}}%
\pgfpathlineto{\pgfqpoint{2.057816in}{1.538128in}}%
\pgfpathlineto{\pgfqpoint{2.203122in}{1.538128in}}%
\pgfpathlineto{\pgfqpoint{2.203122in}{1.443752in}}%
\pgfpathmoveto{\pgfqpoint{2.057816in}{1.538128in}}%
\pgfpathlineto{\pgfqpoint{2.057816in}{1.538128in}}%
\pgfpathlineto{\pgfqpoint{2.057816in}{1.632499in}}%
\pgfpathlineto{\pgfqpoint{2.203122in}{1.632499in}}%
\pgfpathlineto{\pgfqpoint{2.203122in}{1.538128in}}%
\pgfpathmoveto{\pgfqpoint{2.057816in}{1.632499in}}%
\pgfpathlineto{\pgfqpoint{2.057816in}{1.632499in}}%
\pgfpathlineto{\pgfqpoint{2.057816in}{1.726877in}}%
\pgfpathlineto{\pgfqpoint{2.203122in}{1.726877in}}%
\pgfpathlineto{\pgfqpoint{2.203122in}{1.632499in}}%
\pgfpathmoveto{\pgfqpoint{2.057816in}{1.726877in}}%
\pgfpathlineto{\pgfqpoint{2.057816in}{1.726877in}}%
\pgfpathlineto{\pgfqpoint{2.057816in}{1.821249in}}%
\pgfpathlineto{\pgfqpoint{2.203122in}{1.821249in}}%
\pgfpathlineto{\pgfqpoint{2.203122in}{1.726877in}}%
\pgfpathmoveto{\pgfqpoint{2.057816in}{1.821249in}}%
\pgfpathlineto{\pgfqpoint{2.057816in}{1.821249in}}%
\pgfpathlineto{\pgfqpoint{2.057816in}{1.915628in}}%
\pgfpathlineto{\pgfqpoint{2.203122in}{1.915628in}}%
\pgfpathlineto{\pgfqpoint{2.203122in}{1.821249in}}%
\pgfpathmoveto{\pgfqpoint{2.057816in}{1.915628in}}%
\pgfpathlineto{\pgfqpoint{2.057816in}{1.915628in}}%
\pgfpathlineto{\pgfqpoint{2.057816in}{2.009998in}}%
\pgfpathlineto{\pgfqpoint{2.203122in}{2.009998in}}%
\pgfpathlineto{\pgfqpoint{2.203122in}{1.915628in}}%
\pgfpathmoveto{\pgfqpoint{2.057816in}{2.009998in}}%
\pgfpathlineto{\pgfqpoint{2.057816in}{2.009998in}}%
\pgfpathlineto{\pgfqpoint{2.057816in}{2.104374in}}%
\pgfpathlineto{\pgfqpoint{2.203122in}{2.104374in}}%
\pgfpathlineto{\pgfqpoint{2.203122in}{2.009998in}}%
\pgfpathmoveto{\pgfqpoint{2.057816in}{2.104374in}}%
\pgfpathlineto{\pgfqpoint{2.057816in}{2.104374in}}%
\pgfpathlineto{\pgfqpoint{2.057816in}{2.198749in}}%
\pgfpathlineto{\pgfqpoint{2.203122in}{2.198749in}}%
\pgfpathlineto{\pgfqpoint{2.203122in}{2.104374in}}%
\pgfpathmoveto{\pgfqpoint{2.057816in}{2.198749in}}%
\pgfpathlineto{\pgfqpoint{2.057816in}{2.198749in}}%
\pgfpathlineto{\pgfqpoint{2.057816in}{2.293123in}}%
\pgfpathlineto{\pgfqpoint{2.203122in}{2.293123in}}%
\pgfpathlineto{\pgfqpoint{2.203122in}{2.198749in}}%
\pgfpathmoveto{\pgfqpoint{2.057816in}{2.293123in}}%
\pgfpathlineto{\pgfqpoint{2.057816in}{2.293123in}}%
\pgfpathlineto{\pgfqpoint{2.057816in}{2.387498in}}%
\pgfpathlineto{\pgfqpoint{2.203122in}{2.387498in}}%
\pgfpathlineto{\pgfqpoint{2.203122in}{2.293123in}}%
\pgfpathmoveto{\pgfqpoint{2.057816in}{2.387498in}}%
\pgfpathlineto{\pgfqpoint{2.057816in}{2.387498in}}%
\pgfpathlineto{\pgfqpoint{2.057816in}{2.481875in}}%
\pgfpathlineto{\pgfqpoint{2.203122in}{2.481875in}}%
\pgfpathlineto{\pgfqpoint{2.203122in}{2.387498in}}%
\pgfpathmoveto{\pgfqpoint{2.057816in}{2.481875in}}%
\pgfpathlineto{\pgfqpoint{2.057816in}{2.481875in}}%
\pgfpathlineto{\pgfqpoint{2.057816in}{2.576249in}}%
\pgfpathlineto{\pgfqpoint{2.203122in}{2.576249in}}%
\pgfpathlineto{\pgfqpoint{2.203122in}{2.481875in}}%
\pgfpathmoveto{\pgfqpoint{2.057816in}{2.576249in}}%
\pgfpathlineto{\pgfqpoint{2.057816in}{2.576249in}}%
\pgfpathlineto{\pgfqpoint{2.057816in}{2.670624in}}%
\pgfpathlineto{\pgfqpoint{2.203122in}{2.670624in}}%
\pgfpathlineto{\pgfqpoint{2.203122in}{2.576249in}}%
\pgfpathmoveto{\pgfqpoint{2.057816in}{2.670624in}}%
\pgfpathlineto{\pgfqpoint{2.057816in}{2.670624in}}%
\pgfpathlineto{\pgfqpoint{2.057816in}{2.765003in}}%
\pgfpathlineto{\pgfqpoint{2.203122in}{2.765003in}}%
\pgfpathlineto{\pgfqpoint{2.203122in}{2.670624in}}%
\pgfpathmoveto{\pgfqpoint{2.203122in}{0.499998in}}%
\pgfpathlineto{\pgfqpoint{2.203122in}{0.499998in}}%
\pgfpathlineto{\pgfqpoint{2.203122in}{0.594373in}}%
\pgfpathlineto{\pgfqpoint{2.348439in}{0.594373in}}%
\pgfpathlineto{\pgfqpoint{2.348439in}{0.499998in}}%
\pgfpathmoveto{\pgfqpoint{2.203122in}{0.594373in}}%
\pgfpathlineto{\pgfqpoint{2.203122in}{0.594373in}}%
\pgfpathlineto{\pgfqpoint{2.203122in}{0.688753in}}%
\pgfpathlineto{\pgfqpoint{2.348439in}{0.688753in}}%
\pgfpathlineto{\pgfqpoint{2.348439in}{0.594373in}}%
\pgfpathmoveto{\pgfqpoint{2.203122in}{0.688753in}}%
\pgfpathlineto{\pgfqpoint{2.203122in}{0.688753in}}%
\pgfpathlineto{\pgfqpoint{2.203122in}{0.783125in}}%
\pgfpathlineto{\pgfqpoint{2.348439in}{0.783125in}}%
\pgfpathlineto{\pgfqpoint{2.348439in}{0.688753in}}%
\pgfpathmoveto{\pgfqpoint{2.203122in}{0.783125in}}%
\pgfpathlineto{\pgfqpoint{2.203122in}{0.783125in}}%
\pgfpathlineto{\pgfqpoint{2.203122in}{0.877501in}}%
\pgfpathlineto{\pgfqpoint{2.348439in}{0.877501in}}%
\pgfpathlineto{\pgfqpoint{2.348439in}{0.783125in}}%
\pgfpathmoveto{\pgfqpoint{2.203122in}{0.877501in}}%
\pgfpathlineto{\pgfqpoint{2.203122in}{0.877501in}}%
\pgfpathlineto{\pgfqpoint{2.203122in}{0.971874in}}%
\pgfpathlineto{\pgfqpoint{2.348439in}{0.971874in}}%
\pgfpathlineto{\pgfqpoint{2.348439in}{0.877501in}}%
\pgfpathmoveto{\pgfqpoint{2.203122in}{0.971874in}}%
\pgfpathlineto{\pgfqpoint{2.203122in}{0.971874in}}%
\pgfpathlineto{\pgfqpoint{2.203122in}{1.066247in}}%
\pgfpathlineto{\pgfqpoint{2.348439in}{1.066247in}}%
\pgfpathlineto{\pgfqpoint{2.348439in}{0.971874in}}%
\pgfpathmoveto{\pgfqpoint{2.203122in}{1.066247in}}%
\pgfpathlineto{\pgfqpoint{2.203122in}{1.066247in}}%
\pgfpathlineto{\pgfqpoint{2.203122in}{1.160624in}}%
\pgfpathlineto{\pgfqpoint{2.348439in}{1.160624in}}%
\pgfpathlineto{\pgfqpoint{2.348439in}{1.066247in}}%
\pgfpathmoveto{\pgfqpoint{2.203122in}{1.160624in}}%
\pgfpathlineto{\pgfqpoint{2.203122in}{1.160624in}}%
\pgfpathlineto{\pgfqpoint{2.203122in}{1.254999in}}%
\pgfpathlineto{\pgfqpoint{2.348439in}{1.254999in}}%
\pgfpathlineto{\pgfqpoint{2.348439in}{1.160624in}}%
\pgfpathmoveto{\pgfqpoint{2.203122in}{1.254999in}}%
\pgfpathlineto{\pgfqpoint{2.203122in}{1.254999in}}%
\pgfpathlineto{\pgfqpoint{2.203122in}{1.349373in}}%
\pgfpathlineto{\pgfqpoint{2.348439in}{1.349373in}}%
\pgfpathlineto{\pgfqpoint{2.348439in}{1.254999in}}%
\pgfpathmoveto{\pgfqpoint{2.203122in}{1.349373in}}%
\pgfpathlineto{\pgfqpoint{2.203122in}{1.349373in}}%
\pgfpathlineto{\pgfqpoint{2.203122in}{1.443752in}}%
\pgfpathlineto{\pgfqpoint{2.348439in}{1.443752in}}%
\pgfpathlineto{\pgfqpoint{2.348439in}{1.349373in}}%
\pgfpathmoveto{\pgfqpoint{2.203122in}{1.443752in}}%
\pgfpathlineto{\pgfqpoint{2.203122in}{1.443752in}}%
\pgfpathlineto{\pgfqpoint{2.203122in}{1.538128in}}%
\pgfpathlineto{\pgfqpoint{2.348439in}{1.538128in}}%
\pgfpathlineto{\pgfqpoint{2.348439in}{1.443752in}}%
\pgfpathmoveto{\pgfqpoint{2.203122in}{1.538128in}}%
\pgfpathlineto{\pgfqpoint{2.203122in}{1.538128in}}%
\pgfpathlineto{\pgfqpoint{2.203122in}{1.632499in}}%
\pgfpathlineto{\pgfqpoint{2.348439in}{1.632499in}}%
\pgfpathlineto{\pgfqpoint{2.348439in}{1.538128in}}%
\pgfpathmoveto{\pgfqpoint{2.203122in}{1.632499in}}%
\pgfpathlineto{\pgfqpoint{2.203122in}{1.632499in}}%
\pgfpathlineto{\pgfqpoint{2.203122in}{1.726877in}}%
\pgfpathlineto{\pgfqpoint{2.348439in}{1.726877in}}%
\pgfpathlineto{\pgfqpoint{2.348439in}{1.632499in}}%
\pgfpathmoveto{\pgfqpoint{2.203122in}{1.726877in}}%
\pgfpathlineto{\pgfqpoint{2.203122in}{1.726877in}}%
\pgfpathlineto{\pgfqpoint{2.203122in}{1.821249in}}%
\pgfpathlineto{\pgfqpoint{2.348439in}{1.821249in}}%
\pgfpathlineto{\pgfqpoint{2.348439in}{1.726877in}}%
\pgfpathmoveto{\pgfqpoint{2.203122in}{1.821249in}}%
\pgfpathlineto{\pgfqpoint{2.203122in}{1.821249in}}%
\pgfpathlineto{\pgfqpoint{2.203122in}{1.915628in}}%
\pgfpathlineto{\pgfqpoint{2.348439in}{1.915628in}}%
\pgfpathlineto{\pgfqpoint{2.348439in}{1.821249in}}%
\pgfpathmoveto{\pgfqpoint{2.203122in}{1.915628in}}%
\pgfpathlineto{\pgfqpoint{2.203122in}{1.915628in}}%
\pgfpathlineto{\pgfqpoint{2.203122in}{2.009998in}}%
\pgfpathlineto{\pgfqpoint{2.348439in}{2.009998in}}%
\pgfpathlineto{\pgfqpoint{2.348439in}{1.915628in}}%
\pgfpathmoveto{\pgfqpoint{2.203122in}{2.009998in}}%
\pgfpathlineto{\pgfqpoint{2.203122in}{2.009998in}}%
\pgfpathlineto{\pgfqpoint{2.203122in}{2.104374in}}%
\pgfpathlineto{\pgfqpoint{2.348439in}{2.104374in}}%
\pgfpathlineto{\pgfqpoint{2.348439in}{2.009998in}}%
\pgfpathmoveto{\pgfqpoint{2.203122in}{2.104374in}}%
\pgfpathlineto{\pgfqpoint{2.203122in}{2.104374in}}%
\pgfpathlineto{\pgfqpoint{2.203122in}{2.198749in}}%
\pgfpathlineto{\pgfqpoint{2.348439in}{2.198749in}}%
\pgfpathlineto{\pgfqpoint{2.348439in}{2.104374in}}%
\pgfpathmoveto{\pgfqpoint{2.203122in}{2.198749in}}%
\pgfpathlineto{\pgfqpoint{2.203122in}{2.198749in}}%
\pgfpathlineto{\pgfqpoint{2.203122in}{2.293123in}}%
\pgfpathlineto{\pgfqpoint{2.348439in}{2.293123in}}%
\pgfpathlineto{\pgfqpoint{2.348439in}{2.198749in}}%
\pgfpathmoveto{\pgfqpoint{2.203122in}{2.293123in}}%
\pgfpathlineto{\pgfqpoint{2.203122in}{2.293123in}}%
\pgfpathlineto{\pgfqpoint{2.203122in}{2.387498in}}%
\pgfpathlineto{\pgfqpoint{2.348439in}{2.387498in}}%
\pgfpathlineto{\pgfqpoint{2.348439in}{2.293123in}}%
\pgfpathmoveto{\pgfqpoint{2.203122in}{2.387498in}}%
\pgfpathlineto{\pgfqpoint{2.203122in}{2.387498in}}%
\pgfpathlineto{\pgfqpoint{2.203122in}{2.481875in}}%
\pgfpathlineto{\pgfqpoint{2.348439in}{2.481875in}}%
\pgfpathlineto{\pgfqpoint{2.348439in}{2.387498in}}%
\pgfpathmoveto{\pgfqpoint{2.203122in}{2.481875in}}%
\pgfpathlineto{\pgfqpoint{2.203122in}{2.481875in}}%
\pgfpathlineto{\pgfqpoint{2.203122in}{2.576249in}}%
\pgfpathlineto{\pgfqpoint{2.348439in}{2.576249in}}%
\pgfpathlineto{\pgfqpoint{2.348439in}{2.481875in}}%
\pgfpathmoveto{\pgfqpoint{2.348439in}{0.499998in}}%
\pgfpathlineto{\pgfqpoint{2.348439in}{0.499998in}}%
\pgfpathlineto{\pgfqpoint{2.348439in}{0.594373in}}%
\pgfpathlineto{\pgfqpoint{2.493746in}{0.594373in}}%
\pgfpathlineto{\pgfqpoint{2.493746in}{0.499998in}}%
\pgfpathmoveto{\pgfqpoint{2.348439in}{0.594373in}}%
\pgfpathlineto{\pgfqpoint{2.348439in}{0.594373in}}%
\pgfpathlineto{\pgfqpoint{2.348439in}{0.688753in}}%
\pgfpathlineto{\pgfqpoint{2.493746in}{0.688753in}}%
\pgfpathlineto{\pgfqpoint{2.493746in}{0.594373in}}%
\pgfpathmoveto{\pgfqpoint{2.348439in}{0.688753in}}%
\pgfpathlineto{\pgfqpoint{2.348439in}{0.688753in}}%
\pgfpathlineto{\pgfqpoint{2.348439in}{0.783125in}}%
\pgfpathlineto{\pgfqpoint{2.493746in}{0.783125in}}%
\pgfpathlineto{\pgfqpoint{2.493746in}{0.688753in}}%
\pgfpathmoveto{\pgfqpoint{2.348439in}{0.783125in}}%
\pgfpathlineto{\pgfqpoint{2.348439in}{0.783125in}}%
\pgfpathlineto{\pgfqpoint{2.348439in}{0.877501in}}%
\pgfpathlineto{\pgfqpoint{2.493746in}{0.877501in}}%
\pgfpathlineto{\pgfqpoint{2.493746in}{0.783125in}}%
\pgfpathmoveto{\pgfqpoint{2.348439in}{0.877501in}}%
\pgfpathlineto{\pgfqpoint{2.348439in}{0.877501in}}%
\pgfpathlineto{\pgfqpoint{2.348439in}{0.971874in}}%
\pgfpathlineto{\pgfqpoint{2.493746in}{0.971874in}}%
\pgfpathlineto{\pgfqpoint{2.493746in}{0.877501in}}%
\pgfpathmoveto{\pgfqpoint{2.348439in}{0.971874in}}%
\pgfpathlineto{\pgfqpoint{2.348439in}{0.971874in}}%
\pgfpathlineto{\pgfqpoint{2.348439in}{1.066247in}}%
\pgfpathlineto{\pgfqpoint{2.493746in}{1.066247in}}%
\pgfpathlineto{\pgfqpoint{2.493746in}{0.971874in}}%
\pgfpathmoveto{\pgfqpoint{2.348439in}{1.066247in}}%
\pgfpathlineto{\pgfqpoint{2.348439in}{1.066247in}}%
\pgfpathlineto{\pgfqpoint{2.348439in}{1.160624in}}%
\pgfpathlineto{\pgfqpoint{2.493746in}{1.160624in}}%
\pgfpathlineto{\pgfqpoint{2.493746in}{1.066247in}}%
\pgfpathmoveto{\pgfqpoint{2.348439in}{1.160624in}}%
\pgfpathlineto{\pgfqpoint{2.348439in}{1.160624in}}%
\pgfpathlineto{\pgfqpoint{2.348439in}{1.254999in}}%
\pgfpathlineto{\pgfqpoint{2.493746in}{1.254999in}}%
\pgfpathlineto{\pgfqpoint{2.493746in}{1.160624in}}%
\pgfpathmoveto{\pgfqpoint{2.348439in}{1.254999in}}%
\pgfpathlineto{\pgfqpoint{2.348439in}{1.254999in}}%
\pgfpathlineto{\pgfqpoint{2.348439in}{1.349373in}}%
\pgfpathlineto{\pgfqpoint{2.493746in}{1.349373in}}%
\pgfpathlineto{\pgfqpoint{2.493746in}{1.254999in}}%
\pgfpathmoveto{\pgfqpoint{2.348439in}{1.349373in}}%
\pgfpathlineto{\pgfqpoint{2.348439in}{1.349373in}}%
\pgfpathlineto{\pgfqpoint{2.348439in}{1.443752in}}%
\pgfpathlineto{\pgfqpoint{2.493746in}{1.443752in}}%
\pgfpathlineto{\pgfqpoint{2.493746in}{1.349373in}}%
\pgfpathmoveto{\pgfqpoint{2.348439in}{1.443752in}}%
\pgfpathlineto{\pgfqpoint{2.348439in}{1.443752in}}%
\pgfpathlineto{\pgfqpoint{2.348439in}{1.538128in}}%
\pgfpathlineto{\pgfqpoint{2.493746in}{1.538128in}}%
\pgfpathlineto{\pgfqpoint{2.493746in}{1.443752in}}%
\pgfpathmoveto{\pgfqpoint{2.348439in}{1.538128in}}%
\pgfpathlineto{\pgfqpoint{2.348439in}{1.538128in}}%
\pgfpathlineto{\pgfqpoint{2.348439in}{1.632499in}}%
\pgfpathlineto{\pgfqpoint{2.493746in}{1.632499in}}%
\pgfpathlineto{\pgfqpoint{2.493746in}{1.538128in}}%
\pgfpathmoveto{\pgfqpoint{2.348439in}{1.632499in}}%
\pgfpathlineto{\pgfqpoint{2.348439in}{1.632499in}}%
\pgfpathlineto{\pgfqpoint{2.348439in}{1.726877in}}%
\pgfpathlineto{\pgfqpoint{2.493746in}{1.726877in}}%
\pgfpathlineto{\pgfqpoint{2.493746in}{1.632499in}}%
\pgfpathmoveto{\pgfqpoint{2.348439in}{1.726877in}}%
\pgfpathlineto{\pgfqpoint{2.348439in}{1.726877in}}%
\pgfpathlineto{\pgfqpoint{2.348439in}{1.821249in}}%
\pgfpathlineto{\pgfqpoint{2.493746in}{1.821249in}}%
\pgfpathlineto{\pgfqpoint{2.493746in}{1.726877in}}%
\pgfpathmoveto{\pgfqpoint{2.348439in}{1.821249in}}%
\pgfpathlineto{\pgfqpoint{2.348439in}{1.821249in}}%
\pgfpathlineto{\pgfqpoint{2.348439in}{1.915628in}}%
\pgfpathlineto{\pgfqpoint{2.493746in}{1.915628in}}%
\pgfpathlineto{\pgfqpoint{2.493746in}{1.821249in}}%
\pgfpathmoveto{\pgfqpoint{2.348439in}{1.915628in}}%
\pgfpathlineto{\pgfqpoint{2.348439in}{1.915628in}}%
\pgfpathlineto{\pgfqpoint{2.348439in}{2.009998in}}%
\pgfpathlineto{\pgfqpoint{2.493746in}{2.009998in}}%
\pgfpathlineto{\pgfqpoint{2.493746in}{1.915628in}}%
\pgfpathmoveto{\pgfqpoint{2.348439in}{2.009998in}}%
\pgfpathlineto{\pgfqpoint{2.348439in}{2.009998in}}%
\pgfpathlineto{\pgfqpoint{2.348439in}{2.104374in}}%
\pgfpathlineto{\pgfqpoint{2.493746in}{2.104374in}}%
\pgfpathlineto{\pgfqpoint{2.493746in}{2.009998in}}%
\pgfpathmoveto{\pgfqpoint{2.348439in}{2.104374in}}%
\pgfpathlineto{\pgfqpoint{2.348439in}{2.104374in}}%
\pgfpathlineto{\pgfqpoint{2.348439in}{2.198749in}}%
\pgfpathlineto{\pgfqpoint{2.493746in}{2.198749in}}%
\pgfpathlineto{\pgfqpoint{2.493746in}{2.104374in}}%
\pgfpathmoveto{\pgfqpoint{2.348439in}{2.198749in}}%
\pgfpathlineto{\pgfqpoint{2.348439in}{2.198749in}}%
\pgfpathlineto{\pgfqpoint{2.348439in}{2.293123in}}%
\pgfpathlineto{\pgfqpoint{2.493746in}{2.293123in}}%
\pgfpathlineto{\pgfqpoint{2.493746in}{2.198749in}}%
\pgfpathmoveto{\pgfqpoint{2.348439in}{2.293123in}}%
\pgfpathlineto{\pgfqpoint{2.348439in}{2.293123in}}%
\pgfpathlineto{\pgfqpoint{2.348439in}{2.387498in}}%
\pgfpathlineto{\pgfqpoint{2.493746in}{2.387498in}}%
\pgfpathlineto{\pgfqpoint{2.493746in}{2.293123in}}%
\pgfpathmoveto{\pgfqpoint{2.348439in}{2.387498in}}%
\pgfpathlineto{\pgfqpoint{2.348439in}{2.387498in}}%
\pgfpathlineto{\pgfqpoint{2.348439in}{2.481875in}}%
\pgfpathlineto{\pgfqpoint{2.493746in}{2.481875in}}%
\pgfpathlineto{\pgfqpoint{2.493746in}{2.387498in}}%
\pgfpathmoveto{\pgfqpoint{2.493746in}{0.499998in}}%
\pgfpathlineto{\pgfqpoint{2.493746in}{0.499998in}}%
\pgfpathlineto{\pgfqpoint{2.493746in}{0.594373in}}%
\pgfpathlineto{\pgfqpoint{2.639058in}{0.594373in}}%
\pgfpathlineto{\pgfqpoint{2.639058in}{0.499998in}}%
\pgfpathmoveto{\pgfqpoint{2.493746in}{0.594373in}}%
\pgfpathlineto{\pgfqpoint{2.493746in}{0.594373in}}%
\pgfpathlineto{\pgfqpoint{2.493746in}{0.688753in}}%
\pgfpathlineto{\pgfqpoint{2.639058in}{0.688753in}}%
\pgfpathlineto{\pgfqpoint{2.639058in}{0.594373in}}%
\pgfpathmoveto{\pgfqpoint{2.493746in}{0.688753in}}%
\pgfpathlineto{\pgfqpoint{2.493746in}{0.688753in}}%
\pgfpathlineto{\pgfqpoint{2.493746in}{0.783125in}}%
\pgfpathlineto{\pgfqpoint{2.639058in}{0.783125in}}%
\pgfpathlineto{\pgfqpoint{2.639058in}{0.688753in}}%
\pgfpathmoveto{\pgfqpoint{2.493746in}{0.783125in}}%
\pgfpathlineto{\pgfqpoint{2.493746in}{0.783125in}}%
\pgfpathlineto{\pgfqpoint{2.493746in}{0.877501in}}%
\pgfpathlineto{\pgfqpoint{2.639058in}{0.877501in}}%
\pgfpathlineto{\pgfqpoint{2.639058in}{0.783125in}}%
\pgfpathmoveto{\pgfqpoint{2.493746in}{0.877501in}}%
\pgfpathlineto{\pgfqpoint{2.493746in}{0.877501in}}%
\pgfpathlineto{\pgfqpoint{2.493746in}{0.971874in}}%
\pgfpathlineto{\pgfqpoint{2.639058in}{0.971874in}}%
\pgfpathlineto{\pgfqpoint{2.639058in}{0.877501in}}%
\pgfpathmoveto{\pgfqpoint{2.493746in}{0.971874in}}%
\pgfpathlineto{\pgfqpoint{2.493746in}{0.971874in}}%
\pgfpathlineto{\pgfqpoint{2.493746in}{1.066247in}}%
\pgfpathlineto{\pgfqpoint{2.639058in}{1.066247in}}%
\pgfpathlineto{\pgfqpoint{2.639058in}{0.971874in}}%
\pgfpathmoveto{\pgfqpoint{2.493746in}{1.066247in}}%
\pgfpathlineto{\pgfqpoint{2.493746in}{1.066247in}}%
\pgfpathlineto{\pgfqpoint{2.493746in}{1.160624in}}%
\pgfpathlineto{\pgfqpoint{2.639058in}{1.160624in}}%
\pgfpathlineto{\pgfqpoint{2.639058in}{1.066247in}}%
\pgfpathmoveto{\pgfqpoint{2.493746in}{1.160624in}}%
\pgfpathlineto{\pgfqpoint{2.493746in}{1.160624in}}%
\pgfpathlineto{\pgfqpoint{2.493746in}{1.254999in}}%
\pgfpathlineto{\pgfqpoint{2.639058in}{1.254999in}}%
\pgfpathlineto{\pgfqpoint{2.639058in}{1.160624in}}%
\pgfpathmoveto{\pgfqpoint{2.493746in}{1.254999in}}%
\pgfpathlineto{\pgfqpoint{2.493746in}{1.254999in}}%
\pgfpathlineto{\pgfqpoint{2.493746in}{1.349373in}}%
\pgfpathlineto{\pgfqpoint{2.639058in}{1.349373in}}%
\pgfpathlineto{\pgfqpoint{2.639058in}{1.254999in}}%
\pgfpathmoveto{\pgfqpoint{2.493746in}{1.349373in}}%
\pgfpathlineto{\pgfqpoint{2.493746in}{1.349373in}}%
\pgfpathlineto{\pgfqpoint{2.493746in}{1.443752in}}%
\pgfpathlineto{\pgfqpoint{2.639058in}{1.443752in}}%
\pgfpathlineto{\pgfqpoint{2.639058in}{1.349373in}}%
\pgfpathmoveto{\pgfqpoint{2.493746in}{1.443752in}}%
\pgfpathlineto{\pgfqpoint{2.493746in}{1.443752in}}%
\pgfpathlineto{\pgfqpoint{2.493746in}{1.538128in}}%
\pgfpathlineto{\pgfqpoint{2.639058in}{1.538128in}}%
\pgfpathlineto{\pgfqpoint{2.639058in}{1.443752in}}%
\pgfpathmoveto{\pgfqpoint{2.493746in}{1.538128in}}%
\pgfpathlineto{\pgfqpoint{2.493746in}{1.538128in}}%
\pgfpathlineto{\pgfqpoint{2.493746in}{1.632499in}}%
\pgfpathlineto{\pgfqpoint{2.639058in}{1.632499in}}%
\pgfpathlineto{\pgfqpoint{2.639058in}{1.538128in}}%
\pgfpathmoveto{\pgfqpoint{2.493746in}{1.632499in}}%
\pgfpathlineto{\pgfqpoint{2.493746in}{1.632499in}}%
\pgfpathlineto{\pgfqpoint{2.493746in}{1.726877in}}%
\pgfpathlineto{\pgfqpoint{2.639058in}{1.726877in}}%
\pgfpathlineto{\pgfqpoint{2.639058in}{1.632499in}}%
\pgfpathmoveto{\pgfqpoint{2.493746in}{1.726877in}}%
\pgfpathlineto{\pgfqpoint{2.493746in}{1.726877in}}%
\pgfpathlineto{\pgfqpoint{2.493746in}{1.821249in}}%
\pgfpathlineto{\pgfqpoint{2.639058in}{1.821249in}}%
\pgfpathlineto{\pgfqpoint{2.639058in}{1.726877in}}%
\pgfpathmoveto{\pgfqpoint{2.493746in}{1.821249in}}%
\pgfpathlineto{\pgfqpoint{2.493746in}{1.821249in}}%
\pgfpathlineto{\pgfqpoint{2.493746in}{1.915628in}}%
\pgfpathlineto{\pgfqpoint{2.639058in}{1.915628in}}%
\pgfpathlineto{\pgfqpoint{2.639058in}{1.821249in}}%
\pgfpathmoveto{\pgfqpoint{2.493746in}{1.915628in}}%
\pgfpathlineto{\pgfqpoint{2.493746in}{1.915628in}}%
\pgfpathlineto{\pgfqpoint{2.493746in}{2.009998in}}%
\pgfpathlineto{\pgfqpoint{2.639058in}{2.009998in}}%
\pgfpathlineto{\pgfqpoint{2.639058in}{1.915628in}}%
\pgfpathmoveto{\pgfqpoint{2.493746in}{2.009998in}}%
\pgfpathlineto{\pgfqpoint{2.493746in}{2.009998in}}%
\pgfpathlineto{\pgfqpoint{2.493746in}{2.104374in}}%
\pgfpathlineto{\pgfqpoint{2.639058in}{2.104374in}}%
\pgfpathlineto{\pgfqpoint{2.639058in}{2.009998in}}%
\pgfpathmoveto{\pgfqpoint{2.493746in}{2.104374in}}%
\pgfpathlineto{\pgfqpoint{2.493746in}{2.104374in}}%
\pgfpathlineto{\pgfqpoint{2.493746in}{2.198749in}}%
\pgfpathlineto{\pgfqpoint{2.639058in}{2.198749in}}%
\pgfpathlineto{\pgfqpoint{2.639058in}{2.104374in}}%
\pgfpathmoveto{\pgfqpoint{2.493746in}{2.198749in}}%
\pgfpathlineto{\pgfqpoint{2.493746in}{2.198749in}}%
\pgfpathlineto{\pgfqpoint{2.493746in}{2.293123in}}%
\pgfpathlineto{\pgfqpoint{2.639058in}{2.293123in}}%
\pgfpathlineto{\pgfqpoint{2.639058in}{2.198749in}}%
\pgfpathmoveto{\pgfqpoint{2.493746in}{2.293123in}}%
\pgfpathlineto{\pgfqpoint{2.493746in}{2.293123in}}%
\pgfpathlineto{\pgfqpoint{2.493746in}{2.387498in}}%
\pgfpathlineto{\pgfqpoint{2.639058in}{2.387498in}}%
\pgfpathlineto{\pgfqpoint{2.639058in}{2.293123in}}%
\pgfpathmoveto{\pgfqpoint{2.639058in}{0.499998in}}%
\pgfpathlineto{\pgfqpoint{2.639058in}{0.499998in}}%
\pgfpathlineto{\pgfqpoint{2.639058in}{0.594373in}}%
\pgfpathlineto{\pgfqpoint{2.784372in}{0.594373in}}%
\pgfpathlineto{\pgfqpoint{2.784372in}{0.499998in}}%
\pgfpathmoveto{\pgfqpoint{2.639058in}{0.594373in}}%
\pgfpathlineto{\pgfqpoint{2.639058in}{0.594373in}}%
\pgfpathlineto{\pgfqpoint{2.639058in}{0.688753in}}%
\pgfpathlineto{\pgfqpoint{2.784372in}{0.688753in}}%
\pgfpathlineto{\pgfqpoint{2.784372in}{0.594373in}}%
\pgfpathmoveto{\pgfqpoint{2.639058in}{0.688753in}}%
\pgfpathlineto{\pgfqpoint{2.639058in}{0.688753in}}%
\pgfpathlineto{\pgfqpoint{2.639058in}{0.783125in}}%
\pgfpathlineto{\pgfqpoint{2.784372in}{0.783125in}}%
\pgfpathlineto{\pgfqpoint{2.784372in}{0.688753in}}%
\pgfpathmoveto{\pgfqpoint{2.639058in}{0.783125in}}%
\pgfpathlineto{\pgfqpoint{2.639058in}{0.783125in}}%
\pgfpathlineto{\pgfqpoint{2.639058in}{0.877501in}}%
\pgfpathlineto{\pgfqpoint{2.784372in}{0.877501in}}%
\pgfpathlineto{\pgfqpoint{2.784372in}{0.783125in}}%
\pgfpathmoveto{\pgfqpoint{2.639058in}{0.877501in}}%
\pgfpathlineto{\pgfqpoint{2.639058in}{0.877501in}}%
\pgfpathlineto{\pgfqpoint{2.639058in}{0.971874in}}%
\pgfpathlineto{\pgfqpoint{2.784372in}{0.971874in}}%
\pgfpathlineto{\pgfqpoint{2.784372in}{0.877501in}}%
\pgfpathmoveto{\pgfqpoint{2.639058in}{0.971874in}}%
\pgfpathlineto{\pgfqpoint{2.639058in}{0.971874in}}%
\pgfpathlineto{\pgfqpoint{2.639058in}{1.066247in}}%
\pgfpathlineto{\pgfqpoint{2.784372in}{1.066247in}}%
\pgfpathlineto{\pgfqpoint{2.784372in}{0.971874in}}%
\pgfpathmoveto{\pgfqpoint{2.639058in}{1.066247in}}%
\pgfpathlineto{\pgfqpoint{2.639058in}{1.066247in}}%
\pgfpathlineto{\pgfqpoint{2.639058in}{1.160624in}}%
\pgfpathlineto{\pgfqpoint{2.784372in}{1.160624in}}%
\pgfpathlineto{\pgfqpoint{2.784372in}{1.066247in}}%
\pgfpathmoveto{\pgfqpoint{2.639058in}{1.160624in}}%
\pgfpathlineto{\pgfqpoint{2.639058in}{1.160624in}}%
\pgfpathlineto{\pgfqpoint{2.639058in}{1.254999in}}%
\pgfpathlineto{\pgfqpoint{2.784372in}{1.254999in}}%
\pgfpathlineto{\pgfqpoint{2.784372in}{1.160624in}}%
\pgfpathmoveto{\pgfqpoint{2.639058in}{1.254999in}}%
\pgfpathlineto{\pgfqpoint{2.639058in}{1.254999in}}%
\pgfpathlineto{\pgfqpoint{2.639058in}{1.349373in}}%
\pgfpathlineto{\pgfqpoint{2.784372in}{1.349373in}}%
\pgfpathlineto{\pgfqpoint{2.784372in}{1.254999in}}%
\pgfpathmoveto{\pgfqpoint{2.639058in}{1.349373in}}%
\pgfpathlineto{\pgfqpoint{2.639058in}{1.349373in}}%
\pgfpathlineto{\pgfqpoint{2.639058in}{1.443752in}}%
\pgfpathlineto{\pgfqpoint{2.784372in}{1.443752in}}%
\pgfpathlineto{\pgfqpoint{2.784372in}{1.349373in}}%
\pgfpathmoveto{\pgfqpoint{2.639058in}{1.443752in}}%
\pgfpathlineto{\pgfqpoint{2.639058in}{1.443752in}}%
\pgfpathlineto{\pgfqpoint{2.639058in}{1.538128in}}%
\pgfpathlineto{\pgfqpoint{2.784372in}{1.538128in}}%
\pgfpathlineto{\pgfqpoint{2.784372in}{1.443752in}}%
\pgfpathmoveto{\pgfqpoint{2.639058in}{1.538128in}}%
\pgfpathlineto{\pgfqpoint{2.639058in}{1.538128in}}%
\pgfpathlineto{\pgfqpoint{2.639058in}{1.632499in}}%
\pgfpathlineto{\pgfqpoint{2.784372in}{1.632499in}}%
\pgfpathlineto{\pgfqpoint{2.784372in}{1.538128in}}%
\pgfpathmoveto{\pgfqpoint{2.639058in}{1.632499in}}%
\pgfpathlineto{\pgfqpoint{2.639058in}{1.632499in}}%
\pgfpathlineto{\pgfqpoint{2.639058in}{1.726877in}}%
\pgfpathlineto{\pgfqpoint{2.784372in}{1.726877in}}%
\pgfpathlineto{\pgfqpoint{2.784372in}{1.632499in}}%
\pgfpathmoveto{\pgfqpoint{2.639058in}{1.726877in}}%
\pgfpathlineto{\pgfqpoint{2.639058in}{1.726877in}}%
\pgfpathlineto{\pgfqpoint{2.639058in}{1.821249in}}%
\pgfpathlineto{\pgfqpoint{2.784372in}{1.821249in}}%
\pgfpathlineto{\pgfqpoint{2.784372in}{1.726877in}}%
\pgfpathmoveto{\pgfqpoint{2.639058in}{1.821249in}}%
\pgfpathlineto{\pgfqpoint{2.639058in}{1.821249in}}%
\pgfpathlineto{\pgfqpoint{2.639058in}{1.915628in}}%
\pgfpathlineto{\pgfqpoint{2.784372in}{1.915628in}}%
\pgfpathlineto{\pgfqpoint{2.784372in}{1.821249in}}%
\pgfpathmoveto{\pgfqpoint{2.639058in}{1.915628in}}%
\pgfpathlineto{\pgfqpoint{2.639058in}{1.915628in}}%
\pgfpathlineto{\pgfqpoint{2.639058in}{2.009998in}}%
\pgfpathlineto{\pgfqpoint{2.784372in}{2.009998in}}%
\pgfpathlineto{\pgfqpoint{2.784372in}{1.915628in}}%
\pgfpathmoveto{\pgfqpoint{2.639058in}{2.009998in}}%
\pgfpathlineto{\pgfqpoint{2.639058in}{2.009998in}}%
\pgfpathlineto{\pgfqpoint{2.639058in}{2.104374in}}%
\pgfpathlineto{\pgfqpoint{2.784372in}{2.104374in}}%
\pgfpathlineto{\pgfqpoint{2.784372in}{2.009998in}}%
\pgfpathmoveto{\pgfqpoint{2.639058in}{2.104374in}}%
\pgfpathlineto{\pgfqpoint{2.639058in}{2.104374in}}%
\pgfpathlineto{\pgfqpoint{2.639058in}{2.198749in}}%
\pgfpathlineto{\pgfqpoint{2.784372in}{2.198749in}}%
\pgfpathlineto{\pgfqpoint{2.784372in}{2.104374in}}%
\pgfpathmoveto{\pgfqpoint{2.784372in}{0.499998in}}%
\pgfpathlineto{\pgfqpoint{2.784372in}{0.499998in}}%
\pgfpathlineto{\pgfqpoint{2.784372in}{0.594373in}}%
\pgfpathlineto{\pgfqpoint{2.929689in}{0.594373in}}%
\pgfpathlineto{\pgfqpoint{2.929689in}{0.499998in}}%
\pgfpathmoveto{\pgfqpoint{2.784372in}{0.594373in}}%
\pgfpathlineto{\pgfqpoint{2.784372in}{0.594373in}}%
\pgfpathlineto{\pgfqpoint{2.784372in}{0.688753in}}%
\pgfpathlineto{\pgfqpoint{2.929689in}{0.688753in}}%
\pgfpathlineto{\pgfqpoint{2.929689in}{0.594373in}}%
\pgfpathmoveto{\pgfqpoint{2.784372in}{0.688753in}}%
\pgfpathlineto{\pgfqpoint{2.784372in}{0.688753in}}%
\pgfpathlineto{\pgfqpoint{2.784372in}{0.783125in}}%
\pgfpathlineto{\pgfqpoint{2.929689in}{0.783125in}}%
\pgfpathlineto{\pgfqpoint{2.929689in}{0.688753in}}%
\pgfpathmoveto{\pgfqpoint{2.784372in}{0.783125in}}%
\pgfpathlineto{\pgfqpoint{2.784372in}{0.783125in}}%
\pgfpathlineto{\pgfqpoint{2.784372in}{0.877501in}}%
\pgfpathlineto{\pgfqpoint{2.929689in}{0.877501in}}%
\pgfpathlineto{\pgfqpoint{2.929689in}{0.783125in}}%
\pgfpathmoveto{\pgfqpoint{2.784372in}{0.877501in}}%
\pgfpathlineto{\pgfqpoint{2.784372in}{0.877501in}}%
\pgfpathlineto{\pgfqpoint{2.784372in}{0.971874in}}%
\pgfpathlineto{\pgfqpoint{2.929689in}{0.971874in}}%
\pgfpathlineto{\pgfqpoint{2.929689in}{0.877501in}}%
\pgfpathmoveto{\pgfqpoint{2.784372in}{0.971874in}}%
\pgfpathlineto{\pgfqpoint{2.784372in}{0.971874in}}%
\pgfpathlineto{\pgfqpoint{2.784372in}{1.066247in}}%
\pgfpathlineto{\pgfqpoint{2.929689in}{1.066247in}}%
\pgfpathlineto{\pgfqpoint{2.929689in}{0.971874in}}%
\pgfpathmoveto{\pgfqpoint{2.784372in}{1.066247in}}%
\pgfpathlineto{\pgfqpoint{2.784372in}{1.066247in}}%
\pgfpathlineto{\pgfqpoint{2.784372in}{1.160624in}}%
\pgfpathlineto{\pgfqpoint{2.929689in}{1.160624in}}%
\pgfpathlineto{\pgfqpoint{2.929689in}{1.066247in}}%
\pgfpathmoveto{\pgfqpoint{2.784372in}{1.160624in}}%
\pgfpathlineto{\pgfqpoint{2.784372in}{1.160624in}}%
\pgfpathlineto{\pgfqpoint{2.784372in}{1.254999in}}%
\pgfpathlineto{\pgfqpoint{2.929689in}{1.254999in}}%
\pgfpathlineto{\pgfqpoint{2.929689in}{1.160624in}}%
\pgfpathmoveto{\pgfqpoint{2.784372in}{1.254999in}}%
\pgfpathlineto{\pgfqpoint{2.784372in}{1.254999in}}%
\pgfpathlineto{\pgfqpoint{2.784372in}{1.349373in}}%
\pgfpathlineto{\pgfqpoint{2.929689in}{1.349373in}}%
\pgfpathlineto{\pgfqpoint{2.929689in}{1.254999in}}%
\pgfpathmoveto{\pgfqpoint{2.784372in}{1.349373in}}%
\pgfpathlineto{\pgfqpoint{2.784372in}{1.349373in}}%
\pgfpathlineto{\pgfqpoint{2.784372in}{1.443752in}}%
\pgfpathlineto{\pgfqpoint{2.929689in}{1.443752in}}%
\pgfpathlineto{\pgfqpoint{2.929689in}{1.349373in}}%
\pgfpathmoveto{\pgfqpoint{2.784372in}{1.443752in}}%
\pgfpathlineto{\pgfqpoint{2.784372in}{1.443752in}}%
\pgfpathlineto{\pgfqpoint{2.784372in}{1.538128in}}%
\pgfpathlineto{\pgfqpoint{2.929689in}{1.538128in}}%
\pgfpathlineto{\pgfqpoint{2.929689in}{1.443752in}}%
\pgfpathmoveto{\pgfqpoint{2.784372in}{1.538128in}}%
\pgfpathlineto{\pgfqpoint{2.784372in}{1.538128in}}%
\pgfpathlineto{\pgfqpoint{2.784372in}{1.632499in}}%
\pgfpathlineto{\pgfqpoint{2.929689in}{1.632499in}}%
\pgfpathlineto{\pgfqpoint{2.929689in}{1.538128in}}%
\pgfpathmoveto{\pgfqpoint{2.784372in}{1.632499in}}%
\pgfpathlineto{\pgfqpoint{2.784372in}{1.632499in}}%
\pgfpathlineto{\pgfqpoint{2.784372in}{1.726877in}}%
\pgfpathlineto{\pgfqpoint{2.929689in}{1.726877in}}%
\pgfpathlineto{\pgfqpoint{2.929689in}{1.632499in}}%
\pgfpathmoveto{\pgfqpoint{2.784372in}{1.726877in}}%
\pgfpathlineto{\pgfqpoint{2.784372in}{1.726877in}}%
\pgfpathlineto{\pgfqpoint{2.784372in}{1.821249in}}%
\pgfpathlineto{\pgfqpoint{2.929689in}{1.821249in}}%
\pgfpathlineto{\pgfqpoint{2.929689in}{1.726877in}}%
\pgfpathmoveto{\pgfqpoint{2.784372in}{1.821249in}}%
\pgfpathlineto{\pgfqpoint{2.784372in}{1.821249in}}%
\pgfpathlineto{\pgfqpoint{2.784372in}{1.915628in}}%
\pgfpathlineto{\pgfqpoint{2.929689in}{1.915628in}}%
\pgfpathlineto{\pgfqpoint{2.929689in}{1.821249in}}%
\pgfpathmoveto{\pgfqpoint{2.784372in}{1.915628in}}%
\pgfpathlineto{\pgfqpoint{2.784372in}{1.915628in}}%
\pgfpathlineto{\pgfqpoint{2.784372in}{2.009998in}}%
\pgfpathlineto{\pgfqpoint{2.929689in}{2.009998in}}%
\pgfpathlineto{\pgfqpoint{2.929689in}{1.915628in}}%
\pgfpathmoveto{\pgfqpoint{2.784372in}{2.009998in}}%
\pgfpathlineto{\pgfqpoint{2.784372in}{2.009998in}}%
\pgfpathlineto{\pgfqpoint{2.784372in}{2.104374in}}%
\pgfpathlineto{\pgfqpoint{2.929689in}{2.104374in}}%
\pgfpathlineto{\pgfqpoint{2.929689in}{2.009998in}}%
\pgfpathmoveto{\pgfqpoint{2.929689in}{0.499998in}}%
\pgfpathlineto{\pgfqpoint{2.929689in}{0.499998in}}%
\pgfpathlineto{\pgfqpoint{2.929689in}{0.594373in}}%
\pgfpathlineto{\pgfqpoint{3.074997in}{0.594373in}}%
\pgfpathlineto{\pgfqpoint{3.074997in}{0.499998in}}%
\pgfpathmoveto{\pgfqpoint{2.929689in}{0.594373in}}%
\pgfpathlineto{\pgfqpoint{2.929689in}{0.594373in}}%
\pgfpathlineto{\pgfqpoint{2.929689in}{0.688753in}}%
\pgfpathlineto{\pgfqpoint{3.074997in}{0.688753in}}%
\pgfpathlineto{\pgfqpoint{3.074997in}{0.594373in}}%
\pgfpathmoveto{\pgfqpoint{2.929689in}{0.688753in}}%
\pgfpathlineto{\pgfqpoint{2.929689in}{0.688753in}}%
\pgfpathlineto{\pgfqpoint{2.929689in}{0.783125in}}%
\pgfpathlineto{\pgfqpoint{3.074997in}{0.783125in}}%
\pgfpathlineto{\pgfqpoint{3.074997in}{0.688753in}}%
\pgfpathmoveto{\pgfqpoint{2.929689in}{0.783125in}}%
\pgfpathlineto{\pgfqpoint{2.929689in}{0.783125in}}%
\pgfpathlineto{\pgfqpoint{2.929689in}{0.877501in}}%
\pgfpathlineto{\pgfqpoint{3.074997in}{0.877501in}}%
\pgfpathlineto{\pgfqpoint{3.074997in}{0.783125in}}%
\pgfpathmoveto{\pgfqpoint{2.929689in}{0.877501in}}%
\pgfpathlineto{\pgfqpoint{2.929689in}{0.877501in}}%
\pgfpathlineto{\pgfqpoint{2.929689in}{0.971874in}}%
\pgfpathlineto{\pgfqpoint{3.074997in}{0.971874in}}%
\pgfpathlineto{\pgfqpoint{3.074997in}{0.877501in}}%
\pgfpathmoveto{\pgfqpoint{2.929689in}{0.971874in}}%
\pgfpathlineto{\pgfqpoint{2.929689in}{0.971874in}}%
\pgfpathlineto{\pgfqpoint{2.929689in}{1.066247in}}%
\pgfpathlineto{\pgfqpoint{3.074997in}{1.066247in}}%
\pgfpathlineto{\pgfqpoint{3.074997in}{0.971874in}}%
\pgfpathmoveto{\pgfqpoint{2.929689in}{1.066247in}}%
\pgfpathlineto{\pgfqpoint{2.929689in}{1.066247in}}%
\pgfpathlineto{\pgfqpoint{2.929689in}{1.160624in}}%
\pgfpathlineto{\pgfqpoint{3.074997in}{1.160624in}}%
\pgfpathlineto{\pgfqpoint{3.074997in}{1.066247in}}%
\pgfpathmoveto{\pgfqpoint{2.929689in}{1.160624in}}%
\pgfpathlineto{\pgfqpoint{2.929689in}{1.160624in}}%
\pgfpathlineto{\pgfqpoint{2.929689in}{1.254999in}}%
\pgfpathlineto{\pgfqpoint{3.074997in}{1.254999in}}%
\pgfpathlineto{\pgfqpoint{3.074997in}{1.160624in}}%
\pgfpathmoveto{\pgfqpoint{2.929689in}{1.254999in}}%
\pgfpathlineto{\pgfqpoint{2.929689in}{1.254999in}}%
\pgfpathlineto{\pgfqpoint{2.929689in}{1.349373in}}%
\pgfpathlineto{\pgfqpoint{3.074997in}{1.349373in}}%
\pgfpathlineto{\pgfqpoint{3.074997in}{1.254999in}}%
\pgfpathmoveto{\pgfqpoint{2.929689in}{1.349373in}}%
\pgfpathlineto{\pgfqpoint{2.929689in}{1.349373in}}%
\pgfpathlineto{\pgfqpoint{2.929689in}{1.443752in}}%
\pgfpathlineto{\pgfqpoint{3.074997in}{1.443752in}}%
\pgfpathlineto{\pgfqpoint{3.074997in}{1.349373in}}%
\pgfpathmoveto{\pgfqpoint{2.929689in}{1.443752in}}%
\pgfpathlineto{\pgfqpoint{2.929689in}{1.443752in}}%
\pgfpathlineto{\pgfqpoint{2.929689in}{1.538128in}}%
\pgfpathlineto{\pgfqpoint{3.074997in}{1.538128in}}%
\pgfpathlineto{\pgfqpoint{3.074997in}{1.443752in}}%
\pgfpathmoveto{\pgfqpoint{2.929689in}{1.538128in}}%
\pgfpathlineto{\pgfqpoint{2.929689in}{1.538128in}}%
\pgfpathlineto{\pgfqpoint{2.929689in}{1.632499in}}%
\pgfpathlineto{\pgfqpoint{3.074997in}{1.632499in}}%
\pgfpathlineto{\pgfqpoint{3.074997in}{1.538128in}}%
\pgfpathmoveto{\pgfqpoint{2.929689in}{1.632499in}}%
\pgfpathlineto{\pgfqpoint{2.929689in}{1.632499in}}%
\pgfpathlineto{\pgfqpoint{2.929689in}{1.726877in}}%
\pgfpathlineto{\pgfqpoint{3.074997in}{1.726877in}}%
\pgfpathlineto{\pgfqpoint{3.074997in}{1.632499in}}%
\pgfpathmoveto{\pgfqpoint{2.929689in}{1.726877in}}%
\pgfpathlineto{\pgfqpoint{2.929689in}{1.726877in}}%
\pgfpathlineto{\pgfqpoint{2.929689in}{1.821249in}}%
\pgfpathlineto{\pgfqpoint{3.074997in}{1.821249in}}%
\pgfpathlineto{\pgfqpoint{3.074997in}{1.726877in}}%
\pgfpathmoveto{\pgfqpoint{2.929689in}{1.821249in}}%
\pgfpathlineto{\pgfqpoint{2.929689in}{1.821249in}}%
\pgfpathlineto{\pgfqpoint{2.929689in}{1.915628in}}%
\pgfpathlineto{\pgfqpoint{3.074997in}{1.915628in}}%
\pgfpathlineto{\pgfqpoint{3.074997in}{1.821249in}}%
\pgfpathmoveto{\pgfqpoint{2.929689in}{1.915628in}}%
\pgfpathlineto{\pgfqpoint{2.929689in}{1.915628in}}%
\pgfpathlineto{\pgfqpoint{2.929689in}{2.009998in}}%
\pgfpathlineto{\pgfqpoint{3.074997in}{2.009998in}}%
\pgfpathlineto{\pgfqpoint{3.074997in}{1.915628in}}%
\pgfpathmoveto{\pgfqpoint{3.074997in}{0.499998in}}%
\pgfpathlineto{\pgfqpoint{3.074997in}{0.499998in}}%
\pgfpathlineto{\pgfqpoint{3.074997in}{0.594373in}}%
\pgfpathlineto{\pgfqpoint{3.220311in}{0.594373in}}%
\pgfpathlineto{\pgfqpoint{3.220311in}{0.499998in}}%
\pgfpathmoveto{\pgfqpoint{3.074997in}{0.594373in}}%
\pgfpathlineto{\pgfqpoint{3.074997in}{0.594373in}}%
\pgfpathlineto{\pgfqpoint{3.074997in}{0.688753in}}%
\pgfpathlineto{\pgfqpoint{3.220311in}{0.688753in}}%
\pgfpathlineto{\pgfqpoint{3.220311in}{0.594373in}}%
\pgfpathmoveto{\pgfqpoint{3.074997in}{0.688753in}}%
\pgfpathlineto{\pgfqpoint{3.074997in}{0.688753in}}%
\pgfpathlineto{\pgfqpoint{3.074997in}{0.783125in}}%
\pgfpathlineto{\pgfqpoint{3.220311in}{0.783125in}}%
\pgfpathlineto{\pgfqpoint{3.220311in}{0.688753in}}%
\pgfpathmoveto{\pgfqpoint{3.074997in}{0.783125in}}%
\pgfpathlineto{\pgfqpoint{3.074997in}{0.783125in}}%
\pgfpathlineto{\pgfqpoint{3.074997in}{0.877501in}}%
\pgfpathlineto{\pgfqpoint{3.220311in}{0.877501in}}%
\pgfpathlineto{\pgfqpoint{3.220311in}{0.783125in}}%
\pgfpathmoveto{\pgfqpoint{3.074997in}{0.877501in}}%
\pgfpathlineto{\pgfqpoint{3.074997in}{0.877501in}}%
\pgfpathlineto{\pgfqpoint{3.074997in}{0.971874in}}%
\pgfpathlineto{\pgfqpoint{3.220311in}{0.971874in}}%
\pgfpathlineto{\pgfqpoint{3.220311in}{0.877501in}}%
\pgfpathmoveto{\pgfqpoint{3.074997in}{0.971874in}}%
\pgfpathlineto{\pgfqpoint{3.074997in}{0.971874in}}%
\pgfpathlineto{\pgfqpoint{3.074997in}{1.066247in}}%
\pgfpathlineto{\pgfqpoint{3.220311in}{1.066247in}}%
\pgfpathlineto{\pgfqpoint{3.220311in}{0.971874in}}%
\pgfpathmoveto{\pgfqpoint{3.074997in}{1.066247in}}%
\pgfpathlineto{\pgfqpoint{3.074997in}{1.066247in}}%
\pgfpathlineto{\pgfqpoint{3.074997in}{1.160624in}}%
\pgfpathlineto{\pgfqpoint{3.220311in}{1.160624in}}%
\pgfpathlineto{\pgfqpoint{3.220311in}{1.066247in}}%
\pgfpathmoveto{\pgfqpoint{3.074997in}{1.160624in}}%
\pgfpathlineto{\pgfqpoint{3.074997in}{1.160624in}}%
\pgfpathlineto{\pgfqpoint{3.074997in}{1.254999in}}%
\pgfpathlineto{\pgfqpoint{3.220311in}{1.254999in}}%
\pgfpathlineto{\pgfqpoint{3.220311in}{1.160624in}}%
\pgfpathmoveto{\pgfqpoint{3.074997in}{1.254999in}}%
\pgfpathlineto{\pgfqpoint{3.074997in}{1.254999in}}%
\pgfpathlineto{\pgfqpoint{3.074997in}{1.349373in}}%
\pgfpathlineto{\pgfqpoint{3.220311in}{1.349373in}}%
\pgfpathlineto{\pgfqpoint{3.220311in}{1.254999in}}%
\pgfpathmoveto{\pgfqpoint{3.074997in}{1.349373in}}%
\pgfpathlineto{\pgfqpoint{3.074997in}{1.349373in}}%
\pgfpathlineto{\pgfqpoint{3.074997in}{1.443752in}}%
\pgfpathlineto{\pgfqpoint{3.220311in}{1.443752in}}%
\pgfpathlineto{\pgfqpoint{3.220311in}{1.349373in}}%
\pgfpathmoveto{\pgfqpoint{3.074997in}{1.443752in}}%
\pgfpathlineto{\pgfqpoint{3.074997in}{1.443752in}}%
\pgfpathlineto{\pgfqpoint{3.074997in}{1.538128in}}%
\pgfpathlineto{\pgfqpoint{3.220311in}{1.538128in}}%
\pgfpathlineto{\pgfqpoint{3.220311in}{1.443752in}}%
\pgfpathmoveto{\pgfqpoint{3.074997in}{1.538128in}}%
\pgfpathlineto{\pgfqpoint{3.074997in}{1.538128in}}%
\pgfpathlineto{\pgfqpoint{3.074997in}{1.632499in}}%
\pgfpathlineto{\pgfqpoint{3.220311in}{1.632499in}}%
\pgfpathlineto{\pgfqpoint{3.220311in}{1.538128in}}%
\pgfpathmoveto{\pgfqpoint{3.074997in}{1.632499in}}%
\pgfpathlineto{\pgfqpoint{3.074997in}{1.632499in}}%
\pgfpathlineto{\pgfqpoint{3.074997in}{1.726877in}}%
\pgfpathlineto{\pgfqpoint{3.220311in}{1.726877in}}%
\pgfpathlineto{\pgfqpoint{3.220311in}{1.632499in}}%
\pgfpathmoveto{\pgfqpoint{3.074997in}{1.726877in}}%
\pgfpathlineto{\pgfqpoint{3.074997in}{1.726877in}}%
\pgfpathlineto{\pgfqpoint{3.074997in}{1.821249in}}%
\pgfpathlineto{\pgfqpoint{3.220311in}{1.821249in}}%
\pgfpathlineto{\pgfqpoint{3.220311in}{1.726877in}}%
\pgfpathmoveto{\pgfqpoint{3.220311in}{0.499998in}}%
\pgfpathlineto{\pgfqpoint{3.220311in}{0.499998in}}%
\pgfpathlineto{\pgfqpoint{3.220311in}{0.594373in}}%
\pgfpathlineto{\pgfqpoint{3.365628in}{0.594373in}}%
\pgfpathlineto{\pgfqpoint{3.365628in}{0.499998in}}%
\pgfpathmoveto{\pgfqpoint{3.220311in}{0.594373in}}%
\pgfpathlineto{\pgfqpoint{3.220311in}{0.594373in}}%
\pgfpathlineto{\pgfqpoint{3.220311in}{0.688753in}}%
\pgfpathlineto{\pgfqpoint{3.365628in}{0.688753in}}%
\pgfpathlineto{\pgfqpoint{3.365628in}{0.594373in}}%
\pgfpathmoveto{\pgfqpoint{3.220311in}{0.688753in}}%
\pgfpathlineto{\pgfqpoint{3.220311in}{0.688753in}}%
\pgfpathlineto{\pgfqpoint{3.220311in}{0.783125in}}%
\pgfpathlineto{\pgfqpoint{3.365628in}{0.783125in}}%
\pgfpathlineto{\pgfqpoint{3.365628in}{0.688753in}}%
\pgfpathmoveto{\pgfqpoint{3.220311in}{0.783125in}}%
\pgfpathlineto{\pgfqpoint{3.220311in}{0.783125in}}%
\pgfpathlineto{\pgfqpoint{3.220311in}{0.877501in}}%
\pgfpathlineto{\pgfqpoint{3.365628in}{0.877501in}}%
\pgfpathlineto{\pgfqpoint{3.365628in}{0.783125in}}%
\pgfpathmoveto{\pgfqpoint{3.220311in}{0.877501in}}%
\pgfpathlineto{\pgfqpoint{3.220311in}{0.877501in}}%
\pgfpathlineto{\pgfqpoint{3.220311in}{0.971874in}}%
\pgfpathlineto{\pgfqpoint{3.365628in}{0.971874in}}%
\pgfpathlineto{\pgfqpoint{3.365628in}{0.877501in}}%
\pgfpathmoveto{\pgfqpoint{3.220311in}{0.971874in}}%
\pgfpathlineto{\pgfqpoint{3.220311in}{0.971874in}}%
\pgfpathlineto{\pgfqpoint{3.220311in}{1.066247in}}%
\pgfpathlineto{\pgfqpoint{3.365628in}{1.066247in}}%
\pgfpathlineto{\pgfqpoint{3.365628in}{0.971874in}}%
\pgfpathmoveto{\pgfqpoint{3.220311in}{1.066247in}}%
\pgfpathlineto{\pgfqpoint{3.220311in}{1.066247in}}%
\pgfpathlineto{\pgfqpoint{3.220311in}{1.160624in}}%
\pgfpathlineto{\pgfqpoint{3.365628in}{1.160624in}}%
\pgfpathlineto{\pgfqpoint{3.365628in}{1.066247in}}%
\pgfpathmoveto{\pgfqpoint{3.220311in}{1.160624in}}%
\pgfpathlineto{\pgfqpoint{3.220311in}{1.160624in}}%
\pgfpathlineto{\pgfqpoint{3.220311in}{1.254999in}}%
\pgfpathlineto{\pgfqpoint{3.365628in}{1.254999in}}%
\pgfpathlineto{\pgfqpoint{3.365628in}{1.160624in}}%
\pgfpathmoveto{\pgfqpoint{3.220311in}{1.254999in}}%
\pgfpathlineto{\pgfqpoint{3.220311in}{1.254999in}}%
\pgfpathlineto{\pgfqpoint{3.220311in}{1.349373in}}%
\pgfpathlineto{\pgfqpoint{3.365628in}{1.349373in}}%
\pgfpathlineto{\pgfqpoint{3.365628in}{1.254999in}}%
\pgfpathmoveto{\pgfqpoint{3.220311in}{1.349373in}}%
\pgfpathlineto{\pgfqpoint{3.220311in}{1.349373in}}%
\pgfpathlineto{\pgfqpoint{3.220311in}{1.443752in}}%
\pgfpathlineto{\pgfqpoint{3.365628in}{1.443752in}}%
\pgfpathlineto{\pgfqpoint{3.365628in}{1.349373in}}%
\pgfpathmoveto{\pgfqpoint{3.220311in}{1.443752in}}%
\pgfpathlineto{\pgfqpoint{3.220311in}{1.443752in}}%
\pgfpathlineto{\pgfqpoint{3.220311in}{1.538128in}}%
\pgfpathlineto{\pgfqpoint{3.365628in}{1.538128in}}%
\pgfpathlineto{\pgfqpoint{3.365628in}{1.443752in}}%
\pgfpathmoveto{\pgfqpoint{3.220311in}{1.538128in}}%
\pgfpathlineto{\pgfqpoint{3.220311in}{1.538128in}}%
\pgfpathlineto{\pgfqpoint{3.220311in}{1.632499in}}%
\pgfpathlineto{\pgfqpoint{3.365628in}{1.632499in}}%
\pgfpathlineto{\pgfqpoint{3.365628in}{1.538128in}}%
\pgfpathmoveto{\pgfqpoint{3.220311in}{1.632499in}}%
\pgfpathlineto{\pgfqpoint{3.220311in}{1.632499in}}%
\pgfpathlineto{\pgfqpoint{3.220311in}{1.726877in}}%
\pgfpathlineto{\pgfqpoint{3.365628in}{1.726877in}}%
\pgfpathlineto{\pgfqpoint{3.365628in}{1.632499in}}%
\pgfpathmoveto{\pgfqpoint{3.365628in}{0.594373in}}%
\pgfpathlineto{\pgfqpoint{3.365628in}{0.594373in}}%
\pgfpathlineto{\pgfqpoint{3.365628in}{0.688753in}}%
\pgfpathlineto{\pgfqpoint{3.510939in}{0.688753in}}%
\pgfpathlineto{\pgfqpoint{3.510939in}{0.594373in}}%
\pgfpathmoveto{\pgfqpoint{3.365628in}{0.688753in}}%
\pgfpathlineto{\pgfqpoint{3.365628in}{0.688753in}}%
\pgfpathlineto{\pgfqpoint{3.365628in}{0.783125in}}%
\pgfpathlineto{\pgfqpoint{3.510939in}{0.783125in}}%
\pgfpathlineto{\pgfqpoint{3.510939in}{0.688753in}}%
\pgfpathmoveto{\pgfqpoint{3.365628in}{0.783125in}}%
\pgfpathlineto{\pgfqpoint{3.365628in}{0.783125in}}%
\pgfpathlineto{\pgfqpoint{3.365628in}{0.877501in}}%
\pgfpathlineto{\pgfqpoint{3.510939in}{0.877501in}}%
\pgfpathlineto{\pgfqpoint{3.510939in}{0.783125in}}%
\pgfpathmoveto{\pgfqpoint{3.365628in}{0.877501in}}%
\pgfpathlineto{\pgfqpoint{3.365628in}{0.877501in}}%
\pgfpathlineto{\pgfqpoint{3.365628in}{0.971874in}}%
\pgfpathlineto{\pgfqpoint{3.510939in}{0.971874in}}%
\pgfpathlineto{\pgfqpoint{3.510939in}{0.877501in}}%
\pgfpathmoveto{\pgfqpoint{3.365628in}{0.971874in}}%
\pgfpathlineto{\pgfqpoint{3.365628in}{0.971874in}}%
\pgfpathlineto{\pgfqpoint{3.365628in}{1.066247in}}%
\pgfpathlineto{\pgfqpoint{3.510939in}{1.066247in}}%
\pgfpathlineto{\pgfqpoint{3.510939in}{0.971874in}}%
\pgfpathmoveto{\pgfqpoint{3.365628in}{1.066247in}}%
\pgfpathlineto{\pgfqpoint{3.365628in}{1.066247in}}%
\pgfpathlineto{\pgfqpoint{3.365628in}{1.160624in}}%
\pgfpathlineto{\pgfqpoint{3.510939in}{1.160624in}}%
\pgfpathlineto{\pgfqpoint{3.510939in}{1.066247in}}%
\pgfpathmoveto{\pgfqpoint{3.365628in}{1.160624in}}%
\pgfpathlineto{\pgfqpoint{3.365628in}{1.160624in}}%
\pgfpathlineto{\pgfqpoint{3.365628in}{1.254999in}}%
\pgfpathlineto{\pgfqpoint{3.510939in}{1.254999in}}%
\pgfpathlineto{\pgfqpoint{3.510939in}{1.160624in}}%
\pgfpathmoveto{\pgfqpoint{3.365628in}{1.254999in}}%
\pgfpathlineto{\pgfqpoint{3.365628in}{1.254999in}}%
\pgfpathlineto{\pgfqpoint{3.365628in}{1.349373in}}%
\pgfpathlineto{\pgfqpoint{3.510939in}{1.349373in}}%
\pgfpathlineto{\pgfqpoint{3.510939in}{1.254999in}}%
\pgfpathmoveto{\pgfqpoint{3.365628in}{1.349373in}}%
\pgfpathlineto{\pgfqpoint{3.365628in}{1.349373in}}%
\pgfpathlineto{\pgfqpoint{3.365628in}{1.443752in}}%
\pgfpathlineto{\pgfqpoint{3.510939in}{1.443752in}}%
\pgfpathlineto{\pgfqpoint{3.510939in}{1.349373in}}%
\pgfpathmoveto{\pgfqpoint{3.365628in}{1.443752in}}%
\pgfpathlineto{\pgfqpoint{3.365628in}{1.443752in}}%
\pgfpathlineto{\pgfqpoint{3.365628in}{1.538128in}}%
\pgfpathlineto{\pgfqpoint{3.510939in}{1.538128in}}%
\pgfpathlineto{\pgfqpoint{3.510939in}{1.443752in}}%
\pgfpathmoveto{\pgfqpoint{3.365628in}{1.538128in}}%
\pgfpathlineto{\pgfqpoint{3.365628in}{1.538128in}}%
\pgfpathlineto{\pgfqpoint{3.365628in}{1.632499in}}%
\pgfpathlineto{\pgfqpoint{3.510939in}{1.632499in}}%
\pgfpathlineto{\pgfqpoint{3.510939in}{1.538128in}}%
\pgfpathmoveto{\pgfqpoint{3.510939in}{0.688753in}}%
\pgfpathlineto{\pgfqpoint{3.510939in}{0.688753in}}%
\pgfpathlineto{\pgfqpoint{3.510939in}{0.783125in}}%
\pgfpathlineto{\pgfqpoint{3.656250in}{0.783125in}}%
\pgfpathlineto{\pgfqpoint{3.656250in}{0.688753in}}%
\pgfpathmoveto{\pgfqpoint{3.510939in}{0.783125in}}%
\pgfpathlineto{\pgfqpoint{3.510939in}{0.783125in}}%
\pgfpathlineto{\pgfqpoint{3.510939in}{0.877501in}}%
\pgfpathlineto{\pgfqpoint{3.656250in}{0.877501in}}%
\pgfpathlineto{\pgfqpoint{3.656250in}{0.783125in}}%
\pgfpathmoveto{\pgfqpoint{3.510939in}{0.877501in}}%
\pgfpathlineto{\pgfqpoint{3.510939in}{0.877501in}}%
\pgfpathlineto{\pgfqpoint{3.510939in}{0.971874in}}%
\pgfpathlineto{\pgfqpoint{3.656250in}{0.971874in}}%
\pgfpathlineto{\pgfqpoint{3.656250in}{0.877501in}}%
\pgfpathmoveto{\pgfqpoint{3.510939in}{0.971874in}}%
\pgfpathlineto{\pgfqpoint{3.510939in}{0.971874in}}%
\pgfpathlineto{\pgfqpoint{3.510939in}{1.066247in}}%
\pgfpathlineto{\pgfqpoint{3.656250in}{1.066247in}}%
\pgfpathlineto{\pgfqpoint{3.656250in}{0.971874in}}%
\pgfpathmoveto{\pgfqpoint{3.510939in}{1.066247in}}%
\pgfpathlineto{\pgfqpoint{3.510939in}{1.066247in}}%
\pgfpathlineto{\pgfqpoint{3.510939in}{1.160624in}}%
\pgfpathlineto{\pgfqpoint{3.656250in}{1.160624in}}%
\pgfpathlineto{\pgfqpoint{3.656250in}{1.066247in}}%
\pgfpathmoveto{\pgfqpoint{3.510939in}{1.160624in}}%
\pgfpathlineto{\pgfqpoint{3.510939in}{1.160624in}}%
\pgfpathlineto{\pgfqpoint{3.510939in}{1.254999in}}%
\pgfpathlineto{\pgfqpoint{3.656250in}{1.254999in}}%
\pgfpathlineto{\pgfqpoint{3.656250in}{1.160624in}}%
\pgfpathmoveto{\pgfqpoint{3.510939in}{1.254999in}}%
\pgfpathlineto{\pgfqpoint{3.510939in}{1.254999in}}%
\pgfpathlineto{\pgfqpoint{3.510939in}{1.349373in}}%
\pgfpathlineto{\pgfqpoint{3.656250in}{1.349373in}}%
\pgfpathlineto{\pgfqpoint{3.656250in}{1.254999in}}%
\pgfpathmoveto{\pgfqpoint{3.510939in}{1.349373in}}%
\pgfpathlineto{\pgfqpoint{3.510939in}{1.349373in}}%
\pgfpathlineto{\pgfqpoint{3.510939in}{1.443752in}}%
\pgfpathlineto{\pgfqpoint{3.656250in}{1.443752in}}%
\pgfpathlineto{\pgfqpoint{3.656250in}{1.349373in}}%
\pgfpathmoveto{\pgfqpoint{3.656250in}{0.783125in}}%
\pgfpathlineto{\pgfqpoint{3.656250in}{0.783125in}}%
\pgfpathlineto{\pgfqpoint{3.656250in}{0.877501in}}%
\pgfpathlineto{\pgfqpoint{3.801566in}{0.877501in}}%
\pgfpathlineto{\pgfqpoint{3.801566in}{0.783125in}}%
\pgfpathmoveto{\pgfqpoint{3.656250in}{0.877501in}}%
\pgfpathlineto{\pgfqpoint{3.656250in}{0.877501in}}%
\pgfpathlineto{\pgfqpoint{3.656250in}{0.971874in}}%
\pgfpathlineto{\pgfqpoint{3.801566in}{0.971874in}}%
\pgfpathlineto{\pgfqpoint{3.801566in}{0.877501in}}%
\pgfpathmoveto{\pgfqpoint{3.656250in}{0.971874in}}%
\pgfpathlineto{\pgfqpoint{3.656250in}{0.971874in}}%
\pgfpathlineto{\pgfqpoint{3.656250in}{1.066247in}}%
\pgfpathlineto{\pgfqpoint{3.801566in}{1.066247in}}%
\pgfpathlineto{\pgfqpoint{3.801566in}{0.971874in}}%
\pgfpathmoveto{\pgfqpoint{3.656250in}{1.066247in}}%
\pgfpathlineto{\pgfqpoint{3.656250in}{1.066247in}}%
\pgfpathlineto{\pgfqpoint{3.656250in}{1.160624in}}%
\pgfpathlineto{\pgfqpoint{3.801566in}{1.160624in}}%
\pgfpathlineto{\pgfqpoint{3.801566in}{1.066247in}}%
\pgfpathmoveto{\pgfqpoint{3.656250in}{1.160624in}}%
\pgfpathlineto{\pgfqpoint{3.656250in}{1.160624in}}%
\pgfpathlineto{\pgfqpoint{3.656250in}{1.254999in}}%
\pgfpathlineto{\pgfqpoint{3.801566in}{1.254999in}}%
\pgfpathlineto{\pgfqpoint{3.801566in}{1.160624in}}%
\pgfpathmoveto{\pgfqpoint{3.656250in}{1.254999in}}%
\pgfpathlineto{\pgfqpoint{3.656250in}{1.254999in}}%
\pgfpathlineto{\pgfqpoint{3.656250in}{1.349373in}}%
\pgfpathlineto{\pgfqpoint{3.801566in}{1.349373in}}%
\pgfpathlineto{\pgfqpoint{3.801566in}{1.254999in}}%
\pgfpathmoveto{\pgfqpoint{3.801566in}{0.877501in}}%
\pgfpathlineto{\pgfqpoint{3.801566in}{0.877501in}}%
\pgfpathlineto{\pgfqpoint{3.801566in}{0.971874in}}%
\pgfpathlineto{\pgfqpoint{3.946873in}{0.971874in}}%
\pgfpathlineto{\pgfqpoint{3.946873in}{0.877501in}}%
\pgfpathmoveto{\pgfqpoint{3.801566in}{0.971874in}}%
\pgfpathlineto{\pgfqpoint{3.801566in}{0.971874in}}%
\pgfpathlineto{\pgfqpoint{3.801566in}{1.066247in}}%
\pgfpathlineto{\pgfqpoint{3.946873in}{1.066247in}}%
\pgfpathlineto{\pgfqpoint{3.946873in}{0.971874in}}%
\pgfpathmoveto{\pgfqpoint{3.801566in}{1.066247in}}%
\pgfpathlineto{\pgfqpoint{3.801566in}{1.066247in}}%
\pgfpathlineto{\pgfqpoint{3.801566in}{1.160624in}}%
\pgfpathlineto{\pgfqpoint{3.946873in}{1.160624in}}%
\pgfpathlineto{\pgfqpoint{3.946873in}{1.066247in}}%
\pgfpathmoveto{\pgfqpoint{3.801566in}{1.160624in}}%
\pgfpathlineto{\pgfqpoint{3.801566in}{1.160624in}}%
\pgfpathlineto{\pgfqpoint{3.801566in}{1.254999in}}%
\pgfpathlineto{\pgfqpoint{3.946873in}{1.254999in}}%
\pgfpathlineto{\pgfqpoint{3.946873in}{1.160624in}}%
\pgfpathmoveto{\pgfqpoint{1.331252in}{3.331248in}}%
\pgfpathlineto{\pgfqpoint{1.331252in}{3.331248in}}%
\pgfpathlineto{\pgfqpoint{1.331252in}{3.378436in}}%
\pgfpathlineto{\pgfqpoint{1.403906in}{3.378436in}}%
\pgfpathlineto{\pgfqpoint{1.403906in}{3.331248in}}%
\pgfpathmoveto{\pgfqpoint{1.331252in}{3.378436in}}%
\pgfpathlineto{\pgfqpoint{1.331252in}{3.378436in}}%
\pgfpathlineto{\pgfqpoint{1.331252in}{3.425623in}}%
\pgfpathlineto{\pgfqpoint{1.403906in}{3.425623in}}%
\pgfpathlineto{\pgfqpoint{1.403906in}{3.378436in}}%
\pgfpathmoveto{\pgfqpoint{1.403906in}{3.331248in}}%
\pgfpathlineto{\pgfqpoint{1.403906in}{3.331248in}}%
\pgfpathlineto{\pgfqpoint{1.403906in}{3.378436in}}%
\pgfpathlineto{\pgfqpoint{1.476560in}{3.378436in}}%
\pgfpathlineto{\pgfqpoint{1.476560in}{3.331248in}}%
\pgfpathmoveto{\pgfqpoint{1.476560in}{3.236876in}}%
\pgfpathlineto{\pgfqpoint{1.476560in}{3.236876in}}%
\pgfpathlineto{\pgfqpoint{1.476560in}{3.284062in}}%
\pgfpathlineto{\pgfqpoint{1.549220in}{3.284062in}}%
\pgfpathlineto{\pgfqpoint{1.549220in}{3.236876in}}%
\pgfpathmoveto{\pgfqpoint{1.476560in}{3.284062in}}%
\pgfpathlineto{\pgfqpoint{1.476560in}{3.284062in}}%
\pgfpathlineto{\pgfqpoint{1.476560in}{3.331248in}}%
\pgfpathlineto{\pgfqpoint{1.549220in}{3.331248in}}%
\pgfpathlineto{\pgfqpoint{1.549220in}{3.284062in}}%
\pgfpathmoveto{\pgfqpoint{1.621879in}{3.142498in}}%
\pgfpathlineto{\pgfqpoint{1.621879in}{3.142498in}}%
\pgfpathlineto{\pgfqpoint{1.621879in}{3.189687in}}%
\pgfpathlineto{\pgfqpoint{1.694534in}{3.189687in}}%
\pgfpathlineto{\pgfqpoint{1.694534in}{3.142498in}}%
\pgfpathmoveto{\pgfqpoint{1.767189in}{2.953749in}}%
\pgfpathlineto{\pgfqpoint{1.767189in}{2.953749in}}%
\pgfpathlineto{\pgfqpoint{1.767189in}{3.000937in}}%
\pgfpathlineto{\pgfqpoint{1.839844in}{3.000937in}}%
\pgfpathlineto{\pgfqpoint{1.839844in}{2.953749in}}%
\pgfpathmoveto{\pgfqpoint{1.767189in}{3.000937in}}%
\pgfpathlineto{\pgfqpoint{1.767189in}{3.000937in}}%
\pgfpathlineto{\pgfqpoint{1.767189in}{3.048124in}}%
\pgfpathlineto{\pgfqpoint{1.839844in}{3.048124in}}%
\pgfpathlineto{\pgfqpoint{1.839844in}{3.000937in}}%
\pgfpathmoveto{\pgfqpoint{1.839844in}{2.953749in}}%
\pgfpathlineto{\pgfqpoint{1.839844in}{2.953749in}}%
\pgfpathlineto{\pgfqpoint{1.839844in}{3.000937in}}%
\pgfpathlineto{\pgfqpoint{1.912499in}{3.000937in}}%
\pgfpathlineto{\pgfqpoint{1.912499in}{2.953749in}}%
\pgfpathmoveto{\pgfqpoint{1.912499in}{2.859372in}}%
\pgfpathlineto{\pgfqpoint{1.912499in}{2.859372in}}%
\pgfpathlineto{\pgfqpoint{1.912499in}{2.906561in}}%
\pgfpathlineto{\pgfqpoint{1.985158in}{2.906561in}}%
\pgfpathlineto{\pgfqpoint{1.985158in}{2.859372in}}%
\pgfpathmoveto{\pgfqpoint{2.057816in}{2.765003in}}%
\pgfpathlineto{\pgfqpoint{2.057816in}{2.765003in}}%
\pgfpathlineto{\pgfqpoint{2.057816in}{2.812188in}}%
\pgfpathlineto{\pgfqpoint{2.130469in}{2.812188in}}%
\pgfpathlineto{\pgfqpoint{2.130469in}{2.765003in}}%
\pgfpathmoveto{\pgfqpoint{2.203122in}{2.576249in}}%
\pgfpathlineto{\pgfqpoint{2.203122in}{2.576249in}}%
\pgfpathlineto{\pgfqpoint{2.203122in}{2.623436in}}%
\pgfpathlineto{\pgfqpoint{2.275780in}{2.623436in}}%
\pgfpathlineto{\pgfqpoint{2.275780in}{2.576249in}}%
\pgfpathmoveto{\pgfqpoint{2.203122in}{2.623436in}}%
\pgfpathlineto{\pgfqpoint{2.203122in}{2.623436in}}%
\pgfpathlineto{\pgfqpoint{2.203122in}{2.670624in}}%
\pgfpathlineto{\pgfqpoint{2.275780in}{2.670624in}}%
\pgfpathlineto{\pgfqpoint{2.275780in}{2.623436in}}%
\pgfpathmoveto{\pgfqpoint{2.275780in}{2.576249in}}%
\pgfpathlineto{\pgfqpoint{2.275780in}{2.576249in}}%
\pgfpathlineto{\pgfqpoint{2.275780in}{2.623436in}}%
\pgfpathlineto{\pgfqpoint{2.348439in}{2.623436in}}%
\pgfpathlineto{\pgfqpoint{2.348439in}{2.576249in}}%
\pgfpathmoveto{\pgfqpoint{2.348439in}{2.481875in}}%
\pgfpathlineto{\pgfqpoint{2.348439in}{2.481875in}}%
\pgfpathlineto{\pgfqpoint{2.348439in}{2.529062in}}%
\pgfpathlineto{\pgfqpoint{2.421093in}{2.529062in}}%
\pgfpathlineto{\pgfqpoint{2.421093in}{2.481875in}}%
\pgfpathmoveto{\pgfqpoint{2.348439in}{2.529062in}}%
\pgfpathlineto{\pgfqpoint{2.348439in}{2.529062in}}%
\pgfpathlineto{\pgfqpoint{2.348439in}{2.576249in}}%
\pgfpathlineto{\pgfqpoint{2.421093in}{2.576249in}}%
\pgfpathlineto{\pgfqpoint{2.421093in}{2.529062in}}%
\pgfpathmoveto{\pgfqpoint{2.493746in}{2.387498in}}%
\pgfpathlineto{\pgfqpoint{2.493746in}{2.387498in}}%
\pgfpathlineto{\pgfqpoint{2.493746in}{2.434687in}}%
\pgfpathlineto{\pgfqpoint{2.566402in}{2.434687in}}%
\pgfpathlineto{\pgfqpoint{2.566402in}{2.387498in}}%
\pgfpathmoveto{\pgfqpoint{2.639058in}{2.198749in}}%
\pgfpathlineto{\pgfqpoint{2.639058in}{2.198749in}}%
\pgfpathlineto{\pgfqpoint{2.639058in}{2.245936in}}%
\pgfpathlineto{\pgfqpoint{2.711715in}{2.245936in}}%
\pgfpathlineto{\pgfqpoint{2.711715in}{2.198749in}}%
\pgfpathmoveto{\pgfqpoint{2.639058in}{2.245936in}}%
\pgfpathlineto{\pgfqpoint{2.639058in}{2.245936in}}%
\pgfpathlineto{\pgfqpoint{2.639058in}{2.293123in}}%
\pgfpathlineto{\pgfqpoint{2.711715in}{2.293123in}}%
\pgfpathlineto{\pgfqpoint{2.711715in}{2.245936in}}%
\pgfpathmoveto{\pgfqpoint{2.711715in}{2.198749in}}%
\pgfpathlineto{\pgfqpoint{2.711715in}{2.198749in}}%
\pgfpathlineto{\pgfqpoint{2.711715in}{2.245936in}}%
\pgfpathlineto{\pgfqpoint{2.784372in}{2.245936in}}%
\pgfpathlineto{\pgfqpoint{2.784372in}{2.198749in}}%
\pgfpathmoveto{\pgfqpoint{2.784372in}{2.104374in}}%
\pgfpathlineto{\pgfqpoint{2.784372in}{2.104374in}}%
\pgfpathlineto{\pgfqpoint{2.784372in}{2.151562in}}%
\pgfpathlineto{\pgfqpoint{2.857030in}{2.151562in}}%
\pgfpathlineto{\pgfqpoint{2.857030in}{2.104374in}}%
\pgfpathmoveto{\pgfqpoint{2.784372in}{2.151562in}}%
\pgfpathlineto{\pgfqpoint{2.784372in}{2.151562in}}%
\pgfpathlineto{\pgfqpoint{2.784372in}{2.198749in}}%
\pgfpathlineto{\pgfqpoint{2.857030in}{2.198749in}}%
\pgfpathlineto{\pgfqpoint{2.857030in}{2.151562in}}%
\pgfpathmoveto{\pgfqpoint{2.929689in}{2.009998in}}%
\pgfpathlineto{\pgfqpoint{2.929689in}{2.009998in}}%
\pgfpathlineto{\pgfqpoint{2.929689in}{2.057186in}}%
\pgfpathlineto{\pgfqpoint{3.002343in}{2.057186in}}%
\pgfpathlineto{\pgfqpoint{3.002343in}{2.009998in}}%
\pgfpathmoveto{\pgfqpoint{3.074997in}{1.821249in}}%
\pgfpathlineto{\pgfqpoint{3.074997in}{1.821249in}}%
\pgfpathlineto{\pgfqpoint{3.074997in}{1.868438in}}%
\pgfpathlineto{\pgfqpoint{3.147654in}{1.868438in}}%
\pgfpathlineto{\pgfqpoint{3.147654in}{1.821249in}}%
\pgfpathmoveto{\pgfqpoint{3.074997in}{1.868438in}}%
\pgfpathlineto{\pgfqpoint{3.074997in}{1.868438in}}%
\pgfpathlineto{\pgfqpoint{3.074997in}{1.915628in}}%
\pgfpathlineto{\pgfqpoint{3.147654in}{1.915628in}}%
\pgfpathlineto{\pgfqpoint{3.147654in}{1.868438in}}%
\pgfpathmoveto{\pgfqpoint{3.147654in}{1.821249in}}%
\pgfpathlineto{\pgfqpoint{3.147654in}{1.821249in}}%
\pgfpathlineto{\pgfqpoint{3.147654in}{1.868438in}}%
\pgfpathlineto{\pgfqpoint{3.220311in}{1.868438in}}%
\pgfpathlineto{\pgfqpoint{3.220311in}{1.821249in}}%
\pgfpathmoveto{\pgfqpoint{3.220311in}{1.726877in}}%
\pgfpathlineto{\pgfqpoint{3.220311in}{1.726877in}}%
\pgfpathlineto{\pgfqpoint{3.220311in}{1.774063in}}%
\pgfpathlineto{\pgfqpoint{3.292969in}{1.774063in}}%
\pgfpathlineto{\pgfqpoint{3.292969in}{1.726877in}}%
\pgfpathmoveto{\pgfqpoint{3.220311in}{1.774063in}}%
\pgfpathlineto{\pgfqpoint{3.220311in}{1.774063in}}%
\pgfpathlineto{\pgfqpoint{3.220311in}{1.821249in}}%
\pgfpathlineto{\pgfqpoint{3.292969in}{1.821249in}}%
\pgfpathlineto{\pgfqpoint{3.292969in}{1.774063in}}%
\pgfpathmoveto{\pgfqpoint{3.365628in}{0.499998in}}%
\pgfpathlineto{\pgfqpoint{3.365628in}{0.499998in}}%
\pgfpathlineto{\pgfqpoint{3.365628in}{0.547186in}}%
\pgfpathlineto{\pgfqpoint{3.438283in}{0.547186in}}%
\pgfpathlineto{\pgfqpoint{3.438283in}{0.499998in}}%
\pgfpathmoveto{\pgfqpoint{3.365628in}{0.547186in}}%
\pgfpathlineto{\pgfqpoint{3.365628in}{0.547186in}}%
\pgfpathlineto{\pgfqpoint{3.365628in}{0.594373in}}%
\pgfpathlineto{\pgfqpoint{3.438283in}{0.594373in}}%
\pgfpathlineto{\pgfqpoint{3.438283in}{0.547186in}}%
\pgfpathmoveto{\pgfqpoint{3.438283in}{0.547186in}}%
\pgfpathlineto{\pgfqpoint{3.438283in}{0.547186in}}%
\pgfpathlineto{\pgfqpoint{3.438283in}{0.594373in}}%
\pgfpathlineto{\pgfqpoint{3.510939in}{0.594373in}}%
\pgfpathlineto{\pgfqpoint{3.510939in}{0.547186in}}%
\pgfpathmoveto{\pgfqpoint{3.365628in}{1.632499in}}%
\pgfpathlineto{\pgfqpoint{3.365628in}{1.632499in}}%
\pgfpathlineto{\pgfqpoint{3.365628in}{1.679688in}}%
\pgfpathlineto{\pgfqpoint{3.438283in}{1.679688in}}%
\pgfpathlineto{\pgfqpoint{3.438283in}{1.632499in}}%
\pgfpathmoveto{\pgfqpoint{3.510939in}{0.641563in}}%
\pgfpathlineto{\pgfqpoint{3.510939in}{0.641563in}}%
\pgfpathlineto{\pgfqpoint{3.510939in}{0.688753in}}%
\pgfpathlineto{\pgfqpoint{3.583594in}{0.688753in}}%
\pgfpathlineto{\pgfqpoint{3.583594in}{0.641563in}}%
\pgfpathmoveto{\pgfqpoint{3.510939in}{1.443752in}}%
\pgfpathlineto{\pgfqpoint{3.510939in}{1.443752in}}%
\pgfpathlineto{\pgfqpoint{3.510939in}{1.490940in}}%
\pgfpathlineto{\pgfqpoint{3.583594in}{1.490940in}}%
\pgfpathlineto{\pgfqpoint{3.583594in}{1.443752in}}%
\pgfpathmoveto{\pgfqpoint{3.510939in}{1.490940in}}%
\pgfpathlineto{\pgfqpoint{3.510939in}{1.490940in}}%
\pgfpathlineto{\pgfqpoint{3.510939in}{1.538128in}}%
\pgfpathlineto{\pgfqpoint{3.583594in}{1.538128in}}%
\pgfpathlineto{\pgfqpoint{3.583594in}{1.490940in}}%
\pgfpathmoveto{\pgfqpoint{3.583594in}{1.443752in}}%
\pgfpathlineto{\pgfqpoint{3.583594in}{1.443752in}}%
\pgfpathlineto{\pgfqpoint{3.583594in}{1.490940in}}%
\pgfpathlineto{\pgfqpoint{3.656250in}{1.490940in}}%
\pgfpathlineto{\pgfqpoint{3.656250in}{1.443752in}}%
\pgfpathmoveto{\pgfqpoint{3.656250in}{0.735939in}}%
\pgfpathlineto{\pgfqpoint{3.656250in}{0.735939in}}%
\pgfpathlineto{\pgfqpoint{3.656250in}{0.783125in}}%
\pgfpathlineto{\pgfqpoint{3.728908in}{0.783125in}}%
\pgfpathlineto{\pgfqpoint{3.728908in}{0.735939in}}%
\pgfpathmoveto{\pgfqpoint{3.656250in}{1.349373in}}%
\pgfpathlineto{\pgfqpoint{3.656250in}{1.349373in}}%
\pgfpathlineto{\pgfqpoint{3.656250in}{1.396562in}}%
\pgfpathlineto{\pgfqpoint{3.728908in}{1.396562in}}%
\pgfpathlineto{\pgfqpoint{3.728908in}{1.349373in}}%
\pgfpathmoveto{\pgfqpoint{3.801566in}{0.830313in}}%
\pgfpathlineto{\pgfqpoint{3.801566in}{0.830313in}}%
\pgfpathlineto{\pgfqpoint{3.801566in}{0.877501in}}%
\pgfpathlineto{\pgfqpoint{3.874220in}{0.877501in}}%
\pgfpathlineto{\pgfqpoint{3.874220in}{0.830313in}}%
\pgfpathmoveto{\pgfqpoint{3.801566in}{1.254999in}}%
\pgfpathlineto{\pgfqpoint{3.801566in}{1.254999in}}%
\pgfpathlineto{\pgfqpoint{3.801566in}{1.302186in}}%
\pgfpathlineto{\pgfqpoint{3.874220in}{1.302186in}}%
\pgfpathlineto{\pgfqpoint{3.874220in}{1.254999in}}%
\pgfpathmoveto{\pgfqpoint{3.946873in}{0.971874in}}%
\pgfpathlineto{\pgfqpoint{3.946873in}{0.971874in}}%
\pgfpathlineto{\pgfqpoint{3.946873in}{1.019060in}}%
\pgfpathlineto{\pgfqpoint{4.019529in}{1.019060in}}%
\pgfpathlineto{\pgfqpoint{4.019529in}{0.971874in}}%
\pgfpathmoveto{\pgfqpoint{3.946873in}{1.019060in}}%
\pgfpathlineto{\pgfqpoint{3.946873in}{1.019060in}}%
\pgfpathlineto{\pgfqpoint{3.946873in}{1.066247in}}%
\pgfpathlineto{\pgfqpoint{4.019529in}{1.066247in}}%
\pgfpathlineto{\pgfqpoint{4.019529in}{1.019060in}}%
\pgfpathmoveto{\pgfqpoint{4.019529in}{1.019060in}}%
\pgfpathlineto{\pgfqpoint{4.019529in}{1.019060in}}%
\pgfpathlineto{\pgfqpoint{4.019529in}{1.066247in}}%
\pgfpathlineto{\pgfqpoint{4.092184in}{1.066247in}}%
\pgfpathlineto{\pgfqpoint{4.092184in}{1.019060in}}%
\pgfpathmoveto{\pgfqpoint{3.946873in}{1.066247in}}%
\pgfpathlineto{\pgfqpoint{3.946873in}{1.066247in}}%
\pgfpathlineto{\pgfqpoint{3.946873in}{1.113436in}}%
\pgfpathlineto{\pgfqpoint{4.019529in}{1.113436in}}%
\pgfpathlineto{\pgfqpoint{4.019529in}{1.066247in}}%
\pgfpathmoveto{\pgfqpoint{3.946873in}{1.113436in}}%
\pgfpathlineto{\pgfqpoint{3.946873in}{1.113436in}}%
\pgfpathlineto{\pgfqpoint{3.946873in}{1.160624in}}%
\pgfpathlineto{\pgfqpoint{4.019529in}{1.160624in}}%
\pgfpathlineto{\pgfqpoint{4.019529in}{1.113436in}}%
\pgfpathmoveto{\pgfqpoint{4.019529in}{1.066247in}}%
\pgfpathlineto{\pgfqpoint{4.019529in}{1.066247in}}%
\pgfpathlineto{\pgfqpoint{4.019529in}{1.113436in}}%
\pgfpathlineto{\pgfqpoint{4.092184in}{1.113436in}}%
\pgfpathlineto{\pgfqpoint{4.092184in}{1.066247in}}%
\pgfpathmoveto{\pgfqpoint{1.403906in}{3.378436in}}%
\pgfpathlineto{\pgfqpoint{1.403906in}{3.378436in}}%
\pgfpathlineto{\pgfqpoint{1.403906in}{3.402029in}}%
\pgfpathlineto{\pgfqpoint{1.440233in}{3.402029in}}%
\pgfpathlineto{\pgfqpoint{1.440233in}{3.378436in}}%
\pgfpathmoveto{\pgfqpoint{1.403906in}{3.402029in}}%
\pgfpathlineto{\pgfqpoint{1.403906in}{3.402029in}}%
\pgfpathlineto{\pgfqpoint{1.403906in}{3.425623in}}%
\pgfpathlineto{\pgfqpoint{1.440233in}{3.425623in}}%
\pgfpathlineto{\pgfqpoint{1.440233in}{3.402029in}}%
\pgfpathmoveto{\pgfqpoint{1.331252in}{3.425623in}}%
\pgfpathlineto{\pgfqpoint{1.331252in}{3.425623in}}%
\pgfpathlineto{\pgfqpoint{1.331252in}{3.449217in}}%
\pgfpathlineto{\pgfqpoint{1.367579in}{3.449217in}}%
\pgfpathlineto{\pgfqpoint{1.367579in}{3.425623in}}%
\pgfpathmoveto{\pgfqpoint{1.331252in}{3.449217in}}%
\pgfpathlineto{\pgfqpoint{1.331252in}{3.449217in}}%
\pgfpathlineto{\pgfqpoint{1.331252in}{3.472810in}}%
\pgfpathlineto{\pgfqpoint{1.367579in}{3.472810in}}%
\pgfpathlineto{\pgfqpoint{1.367579in}{3.449217in}}%
\pgfpathmoveto{\pgfqpoint{1.367579in}{3.425623in}}%
\pgfpathlineto{\pgfqpoint{1.367579in}{3.425623in}}%
\pgfpathlineto{\pgfqpoint{1.367579in}{3.449217in}}%
\pgfpathlineto{\pgfqpoint{1.403906in}{3.449217in}}%
\pgfpathlineto{\pgfqpoint{1.403906in}{3.425623in}}%
\pgfpathmoveto{\pgfqpoint{1.549220in}{3.236876in}}%
\pgfpathlineto{\pgfqpoint{1.549220in}{3.236876in}}%
\pgfpathlineto{\pgfqpoint{1.549220in}{3.260469in}}%
\pgfpathlineto{\pgfqpoint{1.585549in}{3.260469in}}%
\pgfpathlineto{\pgfqpoint{1.585549in}{3.236876in}}%
\pgfpathmoveto{\pgfqpoint{1.549220in}{3.260469in}}%
\pgfpathlineto{\pgfqpoint{1.549220in}{3.260469in}}%
\pgfpathlineto{\pgfqpoint{1.549220in}{3.284062in}}%
\pgfpathlineto{\pgfqpoint{1.585549in}{3.284062in}}%
\pgfpathlineto{\pgfqpoint{1.585549in}{3.260469in}}%
\pgfpathmoveto{\pgfqpoint{1.585549in}{3.236876in}}%
\pgfpathlineto{\pgfqpoint{1.585549in}{3.236876in}}%
\pgfpathlineto{\pgfqpoint{1.585549in}{3.260469in}}%
\pgfpathlineto{\pgfqpoint{1.621879in}{3.260469in}}%
\pgfpathlineto{\pgfqpoint{1.621879in}{3.236876in}}%
\pgfpathmoveto{\pgfqpoint{1.476560in}{3.331248in}}%
\pgfpathlineto{\pgfqpoint{1.476560in}{3.331248in}}%
\pgfpathlineto{\pgfqpoint{1.476560in}{3.354842in}}%
\pgfpathlineto{\pgfqpoint{1.512890in}{3.354842in}}%
\pgfpathlineto{\pgfqpoint{1.512890in}{3.331248in}}%
\pgfpathmoveto{\pgfqpoint{1.621879in}{3.189687in}}%
\pgfpathlineto{\pgfqpoint{1.621879in}{3.189687in}}%
\pgfpathlineto{\pgfqpoint{1.621879in}{3.213282in}}%
\pgfpathlineto{\pgfqpoint{1.658206in}{3.213282in}}%
\pgfpathlineto{\pgfqpoint{1.658206in}{3.189687in}}%
\pgfpathmoveto{\pgfqpoint{1.694534in}{3.142498in}}%
\pgfpathlineto{\pgfqpoint{1.694534in}{3.142498in}}%
\pgfpathlineto{\pgfqpoint{1.694534in}{3.166093in}}%
\pgfpathlineto{\pgfqpoint{1.730861in}{3.166093in}}%
\pgfpathlineto{\pgfqpoint{1.730861in}{3.142498in}}%
\pgfpathmoveto{\pgfqpoint{1.839844in}{3.000937in}}%
\pgfpathlineto{\pgfqpoint{1.839844in}{3.000937in}}%
\pgfpathlineto{\pgfqpoint{1.839844in}{3.024530in}}%
\pgfpathlineto{\pgfqpoint{1.876171in}{3.024530in}}%
\pgfpathlineto{\pgfqpoint{1.876171in}{3.000937in}}%
\pgfpathmoveto{\pgfqpoint{1.839844in}{3.024530in}}%
\pgfpathlineto{\pgfqpoint{1.839844in}{3.024530in}}%
\pgfpathlineto{\pgfqpoint{1.839844in}{3.048124in}}%
\pgfpathlineto{\pgfqpoint{1.876171in}{3.048124in}}%
\pgfpathlineto{\pgfqpoint{1.876171in}{3.024530in}}%
\pgfpathmoveto{\pgfqpoint{1.767189in}{3.048124in}}%
\pgfpathlineto{\pgfqpoint{1.767189in}{3.048124in}}%
\pgfpathlineto{\pgfqpoint{1.767189in}{3.071718in}}%
\pgfpathlineto{\pgfqpoint{1.803516in}{3.071718in}}%
\pgfpathlineto{\pgfqpoint{1.803516in}{3.048124in}}%
\pgfpathmoveto{\pgfqpoint{1.767189in}{3.071718in}}%
\pgfpathlineto{\pgfqpoint{1.767189in}{3.071718in}}%
\pgfpathlineto{\pgfqpoint{1.767189in}{3.095311in}}%
\pgfpathlineto{\pgfqpoint{1.803516in}{3.095311in}}%
\pgfpathlineto{\pgfqpoint{1.803516in}{3.071718in}}%
\pgfpathmoveto{\pgfqpoint{1.803516in}{3.048124in}}%
\pgfpathlineto{\pgfqpoint{1.803516in}{3.048124in}}%
\pgfpathlineto{\pgfqpoint{1.803516in}{3.071718in}}%
\pgfpathlineto{\pgfqpoint{1.839844in}{3.071718in}}%
\pgfpathlineto{\pgfqpoint{1.839844in}{3.048124in}}%
\pgfpathmoveto{\pgfqpoint{1.912499in}{2.906561in}}%
\pgfpathlineto{\pgfqpoint{1.912499in}{2.906561in}}%
\pgfpathlineto{\pgfqpoint{1.912499in}{2.930155in}}%
\pgfpathlineto{\pgfqpoint{1.948828in}{2.930155in}}%
\pgfpathlineto{\pgfqpoint{1.948828in}{2.906561in}}%
\pgfpathmoveto{\pgfqpoint{1.912499in}{2.930155in}}%
\pgfpathlineto{\pgfqpoint{1.912499in}{2.930155in}}%
\pgfpathlineto{\pgfqpoint{1.912499in}{2.953749in}}%
\pgfpathlineto{\pgfqpoint{1.948828in}{2.953749in}}%
\pgfpathlineto{\pgfqpoint{1.948828in}{2.930155in}}%
\pgfpathmoveto{\pgfqpoint{1.948828in}{2.906561in}}%
\pgfpathlineto{\pgfqpoint{1.948828in}{2.906561in}}%
\pgfpathlineto{\pgfqpoint{1.948828in}{2.930155in}}%
\pgfpathlineto{\pgfqpoint{1.985158in}{2.930155in}}%
\pgfpathlineto{\pgfqpoint{1.985158in}{2.906561in}}%
\pgfpathmoveto{\pgfqpoint{1.985158in}{2.859372in}}%
\pgfpathlineto{\pgfqpoint{1.985158in}{2.859372in}}%
\pgfpathlineto{\pgfqpoint{1.985158in}{2.882966in}}%
\pgfpathlineto{\pgfqpoint{2.021487in}{2.882966in}}%
\pgfpathlineto{\pgfqpoint{2.021487in}{2.859372in}}%
\pgfpathmoveto{\pgfqpoint{1.985158in}{2.882966in}}%
\pgfpathlineto{\pgfqpoint{1.985158in}{2.882966in}}%
\pgfpathlineto{\pgfqpoint{1.985158in}{2.906561in}}%
\pgfpathlineto{\pgfqpoint{2.021487in}{2.906561in}}%
\pgfpathlineto{\pgfqpoint{2.021487in}{2.882966in}}%
\pgfpathmoveto{\pgfqpoint{2.021487in}{2.859372in}}%
\pgfpathlineto{\pgfqpoint{2.021487in}{2.859372in}}%
\pgfpathlineto{\pgfqpoint{2.021487in}{2.882966in}}%
\pgfpathlineto{\pgfqpoint{2.057816in}{2.882966in}}%
\pgfpathlineto{\pgfqpoint{2.057816in}{2.859372in}}%
\pgfpathmoveto{\pgfqpoint{1.912499in}{2.953749in}}%
\pgfpathlineto{\pgfqpoint{1.912499in}{2.953749in}}%
\pgfpathlineto{\pgfqpoint{1.912499in}{2.977343in}}%
\pgfpathlineto{\pgfqpoint{1.948828in}{2.977343in}}%
\pgfpathlineto{\pgfqpoint{1.948828in}{2.953749in}}%
\pgfpathmoveto{\pgfqpoint{2.057816in}{2.812188in}}%
\pgfpathlineto{\pgfqpoint{2.057816in}{2.812188in}}%
\pgfpathlineto{\pgfqpoint{2.057816in}{2.835780in}}%
\pgfpathlineto{\pgfqpoint{2.094143in}{2.835780in}}%
\pgfpathlineto{\pgfqpoint{2.094143in}{2.812188in}}%
\pgfpathmoveto{\pgfqpoint{2.057816in}{2.835780in}}%
\pgfpathlineto{\pgfqpoint{2.057816in}{2.835780in}}%
\pgfpathlineto{\pgfqpoint{2.057816in}{2.859372in}}%
\pgfpathlineto{\pgfqpoint{2.094143in}{2.859372in}}%
\pgfpathlineto{\pgfqpoint{2.094143in}{2.835780in}}%
\pgfpathmoveto{\pgfqpoint{2.130469in}{2.765003in}}%
\pgfpathlineto{\pgfqpoint{2.130469in}{2.765003in}}%
\pgfpathlineto{\pgfqpoint{2.130469in}{2.788595in}}%
\pgfpathlineto{\pgfqpoint{2.166795in}{2.788595in}}%
\pgfpathlineto{\pgfqpoint{2.166795in}{2.765003in}}%
\pgfpathmoveto{\pgfqpoint{2.275780in}{2.623436in}}%
\pgfpathlineto{\pgfqpoint{2.275780in}{2.623436in}}%
\pgfpathlineto{\pgfqpoint{2.275780in}{2.647030in}}%
\pgfpathlineto{\pgfqpoint{2.312110in}{2.647030in}}%
\pgfpathlineto{\pgfqpoint{2.312110in}{2.623436in}}%
\pgfpathmoveto{\pgfqpoint{2.275780in}{2.647030in}}%
\pgfpathlineto{\pgfqpoint{2.275780in}{2.647030in}}%
\pgfpathlineto{\pgfqpoint{2.275780in}{2.670624in}}%
\pgfpathlineto{\pgfqpoint{2.312110in}{2.670624in}}%
\pgfpathlineto{\pgfqpoint{2.312110in}{2.647030in}}%
\pgfpathmoveto{\pgfqpoint{2.203122in}{2.670624in}}%
\pgfpathlineto{\pgfqpoint{2.203122in}{2.670624in}}%
\pgfpathlineto{\pgfqpoint{2.203122in}{2.694218in}}%
\pgfpathlineto{\pgfqpoint{2.239451in}{2.694218in}}%
\pgfpathlineto{\pgfqpoint{2.239451in}{2.670624in}}%
\pgfpathmoveto{\pgfqpoint{2.203122in}{2.694218in}}%
\pgfpathlineto{\pgfqpoint{2.203122in}{2.694218in}}%
\pgfpathlineto{\pgfqpoint{2.203122in}{2.717813in}}%
\pgfpathlineto{\pgfqpoint{2.239451in}{2.717813in}}%
\pgfpathlineto{\pgfqpoint{2.239451in}{2.694218in}}%
\pgfpathmoveto{\pgfqpoint{2.239451in}{2.670624in}}%
\pgfpathlineto{\pgfqpoint{2.239451in}{2.670624in}}%
\pgfpathlineto{\pgfqpoint{2.239451in}{2.694218in}}%
\pgfpathlineto{\pgfqpoint{2.275780in}{2.694218in}}%
\pgfpathlineto{\pgfqpoint{2.275780in}{2.670624in}}%
\pgfpathmoveto{\pgfqpoint{2.421093in}{2.481875in}}%
\pgfpathlineto{\pgfqpoint{2.421093in}{2.481875in}}%
\pgfpathlineto{\pgfqpoint{2.421093in}{2.505469in}}%
\pgfpathlineto{\pgfqpoint{2.457419in}{2.505469in}}%
\pgfpathlineto{\pgfqpoint{2.457419in}{2.481875in}}%
\pgfpathmoveto{\pgfqpoint{2.421093in}{2.505469in}}%
\pgfpathlineto{\pgfqpoint{2.421093in}{2.505469in}}%
\pgfpathlineto{\pgfqpoint{2.421093in}{2.529062in}}%
\pgfpathlineto{\pgfqpoint{2.457419in}{2.529062in}}%
\pgfpathlineto{\pgfqpoint{2.457419in}{2.505469in}}%
\pgfpathmoveto{\pgfqpoint{2.457419in}{2.481875in}}%
\pgfpathlineto{\pgfqpoint{2.457419in}{2.481875in}}%
\pgfpathlineto{\pgfqpoint{2.457419in}{2.505469in}}%
\pgfpathlineto{\pgfqpoint{2.493746in}{2.505469in}}%
\pgfpathlineto{\pgfqpoint{2.493746in}{2.481875in}}%
\pgfpathmoveto{\pgfqpoint{2.348439in}{2.576249in}}%
\pgfpathlineto{\pgfqpoint{2.348439in}{2.576249in}}%
\pgfpathlineto{\pgfqpoint{2.348439in}{2.599843in}}%
\pgfpathlineto{\pgfqpoint{2.384766in}{2.599843in}}%
\pgfpathlineto{\pgfqpoint{2.384766in}{2.576249in}}%
\pgfpathmoveto{\pgfqpoint{2.493746in}{2.434687in}}%
\pgfpathlineto{\pgfqpoint{2.493746in}{2.434687in}}%
\pgfpathlineto{\pgfqpoint{2.493746in}{2.458281in}}%
\pgfpathlineto{\pgfqpoint{2.530074in}{2.458281in}}%
\pgfpathlineto{\pgfqpoint{2.530074in}{2.434687in}}%
\pgfpathmoveto{\pgfqpoint{2.493746in}{2.458281in}}%
\pgfpathlineto{\pgfqpoint{2.493746in}{2.458281in}}%
\pgfpathlineto{\pgfqpoint{2.493746in}{2.481875in}}%
\pgfpathlineto{\pgfqpoint{2.530074in}{2.481875in}}%
\pgfpathlineto{\pgfqpoint{2.530074in}{2.458281in}}%
\pgfpathmoveto{\pgfqpoint{2.566402in}{2.387498in}}%
\pgfpathlineto{\pgfqpoint{2.566402in}{2.387498in}}%
\pgfpathlineto{\pgfqpoint{2.566402in}{2.411093in}}%
\pgfpathlineto{\pgfqpoint{2.602730in}{2.411093in}}%
\pgfpathlineto{\pgfqpoint{2.602730in}{2.387498in}}%
\pgfpathmoveto{\pgfqpoint{2.711715in}{2.245936in}}%
\pgfpathlineto{\pgfqpoint{2.711715in}{2.245936in}}%
\pgfpathlineto{\pgfqpoint{2.711715in}{2.269530in}}%
\pgfpathlineto{\pgfqpoint{2.748043in}{2.269530in}}%
\pgfpathlineto{\pgfqpoint{2.748043in}{2.245936in}}%
\pgfpathmoveto{\pgfqpoint{2.711715in}{2.269530in}}%
\pgfpathlineto{\pgfqpoint{2.711715in}{2.269530in}}%
\pgfpathlineto{\pgfqpoint{2.711715in}{2.293123in}}%
\pgfpathlineto{\pgfqpoint{2.748043in}{2.293123in}}%
\pgfpathlineto{\pgfqpoint{2.748043in}{2.269530in}}%
\pgfpathmoveto{\pgfqpoint{2.639058in}{2.293123in}}%
\pgfpathlineto{\pgfqpoint{2.639058in}{2.293123in}}%
\pgfpathlineto{\pgfqpoint{2.639058in}{2.316717in}}%
\pgfpathlineto{\pgfqpoint{2.675387in}{2.316717in}}%
\pgfpathlineto{\pgfqpoint{2.675387in}{2.293123in}}%
\pgfpathmoveto{\pgfqpoint{2.639058in}{2.316717in}}%
\pgfpathlineto{\pgfqpoint{2.639058in}{2.316717in}}%
\pgfpathlineto{\pgfqpoint{2.639058in}{2.340311in}}%
\pgfpathlineto{\pgfqpoint{2.675387in}{2.340311in}}%
\pgfpathlineto{\pgfqpoint{2.675387in}{2.316717in}}%
\pgfpathmoveto{\pgfqpoint{2.675387in}{2.293123in}}%
\pgfpathlineto{\pgfqpoint{2.675387in}{2.293123in}}%
\pgfpathlineto{\pgfqpoint{2.675387in}{2.316717in}}%
\pgfpathlineto{\pgfqpoint{2.711715in}{2.316717in}}%
\pgfpathlineto{\pgfqpoint{2.711715in}{2.293123in}}%
\pgfpathmoveto{\pgfqpoint{2.857030in}{2.104374in}}%
\pgfpathlineto{\pgfqpoint{2.857030in}{2.104374in}}%
\pgfpathlineto{\pgfqpoint{2.857030in}{2.127968in}}%
\pgfpathlineto{\pgfqpoint{2.893360in}{2.127968in}}%
\pgfpathlineto{\pgfqpoint{2.893360in}{2.104374in}}%
\pgfpathmoveto{\pgfqpoint{2.857030in}{2.127968in}}%
\pgfpathlineto{\pgfqpoint{2.857030in}{2.127968in}}%
\pgfpathlineto{\pgfqpoint{2.857030in}{2.151562in}}%
\pgfpathlineto{\pgfqpoint{2.893360in}{2.151562in}}%
\pgfpathlineto{\pgfqpoint{2.893360in}{2.127968in}}%
\pgfpathmoveto{\pgfqpoint{2.893360in}{2.104374in}}%
\pgfpathlineto{\pgfqpoint{2.893360in}{2.104374in}}%
\pgfpathlineto{\pgfqpoint{2.893360in}{2.127968in}}%
\pgfpathlineto{\pgfqpoint{2.929689in}{2.127968in}}%
\pgfpathlineto{\pgfqpoint{2.929689in}{2.104374in}}%
\pgfpathmoveto{\pgfqpoint{2.784372in}{2.198749in}}%
\pgfpathlineto{\pgfqpoint{2.784372in}{2.198749in}}%
\pgfpathlineto{\pgfqpoint{2.784372in}{2.222343in}}%
\pgfpathlineto{\pgfqpoint{2.820701in}{2.222343in}}%
\pgfpathlineto{\pgfqpoint{2.820701in}{2.198749in}}%
\pgfpathmoveto{\pgfqpoint{2.929689in}{2.057186in}}%
\pgfpathlineto{\pgfqpoint{2.929689in}{2.057186in}}%
\pgfpathlineto{\pgfqpoint{2.929689in}{2.080780in}}%
\pgfpathlineto{\pgfqpoint{2.966016in}{2.080780in}}%
\pgfpathlineto{\pgfqpoint{2.966016in}{2.057186in}}%
\pgfpathmoveto{\pgfqpoint{2.929689in}{2.080780in}}%
\pgfpathlineto{\pgfqpoint{2.929689in}{2.080780in}}%
\pgfpathlineto{\pgfqpoint{2.929689in}{2.104374in}}%
\pgfpathlineto{\pgfqpoint{2.966016in}{2.104374in}}%
\pgfpathlineto{\pgfqpoint{2.966016in}{2.080780in}}%
\pgfpathmoveto{\pgfqpoint{3.002343in}{2.009998in}}%
\pgfpathlineto{\pgfqpoint{3.002343in}{2.009998in}}%
\pgfpathlineto{\pgfqpoint{3.002343in}{2.033592in}}%
\pgfpathlineto{\pgfqpoint{3.038670in}{2.033592in}}%
\pgfpathlineto{\pgfqpoint{3.038670in}{2.009998in}}%
\pgfpathmoveto{\pgfqpoint{3.147654in}{1.868438in}}%
\pgfpathlineto{\pgfqpoint{3.147654in}{1.868438in}}%
\pgfpathlineto{\pgfqpoint{3.147654in}{1.892033in}}%
\pgfpathlineto{\pgfqpoint{3.183982in}{1.892033in}}%
\pgfpathlineto{\pgfqpoint{3.183982in}{1.868438in}}%
\pgfpathmoveto{\pgfqpoint{3.074997in}{1.915628in}}%
\pgfpathlineto{\pgfqpoint{3.074997in}{1.915628in}}%
\pgfpathlineto{\pgfqpoint{3.074997in}{1.939220in}}%
\pgfpathlineto{\pgfqpoint{3.111325in}{1.939220in}}%
\pgfpathlineto{\pgfqpoint{3.111325in}{1.915628in}}%
\pgfpathmoveto{\pgfqpoint{3.074997in}{1.939220in}}%
\pgfpathlineto{\pgfqpoint{3.074997in}{1.939220in}}%
\pgfpathlineto{\pgfqpoint{3.074997in}{1.962813in}}%
\pgfpathlineto{\pgfqpoint{3.111325in}{1.962813in}}%
\pgfpathlineto{\pgfqpoint{3.111325in}{1.939220in}}%
\pgfpathmoveto{\pgfqpoint{3.111325in}{1.915628in}}%
\pgfpathlineto{\pgfqpoint{3.111325in}{1.915628in}}%
\pgfpathlineto{\pgfqpoint{3.111325in}{1.939220in}}%
\pgfpathlineto{\pgfqpoint{3.147654in}{1.939220in}}%
\pgfpathlineto{\pgfqpoint{3.147654in}{1.915628in}}%
\pgfpathmoveto{\pgfqpoint{3.292969in}{1.726877in}}%
\pgfpathlineto{\pgfqpoint{3.292969in}{1.726877in}}%
\pgfpathlineto{\pgfqpoint{3.292969in}{1.750470in}}%
\pgfpathlineto{\pgfqpoint{3.329298in}{1.750470in}}%
\pgfpathlineto{\pgfqpoint{3.329298in}{1.726877in}}%
\pgfpathmoveto{\pgfqpoint{3.292969in}{1.750470in}}%
\pgfpathlineto{\pgfqpoint{3.292969in}{1.750470in}}%
\pgfpathlineto{\pgfqpoint{3.292969in}{1.774063in}}%
\pgfpathlineto{\pgfqpoint{3.329298in}{1.774063in}}%
\pgfpathlineto{\pgfqpoint{3.329298in}{1.750470in}}%
\pgfpathmoveto{\pgfqpoint{3.329298in}{1.726877in}}%
\pgfpathlineto{\pgfqpoint{3.329298in}{1.726877in}}%
\pgfpathlineto{\pgfqpoint{3.329298in}{1.750470in}}%
\pgfpathlineto{\pgfqpoint{3.365628in}{1.750470in}}%
\pgfpathlineto{\pgfqpoint{3.365628in}{1.726877in}}%
\pgfpathmoveto{\pgfqpoint{3.220311in}{1.821249in}}%
\pgfpathlineto{\pgfqpoint{3.220311in}{1.821249in}}%
\pgfpathlineto{\pgfqpoint{3.220311in}{1.844843in}}%
\pgfpathlineto{\pgfqpoint{3.256640in}{1.844843in}}%
\pgfpathlineto{\pgfqpoint{3.256640in}{1.821249in}}%
\pgfpathmoveto{\pgfqpoint{3.438283in}{0.523592in}}%
\pgfpathlineto{\pgfqpoint{3.438283in}{0.523592in}}%
\pgfpathlineto{\pgfqpoint{3.438283in}{0.547186in}}%
\pgfpathlineto{\pgfqpoint{3.474611in}{0.547186in}}%
\pgfpathlineto{\pgfqpoint{3.474611in}{0.523592in}}%
\pgfpathmoveto{\pgfqpoint{3.365628in}{1.679688in}}%
\pgfpathlineto{\pgfqpoint{3.365628in}{1.679688in}}%
\pgfpathlineto{\pgfqpoint{3.365628in}{1.703283in}}%
\pgfpathlineto{\pgfqpoint{3.401955in}{1.703283in}}%
\pgfpathlineto{\pgfqpoint{3.401955in}{1.679688in}}%
\pgfpathmoveto{\pgfqpoint{3.438283in}{1.632499in}}%
\pgfpathlineto{\pgfqpoint{3.438283in}{1.632499in}}%
\pgfpathlineto{\pgfqpoint{3.438283in}{1.656094in}}%
\pgfpathlineto{\pgfqpoint{3.474611in}{1.656094in}}%
\pgfpathlineto{\pgfqpoint{3.474611in}{1.632499in}}%
\pgfpathmoveto{\pgfqpoint{3.510939in}{0.570780in}}%
\pgfpathlineto{\pgfqpoint{3.510939in}{0.570780in}}%
\pgfpathlineto{\pgfqpoint{3.510939in}{0.594373in}}%
\pgfpathlineto{\pgfqpoint{3.547266in}{0.594373in}}%
\pgfpathlineto{\pgfqpoint{3.547266in}{0.570780in}}%
\pgfpathmoveto{\pgfqpoint{3.510939in}{0.594373in}}%
\pgfpathlineto{\pgfqpoint{3.510939in}{0.594373in}}%
\pgfpathlineto{\pgfqpoint{3.510939in}{0.617968in}}%
\pgfpathlineto{\pgfqpoint{3.547266in}{0.617968in}}%
\pgfpathlineto{\pgfqpoint{3.547266in}{0.594373in}}%
\pgfpathmoveto{\pgfqpoint{3.510939in}{0.617968in}}%
\pgfpathlineto{\pgfqpoint{3.510939in}{0.617968in}}%
\pgfpathlineto{\pgfqpoint{3.510939in}{0.641563in}}%
\pgfpathlineto{\pgfqpoint{3.547266in}{0.641563in}}%
\pgfpathlineto{\pgfqpoint{3.547266in}{0.617968in}}%
\pgfpathmoveto{\pgfqpoint{3.547266in}{0.617968in}}%
\pgfpathlineto{\pgfqpoint{3.547266in}{0.617968in}}%
\pgfpathlineto{\pgfqpoint{3.547266in}{0.641563in}}%
\pgfpathlineto{\pgfqpoint{3.583594in}{0.641563in}}%
\pgfpathlineto{\pgfqpoint{3.583594in}{0.617968in}}%
\pgfpathmoveto{\pgfqpoint{3.583594in}{0.641563in}}%
\pgfpathlineto{\pgfqpoint{3.583594in}{0.641563in}}%
\pgfpathlineto{\pgfqpoint{3.583594in}{0.665158in}}%
\pgfpathlineto{\pgfqpoint{3.619922in}{0.665158in}}%
\pgfpathlineto{\pgfqpoint{3.619922in}{0.641563in}}%
\pgfpathmoveto{\pgfqpoint{3.583594in}{0.665158in}}%
\pgfpathlineto{\pgfqpoint{3.583594in}{0.665158in}}%
\pgfpathlineto{\pgfqpoint{3.583594in}{0.688753in}}%
\pgfpathlineto{\pgfqpoint{3.619922in}{0.688753in}}%
\pgfpathlineto{\pgfqpoint{3.619922in}{0.665158in}}%
\pgfpathmoveto{\pgfqpoint{3.619922in}{0.665158in}}%
\pgfpathlineto{\pgfqpoint{3.619922in}{0.665158in}}%
\pgfpathlineto{\pgfqpoint{3.619922in}{0.688753in}}%
\pgfpathlineto{\pgfqpoint{3.656250in}{0.688753in}}%
\pgfpathlineto{\pgfqpoint{3.656250in}{0.665158in}}%
\pgfpathmoveto{\pgfqpoint{3.583594in}{1.490940in}}%
\pgfpathlineto{\pgfqpoint{3.583594in}{1.490940in}}%
\pgfpathlineto{\pgfqpoint{3.583594in}{1.514534in}}%
\pgfpathlineto{\pgfqpoint{3.619922in}{1.514534in}}%
\pgfpathlineto{\pgfqpoint{3.619922in}{1.490940in}}%
\pgfpathmoveto{\pgfqpoint{3.510939in}{1.538128in}}%
\pgfpathlineto{\pgfqpoint{3.510939in}{1.538128in}}%
\pgfpathlineto{\pgfqpoint{3.510939in}{1.561720in}}%
\pgfpathlineto{\pgfqpoint{3.547266in}{1.561720in}}%
\pgfpathlineto{\pgfqpoint{3.547266in}{1.538128in}}%
\pgfpathmoveto{\pgfqpoint{3.510939in}{1.561720in}}%
\pgfpathlineto{\pgfqpoint{3.510939in}{1.561720in}}%
\pgfpathlineto{\pgfqpoint{3.510939in}{1.585313in}}%
\pgfpathlineto{\pgfqpoint{3.547266in}{1.585313in}}%
\pgfpathlineto{\pgfqpoint{3.547266in}{1.561720in}}%
\pgfpathmoveto{\pgfqpoint{3.547266in}{1.538128in}}%
\pgfpathlineto{\pgfqpoint{3.547266in}{1.538128in}}%
\pgfpathlineto{\pgfqpoint{3.547266in}{1.561720in}}%
\pgfpathlineto{\pgfqpoint{3.583594in}{1.561720in}}%
\pgfpathlineto{\pgfqpoint{3.583594in}{1.538128in}}%
\pgfpathmoveto{\pgfqpoint{3.656250in}{0.688753in}}%
\pgfpathlineto{\pgfqpoint{3.656250in}{0.688753in}}%
\pgfpathlineto{\pgfqpoint{3.656250in}{0.712346in}}%
\pgfpathlineto{\pgfqpoint{3.692579in}{0.712346in}}%
\pgfpathlineto{\pgfqpoint{3.692579in}{0.688753in}}%
\pgfpathmoveto{\pgfqpoint{3.656250in}{0.712346in}}%
\pgfpathlineto{\pgfqpoint{3.656250in}{0.712346in}}%
\pgfpathlineto{\pgfqpoint{3.656250in}{0.735939in}}%
\pgfpathlineto{\pgfqpoint{3.692579in}{0.735939in}}%
\pgfpathlineto{\pgfqpoint{3.692579in}{0.712346in}}%
\pgfpathmoveto{\pgfqpoint{3.692579in}{0.712346in}}%
\pgfpathlineto{\pgfqpoint{3.692579in}{0.712346in}}%
\pgfpathlineto{\pgfqpoint{3.692579in}{0.735939in}}%
\pgfpathlineto{\pgfqpoint{3.728908in}{0.735939in}}%
\pgfpathlineto{\pgfqpoint{3.728908in}{0.712346in}}%
\pgfpathmoveto{\pgfqpoint{3.728908in}{0.759532in}}%
\pgfpathlineto{\pgfqpoint{3.728908in}{0.759532in}}%
\pgfpathlineto{\pgfqpoint{3.728908in}{0.783125in}}%
\pgfpathlineto{\pgfqpoint{3.765237in}{0.783125in}}%
\pgfpathlineto{\pgfqpoint{3.765237in}{0.759532in}}%
\pgfpathmoveto{\pgfqpoint{3.656250in}{1.396562in}}%
\pgfpathlineto{\pgfqpoint{3.656250in}{1.396562in}}%
\pgfpathlineto{\pgfqpoint{3.656250in}{1.420157in}}%
\pgfpathlineto{\pgfqpoint{3.692579in}{1.420157in}}%
\pgfpathlineto{\pgfqpoint{3.692579in}{1.396562in}}%
\pgfpathmoveto{\pgfqpoint{3.656250in}{1.420157in}}%
\pgfpathlineto{\pgfqpoint{3.656250in}{1.420157in}}%
\pgfpathlineto{\pgfqpoint{3.656250in}{1.443752in}}%
\pgfpathlineto{\pgfqpoint{3.692579in}{1.443752in}}%
\pgfpathlineto{\pgfqpoint{3.692579in}{1.420157in}}%
\pgfpathmoveto{\pgfqpoint{3.692579in}{1.396562in}}%
\pgfpathlineto{\pgfqpoint{3.692579in}{1.396562in}}%
\pgfpathlineto{\pgfqpoint{3.692579in}{1.420157in}}%
\pgfpathlineto{\pgfqpoint{3.728908in}{1.420157in}}%
\pgfpathlineto{\pgfqpoint{3.728908in}{1.396562in}}%
\pgfpathmoveto{\pgfqpoint{3.728908in}{1.349373in}}%
\pgfpathlineto{\pgfqpoint{3.728908in}{1.349373in}}%
\pgfpathlineto{\pgfqpoint{3.728908in}{1.372968in}}%
\pgfpathlineto{\pgfqpoint{3.765237in}{1.372968in}}%
\pgfpathlineto{\pgfqpoint{3.765237in}{1.349373in}}%
\pgfpathmoveto{\pgfqpoint{3.728908in}{1.372968in}}%
\pgfpathlineto{\pgfqpoint{3.728908in}{1.372968in}}%
\pgfpathlineto{\pgfqpoint{3.728908in}{1.396562in}}%
\pgfpathlineto{\pgfqpoint{3.765237in}{1.396562in}}%
\pgfpathlineto{\pgfqpoint{3.765237in}{1.372968in}}%
\pgfpathmoveto{\pgfqpoint{3.765237in}{1.349373in}}%
\pgfpathlineto{\pgfqpoint{3.765237in}{1.349373in}}%
\pgfpathlineto{\pgfqpoint{3.765237in}{1.372968in}}%
\pgfpathlineto{\pgfqpoint{3.801566in}{1.372968in}}%
\pgfpathlineto{\pgfqpoint{3.801566in}{1.349373in}}%
\pgfpathmoveto{\pgfqpoint{3.656250in}{1.443752in}}%
\pgfpathlineto{\pgfqpoint{3.656250in}{1.443752in}}%
\pgfpathlineto{\pgfqpoint{3.656250in}{1.467346in}}%
\pgfpathlineto{\pgfqpoint{3.692579in}{1.467346in}}%
\pgfpathlineto{\pgfqpoint{3.692579in}{1.443752in}}%
\pgfpathmoveto{\pgfqpoint{3.801566in}{0.806719in}}%
\pgfpathlineto{\pgfqpoint{3.801566in}{0.806719in}}%
\pgfpathlineto{\pgfqpoint{3.801566in}{0.830313in}}%
\pgfpathlineto{\pgfqpoint{3.837893in}{0.830313in}}%
\pgfpathlineto{\pgfqpoint{3.837893in}{0.806719in}}%
\pgfpathmoveto{\pgfqpoint{3.874220in}{0.853907in}}%
\pgfpathlineto{\pgfqpoint{3.874220in}{0.853907in}}%
\pgfpathlineto{\pgfqpoint{3.874220in}{0.877501in}}%
\pgfpathlineto{\pgfqpoint{3.910546in}{0.877501in}}%
\pgfpathlineto{\pgfqpoint{3.910546in}{0.853907in}}%
\pgfpathmoveto{\pgfqpoint{3.801566in}{1.302186in}}%
\pgfpathlineto{\pgfqpoint{3.801566in}{1.302186in}}%
\pgfpathlineto{\pgfqpoint{3.801566in}{1.325779in}}%
\pgfpathlineto{\pgfqpoint{3.837893in}{1.325779in}}%
\pgfpathlineto{\pgfqpoint{3.837893in}{1.302186in}}%
\pgfpathmoveto{\pgfqpoint{3.801566in}{1.325779in}}%
\pgfpathlineto{\pgfqpoint{3.801566in}{1.325779in}}%
\pgfpathlineto{\pgfqpoint{3.801566in}{1.349373in}}%
\pgfpathlineto{\pgfqpoint{3.837893in}{1.349373in}}%
\pgfpathlineto{\pgfqpoint{3.837893in}{1.325779in}}%
\pgfpathmoveto{\pgfqpoint{3.874220in}{1.254999in}}%
\pgfpathlineto{\pgfqpoint{3.874220in}{1.254999in}}%
\pgfpathlineto{\pgfqpoint{3.874220in}{1.278593in}}%
\pgfpathlineto{\pgfqpoint{3.910546in}{1.278593in}}%
\pgfpathlineto{\pgfqpoint{3.910546in}{1.254999in}}%
\pgfpathmoveto{\pgfqpoint{3.946873in}{0.924687in}}%
\pgfpathlineto{\pgfqpoint{3.946873in}{0.924687in}}%
\pgfpathlineto{\pgfqpoint{3.946873in}{0.948280in}}%
\pgfpathlineto{\pgfqpoint{3.983201in}{0.948280in}}%
\pgfpathlineto{\pgfqpoint{3.983201in}{0.924687in}}%
\pgfpathmoveto{\pgfqpoint{3.946873in}{0.948280in}}%
\pgfpathlineto{\pgfqpoint{3.946873in}{0.948280in}}%
\pgfpathlineto{\pgfqpoint{3.946873in}{0.971874in}}%
\pgfpathlineto{\pgfqpoint{3.983201in}{0.971874in}}%
\pgfpathlineto{\pgfqpoint{3.983201in}{0.948280in}}%
\pgfpathmoveto{\pgfqpoint{3.983201in}{0.948280in}}%
\pgfpathlineto{\pgfqpoint{3.983201in}{0.948280in}}%
\pgfpathlineto{\pgfqpoint{3.983201in}{0.971874in}}%
\pgfpathlineto{\pgfqpoint{4.019529in}{0.971874in}}%
\pgfpathlineto{\pgfqpoint{4.019529in}{0.948280in}}%
\pgfpathmoveto{\pgfqpoint{4.019529in}{0.971874in}}%
\pgfpathlineto{\pgfqpoint{4.019529in}{0.971874in}}%
\pgfpathlineto{\pgfqpoint{4.019529in}{0.995467in}}%
\pgfpathlineto{\pgfqpoint{4.055856in}{0.995467in}}%
\pgfpathlineto{\pgfqpoint{4.055856in}{0.971874in}}%
\pgfpathmoveto{\pgfqpoint{4.019529in}{0.995467in}}%
\pgfpathlineto{\pgfqpoint{4.019529in}{0.995467in}}%
\pgfpathlineto{\pgfqpoint{4.019529in}{1.019060in}}%
\pgfpathlineto{\pgfqpoint{4.055856in}{1.019060in}}%
\pgfpathlineto{\pgfqpoint{4.055856in}{0.995467in}}%
\pgfpathmoveto{\pgfqpoint{4.055856in}{0.995467in}}%
\pgfpathlineto{\pgfqpoint{4.055856in}{0.995467in}}%
\pgfpathlineto{\pgfqpoint{4.055856in}{1.019060in}}%
\pgfpathlineto{\pgfqpoint{4.092184in}{1.019060in}}%
\pgfpathlineto{\pgfqpoint{4.092184in}{0.995467in}}%
\pgfpathmoveto{\pgfqpoint{4.019529in}{1.113436in}}%
\pgfpathlineto{\pgfqpoint{4.019529in}{1.113436in}}%
\pgfpathlineto{\pgfqpoint{4.019529in}{1.137030in}}%
\pgfpathlineto{\pgfqpoint{4.055856in}{1.137030in}}%
\pgfpathlineto{\pgfqpoint{4.055856in}{1.113436in}}%
\pgfpathmoveto{\pgfqpoint{4.019529in}{1.137030in}}%
\pgfpathlineto{\pgfqpoint{4.019529in}{1.137030in}}%
\pgfpathlineto{\pgfqpoint{4.019529in}{1.160624in}}%
\pgfpathlineto{\pgfqpoint{4.055856in}{1.160624in}}%
\pgfpathlineto{\pgfqpoint{4.055856in}{1.137030in}}%
\pgfpathmoveto{\pgfqpoint{3.946873in}{1.160624in}}%
\pgfpathlineto{\pgfqpoint{3.946873in}{1.160624in}}%
\pgfpathlineto{\pgfqpoint{3.946873in}{1.184218in}}%
\pgfpathlineto{\pgfqpoint{3.983201in}{1.184218in}}%
\pgfpathlineto{\pgfqpoint{3.983201in}{1.160624in}}%
\pgfpathmoveto{\pgfqpoint{3.946873in}{1.184218in}}%
\pgfpathlineto{\pgfqpoint{3.946873in}{1.184218in}}%
\pgfpathlineto{\pgfqpoint{3.946873in}{1.207812in}}%
\pgfpathlineto{\pgfqpoint{3.983201in}{1.207812in}}%
\pgfpathlineto{\pgfqpoint{3.983201in}{1.184218in}}%
\pgfpathmoveto{\pgfqpoint{3.983201in}{1.160624in}}%
\pgfpathlineto{\pgfqpoint{3.983201in}{1.160624in}}%
\pgfpathlineto{\pgfqpoint{3.983201in}{1.184218in}}%
\pgfpathlineto{\pgfqpoint{4.019529in}{1.184218in}}%
\pgfpathlineto{\pgfqpoint{4.019529in}{1.160624in}}%
\pgfpathmoveto{\pgfqpoint{4.092184in}{1.019060in}}%
\pgfpathlineto{\pgfqpoint{4.092184in}{1.019060in}}%
\pgfpathlineto{\pgfqpoint{4.092184in}{1.042654in}}%
\pgfpathlineto{\pgfqpoint{4.128512in}{1.042654in}}%
\pgfpathlineto{\pgfqpoint{4.128512in}{1.019060in}}%
\pgfpathmoveto{\pgfqpoint{4.092184in}{1.042654in}}%
\pgfpathlineto{\pgfqpoint{4.092184in}{1.042654in}}%
\pgfpathlineto{\pgfqpoint{4.092184in}{1.066247in}}%
\pgfpathlineto{\pgfqpoint{4.128512in}{1.066247in}}%
\pgfpathlineto{\pgfqpoint{4.128512in}{1.042654in}}%
\pgfpathmoveto{\pgfqpoint{4.092184in}{1.066247in}}%
\pgfpathlineto{\pgfqpoint{4.092184in}{1.066247in}}%
\pgfpathlineto{\pgfqpoint{4.092184in}{1.089842in}}%
\pgfpathlineto{\pgfqpoint{4.128512in}{1.089842in}}%
\pgfpathlineto{\pgfqpoint{4.128512in}{1.066247in}}%
\pgfpathmoveto{\pgfqpoint{1.440233in}{3.378436in}}%
\pgfpathlineto{\pgfqpoint{1.440233in}{3.378436in}}%
\pgfpathlineto{\pgfqpoint{1.440233in}{3.390233in}}%
\pgfpathlineto{\pgfqpoint{1.458397in}{3.390233in}}%
\pgfpathlineto{\pgfqpoint{1.458397in}{3.378436in}}%
\pgfpathmoveto{\pgfqpoint{1.440233in}{3.390233in}}%
\pgfpathlineto{\pgfqpoint{1.440233in}{3.390233in}}%
\pgfpathlineto{\pgfqpoint{1.440233in}{3.402029in}}%
\pgfpathlineto{\pgfqpoint{1.458397in}{3.402029in}}%
\pgfpathlineto{\pgfqpoint{1.458397in}{3.390233in}}%
\pgfpathmoveto{\pgfqpoint{1.458397in}{3.378436in}}%
\pgfpathlineto{\pgfqpoint{1.458397in}{3.378436in}}%
\pgfpathlineto{\pgfqpoint{1.458397in}{3.390233in}}%
\pgfpathlineto{\pgfqpoint{1.476560in}{3.390233in}}%
\pgfpathlineto{\pgfqpoint{1.476560in}{3.378436in}}%
\pgfpathmoveto{\pgfqpoint{1.367579in}{3.449217in}}%
\pgfpathlineto{\pgfqpoint{1.367579in}{3.449217in}}%
\pgfpathlineto{\pgfqpoint{1.367579in}{3.461014in}}%
\pgfpathlineto{\pgfqpoint{1.385743in}{3.461014in}}%
\pgfpathlineto{\pgfqpoint{1.385743in}{3.449217in}}%
\pgfpathmoveto{\pgfqpoint{1.367579in}{3.461014in}}%
\pgfpathlineto{\pgfqpoint{1.367579in}{3.461014in}}%
\pgfpathlineto{\pgfqpoint{1.367579in}{3.472810in}}%
\pgfpathlineto{\pgfqpoint{1.385743in}{3.472810in}}%
\pgfpathlineto{\pgfqpoint{1.385743in}{3.461014in}}%
\pgfpathmoveto{\pgfqpoint{1.331252in}{3.472810in}}%
\pgfpathlineto{\pgfqpoint{1.331252in}{3.472810in}}%
\pgfpathlineto{\pgfqpoint{1.331252in}{3.484607in}}%
\pgfpathlineto{\pgfqpoint{1.349416in}{3.484607in}}%
\pgfpathlineto{\pgfqpoint{1.349416in}{3.472810in}}%
\pgfpathmoveto{\pgfqpoint{1.331252in}{3.484607in}}%
\pgfpathlineto{\pgfqpoint{1.331252in}{3.484607in}}%
\pgfpathlineto{\pgfqpoint{1.331252in}{3.496404in}}%
\pgfpathlineto{\pgfqpoint{1.349416in}{3.496404in}}%
\pgfpathlineto{\pgfqpoint{1.349416in}{3.484607in}}%
\pgfpathmoveto{\pgfqpoint{1.349416in}{3.472810in}}%
\pgfpathlineto{\pgfqpoint{1.349416in}{3.472810in}}%
\pgfpathlineto{\pgfqpoint{1.349416in}{3.484607in}}%
\pgfpathlineto{\pgfqpoint{1.367579in}{3.484607in}}%
\pgfpathlineto{\pgfqpoint{1.367579in}{3.472810in}}%
\pgfpathmoveto{\pgfqpoint{1.403906in}{3.425623in}}%
\pgfpathlineto{\pgfqpoint{1.403906in}{3.425623in}}%
\pgfpathlineto{\pgfqpoint{1.403906in}{3.437420in}}%
\pgfpathlineto{\pgfqpoint{1.422070in}{3.437420in}}%
\pgfpathlineto{\pgfqpoint{1.422070in}{3.425623in}}%
\pgfpathmoveto{\pgfqpoint{1.585549in}{3.260469in}}%
\pgfpathlineto{\pgfqpoint{1.585549in}{3.260469in}}%
\pgfpathlineto{\pgfqpoint{1.585549in}{3.272266in}}%
\pgfpathlineto{\pgfqpoint{1.603714in}{3.272266in}}%
\pgfpathlineto{\pgfqpoint{1.603714in}{3.260469in}}%
\pgfpathmoveto{\pgfqpoint{1.549220in}{3.284062in}}%
\pgfpathlineto{\pgfqpoint{1.549220in}{3.284062in}}%
\pgfpathlineto{\pgfqpoint{1.549220in}{3.295859in}}%
\pgfpathlineto{\pgfqpoint{1.567384in}{3.295859in}}%
\pgfpathlineto{\pgfqpoint{1.567384in}{3.284062in}}%
\pgfpathmoveto{\pgfqpoint{1.549220in}{3.295859in}}%
\pgfpathlineto{\pgfqpoint{1.549220in}{3.295859in}}%
\pgfpathlineto{\pgfqpoint{1.549220in}{3.307655in}}%
\pgfpathlineto{\pgfqpoint{1.567384in}{3.307655in}}%
\pgfpathlineto{\pgfqpoint{1.567384in}{3.295859in}}%
\pgfpathmoveto{\pgfqpoint{1.567384in}{3.284062in}}%
\pgfpathlineto{\pgfqpoint{1.567384in}{3.284062in}}%
\pgfpathlineto{\pgfqpoint{1.567384in}{3.295859in}}%
\pgfpathlineto{\pgfqpoint{1.585549in}{3.295859in}}%
\pgfpathlineto{\pgfqpoint{1.585549in}{3.284062in}}%
\pgfpathmoveto{\pgfqpoint{1.476560in}{3.354842in}}%
\pgfpathlineto{\pgfqpoint{1.476560in}{3.354842in}}%
\pgfpathlineto{\pgfqpoint{1.476560in}{3.366639in}}%
\pgfpathlineto{\pgfqpoint{1.494725in}{3.366639in}}%
\pgfpathlineto{\pgfqpoint{1.494725in}{3.354842in}}%
\pgfpathmoveto{\pgfqpoint{1.476560in}{3.366639in}}%
\pgfpathlineto{\pgfqpoint{1.476560in}{3.366639in}}%
\pgfpathlineto{\pgfqpoint{1.476560in}{3.378436in}}%
\pgfpathlineto{\pgfqpoint{1.494725in}{3.378436in}}%
\pgfpathlineto{\pgfqpoint{1.494725in}{3.366639in}}%
\pgfpathmoveto{\pgfqpoint{1.512890in}{3.331248in}}%
\pgfpathlineto{\pgfqpoint{1.512890in}{3.331248in}}%
\pgfpathlineto{\pgfqpoint{1.512890in}{3.343045in}}%
\pgfpathlineto{\pgfqpoint{1.531055in}{3.343045in}}%
\pgfpathlineto{\pgfqpoint{1.531055in}{3.331248in}}%
\pgfpathmoveto{\pgfqpoint{1.621879in}{3.213282in}}%
\pgfpathlineto{\pgfqpoint{1.621879in}{3.213282in}}%
\pgfpathlineto{\pgfqpoint{1.621879in}{3.225079in}}%
\pgfpathlineto{\pgfqpoint{1.640043in}{3.225079in}}%
\pgfpathlineto{\pgfqpoint{1.640043in}{3.213282in}}%
\pgfpathmoveto{\pgfqpoint{1.621879in}{3.225079in}}%
\pgfpathlineto{\pgfqpoint{1.621879in}{3.225079in}}%
\pgfpathlineto{\pgfqpoint{1.621879in}{3.236876in}}%
\pgfpathlineto{\pgfqpoint{1.640043in}{3.236876in}}%
\pgfpathlineto{\pgfqpoint{1.640043in}{3.225079in}}%
\pgfpathmoveto{\pgfqpoint{1.640043in}{3.213282in}}%
\pgfpathlineto{\pgfqpoint{1.640043in}{3.213282in}}%
\pgfpathlineto{\pgfqpoint{1.640043in}{3.225079in}}%
\pgfpathlineto{\pgfqpoint{1.658206in}{3.225079in}}%
\pgfpathlineto{\pgfqpoint{1.658206in}{3.213282in}}%
\pgfpathmoveto{\pgfqpoint{1.658206in}{3.189687in}}%
\pgfpathlineto{\pgfqpoint{1.658206in}{3.189687in}}%
\pgfpathlineto{\pgfqpoint{1.658206in}{3.201485in}}%
\pgfpathlineto{\pgfqpoint{1.676370in}{3.201485in}}%
\pgfpathlineto{\pgfqpoint{1.676370in}{3.189687in}}%
\pgfpathmoveto{\pgfqpoint{1.658206in}{3.201485in}}%
\pgfpathlineto{\pgfqpoint{1.658206in}{3.201485in}}%
\pgfpathlineto{\pgfqpoint{1.658206in}{3.213282in}}%
\pgfpathlineto{\pgfqpoint{1.676370in}{3.213282in}}%
\pgfpathlineto{\pgfqpoint{1.676370in}{3.201485in}}%
\pgfpathmoveto{\pgfqpoint{1.676370in}{3.189687in}}%
\pgfpathlineto{\pgfqpoint{1.676370in}{3.189687in}}%
\pgfpathlineto{\pgfqpoint{1.676370in}{3.201485in}}%
\pgfpathlineto{\pgfqpoint{1.694534in}{3.201485in}}%
\pgfpathlineto{\pgfqpoint{1.694534in}{3.189687in}}%
\pgfpathmoveto{\pgfqpoint{1.694534in}{3.166093in}}%
\pgfpathlineto{\pgfqpoint{1.694534in}{3.166093in}}%
\pgfpathlineto{\pgfqpoint{1.694534in}{3.177890in}}%
\pgfpathlineto{\pgfqpoint{1.712698in}{3.177890in}}%
\pgfpathlineto{\pgfqpoint{1.712698in}{3.166093in}}%
\pgfpathmoveto{\pgfqpoint{1.730861in}{3.142498in}}%
\pgfpathlineto{\pgfqpoint{1.730861in}{3.142498in}}%
\pgfpathlineto{\pgfqpoint{1.730861in}{3.154296in}}%
\pgfpathlineto{\pgfqpoint{1.749025in}{3.154296in}}%
\pgfpathlineto{\pgfqpoint{1.749025in}{3.142498in}}%
\pgfpathmoveto{\pgfqpoint{1.621879in}{3.236876in}}%
\pgfpathlineto{\pgfqpoint{1.621879in}{3.236876in}}%
\pgfpathlineto{\pgfqpoint{1.621879in}{3.248673in}}%
\pgfpathlineto{\pgfqpoint{1.640043in}{3.248673in}}%
\pgfpathlineto{\pgfqpoint{1.640043in}{3.236876in}}%
\pgfpathmoveto{\pgfqpoint{1.876171in}{3.000937in}}%
\pgfpathlineto{\pgfqpoint{1.876171in}{3.000937in}}%
\pgfpathlineto{\pgfqpoint{1.876171in}{3.012733in}}%
\pgfpathlineto{\pgfqpoint{1.894335in}{3.012733in}}%
\pgfpathlineto{\pgfqpoint{1.894335in}{3.000937in}}%
\pgfpathmoveto{\pgfqpoint{1.876171in}{3.012733in}}%
\pgfpathlineto{\pgfqpoint{1.876171in}{3.012733in}}%
\pgfpathlineto{\pgfqpoint{1.876171in}{3.024530in}}%
\pgfpathlineto{\pgfqpoint{1.894335in}{3.024530in}}%
\pgfpathlineto{\pgfqpoint{1.894335in}{3.012733in}}%
\pgfpathmoveto{\pgfqpoint{1.894335in}{3.000937in}}%
\pgfpathlineto{\pgfqpoint{1.894335in}{3.000937in}}%
\pgfpathlineto{\pgfqpoint{1.894335in}{3.012733in}}%
\pgfpathlineto{\pgfqpoint{1.912499in}{3.012733in}}%
\pgfpathlineto{\pgfqpoint{1.912499in}{3.000937in}}%
\pgfpathmoveto{\pgfqpoint{1.803516in}{3.071718in}}%
\pgfpathlineto{\pgfqpoint{1.803516in}{3.071718in}}%
\pgfpathlineto{\pgfqpoint{1.803516in}{3.083514in}}%
\pgfpathlineto{\pgfqpoint{1.821680in}{3.083514in}}%
\pgfpathlineto{\pgfqpoint{1.821680in}{3.071718in}}%
\pgfpathmoveto{\pgfqpoint{1.803516in}{3.083514in}}%
\pgfpathlineto{\pgfqpoint{1.803516in}{3.083514in}}%
\pgfpathlineto{\pgfqpoint{1.803516in}{3.095311in}}%
\pgfpathlineto{\pgfqpoint{1.821680in}{3.095311in}}%
\pgfpathlineto{\pgfqpoint{1.821680in}{3.083514in}}%
\pgfpathmoveto{\pgfqpoint{1.767189in}{3.095311in}}%
\pgfpathlineto{\pgfqpoint{1.767189in}{3.095311in}}%
\pgfpathlineto{\pgfqpoint{1.767189in}{3.107108in}}%
\pgfpathlineto{\pgfqpoint{1.785353in}{3.107108in}}%
\pgfpathlineto{\pgfqpoint{1.785353in}{3.095311in}}%
\pgfpathmoveto{\pgfqpoint{1.767189in}{3.107108in}}%
\pgfpathlineto{\pgfqpoint{1.767189in}{3.107108in}}%
\pgfpathlineto{\pgfqpoint{1.767189in}{3.118905in}}%
\pgfpathlineto{\pgfqpoint{1.785353in}{3.118905in}}%
\pgfpathlineto{\pgfqpoint{1.785353in}{3.107108in}}%
\pgfpathmoveto{\pgfqpoint{1.785353in}{3.095311in}}%
\pgfpathlineto{\pgfqpoint{1.785353in}{3.095311in}}%
\pgfpathlineto{\pgfqpoint{1.785353in}{3.107108in}}%
\pgfpathlineto{\pgfqpoint{1.803516in}{3.107108in}}%
\pgfpathlineto{\pgfqpoint{1.803516in}{3.095311in}}%
\pgfpathmoveto{\pgfqpoint{1.839844in}{3.048124in}}%
\pgfpathlineto{\pgfqpoint{1.839844in}{3.048124in}}%
\pgfpathlineto{\pgfqpoint{1.839844in}{3.059921in}}%
\pgfpathlineto{\pgfqpoint{1.858008in}{3.059921in}}%
\pgfpathlineto{\pgfqpoint{1.858008in}{3.048124in}}%
\pgfpathmoveto{\pgfqpoint{1.948828in}{2.930155in}}%
\pgfpathlineto{\pgfqpoint{1.948828in}{2.930155in}}%
\pgfpathlineto{\pgfqpoint{1.948828in}{2.941952in}}%
\pgfpathlineto{\pgfqpoint{1.966993in}{2.941952in}}%
\pgfpathlineto{\pgfqpoint{1.966993in}{2.930155in}}%
\pgfpathmoveto{\pgfqpoint{1.948828in}{2.941952in}}%
\pgfpathlineto{\pgfqpoint{1.948828in}{2.941952in}}%
\pgfpathlineto{\pgfqpoint{1.948828in}{2.953749in}}%
\pgfpathlineto{\pgfqpoint{1.966993in}{2.953749in}}%
\pgfpathlineto{\pgfqpoint{1.966993in}{2.941952in}}%
\pgfpathmoveto{\pgfqpoint{1.966993in}{2.930155in}}%
\pgfpathlineto{\pgfqpoint{1.966993in}{2.930155in}}%
\pgfpathlineto{\pgfqpoint{1.966993in}{2.941952in}}%
\pgfpathlineto{\pgfqpoint{1.985158in}{2.941952in}}%
\pgfpathlineto{\pgfqpoint{1.985158in}{2.930155in}}%
\pgfpathmoveto{\pgfqpoint{2.021487in}{2.882966in}}%
\pgfpathlineto{\pgfqpoint{2.021487in}{2.882966in}}%
\pgfpathlineto{\pgfqpoint{2.021487in}{2.894764in}}%
\pgfpathlineto{\pgfqpoint{2.039652in}{2.894764in}}%
\pgfpathlineto{\pgfqpoint{2.039652in}{2.882966in}}%
\pgfpathmoveto{\pgfqpoint{1.985158in}{2.906561in}}%
\pgfpathlineto{\pgfqpoint{1.985158in}{2.906561in}}%
\pgfpathlineto{\pgfqpoint{1.985158in}{2.918358in}}%
\pgfpathlineto{\pgfqpoint{2.003322in}{2.918358in}}%
\pgfpathlineto{\pgfqpoint{2.003322in}{2.906561in}}%
\pgfpathmoveto{\pgfqpoint{1.985158in}{2.918358in}}%
\pgfpathlineto{\pgfqpoint{1.985158in}{2.918358in}}%
\pgfpathlineto{\pgfqpoint{1.985158in}{2.930155in}}%
\pgfpathlineto{\pgfqpoint{2.003322in}{2.930155in}}%
\pgfpathlineto{\pgfqpoint{2.003322in}{2.918358in}}%
\pgfpathmoveto{\pgfqpoint{2.003322in}{2.906561in}}%
\pgfpathlineto{\pgfqpoint{2.003322in}{2.906561in}}%
\pgfpathlineto{\pgfqpoint{2.003322in}{2.918358in}}%
\pgfpathlineto{\pgfqpoint{2.021487in}{2.918358in}}%
\pgfpathlineto{\pgfqpoint{2.021487in}{2.906561in}}%
\pgfpathmoveto{\pgfqpoint{1.912499in}{2.977343in}}%
\pgfpathlineto{\pgfqpoint{1.912499in}{2.977343in}}%
\pgfpathlineto{\pgfqpoint{1.912499in}{2.989140in}}%
\pgfpathlineto{\pgfqpoint{1.930663in}{2.989140in}}%
\pgfpathlineto{\pgfqpoint{1.930663in}{2.977343in}}%
\pgfpathmoveto{\pgfqpoint{1.912499in}{2.989140in}}%
\pgfpathlineto{\pgfqpoint{1.912499in}{2.989140in}}%
\pgfpathlineto{\pgfqpoint{1.912499in}{3.000937in}}%
\pgfpathlineto{\pgfqpoint{1.930663in}{3.000937in}}%
\pgfpathlineto{\pgfqpoint{1.930663in}{2.989140in}}%
\pgfpathmoveto{\pgfqpoint{1.948828in}{2.953749in}}%
\pgfpathlineto{\pgfqpoint{1.948828in}{2.953749in}}%
\pgfpathlineto{\pgfqpoint{1.948828in}{2.965546in}}%
\pgfpathlineto{\pgfqpoint{1.966993in}{2.965546in}}%
\pgfpathlineto{\pgfqpoint{1.966993in}{2.953749in}}%
\pgfpathmoveto{\pgfqpoint{2.094143in}{2.812188in}}%
\pgfpathlineto{\pgfqpoint{2.094143in}{2.812188in}}%
\pgfpathlineto{\pgfqpoint{2.094143in}{2.823984in}}%
\pgfpathlineto{\pgfqpoint{2.112306in}{2.823984in}}%
\pgfpathlineto{\pgfqpoint{2.112306in}{2.812188in}}%
\pgfpathmoveto{\pgfqpoint{2.094143in}{2.823984in}}%
\pgfpathlineto{\pgfqpoint{2.094143in}{2.823984in}}%
\pgfpathlineto{\pgfqpoint{2.094143in}{2.835780in}}%
\pgfpathlineto{\pgfqpoint{2.112306in}{2.835780in}}%
\pgfpathlineto{\pgfqpoint{2.112306in}{2.823984in}}%
\pgfpathmoveto{\pgfqpoint{2.112306in}{2.812188in}}%
\pgfpathlineto{\pgfqpoint{2.112306in}{2.812188in}}%
\pgfpathlineto{\pgfqpoint{2.112306in}{2.823984in}}%
\pgfpathlineto{\pgfqpoint{2.130469in}{2.823984in}}%
\pgfpathlineto{\pgfqpoint{2.130469in}{2.812188in}}%
\pgfpathmoveto{\pgfqpoint{2.130469in}{2.788595in}}%
\pgfpathlineto{\pgfqpoint{2.130469in}{2.788595in}}%
\pgfpathlineto{\pgfqpoint{2.130469in}{2.800391in}}%
\pgfpathlineto{\pgfqpoint{2.148632in}{2.800391in}}%
\pgfpathlineto{\pgfqpoint{2.148632in}{2.788595in}}%
\pgfpathmoveto{\pgfqpoint{2.130469in}{2.800391in}}%
\pgfpathlineto{\pgfqpoint{2.130469in}{2.800391in}}%
\pgfpathlineto{\pgfqpoint{2.130469in}{2.812188in}}%
\pgfpathlineto{\pgfqpoint{2.148632in}{2.812188in}}%
\pgfpathlineto{\pgfqpoint{2.148632in}{2.800391in}}%
\pgfpathmoveto{\pgfqpoint{2.166795in}{2.765003in}}%
\pgfpathlineto{\pgfqpoint{2.166795in}{2.765003in}}%
\pgfpathlineto{\pgfqpoint{2.166795in}{2.776799in}}%
\pgfpathlineto{\pgfqpoint{2.184958in}{2.776799in}}%
\pgfpathlineto{\pgfqpoint{2.184958in}{2.765003in}}%
\pgfpathmoveto{\pgfqpoint{2.057816in}{2.859372in}}%
\pgfpathlineto{\pgfqpoint{2.057816in}{2.859372in}}%
\pgfpathlineto{\pgfqpoint{2.057816in}{2.871169in}}%
\pgfpathlineto{\pgfqpoint{2.075979in}{2.871169in}}%
\pgfpathlineto{\pgfqpoint{2.075979in}{2.859372in}}%
\pgfpathmoveto{\pgfqpoint{2.312110in}{2.623436in}}%
\pgfpathlineto{\pgfqpoint{2.312110in}{2.623436in}}%
\pgfpathlineto{\pgfqpoint{2.312110in}{2.635233in}}%
\pgfpathlineto{\pgfqpoint{2.330274in}{2.635233in}}%
\pgfpathlineto{\pgfqpoint{2.330274in}{2.623436in}}%
\pgfpathmoveto{\pgfqpoint{2.312110in}{2.635233in}}%
\pgfpathlineto{\pgfqpoint{2.312110in}{2.635233in}}%
\pgfpathlineto{\pgfqpoint{2.312110in}{2.647030in}}%
\pgfpathlineto{\pgfqpoint{2.330274in}{2.647030in}}%
\pgfpathlineto{\pgfqpoint{2.330274in}{2.635233in}}%
\pgfpathmoveto{\pgfqpoint{2.330274in}{2.623436in}}%
\pgfpathlineto{\pgfqpoint{2.330274in}{2.623436in}}%
\pgfpathlineto{\pgfqpoint{2.330274in}{2.635233in}}%
\pgfpathlineto{\pgfqpoint{2.348439in}{2.635233in}}%
\pgfpathlineto{\pgfqpoint{2.348439in}{2.623436in}}%
\pgfpathmoveto{\pgfqpoint{2.239451in}{2.694218in}}%
\pgfpathlineto{\pgfqpoint{2.239451in}{2.694218in}}%
\pgfpathlineto{\pgfqpoint{2.239451in}{2.706016in}}%
\pgfpathlineto{\pgfqpoint{2.257616in}{2.706016in}}%
\pgfpathlineto{\pgfqpoint{2.257616in}{2.694218in}}%
\pgfpathmoveto{\pgfqpoint{2.239451in}{2.706016in}}%
\pgfpathlineto{\pgfqpoint{2.239451in}{2.706016in}}%
\pgfpathlineto{\pgfqpoint{2.239451in}{2.717813in}}%
\pgfpathlineto{\pgfqpoint{2.257616in}{2.717813in}}%
\pgfpathlineto{\pgfqpoint{2.257616in}{2.706016in}}%
\pgfpathmoveto{\pgfqpoint{2.203122in}{2.717813in}}%
\pgfpathlineto{\pgfqpoint{2.203122in}{2.717813in}}%
\pgfpathlineto{\pgfqpoint{2.203122in}{2.729611in}}%
\pgfpathlineto{\pgfqpoint{2.221286in}{2.729611in}}%
\pgfpathlineto{\pgfqpoint{2.221286in}{2.717813in}}%
\pgfpathmoveto{\pgfqpoint{2.203122in}{2.729611in}}%
\pgfpathlineto{\pgfqpoint{2.203122in}{2.729611in}}%
\pgfpathlineto{\pgfqpoint{2.203122in}{2.741408in}}%
\pgfpathlineto{\pgfqpoint{2.221286in}{2.741408in}}%
\pgfpathlineto{\pgfqpoint{2.221286in}{2.729611in}}%
\pgfpathmoveto{\pgfqpoint{2.221286in}{2.717813in}}%
\pgfpathlineto{\pgfqpoint{2.221286in}{2.717813in}}%
\pgfpathlineto{\pgfqpoint{2.221286in}{2.729611in}}%
\pgfpathlineto{\pgfqpoint{2.239451in}{2.729611in}}%
\pgfpathlineto{\pgfqpoint{2.239451in}{2.717813in}}%
\pgfpathmoveto{\pgfqpoint{2.275780in}{2.670624in}}%
\pgfpathlineto{\pgfqpoint{2.275780in}{2.670624in}}%
\pgfpathlineto{\pgfqpoint{2.275780in}{2.682421in}}%
\pgfpathlineto{\pgfqpoint{2.293945in}{2.682421in}}%
\pgfpathlineto{\pgfqpoint{2.293945in}{2.670624in}}%
\pgfpathmoveto{\pgfqpoint{2.457419in}{2.505469in}}%
\pgfpathlineto{\pgfqpoint{2.457419in}{2.505469in}}%
\pgfpathlineto{\pgfqpoint{2.457419in}{2.517266in}}%
\pgfpathlineto{\pgfqpoint{2.475583in}{2.517266in}}%
\pgfpathlineto{\pgfqpoint{2.475583in}{2.505469in}}%
\pgfpathmoveto{\pgfqpoint{2.457419in}{2.517266in}}%
\pgfpathlineto{\pgfqpoint{2.457419in}{2.517266in}}%
\pgfpathlineto{\pgfqpoint{2.457419in}{2.529062in}}%
\pgfpathlineto{\pgfqpoint{2.475583in}{2.529062in}}%
\pgfpathlineto{\pgfqpoint{2.475583in}{2.517266in}}%
\pgfpathmoveto{\pgfqpoint{2.421093in}{2.529062in}}%
\pgfpathlineto{\pgfqpoint{2.421093in}{2.529062in}}%
\pgfpathlineto{\pgfqpoint{2.421093in}{2.540859in}}%
\pgfpathlineto{\pgfqpoint{2.439256in}{2.540859in}}%
\pgfpathlineto{\pgfqpoint{2.439256in}{2.529062in}}%
\pgfpathmoveto{\pgfqpoint{2.421093in}{2.540859in}}%
\pgfpathlineto{\pgfqpoint{2.421093in}{2.540859in}}%
\pgfpathlineto{\pgfqpoint{2.421093in}{2.552656in}}%
\pgfpathlineto{\pgfqpoint{2.439256in}{2.552656in}}%
\pgfpathlineto{\pgfqpoint{2.439256in}{2.540859in}}%
\pgfpathmoveto{\pgfqpoint{2.439256in}{2.529062in}}%
\pgfpathlineto{\pgfqpoint{2.439256in}{2.529062in}}%
\pgfpathlineto{\pgfqpoint{2.439256in}{2.540859in}}%
\pgfpathlineto{\pgfqpoint{2.457419in}{2.540859in}}%
\pgfpathlineto{\pgfqpoint{2.457419in}{2.529062in}}%
\pgfpathmoveto{\pgfqpoint{2.348439in}{2.599843in}}%
\pgfpathlineto{\pgfqpoint{2.348439in}{2.599843in}}%
\pgfpathlineto{\pgfqpoint{2.348439in}{2.611640in}}%
\pgfpathlineto{\pgfqpoint{2.366602in}{2.611640in}}%
\pgfpathlineto{\pgfqpoint{2.366602in}{2.599843in}}%
\pgfpathmoveto{\pgfqpoint{2.348439in}{2.611640in}}%
\pgfpathlineto{\pgfqpoint{2.348439in}{2.611640in}}%
\pgfpathlineto{\pgfqpoint{2.348439in}{2.623436in}}%
\pgfpathlineto{\pgfqpoint{2.366602in}{2.623436in}}%
\pgfpathlineto{\pgfqpoint{2.366602in}{2.611640in}}%
\pgfpathmoveto{\pgfqpoint{2.384766in}{2.576249in}}%
\pgfpathlineto{\pgfqpoint{2.384766in}{2.576249in}}%
\pgfpathlineto{\pgfqpoint{2.384766in}{2.588046in}}%
\pgfpathlineto{\pgfqpoint{2.402929in}{2.588046in}}%
\pgfpathlineto{\pgfqpoint{2.402929in}{2.576249in}}%
\pgfpathmoveto{\pgfqpoint{2.530074in}{2.434687in}}%
\pgfpathlineto{\pgfqpoint{2.530074in}{2.434687in}}%
\pgfpathlineto{\pgfqpoint{2.530074in}{2.446484in}}%
\pgfpathlineto{\pgfqpoint{2.548238in}{2.446484in}}%
\pgfpathlineto{\pgfqpoint{2.548238in}{2.434687in}}%
\pgfpathmoveto{\pgfqpoint{2.530074in}{2.446484in}}%
\pgfpathlineto{\pgfqpoint{2.530074in}{2.446484in}}%
\pgfpathlineto{\pgfqpoint{2.530074in}{2.458281in}}%
\pgfpathlineto{\pgfqpoint{2.548238in}{2.458281in}}%
\pgfpathlineto{\pgfqpoint{2.548238in}{2.446484in}}%
\pgfpathmoveto{\pgfqpoint{2.548238in}{2.434687in}}%
\pgfpathlineto{\pgfqpoint{2.548238in}{2.434687in}}%
\pgfpathlineto{\pgfqpoint{2.548238in}{2.446484in}}%
\pgfpathlineto{\pgfqpoint{2.566402in}{2.446484in}}%
\pgfpathlineto{\pgfqpoint{2.566402in}{2.434687in}}%
\pgfpathmoveto{\pgfqpoint{2.566402in}{2.411093in}}%
\pgfpathlineto{\pgfqpoint{2.566402in}{2.411093in}}%
\pgfpathlineto{\pgfqpoint{2.566402in}{2.422890in}}%
\pgfpathlineto{\pgfqpoint{2.584566in}{2.422890in}}%
\pgfpathlineto{\pgfqpoint{2.584566in}{2.411093in}}%
\pgfpathmoveto{\pgfqpoint{2.566402in}{2.422890in}}%
\pgfpathlineto{\pgfqpoint{2.566402in}{2.422890in}}%
\pgfpathlineto{\pgfqpoint{2.566402in}{2.434687in}}%
\pgfpathlineto{\pgfqpoint{2.584566in}{2.434687in}}%
\pgfpathlineto{\pgfqpoint{2.584566in}{2.422890in}}%
\pgfpathmoveto{\pgfqpoint{2.602730in}{2.387498in}}%
\pgfpathlineto{\pgfqpoint{2.602730in}{2.387498in}}%
\pgfpathlineto{\pgfqpoint{2.602730in}{2.399295in}}%
\pgfpathlineto{\pgfqpoint{2.620894in}{2.399295in}}%
\pgfpathlineto{\pgfqpoint{2.620894in}{2.387498in}}%
\pgfpathmoveto{\pgfqpoint{2.493746in}{2.481875in}}%
\pgfpathlineto{\pgfqpoint{2.493746in}{2.481875in}}%
\pgfpathlineto{\pgfqpoint{2.493746in}{2.493672in}}%
\pgfpathlineto{\pgfqpoint{2.511910in}{2.493672in}}%
\pgfpathlineto{\pgfqpoint{2.511910in}{2.481875in}}%
\pgfpathmoveto{\pgfqpoint{2.748043in}{2.245936in}}%
\pgfpathlineto{\pgfqpoint{2.748043in}{2.245936in}}%
\pgfpathlineto{\pgfqpoint{2.748043in}{2.257733in}}%
\pgfpathlineto{\pgfqpoint{2.766207in}{2.257733in}}%
\pgfpathlineto{\pgfqpoint{2.766207in}{2.245936in}}%
\pgfpathmoveto{\pgfqpoint{2.748043in}{2.257733in}}%
\pgfpathlineto{\pgfqpoint{2.748043in}{2.257733in}}%
\pgfpathlineto{\pgfqpoint{2.748043in}{2.269530in}}%
\pgfpathlineto{\pgfqpoint{2.766207in}{2.269530in}}%
\pgfpathlineto{\pgfqpoint{2.766207in}{2.257733in}}%
\pgfpathmoveto{\pgfqpoint{2.766207in}{2.245936in}}%
\pgfpathlineto{\pgfqpoint{2.766207in}{2.245936in}}%
\pgfpathlineto{\pgfqpoint{2.766207in}{2.257733in}}%
\pgfpathlineto{\pgfqpoint{2.784372in}{2.257733in}}%
\pgfpathlineto{\pgfqpoint{2.784372in}{2.245936in}}%
\pgfpathmoveto{\pgfqpoint{2.675387in}{2.316717in}}%
\pgfpathlineto{\pgfqpoint{2.675387in}{2.316717in}}%
\pgfpathlineto{\pgfqpoint{2.675387in}{2.328514in}}%
\pgfpathlineto{\pgfqpoint{2.693551in}{2.328514in}}%
\pgfpathlineto{\pgfqpoint{2.693551in}{2.316717in}}%
\pgfpathmoveto{\pgfqpoint{2.675387in}{2.328514in}}%
\pgfpathlineto{\pgfqpoint{2.675387in}{2.328514in}}%
\pgfpathlineto{\pgfqpoint{2.675387in}{2.340311in}}%
\pgfpathlineto{\pgfqpoint{2.693551in}{2.340311in}}%
\pgfpathlineto{\pgfqpoint{2.693551in}{2.328514in}}%
\pgfpathmoveto{\pgfqpoint{2.639058in}{2.340311in}}%
\pgfpathlineto{\pgfqpoint{2.639058in}{2.340311in}}%
\pgfpathlineto{\pgfqpoint{2.639058in}{2.352108in}}%
\pgfpathlineto{\pgfqpoint{2.657223in}{2.352108in}}%
\pgfpathlineto{\pgfqpoint{2.657223in}{2.340311in}}%
\pgfpathmoveto{\pgfqpoint{2.639058in}{2.352108in}}%
\pgfpathlineto{\pgfqpoint{2.639058in}{2.352108in}}%
\pgfpathlineto{\pgfqpoint{2.639058in}{2.363905in}}%
\pgfpathlineto{\pgfqpoint{2.657223in}{2.363905in}}%
\pgfpathlineto{\pgfqpoint{2.657223in}{2.352108in}}%
\pgfpathmoveto{\pgfqpoint{2.657223in}{2.340311in}}%
\pgfpathlineto{\pgfqpoint{2.657223in}{2.340311in}}%
\pgfpathlineto{\pgfqpoint{2.657223in}{2.352108in}}%
\pgfpathlineto{\pgfqpoint{2.675387in}{2.352108in}}%
\pgfpathlineto{\pgfqpoint{2.675387in}{2.340311in}}%
\pgfpathmoveto{\pgfqpoint{2.711715in}{2.293123in}}%
\pgfpathlineto{\pgfqpoint{2.711715in}{2.293123in}}%
\pgfpathlineto{\pgfqpoint{2.711715in}{2.304920in}}%
\pgfpathlineto{\pgfqpoint{2.729879in}{2.304920in}}%
\pgfpathlineto{\pgfqpoint{2.729879in}{2.293123in}}%
\pgfpathmoveto{\pgfqpoint{2.893360in}{2.127968in}}%
\pgfpathlineto{\pgfqpoint{2.893360in}{2.127968in}}%
\pgfpathlineto{\pgfqpoint{2.893360in}{2.139765in}}%
\pgfpathlineto{\pgfqpoint{2.911525in}{2.139765in}}%
\pgfpathlineto{\pgfqpoint{2.911525in}{2.127968in}}%
\pgfpathmoveto{\pgfqpoint{2.857030in}{2.151562in}}%
\pgfpathlineto{\pgfqpoint{2.857030in}{2.151562in}}%
\pgfpathlineto{\pgfqpoint{2.857030in}{2.163359in}}%
\pgfpathlineto{\pgfqpoint{2.875195in}{2.163359in}}%
\pgfpathlineto{\pgfqpoint{2.875195in}{2.151562in}}%
\pgfpathmoveto{\pgfqpoint{2.857030in}{2.163359in}}%
\pgfpathlineto{\pgfqpoint{2.857030in}{2.163359in}}%
\pgfpathlineto{\pgfqpoint{2.857030in}{2.175156in}}%
\pgfpathlineto{\pgfqpoint{2.875195in}{2.175156in}}%
\pgfpathlineto{\pgfqpoint{2.875195in}{2.163359in}}%
\pgfpathmoveto{\pgfqpoint{2.875195in}{2.151562in}}%
\pgfpathlineto{\pgfqpoint{2.875195in}{2.151562in}}%
\pgfpathlineto{\pgfqpoint{2.875195in}{2.163359in}}%
\pgfpathlineto{\pgfqpoint{2.893360in}{2.163359in}}%
\pgfpathlineto{\pgfqpoint{2.893360in}{2.151562in}}%
\pgfpathmoveto{\pgfqpoint{2.784372in}{2.222343in}}%
\pgfpathlineto{\pgfqpoint{2.784372in}{2.222343in}}%
\pgfpathlineto{\pgfqpoint{2.784372in}{2.234139in}}%
\pgfpathlineto{\pgfqpoint{2.802536in}{2.234139in}}%
\pgfpathlineto{\pgfqpoint{2.802536in}{2.222343in}}%
\pgfpathmoveto{\pgfqpoint{2.784372in}{2.234139in}}%
\pgfpathlineto{\pgfqpoint{2.784372in}{2.234139in}}%
\pgfpathlineto{\pgfqpoint{2.784372in}{2.245936in}}%
\pgfpathlineto{\pgfqpoint{2.802536in}{2.245936in}}%
\pgfpathlineto{\pgfqpoint{2.802536in}{2.234139in}}%
\pgfpathmoveto{\pgfqpoint{2.820701in}{2.198749in}}%
\pgfpathlineto{\pgfqpoint{2.820701in}{2.198749in}}%
\pgfpathlineto{\pgfqpoint{2.820701in}{2.210546in}}%
\pgfpathlineto{\pgfqpoint{2.838866in}{2.210546in}}%
\pgfpathlineto{\pgfqpoint{2.838866in}{2.198749in}}%
\pgfpathmoveto{\pgfqpoint{2.966016in}{2.057186in}}%
\pgfpathlineto{\pgfqpoint{2.966016in}{2.057186in}}%
\pgfpathlineto{\pgfqpoint{2.966016in}{2.068983in}}%
\pgfpathlineto{\pgfqpoint{2.984180in}{2.068983in}}%
\pgfpathlineto{\pgfqpoint{2.984180in}{2.057186in}}%
\pgfpathmoveto{\pgfqpoint{2.966016in}{2.068983in}}%
\pgfpathlineto{\pgfqpoint{2.966016in}{2.068983in}}%
\pgfpathlineto{\pgfqpoint{2.966016in}{2.080780in}}%
\pgfpathlineto{\pgfqpoint{2.984180in}{2.080780in}}%
\pgfpathlineto{\pgfqpoint{2.984180in}{2.068983in}}%
\pgfpathmoveto{\pgfqpoint{2.984180in}{2.057186in}}%
\pgfpathlineto{\pgfqpoint{2.984180in}{2.057186in}}%
\pgfpathlineto{\pgfqpoint{2.984180in}{2.068983in}}%
\pgfpathlineto{\pgfqpoint{3.002343in}{2.068983in}}%
\pgfpathlineto{\pgfqpoint{3.002343in}{2.057186in}}%
\pgfpathmoveto{\pgfqpoint{3.002343in}{2.033592in}}%
\pgfpathlineto{\pgfqpoint{3.002343in}{2.033592in}}%
\pgfpathlineto{\pgfqpoint{3.002343in}{2.045389in}}%
\pgfpathlineto{\pgfqpoint{3.020507in}{2.045389in}}%
\pgfpathlineto{\pgfqpoint{3.020507in}{2.033592in}}%
\pgfpathmoveto{\pgfqpoint{3.002343in}{2.045389in}}%
\pgfpathlineto{\pgfqpoint{3.002343in}{2.045389in}}%
\pgfpathlineto{\pgfqpoint{3.002343in}{2.057186in}}%
\pgfpathlineto{\pgfqpoint{3.020507in}{2.057186in}}%
\pgfpathlineto{\pgfqpoint{3.020507in}{2.045389in}}%
\pgfpathmoveto{\pgfqpoint{3.038670in}{2.009998in}}%
\pgfpathlineto{\pgfqpoint{3.038670in}{2.009998in}}%
\pgfpathlineto{\pgfqpoint{3.038670in}{2.021795in}}%
\pgfpathlineto{\pgfqpoint{3.056834in}{2.021795in}}%
\pgfpathlineto{\pgfqpoint{3.056834in}{2.009998in}}%
\pgfpathmoveto{\pgfqpoint{2.929689in}{2.104374in}}%
\pgfpathlineto{\pgfqpoint{2.929689in}{2.104374in}}%
\pgfpathlineto{\pgfqpoint{2.929689in}{2.116171in}}%
\pgfpathlineto{\pgfqpoint{2.947853in}{2.116171in}}%
\pgfpathlineto{\pgfqpoint{2.947853in}{2.104374in}}%
\pgfpathmoveto{\pgfqpoint{3.147654in}{1.892033in}}%
\pgfpathlineto{\pgfqpoint{3.147654in}{1.892033in}}%
\pgfpathlineto{\pgfqpoint{3.147654in}{1.903830in}}%
\pgfpathlineto{\pgfqpoint{3.165818in}{1.903830in}}%
\pgfpathlineto{\pgfqpoint{3.165818in}{1.892033in}}%
\pgfpathmoveto{\pgfqpoint{3.147654in}{1.903830in}}%
\pgfpathlineto{\pgfqpoint{3.147654in}{1.903830in}}%
\pgfpathlineto{\pgfqpoint{3.147654in}{1.915628in}}%
\pgfpathlineto{\pgfqpoint{3.165818in}{1.915628in}}%
\pgfpathlineto{\pgfqpoint{3.165818in}{1.903830in}}%
\pgfpathmoveto{\pgfqpoint{3.165818in}{1.892033in}}%
\pgfpathlineto{\pgfqpoint{3.165818in}{1.892033in}}%
\pgfpathlineto{\pgfqpoint{3.165818in}{1.903830in}}%
\pgfpathlineto{\pgfqpoint{3.183982in}{1.903830in}}%
\pgfpathlineto{\pgfqpoint{3.183982in}{1.892033in}}%
\pgfpathmoveto{\pgfqpoint{3.183982in}{1.868438in}}%
\pgfpathlineto{\pgfqpoint{3.183982in}{1.868438in}}%
\pgfpathlineto{\pgfqpoint{3.183982in}{1.880235in}}%
\pgfpathlineto{\pgfqpoint{3.202146in}{1.880235in}}%
\pgfpathlineto{\pgfqpoint{3.202146in}{1.868438in}}%
\pgfpathmoveto{\pgfqpoint{3.183982in}{1.880235in}}%
\pgfpathlineto{\pgfqpoint{3.183982in}{1.880235in}}%
\pgfpathlineto{\pgfqpoint{3.183982in}{1.892033in}}%
\pgfpathlineto{\pgfqpoint{3.202146in}{1.892033in}}%
\pgfpathlineto{\pgfqpoint{3.202146in}{1.880235in}}%
\pgfpathmoveto{\pgfqpoint{3.202146in}{1.868438in}}%
\pgfpathlineto{\pgfqpoint{3.202146in}{1.868438in}}%
\pgfpathlineto{\pgfqpoint{3.202146in}{1.880235in}}%
\pgfpathlineto{\pgfqpoint{3.220311in}{1.880235in}}%
\pgfpathlineto{\pgfqpoint{3.220311in}{1.868438in}}%
\pgfpathmoveto{\pgfqpoint{3.111325in}{1.939220in}}%
\pgfpathlineto{\pgfqpoint{3.111325in}{1.939220in}}%
\pgfpathlineto{\pgfqpoint{3.111325in}{1.951016in}}%
\pgfpathlineto{\pgfqpoint{3.129490in}{1.951016in}}%
\pgfpathlineto{\pgfqpoint{3.129490in}{1.939220in}}%
\pgfpathmoveto{\pgfqpoint{3.111325in}{1.951016in}}%
\pgfpathlineto{\pgfqpoint{3.111325in}{1.951016in}}%
\pgfpathlineto{\pgfqpoint{3.111325in}{1.962813in}}%
\pgfpathlineto{\pgfqpoint{3.129490in}{1.962813in}}%
\pgfpathlineto{\pgfqpoint{3.129490in}{1.951016in}}%
\pgfpathmoveto{\pgfqpoint{3.074997in}{1.962813in}}%
\pgfpathlineto{\pgfqpoint{3.074997in}{1.962813in}}%
\pgfpathlineto{\pgfqpoint{3.074997in}{1.974609in}}%
\pgfpathlineto{\pgfqpoint{3.093161in}{1.974609in}}%
\pgfpathlineto{\pgfqpoint{3.093161in}{1.962813in}}%
\pgfpathmoveto{\pgfqpoint{3.074997in}{1.974609in}}%
\pgfpathlineto{\pgfqpoint{3.074997in}{1.974609in}}%
\pgfpathlineto{\pgfqpoint{3.074997in}{1.986405in}}%
\pgfpathlineto{\pgfqpoint{3.093161in}{1.986405in}}%
\pgfpathlineto{\pgfqpoint{3.093161in}{1.974609in}}%
\pgfpathmoveto{\pgfqpoint{3.093161in}{1.962813in}}%
\pgfpathlineto{\pgfqpoint{3.093161in}{1.962813in}}%
\pgfpathlineto{\pgfqpoint{3.093161in}{1.974609in}}%
\pgfpathlineto{\pgfqpoint{3.111325in}{1.974609in}}%
\pgfpathlineto{\pgfqpoint{3.111325in}{1.962813in}}%
\pgfpathmoveto{\pgfqpoint{3.147654in}{1.915628in}}%
\pgfpathlineto{\pgfqpoint{3.147654in}{1.915628in}}%
\pgfpathlineto{\pgfqpoint{3.147654in}{1.927424in}}%
\pgfpathlineto{\pgfqpoint{3.165818in}{1.927424in}}%
\pgfpathlineto{\pgfqpoint{3.165818in}{1.915628in}}%
\pgfpathmoveto{\pgfqpoint{3.329298in}{1.750470in}}%
\pgfpathlineto{\pgfqpoint{3.329298in}{1.750470in}}%
\pgfpathlineto{\pgfqpoint{3.329298in}{1.762266in}}%
\pgfpathlineto{\pgfqpoint{3.347463in}{1.762266in}}%
\pgfpathlineto{\pgfqpoint{3.347463in}{1.750470in}}%
\pgfpathmoveto{\pgfqpoint{3.292969in}{1.774063in}}%
\pgfpathlineto{\pgfqpoint{3.292969in}{1.774063in}}%
\pgfpathlineto{\pgfqpoint{3.292969in}{1.785859in}}%
\pgfpathlineto{\pgfqpoint{3.311134in}{1.785859in}}%
\pgfpathlineto{\pgfqpoint{3.311134in}{1.774063in}}%
\pgfpathmoveto{\pgfqpoint{3.292969in}{1.785859in}}%
\pgfpathlineto{\pgfqpoint{3.292969in}{1.785859in}}%
\pgfpathlineto{\pgfqpoint{3.292969in}{1.797656in}}%
\pgfpathlineto{\pgfqpoint{3.311134in}{1.797656in}}%
\pgfpathlineto{\pgfqpoint{3.311134in}{1.785859in}}%
\pgfpathmoveto{\pgfqpoint{3.311134in}{1.774063in}}%
\pgfpathlineto{\pgfqpoint{3.311134in}{1.774063in}}%
\pgfpathlineto{\pgfqpoint{3.311134in}{1.785859in}}%
\pgfpathlineto{\pgfqpoint{3.329298in}{1.785859in}}%
\pgfpathlineto{\pgfqpoint{3.329298in}{1.774063in}}%
\pgfpathmoveto{\pgfqpoint{3.220311in}{1.844843in}}%
\pgfpathlineto{\pgfqpoint{3.220311in}{1.844843in}}%
\pgfpathlineto{\pgfqpoint{3.220311in}{1.856641in}}%
\pgfpathlineto{\pgfqpoint{3.238475in}{1.856641in}}%
\pgfpathlineto{\pgfqpoint{3.238475in}{1.844843in}}%
\pgfpathmoveto{\pgfqpoint{3.220311in}{1.856641in}}%
\pgfpathlineto{\pgfqpoint{3.220311in}{1.856641in}}%
\pgfpathlineto{\pgfqpoint{3.220311in}{1.868438in}}%
\pgfpathlineto{\pgfqpoint{3.238475in}{1.868438in}}%
\pgfpathlineto{\pgfqpoint{3.238475in}{1.856641in}}%
\pgfpathmoveto{\pgfqpoint{3.256640in}{1.821249in}}%
\pgfpathlineto{\pgfqpoint{3.256640in}{1.821249in}}%
\pgfpathlineto{\pgfqpoint{3.256640in}{1.833046in}}%
\pgfpathlineto{\pgfqpoint{3.274805in}{1.833046in}}%
\pgfpathlineto{\pgfqpoint{3.274805in}{1.821249in}}%
\pgfpathmoveto{\pgfqpoint{3.438283in}{0.499998in}}%
\pgfpathlineto{\pgfqpoint{3.438283in}{0.499998in}}%
\pgfpathlineto{\pgfqpoint{3.438283in}{0.511795in}}%
\pgfpathlineto{\pgfqpoint{3.456447in}{0.511795in}}%
\pgfpathlineto{\pgfqpoint{3.456447in}{0.499998in}}%
\pgfpathmoveto{\pgfqpoint{3.438283in}{0.511795in}}%
\pgfpathlineto{\pgfqpoint{3.438283in}{0.511795in}}%
\pgfpathlineto{\pgfqpoint{3.438283in}{0.523592in}}%
\pgfpathlineto{\pgfqpoint{3.456447in}{0.523592in}}%
\pgfpathlineto{\pgfqpoint{3.456447in}{0.511795in}}%
\pgfpathmoveto{\pgfqpoint{3.456447in}{0.511795in}}%
\pgfpathlineto{\pgfqpoint{3.456447in}{0.511795in}}%
\pgfpathlineto{\pgfqpoint{3.456447in}{0.523592in}}%
\pgfpathlineto{\pgfqpoint{3.474611in}{0.523592in}}%
\pgfpathlineto{\pgfqpoint{3.474611in}{0.511795in}}%
\pgfpathmoveto{\pgfqpoint{3.474611in}{0.535389in}}%
\pgfpathlineto{\pgfqpoint{3.474611in}{0.535389in}}%
\pgfpathlineto{\pgfqpoint{3.474611in}{0.547186in}}%
\pgfpathlineto{\pgfqpoint{3.492775in}{0.547186in}}%
\pgfpathlineto{\pgfqpoint{3.492775in}{0.535389in}}%
\pgfpathmoveto{\pgfqpoint{3.365628in}{1.703283in}}%
\pgfpathlineto{\pgfqpoint{3.365628in}{1.703283in}}%
\pgfpathlineto{\pgfqpoint{3.365628in}{1.715080in}}%
\pgfpathlineto{\pgfqpoint{3.383792in}{1.715080in}}%
\pgfpathlineto{\pgfqpoint{3.383792in}{1.703283in}}%
\pgfpathmoveto{\pgfqpoint{3.365628in}{1.715080in}}%
\pgfpathlineto{\pgfqpoint{3.365628in}{1.715080in}}%
\pgfpathlineto{\pgfqpoint{3.365628in}{1.726877in}}%
\pgfpathlineto{\pgfqpoint{3.383792in}{1.726877in}}%
\pgfpathlineto{\pgfqpoint{3.383792in}{1.715080in}}%
\pgfpathmoveto{\pgfqpoint{3.383792in}{1.703283in}}%
\pgfpathlineto{\pgfqpoint{3.383792in}{1.703283in}}%
\pgfpathlineto{\pgfqpoint{3.383792in}{1.715080in}}%
\pgfpathlineto{\pgfqpoint{3.401955in}{1.715080in}}%
\pgfpathlineto{\pgfqpoint{3.401955in}{1.703283in}}%
\pgfpathmoveto{\pgfqpoint{3.401955in}{1.679688in}}%
\pgfpathlineto{\pgfqpoint{3.401955in}{1.679688in}}%
\pgfpathlineto{\pgfqpoint{3.401955in}{1.691485in}}%
\pgfpathlineto{\pgfqpoint{3.420119in}{1.691485in}}%
\pgfpathlineto{\pgfqpoint{3.420119in}{1.679688in}}%
\pgfpathmoveto{\pgfqpoint{3.401955in}{1.691485in}}%
\pgfpathlineto{\pgfqpoint{3.401955in}{1.691485in}}%
\pgfpathlineto{\pgfqpoint{3.401955in}{1.703283in}}%
\pgfpathlineto{\pgfqpoint{3.420119in}{1.703283in}}%
\pgfpathlineto{\pgfqpoint{3.420119in}{1.691485in}}%
\pgfpathmoveto{\pgfqpoint{3.420119in}{1.679688in}}%
\pgfpathlineto{\pgfqpoint{3.420119in}{1.679688in}}%
\pgfpathlineto{\pgfqpoint{3.420119in}{1.691485in}}%
\pgfpathlineto{\pgfqpoint{3.438283in}{1.691485in}}%
\pgfpathlineto{\pgfqpoint{3.438283in}{1.679688in}}%
\pgfpathmoveto{\pgfqpoint{3.438283in}{1.656094in}}%
\pgfpathlineto{\pgfqpoint{3.438283in}{1.656094in}}%
\pgfpathlineto{\pgfqpoint{3.438283in}{1.667891in}}%
\pgfpathlineto{\pgfqpoint{3.456447in}{1.667891in}}%
\pgfpathlineto{\pgfqpoint{3.456447in}{1.656094in}}%
\pgfpathmoveto{\pgfqpoint{3.474611in}{1.632499in}}%
\pgfpathlineto{\pgfqpoint{3.474611in}{1.632499in}}%
\pgfpathlineto{\pgfqpoint{3.474611in}{1.644296in}}%
\pgfpathlineto{\pgfqpoint{3.492775in}{1.644296in}}%
\pgfpathlineto{\pgfqpoint{3.492775in}{1.632499in}}%
\pgfpathmoveto{\pgfqpoint{3.365628in}{1.726877in}}%
\pgfpathlineto{\pgfqpoint{3.365628in}{1.726877in}}%
\pgfpathlineto{\pgfqpoint{3.365628in}{1.738674in}}%
\pgfpathlineto{\pgfqpoint{3.383792in}{1.738674in}}%
\pgfpathlineto{\pgfqpoint{3.383792in}{1.726877in}}%
\pgfpathmoveto{\pgfqpoint{3.510939in}{0.558983in}}%
\pgfpathlineto{\pgfqpoint{3.510939in}{0.558983in}}%
\pgfpathlineto{\pgfqpoint{3.510939in}{0.570780in}}%
\pgfpathlineto{\pgfqpoint{3.529103in}{0.570780in}}%
\pgfpathlineto{\pgfqpoint{3.529103in}{0.558983in}}%
\pgfpathmoveto{\pgfqpoint{3.547266in}{0.582576in}}%
\pgfpathlineto{\pgfqpoint{3.547266in}{0.582576in}}%
\pgfpathlineto{\pgfqpoint{3.547266in}{0.594373in}}%
\pgfpathlineto{\pgfqpoint{3.565430in}{0.594373in}}%
\pgfpathlineto{\pgfqpoint{3.565430in}{0.582576in}}%
\pgfpathmoveto{\pgfqpoint{3.547266in}{0.594373in}}%
\pgfpathlineto{\pgfqpoint{3.547266in}{0.594373in}}%
\pgfpathlineto{\pgfqpoint{3.547266in}{0.606171in}}%
\pgfpathlineto{\pgfqpoint{3.565430in}{0.606171in}}%
\pgfpathlineto{\pgfqpoint{3.565430in}{0.594373in}}%
\pgfpathmoveto{\pgfqpoint{3.547266in}{0.606171in}}%
\pgfpathlineto{\pgfqpoint{3.547266in}{0.606171in}}%
\pgfpathlineto{\pgfqpoint{3.547266in}{0.617968in}}%
\pgfpathlineto{\pgfqpoint{3.565430in}{0.617968in}}%
\pgfpathlineto{\pgfqpoint{3.565430in}{0.606171in}}%
\pgfpathmoveto{\pgfqpoint{3.565430in}{0.606171in}}%
\pgfpathlineto{\pgfqpoint{3.565430in}{0.606171in}}%
\pgfpathlineto{\pgfqpoint{3.565430in}{0.617968in}}%
\pgfpathlineto{\pgfqpoint{3.583594in}{0.617968in}}%
\pgfpathlineto{\pgfqpoint{3.583594in}{0.606171in}}%
\pgfpathmoveto{\pgfqpoint{3.583594in}{0.617968in}}%
\pgfpathlineto{\pgfqpoint{3.583594in}{0.617968in}}%
\pgfpathlineto{\pgfqpoint{3.583594in}{0.629766in}}%
\pgfpathlineto{\pgfqpoint{3.601758in}{0.629766in}}%
\pgfpathlineto{\pgfqpoint{3.601758in}{0.617968in}}%
\pgfpathmoveto{\pgfqpoint{3.583594in}{0.629766in}}%
\pgfpathlineto{\pgfqpoint{3.583594in}{0.629766in}}%
\pgfpathlineto{\pgfqpoint{3.583594in}{0.641563in}}%
\pgfpathlineto{\pgfqpoint{3.601758in}{0.641563in}}%
\pgfpathlineto{\pgfqpoint{3.601758in}{0.629766in}}%
\pgfpathmoveto{\pgfqpoint{3.601758in}{0.629766in}}%
\pgfpathlineto{\pgfqpoint{3.601758in}{0.629766in}}%
\pgfpathlineto{\pgfqpoint{3.601758in}{0.641563in}}%
\pgfpathlineto{\pgfqpoint{3.619922in}{0.641563in}}%
\pgfpathlineto{\pgfqpoint{3.619922in}{0.629766in}}%
\pgfpathmoveto{\pgfqpoint{3.619922in}{0.641563in}}%
\pgfpathlineto{\pgfqpoint{3.619922in}{0.641563in}}%
\pgfpathlineto{\pgfqpoint{3.619922in}{0.653360in}}%
\pgfpathlineto{\pgfqpoint{3.638086in}{0.653360in}}%
\pgfpathlineto{\pgfqpoint{3.638086in}{0.641563in}}%
\pgfpathmoveto{\pgfqpoint{3.619922in}{0.653360in}}%
\pgfpathlineto{\pgfqpoint{3.619922in}{0.653360in}}%
\pgfpathlineto{\pgfqpoint{3.619922in}{0.665158in}}%
\pgfpathlineto{\pgfqpoint{3.638086in}{0.665158in}}%
\pgfpathlineto{\pgfqpoint{3.638086in}{0.653360in}}%
\pgfpathmoveto{\pgfqpoint{3.638086in}{0.653360in}}%
\pgfpathlineto{\pgfqpoint{3.638086in}{0.653360in}}%
\pgfpathlineto{\pgfqpoint{3.638086in}{0.665158in}}%
\pgfpathlineto{\pgfqpoint{3.656250in}{0.665158in}}%
\pgfpathlineto{\pgfqpoint{3.656250in}{0.653360in}}%
\pgfpathmoveto{\pgfqpoint{3.583594in}{1.514534in}}%
\pgfpathlineto{\pgfqpoint{3.583594in}{1.514534in}}%
\pgfpathlineto{\pgfqpoint{3.583594in}{1.526331in}}%
\pgfpathlineto{\pgfqpoint{3.601758in}{1.526331in}}%
\pgfpathlineto{\pgfqpoint{3.601758in}{1.514534in}}%
\pgfpathmoveto{\pgfqpoint{3.583594in}{1.526331in}}%
\pgfpathlineto{\pgfqpoint{3.583594in}{1.526331in}}%
\pgfpathlineto{\pgfqpoint{3.583594in}{1.538128in}}%
\pgfpathlineto{\pgfqpoint{3.601758in}{1.538128in}}%
\pgfpathlineto{\pgfqpoint{3.601758in}{1.526331in}}%
\pgfpathmoveto{\pgfqpoint{3.601758in}{1.514534in}}%
\pgfpathlineto{\pgfqpoint{3.601758in}{1.514534in}}%
\pgfpathlineto{\pgfqpoint{3.601758in}{1.526331in}}%
\pgfpathlineto{\pgfqpoint{3.619922in}{1.526331in}}%
\pgfpathlineto{\pgfqpoint{3.619922in}{1.514534in}}%
\pgfpathmoveto{\pgfqpoint{3.619922in}{1.490940in}}%
\pgfpathlineto{\pgfqpoint{3.619922in}{1.490940in}}%
\pgfpathlineto{\pgfqpoint{3.619922in}{1.502737in}}%
\pgfpathlineto{\pgfqpoint{3.638086in}{1.502737in}}%
\pgfpathlineto{\pgfqpoint{3.638086in}{1.490940in}}%
\pgfpathmoveto{\pgfqpoint{3.619922in}{1.502737in}}%
\pgfpathlineto{\pgfqpoint{3.619922in}{1.502737in}}%
\pgfpathlineto{\pgfqpoint{3.619922in}{1.514534in}}%
\pgfpathlineto{\pgfqpoint{3.638086in}{1.514534in}}%
\pgfpathlineto{\pgfqpoint{3.638086in}{1.502737in}}%
\pgfpathmoveto{\pgfqpoint{3.638086in}{1.490940in}}%
\pgfpathlineto{\pgfqpoint{3.638086in}{1.490940in}}%
\pgfpathlineto{\pgfqpoint{3.638086in}{1.502737in}}%
\pgfpathlineto{\pgfqpoint{3.656250in}{1.502737in}}%
\pgfpathlineto{\pgfqpoint{3.656250in}{1.490940in}}%
\pgfpathmoveto{\pgfqpoint{3.547266in}{1.561720in}}%
\pgfpathlineto{\pgfqpoint{3.547266in}{1.561720in}}%
\pgfpathlineto{\pgfqpoint{3.547266in}{1.573517in}}%
\pgfpathlineto{\pgfqpoint{3.565430in}{1.573517in}}%
\pgfpathlineto{\pgfqpoint{3.565430in}{1.561720in}}%
\pgfpathmoveto{\pgfqpoint{3.510939in}{1.585313in}}%
\pgfpathlineto{\pgfqpoint{3.510939in}{1.585313in}}%
\pgfpathlineto{\pgfqpoint{3.510939in}{1.597110in}}%
\pgfpathlineto{\pgfqpoint{3.529103in}{1.597110in}}%
\pgfpathlineto{\pgfqpoint{3.529103in}{1.585313in}}%
\pgfpathmoveto{\pgfqpoint{3.510939in}{1.597110in}}%
\pgfpathlineto{\pgfqpoint{3.510939in}{1.597110in}}%
\pgfpathlineto{\pgfqpoint{3.510939in}{1.608906in}}%
\pgfpathlineto{\pgfqpoint{3.529103in}{1.608906in}}%
\pgfpathlineto{\pgfqpoint{3.529103in}{1.597110in}}%
\pgfpathmoveto{\pgfqpoint{3.529103in}{1.585313in}}%
\pgfpathlineto{\pgfqpoint{3.529103in}{1.585313in}}%
\pgfpathlineto{\pgfqpoint{3.529103in}{1.597110in}}%
\pgfpathlineto{\pgfqpoint{3.547266in}{1.597110in}}%
\pgfpathlineto{\pgfqpoint{3.547266in}{1.585313in}}%
\pgfpathmoveto{\pgfqpoint{3.583594in}{1.538128in}}%
\pgfpathlineto{\pgfqpoint{3.583594in}{1.538128in}}%
\pgfpathlineto{\pgfqpoint{3.583594in}{1.549924in}}%
\pgfpathlineto{\pgfqpoint{3.601758in}{1.549924in}}%
\pgfpathlineto{\pgfqpoint{3.601758in}{1.538128in}}%
\pgfpathmoveto{\pgfqpoint{3.656250in}{0.665158in}}%
\pgfpathlineto{\pgfqpoint{3.656250in}{0.665158in}}%
\pgfpathlineto{\pgfqpoint{3.656250in}{0.676955in}}%
\pgfpathlineto{\pgfqpoint{3.674415in}{0.676955in}}%
\pgfpathlineto{\pgfqpoint{3.674415in}{0.665158in}}%
\pgfpathmoveto{\pgfqpoint{3.656250in}{0.676955in}}%
\pgfpathlineto{\pgfqpoint{3.656250in}{0.676955in}}%
\pgfpathlineto{\pgfqpoint{3.656250in}{0.688753in}}%
\pgfpathlineto{\pgfqpoint{3.674415in}{0.688753in}}%
\pgfpathlineto{\pgfqpoint{3.674415in}{0.676955in}}%
\pgfpathmoveto{\pgfqpoint{3.692579in}{0.700549in}}%
\pgfpathlineto{\pgfqpoint{3.692579in}{0.700549in}}%
\pgfpathlineto{\pgfqpoint{3.692579in}{0.712346in}}%
\pgfpathlineto{\pgfqpoint{3.710744in}{0.712346in}}%
\pgfpathlineto{\pgfqpoint{3.710744in}{0.700549in}}%
\pgfpathmoveto{\pgfqpoint{3.728908in}{0.724142in}}%
\pgfpathlineto{\pgfqpoint{3.728908in}{0.724142in}}%
\pgfpathlineto{\pgfqpoint{3.728908in}{0.735939in}}%
\pgfpathlineto{\pgfqpoint{3.747073in}{0.735939in}}%
\pgfpathlineto{\pgfqpoint{3.747073in}{0.724142in}}%
\pgfpathmoveto{\pgfqpoint{3.728908in}{0.735939in}}%
\pgfpathlineto{\pgfqpoint{3.728908in}{0.735939in}}%
\pgfpathlineto{\pgfqpoint{3.728908in}{0.747736in}}%
\pgfpathlineto{\pgfqpoint{3.747073in}{0.747736in}}%
\pgfpathlineto{\pgfqpoint{3.747073in}{0.735939in}}%
\pgfpathmoveto{\pgfqpoint{3.728908in}{0.747736in}}%
\pgfpathlineto{\pgfqpoint{3.728908in}{0.747736in}}%
\pgfpathlineto{\pgfqpoint{3.728908in}{0.759532in}}%
\pgfpathlineto{\pgfqpoint{3.747073in}{0.759532in}}%
\pgfpathlineto{\pgfqpoint{3.747073in}{0.747736in}}%
\pgfpathmoveto{\pgfqpoint{3.747073in}{0.747736in}}%
\pgfpathlineto{\pgfqpoint{3.747073in}{0.747736in}}%
\pgfpathlineto{\pgfqpoint{3.747073in}{0.759532in}}%
\pgfpathlineto{\pgfqpoint{3.765237in}{0.759532in}}%
\pgfpathlineto{\pgfqpoint{3.765237in}{0.747736in}}%
\pgfpathmoveto{\pgfqpoint{3.765237in}{0.759532in}}%
\pgfpathlineto{\pgfqpoint{3.765237in}{0.759532in}}%
\pgfpathlineto{\pgfqpoint{3.765237in}{0.771329in}}%
\pgfpathlineto{\pgfqpoint{3.783402in}{0.771329in}}%
\pgfpathlineto{\pgfqpoint{3.783402in}{0.759532in}}%
\pgfpathmoveto{\pgfqpoint{3.765237in}{0.771329in}}%
\pgfpathlineto{\pgfqpoint{3.765237in}{0.771329in}}%
\pgfpathlineto{\pgfqpoint{3.765237in}{0.783125in}}%
\pgfpathlineto{\pgfqpoint{3.783402in}{0.783125in}}%
\pgfpathlineto{\pgfqpoint{3.783402in}{0.771329in}}%
\pgfpathmoveto{\pgfqpoint{3.783402in}{0.771329in}}%
\pgfpathlineto{\pgfqpoint{3.783402in}{0.771329in}}%
\pgfpathlineto{\pgfqpoint{3.783402in}{0.783125in}}%
\pgfpathlineto{\pgfqpoint{3.801566in}{0.783125in}}%
\pgfpathlineto{\pgfqpoint{3.801566in}{0.771329in}}%
\pgfpathmoveto{\pgfqpoint{3.692579in}{1.420157in}}%
\pgfpathlineto{\pgfqpoint{3.692579in}{1.420157in}}%
\pgfpathlineto{\pgfqpoint{3.692579in}{1.431955in}}%
\pgfpathlineto{\pgfqpoint{3.710744in}{1.431955in}}%
\pgfpathlineto{\pgfqpoint{3.710744in}{1.420157in}}%
\pgfpathmoveto{\pgfqpoint{3.692579in}{1.431955in}}%
\pgfpathlineto{\pgfqpoint{3.692579in}{1.431955in}}%
\pgfpathlineto{\pgfqpoint{3.692579in}{1.443752in}}%
\pgfpathlineto{\pgfqpoint{3.710744in}{1.443752in}}%
\pgfpathlineto{\pgfqpoint{3.710744in}{1.431955in}}%
\pgfpathmoveto{\pgfqpoint{3.710744in}{1.420157in}}%
\pgfpathlineto{\pgfqpoint{3.710744in}{1.420157in}}%
\pgfpathlineto{\pgfqpoint{3.710744in}{1.431955in}}%
\pgfpathlineto{\pgfqpoint{3.728908in}{1.431955in}}%
\pgfpathlineto{\pgfqpoint{3.728908in}{1.420157in}}%
\pgfpathmoveto{\pgfqpoint{3.765237in}{1.372968in}}%
\pgfpathlineto{\pgfqpoint{3.765237in}{1.372968in}}%
\pgfpathlineto{\pgfqpoint{3.765237in}{1.384765in}}%
\pgfpathlineto{\pgfqpoint{3.783402in}{1.384765in}}%
\pgfpathlineto{\pgfqpoint{3.783402in}{1.372968in}}%
\pgfpathmoveto{\pgfqpoint{3.728908in}{1.396562in}}%
\pgfpathlineto{\pgfqpoint{3.728908in}{1.396562in}}%
\pgfpathlineto{\pgfqpoint{3.728908in}{1.408360in}}%
\pgfpathlineto{\pgfqpoint{3.747073in}{1.408360in}}%
\pgfpathlineto{\pgfqpoint{3.747073in}{1.396562in}}%
\pgfpathmoveto{\pgfqpoint{3.728908in}{1.408360in}}%
\pgfpathlineto{\pgfqpoint{3.728908in}{1.408360in}}%
\pgfpathlineto{\pgfqpoint{3.728908in}{1.420157in}}%
\pgfpathlineto{\pgfqpoint{3.747073in}{1.420157in}}%
\pgfpathlineto{\pgfqpoint{3.747073in}{1.408360in}}%
\pgfpathmoveto{\pgfqpoint{3.747073in}{1.396562in}}%
\pgfpathlineto{\pgfqpoint{3.747073in}{1.396562in}}%
\pgfpathlineto{\pgfqpoint{3.747073in}{1.408360in}}%
\pgfpathlineto{\pgfqpoint{3.765237in}{1.408360in}}%
\pgfpathlineto{\pgfqpoint{3.765237in}{1.396562in}}%
\pgfpathmoveto{\pgfqpoint{3.656250in}{1.467346in}}%
\pgfpathlineto{\pgfqpoint{3.656250in}{1.467346in}}%
\pgfpathlineto{\pgfqpoint{3.656250in}{1.479143in}}%
\pgfpathlineto{\pgfqpoint{3.674415in}{1.479143in}}%
\pgfpathlineto{\pgfqpoint{3.674415in}{1.467346in}}%
\pgfpathmoveto{\pgfqpoint{3.692579in}{1.443752in}}%
\pgfpathlineto{\pgfqpoint{3.692579in}{1.443752in}}%
\pgfpathlineto{\pgfqpoint{3.692579in}{1.455549in}}%
\pgfpathlineto{\pgfqpoint{3.710744in}{1.455549in}}%
\pgfpathlineto{\pgfqpoint{3.710744in}{1.443752in}}%
\pgfpathmoveto{\pgfqpoint{3.801566in}{0.783125in}}%
\pgfpathlineto{\pgfqpoint{3.801566in}{0.783125in}}%
\pgfpathlineto{\pgfqpoint{3.801566in}{0.794922in}}%
\pgfpathlineto{\pgfqpoint{3.819730in}{0.794922in}}%
\pgfpathlineto{\pgfqpoint{3.819730in}{0.783125in}}%
\pgfpathmoveto{\pgfqpoint{3.801566in}{0.794922in}}%
\pgfpathlineto{\pgfqpoint{3.801566in}{0.794922in}}%
\pgfpathlineto{\pgfqpoint{3.801566in}{0.806719in}}%
\pgfpathlineto{\pgfqpoint{3.819730in}{0.806719in}}%
\pgfpathlineto{\pgfqpoint{3.819730in}{0.794922in}}%
\pgfpathmoveto{\pgfqpoint{3.819730in}{0.794922in}}%
\pgfpathlineto{\pgfqpoint{3.819730in}{0.794922in}}%
\pgfpathlineto{\pgfqpoint{3.819730in}{0.806719in}}%
\pgfpathlineto{\pgfqpoint{3.837893in}{0.806719in}}%
\pgfpathlineto{\pgfqpoint{3.837893in}{0.794922in}}%
\pgfpathmoveto{\pgfqpoint{3.837893in}{0.818516in}}%
\pgfpathlineto{\pgfqpoint{3.837893in}{0.818516in}}%
\pgfpathlineto{\pgfqpoint{3.837893in}{0.830313in}}%
\pgfpathlineto{\pgfqpoint{3.856056in}{0.830313in}}%
\pgfpathlineto{\pgfqpoint{3.856056in}{0.818516in}}%
\pgfpathmoveto{\pgfqpoint{3.874220in}{0.842110in}}%
\pgfpathlineto{\pgfqpoint{3.874220in}{0.842110in}}%
\pgfpathlineto{\pgfqpoint{3.874220in}{0.853907in}}%
\pgfpathlineto{\pgfqpoint{3.892383in}{0.853907in}}%
\pgfpathlineto{\pgfqpoint{3.892383in}{0.842110in}}%
\pgfpathmoveto{\pgfqpoint{3.910546in}{0.865704in}}%
\pgfpathlineto{\pgfqpoint{3.910546in}{0.865704in}}%
\pgfpathlineto{\pgfqpoint{3.910546in}{0.877501in}}%
\pgfpathlineto{\pgfqpoint{3.928710in}{0.877501in}}%
\pgfpathlineto{\pgfqpoint{3.928710in}{0.865704in}}%
\pgfpathmoveto{\pgfqpoint{3.837893in}{1.302186in}}%
\pgfpathlineto{\pgfqpoint{3.837893in}{1.302186in}}%
\pgfpathlineto{\pgfqpoint{3.837893in}{1.313983in}}%
\pgfpathlineto{\pgfqpoint{3.856056in}{1.313983in}}%
\pgfpathlineto{\pgfqpoint{3.856056in}{1.302186in}}%
\pgfpathmoveto{\pgfqpoint{3.837893in}{1.313983in}}%
\pgfpathlineto{\pgfqpoint{3.837893in}{1.313983in}}%
\pgfpathlineto{\pgfqpoint{3.837893in}{1.325779in}}%
\pgfpathlineto{\pgfqpoint{3.856056in}{1.325779in}}%
\pgfpathlineto{\pgfqpoint{3.856056in}{1.313983in}}%
\pgfpathmoveto{\pgfqpoint{3.856056in}{1.302186in}}%
\pgfpathlineto{\pgfqpoint{3.856056in}{1.302186in}}%
\pgfpathlineto{\pgfqpoint{3.856056in}{1.313983in}}%
\pgfpathlineto{\pgfqpoint{3.874220in}{1.313983in}}%
\pgfpathlineto{\pgfqpoint{3.874220in}{1.302186in}}%
\pgfpathmoveto{\pgfqpoint{3.874220in}{1.278593in}}%
\pgfpathlineto{\pgfqpoint{3.874220in}{1.278593in}}%
\pgfpathlineto{\pgfqpoint{3.874220in}{1.290389in}}%
\pgfpathlineto{\pgfqpoint{3.892383in}{1.290389in}}%
\pgfpathlineto{\pgfqpoint{3.892383in}{1.278593in}}%
\pgfpathmoveto{\pgfqpoint{3.874220in}{1.290389in}}%
\pgfpathlineto{\pgfqpoint{3.874220in}{1.290389in}}%
\pgfpathlineto{\pgfqpoint{3.874220in}{1.302186in}}%
\pgfpathlineto{\pgfqpoint{3.892383in}{1.302186in}}%
\pgfpathlineto{\pgfqpoint{3.892383in}{1.290389in}}%
\pgfpathmoveto{\pgfqpoint{3.910546in}{1.254999in}}%
\pgfpathlineto{\pgfqpoint{3.910546in}{1.254999in}}%
\pgfpathlineto{\pgfqpoint{3.910546in}{1.266796in}}%
\pgfpathlineto{\pgfqpoint{3.928710in}{1.266796in}}%
\pgfpathlineto{\pgfqpoint{3.928710in}{1.254999in}}%
\pgfpathmoveto{\pgfqpoint{3.801566in}{1.349373in}}%
\pgfpathlineto{\pgfqpoint{3.801566in}{1.349373in}}%
\pgfpathlineto{\pgfqpoint{3.801566in}{1.361170in}}%
\pgfpathlineto{\pgfqpoint{3.819730in}{1.361170in}}%
\pgfpathlineto{\pgfqpoint{3.819730in}{1.349373in}}%
\pgfpathmoveto{\pgfqpoint{3.946873in}{0.901094in}}%
\pgfpathlineto{\pgfqpoint{3.946873in}{0.901094in}}%
\pgfpathlineto{\pgfqpoint{3.946873in}{0.912891in}}%
\pgfpathlineto{\pgfqpoint{3.965037in}{0.912891in}}%
\pgfpathlineto{\pgfqpoint{3.965037in}{0.901094in}}%
\pgfpathmoveto{\pgfqpoint{3.946873in}{0.912891in}}%
\pgfpathlineto{\pgfqpoint{3.946873in}{0.912891in}}%
\pgfpathlineto{\pgfqpoint{3.946873in}{0.924687in}}%
\pgfpathlineto{\pgfqpoint{3.965037in}{0.924687in}}%
\pgfpathlineto{\pgfqpoint{3.965037in}{0.912891in}}%
\pgfpathmoveto{\pgfqpoint{3.965037in}{0.912891in}}%
\pgfpathlineto{\pgfqpoint{3.965037in}{0.912891in}}%
\pgfpathlineto{\pgfqpoint{3.965037in}{0.924687in}}%
\pgfpathlineto{\pgfqpoint{3.983201in}{0.924687in}}%
\pgfpathlineto{\pgfqpoint{3.983201in}{0.912891in}}%
\pgfpathmoveto{\pgfqpoint{3.983201in}{0.924687in}}%
\pgfpathlineto{\pgfqpoint{3.983201in}{0.924687in}}%
\pgfpathlineto{\pgfqpoint{3.983201in}{0.936484in}}%
\pgfpathlineto{\pgfqpoint{4.001365in}{0.936484in}}%
\pgfpathlineto{\pgfqpoint{4.001365in}{0.924687in}}%
\pgfpathmoveto{\pgfqpoint{3.983201in}{0.936484in}}%
\pgfpathlineto{\pgfqpoint{3.983201in}{0.936484in}}%
\pgfpathlineto{\pgfqpoint{3.983201in}{0.948280in}}%
\pgfpathlineto{\pgfqpoint{4.001365in}{0.948280in}}%
\pgfpathlineto{\pgfqpoint{4.001365in}{0.936484in}}%
\pgfpathmoveto{\pgfqpoint{4.001365in}{0.936484in}}%
\pgfpathlineto{\pgfqpoint{4.001365in}{0.936484in}}%
\pgfpathlineto{\pgfqpoint{4.001365in}{0.948280in}}%
\pgfpathlineto{\pgfqpoint{4.019529in}{0.948280in}}%
\pgfpathlineto{\pgfqpoint{4.019529in}{0.936484in}}%
\pgfpathmoveto{\pgfqpoint{4.019529in}{0.948280in}}%
\pgfpathlineto{\pgfqpoint{4.019529in}{0.948280in}}%
\pgfpathlineto{\pgfqpoint{4.019529in}{0.960077in}}%
\pgfpathlineto{\pgfqpoint{4.037692in}{0.960077in}}%
\pgfpathlineto{\pgfqpoint{4.037692in}{0.948280in}}%
\pgfpathmoveto{\pgfqpoint{4.019529in}{0.960077in}}%
\pgfpathlineto{\pgfqpoint{4.019529in}{0.960077in}}%
\pgfpathlineto{\pgfqpoint{4.019529in}{0.971874in}}%
\pgfpathlineto{\pgfqpoint{4.037692in}{0.971874in}}%
\pgfpathlineto{\pgfqpoint{4.037692in}{0.960077in}}%
\pgfpathmoveto{\pgfqpoint{4.055856in}{0.983670in}}%
\pgfpathlineto{\pgfqpoint{4.055856in}{0.983670in}}%
\pgfpathlineto{\pgfqpoint{4.055856in}{0.995467in}}%
\pgfpathlineto{\pgfqpoint{4.074020in}{0.995467in}}%
\pgfpathlineto{\pgfqpoint{4.074020in}{0.983670in}}%
\pgfpathmoveto{\pgfqpoint{4.055856in}{1.113436in}}%
\pgfpathlineto{\pgfqpoint{4.055856in}{1.113436in}}%
\pgfpathlineto{\pgfqpoint{4.055856in}{1.125233in}}%
\pgfpathlineto{\pgfqpoint{4.074020in}{1.125233in}}%
\pgfpathlineto{\pgfqpoint{4.074020in}{1.113436in}}%
\pgfpathmoveto{\pgfqpoint{4.055856in}{1.125233in}}%
\pgfpathlineto{\pgfqpoint{4.055856in}{1.125233in}}%
\pgfpathlineto{\pgfqpoint{4.055856in}{1.137030in}}%
\pgfpathlineto{\pgfqpoint{4.074020in}{1.137030in}}%
\pgfpathlineto{\pgfqpoint{4.074020in}{1.125233in}}%
\pgfpathmoveto{\pgfqpoint{4.074020in}{1.113436in}}%
\pgfpathlineto{\pgfqpoint{4.074020in}{1.113436in}}%
\pgfpathlineto{\pgfqpoint{4.074020in}{1.125233in}}%
\pgfpathlineto{\pgfqpoint{4.092184in}{1.125233in}}%
\pgfpathlineto{\pgfqpoint{4.092184in}{1.113436in}}%
\pgfpathmoveto{\pgfqpoint{3.983201in}{1.184218in}}%
\pgfpathlineto{\pgfqpoint{3.983201in}{1.184218in}}%
\pgfpathlineto{\pgfqpoint{3.983201in}{1.196015in}}%
\pgfpathlineto{\pgfqpoint{4.001365in}{1.196015in}}%
\pgfpathlineto{\pgfqpoint{4.001365in}{1.184218in}}%
\pgfpathmoveto{\pgfqpoint{3.983201in}{1.196015in}}%
\pgfpathlineto{\pgfqpoint{3.983201in}{1.196015in}}%
\pgfpathlineto{\pgfqpoint{3.983201in}{1.207812in}}%
\pgfpathlineto{\pgfqpoint{4.001365in}{1.207812in}}%
\pgfpathlineto{\pgfqpoint{4.001365in}{1.196015in}}%
\pgfpathmoveto{\pgfqpoint{3.946873in}{1.207812in}}%
\pgfpathlineto{\pgfqpoint{3.946873in}{1.207812in}}%
\pgfpathlineto{\pgfqpoint{3.946873in}{1.219609in}}%
\pgfpathlineto{\pgfqpoint{3.965037in}{1.219609in}}%
\pgfpathlineto{\pgfqpoint{3.965037in}{1.207812in}}%
\pgfpathmoveto{\pgfqpoint{3.946873in}{1.219609in}}%
\pgfpathlineto{\pgfqpoint{3.946873in}{1.219609in}}%
\pgfpathlineto{\pgfqpoint{3.946873in}{1.231406in}}%
\pgfpathlineto{\pgfqpoint{3.965037in}{1.231406in}}%
\pgfpathlineto{\pgfqpoint{3.965037in}{1.219609in}}%
\pgfpathmoveto{\pgfqpoint{3.965037in}{1.207812in}}%
\pgfpathlineto{\pgfqpoint{3.965037in}{1.207812in}}%
\pgfpathlineto{\pgfqpoint{3.965037in}{1.219609in}}%
\pgfpathlineto{\pgfqpoint{3.983201in}{1.219609in}}%
\pgfpathlineto{\pgfqpoint{3.983201in}{1.207812in}}%
\pgfpathmoveto{\pgfqpoint{4.019529in}{1.160624in}}%
\pgfpathlineto{\pgfqpoint{4.019529in}{1.160624in}}%
\pgfpathlineto{\pgfqpoint{4.019529in}{1.172421in}}%
\pgfpathlineto{\pgfqpoint{4.037692in}{1.172421in}}%
\pgfpathlineto{\pgfqpoint{4.037692in}{1.160624in}}%
\pgfpathmoveto{\pgfqpoint{4.092184in}{1.007264in}}%
\pgfpathlineto{\pgfqpoint{4.092184in}{1.007264in}}%
\pgfpathlineto{\pgfqpoint{4.092184in}{1.019060in}}%
\pgfpathlineto{\pgfqpoint{4.110348in}{1.019060in}}%
\pgfpathlineto{\pgfqpoint{4.110348in}{1.007264in}}%
\pgfpathmoveto{\pgfqpoint{4.128512in}{1.042654in}}%
\pgfpathlineto{\pgfqpoint{4.128512in}{1.042654in}}%
\pgfpathlineto{\pgfqpoint{4.128512in}{1.054451in}}%
\pgfpathlineto{\pgfqpoint{4.146676in}{1.054451in}}%
\pgfpathlineto{\pgfqpoint{4.146676in}{1.042654in}}%
\pgfpathmoveto{\pgfqpoint{4.128512in}{1.054451in}}%
\pgfpathlineto{\pgfqpoint{4.128512in}{1.054451in}}%
\pgfpathlineto{\pgfqpoint{4.128512in}{1.066247in}}%
\pgfpathlineto{\pgfqpoint{4.146676in}{1.066247in}}%
\pgfpathlineto{\pgfqpoint{4.146676in}{1.054451in}}%
\pgfpathmoveto{\pgfqpoint{4.146676in}{1.054451in}}%
\pgfpathlineto{\pgfqpoint{4.146676in}{1.054451in}}%
\pgfpathlineto{\pgfqpoint{4.146676in}{1.066247in}}%
\pgfpathlineto{\pgfqpoint{4.164840in}{1.066247in}}%
\pgfpathlineto{\pgfqpoint{4.164840in}{1.054451in}}%
\pgfpathmoveto{\pgfqpoint{4.092184in}{1.089842in}}%
\pgfpathlineto{\pgfqpoint{4.092184in}{1.089842in}}%
\pgfpathlineto{\pgfqpoint{4.092184in}{1.101639in}}%
\pgfpathlineto{\pgfqpoint{4.110348in}{1.101639in}}%
\pgfpathlineto{\pgfqpoint{4.110348in}{1.089842in}}%
\pgfpathmoveto{\pgfqpoint{4.092184in}{1.101639in}}%
\pgfpathlineto{\pgfqpoint{4.092184in}{1.101639in}}%
\pgfpathlineto{\pgfqpoint{4.092184in}{1.113436in}}%
\pgfpathlineto{\pgfqpoint{4.110348in}{1.113436in}}%
\pgfpathlineto{\pgfqpoint{4.110348in}{1.101639in}}%
\pgfpathmoveto{\pgfqpoint{4.128512in}{1.066247in}}%
\pgfpathlineto{\pgfqpoint{4.128512in}{1.066247in}}%
\pgfpathlineto{\pgfqpoint{4.128512in}{1.078044in}}%
\pgfpathlineto{\pgfqpoint{4.146676in}{1.078044in}}%
\pgfpathlineto{\pgfqpoint{4.146676in}{1.066247in}}%
\pgfpathmoveto{\pgfqpoint{1.458397in}{3.390233in}}%
\pgfpathlineto{\pgfqpoint{1.458397in}{3.390233in}}%
\pgfpathlineto{\pgfqpoint{1.458397in}{3.396131in}}%
\pgfpathlineto{\pgfqpoint{1.467478in}{3.396131in}}%
\pgfpathlineto{\pgfqpoint{1.467478in}{3.390233in}}%
\pgfpathmoveto{\pgfqpoint{1.458397in}{3.396131in}}%
\pgfpathlineto{\pgfqpoint{1.458397in}{3.396131in}}%
\pgfpathlineto{\pgfqpoint{1.458397in}{3.402029in}}%
\pgfpathlineto{\pgfqpoint{1.467478in}{3.402029in}}%
\pgfpathlineto{\pgfqpoint{1.467478in}{3.396131in}}%
\pgfpathmoveto{\pgfqpoint{1.440233in}{3.402029in}}%
\pgfpathlineto{\pgfqpoint{1.440233in}{3.402029in}}%
\pgfpathlineto{\pgfqpoint{1.440233in}{3.407928in}}%
\pgfpathlineto{\pgfqpoint{1.449315in}{3.407928in}}%
\pgfpathlineto{\pgfqpoint{1.449315in}{3.402029in}}%
\pgfpathmoveto{\pgfqpoint{1.440233in}{3.407928in}}%
\pgfpathlineto{\pgfqpoint{1.440233in}{3.407928in}}%
\pgfpathlineto{\pgfqpoint{1.440233in}{3.413826in}}%
\pgfpathlineto{\pgfqpoint{1.449315in}{3.413826in}}%
\pgfpathlineto{\pgfqpoint{1.449315in}{3.407928in}}%
\pgfpathmoveto{\pgfqpoint{1.449315in}{3.402029in}}%
\pgfpathlineto{\pgfqpoint{1.449315in}{3.402029in}}%
\pgfpathlineto{\pgfqpoint{1.449315in}{3.407928in}}%
\pgfpathlineto{\pgfqpoint{1.458397in}{3.407928in}}%
\pgfpathlineto{\pgfqpoint{1.458397in}{3.402029in}}%
\pgfpathmoveto{\pgfqpoint{1.385743in}{3.449217in}}%
\pgfpathlineto{\pgfqpoint{1.385743in}{3.449217in}}%
\pgfpathlineto{\pgfqpoint{1.385743in}{3.455115in}}%
\pgfpathlineto{\pgfqpoint{1.394824in}{3.455115in}}%
\pgfpathlineto{\pgfqpoint{1.394824in}{3.449217in}}%
\pgfpathmoveto{\pgfqpoint{1.385743in}{3.455115in}}%
\pgfpathlineto{\pgfqpoint{1.385743in}{3.455115in}}%
\pgfpathlineto{\pgfqpoint{1.385743in}{3.461014in}}%
\pgfpathlineto{\pgfqpoint{1.394824in}{3.461014in}}%
\pgfpathlineto{\pgfqpoint{1.394824in}{3.455115in}}%
\pgfpathmoveto{\pgfqpoint{1.394824in}{3.449217in}}%
\pgfpathlineto{\pgfqpoint{1.394824in}{3.449217in}}%
\pgfpathlineto{\pgfqpoint{1.394824in}{3.455115in}}%
\pgfpathlineto{\pgfqpoint{1.403906in}{3.455115in}}%
\pgfpathlineto{\pgfqpoint{1.403906in}{3.449217in}}%
\pgfpathmoveto{\pgfqpoint{1.349416in}{3.484607in}}%
\pgfpathlineto{\pgfqpoint{1.349416in}{3.484607in}}%
\pgfpathlineto{\pgfqpoint{1.349416in}{3.490506in}}%
\pgfpathlineto{\pgfqpoint{1.358497in}{3.490506in}}%
\pgfpathlineto{\pgfqpoint{1.358497in}{3.484607in}}%
\pgfpathmoveto{\pgfqpoint{1.349416in}{3.490506in}}%
\pgfpathlineto{\pgfqpoint{1.349416in}{3.490506in}}%
\pgfpathlineto{\pgfqpoint{1.349416in}{3.496404in}}%
\pgfpathlineto{\pgfqpoint{1.358497in}{3.496404in}}%
\pgfpathlineto{\pgfqpoint{1.358497in}{3.490506in}}%
\pgfpathmoveto{\pgfqpoint{1.331252in}{3.496404in}}%
\pgfpathlineto{\pgfqpoint{1.331252in}{3.496404in}}%
\pgfpathlineto{\pgfqpoint{1.331252in}{3.502303in}}%
\pgfpathlineto{\pgfqpoint{1.340334in}{3.502303in}}%
\pgfpathlineto{\pgfqpoint{1.340334in}{3.496404in}}%
\pgfpathmoveto{\pgfqpoint{1.331252in}{3.502303in}}%
\pgfpathlineto{\pgfqpoint{1.331252in}{3.502303in}}%
\pgfpathlineto{\pgfqpoint{1.331252in}{3.508201in}}%
\pgfpathlineto{\pgfqpoint{1.340334in}{3.508201in}}%
\pgfpathlineto{\pgfqpoint{1.340334in}{3.502303in}}%
\pgfpathmoveto{\pgfqpoint{1.340334in}{3.496404in}}%
\pgfpathlineto{\pgfqpoint{1.340334in}{3.496404in}}%
\pgfpathlineto{\pgfqpoint{1.340334in}{3.502303in}}%
\pgfpathlineto{\pgfqpoint{1.349416in}{3.502303in}}%
\pgfpathlineto{\pgfqpoint{1.349416in}{3.496404in}}%
\pgfpathmoveto{\pgfqpoint{1.367579in}{3.472810in}}%
\pgfpathlineto{\pgfqpoint{1.367579in}{3.472810in}}%
\pgfpathlineto{\pgfqpoint{1.367579in}{3.478709in}}%
\pgfpathlineto{\pgfqpoint{1.376661in}{3.478709in}}%
\pgfpathlineto{\pgfqpoint{1.376661in}{3.472810in}}%
\pgfpathmoveto{\pgfqpoint{1.403906in}{3.437420in}}%
\pgfpathlineto{\pgfqpoint{1.403906in}{3.437420in}}%
\pgfpathlineto{\pgfqpoint{1.403906in}{3.443318in}}%
\pgfpathlineto{\pgfqpoint{1.412988in}{3.443318in}}%
\pgfpathlineto{\pgfqpoint{1.412988in}{3.437420in}}%
\pgfpathmoveto{\pgfqpoint{1.403906in}{3.443318in}}%
\pgfpathlineto{\pgfqpoint{1.403906in}{3.443318in}}%
\pgfpathlineto{\pgfqpoint{1.403906in}{3.449217in}}%
\pgfpathlineto{\pgfqpoint{1.412988in}{3.449217in}}%
\pgfpathlineto{\pgfqpoint{1.412988in}{3.443318in}}%
\pgfpathmoveto{\pgfqpoint{1.422070in}{3.425623in}}%
\pgfpathlineto{\pgfqpoint{1.422070in}{3.425623in}}%
\pgfpathlineto{\pgfqpoint{1.422070in}{3.431522in}}%
\pgfpathlineto{\pgfqpoint{1.431151in}{3.431522in}}%
\pgfpathlineto{\pgfqpoint{1.431151in}{3.425623in}}%
\pgfpathmoveto{\pgfqpoint{1.585549in}{3.272266in}}%
\pgfpathlineto{\pgfqpoint{1.585549in}{3.272266in}}%
\pgfpathlineto{\pgfqpoint{1.585549in}{3.278164in}}%
\pgfpathlineto{\pgfqpoint{1.594632in}{3.278164in}}%
\pgfpathlineto{\pgfqpoint{1.594632in}{3.272266in}}%
\pgfpathmoveto{\pgfqpoint{1.585549in}{3.278164in}}%
\pgfpathlineto{\pgfqpoint{1.585549in}{3.278164in}}%
\pgfpathlineto{\pgfqpoint{1.585549in}{3.284062in}}%
\pgfpathlineto{\pgfqpoint{1.594632in}{3.284062in}}%
\pgfpathlineto{\pgfqpoint{1.594632in}{3.278164in}}%
\pgfpathmoveto{\pgfqpoint{1.594632in}{3.272266in}}%
\pgfpathlineto{\pgfqpoint{1.594632in}{3.272266in}}%
\pgfpathlineto{\pgfqpoint{1.594632in}{3.278164in}}%
\pgfpathlineto{\pgfqpoint{1.603714in}{3.278164in}}%
\pgfpathlineto{\pgfqpoint{1.603714in}{3.272266in}}%
\pgfpathmoveto{\pgfqpoint{1.603714in}{3.260469in}}%
\pgfpathlineto{\pgfqpoint{1.603714in}{3.260469in}}%
\pgfpathlineto{\pgfqpoint{1.603714in}{3.266368in}}%
\pgfpathlineto{\pgfqpoint{1.612797in}{3.266368in}}%
\pgfpathlineto{\pgfqpoint{1.612797in}{3.260469in}}%
\pgfpathmoveto{\pgfqpoint{1.603714in}{3.266368in}}%
\pgfpathlineto{\pgfqpoint{1.603714in}{3.266368in}}%
\pgfpathlineto{\pgfqpoint{1.603714in}{3.272266in}}%
\pgfpathlineto{\pgfqpoint{1.612797in}{3.272266in}}%
\pgfpathlineto{\pgfqpoint{1.612797in}{3.266368in}}%
\pgfpathmoveto{\pgfqpoint{1.612797in}{3.260469in}}%
\pgfpathlineto{\pgfqpoint{1.612797in}{3.260469in}}%
\pgfpathlineto{\pgfqpoint{1.612797in}{3.266368in}}%
\pgfpathlineto{\pgfqpoint{1.621879in}{3.266368in}}%
\pgfpathlineto{\pgfqpoint{1.621879in}{3.260469in}}%
\pgfpathmoveto{\pgfqpoint{1.567384in}{3.295859in}}%
\pgfpathlineto{\pgfqpoint{1.567384in}{3.295859in}}%
\pgfpathlineto{\pgfqpoint{1.567384in}{3.301757in}}%
\pgfpathlineto{\pgfqpoint{1.576467in}{3.301757in}}%
\pgfpathlineto{\pgfqpoint{1.576467in}{3.295859in}}%
\pgfpathmoveto{\pgfqpoint{1.549220in}{3.307655in}}%
\pgfpathlineto{\pgfqpoint{1.549220in}{3.307655in}}%
\pgfpathlineto{\pgfqpoint{1.549220in}{3.313554in}}%
\pgfpathlineto{\pgfqpoint{1.558302in}{3.313554in}}%
\pgfpathlineto{\pgfqpoint{1.558302in}{3.307655in}}%
\pgfpathmoveto{\pgfqpoint{1.549220in}{3.313554in}}%
\pgfpathlineto{\pgfqpoint{1.549220in}{3.313554in}}%
\pgfpathlineto{\pgfqpoint{1.549220in}{3.319452in}}%
\pgfpathlineto{\pgfqpoint{1.558302in}{3.319452in}}%
\pgfpathlineto{\pgfqpoint{1.558302in}{3.313554in}}%
\pgfpathmoveto{\pgfqpoint{1.558302in}{3.307655in}}%
\pgfpathlineto{\pgfqpoint{1.558302in}{3.307655in}}%
\pgfpathlineto{\pgfqpoint{1.558302in}{3.313554in}}%
\pgfpathlineto{\pgfqpoint{1.567384in}{3.313554in}}%
\pgfpathlineto{\pgfqpoint{1.567384in}{3.307655in}}%
\pgfpathmoveto{\pgfqpoint{1.585549in}{3.284062in}}%
\pgfpathlineto{\pgfqpoint{1.585549in}{3.284062in}}%
\pgfpathlineto{\pgfqpoint{1.585549in}{3.289961in}}%
\pgfpathlineto{\pgfqpoint{1.594632in}{3.289961in}}%
\pgfpathlineto{\pgfqpoint{1.594632in}{3.284062in}}%
\pgfpathmoveto{\pgfqpoint{1.494725in}{3.354842in}}%
\pgfpathlineto{\pgfqpoint{1.494725in}{3.354842in}}%
\pgfpathlineto{\pgfqpoint{1.494725in}{3.360740in}}%
\pgfpathlineto{\pgfqpoint{1.503807in}{3.360740in}}%
\pgfpathlineto{\pgfqpoint{1.503807in}{3.354842in}}%
\pgfpathmoveto{\pgfqpoint{1.494725in}{3.360740in}}%
\pgfpathlineto{\pgfqpoint{1.494725in}{3.360740in}}%
\pgfpathlineto{\pgfqpoint{1.494725in}{3.366639in}}%
\pgfpathlineto{\pgfqpoint{1.503807in}{3.366639in}}%
\pgfpathlineto{\pgfqpoint{1.503807in}{3.360740in}}%
\pgfpathmoveto{\pgfqpoint{1.503807in}{3.354842in}}%
\pgfpathlineto{\pgfqpoint{1.503807in}{3.354842in}}%
\pgfpathlineto{\pgfqpoint{1.503807in}{3.360740in}}%
\pgfpathlineto{\pgfqpoint{1.512890in}{3.360740in}}%
\pgfpathlineto{\pgfqpoint{1.512890in}{3.354842in}}%
\pgfpathmoveto{\pgfqpoint{1.512890in}{3.343045in}}%
\pgfpathlineto{\pgfqpoint{1.512890in}{3.343045in}}%
\pgfpathlineto{\pgfqpoint{1.512890in}{3.348944in}}%
\pgfpathlineto{\pgfqpoint{1.521972in}{3.348944in}}%
\pgfpathlineto{\pgfqpoint{1.521972in}{3.343045in}}%
\pgfpathmoveto{\pgfqpoint{1.512890in}{3.348944in}}%
\pgfpathlineto{\pgfqpoint{1.512890in}{3.348944in}}%
\pgfpathlineto{\pgfqpoint{1.512890in}{3.354842in}}%
\pgfpathlineto{\pgfqpoint{1.521972in}{3.354842in}}%
\pgfpathlineto{\pgfqpoint{1.521972in}{3.348944in}}%
\pgfpathmoveto{\pgfqpoint{1.531055in}{3.331248in}}%
\pgfpathlineto{\pgfqpoint{1.531055in}{3.331248in}}%
\pgfpathlineto{\pgfqpoint{1.531055in}{3.337147in}}%
\pgfpathlineto{\pgfqpoint{1.540137in}{3.337147in}}%
\pgfpathlineto{\pgfqpoint{1.540137in}{3.331248in}}%
\pgfpathmoveto{\pgfqpoint{1.476560in}{3.378436in}}%
\pgfpathlineto{\pgfqpoint{1.476560in}{3.378436in}}%
\pgfpathlineto{\pgfqpoint{1.476560in}{3.384334in}}%
\pgfpathlineto{\pgfqpoint{1.485642in}{3.384334in}}%
\pgfpathlineto{\pgfqpoint{1.485642in}{3.378436in}}%
\pgfpathmoveto{\pgfqpoint{1.640043in}{3.225079in}}%
\pgfpathlineto{\pgfqpoint{1.640043in}{3.225079in}}%
\pgfpathlineto{\pgfqpoint{1.640043in}{3.230978in}}%
\pgfpathlineto{\pgfqpoint{1.649125in}{3.230978in}}%
\pgfpathlineto{\pgfqpoint{1.649125in}{3.225079in}}%
\pgfpathmoveto{\pgfqpoint{1.640043in}{3.230978in}}%
\pgfpathlineto{\pgfqpoint{1.640043in}{3.230978in}}%
\pgfpathlineto{\pgfqpoint{1.640043in}{3.236876in}}%
\pgfpathlineto{\pgfqpoint{1.649125in}{3.236876in}}%
\pgfpathlineto{\pgfqpoint{1.649125in}{3.230978in}}%
\pgfpathmoveto{\pgfqpoint{1.649125in}{3.225079in}}%
\pgfpathlineto{\pgfqpoint{1.649125in}{3.225079in}}%
\pgfpathlineto{\pgfqpoint{1.649125in}{3.230978in}}%
\pgfpathlineto{\pgfqpoint{1.658206in}{3.230978in}}%
\pgfpathlineto{\pgfqpoint{1.658206in}{3.225079in}}%
\pgfpathmoveto{\pgfqpoint{1.676370in}{3.201485in}}%
\pgfpathlineto{\pgfqpoint{1.676370in}{3.201485in}}%
\pgfpathlineto{\pgfqpoint{1.676370in}{3.207383in}}%
\pgfpathlineto{\pgfqpoint{1.685452in}{3.207383in}}%
\pgfpathlineto{\pgfqpoint{1.685452in}{3.201485in}}%
\pgfpathmoveto{\pgfqpoint{1.658206in}{3.213282in}}%
\pgfpathlineto{\pgfqpoint{1.658206in}{3.213282in}}%
\pgfpathlineto{\pgfqpoint{1.658206in}{3.219181in}}%
\pgfpathlineto{\pgfqpoint{1.667288in}{3.219181in}}%
\pgfpathlineto{\pgfqpoint{1.667288in}{3.213282in}}%
\pgfpathmoveto{\pgfqpoint{1.658206in}{3.219181in}}%
\pgfpathlineto{\pgfqpoint{1.658206in}{3.219181in}}%
\pgfpathlineto{\pgfqpoint{1.658206in}{3.225079in}}%
\pgfpathlineto{\pgfqpoint{1.667288in}{3.225079in}}%
\pgfpathlineto{\pgfqpoint{1.667288in}{3.219181in}}%
\pgfpathmoveto{\pgfqpoint{1.667288in}{3.213282in}}%
\pgfpathlineto{\pgfqpoint{1.667288in}{3.213282in}}%
\pgfpathlineto{\pgfqpoint{1.667288in}{3.219181in}}%
\pgfpathlineto{\pgfqpoint{1.676370in}{3.219181in}}%
\pgfpathlineto{\pgfqpoint{1.676370in}{3.213282in}}%
\pgfpathmoveto{\pgfqpoint{1.694534in}{3.177890in}}%
\pgfpathlineto{\pgfqpoint{1.694534in}{3.177890in}}%
\pgfpathlineto{\pgfqpoint{1.694534in}{3.183789in}}%
\pgfpathlineto{\pgfqpoint{1.703616in}{3.183789in}}%
\pgfpathlineto{\pgfqpoint{1.703616in}{3.177890in}}%
\pgfpathmoveto{\pgfqpoint{1.694534in}{3.183789in}}%
\pgfpathlineto{\pgfqpoint{1.694534in}{3.183789in}}%
\pgfpathlineto{\pgfqpoint{1.694534in}{3.189687in}}%
\pgfpathlineto{\pgfqpoint{1.703616in}{3.189687in}}%
\pgfpathlineto{\pgfqpoint{1.703616in}{3.183789in}}%
\pgfpathmoveto{\pgfqpoint{1.703616in}{3.177890in}}%
\pgfpathlineto{\pgfqpoint{1.703616in}{3.177890in}}%
\pgfpathlineto{\pgfqpoint{1.703616in}{3.183789in}}%
\pgfpathlineto{\pgfqpoint{1.712698in}{3.183789in}}%
\pgfpathlineto{\pgfqpoint{1.712698in}{3.177890in}}%
\pgfpathmoveto{\pgfqpoint{1.712698in}{3.166093in}}%
\pgfpathlineto{\pgfqpoint{1.712698in}{3.166093in}}%
\pgfpathlineto{\pgfqpoint{1.712698in}{3.171992in}}%
\pgfpathlineto{\pgfqpoint{1.721780in}{3.171992in}}%
\pgfpathlineto{\pgfqpoint{1.721780in}{3.166093in}}%
\pgfpathmoveto{\pgfqpoint{1.712698in}{3.171992in}}%
\pgfpathlineto{\pgfqpoint{1.712698in}{3.171992in}}%
\pgfpathlineto{\pgfqpoint{1.712698in}{3.177890in}}%
\pgfpathlineto{\pgfqpoint{1.721780in}{3.177890in}}%
\pgfpathlineto{\pgfqpoint{1.721780in}{3.171992in}}%
\pgfpathmoveto{\pgfqpoint{1.721780in}{3.166093in}}%
\pgfpathlineto{\pgfqpoint{1.721780in}{3.166093in}}%
\pgfpathlineto{\pgfqpoint{1.721780in}{3.171992in}}%
\pgfpathlineto{\pgfqpoint{1.730861in}{3.171992in}}%
\pgfpathlineto{\pgfqpoint{1.730861in}{3.166093in}}%
\pgfpathmoveto{\pgfqpoint{1.730861in}{3.154296in}}%
\pgfpathlineto{\pgfqpoint{1.730861in}{3.154296in}}%
\pgfpathlineto{\pgfqpoint{1.730861in}{3.160194in}}%
\pgfpathlineto{\pgfqpoint{1.739943in}{3.160194in}}%
\pgfpathlineto{\pgfqpoint{1.739943in}{3.154296in}}%
\pgfpathmoveto{\pgfqpoint{1.749025in}{3.142498in}}%
\pgfpathlineto{\pgfqpoint{1.749025in}{3.142498in}}%
\pgfpathlineto{\pgfqpoint{1.749025in}{3.148397in}}%
\pgfpathlineto{\pgfqpoint{1.758107in}{3.148397in}}%
\pgfpathlineto{\pgfqpoint{1.758107in}{3.142498in}}%
\pgfpathmoveto{\pgfqpoint{1.694534in}{3.189687in}}%
\pgfpathlineto{\pgfqpoint{1.694534in}{3.189687in}}%
\pgfpathlineto{\pgfqpoint{1.694534in}{3.195586in}}%
\pgfpathlineto{\pgfqpoint{1.703616in}{3.195586in}}%
\pgfpathlineto{\pgfqpoint{1.703616in}{3.189687in}}%
\pgfpathmoveto{\pgfqpoint{1.621879in}{3.248673in}}%
\pgfpathlineto{\pgfqpoint{1.621879in}{3.248673in}}%
\pgfpathlineto{\pgfqpoint{1.621879in}{3.254571in}}%
\pgfpathlineto{\pgfqpoint{1.630961in}{3.254571in}}%
\pgfpathlineto{\pgfqpoint{1.630961in}{3.248673in}}%
\pgfpathmoveto{\pgfqpoint{1.640043in}{3.236876in}}%
\pgfpathlineto{\pgfqpoint{1.640043in}{3.236876in}}%
\pgfpathlineto{\pgfqpoint{1.640043in}{3.242775in}}%
\pgfpathlineto{\pgfqpoint{1.649125in}{3.242775in}}%
\pgfpathlineto{\pgfqpoint{1.649125in}{3.236876in}}%
\pgfpathmoveto{\pgfqpoint{1.894335in}{3.012733in}}%
\pgfpathlineto{\pgfqpoint{1.894335in}{3.012733in}}%
\pgfpathlineto{\pgfqpoint{1.894335in}{3.018632in}}%
\pgfpathlineto{\pgfqpoint{1.903417in}{3.018632in}}%
\pgfpathlineto{\pgfqpoint{1.903417in}{3.012733in}}%
\pgfpathmoveto{\pgfqpoint{1.894335in}{3.018632in}}%
\pgfpathlineto{\pgfqpoint{1.894335in}{3.018632in}}%
\pgfpathlineto{\pgfqpoint{1.894335in}{3.024530in}}%
\pgfpathlineto{\pgfqpoint{1.903417in}{3.024530in}}%
\pgfpathlineto{\pgfqpoint{1.903417in}{3.018632in}}%
\pgfpathmoveto{\pgfqpoint{1.876171in}{3.024530in}}%
\pgfpathlineto{\pgfqpoint{1.876171in}{3.024530in}}%
\pgfpathlineto{\pgfqpoint{1.876171in}{3.030429in}}%
\pgfpathlineto{\pgfqpoint{1.885253in}{3.030429in}}%
\pgfpathlineto{\pgfqpoint{1.885253in}{3.024530in}}%
\pgfpathmoveto{\pgfqpoint{1.876171in}{3.030429in}}%
\pgfpathlineto{\pgfqpoint{1.876171in}{3.030429in}}%
\pgfpathlineto{\pgfqpoint{1.876171in}{3.036327in}}%
\pgfpathlineto{\pgfqpoint{1.885253in}{3.036327in}}%
\pgfpathlineto{\pgfqpoint{1.885253in}{3.030429in}}%
\pgfpathmoveto{\pgfqpoint{1.885253in}{3.024530in}}%
\pgfpathlineto{\pgfqpoint{1.885253in}{3.024530in}}%
\pgfpathlineto{\pgfqpoint{1.885253in}{3.030429in}}%
\pgfpathlineto{\pgfqpoint{1.894335in}{3.030429in}}%
\pgfpathlineto{\pgfqpoint{1.894335in}{3.024530in}}%
\pgfpathmoveto{\pgfqpoint{1.821680in}{3.071718in}}%
\pgfpathlineto{\pgfqpoint{1.821680in}{3.071718in}}%
\pgfpathlineto{\pgfqpoint{1.821680in}{3.077616in}}%
\pgfpathlineto{\pgfqpoint{1.830762in}{3.077616in}}%
\pgfpathlineto{\pgfqpoint{1.830762in}{3.071718in}}%
\pgfpathmoveto{\pgfqpoint{1.821680in}{3.077616in}}%
\pgfpathlineto{\pgfqpoint{1.821680in}{3.077616in}}%
\pgfpathlineto{\pgfqpoint{1.821680in}{3.083514in}}%
\pgfpathlineto{\pgfqpoint{1.830762in}{3.083514in}}%
\pgfpathlineto{\pgfqpoint{1.830762in}{3.077616in}}%
\pgfpathmoveto{\pgfqpoint{1.830762in}{3.071718in}}%
\pgfpathlineto{\pgfqpoint{1.830762in}{3.071718in}}%
\pgfpathlineto{\pgfqpoint{1.830762in}{3.077616in}}%
\pgfpathlineto{\pgfqpoint{1.839844in}{3.077616in}}%
\pgfpathlineto{\pgfqpoint{1.839844in}{3.071718in}}%
\pgfpathmoveto{\pgfqpoint{1.785353in}{3.107108in}}%
\pgfpathlineto{\pgfqpoint{1.785353in}{3.107108in}}%
\pgfpathlineto{\pgfqpoint{1.785353in}{3.113006in}}%
\pgfpathlineto{\pgfqpoint{1.794434in}{3.113006in}}%
\pgfpathlineto{\pgfqpoint{1.794434in}{3.107108in}}%
\pgfpathmoveto{\pgfqpoint{1.785353in}{3.113006in}}%
\pgfpathlineto{\pgfqpoint{1.785353in}{3.113006in}}%
\pgfpathlineto{\pgfqpoint{1.785353in}{3.118905in}}%
\pgfpathlineto{\pgfqpoint{1.794434in}{3.118905in}}%
\pgfpathlineto{\pgfqpoint{1.794434in}{3.113006in}}%
\pgfpathmoveto{\pgfqpoint{1.767189in}{3.118905in}}%
\pgfpathlineto{\pgfqpoint{1.767189in}{3.118905in}}%
\pgfpathlineto{\pgfqpoint{1.767189in}{3.124803in}}%
\pgfpathlineto{\pgfqpoint{1.776271in}{3.124803in}}%
\pgfpathlineto{\pgfqpoint{1.776271in}{3.118905in}}%
\pgfpathmoveto{\pgfqpoint{1.767189in}{3.124803in}}%
\pgfpathlineto{\pgfqpoint{1.767189in}{3.124803in}}%
\pgfpathlineto{\pgfqpoint{1.767189in}{3.130702in}}%
\pgfpathlineto{\pgfqpoint{1.776271in}{3.130702in}}%
\pgfpathlineto{\pgfqpoint{1.776271in}{3.124803in}}%
\pgfpathmoveto{\pgfqpoint{1.776271in}{3.118905in}}%
\pgfpathlineto{\pgfqpoint{1.776271in}{3.118905in}}%
\pgfpathlineto{\pgfqpoint{1.776271in}{3.124803in}}%
\pgfpathlineto{\pgfqpoint{1.785353in}{3.124803in}}%
\pgfpathlineto{\pgfqpoint{1.785353in}{3.118905in}}%
\pgfpathmoveto{\pgfqpoint{1.803516in}{3.095311in}}%
\pgfpathlineto{\pgfqpoint{1.803516in}{3.095311in}}%
\pgfpathlineto{\pgfqpoint{1.803516in}{3.101210in}}%
\pgfpathlineto{\pgfqpoint{1.812598in}{3.101210in}}%
\pgfpathlineto{\pgfqpoint{1.812598in}{3.095311in}}%
\pgfpathmoveto{\pgfqpoint{1.839844in}{3.059921in}}%
\pgfpathlineto{\pgfqpoint{1.839844in}{3.059921in}}%
\pgfpathlineto{\pgfqpoint{1.839844in}{3.065819in}}%
\pgfpathlineto{\pgfqpoint{1.848926in}{3.065819in}}%
\pgfpathlineto{\pgfqpoint{1.848926in}{3.059921in}}%
\pgfpathmoveto{\pgfqpoint{1.839844in}{3.065819in}}%
\pgfpathlineto{\pgfqpoint{1.839844in}{3.065819in}}%
\pgfpathlineto{\pgfqpoint{1.839844in}{3.071718in}}%
\pgfpathlineto{\pgfqpoint{1.848926in}{3.071718in}}%
\pgfpathlineto{\pgfqpoint{1.848926in}{3.065819in}}%
\pgfpathmoveto{\pgfqpoint{1.858008in}{3.048124in}}%
\pgfpathlineto{\pgfqpoint{1.858008in}{3.048124in}}%
\pgfpathlineto{\pgfqpoint{1.858008in}{3.054023in}}%
\pgfpathlineto{\pgfqpoint{1.867089in}{3.054023in}}%
\pgfpathlineto{\pgfqpoint{1.867089in}{3.048124in}}%
\pgfpathmoveto{\pgfqpoint{1.966993in}{2.941952in}}%
\pgfpathlineto{\pgfqpoint{1.966993in}{2.941952in}}%
\pgfpathlineto{\pgfqpoint{1.966993in}{2.947850in}}%
\pgfpathlineto{\pgfqpoint{1.976075in}{2.947850in}}%
\pgfpathlineto{\pgfqpoint{1.976075in}{2.941952in}}%
\pgfpathmoveto{\pgfqpoint{1.966993in}{2.947850in}}%
\pgfpathlineto{\pgfqpoint{1.966993in}{2.947850in}}%
\pgfpathlineto{\pgfqpoint{1.966993in}{2.953749in}}%
\pgfpathlineto{\pgfqpoint{1.976075in}{2.953749in}}%
\pgfpathlineto{\pgfqpoint{1.976075in}{2.947850in}}%
\pgfpathmoveto{\pgfqpoint{1.976075in}{2.941952in}}%
\pgfpathlineto{\pgfqpoint{1.976075in}{2.941952in}}%
\pgfpathlineto{\pgfqpoint{1.976075in}{2.947850in}}%
\pgfpathlineto{\pgfqpoint{1.985158in}{2.947850in}}%
\pgfpathlineto{\pgfqpoint{1.985158in}{2.941952in}}%
\pgfpathmoveto{\pgfqpoint{2.021487in}{2.894764in}}%
\pgfpathlineto{\pgfqpoint{2.021487in}{2.894764in}}%
\pgfpathlineto{\pgfqpoint{2.021487in}{2.900662in}}%
\pgfpathlineto{\pgfqpoint{2.030569in}{2.900662in}}%
\pgfpathlineto{\pgfqpoint{2.030569in}{2.894764in}}%
\pgfpathmoveto{\pgfqpoint{2.021487in}{2.900662in}}%
\pgfpathlineto{\pgfqpoint{2.021487in}{2.900662in}}%
\pgfpathlineto{\pgfqpoint{2.021487in}{2.906561in}}%
\pgfpathlineto{\pgfqpoint{2.030569in}{2.906561in}}%
\pgfpathlineto{\pgfqpoint{2.030569in}{2.900662in}}%
\pgfpathmoveto{\pgfqpoint{2.030569in}{2.894764in}}%
\pgfpathlineto{\pgfqpoint{2.030569in}{2.894764in}}%
\pgfpathlineto{\pgfqpoint{2.030569in}{2.900662in}}%
\pgfpathlineto{\pgfqpoint{2.039652in}{2.900662in}}%
\pgfpathlineto{\pgfqpoint{2.039652in}{2.894764in}}%
\pgfpathmoveto{\pgfqpoint{2.039652in}{2.882966in}}%
\pgfpathlineto{\pgfqpoint{2.039652in}{2.882966in}}%
\pgfpathlineto{\pgfqpoint{2.039652in}{2.888865in}}%
\pgfpathlineto{\pgfqpoint{2.048734in}{2.888865in}}%
\pgfpathlineto{\pgfqpoint{2.048734in}{2.882966in}}%
\pgfpathmoveto{\pgfqpoint{2.039652in}{2.888865in}}%
\pgfpathlineto{\pgfqpoint{2.039652in}{2.888865in}}%
\pgfpathlineto{\pgfqpoint{2.039652in}{2.894764in}}%
\pgfpathlineto{\pgfqpoint{2.048734in}{2.894764in}}%
\pgfpathlineto{\pgfqpoint{2.048734in}{2.888865in}}%
\pgfpathmoveto{\pgfqpoint{2.048734in}{2.882966in}}%
\pgfpathlineto{\pgfqpoint{2.048734in}{2.882966in}}%
\pgfpathlineto{\pgfqpoint{2.048734in}{2.888865in}}%
\pgfpathlineto{\pgfqpoint{2.057816in}{2.888865in}}%
\pgfpathlineto{\pgfqpoint{2.057816in}{2.882966in}}%
\pgfpathmoveto{\pgfqpoint{2.003322in}{2.918358in}}%
\pgfpathlineto{\pgfqpoint{2.003322in}{2.918358in}}%
\pgfpathlineto{\pgfqpoint{2.003322in}{2.924256in}}%
\pgfpathlineto{\pgfqpoint{2.012405in}{2.924256in}}%
\pgfpathlineto{\pgfqpoint{2.012405in}{2.918358in}}%
\pgfpathmoveto{\pgfqpoint{1.985158in}{2.930155in}}%
\pgfpathlineto{\pgfqpoint{1.985158in}{2.930155in}}%
\pgfpathlineto{\pgfqpoint{1.985158in}{2.936053in}}%
\pgfpathlineto{\pgfqpoint{1.994240in}{2.936053in}}%
\pgfpathlineto{\pgfqpoint{1.994240in}{2.930155in}}%
\pgfpathmoveto{\pgfqpoint{1.985158in}{2.936053in}}%
\pgfpathlineto{\pgfqpoint{1.985158in}{2.936053in}}%
\pgfpathlineto{\pgfqpoint{1.985158in}{2.941952in}}%
\pgfpathlineto{\pgfqpoint{1.994240in}{2.941952in}}%
\pgfpathlineto{\pgfqpoint{1.994240in}{2.936053in}}%
\pgfpathmoveto{\pgfqpoint{1.994240in}{2.930155in}}%
\pgfpathlineto{\pgfqpoint{1.994240in}{2.930155in}}%
\pgfpathlineto{\pgfqpoint{1.994240in}{2.936053in}}%
\pgfpathlineto{\pgfqpoint{2.003322in}{2.936053in}}%
\pgfpathlineto{\pgfqpoint{2.003322in}{2.930155in}}%
\pgfpathmoveto{\pgfqpoint{2.021487in}{2.906561in}}%
\pgfpathlineto{\pgfqpoint{2.021487in}{2.906561in}}%
\pgfpathlineto{\pgfqpoint{2.021487in}{2.912459in}}%
\pgfpathlineto{\pgfqpoint{2.030569in}{2.912459in}}%
\pgfpathlineto{\pgfqpoint{2.030569in}{2.906561in}}%
\pgfpathmoveto{\pgfqpoint{1.930663in}{2.977343in}}%
\pgfpathlineto{\pgfqpoint{1.930663in}{2.977343in}}%
\pgfpathlineto{\pgfqpoint{1.930663in}{2.983241in}}%
\pgfpathlineto{\pgfqpoint{1.939746in}{2.983241in}}%
\pgfpathlineto{\pgfqpoint{1.939746in}{2.977343in}}%
\pgfpathmoveto{\pgfqpoint{1.930663in}{2.983241in}}%
\pgfpathlineto{\pgfqpoint{1.930663in}{2.983241in}}%
\pgfpathlineto{\pgfqpoint{1.930663in}{2.989140in}}%
\pgfpathlineto{\pgfqpoint{1.939746in}{2.989140in}}%
\pgfpathlineto{\pgfqpoint{1.939746in}{2.983241in}}%
\pgfpathmoveto{\pgfqpoint{1.939746in}{2.977343in}}%
\pgfpathlineto{\pgfqpoint{1.939746in}{2.977343in}}%
\pgfpathlineto{\pgfqpoint{1.939746in}{2.983241in}}%
\pgfpathlineto{\pgfqpoint{1.948828in}{2.983241in}}%
\pgfpathlineto{\pgfqpoint{1.948828in}{2.977343in}}%
\pgfpathmoveto{\pgfqpoint{1.948828in}{2.965546in}}%
\pgfpathlineto{\pgfqpoint{1.948828in}{2.965546in}}%
\pgfpathlineto{\pgfqpoint{1.948828in}{2.971444in}}%
\pgfpathlineto{\pgfqpoint{1.957910in}{2.971444in}}%
\pgfpathlineto{\pgfqpoint{1.957910in}{2.965546in}}%
\pgfpathmoveto{\pgfqpoint{1.948828in}{2.971444in}}%
\pgfpathlineto{\pgfqpoint{1.948828in}{2.971444in}}%
\pgfpathlineto{\pgfqpoint{1.948828in}{2.977343in}}%
\pgfpathlineto{\pgfqpoint{1.957910in}{2.977343in}}%
\pgfpathlineto{\pgfqpoint{1.957910in}{2.971444in}}%
\pgfpathmoveto{\pgfqpoint{1.966993in}{2.953749in}}%
\pgfpathlineto{\pgfqpoint{1.966993in}{2.953749in}}%
\pgfpathlineto{\pgfqpoint{1.966993in}{2.959647in}}%
\pgfpathlineto{\pgfqpoint{1.976075in}{2.959647in}}%
\pgfpathlineto{\pgfqpoint{1.976075in}{2.953749in}}%
\pgfpathmoveto{\pgfqpoint{1.912499in}{3.000937in}}%
\pgfpathlineto{\pgfqpoint{1.912499in}{3.000937in}}%
\pgfpathlineto{\pgfqpoint{1.912499in}{3.006835in}}%
\pgfpathlineto{\pgfqpoint{1.921581in}{3.006835in}}%
\pgfpathlineto{\pgfqpoint{1.921581in}{3.000937in}}%
\pgfpathmoveto{\pgfqpoint{2.112306in}{2.823984in}}%
\pgfpathlineto{\pgfqpoint{2.112306in}{2.823984in}}%
\pgfpathlineto{\pgfqpoint{2.112306in}{2.829882in}}%
\pgfpathlineto{\pgfqpoint{2.121387in}{2.829882in}}%
\pgfpathlineto{\pgfqpoint{2.121387in}{2.823984in}}%
\pgfpathmoveto{\pgfqpoint{2.112306in}{2.829882in}}%
\pgfpathlineto{\pgfqpoint{2.112306in}{2.829882in}}%
\pgfpathlineto{\pgfqpoint{2.112306in}{2.835780in}}%
\pgfpathlineto{\pgfqpoint{2.121387in}{2.835780in}}%
\pgfpathlineto{\pgfqpoint{2.121387in}{2.829882in}}%
\pgfpathmoveto{\pgfqpoint{2.094143in}{2.835780in}}%
\pgfpathlineto{\pgfqpoint{2.094143in}{2.835780in}}%
\pgfpathlineto{\pgfqpoint{2.094143in}{2.841678in}}%
\pgfpathlineto{\pgfqpoint{2.103224in}{2.841678in}}%
\pgfpathlineto{\pgfqpoint{2.103224in}{2.835780in}}%
\pgfpathmoveto{\pgfqpoint{2.094143in}{2.841678in}}%
\pgfpathlineto{\pgfqpoint{2.094143in}{2.841678in}}%
\pgfpathlineto{\pgfqpoint{2.094143in}{2.847576in}}%
\pgfpathlineto{\pgfqpoint{2.103224in}{2.847576in}}%
\pgfpathlineto{\pgfqpoint{2.103224in}{2.841678in}}%
\pgfpathmoveto{\pgfqpoint{2.103224in}{2.835780in}}%
\pgfpathlineto{\pgfqpoint{2.103224in}{2.835780in}}%
\pgfpathlineto{\pgfqpoint{2.103224in}{2.841678in}}%
\pgfpathlineto{\pgfqpoint{2.112306in}{2.841678in}}%
\pgfpathlineto{\pgfqpoint{2.112306in}{2.835780in}}%
\pgfpathmoveto{\pgfqpoint{2.148632in}{2.788595in}}%
\pgfpathlineto{\pgfqpoint{2.148632in}{2.788595in}}%
\pgfpathlineto{\pgfqpoint{2.148632in}{2.794493in}}%
\pgfpathlineto{\pgfqpoint{2.157714in}{2.794493in}}%
\pgfpathlineto{\pgfqpoint{2.157714in}{2.788595in}}%
\pgfpathmoveto{\pgfqpoint{2.148632in}{2.794493in}}%
\pgfpathlineto{\pgfqpoint{2.148632in}{2.794493in}}%
\pgfpathlineto{\pgfqpoint{2.148632in}{2.800391in}}%
\pgfpathlineto{\pgfqpoint{2.157714in}{2.800391in}}%
\pgfpathlineto{\pgfqpoint{2.157714in}{2.794493in}}%
\pgfpathmoveto{\pgfqpoint{2.157714in}{2.788595in}}%
\pgfpathlineto{\pgfqpoint{2.157714in}{2.788595in}}%
\pgfpathlineto{\pgfqpoint{2.157714in}{2.794493in}}%
\pgfpathlineto{\pgfqpoint{2.166795in}{2.794493in}}%
\pgfpathlineto{\pgfqpoint{2.166795in}{2.788595in}}%
\pgfpathmoveto{\pgfqpoint{2.166795in}{2.776799in}}%
\pgfpathlineto{\pgfqpoint{2.166795in}{2.776799in}}%
\pgfpathlineto{\pgfqpoint{2.166795in}{2.782697in}}%
\pgfpathlineto{\pgfqpoint{2.175877in}{2.782697in}}%
\pgfpathlineto{\pgfqpoint{2.175877in}{2.776799in}}%
\pgfpathmoveto{\pgfqpoint{2.166795in}{2.782697in}}%
\pgfpathlineto{\pgfqpoint{2.166795in}{2.782697in}}%
\pgfpathlineto{\pgfqpoint{2.166795in}{2.788595in}}%
\pgfpathlineto{\pgfqpoint{2.175877in}{2.788595in}}%
\pgfpathlineto{\pgfqpoint{2.175877in}{2.782697in}}%
\pgfpathmoveto{\pgfqpoint{2.184958in}{2.765003in}}%
\pgfpathlineto{\pgfqpoint{2.184958in}{2.765003in}}%
\pgfpathlineto{\pgfqpoint{2.184958in}{2.770901in}}%
\pgfpathlineto{\pgfqpoint{2.194040in}{2.770901in}}%
\pgfpathlineto{\pgfqpoint{2.194040in}{2.765003in}}%
\pgfpathmoveto{\pgfqpoint{2.130469in}{2.812188in}}%
\pgfpathlineto{\pgfqpoint{2.130469in}{2.812188in}}%
\pgfpathlineto{\pgfqpoint{2.130469in}{2.818086in}}%
\pgfpathlineto{\pgfqpoint{2.139550in}{2.818086in}}%
\pgfpathlineto{\pgfqpoint{2.139550in}{2.812188in}}%
\pgfpathmoveto{\pgfqpoint{2.057816in}{2.871169in}}%
\pgfpathlineto{\pgfqpoint{2.057816in}{2.871169in}}%
\pgfpathlineto{\pgfqpoint{2.057816in}{2.877068in}}%
\pgfpathlineto{\pgfqpoint{2.066898in}{2.877068in}}%
\pgfpathlineto{\pgfqpoint{2.066898in}{2.871169in}}%
\pgfpathmoveto{\pgfqpoint{2.075979in}{2.859372in}}%
\pgfpathlineto{\pgfqpoint{2.075979in}{2.859372in}}%
\pgfpathlineto{\pgfqpoint{2.075979in}{2.865271in}}%
\pgfpathlineto{\pgfqpoint{2.085061in}{2.865271in}}%
\pgfpathlineto{\pgfqpoint{2.085061in}{2.859372in}}%
\pgfpathmoveto{\pgfqpoint{2.330274in}{2.635233in}}%
\pgfpathlineto{\pgfqpoint{2.330274in}{2.635233in}}%
\pgfpathlineto{\pgfqpoint{2.330274in}{2.641132in}}%
\pgfpathlineto{\pgfqpoint{2.339357in}{2.641132in}}%
\pgfpathlineto{\pgfqpoint{2.339357in}{2.635233in}}%
\pgfpathmoveto{\pgfqpoint{2.330274in}{2.641132in}}%
\pgfpathlineto{\pgfqpoint{2.330274in}{2.641132in}}%
\pgfpathlineto{\pgfqpoint{2.330274in}{2.647030in}}%
\pgfpathlineto{\pgfqpoint{2.339357in}{2.647030in}}%
\pgfpathlineto{\pgfqpoint{2.339357in}{2.641132in}}%
\pgfpathmoveto{\pgfqpoint{2.312110in}{2.647030in}}%
\pgfpathlineto{\pgfqpoint{2.312110in}{2.647030in}}%
\pgfpathlineto{\pgfqpoint{2.312110in}{2.652928in}}%
\pgfpathlineto{\pgfqpoint{2.321192in}{2.652928in}}%
\pgfpathlineto{\pgfqpoint{2.321192in}{2.647030in}}%
\pgfpathmoveto{\pgfqpoint{2.312110in}{2.652928in}}%
\pgfpathlineto{\pgfqpoint{2.312110in}{2.652928in}}%
\pgfpathlineto{\pgfqpoint{2.312110in}{2.658827in}}%
\pgfpathlineto{\pgfqpoint{2.321192in}{2.658827in}}%
\pgfpathlineto{\pgfqpoint{2.321192in}{2.652928in}}%
\pgfpathmoveto{\pgfqpoint{2.321192in}{2.647030in}}%
\pgfpathlineto{\pgfqpoint{2.321192in}{2.647030in}}%
\pgfpathlineto{\pgfqpoint{2.321192in}{2.652928in}}%
\pgfpathlineto{\pgfqpoint{2.330274in}{2.652928in}}%
\pgfpathlineto{\pgfqpoint{2.330274in}{2.647030in}}%
\pgfpathmoveto{\pgfqpoint{2.257616in}{2.694218in}}%
\pgfpathlineto{\pgfqpoint{2.257616in}{2.694218in}}%
\pgfpathlineto{\pgfqpoint{2.257616in}{2.700117in}}%
\pgfpathlineto{\pgfqpoint{2.266698in}{2.700117in}}%
\pgfpathlineto{\pgfqpoint{2.266698in}{2.694218in}}%
\pgfpathmoveto{\pgfqpoint{2.257616in}{2.700117in}}%
\pgfpathlineto{\pgfqpoint{2.257616in}{2.700117in}}%
\pgfpathlineto{\pgfqpoint{2.257616in}{2.706016in}}%
\pgfpathlineto{\pgfqpoint{2.266698in}{2.706016in}}%
\pgfpathlineto{\pgfqpoint{2.266698in}{2.700117in}}%
\pgfpathmoveto{\pgfqpoint{2.266698in}{2.694218in}}%
\pgfpathlineto{\pgfqpoint{2.266698in}{2.694218in}}%
\pgfpathlineto{\pgfqpoint{2.266698in}{2.700117in}}%
\pgfpathlineto{\pgfqpoint{2.275780in}{2.700117in}}%
\pgfpathlineto{\pgfqpoint{2.275780in}{2.694218in}}%
\pgfpathmoveto{\pgfqpoint{2.221286in}{2.729611in}}%
\pgfpathlineto{\pgfqpoint{2.221286in}{2.729611in}}%
\pgfpathlineto{\pgfqpoint{2.221286in}{2.735509in}}%
\pgfpathlineto{\pgfqpoint{2.230369in}{2.735509in}}%
\pgfpathlineto{\pgfqpoint{2.230369in}{2.729611in}}%
\pgfpathmoveto{\pgfqpoint{2.221286in}{2.735509in}}%
\pgfpathlineto{\pgfqpoint{2.221286in}{2.735509in}}%
\pgfpathlineto{\pgfqpoint{2.221286in}{2.741408in}}%
\pgfpathlineto{\pgfqpoint{2.230369in}{2.741408in}}%
\pgfpathlineto{\pgfqpoint{2.230369in}{2.735509in}}%
\pgfpathmoveto{\pgfqpoint{2.203122in}{2.741408in}}%
\pgfpathlineto{\pgfqpoint{2.203122in}{2.741408in}}%
\pgfpathlineto{\pgfqpoint{2.203122in}{2.747307in}}%
\pgfpathlineto{\pgfqpoint{2.212204in}{2.747307in}}%
\pgfpathlineto{\pgfqpoint{2.212204in}{2.741408in}}%
\pgfpathmoveto{\pgfqpoint{2.203122in}{2.747307in}}%
\pgfpathlineto{\pgfqpoint{2.203122in}{2.747307in}}%
\pgfpathlineto{\pgfqpoint{2.203122in}{2.753205in}}%
\pgfpathlineto{\pgfqpoint{2.212204in}{2.753205in}}%
\pgfpathlineto{\pgfqpoint{2.212204in}{2.747307in}}%
\pgfpathmoveto{\pgfqpoint{2.212204in}{2.741408in}}%
\pgfpathlineto{\pgfqpoint{2.212204in}{2.741408in}}%
\pgfpathlineto{\pgfqpoint{2.212204in}{2.747307in}}%
\pgfpathlineto{\pgfqpoint{2.221286in}{2.747307in}}%
\pgfpathlineto{\pgfqpoint{2.221286in}{2.741408in}}%
\pgfpathmoveto{\pgfqpoint{2.239451in}{2.717813in}}%
\pgfpathlineto{\pgfqpoint{2.239451in}{2.717813in}}%
\pgfpathlineto{\pgfqpoint{2.239451in}{2.723712in}}%
\pgfpathlineto{\pgfqpoint{2.248533in}{2.723712in}}%
\pgfpathlineto{\pgfqpoint{2.248533in}{2.717813in}}%
\pgfpathmoveto{\pgfqpoint{2.275780in}{2.682421in}}%
\pgfpathlineto{\pgfqpoint{2.275780in}{2.682421in}}%
\pgfpathlineto{\pgfqpoint{2.275780in}{2.688320in}}%
\pgfpathlineto{\pgfqpoint{2.284863in}{2.688320in}}%
\pgfpathlineto{\pgfqpoint{2.284863in}{2.682421in}}%
\pgfpathmoveto{\pgfqpoint{2.275780in}{2.688320in}}%
\pgfpathlineto{\pgfqpoint{2.275780in}{2.688320in}}%
\pgfpathlineto{\pgfqpoint{2.275780in}{2.694218in}}%
\pgfpathlineto{\pgfqpoint{2.284863in}{2.694218in}}%
\pgfpathlineto{\pgfqpoint{2.284863in}{2.688320in}}%
\pgfpathmoveto{\pgfqpoint{2.293945in}{2.670624in}}%
\pgfpathlineto{\pgfqpoint{2.293945in}{2.670624in}}%
\pgfpathlineto{\pgfqpoint{2.293945in}{2.676522in}}%
\pgfpathlineto{\pgfqpoint{2.303027in}{2.676522in}}%
\pgfpathlineto{\pgfqpoint{2.303027in}{2.670624in}}%
\pgfpathmoveto{\pgfqpoint{2.475583in}{2.505469in}}%
\pgfpathlineto{\pgfqpoint{2.475583in}{2.505469in}}%
\pgfpathlineto{\pgfqpoint{2.475583in}{2.511367in}}%
\pgfpathlineto{\pgfqpoint{2.484664in}{2.511367in}}%
\pgfpathlineto{\pgfqpoint{2.484664in}{2.505469in}}%
\pgfpathmoveto{\pgfqpoint{2.475583in}{2.511367in}}%
\pgfpathlineto{\pgfqpoint{2.475583in}{2.511367in}}%
\pgfpathlineto{\pgfqpoint{2.475583in}{2.517266in}}%
\pgfpathlineto{\pgfqpoint{2.484664in}{2.517266in}}%
\pgfpathlineto{\pgfqpoint{2.484664in}{2.511367in}}%
\pgfpathmoveto{\pgfqpoint{2.484664in}{2.505469in}}%
\pgfpathlineto{\pgfqpoint{2.484664in}{2.505469in}}%
\pgfpathlineto{\pgfqpoint{2.484664in}{2.511367in}}%
\pgfpathlineto{\pgfqpoint{2.493746in}{2.511367in}}%
\pgfpathlineto{\pgfqpoint{2.493746in}{2.505469in}}%
\pgfpathmoveto{\pgfqpoint{2.439256in}{2.540859in}}%
\pgfpathlineto{\pgfqpoint{2.439256in}{2.540859in}}%
\pgfpathlineto{\pgfqpoint{2.439256in}{2.546757in}}%
\pgfpathlineto{\pgfqpoint{2.448338in}{2.546757in}}%
\pgfpathlineto{\pgfqpoint{2.448338in}{2.540859in}}%
\pgfpathmoveto{\pgfqpoint{2.439256in}{2.546757in}}%
\pgfpathlineto{\pgfqpoint{2.439256in}{2.546757in}}%
\pgfpathlineto{\pgfqpoint{2.439256in}{2.552656in}}%
\pgfpathlineto{\pgfqpoint{2.448338in}{2.552656in}}%
\pgfpathlineto{\pgfqpoint{2.448338in}{2.546757in}}%
\pgfpathmoveto{\pgfqpoint{2.421093in}{2.552656in}}%
\pgfpathlineto{\pgfqpoint{2.421093in}{2.552656in}}%
\pgfpathlineto{\pgfqpoint{2.421093in}{2.558554in}}%
\pgfpathlineto{\pgfqpoint{2.430174in}{2.558554in}}%
\pgfpathlineto{\pgfqpoint{2.430174in}{2.552656in}}%
\pgfpathmoveto{\pgfqpoint{2.421093in}{2.558554in}}%
\pgfpathlineto{\pgfqpoint{2.421093in}{2.558554in}}%
\pgfpathlineto{\pgfqpoint{2.421093in}{2.564453in}}%
\pgfpathlineto{\pgfqpoint{2.430174in}{2.564453in}}%
\pgfpathlineto{\pgfqpoint{2.430174in}{2.558554in}}%
\pgfpathmoveto{\pgfqpoint{2.430174in}{2.552656in}}%
\pgfpathlineto{\pgfqpoint{2.430174in}{2.552656in}}%
\pgfpathlineto{\pgfqpoint{2.430174in}{2.558554in}}%
\pgfpathlineto{\pgfqpoint{2.439256in}{2.558554in}}%
\pgfpathlineto{\pgfqpoint{2.439256in}{2.552656in}}%
\pgfpathmoveto{\pgfqpoint{2.457419in}{2.529062in}}%
\pgfpathlineto{\pgfqpoint{2.457419in}{2.529062in}}%
\pgfpathlineto{\pgfqpoint{2.457419in}{2.534961in}}%
\pgfpathlineto{\pgfqpoint{2.466501in}{2.534961in}}%
\pgfpathlineto{\pgfqpoint{2.466501in}{2.529062in}}%
\pgfpathmoveto{\pgfqpoint{2.366602in}{2.599843in}}%
\pgfpathlineto{\pgfqpoint{2.366602in}{2.599843in}}%
\pgfpathlineto{\pgfqpoint{2.366602in}{2.605741in}}%
\pgfpathlineto{\pgfqpoint{2.375684in}{2.605741in}}%
\pgfpathlineto{\pgfqpoint{2.375684in}{2.599843in}}%
\pgfpathmoveto{\pgfqpoint{2.366602in}{2.605741in}}%
\pgfpathlineto{\pgfqpoint{2.366602in}{2.605741in}}%
\pgfpathlineto{\pgfqpoint{2.366602in}{2.611640in}}%
\pgfpathlineto{\pgfqpoint{2.375684in}{2.611640in}}%
\pgfpathlineto{\pgfqpoint{2.375684in}{2.605741in}}%
\pgfpathmoveto{\pgfqpoint{2.375684in}{2.599843in}}%
\pgfpathlineto{\pgfqpoint{2.375684in}{2.599843in}}%
\pgfpathlineto{\pgfqpoint{2.375684in}{2.605741in}}%
\pgfpathlineto{\pgfqpoint{2.384766in}{2.605741in}}%
\pgfpathlineto{\pgfqpoint{2.384766in}{2.599843in}}%
\pgfpathmoveto{\pgfqpoint{2.384766in}{2.588046in}}%
\pgfpathlineto{\pgfqpoint{2.384766in}{2.588046in}}%
\pgfpathlineto{\pgfqpoint{2.384766in}{2.593945in}}%
\pgfpathlineto{\pgfqpoint{2.393848in}{2.593945in}}%
\pgfpathlineto{\pgfqpoint{2.393848in}{2.588046in}}%
\pgfpathmoveto{\pgfqpoint{2.384766in}{2.593945in}}%
\pgfpathlineto{\pgfqpoint{2.384766in}{2.593945in}}%
\pgfpathlineto{\pgfqpoint{2.384766in}{2.599843in}}%
\pgfpathlineto{\pgfqpoint{2.393848in}{2.599843in}}%
\pgfpathlineto{\pgfqpoint{2.393848in}{2.593945in}}%
\pgfpathmoveto{\pgfqpoint{2.402929in}{2.576249in}}%
\pgfpathlineto{\pgfqpoint{2.402929in}{2.576249in}}%
\pgfpathlineto{\pgfqpoint{2.402929in}{2.582148in}}%
\pgfpathlineto{\pgfqpoint{2.412011in}{2.582148in}}%
\pgfpathlineto{\pgfqpoint{2.412011in}{2.576249in}}%
\pgfpathmoveto{\pgfqpoint{2.348439in}{2.623436in}}%
\pgfpathlineto{\pgfqpoint{2.348439in}{2.623436in}}%
\pgfpathlineto{\pgfqpoint{2.348439in}{2.629335in}}%
\pgfpathlineto{\pgfqpoint{2.357521in}{2.629335in}}%
\pgfpathlineto{\pgfqpoint{2.357521in}{2.623436in}}%
\pgfpathmoveto{\pgfqpoint{2.548238in}{2.446484in}}%
\pgfpathlineto{\pgfqpoint{2.548238in}{2.446484in}}%
\pgfpathlineto{\pgfqpoint{2.548238in}{2.452382in}}%
\pgfpathlineto{\pgfqpoint{2.557320in}{2.452382in}}%
\pgfpathlineto{\pgfqpoint{2.557320in}{2.446484in}}%
\pgfpathmoveto{\pgfqpoint{2.548238in}{2.452382in}}%
\pgfpathlineto{\pgfqpoint{2.548238in}{2.452382in}}%
\pgfpathlineto{\pgfqpoint{2.548238in}{2.458281in}}%
\pgfpathlineto{\pgfqpoint{2.557320in}{2.458281in}}%
\pgfpathlineto{\pgfqpoint{2.557320in}{2.452382in}}%
\pgfpathmoveto{\pgfqpoint{2.530074in}{2.458281in}}%
\pgfpathlineto{\pgfqpoint{2.530074in}{2.458281in}}%
\pgfpathlineto{\pgfqpoint{2.530074in}{2.464180in}}%
\pgfpathlineto{\pgfqpoint{2.539156in}{2.464180in}}%
\pgfpathlineto{\pgfqpoint{2.539156in}{2.458281in}}%
\pgfpathmoveto{\pgfqpoint{2.530074in}{2.464180in}}%
\pgfpathlineto{\pgfqpoint{2.530074in}{2.464180in}}%
\pgfpathlineto{\pgfqpoint{2.530074in}{2.470078in}}%
\pgfpathlineto{\pgfqpoint{2.539156in}{2.470078in}}%
\pgfpathlineto{\pgfqpoint{2.539156in}{2.464180in}}%
\pgfpathmoveto{\pgfqpoint{2.539156in}{2.458281in}}%
\pgfpathlineto{\pgfqpoint{2.539156in}{2.458281in}}%
\pgfpathlineto{\pgfqpoint{2.539156in}{2.464180in}}%
\pgfpathlineto{\pgfqpoint{2.548238in}{2.464180in}}%
\pgfpathlineto{\pgfqpoint{2.548238in}{2.458281in}}%
\pgfpathmoveto{\pgfqpoint{2.584566in}{2.411093in}}%
\pgfpathlineto{\pgfqpoint{2.584566in}{2.411093in}}%
\pgfpathlineto{\pgfqpoint{2.584566in}{2.416991in}}%
\pgfpathlineto{\pgfqpoint{2.593648in}{2.416991in}}%
\pgfpathlineto{\pgfqpoint{2.593648in}{2.411093in}}%
\pgfpathmoveto{\pgfqpoint{2.584566in}{2.416991in}}%
\pgfpathlineto{\pgfqpoint{2.584566in}{2.416991in}}%
\pgfpathlineto{\pgfqpoint{2.584566in}{2.422890in}}%
\pgfpathlineto{\pgfqpoint{2.593648in}{2.422890in}}%
\pgfpathlineto{\pgfqpoint{2.593648in}{2.416991in}}%
\pgfpathmoveto{\pgfqpoint{2.593648in}{2.411093in}}%
\pgfpathlineto{\pgfqpoint{2.593648in}{2.411093in}}%
\pgfpathlineto{\pgfqpoint{2.593648in}{2.416991in}}%
\pgfpathlineto{\pgfqpoint{2.602730in}{2.416991in}}%
\pgfpathlineto{\pgfqpoint{2.602730in}{2.411093in}}%
\pgfpathmoveto{\pgfqpoint{2.602730in}{2.399295in}}%
\pgfpathlineto{\pgfqpoint{2.602730in}{2.399295in}}%
\pgfpathlineto{\pgfqpoint{2.602730in}{2.405194in}}%
\pgfpathlineto{\pgfqpoint{2.611812in}{2.405194in}}%
\pgfpathlineto{\pgfqpoint{2.611812in}{2.399295in}}%
\pgfpathmoveto{\pgfqpoint{2.602730in}{2.405194in}}%
\pgfpathlineto{\pgfqpoint{2.602730in}{2.405194in}}%
\pgfpathlineto{\pgfqpoint{2.602730in}{2.411093in}}%
\pgfpathlineto{\pgfqpoint{2.611812in}{2.411093in}}%
\pgfpathlineto{\pgfqpoint{2.611812in}{2.405194in}}%
\pgfpathmoveto{\pgfqpoint{2.620894in}{2.387498in}}%
\pgfpathlineto{\pgfqpoint{2.620894in}{2.387498in}}%
\pgfpathlineto{\pgfqpoint{2.620894in}{2.393397in}}%
\pgfpathlineto{\pgfqpoint{2.629976in}{2.393397in}}%
\pgfpathlineto{\pgfqpoint{2.629976in}{2.387498in}}%
\pgfpathmoveto{\pgfqpoint{2.566402in}{2.434687in}}%
\pgfpathlineto{\pgfqpoint{2.566402in}{2.434687in}}%
\pgfpathlineto{\pgfqpoint{2.566402in}{2.440585in}}%
\pgfpathlineto{\pgfqpoint{2.575484in}{2.440585in}}%
\pgfpathlineto{\pgfqpoint{2.575484in}{2.434687in}}%
\pgfpathmoveto{\pgfqpoint{2.493746in}{2.493672in}}%
\pgfpathlineto{\pgfqpoint{2.493746in}{2.493672in}}%
\pgfpathlineto{\pgfqpoint{2.493746in}{2.499570in}}%
\pgfpathlineto{\pgfqpoint{2.502828in}{2.499570in}}%
\pgfpathlineto{\pgfqpoint{2.502828in}{2.493672in}}%
\pgfpathmoveto{\pgfqpoint{2.493746in}{2.499570in}}%
\pgfpathlineto{\pgfqpoint{2.493746in}{2.499570in}}%
\pgfpathlineto{\pgfqpoint{2.493746in}{2.505469in}}%
\pgfpathlineto{\pgfqpoint{2.502828in}{2.505469in}}%
\pgfpathlineto{\pgfqpoint{2.502828in}{2.499570in}}%
\pgfpathmoveto{\pgfqpoint{2.511910in}{2.481875in}}%
\pgfpathlineto{\pgfqpoint{2.511910in}{2.481875in}}%
\pgfpathlineto{\pgfqpoint{2.511910in}{2.487774in}}%
\pgfpathlineto{\pgfqpoint{2.520992in}{2.487774in}}%
\pgfpathlineto{\pgfqpoint{2.520992in}{2.481875in}}%
\pgfpathmoveto{\pgfqpoint{2.766207in}{2.257733in}}%
\pgfpathlineto{\pgfqpoint{2.766207in}{2.257733in}}%
\pgfpathlineto{\pgfqpoint{2.766207in}{2.263631in}}%
\pgfpathlineto{\pgfqpoint{2.775290in}{2.263631in}}%
\pgfpathlineto{\pgfqpoint{2.775290in}{2.257733in}}%
\pgfpathmoveto{\pgfqpoint{2.766207in}{2.263631in}}%
\pgfpathlineto{\pgfqpoint{2.766207in}{2.263631in}}%
\pgfpathlineto{\pgfqpoint{2.766207in}{2.269530in}}%
\pgfpathlineto{\pgfqpoint{2.775290in}{2.269530in}}%
\pgfpathlineto{\pgfqpoint{2.775290in}{2.263631in}}%
\pgfpathmoveto{\pgfqpoint{2.748043in}{2.269530in}}%
\pgfpathlineto{\pgfqpoint{2.748043in}{2.269530in}}%
\pgfpathlineto{\pgfqpoint{2.748043in}{2.275428in}}%
\pgfpathlineto{\pgfqpoint{2.757125in}{2.275428in}}%
\pgfpathlineto{\pgfqpoint{2.757125in}{2.269530in}}%
\pgfpathmoveto{\pgfqpoint{2.748043in}{2.275428in}}%
\pgfpathlineto{\pgfqpoint{2.748043in}{2.275428in}}%
\pgfpathlineto{\pgfqpoint{2.748043in}{2.281326in}}%
\pgfpathlineto{\pgfqpoint{2.757125in}{2.281326in}}%
\pgfpathlineto{\pgfqpoint{2.757125in}{2.275428in}}%
\pgfpathmoveto{\pgfqpoint{2.757125in}{2.269530in}}%
\pgfpathlineto{\pgfqpoint{2.757125in}{2.269530in}}%
\pgfpathlineto{\pgfqpoint{2.757125in}{2.275428in}}%
\pgfpathlineto{\pgfqpoint{2.766207in}{2.275428in}}%
\pgfpathlineto{\pgfqpoint{2.766207in}{2.269530in}}%
\pgfpathmoveto{\pgfqpoint{2.693551in}{2.316717in}}%
\pgfpathlineto{\pgfqpoint{2.693551in}{2.316717in}}%
\pgfpathlineto{\pgfqpoint{2.693551in}{2.322615in}}%
\pgfpathlineto{\pgfqpoint{2.702633in}{2.322615in}}%
\pgfpathlineto{\pgfqpoint{2.702633in}{2.316717in}}%
\pgfpathmoveto{\pgfqpoint{2.693551in}{2.322615in}}%
\pgfpathlineto{\pgfqpoint{2.693551in}{2.322615in}}%
\pgfpathlineto{\pgfqpoint{2.693551in}{2.328514in}}%
\pgfpathlineto{\pgfqpoint{2.702633in}{2.328514in}}%
\pgfpathlineto{\pgfqpoint{2.702633in}{2.322615in}}%
\pgfpathmoveto{\pgfqpoint{2.702633in}{2.316717in}}%
\pgfpathlineto{\pgfqpoint{2.702633in}{2.316717in}}%
\pgfpathlineto{\pgfqpoint{2.702633in}{2.322615in}}%
\pgfpathlineto{\pgfqpoint{2.711715in}{2.322615in}}%
\pgfpathlineto{\pgfqpoint{2.711715in}{2.316717in}}%
\pgfpathmoveto{\pgfqpoint{2.657223in}{2.352108in}}%
\pgfpathlineto{\pgfqpoint{2.657223in}{2.352108in}}%
\pgfpathlineto{\pgfqpoint{2.657223in}{2.358006in}}%
\pgfpathlineto{\pgfqpoint{2.666305in}{2.358006in}}%
\pgfpathlineto{\pgfqpoint{2.666305in}{2.352108in}}%
\pgfpathmoveto{\pgfqpoint{2.657223in}{2.358006in}}%
\pgfpathlineto{\pgfqpoint{2.657223in}{2.358006in}}%
\pgfpathlineto{\pgfqpoint{2.657223in}{2.363905in}}%
\pgfpathlineto{\pgfqpoint{2.666305in}{2.363905in}}%
\pgfpathlineto{\pgfqpoint{2.666305in}{2.358006in}}%
\pgfpathmoveto{\pgfqpoint{2.639058in}{2.363905in}}%
\pgfpathlineto{\pgfqpoint{2.639058in}{2.363905in}}%
\pgfpathlineto{\pgfqpoint{2.639058in}{2.369803in}}%
\pgfpathlineto{\pgfqpoint{2.648140in}{2.369803in}}%
\pgfpathlineto{\pgfqpoint{2.648140in}{2.363905in}}%
\pgfpathmoveto{\pgfqpoint{2.639058in}{2.369803in}}%
\pgfpathlineto{\pgfqpoint{2.639058in}{2.369803in}}%
\pgfpathlineto{\pgfqpoint{2.639058in}{2.375701in}}%
\pgfpathlineto{\pgfqpoint{2.648140in}{2.375701in}}%
\pgfpathlineto{\pgfqpoint{2.648140in}{2.369803in}}%
\pgfpathmoveto{\pgfqpoint{2.648140in}{2.363905in}}%
\pgfpathlineto{\pgfqpoint{2.648140in}{2.363905in}}%
\pgfpathlineto{\pgfqpoint{2.648140in}{2.369803in}}%
\pgfpathlineto{\pgfqpoint{2.657223in}{2.369803in}}%
\pgfpathlineto{\pgfqpoint{2.657223in}{2.363905in}}%
\pgfpathmoveto{\pgfqpoint{2.675387in}{2.340311in}}%
\pgfpathlineto{\pgfqpoint{2.675387in}{2.340311in}}%
\pgfpathlineto{\pgfqpoint{2.675387in}{2.346209in}}%
\pgfpathlineto{\pgfqpoint{2.684469in}{2.346209in}}%
\pgfpathlineto{\pgfqpoint{2.684469in}{2.340311in}}%
\pgfpathmoveto{\pgfqpoint{2.711715in}{2.304920in}}%
\pgfpathlineto{\pgfqpoint{2.711715in}{2.304920in}}%
\pgfpathlineto{\pgfqpoint{2.711715in}{2.310818in}}%
\pgfpathlineto{\pgfqpoint{2.720797in}{2.310818in}}%
\pgfpathlineto{\pgfqpoint{2.720797in}{2.304920in}}%
\pgfpathmoveto{\pgfqpoint{2.711715in}{2.310818in}}%
\pgfpathlineto{\pgfqpoint{2.711715in}{2.310818in}}%
\pgfpathlineto{\pgfqpoint{2.711715in}{2.316717in}}%
\pgfpathlineto{\pgfqpoint{2.720797in}{2.316717in}}%
\pgfpathlineto{\pgfqpoint{2.720797in}{2.310818in}}%
\pgfpathmoveto{\pgfqpoint{2.729879in}{2.293123in}}%
\pgfpathlineto{\pgfqpoint{2.729879in}{2.293123in}}%
\pgfpathlineto{\pgfqpoint{2.729879in}{2.299021in}}%
\pgfpathlineto{\pgfqpoint{2.738961in}{2.299021in}}%
\pgfpathlineto{\pgfqpoint{2.738961in}{2.293123in}}%
\pgfpathmoveto{\pgfqpoint{2.893360in}{2.139765in}}%
\pgfpathlineto{\pgfqpoint{2.893360in}{2.139765in}}%
\pgfpathlineto{\pgfqpoint{2.893360in}{2.145663in}}%
\pgfpathlineto{\pgfqpoint{2.902442in}{2.145663in}}%
\pgfpathlineto{\pgfqpoint{2.902442in}{2.139765in}}%
\pgfpathmoveto{\pgfqpoint{2.893360in}{2.145663in}}%
\pgfpathlineto{\pgfqpoint{2.893360in}{2.145663in}}%
\pgfpathlineto{\pgfqpoint{2.893360in}{2.151562in}}%
\pgfpathlineto{\pgfqpoint{2.902442in}{2.151562in}}%
\pgfpathlineto{\pgfqpoint{2.902442in}{2.145663in}}%
\pgfpathmoveto{\pgfqpoint{2.902442in}{2.139765in}}%
\pgfpathlineto{\pgfqpoint{2.902442in}{2.139765in}}%
\pgfpathlineto{\pgfqpoint{2.902442in}{2.145663in}}%
\pgfpathlineto{\pgfqpoint{2.911525in}{2.145663in}}%
\pgfpathlineto{\pgfqpoint{2.911525in}{2.139765in}}%
\pgfpathmoveto{\pgfqpoint{2.911525in}{2.127968in}}%
\pgfpathlineto{\pgfqpoint{2.911525in}{2.127968in}}%
\pgfpathlineto{\pgfqpoint{2.911525in}{2.133866in}}%
\pgfpathlineto{\pgfqpoint{2.920607in}{2.133866in}}%
\pgfpathlineto{\pgfqpoint{2.920607in}{2.127968in}}%
\pgfpathmoveto{\pgfqpoint{2.911525in}{2.133866in}}%
\pgfpathlineto{\pgfqpoint{2.911525in}{2.133866in}}%
\pgfpathlineto{\pgfqpoint{2.911525in}{2.139765in}}%
\pgfpathlineto{\pgfqpoint{2.920607in}{2.139765in}}%
\pgfpathlineto{\pgfqpoint{2.920607in}{2.133866in}}%
\pgfpathmoveto{\pgfqpoint{2.920607in}{2.127968in}}%
\pgfpathlineto{\pgfqpoint{2.920607in}{2.127968in}}%
\pgfpathlineto{\pgfqpoint{2.920607in}{2.133866in}}%
\pgfpathlineto{\pgfqpoint{2.929689in}{2.133866in}}%
\pgfpathlineto{\pgfqpoint{2.929689in}{2.127968in}}%
\pgfpathmoveto{\pgfqpoint{2.875195in}{2.163359in}}%
\pgfpathlineto{\pgfqpoint{2.875195in}{2.163359in}}%
\pgfpathlineto{\pgfqpoint{2.875195in}{2.169257in}}%
\pgfpathlineto{\pgfqpoint{2.884278in}{2.169257in}}%
\pgfpathlineto{\pgfqpoint{2.884278in}{2.163359in}}%
\pgfpathmoveto{\pgfqpoint{2.875195in}{2.169257in}}%
\pgfpathlineto{\pgfqpoint{2.875195in}{2.169257in}}%
\pgfpathlineto{\pgfqpoint{2.875195in}{2.175156in}}%
\pgfpathlineto{\pgfqpoint{2.884278in}{2.175156in}}%
\pgfpathlineto{\pgfqpoint{2.884278in}{2.169257in}}%
\pgfpathmoveto{\pgfqpoint{2.857030in}{2.175156in}}%
\pgfpathlineto{\pgfqpoint{2.857030in}{2.175156in}}%
\pgfpathlineto{\pgfqpoint{2.857030in}{2.181054in}}%
\pgfpathlineto{\pgfqpoint{2.866113in}{2.181054in}}%
\pgfpathlineto{\pgfqpoint{2.866113in}{2.175156in}}%
\pgfpathmoveto{\pgfqpoint{2.857030in}{2.181054in}}%
\pgfpathlineto{\pgfqpoint{2.857030in}{2.181054in}}%
\pgfpathlineto{\pgfqpoint{2.857030in}{2.186952in}}%
\pgfpathlineto{\pgfqpoint{2.866113in}{2.186952in}}%
\pgfpathlineto{\pgfqpoint{2.866113in}{2.181054in}}%
\pgfpathmoveto{\pgfqpoint{2.866113in}{2.175156in}}%
\pgfpathlineto{\pgfqpoint{2.866113in}{2.175156in}}%
\pgfpathlineto{\pgfqpoint{2.866113in}{2.181054in}}%
\pgfpathlineto{\pgfqpoint{2.875195in}{2.181054in}}%
\pgfpathlineto{\pgfqpoint{2.875195in}{2.175156in}}%
\pgfpathmoveto{\pgfqpoint{2.893360in}{2.151562in}}%
\pgfpathlineto{\pgfqpoint{2.893360in}{2.151562in}}%
\pgfpathlineto{\pgfqpoint{2.893360in}{2.157460in}}%
\pgfpathlineto{\pgfqpoint{2.902442in}{2.157460in}}%
\pgfpathlineto{\pgfqpoint{2.902442in}{2.151562in}}%
\pgfpathmoveto{\pgfqpoint{2.802536in}{2.222343in}}%
\pgfpathlineto{\pgfqpoint{2.802536in}{2.222343in}}%
\pgfpathlineto{\pgfqpoint{2.802536in}{2.228241in}}%
\pgfpathlineto{\pgfqpoint{2.811619in}{2.228241in}}%
\pgfpathlineto{\pgfqpoint{2.811619in}{2.222343in}}%
\pgfpathmoveto{\pgfqpoint{2.802536in}{2.228241in}}%
\pgfpathlineto{\pgfqpoint{2.802536in}{2.228241in}}%
\pgfpathlineto{\pgfqpoint{2.802536in}{2.234139in}}%
\pgfpathlineto{\pgfqpoint{2.811619in}{2.234139in}}%
\pgfpathlineto{\pgfqpoint{2.811619in}{2.228241in}}%
\pgfpathmoveto{\pgfqpoint{2.811619in}{2.222343in}}%
\pgfpathlineto{\pgfqpoint{2.811619in}{2.222343in}}%
\pgfpathlineto{\pgfqpoint{2.811619in}{2.228241in}}%
\pgfpathlineto{\pgfqpoint{2.820701in}{2.228241in}}%
\pgfpathlineto{\pgfqpoint{2.820701in}{2.222343in}}%
\pgfpathmoveto{\pgfqpoint{2.820701in}{2.210546in}}%
\pgfpathlineto{\pgfqpoint{2.820701in}{2.210546in}}%
\pgfpathlineto{\pgfqpoint{2.820701in}{2.216444in}}%
\pgfpathlineto{\pgfqpoint{2.829783in}{2.216444in}}%
\pgfpathlineto{\pgfqpoint{2.829783in}{2.210546in}}%
\pgfpathmoveto{\pgfqpoint{2.820701in}{2.216444in}}%
\pgfpathlineto{\pgfqpoint{2.820701in}{2.216444in}}%
\pgfpathlineto{\pgfqpoint{2.820701in}{2.222343in}}%
\pgfpathlineto{\pgfqpoint{2.829783in}{2.222343in}}%
\pgfpathlineto{\pgfqpoint{2.829783in}{2.216444in}}%
\pgfpathmoveto{\pgfqpoint{2.838866in}{2.198749in}}%
\pgfpathlineto{\pgfqpoint{2.838866in}{2.198749in}}%
\pgfpathlineto{\pgfqpoint{2.838866in}{2.204648in}}%
\pgfpathlineto{\pgfqpoint{2.847948in}{2.204648in}}%
\pgfpathlineto{\pgfqpoint{2.847948in}{2.198749in}}%
\pgfpathmoveto{\pgfqpoint{2.784372in}{2.245936in}}%
\pgfpathlineto{\pgfqpoint{2.784372in}{2.245936in}}%
\pgfpathlineto{\pgfqpoint{2.784372in}{2.251835in}}%
\pgfpathlineto{\pgfqpoint{2.793454in}{2.251835in}}%
\pgfpathlineto{\pgfqpoint{2.793454in}{2.245936in}}%
\pgfpathmoveto{\pgfqpoint{2.984180in}{2.068983in}}%
\pgfpathlineto{\pgfqpoint{2.984180in}{2.068983in}}%
\pgfpathlineto{\pgfqpoint{2.984180in}{2.074882in}}%
\pgfpathlineto{\pgfqpoint{2.993261in}{2.074882in}}%
\pgfpathlineto{\pgfqpoint{2.993261in}{2.068983in}}%
\pgfpathmoveto{\pgfqpoint{2.984180in}{2.074882in}}%
\pgfpathlineto{\pgfqpoint{2.984180in}{2.074882in}}%
\pgfpathlineto{\pgfqpoint{2.984180in}{2.080780in}}%
\pgfpathlineto{\pgfqpoint{2.993261in}{2.080780in}}%
\pgfpathlineto{\pgfqpoint{2.993261in}{2.074882in}}%
\pgfpathmoveto{\pgfqpoint{2.966016in}{2.080780in}}%
\pgfpathlineto{\pgfqpoint{2.966016in}{2.080780in}}%
\pgfpathlineto{\pgfqpoint{2.966016in}{2.086679in}}%
\pgfpathlineto{\pgfqpoint{2.975098in}{2.086679in}}%
\pgfpathlineto{\pgfqpoint{2.975098in}{2.080780in}}%
\pgfpathmoveto{\pgfqpoint{2.966016in}{2.086679in}}%
\pgfpathlineto{\pgfqpoint{2.966016in}{2.086679in}}%
\pgfpathlineto{\pgfqpoint{2.966016in}{2.092577in}}%
\pgfpathlineto{\pgfqpoint{2.975098in}{2.092577in}}%
\pgfpathlineto{\pgfqpoint{2.975098in}{2.086679in}}%
\pgfpathmoveto{\pgfqpoint{2.975098in}{2.080780in}}%
\pgfpathlineto{\pgfqpoint{2.975098in}{2.080780in}}%
\pgfpathlineto{\pgfqpoint{2.975098in}{2.086679in}}%
\pgfpathlineto{\pgfqpoint{2.984180in}{2.086679in}}%
\pgfpathlineto{\pgfqpoint{2.984180in}{2.080780in}}%
\pgfpathmoveto{\pgfqpoint{3.020507in}{2.033592in}}%
\pgfpathlineto{\pgfqpoint{3.020507in}{2.033592in}}%
\pgfpathlineto{\pgfqpoint{3.020507in}{2.039490in}}%
\pgfpathlineto{\pgfqpoint{3.029588in}{2.039490in}}%
\pgfpathlineto{\pgfqpoint{3.029588in}{2.033592in}}%
\pgfpathmoveto{\pgfqpoint{3.020507in}{2.039490in}}%
\pgfpathlineto{\pgfqpoint{3.020507in}{2.039490in}}%
\pgfpathlineto{\pgfqpoint{3.020507in}{2.045389in}}%
\pgfpathlineto{\pgfqpoint{3.029588in}{2.045389in}}%
\pgfpathlineto{\pgfqpoint{3.029588in}{2.039490in}}%
\pgfpathmoveto{\pgfqpoint{3.029588in}{2.033592in}}%
\pgfpathlineto{\pgfqpoint{3.029588in}{2.033592in}}%
\pgfpathlineto{\pgfqpoint{3.029588in}{2.039490in}}%
\pgfpathlineto{\pgfqpoint{3.038670in}{2.039490in}}%
\pgfpathlineto{\pgfqpoint{3.038670in}{2.033592in}}%
\pgfpathmoveto{\pgfqpoint{3.038670in}{2.021795in}}%
\pgfpathlineto{\pgfqpoint{3.038670in}{2.021795in}}%
\pgfpathlineto{\pgfqpoint{3.038670in}{2.027693in}}%
\pgfpathlineto{\pgfqpoint{3.047752in}{2.027693in}}%
\pgfpathlineto{\pgfqpoint{3.047752in}{2.021795in}}%
\pgfpathmoveto{\pgfqpoint{3.038670in}{2.027693in}}%
\pgfpathlineto{\pgfqpoint{3.038670in}{2.027693in}}%
\pgfpathlineto{\pgfqpoint{3.038670in}{2.033592in}}%
\pgfpathlineto{\pgfqpoint{3.047752in}{2.033592in}}%
\pgfpathlineto{\pgfqpoint{3.047752in}{2.027693in}}%
\pgfpathmoveto{\pgfqpoint{3.056834in}{2.009998in}}%
\pgfpathlineto{\pgfqpoint{3.056834in}{2.009998in}}%
\pgfpathlineto{\pgfqpoint{3.056834in}{2.015896in}}%
\pgfpathlineto{\pgfqpoint{3.065915in}{2.015896in}}%
\pgfpathlineto{\pgfqpoint{3.065915in}{2.009998in}}%
\pgfpathmoveto{\pgfqpoint{3.002343in}{2.057186in}}%
\pgfpathlineto{\pgfqpoint{3.002343in}{2.057186in}}%
\pgfpathlineto{\pgfqpoint{3.002343in}{2.063084in}}%
\pgfpathlineto{\pgfqpoint{3.011425in}{2.063084in}}%
\pgfpathlineto{\pgfqpoint{3.011425in}{2.057186in}}%
\pgfpathmoveto{\pgfqpoint{2.929689in}{2.116171in}}%
\pgfpathlineto{\pgfqpoint{2.929689in}{2.116171in}}%
\pgfpathlineto{\pgfqpoint{2.929689in}{2.122069in}}%
\pgfpathlineto{\pgfqpoint{2.938771in}{2.122069in}}%
\pgfpathlineto{\pgfqpoint{2.938771in}{2.116171in}}%
\pgfpathmoveto{\pgfqpoint{2.947853in}{2.104374in}}%
\pgfpathlineto{\pgfqpoint{2.947853in}{2.104374in}}%
\pgfpathlineto{\pgfqpoint{2.947853in}{2.110273in}}%
\pgfpathlineto{\pgfqpoint{2.956934in}{2.110273in}}%
\pgfpathlineto{\pgfqpoint{2.956934in}{2.104374in}}%
\pgfpathmoveto{\pgfqpoint{3.165818in}{1.903830in}}%
\pgfpathlineto{\pgfqpoint{3.165818in}{1.903830in}}%
\pgfpathlineto{\pgfqpoint{3.165818in}{1.909729in}}%
\pgfpathlineto{\pgfqpoint{3.174900in}{1.909729in}}%
\pgfpathlineto{\pgfqpoint{3.174900in}{1.903830in}}%
\pgfpathmoveto{\pgfqpoint{3.165818in}{1.909729in}}%
\pgfpathlineto{\pgfqpoint{3.165818in}{1.909729in}}%
\pgfpathlineto{\pgfqpoint{3.165818in}{1.915628in}}%
\pgfpathlineto{\pgfqpoint{3.174900in}{1.915628in}}%
\pgfpathlineto{\pgfqpoint{3.174900in}{1.909729in}}%
\pgfpathmoveto{\pgfqpoint{3.174900in}{1.903830in}}%
\pgfpathlineto{\pgfqpoint{3.174900in}{1.903830in}}%
\pgfpathlineto{\pgfqpoint{3.174900in}{1.909729in}}%
\pgfpathlineto{\pgfqpoint{3.183982in}{1.909729in}}%
\pgfpathlineto{\pgfqpoint{3.183982in}{1.903830in}}%
\pgfpathmoveto{\pgfqpoint{3.202146in}{1.880235in}}%
\pgfpathlineto{\pgfqpoint{3.202146in}{1.880235in}}%
\pgfpathlineto{\pgfqpoint{3.202146in}{1.886134in}}%
\pgfpathlineto{\pgfqpoint{3.211229in}{1.886134in}}%
\pgfpathlineto{\pgfqpoint{3.211229in}{1.880235in}}%
\pgfpathmoveto{\pgfqpoint{3.202146in}{1.886134in}}%
\pgfpathlineto{\pgfqpoint{3.202146in}{1.886134in}}%
\pgfpathlineto{\pgfqpoint{3.202146in}{1.892033in}}%
\pgfpathlineto{\pgfqpoint{3.211229in}{1.892033in}}%
\pgfpathlineto{\pgfqpoint{3.211229in}{1.886134in}}%
\pgfpathmoveto{\pgfqpoint{3.183982in}{1.892033in}}%
\pgfpathlineto{\pgfqpoint{3.183982in}{1.892033in}}%
\pgfpathlineto{\pgfqpoint{3.183982in}{1.897932in}}%
\pgfpathlineto{\pgfqpoint{3.193064in}{1.897932in}}%
\pgfpathlineto{\pgfqpoint{3.193064in}{1.892033in}}%
\pgfpathmoveto{\pgfqpoint{3.183982in}{1.897932in}}%
\pgfpathlineto{\pgfqpoint{3.183982in}{1.897932in}}%
\pgfpathlineto{\pgfqpoint{3.183982in}{1.903830in}}%
\pgfpathlineto{\pgfqpoint{3.193064in}{1.903830in}}%
\pgfpathlineto{\pgfqpoint{3.193064in}{1.897932in}}%
\pgfpathmoveto{\pgfqpoint{3.193064in}{1.892033in}}%
\pgfpathlineto{\pgfqpoint{3.193064in}{1.892033in}}%
\pgfpathlineto{\pgfqpoint{3.193064in}{1.897932in}}%
\pgfpathlineto{\pgfqpoint{3.202146in}{1.897932in}}%
\pgfpathlineto{\pgfqpoint{3.202146in}{1.892033in}}%
\pgfpathmoveto{\pgfqpoint{3.129490in}{1.939220in}}%
\pgfpathlineto{\pgfqpoint{3.129490in}{1.939220in}}%
\pgfpathlineto{\pgfqpoint{3.129490in}{1.945118in}}%
\pgfpathlineto{\pgfqpoint{3.138572in}{1.945118in}}%
\pgfpathlineto{\pgfqpoint{3.138572in}{1.939220in}}%
\pgfpathmoveto{\pgfqpoint{3.129490in}{1.945118in}}%
\pgfpathlineto{\pgfqpoint{3.129490in}{1.945118in}}%
\pgfpathlineto{\pgfqpoint{3.129490in}{1.951016in}}%
\pgfpathlineto{\pgfqpoint{3.138572in}{1.951016in}}%
\pgfpathlineto{\pgfqpoint{3.138572in}{1.945118in}}%
\pgfpathmoveto{\pgfqpoint{3.138572in}{1.939220in}}%
\pgfpathlineto{\pgfqpoint{3.138572in}{1.939220in}}%
\pgfpathlineto{\pgfqpoint{3.138572in}{1.945118in}}%
\pgfpathlineto{\pgfqpoint{3.147654in}{1.945118in}}%
\pgfpathlineto{\pgfqpoint{3.147654in}{1.939220in}}%
\pgfpathmoveto{\pgfqpoint{3.093161in}{1.974609in}}%
\pgfpathlineto{\pgfqpoint{3.093161in}{1.974609in}}%
\pgfpathlineto{\pgfqpoint{3.093161in}{1.980507in}}%
\pgfpathlineto{\pgfqpoint{3.102243in}{1.980507in}}%
\pgfpathlineto{\pgfqpoint{3.102243in}{1.974609in}}%
\pgfpathmoveto{\pgfqpoint{3.093161in}{1.980507in}}%
\pgfpathlineto{\pgfqpoint{3.093161in}{1.980507in}}%
\pgfpathlineto{\pgfqpoint{3.093161in}{1.986405in}}%
\pgfpathlineto{\pgfqpoint{3.102243in}{1.986405in}}%
\pgfpathlineto{\pgfqpoint{3.102243in}{1.980507in}}%
\pgfpathmoveto{\pgfqpoint{3.074997in}{1.986405in}}%
\pgfpathlineto{\pgfqpoint{3.074997in}{1.986405in}}%
\pgfpathlineto{\pgfqpoint{3.074997in}{1.992303in}}%
\pgfpathlineto{\pgfqpoint{3.084079in}{1.992303in}}%
\pgfpathlineto{\pgfqpoint{3.084079in}{1.986405in}}%
\pgfpathmoveto{\pgfqpoint{3.074997in}{1.992303in}}%
\pgfpathlineto{\pgfqpoint{3.074997in}{1.992303in}}%
\pgfpathlineto{\pgfqpoint{3.074997in}{1.998202in}}%
\pgfpathlineto{\pgfqpoint{3.084079in}{1.998202in}}%
\pgfpathlineto{\pgfqpoint{3.084079in}{1.992303in}}%
\pgfpathmoveto{\pgfqpoint{3.084079in}{1.986405in}}%
\pgfpathlineto{\pgfqpoint{3.084079in}{1.986405in}}%
\pgfpathlineto{\pgfqpoint{3.084079in}{1.992303in}}%
\pgfpathlineto{\pgfqpoint{3.093161in}{1.992303in}}%
\pgfpathlineto{\pgfqpoint{3.093161in}{1.986405in}}%
\pgfpathmoveto{\pgfqpoint{3.111325in}{1.962813in}}%
\pgfpathlineto{\pgfqpoint{3.111325in}{1.962813in}}%
\pgfpathlineto{\pgfqpoint{3.111325in}{1.968711in}}%
\pgfpathlineto{\pgfqpoint{3.120408in}{1.968711in}}%
\pgfpathlineto{\pgfqpoint{3.120408in}{1.962813in}}%
\pgfpathmoveto{\pgfqpoint{3.147654in}{1.927424in}}%
\pgfpathlineto{\pgfqpoint{3.147654in}{1.927424in}}%
\pgfpathlineto{\pgfqpoint{3.147654in}{1.933322in}}%
\pgfpathlineto{\pgfqpoint{3.156736in}{1.933322in}}%
\pgfpathlineto{\pgfqpoint{3.156736in}{1.927424in}}%
\pgfpathmoveto{\pgfqpoint{3.147654in}{1.933322in}}%
\pgfpathlineto{\pgfqpoint{3.147654in}{1.933322in}}%
\pgfpathlineto{\pgfqpoint{3.147654in}{1.939220in}}%
\pgfpathlineto{\pgfqpoint{3.156736in}{1.939220in}}%
\pgfpathlineto{\pgfqpoint{3.156736in}{1.933322in}}%
\pgfpathmoveto{\pgfqpoint{3.165818in}{1.915628in}}%
\pgfpathlineto{\pgfqpoint{3.165818in}{1.915628in}}%
\pgfpathlineto{\pgfqpoint{3.165818in}{1.921526in}}%
\pgfpathlineto{\pgfqpoint{3.174900in}{1.921526in}}%
\pgfpathlineto{\pgfqpoint{3.174900in}{1.915628in}}%
\pgfpathmoveto{\pgfqpoint{3.329298in}{1.762266in}}%
\pgfpathlineto{\pgfqpoint{3.329298in}{1.762266in}}%
\pgfpathlineto{\pgfqpoint{3.329298in}{1.768165in}}%
\pgfpathlineto{\pgfqpoint{3.338381in}{1.768165in}}%
\pgfpathlineto{\pgfqpoint{3.338381in}{1.762266in}}%
\pgfpathmoveto{\pgfqpoint{3.329298in}{1.768165in}}%
\pgfpathlineto{\pgfqpoint{3.329298in}{1.768165in}}%
\pgfpathlineto{\pgfqpoint{3.329298in}{1.774063in}}%
\pgfpathlineto{\pgfqpoint{3.338381in}{1.774063in}}%
\pgfpathlineto{\pgfqpoint{3.338381in}{1.768165in}}%
\pgfpathmoveto{\pgfqpoint{3.338381in}{1.762266in}}%
\pgfpathlineto{\pgfqpoint{3.338381in}{1.762266in}}%
\pgfpathlineto{\pgfqpoint{3.338381in}{1.768165in}}%
\pgfpathlineto{\pgfqpoint{3.347463in}{1.768165in}}%
\pgfpathlineto{\pgfqpoint{3.347463in}{1.762266in}}%
\pgfpathmoveto{\pgfqpoint{3.347463in}{1.750470in}}%
\pgfpathlineto{\pgfqpoint{3.347463in}{1.750470in}}%
\pgfpathlineto{\pgfqpoint{3.347463in}{1.756368in}}%
\pgfpathlineto{\pgfqpoint{3.356545in}{1.756368in}}%
\pgfpathlineto{\pgfqpoint{3.356545in}{1.750470in}}%
\pgfpathmoveto{\pgfqpoint{3.347463in}{1.756368in}}%
\pgfpathlineto{\pgfqpoint{3.347463in}{1.756368in}}%
\pgfpathlineto{\pgfqpoint{3.347463in}{1.762266in}}%
\pgfpathlineto{\pgfqpoint{3.356545in}{1.762266in}}%
\pgfpathlineto{\pgfqpoint{3.356545in}{1.756368in}}%
\pgfpathmoveto{\pgfqpoint{3.356545in}{1.750470in}}%
\pgfpathlineto{\pgfqpoint{3.356545in}{1.750470in}}%
\pgfpathlineto{\pgfqpoint{3.356545in}{1.756368in}}%
\pgfpathlineto{\pgfqpoint{3.365628in}{1.756368in}}%
\pgfpathlineto{\pgfqpoint{3.365628in}{1.750470in}}%
\pgfpathmoveto{\pgfqpoint{3.311134in}{1.785859in}}%
\pgfpathlineto{\pgfqpoint{3.311134in}{1.785859in}}%
\pgfpathlineto{\pgfqpoint{3.311134in}{1.791757in}}%
\pgfpathlineto{\pgfqpoint{3.320216in}{1.791757in}}%
\pgfpathlineto{\pgfqpoint{3.320216in}{1.785859in}}%
\pgfpathmoveto{\pgfqpoint{3.292969in}{1.797656in}}%
\pgfpathlineto{\pgfqpoint{3.292969in}{1.797656in}}%
\pgfpathlineto{\pgfqpoint{3.292969in}{1.803554in}}%
\pgfpathlineto{\pgfqpoint{3.302051in}{1.803554in}}%
\pgfpathlineto{\pgfqpoint{3.302051in}{1.797656in}}%
\pgfpathmoveto{\pgfqpoint{3.292969in}{1.803554in}}%
\pgfpathlineto{\pgfqpoint{3.292969in}{1.803554in}}%
\pgfpathlineto{\pgfqpoint{3.292969in}{1.809452in}}%
\pgfpathlineto{\pgfqpoint{3.302051in}{1.809452in}}%
\pgfpathlineto{\pgfqpoint{3.302051in}{1.803554in}}%
\pgfpathmoveto{\pgfqpoint{3.302051in}{1.797656in}}%
\pgfpathlineto{\pgfqpoint{3.302051in}{1.797656in}}%
\pgfpathlineto{\pgfqpoint{3.302051in}{1.803554in}}%
\pgfpathlineto{\pgfqpoint{3.311134in}{1.803554in}}%
\pgfpathlineto{\pgfqpoint{3.311134in}{1.797656in}}%
\pgfpathmoveto{\pgfqpoint{3.329298in}{1.774063in}}%
\pgfpathlineto{\pgfqpoint{3.329298in}{1.774063in}}%
\pgfpathlineto{\pgfqpoint{3.329298in}{1.779961in}}%
\pgfpathlineto{\pgfqpoint{3.338381in}{1.779961in}}%
\pgfpathlineto{\pgfqpoint{3.338381in}{1.774063in}}%
\pgfpathmoveto{\pgfqpoint{3.238475in}{1.844843in}}%
\pgfpathlineto{\pgfqpoint{3.238475in}{1.844843in}}%
\pgfpathlineto{\pgfqpoint{3.238475in}{1.850742in}}%
\pgfpathlineto{\pgfqpoint{3.247558in}{1.850742in}}%
\pgfpathlineto{\pgfqpoint{3.247558in}{1.844843in}}%
\pgfpathmoveto{\pgfqpoint{3.238475in}{1.850742in}}%
\pgfpathlineto{\pgfqpoint{3.238475in}{1.850742in}}%
\pgfpathlineto{\pgfqpoint{3.238475in}{1.856641in}}%
\pgfpathlineto{\pgfqpoint{3.247558in}{1.856641in}}%
\pgfpathlineto{\pgfqpoint{3.247558in}{1.850742in}}%
\pgfpathmoveto{\pgfqpoint{3.247558in}{1.844843in}}%
\pgfpathlineto{\pgfqpoint{3.247558in}{1.844843in}}%
\pgfpathlineto{\pgfqpoint{3.247558in}{1.850742in}}%
\pgfpathlineto{\pgfqpoint{3.256640in}{1.850742in}}%
\pgfpathlineto{\pgfqpoint{3.256640in}{1.844843in}}%
\pgfpathmoveto{\pgfqpoint{3.256640in}{1.833046in}}%
\pgfpathlineto{\pgfqpoint{3.256640in}{1.833046in}}%
\pgfpathlineto{\pgfqpoint{3.256640in}{1.838945in}}%
\pgfpathlineto{\pgfqpoint{3.265722in}{1.838945in}}%
\pgfpathlineto{\pgfqpoint{3.265722in}{1.833046in}}%
\pgfpathmoveto{\pgfqpoint{3.256640in}{1.838945in}}%
\pgfpathlineto{\pgfqpoint{3.256640in}{1.838945in}}%
\pgfpathlineto{\pgfqpoint{3.256640in}{1.844843in}}%
\pgfpathlineto{\pgfqpoint{3.265722in}{1.844843in}}%
\pgfpathlineto{\pgfqpoint{3.265722in}{1.838945in}}%
\pgfpathmoveto{\pgfqpoint{3.274805in}{1.821249in}}%
\pgfpathlineto{\pgfqpoint{3.274805in}{1.821249in}}%
\pgfpathlineto{\pgfqpoint{3.274805in}{1.827147in}}%
\pgfpathlineto{\pgfqpoint{3.283887in}{1.827147in}}%
\pgfpathlineto{\pgfqpoint{3.283887in}{1.821249in}}%
\pgfpathmoveto{\pgfqpoint{3.220311in}{1.868438in}}%
\pgfpathlineto{\pgfqpoint{3.220311in}{1.868438in}}%
\pgfpathlineto{\pgfqpoint{3.220311in}{1.874337in}}%
\pgfpathlineto{\pgfqpoint{3.229393in}{1.874337in}}%
\pgfpathlineto{\pgfqpoint{3.229393in}{1.868438in}}%
\pgfpathmoveto{\pgfqpoint{3.456447in}{0.505897in}}%
\pgfpathlineto{\pgfqpoint{3.456447in}{0.505897in}}%
\pgfpathlineto{\pgfqpoint{3.456447in}{0.511795in}}%
\pgfpathlineto{\pgfqpoint{3.465529in}{0.511795in}}%
\pgfpathlineto{\pgfqpoint{3.465529in}{0.505897in}}%
\pgfpathmoveto{\pgfqpoint{3.474611in}{0.517694in}}%
\pgfpathlineto{\pgfqpoint{3.474611in}{0.517694in}}%
\pgfpathlineto{\pgfqpoint{3.474611in}{0.523592in}}%
\pgfpathlineto{\pgfqpoint{3.483693in}{0.523592in}}%
\pgfpathlineto{\pgfqpoint{3.483693in}{0.517694in}}%
\pgfpathmoveto{\pgfqpoint{3.474611in}{0.523592in}}%
\pgfpathlineto{\pgfqpoint{3.474611in}{0.523592in}}%
\pgfpathlineto{\pgfqpoint{3.474611in}{0.529491in}}%
\pgfpathlineto{\pgfqpoint{3.483693in}{0.529491in}}%
\pgfpathlineto{\pgfqpoint{3.483693in}{0.523592in}}%
\pgfpathmoveto{\pgfqpoint{3.474611in}{0.529491in}}%
\pgfpathlineto{\pgfqpoint{3.474611in}{0.529491in}}%
\pgfpathlineto{\pgfqpoint{3.474611in}{0.535389in}}%
\pgfpathlineto{\pgfqpoint{3.483693in}{0.535389in}}%
\pgfpathlineto{\pgfqpoint{3.483693in}{0.529491in}}%
\pgfpathmoveto{\pgfqpoint{3.483693in}{0.529491in}}%
\pgfpathlineto{\pgfqpoint{3.483693in}{0.529491in}}%
\pgfpathlineto{\pgfqpoint{3.483693in}{0.535389in}}%
\pgfpathlineto{\pgfqpoint{3.492775in}{0.535389in}}%
\pgfpathlineto{\pgfqpoint{3.492775in}{0.529491in}}%
\pgfpathmoveto{\pgfqpoint{3.492775in}{0.535389in}}%
\pgfpathlineto{\pgfqpoint{3.492775in}{0.535389in}}%
\pgfpathlineto{\pgfqpoint{3.492775in}{0.541287in}}%
\pgfpathlineto{\pgfqpoint{3.501857in}{0.541287in}}%
\pgfpathlineto{\pgfqpoint{3.501857in}{0.535389in}}%
\pgfpathmoveto{\pgfqpoint{3.492775in}{0.541287in}}%
\pgfpathlineto{\pgfqpoint{3.492775in}{0.541287in}}%
\pgfpathlineto{\pgfqpoint{3.492775in}{0.547186in}}%
\pgfpathlineto{\pgfqpoint{3.501857in}{0.547186in}}%
\pgfpathlineto{\pgfqpoint{3.501857in}{0.541287in}}%
\pgfpathmoveto{\pgfqpoint{3.501857in}{0.541287in}}%
\pgfpathlineto{\pgfqpoint{3.501857in}{0.541287in}}%
\pgfpathlineto{\pgfqpoint{3.501857in}{0.547186in}}%
\pgfpathlineto{\pgfqpoint{3.510939in}{0.547186in}}%
\pgfpathlineto{\pgfqpoint{3.510939in}{0.541287in}}%
\pgfpathmoveto{\pgfqpoint{3.383792in}{1.715080in}}%
\pgfpathlineto{\pgfqpoint{3.383792in}{1.715080in}}%
\pgfpathlineto{\pgfqpoint{3.383792in}{1.720979in}}%
\pgfpathlineto{\pgfqpoint{3.392873in}{1.720979in}}%
\pgfpathlineto{\pgfqpoint{3.392873in}{1.715080in}}%
\pgfpathmoveto{\pgfqpoint{3.383792in}{1.720979in}}%
\pgfpathlineto{\pgfqpoint{3.383792in}{1.720979in}}%
\pgfpathlineto{\pgfqpoint{3.383792in}{1.726877in}}%
\pgfpathlineto{\pgfqpoint{3.392873in}{1.726877in}}%
\pgfpathlineto{\pgfqpoint{3.392873in}{1.720979in}}%
\pgfpathmoveto{\pgfqpoint{3.392873in}{1.715080in}}%
\pgfpathlineto{\pgfqpoint{3.392873in}{1.715080in}}%
\pgfpathlineto{\pgfqpoint{3.392873in}{1.720979in}}%
\pgfpathlineto{\pgfqpoint{3.401955in}{1.720979in}}%
\pgfpathlineto{\pgfqpoint{3.401955in}{1.715080in}}%
\pgfpathmoveto{\pgfqpoint{3.420119in}{1.691485in}}%
\pgfpathlineto{\pgfqpoint{3.420119in}{1.691485in}}%
\pgfpathlineto{\pgfqpoint{3.420119in}{1.697384in}}%
\pgfpathlineto{\pgfqpoint{3.429201in}{1.697384in}}%
\pgfpathlineto{\pgfqpoint{3.429201in}{1.691485in}}%
\pgfpathmoveto{\pgfqpoint{3.401955in}{1.703283in}}%
\pgfpathlineto{\pgfqpoint{3.401955in}{1.703283in}}%
\pgfpathlineto{\pgfqpoint{3.401955in}{1.709181in}}%
\pgfpathlineto{\pgfqpoint{3.411037in}{1.709181in}}%
\pgfpathlineto{\pgfqpoint{3.411037in}{1.703283in}}%
\pgfpathmoveto{\pgfqpoint{3.401955in}{1.709181in}}%
\pgfpathlineto{\pgfqpoint{3.401955in}{1.709181in}}%
\pgfpathlineto{\pgfqpoint{3.401955in}{1.715080in}}%
\pgfpathlineto{\pgfqpoint{3.411037in}{1.715080in}}%
\pgfpathlineto{\pgfqpoint{3.411037in}{1.709181in}}%
\pgfpathmoveto{\pgfqpoint{3.411037in}{1.703283in}}%
\pgfpathlineto{\pgfqpoint{3.411037in}{1.703283in}}%
\pgfpathlineto{\pgfqpoint{3.411037in}{1.709181in}}%
\pgfpathlineto{\pgfqpoint{3.420119in}{1.709181in}}%
\pgfpathlineto{\pgfqpoint{3.420119in}{1.703283in}}%
\pgfpathmoveto{\pgfqpoint{3.438283in}{1.667891in}}%
\pgfpathlineto{\pgfqpoint{3.438283in}{1.667891in}}%
\pgfpathlineto{\pgfqpoint{3.438283in}{1.673789in}}%
\pgfpathlineto{\pgfqpoint{3.447365in}{1.673789in}}%
\pgfpathlineto{\pgfqpoint{3.447365in}{1.667891in}}%
\pgfpathmoveto{\pgfqpoint{3.438283in}{1.673789in}}%
\pgfpathlineto{\pgfqpoint{3.438283in}{1.673789in}}%
\pgfpathlineto{\pgfqpoint{3.438283in}{1.679688in}}%
\pgfpathlineto{\pgfqpoint{3.447365in}{1.679688in}}%
\pgfpathlineto{\pgfqpoint{3.447365in}{1.673789in}}%
\pgfpathmoveto{\pgfqpoint{3.447365in}{1.667891in}}%
\pgfpathlineto{\pgfqpoint{3.447365in}{1.667891in}}%
\pgfpathlineto{\pgfqpoint{3.447365in}{1.673789in}}%
\pgfpathlineto{\pgfqpoint{3.456447in}{1.673789in}}%
\pgfpathlineto{\pgfqpoint{3.456447in}{1.667891in}}%
\pgfpathmoveto{\pgfqpoint{3.456447in}{1.656094in}}%
\pgfpathlineto{\pgfqpoint{3.456447in}{1.656094in}}%
\pgfpathlineto{\pgfqpoint{3.456447in}{1.661992in}}%
\pgfpathlineto{\pgfqpoint{3.465529in}{1.661992in}}%
\pgfpathlineto{\pgfqpoint{3.465529in}{1.656094in}}%
\pgfpathmoveto{\pgfqpoint{3.456447in}{1.661992in}}%
\pgfpathlineto{\pgfqpoint{3.456447in}{1.661992in}}%
\pgfpathlineto{\pgfqpoint{3.456447in}{1.667891in}}%
\pgfpathlineto{\pgfqpoint{3.465529in}{1.667891in}}%
\pgfpathlineto{\pgfqpoint{3.465529in}{1.661992in}}%
\pgfpathmoveto{\pgfqpoint{3.465529in}{1.656094in}}%
\pgfpathlineto{\pgfqpoint{3.465529in}{1.656094in}}%
\pgfpathlineto{\pgfqpoint{3.465529in}{1.661992in}}%
\pgfpathlineto{\pgfqpoint{3.474611in}{1.661992in}}%
\pgfpathlineto{\pgfqpoint{3.474611in}{1.656094in}}%
\pgfpathmoveto{\pgfqpoint{3.474611in}{1.644296in}}%
\pgfpathlineto{\pgfqpoint{3.474611in}{1.644296in}}%
\pgfpathlineto{\pgfqpoint{3.474611in}{1.650195in}}%
\pgfpathlineto{\pgfqpoint{3.483693in}{1.650195in}}%
\pgfpathlineto{\pgfqpoint{3.483693in}{1.644296in}}%
\pgfpathmoveto{\pgfqpoint{3.492775in}{1.632499in}}%
\pgfpathlineto{\pgfqpoint{3.492775in}{1.632499in}}%
\pgfpathlineto{\pgfqpoint{3.492775in}{1.638398in}}%
\pgfpathlineto{\pgfqpoint{3.501857in}{1.638398in}}%
\pgfpathlineto{\pgfqpoint{3.501857in}{1.632499in}}%
\pgfpathmoveto{\pgfqpoint{3.438283in}{1.679688in}}%
\pgfpathlineto{\pgfqpoint{3.438283in}{1.679688in}}%
\pgfpathlineto{\pgfqpoint{3.438283in}{1.685587in}}%
\pgfpathlineto{\pgfqpoint{3.447365in}{1.685587in}}%
\pgfpathlineto{\pgfqpoint{3.447365in}{1.679688in}}%
\pgfpathmoveto{\pgfqpoint{3.365628in}{1.738674in}}%
\pgfpathlineto{\pgfqpoint{3.365628in}{1.738674in}}%
\pgfpathlineto{\pgfqpoint{3.365628in}{1.744572in}}%
\pgfpathlineto{\pgfqpoint{3.374710in}{1.744572in}}%
\pgfpathlineto{\pgfqpoint{3.374710in}{1.738674in}}%
\pgfpathmoveto{\pgfqpoint{3.383792in}{1.726877in}}%
\pgfpathlineto{\pgfqpoint{3.383792in}{1.726877in}}%
\pgfpathlineto{\pgfqpoint{3.383792in}{1.732775in}}%
\pgfpathlineto{\pgfqpoint{3.392873in}{1.732775in}}%
\pgfpathlineto{\pgfqpoint{3.392873in}{1.726877in}}%
\pgfpathmoveto{\pgfqpoint{3.510939in}{0.547186in}}%
\pgfpathlineto{\pgfqpoint{3.510939in}{0.547186in}}%
\pgfpathlineto{\pgfqpoint{3.510939in}{0.553084in}}%
\pgfpathlineto{\pgfqpoint{3.520021in}{0.553084in}}%
\pgfpathlineto{\pgfqpoint{3.520021in}{0.547186in}}%
\pgfpathmoveto{\pgfqpoint{3.510939in}{0.553084in}}%
\pgfpathlineto{\pgfqpoint{3.510939in}{0.553084in}}%
\pgfpathlineto{\pgfqpoint{3.510939in}{0.558983in}}%
\pgfpathlineto{\pgfqpoint{3.520021in}{0.558983in}}%
\pgfpathlineto{\pgfqpoint{3.520021in}{0.553084in}}%
\pgfpathmoveto{\pgfqpoint{3.520021in}{0.553084in}}%
\pgfpathlineto{\pgfqpoint{3.520021in}{0.553084in}}%
\pgfpathlineto{\pgfqpoint{3.520021in}{0.558983in}}%
\pgfpathlineto{\pgfqpoint{3.529103in}{0.558983in}}%
\pgfpathlineto{\pgfqpoint{3.529103in}{0.553084in}}%
\pgfpathmoveto{\pgfqpoint{3.529103in}{0.564881in}}%
\pgfpathlineto{\pgfqpoint{3.529103in}{0.564881in}}%
\pgfpathlineto{\pgfqpoint{3.529103in}{0.570780in}}%
\pgfpathlineto{\pgfqpoint{3.538185in}{0.570780in}}%
\pgfpathlineto{\pgfqpoint{3.538185in}{0.564881in}}%
\pgfpathmoveto{\pgfqpoint{3.547266in}{0.576678in}}%
\pgfpathlineto{\pgfqpoint{3.547266in}{0.576678in}}%
\pgfpathlineto{\pgfqpoint{3.547266in}{0.582576in}}%
\pgfpathlineto{\pgfqpoint{3.556348in}{0.582576in}}%
\pgfpathlineto{\pgfqpoint{3.556348in}{0.576678in}}%
\pgfpathmoveto{\pgfqpoint{3.565430in}{0.588475in}}%
\pgfpathlineto{\pgfqpoint{3.565430in}{0.588475in}}%
\pgfpathlineto{\pgfqpoint{3.565430in}{0.594373in}}%
\pgfpathlineto{\pgfqpoint{3.574512in}{0.594373in}}%
\pgfpathlineto{\pgfqpoint{3.574512in}{0.588475in}}%
\pgfpathmoveto{\pgfqpoint{3.565430in}{0.594373in}}%
\pgfpathlineto{\pgfqpoint{3.565430in}{0.594373in}}%
\pgfpathlineto{\pgfqpoint{3.565430in}{0.600272in}}%
\pgfpathlineto{\pgfqpoint{3.574512in}{0.600272in}}%
\pgfpathlineto{\pgfqpoint{3.574512in}{0.594373in}}%
\pgfpathmoveto{\pgfqpoint{3.565430in}{0.600272in}}%
\pgfpathlineto{\pgfqpoint{3.565430in}{0.600272in}}%
\pgfpathlineto{\pgfqpoint{3.565430in}{0.606171in}}%
\pgfpathlineto{\pgfqpoint{3.574512in}{0.606171in}}%
\pgfpathlineto{\pgfqpoint{3.574512in}{0.600272in}}%
\pgfpathmoveto{\pgfqpoint{3.574512in}{0.600272in}}%
\pgfpathlineto{\pgfqpoint{3.574512in}{0.600272in}}%
\pgfpathlineto{\pgfqpoint{3.574512in}{0.606171in}}%
\pgfpathlineto{\pgfqpoint{3.583594in}{0.606171in}}%
\pgfpathlineto{\pgfqpoint{3.583594in}{0.600272in}}%
\pgfpathmoveto{\pgfqpoint{3.583594in}{0.606171in}}%
\pgfpathlineto{\pgfqpoint{3.583594in}{0.606171in}}%
\pgfpathlineto{\pgfqpoint{3.583594in}{0.612069in}}%
\pgfpathlineto{\pgfqpoint{3.592676in}{0.612069in}}%
\pgfpathlineto{\pgfqpoint{3.592676in}{0.606171in}}%
\pgfpathmoveto{\pgfqpoint{3.583594in}{0.612069in}}%
\pgfpathlineto{\pgfqpoint{3.583594in}{0.612069in}}%
\pgfpathlineto{\pgfqpoint{3.583594in}{0.617968in}}%
\pgfpathlineto{\pgfqpoint{3.592676in}{0.617968in}}%
\pgfpathlineto{\pgfqpoint{3.592676in}{0.612069in}}%
\pgfpathmoveto{\pgfqpoint{3.592676in}{0.612069in}}%
\pgfpathlineto{\pgfqpoint{3.592676in}{0.612069in}}%
\pgfpathlineto{\pgfqpoint{3.592676in}{0.617968in}}%
\pgfpathlineto{\pgfqpoint{3.601758in}{0.617968in}}%
\pgfpathlineto{\pgfqpoint{3.601758in}{0.612069in}}%
\pgfpathmoveto{\pgfqpoint{3.601758in}{0.617968in}}%
\pgfpathlineto{\pgfqpoint{3.601758in}{0.617968in}}%
\pgfpathlineto{\pgfqpoint{3.601758in}{0.623867in}}%
\pgfpathlineto{\pgfqpoint{3.610840in}{0.623867in}}%
\pgfpathlineto{\pgfqpoint{3.610840in}{0.617968in}}%
\pgfpathmoveto{\pgfqpoint{3.601758in}{0.623867in}}%
\pgfpathlineto{\pgfqpoint{3.601758in}{0.623867in}}%
\pgfpathlineto{\pgfqpoint{3.601758in}{0.629766in}}%
\pgfpathlineto{\pgfqpoint{3.610840in}{0.629766in}}%
\pgfpathlineto{\pgfqpoint{3.610840in}{0.623867in}}%
\pgfpathmoveto{\pgfqpoint{3.610840in}{0.623867in}}%
\pgfpathlineto{\pgfqpoint{3.610840in}{0.623867in}}%
\pgfpathlineto{\pgfqpoint{3.610840in}{0.629766in}}%
\pgfpathlineto{\pgfqpoint{3.619922in}{0.629766in}}%
\pgfpathlineto{\pgfqpoint{3.619922in}{0.623867in}}%
\pgfpathmoveto{\pgfqpoint{3.619922in}{0.635664in}}%
\pgfpathlineto{\pgfqpoint{3.619922in}{0.635664in}}%
\pgfpathlineto{\pgfqpoint{3.619922in}{0.641563in}}%
\pgfpathlineto{\pgfqpoint{3.629004in}{0.641563in}}%
\pgfpathlineto{\pgfqpoint{3.629004in}{0.635664in}}%
\pgfpathmoveto{\pgfqpoint{3.638086in}{0.647462in}}%
\pgfpathlineto{\pgfqpoint{3.638086in}{0.647462in}}%
\pgfpathlineto{\pgfqpoint{3.638086in}{0.653360in}}%
\pgfpathlineto{\pgfqpoint{3.647168in}{0.653360in}}%
\pgfpathlineto{\pgfqpoint{3.647168in}{0.647462in}}%
\pgfpathmoveto{\pgfqpoint{3.601758in}{1.526331in}}%
\pgfpathlineto{\pgfqpoint{3.601758in}{1.526331in}}%
\pgfpathlineto{\pgfqpoint{3.601758in}{1.532229in}}%
\pgfpathlineto{\pgfqpoint{3.610840in}{1.532229in}}%
\pgfpathlineto{\pgfqpoint{3.610840in}{1.526331in}}%
\pgfpathmoveto{\pgfqpoint{3.601758in}{1.532229in}}%
\pgfpathlineto{\pgfqpoint{3.601758in}{1.532229in}}%
\pgfpathlineto{\pgfqpoint{3.601758in}{1.538128in}}%
\pgfpathlineto{\pgfqpoint{3.610840in}{1.538128in}}%
\pgfpathlineto{\pgfqpoint{3.610840in}{1.532229in}}%
\pgfpathmoveto{\pgfqpoint{3.610840in}{1.526331in}}%
\pgfpathlineto{\pgfqpoint{3.610840in}{1.526331in}}%
\pgfpathlineto{\pgfqpoint{3.610840in}{1.532229in}}%
\pgfpathlineto{\pgfqpoint{3.619922in}{1.532229in}}%
\pgfpathlineto{\pgfqpoint{3.619922in}{1.526331in}}%
\pgfpathmoveto{\pgfqpoint{3.638086in}{1.502737in}}%
\pgfpathlineto{\pgfqpoint{3.638086in}{1.502737in}}%
\pgfpathlineto{\pgfqpoint{3.638086in}{1.508635in}}%
\pgfpathlineto{\pgfqpoint{3.647168in}{1.508635in}}%
\pgfpathlineto{\pgfqpoint{3.647168in}{1.502737in}}%
\pgfpathmoveto{\pgfqpoint{3.619922in}{1.514534in}}%
\pgfpathlineto{\pgfqpoint{3.619922in}{1.514534in}}%
\pgfpathlineto{\pgfqpoint{3.619922in}{1.520432in}}%
\pgfpathlineto{\pgfqpoint{3.629004in}{1.520432in}}%
\pgfpathlineto{\pgfqpoint{3.629004in}{1.514534in}}%
\pgfpathmoveto{\pgfqpoint{3.619922in}{1.520432in}}%
\pgfpathlineto{\pgfqpoint{3.619922in}{1.520432in}}%
\pgfpathlineto{\pgfqpoint{3.619922in}{1.526331in}}%
\pgfpathlineto{\pgfqpoint{3.629004in}{1.526331in}}%
\pgfpathlineto{\pgfqpoint{3.629004in}{1.520432in}}%
\pgfpathmoveto{\pgfqpoint{3.629004in}{1.514534in}}%
\pgfpathlineto{\pgfqpoint{3.629004in}{1.514534in}}%
\pgfpathlineto{\pgfqpoint{3.629004in}{1.520432in}}%
\pgfpathlineto{\pgfqpoint{3.638086in}{1.520432in}}%
\pgfpathlineto{\pgfqpoint{3.638086in}{1.514534in}}%
\pgfpathmoveto{\pgfqpoint{3.547266in}{1.573517in}}%
\pgfpathlineto{\pgfqpoint{3.547266in}{1.573517in}}%
\pgfpathlineto{\pgfqpoint{3.547266in}{1.579415in}}%
\pgfpathlineto{\pgfqpoint{3.556348in}{1.579415in}}%
\pgfpathlineto{\pgfqpoint{3.556348in}{1.573517in}}%
\pgfpathmoveto{\pgfqpoint{3.547266in}{1.579415in}}%
\pgfpathlineto{\pgfqpoint{3.547266in}{1.579415in}}%
\pgfpathlineto{\pgfqpoint{3.547266in}{1.585313in}}%
\pgfpathlineto{\pgfqpoint{3.556348in}{1.585313in}}%
\pgfpathlineto{\pgfqpoint{3.556348in}{1.579415in}}%
\pgfpathmoveto{\pgfqpoint{3.556348in}{1.573517in}}%
\pgfpathlineto{\pgfqpoint{3.556348in}{1.573517in}}%
\pgfpathlineto{\pgfqpoint{3.556348in}{1.579415in}}%
\pgfpathlineto{\pgfqpoint{3.565430in}{1.579415in}}%
\pgfpathlineto{\pgfqpoint{3.565430in}{1.573517in}}%
\pgfpathmoveto{\pgfqpoint{3.565430in}{1.561720in}}%
\pgfpathlineto{\pgfqpoint{3.565430in}{1.561720in}}%
\pgfpathlineto{\pgfqpoint{3.565430in}{1.567619in}}%
\pgfpathlineto{\pgfqpoint{3.574512in}{1.567619in}}%
\pgfpathlineto{\pgfqpoint{3.574512in}{1.561720in}}%
\pgfpathmoveto{\pgfqpoint{3.565430in}{1.567619in}}%
\pgfpathlineto{\pgfqpoint{3.565430in}{1.567619in}}%
\pgfpathlineto{\pgfqpoint{3.565430in}{1.573517in}}%
\pgfpathlineto{\pgfqpoint{3.574512in}{1.573517in}}%
\pgfpathlineto{\pgfqpoint{3.574512in}{1.567619in}}%
\pgfpathmoveto{\pgfqpoint{3.574512in}{1.561720in}}%
\pgfpathlineto{\pgfqpoint{3.574512in}{1.561720in}}%
\pgfpathlineto{\pgfqpoint{3.574512in}{1.567619in}}%
\pgfpathlineto{\pgfqpoint{3.583594in}{1.567619in}}%
\pgfpathlineto{\pgfqpoint{3.583594in}{1.561720in}}%
\pgfpathmoveto{\pgfqpoint{3.529103in}{1.597110in}}%
\pgfpathlineto{\pgfqpoint{3.529103in}{1.597110in}}%
\pgfpathlineto{\pgfqpoint{3.529103in}{1.603008in}}%
\pgfpathlineto{\pgfqpoint{3.538185in}{1.603008in}}%
\pgfpathlineto{\pgfqpoint{3.538185in}{1.597110in}}%
\pgfpathmoveto{\pgfqpoint{3.510939in}{1.608906in}}%
\pgfpathlineto{\pgfqpoint{3.510939in}{1.608906in}}%
\pgfpathlineto{\pgfqpoint{3.510939in}{1.614804in}}%
\pgfpathlineto{\pgfqpoint{3.520021in}{1.614804in}}%
\pgfpathlineto{\pgfqpoint{3.520021in}{1.608906in}}%
\pgfpathmoveto{\pgfqpoint{3.510939in}{1.614804in}}%
\pgfpathlineto{\pgfqpoint{3.510939in}{1.614804in}}%
\pgfpathlineto{\pgfqpoint{3.510939in}{1.620703in}}%
\pgfpathlineto{\pgfqpoint{3.520021in}{1.620703in}}%
\pgfpathlineto{\pgfqpoint{3.520021in}{1.614804in}}%
\pgfpathmoveto{\pgfqpoint{3.520021in}{1.608906in}}%
\pgfpathlineto{\pgfqpoint{3.520021in}{1.608906in}}%
\pgfpathlineto{\pgfqpoint{3.520021in}{1.614804in}}%
\pgfpathlineto{\pgfqpoint{3.529103in}{1.614804in}}%
\pgfpathlineto{\pgfqpoint{3.529103in}{1.608906in}}%
\pgfpathmoveto{\pgfqpoint{3.547266in}{1.585313in}}%
\pgfpathlineto{\pgfqpoint{3.547266in}{1.585313in}}%
\pgfpathlineto{\pgfqpoint{3.547266in}{1.591212in}}%
\pgfpathlineto{\pgfqpoint{3.556348in}{1.591212in}}%
\pgfpathlineto{\pgfqpoint{3.556348in}{1.585313in}}%
\pgfpathmoveto{\pgfqpoint{3.583594in}{1.549924in}}%
\pgfpathlineto{\pgfqpoint{3.583594in}{1.549924in}}%
\pgfpathlineto{\pgfqpoint{3.583594in}{1.555822in}}%
\pgfpathlineto{\pgfqpoint{3.592676in}{1.555822in}}%
\pgfpathlineto{\pgfqpoint{3.592676in}{1.549924in}}%
\pgfpathmoveto{\pgfqpoint{3.601758in}{1.538128in}}%
\pgfpathlineto{\pgfqpoint{3.601758in}{1.538128in}}%
\pgfpathlineto{\pgfqpoint{3.601758in}{1.544026in}}%
\pgfpathlineto{\pgfqpoint{3.610840in}{1.544026in}}%
\pgfpathlineto{\pgfqpoint{3.610840in}{1.538128in}}%
\pgfpathmoveto{\pgfqpoint{3.656250in}{0.659259in}}%
\pgfpathlineto{\pgfqpoint{3.656250in}{0.659259in}}%
\pgfpathlineto{\pgfqpoint{3.656250in}{0.665158in}}%
\pgfpathlineto{\pgfqpoint{3.665333in}{0.665158in}}%
\pgfpathlineto{\pgfqpoint{3.665333in}{0.659259in}}%
\pgfpathmoveto{\pgfqpoint{3.674415in}{0.676955in}}%
\pgfpathlineto{\pgfqpoint{3.674415in}{0.676955in}}%
\pgfpathlineto{\pgfqpoint{3.674415in}{0.682854in}}%
\pgfpathlineto{\pgfqpoint{3.683497in}{0.682854in}}%
\pgfpathlineto{\pgfqpoint{3.683497in}{0.676955in}}%
\pgfpathmoveto{\pgfqpoint{3.674415in}{0.682854in}}%
\pgfpathlineto{\pgfqpoint{3.674415in}{0.682854in}}%
\pgfpathlineto{\pgfqpoint{3.674415in}{0.688753in}}%
\pgfpathlineto{\pgfqpoint{3.683497in}{0.688753in}}%
\pgfpathlineto{\pgfqpoint{3.683497in}{0.682854in}}%
\pgfpathmoveto{\pgfqpoint{3.683497in}{0.682854in}}%
\pgfpathlineto{\pgfqpoint{3.683497in}{0.682854in}}%
\pgfpathlineto{\pgfqpoint{3.683497in}{0.688753in}}%
\pgfpathlineto{\pgfqpoint{3.692579in}{0.688753in}}%
\pgfpathlineto{\pgfqpoint{3.692579in}{0.682854in}}%
\pgfpathmoveto{\pgfqpoint{3.692579in}{0.688753in}}%
\pgfpathlineto{\pgfqpoint{3.692579in}{0.688753in}}%
\pgfpathlineto{\pgfqpoint{3.692579in}{0.694651in}}%
\pgfpathlineto{\pgfqpoint{3.701662in}{0.694651in}}%
\pgfpathlineto{\pgfqpoint{3.701662in}{0.688753in}}%
\pgfpathmoveto{\pgfqpoint{3.692579in}{0.694651in}}%
\pgfpathlineto{\pgfqpoint{3.692579in}{0.694651in}}%
\pgfpathlineto{\pgfqpoint{3.692579in}{0.700549in}}%
\pgfpathlineto{\pgfqpoint{3.701662in}{0.700549in}}%
\pgfpathlineto{\pgfqpoint{3.701662in}{0.694651in}}%
\pgfpathmoveto{\pgfqpoint{3.701662in}{0.694651in}}%
\pgfpathlineto{\pgfqpoint{3.701662in}{0.694651in}}%
\pgfpathlineto{\pgfqpoint{3.701662in}{0.700549in}}%
\pgfpathlineto{\pgfqpoint{3.710744in}{0.700549in}}%
\pgfpathlineto{\pgfqpoint{3.710744in}{0.694651in}}%
\pgfpathmoveto{\pgfqpoint{3.710744in}{0.700549in}}%
\pgfpathlineto{\pgfqpoint{3.710744in}{0.700549in}}%
\pgfpathlineto{\pgfqpoint{3.710744in}{0.706448in}}%
\pgfpathlineto{\pgfqpoint{3.719826in}{0.706448in}}%
\pgfpathlineto{\pgfqpoint{3.719826in}{0.700549in}}%
\pgfpathmoveto{\pgfqpoint{3.710744in}{0.706448in}}%
\pgfpathlineto{\pgfqpoint{3.710744in}{0.706448in}}%
\pgfpathlineto{\pgfqpoint{3.710744in}{0.712346in}}%
\pgfpathlineto{\pgfqpoint{3.719826in}{0.712346in}}%
\pgfpathlineto{\pgfqpoint{3.719826in}{0.706448in}}%
\pgfpathmoveto{\pgfqpoint{3.728908in}{0.718244in}}%
\pgfpathlineto{\pgfqpoint{3.728908in}{0.718244in}}%
\pgfpathlineto{\pgfqpoint{3.728908in}{0.724142in}}%
\pgfpathlineto{\pgfqpoint{3.737990in}{0.724142in}}%
\pgfpathlineto{\pgfqpoint{3.737990in}{0.718244in}}%
\pgfpathmoveto{\pgfqpoint{3.747073in}{0.730041in}}%
\pgfpathlineto{\pgfqpoint{3.747073in}{0.730041in}}%
\pgfpathlineto{\pgfqpoint{3.747073in}{0.735939in}}%
\pgfpathlineto{\pgfqpoint{3.756155in}{0.735939in}}%
\pgfpathlineto{\pgfqpoint{3.756155in}{0.730041in}}%
\pgfpathmoveto{\pgfqpoint{3.747073in}{0.735939in}}%
\pgfpathlineto{\pgfqpoint{3.747073in}{0.735939in}}%
\pgfpathlineto{\pgfqpoint{3.747073in}{0.741837in}}%
\pgfpathlineto{\pgfqpoint{3.756155in}{0.741837in}}%
\pgfpathlineto{\pgfqpoint{3.756155in}{0.735939in}}%
\pgfpathmoveto{\pgfqpoint{3.747073in}{0.741837in}}%
\pgfpathlineto{\pgfqpoint{3.747073in}{0.741837in}}%
\pgfpathlineto{\pgfqpoint{3.747073in}{0.747736in}}%
\pgfpathlineto{\pgfqpoint{3.756155in}{0.747736in}}%
\pgfpathlineto{\pgfqpoint{3.756155in}{0.741837in}}%
\pgfpathmoveto{\pgfqpoint{3.756155in}{0.741837in}}%
\pgfpathlineto{\pgfqpoint{3.756155in}{0.741837in}}%
\pgfpathlineto{\pgfqpoint{3.756155in}{0.747736in}}%
\pgfpathlineto{\pgfqpoint{3.765237in}{0.747736in}}%
\pgfpathlineto{\pgfqpoint{3.765237in}{0.741837in}}%
\pgfpathmoveto{\pgfqpoint{3.765237in}{0.747736in}}%
\pgfpathlineto{\pgfqpoint{3.765237in}{0.747736in}}%
\pgfpathlineto{\pgfqpoint{3.765237in}{0.753634in}}%
\pgfpathlineto{\pgfqpoint{3.774319in}{0.753634in}}%
\pgfpathlineto{\pgfqpoint{3.774319in}{0.747736in}}%
\pgfpathmoveto{\pgfqpoint{3.765237in}{0.753634in}}%
\pgfpathlineto{\pgfqpoint{3.765237in}{0.753634in}}%
\pgfpathlineto{\pgfqpoint{3.765237in}{0.759532in}}%
\pgfpathlineto{\pgfqpoint{3.774319in}{0.759532in}}%
\pgfpathlineto{\pgfqpoint{3.774319in}{0.753634in}}%
\pgfpathmoveto{\pgfqpoint{3.774319in}{0.753634in}}%
\pgfpathlineto{\pgfqpoint{3.774319in}{0.753634in}}%
\pgfpathlineto{\pgfqpoint{3.774319in}{0.759532in}}%
\pgfpathlineto{\pgfqpoint{3.783402in}{0.759532in}}%
\pgfpathlineto{\pgfqpoint{3.783402in}{0.753634in}}%
\pgfpathmoveto{\pgfqpoint{3.783402in}{0.759532in}}%
\pgfpathlineto{\pgfqpoint{3.783402in}{0.759532in}}%
\pgfpathlineto{\pgfqpoint{3.783402in}{0.765430in}}%
\pgfpathlineto{\pgfqpoint{3.792484in}{0.765430in}}%
\pgfpathlineto{\pgfqpoint{3.792484in}{0.759532in}}%
\pgfpathmoveto{\pgfqpoint{3.783402in}{0.765430in}}%
\pgfpathlineto{\pgfqpoint{3.783402in}{0.765430in}}%
\pgfpathlineto{\pgfqpoint{3.783402in}{0.771329in}}%
\pgfpathlineto{\pgfqpoint{3.792484in}{0.771329in}}%
\pgfpathlineto{\pgfqpoint{3.792484in}{0.765430in}}%
\pgfpathmoveto{\pgfqpoint{3.792484in}{0.765430in}}%
\pgfpathlineto{\pgfqpoint{3.792484in}{0.765430in}}%
\pgfpathlineto{\pgfqpoint{3.792484in}{0.771329in}}%
\pgfpathlineto{\pgfqpoint{3.801566in}{0.771329in}}%
\pgfpathlineto{\pgfqpoint{3.801566in}{0.765430in}}%
\pgfpathmoveto{\pgfqpoint{3.710744in}{1.431955in}}%
\pgfpathlineto{\pgfqpoint{3.710744in}{1.431955in}}%
\pgfpathlineto{\pgfqpoint{3.710744in}{1.437853in}}%
\pgfpathlineto{\pgfqpoint{3.719826in}{1.437853in}}%
\pgfpathlineto{\pgfqpoint{3.719826in}{1.431955in}}%
\pgfpathmoveto{\pgfqpoint{3.710744in}{1.437853in}}%
\pgfpathlineto{\pgfqpoint{3.710744in}{1.437853in}}%
\pgfpathlineto{\pgfqpoint{3.710744in}{1.443752in}}%
\pgfpathlineto{\pgfqpoint{3.719826in}{1.443752in}}%
\pgfpathlineto{\pgfqpoint{3.719826in}{1.437853in}}%
\pgfpathmoveto{\pgfqpoint{3.719826in}{1.431955in}}%
\pgfpathlineto{\pgfqpoint{3.719826in}{1.431955in}}%
\pgfpathlineto{\pgfqpoint{3.719826in}{1.437853in}}%
\pgfpathlineto{\pgfqpoint{3.728908in}{1.437853in}}%
\pgfpathlineto{\pgfqpoint{3.728908in}{1.431955in}}%
\pgfpathmoveto{\pgfqpoint{3.765237in}{1.384765in}}%
\pgfpathlineto{\pgfqpoint{3.765237in}{1.384765in}}%
\pgfpathlineto{\pgfqpoint{3.765237in}{1.390664in}}%
\pgfpathlineto{\pgfqpoint{3.774319in}{1.390664in}}%
\pgfpathlineto{\pgfqpoint{3.774319in}{1.384765in}}%
\pgfpathmoveto{\pgfqpoint{3.765237in}{1.390664in}}%
\pgfpathlineto{\pgfqpoint{3.765237in}{1.390664in}}%
\pgfpathlineto{\pgfqpoint{3.765237in}{1.396562in}}%
\pgfpathlineto{\pgfqpoint{3.774319in}{1.396562in}}%
\pgfpathlineto{\pgfqpoint{3.774319in}{1.390664in}}%
\pgfpathmoveto{\pgfqpoint{3.774319in}{1.384765in}}%
\pgfpathlineto{\pgfqpoint{3.774319in}{1.384765in}}%
\pgfpathlineto{\pgfqpoint{3.774319in}{1.390664in}}%
\pgfpathlineto{\pgfqpoint{3.783402in}{1.390664in}}%
\pgfpathlineto{\pgfqpoint{3.783402in}{1.384765in}}%
\pgfpathmoveto{\pgfqpoint{3.783402in}{1.372968in}}%
\pgfpathlineto{\pgfqpoint{3.783402in}{1.372968in}}%
\pgfpathlineto{\pgfqpoint{3.783402in}{1.378866in}}%
\pgfpathlineto{\pgfqpoint{3.792484in}{1.378866in}}%
\pgfpathlineto{\pgfqpoint{3.792484in}{1.372968in}}%
\pgfpathmoveto{\pgfqpoint{3.783402in}{1.378866in}}%
\pgfpathlineto{\pgfqpoint{3.783402in}{1.378866in}}%
\pgfpathlineto{\pgfqpoint{3.783402in}{1.384765in}}%
\pgfpathlineto{\pgfqpoint{3.792484in}{1.384765in}}%
\pgfpathlineto{\pgfqpoint{3.792484in}{1.378866in}}%
\pgfpathmoveto{\pgfqpoint{3.792484in}{1.372968in}}%
\pgfpathlineto{\pgfqpoint{3.792484in}{1.372968in}}%
\pgfpathlineto{\pgfqpoint{3.792484in}{1.378866in}}%
\pgfpathlineto{\pgfqpoint{3.801566in}{1.378866in}}%
\pgfpathlineto{\pgfqpoint{3.801566in}{1.372968in}}%
\pgfpathmoveto{\pgfqpoint{3.747073in}{1.408360in}}%
\pgfpathlineto{\pgfqpoint{3.747073in}{1.408360in}}%
\pgfpathlineto{\pgfqpoint{3.747073in}{1.414258in}}%
\pgfpathlineto{\pgfqpoint{3.756155in}{1.414258in}}%
\pgfpathlineto{\pgfqpoint{3.756155in}{1.408360in}}%
\pgfpathmoveto{\pgfqpoint{3.728908in}{1.420157in}}%
\pgfpathlineto{\pgfqpoint{3.728908in}{1.420157in}}%
\pgfpathlineto{\pgfqpoint{3.728908in}{1.426056in}}%
\pgfpathlineto{\pgfqpoint{3.737990in}{1.426056in}}%
\pgfpathlineto{\pgfqpoint{3.737990in}{1.420157in}}%
\pgfpathmoveto{\pgfqpoint{3.728908in}{1.426056in}}%
\pgfpathlineto{\pgfqpoint{3.728908in}{1.426056in}}%
\pgfpathlineto{\pgfqpoint{3.728908in}{1.431955in}}%
\pgfpathlineto{\pgfqpoint{3.737990in}{1.431955in}}%
\pgfpathlineto{\pgfqpoint{3.737990in}{1.426056in}}%
\pgfpathmoveto{\pgfqpoint{3.737990in}{1.420157in}}%
\pgfpathlineto{\pgfqpoint{3.737990in}{1.420157in}}%
\pgfpathlineto{\pgfqpoint{3.737990in}{1.426056in}}%
\pgfpathlineto{\pgfqpoint{3.747073in}{1.426056in}}%
\pgfpathlineto{\pgfqpoint{3.747073in}{1.420157in}}%
\pgfpathmoveto{\pgfqpoint{3.765237in}{1.396562in}}%
\pgfpathlineto{\pgfqpoint{3.765237in}{1.396562in}}%
\pgfpathlineto{\pgfqpoint{3.765237in}{1.402461in}}%
\pgfpathlineto{\pgfqpoint{3.774319in}{1.402461in}}%
\pgfpathlineto{\pgfqpoint{3.774319in}{1.396562in}}%
\pgfpathmoveto{\pgfqpoint{3.656250in}{1.479143in}}%
\pgfpathlineto{\pgfqpoint{3.656250in}{1.479143in}}%
\pgfpathlineto{\pgfqpoint{3.656250in}{1.485041in}}%
\pgfpathlineto{\pgfqpoint{3.665333in}{1.485041in}}%
\pgfpathlineto{\pgfqpoint{3.665333in}{1.479143in}}%
\pgfpathmoveto{\pgfqpoint{3.656250in}{1.485041in}}%
\pgfpathlineto{\pgfqpoint{3.656250in}{1.485041in}}%
\pgfpathlineto{\pgfqpoint{3.656250in}{1.490940in}}%
\pgfpathlineto{\pgfqpoint{3.665333in}{1.490940in}}%
\pgfpathlineto{\pgfqpoint{3.665333in}{1.485041in}}%
\pgfpathmoveto{\pgfqpoint{3.665333in}{1.479143in}}%
\pgfpathlineto{\pgfqpoint{3.665333in}{1.479143in}}%
\pgfpathlineto{\pgfqpoint{3.665333in}{1.485041in}}%
\pgfpathlineto{\pgfqpoint{3.674415in}{1.485041in}}%
\pgfpathlineto{\pgfqpoint{3.674415in}{1.479143in}}%
\pgfpathmoveto{\pgfqpoint{3.674415in}{1.467346in}}%
\pgfpathlineto{\pgfqpoint{3.674415in}{1.467346in}}%
\pgfpathlineto{\pgfqpoint{3.674415in}{1.473244in}}%
\pgfpathlineto{\pgfqpoint{3.683497in}{1.473244in}}%
\pgfpathlineto{\pgfqpoint{3.683497in}{1.467346in}}%
\pgfpathmoveto{\pgfqpoint{3.674415in}{1.473244in}}%
\pgfpathlineto{\pgfqpoint{3.674415in}{1.473244in}}%
\pgfpathlineto{\pgfqpoint{3.674415in}{1.479143in}}%
\pgfpathlineto{\pgfqpoint{3.683497in}{1.479143in}}%
\pgfpathlineto{\pgfqpoint{3.683497in}{1.473244in}}%
\pgfpathmoveto{\pgfqpoint{3.683497in}{1.467346in}}%
\pgfpathlineto{\pgfqpoint{3.683497in}{1.467346in}}%
\pgfpathlineto{\pgfqpoint{3.683497in}{1.473244in}}%
\pgfpathlineto{\pgfqpoint{3.692579in}{1.473244in}}%
\pgfpathlineto{\pgfqpoint{3.692579in}{1.467346in}}%
\pgfpathmoveto{\pgfqpoint{3.692579in}{1.455549in}}%
\pgfpathlineto{\pgfqpoint{3.692579in}{1.455549in}}%
\pgfpathlineto{\pgfqpoint{3.692579in}{1.461447in}}%
\pgfpathlineto{\pgfqpoint{3.701662in}{1.461447in}}%
\pgfpathlineto{\pgfqpoint{3.701662in}{1.455549in}}%
\pgfpathmoveto{\pgfqpoint{3.710744in}{1.443752in}}%
\pgfpathlineto{\pgfqpoint{3.710744in}{1.443752in}}%
\pgfpathlineto{\pgfqpoint{3.710744in}{1.449650in}}%
\pgfpathlineto{\pgfqpoint{3.719826in}{1.449650in}}%
\pgfpathlineto{\pgfqpoint{3.719826in}{1.443752in}}%
\pgfpathmoveto{\pgfqpoint{3.656250in}{1.490940in}}%
\pgfpathlineto{\pgfqpoint{3.656250in}{1.490940in}}%
\pgfpathlineto{\pgfqpoint{3.656250in}{1.496838in}}%
\pgfpathlineto{\pgfqpoint{3.665333in}{1.496838in}}%
\pgfpathlineto{\pgfqpoint{3.665333in}{1.490940in}}%
\pgfpathmoveto{\pgfqpoint{3.801566in}{0.777227in}}%
\pgfpathlineto{\pgfqpoint{3.801566in}{0.777227in}}%
\pgfpathlineto{\pgfqpoint{3.801566in}{0.783125in}}%
\pgfpathlineto{\pgfqpoint{3.810648in}{0.783125in}}%
\pgfpathlineto{\pgfqpoint{3.810648in}{0.777227in}}%
\pgfpathmoveto{\pgfqpoint{3.819730in}{0.789024in}}%
\pgfpathlineto{\pgfqpoint{3.819730in}{0.789024in}}%
\pgfpathlineto{\pgfqpoint{3.819730in}{0.794922in}}%
\pgfpathlineto{\pgfqpoint{3.828811in}{0.794922in}}%
\pgfpathlineto{\pgfqpoint{3.828811in}{0.789024in}}%
\pgfpathmoveto{\pgfqpoint{3.837893in}{0.800821in}}%
\pgfpathlineto{\pgfqpoint{3.837893in}{0.800821in}}%
\pgfpathlineto{\pgfqpoint{3.837893in}{0.806719in}}%
\pgfpathlineto{\pgfqpoint{3.846975in}{0.806719in}}%
\pgfpathlineto{\pgfqpoint{3.846975in}{0.800821in}}%
\pgfpathmoveto{\pgfqpoint{3.837893in}{0.806719in}}%
\pgfpathlineto{\pgfqpoint{3.837893in}{0.806719in}}%
\pgfpathlineto{\pgfqpoint{3.837893in}{0.812618in}}%
\pgfpathlineto{\pgfqpoint{3.846975in}{0.812618in}}%
\pgfpathlineto{\pgfqpoint{3.846975in}{0.806719in}}%
\pgfpathmoveto{\pgfqpoint{3.837893in}{0.812618in}}%
\pgfpathlineto{\pgfqpoint{3.837893in}{0.812618in}}%
\pgfpathlineto{\pgfqpoint{3.837893in}{0.818516in}}%
\pgfpathlineto{\pgfqpoint{3.846975in}{0.818516in}}%
\pgfpathlineto{\pgfqpoint{3.846975in}{0.812618in}}%
\pgfpathmoveto{\pgfqpoint{3.846975in}{0.812618in}}%
\pgfpathlineto{\pgfqpoint{3.846975in}{0.812618in}}%
\pgfpathlineto{\pgfqpoint{3.846975in}{0.818516in}}%
\pgfpathlineto{\pgfqpoint{3.856056in}{0.818516in}}%
\pgfpathlineto{\pgfqpoint{3.856056in}{0.812618in}}%
\pgfpathmoveto{\pgfqpoint{3.856056in}{0.818516in}}%
\pgfpathlineto{\pgfqpoint{3.856056in}{0.818516in}}%
\pgfpathlineto{\pgfqpoint{3.856056in}{0.824415in}}%
\pgfpathlineto{\pgfqpoint{3.865138in}{0.824415in}}%
\pgfpathlineto{\pgfqpoint{3.865138in}{0.818516in}}%
\pgfpathmoveto{\pgfqpoint{3.856056in}{0.824415in}}%
\pgfpathlineto{\pgfqpoint{3.856056in}{0.824415in}}%
\pgfpathlineto{\pgfqpoint{3.856056in}{0.830313in}}%
\pgfpathlineto{\pgfqpoint{3.865138in}{0.830313in}}%
\pgfpathlineto{\pgfqpoint{3.865138in}{0.824415in}}%
\pgfpathmoveto{\pgfqpoint{3.865138in}{0.824415in}}%
\pgfpathlineto{\pgfqpoint{3.865138in}{0.824415in}}%
\pgfpathlineto{\pgfqpoint{3.865138in}{0.830313in}}%
\pgfpathlineto{\pgfqpoint{3.874220in}{0.830313in}}%
\pgfpathlineto{\pgfqpoint{3.874220in}{0.824415in}}%
\pgfpathmoveto{\pgfqpoint{3.874220in}{0.830313in}}%
\pgfpathlineto{\pgfqpoint{3.874220in}{0.830313in}}%
\pgfpathlineto{\pgfqpoint{3.874220in}{0.836212in}}%
\pgfpathlineto{\pgfqpoint{3.883301in}{0.836212in}}%
\pgfpathlineto{\pgfqpoint{3.883301in}{0.830313in}}%
\pgfpathmoveto{\pgfqpoint{3.874220in}{0.836212in}}%
\pgfpathlineto{\pgfqpoint{3.874220in}{0.836212in}}%
\pgfpathlineto{\pgfqpoint{3.874220in}{0.842110in}}%
\pgfpathlineto{\pgfqpoint{3.883301in}{0.842110in}}%
\pgfpathlineto{\pgfqpoint{3.883301in}{0.836212in}}%
\pgfpathmoveto{\pgfqpoint{3.883301in}{0.836212in}}%
\pgfpathlineto{\pgfqpoint{3.883301in}{0.836212in}}%
\pgfpathlineto{\pgfqpoint{3.883301in}{0.842110in}}%
\pgfpathlineto{\pgfqpoint{3.892383in}{0.842110in}}%
\pgfpathlineto{\pgfqpoint{3.892383in}{0.836212in}}%
\pgfpathmoveto{\pgfqpoint{3.892383in}{0.842110in}}%
\pgfpathlineto{\pgfqpoint{3.892383in}{0.842110in}}%
\pgfpathlineto{\pgfqpoint{3.892383in}{0.848008in}}%
\pgfpathlineto{\pgfqpoint{3.901465in}{0.848008in}}%
\pgfpathlineto{\pgfqpoint{3.901465in}{0.842110in}}%
\pgfpathmoveto{\pgfqpoint{3.892383in}{0.848008in}}%
\pgfpathlineto{\pgfqpoint{3.892383in}{0.848008in}}%
\pgfpathlineto{\pgfqpoint{3.892383in}{0.853907in}}%
\pgfpathlineto{\pgfqpoint{3.901465in}{0.853907in}}%
\pgfpathlineto{\pgfqpoint{3.901465in}{0.848008in}}%
\pgfpathmoveto{\pgfqpoint{3.910546in}{0.859805in}}%
\pgfpathlineto{\pgfqpoint{3.910546in}{0.859805in}}%
\pgfpathlineto{\pgfqpoint{3.910546in}{0.865704in}}%
\pgfpathlineto{\pgfqpoint{3.919628in}{0.865704in}}%
\pgfpathlineto{\pgfqpoint{3.919628in}{0.859805in}}%
\pgfpathmoveto{\pgfqpoint{3.928710in}{0.871602in}}%
\pgfpathlineto{\pgfqpoint{3.928710in}{0.871602in}}%
\pgfpathlineto{\pgfqpoint{3.928710in}{0.877501in}}%
\pgfpathlineto{\pgfqpoint{3.937791in}{0.877501in}}%
\pgfpathlineto{\pgfqpoint{3.937791in}{0.871602in}}%
\pgfpathmoveto{\pgfqpoint{3.856056in}{1.313983in}}%
\pgfpathlineto{\pgfqpoint{3.856056in}{1.313983in}}%
\pgfpathlineto{\pgfqpoint{3.856056in}{1.319881in}}%
\pgfpathlineto{\pgfqpoint{3.865138in}{1.319881in}}%
\pgfpathlineto{\pgfqpoint{3.865138in}{1.313983in}}%
\pgfpathmoveto{\pgfqpoint{3.856056in}{1.319881in}}%
\pgfpathlineto{\pgfqpoint{3.856056in}{1.319881in}}%
\pgfpathlineto{\pgfqpoint{3.856056in}{1.325779in}}%
\pgfpathlineto{\pgfqpoint{3.865138in}{1.325779in}}%
\pgfpathlineto{\pgfqpoint{3.865138in}{1.319881in}}%
\pgfpathmoveto{\pgfqpoint{3.837893in}{1.325779in}}%
\pgfpathlineto{\pgfqpoint{3.837893in}{1.325779in}}%
\pgfpathlineto{\pgfqpoint{3.837893in}{1.331678in}}%
\pgfpathlineto{\pgfqpoint{3.846975in}{1.331678in}}%
\pgfpathlineto{\pgfqpoint{3.846975in}{1.325779in}}%
\pgfpathmoveto{\pgfqpoint{3.837893in}{1.331678in}}%
\pgfpathlineto{\pgfqpoint{3.837893in}{1.331678in}}%
\pgfpathlineto{\pgfqpoint{3.837893in}{1.337576in}}%
\pgfpathlineto{\pgfqpoint{3.846975in}{1.337576in}}%
\pgfpathlineto{\pgfqpoint{3.846975in}{1.331678in}}%
\pgfpathmoveto{\pgfqpoint{3.846975in}{1.325779in}}%
\pgfpathlineto{\pgfqpoint{3.846975in}{1.325779in}}%
\pgfpathlineto{\pgfqpoint{3.846975in}{1.331678in}}%
\pgfpathlineto{\pgfqpoint{3.856056in}{1.331678in}}%
\pgfpathlineto{\pgfqpoint{3.856056in}{1.325779in}}%
\pgfpathmoveto{\pgfqpoint{3.892383in}{1.278593in}}%
\pgfpathlineto{\pgfqpoint{3.892383in}{1.278593in}}%
\pgfpathlineto{\pgfqpoint{3.892383in}{1.284491in}}%
\pgfpathlineto{\pgfqpoint{3.901465in}{1.284491in}}%
\pgfpathlineto{\pgfqpoint{3.901465in}{1.278593in}}%
\pgfpathmoveto{\pgfqpoint{3.892383in}{1.284491in}}%
\pgfpathlineto{\pgfqpoint{3.892383in}{1.284491in}}%
\pgfpathlineto{\pgfqpoint{3.892383in}{1.290389in}}%
\pgfpathlineto{\pgfqpoint{3.901465in}{1.290389in}}%
\pgfpathlineto{\pgfqpoint{3.901465in}{1.284491in}}%
\pgfpathmoveto{\pgfqpoint{3.901465in}{1.278593in}}%
\pgfpathlineto{\pgfqpoint{3.901465in}{1.278593in}}%
\pgfpathlineto{\pgfqpoint{3.901465in}{1.284491in}}%
\pgfpathlineto{\pgfqpoint{3.910546in}{1.284491in}}%
\pgfpathlineto{\pgfqpoint{3.910546in}{1.278593in}}%
\pgfpathmoveto{\pgfqpoint{3.910546in}{1.266796in}}%
\pgfpathlineto{\pgfqpoint{3.910546in}{1.266796in}}%
\pgfpathlineto{\pgfqpoint{3.910546in}{1.272694in}}%
\pgfpathlineto{\pgfqpoint{3.919628in}{1.272694in}}%
\pgfpathlineto{\pgfqpoint{3.919628in}{1.266796in}}%
\pgfpathmoveto{\pgfqpoint{3.910546in}{1.272694in}}%
\pgfpathlineto{\pgfqpoint{3.910546in}{1.272694in}}%
\pgfpathlineto{\pgfqpoint{3.910546in}{1.278593in}}%
\pgfpathlineto{\pgfqpoint{3.919628in}{1.278593in}}%
\pgfpathlineto{\pgfqpoint{3.919628in}{1.272694in}}%
\pgfpathmoveto{\pgfqpoint{3.928710in}{1.254999in}}%
\pgfpathlineto{\pgfqpoint{3.928710in}{1.254999in}}%
\pgfpathlineto{\pgfqpoint{3.928710in}{1.260898in}}%
\pgfpathlineto{\pgfqpoint{3.937791in}{1.260898in}}%
\pgfpathlineto{\pgfqpoint{3.937791in}{1.254999in}}%
\pgfpathmoveto{\pgfqpoint{3.874220in}{1.302186in}}%
\pgfpathlineto{\pgfqpoint{3.874220in}{1.302186in}}%
\pgfpathlineto{\pgfqpoint{3.874220in}{1.308084in}}%
\pgfpathlineto{\pgfqpoint{3.883301in}{1.308084in}}%
\pgfpathlineto{\pgfqpoint{3.883301in}{1.302186in}}%
\pgfpathmoveto{\pgfqpoint{3.801566in}{1.361170in}}%
\pgfpathlineto{\pgfqpoint{3.801566in}{1.361170in}}%
\pgfpathlineto{\pgfqpoint{3.801566in}{1.367069in}}%
\pgfpathlineto{\pgfqpoint{3.810648in}{1.367069in}}%
\pgfpathlineto{\pgfqpoint{3.810648in}{1.361170in}}%
\pgfpathmoveto{\pgfqpoint{3.819730in}{1.349373in}}%
\pgfpathlineto{\pgfqpoint{3.819730in}{1.349373in}}%
\pgfpathlineto{\pgfqpoint{3.819730in}{1.355271in}}%
\pgfpathlineto{\pgfqpoint{3.828811in}{1.355271in}}%
\pgfpathlineto{\pgfqpoint{3.828811in}{1.349373in}}%
\pgfpathmoveto{\pgfqpoint{3.946873in}{0.889297in}}%
\pgfpathlineto{\pgfqpoint{3.946873in}{0.889297in}}%
\pgfpathlineto{\pgfqpoint{3.946873in}{0.895196in}}%
\pgfpathlineto{\pgfqpoint{3.955955in}{0.895196in}}%
\pgfpathlineto{\pgfqpoint{3.955955in}{0.889297in}}%
\pgfpathmoveto{\pgfqpoint{3.946873in}{0.895196in}}%
\pgfpathlineto{\pgfqpoint{3.946873in}{0.895196in}}%
\pgfpathlineto{\pgfqpoint{3.946873in}{0.901094in}}%
\pgfpathlineto{\pgfqpoint{3.955955in}{0.901094in}}%
\pgfpathlineto{\pgfqpoint{3.955955in}{0.895196in}}%
\pgfpathmoveto{\pgfqpoint{3.955955in}{0.895196in}}%
\pgfpathlineto{\pgfqpoint{3.955955in}{0.895196in}}%
\pgfpathlineto{\pgfqpoint{3.955955in}{0.901094in}}%
\pgfpathlineto{\pgfqpoint{3.965037in}{0.901094in}}%
\pgfpathlineto{\pgfqpoint{3.965037in}{0.895196in}}%
\pgfpathmoveto{\pgfqpoint{3.965037in}{0.901094in}}%
\pgfpathlineto{\pgfqpoint{3.965037in}{0.901094in}}%
\pgfpathlineto{\pgfqpoint{3.965037in}{0.906992in}}%
\pgfpathlineto{\pgfqpoint{3.974119in}{0.906992in}}%
\pgfpathlineto{\pgfqpoint{3.974119in}{0.901094in}}%
\pgfpathmoveto{\pgfqpoint{3.965037in}{0.906992in}}%
\pgfpathlineto{\pgfqpoint{3.965037in}{0.906992in}}%
\pgfpathlineto{\pgfqpoint{3.965037in}{0.912891in}}%
\pgfpathlineto{\pgfqpoint{3.974119in}{0.912891in}}%
\pgfpathlineto{\pgfqpoint{3.974119in}{0.906992in}}%
\pgfpathmoveto{\pgfqpoint{3.974119in}{0.906992in}}%
\pgfpathlineto{\pgfqpoint{3.974119in}{0.906992in}}%
\pgfpathlineto{\pgfqpoint{3.974119in}{0.912891in}}%
\pgfpathlineto{\pgfqpoint{3.983201in}{0.912891in}}%
\pgfpathlineto{\pgfqpoint{3.983201in}{0.906992in}}%
\pgfpathmoveto{\pgfqpoint{3.983201in}{0.912891in}}%
\pgfpathlineto{\pgfqpoint{3.983201in}{0.912891in}}%
\pgfpathlineto{\pgfqpoint{3.983201in}{0.918789in}}%
\pgfpathlineto{\pgfqpoint{3.992283in}{0.918789in}}%
\pgfpathlineto{\pgfqpoint{3.992283in}{0.912891in}}%
\pgfpathmoveto{\pgfqpoint{3.983201in}{0.918789in}}%
\pgfpathlineto{\pgfqpoint{3.983201in}{0.918789in}}%
\pgfpathlineto{\pgfqpoint{3.983201in}{0.924687in}}%
\pgfpathlineto{\pgfqpoint{3.992283in}{0.924687in}}%
\pgfpathlineto{\pgfqpoint{3.992283in}{0.918789in}}%
\pgfpathmoveto{\pgfqpoint{4.001365in}{0.930585in}}%
\pgfpathlineto{\pgfqpoint{4.001365in}{0.930585in}}%
\pgfpathlineto{\pgfqpoint{4.001365in}{0.936484in}}%
\pgfpathlineto{\pgfqpoint{4.010447in}{0.936484in}}%
\pgfpathlineto{\pgfqpoint{4.010447in}{0.930585in}}%
\pgfpathmoveto{\pgfqpoint{4.019529in}{0.942382in}}%
\pgfpathlineto{\pgfqpoint{4.019529in}{0.942382in}}%
\pgfpathlineto{\pgfqpoint{4.019529in}{0.948280in}}%
\pgfpathlineto{\pgfqpoint{4.028611in}{0.948280in}}%
\pgfpathlineto{\pgfqpoint{4.028611in}{0.942382in}}%
\pgfpathmoveto{\pgfqpoint{4.037692in}{0.960077in}}%
\pgfpathlineto{\pgfqpoint{4.037692in}{0.960077in}}%
\pgfpathlineto{\pgfqpoint{4.037692in}{0.965975in}}%
\pgfpathlineto{\pgfqpoint{4.046774in}{0.965975in}}%
\pgfpathlineto{\pgfqpoint{4.046774in}{0.960077in}}%
\pgfpathmoveto{\pgfqpoint{4.037692in}{0.965975in}}%
\pgfpathlineto{\pgfqpoint{4.037692in}{0.965975in}}%
\pgfpathlineto{\pgfqpoint{4.037692in}{0.971874in}}%
\pgfpathlineto{\pgfqpoint{4.046774in}{0.971874in}}%
\pgfpathlineto{\pgfqpoint{4.046774in}{0.965975in}}%
\pgfpathmoveto{\pgfqpoint{4.046774in}{0.965975in}}%
\pgfpathlineto{\pgfqpoint{4.046774in}{0.965975in}}%
\pgfpathlineto{\pgfqpoint{4.046774in}{0.971874in}}%
\pgfpathlineto{\pgfqpoint{4.055856in}{0.971874in}}%
\pgfpathlineto{\pgfqpoint{4.055856in}{0.965975in}}%
\pgfpathmoveto{\pgfqpoint{4.055856in}{0.971874in}}%
\pgfpathlineto{\pgfqpoint{4.055856in}{0.971874in}}%
\pgfpathlineto{\pgfqpoint{4.055856in}{0.977772in}}%
\pgfpathlineto{\pgfqpoint{4.064938in}{0.977772in}}%
\pgfpathlineto{\pgfqpoint{4.064938in}{0.971874in}}%
\pgfpathmoveto{\pgfqpoint{4.055856in}{0.977772in}}%
\pgfpathlineto{\pgfqpoint{4.055856in}{0.977772in}}%
\pgfpathlineto{\pgfqpoint{4.055856in}{0.983670in}}%
\pgfpathlineto{\pgfqpoint{4.064938in}{0.983670in}}%
\pgfpathlineto{\pgfqpoint{4.064938in}{0.977772in}}%
\pgfpathmoveto{\pgfqpoint{4.064938in}{0.977772in}}%
\pgfpathlineto{\pgfqpoint{4.064938in}{0.977772in}}%
\pgfpathlineto{\pgfqpoint{4.064938in}{0.983670in}}%
\pgfpathlineto{\pgfqpoint{4.074020in}{0.983670in}}%
\pgfpathlineto{\pgfqpoint{4.074020in}{0.977772in}}%
\pgfpathmoveto{\pgfqpoint{4.074020in}{0.983670in}}%
\pgfpathlineto{\pgfqpoint{4.074020in}{0.983670in}}%
\pgfpathlineto{\pgfqpoint{4.074020in}{0.989569in}}%
\pgfpathlineto{\pgfqpoint{4.083102in}{0.989569in}}%
\pgfpathlineto{\pgfqpoint{4.083102in}{0.983670in}}%
\pgfpathmoveto{\pgfqpoint{4.074020in}{0.989569in}}%
\pgfpathlineto{\pgfqpoint{4.074020in}{0.989569in}}%
\pgfpathlineto{\pgfqpoint{4.074020in}{0.995467in}}%
\pgfpathlineto{\pgfqpoint{4.083102in}{0.995467in}}%
\pgfpathlineto{\pgfqpoint{4.083102in}{0.989569in}}%
\pgfpathmoveto{\pgfqpoint{4.074020in}{1.125233in}}%
\pgfpathlineto{\pgfqpoint{4.074020in}{1.125233in}}%
\pgfpathlineto{\pgfqpoint{4.074020in}{1.131132in}}%
\pgfpathlineto{\pgfqpoint{4.083102in}{1.131132in}}%
\pgfpathlineto{\pgfqpoint{4.083102in}{1.125233in}}%
\pgfpathmoveto{\pgfqpoint{4.074020in}{1.131132in}}%
\pgfpathlineto{\pgfqpoint{4.074020in}{1.131132in}}%
\pgfpathlineto{\pgfqpoint{4.074020in}{1.137030in}}%
\pgfpathlineto{\pgfqpoint{4.083102in}{1.137030in}}%
\pgfpathlineto{\pgfqpoint{4.083102in}{1.131132in}}%
\pgfpathmoveto{\pgfqpoint{4.055856in}{1.137030in}}%
\pgfpathlineto{\pgfqpoint{4.055856in}{1.137030in}}%
\pgfpathlineto{\pgfqpoint{4.055856in}{1.142929in}}%
\pgfpathlineto{\pgfqpoint{4.064938in}{1.142929in}}%
\pgfpathlineto{\pgfqpoint{4.064938in}{1.137030in}}%
\pgfpathmoveto{\pgfqpoint{4.055856in}{1.142929in}}%
\pgfpathlineto{\pgfqpoint{4.055856in}{1.142929in}}%
\pgfpathlineto{\pgfqpoint{4.055856in}{1.148827in}}%
\pgfpathlineto{\pgfqpoint{4.064938in}{1.148827in}}%
\pgfpathlineto{\pgfqpoint{4.064938in}{1.142929in}}%
\pgfpathmoveto{\pgfqpoint{4.064938in}{1.137030in}}%
\pgfpathlineto{\pgfqpoint{4.064938in}{1.137030in}}%
\pgfpathlineto{\pgfqpoint{4.064938in}{1.142929in}}%
\pgfpathlineto{\pgfqpoint{4.074020in}{1.142929in}}%
\pgfpathlineto{\pgfqpoint{4.074020in}{1.137030in}}%
\pgfpathmoveto{\pgfqpoint{4.001365in}{1.184218in}}%
\pgfpathlineto{\pgfqpoint{4.001365in}{1.184218in}}%
\pgfpathlineto{\pgfqpoint{4.001365in}{1.190117in}}%
\pgfpathlineto{\pgfqpoint{4.010447in}{1.190117in}}%
\pgfpathlineto{\pgfqpoint{4.010447in}{1.184218in}}%
\pgfpathmoveto{\pgfqpoint{4.001365in}{1.190117in}}%
\pgfpathlineto{\pgfqpoint{4.001365in}{1.190117in}}%
\pgfpathlineto{\pgfqpoint{4.001365in}{1.196015in}}%
\pgfpathlineto{\pgfqpoint{4.010447in}{1.196015in}}%
\pgfpathlineto{\pgfqpoint{4.010447in}{1.190117in}}%
\pgfpathmoveto{\pgfqpoint{4.010447in}{1.184218in}}%
\pgfpathlineto{\pgfqpoint{4.010447in}{1.184218in}}%
\pgfpathlineto{\pgfqpoint{4.010447in}{1.190117in}}%
\pgfpathlineto{\pgfqpoint{4.019529in}{1.190117in}}%
\pgfpathlineto{\pgfqpoint{4.019529in}{1.184218in}}%
\pgfpathmoveto{\pgfqpoint{3.965037in}{1.219609in}}%
\pgfpathlineto{\pgfqpoint{3.965037in}{1.219609in}}%
\pgfpathlineto{\pgfqpoint{3.965037in}{1.225507in}}%
\pgfpathlineto{\pgfqpoint{3.974119in}{1.225507in}}%
\pgfpathlineto{\pgfqpoint{3.974119in}{1.219609in}}%
\pgfpathmoveto{\pgfqpoint{3.965037in}{1.225507in}}%
\pgfpathlineto{\pgfqpoint{3.965037in}{1.225507in}}%
\pgfpathlineto{\pgfqpoint{3.965037in}{1.231406in}}%
\pgfpathlineto{\pgfqpoint{3.974119in}{1.231406in}}%
\pgfpathlineto{\pgfqpoint{3.974119in}{1.225507in}}%
\pgfpathmoveto{\pgfqpoint{3.946873in}{1.231406in}}%
\pgfpathlineto{\pgfqpoint{3.946873in}{1.231406in}}%
\pgfpathlineto{\pgfqpoint{3.946873in}{1.237304in}}%
\pgfpathlineto{\pgfqpoint{3.955955in}{1.237304in}}%
\pgfpathlineto{\pgfqpoint{3.955955in}{1.231406in}}%
\pgfpathmoveto{\pgfqpoint{3.946873in}{1.237304in}}%
\pgfpathlineto{\pgfqpoint{3.946873in}{1.237304in}}%
\pgfpathlineto{\pgfqpoint{3.946873in}{1.243202in}}%
\pgfpathlineto{\pgfqpoint{3.955955in}{1.243202in}}%
\pgfpathlineto{\pgfqpoint{3.955955in}{1.237304in}}%
\pgfpathmoveto{\pgfqpoint{3.955955in}{1.231406in}}%
\pgfpathlineto{\pgfqpoint{3.955955in}{1.231406in}}%
\pgfpathlineto{\pgfqpoint{3.955955in}{1.237304in}}%
\pgfpathlineto{\pgfqpoint{3.965037in}{1.237304in}}%
\pgfpathlineto{\pgfqpoint{3.965037in}{1.231406in}}%
\pgfpathmoveto{\pgfqpoint{3.983201in}{1.207812in}}%
\pgfpathlineto{\pgfqpoint{3.983201in}{1.207812in}}%
\pgfpathlineto{\pgfqpoint{3.983201in}{1.213710in}}%
\pgfpathlineto{\pgfqpoint{3.992283in}{1.213710in}}%
\pgfpathlineto{\pgfqpoint{3.992283in}{1.207812in}}%
\pgfpathmoveto{\pgfqpoint{4.019529in}{1.172421in}}%
\pgfpathlineto{\pgfqpoint{4.019529in}{1.172421in}}%
\pgfpathlineto{\pgfqpoint{4.019529in}{1.178320in}}%
\pgfpathlineto{\pgfqpoint{4.028611in}{1.178320in}}%
\pgfpathlineto{\pgfqpoint{4.028611in}{1.172421in}}%
\pgfpathmoveto{\pgfqpoint{4.019529in}{1.178320in}}%
\pgfpathlineto{\pgfqpoint{4.019529in}{1.178320in}}%
\pgfpathlineto{\pgfqpoint{4.019529in}{1.184218in}}%
\pgfpathlineto{\pgfqpoint{4.028611in}{1.184218in}}%
\pgfpathlineto{\pgfqpoint{4.028611in}{1.178320in}}%
\pgfpathmoveto{\pgfqpoint{4.037692in}{1.160624in}}%
\pgfpathlineto{\pgfqpoint{4.037692in}{1.160624in}}%
\pgfpathlineto{\pgfqpoint{4.037692in}{1.166523in}}%
\pgfpathlineto{\pgfqpoint{4.046774in}{1.166523in}}%
\pgfpathlineto{\pgfqpoint{4.046774in}{1.160624in}}%
\pgfpathmoveto{\pgfqpoint{4.092184in}{1.001365in}}%
\pgfpathlineto{\pgfqpoint{4.092184in}{1.001365in}}%
\pgfpathlineto{\pgfqpoint{4.092184in}{1.007264in}}%
\pgfpathlineto{\pgfqpoint{4.101266in}{1.007264in}}%
\pgfpathlineto{\pgfqpoint{4.101266in}{1.001365in}}%
\pgfpathmoveto{\pgfqpoint{4.110348in}{1.013162in}}%
\pgfpathlineto{\pgfqpoint{4.110348in}{1.013162in}}%
\pgfpathlineto{\pgfqpoint{4.110348in}{1.019060in}}%
\pgfpathlineto{\pgfqpoint{4.119430in}{1.019060in}}%
\pgfpathlineto{\pgfqpoint{4.119430in}{1.013162in}}%
\pgfpathmoveto{\pgfqpoint{4.128512in}{1.030857in}}%
\pgfpathlineto{\pgfqpoint{4.128512in}{1.030857in}}%
\pgfpathlineto{\pgfqpoint{4.128512in}{1.036756in}}%
\pgfpathlineto{\pgfqpoint{4.137594in}{1.036756in}}%
\pgfpathlineto{\pgfqpoint{4.137594in}{1.030857in}}%
\pgfpathmoveto{\pgfqpoint{4.128512in}{1.036756in}}%
\pgfpathlineto{\pgfqpoint{4.128512in}{1.036756in}}%
\pgfpathlineto{\pgfqpoint{4.128512in}{1.042654in}}%
\pgfpathlineto{\pgfqpoint{4.137594in}{1.042654in}}%
\pgfpathlineto{\pgfqpoint{4.137594in}{1.036756in}}%
\pgfpathmoveto{\pgfqpoint{4.137594in}{1.036756in}}%
\pgfpathlineto{\pgfqpoint{4.137594in}{1.036756in}}%
\pgfpathlineto{\pgfqpoint{4.137594in}{1.042654in}}%
\pgfpathlineto{\pgfqpoint{4.146676in}{1.042654in}}%
\pgfpathlineto{\pgfqpoint{4.146676in}{1.036756in}}%
\pgfpathmoveto{\pgfqpoint{4.146676in}{1.042654in}}%
\pgfpathlineto{\pgfqpoint{4.146676in}{1.042654in}}%
\pgfpathlineto{\pgfqpoint{4.146676in}{1.048552in}}%
\pgfpathlineto{\pgfqpoint{4.155758in}{1.048552in}}%
\pgfpathlineto{\pgfqpoint{4.155758in}{1.042654in}}%
\pgfpathmoveto{\pgfqpoint{4.146676in}{1.048552in}}%
\pgfpathlineto{\pgfqpoint{4.146676in}{1.048552in}}%
\pgfpathlineto{\pgfqpoint{4.146676in}{1.054451in}}%
\pgfpathlineto{\pgfqpoint{4.155758in}{1.054451in}}%
\pgfpathlineto{\pgfqpoint{4.155758in}{1.048552in}}%
\pgfpathmoveto{\pgfqpoint{4.155758in}{1.048552in}}%
\pgfpathlineto{\pgfqpoint{4.155758in}{1.048552in}}%
\pgfpathlineto{\pgfqpoint{4.155758in}{1.054451in}}%
\pgfpathlineto{\pgfqpoint{4.164840in}{1.054451in}}%
\pgfpathlineto{\pgfqpoint{4.164840in}{1.048552in}}%
\pgfpathmoveto{\pgfqpoint{4.110348in}{1.089842in}}%
\pgfpathlineto{\pgfqpoint{4.110348in}{1.089842in}}%
\pgfpathlineto{\pgfqpoint{4.110348in}{1.095740in}}%
\pgfpathlineto{\pgfqpoint{4.119430in}{1.095740in}}%
\pgfpathlineto{\pgfqpoint{4.119430in}{1.089842in}}%
\pgfpathmoveto{\pgfqpoint{4.110348in}{1.095740in}}%
\pgfpathlineto{\pgfqpoint{4.110348in}{1.095740in}}%
\pgfpathlineto{\pgfqpoint{4.110348in}{1.101639in}}%
\pgfpathlineto{\pgfqpoint{4.119430in}{1.101639in}}%
\pgfpathlineto{\pgfqpoint{4.119430in}{1.095740in}}%
\pgfpathmoveto{\pgfqpoint{4.119430in}{1.089842in}}%
\pgfpathlineto{\pgfqpoint{4.119430in}{1.089842in}}%
\pgfpathlineto{\pgfqpoint{4.119430in}{1.095740in}}%
\pgfpathlineto{\pgfqpoint{4.128512in}{1.095740in}}%
\pgfpathlineto{\pgfqpoint{4.128512in}{1.089842in}}%
\pgfpathmoveto{\pgfqpoint{4.128512in}{1.078044in}}%
\pgfpathlineto{\pgfqpoint{4.128512in}{1.078044in}}%
\pgfpathlineto{\pgfqpoint{4.128512in}{1.083943in}}%
\pgfpathlineto{\pgfqpoint{4.137594in}{1.083943in}}%
\pgfpathlineto{\pgfqpoint{4.137594in}{1.078044in}}%
\pgfpathmoveto{\pgfqpoint{4.128512in}{1.083943in}}%
\pgfpathlineto{\pgfqpoint{4.128512in}{1.083943in}}%
\pgfpathlineto{\pgfqpoint{4.128512in}{1.089842in}}%
\pgfpathlineto{\pgfqpoint{4.137594in}{1.089842in}}%
\pgfpathlineto{\pgfqpoint{4.137594in}{1.083943in}}%
\pgfpathmoveto{\pgfqpoint{4.146676in}{1.066247in}}%
\pgfpathlineto{\pgfqpoint{4.146676in}{1.066247in}}%
\pgfpathlineto{\pgfqpoint{4.146676in}{1.072146in}}%
\pgfpathlineto{\pgfqpoint{4.155758in}{1.072146in}}%
\pgfpathlineto{\pgfqpoint{4.155758in}{1.066247in}}%
\pgfpathmoveto{\pgfqpoint{4.092184in}{1.113436in}}%
\pgfpathlineto{\pgfqpoint{4.092184in}{1.113436in}}%
\pgfpathlineto{\pgfqpoint{4.092184in}{1.119334in}}%
\pgfpathlineto{\pgfqpoint{4.101266in}{1.119334in}}%
\pgfpathlineto{\pgfqpoint{4.101266in}{1.113436in}}%
\pgfpathmoveto{\pgfqpoint{1.467478in}{3.390233in}}%
\pgfpathlineto{\pgfqpoint{1.467478in}{3.390233in}}%
\pgfpathlineto{\pgfqpoint{1.467478in}{3.393182in}}%
\pgfpathlineto{\pgfqpoint{1.472019in}{3.393182in}}%
\pgfpathlineto{\pgfqpoint{1.472019in}{3.390233in}}%
\pgfpathmoveto{\pgfqpoint{1.467478in}{3.393182in}}%
\pgfpathlineto{\pgfqpoint{1.467478in}{3.393182in}}%
\pgfpathlineto{\pgfqpoint{1.467478in}{3.396131in}}%
\pgfpathlineto{\pgfqpoint{1.472019in}{3.396131in}}%
\pgfpathlineto{\pgfqpoint{1.472019in}{3.393182in}}%
\pgfpathmoveto{\pgfqpoint{1.472019in}{3.390233in}}%
\pgfpathlineto{\pgfqpoint{1.472019in}{3.390233in}}%
\pgfpathlineto{\pgfqpoint{1.472019in}{3.393182in}}%
\pgfpathlineto{\pgfqpoint{1.476560in}{3.393182in}}%
\pgfpathlineto{\pgfqpoint{1.476560in}{3.390233in}}%
\pgfpathmoveto{\pgfqpoint{1.472019in}{3.393182in}}%
\pgfpathlineto{\pgfqpoint{1.472019in}{3.393182in}}%
\pgfpathlineto{\pgfqpoint{1.472019in}{3.396131in}}%
\pgfpathlineto{\pgfqpoint{1.476560in}{3.396131in}}%
\pgfpathlineto{\pgfqpoint{1.476560in}{3.393182in}}%
\pgfpathmoveto{\pgfqpoint{1.467478in}{3.396131in}}%
\pgfpathlineto{\pgfqpoint{1.467478in}{3.396131in}}%
\pgfpathlineto{\pgfqpoint{1.467478in}{3.399080in}}%
\pgfpathlineto{\pgfqpoint{1.472019in}{3.399080in}}%
\pgfpathlineto{\pgfqpoint{1.472019in}{3.396131in}}%
\pgfpathmoveto{\pgfqpoint{1.467478in}{3.399080in}}%
\pgfpathlineto{\pgfqpoint{1.467478in}{3.399080in}}%
\pgfpathlineto{\pgfqpoint{1.467478in}{3.402029in}}%
\pgfpathlineto{\pgfqpoint{1.472019in}{3.402029in}}%
\pgfpathlineto{\pgfqpoint{1.472019in}{3.399080in}}%
\pgfpathmoveto{\pgfqpoint{1.472019in}{3.396131in}}%
\pgfpathlineto{\pgfqpoint{1.472019in}{3.396131in}}%
\pgfpathlineto{\pgfqpoint{1.472019in}{3.399080in}}%
\pgfpathlineto{\pgfqpoint{1.476560in}{3.399080in}}%
\pgfpathlineto{\pgfqpoint{1.476560in}{3.396131in}}%
\pgfpathmoveto{\pgfqpoint{1.449315in}{3.407928in}}%
\pgfpathlineto{\pgfqpoint{1.449315in}{3.407928in}}%
\pgfpathlineto{\pgfqpoint{1.449315in}{3.410877in}}%
\pgfpathlineto{\pgfqpoint{1.453856in}{3.410877in}}%
\pgfpathlineto{\pgfqpoint{1.453856in}{3.407928in}}%
\pgfpathmoveto{\pgfqpoint{1.449315in}{3.410877in}}%
\pgfpathlineto{\pgfqpoint{1.449315in}{3.410877in}}%
\pgfpathlineto{\pgfqpoint{1.449315in}{3.413826in}}%
\pgfpathlineto{\pgfqpoint{1.453856in}{3.413826in}}%
\pgfpathlineto{\pgfqpoint{1.453856in}{3.410877in}}%
\pgfpathmoveto{\pgfqpoint{1.453856in}{3.407928in}}%
\pgfpathlineto{\pgfqpoint{1.453856in}{3.407928in}}%
\pgfpathlineto{\pgfqpoint{1.453856in}{3.410877in}}%
\pgfpathlineto{\pgfqpoint{1.458397in}{3.410877in}}%
\pgfpathlineto{\pgfqpoint{1.458397in}{3.407928in}}%
\pgfpathmoveto{\pgfqpoint{1.453856in}{3.410877in}}%
\pgfpathlineto{\pgfqpoint{1.453856in}{3.410877in}}%
\pgfpathlineto{\pgfqpoint{1.453856in}{3.413826in}}%
\pgfpathlineto{\pgfqpoint{1.458397in}{3.413826in}}%
\pgfpathlineto{\pgfqpoint{1.458397in}{3.410877in}}%
\pgfpathmoveto{\pgfqpoint{1.440233in}{3.413826in}}%
\pgfpathlineto{\pgfqpoint{1.440233in}{3.413826in}}%
\pgfpathlineto{\pgfqpoint{1.440233in}{3.416776in}}%
\pgfpathlineto{\pgfqpoint{1.444774in}{3.416776in}}%
\pgfpathlineto{\pgfqpoint{1.444774in}{3.413826in}}%
\pgfpathmoveto{\pgfqpoint{1.440233in}{3.416776in}}%
\pgfpathlineto{\pgfqpoint{1.440233in}{3.416776in}}%
\pgfpathlineto{\pgfqpoint{1.440233in}{3.419725in}}%
\pgfpathlineto{\pgfqpoint{1.444774in}{3.419725in}}%
\pgfpathlineto{\pgfqpoint{1.444774in}{3.416776in}}%
\pgfpathmoveto{\pgfqpoint{1.444774in}{3.413826in}}%
\pgfpathlineto{\pgfqpoint{1.444774in}{3.413826in}}%
\pgfpathlineto{\pgfqpoint{1.444774in}{3.416776in}}%
\pgfpathlineto{\pgfqpoint{1.449315in}{3.416776in}}%
\pgfpathlineto{\pgfqpoint{1.449315in}{3.413826in}}%
\pgfpathmoveto{\pgfqpoint{1.444774in}{3.416776in}}%
\pgfpathlineto{\pgfqpoint{1.444774in}{3.416776in}}%
\pgfpathlineto{\pgfqpoint{1.444774in}{3.419725in}}%
\pgfpathlineto{\pgfqpoint{1.449315in}{3.419725in}}%
\pgfpathlineto{\pgfqpoint{1.449315in}{3.416776in}}%
\pgfpathmoveto{\pgfqpoint{1.440233in}{3.419725in}}%
\pgfpathlineto{\pgfqpoint{1.440233in}{3.419725in}}%
\pgfpathlineto{\pgfqpoint{1.440233in}{3.422674in}}%
\pgfpathlineto{\pgfqpoint{1.444774in}{3.422674in}}%
\pgfpathlineto{\pgfqpoint{1.444774in}{3.419725in}}%
\pgfpathmoveto{\pgfqpoint{1.440233in}{3.422674in}}%
\pgfpathlineto{\pgfqpoint{1.440233in}{3.422674in}}%
\pgfpathlineto{\pgfqpoint{1.440233in}{3.425623in}}%
\pgfpathlineto{\pgfqpoint{1.444774in}{3.425623in}}%
\pgfpathlineto{\pgfqpoint{1.444774in}{3.422674in}}%
\pgfpathmoveto{\pgfqpoint{1.444774in}{3.419725in}}%
\pgfpathlineto{\pgfqpoint{1.444774in}{3.419725in}}%
\pgfpathlineto{\pgfqpoint{1.444774in}{3.422674in}}%
\pgfpathlineto{\pgfqpoint{1.449315in}{3.422674in}}%
\pgfpathlineto{\pgfqpoint{1.449315in}{3.419725in}}%
\pgfpathmoveto{\pgfqpoint{1.449315in}{3.413826in}}%
\pgfpathlineto{\pgfqpoint{1.449315in}{3.413826in}}%
\pgfpathlineto{\pgfqpoint{1.449315in}{3.416776in}}%
\pgfpathlineto{\pgfqpoint{1.453856in}{3.416776in}}%
\pgfpathlineto{\pgfqpoint{1.453856in}{3.413826in}}%
\pgfpathmoveto{\pgfqpoint{1.449315in}{3.416776in}}%
\pgfpathlineto{\pgfqpoint{1.449315in}{3.416776in}}%
\pgfpathlineto{\pgfqpoint{1.449315in}{3.419725in}}%
\pgfpathlineto{\pgfqpoint{1.453856in}{3.419725in}}%
\pgfpathlineto{\pgfqpoint{1.453856in}{3.416776in}}%
\pgfpathmoveto{\pgfqpoint{1.453856in}{3.413826in}}%
\pgfpathlineto{\pgfqpoint{1.453856in}{3.413826in}}%
\pgfpathlineto{\pgfqpoint{1.453856in}{3.416776in}}%
\pgfpathlineto{\pgfqpoint{1.458397in}{3.416776in}}%
\pgfpathlineto{\pgfqpoint{1.458397in}{3.413826in}}%
\pgfpathmoveto{\pgfqpoint{1.458397in}{3.402029in}}%
\pgfpathlineto{\pgfqpoint{1.458397in}{3.402029in}}%
\pgfpathlineto{\pgfqpoint{1.458397in}{3.404979in}}%
\pgfpathlineto{\pgfqpoint{1.462937in}{3.404979in}}%
\pgfpathlineto{\pgfqpoint{1.462937in}{3.402029in}}%
\pgfpathmoveto{\pgfqpoint{1.458397in}{3.404979in}}%
\pgfpathlineto{\pgfqpoint{1.458397in}{3.404979in}}%
\pgfpathlineto{\pgfqpoint{1.458397in}{3.407928in}}%
\pgfpathlineto{\pgfqpoint{1.462937in}{3.407928in}}%
\pgfpathlineto{\pgfqpoint{1.462937in}{3.404979in}}%
\pgfpathmoveto{\pgfqpoint{1.462937in}{3.402029in}}%
\pgfpathlineto{\pgfqpoint{1.462937in}{3.402029in}}%
\pgfpathlineto{\pgfqpoint{1.462937in}{3.404979in}}%
\pgfpathlineto{\pgfqpoint{1.467478in}{3.404979in}}%
\pgfpathlineto{\pgfqpoint{1.467478in}{3.402029in}}%
\pgfpathmoveto{\pgfqpoint{1.462937in}{3.404979in}}%
\pgfpathlineto{\pgfqpoint{1.462937in}{3.404979in}}%
\pgfpathlineto{\pgfqpoint{1.462937in}{3.407928in}}%
\pgfpathlineto{\pgfqpoint{1.467478in}{3.407928in}}%
\pgfpathlineto{\pgfqpoint{1.467478in}{3.404979in}}%
\pgfpathmoveto{\pgfqpoint{1.458397in}{3.407928in}}%
\pgfpathlineto{\pgfqpoint{1.458397in}{3.407928in}}%
\pgfpathlineto{\pgfqpoint{1.458397in}{3.410877in}}%
\pgfpathlineto{\pgfqpoint{1.462937in}{3.410877in}}%
\pgfpathlineto{\pgfqpoint{1.462937in}{3.407928in}}%
\pgfpathmoveto{\pgfqpoint{1.467478in}{3.402029in}}%
\pgfpathlineto{\pgfqpoint{1.467478in}{3.402029in}}%
\pgfpathlineto{\pgfqpoint{1.467478in}{3.404979in}}%
\pgfpathlineto{\pgfqpoint{1.472019in}{3.404979in}}%
\pgfpathlineto{\pgfqpoint{1.472019in}{3.402029in}}%
\pgfpathmoveto{\pgfqpoint{1.394824in}{3.455115in}}%
\pgfpathlineto{\pgfqpoint{1.394824in}{3.455115in}}%
\pgfpathlineto{\pgfqpoint{1.394824in}{3.458064in}}%
\pgfpathlineto{\pgfqpoint{1.399365in}{3.458064in}}%
\pgfpathlineto{\pgfqpoint{1.399365in}{3.455115in}}%
\pgfpathmoveto{\pgfqpoint{1.394824in}{3.458064in}}%
\pgfpathlineto{\pgfqpoint{1.394824in}{3.458064in}}%
\pgfpathlineto{\pgfqpoint{1.394824in}{3.461014in}}%
\pgfpathlineto{\pgfqpoint{1.399365in}{3.461014in}}%
\pgfpathlineto{\pgfqpoint{1.399365in}{3.458064in}}%
\pgfpathmoveto{\pgfqpoint{1.399365in}{3.455115in}}%
\pgfpathlineto{\pgfqpoint{1.399365in}{3.455115in}}%
\pgfpathlineto{\pgfqpoint{1.399365in}{3.458064in}}%
\pgfpathlineto{\pgfqpoint{1.403906in}{3.458064in}}%
\pgfpathlineto{\pgfqpoint{1.403906in}{3.455115in}}%
\pgfpathmoveto{\pgfqpoint{1.399365in}{3.458064in}}%
\pgfpathlineto{\pgfqpoint{1.399365in}{3.458064in}}%
\pgfpathlineto{\pgfqpoint{1.399365in}{3.461014in}}%
\pgfpathlineto{\pgfqpoint{1.403906in}{3.461014in}}%
\pgfpathlineto{\pgfqpoint{1.403906in}{3.458064in}}%
\pgfpathmoveto{\pgfqpoint{1.385743in}{3.461014in}}%
\pgfpathlineto{\pgfqpoint{1.385743in}{3.461014in}}%
\pgfpathlineto{\pgfqpoint{1.385743in}{3.463963in}}%
\pgfpathlineto{\pgfqpoint{1.390284in}{3.463963in}}%
\pgfpathlineto{\pgfqpoint{1.390284in}{3.461014in}}%
\pgfpathmoveto{\pgfqpoint{1.385743in}{3.463963in}}%
\pgfpathlineto{\pgfqpoint{1.385743in}{3.463963in}}%
\pgfpathlineto{\pgfqpoint{1.385743in}{3.466912in}}%
\pgfpathlineto{\pgfqpoint{1.390284in}{3.466912in}}%
\pgfpathlineto{\pgfqpoint{1.390284in}{3.463963in}}%
\pgfpathmoveto{\pgfqpoint{1.390284in}{3.461014in}}%
\pgfpathlineto{\pgfqpoint{1.390284in}{3.461014in}}%
\pgfpathlineto{\pgfqpoint{1.390284in}{3.463963in}}%
\pgfpathlineto{\pgfqpoint{1.394824in}{3.463963in}}%
\pgfpathlineto{\pgfqpoint{1.394824in}{3.461014in}}%
\pgfpathmoveto{\pgfqpoint{1.390284in}{3.463963in}}%
\pgfpathlineto{\pgfqpoint{1.390284in}{3.463963in}}%
\pgfpathlineto{\pgfqpoint{1.390284in}{3.466912in}}%
\pgfpathlineto{\pgfqpoint{1.394824in}{3.466912in}}%
\pgfpathlineto{\pgfqpoint{1.394824in}{3.463963in}}%
\pgfpathmoveto{\pgfqpoint{1.385743in}{3.466912in}}%
\pgfpathlineto{\pgfqpoint{1.385743in}{3.466912in}}%
\pgfpathlineto{\pgfqpoint{1.385743in}{3.469861in}}%
\pgfpathlineto{\pgfqpoint{1.390284in}{3.469861in}}%
\pgfpathlineto{\pgfqpoint{1.390284in}{3.466912in}}%
\pgfpathmoveto{\pgfqpoint{1.385743in}{3.469861in}}%
\pgfpathlineto{\pgfqpoint{1.385743in}{3.469861in}}%
\pgfpathlineto{\pgfqpoint{1.385743in}{3.472810in}}%
\pgfpathlineto{\pgfqpoint{1.390284in}{3.472810in}}%
\pgfpathlineto{\pgfqpoint{1.390284in}{3.469861in}}%
\pgfpathmoveto{\pgfqpoint{1.390284in}{3.466912in}}%
\pgfpathlineto{\pgfqpoint{1.390284in}{3.466912in}}%
\pgfpathlineto{\pgfqpoint{1.390284in}{3.469861in}}%
\pgfpathlineto{\pgfqpoint{1.394824in}{3.469861in}}%
\pgfpathlineto{\pgfqpoint{1.394824in}{3.466912in}}%
\pgfpathmoveto{\pgfqpoint{1.394824in}{3.461014in}}%
\pgfpathlineto{\pgfqpoint{1.394824in}{3.461014in}}%
\pgfpathlineto{\pgfqpoint{1.394824in}{3.463963in}}%
\pgfpathlineto{\pgfqpoint{1.399365in}{3.463963in}}%
\pgfpathlineto{\pgfqpoint{1.399365in}{3.461014in}}%
\pgfpathmoveto{\pgfqpoint{1.394824in}{3.463963in}}%
\pgfpathlineto{\pgfqpoint{1.394824in}{3.463963in}}%
\pgfpathlineto{\pgfqpoint{1.394824in}{3.466912in}}%
\pgfpathlineto{\pgfqpoint{1.399365in}{3.466912in}}%
\pgfpathlineto{\pgfqpoint{1.399365in}{3.463963in}}%
\pgfpathmoveto{\pgfqpoint{1.399365in}{3.461014in}}%
\pgfpathlineto{\pgfqpoint{1.399365in}{3.461014in}}%
\pgfpathlineto{\pgfqpoint{1.399365in}{3.463963in}}%
\pgfpathlineto{\pgfqpoint{1.403906in}{3.463963in}}%
\pgfpathlineto{\pgfqpoint{1.403906in}{3.461014in}}%
\pgfpathmoveto{\pgfqpoint{1.358497in}{3.484607in}}%
\pgfpathlineto{\pgfqpoint{1.358497in}{3.484607in}}%
\pgfpathlineto{\pgfqpoint{1.358497in}{3.487557in}}%
\pgfpathlineto{\pgfqpoint{1.363038in}{3.487557in}}%
\pgfpathlineto{\pgfqpoint{1.363038in}{3.484607in}}%
\pgfpathmoveto{\pgfqpoint{1.358497in}{3.487557in}}%
\pgfpathlineto{\pgfqpoint{1.358497in}{3.487557in}}%
\pgfpathlineto{\pgfqpoint{1.358497in}{3.490506in}}%
\pgfpathlineto{\pgfqpoint{1.363038in}{3.490506in}}%
\pgfpathlineto{\pgfqpoint{1.363038in}{3.487557in}}%
\pgfpathmoveto{\pgfqpoint{1.363038in}{3.484607in}}%
\pgfpathlineto{\pgfqpoint{1.363038in}{3.484607in}}%
\pgfpathlineto{\pgfqpoint{1.363038in}{3.487557in}}%
\pgfpathlineto{\pgfqpoint{1.367579in}{3.487557in}}%
\pgfpathlineto{\pgfqpoint{1.367579in}{3.484607in}}%
\pgfpathmoveto{\pgfqpoint{1.363038in}{3.487557in}}%
\pgfpathlineto{\pgfqpoint{1.363038in}{3.487557in}}%
\pgfpathlineto{\pgfqpoint{1.363038in}{3.490506in}}%
\pgfpathlineto{\pgfqpoint{1.367579in}{3.490506in}}%
\pgfpathlineto{\pgfqpoint{1.367579in}{3.487557in}}%
\pgfpathmoveto{\pgfqpoint{1.358497in}{3.490506in}}%
\pgfpathlineto{\pgfqpoint{1.358497in}{3.490506in}}%
\pgfpathlineto{\pgfqpoint{1.358497in}{3.493455in}}%
\pgfpathlineto{\pgfqpoint{1.363038in}{3.493455in}}%
\pgfpathlineto{\pgfqpoint{1.363038in}{3.490506in}}%
\pgfpathmoveto{\pgfqpoint{1.358497in}{3.493455in}}%
\pgfpathlineto{\pgfqpoint{1.358497in}{3.493455in}}%
\pgfpathlineto{\pgfqpoint{1.358497in}{3.496404in}}%
\pgfpathlineto{\pgfqpoint{1.363038in}{3.496404in}}%
\pgfpathlineto{\pgfqpoint{1.363038in}{3.493455in}}%
\pgfpathmoveto{\pgfqpoint{1.363038in}{3.490506in}}%
\pgfpathlineto{\pgfqpoint{1.363038in}{3.490506in}}%
\pgfpathlineto{\pgfqpoint{1.363038in}{3.493455in}}%
\pgfpathlineto{\pgfqpoint{1.367579in}{3.493455in}}%
\pgfpathlineto{\pgfqpoint{1.367579in}{3.490506in}}%
\pgfpathmoveto{\pgfqpoint{1.340334in}{3.502303in}}%
\pgfpathlineto{\pgfqpoint{1.340334in}{3.502303in}}%
\pgfpathlineto{\pgfqpoint{1.340334in}{3.505252in}}%
\pgfpathlineto{\pgfqpoint{1.344875in}{3.505252in}}%
\pgfpathlineto{\pgfqpoint{1.344875in}{3.502303in}}%
\pgfpathmoveto{\pgfqpoint{1.340334in}{3.505252in}}%
\pgfpathlineto{\pgfqpoint{1.340334in}{3.505252in}}%
\pgfpathlineto{\pgfqpoint{1.340334in}{3.508201in}}%
\pgfpathlineto{\pgfqpoint{1.344875in}{3.508201in}}%
\pgfpathlineto{\pgfqpoint{1.344875in}{3.505252in}}%
\pgfpathmoveto{\pgfqpoint{1.344875in}{3.502303in}}%
\pgfpathlineto{\pgfqpoint{1.344875in}{3.502303in}}%
\pgfpathlineto{\pgfqpoint{1.344875in}{3.505252in}}%
\pgfpathlineto{\pgfqpoint{1.349416in}{3.505252in}}%
\pgfpathlineto{\pgfqpoint{1.349416in}{3.502303in}}%
\pgfpathmoveto{\pgfqpoint{1.344875in}{3.505252in}}%
\pgfpathlineto{\pgfqpoint{1.344875in}{3.505252in}}%
\pgfpathlineto{\pgfqpoint{1.344875in}{3.508201in}}%
\pgfpathlineto{\pgfqpoint{1.349416in}{3.508201in}}%
\pgfpathlineto{\pgfqpoint{1.349416in}{3.505252in}}%
\pgfpathmoveto{\pgfqpoint{1.331252in}{3.508201in}}%
\pgfpathlineto{\pgfqpoint{1.331252in}{3.508201in}}%
\pgfpathlineto{\pgfqpoint{1.331252in}{3.511150in}}%
\pgfpathlineto{\pgfqpoint{1.335793in}{3.511150in}}%
\pgfpathlineto{\pgfqpoint{1.335793in}{3.508201in}}%
\pgfpathmoveto{\pgfqpoint{1.331252in}{3.511150in}}%
\pgfpathlineto{\pgfqpoint{1.331252in}{3.511150in}}%
\pgfpathlineto{\pgfqpoint{1.331252in}{3.514099in}}%
\pgfpathlineto{\pgfqpoint{1.335793in}{3.514099in}}%
\pgfpathlineto{\pgfqpoint{1.335793in}{3.511150in}}%
\pgfpathmoveto{\pgfqpoint{1.335793in}{3.508201in}}%
\pgfpathlineto{\pgfqpoint{1.335793in}{3.508201in}}%
\pgfpathlineto{\pgfqpoint{1.335793in}{3.511150in}}%
\pgfpathlineto{\pgfqpoint{1.340334in}{3.511150in}}%
\pgfpathlineto{\pgfqpoint{1.340334in}{3.508201in}}%
\pgfpathmoveto{\pgfqpoint{1.335793in}{3.511150in}}%
\pgfpathlineto{\pgfqpoint{1.335793in}{3.511150in}}%
\pgfpathlineto{\pgfqpoint{1.335793in}{3.514099in}}%
\pgfpathlineto{\pgfqpoint{1.340334in}{3.514099in}}%
\pgfpathlineto{\pgfqpoint{1.340334in}{3.511150in}}%
\pgfpathmoveto{\pgfqpoint{1.331252in}{3.514099in}}%
\pgfpathlineto{\pgfqpoint{1.331252in}{3.514099in}}%
\pgfpathlineto{\pgfqpoint{1.331252in}{3.517049in}}%
\pgfpathlineto{\pgfqpoint{1.335793in}{3.517049in}}%
\pgfpathlineto{\pgfqpoint{1.335793in}{3.514099in}}%
\pgfpathmoveto{\pgfqpoint{1.331252in}{3.517049in}}%
\pgfpathlineto{\pgfqpoint{1.331252in}{3.517049in}}%
\pgfpathlineto{\pgfqpoint{1.331252in}{3.519998in}}%
\pgfpathlineto{\pgfqpoint{1.335793in}{3.519998in}}%
\pgfpathlineto{\pgfqpoint{1.335793in}{3.517049in}}%
\pgfpathmoveto{\pgfqpoint{1.335793in}{3.514099in}}%
\pgfpathlineto{\pgfqpoint{1.335793in}{3.514099in}}%
\pgfpathlineto{\pgfqpoint{1.335793in}{3.517049in}}%
\pgfpathlineto{\pgfqpoint{1.340334in}{3.517049in}}%
\pgfpathlineto{\pgfqpoint{1.340334in}{3.514099in}}%
\pgfpathmoveto{\pgfqpoint{1.340334in}{3.508201in}}%
\pgfpathlineto{\pgfqpoint{1.340334in}{3.508201in}}%
\pgfpathlineto{\pgfqpoint{1.340334in}{3.511150in}}%
\pgfpathlineto{\pgfqpoint{1.344875in}{3.511150in}}%
\pgfpathlineto{\pgfqpoint{1.344875in}{3.508201in}}%
\pgfpathmoveto{\pgfqpoint{1.340334in}{3.511150in}}%
\pgfpathlineto{\pgfqpoint{1.340334in}{3.511150in}}%
\pgfpathlineto{\pgfqpoint{1.340334in}{3.514099in}}%
\pgfpathlineto{\pgfqpoint{1.344875in}{3.514099in}}%
\pgfpathlineto{\pgfqpoint{1.344875in}{3.511150in}}%
\pgfpathmoveto{\pgfqpoint{1.344875in}{3.508201in}}%
\pgfpathlineto{\pgfqpoint{1.344875in}{3.508201in}}%
\pgfpathlineto{\pgfqpoint{1.344875in}{3.511150in}}%
\pgfpathlineto{\pgfqpoint{1.349416in}{3.511150in}}%
\pgfpathlineto{\pgfqpoint{1.349416in}{3.508201in}}%
\pgfpathmoveto{\pgfqpoint{1.349416in}{3.496404in}}%
\pgfpathlineto{\pgfqpoint{1.349416in}{3.496404in}}%
\pgfpathlineto{\pgfqpoint{1.349416in}{3.499353in}}%
\pgfpathlineto{\pgfqpoint{1.353957in}{3.499353in}}%
\pgfpathlineto{\pgfqpoint{1.353957in}{3.496404in}}%
\pgfpathmoveto{\pgfqpoint{1.349416in}{3.499353in}}%
\pgfpathlineto{\pgfqpoint{1.349416in}{3.499353in}}%
\pgfpathlineto{\pgfqpoint{1.349416in}{3.502303in}}%
\pgfpathlineto{\pgfqpoint{1.353957in}{3.502303in}}%
\pgfpathlineto{\pgfqpoint{1.353957in}{3.499353in}}%
\pgfpathmoveto{\pgfqpoint{1.353957in}{3.496404in}}%
\pgfpathlineto{\pgfqpoint{1.353957in}{3.496404in}}%
\pgfpathlineto{\pgfqpoint{1.353957in}{3.499353in}}%
\pgfpathlineto{\pgfqpoint{1.358497in}{3.499353in}}%
\pgfpathlineto{\pgfqpoint{1.358497in}{3.496404in}}%
\pgfpathmoveto{\pgfqpoint{1.353957in}{3.499353in}}%
\pgfpathlineto{\pgfqpoint{1.353957in}{3.499353in}}%
\pgfpathlineto{\pgfqpoint{1.353957in}{3.502303in}}%
\pgfpathlineto{\pgfqpoint{1.358497in}{3.502303in}}%
\pgfpathlineto{\pgfqpoint{1.358497in}{3.499353in}}%
\pgfpathmoveto{\pgfqpoint{1.349416in}{3.502303in}}%
\pgfpathlineto{\pgfqpoint{1.349416in}{3.502303in}}%
\pgfpathlineto{\pgfqpoint{1.349416in}{3.505252in}}%
\pgfpathlineto{\pgfqpoint{1.353957in}{3.505252in}}%
\pgfpathlineto{\pgfqpoint{1.353957in}{3.502303in}}%
\pgfpathmoveto{\pgfqpoint{1.358497in}{3.496404in}}%
\pgfpathlineto{\pgfqpoint{1.358497in}{3.496404in}}%
\pgfpathlineto{\pgfqpoint{1.358497in}{3.499353in}}%
\pgfpathlineto{\pgfqpoint{1.363038in}{3.499353in}}%
\pgfpathlineto{\pgfqpoint{1.363038in}{3.496404in}}%
\pgfpathmoveto{\pgfqpoint{1.367579in}{3.478709in}}%
\pgfpathlineto{\pgfqpoint{1.367579in}{3.478709in}}%
\pgfpathlineto{\pgfqpoint{1.367579in}{3.481658in}}%
\pgfpathlineto{\pgfqpoint{1.372120in}{3.481658in}}%
\pgfpathlineto{\pgfqpoint{1.372120in}{3.478709in}}%
\pgfpathmoveto{\pgfqpoint{1.367579in}{3.481658in}}%
\pgfpathlineto{\pgfqpoint{1.367579in}{3.481658in}}%
\pgfpathlineto{\pgfqpoint{1.367579in}{3.484607in}}%
\pgfpathlineto{\pgfqpoint{1.372120in}{3.484607in}}%
\pgfpathlineto{\pgfqpoint{1.372120in}{3.481658in}}%
\pgfpathmoveto{\pgfqpoint{1.372120in}{3.478709in}}%
\pgfpathlineto{\pgfqpoint{1.372120in}{3.478709in}}%
\pgfpathlineto{\pgfqpoint{1.372120in}{3.481658in}}%
\pgfpathlineto{\pgfqpoint{1.376661in}{3.481658in}}%
\pgfpathlineto{\pgfqpoint{1.376661in}{3.478709in}}%
\pgfpathmoveto{\pgfqpoint{1.372120in}{3.481658in}}%
\pgfpathlineto{\pgfqpoint{1.372120in}{3.481658in}}%
\pgfpathlineto{\pgfqpoint{1.372120in}{3.484607in}}%
\pgfpathlineto{\pgfqpoint{1.376661in}{3.484607in}}%
\pgfpathlineto{\pgfqpoint{1.376661in}{3.481658in}}%
\pgfpathmoveto{\pgfqpoint{1.376661in}{3.472810in}}%
\pgfpathlineto{\pgfqpoint{1.376661in}{3.472810in}}%
\pgfpathlineto{\pgfqpoint{1.376661in}{3.475760in}}%
\pgfpathlineto{\pgfqpoint{1.381202in}{3.475760in}}%
\pgfpathlineto{\pgfqpoint{1.381202in}{3.472810in}}%
\pgfpathmoveto{\pgfqpoint{1.376661in}{3.475760in}}%
\pgfpathlineto{\pgfqpoint{1.376661in}{3.475760in}}%
\pgfpathlineto{\pgfqpoint{1.376661in}{3.478709in}}%
\pgfpathlineto{\pgfqpoint{1.381202in}{3.478709in}}%
\pgfpathlineto{\pgfqpoint{1.381202in}{3.475760in}}%
\pgfpathmoveto{\pgfqpoint{1.381202in}{3.472810in}}%
\pgfpathlineto{\pgfqpoint{1.381202in}{3.472810in}}%
\pgfpathlineto{\pgfqpoint{1.381202in}{3.475760in}}%
\pgfpathlineto{\pgfqpoint{1.385743in}{3.475760in}}%
\pgfpathlineto{\pgfqpoint{1.385743in}{3.472810in}}%
\pgfpathmoveto{\pgfqpoint{1.381202in}{3.475760in}}%
\pgfpathlineto{\pgfqpoint{1.381202in}{3.475760in}}%
\pgfpathlineto{\pgfqpoint{1.381202in}{3.478709in}}%
\pgfpathlineto{\pgfqpoint{1.385743in}{3.478709in}}%
\pgfpathlineto{\pgfqpoint{1.385743in}{3.475760in}}%
\pgfpathmoveto{\pgfqpoint{1.376661in}{3.478709in}}%
\pgfpathlineto{\pgfqpoint{1.376661in}{3.478709in}}%
\pgfpathlineto{\pgfqpoint{1.376661in}{3.481658in}}%
\pgfpathlineto{\pgfqpoint{1.381202in}{3.481658in}}%
\pgfpathlineto{\pgfqpoint{1.381202in}{3.478709in}}%
\pgfpathmoveto{\pgfqpoint{1.367579in}{3.484607in}}%
\pgfpathlineto{\pgfqpoint{1.367579in}{3.484607in}}%
\pgfpathlineto{\pgfqpoint{1.367579in}{3.487557in}}%
\pgfpathlineto{\pgfqpoint{1.372120in}{3.487557in}}%
\pgfpathlineto{\pgfqpoint{1.372120in}{3.484607in}}%
\pgfpathmoveto{\pgfqpoint{1.367579in}{3.487557in}}%
\pgfpathlineto{\pgfqpoint{1.367579in}{3.487557in}}%
\pgfpathlineto{\pgfqpoint{1.367579in}{3.490506in}}%
\pgfpathlineto{\pgfqpoint{1.372120in}{3.490506in}}%
\pgfpathlineto{\pgfqpoint{1.372120in}{3.487557in}}%
\pgfpathmoveto{\pgfqpoint{1.372120in}{3.484607in}}%
\pgfpathlineto{\pgfqpoint{1.372120in}{3.484607in}}%
\pgfpathlineto{\pgfqpoint{1.372120in}{3.487557in}}%
\pgfpathlineto{\pgfqpoint{1.376661in}{3.487557in}}%
\pgfpathlineto{\pgfqpoint{1.376661in}{3.484607in}}%
\pgfpathmoveto{\pgfqpoint{1.385743in}{3.472810in}}%
\pgfpathlineto{\pgfqpoint{1.385743in}{3.472810in}}%
\pgfpathlineto{\pgfqpoint{1.385743in}{3.475760in}}%
\pgfpathlineto{\pgfqpoint{1.390284in}{3.475760in}}%
\pgfpathlineto{\pgfqpoint{1.390284in}{3.472810in}}%
\pgfpathmoveto{\pgfqpoint{1.412988in}{3.437420in}}%
\pgfpathlineto{\pgfqpoint{1.412988in}{3.437420in}}%
\pgfpathlineto{\pgfqpoint{1.412988in}{3.440369in}}%
\pgfpathlineto{\pgfqpoint{1.417529in}{3.440369in}}%
\pgfpathlineto{\pgfqpoint{1.417529in}{3.437420in}}%
\pgfpathmoveto{\pgfqpoint{1.412988in}{3.440369in}}%
\pgfpathlineto{\pgfqpoint{1.412988in}{3.440369in}}%
\pgfpathlineto{\pgfqpoint{1.412988in}{3.443318in}}%
\pgfpathlineto{\pgfqpoint{1.417529in}{3.443318in}}%
\pgfpathlineto{\pgfqpoint{1.417529in}{3.440369in}}%
\pgfpathmoveto{\pgfqpoint{1.417529in}{3.437420in}}%
\pgfpathlineto{\pgfqpoint{1.417529in}{3.437420in}}%
\pgfpathlineto{\pgfqpoint{1.417529in}{3.440369in}}%
\pgfpathlineto{\pgfqpoint{1.422070in}{3.440369in}}%
\pgfpathlineto{\pgfqpoint{1.422070in}{3.437420in}}%
\pgfpathmoveto{\pgfqpoint{1.417529in}{3.440369in}}%
\pgfpathlineto{\pgfqpoint{1.417529in}{3.440369in}}%
\pgfpathlineto{\pgfqpoint{1.417529in}{3.443318in}}%
\pgfpathlineto{\pgfqpoint{1.422070in}{3.443318in}}%
\pgfpathlineto{\pgfqpoint{1.422070in}{3.440369in}}%
\pgfpathmoveto{\pgfqpoint{1.412988in}{3.443318in}}%
\pgfpathlineto{\pgfqpoint{1.412988in}{3.443318in}}%
\pgfpathlineto{\pgfqpoint{1.412988in}{3.446268in}}%
\pgfpathlineto{\pgfqpoint{1.417529in}{3.446268in}}%
\pgfpathlineto{\pgfqpoint{1.417529in}{3.443318in}}%
\pgfpathmoveto{\pgfqpoint{1.412988in}{3.446268in}}%
\pgfpathlineto{\pgfqpoint{1.412988in}{3.446268in}}%
\pgfpathlineto{\pgfqpoint{1.412988in}{3.449217in}}%
\pgfpathlineto{\pgfqpoint{1.417529in}{3.449217in}}%
\pgfpathlineto{\pgfqpoint{1.417529in}{3.446268in}}%
\pgfpathmoveto{\pgfqpoint{1.417529in}{3.443318in}}%
\pgfpathlineto{\pgfqpoint{1.417529in}{3.443318in}}%
\pgfpathlineto{\pgfqpoint{1.417529in}{3.446268in}}%
\pgfpathlineto{\pgfqpoint{1.422070in}{3.446268in}}%
\pgfpathlineto{\pgfqpoint{1.422070in}{3.443318in}}%
\pgfpathmoveto{\pgfqpoint{1.422070in}{3.431522in}}%
\pgfpathlineto{\pgfqpoint{1.422070in}{3.431522in}}%
\pgfpathlineto{\pgfqpoint{1.422070in}{3.434471in}}%
\pgfpathlineto{\pgfqpoint{1.426610in}{3.434471in}}%
\pgfpathlineto{\pgfqpoint{1.426610in}{3.431522in}}%
\pgfpathmoveto{\pgfqpoint{1.422070in}{3.434471in}}%
\pgfpathlineto{\pgfqpoint{1.422070in}{3.434471in}}%
\pgfpathlineto{\pgfqpoint{1.422070in}{3.437420in}}%
\pgfpathlineto{\pgfqpoint{1.426610in}{3.437420in}}%
\pgfpathlineto{\pgfqpoint{1.426610in}{3.434471in}}%
\pgfpathmoveto{\pgfqpoint{1.426610in}{3.431522in}}%
\pgfpathlineto{\pgfqpoint{1.426610in}{3.431522in}}%
\pgfpathlineto{\pgfqpoint{1.426610in}{3.434471in}}%
\pgfpathlineto{\pgfqpoint{1.431151in}{3.434471in}}%
\pgfpathlineto{\pgfqpoint{1.431151in}{3.431522in}}%
\pgfpathmoveto{\pgfqpoint{1.426610in}{3.434471in}}%
\pgfpathlineto{\pgfqpoint{1.426610in}{3.434471in}}%
\pgfpathlineto{\pgfqpoint{1.426610in}{3.437420in}}%
\pgfpathlineto{\pgfqpoint{1.431151in}{3.437420in}}%
\pgfpathlineto{\pgfqpoint{1.431151in}{3.434471in}}%
\pgfpathmoveto{\pgfqpoint{1.431151in}{3.425623in}}%
\pgfpathlineto{\pgfqpoint{1.431151in}{3.425623in}}%
\pgfpathlineto{\pgfqpoint{1.431151in}{3.428572in}}%
\pgfpathlineto{\pgfqpoint{1.435692in}{3.428572in}}%
\pgfpathlineto{\pgfqpoint{1.435692in}{3.425623in}}%
\pgfpathmoveto{\pgfqpoint{1.431151in}{3.428572in}}%
\pgfpathlineto{\pgfqpoint{1.431151in}{3.428572in}}%
\pgfpathlineto{\pgfqpoint{1.431151in}{3.431522in}}%
\pgfpathlineto{\pgfqpoint{1.435692in}{3.431522in}}%
\pgfpathlineto{\pgfqpoint{1.435692in}{3.428572in}}%
\pgfpathmoveto{\pgfqpoint{1.435692in}{3.425623in}}%
\pgfpathlineto{\pgfqpoint{1.435692in}{3.425623in}}%
\pgfpathlineto{\pgfqpoint{1.435692in}{3.428572in}}%
\pgfpathlineto{\pgfqpoint{1.440233in}{3.428572in}}%
\pgfpathlineto{\pgfqpoint{1.440233in}{3.425623in}}%
\pgfpathmoveto{\pgfqpoint{1.435692in}{3.428572in}}%
\pgfpathlineto{\pgfqpoint{1.435692in}{3.428572in}}%
\pgfpathlineto{\pgfqpoint{1.435692in}{3.431522in}}%
\pgfpathlineto{\pgfqpoint{1.440233in}{3.431522in}}%
\pgfpathlineto{\pgfqpoint{1.440233in}{3.428572in}}%
\pgfpathmoveto{\pgfqpoint{1.431151in}{3.431522in}}%
\pgfpathlineto{\pgfqpoint{1.431151in}{3.431522in}}%
\pgfpathlineto{\pgfqpoint{1.431151in}{3.434471in}}%
\pgfpathlineto{\pgfqpoint{1.435692in}{3.434471in}}%
\pgfpathlineto{\pgfqpoint{1.435692in}{3.431522in}}%
\pgfpathmoveto{\pgfqpoint{1.422070in}{3.437420in}}%
\pgfpathlineto{\pgfqpoint{1.422070in}{3.437420in}}%
\pgfpathlineto{\pgfqpoint{1.422070in}{3.440369in}}%
\pgfpathlineto{\pgfqpoint{1.426610in}{3.440369in}}%
\pgfpathlineto{\pgfqpoint{1.426610in}{3.437420in}}%
\pgfpathmoveto{\pgfqpoint{1.422070in}{3.440369in}}%
\pgfpathlineto{\pgfqpoint{1.422070in}{3.440369in}}%
\pgfpathlineto{\pgfqpoint{1.422070in}{3.443318in}}%
\pgfpathlineto{\pgfqpoint{1.426610in}{3.443318in}}%
\pgfpathlineto{\pgfqpoint{1.426610in}{3.440369in}}%
\pgfpathmoveto{\pgfqpoint{1.426610in}{3.437420in}}%
\pgfpathlineto{\pgfqpoint{1.426610in}{3.437420in}}%
\pgfpathlineto{\pgfqpoint{1.426610in}{3.440369in}}%
\pgfpathlineto{\pgfqpoint{1.431151in}{3.440369in}}%
\pgfpathlineto{\pgfqpoint{1.431151in}{3.437420in}}%
\pgfpathmoveto{\pgfqpoint{1.403906in}{3.449217in}}%
\pgfpathlineto{\pgfqpoint{1.403906in}{3.449217in}}%
\pgfpathlineto{\pgfqpoint{1.403906in}{3.452166in}}%
\pgfpathlineto{\pgfqpoint{1.408447in}{3.452166in}}%
\pgfpathlineto{\pgfqpoint{1.408447in}{3.449217in}}%
\pgfpathmoveto{\pgfqpoint{1.403906in}{3.452166in}}%
\pgfpathlineto{\pgfqpoint{1.403906in}{3.452166in}}%
\pgfpathlineto{\pgfqpoint{1.403906in}{3.455115in}}%
\pgfpathlineto{\pgfqpoint{1.408447in}{3.455115in}}%
\pgfpathlineto{\pgfqpoint{1.408447in}{3.452166in}}%
\pgfpathmoveto{\pgfqpoint{1.408447in}{3.449217in}}%
\pgfpathlineto{\pgfqpoint{1.408447in}{3.449217in}}%
\pgfpathlineto{\pgfqpoint{1.408447in}{3.452166in}}%
\pgfpathlineto{\pgfqpoint{1.412988in}{3.452166in}}%
\pgfpathlineto{\pgfqpoint{1.412988in}{3.449217in}}%
\pgfpathmoveto{\pgfqpoint{1.408447in}{3.452166in}}%
\pgfpathlineto{\pgfqpoint{1.408447in}{3.452166in}}%
\pgfpathlineto{\pgfqpoint{1.408447in}{3.455115in}}%
\pgfpathlineto{\pgfqpoint{1.412988in}{3.455115in}}%
\pgfpathlineto{\pgfqpoint{1.412988in}{3.452166in}}%
\pgfpathmoveto{\pgfqpoint{1.403906in}{3.455115in}}%
\pgfpathlineto{\pgfqpoint{1.403906in}{3.455115in}}%
\pgfpathlineto{\pgfqpoint{1.403906in}{3.458064in}}%
\pgfpathlineto{\pgfqpoint{1.408447in}{3.458064in}}%
\pgfpathlineto{\pgfqpoint{1.408447in}{3.455115in}}%
\pgfpathmoveto{\pgfqpoint{1.412988in}{3.449217in}}%
\pgfpathlineto{\pgfqpoint{1.412988in}{3.449217in}}%
\pgfpathlineto{\pgfqpoint{1.412988in}{3.452166in}}%
\pgfpathlineto{\pgfqpoint{1.417529in}{3.452166in}}%
\pgfpathlineto{\pgfqpoint{1.417529in}{3.449217in}}%
\pgfpathmoveto{\pgfqpoint{1.440233in}{3.425623in}}%
\pgfpathlineto{\pgfqpoint{1.440233in}{3.425623in}}%
\pgfpathlineto{\pgfqpoint{1.440233in}{3.428572in}}%
\pgfpathlineto{\pgfqpoint{1.444774in}{3.428572in}}%
\pgfpathlineto{\pgfqpoint{1.444774in}{3.425623in}}%
\pgfpathmoveto{\pgfqpoint{1.594632in}{3.278164in}}%
\pgfpathlineto{\pgfqpoint{1.594632in}{3.278164in}}%
\pgfpathlineto{\pgfqpoint{1.594632in}{3.281113in}}%
\pgfpathlineto{\pgfqpoint{1.599173in}{3.281113in}}%
\pgfpathlineto{\pgfqpoint{1.599173in}{3.278164in}}%
\pgfpathmoveto{\pgfqpoint{1.594632in}{3.281113in}}%
\pgfpathlineto{\pgfqpoint{1.594632in}{3.281113in}}%
\pgfpathlineto{\pgfqpoint{1.594632in}{3.284062in}}%
\pgfpathlineto{\pgfqpoint{1.599173in}{3.284062in}}%
\pgfpathlineto{\pgfqpoint{1.599173in}{3.281113in}}%
\pgfpathmoveto{\pgfqpoint{1.599173in}{3.278164in}}%
\pgfpathlineto{\pgfqpoint{1.599173in}{3.278164in}}%
\pgfpathlineto{\pgfqpoint{1.599173in}{3.281113in}}%
\pgfpathlineto{\pgfqpoint{1.603714in}{3.281113in}}%
\pgfpathlineto{\pgfqpoint{1.603714in}{3.278164in}}%
\pgfpathmoveto{\pgfqpoint{1.599173in}{3.281113in}}%
\pgfpathlineto{\pgfqpoint{1.599173in}{3.281113in}}%
\pgfpathlineto{\pgfqpoint{1.599173in}{3.284062in}}%
\pgfpathlineto{\pgfqpoint{1.603714in}{3.284062in}}%
\pgfpathlineto{\pgfqpoint{1.603714in}{3.281113in}}%
\pgfpathmoveto{\pgfqpoint{1.612797in}{3.266368in}}%
\pgfpathlineto{\pgfqpoint{1.612797in}{3.266368in}}%
\pgfpathlineto{\pgfqpoint{1.612797in}{3.269317in}}%
\pgfpathlineto{\pgfqpoint{1.617338in}{3.269317in}}%
\pgfpathlineto{\pgfqpoint{1.617338in}{3.266368in}}%
\pgfpathmoveto{\pgfqpoint{1.612797in}{3.269317in}}%
\pgfpathlineto{\pgfqpoint{1.612797in}{3.269317in}}%
\pgfpathlineto{\pgfqpoint{1.612797in}{3.272266in}}%
\pgfpathlineto{\pgfqpoint{1.617338in}{3.272266in}}%
\pgfpathlineto{\pgfqpoint{1.617338in}{3.269317in}}%
\pgfpathmoveto{\pgfqpoint{1.617338in}{3.266368in}}%
\pgfpathlineto{\pgfqpoint{1.617338in}{3.266368in}}%
\pgfpathlineto{\pgfqpoint{1.617338in}{3.269317in}}%
\pgfpathlineto{\pgfqpoint{1.621879in}{3.269317in}}%
\pgfpathlineto{\pgfqpoint{1.621879in}{3.266368in}}%
\pgfpathmoveto{\pgfqpoint{1.617338in}{3.269317in}}%
\pgfpathlineto{\pgfqpoint{1.617338in}{3.269317in}}%
\pgfpathlineto{\pgfqpoint{1.617338in}{3.272266in}}%
\pgfpathlineto{\pgfqpoint{1.621879in}{3.272266in}}%
\pgfpathlineto{\pgfqpoint{1.621879in}{3.269317in}}%
\pgfpathmoveto{\pgfqpoint{1.603714in}{3.272266in}}%
\pgfpathlineto{\pgfqpoint{1.603714in}{3.272266in}}%
\pgfpathlineto{\pgfqpoint{1.603714in}{3.275215in}}%
\pgfpathlineto{\pgfqpoint{1.608255in}{3.275215in}}%
\pgfpathlineto{\pgfqpoint{1.608255in}{3.272266in}}%
\pgfpathmoveto{\pgfqpoint{1.603714in}{3.275215in}}%
\pgfpathlineto{\pgfqpoint{1.603714in}{3.275215in}}%
\pgfpathlineto{\pgfqpoint{1.603714in}{3.278164in}}%
\pgfpathlineto{\pgfqpoint{1.608255in}{3.278164in}}%
\pgfpathlineto{\pgfqpoint{1.608255in}{3.275215in}}%
\pgfpathmoveto{\pgfqpoint{1.608255in}{3.272266in}}%
\pgfpathlineto{\pgfqpoint{1.608255in}{3.272266in}}%
\pgfpathlineto{\pgfqpoint{1.608255in}{3.275215in}}%
\pgfpathlineto{\pgfqpoint{1.612797in}{3.275215in}}%
\pgfpathlineto{\pgfqpoint{1.612797in}{3.272266in}}%
\pgfpathmoveto{\pgfqpoint{1.608255in}{3.275215in}}%
\pgfpathlineto{\pgfqpoint{1.608255in}{3.275215in}}%
\pgfpathlineto{\pgfqpoint{1.608255in}{3.278164in}}%
\pgfpathlineto{\pgfqpoint{1.612797in}{3.278164in}}%
\pgfpathlineto{\pgfqpoint{1.612797in}{3.275215in}}%
\pgfpathmoveto{\pgfqpoint{1.603714in}{3.278164in}}%
\pgfpathlineto{\pgfqpoint{1.603714in}{3.278164in}}%
\pgfpathlineto{\pgfqpoint{1.603714in}{3.281113in}}%
\pgfpathlineto{\pgfqpoint{1.608255in}{3.281113in}}%
\pgfpathlineto{\pgfqpoint{1.608255in}{3.278164in}}%
\pgfpathmoveto{\pgfqpoint{1.603714in}{3.281113in}}%
\pgfpathlineto{\pgfqpoint{1.603714in}{3.281113in}}%
\pgfpathlineto{\pgfqpoint{1.603714in}{3.284062in}}%
\pgfpathlineto{\pgfqpoint{1.608255in}{3.284062in}}%
\pgfpathlineto{\pgfqpoint{1.608255in}{3.281113in}}%
\pgfpathmoveto{\pgfqpoint{1.608255in}{3.278164in}}%
\pgfpathlineto{\pgfqpoint{1.608255in}{3.278164in}}%
\pgfpathlineto{\pgfqpoint{1.608255in}{3.281113in}}%
\pgfpathlineto{\pgfqpoint{1.612797in}{3.281113in}}%
\pgfpathlineto{\pgfqpoint{1.612797in}{3.278164in}}%
\pgfpathmoveto{\pgfqpoint{1.612797in}{3.272266in}}%
\pgfpathlineto{\pgfqpoint{1.612797in}{3.272266in}}%
\pgfpathlineto{\pgfqpoint{1.612797in}{3.275215in}}%
\pgfpathlineto{\pgfqpoint{1.617338in}{3.275215in}}%
\pgfpathlineto{\pgfqpoint{1.617338in}{3.272266in}}%
\pgfpathmoveto{\pgfqpoint{1.612797in}{3.275215in}}%
\pgfpathlineto{\pgfqpoint{1.612797in}{3.275215in}}%
\pgfpathlineto{\pgfqpoint{1.612797in}{3.278164in}}%
\pgfpathlineto{\pgfqpoint{1.617338in}{3.278164in}}%
\pgfpathlineto{\pgfqpoint{1.617338in}{3.275215in}}%
\pgfpathmoveto{\pgfqpoint{1.567384in}{3.301757in}}%
\pgfpathlineto{\pgfqpoint{1.567384in}{3.301757in}}%
\pgfpathlineto{\pgfqpoint{1.567384in}{3.304706in}}%
\pgfpathlineto{\pgfqpoint{1.571926in}{3.304706in}}%
\pgfpathlineto{\pgfqpoint{1.571926in}{3.301757in}}%
\pgfpathmoveto{\pgfqpoint{1.567384in}{3.304706in}}%
\pgfpathlineto{\pgfqpoint{1.567384in}{3.304706in}}%
\pgfpathlineto{\pgfqpoint{1.567384in}{3.307655in}}%
\pgfpathlineto{\pgfqpoint{1.571926in}{3.307655in}}%
\pgfpathlineto{\pgfqpoint{1.571926in}{3.304706in}}%
\pgfpathmoveto{\pgfqpoint{1.571926in}{3.301757in}}%
\pgfpathlineto{\pgfqpoint{1.571926in}{3.301757in}}%
\pgfpathlineto{\pgfqpoint{1.571926in}{3.304706in}}%
\pgfpathlineto{\pgfqpoint{1.576467in}{3.304706in}}%
\pgfpathlineto{\pgfqpoint{1.576467in}{3.301757in}}%
\pgfpathmoveto{\pgfqpoint{1.571926in}{3.304706in}}%
\pgfpathlineto{\pgfqpoint{1.571926in}{3.304706in}}%
\pgfpathlineto{\pgfqpoint{1.571926in}{3.307655in}}%
\pgfpathlineto{\pgfqpoint{1.576467in}{3.307655in}}%
\pgfpathlineto{\pgfqpoint{1.576467in}{3.304706in}}%
\pgfpathmoveto{\pgfqpoint{1.576467in}{3.295859in}}%
\pgfpathlineto{\pgfqpoint{1.576467in}{3.295859in}}%
\pgfpathlineto{\pgfqpoint{1.576467in}{3.298808in}}%
\pgfpathlineto{\pgfqpoint{1.581008in}{3.298808in}}%
\pgfpathlineto{\pgfqpoint{1.581008in}{3.295859in}}%
\pgfpathmoveto{\pgfqpoint{1.576467in}{3.298808in}}%
\pgfpathlineto{\pgfqpoint{1.576467in}{3.298808in}}%
\pgfpathlineto{\pgfqpoint{1.576467in}{3.301757in}}%
\pgfpathlineto{\pgfqpoint{1.581008in}{3.301757in}}%
\pgfpathlineto{\pgfqpoint{1.581008in}{3.298808in}}%
\pgfpathmoveto{\pgfqpoint{1.581008in}{3.295859in}}%
\pgfpathlineto{\pgfqpoint{1.581008in}{3.295859in}}%
\pgfpathlineto{\pgfqpoint{1.581008in}{3.298808in}}%
\pgfpathlineto{\pgfqpoint{1.585549in}{3.298808in}}%
\pgfpathlineto{\pgfqpoint{1.585549in}{3.295859in}}%
\pgfpathmoveto{\pgfqpoint{1.581008in}{3.298808in}}%
\pgfpathlineto{\pgfqpoint{1.581008in}{3.298808in}}%
\pgfpathlineto{\pgfqpoint{1.581008in}{3.301757in}}%
\pgfpathlineto{\pgfqpoint{1.585549in}{3.301757in}}%
\pgfpathlineto{\pgfqpoint{1.585549in}{3.298808in}}%
\pgfpathmoveto{\pgfqpoint{1.576467in}{3.301757in}}%
\pgfpathlineto{\pgfqpoint{1.576467in}{3.301757in}}%
\pgfpathlineto{\pgfqpoint{1.576467in}{3.304706in}}%
\pgfpathlineto{\pgfqpoint{1.581008in}{3.304706in}}%
\pgfpathlineto{\pgfqpoint{1.581008in}{3.301757in}}%
\pgfpathmoveto{\pgfqpoint{1.576467in}{3.304706in}}%
\pgfpathlineto{\pgfqpoint{1.576467in}{3.304706in}}%
\pgfpathlineto{\pgfqpoint{1.576467in}{3.307655in}}%
\pgfpathlineto{\pgfqpoint{1.581008in}{3.307655in}}%
\pgfpathlineto{\pgfqpoint{1.581008in}{3.304706in}}%
\pgfpathmoveto{\pgfqpoint{1.581008in}{3.301757in}}%
\pgfpathlineto{\pgfqpoint{1.581008in}{3.301757in}}%
\pgfpathlineto{\pgfqpoint{1.581008in}{3.304706in}}%
\pgfpathlineto{\pgfqpoint{1.585549in}{3.304706in}}%
\pgfpathlineto{\pgfqpoint{1.585549in}{3.301757in}}%
\pgfpathmoveto{\pgfqpoint{1.558302in}{3.313554in}}%
\pgfpathlineto{\pgfqpoint{1.558302in}{3.313554in}}%
\pgfpathlineto{\pgfqpoint{1.558302in}{3.316503in}}%
\pgfpathlineto{\pgfqpoint{1.562843in}{3.316503in}}%
\pgfpathlineto{\pgfqpoint{1.562843in}{3.313554in}}%
\pgfpathmoveto{\pgfqpoint{1.558302in}{3.316503in}}%
\pgfpathlineto{\pgfqpoint{1.558302in}{3.316503in}}%
\pgfpathlineto{\pgfqpoint{1.558302in}{3.319452in}}%
\pgfpathlineto{\pgfqpoint{1.562843in}{3.319452in}}%
\pgfpathlineto{\pgfqpoint{1.562843in}{3.316503in}}%
\pgfpathmoveto{\pgfqpoint{1.562843in}{3.313554in}}%
\pgfpathlineto{\pgfqpoint{1.562843in}{3.313554in}}%
\pgfpathlineto{\pgfqpoint{1.562843in}{3.316503in}}%
\pgfpathlineto{\pgfqpoint{1.567384in}{3.316503in}}%
\pgfpathlineto{\pgfqpoint{1.567384in}{3.313554in}}%
\pgfpathmoveto{\pgfqpoint{1.562843in}{3.316503in}}%
\pgfpathlineto{\pgfqpoint{1.562843in}{3.316503in}}%
\pgfpathlineto{\pgfqpoint{1.562843in}{3.319452in}}%
\pgfpathlineto{\pgfqpoint{1.567384in}{3.319452in}}%
\pgfpathlineto{\pgfqpoint{1.567384in}{3.316503in}}%
\pgfpathmoveto{\pgfqpoint{1.549220in}{3.319452in}}%
\pgfpathlineto{\pgfqpoint{1.549220in}{3.319452in}}%
\pgfpathlineto{\pgfqpoint{1.549220in}{3.322401in}}%
\pgfpathlineto{\pgfqpoint{1.553761in}{3.322401in}}%
\pgfpathlineto{\pgfqpoint{1.553761in}{3.319452in}}%
\pgfpathmoveto{\pgfqpoint{1.549220in}{3.322401in}}%
\pgfpathlineto{\pgfqpoint{1.549220in}{3.322401in}}%
\pgfpathlineto{\pgfqpoint{1.549220in}{3.325350in}}%
\pgfpathlineto{\pgfqpoint{1.553761in}{3.325350in}}%
\pgfpathlineto{\pgfqpoint{1.553761in}{3.322401in}}%
\pgfpathmoveto{\pgfqpoint{1.553761in}{3.319452in}}%
\pgfpathlineto{\pgfqpoint{1.553761in}{3.319452in}}%
\pgfpathlineto{\pgfqpoint{1.553761in}{3.322401in}}%
\pgfpathlineto{\pgfqpoint{1.558302in}{3.322401in}}%
\pgfpathlineto{\pgfqpoint{1.558302in}{3.319452in}}%
\pgfpathmoveto{\pgfqpoint{1.553761in}{3.322401in}}%
\pgfpathlineto{\pgfqpoint{1.553761in}{3.322401in}}%
\pgfpathlineto{\pgfqpoint{1.553761in}{3.325350in}}%
\pgfpathlineto{\pgfqpoint{1.558302in}{3.325350in}}%
\pgfpathlineto{\pgfqpoint{1.558302in}{3.322401in}}%
\pgfpathmoveto{\pgfqpoint{1.549220in}{3.325350in}}%
\pgfpathlineto{\pgfqpoint{1.549220in}{3.325350in}}%
\pgfpathlineto{\pgfqpoint{1.549220in}{3.328299in}}%
\pgfpathlineto{\pgfqpoint{1.553761in}{3.328299in}}%
\pgfpathlineto{\pgfqpoint{1.553761in}{3.325350in}}%
\pgfpathmoveto{\pgfqpoint{1.549220in}{3.328299in}}%
\pgfpathlineto{\pgfqpoint{1.549220in}{3.328299in}}%
\pgfpathlineto{\pgfqpoint{1.549220in}{3.331248in}}%
\pgfpathlineto{\pgfqpoint{1.553761in}{3.331248in}}%
\pgfpathlineto{\pgfqpoint{1.553761in}{3.328299in}}%
\pgfpathmoveto{\pgfqpoint{1.553761in}{3.325350in}}%
\pgfpathlineto{\pgfqpoint{1.553761in}{3.325350in}}%
\pgfpathlineto{\pgfqpoint{1.553761in}{3.328299in}}%
\pgfpathlineto{\pgfqpoint{1.558302in}{3.328299in}}%
\pgfpathlineto{\pgfqpoint{1.558302in}{3.325350in}}%
\pgfpathmoveto{\pgfqpoint{1.558302in}{3.319452in}}%
\pgfpathlineto{\pgfqpoint{1.558302in}{3.319452in}}%
\pgfpathlineto{\pgfqpoint{1.558302in}{3.322401in}}%
\pgfpathlineto{\pgfqpoint{1.562843in}{3.322401in}}%
\pgfpathlineto{\pgfqpoint{1.562843in}{3.319452in}}%
\pgfpathmoveto{\pgfqpoint{1.558302in}{3.322401in}}%
\pgfpathlineto{\pgfqpoint{1.558302in}{3.322401in}}%
\pgfpathlineto{\pgfqpoint{1.558302in}{3.325350in}}%
\pgfpathlineto{\pgfqpoint{1.562843in}{3.325350in}}%
\pgfpathlineto{\pgfqpoint{1.562843in}{3.322401in}}%
\pgfpathmoveto{\pgfqpoint{1.562843in}{3.319452in}}%
\pgfpathlineto{\pgfqpoint{1.562843in}{3.319452in}}%
\pgfpathlineto{\pgfqpoint{1.562843in}{3.322401in}}%
\pgfpathlineto{\pgfqpoint{1.567384in}{3.322401in}}%
\pgfpathlineto{\pgfqpoint{1.567384in}{3.319452in}}%
\pgfpathmoveto{\pgfqpoint{1.567384in}{3.307655in}}%
\pgfpathlineto{\pgfqpoint{1.567384in}{3.307655in}}%
\pgfpathlineto{\pgfqpoint{1.567384in}{3.310604in}}%
\pgfpathlineto{\pgfqpoint{1.571926in}{3.310604in}}%
\pgfpathlineto{\pgfqpoint{1.571926in}{3.307655in}}%
\pgfpathmoveto{\pgfqpoint{1.567384in}{3.310604in}}%
\pgfpathlineto{\pgfqpoint{1.567384in}{3.310604in}}%
\pgfpathlineto{\pgfqpoint{1.567384in}{3.313554in}}%
\pgfpathlineto{\pgfqpoint{1.571926in}{3.313554in}}%
\pgfpathlineto{\pgfqpoint{1.571926in}{3.310604in}}%
\pgfpathmoveto{\pgfqpoint{1.571926in}{3.307655in}}%
\pgfpathlineto{\pgfqpoint{1.571926in}{3.307655in}}%
\pgfpathlineto{\pgfqpoint{1.571926in}{3.310604in}}%
\pgfpathlineto{\pgfqpoint{1.576467in}{3.310604in}}%
\pgfpathlineto{\pgfqpoint{1.576467in}{3.307655in}}%
\pgfpathmoveto{\pgfqpoint{1.571926in}{3.310604in}}%
\pgfpathlineto{\pgfqpoint{1.571926in}{3.310604in}}%
\pgfpathlineto{\pgfqpoint{1.571926in}{3.313554in}}%
\pgfpathlineto{\pgfqpoint{1.576467in}{3.313554in}}%
\pgfpathlineto{\pgfqpoint{1.576467in}{3.310604in}}%
\pgfpathmoveto{\pgfqpoint{1.567384in}{3.313554in}}%
\pgfpathlineto{\pgfqpoint{1.567384in}{3.313554in}}%
\pgfpathlineto{\pgfqpoint{1.567384in}{3.316503in}}%
\pgfpathlineto{\pgfqpoint{1.571926in}{3.316503in}}%
\pgfpathlineto{\pgfqpoint{1.571926in}{3.313554in}}%
\pgfpathmoveto{\pgfqpoint{1.585549in}{3.289961in}}%
\pgfpathlineto{\pgfqpoint{1.585549in}{3.289961in}}%
\pgfpathlineto{\pgfqpoint{1.585549in}{3.292910in}}%
\pgfpathlineto{\pgfqpoint{1.590090in}{3.292910in}}%
\pgfpathlineto{\pgfqpoint{1.590090in}{3.289961in}}%
\pgfpathmoveto{\pgfqpoint{1.585549in}{3.292910in}}%
\pgfpathlineto{\pgfqpoint{1.585549in}{3.292910in}}%
\pgfpathlineto{\pgfqpoint{1.585549in}{3.295859in}}%
\pgfpathlineto{\pgfqpoint{1.590090in}{3.295859in}}%
\pgfpathlineto{\pgfqpoint{1.590090in}{3.292910in}}%
\pgfpathmoveto{\pgfqpoint{1.590090in}{3.289961in}}%
\pgfpathlineto{\pgfqpoint{1.590090in}{3.289961in}}%
\pgfpathlineto{\pgfqpoint{1.590090in}{3.292910in}}%
\pgfpathlineto{\pgfqpoint{1.594632in}{3.292910in}}%
\pgfpathlineto{\pgfqpoint{1.594632in}{3.289961in}}%
\pgfpathmoveto{\pgfqpoint{1.590090in}{3.292910in}}%
\pgfpathlineto{\pgfqpoint{1.590090in}{3.292910in}}%
\pgfpathlineto{\pgfqpoint{1.590090in}{3.295859in}}%
\pgfpathlineto{\pgfqpoint{1.594632in}{3.295859in}}%
\pgfpathlineto{\pgfqpoint{1.594632in}{3.292910in}}%
\pgfpathmoveto{\pgfqpoint{1.594632in}{3.284062in}}%
\pgfpathlineto{\pgfqpoint{1.594632in}{3.284062in}}%
\pgfpathlineto{\pgfqpoint{1.594632in}{3.287012in}}%
\pgfpathlineto{\pgfqpoint{1.599173in}{3.287012in}}%
\pgfpathlineto{\pgfqpoint{1.599173in}{3.284062in}}%
\pgfpathmoveto{\pgfqpoint{1.594632in}{3.287012in}}%
\pgfpathlineto{\pgfqpoint{1.594632in}{3.287012in}}%
\pgfpathlineto{\pgfqpoint{1.594632in}{3.289961in}}%
\pgfpathlineto{\pgfqpoint{1.599173in}{3.289961in}}%
\pgfpathlineto{\pgfqpoint{1.599173in}{3.287012in}}%
\pgfpathmoveto{\pgfqpoint{1.599173in}{3.284062in}}%
\pgfpathlineto{\pgfqpoint{1.599173in}{3.284062in}}%
\pgfpathlineto{\pgfqpoint{1.599173in}{3.287012in}}%
\pgfpathlineto{\pgfqpoint{1.603714in}{3.287012in}}%
\pgfpathlineto{\pgfqpoint{1.603714in}{3.284062in}}%
\pgfpathmoveto{\pgfqpoint{1.599173in}{3.287012in}}%
\pgfpathlineto{\pgfqpoint{1.599173in}{3.287012in}}%
\pgfpathlineto{\pgfqpoint{1.599173in}{3.289961in}}%
\pgfpathlineto{\pgfqpoint{1.603714in}{3.289961in}}%
\pgfpathlineto{\pgfqpoint{1.603714in}{3.287012in}}%
\pgfpathmoveto{\pgfqpoint{1.594632in}{3.289961in}}%
\pgfpathlineto{\pgfqpoint{1.594632in}{3.289961in}}%
\pgfpathlineto{\pgfqpoint{1.594632in}{3.292910in}}%
\pgfpathlineto{\pgfqpoint{1.599173in}{3.292910in}}%
\pgfpathlineto{\pgfqpoint{1.599173in}{3.289961in}}%
\pgfpathmoveto{\pgfqpoint{1.585549in}{3.295859in}}%
\pgfpathlineto{\pgfqpoint{1.585549in}{3.295859in}}%
\pgfpathlineto{\pgfqpoint{1.585549in}{3.298808in}}%
\pgfpathlineto{\pgfqpoint{1.590090in}{3.298808in}}%
\pgfpathlineto{\pgfqpoint{1.590090in}{3.295859in}}%
\pgfpathmoveto{\pgfqpoint{1.585549in}{3.298808in}}%
\pgfpathlineto{\pgfqpoint{1.585549in}{3.298808in}}%
\pgfpathlineto{\pgfqpoint{1.585549in}{3.301757in}}%
\pgfpathlineto{\pgfqpoint{1.590090in}{3.301757in}}%
\pgfpathlineto{\pgfqpoint{1.590090in}{3.298808in}}%
\pgfpathmoveto{\pgfqpoint{1.503807in}{3.360740in}}%
\pgfpathlineto{\pgfqpoint{1.503807in}{3.360740in}}%
\pgfpathlineto{\pgfqpoint{1.503807in}{3.363690in}}%
\pgfpathlineto{\pgfqpoint{1.508349in}{3.363690in}}%
\pgfpathlineto{\pgfqpoint{1.508349in}{3.360740in}}%
\pgfpathmoveto{\pgfqpoint{1.503807in}{3.363690in}}%
\pgfpathlineto{\pgfqpoint{1.503807in}{3.363690in}}%
\pgfpathlineto{\pgfqpoint{1.503807in}{3.366639in}}%
\pgfpathlineto{\pgfqpoint{1.508349in}{3.366639in}}%
\pgfpathlineto{\pgfqpoint{1.508349in}{3.363690in}}%
\pgfpathmoveto{\pgfqpoint{1.508349in}{3.360740in}}%
\pgfpathlineto{\pgfqpoint{1.508349in}{3.360740in}}%
\pgfpathlineto{\pgfqpoint{1.508349in}{3.363690in}}%
\pgfpathlineto{\pgfqpoint{1.512890in}{3.363690in}}%
\pgfpathlineto{\pgfqpoint{1.512890in}{3.360740in}}%
\pgfpathmoveto{\pgfqpoint{1.508349in}{3.363690in}}%
\pgfpathlineto{\pgfqpoint{1.508349in}{3.363690in}}%
\pgfpathlineto{\pgfqpoint{1.508349in}{3.366639in}}%
\pgfpathlineto{\pgfqpoint{1.512890in}{3.366639in}}%
\pgfpathlineto{\pgfqpoint{1.512890in}{3.363690in}}%
\pgfpathmoveto{\pgfqpoint{1.494725in}{3.366639in}}%
\pgfpathlineto{\pgfqpoint{1.494725in}{3.366639in}}%
\pgfpathlineto{\pgfqpoint{1.494725in}{3.369588in}}%
\pgfpathlineto{\pgfqpoint{1.499266in}{3.369588in}}%
\pgfpathlineto{\pgfqpoint{1.499266in}{3.366639in}}%
\pgfpathmoveto{\pgfqpoint{1.494725in}{3.369588in}}%
\pgfpathlineto{\pgfqpoint{1.494725in}{3.369588in}}%
\pgfpathlineto{\pgfqpoint{1.494725in}{3.372537in}}%
\pgfpathlineto{\pgfqpoint{1.499266in}{3.372537in}}%
\pgfpathlineto{\pgfqpoint{1.499266in}{3.369588in}}%
\pgfpathmoveto{\pgfqpoint{1.499266in}{3.366639in}}%
\pgfpathlineto{\pgfqpoint{1.499266in}{3.366639in}}%
\pgfpathlineto{\pgfqpoint{1.499266in}{3.369588in}}%
\pgfpathlineto{\pgfqpoint{1.503807in}{3.369588in}}%
\pgfpathlineto{\pgfqpoint{1.503807in}{3.366639in}}%
\pgfpathmoveto{\pgfqpoint{1.499266in}{3.369588in}}%
\pgfpathlineto{\pgfqpoint{1.499266in}{3.369588in}}%
\pgfpathlineto{\pgfqpoint{1.499266in}{3.372537in}}%
\pgfpathlineto{\pgfqpoint{1.503807in}{3.372537in}}%
\pgfpathlineto{\pgfqpoint{1.503807in}{3.369588in}}%
\pgfpathmoveto{\pgfqpoint{1.494725in}{3.372537in}}%
\pgfpathlineto{\pgfqpoint{1.494725in}{3.372537in}}%
\pgfpathlineto{\pgfqpoint{1.494725in}{3.375487in}}%
\pgfpathlineto{\pgfqpoint{1.499266in}{3.375487in}}%
\pgfpathlineto{\pgfqpoint{1.499266in}{3.372537in}}%
\pgfpathmoveto{\pgfqpoint{1.494725in}{3.375487in}}%
\pgfpathlineto{\pgfqpoint{1.494725in}{3.375487in}}%
\pgfpathlineto{\pgfqpoint{1.494725in}{3.378436in}}%
\pgfpathlineto{\pgfqpoint{1.499266in}{3.378436in}}%
\pgfpathlineto{\pgfqpoint{1.499266in}{3.375487in}}%
\pgfpathmoveto{\pgfqpoint{1.499266in}{3.372537in}}%
\pgfpathlineto{\pgfqpoint{1.499266in}{3.372537in}}%
\pgfpathlineto{\pgfqpoint{1.499266in}{3.375487in}}%
\pgfpathlineto{\pgfqpoint{1.503807in}{3.375487in}}%
\pgfpathlineto{\pgfqpoint{1.503807in}{3.372537in}}%
\pgfpathmoveto{\pgfqpoint{1.503807in}{3.366639in}}%
\pgfpathlineto{\pgfqpoint{1.503807in}{3.366639in}}%
\pgfpathlineto{\pgfqpoint{1.503807in}{3.369588in}}%
\pgfpathlineto{\pgfqpoint{1.508349in}{3.369588in}}%
\pgfpathlineto{\pgfqpoint{1.508349in}{3.366639in}}%
\pgfpathmoveto{\pgfqpoint{1.503807in}{3.369588in}}%
\pgfpathlineto{\pgfqpoint{1.503807in}{3.369588in}}%
\pgfpathlineto{\pgfqpoint{1.503807in}{3.372537in}}%
\pgfpathlineto{\pgfqpoint{1.508349in}{3.372537in}}%
\pgfpathlineto{\pgfqpoint{1.508349in}{3.369588in}}%
\pgfpathmoveto{\pgfqpoint{1.508349in}{3.366639in}}%
\pgfpathlineto{\pgfqpoint{1.508349in}{3.366639in}}%
\pgfpathlineto{\pgfqpoint{1.508349in}{3.369588in}}%
\pgfpathlineto{\pgfqpoint{1.512890in}{3.369588in}}%
\pgfpathlineto{\pgfqpoint{1.512890in}{3.366639in}}%
\pgfpathmoveto{\pgfqpoint{1.521972in}{3.343045in}}%
\pgfpathlineto{\pgfqpoint{1.521972in}{3.343045in}}%
\pgfpathlineto{\pgfqpoint{1.521972in}{3.345994in}}%
\pgfpathlineto{\pgfqpoint{1.526513in}{3.345994in}}%
\pgfpathlineto{\pgfqpoint{1.526513in}{3.343045in}}%
\pgfpathmoveto{\pgfqpoint{1.521972in}{3.345994in}}%
\pgfpathlineto{\pgfqpoint{1.521972in}{3.345994in}}%
\pgfpathlineto{\pgfqpoint{1.521972in}{3.348944in}}%
\pgfpathlineto{\pgfqpoint{1.526513in}{3.348944in}}%
\pgfpathlineto{\pgfqpoint{1.526513in}{3.345994in}}%
\pgfpathmoveto{\pgfqpoint{1.526513in}{3.343045in}}%
\pgfpathlineto{\pgfqpoint{1.526513in}{3.343045in}}%
\pgfpathlineto{\pgfqpoint{1.526513in}{3.345994in}}%
\pgfpathlineto{\pgfqpoint{1.531055in}{3.345994in}}%
\pgfpathlineto{\pgfqpoint{1.531055in}{3.343045in}}%
\pgfpathmoveto{\pgfqpoint{1.526513in}{3.345994in}}%
\pgfpathlineto{\pgfqpoint{1.526513in}{3.345994in}}%
\pgfpathlineto{\pgfqpoint{1.526513in}{3.348944in}}%
\pgfpathlineto{\pgfqpoint{1.531055in}{3.348944in}}%
\pgfpathlineto{\pgfqpoint{1.531055in}{3.345994in}}%
\pgfpathmoveto{\pgfqpoint{1.521972in}{3.348944in}}%
\pgfpathlineto{\pgfqpoint{1.521972in}{3.348944in}}%
\pgfpathlineto{\pgfqpoint{1.521972in}{3.351893in}}%
\pgfpathlineto{\pgfqpoint{1.526513in}{3.351893in}}%
\pgfpathlineto{\pgfqpoint{1.526513in}{3.348944in}}%
\pgfpathmoveto{\pgfqpoint{1.521972in}{3.351893in}}%
\pgfpathlineto{\pgfqpoint{1.521972in}{3.351893in}}%
\pgfpathlineto{\pgfqpoint{1.521972in}{3.354842in}}%
\pgfpathlineto{\pgfqpoint{1.526513in}{3.354842in}}%
\pgfpathlineto{\pgfqpoint{1.526513in}{3.351893in}}%
\pgfpathmoveto{\pgfqpoint{1.526513in}{3.348944in}}%
\pgfpathlineto{\pgfqpoint{1.526513in}{3.348944in}}%
\pgfpathlineto{\pgfqpoint{1.526513in}{3.351893in}}%
\pgfpathlineto{\pgfqpoint{1.531055in}{3.351893in}}%
\pgfpathlineto{\pgfqpoint{1.531055in}{3.348944in}}%
\pgfpathmoveto{\pgfqpoint{1.531055in}{3.337147in}}%
\pgfpathlineto{\pgfqpoint{1.531055in}{3.337147in}}%
\pgfpathlineto{\pgfqpoint{1.531055in}{3.340096in}}%
\pgfpathlineto{\pgfqpoint{1.535596in}{3.340096in}}%
\pgfpathlineto{\pgfqpoint{1.535596in}{3.337147in}}%
\pgfpathmoveto{\pgfqpoint{1.531055in}{3.340096in}}%
\pgfpathlineto{\pgfqpoint{1.531055in}{3.340096in}}%
\pgfpathlineto{\pgfqpoint{1.531055in}{3.343045in}}%
\pgfpathlineto{\pgfqpoint{1.535596in}{3.343045in}}%
\pgfpathlineto{\pgfqpoint{1.535596in}{3.340096in}}%
\pgfpathmoveto{\pgfqpoint{1.535596in}{3.337147in}}%
\pgfpathlineto{\pgfqpoint{1.535596in}{3.337147in}}%
\pgfpathlineto{\pgfqpoint{1.535596in}{3.340096in}}%
\pgfpathlineto{\pgfqpoint{1.540137in}{3.340096in}}%
\pgfpathlineto{\pgfqpoint{1.540137in}{3.337147in}}%
\pgfpathmoveto{\pgfqpoint{1.535596in}{3.340096in}}%
\pgfpathlineto{\pgfqpoint{1.535596in}{3.340096in}}%
\pgfpathlineto{\pgfqpoint{1.535596in}{3.343045in}}%
\pgfpathlineto{\pgfqpoint{1.540137in}{3.343045in}}%
\pgfpathlineto{\pgfqpoint{1.540137in}{3.340096in}}%
\pgfpathmoveto{\pgfqpoint{1.540137in}{3.331248in}}%
\pgfpathlineto{\pgfqpoint{1.540137in}{3.331248in}}%
\pgfpathlineto{\pgfqpoint{1.540137in}{3.334198in}}%
\pgfpathlineto{\pgfqpoint{1.544678in}{3.334198in}}%
\pgfpathlineto{\pgfqpoint{1.544678in}{3.331248in}}%
\pgfpathmoveto{\pgfqpoint{1.540137in}{3.334198in}}%
\pgfpathlineto{\pgfqpoint{1.540137in}{3.334198in}}%
\pgfpathlineto{\pgfqpoint{1.540137in}{3.337147in}}%
\pgfpathlineto{\pgfqpoint{1.544678in}{3.337147in}}%
\pgfpathlineto{\pgfqpoint{1.544678in}{3.334198in}}%
\pgfpathmoveto{\pgfqpoint{1.544678in}{3.331248in}}%
\pgfpathlineto{\pgfqpoint{1.544678in}{3.331248in}}%
\pgfpathlineto{\pgfqpoint{1.544678in}{3.334198in}}%
\pgfpathlineto{\pgfqpoint{1.549220in}{3.334198in}}%
\pgfpathlineto{\pgfqpoint{1.549220in}{3.331248in}}%
\pgfpathmoveto{\pgfqpoint{1.544678in}{3.334198in}}%
\pgfpathlineto{\pgfqpoint{1.544678in}{3.334198in}}%
\pgfpathlineto{\pgfqpoint{1.544678in}{3.337147in}}%
\pgfpathlineto{\pgfqpoint{1.549220in}{3.337147in}}%
\pgfpathlineto{\pgfqpoint{1.549220in}{3.334198in}}%
\pgfpathmoveto{\pgfqpoint{1.540137in}{3.337147in}}%
\pgfpathlineto{\pgfqpoint{1.540137in}{3.337147in}}%
\pgfpathlineto{\pgfqpoint{1.540137in}{3.340096in}}%
\pgfpathlineto{\pgfqpoint{1.544678in}{3.340096in}}%
\pgfpathlineto{\pgfqpoint{1.544678in}{3.337147in}}%
\pgfpathmoveto{\pgfqpoint{1.531055in}{3.343045in}}%
\pgfpathlineto{\pgfqpoint{1.531055in}{3.343045in}}%
\pgfpathlineto{\pgfqpoint{1.531055in}{3.345994in}}%
\pgfpathlineto{\pgfqpoint{1.535596in}{3.345994in}}%
\pgfpathlineto{\pgfqpoint{1.535596in}{3.343045in}}%
\pgfpathmoveto{\pgfqpoint{1.531055in}{3.345994in}}%
\pgfpathlineto{\pgfqpoint{1.531055in}{3.345994in}}%
\pgfpathlineto{\pgfqpoint{1.531055in}{3.348944in}}%
\pgfpathlineto{\pgfqpoint{1.535596in}{3.348944in}}%
\pgfpathlineto{\pgfqpoint{1.535596in}{3.345994in}}%
\pgfpathmoveto{\pgfqpoint{1.535596in}{3.343045in}}%
\pgfpathlineto{\pgfqpoint{1.535596in}{3.343045in}}%
\pgfpathlineto{\pgfqpoint{1.535596in}{3.345994in}}%
\pgfpathlineto{\pgfqpoint{1.540137in}{3.345994in}}%
\pgfpathlineto{\pgfqpoint{1.540137in}{3.343045in}}%
\pgfpathmoveto{\pgfqpoint{1.512890in}{3.354842in}}%
\pgfpathlineto{\pgfqpoint{1.512890in}{3.354842in}}%
\pgfpathlineto{\pgfqpoint{1.512890in}{3.357791in}}%
\pgfpathlineto{\pgfqpoint{1.517431in}{3.357791in}}%
\pgfpathlineto{\pgfqpoint{1.517431in}{3.354842in}}%
\pgfpathmoveto{\pgfqpoint{1.512890in}{3.357791in}}%
\pgfpathlineto{\pgfqpoint{1.512890in}{3.357791in}}%
\pgfpathlineto{\pgfqpoint{1.512890in}{3.360740in}}%
\pgfpathlineto{\pgfqpoint{1.517431in}{3.360740in}}%
\pgfpathlineto{\pgfqpoint{1.517431in}{3.357791in}}%
\pgfpathmoveto{\pgfqpoint{1.517431in}{3.354842in}}%
\pgfpathlineto{\pgfqpoint{1.517431in}{3.354842in}}%
\pgfpathlineto{\pgfqpoint{1.517431in}{3.357791in}}%
\pgfpathlineto{\pgfqpoint{1.521972in}{3.357791in}}%
\pgfpathlineto{\pgfqpoint{1.521972in}{3.354842in}}%
\pgfpathmoveto{\pgfqpoint{1.517431in}{3.357791in}}%
\pgfpathlineto{\pgfqpoint{1.517431in}{3.357791in}}%
\pgfpathlineto{\pgfqpoint{1.517431in}{3.360740in}}%
\pgfpathlineto{\pgfqpoint{1.521972in}{3.360740in}}%
\pgfpathlineto{\pgfqpoint{1.521972in}{3.357791in}}%
\pgfpathmoveto{\pgfqpoint{1.512890in}{3.360740in}}%
\pgfpathlineto{\pgfqpoint{1.512890in}{3.360740in}}%
\pgfpathlineto{\pgfqpoint{1.512890in}{3.363690in}}%
\pgfpathlineto{\pgfqpoint{1.517431in}{3.363690in}}%
\pgfpathlineto{\pgfqpoint{1.517431in}{3.360740in}}%
\pgfpathmoveto{\pgfqpoint{1.521972in}{3.354842in}}%
\pgfpathlineto{\pgfqpoint{1.521972in}{3.354842in}}%
\pgfpathlineto{\pgfqpoint{1.521972in}{3.357791in}}%
\pgfpathlineto{\pgfqpoint{1.526513in}{3.357791in}}%
\pgfpathlineto{\pgfqpoint{1.526513in}{3.354842in}}%
\pgfpathmoveto{\pgfqpoint{1.476560in}{3.384334in}}%
\pgfpathlineto{\pgfqpoint{1.476560in}{3.384334in}}%
\pgfpathlineto{\pgfqpoint{1.476560in}{3.387283in}}%
\pgfpathlineto{\pgfqpoint{1.481101in}{3.387283in}}%
\pgfpathlineto{\pgfqpoint{1.481101in}{3.384334in}}%
\pgfpathmoveto{\pgfqpoint{1.476560in}{3.387283in}}%
\pgfpathlineto{\pgfqpoint{1.476560in}{3.387283in}}%
\pgfpathlineto{\pgfqpoint{1.476560in}{3.390233in}}%
\pgfpathlineto{\pgfqpoint{1.481101in}{3.390233in}}%
\pgfpathlineto{\pgfqpoint{1.481101in}{3.387283in}}%
\pgfpathmoveto{\pgfqpoint{1.481101in}{3.384334in}}%
\pgfpathlineto{\pgfqpoint{1.481101in}{3.384334in}}%
\pgfpathlineto{\pgfqpoint{1.481101in}{3.387283in}}%
\pgfpathlineto{\pgfqpoint{1.485642in}{3.387283in}}%
\pgfpathlineto{\pgfqpoint{1.485642in}{3.384334in}}%
\pgfpathmoveto{\pgfqpoint{1.481101in}{3.387283in}}%
\pgfpathlineto{\pgfqpoint{1.481101in}{3.387283in}}%
\pgfpathlineto{\pgfqpoint{1.481101in}{3.390233in}}%
\pgfpathlineto{\pgfqpoint{1.485642in}{3.390233in}}%
\pgfpathlineto{\pgfqpoint{1.485642in}{3.387283in}}%
\pgfpathmoveto{\pgfqpoint{1.485642in}{3.378436in}}%
\pgfpathlineto{\pgfqpoint{1.485642in}{3.378436in}}%
\pgfpathlineto{\pgfqpoint{1.485642in}{3.381385in}}%
\pgfpathlineto{\pgfqpoint{1.490184in}{3.381385in}}%
\pgfpathlineto{\pgfqpoint{1.490184in}{3.378436in}}%
\pgfpathmoveto{\pgfqpoint{1.485642in}{3.381385in}}%
\pgfpathlineto{\pgfqpoint{1.485642in}{3.381385in}}%
\pgfpathlineto{\pgfqpoint{1.485642in}{3.384334in}}%
\pgfpathlineto{\pgfqpoint{1.490184in}{3.384334in}}%
\pgfpathlineto{\pgfqpoint{1.490184in}{3.381385in}}%
\pgfpathmoveto{\pgfqpoint{1.490184in}{3.378436in}}%
\pgfpathlineto{\pgfqpoint{1.490184in}{3.378436in}}%
\pgfpathlineto{\pgfqpoint{1.490184in}{3.381385in}}%
\pgfpathlineto{\pgfqpoint{1.494725in}{3.381385in}}%
\pgfpathlineto{\pgfqpoint{1.494725in}{3.378436in}}%
\pgfpathmoveto{\pgfqpoint{1.490184in}{3.381385in}}%
\pgfpathlineto{\pgfqpoint{1.490184in}{3.381385in}}%
\pgfpathlineto{\pgfqpoint{1.490184in}{3.384334in}}%
\pgfpathlineto{\pgfqpoint{1.494725in}{3.384334in}}%
\pgfpathlineto{\pgfqpoint{1.494725in}{3.381385in}}%
\pgfpathmoveto{\pgfqpoint{1.485642in}{3.384334in}}%
\pgfpathlineto{\pgfqpoint{1.485642in}{3.384334in}}%
\pgfpathlineto{\pgfqpoint{1.485642in}{3.387283in}}%
\pgfpathlineto{\pgfqpoint{1.490184in}{3.387283in}}%
\pgfpathlineto{\pgfqpoint{1.490184in}{3.384334in}}%
\pgfpathmoveto{\pgfqpoint{1.476560in}{3.390233in}}%
\pgfpathlineto{\pgfqpoint{1.476560in}{3.390233in}}%
\pgfpathlineto{\pgfqpoint{1.476560in}{3.393182in}}%
\pgfpathlineto{\pgfqpoint{1.481101in}{3.393182in}}%
\pgfpathlineto{\pgfqpoint{1.481101in}{3.390233in}}%
\pgfpathmoveto{\pgfqpoint{1.476560in}{3.393182in}}%
\pgfpathlineto{\pgfqpoint{1.476560in}{3.393182in}}%
\pgfpathlineto{\pgfqpoint{1.476560in}{3.396131in}}%
\pgfpathlineto{\pgfqpoint{1.481101in}{3.396131in}}%
\pgfpathlineto{\pgfqpoint{1.481101in}{3.393182in}}%
\pgfpathmoveto{\pgfqpoint{1.481101in}{3.390233in}}%
\pgfpathlineto{\pgfqpoint{1.481101in}{3.390233in}}%
\pgfpathlineto{\pgfqpoint{1.481101in}{3.393182in}}%
\pgfpathlineto{\pgfqpoint{1.485642in}{3.393182in}}%
\pgfpathlineto{\pgfqpoint{1.485642in}{3.390233in}}%
\pgfpathmoveto{\pgfqpoint{1.494725in}{3.378436in}}%
\pgfpathlineto{\pgfqpoint{1.494725in}{3.378436in}}%
\pgfpathlineto{\pgfqpoint{1.494725in}{3.381385in}}%
\pgfpathlineto{\pgfqpoint{1.499266in}{3.381385in}}%
\pgfpathlineto{\pgfqpoint{1.499266in}{3.378436in}}%
\pgfpathmoveto{\pgfqpoint{1.549220in}{3.331248in}}%
\pgfpathlineto{\pgfqpoint{1.549220in}{3.331248in}}%
\pgfpathlineto{\pgfqpoint{1.549220in}{3.334198in}}%
\pgfpathlineto{\pgfqpoint{1.553761in}{3.334198in}}%
\pgfpathlineto{\pgfqpoint{1.553761in}{3.331248in}}%
\pgfpathmoveto{\pgfqpoint{1.649125in}{3.230978in}}%
\pgfpathlineto{\pgfqpoint{1.649125in}{3.230978in}}%
\pgfpathlineto{\pgfqpoint{1.649125in}{3.233927in}}%
\pgfpathlineto{\pgfqpoint{1.653666in}{3.233927in}}%
\pgfpathlineto{\pgfqpoint{1.653666in}{3.230978in}}%
\pgfpathmoveto{\pgfqpoint{1.649125in}{3.233927in}}%
\pgfpathlineto{\pgfqpoint{1.649125in}{3.233927in}}%
\pgfpathlineto{\pgfqpoint{1.649125in}{3.236876in}}%
\pgfpathlineto{\pgfqpoint{1.653666in}{3.236876in}}%
\pgfpathlineto{\pgfqpoint{1.653666in}{3.233927in}}%
\pgfpathmoveto{\pgfqpoint{1.653666in}{3.230978in}}%
\pgfpathlineto{\pgfqpoint{1.653666in}{3.230978in}}%
\pgfpathlineto{\pgfqpoint{1.653666in}{3.233927in}}%
\pgfpathlineto{\pgfqpoint{1.658206in}{3.233927in}}%
\pgfpathlineto{\pgfqpoint{1.658206in}{3.230978in}}%
\pgfpathmoveto{\pgfqpoint{1.653666in}{3.233927in}}%
\pgfpathlineto{\pgfqpoint{1.653666in}{3.233927in}}%
\pgfpathlineto{\pgfqpoint{1.653666in}{3.236876in}}%
\pgfpathlineto{\pgfqpoint{1.658206in}{3.236876in}}%
\pgfpathlineto{\pgfqpoint{1.658206in}{3.233927in}}%
\pgfpathmoveto{\pgfqpoint{1.676370in}{3.207383in}}%
\pgfpathlineto{\pgfqpoint{1.676370in}{3.207383in}}%
\pgfpathlineto{\pgfqpoint{1.676370in}{3.210333in}}%
\pgfpathlineto{\pgfqpoint{1.680911in}{3.210333in}}%
\pgfpathlineto{\pgfqpoint{1.680911in}{3.207383in}}%
\pgfpathmoveto{\pgfqpoint{1.676370in}{3.210333in}}%
\pgfpathlineto{\pgfqpoint{1.676370in}{3.210333in}}%
\pgfpathlineto{\pgfqpoint{1.676370in}{3.213282in}}%
\pgfpathlineto{\pgfqpoint{1.680911in}{3.213282in}}%
\pgfpathlineto{\pgfqpoint{1.680911in}{3.210333in}}%
\pgfpathmoveto{\pgfqpoint{1.680911in}{3.207383in}}%
\pgfpathlineto{\pgfqpoint{1.680911in}{3.207383in}}%
\pgfpathlineto{\pgfqpoint{1.680911in}{3.210333in}}%
\pgfpathlineto{\pgfqpoint{1.685452in}{3.210333in}}%
\pgfpathlineto{\pgfqpoint{1.685452in}{3.207383in}}%
\pgfpathmoveto{\pgfqpoint{1.680911in}{3.210333in}}%
\pgfpathlineto{\pgfqpoint{1.680911in}{3.210333in}}%
\pgfpathlineto{\pgfqpoint{1.680911in}{3.213282in}}%
\pgfpathlineto{\pgfqpoint{1.685452in}{3.213282in}}%
\pgfpathlineto{\pgfqpoint{1.685452in}{3.210333in}}%
\pgfpathmoveto{\pgfqpoint{1.685452in}{3.201485in}}%
\pgfpathlineto{\pgfqpoint{1.685452in}{3.201485in}}%
\pgfpathlineto{\pgfqpoint{1.685452in}{3.204434in}}%
\pgfpathlineto{\pgfqpoint{1.689993in}{3.204434in}}%
\pgfpathlineto{\pgfqpoint{1.689993in}{3.201485in}}%
\pgfpathmoveto{\pgfqpoint{1.685452in}{3.204434in}}%
\pgfpathlineto{\pgfqpoint{1.685452in}{3.204434in}}%
\pgfpathlineto{\pgfqpoint{1.685452in}{3.207383in}}%
\pgfpathlineto{\pgfqpoint{1.689993in}{3.207383in}}%
\pgfpathlineto{\pgfqpoint{1.689993in}{3.204434in}}%
\pgfpathmoveto{\pgfqpoint{1.689993in}{3.201485in}}%
\pgfpathlineto{\pgfqpoint{1.689993in}{3.201485in}}%
\pgfpathlineto{\pgfqpoint{1.689993in}{3.204434in}}%
\pgfpathlineto{\pgfqpoint{1.694534in}{3.204434in}}%
\pgfpathlineto{\pgfqpoint{1.694534in}{3.201485in}}%
\pgfpathmoveto{\pgfqpoint{1.689993in}{3.204434in}}%
\pgfpathlineto{\pgfqpoint{1.689993in}{3.204434in}}%
\pgfpathlineto{\pgfqpoint{1.689993in}{3.207383in}}%
\pgfpathlineto{\pgfqpoint{1.694534in}{3.207383in}}%
\pgfpathlineto{\pgfqpoint{1.694534in}{3.204434in}}%
\pgfpathmoveto{\pgfqpoint{1.685452in}{3.207383in}}%
\pgfpathlineto{\pgfqpoint{1.685452in}{3.207383in}}%
\pgfpathlineto{\pgfqpoint{1.685452in}{3.210333in}}%
\pgfpathlineto{\pgfqpoint{1.689993in}{3.210333in}}%
\pgfpathlineto{\pgfqpoint{1.689993in}{3.207383in}}%
\pgfpathmoveto{\pgfqpoint{1.685452in}{3.210333in}}%
\pgfpathlineto{\pgfqpoint{1.685452in}{3.210333in}}%
\pgfpathlineto{\pgfqpoint{1.685452in}{3.213282in}}%
\pgfpathlineto{\pgfqpoint{1.689993in}{3.213282in}}%
\pgfpathlineto{\pgfqpoint{1.689993in}{3.210333in}}%
\pgfpathmoveto{\pgfqpoint{1.689993in}{3.207383in}}%
\pgfpathlineto{\pgfqpoint{1.689993in}{3.207383in}}%
\pgfpathlineto{\pgfqpoint{1.689993in}{3.210333in}}%
\pgfpathlineto{\pgfqpoint{1.694534in}{3.210333in}}%
\pgfpathlineto{\pgfqpoint{1.694534in}{3.207383in}}%
\pgfpathmoveto{\pgfqpoint{1.667288in}{3.219181in}}%
\pgfpathlineto{\pgfqpoint{1.667288in}{3.219181in}}%
\pgfpathlineto{\pgfqpoint{1.667288in}{3.222130in}}%
\pgfpathlineto{\pgfqpoint{1.671829in}{3.222130in}}%
\pgfpathlineto{\pgfqpoint{1.671829in}{3.219181in}}%
\pgfpathmoveto{\pgfqpoint{1.667288in}{3.222130in}}%
\pgfpathlineto{\pgfqpoint{1.667288in}{3.222130in}}%
\pgfpathlineto{\pgfqpoint{1.667288in}{3.225079in}}%
\pgfpathlineto{\pgfqpoint{1.671829in}{3.225079in}}%
\pgfpathlineto{\pgfqpoint{1.671829in}{3.222130in}}%
\pgfpathmoveto{\pgfqpoint{1.671829in}{3.219181in}}%
\pgfpathlineto{\pgfqpoint{1.671829in}{3.219181in}}%
\pgfpathlineto{\pgfqpoint{1.671829in}{3.222130in}}%
\pgfpathlineto{\pgfqpoint{1.676370in}{3.222130in}}%
\pgfpathlineto{\pgfqpoint{1.676370in}{3.219181in}}%
\pgfpathmoveto{\pgfqpoint{1.671829in}{3.222130in}}%
\pgfpathlineto{\pgfqpoint{1.671829in}{3.222130in}}%
\pgfpathlineto{\pgfqpoint{1.671829in}{3.225079in}}%
\pgfpathlineto{\pgfqpoint{1.676370in}{3.225079in}}%
\pgfpathlineto{\pgfqpoint{1.676370in}{3.222130in}}%
\pgfpathmoveto{\pgfqpoint{1.658206in}{3.225079in}}%
\pgfpathlineto{\pgfqpoint{1.658206in}{3.225079in}}%
\pgfpathlineto{\pgfqpoint{1.658206in}{3.228028in}}%
\pgfpathlineto{\pgfqpoint{1.662747in}{3.228028in}}%
\pgfpathlineto{\pgfqpoint{1.662747in}{3.225079in}}%
\pgfpathmoveto{\pgfqpoint{1.658206in}{3.228028in}}%
\pgfpathlineto{\pgfqpoint{1.658206in}{3.228028in}}%
\pgfpathlineto{\pgfqpoint{1.658206in}{3.230978in}}%
\pgfpathlineto{\pgfqpoint{1.662747in}{3.230978in}}%
\pgfpathlineto{\pgfqpoint{1.662747in}{3.228028in}}%
\pgfpathmoveto{\pgfqpoint{1.662747in}{3.225079in}}%
\pgfpathlineto{\pgfqpoint{1.662747in}{3.225079in}}%
\pgfpathlineto{\pgfqpoint{1.662747in}{3.228028in}}%
\pgfpathlineto{\pgfqpoint{1.667288in}{3.228028in}}%
\pgfpathlineto{\pgfqpoint{1.667288in}{3.225079in}}%
\pgfpathmoveto{\pgfqpoint{1.662747in}{3.228028in}}%
\pgfpathlineto{\pgfqpoint{1.662747in}{3.228028in}}%
\pgfpathlineto{\pgfqpoint{1.662747in}{3.230978in}}%
\pgfpathlineto{\pgfqpoint{1.667288in}{3.230978in}}%
\pgfpathlineto{\pgfqpoint{1.667288in}{3.228028in}}%
\pgfpathmoveto{\pgfqpoint{1.658206in}{3.230978in}}%
\pgfpathlineto{\pgfqpoint{1.658206in}{3.230978in}}%
\pgfpathlineto{\pgfqpoint{1.658206in}{3.233927in}}%
\pgfpathlineto{\pgfqpoint{1.662747in}{3.233927in}}%
\pgfpathlineto{\pgfqpoint{1.662747in}{3.230978in}}%
\pgfpathmoveto{\pgfqpoint{1.658206in}{3.233927in}}%
\pgfpathlineto{\pgfqpoint{1.658206in}{3.233927in}}%
\pgfpathlineto{\pgfqpoint{1.658206in}{3.236876in}}%
\pgfpathlineto{\pgfqpoint{1.662747in}{3.236876in}}%
\pgfpathlineto{\pgfqpoint{1.662747in}{3.233927in}}%
\pgfpathmoveto{\pgfqpoint{1.662747in}{3.230978in}}%
\pgfpathlineto{\pgfqpoint{1.662747in}{3.230978in}}%
\pgfpathlineto{\pgfqpoint{1.662747in}{3.233927in}}%
\pgfpathlineto{\pgfqpoint{1.667288in}{3.233927in}}%
\pgfpathlineto{\pgfqpoint{1.667288in}{3.230978in}}%
\pgfpathmoveto{\pgfqpoint{1.667288in}{3.225079in}}%
\pgfpathlineto{\pgfqpoint{1.667288in}{3.225079in}}%
\pgfpathlineto{\pgfqpoint{1.667288in}{3.228028in}}%
\pgfpathlineto{\pgfqpoint{1.671829in}{3.228028in}}%
\pgfpathlineto{\pgfqpoint{1.671829in}{3.225079in}}%
\pgfpathmoveto{\pgfqpoint{1.667288in}{3.228028in}}%
\pgfpathlineto{\pgfqpoint{1.667288in}{3.228028in}}%
\pgfpathlineto{\pgfqpoint{1.667288in}{3.230978in}}%
\pgfpathlineto{\pgfqpoint{1.671829in}{3.230978in}}%
\pgfpathlineto{\pgfqpoint{1.671829in}{3.228028in}}%
\pgfpathmoveto{\pgfqpoint{1.676370in}{3.213282in}}%
\pgfpathlineto{\pgfqpoint{1.676370in}{3.213282in}}%
\pgfpathlineto{\pgfqpoint{1.676370in}{3.216231in}}%
\pgfpathlineto{\pgfqpoint{1.680911in}{3.216231in}}%
\pgfpathlineto{\pgfqpoint{1.680911in}{3.213282in}}%
\pgfpathmoveto{\pgfqpoint{1.676370in}{3.216231in}}%
\pgfpathlineto{\pgfqpoint{1.676370in}{3.216231in}}%
\pgfpathlineto{\pgfqpoint{1.676370in}{3.219181in}}%
\pgfpathlineto{\pgfqpoint{1.680911in}{3.219181in}}%
\pgfpathlineto{\pgfqpoint{1.680911in}{3.216231in}}%
\pgfpathmoveto{\pgfqpoint{1.680911in}{3.213282in}}%
\pgfpathlineto{\pgfqpoint{1.680911in}{3.213282in}}%
\pgfpathlineto{\pgfqpoint{1.680911in}{3.216231in}}%
\pgfpathlineto{\pgfqpoint{1.685452in}{3.216231in}}%
\pgfpathlineto{\pgfqpoint{1.685452in}{3.213282in}}%
\pgfpathmoveto{\pgfqpoint{1.680911in}{3.216231in}}%
\pgfpathlineto{\pgfqpoint{1.680911in}{3.216231in}}%
\pgfpathlineto{\pgfqpoint{1.680911in}{3.219181in}}%
\pgfpathlineto{\pgfqpoint{1.685452in}{3.219181in}}%
\pgfpathlineto{\pgfqpoint{1.685452in}{3.216231in}}%
\pgfpathmoveto{\pgfqpoint{1.676370in}{3.219181in}}%
\pgfpathlineto{\pgfqpoint{1.676370in}{3.219181in}}%
\pgfpathlineto{\pgfqpoint{1.676370in}{3.222130in}}%
\pgfpathlineto{\pgfqpoint{1.680911in}{3.222130in}}%
\pgfpathlineto{\pgfqpoint{1.680911in}{3.219181in}}%
\pgfpathmoveto{\pgfqpoint{1.703616in}{3.183789in}}%
\pgfpathlineto{\pgfqpoint{1.703616in}{3.183789in}}%
\pgfpathlineto{\pgfqpoint{1.703616in}{3.186738in}}%
\pgfpathlineto{\pgfqpoint{1.708157in}{3.186738in}}%
\pgfpathlineto{\pgfqpoint{1.708157in}{3.183789in}}%
\pgfpathmoveto{\pgfqpoint{1.703616in}{3.186738in}}%
\pgfpathlineto{\pgfqpoint{1.703616in}{3.186738in}}%
\pgfpathlineto{\pgfqpoint{1.703616in}{3.189687in}}%
\pgfpathlineto{\pgfqpoint{1.708157in}{3.189687in}}%
\pgfpathlineto{\pgfqpoint{1.708157in}{3.186738in}}%
\pgfpathmoveto{\pgfqpoint{1.708157in}{3.183789in}}%
\pgfpathlineto{\pgfqpoint{1.708157in}{3.183789in}}%
\pgfpathlineto{\pgfqpoint{1.708157in}{3.186738in}}%
\pgfpathlineto{\pgfqpoint{1.712698in}{3.186738in}}%
\pgfpathlineto{\pgfqpoint{1.712698in}{3.183789in}}%
\pgfpathmoveto{\pgfqpoint{1.708157in}{3.186738in}}%
\pgfpathlineto{\pgfqpoint{1.708157in}{3.186738in}}%
\pgfpathlineto{\pgfqpoint{1.708157in}{3.189687in}}%
\pgfpathlineto{\pgfqpoint{1.712698in}{3.189687in}}%
\pgfpathlineto{\pgfqpoint{1.712698in}{3.186738in}}%
\pgfpathmoveto{\pgfqpoint{1.721780in}{3.171992in}}%
\pgfpathlineto{\pgfqpoint{1.721780in}{3.171992in}}%
\pgfpathlineto{\pgfqpoint{1.721780in}{3.174941in}}%
\pgfpathlineto{\pgfqpoint{1.726320in}{3.174941in}}%
\pgfpathlineto{\pgfqpoint{1.726320in}{3.171992in}}%
\pgfpathmoveto{\pgfqpoint{1.721780in}{3.174941in}}%
\pgfpathlineto{\pgfqpoint{1.721780in}{3.174941in}}%
\pgfpathlineto{\pgfqpoint{1.721780in}{3.177890in}}%
\pgfpathlineto{\pgfqpoint{1.726320in}{3.177890in}}%
\pgfpathlineto{\pgfqpoint{1.726320in}{3.174941in}}%
\pgfpathmoveto{\pgfqpoint{1.726320in}{3.171992in}}%
\pgfpathlineto{\pgfqpoint{1.726320in}{3.171992in}}%
\pgfpathlineto{\pgfqpoint{1.726320in}{3.174941in}}%
\pgfpathlineto{\pgfqpoint{1.730861in}{3.174941in}}%
\pgfpathlineto{\pgfqpoint{1.730861in}{3.171992in}}%
\pgfpathmoveto{\pgfqpoint{1.726320in}{3.174941in}}%
\pgfpathlineto{\pgfqpoint{1.726320in}{3.174941in}}%
\pgfpathlineto{\pgfqpoint{1.726320in}{3.177890in}}%
\pgfpathlineto{\pgfqpoint{1.730861in}{3.177890in}}%
\pgfpathlineto{\pgfqpoint{1.730861in}{3.174941in}}%
\pgfpathmoveto{\pgfqpoint{1.712698in}{3.177890in}}%
\pgfpathlineto{\pgfqpoint{1.712698in}{3.177890in}}%
\pgfpathlineto{\pgfqpoint{1.712698in}{3.180839in}}%
\pgfpathlineto{\pgfqpoint{1.717239in}{3.180839in}}%
\pgfpathlineto{\pgfqpoint{1.717239in}{3.177890in}}%
\pgfpathmoveto{\pgfqpoint{1.712698in}{3.180839in}}%
\pgfpathlineto{\pgfqpoint{1.712698in}{3.180839in}}%
\pgfpathlineto{\pgfqpoint{1.712698in}{3.183789in}}%
\pgfpathlineto{\pgfqpoint{1.717239in}{3.183789in}}%
\pgfpathlineto{\pgfqpoint{1.717239in}{3.180839in}}%
\pgfpathmoveto{\pgfqpoint{1.717239in}{3.177890in}}%
\pgfpathlineto{\pgfqpoint{1.717239in}{3.177890in}}%
\pgfpathlineto{\pgfqpoint{1.717239in}{3.180839in}}%
\pgfpathlineto{\pgfqpoint{1.721780in}{3.180839in}}%
\pgfpathlineto{\pgfqpoint{1.721780in}{3.177890in}}%
\pgfpathmoveto{\pgfqpoint{1.717239in}{3.180839in}}%
\pgfpathlineto{\pgfqpoint{1.717239in}{3.180839in}}%
\pgfpathlineto{\pgfqpoint{1.717239in}{3.183789in}}%
\pgfpathlineto{\pgfqpoint{1.721780in}{3.183789in}}%
\pgfpathlineto{\pgfqpoint{1.721780in}{3.180839in}}%
\pgfpathmoveto{\pgfqpoint{1.712698in}{3.183789in}}%
\pgfpathlineto{\pgfqpoint{1.712698in}{3.183789in}}%
\pgfpathlineto{\pgfqpoint{1.712698in}{3.186738in}}%
\pgfpathlineto{\pgfqpoint{1.717239in}{3.186738in}}%
\pgfpathlineto{\pgfqpoint{1.717239in}{3.183789in}}%
\pgfpathmoveto{\pgfqpoint{1.712698in}{3.186738in}}%
\pgfpathlineto{\pgfqpoint{1.712698in}{3.186738in}}%
\pgfpathlineto{\pgfqpoint{1.712698in}{3.189687in}}%
\pgfpathlineto{\pgfqpoint{1.717239in}{3.189687in}}%
\pgfpathlineto{\pgfqpoint{1.717239in}{3.186738in}}%
\pgfpathmoveto{\pgfqpoint{1.717239in}{3.183789in}}%
\pgfpathlineto{\pgfqpoint{1.717239in}{3.183789in}}%
\pgfpathlineto{\pgfqpoint{1.717239in}{3.186738in}}%
\pgfpathlineto{\pgfqpoint{1.721780in}{3.186738in}}%
\pgfpathlineto{\pgfqpoint{1.721780in}{3.183789in}}%
\pgfpathmoveto{\pgfqpoint{1.721780in}{3.177890in}}%
\pgfpathlineto{\pgfqpoint{1.721780in}{3.177890in}}%
\pgfpathlineto{\pgfqpoint{1.721780in}{3.180839in}}%
\pgfpathlineto{\pgfqpoint{1.726320in}{3.180839in}}%
\pgfpathlineto{\pgfqpoint{1.726320in}{3.177890in}}%
\pgfpathmoveto{\pgfqpoint{1.721780in}{3.180839in}}%
\pgfpathlineto{\pgfqpoint{1.721780in}{3.180839in}}%
\pgfpathlineto{\pgfqpoint{1.721780in}{3.183789in}}%
\pgfpathlineto{\pgfqpoint{1.726320in}{3.183789in}}%
\pgfpathlineto{\pgfqpoint{1.726320in}{3.180839in}}%
\pgfpathmoveto{\pgfqpoint{1.730861in}{3.160194in}}%
\pgfpathlineto{\pgfqpoint{1.730861in}{3.160194in}}%
\pgfpathlineto{\pgfqpoint{1.730861in}{3.163144in}}%
\pgfpathlineto{\pgfqpoint{1.735402in}{3.163144in}}%
\pgfpathlineto{\pgfqpoint{1.735402in}{3.160194in}}%
\pgfpathmoveto{\pgfqpoint{1.730861in}{3.163144in}}%
\pgfpathlineto{\pgfqpoint{1.730861in}{3.163144in}}%
\pgfpathlineto{\pgfqpoint{1.730861in}{3.166093in}}%
\pgfpathlineto{\pgfqpoint{1.735402in}{3.166093in}}%
\pgfpathlineto{\pgfqpoint{1.735402in}{3.163144in}}%
\pgfpathmoveto{\pgfqpoint{1.735402in}{3.160194in}}%
\pgfpathlineto{\pgfqpoint{1.735402in}{3.160194in}}%
\pgfpathlineto{\pgfqpoint{1.735402in}{3.163144in}}%
\pgfpathlineto{\pgfqpoint{1.739943in}{3.163144in}}%
\pgfpathlineto{\pgfqpoint{1.739943in}{3.160194in}}%
\pgfpathmoveto{\pgfqpoint{1.735402in}{3.163144in}}%
\pgfpathlineto{\pgfqpoint{1.735402in}{3.163144in}}%
\pgfpathlineto{\pgfqpoint{1.735402in}{3.166093in}}%
\pgfpathlineto{\pgfqpoint{1.739943in}{3.166093in}}%
\pgfpathlineto{\pgfqpoint{1.739943in}{3.163144in}}%
\pgfpathmoveto{\pgfqpoint{1.739943in}{3.154296in}}%
\pgfpathlineto{\pgfqpoint{1.739943in}{3.154296in}}%
\pgfpathlineto{\pgfqpoint{1.739943in}{3.157245in}}%
\pgfpathlineto{\pgfqpoint{1.744484in}{3.157245in}}%
\pgfpathlineto{\pgfqpoint{1.744484in}{3.154296in}}%
\pgfpathmoveto{\pgfqpoint{1.739943in}{3.157245in}}%
\pgfpathlineto{\pgfqpoint{1.739943in}{3.157245in}}%
\pgfpathlineto{\pgfqpoint{1.739943in}{3.160194in}}%
\pgfpathlineto{\pgfqpoint{1.744484in}{3.160194in}}%
\pgfpathlineto{\pgfqpoint{1.744484in}{3.157245in}}%
\pgfpathmoveto{\pgfqpoint{1.744484in}{3.154296in}}%
\pgfpathlineto{\pgfqpoint{1.744484in}{3.154296in}}%
\pgfpathlineto{\pgfqpoint{1.744484in}{3.157245in}}%
\pgfpathlineto{\pgfqpoint{1.749025in}{3.157245in}}%
\pgfpathlineto{\pgfqpoint{1.749025in}{3.154296in}}%
\pgfpathmoveto{\pgfqpoint{1.744484in}{3.157245in}}%
\pgfpathlineto{\pgfqpoint{1.744484in}{3.157245in}}%
\pgfpathlineto{\pgfqpoint{1.744484in}{3.160194in}}%
\pgfpathlineto{\pgfqpoint{1.749025in}{3.160194in}}%
\pgfpathlineto{\pgfqpoint{1.749025in}{3.157245in}}%
\pgfpathmoveto{\pgfqpoint{1.739943in}{3.160194in}}%
\pgfpathlineto{\pgfqpoint{1.739943in}{3.160194in}}%
\pgfpathlineto{\pgfqpoint{1.739943in}{3.163144in}}%
\pgfpathlineto{\pgfqpoint{1.744484in}{3.163144in}}%
\pgfpathlineto{\pgfqpoint{1.744484in}{3.160194in}}%
\pgfpathmoveto{\pgfqpoint{1.739943in}{3.163144in}}%
\pgfpathlineto{\pgfqpoint{1.739943in}{3.163144in}}%
\pgfpathlineto{\pgfqpoint{1.739943in}{3.166093in}}%
\pgfpathlineto{\pgfqpoint{1.744484in}{3.166093in}}%
\pgfpathlineto{\pgfqpoint{1.744484in}{3.163144in}}%
\pgfpathmoveto{\pgfqpoint{1.744484in}{3.160194in}}%
\pgfpathlineto{\pgfqpoint{1.744484in}{3.160194in}}%
\pgfpathlineto{\pgfqpoint{1.744484in}{3.163144in}}%
\pgfpathlineto{\pgfqpoint{1.749025in}{3.163144in}}%
\pgfpathlineto{\pgfqpoint{1.749025in}{3.160194in}}%
\pgfpathmoveto{\pgfqpoint{1.749025in}{3.148397in}}%
\pgfpathlineto{\pgfqpoint{1.749025in}{3.148397in}}%
\pgfpathlineto{\pgfqpoint{1.749025in}{3.151346in}}%
\pgfpathlineto{\pgfqpoint{1.753566in}{3.151346in}}%
\pgfpathlineto{\pgfqpoint{1.753566in}{3.148397in}}%
\pgfpathmoveto{\pgfqpoint{1.749025in}{3.151346in}}%
\pgfpathlineto{\pgfqpoint{1.749025in}{3.151346in}}%
\pgfpathlineto{\pgfqpoint{1.749025in}{3.154296in}}%
\pgfpathlineto{\pgfqpoint{1.753566in}{3.154296in}}%
\pgfpathlineto{\pgfqpoint{1.753566in}{3.151346in}}%
\pgfpathmoveto{\pgfqpoint{1.753566in}{3.148397in}}%
\pgfpathlineto{\pgfqpoint{1.753566in}{3.148397in}}%
\pgfpathlineto{\pgfqpoint{1.753566in}{3.151346in}}%
\pgfpathlineto{\pgfqpoint{1.758107in}{3.151346in}}%
\pgfpathlineto{\pgfqpoint{1.758107in}{3.148397in}}%
\pgfpathmoveto{\pgfqpoint{1.753566in}{3.151346in}}%
\pgfpathlineto{\pgfqpoint{1.753566in}{3.151346in}}%
\pgfpathlineto{\pgfqpoint{1.753566in}{3.154296in}}%
\pgfpathlineto{\pgfqpoint{1.758107in}{3.154296in}}%
\pgfpathlineto{\pgfqpoint{1.758107in}{3.151346in}}%
\pgfpathmoveto{\pgfqpoint{1.758107in}{3.142498in}}%
\pgfpathlineto{\pgfqpoint{1.758107in}{3.142498in}}%
\pgfpathlineto{\pgfqpoint{1.758107in}{3.145448in}}%
\pgfpathlineto{\pgfqpoint{1.762648in}{3.145448in}}%
\pgfpathlineto{\pgfqpoint{1.762648in}{3.142498in}}%
\pgfpathmoveto{\pgfqpoint{1.758107in}{3.145448in}}%
\pgfpathlineto{\pgfqpoint{1.758107in}{3.145448in}}%
\pgfpathlineto{\pgfqpoint{1.758107in}{3.148397in}}%
\pgfpathlineto{\pgfqpoint{1.762648in}{3.148397in}}%
\pgfpathlineto{\pgfqpoint{1.762648in}{3.145448in}}%
\pgfpathmoveto{\pgfqpoint{1.762648in}{3.142498in}}%
\pgfpathlineto{\pgfqpoint{1.762648in}{3.142498in}}%
\pgfpathlineto{\pgfqpoint{1.762648in}{3.145448in}}%
\pgfpathlineto{\pgfqpoint{1.767189in}{3.145448in}}%
\pgfpathlineto{\pgfqpoint{1.767189in}{3.142498in}}%
\pgfpathmoveto{\pgfqpoint{1.762648in}{3.145448in}}%
\pgfpathlineto{\pgfqpoint{1.762648in}{3.145448in}}%
\pgfpathlineto{\pgfqpoint{1.762648in}{3.148397in}}%
\pgfpathlineto{\pgfqpoint{1.767189in}{3.148397in}}%
\pgfpathlineto{\pgfqpoint{1.767189in}{3.145448in}}%
\pgfpathmoveto{\pgfqpoint{1.758107in}{3.148397in}}%
\pgfpathlineto{\pgfqpoint{1.758107in}{3.148397in}}%
\pgfpathlineto{\pgfqpoint{1.758107in}{3.151346in}}%
\pgfpathlineto{\pgfqpoint{1.762648in}{3.151346in}}%
\pgfpathlineto{\pgfqpoint{1.762648in}{3.148397in}}%
\pgfpathmoveto{\pgfqpoint{1.749025in}{3.154296in}}%
\pgfpathlineto{\pgfqpoint{1.749025in}{3.154296in}}%
\pgfpathlineto{\pgfqpoint{1.749025in}{3.157245in}}%
\pgfpathlineto{\pgfqpoint{1.753566in}{3.157245in}}%
\pgfpathlineto{\pgfqpoint{1.753566in}{3.154296in}}%
\pgfpathmoveto{\pgfqpoint{1.749025in}{3.157245in}}%
\pgfpathlineto{\pgfqpoint{1.749025in}{3.157245in}}%
\pgfpathlineto{\pgfqpoint{1.749025in}{3.160194in}}%
\pgfpathlineto{\pgfqpoint{1.753566in}{3.160194in}}%
\pgfpathlineto{\pgfqpoint{1.753566in}{3.157245in}}%
\pgfpathmoveto{\pgfqpoint{1.730861in}{3.166093in}}%
\pgfpathlineto{\pgfqpoint{1.730861in}{3.166093in}}%
\pgfpathlineto{\pgfqpoint{1.730861in}{3.169042in}}%
\pgfpathlineto{\pgfqpoint{1.735402in}{3.169042in}}%
\pgfpathlineto{\pgfqpoint{1.735402in}{3.166093in}}%
\pgfpathmoveto{\pgfqpoint{1.730861in}{3.169042in}}%
\pgfpathlineto{\pgfqpoint{1.730861in}{3.169042in}}%
\pgfpathlineto{\pgfqpoint{1.730861in}{3.171992in}}%
\pgfpathlineto{\pgfqpoint{1.735402in}{3.171992in}}%
\pgfpathlineto{\pgfqpoint{1.735402in}{3.169042in}}%
\pgfpathmoveto{\pgfqpoint{1.735402in}{3.166093in}}%
\pgfpathlineto{\pgfqpoint{1.735402in}{3.166093in}}%
\pgfpathlineto{\pgfqpoint{1.735402in}{3.169042in}}%
\pgfpathlineto{\pgfqpoint{1.739943in}{3.169042in}}%
\pgfpathlineto{\pgfqpoint{1.739943in}{3.166093in}}%
\pgfpathmoveto{\pgfqpoint{1.735402in}{3.169042in}}%
\pgfpathlineto{\pgfqpoint{1.735402in}{3.169042in}}%
\pgfpathlineto{\pgfqpoint{1.735402in}{3.171992in}}%
\pgfpathlineto{\pgfqpoint{1.739943in}{3.171992in}}%
\pgfpathlineto{\pgfqpoint{1.739943in}{3.169042in}}%
\pgfpathmoveto{\pgfqpoint{1.730861in}{3.171992in}}%
\pgfpathlineto{\pgfqpoint{1.730861in}{3.171992in}}%
\pgfpathlineto{\pgfqpoint{1.730861in}{3.174941in}}%
\pgfpathlineto{\pgfqpoint{1.735402in}{3.174941in}}%
\pgfpathlineto{\pgfqpoint{1.735402in}{3.171992in}}%
\pgfpathmoveto{\pgfqpoint{1.694534in}{3.195586in}}%
\pgfpathlineto{\pgfqpoint{1.694534in}{3.195586in}}%
\pgfpathlineto{\pgfqpoint{1.694534in}{3.198535in}}%
\pgfpathlineto{\pgfqpoint{1.699075in}{3.198535in}}%
\pgfpathlineto{\pgfqpoint{1.699075in}{3.195586in}}%
\pgfpathmoveto{\pgfqpoint{1.694534in}{3.198535in}}%
\pgfpathlineto{\pgfqpoint{1.694534in}{3.198535in}}%
\pgfpathlineto{\pgfqpoint{1.694534in}{3.201485in}}%
\pgfpathlineto{\pgfqpoint{1.699075in}{3.201485in}}%
\pgfpathlineto{\pgfqpoint{1.699075in}{3.198535in}}%
\pgfpathmoveto{\pgfqpoint{1.699075in}{3.195586in}}%
\pgfpathlineto{\pgfqpoint{1.699075in}{3.195586in}}%
\pgfpathlineto{\pgfqpoint{1.699075in}{3.198535in}}%
\pgfpathlineto{\pgfqpoint{1.703616in}{3.198535in}}%
\pgfpathlineto{\pgfqpoint{1.703616in}{3.195586in}}%
\pgfpathmoveto{\pgfqpoint{1.699075in}{3.198535in}}%
\pgfpathlineto{\pgfqpoint{1.699075in}{3.198535in}}%
\pgfpathlineto{\pgfqpoint{1.699075in}{3.201485in}}%
\pgfpathlineto{\pgfqpoint{1.703616in}{3.201485in}}%
\pgfpathlineto{\pgfqpoint{1.703616in}{3.198535in}}%
\pgfpathmoveto{\pgfqpoint{1.703616in}{3.189687in}}%
\pgfpathlineto{\pgfqpoint{1.703616in}{3.189687in}}%
\pgfpathlineto{\pgfqpoint{1.703616in}{3.192637in}}%
\pgfpathlineto{\pgfqpoint{1.708157in}{3.192637in}}%
\pgfpathlineto{\pgfqpoint{1.708157in}{3.189687in}}%
\pgfpathmoveto{\pgfqpoint{1.703616in}{3.192637in}}%
\pgfpathlineto{\pgfqpoint{1.703616in}{3.192637in}}%
\pgfpathlineto{\pgfqpoint{1.703616in}{3.195586in}}%
\pgfpathlineto{\pgfqpoint{1.708157in}{3.195586in}}%
\pgfpathlineto{\pgfqpoint{1.708157in}{3.192637in}}%
\pgfpathmoveto{\pgfqpoint{1.708157in}{3.189687in}}%
\pgfpathlineto{\pgfqpoint{1.708157in}{3.189687in}}%
\pgfpathlineto{\pgfqpoint{1.708157in}{3.192637in}}%
\pgfpathlineto{\pgfqpoint{1.712698in}{3.192637in}}%
\pgfpathlineto{\pgfqpoint{1.712698in}{3.189687in}}%
\pgfpathmoveto{\pgfqpoint{1.708157in}{3.192637in}}%
\pgfpathlineto{\pgfqpoint{1.708157in}{3.192637in}}%
\pgfpathlineto{\pgfqpoint{1.708157in}{3.195586in}}%
\pgfpathlineto{\pgfqpoint{1.712698in}{3.195586in}}%
\pgfpathlineto{\pgfqpoint{1.712698in}{3.192637in}}%
\pgfpathmoveto{\pgfqpoint{1.703616in}{3.195586in}}%
\pgfpathlineto{\pgfqpoint{1.703616in}{3.195586in}}%
\pgfpathlineto{\pgfqpoint{1.703616in}{3.198535in}}%
\pgfpathlineto{\pgfqpoint{1.708157in}{3.198535in}}%
\pgfpathlineto{\pgfqpoint{1.708157in}{3.195586in}}%
\pgfpathmoveto{\pgfqpoint{1.694534in}{3.201485in}}%
\pgfpathlineto{\pgfqpoint{1.694534in}{3.201485in}}%
\pgfpathlineto{\pgfqpoint{1.694534in}{3.204434in}}%
\pgfpathlineto{\pgfqpoint{1.699075in}{3.204434in}}%
\pgfpathlineto{\pgfqpoint{1.699075in}{3.201485in}}%
\pgfpathmoveto{\pgfqpoint{1.694534in}{3.204434in}}%
\pgfpathlineto{\pgfqpoint{1.694534in}{3.204434in}}%
\pgfpathlineto{\pgfqpoint{1.694534in}{3.207383in}}%
\pgfpathlineto{\pgfqpoint{1.699075in}{3.207383in}}%
\pgfpathlineto{\pgfqpoint{1.699075in}{3.204434in}}%
\pgfpathmoveto{\pgfqpoint{1.621879in}{3.254571in}}%
\pgfpathlineto{\pgfqpoint{1.621879in}{3.254571in}}%
\pgfpathlineto{\pgfqpoint{1.621879in}{3.257520in}}%
\pgfpathlineto{\pgfqpoint{1.626420in}{3.257520in}}%
\pgfpathlineto{\pgfqpoint{1.626420in}{3.254571in}}%
\pgfpathmoveto{\pgfqpoint{1.621879in}{3.257520in}}%
\pgfpathlineto{\pgfqpoint{1.621879in}{3.257520in}}%
\pgfpathlineto{\pgfqpoint{1.621879in}{3.260469in}}%
\pgfpathlineto{\pgfqpoint{1.626420in}{3.260469in}}%
\pgfpathlineto{\pgfqpoint{1.626420in}{3.257520in}}%
\pgfpathmoveto{\pgfqpoint{1.626420in}{3.254571in}}%
\pgfpathlineto{\pgfqpoint{1.626420in}{3.254571in}}%
\pgfpathlineto{\pgfqpoint{1.626420in}{3.257520in}}%
\pgfpathlineto{\pgfqpoint{1.630961in}{3.257520in}}%
\pgfpathlineto{\pgfqpoint{1.630961in}{3.254571in}}%
\pgfpathmoveto{\pgfqpoint{1.626420in}{3.257520in}}%
\pgfpathlineto{\pgfqpoint{1.626420in}{3.257520in}}%
\pgfpathlineto{\pgfqpoint{1.626420in}{3.260469in}}%
\pgfpathlineto{\pgfqpoint{1.630961in}{3.260469in}}%
\pgfpathlineto{\pgfqpoint{1.630961in}{3.257520in}}%
\pgfpathmoveto{\pgfqpoint{1.630961in}{3.248673in}}%
\pgfpathlineto{\pgfqpoint{1.630961in}{3.248673in}}%
\pgfpathlineto{\pgfqpoint{1.630961in}{3.251622in}}%
\pgfpathlineto{\pgfqpoint{1.635502in}{3.251622in}}%
\pgfpathlineto{\pgfqpoint{1.635502in}{3.248673in}}%
\pgfpathmoveto{\pgfqpoint{1.630961in}{3.251622in}}%
\pgfpathlineto{\pgfqpoint{1.630961in}{3.251622in}}%
\pgfpathlineto{\pgfqpoint{1.630961in}{3.254571in}}%
\pgfpathlineto{\pgfqpoint{1.635502in}{3.254571in}}%
\pgfpathlineto{\pgfqpoint{1.635502in}{3.251622in}}%
\pgfpathmoveto{\pgfqpoint{1.635502in}{3.248673in}}%
\pgfpathlineto{\pgfqpoint{1.635502in}{3.248673in}}%
\pgfpathlineto{\pgfqpoint{1.635502in}{3.251622in}}%
\pgfpathlineto{\pgfqpoint{1.640043in}{3.251622in}}%
\pgfpathlineto{\pgfqpoint{1.640043in}{3.248673in}}%
\pgfpathmoveto{\pgfqpoint{1.635502in}{3.251622in}}%
\pgfpathlineto{\pgfqpoint{1.635502in}{3.251622in}}%
\pgfpathlineto{\pgfqpoint{1.635502in}{3.254571in}}%
\pgfpathlineto{\pgfqpoint{1.640043in}{3.254571in}}%
\pgfpathlineto{\pgfqpoint{1.640043in}{3.251622in}}%
\pgfpathmoveto{\pgfqpoint{1.630961in}{3.254571in}}%
\pgfpathlineto{\pgfqpoint{1.630961in}{3.254571in}}%
\pgfpathlineto{\pgfqpoint{1.630961in}{3.257520in}}%
\pgfpathlineto{\pgfqpoint{1.635502in}{3.257520in}}%
\pgfpathlineto{\pgfqpoint{1.635502in}{3.254571in}}%
\pgfpathmoveto{\pgfqpoint{1.630961in}{3.257520in}}%
\pgfpathlineto{\pgfqpoint{1.630961in}{3.257520in}}%
\pgfpathlineto{\pgfqpoint{1.630961in}{3.260469in}}%
\pgfpathlineto{\pgfqpoint{1.635502in}{3.260469in}}%
\pgfpathlineto{\pgfqpoint{1.635502in}{3.257520in}}%
\pgfpathmoveto{\pgfqpoint{1.635502in}{3.254571in}}%
\pgfpathlineto{\pgfqpoint{1.635502in}{3.254571in}}%
\pgfpathlineto{\pgfqpoint{1.635502in}{3.257520in}}%
\pgfpathlineto{\pgfqpoint{1.640043in}{3.257520in}}%
\pgfpathlineto{\pgfqpoint{1.640043in}{3.254571in}}%
\pgfpathmoveto{\pgfqpoint{1.640043in}{3.242775in}}%
\pgfpathlineto{\pgfqpoint{1.640043in}{3.242775in}}%
\pgfpathlineto{\pgfqpoint{1.640043in}{3.245724in}}%
\pgfpathlineto{\pgfqpoint{1.644584in}{3.245724in}}%
\pgfpathlineto{\pgfqpoint{1.644584in}{3.242775in}}%
\pgfpathmoveto{\pgfqpoint{1.640043in}{3.245724in}}%
\pgfpathlineto{\pgfqpoint{1.640043in}{3.245724in}}%
\pgfpathlineto{\pgfqpoint{1.640043in}{3.248673in}}%
\pgfpathlineto{\pgfqpoint{1.644584in}{3.248673in}}%
\pgfpathlineto{\pgfqpoint{1.644584in}{3.245724in}}%
\pgfpathmoveto{\pgfqpoint{1.644584in}{3.242775in}}%
\pgfpathlineto{\pgfqpoint{1.644584in}{3.242775in}}%
\pgfpathlineto{\pgfqpoint{1.644584in}{3.245724in}}%
\pgfpathlineto{\pgfqpoint{1.649125in}{3.245724in}}%
\pgfpathlineto{\pgfqpoint{1.649125in}{3.242775in}}%
\pgfpathmoveto{\pgfqpoint{1.644584in}{3.245724in}}%
\pgfpathlineto{\pgfqpoint{1.644584in}{3.245724in}}%
\pgfpathlineto{\pgfqpoint{1.644584in}{3.248673in}}%
\pgfpathlineto{\pgfqpoint{1.649125in}{3.248673in}}%
\pgfpathlineto{\pgfqpoint{1.649125in}{3.245724in}}%
\pgfpathmoveto{\pgfqpoint{1.649125in}{3.236876in}}%
\pgfpathlineto{\pgfqpoint{1.649125in}{3.236876in}}%
\pgfpathlineto{\pgfqpoint{1.649125in}{3.239826in}}%
\pgfpathlineto{\pgfqpoint{1.653666in}{3.239826in}}%
\pgfpathlineto{\pgfqpoint{1.653666in}{3.236876in}}%
\pgfpathmoveto{\pgfqpoint{1.649125in}{3.239826in}}%
\pgfpathlineto{\pgfqpoint{1.649125in}{3.239826in}}%
\pgfpathlineto{\pgfqpoint{1.649125in}{3.242775in}}%
\pgfpathlineto{\pgfqpoint{1.653666in}{3.242775in}}%
\pgfpathlineto{\pgfqpoint{1.653666in}{3.239826in}}%
\pgfpathmoveto{\pgfqpoint{1.653666in}{3.236876in}}%
\pgfpathlineto{\pgfqpoint{1.653666in}{3.236876in}}%
\pgfpathlineto{\pgfqpoint{1.653666in}{3.239826in}}%
\pgfpathlineto{\pgfqpoint{1.658206in}{3.239826in}}%
\pgfpathlineto{\pgfqpoint{1.658206in}{3.236876in}}%
\pgfpathmoveto{\pgfqpoint{1.653666in}{3.239826in}}%
\pgfpathlineto{\pgfqpoint{1.653666in}{3.239826in}}%
\pgfpathlineto{\pgfqpoint{1.653666in}{3.242775in}}%
\pgfpathlineto{\pgfqpoint{1.658206in}{3.242775in}}%
\pgfpathlineto{\pgfqpoint{1.658206in}{3.239826in}}%
\pgfpathmoveto{\pgfqpoint{1.649125in}{3.242775in}}%
\pgfpathlineto{\pgfqpoint{1.649125in}{3.242775in}}%
\pgfpathlineto{\pgfqpoint{1.649125in}{3.245724in}}%
\pgfpathlineto{\pgfqpoint{1.653666in}{3.245724in}}%
\pgfpathlineto{\pgfqpoint{1.653666in}{3.242775in}}%
\pgfpathmoveto{\pgfqpoint{1.640043in}{3.248673in}}%
\pgfpathlineto{\pgfqpoint{1.640043in}{3.248673in}}%
\pgfpathlineto{\pgfqpoint{1.640043in}{3.251622in}}%
\pgfpathlineto{\pgfqpoint{1.644584in}{3.251622in}}%
\pgfpathlineto{\pgfqpoint{1.644584in}{3.248673in}}%
\pgfpathmoveto{\pgfqpoint{1.640043in}{3.251622in}}%
\pgfpathlineto{\pgfqpoint{1.640043in}{3.251622in}}%
\pgfpathlineto{\pgfqpoint{1.640043in}{3.254571in}}%
\pgfpathlineto{\pgfqpoint{1.644584in}{3.254571in}}%
\pgfpathlineto{\pgfqpoint{1.644584in}{3.251622in}}%
\pgfpathmoveto{\pgfqpoint{1.621879in}{3.260469in}}%
\pgfpathlineto{\pgfqpoint{1.621879in}{3.260469in}}%
\pgfpathlineto{\pgfqpoint{1.621879in}{3.263419in}}%
\pgfpathlineto{\pgfqpoint{1.626420in}{3.263419in}}%
\pgfpathlineto{\pgfqpoint{1.626420in}{3.260469in}}%
\pgfpathmoveto{\pgfqpoint{1.621879in}{3.263419in}}%
\pgfpathlineto{\pgfqpoint{1.621879in}{3.263419in}}%
\pgfpathlineto{\pgfqpoint{1.621879in}{3.266368in}}%
\pgfpathlineto{\pgfqpoint{1.626420in}{3.266368in}}%
\pgfpathlineto{\pgfqpoint{1.626420in}{3.263419in}}%
\pgfpathmoveto{\pgfqpoint{1.626420in}{3.260469in}}%
\pgfpathlineto{\pgfqpoint{1.626420in}{3.260469in}}%
\pgfpathlineto{\pgfqpoint{1.626420in}{3.263419in}}%
\pgfpathlineto{\pgfqpoint{1.630961in}{3.263419in}}%
\pgfpathlineto{\pgfqpoint{1.630961in}{3.260469in}}%
\pgfpathmoveto{\pgfqpoint{1.626420in}{3.263419in}}%
\pgfpathlineto{\pgfqpoint{1.626420in}{3.263419in}}%
\pgfpathlineto{\pgfqpoint{1.626420in}{3.266368in}}%
\pgfpathlineto{\pgfqpoint{1.630961in}{3.266368in}}%
\pgfpathlineto{\pgfqpoint{1.630961in}{3.263419in}}%
\pgfpathmoveto{\pgfqpoint{1.621879in}{3.266368in}}%
\pgfpathlineto{\pgfqpoint{1.621879in}{3.266368in}}%
\pgfpathlineto{\pgfqpoint{1.621879in}{3.269317in}}%
\pgfpathlineto{\pgfqpoint{1.626420in}{3.269317in}}%
\pgfpathlineto{\pgfqpoint{1.626420in}{3.266368in}}%
\pgfpathmoveto{\pgfqpoint{1.903417in}{3.012733in}}%
\pgfpathlineto{\pgfqpoint{1.903417in}{3.012733in}}%
\pgfpathlineto{\pgfqpoint{1.903417in}{3.015683in}}%
\pgfpathlineto{\pgfqpoint{1.907958in}{3.015683in}}%
\pgfpathlineto{\pgfqpoint{1.907958in}{3.012733in}}%
\pgfpathmoveto{\pgfqpoint{1.903417in}{3.015683in}}%
\pgfpathlineto{\pgfqpoint{1.903417in}{3.015683in}}%
\pgfpathlineto{\pgfqpoint{1.903417in}{3.018632in}}%
\pgfpathlineto{\pgfqpoint{1.907958in}{3.018632in}}%
\pgfpathlineto{\pgfqpoint{1.907958in}{3.015683in}}%
\pgfpathmoveto{\pgfqpoint{1.907958in}{3.012733in}}%
\pgfpathlineto{\pgfqpoint{1.907958in}{3.012733in}}%
\pgfpathlineto{\pgfqpoint{1.907958in}{3.015683in}}%
\pgfpathlineto{\pgfqpoint{1.912499in}{3.015683in}}%
\pgfpathlineto{\pgfqpoint{1.912499in}{3.012733in}}%
\pgfpathmoveto{\pgfqpoint{1.907958in}{3.015683in}}%
\pgfpathlineto{\pgfqpoint{1.907958in}{3.015683in}}%
\pgfpathlineto{\pgfqpoint{1.907958in}{3.018632in}}%
\pgfpathlineto{\pgfqpoint{1.912499in}{3.018632in}}%
\pgfpathlineto{\pgfqpoint{1.912499in}{3.015683in}}%
\pgfpathmoveto{\pgfqpoint{1.903417in}{3.018632in}}%
\pgfpathlineto{\pgfqpoint{1.903417in}{3.018632in}}%
\pgfpathlineto{\pgfqpoint{1.903417in}{3.021581in}}%
\pgfpathlineto{\pgfqpoint{1.907958in}{3.021581in}}%
\pgfpathlineto{\pgfqpoint{1.907958in}{3.018632in}}%
\pgfpathmoveto{\pgfqpoint{1.903417in}{3.021581in}}%
\pgfpathlineto{\pgfqpoint{1.903417in}{3.021581in}}%
\pgfpathlineto{\pgfqpoint{1.903417in}{3.024530in}}%
\pgfpathlineto{\pgfqpoint{1.907958in}{3.024530in}}%
\pgfpathlineto{\pgfqpoint{1.907958in}{3.021581in}}%
\pgfpathmoveto{\pgfqpoint{1.907958in}{3.018632in}}%
\pgfpathlineto{\pgfqpoint{1.907958in}{3.018632in}}%
\pgfpathlineto{\pgfqpoint{1.907958in}{3.021581in}}%
\pgfpathlineto{\pgfqpoint{1.912499in}{3.021581in}}%
\pgfpathlineto{\pgfqpoint{1.912499in}{3.018632in}}%
\pgfpathmoveto{\pgfqpoint{1.885253in}{3.030429in}}%
\pgfpathlineto{\pgfqpoint{1.885253in}{3.030429in}}%
\pgfpathlineto{\pgfqpoint{1.885253in}{3.033378in}}%
\pgfpathlineto{\pgfqpoint{1.889794in}{3.033378in}}%
\pgfpathlineto{\pgfqpoint{1.889794in}{3.030429in}}%
\pgfpathmoveto{\pgfqpoint{1.885253in}{3.033378in}}%
\pgfpathlineto{\pgfqpoint{1.885253in}{3.033378in}}%
\pgfpathlineto{\pgfqpoint{1.885253in}{3.036327in}}%
\pgfpathlineto{\pgfqpoint{1.889794in}{3.036327in}}%
\pgfpathlineto{\pgfqpoint{1.889794in}{3.033378in}}%
\pgfpathmoveto{\pgfqpoint{1.889794in}{3.030429in}}%
\pgfpathlineto{\pgfqpoint{1.889794in}{3.030429in}}%
\pgfpathlineto{\pgfqpoint{1.889794in}{3.033378in}}%
\pgfpathlineto{\pgfqpoint{1.894335in}{3.033378in}}%
\pgfpathlineto{\pgfqpoint{1.894335in}{3.030429in}}%
\pgfpathmoveto{\pgfqpoint{1.889794in}{3.033378in}}%
\pgfpathlineto{\pgfqpoint{1.889794in}{3.033378in}}%
\pgfpathlineto{\pgfqpoint{1.889794in}{3.036327in}}%
\pgfpathlineto{\pgfqpoint{1.894335in}{3.036327in}}%
\pgfpathlineto{\pgfqpoint{1.894335in}{3.033378in}}%
\pgfpathmoveto{\pgfqpoint{1.876171in}{3.036327in}}%
\pgfpathlineto{\pgfqpoint{1.876171in}{3.036327in}}%
\pgfpathlineto{\pgfqpoint{1.876171in}{3.039276in}}%
\pgfpathlineto{\pgfqpoint{1.880712in}{3.039276in}}%
\pgfpathlineto{\pgfqpoint{1.880712in}{3.036327in}}%
\pgfpathmoveto{\pgfqpoint{1.876171in}{3.039276in}}%
\pgfpathlineto{\pgfqpoint{1.876171in}{3.039276in}}%
\pgfpathlineto{\pgfqpoint{1.876171in}{3.042226in}}%
\pgfpathlineto{\pgfqpoint{1.880712in}{3.042226in}}%
\pgfpathlineto{\pgfqpoint{1.880712in}{3.039276in}}%
\pgfpathmoveto{\pgfqpoint{1.880712in}{3.036327in}}%
\pgfpathlineto{\pgfqpoint{1.880712in}{3.036327in}}%
\pgfpathlineto{\pgfqpoint{1.880712in}{3.039276in}}%
\pgfpathlineto{\pgfqpoint{1.885253in}{3.039276in}}%
\pgfpathlineto{\pgfqpoint{1.885253in}{3.036327in}}%
\pgfpathmoveto{\pgfqpoint{1.880712in}{3.039276in}}%
\pgfpathlineto{\pgfqpoint{1.880712in}{3.039276in}}%
\pgfpathlineto{\pgfqpoint{1.880712in}{3.042226in}}%
\pgfpathlineto{\pgfqpoint{1.885253in}{3.042226in}}%
\pgfpathlineto{\pgfqpoint{1.885253in}{3.039276in}}%
\pgfpathmoveto{\pgfqpoint{1.876171in}{3.042226in}}%
\pgfpathlineto{\pgfqpoint{1.876171in}{3.042226in}}%
\pgfpathlineto{\pgfqpoint{1.876171in}{3.045175in}}%
\pgfpathlineto{\pgfqpoint{1.880712in}{3.045175in}}%
\pgfpathlineto{\pgfqpoint{1.880712in}{3.042226in}}%
\pgfpathmoveto{\pgfqpoint{1.876171in}{3.045175in}}%
\pgfpathlineto{\pgfqpoint{1.876171in}{3.045175in}}%
\pgfpathlineto{\pgfqpoint{1.876171in}{3.048124in}}%
\pgfpathlineto{\pgfqpoint{1.880712in}{3.048124in}}%
\pgfpathlineto{\pgfqpoint{1.880712in}{3.045175in}}%
\pgfpathmoveto{\pgfqpoint{1.880712in}{3.042226in}}%
\pgfpathlineto{\pgfqpoint{1.880712in}{3.042226in}}%
\pgfpathlineto{\pgfqpoint{1.880712in}{3.045175in}}%
\pgfpathlineto{\pgfqpoint{1.885253in}{3.045175in}}%
\pgfpathlineto{\pgfqpoint{1.885253in}{3.042226in}}%
\pgfpathmoveto{\pgfqpoint{1.885253in}{3.036327in}}%
\pgfpathlineto{\pgfqpoint{1.885253in}{3.036327in}}%
\pgfpathlineto{\pgfqpoint{1.885253in}{3.039276in}}%
\pgfpathlineto{\pgfqpoint{1.889794in}{3.039276in}}%
\pgfpathlineto{\pgfqpoint{1.889794in}{3.036327in}}%
\pgfpathmoveto{\pgfqpoint{1.885253in}{3.039276in}}%
\pgfpathlineto{\pgfqpoint{1.885253in}{3.039276in}}%
\pgfpathlineto{\pgfqpoint{1.885253in}{3.042226in}}%
\pgfpathlineto{\pgfqpoint{1.889794in}{3.042226in}}%
\pgfpathlineto{\pgfqpoint{1.889794in}{3.039276in}}%
\pgfpathmoveto{\pgfqpoint{1.889794in}{3.036327in}}%
\pgfpathlineto{\pgfqpoint{1.889794in}{3.036327in}}%
\pgfpathlineto{\pgfqpoint{1.889794in}{3.039276in}}%
\pgfpathlineto{\pgfqpoint{1.894335in}{3.039276in}}%
\pgfpathlineto{\pgfqpoint{1.894335in}{3.036327in}}%
\pgfpathmoveto{\pgfqpoint{1.894335in}{3.024530in}}%
\pgfpathlineto{\pgfqpoint{1.894335in}{3.024530in}}%
\pgfpathlineto{\pgfqpoint{1.894335in}{3.027480in}}%
\pgfpathlineto{\pgfqpoint{1.898876in}{3.027480in}}%
\pgfpathlineto{\pgfqpoint{1.898876in}{3.024530in}}%
\pgfpathmoveto{\pgfqpoint{1.894335in}{3.027480in}}%
\pgfpathlineto{\pgfqpoint{1.894335in}{3.027480in}}%
\pgfpathlineto{\pgfqpoint{1.894335in}{3.030429in}}%
\pgfpathlineto{\pgfqpoint{1.898876in}{3.030429in}}%
\pgfpathlineto{\pgfqpoint{1.898876in}{3.027480in}}%
\pgfpathmoveto{\pgfqpoint{1.898876in}{3.024530in}}%
\pgfpathlineto{\pgfqpoint{1.898876in}{3.024530in}}%
\pgfpathlineto{\pgfqpoint{1.898876in}{3.027480in}}%
\pgfpathlineto{\pgfqpoint{1.903417in}{3.027480in}}%
\pgfpathlineto{\pgfqpoint{1.903417in}{3.024530in}}%
\pgfpathmoveto{\pgfqpoint{1.898876in}{3.027480in}}%
\pgfpathlineto{\pgfqpoint{1.898876in}{3.027480in}}%
\pgfpathlineto{\pgfqpoint{1.898876in}{3.030429in}}%
\pgfpathlineto{\pgfqpoint{1.903417in}{3.030429in}}%
\pgfpathlineto{\pgfqpoint{1.903417in}{3.027480in}}%
\pgfpathmoveto{\pgfqpoint{1.894335in}{3.030429in}}%
\pgfpathlineto{\pgfqpoint{1.894335in}{3.030429in}}%
\pgfpathlineto{\pgfqpoint{1.894335in}{3.033378in}}%
\pgfpathlineto{\pgfqpoint{1.898876in}{3.033378in}}%
\pgfpathlineto{\pgfqpoint{1.898876in}{3.030429in}}%
\pgfpathmoveto{\pgfqpoint{1.903417in}{3.024530in}}%
\pgfpathlineto{\pgfqpoint{1.903417in}{3.024530in}}%
\pgfpathlineto{\pgfqpoint{1.903417in}{3.027480in}}%
\pgfpathlineto{\pgfqpoint{1.907958in}{3.027480in}}%
\pgfpathlineto{\pgfqpoint{1.907958in}{3.024530in}}%
\pgfpathmoveto{\pgfqpoint{1.830762in}{3.077616in}}%
\pgfpathlineto{\pgfqpoint{1.830762in}{3.077616in}}%
\pgfpathlineto{\pgfqpoint{1.830762in}{3.080565in}}%
\pgfpathlineto{\pgfqpoint{1.835303in}{3.080565in}}%
\pgfpathlineto{\pgfqpoint{1.835303in}{3.077616in}}%
\pgfpathmoveto{\pgfqpoint{1.830762in}{3.080565in}}%
\pgfpathlineto{\pgfqpoint{1.830762in}{3.080565in}}%
\pgfpathlineto{\pgfqpoint{1.830762in}{3.083514in}}%
\pgfpathlineto{\pgfqpoint{1.835303in}{3.083514in}}%
\pgfpathlineto{\pgfqpoint{1.835303in}{3.080565in}}%
\pgfpathmoveto{\pgfqpoint{1.835303in}{3.077616in}}%
\pgfpathlineto{\pgfqpoint{1.835303in}{3.077616in}}%
\pgfpathlineto{\pgfqpoint{1.835303in}{3.080565in}}%
\pgfpathlineto{\pgfqpoint{1.839844in}{3.080565in}}%
\pgfpathlineto{\pgfqpoint{1.839844in}{3.077616in}}%
\pgfpathmoveto{\pgfqpoint{1.835303in}{3.080565in}}%
\pgfpathlineto{\pgfqpoint{1.835303in}{3.080565in}}%
\pgfpathlineto{\pgfqpoint{1.835303in}{3.083514in}}%
\pgfpathlineto{\pgfqpoint{1.839844in}{3.083514in}}%
\pgfpathlineto{\pgfqpoint{1.839844in}{3.080565in}}%
\pgfpathmoveto{\pgfqpoint{1.821680in}{3.083514in}}%
\pgfpathlineto{\pgfqpoint{1.821680in}{3.083514in}}%
\pgfpathlineto{\pgfqpoint{1.821680in}{3.086464in}}%
\pgfpathlineto{\pgfqpoint{1.826221in}{3.086464in}}%
\pgfpathlineto{\pgfqpoint{1.826221in}{3.083514in}}%
\pgfpathmoveto{\pgfqpoint{1.821680in}{3.086464in}}%
\pgfpathlineto{\pgfqpoint{1.821680in}{3.086464in}}%
\pgfpathlineto{\pgfqpoint{1.821680in}{3.089413in}}%
\pgfpathlineto{\pgfqpoint{1.826221in}{3.089413in}}%
\pgfpathlineto{\pgfqpoint{1.826221in}{3.086464in}}%
\pgfpathmoveto{\pgfqpoint{1.826221in}{3.083514in}}%
\pgfpathlineto{\pgfqpoint{1.826221in}{3.083514in}}%
\pgfpathlineto{\pgfqpoint{1.826221in}{3.086464in}}%
\pgfpathlineto{\pgfqpoint{1.830762in}{3.086464in}}%
\pgfpathlineto{\pgfqpoint{1.830762in}{3.083514in}}%
\pgfpathmoveto{\pgfqpoint{1.826221in}{3.086464in}}%
\pgfpathlineto{\pgfqpoint{1.826221in}{3.086464in}}%
\pgfpathlineto{\pgfqpoint{1.826221in}{3.089413in}}%
\pgfpathlineto{\pgfqpoint{1.830762in}{3.089413in}}%
\pgfpathlineto{\pgfqpoint{1.830762in}{3.086464in}}%
\pgfpathmoveto{\pgfqpoint{1.821680in}{3.089413in}}%
\pgfpathlineto{\pgfqpoint{1.821680in}{3.089413in}}%
\pgfpathlineto{\pgfqpoint{1.821680in}{3.092362in}}%
\pgfpathlineto{\pgfqpoint{1.826221in}{3.092362in}}%
\pgfpathlineto{\pgfqpoint{1.826221in}{3.089413in}}%
\pgfpathmoveto{\pgfqpoint{1.821680in}{3.092362in}}%
\pgfpathlineto{\pgfqpoint{1.821680in}{3.092362in}}%
\pgfpathlineto{\pgfqpoint{1.821680in}{3.095311in}}%
\pgfpathlineto{\pgfqpoint{1.826221in}{3.095311in}}%
\pgfpathlineto{\pgfqpoint{1.826221in}{3.092362in}}%
\pgfpathmoveto{\pgfqpoint{1.826221in}{3.089413in}}%
\pgfpathlineto{\pgfqpoint{1.826221in}{3.089413in}}%
\pgfpathlineto{\pgfqpoint{1.826221in}{3.092362in}}%
\pgfpathlineto{\pgfqpoint{1.830762in}{3.092362in}}%
\pgfpathlineto{\pgfqpoint{1.830762in}{3.089413in}}%
\pgfpathmoveto{\pgfqpoint{1.830762in}{3.083514in}}%
\pgfpathlineto{\pgfqpoint{1.830762in}{3.083514in}}%
\pgfpathlineto{\pgfqpoint{1.830762in}{3.086464in}}%
\pgfpathlineto{\pgfqpoint{1.835303in}{3.086464in}}%
\pgfpathlineto{\pgfqpoint{1.835303in}{3.083514in}}%
\pgfpathmoveto{\pgfqpoint{1.830762in}{3.086464in}}%
\pgfpathlineto{\pgfqpoint{1.830762in}{3.086464in}}%
\pgfpathlineto{\pgfqpoint{1.830762in}{3.089413in}}%
\pgfpathlineto{\pgfqpoint{1.835303in}{3.089413in}}%
\pgfpathlineto{\pgfqpoint{1.835303in}{3.086464in}}%
\pgfpathmoveto{\pgfqpoint{1.835303in}{3.083514in}}%
\pgfpathlineto{\pgfqpoint{1.835303in}{3.083514in}}%
\pgfpathlineto{\pgfqpoint{1.835303in}{3.086464in}}%
\pgfpathlineto{\pgfqpoint{1.839844in}{3.086464in}}%
\pgfpathlineto{\pgfqpoint{1.839844in}{3.083514in}}%
\pgfpathmoveto{\pgfqpoint{1.794434in}{3.107108in}}%
\pgfpathlineto{\pgfqpoint{1.794434in}{3.107108in}}%
\pgfpathlineto{\pgfqpoint{1.794434in}{3.110057in}}%
\pgfpathlineto{\pgfqpoint{1.798975in}{3.110057in}}%
\pgfpathlineto{\pgfqpoint{1.798975in}{3.107108in}}%
\pgfpathmoveto{\pgfqpoint{1.794434in}{3.110057in}}%
\pgfpathlineto{\pgfqpoint{1.794434in}{3.110057in}}%
\pgfpathlineto{\pgfqpoint{1.794434in}{3.113006in}}%
\pgfpathlineto{\pgfqpoint{1.798975in}{3.113006in}}%
\pgfpathlineto{\pgfqpoint{1.798975in}{3.110057in}}%
\pgfpathmoveto{\pgfqpoint{1.798975in}{3.107108in}}%
\pgfpathlineto{\pgfqpoint{1.798975in}{3.107108in}}%
\pgfpathlineto{\pgfqpoint{1.798975in}{3.110057in}}%
\pgfpathlineto{\pgfqpoint{1.803516in}{3.110057in}}%
\pgfpathlineto{\pgfqpoint{1.803516in}{3.107108in}}%
\pgfpathmoveto{\pgfqpoint{1.798975in}{3.110057in}}%
\pgfpathlineto{\pgfqpoint{1.798975in}{3.110057in}}%
\pgfpathlineto{\pgfqpoint{1.798975in}{3.113006in}}%
\pgfpathlineto{\pgfqpoint{1.803516in}{3.113006in}}%
\pgfpathlineto{\pgfqpoint{1.803516in}{3.110057in}}%
\pgfpathmoveto{\pgfqpoint{1.794434in}{3.113006in}}%
\pgfpathlineto{\pgfqpoint{1.794434in}{3.113006in}}%
\pgfpathlineto{\pgfqpoint{1.794434in}{3.115956in}}%
\pgfpathlineto{\pgfqpoint{1.798975in}{3.115956in}}%
\pgfpathlineto{\pgfqpoint{1.798975in}{3.113006in}}%
\pgfpathmoveto{\pgfqpoint{1.794434in}{3.115956in}}%
\pgfpathlineto{\pgfqpoint{1.794434in}{3.115956in}}%
\pgfpathlineto{\pgfqpoint{1.794434in}{3.118905in}}%
\pgfpathlineto{\pgfqpoint{1.798975in}{3.118905in}}%
\pgfpathlineto{\pgfqpoint{1.798975in}{3.115956in}}%
\pgfpathmoveto{\pgfqpoint{1.798975in}{3.113006in}}%
\pgfpathlineto{\pgfqpoint{1.798975in}{3.113006in}}%
\pgfpathlineto{\pgfqpoint{1.798975in}{3.115956in}}%
\pgfpathlineto{\pgfqpoint{1.803516in}{3.115956in}}%
\pgfpathlineto{\pgfqpoint{1.803516in}{3.113006in}}%
\pgfpathmoveto{\pgfqpoint{1.776271in}{3.124803in}}%
\pgfpathlineto{\pgfqpoint{1.776271in}{3.124803in}}%
\pgfpathlineto{\pgfqpoint{1.776271in}{3.127752in}}%
\pgfpathlineto{\pgfqpoint{1.780812in}{3.127752in}}%
\pgfpathlineto{\pgfqpoint{1.780812in}{3.124803in}}%
\pgfpathmoveto{\pgfqpoint{1.776271in}{3.127752in}}%
\pgfpathlineto{\pgfqpoint{1.776271in}{3.127752in}}%
\pgfpathlineto{\pgfqpoint{1.776271in}{3.130702in}}%
\pgfpathlineto{\pgfqpoint{1.780812in}{3.130702in}}%
\pgfpathlineto{\pgfqpoint{1.780812in}{3.127752in}}%
\pgfpathmoveto{\pgfqpoint{1.780812in}{3.124803in}}%
\pgfpathlineto{\pgfqpoint{1.780812in}{3.124803in}}%
\pgfpathlineto{\pgfqpoint{1.780812in}{3.127752in}}%
\pgfpathlineto{\pgfqpoint{1.785353in}{3.127752in}}%
\pgfpathlineto{\pgfqpoint{1.785353in}{3.124803in}}%
\pgfpathmoveto{\pgfqpoint{1.780812in}{3.127752in}}%
\pgfpathlineto{\pgfqpoint{1.780812in}{3.127752in}}%
\pgfpathlineto{\pgfqpoint{1.780812in}{3.130702in}}%
\pgfpathlineto{\pgfqpoint{1.785353in}{3.130702in}}%
\pgfpathlineto{\pgfqpoint{1.785353in}{3.127752in}}%
\pgfpathmoveto{\pgfqpoint{1.767189in}{3.130702in}}%
\pgfpathlineto{\pgfqpoint{1.767189in}{3.130702in}}%
\pgfpathlineto{\pgfqpoint{1.767189in}{3.133651in}}%
\pgfpathlineto{\pgfqpoint{1.771730in}{3.133651in}}%
\pgfpathlineto{\pgfqpoint{1.771730in}{3.130702in}}%
\pgfpathmoveto{\pgfqpoint{1.767189in}{3.133651in}}%
\pgfpathlineto{\pgfqpoint{1.767189in}{3.133651in}}%
\pgfpathlineto{\pgfqpoint{1.767189in}{3.136600in}}%
\pgfpathlineto{\pgfqpoint{1.771730in}{3.136600in}}%
\pgfpathlineto{\pgfqpoint{1.771730in}{3.133651in}}%
\pgfpathmoveto{\pgfqpoint{1.771730in}{3.130702in}}%
\pgfpathlineto{\pgfqpoint{1.771730in}{3.130702in}}%
\pgfpathlineto{\pgfqpoint{1.771730in}{3.133651in}}%
\pgfpathlineto{\pgfqpoint{1.776271in}{3.133651in}}%
\pgfpathlineto{\pgfqpoint{1.776271in}{3.130702in}}%
\pgfpathmoveto{\pgfqpoint{1.771730in}{3.133651in}}%
\pgfpathlineto{\pgfqpoint{1.771730in}{3.133651in}}%
\pgfpathlineto{\pgfqpoint{1.771730in}{3.136600in}}%
\pgfpathlineto{\pgfqpoint{1.776271in}{3.136600in}}%
\pgfpathlineto{\pgfqpoint{1.776271in}{3.133651in}}%
\pgfpathmoveto{\pgfqpoint{1.767189in}{3.136600in}}%
\pgfpathlineto{\pgfqpoint{1.767189in}{3.136600in}}%
\pgfpathlineto{\pgfqpoint{1.767189in}{3.139549in}}%
\pgfpathlineto{\pgfqpoint{1.771730in}{3.139549in}}%
\pgfpathlineto{\pgfqpoint{1.771730in}{3.136600in}}%
\pgfpathmoveto{\pgfqpoint{1.767189in}{3.139549in}}%
\pgfpathlineto{\pgfqpoint{1.767189in}{3.139549in}}%
\pgfpathlineto{\pgfqpoint{1.767189in}{3.142498in}}%
\pgfpathlineto{\pgfqpoint{1.771730in}{3.142498in}}%
\pgfpathlineto{\pgfqpoint{1.771730in}{3.139549in}}%
\pgfpathmoveto{\pgfqpoint{1.771730in}{3.136600in}}%
\pgfpathlineto{\pgfqpoint{1.771730in}{3.136600in}}%
\pgfpathlineto{\pgfqpoint{1.771730in}{3.139549in}}%
\pgfpathlineto{\pgfqpoint{1.776271in}{3.139549in}}%
\pgfpathlineto{\pgfqpoint{1.776271in}{3.136600in}}%
\pgfpathmoveto{\pgfqpoint{1.776271in}{3.130702in}}%
\pgfpathlineto{\pgfqpoint{1.776271in}{3.130702in}}%
\pgfpathlineto{\pgfqpoint{1.776271in}{3.133651in}}%
\pgfpathlineto{\pgfqpoint{1.780812in}{3.133651in}}%
\pgfpathlineto{\pgfqpoint{1.780812in}{3.130702in}}%
\pgfpathmoveto{\pgfqpoint{1.776271in}{3.133651in}}%
\pgfpathlineto{\pgfqpoint{1.776271in}{3.133651in}}%
\pgfpathlineto{\pgfqpoint{1.776271in}{3.136600in}}%
\pgfpathlineto{\pgfqpoint{1.780812in}{3.136600in}}%
\pgfpathlineto{\pgfqpoint{1.780812in}{3.133651in}}%
\pgfpathmoveto{\pgfqpoint{1.780812in}{3.130702in}}%
\pgfpathlineto{\pgfqpoint{1.780812in}{3.130702in}}%
\pgfpathlineto{\pgfqpoint{1.780812in}{3.133651in}}%
\pgfpathlineto{\pgfqpoint{1.785353in}{3.133651in}}%
\pgfpathlineto{\pgfqpoint{1.785353in}{3.130702in}}%
\pgfpathmoveto{\pgfqpoint{1.785353in}{3.118905in}}%
\pgfpathlineto{\pgfqpoint{1.785353in}{3.118905in}}%
\pgfpathlineto{\pgfqpoint{1.785353in}{3.121854in}}%
\pgfpathlineto{\pgfqpoint{1.789894in}{3.121854in}}%
\pgfpathlineto{\pgfqpoint{1.789894in}{3.118905in}}%
\pgfpathmoveto{\pgfqpoint{1.785353in}{3.121854in}}%
\pgfpathlineto{\pgfqpoint{1.785353in}{3.121854in}}%
\pgfpathlineto{\pgfqpoint{1.785353in}{3.124803in}}%
\pgfpathlineto{\pgfqpoint{1.789894in}{3.124803in}}%
\pgfpathlineto{\pgfqpoint{1.789894in}{3.121854in}}%
\pgfpathmoveto{\pgfqpoint{1.789894in}{3.118905in}}%
\pgfpathlineto{\pgfqpoint{1.789894in}{3.118905in}}%
\pgfpathlineto{\pgfqpoint{1.789894in}{3.121854in}}%
\pgfpathlineto{\pgfqpoint{1.794434in}{3.121854in}}%
\pgfpathlineto{\pgfqpoint{1.794434in}{3.118905in}}%
\pgfpathmoveto{\pgfqpoint{1.789894in}{3.121854in}}%
\pgfpathlineto{\pgfqpoint{1.789894in}{3.121854in}}%
\pgfpathlineto{\pgfqpoint{1.789894in}{3.124803in}}%
\pgfpathlineto{\pgfqpoint{1.794434in}{3.124803in}}%
\pgfpathlineto{\pgfqpoint{1.794434in}{3.121854in}}%
\pgfpathmoveto{\pgfqpoint{1.785353in}{3.124803in}}%
\pgfpathlineto{\pgfqpoint{1.785353in}{3.124803in}}%
\pgfpathlineto{\pgfqpoint{1.785353in}{3.127752in}}%
\pgfpathlineto{\pgfqpoint{1.789894in}{3.127752in}}%
\pgfpathlineto{\pgfqpoint{1.789894in}{3.124803in}}%
\pgfpathmoveto{\pgfqpoint{1.794434in}{3.118905in}}%
\pgfpathlineto{\pgfqpoint{1.794434in}{3.118905in}}%
\pgfpathlineto{\pgfqpoint{1.794434in}{3.121854in}}%
\pgfpathlineto{\pgfqpoint{1.798975in}{3.121854in}}%
\pgfpathlineto{\pgfqpoint{1.798975in}{3.118905in}}%
\pgfpathmoveto{\pgfqpoint{1.803516in}{3.101210in}}%
\pgfpathlineto{\pgfqpoint{1.803516in}{3.101210in}}%
\pgfpathlineto{\pgfqpoint{1.803516in}{3.104159in}}%
\pgfpathlineto{\pgfqpoint{1.808057in}{3.104159in}}%
\pgfpathlineto{\pgfqpoint{1.808057in}{3.101210in}}%
\pgfpathmoveto{\pgfqpoint{1.803516in}{3.104159in}}%
\pgfpathlineto{\pgfqpoint{1.803516in}{3.104159in}}%
\pgfpathlineto{\pgfqpoint{1.803516in}{3.107108in}}%
\pgfpathlineto{\pgfqpoint{1.808057in}{3.107108in}}%
\pgfpathlineto{\pgfqpoint{1.808057in}{3.104159in}}%
\pgfpathmoveto{\pgfqpoint{1.808057in}{3.101210in}}%
\pgfpathlineto{\pgfqpoint{1.808057in}{3.101210in}}%
\pgfpathlineto{\pgfqpoint{1.808057in}{3.104159in}}%
\pgfpathlineto{\pgfqpoint{1.812598in}{3.104159in}}%
\pgfpathlineto{\pgfqpoint{1.812598in}{3.101210in}}%
\pgfpathmoveto{\pgfqpoint{1.808057in}{3.104159in}}%
\pgfpathlineto{\pgfqpoint{1.808057in}{3.104159in}}%
\pgfpathlineto{\pgfqpoint{1.808057in}{3.107108in}}%
\pgfpathlineto{\pgfqpoint{1.812598in}{3.107108in}}%
\pgfpathlineto{\pgfqpoint{1.812598in}{3.104159in}}%
\pgfpathmoveto{\pgfqpoint{1.812598in}{3.095311in}}%
\pgfpathlineto{\pgfqpoint{1.812598in}{3.095311in}}%
\pgfpathlineto{\pgfqpoint{1.812598in}{3.098260in}}%
\pgfpathlineto{\pgfqpoint{1.817139in}{3.098260in}}%
\pgfpathlineto{\pgfqpoint{1.817139in}{3.095311in}}%
\pgfpathmoveto{\pgfqpoint{1.812598in}{3.098260in}}%
\pgfpathlineto{\pgfqpoint{1.812598in}{3.098260in}}%
\pgfpathlineto{\pgfqpoint{1.812598in}{3.101210in}}%
\pgfpathlineto{\pgfqpoint{1.817139in}{3.101210in}}%
\pgfpathlineto{\pgfqpoint{1.817139in}{3.098260in}}%
\pgfpathmoveto{\pgfqpoint{1.817139in}{3.095311in}}%
\pgfpathlineto{\pgfqpoint{1.817139in}{3.095311in}}%
\pgfpathlineto{\pgfqpoint{1.817139in}{3.098260in}}%
\pgfpathlineto{\pgfqpoint{1.821680in}{3.098260in}}%
\pgfpathlineto{\pgfqpoint{1.821680in}{3.095311in}}%
\pgfpathmoveto{\pgfqpoint{1.817139in}{3.098260in}}%
\pgfpathlineto{\pgfqpoint{1.817139in}{3.098260in}}%
\pgfpathlineto{\pgfqpoint{1.817139in}{3.101210in}}%
\pgfpathlineto{\pgfqpoint{1.821680in}{3.101210in}}%
\pgfpathlineto{\pgfqpoint{1.821680in}{3.098260in}}%
\pgfpathmoveto{\pgfqpoint{1.812598in}{3.101210in}}%
\pgfpathlineto{\pgfqpoint{1.812598in}{3.101210in}}%
\pgfpathlineto{\pgfqpoint{1.812598in}{3.104159in}}%
\pgfpathlineto{\pgfqpoint{1.817139in}{3.104159in}}%
\pgfpathlineto{\pgfqpoint{1.817139in}{3.101210in}}%
\pgfpathmoveto{\pgfqpoint{1.803516in}{3.107108in}}%
\pgfpathlineto{\pgfqpoint{1.803516in}{3.107108in}}%
\pgfpathlineto{\pgfqpoint{1.803516in}{3.110057in}}%
\pgfpathlineto{\pgfqpoint{1.808057in}{3.110057in}}%
\pgfpathlineto{\pgfqpoint{1.808057in}{3.107108in}}%
\pgfpathmoveto{\pgfqpoint{1.803516in}{3.110057in}}%
\pgfpathlineto{\pgfqpoint{1.803516in}{3.110057in}}%
\pgfpathlineto{\pgfqpoint{1.803516in}{3.113006in}}%
\pgfpathlineto{\pgfqpoint{1.808057in}{3.113006in}}%
\pgfpathlineto{\pgfqpoint{1.808057in}{3.110057in}}%
\pgfpathmoveto{\pgfqpoint{1.808057in}{3.107108in}}%
\pgfpathlineto{\pgfqpoint{1.808057in}{3.107108in}}%
\pgfpathlineto{\pgfqpoint{1.808057in}{3.110057in}}%
\pgfpathlineto{\pgfqpoint{1.812598in}{3.110057in}}%
\pgfpathlineto{\pgfqpoint{1.812598in}{3.107108in}}%
\pgfpathmoveto{\pgfqpoint{1.821680in}{3.095311in}}%
\pgfpathlineto{\pgfqpoint{1.821680in}{3.095311in}}%
\pgfpathlineto{\pgfqpoint{1.821680in}{3.098260in}}%
\pgfpathlineto{\pgfqpoint{1.826221in}{3.098260in}}%
\pgfpathlineto{\pgfqpoint{1.826221in}{3.095311in}}%
\pgfpathmoveto{\pgfqpoint{1.848926in}{3.059921in}}%
\pgfpathlineto{\pgfqpoint{1.848926in}{3.059921in}}%
\pgfpathlineto{\pgfqpoint{1.848926in}{3.062870in}}%
\pgfpathlineto{\pgfqpoint{1.853467in}{3.062870in}}%
\pgfpathlineto{\pgfqpoint{1.853467in}{3.059921in}}%
\pgfpathmoveto{\pgfqpoint{1.848926in}{3.062870in}}%
\pgfpathlineto{\pgfqpoint{1.848926in}{3.062870in}}%
\pgfpathlineto{\pgfqpoint{1.848926in}{3.065819in}}%
\pgfpathlineto{\pgfqpoint{1.853467in}{3.065819in}}%
\pgfpathlineto{\pgfqpoint{1.853467in}{3.062870in}}%
\pgfpathmoveto{\pgfqpoint{1.853467in}{3.059921in}}%
\pgfpathlineto{\pgfqpoint{1.853467in}{3.059921in}}%
\pgfpathlineto{\pgfqpoint{1.853467in}{3.062870in}}%
\pgfpathlineto{\pgfqpoint{1.858008in}{3.062870in}}%
\pgfpathlineto{\pgfqpoint{1.858008in}{3.059921in}}%
\pgfpathmoveto{\pgfqpoint{1.853467in}{3.062870in}}%
\pgfpathlineto{\pgfqpoint{1.853467in}{3.062870in}}%
\pgfpathlineto{\pgfqpoint{1.853467in}{3.065819in}}%
\pgfpathlineto{\pgfqpoint{1.858008in}{3.065819in}}%
\pgfpathlineto{\pgfqpoint{1.858008in}{3.062870in}}%
\pgfpathmoveto{\pgfqpoint{1.848926in}{3.065819in}}%
\pgfpathlineto{\pgfqpoint{1.848926in}{3.065819in}}%
\pgfpathlineto{\pgfqpoint{1.848926in}{3.068768in}}%
\pgfpathlineto{\pgfqpoint{1.853467in}{3.068768in}}%
\pgfpathlineto{\pgfqpoint{1.853467in}{3.065819in}}%
\pgfpathmoveto{\pgfqpoint{1.848926in}{3.068768in}}%
\pgfpathlineto{\pgfqpoint{1.848926in}{3.068768in}}%
\pgfpathlineto{\pgfqpoint{1.848926in}{3.071718in}}%
\pgfpathlineto{\pgfqpoint{1.853467in}{3.071718in}}%
\pgfpathlineto{\pgfqpoint{1.853467in}{3.068768in}}%
\pgfpathmoveto{\pgfqpoint{1.853467in}{3.065819in}}%
\pgfpathlineto{\pgfqpoint{1.853467in}{3.065819in}}%
\pgfpathlineto{\pgfqpoint{1.853467in}{3.068768in}}%
\pgfpathlineto{\pgfqpoint{1.858008in}{3.068768in}}%
\pgfpathlineto{\pgfqpoint{1.858008in}{3.065819in}}%
\pgfpathmoveto{\pgfqpoint{1.858008in}{3.054023in}}%
\pgfpathlineto{\pgfqpoint{1.858008in}{3.054023in}}%
\pgfpathlineto{\pgfqpoint{1.858008in}{3.056972in}}%
\pgfpathlineto{\pgfqpoint{1.862548in}{3.056972in}}%
\pgfpathlineto{\pgfqpoint{1.862548in}{3.054023in}}%
\pgfpathmoveto{\pgfqpoint{1.858008in}{3.056972in}}%
\pgfpathlineto{\pgfqpoint{1.858008in}{3.056972in}}%
\pgfpathlineto{\pgfqpoint{1.858008in}{3.059921in}}%
\pgfpathlineto{\pgfqpoint{1.862548in}{3.059921in}}%
\pgfpathlineto{\pgfqpoint{1.862548in}{3.056972in}}%
\pgfpathmoveto{\pgfqpoint{1.862548in}{3.054023in}}%
\pgfpathlineto{\pgfqpoint{1.862548in}{3.054023in}}%
\pgfpathlineto{\pgfqpoint{1.862548in}{3.056972in}}%
\pgfpathlineto{\pgfqpoint{1.867089in}{3.056972in}}%
\pgfpathlineto{\pgfqpoint{1.867089in}{3.054023in}}%
\pgfpathmoveto{\pgfqpoint{1.862548in}{3.056972in}}%
\pgfpathlineto{\pgfqpoint{1.862548in}{3.056972in}}%
\pgfpathlineto{\pgfqpoint{1.862548in}{3.059921in}}%
\pgfpathlineto{\pgfqpoint{1.867089in}{3.059921in}}%
\pgfpathlineto{\pgfqpoint{1.867089in}{3.056972in}}%
\pgfpathmoveto{\pgfqpoint{1.867089in}{3.048124in}}%
\pgfpathlineto{\pgfqpoint{1.867089in}{3.048124in}}%
\pgfpathlineto{\pgfqpoint{1.867089in}{3.051073in}}%
\pgfpathlineto{\pgfqpoint{1.871630in}{3.051073in}}%
\pgfpathlineto{\pgfqpoint{1.871630in}{3.048124in}}%
\pgfpathmoveto{\pgfqpoint{1.867089in}{3.051073in}}%
\pgfpathlineto{\pgfqpoint{1.867089in}{3.051073in}}%
\pgfpathlineto{\pgfqpoint{1.867089in}{3.054023in}}%
\pgfpathlineto{\pgfqpoint{1.871630in}{3.054023in}}%
\pgfpathlineto{\pgfqpoint{1.871630in}{3.051073in}}%
\pgfpathmoveto{\pgfqpoint{1.871630in}{3.048124in}}%
\pgfpathlineto{\pgfqpoint{1.871630in}{3.048124in}}%
\pgfpathlineto{\pgfqpoint{1.871630in}{3.051073in}}%
\pgfpathlineto{\pgfqpoint{1.876171in}{3.051073in}}%
\pgfpathlineto{\pgfqpoint{1.876171in}{3.048124in}}%
\pgfpathmoveto{\pgfqpoint{1.871630in}{3.051073in}}%
\pgfpathlineto{\pgfqpoint{1.871630in}{3.051073in}}%
\pgfpathlineto{\pgfqpoint{1.871630in}{3.054023in}}%
\pgfpathlineto{\pgfqpoint{1.876171in}{3.054023in}}%
\pgfpathlineto{\pgfqpoint{1.876171in}{3.051073in}}%
\pgfpathmoveto{\pgfqpoint{1.867089in}{3.054023in}}%
\pgfpathlineto{\pgfqpoint{1.867089in}{3.054023in}}%
\pgfpathlineto{\pgfqpoint{1.867089in}{3.056972in}}%
\pgfpathlineto{\pgfqpoint{1.871630in}{3.056972in}}%
\pgfpathlineto{\pgfqpoint{1.871630in}{3.054023in}}%
\pgfpathmoveto{\pgfqpoint{1.858008in}{3.059921in}}%
\pgfpathlineto{\pgfqpoint{1.858008in}{3.059921in}}%
\pgfpathlineto{\pgfqpoint{1.858008in}{3.062870in}}%
\pgfpathlineto{\pgfqpoint{1.862548in}{3.062870in}}%
\pgfpathlineto{\pgfqpoint{1.862548in}{3.059921in}}%
\pgfpathmoveto{\pgfqpoint{1.858008in}{3.062870in}}%
\pgfpathlineto{\pgfqpoint{1.858008in}{3.062870in}}%
\pgfpathlineto{\pgfqpoint{1.858008in}{3.065819in}}%
\pgfpathlineto{\pgfqpoint{1.862548in}{3.065819in}}%
\pgfpathlineto{\pgfqpoint{1.862548in}{3.062870in}}%
\pgfpathmoveto{\pgfqpoint{1.862548in}{3.059921in}}%
\pgfpathlineto{\pgfqpoint{1.862548in}{3.059921in}}%
\pgfpathlineto{\pgfqpoint{1.862548in}{3.062870in}}%
\pgfpathlineto{\pgfqpoint{1.867089in}{3.062870in}}%
\pgfpathlineto{\pgfqpoint{1.867089in}{3.059921in}}%
\pgfpathmoveto{\pgfqpoint{1.839844in}{3.071718in}}%
\pgfpathlineto{\pgfqpoint{1.839844in}{3.071718in}}%
\pgfpathlineto{\pgfqpoint{1.839844in}{3.074667in}}%
\pgfpathlineto{\pgfqpoint{1.844385in}{3.074667in}}%
\pgfpathlineto{\pgfqpoint{1.844385in}{3.071718in}}%
\pgfpathmoveto{\pgfqpoint{1.839844in}{3.074667in}}%
\pgfpathlineto{\pgfqpoint{1.839844in}{3.074667in}}%
\pgfpathlineto{\pgfqpoint{1.839844in}{3.077616in}}%
\pgfpathlineto{\pgfqpoint{1.844385in}{3.077616in}}%
\pgfpathlineto{\pgfqpoint{1.844385in}{3.074667in}}%
\pgfpathmoveto{\pgfqpoint{1.844385in}{3.071718in}}%
\pgfpathlineto{\pgfqpoint{1.844385in}{3.071718in}}%
\pgfpathlineto{\pgfqpoint{1.844385in}{3.074667in}}%
\pgfpathlineto{\pgfqpoint{1.848926in}{3.074667in}}%
\pgfpathlineto{\pgfqpoint{1.848926in}{3.071718in}}%
\pgfpathmoveto{\pgfqpoint{1.844385in}{3.074667in}}%
\pgfpathlineto{\pgfqpoint{1.844385in}{3.074667in}}%
\pgfpathlineto{\pgfqpoint{1.844385in}{3.077616in}}%
\pgfpathlineto{\pgfqpoint{1.848926in}{3.077616in}}%
\pgfpathlineto{\pgfqpoint{1.848926in}{3.074667in}}%
\pgfpathmoveto{\pgfqpoint{1.839844in}{3.077616in}}%
\pgfpathlineto{\pgfqpoint{1.839844in}{3.077616in}}%
\pgfpathlineto{\pgfqpoint{1.839844in}{3.080565in}}%
\pgfpathlineto{\pgfqpoint{1.844385in}{3.080565in}}%
\pgfpathlineto{\pgfqpoint{1.844385in}{3.077616in}}%
\pgfpathmoveto{\pgfqpoint{1.848926in}{3.071718in}}%
\pgfpathlineto{\pgfqpoint{1.848926in}{3.071718in}}%
\pgfpathlineto{\pgfqpoint{1.848926in}{3.074667in}}%
\pgfpathlineto{\pgfqpoint{1.853467in}{3.074667in}}%
\pgfpathlineto{\pgfqpoint{1.853467in}{3.071718in}}%
\pgfpathmoveto{\pgfqpoint{1.876171in}{3.048124in}}%
\pgfpathlineto{\pgfqpoint{1.876171in}{3.048124in}}%
\pgfpathlineto{\pgfqpoint{1.876171in}{3.051073in}}%
\pgfpathlineto{\pgfqpoint{1.880712in}{3.051073in}}%
\pgfpathlineto{\pgfqpoint{1.880712in}{3.048124in}}%
\pgfpathmoveto{\pgfqpoint{1.767189in}{3.142498in}}%
\pgfpathlineto{\pgfqpoint{1.767189in}{3.142498in}}%
\pgfpathlineto{\pgfqpoint{1.767189in}{3.145448in}}%
\pgfpathlineto{\pgfqpoint{1.771730in}{3.145448in}}%
\pgfpathlineto{\pgfqpoint{1.771730in}{3.142498in}}%
\pgfpathmoveto{\pgfqpoint{1.976075in}{2.947850in}}%
\pgfpathlineto{\pgfqpoint{1.976075in}{2.947850in}}%
\pgfpathlineto{\pgfqpoint{1.976075in}{2.950800in}}%
\pgfpathlineto{\pgfqpoint{1.980616in}{2.950800in}}%
\pgfpathlineto{\pgfqpoint{1.980616in}{2.947850in}}%
\pgfpathmoveto{\pgfqpoint{1.976075in}{2.950800in}}%
\pgfpathlineto{\pgfqpoint{1.976075in}{2.950800in}}%
\pgfpathlineto{\pgfqpoint{1.976075in}{2.953749in}}%
\pgfpathlineto{\pgfqpoint{1.980616in}{2.953749in}}%
\pgfpathlineto{\pgfqpoint{1.980616in}{2.950800in}}%
\pgfpathmoveto{\pgfqpoint{1.980616in}{2.947850in}}%
\pgfpathlineto{\pgfqpoint{1.980616in}{2.947850in}}%
\pgfpathlineto{\pgfqpoint{1.980616in}{2.950800in}}%
\pgfpathlineto{\pgfqpoint{1.985158in}{2.950800in}}%
\pgfpathlineto{\pgfqpoint{1.985158in}{2.947850in}}%
\pgfpathmoveto{\pgfqpoint{1.980616in}{2.950800in}}%
\pgfpathlineto{\pgfqpoint{1.980616in}{2.950800in}}%
\pgfpathlineto{\pgfqpoint{1.980616in}{2.953749in}}%
\pgfpathlineto{\pgfqpoint{1.985158in}{2.953749in}}%
\pgfpathlineto{\pgfqpoint{1.985158in}{2.950800in}}%
\pgfpathmoveto{\pgfqpoint{2.030569in}{2.900662in}}%
\pgfpathlineto{\pgfqpoint{2.030569in}{2.900662in}}%
\pgfpathlineto{\pgfqpoint{2.030569in}{2.903611in}}%
\pgfpathlineto{\pgfqpoint{2.035110in}{2.903611in}}%
\pgfpathlineto{\pgfqpoint{2.035110in}{2.900662in}}%
\pgfpathmoveto{\pgfqpoint{2.030569in}{2.903611in}}%
\pgfpathlineto{\pgfqpoint{2.030569in}{2.903611in}}%
\pgfpathlineto{\pgfqpoint{2.030569in}{2.906561in}}%
\pgfpathlineto{\pgfqpoint{2.035110in}{2.906561in}}%
\pgfpathlineto{\pgfqpoint{2.035110in}{2.903611in}}%
\pgfpathmoveto{\pgfqpoint{2.035110in}{2.900662in}}%
\pgfpathlineto{\pgfqpoint{2.035110in}{2.900662in}}%
\pgfpathlineto{\pgfqpoint{2.035110in}{2.903611in}}%
\pgfpathlineto{\pgfqpoint{2.039652in}{2.903611in}}%
\pgfpathlineto{\pgfqpoint{2.039652in}{2.900662in}}%
\pgfpathmoveto{\pgfqpoint{2.035110in}{2.903611in}}%
\pgfpathlineto{\pgfqpoint{2.035110in}{2.903611in}}%
\pgfpathlineto{\pgfqpoint{2.035110in}{2.906561in}}%
\pgfpathlineto{\pgfqpoint{2.039652in}{2.906561in}}%
\pgfpathlineto{\pgfqpoint{2.039652in}{2.903611in}}%
\pgfpathmoveto{\pgfqpoint{2.048734in}{2.888865in}}%
\pgfpathlineto{\pgfqpoint{2.048734in}{2.888865in}}%
\pgfpathlineto{\pgfqpoint{2.048734in}{2.891814in}}%
\pgfpathlineto{\pgfqpoint{2.053275in}{2.891814in}}%
\pgfpathlineto{\pgfqpoint{2.053275in}{2.888865in}}%
\pgfpathmoveto{\pgfqpoint{2.048734in}{2.891814in}}%
\pgfpathlineto{\pgfqpoint{2.048734in}{2.891814in}}%
\pgfpathlineto{\pgfqpoint{2.048734in}{2.894764in}}%
\pgfpathlineto{\pgfqpoint{2.053275in}{2.894764in}}%
\pgfpathlineto{\pgfqpoint{2.053275in}{2.891814in}}%
\pgfpathmoveto{\pgfqpoint{2.053275in}{2.888865in}}%
\pgfpathlineto{\pgfqpoint{2.053275in}{2.888865in}}%
\pgfpathlineto{\pgfqpoint{2.053275in}{2.891814in}}%
\pgfpathlineto{\pgfqpoint{2.057816in}{2.891814in}}%
\pgfpathlineto{\pgfqpoint{2.057816in}{2.888865in}}%
\pgfpathmoveto{\pgfqpoint{2.053275in}{2.891814in}}%
\pgfpathlineto{\pgfqpoint{2.053275in}{2.891814in}}%
\pgfpathlineto{\pgfqpoint{2.053275in}{2.894764in}}%
\pgfpathlineto{\pgfqpoint{2.057816in}{2.894764in}}%
\pgfpathlineto{\pgfqpoint{2.057816in}{2.891814in}}%
\pgfpathmoveto{\pgfqpoint{2.039652in}{2.894764in}}%
\pgfpathlineto{\pgfqpoint{2.039652in}{2.894764in}}%
\pgfpathlineto{\pgfqpoint{2.039652in}{2.897713in}}%
\pgfpathlineto{\pgfqpoint{2.044193in}{2.897713in}}%
\pgfpathlineto{\pgfqpoint{2.044193in}{2.894764in}}%
\pgfpathmoveto{\pgfqpoint{2.039652in}{2.897713in}}%
\pgfpathlineto{\pgfqpoint{2.039652in}{2.897713in}}%
\pgfpathlineto{\pgfqpoint{2.039652in}{2.900662in}}%
\pgfpathlineto{\pgfqpoint{2.044193in}{2.900662in}}%
\pgfpathlineto{\pgfqpoint{2.044193in}{2.897713in}}%
\pgfpathmoveto{\pgfqpoint{2.044193in}{2.894764in}}%
\pgfpathlineto{\pgfqpoint{2.044193in}{2.894764in}}%
\pgfpathlineto{\pgfqpoint{2.044193in}{2.897713in}}%
\pgfpathlineto{\pgfqpoint{2.048734in}{2.897713in}}%
\pgfpathlineto{\pgfqpoint{2.048734in}{2.894764in}}%
\pgfpathmoveto{\pgfqpoint{2.044193in}{2.897713in}}%
\pgfpathlineto{\pgfqpoint{2.044193in}{2.897713in}}%
\pgfpathlineto{\pgfqpoint{2.044193in}{2.900662in}}%
\pgfpathlineto{\pgfqpoint{2.048734in}{2.900662in}}%
\pgfpathlineto{\pgfqpoint{2.048734in}{2.897713in}}%
\pgfpathmoveto{\pgfqpoint{2.039652in}{2.900662in}}%
\pgfpathlineto{\pgfqpoint{2.039652in}{2.900662in}}%
\pgfpathlineto{\pgfqpoint{2.039652in}{2.903611in}}%
\pgfpathlineto{\pgfqpoint{2.044193in}{2.903611in}}%
\pgfpathlineto{\pgfqpoint{2.044193in}{2.900662in}}%
\pgfpathmoveto{\pgfqpoint{2.039652in}{2.903611in}}%
\pgfpathlineto{\pgfqpoint{2.039652in}{2.903611in}}%
\pgfpathlineto{\pgfqpoint{2.039652in}{2.906561in}}%
\pgfpathlineto{\pgfqpoint{2.044193in}{2.906561in}}%
\pgfpathlineto{\pgfqpoint{2.044193in}{2.903611in}}%
\pgfpathmoveto{\pgfqpoint{2.044193in}{2.900662in}}%
\pgfpathlineto{\pgfqpoint{2.044193in}{2.900662in}}%
\pgfpathlineto{\pgfqpoint{2.044193in}{2.903611in}}%
\pgfpathlineto{\pgfqpoint{2.048734in}{2.903611in}}%
\pgfpathlineto{\pgfqpoint{2.048734in}{2.900662in}}%
\pgfpathmoveto{\pgfqpoint{2.048734in}{2.894764in}}%
\pgfpathlineto{\pgfqpoint{2.048734in}{2.894764in}}%
\pgfpathlineto{\pgfqpoint{2.048734in}{2.897713in}}%
\pgfpathlineto{\pgfqpoint{2.053275in}{2.897713in}}%
\pgfpathlineto{\pgfqpoint{2.053275in}{2.894764in}}%
\pgfpathmoveto{\pgfqpoint{2.048734in}{2.897713in}}%
\pgfpathlineto{\pgfqpoint{2.048734in}{2.897713in}}%
\pgfpathlineto{\pgfqpoint{2.048734in}{2.900662in}}%
\pgfpathlineto{\pgfqpoint{2.053275in}{2.900662in}}%
\pgfpathlineto{\pgfqpoint{2.053275in}{2.897713in}}%
\pgfpathmoveto{\pgfqpoint{2.003322in}{2.924256in}}%
\pgfpathlineto{\pgfqpoint{2.003322in}{2.924256in}}%
\pgfpathlineto{\pgfqpoint{2.003322in}{2.927205in}}%
\pgfpathlineto{\pgfqpoint{2.007863in}{2.927205in}}%
\pgfpathlineto{\pgfqpoint{2.007863in}{2.924256in}}%
\pgfpathmoveto{\pgfqpoint{2.003322in}{2.927205in}}%
\pgfpathlineto{\pgfqpoint{2.003322in}{2.927205in}}%
\pgfpathlineto{\pgfqpoint{2.003322in}{2.930155in}}%
\pgfpathlineto{\pgfqpoint{2.007863in}{2.930155in}}%
\pgfpathlineto{\pgfqpoint{2.007863in}{2.927205in}}%
\pgfpathmoveto{\pgfqpoint{2.007863in}{2.924256in}}%
\pgfpathlineto{\pgfqpoint{2.007863in}{2.924256in}}%
\pgfpathlineto{\pgfqpoint{2.007863in}{2.927205in}}%
\pgfpathlineto{\pgfqpoint{2.012405in}{2.927205in}}%
\pgfpathlineto{\pgfqpoint{2.012405in}{2.924256in}}%
\pgfpathmoveto{\pgfqpoint{2.007863in}{2.927205in}}%
\pgfpathlineto{\pgfqpoint{2.007863in}{2.927205in}}%
\pgfpathlineto{\pgfqpoint{2.007863in}{2.930155in}}%
\pgfpathlineto{\pgfqpoint{2.012405in}{2.930155in}}%
\pgfpathlineto{\pgfqpoint{2.012405in}{2.927205in}}%
\pgfpathmoveto{\pgfqpoint{2.012405in}{2.918358in}}%
\pgfpathlineto{\pgfqpoint{2.012405in}{2.918358in}}%
\pgfpathlineto{\pgfqpoint{2.012405in}{2.921307in}}%
\pgfpathlineto{\pgfqpoint{2.016946in}{2.921307in}}%
\pgfpathlineto{\pgfqpoint{2.016946in}{2.918358in}}%
\pgfpathmoveto{\pgfqpoint{2.012405in}{2.921307in}}%
\pgfpathlineto{\pgfqpoint{2.012405in}{2.921307in}}%
\pgfpathlineto{\pgfqpoint{2.012405in}{2.924256in}}%
\pgfpathlineto{\pgfqpoint{2.016946in}{2.924256in}}%
\pgfpathlineto{\pgfqpoint{2.016946in}{2.921307in}}%
\pgfpathmoveto{\pgfqpoint{2.016946in}{2.918358in}}%
\pgfpathlineto{\pgfqpoint{2.016946in}{2.918358in}}%
\pgfpathlineto{\pgfqpoint{2.016946in}{2.921307in}}%
\pgfpathlineto{\pgfqpoint{2.021487in}{2.921307in}}%
\pgfpathlineto{\pgfqpoint{2.021487in}{2.918358in}}%
\pgfpathmoveto{\pgfqpoint{2.016946in}{2.921307in}}%
\pgfpathlineto{\pgfqpoint{2.016946in}{2.921307in}}%
\pgfpathlineto{\pgfqpoint{2.016946in}{2.924256in}}%
\pgfpathlineto{\pgfqpoint{2.021487in}{2.924256in}}%
\pgfpathlineto{\pgfqpoint{2.021487in}{2.921307in}}%
\pgfpathmoveto{\pgfqpoint{2.012405in}{2.924256in}}%
\pgfpathlineto{\pgfqpoint{2.012405in}{2.924256in}}%
\pgfpathlineto{\pgfqpoint{2.012405in}{2.927205in}}%
\pgfpathlineto{\pgfqpoint{2.016946in}{2.927205in}}%
\pgfpathlineto{\pgfqpoint{2.016946in}{2.924256in}}%
\pgfpathmoveto{\pgfqpoint{2.012405in}{2.927205in}}%
\pgfpathlineto{\pgfqpoint{2.012405in}{2.927205in}}%
\pgfpathlineto{\pgfqpoint{2.012405in}{2.930155in}}%
\pgfpathlineto{\pgfqpoint{2.016946in}{2.930155in}}%
\pgfpathlineto{\pgfqpoint{2.016946in}{2.927205in}}%
\pgfpathmoveto{\pgfqpoint{2.016946in}{2.924256in}}%
\pgfpathlineto{\pgfqpoint{2.016946in}{2.924256in}}%
\pgfpathlineto{\pgfqpoint{2.016946in}{2.927205in}}%
\pgfpathlineto{\pgfqpoint{2.021487in}{2.927205in}}%
\pgfpathlineto{\pgfqpoint{2.021487in}{2.924256in}}%
\pgfpathmoveto{\pgfqpoint{1.994240in}{2.936053in}}%
\pgfpathlineto{\pgfqpoint{1.994240in}{2.936053in}}%
\pgfpathlineto{\pgfqpoint{1.994240in}{2.939003in}}%
\pgfpathlineto{\pgfqpoint{1.998781in}{2.939003in}}%
\pgfpathlineto{\pgfqpoint{1.998781in}{2.936053in}}%
\pgfpathmoveto{\pgfqpoint{1.994240in}{2.939003in}}%
\pgfpathlineto{\pgfqpoint{1.994240in}{2.939003in}}%
\pgfpathlineto{\pgfqpoint{1.994240in}{2.941952in}}%
\pgfpathlineto{\pgfqpoint{1.998781in}{2.941952in}}%
\pgfpathlineto{\pgfqpoint{1.998781in}{2.939003in}}%
\pgfpathmoveto{\pgfqpoint{1.998781in}{2.936053in}}%
\pgfpathlineto{\pgfqpoint{1.998781in}{2.936053in}}%
\pgfpathlineto{\pgfqpoint{1.998781in}{2.939003in}}%
\pgfpathlineto{\pgfqpoint{2.003322in}{2.939003in}}%
\pgfpathlineto{\pgfqpoint{2.003322in}{2.936053in}}%
\pgfpathmoveto{\pgfqpoint{1.998781in}{2.939003in}}%
\pgfpathlineto{\pgfqpoint{1.998781in}{2.939003in}}%
\pgfpathlineto{\pgfqpoint{1.998781in}{2.941952in}}%
\pgfpathlineto{\pgfqpoint{2.003322in}{2.941952in}}%
\pgfpathlineto{\pgfqpoint{2.003322in}{2.939003in}}%
\pgfpathmoveto{\pgfqpoint{1.985158in}{2.941952in}}%
\pgfpathlineto{\pgfqpoint{1.985158in}{2.941952in}}%
\pgfpathlineto{\pgfqpoint{1.985158in}{2.944901in}}%
\pgfpathlineto{\pgfqpoint{1.989699in}{2.944901in}}%
\pgfpathlineto{\pgfqpoint{1.989699in}{2.941952in}}%
\pgfpathmoveto{\pgfqpoint{1.985158in}{2.944901in}}%
\pgfpathlineto{\pgfqpoint{1.985158in}{2.944901in}}%
\pgfpathlineto{\pgfqpoint{1.985158in}{2.947850in}}%
\pgfpathlineto{\pgfqpoint{1.989699in}{2.947850in}}%
\pgfpathlineto{\pgfqpoint{1.989699in}{2.944901in}}%
\pgfpathmoveto{\pgfqpoint{1.989699in}{2.941952in}}%
\pgfpathlineto{\pgfqpoint{1.989699in}{2.941952in}}%
\pgfpathlineto{\pgfqpoint{1.989699in}{2.944901in}}%
\pgfpathlineto{\pgfqpoint{1.994240in}{2.944901in}}%
\pgfpathlineto{\pgfqpoint{1.994240in}{2.941952in}}%
\pgfpathmoveto{\pgfqpoint{1.989699in}{2.944901in}}%
\pgfpathlineto{\pgfqpoint{1.989699in}{2.944901in}}%
\pgfpathlineto{\pgfqpoint{1.989699in}{2.947850in}}%
\pgfpathlineto{\pgfqpoint{1.994240in}{2.947850in}}%
\pgfpathlineto{\pgfqpoint{1.994240in}{2.944901in}}%
\pgfpathmoveto{\pgfqpoint{1.985158in}{2.947850in}}%
\pgfpathlineto{\pgfqpoint{1.985158in}{2.947850in}}%
\pgfpathlineto{\pgfqpoint{1.985158in}{2.950800in}}%
\pgfpathlineto{\pgfqpoint{1.989699in}{2.950800in}}%
\pgfpathlineto{\pgfqpoint{1.989699in}{2.947850in}}%
\pgfpathmoveto{\pgfqpoint{1.985158in}{2.950800in}}%
\pgfpathlineto{\pgfqpoint{1.985158in}{2.950800in}}%
\pgfpathlineto{\pgfqpoint{1.985158in}{2.953749in}}%
\pgfpathlineto{\pgfqpoint{1.989699in}{2.953749in}}%
\pgfpathlineto{\pgfqpoint{1.989699in}{2.950800in}}%
\pgfpathmoveto{\pgfqpoint{1.989699in}{2.947850in}}%
\pgfpathlineto{\pgfqpoint{1.989699in}{2.947850in}}%
\pgfpathlineto{\pgfqpoint{1.989699in}{2.950800in}}%
\pgfpathlineto{\pgfqpoint{1.994240in}{2.950800in}}%
\pgfpathlineto{\pgfqpoint{1.994240in}{2.947850in}}%
\pgfpathmoveto{\pgfqpoint{1.994240in}{2.941952in}}%
\pgfpathlineto{\pgfqpoint{1.994240in}{2.941952in}}%
\pgfpathlineto{\pgfqpoint{1.994240in}{2.944901in}}%
\pgfpathlineto{\pgfqpoint{1.998781in}{2.944901in}}%
\pgfpathlineto{\pgfqpoint{1.998781in}{2.941952in}}%
\pgfpathmoveto{\pgfqpoint{1.994240in}{2.944901in}}%
\pgfpathlineto{\pgfqpoint{1.994240in}{2.944901in}}%
\pgfpathlineto{\pgfqpoint{1.994240in}{2.947850in}}%
\pgfpathlineto{\pgfqpoint{1.998781in}{2.947850in}}%
\pgfpathlineto{\pgfqpoint{1.998781in}{2.944901in}}%
\pgfpathmoveto{\pgfqpoint{2.003322in}{2.930155in}}%
\pgfpathlineto{\pgfqpoint{2.003322in}{2.930155in}}%
\pgfpathlineto{\pgfqpoint{2.003322in}{2.933104in}}%
\pgfpathlineto{\pgfqpoint{2.007863in}{2.933104in}}%
\pgfpathlineto{\pgfqpoint{2.007863in}{2.930155in}}%
\pgfpathmoveto{\pgfqpoint{2.003322in}{2.933104in}}%
\pgfpathlineto{\pgfqpoint{2.003322in}{2.933104in}}%
\pgfpathlineto{\pgfqpoint{2.003322in}{2.936053in}}%
\pgfpathlineto{\pgfqpoint{2.007863in}{2.936053in}}%
\pgfpathlineto{\pgfqpoint{2.007863in}{2.933104in}}%
\pgfpathmoveto{\pgfqpoint{2.007863in}{2.930155in}}%
\pgfpathlineto{\pgfqpoint{2.007863in}{2.930155in}}%
\pgfpathlineto{\pgfqpoint{2.007863in}{2.933104in}}%
\pgfpathlineto{\pgfqpoint{2.012405in}{2.933104in}}%
\pgfpathlineto{\pgfqpoint{2.012405in}{2.930155in}}%
\pgfpathmoveto{\pgfqpoint{2.007863in}{2.933104in}}%
\pgfpathlineto{\pgfqpoint{2.007863in}{2.933104in}}%
\pgfpathlineto{\pgfqpoint{2.007863in}{2.936053in}}%
\pgfpathlineto{\pgfqpoint{2.012405in}{2.936053in}}%
\pgfpathlineto{\pgfqpoint{2.012405in}{2.933104in}}%
\pgfpathmoveto{\pgfqpoint{2.003322in}{2.936053in}}%
\pgfpathlineto{\pgfqpoint{2.003322in}{2.936053in}}%
\pgfpathlineto{\pgfqpoint{2.003322in}{2.939003in}}%
\pgfpathlineto{\pgfqpoint{2.007863in}{2.939003in}}%
\pgfpathlineto{\pgfqpoint{2.007863in}{2.936053in}}%
\pgfpathmoveto{\pgfqpoint{2.021487in}{2.912459in}}%
\pgfpathlineto{\pgfqpoint{2.021487in}{2.912459in}}%
\pgfpathlineto{\pgfqpoint{2.021487in}{2.915408in}}%
\pgfpathlineto{\pgfqpoint{2.026028in}{2.915408in}}%
\pgfpathlineto{\pgfqpoint{2.026028in}{2.912459in}}%
\pgfpathmoveto{\pgfqpoint{2.021487in}{2.915408in}}%
\pgfpathlineto{\pgfqpoint{2.021487in}{2.915408in}}%
\pgfpathlineto{\pgfqpoint{2.021487in}{2.918358in}}%
\pgfpathlineto{\pgfqpoint{2.026028in}{2.918358in}}%
\pgfpathlineto{\pgfqpoint{2.026028in}{2.915408in}}%
\pgfpathmoveto{\pgfqpoint{2.026028in}{2.912459in}}%
\pgfpathlineto{\pgfqpoint{2.026028in}{2.912459in}}%
\pgfpathlineto{\pgfqpoint{2.026028in}{2.915408in}}%
\pgfpathlineto{\pgfqpoint{2.030569in}{2.915408in}}%
\pgfpathlineto{\pgfqpoint{2.030569in}{2.912459in}}%
\pgfpathmoveto{\pgfqpoint{2.026028in}{2.915408in}}%
\pgfpathlineto{\pgfqpoint{2.026028in}{2.915408in}}%
\pgfpathlineto{\pgfqpoint{2.026028in}{2.918358in}}%
\pgfpathlineto{\pgfqpoint{2.030569in}{2.918358in}}%
\pgfpathlineto{\pgfqpoint{2.030569in}{2.915408in}}%
\pgfpathmoveto{\pgfqpoint{2.030569in}{2.906561in}}%
\pgfpathlineto{\pgfqpoint{2.030569in}{2.906561in}}%
\pgfpathlineto{\pgfqpoint{2.030569in}{2.909510in}}%
\pgfpathlineto{\pgfqpoint{2.035110in}{2.909510in}}%
\pgfpathlineto{\pgfqpoint{2.035110in}{2.906561in}}%
\pgfpathmoveto{\pgfqpoint{2.030569in}{2.909510in}}%
\pgfpathlineto{\pgfqpoint{2.030569in}{2.909510in}}%
\pgfpathlineto{\pgfqpoint{2.030569in}{2.912459in}}%
\pgfpathlineto{\pgfqpoint{2.035110in}{2.912459in}}%
\pgfpathlineto{\pgfqpoint{2.035110in}{2.909510in}}%
\pgfpathmoveto{\pgfqpoint{2.035110in}{2.906561in}}%
\pgfpathlineto{\pgfqpoint{2.035110in}{2.906561in}}%
\pgfpathlineto{\pgfqpoint{2.035110in}{2.909510in}}%
\pgfpathlineto{\pgfqpoint{2.039652in}{2.909510in}}%
\pgfpathlineto{\pgfqpoint{2.039652in}{2.906561in}}%
\pgfpathmoveto{\pgfqpoint{2.035110in}{2.909510in}}%
\pgfpathlineto{\pgfqpoint{2.035110in}{2.909510in}}%
\pgfpathlineto{\pgfqpoint{2.035110in}{2.912459in}}%
\pgfpathlineto{\pgfqpoint{2.039652in}{2.912459in}}%
\pgfpathlineto{\pgfqpoint{2.039652in}{2.909510in}}%
\pgfpathmoveto{\pgfqpoint{2.030569in}{2.912459in}}%
\pgfpathlineto{\pgfqpoint{2.030569in}{2.912459in}}%
\pgfpathlineto{\pgfqpoint{2.030569in}{2.915408in}}%
\pgfpathlineto{\pgfqpoint{2.035110in}{2.915408in}}%
\pgfpathlineto{\pgfqpoint{2.035110in}{2.912459in}}%
\pgfpathmoveto{\pgfqpoint{2.021487in}{2.918358in}}%
\pgfpathlineto{\pgfqpoint{2.021487in}{2.918358in}}%
\pgfpathlineto{\pgfqpoint{2.021487in}{2.921307in}}%
\pgfpathlineto{\pgfqpoint{2.026028in}{2.921307in}}%
\pgfpathlineto{\pgfqpoint{2.026028in}{2.918358in}}%
\pgfpathmoveto{\pgfqpoint{2.021487in}{2.921307in}}%
\pgfpathlineto{\pgfqpoint{2.021487in}{2.921307in}}%
\pgfpathlineto{\pgfqpoint{2.021487in}{2.924256in}}%
\pgfpathlineto{\pgfqpoint{2.026028in}{2.924256in}}%
\pgfpathlineto{\pgfqpoint{2.026028in}{2.921307in}}%
\pgfpathmoveto{\pgfqpoint{1.939746in}{2.983241in}}%
\pgfpathlineto{\pgfqpoint{1.939746in}{2.983241in}}%
\pgfpathlineto{\pgfqpoint{1.939746in}{2.986190in}}%
\pgfpathlineto{\pgfqpoint{1.944287in}{2.986190in}}%
\pgfpathlineto{\pgfqpoint{1.944287in}{2.983241in}}%
\pgfpathmoveto{\pgfqpoint{1.939746in}{2.986190in}}%
\pgfpathlineto{\pgfqpoint{1.939746in}{2.986190in}}%
\pgfpathlineto{\pgfqpoint{1.939746in}{2.989140in}}%
\pgfpathlineto{\pgfqpoint{1.944287in}{2.989140in}}%
\pgfpathlineto{\pgfqpoint{1.944287in}{2.986190in}}%
\pgfpathmoveto{\pgfqpoint{1.944287in}{2.983241in}}%
\pgfpathlineto{\pgfqpoint{1.944287in}{2.983241in}}%
\pgfpathlineto{\pgfqpoint{1.944287in}{2.986190in}}%
\pgfpathlineto{\pgfqpoint{1.948828in}{2.986190in}}%
\pgfpathlineto{\pgfqpoint{1.948828in}{2.983241in}}%
\pgfpathmoveto{\pgfqpoint{1.944287in}{2.986190in}}%
\pgfpathlineto{\pgfqpoint{1.944287in}{2.986190in}}%
\pgfpathlineto{\pgfqpoint{1.944287in}{2.989140in}}%
\pgfpathlineto{\pgfqpoint{1.948828in}{2.989140in}}%
\pgfpathlineto{\pgfqpoint{1.948828in}{2.986190in}}%
\pgfpathmoveto{\pgfqpoint{1.930663in}{2.989140in}}%
\pgfpathlineto{\pgfqpoint{1.930663in}{2.989140in}}%
\pgfpathlineto{\pgfqpoint{1.930663in}{2.992089in}}%
\pgfpathlineto{\pgfqpoint{1.935205in}{2.992089in}}%
\pgfpathlineto{\pgfqpoint{1.935205in}{2.989140in}}%
\pgfpathmoveto{\pgfqpoint{1.930663in}{2.992089in}}%
\pgfpathlineto{\pgfqpoint{1.930663in}{2.992089in}}%
\pgfpathlineto{\pgfqpoint{1.930663in}{2.995038in}}%
\pgfpathlineto{\pgfqpoint{1.935205in}{2.995038in}}%
\pgfpathlineto{\pgfqpoint{1.935205in}{2.992089in}}%
\pgfpathmoveto{\pgfqpoint{1.935205in}{2.989140in}}%
\pgfpathlineto{\pgfqpoint{1.935205in}{2.989140in}}%
\pgfpathlineto{\pgfqpoint{1.935205in}{2.992089in}}%
\pgfpathlineto{\pgfqpoint{1.939746in}{2.992089in}}%
\pgfpathlineto{\pgfqpoint{1.939746in}{2.989140in}}%
\pgfpathmoveto{\pgfqpoint{1.935205in}{2.992089in}}%
\pgfpathlineto{\pgfqpoint{1.935205in}{2.992089in}}%
\pgfpathlineto{\pgfqpoint{1.935205in}{2.995038in}}%
\pgfpathlineto{\pgfqpoint{1.939746in}{2.995038in}}%
\pgfpathlineto{\pgfqpoint{1.939746in}{2.992089in}}%
\pgfpathmoveto{\pgfqpoint{1.930663in}{2.995038in}}%
\pgfpathlineto{\pgfqpoint{1.930663in}{2.995038in}}%
\pgfpathlineto{\pgfqpoint{1.930663in}{2.997987in}}%
\pgfpathlineto{\pgfqpoint{1.935205in}{2.997987in}}%
\pgfpathlineto{\pgfqpoint{1.935205in}{2.995038in}}%
\pgfpathmoveto{\pgfqpoint{1.930663in}{2.997987in}}%
\pgfpathlineto{\pgfqpoint{1.930663in}{2.997987in}}%
\pgfpathlineto{\pgfqpoint{1.930663in}{3.000937in}}%
\pgfpathlineto{\pgfqpoint{1.935205in}{3.000937in}}%
\pgfpathlineto{\pgfqpoint{1.935205in}{2.997987in}}%
\pgfpathmoveto{\pgfqpoint{1.935205in}{2.995038in}}%
\pgfpathlineto{\pgfqpoint{1.935205in}{2.995038in}}%
\pgfpathlineto{\pgfqpoint{1.935205in}{2.997987in}}%
\pgfpathlineto{\pgfqpoint{1.939746in}{2.997987in}}%
\pgfpathlineto{\pgfqpoint{1.939746in}{2.995038in}}%
\pgfpathmoveto{\pgfqpoint{1.939746in}{2.989140in}}%
\pgfpathlineto{\pgfqpoint{1.939746in}{2.989140in}}%
\pgfpathlineto{\pgfqpoint{1.939746in}{2.992089in}}%
\pgfpathlineto{\pgfqpoint{1.944287in}{2.992089in}}%
\pgfpathlineto{\pgfqpoint{1.944287in}{2.989140in}}%
\pgfpathmoveto{\pgfqpoint{1.939746in}{2.992089in}}%
\pgfpathlineto{\pgfqpoint{1.939746in}{2.992089in}}%
\pgfpathlineto{\pgfqpoint{1.939746in}{2.995038in}}%
\pgfpathlineto{\pgfqpoint{1.944287in}{2.995038in}}%
\pgfpathlineto{\pgfqpoint{1.944287in}{2.992089in}}%
\pgfpathmoveto{\pgfqpoint{1.944287in}{2.989140in}}%
\pgfpathlineto{\pgfqpoint{1.944287in}{2.989140in}}%
\pgfpathlineto{\pgfqpoint{1.944287in}{2.992089in}}%
\pgfpathlineto{\pgfqpoint{1.948828in}{2.992089in}}%
\pgfpathlineto{\pgfqpoint{1.948828in}{2.989140in}}%
\pgfpathmoveto{\pgfqpoint{1.957910in}{2.965546in}}%
\pgfpathlineto{\pgfqpoint{1.957910in}{2.965546in}}%
\pgfpathlineto{\pgfqpoint{1.957910in}{2.968495in}}%
\pgfpathlineto{\pgfqpoint{1.962452in}{2.968495in}}%
\pgfpathlineto{\pgfqpoint{1.962452in}{2.965546in}}%
\pgfpathmoveto{\pgfqpoint{1.957910in}{2.968495in}}%
\pgfpathlineto{\pgfqpoint{1.957910in}{2.968495in}}%
\pgfpathlineto{\pgfqpoint{1.957910in}{2.971444in}}%
\pgfpathlineto{\pgfqpoint{1.962452in}{2.971444in}}%
\pgfpathlineto{\pgfqpoint{1.962452in}{2.968495in}}%
\pgfpathmoveto{\pgfqpoint{1.962452in}{2.965546in}}%
\pgfpathlineto{\pgfqpoint{1.962452in}{2.965546in}}%
\pgfpathlineto{\pgfqpoint{1.962452in}{2.968495in}}%
\pgfpathlineto{\pgfqpoint{1.966993in}{2.968495in}}%
\pgfpathlineto{\pgfqpoint{1.966993in}{2.965546in}}%
\pgfpathmoveto{\pgfqpoint{1.962452in}{2.968495in}}%
\pgfpathlineto{\pgfqpoint{1.962452in}{2.968495in}}%
\pgfpathlineto{\pgfqpoint{1.962452in}{2.971444in}}%
\pgfpathlineto{\pgfqpoint{1.966993in}{2.971444in}}%
\pgfpathlineto{\pgfqpoint{1.966993in}{2.968495in}}%
\pgfpathmoveto{\pgfqpoint{1.957910in}{2.971444in}}%
\pgfpathlineto{\pgfqpoint{1.957910in}{2.971444in}}%
\pgfpathlineto{\pgfqpoint{1.957910in}{2.974393in}}%
\pgfpathlineto{\pgfqpoint{1.962452in}{2.974393in}}%
\pgfpathlineto{\pgfqpoint{1.962452in}{2.971444in}}%
\pgfpathmoveto{\pgfqpoint{1.957910in}{2.974393in}}%
\pgfpathlineto{\pgfqpoint{1.957910in}{2.974393in}}%
\pgfpathlineto{\pgfqpoint{1.957910in}{2.977343in}}%
\pgfpathlineto{\pgfqpoint{1.962452in}{2.977343in}}%
\pgfpathlineto{\pgfqpoint{1.962452in}{2.974393in}}%
\pgfpathmoveto{\pgfqpoint{1.962452in}{2.971444in}}%
\pgfpathlineto{\pgfqpoint{1.962452in}{2.971444in}}%
\pgfpathlineto{\pgfqpoint{1.962452in}{2.974393in}}%
\pgfpathlineto{\pgfqpoint{1.966993in}{2.974393in}}%
\pgfpathlineto{\pgfqpoint{1.966993in}{2.971444in}}%
\pgfpathmoveto{\pgfqpoint{1.966993in}{2.959647in}}%
\pgfpathlineto{\pgfqpoint{1.966993in}{2.959647in}}%
\pgfpathlineto{\pgfqpoint{1.966993in}{2.962597in}}%
\pgfpathlineto{\pgfqpoint{1.971534in}{2.962597in}}%
\pgfpathlineto{\pgfqpoint{1.971534in}{2.959647in}}%
\pgfpathmoveto{\pgfqpoint{1.966993in}{2.962597in}}%
\pgfpathlineto{\pgfqpoint{1.966993in}{2.962597in}}%
\pgfpathlineto{\pgfqpoint{1.966993in}{2.965546in}}%
\pgfpathlineto{\pgfqpoint{1.971534in}{2.965546in}}%
\pgfpathlineto{\pgfqpoint{1.971534in}{2.962597in}}%
\pgfpathmoveto{\pgfqpoint{1.971534in}{2.959647in}}%
\pgfpathlineto{\pgfqpoint{1.971534in}{2.959647in}}%
\pgfpathlineto{\pgfqpoint{1.971534in}{2.962597in}}%
\pgfpathlineto{\pgfqpoint{1.976075in}{2.962597in}}%
\pgfpathlineto{\pgfqpoint{1.976075in}{2.959647in}}%
\pgfpathmoveto{\pgfqpoint{1.971534in}{2.962597in}}%
\pgfpathlineto{\pgfqpoint{1.971534in}{2.962597in}}%
\pgfpathlineto{\pgfqpoint{1.971534in}{2.965546in}}%
\pgfpathlineto{\pgfqpoint{1.976075in}{2.965546in}}%
\pgfpathlineto{\pgfqpoint{1.976075in}{2.962597in}}%
\pgfpathmoveto{\pgfqpoint{1.976075in}{2.953749in}}%
\pgfpathlineto{\pgfqpoint{1.976075in}{2.953749in}}%
\pgfpathlineto{\pgfqpoint{1.976075in}{2.956698in}}%
\pgfpathlineto{\pgfqpoint{1.980616in}{2.956698in}}%
\pgfpathlineto{\pgfqpoint{1.980616in}{2.953749in}}%
\pgfpathmoveto{\pgfqpoint{1.976075in}{2.956698in}}%
\pgfpathlineto{\pgfqpoint{1.976075in}{2.956698in}}%
\pgfpathlineto{\pgfqpoint{1.976075in}{2.959647in}}%
\pgfpathlineto{\pgfqpoint{1.980616in}{2.959647in}}%
\pgfpathlineto{\pgfqpoint{1.980616in}{2.956698in}}%
\pgfpathmoveto{\pgfqpoint{1.980616in}{2.953749in}}%
\pgfpathlineto{\pgfqpoint{1.980616in}{2.953749in}}%
\pgfpathlineto{\pgfqpoint{1.980616in}{2.956698in}}%
\pgfpathlineto{\pgfqpoint{1.985158in}{2.956698in}}%
\pgfpathlineto{\pgfqpoint{1.985158in}{2.953749in}}%
\pgfpathmoveto{\pgfqpoint{1.980616in}{2.956698in}}%
\pgfpathlineto{\pgfqpoint{1.980616in}{2.956698in}}%
\pgfpathlineto{\pgfqpoint{1.980616in}{2.959647in}}%
\pgfpathlineto{\pgfqpoint{1.985158in}{2.959647in}}%
\pgfpathlineto{\pgfqpoint{1.985158in}{2.956698in}}%
\pgfpathmoveto{\pgfqpoint{1.976075in}{2.959647in}}%
\pgfpathlineto{\pgfqpoint{1.976075in}{2.959647in}}%
\pgfpathlineto{\pgfqpoint{1.976075in}{2.962597in}}%
\pgfpathlineto{\pgfqpoint{1.980616in}{2.962597in}}%
\pgfpathlineto{\pgfqpoint{1.980616in}{2.959647in}}%
\pgfpathmoveto{\pgfqpoint{1.966993in}{2.965546in}}%
\pgfpathlineto{\pgfqpoint{1.966993in}{2.965546in}}%
\pgfpathlineto{\pgfqpoint{1.966993in}{2.968495in}}%
\pgfpathlineto{\pgfqpoint{1.971534in}{2.968495in}}%
\pgfpathlineto{\pgfqpoint{1.971534in}{2.965546in}}%
\pgfpathmoveto{\pgfqpoint{1.966993in}{2.968495in}}%
\pgfpathlineto{\pgfqpoint{1.966993in}{2.968495in}}%
\pgfpathlineto{\pgfqpoint{1.966993in}{2.971444in}}%
\pgfpathlineto{\pgfqpoint{1.971534in}{2.971444in}}%
\pgfpathlineto{\pgfqpoint{1.971534in}{2.968495in}}%
\pgfpathmoveto{\pgfqpoint{1.971534in}{2.965546in}}%
\pgfpathlineto{\pgfqpoint{1.971534in}{2.965546in}}%
\pgfpathlineto{\pgfqpoint{1.971534in}{2.968495in}}%
\pgfpathlineto{\pgfqpoint{1.976075in}{2.968495in}}%
\pgfpathlineto{\pgfqpoint{1.976075in}{2.965546in}}%
\pgfpathmoveto{\pgfqpoint{1.948828in}{2.977343in}}%
\pgfpathlineto{\pgfqpoint{1.948828in}{2.977343in}}%
\pgfpathlineto{\pgfqpoint{1.948828in}{2.980292in}}%
\pgfpathlineto{\pgfqpoint{1.953369in}{2.980292in}}%
\pgfpathlineto{\pgfqpoint{1.953369in}{2.977343in}}%
\pgfpathmoveto{\pgfqpoint{1.948828in}{2.980292in}}%
\pgfpathlineto{\pgfqpoint{1.948828in}{2.980292in}}%
\pgfpathlineto{\pgfqpoint{1.948828in}{2.983241in}}%
\pgfpathlineto{\pgfqpoint{1.953369in}{2.983241in}}%
\pgfpathlineto{\pgfqpoint{1.953369in}{2.980292in}}%
\pgfpathmoveto{\pgfqpoint{1.953369in}{2.977343in}}%
\pgfpathlineto{\pgfqpoint{1.953369in}{2.977343in}}%
\pgfpathlineto{\pgfqpoint{1.953369in}{2.980292in}}%
\pgfpathlineto{\pgfqpoint{1.957910in}{2.980292in}}%
\pgfpathlineto{\pgfqpoint{1.957910in}{2.977343in}}%
\pgfpathmoveto{\pgfqpoint{1.953369in}{2.980292in}}%
\pgfpathlineto{\pgfqpoint{1.953369in}{2.980292in}}%
\pgfpathlineto{\pgfqpoint{1.953369in}{2.983241in}}%
\pgfpathlineto{\pgfqpoint{1.957910in}{2.983241in}}%
\pgfpathlineto{\pgfqpoint{1.957910in}{2.980292in}}%
\pgfpathmoveto{\pgfqpoint{1.948828in}{2.983241in}}%
\pgfpathlineto{\pgfqpoint{1.948828in}{2.983241in}}%
\pgfpathlineto{\pgfqpoint{1.948828in}{2.986190in}}%
\pgfpathlineto{\pgfqpoint{1.953369in}{2.986190in}}%
\pgfpathlineto{\pgfqpoint{1.953369in}{2.983241in}}%
\pgfpathmoveto{\pgfqpoint{1.957910in}{2.977343in}}%
\pgfpathlineto{\pgfqpoint{1.957910in}{2.977343in}}%
\pgfpathlineto{\pgfqpoint{1.957910in}{2.980292in}}%
\pgfpathlineto{\pgfqpoint{1.962452in}{2.980292in}}%
\pgfpathlineto{\pgfqpoint{1.962452in}{2.977343in}}%
\pgfpathmoveto{\pgfqpoint{1.912499in}{3.006835in}}%
\pgfpathlineto{\pgfqpoint{1.912499in}{3.006835in}}%
\pgfpathlineto{\pgfqpoint{1.912499in}{3.009784in}}%
\pgfpathlineto{\pgfqpoint{1.917040in}{3.009784in}}%
\pgfpathlineto{\pgfqpoint{1.917040in}{3.006835in}}%
\pgfpathmoveto{\pgfqpoint{1.912499in}{3.009784in}}%
\pgfpathlineto{\pgfqpoint{1.912499in}{3.009784in}}%
\pgfpathlineto{\pgfqpoint{1.912499in}{3.012733in}}%
\pgfpathlineto{\pgfqpoint{1.917040in}{3.012733in}}%
\pgfpathlineto{\pgfqpoint{1.917040in}{3.009784in}}%
\pgfpathmoveto{\pgfqpoint{1.917040in}{3.006835in}}%
\pgfpathlineto{\pgfqpoint{1.917040in}{3.006835in}}%
\pgfpathlineto{\pgfqpoint{1.917040in}{3.009784in}}%
\pgfpathlineto{\pgfqpoint{1.921581in}{3.009784in}}%
\pgfpathlineto{\pgfqpoint{1.921581in}{3.006835in}}%
\pgfpathmoveto{\pgfqpoint{1.917040in}{3.009784in}}%
\pgfpathlineto{\pgfqpoint{1.917040in}{3.009784in}}%
\pgfpathlineto{\pgfqpoint{1.917040in}{3.012733in}}%
\pgfpathlineto{\pgfqpoint{1.921581in}{3.012733in}}%
\pgfpathlineto{\pgfqpoint{1.921581in}{3.009784in}}%
\pgfpathmoveto{\pgfqpoint{1.921581in}{3.000937in}}%
\pgfpathlineto{\pgfqpoint{1.921581in}{3.000937in}}%
\pgfpathlineto{\pgfqpoint{1.921581in}{3.003886in}}%
\pgfpathlineto{\pgfqpoint{1.926122in}{3.003886in}}%
\pgfpathlineto{\pgfqpoint{1.926122in}{3.000937in}}%
\pgfpathmoveto{\pgfqpoint{1.921581in}{3.003886in}}%
\pgfpathlineto{\pgfqpoint{1.921581in}{3.003886in}}%
\pgfpathlineto{\pgfqpoint{1.921581in}{3.006835in}}%
\pgfpathlineto{\pgfqpoint{1.926122in}{3.006835in}}%
\pgfpathlineto{\pgfqpoint{1.926122in}{3.003886in}}%
\pgfpathmoveto{\pgfqpoint{1.926122in}{3.000937in}}%
\pgfpathlineto{\pgfqpoint{1.926122in}{3.000937in}}%
\pgfpathlineto{\pgfqpoint{1.926122in}{3.003886in}}%
\pgfpathlineto{\pgfqpoint{1.930663in}{3.003886in}}%
\pgfpathlineto{\pgfqpoint{1.930663in}{3.000937in}}%
\pgfpathmoveto{\pgfqpoint{1.926122in}{3.003886in}}%
\pgfpathlineto{\pgfqpoint{1.926122in}{3.003886in}}%
\pgfpathlineto{\pgfqpoint{1.926122in}{3.006835in}}%
\pgfpathlineto{\pgfqpoint{1.930663in}{3.006835in}}%
\pgfpathlineto{\pgfqpoint{1.930663in}{3.003886in}}%
\pgfpathmoveto{\pgfqpoint{1.921581in}{3.006835in}}%
\pgfpathlineto{\pgfqpoint{1.921581in}{3.006835in}}%
\pgfpathlineto{\pgfqpoint{1.921581in}{3.009784in}}%
\pgfpathlineto{\pgfqpoint{1.926122in}{3.009784in}}%
\pgfpathlineto{\pgfqpoint{1.926122in}{3.006835in}}%
\pgfpathmoveto{\pgfqpoint{1.912499in}{3.012733in}}%
\pgfpathlineto{\pgfqpoint{1.912499in}{3.012733in}}%
\pgfpathlineto{\pgfqpoint{1.912499in}{3.015683in}}%
\pgfpathlineto{\pgfqpoint{1.917040in}{3.015683in}}%
\pgfpathlineto{\pgfqpoint{1.917040in}{3.012733in}}%
\pgfpathmoveto{\pgfqpoint{1.912499in}{3.015683in}}%
\pgfpathlineto{\pgfqpoint{1.912499in}{3.015683in}}%
\pgfpathlineto{\pgfqpoint{1.912499in}{3.018632in}}%
\pgfpathlineto{\pgfqpoint{1.917040in}{3.018632in}}%
\pgfpathlineto{\pgfqpoint{1.917040in}{3.015683in}}%
\pgfpathmoveto{\pgfqpoint{1.917040in}{3.012733in}}%
\pgfpathlineto{\pgfqpoint{1.917040in}{3.012733in}}%
\pgfpathlineto{\pgfqpoint{1.917040in}{3.015683in}}%
\pgfpathlineto{\pgfqpoint{1.921581in}{3.015683in}}%
\pgfpathlineto{\pgfqpoint{1.921581in}{3.012733in}}%
\pgfpathmoveto{\pgfqpoint{1.930663in}{3.000937in}}%
\pgfpathlineto{\pgfqpoint{1.930663in}{3.000937in}}%
\pgfpathlineto{\pgfqpoint{1.930663in}{3.003886in}}%
\pgfpathlineto{\pgfqpoint{1.935205in}{3.003886in}}%
\pgfpathlineto{\pgfqpoint{1.935205in}{3.000937in}}%
\pgfpathmoveto{\pgfqpoint{2.121387in}{2.823984in}}%
\pgfpathlineto{\pgfqpoint{2.121387in}{2.823984in}}%
\pgfpathlineto{\pgfqpoint{2.121387in}{2.826933in}}%
\pgfpathlineto{\pgfqpoint{2.125928in}{2.826933in}}%
\pgfpathlineto{\pgfqpoint{2.125928in}{2.823984in}}%
\pgfpathmoveto{\pgfqpoint{2.121387in}{2.826933in}}%
\pgfpathlineto{\pgfqpoint{2.121387in}{2.826933in}}%
\pgfpathlineto{\pgfqpoint{2.121387in}{2.829882in}}%
\pgfpathlineto{\pgfqpoint{2.125928in}{2.829882in}}%
\pgfpathlineto{\pgfqpoint{2.125928in}{2.826933in}}%
\pgfpathmoveto{\pgfqpoint{2.125928in}{2.823984in}}%
\pgfpathlineto{\pgfqpoint{2.125928in}{2.823984in}}%
\pgfpathlineto{\pgfqpoint{2.125928in}{2.826933in}}%
\pgfpathlineto{\pgfqpoint{2.130469in}{2.826933in}}%
\pgfpathlineto{\pgfqpoint{2.130469in}{2.823984in}}%
\pgfpathmoveto{\pgfqpoint{2.125928in}{2.826933in}}%
\pgfpathlineto{\pgfqpoint{2.125928in}{2.826933in}}%
\pgfpathlineto{\pgfqpoint{2.125928in}{2.829882in}}%
\pgfpathlineto{\pgfqpoint{2.130469in}{2.829882in}}%
\pgfpathlineto{\pgfqpoint{2.130469in}{2.826933in}}%
\pgfpathmoveto{\pgfqpoint{2.121387in}{2.829882in}}%
\pgfpathlineto{\pgfqpoint{2.121387in}{2.829882in}}%
\pgfpathlineto{\pgfqpoint{2.121387in}{2.832831in}}%
\pgfpathlineto{\pgfqpoint{2.125928in}{2.832831in}}%
\pgfpathlineto{\pgfqpoint{2.125928in}{2.829882in}}%
\pgfpathmoveto{\pgfqpoint{2.121387in}{2.832831in}}%
\pgfpathlineto{\pgfqpoint{2.121387in}{2.832831in}}%
\pgfpathlineto{\pgfqpoint{2.121387in}{2.835780in}}%
\pgfpathlineto{\pgfqpoint{2.125928in}{2.835780in}}%
\pgfpathlineto{\pgfqpoint{2.125928in}{2.832831in}}%
\pgfpathmoveto{\pgfqpoint{2.125928in}{2.829882in}}%
\pgfpathlineto{\pgfqpoint{2.125928in}{2.829882in}}%
\pgfpathlineto{\pgfqpoint{2.125928in}{2.832831in}}%
\pgfpathlineto{\pgfqpoint{2.130469in}{2.832831in}}%
\pgfpathlineto{\pgfqpoint{2.130469in}{2.829882in}}%
\pgfpathmoveto{\pgfqpoint{2.103224in}{2.841678in}}%
\pgfpathlineto{\pgfqpoint{2.103224in}{2.841678in}}%
\pgfpathlineto{\pgfqpoint{2.103224in}{2.844627in}}%
\pgfpathlineto{\pgfqpoint{2.107765in}{2.844627in}}%
\pgfpathlineto{\pgfqpoint{2.107765in}{2.841678in}}%
\pgfpathmoveto{\pgfqpoint{2.103224in}{2.844627in}}%
\pgfpathlineto{\pgfqpoint{2.103224in}{2.844627in}}%
\pgfpathlineto{\pgfqpoint{2.103224in}{2.847576in}}%
\pgfpathlineto{\pgfqpoint{2.107765in}{2.847576in}}%
\pgfpathlineto{\pgfqpoint{2.107765in}{2.844627in}}%
\pgfpathmoveto{\pgfqpoint{2.107765in}{2.841678in}}%
\pgfpathlineto{\pgfqpoint{2.107765in}{2.841678in}}%
\pgfpathlineto{\pgfqpoint{2.107765in}{2.844627in}}%
\pgfpathlineto{\pgfqpoint{2.112306in}{2.844627in}}%
\pgfpathlineto{\pgfqpoint{2.112306in}{2.841678in}}%
\pgfpathmoveto{\pgfqpoint{2.107765in}{2.844627in}}%
\pgfpathlineto{\pgfqpoint{2.107765in}{2.844627in}}%
\pgfpathlineto{\pgfqpoint{2.107765in}{2.847576in}}%
\pgfpathlineto{\pgfqpoint{2.112306in}{2.847576in}}%
\pgfpathlineto{\pgfqpoint{2.112306in}{2.844627in}}%
\pgfpathmoveto{\pgfqpoint{2.094143in}{2.847576in}}%
\pgfpathlineto{\pgfqpoint{2.094143in}{2.847576in}}%
\pgfpathlineto{\pgfqpoint{2.094143in}{2.850525in}}%
\pgfpathlineto{\pgfqpoint{2.098683in}{2.850525in}}%
\pgfpathlineto{\pgfqpoint{2.098683in}{2.847576in}}%
\pgfpathmoveto{\pgfqpoint{2.094143in}{2.850525in}}%
\pgfpathlineto{\pgfqpoint{2.094143in}{2.850525in}}%
\pgfpathlineto{\pgfqpoint{2.094143in}{2.853474in}}%
\pgfpathlineto{\pgfqpoint{2.098683in}{2.853474in}}%
\pgfpathlineto{\pgfqpoint{2.098683in}{2.850525in}}%
\pgfpathmoveto{\pgfqpoint{2.098683in}{2.847576in}}%
\pgfpathlineto{\pgfqpoint{2.098683in}{2.847576in}}%
\pgfpathlineto{\pgfqpoint{2.098683in}{2.850525in}}%
\pgfpathlineto{\pgfqpoint{2.103224in}{2.850525in}}%
\pgfpathlineto{\pgfqpoint{2.103224in}{2.847576in}}%
\pgfpathmoveto{\pgfqpoint{2.098683in}{2.850525in}}%
\pgfpathlineto{\pgfqpoint{2.098683in}{2.850525in}}%
\pgfpathlineto{\pgfqpoint{2.098683in}{2.853474in}}%
\pgfpathlineto{\pgfqpoint{2.103224in}{2.853474in}}%
\pgfpathlineto{\pgfqpoint{2.103224in}{2.850525in}}%
\pgfpathmoveto{\pgfqpoint{2.094143in}{2.853474in}}%
\pgfpathlineto{\pgfqpoint{2.094143in}{2.853474in}}%
\pgfpathlineto{\pgfqpoint{2.094143in}{2.856423in}}%
\pgfpathlineto{\pgfqpoint{2.098683in}{2.856423in}}%
\pgfpathlineto{\pgfqpoint{2.098683in}{2.853474in}}%
\pgfpathmoveto{\pgfqpoint{2.094143in}{2.856423in}}%
\pgfpathlineto{\pgfqpoint{2.094143in}{2.856423in}}%
\pgfpathlineto{\pgfqpoint{2.094143in}{2.859372in}}%
\pgfpathlineto{\pgfqpoint{2.098683in}{2.859372in}}%
\pgfpathlineto{\pgfqpoint{2.098683in}{2.856423in}}%
\pgfpathmoveto{\pgfqpoint{2.098683in}{2.853474in}}%
\pgfpathlineto{\pgfqpoint{2.098683in}{2.853474in}}%
\pgfpathlineto{\pgfqpoint{2.098683in}{2.856423in}}%
\pgfpathlineto{\pgfqpoint{2.103224in}{2.856423in}}%
\pgfpathlineto{\pgfqpoint{2.103224in}{2.853474in}}%
\pgfpathmoveto{\pgfqpoint{2.103224in}{2.847576in}}%
\pgfpathlineto{\pgfqpoint{2.103224in}{2.847576in}}%
\pgfpathlineto{\pgfqpoint{2.103224in}{2.850525in}}%
\pgfpathlineto{\pgfqpoint{2.107765in}{2.850525in}}%
\pgfpathlineto{\pgfqpoint{2.107765in}{2.847576in}}%
\pgfpathmoveto{\pgfqpoint{2.103224in}{2.850525in}}%
\pgfpathlineto{\pgfqpoint{2.103224in}{2.850525in}}%
\pgfpathlineto{\pgfqpoint{2.103224in}{2.853474in}}%
\pgfpathlineto{\pgfqpoint{2.107765in}{2.853474in}}%
\pgfpathlineto{\pgfqpoint{2.107765in}{2.850525in}}%
\pgfpathmoveto{\pgfqpoint{2.107765in}{2.847576in}}%
\pgfpathlineto{\pgfqpoint{2.107765in}{2.847576in}}%
\pgfpathlineto{\pgfqpoint{2.107765in}{2.850525in}}%
\pgfpathlineto{\pgfqpoint{2.112306in}{2.850525in}}%
\pgfpathlineto{\pgfqpoint{2.112306in}{2.847576in}}%
\pgfpathmoveto{\pgfqpoint{2.112306in}{2.835780in}}%
\pgfpathlineto{\pgfqpoint{2.112306in}{2.835780in}}%
\pgfpathlineto{\pgfqpoint{2.112306in}{2.838729in}}%
\pgfpathlineto{\pgfqpoint{2.116847in}{2.838729in}}%
\pgfpathlineto{\pgfqpoint{2.116847in}{2.835780in}}%
\pgfpathmoveto{\pgfqpoint{2.112306in}{2.838729in}}%
\pgfpathlineto{\pgfqpoint{2.112306in}{2.838729in}}%
\pgfpathlineto{\pgfqpoint{2.112306in}{2.841678in}}%
\pgfpathlineto{\pgfqpoint{2.116847in}{2.841678in}}%
\pgfpathlineto{\pgfqpoint{2.116847in}{2.838729in}}%
\pgfpathmoveto{\pgfqpoint{2.116847in}{2.835780in}}%
\pgfpathlineto{\pgfqpoint{2.116847in}{2.835780in}}%
\pgfpathlineto{\pgfqpoint{2.116847in}{2.838729in}}%
\pgfpathlineto{\pgfqpoint{2.121387in}{2.838729in}}%
\pgfpathlineto{\pgfqpoint{2.121387in}{2.835780in}}%
\pgfpathmoveto{\pgfqpoint{2.116847in}{2.838729in}}%
\pgfpathlineto{\pgfqpoint{2.116847in}{2.838729in}}%
\pgfpathlineto{\pgfqpoint{2.116847in}{2.841678in}}%
\pgfpathlineto{\pgfqpoint{2.121387in}{2.841678in}}%
\pgfpathlineto{\pgfqpoint{2.121387in}{2.838729in}}%
\pgfpathmoveto{\pgfqpoint{2.112306in}{2.841678in}}%
\pgfpathlineto{\pgfqpoint{2.112306in}{2.841678in}}%
\pgfpathlineto{\pgfqpoint{2.112306in}{2.844627in}}%
\pgfpathlineto{\pgfqpoint{2.116847in}{2.844627in}}%
\pgfpathlineto{\pgfqpoint{2.116847in}{2.841678in}}%
\pgfpathmoveto{\pgfqpoint{2.121387in}{2.835780in}}%
\pgfpathlineto{\pgfqpoint{2.121387in}{2.835780in}}%
\pgfpathlineto{\pgfqpoint{2.121387in}{2.838729in}}%
\pgfpathlineto{\pgfqpoint{2.125928in}{2.838729in}}%
\pgfpathlineto{\pgfqpoint{2.125928in}{2.835780in}}%
\pgfpathmoveto{\pgfqpoint{2.157714in}{2.794493in}}%
\pgfpathlineto{\pgfqpoint{2.157714in}{2.794493in}}%
\pgfpathlineto{\pgfqpoint{2.157714in}{2.797442in}}%
\pgfpathlineto{\pgfqpoint{2.162254in}{2.797442in}}%
\pgfpathlineto{\pgfqpoint{2.162254in}{2.794493in}}%
\pgfpathmoveto{\pgfqpoint{2.157714in}{2.797442in}}%
\pgfpathlineto{\pgfqpoint{2.157714in}{2.797442in}}%
\pgfpathlineto{\pgfqpoint{2.157714in}{2.800391in}}%
\pgfpathlineto{\pgfqpoint{2.162254in}{2.800391in}}%
\pgfpathlineto{\pgfqpoint{2.162254in}{2.797442in}}%
\pgfpathmoveto{\pgfqpoint{2.162254in}{2.794493in}}%
\pgfpathlineto{\pgfqpoint{2.162254in}{2.794493in}}%
\pgfpathlineto{\pgfqpoint{2.162254in}{2.797442in}}%
\pgfpathlineto{\pgfqpoint{2.166795in}{2.797442in}}%
\pgfpathlineto{\pgfqpoint{2.166795in}{2.794493in}}%
\pgfpathmoveto{\pgfqpoint{2.162254in}{2.797442in}}%
\pgfpathlineto{\pgfqpoint{2.162254in}{2.797442in}}%
\pgfpathlineto{\pgfqpoint{2.162254in}{2.800391in}}%
\pgfpathlineto{\pgfqpoint{2.166795in}{2.800391in}}%
\pgfpathlineto{\pgfqpoint{2.166795in}{2.797442in}}%
\pgfpathmoveto{\pgfqpoint{2.148632in}{2.800391in}}%
\pgfpathlineto{\pgfqpoint{2.148632in}{2.800391in}}%
\pgfpathlineto{\pgfqpoint{2.148632in}{2.803340in}}%
\pgfpathlineto{\pgfqpoint{2.153173in}{2.803340in}}%
\pgfpathlineto{\pgfqpoint{2.153173in}{2.800391in}}%
\pgfpathmoveto{\pgfqpoint{2.148632in}{2.803340in}}%
\pgfpathlineto{\pgfqpoint{2.148632in}{2.803340in}}%
\pgfpathlineto{\pgfqpoint{2.148632in}{2.806289in}}%
\pgfpathlineto{\pgfqpoint{2.153173in}{2.806289in}}%
\pgfpathlineto{\pgfqpoint{2.153173in}{2.803340in}}%
\pgfpathmoveto{\pgfqpoint{2.153173in}{2.800391in}}%
\pgfpathlineto{\pgfqpoint{2.153173in}{2.800391in}}%
\pgfpathlineto{\pgfqpoint{2.153173in}{2.803340in}}%
\pgfpathlineto{\pgfqpoint{2.157714in}{2.803340in}}%
\pgfpathlineto{\pgfqpoint{2.157714in}{2.800391in}}%
\pgfpathmoveto{\pgfqpoint{2.153173in}{2.803340in}}%
\pgfpathlineto{\pgfqpoint{2.153173in}{2.803340in}}%
\pgfpathlineto{\pgfqpoint{2.153173in}{2.806289in}}%
\pgfpathlineto{\pgfqpoint{2.157714in}{2.806289in}}%
\pgfpathlineto{\pgfqpoint{2.157714in}{2.803340in}}%
\pgfpathmoveto{\pgfqpoint{2.148632in}{2.806289in}}%
\pgfpathlineto{\pgfqpoint{2.148632in}{2.806289in}}%
\pgfpathlineto{\pgfqpoint{2.148632in}{2.809239in}}%
\pgfpathlineto{\pgfqpoint{2.153173in}{2.809239in}}%
\pgfpathlineto{\pgfqpoint{2.153173in}{2.806289in}}%
\pgfpathmoveto{\pgfqpoint{2.148632in}{2.809239in}}%
\pgfpathlineto{\pgfqpoint{2.148632in}{2.809239in}}%
\pgfpathlineto{\pgfqpoint{2.148632in}{2.812188in}}%
\pgfpathlineto{\pgfqpoint{2.153173in}{2.812188in}}%
\pgfpathlineto{\pgfqpoint{2.153173in}{2.809239in}}%
\pgfpathmoveto{\pgfqpoint{2.153173in}{2.806289in}}%
\pgfpathlineto{\pgfqpoint{2.153173in}{2.806289in}}%
\pgfpathlineto{\pgfqpoint{2.153173in}{2.809239in}}%
\pgfpathlineto{\pgfqpoint{2.157714in}{2.809239in}}%
\pgfpathlineto{\pgfqpoint{2.157714in}{2.806289in}}%
\pgfpathmoveto{\pgfqpoint{2.157714in}{2.800391in}}%
\pgfpathlineto{\pgfqpoint{2.157714in}{2.800391in}}%
\pgfpathlineto{\pgfqpoint{2.157714in}{2.803340in}}%
\pgfpathlineto{\pgfqpoint{2.162254in}{2.803340in}}%
\pgfpathlineto{\pgfqpoint{2.162254in}{2.800391in}}%
\pgfpathmoveto{\pgfqpoint{2.157714in}{2.803340in}}%
\pgfpathlineto{\pgfqpoint{2.157714in}{2.803340in}}%
\pgfpathlineto{\pgfqpoint{2.157714in}{2.806289in}}%
\pgfpathlineto{\pgfqpoint{2.162254in}{2.806289in}}%
\pgfpathlineto{\pgfqpoint{2.162254in}{2.803340in}}%
\pgfpathmoveto{\pgfqpoint{2.162254in}{2.800391in}}%
\pgfpathlineto{\pgfqpoint{2.162254in}{2.800391in}}%
\pgfpathlineto{\pgfqpoint{2.162254in}{2.803340in}}%
\pgfpathlineto{\pgfqpoint{2.166795in}{2.803340in}}%
\pgfpathlineto{\pgfqpoint{2.166795in}{2.800391in}}%
\pgfpathmoveto{\pgfqpoint{2.175877in}{2.776799in}}%
\pgfpathlineto{\pgfqpoint{2.175877in}{2.776799in}}%
\pgfpathlineto{\pgfqpoint{2.175877in}{2.779748in}}%
\pgfpathlineto{\pgfqpoint{2.180418in}{2.779748in}}%
\pgfpathlineto{\pgfqpoint{2.180418in}{2.776799in}}%
\pgfpathmoveto{\pgfqpoint{2.175877in}{2.779748in}}%
\pgfpathlineto{\pgfqpoint{2.175877in}{2.779748in}}%
\pgfpathlineto{\pgfqpoint{2.175877in}{2.782697in}}%
\pgfpathlineto{\pgfqpoint{2.180418in}{2.782697in}}%
\pgfpathlineto{\pgfqpoint{2.180418in}{2.779748in}}%
\pgfpathmoveto{\pgfqpoint{2.180418in}{2.776799in}}%
\pgfpathlineto{\pgfqpoint{2.180418in}{2.776799in}}%
\pgfpathlineto{\pgfqpoint{2.180418in}{2.779748in}}%
\pgfpathlineto{\pgfqpoint{2.184958in}{2.779748in}}%
\pgfpathlineto{\pgfqpoint{2.184958in}{2.776799in}}%
\pgfpathmoveto{\pgfqpoint{2.180418in}{2.779748in}}%
\pgfpathlineto{\pgfqpoint{2.180418in}{2.779748in}}%
\pgfpathlineto{\pgfqpoint{2.180418in}{2.782697in}}%
\pgfpathlineto{\pgfqpoint{2.184958in}{2.782697in}}%
\pgfpathlineto{\pgfqpoint{2.184958in}{2.779748in}}%
\pgfpathmoveto{\pgfqpoint{2.175877in}{2.782697in}}%
\pgfpathlineto{\pgfqpoint{2.175877in}{2.782697in}}%
\pgfpathlineto{\pgfqpoint{2.175877in}{2.785646in}}%
\pgfpathlineto{\pgfqpoint{2.180418in}{2.785646in}}%
\pgfpathlineto{\pgfqpoint{2.180418in}{2.782697in}}%
\pgfpathmoveto{\pgfqpoint{2.175877in}{2.785646in}}%
\pgfpathlineto{\pgfqpoint{2.175877in}{2.785646in}}%
\pgfpathlineto{\pgfqpoint{2.175877in}{2.788595in}}%
\pgfpathlineto{\pgfqpoint{2.180418in}{2.788595in}}%
\pgfpathlineto{\pgfqpoint{2.180418in}{2.785646in}}%
\pgfpathmoveto{\pgfqpoint{2.180418in}{2.782697in}}%
\pgfpathlineto{\pgfqpoint{2.180418in}{2.782697in}}%
\pgfpathlineto{\pgfqpoint{2.180418in}{2.785646in}}%
\pgfpathlineto{\pgfqpoint{2.184958in}{2.785646in}}%
\pgfpathlineto{\pgfqpoint{2.184958in}{2.782697in}}%
\pgfpathmoveto{\pgfqpoint{2.184958in}{2.770901in}}%
\pgfpathlineto{\pgfqpoint{2.184958in}{2.770901in}}%
\pgfpathlineto{\pgfqpoint{2.184958in}{2.773850in}}%
\pgfpathlineto{\pgfqpoint{2.189499in}{2.773850in}}%
\pgfpathlineto{\pgfqpoint{2.189499in}{2.770901in}}%
\pgfpathmoveto{\pgfqpoint{2.184958in}{2.773850in}}%
\pgfpathlineto{\pgfqpoint{2.184958in}{2.773850in}}%
\pgfpathlineto{\pgfqpoint{2.184958in}{2.776799in}}%
\pgfpathlineto{\pgfqpoint{2.189499in}{2.776799in}}%
\pgfpathlineto{\pgfqpoint{2.189499in}{2.773850in}}%
\pgfpathmoveto{\pgfqpoint{2.189499in}{2.770901in}}%
\pgfpathlineto{\pgfqpoint{2.189499in}{2.770901in}}%
\pgfpathlineto{\pgfqpoint{2.189499in}{2.773850in}}%
\pgfpathlineto{\pgfqpoint{2.194040in}{2.773850in}}%
\pgfpathlineto{\pgfqpoint{2.194040in}{2.770901in}}%
\pgfpathmoveto{\pgfqpoint{2.189499in}{2.773850in}}%
\pgfpathlineto{\pgfqpoint{2.189499in}{2.773850in}}%
\pgfpathlineto{\pgfqpoint{2.189499in}{2.776799in}}%
\pgfpathlineto{\pgfqpoint{2.194040in}{2.776799in}}%
\pgfpathlineto{\pgfqpoint{2.194040in}{2.773850in}}%
\pgfpathmoveto{\pgfqpoint{2.194040in}{2.765003in}}%
\pgfpathlineto{\pgfqpoint{2.194040in}{2.765003in}}%
\pgfpathlineto{\pgfqpoint{2.194040in}{2.767952in}}%
\pgfpathlineto{\pgfqpoint{2.198581in}{2.767952in}}%
\pgfpathlineto{\pgfqpoint{2.198581in}{2.765003in}}%
\pgfpathmoveto{\pgfqpoint{2.194040in}{2.767952in}}%
\pgfpathlineto{\pgfqpoint{2.194040in}{2.767952in}}%
\pgfpathlineto{\pgfqpoint{2.194040in}{2.770901in}}%
\pgfpathlineto{\pgfqpoint{2.198581in}{2.770901in}}%
\pgfpathlineto{\pgfqpoint{2.198581in}{2.767952in}}%
\pgfpathmoveto{\pgfqpoint{2.198581in}{2.765003in}}%
\pgfpathlineto{\pgfqpoint{2.198581in}{2.765003in}}%
\pgfpathlineto{\pgfqpoint{2.198581in}{2.767952in}}%
\pgfpathlineto{\pgfqpoint{2.203122in}{2.767952in}}%
\pgfpathlineto{\pgfqpoint{2.203122in}{2.765003in}}%
\pgfpathmoveto{\pgfqpoint{2.198581in}{2.767952in}}%
\pgfpathlineto{\pgfqpoint{2.198581in}{2.767952in}}%
\pgfpathlineto{\pgfqpoint{2.198581in}{2.770901in}}%
\pgfpathlineto{\pgfqpoint{2.203122in}{2.770901in}}%
\pgfpathlineto{\pgfqpoint{2.203122in}{2.767952in}}%
\pgfpathmoveto{\pgfqpoint{2.194040in}{2.770901in}}%
\pgfpathlineto{\pgfqpoint{2.194040in}{2.770901in}}%
\pgfpathlineto{\pgfqpoint{2.194040in}{2.773850in}}%
\pgfpathlineto{\pgfqpoint{2.198581in}{2.773850in}}%
\pgfpathlineto{\pgfqpoint{2.198581in}{2.770901in}}%
\pgfpathmoveto{\pgfqpoint{2.184958in}{2.776799in}}%
\pgfpathlineto{\pgfqpoint{2.184958in}{2.776799in}}%
\pgfpathlineto{\pgfqpoint{2.184958in}{2.779748in}}%
\pgfpathlineto{\pgfqpoint{2.189499in}{2.779748in}}%
\pgfpathlineto{\pgfqpoint{2.189499in}{2.776799in}}%
\pgfpathmoveto{\pgfqpoint{2.184958in}{2.779748in}}%
\pgfpathlineto{\pgfqpoint{2.184958in}{2.779748in}}%
\pgfpathlineto{\pgfqpoint{2.184958in}{2.782697in}}%
\pgfpathlineto{\pgfqpoint{2.189499in}{2.782697in}}%
\pgfpathlineto{\pgfqpoint{2.189499in}{2.779748in}}%
\pgfpathmoveto{\pgfqpoint{2.189499in}{2.776799in}}%
\pgfpathlineto{\pgfqpoint{2.189499in}{2.776799in}}%
\pgfpathlineto{\pgfqpoint{2.189499in}{2.779748in}}%
\pgfpathlineto{\pgfqpoint{2.194040in}{2.779748in}}%
\pgfpathlineto{\pgfqpoint{2.194040in}{2.776799in}}%
\pgfpathmoveto{\pgfqpoint{2.166795in}{2.788595in}}%
\pgfpathlineto{\pgfqpoint{2.166795in}{2.788595in}}%
\pgfpathlineto{\pgfqpoint{2.166795in}{2.791544in}}%
\pgfpathlineto{\pgfqpoint{2.171336in}{2.791544in}}%
\pgfpathlineto{\pgfqpoint{2.171336in}{2.788595in}}%
\pgfpathmoveto{\pgfqpoint{2.166795in}{2.791544in}}%
\pgfpathlineto{\pgfqpoint{2.166795in}{2.791544in}}%
\pgfpathlineto{\pgfqpoint{2.166795in}{2.794493in}}%
\pgfpathlineto{\pgfqpoint{2.171336in}{2.794493in}}%
\pgfpathlineto{\pgfqpoint{2.171336in}{2.791544in}}%
\pgfpathmoveto{\pgfqpoint{2.171336in}{2.788595in}}%
\pgfpathlineto{\pgfqpoint{2.171336in}{2.788595in}}%
\pgfpathlineto{\pgfqpoint{2.171336in}{2.791544in}}%
\pgfpathlineto{\pgfqpoint{2.175877in}{2.791544in}}%
\pgfpathlineto{\pgfqpoint{2.175877in}{2.788595in}}%
\pgfpathmoveto{\pgfqpoint{2.171336in}{2.791544in}}%
\pgfpathlineto{\pgfqpoint{2.171336in}{2.791544in}}%
\pgfpathlineto{\pgfqpoint{2.171336in}{2.794493in}}%
\pgfpathlineto{\pgfqpoint{2.175877in}{2.794493in}}%
\pgfpathlineto{\pgfqpoint{2.175877in}{2.791544in}}%
\pgfpathmoveto{\pgfqpoint{2.166795in}{2.794493in}}%
\pgfpathlineto{\pgfqpoint{2.166795in}{2.794493in}}%
\pgfpathlineto{\pgfqpoint{2.166795in}{2.797442in}}%
\pgfpathlineto{\pgfqpoint{2.171336in}{2.797442in}}%
\pgfpathlineto{\pgfqpoint{2.171336in}{2.794493in}}%
\pgfpathmoveto{\pgfqpoint{2.175877in}{2.788595in}}%
\pgfpathlineto{\pgfqpoint{2.175877in}{2.788595in}}%
\pgfpathlineto{\pgfqpoint{2.175877in}{2.791544in}}%
\pgfpathlineto{\pgfqpoint{2.180418in}{2.791544in}}%
\pgfpathlineto{\pgfqpoint{2.180418in}{2.788595in}}%
\pgfpathmoveto{\pgfqpoint{2.130469in}{2.818086in}}%
\pgfpathlineto{\pgfqpoint{2.130469in}{2.818086in}}%
\pgfpathlineto{\pgfqpoint{2.130469in}{2.821035in}}%
\pgfpathlineto{\pgfqpoint{2.135010in}{2.821035in}}%
\pgfpathlineto{\pgfqpoint{2.135010in}{2.818086in}}%
\pgfpathmoveto{\pgfqpoint{2.130469in}{2.821035in}}%
\pgfpathlineto{\pgfqpoint{2.130469in}{2.821035in}}%
\pgfpathlineto{\pgfqpoint{2.130469in}{2.823984in}}%
\pgfpathlineto{\pgfqpoint{2.135010in}{2.823984in}}%
\pgfpathlineto{\pgfqpoint{2.135010in}{2.821035in}}%
\pgfpathmoveto{\pgfqpoint{2.135010in}{2.818086in}}%
\pgfpathlineto{\pgfqpoint{2.135010in}{2.818086in}}%
\pgfpathlineto{\pgfqpoint{2.135010in}{2.821035in}}%
\pgfpathlineto{\pgfqpoint{2.139550in}{2.821035in}}%
\pgfpathlineto{\pgfqpoint{2.139550in}{2.818086in}}%
\pgfpathmoveto{\pgfqpoint{2.135010in}{2.821035in}}%
\pgfpathlineto{\pgfqpoint{2.135010in}{2.821035in}}%
\pgfpathlineto{\pgfqpoint{2.135010in}{2.823984in}}%
\pgfpathlineto{\pgfqpoint{2.139550in}{2.823984in}}%
\pgfpathlineto{\pgfqpoint{2.139550in}{2.821035in}}%
\pgfpathmoveto{\pgfqpoint{2.139550in}{2.812188in}}%
\pgfpathlineto{\pgfqpoint{2.139550in}{2.812188in}}%
\pgfpathlineto{\pgfqpoint{2.139550in}{2.815137in}}%
\pgfpathlineto{\pgfqpoint{2.144091in}{2.815137in}}%
\pgfpathlineto{\pgfqpoint{2.144091in}{2.812188in}}%
\pgfpathmoveto{\pgfqpoint{2.139550in}{2.815137in}}%
\pgfpathlineto{\pgfqpoint{2.139550in}{2.815137in}}%
\pgfpathlineto{\pgfqpoint{2.139550in}{2.818086in}}%
\pgfpathlineto{\pgfqpoint{2.144091in}{2.818086in}}%
\pgfpathlineto{\pgfqpoint{2.144091in}{2.815137in}}%
\pgfpathmoveto{\pgfqpoint{2.144091in}{2.812188in}}%
\pgfpathlineto{\pgfqpoint{2.144091in}{2.812188in}}%
\pgfpathlineto{\pgfqpoint{2.144091in}{2.815137in}}%
\pgfpathlineto{\pgfqpoint{2.148632in}{2.815137in}}%
\pgfpathlineto{\pgfqpoint{2.148632in}{2.812188in}}%
\pgfpathmoveto{\pgfqpoint{2.144091in}{2.815137in}}%
\pgfpathlineto{\pgfqpoint{2.144091in}{2.815137in}}%
\pgfpathlineto{\pgfqpoint{2.144091in}{2.818086in}}%
\pgfpathlineto{\pgfqpoint{2.148632in}{2.818086in}}%
\pgfpathlineto{\pgfqpoint{2.148632in}{2.815137in}}%
\pgfpathmoveto{\pgfqpoint{2.139550in}{2.818086in}}%
\pgfpathlineto{\pgfqpoint{2.139550in}{2.818086in}}%
\pgfpathlineto{\pgfqpoint{2.139550in}{2.821035in}}%
\pgfpathlineto{\pgfqpoint{2.144091in}{2.821035in}}%
\pgfpathlineto{\pgfqpoint{2.144091in}{2.818086in}}%
\pgfpathmoveto{\pgfqpoint{2.130469in}{2.823984in}}%
\pgfpathlineto{\pgfqpoint{2.130469in}{2.823984in}}%
\pgfpathlineto{\pgfqpoint{2.130469in}{2.826933in}}%
\pgfpathlineto{\pgfqpoint{2.135010in}{2.826933in}}%
\pgfpathlineto{\pgfqpoint{2.135010in}{2.823984in}}%
\pgfpathmoveto{\pgfqpoint{2.130469in}{2.826933in}}%
\pgfpathlineto{\pgfqpoint{2.130469in}{2.826933in}}%
\pgfpathlineto{\pgfqpoint{2.130469in}{2.829882in}}%
\pgfpathlineto{\pgfqpoint{2.135010in}{2.829882in}}%
\pgfpathlineto{\pgfqpoint{2.135010in}{2.826933in}}%
\pgfpathmoveto{\pgfqpoint{2.135010in}{2.823984in}}%
\pgfpathlineto{\pgfqpoint{2.135010in}{2.823984in}}%
\pgfpathlineto{\pgfqpoint{2.135010in}{2.826933in}}%
\pgfpathlineto{\pgfqpoint{2.139550in}{2.826933in}}%
\pgfpathlineto{\pgfqpoint{2.139550in}{2.823984in}}%
\pgfpathmoveto{\pgfqpoint{2.148632in}{2.812188in}}%
\pgfpathlineto{\pgfqpoint{2.148632in}{2.812188in}}%
\pgfpathlineto{\pgfqpoint{2.148632in}{2.815137in}}%
\pgfpathlineto{\pgfqpoint{2.153173in}{2.815137in}}%
\pgfpathlineto{\pgfqpoint{2.153173in}{2.812188in}}%
\pgfpathmoveto{\pgfqpoint{2.057816in}{2.877068in}}%
\pgfpathlineto{\pgfqpoint{2.057816in}{2.877068in}}%
\pgfpathlineto{\pgfqpoint{2.057816in}{2.880017in}}%
\pgfpathlineto{\pgfqpoint{2.062357in}{2.880017in}}%
\pgfpathlineto{\pgfqpoint{2.062357in}{2.877068in}}%
\pgfpathmoveto{\pgfqpoint{2.057816in}{2.880017in}}%
\pgfpathlineto{\pgfqpoint{2.057816in}{2.880017in}}%
\pgfpathlineto{\pgfqpoint{2.057816in}{2.882966in}}%
\pgfpathlineto{\pgfqpoint{2.062357in}{2.882966in}}%
\pgfpathlineto{\pgfqpoint{2.062357in}{2.880017in}}%
\pgfpathmoveto{\pgfqpoint{2.062357in}{2.877068in}}%
\pgfpathlineto{\pgfqpoint{2.062357in}{2.877068in}}%
\pgfpathlineto{\pgfqpoint{2.062357in}{2.880017in}}%
\pgfpathlineto{\pgfqpoint{2.066898in}{2.880017in}}%
\pgfpathlineto{\pgfqpoint{2.066898in}{2.877068in}}%
\pgfpathmoveto{\pgfqpoint{2.062357in}{2.880017in}}%
\pgfpathlineto{\pgfqpoint{2.062357in}{2.880017in}}%
\pgfpathlineto{\pgfqpoint{2.062357in}{2.882966in}}%
\pgfpathlineto{\pgfqpoint{2.066898in}{2.882966in}}%
\pgfpathlineto{\pgfqpoint{2.066898in}{2.880017in}}%
\pgfpathmoveto{\pgfqpoint{2.066898in}{2.871169in}}%
\pgfpathlineto{\pgfqpoint{2.066898in}{2.871169in}}%
\pgfpathlineto{\pgfqpoint{2.066898in}{2.874119in}}%
\pgfpathlineto{\pgfqpoint{2.071439in}{2.874119in}}%
\pgfpathlineto{\pgfqpoint{2.071439in}{2.871169in}}%
\pgfpathmoveto{\pgfqpoint{2.066898in}{2.874119in}}%
\pgfpathlineto{\pgfqpoint{2.066898in}{2.874119in}}%
\pgfpathlineto{\pgfqpoint{2.066898in}{2.877068in}}%
\pgfpathlineto{\pgfqpoint{2.071439in}{2.877068in}}%
\pgfpathlineto{\pgfqpoint{2.071439in}{2.874119in}}%
\pgfpathmoveto{\pgfqpoint{2.071439in}{2.871169in}}%
\pgfpathlineto{\pgfqpoint{2.071439in}{2.871169in}}%
\pgfpathlineto{\pgfqpoint{2.071439in}{2.874119in}}%
\pgfpathlineto{\pgfqpoint{2.075979in}{2.874119in}}%
\pgfpathlineto{\pgfqpoint{2.075979in}{2.871169in}}%
\pgfpathmoveto{\pgfqpoint{2.071439in}{2.874119in}}%
\pgfpathlineto{\pgfqpoint{2.071439in}{2.874119in}}%
\pgfpathlineto{\pgfqpoint{2.071439in}{2.877068in}}%
\pgfpathlineto{\pgfqpoint{2.075979in}{2.877068in}}%
\pgfpathlineto{\pgfqpoint{2.075979in}{2.874119in}}%
\pgfpathmoveto{\pgfqpoint{2.066898in}{2.877068in}}%
\pgfpathlineto{\pgfqpoint{2.066898in}{2.877068in}}%
\pgfpathlineto{\pgfqpoint{2.066898in}{2.880017in}}%
\pgfpathlineto{\pgfqpoint{2.071439in}{2.880017in}}%
\pgfpathlineto{\pgfqpoint{2.071439in}{2.877068in}}%
\pgfpathmoveto{\pgfqpoint{2.066898in}{2.880017in}}%
\pgfpathlineto{\pgfqpoint{2.066898in}{2.880017in}}%
\pgfpathlineto{\pgfqpoint{2.066898in}{2.882966in}}%
\pgfpathlineto{\pgfqpoint{2.071439in}{2.882966in}}%
\pgfpathlineto{\pgfqpoint{2.071439in}{2.880017in}}%
\pgfpathmoveto{\pgfqpoint{2.071439in}{2.877068in}}%
\pgfpathlineto{\pgfqpoint{2.071439in}{2.877068in}}%
\pgfpathlineto{\pgfqpoint{2.071439in}{2.880017in}}%
\pgfpathlineto{\pgfqpoint{2.075979in}{2.880017in}}%
\pgfpathlineto{\pgfqpoint{2.075979in}{2.877068in}}%
\pgfpathmoveto{\pgfqpoint{2.075979in}{2.865271in}}%
\pgfpathlineto{\pgfqpoint{2.075979in}{2.865271in}}%
\pgfpathlineto{\pgfqpoint{2.075979in}{2.868220in}}%
\pgfpathlineto{\pgfqpoint{2.080520in}{2.868220in}}%
\pgfpathlineto{\pgfqpoint{2.080520in}{2.865271in}}%
\pgfpathmoveto{\pgfqpoint{2.075979in}{2.868220in}}%
\pgfpathlineto{\pgfqpoint{2.075979in}{2.868220in}}%
\pgfpathlineto{\pgfqpoint{2.075979in}{2.871169in}}%
\pgfpathlineto{\pgfqpoint{2.080520in}{2.871169in}}%
\pgfpathlineto{\pgfqpoint{2.080520in}{2.868220in}}%
\pgfpathmoveto{\pgfqpoint{2.080520in}{2.865271in}}%
\pgfpathlineto{\pgfqpoint{2.080520in}{2.865271in}}%
\pgfpathlineto{\pgfqpoint{2.080520in}{2.868220in}}%
\pgfpathlineto{\pgfqpoint{2.085061in}{2.868220in}}%
\pgfpathlineto{\pgfqpoint{2.085061in}{2.865271in}}%
\pgfpathmoveto{\pgfqpoint{2.080520in}{2.868220in}}%
\pgfpathlineto{\pgfqpoint{2.080520in}{2.868220in}}%
\pgfpathlineto{\pgfqpoint{2.080520in}{2.871169in}}%
\pgfpathlineto{\pgfqpoint{2.085061in}{2.871169in}}%
\pgfpathlineto{\pgfqpoint{2.085061in}{2.868220in}}%
\pgfpathmoveto{\pgfqpoint{2.085061in}{2.859372in}}%
\pgfpathlineto{\pgfqpoint{2.085061in}{2.859372in}}%
\pgfpathlineto{\pgfqpoint{2.085061in}{2.862322in}}%
\pgfpathlineto{\pgfqpoint{2.089602in}{2.862322in}}%
\pgfpathlineto{\pgfqpoint{2.089602in}{2.859372in}}%
\pgfpathmoveto{\pgfqpoint{2.085061in}{2.862322in}}%
\pgfpathlineto{\pgfqpoint{2.085061in}{2.862322in}}%
\pgfpathlineto{\pgfqpoint{2.085061in}{2.865271in}}%
\pgfpathlineto{\pgfqpoint{2.089602in}{2.865271in}}%
\pgfpathlineto{\pgfqpoint{2.089602in}{2.862322in}}%
\pgfpathmoveto{\pgfqpoint{2.089602in}{2.859372in}}%
\pgfpathlineto{\pgfqpoint{2.089602in}{2.859372in}}%
\pgfpathlineto{\pgfqpoint{2.089602in}{2.862322in}}%
\pgfpathlineto{\pgfqpoint{2.094143in}{2.862322in}}%
\pgfpathlineto{\pgfqpoint{2.094143in}{2.859372in}}%
\pgfpathmoveto{\pgfqpoint{2.089602in}{2.862322in}}%
\pgfpathlineto{\pgfqpoint{2.089602in}{2.862322in}}%
\pgfpathlineto{\pgfqpoint{2.089602in}{2.865271in}}%
\pgfpathlineto{\pgfqpoint{2.094143in}{2.865271in}}%
\pgfpathlineto{\pgfqpoint{2.094143in}{2.862322in}}%
\pgfpathmoveto{\pgfqpoint{2.085061in}{2.865271in}}%
\pgfpathlineto{\pgfqpoint{2.085061in}{2.865271in}}%
\pgfpathlineto{\pgfqpoint{2.085061in}{2.868220in}}%
\pgfpathlineto{\pgfqpoint{2.089602in}{2.868220in}}%
\pgfpathlineto{\pgfqpoint{2.089602in}{2.865271in}}%
\pgfpathmoveto{\pgfqpoint{2.075979in}{2.871169in}}%
\pgfpathlineto{\pgfqpoint{2.075979in}{2.871169in}}%
\pgfpathlineto{\pgfqpoint{2.075979in}{2.874119in}}%
\pgfpathlineto{\pgfqpoint{2.080520in}{2.874119in}}%
\pgfpathlineto{\pgfqpoint{2.080520in}{2.871169in}}%
\pgfpathmoveto{\pgfqpoint{2.075979in}{2.874119in}}%
\pgfpathlineto{\pgfqpoint{2.075979in}{2.874119in}}%
\pgfpathlineto{\pgfqpoint{2.075979in}{2.877068in}}%
\pgfpathlineto{\pgfqpoint{2.080520in}{2.877068in}}%
\pgfpathlineto{\pgfqpoint{2.080520in}{2.874119in}}%
\pgfpathmoveto{\pgfqpoint{2.080520in}{2.871169in}}%
\pgfpathlineto{\pgfqpoint{2.080520in}{2.871169in}}%
\pgfpathlineto{\pgfqpoint{2.080520in}{2.874119in}}%
\pgfpathlineto{\pgfqpoint{2.085061in}{2.874119in}}%
\pgfpathlineto{\pgfqpoint{2.085061in}{2.871169in}}%
\pgfpathmoveto{\pgfqpoint{2.057816in}{2.882966in}}%
\pgfpathlineto{\pgfqpoint{2.057816in}{2.882966in}}%
\pgfpathlineto{\pgfqpoint{2.057816in}{2.885916in}}%
\pgfpathlineto{\pgfqpoint{2.062357in}{2.885916in}}%
\pgfpathlineto{\pgfqpoint{2.062357in}{2.882966in}}%
\pgfpathmoveto{\pgfqpoint{2.057816in}{2.885916in}}%
\pgfpathlineto{\pgfqpoint{2.057816in}{2.885916in}}%
\pgfpathlineto{\pgfqpoint{2.057816in}{2.888865in}}%
\pgfpathlineto{\pgfqpoint{2.062357in}{2.888865in}}%
\pgfpathlineto{\pgfqpoint{2.062357in}{2.885916in}}%
\pgfpathmoveto{\pgfqpoint{2.062357in}{2.882966in}}%
\pgfpathlineto{\pgfqpoint{2.062357in}{2.882966in}}%
\pgfpathlineto{\pgfqpoint{2.062357in}{2.885916in}}%
\pgfpathlineto{\pgfqpoint{2.066898in}{2.885916in}}%
\pgfpathlineto{\pgfqpoint{2.066898in}{2.882966in}}%
\pgfpathmoveto{\pgfqpoint{2.062357in}{2.885916in}}%
\pgfpathlineto{\pgfqpoint{2.062357in}{2.885916in}}%
\pgfpathlineto{\pgfqpoint{2.062357in}{2.888865in}}%
\pgfpathlineto{\pgfqpoint{2.066898in}{2.888865in}}%
\pgfpathlineto{\pgfqpoint{2.066898in}{2.885916in}}%
\pgfpathmoveto{\pgfqpoint{2.057816in}{2.888865in}}%
\pgfpathlineto{\pgfqpoint{2.057816in}{2.888865in}}%
\pgfpathlineto{\pgfqpoint{2.057816in}{2.891814in}}%
\pgfpathlineto{\pgfqpoint{2.062357in}{2.891814in}}%
\pgfpathlineto{\pgfqpoint{2.062357in}{2.888865in}}%
\pgfpathmoveto{\pgfqpoint{2.094143in}{2.859372in}}%
\pgfpathlineto{\pgfqpoint{2.094143in}{2.859372in}}%
\pgfpathlineto{\pgfqpoint{2.094143in}{2.862322in}}%
\pgfpathlineto{\pgfqpoint{2.098683in}{2.862322in}}%
\pgfpathlineto{\pgfqpoint{2.098683in}{2.859372in}}%
\pgfpathmoveto{\pgfqpoint{2.339357in}{2.635233in}}%
\pgfpathlineto{\pgfqpoint{2.339357in}{2.635233in}}%
\pgfpathlineto{\pgfqpoint{2.339357in}{2.638182in}}%
\pgfpathlineto{\pgfqpoint{2.343898in}{2.638182in}}%
\pgfpathlineto{\pgfqpoint{2.343898in}{2.635233in}}%
\pgfpathmoveto{\pgfqpoint{2.339357in}{2.638182in}}%
\pgfpathlineto{\pgfqpoint{2.339357in}{2.638182in}}%
\pgfpathlineto{\pgfqpoint{2.339357in}{2.641132in}}%
\pgfpathlineto{\pgfqpoint{2.343898in}{2.641132in}}%
\pgfpathlineto{\pgfqpoint{2.343898in}{2.638182in}}%
\pgfpathmoveto{\pgfqpoint{2.343898in}{2.635233in}}%
\pgfpathlineto{\pgfqpoint{2.343898in}{2.635233in}}%
\pgfpathlineto{\pgfqpoint{2.343898in}{2.638182in}}%
\pgfpathlineto{\pgfqpoint{2.348439in}{2.638182in}}%
\pgfpathlineto{\pgfqpoint{2.348439in}{2.635233in}}%
\pgfpathmoveto{\pgfqpoint{2.343898in}{2.638182in}}%
\pgfpathlineto{\pgfqpoint{2.343898in}{2.638182in}}%
\pgfpathlineto{\pgfqpoint{2.343898in}{2.641132in}}%
\pgfpathlineto{\pgfqpoint{2.348439in}{2.641132in}}%
\pgfpathlineto{\pgfqpoint{2.348439in}{2.638182in}}%
\pgfpathmoveto{\pgfqpoint{2.339357in}{2.641132in}}%
\pgfpathlineto{\pgfqpoint{2.339357in}{2.641132in}}%
\pgfpathlineto{\pgfqpoint{2.339357in}{2.644081in}}%
\pgfpathlineto{\pgfqpoint{2.343898in}{2.644081in}}%
\pgfpathlineto{\pgfqpoint{2.343898in}{2.641132in}}%
\pgfpathmoveto{\pgfqpoint{2.339357in}{2.644081in}}%
\pgfpathlineto{\pgfqpoint{2.339357in}{2.644081in}}%
\pgfpathlineto{\pgfqpoint{2.339357in}{2.647030in}}%
\pgfpathlineto{\pgfqpoint{2.343898in}{2.647030in}}%
\pgfpathlineto{\pgfqpoint{2.343898in}{2.644081in}}%
\pgfpathmoveto{\pgfqpoint{2.343898in}{2.641132in}}%
\pgfpathlineto{\pgfqpoint{2.343898in}{2.641132in}}%
\pgfpathlineto{\pgfqpoint{2.343898in}{2.644081in}}%
\pgfpathlineto{\pgfqpoint{2.348439in}{2.644081in}}%
\pgfpathlineto{\pgfqpoint{2.348439in}{2.641132in}}%
\pgfpathmoveto{\pgfqpoint{2.321192in}{2.652928in}}%
\pgfpathlineto{\pgfqpoint{2.321192in}{2.652928in}}%
\pgfpathlineto{\pgfqpoint{2.321192in}{2.655878in}}%
\pgfpathlineto{\pgfqpoint{2.325733in}{2.655878in}}%
\pgfpathlineto{\pgfqpoint{2.325733in}{2.652928in}}%
\pgfpathmoveto{\pgfqpoint{2.321192in}{2.655878in}}%
\pgfpathlineto{\pgfqpoint{2.321192in}{2.655878in}}%
\pgfpathlineto{\pgfqpoint{2.321192in}{2.658827in}}%
\pgfpathlineto{\pgfqpoint{2.325733in}{2.658827in}}%
\pgfpathlineto{\pgfqpoint{2.325733in}{2.655878in}}%
\pgfpathmoveto{\pgfqpoint{2.325733in}{2.652928in}}%
\pgfpathlineto{\pgfqpoint{2.325733in}{2.652928in}}%
\pgfpathlineto{\pgfqpoint{2.325733in}{2.655878in}}%
\pgfpathlineto{\pgfqpoint{2.330274in}{2.655878in}}%
\pgfpathlineto{\pgfqpoint{2.330274in}{2.652928in}}%
\pgfpathmoveto{\pgfqpoint{2.325733in}{2.655878in}}%
\pgfpathlineto{\pgfqpoint{2.325733in}{2.655878in}}%
\pgfpathlineto{\pgfqpoint{2.325733in}{2.658827in}}%
\pgfpathlineto{\pgfqpoint{2.330274in}{2.658827in}}%
\pgfpathlineto{\pgfqpoint{2.330274in}{2.655878in}}%
\pgfpathmoveto{\pgfqpoint{2.312110in}{2.658827in}}%
\pgfpathlineto{\pgfqpoint{2.312110in}{2.658827in}}%
\pgfpathlineto{\pgfqpoint{2.312110in}{2.661776in}}%
\pgfpathlineto{\pgfqpoint{2.316651in}{2.661776in}}%
\pgfpathlineto{\pgfqpoint{2.316651in}{2.658827in}}%
\pgfpathmoveto{\pgfqpoint{2.312110in}{2.661776in}}%
\pgfpathlineto{\pgfqpoint{2.312110in}{2.661776in}}%
\pgfpathlineto{\pgfqpoint{2.312110in}{2.664725in}}%
\pgfpathlineto{\pgfqpoint{2.316651in}{2.664725in}}%
\pgfpathlineto{\pgfqpoint{2.316651in}{2.661776in}}%
\pgfpathmoveto{\pgfqpoint{2.316651in}{2.658827in}}%
\pgfpathlineto{\pgfqpoint{2.316651in}{2.658827in}}%
\pgfpathlineto{\pgfqpoint{2.316651in}{2.661776in}}%
\pgfpathlineto{\pgfqpoint{2.321192in}{2.661776in}}%
\pgfpathlineto{\pgfqpoint{2.321192in}{2.658827in}}%
\pgfpathmoveto{\pgfqpoint{2.316651in}{2.661776in}}%
\pgfpathlineto{\pgfqpoint{2.316651in}{2.661776in}}%
\pgfpathlineto{\pgfqpoint{2.316651in}{2.664725in}}%
\pgfpathlineto{\pgfqpoint{2.321192in}{2.664725in}}%
\pgfpathlineto{\pgfqpoint{2.321192in}{2.661776in}}%
\pgfpathmoveto{\pgfqpoint{2.312110in}{2.664725in}}%
\pgfpathlineto{\pgfqpoint{2.312110in}{2.664725in}}%
\pgfpathlineto{\pgfqpoint{2.312110in}{2.667674in}}%
\pgfpathlineto{\pgfqpoint{2.316651in}{2.667674in}}%
\pgfpathlineto{\pgfqpoint{2.316651in}{2.664725in}}%
\pgfpathmoveto{\pgfqpoint{2.312110in}{2.667674in}}%
\pgfpathlineto{\pgfqpoint{2.312110in}{2.667674in}}%
\pgfpathlineto{\pgfqpoint{2.312110in}{2.670624in}}%
\pgfpathlineto{\pgfqpoint{2.316651in}{2.670624in}}%
\pgfpathlineto{\pgfqpoint{2.316651in}{2.667674in}}%
\pgfpathmoveto{\pgfqpoint{2.316651in}{2.664725in}}%
\pgfpathlineto{\pgfqpoint{2.316651in}{2.664725in}}%
\pgfpathlineto{\pgfqpoint{2.316651in}{2.667674in}}%
\pgfpathlineto{\pgfqpoint{2.321192in}{2.667674in}}%
\pgfpathlineto{\pgfqpoint{2.321192in}{2.664725in}}%
\pgfpathmoveto{\pgfqpoint{2.321192in}{2.658827in}}%
\pgfpathlineto{\pgfqpoint{2.321192in}{2.658827in}}%
\pgfpathlineto{\pgfqpoint{2.321192in}{2.661776in}}%
\pgfpathlineto{\pgfqpoint{2.325733in}{2.661776in}}%
\pgfpathlineto{\pgfqpoint{2.325733in}{2.658827in}}%
\pgfpathmoveto{\pgfqpoint{2.321192in}{2.661776in}}%
\pgfpathlineto{\pgfqpoint{2.321192in}{2.661776in}}%
\pgfpathlineto{\pgfqpoint{2.321192in}{2.664725in}}%
\pgfpathlineto{\pgfqpoint{2.325733in}{2.664725in}}%
\pgfpathlineto{\pgfqpoint{2.325733in}{2.661776in}}%
\pgfpathmoveto{\pgfqpoint{2.325733in}{2.658827in}}%
\pgfpathlineto{\pgfqpoint{2.325733in}{2.658827in}}%
\pgfpathlineto{\pgfqpoint{2.325733in}{2.661776in}}%
\pgfpathlineto{\pgfqpoint{2.330274in}{2.661776in}}%
\pgfpathlineto{\pgfqpoint{2.330274in}{2.658827in}}%
\pgfpathmoveto{\pgfqpoint{2.330274in}{2.647030in}}%
\pgfpathlineto{\pgfqpoint{2.330274in}{2.647030in}}%
\pgfpathlineto{\pgfqpoint{2.330274in}{2.649979in}}%
\pgfpathlineto{\pgfqpoint{2.334816in}{2.649979in}}%
\pgfpathlineto{\pgfqpoint{2.334816in}{2.647030in}}%
\pgfpathmoveto{\pgfqpoint{2.330274in}{2.649979in}}%
\pgfpathlineto{\pgfqpoint{2.330274in}{2.649979in}}%
\pgfpathlineto{\pgfqpoint{2.330274in}{2.652928in}}%
\pgfpathlineto{\pgfqpoint{2.334816in}{2.652928in}}%
\pgfpathlineto{\pgfqpoint{2.334816in}{2.649979in}}%
\pgfpathmoveto{\pgfqpoint{2.334816in}{2.647030in}}%
\pgfpathlineto{\pgfqpoint{2.334816in}{2.647030in}}%
\pgfpathlineto{\pgfqpoint{2.334816in}{2.649979in}}%
\pgfpathlineto{\pgfqpoint{2.339357in}{2.649979in}}%
\pgfpathlineto{\pgfqpoint{2.339357in}{2.647030in}}%
\pgfpathmoveto{\pgfqpoint{2.334816in}{2.649979in}}%
\pgfpathlineto{\pgfqpoint{2.334816in}{2.649979in}}%
\pgfpathlineto{\pgfqpoint{2.334816in}{2.652928in}}%
\pgfpathlineto{\pgfqpoint{2.339357in}{2.652928in}}%
\pgfpathlineto{\pgfqpoint{2.339357in}{2.649979in}}%
\pgfpathmoveto{\pgfqpoint{2.330274in}{2.652928in}}%
\pgfpathlineto{\pgfqpoint{2.330274in}{2.652928in}}%
\pgfpathlineto{\pgfqpoint{2.330274in}{2.655878in}}%
\pgfpathlineto{\pgfqpoint{2.334816in}{2.655878in}}%
\pgfpathlineto{\pgfqpoint{2.334816in}{2.652928in}}%
\pgfpathmoveto{\pgfqpoint{2.339357in}{2.647030in}}%
\pgfpathlineto{\pgfqpoint{2.339357in}{2.647030in}}%
\pgfpathlineto{\pgfqpoint{2.339357in}{2.649979in}}%
\pgfpathlineto{\pgfqpoint{2.343898in}{2.649979in}}%
\pgfpathlineto{\pgfqpoint{2.343898in}{2.647030in}}%
\pgfpathmoveto{\pgfqpoint{2.266698in}{2.700117in}}%
\pgfpathlineto{\pgfqpoint{2.266698in}{2.700117in}}%
\pgfpathlineto{\pgfqpoint{2.266698in}{2.703066in}}%
\pgfpathlineto{\pgfqpoint{2.271239in}{2.703066in}}%
\pgfpathlineto{\pgfqpoint{2.271239in}{2.700117in}}%
\pgfpathmoveto{\pgfqpoint{2.266698in}{2.703066in}}%
\pgfpathlineto{\pgfqpoint{2.266698in}{2.703066in}}%
\pgfpathlineto{\pgfqpoint{2.266698in}{2.706016in}}%
\pgfpathlineto{\pgfqpoint{2.271239in}{2.706016in}}%
\pgfpathlineto{\pgfqpoint{2.271239in}{2.703066in}}%
\pgfpathmoveto{\pgfqpoint{2.271239in}{2.700117in}}%
\pgfpathlineto{\pgfqpoint{2.271239in}{2.700117in}}%
\pgfpathlineto{\pgfqpoint{2.271239in}{2.703066in}}%
\pgfpathlineto{\pgfqpoint{2.275780in}{2.703066in}}%
\pgfpathlineto{\pgfqpoint{2.275780in}{2.700117in}}%
\pgfpathmoveto{\pgfqpoint{2.271239in}{2.703066in}}%
\pgfpathlineto{\pgfqpoint{2.271239in}{2.703066in}}%
\pgfpathlineto{\pgfqpoint{2.271239in}{2.706016in}}%
\pgfpathlineto{\pgfqpoint{2.275780in}{2.706016in}}%
\pgfpathlineto{\pgfqpoint{2.275780in}{2.703066in}}%
\pgfpathmoveto{\pgfqpoint{2.257616in}{2.706016in}}%
\pgfpathlineto{\pgfqpoint{2.257616in}{2.706016in}}%
\pgfpathlineto{\pgfqpoint{2.257616in}{2.708965in}}%
\pgfpathlineto{\pgfqpoint{2.262157in}{2.708965in}}%
\pgfpathlineto{\pgfqpoint{2.262157in}{2.706016in}}%
\pgfpathmoveto{\pgfqpoint{2.257616in}{2.708965in}}%
\pgfpathlineto{\pgfqpoint{2.257616in}{2.708965in}}%
\pgfpathlineto{\pgfqpoint{2.257616in}{2.711915in}}%
\pgfpathlineto{\pgfqpoint{2.262157in}{2.711915in}}%
\pgfpathlineto{\pgfqpoint{2.262157in}{2.708965in}}%
\pgfpathmoveto{\pgfqpoint{2.262157in}{2.706016in}}%
\pgfpathlineto{\pgfqpoint{2.262157in}{2.706016in}}%
\pgfpathlineto{\pgfqpoint{2.262157in}{2.708965in}}%
\pgfpathlineto{\pgfqpoint{2.266698in}{2.708965in}}%
\pgfpathlineto{\pgfqpoint{2.266698in}{2.706016in}}%
\pgfpathmoveto{\pgfqpoint{2.262157in}{2.708965in}}%
\pgfpathlineto{\pgfqpoint{2.262157in}{2.708965in}}%
\pgfpathlineto{\pgfqpoint{2.262157in}{2.711915in}}%
\pgfpathlineto{\pgfqpoint{2.266698in}{2.711915in}}%
\pgfpathlineto{\pgfqpoint{2.266698in}{2.708965in}}%
\pgfpathmoveto{\pgfqpoint{2.257616in}{2.711915in}}%
\pgfpathlineto{\pgfqpoint{2.257616in}{2.711915in}}%
\pgfpathlineto{\pgfqpoint{2.257616in}{2.714864in}}%
\pgfpathlineto{\pgfqpoint{2.262157in}{2.714864in}}%
\pgfpathlineto{\pgfqpoint{2.262157in}{2.711915in}}%
\pgfpathmoveto{\pgfqpoint{2.257616in}{2.714864in}}%
\pgfpathlineto{\pgfqpoint{2.257616in}{2.714864in}}%
\pgfpathlineto{\pgfqpoint{2.257616in}{2.717813in}}%
\pgfpathlineto{\pgfqpoint{2.262157in}{2.717813in}}%
\pgfpathlineto{\pgfqpoint{2.262157in}{2.714864in}}%
\pgfpathmoveto{\pgfqpoint{2.262157in}{2.711915in}}%
\pgfpathlineto{\pgfqpoint{2.262157in}{2.711915in}}%
\pgfpathlineto{\pgfqpoint{2.262157in}{2.714864in}}%
\pgfpathlineto{\pgfqpoint{2.266698in}{2.714864in}}%
\pgfpathlineto{\pgfqpoint{2.266698in}{2.711915in}}%
\pgfpathmoveto{\pgfqpoint{2.266698in}{2.706016in}}%
\pgfpathlineto{\pgfqpoint{2.266698in}{2.706016in}}%
\pgfpathlineto{\pgfqpoint{2.266698in}{2.708965in}}%
\pgfpathlineto{\pgfqpoint{2.271239in}{2.708965in}}%
\pgfpathlineto{\pgfqpoint{2.271239in}{2.706016in}}%
\pgfpathmoveto{\pgfqpoint{2.266698in}{2.708965in}}%
\pgfpathlineto{\pgfqpoint{2.266698in}{2.708965in}}%
\pgfpathlineto{\pgfqpoint{2.266698in}{2.711915in}}%
\pgfpathlineto{\pgfqpoint{2.271239in}{2.711915in}}%
\pgfpathlineto{\pgfqpoint{2.271239in}{2.708965in}}%
\pgfpathmoveto{\pgfqpoint{2.271239in}{2.706016in}}%
\pgfpathlineto{\pgfqpoint{2.271239in}{2.706016in}}%
\pgfpathlineto{\pgfqpoint{2.271239in}{2.708965in}}%
\pgfpathlineto{\pgfqpoint{2.275780in}{2.708965in}}%
\pgfpathlineto{\pgfqpoint{2.275780in}{2.706016in}}%
\pgfpathmoveto{\pgfqpoint{2.230369in}{2.729611in}}%
\pgfpathlineto{\pgfqpoint{2.230369in}{2.729611in}}%
\pgfpathlineto{\pgfqpoint{2.230369in}{2.732560in}}%
\pgfpathlineto{\pgfqpoint{2.234910in}{2.732560in}}%
\pgfpathlineto{\pgfqpoint{2.234910in}{2.729611in}}%
\pgfpathmoveto{\pgfqpoint{2.230369in}{2.732560in}}%
\pgfpathlineto{\pgfqpoint{2.230369in}{2.732560in}}%
\pgfpathlineto{\pgfqpoint{2.230369in}{2.735509in}}%
\pgfpathlineto{\pgfqpoint{2.234910in}{2.735509in}}%
\pgfpathlineto{\pgfqpoint{2.234910in}{2.732560in}}%
\pgfpathmoveto{\pgfqpoint{2.234910in}{2.729611in}}%
\pgfpathlineto{\pgfqpoint{2.234910in}{2.729611in}}%
\pgfpathlineto{\pgfqpoint{2.234910in}{2.732560in}}%
\pgfpathlineto{\pgfqpoint{2.239451in}{2.732560in}}%
\pgfpathlineto{\pgfqpoint{2.239451in}{2.729611in}}%
\pgfpathmoveto{\pgfqpoint{2.234910in}{2.732560in}}%
\pgfpathlineto{\pgfqpoint{2.234910in}{2.732560in}}%
\pgfpathlineto{\pgfqpoint{2.234910in}{2.735509in}}%
\pgfpathlineto{\pgfqpoint{2.239451in}{2.735509in}}%
\pgfpathlineto{\pgfqpoint{2.239451in}{2.732560in}}%
\pgfpathmoveto{\pgfqpoint{2.230369in}{2.735509in}}%
\pgfpathlineto{\pgfqpoint{2.230369in}{2.735509in}}%
\pgfpathlineto{\pgfqpoint{2.230369in}{2.738459in}}%
\pgfpathlineto{\pgfqpoint{2.234910in}{2.738459in}}%
\pgfpathlineto{\pgfqpoint{2.234910in}{2.735509in}}%
\pgfpathmoveto{\pgfqpoint{2.230369in}{2.738459in}}%
\pgfpathlineto{\pgfqpoint{2.230369in}{2.738459in}}%
\pgfpathlineto{\pgfqpoint{2.230369in}{2.741408in}}%
\pgfpathlineto{\pgfqpoint{2.234910in}{2.741408in}}%
\pgfpathlineto{\pgfqpoint{2.234910in}{2.738459in}}%
\pgfpathmoveto{\pgfqpoint{2.234910in}{2.735509in}}%
\pgfpathlineto{\pgfqpoint{2.234910in}{2.735509in}}%
\pgfpathlineto{\pgfqpoint{2.234910in}{2.738459in}}%
\pgfpathlineto{\pgfqpoint{2.239451in}{2.738459in}}%
\pgfpathlineto{\pgfqpoint{2.239451in}{2.735509in}}%
\pgfpathmoveto{\pgfqpoint{2.212204in}{2.747307in}}%
\pgfpathlineto{\pgfqpoint{2.212204in}{2.747307in}}%
\pgfpathlineto{\pgfqpoint{2.212204in}{2.750256in}}%
\pgfpathlineto{\pgfqpoint{2.216745in}{2.750256in}}%
\pgfpathlineto{\pgfqpoint{2.216745in}{2.747307in}}%
\pgfpathmoveto{\pgfqpoint{2.212204in}{2.750256in}}%
\pgfpathlineto{\pgfqpoint{2.212204in}{2.750256in}}%
\pgfpathlineto{\pgfqpoint{2.212204in}{2.753205in}}%
\pgfpathlineto{\pgfqpoint{2.216745in}{2.753205in}}%
\pgfpathlineto{\pgfqpoint{2.216745in}{2.750256in}}%
\pgfpathmoveto{\pgfqpoint{2.216745in}{2.747307in}}%
\pgfpathlineto{\pgfqpoint{2.216745in}{2.747307in}}%
\pgfpathlineto{\pgfqpoint{2.216745in}{2.750256in}}%
\pgfpathlineto{\pgfqpoint{2.221286in}{2.750256in}}%
\pgfpathlineto{\pgfqpoint{2.221286in}{2.747307in}}%
\pgfpathmoveto{\pgfqpoint{2.216745in}{2.750256in}}%
\pgfpathlineto{\pgfqpoint{2.216745in}{2.750256in}}%
\pgfpathlineto{\pgfqpoint{2.216745in}{2.753205in}}%
\pgfpathlineto{\pgfqpoint{2.221286in}{2.753205in}}%
\pgfpathlineto{\pgfqpoint{2.221286in}{2.750256in}}%
\pgfpathmoveto{\pgfqpoint{2.203122in}{2.753205in}}%
\pgfpathlineto{\pgfqpoint{2.203122in}{2.753205in}}%
\pgfpathlineto{\pgfqpoint{2.203122in}{2.756155in}}%
\pgfpathlineto{\pgfqpoint{2.207663in}{2.756155in}}%
\pgfpathlineto{\pgfqpoint{2.207663in}{2.753205in}}%
\pgfpathmoveto{\pgfqpoint{2.203122in}{2.756155in}}%
\pgfpathlineto{\pgfqpoint{2.203122in}{2.756155in}}%
\pgfpathlineto{\pgfqpoint{2.203122in}{2.759104in}}%
\pgfpathlineto{\pgfqpoint{2.207663in}{2.759104in}}%
\pgfpathlineto{\pgfqpoint{2.207663in}{2.756155in}}%
\pgfpathmoveto{\pgfqpoint{2.207663in}{2.753205in}}%
\pgfpathlineto{\pgfqpoint{2.207663in}{2.753205in}}%
\pgfpathlineto{\pgfqpoint{2.207663in}{2.756155in}}%
\pgfpathlineto{\pgfqpoint{2.212204in}{2.756155in}}%
\pgfpathlineto{\pgfqpoint{2.212204in}{2.753205in}}%
\pgfpathmoveto{\pgfqpoint{2.207663in}{2.756155in}}%
\pgfpathlineto{\pgfqpoint{2.207663in}{2.756155in}}%
\pgfpathlineto{\pgfqpoint{2.207663in}{2.759104in}}%
\pgfpathlineto{\pgfqpoint{2.212204in}{2.759104in}}%
\pgfpathlineto{\pgfqpoint{2.212204in}{2.756155in}}%
\pgfpathmoveto{\pgfqpoint{2.203122in}{2.759104in}}%
\pgfpathlineto{\pgfqpoint{2.203122in}{2.759104in}}%
\pgfpathlineto{\pgfqpoint{2.203122in}{2.762053in}}%
\pgfpathlineto{\pgfqpoint{2.207663in}{2.762053in}}%
\pgfpathlineto{\pgfqpoint{2.207663in}{2.759104in}}%
\pgfpathmoveto{\pgfqpoint{2.203122in}{2.762053in}}%
\pgfpathlineto{\pgfqpoint{2.203122in}{2.762053in}}%
\pgfpathlineto{\pgfqpoint{2.203122in}{2.765003in}}%
\pgfpathlineto{\pgfqpoint{2.207663in}{2.765003in}}%
\pgfpathlineto{\pgfqpoint{2.207663in}{2.762053in}}%
\pgfpathmoveto{\pgfqpoint{2.207663in}{2.759104in}}%
\pgfpathlineto{\pgfqpoint{2.207663in}{2.759104in}}%
\pgfpathlineto{\pgfqpoint{2.207663in}{2.762053in}}%
\pgfpathlineto{\pgfqpoint{2.212204in}{2.762053in}}%
\pgfpathlineto{\pgfqpoint{2.212204in}{2.759104in}}%
\pgfpathmoveto{\pgfqpoint{2.212204in}{2.753205in}}%
\pgfpathlineto{\pgfqpoint{2.212204in}{2.753205in}}%
\pgfpathlineto{\pgfqpoint{2.212204in}{2.756155in}}%
\pgfpathlineto{\pgfqpoint{2.216745in}{2.756155in}}%
\pgfpathlineto{\pgfqpoint{2.216745in}{2.753205in}}%
\pgfpathmoveto{\pgfqpoint{2.212204in}{2.756155in}}%
\pgfpathlineto{\pgfqpoint{2.212204in}{2.756155in}}%
\pgfpathlineto{\pgfqpoint{2.212204in}{2.759104in}}%
\pgfpathlineto{\pgfqpoint{2.216745in}{2.759104in}}%
\pgfpathlineto{\pgfqpoint{2.216745in}{2.756155in}}%
\pgfpathmoveto{\pgfqpoint{2.216745in}{2.753205in}}%
\pgfpathlineto{\pgfqpoint{2.216745in}{2.753205in}}%
\pgfpathlineto{\pgfqpoint{2.216745in}{2.756155in}}%
\pgfpathlineto{\pgfqpoint{2.221286in}{2.756155in}}%
\pgfpathlineto{\pgfqpoint{2.221286in}{2.753205in}}%
\pgfpathmoveto{\pgfqpoint{2.221286in}{2.741408in}}%
\pgfpathlineto{\pgfqpoint{2.221286in}{2.741408in}}%
\pgfpathlineto{\pgfqpoint{2.221286in}{2.744357in}}%
\pgfpathlineto{\pgfqpoint{2.225827in}{2.744357in}}%
\pgfpathlineto{\pgfqpoint{2.225827in}{2.741408in}}%
\pgfpathmoveto{\pgfqpoint{2.221286in}{2.744357in}}%
\pgfpathlineto{\pgfqpoint{2.221286in}{2.744357in}}%
\pgfpathlineto{\pgfqpoint{2.221286in}{2.747307in}}%
\pgfpathlineto{\pgfqpoint{2.225827in}{2.747307in}}%
\pgfpathlineto{\pgfqpoint{2.225827in}{2.744357in}}%
\pgfpathmoveto{\pgfqpoint{2.225827in}{2.741408in}}%
\pgfpathlineto{\pgfqpoint{2.225827in}{2.741408in}}%
\pgfpathlineto{\pgfqpoint{2.225827in}{2.744357in}}%
\pgfpathlineto{\pgfqpoint{2.230369in}{2.744357in}}%
\pgfpathlineto{\pgfqpoint{2.230369in}{2.741408in}}%
\pgfpathmoveto{\pgfqpoint{2.225827in}{2.744357in}}%
\pgfpathlineto{\pgfqpoint{2.225827in}{2.744357in}}%
\pgfpathlineto{\pgfqpoint{2.225827in}{2.747307in}}%
\pgfpathlineto{\pgfqpoint{2.230369in}{2.747307in}}%
\pgfpathlineto{\pgfqpoint{2.230369in}{2.744357in}}%
\pgfpathmoveto{\pgfqpoint{2.221286in}{2.747307in}}%
\pgfpathlineto{\pgfqpoint{2.221286in}{2.747307in}}%
\pgfpathlineto{\pgfqpoint{2.221286in}{2.750256in}}%
\pgfpathlineto{\pgfqpoint{2.225827in}{2.750256in}}%
\pgfpathlineto{\pgfqpoint{2.225827in}{2.747307in}}%
\pgfpathmoveto{\pgfqpoint{2.230369in}{2.741408in}}%
\pgfpathlineto{\pgfqpoint{2.230369in}{2.741408in}}%
\pgfpathlineto{\pgfqpoint{2.230369in}{2.744357in}}%
\pgfpathlineto{\pgfqpoint{2.234910in}{2.744357in}}%
\pgfpathlineto{\pgfqpoint{2.234910in}{2.741408in}}%
\pgfpathmoveto{\pgfqpoint{2.239451in}{2.723712in}}%
\pgfpathlineto{\pgfqpoint{2.239451in}{2.723712in}}%
\pgfpathlineto{\pgfqpoint{2.239451in}{2.726661in}}%
\pgfpathlineto{\pgfqpoint{2.243992in}{2.726661in}}%
\pgfpathlineto{\pgfqpoint{2.243992in}{2.723712in}}%
\pgfpathmoveto{\pgfqpoint{2.239451in}{2.726661in}}%
\pgfpathlineto{\pgfqpoint{2.239451in}{2.726661in}}%
\pgfpathlineto{\pgfqpoint{2.239451in}{2.729611in}}%
\pgfpathlineto{\pgfqpoint{2.243992in}{2.729611in}}%
\pgfpathlineto{\pgfqpoint{2.243992in}{2.726661in}}%
\pgfpathmoveto{\pgfqpoint{2.243992in}{2.723712in}}%
\pgfpathlineto{\pgfqpoint{2.243992in}{2.723712in}}%
\pgfpathlineto{\pgfqpoint{2.243992in}{2.726661in}}%
\pgfpathlineto{\pgfqpoint{2.248533in}{2.726661in}}%
\pgfpathlineto{\pgfqpoint{2.248533in}{2.723712in}}%
\pgfpathmoveto{\pgfqpoint{2.243992in}{2.726661in}}%
\pgfpathlineto{\pgfqpoint{2.243992in}{2.726661in}}%
\pgfpathlineto{\pgfqpoint{2.243992in}{2.729611in}}%
\pgfpathlineto{\pgfqpoint{2.248533in}{2.729611in}}%
\pgfpathlineto{\pgfqpoint{2.248533in}{2.726661in}}%
\pgfpathmoveto{\pgfqpoint{2.248533in}{2.717813in}}%
\pgfpathlineto{\pgfqpoint{2.248533in}{2.717813in}}%
\pgfpathlineto{\pgfqpoint{2.248533in}{2.720763in}}%
\pgfpathlineto{\pgfqpoint{2.253074in}{2.720763in}}%
\pgfpathlineto{\pgfqpoint{2.253074in}{2.717813in}}%
\pgfpathmoveto{\pgfqpoint{2.248533in}{2.720763in}}%
\pgfpathlineto{\pgfqpoint{2.248533in}{2.720763in}}%
\pgfpathlineto{\pgfqpoint{2.248533in}{2.723712in}}%
\pgfpathlineto{\pgfqpoint{2.253074in}{2.723712in}}%
\pgfpathlineto{\pgfqpoint{2.253074in}{2.720763in}}%
\pgfpathmoveto{\pgfqpoint{2.253074in}{2.717813in}}%
\pgfpathlineto{\pgfqpoint{2.253074in}{2.717813in}}%
\pgfpathlineto{\pgfqpoint{2.253074in}{2.720763in}}%
\pgfpathlineto{\pgfqpoint{2.257616in}{2.720763in}}%
\pgfpathlineto{\pgfqpoint{2.257616in}{2.717813in}}%
\pgfpathmoveto{\pgfqpoint{2.253074in}{2.720763in}}%
\pgfpathlineto{\pgfqpoint{2.253074in}{2.720763in}}%
\pgfpathlineto{\pgfqpoint{2.253074in}{2.723712in}}%
\pgfpathlineto{\pgfqpoint{2.257616in}{2.723712in}}%
\pgfpathlineto{\pgfqpoint{2.257616in}{2.720763in}}%
\pgfpathmoveto{\pgfqpoint{2.248533in}{2.723712in}}%
\pgfpathlineto{\pgfqpoint{2.248533in}{2.723712in}}%
\pgfpathlineto{\pgfqpoint{2.248533in}{2.726661in}}%
\pgfpathlineto{\pgfqpoint{2.253074in}{2.726661in}}%
\pgfpathlineto{\pgfqpoint{2.253074in}{2.723712in}}%
\pgfpathmoveto{\pgfqpoint{2.239451in}{2.729611in}}%
\pgfpathlineto{\pgfqpoint{2.239451in}{2.729611in}}%
\pgfpathlineto{\pgfqpoint{2.239451in}{2.732560in}}%
\pgfpathlineto{\pgfqpoint{2.243992in}{2.732560in}}%
\pgfpathlineto{\pgfqpoint{2.243992in}{2.729611in}}%
\pgfpathmoveto{\pgfqpoint{2.239451in}{2.732560in}}%
\pgfpathlineto{\pgfqpoint{2.239451in}{2.732560in}}%
\pgfpathlineto{\pgfqpoint{2.239451in}{2.735509in}}%
\pgfpathlineto{\pgfqpoint{2.243992in}{2.735509in}}%
\pgfpathlineto{\pgfqpoint{2.243992in}{2.732560in}}%
\pgfpathmoveto{\pgfqpoint{2.243992in}{2.729611in}}%
\pgfpathlineto{\pgfqpoint{2.243992in}{2.729611in}}%
\pgfpathlineto{\pgfqpoint{2.243992in}{2.732560in}}%
\pgfpathlineto{\pgfqpoint{2.248533in}{2.732560in}}%
\pgfpathlineto{\pgfqpoint{2.248533in}{2.729611in}}%
\pgfpathmoveto{\pgfqpoint{2.257616in}{2.717813in}}%
\pgfpathlineto{\pgfqpoint{2.257616in}{2.717813in}}%
\pgfpathlineto{\pgfqpoint{2.257616in}{2.720763in}}%
\pgfpathlineto{\pgfqpoint{2.262157in}{2.720763in}}%
\pgfpathlineto{\pgfqpoint{2.262157in}{2.717813in}}%
\pgfpathmoveto{\pgfqpoint{2.284863in}{2.682421in}}%
\pgfpathlineto{\pgfqpoint{2.284863in}{2.682421in}}%
\pgfpathlineto{\pgfqpoint{2.284863in}{2.685370in}}%
\pgfpathlineto{\pgfqpoint{2.289404in}{2.685370in}}%
\pgfpathlineto{\pgfqpoint{2.289404in}{2.682421in}}%
\pgfpathmoveto{\pgfqpoint{2.284863in}{2.685370in}}%
\pgfpathlineto{\pgfqpoint{2.284863in}{2.685370in}}%
\pgfpathlineto{\pgfqpoint{2.284863in}{2.688320in}}%
\pgfpathlineto{\pgfqpoint{2.289404in}{2.688320in}}%
\pgfpathlineto{\pgfqpoint{2.289404in}{2.685370in}}%
\pgfpathmoveto{\pgfqpoint{2.289404in}{2.682421in}}%
\pgfpathlineto{\pgfqpoint{2.289404in}{2.682421in}}%
\pgfpathlineto{\pgfqpoint{2.289404in}{2.685370in}}%
\pgfpathlineto{\pgfqpoint{2.293945in}{2.685370in}}%
\pgfpathlineto{\pgfqpoint{2.293945in}{2.682421in}}%
\pgfpathmoveto{\pgfqpoint{2.289404in}{2.685370in}}%
\pgfpathlineto{\pgfqpoint{2.289404in}{2.685370in}}%
\pgfpathlineto{\pgfqpoint{2.289404in}{2.688320in}}%
\pgfpathlineto{\pgfqpoint{2.293945in}{2.688320in}}%
\pgfpathlineto{\pgfqpoint{2.293945in}{2.685370in}}%
\pgfpathmoveto{\pgfqpoint{2.284863in}{2.688320in}}%
\pgfpathlineto{\pgfqpoint{2.284863in}{2.688320in}}%
\pgfpathlineto{\pgfqpoint{2.284863in}{2.691269in}}%
\pgfpathlineto{\pgfqpoint{2.289404in}{2.691269in}}%
\pgfpathlineto{\pgfqpoint{2.289404in}{2.688320in}}%
\pgfpathmoveto{\pgfqpoint{2.284863in}{2.691269in}}%
\pgfpathlineto{\pgfqpoint{2.284863in}{2.691269in}}%
\pgfpathlineto{\pgfqpoint{2.284863in}{2.694218in}}%
\pgfpathlineto{\pgfqpoint{2.289404in}{2.694218in}}%
\pgfpathlineto{\pgfqpoint{2.289404in}{2.691269in}}%
\pgfpathmoveto{\pgfqpoint{2.289404in}{2.688320in}}%
\pgfpathlineto{\pgfqpoint{2.289404in}{2.688320in}}%
\pgfpathlineto{\pgfqpoint{2.289404in}{2.691269in}}%
\pgfpathlineto{\pgfqpoint{2.293945in}{2.691269in}}%
\pgfpathlineto{\pgfqpoint{2.293945in}{2.688320in}}%
\pgfpathmoveto{\pgfqpoint{2.293945in}{2.676522in}}%
\pgfpathlineto{\pgfqpoint{2.293945in}{2.676522in}}%
\pgfpathlineto{\pgfqpoint{2.293945in}{2.679472in}}%
\pgfpathlineto{\pgfqpoint{2.298486in}{2.679472in}}%
\pgfpathlineto{\pgfqpoint{2.298486in}{2.676522in}}%
\pgfpathmoveto{\pgfqpoint{2.293945in}{2.679472in}}%
\pgfpathlineto{\pgfqpoint{2.293945in}{2.679472in}}%
\pgfpathlineto{\pgfqpoint{2.293945in}{2.682421in}}%
\pgfpathlineto{\pgfqpoint{2.298486in}{2.682421in}}%
\pgfpathlineto{\pgfqpoint{2.298486in}{2.679472in}}%
\pgfpathmoveto{\pgfqpoint{2.298486in}{2.676522in}}%
\pgfpathlineto{\pgfqpoint{2.298486in}{2.676522in}}%
\pgfpathlineto{\pgfqpoint{2.298486in}{2.679472in}}%
\pgfpathlineto{\pgfqpoint{2.303027in}{2.679472in}}%
\pgfpathlineto{\pgfqpoint{2.303027in}{2.676522in}}%
\pgfpathmoveto{\pgfqpoint{2.298486in}{2.679472in}}%
\pgfpathlineto{\pgfqpoint{2.298486in}{2.679472in}}%
\pgfpathlineto{\pgfqpoint{2.298486in}{2.682421in}}%
\pgfpathlineto{\pgfqpoint{2.303027in}{2.682421in}}%
\pgfpathlineto{\pgfqpoint{2.303027in}{2.679472in}}%
\pgfpathmoveto{\pgfqpoint{2.303027in}{2.670624in}}%
\pgfpathlineto{\pgfqpoint{2.303027in}{2.670624in}}%
\pgfpathlineto{\pgfqpoint{2.303027in}{2.673573in}}%
\pgfpathlineto{\pgfqpoint{2.307569in}{2.673573in}}%
\pgfpathlineto{\pgfqpoint{2.307569in}{2.670624in}}%
\pgfpathmoveto{\pgfqpoint{2.303027in}{2.673573in}}%
\pgfpathlineto{\pgfqpoint{2.303027in}{2.673573in}}%
\pgfpathlineto{\pgfqpoint{2.303027in}{2.676522in}}%
\pgfpathlineto{\pgfqpoint{2.307569in}{2.676522in}}%
\pgfpathlineto{\pgfqpoint{2.307569in}{2.673573in}}%
\pgfpathmoveto{\pgfqpoint{2.307569in}{2.670624in}}%
\pgfpathlineto{\pgfqpoint{2.307569in}{2.670624in}}%
\pgfpathlineto{\pgfqpoint{2.307569in}{2.673573in}}%
\pgfpathlineto{\pgfqpoint{2.312110in}{2.673573in}}%
\pgfpathlineto{\pgfqpoint{2.312110in}{2.670624in}}%
\pgfpathmoveto{\pgfqpoint{2.307569in}{2.673573in}}%
\pgfpathlineto{\pgfqpoint{2.307569in}{2.673573in}}%
\pgfpathlineto{\pgfqpoint{2.307569in}{2.676522in}}%
\pgfpathlineto{\pgfqpoint{2.312110in}{2.676522in}}%
\pgfpathlineto{\pgfqpoint{2.312110in}{2.673573in}}%
\pgfpathmoveto{\pgfqpoint{2.303027in}{2.676522in}}%
\pgfpathlineto{\pgfqpoint{2.303027in}{2.676522in}}%
\pgfpathlineto{\pgfqpoint{2.303027in}{2.679472in}}%
\pgfpathlineto{\pgfqpoint{2.307569in}{2.679472in}}%
\pgfpathlineto{\pgfqpoint{2.307569in}{2.676522in}}%
\pgfpathmoveto{\pgfqpoint{2.293945in}{2.682421in}}%
\pgfpathlineto{\pgfqpoint{2.293945in}{2.682421in}}%
\pgfpathlineto{\pgfqpoint{2.293945in}{2.685370in}}%
\pgfpathlineto{\pgfqpoint{2.298486in}{2.685370in}}%
\pgfpathlineto{\pgfqpoint{2.298486in}{2.682421in}}%
\pgfpathmoveto{\pgfqpoint{2.293945in}{2.685370in}}%
\pgfpathlineto{\pgfqpoint{2.293945in}{2.685370in}}%
\pgfpathlineto{\pgfqpoint{2.293945in}{2.688320in}}%
\pgfpathlineto{\pgfqpoint{2.298486in}{2.688320in}}%
\pgfpathlineto{\pgfqpoint{2.298486in}{2.685370in}}%
\pgfpathmoveto{\pgfqpoint{2.298486in}{2.682421in}}%
\pgfpathlineto{\pgfqpoint{2.298486in}{2.682421in}}%
\pgfpathlineto{\pgfqpoint{2.298486in}{2.685370in}}%
\pgfpathlineto{\pgfqpoint{2.303027in}{2.685370in}}%
\pgfpathlineto{\pgfqpoint{2.303027in}{2.682421in}}%
\pgfpathmoveto{\pgfqpoint{2.275780in}{2.694218in}}%
\pgfpathlineto{\pgfqpoint{2.275780in}{2.694218in}}%
\pgfpathlineto{\pgfqpoint{2.275780in}{2.697168in}}%
\pgfpathlineto{\pgfqpoint{2.280321in}{2.697168in}}%
\pgfpathlineto{\pgfqpoint{2.280321in}{2.694218in}}%
\pgfpathmoveto{\pgfqpoint{2.275780in}{2.697168in}}%
\pgfpathlineto{\pgfqpoint{2.275780in}{2.697168in}}%
\pgfpathlineto{\pgfqpoint{2.275780in}{2.700117in}}%
\pgfpathlineto{\pgfqpoint{2.280321in}{2.700117in}}%
\pgfpathlineto{\pgfqpoint{2.280321in}{2.697168in}}%
\pgfpathmoveto{\pgfqpoint{2.280321in}{2.694218in}}%
\pgfpathlineto{\pgfqpoint{2.280321in}{2.694218in}}%
\pgfpathlineto{\pgfqpoint{2.280321in}{2.697168in}}%
\pgfpathlineto{\pgfqpoint{2.284863in}{2.697168in}}%
\pgfpathlineto{\pgfqpoint{2.284863in}{2.694218in}}%
\pgfpathmoveto{\pgfqpoint{2.280321in}{2.697168in}}%
\pgfpathlineto{\pgfqpoint{2.280321in}{2.697168in}}%
\pgfpathlineto{\pgfqpoint{2.280321in}{2.700117in}}%
\pgfpathlineto{\pgfqpoint{2.284863in}{2.700117in}}%
\pgfpathlineto{\pgfqpoint{2.284863in}{2.697168in}}%
\pgfpathmoveto{\pgfqpoint{2.275780in}{2.700117in}}%
\pgfpathlineto{\pgfqpoint{2.275780in}{2.700117in}}%
\pgfpathlineto{\pgfqpoint{2.275780in}{2.703066in}}%
\pgfpathlineto{\pgfqpoint{2.280321in}{2.703066in}}%
\pgfpathlineto{\pgfqpoint{2.280321in}{2.700117in}}%
\pgfpathmoveto{\pgfqpoint{2.284863in}{2.694218in}}%
\pgfpathlineto{\pgfqpoint{2.284863in}{2.694218in}}%
\pgfpathlineto{\pgfqpoint{2.284863in}{2.697168in}}%
\pgfpathlineto{\pgfqpoint{2.289404in}{2.697168in}}%
\pgfpathlineto{\pgfqpoint{2.289404in}{2.694218in}}%
\pgfpathmoveto{\pgfqpoint{2.312110in}{2.670624in}}%
\pgfpathlineto{\pgfqpoint{2.312110in}{2.670624in}}%
\pgfpathlineto{\pgfqpoint{2.312110in}{2.673573in}}%
\pgfpathlineto{\pgfqpoint{2.316651in}{2.673573in}}%
\pgfpathlineto{\pgfqpoint{2.316651in}{2.670624in}}%
\pgfpathmoveto{\pgfqpoint{2.203122in}{2.765003in}}%
\pgfpathlineto{\pgfqpoint{2.203122in}{2.765003in}}%
\pgfpathlineto{\pgfqpoint{2.203122in}{2.767952in}}%
\pgfpathlineto{\pgfqpoint{2.207663in}{2.767952in}}%
\pgfpathlineto{\pgfqpoint{2.207663in}{2.765003in}}%
\pgfpathmoveto{\pgfqpoint{2.484664in}{2.511367in}}%
\pgfpathlineto{\pgfqpoint{2.484664in}{2.511367in}}%
\pgfpathlineto{\pgfqpoint{2.484664in}{2.514316in}}%
\pgfpathlineto{\pgfqpoint{2.489205in}{2.514316in}}%
\pgfpathlineto{\pgfqpoint{2.489205in}{2.511367in}}%
\pgfpathmoveto{\pgfqpoint{2.484664in}{2.514316in}}%
\pgfpathlineto{\pgfqpoint{2.484664in}{2.514316in}}%
\pgfpathlineto{\pgfqpoint{2.484664in}{2.517266in}}%
\pgfpathlineto{\pgfqpoint{2.489205in}{2.517266in}}%
\pgfpathlineto{\pgfqpoint{2.489205in}{2.514316in}}%
\pgfpathmoveto{\pgfqpoint{2.489205in}{2.511367in}}%
\pgfpathlineto{\pgfqpoint{2.489205in}{2.511367in}}%
\pgfpathlineto{\pgfqpoint{2.489205in}{2.514316in}}%
\pgfpathlineto{\pgfqpoint{2.493746in}{2.514316in}}%
\pgfpathlineto{\pgfqpoint{2.493746in}{2.511367in}}%
\pgfpathmoveto{\pgfqpoint{2.489205in}{2.514316in}}%
\pgfpathlineto{\pgfqpoint{2.489205in}{2.514316in}}%
\pgfpathlineto{\pgfqpoint{2.489205in}{2.517266in}}%
\pgfpathlineto{\pgfqpoint{2.493746in}{2.517266in}}%
\pgfpathlineto{\pgfqpoint{2.493746in}{2.514316in}}%
\pgfpathmoveto{\pgfqpoint{2.475583in}{2.517266in}}%
\pgfpathlineto{\pgfqpoint{2.475583in}{2.517266in}}%
\pgfpathlineto{\pgfqpoint{2.475583in}{2.520215in}}%
\pgfpathlineto{\pgfqpoint{2.480124in}{2.520215in}}%
\pgfpathlineto{\pgfqpoint{2.480124in}{2.517266in}}%
\pgfpathmoveto{\pgfqpoint{2.475583in}{2.520215in}}%
\pgfpathlineto{\pgfqpoint{2.475583in}{2.520215in}}%
\pgfpathlineto{\pgfqpoint{2.475583in}{2.523164in}}%
\pgfpathlineto{\pgfqpoint{2.480124in}{2.523164in}}%
\pgfpathlineto{\pgfqpoint{2.480124in}{2.520215in}}%
\pgfpathmoveto{\pgfqpoint{2.480124in}{2.517266in}}%
\pgfpathlineto{\pgfqpoint{2.480124in}{2.517266in}}%
\pgfpathlineto{\pgfqpoint{2.480124in}{2.520215in}}%
\pgfpathlineto{\pgfqpoint{2.484664in}{2.520215in}}%
\pgfpathlineto{\pgfqpoint{2.484664in}{2.517266in}}%
\pgfpathmoveto{\pgfqpoint{2.480124in}{2.520215in}}%
\pgfpathlineto{\pgfqpoint{2.480124in}{2.520215in}}%
\pgfpathlineto{\pgfqpoint{2.480124in}{2.523164in}}%
\pgfpathlineto{\pgfqpoint{2.484664in}{2.523164in}}%
\pgfpathlineto{\pgfqpoint{2.484664in}{2.520215in}}%
\pgfpathmoveto{\pgfqpoint{2.475583in}{2.523164in}}%
\pgfpathlineto{\pgfqpoint{2.475583in}{2.523164in}}%
\pgfpathlineto{\pgfqpoint{2.475583in}{2.526113in}}%
\pgfpathlineto{\pgfqpoint{2.480124in}{2.526113in}}%
\pgfpathlineto{\pgfqpoint{2.480124in}{2.523164in}}%
\pgfpathmoveto{\pgfqpoint{2.475583in}{2.526113in}}%
\pgfpathlineto{\pgfqpoint{2.475583in}{2.526113in}}%
\pgfpathlineto{\pgfqpoint{2.475583in}{2.529062in}}%
\pgfpathlineto{\pgfqpoint{2.480124in}{2.529062in}}%
\pgfpathlineto{\pgfqpoint{2.480124in}{2.526113in}}%
\pgfpathmoveto{\pgfqpoint{2.480124in}{2.523164in}}%
\pgfpathlineto{\pgfqpoint{2.480124in}{2.523164in}}%
\pgfpathlineto{\pgfqpoint{2.480124in}{2.526113in}}%
\pgfpathlineto{\pgfqpoint{2.484664in}{2.526113in}}%
\pgfpathlineto{\pgfqpoint{2.484664in}{2.523164in}}%
\pgfpathmoveto{\pgfqpoint{2.484664in}{2.517266in}}%
\pgfpathlineto{\pgfqpoint{2.484664in}{2.517266in}}%
\pgfpathlineto{\pgfqpoint{2.484664in}{2.520215in}}%
\pgfpathlineto{\pgfqpoint{2.489205in}{2.520215in}}%
\pgfpathlineto{\pgfqpoint{2.489205in}{2.517266in}}%
\pgfpathmoveto{\pgfqpoint{2.484664in}{2.520215in}}%
\pgfpathlineto{\pgfqpoint{2.484664in}{2.520215in}}%
\pgfpathlineto{\pgfqpoint{2.484664in}{2.523164in}}%
\pgfpathlineto{\pgfqpoint{2.489205in}{2.523164in}}%
\pgfpathlineto{\pgfqpoint{2.489205in}{2.520215in}}%
\pgfpathmoveto{\pgfqpoint{2.489205in}{2.517266in}}%
\pgfpathlineto{\pgfqpoint{2.489205in}{2.517266in}}%
\pgfpathlineto{\pgfqpoint{2.489205in}{2.520215in}}%
\pgfpathlineto{\pgfqpoint{2.493746in}{2.520215in}}%
\pgfpathlineto{\pgfqpoint{2.493746in}{2.517266in}}%
\pgfpathmoveto{\pgfqpoint{2.448338in}{2.540859in}}%
\pgfpathlineto{\pgfqpoint{2.448338in}{2.540859in}}%
\pgfpathlineto{\pgfqpoint{2.448338in}{2.543808in}}%
\pgfpathlineto{\pgfqpoint{2.452878in}{2.543808in}}%
\pgfpathlineto{\pgfqpoint{2.452878in}{2.540859in}}%
\pgfpathmoveto{\pgfqpoint{2.448338in}{2.543808in}}%
\pgfpathlineto{\pgfqpoint{2.448338in}{2.543808in}}%
\pgfpathlineto{\pgfqpoint{2.448338in}{2.546757in}}%
\pgfpathlineto{\pgfqpoint{2.452878in}{2.546757in}}%
\pgfpathlineto{\pgfqpoint{2.452878in}{2.543808in}}%
\pgfpathmoveto{\pgfqpoint{2.452878in}{2.540859in}}%
\pgfpathlineto{\pgfqpoint{2.452878in}{2.540859in}}%
\pgfpathlineto{\pgfqpoint{2.452878in}{2.543808in}}%
\pgfpathlineto{\pgfqpoint{2.457419in}{2.543808in}}%
\pgfpathlineto{\pgfqpoint{2.457419in}{2.540859in}}%
\pgfpathmoveto{\pgfqpoint{2.452878in}{2.543808in}}%
\pgfpathlineto{\pgfqpoint{2.452878in}{2.543808in}}%
\pgfpathlineto{\pgfqpoint{2.452878in}{2.546757in}}%
\pgfpathlineto{\pgfqpoint{2.457419in}{2.546757in}}%
\pgfpathlineto{\pgfqpoint{2.457419in}{2.543808in}}%
\pgfpathmoveto{\pgfqpoint{2.448338in}{2.546757in}}%
\pgfpathlineto{\pgfqpoint{2.448338in}{2.546757in}}%
\pgfpathlineto{\pgfqpoint{2.448338in}{2.549707in}}%
\pgfpathlineto{\pgfqpoint{2.452878in}{2.549707in}}%
\pgfpathlineto{\pgfqpoint{2.452878in}{2.546757in}}%
\pgfpathmoveto{\pgfqpoint{2.448338in}{2.549707in}}%
\pgfpathlineto{\pgfqpoint{2.448338in}{2.549707in}}%
\pgfpathlineto{\pgfqpoint{2.448338in}{2.552656in}}%
\pgfpathlineto{\pgfqpoint{2.452878in}{2.552656in}}%
\pgfpathlineto{\pgfqpoint{2.452878in}{2.549707in}}%
\pgfpathmoveto{\pgfqpoint{2.452878in}{2.546757in}}%
\pgfpathlineto{\pgfqpoint{2.452878in}{2.546757in}}%
\pgfpathlineto{\pgfqpoint{2.452878in}{2.549707in}}%
\pgfpathlineto{\pgfqpoint{2.457419in}{2.549707in}}%
\pgfpathlineto{\pgfqpoint{2.457419in}{2.546757in}}%
\pgfpathmoveto{\pgfqpoint{2.430174in}{2.558554in}}%
\pgfpathlineto{\pgfqpoint{2.430174in}{2.558554in}}%
\pgfpathlineto{\pgfqpoint{2.430174in}{2.561503in}}%
\pgfpathlineto{\pgfqpoint{2.434715in}{2.561503in}}%
\pgfpathlineto{\pgfqpoint{2.434715in}{2.558554in}}%
\pgfpathmoveto{\pgfqpoint{2.430174in}{2.561503in}}%
\pgfpathlineto{\pgfqpoint{2.430174in}{2.561503in}}%
\pgfpathlineto{\pgfqpoint{2.430174in}{2.564453in}}%
\pgfpathlineto{\pgfqpoint{2.434715in}{2.564453in}}%
\pgfpathlineto{\pgfqpoint{2.434715in}{2.561503in}}%
\pgfpathmoveto{\pgfqpoint{2.434715in}{2.558554in}}%
\pgfpathlineto{\pgfqpoint{2.434715in}{2.558554in}}%
\pgfpathlineto{\pgfqpoint{2.434715in}{2.561503in}}%
\pgfpathlineto{\pgfqpoint{2.439256in}{2.561503in}}%
\pgfpathlineto{\pgfqpoint{2.439256in}{2.558554in}}%
\pgfpathmoveto{\pgfqpoint{2.434715in}{2.561503in}}%
\pgfpathlineto{\pgfqpoint{2.434715in}{2.561503in}}%
\pgfpathlineto{\pgfqpoint{2.434715in}{2.564453in}}%
\pgfpathlineto{\pgfqpoint{2.439256in}{2.564453in}}%
\pgfpathlineto{\pgfqpoint{2.439256in}{2.561503in}}%
\pgfpathmoveto{\pgfqpoint{2.421093in}{2.564453in}}%
\pgfpathlineto{\pgfqpoint{2.421093in}{2.564453in}}%
\pgfpathlineto{\pgfqpoint{2.421093in}{2.567402in}}%
\pgfpathlineto{\pgfqpoint{2.425633in}{2.567402in}}%
\pgfpathlineto{\pgfqpoint{2.425633in}{2.564453in}}%
\pgfpathmoveto{\pgfqpoint{2.421093in}{2.567402in}}%
\pgfpathlineto{\pgfqpoint{2.421093in}{2.567402in}}%
\pgfpathlineto{\pgfqpoint{2.421093in}{2.570351in}}%
\pgfpathlineto{\pgfqpoint{2.425633in}{2.570351in}}%
\pgfpathlineto{\pgfqpoint{2.425633in}{2.567402in}}%
\pgfpathmoveto{\pgfqpoint{2.425633in}{2.564453in}}%
\pgfpathlineto{\pgfqpoint{2.425633in}{2.564453in}}%
\pgfpathlineto{\pgfqpoint{2.425633in}{2.567402in}}%
\pgfpathlineto{\pgfqpoint{2.430174in}{2.567402in}}%
\pgfpathlineto{\pgfqpoint{2.430174in}{2.564453in}}%
\pgfpathmoveto{\pgfqpoint{2.425633in}{2.567402in}}%
\pgfpathlineto{\pgfqpoint{2.425633in}{2.567402in}}%
\pgfpathlineto{\pgfqpoint{2.425633in}{2.570351in}}%
\pgfpathlineto{\pgfqpoint{2.430174in}{2.570351in}}%
\pgfpathlineto{\pgfqpoint{2.430174in}{2.567402in}}%
\pgfpathmoveto{\pgfqpoint{2.421093in}{2.570351in}}%
\pgfpathlineto{\pgfqpoint{2.421093in}{2.570351in}}%
\pgfpathlineto{\pgfqpoint{2.421093in}{2.573300in}}%
\pgfpathlineto{\pgfqpoint{2.425633in}{2.573300in}}%
\pgfpathlineto{\pgfqpoint{2.425633in}{2.570351in}}%
\pgfpathmoveto{\pgfqpoint{2.421093in}{2.573300in}}%
\pgfpathlineto{\pgfqpoint{2.421093in}{2.573300in}}%
\pgfpathlineto{\pgfqpoint{2.421093in}{2.576249in}}%
\pgfpathlineto{\pgfqpoint{2.425633in}{2.576249in}}%
\pgfpathlineto{\pgfqpoint{2.425633in}{2.573300in}}%
\pgfpathmoveto{\pgfqpoint{2.425633in}{2.570351in}}%
\pgfpathlineto{\pgfqpoint{2.425633in}{2.570351in}}%
\pgfpathlineto{\pgfqpoint{2.425633in}{2.573300in}}%
\pgfpathlineto{\pgfqpoint{2.430174in}{2.573300in}}%
\pgfpathlineto{\pgfqpoint{2.430174in}{2.570351in}}%
\pgfpathmoveto{\pgfqpoint{2.430174in}{2.564453in}}%
\pgfpathlineto{\pgfqpoint{2.430174in}{2.564453in}}%
\pgfpathlineto{\pgfqpoint{2.430174in}{2.567402in}}%
\pgfpathlineto{\pgfqpoint{2.434715in}{2.567402in}}%
\pgfpathlineto{\pgfqpoint{2.434715in}{2.564453in}}%
\pgfpathmoveto{\pgfqpoint{2.430174in}{2.567402in}}%
\pgfpathlineto{\pgfqpoint{2.430174in}{2.567402in}}%
\pgfpathlineto{\pgfqpoint{2.430174in}{2.570351in}}%
\pgfpathlineto{\pgfqpoint{2.434715in}{2.570351in}}%
\pgfpathlineto{\pgfqpoint{2.434715in}{2.567402in}}%
\pgfpathmoveto{\pgfqpoint{2.434715in}{2.564453in}}%
\pgfpathlineto{\pgfqpoint{2.434715in}{2.564453in}}%
\pgfpathlineto{\pgfqpoint{2.434715in}{2.567402in}}%
\pgfpathlineto{\pgfqpoint{2.439256in}{2.567402in}}%
\pgfpathlineto{\pgfqpoint{2.439256in}{2.564453in}}%
\pgfpathmoveto{\pgfqpoint{2.439256in}{2.552656in}}%
\pgfpathlineto{\pgfqpoint{2.439256in}{2.552656in}}%
\pgfpathlineto{\pgfqpoint{2.439256in}{2.555605in}}%
\pgfpathlineto{\pgfqpoint{2.443797in}{2.555605in}}%
\pgfpathlineto{\pgfqpoint{2.443797in}{2.552656in}}%
\pgfpathmoveto{\pgfqpoint{2.439256in}{2.555605in}}%
\pgfpathlineto{\pgfqpoint{2.439256in}{2.555605in}}%
\pgfpathlineto{\pgfqpoint{2.439256in}{2.558554in}}%
\pgfpathlineto{\pgfqpoint{2.443797in}{2.558554in}}%
\pgfpathlineto{\pgfqpoint{2.443797in}{2.555605in}}%
\pgfpathmoveto{\pgfqpoint{2.443797in}{2.552656in}}%
\pgfpathlineto{\pgfqpoint{2.443797in}{2.552656in}}%
\pgfpathlineto{\pgfqpoint{2.443797in}{2.555605in}}%
\pgfpathlineto{\pgfqpoint{2.448338in}{2.555605in}}%
\pgfpathlineto{\pgfqpoint{2.448338in}{2.552656in}}%
\pgfpathmoveto{\pgfqpoint{2.443797in}{2.555605in}}%
\pgfpathlineto{\pgfqpoint{2.443797in}{2.555605in}}%
\pgfpathlineto{\pgfqpoint{2.443797in}{2.558554in}}%
\pgfpathlineto{\pgfqpoint{2.448338in}{2.558554in}}%
\pgfpathlineto{\pgfqpoint{2.448338in}{2.555605in}}%
\pgfpathmoveto{\pgfqpoint{2.439256in}{2.558554in}}%
\pgfpathlineto{\pgfqpoint{2.439256in}{2.558554in}}%
\pgfpathlineto{\pgfqpoint{2.439256in}{2.561503in}}%
\pgfpathlineto{\pgfqpoint{2.443797in}{2.561503in}}%
\pgfpathlineto{\pgfqpoint{2.443797in}{2.558554in}}%
\pgfpathmoveto{\pgfqpoint{2.448338in}{2.552656in}}%
\pgfpathlineto{\pgfqpoint{2.448338in}{2.552656in}}%
\pgfpathlineto{\pgfqpoint{2.448338in}{2.555605in}}%
\pgfpathlineto{\pgfqpoint{2.452878in}{2.555605in}}%
\pgfpathlineto{\pgfqpoint{2.452878in}{2.552656in}}%
\pgfpathmoveto{\pgfqpoint{2.457419in}{2.534961in}}%
\pgfpathlineto{\pgfqpoint{2.457419in}{2.534961in}}%
\pgfpathlineto{\pgfqpoint{2.457419in}{2.537910in}}%
\pgfpathlineto{\pgfqpoint{2.461960in}{2.537910in}}%
\pgfpathlineto{\pgfqpoint{2.461960in}{2.534961in}}%
\pgfpathmoveto{\pgfqpoint{2.457419in}{2.537910in}}%
\pgfpathlineto{\pgfqpoint{2.457419in}{2.537910in}}%
\pgfpathlineto{\pgfqpoint{2.457419in}{2.540859in}}%
\pgfpathlineto{\pgfqpoint{2.461960in}{2.540859in}}%
\pgfpathlineto{\pgfqpoint{2.461960in}{2.537910in}}%
\pgfpathmoveto{\pgfqpoint{2.461960in}{2.534961in}}%
\pgfpathlineto{\pgfqpoint{2.461960in}{2.534961in}}%
\pgfpathlineto{\pgfqpoint{2.461960in}{2.537910in}}%
\pgfpathlineto{\pgfqpoint{2.466501in}{2.537910in}}%
\pgfpathlineto{\pgfqpoint{2.466501in}{2.534961in}}%
\pgfpathmoveto{\pgfqpoint{2.461960in}{2.537910in}}%
\pgfpathlineto{\pgfqpoint{2.461960in}{2.537910in}}%
\pgfpathlineto{\pgfqpoint{2.461960in}{2.540859in}}%
\pgfpathlineto{\pgfqpoint{2.466501in}{2.540859in}}%
\pgfpathlineto{\pgfqpoint{2.466501in}{2.537910in}}%
\pgfpathmoveto{\pgfqpoint{2.466501in}{2.529062in}}%
\pgfpathlineto{\pgfqpoint{2.466501in}{2.529062in}}%
\pgfpathlineto{\pgfqpoint{2.466501in}{2.532011in}}%
\pgfpathlineto{\pgfqpoint{2.471042in}{2.532011in}}%
\pgfpathlineto{\pgfqpoint{2.471042in}{2.529062in}}%
\pgfpathmoveto{\pgfqpoint{2.466501in}{2.532011in}}%
\pgfpathlineto{\pgfqpoint{2.466501in}{2.532011in}}%
\pgfpathlineto{\pgfqpoint{2.466501in}{2.534961in}}%
\pgfpathlineto{\pgfqpoint{2.471042in}{2.534961in}}%
\pgfpathlineto{\pgfqpoint{2.471042in}{2.532011in}}%
\pgfpathmoveto{\pgfqpoint{2.471042in}{2.529062in}}%
\pgfpathlineto{\pgfqpoint{2.471042in}{2.529062in}}%
\pgfpathlineto{\pgfqpoint{2.471042in}{2.532011in}}%
\pgfpathlineto{\pgfqpoint{2.475583in}{2.532011in}}%
\pgfpathlineto{\pgfqpoint{2.475583in}{2.529062in}}%
\pgfpathmoveto{\pgfqpoint{2.471042in}{2.532011in}}%
\pgfpathlineto{\pgfqpoint{2.471042in}{2.532011in}}%
\pgfpathlineto{\pgfqpoint{2.471042in}{2.534961in}}%
\pgfpathlineto{\pgfqpoint{2.475583in}{2.534961in}}%
\pgfpathlineto{\pgfqpoint{2.475583in}{2.532011in}}%
\pgfpathmoveto{\pgfqpoint{2.466501in}{2.534961in}}%
\pgfpathlineto{\pgfqpoint{2.466501in}{2.534961in}}%
\pgfpathlineto{\pgfqpoint{2.466501in}{2.537910in}}%
\pgfpathlineto{\pgfqpoint{2.471042in}{2.537910in}}%
\pgfpathlineto{\pgfqpoint{2.471042in}{2.534961in}}%
\pgfpathmoveto{\pgfqpoint{2.457419in}{2.540859in}}%
\pgfpathlineto{\pgfqpoint{2.457419in}{2.540859in}}%
\pgfpathlineto{\pgfqpoint{2.457419in}{2.543808in}}%
\pgfpathlineto{\pgfqpoint{2.461960in}{2.543808in}}%
\pgfpathlineto{\pgfqpoint{2.461960in}{2.540859in}}%
\pgfpathmoveto{\pgfqpoint{2.457419in}{2.543808in}}%
\pgfpathlineto{\pgfqpoint{2.457419in}{2.543808in}}%
\pgfpathlineto{\pgfqpoint{2.457419in}{2.546757in}}%
\pgfpathlineto{\pgfqpoint{2.461960in}{2.546757in}}%
\pgfpathlineto{\pgfqpoint{2.461960in}{2.543808in}}%
\pgfpathmoveto{\pgfqpoint{2.461960in}{2.540859in}}%
\pgfpathlineto{\pgfqpoint{2.461960in}{2.540859in}}%
\pgfpathlineto{\pgfqpoint{2.461960in}{2.543808in}}%
\pgfpathlineto{\pgfqpoint{2.466501in}{2.543808in}}%
\pgfpathlineto{\pgfqpoint{2.466501in}{2.540859in}}%
\pgfpathmoveto{\pgfqpoint{2.475583in}{2.529062in}}%
\pgfpathlineto{\pgfqpoint{2.475583in}{2.529062in}}%
\pgfpathlineto{\pgfqpoint{2.475583in}{2.532011in}}%
\pgfpathlineto{\pgfqpoint{2.480124in}{2.532011in}}%
\pgfpathlineto{\pgfqpoint{2.480124in}{2.529062in}}%
\pgfpathmoveto{\pgfqpoint{2.375684in}{2.605741in}}%
\pgfpathlineto{\pgfqpoint{2.375684in}{2.605741in}}%
\pgfpathlineto{\pgfqpoint{2.375684in}{2.608691in}}%
\pgfpathlineto{\pgfqpoint{2.380225in}{2.608691in}}%
\pgfpathlineto{\pgfqpoint{2.380225in}{2.605741in}}%
\pgfpathmoveto{\pgfqpoint{2.375684in}{2.608691in}}%
\pgfpathlineto{\pgfqpoint{2.375684in}{2.608691in}}%
\pgfpathlineto{\pgfqpoint{2.375684in}{2.611640in}}%
\pgfpathlineto{\pgfqpoint{2.380225in}{2.611640in}}%
\pgfpathlineto{\pgfqpoint{2.380225in}{2.608691in}}%
\pgfpathmoveto{\pgfqpoint{2.380225in}{2.605741in}}%
\pgfpathlineto{\pgfqpoint{2.380225in}{2.605741in}}%
\pgfpathlineto{\pgfqpoint{2.380225in}{2.608691in}}%
\pgfpathlineto{\pgfqpoint{2.384766in}{2.608691in}}%
\pgfpathlineto{\pgfqpoint{2.384766in}{2.605741in}}%
\pgfpathmoveto{\pgfqpoint{2.380225in}{2.608691in}}%
\pgfpathlineto{\pgfqpoint{2.380225in}{2.608691in}}%
\pgfpathlineto{\pgfqpoint{2.380225in}{2.611640in}}%
\pgfpathlineto{\pgfqpoint{2.384766in}{2.611640in}}%
\pgfpathlineto{\pgfqpoint{2.384766in}{2.608691in}}%
\pgfpathmoveto{\pgfqpoint{2.366602in}{2.611640in}}%
\pgfpathlineto{\pgfqpoint{2.366602in}{2.611640in}}%
\pgfpathlineto{\pgfqpoint{2.366602in}{2.614589in}}%
\pgfpathlineto{\pgfqpoint{2.371143in}{2.614589in}}%
\pgfpathlineto{\pgfqpoint{2.371143in}{2.611640in}}%
\pgfpathmoveto{\pgfqpoint{2.366602in}{2.614589in}}%
\pgfpathlineto{\pgfqpoint{2.366602in}{2.614589in}}%
\pgfpathlineto{\pgfqpoint{2.366602in}{2.617538in}}%
\pgfpathlineto{\pgfqpoint{2.371143in}{2.617538in}}%
\pgfpathlineto{\pgfqpoint{2.371143in}{2.614589in}}%
\pgfpathmoveto{\pgfqpoint{2.371143in}{2.611640in}}%
\pgfpathlineto{\pgfqpoint{2.371143in}{2.611640in}}%
\pgfpathlineto{\pgfqpoint{2.371143in}{2.614589in}}%
\pgfpathlineto{\pgfqpoint{2.375684in}{2.614589in}}%
\pgfpathlineto{\pgfqpoint{2.375684in}{2.611640in}}%
\pgfpathmoveto{\pgfqpoint{2.371143in}{2.614589in}}%
\pgfpathlineto{\pgfqpoint{2.371143in}{2.614589in}}%
\pgfpathlineto{\pgfqpoint{2.371143in}{2.617538in}}%
\pgfpathlineto{\pgfqpoint{2.375684in}{2.617538in}}%
\pgfpathlineto{\pgfqpoint{2.375684in}{2.614589in}}%
\pgfpathmoveto{\pgfqpoint{2.366602in}{2.617538in}}%
\pgfpathlineto{\pgfqpoint{2.366602in}{2.617538in}}%
\pgfpathlineto{\pgfqpoint{2.366602in}{2.620487in}}%
\pgfpathlineto{\pgfqpoint{2.371143in}{2.620487in}}%
\pgfpathlineto{\pgfqpoint{2.371143in}{2.617538in}}%
\pgfpathmoveto{\pgfqpoint{2.366602in}{2.620487in}}%
\pgfpathlineto{\pgfqpoint{2.366602in}{2.620487in}}%
\pgfpathlineto{\pgfqpoint{2.366602in}{2.623436in}}%
\pgfpathlineto{\pgfqpoint{2.371143in}{2.623436in}}%
\pgfpathlineto{\pgfqpoint{2.371143in}{2.620487in}}%
\pgfpathmoveto{\pgfqpoint{2.371143in}{2.617538in}}%
\pgfpathlineto{\pgfqpoint{2.371143in}{2.617538in}}%
\pgfpathlineto{\pgfqpoint{2.371143in}{2.620487in}}%
\pgfpathlineto{\pgfqpoint{2.375684in}{2.620487in}}%
\pgfpathlineto{\pgfqpoint{2.375684in}{2.617538in}}%
\pgfpathmoveto{\pgfqpoint{2.375684in}{2.611640in}}%
\pgfpathlineto{\pgfqpoint{2.375684in}{2.611640in}}%
\pgfpathlineto{\pgfqpoint{2.375684in}{2.614589in}}%
\pgfpathlineto{\pgfqpoint{2.380225in}{2.614589in}}%
\pgfpathlineto{\pgfqpoint{2.380225in}{2.611640in}}%
\pgfpathmoveto{\pgfqpoint{2.375684in}{2.614589in}}%
\pgfpathlineto{\pgfqpoint{2.375684in}{2.614589in}}%
\pgfpathlineto{\pgfqpoint{2.375684in}{2.617538in}}%
\pgfpathlineto{\pgfqpoint{2.380225in}{2.617538in}}%
\pgfpathlineto{\pgfqpoint{2.380225in}{2.614589in}}%
\pgfpathmoveto{\pgfqpoint{2.380225in}{2.611640in}}%
\pgfpathlineto{\pgfqpoint{2.380225in}{2.611640in}}%
\pgfpathlineto{\pgfqpoint{2.380225in}{2.614589in}}%
\pgfpathlineto{\pgfqpoint{2.384766in}{2.614589in}}%
\pgfpathlineto{\pgfqpoint{2.384766in}{2.611640in}}%
\pgfpathmoveto{\pgfqpoint{2.393848in}{2.588046in}}%
\pgfpathlineto{\pgfqpoint{2.393848in}{2.588046in}}%
\pgfpathlineto{\pgfqpoint{2.393848in}{2.590995in}}%
\pgfpathlineto{\pgfqpoint{2.398388in}{2.590995in}}%
\pgfpathlineto{\pgfqpoint{2.398388in}{2.588046in}}%
\pgfpathmoveto{\pgfqpoint{2.393848in}{2.590995in}}%
\pgfpathlineto{\pgfqpoint{2.393848in}{2.590995in}}%
\pgfpathlineto{\pgfqpoint{2.393848in}{2.593945in}}%
\pgfpathlineto{\pgfqpoint{2.398388in}{2.593945in}}%
\pgfpathlineto{\pgfqpoint{2.398388in}{2.590995in}}%
\pgfpathmoveto{\pgfqpoint{2.398388in}{2.588046in}}%
\pgfpathlineto{\pgfqpoint{2.398388in}{2.588046in}}%
\pgfpathlineto{\pgfqpoint{2.398388in}{2.590995in}}%
\pgfpathlineto{\pgfqpoint{2.402929in}{2.590995in}}%
\pgfpathlineto{\pgfqpoint{2.402929in}{2.588046in}}%
\pgfpathmoveto{\pgfqpoint{2.398388in}{2.590995in}}%
\pgfpathlineto{\pgfqpoint{2.398388in}{2.590995in}}%
\pgfpathlineto{\pgfqpoint{2.398388in}{2.593945in}}%
\pgfpathlineto{\pgfqpoint{2.402929in}{2.593945in}}%
\pgfpathlineto{\pgfqpoint{2.402929in}{2.590995in}}%
\pgfpathmoveto{\pgfqpoint{2.393848in}{2.593945in}}%
\pgfpathlineto{\pgfqpoint{2.393848in}{2.593945in}}%
\pgfpathlineto{\pgfqpoint{2.393848in}{2.596894in}}%
\pgfpathlineto{\pgfqpoint{2.398388in}{2.596894in}}%
\pgfpathlineto{\pgfqpoint{2.398388in}{2.593945in}}%
\pgfpathmoveto{\pgfqpoint{2.393848in}{2.596894in}}%
\pgfpathlineto{\pgfqpoint{2.393848in}{2.596894in}}%
\pgfpathlineto{\pgfqpoint{2.393848in}{2.599843in}}%
\pgfpathlineto{\pgfqpoint{2.398388in}{2.599843in}}%
\pgfpathlineto{\pgfqpoint{2.398388in}{2.596894in}}%
\pgfpathmoveto{\pgfqpoint{2.398388in}{2.593945in}}%
\pgfpathlineto{\pgfqpoint{2.398388in}{2.593945in}}%
\pgfpathlineto{\pgfqpoint{2.398388in}{2.596894in}}%
\pgfpathlineto{\pgfqpoint{2.402929in}{2.596894in}}%
\pgfpathlineto{\pgfqpoint{2.402929in}{2.593945in}}%
\pgfpathmoveto{\pgfqpoint{2.402929in}{2.582148in}}%
\pgfpathlineto{\pgfqpoint{2.402929in}{2.582148in}}%
\pgfpathlineto{\pgfqpoint{2.402929in}{2.585097in}}%
\pgfpathlineto{\pgfqpoint{2.407470in}{2.585097in}}%
\pgfpathlineto{\pgfqpoint{2.407470in}{2.582148in}}%
\pgfpathmoveto{\pgfqpoint{2.402929in}{2.585097in}}%
\pgfpathlineto{\pgfqpoint{2.402929in}{2.585097in}}%
\pgfpathlineto{\pgfqpoint{2.402929in}{2.588046in}}%
\pgfpathlineto{\pgfqpoint{2.407470in}{2.588046in}}%
\pgfpathlineto{\pgfqpoint{2.407470in}{2.585097in}}%
\pgfpathmoveto{\pgfqpoint{2.407470in}{2.582148in}}%
\pgfpathlineto{\pgfqpoint{2.407470in}{2.582148in}}%
\pgfpathlineto{\pgfqpoint{2.407470in}{2.585097in}}%
\pgfpathlineto{\pgfqpoint{2.412011in}{2.585097in}}%
\pgfpathlineto{\pgfqpoint{2.412011in}{2.582148in}}%
\pgfpathmoveto{\pgfqpoint{2.407470in}{2.585097in}}%
\pgfpathlineto{\pgfqpoint{2.407470in}{2.585097in}}%
\pgfpathlineto{\pgfqpoint{2.407470in}{2.588046in}}%
\pgfpathlineto{\pgfqpoint{2.412011in}{2.588046in}}%
\pgfpathlineto{\pgfqpoint{2.412011in}{2.585097in}}%
\pgfpathmoveto{\pgfqpoint{2.412011in}{2.576249in}}%
\pgfpathlineto{\pgfqpoint{2.412011in}{2.576249in}}%
\pgfpathlineto{\pgfqpoint{2.412011in}{2.579199in}}%
\pgfpathlineto{\pgfqpoint{2.416552in}{2.579199in}}%
\pgfpathlineto{\pgfqpoint{2.416552in}{2.576249in}}%
\pgfpathmoveto{\pgfqpoint{2.412011in}{2.579199in}}%
\pgfpathlineto{\pgfqpoint{2.412011in}{2.579199in}}%
\pgfpathlineto{\pgfqpoint{2.412011in}{2.582148in}}%
\pgfpathlineto{\pgfqpoint{2.416552in}{2.582148in}}%
\pgfpathlineto{\pgfqpoint{2.416552in}{2.579199in}}%
\pgfpathmoveto{\pgfqpoint{2.416552in}{2.576249in}}%
\pgfpathlineto{\pgfqpoint{2.416552in}{2.576249in}}%
\pgfpathlineto{\pgfqpoint{2.416552in}{2.579199in}}%
\pgfpathlineto{\pgfqpoint{2.421093in}{2.579199in}}%
\pgfpathlineto{\pgfqpoint{2.421093in}{2.576249in}}%
\pgfpathmoveto{\pgfqpoint{2.416552in}{2.579199in}}%
\pgfpathlineto{\pgfqpoint{2.416552in}{2.579199in}}%
\pgfpathlineto{\pgfqpoint{2.416552in}{2.582148in}}%
\pgfpathlineto{\pgfqpoint{2.421093in}{2.582148in}}%
\pgfpathlineto{\pgfqpoint{2.421093in}{2.579199in}}%
\pgfpathmoveto{\pgfqpoint{2.412011in}{2.582148in}}%
\pgfpathlineto{\pgfqpoint{2.412011in}{2.582148in}}%
\pgfpathlineto{\pgfqpoint{2.412011in}{2.585097in}}%
\pgfpathlineto{\pgfqpoint{2.416552in}{2.585097in}}%
\pgfpathlineto{\pgfqpoint{2.416552in}{2.582148in}}%
\pgfpathmoveto{\pgfqpoint{2.402929in}{2.588046in}}%
\pgfpathlineto{\pgfqpoint{2.402929in}{2.588046in}}%
\pgfpathlineto{\pgfqpoint{2.402929in}{2.590995in}}%
\pgfpathlineto{\pgfqpoint{2.407470in}{2.590995in}}%
\pgfpathlineto{\pgfqpoint{2.407470in}{2.588046in}}%
\pgfpathmoveto{\pgfqpoint{2.402929in}{2.590995in}}%
\pgfpathlineto{\pgfqpoint{2.402929in}{2.590995in}}%
\pgfpathlineto{\pgfqpoint{2.402929in}{2.593945in}}%
\pgfpathlineto{\pgfqpoint{2.407470in}{2.593945in}}%
\pgfpathlineto{\pgfqpoint{2.407470in}{2.590995in}}%
\pgfpathmoveto{\pgfqpoint{2.407470in}{2.588046in}}%
\pgfpathlineto{\pgfqpoint{2.407470in}{2.588046in}}%
\pgfpathlineto{\pgfqpoint{2.407470in}{2.590995in}}%
\pgfpathlineto{\pgfqpoint{2.412011in}{2.590995in}}%
\pgfpathlineto{\pgfqpoint{2.412011in}{2.588046in}}%
\pgfpathmoveto{\pgfqpoint{2.384766in}{2.599843in}}%
\pgfpathlineto{\pgfqpoint{2.384766in}{2.599843in}}%
\pgfpathlineto{\pgfqpoint{2.384766in}{2.602792in}}%
\pgfpathlineto{\pgfqpoint{2.389307in}{2.602792in}}%
\pgfpathlineto{\pgfqpoint{2.389307in}{2.599843in}}%
\pgfpathmoveto{\pgfqpoint{2.384766in}{2.602792in}}%
\pgfpathlineto{\pgfqpoint{2.384766in}{2.602792in}}%
\pgfpathlineto{\pgfqpoint{2.384766in}{2.605741in}}%
\pgfpathlineto{\pgfqpoint{2.389307in}{2.605741in}}%
\pgfpathlineto{\pgfqpoint{2.389307in}{2.602792in}}%
\pgfpathmoveto{\pgfqpoint{2.389307in}{2.599843in}}%
\pgfpathlineto{\pgfqpoint{2.389307in}{2.599843in}}%
\pgfpathlineto{\pgfqpoint{2.389307in}{2.602792in}}%
\pgfpathlineto{\pgfqpoint{2.393848in}{2.602792in}}%
\pgfpathlineto{\pgfqpoint{2.393848in}{2.599843in}}%
\pgfpathmoveto{\pgfqpoint{2.389307in}{2.602792in}}%
\pgfpathlineto{\pgfqpoint{2.389307in}{2.602792in}}%
\pgfpathlineto{\pgfqpoint{2.389307in}{2.605741in}}%
\pgfpathlineto{\pgfqpoint{2.393848in}{2.605741in}}%
\pgfpathlineto{\pgfqpoint{2.393848in}{2.602792in}}%
\pgfpathmoveto{\pgfqpoint{2.384766in}{2.605741in}}%
\pgfpathlineto{\pgfqpoint{2.384766in}{2.605741in}}%
\pgfpathlineto{\pgfqpoint{2.384766in}{2.608691in}}%
\pgfpathlineto{\pgfqpoint{2.389307in}{2.608691in}}%
\pgfpathlineto{\pgfqpoint{2.389307in}{2.605741in}}%
\pgfpathmoveto{\pgfqpoint{2.393848in}{2.599843in}}%
\pgfpathlineto{\pgfqpoint{2.393848in}{2.599843in}}%
\pgfpathlineto{\pgfqpoint{2.393848in}{2.602792in}}%
\pgfpathlineto{\pgfqpoint{2.398388in}{2.602792in}}%
\pgfpathlineto{\pgfqpoint{2.398388in}{2.599843in}}%
\pgfpathmoveto{\pgfqpoint{2.348439in}{2.629335in}}%
\pgfpathlineto{\pgfqpoint{2.348439in}{2.629335in}}%
\pgfpathlineto{\pgfqpoint{2.348439in}{2.632284in}}%
\pgfpathlineto{\pgfqpoint{2.352980in}{2.632284in}}%
\pgfpathlineto{\pgfqpoint{2.352980in}{2.629335in}}%
\pgfpathmoveto{\pgfqpoint{2.348439in}{2.632284in}}%
\pgfpathlineto{\pgfqpoint{2.348439in}{2.632284in}}%
\pgfpathlineto{\pgfqpoint{2.348439in}{2.635233in}}%
\pgfpathlineto{\pgfqpoint{2.352980in}{2.635233in}}%
\pgfpathlineto{\pgfqpoint{2.352980in}{2.632284in}}%
\pgfpathmoveto{\pgfqpoint{2.352980in}{2.629335in}}%
\pgfpathlineto{\pgfqpoint{2.352980in}{2.629335in}}%
\pgfpathlineto{\pgfqpoint{2.352980in}{2.632284in}}%
\pgfpathlineto{\pgfqpoint{2.357521in}{2.632284in}}%
\pgfpathlineto{\pgfqpoint{2.357521in}{2.629335in}}%
\pgfpathmoveto{\pgfqpoint{2.352980in}{2.632284in}}%
\pgfpathlineto{\pgfqpoint{2.352980in}{2.632284in}}%
\pgfpathlineto{\pgfqpoint{2.352980in}{2.635233in}}%
\pgfpathlineto{\pgfqpoint{2.357521in}{2.635233in}}%
\pgfpathlineto{\pgfqpoint{2.357521in}{2.632284in}}%
\pgfpathmoveto{\pgfqpoint{2.357521in}{2.623436in}}%
\pgfpathlineto{\pgfqpoint{2.357521in}{2.623436in}}%
\pgfpathlineto{\pgfqpoint{2.357521in}{2.626386in}}%
\pgfpathlineto{\pgfqpoint{2.362062in}{2.626386in}}%
\pgfpathlineto{\pgfqpoint{2.362062in}{2.623436in}}%
\pgfpathmoveto{\pgfqpoint{2.357521in}{2.626386in}}%
\pgfpathlineto{\pgfqpoint{2.357521in}{2.626386in}}%
\pgfpathlineto{\pgfqpoint{2.357521in}{2.629335in}}%
\pgfpathlineto{\pgfqpoint{2.362062in}{2.629335in}}%
\pgfpathlineto{\pgfqpoint{2.362062in}{2.626386in}}%
\pgfpathmoveto{\pgfqpoint{2.362062in}{2.623436in}}%
\pgfpathlineto{\pgfqpoint{2.362062in}{2.623436in}}%
\pgfpathlineto{\pgfqpoint{2.362062in}{2.626386in}}%
\pgfpathlineto{\pgfqpoint{2.366602in}{2.626386in}}%
\pgfpathlineto{\pgfqpoint{2.366602in}{2.623436in}}%
\pgfpathmoveto{\pgfqpoint{2.362062in}{2.626386in}}%
\pgfpathlineto{\pgfqpoint{2.362062in}{2.626386in}}%
\pgfpathlineto{\pgfqpoint{2.362062in}{2.629335in}}%
\pgfpathlineto{\pgfqpoint{2.366602in}{2.629335in}}%
\pgfpathlineto{\pgfqpoint{2.366602in}{2.626386in}}%
\pgfpathmoveto{\pgfqpoint{2.357521in}{2.629335in}}%
\pgfpathlineto{\pgfqpoint{2.357521in}{2.629335in}}%
\pgfpathlineto{\pgfqpoint{2.357521in}{2.632284in}}%
\pgfpathlineto{\pgfqpoint{2.362062in}{2.632284in}}%
\pgfpathlineto{\pgfqpoint{2.362062in}{2.629335in}}%
\pgfpathmoveto{\pgfqpoint{2.348439in}{2.635233in}}%
\pgfpathlineto{\pgfqpoint{2.348439in}{2.635233in}}%
\pgfpathlineto{\pgfqpoint{2.348439in}{2.638182in}}%
\pgfpathlineto{\pgfqpoint{2.352980in}{2.638182in}}%
\pgfpathlineto{\pgfqpoint{2.352980in}{2.635233in}}%
\pgfpathmoveto{\pgfqpoint{2.348439in}{2.638182in}}%
\pgfpathlineto{\pgfqpoint{2.348439in}{2.638182in}}%
\pgfpathlineto{\pgfqpoint{2.348439in}{2.641132in}}%
\pgfpathlineto{\pgfqpoint{2.352980in}{2.641132in}}%
\pgfpathlineto{\pgfqpoint{2.352980in}{2.638182in}}%
\pgfpathmoveto{\pgfqpoint{2.366602in}{2.623436in}}%
\pgfpathlineto{\pgfqpoint{2.366602in}{2.623436in}}%
\pgfpathlineto{\pgfqpoint{2.366602in}{2.626386in}}%
\pgfpathlineto{\pgfqpoint{2.371143in}{2.626386in}}%
\pgfpathlineto{\pgfqpoint{2.371143in}{2.623436in}}%
\pgfpathmoveto{\pgfqpoint{2.421093in}{2.576249in}}%
\pgfpathlineto{\pgfqpoint{2.421093in}{2.576249in}}%
\pgfpathlineto{\pgfqpoint{2.421093in}{2.579199in}}%
\pgfpathlineto{\pgfqpoint{2.425633in}{2.579199in}}%
\pgfpathlineto{\pgfqpoint{2.425633in}{2.576249in}}%
\pgfpathmoveto{\pgfqpoint{2.557320in}{2.446484in}}%
\pgfpathlineto{\pgfqpoint{2.557320in}{2.446484in}}%
\pgfpathlineto{\pgfqpoint{2.557320in}{2.449433in}}%
\pgfpathlineto{\pgfqpoint{2.561861in}{2.449433in}}%
\pgfpathlineto{\pgfqpoint{2.561861in}{2.446484in}}%
\pgfpathmoveto{\pgfqpoint{2.557320in}{2.449433in}}%
\pgfpathlineto{\pgfqpoint{2.557320in}{2.449433in}}%
\pgfpathlineto{\pgfqpoint{2.557320in}{2.452382in}}%
\pgfpathlineto{\pgfqpoint{2.561861in}{2.452382in}}%
\pgfpathlineto{\pgfqpoint{2.561861in}{2.449433in}}%
\pgfpathmoveto{\pgfqpoint{2.561861in}{2.446484in}}%
\pgfpathlineto{\pgfqpoint{2.561861in}{2.446484in}}%
\pgfpathlineto{\pgfqpoint{2.561861in}{2.449433in}}%
\pgfpathlineto{\pgfqpoint{2.566402in}{2.449433in}}%
\pgfpathlineto{\pgfqpoint{2.566402in}{2.446484in}}%
\pgfpathmoveto{\pgfqpoint{2.561861in}{2.449433in}}%
\pgfpathlineto{\pgfqpoint{2.561861in}{2.449433in}}%
\pgfpathlineto{\pgfqpoint{2.561861in}{2.452382in}}%
\pgfpathlineto{\pgfqpoint{2.566402in}{2.452382in}}%
\pgfpathlineto{\pgfqpoint{2.566402in}{2.449433in}}%
\pgfpathmoveto{\pgfqpoint{2.557320in}{2.452382in}}%
\pgfpathlineto{\pgfqpoint{2.557320in}{2.452382in}}%
\pgfpathlineto{\pgfqpoint{2.557320in}{2.455332in}}%
\pgfpathlineto{\pgfqpoint{2.561861in}{2.455332in}}%
\pgfpathlineto{\pgfqpoint{2.561861in}{2.452382in}}%
\pgfpathmoveto{\pgfqpoint{2.557320in}{2.455332in}}%
\pgfpathlineto{\pgfqpoint{2.557320in}{2.455332in}}%
\pgfpathlineto{\pgfqpoint{2.557320in}{2.458281in}}%
\pgfpathlineto{\pgfqpoint{2.561861in}{2.458281in}}%
\pgfpathlineto{\pgfqpoint{2.561861in}{2.455332in}}%
\pgfpathmoveto{\pgfqpoint{2.561861in}{2.452382in}}%
\pgfpathlineto{\pgfqpoint{2.561861in}{2.452382in}}%
\pgfpathlineto{\pgfqpoint{2.561861in}{2.455332in}}%
\pgfpathlineto{\pgfqpoint{2.566402in}{2.455332in}}%
\pgfpathlineto{\pgfqpoint{2.566402in}{2.452382in}}%
\pgfpathmoveto{\pgfqpoint{2.539156in}{2.464180in}}%
\pgfpathlineto{\pgfqpoint{2.539156in}{2.464180in}}%
\pgfpathlineto{\pgfqpoint{2.539156in}{2.467129in}}%
\pgfpathlineto{\pgfqpoint{2.543697in}{2.467129in}}%
\pgfpathlineto{\pgfqpoint{2.543697in}{2.464180in}}%
\pgfpathmoveto{\pgfqpoint{2.539156in}{2.467129in}}%
\pgfpathlineto{\pgfqpoint{2.539156in}{2.467129in}}%
\pgfpathlineto{\pgfqpoint{2.539156in}{2.470078in}}%
\pgfpathlineto{\pgfqpoint{2.543697in}{2.470078in}}%
\pgfpathlineto{\pgfqpoint{2.543697in}{2.467129in}}%
\pgfpathmoveto{\pgfqpoint{2.543697in}{2.464180in}}%
\pgfpathlineto{\pgfqpoint{2.543697in}{2.464180in}}%
\pgfpathlineto{\pgfqpoint{2.543697in}{2.467129in}}%
\pgfpathlineto{\pgfqpoint{2.548238in}{2.467129in}}%
\pgfpathlineto{\pgfqpoint{2.548238in}{2.464180in}}%
\pgfpathmoveto{\pgfqpoint{2.543697in}{2.467129in}}%
\pgfpathlineto{\pgfqpoint{2.543697in}{2.467129in}}%
\pgfpathlineto{\pgfqpoint{2.543697in}{2.470078in}}%
\pgfpathlineto{\pgfqpoint{2.548238in}{2.470078in}}%
\pgfpathlineto{\pgfqpoint{2.548238in}{2.467129in}}%
\pgfpathmoveto{\pgfqpoint{2.530074in}{2.470078in}}%
\pgfpathlineto{\pgfqpoint{2.530074in}{2.470078in}}%
\pgfpathlineto{\pgfqpoint{2.530074in}{2.473027in}}%
\pgfpathlineto{\pgfqpoint{2.534615in}{2.473027in}}%
\pgfpathlineto{\pgfqpoint{2.534615in}{2.470078in}}%
\pgfpathmoveto{\pgfqpoint{2.530074in}{2.473027in}}%
\pgfpathlineto{\pgfqpoint{2.530074in}{2.473027in}}%
\pgfpathlineto{\pgfqpoint{2.530074in}{2.475977in}}%
\pgfpathlineto{\pgfqpoint{2.534615in}{2.475977in}}%
\pgfpathlineto{\pgfqpoint{2.534615in}{2.473027in}}%
\pgfpathmoveto{\pgfqpoint{2.534615in}{2.470078in}}%
\pgfpathlineto{\pgfqpoint{2.534615in}{2.470078in}}%
\pgfpathlineto{\pgfqpoint{2.534615in}{2.473027in}}%
\pgfpathlineto{\pgfqpoint{2.539156in}{2.473027in}}%
\pgfpathlineto{\pgfqpoint{2.539156in}{2.470078in}}%
\pgfpathmoveto{\pgfqpoint{2.534615in}{2.473027in}}%
\pgfpathlineto{\pgfqpoint{2.534615in}{2.473027in}}%
\pgfpathlineto{\pgfqpoint{2.534615in}{2.475977in}}%
\pgfpathlineto{\pgfqpoint{2.539156in}{2.475977in}}%
\pgfpathlineto{\pgfqpoint{2.539156in}{2.473027in}}%
\pgfpathmoveto{\pgfqpoint{2.530074in}{2.475977in}}%
\pgfpathlineto{\pgfqpoint{2.530074in}{2.475977in}}%
\pgfpathlineto{\pgfqpoint{2.530074in}{2.478926in}}%
\pgfpathlineto{\pgfqpoint{2.534615in}{2.478926in}}%
\pgfpathlineto{\pgfqpoint{2.534615in}{2.475977in}}%
\pgfpathmoveto{\pgfqpoint{2.530074in}{2.478926in}}%
\pgfpathlineto{\pgfqpoint{2.530074in}{2.478926in}}%
\pgfpathlineto{\pgfqpoint{2.530074in}{2.481875in}}%
\pgfpathlineto{\pgfqpoint{2.534615in}{2.481875in}}%
\pgfpathlineto{\pgfqpoint{2.534615in}{2.478926in}}%
\pgfpathmoveto{\pgfqpoint{2.534615in}{2.475977in}}%
\pgfpathlineto{\pgfqpoint{2.534615in}{2.475977in}}%
\pgfpathlineto{\pgfqpoint{2.534615in}{2.478926in}}%
\pgfpathlineto{\pgfqpoint{2.539156in}{2.478926in}}%
\pgfpathlineto{\pgfqpoint{2.539156in}{2.475977in}}%
\pgfpathmoveto{\pgfqpoint{2.539156in}{2.470078in}}%
\pgfpathlineto{\pgfqpoint{2.539156in}{2.470078in}}%
\pgfpathlineto{\pgfqpoint{2.539156in}{2.473027in}}%
\pgfpathlineto{\pgfqpoint{2.543697in}{2.473027in}}%
\pgfpathlineto{\pgfqpoint{2.543697in}{2.470078in}}%
\pgfpathmoveto{\pgfqpoint{2.539156in}{2.473027in}}%
\pgfpathlineto{\pgfqpoint{2.539156in}{2.473027in}}%
\pgfpathlineto{\pgfqpoint{2.539156in}{2.475977in}}%
\pgfpathlineto{\pgfqpoint{2.543697in}{2.475977in}}%
\pgfpathlineto{\pgfqpoint{2.543697in}{2.473027in}}%
\pgfpathmoveto{\pgfqpoint{2.543697in}{2.470078in}}%
\pgfpathlineto{\pgfqpoint{2.543697in}{2.470078in}}%
\pgfpathlineto{\pgfqpoint{2.543697in}{2.473027in}}%
\pgfpathlineto{\pgfqpoint{2.548238in}{2.473027in}}%
\pgfpathlineto{\pgfqpoint{2.548238in}{2.470078in}}%
\pgfpathmoveto{\pgfqpoint{2.548238in}{2.458281in}}%
\pgfpathlineto{\pgfqpoint{2.548238in}{2.458281in}}%
\pgfpathlineto{\pgfqpoint{2.548238in}{2.461230in}}%
\pgfpathlineto{\pgfqpoint{2.552779in}{2.461230in}}%
\pgfpathlineto{\pgfqpoint{2.552779in}{2.458281in}}%
\pgfpathmoveto{\pgfqpoint{2.548238in}{2.461230in}}%
\pgfpathlineto{\pgfqpoint{2.548238in}{2.461230in}}%
\pgfpathlineto{\pgfqpoint{2.548238in}{2.464180in}}%
\pgfpathlineto{\pgfqpoint{2.552779in}{2.464180in}}%
\pgfpathlineto{\pgfqpoint{2.552779in}{2.461230in}}%
\pgfpathmoveto{\pgfqpoint{2.552779in}{2.458281in}}%
\pgfpathlineto{\pgfqpoint{2.552779in}{2.458281in}}%
\pgfpathlineto{\pgfqpoint{2.552779in}{2.461230in}}%
\pgfpathlineto{\pgfqpoint{2.557320in}{2.461230in}}%
\pgfpathlineto{\pgfqpoint{2.557320in}{2.458281in}}%
\pgfpathmoveto{\pgfqpoint{2.552779in}{2.461230in}}%
\pgfpathlineto{\pgfqpoint{2.552779in}{2.461230in}}%
\pgfpathlineto{\pgfqpoint{2.552779in}{2.464180in}}%
\pgfpathlineto{\pgfqpoint{2.557320in}{2.464180in}}%
\pgfpathlineto{\pgfqpoint{2.557320in}{2.461230in}}%
\pgfpathmoveto{\pgfqpoint{2.548238in}{2.464180in}}%
\pgfpathlineto{\pgfqpoint{2.548238in}{2.464180in}}%
\pgfpathlineto{\pgfqpoint{2.548238in}{2.467129in}}%
\pgfpathlineto{\pgfqpoint{2.552779in}{2.467129in}}%
\pgfpathlineto{\pgfqpoint{2.552779in}{2.464180in}}%
\pgfpathmoveto{\pgfqpoint{2.557320in}{2.458281in}}%
\pgfpathlineto{\pgfqpoint{2.557320in}{2.458281in}}%
\pgfpathlineto{\pgfqpoint{2.557320in}{2.461230in}}%
\pgfpathlineto{\pgfqpoint{2.561861in}{2.461230in}}%
\pgfpathlineto{\pgfqpoint{2.561861in}{2.458281in}}%
\pgfpathmoveto{\pgfqpoint{2.593648in}{2.416991in}}%
\pgfpathlineto{\pgfqpoint{2.593648in}{2.416991in}}%
\pgfpathlineto{\pgfqpoint{2.593648in}{2.419940in}}%
\pgfpathlineto{\pgfqpoint{2.598189in}{2.419940in}}%
\pgfpathlineto{\pgfqpoint{2.598189in}{2.416991in}}%
\pgfpathmoveto{\pgfqpoint{2.593648in}{2.419940in}}%
\pgfpathlineto{\pgfqpoint{2.593648in}{2.419940in}}%
\pgfpathlineto{\pgfqpoint{2.593648in}{2.422890in}}%
\pgfpathlineto{\pgfqpoint{2.598189in}{2.422890in}}%
\pgfpathlineto{\pgfqpoint{2.598189in}{2.419940in}}%
\pgfpathmoveto{\pgfqpoint{2.598189in}{2.416991in}}%
\pgfpathlineto{\pgfqpoint{2.598189in}{2.416991in}}%
\pgfpathlineto{\pgfqpoint{2.598189in}{2.419940in}}%
\pgfpathlineto{\pgfqpoint{2.602730in}{2.419940in}}%
\pgfpathlineto{\pgfqpoint{2.602730in}{2.416991in}}%
\pgfpathmoveto{\pgfqpoint{2.598189in}{2.419940in}}%
\pgfpathlineto{\pgfqpoint{2.598189in}{2.419940in}}%
\pgfpathlineto{\pgfqpoint{2.598189in}{2.422890in}}%
\pgfpathlineto{\pgfqpoint{2.602730in}{2.422890in}}%
\pgfpathlineto{\pgfqpoint{2.602730in}{2.419940in}}%
\pgfpathmoveto{\pgfqpoint{2.584566in}{2.422890in}}%
\pgfpathlineto{\pgfqpoint{2.584566in}{2.422890in}}%
\pgfpathlineto{\pgfqpoint{2.584566in}{2.425839in}}%
\pgfpathlineto{\pgfqpoint{2.589107in}{2.425839in}}%
\pgfpathlineto{\pgfqpoint{2.589107in}{2.422890in}}%
\pgfpathmoveto{\pgfqpoint{2.584566in}{2.425839in}}%
\pgfpathlineto{\pgfqpoint{2.584566in}{2.425839in}}%
\pgfpathlineto{\pgfqpoint{2.584566in}{2.428788in}}%
\pgfpathlineto{\pgfqpoint{2.589107in}{2.428788in}}%
\pgfpathlineto{\pgfqpoint{2.589107in}{2.425839in}}%
\pgfpathmoveto{\pgfqpoint{2.589107in}{2.422890in}}%
\pgfpathlineto{\pgfqpoint{2.589107in}{2.422890in}}%
\pgfpathlineto{\pgfqpoint{2.589107in}{2.425839in}}%
\pgfpathlineto{\pgfqpoint{2.593648in}{2.425839in}}%
\pgfpathlineto{\pgfqpoint{2.593648in}{2.422890in}}%
\pgfpathmoveto{\pgfqpoint{2.589107in}{2.425839in}}%
\pgfpathlineto{\pgfqpoint{2.589107in}{2.425839in}}%
\pgfpathlineto{\pgfqpoint{2.589107in}{2.428788in}}%
\pgfpathlineto{\pgfqpoint{2.593648in}{2.428788in}}%
\pgfpathlineto{\pgfqpoint{2.593648in}{2.425839in}}%
\pgfpathmoveto{\pgfqpoint{2.584566in}{2.428788in}}%
\pgfpathlineto{\pgfqpoint{2.584566in}{2.428788in}}%
\pgfpathlineto{\pgfqpoint{2.584566in}{2.431738in}}%
\pgfpathlineto{\pgfqpoint{2.589107in}{2.431738in}}%
\pgfpathlineto{\pgfqpoint{2.589107in}{2.428788in}}%
\pgfpathmoveto{\pgfqpoint{2.584566in}{2.431738in}}%
\pgfpathlineto{\pgfqpoint{2.584566in}{2.431738in}}%
\pgfpathlineto{\pgfqpoint{2.584566in}{2.434687in}}%
\pgfpathlineto{\pgfqpoint{2.589107in}{2.434687in}}%
\pgfpathlineto{\pgfqpoint{2.589107in}{2.431738in}}%
\pgfpathmoveto{\pgfqpoint{2.589107in}{2.428788in}}%
\pgfpathlineto{\pgfqpoint{2.589107in}{2.428788in}}%
\pgfpathlineto{\pgfqpoint{2.589107in}{2.431738in}}%
\pgfpathlineto{\pgfqpoint{2.593648in}{2.431738in}}%
\pgfpathlineto{\pgfqpoint{2.593648in}{2.428788in}}%
\pgfpathmoveto{\pgfqpoint{2.593648in}{2.422890in}}%
\pgfpathlineto{\pgfqpoint{2.593648in}{2.422890in}}%
\pgfpathlineto{\pgfqpoint{2.593648in}{2.425839in}}%
\pgfpathlineto{\pgfqpoint{2.598189in}{2.425839in}}%
\pgfpathlineto{\pgfqpoint{2.598189in}{2.422890in}}%
\pgfpathmoveto{\pgfqpoint{2.593648in}{2.425839in}}%
\pgfpathlineto{\pgfqpoint{2.593648in}{2.425839in}}%
\pgfpathlineto{\pgfqpoint{2.593648in}{2.428788in}}%
\pgfpathlineto{\pgfqpoint{2.598189in}{2.428788in}}%
\pgfpathlineto{\pgfqpoint{2.598189in}{2.425839in}}%
\pgfpathmoveto{\pgfqpoint{2.598189in}{2.422890in}}%
\pgfpathlineto{\pgfqpoint{2.598189in}{2.422890in}}%
\pgfpathlineto{\pgfqpoint{2.598189in}{2.425839in}}%
\pgfpathlineto{\pgfqpoint{2.602730in}{2.425839in}}%
\pgfpathlineto{\pgfqpoint{2.602730in}{2.422890in}}%
\pgfpathmoveto{\pgfqpoint{2.611812in}{2.399295in}}%
\pgfpathlineto{\pgfqpoint{2.611812in}{2.399295in}}%
\pgfpathlineto{\pgfqpoint{2.611812in}{2.402245in}}%
\pgfpathlineto{\pgfqpoint{2.616353in}{2.402245in}}%
\pgfpathlineto{\pgfqpoint{2.616353in}{2.399295in}}%
\pgfpathmoveto{\pgfqpoint{2.611812in}{2.402245in}}%
\pgfpathlineto{\pgfqpoint{2.611812in}{2.402245in}}%
\pgfpathlineto{\pgfqpoint{2.611812in}{2.405194in}}%
\pgfpathlineto{\pgfqpoint{2.616353in}{2.405194in}}%
\pgfpathlineto{\pgfqpoint{2.616353in}{2.402245in}}%
\pgfpathmoveto{\pgfqpoint{2.616353in}{2.399295in}}%
\pgfpathlineto{\pgfqpoint{2.616353in}{2.399295in}}%
\pgfpathlineto{\pgfqpoint{2.616353in}{2.402245in}}%
\pgfpathlineto{\pgfqpoint{2.620894in}{2.402245in}}%
\pgfpathlineto{\pgfqpoint{2.620894in}{2.399295in}}%
\pgfpathmoveto{\pgfqpoint{2.616353in}{2.402245in}}%
\pgfpathlineto{\pgfqpoint{2.616353in}{2.402245in}}%
\pgfpathlineto{\pgfqpoint{2.616353in}{2.405194in}}%
\pgfpathlineto{\pgfqpoint{2.620894in}{2.405194in}}%
\pgfpathlineto{\pgfqpoint{2.620894in}{2.402245in}}%
\pgfpathmoveto{\pgfqpoint{2.611812in}{2.405194in}}%
\pgfpathlineto{\pgfqpoint{2.611812in}{2.405194in}}%
\pgfpathlineto{\pgfqpoint{2.611812in}{2.408143in}}%
\pgfpathlineto{\pgfqpoint{2.616353in}{2.408143in}}%
\pgfpathlineto{\pgfqpoint{2.616353in}{2.405194in}}%
\pgfpathmoveto{\pgfqpoint{2.611812in}{2.408143in}}%
\pgfpathlineto{\pgfqpoint{2.611812in}{2.408143in}}%
\pgfpathlineto{\pgfqpoint{2.611812in}{2.411093in}}%
\pgfpathlineto{\pgfqpoint{2.616353in}{2.411093in}}%
\pgfpathlineto{\pgfqpoint{2.616353in}{2.408143in}}%
\pgfpathmoveto{\pgfqpoint{2.616353in}{2.405194in}}%
\pgfpathlineto{\pgfqpoint{2.616353in}{2.405194in}}%
\pgfpathlineto{\pgfqpoint{2.616353in}{2.408143in}}%
\pgfpathlineto{\pgfqpoint{2.620894in}{2.408143in}}%
\pgfpathlineto{\pgfqpoint{2.620894in}{2.405194in}}%
\pgfpathmoveto{\pgfqpoint{2.620894in}{2.393397in}}%
\pgfpathlineto{\pgfqpoint{2.620894in}{2.393397in}}%
\pgfpathlineto{\pgfqpoint{2.620894in}{2.396346in}}%
\pgfpathlineto{\pgfqpoint{2.625435in}{2.396346in}}%
\pgfpathlineto{\pgfqpoint{2.625435in}{2.393397in}}%
\pgfpathmoveto{\pgfqpoint{2.620894in}{2.396346in}}%
\pgfpathlineto{\pgfqpoint{2.620894in}{2.396346in}}%
\pgfpathlineto{\pgfqpoint{2.620894in}{2.399295in}}%
\pgfpathlineto{\pgfqpoint{2.625435in}{2.399295in}}%
\pgfpathlineto{\pgfqpoint{2.625435in}{2.396346in}}%
\pgfpathmoveto{\pgfqpoint{2.625435in}{2.393397in}}%
\pgfpathlineto{\pgfqpoint{2.625435in}{2.393397in}}%
\pgfpathlineto{\pgfqpoint{2.625435in}{2.396346in}}%
\pgfpathlineto{\pgfqpoint{2.629976in}{2.396346in}}%
\pgfpathlineto{\pgfqpoint{2.629976in}{2.393397in}}%
\pgfpathmoveto{\pgfqpoint{2.625435in}{2.396346in}}%
\pgfpathlineto{\pgfqpoint{2.625435in}{2.396346in}}%
\pgfpathlineto{\pgfqpoint{2.625435in}{2.399295in}}%
\pgfpathlineto{\pgfqpoint{2.629976in}{2.399295in}}%
\pgfpathlineto{\pgfqpoint{2.629976in}{2.396346in}}%
\pgfpathmoveto{\pgfqpoint{2.629976in}{2.387498in}}%
\pgfpathlineto{\pgfqpoint{2.629976in}{2.387498in}}%
\pgfpathlineto{\pgfqpoint{2.629976in}{2.390448in}}%
\pgfpathlineto{\pgfqpoint{2.634517in}{2.390448in}}%
\pgfpathlineto{\pgfqpoint{2.634517in}{2.387498in}}%
\pgfpathmoveto{\pgfqpoint{2.629976in}{2.390448in}}%
\pgfpathlineto{\pgfqpoint{2.629976in}{2.390448in}}%
\pgfpathlineto{\pgfqpoint{2.629976in}{2.393397in}}%
\pgfpathlineto{\pgfqpoint{2.634517in}{2.393397in}}%
\pgfpathlineto{\pgfqpoint{2.634517in}{2.390448in}}%
\pgfpathmoveto{\pgfqpoint{2.634517in}{2.387498in}}%
\pgfpathlineto{\pgfqpoint{2.634517in}{2.387498in}}%
\pgfpathlineto{\pgfqpoint{2.634517in}{2.390448in}}%
\pgfpathlineto{\pgfqpoint{2.639058in}{2.390448in}}%
\pgfpathlineto{\pgfqpoint{2.639058in}{2.387498in}}%
\pgfpathmoveto{\pgfqpoint{2.634517in}{2.390448in}}%
\pgfpathlineto{\pgfqpoint{2.634517in}{2.390448in}}%
\pgfpathlineto{\pgfqpoint{2.634517in}{2.393397in}}%
\pgfpathlineto{\pgfqpoint{2.639058in}{2.393397in}}%
\pgfpathlineto{\pgfqpoint{2.639058in}{2.390448in}}%
\pgfpathmoveto{\pgfqpoint{2.629976in}{2.393397in}}%
\pgfpathlineto{\pgfqpoint{2.629976in}{2.393397in}}%
\pgfpathlineto{\pgfqpoint{2.629976in}{2.396346in}}%
\pgfpathlineto{\pgfqpoint{2.634517in}{2.396346in}}%
\pgfpathlineto{\pgfqpoint{2.634517in}{2.393397in}}%
\pgfpathmoveto{\pgfqpoint{2.620894in}{2.399295in}}%
\pgfpathlineto{\pgfqpoint{2.620894in}{2.399295in}}%
\pgfpathlineto{\pgfqpoint{2.620894in}{2.402245in}}%
\pgfpathlineto{\pgfqpoint{2.625435in}{2.402245in}}%
\pgfpathlineto{\pgfqpoint{2.625435in}{2.399295in}}%
\pgfpathmoveto{\pgfqpoint{2.620894in}{2.402245in}}%
\pgfpathlineto{\pgfqpoint{2.620894in}{2.402245in}}%
\pgfpathlineto{\pgfqpoint{2.620894in}{2.405194in}}%
\pgfpathlineto{\pgfqpoint{2.625435in}{2.405194in}}%
\pgfpathlineto{\pgfqpoint{2.625435in}{2.402245in}}%
\pgfpathmoveto{\pgfqpoint{2.625435in}{2.399295in}}%
\pgfpathlineto{\pgfqpoint{2.625435in}{2.399295in}}%
\pgfpathlineto{\pgfqpoint{2.625435in}{2.402245in}}%
\pgfpathlineto{\pgfqpoint{2.629976in}{2.402245in}}%
\pgfpathlineto{\pgfqpoint{2.629976in}{2.399295in}}%
\pgfpathmoveto{\pgfqpoint{2.602730in}{2.411093in}}%
\pgfpathlineto{\pgfqpoint{2.602730in}{2.411093in}}%
\pgfpathlineto{\pgfqpoint{2.602730in}{2.414042in}}%
\pgfpathlineto{\pgfqpoint{2.607271in}{2.414042in}}%
\pgfpathlineto{\pgfqpoint{2.607271in}{2.411093in}}%
\pgfpathmoveto{\pgfqpoint{2.602730in}{2.414042in}}%
\pgfpathlineto{\pgfqpoint{2.602730in}{2.414042in}}%
\pgfpathlineto{\pgfqpoint{2.602730in}{2.416991in}}%
\pgfpathlineto{\pgfqpoint{2.607271in}{2.416991in}}%
\pgfpathlineto{\pgfqpoint{2.607271in}{2.414042in}}%
\pgfpathmoveto{\pgfqpoint{2.607271in}{2.411093in}}%
\pgfpathlineto{\pgfqpoint{2.607271in}{2.411093in}}%
\pgfpathlineto{\pgfqpoint{2.607271in}{2.414042in}}%
\pgfpathlineto{\pgfqpoint{2.611812in}{2.414042in}}%
\pgfpathlineto{\pgfqpoint{2.611812in}{2.411093in}}%
\pgfpathmoveto{\pgfqpoint{2.607271in}{2.414042in}}%
\pgfpathlineto{\pgfqpoint{2.607271in}{2.414042in}}%
\pgfpathlineto{\pgfqpoint{2.607271in}{2.416991in}}%
\pgfpathlineto{\pgfqpoint{2.611812in}{2.416991in}}%
\pgfpathlineto{\pgfqpoint{2.611812in}{2.414042in}}%
\pgfpathmoveto{\pgfqpoint{2.602730in}{2.416991in}}%
\pgfpathlineto{\pgfqpoint{2.602730in}{2.416991in}}%
\pgfpathlineto{\pgfqpoint{2.602730in}{2.419940in}}%
\pgfpathlineto{\pgfqpoint{2.607271in}{2.419940in}}%
\pgfpathlineto{\pgfqpoint{2.607271in}{2.416991in}}%
\pgfpathmoveto{\pgfqpoint{2.611812in}{2.411093in}}%
\pgfpathlineto{\pgfqpoint{2.611812in}{2.411093in}}%
\pgfpathlineto{\pgfqpoint{2.611812in}{2.414042in}}%
\pgfpathlineto{\pgfqpoint{2.616353in}{2.414042in}}%
\pgfpathlineto{\pgfqpoint{2.616353in}{2.411093in}}%
\pgfpathmoveto{\pgfqpoint{2.566402in}{2.440585in}}%
\pgfpathlineto{\pgfqpoint{2.566402in}{2.440585in}}%
\pgfpathlineto{\pgfqpoint{2.566402in}{2.443535in}}%
\pgfpathlineto{\pgfqpoint{2.570943in}{2.443535in}}%
\pgfpathlineto{\pgfqpoint{2.570943in}{2.440585in}}%
\pgfpathmoveto{\pgfqpoint{2.566402in}{2.443535in}}%
\pgfpathlineto{\pgfqpoint{2.566402in}{2.443535in}}%
\pgfpathlineto{\pgfqpoint{2.566402in}{2.446484in}}%
\pgfpathlineto{\pgfqpoint{2.570943in}{2.446484in}}%
\pgfpathlineto{\pgfqpoint{2.570943in}{2.443535in}}%
\pgfpathmoveto{\pgfqpoint{2.570943in}{2.440585in}}%
\pgfpathlineto{\pgfqpoint{2.570943in}{2.440585in}}%
\pgfpathlineto{\pgfqpoint{2.570943in}{2.443535in}}%
\pgfpathlineto{\pgfqpoint{2.575484in}{2.443535in}}%
\pgfpathlineto{\pgfqpoint{2.575484in}{2.440585in}}%
\pgfpathmoveto{\pgfqpoint{2.570943in}{2.443535in}}%
\pgfpathlineto{\pgfqpoint{2.570943in}{2.443535in}}%
\pgfpathlineto{\pgfqpoint{2.570943in}{2.446484in}}%
\pgfpathlineto{\pgfqpoint{2.575484in}{2.446484in}}%
\pgfpathlineto{\pgfqpoint{2.575484in}{2.443535in}}%
\pgfpathmoveto{\pgfqpoint{2.575484in}{2.434687in}}%
\pgfpathlineto{\pgfqpoint{2.575484in}{2.434687in}}%
\pgfpathlineto{\pgfqpoint{2.575484in}{2.437636in}}%
\pgfpathlineto{\pgfqpoint{2.580025in}{2.437636in}}%
\pgfpathlineto{\pgfqpoint{2.580025in}{2.434687in}}%
\pgfpathmoveto{\pgfqpoint{2.575484in}{2.437636in}}%
\pgfpathlineto{\pgfqpoint{2.575484in}{2.437636in}}%
\pgfpathlineto{\pgfqpoint{2.575484in}{2.440585in}}%
\pgfpathlineto{\pgfqpoint{2.580025in}{2.440585in}}%
\pgfpathlineto{\pgfqpoint{2.580025in}{2.437636in}}%
\pgfpathmoveto{\pgfqpoint{2.580025in}{2.434687in}}%
\pgfpathlineto{\pgfqpoint{2.580025in}{2.434687in}}%
\pgfpathlineto{\pgfqpoint{2.580025in}{2.437636in}}%
\pgfpathlineto{\pgfqpoint{2.584566in}{2.437636in}}%
\pgfpathlineto{\pgfqpoint{2.584566in}{2.434687in}}%
\pgfpathmoveto{\pgfqpoint{2.580025in}{2.437636in}}%
\pgfpathlineto{\pgfqpoint{2.580025in}{2.437636in}}%
\pgfpathlineto{\pgfqpoint{2.580025in}{2.440585in}}%
\pgfpathlineto{\pgfqpoint{2.584566in}{2.440585in}}%
\pgfpathlineto{\pgfqpoint{2.584566in}{2.437636in}}%
\pgfpathmoveto{\pgfqpoint{2.575484in}{2.440585in}}%
\pgfpathlineto{\pgfqpoint{2.575484in}{2.440585in}}%
\pgfpathlineto{\pgfqpoint{2.575484in}{2.443535in}}%
\pgfpathlineto{\pgfqpoint{2.580025in}{2.443535in}}%
\pgfpathlineto{\pgfqpoint{2.580025in}{2.440585in}}%
\pgfpathmoveto{\pgfqpoint{2.566402in}{2.446484in}}%
\pgfpathlineto{\pgfqpoint{2.566402in}{2.446484in}}%
\pgfpathlineto{\pgfqpoint{2.566402in}{2.449433in}}%
\pgfpathlineto{\pgfqpoint{2.570943in}{2.449433in}}%
\pgfpathlineto{\pgfqpoint{2.570943in}{2.446484in}}%
\pgfpathmoveto{\pgfqpoint{2.566402in}{2.449433in}}%
\pgfpathlineto{\pgfqpoint{2.566402in}{2.449433in}}%
\pgfpathlineto{\pgfqpoint{2.566402in}{2.452382in}}%
\pgfpathlineto{\pgfqpoint{2.570943in}{2.452382in}}%
\pgfpathlineto{\pgfqpoint{2.570943in}{2.449433in}}%
\pgfpathmoveto{\pgfqpoint{2.570943in}{2.446484in}}%
\pgfpathlineto{\pgfqpoint{2.570943in}{2.446484in}}%
\pgfpathlineto{\pgfqpoint{2.570943in}{2.449433in}}%
\pgfpathlineto{\pgfqpoint{2.575484in}{2.449433in}}%
\pgfpathlineto{\pgfqpoint{2.575484in}{2.446484in}}%
\pgfpathmoveto{\pgfqpoint{2.584566in}{2.434687in}}%
\pgfpathlineto{\pgfqpoint{2.584566in}{2.434687in}}%
\pgfpathlineto{\pgfqpoint{2.584566in}{2.437636in}}%
\pgfpathlineto{\pgfqpoint{2.589107in}{2.437636in}}%
\pgfpathlineto{\pgfqpoint{2.589107in}{2.434687in}}%
\pgfpathmoveto{\pgfqpoint{2.502828in}{2.493672in}}%
\pgfpathlineto{\pgfqpoint{2.502828in}{2.493672in}}%
\pgfpathlineto{\pgfqpoint{2.502828in}{2.496621in}}%
\pgfpathlineto{\pgfqpoint{2.507369in}{2.496621in}}%
\pgfpathlineto{\pgfqpoint{2.507369in}{2.493672in}}%
\pgfpathmoveto{\pgfqpoint{2.502828in}{2.496621in}}%
\pgfpathlineto{\pgfqpoint{2.502828in}{2.496621in}}%
\pgfpathlineto{\pgfqpoint{2.502828in}{2.499570in}}%
\pgfpathlineto{\pgfqpoint{2.507369in}{2.499570in}}%
\pgfpathlineto{\pgfqpoint{2.507369in}{2.496621in}}%
\pgfpathmoveto{\pgfqpoint{2.507369in}{2.493672in}}%
\pgfpathlineto{\pgfqpoint{2.507369in}{2.493672in}}%
\pgfpathlineto{\pgfqpoint{2.507369in}{2.496621in}}%
\pgfpathlineto{\pgfqpoint{2.511910in}{2.496621in}}%
\pgfpathlineto{\pgfqpoint{2.511910in}{2.493672in}}%
\pgfpathmoveto{\pgfqpoint{2.507369in}{2.496621in}}%
\pgfpathlineto{\pgfqpoint{2.507369in}{2.496621in}}%
\pgfpathlineto{\pgfqpoint{2.507369in}{2.499570in}}%
\pgfpathlineto{\pgfqpoint{2.511910in}{2.499570in}}%
\pgfpathlineto{\pgfqpoint{2.511910in}{2.496621in}}%
\pgfpathmoveto{\pgfqpoint{2.502828in}{2.499570in}}%
\pgfpathlineto{\pgfqpoint{2.502828in}{2.499570in}}%
\pgfpathlineto{\pgfqpoint{2.502828in}{2.502520in}}%
\pgfpathlineto{\pgfqpoint{2.507369in}{2.502520in}}%
\pgfpathlineto{\pgfqpoint{2.507369in}{2.499570in}}%
\pgfpathmoveto{\pgfqpoint{2.502828in}{2.502520in}}%
\pgfpathlineto{\pgfqpoint{2.502828in}{2.502520in}}%
\pgfpathlineto{\pgfqpoint{2.502828in}{2.505469in}}%
\pgfpathlineto{\pgfqpoint{2.507369in}{2.505469in}}%
\pgfpathlineto{\pgfqpoint{2.507369in}{2.502520in}}%
\pgfpathmoveto{\pgfqpoint{2.507369in}{2.499570in}}%
\pgfpathlineto{\pgfqpoint{2.507369in}{2.499570in}}%
\pgfpathlineto{\pgfqpoint{2.507369in}{2.502520in}}%
\pgfpathlineto{\pgfqpoint{2.511910in}{2.502520in}}%
\pgfpathlineto{\pgfqpoint{2.511910in}{2.499570in}}%
\pgfpathmoveto{\pgfqpoint{2.511910in}{2.487774in}}%
\pgfpathlineto{\pgfqpoint{2.511910in}{2.487774in}}%
\pgfpathlineto{\pgfqpoint{2.511910in}{2.490723in}}%
\pgfpathlineto{\pgfqpoint{2.516451in}{2.490723in}}%
\pgfpathlineto{\pgfqpoint{2.516451in}{2.487774in}}%
\pgfpathmoveto{\pgfqpoint{2.511910in}{2.490723in}}%
\pgfpathlineto{\pgfqpoint{2.511910in}{2.490723in}}%
\pgfpathlineto{\pgfqpoint{2.511910in}{2.493672in}}%
\pgfpathlineto{\pgfqpoint{2.516451in}{2.493672in}}%
\pgfpathlineto{\pgfqpoint{2.516451in}{2.490723in}}%
\pgfpathmoveto{\pgfqpoint{2.516451in}{2.487774in}}%
\pgfpathlineto{\pgfqpoint{2.516451in}{2.487774in}}%
\pgfpathlineto{\pgfqpoint{2.516451in}{2.490723in}}%
\pgfpathlineto{\pgfqpoint{2.520992in}{2.490723in}}%
\pgfpathlineto{\pgfqpoint{2.520992in}{2.487774in}}%
\pgfpathmoveto{\pgfqpoint{2.516451in}{2.490723in}}%
\pgfpathlineto{\pgfqpoint{2.516451in}{2.490723in}}%
\pgfpathlineto{\pgfqpoint{2.516451in}{2.493672in}}%
\pgfpathlineto{\pgfqpoint{2.520992in}{2.493672in}}%
\pgfpathlineto{\pgfqpoint{2.520992in}{2.490723in}}%
\pgfpathmoveto{\pgfqpoint{2.520992in}{2.481875in}}%
\pgfpathlineto{\pgfqpoint{2.520992in}{2.481875in}}%
\pgfpathlineto{\pgfqpoint{2.520992in}{2.484824in}}%
\pgfpathlineto{\pgfqpoint{2.525533in}{2.484824in}}%
\pgfpathlineto{\pgfqpoint{2.525533in}{2.481875in}}%
\pgfpathmoveto{\pgfqpoint{2.520992in}{2.484824in}}%
\pgfpathlineto{\pgfqpoint{2.520992in}{2.484824in}}%
\pgfpathlineto{\pgfqpoint{2.520992in}{2.487774in}}%
\pgfpathlineto{\pgfqpoint{2.525533in}{2.487774in}}%
\pgfpathlineto{\pgfqpoint{2.525533in}{2.484824in}}%
\pgfpathmoveto{\pgfqpoint{2.525533in}{2.481875in}}%
\pgfpathlineto{\pgfqpoint{2.525533in}{2.481875in}}%
\pgfpathlineto{\pgfqpoint{2.525533in}{2.484824in}}%
\pgfpathlineto{\pgfqpoint{2.530074in}{2.484824in}}%
\pgfpathlineto{\pgfqpoint{2.530074in}{2.481875in}}%
\pgfpathmoveto{\pgfqpoint{2.525533in}{2.484824in}}%
\pgfpathlineto{\pgfqpoint{2.525533in}{2.484824in}}%
\pgfpathlineto{\pgfqpoint{2.525533in}{2.487774in}}%
\pgfpathlineto{\pgfqpoint{2.530074in}{2.487774in}}%
\pgfpathlineto{\pgfqpoint{2.530074in}{2.484824in}}%
\pgfpathmoveto{\pgfqpoint{2.520992in}{2.487774in}}%
\pgfpathlineto{\pgfqpoint{2.520992in}{2.487774in}}%
\pgfpathlineto{\pgfqpoint{2.520992in}{2.490723in}}%
\pgfpathlineto{\pgfqpoint{2.525533in}{2.490723in}}%
\pgfpathlineto{\pgfqpoint{2.525533in}{2.487774in}}%
\pgfpathmoveto{\pgfqpoint{2.511910in}{2.493672in}}%
\pgfpathlineto{\pgfqpoint{2.511910in}{2.493672in}}%
\pgfpathlineto{\pgfqpoint{2.511910in}{2.496621in}}%
\pgfpathlineto{\pgfqpoint{2.516451in}{2.496621in}}%
\pgfpathlineto{\pgfqpoint{2.516451in}{2.493672in}}%
\pgfpathmoveto{\pgfqpoint{2.511910in}{2.496621in}}%
\pgfpathlineto{\pgfqpoint{2.511910in}{2.496621in}}%
\pgfpathlineto{\pgfqpoint{2.511910in}{2.499570in}}%
\pgfpathlineto{\pgfqpoint{2.516451in}{2.499570in}}%
\pgfpathlineto{\pgfqpoint{2.516451in}{2.496621in}}%
\pgfpathmoveto{\pgfqpoint{2.516451in}{2.493672in}}%
\pgfpathlineto{\pgfqpoint{2.516451in}{2.493672in}}%
\pgfpathlineto{\pgfqpoint{2.516451in}{2.496621in}}%
\pgfpathlineto{\pgfqpoint{2.520992in}{2.496621in}}%
\pgfpathlineto{\pgfqpoint{2.520992in}{2.493672in}}%
\pgfpathmoveto{\pgfqpoint{2.493746in}{2.505469in}}%
\pgfpathlineto{\pgfqpoint{2.493746in}{2.505469in}}%
\pgfpathlineto{\pgfqpoint{2.493746in}{2.508418in}}%
\pgfpathlineto{\pgfqpoint{2.498287in}{2.508418in}}%
\pgfpathlineto{\pgfqpoint{2.498287in}{2.505469in}}%
\pgfpathmoveto{\pgfqpoint{2.493746in}{2.508418in}}%
\pgfpathlineto{\pgfqpoint{2.493746in}{2.508418in}}%
\pgfpathlineto{\pgfqpoint{2.493746in}{2.511367in}}%
\pgfpathlineto{\pgfqpoint{2.498287in}{2.511367in}}%
\pgfpathlineto{\pgfqpoint{2.498287in}{2.508418in}}%
\pgfpathmoveto{\pgfqpoint{2.498287in}{2.505469in}}%
\pgfpathlineto{\pgfqpoint{2.498287in}{2.505469in}}%
\pgfpathlineto{\pgfqpoint{2.498287in}{2.508418in}}%
\pgfpathlineto{\pgfqpoint{2.502828in}{2.508418in}}%
\pgfpathlineto{\pgfqpoint{2.502828in}{2.505469in}}%
\pgfpathmoveto{\pgfqpoint{2.498287in}{2.508418in}}%
\pgfpathlineto{\pgfqpoint{2.498287in}{2.508418in}}%
\pgfpathlineto{\pgfqpoint{2.498287in}{2.511367in}}%
\pgfpathlineto{\pgfqpoint{2.502828in}{2.511367in}}%
\pgfpathlineto{\pgfqpoint{2.502828in}{2.508418in}}%
\pgfpathmoveto{\pgfqpoint{2.493746in}{2.511367in}}%
\pgfpathlineto{\pgfqpoint{2.493746in}{2.511367in}}%
\pgfpathlineto{\pgfqpoint{2.493746in}{2.514316in}}%
\pgfpathlineto{\pgfqpoint{2.498287in}{2.514316in}}%
\pgfpathlineto{\pgfqpoint{2.498287in}{2.511367in}}%
\pgfpathmoveto{\pgfqpoint{2.502828in}{2.505469in}}%
\pgfpathlineto{\pgfqpoint{2.502828in}{2.505469in}}%
\pgfpathlineto{\pgfqpoint{2.502828in}{2.508418in}}%
\pgfpathlineto{\pgfqpoint{2.507369in}{2.508418in}}%
\pgfpathlineto{\pgfqpoint{2.507369in}{2.505469in}}%
\pgfpathmoveto{\pgfqpoint{2.530074in}{2.481875in}}%
\pgfpathlineto{\pgfqpoint{2.530074in}{2.481875in}}%
\pgfpathlineto{\pgfqpoint{2.530074in}{2.484824in}}%
\pgfpathlineto{\pgfqpoint{2.534615in}{2.484824in}}%
\pgfpathlineto{\pgfqpoint{2.534615in}{2.481875in}}%
\pgfpathmoveto{\pgfqpoint{2.775290in}{2.257733in}}%
\pgfpathlineto{\pgfqpoint{2.775290in}{2.257733in}}%
\pgfpathlineto{\pgfqpoint{2.775290in}{2.260682in}}%
\pgfpathlineto{\pgfqpoint{2.779831in}{2.260682in}}%
\pgfpathlineto{\pgfqpoint{2.779831in}{2.257733in}}%
\pgfpathmoveto{\pgfqpoint{2.775290in}{2.260682in}}%
\pgfpathlineto{\pgfqpoint{2.775290in}{2.260682in}}%
\pgfpathlineto{\pgfqpoint{2.775290in}{2.263631in}}%
\pgfpathlineto{\pgfqpoint{2.779831in}{2.263631in}}%
\pgfpathlineto{\pgfqpoint{2.779831in}{2.260682in}}%
\pgfpathmoveto{\pgfqpoint{2.779831in}{2.257733in}}%
\pgfpathlineto{\pgfqpoint{2.779831in}{2.257733in}}%
\pgfpathlineto{\pgfqpoint{2.779831in}{2.260682in}}%
\pgfpathlineto{\pgfqpoint{2.784372in}{2.260682in}}%
\pgfpathlineto{\pgfqpoint{2.784372in}{2.257733in}}%
\pgfpathmoveto{\pgfqpoint{2.779831in}{2.260682in}}%
\pgfpathlineto{\pgfqpoint{2.779831in}{2.260682in}}%
\pgfpathlineto{\pgfqpoint{2.779831in}{2.263631in}}%
\pgfpathlineto{\pgfqpoint{2.784372in}{2.263631in}}%
\pgfpathlineto{\pgfqpoint{2.784372in}{2.260682in}}%
\pgfpathmoveto{\pgfqpoint{2.775290in}{2.263631in}}%
\pgfpathlineto{\pgfqpoint{2.775290in}{2.263631in}}%
\pgfpathlineto{\pgfqpoint{2.775290in}{2.266580in}}%
\pgfpathlineto{\pgfqpoint{2.779831in}{2.266580in}}%
\pgfpathlineto{\pgfqpoint{2.779831in}{2.263631in}}%
\pgfpathmoveto{\pgfqpoint{2.775290in}{2.266580in}}%
\pgfpathlineto{\pgfqpoint{2.775290in}{2.266580in}}%
\pgfpathlineto{\pgfqpoint{2.775290in}{2.269530in}}%
\pgfpathlineto{\pgfqpoint{2.779831in}{2.269530in}}%
\pgfpathlineto{\pgfqpoint{2.779831in}{2.266580in}}%
\pgfpathmoveto{\pgfqpoint{2.779831in}{2.263631in}}%
\pgfpathlineto{\pgfqpoint{2.779831in}{2.263631in}}%
\pgfpathlineto{\pgfqpoint{2.779831in}{2.266580in}}%
\pgfpathlineto{\pgfqpoint{2.784372in}{2.266580in}}%
\pgfpathlineto{\pgfqpoint{2.784372in}{2.263631in}}%
\pgfpathmoveto{\pgfqpoint{2.757125in}{2.275428in}}%
\pgfpathlineto{\pgfqpoint{2.757125in}{2.275428in}}%
\pgfpathlineto{\pgfqpoint{2.757125in}{2.278377in}}%
\pgfpathlineto{\pgfqpoint{2.761666in}{2.278377in}}%
\pgfpathlineto{\pgfqpoint{2.761666in}{2.275428in}}%
\pgfpathmoveto{\pgfqpoint{2.757125in}{2.278377in}}%
\pgfpathlineto{\pgfqpoint{2.757125in}{2.278377in}}%
\pgfpathlineto{\pgfqpoint{2.757125in}{2.281326in}}%
\pgfpathlineto{\pgfqpoint{2.761666in}{2.281326in}}%
\pgfpathlineto{\pgfqpoint{2.761666in}{2.278377in}}%
\pgfpathmoveto{\pgfqpoint{2.761666in}{2.275428in}}%
\pgfpathlineto{\pgfqpoint{2.761666in}{2.275428in}}%
\pgfpathlineto{\pgfqpoint{2.761666in}{2.278377in}}%
\pgfpathlineto{\pgfqpoint{2.766207in}{2.278377in}}%
\pgfpathlineto{\pgfqpoint{2.766207in}{2.275428in}}%
\pgfpathmoveto{\pgfqpoint{2.761666in}{2.278377in}}%
\pgfpathlineto{\pgfqpoint{2.761666in}{2.278377in}}%
\pgfpathlineto{\pgfqpoint{2.761666in}{2.281326in}}%
\pgfpathlineto{\pgfqpoint{2.766207in}{2.281326in}}%
\pgfpathlineto{\pgfqpoint{2.766207in}{2.278377in}}%
\pgfpathmoveto{\pgfqpoint{2.748043in}{2.281326in}}%
\pgfpathlineto{\pgfqpoint{2.748043in}{2.281326in}}%
\pgfpathlineto{\pgfqpoint{2.748043in}{2.284276in}}%
\pgfpathlineto{\pgfqpoint{2.752584in}{2.284276in}}%
\pgfpathlineto{\pgfqpoint{2.752584in}{2.281326in}}%
\pgfpathmoveto{\pgfqpoint{2.748043in}{2.284276in}}%
\pgfpathlineto{\pgfqpoint{2.748043in}{2.284276in}}%
\pgfpathlineto{\pgfqpoint{2.748043in}{2.287225in}}%
\pgfpathlineto{\pgfqpoint{2.752584in}{2.287225in}}%
\pgfpathlineto{\pgfqpoint{2.752584in}{2.284276in}}%
\pgfpathmoveto{\pgfqpoint{2.752584in}{2.281326in}}%
\pgfpathlineto{\pgfqpoint{2.752584in}{2.281326in}}%
\pgfpathlineto{\pgfqpoint{2.752584in}{2.284276in}}%
\pgfpathlineto{\pgfqpoint{2.757125in}{2.284276in}}%
\pgfpathlineto{\pgfqpoint{2.757125in}{2.281326in}}%
\pgfpathmoveto{\pgfqpoint{2.752584in}{2.284276in}}%
\pgfpathlineto{\pgfqpoint{2.752584in}{2.284276in}}%
\pgfpathlineto{\pgfqpoint{2.752584in}{2.287225in}}%
\pgfpathlineto{\pgfqpoint{2.757125in}{2.287225in}}%
\pgfpathlineto{\pgfqpoint{2.757125in}{2.284276in}}%
\pgfpathmoveto{\pgfqpoint{2.748043in}{2.287225in}}%
\pgfpathlineto{\pgfqpoint{2.748043in}{2.287225in}}%
\pgfpathlineto{\pgfqpoint{2.748043in}{2.290174in}}%
\pgfpathlineto{\pgfqpoint{2.752584in}{2.290174in}}%
\pgfpathlineto{\pgfqpoint{2.752584in}{2.287225in}}%
\pgfpathmoveto{\pgfqpoint{2.748043in}{2.290174in}}%
\pgfpathlineto{\pgfqpoint{2.748043in}{2.290174in}}%
\pgfpathlineto{\pgfqpoint{2.748043in}{2.293123in}}%
\pgfpathlineto{\pgfqpoint{2.752584in}{2.293123in}}%
\pgfpathlineto{\pgfqpoint{2.752584in}{2.290174in}}%
\pgfpathmoveto{\pgfqpoint{2.752584in}{2.287225in}}%
\pgfpathlineto{\pgfqpoint{2.752584in}{2.287225in}}%
\pgfpathlineto{\pgfqpoint{2.752584in}{2.290174in}}%
\pgfpathlineto{\pgfqpoint{2.757125in}{2.290174in}}%
\pgfpathlineto{\pgfqpoint{2.757125in}{2.287225in}}%
\pgfpathmoveto{\pgfqpoint{2.757125in}{2.281326in}}%
\pgfpathlineto{\pgfqpoint{2.757125in}{2.281326in}}%
\pgfpathlineto{\pgfqpoint{2.757125in}{2.284276in}}%
\pgfpathlineto{\pgfqpoint{2.761666in}{2.284276in}}%
\pgfpathlineto{\pgfqpoint{2.761666in}{2.281326in}}%
\pgfpathmoveto{\pgfqpoint{2.757125in}{2.284276in}}%
\pgfpathlineto{\pgfqpoint{2.757125in}{2.284276in}}%
\pgfpathlineto{\pgfqpoint{2.757125in}{2.287225in}}%
\pgfpathlineto{\pgfqpoint{2.761666in}{2.287225in}}%
\pgfpathlineto{\pgfqpoint{2.761666in}{2.284276in}}%
\pgfpathmoveto{\pgfqpoint{2.761666in}{2.281326in}}%
\pgfpathlineto{\pgfqpoint{2.761666in}{2.281326in}}%
\pgfpathlineto{\pgfqpoint{2.761666in}{2.284276in}}%
\pgfpathlineto{\pgfqpoint{2.766207in}{2.284276in}}%
\pgfpathlineto{\pgfqpoint{2.766207in}{2.281326in}}%
\pgfpathmoveto{\pgfqpoint{2.766207in}{2.269530in}}%
\pgfpathlineto{\pgfqpoint{2.766207in}{2.269530in}}%
\pgfpathlineto{\pgfqpoint{2.766207in}{2.272479in}}%
\pgfpathlineto{\pgfqpoint{2.770748in}{2.272479in}}%
\pgfpathlineto{\pgfqpoint{2.770748in}{2.269530in}}%
\pgfpathmoveto{\pgfqpoint{2.766207in}{2.272479in}}%
\pgfpathlineto{\pgfqpoint{2.766207in}{2.272479in}}%
\pgfpathlineto{\pgfqpoint{2.766207in}{2.275428in}}%
\pgfpathlineto{\pgfqpoint{2.770748in}{2.275428in}}%
\pgfpathlineto{\pgfqpoint{2.770748in}{2.272479in}}%
\pgfpathmoveto{\pgfqpoint{2.770748in}{2.269530in}}%
\pgfpathlineto{\pgfqpoint{2.770748in}{2.269530in}}%
\pgfpathlineto{\pgfqpoint{2.770748in}{2.272479in}}%
\pgfpathlineto{\pgfqpoint{2.775290in}{2.272479in}}%
\pgfpathlineto{\pgfqpoint{2.775290in}{2.269530in}}%
\pgfpathmoveto{\pgfqpoint{2.770748in}{2.272479in}}%
\pgfpathlineto{\pgfqpoint{2.770748in}{2.272479in}}%
\pgfpathlineto{\pgfqpoint{2.770748in}{2.275428in}}%
\pgfpathlineto{\pgfqpoint{2.775290in}{2.275428in}}%
\pgfpathlineto{\pgfqpoint{2.775290in}{2.272479in}}%
\pgfpathmoveto{\pgfqpoint{2.766207in}{2.275428in}}%
\pgfpathlineto{\pgfqpoint{2.766207in}{2.275428in}}%
\pgfpathlineto{\pgfqpoint{2.766207in}{2.278377in}}%
\pgfpathlineto{\pgfqpoint{2.770748in}{2.278377in}}%
\pgfpathlineto{\pgfqpoint{2.770748in}{2.275428in}}%
\pgfpathmoveto{\pgfqpoint{2.775290in}{2.269530in}}%
\pgfpathlineto{\pgfqpoint{2.775290in}{2.269530in}}%
\pgfpathlineto{\pgfqpoint{2.775290in}{2.272479in}}%
\pgfpathlineto{\pgfqpoint{2.779831in}{2.272479in}}%
\pgfpathlineto{\pgfqpoint{2.779831in}{2.269530in}}%
\pgfpathmoveto{\pgfqpoint{2.702633in}{2.322615in}}%
\pgfpathlineto{\pgfqpoint{2.702633in}{2.322615in}}%
\pgfpathlineto{\pgfqpoint{2.702633in}{2.325565in}}%
\pgfpathlineto{\pgfqpoint{2.707174in}{2.325565in}}%
\pgfpathlineto{\pgfqpoint{2.707174in}{2.322615in}}%
\pgfpathmoveto{\pgfqpoint{2.702633in}{2.325565in}}%
\pgfpathlineto{\pgfqpoint{2.702633in}{2.325565in}}%
\pgfpathlineto{\pgfqpoint{2.702633in}{2.328514in}}%
\pgfpathlineto{\pgfqpoint{2.707174in}{2.328514in}}%
\pgfpathlineto{\pgfqpoint{2.707174in}{2.325565in}}%
\pgfpathmoveto{\pgfqpoint{2.707174in}{2.322615in}}%
\pgfpathlineto{\pgfqpoint{2.707174in}{2.322615in}}%
\pgfpathlineto{\pgfqpoint{2.707174in}{2.325565in}}%
\pgfpathlineto{\pgfqpoint{2.711715in}{2.325565in}}%
\pgfpathlineto{\pgfqpoint{2.711715in}{2.322615in}}%
\pgfpathmoveto{\pgfqpoint{2.707174in}{2.325565in}}%
\pgfpathlineto{\pgfqpoint{2.707174in}{2.325565in}}%
\pgfpathlineto{\pgfqpoint{2.707174in}{2.328514in}}%
\pgfpathlineto{\pgfqpoint{2.711715in}{2.328514in}}%
\pgfpathlineto{\pgfqpoint{2.711715in}{2.325565in}}%
\pgfpathmoveto{\pgfqpoint{2.693551in}{2.328514in}}%
\pgfpathlineto{\pgfqpoint{2.693551in}{2.328514in}}%
\pgfpathlineto{\pgfqpoint{2.693551in}{2.331463in}}%
\pgfpathlineto{\pgfqpoint{2.698092in}{2.331463in}}%
\pgfpathlineto{\pgfqpoint{2.698092in}{2.328514in}}%
\pgfpathmoveto{\pgfqpoint{2.693551in}{2.331463in}}%
\pgfpathlineto{\pgfqpoint{2.693551in}{2.331463in}}%
\pgfpathlineto{\pgfqpoint{2.693551in}{2.334412in}}%
\pgfpathlineto{\pgfqpoint{2.698092in}{2.334412in}}%
\pgfpathlineto{\pgfqpoint{2.698092in}{2.331463in}}%
\pgfpathmoveto{\pgfqpoint{2.698092in}{2.328514in}}%
\pgfpathlineto{\pgfqpoint{2.698092in}{2.328514in}}%
\pgfpathlineto{\pgfqpoint{2.698092in}{2.331463in}}%
\pgfpathlineto{\pgfqpoint{2.702633in}{2.331463in}}%
\pgfpathlineto{\pgfqpoint{2.702633in}{2.328514in}}%
\pgfpathmoveto{\pgfqpoint{2.698092in}{2.331463in}}%
\pgfpathlineto{\pgfqpoint{2.698092in}{2.331463in}}%
\pgfpathlineto{\pgfqpoint{2.698092in}{2.334412in}}%
\pgfpathlineto{\pgfqpoint{2.702633in}{2.334412in}}%
\pgfpathlineto{\pgfqpoint{2.702633in}{2.331463in}}%
\pgfpathmoveto{\pgfqpoint{2.693551in}{2.334412in}}%
\pgfpathlineto{\pgfqpoint{2.693551in}{2.334412in}}%
\pgfpathlineto{\pgfqpoint{2.693551in}{2.337361in}}%
\pgfpathlineto{\pgfqpoint{2.698092in}{2.337361in}}%
\pgfpathlineto{\pgfqpoint{2.698092in}{2.334412in}}%
\pgfpathmoveto{\pgfqpoint{2.693551in}{2.337361in}}%
\pgfpathlineto{\pgfqpoint{2.693551in}{2.337361in}}%
\pgfpathlineto{\pgfqpoint{2.693551in}{2.340311in}}%
\pgfpathlineto{\pgfqpoint{2.698092in}{2.340311in}}%
\pgfpathlineto{\pgfqpoint{2.698092in}{2.337361in}}%
\pgfpathmoveto{\pgfqpoint{2.698092in}{2.334412in}}%
\pgfpathlineto{\pgfqpoint{2.698092in}{2.334412in}}%
\pgfpathlineto{\pgfqpoint{2.698092in}{2.337361in}}%
\pgfpathlineto{\pgfqpoint{2.702633in}{2.337361in}}%
\pgfpathlineto{\pgfqpoint{2.702633in}{2.334412in}}%
\pgfpathmoveto{\pgfqpoint{2.702633in}{2.328514in}}%
\pgfpathlineto{\pgfqpoint{2.702633in}{2.328514in}}%
\pgfpathlineto{\pgfqpoint{2.702633in}{2.331463in}}%
\pgfpathlineto{\pgfqpoint{2.707174in}{2.331463in}}%
\pgfpathlineto{\pgfqpoint{2.707174in}{2.328514in}}%
\pgfpathmoveto{\pgfqpoint{2.702633in}{2.331463in}}%
\pgfpathlineto{\pgfqpoint{2.702633in}{2.331463in}}%
\pgfpathlineto{\pgfqpoint{2.702633in}{2.334412in}}%
\pgfpathlineto{\pgfqpoint{2.707174in}{2.334412in}}%
\pgfpathlineto{\pgfqpoint{2.707174in}{2.331463in}}%
\pgfpathmoveto{\pgfqpoint{2.707174in}{2.328514in}}%
\pgfpathlineto{\pgfqpoint{2.707174in}{2.328514in}}%
\pgfpathlineto{\pgfqpoint{2.707174in}{2.331463in}}%
\pgfpathlineto{\pgfqpoint{2.711715in}{2.331463in}}%
\pgfpathlineto{\pgfqpoint{2.711715in}{2.328514in}}%
\pgfpathmoveto{\pgfqpoint{2.666305in}{2.352108in}}%
\pgfpathlineto{\pgfqpoint{2.666305in}{2.352108in}}%
\pgfpathlineto{\pgfqpoint{2.666305in}{2.355057in}}%
\pgfpathlineto{\pgfqpoint{2.670846in}{2.355057in}}%
\pgfpathlineto{\pgfqpoint{2.670846in}{2.352108in}}%
\pgfpathmoveto{\pgfqpoint{2.666305in}{2.355057in}}%
\pgfpathlineto{\pgfqpoint{2.666305in}{2.355057in}}%
\pgfpathlineto{\pgfqpoint{2.666305in}{2.358006in}}%
\pgfpathlineto{\pgfqpoint{2.670846in}{2.358006in}}%
\pgfpathlineto{\pgfqpoint{2.670846in}{2.355057in}}%
\pgfpathmoveto{\pgfqpoint{2.670846in}{2.352108in}}%
\pgfpathlineto{\pgfqpoint{2.670846in}{2.352108in}}%
\pgfpathlineto{\pgfqpoint{2.670846in}{2.355057in}}%
\pgfpathlineto{\pgfqpoint{2.675387in}{2.355057in}}%
\pgfpathlineto{\pgfqpoint{2.675387in}{2.352108in}}%
\pgfpathmoveto{\pgfqpoint{2.670846in}{2.355057in}}%
\pgfpathlineto{\pgfqpoint{2.670846in}{2.355057in}}%
\pgfpathlineto{\pgfqpoint{2.670846in}{2.358006in}}%
\pgfpathlineto{\pgfqpoint{2.675387in}{2.358006in}}%
\pgfpathlineto{\pgfqpoint{2.675387in}{2.355057in}}%
\pgfpathmoveto{\pgfqpoint{2.666305in}{2.358006in}}%
\pgfpathlineto{\pgfqpoint{2.666305in}{2.358006in}}%
\pgfpathlineto{\pgfqpoint{2.666305in}{2.360955in}}%
\pgfpathlineto{\pgfqpoint{2.670846in}{2.360955in}}%
\pgfpathlineto{\pgfqpoint{2.670846in}{2.358006in}}%
\pgfpathmoveto{\pgfqpoint{2.666305in}{2.360955in}}%
\pgfpathlineto{\pgfqpoint{2.666305in}{2.360955in}}%
\pgfpathlineto{\pgfqpoint{2.666305in}{2.363905in}}%
\pgfpathlineto{\pgfqpoint{2.670846in}{2.363905in}}%
\pgfpathlineto{\pgfqpoint{2.670846in}{2.360955in}}%
\pgfpathmoveto{\pgfqpoint{2.670846in}{2.358006in}}%
\pgfpathlineto{\pgfqpoint{2.670846in}{2.358006in}}%
\pgfpathlineto{\pgfqpoint{2.670846in}{2.360955in}}%
\pgfpathlineto{\pgfqpoint{2.675387in}{2.360955in}}%
\pgfpathlineto{\pgfqpoint{2.675387in}{2.358006in}}%
\pgfpathmoveto{\pgfqpoint{2.648140in}{2.369803in}}%
\pgfpathlineto{\pgfqpoint{2.648140in}{2.369803in}}%
\pgfpathlineto{\pgfqpoint{2.648140in}{2.372752in}}%
\pgfpathlineto{\pgfqpoint{2.652681in}{2.372752in}}%
\pgfpathlineto{\pgfqpoint{2.652681in}{2.369803in}}%
\pgfpathmoveto{\pgfqpoint{2.648140in}{2.372752in}}%
\pgfpathlineto{\pgfqpoint{2.648140in}{2.372752in}}%
\pgfpathlineto{\pgfqpoint{2.648140in}{2.375701in}}%
\pgfpathlineto{\pgfqpoint{2.652681in}{2.375701in}}%
\pgfpathlineto{\pgfqpoint{2.652681in}{2.372752in}}%
\pgfpathmoveto{\pgfqpoint{2.652681in}{2.369803in}}%
\pgfpathlineto{\pgfqpoint{2.652681in}{2.369803in}}%
\pgfpathlineto{\pgfqpoint{2.652681in}{2.372752in}}%
\pgfpathlineto{\pgfqpoint{2.657223in}{2.372752in}}%
\pgfpathlineto{\pgfqpoint{2.657223in}{2.369803in}}%
\pgfpathmoveto{\pgfqpoint{2.652681in}{2.372752in}}%
\pgfpathlineto{\pgfqpoint{2.652681in}{2.372752in}}%
\pgfpathlineto{\pgfqpoint{2.652681in}{2.375701in}}%
\pgfpathlineto{\pgfqpoint{2.657223in}{2.375701in}}%
\pgfpathlineto{\pgfqpoint{2.657223in}{2.372752in}}%
\pgfpathmoveto{\pgfqpoint{2.639058in}{2.375701in}}%
\pgfpathlineto{\pgfqpoint{2.639058in}{2.375701in}}%
\pgfpathlineto{\pgfqpoint{2.639058in}{2.378651in}}%
\pgfpathlineto{\pgfqpoint{2.643599in}{2.378651in}}%
\pgfpathlineto{\pgfqpoint{2.643599in}{2.375701in}}%
\pgfpathmoveto{\pgfqpoint{2.639058in}{2.378651in}}%
\pgfpathlineto{\pgfqpoint{2.639058in}{2.378651in}}%
\pgfpathlineto{\pgfqpoint{2.639058in}{2.381600in}}%
\pgfpathlineto{\pgfqpoint{2.643599in}{2.381600in}}%
\pgfpathlineto{\pgfqpoint{2.643599in}{2.378651in}}%
\pgfpathmoveto{\pgfqpoint{2.643599in}{2.375701in}}%
\pgfpathlineto{\pgfqpoint{2.643599in}{2.375701in}}%
\pgfpathlineto{\pgfqpoint{2.643599in}{2.378651in}}%
\pgfpathlineto{\pgfqpoint{2.648140in}{2.378651in}}%
\pgfpathlineto{\pgfqpoint{2.648140in}{2.375701in}}%
\pgfpathmoveto{\pgfqpoint{2.643599in}{2.378651in}}%
\pgfpathlineto{\pgfqpoint{2.643599in}{2.378651in}}%
\pgfpathlineto{\pgfqpoint{2.643599in}{2.381600in}}%
\pgfpathlineto{\pgfqpoint{2.648140in}{2.381600in}}%
\pgfpathlineto{\pgfqpoint{2.648140in}{2.378651in}}%
\pgfpathmoveto{\pgfqpoint{2.639058in}{2.381600in}}%
\pgfpathlineto{\pgfqpoint{2.639058in}{2.381600in}}%
\pgfpathlineto{\pgfqpoint{2.639058in}{2.384549in}}%
\pgfpathlineto{\pgfqpoint{2.643599in}{2.384549in}}%
\pgfpathlineto{\pgfqpoint{2.643599in}{2.381600in}}%
\pgfpathmoveto{\pgfqpoint{2.639058in}{2.384549in}}%
\pgfpathlineto{\pgfqpoint{2.639058in}{2.384549in}}%
\pgfpathlineto{\pgfqpoint{2.639058in}{2.387498in}}%
\pgfpathlineto{\pgfqpoint{2.643599in}{2.387498in}}%
\pgfpathlineto{\pgfqpoint{2.643599in}{2.384549in}}%
\pgfpathmoveto{\pgfqpoint{2.643599in}{2.381600in}}%
\pgfpathlineto{\pgfqpoint{2.643599in}{2.381600in}}%
\pgfpathlineto{\pgfqpoint{2.643599in}{2.384549in}}%
\pgfpathlineto{\pgfqpoint{2.648140in}{2.384549in}}%
\pgfpathlineto{\pgfqpoint{2.648140in}{2.381600in}}%
\pgfpathmoveto{\pgfqpoint{2.648140in}{2.375701in}}%
\pgfpathlineto{\pgfqpoint{2.648140in}{2.375701in}}%
\pgfpathlineto{\pgfqpoint{2.648140in}{2.378651in}}%
\pgfpathlineto{\pgfqpoint{2.652681in}{2.378651in}}%
\pgfpathlineto{\pgfqpoint{2.652681in}{2.375701in}}%
\pgfpathmoveto{\pgfqpoint{2.648140in}{2.378651in}}%
\pgfpathlineto{\pgfqpoint{2.648140in}{2.378651in}}%
\pgfpathlineto{\pgfqpoint{2.648140in}{2.381600in}}%
\pgfpathlineto{\pgfqpoint{2.652681in}{2.381600in}}%
\pgfpathlineto{\pgfqpoint{2.652681in}{2.378651in}}%
\pgfpathmoveto{\pgfqpoint{2.652681in}{2.375701in}}%
\pgfpathlineto{\pgfqpoint{2.652681in}{2.375701in}}%
\pgfpathlineto{\pgfqpoint{2.652681in}{2.378651in}}%
\pgfpathlineto{\pgfqpoint{2.657223in}{2.378651in}}%
\pgfpathlineto{\pgfqpoint{2.657223in}{2.375701in}}%
\pgfpathmoveto{\pgfqpoint{2.657223in}{2.363905in}}%
\pgfpathlineto{\pgfqpoint{2.657223in}{2.363905in}}%
\pgfpathlineto{\pgfqpoint{2.657223in}{2.366854in}}%
\pgfpathlineto{\pgfqpoint{2.661764in}{2.366854in}}%
\pgfpathlineto{\pgfqpoint{2.661764in}{2.363905in}}%
\pgfpathmoveto{\pgfqpoint{2.657223in}{2.366854in}}%
\pgfpathlineto{\pgfqpoint{2.657223in}{2.366854in}}%
\pgfpathlineto{\pgfqpoint{2.657223in}{2.369803in}}%
\pgfpathlineto{\pgfqpoint{2.661764in}{2.369803in}}%
\pgfpathlineto{\pgfqpoint{2.661764in}{2.366854in}}%
\pgfpathmoveto{\pgfqpoint{2.661764in}{2.363905in}}%
\pgfpathlineto{\pgfqpoint{2.661764in}{2.363905in}}%
\pgfpathlineto{\pgfqpoint{2.661764in}{2.366854in}}%
\pgfpathlineto{\pgfqpoint{2.666305in}{2.366854in}}%
\pgfpathlineto{\pgfqpoint{2.666305in}{2.363905in}}%
\pgfpathmoveto{\pgfqpoint{2.661764in}{2.366854in}}%
\pgfpathlineto{\pgfqpoint{2.661764in}{2.366854in}}%
\pgfpathlineto{\pgfqpoint{2.661764in}{2.369803in}}%
\pgfpathlineto{\pgfqpoint{2.666305in}{2.369803in}}%
\pgfpathlineto{\pgfqpoint{2.666305in}{2.366854in}}%
\pgfpathmoveto{\pgfqpoint{2.657223in}{2.369803in}}%
\pgfpathlineto{\pgfqpoint{2.657223in}{2.369803in}}%
\pgfpathlineto{\pgfqpoint{2.657223in}{2.372752in}}%
\pgfpathlineto{\pgfqpoint{2.661764in}{2.372752in}}%
\pgfpathlineto{\pgfqpoint{2.661764in}{2.369803in}}%
\pgfpathmoveto{\pgfqpoint{2.666305in}{2.363905in}}%
\pgfpathlineto{\pgfqpoint{2.666305in}{2.363905in}}%
\pgfpathlineto{\pgfqpoint{2.666305in}{2.366854in}}%
\pgfpathlineto{\pgfqpoint{2.670846in}{2.366854in}}%
\pgfpathlineto{\pgfqpoint{2.670846in}{2.363905in}}%
\pgfpathmoveto{\pgfqpoint{2.675387in}{2.346209in}}%
\pgfpathlineto{\pgfqpoint{2.675387in}{2.346209in}}%
\pgfpathlineto{\pgfqpoint{2.675387in}{2.349158in}}%
\pgfpathlineto{\pgfqpoint{2.679928in}{2.349158in}}%
\pgfpathlineto{\pgfqpoint{2.679928in}{2.346209in}}%
\pgfpathmoveto{\pgfqpoint{2.675387in}{2.349158in}}%
\pgfpathlineto{\pgfqpoint{2.675387in}{2.349158in}}%
\pgfpathlineto{\pgfqpoint{2.675387in}{2.352108in}}%
\pgfpathlineto{\pgfqpoint{2.679928in}{2.352108in}}%
\pgfpathlineto{\pgfqpoint{2.679928in}{2.349158in}}%
\pgfpathmoveto{\pgfqpoint{2.679928in}{2.346209in}}%
\pgfpathlineto{\pgfqpoint{2.679928in}{2.346209in}}%
\pgfpathlineto{\pgfqpoint{2.679928in}{2.349158in}}%
\pgfpathlineto{\pgfqpoint{2.684469in}{2.349158in}}%
\pgfpathlineto{\pgfqpoint{2.684469in}{2.346209in}}%
\pgfpathmoveto{\pgfqpoint{2.679928in}{2.349158in}}%
\pgfpathlineto{\pgfqpoint{2.679928in}{2.349158in}}%
\pgfpathlineto{\pgfqpoint{2.679928in}{2.352108in}}%
\pgfpathlineto{\pgfqpoint{2.684469in}{2.352108in}}%
\pgfpathlineto{\pgfqpoint{2.684469in}{2.349158in}}%
\pgfpathmoveto{\pgfqpoint{2.684469in}{2.340311in}}%
\pgfpathlineto{\pgfqpoint{2.684469in}{2.340311in}}%
\pgfpathlineto{\pgfqpoint{2.684469in}{2.343260in}}%
\pgfpathlineto{\pgfqpoint{2.689010in}{2.343260in}}%
\pgfpathlineto{\pgfqpoint{2.689010in}{2.340311in}}%
\pgfpathmoveto{\pgfqpoint{2.684469in}{2.343260in}}%
\pgfpathlineto{\pgfqpoint{2.684469in}{2.343260in}}%
\pgfpathlineto{\pgfqpoint{2.684469in}{2.346209in}}%
\pgfpathlineto{\pgfqpoint{2.689010in}{2.346209in}}%
\pgfpathlineto{\pgfqpoint{2.689010in}{2.343260in}}%
\pgfpathmoveto{\pgfqpoint{2.689010in}{2.340311in}}%
\pgfpathlineto{\pgfqpoint{2.689010in}{2.340311in}}%
\pgfpathlineto{\pgfqpoint{2.689010in}{2.343260in}}%
\pgfpathlineto{\pgfqpoint{2.693551in}{2.343260in}}%
\pgfpathlineto{\pgfqpoint{2.693551in}{2.340311in}}%
\pgfpathmoveto{\pgfqpoint{2.689010in}{2.343260in}}%
\pgfpathlineto{\pgfqpoint{2.689010in}{2.343260in}}%
\pgfpathlineto{\pgfqpoint{2.689010in}{2.346209in}}%
\pgfpathlineto{\pgfqpoint{2.693551in}{2.346209in}}%
\pgfpathlineto{\pgfqpoint{2.693551in}{2.343260in}}%
\pgfpathmoveto{\pgfqpoint{2.684469in}{2.346209in}}%
\pgfpathlineto{\pgfqpoint{2.684469in}{2.346209in}}%
\pgfpathlineto{\pgfqpoint{2.684469in}{2.349158in}}%
\pgfpathlineto{\pgfqpoint{2.689010in}{2.349158in}}%
\pgfpathlineto{\pgfqpoint{2.689010in}{2.346209in}}%
\pgfpathmoveto{\pgfqpoint{2.675387in}{2.352108in}}%
\pgfpathlineto{\pgfqpoint{2.675387in}{2.352108in}}%
\pgfpathlineto{\pgfqpoint{2.675387in}{2.355057in}}%
\pgfpathlineto{\pgfqpoint{2.679928in}{2.355057in}}%
\pgfpathlineto{\pgfqpoint{2.679928in}{2.352108in}}%
\pgfpathmoveto{\pgfqpoint{2.675387in}{2.355057in}}%
\pgfpathlineto{\pgfqpoint{2.675387in}{2.355057in}}%
\pgfpathlineto{\pgfqpoint{2.675387in}{2.358006in}}%
\pgfpathlineto{\pgfqpoint{2.679928in}{2.358006in}}%
\pgfpathlineto{\pgfqpoint{2.679928in}{2.355057in}}%
\pgfpathmoveto{\pgfqpoint{2.679928in}{2.352108in}}%
\pgfpathlineto{\pgfqpoint{2.679928in}{2.352108in}}%
\pgfpathlineto{\pgfqpoint{2.679928in}{2.355057in}}%
\pgfpathlineto{\pgfqpoint{2.684469in}{2.355057in}}%
\pgfpathlineto{\pgfqpoint{2.684469in}{2.352108in}}%
\pgfpathmoveto{\pgfqpoint{2.693551in}{2.340311in}}%
\pgfpathlineto{\pgfqpoint{2.693551in}{2.340311in}}%
\pgfpathlineto{\pgfqpoint{2.693551in}{2.343260in}}%
\pgfpathlineto{\pgfqpoint{2.698092in}{2.343260in}}%
\pgfpathlineto{\pgfqpoint{2.698092in}{2.340311in}}%
\pgfpathmoveto{\pgfqpoint{2.720797in}{2.304920in}}%
\pgfpathlineto{\pgfqpoint{2.720797in}{2.304920in}}%
\pgfpathlineto{\pgfqpoint{2.720797in}{2.307869in}}%
\pgfpathlineto{\pgfqpoint{2.725338in}{2.307869in}}%
\pgfpathlineto{\pgfqpoint{2.725338in}{2.304920in}}%
\pgfpathmoveto{\pgfqpoint{2.720797in}{2.307869in}}%
\pgfpathlineto{\pgfqpoint{2.720797in}{2.307869in}}%
\pgfpathlineto{\pgfqpoint{2.720797in}{2.310818in}}%
\pgfpathlineto{\pgfqpoint{2.725338in}{2.310818in}}%
\pgfpathlineto{\pgfqpoint{2.725338in}{2.307869in}}%
\pgfpathmoveto{\pgfqpoint{2.725338in}{2.304920in}}%
\pgfpathlineto{\pgfqpoint{2.725338in}{2.304920in}}%
\pgfpathlineto{\pgfqpoint{2.725338in}{2.307869in}}%
\pgfpathlineto{\pgfqpoint{2.729879in}{2.307869in}}%
\pgfpathlineto{\pgfqpoint{2.729879in}{2.304920in}}%
\pgfpathmoveto{\pgfqpoint{2.725338in}{2.307869in}}%
\pgfpathlineto{\pgfqpoint{2.725338in}{2.307869in}}%
\pgfpathlineto{\pgfqpoint{2.725338in}{2.310818in}}%
\pgfpathlineto{\pgfqpoint{2.729879in}{2.310818in}}%
\pgfpathlineto{\pgfqpoint{2.729879in}{2.307869in}}%
\pgfpathmoveto{\pgfqpoint{2.720797in}{2.310818in}}%
\pgfpathlineto{\pgfqpoint{2.720797in}{2.310818in}}%
\pgfpathlineto{\pgfqpoint{2.720797in}{2.313768in}}%
\pgfpathlineto{\pgfqpoint{2.725338in}{2.313768in}}%
\pgfpathlineto{\pgfqpoint{2.725338in}{2.310818in}}%
\pgfpathmoveto{\pgfqpoint{2.720797in}{2.313768in}}%
\pgfpathlineto{\pgfqpoint{2.720797in}{2.313768in}}%
\pgfpathlineto{\pgfqpoint{2.720797in}{2.316717in}}%
\pgfpathlineto{\pgfqpoint{2.725338in}{2.316717in}}%
\pgfpathlineto{\pgfqpoint{2.725338in}{2.313768in}}%
\pgfpathmoveto{\pgfqpoint{2.725338in}{2.310818in}}%
\pgfpathlineto{\pgfqpoint{2.725338in}{2.310818in}}%
\pgfpathlineto{\pgfqpoint{2.725338in}{2.313768in}}%
\pgfpathlineto{\pgfqpoint{2.729879in}{2.313768in}}%
\pgfpathlineto{\pgfqpoint{2.729879in}{2.310818in}}%
\pgfpathmoveto{\pgfqpoint{2.729879in}{2.299021in}}%
\pgfpathlineto{\pgfqpoint{2.729879in}{2.299021in}}%
\pgfpathlineto{\pgfqpoint{2.729879in}{2.301971in}}%
\pgfpathlineto{\pgfqpoint{2.734420in}{2.301971in}}%
\pgfpathlineto{\pgfqpoint{2.734420in}{2.299021in}}%
\pgfpathmoveto{\pgfqpoint{2.729879in}{2.301971in}}%
\pgfpathlineto{\pgfqpoint{2.729879in}{2.301971in}}%
\pgfpathlineto{\pgfqpoint{2.729879in}{2.304920in}}%
\pgfpathlineto{\pgfqpoint{2.734420in}{2.304920in}}%
\pgfpathlineto{\pgfqpoint{2.734420in}{2.301971in}}%
\pgfpathmoveto{\pgfqpoint{2.734420in}{2.299021in}}%
\pgfpathlineto{\pgfqpoint{2.734420in}{2.299021in}}%
\pgfpathlineto{\pgfqpoint{2.734420in}{2.301971in}}%
\pgfpathlineto{\pgfqpoint{2.738961in}{2.301971in}}%
\pgfpathlineto{\pgfqpoint{2.738961in}{2.299021in}}%
\pgfpathmoveto{\pgfqpoint{2.734420in}{2.301971in}}%
\pgfpathlineto{\pgfqpoint{2.734420in}{2.301971in}}%
\pgfpathlineto{\pgfqpoint{2.734420in}{2.304920in}}%
\pgfpathlineto{\pgfqpoint{2.738961in}{2.304920in}}%
\pgfpathlineto{\pgfqpoint{2.738961in}{2.301971in}}%
\pgfpathmoveto{\pgfqpoint{2.738961in}{2.293123in}}%
\pgfpathlineto{\pgfqpoint{2.738961in}{2.293123in}}%
\pgfpathlineto{\pgfqpoint{2.738961in}{2.296072in}}%
\pgfpathlineto{\pgfqpoint{2.743502in}{2.296072in}}%
\pgfpathlineto{\pgfqpoint{2.743502in}{2.293123in}}%
\pgfpathmoveto{\pgfqpoint{2.738961in}{2.296072in}}%
\pgfpathlineto{\pgfqpoint{2.738961in}{2.296072in}}%
\pgfpathlineto{\pgfqpoint{2.738961in}{2.299021in}}%
\pgfpathlineto{\pgfqpoint{2.743502in}{2.299021in}}%
\pgfpathlineto{\pgfqpoint{2.743502in}{2.296072in}}%
\pgfpathmoveto{\pgfqpoint{2.743502in}{2.293123in}}%
\pgfpathlineto{\pgfqpoint{2.743502in}{2.293123in}}%
\pgfpathlineto{\pgfqpoint{2.743502in}{2.296072in}}%
\pgfpathlineto{\pgfqpoint{2.748043in}{2.296072in}}%
\pgfpathlineto{\pgfqpoint{2.748043in}{2.293123in}}%
\pgfpathmoveto{\pgfqpoint{2.743502in}{2.296072in}}%
\pgfpathlineto{\pgfqpoint{2.743502in}{2.296072in}}%
\pgfpathlineto{\pgfqpoint{2.743502in}{2.299021in}}%
\pgfpathlineto{\pgfqpoint{2.748043in}{2.299021in}}%
\pgfpathlineto{\pgfqpoint{2.748043in}{2.296072in}}%
\pgfpathmoveto{\pgfqpoint{2.738961in}{2.299021in}}%
\pgfpathlineto{\pgfqpoint{2.738961in}{2.299021in}}%
\pgfpathlineto{\pgfqpoint{2.738961in}{2.301971in}}%
\pgfpathlineto{\pgfqpoint{2.743502in}{2.301971in}}%
\pgfpathlineto{\pgfqpoint{2.743502in}{2.299021in}}%
\pgfpathmoveto{\pgfqpoint{2.729879in}{2.304920in}}%
\pgfpathlineto{\pgfqpoint{2.729879in}{2.304920in}}%
\pgfpathlineto{\pgfqpoint{2.729879in}{2.307869in}}%
\pgfpathlineto{\pgfqpoint{2.734420in}{2.307869in}}%
\pgfpathlineto{\pgfqpoint{2.734420in}{2.304920in}}%
\pgfpathmoveto{\pgfqpoint{2.729879in}{2.307869in}}%
\pgfpathlineto{\pgfqpoint{2.729879in}{2.307869in}}%
\pgfpathlineto{\pgfqpoint{2.729879in}{2.310818in}}%
\pgfpathlineto{\pgfqpoint{2.734420in}{2.310818in}}%
\pgfpathlineto{\pgfqpoint{2.734420in}{2.307869in}}%
\pgfpathmoveto{\pgfqpoint{2.734420in}{2.304920in}}%
\pgfpathlineto{\pgfqpoint{2.734420in}{2.304920in}}%
\pgfpathlineto{\pgfqpoint{2.734420in}{2.307869in}}%
\pgfpathlineto{\pgfqpoint{2.738961in}{2.307869in}}%
\pgfpathlineto{\pgfqpoint{2.738961in}{2.304920in}}%
\pgfpathmoveto{\pgfqpoint{2.711715in}{2.316717in}}%
\pgfpathlineto{\pgfqpoint{2.711715in}{2.316717in}}%
\pgfpathlineto{\pgfqpoint{2.711715in}{2.319666in}}%
\pgfpathlineto{\pgfqpoint{2.716256in}{2.319666in}}%
\pgfpathlineto{\pgfqpoint{2.716256in}{2.316717in}}%
\pgfpathmoveto{\pgfqpoint{2.711715in}{2.319666in}}%
\pgfpathlineto{\pgfqpoint{2.711715in}{2.319666in}}%
\pgfpathlineto{\pgfqpoint{2.711715in}{2.322615in}}%
\pgfpathlineto{\pgfqpoint{2.716256in}{2.322615in}}%
\pgfpathlineto{\pgfqpoint{2.716256in}{2.319666in}}%
\pgfpathmoveto{\pgfqpoint{2.716256in}{2.316717in}}%
\pgfpathlineto{\pgfqpoint{2.716256in}{2.316717in}}%
\pgfpathlineto{\pgfqpoint{2.716256in}{2.319666in}}%
\pgfpathlineto{\pgfqpoint{2.720797in}{2.319666in}}%
\pgfpathlineto{\pgfqpoint{2.720797in}{2.316717in}}%
\pgfpathmoveto{\pgfqpoint{2.716256in}{2.319666in}}%
\pgfpathlineto{\pgfqpoint{2.716256in}{2.319666in}}%
\pgfpathlineto{\pgfqpoint{2.716256in}{2.322615in}}%
\pgfpathlineto{\pgfqpoint{2.720797in}{2.322615in}}%
\pgfpathlineto{\pgfqpoint{2.720797in}{2.319666in}}%
\pgfpathmoveto{\pgfqpoint{2.711715in}{2.322615in}}%
\pgfpathlineto{\pgfqpoint{2.711715in}{2.322615in}}%
\pgfpathlineto{\pgfqpoint{2.711715in}{2.325565in}}%
\pgfpathlineto{\pgfqpoint{2.716256in}{2.325565in}}%
\pgfpathlineto{\pgfqpoint{2.716256in}{2.322615in}}%
\pgfpathmoveto{\pgfqpoint{2.720797in}{2.316717in}}%
\pgfpathlineto{\pgfqpoint{2.720797in}{2.316717in}}%
\pgfpathlineto{\pgfqpoint{2.720797in}{2.319666in}}%
\pgfpathlineto{\pgfqpoint{2.725338in}{2.319666in}}%
\pgfpathlineto{\pgfqpoint{2.725338in}{2.316717in}}%
\pgfpathmoveto{\pgfqpoint{2.748043in}{2.293123in}}%
\pgfpathlineto{\pgfqpoint{2.748043in}{2.293123in}}%
\pgfpathlineto{\pgfqpoint{2.748043in}{2.296072in}}%
\pgfpathlineto{\pgfqpoint{2.752584in}{2.296072in}}%
\pgfpathlineto{\pgfqpoint{2.752584in}{2.293123in}}%
\pgfpathmoveto{\pgfqpoint{2.639058in}{2.387498in}}%
\pgfpathlineto{\pgfqpoint{2.639058in}{2.387498in}}%
\pgfpathlineto{\pgfqpoint{2.639058in}{2.390448in}}%
\pgfpathlineto{\pgfqpoint{2.643599in}{2.390448in}}%
\pgfpathlineto{\pgfqpoint{2.643599in}{2.387498in}}%
\pgfpathmoveto{\pgfqpoint{2.902442in}{2.145663in}}%
\pgfpathlineto{\pgfqpoint{2.902442in}{2.145663in}}%
\pgfpathlineto{\pgfqpoint{2.902442in}{2.148612in}}%
\pgfpathlineto{\pgfqpoint{2.906983in}{2.148612in}}%
\pgfpathlineto{\pgfqpoint{2.906983in}{2.145663in}}%
\pgfpathmoveto{\pgfqpoint{2.902442in}{2.148612in}}%
\pgfpathlineto{\pgfqpoint{2.902442in}{2.148612in}}%
\pgfpathlineto{\pgfqpoint{2.902442in}{2.151562in}}%
\pgfpathlineto{\pgfqpoint{2.906983in}{2.151562in}}%
\pgfpathlineto{\pgfqpoint{2.906983in}{2.148612in}}%
\pgfpathmoveto{\pgfqpoint{2.906983in}{2.145663in}}%
\pgfpathlineto{\pgfqpoint{2.906983in}{2.145663in}}%
\pgfpathlineto{\pgfqpoint{2.906983in}{2.148612in}}%
\pgfpathlineto{\pgfqpoint{2.911525in}{2.148612in}}%
\pgfpathlineto{\pgfqpoint{2.911525in}{2.145663in}}%
\pgfpathmoveto{\pgfqpoint{2.906983in}{2.148612in}}%
\pgfpathlineto{\pgfqpoint{2.906983in}{2.148612in}}%
\pgfpathlineto{\pgfqpoint{2.906983in}{2.151562in}}%
\pgfpathlineto{\pgfqpoint{2.911525in}{2.151562in}}%
\pgfpathlineto{\pgfqpoint{2.911525in}{2.148612in}}%
\pgfpathmoveto{\pgfqpoint{2.920607in}{2.133866in}}%
\pgfpathlineto{\pgfqpoint{2.920607in}{2.133866in}}%
\pgfpathlineto{\pgfqpoint{2.920607in}{2.136816in}}%
\pgfpathlineto{\pgfqpoint{2.925148in}{2.136816in}}%
\pgfpathlineto{\pgfqpoint{2.925148in}{2.133866in}}%
\pgfpathmoveto{\pgfqpoint{2.920607in}{2.136816in}}%
\pgfpathlineto{\pgfqpoint{2.920607in}{2.136816in}}%
\pgfpathlineto{\pgfqpoint{2.920607in}{2.139765in}}%
\pgfpathlineto{\pgfqpoint{2.925148in}{2.139765in}}%
\pgfpathlineto{\pgfqpoint{2.925148in}{2.136816in}}%
\pgfpathmoveto{\pgfqpoint{2.925148in}{2.133866in}}%
\pgfpathlineto{\pgfqpoint{2.925148in}{2.133866in}}%
\pgfpathlineto{\pgfqpoint{2.925148in}{2.136816in}}%
\pgfpathlineto{\pgfqpoint{2.929689in}{2.136816in}}%
\pgfpathlineto{\pgfqpoint{2.929689in}{2.133866in}}%
\pgfpathmoveto{\pgfqpoint{2.925148in}{2.136816in}}%
\pgfpathlineto{\pgfqpoint{2.925148in}{2.136816in}}%
\pgfpathlineto{\pgfqpoint{2.925148in}{2.139765in}}%
\pgfpathlineto{\pgfqpoint{2.929689in}{2.139765in}}%
\pgfpathlineto{\pgfqpoint{2.929689in}{2.136816in}}%
\pgfpathmoveto{\pgfqpoint{2.911525in}{2.139765in}}%
\pgfpathlineto{\pgfqpoint{2.911525in}{2.139765in}}%
\pgfpathlineto{\pgfqpoint{2.911525in}{2.142714in}}%
\pgfpathlineto{\pgfqpoint{2.916066in}{2.142714in}}%
\pgfpathlineto{\pgfqpoint{2.916066in}{2.139765in}}%
\pgfpathmoveto{\pgfqpoint{2.911525in}{2.142714in}}%
\pgfpathlineto{\pgfqpoint{2.911525in}{2.142714in}}%
\pgfpathlineto{\pgfqpoint{2.911525in}{2.145663in}}%
\pgfpathlineto{\pgfqpoint{2.916066in}{2.145663in}}%
\pgfpathlineto{\pgfqpoint{2.916066in}{2.142714in}}%
\pgfpathmoveto{\pgfqpoint{2.916066in}{2.139765in}}%
\pgfpathlineto{\pgfqpoint{2.916066in}{2.139765in}}%
\pgfpathlineto{\pgfqpoint{2.916066in}{2.142714in}}%
\pgfpathlineto{\pgfqpoint{2.920607in}{2.142714in}}%
\pgfpathlineto{\pgfqpoint{2.920607in}{2.139765in}}%
\pgfpathmoveto{\pgfqpoint{2.916066in}{2.142714in}}%
\pgfpathlineto{\pgfqpoint{2.916066in}{2.142714in}}%
\pgfpathlineto{\pgfqpoint{2.916066in}{2.145663in}}%
\pgfpathlineto{\pgfqpoint{2.920607in}{2.145663in}}%
\pgfpathlineto{\pgfqpoint{2.920607in}{2.142714in}}%
\pgfpathmoveto{\pgfqpoint{2.911525in}{2.145663in}}%
\pgfpathlineto{\pgfqpoint{2.911525in}{2.145663in}}%
\pgfpathlineto{\pgfqpoint{2.911525in}{2.148612in}}%
\pgfpathlineto{\pgfqpoint{2.916066in}{2.148612in}}%
\pgfpathlineto{\pgfqpoint{2.916066in}{2.145663in}}%
\pgfpathmoveto{\pgfqpoint{2.911525in}{2.148612in}}%
\pgfpathlineto{\pgfqpoint{2.911525in}{2.148612in}}%
\pgfpathlineto{\pgfqpoint{2.911525in}{2.151562in}}%
\pgfpathlineto{\pgfqpoint{2.916066in}{2.151562in}}%
\pgfpathlineto{\pgfqpoint{2.916066in}{2.148612in}}%
\pgfpathmoveto{\pgfqpoint{2.916066in}{2.145663in}}%
\pgfpathlineto{\pgfqpoint{2.916066in}{2.145663in}}%
\pgfpathlineto{\pgfqpoint{2.916066in}{2.148612in}}%
\pgfpathlineto{\pgfqpoint{2.920607in}{2.148612in}}%
\pgfpathlineto{\pgfqpoint{2.920607in}{2.145663in}}%
\pgfpathmoveto{\pgfqpoint{2.920607in}{2.139765in}}%
\pgfpathlineto{\pgfqpoint{2.920607in}{2.139765in}}%
\pgfpathlineto{\pgfqpoint{2.920607in}{2.142714in}}%
\pgfpathlineto{\pgfqpoint{2.925148in}{2.142714in}}%
\pgfpathlineto{\pgfqpoint{2.925148in}{2.139765in}}%
\pgfpathmoveto{\pgfqpoint{2.920607in}{2.142714in}}%
\pgfpathlineto{\pgfqpoint{2.920607in}{2.142714in}}%
\pgfpathlineto{\pgfqpoint{2.920607in}{2.145663in}}%
\pgfpathlineto{\pgfqpoint{2.925148in}{2.145663in}}%
\pgfpathlineto{\pgfqpoint{2.925148in}{2.142714in}}%
\pgfpathmoveto{\pgfqpoint{2.884278in}{2.163359in}}%
\pgfpathlineto{\pgfqpoint{2.884278in}{2.163359in}}%
\pgfpathlineto{\pgfqpoint{2.884278in}{2.166308in}}%
\pgfpathlineto{\pgfqpoint{2.888819in}{2.166308in}}%
\pgfpathlineto{\pgfqpoint{2.888819in}{2.163359in}}%
\pgfpathmoveto{\pgfqpoint{2.884278in}{2.166308in}}%
\pgfpathlineto{\pgfqpoint{2.884278in}{2.166308in}}%
\pgfpathlineto{\pgfqpoint{2.884278in}{2.169257in}}%
\pgfpathlineto{\pgfqpoint{2.888819in}{2.169257in}}%
\pgfpathlineto{\pgfqpoint{2.888819in}{2.166308in}}%
\pgfpathmoveto{\pgfqpoint{2.888819in}{2.163359in}}%
\pgfpathlineto{\pgfqpoint{2.888819in}{2.163359in}}%
\pgfpathlineto{\pgfqpoint{2.888819in}{2.166308in}}%
\pgfpathlineto{\pgfqpoint{2.893360in}{2.166308in}}%
\pgfpathlineto{\pgfqpoint{2.893360in}{2.163359in}}%
\pgfpathmoveto{\pgfqpoint{2.888819in}{2.166308in}}%
\pgfpathlineto{\pgfqpoint{2.888819in}{2.166308in}}%
\pgfpathlineto{\pgfqpoint{2.888819in}{2.169257in}}%
\pgfpathlineto{\pgfqpoint{2.893360in}{2.169257in}}%
\pgfpathlineto{\pgfqpoint{2.893360in}{2.166308in}}%
\pgfpathmoveto{\pgfqpoint{2.884278in}{2.169257in}}%
\pgfpathlineto{\pgfqpoint{2.884278in}{2.169257in}}%
\pgfpathlineto{\pgfqpoint{2.884278in}{2.172206in}}%
\pgfpathlineto{\pgfqpoint{2.888819in}{2.172206in}}%
\pgfpathlineto{\pgfqpoint{2.888819in}{2.169257in}}%
\pgfpathmoveto{\pgfqpoint{2.884278in}{2.172206in}}%
\pgfpathlineto{\pgfqpoint{2.884278in}{2.172206in}}%
\pgfpathlineto{\pgfqpoint{2.884278in}{2.175156in}}%
\pgfpathlineto{\pgfqpoint{2.888819in}{2.175156in}}%
\pgfpathlineto{\pgfqpoint{2.888819in}{2.172206in}}%
\pgfpathmoveto{\pgfqpoint{2.888819in}{2.169257in}}%
\pgfpathlineto{\pgfqpoint{2.888819in}{2.169257in}}%
\pgfpathlineto{\pgfqpoint{2.888819in}{2.172206in}}%
\pgfpathlineto{\pgfqpoint{2.893360in}{2.172206in}}%
\pgfpathlineto{\pgfqpoint{2.893360in}{2.169257in}}%
\pgfpathmoveto{\pgfqpoint{2.866113in}{2.181054in}}%
\pgfpathlineto{\pgfqpoint{2.866113in}{2.181054in}}%
\pgfpathlineto{\pgfqpoint{2.866113in}{2.184003in}}%
\pgfpathlineto{\pgfqpoint{2.870654in}{2.184003in}}%
\pgfpathlineto{\pgfqpoint{2.870654in}{2.181054in}}%
\pgfpathmoveto{\pgfqpoint{2.866113in}{2.184003in}}%
\pgfpathlineto{\pgfqpoint{2.866113in}{2.184003in}}%
\pgfpathlineto{\pgfqpoint{2.866113in}{2.186952in}}%
\pgfpathlineto{\pgfqpoint{2.870654in}{2.186952in}}%
\pgfpathlineto{\pgfqpoint{2.870654in}{2.184003in}}%
\pgfpathmoveto{\pgfqpoint{2.870654in}{2.181054in}}%
\pgfpathlineto{\pgfqpoint{2.870654in}{2.181054in}}%
\pgfpathlineto{\pgfqpoint{2.870654in}{2.184003in}}%
\pgfpathlineto{\pgfqpoint{2.875195in}{2.184003in}}%
\pgfpathlineto{\pgfqpoint{2.875195in}{2.181054in}}%
\pgfpathmoveto{\pgfqpoint{2.870654in}{2.184003in}}%
\pgfpathlineto{\pgfqpoint{2.870654in}{2.184003in}}%
\pgfpathlineto{\pgfqpoint{2.870654in}{2.186952in}}%
\pgfpathlineto{\pgfqpoint{2.875195in}{2.186952in}}%
\pgfpathlineto{\pgfqpoint{2.875195in}{2.184003in}}%
\pgfpathmoveto{\pgfqpoint{2.857030in}{2.186952in}}%
\pgfpathlineto{\pgfqpoint{2.857030in}{2.186952in}}%
\pgfpathlineto{\pgfqpoint{2.857030in}{2.189902in}}%
\pgfpathlineto{\pgfqpoint{2.861572in}{2.189902in}}%
\pgfpathlineto{\pgfqpoint{2.861572in}{2.186952in}}%
\pgfpathmoveto{\pgfqpoint{2.857030in}{2.189902in}}%
\pgfpathlineto{\pgfqpoint{2.857030in}{2.189902in}}%
\pgfpathlineto{\pgfqpoint{2.857030in}{2.192851in}}%
\pgfpathlineto{\pgfqpoint{2.861572in}{2.192851in}}%
\pgfpathlineto{\pgfqpoint{2.861572in}{2.189902in}}%
\pgfpathmoveto{\pgfqpoint{2.861572in}{2.186952in}}%
\pgfpathlineto{\pgfqpoint{2.861572in}{2.186952in}}%
\pgfpathlineto{\pgfqpoint{2.861572in}{2.189902in}}%
\pgfpathlineto{\pgfqpoint{2.866113in}{2.189902in}}%
\pgfpathlineto{\pgfqpoint{2.866113in}{2.186952in}}%
\pgfpathmoveto{\pgfqpoint{2.861572in}{2.189902in}}%
\pgfpathlineto{\pgfqpoint{2.861572in}{2.189902in}}%
\pgfpathlineto{\pgfqpoint{2.861572in}{2.192851in}}%
\pgfpathlineto{\pgfqpoint{2.866113in}{2.192851in}}%
\pgfpathlineto{\pgfqpoint{2.866113in}{2.189902in}}%
\pgfpathmoveto{\pgfqpoint{2.857030in}{2.192851in}}%
\pgfpathlineto{\pgfqpoint{2.857030in}{2.192851in}}%
\pgfpathlineto{\pgfqpoint{2.857030in}{2.195800in}}%
\pgfpathlineto{\pgfqpoint{2.861572in}{2.195800in}}%
\pgfpathlineto{\pgfqpoint{2.861572in}{2.192851in}}%
\pgfpathmoveto{\pgfqpoint{2.857030in}{2.195800in}}%
\pgfpathlineto{\pgfqpoint{2.857030in}{2.195800in}}%
\pgfpathlineto{\pgfqpoint{2.857030in}{2.198749in}}%
\pgfpathlineto{\pgfqpoint{2.861572in}{2.198749in}}%
\pgfpathlineto{\pgfqpoint{2.861572in}{2.195800in}}%
\pgfpathmoveto{\pgfqpoint{2.861572in}{2.192851in}}%
\pgfpathlineto{\pgfqpoint{2.861572in}{2.192851in}}%
\pgfpathlineto{\pgfqpoint{2.861572in}{2.195800in}}%
\pgfpathlineto{\pgfqpoint{2.866113in}{2.195800in}}%
\pgfpathlineto{\pgfqpoint{2.866113in}{2.192851in}}%
\pgfpathmoveto{\pgfqpoint{2.866113in}{2.186952in}}%
\pgfpathlineto{\pgfqpoint{2.866113in}{2.186952in}}%
\pgfpathlineto{\pgfqpoint{2.866113in}{2.189902in}}%
\pgfpathlineto{\pgfqpoint{2.870654in}{2.189902in}}%
\pgfpathlineto{\pgfqpoint{2.870654in}{2.186952in}}%
\pgfpathmoveto{\pgfqpoint{2.866113in}{2.189902in}}%
\pgfpathlineto{\pgfqpoint{2.866113in}{2.189902in}}%
\pgfpathlineto{\pgfqpoint{2.866113in}{2.192851in}}%
\pgfpathlineto{\pgfqpoint{2.870654in}{2.192851in}}%
\pgfpathlineto{\pgfqpoint{2.870654in}{2.189902in}}%
\pgfpathmoveto{\pgfqpoint{2.870654in}{2.186952in}}%
\pgfpathlineto{\pgfqpoint{2.870654in}{2.186952in}}%
\pgfpathlineto{\pgfqpoint{2.870654in}{2.189902in}}%
\pgfpathlineto{\pgfqpoint{2.875195in}{2.189902in}}%
\pgfpathlineto{\pgfqpoint{2.875195in}{2.186952in}}%
\pgfpathmoveto{\pgfqpoint{2.875195in}{2.175156in}}%
\pgfpathlineto{\pgfqpoint{2.875195in}{2.175156in}}%
\pgfpathlineto{\pgfqpoint{2.875195in}{2.178105in}}%
\pgfpathlineto{\pgfqpoint{2.879736in}{2.178105in}}%
\pgfpathlineto{\pgfqpoint{2.879736in}{2.175156in}}%
\pgfpathmoveto{\pgfqpoint{2.875195in}{2.178105in}}%
\pgfpathlineto{\pgfqpoint{2.875195in}{2.178105in}}%
\pgfpathlineto{\pgfqpoint{2.875195in}{2.181054in}}%
\pgfpathlineto{\pgfqpoint{2.879736in}{2.181054in}}%
\pgfpathlineto{\pgfqpoint{2.879736in}{2.178105in}}%
\pgfpathmoveto{\pgfqpoint{2.879736in}{2.175156in}}%
\pgfpathlineto{\pgfqpoint{2.879736in}{2.175156in}}%
\pgfpathlineto{\pgfqpoint{2.879736in}{2.178105in}}%
\pgfpathlineto{\pgfqpoint{2.884278in}{2.178105in}}%
\pgfpathlineto{\pgfqpoint{2.884278in}{2.175156in}}%
\pgfpathmoveto{\pgfqpoint{2.879736in}{2.178105in}}%
\pgfpathlineto{\pgfqpoint{2.879736in}{2.178105in}}%
\pgfpathlineto{\pgfqpoint{2.879736in}{2.181054in}}%
\pgfpathlineto{\pgfqpoint{2.884278in}{2.181054in}}%
\pgfpathlineto{\pgfqpoint{2.884278in}{2.178105in}}%
\pgfpathmoveto{\pgfqpoint{2.875195in}{2.181054in}}%
\pgfpathlineto{\pgfqpoint{2.875195in}{2.181054in}}%
\pgfpathlineto{\pgfqpoint{2.875195in}{2.184003in}}%
\pgfpathlineto{\pgfqpoint{2.879736in}{2.184003in}}%
\pgfpathlineto{\pgfqpoint{2.879736in}{2.181054in}}%
\pgfpathmoveto{\pgfqpoint{2.884278in}{2.175156in}}%
\pgfpathlineto{\pgfqpoint{2.884278in}{2.175156in}}%
\pgfpathlineto{\pgfqpoint{2.884278in}{2.178105in}}%
\pgfpathlineto{\pgfqpoint{2.888819in}{2.178105in}}%
\pgfpathlineto{\pgfqpoint{2.888819in}{2.175156in}}%
\pgfpathmoveto{\pgfqpoint{2.893360in}{2.157460in}}%
\pgfpathlineto{\pgfqpoint{2.893360in}{2.157460in}}%
\pgfpathlineto{\pgfqpoint{2.893360in}{2.160409in}}%
\pgfpathlineto{\pgfqpoint{2.897901in}{2.160409in}}%
\pgfpathlineto{\pgfqpoint{2.897901in}{2.157460in}}%
\pgfpathmoveto{\pgfqpoint{2.893360in}{2.160409in}}%
\pgfpathlineto{\pgfqpoint{2.893360in}{2.160409in}}%
\pgfpathlineto{\pgfqpoint{2.893360in}{2.163359in}}%
\pgfpathlineto{\pgfqpoint{2.897901in}{2.163359in}}%
\pgfpathlineto{\pgfqpoint{2.897901in}{2.160409in}}%
\pgfpathmoveto{\pgfqpoint{2.897901in}{2.157460in}}%
\pgfpathlineto{\pgfqpoint{2.897901in}{2.157460in}}%
\pgfpathlineto{\pgfqpoint{2.897901in}{2.160409in}}%
\pgfpathlineto{\pgfqpoint{2.902442in}{2.160409in}}%
\pgfpathlineto{\pgfqpoint{2.902442in}{2.157460in}}%
\pgfpathmoveto{\pgfqpoint{2.897901in}{2.160409in}}%
\pgfpathlineto{\pgfqpoint{2.897901in}{2.160409in}}%
\pgfpathlineto{\pgfqpoint{2.897901in}{2.163359in}}%
\pgfpathlineto{\pgfqpoint{2.902442in}{2.163359in}}%
\pgfpathlineto{\pgfqpoint{2.902442in}{2.160409in}}%
\pgfpathmoveto{\pgfqpoint{2.902442in}{2.151562in}}%
\pgfpathlineto{\pgfqpoint{2.902442in}{2.151562in}}%
\pgfpathlineto{\pgfqpoint{2.902442in}{2.154511in}}%
\pgfpathlineto{\pgfqpoint{2.906983in}{2.154511in}}%
\pgfpathlineto{\pgfqpoint{2.906983in}{2.151562in}}%
\pgfpathmoveto{\pgfqpoint{2.902442in}{2.154511in}}%
\pgfpathlineto{\pgfqpoint{2.902442in}{2.154511in}}%
\pgfpathlineto{\pgfqpoint{2.902442in}{2.157460in}}%
\pgfpathlineto{\pgfqpoint{2.906983in}{2.157460in}}%
\pgfpathlineto{\pgfqpoint{2.906983in}{2.154511in}}%
\pgfpathmoveto{\pgfqpoint{2.906983in}{2.151562in}}%
\pgfpathlineto{\pgfqpoint{2.906983in}{2.151562in}}%
\pgfpathlineto{\pgfqpoint{2.906983in}{2.154511in}}%
\pgfpathlineto{\pgfqpoint{2.911525in}{2.154511in}}%
\pgfpathlineto{\pgfqpoint{2.911525in}{2.151562in}}%
\pgfpathmoveto{\pgfqpoint{2.906983in}{2.154511in}}%
\pgfpathlineto{\pgfqpoint{2.906983in}{2.154511in}}%
\pgfpathlineto{\pgfqpoint{2.906983in}{2.157460in}}%
\pgfpathlineto{\pgfqpoint{2.911525in}{2.157460in}}%
\pgfpathlineto{\pgfqpoint{2.911525in}{2.154511in}}%
\pgfpathmoveto{\pgfqpoint{2.902442in}{2.157460in}}%
\pgfpathlineto{\pgfqpoint{2.902442in}{2.157460in}}%
\pgfpathlineto{\pgfqpoint{2.902442in}{2.160409in}}%
\pgfpathlineto{\pgfqpoint{2.906983in}{2.160409in}}%
\pgfpathlineto{\pgfqpoint{2.906983in}{2.157460in}}%
\pgfpathmoveto{\pgfqpoint{2.893360in}{2.163359in}}%
\pgfpathlineto{\pgfqpoint{2.893360in}{2.163359in}}%
\pgfpathlineto{\pgfqpoint{2.893360in}{2.166308in}}%
\pgfpathlineto{\pgfqpoint{2.897901in}{2.166308in}}%
\pgfpathlineto{\pgfqpoint{2.897901in}{2.163359in}}%
\pgfpathmoveto{\pgfqpoint{2.893360in}{2.166308in}}%
\pgfpathlineto{\pgfqpoint{2.893360in}{2.166308in}}%
\pgfpathlineto{\pgfqpoint{2.893360in}{2.169257in}}%
\pgfpathlineto{\pgfqpoint{2.897901in}{2.169257in}}%
\pgfpathlineto{\pgfqpoint{2.897901in}{2.166308in}}%
\pgfpathmoveto{\pgfqpoint{2.897901in}{2.163359in}}%
\pgfpathlineto{\pgfqpoint{2.897901in}{2.163359in}}%
\pgfpathlineto{\pgfqpoint{2.897901in}{2.166308in}}%
\pgfpathlineto{\pgfqpoint{2.902442in}{2.166308in}}%
\pgfpathlineto{\pgfqpoint{2.902442in}{2.163359in}}%
\pgfpathmoveto{\pgfqpoint{2.811619in}{2.228241in}}%
\pgfpathlineto{\pgfqpoint{2.811619in}{2.228241in}}%
\pgfpathlineto{\pgfqpoint{2.811619in}{2.231190in}}%
\pgfpathlineto{\pgfqpoint{2.816160in}{2.231190in}}%
\pgfpathlineto{\pgfqpoint{2.816160in}{2.228241in}}%
\pgfpathmoveto{\pgfqpoint{2.811619in}{2.231190in}}%
\pgfpathlineto{\pgfqpoint{2.811619in}{2.231190in}}%
\pgfpathlineto{\pgfqpoint{2.811619in}{2.234139in}}%
\pgfpathlineto{\pgfqpoint{2.816160in}{2.234139in}}%
\pgfpathlineto{\pgfqpoint{2.816160in}{2.231190in}}%
\pgfpathmoveto{\pgfqpoint{2.816160in}{2.228241in}}%
\pgfpathlineto{\pgfqpoint{2.816160in}{2.228241in}}%
\pgfpathlineto{\pgfqpoint{2.816160in}{2.231190in}}%
\pgfpathlineto{\pgfqpoint{2.820701in}{2.231190in}}%
\pgfpathlineto{\pgfqpoint{2.820701in}{2.228241in}}%
\pgfpathmoveto{\pgfqpoint{2.816160in}{2.231190in}}%
\pgfpathlineto{\pgfqpoint{2.816160in}{2.231190in}}%
\pgfpathlineto{\pgfqpoint{2.816160in}{2.234139in}}%
\pgfpathlineto{\pgfqpoint{2.820701in}{2.234139in}}%
\pgfpathlineto{\pgfqpoint{2.820701in}{2.231190in}}%
\pgfpathmoveto{\pgfqpoint{2.802536in}{2.234139in}}%
\pgfpathlineto{\pgfqpoint{2.802536in}{2.234139in}}%
\pgfpathlineto{\pgfqpoint{2.802536in}{2.237089in}}%
\pgfpathlineto{\pgfqpoint{2.807077in}{2.237089in}}%
\pgfpathlineto{\pgfqpoint{2.807077in}{2.234139in}}%
\pgfpathmoveto{\pgfqpoint{2.802536in}{2.237089in}}%
\pgfpathlineto{\pgfqpoint{2.802536in}{2.237089in}}%
\pgfpathlineto{\pgfqpoint{2.802536in}{2.240038in}}%
\pgfpathlineto{\pgfqpoint{2.807077in}{2.240038in}}%
\pgfpathlineto{\pgfqpoint{2.807077in}{2.237089in}}%
\pgfpathmoveto{\pgfqpoint{2.807077in}{2.234139in}}%
\pgfpathlineto{\pgfqpoint{2.807077in}{2.234139in}}%
\pgfpathlineto{\pgfqpoint{2.807077in}{2.237089in}}%
\pgfpathlineto{\pgfqpoint{2.811619in}{2.237089in}}%
\pgfpathlineto{\pgfqpoint{2.811619in}{2.234139in}}%
\pgfpathmoveto{\pgfqpoint{2.807077in}{2.237089in}}%
\pgfpathlineto{\pgfqpoint{2.807077in}{2.237089in}}%
\pgfpathlineto{\pgfqpoint{2.807077in}{2.240038in}}%
\pgfpathlineto{\pgfqpoint{2.811619in}{2.240038in}}%
\pgfpathlineto{\pgfqpoint{2.811619in}{2.237089in}}%
\pgfpathmoveto{\pgfqpoint{2.802536in}{2.240038in}}%
\pgfpathlineto{\pgfqpoint{2.802536in}{2.240038in}}%
\pgfpathlineto{\pgfqpoint{2.802536in}{2.242987in}}%
\pgfpathlineto{\pgfqpoint{2.807077in}{2.242987in}}%
\pgfpathlineto{\pgfqpoint{2.807077in}{2.240038in}}%
\pgfpathmoveto{\pgfqpoint{2.802536in}{2.242987in}}%
\pgfpathlineto{\pgfqpoint{2.802536in}{2.242987in}}%
\pgfpathlineto{\pgfqpoint{2.802536in}{2.245936in}}%
\pgfpathlineto{\pgfqpoint{2.807077in}{2.245936in}}%
\pgfpathlineto{\pgfqpoint{2.807077in}{2.242987in}}%
\pgfpathmoveto{\pgfqpoint{2.807077in}{2.240038in}}%
\pgfpathlineto{\pgfqpoint{2.807077in}{2.240038in}}%
\pgfpathlineto{\pgfqpoint{2.807077in}{2.242987in}}%
\pgfpathlineto{\pgfqpoint{2.811619in}{2.242987in}}%
\pgfpathlineto{\pgfqpoint{2.811619in}{2.240038in}}%
\pgfpathmoveto{\pgfqpoint{2.811619in}{2.234139in}}%
\pgfpathlineto{\pgfqpoint{2.811619in}{2.234139in}}%
\pgfpathlineto{\pgfqpoint{2.811619in}{2.237089in}}%
\pgfpathlineto{\pgfqpoint{2.816160in}{2.237089in}}%
\pgfpathlineto{\pgfqpoint{2.816160in}{2.234139in}}%
\pgfpathmoveto{\pgfqpoint{2.811619in}{2.237089in}}%
\pgfpathlineto{\pgfqpoint{2.811619in}{2.237089in}}%
\pgfpathlineto{\pgfqpoint{2.811619in}{2.240038in}}%
\pgfpathlineto{\pgfqpoint{2.816160in}{2.240038in}}%
\pgfpathlineto{\pgfqpoint{2.816160in}{2.237089in}}%
\pgfpathmoveto{\pgfqpoint{2.816160in}{2.234139in}}%
\pgfpathlineto{\pgfqpoint{2.816160in}{2.234139in}}%
\pgfpathlineto{\pgfqpoint{2.816160in}{2.237089in}}%
\pgfpathlineto{\pgfqpoint{2.820701in}{2.237089in}}%
\pgfpathlineto{\pgfqpoint{2.820701in}{2.234139in}}%
\pgfpathmoveto{\pgfqpoint{2.829783in}{2.210546in}}%
\pgfpathlineto{\pgfqpoint{2.829783in}{2.210546in}}%
\pgfpathlineto{\pgfqpoint{2.829783in}{2.213495in}}%
\pgfpathlineto{\pgfqpoint{2.834325in}{2.213495in}}%
\pgfpathlineto{\pgfqpoint{2.834325in}{2.210546in}}%
\pgfpathmoveto{\pgfqpoint{2.829783in}{2.213495in}}%
\pgfpathlineto{\pgfqpoint{2.829783in}{2.213495in}}%
\pgfpathlineto{\pgfqpoint{2.829783in}{2.216444in}}%
\pgfpathlineto{\pgfqpoint{2.834325in}{2.216444in}}%
\pgfpathlineto{\pgfqpoint{2.834325in}{2.213495in}}%
\pgfpathmoveto{\pgfqpoint{2.834325in}{2.210546in}}%
\pgfpathlineto{\pgfqpoint{2.834325in}{2.210546in}}%
\pgfpathlineto{\pgfqpoint{2.834325in}{2.213495in}}%
\pgfpathlineto{\pgfqpoint{2.838866in}{2.213495in}}%
\pgfpathlineto{\pgfqpoint{2.838866in}{2.210546in}}%
\pgfpathmoveto{\pgfqpoint{2.834325in}{2.213495in}}%
\pgfpathlineto{\pgfqpoint{2.834325in}{2.213495in}}%
\pgfpathlineto{\pgfqpoint{2.834325in}{2.216444in}}%
\pgfpathlineto{\pgfqpoint{2.838866in}{2.216444in}}%
\pgfpathlineto{\pgfqpoint{2.838866in}{2.213495in}}%
\pgfpathmoveto{\pgfqpoint{2.829783in}{2.216444in}}%
\pgfpathlineto{\pgfqpoint{2.829783in}{2.216444in}}%
\pgfpathlineto{\pgfqpoint{2.829783in}{2.219394in}}%
\pgfpathlineto{\pgfqpoint{2.834325in}{2.219394in}}%
\pgfpathlineto{\pgfqpoint{2.834325in}{2.216444in}}%
\pgfpathmoveto{\pgfqpoint{2.829783in}{2.219394in}}%
\pgfpathlineto{\pgfqpoint{2.829783in}{2.219394in}}%
\pgfpathlineto{\pgfqpoint{2.829783in}{2.222343in}}%
\pgfpathlineto{\pgfqpoint{2.834325in}{2.222343in}}%
\pgfpathlineto{\pgfqpoint{2.834325in}{2.219394in}}%
\pgfpathmoveto{\pgfqpoint{2.834325in}{2.216444in}}%
\pgfpathlineto{\pgfqpoint{2.834325in}{2.216444in}}%
\pgfpathlineto{\pgfqpoint{2.834325in}{2.219394in}}%
\pgfpathlineto{\pgfqpoint{2.838866in}{2.219394in}}%
\pgfpathlineto{\pgfqpoint{2.838866in}{2.216444in}}%
\pgfpathmoveto{\pgfqpoint{2.838866in}{2.204648in}}%
\pgfpathlineto{\pgfqpoint{2.838866in}{2.204648in}}%
\pgfpathlineto{\pgfqpoint{2.838866in}{2.207597in}}%
\pgfpathlineto{\pgfqpoint{2.843407in}{2.207597in}}%
\pgfpathlineto{\pgfqpoint{2.843407in}{2.204648in}}%
\pgfpathmoveto{\pgfqpoint{2.838866in}{2.207597in}}%
\pgfpathlineto{\pgfqpoint{2.838866in}{2.207597in}}%
\pgfpathlineto{\pgfqpoint{2.838866in}{2.210546in}}%
\pgfpathlineto{\pgfqpoint{2.843407in}{2.210546in}}%
\pgfpathlineto{\pgfqpoint{2.843407in}{2.207597in}}%
\pgfpathmoveto{\pgfqpoint{2.843407in}{2.204648in}}%
\pgfpathlineto{\pgfqpoint{2.843407in}{2.204648in}}%
\pgfpathlineto{\pgfqpoint{2.843407in}{2.207597in}}%
\pgfpathlineto{\pgfqpoint{2.847948in}{2.207597in}}%
\pgfpathlineto{\pgfqpoint{2.847948in}{2.204648in}}%
\pgfpathmoveto{\pgfqpoint{2.843407in}{2.207597in}}%
\pgfpathlineto{\pgfqpoint{2.843407in}{2.207597in}}%
\pgfpathlineto{\pgfqpoint{2.843407in}{2.210546in}}%
\pgfpathlineto{\pgfqpoint{2.847948in}{2.210546in}}%
\pgfpathlineto{\pgfqpoint{2.847948in}{2.207597in}}%
\pgfpathmoveto{\pgfqpoint{2.847948in}{2.198749in}}%
\pgfpathlineto{\pgfqpoint{2.847948in}{2.198749in}}%
\pgfpathlineto{\pgfqpoint{2.847948in}{2.201699in}}%
\pgfpathlineto{\pgfqpoint{2.852489in}{2.201699in}}%
\pgfpathlineto{\pgfqpoint{2.852489in}{2.198749in}}%
\pgfpathmoveto{\pgfqpoint{2.847948in}{2.201699in}}%
\pgfpathlineto{\pgfqpoint{2.847948in}{2.201699in}}%
\pgfpathlineto{\pgfqpoint{2.847948in}{2.204648in}}%
\pgfpathlineto{\pgfqpoint{2.852489in}{2.204648in}}%
\pgfpathlineto{\pgfqpoint{2.852489in}{2.201699in}}%
\pgfpathmoveto{\pgfqpoint{2.852489in}{2.198749in}}%
\pgfpathlineto{\pgfqpoint{2.852489in}{2.198749in}}%
\pgfpathlineto{\pgfqpoint{2.852489in}{2.201699in}}%
\pgfpathlineto{\pgfqpoint{2.857030in}{2.201699in}}%
\pgfpathlineto{\pgfqpoint{2.857030in}{2.198749in}}%
\pgfpathmoveto{\pgfqpoint{2.852489in}{2.201699in}}%
\pgfpathlineto{\pgfqpoint{2.852489in}{2.201699in}}%
\pgfpathlineto{\pgfqpoint{2.852489in}{2.204648in}}%
\pgfpathlineto{\pgfqpoint{2.857030in}{2.204648in}}%
\pgfpathlineto{\pgfqpoint{2.857030in}{2.201699in}}%
\pgfpathmoveto{\pgfqpoint{2.847948in}{2.204648in}}%
\pgfpathlineto{\pgfqpoint{2.847948in}{2.204648in}}%
\pgfpathlineto{\pgfqpoint{2.847948in}{2.207597in}}%
\pgfpathlineto{\pgfqpoint{2.852489in}{2.207597in}}%
\pgfpathlineto{\pgfqpoint{2.852489in}{2.204648in}}%
\pgfpathmoveto{\pgfqpoint{2.838866in}{2.210546in}}%
\pgfpathlineto{\pgfqpoint{2.838866in}{2.210546in}}%
\pgfpathlineto{\pgfqpoint{2.838866in}{2.213495in}}%
\pgfpathlineto{\pgfqpoint{2.843407in}{2.213495in}}%
\pgfpathlineto{\pgfqpoint{2.843407in}{2.210546in}}%
\pgfpathmoveto{\pgfqpoint{2.838866in}{2.213495in}}%
\pgfpathlineto{\pgfqpoint{2.838866in}{2.213495in}}%
\pgfpathlineto{\pgfqpoint{2.838866in}{2.216444in}}%
\pgfpathlineto{\pgfqpoint{2.843407in}{2.216444in}}%
\pgfpathlineto{\pgfqpoint{2.843407in}{2.213495in}}%
\pgfpathmoveto{\pgfqpoint{2.843407in}{2.210546in}}%
\pgfpathlineto{\pgfqpoint{2.843407in}{2.210546in}}%
\pgfpathlineto{\pgfqpoint{2.843407in}{2.213495in}}%
\pgfpathlineto{\pgfqpoint{2.847948in}{2.213495in}}%
\pgfpathlineto{\pgfqpoint{2.847948in}{2.210546in}}%
\pgfpathmoveto{\pgfqpoint{2.820701in}{2.222343in}}%
\pgfpathlineto{\pgfqpoint{2.820701in}{2.222343in}}%
\pgfpathlineto{\pgfqpoint{2.820701in}{2.225292in}}%
\pgfpathlineto{\pgfqpoint{2.825242in}{2.225292in}}%
\pgfpathlineto{\pgfqpoint{2.825242in}{2.222343in}}%
\pgfpathmoveto{\pgfqpoint{2.820701in}{2.225292in}}%
\pgfpathlineto{\pgfqpoint{2.820701in}{2.225292in}}%
\pgfpathlineto{\pgfqpoint{2.820701in}{2.228241in}}%
\pgfpathlineto{\pgfqpoint{2.825242in}{2.228241in}}%
\pgfpathlineto{\pgfqpoint{2.825242in}{2.225292in}}%
\pgfpathmoveto{\pgfqpoint{2.825242in}{2.222343in}}%
\pgfpathlineto{\pgfqpoint{2.825242in}{2.222343in}}%
\pgfpathlineto{\pgfqpoint{2.825242in}{2.225292in}}%
\pgfpathlineto{\pgfqpoint{2.829783in}{2.225292in}}%
\pgfpathlineto{\pgfqpoint{2.829783in}{2.222343in}}%
\pgfpathmoveto{\pgfqpoint{2.825242in}{2.225292in}}%
\pgfpathlineto{\pgfqpoint{2.825242in}{2.225292in}}%
\pgfpathlineto{\pgfqpoint{2.825242in}{2.228241in}}%
\pgfpathlineto{\pgfqpoint{2.829783in}{2.228241in}}%
\pgfpathlineto{\pgfqpoint{2.829783in}{2.225292in}}%
\pgfpathmoveto{\pgfqpoint{2.820701in}{2.228241in}}%
\pgfpathlineto{\pgfqpoint{2.820701in}{2.228241in}}%
\pgfpathlineto{\pgfqpoint{2.820701in}{2.231190in}}%
\pgfpathlineto{\pgfqpoint{2.825242in}{2.231190in}}%
\pgfpathlineto{\pgfqpoint{2.825242in}{2.228241in}}%
\pgfpathmoveto{\pgfqpoint{2.829783in}{2.222343in}}%
\pgfpathlineto{\pgfqpoint{2.829783in}{2.222343in}}%
\pgfpathlineto{\pgfqpoint{2.829783in}{2.225292in}}%
\pgfpathlineto{\pgfqpoint{2.834325in}{2.225292in}}%
\pgfpathlineto{\pgfqpoint{2.834325in}{2.222343in}}%
\pgfpathmoveto{\pgfqpoint{2.784372in}{2.251835in}}%
\pgfpathlineto{\pgfqpoint{2.784372in}{2.251835in}}%
\pgfpathlineto{\pgfqpoint{2.784372in}{2.254784in}}%
\pgfpathlineto{\pgfqpoint{2.788913in}{2.254784in}}%
\pgfpathlineto{\pgfqpoint{2.788913in}{2.251835in}}%
\pgfpathmoveto{\pgfqpoint{2.784372in}{2.254784in}}%
\pgfpathlineto{\pgfqpoint{2.784372in}{2.254784in}}%
\pgfpathlineto{\pgfqpoint{2.784372in}{2.257733in}}%
\pgfpathlineto{\pgfqpoint{2.788913in}{2.257733in}}%
\pgfpathlineto{\pgfqpoint{2.788913in}{2.254784in}}%
\pgfpathmoveto{\pgfqpoint{2.788913in}{2.251835in}}%
\pgfpathlineto{\pgfqpoint{2.788913in}{2.251835in}}%
\pgfpathlineto{\pgfqpoint{2.788913in}{2.254784in}}%
\pgfpathlineto{\pgfqpoint{2.793454in}{2.254784in}}%
\pgfpathlineto{\pgfqpoint{2.793454in}{2.251835in}}%
\pgfpathmoveto{\pgfqpoint{2.788913in}{2.254784in}}%
\pgfpathlineto{\pgfqpoint{2.788913in}{2.254784in}}%
\pgfpathlineto{\pgfqpoint{2.788913in}{2.257733in}}%
\pgfpathlineto{\pgfqpoint{2.793454in}{2.257733in}}%
\pgfpathlineto{\pgfqpoint{2.793454in}{2.254784in}}%
\pgfpathmoveto{\pgfqpoint{2.793454in}{2.245936in}}%
\pgfpathlineto{\pgfqpoint{2.793454in}{2.245936in}}%
\pgfpathlineto{\pgfqpoint{2.793454in}{2.248885in}}%
\pgfpathlineto{\pgfqpoint{2.797995in}{2.248885in}}%
\pgfpathlineto{\pgfqpoint{2.797995in}{2.245936in}}%
\pgfpathmoveto{\pgfqpoint{2.793454in}{2.248885in}}%
\pgfpathlineto{\pgfqpoint{2.793454in}{2.248885in}}%
\pgfpathlineto{\pgfqpoint{2.793454in}{2.251835in}}%
\pgfpathlineto{\pgfqpoint{2.797995in}{2.251835in}}%
\pgfpathlineto{\pgfqpoint{2.797995in}{2.248885in}}%
\pgfpathmoveto{\pgfqpoint{2.797995in}{2.245936in}}%
\pgfpathlineto{\pgfqpoint{2.797995in}{2.245936in}}%
\pgfpathlineto{\pgfqpoint{2.797995in}{2.248885in}}%
\pgfpathlineto{\pgfqpoint{2.802536in}{2.248885in}}%
\pgfpathlineto{\pgfqpoint{2.802536in}{2.245936in}}%
\pgfpathmoveto{\pgfqpoint{2.797995in}{2.248885in}}%
\pgfpathlineto{\pgfqpoint{2.797995in}{2.248885in}}%
\pgfpathlineto{\pgfqpoint{2.797995in}{2.251835in}}%
\pgfpathlineto{\pgfqpoint{2.802536in}{2.251835in}}%
\pgfpathlineto{\pgfqpoint{2.802536in}{2.248885in}}%
\pgfpathmoveto{\pgfqpoint{2.793454in}{2.251835in}}%
\pgfpathlineto{\pgfqpoint{2.793454in}{2.251835in}}%
\pgfpathlineto{\pgfqpoint{2.793454in}{2.254784in}}%
\pgfpathlineto{\pgfqpoint{2.797995in}{2.254784in}}%
\pgfpathlineto{\pgfqpoint{2.797995in}{2.251835in}}%
\pgfpathmoveto{\pgfqpoint{2.784372in}{2.257733in}}%
\pgfpathlineto{\pgfqpoint{2.784372in}{2.257733in}}%
\pgfpathlineto{\pgfqpoint{2.784372in}{2.260682in}}%
\pgfpathlineto{\pgfqpoint{2.788913in}{2.260682in}}%
\pgfpathlineto{\pgfqpoint{2.788913in}{2.257733in}}%
\pgfpathmoveto{\pgfqpoint{2.784372in}{2.260682in}}%
\pgfpathlineto{\pgfqpoint{2.784372in}{2.260682in}}%
\pgfpathlineto{\pgfqpoint{2.784372in}{2.263631in}}%
\pgfpathlineto{\pgfqpoint{2.788913in}{2.263631in}}%
\pgfpathlineto{\pgfqpoint{2.788913in}{2.260682in}}%
\pgfpathmoveto{\pgfqpoint{2.788913in}{2.257733in}}%
\pgfpathlineto{\pgfqpoint{2.788913in}{2.257733in}}%
\pgfpathlineto{\pgfqpoint{2.788913in}{2.260682in}}%
\pgfpathlineto{\pgfqpoint{2.793454in}{2.260682in}}%
\pgfpathlineto{\pgfqpoint{2.793454in}{2.257733in}}%
\pgfpathmoveto{\pgfqpoint{2.802536in}{2.245936in}}%
\pgfpathlineto{\pgfqpoint{2.802536in}{2.245936in}}%
\pgfpathlineto{\pgfqpoint{2.802536in}{2.248885in}}%
\pgfpathlineto{\pgfqpoint{2.807077in}{2.248885in}}%
\pgfpathlineto{\pgfqpoint{2.807077in}{2.245936in}}%
\pgfpathmoveto{\pgfqpoint{2.857030in}{2.198749in}}%
\pgfpathlineto{\pgfqpoint{2.857030in}{2.198749in}}%
\pgfpathlineto{\pgfqpoint{2.857030in}{2.201699in}}%
\pgfpathlineto{\pgfqpoint{2.861572in}{2.201699in}}%
\pgfpathlineto{\pgfqpoint{2.861572in}{2.198749in}}%
\pgfpathmoveto{\pgfqpoint{2.993261in}{2.068983in}}%
\pgfpathlineto{\pgfqpoint{2.993261in}{2.068983in}}%
\pgfpathlineto{\pgfqpoint{2.993261in}{2.071932in}}%
\pgfpathlineto{\pgfqpoint{2.997802in}{2.071932in}}%
\pgfpathlineto{\pgfqpoint{2.997802in}{2.068983in}}%
\pgfpathmoveto{\pgfqpoint{2.993261in}{2.071932in}}%
\pgfpathlineto{\pgfqpoint{2.993261in}{2.071932in}}%
\pgfpathlineto{\pgfqpoint{2.993261in}{2.074882in}}%
\pgfpathlineto{\pgfqpoint{2.997802in}{2.074882in}}%
\pgfpathlineto{\pgfqpoint{2.997802in}{2.071932in}}%
\pgfpathmoveto{\pgfqpoint{2.997802in}{2.068983in}}%
\pgfpathlineto{\pgfqpoint{2.997802in}{2.068983in}}%
\pgfpathlineto{\pgfqpoint{2.997802in}{2.071932in}}%
\pgfpathlineto{\pgfqpoint{3.002343in}{2.071932in}}%
\pgfpathlineto{\pgfqpoint{3.002343in}{2.068983in}}%
\pgfpathmoveto{\pgfqpoint{2.997802in}{2.071932in}}%
\pgfpathlineto{\pgfqpoint{2.997802in}{2.071932in}}%
\pgfpathlineto{\pgfqpoint{2.997802in}{2.074882in}}%
\pgfpathlineto{\pgfqpoint{3.002343in}{2.074882in}}%
\pgfpathlineto{\pgfqpoint{3.002343in}{2.071932in}}%
\pgfpathmoveto{\pgfqpoint{2.993261in}{2.074882in}}%
\pgfpathlineto{\pgfqpoint{2.993261in}{2.074882in}}%
\pgfpathlineto{\pgfqpoint{2.993261in}{2.077831in}}%
\pgfpathlineto{\pgfqpoint{2.997802in}{2.077831in}}%
\pgfpathlineto{\pgfqpoint{2.997802in}{2.074882in}}%
\pgfpathmoveto{\pgfqpoint{2.993261in}{2.077831in}}%
\pgfpathlineto{\pgfqpoint{2.993261in}{2.077831in}}%
\pgfpathlineto{\pgfqpoint{2.993261in}{2.080780in}}%
\pgfpathlineto{\pgfqpoint{2.997802in}{2.080780in}}%
\pgfpathlineto{\pgfqpoint{2.997802in}{2.077831in}}%
\pgfpathmoveto{\pgfqpoint{2.997802in}{2.074882in}}%
\pgfpathlineto{\pgfqpoint{2.997802in}{2.074882in}}%
\pgfpathlineto{\pgfqpoint{2.997802in}{2.077831in}}%
\pgfpathlineto{\pgfqpoint{3.002343in}{2.077831in}}%
\pgfpathlineto{\pgfqpoint{3.002343in}{2.074882in}}%
\pgfpathmoveto{\pgfqpoint{2.975098in}{2.086679in}}%
\pgfpathlineto{\pgfqpoint{2.975098in}{2.086679in}}%
\pgfpathlineto{\pgfqpoint{2.975098in}{2.089628in}}%
\pgfpathlineto{\pgfqpoint{2.979639in}{2.089628in}}%
\pgfpathlineto{\pgfqpoint{2.979639in}{2.086679in}}%
\pgfpathmoveto{\pgfqpoint{2.975098in}{2.089628in}}%
\pgfpathlineto{\pgfqpoint{2.975098in}{2.089628in}}%
\pgfpathlineto{\pgfqpoint{2.975098in}{2.092577in}}%
\pgfpathlineto{\pgfqpoint{2.979639in}{2.092577in}}%
\pgfpathlineto{\pgfqpoint{2.979639in}{2.089628in}}%
\pgfpathmoveto{\pgfqpoint{2.979639in}{2.086679in}}%
\pgfpathlineto{\pgfqpoint{2.979639in}{2.086679in}}%
\pgfpathlineto{\pgfqpoint{2.979639in}{2.089628in}}%
\pgfpathlineto{\pgfqpoint{2.984180in}{2.089628in}}%
\pgfpathlineto{\pgfqpoint{2.984180in}{2.086679in}}%
\pgfpathmoveto{\pgfqpoint{2.979639in}{2.089628in}}%
\pgfpathlineto{\pgfqpoint{2.979639in}{2.089628in}}%
\pgfpathlineto{\pgfqpoint{2.979639in}{2.092577in}}%
\pgfpathlineto{\pgfqpoint{2.984180in}{2.092577in}}%
\pgfpathlineto{\pgfqpoint{2.984180in}{2.089628in}}%
\pgfpathmoveto{\pgfqpoint{2.966016in}{2.092577in}}%
\pgfpathlineto{\pgfqpoint{2.966016in}{2.092577in}}%
\pgfpathlineto{\pgfqpoint{2.966016in}{2.095526in}}%
\pgfpathlineto{\pgfqpoint{2.970557in}{2.095526in}}%
\pgfpathlineto{\pgfqpoint{2.970557in}{2.092577in}}%
\pgfpathmoveto{\pgfqpoint{2.966016in}{2.095526in}}%
\pgfpathlineto{\pgfqpoint{2.966016in}{2.095526in}}%
\pgfpathlineto{\pgfqpoint{2.966016in}{2.098476in}}%
\pgfpathlineto{\pgfqpoint{2.970557in}{2.098476in}}%
\pgfpathlineto{\pgfqpoint{2.970557in}{2.095526in}}%
\pgfpathmoveto{\pgfqpoint{2.970557in}{2.092577in}}%
\pgfpathlineto{\pgfqpoint{2.970557in}{2.092577in}}%
\pgfpathlineto{\pgfqpoint{2.970557in}{2.095526in}}%
\pgfpathlineto{\pgfqpoint{2.975098in}{2.095526in}}%
\pgfpathlineto{\pgfqpoint{2.975098in}{2.092577in}}%
\pgfpathmoveto{\pgfqpoint{2.970557in}{2.095526in}}%
\pgfpathlineto{\pgfqpoint{2.970557in}{2.095526in}}%
\pgfpathlineto{\pgfqpoint{2.970557in}{2.098476in}}%
\pgfpathlineto{\pgfqpoint{2.975098in}{2.098476in}}%
\pgfpathlineto{\pgfqpoint{2.975098in}{2.095526in}}%
\pgfpathmoveto{\pgfqpoint{2.966016in}{2.098476in}}%
\pgfpathlineto{\pgfqpoint{2.966016in}{2.098476in}}%
\pgfpathlineto{\pgfqpoint{2.966016in}{2.101425in}}%
\pgfpathlineto{\pgfqpoint{2.970557in}{2.101425in}}%
\pgfpathlineto{\pgfqpoint{2.970557in}{2.098476in}}%
\pgfpathmoveto{\pgfqpoint{2.966016in}{2.101425in}}%
\pgfpathlineto{\pgfqpoint{2.966016in}{2.101425in}}%
\pgfpathlineto{\pgfqpoint{2.966016in}{2.104374in}}%
\pgfpathlineto{\pgfqpoint{2.970557in}{2.104374in}}%
\pgfpathlineto{\pgfqpoint{2.970557in}{2.101425in}}%
\pgfpathmoveto{\pgfqpoint{2.970557in}{2.098476in}}%
\pgfpathlineto{\pgfqpoint{2.970557in}{2.098476in}}%
\pgfpathlineto{\pgfqpoint{2.970557in}{2.101425in}}%
\pgfpathlineto{\pgfqpoint{2.975098in}{2.101425in}}%
\pgfpathlineto{\pgfqpoint{2.975098in}{2.098476in}}%
\pgfpathmoveto{\pgfqpoint{2.975098in}{2.092577in}}%
\pgfpathlineto{\pgfqpoint{2.975098in}{2.092577in}}%
\pgfpathlineto{\pgfqpoint{2.975098in}{2.095526in}}%
\pgfpathlineto{\pgfqpoint{2.979639in}{2.095526in}}%
\pgfpathlineto{\pgfqpoint{2.979639in}{2.092577in}}%
\pgfpathmoveto{\pgfqpoint{2.975098in}{2.095526in}}%
\pgfpathlineto{\pgfqpoint{2.975098in}{2.095526in}}%
\pgfpathlineto{\pgfqpoint{2.975098in}{2.098476in}}%
\pgfpathlineto{\pgfqpoint{2.979639in}{2.098476in}}%
\pgfpathlineto{\pgfqpoint{2.979639in}{2.095526in}}%
\pgfpathmoveto{\pgfqpoint{2.979639in}{2.092577in}}%
\pgfpathlineto{\pgfqpoint{2.979639in}{2.092577in}}%
\pgfpathlineto{\pgfqpoint{2.979639in}{2.095526in}}%
\pgfpathlineto{\pgfqpoint{2.984180in}{2.095526in}}%
\pgfpathlineto{\pgfqpoint{2.984180in}{2.092577in}}%
\pgfpathmoveto{\pgfqpoint{2.984180in}{2.080780in}}%
\pgfpathlineto{\pgfqpoint{2.984180in}{2.080780in}}%
\pgfpathlineto{\pgfqpoint{2.984180in}{2.083729in}}%
\pgfpathlineto{\pgfqpoint{2.988721in}{2.083729in}}%
\pgfpathlineto{\pgfqpoint{2.988721in}{2.080780in}}%
\pgfpathmoveto{\pgfqpoint{2.984180in}{2.083729in}}%
\pgfpathlineto{\pgfqpoint{2.984180in}{2.083729in}}%
\pgfpathlineto{\pgfqpoint{2.984180in}{2.086679in}}%
\pgfpathlineto{\pgfqpoint{2.988721in}{2.086679in}}%
\pgfpathlineto{\pgfqpoint{2.988721in}{2.083729in}}%
\pgfpathmoveto{\pgfqpoint{2.988721in}{2.080780in}}%
\pgfpathlineto{\pgfqpoint{2.988721in}{2.080780in}}%
\pgfpathlineto{\pgfqpoint{2.988721in}{2.083729in}}%
\pgfpathlineto{\pgfqpoint{2.993261in}{2.083729in}}%
\pgfpathlineto{\pgfqpoint{2.993261in}{2.080780in}}%
\pgfpathmoveto{\pgfqpoint{2.988721in}{2.083729in}}%
\pgfpathlineto{\pgfqpoint{2.988721in}{2.083729in}}%
\pgfpathlineto{\pgfqpoint{2.988721in}{2.086679in}}%
\pgfpathlineto{\pgfqpoint{2.993261in}{2.086679in}}%
\pgfpathlineto{\pgfqpoint{2.993261in}{2.083729in}}%
\pgfpathmoveto{\pgfqpoint{2.984180in}{2.086679in}}%
\pgfpathlineto{\pgfqpoint{2.984180in}{2.086679in}}%
\pgfpathlineto{\pgfqpoint{2.984180in}{2.089628in}}%
\pgfpathlineto{\pgfqpoint{2.988721in}{2.089628in}}%
\pgfpathlineto{\pgfqpoint{2.988721in}{2.086679in}}%
\pgfpathmoveto{\pgfqpoint{2.993261in}{2.080780in}}%
\pgfpathlineto{\pgfqpoint{2.993261in}{2.080780in}}%
\pgfpathlineto{\pgfqpoint{2.993261in}{2.083729in}}%
\pgfpathlineto{\pgfqpoint{2.997802in}{2.083729in}}%
\pgfpathlineto{\pgfqpoint{2.997802in}{2.080780in}}%
\pgfpathmoveto{\pgfqpoint{3.029588in}{2.039490in}}%
\pgfpathlineto{\pgfqpoint{3.029588in}{2.039490in}}%
\pgfpathlineto{\pgfqpoint{3.029588in}{2.042440in}}%
\pgfpathlineto{\pgfqpoint{3.034129in}{2.042440in}}%
\pgfpathlineto{\pgfqpoint{3.034129in}{2.039490in}}%
\pgfpathmoveto{\pgfqpoint{3.029588in}{2.042440in}}%
\pgfpathlineto{\pgfqpoint{3.029588in}{2.042440in}}%
\pgfpathlineto{\pgfqpoint{3.029588in}{2.045389in}}%
\pgfpathlineto{\pgfqpoint{3.034129in}{2.045389in}}%
\pgfpathlineto{\pgfqpoint{3.034129in}{2.042440in}}%
\pgfpathmoveto{\pgfqpoint{3.034129in}{2.039490in}}%
\pgfpathlineto{\pgfqpoint{3.034129in}{2.039490in}}%
\pgfpathlineto{\pgfqpoint{3.034129in}{2.042440in}}%
\pgfpathlineto{\pgfqpoint{3.038670in}{2.042440in}}%
\pgfpathlineto{\pgfqpoint{3.038670in}{2.039490in}}%
\pgfpathmoveto{\pgfqpoint{3.034129in}{2.042440in}}%
\pgfpathlineto{\pgfqpoint{3.034129in}{2.042440in}}%
\pgfpathlineto{\pgfqpoint{3.034129in}{2.045389in}}%
\pgfpathlineto{\pgfqpoint{3.038670in}{2.045389in}}%
\pgfpathlineto{\pgfqpoint{3.038670in}{2.042440in}}%
\pgfpathmoveto{\pgfqpoint{3.020507in}{2.045389in}}%
\pgfpathlineto{\pgfqpoint{3.020507in}{2.045389in}}%
\pgfpathlineto{\pgfqpoint{3.020507in}{2.048338in}}%
\pgfpathlineto{\pgfqpoint{3.025047in}{2.048338in}}%
\pgfpathlineto{\pgfqpoint{3.025047in}{2.045389in}}%
\pgfpathmoveto{\pgfqpoint{3.020507in}{2.048338in}}%
\pgfpathlineto{\pgfqpoint{3.020507in}{2.048338in}}%
\pgfpathlineto{\pgfqpoint{3.020507in}{2.051287in}}%
\pgfpathlineto{\pgfqpoint{3.025047in}{2.051287in}}%
\pgfpathlineto{\pgfqpoint{3.025047in}{2.048338in}}%
\pgfpathmoveto{\pgfqpoint{3.025047in}{2.045389in}}%
\pgfpathlineto{\pgfqpoint{3.025047in}{2.045389in}}%
\pgfpathlineto{\pgfqpoint{3.025047in}{2.048338in}}%
\pgfpathlineto{\pgfqpoint{3.029588in}{2.048338in}}%
\pgfpathlineto{\pgfqpoint{3.029588in}{2.045389in}}%
\pgfpathmoveto{\pgfqpoint{3.025047in}{2.048338in}}%
\pgfpathlineto{\pgfqpoint{3.025047in}{2.048338in}}%
\pgfpathlineto{\pgfqpoint{3.025047in}{2.051287in}}%
\pgfpathlineto{\pgfqpoint{3.029588in}{2.051287in}}%
\pgfpathlineto{\pgfqpoint{3.029588in}{2.048338in}}%
\pgfpathmoveto{\pgfqpoint{3.020507in}{2.051287in}}%
\pgfpathlineto{\pgfqpoint{3.020507in}{2.051287in}}%
\pgfpathlineto{\pgfqpoint{3.020507in}{2.054237in}}%
\pgfpathlineto{\pgfqpoint{3.025047in}{2.054237in}}%
\pgfpathlineto{\pgfqpoint{3.025047in}{2.051287in}}%
\pgfpathmoveto{\pgfqpoint{3.020507in}{2.054237in}}%
\pgfpathlineto{\pgfqpoint{3.020507in}{2.054237in}}%
\pgfpathlineto{\pgfqpoint{3.020507in}{2.057186in}}%
\pgfpathlineto{\pgfqpoint{3.025047in}{2.057186in}}%
\pgfpathlineto{\pgfqpoint{3.025047in}{2.054237in}}%
\pgfpathmoveto{\pgfqpoint{3.025047in}{2.051287in}}%
\pgfpathlineto{\pgfqpoint{3.025047in}{2.051287in}}%
\pgfpathlineto{\pgfqpoint{3.025047in}{2.054237in}}%
\pgfpathlineto{\pgfqpoint{3.029588in}{2.054237in}}%
\pgfpathlineto{\pgfqpoint{3.029588in}{2.051287in}}%
\pgfpathmoveto{\pgfqpoint{3.029588in}{2.045389in}}%
\pgfpathlineto{\pgfqpoint{3.029588in}{2.045389in}}%
\pgfpathlineto{\pgfqpoint{3.029588in}{2.048338in}}%
\pgfpathlineto{\pgfqpoint{3.034129in}{2.048338in}}%
\pgfpathlineto{\pgfqpoint{3.034129in}{2.045389in}}%
\pgfpathmoveto{\pgfqpoint{3.029588in}{2.048338in}}%
\pgfpathlineto{\pgfqpoint{3.029588in}{2.048338in}}%
\pgfpathlineto{\pgfqpoint{3.029588in}{2.051287in}}%
\pgfpathlineto{\pgfqpoint{3.034129in}{2.051287in}}%
\pgfpathlineto{\pgfqpoint{3.034129in}{2.048338in}}%
\pgfpathmoveto{\pgfqpoint{3.034129in}{2.045389in}}%
\pgfpathlineto{\pgfqpoint{3.034129in}{2.045389in}}%
\pgfpathlineto{\pgfqpoint{3.034129in}{2.048338in}}%
\pgfpathlineto{\pgfqpoint{3.038670in}{2.048338in}}%
\pgfpathlineto{\pgfqpoint{3.038670in}{2.045389in}}%
\pgfpathmoveto{\pgfqpoint{3.047752in}{2.021795in}}%
\pgfpathlineto{\pgfqpoint{3.047752in}{2.021795in}}%
\pgfpathlineto{\pgfqpoint{3.047752in}{2.024744in}}%
\pgfpathlineto{\pgfqpoint{3.052293in}{2.024744in}}%
\pgfpathlineto{\pgfqpoint{3.052293in}{2.021795in}}%
\pgfpathmoveto{\pgfqpoint{3.047752in}{2.024744in}}%
\pgfpathlineto{\pgfqpoint{3.047752in}{2.024744in}}%
\pgfpathlineto{\pgfqpoint{3.047752in}{2.027693in}}%
\pgfpathlineto{\pgfqpoint{3.052293in}{2.027693in}}%
\pgfpathlineto{\pgfqpoint{3.052293in}{2.024744in}}%
\pgfpathmoveto{\pgfqpoint{3.052293in}{2.021795in}}%
\pgfpathlineto{\pgfqpoint{3.052293in}{2.021795in}}%
\pgfpathlineto{\pgfqpoint{3.052293in}{2.024744in}}%
\pgfpathlineto{\pgfqpoint{3.056834in}{2.024744in}}%
\pgfpathlineto{\pgfqpoint{3.056834in}{2.021795in}}%
\pgfpathmoveto{\pgfqpoint{3.052293in}{2.024744in}}%
\pgfpathlineto{\pgfqpoint{3.052293in}{2.024744in}}%
\pgfpathlineto{\pgfqpoint{3.052293in}{2.027693in}}%
\pgfpathlineto{\pgfqpoint{3.056834in}{2.027693in}}%
\pgfpathlineto{\pgfqpoint{3.056834in}{2.024744in}}%
\pgfpathmoveto{\pgfqpoint{3.047752in}{2.027693in}}%
\pgfpathlineto{\pgfqpoint{3.047752in}{2.027693in}}%
\pgfpathlineto{\pgfqpoint{3.047752in}{2.030643in}}%
\pgfpathlineto{\pgfqpoint{3.052293in}{2.030643in}}%
\pgfpathlineto{\pgfqpoint{3.052293in}{2.027693in}}%
\pgfpathmoveto{\pgfqpoint{3.047752in}{2.030643in}}%
\pgfpathlineto{\pgfqpoint{3.047752in}{2.030643in}}%
\pgfpathlineto{\pgfqpoint{3.047752in}{2.033592in}}%
\pgfpathlineto{\pgfqpoint{3.052293in}{2.033592in}}%
\pgfpathlineto{\pgfqpoint{3.052293in}{2.030643in}}%
\pgfpathmoveto{\pgfqpoint{3.052293in}{2.027693in}}%
\pgfpathlineto{\pgfqpoint{3.052293in}{2.027693in}}%
\pgfpathlineto{\pgfqpoint{3.052293in}{2.030643in}}%
\pgfpathlineto{\pgfqpoint{3.056834in}{2.030643in}}%
\pgfpathlineto{\pgfqpoint{3.056834in}{2.027693in}}%
\pgfpathmoveto{\pgfqpoint{3.056834in}{2.015896in}}%
\pgfpathlineto{\pgfqpoint{3.056834in}{2.015896in}}%
\pgfpathlineto{\pgfqpoint{3.056834in}{2.018846in}}%
\pgfpathlineto{\pgfqpoint{3.061374in}{2.018846in}}%
\pgfpathlineto{\pgfqpoint{3.061374in}{2.015896in}}%
\pgfpathmoveto{\pgfqpoint{3.056834in}{2.018846in}}%
\pgfpathlineto{\pgfqpoint{3.056834in}{2.018846in}}%
\pgfpathlineto{\pgfqpoint{3.056834in}{2.021795in}}%
\pgfpathlineto{\pgfqpoint{3.061374in}{2.021795in}}%
\pgfpathlineto{\pgfqpoint{3.061374in}{2.018846in}}%
\pgfpathmoveto{\pgfqpoint{3.061374in}{2.015896in}}%
\pgfpathlineto{\pgfqpoint{3.061374in}{2.015896in}}%
\pgfpathlineto{\pgfqpoint{3.061374in}{2.018846in}}%
\pgfpathlineto{\pgfqpoint{3.065915in}{2.018846in}}%
\pgfpathlineto{\pgfqpoint{3.065915in}{2.015896in}}%
\pgfpathmoveto{\pgfqpoint{3.061374in}{2.018846in}}%
\pgfpathlineto{\pgfqpoint{3.061374in}{2.018846in}}%
\pgfpathlineto{\pgfqpoint{3.061374in}{2.021795in}}%
\pgfpathlineto{\pgfqpoint{3.065915in}{2.021795in}}%
\pgfpathlineto{\pgfqpoint{3.065915in}{2.018846in}}%
\pgfpathmoveto{\pgfqpoint{3.065915in}{2.009998in}}%
\pgfpathlineto{\pgfqpoint{3.065915in}{2.009998in}}%
\pgfpathlineto{\pgfqpoint{3.065915in}{2.012947in}}%
\pgfpathlineto{\pgfqpoint{3.070456in}{2.012947in}}%
\pgfpathlineto{\pgfqpoint{3.070456in}{2.009998in}}%
\pgfpathmoveto{\pgfqpoint{3.065915in}{2.012947in}}%
\pgfpathlineto{\pgfqpoint{3.065915in}{2.012947in}}%
\pgfpathlineto{\pgfqpoint{3.065915in}{2.015896in}}%
\pgfpathlineto{\pgfqpoint{3.070456in}{2.015896in}}%
\pgfpathlineto{\pgfqpoint{3.070456in}{2.012947in}}%
\pgfpathmoveto{\pgfqpoint{3.070456in}{2.009998in}}%
\pgfpathlineto{\pgfqpoint{3.070456in}{2.009998in}}%
\pgfpathlineto{\pgfqpoint{3.070456in}{2.012947in}}%
\pgfpathlineto{\pgfqpoint{3.074997in}{2.012947in}}%
\pgfpathlineto{\pgfqpoint{3.074997in}{2.009998in}}%
\pgfpathmoveto{\pgfqpoint{3.070456in}{2.012947in}}%
\pgfpathlineto{\pgfqpoint{3.070456in}{2.012947in}}%
\pgfpathlineto{\pgfqpoint{3.070456in}{2.015896in}}%
\pgfpathlineto{\pgfqpoint{3.074997in}{2.015896in}}%
\pgfpathlineto{\pgfqpoint{3.074997in}{2.012947in}}%
\pgfpathmoveto{\pgfqpoint{3.065915in}{2.015896in}}%
\pgfpathlineto{\pgfqpoint{3.065915in}{2.015896in}}%
\pgfpathlineto{\pgfqpoint{3.065915in}{2.018846in}}%
\pgfpathlineto{\pgfqpoint{3.070456in}{2.018846in}}%
\pgfpathlineto{\pgfqpoint{3.070456in}{2.015896in}}%
\pgfpathmoveto{\pgfqpoint{3.056834in}{2.021795in}}%
\pgfpathlineto{\pgfqpoint{3.056834in}{2.021795in}}%
\pgfpathlineto{\pgfqpoint{3.056834in}{2.024744in}}%
\pgfpathlineto{\pgfqpoint{3.061374in}{2.024744in}}%
\pgfpathlineto{\pgfqpoint{3.061374in}{2.021795in}}%
\pgfpathmoveto{\pgfqpoint{3.056834in}{2.024744in}}%
\pgfpathlineto{\pgfqpoint{3.056834in}{2.024744in}}%
\pgfpathlineto{\pgfqpoint{3.056834in}{2.027693in}}%
\pgfpathlineto{\pgfqpoint{3.061374in}{2.027693in}}%
\pgfpathlineto{\pgfqpoint{3.061374in}{2.024744in}}%
\pgfpathmoveto{\pgfqpoint{3.061374in}{2.021795in}}%
\pgfpathlineto{\pgfqpoint{3.061374in}{2.021795in}}%
\pgfpathlineto{\pgfqpoint{3.061374in}{2.024744in}}%
\pgfpathlineto{\pgfqpoint{3.065915in}{2.024744in}}%
\pgfpathlineto{\pgfqpoint{3.065915in}{2.021795in}}%
\pgfpathmoveto{\pgfqpoint{3.038670in}{2.033592in}}%
\pgfpathlineto{\pgfqpoint{3.038670in}{2.033592in}}%
\pgfpathlineto{\pgfqpoint{3.038670in}{2.036541in}}%
\pgfpathlineto{\pgfqpoint{3.043211in}{2.036541in}}%
\pgfpathlineto{\pgfqpoint{3.043211in}{2.033592in}}%
\pgfpathmoveto{\pgfqpoint{3.038670in}{2.036541in}}%
\pgfpathlineto{\pgfqpoint{3.038670in}{2.036541in}}%
\pgfpathlineto{\pgfqpoint{3.038670in}{2.039490in}}%
\pgfpathlineto{\pgfqpoint{3.043211in}{2.039490in}}%
\pgfpathlineto{\pgfqpoint{3.043211in}{2.036541in}}%
\pgfpathmoveto{\pgfqpoint{3.043211in}{2.033592in}}%
\pgfpathlineto{\pgfqpoint{3.043211in}{2.033592in}}%
\pgfpathlineto{\pgfqpoint{3.043211in}{2.036541in}}%
\pgfpathlineto{\pgfqpoint{3.047752in}{2.036541in}}%
\pgfpathlineto{\pgfqpoint{3.047752in}{2.033592in}}%
\pgfpathmoveto{\pgfqpoint{3.043211in}{2.036541in}}%
\pgfpathlineto{\pgfqpoint{3.043211in}{2.036541in}}%
\pgfpathlineto{\pgfqpoint{3.043211in}{2.039490in}}%
\pgfpathlineto{\pgfqpoint{3.047752in}{2.039490in}}%
\pgfpathlineto{\pgfqpoint{3.047752in}{2.036541in}}%
\pgfpathmoveto{\pgfqpoint{3.038670in}{2.039490in}}%
\pgfpathlineto{\pgfqpoint{3.038670in}{2.039490in}}%
\pgfpathlineto{\pgfqpoint{3.038670in}{2.042440in}}%
\pgfpathlineto{\pgfqpoint{3.043211in}{2.042440in}}%
\pgfpathlineto{\pgfqpoint{3.043211in}{2.039490in}}%
\pgfpathmoveto{\pgfqpoint{3.047752in}{2.033592in}}%
\pgfpathlineto{\pgfqpoint{3.047752in}{2.033592in}}%
\pgfpathlineto{\pgfqpoint{3.047752in}{2.036541in}}%
\pgfpathlineto{\pgfqpoint{3.052293in}{2.036541in}}%
\pgfpathlineto{\pgfqpoint{3.052293in}{2.033592in}}%
\pgfpathmoveto{\pgfqpoint{3.002343in}{2.063084in}}%
\pgfpathlineto{\pgfqpoint{3.002343in}{2.063084in}}%
\pgfpathlineto{\pgfqpoint{3.002343in}{2.066034in}}%
\pgfpathlineto{\pgfqpoint{3.006884in}{2.066034in}}%
\pgfpathlineto{\pgfqpoint{3.006884in}{2.063084in}}%
\pgfpathmoveto{\pgfqpoint{3.002343in}{2.066034in}}%
\pgfpathlineto{\pgfqpoint{3.002343in}{2.066034in}}%
\pgfpathlineto{\pgfqpoint{3.002343in}{2.068983in}}%
\pgfpathlineto{\pgfqpoint{3.006884in}{2.068983in}}%
\pgfpathlineto{\pgfqpoint{3.006884in}{2.066034in}}%
\pgfpathmoveto{\pgfqpoint{3.006884in}{2.063084in}}%
\pgfpathlineto{\pgfqpoint{3.006884in}{2.063084in}}%
\pgfpathlineto{\pgfqpoint{3.006884in}{2.066034in}}%
\pgfpathlineto{\pgfqpoint{3.011425in}{2.066034in}}%
\pgfpathlineto{\pgfqpoint{3.011425in}{2.063084in}}%
\pgfpathmoveto{\pgfqpoint{3.006884in}{2.066034in}}%
\pgfpathlineto{\pgfqpoint{3.006884in}{2.066034in}}%
\pgfpathlineto{\pgfqpoint{3.006884in}{2.068983in}}%
\pgfpathlineto{\pgfqpoint{3.011425in}{2.068983in}}%
\pgfpathlineto{\pgfqpoint{3.011425in}{2.066034in}}%
\pgfpathmoveto{\pgfqpoint{3.011425in}{2.057186in}}%
\pgfpathlineto{\pgfqpoint{3.011425in}{2.057186in}}%
\pgfpathlineto{\pgfqpoint{3.011425in}{2.060135in}}%
\pgfpathlineto{\pgfqpoint{3.015966in}{2.060135in}}%
\pgfpathlineto{\pgfqpoint{3.015966in}{2.057186in}}%
\pgfpathmoveto{\pgfqpoint{3.011425in}{2.060135in}}%
\pgfpathlineto{\pgfqpoint{3.011425in}{2.060135in}}%
\pgfpathlineto{\pgfqpoint{3.011425in}{2.063084in}}%
\pgfpathlineto{\pgfqpoint{3.015966in}{2.063084in}}%
\pgfpathlineto{\pgfqpoint{3.015966in}{2.060135in}}%
\pgfpathmoveto{\pgfqpoint{3.015966in}{2.057186in}}%
\pgfpathlineto{\pgfqpoint{3.015966in}{2.057186in}}%
\pgfpathlineto{\pgfqpoint{3.015966in}{2.060135in}}%
\pgfpathlineto{\pgfqpoint{3.020507in}{2.060135in}}%
\pgfpathlineto{\pgfqpoint{3.020507in}{2.057186in}}%
\pgfpathmoveto{\pgfqpoint{3.015966in}{2.060135in}}%
\pgfpathlineto{\pgfqpoint{3.015966in}{2.060135in}}%
\pgfpathlineto{\pgfqpoint{3.015966in}{2.063084in}}%
\pgfpathlineto{\pgfqpoint{3.020507in}{2.063084in}}%
\pgfpathlineto{\pgfqpoint{3.020507in}{2.060135in}}%
\pgfpathmoveto{\pgfqpoint{3.011425in}{2.063084in}}%
\pgfpathlineto{\pgfqpoint{3.011425in}{2.063084in}}%
\pgfpathlineto{\pgfqpoint{3.011425in}{2.066034in}}%
\pgfpathlineto{\pgfqpoint{3.015966in}{2.066034in}}%
\pgfpathlineto{\pgfqpoint{3.015966in}{2.063084in}}%
\pgfpathmoveto{\pgfqpoint{3.002343in}{2.068983in}}%
\pgfpathlineto{\pgfqpoint{3.002343in}{2.068983in}}%
\pgfpathlineto{\pgfqpoint{3.002343in}{2.071932in}}%
\pgfpathlineto{\pgfqpoint{3.006884in}{2.071932in}}%
\pgfpathlineto{\pgfqpoint{3.006884in}{2.068983in}}%
\pgfpathmoveto{\pgfqpoint{3.002343in}{2.071932in}}%
\pgfpathlineto{\pgfqpoint{3.002343in}{2.071932in}}%
\pgfpathlineto{\pgfqpoint{3.002343in}{2.074882in}}%
\pgfpathlineto{\pgfqpoint{3.006884in}{2.074882in}}%
\pgfpathlineto{\pgfqpoint{3.006884in}{2.071932in}}%
\pgfpathmoveto{\pgfqpoint{3.006884in}{2.068983in}}%
\pgfpathlineto{\pgfqpoint{3.006884in}{2.068983in}}%
\pgfpathlineto{\pgfqpoint{3.006884in}{2.071932in}}%
\pgfpathlineto{\pgfqpoint{3.011425in}{2.071932in}}%
\pgfpathlineto{\pgfqpoint{3.011425in}{2.068983in}}%
\pgfpathmoveto{\pgfqpoint{3.020507in}{2.057186in}}%
\pgfpathlineto{\pgfqpoint{3.020507in}{2.057186in}}%
\pgfpathlineto{\pgfqpoint{3.020507in}{2.060135in}}%
\pgfpathlineto{\pgfqpoint{3.025047in}{2.060135in}}%
\pgfpathlineto{\pgfqpoint{3.025047in}{2.057186in}}%
\pgfpathmoveto{\pgfqpoint{2.929689in}{2.122069in}}%
\pgfpathlineto{\pgfqpoint{2.929689in}{2.122069in}}%
\pgfpathlineto{\pgfqpoint{2.929689in}{2.125019in}}%
\pgfpathlineto{\pgfqpoint{2.934230in}{2.125019in}}%
\pgfpathlineto{\pgfqpoint{2.934230in}{2.122069in}}%
\pgfpathmoveto{\pgfqpoint{2.929689in}{2.125019in}}%
\pgfpathlineto{\pgfqpoint{2.929689in}{2.125019in}}%
\pgfpathlineto{\pgfqpoint{2.929689in}{2.127968in}}%
\pgfpathlineto{\pgfqpoint{2.934230in}{2.127968in}}%
\pgfpathlineto{\pgfqpoint{2.934230in}{2.125019in}}%
\pgfpathmoveto{\pgfqpoint{2.934230in}{2.122069in}}%
\pgfpathlineto{\pgfqpoint{2.934230in}{2.122069in}}%
\pgfpathlineto{\pgfqpoint{2.934230in}{2.125019in}}%
\pgfpathlineto{\pgfqpoint{2.938771in}{2.125019in}}%
\pgfpathlineto{\pgfqpoint{2.938771in}{2.122069in}}%
\pgfpathmoveto{\pgfqpoint{2.934230in}{2.125019in}}%
\pgfpathlineto{\pgfqpoint{2.934230in}{2.125019in}}%
\pgfpathlineto{\pgfqpoint{2.934230in}{2.127968in}}%
\pgfpathlineto{\pgfqpoint{2.938771in}{2.127968in}}%
\pgfpathlineto{\pgfqpoint{2.938771in}{2.125019in}}%
\pgfpathmoveto{\pgfqpoint{2.938771in}{2.116171in}}%
\pgfpathlineto{\pgfqpoint{2.938771in}{2.116171in}}%
\pgfpathlineto{\pgfqpoint{2.938771in}{2.119120in}}%
\pgfpathlineto{\pgfqpoint{2.943312in}{2.119120in}}%
\pgfpathlineto{\pgfqpoint{2.943312in}{2.116171in}}%
\pgfpathmoveto{\pgfqpoint{2.938771in}{2.119120in}}%
\pgfpathlineto{\pgfqpoint{2.938771in}{2.119120in}}%
\pgfpathlineto{\pgfqpoint{2.938771in}{2.122069in}}%
\pgfpathlineto{\pgfqpoint{2.943312in}{2.122069in}}%
\pgfpathlineto{\pgfqpoint{2.943312in}{2.119120in}}%
\pgfpathmoveto{\pgfqpoint{2.943312in}{2.116171in}}%
\pgfpathlineto{\pgfqpoint{2.943312in}{2.116171in}}%
\pgfpathlineto{\pgfqpoint{2.943312in}{2.119120in}}%
\pgfpathlineto{\pgfqpoint{2.947853in}{2.119120in}}%
\pgfpathlineto{\pgfqpoint{2.947853in}{2.116171in}}%
\pgfpathmoveto{\pgfqpoint{2.943312in}{2.119120in}}%
\pgfpathlineto{\pgfqpoint{2.943312in}{2.119120in}}%
\pgfpathlineto{\pgfqpoint{2.943312in}{2.122069in}}%
\pgfpathlineto{\pgfqpoint{2.947853in}{2.122069in}}%
\pgfpathlineto{\pgfqpoint{2.947853in}{2.119120in}}%
\pgfpathmoveto{\pgfqpoint{2.938771in}{2.122069in}}%
\pgfpathlineto{\pgfqpoint{2.938771in}{2.122069in}}%
\pgfpathlineto{\pgfqpoint{2.938771in}{2.125019in}}%
\pgfpathlineto{\pgfqpoint{2.943312in}{2.125019in}}%
\pgfpathlineto{\pgfqpoint{2.943312in}{2.122069in}}%
\pgfpathmoveto{\pgfqpoint{2.938771in}{2.125019in}}%
\pgfpathlineto{\pgfqpoint{2.938771in}{2.125019in}}%
\pgfpathlineto{\pgfqpoint{2.938771in}{2.127968in}}%
\pgfpathlineto{\pgfqpoint{2.943312in}{2.127968in}}%
\pgfpathlineto{\pgfqpoint{2.943312in}{2.125019in}}%
\pgfpathmoveto{\pgfqpoint{2.943312in}{2.122069in}}%
\pgfpathlineto{\pgfqpoint{2.943312in}{2.122069in}}%
\pgfpathlineto{\pgfqpoint{2.943312in}{2.125019in}}%
\pgfpathlineto{\pgfqpoint{2.947853in}{2.125019in}}%
\pgfpathlineto{\pgfqpoint{2.947853in}{2.122069in}}%
\pgfpathmoveto{\pgfqpoint{2.947853in}{2.110273in}}%
\pgfpathlineto{\pgfqpoint{2.947853in}{2.110273in}}%
\pgfpathlineto{\pgfqpoint{2.947853in}{2.113222in}}%
\pgfpathlineto{\pgfqpoint{2.952394in}{2.113222in}}%
\pgfpathlineto{\pgfqpoint{2.952394in}{2.110273in}}%
\pgfpathmoveto{\pgfqpoint{2.947853in}{2.113222in}}%
\pgfpathlineto{\pgfqpoint{2.947853in}{2.113222in}}%
\pgfpathlineto{\pgfqpoint{2.947853in}{2.116171in}}%
\pgfpathlineto{\pgfqpoint{2.952394in}{2.116171in}}%
\pgfpathlineto{\pgfqpoint{2.952394in}{2.113222in}}%
\pgfpathmoveto{\pgfqpoint{2.952394in}{2.110273in}}%
\pgfpathlineto{\pgfqpoint{2.952394in}{2.110273in}}%
\pgfpathlineto{\pgfqpoint{2.952394in}{2.113222in}}%
\pgfpathlineto{\pgfqpoint{2.956934in}{2.113222in}}%
\pgfpathlineto{\pgfqpoint{2.956934in}{2.110273in}}%
\pgfpathmoveto{\pgfqpoint{2.952394in}{2.113222in}}%
\pgfpathlineto{\pgfqpoint{2.952394in}{2.113222in}}%
\pgfpathlineto{\pgfqpoint{2.952394in}{2.116171in}}%
\pgfpathlineto{\pgfqpoint{2.956934in}{2.116171in}}%
\pgfpathlineto{\pgfqpoint{2.956934in}{2.113222in}}%
\pgfpathmoveto{\pgfqpoint{2.956934in}{2.104374in}}%
\pgfpathlineto{\pgfqpoint{2.956934in}{2.104374in}}%
\pgfpathlineto{\pgfqpoint{2.956934in}{2.107323in}}%
\pgfpathlineto{\pgfqpoint{2.961475in}{2.107323in}}%
\pgfpathlineto{\pgfqpoint{2.961475in}{2.104374in}}%
\pgfpathmoveto{\pgfqpoint{2.956934in}{2.107323in}}%
\pgfpathlineto{\pgfqpoint{2.956934in}{2.107323in}}%
\pgfpathlineto{\pgfqpoint{2.956934in}{2.110273in}}%
\pgfpathlineto{\pgfqpoint{2.961475in}{2.110273in}}%
\pgfpathlineto{\pgfqpoint{2.961475in}{2.107323in}}%
\pgfpathmoveto{\pgfqpoint{2.961475in}{2.104374in}}%
\pgfpathlineto{\pgfqpoint{2.961475in}{2.104374in}}%
\pgfpathlineto{\pgfqpoint{2.961475in}{2.107323in}}%
\pgfpathlineto{\pgfqpoint{2.966016in}{2.107323in}}%
\pgfpathlineto{\pgfqpoint{2.966016in}{2.104374in}}%
\pgfpathmoveto{\pgfqpoint{2.961475in}{2.107323in}}%
\pgfpathlineto{\pgfqpoint{2.961475in}{2.107323in}}%
\pgfpathlineto{\pgfqpoint{2.961475in}{2.110273in}}%
\pgfpathlineto{\pgfqpoint{2.966016in}{2.110273in}}%
\pgfpathlineto{\pgfqpoint{2.966016in}{2.107323in}}%
\pgfpathmoveto{\pgfqpoint{2.956934in}{2.110273in}}%
\pgfpathlineto{\pgfqpoint{2.956934in}{2.110273in}}%
\pgfpathlineto{\pgfqpoint{2.956934in}{2.113222in}}%
\pgfpathlineto{\pgfqpoint{2.961475in}{2.113222in}}%
\pgfpathlineto{\pgfqpoint{2.961475in}{2.110273in}}%
\pgfpathmoveto{\pgfqpoint{2.947853in}{2.116171in}}%
\pgfpathlineto{\pgfqpoint{2.947853in}{2.116171in}}%
\pgfpathlineto{\pgfqpoint{2.947853in}{2.119120in}}%
\pgfpathlineto{\pgfqpoint{2.952394in}{2.119120in}}%
\pgfpathlineto{\pgfqpoint{2.952394in}{2.116171in}}%
\pgfpathmoveto{\pgfqpoint{2.947853in}{2.119120in}}%
\pgfpathlineto{\pgfqpoint{2.947853in}{2.119120in}}%
\pgfpathlineto{\pgfqpoint{2.947853in}{2.122069in}}%
\pgfpathlineto{\pgfqpoint{2.952394in}{2.122069in}}%
\pgfpathlineto{\pgfqpoint{2.952394in}{2.119120in}}%
\pgfpathmoveto{\pgfqpoint{2.929689in}{2.127968in}}%
\pgfpathlineto{\pgfqpoint{2.929689in}{2.127968in}}%
\pgfpathlineto{\pgfqpoint{2.929689in}{2.130917in}}%
\pgfpathlineto{\pgfqpoint{2.934230in}{2.130917in}}%
\pgfpathlineto{\pgfqpoint{2.934230in}{2.127968in}}%
\pgfpathmoveto{\pgfqpoint{2.929689in}{2.130917in}}%
\pgfpathlineto{\pgfqpoint{2.929689in}{2.130917in}}%
\pgfpathlineto{\pgfqpoint{2.929689in}{2.133866in}}%
\pgfpathlineto{\pgfqpoint{2.934230in}{2.133866in}}%
\pgfpathlineto{\pgfqpoint{2.934230in}{2.130917in}}%
\pgfpathmoveto{\pgfqpoint{2.934230in}{2.127968in}}%
\pgfpathlineto{\pgfqpoint{2.934230in}{2.127968in}}%
\pgfpathlineto{\pgfqpoint{2.934230in}{2.130917in}}%
\pgfpathlineto{\pgfqpoint{2.938771in}{2.130917in}}%
\pgfpathlineto{\pgfqpoint{2.938771in}{2.127968in}}%
\pgfpathmoveto{\pgfqpoint{2.934230in}{2.130917in}}%
\pgfpathlineto{\pgfqpoint{2.934230in}{2.130917in}}%
\pgfpathlineto{\pgfqpoint{2.934230in}{2.133866in}}%
\pgfpathlineto{\pgfqpoint{2.938771in}{2.133866in}}%
\pgfpathlineto{\pgfqpoint{2.938771in}{2.130917in}}%
\pgfpathmoveto{\pgfqpoint{2.929689in}{2.133866in}}%
\pgfpathlineto{\pgfqpoint{2.929689in}{2.133866in}}%
\pgfpathlineto{\pgfqpoint{2.929689in}{2.136816in}}%
\pgfpathlineto{\pgfqpoint{2.934230in}{2.136816in}}%
\pgfpathlineto{\pgfqpoint{2.934230in}{2.133866in}}%
\pgfpathmoveto{\pgfqpoint{2.966016in}{2.104374in}}%
\pgfpathlineto{\pgfqpoint{2.966016in}{2.104374in}}%
\pgfpathlineto{\pgfqpoint{2.966016in}{2.107323in}}%
\pgfpathlineto{\pgfqpoint{2.970557in}{2.107323in}}%
\pgfpathlineto{\pgfqpoint{2.970557in}{2.104374in}}%
\pgfpathmoveto{\pgfqpoint{3.174900in}{1.909729in}}%
\pgfpathlineto{\pgfqpoint{3.174900in}{1.909729in}}%
\pgfpathlineto{\pgfqpoint{3.174900in}{1.912678in}}%
\pgfpathlineto{\pgfqpoint{3.179441in}{1.912678in}}%
\pgfpathlineto{\pgfqpoint{3.179441in}{1.909729in}}%
\pgfpathmoveto{\pgfqpoint{3.174900in}{1.912678in}}%
\pgfpathlineto{\pgfqpoint{3.174900in}{1.912678in}}%
\pgfpathlineto{\pgfqpoint{3.174900in}{1.915628in}}%
\pgfpathlineto{\pgfqpoint{3.179441in}{1.915628in}}%
\pgfpathlineto{\pgfqpoint{3.179441in}{1.912678in}}%
\pgfpathmoveto{\pgfqpoint{3.179441in}{1.909729in}}%
\pgfpathlineto{\pgfqpoint{3.179441in}{1.909729in}}%
\pgfpathlineto{\pgfqpoint{3.179441in}{1.912678in}}%
\pgfpathlineto{\pgfqpoint{3.183982in}{1.912678in}}%
\pgfpathlineto{\pgfqpoint{3.183982in}{1.909729in}}%
\pgfpathmoveto{\pgfqpoint{3.179441in}{1.912678in}}%
\pgfpathlineto{\pgfqpoint{3.179441in}{1.912678in}}%
\pgfpathlineto{\pgfqpoint{3.179441in}{1.915628in}}%
\pgfpathlineto{\pgfqpoint{3.183982in}{1.915628in}}%
\pgfpathlineto{\pgfqpoint{3.183982in}{1.912678in}}%
\pgfpathmoveto{\pgfqpoint{3.211229in}{1.880235in}}%
\pgfpathlineto{\pgfqpoint{3.211229in}{1.880235in}}%
\pgfpathlineto{\pgfqpoint{3.211229in}{1.883185in}}%
\pgfpathlineto{\pgfqpoint{3.215770in}{1.883185in}}%
\pgfpathlineto{\pgfqpoint{3.215770in}{1.880235in}}%
\pgfpathmoveto{\pgfqpoint{3.211229in}{1.883185in}}%
\pgfpathlineto{\pgfqpoint{3.211229in}{1.883185in}}%
\pgfpathlineto{\pgfqpoint{3.211229in}{1.886134in}}%
\pgfpathlineto{\pgfqpoint{3.215770in}{1.886134in}}%
\pgfpathlineto{\pgfqpoint{3.215770in}{1.883185in}}%
\pgfpathmoveto{\pgfqpoint{3.215770in}{1.880235in}}%
\pgfpathlineto{\pgfqpoint{3.215770in}{1.880235in}}%
\pgfpathlineto{\pgfqpoint{3.215770in}{1.883185in}}%
\pgfpathlineto{\pgfqpoint{3.220311in}{1.883185in}}%
\pgfpathlineto{\pgfqpoint{3.220311in}{1.880235in}}%
\pgfpathmoveto{\pgfqpoint{3.215770in}{1.883185in}}%
\pgfpathlineto{\pgfqpoint{3.215770in}{1.883185in}}%
\pgfpathlineto{\pgfqpoint{3.215770in}{1.886134in}}%
\pgfpathlineto{\pgfqpoint{3.220311in}{1.886134in}}%
\pgfpathlineto{\pgfqpoint{3.220311in}{1.883185in}}%
\pgfpathmoveto{\pgfqpoint{3.211229in}{1.886134in}}%
\pgfpathlineto{\pgfqpoint{3.211229in}{1.886134in}}%
\pgfpathlineto{\pgfqpoint{3.211229in}{1.889084in}}%
\pgfpathlineto{\pgfqpoint{3.215770in}{1.889084in}}%
\pgfpathlineto{\pgfqpoint{3.215770in}{1.886134in}}%
\pgfpathmoveto{\pgfqpoint{3.211229in}{1.889084in}}%
\pgfpathlineto{\pgfqpoint{3.211229in}{1.889084in}}%
\pgfpathlineto{\pgfqpoint{3.211229in}{1.892033in}}%
\pgfpathlineto{\pgfqpoint{3.215770in}{1.892033in}}%
\pgfpathlineto{\pgfqpoint{3.215770in}{1.889084in}}%
\pgfpathmoveto{\pgfqpoint{3.215770in}{1.886134in}}%
\pgfpathlineto{\pgfqpoint{3.215770in}{1.886134in}}%
\pgfpathlineto{\pgfqpoint{3.215770in}{1.889084in}}%
\pgfpathlineto{\pgfqpoint{3.220311in}{1.889084in}}%
\pgfpathlineto{\pgfqpoint{3.220311in}{1.886134in}}%
\pgfpathmoveto{\pgfqpoint{3.193064in}{1.897932in}}%
\pgfpathlineto{\pgfqpoint{3.193064in}{1.897932in}}%
\pgfpathlineto{\pgfqpoint{3.193064in}{1.900881in}}%
\pgfpathlineto{\pgfqpoint{3.197605in}{1.900881in}}%
\pgfpathlineto{\pgfqpoint{3.197605in}{1.897932in}}%
\pgfpathmoveto{\pgfqpoint{3.193064in}{1.900881in}}%
\pgfpathlineto{\pgfqpoint{3.193064in}{1.900881in}}%
\pgfpathlineto{\pgfqpoint{3.193064in}{1.903830in}}%
\pgfpathlineto{\pgfqpoint{3.197605in}{1.903830in}}%
\pgfpathlineto{\pgfqpoint{3.197605in}{1.900881in}}%
\pgfpathmoveto{\pgfqpoint{3.197605in}{1.897932in}}%
\pgfpathlineto{\pgfqpoint{3.197605in}{1.897932in}}%
\pgfpathlineto{\pgfqpoint{3.197605in}{1.900881in}}%
\pgfpathlineto{\pgfqpoint{3.202146in}{1.900881in}}%
\pgfpathlineto{\pgfqpoint{3.202146in}{1.897932in}}%
\pgfpathmoveto{\pgfqpoint{3.197605in}{1.900881in}}%
\pgfpathlineto{\pgfqpoint{3.197605in}{1.900881in}}%
\pgfpathlineto{\pgfqpoint{3.197605in}{1.903830in}}%
\pgfpathlineto{\pgfqpoint{3.202146in}{1.903830in}}%
\pgfpathlineto{\pgfqpoint{3.202146in}{1.900881in}}%
\pgfpathmoveto{\pgfqpoint{3.183982in}{1.903830in}}%
\pgfpathlineto{\pgfqpoint{3.183982in}{1.903830in}}%
\pgfpathlineto{\pgfqpoint{3.183982in}{1.906780in}}%
\pgfpathlineto{\pgfqpoint{3.188523in}{1.906780in}}%
\pgfpathlineto{\pgfqpoint{3.188523in}{1.903830in}}%
\pgfpathmoveto{\pgfqpoint{3.183982in}{1.906780in}}%
\pgfpathlineto{\pgfqpoint{3.183982in}{1.906780in}}%
\pgfpathlineto{\pgfqpoint{3.183982in}{1.909729in}}%
\pgfpathlineto{\pgfqpoint{3.188523in}{1.909729in}}%
\pgfpathlineto{\pgfqpoint{3.188523in}{1.906780in}}%
\pgfpathmoveto{\pgfqpoint{3.188523in}{1.903830in}}%
\pgfpathlineto{\pgfqpoint{3.188523in}{1.903830in}}%
\pgfpathlineto{\pgfqpoint{3.188523in}{1.906780in}}%
\pgfpathlineto{\pgfqpoint{3.193064in}{1.906780in}}%
\pgfpathlineto{\pgfqpoint{3.193064in}{1.903830in}}%
\pgfpathmoveto{\pgfqpoint{3.188523in}{1.906780in}}%
\pgfpathlineto{\pgfqpoint{3.188523in}{1.906780in}}%
\pgfpathlineto{\pgfqpoint{3.188523in}{1.909729in}}%
\pgfpathlineto{\pgfqpoint{3.193064in}{1.909729in}}%
\pgfpathlineto{\pgfqpoint{3.193064in}{1.906780in}}%
\pgfpathmoveto{\pgfqpoint{3.183982in}{1.909729in}}%
\pgfpathlineto{\pgfqpoint{3.183982in}{1.909729in}}%
\pgfpathlineto{\pgfqpoint{3.183982in}{1.912678in}}%
\pgfpathlineto{\pgfqpoint{3.188523in}{1.912678in}}%
\pgfpathlineto{\pgfqpoint{3.188523in}{1.909729in}}%
\pgfpathmoveto{\pgfqpoint{3.183982in}{1.912678in}}%
\pgfpathlineto{\pgfqpoint{3.183982in}{1.912678in}}%
\pgfpathlineto{\pgfqpoint{3.183982in}{1.915628in}}%
\pgfpathlineto{\pgfqpoint{3.188523in}{1.915628in}}%
\pgfpathlineto{\pgfqpoint{3.188523in}{1.912678in}}%
\pgfpathmoveto{\pgfqpoint{3.188523in}{1.909729in}}%
\pgfpathlineto{\pgfqpoint{3.188523in}{1.909729in}}%
\pgfpathlineto{\pgfqpoint{3.188523in}{1.912678in}}%
\pgfpathlineto{\pgfqpoint{3.193064in}{1.912678in}}%
\pgfpathlineto{\pgfqpoint{3.193064in}{1.909729in}}%
\pgfpathmoveto{\pgfqpoint{3.193064in}{1.903830in}}%
\pgfpathlineto{\pgfqpoint{3.193064in}{1.903830in}}%
\pgfpathlineto{\pgfqpoint{3.193064in}{1.906780in}}%
\pgfpathlineto{\pgfqpoint{3.197605in}{1.906780in}}%
\pgfpathlineto{\pgfqpoint{3.197605in}{1.903830in}}%
\pgfpathmoveto{\pgfqpoint{3.193064in}{1.906780in}}%
\pgfpathlineto{\pgfqpoint{3.193064in}{1.906780in}}%
\pgfpathlineto{\pgfqpoint{3.193064in}{1.909729in}}%
\pgfpathlineto{\pgfqpoint{3.197605in}{1.909729in}}%
\pgfpathlineto{\pgfqpoint{3.197605in}{1.906780in}}%
\pgfpathmoveto{\pgfqpoint{3.202146in}{1.892033in}}%
\pgfpathlineto{\pgfqpoint{3.202146in}{1.892033in}}%
\pgfpathlineto{\pgfqpoint{3.202146in}{1.894982in}}%
\pgfpathlineto{\pgfqpoint{3.206688in}{1.894982in}}%
\pgfpathlineto{\pgfqpoint{3.206688in}{1.892033in}}%
\pgfpathmoveto{\pgfqpoint{3.202146in}{1.894982in}}%
\pgfpathlineto{\pgfqpoint{3.202146in}{1.894982in}}%
\pgfpathlineto{\pgfqpoint{3.202146in}{1.897932in}}%
\pgfpathlineto{\pgfqpoint{3.206688in}{1.897932in}}%
\pgfpathlineto{\pgfqpoint{3.206688in}{1.894982in}}%
\pgfpathmoveto{\pgfqpoint{3.206688in}{1.892033in}}%
\pgfpathlineto{\pgfqpoint{3.206688in}{1.892033in}}%
\pgfpathlineto{\pgfqpoint{3.206688in}{1.894982in}}%
\pgfpathlineto{\pgfqpoint{3.211229in}{1.894982in}}%
\pgfpathlineto{\pgfqpoint{3.211229in}{1.892033in}}%
\pgfpathmoveto{\pgfqpoint{3.206688in}{1.894982in}}%
\pgfpathlineto{\pgfqpoint{3.206688in}{1.894982in}}%
\pgfpathlineto{\pgfqpoint{3.206688in}{1.897932in}}%
\pgfpathlineto{\pgfqpoint{3.211229in}{1.897932in}}%
\pgfpathlineto{\pgfqpoint{3.211229in}{1.894982in}}%
\pgfpathmoveto{\pgfqpoint{3.202146in}{1.897932in}}%
\pgfpathlineto{\pgfqpoint{3.202146in}{1.897932in}}%
\pgfpathlineto{\pgfqpoint{3.202146in}{1.900881in}}%
\pgfpathlineto{\pgfqpoint{3.206688in}{1.900881in}}%
\pgfpathlineto{\pgfqpoint{3.206688in}{1.897932in}}%
\pgfpathmoveto{\pgfqpoint{3.211229in}{1.892033in}}%
\pgfpathlineto{\pgfqpoint{3.211229in}{1.892033in}}%
\pgfpathlineto{\pgfqpoint{3.211229in}{1.894982in}}%
\pgfpathlineto{\pgfqpoint{3.215770in}{1.894982in}}%
\pgfpathlineto{\pgfqpoint{3.215770in}{1.892033in}}%
\pgfpathmoveto{\pgfqpoint{3.138572in}{1.945118in}}%
\pgfpathlineto{\pgfqpoint{3.138572in}{1.945118in}}%
\pgfpathlineto{\pgfqpoint{3.138572in}{1.948067in}}%
\pgfpathlineto{\pgfqpoint{3.143113in}{1.948067in}}%
\pgfpathlineto{\pgfqpoint{3.143113in}{1.945118in}}%
\pgfpathmoveto{\pgfqpoint{3.138572in}{1.948067in}}%
\pgfpathlineto{\pgfqpoint{3.138572in}{1.948067in}}%
\pgfpathlineto{\pgfqpoint{3.138572in}{1.951016in}}%
\pgfpathlineto{\pgfqpoint{3.143113in}{1.951016in}}%
\pgfpathlineto{\pgfqpoint{3.143113in}{1.948067in}}%
\pgfpathmoveto{\pgfqpoint{3.143113in}{1.945118in}}%
\pgfpathlineto{\pgfqpoint{3.143113in}{1.945118in}}%
\pgfpathlineto{\pgfqpoint{3.143113in}{1.948067in}}%
\pgfpathlineto{\pgfqpoint{3.147654in}{1.948067in}}%
\pgfpathlineto{\pgfqpoint{3.147654in}{1.945118in}}%
\pgfpathmoveto{\pgfqpoint{3.143113in}{1.948067in}}%
\pgfpathlineto{\pgfqpoint{3.143113in}{1.948067in}}%
\pgfpathlineto{\pgfqpoint{3.143113in}{1.951016in}}%
\pgfpathlineto{\pgfqpoint{3.147654in}{1.951016in}}%
\pgfpathlineto{\pgfqpoint{3.147654in}{1.948067in}}%
\pgfpathmoveto{\pgfqpoint{3.129490in}{1.951016in}}%
\pgfpathlineto{\pgfqpoint{3.129490in}{1.951016in}}%
\pgfpathlineto{\pgfqpoint{3.129490in}{1.953966in}}%
\pgfpathlineto{\pgfqpoint{3.134031in}{1.953966in}}%
\pgfpathlineto{\pgfqpoint{3.134031in}{1.951016in}}%
\pgfpathmoveto{\pgfqpoint{3.129490in}{1.953966in}}%
\pgfpathlineto{\pgfqpoint{3.129490in}{1.953966in}}%
\pgfpathlineto{\pgfqpoint{3.129490in}{1.956915in}}%
\pgfpathlineto{\pgfqpoint{3.134031in}{1.956915in}}%
\pgfpathlineto{\pgfqpoint{3.134031in}{1.953966in}}%
\pgfpathmoveto{\pgfqpoint{3.134031in}{1.951016in}}%
\pgfpathlineto{\pgfqpoint{3.134031in}{1.951016in}}%
\pgfpathlineto{\pgfqpoint{3.134031in}{1.953966in}}%
\pgfpathlineto{\pgfqpoint{3.138572in}{1.953966in}}%
\pgfpathlineto{\pgfqpoint{3.138572in}{1.951016in}}%
\pgfpathmoveto{\pgfqpoint{3.134031in}{1.953966in}}%
\pgfpathlineto{\pgfqpoint{3.134031in}{1.953966in}}%
\pgfpathlineto{\pgfqpoint{3.134031in}{1.956915in}}%
\pgfpathlineto{\pgfqpoint{3.138572in}{1.956915in}}%
\pgfpathlineto{\pgfqpoint{3.138572in}{1.953966in}}%
\pgfpathmoveto{\pgfqpoint{3.129490in}{1.956915in}}%
\pgfpathlineto{\pgfqpoint{3.129490in}{1.956915in}}%
\pgfpathlineto{\pgfqpoint{3.129490in}{1.959864in}}%
\pgfpathlineto{\pgfqpoint{3.134031in}{1.959864in}}%
\pgfpathlineto{\pgfqpoint{3.134031in}{1.956915in}}%
\pgfpathmoveto{\pgfqpoint{3.129490in}{1.959864in}}%
\pgfpathlineto{\pgfqpoint{3.129490in}{1.959864in}}%
\pgfpathlineto{\pgfqpoint{3.129490in}{1.962813in}}%
\pgfpathlineto{\pgfqpoint{3.134031in}{1.962813in}}%
\pgfpathlineto{\pgfqpoint{3.134031in}{1.959864in}}%
\pgfpathmoveto{\pgfqpoint{3.134031in}{1.956915in}}%
\pgfpathlineto{\pgfqpoint{3.134031in}{1.956915in}}%
\pgfpathlineto{\pgfqpoint{3.134031in}{1.959864in}}%
\pgfpathlineto{\pgfqpoint{3.138572in}{1.959864in}}%
\pgfpathlineto{\pgfqpoint{3.138572in}{1.956915in}}%
\pgfpathmoveto{\pgfqpoint{3.138572in}{1.951016in}}%
\pgfpathlineto{\pgfqpoint{3.138572in}{1.951016in}}%
\pgfpathlineto{\pgfqpoint{3.138572in}{1.953966in}}%
\pgfpathlineto{\pgfqpoint{3.143113in}{1.953966in}}%
\pgfpathlineto{\pgfqpoint{3.143113in}{1.951016in}}%
\pgfpathmoveto{\pgfqpoint{3.138572in}{1.953966in}}%
\pgfpathlineto{\pgfqpoint{3.138572in}{1.953966in}}%
\pgfpathlineto{\pgfqpoint{3.138572in}{1.956915in}}%
\pgfpathlineto{\pgfqpoint{3.143113in}{1.956915in}}%
\pgfpathlineto{\pgfqpoint{3.143113in}{1.953966in}}%
\pgfpathmoveto{\pgfqpoint{3.143113in}{1.951016in}}%
\pgfpathlineto{\pgfqpoint{3.143113in}{1.951016in}}%
\pgfpathlineto{\pgfqpoint{3.143113in}{1.953966in}}%
\pgfpathlineto{\pgfqpoint{3.147654in}{1.953966in}}%
\pgfpathlineto{\pgfqpoint{3.147654in}{1.951016in}}%
\pgfpathmoveto{\pgfqpoint{3.102243in}{1.974609in}}%
\pgfpathlineto{\pgfqpoint{3.102243in}{1.974609in}}%
\pgfpathlineto{\pgfqpoint{3.102243in}{1.977558in}}%
\pgfpathlineto{\pgfqpoint{3.106784in}{1.977558in}}%
\pgfpathlineto{\pgfqpoint{3.106784in}{1.974609in}}%
\pgfpathmoveto{\pgfqpoint{3.102243in}{1.977558in}}%
\pgfpathlineto{\pgfqpoint{3.102243in}{1.977558in}}%
\pgfpathlineto{\pgfqpoint{3.102243in}{1.980507in}}%
\pgfpathlineto{\pgfqpoint{3.106784in}{1.980507in}}%
\pgfpathlineto{\pgfqpoint{3.106784in}{1.977558in}}%
\pgfpathmoveto{\pgfqpoint{3.106784in}{1.974609in}}%
\pgfpathlineto{\pgfqpoint{3.106784in}{1.974609in}}%
\pgfpathlineto{\pgfqpoint{3.106784in}{1.977558in}}%
\pgfpathlineto{\pgfqpoint{3.111325in}{1.977558in}}%
\pgfpathlineto{\pgfqpoint{3.111325in}{1.974609in}}%
\pgfpathmoveto{\pgfqpoint{3.106784in}{1.977558in}}%
\pgfpathlineto{\pgfqpoint{3.106784in}{1.977558in}}%
\pgfpathlineto{\pgfqpoint{3.106784in}{1.980507in}}%
\pgfpathlineto{\pgfqpoint{3.111325in}{1.980507in}}%
\pgfpathlineto{\pgfqpoint{3.111325in}{1.977558in}}%
\pgfpathmoveto{\pgfqpoint{3.102243in}{1.980507in}}%
\pgfpathlineto{\pgfqpoint{3.102243in}{1.980507in}}%
\pgfpathlineto{\pgfqpoint{3.102243in}{1.983456in}}%
\pgfpathlineto{\pgfqpoint{3.106784in}{1.983456in}}%
\pgfpathlineto{\pgfqpoint{3.106784in}{1.980507in}}%
\pgfpathmoveto{\pgfqpoint{3.102243in}{1.983456in}}%
\pgfpathlineto{\pgfqpoint{3.102243in}{1.983456in}}%
\pgfpathlineto{\pgfqpoint{3.102243in}{1.986405in}}%
\pgfpathlineto{\pgfqpoint{3.106784in}{1.986405in}}%
\pgfpathlineto{\pgfqpoint{3.106784in}{1.983456in}}%
\pgfpathmoveto{\pgfqpoint{3.106784in}{1.980507in}}%
\pgfpathlineto{\pgfqpoint{3.106784in}{1.980507in}}%
\pgfpathlineto{\pgfqpoint{3.106784in}{1.983456in}}%
\pgfpathlineto{\pgfqpoint{3.111325in}{1.983456in}}%
\pgfpathlineto{\pgfqpoint{3.111325in}{1.980507in}}%
\pgfpathmoveto{\pgfqpoint{3.084079in}{1.992303in}}%
\pgfpathlineto{\pgfqpoint{3.084079in}{1.992303in}}%
\pgfpathlineto{\pgfqpoint{3.084079in}{1.995252in}}%
\pgfpathlineto{\pgfqpoint{3.088620in}{1.995252in}}%
\pgfpathlineto{\pgfqpoint{3.088620in}{1.992303in}}%
\pgfpathmoveto{\pgfqpoint{3.084079in}{1.995252in}}%
\pgfpathlineto{\pgfqpoint{3.084079in}{1.995252in}}%
\pgfpathlineto{\pgfqpoint{3.084079in}{1.998202in}}%
\pgfpathlineto{\pgfqpoint{3.088620in}{1.998202in}}%
\pgfpathlineto{\pgfqpoint{3.088620in}{1.995252in}}%
\pgfpathmoveto{\pgfqpoint{3.088620in}{1.992303in}}%
\pgfpathlineto{\pgfqpoint{3.088620in}{1.992303in}}%
\pgfpathlineto{\pgfqpoint{3.088620in}{1.995252in}}%
\pgfpathlineto{\pgfqpoint{3.093161in}{1.995252in}}%
\pgfpathlineto{\pgfqpoint{3.093161in}{1.992303in}}%
\pgfpathmoveto{\pgfqpoint{3.088620in}{1.995252in}}%
\pgfpathlineto{\pgfqpoint{3.088620in}{1.995252in}}%
\pgfpathlineto{\pgfqpoint{3.088620in}{1.998202in}}%
\pgfpathlineto{\pgfqpoint{3.093161in}{1.998202in}}%
\pgfpathlineto{\pgfqpoint{3.093161in}{1.995252in}}%
\pgfpathmoveto{\pgfqpoint{3.074997in}{1.998202in}}%
\pgfpathlineto{\pgfqpoint{3.074997in}{1.998202in}}%
\pgfpathlineto{\pgfqpoint{3.074997in}{2.001151in}}%
\pgfpathlineto{\pgfqpoint{3.079538in}{2.001151in}}%
\pgfpathlineto{\pgfqpoint{3.079538in}{1.998202in}}%
\pgfpathmoveto{\pgfqpoint{3.074997in}{2.001151in}}%
\pgfpathlineto{\pgfqpoint{3.074997in}{2.001151in}}%
\pgfpathlineto{\pgfqpoint{3.074997in}{2.004100in}}%
\pgfpathlineto{\pgfqpoint{3.079538in}{2.004100in}}%
\pgfpathlineto{\pgfqpoint{3.079538in}{2.001151in}}%
\pgfpathmoveto{\pgfqpoint{3.079538in}{1.998202in}}%
\pgfpathlineto{\pgfqpoint{3.079538in}{1.998202in}}%
\pgfpathlineto{\pgfqpoint{3.079538in}{2.001151in}}%
\pgfpathlineto{\pgfqpoint{3.084079in}{2.001151in}}%
\pgfpathlineto{\pgfqpoint{3.084079in}{1.998202in}}%
\pgfpathmoveto{\pgfqpoint{3.079538in}{2.001151in}}%
\pgfpathlineto{\pgfqpoint{3.079538in}{2.001151in}}%
\pgfpathlineto{\pgfqpoint{3.079538in}{2.004100in}}%
\pgfpathlineto{\pgfqpoint{3.084079in}{2.004100in}}%
\pgfpathlineto{\pgfqpoint{3.084079in}{2.001151in}}%
\pgfpathmoveto{\pgfqpoint{3.074997in}{2.004100in}}%
\pgfpathlineto{\pgfqpoint{3.074997in}{2.004100in}}%
\pgfpathlineto{\pgfqpoint{3.074997in}{2.007049in}}%
\pgfpathlineto{\pgfqpoint{3.079538in}{2.007049in}}%
\pgfpathlineto{\pgfqpoint{3.079538in}{2.004100in}}%
\pgfpathmoveto{\pgfqpoint{3.074997in}{2.007049in}}%
\pgfpathlineto{\pgfqpoint{3.074997in}{2.007049in}}%
\pgfpathlineto{\pgfqpoint{3.074997in}{2.009998in}}%
\pgfpathlineto{\pgfqpoint{3.079538in}{2.009998in}}%
\pgfpathlineto{\pgfqpoint{3.079538in}{2.007049in}}%
\pgfpathmoveto{\pgfqpoint{3.079538in}{2.004100in}}%
\pgfpathlineto{\pgfqpoint{3.079538in}{2.004100in}}%
\pgfpathlineto{\pgfqpoint{3.079538in}{2.007049in}}%
\pgfpathlineto{\pgfqpoint{3.084079in}{2.007049in}}%
\pgfpathlineto{\pgfqpoint{3.084079in}{2.004100in}}%
\pgfpathmoveto{\pgfqpoint{3.084079in}{1.998202in}}%
\pgfpathlineto{\pgfqpoint{3.084079in}{1.998202in}}%
\pgfpathlineto{\pgfqpoint{3.084079in}{2.001151in}}%
\pgfpathlineto{\pgfqpoint{3.088620in}{2.001151in}}%
\pgfpathlineto{\pgfqpoint{3.088620in}{1.998202in}}%
\pgfpathmoveto{\pgfqpoint{3.084079in}{2.001151in}}%
\pgfpathlineto{\pgfqpoint{3.084079in}{2.001151in}}%
\pgfpathlineto{\pgfqpoint{3.084079in}{2.004100in}}%
\pgfpathlineto{\pgfqpoint{3.088620in}{2.004100in}}%
\pgfpathlineto{\pgfqpoint{3.088620in}{2.001151in}}%
\pgfpathmoveto{\pgfqpoint{3.088620in}{1.998202in}}%
\pgfpathlineto{\pgfqpoint{3.088620in}{1.998202in}}%
\pgfpathlineto{\pgfqpoint{3.088620in}{2.001151in}}%
\pgfpathlineto{\pgfqpoint{3.093161in}{2.001151in}}%
\pgfpathlineto{\pgfqpoint{3.093161in}{1.998202in}}%
\pgfpathmoveto{\pgfqpoint{3.093161in}{1.986405in}}%
\pgfpathlineto{\pgfqpoint{3.093161in}{1.986405in}}%
\pgfpathlineto{\pgfqpoint{3.093161in}{1.989354in}}%
\pgfpathlineto{\pgfqpoint{3.097702in}{1.989354in}}%
\pgfpathlineto{\pgfqpoint{3.097702in}{1.986405in}}%
\pgfpathmoveto{\pgfqpoint{3.093161in}{1.989354in}}%
\pgfpathlineto{\pgfqpoint{3.093161in}{1.989354in}}%
\pgfpathlineto{\pgfqpoint{3.093161in}{1.992303in}}%
\pgfpathlineto{\pgfqpoint{3.097702in}{1.992303in}}%
\pgfpathlineto{\pgfqpoint{3.097702in}{1.989354in}}%
\pgfpathmoveto{\pgfqpoint{3.097702in}{1.986405in}}%
\pgfpathlineto{\pgfqpoint{3.097702in}{1.986405in}}%
\pgfpathlineto{\pgfqpoint{3.097702in}{1.989354in}}%
\pgfpathlineto{\pgfqpoint{3.102243in}{1.989354in}}%
\pgfpathlineto{\pgfqpoint{3.102243in}{1.986405in}}%
\pgfpathmoveto{\pgfqpoint{3.097702in}{1.989354in}}%
\pgfpathlineto{\pgfqpoint{3.097702in}{1.989354in}}%
\pgfpathlineto{\pgfqpoint{3.097702in}{1.992303in}}%
\pgfpathlineto{\pgfqpoint{3.102243in}{1.992303in}}%
\pgfpathlineto{\pgfqpoint{3.102243in}{1.989354in}}%
\pgfpathmoveto{\pgfqpoint{3.093161in}{1.992303in}}%
\pgfpathlineto{\pgfqpoint{3.093161in}{1.992303in}}%
\pgfpathlineto{\pgfqpoint{3.093161in}{1.995252in}}%
\pgfpathlineto{\pgfqpoint{3.097702in}{1.995252in}}%
\pgfpathlineto{\pgfqpoint{3.097702in}{1.992303in}}%
\pgfpathmoveto{\pgfqpoint{3.102243in}{1.986405in}}%
\pgfpathlineto{\pgfqpoint{3.102243in}{1.986405in}}%
\pgfpathlineto{\pgfqpoint{3.102243in}{1.989354in}}%
\pgfpathlineto{\pgfqpoint{3.106784in}{1.989354in}}%
\pgfpathlineto{\pgfqpoint{3.106784in}{1.986405in}}%
\pgfpathmoveto{\pgfqpoint{3.111325in}{1.968711in}}%
\pgfpathlineto{\pgfqpoint{3.111325in}{1.968711in}}%
\pgfpathlineto{\pgfqpoint{3.111325in}{1.971660in}}%
\pgfpathlineto{\pgfqpoint{3.115866in}{1.971660in}}%
\pgfpathlineto{\pgfqpoint{3.115866in}{1.968711in}}%
\pgfpathmoveto{\pgfqpoint{3.111325in}{1.971660in}}%
\pgfpathlineto{\pgfqpoint{3.111325in}{1.971660in}}%
\pgfpathlineto{\pgfqpoint{3.111325in}{1.974609in}}%
\pgfpathlineto{\pgfqpoint{3.115866in}{1.974609in}}%
\pgfpathlineto{\pgfqpoint{3.115866in}{1.971660in}}%
\pgfpathmoveto{\pgfqpoint{3.115866in}{1.968711in}}%
\pgfpathlineto{\pgfqpoint{3.115866in}{1.968711in}}%
\pgfpathlineto{\pgfqpoint{3.115866in}{1.971660in}}%
\pgfpathlineto{\pgfqpoint{3.120408in}{1.971660in}}%
\pgfpathlineto{\pgfqpoint{3.120408in}{1.968711in}}%
\pgfpathmoveto{\pgfqpoint{3.115866in}{1.971660in}}%
\pgfpathlineto{\pgfqpoint{3.115866in}{1.971660in}}%
\pgfpathlineto{\pgfqpoint{3.115866in}{1.974609in}}%
\pgfpathlineto{\pgfqpoint{3.120408in}{1.974609in}}%
\pgfpathlineto{\pgfqpoint{3.120408in}{1.971660in}}%
\pgfpathmoveto{\pgfqpoint{3.120408in}{1.962813in}}%
\pgfpathlineto{\pgfqpoint{3.120408in}{1.962813in}}%
\pgfpathlineto{\pgfqpoint{3.120408in}{1.965762in}}%
\pgfpathlineto{\pgfqpoint{3.124949in}{1.965762in}}%
\pgfpathlineto{\pgfqpoint{3.124949in}{1.962813in}}%
\pgfpathmoveto{\pgfqpoint{3.120408in}{1.965762in}}%
\pgfpathlineto{\pgfqpoint{3.120408in}{1.965762in}}%
\pgfpathlineto{\pgfqpoint{3.120408in}{1.968711in}}%
\pgfpathlineto{\pgfqpoint{3.124949in}{1.968711in}}%
\pgfpathlineto{\pgfqpoint{3.124949in}{1.965762in}}%
\pgfpathmoveto{\pgfqpoint{3.124949in}{1.962813in}}%
\pgfpathlineto{\pgfqpoint{3.124949in}{1.962813in}}%
\pgfpathlineto{\pgfqpoint{3.124949in}{1.965762in}}%
\pgfpathlineto{\pgfqpoint{3.129490in}{1.965762in}}%
\pgfpathlineto{\pgfqpoint{3.129490in}{1.962813in}}%
\pgfpathmoveto{\pgfqpoint{3.124949in}{1.965762in}}%
\pgfpathlineto{\pgfqpoint{3.124949in}{1.965762in}}%
\pgfpathlineto{\pgfqpoint{3.124949in}{1.968711in}}%
\pgfpathlineto{\pgfqpoint{3.129490in}{1.968711in}}%
\pgfpathlineto{\pgfqpoint{3.129490in}{1.965762in}}%
\pgfpathmoveto{\pgfqpoint{3.120408in}{1.968711in}}%
\pgfpathlineto{\pgfqpoint{3.120408in}{1.968711in}}%
\pgfpathlineto{\pgfqpoint{3.120408in}{1.971660in}}%
\pgfpathlineto{\pgfqpoint{3.124949in}{1.971660in}}%
\pgfpathlineto{\pgfqpoint{3.124949in}{1.968711in}}%
\pgfpathmoveto{\pgfqpoint{3.111325in}{1.974609in}}%
\pgfpathlineto{\pgfqpoint{3.111325in}{1.974609in}}%
\pgfpathlineto{\pgfqpoint{3.111325in}{1.977558in}}%
\pgfpathlineto{\pgfqpoint{3.115866in}{1.977558in}}%
\pgfpathlineto{\pgfqpoint{3.115866in}{1.974609in}}%
\pgfpathmoveto{\pgfqpoint{3.111325in}{1.977558in}}%
\pgfpathlineto{\pgfqpoint{3.111325in}{1.977558in}}%
\pgfpathlineto{\pgfqpoint{3.111325in}{1.980507in}}%
\pgfpathlineto{\pgfqpoint{3.115866in}{1.980507in}}%
\pgfpathlineto{\pgfqpoint{3.115866in}{1.977558in}}%
\pgfpathmoveto{\pgfqpoint{3.115866in}{1.974609in}}%
\pgfpathlineto{\pgfqpoint{3.115866in}{1.974609in}}%
\pgfpathlineto{\pgfqpoint{3.115866in}{1.977558in}}%
\pgfpathlineto{\pgfqpoint{3.120408in}{1.977558in}}%
\pgfpathlineto{\pgfqpoint{3.120408in}{1.974609in}}%
\pgfpathmoveto{\pgfqpoint{3.129490in}{1.962813in}}%
\pgfpathlineto{\pgfqpoint{3.129490in}{1.962813in}}%
\pgfpathlineto{\pgfqpoint{3.129490in}{1.965762in}}%
\pgfpathlineto{\pgfqpoint{3.134031in}{1.965762in}}%
\pgfpathlineto{\pgfqpoint{3.134031in}{1.962813in}}%
\pgfpathmoveto{\pgfqpoint{3.156736in}{1.927424in}}%
\pgfpathlineto{\pgfqpoint{3.156736in}{1.927424in}}%
\pgfpathlineto{\pgfqpoint{3.156736in}{1.930373in}}%
\pgfpathlineto{\pgfqpoint{3.161277in}{1.930373in}}%
\pgfpathlineto{\pgfqpoint{3.161277in}{1.927424in}}%
\pgfpathmoveto{\pgfqpoint{3.156736in}{1.930373in}}%
\pgfpathlineto{\pgfqpoint{3.156736in}{1.930373in}}%
\pgfpathlineto{\pgfqpoint{3.156736in}{1.933322in}}%
\pgfpathlineto{\pgfqpoint{3.161277in}{1.933322in}}%
\pgfpathlineto{\pgfqpoint{3.161277in}{1.930373in}}%
\pgfpathmoveto{\pgfqpoint{3.161277in}{1.927424in}}%
\pgfpathlineto{\pgfqpoint{3.161277in}{1.927424in}}%
\pgfpathlineto{\pgfqpoint{3.161277in}{1.930373in}}%
\pgfpathlineto{\pgfqpoint{3.165818in}{1.930373in}}%
\pgfpathlineto{\pgfqpoint{3.165818in}{1.927424in}}%
\pgfpathmoveto{\pgfqpoint{3.161277in}{1.930373in}}%
\pgfpathlineto{\pgfqpoint{3.161277in}{1.930373in}}%
\pgfpathlineto{\pgfqpoint{3.161277in}{1.933322in}}%
\pgfpathlineto{\pgfqpoint{3.165818in}{1.933322in}}%
\pgfpathlineto{\pgfqpoint{3.165818in}{1.930373in}}%
\pgfpathmoveto{\pgfqpoint{3.156736in}{1.933322in}}%
\pgfpathlineto{\pgfqpoint{3.156736in}{1.933322in}}%
\pgfpathlineto{\pgfqpoint{3.156736in}{1.936271in}}%
\pgfpathlineto{\pgfqpoint{3.161277in}{1.936271in}}%
\pgfpathlineto{\pgfqpoint{3.161277in}{1.933322in}}%
\pgfpathmoveto{\pgfqpoint{3.156736in}{1.936271in}}%
\pgfpathlineto{\pgfqpoint{3.156736in}{1.936271in}}%
\pgfpathlineto{\pgfqpoint{3.156736in}{1.939220in}}%
\pgfpathlineto{\pgfqpoint{3.161277in}{1.939220in}}%
\pgfpathlineto{\pgfqpoint{3.161277in}{1.936271in}}%
\pgfpathmoveto{\pgfqpoint{3.161277in}{1.933322in}}%
\pgfpathlineto{\pgfqpoint{3.161277in}{1.933322in}}%
\pgfpathlineto{\pgfqpoint{3.161277in}{1.936271in}}%
\pgfpathlineto{\pgfqpoint{3.165818in}{1.936271in}}%
\pgfpathlineto{\pgfqpoint{3.165818in}{1.933322in}}%
\pgfpathmoveto{\pgfqpoint{3.165818in}{1.921526in}}%
\pgfpathlineto{\pgfqpoint{3.165818in}{1.921526in}}%
\pgfpathlineto{\pgfqpoint{3.165818in}{1.924475in}}%
\pgfpathlineto{\pgfqpoint{3.170359in}{1.924475in}}%
\pgfpathlineto{\pgfqpoint{3.170359in}{1.921526in}}%
\pgfpathmoveto{\pgfqpoint{3.165818in}{1.924475in}}%
\pgfpathlineto{\pgfqpoint{3.165818in}{1.924475in}}%
\pgfpathlineto{\pgfqpoint{3.165818in}{1.927424in}}%
\pgfpathlineto{\pgfqpoint{3.170359in}{1.927424in}}%
\pgfpathlineto{\pgfqpoint{3.170359in}{1.924475in}}%
\pgfpathmoveto{\pgfqpoint{3.170359in}{1.921526in}}%
\pgfpathlineto{\pgfqpoint{3.170359in}{1.921526in}}%
\pgfpathlineto{\pgfqpoint{3.170359in}{1.924475in}}%
\pgfpathlineto{\pgfqpoint{3.174900in}{1.924475in}}%
\pgfpathlineto{\pgfqpoint{3.174900in}{1.921526in}}%
\pgfpathmoveto{\pgfqpoint{3.170359in}{1.924475in}}%
\pgfpathlineto{\pgfqpoint{3.170359in}{1.924475in}}%
\pgfpathlineto{\pgfqpoint{3.170359in}{1.927424in}}%
\pgfpathlineto{\pgfqpoint{3.174900in}{1.927424in}}%
\pgfpathlineto{\pgfqpoint{3.174900in}{1.924475in}}%
\pgfpathmoveto{\pgfqpoint{3.174900in}{1.915628in}}%
\pgfpathlineto{\pgfqpoint{3.174900in}{1.915628in}}%
\pgfpathlineto{\pgfqpoint{3.174900in}{1.918577in}}%
\pgfpathlineto{\pgfqpoint{3.179441in}{1.918577in}}%
\pgfpathlineto{\pgfqpoint{3.179441in}{1.915628in}}%
\pgfpathmoveto{\pgfqpoint{3.174900in}{1.918577in}}%
\pgfpathlineto{\pgfqpoint{3.174900in}{1.918577in}}%
\pgfpathlineto{\pgfqpoint{3.174900in}{1.921526in}}%
\pgfpathlineto{\pgfqpoint{3.179441in}{1.921526in}}%
\pgfpathlineto{\pgfqpoint{3.179441in}{1.918577in}}%
\pgfpathmoveto{\pgfqpoint{3.179441in}{1.915628in}}%
\pgfpathlineto{\pgfqpoint{3.179441in}{1.915628in}}%
\pgfpathlineto{\pgfqpoint{3.179441in}{1.918577in}}%
\pgfpathlineto{\pgfqpoint{3.183982in}{1.918577in}}%
\pgfpathlineto{\pgfqpoint{3.183982in}{1.915628in}}%
\pgfpathmoveto{\pgfqpoint{3.179441in}{1.918577in}}%
\pgfpathlineto{\pgfqpoint{3.179441in}{1.918577in}}%
\pgfpathlineto{\pgfqpoint{3.179441in}{1.921526in}}%
\pgfpathlineto{\pgfqpoint{3.183982in}{1.921526in}}%
\pgfpathlineto{\pgfqpoint{3.183982in}{1.918577in}}%
\pgfpathmoveto{\pgfqpoint{3.174900in}{1.921526in}}%
\pgfpathlineto{\pgfqpoint{3.174900in}{1.921526in}}%
\pgfpathlineto{\pgfqpoint{3.174900in}{1.924475in}}%
\pgfpathlineto{\pgfqpoint{3.179441in}{1.924475in}}%
\pgfpathlineto{\pgfqpoint{3.179441in}{1.921526in}}%
\pgfpathmoveto{\pgfqpoint{3.165818in}{1.927424in}}%
\pgfpathlineto{\pgfqpoint{3.165818in}{1.927424in}}%
\pgfpathlineto{\pgfqpoint{3.165818in}{1.930373in}}%
\pgfpathlineto{\pgfqpoint{3.170359in}{1.930373in}}%
\pgfpathlineto{\pgfqpoint{3.170359in}{1.927424in}}%
\pgfpathmoveto{\pgfqpoint{3.165818in}{1.930373in}}%
\pgfpathlineto{\pgfqpoint{3.165818in}{1.930373in}}%
\pgfpathlineto{\pgfqpoint{3.165818in}{1.933322in}}%
\pgfpathlineto{\pgfqpoint{3.170359in}{1.933322in}}%
\pgfpathlineto{\pgfqpoint{3.170359in}{1.930373in}}%
\pgfpathmoveto{\pgfqpoint{3.147654in}{1.939220in}}%
\pgfpathlineto{\pgfqpoint{3.147654in}{1.939220in}}%
\pgfpathlineto{\pgfqpoint{3.147654in}{1.942169in}}%
\pgfpathlineto{\pgfqpoint{3.152195in}{1.942169in}}%
\pgfpathlineto{\pgfqpoint{3.152195in}{1.939220in}}%
\pgfpathmoveto{\pgfqpoint{3.147654in}{1.942169in}}%
\pgfpathlineto{\pgfqpoint{3.147654in}{1.942169in}}%
\pgfpathlineto{\pgfqpoint{3.147654in}{1.945118in}}%
\pgfpathlineto{\pgfqpoint{3.152195in}{1.945118in}}%
\pgfpathlineto{\pgfqpoint{3.152195in}{1.942169in}}%
\pgfpathmoveto{\pgfqpoint{3.152195in}{1.939220in}}%
\pgfpathlineto{\pgfqpoint{3.152195in}{1.939220in}}%
\pgfpathlineto{\pgfqpoint{3.152195in}{1.942169in}}%
\pgfpathlineto{\pgfqpoint{3.156736in}{1.942169in}}%
\pgfpathlineto{\pgfqpoint{3.156736in}{1.939220in}}%
\pgfpathmoveto{\pgfqpoint{3.152195in}{1.942169in}}%
\pgfpathlineto{\pgfqpoint{3.152195in}{1.942169in}}%
\pgfpathlineto{\pgfqpoint{3.152195in}{1.945118in}}%
\pgfpathlineto{\pgfqpoint{3.156736in}{1.945118in}}%
\pgfpathlineto{\pgfqpoint{3.156736in}{1.942169in}}%
\pgfpathmoveto{\pgfqpoint{3.147654in}{1.945118in}}%
\pgfpathlineto{\pgfqpoint{3.147654in}{1.945118in}}%
\pgfpathlineto{\pgfqpoint{3.147654in}{1.948067in}}%
\pgfpathlineto{\pgfqpoint{3.152195in}{1.948067in}}%
\pgfpathlineto{\pgfqpoint{3.152195in}{1.945118in}}%
\pgfpathmoveto{\pgfqpoint{3.156736in}{1.939220in}}%
\pgfpathlineto{\pgfqpoint{3.156736in}{1.939220in}}%
\pgfpathlineto{\pgfqpoint{3.156736in}{1.942169in}}%
\pgfpathlineto{\pgfqpoint{3.161277in}{1.942169in}}%
\pgfpathlineto{\pgfqpoint{3.161277in}{1.939220in}}%
\pgfpathmoveto{\pgfqpoint{3.074997in}{2.009998in}}%
\pgfpathlineto{\pgfqpoint{3.074997in}{2.009998in}}%
\pgfpathlineto{\pgfqpoint{3.074997in}{2.012947in}}%
\pgfpathlineto{\pgfqpoint{3.079538in}{2.012947in}}%
\pgfpathlineto{\pgfqpoint{3.079538in}{2.009998in}}%
\pgfpathmoveto{\pgfqpoint{3.338381in}{1.768165in}}%
\pgfpathlineto{\pgfqpoint{3.338381in}{1.768165in}}%
\pgfpathlineto{\pgfqpoint{3.338381in}{1.771114in}}%
\pgfpathlineto{\pgfqpoint{3.342922in}{1.771114in}}%
\pgfpathlineto{\pgfqpoint{3.342922in}{1.768165in}}%
\pgfpathmoveto{\pgfqpoint{3.338381in}{1.771114in}}%
\pgfpathlineto{\pgfqpoint{3.338381in}{1.771114in}}%
\pgfpathlineto{\pgfqpoint{3.338381in}{1.774063in}}%
\pgfpathlineto{\pgfqpoint{3.342922in}{1.774063in}}%
\pgfpathlineto{\pgfqpoint{3.342922in}{1.771114in}}%
\pgfpathmoveto{\pgfqpoint{3.342922in}{1.768165in}}%
\pgfpathlineto{\pgfqpoint{3.342922in}{1.768165in}}%
\pgfpathlineto{\pgfqpoint{3.342922in}{1.771114in}}%
\pgfpathlineto{\pgfqpoint{3.347463in}{1.771114in}}%
\pgfpathlineto{\pgfqpoint{3.347463in}{1.768165in}}%
\pgfpathmoveto{\pgfqpoint{3.342922in}{1.771114in}}%
\pgfpathlineto{\pgfqpoint{3.342922in}{1.771114in}}%
\pgfpathlineto{\pgfqpoint{3.342922in}{1.774063in}}%
\pgfpathlineto{\pgfqpoint{3.347463in}{1.774063in}}%
\pgfpathlineto{\pgfqpoint{3.347463in}{1.771114in}}%
\pgfpathmoveto{\pgfqpoint{3.356545in}{1.756368in}}%
\pgfpathlineto{\pgfqpoint{3.356545in}{1.756368in}}%
\pgfpathlineto{\pgfqpoint{3.356545in}{1.759317in}}%
\pgfpathlineto{\pgfqpoint{3.361087in}{1.759317in}}%
\pgfpathlineto{\pgfqpoint{3.361087in}{1.756368in}}%
\pgfpathmoveto{\pgfqpoint{3.356545in}{1.759317in}}%
\pgfpathlineto{\pgfqpoint{3.356545in}{1.759317in}}%
\pgfpathlineto{\pgfqpoint{3.356545in}{1.762266in}}%
\pgfpathlineto{\pgfqpoint{3.361087in}{1.762266in}}%
\pgfpathlineto{\pgfqpoint{3.361087in}{1.759317in}}%
\pgfpathmoveto{\pgfqpoint{3.361087in}{1.756368in}}%
\pgfpathlineto{\pgfqpoint{3.361087in}{1.756368in}}%
\pgfpathlineto{\pgfqpoint{3.361087in}{1.759317in}}%
\pgfpathlineto{\pgfqpoint{3.365628in}{1.759317in}}%
\pgfpathlineto{\pgfqpoint{3.365628in}{1.756368in}}%
\pgfpathmoveto{\pgfqpoint{3.361087in}{1.759317in}}%
\pgfpathlineto{\pgfqpoint{3.361087in}{1.759317in}}%
\pgfpathlineto{\pgfqpoint{3.361087in}{1.762266in}}%
\pgfpathlineto{\pgfqpoint{3.365628in}{1.762266in}}%
\pgfpathlineto{\pgfqpoint{3.365628in}{1.759317in}}%
\pgfpathmoveto{\pgfqpoint{3.347463in}{1.762266in}}%
\pgfpathlineto{\pgfqpoint{3.347463in}{1.762266in}}%
\pgfpathlineto{\pgfqpoint{3.347463in}{1.765216in}}%
\pgfpathlineto{\pgfqpoint{3.352004in}{1.765216in}}%
\pgfpathlineto{\pgfqpoint{3.352004in}{1.762266in}}%
\pgfpathmoveto{\pgfqpoint{3.347463in}{1.765216in}}%
\pgfpathlineto{\pgfqpoint{3.347463in}{1.765216in}}%
\pgfpathlineto{\pgfqpoint{3.347463in}{1.768165in}}%
\pgfpathlineto{\pgfqpoint{3.352004in}{1.768165in}}%
\pgfpathlineto{\pgfqpoint{3.352004in}{1.765216in}}%
\pgfpathmoveto{\pgfqpoint{3.352004in}{1.762266in}}%
\pgfpathlineto{\pgfqpoint{3.352004in}{1.762266in}}%
\pgfpathlineto{\pgfqpoint{3.352004in}{1.765216in}}%
\pgfpathlineto{\pgfqpoint{3.356545in}{1.765216in}}%
\pgfpathlineto{\pgfqpoint{3.356545in}{1.762266in}}%
\pgfpathmoveto{\pgfqpoint{3.352004in}{1.765216in}}%
\pgfpathlineto{\pgfqpoint{3.352004in}{1.765216in}}%
\pgfpathlineto{\pgfqpoint{3.352004in}{1.768165in}}%
\pgfpathlineto{\pgfqpoint{3.356545in}{1.768165in}}%
\pgfpathlineto{\pgfqpoint{3.356545in}{1.765216in}}%
\pgfpathmoveto{\pgfqpoint{3.347463in}{1.768165in}}%
\pgfpathlineto{\pgfqpoint{3.347463in}{1.768165in}}%
\pgfpathlineto{\pgfqpoint{3.347463in}{1.771114in}}%
\pgfpathlineto{\pgfqpoint{3.352004in}{1.771114in}}%
\pgfpathlineto{\pgfqpoint{3.352004in}{1.768165in}}%
\pgfpathmoveto{\pgfqpoint{3.347463in}{1.771114in}}%
\pgfpathlineto{\pgfqpoint{3.347463in}{1.771114in}}%
\pgfpathlineto{\pgfqpoint{3.347463in}{1.774063in}}%
\pgfpathlineto{\pgfqpoint{3.352004in}{1.774063in}}%
\pgfpathlineto{\pgfqpoint{3.352004in}{1.771114in}}%
\pgfpathmoveto{\pgfqpoint{3.352004in}{1.768165in}}%
\pgfpathlineto{\pgfqpoint{3.352004in}{1.768165in}}%
\pgfpathlineto{\pgfqpoint{3.352004in}{1.771114in}}%
\pgfpathlineto{\pgfqpoint{3.356545in}{1.771114in}}%
\pgfpathlineto{\pgfqpoint{3.356545in}{1.768165in}}%
\pgfpathmoveto{\pgfqpoint{3.356545in}{1.762266in}}%
\pgfpathlineto{\pgfqpoint{3.356545in}{1.762266in}}%
\pgfpathlineto{\pgfqpoint{3.356545in}{1.765216in}}%
\pgfpathlineto{\pgfqpoint{3.361087in}{1.765216in}}%
\pgfpathlineto{\pgfqpoint{3.361087in}{1.762266in}}%
\pgfpathmoveto{\pgfqpoint{3.356545in}{1.765216in}}%
\pgfpathlineto{\pgfqpoint{3.356545in}{1.765216in}}%
\pgfpathlineto{\pgfqpoint{3.356545in}{1.768165in}}%
\pgfpathlineto{\pgfqpoint{3.361087in}{1.768165in}}%
\pgfpathlineto{\pgfqpoint{3.361087in}{1.765216in}}%
\pgfpathmoveto{\pgfqpoint{3.311134in}{1.791757in}}%
\pgfpathlineto{\pgfqpoint{3.311134in}{1.791757in}}%
\pgfpathlineto{\pgfqpoint{3.311134in}{1.794707in}}%
\pgfpathlineto{\pgfqpoint{3.315675in}{1.794707in}}%
\pgfpathlineto{\pgfqpoint{3.315675in}{1.791757in}}%
\pgfpathmoveto{\pgfqpoint{3.311134in}{1.794707in}}%
\pgfpathlineto{\pgfqpoint{3.311134in}{1.794707in}}%
\pgfpathlineto{\pgfqpoint{3.311134in}{1.797656in}}%
\pgfpathlineto{\pgfqpoint{3.315675in}{1.797656in}}%
\pgfpathlineto{\pgfqpoint{3.315675in}{1.794707in}}%
\pgfpathmoveto{\pgfqpoint{3.315675in}{1.791757in}}%
\pgfpathlineto{\pgfqpoint{3.315675in}{1.791757in}}%
\pgfpathlineto{\pgfqpoint{3.315675in}{1.794707in}}%
\pgfpathlineto{\pgfqpoint{3.320216in}{1.794707in}}%
\pgfpathlineto{\pgfqpoint{3.320216in}{1.791757in}}%
\pgfpathmoveto{\pgfqpoint{3.315675in}{1.794707in}}%
\pgfpathlineto{\pgfqpoint{3.315675in}{1.794707in}}%
\pgfpathlineto{\pgfqpoint{3.315675in}{1.797656in}}%
\pgfpathlineto{\pgfqpoint{3.320216in}{1.797656in}}%
\pgfpathlineto{\pgfqpoint{3.320216in}{1.794707in}}%
\pgfpathmoveto{\pgfqpoint{3.320216in}{1.785859in}}%
\pgfpathlineto{\pgfqpoint{3.320216in}{1.785859in}}%
\pgfpathlineto{\pgfqpoint{3.320216in}{1.788808in}}%
\pgfpathlineto{\pgfqpoint{3.324757in}{1.788808in}}%
\pgfpathlineto{\pgfqpoint{3.324757in}{1.785859in}}%
\pgfpathmoveto{\pgfqpoint{3.320216in}{1.788808in}}%
\pgfpathlineto{\pgfqpoint{3.320216in}{1.788808in}}%
\pgfpathlineto{\pgfqpoint{3.320216in}{1.791757in}}%
\pgfpathlineto{\pgfqpoint{3.324757in}{1.791757in}}%
\pgfpathlineto{\pgfqpoint{3.324757in}{1.788808in}}%
\pgfpathmoveto{\pgfqpoint{3.324757in}{1.785859in}}%
\pgfpathlineto{\pgfqpoint{3.324757in}{1.785859in}}%
\pgfpathlineto{\pgfqpoint{3.324757in}{1.788808in}}%
\pgfpathlineto{\pgfqpoint{3.329298in}{1.788808in}}%
\pgfpathlineto{\pgfqpoint{3.329298in}{1.785859in}}%
\pgfpathmoveto{\pgfqpoint{3.324757in}{1.788808in}}%
\pgfpathlineto{\pgfqpoint{3.324757in}{1.788808in}}%
\pgfpathlineto{\pgfqpoint{3.324757in}{1.791757in}}%
\pgfpathlineto{\pgfqpoint{3.329298in}{1.791757in}}%
\pgfpathlineto{\pgfqpoint{3.329298in}{1.788808in}}%
\pgfpathmoveto{\pgfqpoint{3.320216in}{1.791757in}}%
\pgfpathlineto{\pgfqpoint{3.320216in}{1.791757in}}%
\pgfpathlineto{\pgfqpoint{3.320216in}{1.794707in}}%
\pgfpathlineto{\pgfqpoint{3.324757in}{1.794707in}}%
\pgfpathlineto{\pgfqpoint{3.324757in}{1.791757in}}%
\pgfpathmoveto{\pgfqpoint{3.320216in}{1.794707in}}%
\pgfpathlineto{\pgfqpoint{3.320216in}{1.794707in}}%
\pgfpathlineto{\pgfqpoint{3.320216in}{1.797656in}}%
\pgfpathlineto{\pgfqpoint{3.324757in}{1.797656in}}%
\pgfpathlineto{\pgfqpoint{3.324757in}{1.794707in}}%
\pgfpathmoveto{\pgfqpoint{3.324757in}{1.791757in}}%
\pgfpathlineto{\pgfqpoint{3.324757in}{1.791757in}}%
\pgfpathlineto{\pgfqpoint{3.324757in}{1.794707in}}%
\pgfpathlineto{\pgfqpoint{3.329298in}{1.794707in}}%
\pgfpathlineto{\pgfqpoint{3.329298in}{1.791757in}}%
\pgfpathmoveto{\pgfqpoint{3.302051in}{1.803554in}}%
\pgfpathlineto{\pgfqpoint{3.302051in}{1.803554in}}%
\pgfpathlineto{\pgfqpoint{3.302051in}{1.806503in}}%
\pgfpathlineto{\pgfqpoint{3.306593in}{1.806503in}}%
\pgfpathlineto{\pgfqpoint{3.306593in}{1.803554in}}%
\pgfpathmoveto{\pgfqpoint{3.302051in}{1.806503in}}%
\pgfpathlineto{\pgfqpoint{3.302051in}{1.806503in}}%
\pgfpathlineto{\pgfqpoint{3.302051in}{1.809452in}}%
\pgfpathlineto{\pgfqpoint{3.306593in}{1.809452in}}%
\pgfpathlineto{\pgfqpoint{3.306593in}{1.806503in}}%
\pgfpathmoveto{\pgfqpoint{3.306593in}{1.803554in}}%
\pgfpathlineto{\pgfqpoint{3.306593in}{1.803554in}}%
\pgfpathlineto{\pgfqpoint{3.306593in}{1.806503in}}%
\pgfpathlineto{\pgfqpoint{3.311134in}{1.806503in}}%
\pgfpathlineto{\pgfqpoint{3.311134in}{1.803554in}}%
\pgfpathmoveto{\pgfqpoint{3.306593in}{1.806503in}}%
\pgfpathlineto{\pgfqpoint{3.306593in}{1.806503in}}%
\pgfpathlineto{\pgfqpoint{3.306593in}{1.809452in}}%
\pgfpathlineto{\pgfqpoint{3.311134in}{1.809452in}}%
\pgfpathlineto{\pgfqpoint{3.311134in}{1.806503in}}%
\pgfpathmoveto{\pgfqpoint{3.292969in}{1.809452in}}%
\pgfpathlineto{\pgfqpoint{3.292969in}{1.809452in}}%
\pgfpathlineto{\pgfqpoint{3.292969in}{1.812401in}}%
\pgfpathlineto{\pgfqpoint{3.297510in}{1.812401in}}%
\pgfpathlineto{\pgfqpoint{3.297510in}{1.809452in}}%
\pgfpathmoveto{\pgfqpoint{3.292969in}{1.812401in}}%
\pgfpathlineto{\pgfqpoint{3.292969in}{1.812401in}}%
\pgfpathlineto{\pgfqpoint{3.292969in}{1.815350in}}%
\pgfpathlineto{\pgfqpoint{3.297510in}{1.815350in}}%
\pgfpathlineto{\pgfqpoint{3.297510in}{1.812401in}}%
\pgfpathmoveto{\pgfqpoint{3.297510in}{1.809452in}}%
\pgfpathlineto{\pgfqpoint{3.297510in}{1.809452in}}%
\pgfpathlineto{\pgfqpoint{3.297510in}{1.812401in}}%
\pgfpathlineto{\pgfqpoint{3.302051in}{1.812401in}}%
\pgfpathlineto{\pgfqpoint{3.302051in}{1.809452in}}%
\pgfpathmoveto{\pgfqpoint{3.297510in}{1.812401in}}%
\pgfpathlineto{\pgfqpoint{3.297510in}{1.812401in}}%
\pgfpathlineto{\pgfqpoint{3.297510in}{1.815350in}}%
\pgfpathlineto{\pgfqpoint{3.302051in}{1.815350in}}%
\pgfpathlineto{\pgfqpoint{3.302051in}{1.812401in}}%
\pgfpathmoveto{\pgfqpoint{3.292969in}{1.815350in}}%
\pgfpathlineto{\pgfqpoint{3.292969in}{1.815350in}}%
\pgfpathlineto{\pgfqpoint{3.292969in}{1.818299in}}%
\pgfpathlineto{\pgfqpoint{3.297510in}{1.818299in}}%
\pgfpathlineto{\pgfqpoint{3.297510in}{1.815350in}}%
\pgfpathmoveto{\pgfqpoint{3.292969in}{1.818299in}}%
\pgfpathlineto{\pgfqpoint{3.292969in}{1.818299in}}%
\pgfpathlineto{\pgfqpoint{3.292969in}{1.821249in}}%
\pgfpathlineto{\pgfqpoint{3.297510in}{1.821249in}}%
\pgfpathlineto{\pgfqpoint{3.297510in}{1.818299in}}%
\pgfpathmoveto{\pgfqpoint{3.297510in}{1.815350in}}%
\pgfpathlineto{\pgfqpoint{3.297510in}{1.815350in}}%
\pgfpathlineto{\pgfqpoint{3.297510in}{1.818299in}}%
\pgfpathlineto{\pgfqpoint{3.302051in}{1.818299in}}%
\pgfpathlineto{\pgfqpoint{3.302051in}{1.815350in}}%
\pgfpathmoveto{\pgfqpoint{3.302051in}{1.809452in}}%
\pgfpathlineto{\pgfqpoint{3.302051in}{1.809452in}}%
\pgfpathlineto{\pgfqpoint{3.302051in}{1.812401in}}%
\pgfpathlineto{\pgfqpoint{3.306593in}{1.812401in}}%
\pgfpathlineto{\pgfqpoint{3.306593in}{1.809452in}}%
\pgfpathmoveto{\pgfqpoint{3.302051in}{1.812401in}}%
\pgfpathlineto{\pgfqpoint{3.302051in}{1.812401in}}%
\pgfpathlineto{\pgfqpoint{3.302051in}{1.815350in}}%
\pgfpathlineto{\pgfqpoint{3.306593in}{1.815350in}}%
\pgfpathlineto{\pgfqpoint{3.306593in}{1.812401in}}%
\pgfpathmoveto{\pgfqpoint{3.306593in}{1.809452in}}%
\pgfpathlineto{\pgfqpoint{3.306593in}{1.809452in}}%
\pgfpathlineto{\pgfqpoint{3.306593in}{1.812401in}}%
\pgfpathlineto{\pgfqpoint{3.311134in}{1.812401in}}%
\pgfpathlineto{\pgfqpoint{3.311134in}{1.809452in}}%
\pgfpathmoveto{\pgfqpoint{3.311134in}{1.797656in}}%
\pgfpathlineto{\pgfqpoint{3.311134in}{1.797656in}}%
\pgfpathlineto{\pgfqpoint{3.311134in}{1.800605in}}%
\pgfpathlineto{\pgfqpoint{3.315675in}{1.800605in}}%
\pgfpathlineto{\pgfqpoint{3.315675in}{1.797656in}}%
\pgfpathmoveto{\pgfqpoint{3.311134in}{1.800605in}}%
\pgfpathlineto{\pgfqpoint{3.311134in}{1.800605in}}%
\pgfpathlineto{\pgfqpoint{3.311134in}{1.803554in}}%
\pgfpathlineto{\pgfqpoint{3.315675in}{1.803554in}}%
\pgfpathlineto{\pgfqpoint{3.315675in}{1.800605in}}%
\pgfpathmoveto{\pgfqpoint{3.315675in}{1.797656in}}%
\pgfpathlineto{\pgfqpoint{3.315675in}{1.797656in}}%
\pgfpathlineto{\pgfqpoint{3.315675in}{1.800605in}}%
\pgfpathlineto{\pgfqpoint{3.320216in}{1.800605in}}%
\pgfpathlineto{\pgfqpoint{3.320216in}{1.797656in}}%
\pgfpathmoveto{\pgfqpoint{3.315675in}{1.800605in}}%
\pgfpathlineto{\pgfqpoint{3.315675in}{1.800605in}}%
\pgfpathlineto{\pgfqpoint{3.315675in}{1.803554in}}%
\pgfpathlineto{\pgfqpoint{3.320216in}{1.803554in}}%
\pgfpathlineto{\pgfqpoint{3.320216in}{1.800605in}}%
\pgfpathmoveto{\pgfqpoint{3.311134in}{1.803554in}}%
\pgfpathlineto{\pgfqpoint{3.311134in}{1.803554in}}%
\pgfpathlineto{\pgfqpoint{3.311134in}{1.806503in}}%
\pgfpathlineto{\pgfqpoint{3.315675in}{1.806503in}}%
\pgfpathlineto{\pgfqpoint{3.315675in}{1.803554in}}%
\pgfpathmoveto{\pgfqpoint{3.329298in}{1.779961in}}%
\pgfpathlineto{\pgfqpoint{3.329298in}{1.779961in}}%
\pgfpathlineto{\pgfqpoint{3.329298in}{1.782910in}}%
\pgfpathlineto{\pgfqpoint{3.333840in}{1.782910in}}%
\pgfpathlineto{\pgfqpoint{3.333840in}{1.779961in}}%
\pgfpathmoveto{\pgfqpoint{3.329298in}{1.782910in}}%
\pgfpathlineto{\pgfqpoint{3.329298in}{1.782910in}}%
\pgfpathlineto{\pgfqpoint{3.329298in}{1.785859in}}%
\pgfpathlineto{\pgfqpoint{3.333840in}{1.785859in}}%
\pgfpathlineto{\pgfqpoint{3.333840in}{1.782910in}}%
\pgfpathmoveto{\pgfqpoint{3.333840in}{1.779961in}}%
\pgfpathlineto{\pgfqpoint{3.333840in}{1.779961in}}%
\pgfpathlineto{\pgfqpoint{3.333840in}{1.782910in}}%
\pgfpathlineto{\pgfqpoint{3.338381in}{1.782910in}}%
\pgfpathlineto{\pgfqpoint{3.338381in}{1.779961in}}%
\pgfpathmoveto{\pgfqpoint{3.333840in}{1.782910in}}%
\pgfpathlineto{\pgfqpoint{3.333840in}{1.782910in}}%
\pgfpathlineto{\pgfqpoint{3.333840in}{1.785859in}}%
\pgfpathlineto{\pgfqpoint{3.338381in}{1.785859in}}%
\pgfpathlineto{\pgfqpoint{3.338381in}{1.782910in}}%
\pgfpathmoveto{\pgfqpoint{3.338381in}{1.774063in}}%
\pgfpathlineto{\pgfqpoint{3.338381in}{1.774063in}}%
\pgfpathlineto{\pgfqpoint{3.338381in}{1.777012in}}%
\pgfpathlineto{\pgfqpoint{3.342922in}{1.777012in}}%
\pgfpathlineto{\pgfqpoint{3.342922in}{1.774063in}}%
\pgfpathmoveto{\pgfqpoint{3.338381in}{1.777012in}}%
\pgfpathlineto{\pgfqpoint{3.338381in}{1.777012in}}%
\pgfpathlineto{\pgfqpoint{3.338381in}{1.779961in}}%
\pgfpathlineto{\pgfqpoint{3.342922in}{1.779961in}}%
\pgfpathlineto{\pgfqpoint{3.342922in}{1.777012in}}%
\pgfpathmoveto{\pgfqpoint{3.342922in}{1.774063in}}%
\pgfpathlineto{\pgfqpoint{3.342922in}{1.774063in}}%
\pgfpathlineto{\pgfqpoint{3.342922in}{1.777012in}}%
\pgfpathlineto{\pgfqpoint{3.347463in}{1.777012in}}%
\pgfpathlineto{\pgfqpoint{3.347463in}{1.774063in}}%
\pgfpathmoveto{\pgfqpoint{3.342922in}{1.777012in}}%
\pgfpathlineto{\pgfqpoint{3.342922in}{1.777012in}}%
\pgfpathlineto{\pgfqpoint{3.342922in}{1.779961in}}%
\pgfpathlineto{\pgfqpoint{3.347463in}{1.779961in}}%
\pgfpathlineto{\pgfqpoint{3.347463in}{1.777012in}}%
\pgfpathmoveto{\pgfqpoint{3.338381in}{1.779961in}}%
\pgfpathlineto{\pgfqpoint{3.338381in}{1.779961in}}%
\pgfpathlineto{\pgfqpoint{3.338381in}{1.782910in}}%
\pgfpathlineto{\pgfqpoint{3.342922in}{1.782910in}}%
\pgfpathlineto{\pgfqpoint{3.342922in}{1.779961in}}%
\pgfpathmoveto{\pgfqpoint{3.329298in}{1.785859in}}%
\pgfpathlineto{\pgfqpoint{3.329298in}{1.785859in}}%
\pgfpathlineto{\pgfqpoint{3.329298in}{1.788808in}}%
\pgfpathlineto{\pgfqpoint{3.333840in}{1.788808in}}%
\pgfpathlineto{\pgfqpoint{3.333840in}{1.785859in}}%
\pgfpathmoveto{\pgfqpoint{3.329298in}{1.788808in}}%
\pgfpathlineto{\pgfqpoint{3.329298in}{1.788808in}}%
\pgfpathlineto{\pgfqpoint{3.329298in}{1.791757in}}%
\pgfpathlineto{\pgfqpoint{3.333840in}{1.791757in}}%
\pgfpathlineto{\pgfqpoint{3.333840in}{1.788808in}}%
\pgfpathmoveto{\pgfqpoint{3.247558in}{1.850742in}}%
\pgfpathlineto{\pgfqpoint{3.247558in}{1.850742in}}%
\pgfpathlineto{\pgfqpoint{3.247558in}{1.853691in}}%
\pgfpathlineto{\pgfqpoint{3.252099in}{1.853691in}}%
\pgfpathlineto{\pgfqpoint{3.252099in}{1.850742in}}%
\pgfpathmoveto{\pgfqpoint{3.247558in}{1.853691in}}%
\pgfpathlineto{\pgfqpoint{3.247558in}{1.853691in}}%
\pgfpathlineto{\pgfqpoint{3.247558in}{1.856641in}}%
\pgfpathlineto{\pgfqpoint{3.252099in}{1.856641in}}%
\pgfpathlineto{\pgfqpoint{3.252099in}{1.853691in}}%
\pgfpathmoveto{\pgfqpoint{3.252099in}{1.850742in}}%
\pgfpathlineto{\pgfqpoint{3.252099in}{1.850742in}}%
\pgfpathlineto{\pgfqpoint{3.252099in}{1.853691in}}%
\pgfpathlineto{\pgfqpoint{3.256640in}{1.853691in}}%
\pgfpathlineto{\pgfqpoint{3.256640in}{1.850742in}}%
\pgfpathmoveto{\pgfqpoint{3.252099in}{1.853691in}}%
\pgfpathlineto{\pgfqpoint{3.252099in}{1.853691in}}%
\pgfpathlineto{\pgfqpoint{3.252099in}{1.856641in}}%
\pgfpathlineto{\pgfqpoint{3.256640in}{1.856641in}}%
\pgfpathlineto{\pgfqpoint{3.256640in}{1.853691in}}%
\pgfpathmoveto{\pgfqpoint{3.238475in}{1.856641in}}%
\pgfpathlineto{\pgfqpoint{3.238475in}{1.856641in}}%
\pgfpathlineto{\pgfqpoint{3.238475in}{1.859590in}}%
\pgfpathlineto{\pgfqpoint{3.243016in}{1.859590in}}%
\pgfpathlineto{\pgfqpoint{3.243016in}{1.856641in}}%
\pgfpathmoveto{\pgfqpoint{3.238475in}{1.859590in}}%
\pgfpathlineto{\pgfqpoint{3.238475in}{1.859590in}}%
\pgfpathlineto{\pgfqpoint{3.238475in}{1.862539in}}%
\pgfpathlineto{\pgfqpoint{3.243016in}{1.862539in}}%
\pgfpathlineto{\pgfqpoint{3.243016in}{1.859590in}}%
\pgfpathmoveto{\pgfqpoint{3.243016in}{1.856641in}}%
\pgfpathlineto{\pgfqpoint{3.243016in}{1.856641in}}%
\pgfpathlineto{\pgfqpoint{3.243016in}{1.859590in}}%
\pgfpathlineto{\pgfqpoint{3.247558in}{1.859590in}}%
\pgfpathlineto{\pgfqpoint{3.247558in}{1.856641in}}%
\pgfpathmoveto{\pgfqpoint{3.243016in}{1.859590in}}%
\pgfpathlineto{\pgfqpoint{3.243016in}{1.859590in}}%
\pgfpathlineto{\pgfqpoint{3.243016in}{1.862539in}}%
\pgfpathlineto{\pgfqpoint{3.247558in}{1.862539in}}%
\pgfpathlineto{\pgfqpoint{3.247558in}{1.859590in}}%
\pgfpathmoveto{\pgfqpoint{3.238475in}{1.862539in}}%
\pgfpathlineto{\pgfqpoint{3.238475in}{1.862539in}}%
\pgfpathlineto{\pgfqpoint{3.238475in}{1.865489in}}%
\pgfpathlineto{\pgfqpoint{3.243016in}{1.865489in}}%
\pgfpathlineto{\pgfqpoint{3.243016in}{1.862539in}}%
\pgfpathmoveto{\pgfqpoint{3.238475in}{1.865489in}}%
\pgfpathlineto{\pgfqpoint{3.238475in}{1.865489in}}%
\pgfpathlineto{\pgfqpoint{3.238475in}{1.868438in}}%
\pgfpathlineto{\pgfqpoint{3.243016in}{1.868438in}}%
\pgfpathlineto{\pgfqpoint{3.243016in}{1.865489in}}%
\pgfpathmoveto{\pgfqpoint{3.243016in}{1.862539in}}%
\pgfpathlineto{\pgfqpoint{3.243016in}{1.862539in}}%
\pgfpathlineto{\pgfqpoint{3.243016in}{1.865489in}}%
\pgfpathlineto{\pgfqpoint{3.247558in}{1.865489in}}%
\pgfpathlineto{\pgfqpoint{3.247558in}{1.862539in}}%
\pgfpathmoveto{\pgfqpoint{3.247558in}{1.856641in}}%
\pgfpathlineto{\pgfqpoint{3.247558in}{1.856641in}}%
\pgfpathlineto{\pgfqpoint{3.247558in}{1.859590in}}%
\pgfpathlineto{\pgfqpoint{3.252099in}{1.859590in}}%
\pgfpathlineto{\pgfqpoint{3.252099in}{1.856641in}}%
\pgfpathmoveto{\pgfqpoint{3.247558in}{1.859590in}}%
\pgfpathlineto{\pgfqpoint{3.247558in}{1.859590in}}%
\pgfpathlineto{\pgfqpoint{3.247558in}{1.862539in}}%
\pgfpathlineto{\pgfqpoint{3.252099in}{1.862539in}}%
\pgfpathlineto{\pgfqpoint{3.252099in}{1.859590in}}%
\pgfpathmoveto{\pgfqpoint{3.252099in}{1.856641in}}%
\pgfpathlineto{\pgfqpoint{3.252099in}{1.856641in}}%
\pgfpathlineto{\pgfqpoint{3.252099in}{1.859590in}}%
\pgfpathlineto{\pgfqpoint{3.256640in}{1.859590in}}%
\pgfpathlineto{\pgfqpoint{3.256640in}{1.856641in}}%
\pgfpathmoveto{\pgfqpoint{3.265722in}{1.833046in}}%
\pgfpathlineto{\pgfqpoint{3.265722in}{1.833046in}}%
\pgfpathlineto{\pgfqpoint{3.265722in}{1.835995in}}%
\pgfpathlineto{\pgfqpoint{3.270263in}{1.835995in}}%
\pgfpathlineto{\pgfqpoint{3.270263in}{1.833046in}}%
\pgfpathmoveto{\pgfqpoint{3.265722in}{1.835995in}}%
\pgfpathlineto{\pgfqpoint{3.265722in}{1.835995in}}%
\pgfpathlineto{\pgfqpoint{3.265722in}{1.838945in}}%
\pgfpathlineto{\pgfqpoint{3.270263in}{1.838945in}}%
\pgfpathlineto{\pgfqpoint{3.270263in}{1.835995in}}%
\pgfpathmoveto{\pgfqpoint{3.270263in}{1.833046in}}%
\pgfpathlineto{\pgfqpoint{3.270263in}{1.833046in}}%
\pgfpathlineto{\pgfqpoint{3.270263in}{1.835995in}}%
\pgfpathlineto{\pgfqpoint{3.274805in}{1.835995in}}%
\pgfpathlineto{\pgfqpoint{3.274805in}{1.833046in}}%
\pgfpathmoveto{\pgfqpoint{3.270263in}{1.835995in}}%
\pgfpathlineto{\pgfqpoint{3.270263in}{1.835995in}}%
\pgfpathlineto{\pgfqpoint{3.270263in}{1.838945in}}%
\pgfpathlineto{\pgfqpoint{3.274805in}{1.838945in}}%
\pgfpathlineto{\pgfqpoint{3.274805in}{1.835995in}}%
\pgfpathmoveto{\pgfqpoint{3.265722in}{1.838945in}}%
\pgfpathlineto{\pgfqpoint{3.265722in}{1.838945in}}%
\pgfpathlineto{\pgfqpoint{3.265722in}{1.841894in}}%
\pgfpathlineto{\pgfqpoint{3.270263in}{1.841894in}}%
\pgfpathlineto{\pgfqpoint{3.270263in}{1.838945in}}%
\pgfpathmoveto{\pgfqpoint{3.265722in}{1.841894in}}%
\pgfpathlineto{\pgfqpoint{3.265722in}{1.841894in}}%
\pgfpathlineto{\pgfqpoint{3.265722in}{1.844843in}}%
\pgfpathlineto{\pgfqpoint{3.270263in}{1.844843in}}%
\pgfpathlineto{\pgfqpoint{3.270263in}{1.841894in}}%
\pgfpathmoveto{\pgfqpoint{3.270263in}{1.838945in}}%
\pgfpathlineto{\pgfqpoint{3.270263in}{1.838945in}}%
\pgfpathlineto{\pgfqpoint{3.270263in}{1.841894in}}%
\pgfpathlineto{\pgfqpoint{3.274805in}{1.841894in}}%
\pgfpathlineto{\pgfqpoint{3.274805in}{1.838945in}}%
\pgfpathmoveto{\pgfqpoint{3.274805in}{1.827147in}}%
\pgfpathlineto{\pgfqpoint{3.274805in}{1.827147in}}%
\pgfpathlineto{\pgfqpoint{3.274805in}{1.830097in}}%
\pgfpathlineto{\pgfqpoint{3.279346in}{1.830097in}}%
\pgfpathlineto{\pgfqpoint{3.279346in}{1.827147in}}%
\pgfpathmoveto{\pgfqpoint{3.274805in}{1.830097in}}%
\pgfpathlineto{\pgfqpoint{3.274805in}{1.830097in}}%
\pgfpathlineto{\pgfqpoint{3.274805in}{1.833046in}}%
\pgfpathlineto{\pgfqpoint{3.279346in}{1.833046in}}%
\pgfpathlineto{\pgfqpoint{3.279346in}{1.830097in}}%
\pgfpathmoveto{\pgfqpoint{3.279346in}{1.827147in}}%
\pgfpathlineto{\pgfqpoint{3.279346in}{1.827147in}}%
\pgfpathlineto{\pgfqpoint{3.279346in}{1.830097in}}%
\pgfpathlineto{\pgfqpoint{3.283887in}{1.830097in}}%
\pgfpathlineto{\pgfqpoint{3.283887in}{1.827147in}}%
\pgfpathmoveto{\pgfqpoint{3.279346in}{1.830097in}}%
\pgfpathlineto{\pgfqpoint{3.279346in}{1.830097in}}%
\pgfpathlineto{\pgfqpoint{3.279346in}{1.833046in}}%
\pgfpathlineto{\pgfqpoint{3.283887in}{1.833046in}}%
\pgfpathlineto{\pgfqpoint{3.283887in}{1.830097in}}%
\pgfpathmoveto{\pgfqpoint{3.283887in}{1.821249in}}%
\pgfpathlineto{\pgfqpoint{3.283887in}{1.821249in}}%
\pgfpathlineto{\pgfqpoint{3.283887in}{1.824198in}}%
\pgfpathlineto{\pgfqpoint{3.288428in}{1.824198in}}%
\pgfpathlineto{\pgfqpoint{3.288428in}{1.821249in}}%
\pgfpathmoveto{\pgfqpoint{3.283887in}{1.824198in}}%
\pgfpathlineto{\pgfqpoint{3.283887in}{1.824198in}}%
\pgfpathlineto{\pgfqpoint{3.283887in}{1.827147in}}%
\pgfpathlineto{\pgfqpoint{3.288428in}{1.827147in}}%
\pgfpathlineto{\pgfqpoint{3.288428in}{1.824198in}}%
\pgfpathmoveto{\pgfqpoint{3.288428in}{1.821249in}}%
\pgfpathlineto{\pgfqpoint{3.288428in}{1.821249in}}%
\pgfpathlineto{\pgfqpoint{3.288428in}{1.824198in}}%
\pgfpathlineto{\pgfqpoint{3.292969in}{1.824198in}}%
\pgfpathlineto{\pgfqpoint{3.292969in}{1.821249in}}%
\pgfpathmoveto{\pgfqpoint{3.288428in}{1.824198in}}%
\pgfpathlineto{\pgfqpoint{3.288428in}{1.824198in}}%
\pgfpathlineto{\pgfqpoint{3.288428in}{1.827147in}}%
\pgfpathlineto{\pgfqpoint{3.292969in}{1.827147in}}%
\pgfpathlineto{\pgfqpoint{3.292969in}{1.824198in}}%
\pgfpathmoveto{\pgfqpoint{3.283887in}{1.827147in}}%
\pgfpathlineto{\pgfqpoint{3.283887in}{1.827147in}}%
\pgfpathlineto{\pgfqpoint{3.283887in}{1.830097in}}%
\pgfpathlineto{\pgfqpoint{3.288428in}{1.830097in}}%
\pgfpathlineto{\pgfqpoint{3.288428in}{1.827147in}}%
\pgfpathmoveto{\pgfqpoint{3.274805in}{1.833046in}}%
\pgfpathlineto{\pgfqpoint{3.274805in}{1.833046in}}%
\pgfpathlineto{\pgfqpoint{3.274805in}{1.835995in}}%
\pgfpathlineto{\pgfqpoint{3.279346in}{1.835995in}}%
\pgfpathlineto{\pgfqpoint{3.279346in}{1.833046in}}%
\pgfpathmoveto{\pgfqpoint{3.274805in}{1.835995in}}%
\pgfpathlineto{\pgfqpoint{3.274805in}{1.835995in}}%
\pgfpathlineto{\pgfqpoint{3.274805in}{1.838945in}}%
\pgfpathlineto{\pgfqpoint{3.279346in}{1.838945in}}%
\pgfpathlineto{\pgfqpoint{3.279346in}{1.835995in}}%
\pgfpathmoveto{\pgfqpoint{3.279346in}{1.833046in}}%
\pgfpathlineto{\pgfqpoint{3.279346in}{1.833046in}}%
\pgfpathlineto{\pgfqpoint{3.279346in}{1.835995in}}%
\pgfpathlineto{\pgfqpoint{3.283887in}{1.835995in}}%
\pgfpathlineto{\pgfqpoint{3.283887in}{1.833046in}}%
\pgfpathmoveto{\pgfqpoint{3.256640in}{1.844843in}}%
\pgfpathlineto{\pgfqpoint{3.256640in}{1.844843in}}%
\pgfpathlineto{\pgfqpoint{3.256640in}{1.847793in}}%
\pgfpathlineto{\pgfqpoint{3.261181in}{1.847793in}}%
\pgfpathlineto{\pgfqpoint{3.261181in}{1.844843in}}%
\pgfpathmoveto{\pgfqpoint{3.256640in}{1.847793in}}%
\pgfpathlineto{\pgfqpoint{3.256640in}{1.847793in}}%
\pgfpathlineto{\pgfqpoint{3.256640in}{1.850742in}}%
\pgfpathlineto{\pgfqpoint{3.261181in}{1.850742in}}%
\pgfpathlineto{\pgfqpoint{3.261181in}{1.847793in}}%
\pgfpathmoveto{\pgfqpoint{3.261181in}{1.844843in}}%
\pgfpathlineto{\pgfqpoint{3.261181in}{1.844843in}}%
\pgfpathlineto{\pgfqpoint{3.261181in}{1.847793in}}%
\pgfpathlineto{\pgfqpoint{3.265722in}{1.847793in}}%
\pgfpathlineto{\pgfqpoint{3.265722in}{1.844843in}}%
\pgfpathmoveto{\pgfqpoint{3.261181in}{1.847793in}}%
\pgfpathlineto{\pgfqpoint{3.261181in}{1.847793in}}%
\pgfpathlineto{\pgfqpoint{3.261181in}{1.850742in}}%
\pgfpathlineto{\pgfqpoint{3.265722in}{1.850742in}}%
\pgfpathlineto{\pgfqpoint{3.265722in}{1.847793in}}%
\pgfpathmoveto{\pgfqpoint{3.256640in}{1.850742in}}%
\pgfpathlineto{\pgfqpoint{3.256640in}{1.850742in}}%
\pgfpathlineto{\pgfqpoint{3.256640in}{1.853691in}}%
\pgfpathlineto{\pgfqpoint{3.261181in}{1.853691in}}%
\pgfpathlineto{\pgfqpoint{3.261181in}{1.850742in}}%
\pgfpathmoveto{\pgfqpoint{3.265722in}{1.844843in}}%
\pgfpathlineto{\pgfqpoint{3.265722in}{1.844843in}}%
\pgfpathlineto{\pgfqpoint{3.265722in}{1.847793in}}%
\pgfpathlineto{\pgfqpoint{3.270263in}{1.847793in}}%
\pgfpathlineto{\pgfqpoint{3.270263in}{1.844843in}}%
\pgfpathmoveto{\pgfqpoint{3.220311in}{1.874337in}}%
\pgfpathlineto{\pgfqpoint{3.220311in}{1.874337in}}%
\pgfpathlineto{\pgfqpoint{3.220311in}{1.877286in}}%
\pgfpathlineto{\pgfqpoint{3.224852in}{1.877286in}}%
\pgfpathlineto{\pgfqpoint{3.224852in}{1.874337in}}%
\pgfpathmoveto{\pgfqpoint{3.220311in}{1.877286in}}%
\pgfpathlineto{\pgfqpoint{3.220311in}{1.877286in}}%
\pgfpathlineto{\pgfqpoint{3.220311in}{1.880235in}}%
\pgfpathlineto{\pgfqpoint{3.224852in}{1.880235in}}%
\pgfpathlineto{\pgfqpoint{3.224852in}{1.877286in}}%
\pgfpathmoveto{\pgfqpoint{3.224852in}{1.874337in}}%
\pgfpathlineto{\pgfqpoint{3.224852in}{1.874337in}}%
\pgfpathlineto{\pgfqpoint{3.224852in}{1.877286in}}%
\pgfpathlineto{\pgfqpoint{3.229393in}{1.877286in}}%
\pgfpathlineto{\pgfqpoint{3.229393in}{1.874337in}}%
\pgfpathmoveto{\pgfqpoint{3.224852in}{1.877286in}}%
\pgfpathlineto{\pgfqpoint{3.224852in}{1.877286in}}%
\pgfpathlineto{\pgfqpoint{3.224852in}{1.880235in}}%
\pgfpathlineto{\pgfqpoint{3.229393in}{1.880235in}}%
\pgfpathlineto{\pgfqpoint{3.229393in}{1.877286in}}%
\pgfpathmoveto{\pgfqpoint{3.229393in}{1.868438in}}%
\pgfpathlineto{\pgfqpoint{3.229393in}{1.868438in}}%
\pgfpathlineto{\pgfqpoint{3.229393in}{1.871387in}}%
\pgfpathlineto{\pgfqpoint{3.233934in}{1.871387in}}%
\pgfpathlineto{\pgfqpoint{3.233934in}{1.868438in}}%
\pgfpathmoveto{\pgfqpoint{3.229393in}{1.871387in}}%
\pgfpathlineto{\pgfqpoint{3.229393in}{1.871387in}}%
\pgfpathlineto{\pgfqpoint{3.229393in}{1.874337in}}%
\pgfpathlineto{\pgfqpoint{3.233934in}{1.874337in}}%
\pgfpathlineto{\pgfqpoint{3.233934in}{1.871387in}}%
\pgfpathmoveto{\pgfqpoint{3.233934in}{1.868438in}}%
\pgfpathlineto{\pgfqpoint{3.233934in}{1.868438in}}%
\pgfpathlineto{\pgfqpoint{3.233934in}{1.871387in}}%
\pgfpathlineto{\pgfqpoint{3.238475in}{1.871387in}}%
\pgfpathlineto{\pgfqpoint{3.238475in}{1.868438in}}%
\pgfpathmoveto{\pgfqpoint{3.233934in}{1.871387in}}%
\pgfpathlineto{\pgfqpoint{3.233934in}{1.871387in}}%
\pgfpathlineto{\pgfqpoint{3.233934in}{1.874337in}}%
\pgfpathlineto{\pgfqpoint{3.238475in}{1.874337in}}%
\pgfpathlineto{\pgfqpoint{3.238475in}{1.871387in}}%
\pgfpathmoveto{\pgfqpoint{3.229393in}{1.874337in}}%
\pgfpathlineto{\pgfqpoint{3.229393in}{1.874337in}}%
\pgfpathlineto{\pgfqpoint{3.229393in}{1.877286in}}%
\pgfpathlineto{\pgfqpoint{3.233934in}{1.877286in}}%
\pgfpathlineto{\pgfqpoint{3.233934in}{1.874337in}}%
\pgfpathmoveto{\pgfqpoint{3.220311in}{1.880235in}}%
\pgfpathlineto{\pgfqpoint{3.220311in}{1.880235in}}%
\pgfpathlineto{\pgfqpoint{3.220311in}{1.883185in}}%
\pgfpathlineto{\pgfqpoint{3.224852in}{1.883185in}}%
\pgfpathlineto{\pgfqpoint{3.224852in}{1.880235in}}%
\pgfpathmoveto{\pgfqpoint{3.220311in}{1.883185in}}%
\pgfpathlineto{\pgfqpoint{3.220311in}{1.883185in}}%
\pgfpathlineto{\pgfqpoint{3.220311in}{1.886134in}}%
\pgfpathlineto{\pgfqpoint{3.224852in}{1.886134in}}%
\pgfpathlineto{\pgfqpoint{3.224852in}{1.883185in}}%
\pgfpathmoveto{\pgfqpoint{3.224852in}{1.880235in}}%
\pgfpathlineto{\pgfqpoint{3.224852in}{1.880235in}}%
\pgfpathlineto{\pgfqpoint{3.224852in}{1.883185in}}%
\pgfpathlineto{\pgfqpoint{3.229393in}{1.883185in}}%
\pgfpathlineto{\pgfqpoint{3.229393in}{1.880235in}}%
\pgfpathmoveto{\pgfqpoint{3.238475in}{1.868438in}}%
\pgfpathlineto{\pgfqpoint{3.238475in}{1.868438in}}%
\pgfpathlineto{\pgfqpoint{3.238475in}{1.871387in}}%
\pgfpathlineto{\pgfqpoint{3.243016in}{1.871387in}}%
\pgfpathlineto{\pgfqpoint{3.243016in}{1.868438in}}%
\pgfpathmoveto{\pgfqpoint{3.292969in}{1.821249in}}%
\pgfpathlineto{\pgfqpoint{3.292969in}{1.821249in}}%
\pgfpathlineto{\pgfqpoint{3.292969in}{1.824198in}}%
\pgfpathlineto{\pgfqpoint{3.297510in}{1.824198in}}%
\pgfpathlineto{\pgfqpoint{3.297510in}{1.821249in}}%
\pgfpathmoveto{\pgfqpoint{3.456447in}{0.499998in}}%
\pgfpathlineto{\pgfqpoint{3.456447in}{0.499998in}}%
\pgfpathlineto{\pgfqpoint{3.456447in}{0.502948in}}%
\pgfpathlineto{\pgfqpoint{3.460988in}{0.502948in}}%
\pgfpathlineto{\pgfqpoint{3.460988in}{0.499998in}}%
\pgfpathmoveto{\pgfqpoint{3.456447in}{0.502948in}}%
\pgfpathlineto{\pgfqpoint{3.456447in}{0.502948in}}%
\pgfpathlineto{\pgfqpoint{3.456447in}{0.505897in}}%
\pgfpathlineto{\pgfqpoint{3.460988in}{0.505897in}}%
\pgfpathlineto{\pgfqpoint{3.460988in}{0.502948in}}%
\pgfpathmoveto{\pgfqpoint{3.460988in}{0.499998in}}%
\pgfpathlineto{\pgfqpoint{3.460988in}{0.499998in}}%
\pgfpathlineto{\pgfqpoint{3.460988in}{0.502948in}}%
\pgfpathlineto{\pgfqpoint{3.465529in}{0.502948in}}%
\pgfpathlineto{\pgfqpoint{3.465529in}{0.499998in}}%
\pgfpathmoveto{\pgfqpoint{3.460988in}{0.502948in}}%
\pgfpathlineto{\pgfqpoint{3.460988in}{0.502948in}}%
\pgfpathlineto{\pgfqpoint{3.460988in}{0.505897in}}%
\pgfpathlineto{\pgfqpoint{3.465529in}{0.505897in}}%
\pgfpathlineto{\pgfqpoint{3.465529in}{0.502948in}}%
\pgfpathmoveto{\pgfqpoint{3.465529in}{0.499998in}}%
\pgfpathlineto{\pgfqpoint{3.465529in}{0.499998in}}%
\pgfpathlineto{\pgfqpoint{3.465529in}{0.502948in}}%
\pgfpathlineto{\pgfqpoint{3.470070in}{0.502948in}}%
\pgfpathlineto{\pgfqpoint{3.470070in}{0.499998in}}%
\pgfpathmoveto{\pgfqpoint{3.465529in}{0.502948in}}%
\pgfpathlineto{\pgfqpoint{3.465529in}{0.502948in}}%
\pgfpathlineto{\pgfqpoint{3.465529in}{0.505897in}}%
\pgfpathlineto{\pgfqpoint{3.470070in}{0.505897in}}%
\pgfpathlineto{\pgfqpoint{3.470070in}{0.502948in}}%
\pgfpathmoveto{\pgfqpoint{3.465529in}{0.505897in}}%
\pgfpathlineto{\pgfqpoint{3.465529in}{0.505897in}}%
\pgfpathlineto{\pgfqpoint{3.465529in}{0.508846in}}%
\pgfpathlineto{\pgfqpoint{3.470070in}{0.508846in}}%
\pgfpathlineto{\pgfqpoint{3.470070in}{0.505897in}}%
\pgfpathmoveto{\pgfqpoint{3.465529in}{0.508846in}}%
\pgfpathlineto{\pgfqpoint{3.465529in}{0.508846in}}%
\pgfpathlineto{\pgfqpoint{3.465529in}{0.511795in}}%
\pgfpathlineto{\pgfqpoint{3.470070in}{0.511795in}}%
\pgfpathlineto{\pgfqpoint{3.470070in}{0.508846in}}%
\pgfpathmoveto{\pgfqpoint{3.470070in}{0.505897in}}%
\pgfpathlineto{\pgfqpoint{3.470070in}{0.505897in}}%
\pgfpathlineto{\pgfqpoint{3.470070in}{0.508846in}}%
\pgfpathlineto{\pgfqpoint{3.474611in}{0.508846in}}%
\pgfpathlineto{\pgfqpoint{3.474611in}{0.505897in}}%
\pgfpathmoveto{\pgfqpoint{3.470070in}{0.508846in}}%
\pgfpathlineto{\pgfqpoint{3.470070in}{0.508846in}}%
\pgfpathlineto{\pgfqpoint{3.470070in}{0.511795in}}%
\pgfpathlineto{\pgfqpoint{3.474611in}{0.511795in}}%
\pgfpathlineto{\pgfqpoint{3.474611in}{0.508846in}}%
\pgfpathmoveto{\pgfqpoint{3.474611in}{0.508846in}}%
\pgfpathlineto{\pgfqpoint{3.474611in}{0.508846in}}%
\pgfpathlineto{\pgfqpoint{3.474611in}{0.511795in}}%
\pgfpathlineto{\pgfqpoint{3.479152in}{0.511795in}}%
\pgfpathlineto{\pgfqpoint{3.479152in}{0.508846in}}%
\pgfpathmoveto{\pgfqpoint{3.474611in}{0.511795in}}%
\pgfpathlineto{\pgfqpoint{3.474611in}{0.511795in}}%
\pgfpathlineto{\pgfqpoint{3.474611in}{0.514744in}}%
\pgfpathlineto{\pgfqpoint{3.479152in}{0.514744in}}%
\pgfpathlineto{\pgfqpoint{3.479152in}{0.511795in}}%
\pgfpathmoveto{\pgfqpoint{3.474611in}{0.514744in}}%
\pgfpathlineto{\pgfqpoint{3.474611in}{0.514744in}}%
\pgfpathlineto{\pgfqpoint{3.474611in}{0.517694in}}%
\pgfpathlineto{\pgfqpoint{3.479152in}{0.517694in}}%
\pgfpathlineto{\pgfqpoint{3.479152in}{0.514744in}}%
\pgfpathmoveto{\pgfqpoint{3.479152in}{0.511795in}}%
\pgfpathlineto{\pgfqpoint{3.479152in}{0.511795in}}%
\pgfpathlineto{\pgfqpoint{3.479152in}{0.514744in}}%
\pgfpathlineto{\pgfqpoint{3.483693in}{0.514744in}}%
\pgfpathlineto{\pgfqpoint{3.483693in}{0.511795in}}%
\pgfpathmoveto{\pgfqpoint{3.479152in}{0.514744in}}%
\pgfpathlineto{\pgfqpoint{3.479152in}{0.514744in}}%
\pgfpathlineto{\pgfqpoint{3.479152in}{0.517694in}}%
\pgfpathlineto{\pgfqpoint{3.483693in}{0.517694in}}%
\pgfpathlineto{\pgfqpoint{3.483693in}{0.514744in}}%
\pgfpathmoveto{\pgfqpoint{3.483693in}{0.514744in}}%
\pgfpathlineto{\pgfqpoint{3.483693in}{0.514744in}}%
\pgfpathlineto{\pgfqpoint{3.483693in}{0.517694in}}%
\pgfpathlineto{\pgfqpoint{3.488234in}{0.517694in}}%
\pgfpathlineto{\pgfqpoint{3.488234in}{0.514744in}}%
\pgfpathmoveto{\pgfqpoint{3.483693in}{0.517694in}}%
\pgfpathlineto{\pgfqpoint{3.483693in}{0.517694in}}%
\pgfpathlineto{\pgfqpoint{3.483693in}{0.520643in}}%
\pgfpathlineto{\pgfqpoint{3.488234in}{0.520643in}}%
\pgfpathlineto{\pgfqpoint{3.488234in}{0.517694in}}%
\pgfpathmoveto{\pgfqpoint{3.483693in}{0.520643in}}%
\pgfpathlineto{\pgfqpoint{3.483693in}{0.520643in}}%
\pgfpathlineto{\pgfqpoint{3.483693in}{0.523592in}}%
\pgfpathlineto{\pgfqpoint{3.488234in}{0.523592in}}%
\pgfpathlineto{\pgfqpoint{3.488234in}{0.520643in}}%
\pgfpathmoveto{\pgfqpoint{3.488234in}{0.517694in}}%
\pgfpathlineto{\pgfqpoint{3.488234in}{0.517694in}}%
\pgfpathlineto{\pgfqpoint{3.488234in}{0.520643in}}%
\pgfpathlineto{\pgfqpoint{3.492775in}{0.520643in}}%
\pgfpathlineto{\pgfqpoint{3.492775in}{0.517694in}}%
\pgfpathmoveto{\pgfqpoint{3.488234in}{0.520643in}}%
\pgfpathlineto{\pgfqpoint{3.488234in}{0.520643in}}%
\pgfpathlineto{\pgfqpoint{3.488234in}{0.523592in}}%
\pgfpathlineto{\pgfqpoint{3.492775in}{0.523592in}}%
\pgfpathlineto{\pgfqpoint{3.492775in}{0.520643in}}%
\pgfpathmoveto{\pgfqpoint{3.483693in}{0.523592in}}%
\pgfpathlineto{\pgfqpoint{3.483693in}{0.523592in}}%
\pgfpathlineto{\pgfqpoint{3.483693in}{0.526541in}}%
\pgfpathlineto{\pgfqpoint{3.488234in}{0.526541in}}%
\pgfpathlineto{\pgfqpoint{3.488234in}{0.523592in}}%
\pgfpathmoveto{\pgfqpoint{3.483693in}{0.526541in}}%
\pgfpathlineto{\pgfqpoint{3.483693in}{0.526541in}}%
\pgfpathlineto{\pgfqpoint{3.483693in}{0.529491in}}%
\pgfpathlineto{\pgfqpoint{3.488234in}{0.529491in}}%
\pgfpathlineto{\pgfqpoint{3.488234in}{0.526541in}}%
\pgfpathmoveto{\pgfqpoint{3.488234in}{0.523592in}}%
\pgfpathlineto{\pgfqpoint{3.488234in}{0.523592in}}%
\pgfpathlineto{\pgfqpoint{3.488234in}{0.526541in}}%
\pgfpathlineto{\pgfqpoint{3.492775in}{0.526541in}}%
\pgfpathlineto{\pgfqpoint{3.492775in}{0.523592in}}%
\pgfpathmoveto{\pgfqpoint{3.488234in}{0.526541in}}%
\pgfpathlineto{\pgfqpoint{3.488234in}{0.526541in}}%
\pgfpathlineto{\pgfqpoint{3.488234in}{0.529491in}}%
\pgfpathlineto{\pgfqpoint{3.492775in}{0.529491in}}%
\pgfpathlineto{\pgfqpoint{3.492775in}{0.526541in}}%
\pgfpathmoveto{\pgfqpoint{3.492775in}{0.523592in}}%
\pgfpathlineto{\pgfqpoint{3.492775in}{0.523592in}}%
\pgfpathlineto{\pgfqpoint{3.492775in}{0.526541in}}%
\pgfpathlineto{\pgfqpoint{3.497316in}{0.526541in}}%
\pgfpathlineto{\pgfqpoint{3.497316in}{0.523592in}}%
\pgfpathmoveto{\pgfqpoint{3.492775in}{0.526541in}}%
\pgfpathlineto{\pgfqpoint{3.492775in}{0.526541in}}%
\pgfpathlineto{\pgfqpoint{3.492775in}{0.529491in}}%
\pgfpathlineto{\pgfqpoint{3.497316in}{0.529491in}}%
\pgfpathlineto{\pgfqpoint{3.497316in}{0.526541in}}%
\pgfpathmoveto{\pgfqpoint{3.497316in}{0.526541in}}%
\pgfpathlineto{\pgfqpoint{3.497316in}{0.526541in}}%
\pgfpathlineto{\pgfqpoint{3.497316in}{0.529491in}}%
\pgfpathlineto{\pgfqpoint{3.501857in}{0.529491in}}%
\pgfpathlineto{\pgfqpoint{3.501857in}{0.526541in}}%
\pgfpathmoveto{\pgfqpoint{3.492775in}{0.529491in}}%
\pgfpathlineto{\pgfqpoint{3.492775in}{0.529491in}}%
\pgfpathlineto{\pgfqpoint{3.492775in}{0.532440in}}%
\pgfpathlineto{\pgfqpoint{3.497316in}{0.532440in}}%
\pgfpathlineto{\pgfqpoint{3.497316in}{0.529491in}}%
\pgfpathmoveto{\pgfqpoint{3.492775in}{0.532440in}}%
\pgfpathlineto{\pgfqpoint{3.492775in}{0.532440in}}%
\pgfpathlineto{\pgfqpoint{3.492775in}{0.535389in}}%
\pgfpathlineto{\pgfqpoint{3.497316in}{0.535389in}}%
\pgfpathlineto{\pgfqpoint{3.497316in}{0.532440in}}%
\pgfpathmoveto{\pgfqpoint{3.497316in}{0.529491in}}%
\pgfpathlineto{\pgfqpoint{3.497316in}{0.529491in}}%
\pgfpathlineto{\pgfqpoint{3.497316in}{0.532440in}}%
\pgfpathlineto{\pgfqpoint{3.501857in}{0.532440in}}%
\pgfpathlineto{\pgfqpoint{3.501857in}{0.529491in}}%
\pgfpathmoveto{\pgfqpoint{3.497316in}{0.532440in}}%
\pgfpathlineto{\pgfqpoint{3.497316in}{0.532440in}}%
\pgfpathlineto{\pgfqpoint{3.497316in}{0.535389in}}%
\pgfpathlineto{\pgfqpoint{3.501857in}{0.535389in}}%
\pgfpathlineto{\pgfqpoint{3.501857in}{0.532440in}}%
\pgfpathmoveto{\pgfqpoint{3.501857in}{0.529491in}}%
\pgfpathlineto{\pgfqpoint{3.501857in}{0.529491in}}%
\pgfpathlineto{\pgfqpoint{3.501857in}{0.532440in}}%
\pgfpathlineto{\pgfqpoint{3.506398in}{0.532440in}}%
\pgfpathlineto{\pgfqpoint{3.506398in}{0.529491in}}%
\pgfpathmoveto{\pgfqpoint{3.501857in}{0.532440in}}%
\pgfpathlineto{\pgfqpoint{3.501857in}{0.532440in}}%
\pgfpathlineto{\pgfqpoint{3.501857in}{0.535389in}}%
\pgfpathlineto{\pgfqpoint{3.506398in}{0.535389in}}%
\pgfpathlineto{\pgfqpoint{3.506398in}{0.532440in}}%
\pgfpathmoveto{\pgfqpoint{3.506398in}{0.532440in}}%
\pgfpathlineto{\pgfqpoint{3.506398in}{0.532440in}}%
\pgfpathlineto{\pgfqpoint{3.506398in}{0.535389in}}%
\pgfpathlineto{\pgfqpoint{3.510939in}{0.535389in}}%
\pgfpathlineto{\pgfqpoint{3.510939in}{0.532440in}}%
\pgfpathmoveto{\pgfqpoint{3.501857in}{0.535389in}}%
\pgfpathlineto{\pgfqpoint{3.501857in}{0.535389in}}%
\pgfpathlineto{\pgfqpoint{3.501857in}{0.538338in}}%
\pgfpathlineto{\pgfqpoint{3.506398in}{0.538338in}}%
\pgfpathlineto{\pgfqpoint{3.506398in}{0.535389in}}%
\pgfpathmoveto{\pgfqpoint{3.501857in}{0.538338in}}%
\pgfpathlineto{\pgfqpoint{3.501857in}{0.538338in}}%
\pgfpathlineto{\pgfqpoint{3.501857in}{0.541287in}}%
\pgfpathlineto{\pgfqpoint{3.506398in}{0.541287in}}%
\pgfpathlineto{\pgfqpoint{3.506398in}{0.538338in}}%
\pgfpathmoveto{\pgfqpoint{3.506398in}{0.535389in}}%
\pgfpathlineto{\pgfqpoint{3.506398in}{0.535389in}}%
\pgfpathlineto{\pgfqpoint{3.506398in}{0.538338in}}%
\pgfpathlineto{\pgfqpoint{3.510939in}{0.538338in}}%
\pgfpathlineto{\pgfqpoint{3.510939in}{0.535389in}}%
\pgfpathmoveto{\pgfqpoint{3.506398in}{0.538338in}}%
\pgfpathlineto{\pgfqpoint{3.506398in}{0.538338in}}%
\pgfpathlineto{\pgfqpoint{3.506398in}{0.541287in}}%
\pgfpathlineto{\pgfqpoint{3.510939in}{0.541287in}}%
\pgfpathlineto{\pgfqpoint{3.510939in}{0.538338in}}%
\pgfpathmoveto{\pgfqpoint{3.392873in}{1.720979in}}%
\pgfpathlineto{\pgfqpoint{3.392873in}{1.720979in}}%
\pgfpathlineto{\pgfqpoint{3.392873in}{1.723928in}}%
\pgfpathlineto{\pgfqpoint{3.397414in}{1.723928in}}%
\pgfpathlineto{\pgfqpoint{3.397414in}{1.720979in}}%
\pgfpathmoveto{\pgfqpoint{3.392873in}{1.723928in}}%
\pgfpathlineto{\pgfqpoint{3.392873in}{1.723928in}}%
\pgfpathlineto{\pgfqpoint{3.392873in}{1.726877in}}%
\pgfpathlineto{\pgfqpoint{3.397414in}{1.726877in}}%
\pgfpathlineto{\pgfqpoint{3.397414in}{1.723928in}}%
\pgfpathmoveto{\pgfqpoint{3.397414in}{1.720979in}}%
\pgfpathlineto{\pgfqpoint{3.397414in}{1.720979in}}%
\pgfpathlineto{\pgfqpoint{3.397414in}{1.723928in}}%
\pgfpathlineto{\pgfqpoint{3.401955in}{1.723928in}}%
\pgfpathlineto{\pgfqpoint{3.401955in}{1.720979in}}%
\pgfpathmoveto{\pgfqpoint{3.397414in}{1.723928in}}%
\pgfpathlineto{\pgfqpoint{3.397414in}{1.723928in}}%
\pgfpathlineto{\pgfqpoint{3.397414in}{1.726877in}}%
\pgfpathlineto{\pgfqpoint{3.401955in}{1.726877in}}%
\pgfpathlineto{\pgfqpoint{3.401955in}{1.723928in}}%
\pgfpathmoveto{\pgfqpoint{3.420119in}{1.697384in}}%
\pgfpathlineto{\pgfqpoint{3.420119in}{1.697384in}}%
\pgfpathlineto{\pgfqpoint{3.420119in}{1.700333in}}%
\pgfpathlineto{\pgfqpoint{3.424660in}{1.700333in}}%
\pgfpathlineto{\pgfqpoint{3.424660in}{1.697384in}}%
\pgfpathmoveto{\pgfqpoint{3.420119in}{1.700333in}}%
\pgfpathlineto{\pgfqpoint{3.420119in}{1.700333in}}%
\pgfpathlineto{\pgfqpoint{3.420119in}{1.703283in}}%
\pgfpathlineto{\pgfqpoint{3.424660in}{1.703283in}}%
\pgfpathlineto{\pgfqpoint{3.424660in}{1.700333in}}%
\pgfpathmoveto{\pgfqpoint{3.424660in}{1.697384in}}%
\pgfpathlineto{\pgfqpoint{3.424660in}{1.697384in}}%
\pgfpathlineto{\pgfqpoint{3.424660in}{1.700333in}}%
\pgfpathlineto{\pgfqpoint{3.429201in}{1.700333in}}%
\pgfpathlineto{\pgfqpoint{3.429201in}{1.697384in}}%
\pgfpathmoveto{\pgfqpoint{3.424660in}{1.700333in}}%
\pgfpathlineto{\pgfqpoint{3.424660in}{1.700333in}}%
\pgfpathlineto{\pgfqpoint{3.424660in}{1.703283in}}%
\pgfpathlineto{\pgfqpoint{3.429201in}{1.703283in}}%
\pgfpathlineto{\pgfqpoint{3.429201in}{1.700333in}}%
\pgfpathmoveto{\pgfqpoint{3.429201in}{1.691485in}}%
\pgfpathlineto{\pgfqpoint{3.429201in}{1.691485in}}%
\pgfpathlineto{\pgfqpoint{3.429201in}{1.694435in}}%
\pgfpathlineto{\pgfqpoint{3.433742in}{1.694435in}}%
\pgfpathlineto{\pgfqpoint{3.433742in}{1.691485in}}%
\pgfpathmoveto{\pgfqpoint{3.429201in}{1.694435in}}%
\pgfpathlineto{\pgfqpoint{3.429201in}{1.694435in}}%
\pgfpathlineto{\pgfqpoint{3.429201in}{1.697384in}}%
\pgfpathlineto{\pgfqpoint{3.433742in}{1.697384in}}%
\pgfpathlineto{\pgfqpoint{3.433742in}{1.694435in}}%
\pgfpathmoveto{\pgfqpoint{3.433742in}{1.691485in}}%
\pgfpathlineto{\pgfqpoint{3.433742in}{1.691485in}}%
\pgfpathlineto{\pgfqpoint{3.433742in}{1.694435in}}%
\pgfpathlineto{\pgfqpoint{3.438283in}{1.694435in}}%
\pgfpathlineto{\pgfqpoint{3.438283in}{1.691485in}}%
\pgfpathmoveto{\pgfqpoint{3.433742in}{1.694435in}}%
\pgfpathlineto{\pgfqpoint{3.433742in}{1.694435in}}%
\pgfpathlineto{\pgfqpoint{3.433742in}{1.697384in}}%
\pgfpathlineto{\pgfqpoint{3.438283in}{1.697384in}}%
\pgfpathlineto{\pgfqpoint{3.438283in}{1.694435in}}%
\pgfpathmoveto{\pgfqpoint{3.429201in}{1.697384in}}%
\pgfpathlineto{\pgfqpoint{3.429201in}{1.697384in}}%
\pgfpathlineto{\pgfqpoint{3.429201in}{1.700333in}}%
\pgfpathlineto{\pgfqpoint{3.433742in}{1.700333in}}%
\pgfpathlineto{\pgfqpoint{3.433742in}{1.697384in}}%
\pgfpathmoveto{\pgfqpoint{3.429201in}{1.700333in}}%
\pgfpathlineto{\pgfqpoint{3.429201in}{1.700333in}}%
\pgfpathlineto{\pgfqpoint{3.429201in}{1.703283in}}%
\pgfpathlineto{\pgfqpoint{3.433742in}{1.703283in}}%
\pgfpathlineto{\pgfqpoint{3.433742in}{1.700333in}}%
\pgfpathmoveto{\pgfqpoint{3.433742in}{1.697384in}}%
\pgfpathlineto{\pgfqpoint{3.433742in}{1.697384in}}%
\pgfpathlineto{\pgfqpoint{3.433742in}{1.700333in}}%
\pgfpathlineto{\pgfqpoint{3.438283in}{1.700333in}}%
\pgfpathlineto{\pgfqpoint{3.438283in}{1.697384in}}%
\pgfpathmoveto{\pgfqpoint{3.411037in}{1.709181in}}%
\pgfpathlineto{\pgfqpoint{3.411037in}{1.709181in}}%
\pgfpathlineto{\pgfqpoint{3.411037in}{1.712131in}}%
\pgfpathlineto{\pgfqpoint{3.415578in}{1.712131in}}%
\pgfpathlineto{\pgfqpoint{3.415578in}{1.709181in}}%
\pgfpathmoveto{\pgfqpoint{3.411037in}{1.712131in}}%
\pgfpathlineto{\pgfqpoint{3.411037in}{1.712131in}}%
\pgfpathlineto{\pgfqpoint{3.411037in}{1.715080in}}%
\pgfpathlineto{\pgfqpoint{3.415578in}{1.715080in}}%
\pgfpathlineto{\pgfqpoint{3.415578in}{1.712131in}}%
\pgfpathmoveto{\pgfqpoint{3.415578in}{1.709181in}}%
\pgfpathlineto{\pgfqpoint{3.415578in}{1.709181in}}%
\pgfpathlineto{\pgfqpoint{3.415578in}{1.712131in}}%
\pgfpathlineto{\pgfqpoint{3.420119in}{1.712131in}}%
\pgfpathlineto{\pgfqpoint{3.420119in}{1.709181in}}%
\pgfpathmoveto{\pgfqpoint{3.415578in}{1.712131in}}%
\pgfpathlineto{\pgfqpoint{3.415578in}{1.712131in}}%
\pgfpathlineto{\pgfqpoint{3.415578in}{1.715080in}}%
\pgfpathlineto{\pgfqpoint{3.420119in}{1.715080in}}%
\pgfpathlineto{\pgfqpoint{3.420119in}{1.712131in}}%
\pgfpathmoveto{\pgfqpoint{3.401955in}{1.715080in}}%
\pgfpathlineto{\pgfqpoint{3.401955in}{1.715080in}}%
\pgfpathlineto{\pgfqpoint{3.401955in}{1.718029in}}%
\pgfpathlineto{\pgfqpoint{3.406496in}{1.718029in}}%
\pgfpathlineto{\pgfqpoint{3.406496in}{1.715080in}}%
\pgfpathmoveto{\pgfqpoint{3.401955in}{1.718029in}}%
\pgfpathlineto{\pgfqpoint{3.401955in}{1.718029in}}%
\pgfpathlineto{\pgfqpoint{3.401955in}{1.720979in}}%
\pgfpathlineto{\pgfqpoint{3.406496in}{1.720979in}}%
\pgfpathlineto{\pgfqpoint{3.406496in}{1.718029in}}%
\pgfpathmoveto{\pgfqpoint{3.406496in}{1.715080in}}%
\pgfpathlineto{\pgfqpoint{3.406496in}{1.715080in}}%
\pgfpathlineto{\pgfqpoint{3.406496in}{1.718029in}}%
\pgfpathlineto{\pgfqpoint{3.411037in}{1.718029in}}%
\pgfpathlineto{\pgfqpoint{3.411037in}{1.715080in}}%
\pgfpathmoveto{\pgfqpoint{3.406496in}{1.718029in}}%
\pgfpathlineto{\pgfqpoint{3.406496in}{1.718029in}}%
\pgfpathlineto{\pgfqpoint{3.406496in}{1.720979in}}%
\pgfpathlineto{\pgfqpoint{3.411037in}{1.720979in}}%
\pgfpathlineto{\pgfqpoint{3.411037in}{1.718029in}}%
\pgfpathmoveto{\pgfqpoint{3.401955in}{1.720979in}}%
\pgfpathlineto{\pgfqpoint{3.401955in}{1.720979in}}%
\pgfpathlineto{\pgfqpoint{3.401955in}{1.723928in}}%
\pgfpathlineto{\pgfqpoint{3.406496in}{1.723928in}}%
\pgfpathlineto{\pgfqpoint{3.406496in}{1.720979in}}%
\pgfpathmoveto{\pgfqpoint{3.401955in}{1.723928in}}%
\pgfpathlineto{\pgfqpoint{3.401955in}{1.723928in}}%
\pgfpathlineto{\pgfqpoint{3.401955in}{1.726877in}}%
\pgfpathlineto{\pgfqpoint{3.406496in}{1.726877in}}%
\pgfpathlineto{\pgfqpoint{3.406496in}{1.723928in}}%
\pgfpathmoveto{\pgfqpoint{3.406496in}{1.720979in}}%
\pgfpathlineto{\pgfqpoint{3.406496in}{1.720979in}}%
\pgfpathlineto{\pgfqpoint{3.406496in}{1.723928in}}%
\pgfpathlineto{\pgfqpoint{3.411037in}{1.723928in}}%
\pgfpathlineto{\pgfqpoint{3.411037in}{1.720979in}}%
\pgfpathmoveto{\pgfqpoint{3.411037in}{1.715080in}}%
\pgfpathlineto{\pgfqpoint{3.411037in}{1.715080in}}%
\pgfpathlineto{\pgfqpoint{3.411037in}{1.718029in}}%
\pgfpathlineto{\pgfqpoint{3.415578in}{1.718029in}}%
\pgfpathlineto{\pgfqpoint{3.415578in}{1.715080in}}%
\pgfpathmoveto{\pgfqpoint{3.411037in}{1.718029in}}%
\pgfpathlineto{\pgfqpoint{3.411037in}{1.718029in}}%
\pgfpathlineto{\pgfqpoint{3.411037in}{1.720979in}}%
\pgfpathlineto{\pgfqpoint{3.415578in}{1.720979in}}%
\pgfpathlineto{\pgfqpoint{3.415578in}{1.718029in}}%
\pgfpathmoveto{\pgfqpoint{3.420119in}{1.703283in}}%
\pgfpathlineto{\pgfqpoint{3.420119in}{1.703283in}}%
\pgfpathlineto{\pgfqpoint{3.420119in}{1.706232in}}%
\pgfpathlineto{\pgfqpoint{3.424660in}{1.706232in}}%
\pgfpathlineto{\pgfqpoint{3.424660in}{1.703283in}}%
\pgfpathmoveto{\pgfqpoint{3.420119in}{1.706232in}}%
\pgfpathlineto{\pgfqpoint{3.420119in}{1.706232in}}%
\pgfpathlineto{\pgfqpoint{3.420119in}{1.709181in}}%
\pgfpathlineto{\pgfqpoint{3.424660in}{1.709181in}}%
\pgfpathlineto{\pgfqpoint{3.424660in}{1.706232in}}%
\pgfpathmoveto{\pgfqpoint{3.424660in}{1.703283in}}%
\pgfpathlineto{\pgfqpoint{3.424660in}{1.703283in}}%
\pgfpathlineto{\pgfqpoint{3.424660in}{1.706232in}}%
\pgfpathlineto{\pgfqpoint{3.429201in}{1.706232in}}%
\pgfpathlineto{\pgfqpoint{3.429201in}{1.703283in}}%
\pgfpathmoveto{\pgfqpoint{3.424660in}{1.706232in}}%
\pgfpathlineto{\pgfqpoint{3.424660in}{1.706232in}}%
\pgfpathlineto{\pgfqpoint{3.424660in}{1.709181in}}%
\pgfpathlineto{\pgfqpoint{3.429201in}{1.709181in}}%
\pgfpathlineto{\pgfqpoint{3.429201in}{1.706232in}}%
\pgfpathmoveto{\pgfqpoint{3.420119in}{1.709181in}}%
\pgfpathlineto{\pgfqpoint{3.420119in}{1.709181in}}%
\pgfpathlineto{\pgfqpoint{3.420119in}{1.712131in}}%
\pgfpathlineto{\pgfqpoint{3.424660in}{1.712131in}}%
\pgfpathlineto{\pgfqpoint{3.424660in}{1.709181in}}%
\pgfpathmoveto{\pgfqpoint{3.447365in}{1.673789in}}%
\pgfpathlineto{\pgfqpoint{3.447365in}{1.673789in}}%
\pgfpathlineto{\pgfqpoint{3.447365in}{1.676739in}}%
\pgfpathlineto{\pgfqpoint{3.451906in}{1.676739in}}%
\pgfpathlineto{\pgfqpoint{3.451906in}{1.673789in}}%
\pgfpathmoveto{\pgfqpoint{3.447365in}{1.676739in}}%
\pgfpathlineto{\pgfqpoint{3.447365in}{1.676739in}}%
\pgfpathlineto{\pgfqpoint{3.447365in}{1.679688in}}%
\pgfpathlineto{\pgfqpoint{3.451906in}{1.679688in}}%
\pgfpathlineto{\pgfqpoint{3.451906in}{1.676739in}}%
\pgfpathmoveto{\pgfqpoint{3.451906in}{1.673789in}}%
\pgfpathlineto{\pgfqpoint{3.451906in}{1.673789in}}%
\pgfpathlineto{\pgfqpoint{3.451906in}{1.676739in}}%
\pgfpathlineto{\pgfqpoint{3.456447in}{1.676739in}}%
\pgfpathlineto{\pgfqpoint{3.456447in}{1.673789in}}%
\pgfpathmoveto{\pgfqpoint{3.451906in}{1.676739in}}%
\pgfpathlineto{\pgfqpoint{3.451906in}{1.676739in}}%
\pgfpathlineto{\pgfqpoint{3.451906in}{1.679688in}}%
\pgfpathlineto{\pgfqpoint{3.456447in}{1.679688in}}%
\pgfpathlineto{\pgfqpoint{3.456447in}{1.676739in}}%
\pgfpathmoveto{\pgfqpoint{3.465529in}{1.661992in}}%
\pgfpathlineto{\pgfqpoint{3.465529in}{1.661992in}}%
\pgfpathlineto{\pgfqpoint{3.465529in}{1.664942in}}%
\pgfpathlineto{\pgfqpoint{3.470070in}{1.664942in}}%
\pgfpathlineto{\pgfqpoint{3.470070in}{1.661992in}}%
\pgfpathmoveto{\pgfqpoint{3.465529in}{1.664942in}}%
\pgfpathlineto{\pgfqpoint{3.465529in}{1.664942in}}%
\pgfpathlineto{\pgfqpoint{3.465529in}{1.667891in}}%
\pgfpathlineto{\pgfqpoint{3.470070in}{1.667891in}}%
\pgfpathlineto{\pgfqpoint{3.470070in}{1.664942in}}%
\pgfpathmoveto{\pgfqpoint{3.470070in}{1.661992in}}%
\pgfpathlineto{\pgfqpoint{3.470070in}{1.661992in}}%
\pgfpathlineto{\pgfqpoint{3.470070in}{1.664942in}}%
\pgfpathlineto{\pgfqpoint{3.474611in}{1.664942in}}%
\pgfpathlineto{\pgfqpoint{3.474611in}{1.661992in}}%
\pgfpathmoveto{\pgfqpoint{3.470070in}{1.664942in}}%
\pgfpathlineto{\pgfqpoint{3.470070in}{1.664942in}}%
\pgfpathlineto{\pgfqpoint{3.470070in}{1.667891in}}%
\pgfpathlineto{\pgfqpoint{3.474611in}{1.667891in}}%
\pgfpathlineto{\pgfqpoint{3.474611in}{1.664942in}}%
\pgfpathmoveto{\pgfqpoint{3.456447in}{1.667891in}}%
\pgfpathlineto{\pgfqpoint{3.456447in}{1.667891in}}%
\pgfpathlineto{\pgfqpoint{3.456447in}{1.670840in}}%
\pgfpathlineto{\pgfqpoint{3.460988in}{1.670840in}}%
\pgfpathlineto{\pgfqpoint{3.460988in}{1.667891in}}%
\pgfpathmoveto{\pgfqpoint{3.456447in}{1.670840in}}%
\pgfpathlineto{\pgfqpoint{3.456447in}{1.670840in}}%
\pgfpathlineto{\pgfqpoint{3.456447in}{1.673789in}}%
\pgfpathlineto{\pgfqpoint{3.460988in}{1.673789in}}%
\pgfpathlineto{\pgfqpoint{3.460988in}{1.670840in}}%
\pgfpathmoveto{\pgfqpoint{3.460988in}{1.667891in}}%
\pgfpathlineto{\pgfqpoint{3.460988in}{1.667891in}}%
\pgfpathlineto{\pgfqpoint{3.460988in}{1.670840in}}%
\pgfpathlineto{\pgfqpoint{3.465529in}{1.670840in}}%
\pgfpathlineto{\pgfqpoint{3.465529in}{1.667891in}}%
\pgfpathmoveto{\pgfqpoint{3.460988in}{1.670840in}}%
\pgfpathlineto{\pgfqpoint{3.460988in}{1.670840in}}%
\pgfpathlineto{\pgfqpoint{3.460988in}{1.673789in}}%
\pgfpathlineto{\pgfqpoint{3.465529in}{1.673789in}}%
\pgfpathlineto{\pgfqpoint{3.465529in}{1.670840in}}%
\pgfpathmoveto{\pgfqpoint{3.456447in}{1.673789in}}%
\pgfpathlineto{\pgfqpoint{3.456447in}{1.673789in}}%
\pgfpathlineto{\pgfqpoint{3.456447in}{1.676739in}}%
\pgfpathlineto{\pgfqpoint{3.460988in}{1.676739in}}%
\pgfpathlineto{\pgfqpoint{3.460988in}{1.673789in}}%
\pgfpathmoveto{\pgfqpoint{3.456447in}{1.676739in}}%
\pgfpathlineto{\pgfqpoint{3.456447in}{1.676739in}}%
\pgfpathlineto{\pgfqpoint{3.456447in}{1.679688in}}%
\pgfpathlineto{\pgfqpoint{3.460988in}{1.679688in}}%
\pgfpathlineto{\pgfqpoint{3.460988in}{1.676739in}}%
\pgfpathmoveto{\pgfqpoint{3.460988in}{1.673789in}}%
\pgfpathlineto{\pgfqpoint{3.460988in}{1.673789in}}%
\pgfpathlineto{\pgfqpoint{3.460988in}{1.676739in}}%
\pgfpathlineto{\pgfqpoint{3.465529in}{1.676739in}}%
\pgfpathlineto{\pgfqpoint{3.465529in}{1.673789in}}%
\pgfpathmoveto{\pgfqpoint{3.465529in}{1.667891in}}%
\pgfpathlineto{\pgfqpoint{3.465529in}{1.667891in}}%
\pgfpathlineto{\pgfqpoint{3.465529in}{1.670840in}}%
\pgfpathlineto{\pgfqpoint{3.470070in}{1.670840in}}%
\pgfpathlineto{\pgfqpoint{3.470070in}{1.667891in}}%
\pgfpathmoveto{\pgfqpoint{3.465529in}{1.670840in}}%
\pgfpathlineto{\pgfqpoint{3.465529in}{1.670840in}}%
\pgfpathlineto{\pgfqpoint{3.465529in}{1.673789in}}%
\pgfpathlineto{\pgfqpoint{3.470070in}{1.673789in}}%
\pgfpathlineto{\pgfqpoint{3.470070in}{1.670840in}}%
\pgfpathmoveto{\pgfqpoint{3.474611in}{1.650195in}}%
\pgfpathlineto{\pgfqpoint{3.474611in}{1.650195in}}%
\pgfpathlineto{\pgfqpoint{3.474611in}{1.653144in}}%
\pgfpathlineto{\pgfqpoint{3.479152in}{1.653144in}}%
\pgfpathlineto{\pgfqpoint{3.479152in}{1.650195in}}%
\pgfpathmoveto{\pgfqpoint{3.474611in}{1.653144in}}%
\pgfpathlineto{\pgfqpoint{3.474611in}{1.653144in}}%
\pgfpathlineto{\pgfqpoint{3.474611in}{1.656094in}}%
\pgfpathlineto{\pgfqpoint{3.479152in}{1.656094in}}%
\pgfpathlineto{\pgfqpoint{3.479152in}{1.653144in}}%
\pgfpathmoveto{\pgfqpoint{3.479152in}{1.650195in}}%
\pgfpathlineto{\pgfqpoint{3.479152in}{1.650195in}}%
\pgfpathlineto{\pgfqpoint{3.479152in}{1.653144in}}%
\pgfpathlineto{\pgfqpoint{3.483693in}{1.653144in}}%
\pgfpathlineto{\pgfqpoint{3.483693in}{1.650195in}}%
\pgfpathmoveto{\pgfqpoint{3.479152in}{1.653144in}}%
\pgfpathlineto{\pgfqpoint{3.479152in}{1.653144in}}%
\pgfpathlineto{\pgfqpoint{3.479152in}{1.656094in}}%
\pgfpathlineto{\pgfqpoint{3.483693in}{1.656094in}}%
\pgfpathlineto{\pgfqpoint{3.483693in}{1.653144in}}%
\pgfpathmoveto{\pgfqpoint{3.483693in}{1.644296in}}%
\pgfpathlineto{\pgfqpoint{3.483693in}{1.644296in}}%
\pgfpathlineto{\pgfqpoint{3.483693in}{1.647246in}}%
\pgfpathlineto{\pgfqpoint{3.488234in}{1.647246in}}%
\pgfpathlineto{\pgfqpoint{3.488234in}{1.644296in}}%
\pgfpathmoveto{\pgfqpoint{3.483693in}{1.647246in}}%
\pgfpathlineto{\pgfqpoint{3.483693in}{1.647246in}}%
\pgfpathlineto{\pgfqpoint{3.483693in}{1.650195in}}%
\pgfpathlineto{\pgfqpoint{3.488234in}{1.650195in}}%
\pgfpathlineto{\pgfqpoint{3.488234in}{1.647246in}}%
\pgfpathmoveto{\pgfqpoint{3.488234in}{1.644296in}}%
\pgfpathlineto{\pgfqpoint{3.488234in}{1.644296in}}%
\pgfpathlineto{\pgfqpoint{3.488234in}{1.647246in}}%
\pgfpathlineto{\pgfqpoint{3.492775in}{1.647246in}}%
\pgfpathlineto{\pgfqpoint{3.492775in}{1.644296in}}%
\pgfpathmoveto{\pgfqpoint{3.488234in}{1.647246in}}%
\pgfpathlineto{\pgfqpoint{3.488234in}{1.647246in}}%
\pgfpathlineto{\pgfqpoint{3.488234in}{1.650195in}}%
\pgfpathlineto{\pgfqpoint{3.492775in}{1.650195in}}%
\pgfpathlineto{\pgfqpoint{3.492775in}{1.647246in}}%
\pgfpathmoveto{\pgfqpoint{3.483693in}{1.650195in}}%
\pgfpathlineto{\pgfqpoint{3.483693in}{1.650195in}}%
\pgfpathlineto{\pgfqpoint{3.483693in}{1.653144in}}%
\pgfpathlineto{\pgfqpoint{3.488234in}{1.653144in}}%
\pgfpathlineto{\pgfqpoint{3.488234in}{1.650195in}}%
\pgfpathmoveto{\pgfqpoint{3.483693in}{1.653144in}}%
\pgfpathlineto{\pgfqpoint{3.483693in}{1.653144in}}%
\pgfpathlineto{\pgfqpoint{3.483693in}{1.656094in}}%
\pgfpathlineto{\pgfqpoint{3.488234in}{1.656094in}}%
\pgfpathlineto{\pgfqpoint{3.488234in}{1.653144in}}%
\pgfpathmoveto{\pgfqpoint{3.488234in}{1.650195in}}%
\pgfpathlineto{\pgfqpoint{3.488234in}{1.650195in}}%
\pgfpathlineto{\pgfqpoint{3.488234in}{1.653144in}}%
\pgfpathlineto{\pgfqpoint{3.492775in}{1.653144in}}%
\pgfpathlineto{\pgfqpoint{3.492775in}{1.650195in}}%
\pgfpathmoveto{\pgfqpoint{3.492775in}{1.638398in}}%
\pgfpathlineto{\pgfqpoint{3.492775in}{1.638398in}}%
\pgfpathlineto{\pgfqpoint{3.492775in}{1.641347in}}%
\pgfpathlineto{\pgfqpoint{3.497316in}{1.641347in}}%
\pgfpathlineto{\pgfqpoint{3.497316in}{1.638398in}}%
\pgfpathmoveto{\pgfqpoint{3.492775in}{1.641347in}}%
\pgfpathlineto{\pgfqpoint{3.492775in}{1.641347in}}%
\pgfpathlineto{\pgfqpoint{3.492775in}{1.644296in}}%
\pgfpathlineto{\pgfqpoint{3.497316in}{1.644296in}}%
\pgfpathlineto{\pgfqpoint{3.497316in}{1.641347in}}%
\pgfpathmoveto{\pgfqpoint{3.497316in}{1.638398in}}%
\pgfpathlineto{\pgfqpoint{3.497316in}{1.638398in}}%
\pgfpathlineto{\pgfqpoint{3.497316in}{1.641347in}}%
\pgfpathlineto{\pgfqpoint{3.501857in}{1.641347in}}%
\pgfpathlineto{\pgfqpoint{3.501857in}{1.638398in}}%
\pgfpathmoveto{\pgfqpoint{3.497316in}{1.641347in}}%
\pgfpathlineto{\pgfqpoint{3.497316in}{1.641347in}}%
\pgfpathlineto{\pgfqpoint{3.497316in}{1.644296in}}%
\pgfpathlineto{\pgfqpoint{3.501857in}{1.644296in}}%
\pgfpathlineto{\pgfqpoint{3.501857in}{1.641347in}}%
\pgfpathmoveto{\pgfqpoint{3.501857in}{1.632499in}}%
\pgfpathlineto{\pgfqpoint{3.501857in}{1.632499in}}%
\pgfpathlineto{\pgfqpoint{3.501857in}{1.635448in}}%
\pgfpathlineto{\pgfqpoint{3.506398in}{1.635448in}}%
\pgfpathlineto{\pgfqpoint{3.506398in}{1.632499in}}%
\pgfpathmoveto{\pgfqpoint{3.501857in}{1.635448in}}%
\pgfpathlineto{\pgfqpoint{3.501857in}{1.635448in}}%
\pgfpathlineto{\pgfqpoint{3.501857in}{1.638398in}}%
\pgfpathlineto{\pgfqpoint{3.506398in}{1.638398in}}%
\pgfpathlineto{\pgfqpoint{3.506398in}{1.635448in}}%
\pgfpathmoveto{\pgfqpoint{3.506398in}{1.632499in}}%
\pgfpathlineto{\pgfqpoint{3.506398in}{1.632499in}}%
\pgfpathlineto{\pgfqpoint{3.506398in}{1.635448in}}%
\pgfpathlineto{\pgfqpoint{3.510939in}{1.635448in}}%
\pgfpathlineto{\pgfqpoint{3.510939in}{1.632499in}}%
\pgfpathmoveto{\pgfqpoint{3.506398in}{1.635448in}}%
\pgfpathlineto{\pgfqpoint{3.506398in}{1.635448in}}%
\pgfpathlineto{\pgfqpoint{3.506398in}{1.638398in}}%
\pgfpathlineto{\pgfqpoint{3.510939in}{1.638398in}}%
\pgfpathlineto{\pgfqpoint{3.510939in}{1.635448in}}%
\pgfpathmoveto{\pgfqpoint{3.501857in}{1.638398in}}%
\pgfpathlineto{\pgfqpoint{3.501857in}{1.638398in}}%
\pgfpathlineto{\pgfqpoint{3.501857in}{1.641347in}}%
\pgfpathlineto{\pgfqpoint{3.506398in}{1.641347in}}%
\pgfpathlineto{\pgfqpoint{3.506398in}{1.638398in}}%
\pgfpathmoveto{\pgfqpoint{3.492775in}{1.644296in}}%
\pgfpathlineto{\pgfqpoint{3.492775in}{1.644296in}}%
\pgfpathlineto{\pgfqpoint{3.492775in}{1.647246in}}%
\pgfpathlineto{\pgfqpoint{3.497316in}{1.647246in}}%
\pgfpathlineto{\pgfqpoint{3.497316in}{1.644296in}}%
\pgfpathmoveto{\pgfqpoint{3.492775in}{1.647246in}}%
\pgfpathlineto{\pgfqpoint{3.492775in}{1.647246in}}%
\pgfpathlineto{\pgfqpoint{3.492775in}{1.650195in}}%
\pgfpathlineto{\pgfqpoint{3.497316in}{1.650195in}}%
\pgfpathlineto{\pgfqpoint{3.497316in}{1.647246in}}%
\pgfpathmoveto{\pgfqpoint{3.474611in}{1.656094in}}%
\pgfpathlineto{\pgfqpoint{3.474611in}{1.656094in}}%
\pgfpathlineto{\pgfqpoint{3.474611in}{1.659043in}}%
\pgfpathlineto{\pgfqpoint{3.479152in}{1.659043in}}%
\pgfpathlineto{\pgfqpoint{3.479152in}{1.656094in}}%
\pgfpathmoveto{\pgfqpoint{3.474611in}{1.659043in}}%
\pgfpathlineto{\pgfqpoint{3.474611in}{1.659043in}}%
\pgfpathlineto{\pgfqpoint{3.474611in}{1.661992in}}%
\pgfpathlineto{\pgfqpoint{3.479152in}{1.661992in}}%
\pgfpathlineto{\pgfqpoint{3.479152in}{1.659043in}}%
\pgfpathmoveto{\pgfqpoint{3.479152in}{1.656094in}}%
\pgfpathlineto{\pgfqpoint{3.479152in}{1.656094in}}%
\pgfpathlineto{\pgfqpoint{3.479152in}{1.659043in}}%
\pgfpathlineto{\pgfqpoint{3.483693in}{1.659043in}}%
\pgfpathlineto{\pgfqpoint{3.483693in}{1.656094in}}%
\pgfpathmoveto{\pgfqpoint{3.479152in}{1.659043in}}%
\pgfpathlineto{\pgfqpoint{3.479152in}{1.659043in}}%
\pgfpathlineto{\pgfqpoint{3.479152in}{1.661992in}}%
\pgfpathlineto{\pgfqpoint{3.483693in}{1.661992in}}%
\pgfpathlineto{\pgfqpoint{3.483693in}{1.659043in}}%
\pgfpathmoveto{\pgfqpoint{3.474611in}{1.661992in}}%
\pgfpathlineto{\pgfqpoint{3.474611in}{1.661992in}}%
\pgfpathlineto{\pgfqpoint{3.474611in}{1.664942in}}%
\pgfpathlineto{\pgfqpoint{3.479152in}{1.664942in}}%
\pgfpathlineto{\pgfqpoint{3.479152in}{1.661992in}}%
\pgfpathmoveto{\pgfqpoint{3.438283in}{1.685587in}}%
\pgfpathlineto{\pgfqpoint{3.438283in}{1.685587in}}%
\pgfpathlineto{\pgfqpoint{3.438283in}{1.688536in}}%
\pgfpathlineto{\pgfqpoint{3.442824in}{1.688536in}}%
\pgfpathlineto{\pgfqpoint{3.442824in}{1.685587in}}%
\pgfpathmoveto{\pgfqpoint{3.438283in}{1.688536in}}%
\pgfpathlineto{\pgfqpoint{3.438283in}{1.688536in}}%
\pgfpathlineto{\pgfqpoint{3.438283in}{1.691485in}}%
\pgfpathlineto{\pgfqpoint{3.442824in}{1.691485in}}%
\pgfpathlineto{\pgfqpoint{3.442824in}{1.688536in}}%
\pgfpathmoveto{\pgfqpoint{3.442824in}{1.685587in}}%
\pgfpathlineto{\pgfqpoint{3.442824in}{1.685587in}}%
\pgfpathlineto{\pgfqpoint{3.442824in}{1.688536in}}%
\pgfpathlineto{\pgfqpoint{3.447365in}{1.688536in}}%
\pgfpathlineto{\pgfqpoint{3.447365in}{1.685587in}}%
\pgfpathmoveto{\pgfqpoint{3.442824in}{1.688536in}}%
\pgfpathlineto{\pgfqpoint{3.442824in}{1.688536in}}%
\pgfpathlineto{\pgfqpoint{3.442824in}{1.691485in}}%
\pgfpathlineto{\pgfqpoint{3.447365in}{1.691485in}}%
\pgfpathlineto{\pgfqpoint{3.447365in}{1.688536in}}%
\pgfpathmoveto{\pgfqpoint{3.447365in}{1.679688in}}%
\pgfpathlineto{\pgfqpoint{3.447365in}{1.679688in}}%
\pgfpathlineto{\pgfqpoint{3.447365in}{1.682637in}}%
\pgfpathlineto{\pgfqpoint{3.451906in}{1.682637in}}%
\pgfpathlineto{\pgfqpoint{3.451906in}{1.679688in}}%
\pgfpathmoveto{\pgfqpoint{3.447365in}{1.682637in}}%
\pgfpathlineto{\pgfqpoint{3.447365in}{1.682637in}}%
\pgfpathlineto{\pgfqpoint{3.447365in}{1.685587in}}%
\pgfpathlineto{\pgfqpoint{3.451906in}{1.685587in}}%
\pgfpathlineto{\pgfqpoint{3.451906in}{1.682637in}}%
\pgfpathmoveto{\pgfqpoint{3.451906in}{1.679688in}}%
\pgfpathlineto{\pgfqpoint{3.451906in}{1.679688in}}%
\pgfpathlineto{\pgfqpoint{3.451906in}{1.682637in}}%
\pgfpathlineto{\pgfqpoint{3.456447in}{1.682637in}}%
\pgfpathlineto{\pgfqpoint{3.456447in}{1.679688in}}%
\pgfpathmoveto{\pgfqpoint{3.451906in}{1.682637in}}%
\pgfpathlineto{\pgfqpoint{3.451906in}{1.682637in}}%
\pgfpathlineto{\pgfqpoint{3.451906in}{1.685587in}}%
\pgfpathlineto{\pgfqpoint{3.456447in}{1.685587in}}%
\pgfpathlineto{\pgfqpoint{3.456447in}{1.682637in}}%
\pgfpathmoveto{\pgfqpoint{3.447365in}{1.685587in}}%
\pgfpathlineto{\pgfqpoint{3.447365in}{1.685587in}}%
\pgfpathlineto{\pgfqpoint{3.447365in}{1.688536in}}%
\pgfpathlineto{\pgfqpoint{3.451906in}{1.688536in}}%
\pgfpathlineto{\pgfqpoint{3.451906in}{1.685587in}}%
\pgfpathmoveto{\pgfqpoint{3.438283in}{1.691485in}}%
\pgfpathlineto{\pgfqpoint{3.438283in}{1.691485in}}%
\pgfpathlineto{\pgfqpoint{3.438283in}{1.694435in}}%
\pgfpathlineto{\pgfqpoint{3.442824in}{1.694435in}}%
\pgfpathlineto{\pgfqpoint{3.442824in}{1.691485in}}%
\pgfpathmoveto{\pgfqpoint{3.438283in}{1.694435in}}%
\pgfpathlineto{\pgfqpoint{3.438283in}{1.694435in}}%
\pgfpathlineto{\pgfqpoint{3.438283in}{1.697384in}}%
\pgfpathlineto{\pgfqpoint{3.442824in}{1.697384in}}%
\pgfpathlineto{\pgfqpoint{3.442824in}{1.694435in}}%
\pgfpathmoveto{\pgfqpoint{3.365628in}{1.744572in}}%
\pgfpathlineto{\pgfqpoint{3.365628in}{1.744572in}}%
\pgfpathlineto{\pgfqpoint{3.365628in}{1.747521in}}%
\pgfpathlineto{\pgfqpoint{3.370169in}{1.747521in}}%
\pgfpathlineto{\pgfqpoint{3.370169in}{1.744572in}}%
\pgfpathmoveto{\pgfqpoint{3.365628in}{1.747521in}}%
\pgfpathlineto{\pgfqpoint{3.365628in}{1.747521in}}%
\pgfpathlineto{\pgfqpoint{3.365628in}{1.750470in}}%
\pgfpathlineto{\pgfqpoint{3.370169in}{1.750470in}}%
\pgfpathlineto{\pgfqpoint{3.370169in}{1.747521in}}%
\pgfpathmoveto{\pgfqpoint{3.370169in}{1.744572in}}%
\pgfpathlineto{\pgfqpoint{3.370169in}{1.744572in}}%
\pgfpathlineto{\pgfqpoint{3.370169in}{1.747521in}}%
\pgfpathlineto{\pgfqpoint{3.374710in}{1.747521in}}%
\pgfpathlineto{\pgfqpoint{3.374710in}{1.744572in}}%
\pgfpathmoveto{\pgfqpoint{3.370169in}{1.747521in}}%
\pgfpathlineto{\pgfqpoint{3.370169in}{1.747521in}}%
\pgfpathlineto{\pgfqpoint{3.370169in}{1.750470in}}%
\pgfpathlineto{\pgfqpoint{3.374710in}{1.750470in}}%
\pgfpathlineto{\pgfqpoint{3.374710in}{1.747521in}}%
\pgfpathmoveto{\pgfqpoint{3.374710in}{1.738674in}}%
\pgfpathlineto{\pgfqpoint{3.374710in}{1.738674in}}%
\pgfpathlineto{\pgfqpoint{3.374710in}{1.741623in}}%
\pgfpathlineto{\pgfqpoint{3.379251in}{1.741623in}}%
\pgfpathlineto{\pgfqpoint{3.379251in}{1.738674in}}%
\pgfpathmoveto{\pgfqpoint{3.374710in}{1.741623in}}%
\pgfpathlineto{\pgfqpoint{3.374710in}{1.741623in}}%
\pgfpathlineto{\pgfqpoint{3.374710in}{1.744572in}}%
\pgfpathlineto{\pgfqpoint{3.379251in}{1.744572in}}%
\pgfpathlineto{\pgfqpoint{3.379251in}{1.741623in}}%
\pgfpathmoveto{\pgfqpoint{3.379251in}{1.738674in}}%
\pgfpathlineto{\pgfqpoint{3.379251in}{1.738674in}}%
\pgfpathlineto{\pgfqpoint{3.379251in}{1.741623in}}%
\pgfpathlineto{\pgfqpoint{3.383792in}{1.741623in}}%
\pgfpathlineto{\pgfqpoint{3.383792in}{1.738674in}}%
\pgfpathmoveto{\pgfqpoint{3.379251in}{1.741623in}}%
\pgfpathlineto{\pgfqpoint{3.379251in}{1.741623in}}%
\pgfpathlineto{\pgfqpoint{3.379251in}{1.744572in}}%
\pgfpathlineto{\pgfqpoint{3.383792in}{1.744572in}}%
\pgfpathlineto{\pgfqpoint{3.383792in}{1.741623in}}%
\pgfpathmoveto{\pgfqpoint{3.374710in}{1.744572in}}%
\pgfpathlineto{\pgfqpoint{3.374710in}{1.744572in}}%
\pgfpathlineto{\pgfqpoint{3.374710in}{1.747521in}}%
\pgfpathlineto{\pgfqpoint{3.379251in}{1.747521in}}%
\pgfpathlineto{\pgfqpoint{3.379251in}{1.744572in}}%
\pgfpathmoveto{\pgfqpoint{3.374710in}{1.747521in}}%
\pgfpathlineto{\pgfqpoint{3.374710in}{1.747521in}}%
\pgfpathlineto{\pgfqpoint{3.374710in}{1.750470in}}%
\pgfpathlineto{\pgfqpoint{3.379251in}{1.750470in}}%
\pgfpathlineto{\pgfqpoint{3.379251in}{1.747521in}}%
\pgfpathmoveto{\pgfqpoint{3.379251in}{1.744572in}}%
\pgfpathlineto{\pgfqpoint{3.379251in}{1.744572in}}%
\pgfpathlineto{\pgfqpoint{3.379251in}{1.747521in}}%
\pgfpathlineto{\pgfqpoint{3.383792in}{1.747521in}}%
\pgfpathlineto{\pgfqpoint{3.383792in}{1.744572in}}%
\pgfpathmoveto{\pgfqpoint{3.383792in}{1.732775in}}%
\pgfpathlineto{\pgfqpoint{3.383792in}{1.732775in}}%
\pgfpathlineto{\pgfqpoint{3.383792in}{1.735725in}}%
\pgfpathlineto{\pgfqpoint{3.388333in}{1.735725in}}%
\pgfpathlineto{\pgfqpoint{3.388333in}{1.732775in}}%
\pgfpathmoveto{\pgfqpoint{3.383792in}{1.735725in}}%
\pgfpathlineto{\pgfqpoint{3.383792in}{1.735725in}}%
\pgfpathlineto{\pgfqpoint{3.383792in}{1.738674in}}%
\pgfpathlineto{\pgfqpoint{3.388333in}{1.738674in}}%
\pgfpathlineto{\pgfqpoint{3.388333in}{1.735725in}}%
\pgfpathmoveto{\pgfqpoint{3.388333in}{1.732775in}}%
\pgfpathlineto{\pgfqpoint{3.388333in}{1.732775in}}%
\pgfpathlineto{\pgfqpoint{3.388333in}{1.735725in}}%
\pgfpathlineto{\pgfqpoint{3.392873in}{1.735725in}}%
\pgfpathlineto{\pgfqpoint{3.392873in}{1.732775in}}%
\pgfpathmoveto{\pgfqpoint{3.388333in}{1.735725in}}%
\pgfpathlineto{\pgfqpoint{3.388333in}{1.735725in}}%
\pgfpathlineto{\pgfqpoint{3.388333in}{1.738674in}}%
\pgfpathlineto{\pgfqpoint{3.392873in}{1.738674in}}%
\pgfpathlineto{\pgfqpoint{3.392873in}{1.735725in}}%
\pgfpathmoveto{\pgfqpoint{3.392873in}{1.726877in}}%
\pgfpathlineto{\pgfqpoint{3.392873in}{1.726877in}}%
\pgfpathlineto{\pgfqpoint{3.392873in}{1.729826in}}%
\pgfpathlineto{\pgfqpoint{3.397414in}{1.729826in}}%
\pgfpathlineto{\pgfqpoint{3.397414in}{1.726877in}}%
\pgfpathmoveto{\pgfqpoint{3.392873in}{1.729826in}}%
\pgfpathlineto{\pgfqpoint{3.392873in}{1.729826in}}%
\pgfpathlineto{\pgfqpoint{3.392873in}{1.732775in}}%
\pgfpathlineto{\pgfqpoint{3.397414in}{1.732775in}}%
\pgfpathlineto{\pgfqpoint{3.397414in}{1.729826in}}%
\pgfpathmoveto{\pgfqpoint{3.397414in}{1.726877in}}%
\pgfpathlineto{\pgfqpoint{3.397414in}{1.726877in}}%
\pgfpathlineto{\pgfqpoint{3.397414in}{1.729826in}}%
\pgfpathlineto{\pgfqpoint{3.401955in}{1.729826in}}%
\pgfpathlineto{\pgfqpoint{3.401955in}{1.726877in}}%
\pgfpathmoveto{\pgfqpoint{3.397414in}{1.729826in}}%
\pgfpathlineto{\pgfqpoint{3.397414in}{1.729826in}}%
\pgfpathlineto{\pgfqpoint{3.397414in}{1.732775in}}%
\pgfpathlineto{\pgfqpoint{3.401955in}{1.732775in}}%
\pgfpathlineto{\pgfqpoint{3.401955in}{1.729826in}}%
\pgfpathmoveto{\pgfqpoint{3.392873in}{1.732775in}}%
\pgfpathlineto{\pgfqpoint{3.392873in}{1.732775in}}%
\pgfpathlineto{\pgfqpoint{3.392873in}{1.735725in}}%
\pgfpathlineto{\pgfqpoint{3.397414in}{1.735725in}}%
\pgfpathlineto{\pgfqpoint{3.397414in}{1.732775in}}%
\pgfpathmoveto{\pgfqpoint{3.383792in}{1.738674in}}%
\pgfpathlineto{\pgfqpoint{3.383792in}{1.738674in}}%
\pgfpathlineto{\pgfqpoint{3.383792in}{1.741623in}}%
\pgfpathlineto{\pgfqpoint{3.388333in}{1.741623in}}%
\pgfpathlineto{\pgfqpoint{3.388333in}{1.738674in}}%
\pgfpathmoveto{\pgfqpoint{3.383792in}{1.741623in}}%
\pgfpathlineto{\pgfqpoint{3.383792in}{1.741623in}}%
\pgfpathlineto{\pgfqpoint{3.383792in}{1.744572in}}%
\pgfpathlineto{\pgfqpoint{3.388333in}{1.744572in}}%
\pgfpathlineto{\pgfqpoint{3.388333in}{1.741623in}}%
\pgfpathmoveto{\pgfqpoint{3.365628in}{1.750470in}}%
\pgfpathlineto{\pgfqpoint{3.365628in}{1.750470in}}%
\pgfpathlineto{\pgfqpoint{3.365628in}{1.753419in}}%
\pgfpathlineto{\pgfqpoint{3.370169in}{1.753419in}}%
\pgfpathlineto{\pgfqpoint{3.370169in}{1.750470in}}%
\pgfpathmoveto{\pgfqpoint{3.365628in}{1.753419in}}%
\pgfpathlineto{\pgfqpoint{3.365628in}{1.753419in}}%
\pgfpathlineto{\pgfqpoint{3.365628in}{1.756368in}}%
\pgfpathlineto{\pgfqpoint{3.370169in}{1.756368in}}%
\pgfpathlineto{\pgfqpoint{3.370169in}{1.753419in}}%
\pgfpathmoveto{\pgfqpoint{3.370169in}{1.750470in}}%
\pgfpathlineto{\pgfqpoint{3.370169in}{1.750470in}}%
\pgfpathlineto{\pgfqpoint{3.370169in}{1.753419in}}%
\pgfpathlineto{\pgfqpoint{3.374710in}{1.753419in}}%
\pgfpathlineto{\pgfqpoint{3.374710in}{1.750470in}}%
\pgfpathmoveto{\pgfqpoint{3.370169in}{1.753419in}}%
\pgfpathlineto{\pgfqpoint{3.370169in}{1.753419in}}%
\pgfpathlineto{\pgfqpoint{3.370169in}{1.756368in}}%
\pgfpathlineto{\pgfqpoint{3.374710in}{1.756368in}}%
\pgfpathlineto{\pgfqpoint{3.374710in}{1.753419in}}%
\pgfpathmoveto{\pgfqpoint{3.365628in}{1.756368in}}%
\pgfpathlineto{\pgfqpoint{3.365628in}{1.756368in}}%
\pgfpathlineto{\pgfqpoint{3.365628in}{1.759317in}}%
\pgfpathlineto{\pgfqpoint{3.370169in}{1.759317in}}%
\pgfpathlineto{\pgfqpoint{3.370169in}{1.756368in}}%
\pgfpathmoveto{\pgfqpoint{3.510939in}{0.535389in}}%
\pgfpathlineto{\pgfqpoint{3.510939in}{0.535389in}}%
\pgfpathlineto{\pgfqpoint{3.510939in}{0.538338in}}%
\pgfpathlineto{\pgfqpoint{3.515480in}{0.538338in}}%
\pgfpathlineto{\pgfqpoint{3.515480in}{0.535389in}}%
\pgfpathmoveto{\pgfqpoint{3.510939in}{0.538338in}}%
\pgfpathlineto{\pgfqpoint{3.510939in}{0.538338in}}%
\pgfpathlineto{\pgfqpoint{3.510939in}{0.541287in}}%
\pgfpathlineto{\pgfqpoint{3.515480in}{0.541287in}}%
\pgfpathlineto{\pgfqpoint{3.515480in}{0.538338in}}%
\pgfpathmoveto{\pgfqpoint{3.510939in}{0.541287in}}%
\pgfpathlineto{\pgfqpoint{3.510939in}{0.541287in}}%
\pgfpathlineto{\pgfqpoint{3.510939in}{0.544237in}}%
\pgfpathlineto{\pgfqpoint{3.515480in}{0.544237in}}%
\pgfpathlineto{\pgfqpoint{3.515480in}{0.541287in}}%
\pgfpathmoveto{\pgfqpoint{3.510939in}{0.544237in}}%
\pgfpathlineto{\pgfqpoint{3.510939in}{0.544237in}}%
\pgfpathlineto{\pgfqpoint{3.510939in}{0.547186in}}%
\pgfpathlineto{\pgfqpoint{3.515480in}{0.547186in}}%
\pgfpathlineto{\pgfqpoint{3.515480in}{0.544237in}}%
\pgfpathmoveto{\pgfqpoint{3.515480in}{0.541287in}}%
\pgfpathlineto{\pgfqpoint{3.515480in}{0.541287in}}%
\pgfpathlineto{\pgfqpoint{3.515480in}{0.544237in}}%
\pgfpathlineto{\pgfqpoint{3.520021in}{0.544237in}}%
\pgfpathlineto{\pgfqpoint{3.520021in}{0.541287in}}%
\pgfpathmoveto{\pgfqpoint{3.515480in}{0.544237in}}%
\pgfpathlineto{\pgfqpoint{3.515480in}{0.544237in}}%
\pgfpathlineto{\pgfqpoint{3.515480in}{0.547186in}}%
\pgfpathlineto{\pgfqpoint{3.520021in}{0.547186in}}%
\pgfpathlineto{\pgfqpoint{3.520021in}{0.544237in}}%
\pgfpathmoveto{\pgfqpoint{3.520021in}{0.544237in}}%
\pgfpathlineto{\pgfqpoint{3.520021in}{0.544237in}}%
\pgfpathlineto{\pgfqpoint{3.520021in}{0.547186in}}%
\pgfpathlineto{\pgfqpoint{3.524562in}{0.547186in}}%
\pgfpathlineto{\pgfqpoint{3.524562in}{0.544237in}}%
\pgfpathmoveto{\pgfqpoint{3.520021in}{0.547186in}}%
\pgfpathlineto{\pgfqpoint{3.520021in}{0.547186in}}%
\pgfpathlineto{\pgfqpoint{3.520021in}{0.550135in}}%
\pgfpathlineto{\pgfqpoint{3.524562in}{0.550135in}}%
\pgfpathlineto{\pgfqpoint{3.524562in}{0.547186in}}%
\pgfpathmoveto{\pgfqpoint{3.520021in}{0.550135in}}%
\pgfpathlineto{\pgfqpoint{3.520021in}{0.550135in}}%
\pgfpathlineto{\pgfqpoint{3.520021in}{0.553084in}}%
\pgfpathlineto{\pgfqpoint{3.524562in}{0.553084in}}%
\pgfpathlineto{\pgfqpoint{3.524562in}{0.550135in}}%
\pgfpathmoveto{\pgfqpoint{3.524562in}{0.547186in}}%
\pgfpathlineto{\pgfqpoint{3.524562in}{0.547186in}}%
\pgfpathlineto{\pgfqpoint{3.524562in}{0.550135in}}%
\pgfpathlineto{\pgfqpoint{3.529103in}{0.550135in}}%
\pgfpathlineto{\pgfqpoint{3.529103in}{0.547186in}}%
\pgfpathmoveto{\pgfqpoint{3.524562in}{0.550135in}}%
\pgfpathlineto{\pgfqpoint{3.524562in}{0.550135in}}%
\pgfpathlineto{\pgfqpoint{3.524562in}{0.553084in}}%
\pgfpathlineto{\pgfqpoint{3.529103in}{0.553084in}}%
\pgfpathlineto{\pgfqpoint{3.529103in}{0.550135in}}%
\pgfpathmoveto{\pgfqpoint{3.529103in}{0.550135in}}%
\pgfpathlineto{\pgfqpoint{3.529103in}{0.550135in}}%
\pgfpathlineto{\pgfqpoint{3.529103in}{0.553084in}}%
\pgfpathlineto{\pgfqpoint{3.533644in}{0.553084in}}%
\pgfpathlineto{\pgfqpoint{3.533644in}{0.550135in}}%
\pgfpathmoveto{\pgfqpoint{3.529103in}{0.553084in}}%
\pgfpathlineto{\pgfqpoint{3.529103in}{0.553084in}}%
\pgfpathlineto{\pgfqpoint{3.529103in}{0.556033in}}%
\pgfpathlineto{\pgfqpoint{3.533644in}{0.556033in}}%
\pgfpathlineto{\pgfqpoint{3.533644in}{0.553084in}}%
\pgfpathmoveto{\pgfqpoint{3.529103in}{0.556033in}}%
\pgfpathlineto{\pgfqpoint{3.529103in}{0.556033in}}%
\pgfpathlineto{\pgfqpoint{3.529103in}{0.558983in}}%
\pgfpathlineto{\pgfqpoint{3.533644in}{0.558983in}}%
\pgfpathlineto{\pgfqpoint{3.533644in}{0.556033in}}%
\pgfpathmoveto{\pgfqpoint{3.533644in}{0.553084in}}%
\pgfpathlineto{\pgfqpoint{3.533644in}{0.553084in}}%
\pgfpathlineto{\pgfqpoint{3.533644in}{0.556033in}}%
\pgfpathlineto{\pgfqpoint{3.538185in}{0.556033in}}%
\pgfpathlineto{\pgfqpoint{3.538185in}{0.553084in}}%
\pgfpathmoveto{\pgfqpoint{3.533644in}{0.556033in}}%
\pgfpathlineto{\pgfqpoint{3.533644in}{0.556033in}}%
\pgfpathlineto{\pgfqpoint{3.533644in}{0.558983in}}%
\pgfpathlineto{\pgfqpoint{3.538185in}{0.558983in}}%
\pgfpathlineto{\pgfqpoint{3.538185in}{0.556033in}}%
\pgfpathmoveto{\pgfqpoint{3.529103in}{0.558983in}}%
\pgfpathlineto{\pgfqpoint{3.529103in}{0.558983in}}%
\pgfpathlineto{\pgfqpoint{3.529103in}{0.561932in}}%
\pgfpathlineto{\pgfqpoint{3.533644in}{0.561932in}}%
\pgfpathlineto{\pgfqpoint{3.533644in}{0.558983in}}%
\pgfpathmoveto{\pgfqpoint{3.529103in}{0.561932in}}%
\pgfpathlineto{\pgfqpoint{3.529103in}{0.561932in}}%
\pgfpathlineto{\pgfqpoint{3.529103in}{0.564881in}}%
\pgfpathlineto{\pgfqpoint{3.533644in}{0.564881in}}%
\pgfpathlineto{\pgfqpoint{3.533644in}{0.561932in}}%
\pgfpathmoveto{\pgfqpoint{3.533644in}{0.558983in}}%
\pgfpathlineto{\pgfqpoint{3.533644in}{0.558983in}}%
\pgfpathlineto{\pgfqpoint{3.533644in}{0.561932in}}%
\pgfpathlineto{\pgfqpoint{3.538185in}{0.561932in}}%
\pgfpathlineto{\pgfqpoint{3.538185in}{0.558983in}}%
\pgfpathmoveto{\pgfqpoint{3.533644in}{0.561932in}}%
\pgfpathlineto{\pgfqpoint{3.533644in}{0.561932in}}%
\pgfpathlineto{\pgfqpoint{3.533644in}{0.564881in}}%
\pgfpathlineto{\pgfqpoint{3.538185in}{0.564881in}}%
\pgfpathlineto{\pgfqpoint{3.538185in}{0.561932in}}%
\pgfpathmoveto{\pgfqpoint{3.538185in}{0.558983in}}%
\pgfpathlineto{\pgfqpoint{3.538185in}{0.558983in}}%
\pgfpathlineto{\pgfqpoint{3.538185in}{0.561932in}}%
\pgfpathlineto{\pgfqpoint{3.542725in}{0.561932in}}%
\pgfpathlineto{\pgfqpoint{3.542725in}{0.558983in}}%
\pgfpathmoveto{\pgfqpoint{3.538185in}{0.561932in}}%
\pgfpathlineto{\pgfqpoint{3.538185in}{0.561932in}}%
\pgfpathlineto{\pgfqpoint{3.538185in}{0.564881in}}%
\pgfpathlineto{\pgfqpoint{3.542725in}{0.564881in}}%
\pgfpathlineto{\pgfqpoint{3.542725in}{0.561932in}}%
\pgfpathmoveto{\pgfqpoint{3.542725in}{0.561932in}}%
\pgfpathlineto{\pgfqpoint{3.542725in}{0.561932in}}%
\pgfpathlineto{\pgfqpoint{3.542725in}{0.564881in}}%
\pgfpathlineto{\pgfqpoint{3.547266in}{0.564881in}}%
\pgfpathlineto{\pgfqpoint{3.547266in}{0.561932in}}%
\pgfpathmoveto{\pgfqpoint{3.538185in}{0.564881in}}%
\pgfpathlineto{\pgfqpoint{3.538185in}{0.564881in}}%
\pgfpathlineto{\pgfqpoint{3.538185in}{0.567830in}}%
\pgfpathlineto{\pgfqpoint{3.542725in}{0.567830in}}%
\pgfpathlineto{\pgfqpoint{3.542725in}{0.564881in}}%
\pgfpathmoveto{\pgfqpoint{3.538185in}{0.567830in}}%
\pgfpathlineto{\pgfqpoint{3.538185in}{0.567830in}}%
\pgfpathlineto{\pgfqpoint{3.538185in}{0.570780in}}%
\pgfpathlineto{\pgfqpoint{3.542725in}{0.570780in}}%
\pgfpathlineto{\pgfqpoint{3.542725in}{0.567830in}}%
\pgfpathmoveto{\pgfqpoint{3.542725in}{0.564881in}}%
\pgfpathlineto{\pgfqpoint{3.542725in}{0.564881in}}%
\pgfpathlineto{\pgfqpoint{3.542725in}{0.567830in}}%
\pgfpathlineto{\pgfqpoint{3.547266in}{0.567830in}}%
\pgfpathlineto{\pgfqpoint{3.547266in}{0.564881in}}%
\pgfpathmoveto{\pgfqpoint{3.542725in}{0.567830in}}%
\pgfpathlineto{\pgfqpoint{3.542725in}{0.567830in}}%
\pgfpathlineto{\pgfqpoint{3.542725in}{0.570780in}}%
\pgfpathlineto{\pgfqpoint{3.547266in}{0.570780in}}%
\pgfpathlineto{\pgfqpoint{3.547266in}{0.567830in}}%
\pgfpathmoveto{\pgfqpoint{3.547266in}{0.564881in}}%
\pgfpathlineto{\pgfqpoint{3.547266in}{0.564881in}}%
\pgfpathlineto{\pgfqpoint{3.547266in}{0.567830in}}%
\pgfpathlineto{\pgfqpoint{3.551807in}{0.567830in}}%
\pgfpathlineto{\pgfqpoint{3.551807in}{0.564881in}}%
\pgfpathmoveto{\pgfqpoint{3.547266in}{0.567830in}}%
\pgfpathlineto{\pgfqpoint{3.547266in}{0.567830in}}%
\pgfpathlineto{\pgfqpoint{3.547266in}{0.570780in}}%
\pgfpathlineto{\pgfqpoint{3.551807in}{0.570780in}}%
\pgfpathlineto{\pgfqpoint{3.551807in}{0.567830in}}%
\pgfpathmoveto{\pgfqpoint{3.551807in}{0.567830in}}%
\pgfpathlineto{\pgfqpoint{3.551807in}{0.567830in}}%
\pgfpathlineto{\pgfqpoint{3.551807in}{0.570780in}}%
\pgfpathlineto{\pgfqpoint{3.556348in}{0.570780in}}%
\pgfpathlineto{\pgfqpoint{3.556348in}{0.567830in}}%
\pgfpathmoveto{\pgfqpoint{3.547266in}{0.570780in}}%
\pgfpathlineto{\pgfqpoint{3.547266in}{0.570780in}}%
\pgfpathlineto{\pgfqpoint{3.547266in}{0.573729in}}%
\pgfpathlineto{\pgfqpoint{3.551807in}{0.573729in}}%
\pgfpathlineto{\pgfqpoint{3.551807in}{0.570780in}}%
\pgfpathmoveto{\pgfqpoint{3.547266in}{0.573729in}}%
\pgfpathlineto{\pgfqpoint{3.547266in}{0.573729in}}%
\pgfpathlineto{\pgfqpoint{3.547266in}{0.576678in}}%
\pgfpathlineto{\pgfqpoint{3.551807in}{0.576678in}}%
\pgfpathlineto{\pgfqpoint{3.551807in}{0.573729in}}%
\pgfpathmoveto{\pgfqpoint{3.551807in}{0.570780in}}%
\pgfpathlineto{\pgfqpoint{3.551807in}{0.570780in}}%
\pgfpathlineto{\pgfqpoint{3.551807in}{0.573729in}}%
\pgfpathlineto{\pgfqpoint{3.556348in}{0.573729in}}%
\pgfpathlineto{\pgfqpoint{3.556348in}{0.570780in}}%
\pgfpathmoveto{\pgfqpoint{3.551807in}{0.573729in}}%
\pgfpathlineto{\pgfqpoint{3.551807in}{0.573729in}}%
\pgfpathlineto{\pgfqpoint{3.551807in}{0.576678in}}%
\pgfpathlineto{\pgfqpoint{3.556348in}{0.576678in}}%
\pgfpathlineto{\pgfqpoint{3.556348in}{0.573729in}}%
\pgfpathmoveto{\pgfqpoint{3.556348in}{0.570780in}}%
\pgfpathlineto{\pgfqpoint{3.556348in}{0.570780in}}%
\pgfpathlineto{\pgfqpoint{3.556348in}{0.573729in}}%
\pgfpathlineto{\pgfqpoint{3.560889in}{0.573729in}}%
\pgfpathlineto{\pgfqpoint{3.560889in}{0.570780in}}%
\pgfpathmoveto{\pgfqpoint{3.556348in}{0.573729in}}%
\pgfpathlineto{\pgfqpoint{3.556348in}{0.573729in}}%
\pgfpathlineto{\pgfqpoint{3.556348in}{0.576678in}}%
\pgfpathlineto{\pgfqpoint{3.560889in}{0.576678in}}%
\pgfpathlineto{\pgfqpoint{3.560889in}{0.573729in}}%
\pgfpathmoveto{\pgfqpoint{3.556348in}{0.576678in}}%
\pgfpathlineto{\pgfqpoint{3.556348in}{0.576678in}}%
\pgfpathlineto{\pgfqpoint{3.556348in}{0.579627in}}%
\pgfpathlineto{\pgfqpoint{3.560889in}{0.579627in}}%
\pgfpathlineto{\pgfqpoint{3.560889in}{0.576678in}}%
\pgfpathmoveto{\pgfqpoint{3.556348in}{0.579627in}}%
\pgfpathlineto{\pgfqpoint{3.556348in}{0.579627in}}%
\pgfpathlineto{\pgfqpoint{3.556348in}{0.582576in}}%
\pgfpathlineto{\pgfqpoint{3.560889in}{0.582576in}}%
\pgfpathlineto{\pgfqpoint{3.560889in}{0.579627in}}%
\pgfpathmoveto{\pgfqpoint{3.560889in}{0.576678in}}%
\pgfpathlineto{\pgfqpoint{3.560889in}{0.576678in}}%
\pgfpathlineto{\pgfqpoint{3.560889in}{0.579627in}}%
\pgfpathlineto{\pgfqpoint{3.565430in}{0.579627in}}%
\pgfpathlineto{\pgfqpoint{3.565430in}{0.576678in}}%
\pgfpathmoveto{\pgfqpoint{3.560889in}{0.579627in}}%
\pgfpathlineto{\pgfqpoint{3.560889in}{0.579627in}}%
\pgfpathlineto{\pgfqpoint{3.560889in}{0.582576in}}%
\pgfpathlineto{\pgfqpoint{3.565430in}{0.582576in}}%
\pgfpathlineto{\pgfqpoint{3.565430in}{0.579627in}}%
\pgfpathmoveto{\pgfqpoint{3.565430in}{0.579627in}}%
\pgfpathlineto{\pgfqpoint{3.565430in}{0.579627in}}%
\pgfpathlineto{\pgfqpoint{3.565430in}{0.582576in}}%
\pgfpathlineto{\pgfqpoint{3.569971in}{0.582576in}}%
\pgfpathlineto{\pgfqpoint{3.569971in}{0.579627in}}%
\pgfpathmoveto{\pgfqpoint{3.565430in}{0.582576in}}%
\pgfpathlineto{\pgfqpoint{3.565430in}{0.582576in}}%
\pgfpathlineto{\pgfqpoint{3.565430in}{0.585526in}}%
\pgfpathlineto{\pgfqpoint{3.569971in}{0.585526in}}%
\pgfpathlineto{\pgfqpoint{3.569971in}{0.582576in}}%
\pgfpathmoveto{\pgfqpoint{3.565430in}{0.585526in}}%
\pgfpathlineto{\pgfqpoint{3.565430in}{0.585526in}}%
\pgfpathlineto{\pgfqpoint{3.565430in}{0.588475in}}%
\pgfpathlineto{\pgfqpoint{3.569971in}{0.588475in}}%
\pgfpathlineto{\pgfqpoint{3.569971in}{0.585526in}}%
\pgfpathmoveto{\pgfqpoint{3.569971in}{0.582576in}}%
\pgfpathlineto{\pgfqpoint{3.569971in}{0.582576in}}%
\pgfpathlineto{\pgfqpoint{3.569971in}{0.585526in}}%
\pgfpathlineto{\pgfqpoint{3.574512in}{0.585526in}}%
\pgfpathlineto{\pgfqpoint{3.574512in}{0.582576in}}%
\pgfpathmoveto{\pgfqpoint{3.569971in}{0.585526in}}%
\pgfpathlineto{\pgfqpoint{3.569971in}{0.585526in}}%
\pgfpathlineto{\pgfqpoint{3.569971in}{0.588475in}}%
\pgfpathlineto{\pgfqpoint{3.574512in}{0.588475in}}%
\pgfpathlineto{\pgfqpoint{3.574512in}{0.585526in}}%
\pgfpathmoveto{\pgfqpoint{3.574512in}{0.585526in}}%
\pgfpathlineto{\pgfqpoint{3.574512in}{0.585526in}}%
\pgfpathlineto{\pgfqpoint{3.574512in}{0.588475in}}%
\pgfpathlineto{\pgfqpoint{3.579053in}{0.588475in}}%
\pgfpathlineto{\pgfqpoint{3.579053in}{0.585526in}}%
\pgfpathmoveto{\pgfqpoint{3.574512in}{0.588475in}}%
\pgfpathlineto{\pgfqpoint{3.574512in}{0.588475in}}%
\pgfpathlineto{\pgfqpoint{3.574512in}{0.591424in}}%
\pgfpathlineto{\pgfqpoint{3.579053in}{0.591424in}}%
\pgfpathlineto{\pgfqpoint{3.579053in}{0.588475in}}%
\pgfpathmoveto{\pgfqpoint{3.574512in}{0.591424in}}%
\pgfpathlineto{\pgfqpoint{3.574512in}{0.591424in}}%
\pgfpathlineto{\pgfqpoint{3.574512in}{0.594373in}}%
\pgfpathlineto{\pgfqpoint{3.579053in}{0.594373in}}%
\pgfpathlineto{\pgfqpoint{3.579053in}{0.591424in}}%
\pgfpathmoveto{\pgfqpoint{3.579053in}{0.588475in}}%
\pgfpathlineto{\pgfqpoint{3.579053in}{0.588475in}}%
\pgfpathlineto{\pgfqpoint{3.579053in}{0.591424in}}%
\pgfpathlineto{\pgfqpoint{3.583594in}{0.591424in}}%
\pgfpathlineto{\pgfqpoint{3.583594in}{0.588475in}}%
\pgfpathmoveto{\pgfqpoint{3.579053in}{0.591424in}}%
\pgfpathlineto{\pgfqpoint{3.579053in}{0.591424in}}%
\pgfpathlineto{\pgfqpoint{3.579053in}{0.594373in}}%
\pgfpathlineto{\pgfqpoint{3.583594in}{0.594373in}}%
\pgfpathlineto{\pgfqpoint{3.583594in}{0.591424in}}%
\pgfpathmoveto{\pgfqpoint{3.574512in}{0.594373in}}%
\pgfpathlineto{\pgfqpoint{3.574512in}{0.594373in}}%
\pgfpathlineto{\pgfqpoint{3.574512in}{0.597323in}}%
\pgfpathlineto{\pgfqpoint{3.579053in}{0.597323in}}%
\pgfpathlineto{\pgfqpoint{3.579053in}{0.594373in}}%
\pgfpathmoveto{\pgfqpoint{3.574512in}{0.597323in}}%
\pgfpathlineto{\pgfqpoint{3.574512in}{0.597323in}}%
\pgfpathlineto{\pgfqpoint{3.574512in}{0.600272in}}%
\pgfpathlineto{\pgfqpoint{3.579053in}{0.600272in}}%
\pgfpathlineto{\pgfqpoint{3.579053in}{0.597323in}}%
\pgfpathmoveto{\pgfqpoint{3.579053in}{0.594373in}}%
\pgfpathlineto{\pgfqpoint{3.579053in}{0.594373in}}%
\pgfpathlineto{\pgfqpoint{3.579053in}{0.597323in}}%
\pgfpathlineto{\pgfqpoint{3.583594in}{0.597323in}}%
\pgfpathlineto{\pgfqpoint{3.583594in}{0.594373in}}%
\pgfpathmoveto{\pgfqpoint{3.579053in}{0.597323in}}%
\pgfpathlineto{\pgfqpoint{3.579053in}{0.597323in}}%
\pgfpathlineto{\pgfqpoint{3.579053in}{0.600272in}}%
\pgfpathlineto{\pgfqpoint{3.583594in}{0.600272in}}%
\pgfpathlineto{\pgfqpoint{3.583594in}{0.597323in}}%
\pgfpathmoveto{\pgfqpoint{3.583594in}{0.594373in}}%
\pgfpathlineto{\pgfqpoint{3.583594in}{0.594373in}}%
\pgfpathlineto{\pgfqpoint{3.583594in}{0.597323in}}%
\pgfpathlineto{\pgfqpoint{3.588135in}{0.597323in}}%
\pgfpathlineto{\pgfqpoint{3.588135in}{0.594373in}}%
\pgfpathmoveto{\pgfqpoint{3.583594in}{0.597323in}}%
\pgfpathlineto{\pgfqpoint{3.583594in}{0.597323in}}%
\pgfpathlineto{\pgfqpoint{3.583594in}{0.600272in}}%
\pgfpathlineto{\pgfqpoint{3.588135in}{0.600272in}}%
\pgfpathlineto{\pgfqpoint{3.588135in}{0.597323in}}%
\pgfpathmoveto{\pgfqpoint{3.588135in}{0.597323in}}%
\pgfpathlineto{\pgfqpoint{3.588135in}{0.597323in}}%
\pgfpathlineto{\pgfqpoint{3.588135in}{0.600272in}}%
\pgfpathlineto{\pgfqpoint{3.592676in}{0.600272in}}%
\pgfpathlineto{\pgfqpoint{3.592676in}{0.597323in}}%
\pgfpathmoveto{\pgfqpoint{3.583594in}{0.600272in}}%
\pgfpathlineto{\pgfqpoint{3.583594in}{0.600272in}}%
\pgfpathlineto{\pgfqpoint{3.583594in}{0.603221in}}%
\pgfpathlineto{\pgfqpoint{3.588135in}{0.603221in}}%
\pgfpathlineto{\pgfqpoint{3.588135in}{0.600272in}}%
\pgfpathmoveto{\pgfqpoint{3.583594in}{0.603221in}}%
\pgfpathlineto{\pgfqpoint{3.583594in}{0.603221in}}%
\pgfpathlineto{\pgfqpoint{3.583594in}{0.606171in}}%
\pgfpathlineto{\pgfqpoint{3.588135in}{0.606171in}}%
\pgfpathlineto{\pgfqpoint{3.588135in}{0.603221in}}%
\pgfpathmoveto{\pgfqpoint{3.588135in}{0.600272in}}%
\pgfpathlineto{\pgfqpoint{3.588135in}{0.600272in}}%
\pgfpathlineto{\pgfqpoint{3.588135in}{0.603221in}}%
\pgfpathlineto{\pgfqpoint{3.592676in}{0.603221in}}%
\pgfpathlineto{\pgfqpoint{3.592676in}{0.600272in}}%
\pgfpathmoveto{\pgfqpoint{3.588135in}{0.603221in}}%
\pgfpathlineto{\pgfqpoint{3.588135in}{0.603221in}}%
\pgfpathlineto{\pgfqpoint{3.588135in}{0.606171in}}%
\pgfpathlineto{\pgfqpoint{3.592676in}{0.606171in}}%
\pgfpathlineto{\pgfqpoint{3.592676in}{0.603221in}}%
\pgfpathmoveto{\pgfqpoint{3.592676in}{0.600272in}}%
\pgfpathlineto{\pgfqpoint{3.592676in}{0.600272in}}%
\pgfpathlineto{\pgfqpoint{3.592676in}{0.603221in}}%
\pgfpathlineto{\pgfqpoint{3.597217in}{0.603221in}}%
\pgfpathlineto{\pgfqpoint{3.597217in}{0.600272in}}%
\pgfpathmoveto{\pgfqpoint{3.592676in}{0.603221in}}%
\pgfpathlineto{\pgfqpoint{3.592676in}{0.603221in}}%
\pgfpathlineto{\pgfqpoint{3.592676in}{0.606171in}}%
\pgfpathlineto{\pgfqpoint{3.597217in}{0.606171in}}%
\pgfpathlineto{\pgfqpoint{3.597217in}{0.603221in}}%
\pgfpathmoveto{\pgfqpoint{3.597217in}{0.603221in}}%
\pgfpathlineto{\pgfqpoint{3.597217in}{0.603221in}}%
\pgfpathlineto{\pgfqpoint{3.597217in}{0.606171in}}%
\pgfpathlineto{\pgfqpoint{3.601758in}{0.606171in}}%
\pgfpathlineto{\pgfqpoint{3.601758in}{0.603221in}}%
\pgfpathmoveto{\pgfqpoint{3.592676in}{0.606171in}}%
\pgfpathlineto{\pgfqpoint{3.592676in}{0.606171in}}%
\pgfpathlineto{\pgfqpoint{3.592676in}{0.609120in}}%
\pgfpathlineto{\pgfqpoint{3.597217in}{0.609120in}}%
\pgfpathlineto{\pgfqpoint{3.597217in}{0.606171in}}%
\pgfpathmoveto{\pgfqpoint{3.592676in}{0.609120in}}%
\pgfpathlineto{\pgfqpoint{3.592676in}{0.609120in}}%
\pgfpathlineto{\pgfqpoint{3.592676in}{0.612069in}}%
\pgfpathlineto{\pgfqpoint{3.597217in}{0.612069in}}%
\pgfpathlineto{\pgfqpoint{3.597217in}{0.609120in}}%
\pgfpathmoveto{\pgfqpoint{3.597217in}{0.606171in}}%
\pgfpathlineto{\pgfqpoint{3.597217in}{0.606171in}}%
\pgfpathlineto{\pgfqpoint{3.597217in}{0.609120in}}%
\pgfpathlineto{\pgfqpoint{3.601758in}{0.609120in}}%
\pgfpathlineto{\pgfqpoint{3.601758in}{0.606171in}}%
\pgfpathmoveto{\pgfqpoint{3.597217in}{0.609120in}}%
\pgfpathlineto{\pgfqpoint{3.597217in}{0.609120in}}%
\pgfpathlineto{\pgfqpoint{3.597217in}{0.612069in}}%
\pgfpathlineto{\pgfqpoint{3.601758in}{0.612069in}}%
\pgfpathlineto{\pgfqpoint{3.601758in}{0.609120in}}%
\pgfpathmoveto{\pgfqpoint{3.601758in}{0.606171in}}%
\pgfpathlineto{\pgfqpoint{3.601758in}{0.606171in}}%
\pgfpathlineto{\pgfqpoint{3.601758in}{0.609120in}}%
\pgfpathlineto{\pgfqpoint{3.606299in}{0.609120in}}%
\pgfpathlineto{\pgfqpoint{3.606299in}{0.606171in}}%
\pgfpathmoveto{\pgfqpoint{3.601758in}{0.609120in}}%
\pgfpathlineto{\pgfqpoint{3.601758in}{0.609120in}}%
\pgfpathlineto{\pgfqpoint{3.601758in}{0.612069in}}%
\pgfpathlineto{\pgfqpoint{3.606299in}{0.612069in}}%
\pgfpathlineto{\pgfqpoint{3.606299in}{0.609120in}}%
\pgfpathmoveto{\pgfqpoint{3.601758in}{0.612069in}}%
\pgfpathlineto{\pgfqpoint{3.601758in}{0.612069in}}%
\pgfpathlineto{\pgfqpoint{3.601758in}{0.615019in}}%
\pgfpathlineto{\pgfqpoint{3.606299in}{0.615019in}}%
\pgfpathlineto{\pgfqpoint{3.606299in}{0.612069in}}%
\pgfpathmoveto{\pgfqpoint{3.601758in}{0.615019in}}%
\pgfpathlineto{\pgfqpoint{3.601758in}{0.615019in}}%
\pgfpathlineto{\pgfqpoint{3.601758in}{0.617968in}}%
\pgfpathlineto{\pgfqpoint{3.606299in}{0.617968in}}%
\pgfpathlineto{\pgfqpoint{3.606299in}{0.615019in}}%
\pgfpathmoveto{\pgfqpoint{3.606299in}{0.612069in}}%
\pgfpathlineto{\pgfqpoint{3.606299in}{0.612069in}}%
\pgfpathlineto{\pgfqpoint{3.606299in}{0.615019in}}%
\pgfpathlineto{\pgfqpoint{3.610840in}{0.615019in}}%
\pgfpathlineto{\pgfqpoint{3.610840in}{0.612069in}}%
\pgfpathmoveto{\pgfqpoint{3.606299in}{0.615019in}}%
\pgfpathlineto{\pgfqpoint{3.606299in}{0.615019in}}%
\pgfpathlineto{\pgfqpoint{3.606299in}{0.617968in}}%
\pgfpathlineto{\pgfqpoint{3.610840in}{0.617968in}}%
\pgfpathlineto{\pgfqpoint{3.610840in}{0.615019in}}%
\pgfpathmoveto{\pgfqpoint{3.610840in}{0.615019in}}%
\pgfpathlineto{\pgfqpoint{3.610840in}{0.615019in}}%
\pgfpathlineto{\pgfqpoint{3.610840in}{0.617968in}}%
\pgfpathlineto{\pgfqpoint{3.615381in}{0.617968in}}%
\pgfpathlineto{\pgfqpoint{3.615381in}{0.615019in}}%
\pgfpathmoveto{\pgfqpoint{3.610840in}{0.617968in}}%
\pgfpathlineto{\pgfqpoint{3.610840in}{0.617968in}}%
\pgfpathlineto{\pgfqpoint{3.610840in}{0.620917in}}%
\pgfpathlineto{\pgfqpoint{3.615381in}{0.620917in}}%
\pgfpathlineto{\pgfqpoint{3.615381in}{0.617968in}}%
\pgfpathmoveto{\pgfqpoint{3.610840in}{0.620917in}}%
\pgfpathlineto{\pgfqpoint{3.610840in}{0.620917in}}%
\pgfpathlineto{\pgfqpoint{3.610840in}{0.623867in}}%
\pgfpathlineto{\pgfqpoint{3.615381in}{0.623867in}}%
\pgfpathlineto{\pgfqpoint{3.615381in}{0.620917in}}%
\pgfpathmoveto{\pgfqpoint{3.615381in}{0.617968in}}%
\pgfpathlineto{\pgfqpoint{3.615381in}{0.617968in}}%
\pgfpathlineto{\pgfqpoint{3.615381in}{0.620917in}}%
\pgfpathlineto{\pgfqpoint{3.619922in}{0.620917in}}%
\pgfpathlineto{\pgfqpoint{3.619922in}{0.617968in}}%
\pgfpathmoveto{\pgfqpoint{3.615381in}{0.620917in}}%
\pgfpathlineto{\pgfqpoint{3.615381in}{0.620917in}}%
\pgfpathlineto{\pgfqpoint{3.615381in}{0.623867in}}%
\pgfpathlineto{\pgfqpoint{3.619922in}{0.623867in}}%
\pgfpathlineto{\pgfqpoint{3.619922in}{0.620917in}}%
\pgfpathmoveto{\pgfqpoint{3.619922in}{0.620917in}}%
\pgfpathlineto{\pgfqpoint{3.619922in}{0.620917in}}%
\pgfpathlineto{\pgfqpoint{3.619922in}{0.623867in}}%
\pgfpathlineto{\pgfqpoint{3.624463in}{0.623867in}}%
\pgfpathlineto{\pgfqpoint{3.624463in}{0.620917in}}%
\pgfpathmoveto{\pgfqpoint{3.619922in}{0.623867in}}%
\pgfpathlineto{\pgfqpoint{3.619922in}{0.623867in}}%
\pgfpathlineto{\pgfqpoint{3.619922in}{0.626816in}}%
\pgfpathlineto{\pgfqpoint{3.624463in}{0.626816in}}%
\pgfpathlineto{\pgfqpoint{3.624463in}{0.623867in}}%
\pgfpathmoveto{\pgfqpoint{3.619922in}{0.626816in}}%
\pgfpathlineto{\pgfqpoint{3.619922in}{0.626816in}}%
\pgfpathlineto{\pgfqpoint{3.619922in}{0.629766in}}%
\pgfpathlineto{\pgfqpoint{3.624463in}{0.629766in}}%
\pgfpathlineto{\pgfqpoint{3.624463in}{0.626816in}}%
\pgfpathmoveto{\pgfqpoint{3.624463in}{0.623867in}}%
\pgfpathlineto{\pgfqpoint{3.624463in}{0.623867in}}%
\pgfpathlineto{\pgfqpoint{3.624463in}{0.626816in}}%
\pgfpathlineto{\pgfqpoint{3.629004in}{0.626816in}}%
\pgfpathlineto{\pgfqpoint{3.629004in}{0.623867in}}%
\pgfpathmoveto{\pgfqpoint{3.624463in}{0.626816in}}%
\pgfpathlineto{\pgfqpoint{3.624463in}{0.626816in}}%
\pgfpathlineto{\pgfqpoint{3.624463in}{0.629766in}}%
\pgfpathlineto{\pgfqpoint{3.629004in}{0.629766in}}%
\pgfpathlineto{\pgfqpoint{3.629004in}{0.626816in}}%
\pgfpathmoveto{\pgfqpoint{3.619922in}{0.629766in}}%
\pgfpathlineto{\pgfqpoint{3.619922in}{0.629766in}}%
\pgfpathlineto{\pgfqpoint{3.619922in}{0.632715in}}%
\pgfpathlineto{\pgfqpoint{3.624463in}{0.632715in}}%
\pgfpathlineto{\pgfqpoint{3.624463in}{0.629766in}}%
\pgfpathmoveto{\pgfqpoint{3.619922in}{0.632715in}}%
\pgfpathlineto{\pgfqpoint{3.619922in}{0.632715in}}%
\pgfpathlineto{\pgfqpoint{3.619922in}{0.635664in}}%
\pgfpathlineto{\pgfqpoint{3.624463in}{0.635664in}}%
\pgfpathlineto{\pgfqpoint{3.624463in}{0.632715in}}%
\pgfpathmoveto{\pgfqpoint{3.624463in}{0.629766in}}%
\pgfpathlineto{\pgfqpoint{3.624463in}{0.629766in}}%
\pgfpathlineto{\pgfqpoint{3.624463in}{0.632715in}}%
\pgfpathlineto{\pgfqpoint{3.629004in}{0.632715in}}%
\pgfpathlineto{\pgfqpoint{3.629004in}{0.629766in}}%
\pgfpathmoveto{\pgfqpoint{3.624463in}{0.632715in}}%
\pgfpathlineto{\pgfqpoint{3.624463in}{0.632715in}}%
\pgfpathlineto{\pgfqpoint{3.624463in}{0.635664in}}%
\pgfpathlineto{\pgfqpoint{3.629004in}{0.635664in}}%
\pgfpathlineto{\pgfqpoint{3.629004in}{0.632715in}}%
\pgfpathmoveto{\pgfqpoint{3.629004in}{0.629766in}}%
\pgfpathlineto{\pgfqpoint{3.629004in}{0.629766in}}%
\pgfpathlineto{\pgfqpoint{3.629004in}{0.632715in}}%
\pgfpathlineto{\pgfqpoint{3.633545in}{0.632715in}}%
\pgfpathlineto{\pgfqpoint{3.633545in}{0.629766in}}%
\pgfpathmoveto{\pgfqpoint{3.629004in}{0.632715in}}%
\pgfpathlineto{\pgfqpoint{3.629004in}{0.632715in}}%
\pgfpathlineto{\pgfqpoint{3.629004in}{0.635664in}}%
\pgfpathlineto{\pgfqpoint{3.633545in}{0.635664in}}%
\pgfpathlineto{\pgfqpoint{3.633545in}{0.632715in}}%
\pgfpathmoveto{\pgfqpoint{3.633545in}{0.632715in}}%
\pgfpathlineto{\pgfqpoint{3.633545in}{0.632715in}}%
\pgfpathlineto{\pgfqpoint{3.633545in}{0.635664in}}%
\pgfpathlineto{\pgfqpoint{3.638086in}{0.635664in}}%
\pgfpathlineto{\pgfqpoint{3.638086in}{0.632715in}}%
\pgfpathmoveto{\pgfqpoint{3.629004in}{0.635664in}}%
\pgfpathlineto{\pgfqpoint{3.629004in}{0.635664in}}%
\pgfpathlineto{\pgfqpoint{3.629004in}{0.638614in}}%
\pgfpathlineto{\pgfqpoint{3.633545in}{0.638614in}}%
\pgfpathlineto{\pgfqpoint{3.633545in}{0.635664in}}%
\pgfpathmoveto{\pgfqpoint{3.629004in}{0.638614in}}%
\pgfpathlineto{\pgfqpoint{3.629004in}{0.638614in}}%
\pgfpathlineto{\pgfqpoint{3.629004in}{0.641563in}}%
\pgfpathlineto{\pgfqpoint{3.633545in}{0.641563in}}%
\pgfpathlineto{\pgfqpoint{3.633545in}{0.638614in}}%
\pgfpathmoveto{\pgfqpoint{3.633545in}{0.635664in}}%
\pgfpathlineto{\pgfqpoint{3.633545in}{0.635664in}}%
\pgfpathlineto{\pgfqpoint{3.633545in}{0.638614in}}%
\pgfpathlineto{\pgfqpoint{3.638086in}{0.638614in}}%
\pgfpathlineto{\pgfqpoint{3.638086in}{0.635664in}}%
\pgfpathmoveto{\pgfqpoint{3.633545in}{0.638614in}}%
\pgfpathlineto{\pgfqpoint{3.633545in}{0.638614in}}%
\pgfpathlineto{\pgfqpoint{3.633545in}{0.641563in}}%
\pgfpathlineto{\pgfqpoint{3.638086in}{0.641563in}}%
\pgfpathlineto{\pgfqpoint{3.638086in}{0.638614in}}%
\pgfpathmoveto{\pgfqpoint{3.638086in}{0.635664in}}%
\pgfpathlineto{\pgfqpoint{3.638086in}{0.635664in}}%
\pgfpathlineto{\pgfqpoint{3.638086in}{0.638614in}}%
\pgfpathlineto{\pgfqpoint{3.642627in}{0.638614in}}%
\pgfpathlineto{\pgfqpoint{3.642627in}{0.635664in}}%
\pgfpathmoveto{\pgfqpoint{3.638086in}{0.638614in}}%
\pgfpathlineto{\pgfqpoint{3.638086in}{0.638614in}}%
\pgfpathlineto{\pgfqpoint{3.638086in}{0.641563in}}%
\pgfpathlineto{\pgfqpoint{3.642627in}{0.641563in}}%
\pgfpathlineto{\pgfqpoint{3.642627in}{0.638614in}}%
\pgfpathmoveto{\pgfqpoint{3.642627in}{0.638614in}}%
\pgfpathlineto{\pgfqpoint{3.642627in}{0.638614in}}%
\pgfpathlineto{\pgfqpoint{3.642627in}{0.641563in}}%
\pgfpathlineto{\pgfqpoint{3.647168in}{0.641563in}}%
\pgfpathlineto{\pgfqpoint{3.647168in}{0.638614in}}%
\pgfpathmoveto{\pgfqpoint{3.638086in}{0.641563in}}%
\pgfpathlineto{\pgfqpoint{3.638086in}{0.641563in}}%
\pgfpathlineto{\pgfqpoint{3.638086in}{0.644512in}}%
\pgfpathlineto{\pgfqpoint{3.642627in}{0.644512in}}%
\pgfpathlineto{\pgfqpoint{3.642627in}{0.641563in}}%
\pgfpathmoveto{\pgfqpoint{3.638086in}{0.644512in}}%
\pgfpathlineto{\pgfqpoint{3.638086in}{0.644512in}}%
\pgfpathlineto{\pgfqpoint{3.638086in}{0.647462in}}%
\pgfpathlineto{\pgfqpoint{3.642627in}{0.647462in}}%
\pgfpathlineto{\pgfqpoint{3.642627in}{0.644512in}}%
\pgfpathmoveto{\pgfqpoint{3.642627in}{0.641563in}}%
\pgfpathlineto{\pgfqpoint{3.642627in}{0.641563in}}%
\pgfpathlineto{\pgfqpoint{3.642627in}{0.644512in}}%
\pgfpathlineto{\pgfqpoint{3.647168in}{0.644512in}}%
\pgfpathlineto{\pgfqpoint{3.647168in}{0.641563in}}%
\pgfpathmoveto{\pgfqpoint{3.642627in}{0.644512in}}%
\pgfpathlineto{\pgfqpoint{3.642627in}{0.644512in}}%
\pgfpathlineto{\pgfqpoint{3.642627in}{0.647462in}}%
\pgfpathlineto{\pgfqpoint{3.647168in}{0.647462in}}%
\pgfpathlineto{\pgfqpoint{3.647168in}{0.644512in}}%
\pgfpathmoveto{\pgfqpoint{3.647168in}{0.641563in}}%
\pgfpathlineto{\pgfqpoint{3.647168in}{0.641563in}}%
\pgfpathlineto{\pgfqpoint{3.647168in}{0.644512in}}%
\pgfpathlineto{\pgfqpoint{3.651709in}{0.644512in}}%
\pgfpathlineto{\pgfqpoint{3.651709in}{0.641563in}}%
\pgfpathmoveto{\pgfqpoint{3.647168in}{0.644512in}}%
\pgfpathlineto{\pgfqpoint{3.647168in}{0.644512in}}%
\pgfpathlineto{\pgfqpoint{3.647168in}{0.647462in}}%
\pgfpathlineto{\pgfqpoint{3.651709in}{0.647462in}}%
\pgfpathlineto{\pgfqpoint{3.651709in}{0.644512in}}%
\pgfpathmoveto{\pgfqpoint{3.651709in}{0.644512in}}%
\pgfpathlineto{\pgfqpoint{3.651709in}{0.644512in}}%
\pgfpathlineto{\pgfqpoint{3.651709in}{0.647462in}}%
\pgfpathlineto{\pgfqpoint{3.656250in}{0.647462in}}%
\pgfpathlineto{\pgfqpoint{3.656250in}{0.644512in}}%
\pgfpathmoveto{\pgfqpoint{3.647168in}{0.647462in}}%
\pgfpathlineto{\pgfqpoint{3.647168in}{0.647462in}}%
\pgfpathlineto{\pgfqpoint{3.647168in}{0.650411in}}%
\pgfpathlineto{\pgfqpoint{3.651709in}{0.650411in}}%
\pgfpathlineto{\pgfqpoint{3.651709in}{0.647462in}}%
\pgfpathmoveto{\pgfqpoint{3.647168in}{0.650411in}}%
\pgfpathlineto{\pgfqpoint{3.647168in}{0.650411in}}%
\pgfpathlineto{\pgfqpoint{3.647168in}{0.653360in}}%
\pgfpathlineto{\pgfqpoint{3.651709in}{0.653360in}}%
\pgfpathlineto{\pgfqpoint{3.651709in}{0.650411in}}%
\pgfpathmoveto{\pgfqpoint{3.651709in}{0.647462in}}%
\pgfpathlineto{\pgfqpoint{3.651709in}{0.647462in}}%
\pgfpathlineto{\pgfqpoint{3.651709in}{0.650411in}}%
\pgfpathlineto{\pgfqpoint{3.656250in}{0.650411in}}%
\pgfpathlineto{\pgfqpoint{3.656250in}{0.647462in}}%
\pgfpathmoveto{\pgfqpoint{3.651709in}{0.650411in}}%
\pgfpathlineto{\pgfqpoint{3.651709in}{0.650411in}}%
\pgfpathlineto{\pgfqpoint{3.651709in}{0.653360in}}%
\pgfpathlineto{\pgfqpoint{3.656250in}{0.653360in}}%
\pgfpathlineto{\pgfqpoint{3.656250in}{0.650411in}}%
\pgfpathmoveto{\pgfqpoint{3.610840in}{1.532229in}}%
\pgfpathlineto{\pgfqpoint{3.610840in}{1.532229in}}%
\pgfpathlineto{\pgfqpoint{3.610840in}{1.535178in}}%
\pgfpathlineto{\pgfqpoint{3.615381in}{1.535178in}}%
\pgfpathlineto{\pgfqpoint{3.615381in}{1.532229in}}%
\pgfpathmoveto{\pgfqpoint{3.610840in}{1.535178in}}%
\pgfpathlineto{\pgfqpoint{3.610840in}{1.535178in}}%
\pgfpathlineto{\pgfqpoint{3.610840in}{1.538128in}}%
\pgfpathlineto{\pgfqpoint{3.615381in}{1.538128in}}%
\pgfpathlineto{\pgfqpoint{3.615381in}{1.535178in}}%
\pgfpathmoveto{\pgfqpoint{3.615381in}{1.532229in}}%
\pgfpathlineto{\pgfqpoint{3.615381in}{1.532229in}}%
\pgfpathlineto{\pgfqpoint{3.615381in}{1.535178in}}%
\pgfpathlineto{\pgfqpoint{3.619922in}{1.535178in}}%
\pgfpathlineto{\pgfqpoint{3.619922in}{1.532229in}}%
\pgfpathmoveto{\pgfqpoint{3.615381in}{1.535178in}}%
\pgfpathlineto{\pgfqpoint{3.615381in}{1.535178in}}%
\pgfpathlineto{\pgfqpoint{3.615381in}{1.538128in}}%
\pgfpathlineto{\pgfqpoint{3.619922in}{1.538128in}}%
\pgfpathlineto{\pgfqpoint{3.619922in}{1.535178in}}%
\pgfpathmoveto{\pgfqpoint{3.638086in}{1.508635in}}%
\pgfpathlineto{\pgfqpoint{3.638086in}{1.508635in}}%
\pgfpathlineto{\pgfqpoint{3.638086in}{1.511584in}}%
\pgfpathlineto{\pgfqpoint{3.642627in}{1.511584in}}%
\pgfpathlineto{\pgfqpoint{3.642627in}{1.508635in}}%
\pgfpathmoveto{\pgfqpoint{3.638086in}{1.511584in}}%
\pgfpathlineto{\pgfqpoint{3.638086in}{1.511584in}}%
\pgfpathlineto{\pgfqpoint{3.638086in}{1.514534in}}%
\pgfpathlineto{\pgfqpoint{3.642627in}{1.514534in}}%
\pgfpathlineto{\pgfqpoint{3.642627in}{1.511584in}}%
\pgfpathmoveto{\pgfqpoint{3.642627in}{1.508635in}}%
\pgfpathlineto{\pgfqpoint{3.642627in}{1.508635in}}%
\pgfpathlineto{\pgfqpoint{3.642627in}{1.511584in}}%
\pgfpathlineto{\pgfqpoint{3.647168in}{1.511584in}}%
\pgfpathlineto{\pgfqpoint{3.647168in}{1.508635in}}%
\pgfpathmoveto{\pgfqpoint{3.642627in}{1.511584in}}%
\pgfpathlineto{\pgfqpoint{3.642627in}{1.511584in}}%
\pgfpathlineto{\pgfqpoint{3.642627in}{1.514534in}}%
\pgfpathlineto{\pgfqpoint{3.647168in}{1.514534in}}%
\pgfpathlineto{\pgfqpoint{3.647168in}{1.511584in}}%
\pgfpathmoveto{\pgfqpoint{3.647168in}{1.502737in}}%
\pgfpathlineto{\pgfqpoint{3.647168in}{1.502737in}}%
\pgfpathlineto{\pgfqpoint{3.647168in}{1.505686in}}%
\pgfpathlineto{\pgfqpoint{3.651709in}{1.505686in}}%
\pgfpathlineto{\pgfqpoint{3.651709in}{1.502737in}}%
\pgfpathmoveto{\pgfqpoint{3.647168in}{1.505686in}}%
\pgfpathlineto{\pgfqpoint{3.647168in}{1.505686in}}%
\pgfpathlineto{\pgfqpoint{3.647168in}{1.508635in}}%
\pgfpathlineto{\pgfqpoint{3.651709in}{1.508635in}}%
\pgfpathlineto{\pgfqpoint{3.651709in}{1.505686in}}%
\pgfpathmoveto{\pgfqpoint{3.651709in}{1.502737in}}%
\pgfpathlineto{\pgfqpoint{3.651709in}{1.502737in}}%
\pgfpathlineto{\pgfqpoint{3.651709in}{1.505686in}}%
\pgfpathlineto{\pgfqpoint{3.656250in}{1.505686in}}%
\pgfpathlineto{\pgfqpoint{3.656250in}{1.502737in}}%
\pgfpathmoveto{\pgfqpoint{3.651709in}{1.505686in}}%
\pgfpathlineto{\pgfqpoint{3.651709in}{1.505686in}}%
\pgfpathlineto{\pgfqpoint{3.651709in}{1.508635in}}%
\pgfpathlineto{\pgfqpoint{3.656250in}{1.508635in}}%
\pgfpathlineto{\pgfqpoint{3.656250in}{1.505686in}}%
\pgfpathmoveto{\pgfqpoint{3.647168in}{1.508635in}}%
\pgfpathlineto{\pgfqpoint{3.647168in}{1.508635in}}%
\pgfpathlineto{\pgfqpoint{3.647168in}{1.511584in}}%
\pgfpathlineto{\pgfqpoint{3.651709in}{1.511584in}}%
\pgfpathlineto{\pgfqpoint{3.651709in}{1.508635in}}%
\pgfpathmoveto{\pgfqpoint{3.647168in}{1.511584in}}%
\pgfpathlineto{\pgfqpoint{3.647168in}{1.511584in}}%
\pgfpathlineto{\pgfqpoint{3.647168in}{1.514534in}}%
\pgfpathlineto{\pgfqpoint{3.651709in}{1.514534in}}%
\pgfpathlineto{\pgfqpoint{3.651709in}{1.511584in}}%
\pgfpathmoveto{\pgfqpoint{3.651709in}{1.508635in}}%
\pgfpathlineto{\pgfqpoint{3.651709in}{1.508635in}}%
\pgfpathlineto{\pgfqpoint{3.651709in}{1.511584in}}%
\pgfpathlineto{\pgfqpoint{3.656250in}{1.511584in}}%
\pgfpathlineto{\pgfqpoint{3.656250in}{1.508635in}}%
\pgfpathmoveto{\pgfqpoint{3.629004in}{1.520432in}}%
\pgfpathlineto{\pgfqpoint{3.629004in}{1.520432in}}%
\pgfpathlineto{\pgfqpoint{3.629004in}{1.523381in}}%
\pgfpathlineto{\pgfqpoint{3.633545in}{1.523381in}}%
\pgfpathlineto{\pgfqpoint{3.633545in}{1.520432in}}%
\pgfpathmoveto{\pgfqpoint{3.629004in}{1.523381in}}%
\pgfpathlineto{\pgfqpoint{3.629004in}{1.523381in}}%
\pgfpathlineto{\pgfqpoint{3.629004in}{1.526331in}}%
\pgfpathlineto{\pgfqpoint{3.633545in}{1.526331in}}%
\pgfpathlineto{\pgfqpoint{3.633545in}{1.523381in}}%
\pgfpathmoveto{\pgfqpoint{3.633545in}{1.520432in}}%
\pgfpathlineto{\pgfqpoint{3.633545in}{1.520432in}}%
\pgfpathlineto{\pgfqpoint{3.633545in}{1.523381in}}%
\pgfpathlineto{\pgfqpoint{3.638086in}{1.523381in}}%
\pgfpathlineto{\pgfqpoint{3.638086in}{1.520432in}}%
\pgfpathmoveto{\pgfqpoint{3.633545in}{1.523381in}}%
\pgfpathlineto{\pgfqpoint{3.633545in}{1.523381in}}%
\pgfpathlineto{\pgfqpoint{3.633545in}{1.526331in}}%
\pgfpathlineto{\pgfqpoint{3.638086in}{1.526331in}}%
\pgfpathlineto{\pgfqpoint{3.638086in}{1.523381in}}%
\pgfpathmoveto{\pgfqpoint{3.619922in}{1.526331in}}%
\pgfpathlineto{\pgfqpoint{3.619922in}{1.526331in}}%
\pgfpathlineto{\pgfqpoint{3.619922in}{1.529280in}}%
\pgfpathlineto{\pgfqpoint{3.624463in}{1.529280in}}%
\pgfpathlineto{\pgfqpoint{3.624463in}{1.526331in}}%
\pgfpathmoveto{\pgfqpoint{3.619922in}{1.529280in}}%
\pgfpathlineto{\pgfqpoint{3.619922in}{1.529280in}}%
\pgfpathlineto{\pgfqpoint{3.619922in}{1.532229in}}%
\pgfpathlineto{\pgfqpoint{3.624463in}{1.532229in}}%
\pgfpathlineto{\pgfqpoint{3.624463in}{1.529280in}}%
\pgfpathmoveto{\pgfqpoint{3.624463in}{1.526331in}}%
\pgfpathlineto{\pgfqpoint{3.624463in}{1.526331in}}%
\pgfpathlineto{\pgfqpoint{3.624463in}{1.529280in}}%
\pgfpathlineto{\pgfqpoint{3.629004in}{1.529280in}}%
\pgfpathlineto{\pgfqpoint{3.629004in}{1.526331in}}%
\pgfpathmoveto{\pgfqpoint{3.624463in}{1.529280in}}%
\pgfpathlineto{\pgfqpoint{3.624463in}{1.529280in}}%
\pgfpathlineto{\pgfqpoint{3.624463in}{1.532229in}}%
\pgfpathlineto{\pgfqpoint{3.629004in}{1.532229in}}%
\pgfpathlineto{\pgfqpoint{3.629004in}{1.529280in}}%
\pgfpathmoveto{\pgfqpoint{3.619922in}{1.532229in}}%
\pgfpathlineto{\pgfqpoint{3.619922in}{1.532229in}}%
\pgfpathlineto{\pgfqpoint{3.619922in}{1.535178in}}%
\pgfpathlineto{\pgfqpoint{3.624463in}{1.535178in}}%
\pgfpathlineto{\pgfqpoint{3.624463in}{1.532229in}}%
\pgfpathmoveto{\pgfqpoint{3.619922in}{1.535178in}}%
\pgfpathlineto{\pgfqpoint{3.619922in}{1.535178in}}%
\pgfpathlineto{\pgfqpoint{3.619922in}{1.538128in}}%
\pgfpathlineto{\pgfqpoint{3.624463in}{1.538128in}}%
\pgfpathlineto{\pgfqpoint{3.624463in}{1.535178in}}%
\pgfpathmoveto{\pgfqpoint{3.624463in}{1.532229in}}%
\pgfpathlineto{\pgfqpoint{3.624463in}{1.532229in}}%
\pgfpathlineto{\pgfqpoint{3.624463in}{1.535178in}}%
\pgfpathlineto{\pgfqpoint{3.629004in}{1.535178in}}%
\pgfpathlineto{\pgfqpoint{3.629004in}{1.532229in}}%
\pgfpathmoveto{\pgfqpoint{3.629004in}{1.526331in}}%
\pgfpathlineto{\pgfqpoint{3.629004in}{1.526331in}}%
\pgfpathlineto{\pgfqpoint{3.629004in}{1.529280in}}%
\pgfpathlineto{\pgfqpoint{3.633545in}{1.529280in}}%
\pgfpathlineto{\pgfqpoint{3.633545in}{1.526331in}}%
\pgfpathmoveto{\pgfqpoint{3.629004in}{1.529280in}}%
\pgfpathlineto{\pgfqpoint{3.629004in}{1.529280in}}%
\pgfpathlineto{\pgfqpoint{3.629004in}{1.532229in}}%
\pgfpathlineto{\pgfqpoint{3.633545in}{1.532229in}}%
\pgfpathlineto{\pgfqpoint{3.633545in}{1.529280in}}%
\pgfpathmoveto{\pgfqpoint{3.638086in}{1.514534in}}%
\pgfpathlineto{\pgfqpoint{3.638086in}{1.514534in}}%
\pgfpathlineto{\pgfqpoint{3.638086in}{1.517483in}}%
\pgfpathlineto{\pgfqpoint{3.642627in}{1.517483in}}%
\pgfpathlineto{\pgfqpoint{3.642627in}{1.514534in}}%
\pgfpathmoveto{\pgfqpoint{3.638086in}{1.517483in}}%
\pgfpathlineto{\pgfqpoint{3.638086in}{1.517483in}}%
\pgfpathlineto{\pgfqpoint{3.638086in}{1.520432in}}%
\pgfpathlineto{\pgfqpoint{3.642627in}{1.520432in}}%
\pgfpathlineto{\pgfqpoint{3.642627in}{1.517483in}}%
\pgfpathmoveto{\pgfqpoint{3.642627in}{1.514534in}}%
\pgfpathlineto{\pgfqpoint{3.642627in}{1.514534in}}%
\pgfpathlineto{\pgfqpoint{3.642627in}{1.517483in}}%
\pgfpathlineto{\pgfqpoint{3.647168in}{1.517483in}}%
\pgfpathlineto{\pgfqpoint{3.647168in}{1.514534in}}%
\pgfpathmoveto{\pgfqpoint{3.642627in}{1.517483in}}%
\pgfpathlineto{\pgfqpoint{3.642627in}{1.517483in}}%
\pgfpathlineto{\pgfqpoint{3.642627in}{1.520432in}}%
\pgfpathlineto{\pgfqpoint{3.647168in}{1.520432in}}%
\pgfpathlineto{\pgfqpoint{3.647168in}{1.517483in}}%
\pgfpathmoveto{\pgfqpoint{3.638086in}{1.520432in}}%
\pgfpathlineto{\pgfqpoint{3.638086in}{1.520432in}}%
\pgfpathlineto{\pgfqpoint{3.638086in}{1.523381in}}%
\pgfpathlineto{\pgfqpoint{3.642627in}{1.523381in}}%
\pgfpathlineto{\pgfqpoint{3.642627in}{1.520432in}}%
\pgfpathmoveto{\pgfqpoint{3.556348in}{1.579415in}}%
\pgfpathlineto{\pgfqpoint{3.556348in}{1.579415in}}%
\pgfpathlineto{\pgfqpoint{3.556348in}{1.582364in}}%
\pgfpathlineto{\pgfqpoint{3.560889in}{1.582364in}}%
\pgfpathlineto{\pgfqpoint{3.560889in}{1.579415in}}%
\pgfpathmoveto{\pgfqpoint{3.556348in}{1.582364in}}%
\pgfpathlineto{\pgfqpoint{3.556348in}{1.582364in}}%
\pgfpathlineto{\pgfqpoint{3.556348in}{1.585313in}}%
\pgfpathlineto{\pgfqpoint{3.560889in}{1.585313in}}%
\pgfpathlineto{\pgfqpoint{3.560889in}{1.582364in}}%
\pgfpathmoveto{\pgfqpoint{3.560889in}{1.579415in}}%
\pgfpathlineto{\pgfqpoint{3.560889in}{1.579415in}}%
\pgfpathlineto{\pgfqpoint{3.560889in}{1.582364in}}%
\pgfpathlineto{\pgfqpoint{3.565430in}{1.582364in}}%
\pgfpathlineto{\pgfqpoint{3.565430in}{1.579415in}}%
\pgfpathmoveto{\pgfqpoint{3.560889in}{1.582364in}}%
\pgfpathlineto{\pgfqpoint{3.560889in}{1.582364in}}%
\pgfpathlineto{\pgfqpoint{3.560889in}{1.585313in}}%
\pgfpathlineto{\pgfqpoint{3.565430in}{1.585313in}}%
\pgfpathlineto{\pgfqpoint{3.565430in}{1.582364in}}%
\pgfpathmoveto{\pgfqpoint{3.574512in}{1.567619in}}%
\pgfpathlineto{\pgfqpoint{3.574512in}{1.567619in}}%
\pgfpathlineto{\pgfqpoint{3.574512in}{1.570568in}}%
\pgfpathlineto{\pgfqpoint{3.579053in}{1.570568in}}%
\pgfpathlineto{\pgfqpoint{3.579053in}{1.567619in}}%
\pgfpathmoveto{\pgfqpoint{3.574512in}{1.570568in}}%
\pgfpathlineto{\pgfqpoint{3.574512in}{1.570568in}}%
\pgfpathlineto{\pgfqpoint{3.574512in}{1.573517in}}%
\pgfpathlineto{\pgfqpoint{3.579053in}{1.573517in}}%
\pgfpathlineto{\pgfqpoint{3.579053in}{1.570568in}}%
\pgfpathmoveto{\pgfqpoint{3.579053in}{1.567619in}}%
\pgfpathlineto{\pgfqpoint{3.579053in}{1.567619in}}%
\pgfpathlineto{\pgfqpoint{3.579053in}{1.570568in}}%
\pgfpathlineto{\pgfqpoint{3.583594in}{1.570568in}}%
\pgfpathlineto{\pgfqpoint{3.583594in}{1.567619in}}%
\pgfpathmoveto{\pgfqpoint{3.579053in}{1.570568in}}%
\pgfpathlineto{\pgfqpoint{3.579053in}{1.570568in}}%
\pgfpathlineto{\pgfqpoint{3.579053in}{1.573517in}}%
\pgfpathlineto{\pgfqpoint{3.583594in}{1.573517in}}%
\pgfpathlineto{\pgfqpoint{3.583594in}{1.570568in}}%
\pgfpathmoveto{\pgfqpoint{3.565430in}{1.573517in}}%
\pgfpathlineto{\pgfqpoint{3.565430in}{1.573517in}}%
\pgfpathlineto{\pgfqpoint{3.565430in}{1.576466in}}%
\pgfpathlineto{\pgfqpoint{3.569971in}{1.576466in}}%
\pgfpathlineto{\pgfqpoint{3.569971in}{1.573517in}}%
\pgfpathmoveto{\pgfqpoint{3.565430in}{1.576466in}}%
\pgfpathlineto{\pgfqpoint{3.565430in}{1.576466in}}%
\pgfpathlineto{\pgfqpoint{3.565430in}{1.579415in}}%
\pgfpathlineto{\pgfqpoint{3.569971in}{1.579415in}}%
\pgfpathlineto{\pgfqpoint{3.569971in}{1.576466in}}%
\pgfpathmoveto{\pgfqpoint{3.569971in}{1.573517in}}%
\pgfpathlineto{\pgfqpoint{3.569971in}{1.573517in}}%
\pgfpathlineto{\pgfqpoint{3.569971in}{1.576466in}}%
\pgfpathlineto{\pgfqpoint{3.574512in}{1.576466in}}%
\pgfpathlineto{\pgfqpoint{3.574512in}{1.573517in}}%
\pgfpathmoveto{\pgfqpoint{3.569971in}{1.576466in}}%
\pgfpathlineto{\pgfqpoint{3.569971in}{1.576466in}}%
\pgfpathlineto{\pgfqpoint{3.569971in}{1.579415in}}%
\pgfpathlineto{\pgfqpoint{3.574512in}{1.579415in}}%
\pgfpathlineto{\pgfqpoint{3.574512in}{1.576466in}}%
\pgfpathmoveto{\pgfqpoint{3.565430in}{1.579415in}}%
\pgfpathlineto{\pgfqpoint{3.565430in}{1.579415in}}%
\pgfpathlineto{\pgfqpoint{3.565430in}{1.582364in}}%
\pgfpathlineto{\pgfqpoint{3.569971in}{1.582364in}}%
\pgfpathlineto{\pgfqpoint{3.569971in}{1.579415in}}%
\pgfpathmoveto{\pgfqpoint{3.565430in}{1.582364in}}%
\pgfpathlineto{\pgfqpoint{3.565430in}{1.582364in}}%
\pgfpathlineto{\pgfqpoint{3.565430in}{1.585313in}}%
\pgfpathlineto{\pgfqpoint{3.569971in}{1.585313in}}%
\pgfpathlineto{\pgfqpoint{3.569971in}{1.582364in}}%
\pgfpathmoveto{\pgfqpoint{3.569971in}{1.579415in}}%
\pgfpathlineto{\pgfqpoint{3.569971in}{1.579415in}}%
\pgfpathlineto{\pgfqpoint{3.569971in}{1.582364in}}%
\pgfpathlineto{\pgfqpoint{3.574512in}{1.582364in}}%
\pgfpathlineto{\pgfqpoint{3.574512in}{1.579415in}}%
\pgfpathmoveto{\pgfqpoint{3.574512in}{1.573517in}}%
\pgfpathlineto{\pgfqpoint{3.574512in}{1.573517in}}%
\pgfpathlineto{\pgfqpoint{3.574512in}{1.576466in}}%
\pgfpathlineto{\pgfqpoint{3.579053in}{1.576466in}}%
\pgfpathlineto{\pgfqpoint{3.579053in}{1.573517in}}%
\pgfpathmoveto{\pgfqpoint{3.574512in}{1.576466in}}%
\pgfpathlineto{\pgfqpoint{3.574512in}{1.576466in}}%
\pgfpathlineto{\pgfqpoint{3.574512in}{1.579415in}}%
\pgfpathlineto{\pgfqpoint{3.579053in}{1.579415in}}%
\pgfpathlineto{\pgfqpoint{3.579053in}{1.576466in}}%
\pgfpathmoveto{\pgfqpoint{3.529103in}{1.603008in}}%
\pgfpathlineto{\pgfqpoint{3.529103in}{1.603008in}}%
\pgfpathlineto{\pgfqpoint{3.529103in}{1.605957in}}%
\pgfpathlineto{\pgfqpoint{3.533644in}{1.605957in}}%
\pgfpathlineto{\pgfqpoint{3.533644in}{1.603008in}}%
\pgfpathmoveto{\pgfqpoint{3.529103in}{1.605957in}}%
\pgfpathlineto{\pgfqpoint{3.529103in}{1.605957in}}%
\pgfpathlineto{\pgfqpoint{3.529103in}{1.608906in}}%
\pgfpathlineto{\pgfqpoint{3.533644in}{1.608906in}}%
\pgfpathlineto{\pgfqpoint{3.533644in}{1.605957in}}%
\pgfpathmoveto{\pgfqpoint{3.533644in}{1.603008in}}%
\pgfpathlineto{\pgfqpoint{3.533644in}{1.603008in}}%
\pgfpathlineto{\pgfqpoint{3.533644in}{1.605957in}}%
\pgfpathlineto{\pgfqpoint{3.538185in}{1.605957in}}%
\pgfpathlineto{\pgfqpoint{3.538185in}{1.603008in}}%
\pgfpathmoveto{\pgfqpoint{3.533644in}{1.605957in}}%
\pgfpathlineto{\pgfqpoint{3.533644in}{1.605957in}}%
\pgfpathlineto{\pgfqpoint{3.533644in}{1.608906in}}%
\pgfpathlineto{\pgfqpoint{3.538185in}{1.608906in}}%
\pgfpathlineto{\pgfqpoint{3.538185in}{1.605957in}}%
\pgfpathmoveto{\pgfqpoint{3.538185in}{1.597110in}}%
\pgfpathlineto{\pgfqpoint{3.538185in}{1.597110in}}%
\pgfpathlineto{\pgfqpoint{3.538185in}{1.600059in}}%
\pgfpathlineto{\pgfqpoint{3.542725in}{1.600059in}}%
\pgfpathlineto{\pgfqpoint{3.542725in}{1.597110in}}%
\pgfpathmoveto{\pgfqpoint{3.538185in}{1.600059in}}%
\pgfpathlineto{\pgfqpoint{3.538185in}{1.600059in}}%
\pgfpathlineto{\pgfqpoint{3.538185in}{1.603008in}}%
\pgfpathlineto{\pgfqpoint{3.542725in}{1.603008in}}%
\pgfpathlineto{\pgfqpoint{3.542725in}{1.600059in}}%
\pgfpathmoveto{\pgfqpoint{3.542725in}{1.597110in}}%
\pgfpathlineto{\pgfqpoint{3.542725in}{1.597110in}}%
\pgfpathlineto{\pgfqpoint{3.542725in}{1.600059in}}%
\pgfpathlineto{\pgfqpoint{3.547266in}{1.600059in}}%
\pgfpathlineto{\pgfqpoint{3.547266in}{1.597110in}}%
\pgfpathmoveto{\pgfqpoint{3.542725in}{1.600059in}}%
\pgfpathlineto{\pgfqpoint{3.542725in}{1.600059in}}%
\pgfpathlineto{\pgfqpoint{3.542725in}{1.603008in}}%
\pgfpathlineto{\pgfqpoint{3.547266in}{1.603008in}}%
\pgfpathlineto{\pgfqpoint{3.547266in}{1.600059in}}%
\pgfpathmoveto{\pgfqpoint{3.538185in}{1.603008in}}%
\pgfpathlineto{\pgfqpoint{3.538185in}{1.603008in}}%
\pgfpathlineto{\pgfqpoint{3.538185in}{1.605957in}}%
\pgfpathlineto{\pgfqpoint{3.542725in}{1.605957in}}%
\pgfpathlineto{\pgfqpoint{3.542725in}{1.603008in}}%
\pgfpathmoveto{\pgfqpoint{3.538185in}{1.605957in}}%
\pgfpathlineto{\pgfqpoint{3.538185in}{1.605957in}}%
\pgfpathlineto{\pgfqpoint{3.538185in}{1.608906in}}%
\pgfpathlineto{\pgfqpoint{3.542725in}{1.608906in}}%
\pgfpathlineto{\pgfqpoint{3.542725in}{1.605957in}}%
\pgfpathmoveto{\pgfqpoint{3.542725in}{1.603008in}}%
\pgfpathlineto{\pgfqpoint{3.542725in}{1.603008in}}%
\pgfpathlineto{\pgfqpoint{3.542725in}{1.605957in}}%
\pgfpathlineto{\pgfqpoint{3.547266in}{1.605957in}}%
\pgfpathlineto{\pgfqpoint{3.547266in}{1.603008in}}%
\pgfpathmoveto{\pgfqpoint{3.520021in}{1.614804in}}%
\pgfpathlineto{\pgfqpoint{3.520021in}{1.614804in}}%
\pgfpathlineto{\pgfqpoint{3.520021in}{1.617754in}}%
\pgfpathlineto{\pgfqpoint{3.524562in}{1.617754in}}%
\pgfpathlineto{\pgfqpoint{3.524562in}{1.614804in}}%
\pgfpathmoveto{\pgfqpoint{3.520021in}{1.617754in}}%
\pgfpathlineto{\pgfqpoint{3.520021in}{1.617754in}}%
\pgfpathlineto{\pgfqpoint{3.520021in}{1.620703in}}%
\pgfpathlineto{\pgfqpoint{3.524562in}{1.620703in}}%
\pgfpathlineto{\pgfqpoint{3.524562in}{1.617754in}}%
\pgfpathmoveto{\pgfqpoint{3.524562in}{1.614804in}}%
\pgfpathlineto{\pgfqpoint{3.524562in}{1.614804in}}%
\pgfpathlineto{\pgfqpoint{3.524562in}{1.617754in}}%
\pgfpathlineto{\pgfqpoint{3.529103in}{1.617754in}}%
\pgfpathlineto{\pgfqpoint{3.529103in}{1.614804in}}%
\pgfpathmoveto{\pgfqpoint{3.524562in}{1.617754in}}%
\pgfpathlineto{\pgfqpoint{3.524562in}{1.617754in}}%
\pgfpathlineto{\pgfqpoint{3.524562in}{1.620703in}}%
\pgfpathlineto{\pgfqpoint{3.529103in}{1.620703in}}%
\pgfpathlineto{\pgfqpoint{3.529103in}{1.617754in}}%
\pgfpathmoveto{\pgfqpoint{3.510939in}{1.620703in}}%
\pgfpathlineto{\pgfqpoint{3.510939in}{1.620703in}}%
\pgfpathlineto{\pgfqpoint{3.510939in}{1.623652in}}%
\pgfpathlineto{\pgfqpoint{3.515480in}{1.623652in}}%
\pgfpathlineto{\pgfqpoint{3.515480in}{1.620703in}}%
\pgfpathmoveto{\pgfqpoint{3.510939in}{1.623652in}}%
\pgfpathlineto{\pgfqpoint{3.510939in}{1.623652in}}%
\pgfpathlineto{\pgfqpoint{3.510939in}{1.626601in}}%
\pgfpathlineto{\pgfqpoint{3.515480in}{1.626601in}}%
\pgfpathlineto{\pgfqpoint{3.515480in}{1.623652in}}%
\pgfpathmoveto{\pgfqpoint{3.515480in}{1.620703in}}%
\pgfpathlineto{\pgfqpoint{3.515480in}{1.620703in}}%
\pgfpathlineto{\pgfqpoint{3.515480in}{1.623652in}}%
\pgfpathlineto{\pgfqpoint{3.520021in}{1.623652in}}%
\pgfpathlineto{\pgfqpoint{3.520021in}{1.620703in}}%
\pgfpathmoveto{\pgfqpoint{3.515480in}{1.623652in}}%
\pgfpathlineto{\pgfqpoint{3.515480in}{1.623652in}}%
\pgfpathlineto{\pgfqpoint{3.515480in}{1.626601in}}%
\pgfpathlineto{\pgfqpoint{3.520021in}{1.626601in}}%
\pgfpathlineto{\pgfqpoint{3.520021in}{1.623652in}}%
\pgfpathmoveto{\pgfqpoint{3.510939in}{1.626601in}}%
\pgfpathlineto{\pgfqpoint{3.510939in}{1.626601in}}%
\pgfpathlineto{\pgfqpoint{3.510939in}{1.629550in}}%
\pgfpathlineto{\pgfqpoint{3.515480in}{1.629550in}}%
\pgfpathlineto{\pgfqpoint{3.515480in}{1.626601in}}%
\pgfpathmoveto{\pgfqpoint{3.510939in}{1.629550in}}%
\pgfpathlineto{\pgfqpoint{3.510939in}{1.629550in}}%
\pgfpathlineto{\pgfqpoint{3.510939in}{1.632499in}}%
\pgfpathlineto{\pgfqpoint{3.515480in}{1.632499in}}%
\pgfpathlineto{\pgfqpoint{3.515480in}{1.629550in}}%
\pgfpathmoveto{\pgfqpoint{3.515480in}{1.626601in}}%
\pgfpathlineto{\pgfqpoint{3.515480in}{1.626601in}}%
\pgfpathlineto{\pgfqpoint{3.515480in}{1.629550in}}%
\pgfpathlineto{\pgfqpoint{3.520021in}{1.629550in}}%
\pgfpathlineto{\pgfqpoint{3.520021in}{1.626601in}}%
\pgfpathmoveto{\pgfqpoint{3.520021in}{1.620703in}}%
\pgfpathlineto{\pgfqpoint{3.520021in}{1.620703in}}%
\pgfpathlineto{\pgfqpoint{3.520021in}{1.623652in}}%
\pgfpathlineto{\pgfqpoint{3.524562in}{1.623652in}}%
\pgfpathlineto{\pgfqpoint{3.524562in}{1.620703in}}%
\pgfpathmoveto{\pgfqpoint{3.520021in}{1.623652in}}%
\pgfpathlineto{\pgfqpoint{3.520021in}{1.623652in}}%
\pgfpathlineto{\pgfqpoint{3.520021in}{1.626601in}}%
\pgfpathlineto{\pgfqpoint{3.524562in}{1.626601in}}%
\pgfpathlineto{\pgfqpoint{3.524562in}{1.623652in}}%
\pgfpathmoveto{\pgfqpoint{3.529103in}{1.608906in}}%
\pgfpathlineto{\pgfqpoint{3.529103in}{1.608906in}}%
\pgfpathlineto{\pgfqpoint{3.529103in}{1.611855in}}%
\pgfpathlineto{\pgfqpoint{3.533644in}{1.611855in}}%
\pgfpathlineto{\pgfqpoint{3.533644in}{1.608906in}}%
\pgfpathmoveto{\pgfqpoint{3.529103in}{1.611855in}}%
\pgfpathlineto{\pgfqpoint{3.529103in}{1.611855in}}%
\pgfpathlineto{\pgfqpoint{3.529103in}{1.614804in}}%
\pgfpathlineto{\pgfqpoint{3.533644in}{1.614804in}}%
\pgfpathlineto{\pgfqpoint{3.533644in}{1.611855in}}%
\pgfpathmoveto{\pgfqpoint{3.533644in}{1.608906in}}%
\pgfpathlineto{\pgfqpoint{3.533644in}{1.608906in}}%
\pgfpathlineto{\pgfqpoint{3.533644in}{1.611855in}}%
\pgfpathlineto{\pgfqpoint{3.538185in}{1.611855in}}%
\pgfpathlineto{\pgfqpoint{3.538185in}{1.608906in}}%
\pgfpathmoveto{\pgfqpoint{3.533644in}{1.611855in}}%
\pgfpathlineto{\pgfqpoint{3.533644in}{1.611855in}}%
\pgfpathlineto{\pgfqpoint{3.533644in}{1.614804in}}%
\pgfpathlineto{\pgfqpoint{3.538185in}{1.614804in}}%
\pgfpathlineto{\pgfqpoint{3.538185in}{1.611855in}}%
\pgfpathmoveto{\pgfqpoint{3.529103in}{1.614804in}}%
\pgfpathlineto{\pgfqpoint{3.529103in}{1.614804in}}%
\pgfpathlineto{\pgfqpoint{3.529103in}{1.617754in}}%
\pgfpathlineto{\pgfqpoint{3.533644in}{1.617754in}}%
\pgfpathlineto{\pgfqpoint{3.533644in}{1.614804in}}%
\pgfpathmoveto{\pgfqpoint{3.547266in}{1.591212in}}%
\pgfpathlineto{\pgfqpoint{3.547266in}{1.591212in}}%
\pgfpathlineto{\pgfqpoint{3.547266in}{1.594161in}}%
\pgfpathlineto{\pgfqpoint{3.551807in}{1.594161in}}%
\pgfpathlineto{\pgfqpoint{3.551807in}{1.591212in}}%
\pgfpathmoveto{\pgfqpoint{3.547266in}{1.594161in}}%
\pgfpathlineto{\pgfqpoint{3.547266in}{1.594161in}}%
\pgfpathlineto{\pgfqpoint{3.547266in}{1.597110in}}%
\pgfpathlineto{\pgfqpoint{3.551807in}{1.597110in}}%
\pgfpathlineto{\pgfqpoint{3.551807in}{1.594161in}}%
\pgfpathmoveto{\pgfqpoint{3.551807in}{1.591212in}}%
\pgfpathlineto{\pgfqpoint{3.551807in}{1.591212in}}%
\pgfpathlineto{\pgfqpoint{3.551807in}{1.594161in}}%
\pgfpathlineto{\pgfqpoint{3.556348in}{1.594161in}}%
\pgfpathlineto{\pgfqpoint{3.556348in}{1.591212in}}%
\pgfpathmoveto{\pgfqpoint{3.551807in}{1.594161in}}%
\pgfpathlineto{\pgfqpoint{3.551807in}{1.594161in}}%
\pgfpathlineto{\pgfqpoint{3.551807in}{1.597110in}}%
\pgfpathlineto{\pgfqpoint{3.556348in}{1.597110in}}%
\pgfpathlineto{\pgfqpoint{3.556348in}{1.594161in}}%
\pgfpathmoveto{\pgfqpoint{3.556348in}{1.585313in}}%
\pgfpathlineto{\pgfqpoint{3.556348in}{1.585313in}}%
\pgfpathlineto{\pgfqpoint{3.556348in}{1.588262in}}%
\pgfpathlineto{\pgfqpoint{3.560889in}{1.588262in}}%
\pgfpathlineto{\pgfqpoint{3.560889in}{1.585313in}}%
\pgfpathmoveto{\pgfqpoint{3.556348in}{1.588262in}}%
\pgfpathlineto{\pgfqpoint{3.556348in}{1.588262in}}%
\pgfpathlineto{\pgfqpoint{3.556348in}{1.591212in}}%
\pgfpathlineto{\pgfqpoint{3.560889in}{1.591212in}}%
\pgfpathlineto{\pgfqpoint{3.560889in}{1.588262in}}%
\pgfpathmoveto{\pgfqpoint{3.560889in}{1.585313in}}%
\pgfpathlineto{\pgfqpoint{3.560889in}{1.585313in}}%
\pgfpathlineto{\pgfqpoint{3.560889in}{1.588262in}}%
\pgfpathlineto{\pgfqpoint{3.565430in}{1.588262in}}%
\pgfpathlineto{\pgfqpoint{3.565430in}{1.585313in}}%
\pgfpathmoveto{\pgfqpoint{3.560889in}{1.588262in}}%
\pgfpathlineto{\pgfqpoint{3.560889in}{1.588262in}}%
\pgfpathlineto{\pgfqpoint{3.560889in}{1.591212in}}%
\pgfpathlineto{\pgfqpoint{3.565430in}{1.591212in}}%
\pgfpathlineto{\pgfqpoint{3.565430in}{1.588262in}}%
\pgfpathmoveto{\pgfqpoint{3.556348in}{1.591212in}}%
\pgfpathlineto{\pgfqpoint{3.556348in}{1.591212in}}%
\pgfpathlineto{\pgfqpoint{3.556348in}{1.594161in}}%
\pgfpathlineto{\pgfqpoint{3.560889in}{1.594161in}}%
\pgfpathlineto{\pgfqpoint{3.560889in}{1.591212in}}%
\pgfpathmoveto{\pgfqpoint{3.547266in}{1.597110in}}%
\pgfpathlineto{\pgfqpoint{3.547266in}{1.597110in}}%
\pgfpathlineto{\pgfqpoint{3.547266in}{1.600059in}}%
\pgfpathlineto{\pgfqpoint{3.551807in}{1.600059in}}%
\pgfpathlineto{\pgfqpoint{3.551807in}{1.597110in}}%
\pgfpathmoveto{\pgfqpoint{3.547266in}{1.600059in}}%
\pgfpathlineto{\pgfqpoint{3.547266in}{1.600059in}}%
\pgfpathlineto{\pgfqpoint{3.547266in}{1.603008in}}%
\pgfpathlineto{\pgfqpoint{3.551807in}{1.603008in}}%
\pgfpathlineto{\pgfqpoint{3.551807in}{1.600059in}}%
\pgfpathmoveto{\pgfqpoint{3.583594in}{1.555822in}}%
\pgfpathlineto{\pgfqpoint{3.583594in}{1.555822in}}%
\pgfpathlineto{\pgfqpoint{3.583594in}{1.558771in}}%
\pgfpathlineto{\pgfqpoint{3.588135in}{1.558771in}}%
\pgfpathlineto{\pgfqpoint{3.588135in}{1.555822in}}%
\pgfpathmoveto{\pgfqpoint{3.583594in}{1.558771in}}%
\pgfpathlineto{\pgfqpoint{3.583594in}{1.558771in}}%
\pgfpathlineto{\pgfqpoint{3.583594in}{1.561720in}}%
\pgfpathlineto{\pgfqpoint{3.588135in}{1.561720in}}%
\pgfpathlineto{\pgfqpoint{3.588135in}{1.558771in}}%
\pgfpathmoveto{\pgfqpoint{3.588135in}{1.555822in}}%
\pgfpathlineto{\pgfqpoint{3.588135in}{1.555822in}}%
\pgfpathlineto{\pgfqpoint{3.588135in}{1.558771in}}%
\pgfpathlineto{\pgfqpoint{3.592676in}{1.558771in}}%
\pgfpathlineto{\pgfqpoint{3.592676in}{1.555822in}}%
\pgfpathmoveto{\pgfqpoint{3.588135in}{1.558771in}}%
\pgfpathlineto{\pgfqpoint{3.588135in}{1.558771in}}%
\pgfpathlineto{\pgfqpoint{3.588135in}{1.561720in}}%
\pgfpathlineto{\pgfqpoint{3.592676in}{1.561720in}}%
\pgfpathlineto{\pgfqpoint{3.592676in}{1.558771in}}%
\pgfpathmoveto{\pgfqpoint{3.592676in}{1.549924in}}%
\pgfpathlineto{\pgfqpoint{3.592676in}{1.549924in}}%
\pgfpathlineto{\pgfqpoint{3.592676in}{1.552873in}}%
\pgfpathlineto{\pgfqpoint{3.597217in}{1.552873in}}%
\pgfpathlineto{\pgfqpoint{3.597217in}{1.549924in}}%
\pgfpathmoveto{\pgfqpoint{3.592676in}{1.552873in}}%
\pgfpathlineto{\pgfqpoint{3.592676in}{1.552873in}}%
\pgfpathlineto{\pgfqpoint{3.592676in}{1.555822in}}%
\pgfpathlineto{\pgfqpoint{3.597217in}{1.555822in}}%
\pgfpathlineto{\pgfqpoint{3.597217in}{1.552873in}}%
\pgfpathmoveto{\pgfqpoint{3.597217in}{1.549924in}}%
\pgfpathlineto{\pgfqpoint{3.597217in}{1.549924in}}%
\pgfpathlineto{\pgfqpoint{3.597217in}{1.552873in}}%
\pgfpathlineto{\pgfqpoint{3.601758in}{1.552873in}}%
\pgfpathlineto{\pgfqpoint{3.601758in}{1.549924in}}%
\pgfpathmoveto{\pgfqpoint{3.597217in}{1.552873in}}%
\pgfpathlineto{\pgfqpoint{3.597217in}{1.552873in}}%
\pgfpathlineto{\pgfqpoint{3.597217in}{1.555822in}}%
\pgfpathlineto{\pgfqpoint{3.601758in}{1.555822in}}%
\pgfpathlineto{\pgfqpoint{3.601758in}{1.552873in}}%
\pgfpathmoveto{\pgfqpoint{3.592676in}{1.555822in}}%
\pgfpathlineto{\pgfqpoint{3.592676in}{1.555822in}}%
\pgfpathlineto{\pgfqpoint{3.592676in}{1.558771in}}%
\pgfpathlineto{\pgfqpoint{3.597217in}{1.558771in}}%
\pgfpathlineto{\pgfqpoint{3.597217in}{1.555822in}}%
\pgfpathmoveto{\pgfqpoint{3.592676in}{1.558771in}}%
\pgfpathlineto{\pgfqpoint{3.592676in}{1.558771in}}%
\pgfpathlineto{\pgfqpoint{3.592676in}{1.561720in}}%
\pgfpathlineto{\pgfqpoint{3.597217in}{1.561720in}}%
\pgfpathlineto{\pgfqpoint{3.597217in}{1.558771in}}%
\pgfpathmoveto{\pgfqpoint{3.597217in}{1.555822in}}%
\pgfpathlineto{\pgfqpoint{3.597217in}{1.555822in}}%
\pgfpathlineto{\pgfqpoint{3.597217in}{1.558771in}}%
\pgfpathlineto{\pgfqpoint{3.601758in}{1.558771in}}%
\pgfpathlineto{\pgfqpoint{3.601758in}{1.555822in}}%
\pgfpathmoveto{\pgfqpoint{3.601758in}{1.544026in}}%
\pgfpathlineto{\pgfqpoint{3.601758in}{1.544026in}}%
\pgfpathlineto{\pgfqpoint{3.601758in}{1.546975in}}%
\pgfpathlineto{\pgfqpoint{3.606299in}{1.546975in}}%
\pgfpathlineto{\pgfqpoint{3.606299in}{1.544026in}}%
\pgfpathmoveto{\pgfqpoint{3.601758in}{1.546975in}}%
\pgfpathlineto{\pgfqpoint{3.601758in}{1.546975in}}%
\pgfpathlineto{\pgfqpoint{3.601758in}{1.549924in}}%
\pgfpathlineto{\pgfqpoint{3.606299in}{1.549924in}}%
\pgfpathlineto{\pgfqpoint{3.606299in}{1.546975in}}%
\pgfpathmoveto{\pgfqpoint{3.606299in}{1.544026in}}%
\pgfpathlineto{\pgfqpoint{3.606299in}{1.544026in}}%
\pgfpathlineto{\pgfqpoint{3.606299in}{1.546975in}}%
\pgfpathlineto{\pgfqpoint{3.610840in}{1.546975in}}%
\pgfpathlineto{\pgfqpoint{3.610840in}{1.544026in}}%
\pgfpathmoveto{\pgfqpoint{3.606299in}{1.546975in}}%
\pgfpathlineto{\pgfqpoint{3.606299in}{1.546975in}}%
\pgfpathlineto{\pgfqpoint{3.606299in}{1.549924in}}%
\pgfpathlineto{\pgfqpoint{3.610840in}{1.549924in}}%
\pgfpathlineto{\pgfqpoint{3.610840in}{1.546975in}}%
\pgfpathmoveto{\pgfqpoint{3.610840in}{1.538128in}}%
\pgfpathlineto{\pgfqpoint{3.610840in}{1.538128in}}%
\pgfpathlineto{\pgfqpoint{3.610840in}{1.541077in}}%
\pgfpathlineto{\pgfqpoint{3.615381in}{1.541077in}}%
\pgfpathlineto{\pgfqpoint{3.615381in}{1.538128in}}%
\pgfpathmoveto{\pgfqpoint{3.610840in}{1.541077in}}%
\pgfpathlineto{\pgfqpoint{3.610840in}{1.541077in}}%
\pgfpathlineto{\pgfqpoint{3.610840in}{1.544026in}}%
\pgfpathlineto{\pgfqpoint{3.615381in}{1.544026in}}%
\pgfpathlineto{\pgfqpoint{3.615381in}{1.541077in}}%
\pgfpathmoveto{\pgfqpoint{3.615381in}{1.538128in}}%
\pgfpathlineto{\pgfqpoint{3.615381in}{1.538128in}}%
\pgfpathlineto{\pgfqpoint{3.615381in}{1.541077in}}%
\pgfpathlineto{\pgfqpoint{3.619922in}{1.541077in}}%
\pgfpathlineto{\pgfqpoint{3.619922in}{1.538128in}}%
\pgfpathmoveto{\pgfqpoint{3.615381in}{1.541077in}}%
\pgfpathlineto{\pgfqpoint{3.615381in}{1.541077in}}%
\pgfpathlineto{\pgfqpoint{3.615381in}{1.544026in}}%
\pgfpathlineto{\pgfqpoint{3.619922in}{1.544026in}}%
\pgfpathlineto{\pgfqpoint{3.619922in}{1.541077in}}%
\pgfpathmoveto{\pgfqpoint{3.610840in}{1.544026in}}%
\pgfpathlineto{\pgfqpoint{3.610840in}{1.544026in}}%
\pgfpathlineto{\pgfqpoint{3.610840in}{1.546975in}}%
\pgfpathlineto{\pgfqpoint{3.615381in}{1.546975in}}%
\pgfpathlineto{\pgfqpoint{3.615381in}{1.544026in}}%
\pgfpathmoveto{\pgfqpoint{3.601758in}{1.549924in}}%
\pgfpathlineto{\pgfqpoint{3.601758in}{1.549924in}}%
\pgfpathlineto{\pgfqpoint{3.601758in}{1.552873in}}%
\pgfpathlineto{\pgfqpoint{3.606299in}{1.552873in}}%
\pgfpathlineto{\pgfqpoint{3.606299in}{1.549924in}}%
\pgfpathmoveto{\pgfqpoint{3.601758in}{1.552873in}}%
\pgfpathlineto{\pgfqpoint{3.601758in}{1.552873in}}%
\pgfpathlineto{\pgfqpoint{3.601758in}{1.555822in}}%
\pgfpathlineto{\pgfqpoint{3.606299in}{1.555822in}}%
\pgfpathlineto{\pgfqpoint{3.606299in}{1.552873in}}%
\pgfpathmoveto{\pgfqpoint{3.583594in}{1.561720in}}%
\pgfpathlineto{\pgfqpoint{3.583594in}{1.561720in}}%
\pgfpathlineto{\pgfqpoint{3.583594in}{1.564670in}}%
\pgfpathlineto{\pgfqpoint{3.588135in}{1.564670in}}%
\pgfpathlineto{\pgfqpoint{3.588135in}{1.561720in}}%
\pgfpathmoveto{\pgfqpoint{3.583594in}{1.564670in}}%
\pgfpathlineto{\pgfqpoint{3.583594in}{1.564670in}}%
\pgfpathlineto{\pgfqpoint{3.583594in}{1.567619in}}%
\pgfpathlineto{\pgfqpoint{3.588135in}{1.567619in}}%
\pgfpathlineto{\pgfqpoint{3.588135in}{1.564670in}}%
\pgfpathmoveto{\pgfqpoint{3.588135in}{1.561720in}}%
\pgfpathlineto{\pgfqpoint{3.588135in}{1.561720in}}%
\pgfpathlineto{\pgfqpoint{3.588135in}{1.564670in}}%
\pgfpathlineto{\pgfqpoint{3.592676in}{1.564670in}}%
\pgfpathlineto{\pgfqpoint{3.592676in}{1.561720in}}%
\pgfpathmoveto{\pgfqpoint{3.588135in}{1.564670in}}%
\pgfpathlineto{\pgfqpoint{3.588135in}{1.564670in}}%
\pgfpathlineto{\pgfqpoint{3.588135in}{1.567619in}}%
\pgfpathlineto{\pgfqpoint{3.592676in}{1.567619in}}%
\pgfpathlineto{\pgfqpoint{3.592676in}{1.564670in}}%
\pgfpathmoveto{\pgfqpoint{3.583594in}{1.567619in}}%
\pgfpathlineto{\pgfqpoint{3.583594in}{1.567619in}}%
\pgfpathlineto{\pgfqpoint{3.583594in}{1.570568in}}%
\pgfpathlineto{\pgfqpoint{3.588135in}{1.570568in}}%
\pgfpathlineto{\pgfqpoint{3.588135in}{1.567619in}}%
\pgfpathmoveto{\pgfqpoint{3.510939in}{1.632499in}}%
\pgfpathlineto{\pgfqpoint{3.510939in}{1.632499in}}%
\pgfpathlineto{\pgfqpoint{3.510939in}{1.635448in}}%
\pgfpathlineto{\pgfqpoint{3.515480in}{1.635448in}}%
\pgfpathlineto{\pgfqpoint{3.515480in}{1.632499in}}%
\pgfpathmoveto{\pgfqpoint{3.656250in}{0.650411in}}%
\pgfpathlineto{\pgfqpoint{3.656250in}{0.650411in}}%
\pgfpathlineto{\pgfqpoint{3.656250in}{0.653360in}}%
\pgfpathlineto{\pgfqpoint{3.660791in}{0.653360in}}%
\pgfpathlineto{\pgfqpoint{3.660791in}{0.650411in}}%
\pgfpathmoveto{\pgfqpoint{3.656250in}{0.653360in}}%
\pgfpathlineto{\pgfqpoint{3.656250in}{0.653360in}}%
\pgfpathlineto{\pgfqpoint{3.656250in}{0.656310in}}%
\pgfpathlineto{\pgfqpoint{3.660791in}{0.656310in}}%
\pgfpathlineto{\pgfqpoint{3.660791in}{0.653360in}}%
\pgfpathmoveto{\pgfqpoint{3.656250in}{0.656310in}}%
\pgfpathlineto{\pgfqpoint{3.656250in}{0.656310in}}%
\pgfpathlineto{\pgfqpoint{3.656250in}{0.659259in}}%
\pgfpathlineto{\pgfqpoint{3.660791in}{0.659259in}}%
\pgfpathlineto{\pgfqpoint{3.660791in}{0.656310in}}%
\pgfpathmoveto{\pgfqpoint{3.660791in}{0.653360in}}%
\pgfpathlineto{\pgfqpoint{3.660791in}{0.653360in}}%
\pgfpathlineto{\pgfqpoint{3.660791in}{0.656310in}}%
\pgfpathlineto{\pgfqpoint{3.665333in}{0.656310in}}%
\pgfpathlineto{\pgfqpoint{3.665333in}{0.653360in}}%
\pgfpathmoveto{\pgfqpoint{3.660791in}{0.656310in}}%
\pgfpathlineto{\pgfqpoint{3.660791in}{0.656310in}}%
\pgfpathlineto{\pgfqpoint{3.660791in}{0.659259in}}%
\pgfpathlineto{\pgfqpoint{3.665333in}{0.659259in}}%
\pgfpathlineto{\pgfqpoint{3.665333in}{0.656310in}}%
\pgfpathmoveto{\pgfqpoint{3.665333in}{0.656310in}}%
\pgfpathlineto{\pgfqpoint{3.665333in}{0.656310in}}%
\pgfpathlineto{\pgfqpoint{3.665333in}{0.659259in}}%
\pgfpathlineto{\pgfqpoint{3.669874in}{0.659259in}}%
\pgfpathlineto{\pgfqpoint{3.669874in}{0.656310in}}%
\pgfpathmoveto{\pgfqpoint{3.665333in}{0.659259in}}%
\pgfpathlineto{\pgfqpoint{3.665333in}{0.659259in}}%
\pgfpathlineto{\pgfqpoint{3.665333in}{0.662208in}}%
\pgfpathlineto{\pgfqpoint{3.669874in}{0.662208in}}%
\pgfpathlineto{\pgfqpoint{3.669874in}{0.659259in}}%
\pgfpathmoveto{\pgfqpoint{3.665333in}{0.662208in}}%
\pgfpathlineto{\pgfqpoint{3.665333in}{0.662208in}}%
\pgfpathlineto{\pgfqpoint{3.665333in}{0.665158in}}%
\pgfpathlineto{\pgfqpoint{3.669874in}{0.665158in}}%
\pgfpathlineto{\pgfqpoint{3.669874in}{0.662208in}}%
\pgfpathmoveto{\pgfqpoint{3.669874in}{0.659259in}}%
\pgfpathlineto{\pgfqpoint{3.669874in}{0.659259in}}%
\pgfpathlineto{\pgfqpoint{3.669874in}{0.662208in}}%
\pgfpathlineto{\pgfqpoint{3.674415in}{0.662208in}}%
\pgfpathlineto{\pgfqpoint{3.674415in}{0.659259in}}%
\pgfpathmoveto{\pgfqpoint{3.669874in}{0.662208in}}%
\pgfpathlineto{\pgfqpoint{3.669874in}{0.662208in}}%
\pgfpathlineto{\pgfqpoint{3.669874in}{0.665158in}}%
\pgfpathlineto{\pgfqpoint{3.674415in}{0.665158in}}%
\pgfpathlineto{\pgfqpoint{3.674415in}{0.662208in}}%
\pgfpathmoveto{\pgfqpoint{3.674415in}{0.662208in}}%
\pgfpathlineto{\pgfqpoint{3.674415in}{0.662208in}}%
\pgfpathlineto{\pgfqpoint{3.674415in}{0.665158in}}%
\pgfpathlineto{\pgfqpoint{3.678956in}{0.665158in}}%
\pgfpathlineto{\pgfqpoint{3.678956in}{0.662208in}}%
\pgfpathmoveto{\pgfqpoint{3.674415in}{0.665158in}}%
\pgfpathlineto{\pgfqpoint{3.674415in}{0.665158in}}%
\pgfpathlineto{\pgfqpoint{3.674415in}{0.668107in}}%
\pgfpathlineto{\pgfqpoint{3.678956in}{0.668107in}}%
\pgfpathlineto{\pgfqpoint{3.678956in}{0.665158in}}%
\pgfpathmoveto{\pgfqpoint{3.674415in}{0.668107in}}%
\pgfpathlineto{\pgfqpoint{3.674415in}{0.668107in}}%
\pgfpathlineto{\pgfqpoint{3.674415in}{0.671057in}}%
\pgfpathlineto{\pgfqpoint{3.678956in}{0.671057in}}%
\pgfpathlineto{\pgfqpoint{3.678956in}{0.668107in}}%
\pgfpathmoveto{\pgfqpoint{3.678956in}{0.668107in}}%
\pgfpathlineto{\pgfqpoint{3.678956in}{0.668107in}}%
\pgfpathlineto{\pgfqpoint{3.678956in}{0.671057in}}%
\pgfpathlineto{\pgfqpoint{3.683497in}{0.671057in}}%
\pgfpathlineto{\pgfqpoint{3.683497in}{0.668107in}}%
\pgfpathmoveto{\pgfqpoint{3.674415in}{0.671057in}}%
\pgfpathlineto{\pgfqpoint{3.674415in}{0.671057in}}%
\pgfpathlineto{\pgfqpoint{3.674415in}{0.674006in}}%
\pgfpathlineto{\pgfqpoint{3.678956in}{0.674006in}}%
\pgfpathlineto{\pgfqpoint{3.678956in}{0.671057in}}%
\pgfpathmoveto{\pgfqpoint{3.674415in}{0.674006in}}%
\pgfpathlineto{\pgfqpoint{3.674415in}{0.674006in}}%
\pgfpathlineto{\pgfqpoint{3.674415in}{0.676955in}}%
\pgfpathlineto{\pgfqpoint{3.678956in}{0.676955in}}%
\pgfpathlineto{\pgfqpoint{3.678956in}{0.674006in}}%
\pgfpathmoveto{\pgfqpoint{3.678956in}{0.671057in}}%
\pgfpathlineto{\pgfqpoint{3.678956in}{0.671057in}}%
\pgfpathlineto{\pgfqpoint{3.678956in}{0.674006in}}%
\pgfpathlineto{\pgfqpoint{3.683497in}{0.674006in}}%
\pgfpathlineto{\pgfqpoint{3.683497in}{0.671057in}}%
\pgfpathmoveto{\pgfqpoint{3.678956in}{0.674006in}}%
\pgfpathlineto{\pgfqpoint{3.678956in}{0.674006in}}%
\pgfpathlineto{\pgfqpoint{3.678956in}{0.676955in}}%
\pgfpathlineto{\pgfqpoint{3.683497in}{0.676955in}}%
\pgfpathlineto{\pgfqpoint{3.683497in}{0.674006in}}%
\pgfpathmoveto{\pgfqpoint{3.683497in}{0.671057in}}%
\pgfpathlineto{\pgfqpoint{3.683497in}{0.671057in}}%
\pgfpathlineto{\pgfqpoint{3.683497in}{0.674006in}}%
\pgfpathlineto{\pgfqpoint{3.688038in}{0.674006in}}%
\pgfpathlineto{\pgfqpoint{3.688038in}{0.671057in}}%
\pgfpathmoveto{\pgfqpoint{3.683497in}{0.674006in}}%
\pgfpathlineto{\pgfqpoint{3.683497in}{0.674006in}}%
\pgfpathlineto{\pgfqpoint{3.683497in}{0.676955in}}%
\pgfpathlineto{\pgfqpoint{3.688038in}{0.676955in}}%
\pgfpathlineto{\pgfqpoint{3.688038in}{0.674006in}}%
\pgfpathmoveto{\pgfqpoint{3.688038in}{0.674006in}}%
\pgfpathlineto{\pgfqpoint{3.688038in}{0.674006in}}%
\pgfpathlineto{\pgfqpoint{3.688038in}{0.676955in}}%
\pgfpathlineto{\pgfqpoint{3.692579in}{0.676955in}}%
\pgfpathlineto{\pgfqpoint{3.692579in}{0.674006in}}%
\pgfpathmoveto{\pgfqpoint{3.683497in}{0.676955in}}%
\pgfpathlineto{\pgfqpoint{3.683497in}{0.676955in}}%
\pgfpathlineto{\pgfqpoint{3.683497in}{0.679905in}}%
\pgfpathlineto{\pgfqpoint{3.688038in}{0.679905in}}%
\pgfpathlineto{\pgfqpoint{3.688038in}{0.676955in}}%
\pgfpathmoveto{\pgfqpoint{3.683497in}{0.679905in}}%
\pgfpathlineto{\pgfqpoint{3.683497in}{0.679905in}}%
\pgfpathlineto{\pgfqpoint{3.683497in}{0.682854in}}%
\pgfpathlineto{\pgfqpoint{3.688038in}{0.682854in}}%
\pgfpathlineto{\pgfqpoint{3.688038in}{0.679905in}}%
\pgfpathmoveto{\pgfqpoint{3.688038in}{0.676955in}}%
\pgfpathlineto{\pgfqpoint{3.688038in}{0.676955in}}%
\pgfpathlineto{\pgfqpoint{3.688038in}{0.679905in}}%
\pgfpathlineto{\pgfqpoint{3.692579in}{0.679905in}}%
\pgfpathlineto{\pgfqpoint{3.692579in}{0.676955in}}%
\pgfpathmoveto{\pgfqpoint{3.688038in}{0.679905in}}%
\pgfpathlineto{\pgfqpoint{3.688038in}{0.679905in}}%
\pgfpathlineto{\pgfqpoint{3.688038in}{0.682854in}}%
\pgfpathlineto{\pgfqpoint{3.692579in}{0.682854in}}%
\pgfpathlineto{\pgfqpoint{3.692579in}{0.679905in}}%
\pgfpathmoveto{\pgfqpoint{3.692579in}{0.676955in}}%
\pgfpathlineto{\pgfqpoint{3.692579in}{0.676955in}}%
\pgfpathlineto{\pgfqpoint{3.692579in}{0.679905in}}%
\pgfpathlineto{\pgfqpoint{3.697120in}{0.679905in}}%
\pgfpathlineto{\pgfqpoint{3.697120in}{0.676955in}}%
\pgfpathmoveto{\pgfqpoint{3.692579in}{0.679905in}}%
\pgfpathlineto{\pgfqpoint{3.692579in}{0.679905in}}%
\pgfpathlineto{\pgfqpoint{3.692579in}{0.682854in}}%
\pgfpathlineto{\pgfqpoint{3.697120in}{0.682854in}}%
\pgfpathlineto{\pgfqpoint{3.697120in}{0.679905in}}%
\pgfpathmoveto{\pgfqpoint{3.697120in}{0.679905in}}%
\pgfpathlineto{\pgfqpoint{3.697120in}{0.679905in}}%
\pgfpathlineto{\pgfqpoint{3.697120in}{0.682854in}}%
\pgfpathlineto{\pgfqpoint{3.701662in}{0.682854in}}%
\pgfpathlineto{\pgfqpoint{3.701662in}{0.679905in}}%
\pgfpathmoveto{\pgfqpoint{3.692579in}{0.682854in}}%
\pgfpathlineto{\pgfqpoint{3.692579in}{0.682854in}}%
\pgfpathlineto{\pgfqpoint{3.692579in}{0.685803in}}%
\pgfpathlineto{\pgfqpoint{3.697120in}{0.685803in}}%
\pgfpathlineto{\pgfqpoint{3.697120in}{0.682854in}}%
\pgfpathmoveto{\pgfqpoint{3.692579in}{0.685803in}}%
\pgfpathlineto{\pgfqpoint{3.692579in}{0.685803in}}%
\pgfpathlineto{\pgfqpoint{3.692579in}{0.688753in}}%
\pgfpathlineto{\pgfqpoint{3.697120in}{0.688753in}}%
\pgfpathlineto{\pgfqpoint{3.697120in}{0.685803in}}%
\pgfpathmoveto{\pgfqpoint{3.697120in}{0.682854in}}%
\pgfpathlineto{\pgfqpoint{3.697120in}{0.682854in}}%
\pgfpathlineto{\pgfqpoint{3.697120in}{0.685803in}}%
\pgfpathlineto{\pgfqpoint{3.701662in}{0.685803in}}%
\pgfpathlineto{\pgfqpoint{3.701662in}{0.682854in}}%
\pgfpathmoveto{\pgfqpoint{3.697120in}{0.685803in}}%
\pgfpathlineto{\pgfqpoint{3.697120in}{0.685803in}}%
\pgfpathlineto{\pgfqpoint{3.697120in}{0.688753in}}%
\pgfpathlineto{\pgfqpoint{3.701662in}{0.688753in}}%
\pgfpathlineto{\pgfqpoint{3.701662in}{0.685803in}}%
\pgfpathmoveto{\pgfqpoint{3.701662in}{0.685803in}}%
\pgfpathlineto{\pgfqpoint{3.701662in}{0.685803in}}%
\pgfpathlineto{\pgfqpoint{3.701662in}{0.688753in}}%
\pgfpathlineto{\pgfqpoint{3.706203in}{0.688753in}}%
\pgfpathlineto{\pgfqpoint{3.706203in}{0.685803in}}%
\pgfpathmoveto{\pgfqpoint{3.701662in}{0.688753in}}%
\pgfpathlineto{\pgfqpoint{3.701662in}{0.688753in}}%
\pgfpathlineto{\pgfqpoint{3.701662in}{0.691702in}}%
\pgfpathlineto{\pgfqpoint{3.706203in}{0.691702in}}%
\pgfpathlineto{\pgfqpoint{3.706203in}{0.688753in}}%
\pgfpathmoveto{\pgfqpoint{3.701662in}{0.691702in}}%
\pgfpathlineto{\pgfqpoint{3.701662in}{0.691702in}}%
\pgfpathlineto{\pgfqpoint{3.701662in}{0.694651in}}%
\pgfpathlineto{\pgfqpoint{3.706203in}{0.694651in}}%
\pgfpathlineto{\pgfqpoint{3.706203in}{0.691702in}}%
\pgfpathmoveto{\pgfqpoint{3.706203in}{0.688753in}}%
\pgfpathlineto{\pgfqpoint{3.706203in}{0.688753in}}%
\pgfpathlineto{\pgfqpoint{3.706203in}{0.691702in}}%
\pgfpathlineto{\pgfqpoint{3.710744in}{0.691702in}}%
\pgfpathlineto{\pgfqpoint{3.710744in}{0.688753in}}%
\pgfpathmoveto{\pgfqpoint{3.706203in}{0.691702in}}%
\pgfpathlineto{\pgfqpoint{3.706203in}{0.691702in}}%
\pgfpathlineto{\pgfqpoint{3.706203in}{0.694651in}}%
\pgfpathlineto{\pgfqpoint{3.710744in}{0.694651in}}%
\pgfpathlineto{\pgfqpoint{3.710744in}{0.691702in}}%
\pgfpathmoveto{\pgfqpoint{3.710744in}{0.691702in}}%
\pgfpathlineto{\pgfqpoint{3.710744in}{0.691702in}}%
\pgfpathlineto{\pgfqpoint{3.710744in}{0.694651in}}%
\pgfpathlineto{\pgfqpoint{3.715285in}{0.694651in}}%
\pgfpathlineto{\pgfqpoint{3.715285in}{0.691702in}}%
\pgfpathmoveto{\pgfqpoint{3.710744in}{0.694651in}}%
\pgfpathlineto{\pgfqpoint{3.710744in}{0.694651in}}%
\pgfpathlineto{\pgfqpoint{3.710744in}{0.697600in}}%
\pgfpathlineto{\pgfqpoint{3.715285in}{0.697600in}}%
\pgfpathlineto{\pgfqpoint{3.715285in}{0.694651in}}%
\pgfpathmoveto{\pgfqpoint{3.710744in}{0.697600in}}%
\pgfpathlineto{\pgfqpoint{3.710744in}{0.697600in}}%
\pgfpathlineto{\pgfqpoint{3.710744in}{0.700549in}}%
\pgfpathlineto{\pgfqpoint{3.715285in}{0.700549in}}%
\pgfpathlineto{\pgfqpoint{3.715285in}{0.697600in}}%
\pgfpathmoveto{\pgfqpoint{3.715285in}{0.694651in}}%
\pgfpathlineto{\pgfqpoint{3.715285in}{0.694651in}}%
\pgfpathlineto{\pgfqpoint{3.715285in}{0.697600in}}%
\pgfpathlineto{\pgfqpoint{3.719826in}{0.697600in}}%
\pgfpathlineto{\pgfqpoint{3.719826in}{0.694651in}}%
\pgfpathmoveto{\pgfqpoint{3.715285in}{0.697600in}}%
\pgfpathlineto{\pgfqpoint{3.715285in}{0.697600in}}%
\pgfpathlineto{\pgfqpoint{3.715285in}{0.700549in}}%
\pgfpathlineto{\pgfqpoint{3.719826in}{0.700549in}}%
\pgfpathlineto{\pgfqpoint{3.719826in}{0.697600in}}%
\pgfpathmoveto{\pgfqpoint{3.719826in}{0.697600in}}%
\pgfpathlineto{\pgfqpoint{3.719826in}{0.697600in}}%
\pgfpathlineto{\pgfqpoint{3.719826in}{0.700549in}}%
\pgfpathlineto{\pgfqpoint{3.724367in}{0.700549in}}%
\pgfpathlineto{\pgfqpoint{3.724367in}{0.697600in}}%
\pgfpathmoveto{\pgfqpoint{3.719826in}{0.700549in}}%
\pgfpathlineto{\pgfqpoint{3.719826in}{0.700549in}}%
\pgfpathlineto{\pgfqpoint{3.719826in}{0.703498in}}%
\pgfpathlineto{\pgfqpoint{3.724367in}{0.703498in}}%
\pgfpathlineto{\pgfqpoint{3.724367in}{0.700549in}}%
\pgfpathmoveto{\pgfqpoint{3.719826in}{0.703498in}}%
\pgfpathlineto{\pgfqpoint{3.719826in}{0.703498in}}%
\pgfpathlineto{\pgfqpoint{3.719826in}{0.706448in}}%
\pgfpathlineto{\pgfqpoint{3.724367in}{0.706448in}}%
\pgfpathlineto{\pgfqpoint{3.724367in}{0.703498in}}%
\pgfpathmoveto{\pgfqpoint{3.724367in}{0.703498in}}%
\pgfpathlineto{\pgfqpoint{3.724367in}{0.703498in}}%
\pgfpathlineto{\pgfqpoint{3.724367in}{0.706448in}}%
\pgfpathlineto{\pgfqpoint{3.728908in}{0.706448in}}%
\pgfpathlineto{\pgfqpoint{3.728908in}{0.703498in}}%
\pgfpathmoveto{\pgfqpoint{3.719826in}{0.706448in}}%
\pgfpathlineto{\pgfqpoint{3.719826in}{0.706448in}}%
\pgfpathlineto{\pgfqpoint{3.719826in}{0.709397in}}%
\pgfpathlineto{\pgfqpoint{3.724367in}{0.709397in}}%
\pgfpathlineto{\pgfqpoint{3.724367in}{0.706448in}}%
\pgfpathmoveto{\pgfqpoint{3.719826in}{0.709397in}}%
\pgfpathlineto{\pgfqpoint{3.719826in}{0.709397in}}%
\pgfpathlineto{\pgfqpoint{3.719826in}{0.712346in}}%
\pgfpathlineto{\pgfqpoint{3.724367in}{0.712346in}}%
\pgfpathlineto{\pgfqpoint{3.724367in}{0.709397in}}%
\pgfpathmoveto{\pgfqpoint{3.724367in}{0.706448in}}%
\pgfpathlineto{\pgfqpoint{3.724367in}{0.706448in}}%
\pgfpathlineto{\pgfqpoint{3.724367in}{0.709397in}}%
\pgfpathlineto{\pgfqpoint{3.728908in}{0.709397in}}%
\pgfpathlineto{\pgfqpoint{3.728908in}{0.706448in}}%
\pgfpathmoveto{\pgfqpoint{3.724367in}{0.709397in}}%
\pgfpathlineto{\pgfqpoint{3.724367in}{0.709397in}}%
\pgfpathlineto{\pgfqpoint{3.724367in}{0.712346in}}%
\pgfpathlineto{\pgfqpoint{3.728908in}{0.712346in}}%
\pgfpathlineto{\pgfqpoint{3.728908in}{0.709397in}}%
\pgfpathmoveto{\pgfqpoint{3.728908in}{0.706448in}}%
\pgfpathlineto{\pgfqpoint{3.728908in}{0.706448in}}%
\pgfpathlineto{\pgfqpoint{3.728908in}{0.709397in}}%
\pgfpathlineto{\pgfqpoint{3.733449in}{0.709397in}}%
\pgfpathlineto{\pgfqpoint{3.733449in}{0.706448in}}%
\pgfpathmoveto{\pgfqpoint{3.728908in}{0.709397in}}%
\pgfpathlineto{\pgfqpoint{3.728908in}{0.709397in}}%
\pgfpathlineto{\pgfqpoint{3.728908in}{0.712346in}}%
\pgfpathlineto{\pgfqpoint{3.733449in}{0.712346in}}%
\pgfpathlineto{\pgfqpoint{3.733449in}{0.709397in}}%
\pgfpathmoveto{\pgfqpoint{3.733449in}{0.709397in}}%
\pgfpathlineto{\pgfqpoint{3.733449in}{0.709397in}}%
\pgfpathlineto{\pgfqpoint{3.733449in}{0.712346in}}%
\pgfpathlineto{\pgfqpoint{3.737990in}{0.712346in}}%
\pgfpathlineto{\pgfqpoint{3.737990in}{0.709397in}}%
\pgfpathmoveto{\pgfqpoint{3.728908in}{0.712346in}}%
\pgfpathlineto{\pgfqpoint{3.728908in}{0.712346in}}%
\pgfpathlineto{\pgfqpoint{3.728908in}{0.715295in}}%
\pgfpathlineto{\pgfqpoint{3.733449in}{0.715295in}}%
\pgfpathlineto{\pgfqpoint{3.733449in}{0.712346in}}%
\pgfpathmoveto{\pgfqpoint{3.728908in}{0.715295in}}%
\pgfpathlineto{\pgfqpoint{3.728908in}{0.715295in}}%
\pgfpathlineto{\pgfqpoint{3.728908in}{0.718244in}}%
\pgfpathlineto{\pgfqpoint{3.733449in}{0.718244in}}%
\pgfpathlineto{\pgfqpoint{3.733449in}{0.715295in}}%
\pgfpathmoveto{\pgfqpoint{3.733449in}{0.712346in}}%
\pgfpathlineto{\pgfqpoint{3.733449in}{0.712346in}}%
\pgfpathlineto{\pgfqpoint{3.733449in}{0.715295in}}%
\pgfpathlineto{\pgfqpoint{3.737990in}{0.715295in}}%
\pgfpathlineto{\pgfqpoint{3.737990in}{0.712346in}}%
\pgfpathmoveto{\pgfqpoint{3.733449in}{0.715295in}}%
\pgfpathlineto{\pgfqpoint{3.733449in}{0.715295in}}%
\pgfpathlineto{\pgfqpoint{3.733449in}{0.718244in}}%
\pgfpathlineto{\pgfqpoint{3.737990in}{0.718244in}}%
\pgfpathlineto{\pgfqpoint{3.737990in}{0.715295in}}%
\pgfpathmoveto{\pgfqpoint{3.737990in}{0.712346in}}%
\pgfpathlineto{\pgfqpoint{3.737990in}{0.712346in}}%
\pgfpathlineto{\pgfqpoint{3.737990in}{0.715295in}}%
\pgfpathlineto{\pgfqpoint{3.742532in}{0.715295in}}%
\pgfpathlineto{\pgfqpoint{3.742532in}{0.712346in}}%
\pgfpathmoveto{\pgfqpoint{3.737990in}{0.715295in}}%
\pgfpathlineto{\pgfqpoint{3.737990in}{0.715295in}}%
\pgfpathlineto{\pgfqpoint{3.737990in}{0.718244in}}%
\pgfpathlineto{\pgfqpoint{3.742532in}{0.718244in}}%
\pgfpathlineto{\pgfqpoint{3.742532in}{0.715295in}}%
\pgfpathmoveto{\pgfqpoint{3.742532in}{0.715295in}}%
\pgfpathlineto{\pgfqpoint{3.742532in}{0.715295in}}%
\pgfpathlineto{\pgfqpoint{3.742532in}{0.718244in}}%
\pgfpathlineto{\pgfqpoint{3.747073in}{0.718244in}}%
\pgfpathlineto{\pgfqpoint{3.747073in}{0.715295in}}%
\pgfpathmoveto{\pgfqpoint{3.737990in}{0.718244in}}%
\pgfpathlineto{\pgfqpoint{3.737990in}{0.718244in}}%
\pgfpathlineto{\pgfqpoint{3.737990in}{0.721193in}}%
\pgfpathlineto{\pgfqpoint{3.742532in}{0.721193in}}%
\pgfpathlineto{\pgfqpoint{3.742532in}{0.718244in}}%
\pgfpathmoveto{\pgfqpoint{3.737990in}{0.721193in}}%
\pgfpathlineto{\pgfqpoint{3.737990in}{0.721193in}}%
\pgfpathlineto{\pgfqpoint{3.737990in}{0.724142in}}%
\pgfpathlineto{\pgfqpoint{3.742532in}{0.724142in}}%
\pgfpathlineto{\pgfqpoint{3.742532in}{0.721193in}}%
\pgfpathmoveto{\pgfqpoint{3.742532in}{0.718244in}}%
\pgfpathlineto{\pgfqpoint{3.742532in}{0.718244in}}%
\pgfpathlineto{\pgfqpoint{3.742532in}{0.721193in}}%
\pgfpathlineto{\pgfqpoint{3.747073in}{0.721193in}}%
\pgfpathlineto{\pgfqpoint{3.747073in}{0.718244in}}%
\pgfpathmoveto{\pgfqpoint{3.742532in}{0.721193in}}%
\pgfpathlineto{\pgfqpoint{3.742532in}{0.721193in}}%
\pgfpathlineto{\pgfqpoint{3.742532in}{0.724142in}}%
\pgfpathlineto{\pgfqpoint{3.747073in}{0.724142in}}%
\pgfpathlineto{\pgfqpoint{3.747073in}{0.721193in}}%
\pgfpathmoveto{\pgfqpoint{3.747073in}{0.721193in}}%
\pgfpathlineto{\pgfqpoint{3.747073in}{0.721193in}}%
\pgfpathlineto{\pgfqpoint{3.747073in}{0.724142in}}%
\pgfpathlineto{\pgfqpoint{3.751614in}{0.724142in}}%
\pgfpathlineto{\pgfqpoint{3.751614in}{0.721193in}}%
\pgfpathmoveto{\pgfqpoint{3.747073in}{0.724142in}}%
\pgfpathlineto{\pgfqpoint{3.747073in}{0.724142in}}%
\pgfpathlineto{\pgfqpoint{3.747073in}{0.727092in}}%
\pgfpathlineto{\pgfqpoint{3.751614in}{0.727092in}}%
\pgfpathlineto{\pgfqpoint{3.751614in}{0.724142in}}%
\pgfpathmoveto{\pgfqpoint{3.747073in}{0.727092in}}%
\pgfpathlineto{\pgfqpoint{3.747073in}{0.727092in}}%
\pgfpathlineto{\pgfqpoint{3.747073in}{0.730041in}}%
\pgfpathlineto{\pgfqpoint{3.751614in}{0.730041in}}%
\pgfpathlineto{\pgfqpoint{3.751614in}{0.727092in}}%
\pgfpathmoveto{\pgfqpoint{3.751614in}{0.724142in}}%
\pgfpathlineto{\pgfqpoint{3.751614in}{0.724142in}}%
\pgfpathlineto{\pgfqpoint{3.751614in}{0.727092in}}%
\pgfpathlineto{\pgfqpoint{3.756155in}{0.727092in}}%
\pgfpathlineto{\pgfqpoint{3.756155in}{0.724142in}}%
\pgfpathmoveto{\pgfqpoint{3.751614in}{0.727092in}}%
\pgfpathlineto{\pgfqpoint{3.751614in}{0.727092in}}%
\pgfpathlineto{\pgfqpoint{3.751614in}{0.730041in}}%
\pgfpathlineto{\pgfqpoint{3.756155in}{0.730041in}}%
\pgfpathlineto{\pgfqpoint{3.756155in}{0.727092in}}%
\pgfpathmoveto{\pgfqpoint{3.756155in}{0.727092in}}%
\pgfpathlineto{\pgfqpoint{3.756155in}{0.727092in}}%
\pgfpathlineto{\pgfqpoint{3.756155in}{0.730041in}}%
\pgfpathlineto{\pgfqpoint{3.760696in}{0.730041in}}%
\pgfpathlineto{\pgfqpoint{3.760696in}{0.727092in}}%
\pgfpathmoveto{\pgfqpoint{3.756155in}{0.730041in}}%
\pgfpathlineto{\pgfqpoint{3.756155in}{0.730041in}}%
\pgfpathlineto{\pgfqpoint{3.756155in}{0.732990in}}%
\pgfpathlineto{\pgfqpoint{3.760696in}{0.732990in}}%
\pgfpathlineto{\pgfqpoint{3.760696in}{0.730041in}}%
\pgfpathmoveto{\pgfqpoint{3.756155in}{0.732990in}}%
\pgfpathlineto{\pgfqpoint{3.756155in}{0.732990in}}%
\pgfpathlineto{\pgfqpoint{3.756155in}{0.735939in}}%
\pgfpathlineto{\pgfqpoint{3.760696in}{0.735939in}}%
\pgfpathlineto{\pgfqpoint{3.760696in}{0.732990in}}%
\pgfpathmoveto{\pgfqpoint{3.760696in}{0.730041in}}%
\pgfpathlineto{\pgfqpoint{3.760696in}{0.730041in}}%
\pgfpathlineto{\pgfqpoint{3.760696in}{0.732990in}}%
\pgfpathlineto{\pgfqpoint{3.765237in}{0.732990in}}%
\pgfpathlineto{\pgfqpoint{3.765237in}{0.730041in}}%
\pgfpathmoveto{\pgfqpoint{3.760696in}{0.732990in}}%
\pgfpathlineto{\pgfqpoint{3.760696in}{0.732990in}}%
\pgfpathlineto{\pgfqpoint{3.760696in}{0.735939in}}%
\pgfpathlineto{\pgfqpoint{3.765237in}{0.735939in}}%
\pgfpathlineto{\pgfqpoint{3.765237in}{0.732990in}}%
\pgfpathmoveto{\pgfqpoint{3.756155in}{0.735939in}}%
\pgfpathlineto{\pgfqpoint{3.756155in}{0.735939in}}%
\pgfpathlineto{\pgfqpoint{3.756155in}{0.738888in}}%
\pgfpathlineto{\pgfqpoint{3.760696in}{0.738888in}}%
\pgfpathlineto{\pgfqpoint{3.760696in}{0.735939in}}%
\pgfpathmoveto{\pgfqpoint{3.756155in}{0.738888in}}%
\pgfpathlineto{\pgfqpoint{3.756155in}{0.738888in}}%
\pgfpathlineto{\pgfqpoint{3.756155in}{0.741837in}}%
\pgfpathlineto{\pgfqpoint{3.760696in}{0.741837in}}%
\pgfpathlineto{\pgfqpoint{3.760696in}{0.738888in}}%
\pgfpathmoveto{\pgfqpoint{3.760696in}{0.735939in}}%
\pgfpathlineto{\pgfqpoint{3.760696in}{0.735939in}}%
\pgfpathlineto{\pgfqpoint{3.760696in}{0.738888in}}%
\pgfpathlineto{\pgfqpoint{3.765237in}{0.738888in}}%
\pgfpathlineto{\pgfqpoint{3.765237in}{0.735939in}}%
\pgfpathmoveto{\pgfqpoint{3.760696in}{0.738888in}}%
\pgfpathlineto{\pgfqpoint{3.760696in}{0.738888in}}%
\pgfpathlineto{\pgfqpoint{3.760696in}{0.741837in}}%
\pgfpathlineto{\pgfqpoint{3.765237in}{0.741837in}}%
\pgfpathlineto{\pgfqpoint{3.765237in}{0.738888in}}%
\pgfpathmoveto{\pgfqpoint{3.765237in}{0.735939in}}%
\pgfpathlineto{\pgfqpoint{3.765237in}{0.735939in}}%
\pgfpathlineto{\pgfqpoint{3.765237in}{0.738888in}}%
\pgfpathlineto{\pgfqpoint{3.769778in}{0.738888in}}%
\pgfpathlineto{\pgfqpoint{3.769778in}{0.735939in}}%
\pgfpathmoveto{\pgfqpoint{3.765237in}{0.738888in}}%
\pgfpathlineto{\pgfqpoint{3.765237in}{0.738888in}}%
\pgfpathlineto{\pgfqpoint{3.765237in}{0.741837in}}%
\pgfpathlineto{\pgfqpoint{3.769778in}{0.741837in}}%
\pgfpathlineto{\pgfqpoint{3.769778in}{0.738888in}}%
\pgfpathmoveto{\pgfqpoint{3.769778in}{0.738888in}}%
\pgfpathlineto{\pgfqpoint{3.769778in}{0.738888in}}%
\pgfpathlineto{\pgfqpoint{3.769778in}{0.741837in}}%
\pgfpathlineto{\pgfqpoint{3.774319in}{0.741837in}}%
\pgfpathlineto{\pgfqpoint{3.774319in}{0.738888in}}%
\pgfpathmoveto{\pgfqpoint{3.765237in}{0.741837in}}%
\pgfpathlineto{\pgfqpoint{3.765237in}{0.741837in}}%
\pgfpathlineto{\pgfqpoint{3.765237in}{0.744786in}}%
\pgfpathlineto{\pgfqpoint{3.769778in}{0.744786in}}%
\pgfpathlineto{\pgfqpoint{3.769778in}{0.741837in}}%
\pgfpathmoveto{\pgfqpoint{3.765237in}{0.744786in}}%
\pgfpathlineto{\pgfqpoint{3.765237in}{0.744786in}}%
\pgfpathlineto{\pgfqpoint{3.765237in}{0.747736in}}%
\pgfpathlineto{\pgfqpoint{3.769778in}{0.747736in}}%
\pgfpathlineto{\pgfqpoint{3.769778in}{0.744786in}}%
\pgfpathmoveto{\pgfqpoint{3.769778in}{0.741837in}}%
\pgfpathlineto{\pgfqpoint{3.769778in}{0.741837in}}%
\pgfpathlineto{\pgfqpoint{3.769778in}{0.744786in}}%
\pgfpathlineto{\pgfqpoint{3.774319in}{0.744786in}}%
\pgfpathlineto{\pgfqpoint{3.774319in}{0.741837in}}%
\pgfpathmoveto{\pgfqpoint{3.769778in}{0.744786in}}%
\pgfpathlineto{\pgfqpoint{3.769778in}{0.744786in}}%
\pgfpathlineto{\pgfqpoint{3.769778in}{0.747736in}}%
\pgfpathlineto{\pgfqpoint{3.774319in}{0.747736in}}%
\pgfpathlineto{\pgfqpoint{3.774319in}{0.744786in}}%
\pgfpathmoveto{\pgfqpoint{3.774319in}{0.741837in}}%
\pgfpathlineto{\pgfqpoint{3.774319in}{0.741837in}}%
\pgfpathlineto{\pgfqpoint{3.774319in}{0.744786in}}%
\pgfpathlineto{\pgfqpoint{3.778861in}{0.744786in}}%
\pgfpathlineto{\pgfqpoint{3.778861in}{0.741837in}}%
\pgfpathmoveto{\pgfqpoint{3.774319in}{0.744786in}}%
\pgfpathlineto{\pgfqpoint{3.774319in}{0.744786in}}%
\pgfpathlineto{\pgfqpoint{3.774319in}{0.747736in}}%
\pgfpathlineto{\pgfqpoint{3.778861in}{0.747736in}}%
\pgfpathlineto{\pgfqpoint{3.778861in}{0.744786in}}%
\pgfpathmoveto{\pgfqpoint{3.778861in}{0.744786in}}%
\pgfpathlineto{\pgfqpoint{3.778861in}{0.744786in}}%
\pgfpathlineto{\pgfqpoint{3.778861in}{0.747736in}}%
\pgfpathlineto{\pgfqpoint{3.783402in}{0.747736in}}%
\pgfpathlineto{\pgfqpoint{3.783402in}{0.744786in}}%
\pgfpathmoveto{\pgfqpoint{3.774319in}{0.747736in}}%
\pgfpathlineto{\pgfqpoint{3.774319in}{0.747736in}}%
\pgfpathlineto{\pgfqpoint{3.774319in}{0.750685in}}%
\pgfpathlineto{\pgfqpoint{3.778861in}{0.750685in}}%
\pgfpathlineto{\pgfqpoint{3.778861in}{0.747736in}}%
\pgfpathmoveto{\pgfqpoint{3.774319in}{0.750685in}}%
\pgfpathlineto{\pgfqpoint{3.774319in}{0.750685in}}%
\pgfpathlineto{\pgfqpoint{3.774319in}{0.753634in}}%
\pgfpathlineto{\pgfqpoint{3.778861in}{0.753634in}}%
\pgfpathlineto{\pgfqpoint{3.778861in}{0.750685in}}%
\pgfpathmoveto{\pgfqpoint{3.778861in}{0.747736in}}%
\pgfpathlineto{\pgfqpoint{3.778861in}{0.747736in}}%
\pgfpathlineto{\pgfqpoint{3.778861in}{0.750685in}}%
\pgfpathlineto{\pgfqpoint{3.783402in}{0.750685in}}%
\pgfpathlineto{\pgfqpoint{3.783402in}{0.747736in}}%
\pgfpathmoveto{\pgfqpoint{3.778861in}{0.750685in}}%
\pgfpathlineto{\pgfqpoint{3.778861in}{0.750685in}}%
\pgfpathlineto{\pgfqpoint{3.778861in}{0.753634in}}%
\pgfpathlineto{\pgfqpoint{3.783402in}{0.753634in}}%
\pgfpathlineto{\pgfqpoint{3.783402in}{0.750685in}}%
\pgfpathmoveto{\pgfqpoint{3.783402in}{0.747736in}}%
\pgfpathlineto{\pgfqpoint{3.783402in}{0.747736in}}%
\pgfpathlineto{\pgfqpoint{3.783402in}{0.750685in}}%
\pgfpathlineto{\pgfqpoint{3.787943in}{0.750685in}}%
\pgfpathlineto{\pgfqpoint{3.787943in}{0.747736in}}%
\pgfpathmoveto{\pgfqpoint{3.783402in}{0.750685in}}%
\pgfpathlineto{\pgfqpoint{3.783402in}{0.750685in}}%
\pgfpathlineto{\pgfqpoint{3.783402in}{0.753634in}}%
\pgfpathlineto{\pgfqpoint{3.787943in}{0.753634in}}%
\pgfpathlineto{\pgfqpoint{3.787943in}{0.750685in}}%
\pgfpathmoveto{\pgfqpoint{3.783402in}{0.753634in}}%
\pgfpathlineto{\pgfqpoint{3.783402in}{0.753634in}}%
\pgfpathlineto{\pgfqpoint{3.783402in}{0.756583in}}%
\pgfpathlineto{\pgfqpoint{3.787943in}{0.756583in}}%
\pgfpathlineto{\pgfqpoint{3.787943in}{0.753634in}}%
\pgfpathmoveto{\pgfqpoint{3.783402in}{0.756583in}}%
\pgfpathlineto{\pgfqpoint{3.783402in}{0.756583in}}%
\pgfpathlineto{\pgfqpoint{3.783402in}{0.759532in}}%
\pgfpathlineto{\pgfqpoint{3.787943in}{0.759532in}}%
\pgfpathlineto{\pgfqpoint{3.787943in}{0.756583in}}%
\pgfpathmoveto{\pgfqpoint{3.787943in}{0.753634in}}%
\pgfpathlineto{\pgfqpoint{3.787943in}{0.753634in}}%
\pgfpathlineto{\pgfqpoint{3.787943in}{0.756583in}}%
\pgfpathlineto{\pgfqpoint{3.792484in}{0.756583in}}%
\pgfpathlineto{\pgfqpoint{3.792484in}{0.753634in}}%
\pgfpathmoveto{\pgfqpoint{3.787943in}{0.756583in}}%
\pgfpathlineto{\pgfqpoint{3.787943in}{0.756583in}}%
\pgfpathlineto{\pgfqpoint{3.787943in}{0.759532in}}%
\pgfpathlineto{\pgfqpoint{3.792484in}{0.759532in}}%
\pgfpathlineto{\pgfqpoint{3.792484in}{0.756583in}}%
\pgfpathmoveto{\pgfqpoint{3.792484in}{0.756583in}}%
\pgfpathlineto{\pgfqpoint{3.792484in}{0.756583in}}%
\pgfpathlineto{\pgfqpoint{3.792484in}{0.759532in}}%
\pgfpathlineto{\pgfqpoint{3.797025in}{0.759532in}}%
\pgfpathlineto{\pgfqpoint{3.797025in}{0.756583in}}%
\pgfpathmoveto{\pgfqpoint{3.792484in}{0.759532in}}%
\pgfpathlineto{\pgfqpoint{3.792484in}{0.759532in}}%
\pgfpathlineto{\pgfqpoint{3.792484in}{0.762481in}}%
\pgfpathlineto{\pgfqpoint{3.797025in}{0.762481in}}%
\pgfpathlineto{\pgfqpoint{3.797025in}{0.759532in}}%
\pgfpathmoveto{\pgfqpoint{3.792484in}{0.762481in}}%
\pgfpathlineto{\pgfqpoint{3.792484in}{0.762481in}}%
\pgfpathlineto{\pgfqpoint{3.792484in}{0.765430in}}%
\pgfpathlineto{\pgfqpoint{3.797025in}{0.765430in}}%
\pgfpathlineto{\pgfqpoint{3.797025in}{0.762481in}}%
\pgfpathmoveto{\pgfqpoint{3.797025in}{0.759532in}}%
\pgfpathlineto{\pgfqpoint{3.797025in}{0.759532in}}%
\pgfpathlineto{\pgfqpoint{3.797025in}{0.762481in}}%
\pgfpathlineto{\pgfqpoint{3.801566in}{0.762481in}}%
\pgfpathlineto{\pgfqpoint{3.801566in}{0.759532in}}%
\pgfpathmoveto{\pgfqpoint{3.797025in}{0.762481in}}%
\pgfpathlineto{\pgfqpoint{3.797025in}{0.762481in}}%
\pgfpathlineto{\pgfqpoint{3.797025in}{0.765430in}}%
\pgfpathlineto{\pgfqpoint{3.801566in}{0.765430in}}%
\pgfpathlineto{\pgfqpoint{3.801566in}{0.762481in}}%
\pgfpathmoveto{\pgfqpoint{3.719826in}{1.437853in}}%
\pgfpathlineto{\pgfqpoint{3.719826in}{1.437853in}}%
\pgfpathlineto{\pgfqpoint{3.719826in}{1.440803in}}%
\pgfpathlineto{\pgfqpoint{3.724367in}{1.440803in}}%
\pgfpathlineto{\pgfqpoint{3.724367in}{1.437853in}}%
\pgfpathmoveto{\pgfqpoint{3.719826in}{1.440803in}}%
\pgfpathlineto{\pgfqpoint{3.719826in}{1.440803in}}%
\pgfpathlineto{\pgfqpoint{3.719826in}{1.443752in}}%
\pgfpathlineto{\pgfqpoint{3.724367in}{1.443752in}}%
\pgfpathlineto{\pgfqpoint{3.724367in}{1.440803in}}%
\pgfpathmoveto{\pgfqpoint{3.724367in}{1.437853in}}%
\pgfpathlineto{\pgfqpoint{3.724367in}{1.437853in}}%
\pgfpathlineto{\pgfqpoint{3.724367in}{1.440803in}}%
\pgfpathlineto{\pgfqpoint{3.728908in}{1.440803in}}%
\pgfpathlineto{\pgfqpoint{3.728908in}{1.437853in}}%
\pgfpathmoveto{\pgfqpoint{3.724367in}{1.440803in}}%
\pgfpathlineto{\pgfqpoint{3.724367in}{1.440803in}}%
\pgfpathlineto{\pgfqpoint{3.724367in}{1.443752in}}%
\pgfpathlineto{\pgfqpoint{3.728908in}{1.443752in}}%
\pgfpathlineto{\pgfqpoint{3.728908in}{1.440803in}}%
\pgfpathmoveto{\pgfqpoint{3.774319in}{1.390664in}}%
\pgfpathlineto{\pgfqpoint{3.774319in}{1.390664in}}%
\pgfpathlineto{\pgfqpoint{3.774319in}{1.393613in}}%
\pgfpathlineto{\pgfqpoint{3.778861in}{1.393613in}}%
\pgfpathlineto{\pgfqpoint{3.778861in}{1.390664in}}%
\pgfpathmoveto{\pgfqpoint{3.774319in}{1.393613in}}%
\pgfpathlineto{\pgfqpoint{3.774319in}{1.393613in}}%
\pgfpathlineto{\pgfqpoint{3.774319in}{1.396562in}}%
\pgfpathlineto{\pgfqpoint{3.778861in}{1.396562in}}%
\pgfpathlineto{\pgfqpoint{3.778861in}{1.393613in}}%
\pgfpathmoveto{\pgfqpoint{3.778861in}{1.390664in}}%
\pgfpathlineto{\pgfqpoint{3.778861in}{1.390664in}}%
\pgfpathlineto{\pgfqpoint{3.778861in}{1.393613in}}%
\pgfpathlineto{\pgfqpoint{3.783402in}{1.393613in}}%
\pgfpathlineto{\pgfqpoint{3.783402in}{1.390664in}}%
\pgfpathmoveto{\pgfqpoint{3.778861in}{1.393613in}}%
\pgfpathlineto{\pgfqpoint{3.778861in}{1.393613in}}%
\pgfpathlineto{\pgfqpoint{3.778861in}{1.396562in}}%
\pgfpathlineto{\pgfqpoint{3.783402in}{1.396562in}}%
\pgfpathlineto{\pgfqpoint{3.783402in}{1.393613in}}%
\pgfpathmoveto{\pgfqpoint{3.792484in}{1.378866in}}%
\pgfpathlineto{\pgfqpoint{3.792484in}{1.378866in}}%
\pgfpathlineto{\pgfqpoint{3.792484in}{1.381816in}}%
\pgfpathlineto{\pgfqpoint{3.797025in}{1.381816in}}%
\pgfpathlineto{\pgfqpoint{3.797025in}{1.378866in}}%
\pgfpathmoveto{\pgfqpoint{3.792484in}{1.381816in}}%
\pgfpathlineto{\pgfqpoint{3.792484in}{1.381816in}}%
\pgfpathlineto{\pgfqpoint{3.792484in}{1.384765in}}%
\pgfpathlineto{\pgfqpoint{3.797025in}{1.384765in}}%
\pgfpathlineto{\pgfqpoint{3.797025in}{1.381816in}}%
\pgfpathmoveto{\pgfqpoint{3.797025in}{1.378866in}}%
\pgfpathlineto{\pgfqpoint{3.797025in}{1.378866in}}%
\pgfpathlineto{\pgfqpoint{3.797025in}{1.381816in}}%
\pgfpathlineto{\pgfqpoint{3.801566in}{1.381816in}}%
\pgfpathlineto{\pgfqpoint{3.801566in}{1.378866in}}%
\pgfpathmoveto{\pgfqpoint{3.797025in}{1.381816in}}%
\pgfpathlineto{\pgfqpoint{3.797025in}{1.381816in}}%
\pgfpathlineto{\pgfqpoint{3.797025in}{1.384765in}}%
\pgfpathlineto{\pgfqpoint{3.801566in}{1.384765in}}%
\pgfpathlineto{\pgfqpoint{3.801566in}{1.381816in}}%
\pgfpathmoveto{\pgfqpoint{3.783402in}{1.384765in}}%
\pgfpathlineto{\pgfqpoint{3.783402in}{1.384765in}}%
\pgfpathlineto{\pgfqpoint{3.783402in}{1.387714in}}%
\pgfpathlineto{\pgfqpoint{3.787943in}{1.387714in}}%
\pgfpathlineto{\pgfqpoint{3.787943in}{1.384765in}}%
\pgfpathmoveto{\pgfqpoint{3.783402in}{1.387714in}}%
\pgfpathlineto{\pgfqpoint{3.783402in}{1.387714in}}%
\pgfpathlineto{\pgfqpoint{3.783402in}{1.390664in}}%
\pgfpathlineto{\pgfqpoint{3.787943in}{1.390664in}}%
\pgfpathlineto{\pgfqpoint{3.787943in}{1.387714in}}%
\pgfpathmoveto{\pgfqpoint{3.787943in}{1.384765in}}%
\pgfpathlineto{\pgfqpoint{3.787943in}{1.384765in}}%
\pgfpathlineto{\pgfqpoint{3.787943in}{1.387714in}}%
\pgfpathlineto{\pgfqpoint{3.792484in}{1.387714in}}%
\pgfpathlineto{\pgfqpoint{3.792484in}{1.384765in}}%
\pgfpathmoveto{\pgfqpoint{3.787943in}{1.387714in}}%
\pgfpathlineto{\pgfqpoint{3.787943in}{1.387714in}}%
\pgfpathlineto{\pgfqpoint{3.787943in}{1.390664in}}%
\pgfpathlineto{\pgfqpoint{3.792484in}{1.390664in}}%
\pgfpathlineto{\pgfqpoint{3.792484in}{1.387714in}}%
\pgfpathmoveto{\pgfqpoint{3.783402in}{1.390664in}}%
\pgfpathlineto{\pgfqpoint{3.783402in}{1.390664in}}%
\pgfpathlineto{\pgfqpoint{3.783402in}{1.393613in}}%
\pgfpathlineto{\pgfqpoint{3.787943in}{1.393613in}}%
\pgfpathlineto{\pgfqpoint{3.787943in}{1.390664in}}%
\pgfpathmoveto{\pgfqpoint{3.783402in}{1.393613in}}%
\pgfpathlineto{\pgfqpoint{3.783402in}{1.393613in}}%
\pgfpathlineto{\pgfqpoint{3.783402in}{1.396562in}}%
\pgfpathlineto{\pgfqpoint{3.787943in}{1.396562in}}%
\pgfpathlineto{\pgfqpoint{3.787943in}{1.393613in}}%
\pgfpathmoveto{\pgfqpoint{3.787943in}{1.390664in}}%
\pgfpathlineto{\pgfqpoint{3.787943in}{1.390664in}}%
\pgfpathlineto{\pgfqpoint{3.787943in}{1.393613in}}%
\pgfpathlineto{\pgfqpoint{3.792484in}{1.393613in}}%
\pgfpathlineto{\pgfqpoint{3.792484in}{1.390664in}}%
\pgfpathmoveto{\pgfqpoint{3.792484in}{1.384765in}}%
\pgfpathlineto{\pgfqpoint{3.792484in}{1.384765in}}%
\pgfpathlineto{\pgfqpoint{3.792484in}{1.387714in}}%
\pgfpathlineto{\pgfqpoint{3.797025in}{1.387714in}}%
\pgfpathlineto{\pgfqpoint{3.797025in}{1.384765in}}%
\pgfpathmoveto{\pgfqpoint{3.792484in}{1.387714in}}%
\pgfpathlineto{\pgfqpoint{3.792484in}{1.387714in}}%
\pgfpathlineto{\pgfqpoint{3.792484in}{1.390664in}}%
\pgfpathlineto{\pgfqpoint{3.797025in}{1.390664in}}%
\pgfpathlineto{\pgfqpoint{3.797025in}{1.387714in}}%
\pgfpathmoveto{\pgfqpoint{3.747073in}{1.414258in}}%
\pgfpathlineto{\pgfqpoint{3.747073in}{1.414258in}}%
\pgfpathlineto{\pgfqpoint{3.747073in}{1.417208in}}%
\pgfpathlineto{\pgfqpoint{3.751614in}{1.417208in}}%
\pgfpathlineto{\pgfqpoint{3.751614in}{1.414258in}}%
\pgfpathmoveto{\pgfqpoint{3.747073in}{1.417208in}}%
\pgfpathlineto{\pgfqpoint{3.747073in}{1.417208in}}%
\pgfpathlineto{\pgfqpoint{3.747073in}{1.420157in}}%
\pgfpathlineto{\pgfqpoint{3.751614in}{1.420157in}}%
\pgfpathlineto{\pgfqpoint{3.751614in}{1.417208in}}%
\pgfpathmoveto{\pgfqpoint{3.751614in}{1.414258in}}%
\pgfpathlineto{\pgfqpoint{3.751614in}{1.414258in}}%
\pgfpathlineto{\pgfqpoint{3.751614in}{1.417208in}}%
\pgfpathlineto{\pgfqpoint{3.756155in}{1.417208in}}%
\pgfpathlineto{\pgfqpoint{3.756155in}{1.414258in}}%
\pgfpathmoveto{\pgfqpoint{3.751614in}{1.417208in}}%
\pgfpathlineto{\pgfqpoint{3.751614in}{1.417208in}}%
\pgfpathlineto{\pgfqpoint{3.751614in}{1.420157in}}%
\pgfpathlineto{\pgfqpoint{3.756155in}{1.420157in}}%
\pgfpathlineto{\pgfqpoint{3.756155in}{1.417208in}}%
\pgfpathmoveto{\pgfqpoint{3.756155in}{1.408360in}}%
\pgfpathlineto{\pgfqpoint{3.756155in}{1.408360in}}%
\pgfpathlineto{\pgfqpoint{3.756155in}{1.411309in}}%
\pgfpathlineto{\pgfqpoint{3.760696in}{1.411309in}}%
\pgfpathlineto{\pgfqpoint{3.760696in}{1.408360in}}%
\pgfpathmoveto{\pgfqpoint{3.756155in}{1.411309in}}%
\pgfpathlineto{\pgfqpoint{3.756155in}{1.411309in}}%
\pgfpathlineto{\pgfqpoint{3.756155in}{1.414258in}}%
\pgfpathlineto{\pgfqpoint{3.760696in}{1.414258in}}%
\pgfpathlineto{\pgfqpoint{3.760696in}{1.411309in}}%
\pgfpathmoveto{\pgfqpoint{3.760696in}{1.408360in}}%
\pgfpathlineto{\pgfqpoint{3.760696in}{1.408360in}}%
\pgfpathlineto{\pgfqpoint{3.760696in}{1.411309in}}%
\pgfpathlineto{\pgfqpoint{3.765237in}{1.411309in}}%
\pgfpathlineto{\pgfqpoint{3.765237in}{1.408360in}}%
\pgfpathmoveto{\pgfqpoint{3.760696in}{1.411309in}}%
\pgfpathlineto{\pgfqpoint{3.760696in}{1.411309in}}%
\pgfpathlineto{\pgfqpoint{3.760696in}{1.414258in}}%
\pgfpathlineto{\pgfqpoint{3.765237in}{1.414258in}}%
\pgfpathlineto{\pgfqpoint{3.765237in}{1.411309in}}%
\pgfpathmoveto{\pgfqpoint{3.756155in}{1.414258in}}%
\pgfpathlineto{\pgfqpoint{3.756155in}{1.414258in}}%
\pgfpathlineto{\pgfqpoint{3.756155in}{1.417208in}}%
\pgfpathlineto{\pgfqpoint{3.760696in}{1.417208in}}%
\pgfpathlineto{\pgfqpoint{3.760696in}{1.414258in}}%
\pgfpathmoveto{\pgfqpoint{3.756155in}{1.417208in}}%
\pgfpathlineto{\pgfqpoint{3.756155in}{1.417208in}}%
\pgfpathlineto{\pgfqpoint{3.756155in}{1.420157in}}%
\pgfpathlineto{\pgfqpoint{3.760696in}{1.420157in}}%
\pgfpathlineto{\pgfqpoint{3.760696in}{1.417208in}}%
\pgfpathmoveto{\pgfqpoint{3.760696in}{1.414258in}}%
\pgfpathlineto{\pgfqpoint{3.760696in}{1.414258in}}%
\pgfpathlineto{\pgfqpoint{3.760696in}{1.417208in}}%
\pgfpathlineto{\pgfqpoint{3.765237in}{1.417208in}}%
\pgfpathlineto{\pgfqpoint{3.765237in}{1.414258in}}%
\pgfpathmoveto{\pgfqpoint{3.737990in}{1.426056in}}%
\pgfpathlineto{\pgfqpoint{3.737990in}{1.426056in}}%
\pgfpathlineto{\pgfqpoint{3.737990in}{1.429005in}}%
\pgfpathlineto{\pgfqpoint{3.742532in}{1.429005in}}%
\pgfpathlineto{\pgfqpoint{3.742532in}{1.426056in}}%
\pgfpathmoveto{\pgfqpoint{3.737990in}{1.429005in}}%
\pgfpathlineto{\pgfqpoint{3.737990in}{1.429005in}}%
\pgfpathlineto{\pgfqpoint{3.737990in}{1.431955in}}%
\pgfpathlineto{\pgfqpoint{3.742532in}{1.431955in}}%
\pgfpathlineto{\pgfqpoint{3.742532in}{1.429005in}}%
\pgfpathmoveto{\pgfqpoint{3.742532in}{1.426056in}}%
\pgfpathlineto{\pgfqpoint{3.742532in}{1.426056in}}%
\pgfpathlineto{\pgfqpoint{3.742532in}{1.429005in}}%
\pgfpathlineto{\pgfqpoint{3.747073in}{1.429005in}}%
\pgfpathlineto{\pgfqpoint{3.747073in}{1.426056in}}%
\pgfpathmoveto{\pgfqpoint{3.742532in}{1.429005in}}%
\pgfpathlineto{\pgfqpoint{3.742532in}{1.429005in}}%
\pgfpathlineto{\pgfqpoint{3.742532in}{1.431955in}}%
\pgfpathlineto{\pgfqpoint{3.747073in}{1.431955in}}%
\pgfpathlineto{\pgfqpoint{3.747073in}{1.429005in}}%
\pgfpathmoveto{\pgfqpoint{3.728908in}{1.431955in}}%
\pgfpathlineto{\pgfqpoint{3.728908in}{1.431955in}}%
\pgfpathlineto{\pgfqpoint{3.728908in}{1.434904in}}%
\pgfpathlineto{\pgfqpoint{3.733449in}{1.434904in}}%
\pgfpathlineto{\pgfqpoint{3.733449in}{1.431955in}}%
\pgfpathmoveto{\pgfqpoint{3.728908in}{1.434904in}}%
\pgfpathlineto{\pgfqpoint{3.728908in}{1.434904in}}%
\pgfpathlineto{\pgfqpoint{3.728908in}{1.437853in}}%
\pgfpathlineto{\pgfqpoint{3.733449in}{1.437853in}}%
\pgfpathlineto{\pgfqpoint{3.733449in}{1.434904in}}%
\pgfpathmoveto{\pgfqpoint{3.733449in}{1.431955in}}%
\pgfpathlineto{\pgfqpoint{3.733449in}{1.431955in}}%
\pgfpathlineto{\pgfqpoint{3.733449in}{1.434904in}}%
\pgfpathlineto{\pgfqpoint{3.737990in}{1.434904in}}%
\pgfpathlineto{\pgfqpoint{3.737990in}{1.431955in}}%
\pgfpathmoveto{\pgfqpoint{3.733449in}{1.434904in}}%
\pgfpathlineto{\pgfqpoint{3.733449in}{1.434904in}}%
\pgfpathlineto{\pgfqpoint{3.733449in}{1.437853in}}%
\pgfpathlineto{\pgfqpoint{3.737990in}{1.437853in}}%
\pgfpathlineto{\pgfqpoint{3.737990in}{1.434904in}}%
\pgfpathmoveto{\pgfqpoint{3.728908in}{1.437853in}}%
\pgfpathlineto{\pgfqpoint{3.728908in}{1.437853in}}%
\pgfpathlineto{\pgfqpoint{3.728908in}{1.440803in}}%
\pgfpathlineto{\pgfqpoint{3.733449in}{1.440803in}}%
\pgfpathlineto{\pgfqpoint{3.733449in}{1.437853in}}%
\pgfpathmoveto{\pgfqpoint{3.728908in}{1.440803in}}%
\pgfpathlineto{\pgfqpoint{3.728908in}{1.440803in}}%
\pgfpathlineto{\pgfqpoint{3.728908in}{1.443752in}}%
\pgfpathlineto{\pgfqpoint{3.733449in}{1.443752in}}%
\pgfpathlineto{\pgfqpoint{3.733449in}{1.440803in}}%
\pgfpathmoveto{\pgfqpoint{3.733449in}{1.437853in}}%
\pgfpathlineto{\pgfqpoint{3.733449in}{1.437853in}}%
\pgfpathlineto{\pgfqpoint{3.733449in}{1.440803in}}%
\pgfpathlineto{\pgfqpoint{3.737990in}{1.440803in}}%
\pgfpathlineto{\pgfqpoint{3.737990in}{1.437853in}}%
\pgfpathmoveto{\pgfqpoint{3.737990in}{1.431955in}}%
\pgfpathlineto{\pgfqpoint{3.737990in}{1.431955in}}%
\pgfpathlineto{\pgfqpoint{3.737990in}{1.434904in}}%
\pgfpathlineto{\pgfqpoint{3.742532in}{1.434904in}}%
\pgfpathlineto{\pgfqpoint{3.742532in}{1.431955in}}%
\pgfpathmoveto{\pgfqpoint{3.737990in}{1.434904in}}%
\pgfpathlineto{\pgfqpoint{3.737990in}{1.434904in}}%
\pgfpathlineto{\pgfqpoint{3.737990in}{1.437853in}}%
\pgfpathlineto{\pgfqpoint{3.742532in}{1.437853in}}%
\pgfpathlineto{\pgfqpoint{3.742532in}{1.434904in}}%
\pgfpathmoveto{\pgfqpoint{3.747073in}{1.420157in}}%
\pgfpathlineto{\pgfqpoint{3.747073in}{1.420157in}}%
\pgfpathlineto{\pgfqpoint{3.747073in}{1.423106in}}%
\pgfpathlineto{\pgfqpoint{3.751614in}{1.423106in}}%
\pgfpathlineto{\pgfqpoint{3.751614in}{1.420157in}}%
\pgfpathmoveto{\pgfqpoint{3.747073in}{1.423106in}}%
\pgfpathlineto{\pgfqpoint{3.747073in}{1.423106in}}%
\pgfpathlineto{\pgfqpoint{3.747073in}{1.426056in}}%
\pgfpathlineto{\pgfqpoint{3.751614in}{1.426056in}}%
\pgfpathlineto{\pgfqpoint{3.751614in}{1.423106in}}%
\pgfpathmoveto{\pgfqpoint{3.751614in}{1.420157in}}%
\pgfpathlineto{\pgfqpoint{3.751614in}{1.420157in}}%
\pgfpathlineto{\pgfqpoint{3.751614in}{1.423106in}}%
\pgfpathlineto{\pgfqpoint{3.756155in}{1.423106in}}%
\pgfpathlineto{\pgfqpoint{3.756155in}{1.420157in}}%
\pgfpathmoveto{\pgfqpoint{3.751614in}{1.423106in}}%
\pgfpathlineto{\pgfqpoint{3.751614in}{1.423106in}}%
\pgfpathlineto{\pgfqpoint{3.751614in}{1.426056in}}%
\pgfpathlineto{\pgfqpoint{3.756155in}{1.426056in}}%
\pgfpathlineto{\pgfqpoint{3.756155in}{1.423106in}}%
\pgfpathmoveto{\pgfqpoint{3.747073in}{1.426056in}}%
\pgfpathlineto{\pgfqpoint{3.747073in}{1.426056in}}%
\pgfpathlineto{\pgfqpoint{3.747073in}{1.429005in}}%
\pgfpathlineto{\pgfqpoint{3.751614in}{1.429005in}}%
\pgfpathlineto{\pgfqpoint{3.751614in}{1.426056in}}%
\pgfpathmoveto{\pgfqpoint{3.765237in}{1.402461in}}%
\pgfpathlineto{\pgfqpoint{3.765237in}{1.402461in}}%
\pgfpathlineto{\pgfqpoint{3.765237in}{1.405410in}}%
\pgfpathlineto{\pgfqpoint{3.769778in}{1.405410in}}%
\pgfpathlineto{\pgfqpoint{3.769778in}{1.402461in}}%
\pgfpathmoveto{\pgfqpoint{3.765237in}{1.405410in}}%
\pgfpathlineto{\pgfqpoint{3.765237in}{1.405410in}}%
\pgfpathlineto{\pgfqpoint{3.765237in}{1.408360in}}%
\pgfpathlineto{\pgfqpoint{3.769778in}{1.408360in}}%
\pgfpathlineto{\pgfqpoint{3.769778in}{1.405410in}}%
\pgfpathmoveto{\pgfqpoint{3.769778in}{1.402461in}}%
\pgfpathlineto{\pgfqpoint{3.769778in}{1.402461in}}%
\pgfpathlineto{\pgfqpoint{3.769778in}{1.405410in}}%
\pgfpathlineto{\pgfqpoint{3.774319in}{1.405410in}}%
\pgfpathlineto{\pgfqpoint{3.774319in}{1.402461in}}%
\pgfpathmoveto{\pgfqpoint{3.769778in}{1.405410in}}%
\pgfpathlineto{\pgfqpoint{3.769778in}{1.405410in}}%
\pgfpathlineto{\pgfqpoint{3.769778in}{1.408360in}}%
\pgfpathlineto{\pgfqpoint{3.774319in}{1.408360in}}%
\pgfpathlineto{\pgfqpoint{3.774319in}{1.405410in}}%
\pgfpathmoveto{\pgfqpoint{3.774319in}{1.396562in}}%
\pgfpathlineto{\pgfqpoint{3.774319in}{1.396562in}}%
\pgfpathlineto{\pgfqpoint{3.774319in}{1.399512in}}%
\pgfpathlineto{\pgfqpoint{3.778861in}{1.399512in}}%
\pgfpathlineto{\pgfqpoint{3.778861in}{1.396562in}}%
\pgfpathmoveto{\pgfqpoint{3.774319in}{1.399512in}}%
\pgfpathlineto{\pgfqpoint{3.774319in}{1.399512in}}%
\pgfpathlineto{\pgfqpoint{3.774319in}{1.402461in}}%
\pgfpathlineto{\pgfqpoint{3.778861in}{1.402461in}}%
\pgfpathlineto{\pgfqpoint{3.778861in}{1.399512in}}%
\pgfpathmoveto{\pgfqpoint{3.778861in}{1.396562in}}%
\pgfpathlineto{\pgfqpoint{3.778861in}{1.396562in}}%
\pgfpathlineto{\pgfqpoint{3.778861in}{1.399512in}}%
\pgfpathlineto{\pgfqpoint{3.783402in}{1.399512in}}%
\pgfpathlineto{\pgfqpoint{3.783402in}{1.396562in}}%
\pgfpathmoveto{\pgfqpoint{3.778861in}{1.399512in}}%
\pgfpathlineto{\pgfqpoint{3.778861in}{1.399512in}}%
\pgfpathlineto{\pgfqpoint{3.778861in}{1.402461in}}%
\pgfpathlineto{\pgfqpoint{3.783402in}{1.402461in}}%
\pgfpathlineto{\pgfqpoint{3.783402in}{1.399512in}}%
\pgfpathmoveto{\pgfqpoint{3.774319in}{1.402461in}}%
\pgfpathlineto{\pgfqpoint{3.774319in}{1.402461in}}%
\pgfpathlineto{\pgfqpoint{3.774319in}{1.405410in}}%
\pgfpathlineto{\pgfqpoint{3.778861in}{1.405410in}}%
\pgfpathlineto{\pgfqpoint{3.778861in}{1.402461in}}%
\pgfpathmoveto{\pgfqpoint{3.765237in}{1.408360in}}%
\pgfpathlineto{\pgfqpoint{3.765237in}{1.408360in}}%
\pgfpathlineto{\pgfqpoint{3.765237in}{1.411309in}}%
\pgfpathlineto{\pgfqpoint{3.769778in}{1.411309in}}%
\pgfpathlineto{\pgfqpoint{3.769778in}{1.408360in}}%
\pgfpathmoveto{\pgfqpoint{3.765237in}{1.411309in}}%
\pgfpathlineto{\pgfqpoint{3.765237in}{1.411309in}}%
\pgfpathlineto{\pgfqpoint{3.765237in}{1.414258in}}%
\pgfpathlineto{\pgfqpoint{3.769778in}{1.414258in}}%
\pgfpathlineto{\pgfqpoint{3.769778in}{1.411309in}}%
\pgfpathmoveto{\pgfqpoint{3.665333in}{1.485041in}}%
\pgfpathlineto{\pgfqpoint{3.665333in}{1.485041in}}%
\pgfpathlineto{\pgfqpoint{3.665333in}{1.487991in}}%
\pgfpathlineto{\pgfqpoint{3.669874in}{1.487991in}}%
\pgfpathlineto{\pgfqpoint{3.669874in}{1.485041in}}%
\pgfpathmoveto{\pgfqpoint{3.665333in}{1.487991in}}%
\pgfpathlineto{\pgfqpoint{3.665333in}{1.487991in}}%
\pgfpathlineto{\pgfqpoint{3.665333in}{1.490940in}}%
\pgfpathlineto{\pgfqpoint{3.669874in}{1.490940in}}%
\pgfpathlineto{\pgfqpoint{3.669874in}{1.487991in}}%
\pgfpathmoveto{\pgfqpoint{3.669874in}{1.485041in}}%
\pgfpathlineto{\pgfqpoint{3.669874in}{1.485041in}}%
\pgfpathlineto{\pgfqpoint{3.669874in}{1.487991in}}%
\pgfpathlineto{\pgfqpoint{3.674415in}{1.487991in}}%
\pgfpathlineto{\pgfqpoint{3.674415in}{1.485041in}}%
\pgfpathmoveto{\pgfqpoint{3.669874in}{1.487991in}}%
\pgfpathlineto{\pgfqpoint{3.669874in}{1.487991in}}%
\pgfpathlineto{\pgfqpoint{3.669874in}{1.490940in}}%
\pgfpathlineto{\pgfqpoint{3.674415in}{1.490940in}}%
\pgfpathlineto{\pgfqpoint{3.674415in}{1.487991in}}%
\pgfpathmoveto{\pgfqpoint{3.683497in}{1.473244in}}%
\pgfpathlineto{\pgfqpoint{3.683497in}{1.473244in}}%
\pgfpathlineto{\pgfqpoint{3.683497in}{1.476194in}}%
\pgfpathlineto{\pgfqpoint{3.688038in}{1.476194in}}%
\pgfpathlineto{\pgfqpoint{3.688038in}{1.473244in}}%
\pgfpathmoveto{\pgfqpoint{3.683497in}{1.476194in}}%
\pgfpathlineto{\pgfqpoint{3.683497in}{1.476194in}}%
\pgfpathlineto{\pgfqpoint{3.683497in}{1.479143in}}%
\pgfpathlineto{\pgfqpoint{3.688038in}{1.479143in}}%
\pgfpathlineto{\pgfqpoint{3.688038in}{1.476194in}}%
\pgfpathmoveto{\pgfqpoint{3.688038in}{1.473244in}}%
\pgfpathlineto{\pgfqpoint{3.688038in}{1.473244in}}%
\pgfpathlineto{\pgfqpoint{3.688038in}{1.476194in}}%
\pgfpathlineto{\pgfqpoint{3.692579in}{1.476194in}}%
\pgfpathlineto{\pgfqpoint{3.692579in}{1.473244in}}%
\pgfpathmoveto{\pgfqpoint{3.688038in}{1.476194in}}%
\pgfpathlineto{\pgfqpoint{3.688038in}{1.476194in}}%
\pgfpathlineto{\pgfqpoint{3.688038in}{1.479143in}}%
\pgfpathlineto{\pgfqpoint{3.692579in}{1.479143in}}%
\pgfpathlineto{\pgfqpoint{3.692579in}{1.476194in}}%
\pgfpathmoveto{\pgfqpoint{3.674415in}{1.479143in}}%
\pgfpathlineto{\pgfqpoint{3.674415in}{1.479143in}}%
\pgfpathlineto{\pgfqpoint{3.674415in}{1.482092in}}%
\pgfpathlineto{\pgfqpoint{3.678956in}{1.482092in}}%
\pgfpathlineto{\pgfqpoint{3.678956in}{1.479143in}}%
\pgfpathmoveto{\pgfqpoint{3.674415in}{1.482092in}}%
\pgfpathlineto{\pgfqpoint{3.674415in}{1.482092in}}%
\pgfpathlineto{\pgfqpoint{3.674415in}{1.485041in}}%
\pgfpathlineto{\pgfqpoint{3.678956in}{1.485041in}}%
\pgfpathlineto{\pgfqpoint{3.678956in}{1.482092in}}%
\pgfpathmoveto{\pgfqpoint{3.678956in}{1.479143in}}%
\pgfpathlineto{\pgfqpoint{3.678956in}{1.479143in}}%
\pgfpathlineto{\pgfqpoint{3.678956in}{1.482092in}}%
\pgfpathlineto{\pgfqpoint{3.683497in}{1.482092in}}%
\pgfpathlineto{\pgfqpoint{3.683497in}{1.479143in}}%
\pgfpathmoveto{\pgfqpoint{3.678956in}{1.482092in}}%
\pgfpathlineto{\pgfqpoint{3.678956in}{1.482092in}}%
\pgfpathlineto{\pgfqpoint{3.678956in}{1.485041in}}%
\pgfpathlineto{\pgfqpoint{3.683497in}{1.485041in}}%
\pgfpathlineto{\pgfqpoint{3.683497in}{1.482092in}}%
\pgfpathmoveto{\pgfqpoint{3.674415in}{1.485041in}}%
\pgfpathlineto{\pgfqpoint{3.674415in}{1.485041in}}%
\pgfpathlineto{\pgfqpoint{3.674415in}{1.487991in}}%
\pgfpathlineto{\pgfqpoint{3.678956in}{1.487991in}}%
\pgfpathlineto{\pgfqpoint{3.678956in}{1.485041in}}%
\pgfpathmoveto{\pgfqpoint{3.674415in}{1.487991in}}%
\pgfpathlineto{\pgfqpoint{3.674415in}{1.487991in}}%
\pgfpathlineto{\pgfqpoint{3.674415in}{1.490940in}}%
\pgfpathlineto{\pgfqpoint{3.678956in}{1.490940in}}%
\pgfpathlineto{\pgfqpoint{3.678956in}{1.487991in}}%
\pgfpathmoveto{\pgfqpoint{3.678956in}{1.485041in}}%
\pgfpathlineto{\pgfqpoint{3.678956in}{1.485041in}}%
\pgfpathlineto{\pgfqpoint{3.678956in}{1.487991in}}%
\pgfpathlineto{\pgfqpoint{3.683497in}{1.487991in}}%
\pgfpathlineto{\pgfqpoint{3.683497in}{1.485041in}}%
\pgfpathmoveto{\pgfqpoint{3.683497in}{1.479143in}}%
\pgfpathlineto{\pgfqpoint{3.683497in}{1.479143in}}%
\pgfpathlineto{\pgfqpoint{3.683497in}{1.482092in}}%
\pgfpathlineto{\pgfqpoint{3.688038in}{1.482092in}}%
\pgfpathlineto{\pgfqpoint{3.688038in}{1.479143in}}%
\pgfpathmoveto{\pgfqpoint{3.683497in}{1.482092in}}%
\pgfpathlineto{\pgfqpoint{3.683497in}{1.482092in}}%
\pgfpathlineto{\pgfqpoint{3.683497in}{1.485041in}}%
\pgfpathlineto{\pgfqpoint{3.688038in}{1.485041in}}%
\pgfpathlineto{\pgfqpoint{3.688038in}{1.482092in}}%
\pgfpathmoveto{\pgfqpoint{3.692579in}{1.461447in}}%
\pgfpathlineto{\pgfqpoint{3.692579in}{1.461447in}}%
\pgfpathlineto{\pgfqpoint{3.692579in}{1.464397in}}%
\pgfpathlineto{\pgfqpoint{3.697120in}{1.464397in}}%
\pgfpathlineto{\pgfqpoint{3.697120in}{1.461447in}}%
\pgfpathmoveto{\pgfqpoint{3.692579in}{1.464397in}}%
\pgfpathlineto{\pgfqpoint{3.692579in}{1.464397in}}%
\pgfpathlineto{\pgfqpoint{3.692579in}{1.467346in}}%
\pgfpathlineto{\pgfqpoint{3.697120in}{1.467346in}}%
\pgfpathlineto{\pgfqpoint{3.697120in}{1.464397in}}%
\pgfpathmoveto{\pgfqpoint{3.697120in}{1.461447in}}%
\pgfpathlineto{\pgfqpoint{3.697120in}{1.461447in}}%
\pgfpathlineto{\pgfqpoint{3.697120in}{1.464397in}}%
\pgfpathlineto{\pgfqpoint{3.701662in}{1.464397in}}%
\pgfpathlineto{\pgfqpoint{3.701662in}{1.461447in}}%
\pgfpathmoveto{\pgfqpoint{3.697120in}{1.464397in}}%
\pgfpathlineto{\pgfqpoint{3.697120in}{1.464397in}}%
\pgfpathlineto{\pgfqpoint{3.697120in}{1.467346in}}%
\pgfpathlineto{\pgfqpoint{3.701662in}{1.467346in}}%
\pgfpathlineto{\pgfqpoint{3.701662in}{1.464397in}}%
\pgfpathmoveto{\pgfqpoint{3.701662in}{1.455549in}}%
\pgfpathlineto{\pgfqpoint{3.701662in}{1.455549in}}%
\pgfpathlineto{\pgfqpoint{3.701662in}{1.458498in}}%
\pgfpathlineto{\pgfqpoint{3.706203in}{1.458498in}}%
\pgfpathlineto{\pgfqpoint{3.706203in}{1.455549in}}%
\pgfpathmoveto{\pgfqpoint{3.701662in}{1.458498in}}%
\pgfpathlineto{\pgfqpoint{3.701662in}{1.458498in}}%
\pgfpathlineto{\pgfqpoint{3.701662in}{1.461447in}}%
\pgfpathlineto{\pgfqpoint{3.706203in}{1.461447in}}%
\pgfpathlineto{\pgfqpoint{3.706203in}{1.458498in}}%
\pgfpathmoveto{\pgfqpoint{3.706203in}{1.455549in}}%
\pgfpathlineto{\pgfqpoint{3.706203in}{1.455549in}}%
\pgfpathlineto{\pgfqpoint{3.706203in}{1.458498in}}%
\pgfpathlineto{\pgfqpoint{3.710744in}{1.458498in}}%
\pgfpathlineto{\pgfqpoint{3.710744in}{1.455549in}}%
\pgfpathmoveto{\pgfqpoint{3.706203in}{1.458498in}}%
\pgfpathlineto{\pgfqpoint{3.706203in}{1.458498in}}%
\pgfpathlineto{\pgfqpoint{3.706203in}{1.461447in}}%
\pgfpathlineto{\pgfqpoint{3.710744in}{1.461447in}}%
\pgfpathlineto{\pgfqpoint{3.710744in}{1.458498in}}%
\pgfpathmoveto{\pgfqpoint{3.701662in}{1.461447in}}%
\pgfpathlineto{\pgfqpoint{3.701662in}{1.461447in}}%
\pgfpathlineto{\pgfqpoint{3.701662in}{1.464397in}}%
\pgfpathlineto{\pgfqpoint{3.706203in}{1.464397in}}%
\pgfpathlineto{\pgfqpoint{3.706203in}{1.461447in}}%
\pgfpathmoveto{\pgfqpoint{3.701662in}{1.464397in}}%
\pgfpathlineto{\pgfqpoint{3.701662in}{1.464397in}}%
\pgfpathlineto{\pgfqpoint{3.701662in}{1.467346in}}%
\pgfpathlineto{\pgfqpoint{3.706203in}{1.467346in}}%
\pgfpathlineto{\pgfqpoint{3.706203in}{1.464397in}}%
\pgfpathmoveto{\pgfqpoint{3.706203in}{1.461447in}}%
\pgfpathlineto{\pgfqpoint{3.706203in}{1.461447in}}%
\pgfpathlineto{\pgfqpoint{3.706203in}{1.464397in}}%
\pgfpathlineto{\pgfqpoint{3.710744in}{1.464397in}}%
\pgfpathlineto{\pgfqpoint{3.710744in}{1.461447in}}%
\pgfpathmoveto{\pgfqpoint{3.710744in}{1.449650in}}%
\pgfpathlineto{\pgfqpoint{3.710744in}{1.449650in}}%
\pgfpathlineto{\pgfqpoint{3.710744in}{1.452600in}}%
\pgfpathlineto{\pgfqpoint{3.715285in}{1.452600in}}%
\pgfpathlineto{\pgfqpoint{3.715285in}{1.449650in}}%
\pgfpathmoveto{\pgfqpoint{3.710744in}{1.452600in}}%
\pgfpathlineto{\pgfqpoint{3.710744in}{1.452600in}}%
\pgfpathlineto{\pgfqpoint{3.710744in}{1.455549in}}%
\pgfpathlineto{\pgfqpoint{3.715285in}{1.455549in}}%
\pgfpathlineto{\pgfqpoint{3.715285in}{1.452600in}}%
\pgfpathmoveto{\pgfqpoint{3.715285in}{1.449650in}}%
\pgfpathlineto{\pgfqpoint{3.715285in}{1.449650in}}%
\pgfpathlineto{\pgfqpoint{3.715285in}{1.452600in}}%
\pgfpathlineto{\pgfqpoint{3.719826in}{1.452600in}}%
\pgfpathlineto{\pgfqpoint{3.719826in}{1.449650in}}%
\pgfpathmoveto{\pgfqpoint{3.715285in}{1.452600in}}%
\pgfpathlineto{\pgfqpoint{3.715285in}{1.452600in}}%
\pgfpathlineto{\pgfqpoint{3.715285in}{1.455549in}}%
\pgfpathlineto{\pgfqpoint{3.719826in}{1.455549in}}%
\pgfpathlineto{\pgfqpoint{3.719826in}{1.452600in}}%
\pgfpathmoveto{\pgfqpoint{3.719826in}{1.443752in}}%
\pgfpathlineto{\pgfqpoint{3.719826in}{1.443752in}}%
\pgfpathlineto{\pgfqpoint{3.719826in}{1.446701in}}%
\pgfpathlineto{\pgfqpoint{3.724367in}{1.446701in}}%
\pgfpathlineto{\pgfqpoint{3.724367in}{1.443752in}}%
\pgfpathmoveto{\pgfqpoint{3.719826in}{1.446701in}}%
\pgfpathlineto{\pgfqpoint{3.719826in}{1.446701in}}%
\pgfpathlineto{\pgfqpoint{3.719826in}{1.449650in}}%
\pgfpathlineto{\pgfqpoint{3.724367in}{1.449650in}}%
\pgfpathlineto{\pgfqpoint{3.724367in}{1.446701in}}%
\pgfpathmoveto{\pgfqpoint{3.724367in}{1.443752in}}%
\pgfpathlineto{\pgfqpoint{3.724367in}{1.443752in}}%
\pgfpathlineto{\pgfqpoint{3.724367in}{1.446701in}}%
\pgfpathlineto{\pgfqpoint{3.728908in}{1.446701in}}%
\pgfpathlineto{\pgfqpoint{3.728908in}{1.443752in}}%
\pgfpathmoveto{\pgfqpoint{3.724367in}{1.446701in}}%
\pgfpathlineto{\pgfqpoint{3.724367in}{1.446701in}}%
\pgfpathlineto{\pgfqpoint{3.724367in}{1.449650in}}%
\pgfpathlineto{\pgfqpoint{3.728908in}{1.449650in}}%
\pgfpathlineto{\pgfqpoint{3.728908in}{1.446701in}}%
\pgfpathmoveto{\pgfqpoint{3.719826in}{1.449650in}}%
\pgfpathlineto{\pgfqpoint{3.719826in}{1.449650in}}%
\pgfpathlineto{\pgfqpoint{3.719826in}{1.452600in}}%
\pgfpathlineto{\pgfqpoint{3.724367in}{1.452600in}}%
\pgfpathlineto{\pgfqpoint{3.724367in}{1.449650in}}%
\pgfpathmoveto{\pgfqpoint{3.710744in}{1.455549in}}%
\pgfpathlineto{\pgfqpoint{3.710744in}{1.455549in}}%
\pgfpathlineto{\pgfqpoint{3.710744in}{1.458498in}}%
\pgfpathlineto{\pgfqpoint{3.715285in}{1.458498in}}%
\pgfpathlineto{\pgfqpoint{3.715285in}{1.455549in}}%
\pgfpathmoveto{\pgfqpoint{3.710744in}{1.458498in}}%
\pgfpathlineto{\pgfqpoint{3.710744in}{1.458498in}}%
\pgfpathlineto{\pgfqpoint{3.710744in}{1.461447in}}%
\pgfpathlineto{\pgfqpoint{3.715285in}{1.461447in}}%
\pgfpathlineto{\pgfqpoint{3.715285in}{1.458498in}}%
\pgfpathmoveto{\pgfqpoint{3.692579in}{1.467346in}}%
\pgfpathlineto{\pgfqpoint{3.692579in}{1.467346in}}%
\pgfpathlineto{\pgfqpoint{3.692579in}{1.470295in}}%
\pgfpathlineto{\pgfqpoint{3.697120in}{1.470295in}}%
\pgfpathlineto{\pgfqpoint{3.697120in}{1.467346in}}%
\pgfpathmoveto{\pgfqpoint{3.692579in}{1.470295in}}%
\pgfpathlineto{\pgfqpoint{3.692579in}{1.470295in}}%
\pgfpathlineto{\pgfqpoint{3.692579in}{1.473244in}}%
\pgfpathlineto{\pgfqpoint{3.697120in}{1.473244in}}%
\pgfpathlineto{\pgfqpoint{3.697120in}{1.470295in}}%
\pgfpathmoveto{\pgfqpoint{3.697120in}{1.467346in}}%
\pgfpathlineto{\pgfqpoint{3.697120in}{1.467346in}}%
\pgfpathlineto{\pgfqpoint{3.697120in}{1.470295in}}%
\pgfpathlineto{\pgfqpoint{3.701662in}{1.470295in}}%
\pgfpathlineto{\pgfqpoint{3.701662in}{1.467346in}}%
\pgfpathmoveto{\pgfqpoint{3.697120in}{1.470295in}}%
\pgfpathlineto{\pgfqpoint{3.697120in}{1.470295in}}%
\pgfpathlineto{\pgfqpoint{3.697120in}{1.473244in}}%
\pgfpathlineto{\pgfqpoint{3.701662in}{1.473244in}}%
\pgfpathlineto{\pgfqpoint{3.701662in}{1.470295in}}%
\pgfpathmoveto{\pgfqpoint{3.692579in}{1.473244in}}%
\pgfpathlineto{\pgfqpoint{3.692579in}{1.473244in}}%
\pgfpathlineto{\pgfqpoint{3.692579in}{1.476194in}}%
\pgfpathlineto{\pgfqpoint{3.697120in}{1.476194in}}%
\pgfpathlineto{\pgfqpoint{3.697120in}{1.473244in}}%
\pgfpathmoveto{\pgfqpoint{3.656250in}{1.496838in}}%
\pgfpathlineto{\pgfqpoint{3.656250in}{1.496838in}}%
\pgfpathlineto{\pgfqpoint{3.656250in}{1.499787in}}%
\pgfpathlineto{\pgfqpoint{3.660791in}{1.499787in}}%
\pgfpathlineto{\pgfqpoint{3.660791in}{1.496838in}}%
\pgfpathmoveto{\pgfqpoint{3.656250in}{1.499787in}}%
\pgfpathlineto{\pgfqpoint{3.656250in}{1.499787in}}%
\pgfpathlineto{\pgfqpoint{3.656250in}{1.502737in}}%
\pgfpathlineto{\pgfqpoint{3.660791in}{1.502737in}}%
\pgfpathlineto{\pgfqpoint{3.660791in}{1.499787in}}%
\pgfpathmoveto{\pgfqpoint{3.660791in}{1.496838in}}%
\pgfpathlineto{\pgfqpoint{3.660791in}{1.496838in}}%
\pgfpathlineto{\pgfqpoint{3.660791in}{1.499787in}}%
\pgfpathlineto{\pgfqpoint{3.665333in}{1.499787in}}%
\pgfpathlineto{\pgfqpoint{3.665333in}{1.496838in}}%
\pgfpathmoveto{\pgfqpoint{3.660791in}{1.499787in}}%
\pgfpathlineto{\pgfqpoint{3.660791in}{1.499787in}}%
\pgfpathlineto{\pgfqpoint{3.660791in}{1.502737in}}%
\pgfpathlineto{\pgfqpoint{3.665333in}{1.502737in}}%
\pgfpathlineto{\pgfqpoint{3.665333in}{1.499787in}}%
\pgfpathmoveto{\pgfqpoint{3.665333in}{1.490940in}}%
\pgfpathlineto{\pgfqpoint{3.665333in}{1.490940in}}%
\pgfpathlineto{\pgfqpoint{3.665333in}{1.493889in}}%
\pgfpathlineto{\pgfqpoint{3.669874in}{1.493889in}}%
\pgfpathlineto{\pgfqpoint{3.669874in}{1.490940in}}%
\pgfpathmoveto{\pgfqpoint{3.665333in}{1.493889in}}%
\pgfpathlineto{\pgfqpoint{3.665333in}{1.493889in}}%
\pgfpathlineto{\pgfqpoint{3.665333in}{1.496838in}}%
\pgfpathlineto{\pgfqpoint{3.669874in}{1.496838in}}%
\pgfpathlineto{\pgfqpoint{3.669874in}{1.493889in}}%
\pgfpathmoveto{\pgfqpoint{3.669874in}{1.490940in}}%
\pgfpathlineto{\pgfqpoint{3.669874in}{1.490940in}}%
\pgfpathlineto{\pgfqpoint{3.669874in}{1.493889in}}%
\pgfpathlineto{\pgfqpoint{3.674415in}{1.493889in}}%
\pgfpathlineto{\pgfqpoint{3.674415in}{1.490940in}}%
\pgfpathmoveto{\pgfqpoint{3.669874in}{1.493889in}}%
\pgfpathlineto{\pgfqpoint{3.669874in}{1.493889in}}%
\pgfpathlineto{\pgfqpoint{3.669874in}{1.496838in}}%
\pgfpathlineto{\pgfqpoint{3.674415in}{1.496838in}}%
\pgfpathlineto{\pgfqpoint{3.674415in}{1.493889in}}%
\pgfpathmoveto{\pgfqpoint{3.665333in}{1.496838in}}%
\pgfpathlineto{\pgfqpoint{3.665333in}{1.496838in}}%
\pgfpathlineto{\pgfqpoint{3.665333in}{1.499787in}}%
\pgfpathlineto{\pgfqpoint{3.669874in}{1.499787in}}%
\pgfpathlineto{\pgfqpoint{3.669874in}{1.496838in}}%
\pgfpathmoveto{\pgfqpoint{3.656250in}{1.502737in}}%
\pgfpathlineto{\pgfqpoint{3.656250in}{1.502737in}}%
\pgfpathlineto{\pgfqpoint{3.656250in}{1.505686in}}%
\pgfpathlineto{\pgfqpoint{3.660791in}{1.505686in}}%
\pgfpathlineto{\pgfqpoint{3.660791in}{1.502737in}}%
\pgfpathmoveto{\pgfqpoint{3.656250in}{1.505686in}}%
\pgfpathlineto{\pgfqpoint{3.656250in}{1.505686in}}%
\pgfpathlineto{\pgfqpoint{3.656250in}{1.508635in}}%
\pgfpathlineto{\pgfqpoint{3.660791in}{1.508635in}}%
\pgfpathlineto{\pgfqpoint{3.660791in}{1.505686in}}%
\pgfpathmoveto{\pgfqpoint{3.801566in}{0.762481in}}%
\pgfpathlineto{\pgfqpoint{3.801566in}{0.762481in}}%
\pgfpathlineto{\pgfqpoint{3.801566in}{0.765430in}}%
\pgfpathlineto{\pgfqpoint{3.806107in}{0.765430in}}%
\pgfpathlineto{\pgfqpoint{3.806107in}{0.762481in}}%
\pgfpathmoveto{\pgfqpoint{3.801566in}{0.765430in}}%
\pgfpathlineto{\pgfqpoint{3.801566in}{0.765430in}}%
\pgfpathlineto{\pgfqpoint{3.801566in}{0.768380in}}%
\pgfpathlineto{\pgfqpoint{3.806107in}{0.768380in}}%
\pgfpathlineto{\pgfqpoint{3.806107in}{0.765430in}}%
\pgfpathmoveto{\pgfqpoint{3.801566in}{0.768380in}}%
\pgfpathlineto{\pgfqpoint{3.801566in}{0.768380in}}%
\pgfpathlineto{\pgfqpoint{3.801566in}{0.771329in}}%
\pgfpathlineto{\pgfqpoint{3.806107in}{0.771329in}}%
\pgfpathlineto{\pgfqpoint{3.806107in}{0.768380in}}%
\pgfpathmoveto{\pgfqpoint{3.806107in}{0.765430in}}%
\pgfpathlineto{\pgfqpoint{3.806107in}{0.765430in}}%
\pgfpathlineto{\pgfqpoint{3.806107in}{0.768380in}}%
\pgfpathlineto{\pgfqpoint{3.810648in}{0.768380in}}%
\pgfpathlineto{\pgfqpoint{3.810648in}{0.765430in}}%
\pgfpathmoveto{\pgfqpoint{3.806107in}{0.768380in}}%
\pgfpathlineto{\pgfqpoint{3.806107in}{0.768380in}}%
\pgfpathlineto{\pgfqpoint{3.806107in}{0.771329in}}%
\pgfpathlineto{\pgfqpoint{3.810648in}{0.771329in}}%
\pgfpathlineto{\pgfqpoint{3.810648in}{0.768380in}}%
\pgfpathmoveto{\pgfqpoint{3.801566in}{0.771329in}}%
\pgfpathlineto{\pgfqpoint{3.801566in}{0.771329in}}%
\pgfpathlineto{\pgfqpoint{3.801566in}{0.774278in}}%
\pgfpathlineto{\pgfqpoint{3.806107in}{0.774278in}}%
\pgfpathlineto{\pgfqpoint{3.806107in}{0.771329in}}%
\pgfpathmoveto{\pgfqpoint{3.801566in}{0.774278in}}%
\pgfpathlineto{\pgfqpoint{3.801566in}{0.774278in}}%
\pgfpathlineto{\pgfqpoint{3.801566in}{0.777227in}}%
\pgfpathlineto{\pgfqpoint{3.806107in}{0.777227in}}%
\pgfpathlineto{\pgfqpoint{3.806107in}{0.774278in}}%
\pgfpathmoveto{\pgfqpoint{3.806107in}{0.771329in}}%
\pgfpathlineto{\pgfqpoint{3.806107in}{0.771329in}}%
\pgfpathlineto{\pgfqpoint{3.806107in}{0.774278in}}%
\pgfpathlineto{\pgfqpoint{3.810648in}{0.774278in}}%
\pgfpathlineto{\pgfqpoint{3.810648in}{0.771329in}}%
\pgfpathmoveto{\pgfqpoint{3.806107in}{0.774278in}}%
\pgfpathlineto{\pgfqpoint{3.806107in}{0.774278in}}%
\pgfpathlineto{\pgfqpoint{3.806107in}{0.777227in}}%
\pgfpathlineto{\pgfqpoint{3.810648in}{0.777227in}}%
\pgfpathlineto{\pgfqpoint{3.810648in}{0.774278in}}%
\pgfpathmoveto{\pgfqpoint{3.810648in}{0.771329in}}%
\pgfpathlineto{\pgfqpoint{3.810648in}{0.771329in}}%
\pgfpathlineto{\pgfqpoint{3.810648in}{0.774278in}}%
\pgfpathlineto{\pgfqpoint{3.815189in}{0.774278in}}%
\pgfpathlineto{\pgfqpoint{3.815189in}{0.771329in}}%
\pgfpathmoveto{\pgfqpoint{3.810648in}{0.774278in}}%
\pgfpathlineto{\pgfqpoint{3.810648in}{0.774278in}}%
\pgfpathlineto{\pgfqpoint{3.810648in}{0.777227in}}%
\pgfpathlineto{\pgfqpoint{3.815189in}{0.777227in}}%
\pgfpathlineto{\pgfqpoint{3.815189in}{0.774278in}}%
\pgfpathmoveto{\pgfqpoint{3.815189in}{0.774278in}}%
\pgfpathlineto{\pgfqpoint{3.815189in}{0.774278in}}%
\pgfpathlineto{\pgfqpoint{3.815189in}{0.777227in}}%
\pgfpathlineto{\pgfqpoint{3.819730in}{0.777227in}}%
\pgfpathlineto{\pgfqpoint{3.819730in}{0.774278in}}%
\pgfpathmoveto{\pgfqpoint{3.810648in}{0.777227in}}%
\pgfpathlineto{\pgfqpoint{3.810648in}{0.777227in}}%
\pgfpathlineto{\pgfqpoint{3.810648in}{0.780176in}}%
\pgfpathlineto{\pgfqpoint{3.815189in}{0.780176in}}%
\pgfpathlineto{\pgfqpoint{3.815189in}{0.777227in}}%
\pgfpathmoveto{\pgfqpoint{3.810648in}{0.780176in}}%
\pgfpathlineto{\pgfqpoint{3.810648in}{0.780176in}}%
\pgfpathlineto{\pgfqpoint{3.810648in}{0.783125in}}%
\pgfpathlineto{\pgfqpoint{3.815189in}{0.783125in}}%
\pgfpathlineto{\pgfqpoint{3.815189in}{0.780176in}}%
\pgfpathmoveto{\pgfqpoint{3.815189in}{0.777227in}}%
\pgfpathlineto{\pgfqpoint{3.815189in}{0.777227in}}%
\pgfpathlineto{\pgfqpoint{3.815189in}{0.780176in}}%
\pgfpathlineto{\pgfqpoint{3.819730in}{0.780176in}}%
\pgfpathlineto{\pgfqpoint{3.819730in}{0.777227in}}%
\pgfpathmoveto{\pgfqpoint{3.815189in}{0.780176in}}%
\pgfpathlineto{\pgfqpoint{3.815189in}{0.780176in}}%
\pgfpathlineto{\pgfqpoint{3.815189in}{0.783125in}}%
\pgfpathlineto{\pgfqpoint{3.819730in}{0.783125in}}%
\pgfpathlineto{\pgfqpoint{3.819730in}{0.780176in}}%
\pgfpathmoveto{\pgfqpoint{3.819730in}{0.777227in}}%
\pgfpathlineto{\pgfqpoint{3.819730in}{0.777227in}}%
\pgfpathlineto{\pgfqpoint{3.819730in}{0.780176in}}%
\pgfpathlineto{\pgfqpoint{3.824270in}{0.780176in}}%
\pgfpathlineto{\pgfqpoint{3.824270in}{0.777227in}}%
\pgfpathmoveto{\pgfqpoint{3.819730in}{0.780176in}}%
\pgfpathlineto{\pgfqpoint{3.819730in}{0.780176in}}%
\pgfpathlineto{\pgfqpoint{3.819730in}{0.783125in}}%
\pgfpathlineto{\pgfqpoint{3.824270in}{0.783125in}}%
\pgfpathlineto{\pgfqpoint{3.824270in}{0.780176in}}%
\pgfpathmoveto{\pgfqpoint{3.824270in}{0.780176in}}%
\pgfpathlineto{\pgfqpoint{3.824270in}{0.780176in}}%
\pgfpathlineto{\pgfqpoint{3.824270in}{0.783125in}}%
\pgfpathlineto{\pgfqpoint{3.828811in}{0.783125in}}%
\pgfpathlineto{\pgfqpoint{3.828811in}{0.780176in}}%
\pgfpathmoveto{\pgfqpoint{3.819730in}{0.783125in}}%
\pgfpathlineto{\pgfqpoint{3.819730in}{0.783125in}}%
\pgfpathlineto{\pgfqpoint{3.819730in}{0.786075in}}%
\pgfpathlineto{\pgfqpoint{3.824270in}{0.786075in}}%
\pgfpathlineto{\pgfqpoint{3.824270in}{0.783125in}}%
\pgfpathmoveto{\pgfqpoint{3.819730in}{0.786075in}}%
\pgfpathlineto{\pgfqpoint{3.819730in}{0.786075in}}%
\pgfpathlineto{\pgfqpoint{3.819730in}{0.789024in}}%
\pgfpathlineto{\pgfqpoint{3.824270in}{0.789024in}}%
\pgfpathlineto{\pgfqpoint{3.824270in}{0.786075in}}%
\pgfpathmoveto{\pgfqpoint{3.824270in}{0.783125in}}%
\pgfpathlineto{\pgfqpoint{3.824270in}{0.783125in}}%
\pgfpathlineto{\pgfqpoint{3.824270in}{0.786075in}}%
\pgfpathlineto{\pgfqpoint{3.828811in}{0.786075in}}%
\pgfpathlineto{\pgfqpoint{3.828811in}{0.783125in}}%
\pgfpathmoveto{\pgfqpoint{3.824270in}{0.786075in}}%
\pgfpathlineto{\pgfqpoint{3.824270in}{0.786075in}}%
\pgfpathlineto{\pgfqpoint{3.824270in}{0.789024in}}%
\pgfpathlineto{\pgfqpoint{3.828811in}{0.789024in}}%
\pgfpathlineto{\pgfqpoint{3.828811in}{0.786075in}}%
\pgfpathmoveto{\pgfqpoint{3.828811in}{0.783125in}}%
\pgfpathlineto{\pgfqpoint{3.828811in}{0.783125in}}%
\pgfpathlineto{\pgfqpoint{3.828811in}{0.786075in}}%
\pgfpathlineto{\pgfqpoint{3.833352in}{0.786075in}}%
\pgfpathlineto{\pgfqpoint{3.833352in}{0.783125in}}%
\pgfpathmoveto{\pgfqpoint{3.828811in}{0.786075in}}%
\pgfpathlineto{\pgfqpoint{3.828811in}{0.786075in}}%
\pgfpathlineto{\pgfqpoint{3.828811in}{0.789024in}}%
\pgfpathlineto{\pgfqpoint{3.833352in}{0.789024in}}%
\pgfpathlineto{\pgfqpoint{3.833352in}{0.786075in}}%
\pgfpathmoveto{\pgfqpoint{3.828811in}{0.789024in}}%
\pgfpathlineto{\pgfqpoint{3.828811in}{0.789024in}}%
\pgfpathlineto{\pgfqpoint{3.828811in}{0.791973in}}%
\pgfpathlineto{\pgfqpoint{3.833352in}{0.791973in}}%
\pgfpathlineto{\pgfqpoint{3.833352in}{0.789024in}}%
\pgfpathmoveto{\pgfqpoint{3.828811in}{0.791973in}}%
\pgfpathlineto{\pgfqpoint{3.828811in}{0.791973in}}%
\pgfpathlineto{\pgfqpoint{3.828811in}{0.794922in}}%
\pgfpathlineto{\pgfqpoint{3.833352in}{0.794922in}}%
\pgfpathlineto{\pgfqpoint{3.833352in}{0.791973in}}%
\pgfpathmoveto{\pgfqpoint{3.833352in}{0.789024in}}%
\pgfpathlineto{\pgfqpoint{3.833352in}{0.789024in}}%
\pgfpathlineto{\pgfqpoint{3.833352in}{0.791973in}}%
\pgfpathlineto{\pgfqpoint{3.837893in}{0.791973in}}%
\pgfpathlineto{\pgfqpoint{3.837893in}{0.789024in}}%
\pgfpathmoveto{\pgfqpoint{3.833352in}{0.791973in}}%
\pgfpathlineto{\pgfqpoint{3.833352in}{0.791973in}}%
\pgfpathlineto{\pgfqpoint{3.833352in}{0.794922in}}%
\pgfpathlineto{\pgfqpoint{3.837893in}{0.794922in}}%
\pgfpathlineto{\pgfqpoint{3.837893in}{0.791973in}}%
\pgfpathmoveto{\pgfqpoint{3.837893in}{0.791973in}}%
\pgfpathlineto{\pgfqpoint{3.837893in}{0.791973in}}%
\pgfpathlineto{\pgfqpoint{3.837893in}{0.794922in}}%
\pgfpathlineto{\pgfqpoint{3.842434in}{0.794922in}}%
\pgfpathlineto{\pgfqpoint{3.842434in}{0.791973in}}%
\pgfpathmoveto{\pgfqpoint{3.837893in}{0.794922in}}%
\pgfpathlineto{\pgfqpoint{3.837893in}{0.794922in}}%
\pgfpathlineto{\pgfqpoint{3.837893in}{0.797871in}}%
\pgfpathlineto{\pgfqpoint{3.842434in}{0.797871in}}%
\pgfpathlineto{\pgfqpoint{3.842434in}{0.794922in}}%
\pgfpathmoveto{\pgfqpoint{3.837893in}{0.797871in}}%
\pgfpathlineto{\pgfqpoint{3.837893in}{0.797871in}}%
\pgfpathlineto{\pgfqpoint{3.837893in}{0.800821in}}%
\pgfpathlineto{\pgfqpoint{3.842434in}{0.800821in}}%
\pgfpathlineto{\pgfqpoint{3.842434in}{0.797871in}}%
\pgfpathmoveto{\pgfqpoint{3.842434in}{0.794922in}}%
\pgfpathlineto{\pgfqpoint{3.842434in}{0.794922in}}%
\pgfpathlineto{\pgfqpoint{3.842434in}{0.797871in}}%
\pgfpathlineto{\pgfqpoint{3.846975in}{0.797871in}}%
\pgfpathlineto{\pgfqpoint{3.846975in}{0.794922in}}%
\pgfpathmoveto{\pgfqpoint{3.842434in}{0.797871in}}%
\pgfpathlineto{\pgfqpoint{3.842434in}{0.797871in}}%
\pgfpathlineto{\pgfqpoint{3.842434in}{0.800821in}}%
\pgfpathlineto{\pgfqpoint{3.846975in}{0.800821in}}%
\pgfpathlineto{\pgfqpoint{3.846975in}{0.797871in}}%
\pgfpathmoveto{\pgfqpoint{3.846975in}{0.797871in}}%
\pgfpathlineto{\pgfqpoint{3.846975in}{0.797871in}}%
\pgfpathlineto{\pgfqpoint{3.846975in}{0.800821in}}%
\pgfpathlineto{\pgfqpoint{3.851515in}{0.800821in}}%
\pgfpathlineto{\pgfqpoint{3.851515in}{0.797871in}}%
\pgfpathmoveto{\pgfqpoint{3.846975in}{0.800821in}}%
\pgfpathlineto{\pgfqpoint{3.846975in}{0.800821in}}%
\pgfpathlineto{\pgfqpoint{3.846975in}{0.803770in}}%
\pgfpathlineto{\pgfqpoint{3.851515in}{0.803770in}}%
\pgfpathlineto{\pgfqpoint{3.851515in}{0.800821in}}%
\pgfpathmoveto{\pgfqpoint{3.846975in}{0.803770in}}%
\pgfpathlineto{\pgfqpoint{3.846975in}{0.803770in}}%
\pgfpathlineto{\pgfqpoint{3.846975in}{0.806719in}}%
\pgfpathlineto{\pgfqpoint{3.851515in}{0.806719in}}%
\pgfpathlineto{\pgfqpoint{3.851515in}{0.803770in}}%
\pgfpathmoveto{\pgfqpoint{3.851515in}{0.800821in}}%
\pgfpathlineto{\pgfqpoint{3.851515in}{0.800821in}}%
\pgfpathlineto{\pgfqpoint{3.851515in}{0.803770in}}%
\pgfpathlineto{\pgfqpoint{3.856056in}{0.803770in}}%
\pgfpathlineto{\pgfqpoint{3.856056in}{0.800821in}}%
\pgfpathmoveto{\pgfqpoint{3.851515in}{0.803770in}}%
\pgfpathlineto{\pgfqpoint{3.851515in}{0.803770in}}%
\pgfpathlineto{\pgfqpoint{3.851515in}{0.806719in}}%
\pgfpathlineto{\pgfqpoint{3.856056in}{0.806719in}}%
\pgfpathlineto{\pgfqpoint{3.856056in}{0.803770in}}%
\pgfpathmoveto{\pgfqpoint{3.846975in}{0.806719in}}%
\pgfpathlineto{\pgfqpoint{3.846975in}{0.806719in}}%
\pgfpathlineto{\pgfqpoint{3.846975in}{0.809668in}}%
\pgfpathlineto{\pgfqpoint{3.851515in}{0.809668in}}%
\pgfpathlineto{\pgfqpoint{3.851515in}{0.806719in}}%
\pgfpathmoveto{\pgfqpoint{3.846975in}{0.809668in}}%
\pgfpathlineto{\pgfqpoint{3.846975in}{0.809668in}}%
\pgfpathlineto{\pgfqpoint{3.846975in}{0.812618in}}%
\pgfpathlineto{\pgfqpoint{3.851515in}{0.812618in}}%
\pgfpathlineto{\pgfqpoint{3.851515in}{0.809668in}}%
\pgfpathmoveto{\pgfqpoint{3.851515in}{0.806719in}}%
\pgfpathlineto{\pgfqpoint{3.851515in}{0.806719in}}%
\pgfpathlineto{\pgfqpoint{3.851515in}{0.809668in}}%
\pgfpathlineto{\pgfqpoint{3.856056in}{0.809668in}}%
\pgfpathlineto{\pgfqpoint{3.856056in}{0.806719in}}%
\pgfpathmoveto{\pgfqpoint{3.851515in}{0.809668in}}%
\pgfpathlineto{\pgfqpoint{3.851515in}{0.809668in}}%
\pgfpathlineto{\pgfqpoint{3.851515in}{0.812618in}}%
\pgfpathlineto{\pgfqpoint{3.856056in}{0.812618in}}%
\pgfpathlineto{\pgfqpoint{3.856056in}{0.809668in}}%
\pgfpathmoveto{\pgfqpoint{3.856056in}{0.806719in}}%
\pgfpathlineto{\pgfqpoint{3.856056in}{0.806719in}}%
\pgfpathlineto{\pgfqpoint{3.856056in}{0.809668in}}%
\pgfpathlineto{\pgfqpoint{3.860597in}{0.809668in}}%
\pgfpathlineto{\pgfqpoint{3.860597in}{0.806719in}}%
\pgfpathmoveto{\pgfqpoint{3.856056in}{0.809668in}}%
\pgfpathlineto{\pgfqpoint{3.856056in}{0.809668in}}%
\pgfpathlineto{\pgfqpoint{3.856056in}{0.812618in}}%
\pgfpathlineto{\pgfqpoint{3.860597in}{0.812618in}}%
\pgfpathlineto{\pgfqpoint{3.860597in}{0.809668in}}%
\pgfpathmoveto{\pgfqpoint{3.860597in}{0.809668in}}%
\pgfpathlineto{\pgfqpoint{3.860597in}{0.809668in}}%
\pgfpathlineto{\pgfqpoint{3.860597in}{0.812618in}}%
\pgfpathlineto{\pgfqpoint{3.865138in}{0.812618in}}%
\pgfpathlineto{\pgfqpoint{3.865138in}{0.809668in}}%
\pgfpathmoveto{\pgfqpoint{3.856056in}{0.812618in}}%
\pgfpathlineto{\pgfqpoint{3.856056in}{0.812618in}}%
\pgfpathlineto{\pgfqpoint{3.856056in}{0.815567in}}%
\pgfpathlineto{\pgfqpoint{3.860597in}{0.815567in}}%
\pgfpathlineto{\pgfqpoint{3.860597in}{0.812618in}}%
\pgfpathmoveto{\pgfqpoint{3.856056in}{0.815567in}}%
\pgfpathlineto{\pgfqpoint{3.856056in}{0.815567in}}%
\pgfpathlineto{\pgfqpoint{3.856056in}{0.818516in}}%
\pgfpathlineto{\pgfqpoint{3.860597in}{0.818516in}}%
\pgfpathlineto{\pgfqpoint{3.860597in}{0.815567in}}%
\pgfpathmoveto{\pgfqpoint{3.860597in}{0.812618in}}%
\pgfpathlineto{\pgfqpoint{3.860597in}{0.812618in}}%
\pgfpathlineto{\pgfqpoint{3.860597in}{0.815567in}}%
\pgfpathlineto{\pgfqpoint{3.865138in}{0.815567in}}%
\pgfpathlineto{\pgfqpoint{3.865138in}{0.812618in}}%
\pgfpathmoveto{\pgfqpoint{3.860597in}{0.815567in}}%
\pgfpathlineto{\pgfqpoint{3.860597in}{0.815567in}}%
\pgfpathlineto{\pgfqpoint{3.860597in}{0.818516in}}%
\pgfpathlineto{\pgfqpoint{3.865138in}{0.818516in}}%
\pgfpathlineto{\pgfqpoint{3.865138in}{0.815567in}}%
\pgfpathmoveto{\pgfqpoint{3.865138in}{0.812618in}}%
\pgfpathlineto{\pgfqpoint{3.865138in}{0.812618in}}%
\pgfpathlineto{\pgfqpoint{3.865138in}{0.815567in}}%
\pgfpathlineto{\pgfqpoint{3.869679in}{0.815567in}}%
\pgfpathlineto{\pgfqpoint{3.869679in}{0.812618in}}%
\pgfpathmoveto{\pgfqpoint{3.865138in}{0.815567in}}%
\pgfpathlineto{\pgfqpoint{3.865138in}{0.815567in}}%
\pgfpathlineto{\pgfqpoint{3.865138in}{0.818516in}}%
\pgfpathlineto{\pgfqpoint{3.869679in}{0.818516in}}%
\pgfpathlineto{\pgfqpoint{3.869679in}{0.815567in}}%
\pgfpathmoveto{\pgfqpoint{3.869679in}{0.815567in}}%
\pgfpathlineto{\pgfqpoint{3.869679in}{0.815567in}}%
\pgfpathlineto{\pgfqpoint{3.869679in}{0.818516in}}%
\pgfpathlineto{\pgfqpoint{3.874220in}{0.818516in}}%
\pgfpathlineto{\pgfqpoint{3.874220in}{0.815567in}}%
\pgfpathmoveto{\pgfqpoint{3.865138in}{0.818516in}}%
\pgfpathlineto{\pgfqpoint{3.865138in}{0.818516in}}%
\pgfpathlineto{\pgfqpoint{3.865138in}{0.821465in}}%
\pgfpathlineto{\pgfqpoint{3.869679in}{0.821465in}}%
\pgfpathlineto{\pgfqpoint{3.869679in}{0.818516in}}%
\pgfpathmoveto{\pgfqpoint{3.865138in}{0.821465in}}%
\pgfpathlineto{\pgfqpoint{3.865138in}{0.821465in}}%
\pgfpathlineto{\pgfqpoint{3.865138in}{0.824415in}}%
\pgfpathlineto{\pgfqpoint{3.869679in}{0.824415in}}%
\pgfpathlineto{\pgfqpoint{3.869679in}{0.821465in}}%
\pgfpathmoveto{\pgfqpoint{3.869679in}{0.818516in}}%
\pgfpathlineto{\pgfqpoint{3.869679in}{0.818516in}}%
\pgfpathlineto{\pgfqpoint{3.869679in}{0.821465in}}%
\pgfpathlineto{\pgfqpoint{3.874220in}{0.821465in}}%
\pgfpathlineto{\pgfqpoint{3.874220in}{0.818516in}}%
\pgfpathmoveto{\pgfqpoint{3.869679in}{0.821465in}}%
\pgfpathlineto{\pgfqpoint{3.869679in}{0.821465in}}%
\pgfpathlineto{\pgfqpoint{3.869679in}{0.824415in}}%
\pgfpathlineto{\pgfqpoint{3.874220in}{0.824415in}}%
\pgfpathlineto{\pgfqpoint{3.874220in}{0.821465in}}%
\pgfpathmoveto{\pgfqpoint{3.874220in}{0.818516in}}%
\pgfpathlineto{\pgfqpoint{3.874220in}{0.818516in}}%
\pgfpathlineto{\pgfqpoint{3.874220in}{0.821465in}}%
\pgfpathlineto{\pgfqpoint{3.878760in}{0.821465in}}%
\pgfpathlineto{\pgfqpoint{3.878760in}{0.818516in}}%
\pgfpathmoveto{\pgfqpoint{3.874220in}{0.821465in}}%
\pgfpathlineto{\pgfqpoint{3.874220in}{0.821465in}}%
\pgfpathlineto{\pgfqpoint{3.874220in}{0.824415in}}%
\pgfpathlineto{\pgfqpoint{3.878760in}{0.824415in}}%
\pgfpathlineto{\pgfqpoint{3.878760in}{0.821465in}}%
\pgfpathmoveto{\pgfqpoint{3.874220in}{0.824415in}}%
\pgfpathlineto{\pgfqpoint{3.874220in}{0.824415in}}%
\pgfpathlineto{\pgfqpoint{3.874220in}{0.827364in}}%
\pgfpathlineto{\pgfqpoint{3.878760in}{0.827364in}}%
\pgfpathlineto{\pgfqpoint{3.878760in}{0.824415in}}%
\pgfpathmoveto{\pgfqpoint{3.874220in}{0.827364in}}%
\pgfpathlineto{\pgfqpoint{3.874220in}{0.827364in}}%
\pgfpathlineto{\pgfqpoint{3.874220in}{0.830313in}}%
\pgfpathlineto{\pgfqpoint{3.878760in}{0.830313in}}%
\pgfpathlineto{\pgfqpoint{3.878760in}{0.827364in}}%
\pgfpathmoveto{\pgfqpoint{3.878760in}{0.824415in}}%
\pgfpathlineto{\pgfqpoint{3.878760in}{0.824415in}}%
\pgfpathlineto{\pgfqpoint{3.878760in}{0.827364in}}%
\pgfpathlineto{\pgfqpoint{3.883301in}{0.827364in}}%
\pgfpathlineto{\pgfqpoint{3.883301in}{0.824415in}}%
\pgfpathmoveto{\pgfqpoint{3.878760in}{0.827364in}}%
\pgfpathlineto{\pgfqpoint{3.878760in}{0.827364in}}%
\pgfpathlineto{\pgfqpoint{3.878760in}{0.830313in}}%
\pgfpathlineto{\pgfqpoint{3.883301in}{0.830313in}}%
\pgfpathlineto{\pgfqpoint{3.883301in}{0.827364in}}%
\pgfpathmoveto{\pgfqpoint{3.883301in}{0.827364in}}%
\pgfpathlineto{\pgfqpoint{3.883301in}{0.827364in}}%
\pgfpathlineto{\pgfqpoint{3.883301in}{0.830313in}}%
\pgfpathlineto{\pgfqpoint{3.887842in}{0.830313in}}%
\pgfpathlineto{\pgfqpoint{3.887842in}{0.827364in}}%
\pgfpathmoveto{\pgfqpoint{3.883301in}{0.830313in}}%
\pgfpathlineto{\pgfqpoint{3.883301in}{0.830313in}}%
\pgfpathlineto{\pgfqpoint{3.883301in}{0.833262in}}%
\pgfpathlineto{\pgfqpoint{3.887842in}{0.833262in}}%
\pgfpathlineto{\pgfqpoint{3.887842in}{0.830313in}}%
\pgfpathmoveto{\pgfqpoint{3.883301in}{0.833262in}}%
\pgfpathlineto{\pgfqpoint{3.883301in}{0.833262in}}%
\pgfpathlineto{\pgfqpoint{3.883301in}{0.836212in}}%
\pgfpathlineto{\pgfqpoint{3.887842in}{0.836212in}}%
\pgfpathlineto{\pgfqpoint{3.887842in}{0.833262in}}%
\pgfpathmoveto{\pgfqpoint{3.887842in}{0.830313in}}%
\pgfpathlineto{\pgfqpoint{3.887842in}{0.830313in}}%
\pgfpathlineto{\pgfqpoint{3.887842in}{0.833262in}}%
\pgfpathlineto{\pgfqpoint{3.892383in}{0.833262in}}%
\pgfpathlineto{\pgfqpoint{3.892383in}{0.830313in}}%
\pgfpathmoveto{\pgfqpoint{3.887842in}{0.833262in}}%
\pgfpathlineto{\pgfqpoint{3.887842in}{0.833262in}}%
\pgfpathlineto{\pgfqpoint{3.887842in}{0.836212in}}%
\pgfpathlineto{\pgfqpoint{3.892383in}{0.836212in}}%
\pgfpathlineto{\pgfqpoint{3.892383in}{0.833262in}}%
\pgfpathmoveto{\pgfqpoint{3.892383in}{0.833262in}}%
\pgfpathlineto{\pgfqpoint{3.892383in}{0.833262in}}%
\pgfpathlineto{\pgfqpoint{3.892383in}{0.836212in}}%
\pgfpathlineto{\pgfqpoint{3.896924in}{0.836212in}}%
\pgfpathlineto{\pgfqpoint{3.896924in}{0.833262in}}%
\pgfpathmoveto{\pgfqpoint{3.892383in}{0.836212in}}%
\pgfpathlineto{\pgfqpoint{3.892383in}{0.836212in}}%
\pgfpathlineto{\pgfqpoint{3.892383in}{0.839161in}}%
\pgfpathlineto{\pgfqpoint{3.896924in}{0.839161in}}%
\pgfpathlineto{\pgfqpoint{3.896924in}{0.836212in}}%
\pgfpathmoveto{\pgfqpoint{3.892383in}{0.839161in}}%
\pgfpathlineto{\pgfqpoint{3.892383in}{0.839161in}}%
\pgfpathlineto{\pgfqpoint{3.892383in}{0.842110in}}%
\pgfpathlineto{\pgfqpoint{3.896924in}{0.842110in}}%
\pgfpathlineto{\pgfqpoint{3.896924in}{0.839161in}}%
\pgfpathmoveto{\pgfqpoint{3.896924in}{0.836212in}}%
\pgfpathlineto{\pgfqpoint{3.896924in}{0.836212in}}%
\pgfpathlineto{\pgfqpoint{3.896924in}{0.839161in}}%
\pgfpathlineto{\pgfqpoint{3.901465in}{0.839161in}}%
\pgfpathlineto{\pgfqpoint{3.901465in}{0.836212in}}%
\pgfpathmoveto{\pgfqpoint{3.896924in}{0.839161in}}%
\pgfpathlineto{\pgfqpoint{3.896924in}{0.839161in}}%
\pgfpathlineto{\pgfqpoint{3.896924in}{0.842110in}}%
\pgfpathlineto{\pgfqpoint{3.901465in}{0.842110in}}%
\pgfpathlineto{\pgfqpoint{3.901465in}{0.839161in}}%
\pgfpathmoveto{\pgfqpoint{3.901465in}{0.839161in}}%
\pgfpathlineto{\pgfqpoint{3.901465in}{0.839161in}}%
\pgfpathlineto{\pgfqpoint{3.901465in}{0.842110in}}%
\pgfpathlineto{\pgfqpoint{3.906006in}{0.842110in}}%
\pgfpathlineto{\pgfqpoint{3.906006in}{0.839161in}}%
\pgfpathmoveto{\pgfqpoint{3.901465in}{0.842110in}}%
\pgfpathlineto{\pgfqpoint{3.901465in}{0.842110in}}%
\pgfpathlineto{\pgfqpoint{3.901465in}{0.845059in}}%
\pgfpathlineto{\pgfqpoint{3.906006in}{0.845059in}}%
\pgfpathlineto{\pgfqpoint{3.906006in}{0.842110in}}%
\pgfpathmoveto{\pgfqpoint{3.901465in}{0.845059in}}%
\pgfpathlineto{\pgfqpoint{3.901465in}{0.845059in}}%
\pgfpathlineto{\pgfqpoint{3.901465in}{0.848008in}}%
\pgfpathlineto{\pgfqpoint{3.906006in}{0.848008in}}%
\pgfpathlineto{\pgfqpoint{3.906006in}{0.845059in}}%
\pgfpathmoveto{\pgfqpoint{3.906006in}{0.845059in}}%
\pgfpathlineto{\pgfqpoint{3.906006in}{0.845059in}}%
\pgfpathlineto{\pgfqpoint{3.906006in}{0.848008in}}%
\pgfpathlineto{\pgfqpoint{3.910546in}{0.848008in}}%
\pgfpathlineto{\pgfqpoint{3.910546in}{0.845059in}}%
\pgfpathmoveto{\pgfqpoint{3.901465in}{0.848008in}}%
\pgfpathlineto{\pgfqpoint{3.901465in}{0.848008in}}%
\pgfpathlineto{\pgfqpoint{3.901465in}{0.850958in}}%
\pgfpathlineto{\pgfqpoint{3.906006in}{0.850958in}}%
\pgfpathlineto{\pgfqpoint{3.906006in}{0.848008in}}%
\pgfpathmoveto{\pgfqpoint{3.901465in}{0.850958in}}%
\pgfpathlineto{\pgfqpoint{3.901465in}{0.850958in}}%
\pgfpathlineto{\pgfqpoint{3.901465in}{0.853907in}}%
\pgfpathlineto{\pgfqpoint{3.906006in}{0.853907in}}%
\pgfpathlineto{\pgfqpoint{3.906006in}{0.850958in}}%
\pgfpathmoveto{\pgfqpoint{3.906006in}{0.848008in}}%
\pgfpathlineto{\pgfqpoint{3.906006in}{0.848008in}}%
\pgfpathlineto{\pgfqpoint{3.906006in}{0.850958in}}%
\pgfpathlineto{\pgfqpoint{3.910546in}{0.850958in}}%
\pgfpathlineto{\pgfqpoint{3.910546in}{0.848008in}}%
\pgfpathmoveto{\pgfqpoint{3.906006in}{0.850958in}}%
\pgfpathlineto{\pgfqpoint{3.906006in}{0.850958in}}%
\pgfpathlineto{\pgfqpoint{3.906006in}{0.853907in}}%
\pgfpathlineto{\pgfqpoint{3.910546in}{0.853907in}}%
\pgfpathlineto{\pgfqpoint{3.910546in}{0.850958in}}%
\pgfpathmoveto{\pgfqpoint{3.910546in}{0.848008in}}%
\pgfpathlineto{\pgfqpoint{3.910546in}{0.848008in}}%
\pgfpathlineto{\pgfqpoint{3.910546in}{0.850958in}}%
\pgfpathlineto{\pgfqpoint{3.915087in}{0.850958in}}%
\pgfpathlineto{\pgfqpoint{3.915087in}{0.848008in}}%
\pgfpathmoveto{\pgfqpoint{3.910546in}{0.850958in}}%
\pgfpathlineto{\pgfqpoint{3.910546in}{0.850958in}}%
\pgfpathlineto{\pgfqpoint{3.910546in}{0.853907in}}%
\pgfpathlineto{\pgfqpoint{3.915087in}{0.853907in}}%
\pgfpathlineto{\pgfqpoint{3.915087in}{0.850958in}}%
\pgfpathmoveto{\pgfqpoint{3.915087in}{0.850958in}}%
\pgfpathlineto{\pgfqpoint{3.915087in}{0.850958in}}%
\pgfpathlineto{\pgfqpoint{3.915087in}{0.853907in}}%
\pgfpathlineto{\pgfqpoint{3.919628in}{0.853907in}}%
\pgfpathlineto{\pgfqpoint{3.919628in}{0.850958in}}%
\pgfpathmoveto{\pgfqpoint{3.910546in}{0.853907in}}%
\pgfpathlineto{\pgfqpoint{3.910546in}{0.853907in}}%
\pgfpathlineto{\pgfqpoint{3.910546in}{0.856856in}}%
\pgfpathlineto{\pgfqpoint{3.915087in}{0.856856in}}%
\pgfpathlineto{\pgfqpoint{3.915087in}{0.853907in}}%
\pgfpathmoveto{\pgfqpoint{3.910546in}{0.856856in}}%
\pgfpathlineto{\pgfqpoint{3.910546in}{0.856856in}}%
\pgfpathlineto{\pgfqpoint{3.910546in}{0.859805in}}%
\pgfpathlineto{\pgfqpoint{3.915087in}{0.859805in}}%
\pgfpathlineto{\pgfqpoint{3.915087in}{0.856856in}}%
\pgfpathmoveto{\pgfqpoint{3.915087in}{0.853907in}}%
\pgfpathlineto{\pgfqpoint{3.915087in}{0.853907in}}%
\pgfpathlineto{\pgfqpoint{3.915087in}{0.856856in}}%
\pgfpathlineto{\pgfqpoint{3.919628in}{0.856856in}}%
\pgfpathlineto{\pgfqpoint{3.919628in}{0.853907in}}%
\pgfpathmoveto{\pgfqpoint{3.915087in}{0.856856in}}%
\pgfpathlineto{\pgfqpoint{3.915087in}{0.856856in}}%
\pgfpathlineto{\pgfqpoint{3.915087in}{0.859805in}}%
\pgfpathlineto{\pgfqpoint{3.919628in}{0.859805in}}%
\pgfpathlineto{\pgfqpoint{3.919628in}{0.856856in}}%
\pgfpathmoveto{\pgfqpoint{3.919628in}{0.853907in}}%
\pgfpathlineto{\pgfqpoint{3.919628in}{0.853907in}}%
\pgfpathlineto{\pgfqpoint{3.919628in}{0.856856in}}%
\pgfpathlineto{\pgfqpoint{3.924169in}{0.856856in}}%
\pgfpathlineto{\pgfqpoint{3.924169in}{0.853907in}}%
\pgfpathmoveto{\pgfqpoint{3.919628in}{0.856856in}}%
\pgfpathlineto{\pgfqpoint{3.919628in}{0.856856in}}%
\pgfpathlineto{\pgfqpoint{3.919628in}{0.859805in}}%
\pgfpathlineto{\pgfqpoint{3.924169in}{0.859805in}}%
\pgfpathlineto{\pgfqpoint{3.924169in}{0.856856in}}%
\pgfpathmoveto{\pgfqpoint{3.924169in}{0.856856in}}%
\pgfpathlineto{\pgfqpoint{3.924169in}{0.856856in}}%
\pgfpathlineto{\pgfqpoint{3.924169in}{0.859805in}}%
\pgfpathlineto{\pgfqpoint{3.928710in}{0.859805in}}%
\pgfpathlineto{\pgfqpoint{3.928710in}{0.856856in}}%
\pgfpathmoveto{\pgfqpoint{3.919628in}{0.859805in}}%
\pgfpathlineto{\pgfqpoint{3.919628in}{0.859805in}}%
\pgfpathlineto{\pgfqpoint{3.919628in}{0.862755in}}%
\pgfpathlineto{\pgfqpoint{3.924169in}{0.862755in}}%
\pgfpathlineto{\pgfqpoint{3.924169in}{0.859805in}}%
\pgfpathmoveto{\pgfqpoint{3.919628in}{0.862755in}}%
\pgfpathlineto{\pgfqpoint{3.919628in}{0.862755in}}%
\pgfpathlineto{\pgfqpoint{3.919628in}{0.865704in}}%
\pgfpathlineto{\pgfqpoint{3.924169in}{0.865704in}}%
\pgfpathlineto{\pgfqpoint{3.924169in}{0.862755in}}%
\pgfpathmoveto{\pgfqpoint{3.924169in}{0.859805in}}%
\pgfpathlineto{\pgfqpoint{3.924169in}{0.859805in}}%
\pgfpathlineto{\pgfqpoint{3.924169in}{0.862755in}}%
\pgfpathlineto{\pgfqpoint{3.928710in}{0.862755in}}%
\pgfpathlineto{\pgfqpoint{3.928710in}{0.859805in}}%
\pgfpathmoveto{\pgfqpoint{3.924169in}{0.862755in}}%
\pgfpathlineto{\pgfqpoint{3.924169in}{0.862755in}}%
\pgfpathlineto{\pgfqpoint{3.924169in}{0.865704in}}%
\pgfpathlineto{\pgfqpoint{3.928710in}{0.865704in}}%
\pgfpathlineto{\pgfqpoint{3.928710in}{0.862755in}}%
\pgfpathmoveto{\pgfqpoint{3.928710in}{0.862755in}}%
\pgfpathlineto{\pgfqpoint{3.928710in}{0.862755in}}%
\pgfpathlineto{\pgfqpoint{3.928710in}{0.865704in}}%
\pgfpathlineto{\pgfqpoint{3.933251in}{0.865704in}}%
\pgfpathlineto{\pgfqpoint{3.933251in}{0.862755in}}%
\pgfpathmoveto{\pgfqpoint{3.928710in}{0.865704in}}%
\pgfpathlineto{\pgfqpoint{3.928710in}{0.865704in}}%
\pgfpathlineto{\pgfqpoint{3.928710in}{0.868653in}}%
\pgfpathlineto{\pgfqpoint{3.933251in}{0.868653in}}%
\pgfpathlineto{\pgfqpoint{3.933251in}{0.865704in}}%
\pgfpathmoveto{\pgfqpoint{3.928710in}{0.868653in}}%
\pgfpathlineto{\pgfqpoint{3.928710in}{0.868653in}}%
\pgfpathlineto{\pgfqpoint{3.928710in}{0.871602in}}%
\pgfpathlineto{\pgfqpoint{3.933251in}{0.871602in}}%
\pgfpathlineto{\pgfqpoint{3.933251in}{0.868653in}}%
\pgfpathmoveto{\pgfqpoint{3.933251in}{0.865704in}}%
\pgfpathlineto{\pgfqpoint{3.933251in}{0.865704in}}%
\pgfpathlineto{\pgfqpoint{3.933251in}{0.868653in}}%
\pgfpathlineto{\pgfqpoint{3.937791in}{0.868653in}}%
\pgfpathlineto{\pgfqpoint{3.937791in}{0.865704in}}%
\pgfpathmoveto{\pgfqpoint{3.933251in}{0.868653in}}%
\pgfpathlineto{\pgfqpoint{3.933251in}{0.868653in}}%
\pgfpathlineto{\pgfqpoint{3.933251in}{0.871602in}}%
\pgfpathlineto{\pgfqpoint{3.937791in}{0.871602in}}%
\pgfpathlineto{\pgfqpoint{3.937791in}{0.868653in}}%
\pgfpathmoveto{\pgfqpoint{3.937791in}{0.868653in}}%
\pgfpathlineto{\pgfqpoint{3.937791in}{0.868653in}}%
\pgfpathlineto{\pgfqpoint{3.937791in}{0.871602in}}%
\pgfpathlineto{\pgfqpoint{3.942332in}{0.871602in}}%
\pgfpathlineto{\pgfqpoint{3.942332in}{0.868653in}}%
\pgfpathmoveto{\pgfqpoint{3.937791in}{0.871602in}}%
\pgfpathlineto{\pgfqpoint{3.937791in}{0.871602in}}%
\pgfpathlineto{\pgfqpoint{3.937791in}{0.874552in}}%
\pgfpathlineto{\pgfqpoint{3.942332in}{0.874552in}}%
\pgfpathlineto{\pgfqpoint{3.942332in}{0.871602in}}%
\pgfpathmoveto{\pgfqpoint{3.937791in}{0.874552in}}%
\pgfpathlineto{\pgfqpoint{3.937791in}{0.874552in}}%
\pgfpathlineto{\pgfqpoint{3.937791in}{0.877501in}}%
\pgfpathlineto{\pgfqpoint{3.942332in}{0.877501in}}%
\pgfpathlineto{\pgfqpoint{3.942332in}{0.874552in}}%
\pgfpathmoveto{\pgfqpoint{3.942332in}{0.871602in}}%
\pgfpathlineto{\pgfqpoint{3.942332in}{0.871602in}}%
\pgfpathlineto{\pgfqpoint{3.942332in}{0.874552in}}%
\pgfpathlineto{\pgfqpoint{3.946873in}{0.874552in}}%
\pgfpathlineto{\pgfqpoint{3.946873in}{0.871602in}}%
\pgfpathmoveto{\pgfqpoint{3.942332in}{0.874552in}}%
\pgfpathlineto{\pgfqpoint{3.942332in}{0.874552in}}%
\pgfpathlineto{\pgfqpoint{3.942332in}{0.877501in}}%
\pgfpathlineto{\pgfqpoint{3.946873in}{0.877501in}}%
\pgfpathlineto{\pgfqpoint{3.946873in}{0.874552in}}%
\pgfpathmoveto{\pgfqpoint{3.865138in}{1.313983in}}%
\pgfpathlineto{\pgfqpoint{3.865138in}{1.313983in}}%
\pgfpathlineto{\pgfqpoint{3.865138in}{1.316932in}}%
\pgfpathlineto{\pgfqpoint{3.869679in}{1.316932in}}%
\pgfpathlineto{\pgfqpoint{3.869679in}{1.313983in}}%
\pgfpathmoveto{\pgfqpoint{3.865138in}{1.316932in}}%
\pgfpathlineto{\pgfqpoint{3.865138in}{1.316932in}}%
\pgfpathlineto{\pgfqpoint{3.865138in}{1.319881in}}%
\pgfpathlineto{\pgfqpoint{3.869679in}{1.319881in}}%
\pgfpathlineto{\pgfqpoint{3.869679in}{1.316932in}}%
\pgfpathmoveto{\pgfqpoint{3.869679in}{1.313983in}}%
\pgfpathlineto{\pgfqpoint{3.869679in}{1.313983in}}%
\pgfpathlineto{\pgfqpoint{3.869679in}{1.316932in}}%
\pgfpathlineto{\pgfqpoint{3.874220in}{1.316932in}}%
\pgfpathlineto{\pgfqpoint{3.874220in}{1.313983in}}%
\pgfpathmoveto{\pgfqpoint{3.869679in}{1.316932in}}%
\pgfpathlineto{\pgfqpoint{3.869679in}{1.316932in}}%
\pgfpathlineto{\pgfqpoint{3.869679in}{1.319881in}}%
\pgfpathlineto{\pgfqpoint{3.874220in}{1.319881in}}%
\pgfpathlineto{\pgfqpoint{3.874220in}{1.316932in}}%
\pgfpathmoveto{\pgfqpoint{3.865138in}{1.319881in}}%
\pgfpathlineto{\pgfqpoint{3.865138in}{1.319881in}}%
\pgfpathlineto{\pgfqpoint{3.865138in}{1.322830in}}%
\pgfpathlineto{\pgfqpoint{3.869679in}{1.322830in}}%
\pgfpathlineto{\pgfqpoint{3.869679in}{1.319881in}}%
\pgfpathmoveto{\pgfqpoint{3.865138in}{1.322830in}}%
\pgfpathlineto{\pgfqpoint{3.865138in}{1.322830in}}%
\pgfpathlineto{\pgfqpoint{3.865138in}{1.325779in}}%
\pgfpathlineto{\pgfqpoint{3.869679in}{1.325779in}}%
\pgfpathlineto{\pgfqpoint{3.869679in}{1.322830in}}%
\pgfpathmoveto{\pgfqpoint{3.869679in}{1.319881in}}%
\pgfpathlineto{\pgfqpoint{3.869679in}{1.319881in}}%
\pgfpathlineto{\pgfqpoint{3.869679in}{1.322830in}}%
\pgfpathlineto{\pgfqpoint{3.874220in}{1.322830in}}%
\pgfpathlineto{\pgfqpoint{3.874220in}{1.319881in}}%
\pgfpathmoveto{\pgfqpoint{3.846975in}{1.331678in}}%
\pgfpathlineto{\pgfqpoint{3.846975in}{1.331678in}}%
\pgfpathlineto{\pgfqpoint{3.846975in}{1.334627in}}%
\pgfpathlineto{\pgfqpoint{3.851515in}{1.334627in}}%
\pgfpathlineto{\pgfqpoint{3.851515in}{1.331678in}}%
\pgfpathmoveto{\pgfqpoint{3.846975in}{1.334627in}}%
\pgfpathlineto{\pgfqpoint{3.846975in}{1.334627in}}%
\pgfpathlineto{\pgfqpoint{3.846975in}{1.337576in}}%
\pgfpathlineto{\pgfqpoint{3.851515in}{1.337576in}}%
\pgfpathlineto{\pgfqpoint{3.851515in}{1.334627in}}%
\pgfpathmoveto{\pgfqpoint{3.851515in}{1.331678in}}%
\pgfpathlineto{\pgfqpoint{3.851515in}{1.331678in}}%
\pgfpathlineto{\pgfqpoint{3.851515in}{1.334627in}}%
\pgfpathlineto{\pgfqpoint{3.856056in}{1.334627in}}%
\pgfpathlineto{\pgfqpoint{3.856056in}{1.331678in}}%
\pgfpathmoveto{\pgfqpoint{3.851515in}{1.334627in}}%
\pgfpathlineto{\pgfqpoint{3.851515in}{1.334627in}}%
\pgfpathlineto{\pgfqpoint{3.851515in}{1.337576in}}%
\pgfpathlineto{\pgfqpoint{3.856056in}{1.337576in}}%
\pgfpathlineto{\pgfqpoint{3.856056in}{1.334627in}}%
\pgfpathmoveto{\pgfqpoint{3.837893in}{1.337576in}}%
\pgfpathlineto{\pgfqpoint{3.837893in}{1.337576in}}%
\pgfpathlineto{\pgfqpoint{3.837893in}{1.340525in}}%
\pgfpathlineto{\pgfqpoint{3.842434in}{1.340525in}}%
\pgfpathlineto{\pgfqpoint{3.842434in}{1.337576in}}%
\pgfpathmoveto{\pgfqpoint{3.837893in}{1.340525in}}%
\pgfpathlineto{\pgfqpoint{3.837893in}{1.340525in}}%
\pgfpathlineto{\pgfqpoint{3.837893in}{1.343474in}}%
\pgfpathlineto{\pgfqpoint{3.842434in}{1.343474in}}%
\pgfpathlineto{\pgfqpoint{3.842434in}{1.340525in}}%
\pgfpathmoveto{\pgfqpoint{3.842434in}{1.337576in}}%
\pgfpathlineto{\pgfqpoint{3.842434in}{1.337576in}}%
\pgfpathlineto{\pgfqpoint{3.842434in}{1.340525in}}%
\pgfpathlineto{\pgfqpoint{3.846975in}{1.340525in}}%
\pgfpathlineto{\pgfqpoint{3.846975in}{1.337576in}}%
\pgfpathmoveto{\pgfqpoint{3.842434in}{1.340525in}}%
\pgfpathlineto{\pgfqpoint{3.842434in}{1.340525in}}%
\pgfpathlineto{\pgfqpoint{3.842434in}{1.343474in}}%
\pgfpathlineto{\pgfqpoint{3.846975in}{1.343474in}}%
\pgfpathlineto{\pgfqpoint{3.846975in}{1.340525in}}%
\pgfpathmoveto{\pgfqpoint{3.837893in}{1.343474in}}%
\pgfpathlineto{\pgfqpoint{3.837893in}{1.343474in}}%
\pgfpathlineto{\pgfqpoint{3.837893in}{1.346424in}}%
\pgfpathlineto{\pgfqpoint{3.842434in}{1.346424in}}%
\pgfpathlineto{\pgfqpoint{3.842434in}{1.343474in}}%
\pgfpathmoveto{\pgfqpoint{3.837893in}{1.346424in}}%
\pgfpathlineto{\pgfqpoint{3.837893in}{1.346424in}}%
\pgfpathlineto{\pgfqpoint{3.837893in}{1.349373in}}%
\pgfpathlineto{\pgfqpoint{3.842434in}{1.349373in}}%
\pgfpathlineto{\pgfqpoint{3.842434in}{1.346424in}}%
\pgfpathmoveto{\pgfqpoint{3.842434in}{1.343474in}}%
\pgfpathlineto{\pgfqpoint{3.842434in}{1.343474in}}%
\pgfpathlineto{\pgfqpoint{3.842434in}{1.346424in}}%
\pgfpathlineto{\pgfqpoint{3.846975in}{1.346424in}}%
\pgfpathlineto{\pgfqpoint{3.846975in}{1.343474in}}%
\pgfpathmoveto{\pgfqpoint{3.846975in}{1.337576in}}%
\pgfpathlineto{\pgfqpoint{3.846975in}{1.337576in}}%
\pgfpathlineto{\pgfqpoint{3.846975in}{1.340525in}}%
\pgfpathlineto{\pgfqpoint{3.851515in}{1.340525in}}%
\pgfpathlineto{\pgfqpoint{3.851515in}{1.337576in}}%
\pgfpathmoveto{\pgfqpoint{3.846975in}{1.340525in}}%
\pgfpathlineto{\pgfqpoint{3.846975in}{1.340525in}}%
\pgfpathlineto{\pgfqpoint{3.846975in}{1.343474in}}%
\pgfpathlineto{\pgfqpoint{3.851515in}{1.343474in}}%
\pgfpathlineto{\pgfqpoint{3.851515in}{1.340525in}}%
\pgfpathmoveto{\pgfqpoint{3.851515in}{1.337576in}}%
\pgfpathlineto{\pgfqpoint{3.851515in}{1.337576in}}%
\pgfpathlineto{\pgfqpoint{3.851515in}{1.340525in}}%
\pgfpathlineto{\pgfqpoint{3.856056in}{1.340525in}}%
\pgfpathlineto{\pgfqpoint{3.856056in}{1.337576in}}%
\pgfpathmoveto{\pgfqpoint{3.856056in}{1.325779in}}%
\pgfpathlineto{\pgfqpoint{3.856056in}{1.325779in}}%
\pgfpathlineto{\pgfqpoint{3.856056in}{1.328729in}}%
\pgfpathlineto{\pgfqpoint{3.860597in}{1.328729in}}%
\pgfpathlineto{\pgfqpoint{3.860597in}{1.325779in}}%
\pgfpathmoveto{\pgfqpoint{3.856056in}{1.328729in}}%
\pgfpathlineto{\pgfqpoint{3.856056in}{1.328729in}}%
\pgfpathlineto{\pgfqpoint{3.856056in}{1.331678in}}%
\pgfpathlineto{\pgfqpoint{3.860597in}{1.331678in}}%
\pgfpathlineto{\pgfqpoint{3.860597in}{1.328729in}}%
\pgfpathmoveto{\pgfqpoint{3.860597in}{1.325779in}}%
\pgfpathlineto{\pgfqpoint{3.860597in}{1.325779in}}%
\pgfpathlineto{\pgfqpoint{3.860597in}{1.328729in}}%
\pgfpathlineto{\pgfqpoint{3.865138in}{1.328729in}}%
\pgfpathlineto{\pgfqpoint{3.865138in}{1.325779in}}%
\pgfpathmoveto{\pgfqpoint{3.860597in}{1.328729in}}%
\pgfpathlineto{\pgfqpoint{3.860597in}{1.328729in}}%
\pgfpathlineto{\pgfqpoint{3.860597in}{1.331678in}}%
\pgfpathlineto{\pgfqpoint{3.865138in}{1.331678in}}%
\pgfpathlineto{\pgfqpoint{3.865138in}{1.328729in}}%
\pgfpathmoveto{\pgfqpoint{3.856056in}{1.331678in}}%
\pgfpathlineto{\pgfqpoint{3.856056in}{1.331678in}}%
\pgfpathlineto{\pgfqpoint{3.856056in}{1.334627in}}%
\pgfpathlineto{\pgfqpoint{3.860597in}{1.334627in}}%
\pgfpathlineto{\pgfqpoint{3.860597in}{1.331678in}}%
\pgfpathmoveto{\pgfqpoint{3.865138in}{1.325779in}}%
\pgfpathlineto{\pgfqpoint{3.865138in}{1.325779in}}%
\pgfpathlineto{\pgfqpoint{3.865138in}{1.328729in}}%
\pgfpathlineto{\pgfqpoint{3.869679in}{1.328729in}}%
\pgfpathlineto{\pgfqpoint{3.869679in}{1.325779in}}%
\pgfpathmoveto{\pgfqpoint{3.901465in}{1.284491in}}%
\pgfpathlineto{\pgfqpoint{3.901465in}{1.284491in}}%
\pgfpathlineto{\pgfqpoint{3.901465in}{1.287440in}}%
\pgfpathlineto{\pgfqpoint{3.906006in}{1.287440in}}%
\pgfpathlineto{\pgfqpoint{3.906006in}{1.284491in}}%
\pgfpathmoveto{\pgfqpoint{3.901465in}{1.287440in}}%
\pgfpathlineto{\pgfqpoint{3.901465in}{1.287440in}}%
\pgfpathlineto{\pgfqpoint{3.901465in}{1.290389in}}%
\pgfpathlineto{\pgfqpoint{3.906006in}{1.290389in}}%
\pgfpathlineto{\pgfqpoint{3.906006in}{1.287440in}}%
\pgfpathmoveto{\pgfqpoint{3.906006in}{1.284491in}}%
\pgfpathlineto{\pgfqpoint{3.906006in}{1.284491in}}%
\pgfpathlineto{\pgfqpoint{3.906006in}{1.287440in}}%
\pgfpathlineto{\pgfqpoint{3.910546in}{1.287440in}}%
\pgfpathlineto{\pgfqpoint{3.910546in}{1.284491in}}%
\pgfpathmoveto{\pgfqpoint{3.906006in}{1.287440in}}%
\pgfpathlineto{\pgfqpoint{3.906006in}{1.287440in}}%
\pgfpathlineto{\pgfqpoint{3.906006in}{1.290389in}}%
\pgfpathlineto{\pgfqpoint{3.910546in}{1.290389in}}%
\pgfpathlineto{\pgfqpoint{3.910546in}{1.287440in}}%
\pgfpathmoveto{\pgfqpoint{3.892383in}{1.290389in}}%
\pgfpathlineto{\pgfqpoint{3.892383in}{1.290389in}}%
\pgfpathlineto{\pgfqpoint{3.892383in}{1.293339in}}%
\pgfpathlineto{\pgfqpoint{3.896924in}{1.293339in}}%
\pgfpathlineto{\pgfqpoint{3.896924in}{1.290389in}}%
\pgfpathmoveto{\pgfqpoint{3.892383in}{1.293339in}}%
\pgfpathlineto{\pgfqpoint{3.892383in}{1.293339in}}%
\pgfpathlineto{\pgfqpoint{3.892383in}{1.296288in}}%
\pgfpathlineto{\pgfqpoint{3.896924in}{1.296288in}}%
\pgfpathlineto{\pgfqpoint{3.896924in}{1.293339in}}%
\pgfpathmoveto{\pgfqpoint{3.896924in}{1.290389in}}%
\pgfpathlineto{\pgfqpoint{3.896924in}{1.290389in}}%
\pgfpathlineto{\pgfqpoint{3.896924in}{1.293339in}}%
\pgfpathlineto{\pgfqpoint{3.901465in}{1.293339in}}%
\pgfpathlineto{\pgfqpoint{3.901465in}{1.290389in}}%
\pgfpathmoveto{\pgfqpoint{3.896924in}{1.293339in}}%
\pgfpathlineto{\pgfqpoint{3.896924in}{1.293339in}}%
\pgfpathlineto{\pgfqpoint{3.896924in}{1.296288in}}%
\pgfpathlineto{\pgfqpoint{3.901465in}{1.296288in}}%
\pgfpathlineto{\pgfqpoint{3.901465in}{1.293339in}}%
\pgfpathmoveto{\pgfqpoint{3.892383in}{1.296288in}}%
\pgfpathlineto{\pgfqpoint{3.892383in}{1.296288in}}%
\pgfpathlineto{\pgfqpoint{3.892383in}{1.299237in}}%
\pgfpathlineto{\pgfqpoint{3.896924in}{1.299237in}}%
\pgfpathlineto{\pgfqpoint{3.896924in}{1.296288in}}%
\pgfpathmoveto{\pgfqpoint{3.892383in}{1.299237in}}%
\pgfpathlineto{\pgfqpoint{3.892383in}{1.299237in}}%
\pgfpathlineto{\pgfqpoint{3.892383in}{1.302186in}}%
\pgfpathlineto{\pgfqpoint{3.896924in}{1.302186in}}%
\pgfpathlineto{\pgfqpoint{3.896924in}{1.299237in}}%
\pgfpathmoveto{\pgfqpoint{3.896924in}{1.296288in}}%
\pgfpathlineto{\pgfqpoint{3.896924in}{1.296288in}}%
\pgfpathlineto{\pgfqpoint{3.896924in}{1.299237in}}%
\pgfpathlineto{\pgfqpoint{3.901465in}{1.299237in}}%
\pgfpathlineto{\pgfqpoint{3.901465in}{1.296288in}}%
\pgfpathmoveto{\pgfqpoint{3.901465in}{1.290389in}}%
\pgfpathlineto{\pgfqpoint{3.901465in}{1.290389in}}%
\pgfpathlineto{\pgfqpoint{3.901465in}{1.293339in}}%
\pgfpathlineto{\pgfqpoint{3.906006in}{1.293339in}}%
\pgfpathlineto{\pgfqpoint{3.906006in}{1.290389in}}%
\pgfpathmoveto{\pgfqpoint{3.901465in}{1.293339in}}%
\pgfpathlineto{\pgfqpoint{3.901465in}{1.293339in}}%
\pgfpathlineto{\pgfqpoint{3.901465in}{1.296288in}}%
\pgfpathlineto{\pgfqpoint{3.906006in}{1.296288in}}%
\pgfpathlineto{\pgfqpoint{3.906006in}{1.293339in}}%
\pgfpathmoveto{\pgfqpoint{3.906006in}{1.290389in}}%
\pgfpathlineto{\pgfqpoint{3.906006in}{1.290389in}}%
\pgfpathlineto{\pgfqpoint{3.906006in}{1.293339in}}%
\pgfpathlineto{\pgfqpoint{3.910546in}{1.293339in}}%
\pgfpathlineto{\pgfqpoint{3.910546in}{1.290389in}}%
\pgfpathmoveto{\pgfqpoint{3.919628in}{1.266796in}}%
\pgfpathlineto{\pgfqpoint{3.919628in}{1.266796in}}%
\pgfpathlineto{\pgfqpoint{3.919628in}{1.269745in}}%
\pgfpathlineto{\pgfqpoint{3.924169in}{1.269745in}}%
\pgfpathlineto{\pgfqpoint{3.924169in}{1.266796in}}%
\pgfpathmoveto{\pgfqpoint{3.919628in}{1.269745in}}%
\pgfpathlineto{\pgfqpoint{3.919628in}{1.269745in}}%
\pgfpathlineto{\pgfqpoint{3.919628in}{1.272694in}}%
\pgfpathlineto{\pgfqpoint{3.924169in}{1.272694in}}%
\pgfpathlineto{\pgfqpoint{3.924169in}{1.269745in}}%
\pgfpathmoveto{\pgfqpoint{3.924169in}{1.266796in}}%
\pgfpathlineto{\pgfqpoint{3.924169in}{1.266796in}}%
\pgfpathlineto{\pgfqpoint{3.924169in}{1.269745in}}%
\pgfpathlineto{\pgfqpoint{3.928710in}{1.269745in}}%
\pgfpathlineto{\pgfqpoint{3.928710in}{1.266796in}}%
\pgfpathmoveto{\pgfqpoint{3.924169in}{1.269745in}}%
\pgfpathlineto{\pgfqpoint{3.924169in}{1.269745in}}%
\pgfpathlineto{\pgfqpoint{3.924169in}{1.272694in}}%
\pgfpathlineto{\pgfqpoint{3.928710in}{1.272694in}}%
\pgfpathlineto{\pgfqpoint{3.928710in}{1.269745in}}%
\pgfpathmoveto{\pgfqpoint{3.919628in}{1.272694in}}%
\pgfpathlineto{\pgfqpoint{3.919628in}{1.272694in}}%
\pgfpathlineto{\pgfqpoint{3.919628in}{1.275644in}}%
\pgfpathlineto{\pgfqpoint{3.924169in}{1.275644in}}%
\pgfpathlineto{\pgfqpoint{3.924169in}{1.272694in}}%
\pgfpathmoveto{\pgfqpoint{3.919628in}{1.275644in}}%
\pgfpathlineto{\pgfqpoint{3.919628in}{1.275644in}}%
\pgfpathlineto{\pgfqpoint{3.919628in}{1.278593in}}%
\pgfpathlineto{\pgfqpoint{3.924169in}{1.278593in}}%
\pgfpathlineto{\pgfqpoint{3.924169in}{1.275644in}}%
\pgfpathmoveto{\pgfqpoint{3.924169in}{1.272694in}}%
\pgfpathlineto{\pgfqpoint{3.924169in}{1.272694in}}%
\pgfpathlineto{\pgfqpoint{3.924169in}{1.275644in}}%
\pgfpathlineto{\pgfqpoint{3.928710in}{1.275644in}}%
\pgfpathlineto{\pgfqpoint{3.928710in}{1.272694in}}%
\pgfpathmoveto{\pgfqpoint{3.928710in}{1.260898in}}%
\pgfpathlineto{\pgfqpoint{3.928710in}{1.260898in}}%
\pgfpathlineto{\pgfqpoint{3.928710in}{1.263847in}}%
\pgfpathlineto{\pgfqpoint{3.933251in}{1.263847in}}%
\pgfpathlineto{\pgfqpoint{3.933251in}{1.260898in}}%
\pgfpathmoveto{\pgfqpoint{3.928710in}{1.263847in}}%
\pgfpathlineto{\pgfqpoint{3.928710in}{1.263847in}}%
\pgfpathlineto{\pgfqpoint{3.928710in}{1.266796in}}%
\pgfpathlineto{\pgfqpoint{3.933251in}{1.266796in}}%
\pgfpathlineto{\pgfqpoint{3.933251in}{1.263847in}}%
\pgfpathmoveto{\pgfqpoint{3.933251in}{1.260898in}}%
\pgfpathlineto{\pgfqpoint{3.933251in}{1.260898in}}%
\pgfpathlineto{\pgfqpoint{3.933251in}{1.263847in}}%
\pgfpathlineto{\pgfqpoint{3.937791in}{1.263847in}}%
\pgfpathlineto{\pgfqpoint{3.937791in}{1.260898in}}%
\pgfpathmoveto{\pgfqpoint{3.933251in}{1.263847in}}%
\pgfpathlineto{\pgfqpoint{3.933251in}{1.263847in}}%
\pgfpathlineto{\pgfqpoint{3.933251in}{1.266796in}}%
\pgfpathlineto{\pgfqpoint{3.937791in}{1.266796in}}%
\pgfpathlineto{\pgfqpoint{3.937791in}{1.263847in}}%
\pgfpathmoveto{\pgfqpoint{3.937791in}{1.254999in}}%
\pgfpathlineto{\pgfqpoint{3.937791in}{1.254999in}}%
\pgfpathlineto{\pgfqpoint{3.937791in}{1.257949in}}%
\pgfpathlineto{\pgfqpoint{3.942332in}{1.257949in}}%
\pgfpathlineto{\pgfqpoint{3.942332in}{1.254999in}}%
\pgfpathmoveto{\pgfqpoint{3.937791in}{1.257949in}}%
\pgfpathlineto{\pgfqpoint{3.937791in}{1.257949in}}%
\pgfpathlineto{\pgfqpoint{3.937791in}{1.260898in}}%
\pgfpathlineto{\pgfqpoint{3.942332in}{1.260898in}}%
\pgfpathlineto{\pgfqpoint{3.942332in}{1.257949in}}%
\pgfpathmoveto{\pgfqpoint{3.942332in}{1.254999in}}%
\pgfpathlineto{\pgfqpoint{3.942332in}{1.254999in}}%
\pgfpathlineto{\pgfqpoint{3.942332in}{1.257949in}}%
\pgfpathlineto{\pgfqpoint{3.946873in}{1.257949in}}%
\pgfpathlineto{\pgfqpoint{3.946873in}{1.254999in}}%
\pgfpathmoveto{\pgfqpoint{3.942332in}{1.257949in}}%
\pgfpathlineto{\pgfqpoint{3.942332in}{1.257949in}}%
\pgfpathlineto{\pgfqpoint{3.942332in}{1.260898in}}%
\pgfpathlineto{\pgfqpoint{3.946873in}{1.260898in}}%
\pgfpathlineto{\pgfqpoint{3.946873in}{1.257949in}}%
\pgfpathmoveto{\pgfqpoint{3.937791in}{1.260898in}}%
\pgfpathlineto{\pgfqpoint{3.937791in}{1.260898in}}%
\pgfpathlineto{\pgfqpoint{3.937791in}{1.263847in}}%
\pgfpathlineto{\pgfqpoint{3.942332in}{1.263847in}}%
\pgfpathlineto{\pgfqpoint{3.942332in}{1.260898in}}%
\pgfpathmoveto{\pgfqpoint{3.928710in}{1.266796in}}%
\pgfpathlineto{\pgfqpoint{3.928710in}{1.266796in}}%
\pgfpathlineto{\pgfqpoint{3.928710in}{1.269745in}}%
\pgfpathlineto{\pgfqpoint{3.933251in}{1.269745in}}%
\pgfpathlineto{\pgfqpoint{3.933251in}{1.266796in}}%
\pgfpathmoveto{\pgfqpoint{3.928710in}{1.269745in}}%
\pgfpathlineto{\pgfqpoint{3.928710in}{1.269745in}}%
\pgfpathlineto{\pgfqpoint{3.928710in}{1.272694in}}%
\pgfpathlineto{\pgfqpoint{3.933251in}{1.272694in}}%
\pgfpathlineto{\pgfqpoint{3.933251in}{1.269745in}}%
\pgfpathmoveto{\pgfqpoint{3.933251in}{1.266796in}}%
\pgfpathlineto{\pgfqpoint{3.933251in}{1.266796in}}%
\pgfpathlineto{\pgfqpoint{3.933251in}{1.269745in}}%
\pgfpathlineto{\pgfqpoint{3.937791in}{1.269745in}}%
\pgfpathlineto{\pgfqpoint{3.937791in}{1.266796in}}%
\pgfpathmoveto{\pgfqpoint{3.910546in}{1.278593in}}%
\pgfpathlineto{\pgfqpoint{3.910546in}{1.278593in}}%
\pgfpathlineto{\pgfqpoint{3.910546in}{1.281542in}}%
\pgfpathlineto{\pgfqpoint{3.915087in}{1.281542in}}%
\pgfpathlineto{\pgfqpoint{3.915087in}{1.278593in}}%
\pgfpathmoveto{\pgfqpoint{3.910546in}{1.281542in}}%
\pgfpathlineto{\pgfqpoint{3.910546in}{1.281542in}}%
\pgfpathlineto{\pgfqpoint{3.910546in}{1.284491in}}%
\pgfpathlineto{\pgfqpoint{3.915087in}{1.284491in}}%
\pgfpathlineto{\pgfqpoint{3.915087in}{1.281542in}}%
\pgfpathmoveto{\pgfqpoint{3.915087in}{1.278593in}}%
\pgfpathlineto{\pgfqpoint{3.915087in}{1.278593in}}%
\pgfpathlineto{\pgfqpoint{3.915087in}{1.281542in}}%
\pgfpathlineto{\pgfqpoint{3.919628in}{1.281542in}}%
\pgfpathlineto{\pgfqpoint{3.919628in}{1.278593in}}%
\pgfpathmoveto{\pgfqpoint{3.915087in}{1.281542in}}%
\pgfpathlineto{\pgfqpoint{3.915087in}{1.281542in}}%
\pgfpathlineto{\pgfqpoint{3.915087in}{1.284491in}}%
\pgfpathlineto{\pgfqpoint{3.919628in}{1.284491in}}%
\pgfpathlineto{\pgfqpoint{3.919628in}{1.281542in}}%
\pgfpathmoveto{\pgfqpoint{3.910546in}{1.284491in}}%
\pgfpathlineto{\pgfqpoint{3.910546in}{1.284491in}}%
\pgfpathlineto{\pgfqpoint{3.910546in}{1.287440in}}%
\pgfpathlineto{\pgfqpoint{3.915087in}{1.287440in}}%
\pgfpathlineto{\pgfqpoint{3.915087in}{1.284491in}}%
\pgfpathmoveto{\pgfqpoint{3.919628in}{1.278593in}}%
\pgfpathlineto{\pgfqpoint{3.919628in}{1.278593in}}%
\pgfpathlineto{\pgfqpoint{3.919628in}{1.281542in}}%
\pgfpathlineto{\pgfqpoint{3.924169in}{1.281542in}}%
\pgfpathlineto{\pgfqpoint{3.924169in}{1.278593in}}%
\pgfpathmoveto{\pgfqpoint{3.874220in}{1.308084in}}%
\pgfpathlineto{\pgfqpoint{3.874220in}{1.308084in}}%
\pgfpathlineto{\pgfqpoint{3.874220in}{1.311034in}}%
\pgfpathlineto{\pgfqpoint{3.878760in}{1.311034in}}%
\pgfpathlineto{\pgfqpoint{3.878760in}{1.308084in}}%
\pgfpathmoveto{\pgfqpoint{3.874220in}{1.311034in}}%
\pgfpathlineto{\pgfqpoint{3.874220in}{1.311034in}}%
\pgfpathlineto{\pgfqpoint{3.874220in}{1.313983in}}%
\pgfpathlineto{\pgfqpoint{3.878760in}{1.313983in}}%
\pgfpathlineto{\pgfqpoint{3.878760in}{1.311034in}}%
\pgfpathmoveto{\pgfqpoint{3.878760in}{1.308084in}}%
\pgfpathlineto{\pgfqpoint{3.878760in}{1.308084in}}%
\pgfpathlineto{\pgfqpoint{3.878760in}{1.311034in}}%
\pgfpathlineto{\pgfqpoint{3.883301in}{1.311034in}}%
\pgfpathlineto{\pgfqpoint{3.883301in}{1.308084in}}%
\pgfpathmoveto{\pgfqpoint{3.878760in}{1.311034in}}%
\pgfpathlineto{\pgfqpoint{3.878760in}{1.311034in}}%
\pgfpathlineto{\pgfqpoint{3.878760in}{1.313983in}}%
\pgfpathlineto{\pgfqpoint{3.883301in}{1.313983in}}%
\pgfpathlineto{\pgfqpoint{3.883301in}{1.311034in}}%
\pgfpathmoveto{\pgfqpoint{3.883301in}{1.302186in}}%
\pgfpathlineto{\pgfqpoint{3.883301in}{1.302186in}}%
\pgfpathlineto{\pgfqpoint{3.883301in}{1.305135in}}%
\pgfpathlineto{\pgfqpoint{3.887842in}{1.305135in}}%
\pgfpathlineto{\pgfqpoint{3.887842in}{1.302186in}}%
\pgfpathmoveto{\pgfqpoint{3.883301in}{1.305135in}}%
\pgfpathlineto{\pgfqpoint{3.883301in}{1.305135in}}%
\pgfpathlineto{\pgfqpoint{3.883301in}{1.308084in}}%
\pgfpathlineto{\pgfqpoint{3.887842in}{1.308084in}}%
\pgfpathlineto{\pgfqpoint{3.887842in}{1.305135in}}%
\pgfpathmoveto{\pgfqpoint{3.887842in}{1.302186in}}%
\pgfpathlineto{\pgfqpoint{3.887842in}{1.302186in}}%
\pgfpathlineto{\pgfqpoint{3.887842in}{1.305135in}}%
\pgfpathlineto{\pgfqpoint{3.892383in}{1.305135in}}%
\pgfpathlineto{\pgfqpoint{3.892383in}{1.302186in}}%
\pgfpathmoveto{\pgfqpoint{3.887842in}{1.305135in}}%
\pgfpathlineto{\pgfqpoint{3.887842in}{1.305135in}}%
\pgfpathlineto{\pgfqpoint{3.887842in}{1.308084in}}%
\pgfpathlineto{\pgfqpoint{3.892383in}{1.308084in}}%
\pgfpathlineto{\pgfqpoint{3.892383in}{1.305135in}}%
\pgfpathmoveto{\pgfqpoint{3.883301in}{1.308084in}}%
\pgfpathlineto{\pgfqpoint{3.883301in}{1.308084in}}%
\pgfpathlineto{\pgfqpoint{3.883301in}{1.311034in}}%
\pgfpathlineto{\pgfqpoint{3.887842in}{1.311034in}}%
\pgfpathlineto{\pgfqpoint{3.887842in}{1.308084in}}%
\pgfpathmoveto{\pgfqpoint{3.874220in}{1.313983in}}%
\pgfpathlineto{\pgfqpoint{3.874220in}{1.313983in}}%
\pgfpathlineto{\pgfqpoint{3.874220in}{1.316932in}}%
\pgfpathlineto{\pgfqpoint{3.878760in}{1.316932in}}%
\pgfpathlineto{\pgfqpoint{3.878760in}{1.313983in}}%
\pgfpathmoveto{\pgfqpoint{3.874220in}{1.316932in}}%
\pgfpathlineto{\pgfqpoint{3.874220in}{1.316932in}}%
\pgfpathlineto{\pgfqpoint{3.874220in}{1.319881in}}%
\pgfpathlineto{\pgfqpoint{3.878760in}{1.319881in}}%
\pgfpathlineto{\pgfqpoint{3.878760in}{1.316932in}}%
\pgfpathmoveto{\pgfqpoint{3.878760in}{1.313983in}}%
\pgfpathlineto{\pgfqpoint{3.878760in}{1.313983in}}%
\pgfpathlineto{\pgfqpoint{3.878760in}{1.316932in}}%
\pgfpathlineto{\pgfqpoint{3.883301in}{1.316932in}}%
\pgfpathlineto{\pgfqpoint{3.883301in}{1.313983in}}%
\pgfpathmoveto{\pgfqpoint{3.892383in}{1.302186in}}%
\pgfpathlineto{\pgfqpoint{3.892383in}{1.302186in}}%
\pgfpathlineto{\pgfqpoint{3.892383in}{1.305135in}}%
\pgfpathlineto{\pgfqpoint{3.896924in}{1.305135in}}%
\pgfpathlineto{\pgfqpoint{3.896924in}{1.302186in}}%
\pgfpathmoveto{\pgfqpoint{3.801566in}{1.367069in}}%
\pgfpathlineto{\pgfqpoint{3.801566in}{1.367069in}}%
\pgfpathlineto{\pgfqpoint{3.801566in}{1.370018in}}%
\pgfpathlineto{\pgfqpoint{3.806107in}{1.370018in}}%
\pgfpathlineto{\pgfqpoint{3.806107in}{1.367069in}}%
\pgfpathmoveto{\pgfqpoint{3.801566in}{1.370018in}}%
\pgfpathlineto{\pgfqpoint{3.801566in}{1.370018in}}%
\pgfpathlineto{\pgfqpoint{3.801566in}{1.372968in}}%
\pgfpathlineto{\pgfqpoint{3.806107in}{1.372968in}}%
\pgfpathlineto{\pgfqpoint{3.806107in}{1.370018in}}%
\pgfpathmoveto{\pgfqpoint{3.806107in}{1.367069in}}%
\pgfpathlineto{\pgfqpoint{3.806107in}{1.367069in}}%
\pgfpathlineto{\pgfqpoint{3.806107in}{1.370018in}}%
\pgfpathlineto{\pgfqpoint{3.810648in}{1.370018in}}%
\pgfpathlineto{\pgfqpoint{3.810648in}{1.367069in}}%
\pgfpathmoveto{\pgfqpoint{3.806107in}{1.370018in}}%
\pgfpathlineto{\pgfqpoint{3.806107in}{1.370018in}}%
\pgfpathlineto{\pgfqpoint{3.806107in}{1.372968in}}%
\pgfpathlineto{\pgfqpoint{3.810648in}{1.372968in}}%
\pgfpathlineto{\pgfqpoint{3.810648in}{1.370018in}}%
\pgfpathmoveto{\pgfqpoint{3.810648in}{1.361170in}}%
\pgfpathlineto{\pgfqpoint{3.810648in}{1.361170in}}%
\pgfpathlineto{\pgfqpoint{3.810648in}{1.364119in}}%
\pgfpathlineto{\pgfqpoint{3.815189in}{1.364119in}}%
\pgfpathlineto{\pgfqpoint{3.815189in}{1.361170in}}%
\pgfpathmoveto{\pgfqpoint{3.810648in}{1.364119in}}%
\pgfpathlineto{\pgfqpoint{3.810648in}{1.364119in}}%
\pgfpathlineto{\pgfqpoint{3.810648in}{1.367069in}}%
\pgfpathlineto{\pgfqpoint{3.815189in}{1.367069in}}%
\pgfpathlineto{\pgfqpoint{3.815189in}{1.364119in}}%
\pgfpathmoveto{\pgfqpoint{3.815189in}{1.361170in}}%
\pgfpathlineto{\pgfqpoint{3.815189in}{1.361170in}}%
\pgfpathlineto{\pgfqpoint{3.815189in}{1.364119in}}%
\pgfpathlineto{\pgfqpoint{3.819730in}{1.364119in}}%
\pgfpathlineto{\pgfqpoint{3.819730in}{1.361170in}}%
\pgfpathmoveto{\pgfqpoint{3.815189in}{1.364119in}}%
\pgfpathlineto{\pgfqpoint{3.815189in}{1.364119in}}%
\pgfpathlineto{\pgfqpoint{3.815189in}{1.367069in}}%
\pgfpathlineto{\pgfqpoint{3.819730in}{1.367069in}}%
\pgfpathlineto{\pgfqpoint{3.819730in}{1.364119in}}%
\pgfpathmoveto{\pgfqpoint{3.810648in}{1.367069in}}%
\pgfpathlineto{\pgfqpoint{3.810648in}{1.367069in}}%
\pgfpathlineto{\pgfqpoint{3.810648in}{1.370018in}}%
\pgfpathlineto{\pgfqpoint{3.815189in}{1.370018in}}%
\pgfpathlineto{\pgfqpoint{3.815189in}{1.367069in}}%
\pgfpathmoveto{\pgfqpoint{3.810648in}{1.370018in}}%
\pgfpathlineto{\pgfqpoint{3.810648in}{1.370018in}}%
\pgfpathlineto{\pgfqpoint{3.810648in}{1.372968in}}%
\pgfpathlineto{\pgfqpoint{3.815189in}{1.372968in}}%
\pgfpathlineto{\pgfqpoint{3.815189in}{1.370018in}}%
\pgfpathmoveto{\pgfqpoint{3.815189in}{1.367069in}}%
\pgfpathlineto{\pgfqpoint{3.815189in}{1.367069in}}%
\pgfpathlineto{\pgfqpoint{3.815189in}{1.370018in}}%
\pgfpathlineto{\pgfqpoint{3.819730in}{1.370018in}}%
\pgfpathlineto{\pgfqpoint{3.819730in}{1.367069in}}%
\pgfpathmoveto{\pgfqpoint{3.819730in}{1.355271in}}%
\pgfpathlineto{\pgfqpoint{3.819730in}{1.355271in}}%
\pgfpathlineto{\pgfqpoint{3.819730in}{1.358221in}}%
\pgfpathlineto{\pgfqpoint{3.824270in}{1.358221in}}%
\pgfpathlineto{\pgfqpoint{3.824270in}{1.355271in}}%
\pgfpathmoveto{\pgfqpoint{3.819730in}{1.358221in}}%
\pgfpathlineto{\pgfqpoint{3.819730in}{1.358221in}}%
\pgfpathlineto{\pgfqpoint{3.819730in}{1.361170in}}%
\pgfpathlineto{\pgfqpoint{3.824270in}{1.361170in}}%
\pgfpathlineto{\pgfqpoint{3.824270in}{1.358221in}}%
\pgfpathmoveto{\pgfqpoint{3.824270in}{1.355271in}}%
\pgfpathlineto{\pgfqpoint{3.824270in}{1.355271in}}%
\pgfpathlineto{\pgfqpoint{3.824270in}{1.358221in}}%
\pgfpathlineto{\pgfqpoint{3.828811in}{1.358221in}}%
\pgfpathlineto{\pgfqpoint{3.828811in}{1.355271in}}%
\pgfpathmoveto{\pgfqpoint{3.824270in}{1.358221in}}%
\pgfpathlineto{\pgfqpoint{3.824270in}{1.358221in}}%
\pgfpathlineto{\pgfqpoint{3.824270in}{1.361170in}}%
\pgfpathlineto{\pgfqpoint{3.828811in}{1.361170in}}%
\pgfpathlineto{\pgfqpoint{3.828811in}{1.358221in}}%
\pgfpathmoveto{\pgfqpoint{3.828811in}{1.349373in}}%
\pgfpathlineto{\pgfqpoint{3.828811in}{1.349373in}}%
\pgfpathlineto{\pgfqpoint{3.828811in}{1.352322in}}%
\pgfpathlineto{\pgfqpoint{3.833352in}{1.352322in}}%
\pgfpathlineto{\pgfqpoint{3.833352in}{1.349373in}}%
\pgfpathmoveto{\pgfqpoint{3.828811in}{1.352322in}}%
\pgfpathlineto{\pgfqpoint{3.828811in}{1.352322in}}%
\pgfpathlineto{\pgfqpoint{3.828811in}{1.355271in}}%
\pgfpathlineto{\pgfqpoint{3.833352in}{1.355271in}}%
\pgfpathlineto{\pgfqpoint{3.833352in}{1.352322in}}%
\pgfpathmoveto{\pgfqpoint{3.833352in}{1.349373in}}%
\pgfpathlineto{\pgfqpoint{3.833352in}{1.349373in}}%
\pgfpathlineto{\pgfqpoint{3.833352in}{1.352322in}}%
\pgfpathlineto{\pgfqpoint{3.837893in}{1.352322in}}%
\pgfpathlineto{\pgfqpoint{3.837893in}{1.349373in}}%
\pgfpathmoveto{\pgfqpoint{3.833352in}{1.352322in}}%
\pgfpathlineto{\pgfqpoint{3.833352in}{1.352322in}}%
\pgfpathlineto{\pgfqpoint{3.833352in}{1.355271in}}%
\pgfpathlineto{\pgfqpoint{3.837893in}{1.355271in}}%
\pgfpathlineto{\pgfqpoint{3.837893in}{1.352322in}}%
\pgfpathmoveto{\pgfqpoint{3.828811in}{1.355271in}}%
\pgfpathlineto{\pgfqpoint{3.828811in}{1.355271in}}%
\pgfpathlineto{\pgfqpoint{3.828811in}{1.358221in}}%
\pgfpathlineto{\pgfqpoint{3.833352in}{1.358221in}}%
\pgfpathlineto{\pgfqpoint{3.833352in}{1.355271in}}%
\pgfpathmoveto{\pgfqpoint{3.819730in}{1.361170in}}%
\pgfpathlineto{\pgfqpoint{3.819730in}{1.361170in}}%
\pgfpathlineto{\pgfqpoint{3.819730in}{1.364119in}}%
\pgfpathlineto{\pgfqpoint{3.824270in}{1.364119in}}%
\pgfpathlineto{\pgfqpoint{3.824270in}{1.361170in}}%
\pgfpathmoveto{\pgfqpoint{3.819730in}{1.364119in}}%
\pgfpathlineto{\pgfqpoint{3.819730in}{1.364119in}}%
\pgfpathlineto{\pgfqpoint{3.819730in}{1.367069in}}%
\pgfpathlineto{\pgfqpoint{3.824270in}{1.367069in}}%
\pgfpathlineto{\pgfqpoint{3.824270in}{1.364119in}}%
\pgfpathmoveto{\pgfqpoint{3.801566in}{1.372968in}}%
\pgfpathlineto{\pgfqpoint{3.801566in}{1.372968in}}%
\pgfpathlineto{\pgfqpoint{3.801566in}{1.375917in}}%
\pgfpathlineto{\pgfqpoint{3.806107in}{1.375917in}}%
\pgfpathlineto{\pgfqpoint{3.806107in}{1.372968in}}%
\pgfpathmoveto{\pgfqpoint{3.801566in}{1.375917in}}%
\pgfpathlineto{\pgfqpoint{3.801566in}{1.375917in}}%
\pgfpathlineto{\pgfqpoint{3.801566in}{1.378866in}}%
\pgfpathlineto{\pgfqpoint{3.806107in}{1.378866in}}%
\pgfpathlineto{\pgfqpoint{3.806107in}{1.375917in}}%
\pgfpathmoveto{\pgfqpoint{3.806107in}{1.372968in}}%
\pgfpathlineto{\pgfqpoint{3.806107in}{1.372968in}}%
\pgfpathlineto{\pgfqpoint{3.806107in}{1.375917in}}%
\pgfpathlineto{\pgfqpoint{3.810648in}{1.375917in}}%
\pgfpathlineto{\pgfqpoint{3.810648in}{1.372968in}}%
\pgfpathmoveto{\pgfqpoint{3.806107in}{1.375917in}}%
\pgfpathlineto{\pgfqpoint{3.806107in}{1.375917in}}%
\pgfpathlineto{\pgfqpoint{3.806107in}{1.378866in}}%
\pgfpathlineto{\pgfqpoint{3.810648in}{1.378866in}}%
\pgfpathlineto{\pgfqpoint{3.810648in}{1.375917in}}%
\pgfpathmoveto{\pgfqpoint{3.801566in}{1.378866in}}%
\pgfpathlineto{\pgfqpoint{3.801566in}{1.378866in}}%
\pgfpathlineto{\pgfqpoint{3.801566in}{1.381816in}}%
\pgfpathlineto{\pgfqpoint{3.806107in}{1.381816in}}%
\pgfpathlineto{\pgfqpoint{3.806107in}{1.378866in}}%
\pgfpathmoveto{\pgfqpoint{3.837893in}{1.349373in}}%
\pgfpathlineto{\pgfqpoint{3.837893in}{1.349373in}}%
\pgfpathlineto{\pgfqpoint{3.837893in}{1.352322in}}%
\pgfpathlineto{\pgfqpoint{3.842434in}{1.352322in}}%
\pgfpathlineto{\pgfqpoint{3.842434in}{1.349373in}}%
\pgfpathmoveto{\pgfqpoint{3.946873in}{0.874552in}}%
\pgfpathlineto{\pgfqpoint{3.946873in}{0.874552in}}%
\pgfpathlineto{\pgfqpoint{3.946873in}{0.877501in}}%
\pgfpathlineto{\pgfqpoint{3.951414in}{0.877501in}}%
\pgfpathlineto{\pgfqpoint{3.951414in}{0.874552in}}%
\pgfpathmoveto{\pgfqpoint{3.946873in}{0.877501in}}%
\pgfpathlineto{\pgfqpoint{3.946873in}{0.877501in}}%
\pgfpathlineto{\pgfqpoint{3.946873in}{0.880450in}}%
\pgfpathlineto{\pgfqpoint{3.951414in}{0.880450in}}%
\pgfpathlineto{\pgfqpoint{3.951414in}{0.877501in}}%
\pgfpathmoveto{\pgfqpoint{3.946873in}{0.880450in}}%
\pgfpathlineto{\pgfqpoint{3.946873in}{0.880450in}}%
\pgfpathlineto{\pgfqpoint{3.946873in}{0.883399in}}%
\pgfpathlineto{\pgfqpoint{3.951414in}{0.883399in}}%
\pgfpathlineto{\pgfqpoint{3.951414in}{0.880450in}}%
\pgfpathmoveto{\pgfqpoint{3.951414in}{0.880450in}}%
\pgfpathlineto{\pgfqpoint{3.951414in}{0.880450in}}%
\pgfpathlineto{\pgfqpoint{3.951414in}{0.883399in}}%
\pgfpathlineto{\pgfqpoint{3.955955in}{0.883399in}}%
\pgfpathlineto{\pgfqpoint{3.955955in}{0.880450in}}%
\pgfpathmoveto{\pgfqpoint{3.946873in}{0.883399in}}%
\pgfpathlineto{\pgfqpoint{3.946873in}{0.883399in}}%
\pgfpathlineto{\pgfqpoint{3.946873in}{0.886348in}}%
\pgfpathlineto{\pgfqpoint{3.951414in}{0.886348in}}%
\pgfpathlineto{\pgfqpoint{3.951414in}{0.883399in}}%
\pgfpathmoveto{\pgfqpoint{3.946873in}{0.886348in}}%
\pgfpathlineto{\pgfqpoint{3.946873in}{0.886348in}}%
\pgfpathlineto{\pgfqpoint{3.946873in}{0.889297in}}%
\pgfpathlineto{\pgfqpoint{3.951414in}{0.889297in}}%
\pgfpathlineto{\pgfqpoint{3.951414in}{0.886348in}}%
\pgfpathmoveto{\pgfqpoint{3.951414in}{0.883399in}}%
\pgfpathlineto{\pgfqpoint{3.951414in}{0.883399in}}%
\pgfpathlineto{\pgfqpoint{3.951414in}{0.886348in}}%
\pgfpathlineto{\pgfqpoint{3.955955in}{0.886348in}}%
\pgfpathlineto{\pgfqpoint{3.955955in}{0.883399in}}%
\pgfpathmoveto{\pgfqpoint{3.951414in}{0.886348in}}%
\pgfpathlineto{\pgfqpoint{3.951414in}{0.886348in}}%
\pgfpathlineto{\pgfqpoint{3.951414in}{0.889297in}}%
\pgfpathlineto{\pgfqpoint{3.955955in}{0.889297in}}%
\pgfpathlineto{\pgfqpoint{3.955955in}{0.886348in}}%
\pgfpathmoveto{\pgfqpoint{3.955955in}{0.883399in}}%
\pgfpathlineto{\pgfqpoint{3.955955in}{0.883399in}}%
\pgfpathlineto{\pgfqpoint{3.955955in}{0.886348in}}%
\pgfpathlineto{\pgfqpoint{3.960496in}{0.886348in}}%
\pgfpathlineto{\pgfqpoint{3.960496in}{0.883399in}}%
\pgfpathmoveto{\pgfqpoint{3.955955in}{0.886348in}}%
\pgfpathlineto{\pgfqpoint{3.955955in}{0.886348in}}%
\pgfpathlineto{\pgfqpoint{3.955955in}{0.889297in}}%
\pgfpathlineto{\pgfqpoint{3.960496in}{0.889297in}}%
\pgfpathlineto{\pgfqpoint{3.960496in}{0.886348in}}%
\pgfpathmoveto{\pgfqpoint{3.960496in}{0.886348in}}%
\pgfpathlineto{\pgfqpoint{3.960496in}{0.886348in}}%
\pgfpathlineto{\pgfqpoint{3.960496in}{0.889297in}}%
\pgfpathlineto{\pgfqpoint{3.965037in}{0.889297in}}%
\pgfpathlineto{\pgfqpoint{3.965037in}{0.886348in}}%
\pgfpathmoveto{\pgfqpoint{3.955955in}{0.889297in}}%
\pgfpathlineto{\pgfqpoint{3.955955in}{0.889297in}}%
\pgfpathlineto{\pgfqpoint{3.955955in}{0.892247in}}%
\pgfpathlineto{\pgfqpoint{3.960496in}{0.892247in}}%
\pgfpathlineto{\pgfqpoint{3.960496in}{0.889297in}}%
\pgfpathmoveto{\pgfqpoint{3.955955in}{0.892247in}}%
\pgfpathlineto{\pgfqpoint{3.955955in}{0.892247in}}%
\pgfpathlineto{\pgfqpoint{3.955955in}{0.895196in}}%
\pgfpathlineto{\pgfqpoint{3.960496in}{0.895196in}}%
\pgfpathlineto{\pgfqpoint{3.960496in}{0.892247in}}%
\pgfpathmoveto{\pgfqpoint{3.960496in}{0.889297in}}%
\pgfpathlineto{\pgfqpoint{3.960496in}{0.889297in}}%
\pgfpathlineto{\pgfqpoint{3.960496in}{0.892247in}}%
\pgfpathlineto{\pgfqpoint{3.965037in}{0.892247in}}%
\pgfpathlineto{\pgfqpoint{3.965037in}{0.889297in}}%
\pgfpathmoveto{\pgfqpoint{3.960496in}{0.892247in}}%
\pgfpathlineto{\pgfqpoint{3.960496in}{0.892247in}}%
\pgfpathlineto{\pgfqpoint{3.960496in}{0.895196in}}%
\pgfpathlineto{\pgfqpoint{3.965037in}{0.895196in}}%
\pgfpathlineto{\pgfqpoint{3.965037in}{0.892247in}}%
\pgfpathmoveto{\pgfqpoint{3.965037in}{0.889297in}}%
\pgfpathlineto{\pgfqpoint{3.965037in}{0.889297in}}%
\pgfpathlineto{\pgfqpoint{3.965037in}{0.892247in}}%
\pgfpathlineto{\pgfqpoint{3.969578in}{0.892247in}}%
\pgfpathlineto{\pgfqpoint{3.969578in}{0.889297in}}%
\pgfpathmoveto{\pgfqpoint{3.965037in}{0.892247in}}%
\pgfpathlineto{\pgfqpoint{3.965037in}{0.892247in}}%
\pgfpathlineto{\pgfqpoint{3.965037in}{0.895196in}}%
\pgfpathlineto{\pgfqpoint{3.969578in}{0.895196in}}%
\pgfpathlineto{\pgfqpoint{3.969578in}{0.892247in}}%
\pgfpathmoveto{\pgfqpoint{3.969578in}{0.892247in}}%
\pgfpathlineto{\pgfqpoint{3.969578in}{0.892247in}}%
\pgfpathlineto{\pgfqpoint{3.969578in}{0.895196in}}%
\pgfpathlineto{\pgfqpoint{3.974119in}{0.895196in}}%
\pgfpathlineto{\pgfqpoint{3.974119in}{0.892247in}}%
\pgfpathmoveto{\pgfqpoint{3.965037in}{0.895196in}}%
\pgfpathlineto{\pgfqpoint{3.965037in}{0.895196in}}%
\pgfpathlineto{\pgfqpoint{3.965037in}{0.898145in}}%
\pgfpathlineto{\pgfqpoint{3.969578in}{0.898145in}}%
\pgfpathlineto{\pgfqpoint{3.969578in}{0.895196in}}%
\pgfpathmoveto{\pgfqpoint{3.965037in}{0.898145in}}%
\pgfpathlineto{\pgfqpoint{3.965037in}{0.898145in}}%
\pgfpathlineto{\pgfqpoint{3.965037in}{0.901094in}}%
\pgfpathlineto{\pgfqpoint{3.969578in}{0.901094in}}%
\pgfpathlineto{\pgfqpoint{3.969578in}{0.898145in}}%
\pgfpathmoveto{\pgfqpoint{3.969578in}{0.895196in}}%
\pgfpathlineto{\pgfqpoint{3.969578in}{0.895196in}}%
\pgfpathlineto{\pgfqpoint{3.969578in}{0.898145in}}%
\pgfpathlineto{\pgfqpoint{3.974119in}{0.898145in}}%
\pgfpathlineto{\pgfqpoint{3.974119in}{0.895196in}}%
\pgfpathmoveto{\pgfqpoint{3.969578in}{0.898145in}}%
\pgfpathlineto{\pgfqpoint{3.969578in}{0.898145in}}%
\pgfpathlineto{\pgfqpoint{3.969578in}{0.901094in}}%
\pgfpathlineto{\pgfqpoint{3.974119in}{0.901094in}}%
\pgfpathlineto{\pgfqpoint{3.974119in}{0.898145in}}%
\pgfpathmoveto{\pgfqpoint{3.974119in}{0.898145in}}%
\pgfpathlineto{\pgfqpoint{3.974119in}{0.898145in}}%
\pgfpathlineto{\pgfqpoint{3.974119in}{0.901094in}}%
\pgfpathlineto{\pgfqpoint{3.978660in}{0.901094in}}%
\pgfpathlineto{\pgfqpoint{3.978660in}{0.898145in}}%
\pgfpathmoveto{\pgfqpoint{3.974119in}{0.901094in}}%
\pgfpathlineto{\pgfqpoint{3.974119in}{0.901094in}}%
\pgfpathlineto{\pgfqpoint{3.974119in}{0.904043in}}%
\pgfpathlineto{\pgfqpoint{3.978660in}{0.904043in}}%
\pgfpathlineto{\pgfqpoint{3.978660in}{0.901094in}}%
\pgfpathmoveto{\pgfqpoint{3.974119in}{0.904043in}}%
\pgfpathlineto{\pgfqpoint{3.974119in}{0.904043in}}%
\pgfpathlineto{\pgfqpoint{3.974119in}{0.906992in}}%
\pgfpathlineto{\pgfqpoint{3.978660in}{0.906992in}}%
\pgfpathlineto{\pgfqpoint{3.978660in}{0.904043in}}%
\pgfpathmoveto{\pgfqpoint{3.978660in}{0.901094in}}%
\pgfpathlineto{\pgfqpoint{3.978660in}{0.901094in}}%
\pgfpathlineto{\pgfqpoint{3.978660in}{0.904043in}}%
\pgfpathlineto{\pgfqpoint{3.983201in}{0.904043in}}%
\pgfpathlineto{\pgfqpoint{3.983201in}{0.901094in}}%
\pgfpathmoveto{\pgfqpoint{3.978660in}{0.904043in}}%
\pgfpathlineto{\pgfqpoint{3.978660in}{0.904043in}}%
\pgfpathlineto{\pgfqpoint{3.978660in}{0.906992in}}%
\pgfpathlineto{\pgfqpoint{3.983201in}{0.906992in}}%
\pgfpathlineto{\pgfqpoint{3.983201in}{0.904043in}}%
\pgfpathmoveto{\pgfqpoint{3.983201in}{0.904043in}}%
\pgfpathlineto{\pgfqpoint{3.983201in}{0.904043in}}%
\pgfpathlineto{\pgfqpoint{3.983201in}{0.906992in}}%
\pgfpathlineto{\pgfqpoint{3.987742in}{0.906992in}}%
\pgfpathlineto{\pgfqpoint{3.987742in}{0.904043in}}%
\pgfpathmoveto{\pgfqpoint{3.983201in}{0.906992in}}%
\pgfpathlineto{\pgfqpoint{3.983201in}{0.906992in}}%
\pgfpathlineto{\pgfqpoint{3.983201in}{0.909941in}}%
\pgfpathlineto{\pgfqpoint{3.987742in}{0.909941in}}%
\pgfpathlineto{\pgfqpoint{3.987742in}{0.906992in}}%
\pgfpathmoveto{\pgfqpoint{3.983201in}{0.909941in}}%
\pgfpathlineto{\pgfqpoint{3.983201in}{0.909941in}}%
\pgfpathlineto{\pgfqpoint{3.983201in}{0.912891in}}%
\pgfpathlineto{\pgfqpoint{3.987742in}{0.912891in}}%
\pgfpathlineto{\pgfqpoint{3.987742in}{0.909941in}}%
\pgfpathmoveto{\pgfqpoint{3.987742in}{0.906992in}}%
\pgfpathlineto{\pgfqpoint{3.987742in}{0.906992in}}%
\pgfpathlineto{\pgfqpoint{3.987742in}{0.909941in}}%
\pgfpathlineto{\pgfqpoint{3.992283in}{0.909941in}}%
\pgfpathlineto{\pgfqpoint{3.992283in}{0.906992in}}%
\pgfpathmoveto{\pgfqpoint{3.987742in}{0.909941in}}%
\pgfpathlineto{\pgfqpoint{3.987742in}{0.909941in}}%
\pgfpathlineto{\pgfqpoint{3.987742in}{0.912891in}}%
\pgfpathlineto{\pgfqpoint{3.992283in}{0.912891in}}%
\pgfpathlineto{\pgfqpoint{3.992283in}{0.909941in}}%
\pgfpathmoveto{\pgfqpoint{3.992283in}{0.909941in}}%
\pgfpathlineto{\pgfqpoint{3.992283in}{0.909941in}}%
\pgfpathlineto{\pgfqpoint{3.992283in}{0.912891in}}%
\pgfpathlineto{\pgfqpoint{3.996824in}{0.912891in}}%
\pgfpathlineto{\pgfqpoint{3.996824in}{0.909941in}}%
\pgfpathmoveto{\pgfqpoint{3.992283in}{0.912891in}}%
\pgfpathlineto{\pgfqpoint{3.992283in}{0.912891in}}%
\pgfpathlineto{\pgfqpoint{3.992283in}{0.915840in}}%
\pgfpathlineto{\pgfqpoint{3.996824in}{0.915840in}}%
\pgfpathlineto{\pgfqpoint{3.996824in}{0.912891in}}%
\pgfpathmoveto{\pgfqpoint{3.992283in}{0.915840in}}%
\pgfpathlineto{\pgfqpoint{3.992283in}{0.915840in}}%
\pgfpathlineto{\pgfqpoint{3.992283in}{0.918789in}}%
\pgfpathlineto{\pgfqpoint{3.996824in}{0.918789in}}%
\pgfpathlineto{\pgfqpoint{3.996824in}{0.915840in}}%
\pgfpathmoveto{\pgfqpoint{3.996824in}{0.915840in}}%
\pgfpathlineto{\pgfqpoint{3.996824in}{0.915840in}}%
\pgfpathlineto{\pgfqpoint{3.996824in}{0.918789in}}%
\pgfpathlineto{\pgfqpoint{4.001365in}{0.918789in}}%
\pgfpathlineto{\pgfqpoint{4.001365in}{0.915840in}}%
\pgfpathmoveto{\pgfqpoint{3.992283in}{0.918789in}}%
\pgfpathlineto{\pgfqpoint{3.992283in}{0.918789in}}%
\pgfpathlineto{\pgfqpoint{3.992283in}{0.921738in}}%
\pgfpathlineto{\pgfqpoint{3.996824in}{0.921738in}}%
\pgfpathlineto{\pgfqpoint{3.996824in}{0.918789in}}%
\pgfpathmoveto{\pgfqpoint{3.992283in}{0.921738in}}%
\pgfpathlineto{\pgfqpoint{3.992283in}{0.921738in}}%
\pgfpathlineto{\pgfqpoint{3.992283in}{0.924687in}}%
\pgfpathlineto{\pgfqpoint{3.996824in}{0.924687in}}%
\pgfpathlineto{\pgfqpoint{3.996824in}{0.921738in}}%
\pgfpathmoveto{\pgfqpoint{3.996824in}{0.918789in}}%
\pgfpathlineto{\pgfqpoint{3.996824in}{0.918789in}}%
\pgfpathlineto{\pgfqpoint{3.996824in}{0.921738in}}%
\pgfpathlineto{\pgfqpoint{4.001365in}{0.921738in}}%
\pgfpathlineto{\pgfqpoint{4.001365in}{0.918789in}}%
\pgfpathmoveto{\pgfqpoint{3.996824in}{0.921738in}}%
\pgfpathlineto{\pgfqpoint{3.996824in}{0.921738in}}%
\pgfpathlineto{\pgfqpoint{3.996824in}{0.924687in}}%
\pgfpathlineto{\pgfqpoint{4.001365in}{0.924687in}}%
\pgfpathlineto{\pgfqpoint{4.001365in}{0.921738in}}%
\pgfpathmoveto{\pgfqpoint{4.001365in}{0.918789in}}%
\pgfpathlineto{\pgfqpoint{4.001365in}{0.918789in}}%
\pgfpathlineto{\pgfqpoint{4.001365in}{0.921738in}}%
\pgfpathlineto{\pgfqpoint{4.005906in}{0.921738in}}%
\pgfpathlineto{\pgfqpoint{4.005906in}{0.918789in}}%
\pgfpathmoveto{\pgfqpoint{4.001365in}{0.921738in}}%
\pgfpathlineto{\pgfqpoint{4.001365in}{0.921738in}}%
\pgfpathlineto{\pgfqpoint{4.001365in}{0.924687in}}%
\pgfpathlineto{\pgfqpoint{4.005906in}{0.924687in}}%
\pgfpathlineto{\pgfqpoint{4.005906in}{0.921738in}}%
\pgfpathmoveto{\pgfqpoint{4.005906in}{0.921738in}}%
\pgfpathlineto{\pgfqpoint{4.005906in}{0.921738in}}%
\pgfpathlineto{\pgfqpoint{4.005906in}{0.924687in}}%
\pgfpathlineto{\pgfqpoint{4.010447in}{0.924687in}}%
\pgfpathlineto{\pgfqpoint{4.010447in}{0.921738in}}%
\pgfpathmoveto{\pgfqpoint{4.001365in}{0.924687in}}%
\pgfpathlineto{\pgfqpoint{4.001365in}{0.924687in}}%
\pgfpathlineto{\pgfqpoint{4.001365in}{0.927636in}}%
\pgfpathlineto{\pgfqpoint{4.005906in}{0.927636in}}%
\pgfpathlineto{\pgfqpoint{4.005906in}{0.924687in}}%
\pgfpathmoveto{\pgfqpoint{4.001365in}{0.927636in}}%
\pgfpathlineto{\pgfqpoint{4.001365in}{0.927636in}}%
\pgfpathlineto{\pgfqpoint{4.001365in}{0.930585in}}%
\pgfpathlineto{\pgfqpoint{4.005906in}{0.930585in}}%
\pgfpathlineto{\pgfqpoint{4.005906in}{0.927636in}}%
\pgfpathmoveto{\pgfqpoint{4.005906in}{0.924687in}}%
\pgfpathlineto{\pgfqpoint{4.005906in}{0.924687in}}%
\pgfpathlineto{\pgfqpoint{4.005906in}{0.927636in}}%
\pgfpathlineto{\pgfqpoint{4.010447in}{0.927636in}}%
\pgfpathlineto{\pgfqpoint{4.010447in}{0.924687in}}%
\pgfpathmoveto{\pgfqpoint{4.005906in}{0.927636in}}%
\pgfpathlineto{\pgfqpoint{4.005906in}{0.927636in}}%
\pgfpathlineto{\pgfqpoint{4.005906in}{0.930585in}}%
\pgfpathlineto{\pgfqpoint{4.010447in}{0.930585in}}%
\pgfpathlineto{\pgfqpoint{4.010447in}{0.927636in}}%
\pgfpathmoveto{\pgfqpoint{4.010447in}{0.924687in}}%
\pgfpathlineto{\pgfqpoint{4.010447in}{0.924687in}}%
\pgfpathlineto{\pgfqpoint{4.010447in}{0.927636in}}%
\pgfpathlineto{\pgfqpoint{4.014988in}{0.927636in}}%
\pgfpathlineto{\pgfqpoint{4.014988in}{0.924687in}}%
\pgfpathmoveto{\pgfqpoint{4.010447in}{0.927636in}}%
\pgfpathlineto{\pgfqpoint{4.010447in}{0.927636in}}%
\pgfpathlineto{\pgfqpoint{4.010447in}{0.930585in}}%
\pgfpathlineto{\pgfqpoint{4.014988in}{0.930585in}}%
\pgfpathlineto{\pgfqpoint{4.014988in}{0.927636in}}%
\pgfpathmoveto{\pgfqpoint{4.014988in}{0.927636in}}%
\pgfpathlineto{\pgfqpoint{4.014988in}{0.927636in}}%
\pgfpathlineto{\pgfqpoint{4.014988in}{0.930585in}}%
\pgfpathlineto{\pgfqpoint{4.019529in}{0.930585in}}%
\pgfpathlineto{\pgfqpoint{4.019529in}{0.927636in}}%
\pgfpathmoveto{\pgfqpoint{4.010447in}{0.930585in}}%
\pgfpathlineto{\pgfqpoint{4.010447in}{0.930585in}}%
\pgfpathlineto{\pgfqpoint{4.010447in}{0.933535in}}%
\pgfpathlineto{\pgfqpoint{4.014988in}{0.933535in}}%
\pgfpathlineto{\pgfqpoint{4.014988in}{0.930585in}}%
\pgfpathmoveto{\pgfqpoint{4.010447in}{0.933535in}}%
\pgfpathlineto{\pgfqpoint{4.010447in}{0.933535in}}%
\pgfpathlineto{\pgfqpoint{4.010447in}{0.936484in}}%
\pgfpathlineto{\pgfqpoint{4.014988in}{0.936484in}}%
\pgfpathlineto{\pgfqpoint{4.014988in}{0.933535in}}%
\pgfpathmoveto{\pgfqpoint{4.014988in}{0.930585in}}%
\pgfpathlineto{\pgfqpoint{4.014988in}{0.930585in}}%
\pgfpathlineto{\pgfqpoint{4.014988in}{0.933535in}}%
\pgfpathlineto{\pgfqpoint{4.019529in}{0.933535in}}%
\pgfpathlineto{\pgfqpoint{4.019529in}{0.930585in}}%
\pgfpathmoveto{\pgfqpoint{4.014988in}{0.933535in}}%
\pgfpathlineto{\pgfqpoint{4.014988in}{0.933535in}}%
\pgfpathlineto{\pgfqpoint{4.014988in}{0.936484in}}%
\pgfpathlineto{\pgfqpoint{4.019529in}{0.936484in}}%
\pgfpathlineto{\pgfqpoint{4.019529in}{0.933535in}}%
\pgfpathmoveto{\pgfqpoint{4.019529in}{0.933535in}}%
\pgfpathlineto{\pgfqpoint{4.019529in}{0.933535in}}%
\pgfpathlineto{\pgfqpoint{4.019529in}{0.936484in}}%
\pgfpathlineto{\pgfqpoint{4.024070in}{0.936484in}}%
\pgfpathlineto{\pgfqpoint{4.024070in}{0.933535in}}%
\pgfpathmoveto{\pgfqpoint{4.019529in}{0.936484in}}%
\pgfpathlineto{\pgfqpoint{4.019529in}{0.936484in}}%
\pgfpathlineto{\pgfqpoint{4.019529in}{0.939433in}}%
\pgfpathlineto{\pgfqpoint{4.024070in}{0.939433in}}%
\pgfpathlineto{\pgfqpoint{4.024070in}{0.936484in}}%
\pgfpathmoveto{\pgfqpoint{4.019529in}{0.939433in}}%
\pgfpathlineto{\pgfqpoint{4.019529in}{0.939433in}}%
\pgfpathlineto{\pgfqpoint{4.019529in}{0.942382in}}%
\pgfpathlineto{\pgfqpoint{4.024070in}{0.942382in}}%
\pgfpathlineto{\pgfqpoint{4.024070in}{0.939433in}}%
\pgfpathmoveto{\pgfqpoint{4.024070in}{0.936484in}}%
\pgfpathlineto{\pgfqpoint{4.024070in}{0.936484in}}%
\pgfpathlineto{\pgfqpoint{4.024070in}{0.939433in}}%
\pgfpathlineto{\pgfqpoint{4.028611in}{0.939433in}}%
\pgfpathlineto{\pgfqpoint{4.028611in}{0.936484in}}%
\pgfpathmoveto{\pgfqpoint{4.024070in}{0.939433in}}%
\pgfpathlineto{\pgfqpoint{4.024070in}{0.939433in}}%
\pgfpathlineto{\pgfqpoint{4.024070in}{0.942382in}}%
\pgfpathlineto{\pgfqpoint{4.028611in}{0.942382in}}%
\pgfpathlineto{\pgfqpoint{4.028611in}{0.939433in}}%
\pgfpathmoveto{\pgfqpoint{4.028611in}{0.939433in}}%
\pgfpathlineto{\pgfqpoint{4.028611in}{0.939433in}}%
\pgfpathlineto{\pgfqpoint{4.028611in}{0.942382in}}%
\pgfpathlineto{\pgfqpoint{4.033152in}{0.942382in}}%
\pgfpathlineto{\pgfqpoint{4.033152in}{0.939433in}}%
\pgfpathmoveto{\pgfqpoint{4.028611in}{0.942382in}}%
\pgfpathlineto{\pgfqpoint{4.028611in}{0.942382in}}%
\pgfpathlineto{\pgfqpoint{4.028611in}{0.945331in}}%
\pgfpathlineto{\pgfqpoint{4.033152in}{0.945331in}}%
\pgfpathlineto{\pgfqpoint{4.033152in}{0.942382in}}%
\pgfpathmoveto{\pgfqpoint{4.028611in}{0.945331in}}%
\pgfpathlineto{\pgfqpoint{4.028611in}{0.945331in}}%
\pgfpathlineto{\pgfqpoint{4.028611in}{0.948280in}}%
\pgfpathlineto{\pgfqpoint{4.033152in}{0.948280in}}%
\pgfpathlineto{\pgfqpoint{4.033152in}{0.945331in}}%
\pgfpathmoveto{\pgfqpoint{4.033152in}{0.942382in}}%
\pgfpathlineto{\pgfqpoint{4.033152in}{0.942382in}}%
\pgfpathlineto{\pgfqpoint{4.033152in}{0.945331in}}%
\pgfpathlineto{\pgfqpoint{4.037692in}{0.945331in}}%
\pgfpathlineto{\pgfqpoint{4.037692in}{0.942382in}}%
\pgfpathmoveto{\pgfqpoint{4.033152in}{0.945331in}}%
\pgfpathlineto{\pgfqpoint{4.033152in}{0.945331in}}%
\pgfpathlineto{\pgfqpoint{4.033152in}{0.948280in}}%
\pgfpathlineto{\pgfqpoint{4.037692in}{0.948280in}}%
\pgfpathlineto{\pgfqpoint{4.037692in}{0.945331in}}%
\pgfpathmoveto{\pgfqpoint{4.037692in}{0.945331in}}%
\pgfpathlineto{\pgfqpoint{4.037692in}{0.945331in}}%
\pgfpathlineto{\pgfqpoint{4.037692in}{0.948280in}}%
\pgfpathlineto{\pgfqpoint{4.042233in}{0.948280in}}%
\pgfpathlineto{\pgfqpoint{4.042233in}{0.945331in}}%
\pgfpathmoveto{\pgfqpoint{4.037692in}{0.948280in}}%
\pgfpathlineto{\pgfqpoint{4.037692in}{0.948280in}}%
\pgfpathlineto{\pgfqpoint{4.037692in}{0.951230in}}%
\pgfpathlineto{\pgfqpoint{4.042233in}{0.951230in}}%
\pgfpathlineto{\pgfqpoint{4.042233in}{0.948280in}}%
\pgfpathmoveto{\pgfqpoint{4.037692in}{0.951230in}}%
\pgfpathlineto{\pgfqpoint{4.037692in}{0.951230in}}%
\pgfpathlineto{\pgfqpoint{4.037692in}{0.954179in}}%
\pgfpathlineto{\pgfqpoint{4.042233in}{0.954179in}}%
\pgfpathlineto{\pgfqpoint{4.042233in}{0.951230in}}%
\pgfpathmoveto{\pgfqpoint{4.042233in}{0.951230in}}%
\pgfpathlineto{\pgfqpoint{4.042233in}{0.951230in}}%
\pgfpathlineto{\pgfqpoint{4.042233in}{0.954179in}}%
\pgfpathlineto{\pgfqpoint{4.046774in}{0.954179in}}%
\pgfpathlineto{\pgfqpoint{4.046774in}{0.951230in}}%
\pgfpathmoveto{\pgfqpoint{4.037692in}{0.954179in}}%
\pgfpathlineto{\pgfqpoint{4.037692in}{0.954179in}}%
\pgfpathlineto{\pgfqpoint{4.037692in}{0.957128in}}%
\pgfpathlineto{\pgfqpoint{4.042233in}{0.957128in}}%
\pgfpathlineto{\pgfqpoint{4.042233in}{0.954179in}}%
\pgfpathmoveto{\pgfqpoint{4.037692in}{0.957128in}}%
\pgfpathlineto{\pgfqpoint{4.037692in}{0.957128in}}%
\pgfpathlineto{\pgfqpoint{4.037692in}{0.960077in}}%
\pgfpathlineto{\pgfqpoint{4.042233in}{0.960077in}}%
\pgfpathlineto{\pgfqpoint{4.042233in}{0.957128in}}%
\pgfpathmoveto{\pgfqpoint{4.042233in}{0.954179in}}%
\pgfpathlineto{\pgfqpoint{4.042233in}{0.954179in}}%
\pgfpathlineto{\pgfqpoint{4.042233in}{0.957128in}}%
\pgfpathlineto{\pgfqpoint{4.046774in}{0.957128in}}%
\pgfpathlineto{\pgfqpoint{4.046774in}{0.954179in}}%
\pgfpathmoveto{\pgfqpoint{4.042233in}{0.957128in}}%
\pgfpathlineto{\pgfqpoint{4.042233in}{0.957128in}}%
\pgfpathlineto{\pgfqpoint{4.042233in}{0.960077in}}%
\pgfpathlineto{\pgfqpoint{4.046774in}{0.960077in}}%
\pgfpathlineto{\pgfqpoint{4.046774in}{0.957128in}}%
\pgfpathmoveto{\pgfqpoint{4.046774in}{0.954179in}}%
\pgfpathlineto{\pgfqpoint{4.046774in}{0.954179in}}%
\pgfpathlineto{\pgfqpoint{4.046774in}{0.957128in}}%
\pgfpathlineto{\pgfqpoint{4.051315in}{0.957128in}}%
\pgfpathlineto{\pgfqpoint{4.051315in}{0.954179in}}%
\pgfpathmoveto{\pgfqpoint{4.046774in}{0.957128in}}%
\pgfpathlineto{\pgfqpoint{4.046774in}{0.957128in}}%
\pgfpathlineto{\pgfqpoint{4.046774in}{0.960077in}}%
\pgfpathlineto{\pgfqpoint{4.051315in}{0.960077in}}%
\pgfpathlineto{\pgfqpoint{4.051315in}{0.957128in}}%
\pgfpathmoveto{\pgfqpoint{4.051315in}{0.957128in}}%
\pgfpathlineto{\pgfqpoint{4.051315in}{0.957128in}}%
\pgfpathlineto{\pgfqpoint{4.051315in}{0.960077in}}%
\pgfpathlineto{\pgfqpoint{4.055856in}{0.960077in}}%
\pgfpathlineto{\pgfqpoint{4.055856in}{0.957128in}}%
\pgfpathmoveto{\pgfqpoint{4.046774in}{0.960077in}}%
\pgfpathlineto{\pgfqpoint{4.046774in}{0.960077in}}%
\pgfpathlineto{\pgfqpoint{4.046774in}{0.963026in}}%
\pgfpathlineto{\pgfqpoint{4.051315in}{0.963026in}}%
\pgfpathlineto{\pgfqpoint{4.051315in}{0.960077in}}%
\pgfpathmoveto{\pgfqpoint{4.046774in}{0.963026in}}%
\pgfpathlineto{\pgfqpoint{4.046774in}{0.963026in}}%
\pgfpathlineto{\pgfqpoint{4.046774in}{0.965975in}}%
\pgfpathlineto{\pgfqpoint{4.051315in}{0.965975in}}%
\pgfpathlineto{\pgfqpoint{4.051315in}{0.963026in}}%
\pgfpathmoveto{\pgfqpoint{4.051315in}{0.960077in}}%
\pgfpathlineto{\pgfqpoint{4.051315in}{0.960077in}}%
\pgfpathlineto{\pgfqpoint{4.051315in}{0.963026in}}%
\pgfpathlineto{\pgfqpoint{4.055856in}{0.963026in}}%
\pgfpathlineto{\pgfqpoint{4.055856in}{0.960077in}}%
\pgfpathmoveto{\pgfqpoint{4.051315in}{0.963026in}}%
\pgfpathlineto{\pgfqpoint{4.051315in}{0.963026in}}%
\pgfpathlineto{\pgfqpoint{4.051315in}{0.965975in}}%
\pgfpathlineto{\pgfqpoint{4.055856in}{0.965975in}}%
\pgfpathlineto{\pgfqpoint{4.055856in}{0.963026in}}%
\pgfpathmoveto{\pgfqpoint{4.055856in}{0.960077in}}%
\pgfpathlineto{\pgfqpoint{4.055856in}{0.960077in}}%
\pgfpathlineto{\pgfqpoint{4.055856in}{0.963026in}}%
\pgfpathlineto{\pgfqpoint{4.060397in}{0.963026in}}%
\pgfpathlineto{\pgfqpoint{4.060397in}{0.960077in}}%
\pgfpathmoveto{\pgfqpoint{4.055856in}{0.963026in}}%
\pgfpathlineto{\pgfqpoint{4.055856in}{0.963026in}}%
\pgfpathlineto{\pgfqpoint{4.055856in}{0.965975in}}%
\pgfpathlineto{\pgfqpoint{4.060397in}{0.965975in}}%
\pgfpathlineto{\pgfqpoint{4.060397in}{0.963026in}}%
\pgfpathmoveto{\pgfqpoint{4.060397in}{0.963026in}}%
\pgfpathlineto{\pgfqpoint{4.060397in}{0.963026in}}%
\pgfpathlineto{\pgfqpoint{4.060397in}{0.965975in}}%
\pgfpathlineto{\pgfqpoint{4.064938in}{0.965975in}}%
\pgfpathlineto{\pgfqpoint{4.064938in}{0.963026in}}%
\pgfpathmoveto{\pgfqpoint{4.055856in}{0.965975in}}%
\pgfpathlineto{\pgfqpoint{4.055856in}{0.965975in}}%
\pgfpathlineto{\pgfqpoint{4.055856in}{0.968924in}}%
\pgfpathlineto{\pgfqpoint{4.060397in}{0.968924in}}%
\pgfpathlineto{\pgfqpoint{4.060397in}{0.965975in}}%
\pgfpathmoveto{\pgfqpoint{4.055856in}{0.968924in}}%
\pgfpathlineto{\pgfqpoint{4.055856in}{0.968924in}}%
\pgfpathlineto{\pgfqpoint{4.055856in}{0.971874in}}%
\pgfpathlineto{\pgfqpoint{4.060397in}{0.971874in}}%
\pgfpathlineto{\pgfqpoint{4.060397in}{0.968924in}}%
\pgfpathmoveto{\pgfqpoint{4.060397in}{0.965975in}}%
\pgfpathlineto{\pgfqpoint{4.060397in}{0.965975in}}%
\pgfpathlineto{\pgfqpoint{4.060397in}{0.968924in}}%
\pgfpathlineto{\pgfqpoint{4.064938in}{0.968924in}}%
\pgfpathlineto{\pgfqpoint{4.064938in}{0.965975in}}%
\pgfpathmoveto{\pgfqpoint{4.060397in}{0.968924in}}%
\pgfpathlineto{\pgfqpoint{4.060397in}{0.968924in}}%
\pgfpathlineto{\pgfqpoint{4.060397in}{0.971874in}}%
\pgfpathlineto{\pgfqpoint{4.064938in}{0.971874in}}%
\pgfpathlineto{\pgfqpoint{4.064938in}{0.968924in}}%
\pgfpathmoveto{\pgfqpoint{4.064938in}{0.968924in}}%
\pgfpathlineto{\pgfqpoint{4.064938in}{0.968924in}}%
\pgfpathlineto{\pgfqpoint{4.064938in}{0.971874in}}%
\pgfpathlineto{\pgfqpoint{4.069479in}{0.971874in}}%
\pgfpathlineto{\pgfqpoint{4.069479in}{0.968924in}}%
\pgfpathmoveto{\pgfqpoint{4.064938in}{0.971874in}}%
\pgfpathlineto{\pgfqpoint{4.064938in}{0.971874in}}%
\pgfpathlineto{\pgfqpoint{4.064938in}{0.974823in}}%
\pgfpathlineto{\pgfqpoint{4.069479in}{0.974823in}}%
\pgfpathlineto{\pgfqpoint{4.069479in}{0.971874in}}%
\pgfpathmoveto{\pgfqpoint{4.064938in}{0.974823in}}%
\pgfpathlineto{\pgfqpoint{4.064938in}{0.974823in}}%
\pgfpathlineto{\pgfqpoint{4.064938in}{0.977772in}}%
\pgfpathlineto{\pgfqpoint{4.069479in}{0.977772in}}%
\pgfpathlineto{\pgfqpoint{4.069479in}{0.974823in}}%
\pgfpathmoveto{\pgfqpoint{4.069479in}{0.971874in}}%
\pgfpathlineto{\pgfqpoint{4.069479in}{0.971874in}}%
\pgfpathlineto{\pgfqpoint{4.069479in}{0.974823in}}%
\pgfpathlineto{\pgfqpoint{4.074020in}{0.974823in}}%
\pgfpathlineto{\pgfqpoint{4.074020in}{0.971874in}}%
\pgfpathmoveto{\pgfqpoint{4.069479in}{0.974823in}}%
\pgfpathlineto{\pgfqpoint{4.069479in}{0.974823in}}%
\pgfpathlineto{\pgfqpoint{4.069479in}{0.977772in}}%
\pgfpathlineto{\pgfqpoint{4.074020in}{0.977772in}}%
\pgfpathlineto{\pgfqpoint{4.074020in}{0.974823in}}%
\pgfpathmoveto{\pgfqpoint{4.074020in}{0.974823in}}%
\pgfpathlineto{\pgfqpoint{4.074020in}{0.974823in}}%
\pgfpathlineto{\pgfqpoint{4.074020in}{0.977772in}}%
\pgfpathlineto{\pgfqpoint{4.078561in}{0.977772in}}%
\pgfpathlineto{\pgfqpoint{4.078561in}{0.974823in}}%
\pgfpathmoveto{\pgfqpoint{4.074020in}{0.977772in}}%
\pgfpathlineto{\pgfqpoint{4.074020in}{0.977772in}}%
\pgfpathlineto{\pgfqpoint{4.074020in}{0.980721in}}%
\pgfpathlineto{\pgfqpoint{4.078561in}{0.980721in}}%
\pgfpathlineto{\pgfqpoint{4.078561in}{0.977772in}}%
\pgfpathmoveto{\pgfqpoint{4.074020in}{0.980721in}}%
\pgfpathlineto{\pgfqpoint{4.074020in}{0.980721in}}%
\pgfpathlineto{\pgfqpoint{4.074020in}{0.983670in}}%
\pgfpathlineto{\pgfqpoint{4.078561in}{0.983670in}}%
\pgfpathlineto{\pgfqpoint{4.078561in}{0.980721in}}%
\pgfpathmoveto{\pgfqpoint{4.078561in}{0.977772in}}%
\pgfpathlineto{\pgfqpoint{4.078561in}{0.977772in}}%
\pgfpathlineto{\pgfqpoint{4.078561in}{0.980721in}}%
\pgfpathlineto{\pgfqpoint{4.083102in}{0.980721in}}%
\pgfpathlineto{\pgfqpoint{4.083102in}{0.977772in}}%
\pgfpathmoveto{\pgfqpoint{4.078561in}{0.980721in}}%
\pgfpathlineto{\pgfqpoint{4.078561in}{0.980721in}}%
\pgfpathlineto{\pgfqpoint{4.078561in}{0.983670in}}%
\pgfpathlineto{\pgfqpoint{4.083102in}{0.983670in}}%
\pgfpathlineto{\pgfqpoint{4.083102in}{0.980721in}}%
\pgfpathmoveto{\pgfqpoint{4.083102in}{0.980721in}}%
\pgfpathlineto{\pgfqpoint{4.083102in}{0.980721in}}%
\pgfpathlineto{\pgfqpoint{4.083102in}{0.983670in}}%
\pgfpathlineto{\pgfqpoint{4.087643in}{0.983670in}}%
\pgfpathlineto{\pgfqpoint{4.087643in}{0.980721in}}%
\pgfpathmoveto{\pgfqpoint{4.083102in}{0.983670in}}%
\pgfpathlineto{\pgfqpoint{4.083102in}{0.983670in}}%
\pgfpathlineto{\pgfqpoint{4.083102in}{0.986619in}}%
\pgfpathlineto{\pgfqpoint{4.087643in}{0.986619in}}%
\pgfpathlineto{\pgfqpoint{4.087643in}{0.983670in}}%
\pgfpathmoveto{\pgfqpoint{4.083102in}{0.986619in}}%
\pgfpathlineto{\pgfqpoint{4.083102in}{0.986619in}}%
\pgfpathlineto{\pgfqpoint{4.083102in}{0.989569in}}%
\pgfpathlineto{\pgfqpoint{4.087643in}{0.989569in}}%
\pgfpathlineto{\pgfqpoint{4.087643in}{0.986619in}}%
\pgfpathmoveto{\pgfqpoint{4.087643in}{0.986619in}}%
\pgfpathlineto{\pgfqpoint{4.087643in}{0.986619in}}%
\pgfpathlineto{\pgfqpoint{4.087643in}{0.989569in}}%
\pgfpathlineto{\pgfqpoint{4.092184in}{0.989569in}}%
\pgfpathlineto{\pgfqpoint{4.092184in}{0.986619in}}%
\pgfpathmoveto{\pgfqpoint{4.083102in}{0.989569in}}%
\pgfpathlineto{\pgfqpoint{4.083102in}{0.989569in}}%
\pgfpathlineto{\pgfqpoint{4.083102in}{0.992518in}}%
\pgfpathlineto{\pgfqpoint{4.087643in}{0.992518in}}%
\pgfpathlineto{\pgfqpoint{4.087643in}{0.989569in}}%
\pgfpathmoveto{\pgfqpoint{4.083102in}{0.992518in}}%
\pgfpathlineto{\pgfqpoint{4.083102in}{0.992518in}}%
\pgfpathlineto{\pgfqpoint{4.083102in}{0.995467in}}%
\pgfpathlineto{\pgfqpoint{4.087643in}{0.995467in}}%
\pgfpathlineto{\pgfqpoint{4.087643in}{0.992518in}}%
\pgfpathmoveto{\pgfqpoint{4.087643in}{0.989569in}}%
\pgfpathlineto{\pgfqpoint{4.087643in}{0.989569in}}%
\pgfpathlineto{\pgfqpoint{4.087643in}{0.992518in}}%
\pgfpathlineto{\pgfqpoint{4.092184in}{0.992518in}}%
\pgfpathlineto{\pgfqpoint{4.092184in}{0.989569in}}%
\pgfpathmoveto{\pgfqpoint{4.087643in}{0.992518in}}%
\pgfpathlineto{\pgfqpoint{4.087643in}{0.992518in}}%
\pgfpathlineto{\pgfqpoint{4.087643in}{0.995467in}}%
\pgfpathlineto{\pgfqpoint{4.092184in}{0.995467in}}%
\pgfpathlineto{\pgfqpoint{4.092184in}{0.992518in}}%
\pgfpathmoveto{\pgfqpoint{4.083102in}{1.125233in}}%
\pgfpathlineto{\pgfqpoint{4.083102in}{1.125233in}}%
\pgfpathlineto{\pgfqpoint{4.083102in}{1.128182in}}%
\pgfpathlineto{\pgfqpoint{4.087643in}{1.128182in}}%
\pgfpathlineto{\pgfqpoint{4.087643in}{1.125233in}}%
\pgfpathmoveto{\pgfqpoint{4.083102in}{1.128182in}}%
\pgfpathlineto{\pgfqpoint{4.083102in}{1.128182in}}%
\pgfpathlineto{\pgfqpoint{4.083102in}{1.131132in}}%
\pgfpathlineto{\pgfqpoint{4.087643in}{1.131132in}}%
\pgfpathlineto{\pgfqpoint{4.087643in}{1.128182in}}%
\pgfpathmoveto{\pgfqpoint{4.087643in}{1.125233in}}%
\pgfpathlineto{\pgfqpoint{4.087643in}{1.125233in}}%
\pgfpathlineto{\pgfqpoint{4.087643in}{1.128182in}}%
\pgfpathlineto{\pgfqpoint{4.092184in}{1.128182in}}%
\pgfpathlineto{\pgfqpoint{4.092184in}{1.125233in}}%
\pgfpathmoveto{\pgfqpoint{4.087643in}{1.128182in}}%
\pgfpathlineto{\pgfqpoint{4.087643in}{1.128182in}}%
\pgfpathlineto{\pgfqpoint{4.087643in}{1.131132in}}%
\pgfpathlineto{\pgfqpoint{4.092184in}{1.131132in}}%
\pgfpathlineto{\pgfqpoint{4.092184in}{1.128182in}}%
\pgfpathmoveto{\pgfqpoint{4.083102in}{1.131132in}}%
\pgfpathlineto{\pgfqpoint{4.083102in}{1.131132in}}%
\pgfpathlineto{\pgfqpoint{4.083102in}{1.134081in}}%
\pgfpathlineto{\pgfqpoint{4.087643in}{1.134081in}}%
\pgfpathlineto{\pgfqpoint{4.087643in}{1.131132in}}%
\pgfpathmoveto{\pgfqpoint{4.083102in}{1.134081in}}%
\pgfpathlineto{\pgfqpoint{4.083102in}{1.134081in}}%
\pgfpathlineto{\pgfqpoint{4.083102in}{1.137030in}}%
\pgfpathlineto{\pgfqpoint{4.087643in}{1.137030in}}%
\pgfpathlineto{\pgfqpoint{4.087643in}{1.134081in}}%
\pgfpathmoveto{\pgfqpoint{4.087643in}{1.131132in}}%
\pgfpathlineto{\pgfqpoint{4.087643in}{1.131132in}}%
\pgfpathlineto{\pgfqpoint{4.087643in}{1.134081in}}%
\pgfpathlineto{\pgfqpoint{4.092184in}{1.134081in}}%
\pgfpathlineto{\pgfqpoint{4.092184in}{1.131132in}}%
\pgfpathmoveto{\pgfqpoint{4.064938in}{1.142929in}}%
\pgfpathlineto{\pgfqpoint{4.064938in}{1.142929in}}%
\pgfpathlineto{\pgfqpoint{4.064938in}{1.145878in}}%
\pgfpathlineto{\pgfqpoint{4.069479in}{1.145878in}}%
\pgfpathlineto{\pgfqpoint{4.069479in}{1.142929in}}%
\pgfpathmoveto{\pgfqpoint{4.064938in}{1.145878in}}%
\pgfpathlineto{\pgfqpoint{4.064938in}{1.145878in}}%
\pgfpathlineto{\pgfqpoint{4.064938in}{1.148827in}}%
\pgfpathlineto{\pgfqpoint{4.069479in}{1.148827in}}%
\pgfpathlineto{\pgfqpoint{4.069479in}{1.145878in}}%
\pgfpathmoveto{\pgfqpoint{4.069479in}{1.142929in}}%
\pgfpathlineto{\pgfqpoint{4.069479in}{1.142929in}}%
\pgfpathlineto{\pgfqpoint{4.069479in}{1.145878in}}%
\pgfpathlineto{\pgfqpoint{4.074020in}{1.145878in}}%
\pgfpathlineto{\pgfqpoint{4.074020in}{1.142929in}}%
\pgfpathmoveto{\pgfqpoint{4.069479in}{1.145878in}}%
\pgfpathlineto{\pgfqpoint{4.069479in}{1.145878in}}%
\pgfpathlineto{\pgfqpoint{4.069479in}{1.148827in}}%
\pgfpathlineto{\pgfqpoint{4.074020in}{1.148827in}}%
\pgfpathlineto{\pgfqpoint{4.074020in}{1.145878in}}%
\pgfpathmoveto{\pgfqpoint{4.055856in}{1.148827in}}%
\pgfpathlineto{\pgfqpoint{4.055856in}{1.148827in}}%
\pgfpathlineto{\pgfqpoint{4.055856in}{1.151776in}}%
\pgfpathlineto{\pgfqpoint{4.060397in}{1.151776in}}%
\pgfpathlineto{\pgfqpoint{4.060397in}{1.148827in}}%
\pgfpathmoveto{\pgfqpoint{4.055856in}{1.151776in}}%
\pgfpathlineto{\pgfqpoint{4.055856in}{1.151776in}}%
\pgfpathlineto{\pgfqpoint{4.055856in}{1.154726in}}%
\pgfpathlineto{\pgfqpoint{4.060397in}{1.154726in}}%
\pgfpathlineto{\pgfqpoint{4.060397in}{1.151776in}}%
\pgfpathmoveto{\pgfqpoint{4.060397in}{1.148827in}}%
\pgfpathlineto{\pgfqpoint{4.060397in}{1.148827in}}%
\pgfpathlineto{\pgfqpoint{4.060397in}{1.151776in}}%
\pgfpathlineto{\pgfqpoint{4.064938in}{1.151776in}}%
\pgfpathlineto{\pgfqpoint{4.064938in}{1.148827in}}%
\pgfpathmoveto{\pgfqpoint{4.060397in}{1.151776in}}%
\pgfpathlineto{\pgfqpoint{4.060397in}{1.151776in}}%
\pgfpathlineto{\pgfqpoint{4.060397in}{1.154726in}}%
\pgfpathlineto{\pgfqpoint{4.064938in}{1.154726in}}%
\pgfpathlineto{\pgfqpoint{4.064938in}{1.151776in}}%
\pgfpathmoveto{\pgfqpoint{4.055856in}{1.154726in}}%
\pgfpathlineto{\pgfqpoint{4.055856in}{1.154726in}}%
\pgfpathlineto{\pgfqpoint{4.055856in}{1.157675in}}%
\pgfpathlineto{\pgfqpoint{4.060397in}{1.157675in}}%
\pgfpathlineto{\pgfqpoint{4.060397in}{1.154726in}}%
\pgfpathmoveto{\pgfqpoint{4.055856in}{1.157675in}}%
\pgfpathlineto{\pgfqpoint{4.055856in}{1.157675in}}%
\pgfpathlineto{\pgfqpoint{4.055856in}{1.160624in}}%
\pgfpathlineto{\pgfqpoint{4.060397in}{1.160624in}}%
\pgfpathlineto{\pgfqpoint{4.060397in}{1.157675in}}%
\pgfpathmoveto{\pgfqpoint{4.060397in}{1.154726in}}%
\pgfpathlineto{\pgfqpoint{4.060397in}{1.154726in}}%
\pgfpathlineto{\pgfqpoint{4.060397in}{1.157675in}}%
\pgfpathlineto{\pgfqpoint{4.064938in}{1.157675in}}%
\pgfpathlineto{\pgfqpoint{4.064938in}{1.154726in}}%
\pgfpathmoveto{\pgfqpoint{4.064938in}{1.148827in}}%
\pgfpathlineto{\pgfqpoint{4.064938in}{1.148827in}}%
\pgfpathlineto{\pgfqpoint{4.064938in}{1.151776in}}%
\pgfpathlineto{\pgfqpoint{4.069479in}{1.151776in}}%
\pgfpathlineto{\pgfqpoint{4.069479in}{1.148827in}}%
\pgfpathmoveto{\pgfqpoint{4.064938in}{1.151776in}}%
\pgfpathlineto{\pgfqpoint{4.064938in}{1.151776in}}%
\pgfpathlineto{\pgfqpoint{4.064938in}{1.154726in}}%
\pgfpathlineto{\pgfqpoint{4.069479in}{1.154726in}}%
\pgfpathlineto{\pgfqpoint{4.069479in}{1.151776in}}%
\pgfpathmoveto{\pgfqpoint{4.069479in}{1.148827in}}%
\pgfpathlineto{\pgfqpoint{4.069479in}{1.148827in}}%
\pgfpathlineto{\pgfqpoint{4.069479in}{1.151776in}}%
\pgfpathlineto{\pgfqpoint{4.074020in}{1.151776in}}%
\pgfpathlineto{\pgfqpoint{4.074020in}{1.148827in}}%
\pgfpathmoveto{\pgfqpoint{4.074020in}{1.137030in}}%
\pgfpathlineto{\pgfqpoint{4.074020in}{1.137030in}}%
\pgfpathlineto{\pgfqpoint{4.074020in}{1.139979in}}%
\pgfpathlineto{\pgfqpoint{4.078561in}{1.139979in}}%
\pgfpathlineto{\pgfqpoint{4.078561in}{1.137030in}}%
\pgfpathmoveto{\pgfqpoint{4.074020in}{1.139979in}}%
\pgfpathlineto{\pgfqpoint{4.074020in}{1.139979in}}%
\pgfpathlineto{\pgfqpoint{4.074020in}{1.142929in}}%
\pgfpathlineto{\pgfqpoint{4.078561in}{1.142929in}}%
\pgfpathlineto{\pgfqpoint{4.078561in}{1.139979in}}%
\pgfpathmoveto{\pgfqpoint{4.078561in}{1.137030in}}%
\pgfpathlineto{\pgfqpoint{4.078561in}{1.137030in}}%
\pgfpathlineto{\pgfqpoint{4.078561in}{1.139979in}}%
\pgfpathlineto{\pgfqpoint{4.083102in}{1.139979in}}%
\pgfpathlineto{\pgfqpoint{4.083102in}{1.137030in}}%
\pgfpathmoveto{\pgfqpoint{4.078561in}{1.139979in}}%
\pgfpathlineto{\pgfqpoint{4.078561in}{1.139979in}}%
\pgfpathlineto{\pgfqpoint{4.078561in}{1.142929in}}%
\pgfpathlineto{\pgfqpoint{4.083102in}{1.142929in}}%
\pgfpathlineto{\pgfqpoint{4.083102in}{1.139979in}}%
\pgfpathmoveto{\pgfqpoint{4.074020in}{1.142929in}}%
\pgfpathlineto{\pgfqpoint{4.074020in}{1.142929in}}%
\pgfpathlineto{\pgfqpoint{4.074020in}{1.145878in}}%
\pgfpathlineto{\pgfqpoint{4.078561in}{1.145878in}}%
\pgfpathlineto{\pgfqpoint{4.078561in}{1.142929in}}%
\pgfpathmoveto{\pgfqpoint{4.083102in}{1.137030in}}%
\pgfpathlineto{\pgfqpoint{4.083102in}{1.137030in}}%
\pgfpathlineto{\pgfqpoint{4.083102in}{1.139979in}}%
\pgfpathlineto{\pgfqpoint{4.087643in}{1.139979in}}%
\pgfpathlineto{\pgfqpoint{4.087643in}{1.137030in}}%
\pgfpathmoveto{\pgfqpoint{4.010447in}{1.190117in}}%
\pgfpathlineto{\pgfqpoint{4.010447in}{1.190117in}}%
\pgfpathlineto{\pgfqpoint{4.010447in}{1.193066in}}%
\pgfpathlineto{\pgfqpoint{4.014988in}{1.193066in}}%
\pgfpathlineto{\pgfqpoint{4.014988in}{1.190117in}}%
\pgfpathmoveto{\pgfqpoint{4.010447in}{1.193066in}}%
\pgfpathlineto{\pgfqpoint{4.010447in}{1.193066in}}%
\pgfpathlineto{\pgfqpoint{4.010447in}{1.196015in}}%
\pgfpathlineto{\pgfqpoint{4.014988in}{1.196015in}}%
\pgfpathlineto{\pgfqpoint{4.014988in}{1.193066in}}%
\pgfpathmoveto{\pgfqpoint{4.014988in}{1.190117in}}%
\pgfpathlineto{\pgfqpoint{4.014988in}{1.190117in}}%
\pgfpathlineto{\pgfqpoint{4.014988in}{1.193066in}}%
\pgfpathlineto{\pgfqpoint{4.019529in}{1.193066in}}%
\pgfpathlineto{\pgfqpoint{4.019529in}{1.190117in}}%
\pgfpathmoveto{\pgfqpoint{4.014988in}{1.193066in}}%
\pgfpathlineto{\pgfqpoint{4.014988in}{1.193066in}}%
\pgfpathlineto{\pgfqpoint{4.014988in}{1.196015in}}%
\pgfpathlineto{\pgfqpoint{4.019529in}{1.196015in}}%
\pgfpathlineto{\pgfqpoint{4.019529in}{1.193066in}}%
\pgfpathmoveto{\pgfqpoint{4.001365in}{1.196015in}}%
\pgfpathlineto{\pgfqpoint{4.001365in}{1.196015in}}%
\pgfpathlineto{\pgfqpoint{4.001365in}{1.198964in}}%
\pgfpathlineto{\pgfqpoint{4.005906in}{1.198964in}}%
\pgfpathlineto{\pgfqpoint{4.005906in}{1.196015in}}%
\pgfpathmoveto{\pgfqpoint{4.001365in}{1.198964in}}%
\pgfpathlineto{\pgfqpoint{4.001365in}{1.198964in}}%
\pgfpathlineto{\pgfqpoint{4.001365in}{1.201913in}}%
\pgfpathlineto{\pgfqpoint{4.005906in}{1.201913in}}%
\pgfpathlineto{\pgfqpoint{4.005906in}{1.198964in}}%
\pgfpathmoveto{\pgfqpoint{4.005906in}{1.196015in}}%
\pgfpathlineto{\pgfqpoint{4.005906in}{1.196015in}}%
\pgfpathlineto{\pgfqpoint{4.005906in}{1.198964in}}%
\pgfpathlineto{\pgfqpoint{4.010447in}{1.198964in}}%
\pgfpathlineto{\pgfqpoint{4.010447in}{1.196015in}}%
\pgfpathmoveto{\pgfqpoint{4.005906in}{1.198964in}}%
\pgfpathlineto{\pgfqpoint{4.005906in}{1.198964in}}%
\pgfpathlineto{\pgfqpoint{4.005906in}{1.201913in}}%
\pgfpathlineto{\pgfqpoint{4.010447in}{1.201913in}}%
\pgfpathlineto{\pgfqpoint{4.010447in}{1.198964in}}%
\pgfpathmoveto{\pgfqpoint{4.001365in}{1.201913in}}%
\pgfpathlineto{\pgfqpoint{4.001365in}{1.201913in}}%
\pgfpathlineto{\pgfqpoint{4.001365in}{1.204863in}}%
\pgfpathlineto{\pgfqpoint{4.005906in}{1.204863in}}%
\pgfpathlineto{\pgfqpoint{4.005906in}{1.201913in}}%
\pgfpathmoveto{\pgfqpoint{4.001365in}{1.204863in}}%
\pgfpathlineto{\pgfqpoint{4.001365in}{1.204863in}}%
\pgfpathlineto{\pgfqpoint{4.001365in}{1.207812in}}%
\pgfpathlineto{\pgfqpoint{4.005906in}{1.207812in}}%
\pgfpathlineto{\pgfqpoint{4.005906in}{1.204863in}}%
\pgfpathmoveto{\pgfqpoint{4.005906in}{1.201913in}}%
\pgfpathlineto{\pgfqpoint{4.005906in}{1.201913in}}%
\pgfpathlineto{\pgfqpoint{4.005906in}{1.204863in}}%
\pgfpathlineto{\pgfqpoint{4.010447in}{1.204863in}}%
\pgfpathlineto{\pgfqpoint{4.010447in}{1.201913in}}%
\pgfpathmoveto{\pgfqpoint{4.010447in}{1.196015in}}%
\pgfpathlineto{\pgfqpoint{4.010447in}{1.196015in}}%
\pgfpathlineto{\pgfqpoint{4.010447in}{1.198964in}}%
\pgfpathlineto{\pgfqpoint{4.014988in}{1.198964in}}%
\pgfpathlineto{\pgfqpoint{4.014988in}{1.196015in}}%
\pgfpathmoveto{\pgfqpoint{4.010447in}{1.198964in}}%
\pgfpathlineto{\pgfqpoint{4.010447in}{1.198964in}}%
\pgfpathlineto{\pgfqpoint{4.010447in}{1.201913in}}%
\pgfpathlineto{\pgfqpoint{4.014988in}{1.201913in}}%
\pgfpathlineto{\pgfqpoint{4.014988in}{1.198964in}}%
\pgfpathmoveto{\pgfqpoint{4.014988in}{1.196015in}}%
\pgfpathlineto{\pgfqpoint{4.014988in}{1.196015in}}%
\pgfpathlineto{\pgfqpoint{4.014988in}{1.198964in}}%
\pgfpathlineto{\pgfqpoint{4.019529in}{1.198964in}}%
\pgfpathlineto{\pgfqpoint{4.019529in}{1.196015in}}%
\pgfpathmoveto{\pgfqpoint{3.974119in}{1.219609in}}%
\pgfpathlineto{\pgfqpoint{3.974119in}{1.219609in}}%
\pgfpathlineto{\pgfqpoint{3.974119in}{1.222558in}}%
\pgfpathlineto{\pgfqpoint{3.978660in}{1.222558in}}%
\pgfpathlineto{\pgfqpoint{3.978660in}{1.219609in}}%
\pgfpathmoveto{\pgfqpoint{3.974119in}{1.222558in}}%
\pgfpathlineto{\pgfqpoint{3.974119in}{1.222558in}}%
\pgfpathlineto{\pgfqpoint{3.974119in}{1.225507in}}%
\pgfpathlineto{\pgfqpoint{3.978660in}{1.225507in}}%
\pgfpathlineto{\pgfqpoint{3.978660in}{1.222558in}}%
\pgfpathmoveto{\pgfqpoint{3.978660in}{1.219609in}}%
\pgfpathlineto{\pgfqpoint{3.978660in}{1.219609in}}%
\pgfpathlineto{\pgfqpoint{3.978660in}{1.222558in}}%
\pgfpathlineto{\pgfqpoint{3.983201in}{1.222558in}}%
\pgfpathlineto{\pgfqpoint{3.983201in}{1.219609in}}%
\pgfpathmoveto{\pgfqpoint{3.978660in}{1.222558in}}%
\pgfpathlineto{\pgfqpoint{3.978660in}{1.222558in}}%
\pgfpathlineto{\pgfqpoint{3.978660in}{1.225507in}}%
\pgfpathlineto{\pgfqpoint{3.983201in}{1.225507in}}%
\pgfpathlineto{\pgfqpoint{3.983201in}{1.222558in}}%
\pgfpathmoveto{\pgfqpoint{3.974119in}{1.225507in}}%
\pgfpathlineto{\pgfqpoint{3.974119in}{1.225507in}}%
\pgfpathlineto{\pgfqpoint{3.974119in}{1.228456in}}%
\pgfpathlineto{\pgfqpoint{3.978660in}{1.228456in}}%
\pgfpathlineto{\pgfqpoint{3.978660in}{1.225507in}}%
\pgfpathmoveto{\pgfqpoint{3.974119in}{1.228456in}}%
\pgfpathlineto{\pgfqpoint{3.974119in}{1.228456in}}%
\pgfpathlineto{\pgfqpoint{3.974119in}{1.231406in}}%
\pgfpathlineto{\pgfqpoint{3.978660in}{1.231406in}}%
\pgfpathlineto{\pgfqpoint{3.978660in}{1.228456in}}%
\pgfpathmoveto{\pgfqpoint{3.978660in}{1.225507in}}%
\pgfpathlineto{\pgfqpoint{3.978660in}{1.225507in}}%
\pgfpathlineto{\pgfqpoint{3.978660in}{1.228456in}}%
\pgfpathlineto{\pgfqpoint{3.983201in}{1.228456in}}%
\pgfpathlineto{\pgfqpoint{3.983201in}{1.225507in}}%
\pgfpathmoveto{\pgfqpoint{3.955955in}{1.237304in}}%
\pgfpathlineto{\pgfqpoint{3.955955in}{1.237304in}}%
\pgfpathlineto{\pgfqpoint{3.955955in}{1.240253in}}%
\pgfpathlineto{\pgfqpoint{3.960496in}{1.240253in}}%
\pgfpathlineto{\pgfqpoint{3.960496in}{1.237304in}}%
\pgfpathmoveto{\pgfqpoint{3.955955in}{1.240253in}}%
\pgfpathlineto{\pgfqpoint{3.955955in}{1.240253in}}%
\pgfpathlineto{\pgfqpoint{3.955955in}{1.243202in}}%
\pgfpathlineto{\pgfqpoint{3.960496in}{1.243202in}}%
\pgfpathlineto{\pgfqpoint{3.960496in}{1.240253in}}%
\pgfpathmoveto{\pgfqpoint{3.960496in}{1.237304in}}%
\pgfpathlineto{\pgfqpoint{3.960496in}{1.237304in}}%
\pgfpathlineto{\pgfqpoint{3.960496in}{1.240253in}}%
\pgfpathlineto{\pgfqpoint{3.965037in}{1.240253in}}%
\pgfpathlineto{\pgfqpoint{3.965037in}{1.237304in}}%
\pgfpathmoveto{\pgfqpoint{3.960496in}{1.240253in}}%
\pgfpathlineto{\pgfqpoint{3.960496in}{1.240253in}}%
\pgfpathlineto{\pgfqpoint{3.960496in}{1.243202in}}%
\pgfpathlineto{\pgfqpoint{3.965037in}{1.243202in}}%
\pgfpathlineto{\pgfqpoint{3.965037in}{1.240253in}}%
\pgfpathmoveto{\pgfqpoint{3.946873in}{1.243202in}}%
\pgfpathlineto{\pgfqpoint{3.946873in}{1.243202in}}%
\pgfpathlineto{\pgfqpoint{3.946873in}{1.246152in}}%
\pgfpathlineto{\pgfqpoint{3.951414in}{1.246152in}}%
\pgfpathlineto{\pgfqpoint{3.951414in}{1.243202in}}%
\pgfpathmoveto{\pgfqpoint{3.946873in}{1.246152in}}%
\pgfpathlineto{\pgfqpoint{3.946873in}{1.246152in}}%
\pgfpathlineto{\pgfqpoint{3.946873in}{1.249101in}}%
\pgfpathlineto{\pgfqpoint{3.951414in}{1.249101in}}%
\pgfpathlineto{\pgfqpoint{3.951414in}{1.246152in}}%
\pgfpathmoveto{\pgfqpoint{3.951414in}{1.243202in}}%
\pgfpathlineto{\pgfqpoint{3.951414in}{1.243202in}}%
\pgfpathlineto{\pgfqpoint{3.951414in}{1.246152in}}%
\pgfpathlineto{\pgfqpoint{3.955955in}{1.246152in}}%
\pgfpathlineto{\pgfqpoint{3.955955in}{1.243202in}}%
\pgfpathmoveto{\pgfqpoint{3.951414in}{1.246152in}}%
\pgfpathlineto{\pgfqpoint{3.951414in}{1.246152in}}%
\pgfpathlineto{\pgfqpoint{3.951414in}{1.249101in}}%
\pgfpathlineto{\pgfqpoint{3.955955in}{1.249101in}}%
\pgfpathlineto{\pgfqpoint{3.955955in}{1.246152in}}%
\pgfpathmoveto{\pgfqpoint{3.946873in}{1.249101in}}%
\pgfpathlineto{\pgfqpoint{3.946873in}{1.249101in}}%
\pgfpathlineto{\pgfqpoint{3.946873in}{1.252050in}}%
\pgfpathlineto{\pgfqpoint{3.951414in}{1.252050in}}%
\pgfpathlineto{\pgfqpoint{3.951414in}{1.249101in}}%
\pgfpathmoveto{\pgfqpoint{3.946873in}{1.252050in}}%
\pgfpathlineto{\pgfqpoint{3.946873in}{1.252050in}}%
\pgfpathlineto{\pgfqpoint{3.946873in}{1.254999in}}%
\pgfpathlineto{\pgfqpoint{3.951414in}{1.254999in}}%
\pgfpathlineto{\pgfqpoint{3.951414in}{1.252050in}}%
\pgfpathmoveto{\pgfqpoint{3.951414in}{1.249101in}}%
\pgfpathlineto{\pgfqpoint{3.951414in}{1.249101in}}%
\pgfpathlineto{\pgfqpoint{3.951414in}{1.252050in}}%
\pgfpathlineto{\pgfqpoint{3.955955in}{1.252050in}}%
\pgfpathlineto{\pgfqpoint{3.955955in}{1.249101in}}%
\pgfpathmoveto{\pgfqpoint{3.955955in}{1.243202in}}%
\pgfpathlineto{\pgfqpoint{3.955955in}{1.243202in}}%
\pgfpathlineto{\pgfqpoint{3.955955in}{1.246152in}}%
\pgfpathlineto{\pgfqpoint{3.960496in}{1.246152in}}%
\pgfpathlineto{\pgfqpoint{3.960496in}{1.243202in}}%
\pgfpathmoveto{\pgfqpoint{3.955955in}{1.246152in}}%
\pgfpathlineto{\pgfqpoint{3.955955in}{1.246152in}}%
\pgfpathlineto{\pgfqpoint{3.955955in}{1.249101in}}%
\pgfpathlineto{\pgfqpoint{3.960496in}{1.249101in}}%
\pgfpathlineto{\pgfqpoint{3.960496in}{1.246152in}}%
\pgfpathmoveto{\pgfqpoint{3.960496in}{1.243202in}}%
\pgfpathlineto{\pgfqpoint{3.960496in}{1.243202in}}%
\pgfpathlineto{\pgfqpoint{3.960496in}{1.246152in}}%
\pgfpathlineto{\pgfqpoint{3.965037in}{1.246152in}}%
\pgfpathlineto{\pgfqpoint{3.965037in}{1.243202in}}%
\pgfpathmoveto{\pgfqpoint{3.965037in}{1.231406in}}%
\pgfpathlineto{\pgfqpoint{3.965037in}{1.231406in}}%
\pgfpathlineto{\pgfqpoint{3.965037in}{1.234355in}}%
\pgfpathlineto{\pgfqpoint{3.969578in}{1.234355in}}%
\pgfpathlineto{\pgfqpoint{3.969578in}{1.231406in}}%
\pgfpathmoveto{\pgfqpoint{3.965037in}{1.234355in}}%
\pgfpathlineto{\pgfqpoint{3.965037in}{1.234355in}}%
\pgfpathlineto{\pgfqpoint{3.965037in}{1.237304in}}%
\pgfpathlineto{\pgfqpoint{3.969578in}{1.237304in}}%
\pgfpathlineto{\pgfqpoint{3.969578in}{1.234355in}}%
\pgfpathmoveto{\pgfqpoint{3.969578in}{1.231406in}}%
\pgfpathlineto{\pgfqpoint{3.969578in}{1.231406in}}%
\pgfpathlineto{\pgfqpoint{3.969578in}{1.234355in}}%
\pgfpathlineto{\pgfqpoint{3.974119in}{1.234355in}}%
\pgfpathlineto{\pgfqpoint{3.974119in}{1.231406in}}%
\pgfpathmoveto{\pgfqpoint{3.969578in}{1.234355in}}%
\pgfpathlineto{\pgfqpoint{3.969578in}{1.234355in}}%
\pgfpathlineto{\pgfqpoint{3.969578in}{1.237304in}}%
\pgfpathlineto{\pgfqpoint{3.974119in}{1.237304in}}%
\pgfpathlineto{\pgfqpoint{3.974119in}{1.234355in}}%
\pgfpathmoveto{\pgfqpoint{3.965037in}{1.237304in}}%
\pgfpathlineto{\pgfqpoint{3.965037in}{1.237304in}}%
\pgfpathlineto{\pgfqpoint{3.965037in}{1.240253in}}%
\pgfpathlineto{\pgfqpoint{3.969578in}{1.240253in}}%
\pgfpathlineto{\pgfqpoint{3.969578in}{1.237304in}}%
\pgfpathmoveto{\pgfqpoint{3.974119in}{1.231406in}}%
\pgfpathlineto{\pgfqpoint{3.974119in}{1.231406in}}%
\pgfpathlineto{\pgfqpoint{3.974119in}{1.234355in}}%
\pgfpathlineto{\pgfqpoint{3.978660in}{1.234355in}}%
\pgfpathlineto{\pgfqpoint{3.978660in}{1.231406in}}%
\pgfpathmoveto{\pgfqpoint{3.983201in}{1.213710in}}%
\pgfpathlineto{\pgfqpoint{3.983201in}{1.213710in}}%
\pgfpathlineto{\pgfqpoint{3.983201in}{1.216659in}}%
\pgfpathlineto{\pgfqpoint{3.987742in}{1.216659in}}%
\pgfpathlineto{\pgfqpoint{3.987742in}{1.213710in}}%
\pgfpathmoveto{\pgfqpoint{3.983201in}{1.216659in}}%
\pgfpathlineto{\pgfqpoint{3.983201in}{1.216659in}}%
\pgfpathlineto{\pgfqpoint{3.983201in}{1.219609in}}%
\pgfpathlineto{\pgfqpoint{3.987742in}{1.219609in}}%
\pgfpathlineto{\pgfqpoint{3.987742in}{1.216659in}}%
\pgfpathmoveto{\pgfqpoint{3.987742in}{1.213710in}}%
\pgfpathlineto{\pgfqpoint{3.987742in}{1.213710in}}%
\pgfpathlineto{\pgfqpoint{3.987742in}{1.216659in}}%
\pgfpathlineto{\pgfqpoint{3.992283in}{1.216659in}}%
\pgfpathlineto{\pgfqpoint{3.992283in}{1.213710in}}%
\pgfpathmoveto{\pgfqpoint{3.987742in}{1.216659in}}%
\pgfpathlineto{\pgfqpoint{3.987742in}{1.216659in}}%
\pgfpathlineto{\pgfqpoint{3.987742in}{1.219609in}}%
\pgfpathlineto{\pgfqpoint{3.992283in}{1.219609in}}%
\pgfpathlineto{\pgfqpoint{3.992283in}{1.216659in}}%
\pgfpathmoveto{\pgfqpoint{3.992283in}{1.207812in}}%
\pgfpathlineto{\pgfqpoint{3.992283in}{1.207812in}}%
\pgfpathlineto{\pgfqpoint{3.992283in}{1.210761in}}%
\pgfpathlineto{\pgfqpoint{3.996824in}{1.210761in}}%
\pgfpathlineto{\pgfqpoint{3.996824in}{1.207812in}}%
\pgfpathmoveto{\pgfqpoint{3.992283in}{1.210761in}}%
\pgfpathlineto{\pgfqpoint{3.992283in}{1.210761in}}%
\pgfpathlineto{\pgfqpoint{3.992283in}{1.213710in}}%
\pgfpathlineto{\pgfqpoint{3.996824in}{1.213710in}}%
\pgfpathlineto{\pgfqpoint{3.996824in}{1.210761in}}%
\pgfpathmoveto{\pgfqpoint{3.996824in}{1.207812in}}%
\pgfpathlineto{\pgfqpoint{3.996824in}{1.207812in}}%
\pgfpathlineto{\pgfqpoint{3.996824in}{1.210761in}}%
\pgfpathlineto{\pgfqpoint{4.001365in}{1.210761in}}%
\pgfpathlineto{\pgfqpoint{4.001365in}{1.207812in}}%
\pgfpathmoveto{\pgfqpoint{3.996824in}{1.210761in}}%
\pgfpathlineto{\pgfqpoint{3.996824in}{1.210761in}}%
\pgfpathlineto{\pgfqpoint{3.996824in}{1.213710in}}%
\pgfpathlineto{\pgfqpoint{4.001365in}{1.213710in}}%
\pgfpathlineto{\pgfqpoint{4.001365in}{1.210761in}}%
\pgfpathmoveto{\pgfqpoint{3.992283in}{1.213710in}}%
\pgfpathlineto{\pgfqpoint{3.992283in}{1.213710in}}%
\pgfpathlineto{\pgfqpoint{3.992283in}{1.216659in}}%
\pgfpathlineto{\pgfqpoint{3.996824in}{1.216659in}}%
\pgfpathlineto{\pgfqpoint{3.996824in}{1.213710in}}%
\pgfpathmoveto{\pgfqpoint{3.983201in}{1.219609in}}%
\pgfpathlineto{\pgfqpoint{3.983201in}{1.219609in}}%
\pgfpathlineto{\pgfqpoint{3.983201in}{1.222558in}}%
\pgfpathlineto{\pgfqpoint{3.987742in}{1.222558in}}%
\pgfpathlineto{\pgfqpoint{3.987742in}{1.219609in}}%
\pgfpathmoveto{\pgfqpoint{3.983201in}{1.222558in}}%
\pgfpathlineto{\pgfqpoint{3.983201in}{1.222558in}}%
\pgfpathlineto{\pgfqpoint{3.983201in}{1.225507in}}%
\pgfpathlineto{\pgfqpoint{3.987742in}{1.225507in}}%
\pgfpathlineto{\pgfqpoint{3.987742in}{1.222558in}}%
\pgfpathmoveto{\pgfqpoint{3.987742in}{1.219609in}}%
\pgfpathlineto{\pgfqpoint{3.987742in}{1.219609in}}%
\pgfpathlineto{\pgfqpoint{3.987742in}{1.222558in}}%
\pgfpathlineto{\pgfqpoint{3.992283in}{1.222558in}}%
\pgfpathlineto{\pgfqpoint{3.992283in}{1.219609in}}%
\pgfpathmoveto{\pgfqpoint{4.001365in}{1.207812in}}%
\pgfpathlineto{\pgfqpoint{4.001365in}{1.207812in}}%
\pgfpathlineto{\pgfqpoint{4.001365in}{1.210761in}}%
\pgfpathlineto{\pgfqpoint{4.005906in}{1.210761in}}%
\pgfpathlineto{\pgfqpoint{4.005906in}{1.207812in}}%
\pgfpathmoveto{\pgfqpoint{4.028611in}{1.172421in}}%
\pgfpathlineto{\pgfqpoint{4.028611in}{1.172421in}}%
\pgfpathlineto{\pgfqpoint{4.028611in}{1.175370in}}%
\pgfpathlineto{\pgfqpoint{4.033152in}{1.175370in}}%
\pgfpathlineto{\pgfqpoint{4.033152in}{1.172421in}}%
\pgfpathmoveto{\pgfqpoint{4.028611in}{1.175370in}}%
\pgfpathlineto{\pgfqpoint{4.028611in}{1.175370in}}%
\pgfpathlineto{\pgfqpoint{4.028611in}{1.178320in}}%
\pgfpathlineto{\pgfqpoint{4.033152in}{1.178320in}}%
\pgfpathlineto{\pgfqpoint{4.033152in}{1.175370in}}%
\pgfpathmoveto{\pgfqpoint{4.033152in}{1.172421in}}%
\pgfpathlineto{\pgfqpoint{4.033152in}{1.172421in}}%
\pgfpathlineto{\pgfqpoint{4.033152in}{1.175370in}}%
\pgfpathlineto{\pgfqpoint{4.037692in}{1.175370in}}%
\pgfpathlineto{\pgfqpoint{4.037692in}{1.172421in}}%
\pgfpathmoveto{\pgfqpoint{4.033152in}{1.175370in}}%
\pgfpathlineto{\pgfqpoint{4.033152in}{1.175370in}}%
\pgfpathlineto{\pgfqpoint{4.033152in}{1.178320in}}%
\pgfpathlineto{\pgfqpoint{4.037692in}{1.178320in}}%
\pgfpathlineto{\pgfqpoint{4.037692in}{1.175370in}}%
\pgfpathmoveto{\pgfqpoint{4.028611in}{1.178320in}}%
\pgfpathlineto{\pgfqpoint{4.028611in}{1.178320in}}%
\pgfpathlineto{\pgfqpoint{4.028611in}{1.181269in}}%
\pgfpathlineto{\pgfqpoint{4.033152in}{1.181269in}}%
\pgfpathlineto{\pgfqpoint{4.033152in}{1.178320in}}%
\pgfpathmoveto{\pgfqpoint{4.028611in}{1.181269in}}%
\pgfpathlineto{\pgfqpoint{4.028611in}{1.181269in}}%
\pgfpathlineto{\pgfqpoint{4.028611in}{1.184218in}}%
\pgfpathlineto{\pgfqpoint{4.033152in}{1.184218in}}%
\pgfpathlineto{\pgfqpoint{4.033152in}{1.181269in}}%
\pgfpathmoveto{\pgfqpoint{4.033152in}{1.178320in}}%
\pgfpathlineto{\pgfqpoint{4.033152in}{1.178320in}}%
\pgfpathlineto{\pgfqpoint{4.033152in}{1.181269in}}%
\pgfpathlineto{\pgfqpoint{4.037692in}{1.181269in}}%
\pgfpathlineto{\pgfqpoint{4.037692in}{1.178320in}}%
\pgfpathmoveto{\pgfqpoint{4.037692in}{1.166523in}}%
\pgfpathlineto{\pgfqpoint{4.037692in}{1.166523in}}%
\pgfpathlineto{\pgfqpoint{4.037692in}{1.169472in}}%
\pgfpathlineto{\pgfqpoint{4.042233in}{1.169472in}}%
\pgfpathlineto{\pgfqpoint{4.042233in}{1.166523in}}%
\pgfpathmoveto{\pgfqpoint{4.037692in}{1.169472in}}%
\pgfpathlineto{\pgfqpoint{4.037692in}{1.169472in}}%
\pgfpathlineto{\pgfqpoint{4.037692in}{1.172421in}}%
\pgfpathlineto{\pgfqpoint{4.042233in}{1.172421in}}%
\pgfpathlineto{\pgfqpoint{4.042233in}{1.169472in}}%
\pgfpathmoveto{\pgfqpoint{4.042233in}{1.166523in}}%
\pgfpathlineto{\pgfqpoint{4.042233in}{1.166523in}}%
\pgfpathlineto{\pgfqpoint{4.042233in}{1.169472in}}%
\pgfpathlineto{\pgfqpoint{4.046774in}{1.169472in}}%
\pgfpathlineto{\pgfqpoint{4.046774in}{1.166523in}}%
\pgfpathmoveto{\pgfqpoint{4.042233in}{1.169472in}}%
\pgfpathlineto{\pgfqpoint{4.042233in}{1.169472in}}%
\pgfpathlineto{\pgfqpoint{4.042233in}{1.172421in}}%
\pgfpathlineto{\pgfqpoint{4.046774in}{1.172421in}}%
\pgfpathlineto{\pgfqpoint{4.046774in}{1.169472in}}%
\pgfpathmoveto{\pgfqpoint{4.046774in}{1.160624in}}%
\pgfpathlineto{\pgfqpoint{4.046774in}{1.160624in}}%
\pgfpathlineto{\pgfqpoint{4.046774in}{1.163574in}}%
\pgfpathlineto{\pgfqpoint{4.051315in}{1.163574in}}%
\pgfpathlineto{\pgfqpoint{4.051315in}{1.160624in}}%
\pgfpathmoveto{\pgfqpoint{4.046774in}{1.163574in}}%
\pgfpathlineto{\pgfqpoint{4.046774in}{1.163574in}}%
\pgfpathlineto{\pgfqpoint{4.046774in}{1.166523in}}%
\pgfpathlineto{\pgfqpoint{4.051315in}{1.166523in}}%
\pgfpathlineto{\pgfqpoint{4.051315in}{1.163574in}}%
\pgfpathmoveto{\pgfqpoint{4.051315in}{1.160624in}}%
\pgfpathlineto{\pgfqpoint{4.051315in}{1.160624in}}%
\pgfpathlineto{\pgfqpoint{4.051315in}{1.163574in}}%
\pgfpathlineto{\pgfqpoint{4.055856in}{1.163574in}}%
\pgfpathlineto{\pgfqpoint{4.055856in}{1.160624in}}%
\pgfpathmoveto{\pgfqpoint{4.051315in}{1.163574in}}%
\pgfpathlineto{\pgfqpoint{4.051315in}{1.163574in}}%
\pgfpathlineto{\pgfqpoint{4.051315in}{1.166523in}}%
\pgfpathlineto{\pgfqpoint{4.055856in}{1.166523in}}%
\pgfpathlineto{\pgfqpoint{4.055856in}{1.163574in}}%
\pgfpathmoveto{\pgfqpoint{4.046774in}{1.166523in}}%
\pgfpathlineto{\pgfqpoint{4.046774in}{1.166523in}}%
\pgfpathlineto{\pgfqpoint{4.046774in}{1.169472in}}%
\pgfpathlineto{\pgfqpoint{4.051315in}{1.169472in}}%
\pgfpathlineto{\pgfqpoint{4.051315in}{1.166523in}}%
\pgfpathmoveto{\pgfqpoint{4.037692in}{1.172421in}}%
\pgfpathlineto{\pgfqpoint{4.037692in}{1.172421in}}%
\pgfpathlineto{\pgfqpoint{4.037692in}{1.175370in}}%
\pgfpathlineto{\pgfqpoint{4.042233in}{1.175370in}}%
\pgfpathlineto{\pgfqpoint{4.042233in}{1.172421in}}%
\pgfpathmoveto{\pgfqpoint{4.037692in}{1.175370in}}%
\pgfpathlineto{\pgfqpoint{4.037692in}{1.175370in}}%
\pgfpathlineto{\pgfqpoint{4.037692in}{1.178320in}}%
\pgfpathlineto{\pgfqpoint{4.042233in}{1.178320in}}%
\pgfpathlineto{\pgfqpoint{4.042233in}{1.175370in}}%
\pgfpathmoveto{\pgfqpoint{4.042233in}{1.172421in}}%
\pgfpathlineto{\pgfqpoint{4.042233in}{1.172421in}}%
\pgfpathlineto{\pgfqpoint{4.042233in}{1.175370in}}%
\pgfpathlineto{\pgfqpoint{4.046774in}{1.175370in}}%
\pgfpathlineto{\pgfqpoint{4.046774in}{1.172421in}}%
\pgfpathmoveto{\pgfqpoint{4.019529in}{1.184218in}}%
\pgfpathlineto{\pgfqpoint{4.019529in}{1.184218in}}%
\pgfpathlineto{\pgfqpoint{4.019529in}{1.187167in}}%
\pgfpathlineto{\pgfqpoint{4.024070in}{1.187167in}}%
\pgfpathlineto{\pgfqpoint{4.024070in}{1.184218in}}%
\pgfpathmoveto{\pgfqpoint{4.019529in}{1.187167in}}%
\pgfpathlineto{\pgfqpoint{4.019529in}{1.187167in}}%
\pgfpathlineto{\pgfqpoint{4.019529in}{1.190117in}}%
\pgfpathlineto{\pgfqpoint{4.024070in}{1.190117in}}%
\pgfpathlineto{\pgfqpoint{4.024070in}{1.187167in}}%
\pgfpathmoveto{\pgfqpoint{4.024070in}{1.184218in}}%
\pgfpathlineto{\pgfqpoint{4.024070in}{1.184218in}}%
\pgfpathlineto{\pgfqpoint{4.024070in}{1.187167in}}%
\pgfpathlineto{\pgfqpoint{4.028611in}{1.187167in}}%
\pgfpathlineto{\pgfqpoint{4.028611in}{1.184218in}}%
\pgfpathmoveto{\pgfqpoint{4.024070in}{1.187167in}}%
\pgfpathlineto{\pgfqpoint{4.024070in}{1.187167in}}%
\pgfpathlineto{\pgfqpoint{4.024070in}{1.190117in}}%
\pgfpathlineto{\pgfqpoint{4.028611in}{1.190117in}}%
\pgfpathlineto{\pgfqpoint{4.028611in}{1.187167in}}%
\pgfpathmoveto{\pgfqpoint{4.019529in}{1.190117in}}%
\pgfpathlineto{\pgfqpoint{4.019529in}{1.190117in}}%
\pgfpathlineto{\pgfqpoint{4.019529in}{1.193066in}}%
\pgfpathlineto{\pgfqpoint{4.024070in}{1.193066in}}%
\pgfpathlineto{\pgfqpoint{4.024070in}{1.190117in}}%
\pgfpathmoveto{\pgfqpoint{4.028611in}{1.184218in}}%
\pgfpathlineto{\pgfqpoint{4.028611in}{1.184218in}}%
\pgfpathlineto{\pgfqpoint{4.028611in}{1.187167in}}%
\pgfpathlineto{\pgfqpoint{4.033152in}{1.187167in}}%
\pgfpathlineto{\pgfqpoint{4.033152in}{1.184218in}}%
\pgfpathmoveto{\pgfqpoint{4.055856in}{1.160624in}}%
\pgfpathlineto{\pgfqpoint{4.055856in}{1.160624in}}%
\pgfpathlineto{\pgfqpoint{4.055856in}{1.163574in}}%
\pgfpathlineto{\pgfqpoint{4.060397in}{1.163574in}}%
\pgfpathlineto{\pgfqpoint{4.060397in}{1.160624in}}%
\pgfpathmoveto{\pgfqpoint{3.946873in}{1.254999in}}%
\pgfpathlineto{\pgfqpoint{3.946873in}{1.254999in}}%
\pgfpathlineto{\pgfqpoint{3.946873in}{1.257949in}}%
\pgfpathlineto{\pgfqpoint{3.951414in}{1.257949in}}%
\pgfpathlineto{\pgfqpoint{3.951414in}{1.254999in}}%
\pgfpathmoveto{\pgfqpoint{4.092184in}{0.989569in}}%
\pgfpathlineto{\pgfqpoint{4.092184in}{0.989569in}}%
\pgfpathlineto{\pgfqpoint{4.092184in}{0.992518in}}%
\pgfpathlineto{\pgfqpoint{4.096725in}{0.992518in}}%
\pgfpathlineto{\pgfqpoint{4.096725in}{0.989569in}}%
\pgfpathmoveto{\pgfqpoint{4.092184in}{0.992518in}}%
\pgfpathlineto{\pgfqpoint{4.092184in}{0.992518in}}%
\pgfpathlineto{\pgfqpoint{4.092184in}{0.995467in}}%
\pgfpathlineto{\pgfqpoint{4.096725in}{0.995467in}}%
\pgfpathlineto{\pgfqpoint{4.096725in}{0.992518in}}%
\pgfpathmoveto{\pgfqpoint{4.096725in}{0.992518in}}%
\pgfpathlineto{\pgfqpoint{4.096725in}{0.992518in}}%
\pgfpathlineto{\pgfqpoint{4.096725in}{0.995467in}}%
\pgfpathlineto{\pgfqpoint{4.101266in}{0.995467in}}%
\pgfpathlineto{\pgfqpoint{4.101266in}{0.992518in}}%
\pgfpathmoveto{\pgfqpoint{4.092184in}{0.995467in}}%
\pgfpathlineto{\pgfqpoint{4.092184in}{0.995467in}}%
\pgfpathlineto{\pgfqpoint{4.092184in}{0.998416in}}%
\pgfpathlineto{\pgfqpoint{4.096725in}{0.998416in}}%
\pgfpathlineto{\pgfqpoint{4.096725in}{0.995467in}}%
\pgfpathmoveto{\pgfqpoint{4.092184in}{0.998416in}}%
\pgfpathlineto{\pgfqpoint{4.092184in}{0.998416in}}%
\pgfpathlineto{\pgfqpoint{4.092184in}{1.001365in}}%
\pgfpathlineto{\pgfqpoint{4.096725in}{1.001365in}}%
\pgfpathlineto{\pgfqpoint{4.096725in}{0.998416in}}%
\pgfpathmoveto{\pgfqpoint{4.096725in}{0.995467in}}%
\pgfpathlineto{\pgfqpoint{4.096725in}{0.995467in}}%
\pgfpathlineto{\pgfqpoint{4.096725in}{0.998416in}}%
\pgfpathlineto{\pgfqpoint{4.101266in}{0.998416in}}%
\pgfpathlineto{\pgfqpoint{4.101266in}{0.995467in}}%
\pgfpathmoveto{\pgfqpoint{4.096725in}{0.998416in}}%
\pgfpathlineto{\pgfqpoint{4.096725in}{0.998416in}}%
\pgfpathlineto{\pgfqpoint{4.096725in}{1.001365in}}%
\pgfpathlineto{\pgfqpoint{4.101266in}{1.001365in}}%
\pgfpathlineto{\pgfqpoint{4.101266in}{0.998416in}}%
\pgfpathmoveto{\pgfqpoint{4.101266in}{0.995467in}}%
\pgfpathlineto{\pgfqpoint{4.101266in}{0.995467in}}%
\pgfpathlineto{\pgfqpoint{4.101266in}{0.998416in}}%
\pgfpathlineto{\pgfqpoint{4.105807in}{0.998416in}}%
\pgfpathlineto{\pgfqpoint{4.105807in}{0.995467in}}%
\pgfpathmoveto{\pgfqpoint{4.101266in}{0.998416in}}%
\pgfpathlineto{\pgfqpoint{4.101266in}{0.998416in}}%
\pgfpathlineto{\pgfqpoint{4.101266in}{1.001365in}}%
\pgfpathlineto{\pgfqpoint{4.105807in}{1.001365in}}%
\pgfpathlineto{\pgfqpoint{4.105807in}{0.998416in}}%
\pgfpathmoveto{\pgfqpoint{4.105807in}{0.998416in}}%
\pgfpathlineto{\pgfqpoint{4.105807in}{0.998416in}}%
\pgfpathlineto{\pgfqpoint{4.105807in}{1.001365in}}%
\pgfpathlineto{\pgfqpoint{4.110348in}{1.001365in}}%
\pgfpathlineto{\pgfqpoint{4.110348in}{0.998416in}}%
\pgfpathmoveto{\pgfqpoint{4.101266in}{1.001365in}}%
\pgfpathlineto{\pgfqpoint{4.101266in}{1.001365in}}%
\pgfpathlineto{\pgfqpoint{4.101266in}{1.004315in}}%
\pgfpathlineto{\pgfqpoint{4.105807in}{1.004315in}}%
\pgfpathlineto{\pgfqpoint{4.105807in}{1.001365in}}%
\pgfpathmoveto{\pgfqpoint{4.101266in}{1.004315in}}%
\pgfpathlineto{\pgfqpoint{4.101266in}{1.004315in}}%
\pgfpathlineto{\pgfqpoint{4.101266in}{1.007264in}}%
\pgfpathlineto{\pgfqpoint{4.105807in}{1.007264in}}%
\pgfpathlineto{\pgfqpoint{4.105807in}{1.004315in}}%
\pgfpathmoveto{\pgfqpoint{4.105807in}{1.001365in}}%
\pgfpathlineto{\pgfqpoint{4.105807in}{1.001365in}}%
\pgfpathlineto{\pgfqpoint{4.105807in}{1.004315in}}%
\pgfpathlineto{\pgfqpoint{4.110348in}{1.004315in}}%
\pgfpathlineto{\pgfqpoint{4.110348in}{1.001365in}}%
\pgfpathmoveto{\pgfqpoint{4.105807in}{1.004315in}}%
\pgfpathlineto{\pgfqpoint{4.105807in}{1.004315in}}%
\pgfpathlineto{\pgfqpoint{4.105807in}{1.007264in}}%
\pgfpathlineto{\pgfqpoint{4.110348in}{1.007264in}}%
\pgfpathlineto{\pgfqpoint{4.110348in}{1.004315in}}%
\pgfpathmoveto{\pgfqpoint{4.110348in}{1.004315in}}%
\pgfpathlineto{\pgfqpoint{4.110348in}{1.004315in}}%
\pgfpathlineto{\pgfqpoint{4.110348in}{1.007264in}}%
\pgfpathlineto{\pgfqpoint{4.114889in}{1.007264in}}%
\pgfpathlineto{\pgfqpoint{4.114889in}{1.004315in}}%
\pgfpathmoveto{\pgfqpoint{4.110348in}{1.007264in}}%
\pgfpathlineto{\pgfqpoint{4.110348in}{1.007264in}}%
\pgfpathlineto{\pgfqpoint{4.110348in}{1.010213in}}%
\pgfpathlineto{\pgfqpoint{4.114889in}{1.010213in}}%
\pgfpathlineto{\pgfqpoint{4.114889in}{1.007264in}}%
\pgfpathmoveto{\pgfqpoint{4.110348in}{1.010213in}}%
\pgfpathlineto{\pgfqpoint{4.110348in}{1.010213in}}%
\pgfpathlineto{\pgfqpoint{4.110348in}{1.013162in}}%
\pgfpathlineto{\pgfqpoint{4.114889in}{1.013162in}}%
\pgfpathlineto{\pgfqpoint{4.114889in}{1.010213in}}%
\pgfpathmoveto{\pgfqpoint{4.114889in}{1.007264in}}%
\pgfpathlineto{\pgfqpoint{4.114889in}{1.007264in}}%
\pgfpathlineto{\pgfqpoint{4.114889in}{1.010213in}}%
\pgfpathlineto{\pgfqpoint{4.119430in}{1.010213in}}%
\pgfpathlineto{\pgfqpoint{4.119430in}{1.007264in}}%
\pgfpathmoveto{\pgfqpoint{4.114889in}{1.010213in}}%
\pgfpathlineto{\pgfqpoint{4.114889in}{1.010213in}}%
\pgfpathlineto{\pgfqpoint{4.114889in}{1.013162in}}%
\pgfpathlineto{\pgfqpoint{4.119430in}{1.013162in}}%
\pgfpathlineto{\pgfqpoint{4.119430in}{1.010213in}}%
\pgfpathmoveto{\pgfqpoint{4.119430in}{1.010213in}}%
\pgfpathlineto{\pgfqpoint{4.119430in}{1.010213in}}%
\pgfpathlineto{\pgfqpoint{4.119430in}{1.013162in}}%
\pgfpathlineto{\pgfqpoint{4.123971in}{1.013162in}}%
\pgfpathlineto{\pgfqpoint{4.123971in}{1.010213in}}%
\pgfpathmoveto{\pgfqpoint{4.119430in}{1.013162in}}%
\pgfpathlineto{\pgfqpoint{4.119430in}{1.013162in}}%
\pgfpathlineto{\pgfqpoint{4.119430in}{1.016111in}}%
\pgfpathlineto{\pgfqpoint{4.123971in}{1.016111in}}%
\pgfpathlineto{\pgfqpoint{4.123971in}{1.013162in}}%
\pgfpathmoveto{\pgfqpoint{4.119430in}{1.016111in}}%
\pgfpathlineto{\pgfqpoint{4.119430in}{1.016111in}}%
\pgfpathlineto{\pgfqpoint{4.119430in}{1.019060in}}%
\pgfpathlineto{\pgfqpoint{4.123971in}{1.019060in}}%
\pgfpathlineto{\pgfqpoint{4.123971in}{1.016111in}}%
\pgfpathmoveto{\pgfqpoint{4.123971in}{1.013162in}}%
\pgfpathlineto{\pgfqpoint{4.123971in}{1.013162in}}%
\pgfpathlineto{\pgfqpoint{4.123971in}{1.016111in}}%
\pgfpathlineto{\pgfqpoint{4.128512in}{1.016111in}}%
\pgfpathlineto{\pgfqpoint{4.128512in}{1.013162in}}%
\pgfpathmoveto{\pgfqpoint{4.123971in}{1.016111in}}%
\pgfpathlineto{\pgfqpoint{4.123971in}{1.016111in}}%
\pgfpathlineto{\pgfqpoint{4.123971in}{1.019060in}}%
\pgfpathlineto{\pgfqpoint{4.128512in}{1.019060in}}%
\pgfpathlineto{\pgfqpoint{4.128512in}{1.016111in}}%
\pgfpathmoveto{\pgfqpoint{4.128512in}{1.016111in}}%
\pgfpathlineto{\pgfqpoint{4.128512in}{1.016111in}}%
\pgfpathlineto{\pgfqpoint{4.128512in}{1.019060in}}%
\pgfpathlineto{\pgfqpoint{4.133053in}{1.019060in}}%
\pgfpathlineto{\pgfqpoint{4.133053in}{1.016111in}}%
\pgfpathmoveto{\pgfqpoint{4.128512in}{1.019060in}}%
\pgfpathlineto{\pgfqpoint{4.128512in}{1.019060in}}%
\pgfpathlineto{\pgfqpoint{4.128512in}{1.022010in}}%
\pgfpathlineto{\pgfqpoint{4.133053in}{1.022010in}}%
\pgfpathlineto{\pgfqpoint{4.133053in}{1.019060in}}%
\pgfpathmoveto{\pgfqpoint{4.128512in}{1.022010in}}%
\pgfpathlineto{\pgfqpoint{4.128512in}{1.022010in}}%
\pgfpathlineto{\pgfqpoint{4.128512in}{1.024959in}}%
\pgfpathlineto{\pgfqpoint{4.133053in}{1.024959in}}%
\pgfpathlineto{\pgfqpoint{4.133053in}{1.022010in}}%
\pgfpathmoveto{\pgfqpoint{4.133053in}{1.022010in}}%
\pgfpathlineto{\pgfqpoint{4.133053in}{1.022010in}}%
\pgfpathlineto{\pgfqpoint{4.133053in}{1.024959in}}%
\pgfpathlineto{\pgfqpoint{4.137594in}{1.024959in}}%
\pgfpathlineto{\pgfqpoint{4.137594in}{1.022010in}}%
\pgfpathmoveto{\pgfqpoint{4.128512in}{1.024959in}}%
\pgfpathlineto{\pgfqpoint{4.128512in}{1.024959in}}%
\pgfpathlineto{\pgfqpoint{4.128512in}{1.027908in}}%
\pgfpathlineto{\pgfqpoint{4.133053in}{1.027908in}}%
\pgfpathlineto{\pgfqpoint{4.133053in}{1.024959in}}%
\pgfpathmoveto{\pgfqpoint{4.128512in}{1.027908in}}%
\pgfpathlineto{\pgfqpoint{4.128512in}{1.027908in}}%
\pgfpathlineto{\pgfqpoint{4.128512in}{1.030857in}}%
\pgfpathlineto{\pgfqpoint{4.133053in}{1.030857in}}%
\pgfpathlineto{\pgfqpoint{4.133053in}{1.027908in}}%
\pgfpathmoveto{\pgfqpoint{4.133053in}{1.024959in}}%
\pgfpathlineto{\pgfqpoint{4.133053in}{1.024959in}}%
\pgfpathlineto{\pgfqpoint{4.133053in}{1.027908in}}%
\pgfpathlineto{\pgfqpoint{4.137594in}{1.027908in}}%
\pgfpathlineto{\pgfqpoint{4.137594in}{1.024959in}}%
\pgfpathmoveto{\pgfqpoint{4.133053in}{1.027908in}}%
\pgfpathlineto{\pgfqpoint{4.133053in}{1.027908in}}%
\pgfpathlineto{\pgfqpoint{4.133053in}{1.030857in}}%
\pgfpathlineto{\pgfqpoint{4.137594in}{1.030857in}}%
\pgfpathlineto{\pgfqpoint{4.137594in}{1.027908in}}%
\pgfpathmoveto{\pgfqpoint{4.137594in}{1.024959in}}%
\pgfpathlineto{\pgfqpoint{4.137594in}{1.024959in}}%
\pgfpathlineto{\pgfqpoint{4.137594in}{1.027908in}}%
\pgfpathlineto{\pgfqpoint{4.142135in}{1.027908in}}%
\pgfpathlineto{\pgfqpoint{4.142135in}{1.024959in}}%
\pgfpathmoveto{\pgfqpoint{4.137594in}{1.027908in}}%
\pgfpathlineto{\pgfqpoint{4.137594in}{1.027908in}}%
\pgfpathlineto{\pgfqpoint{4.137594in}{1.030857in}}%
\pgfpathlineto{\pgfqpoint{4.142135in}{1.030857in}}%
\pgfpathlineto{\pgfqpoint{4.142135in}{1.027908in}}%
\pgfpathmoveto{\pgfqpoint{4.142135in}{1.027908in}}%
\pgfpathlineto{\pgfqpoint{4.142135in}{1.027908in}}%
\pgfpathlineto{\pgfqpoint{4.142135in}{1.030857in}}%
\pgfpathlineto{\pgfqpoint{4.146676in}{1.030857in}}%
\pgfpathlineto{\pgfqpoint{4.146676in}{1.027908in}}%
\pgfpathmoveto{\pgfqpoint{4.137594in}{1.030857in}}%
\pgfpathlineto{\pgfqpoint{4.137594in}{1.030857in}}%
\pgfpathlineto{\pgfqpoint{4.137594in}{1.033806in}}%
\pgfpathlineto{\pgfqpoint{4.142135in}{1.033806in}}%
\pgfpathlineto{\pgfqpoint{4.142135in}{1.030857in}}%
\pgfpathmoveto{\pgfqpoint{4.137594in}{1.033806in}}%
\pgfpathlineto{\pgfqpoint{4.137594in}{1.033806in}}%
\pgfpathlineto{\pgfqpoint{4.137594in}{1.036756in}}%
\pgfpathlineto{\pgfqpoint{4.142135in}{1.036756in}}%
\pgfpathlineto{\pgfqpoint{4.142135in}{1.033806in}}%
\pgfpathmoveto{\pgfqpoint{4.142135in}{1.030857in}}%
\pgfpathlineto{\pgfqpoint{4.142135in}{1.030857in}}%
\pgfpathlineto{\pgfqpoint{4.142135in}{1.033806in}}%
\pgfpathlineto{\pgfqpoint{4.146676in}{1.033806in}}%
\pgfpathlineto{\pgfqpoint{4.146676in}{1.030857in}}%
\pgfpathmoveto{\pgfqpoint{4.142135in}{1.033806in}}%
\pgfpathlineto{\pgfqpoint{4.142135in}{1.033806in}}%
\pgfpathlineto{\pgfqpoint{4.142135in}{1.036756in}}%
\pgfpathlineto{\pgfqpoint{4.146676in}{1.036756in}}%
\pgfpathlineto{\pgfqpoint{4.146676in}{1.033806in}}%
\pgfpathmoveto{\pgfqpoint{4.146676in}{1.030857in}}%
\pgfpathlineto{\pgfqpoint{4.146676in}{1.030857in}}%
\pgfpathlineto{\pgfqpoint{4.146676in}{1.033806in}}%
\pgfpathlineto{\pgfqpoint{4.151217in}{1.033806in}}%
\pgfpathlineto{\pgfqpoint{4.151217in}{1.030857in}}%
\pgfpathmoveto{\pgfqpoint{4.146676in}{1.033806in}}%
\pgfpathlineto{\pgfqpoint{4.146676in}{1.033806in}}%
\pgfpathlineto{\pgfqpoint{4.146676in}{1.036756in}}%
\pgfpathlineto{\pgfqpoint{4.151217in}{1.036756in}}%
\pgfpathlineto{\pgfqpoint{4.151217in}{1.033806in}}%
\pgfpathmoveto{\pgfqpoint{4.151217in}{1.033806in}}%
\pgfpathlineto{\pgfqpoint{4.151217in}{1.033806in}}%
\pgfpathlineto{\pgfqpoint{4.151217in}{1.036756in}}%
\pgfpathlineto{\pgfqpoint{4.155758in}{1.036756in}}%
\pgfpathlineto{\pgfqpoint{4.155758in}{1.033806in}}%
\pgfpathmoveto{\pgfqpoint{4.146676in}{1.036756in}}%
\pgfpathlineto{\pgfqpoint{4.146676in}{1.036756in}}%
\pgfpathlineto{\pgfqpoint{4.146676in}{1.039705in}}%
\pgfpathlineto{\pgfqpoint{4.151217in}{1.039705in}}%
\pgfpathlineto{\pgfqpoint{4.151217in}{1.036756in}}%
\pgfpathmoveto{\pgfqpoint{4.146676in}{1.039705in}}%
\pgfpathlineto{\pgfqpoint{4.146676in}{1.039705in}}%
\pgfpathlineto{\pgfqpoint{4.146676in}{1.042654in}}%
\pgfpathlineto{\pgfqpoint{4.151217in}{1.042654in}}%
\pgfpathlineto{\pgfqpoint{4.151217in}{1.039705in}}%
\pgfpathmoveto{\pgfqpoint{4.151217in}{1.036756in}}%
\pgfpathlineto{\pgfqpoint{4.151217in}{1.036756in}}%
\pgfpathlineto{\pgfqpoint{4.151217in}{1.039705in}}%
\pgfpathlineto{\pgfqpoint{4.155758in}{1.039705in}}%
\pgfpathlineto{\pgfqpoint{4.155758in}{1.036756in}}%
\pgfpathmoveto{\pgfqpoint{4.151217in}{1.039705in}}%
\pgfpathlineto{\pgfqpoint{4.151217in}{1.039705in}}%
\pgfpathlineto{\pgfqpoint{4.151217in}{1.042654in}}%
\pgfpathlineto{\pgfqpoint{4.155758in}{1.042654in}}%
\pgfpathlineto{\pgfqpoint{4.155758in}{1.039705in}}%
\pgfpathmoveto{\pgfqpoint{4.155758in}{1.039705in}}%
\pgfpathlineto{\pgfqpoint{4.155758in}{1.039705in}}%
\pgfpathlineto{\pgfqpoint{4.155758in}{1.042654in}}%
\pgfpathlineto{\pgfqpoint{4.160299in}{1.042654in}}%
\pgfpathlineto{\pgfqpoint{4.160299in}{1.039705in}}%
\pgfpathmoveto{\pgfqpoint{4.155758in}{1.042654in}}%
\pgfpathlineto{\pgfqpoint{4.155758in}{1.042654in}}%
\pgfpathlineto{\pgfqpoint{4.155758in}{1.045603in}}%
\pgfpathlineto{\pgfqpoint{4.160299in}{1.045603in}}%
\pgfpathlineto{\pgfqpoint{4.160299in}{1.042654in}}%
\pgfpathmoveto{\pgfqpoint{4.155758in}{1.045603in}}%
\pgfpathlineto{\pgfqpoint{4.155758in}{1.045603in}}%
\pgfpathlineto{\pgfqpoint{4.155758in}{1.048552in}}%
\pgfpathlineto{\pgfqpoint{4.160299in}{1.048552in}}%
\pgfpathlineto{\pgfqpoint{4.160299in}{1.045603in}}%
\pgfpathmoveto{\pgfqpoint{4.160299in}{1.042654in}}%
\pgfpathlineto{\pgfqpoint{4.160299in}{1.042654in}}%
\pgfpathlineto{\pgfqpoint{4.160299in}{1.045603in}}%
\pgfpathlineto{\pgfqpoint{4.164840in}{1.045603in}}%
\pgfpathlineto{\pgfqpoint{4.164840in}{1.042654in}}%
\pgfpathmoveto{\pgfqpoint{4.160299in}{1.045603in}}%
\pgfpathlineto{\pgfqpoint{4.160299in}{1.045603in}}%
\pgfpathlineto{\pgfqpoint{4.160299in}{1.048552in}}%
\pgfpathlineto{\pgfqpoint{4.164840in}{1.048552in}}%
\pgfpathlineto{\pgfqpoint{4.164840in}{1.045603in}}%
\pgfpathmoveto{\pgfqpoint{4.164840in}{1.045603in}}%
\pgfpathlineto{\pgfqpoint{4.164840in}{1.045603in}}%
\pgfpathlineto{\pgfqpoint{4.164840in}{1.048552in}}%
\pgfpathlineto{\pgfqpoint{4.169381in}{1.048552in}}%
\pgfpathlineto{\pgfqpoint{4.169381in}{1.045603in}}%
\pgfpathmoveto{\pgfqpoint{4.164840in}{1.048552in}}%
\pgfpathlineto{\pgfqpoint{4.164840in}{1.048552in}}%
\pgfpathlineto{\pgfqpoint{4.164840in}{1.051501in}}%
\pgfpathlineto{\pgfqpoint{4.169381in}{1.051501in}}%
\pgfpathlineto{\pgfqpoint{4.169381in}{1.048552in}}%
\pgfpathmoveto{\pgfqpoint{4.164840in}{1.051501in}}%
\pgfpathlineto{\pgfqpoint{4.164840in}{1.051501in}}%
\pgfpathlineto{\pgfqpoint{4.164840in}{1.054451in}}%
\pgfpathlineto{\pgfqpoint{4.169381in}{1.054451in}}%
\pgfpathlineto{\pgfqpoint{4.169381in}{1.051501in}}%
\pgfpathmoveto{\pgfqpoint{4.169381in}{1.048552in}}%
\pgfpathlineto{\pgfqpoint{4.169381in}{1.048552in}}%
\pgfpathlineto{\pgfqpoint{4.169381in}{1.051501in}}%
\pgfpathlineto{\pgfqpoint{4.173922in}{1.051501in}}%
\pgfpathlineto{\pgfqpoint{4.173922in}{1.048552in}}%
\pgfpathmoveto{\pgfqpoint{4.169381in}{1.051501in}}%
\pgfpathlineto{\pgfqpoint{4.169381in}{1.051501in}}%
\pgfpathlineto{\pgfqpoint{4.169381in}{1.054451in}}%
\pgfpathlineto{\pgfqpoint{4.173922in}{1.054451in}}%
\pgfpathlineto{\pgfqpoint{4.173922in}{1.051501in}}%
\pgfpathmoveto{\pgfqpoint{4.173922in}{1.051501in}}%
\pgfpathlineto{\pgfqpoint{4.173922in}{1.051501in}}%
\pgfpathlineto{\pgfqpoint{4.173922in}{1.054451in}}%
\pgfpathlineto{\pgfqpoint{4.178463in}{1.054451in}}%
\pgfpathlineto{\pgfqpoint{4.178463in}{1.051501in}}%
\pgfpathmoveto{\pgfqpoint{4.164840in}{1.054451in}}%
\pgfpathlineto{\pgfqpoint{4.164840in}{1.054451in}}%
\pgfpathlineto{\pgfqpoint{4.164840in}{1.057400in}}%
\pgfpathlineto{\pgfqpoint{4.169381in}{1.057400in}}%
\pgfpathlineto{\pgfqpoint{4.169381in}{1.054451in}}%
\pgfpathmoveto{\pgfqpoint{4.164840in}{1.057400in}}%
\pgfpathlineto{\pgfqpoint{4.164840in}{1.057400in}}%
\pgfpathlineto{\pgfqpoint{4.164840in}{1.060349in}}%
\pgfpathlineto{\pgfqpoint{4.169381in}{1.060349in}}%
\pgfpathlineto{\pgfqpoint{4.169381in}{1.057400in}}%
\pgfpathmoveto{\pgfqpoint{4.169381in}{1.054451in}}%
\pgfpathlineto{\pgfqpoint{4.169381in}{1.054451in}}%
\pgfpathlineto{\pgfqpoint{4.169381in}{1.057400in}}%
\pgfpathlineto{\pgfqpoint{4.173922in}{1.057400in}}%
\pgfpathlineto{\pgfqpoint{4.173922in}{1.054451in}}%
\pgfpathmoveto{\pgfqpoint{4.169381in}{1.057400in}}%
\pgfpathlineto{\pgfqpoint{4.169381in}{1.057400in}}%
\pgfpathlineto{\pgfqpoint{4.169381in}{1.060349in}}%
\pgfpathlineto{\pgfqpoint{4.173922in}{1.060349in}}%
\pgfpathlineto{\pgfqpoint{4.173922in}{1.057400in}}%
\pgfpathmoveto{\pgfqpoint{4.164840in}{1.060349in}}%
\pgfpathlineto{\pgfqpoint{4.164840in}{1.060349in}}%
\pgfpathlineto{\pgfqpoint{4.164840in}{1.063298in}}%
\pgfpathlineto{\pgfqpoint{4.169381in}{1.063298in}}%
\pgfpathlineto{\pgfqpoint{4.169381in}{1.060349in}}%
\pgfpathmoveto{\pgfqpoint{4.164840in}{1.063298in}}%
\pgfpathlineto{\pgfqpoint{4.164840in}{1.063298in}}%
\pgfpathlineto{\pgfqpoint{4.164840in}{1.066247in}}%
\pgfpathlineto{\pgfqpoint{4.169381in}{1.066247in}}%
\pgfpathlineto{\pgfqpoint{4.169381in}{1.063298in}}%
\pgfpathmoveto{\pgfqpoint{4.169381in}{1.060349in}}%
\pgfpathlineto{\pgfqpoint{4.169381in}{1.060349in}}%
\pgfpathlineto{\pgfqpoint{4.169381in}{1.063298in}}%
\pgfpathlineto{\pgfqpoint{4.173922in}{1.063298in}}%
\pgfpathlineto{\pgfqpoint{4.173922in}{1.060349in}}%
\pgfpathmoveto{\pgfqpoint{4.173922in}{1.054451in}}%
\pgfpathlineto{\pgfqpoint{4.173922in}{1.054451in}}%
\pgfpathlineto{\pgfqpoint{4.173922in}{1.057400in}}%
\pgfpathlineto{\pgfqpoint{4.178463in}{1.057400in}}%
\pgfpathlineto{\pgfqpoint{4.178463in}{1.054451in}}%
\pgfpathmoveto{\pgfqpoint{4.173922in}{1.057400in}}%
\pgfpathlineto{\pgfqpoint{4.173922in}{1.057400in}}%
\pgfpathlineto{\pgfqpoint{4.173922in}{1.060349in}}%
\pgfpathlineto{\pgfqpoint{4.178463in}{1.060349in}}%
\pgfpathlineto{\pgfqpoint{4.178463in}{1.057400in}}%
\pgfpathmoveto{\pgfqpoint{4.119430in}{1.095740in}}%
\pgfpathlineto{\pgfqpoint{4.119430in}{1.095740in}}%
\pgfpathlineto{\pgfqpoint{4.119430in}{1.098689in}}%
\pgfpathlineto{\pgfqpoint{4.123971in}{1.098689in}}%
\pgfpathlineto{\pgfqpoint{4.123971in}{1.095740in}}%
\pgfpathmoveto{\pgfqpoint{4.119430in}{1.098689in}}%
\pgfpathlineto{\pgfqpoint{4.119430in}{1.098689in}}%
\pgfpathlineto{\pgfqpoint{4.119430in}{1.101639in}}%
\pgfpathlineto{\pgfqpoint{4.123971in}{1.101639in}}%
\pgfpathlineto{\pgfqpoint{4.123971in}{1.098689in}}%
\pgfpathmoveto{\pgfqpoint{4.123971in}{1.095740in}}%
\pgfpathlineto{\pgfqpoint{4.123971in}{1.095740in}}%
\pgfpathlineto{\pgfqpoint{4.123971in}{1.098689in}}%
\pgfpathlineto{\pgfqpoint{4.128512in}{1.098689in}}%
\pgfpathlineto{\pgfqpoint{4.128512in}{1.095740in}}%
\pgfpathmoveto{\pgfqpoint{4.123971in}{1.098689in}}%
\pgfpathlineto{\pgfqpoint{4.123971in}{1.098689in}}%
\pgfpathlineto{\pgfqpoint{4.123971in}{1.101639in}}%
\pgfpathlineto{\pgfqpoint{4.128512in}{1.101639in}}%
\pgfpathlineto{\pgfqpoint{4.128512in}{1.098689in}}%
\pgfpathmoveto{\pgfqpoint{4.110348in}{1.101639in}}%
\pgfpathlineto{\pgfqpoint{4.110348in}{1.101639in}}%
\pgfpathlineto{\pgfqpoint{4.110348in}{1.104588in}}%
\pgfpathlineto{\pgfqpoint{4.114889in}{1.104588in}}%
\pgfpathlineto{\pgfqpoint{4.114889in}{1.101639in}}%
\pgfpathmoveto{\pgfqpoint{4.110348in}{1.104588in}}%
\pgfpathlineto{\pgfqpoint{4.110348in}{1.104588in}}%
\pgfpathlineto{\pgfqpoint{4.110348in}{1.107537in}}%
\pgfpathlineto{\pgfqpoint{4.114889in}{1.107537in}}%
\pgfpathlineto{\pgfqpoint{4.114889in}{1.104588in}}%
\pgfpathmoveto{\pgfqpoint{4.114889in}{1.101639in}}%
\pgfpathlineto{\pgfqpoint{4.114889in}{1.101639in}}%
\pgfpathlineto{\pgfqpoint{4.114889in}{1.104588in}}%
\pgfpathlineto{\pgfqpoint{4.119430in}{1.104588in}}%
\pgfpathlineto{\pgfqpoint{4.119430in}{1.101639in}}%
\pgfpathmoveto{\pgfqpoint{4.114889in}{1.104588in}}%
\pgfpathlineto{\pgfqpoint{4.114889in}{1.104588in}}%
\pgfpathlineto{\pgfqpoint{4.114889in}{1.107537in}}%
\pgfpathlineto{\pgfqpoint{4.119430in}{1.107537in}}%
\pgfpathlineto{\pgfqpoint{4.119430in}{1.104588in}}%
\pgfpathmoveto{\pgfqpoint{4.110348in}{1.107537in}}%
\pgfpathlineto{\pgfqpoint{4.110348in}{1.107537in}}%
\pgfpathlineto{\pgfqpoint{4.110348in}{1.110487in}}%
\pgfpathlineto{\pgfqpoint{4.114889in}{1.110487in}}%
\pgfpathlineto{\pgfqpoint{4.114889in}{1.107537in}}%
\pgfpathmoveto{\pgfqpoint{4.110348in}{1.110487in}}%
\pgfpathlineto{\pgfqpoint{4.110348in}{1.110487in}}%
\pgfpathlineto{\pgfqpoint{4.110348in}{1.113436in}}%
\pgfpathlineto{\pgfqpoint{4.114889in}{1.113436in}}%
\pgfpathlineto{\pgfqpoint{4.114889in}{1.110487in}}%
\pgfpathmoveto{\pgfqpoint{4.114889in}{1.107537in}}%
\pgfpathlineto{\pgfqpoint{4.114889in}{1.107537in}}%
\pgfpathlineto{\pgfqpoint{4.114889in}{1.110487in}}%
\pgfpathlineto{\pgfqpoint{4.119430in}{1.110487in}}%
\pgfpathlineto{\pgfqpoint{4.119430in}{1.107537in}}%
\pgfpathmoveto{\pgfqpoint{4.119430in}{1.101639in}}%
\pgfpathlineto{\pgfqpoint{4.119430in}{1.101639in}}%
\pgfpathlineto{\pgfqpoint{4.119430in}{1.104588in}}%
\pgfpathlineto{\pgfqpoint{4.123971in}{1.104588in}}%
\pgfpathlineto{\pgfqpoint{4.123971in}{1.101639in}}%
\pgfpathmoveto{\pgfqpoint{4.119430in}{1.104588in}}%
\pgfpathlineto{\pgfqpoint{4.119430in}{1.104588in}}%
\pgfpathlineto{\pgfqpoint{4.119430in}{1.107537in}}%
\pgfpathlineto{\pgfqpoint{4.123971in}{1.107537in}}%
\pgfpathlineto{\pgfqpoint{4.123971in}{1.104588in}}%
\pgfpathmoveto{\pgfqpoint{4.123971in}{1.101639in}}%
\pgfpathlineto{\pgfqpoint{4.123971in}{1.101639in}}%
\pgfpathlineto{\pgfqpoint{4.123971in}{1.104588in}}%
\pgfpathlineto{\pgfqpoint{4.128512in}{1.104588in}}%
\pgfpathlineto{\pgfqpoint{4.128512in}{1.101639in}}%
\pgfpathmoveto{\pgfqpoint{4.137594in}{1.078044in}}%
\pgfpathlineto{\pgfqpoint{4.137594in}{1.078044in}}%
\pgfpathlineto{\pgfqpoint{4.137594in}{1.080994in}}%
\pgfpathlineto{\pgfqpoint{4.142135in}{1.080994in}}%
\pgfpathlineto{\pgfqpoint{4.142135in}{1.078044in}}%
\pgfpathmoveto{\pgfqpoint{4.137594in}{1.080994in}}%
\pgfpathlineto{\pgfqpoint{4.137594in}{1.080994in}}%
\pgfpathlineto{\pgfqpoint{4.137594in}{1.083943in}}%
\pgfpathlineto{\pgfqpoint{4.142135in}{1.083943in}}%
\pgfpathlineto{\pgfqpoint{4.142135in}{1.080994in}}%
\pgfpathmoveto{\pgfqpoint{4.142135in}{1.078044in}}%
\pgfpathlineto{\pgfqpoint{4.142135in}{1.078044in}}%
\pgfpathlineto{\pgfqpoint{4.142135in}{1.080994in}}%
\pgfpathlineto{\pgfqpoint{4.146676in}{1.080994in}}%
\pgfpathlineto{\pgfqpoint{4.146676in}{1.078044in}}%
\pgfpathmoveto{\pgfqpoint{4.142135in}{1.080994in}}%
\pgfpathlineto{\pgfqpoint{4.142135in}{1.080994in}}%
\pgfpathlineto{\pgfqpoint{4.142135in}{1.083943in}}%
\pgfpathlineto{\pgfqpoint{4.146676in}{1.083943in}}%
\pgfpathlineto{\pgfqpoint{4.146676in}{1.080994in}}%
\pgfpathmoveto{\pgfqpoint{4.137594in}{1.083943in}}%
\pgfpathlineto{\pgfqpoint{4.137594in}{1.083943in}}%
\pgfpathlineto{\pgfqpoint{4.137594in}{1.086892in}}%
\pgfpathlineto{\pgfqpoint{4.142135in}{1.086892in}}%
\pgfpathlineto{\pgfqpoint{4.142135in}{1.083943in}}%
\pgfpathmoveto{\pgfqpoint{4.137594in}{1.086892in}}%
\pgfpathlineto{\pgfqpoint{4.137594in}{1.086892in}}%
\pgfpathlineto{\pgfqpoint{4.137594in}{1.089842in}}%
\pgfpathlineto{\pgfqpoint{4.142135in}{1.089842in}}%
\pgfpathlineto{\pgfqpoint{4.142135in}{1.086892in}}%
\pgfpathmoveto{\pgfqpoint{4.142135in}{1.083943in}}%
\pgfpathlineto{\pgfqpoint{4.142135in}{1.083943in}}%
\pgfpathlineto{\pgfqpoint{4.142135in}{1.086892in}}%
\pgfpathlineto{\pgfqpoint{4.146676in}{1.086892in}}%
\pgfpathlineto{\pgfqpoint{4.146676in}{1.083943in}}%
\pgfpathmoveto{\pgfqpoint{4.146676in}{1.072146in}}%
\pgfpathlineto{\pgfqpoint{4.146676in}{1.072146in}}%
\pgfpathlineto{\pgfqpoint{4.146676in}{1.075095in}}%
\pgfpathlineto{\pgfqpoint{4.151217in}{1.075095in}}%
\pgfpathlineto{\pgfqpoint{4.151217in}{1.072146in}}%
\pgfpathmoveto{\pgfqpoint{4.146676in}{1.075095in}}%
\pgfpathlineto{\pgfqpoint{4.146676in}{1.075095in}}%
\pgfpathlineto{\pgfqpoint{4.146676in}{1.078044in}}%
\pgfpathlineto{\pgfqpoint{4.151217in}{1.078044in}}%
\pgfpathlineto{\pgfqpoint{4.151217in}{1.075095in}}%
\pgfpathmoveto{\pgfqpoint{4.151217in}{1.072146in}}%
\pgfpathlineto{\pgfqpoint{4.151217in}{1.072146in}}%
\pgfpathlineto{\pgfqpoint{4.151217in}{1.075095in}}%
\pgfpathlineto{\pgfqpoint{4.155758in}{1.075095in}}%
\pgfpathlineto{\pgfqpoint{4.155758in}{1.072146in}}%
\pgfpathmoveto{\pgfqpoint{4.151217in}{1.075095in}}%
\pgfpathlineto{\pgfqpoint{4.151217in}{1.075095in}}%
\pgfpathlineto{\pgfqpoint{4.151217in}{1.078044in}}%
\pgfpathlineto{\pgfqpoint{4.155758in}{1.078044in}}%
\pgfpathlineto{\pgfqpoint{4.155758in}{1.075095in}}%
\pgfpathmoveto{\pgfqpoint{4.155758in}{1.066247in}}%
\pgfpathlineto{\pgfqpoint{4.155758in}{1.066247in}}%
\pgfpathlineto{\pgfqpoint{4.155758in}{1.069197in}}%
\pgfpathlineto{\pgfqpoint{4.160299in}{1.069197in}}%
\pgfpathlineto{\pgfqpoint{4.160299in}{1.066247in}}%
\pgfpathmoveto{\pgfqpoint{4.155758in}{1.069197in}}%
\pgfpathlineto{\pgfqpoint{4.155758in}{1.069197in}}%
\pgfpathlineto{\pgfqpoint{4.155758in}{1.072146in}}%
\pgfpathlineto{\pgfqpoint{4.160299in}{1.072146in}}%
\pgfpathlineto{\pgfqpoint{4.160299in}{1.069197in}}%
\pgfpathmoveto{\pgfqpoint{4.160299in}{1.066247in}}%
\pgfpathlineto{\pgfqpoint{4.160299in}{1.066247in}}%
\pgfpathlineto{\pgfqpoint{4.160299in}{1.069197in}}%
\pgfpathlineto{\pgfqpoint{4.164840in}{1.069197in}}%
\pgfpathlineto{\pgfqpoint{4.164840in}{1.066247in}}%
\pgfpathmoveto{\pgfqpoint{4.160299in}{1.069197in}}%
\pgfpathlineto{\pgfqpoint{4.160299in}{1.069197in}}%
\pgfpathlineto{\pgfqpoint{4.160299in}{1.072146in}}%
\pgfpathlineto{\pgfqpoint{4.164840in}{1.072146in}}%
\pgfpathlineto{\pgfqpoint{4.164840in}{1.069197in}}%
\pgfpathmoveto{\pgfqpoint{4.155758in}{1.072146in}}%
\pgfpathlineto{\pgfqpoint{4.155758in}{1.072146in}}%
\pgfpathlineto{\pgfqpoint{4.155758in}{1.075095in}}%
\pgfpathlineto{\pgfqpoint{4.160299in}{1.075095in}}%
\pgfpathlineto{\pgfqpoint{4.160299in}{1.072146in}}%
\pgfpathmoveto{\pgfqpoint{4.146676in}{1.078044in}}%
\pgfpathlineto{\pgfqpoint{4.146676in}{1.078044in}}%
\pgfpathlineto{\pgfqpoint{4.146676in}{1.080994in}}%
\pgfpathlineto{\pgfqpoint{4.151217in}{1.080994in}}%
\pgfpathlineto{\pgfqpoint{4.151217in}{1.078044in}}%
\pgfpathmoveto{\pgfqpoint{4.146676in}{1.080994in}}%
\pgfpathlineto{\pgfqpoint{4.146676in}{1.080994in}}%
\pgfpathlineto{\pgfqpoint{4.146676in}{1.083943in}}%
\pgfpathlineto{\pgfqpoint{4.151217in}{1.083943in}}%
\pgfpathlineto{\pgfqpoint{4.151217in}{1.080994in}}%
\pgfpathmoveto{\pgfqpoint{4.151217in}{1.078044in}}%
\pgfpathlineto{\pgfqpoint{4.151217in}{1.078044in}}%
\pgfpathlineto{\pgfqpoint{4.151217in}{1.080994in}}%
\pgfpathlineto{\pgfqpoint{4.155758in}{1.080994in}}%
\pgfpathlineto{\pgfqpoint{4.155758in}{1.078044in}}%
\pgfpathmoveto{\pgfqpoint{4.128512in}{1.089842in}}%
\pgfpathlineto{\pgfqpoint{4.128512in}{1.089842in}}%
\pgfpathlineto{\pgfqpoint{4.128512in}{1.092791in}}%
\pgfpathlineto{\pgfqpoint{4.133053in}{1.092791in}}%
\pgfpathlineto{\pgfqpoint{4.133053in}{1.089842in}}%
\pgfpathmoveto{\pgfqpoint{4.128512in}{1.092791in}}%
\pgfpathlineto{\pgfqpoint{4.128512in}{1.092791in}}%
\pgfpathlineto{\pgfqpoint{4.128512in}{1.095740in}}%
\pgfpathlineto{\pgfqpoint{4.133053in}{1.095740in}}%
\pgfpathlineto{\pgfqpoint{4.133053in}{1.092791in}}%
\pgfpathmoveto{\pgfqpoint{4.133053in}{1.089842in}}%
\pgfpathlineto{\pgfqpoint{4.133053in}{1.089842in}}%
\pgfpathlineto{\pgfqpoint{4.133053in}{1.092791in}}%
\pgfpathlineto{\pgfqpoint{4.137594in}{1.092791in}}%
\pgfpathlineto{\pgfqpoint{4.137594in}{1.089842in}}%
\pgfpathmoveto{\pgfqpoint{4.133053in}{1.092791in}}%
\pgfpathlineto{\pgfqpoint{4.133053in}{1.092791in}}%
\pgfpathlineto{\pgfqpoint{4.133053in}{1.095740in}}%
\pgfpathlineto{\pgfqpoint{4.137594in}{1.095740in}}%
\pgfpathlineto{\pgfqpoint{4.137594in}{1.092791in}}%
\pgfpathmoveto{\pgfqpoint{4.128512in}{1.095740in}}%
\pgfpathlineto{\pgfqpoint{4.128512in}{1.095740in}}%
\pgfpathlineto{\pgfqpoint{4.128512in}{1.098689in}}%
\pgfpathlineto{\pgfqpoint{4.133053in}{1.098689in}}%
\pgfpathlineto{\pgfqpoint{4.133053in}{1.095740in}}%
\pgfpathmoveto{\pgfqpoint{4.137594in}{1.089842in}}%
\pgfpathlineto{\pgfqpoint{4.137594in}{1.089842in}}%
\pgfpathlineto{\pgfqpoint{4.137594in}{1.092791in}}%
\pgfpathlineto{\pgfqpoint{4.142135in}{1.092791in}}%
\pgfpathlineto{\pgfqpoint{4.142135in}{1.089842in}}%
\pgfpathmoveto{\pgfqpoint{4.092184in}{1.119334in}}%
\pgfpathlineto{\pgfqpoint{4.092184in}{1.119334in}}%
\pgfpathlineto{\pgfqpoint{4.092184in}{1.122284in}}%
\pgfpathlineto{\pgfqpoint{4.096725in}{1.122284in}}%
\pgfpathlineto{\pgfqpoint{4.096725in}{1.119334in}}%
\pgfpathmoveto{\pgfqpoint{4.092184in}{1.122284in}}%
\pgfpathlineto{\pgfqpoint{4.092184in}{1.122284in}}%
\pgfpathlineto{\pgfqpoint{4.092184in}{1.125233in}}%
\pgfpathlineto{\pgfqpoint{4.096725in}{1.125233in}}%
\pgfpathlineto{\pgfqpoint{4.096725in}{1.122284in}}%
\pgfpathmoveto{\pgfqpoint{4.096725in}{1.119334in}}%
\pgfpathlineto{\pgfqpoint{4.096725in}{1.119334in}}%
\pgfpathlineto{\pgfqpoint{4.096725in}{1.122284in}}%
\pgfpathlineto{\pgfqpoint{4.101266in}{1.122284in}}%
\pgfpathlineto{\pgfqpoint{4.101266in}{1.119334in}}%
\pgfpathmoveto{\pgfqpoint{4.096725in}{1.122284in}}%
\pgfpathlineto{\pgfqpoint{4.096725in}{1.122284in}}%
\pgfpathlineto{\pgfqpoint{4.096725in}{1.125233in}}%
\pgfpathlineto{\pgfqpoint{4.101266in}{1.125233in}}%
\pgfpathlineto{\pgfqpoint{4.101266in}{1.122284in}}%
\pgfpathmoveto{\pgfqpoint{4.101266in}{1.113436in}}%
\pgfpathlineto{\pgfqpoint{4.101266in}{1.113436in}}%
\pgfpathlineto{\pgfqpoint{4.101266in}{1.116385in}}%
\pgfpathlineto{\pgfqpoint{4.105807in}{1.116385in}}%
\pgfpathlineto{\pgfqpoint{4.105807in}{1.113436in}}%
\pgfpathmoveto{\pgfqpoint{4.101266in}{1.116385in}}%
\pgfpathlineto{\pgfqpoint{4.101266in}{1.116385in}}%
\pgfpathlineto{\pgfqpoint{4.101266in}{1.119334in}}%
\pgfpathlineto{\pgfqpoint{4.105807in}{1.119334in}}%
\pgfpathlineto{\pgfqpoint{4.105807in}{1.116385in}}%
\pgfpathmoveto{\pgfqpoint{4.105807in}{1.113436in}}%
\pgfpathlineto{\pgfqpoint{4.105807in}{1.113436in}}%
\pgfpathlineto{\pgfqpoint{4.105807in}{1.116385in}}%
\pgfpathlineto{\pgfqpoint{4.110348in}{1.116385in}}%
\pgfpathlineto{\pgfqpoint{4.110348in}{1.113436in}}%
\pgfpathmoveto{\pgfqpoint{4.105807in}{1.116385in}}%
\pgfpathlineto{\pgfqpoint{4.105807in}{1.116385in}}%
\pgfpathlineto{\pgfqpoint{4.105807in}{1.119334in}}%
\pgfpathlineto{\pgfqpoint{4.110348in}{1.119334in}}%
\pgfpathlineto{\pgfqpoint{4.110348in}{1.116385in}}%
\pgfpathmoveto{\pgfqpoint{4.101266in}{1.119334in}}%
\pgfpathlineto{\pgfqpoint{4.101266in}{1.119334in}}%
\pgfpathlineto{\pgfqpoint{4.101266in}{1.122284in}}%
\pgfpathlineto{\pgfqpoint{4.105807in}{1.122284in}}%
\pgfpathlineto{\pgfqpoint{4.105807in}{1.119334in}}%
\pgfpathmoveto{\pgfqpoint{4.092184in}{1.125233in}}%
\pgfpathlineto{\pgfqpoint{4.092184in}{1.125233in}}%
\pgfpathlineto{\pgfqpoint{4.092184in}{1.128182in}}%
\pgfpathlineto{\pgfqpoint{4.096725in}{1.128182in}}%
\pgfpathlineto{\pgfqpoint{4.096725in}{1.125233in}}%
\pgfpathmoveto{\pgfqpoint{4.092184in}{1.128182in}}%
\pgfpathlineto{\pgfqpoint{4.092184in}{1.128182in}}%
\pgfpathlineto{\pgfqpoint{4.092184in}{1.131132in}}%
\pgfpathlineto{\pgfqpoint{4.096725in}{1.131132in}}%
\pgfpathlineto{\pgfqpoint{4.096725in}{1.128182in}}%
\pgfpathmoveto{\pgfqpoint{4.096725in}{1.125233in}}%
\pgfpathlineto{\pgfqpoint{4.096725in}{1.125233in}}%
\pgfpathlineto{\pgfqpoint{4.096725in}{1.128182in}}%
\pgfpathlineto{\pgfqpoint{4.101266in}{1.128182in}}%
\pgfpathlineto{\pgfqpoint{4.101266in}{1.125233in}}%
\pgfpathmoveto{\pgfqpoint{4.110348in}{1.113436in}}%
\pgfpathlineto{\pgfqpoint{4.110348in}{1.113436in}}%
\pgfpathlineto{\pgfqpoint{4.110348in}{1.116385in}}%
\pgfpathlineto{\pgfqpoint{4.114889in}{1.116385in}}%
\pgfpathlineto{\pgfqpoint{4.114889in}{1.113436in}}%
\pgfpathmoveto{\pgfqpoint{4.164840in}{1.066247in}}%
\pgfpathlineto{\pgfqpoint{4.164840in}{1.066247in}}%
\pgfpathlineto{\pgfqpoint{4.164840in}{1.069197in}}%
\pgfpathlineto{\pgfqpoint{4.169381in}{1.069197in}}%
\pgfpathlineto{\pgfqpoint{4.169381in}{1.066247in}}%
\pgfpathclose%
\pgfusepath{fill}%
\end{pgfscope}%
\begin{pgfscope}%
\pgfpathrectangle{\pgfqpoint{0.750000in}{0.500000in}}{\pgfqpoint{4.650000in}{3.020000in}}%
\pgfusepath{clip}%
\pgfsetbuttcap%
\pgfsetmiterjoin%
\definecolor{currentfill}{rgb}{1.000000,0.000000,0.000000}%
\pgfsetfillcolor{currentfill}%
\pgfsetlinewidth{0.000000pt}%
\definecolor{currentstroke}{rgb}{0.000000,0.000000,0.000000}%
\pgfsetstrokecolor{currentstroke}%
\pgfsetstrokeopacity{0.000000}%
\pgfsetdash{}{0pt}%
\pgfpathmoveto{\pgfqpoint{3.460989in}{0.500002in}}%
\pgfpathlineto{\pgfqpoint{3.460989in}{0.502952in}}%
\pgfpathlineto{\pgfqpoint{3.465530in}{0.502952in}}%
\pgfpathlineto{\pgfqpoint{3.465530in}{0.500002in}}%
\pgfpathmoveto{\pgfqpoint{3.465530in}{0.500002in}}%
\pgfpathlineto{\pgfqpoint{3.465530in}{0.500002in}}%
\pgfpathlineto{\pgfqpoint{3.465530in}{0.502952in}}%
\pgfpathlineto{\pgfqpoint{3.470071in}{0.502952in}}%
\pgfpathlineto{\pgfqpoint{3.470071in}{0.500002in}}%
\pgfpathmoveto{\pgfqpoint{3.465530in}{0.502952in}}%
\pgfpathlineto{\pgfqpoint{3.465530in}{0.502952in}}%
\pgfpathlineto{\pgfqpoint{3.465530in}{0.505901in}}%
\pgfpathlineto{\pgfqpoint{3.470071in}{0.505901in}}%
\pgfpathlineto{\pgfqpoint{3.470071in}{0.502952in}}%
\pgfpathmoveto{\pgfqpoint{3.470071in}{0.502952in}}%
\pgfpathlineto{\pgfqpoint{3.470071in}{0.502952in}}%
\pgfpathlineto{\pgfqpoint{3.470071in}{0.505901in}}%
\pgfpathlineto{\pgfqpoint{3.474612in}{0.505901in}}%
\pgfpathlineto{\pgfqpoint{3.474612in}{0.502952in}}%
\pgfpathmoveto{\pgfqpoint{3.470071in}{0.505901in}}%
\pgfpathlineto{\pgfqpoint{3.470071in}{0.505901in}}%
\pgfpathlineto{\pgfqpoint{3.470071in}{0.508850in}}%
\pgfpathlineto{\pgfqpoint{3.474612in}{0.508850in}}%
\pgfpathlineto{\pgfqpoint{3.474612in}{0.505901in}}%
\pgfpathmoveto{\pgfqpoint{3.470071in}{0.508850in}}%
\pgfpathlineto{\pgfqpoint{3.470071in}{0.508850in}}%
\pgfpathlineto{\pgfqpoint{3.470071in}{0.511799in}}%
\pgfpathlineto{\pgfqpoint{3.474612in}{0.511799in}}%
\pgfpathlineto{\pgfqpoint{3.474612in}{0.508850in}}%
\pgfpathmoveto{\pgfqpoint{3.474612in}{0.508850in}}%
\pgfpathlineto{\pgfqpoint{3.474612in}{0.508850in}}%
\pgfpathlineto{\pgfqpoint{3.474612in}{0.511799in}}%
\pgfpathlineto{\pgfqpoint{3.479153in}{0.511799in}}%
\pgfpathlineto{\pgfqpoint{3.479153in}{0.508850in}}%
\pgfpathmoveto{\pgfqpoint{3.474612in}{0.511799in}}%
\pgfpathlineto{\pgfqpoint{3.474612in}{0.511799in}}%
\pgfpathlineto{\pgfqpoint{3.474612in}{0.514748in}}%
\pgfpathlineto{\pgfqpoint{3.479153in}{0.514748in}}%
\pgfpathlineto{\pgfqpoint{3.479153in}{0.511799in}}%
\pgfpathmoveto{\pgfqpoint{3.479153in}{0.511799in}}%
\pgfpathlineto{\pgfqpoint{3.479153in}{0.511799in}}%
\pgfpathlineto{\pgfqpoint{3.479153in}{0.514748in}}%
\pgfpathlineto{\pgfqpoint{3.483694in}{0.514748in}}%
\pgfpathlineto{\pgfqpoint{3.483694in}{0.511799in}}%
\pgfpathmoveto{\pgfqpoint{3.479153in}{0.514748in}}%
\pgfpathlineto{\pgfqpoint{3.479153in}{0.514748in}}%
\pgfpathlineto{\pgfqpoint{3.479153in}{0.517698in}}%
\pgfpathlineto{\pgfqpoint{3.483694in}{0.517698in}}%
\pgfpathlineto{\pgfqpoint{3.483694in}{0.514748in}}%
\pgfpathmoveto{\pgfqpoint{3.483694in}{0.514748in}}%
\pgfpathlineto{\pgfqpoint{3.483694in}{0.514748in}}%
\pgfpathlineto{\pgfqpoint{3.483694in}{0.517698in}}%
\pgfpathlineto{\pgfqpoint{3.488236in}{0.517698in}}%
\pgfpathlineto{\pgfqpoint{3.488236in}{0.514748in}}%
\pgfpathmoveto{\pgfqpoint{3.483694in}{0.517698in}}%
\pgfpathlineto{\pgfqpoint{3.483694in}{0.517698in}}%
\pgfpathlineto{\pgfqpoint{3.483694in}{0.520647in}}%
\pgfpathlineto{\pgfqpoint{3.488236in}{0.520647in}}%
\pgfpathlineto{\pgfqpoint{3.488236in}{0.517698in}}%
\pgfpathmoveto{\pgfqpoint{3.488236in}{0.517698in}}%
\pgfpathlineto{\pgfqpoint{3.488236in}{0.517698in}}%
\pgfpathlineto{\pgfqpoint{3.488236in}{0.520647in}}%
\pgfpathlineto{\pgfqpoint{3.492777in}{0.520647in}}%
\pgfpathlineto{\pgfqpoint{3.492777in}{0.517698in}}%
\pgfpathmoveto{\pgfqpoint{3.488236in}{0.520647in}}%
\pgfpathlineto{\pgfqpoint{3.488236in}{0.520647in}}%
\pgfpathlineto{\pgfqpoint{3.488236in}{0.523596in}}%
\pgfpathlineto{\pgfqpoint{3.492777in}{0.523596in}}%
\pgfpathlineto{\pgfqpoint{3.492777in}{0.520647in}}%
\pgfpathmoveto{\pgfqpoint{3.488236in}{0.523596in}}%
\pgfpathlineto{\pgfqpoint{3.488236in}{0.523596in}}%
\pgfpathlineto{\pgfqpoint{3.488236in}{0.526545in}}%
\pgfpathlineto{\pgfqpoint{3.492777in}{0.526545in}}%
\pgfpathlineto{\pgfqpoint{3.492777in}{0.523596in}}%
\pgfpathmoveto{\pgfqpoint{3.492777in}{0.523596in}}%
\pgfpathlineto{\pgfqpoint{3.492777in}{0.523596in}}%
\pgfpathlineto{\pgfqpoint{3.492777in}{0.526545in}}%
\pgfpathlineto{\pgfqpoint{3.497318in}{0.526545in}}%
\pgfpathlineto{\pgfqpoint{3.497318in}{0.523596in}}%
\pgfpathmoveto{\pgfqpoint{3.492777in}{0.526545in}}%
\pgfpathlineto{\pgfqpoint{3.492777in}{0.526545in}}%
\pgfpathlineto{\pgfqpoint{3.492777in}{0.529494in}}%
\pgfpathlineto{\pgfqpoint{3.497318in}{0.529494in}}%
\pgfpathlineto{\pgfqpoint{3.497318in}{0.526545in}}%
\pgfpathmoveto{\pgfqpoint{3.497318in}{0.526545in}}%
\pgfpathlineto{\pgfqpoint{3.497318in}{0.526545in}}%
\pgfpathlineto{\pgfqpoint{3.497318in}{0.529494in}}%
\pgfpathlineto{\pgfqpoint{3.501859in}{0.529494in}}%
\pgfpathlineto{\pgfqpoint{3.501859in}{0.526545in}}%
\pgfpathmoveto{\pgfqpoint{3.497318in}{0.529494in}}%
\pgfpathlineto{\pgfqpoint{3.497318in}{0.529494in}}%
\pgfpathlineto{\pgfqpoint{3.497318in}{0.532443in}}%
\pgfpathlineto{\pgfqpoint{3.501859in}{0.532443in}}%
\pgfpathlineto{\pgfqpoint{3.501859in}{0.529494in}}%
\pgfpathmoveto{\pgfqpoint{3.501859in}{0.529494in}}%
\pgfpathlineto{\pgfqpoint{3.501859in}{0.529494in}}%
\pgfpathlineto{\pgfqpoint{3.501859in}{0.532443in}}%
\pgfpathlineto{\pgfqpoint{3.506400in}{0.532443in}}%
\pgfpathlineto{\pgfqpoint{3.506400in}{0.529494in}}%
\pgfpathmoveto{\pgfqpoint{3.501859in}{0.532443in}}%
\pgfpathlineto{\pgfqpoint{3.501859in}{0.532443in}}%
\pgfpathlineto{\pgfqpoint{3.501859in}{0.535393in}}%
\pgfpathlineto{\pgfqpoint{3.506400in}{0.535393in}}%
\pgfpathlineto{\pgfqpoint{3.506400in}{0.532443in}}%
\pgfpathmoveto{\pgfqpoint{3.506400in}{0.532443in}}%
\pgfpathlineto{\pgfqpoint{3.506400in}{0.532443in}}%
\pgfpathlineto{\pgfqpoint{3.506400in}{0.535393in}}%
\pgfpathlineto{\pgfqpoint{3.510941in}{0.535393in}}%
\pgfpathlineto{\pgfqpoint{3.510941in}{0.532443in}}%
\pgfpathmoveto{\pgfqpoint{3.506400in}{0.535393in}}%
\pgfpathlineto{\pgfqpoint{3.506400in}{0.535393in}}%
\pgfpathlineto{\pgfqpoint{3.506400in}{0.538342in}}%
\pgfpathlineto{\pgfqpoint{3.510941in}{0.538342in}}%
\pgfpathlineto{\pgfqpoint{3.510941in}{0.535393in}}%
\pgfpathmoveto{\pgfqpoint{3.510941in}{0.535393in}}%
\pgfpathlineto{\pgfqpoint{3.510941in}{0.535393in}}%
\pgfpathlineto{\pgfqpoint{3.510941in}{0.538342in}}%
\pgfpathlineto{\pgfqpoint{3.515482in}{0.538342in}}%
\pgfpathlineto{\pgfqpoint{3.515482in}{0.535393in}}%
\pgfpathmoveto{\pgfqpoint{3.510941in}{0.538342in}}%
\pgfpathlineto{\pgfqpoint{3.510941in}{0.538342in}}%
\pgfpathlineto{\pgfqpoint{3.510941in}{0.541291in}}%
\pgfpathlineto{\pgfqpoint{3.515482in}{0.541291in}}%
\pgfpathlineto{\pgfqpoint{3.515482in}{0.538342in}}%
\pgfpathmoveto{\pgfqpoint{3.510941in}{0.541291in}}%
\pgfpathlineto{\pgfqpoint{3.510941in}{0.541291in}}%
\pgfpathlineto{\pgfqpoint{3.510941in}{0.544240in}}%
\pgfpathlineto{\pgfqpoint{3.515482in}{0.544240in}}%
\pgfpathlineto{\pgfqpoint{3.515482in}{0.541291in}}%
\pgfpathmoveto{\pgfqpoint{3.515482in}{0.541291in}}%
\pgfpathlineto{\pgfqpoint{3.515482in}{0.541291in}}%
\pgfpathlineto{\pgfqpoint{3.515482in}{0.544240in}}%
\pgfpathlineto{\pgfqpoint{3.520023in}{0.544240in}}%
\pgfpathlineto{\pgfqpoint{3.520023in}{0.541291in}}%
\pgfpathmoveto{\pgfqpoint{3.515482in}{0.544240in}}%
\pgfpathlineto{\pgfqpoint{3.515482in}{0.544240in}}%
\pgfpathlineto{\pgfqpoint{3.515482in}{0.547189in}}%
\pgfpathlineto{\pgfqpoint{3.520023in}{0.547189in}}%
\pgfpathlineto{\pgfqpoint{3.520023in}{0.544240in}}%
\pgfpathmoveto{\pgfqpoint{3.520023in}{0.544240in}}%
\pgfpathlineto{\pgfqpoint{3.520023in}{0.544240in}}%
\pgfpathlineto{\pgfqpoint{3.520023in}{0.547189in}}%
\pgfpathlineto{\pgfqpoint{3.524564in}{0.547189in}}%
\pgfpathlineto{\pgfqpoint{3.524564in}{0.544240in}}%
\pgfpathmoveto{\pgfqpoint{3.520023in}{0.547189in}}%
\pgfpathlineto{\pgfqpoint{3.520023in}{0.547189in}}%
\pgfpathlineto{\pgfqpoint{3.520023in}{0.550139in}}%
\pgfpathlineto{\pgfqpoint{3.524564in}{0.550139in}}%
\pgfpathlineto{\pgfqpoint{3.524564in}{0.547189in}}%
\pgfpathmoveto{\pgfqpoint{3.524564in}{0.547189in}}%
\pgfpathlineto{\pgfqpoint{3.524564in}{0.547189in}}%
\pgfpathlineto{\pgfqpoint{3.524564in}{0.550139in}}%
\pgfpathlineto{\pgfqpoint{3.529105in}{0.550139in}}%
\pgfpathlineto{\pgfqpoint{3.529105in}{0.547189in}}%
\pgfpathmoveto{\pgfqpoint{3.524564in}{0.550139in}}%
\pgfpathlineto{\pgfqpoint{3.524564in}{0.550139in}}%
\pgfpathlineto{\pgfqpoint{3.524564in}{0.553088in}}%
\pgfpathlineto{\pgfqpoint{3.529105in}{0.553088in}}%
\pgfpathlineto{\pgfqpoint{3.529105in}{0.550139in}}%
\pgfpathmoveto{\pgfqpoint{3.529105in}{0.550139in}}%
\pgfpathlineto{\pgfqpoint{3.529105in}{0.550139in}}%
\pgfpathlineto{\pgfqpoint{3.529105in}{0.553088in}}%
\pgfpathlineto{\pgfqpoint{3.533646in}{0.553088in}}%
\pgfpathlineto{\pgfqpoint{3.533646in}{0.550139in}}%
\pgfpathmoveto{\pgfqpoint{3.529105in}{0.553088in}}%
\pgfpathlineto{\pgfqpoint{3.529105in}{0.553088in}}%
\pgfpathlineto{\pgfqpoint{3.529105in}{0.556037in}}%
\pgfpathlineto{\pgfqpoint{3.533646in}{0.556037in}}%
\pgfpathlineto{\pgfqpoint{3.533646in}{0.553088in}}%
\pgfpathmoveto{\pgfqpoint{3.533646in}{0.553088in}}%
\pgfpathlineto{\pgfqpoint{3.533646in}{0.553088in}}%
\pgfpathlineto{\pgfqpoint{3.533646in}{0.556037in}}%
\pgfpathlineto{\pgfqpoint{3.538186in}{0.556037in}}%
\pgfpathlineto{\pgfqpoint{3.538186in}{0.553088in}}%
\pgfpathmoveto{\pgfqpoint{3.533646in}{0.556037in}}%
\pgfpathlineto{\pgfqpoint{3.533646in}{0.556037in}}%
\pgfpathlineto{\pgfqpoint{3.533646in}{0.558986in}}%
\pgfpathlineto{\pgfqpoint{3.538186in}{0.558986in}}%
\pgfpathlineto{\pgfqpoint{3.538186in}{0.556037in}}%
\pgfpathmoveto{\pgfqpoint{3.533646in}{0.558986in}}%
\pgfpathlineto{\pgfqpoint{3.533646in}{0.558986in}}%
\pgfpathlineto{\pgfqpoint{3.533646in}{0.561935in}}%
\pgfpathlineto{\pgfqpoint{3.538186in}{0.561935in}}%
\pgfpathlineto{\pgfqpoint{3.538186in}{0.558986in}}%
\pgfpathmoveto{\pgfqpoint{3.538186in}{0.558986in}}%
\pgfpathlineto{\pgfqpoint{3.538186in}{0.558986in}}%
\pgfpathlineto{\pgfqpoint{3.538186in}{0.561935in}}%
\pgfpathlineto{\pgfqpoint{3.542727in}{0.561935in}}%
\pgfpathlineto{\pgfqpoint{3.542727in}{0.558986in}}%
\pgfpathmoveto{\pgfqpoint{3.538186in}{0.561935in}}%
\pgfpathlineto{\pgfqpoint{3.538186in}{0.561935in}}%
\pgfpathlineto{\pgfqpoint{3.538186in}{0.564885in}}%
\pgfpathlineto{\pgfqpoint{3.542727in}{0.564885in}}%
\pgfpathlineto{\pgfqpoint{3.542727in}{0.561935in}}%
\pgfpathmoveto{\pgfqpoint{3.542727in}{0.561935in}}%
\pgfpathlineto{\pgfqpoint{3.542727in}{0.561935in}}%
\pgfpathlineto{\pgfqpoint{3.542727in}{0.564885in}}%
\pgfpathlineto{\pgfqpoint{3.547268in}{0.564885in}}%
\pgfpathlineto{\pgfqpoint{3.547268in}{0.561935in}}%
\pgfpathmoveto{\pgfqpoint{3.542727in}{0.564885in}}%
\pgfpathlineto{\pgfqpoint{3.542727in}{0.564885in}}%
\pgfpathlineto{\pgfqpoint{3.542727in}{0.567834in}}%
\pgfpathlineto{\pgfqpoint{3.547268in}{0.567834in}}%
\pgfpathlineto{\pgfqpoint{3.547268in}{0.564885in}}%
\pgfpathmoveto{\pgfqpoint{3.547268in}{0.564885in}}%
\pgfpathlineto{\pgfqpoint{3.547268in}{0.564885in}}%
\pgfpathlineto{\pgfqpoint{3.547268in}{0.567834in}}%
\pgfpathlineto{\pgfqpoint{3.551809in}{0.567834in}}%
\pgfpathlineto{\pgfqpoint{3.551809in}{0.564885in}}%
\pgfpathmoveto{\pgfqpoint{3.547268in}{0.567834in}}%
\pgfpathlineto{\pgfqpoint{3.547268in}{0.567834in}}%
\pgfpathlineto{\pgfqpoint{3.547268in}{0.570783in}}%
\pgfpathlineto{\pgfqpoint{3.551809in}{0.570783in}}%
\pgfpathlineto{\pgfqpoint{3.551809in}{0.567834in}}%
\pgfpathmoveto{\pgfqpoint{3.551809in}{0.567834in}}%
\pgfpathlineto{\pgfqpoint{3.551809in}{0.567834in}}%
\pgfpathlineto{\pgfqpoint{3.551809in}{0.570783in}}%
\pgfpathlineto{\pgfqpoint{3.556350in}{0.570783in}}%
\pgfpathlineto{\pgfqpoint{3.556350in}{0.567834in}}%
\pgfpathmoveto{\pgfqpoint{3.551809in}{0.570783in}}%
\pgfpathlineto{\pgfqpoint{3.551809in}{0.570783in}}%
\pgfpathlineto{\pgfqpoint{3.551809in}{0.573732in}}%
\pgfpathlineto{\pgfqpoint{3.556350in}{0.573732in}}%
\pgfpathlineto{\pgfqpoint{3.556350in}{0.570783in}}%
\pgfpathmoveto{\pgfqpoint{3.556350in}{0.570783in}}%
\pgfpathlineto{\pgfqpoint{3.556350in}{0.570783in}}%
\pgfpathlineto{\pgfqpoint{3.556350in}{0.573732in}}%
\pgfpathlineto{\pgfqpoint{3.560890in}{0.573732in}}%
\pgfpathlineto{\pgfqpoint{3.560890in}{0.570783in}}%
\pgfpathmoveto{\pgfqpoint{3.556350in}{0.573732in}}%
\pgfpathlineto{\pgfqpoint{3.556350in}{0.573732in}}%
\pgfpathlineto{\pgfqpoint{3.556350in}{0.576681in}}%
\pgfpathlineto{\pgfqpoint{3.560890in}{0.576681in}}%
\pgfpathlineto{\pgfqpoint{3.560890in}{0.573732in}}%
\pgfpathmoveto{\pgfqpoint{3.560890in}{0.573732in}}%
\pgfpathlineto{\pgfqpoint{3.560890in}{0.573732in}}%
\pgfpathlineto{\pgfqpoint{3.560890in}{0.576681in}}%
\pgfpathlineto{\pgfqpoint{3.565431in}{0.576681in}}%
\pgfpathlineto{\pgfqpoint{3.565431in}{0.573732in}}%
\pgfpathmoveto{\pgfqpoint{3.560890in}{0.576681in}}%
\pgfpathlineto{\pgfqpoint{3.560890in}{0.576681in}}%
\pgfpathlineto{\pgfqpoint{3.560890in}{0.579631in}}%
\pgfpathlineto{\pgfqpoint{3.565431in}{0.579631in}}%
\pgfpathlineto{\pgfqpoint{3.565431in}{0.576681in}}%
\pgfpathmoveto{\pgfqpoint{3.560890in}{0.579631in}}%
\pgfpathlineto{\pgfqpoint{3.560890in}{0.579631in}}%
\pgfpathlineto{\pgfqpoint{3.560890in}{0.582580in}}%
\pgfpathlineto{\pgfqpoint{3.565431in}{0.582580in}}%
\pgfpathlineto{\pgfqpoint{3.565431in}{0.579631in}}%
\pgfpathmoveto{\pgfqpoint{3.565431in}{0.579631in}}%
\pgfpathlineto{\pgfqpoint{3.565431in}{0.579631in}}%
\pgfpathlineto{\pgfqpoint{3.565431in}{0.582580in}}%
\pgfpathlineto{\pgfqpoint{3.569972in}{0.582580in}}%
\pgfpathlineto{\pgfqpoint{3.569972in}{0.579631in}}%
\pgfpathmoveto{\pgfqpoint{3.565431in}{0.582580in}}%
\pgfpathlineto{\pgfqpoint{3.565431in}{0.582580in}}%
\pgfpathlineto{\pgfqpoint{3.565431in}{0.585529in}}%
\pgfpathlineto{\pgfqpoint{3.569972in}{0.585529in}}%
\pgfpathlineto{\pgfqpoint{3.569972in}{0.582580in}}%
\pgfpathmoveto{\pgfqpoint{3.569972in}{0.582580in}}%
\pgfpathlineto{\pgfqpoint{3.569972in}{0.582580in}}%
\pgfpathlineto{\pgfqpoint{3.569972in}{0.585529in}}%
\pgfpathlineto{\pgfqpoint{3.574513in}{0.585529in}}%
\pgfpathlineto{\pgfqpoint{3.574513in}{0.582580in}}%
\pgfpathmoveto{\pgfqpoint{3.569972in}{0.585529in}}%
\pgfpathlineto{\pgfqpoint{3.569972in}{0.585529in}}%
\pgfpathlineto{\pgfqpoint{3.569972in}{0.588478in}}%
\pgfpathlineto{\pgfqpoint{3.574513in}{0.588478in}}%
\pgfpathlineto{\pgfqpoint{3.574513in}{0.585529in}}%
\pgfpathmoveto{\pgfqpoint{3.574513in}{0.585529in}}%
\pgfpathlineto{\pgfqpoint{3.574513in}{0.585529in}}%
\pgfpathlineto{\pgfqpoint{3.574513in}{0.588478in}}%
\pgfpathlineto{\pgfqpoint{3.579054in}{0.588478in}}%
\pgfpathlineto{\pgfqpoint{3.579054in}{0.585529in}}%
\pgfpathmoveto{\pgfqpoint{3.574513in}{0.588478in}}%
\pgfpathlineto{\pgfqpoint{3.574513in}{0.588478in}}%
\pgfpathlineto{\pgfqpoint{3.574513in}{0.591427in}}%
\pgfpathlineto{\pgfqpoint{3.579054in}{0.591427in}}%
\pgfpathlineto{\pgfqpoint{3.579054in}{0.588478in}}%
\pgfpathmoveto{\pgfqpoint{3.579054in}{0.588478in}}%
\pgfpathlineto{\pgfqpoint{3.579054in}{0.588478in}}%
\pgfpathlineto{\pgfqpoint{3.579054in}{0.591427in}}%
\pgfpathlineto{\pgfqpoint{3.583595in}{0.591427in}}%
\pgfpathlineto{\pgfqpoint{3.583595in}{0.588478in}}%
\pgfpathmoveto{\pgfqpoint{3.579054in}{0.591427in}}%
\pgfpathlineto{\pgfqpoint{3.579054in}{0.591427in}}%
\pgfpathlineto{\pgfqpoint{3.579054in}{0.594377in}}%
\pgfpathlineto{\pgfqpoint{3.583595in}{0.594377in}}%
\pgfpathlineto{\pgfqpoint{3.583595in}{0.591427in}}%
\pgfpathmoveto{\pgfqpoint{3.583595in}{0.591427in}}%
\pgfpathlineto{\pgfqpoint{3.583595in}{0.591427in}}%
\pgfpathlineto{\pgfqpoint{3.583595in}{0.594377in}}%
\pgfpathlineto{\pgfqpoint{3.588135in}{0.594377in}}%
\pgfpathlineto{\pgfqpoint{3.588135in}{0.591427in}}%
\pgfpathmoveto{\pgfqpoint{3.583595in}{0.594377in}}%
\pgfpathlineto{\pgfqpoint{3.583595in}{0.594377in}}%
\pgfpathlineto{\pgfqpoint{3.583595in}{0.597326in}}%
\pgfpathlineto{\pgfqpoint{3.588135in}{0.597326in}}%
\pgfpathlineto{\pgfqpoint{3.588135in}{0.594377in}}%
\pgfpathmoveto{\pgfqpoint{3.583595in}{0.597326in}}%
\pgfpathlineto{\pgfqpoint{3.583595in}{0.597326in}}%
\pgfpathlineto{\pgfqpoint{3.583595in}{0.600275in}}%
\pgfpathlineto{\pgfqpoint{3.588135in}{0.600275in}}%
\pgfpathlineto{\pgfqpoint{3.588135in}{0.597326in}}%
\pgfpathmoveto{\pgfqpoint{3.588135in}{0.597326in}}%
\pgfpathlineto{\pgfqpoint{3.588135in}{0.597326in}}%
\pgfpathlineto{\pgfqpoint{3.588135in}{0.600275in}}%
\pgfpathlineto{\pgfqpoint{3.592676in}{0.600275in}}%
\pgfpathlineto{\pgfqpoint{3.592676in}{0.597326in}}%
\pgfpathmoveto{\pgfqpoint{3.588135in}{0.600275in}}%
\pgfpathlineto{\pgfqpoint{3.588135in}{0.600275in}}%
\pgfpathlineto{\pgfqpoint{3.588135in}{0.603224in}}%
\pgfpathlineto{\pgfqpoint{3.592676in}{0.603224in}}%
\pgfpathlineto{\pgfqpoint{3.592676in}{0.600275in}}%
\pgfpathmoveto{\pgfqpoint{3.592676in}{0.600275in}}%
\pgfpathlineto{\pgfqpoint{3.592676in}{0.600275in}}%
\pgfpathlineto{\pgfqpoint{3.592676in}{0.603224in}}%
\pgfpathlineto{\pgfqpoint{3.597217in}{0.603224in}}%
\pgfpathlineto{\pgfqpoint{3.597217in}{0.600275in}}%
\pgfpathmoveto{\pgfqpoint{3.592676in}{0.603224in}}%
\pgfpathlineto{\pgfqpoint{3.592676in}{0.603224in}}%
\pgfpathlineto{\pgfqpoint{3.592676in}{0.606173in}}%
\pgfpathlineto{\pgfqpoint{3.597217in}{0.606173in}}%
\pgfpathlineto{\pgfqpoint{3.597217in}{0.603224in}}%
\pgfpathmoveto{\pgfqpoint{3.597217in}{0.603224in}}%
\pgfpathlineto{\pgfqpoint{3.597217in}{0.603224in}}%
\pgfpathlineto{\pgfqpoint{3.597217in}{0.606173in}}%
\pgfpathlineto{\pgfqpoint{3.601758in}{0.606173in}}%
\pgfpathlineto{\pgfqpoint{3.601758in}{0.603224in}}%
\pgfpathmoveto{\pgfqpoint{3.597217in}{0.606173in}}%
\pgfpathlineto{\pgfqpoint{3.597217in}{0.606173in}}%
\pgfpathlineto{\pgfqpoint{3.597217in}{0.609122in}}%
\pgfpathlineto{\pgfqpoint{3.601758in}{0.609122in}}%
\pgfpathlineto{\pgfqpoint{3.601758in}{0.606173in}}%
\pgfpathmoveto{\pgfqpoint{3.601758in}{0.606173in}}%
\pgfpathlineto{\pgfqpoint{3.601758in}{0.606173in}}%
\pgfpathlineto{\pgfqpoint{3.601758in}{0.609122in}}%
\pgfpathlineto{\pgfqpoint{3.606299in}{0.609122in}}%
\pgfpathlineto{\pgfqpoint{3.606299in}{0.606173in}}%
\pgfpathmoveto{\pgfqpoint{3.601758in}{0.609122in}}%
\pgfpathlineto{\pgfqpoint{3.601758in}{0.609122in}}%
\pgfpathlineto{\pgfqpoint{3.601758in}{0.612072in}}%
\pgfpathlineto{\pgfqpoint{3.606299in}{0.612072in}}%
\pgfpathlineto{\pgfqpoint{3.606299in}{0.609122in}}%
\pgfpathmoveto{\pgfqpoint{3.606299in}{0.609122in}}%
\pgfpathlineto{\pgfqpoint{3.606299in}{0.609122in}}%
\pgfpathlineto{\pgfqpoint{3.606299in}{0.612072in}}%
\pgfpathlineto{\pgfqpoint{3.610839in}{0.612072in}}%
\pgfpathlineto{\pgfqpoint{3.610839in}{0.609122in}}%
\pgfpathmoveto{\pgfqpoint{3.606299in}{0.612072in}}%
\pgfpathlineto{\pgfqpoint{3.606299in}{0.612072in}}%
\pgfpathlineto{\pgfqpoint{3.606299in}{0.615021in}}%
\pgfpathlineto{\pgfqpoint{3.610839in}{0.615021in}}%
\pgfpathlineto{\pgfqpoint{3.610839in}{0.612072in}}%
\pgfpathmoveto{\pgfqpoint{3.606299in}{0.615021in}}%
\pgfpathlineto{\pgfqpoint{3.606299in}{0.615021in}}%
\pgfpathlineto{\pgfqpoint{3.606299in}{0.617970in}}%
\pgfpathlineto{\pgfqpoint{3.610839in}{0.617970in}}%
\pgfpathlineto{\pgfqpoint{3.610839in}{0.615021in}}%
\pgfpathmoveto{\pgfqpoint{3.610839in}{0.615021in}}%
\pgfpathlineto{\pgfqpoint{3.610839in}{0.615021in}}%
\pgfpathlineto{\pgfqpoint{3.610839in}{0.617970in}}%
\pgfpathlineto{\pgfqpoint{3.615380in}{0.617970in}}%
\pgfpathlineto{\pgfqpoint{3.615380in}{0.615021in}}%
\pgfpathmoveto{\pgfqpoint{3.610839in}{0.617970in}}%
\pgfpathlineto{\pgfqpoint{3.610839in}{0.617970in}}%
\pgfpathlineto{\pgfqpoint{3.610839in}{0.620919in}}%
\pgfpathlineto{\pgfqpoint{3.615380in}{0.620919in}}%
\pgfpathlineto{\pgfqpoint{3.615380in}{0.617970in}}%
\pgfpathmoveto{\pgfqpoint{3.615380in}{0.617970in}}%
\pgfpathlineto{\pgfqpoint{3.615380in}{0.617970in}}%
\pgfpathlineto{\pgfqpoint{3.615380in}{0.620919in}}%
\pgfpathlineto{\pgfqpoint{3.619921in}{0.620919in}}%
\pgfpathlineto{\pgfqpoint{3.619921in}{0.617970in}}%
\pgfpathmoveto{\pgfqpoint{3.615380in}{0.620919in}}%
\pgfpathlineto{\pgfqpoint{3.615380in}{0.620919in}}%
\pgfpathlineto{\pgfqpoint{3.615380in}{0.623868in}}%
\pgfpathlineto{\pgfqpoint{3.619921in}{0.623868in}}%
\pgfpathlineto{\pgfqpoint{3.619921in}{0.620919in}}%
\pgfpathmoveto{\pgfqpoint{3.619921in}{0.620919in}}%
\pgfpathlineto{\pgfqpoint{3.619921in}{0.620919in}}%
\pgfpathlineto{\pgfqpoint{3.619921in}{0.623868in}}%
\pgfpathlineto{\pgfqpoint{3.624462in}{0.623868in}}%
\pgfpathlineto{\pgfqpoint{3.624462in}{0.620919in}}%
\pgfpathmoveto{\pgfqpoint{3.619921in}{0.623868in}}%
\pgfpathlineto{\pgfqpoint{3.619921in}{0.623868in}}%
\pgfpathlineto{\pgfqpoint{3.619921in}{0.626817in}}%
\pgfpathlineto{\pgfqpoint{3.624462in}{0.626817in}}%
\pgfpathlineto{\pgfqpoint{3.624462in}{0.623868in}}%
\pgfpathmoveto{\pgfqpoint{3.624462in}{0.623868in}}%
\pgfpathlineto{\pgfqpoint{3.624462in}{0.623868in}}%
\pgfpathlineto{\pgfqpoint{3.624462in}{0.626817in}}%
\pgfpathlineto{\pgfqpoint{3.629003in}{0.626817in}}%
\pgfpathlineto{\pgfqpoint{3.629003in}{0.623868in}}%
\pgfpathmoveto{\pgfqpoint{3.624462in}{0.626817in}}%
\pgfpathlineto{\pgfqpoint{3.624462in}{0.626817in}}%
\pgfpathlineto{\pgfqpoint{3.624462in}{0.629766in}}%
\pgfpathlineto{\pgfqpoint{3.629003in}{0.629766in}}%
\pgfpathlineto{\pgfqpoint{3.629003in}{0.626817in}}%
\pgfpathmoveto{\pgfqpoint{3.629003in}{0.626817in}}%
\pgfpathlineto{\pgfqpoint{3.629003in}{0.626817in}}%
\pgfpathlineto{\pgfqpoint{3.629003in}{0.629766in}}%
\pgfpathlineto{\pgfqpoint{3.633544in}{0.629766in}}%
\pgfpathlineto{\pgfqpoint{3.633544in}{0.626817in}}%
\pgfpathmoveto{\pgfqpoint{3.629003in}{0.629766in}}%
\pgfpathlineto{\pgfqpoint{3.629003in}{0.629766in}}%
\pgfpathlineto{\pgfqpoint{3.629003in}{0.632716in}}%
\pgfpathlineto{\pgfqpoint{3.633544in}{0.632716in}}%
\pgfpathlineto{\pgfqpoint{3.633544in}{0.629766in}}%
\pgfpathmoveto{\pgfqpoint{3.629003in}{0.632716in}}%
\pgfpathlineto{\pgfqpoint{3.629003in}{0.632716in}}%
\pgfpathlineto{\pgfqpoint{3.629003in}{0.635665in}}%
\pgfpathlineto{\pgfqpoint{3.633544in}{0.635665in}}%
\pgfpathlineto{\pgfqpoint{3.633544in}{0.632716in}}%
\pgfpathmoveto{\pgfqpoint{3.633544in}{0.632716in}}%
\pgfpathlineto{\pgfqpoint{3.633544in}{0.632716in}}%
\pgfpathlineto{\pgfqpoint{3.633544in}{0.635665in}}%
\pgfpathlineto{\pgfqpoint{3.638084in}{0.635665in}}%
\pgfpathlineto{\pgfqpoint{3.638084in}{0.632716in}}%
\pgfpathmoveto{\pgfqpoint{3.633544in}{0.635665in}}%
\pgfpathlineto{\pgfqpoint{3.633544in}{0.635665in}}%
\pgfpathlineto{\pgfqpoint{3.633544in}{0.638614in}}%
\pgfpathlineto{\pgfqpoint{3.638084in}{0.638614in}}%
\pgfpathlineto{\pgfqpoint{3.638084in}{0.635665in}}%
\pgfpathmoveto{\pgfqpoint{3.638084in}{0.635665in}}%
\pgfpathlineto{\pgfqpoint{3.638084in}{0.635665in}}%
\pgfpathlineto{\pgfqpoint{3.638084in}{0.638614in}}%
\pgfpathlineto{\pgfqpoint{3.642625in}{0.638614in}}%
\pgfpathlineto{\pgfqpoint{3.642625in}{0.635665in}}%
\pgfpathmoveto{\pgfqpoint{3.638084in}{0.638614in}}%
\pgfpathlineto{\pgfqpoint{3.638084in}{0.638614in}}%
\pgfpathlineto{\pgfqpoint{3.638084in}{0.641563in}}%
\pgfpathlineto{\pgfqpoint{3.642625in}{0.641563in}}%
\pgfpathlineto{\pgfqpoint{3.642625in}{0.638614in}}%
\pgfpathmoveto{\pgfqpoint{3.642625in}{0.638614in}}%
\pgfpathlineto{\pgfqpoint{3.642625in}{0.638614in}}%
\pgfpathlineto{\pgfqpoint{3.642625in}{0.641563in}}%
\pgfpathlineto{\pgfqpoint{3.647166in}{0.641563in}}%
\pgfpathlineto{\pgfqpoint{3.647166in}{0.638614in}}%
\pgfpathmoveto{\pgfqpoint{3.642625in}{0.641563in}}%
\pgfpathlineto{\pgfqpoint{3.642625in}{0.641563in}}%
\pgfpathlineto{\pgfqpoint{3.642625in}{0.644512in}}%
\pgfpathlineto{\pgfqpoint{3.647166in}{0.644512in}}%
\pgfpathlineto{\pgfqpoint{3.647166in}{0.641563in}}%
\pgfpathmoveto{\pgfqpoint{3.647166in}{0.641563in}}%
\pgfpathlineto{\pgfqpoint{3.647166in}{0.641563in}}%
\pgfpathlineto{\pgfqpoint{3.647166in}{0.644512in}}%
\pgfpathlineto{\pgfqpoint{3.651707in}{0.644512in}}%
\pgfpathlineto{\pgfqpoint{3.651707in}{0.641563in}}%
\pgfpathmoveto{\pgfqpoint{3.647166in}{0.644512in}}%
\pgfpathlineto{\pgfqpoint{3.647166in}{0.644512in}}%
\pgfpathlineto{\pgfqpoint{3.647166in}{0.647461in}}%
\pgfpathlineto{\pgfqpoint{3.651707in}{0.647461in}}%
\pgfpathlineto{\pgfqpoint{3.651707in}{0.644512in}}%
\pgfpathmoveto{\pgfqpoint{3.651707in}{0.644512in}}%
\pgfpathlineto{\pgfqpoint{3.651707in}{0.644512in}}%
\pgfpathlineto{\pgfqpoint{3.651707in}{0.647461in}}%
\pgfpathlineto{\pgfqpoint{3.656248in}{0.647461in}}%
\pgfpathlineto{\pgfqpoint{3.656248in}{0.644512in}}%
\pgfpathmoveto{\pgfqpoint{3.651707in}{0.647461in}}%
\pgfpathlineto{\pgfqpoint{3.651707in}{0.647461in}}%
\pgfpathlineto{\pgfqpoint{3.651707in}{0.650411in}}%
\pgfpathlineto{\pgfqpoint{3.656248in}{0.650411in}}%
\pgfpathlineto{\pgfqpoint{3.656248in}{0.647461in}}%
\pgfpathmoveto{\pgfqpoint{3.651707in}{0.650411in}}%
\pgfpathlineto{\pgfqpoint{3.651707in}{0.650411in}}%
\pgfpathlineto{\pgfqpoint{3.651707in}{0.653360in}}%
\pgfpathlineto{\pgfqpoint{3.656248in}{0.653360in}}%
\pgfpathlineto{\pgfqpoint{3.656248in}{0.650411in}}%
\pgfpathmoveto{\pgfqpoint{3.656248in}{0.650411in}}%
\pgfpathlineto{\pgfqpoint{3.656248in}{0.650411in}}%
\pgfpathlineto{\pgfqpoint{3.656248in}{0.653360in}}%
\pgfpathlineto{\pgfqpoint{3.660789in}{0.653360in}}%
\pgfpathlineto{\pgfqpoint{3.660789in}{0.650411in}}%
\pgfpathmoveto{\pgfqpoint{3.656248in}{0.653360in}}%
\pgfpathlineto{\pgfqpoint{3.656248in}{0.653360in}}%
\pgfpathlineto{\pgfqpoint{3.656248in}{0.656309in}}%
\pgfpathlineto{\pgfqpoint{3.660789in}{0.656309in}}%
\pgfpathlineto{\pgfqpoint{3.660789in}{0.653360in}}%
\pgfpathmoveto{\pgfqpoint{3.660789in}{0.653360in}}%
\pgfpathlineto{\pgfqpoint{3.660789in}{0.653360in}}%
\pgfpathlineto{\pgfqpoint{3.660789in}{0.656309in}}%
\pgfpathlineto{\pgfqpoint{3.665330in}{0.656309in}}%
\pgfpathlineto{\pgfqpoint{3.665330in}{0.653360in}}%
\pgfpathmoveto{\pgfqpoint{3.660789in}{0.656309in}}%
\pgfpathlineto{\pgfqpoint{3.660789in}{0.656309in}}%
\pgfpathlineto{\pgfqpoint{3.660789in}{0.659258in}}%
\pgfpathlineto{\pgfqpoint{3.665330in}{0.659258in}}%
\pgfpathlineto{\pgfqpoint{3.665330in}{0.656309in}}%
\pgfpathmoveto{\pgfqpoint{3.665330in}{0.656309in}}%
\pgfpathlineto{\pgfqpoint{3.665330in}{0.656309in}}%
\pgfpathlineto{\pgfqpoint{3.665330in}{0.659258in}}%
\pgfpathlineto{\pgfqpoint{3.669871in}{0.659258in}}%
\pgfpathlineto{\pgfqpoint{3.669871in}{0.656309in}}%
\pgfpathmoveto{\pgfqpoint{3.665330in}{0.659258in}}%
\pgfpathlineto{\pgfqpoint{3.665330in}{0.659258in}}%
\pgfpathlineto{\pgfqpoint{3.665330in}{0.662207in}}%
\pgfpathlineto{\pgfqpoint{3.669871in}{0.662207in}}%
\pgfpathlineto{\pgfqpoint{3.669871in}{0.659258in}}%
\pgfpathmoveto{\pgfqpoint{3.669871in}{0.659258in}}%
\pgfpathlineto{\pgfqpoint{3.669871in}{0.659258in}}%
\pgfpathlineto{\pgfqpoint{3.669871in}{0.662207in}}%
\pgfpathlineto{\pgfqpoint{3.674412in}{0.662207in}}%
\pgfpathlineto{\pgfqpoint{3.674412in}{0.659258in}}%
\pgfpathmoveto{\pgfqpoint{3.669871in}{0.662207in}}%
\pgfpathlineto{\pgfqpoint{3.669871in}{0.662207in}}%
\pgfpathlineto{\pgfqpoint{3.669871in}{0.665156in}}%
\pgfpathlineto{\pgfqpoint{3.674412in}{0.665156in}}%
\pgfpathlineto{\pgfqpoint{3.674412in}{0.662207in}}%
\pgfpathmoveto{\pgfqpoint{3.674412in}{0.662207in}}%
\pgfpathlineto{\pgfqpoint{3.674412in}{0.662207in}}%
\pgfpathlineto{\pgfqpoint{3.674412in}{0.665156in}}%
\pgfpathlineto{\pgfqpoint{3.678954in}{0.665156in}}%
\pgfpathlineto{\pgfqpoint{3.678954in}{0.662207in}}%
\pgfpathmoveto{\pgfqpoint{3.674412in}{0.665156in}}%
\pgfpathlineto{\pgfqpoint{3.674412in}{0.665156in}}%
\pgfpathlineto{\pgfqpoint{3.674412in}{0.668105in}}%
\pgfpathlineto{\pgfqpoint{3.678954in}{0.668105in}}%
\pgfpathlineto{\pgfqpoint{3.678954in}{0.665156in}}%
\pgfpathmoveto{\pgfqpoint{3.674412in}{0.668105in}}%
\pgfpathlineto{\pgfqpoint{3.674412in}{0.668105in}}%
\pgfpathlineto{\pgfqpoint{3.674412in}{0.671055in}}%
\pgfpathlineto{\pgfqpoint{3.678954in}{0.671055in}}%
\pgfpathlineto{\pgfqpoint{3.678954in}{0.668105in}}%
\pgfpathmoveto{\pgfqpoint{3.678954in}{0.668105in}}%
\pgfpathlineto{\pgfqpoint{3.678954in}{0.668105in}}%
\pgfpathlineto{\pgfqpoint{3.678954in}{0.671055in}}%
\pgfpathlineto{\pgfqpoint{3.683495in}{0.671055in}}%
\pgfpathlineto{\pgfqpoint{3.683495in}{0.668105in}}%
\pgfpathmoveto{\pgfqpoint{3.678954in}{0.671055in}}%
\pgfpathlineto{\pgfqpoint{3.678954in}{0.671055in}}%
\pgfpathlineto{\pgfqpoint{3.678954in}{0.674004in}}%
\pgfpathlineto{\pgfqpoint{3.683495in}{0.674004in}}%
\pgfpathlineto{\pgfqpoint{3.683495in}{0.671055in}}%
\pgfpathmoveto{\pgfqpoint{3.683495in}{0.671055in}}%
\pgfpathlineto{\pgfqpoint{3.683495in}{0.671055in}}%
\pgfpathlineto{\pgfqpoint{3.683495in}{0.674004in}}%
\pgfpathlineto{\pgfqpoint{3.688036in}{0.674004in}}%
\pgfpathlineto{\pgfqpoint{3.688036in}{0.671055in}}%
\pgfpathmoveto{\pgfqpoint{3.683495in}{0.674004in}}%
\pgfpathlineto{\pgfqpoint{3.683495in}{0.674004in}}%
\pgfpathlineto{\pgfqpoint{3.683495in}{0.676953in}}%
\pgfpathlineto{\pgfqpoint{3.688036in}{0.676953in}}%
\pgfpathlineto{\pgfqpoint{3.688036in}{0.674004in}}%
\pgfpathmoveto{\pgfqpoint{3.688036in}{0.674004in}}%
\pgfpathlineto{\pgfqpoint{3.688036in}{0.674004in}}%
\pgfpathlineto{\pgfqpoint{3.688036in}{0.676953in}}%
\pgfpathlineto{\pgfqpoint{3.692577in}{0.676953in}}%
\pgfpathlineto{\pgfqpoint{3.692577in}{0.674004in}}%
\pgfpathmoveto{\pgfqpoint{3.688036in}{0.676953in}}%
\pgfpathlineto{\pgfqpoint{3.688036in}{0.676953in}}%
\pgfpathlineto{\pgfqpoint{3.688036in}{0.679902in}}%
\pgfpathlineto{\pgfqpoint{3.692577in}{0.679902in}}%
\pgfpathlineto{\pgfqpoint{3.692577in}{0.676953in}}%
\pgfpathmoveto{\pgfqpoint{3.692577in}{0.676953in}}%
\pgfpathlineto{\pgfqpoint{3.692577in}{0.676953in}}%
\pgfpathlineto{\pgfqpoint{3.692577in}{0.679902in}}%
\pgfpathlineto{\pgfqpoint{3.697119in}{0.679902in}}%
\pgfpathlineto{\pgfqpoint{3.697119in}{0.676953in}}%
\pgfpathmoveto{\pgfqpoint{3.692577in}{0.679902in}}%
\pgfpathlineto{\pgfqpoint{3.692577in}{0.679902in}}%
\pgfpathlineto{\pgfqpoint{3.692577in}{0.682851in}}%
\pgfpathlineto{\pgfqpoint{3.697119in}{0.682851in}}%
\pgfpathlineto{\pgfqpoint{3.697119in}{0.679902in}}%
\pgfpathmoveto{\pgfqpoint{3.697119in}{0.679902in}}%
\pgfpathlineto{\pgfqpoint{3.697119in}{0.679902in}}%
\pgfpathlineto{\pgfqpoint{3.697119in}{0.682851in}}%
\pgfpathlineto{\pgfqpoint{3.701660in}{0.682851in}}%
\pgfpathlineto{\pgfqpoint{3.701660in}{0.679902in}}%
\pgfpathmoveto{\pgfqpoint{3.697119in}{0.682851in}}%
\pgfpathlineto{\pgfqpoint{3.697119in}{0.682851in}}%
\pgfpathlineto{\pgfqpoint{3.697119in}{0.685800in}}%
\pgfpathlineto{\pgfqpoint{3.701660in}{0.685800in}}%
\pgfpathlineto{\pgfqpoint{3.701660in}{0.682851in}}%
\pgfpathmoveto{\pgfqpoint{3.697119in}{0.685800in}}%
\pgfpathlineto{\pgfqpoint{3.697119in}{0.685800in}}%
\pgfpathlineto{\pgfqpoint{3.697119in}{0.688750in}}%
\pgfpathlineto{\pgfqpoint{3.701660in}{0.688750in}}%
\pgfpathlineto{\pgfqpoint{3.701660in}{0.685800in}}%
\pgfpathmoveto{\pgfqpoint{3.701660in}{0.685800in}}%
\pgfpathlineto{\pgfqpoint{3.701660in}{0.685800in}}%
\pgfpathlineto{\pgfqpoint{3.701660in}{0.688750in}}%
\pgfpathlineto{\pgfqpoint{3.706201in}{0.688750in}}%
\pgfpathlineto{\pgfqpoint{3.706201in}{0.685800in}}%
\pgfpathmoveto{\pgfqpoint{3.701660in}{0.688750in}}%
\pgfpathlineto{\pgfqpoint{3.701660in}{0.688750in}}%
\pgfpathlineto{\pgfqpoint{3.701660in}{0.691699in}}%
\pgfpathlineto{\pgfqpoint{3.706201in}{0.691699in}}%
\pgfpathlineto{\pgfqpoint{3.706201in}{0.688750in}}%
\pgfpathmoveto{\pgfqpoint{3.706201in}{0.688750in}}%
\pgfpathlineto{\pgfqpoint{3.706201in}{0.688750in}}%
\pgfpathlineto{\pgfqpoint{3.706201in}{0.691699in}}%
\pgfpathlineto{\pgfqpoint{3.710742in}{0.691699in}}%
\pgfpathlineto{\pgfqpoint{3.710742in}{0.688750in}}%
\pgfpathmoveto{\pgfqpoint{3.706201in}{0.691699in}}%
\pgfpathlineto{\pgfqpoint{3.706201in}{0.691699in}}%
\pgfpathlineto{\pgfqpoint{3.706201in}{0.694648in}}%
\pgfpathlineto{\pgfqpoint{3.710742in}{0.694648in}}%
\pgfpathlineto{\pgfqpoint{3.710742in}{0.691699in}}%
\pgfpathmoveto{\pgfqpoint{3.710742in}{0.691699in}}%
\pgfpathlineto{\pgfqpoint{3.710742in}{0.691699in}}%
\pgfpathlineto{\pgfqpoint{3.710742in}{0.694648in}}%
\pgfpathlineto{\pgfqpoint{3.715284in}{0.694648in}}%
\pgfpathlineto{\pgfqpoint{3.715284in}{0.691699in}}%
\pgfpathmoveto{\pgfqpoint{3.710742in}{0.694648in}}%
\pgfpathlineto{\pgfqpoint{3.710742in}{0.694648in}}%
\pgfpathlineto{\pgfqpoint{3.710742in}{0.697597in}}%
\pgfpathlineto{\pgfqpoint{3.715284in}{0.697597in}}%
\pgfpathlineto{\pgfqpoint{3.715284in}{0.694648in}}%
\pgfpathmoveto{\pgfqpoint{3.715284in}{0.694648in}}%
\pgfpathlineto{\pgfqpoint{3.715284in}{0.694648in}}%
\pgfpathlineto{\pgfqpoint{3.715284in}{0.697597in}}%
\pgfpathlineto{\pgfqpoint{3.719825in}{0.697597in}}%
\pgfpathlineto{\pgfqpoint{3.719825in}{0.694648in}}%
\pgfpathmoveto{\pgfqpoint{3.715284in}{0.697597in}}%
\pgfpathlineto{\pgfqpoint{3.715284in}{0.697597in}}%
\pgfpathlineto{\pgfqpoint{3.715284in}{0.700547in}}%
\pgfpathlineto{\pgfqpoint{3.719825in}{0.700547in}}%
\pgfpathlineto{\pgfqpoint{3.719825in}{0.697597in}}%
\pgfpathmoveto{\pgfqpoint{3.715284in}{0.700547in}}%
\pgfpathlineto{\pgfqpoint{3.715284in}{0.700547in}}%
\pgfpathlineto{\pgfqpoint{3.715284in}{0.703496in}}%
\pgfpathlineto{\pgfqpoint{3.719825in}{0.703496in}}%
\pgfpathlineto{\pgfqpoint{3.719825in}{0.700547in}}%
\pgfpathmoveto{\pgfqpoint{3.719825in}{0.700547in}}%
\pgfpathlineto{\pgfqpoint{3.719825in}{0.700547in}}%
\pgfpathlineto{\pgfqpoint{3.719825in}{0.703496in}}%
\pgfpathlineto{\pgfqpoint{3.724366in}{0.703496in}}%
\pgfpathlineto{\pgfqpoint{3.724366in}{0.700547in}}%
\pgfpathmoveto{\pgfqpoint{3.719825in}{0.703496in}}%
\pgfpathlineto{\pgfqpoint{3.719825in}{0.703496in}}%
\pgfpathlineto{\pgfqpoint{3.719825in}{0.706445in}}%
\pgfpathlineto{\pgfqpoint{3.724366in}{0.706445in}}%
\pgfpathlineto{\pgfqpoint{3.724366in}{0.703496in}}%
\pgfpathmoveto{\pgfqpoint{3.724366in}{0.703496in}}%
\pgfpathlineto{\pgfqpoint{3.724366in}{0.703496in}}%
\pgfpathlineto{\pgfqpoint{3.724366in}{0.706445in}}%
\pgfpathlineto{\pgfqpoint{3.728907in}{0.706445in}}%
\pgfpathlineto{\pgfqpoint{3.728907in}{0.703496in}}%
\pgfpathmoveto{\pgfqpoint{3.724366in}{0.706445in}}%
\pgfpathlineto{\pgfqpoint{3.724366in}{0.706445in}}%
\pgfpathlineto{\pgfqpoint{3.724366in}{0.709395in}}%
\pgfpathlineto{\pgfqpoint{3.728907in}{0.709395in}}%
\pgfpathlineto{\pgfqpoint{3.728907in}{0.706445in}}%
\pgfpathmoveto{\pgfqpoint{3.728907in}{0.706445in}}%
\pgfpathlineto{\pgfqpoint{3.728907in}{0.706445in}}%
\pgfpathlineto{\pgfqpoint{3.728907in}{0.709395in}}%
\pgfpathlineto{\pgfqpoint{3.733448in}{0.709395in}}%
\pgfpathlineto{\pgfqpoint{3.733448in}{0.706445in}}%
\pgfpathmoveto{\pgfqpoint{3.728907in}{0.709395in}}%
\pgfpathlineto{\pgfqpoint{3.728907in}{0.709395in}}%
\pgfpathlineto{\pgfqpoint{3.728907in}{0.712344in}}%
\pgfpathlineto{\pgfqpoint{3.733448in}{0.712344in}}%
\pgfpathlineto{\pgfqpoint{3.733448in}{0.709395in}}%
\pgfpathmoveto{\pgfqpoint{3.733448in}{0.709395in}}%
\pgfpathlineto{\pgfqpoint{3.733448in}{0.709395in}}%
\pgfpathlineto{\pgfqpoint{3.733448in}{0.712344in}}%
\pgfpathlineto{\pgfqpoint{3.737990in}{0.712344in}}%
\pgfpathlineto{\pgfqpoint{3.737990in}{0.709395in}}%
\pgfpathmoveto{\pgfqpoint{3.733448in}{0.712344in}}%
\pgfpathlineto{\pgfqpoint{3.733448in}{0.712344in}}%
\pgfpathlineto{\pgfqpoint{3.733448in}{0.715293in}}%
\pgfpathlineto{\pgfqpoint{3.737990in}{0.715293in}}%
\pgfpathlineto{\pgfqpoint{3.737990in}{0.712344in}}%
\pgfpathmoveto{\pgfqpoint{3.737990in}{0.712344in}}%
\pgfpathlineto{\pgfqpoint{3.737990in}{0.712344in}}%
\pgfpathlineto{\pgfqpoint{3.737990in}{0.715293in}}%
\pgfpathlineto{\pgfqpoint{3.742531in}{0.715293in}}%
\pgfpathlineto{\pgfqpoint{3.742531in}{0.712344in}}%
\pgfpathmoveto{\pgfqpoint{3.737990in}{0.715293in}}%
\pgfpathlineto{\pgfqpoint{3.737990in}{0.715293in}}%
\pgfpathlineto{\pgfqpoint{3.737990in}{0.718243in}}%
\pgfpathlineto{\pgfqpoint{3.742531in}{0.718243in}}%
\pgfpathlineto{\pgfqpoint{3.742531in}{0.715293in}}%
\pgfpathmoveto{\pgfqpoint{3.737990in}{0.718243in}}%
\pgfpathlineto{\pgfqpoint{3.737990in}{0.718243in}}%
\pgfpathlineto{\pgfqpoint{3.737990in}{0.721192in}}%
\pgfpathlineto{\pgfqpoint{3.742531in}{0.721192in}}%
\pgfpathlineto{\pgfqpoint{3.742531in}{0.718243in}}%
\pgfpathmoveto{\pgfqpoint{3.742531in}{0.718243in}}%
\pgfpathlineto{\pgfqpoint{3.742531in}{0.718243in}}%
\pgfpathlineto{\pgfqpoint{3.742531in}{0.721192in}}%
\pgfpathlineto{\pgfqpoint{3.747072in}{0.721192in}}%
\pgfpathlineto{\pgfqpoint{3.747072in}{0.718243in}}%
\pgfpathmoveto{\pgfqpoint{3.742531in}{0.721192in}}%
\pgfpathlineto{\pgfqpoint{3.742531in}{0.721192in}}%
\pgfpathlineto{\pgfqpoint{3.742531in}{0.724141in}}%
\pgfpathlineto{\pgfqpoint{3.747072in}{0.724141in}}%
\pgfpathlineto{\pgfqpoint{3.747072in}{0.721192in}}%
\pgfpathmoveto{\pgfqpoint{3.747072in}{0.721192in}}%
\pgfpathlineto{\pgfqpoint{3.747072in}{0.721192in}}%
\pgfpathlineto{\pgfqpoint{3.747072in}{0.724141in}}%
\pgfpathlineto{\pgfqpoint{3.751613in}{0.724141in}}%
\pgfpathlineto{\pgfqpoint{3.751613in}{0.721192in}}%
\pgfpathmoveto{\pgfqpoint{3.747072in}{0.724141in}}%
\pgfpathlineto{\pgfqpoint{3.747072in}{0.724141in}}%
\pgfpathlineto{\pgfqpoint{3.747072in}{0.727090in}}%
\pgfpathlineto{\pgfqpoint{3.751613in}{0.727090in}}%
\pgfpathlineto{\pgfqpoint{3.751613in}{0.724141in}}%
\pgfpathmoveto{\pgfqpoint{3.751613in}{0.724141in}}%
\pgfpathlineto{\pgfqpoint{3.751613in}{0.724141in}}%
\pgfpathlineto{\pgfqpoint{3.751613in}{0.727090in}}%
\pgfpathlineto{\pgfqpoint{3.756155in}{0.727090in}}%
\pgfpathlineto{\pgfqpoint{3.756155in}{0.724141in}}%
\pgfpathmoveto{\pgfqpoint{3.751613in}{0.727090in}}%
\pgfpathlineto{\pgfqpoint{3.751613in}{0.727090in}}%
\pgfpathlineto{\pgfqpoint{3.751613in}{0.730040in}}%
\pgfpathlineto{\pgfqpoint{3.756155in}{0.730040in}}%
\pgfpathlineto{\pgfqpoint{3.756155in}{0.727090in}}%
\pgfpathmoveto{\pgfqpoint{3.756155in}{0.727090in}}%
\pgfpathlineto{\pgfqpoint{3.756155in}{0.727090in}}%
\pgfpathlineto{\pgfqpoint{3.756155in}{0.730040in}}%
\pgfpathlineto{\pgfqpoint{3.760696in}{0.730040in}}%
\pgfpathlineto{\pgfqpoint{3.760696in}{0.727090in}}%
\pgfpathmoveto{\pgfqpoint{3.756155in}{0.730040in}}%
\pgfpathlineto{\pgfqpoint{3.756155in}{0.730040in}}%
\pgfpathlineto{\pgfqpoint{3.756155in}{0.732989in}}%
\pgfpathlineto{\pgfqpoint{3.760696in}{0.732989in}}%
\pgfpathlineto{\pgfqpoint{3.760696in}{0.730040in}}%
\pgfpathmoveto{\pgfqpoint{3.760696in}{0.730040in}}%
\pgfpathlineto{\pgfqpoint{3.760696in}{0.730040in}}%
\pgfpathlineto{\pgfqpoint{3.760696in}{0.732989in}}%
\pgfpathlineto{\pgfqpoint{3.765237in}{0.732989in}}%
\pgfpathlineto{\pgfqpoint{3.765237in}{0.730040in}}%
\pgfpathmoveto{\pgfqpoint{3.760696in}{0.732989in}}%
\pgfpathlineto{\pgfqpoint{3.760696in}{0.732989in}}%
\pgfpathlineto{\pgfqpoint{3.760696in}{0.735938in}}%
\pgfpathlineto{\pgfqpoint{3.765237in}{0.735938in}}%
\pgfpathlineto{\pgfqpoint{3.765237in}{0.732989in}}%
\pgfpathmoveto{\pgfqpoint{3.760696in}{0.735938in}}%
\pgfpathlineto{\pgfqpoint{3.760696in}{0.735938in}}%
\pgfpathlineto{\pgfqpoint{3.760696in}{0.738888in}}%
\pgfpathlineto{\pgfqpoint{3.765237in}{0.738888in}}%
\pgfpathlineto{\pgfqpoint{3.765237in}{0.735938in}}%
\pgfpathmoveto{\pgfqpoint{3.765237in}{0.735938in}}%
\pgfpathlineto{\pgfqpoint{3.765237in}{0.735938in}}%
\pgfpathlineto{\pgfqpoint{3.765237in}{0.738888in}}%
\pgfpathlineto{\pgfqpoint{3.769778in}{0.738888in}}%
\pgfpathlineto{\pgfqpoint{3.769778in}{0.735938in}}%
\pgfpathmoveto{\pgfqpoint{3.765237in}{0.738888in}}%
\pgfpathlineto{\pgfqpoint{3.765237in}{0.738888in}}%
\pgfpathlineto{\pgfqpoint{3.765237in}{0.741837in}}%
\pgfpathlineto{\pgfqpoint{3.769778in}{0.741837in}}%
\pgfpathlineto{\pgfqpoint{3.769778in}{0.738888in}}%
\pgfpathmoveto{\pgfqpoint{3.769778in}{0.738888in}}%
\pgfpathlineto{\pgfqpoint{3.769778in}{0.738888in}}%
\pgfpathlineto{\pgfqpoint{3.769778in}{0.741837in}}%
\pgfpathlineto{\pgfqpoint{3.774319in}{0.741837in}}%
\pgfpathlineto{\pgfqpoint{3.774319in}{0.738888in}}%
\pgfpathmoveto{\pgfqpoint{3.769778in}{0.741837in}}%
\pgfpathlineto{\pgfqpoint{3.769778in}{0.741837in}}%
\pgfpathlineto{\pgfqpoint{3.769778in}{0.744786in}}%
\pgfpathlineto{\pgfqpoint{3.774319in}{0.744786in}}%
\pgfpathlineto{\pgfqpoint{3.774319in}{0.741837in}}%
\pgfpathmoveto{\pgfqpoint{3.774319in}{0.741837in}}%
\pgfpathlineto{\pgfqpoint{3.774319in}{0.741837in}}%
\pgfpathlineto{\pgfqpoint{3.774319in}{0.744786in}}%
\pgfpathlineto{\pgfqpoint{3.778861in}{0.744786in}}%
\pgfpathlineto{\pgfqpoint{3.778861in}{0.741837in}}%
\pgfpathmoveto{\pgfqpoint{3.774319in}{0.744786in}}%
\pgfpathlineto{\pgfqpoint{3.774319in}{0.744786in}}%
\pgfpathlineto{\pgfqpoint{3.774319in}{0.747736in}}%
\pgfpathlineto{\pgfqpoint{3.778861in}{0.747736in}}%
\pgfpathlineto{\pgfqpoint{3.778861in}{0.744786in}}%
\pgfpathmoveto{\pgfqpoint{3.778861in}{0.744786in}}%
\pgfpathlineto{\pgfqpoint{3.778861in}{0.744786in}}%
\pgfpathlineto{\pgfqpoint{3.778861in}{0.747736in}}%
\pgfpathlineto{\pgfqpoint{3.783402in}{0.747736in}}%
\pgfpathlineto{\pgfqpoint{3.783402in}{0.744786in}}%
\pgfpathmoveto{\pgfqpoint{3.778861in}{0.747736in}}%
\pgfpathlineto{\pgfqpoint{3.778861in}{0.747736in}}%
\pgfpathlineto{\pgfqpoint{3.778861in}{0.750685in}}%
\pgfpathlineto{\pgfqpoint{3.783402in}{0.750685in}}%
\pgfpathlineto{\pgfqpoint{3.783402in}{0.747736in}}%
\pgfpathmoveto{\pgfqpoint{3.783402in}{0.747736in}}%
\pgfpathlineto{\pgfqpoint{3.783402in}{0.747736in}}%
\pgfpathlineto{\pgfqpoint{3.783402in}{0.750685in}}%
\pgfpathlineto{\pgfqpoint{3.787943in}{0.750685in}}%
\pgfpathlineto{\pgfqpoint{3.787943in}{0.747736in}}%
\pgfpathmoveto{\pgfqpoint{3.783402in}{0.750685in}}%
\pgfpathlineto{\pgfqpoint{3.783402in}{0.750685in}}%
\pgfpathlineto{\pgfqpoint{3.783402in}{0.753634in}}%
\pgfpathlineto{\pgfqpoint{3.787943in}{0.753634in}}%
\pgfpathlineto{\pgfqpoint{3.787943in}{0.750685in}}%
\pgfpathmoveto{\pgfqpoint{3.783402in}{0.753634in}}%
\pgfpathlineto{\pgfqpoint{3.783402in}{0.753634in}}%
\pgfpathlineto{\pgfqpoint{3.783402in}{0.756583in}}%
\pgfpathlineto{\pgfqpoint{3.787943in}{0.756583in}}%
\pgfpathlineto{\pgfqpoint{3.787943in}{0.753634in}}%
\pgfpathmoveto{\pgfqpoint{3.787943in}{0.753634in}}%
\pgfpathlineto{\pgfqpoint{3.787943in}{0.753634in}}%
\pgfpathlineto{\pgfqpoint{3.787943in}{0.756583in}}%
\pgfpathlineto{\pgfqpoint{3.792484in}{0.756583in}}%
\pgfpathlineto{\pgfqpoint{3.792484in}{0.753634in}}%
\pgfpathmoveto{\pgfqpoint{3.787943in}{0.756583in}}%
\pgfpathlineto{\pgfqpoint{3.787943in}{0.756583in}}%
\pgfpathlineto{\pgfqpoint{3.787943in}{0.759533in}}%
\pgfpathlineto{\pgfqpoint{3.792484in}{0.759533in}}%
\pgfpathlineto{\pgfqpoint{3.792484in}{0.756583in}}%
\pgfpathmoveto{\pgfqpoint{3.792484in}{0.756583in}}%
\pgfpathlineto{\pgfqpoint{3.792484in}{0.756583in}}%
\pgfpathlineto{\pgfqpoint{3.792484in}{0.759533in}}%
\pgfpathlineto{\pgfqpoint{3.797026in}{0.759533in}}%
\pgfpathlineto{\pgfqpoint{3.797026in}{0.756583in}}%
\pgfpathmoveto{\pgfqpoint{3.792484in}{0.759533in}}%
\pgfpathlineto{\pgfqpoint{3.792484in}{0.759533in}}%
\pgfpathlineto{\pgfqpoint{3.792484in}{0.762482in}}%
\pgfpathlineto{\pgfqpoint{3.797026in}{0.762482in}}%
\pgfpathlineto{\pgfqpoint{3.797026in}{0.759533in}}%
\pgfpathmoveto{\pgfqpoint{3.797026in}{0.759533in}}%
\pgfpathlineto{\pgfqpoint{3.797026in}{0.759533in}}%
\pgfpathlineto{\pgfqpoint{3.797026in}{0.762482in}}%
\pgfpathlineto{\pgfqpoint{3.801567in}{0.762482in}}%
\pgfpathlineto{\pgfqpoint{3.801567in}{0.759533in}}%
\pgfpathmoveto{\pgfqpoint{3.797026in}{0.762482in}}%
\pgfpathlineto{\pgfqpoint{3.797026in}{0.762482in}}%
\pgfpathlineto{\pgfqpoint{3.797026in}{0.765431in}}%
\pgfpathlineto{\pgfqpoint{3.801567in}{0.765431in}}%
\pgfpathlineto{\pgfqpoint{3.801567in}{0.762482in}}%
\pgfpathmoveto{\pgfqpoint{3.801567in}{0.762482in}}%
\pgfpathlineto{\pgfqpoint{3.801567in}{0.762482in}}%
\pgfpathlineto{\pgfqpoint{3.801567in}{0.765431in}}%
\pgfpathlineto{\pgfqpoint{3.806108in}{0.765431in}}%
\pgfpathlineto{\pgfqpoint{3.806108in}{0.762482in}}%
\pgfpathmoveto{\pgfqpoint{3.801567in}{0.765431in}}%
\pgfpathlineto{\pgfqpoint{3.801567in}{0.765431in}}%
\pgfpathlineto{\pgfqpoint{3.801567in}{0.768381in}}%
\pgfpathlineto{\pgfqpoint{3.806108in}{0.768381in}}%
\pgfpathlineto{\pgfqpoint{3.806108in}{0.765431in}}%
\pgfpathmoveto{\pgfqpoint{3.806108in}{0.765431in}}%
\pgfpathlineto{\pgfqpoint{3.806108in}{0.765431in}}%
\pgfpathlineto{\pgfqpoint{3.806108in}{0.768381in}}%
\pgfpathlineto{\pgfqpoint{3.810649in}{0.768381in}}%
\pgfpathlineto{\pgfqpoint{3.810649in}{0.765431in}}%
\pgfpathmoveto{\pgfqpoint{3.806108in}{0.768381in}}%
\pgfpathlineto{\pgfqpoint{3.806108in}{0.768381in}}%
\pgfpathlineto{\pgfqpoint{3.806108in}{0.771330in}}%
\pgfpathlineto{\pgfqpoint{3.810649in}{0.771330in}}%
\pgfpathlineto{\pgfqpoint{3.810649in}{0.768381in}}%
\pgfpathmoveto{\pgfqpoint{3.806108in}{0.771330in}}%
\pgfpathlineto{\pgfqpoint{3.806108in}{0.771330in}}%
\pgfpathlineto{\pgfqpoint{3.806108in}{0.774279in}}%
\pgfpathlineto{\pgfqpoint{3.810649in}{0.774279in}}%
\pgfpathlineto{\pgfqpoint{3.810649in}{0.771330in}}%
\pgfpathmoveto{\pgfqpoint{3.810649in}{0.771330in}}%
\pgfpathlineto{\pgfqpoint{3.810649in}{0.771330in}}%
\pgfpathlineto{\pgfqpoint{3.810649in}{0.774279in}}%
\pgfpathlineto{\pgfqpoint{3.815190in}{0.774279in}}%
\pgfpathlineto{\pgfqpoint{3.815190in}{0.771330in}}%
\pgfpathmoveto{\pgfqpoint{3.810649in}{0.774279in}}%
\pgfpathlineto{\pgfqpoint{3.810649in}{0.774279in}}%
\pgfpathlineto{\pgfqpoint{3.810649in}{0.777229in}}%
\pgfpathlineto{\pgfqpoint{3.815190in}{0.777229in}}%
\pgfpathlineto{\pgfqpoint{3.815190in}{0.774279in}}%
\pgfpathmoveto{\pgfqpoint{3.815190in}{0.774279in}}%
\pgfpathlineto{\pgfqpoint{3.815190in}{0.774279in}}%
\pgfpathlineto{\pgfqpoint{3.815190in}{0.777229in}}%
\pgfpathlineto{\pgfqpoint{3.819731in}{0.777229in}}%
\pgfpathlineto{\pgfqpoint{3.819731in}{0.774279in}}%
\pgfpathmoveto{\pgfqpoint{3.815190in}{0.777229in}}%
\pgfpathlineto{\pgfqpoint{3.815190in}{0.777229in}}%
\pgfpathlineto{\pgfqpoint{3.815190in}{0.780178in}}%
\pgfpathlineto{\pgfqpoint{3.819731in}{0.780178in}}%
\pgfpathlineto{\pgfqpoint{3.819731in}{0.777229in}}%
\pgfpathmoveto{\pgfqpoint{3.819731in}{0.777229in}}%
\pgfpathlineto{\pgfqpoint{3.819731in}{0.777229in}}%
\pgfpathlineto{\pgfqpoint{3.819731in}{0.780178in}}%
\pgfpathlineto{\pgfqpoint{3.824272in}{0.780178in}}%
\pgfpathlineto{\pgfqpoint{3.824272in}{0.777229in}}%
\pgfpathmoveto{\pgfqpoint{3.819731in}{0.780178in}}%
\pgfpathlineto{\pgfqpoint{3.819731in}{0.780178in}}%
\pgfpathlineto{\pgfqpoint{3.819731in}{0.783127in}}%
\pgfpathlineto{\pgfqpoint{3.824272in}{0.783127in}}%
\pgfpathlineto{\pgfqpoint{3.824272in}{0.780178in}}%
\pgfpathmoveto{\pgfqpoint{3.824272in}{0.780178in}}%
\pgfpathlineto{\pgfqpoint{3.824272in}{0.780178in}}%
\pgfpathlineto{\pgfqpoint{3.824272in}{0.783127in}}%
\pgfpathlineto{\pgfqpoint{3.828813in}{0.783127in}}%
\pgfpathlineto{\pgfqpoint{3.828813in}{0.780178in}}%
\pgfpathmoveto{\pgfqpoint{3.824272in}{0.783127in}}%
\pgfpathlineto{\pgfqpoint{3.824272in}{0.783127in}}%
\pgfpathlineto{\pgfqpoint{3.824272in}{0.786076in}}%
\pgfpathlineto{\pgfqpoint{3.828813in}{0.786076in}}%
\pgfpathlineto{\pgfqpoint{3.828813in}{0.783127in}}%
\pgfpathmoveto{\pgfqpoint{3.828813in}{0.783127in}}%
\pgfpathlineto{\pgfqpoint{3.828813in}{0.783127in}}%
\pgfpathlineto{\pgfqpoint{3.828813in}{0.786076in}}%
\pgfpathlineto{\pgfqpoint{3.833354in}{0.786076in}}%
\pgfpathlineto{\pgfqpoint{3.833354in}{0.783127in}}%
\pgfpathmoveto{\pgfqpoint{3.828813in}{0.786076in}}%
\pgfpathlineto{\pgfqpoint{3.828813in}{0.786076in}}%
\pgfpathlineto{\pgfqpoint{3.828813in}{0.789026in}}%
\pgfpathlineto{\pgfqpoint{3.833354in}{0.789026in}}%
\pgfpathlineto{\pgfqpoint{3.833354in}{0.786076in}}%
\pgfpathmoveto{\pgfqpoint{3.828813in}{0.789026in}}%
\pgfpathlineto{\pgfqpoint{3.828813in}{0.789026in}}%
\pgfpathlineto{\pgfqpoint{3.828813in}{0.791975in}}%
\pgfpathlineto{\pgfqpoint{3.833354in}{0.791975in}}%
\pgfpathlineto{\pgfqpoint{3.833354in}{0.789026in}}%
\pgfpathmoveto{\pgfqpoint{3.833354in}{0.789026in}}%
\pgfpathlineto{\pgfqpoint{3.833354in}{0.789026in}}%
\pgfpathlineto{\pgfqpoint{3.833354in}{0.791975in}}%
\pgfpathlineto{\pgfqpoint{3.837895in}{0.791975in}}%
\pgfpathlineto{\pgfqpoint{3.837895in}{0.789026in}}%
\pgfpathmoveto{\pgfqpoint{3.833354in}{0.791975in}}%
\pgfpathlineto{\pgfqpoint{3.833354in}{0.791975in}}%
\pgfpathlineto{\pgfqpoint{3.833354in}{0.794924in}}%
\pgfpathlineto{\pgfqpoint{3.837895in}{0.794924in}}%
\pgfpathlineto{\pgfqpoint{3.837895in}{0.791975in}}%
\pgfpathmoveto{\pgfqpoint{3.837895in}{0.791975in}}%
\pgfpathlineto{\pgfqpoint{3.837895in}{0.791975in}}%
\pgfpathlineto{\pgfqpoint{3.837895in}{0.794924in}}%
\pgfpathlineto{\pgfqpoint{3.842436in}{0.794924in}}%
\pgfpathlineto{\pgfqpoint{3.842436in}{0.791975in}}%
\pgfpathmoveto{\pgfqpoint{3.837895in}{0.794924in}}%
\pgfpathlineto{\pgfqpoint{3.837895in}{0.794924in}}%
\pgfpathlineto{\pgfqpoint{3.837895in}{0.797873in}}%
\pgfpathlineto{\pgfqpoint{3.842436in}{0.797873in}}%
\pgfpathlineto{\pgfqpoint{3.842436in}{0.794924in}}%
\pgfpathmoveto{\pgfqpoint{3.842436in}{0.794924in}}%
\pgfpathlineto{\pgfqpoint{3.842436in}{0.794924in}}%
\pgfpathlineto{\pgfqpoint{3.842436in}{0.797873in}}%
\pgfpathlineto{\pgfqpoint{3.846977in}{0.797873in}}%
\pgfpathlineto{\pgfqpoint{3.846977in}{0.794924in}}%
\pgfpathmoveto{\pgfqpoint{3.842436in}{0.797873in}}%
\pgfpathlineto{\pgfqpoint{3.842436in}{0.797873in}}%
\pgfpathlineto{\pgfqpoint{3.842436in}{0.800822in}}%
\pgfpathlineto{\pgfqpoint{3.846977in}{0.800822in}}%
\pgfpathlineto{\pgfqpoint{3.846977in}{0.797873in}}%
\pgfpathmoveto{\pgfqpoint{3.846977in}{0.797873in}}%
\pgfpathlineto{\pgfqpoint{3.846977in}{0.797873in}}%
\pgfpathlineto{\pgfqpoint{3.846977in}{0.800822in}}%
\pgfpathlineto{\pgfqpoint{3.851518in}{0.800822in}}%
\pgfpathlineto{\pgfqpoint{3.851518in}{0.797873in}}%
\pgfpathmoveto{\pgfqpoint{3.846977in}{0.800822in}}%
\pgfpathlineto{\pgfqpoint{3.846977in}{0.800822in}}%
\pgfpathlineto{\pgfqpoint{3.846977in}{0.803771in}}%
\pgfpathlineto{\pgfqpoint{3.851518in}{0.803771in}}%
\pgfpathlineto{\pgfqpoint{3.851518in}{0.800822in}}%
\pgfpathmoveto{\pgfqpoint{3.851518in}{0.800822in}}%
\pgfpathlineto{\pgfqpoint{3.851518in}{0.800822in}}%
\pgfpathlineto{\pgfqpoint{3.851518in}{0.803771in}}%
\pgfpathlineto{\pgfqpoint{3.856059in}{0.803771in}}%
\pgfpathlineto{\pgfqpoint{3.856059in}{0.800822in}}%
\pgfpathmoveto{\pgfqpoint{3.851518in}{0.803771in}}%
\pgfpathlineto{\pgfqpoint{3.851518in}{0.803771in}}%
\pgfpathlineto{\pgfqpoint{3.851518in}{0.806721in}}%
\pgfpathlineto{\pgfqpoint{3.856059in}{0.806721in}}%
\pgfpathlineto{\pgfqpoint{3.856059in}{0.803771in}}%
\pgfpathmoveto{\pgfqpoint{3.851518in}{0.806721in}}%
\pgfpathlineto{\pgfqpoint{3.851518in}{0.806721in}}%
\pgfpathlineto{\pgfqpoint{3.851518in}{0.809670in}}%
\pgfpathlineto{\pgfqpoint{3.856059in}{0.809670in}}%
\pgfpathlineto{\pgfqpoint{3.856059in}{0.806721in}}%
\pgfpathmoveto{\pgfqpoint{3.856059in}{0.806721in}}%
\pgfpathlineto{\pgfqpoint{3.856059in}{0.806721in}}%
\pgfpathlineto{\pgfqpoint{3.856059in}{0.809670in}}%
\pgfpathlineto{\pgfqpoint{3.860600in}{0.809670in}}%
\pgfpathlineto{\pgfqpoint{3.860600in}{0.806721in}}%
\pgfpathmoveto{\pgfqpoint{3.856059in}{0.809670in}}%
\pgfpathlineto{\pgfqpoint{3.856059in}{0.809670in}}%
\pgfpathlineto{\pgfqpoint{3.856059in}{0.812619in}}%
\pgfpathlineto{\pgfqpoint{3.860600in}{0.812619in}}%
\pgfpathlineto{\pgfqpoint{3.860600in}{0.809670in}}%
\pgfpathmoveto{\pgfqpoint{3.860600in}{0.809670in}}%
\pgfpathlineto{\pgfqpoint{3.860600in}{0.809670in}}%
\pgfpathlineto{\pgfqpoint{3.860600in}{0.812619in}}%
\pgfpathlineto{\pgfqpoint{3.865141in}{0.812619in}}%
\pgfpathlineto{\pgfqpoint{3.865141in}{0.809670in}}%
\pgfpathmoveto{\pgfqpoint{3.860600in}{0.812619in}}%
\pgfpathlineto{\pgfqpoint{3.860600in}{0.812619in}}%
\pgfpathlineto{\pgfqpoint{3.860600in}{0.815568in}}%
\pgfpathlineto{\pgfqpoint{3.865141in}{0.815568in}}%
\pgfpathlineto{\pgfqpoint{3.865141in}{0.812619in}}%
\pgfpathmoveto{\pgfqpoint{3.865141in}{0.812619in}}%
\pgfpathlineto{\pgfqpoint{3.865141in}{0.812619in}}%
\pgfpathlineto{\pgfqpoint{3.865141in}{0.815568in}}%
\pgfpathlineto{\pgfqpoint{3.869682in}{0.815568in}}%
\pgfpathlineto{\pgfqpoint{3.869682in}{0.812619in}}%
\pgfpathmoveto{\pgfqpoint{3.865141in}{0.815568in}}%
\pgfpathlineto{\pgfqpoint{3.865141in}{0.815568in}}%
\pgfpathlineto{\pgfqpoint{3.865141in}{0.818517in}}%
\pgfpathlineto{\pgfqpoint{3.869682in}{0.818517in}}%
\pgfpathlineto{\pgfqpoint{3.869682in}{0.815568in}}%
\pgfpathmoveto{\pgfqpoint{3.869682in}{0.815568in}}%
\pgfpathlineto{\pgfqpoint{3.869682in}{0.815568in}}%
\pgfpathlineto{\pgfqpoint{3.869682in}{0.818517in}}%
\pgfpathlineto{\pgfqpoint{3.874223in}{0.818517in}}%
\pgfpathlineto{\pgfqpoint{3.874223in}{0.815568in}}%
\pgfpathmoveto{\pgfqpoint{3.869682in}{0.818517in}}%
\pgfpathlineto{\pgfqpoint{3.869682in}{0.818517in}}%
\pgfpathlineto{\pgfqpoint{3.869682in}{0.821466in}}%
\pgfpathlineto{\pgfqpoint{3.874223in}{0.821466in}}%
\pgfpathlineto{\pgfqpoint{3.874223in}{0.818517in}}%
\pgfpathmoveto{\pgfqpoint{3.874223in}{0.818517in}}%
\pgfpathlineto{\pgfqpoint{3.874223in}{0.818517in}}%
\pgfpathlineto{\pgfqpoint{3.874223in}{0.821466in}}%
\pgfpathlineto{\pgfqpoint{3.878764in}{0.821466in}}%
\pgfpathlineto{\pgfqpoint{3.878764in}{0.818517in}}%
\pgfpathmoveto{\pgfqpoint{3.874223in}{0.821466in}}%
\pgfpathlineto{\pgfqpoint{3.874223in}{0.821466in}}%
\pgfpathlineto{\pgfqpoint{3.874223in}{0.824416in}}%
\pgfpathlineto{\pgfqpoint{3.878764in}{0.824416in}}%
\pgfpathlineto{\pgfqpoint{3.878764in}{0.821466in}}%
\pgfpathmoveto{\pgfqpoint{3.874223in}{0.824416in}}%
\pgfpathlineto{\pgfqpoint{3.874223in}{0.824416in}}%
\pgfpathlineto{\pgfqpoint{3.874223in}{0.827365in}}%
\pgfpathlineto{\pgfqpoint{3.878764in}{0.827365in}}%
\pgfpathlineto{\pgfqpoint{3.878764in}{0.824416in}}%
\pgfpathmoveto{\pgfqpoint{3.878764in}{0.824416in}}%
\pgfpathlineto{\pgfqpoint{3.878764in}{0.824416in}}%
\pgfpathlineto{\pgfqpoint{3.878764in}{0.827365in}}%
\pgfpathlineto{\pgfqpoint{3.883305in}{0.827365in}}%
\pgfpathlineto{\pgfqpoint{3.883305in}{0.824416in}}%
\pgfpathmoveto{\pgfqpoint{3.878764in}{0.827365in}}%
\pgfpathlineto{\pgfqpoint{3.878764in}{0.827365in}}%
\pgfpathlineto{\pgfqpoint{3.878764in}{0.830314in}}%
\pgfpathlineto{\pgfqpoint{3.883305in}{0.830314in}}%
\pgfpathlineto{\pgfqpoint{3.883305in}{0.827365in}}%
\pgfpathmoveto{\pgfqpoint{3.883305in}{0.827365in}}%
\pgfpathlineto{\pgfqpoint{3.883305in}{0.827365in}}%
\pgfpathlineto{\pgfqpoint{3.883305in}{0.830314in}}%
\pgfpathlineto{\pgfqpoint{3.887846in}{0.830314in}}%
\pgfpathlineto{\pgfqpoint{3.887846in}{0.827365in}}%
\pgfpathmoveto{\pgfqpoint{3.883305in}{0.830314in}}%
\pgfpathlineto{\pgfqpoint{3.883305in}{0.830314in}}%
\pgfpathlineto{\pgfqpoint{3.883305in}{0.833263in}}%
\pgfpathlineto{\pgfqpoint{3.887846in}{0.833263in}}%
\pgfpathlineto{\pgfqpoint{3.887846in}{0.830314in}}%
\pgfpathmoveto{\pgfqpoint{3.887846in}{0.830314in}}%
\pgfpathlineto{\pgfqpoint{3.887846in}{0.830314in}}%
\pgfpathlineto{\pgfqpoint{3.887846in}{0.833263in}}%
\pgfpathlineto{\pgfqpoint{3.892387in}{0.833263in}}%
\pgfpathlineto{\pgfqpoint{3.892387in}{0.830314in}}%
\pgfpathmoveto{\pgfqpoint{3.887846in}{0.833263in}}%
\pgfpathlineto{\pgfqpoint{3.887846in}{0.833263in}}%
\pgfpathlineto{\pgfqpoint{3.887846in}{0.836212in}}%
\pgfpathlineto{\pgfqpoint{3.892387in}{0.836212in}}%
\pgfpathlineto{\pgfqpoint{3.892387in}{0.833263in}}%
\pgfpathmoveto{\pgfqpoint{3.892387in}{0.833263in}}%
\pgfpathlineto{\pgfqpoint{3.892387in}{0.833263in}}%
\pgfpathlineto{\pgfqpoint{3.892387in}{0.836212in}}%
\pgfpathlineto{\pgfqpoint{3.896928in}{0.836212in}}%
\pgfpathlineto{\pgfqpoint{3.896928in}{0.833263in}}%
\pgfpathmoveto{\pgfqpoint{3.892387in}{0.836212in}}%
\pgfpathlineto{\pgfqpoint{3.892387in}{0.836212in}}%
\pgfpathlineto{\pgfqpoint{3.892387in}{0.839161in}}%
\pgfpathlineto{\pgfqpoint{3.896928in}{0.839161in}}%
\pgfpathlineto{\pgfqpoint{3.896928in}{0.836212in}}%
\pgfpathmoveto{\pgfqpoint{3.896928in}{0.836212in}}%
\pgfpathlineto{\pgfqpoint{3.896928in}{0.836212in}}%
\pgfpathlineto{\pgfqpoint{3.896928in}{0.839161in}}%
\pgfpathlineto{\pgfqpoint{3.901469in}{0.839161in}}%
\pgfpathlineto{\pgfqpoint{3.901469in}{0.836212in}}%
\pgfpathmoveto{\pgfqpoint{3.896928in}{0.839161in}}%
\pgfpathlineto{\pgfqpoint{3.896928in}{0.839161in}}%
\pgfpathlineto{\pgfqpoint{3.896928in}{0.842111in}}%
\pgfpathlineto{\pgfqpoint{3.901469in}{0.842111in}}%
\pgfpathlineto{\pgfqpoint{3.901469in}{0.839161in}}%
\pgfpathmoveto{\pgfqpoint{3.896928in}{0.842111in}}%
\pgfpathlineto{\pgfqpoint{3.896928in}{0.842111in}}%
\pgfpathlineto{\pgfqpoint{3.896928in}{0.845060in}}%
\pgfpathlineto{\pgfqpoint{3.901469in}{0.845060in}}%
\pgfpathlineto{\pgfqpoint{3.901469in}{0.842111in}}%
\pgfpathmoveto{\pgfqpoint{3.901469in}{0.842111in}}%
\pgfpathlineto{\pgfqpoint{3.901469in}{0.842111in}}%
\pgfpathlineto{\pgfqpoint{3.901469in}{0.845060in}}%
\pgfpathlineto{\pgfqpoint{3.906010in}{0.845060in}}%
\pgfpathlineto{\pgfqpoint{3.906010in}{0.842111in}}%
\pgfpathmoveto{\pgfqpoint{3.901469in}{0.845060in}}%
\pgfpathlineto{\pgfqpoint{3.901469in}{0.845060in}}%
\pgfpathlineto{\pgfqpoint{3.901469in}{0.848009in}}%
\pgfpathlineto{\pgfqpoint{3.906010in}{0.848009in}}%
\pgfpathlineto{\pgfqpoint{3.906010in}{0.845060in}}%
\pgfpathmoveto{\pgfqpoint{3.906010in}{0.845060in}}%
\pgfpathlineto{\pgfqpoint{3.906010in}{0.845060in}}%
\pgfpathlineto{\pgfqpoint{3.906010in}{0.848009in}}%
\pgfpathlineto{\pgfqpoint{3.910551in}{0.848009in}}%
\pgfpathlineto{\pgfqpoint{3.910551in}{0.845060in}}%
\pgfpathmoveto{\pgfqpoint{3.906010in}{0.848009in}}%
\pgfpathlineto{\pgfqpoint{3.906010in}{0.848009in}}%
\pgfpathlineto{\pgfqpoint{3.906010in}{0.850958in}}%
\pgfpathlineto{\pgfqpoint{3.910551in}{0.850958in}}%
\pgfpathlineto{\pgfqpoint{3.910551in}{0.848009in}}%
\pgfpathmoveto{\pgfqpoint{3.910551in}{0.848009in}}%
\pgfpathlineto{\pgfqpoint{3.910551in}{0.848009in}}%
\pgfpathlineto{\pgfqpoint{3.910551in}{0.850958in}}%
\pgfpathlineto{\pgfqpoint{3.915092in}{0.850958in}}%
\pgfpathlineto{\pgfqpoint{3.915092in}{0.848009in}}%
\pgfpathmoveto{\pgfqpoint{3.910551in}{0.850958in}}%
\pgfpathlineto{\pgfqpoint{3.910551in}{0.850958in}}%
\pgfpathlineto{\pgfqpoint{3.910551in}{0.853907in}}%
\pgfpathlineto{\pgfqpoint{3.915092in}{0.853907in}}%
\pgfpathlineto{\pgfqpoint{3.915092in}{0.850958in}}%
\pgfpathmoveto{\pgfqpoint{3.915092in}{0.850958in}}%
\pgfpathlineto{\pgfqpoint{3.915092in}{0.850958in}}%
\pgfpathlineto{\pgfqpoint{3.915092in}{0.853907in}}%
\pgfpathlineto{\pgfqpoint{3.919633in}{0.853907in}}%
\pgfpathlineto{\pgfqpoint{3.919633in}{0.850958in}}%
\pgfpathmoveto{\pgfqpoint{3.915092in}{0.853907in}}%
\pgfpathlineto{\pgfqpoint{3.915092in}{0.853907in}}%
\pgfpathlineto{\pgfqpoint{3.915092in}{0.856856in}}%
\pgfpathlineto{\pgfqpoint{3.919633in}{0.856856in}}%
\pgfpathlineto{\pgfqpoint{3.919633in}{0.853907in}}%
\pgfpathmoveto{\pgfqpoint{3.919633in}{0.853907in}}%
\pgfpathlineto{\pgfqpoint{3.919633in}{0.853907in}}%
\pgfpathlineto{\pgfqpoint{3.919633in}{0.856856in}}%
\pgfpathlineto{\pgfqpoint{3.924174in}{0.856856in}}%
\pgfpathlineto{\pgfqpoint{3.924174in}{0.853907in}}%
\pgfpathmoveto{\pgfqpoint{3.919633in}{0.856856in}}%
\pgfpathlineto{\pgfqpoint{3.919633in}{0.856856in}}%
\pgfpathlineto{\pgfqpoint{3.919633in}{0.859806in}}%
\pgfpathlineto{\pgfqpoint{3.924174in}{0.859806in}}%
\pgfpathlineto{\pgfqpoint{3.924174in}{0.856856in}}%
\pgfpathmoveto{\pgfqpoint{3.919633in}{0.859806in}}%
\pgfpathlineto{\pgfqpoint{3.919633in}{0.859806in}}%
\pgfpathlineto{\pgfqpoint{3.919633in}{0.862755in}}%
\pgfpathlineto{\pgfqpoint{3.924174in}{0.862755in}}%
\pgfpathlineto{\pgfqpoint{3.924174in}{0.859806in}}%
\pgfpathmoveto{\pgfqpoint{3.924174in}{0.859806in}}%
\pgfpathlineto{\pgfqpoint{3.924174in}{0.859806in}}%
\pgfpathlineto{\pgfqpoint{3.924174in}{0.862755in}}%
\pgfpathlineto{\pgfqpoint{3.928715in}{0.862755in}}%
\pgfpathlineto{\pgfqpoint{3.928715in}{0.859806in}}%
\pgfpathmoveto{\pgfqpoint{3.924174in}{0.862755in}}%
\pgfpathlineto{\pgfqpoint{3.924174in}{0.862755in}}%
\pgfpathlineto{\pgfqpoint{3.924174in}{0.865704in}}%
\pgfpathlineto{\pgfqpoint{3.928715in}{0.865704in}}%
\pgfpathlineto{\pgfqpoint{3.928715in}{0.862755in}}%
\pgfpathmoveto{\pgfqpoint{3.928715in}{0.862755in}}%
\pgfpathlineto{\pgfqpoint{3.928715in}{0.862755in}}%
\pgfpathlineto{\pgfqpoint{3.928715in}{0.865704in}}%
\pgfpathlineto{\pgfqpoint{3.933256in}{0.865704in}}%
\pgfpathlineto{\pgfqpoint{3.933256in}{0.862755in}}%
\pgfpathmoveto{\pgfqpoint{3.928715in}{0.865704in}}%
\pgfpathlineto{\pgfqpoint{3.928715in}{0.865704in}}%
\pgfpathlineto{\pgfqpoint{3.928715in}{0.868653in}}%
\pgfpathlineto{\pgfqpoint{3.933256in}{0.868653in}}%
\pgfpathlineto{\pgfqpoint{3.933256in}{0.865704in}}%
\pgfpathmoveto{\pgfqpoint{3.933256in}{0.865704in}}%
\pgfpathlineto{\pgfqpoint{3.933256in}{0.865704in}}%
\pgfpathlineto{\pgfqpoint{3.933256in}{0.868653in}}%
\pgfpathlineto{\pgfqpoint{3.937797in}{0.868653in}}%
\pgfpathlineto{\pgfqpoint{3.937797in}{0.865704in}}%
\pgfpathmoveto{\pgfqpoint{3.933256in}{0.868653in}}%
\pgfpathlineto{\pgfqpoint{3.933256in}{0.868653in}}%
\pgfpathlineto{\pgfqpoint{3.933256in}{0.871602in}}%
\pgfpathlineto{\pgfqpoint{3.937797in}{0.871602in}}%
\pgfpathlineto{\pgfqpoint{3.937797in}{0.868653in}}%
\pgfpathmoveto{\pgfqpoint{3.937797in}{0.868653in}}%
\pgfpathlineto{\pgfqpoint{3.937797in}{0.868653in}}%
\pgfpathlineto{\pgfqpoint{3.937797in}{0.871602in}}%
\pgfpathlineto{\pgfqpoint{3.942338in}{0.871602in}}%
\pgfpathlineto{\pgfqpoint{3.942338in}{0.868653in}}%
\pgfpathmoveto{\pgfqpoint{3.937797in}{0.871602in}}%
\pgfpathlineto{\pgfqpoint{3.937797in}{0.871602in}}%
\pgfpathlineto{\pgfqpoint{3.937797in}{0.874552in}}%
\pgfpathlineto{\pgfqpoint{3.942338in}{0.874552in}}%
\pgfpathlineto{\pgfqpoint{3.942338in}{0.871602in}}%
\pgfpathmoveto{\pgfqpoint{3.942338in}{0.871602in}}%
\pgfpathlineto{\pgfqpoint{3.942338in}{0.871602in}}%
\pgfpathlineto{\pgfqpoint{3.942338in}{0.874552in}}%
\pgfpathlineto{\pgfqpoint{3.946879in}{0.874552in}}%
\pgfpathlineto{\pgfqpoint{3.946879in}{0.871602in}}%
\pgfpathmoveto{\pgfqpoint{3.942338in}{0.874552in}}%
\pgfpathlineto{\pgfqpoint{3.942338in}{0.874552in}}%
\pgfpathlineto{\pgfqpoint{3.942338in}{0.877501in}}%
\pgfpathlineto{\pgfqpoint{3.946879in}{0.877501in}}%
\pgfpathlineto{\pgfqpoint{3.946879in}{0.874552in}}%
\pgfpathmoveto{\pgfqpoint{3.942338in}{0.877501in}}%
\pgfpathlineto{\pgfqpoint{3.942338in}{0.877501in}}%
\pgfpathlineto{\pgfqpoint{3.942338in}{0.880450in}}%
\pgfpathlineto{\pgfqpoint{3.946879in}{0.880450in}}%
\pgfpathlineto{\pgfqpoint{3.946879in}{0.877501in}}%
\pgfpathmoveto{\pgfqpoint{3.946879in}{0.877501in}}%
\pgfpathlineto{\pgfqpoint{3.946879in}{0.877501in}}%
\pgfpathlineto{\pgfqpoint{3.946879in}{0.880450in}}%
\pgfpathlineto{\pgfqpoint{3.951420in}{0.880450in}}%
\pgfpathlineto{\pgfqpoint{3.951420in}{0.877501in}}%
\pgfpathmoveto{\pgfqpoint{3.946879in}{0.880450in}}%
\pgfpathlineto{\pgfqpoint{3.946879in}{0.880450in}}%
\pgfpathlineto{\pgfqpoint{3.946879in}{0.883399in}}%
\pgfpathlineto{\pgfqpoint{3.951420in}{0.883399in}}%
\pgfpathlineto{\pgfqpoint{3.951420in}{0.880450in}}%
\pgfpathmoveto{\pgfqpoint{3.951420in}{0.880450in}}%
\pgfpathlineto{\pgfqpoint{3.951420in}{0.880450in}}%
\pgfpathlineto{\pgfqpoint{3.951420in}{0.883399in}}%
\pgfpathlineto{\pgfqpoint{3.955961in}{0.883399in}}%
\pgfpathlineto{\pgfqpoint{3.955961in}{0.880450in}}%
\pgfpathmoveto{\pgfqpoint{3.951420in}{0.883399in}}%
\pgfpathlineto{\pgfqpoint{3.951420in}{0.883399in}}%
\pgfpathlineto{\pgfqpoint{3.951420in}{0.886348in}}%
\pgfpathlineto{\pgfqpoint{3.955961in}{0.886348in}}%
\pgfpathlineto{\pgfqpoint{3.955961in}{0.883399in}}%
\pgfpathmoveto{\pgfqpoint{3.955961in}{0.883399in}}%
\pgfpathlineto{\pgfqpoint{3.955961in}{0.883399in}}%
\pgfpathlineto{\pgfqpoint{3.955961in}{0.886348in}}%
\pgfpathlineto{\pgfqpoint{3.960502in}{0.886348in}}%
\pgfpathlineto{\pgfqpoint{3.960502in}{0.883399in}}%
\pgfpathmoveto{\pgfqpoint{3.955961in}{0.886348in}}%
\pgfpathlineto{\pgfqpoint{3.955961in}{0.886348in}}%
\pgfpathlineto{\pgfqpoint{3.955961in}{0.889298in}}%
\pgfpathlineto{\pgfqpoint{3.960502in}{0.889298in}}%
\pgfpathlineto{\pgfqpoint{3.960502in}{0.886348in}}%
\pgfpathmoveto{\pgfqpoint{3.960502in}{0.886348in}}%
\pgfpathlineto{\pgfqpoint{3.960502in}{0.886348in}}%
\pgfpathlineto{\pgfqpoint{3.960502in}{0.889298in}}%
\pgfpathlineto{\pgfqpoint{3.965043in}{0.889298in}}%
\pgfpathlineto{\pgfqpoint{3.965043in}{0.886348in}}%
\pgfpathmoveto{\pgfqpoint{3.960502in}{0.889298in}}%
\pgfpathlineto{\pgfqpoint{3.960502in}{0.889298in}}%
\pgfpathlineto{\pgfqpoint{3.960502in}{0.892247in}}%
\pgfpathlineto{\pgfqpoint{3.965043in}{0.892247in}}%
\pgfpathlineto{\pgfqpoint{3.965043in}{0.889298in}}%
\pgfpathmoveto{\pgfqpoint{3.965043in}{0.889298in}}%
\pgfpathlineto{\pgfqpoint{3.965043in}{0.889298in}}%
\pgfpathlineto{\pgfqpoint{3.965043in}{0.892247in}}%
\pgfpathlineto{\pgfqpoint{3.969584in}{0.892247in}}%
\pgfpathlineto{\pgfqpoint{3.969584in}{0.889298in}}%
\pgfpathmoveto{\pgfqpoint{3.965043in}{0.892247in}}%
\pgfpathlineto{\pgfqpoint{3.965043in}{0.892247in}}%
\pgfpathlineto{\pgfqpoint{3.965043in}{0.895196in}}%
\pgfpathlineto{\pgfqpoint{3.969584in}{0.895196in}}%
\pgfpathlineto{\pgfqpoint{3.969584in}{0.892247in}}%
\pgfpathmoveto{\pgfqpoint{3.965043in}{0.895196in}}%
\pgfpathlineto{\pgfqpoint{3.965043in}{0.895196in}}%
\pgfpathlineto{\pgfqpoint{3.965043in}{0.898145in}}%
\pgfpathlineto{\pgfqpoint{3.969584in}{0.898145in}}%
\pgfpathlineto{\pgfqpoint{3.969584in}{0.895196in}}%
\pgfpathmoveto{\pgfqpoint{3.969584in}{0.895196in}}%
\pgfpathlineto{\pgfqpoint{3.969584in}{0.895196in}}%
\pgfpathlineto{\pgfqpoint{3.969584in}{0.898145in}}%
\pgfpathlineto{\pgfqpoint{3.974125in}{0.898145in}}%
\pgfpathlineto{\pgfqpoint{3.974125in}{0.895196in}}%
\pgfpathmoveto{\pgfqpoint{3.969584in}{0.898145in}}%
\pgfpathlineto{\pgfqpoint{3.969584in}{0.898145in}}%
\pgfpathlineto{\pgfqpoint{3.969584in}{0.901095in}}%
\pgfpathlineto{\pgfqpoint{3.974125in}{0.901095in}}%
\pgfpathlineto{\pgfqpoint{3.974125in}{0.898145in}}%
\pgfpathmoveto{\pgfqpoint{3.974125in}{0.898145in}}%
\pgfpathlineto{\pgfqpoint{3.974125in}{0.898145in}}%
\pgfpathlineto{\pgfqpoint{3.974125in}{0.901095in}}%
\pgfpathlineto{\pgfqpoint{3.978666in}{0.901095in}}%
\pgfpathlineto{\pgfqpoint{3.978666in}{0.898145in}}%
\pgfpathmoveto{\pgfqpoint{3.974125in}{0.901095in}}%
\pgfpathlineto{\pgfqpoint{3.974125in}{0.901095in}}%
\pgfpathlineto{\pgfqpoint{3.974125in}{0.904044in}}%
\pgfpathlineto{\pgfqpoint{3.978666in}{0.904044in}}%
\pgfpathlineto{\pgfqpoint{3.978666in}{0.901095in}}%
\pgfpathmoveto{\pgfqpoint{3.978666in}{0.901095in}}%
\pgfpathlineto{\pgfqpoint{3.978666in}{0.901095in}}%
\pgfpathlineto{\pgfqpoint{3.978666in}{0.904044in}}%
\pgfpathlineto{\pgfqpoint{3.983207in}{0.904044in}}%
\pgfpathlineto{\pgfqpoint{3.983207in}{0.901095in}}%
\pgfpathmoveto{\pgfqpoint{3.978666in}{0.904044in}}%
\pgfpathlineto{\pgfqpoint{3.978666in}{0.904044in}}%
\pgfpathlineto{\pgfqpoint{3.978666in}{0.906993in}}%
\pgfpathlineto{\pgfqpoint{3.983207in}{0.906993in}}%
\pgfpathlineto{\pgfqpoint{3.983207in}{0.904044in}}%
\pgfpathmoveto{\pgfqpoint{3.983207in}{0.904044in}}%
\pgfpathlineto{\pgfqpoint{3.983207in}{0.904044in}}%
\pgfpathlineto{\pgfqpoint{3.983207in}{0.906993in}}%
\pgfpathlineto{\pgfqpoint{3.987747in}{0.906993in}}%
\pgfpathlineto{\pgfqpoint{3.987747in}{0.904044in}}%
\pgfpathmoveto{\pgfqpoint{3.983207in}{0.906993in}}%
\pgfpathlineto{\pgfqpoint{3.983207in}{0.906993in}}%
\pgfpathlineto{\pgfqpoint{3.983207in}{0.909942in}}%
\pgfpathlineto{\pgfqpoint{3.987747in}{0.909942in}}%
\pgfpathlineto{\pgfqpoint{3.987747in}{0.906993in}}%
\pgfpathmoveto{\pgfqpoint{3.987747in}{0.906993in}}%
\pgfpathlineto{\pgfqpoint{3.987747in}{0.906993in}}%
\pgfpathlineto{\pgfqpoint{3.987747in}{0.909942in}}%
\pgfpathlineto{\pgfqpoint{3.992288in}{0.909942in}}%
\pgfpathlineto{\pgfqpoint{3.992288in}{0.906993in}}%
\pgfpathmoveto{\pgfqpoint{3.987747in}{0.909942in}}%
\pgfpathlineto{\pgfqpoint{3.987747in}{0.909942in}}%
\pgfpathlineto{\pgfqpoint{3.987747in}{0.912892in}}%
\pgfpathlineto{\pgfqpoint{3.992288in}{0.912892in}}%
\pgfpathlineto{\pgfqpoint{3.992288in}{0.909942in}}%
\pgfpathmoveto{\pgfqpoint{3.987747in}{0.912892in}}%
\pgfpathlineto{\pgfqpoint{3.987747in}{0.912892in}}%
\pgfpathlineto{\pgfqpoint{3.987747in}{0.915841in}}%
\pgfpathlineto{\pgfqpoint{3.992288in}{0.915841in}}%
\pgfpathlineto{\pgfqpoint{3.992288in}{0.912892in}}%
\pgfpathmoveto{\pgfqpoint{3.992288in}{0.912892in}}%
\pgfpathlineto{\pgfqpoint{3.992288in}{0.912892in}}%
\pgfpathlineto{\pgfqpoint{3.992288in}{0.915841in}}%
\pgfpathlineto{\pgfqpoint{3.996829in}{0.915841in}}%
\pgfpathlineto{\pgfqpoint{3.996829in}{0.912892in}}%
\pgfpathmoveto{\pgfqpoint{3.992288in}{0.915841in}}%
\pgfpathlineto{\pgfqpoint{3.992288in}{0.915841in}}%
\pgfpathlineto{\pgfqpoint{3.992288in}{0.918790in}}%
\pgfpathlineto{\pgfqpoint{3.996829in}{0.918790in}}%
\pgfpathlineto{\pgfqpoint{3.996829in}{0.915841in}}%
\pgfpathmoveto{\pgfqpoint{3.996829in}{0.915841in}}%
\pgfpathlineto{\pgfqpoint{3.996829in}{0.915841in}}%
\pgfpathlineto{\pgfqpoint{3.996829in}{0.918790in}}%
\pgfpathlineto{\pgfqpoint{4.001370in}{0.918790in}}%
\pgfpathlineto{\pgfqpoint{4.001370in}{0.915841in}}%
\pgfpathmoveto{\pgfqpoint{3.996829in}{0.918790in}}%
\pgfpathlineto{\pgfqpoint{3.996829in}{0.918790in}}%
\pgfpathlineto{\pgfqpoint{3.996829in}{0.921739in}}%
\pgfpathlineto{\pgfqpoint{4.001370in}{0.921739in}}%
\pgfpathlineto{\pgfqpoint{4.001370in}{0.918790in}}%
\pgfpathmoveto{\pgfqpoint{4.001370in}{0.918790in}}%
\pgfpathlineto{\pgfqpoint{4.001370in}{0.918790in}}%
\pgfpathlineto{\pgfqpoint{4.001370in}{0.921739in}}%
\pgfpathlineto{\pgfqpoint{4.005911in}{0.921739in}}%
\pgfpathlineto{\pgfqpoint{4.005911in}{0.918790in}}%
\pgfpathmoveto{\pgfqpoint{4.001370in}{0.921739in}}%
\pgfpathlineto{\pgfqpoint{4.001370in}{0.921739in}}%
\pgfpathlineto{\pgfqpoint{4.001370in}{0.924689in}}%
\pgfpathlineto{\pgfqpoint{4.005911in}{0.924689in}}%
\pgfpathlineto{\pgfqpoint{4.005911in}{0.921739in}}%
\pgfpathmoveto{\pgfqpoint{4.005911in}{0.921739in}}%
\pgfpathlineto{\pgfqpoint{4.005911in}{0.921739in}}%
\pgfpathlineto{\pgfqpoint{4.005911in}{0.924689in}}%
\pgfpathlineto{\pgfqpoint{4.010452in}{0.924689in}}%
\pgfpathlineto{\pgfqpoint{4.010452in}{0.921739in}}%
\pgfpathmoveto{\pgfqpoint{4.005911in}{0.924689in}}%
\pgfpathlineto{\pgfqpoint{4.005911in}{0.924689in}}%
\pgfpathlineto{\pgfqpoint{4.005911in}{0.927638in}}%
\pgfpathlineto{\pgfqpoint{4.010452in}{0.927638in}}%
\pgfpathlineto{\pgfqpoint{4.010452in}{0.924689in}}%
\pgfpathmoveto{\pgfqpoint{4.010452in}{0.924689in}}%
\pgfpathlineto{\pgfqpoint{4.010452in}{0.924689in}}%
\pgfpathlineto{\pgfqpoint{4.010452in}{0.927638in}}%
\pgfpathlineto{\pgfqpoint{4.014993in}{0.927638in}}%
\pgfpathlineto{\pgfqpoint{4.014993in}{0.924689in}}%
\pgfpathmoveto{\pgfqpoint{4.010452in}{0.927638in}}%
\pgfpathlineto{\pgfqpoint{4.010452in}{0.927638in}}%
\pgfpathlineto{\pgfqpoint{4.010452in}{0.930587in}}%
\pgfpathlineto{\pgfqpoint{4.014993in}{0.930587in}}%
\pgfpathlineto{\pgfqpoint{4.014993in}{0.927638in}}%
\pgfpathmoveto{\pgfqpoint{4.010452in}{0.930587in}}%
\pgfpathlineto{\pgfqpoint{4.010452in}{0.930587in}}%
\pgfpathlineto{\pgfqpoint{4.010452in}{0.933536in}}%
\pgfpathlineto{\pgfqpoint{4.014993in}{0.933536in}}%
\pgfpathlineto{\pgfqpoint{4.014993in}{0.930587in}}%
\pgfpathmoveto{\pgfqpoint{4.014993in}{0.930587in}}%
\pgfpathlineto{\pgfqpoint{4.014993in}{0.930587in}}%
\pgfpathlineto{\pgfqpoint{4.014993in}{0.933536in}}%
\pgfpathlineto{\pgfqpoint{4.019534in}{0.933536in}}%
\pgfpathlineto{\pgfqpoint{4.019534in}{0.930587in}}%
\pgfpathmoveto{\pgfqpoint{4.014993in}{0.933536in}}%
\pgfpathlineto{\pgfqpoint{4.014993in}{0.933536in}}%
\pgfpathlineto{\pgfqpoint{4.014993in}{0.936485in}}%
\pgfpathlineto{\pgfqpoint{4.019534in}{0.936485in}}%
\pgfpathlineto{\pgfqpoint{4.019534in}{0.933536in}}%
\pgfpathmoveto{\pgfqpoint{4.019534in}{0.933536in}}%
\pgfpathlineto{\pgfqpoint{4.019534in}{0.933536in}}%
\pgfpathlineto{\pgfqpoint{4.019534in}{0.936485in}}%
\pgfpathlineto{\pgfqpoint{4.024075in}{0.936485in}}%
\pgfpathlineto{\pgfqpoint{4.024075in}{0.933536in}}%
\pgfpathmoveto{\pgfqpoint{4.019534in}{0.936485in}}%
\pgfpathlineto{\pgfqpoint{4.019534in}{0.936485in}}%
\pgfpathlineto{\pgfqpoint{4.019534in}{0.939435in}}%
\pgfpathlineto{\pgfqpoint{4.024075in}{0.939435in}}%
\pgfpathlineto{\pgfqpoint{4.024075in}{0.936485in}}%
\pgfpathmoveto{\pgfqpoint{4.024075in}{0.936485in}}%
\pgfpathlineto{\pgfqpoint{4.024075in}{0.936485in}}%
\pgfpathlineto{\pgfqpoint{4.024075in}{0.939435in}}%
\pgfpathlineto{\pgfqpoint{4.028616in}{0.939435in}}%
\pgfpathlineto{\pgfqpoint{4.028616in}{0.936485in}}%
\pgfpathmoveto{\pgfqpoint{4.024075in}{0.939435in}}%
\pgfpathlineto{\pgfqpoint{4.024075in}{0.939435in}}%
\pgfpathlineto{\pgfqpoint{4.024075in}{0.942384in}}%
\pgfpathlineto{\pgfqpoint{4.028616in}{0.942384in}}%
\pgfpathlineto{\pgfqpoint{4.028616in}{0.939435in}}%
\pgfpathmoveto{\pgfqpoint{4.028616in}{0.939435in}}%
\pgfpathlineto{\pgfqpoint{4.028616in}{0.939435in}}%
\pgfpathlineto{\pgfqpoint{4.028616in}{0.942384in}}%
\pgfpathlineto{\pgfqpoint{4.033157in}{0.942384in}}%
\pgfpathlineto{\pgfqpoint{4.033157in}{0.939435in}}%
\pgfpathmoveto{\pgfqpoint{4.028616in}{0.942384in}}%
\pgfpathlineto{\pgfqpoint{4.028616in}{0.942384in}}%
\pgfpathlineto{\pgfqpoint{4.028616in}{0.945333in}}%
\pgfpathlineto{\pgfqpoint{4.033157in}{0.945333in}}%
\pgfpathlineto{\pgfqpoint{4.033157in}{0.942384in}}%
\pgfpathmoveto{\pgfqpoint{4.033157in}{0.942384in}}%
\pgfpathlineto{\pgfqpoint{4.033157in}{0.942384in}}%
\pgfpathlineto{\pgfqpoint{4.033157in}{0.945333in}}%
\pgfpathlineto{\pgfqpoint{4.037698in}{0.945333in}}%
\pgfpathlineto{\pgfqpoint{4.037698in}{0.942384in}}%
\pgfpathmoveto{\pgfqpoint{4.033157in}{0.945333in}}%
\pgfpathlineto{\pgfqpoint{4.033157in}{0.945333in}}%
\pgfpathlineto{\pgfqpoint{4.033157in}{0.948282in}}%
\pgfpathlineto{\pgfqpoint{4.037698in}{0.948282in}}%
\pgfpathlineto{\pgfqpoint{4.037698in}{0.945333in}}%
\pgfpathmoveto{\pgfqpoint{4.033157in}{0.948282in}}%
\pgfpathlineto{\pgfqpoint{4.033157in}{0.948282in}}%
\pgfpathlineto{\pgfqpoint{4.033157in}{0.951232in}}%
\pgfpathlineto{\pgfqpoint{4.037698in}{0.951232in}}%
\pgfpathlineto{\pgfqpoint{4.037698in}{0.948282in}}%
\pgfpathmoveto{\pgfqpoint{4.037698in}{0.948282in}}%
\pgfpathlineto{\pgfqpoint{4.037698in}{0.948282in}}%
\pgfpathlineto{\pgfqpoint{4.037698in}{0.951232in}}%
\pgfpathlineto{\pgfqpoint{4.042239in}{0.951232in}}%
\pgfpathlineto{\pgfqpoint{4.042239in}{0.948282in}}%
\pgfpathmoveto{\pgfqpoint{4.037698in}{0.951232in}}%
\pgfpathlineto{\pgfqpoint{4.037698in}{0.951232in}}%
\pgfpathlineto{\pgfqpoint{4.037698in}{0.954181in}}%
\pgfpathlineto{\pgfqpoint{4.042239in}{0.954181in}}%
\pgfpathlineto{\pgfqpoint{4.042239in}{0.951232in}}%
\pgfpathmoveto{\pgfqpoint{4.042239in}{0.951232in}}%
\pgfpathlineto{\pgfqpoint{4.042239in}{0.951232in}}%
\pgfpathlineto{\pgfqpoint{4.042239in}{0.954181in}}%
\pgfpathlineto{\pgfqpoint{4.046780in}{0.954181in}}%
\pgfpathlineto{\pgfqpoint{4.046780in}{0.951232in}}%
\pgfpathmoveto{\pgfqpoint{4.042239in}{0.954181in}}%
\pgfpathlineto{\pgfqpoint{4.042239in}{0.954181in}}%
\pgfpathlineto{\pgfqpoint{4.042239in}{0.957130in}}%
\pgfpathlineto{\pgfqpoint{4.046780in}{0.957130in}}%
\pgfpathlineto{\pgfqpoint{4.046780in}{0.954181in}}%
\pgfpathmoveto{\pgfqpoint{4.046780in}{0.954181in}}%
\pgfpathlineto{\pgfqpoint{4.046780in}{0.954181in}}%
\pgfpathlineto{\pgfqpoint{4.046780in}{0.957130in}}%
\pgfpathlineto{\pgfqpoint{4.051321in}{0.957130in}}%
\pgfpathlineto{\pgfqpoint{4.051321in}{0.954181in}}%
\pgfpathmoveto{\pgfqpoint{4.046780in}{0.957130in}}%
\pgfpathlineto{\pgfqpoint{4.046780in}{0.957130in}}%
\pgfpathlineto{\pgfqpoint{4.046780in}{0.960079in}}%
\pgfpathlineto{\pgfqpoint{4.051321in}{0.960079in}}%
\pgfpathlineto{\pgfqpoint{4.051321in}{0.957130in}}%
\pgfpathmoveto{\pgfqpoint{4.051321in}{0.957130in}}%
\pgfpathlineto{\pgfqpoint{4.051321in}{0.957130in}}%
\pgfpathlineto{\pgfqpoint{4.051321in}{0.960079in}}%
\pgfpathlineto{\pgfqpoint{4.055862in}{0.960079in}}%
\pgfpathlineto{\pgfqpoint{4.055862in}{0.957130in}}%
\pgfpathmoveto{\pgfqpoint{4.051321in}{0.960079in}}%
\pgfpathlineto{\pgfqpoint{4.051321in}{0.960079in}}%
\pgfpathlineto{\pgfqpoint{4.051321in}{0.963029in}}%
\pgfpathlineto{\pgfqpoint{4.055862in}{0.963029in}}%
\pgfpathlineto{\pgfqpoint{4.055862in}{0.960079in}}%
\pgfpathmoveto{\pgfqpoint{4.055862in}{0.960079in}}%
\pgfpathlineto{\pgfqpoint{4.055862in}{0.960079in}}%
\pgfpathlineto{\pgfqpoint{4.055862in}{0.963029in}}%
\pgfpathlineto{\pgfqpoint{4.060403in}{0.963029in}}%
\pgfpathlineto{\pgfqpoint{4.060403in}{0.960079in}}%
\pgfpathmoveto{\pgfqpoint{4.055862in}{0.963029in}}%
\pgfpathlineto{\pgfqpoint{4.055862in}{0.963029in}}%
\pgfpathlineto{\pgfqpoint{4.055862in}{0.965978in}}%
\pgfpathlineto{\pgfqpoint{4.060403in}{0.965978in}}%
\pgfpathlineto{\pgfqpoint{4.060403in}{0.963029in}}%
\pgfpathmoveto{\pgfqpoint{4.055862in}{0.965978in}}%
\pgfpathlineto{\pgfqpoint{4.055862in}{0.965978in}}%
\pgfpathlineto{\pgfqpoint{4.055862in}{0.968927in}}%
\pgfpathlineto{\pgfqpoint{4.060403in}{0.968927in}}%
\pgfpathlineto{\pgfqpoint{4.060403in}{0.965978in}}%
\pgfpathmoveto{\pgfqpoint{4.060403in}{0.965978in}}%
\pgfpathlineto{\pgfqpoint{4.060403in}{0.965978in}}%
\pgfpathlineto{\pgfqpoint{4.060403in}{0.968927in}}%
\pgfpathlineto{\pgfqpoint{4.064944in}{0.968927in}}%
\pgfpathlineto{\pgfqpoint{4.064944in}{0.965978in}}%
\pgfpathmoveto{\pgfqpoint{4.060403in}{0.968927in}}%
\pgfpathlineto{\pgfqpoint{4.060403in}{0.968927in}}%
\pgfpathlineto{\pgfqpoint{4.060403in}{0.971876in}}%
\pgfpathlineto{\pgfqpoint{4.064944in}{0.971876in}}%
\pgfpathlineto{\pgfqpoint{4.064944in}{0.968927in}}%
\pgfpathmoveto{\pgfqpoint{4.064944in}{0.968927in}}%
\pgfpathlineto{\pgfqpoint{4.064944in}{0.968927in}}%
\pgfpathlineto{\pgfqpoint{4.064944in}{0.971876in}}%
\pgfpathlineto{\pgfqpoint{4.069485in}{0.971876in}}%
\pgfpathlineto{\pgfqpoint{4.069485in}{0.968927in}}%
\pgfpathmoveto{\pgfqpoint{4.064944in}{0.971876in}}%
\pgfpathlineto{\pgfqpoint{4.064944in}{0.971876in}}%
\pgfpathlineto{\pgfqpoint{4.064944in}{0.974825in}}%
\pgfpathlineto{\pgfqpoint{4.069485in}{0.974825in}}%
\pgfpathlineto{\pgfqpoint{4.069485in}{0.971876in}}%
\pgfpathmoveto{\pgfqpoint{4.069485in}{0.971876in}}%
\pgfpathlineto{\pgfqpoint{4.069485in}{0.971876in}}%
\pgfpathlineto{\pgfqpoint{4.069485in}{0.974825in}}%
\pgfpathlineto{\pgfqpoint{4.074026in}{0.974825in}}%
\pgfpathlineto{\pgfqpoint{4.074026in}{0.971876in}}%
\pgfpathmoveto{\pgfqpoint{4.069485in}{0.974825in}}%
\pgfpathlineto{\pgfqpoint{4.069485in}{0.974825in}}%
\pgfpathlineto{\pgfqpoint{4.069485in}{0.977775in}}%
\pgfpathlineto{\pgfqpoint{4.074026in}{0.977775in}}%
\pgfpathlineto{\pgfqpoint{4.074026in}{0.974825in}}%
\pgfpathmoveto{\pgfqpoint{4.074026in}{0.974825in}}%
\pgfpathlineto{\pgfqpoint{4.074026in}{0.974825in}}%
\pgfpathlineto{\pgfqpoint{4.074026in}{0.977775in}}%
\pgfpathlineto{\pgfqpoint{4.078567in}{0.977775in}}%
\pgfpathlineto{\pgfqpoint{4.078567in}{0.974825in}}%
\pgfpathmoveto{\pgfqpoint{4.074026in}{0.977775in}}%
\pgfpathlineto{\pgfqpoint{4.074026in}{0.977775in}}%
\pgfpathlineto{\pgfqpoint{4.074026in}{0.980724in}}%
\pgfpathlineto{\pgfqpoint{4.078567in}{0.980724in}}%
\pgfpathlineto{\pgfqpoint{4.078567in}{0.977775in}}%
\pgfpathmoveto{\pgfqpoint{4.078567in}{0.977775in}}%
\pgfpathlineto{\pgfqpoint{4.078567in}{0.977775in}}%
\pgfpathlineto{\pgfqpoint{4.078567in}{0.980724in}}%
\pgfpathlineto{\pgfqpoint{4.083108in}{0.980724in}}%
\pgfpathlineto{\pgfqpoint{4.083108in}{0.977775in}}%
\pgfpathmoveto{\pgfqpoint{4.078567in}{0.980724in}}%
\pgfpathlineto{\pgfqpoint{4.078567in}{0.980724in}}%
\pgfpathlineto{\pgfqpoint{4.078567in}{0.983673in}}%
\pgfpathlineto{\pgfqpoint{4.083108in}{0.983673in}}%
\pgfpathlineto{\pgfqpoint{4.083108in}{0.980724in}}%
\pgfpathmoveto{\pgfqpoint{4.078567in}{0.983673in}}%
\pgfpathlineto{\pgfqpoint{4.078567in}{0.983673in}}%
\pgfpathlineto{\pgfqpoint{4.078567in}{0.986622in}}%
\pgfpathlineto{\pgfqpoint{4.083108in}{0.986622in}}%
\pgfpathlineto{\pgfqpoint{4.083108in}{0.983673in}}%
\pgfpathmoveto{\pgfqpoint{4.083108in}{0.983673in}}%
\pgfpathlineto{\pgfqpoint{4.083108in}{0.983673in}}%
\pgfpathlineto{\pgfqpoint{4.083108in}{0.986622in}}%
\pgfpathlineto{\pgfqpoint{4.087649in}{0.986622in}}%
\pgfpathlineto{\pgfqpoint{4.087649in}{0.983673in}}%
\pgfpathmoveto{\pgfqpoint{4.083108in}{0.986622in}}%
\pgfpathlineto{\pgfqpoint{4.083108in}{0.986622in}}%
\pgfpathlineto{\pgfqpoint{4.083108in}{0.989571in}}%
\pgfpathlineto{\pgfqpoint{4.087649in}{0.989571in}}%
\pgfpathlineto{\pgfqpoint{4.087649in}{0.986622in}}%
\pgfpathmoveto{\pgfqpoint{4.087649in}{0.986622in}}%
\pgfpathlineto{\pgfqpoint{4.087649in}{0.986622in}}%
\pgfpathlineto{\pgfqpoint{4.087649in}{0.989571in}}%
\pgfpathlineto{\pgfqpoint{4.092190in}{0.989571in}}%
\pgfpathlineto{\pgfqpoint{4.092190in}{0.986622in}}%
\pgfpathmoveto{\pgfqpoint{4.087649in}{0.989571in}}%
\pgfpathlineto{\pgfqpoint{4.087649in}{0.989571in}}%
\pgfpathlineto{\pgfqpoint{4.087649in}{0.992520in}}%
\pgfpathlineto{\pgfqpoint{4.092190in}{0.992520in}}%
\pgfpathlineto{\pgfqpoint{4.092190in}{0.989571in}}%
\pgfpathmoveto{\pgfqpoint{4.092190in}{0.989571in}}%
\pgfpathlineto{\pgfqpoint{4.092190in}{0.989571in}}%
\pgfpathlineto{\pgfqpoint{4.092190in}{0.992520in}}%
\pgfpathlineto{\pgfqpoint{4.096731in}{0.992520in}}%
\pgfpathlineto{\pgfqpoint{4.096731in}{0.989571in}}%
\pgfpathmoveto{\pgfqpoint{4.092190in}{0.992520in}}%
\pgfpathlineto{\pgfqpoint{4.092190in}{0.992520in}}%
\pgfpathlineto{\pgfqpoint{4.092190in}{0.995469in}}%
\pgfpathlineto{\pgfqpoint{4.096731in}{0.995469in}}%
\pgfpathlineto{\pgfqpoint{4.096731in}{0.992520in}}%
\pgfpathmoveto{\pgfqpoint{4.096731in}{0.992520in}}%
\pgfpathlineto{\pgfqpoint{4.096731in}{0.992520in}}%
\pgfpathlineto{\pgfqpoint{4.096731in}{0.995469in}}%
\pgfpathlineto{\pgfqpoint{4.101272in}{0.995469in}}%
\pgfpathlineto{\pgfqpoint{4.101272in}{0.992520in}}%
\pgfpathmoveto{\pgfqpoint{4.096731in}{0.995469in}}%
\pgfpathlineto{\pgfqpoint{4.096731in}{0.995469in}}%
\pgfpathlineto{\pgfqpoint{4.096731in}{0.998419in}}%
\pgfpathlineto{\pgfqpoint{4.101272in}{0.998419in}}%
\pgfpathlineto{\pgfqpoint{4.101272in}{0.995469in}}%
\pgfpathmoveto{\pgfqpoint{4.101272in}{0.995469in}}%
\pgfpathlineto{\pgfqpoint{4.101272in}{0.995469in}}%
\pgfpathlineto{\pgfqpoint{4.101272in}{0.998419in}}%
\pgfpathlineto{\pgfqpoint{4.105813in}{0.998419in}}%
\pgfpathlineto{\pgfqpoint{4.105813in}{0.995469in}}%
\pgfpathmoveto{\pgfqpoint{4.101272in}{0.998419in}}%
\pgfpathlineto{\pgfqpoint{4.101272in}{0.998419in}}%
\pgfpathlineto{\pgfqpoint{4.101272in}{1.001368in}}%
\pgfpathlineto{\pgfqpoint{4.105813in}{1.001368in}}%
\pgfpathlineto{\pgfqpoint{4.105813in}{0.998419in}}%
\pgfpathmoveto{\pgfqpoint{4.101272in}{1.001368in}}%
\pgfpathlineto{\pgfqpoint{4.101272in}{1.001368in}}%
\pgfpathlineto{\pgfqpoint{4.101272in}{1.004317in}}%
\pgfpathlineto{\pgfqpoint{4.105813in}{1.004317in}}%
\pgfpathlineto{\pgfqpoint{4.105813in}{1.001368in}}%
\pgfpathmoveto{\pgfqpoint{4.105813in}{1.001368in}}%
\pgfpathlineto{\pgfqpoint{4.105813in}{1.001368in}}%
\pgfpathlineto{\pgfqpoint{4.105813in}{1.004317in}}%
\pgfpathlineto{\pgfqpoint{4.110354in}{1.004317in}}%
\pgfpathlineto{\pgfqpoint{4.110354in}{1.001368in}}%
\pgfpathmoveto{\pgfqpoint{4.105813in}{1.004317in}}%
\pgfpathlineto{\pgfqpoint{4.105813in}{1.004317in}}%
\pgfpathlineto{\pgfqpoint{4.105813in}{1.007266in}}%
\pgfpathlineto{\pgfqpoint{4.110354in}{1.007266in}}%
\pgfpathlineto{\pgfqpoint{4.110354in}{1.004317in}}%
\pgfpathmoveto{\pgfqpoint{4.110354in}{1.004317in}}%
\pgfpathlineto{\pgfqpoint{4.110354in}{1.004317in}}%
\pgfpathlineto{\pgfqpoint{4.110354in}{1.007266in}}%
\pgfpathlineto{\pgfqpoint{4.114895in}{1.007266in}}%
\pgfpathlineto{\pgfqpoint{4.114895in}{1.004317in}}%
\pgfpathmoveto{\pgfqpoint{4.110354in}{1.007266in}}%
\pgfpathlineto{\pgfqpoint{4.110354in}{1.007266in}}%
\pgfpathlineto{\pgfqpoint{4.110354in}{1.010215in}}%
\pgfpathlineto{\pgfqpoint{4.114895in}{1.010215in}}%
\pgfpathlineto{\pgfqpoint{4.114895in}{1.007266in}}%
\pgfpathmoveto{\pgfqpoint{4.114895in}{1.007266in}}%
\pgfpathlineto{\pgfqpoint{4.114895in}{1.007266in}}%
\pgfpathlineto{\pgfqpoint{4.114895in}{1.010215in}}%
\pgfpathlineto{\pgfqpoint{4.119436in}{1.010215in}}%
\pgfpathlineto{\pgfqpoint{4.119436in}{1.007266in}}%
\pgfpathmoveto{\pgfqpoint{4.114895in}{1.010215in}}%
\pgfpathlineto{\pgfqpoint{4.114895in}{1.010215in}}%
\pgfpathlineto{\pgfqpoint{4.114895in}{1.013164in}}%
\pgfpathlineto{\pgfqpoint{4.119436in}{1.013164in}}%
\pgfpathlineto{\pgfqpoint{4.119436in}{1.010215in}}%
\pgfpathmoveto{\pgfqpoint{4.119436in}{1.010215in}}%
\pgfpathlineto{\pgfqpoint{4.119436in}{1.010215in}}%
\pgfpathlineto{\pgfqpoint{4.119436in}{1.013164in}}%
\pgfpathlineto{\pgfqpoint{4.123976in}{1.013164in}}%
\pgfpathlineto{\pgfqpoint{4.123976in}{1.010215in}}%
\pgfpathmoveto{\pgfqpoint{4.119436in}{1.013164in}}%
\pgfpathlineto{\pgfqpoint{4.119436in}{1.013164in}}%
\pgfpathlineto{\pgfqpoint{4.119436in}{1.016114in}}%
\pgfpathlineto{\pgfqpoint{4.123976in}{1.016114in}}%
\pgfpathlineto{\pgfqpoint{4.123976in}{1.013164in}}%
\pgfpathmoveto{\pgfqpoint{4.123976in}{1.013164in}}%
\pgfpathlineto{\pgfqpoint{4.123976in}{1.013164in}}%
\pgfpathlineto{\pgfqpoint{4.123976in}{1.016114in}}%
\pgfpathlineto{\pgfqpoint{4.128517in}{1.016114in}}%
\pgfpathlineto{\pgfqpoint{4.128517in}{1.013164in}}%
\pgfpathmoveto{\pgfqpoint{4.123976in}{1.016114in}}%
\pgfpathlineto{\pgfqpoint{4.123976in}{1.016114in}}%
\pgfpathlineto{\pgfqpoint{4.123976in}{1.019063in}}%
\pgfpathlineto{\pgfqpoint{4.128517in}{1.019063in}}%
\pgfpathlineto{\pgfqpoint{4.128517in}{1.016114in}}%
\pgfpathmoveto{\pgfqpoint{4.123976in}{1.019063in}}%
\pgfpathlineto{\pgfqpoint{4.123976in}{1.019063in}}%
\pgfpathlineto{\pgfqpoint{4.123976in}{1.022012in}}%
\pgfpathlineto{\pgfqpoint{4.128517in}{1.022012in}}%
\pgfpathlineto{\pgfqpoint{4.128517in}{1.019063in}}%
\pgfpathmoveto{\pgfqpoint{4.128517in}{1.019063in}}%
\pgfpathlineto{\pgfqpoint{4.128517in}{1.019063in}}%
\pgfpathlineto{\pgfqpoint{4.128517in}{1.022012in}}%
\pgfpathlineto{\pgfqpoint{4.133058in}{1.022012in}}%
\pgfpathlineto{\pgfqpoint{4.133058in}{1.019063in}}%
\pgfpathmoveto{\pgfqpoint{4.128517in}{1.022012in}}%
\pgfpathlineto{\pgfqpoint{4.128517in}{1.022012in}}%
\pgfpathlineto{\pgfqpoint{4.128517in}{1.024961in}}%
\pgfpathlineto{\pgfqpoint{4.133058in}{1.024961in}}%
\pgfpathlineto{\pgfqpoint{4.133058in}{1.022012in}}%
\pgfpathmoveto{\pgfqpoint{4.133058in}{1.022012in}}%
\pgfpathlineto{\pgfqpoint{4.133058in}{1.022012in}}%
\pgfpathlineto{\pgfqpoint{4.133058in}{1.024961in}}%
\pgfpathlineto{\pgfqpoint{4.137599in}{1.024961in}}%
\pgfpathlineto{\pgfqpoint{4.137599in}{1.022012in}}%
\pgfpathmoveto{\pgfqpoint{4.133058in}{1.024961in}}%
\pgfpathlineto{\pgfqpoint{4.133058in}{1.024961in}}%
\pgfpathlineto{\pgfqpoint{4.133058in}{1.027910in}}%
\pgfpathlineto{\pgfqpoint{4.137599in}{1.027910in}}%
\pgfpathlineto{\pgfqpoint{4.137599in}{1.024961in}}%
\pgfpathmoveto{\pgfqpoint{4.137599in}{1.024961in}}%
\pgfpathlineto{\pgfqpoint{4.137599in}{1.024961in}}%
\pgfpathlineto{\pgfqpoint{4.137599in}{1.027910in}}%
\pgfpathlineto{\pgfqpoint{4.142140in}{1.027910in}}%
\pgfpathlineto{\pgfqpoint{4.142140in}{1.024961in}}%
\pgfpathmoveto{\pgfqpoint{4.137599in}{1.027910in}}%
\pgfpathlineto{\pgfqpoint{4.137599in}{1.027910in}}%
\pgfpathlineto{\pgfqpoint{4.137599in}{1.030859in}}%
\pgfpathlineto{\pgfqpoint{4.142140in}{1.030859in}}%
\pgfpathlineto{\pgfqpoint{4.142140in}{1.027910in}}%
\pgfpathmoveto{\pgfqpoint{4.142140in}{1.027910in}}%
\pgfpathlineto{\pgfqpoint{4.142140in}{1.027910in}}%
\pgfpathlineto{\pgfqpoint{4.142140in}{1.030859in}}%
\pgfpathlineto{\pgfqpoint{4.146681in}{1.030859in}}%
\pgfpathlineto{\pgfqpoint{4.146681in}{1.027910in}}%
\pgfpathmoveto{\pgfqpoint{4.142140in}{1.030859in}}%
\pgfpathlineto{\pgfqpoint{4.142140in}{1.030859in}}%
\pgfpathlineto{\pgfqpoint{4.142140in}{1.033808in}}%
\pgfpathlineto{\pgfqpoint{4.146681in}{1.033808in}}%
\pgfpathlineto{\pgfqpoint{4.146681in}{1.030859in}}%
\pgfpathmoveto{\pgfqpoint{4.146681in}{1.030859in}}%
\pgfpathlineto{\pgfqpoint{4.146681in}{1.030859in}}%
\pgfpathlineto{\pgfqpoint{4.146681in}{1.033808in}}%
\pgfpathlineto{\pgfqpoint{4.151222in}{1.033808in}}%
\pgfpathlineto{\pgfqpoint{4.151222in}{1.030859in}}%
\pgfpathmoveto{\pgfqpoint{4.146681in}{1.033808in}}%
\pgfpathlineto{\pgfqpoint{4.146681in}{1.033808in}}%
\pgfpathlineto{\pgfqpoint{4.146681in}{1.036758in}}%
\pgfpathlineto{\pgfqpoint{4.151222in}{1.036758in}}%
\pgfpathlineto{\pgfqpoint{4.151222in}{1.033808in}}%
\pgfpathmoveto{\pgfqpoint{4.146681in}{1.036758in}}%
\pgfpathlineto{\pgfqpoint{4.146681in}{1.036758in}}%
\pgfpathlineto{\pgfqpoint{4.146681in}{1.039707in}}%
\pgfpathlineto{\pgfqpoint{4.151222in}{1.039707in}}%
\pgfpathlineto{\pgfqpoint{4.151222in}{1.036758in}}%
\pgfpathmoveto{\pgfqpoint{4.151222in}{1.036758in}}%
\pgfpathlineto{\pgfqpoint{4.151222in}{1.036758in}}%
\pgfpathlineto{\pgfqpoint{4.151222in}{1.039707in}}%
\pgfpathlineto{\pgfqpoint{4.155763in}{1.039707in}}%
\pgfpathlineto{\pgfqpoint{4.155763in}{1.036758in}}%
\pgfpathmoveto{\pgfqpoint{4.151222in}{1.039707in}}%
\pgfpathlineto{\pgfqpoint{4.151222in}{1.039707in}}%
\pgfpathlineto{\pgfqpoint{4.151222in}{1.042656in}}%
\pgfpathlineto{\pgfqpoint{4.155763in}{1.042656in}}%
\pgfpathlineto{\pgfqpoint{4.155763in}{1.039707in}}%
\pgfpathmoveto{\pgfqpoint{4.155763in}{1.039707in}}%
\pgfpathlineto{\pgfqpoint{4.155763in}{1.039707in}}%
\pgfpathlineto{\pgfqpoint{4.155763in}{1.042656in}}%
\pgfpathlineto{\pgfqpoint{4.160304in}{1.042656in}}%
\pgfpathlineto{\pgfqpoint{4.160304in}{1.039707in}}%
\pgfpathmoveto{\pgfqpoint{4.155763in}{1.042656in}}%
\pgfpathlineto{\pgfqpoint{4.155763in}{1.042656in}}%
\pgfpathlineto{\pgfqpoint{4.155763in}{1.045605in}}%
\pgfpathlineto{\pgfqpoint{4.160304in}{1.045605in}}%
\pgfpathlineto{\pgfqpoint{4.160304in}{1.042656in}}%
\pgfpathmoveto{\pgfqpoint{4.160304in}{1.042656in}}%
\pgfpathlineto{\pgfqpoint{4.160304in}{1.042656in}}%
\pgfpathlineto{\pgfqpoint{4.160304in}{1.045605in}}%
\pgfpathlineto{\pgfqpoint{4.164845in}{1.045605in}}%
\pgfpathlineto{\pgfqpoint{4.164845in}{1.042656in}}%
\pgfpathmoveto{\pgfqpoint{4.160304in}{1.045605in}}%
\pgfpathlineto{\pgfqpoint{4.160304in}{1.045605in}}%
\pgfpathlineto{\pgfqpoint{4.160304in}{1.048554in}}%
\pgfpathlineto{\pgfqpoint{4.164845in}{1.048554in}}%
\pgfpathlineto{\pgfqpoint{4.164845in}{1.045605in}}%
\pgfpathmoveto{\pgfqpoint{4.164845in}{1.045605in}}%
\pgfpathlineto{\pgfqpoint{4.164845in}{1.045605in}}%
\pgfpathlineto{\pgfqpoint{4.164845in}{1.048554in}}%
\pgfpathlineto{\pgfqpoint{4.169386in}{1.048554in}}%
\pgfpathlineto{\pgfqpoint{4.169386in}{1.045605in}}%
\pgfpathmoveto{\pgfqpoint{4.164845in}{1.048554in}}%
\pgfpathlineto{\pgfqpoint{4.164845in}{1.048554in}}%
\pgfpathlineto{\pgfqpoint{4.164845in}{1.051503in}}%
\pgfpathlineto{\pgfqpoint{4.169386in}{1.051503in}}%
\pgfpathlineto{\pgfqpoint{4.169386in}{1.048554in}}%
\pgfpathmoveto{\pgfqpoint{4.169386in}{1.048554in}}%
\pgfpathlineto{\pgfqpoint{4.169386in}{1.048554in}}%
\pgfpathlineto{\pgfqpoint{4.169386in}{1.051503in}}%
\pgfpathlineto{\pgfqpoint{4.173927in}{1.051503in}}%
\pgfpathlineto{\pgfqpoint{4.173927in}{1.048554in}}%
\pgfpathmoveto{\pgfqpoint{4.169386in}{1.051503in}}%
\pgfpathlineto{\pgfqpoint{4.169386in}{1.051503in}}%
\pgfpathlineto{\pgfqpoint{4.169386in}{1.054452in}}%
\pgfpathlineto{\pgfqpoint{4.173927in}{1.054452in}}%
\pgfpathlineto{\pgfqpoint{4.173927in}{1.051503in}}%
\pgfpathmoveto{\pgfqpoint{4.169386in}{1.054452in}}%
\pgfpathlineto{\pgfqpoint{4.169386in}{1.054452in}}%
\pgfpathlineto{\pgfqpoint{4.169386in}{1.057402in}}%
\pgfpathlineto{\pgfqpoint{4.173927in}{1.057402in}}%
\pgfpathlineto{\pgfqpoint{4.173927in}{1.054452in}}%
\pgfpathmoveto{\pgfqpoint{4.173927in}{1.054452in}}%
\pgfpathlineto{\pgfqpoint{4.173927in}{1.054452in}}%
\pgfpathlineto{\pgfqpoint{4.173927in}{1.057402in}}%
\pgfpathlineto{\pgfqpoint{4.178468in}{1.057402in}}%
\pgfpathlineto{\pgfqpoint{4.178468in}{1.054452in}}%
\pgfpathmoveto{\pgfqpoint{4.173927in}{1.057402in}}%
\pgfpathlineto{\pgfqpoint{4.173927in}{1.057402in}}%
\pgfpathlineto{\pgfqpoint{4.173927in}{1.060351in}}%
\pgfpathlineto{\pgfqpoint{4.178468in}{1.060351in}}%
\pgfpathlineto{\pgfqpoint{4.178468in}{1.057402in}}%
\pgfpathmoveto{\pgfqpoint{4.178468in}{1.057402in}}%
\pgfpathlineto{\pgfqpoint{4.178468in}{1.057402in}}%
\pgfpathlineto{\pgfqpoint{4.178468in}{1.060351in}}%
\pgfpathlineto{\pgfqpoint{4.183009in}{1.060351in}}%
\pgfpathlineto{\pgfqpoint{4.183009in}{1.057402in}}%
\pgfpathmoveto{\pgfqpoint{4.178468in}{1.060351in}}%
\pgfpathlineto{\pgfqpoint{4.178468in}{1.060351in}}%
\pgfpathlineto{\pgfqpoint{4.178468in}{1.063300in}}%
\pgfpathlineto{\pgfqpoint{4.183009in}{1.063300in}}%
\pgfpathlineto{\pgfqpoint{4.183009in}{1.060351in}}%
\pgfpathmoveto{\pgfqpoint{4.183009in}{1.060351in}}%
\pgfpathlineto{\pgfqpoint{4.183009in}{1.060351in}}%
\pgfpathlineto{\pgfqpoint{4.183009in}{1.063300in}}%
\pgfpathlineto{\pgfqpoint{4.187550in}{1.063300in}}%
\pgfpathlineto{\pgfqpoint{4.187550in}{1.060351in}}%
\pgfpathmoveto{\pgfqpoint{4.183009in}{1.063300in}}%
\pgfpathlineto{\pgfqpoint{4.183009in}{1.063300in}}%
\pgfpathlineto{\pgfqpoint{4.183009in}{1.066249in}}%
\pgfpathlineto{\pgfqpoint{4.187550in}{1.066249in}}%
\pgfpathlineto{\pgfqpoint{4.187550in}{1.063300in}}%
\pgfpathmoveto{\pgfqpoint{4.187550in}{1.063300in}}%
\pgfpathlineto{\pgfqpoint{4.187550in}{1.063300in}}%
\pgfpathlineto{\pgfqpoint{4.187550in}{1.066249in}}%
\pgfpathlineto{\pgfqpoint{4.192091in}{1.066249in}}%
\pgfpathlineto{\pgfqpoint{4.192091in}{1.063300in}}%
\pgfpathmoveto{\pgfqpoint{4.187550in}{1.066249in}}%
\pgfpathlineto{\pgfqpoint{4.187550in}{1.066249in}}%
\pgfpathlineto{\pgfqpoint{4.187550in}{1.069198in}}%
\pgfpathlineto{\pgfqpoint{4.192091in}{1.069198in}}%
\pgfpathlineto{\pgfqpoint{4.192091in}{1.066249in}}%
\pgfpathmoveto{\pgfqpoint{4.192091in}{1.066249in}}%
\pgfpathlineto{\pgfqpoint{4.192091in}{1.066249in}}%
\pgfpathlineto{\pgfqpoint{4.192091in}{1.069198in}}%
\pgfpathlineto{\pgfqpoint{4.196631in}{1.069198in}}%
\pgfpathlineto{\pgfqpoint{4.196631in}{1.066249in}}%
\pgfpathmoveto{\pgfqpoint{4.192091in}{1.069198in}}%
\pgfpathlineto{\pgfqpoint{4.192091in}{1.069198in}}%
\pgfpathlineto{\pgfqpoint{4.192091in}{1.072147in}}%
\pgfpathlineto{\pgfqpoint{4.196631in}{1.072147in}}%
\pgfpathlineto{\pgfqpoint{4.196631in}{1.069198in}}%
\pgfpathmoveto{\pgfqpoint{4.192091in}{1.072147in}}%
\pgfpathlineto{\pgfqpoint{4.192091in}{1.072147in}}%
\pgfpathlineto{\pgfqpoint{4.192091in}{1.075097in}}%
\pgfpathlineto{\pgfqpoint{4.196631in}{1.075097in}}%
\pgfpathlineto{\pgfqpoint{4.196631in}{1.072147in}}%
\pgfpathmoveto{\pgfqpoint{4.196631in}{1.072147in}}%
\pgfpathlineto{\pgfqpoint{4.196631in}{1.072147in}}%
\pgfpathlineto{\pgfqpoint{4.196631in}{1.075097in}}%
\pgfpathlineto{\pgfqpoint{4.201172in}{1.075097in}}%
\pgfpathlineto{\pgfqpoint{4.201172in}{1.072147in}}%
\pgfpathmoveto{\pgfqpoint{4.196631in}{1.075097in}}%
\pgfpathlineto{\pgfqpoint{4.196631in}{1.075097in}}%
\pgfpathlineto{\pgfqpoint{4.196631in}{1.078046in}}%
\pgfpathlineto{\pgfqpoint{4.201172in}{1.078046in}}%
\pgfpathlineto{\pgfqpoint{4.201172in}{1.075097in}}%
\pgfpathmoveto{\pgfqpoint{4.201172in}{1.075097in}}%
\pgfpathlineto{\pgfqpoint{4.201172in}{1.075097in}}%
\pgfpathlineto{\pgfqpoint{4.201172in}{1.078046in}}%
\pgfpathlineto{\pgfqpoint{4.205713in}{1.078046in}}%
\pgfpathlineto{\pgfqpoint{4.205713in}{1.075097in}}%
\pgfpathmoveto{\pgfqpoint{4.201172in}{1.078046in}}%
\pgfpathlineto{\pgfqpoint{4.201172in}{1.078046in}}%
\pgfpathlineto{\pgfqpoint{4.201172in}{1.080995in}}%
\pgfpathlineto{\pgfqpoint{4.205713in}{1.080995in}}%
\pgfpathlineto{\pgfqpoint{4.205713in}{1.078046in}}%
\pgfpathmoveto{\pgfqpoint{4.205713in}{1.078046in}}%
\pgfpathlineto{\pgfqpoint{4.205713in}{1.078046in}}%
\pgfpathlineto{\pgfqpoint{4.205713in}{1.080995in}}%
\pgfpathlineto{\pgfqpoint{4.210254in}{1.080995in}}%
\pgfpathlineto{\pgfqpoint{4.210254in}{1.078046in}}%
\pgfpathmoveto{\pgfqpoint{4.205713in}{1.080995in}}%
\pgfpathlineto{\pgfqpoint{4.205713in}{1.080995in}}%
\pgfpathlineto{\pgfqpoint{4.205713in}{1.083944in}}%
\pgfpathlineto{\pgfqpoint{4.210254in}{1.083944in}}%
\pgfpathlineto{\pgfqpoint{4.210254in}{1.080995in}}%
\pgfpathmoveto{\pgfqpoint{4.210254in}{1.080995in}}%
\pgfpathlineto{\pgfqpoint{4.210254in}{1.080995in}}%
\pgfpathlineto{\pgfqpoint{4.210254in}{1.083944in}}%
\pgfpathlineto{\pgfqpoint{4.214795in}{1.083944in}}%
\pgfpathlineto{\pgfqpoint{4.214795in}{1.080995in}}%
\pgfpathmoveto{\pgfqpoint{4.210254in}{1.083944in}}%
\pgfpathlineto{\pgfqpoint{4.210254in}{1.083944in}}%
\pgfpathlineto{\pgfqpoint{4.210254in}{1.086894in}}%
\pgfpathlineto{\pgfqpoint{4.214795in}{1.086894in}}%
\pgfpathlineto{\pgfqpoint{4.214795in}{1.083944in}}%
\pgfpathmoveto{\pgfqpoint{4.214795in}{1.083944in}}%
\pgfpathlineto{\pgfqpoint{4.214795in}{1.083944in}}%
\pgfpathlineto{\pgfqpoint{4.214795in}{1.086894in}}%
\pgfpathlineto{\pgfqpoint{4.219336in}{1.086894in}}%
\pgfpathlineto{\pgfqpoint{4.219336in}{1.083944in}}%
\pgfpathmoveto{\pgfqpoint{4.214795in}{1.086894in}}%
\pgfpathlineto{\pgfqpoint{4.214795in}{1.086894in}}%
\pgfpathlineto{\pgfqpoint{4.214795in}{1.089843in}}%
\pgfpathlineto{\pgfqpoint{4.219336in}{1.089843in}}%
\pgfpathlineto{\pgfqpoint{4.219336in}{1.086894in}}%
\pgfpathmoveto{\pgfqpoint{4.214795in}{1.089843in}}%
\pgfpathlineto{\pgfqpoint{4.214795in}{1.089843in}}%
\pgfpathlineto{\pgfqpoint{4.214795in}{1.092792in}}%
\pgfpathlineto{\pgfqpoint{4.219336in}{1.092792in}}%
\pgfpathlineto{\pgfqpoint{4.219336in}{1.089843in}}%
\pgfpathmoveto{\pgfqpoint{4.219336in}{1.089843in}}%
\pgfpathlineto{\pgfqpoint{4.219336in}{1.089843in}}%
\pgfpathlineto{\pgfqpoint{4.219336in}{1.092792in}}%
\pgfpathlineto{\pgfqpoint{4.223877in}{1.092792in}}%
\pgfpathlineto{\pgfqpoint{4.223877in}{1.089843in}}%
\pgfpathmoveto{\pgfqpoint{4.219336in}{1.092792in}}%
\pgfpathlineto{\pgfqpoint{4.219336in}{1.092792in}}%
\pgfpathlineto{\pgfqpoint{4.219336in}{1.095741in}}%
\pgfpathlineto{\pgfqpoint{4.223877in}{1.095741in}}%
\pgfpathlineto{\pgfqpoint{4.223877in}{1.092792in}}%
\pgfpathmoveto{\pgfqpoint{4.223877in}{1.092792in}}%
\pgfpathlineto{\pgfqpoint{4.223877in}{1.092792in}}%
\pgfpathlineto{\pgfqpoint{4.223877in}{1.095741in}}%
\pgfpathlineto{\pgfqpoint{4.228418in}{1.095741in}}%
\pgfpathlineto{\pgfqpoint{4.228418in}{1.092792in}}%
\pgfpathmoveto{\pgfqpoint{4.223877in}{1.095741in}}%
\pgfpathlineto{\pgfqpoint{4.223877in}{1.095741in}}%
\pgfpathlineto{\pgfqpoint{4.223877in}{1.098690in}}%
\pgfpathlineto{\pgfqpoint{4.228418in}{1.098690in}}%
\pgfpathlineto{\pgfqpoint{4.228418in}{1.095741in}}%
\pgfpathmoveto{\pgfqpoint{4.228418in}{1.095741in}}%
\pgfpathlineto{\pgfqpoint{4.228418in}{1.095741in}}%
\pgfpathlineto{\pgfqpoint{4.228418in}{1.098690in}}%
\pgfpathlineto{\pgfqpoint{4.232959in}{1.098690in}}%
\pgfpathlineto{\pgfqpoint{4.232959in}{1.095741in}}%
\pgfpathmoveto{\pgfqpoint{4.228418in}{1.098690in}}%
\pgfpathlineto{\pgfqpoint{4.228418in}{1.098690in}}%
\pgfpathlineto{\pgfqpoint{4.228418in}{1.101640in}}%
\pgfpathlineto{\pgfqpoint{4.232959in}{1.101640in}}%
\pgfpathlineto{\pgfqpoint{4.232959in}{1.098690in}}%
\pgfpathmoveto{\pgfqpoint{4.232959in}{1.098690in}}%
\pgfpathlineto{\pgfqpoint{4.232959in}{1.098690in}}%
\pgfpathlineto{\pgfqpoint{4.232959in}{1.101640in}}%
\pgfpathlineto{\pgfqpoint{4.237500in}{1.101640in}}%
\pgfpathlineto{\pgfqpoint{4.237500in}{1.098690in}}%
\pgfpathmoveto{\pgfqpoint{4.232959in}{1.101640in}}%
\pgfpathlineto{\pgfqpoint{4.232959in}{1.101640in}}%
\pgfpathlineto{\pgfqpoint{4.232959in}{1.104589in}}%
\pgfpathlineto{\pgfqpoint{4.237500in}{1.104589in}}%
\pgfpathlineto{\pgfqpoint{4.237500in}{1.101640in}}%
\pgfpathmoveto{\pgfqpoint{4.237500in}{1.101640in}}%
\pgfpathlineto{\pgfqpoint{4.237500in}{1.101640in}}%
\pgfpathlineto{\pgfqpoint{4.237500in}{1.104589in}}%
\pgfpathlineto{\pgfqpoint{4.242041in}{1.104589in}}%
\pgfpathlineto{\pgfqpoint{4.242041in}{1.101640in}}%
\pgfpathmoveto{\pgfqpoint{4.237500in}{1.104589in}}%
\pgfpathlineto{\pgfqpoint{4.237500in}{1.104589in}}%
\pgfpathlineto{\pgfqpoint{4.237500in}{1.107538in}}%
\pgfpathlineto{\pgfqpoint{4.242041in}{1.107538in}}%
\pgfpathlineto{\pgfqpoint{4.242041in}{1.104589in}}%
\pgfpathmoveto{\pgfqpoint{4.237500in}{1.107538in}}%
\pgfpathlineto{\pgfqpoint{4.237500in}{1.107538in}}%
\pgfpathlineto{\pgfqpoint{4.237500in}{1.110487in}}%
\pgfpathlineto{\pgfqpoint{4.242041in}{1.110487in}}%
\pgfpathlineto{\pgfqpoint{4.242041in}{1.107538in}}%
\pgfpathmoveto{\pgfqpoint{4.242041in}{1.107538in}}%
\pgfpathlineto{\pgfqpoint{4.242041in}{1.107538in}}%
\pgfpathlineto{\pgfqpoint{4.242041in}{1.110487in}}%
\pgfpathlineto{\pgfqpoint{4.246582in}{1.110487in}}%
\pgfpathlineto{\pgfqpoint{4.246582in}{1.107538in}}%
\pgfpathmoveto{\pgfqpoint{4.242041in}{1.110487in}}%
\pgfpathlineto{\pgfqpoint{4.242041in}{1.110487in}}%
\pgfpathlineto{\pgfqpoint{4.242041in}{1.113437in}}%
\pgfpathlineto{\pgfqpoint{4.246582in}{1.113437in}}%
\pgfpathlineto{\pgfqpoint{4.246582in}{1.110487in}}%
\pgfpathmoveto{\pgfqpoint{4.246582in}{1.110487in}}%
\pgfpathlineto{\pgfqpoint{4.246582in}{1.110487in}}%
\pgfpathlineto{\pgfqpoint{4.246582in}{1.113437in}}%
\pgfpathlineto{\pgfqpoint{4.251123in}{1.113437in}}%
\pgfpathlineto{\pgfqpoint{4.251123in}{1.110487in}}%
\pgfpathmoveto{\pgfqpoint{4.246582in}{1.113437in}}%
\pgfpathlineto{\pgfqpoint{4.246582in}{1.113437in}}%
\pgfpathlineto{\pgfqpoint{4.246582in}{1.116386in}}%
\pgfpathlineto{\pgfqpoint{4.251123in}{1.116386in}}%
\pgfpathlineto{\pgfqpoint{4.251123in}{1.113437in}}%
\pgfpathmoveto{\pgfqpoint{4.251123in}{1.113437in}}%
\pgfpathlineto{\pgfqpoint{4.251123in}{1.113437in}}%
\pgfpathlineto{\pgfqpoint{4.251123in}{1.116386in}}%
\pgfpathlineto{\pgfqpoint{4.255664in}{1.116386in}}%
\pgfpathlineto{\pgfqpoint{4.255664in}{1.113437in}}%
\pgfpathmoveto{\pgfqpoint{4.251123in}{1.116386in}}%
\pgfpathlineto{\pgfqpoint{4.251123in}{1.116386in}}%
\pgfpathlineto{\pgfqpoint{4.251123in}{1.119335in}}%
\pgfpathlineto{\pgfqpoint{4.255664in}{1.119335in}}%
\pgfpathlineto{\pgfqpoint{4.255664in}{1.116386in}}%
\pgfpathmoveto{\pgfqpoint{4.255664in}{1.116386in}}%
\pgfpathlineto{\pgfqpoint{4.255664in}{1.116386in}}%
\pgfpathlineto{\pgfqpoint{4.255664in}{1.119335in}}%
\pgfpathlineto{\pgfqpoint{4.260204in}{1.119335in}}%
\pgfpathlineto{\pgfqpoint{4.260204in}{1.116386in}}%
\pgfpathmoveto{\pgfqpoint{4.255664in}{1.119335in}}%
\pgfpathlineto{\pgfqpoint{4.255664in}{1.119335in}}%
\pgfpathlineto{\pgfqpoint{4.255664in}{1.122284in}}%
\pgfpathlineto{\pgfqpoint{4.260204in}{1.122284in}}%
\pgfpathlineto{\pgfqpoint{4.260204in}{1.119335in}}%
\pgfpathmoveto{\pgfqpoint{4.260204in}{1.119335in}}%
\pgfpathlineto{\pgfqpoint{4.260204in}{1.119335in}}%
\pgfpathlineto{\pgfqpoint{4.260204in}{1.122284in}}%
\pgfpathlineto{\pgfqpoint{4.264745in}{1.122284in}}%
\pgfpathlineto{\pgfqpoint{4.264745in}{1.119335in}}%
\pgfpathmoveto{\pgfqpoint{4.260204in}{1.122284in}}%
\pgfpathlineto{\pgfqpoint{4.260204in}{1.122284in}}%
\pgfpathlineto{\pgfqpoint{4.260204in}{1.125233in}}%
\pgfpathlineto{\pgfqpoint{4.264745in}{1.125233in}}%
\pgfpathlineto{\pgfqpoint{4.264745in}{1.122284in}}%
\pgfpathmoveto{\pgfqpoint{4.260204in}{1.125233in}}%
\pgfpathlineto{\pgfqpoint{4.260204in}{1.125233in}}%
\pgfpathlineto{\pgfqpoint{4.260204in}{1.128183in}}%
\pgfpathlineto{\pgfqpoint{4.264745in}{1.128183in}}%
\pgfpathlineto{\pgfqpoint{4.264745in}{1.125233in}}%
\pgfpathmoveto{\pgfqpoint{4.264745in}{1.125233in}}%
\pgfpathlineto{\pgfqpoint{4.264745in}{1.125233in}}%
\pgfpathlineto{\pgfqpoint{4.264745in}{1.128183in}}%
\pgfpathlineto{\pgfqpoint{4.269286in}{1.128183in}}%
\pgfpathlineto{\pgfqpoint{4.269286in}{1.125233in}}%
\pgfpathmoveto{\pgfqpoint{4.264745in}{1.128183in}}%
\pgfpathlineto{\pgfqpoint{4.264745in}{1.128183in}}%
\pgfpathlineto{\pgfqpoint{4.264745in}{1.131132in}}%
\pgfpathlineto{\pgfqpoint{4.269286in}{1.131132in}}%
\pgfpathlineto{\pgfqpoint{4.269286in}{1.128183in}}%
\pgfpathmoveto{\pgfqpoint{4.269286in}{1.128183in}}%
\pgfpathlineto{\pgfqpoint{4.269286in}{1.128183in}}%
\pgfpathlineto{\pgfqpoint{4.269286in}{1.131132in}}%
\pgfpathlineto{\pgfqpoint{4.273827in}{1.131132in}}%
\pgfpathlineto{\pgfqpoint{4.273827in}{1.128183in}}%
\pgfpathmoveto{\pgfqpoint{4.269286in}{1.131132in}}%
\pgfpathlineto{\pgfqpoint{4.269286in}{1.131132in}}%
\pgfpathlineto{\pgfqpoint{4.269286in}{1.134081in}}%
\pgfpathlineto{\pgfqpoint{4.273827in}{1.134081in}}%
\pgfpathlineto{\pgfqpoint{4.273827in}{1.131132in}}%
\pgfpathmoveto{\pgfqpoint{4.273827in}{1.131132in}}%
\pgfpathlineto{\pgfqpoint{4.273827in}{1.131132in}}%
\pgfpathlineto{\pgfqpoint{4.273827in}{1.134081in}}%
\pgfpathlineto{\pgfqpoint{4.278368in}{1.134081in}}%
\pgfpathlineto{\pgfqpoint{4.278368in}{1.131132in}}%
\pgfpathmoveto{\pgfqpoint{4.273827in}{1.134081in}}%
\pgfpathlineto{\pgfqpoint{4.273827in}{1.134081in}}%
\pgfpathlineto{\pgfqpoint{4.273827in}{1.137030in}}%
\pgfpathlineto{\pgfqpoint{4.278368in}{1.137030in}}%
\pgfpathlineto{\pgfqpoint{4.278368in}{1.134081in}}%
\pgfpathmoveto{\pgfqpoint{4.278368in}{1.134081in}}%
\pgfpathlineto{\pgfqpoint{4.278368in}{1.134081in}}%
\pgfpathlineto{\pgfqpoint{4.278368in}{1.137030in}}%
\pgfpathlineto{\pgfqpoint{4.282909in}{1.137030in}}%
\pgfpathlineto{\pgfqpoint{4.282909in}{1.134081in}}%
\pgfpathmoveto{\pgfqpoint{4.278368in}{1.137030in}}%
\pgfpathlineto{\pgfqpoint{4.278368in}{1.137030in}}%
\pgfpathlineto{\pgfqpoint{4.278368in}{1.139980in}}%
\pgfpathlineto{\pgfqpoint{4.282909in}{1.139980in}}%
\pgfpathlineto{\pgfqpoint{4.282909in}{1.137030in}}%
\pgfpathmoveto{\pgfqpoint{4.282909in}{1.137030in}}%
\pgfpathlineto{\pgfqpoint{4.282909in}{1.137030in}}%
\pgfpathlineto{\pgfqpoint{4.282909in}{1.139980in}}%
\pgfpathlineto{\pgfqpoint{4.287450in}{1.139980in}}%
\pgfpathlineto{\pgfqpoint{4.287450in}{1.137030in}}%
\pgfpathmoveto{\pgfqpoint{4.282909in}{1.139980in}}%
\pgfpathlineto{\pgfqpoint{4.282909in}{1.139980in}}%
\pgfpathlineto{\pgfqpoint{4.282909in}{1.142929in}}%
\pgfpathlineto{\pgfqpoint{4.287450in}{1.142929in}}%
\pgfpathlineto{\pgfqpoint{4.287450in}{1.139980in}}%
\pgfpathmoveto{\pgfqpoint{4.287450in}{1.139980in}}%
\pgfpathlineto{\pgfqpoint{4.287450in}{1.139980in}}%
\pgfpathlineto{\pgfqpoint{4.287450in}{1.142929in}}%
\pgfpathlineto{\pgfqpoint{4.291991in}{1.142929in}}%
\pgfpathlineto{\pgfqpoint{4.291991in}{1.139980in}}%
\pgfpathmoveto{\pgfqpoint{4.287450in}{1.142929in}}%
\pgfpathlineto{\pgfqpoint{4.287450in}{1.142929in}}%
\pgfpathlineto{\pgfqpoint{4.287450in}{1.145878in}}%
\pgfpathlineto{\pgfqpoint{4.291991in}{1.145878in}}%
\pgfpathlineto{\pgfqpoint{4.291991in}{1.142929in}}%
\pgfpathmoveto{\pgfqpoint{4.287450in}{1.145878in}}%
\pgfpathlineto{\pgfqpoint{4.287450in}{1.145878in}}%
\pgfpathlineto{\pgfqpoint{4.287450in}{1.148827in}}%
\pgfpathlineto{\pgfqpoint{4.291991in}{1.148827in}}%
\pgfpathlineto{\pgfqpoint{4.291991in}{1.145878in}}%
\pgfpathmoveto{\pgfqpoint{4.291991in}{1.145878in}}%
\pgfpathlineto{\pgfqpoint{4.291991in}{1.145878in}}%
\pgfpathlineto{\pgfqpoint{4.291991in}{1.148827in}}%
\pgfpathlineto{\pgfqpoint{4.296532in}{1.148827in}}%
\pgfpathlineto{\pgfqpoint{4.296532in}{1.145878in}}%
\pgfpathmoveto{\pgfqpoint{4.291991in}{1.148827in}}%
\pgfpathlineto{\pgfqpoint{4.291991in}{1.148827in}}%
\pgfpathlineto{\pgfqpoint{4.291991in}{1.151776in}}%
\pgfpathlineto{\pgfqpoint{4.296532in}{1.151776in}}%
\pgfpathlineto{\pgfqpoint{4.296532in}{1.148827in}}%
\pgfpathmoveto{\pgfqpoint{4.296532in}{1.148827in}}%
\pgfpathlineto{\pgfqpoint{4.296532in}{1.148827in}}%
\pgfpathlineto{\pgfqpoint{4.296532in}{1.151776in}}%
\pgfpathlineto{\pgfqpoint{4.301073in}{1.151776in}}%
\pgfpathlineto{\pgfqpoint{4.301073in}{1.148827in}}%
\pgfpathmoveto{\pgfqpoint{4.296532in}{1.151776in}}%
\pgfpathlineto{\pgfqpoint{4.296532in}{1.151776in}}%
\pgfpathlineto{\pgfqpoint{4.296532in}{1.154726in}}%
\pgfpathlineto{\pgfqpoint{4.301073in}{1.154726in}}%
\pgfpathlineto{\pgfqpoint{4.301073in}{1.151776in}}%
\pgfpathmoveto{\pgfqpoint{4.301073in}{1.151776in}}%
\pgfpathlineto{\pgfqpoint{4.301073in}{1.151776in}}%
\pgfpathlineto{\pgfqpoint{4.301073in}{1.154726in}}%
\pgfpathlineto{\pgfqpoint{4.305614in}{1.154726in}}%
\pgfpathlineto{\pgfqpoint{4.305614in}{1.151776in}}%
\pgfpathmoveto{\pgfqpoint{4.301073in}{1.154726in}}%
\pgfpathlineto{\pgfqpoint{4.301073in}{1.154726in}}%
\pgfpathlineto{\pgfqpoint{4.301073in}{1.157675in}}%
\pgfpathlineto{\pgfqpoint{4.305614in}{1.157675in}}%
\pgfpathlineto{\pgfqpoint{4.305614in}{1.154726in}}%
\pgfpathmoveto{\pgfqpoint{4.305614in}{1.154726in}}%
\pgfpathlineto{\pgfqpoint{4.305614in}{1.154726in}}%
\pgfpathlineto{\pgfqpoint{4.305614in}{1.157675in}}%
\pgfpathlineto{\pgfqpoint{4.310155in}{1.157675in}}%
\pgfpathlineto{\pgfqpoint{4.310155in}{1.154726in}}%
\pgfpathmoveto{\pgfqpoint{4.305614in}{1.157675in}}%
\pgfpathlineto{\pgfqpoint{4.305614in}{1.157675in}}%
\pgfpathlineto{\pgfqpoint{4.305614in}{1.160624in}}%
\pgfpathlineto{\pgfqpoint{4.310155in}{1.160624in}}%
\pgfpathlineto{\pgfqpoint{4.310155in}{1.157675in}}%
\pgfpathmoveto{\pgfqpoint{4.310155in}{1.157675in}}%
\pgfpathlineto{\pgfqpoint{4.310155in}{1.157675in}}%
\pgfpathlineto{\pgfqpoint{4.310155in}{1.160624in}}%
\pgfpathlineto{\pgfqpoint{4.314695in}{1.160624in}}%
\pgfpathlineto{\pgfqpoint{4.314695in}{1.157675in}}%
\pgfpathmoveto{\pgfqpoint{4.310155in}{1.160624in}}%
\pgfpathlineto{\pgfqpoint{4.310155in}{1.160624in}}%
\pgfpathlineto{\pgfqpoint{4.310155in}{1.163573in}}%
\pgfpathlineto{\pgfqpoint{4.314695in}{1.163573in}}%
\pgfpathlineto{\pgfqpoint{4.314695in}{1.160624in}}%
\pgfpathmoveto{\pgfqpoint{4.310155in}{1.163573in}}%
\pgfpathlineto{\pgfqpoint{4.310155in}{1.163573in}}%
\pgfpathlineto{\pgfqpoint{4.310155in}{1.166523in}}%
\pgfpathlineto{\pgfqpoint{4.314695in}{1.166523in}}%
\pgfpathlineto{\pgfqpoint{4.314695in}{1.163573in}}%
\pgfpathmoveto{\pgfqpoint{4.314695in}{1.163573in}}%
\pgfpathlineto{\pgfqpoint{4.314695in}{1.163573in}}%
\pgfpathlineto{\pgfqpoint{4.314695in}{1.166523in}}%
\pgfpathlineto{\pgfqpoint{4.319236in}{1.166523in}}%
\pgfpathlineto{\pgfqpoint{4.319236in}{1.163573in}}%
\pgfpathmoveto{\pgfqpoint{4.314695in}{1.166523in}}%
\pgfpathlineto{\pgfqpoint{4.314695in}{1.166523in}}%
\pgfpathlineto{\pgfqpoint{4.314695in}{1.169472in}}%
\pgfpathlineto{\pgfqpoint{4.319236in}{1.169472in}}%
\pgfpathlineto{\pgfqpoint{4.319236in}{1.166523in}}%
\pgfpathmoveto{\pgfqpoint{4.319236in}{1.166523in}}%
\pgfpathlineto{\pgfqpoint{4.319236in}{1.166523in}}%
\pgfpathlineto{\pgfqpoint{4.319236in}{1.169472in}}%
\pgfpathlineto{\pgfqpoint{4.323777in}{1.169472in}}%
\pgfpathlineto{\pgfqpoint{4.323777in}{1.166523in}}%
\pgfpathmoveto{\pgfqpoint{4.319236in}{1.169472in}}%
\pgfpathlineto{\pgfqpoint{4.319236in}{1.169472in}}%
\pgfpathlineto{\pgfqpoint{4.319236in}{1.172421in}}%
\pgfpathlineto{\pgfqpoint{4.323777in}{1.172421in}}%
\pgfpathlineto{\pgfqpoint{4.323777in}{1.169472in}}%
\pgfpathmoveto{\pgfqpoint{4.323777in}{1.169472in}}%
\pgfpathlineto{\pgfqpoint{4.323777in}{1.169472in}}%
\pgfpathlineto{\pgfqpoint{4.323777in}{1.172421in}}%
\pgfpathlineto{\pgfqpoint{4.328318in}{1.172421in}}%
\pgfpathlineto{\pgfqpoint{4.328318in}{1.169472in}}%
\pgfpathmoveto{\pgfqpoint{4.323777in}{1.172421in}}%
\pgfpathlineto{\pgfqpoint{4.323777in}{1.172421in}}%
\pgfpathlineto{\pgfqpoint{4.323777in}{1.175370in}}%
\pgfpathlineto{\pgfqpoint{4.328318in}{1.175370in}}%
\pgfpathlineto{\pgfqpoint{4.328318in}{1.172421in}}%
\pgfpathmoveto{\pgfqpoint{4.328318in}{1.172421in}}%
\pgfpathlineto{\pgfqpoint{4.328318in}{1.172421in}}%
\pgfpathlineto{\pgfqpoint{4.328318in}{1.175370in}}%
\pgfpathlineto{\pgfqpoint{4.332859in}{1.175370in}}%
\pgfpathlineto{\pgfqpoint{4.332859in}{1.172421in}}%
\pgfpathmoveto{\pgfqpoint{4.328318in}{1.175370in}}%
\pgfpathlineto{\pgfqpoint{4.328318in}{1.175370in}}%
\pgfpathlineto{\pgfqpoint{4.328318in}{1.178319in}}%
\pgfpathlineto{\pgfqpoint{4.332859in}{1.178319in}}%
\pgfpathlineto{\pgfqpoint{4.332859in}{1.175370in}}%
\pgfpathmoveto{\pgfqpoint{4.332859in}{1.175370in}}%
\pgfpathlineto{\pgfqpoint{4.332859in}{1.175370in}}%
\pgfpathlineto{\pgfqpoint{4.332859in}{1.178319in}}%
\pgfpathlineto{\pgfqpoint{4.337400in}{1.178319in}}%
\pgfpathlineto{\pgfqpoint{4.337400in}{1.175370in}}%
\pgfpathmoveto{\pgfqpoint{4.332859in}{1.178319in}}%
\pgfpathlineto{\pgfqpoint{4.332859in}{1.178319in}}%
\pgfpathlineto{\pgfqpoint{4.332859in}{1.181269in}}%
\pgfpathlineto{\pgfqpoint{4.337400in}{1.181269in}}%
\pgfpathlineto{\pgfqpoint{4.337400in}{1.178319in}}%
\pgfpathmoveto{\pgfqpoint{4.332859in}{1.181269in}}%
\pgfpathlineto{\pgfqpoint{4.332859in}{1.181269in}}%
\pgfpathlineto{\pgfqpoint{4.332859in}{1.184218in}}%
\pgfpathlineto{\pgfqpoint{4.337400in}{1.184218in}}%
\pgfpathlineto{\pgfqpoint{4.337400in}{1.181269in}}%
\pgfpathmoveto{\pgfqpoint{4.337400in}{1.181269in}}%
\pgfpathlineto{\pgfqpoint{4.337400in}{1.181269in}}%
\pgfpathlineto{\pgfqpoint{4.337400in}{1.184218in}}%
\pgfpathlineto{\pgfqpoint{4.341941in}{1.184218in}}%
\pgfpathlineto{\pgfqpoint{4.341941in}{1.181269in}}%
\pgfpathmoveto{\pgfqpoint{4.337400in}{1.184218in}}%
\pgfpathlineto{\pgfqpoint{4.337400in}{1.184218in}}%
\pgfpathlineto{\pgfqpoint{4.337400in}{1.187167in}}%
\pgfpathlineto{\pgfqpoint{4.341941in}{1.187167in}}%
\pgfpathlineto{\pgfqpoint{4.341941in}{1.184218in}}%
\pgfpathmoveto{\pgfqpoint{4.341941in}{1.184218in}}%
\pgfpathlineto{\pgfqpoint{4.341941in}{1.184218in}}%
\pgfpathlineto{\pgfqpoint{4.341941in}{1.187167in}}%
\pgfpathlineto{\pgfqpoint{4.346482in}{1.187167in}}%
\pgfpathlineto{\pgfqpoint{4.346482in}{1.184218in}}%
\pgfpathmoveto{\pgfqpoint{4.341941in}{1.187167in}}%
\pgfpathlineto{\pgfqpoint{4.341941in}{1.187167in}}%
\pgfpathlineto{\pgfqpoint{4.341941in}{1.190116in}}%
\pgfpathlineto{\pgfqpoint{4.346482in}{1.190116in}}%
\pgfpathlineto{\pgfqpoint{4.346482in}{1.187167in}}%
\pgfpathmoveto{\pgfqpoint{4.346482in}{1.187167in}}%
\pgfpathlineto{\pgfqpoint{4.346482in}{1.187167in}}%
\pgfpathlineto{\pgfqpoint{4.346482in}{1.190116in}}%
\pgfpathlineto{\pgfqpoint{4.351023in}{1.190116in}}%
\pgfpathlineto{\pgfqpoint{4.351023in}{1.187167in}}%
\pgfpathmoveto{\pgfqpoint{4.346482in}{1.190116in}}%
\pgfpathlineto{\pgfqpoint{4.346482in}{1.190116in}}%
\pgfpathlineto{\pgfqpoint{4.346482in}{1.193066in}}%
\pgfpathlineto{\pgfqpoint{4.351023in}{1.193066in}}%
\pgfpathlineto{\pgfqpoint{4.351023in}{1.190116in}}%
\pgfpathmoveto{\pgfqpoint{4.351023in}{1.190116in}}%
\pgfpathlineto{\pgfqpoint{4.351023in}{1.190116in}}%
\pgfpathlineto{\pgfqpoint{4.351023in}{1.193066in}}%
\pgfpathlineto{\pgfqpoint{4.355564in}{1.193066in}}%
\pgfpathlineto{\pgfqpoint{4.355564in}{1.190116in}}%
\pgfpathmoveto{\pgfqpoint{4.351023in}{1.193066in}}%
\pgfpathlineto{\pgfqpoint{4.351023in}{1.193066in}}%
\pgfpathlineto{\pgfqpoint{4.351023in}{1.196015in}}%
\pgfpathlineto{\pgfqpoint{4.355564in}{1.196015in}}%
\pgfpathlineto{\pgfqpoint{4.355564in}{1.193066in}}%
\pgfpathmoveto{\pgfqpoint{4.355564in}{1.193066in}}%
\pgfpathlineto{\pgfqpoint{4.355564in}{1.193066in}}%
\pgfpathlineto{\pgfqpoint{4.355564in}{1.196015in}}%
\pgfpathlineto{\pgfqpoint{4.360105in}{1.196015in}}%
\pgfpathlineto{\pgfqpoint{4.360105in}{1.193066in}}%
\pgfpathmoveto{\pgfqpoint{4.355564in}{1.196015in}}%
\pgfpathlineto{\pgfqpoint{4.355564in}{1.196015in}}%
\pgfpathlineto{\pgfqpoint{4.355564in}{1.198964in}}%
\pgfpathlineto{\pgfqpoint{4.360105in}{1.198964in}}%
\pgfpathlineto{\pgfqpoint{4.360105in}{1.196015in}}%
\pgfpathmoveto{\pgfqpoint{4.355564in}{1.198964in}}%
\pgfpathlineto{\pgfqpoint{4.355564in}{1.198964in}}%
\pgfpathlineto{\pgfqpoint{4.355564in}{1.201913in}}%
\pgfpathlineto{\pgfqpoint{4.360105in}{1.201913in}}%
\pgfpathlineto{\pgfqpoint{4.360105in}{1.198964in}}%
\pgfpathmoveto{\pgfqpoint{4.360105in}{1.198964in}}%
\pgfpathlineto{\pgfqpoint{4.360105in}{1.198964in}}%
\pgfpathlineto{\pgfqpoint{4.360105in}{1.201913in}}%
\pgfpathlineto{\pgfqpoint{4.364645in}{1.201913in}}%
\pgfpathlineto{\pgfqpoint{4.364645in}{1.198964in}}%
\pgfpathmoveto{\pgfqpoint{4.360105in}{1.201913in}}%
\pgfpathlineto{\pgfqpoint{4.360105in}{1.201913in}}%
\pgfpathlineto{\pgfqpoint{4.360105in}{1.204863in}}%
\pgfpathlineto{\pgfqpoint{4.364645in}{1.204863in}}%
\pgfpathlineto{\pgfqpoint{4.364645in}{1.201913in}}%
\pgfpathmoveto{\pgfqpoint{4.364645in}{1.201913in}}%
\pgfpathlineto{\pgfqpoint{4.364645in}{1.201913in}}%
\pgfpathlineto{\pgfqpoint{4.364645in}{1.204863in}}%
\pgfpathlineto{\pgfqpoint{4.369186in}{1.204863in}}%
\pgfpathlineto{\pgfqpoint{4.369186in}{1.201913in}}%
\pgfpathmoveto{\pgfqpoint{4.364645in}{1.204863in}}%
\pgfpathlineto{\pgfqpoint{4.364645in}{1.204863in}}%
\pgfpathlineto{\pgfqpoint{4.364645in}{1.207812in}}%
\pgfpathlineto{\pgfqpoint{4.369186in}{1.207812in}}%
\pgfpathlineto{\pgfqpoint{4.369186in}{1.204863in}}%
\pgfpathmoveto{\pgfqpoint{4.369186in}{1.204863in}}%
\pgfpathlineto{\pgfqpoint{4.369186in}{1.204863in}}%
\pgfpathlineto{\pgfqpoint{4.369186in}{1.207812in}}%
\pgfpathlineto{\pgfqpoint{4.373727in}{1.207812in}}%
\pgfpathlineto{\pgfqpoint{4.373727in}{1.204863in}}%
\pgfpathmoveto{\pgfqpoint{4.369186in}{1.207812in}}%
\pgfpathlineto{\pgfqpoint{4.369186in}{1.207812in}}%
\pgfpathlineto{\pgfqpoint{4.369186in}{1.210761in}}%
\pgfpathlineto{\pgfqpoint{4.373727in}{1.210761in}}%
\pgfpathlineto{\pgfqpoint{4.373727in}{1.207812in}}%
\pgfpathmoveto{\pgfqpoint{4.373727in}{1.207812in}}%
\pgfpathlineto{\pgfqpoint{4.373727in}{1.207812in}}%
\pgfpathlineto{\pgfqpoint{4.373727in}{1.210761in}}%
\pgfpathlineto{\pgfqpoint{4.378268in}{1.210761in}}%
\pgfpathlineto{\pgfqpoint{4.378268in}{1.207812in}}%
\pgfpathmoveto{\pgfqpoint{4.373727in}{1.210761in}}%
\pgfpathlineto{\pgfqpoint{4.373727in}{1.210761in}}%
\pgfpathlineto{\pgfqpoint{4.373727in}{1.213710in}}%
\pgfpathlineto{\pgfqpoint{4.378268in}{1.213710in}}%
\pgfpathlineto{\pgfqpoint{4.378268in}{1.210761in}}%
\pgfpathmoveto{\pgfqpoint{4.378268in}{1.210761in}}%
\pgfpathlineto{\pgfqpoint{4.378268in}{1.210761in}}%
\pgfpathlineto{\pgfqpoint{4.378268in}{1.213710in}}%
\pgfpathlineto{\pgfqpoint{4.382809in}{1.213710in}}%
\pgfpathlineto{\pgfqpoint{4.382809in}{1.210761in}}%
\pgfpathmoveto{\pgfqpoint{4.378268in}{1.213710in}}%
\pgfpathlineto{\pgfqpoint{4.378268in}{1.213710in}}%
\pgfpathlineto{\pgfqpoint{4.378268in}{1.216659in}}%
\pgfpathlineto{\pgfqpoint{4.382809in}{1.216659in}}%
\pgfpathlineto{\pgfqpoint{4.382809in}{1.213710in}}%
\pgfpathmoveto{\pgfqpoint{4.378268in}{1.216659in}}%
\pgfpathlineto{\pgfqpoint{4.378268in}{1.216659in}}%
\pgfpathlineto{\pgfqpoint{4.378268in}{1.219609in}}%
\pgfpathlineto{\pgfqpoint{4.382809in}{1.219609in}}%
\pgfpathlineto{\pgfqpoint{4.382809in}{1.216659in}}%
\pgfpathmoveto{\pgfqpoint{4.382809in}{1.216659in}}%
\pgfpathlineto{\pgfqpoint{4.382809in}{1.216659in}}%
\pgfpathlineto{\pgfqpoint{4.382809in}{1.219609in}}%
\pgfpathlineto{\pgfqpoint{4.387350in}{1.219609in}}%
\pgfpathlineto{\pgfqpoint{4.387350in}{1.216659in}}%
\pgfpathmoveto{\pgfqpoint{4.382809in}{1.219609in}}%
\pgfpathlineto{\pgfqpoint{4.382809in}{1.219609in}}%
\pgfpathlineto{\pgfqpoint{4.382809in}{1.222558in}}%
\pgfpathlineto{\pgfqpoint{4.387350in}{1.222558in}}%
\pgfpathlineto{\pgfqpoint{4.387350in}{1.219609in}}%
\pgfpathmoveto{\pgfqpoint{4.387350in}{1.219609in}}%
\pgfpathlineto{\pgfqpoint{4.387350in}{1.219609in}}%
\pgfpathlineto{\pgfqpoint{4.387350in}{1.222558in}}%
\pgfpathlineto{\pgfqpoint{4.391892in}{1.222558in}}%
\pgfpathlineto{\pgfqpoint{4.391892in}{1.219609in}}%
\pgfpathmoveto{\pgfqpoint{4.387350in}{1.222558in}}%
\pgfpathlineto{\pgfqpoint{4.387350in}{1.222558in}}%
\pgfpathlineto{\pgfqpoint{4.387350in}{1.225507in}}%
\pgfpathlineto{\pgfqpoint{4.391892in}{1.225507in}}%
\pgfpathlineto{\pgfqpoint{4.391892in}{1.222558in}}%
\pgfpathmoveto{\pgfqpoint{4.391892in}{1.222558in}}%
\pgfpathlineto{\pgfqpoint{4.391892in}{1.222558in}}%
\pgfpathlineto{\pgfqpoint{4.391892in}{1.225507in}}%
\pgfpathlineto{\pgfqpoint{4.396433in}{1.225507in}}%
\pgfpathlineto{\pgfqpoint{4.396433in}{1.222558in}}%
\pgfpathmoveto{\pgfqpoint{4.391892in}{1.225507in}}%
\pgfpathlineto{\pgfqpoint{4.391892in}{1.225507in}}%
\pgfpathlineto{\pgfqpoint{4.391892in}{1.228456in}}%
\pgfpathlineto{\pgfqpoint{4.396433in}{1.228456in}}%
\pgfpathlineto{\pgfqpoint{4.396433in}{1.225507in}}%
\pgfpathmoveto{\pgfqpoint{4.396433in}{1.225507in}}%
\pgfpathlineto{\pgfqpoint{4.396433in}{1.225507in}}%
\pgfpathlineto{\pgfqpoint{4.396433in}{1.228456in}}%
\pgfpathlineto{\pgfqpoint{4.400974in}{1.228456in}}%
\pgfpathlineto{\pgfqpoint{4.400974in}{1.225507in}}%
\pgfpathmoveto{\pgfqpoint{4.396433in}{1.228456in}}%
\pgfpathlineto{\pgfqpoint{4.396433in}{1.228456in}}%
\pgfpathlineto{\pgfqpoint{4.396433in}{1.231406in}}%
\pgfpathlineto{\pgfqpoint{4.400974in}{1.231406in}}%
\pgfpathlineto{\pgfqpoint{4.400974in}{1.228456in}}%
\pgfpathmoveto{\pgfqpoint{4.400974in}{1.228456in}}%
\pgfpathlineto{\pgfqpoint{4.400974in}{1.228456in}}%
\pgfpathlineto{\pgfqpoint{4.400974in}{1.231406in}}%
\pgfpathlineto{\pgfqpoint{4.405515in}{1.231406in}}%
\pgfpathlineto{\pgfqpoint{4.405515in}{1.228456in}}%
\pgfpathmoveto{\pgfqpoint{4.400974in}{1.231406in}}%
\pgfpathlineto{\pgfqpoint{4.400974in}{1.231406in}}%
\pgfpathlineto{\pgfqpoint{4.400974in}{1.234355in}}%
\pgfpathlineto{\pgfqpoint{4.405515in}{1.234355in}}%
\pgfpathlineto{\pgfqpoint{4.405515in}{1.231406in}}%
\pgfpathmoveto{\pgfqpoint{4.400974in}{1.234355in}}%
\pgfpathlineto{\pgfqpoint{4.400974in}{1.234355in}}%
\pgfpathlineto{\pgfqpoint{4.400974in}{1.237304in}}%
\pgfpathlineto{\pgfqpoint{4.405515in}{1.237304in}}%
\pgfpathlineto{\pgfqpoint{4.405515in}{1.234355in}}%
\pgfpathmoveto{\pgfqpoint{4.405515in}{1.234355in}}%
\pgfpathlineto{\pgfqpoint{4.405515in}{1.234355in}}%
\pgfpathlineto{\pgfqpoint{4.405515in}{1.237304in}}%
\pgfpathlineto{\pgfqpoint{4.410056in}{1.237304in}}%
\pgfpathlineto{\pgfqpoint{4.410056in}{1.234355in}}%
\pgfpathmoveto{\pgfqpoint{4.405515in}{1.237304in}}%
\pgfpathlineto{\pgfqpoint{4.405515in}{1.237304in}}%
\pgfpathlineto{\pgfqpoint{4.405515in}{1.240253in}}%
\pgfpathlineto{\pgfqpoint{4.410056in}{1.240253in}}%
\pgfpathlineto{\pgfqpoint{4.410056in}{1.237304in}}%
\pgfpathmoveto{\pgfqpoint{4.410056in}{1.237304in}}%
\pgfpathlineto{\pgfqpoint{4.410056in}{1.237304in}}%
\pgfpathlineto{\pgfqpoint{4.410056in}{1.240253in}}%
\pgfpathlineto{\pgfqpoint{4.414598in}{1.240253in}}%
\pgfpathlineto{\pgfqpoint{4.414598in}{1.237304in}}%
\pgfpathmoveto{\pgfqpoint{4.410056in}{1.240253in}}%
\pgfpathlineto{\pgfqpoint{4.410056in}{1.240253in}}%
\pgfpathlineto{\pgfqpoint{4.410056in}{1.243203in}}%
\pgfpathlineto{\pgfqpoint{4.414598in}{1.243203in}}%
\pgfpathlineto{\pgfqpoint{4.414598in}{1.240253in}}%
\pgfpathmoveto{\pgfqpoint{4.414598in}{1.240253in}}%
\pgfpathlineto{\pgfqpoint{4.414598in}{1.240253in}}%
\pgfpathlineto{\pgfqpoint{4.414598in}{1.243203in}}%
\pgfpathlineto{\pgfqpoint{4.419139in}{1.243203in}}%
\pgfpathlineto{\pgfqpoint{4.419139in}{1.240253in}}%
\pgfpathmoveto{\pgfqpoint{4.414598in}{1.243203in}}%
\pgfpathlineto{\pgfqpoint{4.414598in}{1.243203in}}%
\pgfpathlineto{\pgfqpoint{4.414598in}{1.246152in}}%
\pgfpathlineto{\pgfqpoint{4.419139in}{1.246152in}}%
\pgfpathlineto{\pgfqpoint{4.419139in}{1.243203in}}%
\pgfpathmoveto{\pgfqpoint{4.419139in}{1.243203in}}%
\pgfpathlineto{\pgfqpoint{4.419139in}{1.243203in}}%
\pgfpathlineto{\pgfqpoint{4.419139in}{1.246152in}}%
\pgfpathlineto{\pgfqpoint{4.423680in}{1.246152in}}%
\pgfpathlineto{\pgfqpoint{4.423680in}{1.243203in}}%
\pgfpathmoveto{\pgfqpoint{4.419139in}{1.246152in}}%
\pgfpathlineto{\pgfqpoint{4.419139in}{1.246152in}}%
\pgfpathlineto{\pgfqpoint{4.419139in}{1.249101in}}%
\pgfpathlineto{\pgfqpoint{4.423680in}{1.249101in}}%
\pgfpathlineto{\pgfqpoint{4.423680in}{1.246152in}}%
\pgfpathmoveto{\pgfqpoint{4.423680in}{1.246152in}}%
\pgfpathlineto{\pgfqpoint{4.423680in}{1.246152in}}%
\pgfpathlineto{\pgfqpoint{4.423680in}{1.249101in}}%
\pgfpathlineto{\pgfqpoint{4.428221in}{1.249101in}}%
\pgfpathlineto{\pgfqpoint{4.428221in}{1.246152in}}%
\pgfpathmoveto{\pgfqpoint{4.423680in}{1.249101in}}%
\pgfpathlineto{\pgfqpoint{4.423680in}{1.249101in}}%
\pgfpathlineto{\pgfqpoint{4.423680in}{1.252050in}}%
\pgfpathlineto{\pgfqpoint{4.428221in}{1.252050in}}%
\pgfpathlineto{\pgfqpoint{4.428221in}{1.249101in}}%
\pgfpathmoveto{\pgfqpoint{4.423680in}{1.252050in}}%
\pgfpathlineto{\pgfqpoint{4.423680in}{1.252050in}}%
\pgfpathlineto{\pgfqpoint{4.423680in}{1.254999in}}%
\pgfpathlineto{\pgfqpoint{4.428221in}{1.254999in}}%
\pgfpathlineto{\pgfqpoint{4.428221in}{1.252050in}}%
\pgfpathmoveto{\pgfqpoint{4.428221in}{1.252050in}}%
\pgfpathlineto{\pgfqpoint{4.428221in}{1.252050in}}%
\pgfpathlineto{\pgfqpoint{4.428221in}{1.254999in}}%
\pgfpathlineto{\pgfqpoint{4.432763in}{1.254999in}}%
\pgfpathlineto{\pgfqpoint{4.432763in}{1.252050in}}%
\pgfpathmoveto{\pgfqpoint{4.428221in}{1.254999in}}%
\pgfpathlineto{\pgfqpoint{4.428221in}{1.254999in}}%
\pgfpathlineto{\pgfqpoint{4.428221in}{1.257949in}}%
\pgfpathlineto{\pgfqpoint{4.432763in}{1.257949in}}%
\pgfpathlineto{\pgfqpoint{4.432763in}{1.254999in}}%
\pgfpathmoveto{\pgfqpoint{4.432763in}{1.254999in}}%
\pgfpathlineto{\pgfqpoint{4.432763in}{1.254999in}}%
\pgfpathlineto{\pgfqpoint{4.432763in}{1.257949in}}%
\pgfpathlineto{\pgfqpoint{4.437304in}{1.257949in}}%
\pgfpathlineto{\pgfqpoint{4.437304in}{1.254999in}}%
\pgfpathmoveto{\pgfqpoint{4.432763in}{1.257949in}}%
\pgfpathlineto{\pgfqpoint{4.432763in}{1.257949in}}%
\pgfpathlineto{\pgfqpoint{4.432763in}{1.260898in}}%
\pgfpathlineto{\pgfqpoint{4.437304in}{1.260898in}}%
\pgfpathlineto{\pgfqpoint{4.437304in}{1.257949in}}%
\pgfpathmoveto{\pgfqpoint{4.437304in}{1.257949in}}%
\pgfpathlineto{\pgfqpoint{4.437304in}{1.257949in}}%
\pgfpathlineto{\pgfqpoint{4.437304in}{1.260898in}}%
\pgfpathlineto{\pgfqpoint{4.441845in}{1.260898in}}%
\pgfpathlineto{\pgfqpoint{4.441845in}{1.257949in}}%
\pgfpathmoveto{\pgfqpoint{4.437304in}{1.260898in}}%
\pgfpathlineto{\pgfqpoint{4.437304in}{1.260898in}}%
\pgfpathlineto{\pgfqpoint{4.437304in}{1.263847in}}%
\pgfpathlineto{\pgfqpoint{4.441845in}{1.263847in}}%
\pgfpathlineto{\pgfqpoint{4.441845in}{1.260898in}}%
\pgfpathmoveto{\pgfqpoint{4.441845in}{1.260898in}}%
\pgfpathlineto{\pgfqpoint{4.441845in}{1.260898in}}%
\pgfpathlineto{\pgfqpoint{4.441845in}{1.263847in}}%
\pgfpathlineto{\pgfqpoint{4.446386in}{1.263847in}}%
\pgfpathlineto{\pgfqpoint{4.446386in}{1.260898in}}%
\pgfpathmoveto{\pgfqpoint{4.441845in}{1.263847in}}%
\pgfpathlineto{\pgfqpoint{4.441845in}{1.263847in}}%
\pgfpathlineto{\pgfqpoint{4.441845in}{1.266796in}}%
\pgfpathlineto{\pgfqpoint{4.446386in}{1.266796in}}%
\pgfpathlineto{\pgfqpoint{4.446386in}{1.263847in}}%
\pgfpathmoveto{\pgfqpoint{4.441845in}{1.266796in}}%
\pgfpathlineto{\pgfqpoint{4.441845in}{1.266796in}}%
\pgfpathlineto{\pgfqpoint{4.441845in}{1.269746in}}%
\pgfpathlineto{\pgfqpoint{4.446386in}{1.269746in}}%
\pgfpathlineto{\pgfqpoint{4.446386in}{1.266796in}}%
\pgfpathmoveto{\pgfqpoint{4.446386in}{1.266796in}}%
\pgfpathlineto{\pgfqpoint{4.446386in}{1.266796in}}%
\pgfpathlineto{\pgfqpoint{4.446386in}{1.269746in}}%
\pgfpathlineto{\pgfqpoint{4.450927in}{1.269746in}}%
\pgfpathlineto{\pgfqpoint{4.450927in}{1.266796in}}%
\pgfpathmoveto{\pgfqpoint{4.446386in}{1.269746in}}%
\pgfpathlineto{\pgfqpoint{4.446386in}{1.269746in}}%
\pgfpathlineto{\pgfqpoint{4.446386in}{1.272695in}}%
\pgfpathlineto{\pgfqpoint{4.450927in}{1.272695in}}%
\pgfpathlineto{\pgfqpoint{4.450927in}{1.269746in}}%
\pgfpathmoveto{\pgfqpoint{4.450927in}{1.269746in}}%
\pgfpathlineto{\pgfqpoint{4.450927in}{1.269746in}}%
\pgfpathlineto{\pgfqpoint{4.450927in}{1.272695in}}%
\pgfpathlineto{\pgfqpoint{4.455469in}{1.272695in}}%
\pgfpathlineto{\pgfqpoint{4.455469in}{1.269746in}}%
\pgfpathmoveto{\pgfqpoint{4.450927in}{1.272695in}}%
\pgfpathlineto{\pgfqpoint{4.450927in}{1.272695in}}%
\pgfpathlineto{\pgfqpoint{4.450927in}{1.275644in}}%
\pgfpathlineto{\pgfqpoint{4.455469in}{1.275644in}}%
\pgfpathlineto{\pgfqpoint{4.455469in}{1.272695in}}%
\pgfpathmoveto{\pgfqpoint{4.455469in}{1.272695in}}%
\pgfpathlineto{\pgfqpoint{4.455469in}{1.272695in}}%
\pgfpathlineto{\pgfqpoint{4.455469in}{1.275644in}}%
\pgfpathlineto{\pgfqpoint{4.460010in}{1.275644in}}%
\pgfpathlineto{\pgfqpoint{4.460010in}{1.272695in}}%
\pgfpathmoveto{\pgfqpoint{4.455469in}{1.275644in}}%
\pgfpathlineto{\pgfqpoint{4.455469in}{1.275644in}}%
\pgfpathlineto{\pgfqpoint{4.455469in}{1.278593in}}%
\pgfpathlineto{\pgfqpoint{4.460010in}{1.278593in}}%
\pgfpathlineto{\pgfqpoint{4.460010in}{1.275644in}}%
\pgfpathmoveto{\pgfqpoint{4.460010in}{1.275644in}}%
\pgfpathlineto{\pgfqpoint{4.460010in}{1.275644in}}%
\pgfpathlineto{\pgfqpoint{4.460010in}{1.278593in}}%
\pgfpathlineto{\pgfqpoint{4.464551in}{1.278593in}}%
\pgfpathlineto{\pgfqpoint{4.464551in}{1.275644in}}%
\pgfpathmoveto{\pgfqpoint{4.460010in}{1.278593in}}%
\pgfpathlineto{\pgfqpoint{4.460010in}{1.278593in}}%
\pgfpathlineto{\pgfqpoint{4.460010in}{1.281543in}}%
\pgfpathlineto{\pgfqpoint{4.464551in}{1.281543in}}%
\pgfpathlineto{\pgfqpoint{4.464551in}{1.278593in}}%
\pgfpathmoveto{\pgfqpoint{4.464551in}{1.278593in}}%
\pgfpathlineto{\pgfqpoint{4.464551in}{1.278593in}}%
\pgfpathlineto{\pgfqpoint{4.464551in}{1.281543in}}%
\pgfpathlineto{\pgfqpoint{4.469092in}{1.281543in}}%
\pgfpathlineto{\pgfqpoint{4.469092in}{1.278593in}}%
\pgfpathmoveto{\pgfqpoint{4.464551in}{1.281543in}}%
\pgfpathlineto{\pgfqpoint{4.464551in}{1.281543in}}%
\pgfpathlineto{\pgfqpoint{4.464551in}{1.284492in}}%
\pgfpathlineto{\pgfqpoint{4.469092in}{1.284492in}}%
\pgfpathlineto{\pgfqpoint{4.469092in}{1.281543in}}%
\pgfpathmoveto{\pgfqpoint{4.464551in}{1.284492in}}%
\pgfpathlineto{\pgfqpoint{4.464551in}{1.284492in}}%
\pgfpathlineto{\pgfqpoint{4.464551in}{1.287441in}}%
\pgfpathlineto{\pgfqpoint{4.469092in}{1.287441in}}%
\pgfpathlineto{\pgfqpoint{4.469092in}{1.284492in}}%
\pgfpathmoveto{\pgfqpoint{4.469092in}{1.284492in}}%
\pgfpathlineto{\pgfqpoint{4.469092in}{1.284492in}}%
\pgfpathlineto{\pgfqpoint{4.469092in}{1.287441in}}%
\pgfpathlineto{\pgfqpoint{4.473634in}{1.287441in}}%
\pgfpathlineto{\pgfqpoint{4.473634in}{1.284492in}}%
\pgfpathmoveto{\pgfqpoint{4.469092in}{1.287441in}}%
\pgfpathlineto{\pgfqpoint{4.469092in}{1.287441in}}%
\pgfpathlineto{\pgfqpoint{4.469092in}{1.290390in}}%
\pgfpathlineto{\pgfqpoint{4.473634in}{1.290390in}}%
\pgfpathlineto{\pgfqpoint{4.473634in}{1.287441in}}%
\pgfpathmoveto{\pgfqpoint{4.473634in}{1.287441in}}%
\pgfpathlineto{\pgfqpoint{4.473634in}{1.287441in}}%
\pgfpathlineto{\pgfqpoint{4.473634in}{1.290390in}}%
\pgfpathlineto{\pgfqpoint{4.478175in}{1.290390in}}%
\pgfpathlineto{\pgfqpoint{4.478175in}{1.287441in}}%
\pgfpathmoveto{\pgfqpoint{4.473634in}{1.290390in}}%
\pgfpathlineto{\pgfqpoint{4.473634in}{1.290390in}}%
\pgfpathlineto{\pgfqpoint{4.473634in}{1.293340in}}%
\pgfpathlineto{\pgfqpoint{4.478175in}{1.293340in}}%
\pgfpathlineto{\pgfqpoint{4.478175in}{1.290390in}}%
\pgfpathmoveto{\pgfqpoint{4.478175in}{1.290390in}}%
\pgfpathlineto{\pgfqpoint{4.478175in}{1.290390in}}%
\pgfpathlineto{\pgfqpoint{4.478175in}{1.293340in}}%
\pgfpathlineto{\pgfqpoint{4.482716in}{1.293340in}}%
\pgfpathlineto{\pgfqpoint{4.482716in}{1.290390in}}%
\pgfpathmoveto{\pgfqpoint{4.478175in}{1.293340in}}%
\pgfpathlineto{\pgfqpoint{4.478175in}{1.293340in}}%
\pgfpathlineto{\pgfqpoint{4.478175in}{1.296289in}}%
\pgfpathlineto{\pgfqpoint{4.482716in}{1.296289in}}%
\pgfpathlineto{\pgfqpoint{4.482716in}{1.293340in}}%
\pgfpathmoveto{\pgfqpoint{4.482716in}{1.293340in}}%
\pgfpathlineto{\pgfqpoint{4.482716in}{1.293340in}}%
\pgfpathlineto{\pgfqpoint{4.482716in}{1.296289in}}%
\pgfpathlineto{\pgfqpoint{4.487257in}{1.296289in}}%
\pgfpathlineto{\pgfqpoint{4.487257in}{1.293340in}}%
\pgfpathmoveto{\pgfqpoint{4.482716in}{1.296289in}}%
\pgfpathlineto{\pgfqpoint{4.482716in}{1.296289in}}%
\pgfpathlineto{\pgfqpoint{4.482716in}{1.299238in}}%
\pgfpathlineto{\pgfqpoint{4.487257in}{1.299238in}}%
\pgfpathlineto{\pgfqpoint{4.487257in}{1.296289in}}%
\pgfpathmoveto{\pgfqpoint{4.487257in}{1.296289in}}%
\pgfpathlineto{\pgfqpoint{4.487257in}{1.296289in}}%
\pgfpathlineto{\pgfqpoint{4.487257in}{1.299238in}}%
\pgfpathlineto{\pgfqpoint{4.491798in}{1.299238in}}%
\pgfpathlineto{\pgfqpoint{4.491798in}{1.296289in}}%
\pgfpathmoveto{\pgfqpoint{4.487257in}{1.299238in}}%
\pgfpathlineto{\pgfqpoint{4.487257in}{1.299238in}}%
\pgfpathlineto{\pgfqpoint{4.487257in}{1.302187in}}%
\pgfpathlineto{\pgfqpoint{4.491798in}{1.302187in}}%
\pgfpathlineto{\pgfqpoint{4.491798in}{1.299238in}}%
\pgfpathmoveto{\pgfqpoint{4.487257in}{1.302187in}}%
\pgfpathlineto{\pgfqpoint{4.487257in}{1.302187in}}%
\pgfpathlineto{\pgfqpoint{4.487257in}{1.305137in}}%
\pgfpathlineto{\pgfqpoint{4.491798in}{1.305137in}}%
\pgfpathlineto{\pgfqpoint{4.491798in}{1.302187in}}%
\pgfpathmoveto{\pgfqpoint{4.491798in}{1.302187in}}%
\pgfpathlineto{\pgfqpoint{4.491798in}{1.302187in}}%
\pgfpathlineto{\pgfqpoint{4.491798in}{1.305137in}}%
\pgfpathlineto{\pgfqpoint{4.496340in}{1.305137in}}%
\pgfpathlineto{\pgfqpoint{4.496340in}{1.302187in}}%
\pgfpathmoveto{\pgfqpoint{4.491798in}{1.305137in}}%
\pgfpathlineto{\pgfqpoint{4.491798in}{1.305137in}}%
\pgfpathlineto{\pgfqpoint{4.491798in}{1.308086in}}%
\pgfpathlineto{\pgfqpoint{4.496340in}{1.308086in}}%
\pgfpathlineto{\pgfqpoint{4.496340in}{1.305137in}}%
\pgfpathmoveto{\pgfqpoint{4.496340in}{1.305137in}}%
\pgfpathlineto{\pgfqpoint{4.496340in}{1.305137in}}%
\pgfpathlineto{\pgfqpoint{4.496340in}{1.308086in}}%
\pgfpathlineto{\pgfqpoint{4.500881in}{1.308086in}}%
\pgfpathlineto{\pgfqpoint{4.500881in}{1.305137in}}%
\pgfpathmoveto{\pgfqpoint{4.496340in}{1.308086in}}%
\pgfpathlineto{\pgfqpoint{4.496340in}{1.308086in}}%
\pgfpathlineto{\pgfqpoint{4.496340in}{1.311035in}}%
\pgfpathlineto{\pgfqpoint{4.500881in}{1.311035in}}%
\pgfpathlineto{\pgfqpoint{4.500881in}{1.308086in}}%
\pgfpathmoveto{\pgfqpoint{4.500881in}{1.308086in}}%
\pgfpathlineto{\pgfqpoint{4.500881in}{1.308086in}}%
\pgfpathlineto{\pgfqpoint{4.500881in}{1.311035in}}%
\pgfpathlineto{\pgfqpoint{4.505422in}{1.311035in}}%
\pgfpathlineto{\pgfqpoint{4.505422in}{1.308086in}}%
\pgfpathmoveto{\pgfqpoint{4.500881in}{1.311035in}}%
\pgfpathlineto{\pgfqpoint{4.500881in}{1.311035in}}%
\pgfpathlineto{\pgfqpoint{4.500881in}{1.313984in}}%
\pgfpathlineto{\pgfqpoint{4.505422in}{1.313984in}}%
\pgfpathlineto{\pgfqpoint{4.505422in}{1.311035in}}%
\pgfpathmoveto{\pgfqpoint{4.505422in}{1.311035in}}%
\pgfpathlineto{\pgfqpoint{4.505422in}{1.311035in}}%
\pgfpathlineto{\pgfqpoint{4.505422in}{1.313984in}}%
\pgfpathlineto{\pgfqpoint{4.509963in}{1.313984in}}%
\pgfpathlineto{\pgfqpoint{4.509963in}{1.311035in}}%
\pgfpathmoveto{\pgfqpoint{4.505422in}{1.313984in}}%
\pgfpathlineto{\pgfqpoint{4.505422in}{1.313984in}}%
\pgfpathlineto{\pgfqpoint{4.505422in}{1.316934in}}%
\pgfpathlineto{\pgfqpoint{4.509963in}{1.316934in}}%
\pgfpathlineto{\pgfqpoint{4.509963in}{1.313984in}}%
\pgfpathmoveto{\pgfqpoint{4.509963in}{1.313984in}}%
\pgfpathlineto{\pgfqpoint{4.509963in}{1.313984in}}%
\pgfpathlineto{\pgfqpoint{4.509963in}{1.316934in}}%
\pgfpathlineto{\pgfqpoint{4.514504in}{1.316934in}}%
\pgfpathlineto{\pgfqpoint{4.514504in}{1.313984in}}%
\pgfpathmoveto{\pgfqpoint{4.509963in}{1.316934in}}%
\pgfpathlineto{\pgfqpoint{4.509963in}{1.316934in}}%
\pgfpathlineto{\pgfqpoint{4.509963in}{1.319883in}}%
\pgfpathlineto{\pgfqpoint{4.514504in}{1.319883in}}%
\pgfpathlineto{\pgfqpoint{4.514504in}{1.316934in}}%
\pgfpathmoveto{\pgfqpoint{4.509963in}{1.319883in}}%
\pgfpathlineto{\pgfqpoint{4.509963in}{1.319883in}}%
\pgfpathlineto{\pgfqpoint{4.509963in}{1.322832in}}%
\pgfpathlineto{\pgfqpoint{4.514504in}{1.322832in}}%
\pgfpathlineto{\pgfqpoint{4.514504in}{1.319883in}}%
\pgfpathmoveto{\pgfqpoint{4.514504in}{1.319883in}}%
\pgfpathlineto{\pgfqpoint{4.514504in}{1.319883in}}%
\pgfpathlineto{\pgfqpoint{4.514504in}{1.322832in}}%
\pgfpathlineto{\pgfqpoint{4.519046in}{1.322832in}}%
\pgfpathlineto{\pgfqpoint{4.519046in}{1.319883in}}%
\pgfpathmoveto{\pgfqpoint{4.514504in}{1.322832in}}%
\pgfpathlineto{\pgfqpoint{4.514504in}{1.322832in}}%
\pgfpathlineto{\pgfqpoint{4.514504in}{1.325781in}}%
\pgfpathlineto{\pgfqpoint{4.519046in}{1.325781in}}%
\pgfpathlineto{\pgfqpoint{4.519046in}{1.322832in}}%
\pgfpathmoveto{\pgfqpoint{4.519046in}{1.322832in}}%
\pgfpathlineto{\pgfqpoint{4.519046in}{1.322832in}}%
\pgfpathlineto{\pgfqpoint{4.519046in}{1.325781in}}%
\pgfpathlineto{\pgfqpoint{4.523587in}{1.325781in}}%
\pgfpathlineto{\pgfqpoint{4.523587in}{1.322832in}}%
\pgfpathmoveto{\pgfqpoint{4.519046in}{1.325781in}}%
\pgfpathlineto{\pgfqpoint{4.519046in}{1.325781in}}%
\pgfpathlineto{\pgfqpoint{4.519046in}{1.328731in}}%
\pgfpathlineto{\pgfqpoint{4.523587in}{1.328731in}}%
\pgfpathlineto{\pgfqpoint{4.523587in}{1.325781in}}%
\pgfpathmoveto{\pgfqpoint{4.523587in}{1.325781in}}%
\pgfpathlineto{\pgfqpoint{4.523587in}{1.325781in}}%
\pgfpathlineto{\pgfqpoint{4.523587in}{1.328731in}}%
\pgfpathlineto{\pgfqpoint{4.528128in}{1.328731in}}%
\pgfpathlineto{\pgfqpoint{4.528128in}{1.325781in}}%
\pgfpathmoveto{\pgfqpoint{4.523587in}{1.328731in}}%
\pgfpathlineto{\pgfqpoint{4.523587in}{1.328731in}}%
\pgfpathlineto{\pgfqpoint{4.523587in}{1.331680in}}%
\pgfpathlineto{\pgfqpoint{4.528128in}{1.331680in}}%
\pgfpathlineto{\pgfqpoint{4.528128in}{1.328731in}}%
\pgfpathmoveto{\pgfqpoint{4.528128in}{1.328731in}}%
\pgfpathlineto{\pgfqpoint{4.528128in}{1.328731in}}%
\pgfpathlineto{\pgfqpoint{4.528128in}{1.331680in}}%
\pgfpathlineto{\pgfqpoint{4.532669in}{1.331680in}}%
\pgfpathlineto{\pgfqpoint{4.532669in}{1.328731in}}%
\pgfpathmoveto{\pgfqpoint{4.528128in}{1.331680in}}%
\pgfpathlineto{\pgfqpoint{4.528128in}{1.331680in}}%
\pgfpathlineto{\pgfqpoint{4.528128in}{1.334629in}}%
\pgfpathlineto{\pgfqpoint{4.532669in}{1.334629in}}%
\pgfpathlineto{\pgfqpoint{4.532669in}{1.331680in}}%
\pgfpathmoveto{\pgfqpoint{4.532669in}{1.331680in}}%
\pgfpathlineto{\pgfqpoint{4.532669in}{1.331680in}}%
\pgfpathlineto{\pgfqpoint{4.532669in}{1.334629in}}%
\pgfpathlineto{\pgfqpoint{4.537210in}{1.334629in}}%
\pgfpathlineto{\pgfqpoint{4.537210in}{1.331680in}}%
\pgfpathmoveto{\pgfqpoint{4.532669in}{1.334629in}}%
\pgfpathlineto{\pgfqpoint{4.532669in}{1.334629in}}%
\pgfpathlineto{\pgfqpoint{4.532669in}{1.337578in}}%
\pgfpathlineto{\pgfqpoint{4.537210in}{1.337578in}}%
\pgfpathlineto{\pgfqpoint{4.537210in}{1.334629in}}%
\pgfpathmoveto{\pgfqpoint{4.532669in}{1.337578in}}%
\pgfpathlineto{\pgfqpoint{4.532669in}{1.337578in}}%
\pgfpathlineto{\pgfqpoint{4.532669in}{1.340528in}}%
\pgfpathlineto{\pgfqpoint{4.537210in}{1.340528in}}%
\pgfpathlineto{\pgfqpoint{4.537210in}{1.337578in}}%
\pgfpathmoveto{\pgfqpoint{4.537210in}{1.337578in}}%
\pgfpathlineto{\pgfqpoint{4.537210in}{1.337578in}}%
\pgfpathlineto{\pgfqpoint{4.537210in}{1.340528in}}%
\pgfpathlineto{\pgfqpoint{4.541751in}{1.340528in}}%
\pgfpathlineto{\pgfqpoint{4.541751in}{1.337578in}}%
\pgfpathmoveto{\pgfqpoint{4.537210in}{1.340528in}}%
\pgfpathlineto{\pgfqpoint{4.537210in}{1.340528in}}%
\pgfpathlineto{\pgfqpoint{4.537210in}{1.343477in}}%
\pgfpathlineto{\pgfqpoint{4.541751in}{1.343477in}}%
\pgfpathlineto{\pgfqpoint{4.541751in}{1.340528in}}%
\pgfpathmoveto{\pgfqpoint{4.541751in}{1.340528in}}%
\pgfpathlineto{\pgfqpoint{4.541751in}{1.340528in}}%
\pgfpathlineto{\pgfqpoint{4.541751in}{1.343477in}}%
\pgfpathlineto{\pgfqpoint{4.546292in}{1.343477in}}%
\pgfpathlineto{\pgfqpoint{4.546292in}{1.340528in}}%
\pgfpathmoveto{\pgfqpoint{4.541751in}{1.343477in}}%
\pgfpathlineto{\pgfqpoint{4.541751in}{1.343477in}}%
\pgfpathlineto{\pgfqpoint{4.541751in}{1.346426in}}%
\pgfpathlineto{\pgfqpoint{4.546292in}{1.346426in}}%
\pgfpathlineto{\pgfqpoint{4.546292in}{1.343477in}}%
\pgfpathmoveto{\pgfqpoint{4.546292in}{1.343477in}}%
\pgfpathlineto{\pgfqpoint{4.546292in}{1.343477in}}%
\pgfpathlineto{\pgfqpoint{4.546292in}{1.346426in}}%
\pgfpathlineto{\pgfqpoint{4.550833in}{1.346426in}}%
\pgfpathlineto{\pgfqpoint{4.550833in}{1.343477in}}%
\pgfpathmoveto{\pgfqpoint{4.546292in}{1.346426in}}%
\pgfpathlineto{\pgfqpoint{4.546292in}{1.346426in}}%
\pgfpathlineto{\pgfqpoint{4.546292in}{1.349375in}}%
\pgfpathlineto{\pgfqpoint{4.550833in}{1.349375in}}%
\pgfpathlineto{\pgfqpoint{4.550833in}{1.346426in}}%
\pgfpathmoveto{\pgfqpoint{4.550833in}{1.346426in}}%
\pgfpathlineto{\pgfqpoint{4.550833in}{1.346426in}}%
\pgfpathlineto{\pgfqpoint{4.550833in}{1.349375in}}%
\pgfpathlineto{\pgfqpoint{4.555374in}{1.349375in}}%
\pgfpathlineto{\pgfqpoint{4.555374in}{1.346426in}}%
\pgfpathmoveto{\pgfqpoint{4.550833in}{1.349375in}}%
\pgfpathlineto{\pgfqpoint{4.550833in}{1.349375in}}%
\pgfpathlineto{\pgfqpoint{4.550833in}{1.352324in}}%
\pgfpathlineto{\pgfqpoint{4.555374in}{1.352324in}}%
\pgfpathlineto{\pgfqpoint{4.555374in}{1.349375in}}%
\pgfpathmoveto{\pgfqpoint{4.555374in}{1.349375in}}%
\pgfpathlineto{\pgfqpoint{4.555374in}{1.349375in}}%
\pgfpathlineto{\pgfqpoint{4.555374in}{1.352324in}}%
\pgfpathlineto{\pgfqpoint{4.559914in}{1.352324in}}%
\pgfpathlineto{\pgfqpoint{4.559914in}{1.349375in}}%
\pgfpathmoveto{\pgfqpoint{4.555374in}{1.352324in}}%
\pgfpathlineto{\pgfqpoint{4.555374in}{1.352324in}}%
\pgfpathlineto{\pgfqpoint{4.555374in}{1.355274in}}%
\pgfpathlineto{\pgfqpoint{4.559914in}{1.355274in}}%
\pgfpathlineto{\pgfqpoint{4.559914in}{1.352324in}}%
\pgfpathmoveto{\pgfqpoint{4.555374in}{1.355274in}}%
\pgfpathlineto{\pgfqpoint{4.555374in}{1.355274in}}%
\pgfpathlineto{\pgfqpoint{4.555374in}{1.358223in}}%
\pgfpathlineto{\pgfqpoint{4.559914in}{1.358223in}}%
\pgfpathlineto{\pgfqpoint{4.559914in}{1.355274in}}%
\pgfpathmoveto{\pgfqpoint{4.559914in}{1.355274in}}%
\pgfpathlineto{\pgfqpoint{4.559914in}{1.355274in}}%
\pgfpathlineto{\pgfqpoint{4.559914in}{1.358223in}}%
\pgfpathlineto{\pgfqpoint{4.564455in}{1.358223in}}%
\pgfpathlineto{\pgfqpoint{4.564455in}{1.355274in}}%
\pgfpathmoveto{\pgfqpoint{4.559914in}{1.358223in}}%
\pgfpathlineto{\pgfqpoint{4.559914in}{1.358223in}}%
\pgfpathlineto{\pgfqpoint{4.559914in}{1.361172in}}%
\pgfpathlineto{\pgfqpoint{4.564455in}{1.361172in}}%
\pgfpathlineto{\pgfqpoint{4.564455in}{1.358223in}}%
\pgfpathmoveto{\pgfqpoint{4.564455in}{1.358223in}}%
\pgfpathlineto{\pgfqpoint{4.564455in}{1.358223in}}%
\pgfpathlineto{\pgfqpoint{4.564455in}{1.361172in}}%
\pgfpathlineto{\pgfqpoint{4.568996in}{1.361172in}}%
\pgfpathlineto{\pgfqpoint{4.568996in}{1.358223in}}%
\pgfpathmoveto{\pgfqpoint{4.564455in}{1.361172in}}%
\pgfpathlineto{\pgfqpoint{4.564455in}{1.361172in}}%
\pgfpathlineto{\pgfqpoint{4.564455in}{1.364121in}}%
\pgfpathlineto{\pgfqpoint{4.568996in}{1.364121in}}%
\pgfpathlineto{\pgfqpoint{4.568996in}{1.361172in}}%
\pgfpathmoveto{\pgfqpoint{4.568996in}{1.361172in}}%
\pgfpathlineto{\pgfqpoint{4.568996in}{1.361172in}}%
\pgfpathlineto{\pgfqpoint{4.568996in}{1.364121in}}%
\pgfpathlineto{\pgfqpoint{4.573537in}{1.364121in}}%
\pgfpathlineto{\pgfqpoint{4.573537in}{1.361172in}}%
\pgfpathmoveto{\pgfqpoint{4.568996in}{1.364121in}}%
\pgfpathlineto{\pgfqpoint{4.568996in}{1.364121in}}%
\pgfpathlineto{\pgfqpoint{4.568996in}{1.367070in}}%
\pgfpathlineto{\pgfqpoint{4.573537in}{1.367070in}}%
\pgfpathlineto{\pgfqpoint{4.573537in}{1.364121in}}%
\pgfpathmoveto{\pgfqpoint{4.573537in}{1.364121in}}%
\pgfpathlineto{\pgfqpoint{4.573537in}{1.364121in}}%
\pgfpathlineto{\pgfqpoint{4.573537in}{1.367070in}}%
\pgfpathlineto{\pgfqpoint{4.578078in}{1.367070in}}%
\pgfpathlineto{\pgfqpoint{4.578078in}{1.364121in}}%
\pgfpathmoveto{\pgfqpoint{4.573537in}{1.367070in}}%
\pgfpathlineto{\pgfqpoint{4.573537in}{1.367070in}}%
\pgfpathlineto{\pgfqpoint{4.573537in}{1.370019in}}%
\pgfpathlineto{\pgfqpoint{4.578078in}{1.370019in}}%
\pgfpathlineto{\pgfqpoint{4.578078in}{1.367070in}}%
\pgfpathmoveto{\pgfqpoint{4.578078in}{1.367070in}}%
\pgfpathlineto{\pgfqpoint{4.578078in}{1.367070in}}%
\pgfpathlineto{\pgfqpoint{4.578078in}{1.370019in}}%
\pgfpathlineto{\pgfqpoint{4.582619in}{1.370019in}}%
\pgfpathlineto{\pgfqpoint{4.582619in}{1.367070in}}%
\pgfpathmoveto{\pgfqpoint{4.578078in}{1.370019in}}%
\pgfpathlineto{\pgfqpoint{4.578078in}{1.370019in}}%
\pgfpathlineto{\pgfqpoint{4.578078in}{1.372968in}}%
\pgfpathlineto{\pgfqpoint{4.582619in}{1.372968in}}%
\pgfpathlineto{\pgfqpoint{4.582619in}{1.370019in}}%
\pgfpathmoveto{\pgfqpoint{4.578078in}{1.372968in}}%
\pgfpathlineto{\pgfqpoint{4.578078in}{1.372968in}}%
\pgfpathlineto{\pgfqpoint{4.578078in}{1.375917in}}%
\pgfpathlineto{\pgfqpoint{4.582619in}{1.375917in}}%
\pgfpathlineto{\pgfqpoint{4.582619in}{1.372968in}}%
\pgfpathmoveto{\pgfqpoint{4.582619in}{1.372968in}}%
\pgfpathlineto{\pgfqpoint{4.582619in}{1.372968in}}%
\pgfpathlineto{\pgfqpoint{4.582619in}{1.375917in}}%
\pgfpathlineto{\pgfqpoint{4.587160in}{1.375917in}}%
\pgfpathlineto{\pgfqpoint{4.587160in}{1.372968in}}%
\pgfpathmoveto{\pgfqpoint{4.582619in}{1.375917in}}%
\pgfpathlineto{\pgfqpoint{4.582619in}{1.375917in}}%
\pgfpathlineto{\pgfqpoint{4.582619in}{1.378867in}}%
\pgfpathlineto{\pgfqpoint{4.587160in}{1.378867in}}%
\pgfpathlineto{\pgfqpoint{4.587160in}{1.375917in}}%
\pgfpathmoveto{\pgfqpoint{4.587160in}{1.375917in}}%
\pgfpathlineto{\pgfqpoint{4.587160in}{1.375917in}}%
\pgfpathlineto{\pgfqpoint{4.587160in}{1.378867in}}%
\pgfpathlineto{\pgfqpoint{4.591701in}{1.378867in}}%
\pgfpathlineto{\pgfqpoint{4.591701in}{1.375917in}}%
\pgfpathmoveto{\pgfqpoint{4.587160in}{1.378867in}}%
\pgfpathlineto{\pgfqpoint{4.587160in}{1.378867in}}%
\pgfpathlineto{\pgfqpoint{4.587160in}{1.381816in}}%
\pgfpathlineto{\pgfqpoint{4.591701in}{1.381816in}}%
\pgfpathlineto{\pgfqpoint{4.591701in}{1.378867in}}%
\pgfpathmoveto{\pgfqpoint{4.591701in}{1.378867in}}%
\pgfpathlineto{\pgfqpoint{4.591701in}{1.378867in}}%
\pgfpathlineto{\pgfqpoint{4.591701in}{1.381816in}}%
\pgfpathlineto{\pgfqpoint{4.596242in}{1.381816in}}%
\pgfpathlineto{\pgfqpoint{4.596242in}{1.378867in}}%
\pgfpathmoveto{\pgfqpoint{4.591701in}{1.381816in}}%
\pgfpathlineto{\pgfqpoint{4.591701in}{1.381816in}}%
\pgfpathlineto{\pgfqpoint{4.591701in}{1.384765in}}%
\pgfpathlineto{\pgfqpoint{4.596242in}{1.384765in}}%
\pgfpathlineto{\pgfqpoint{4.596242in}{1.381816in}}%
\pgfpathmoveto{\pgfqpoint{4.596242in}{1.381816in}}%
\pgfpathlineto{\pgfqpoint{4.596242in}{1.381816in}}%
\pgfpathlineto{\pgfqpoint{4.596242in}{1.384765in}}%
\pgfpathlineto{\pgfqpoint{4.600782in}{1.384765in}}%
\pgfpathlineto{\pgfqpoint{4.600782in}{1.381816in}}%
\pgfpathmoveto{\pgfqpoint{4.596242in}{1.384765in}}%
\pgfpathlineto{\pgfqpoint{4.596242in}{1.384765in}}%
\pgfpathlineto{\pgfqpoint{4.596242in}{1.387714in}}%
\pgfpathlineto{\pgfqpoint{4.600782in}{1.387714in}}%
\pgfpathlineto{\pgfqpoint{4.600782in}{1.384765in}}%
\pgfpathmoveto{\pgfqpoint{4.600782in}{1.384765in}}%
\pgfpathlineto{\pgfqpoint{4.600782in}{1.384765in}}%
\pgfpathlineto{\pgfqpoint{4.600782in}{1.387714in}}%
\pgfpathlineto{\pgfqpoint{4.605323in}{1.387714in}}%
\pgfpathlineto{\pgfqpoint{4.605323in}{1.384765in}}%
\pgfpathmoveto{\pgfqpoint{4.600782in}{1.387714in}}%
\pgfpathlineto{\pgfqpoint{4.600782in}{1.387714in}}%
\pgfpathlineto{\pgfqpoint{4.600782in}{1.390663in}}%
\pgfpathlineto{\pgfqpoint{4.605323in}{1.390663in}}%
\pgfpathlineto{\pgfqpoint{4.605323in}{1.387714in}}%
\pgfpathmoveto{\pgfqpoint{4.600782in}{1.390663in}}%
\pgfpathlineto{\pgfqpoint{4.600782in}{1.390663in}}%
\pgfpathlineto{\pgfqpoint{4.600782in}{1.393612in}}%
\pgfpathlineto{\pgfqpoint{4.605323in}{1.393612in}}%
\pgfpathlineto{\pgfqpoint{4.605323in}{1.390663in}}%
\pgfpathmoveto{\pgfqpoint{4.605323in}{1.390663in}}%
\pgfpathlineto{\pgfqpoint{4.605323in}{1.390663in}}%
\pgfpathlineto{\pgfqpoint{4.605323in}{1.393612in}}%
\pgfpathlineto{\pgfqpoint{4.609864in}{1.393612in}}%
\pgfpathlineto{\pgfqpoint{4.609864in}{1.390663in}}%
\pgfpathmoveto{\pgfqpoint{4.605323in}{1.393612in}}%
\pgfpathlineto{\pgfqpoint{4.605323in}{1.393612in}}%
\pgfpathlineto{\pgfqpoint{4.605323in}{1.396561in}}%
\pgfpathlineto{\pgfqpoint{4.609864in}{1.396561in}}%
\pgfpathlineto{\pgfqpoint{4.609864in}{1.393612in}}%
\pgfpathmoveto{\pgfqpoint{4.609864in}{1.393612in}}%
\pgfpathlineto{\pgfqpoint{4.609864in}{1.393612in}}%
\pgfpathlineto{\pgfqpoint{4.609864in}{1.396561in}}%
\pgfpathlineto{\pgfqpoint{4.614405in}{1.396561in}}%
\pgfpathlineto{\pgfqpoint{4.614405in}{1.393612in}}%
\pgfpathmoveto{\pgfqpoint{4.609864in}{1.396561in}}%
\pgfpathlineto{\pgfqpoint{4.609864in}{1.396561in}}%
\pgfpathlineto{\pgfqpoint{4.609864in}{1.399510in}}%
\pgfpathlineto{\pgfqpoint{4.614405in}{1.399510in}}%
\pgfpathlineto{\pgfqpoint{4.614405in}{1.396561in}}%
\pgfpathmoveto{\pgfqpoint{4.614405in}{1.396561in}}%
\pgfpathlineto{\pgfqpoint{4.614405in}{1.396561in}}%
\pgfpathlineto{\pgfqpoint{4.614405in}{1.399510in}}%
\pgfpathlineto{\pgfqpoint{4.618946in}{1.399510in}}%
\pgfpathlineto{\pgfqpoint{4.618946in}{1.396561in}}%
\pgfpathmoveto{\pgfqpoint{4.614405in}{1.399510in}}%
\pgfpathlineto{\pgfqpoint{4.614405in}{1.399510in}}%
\pgfpathlineto{\pgfqpoint{4.614405in}{1.402460in}}%
\pgfpathlineto{\pgfqpoint{4.618946in}{1.402460in}}%
\pgfpathlineto{\pgfqpoint{4.618946in}{1.399510in}}%
\pgfpathmoveto{\pgfqpoint{4.618946in}{1.399510in}}%
\pgfpathlineto{\pgfqpoint{4.618946in}{1.399510in}}%
\pgfpathlineto{\pgfqpoint{4.618946in}{1.402460in}}%
\pgfpathlineto{\pgfqpoint{4.623487in}{1.402460in}}%
\pgfpathlineto{\pgfqpoint{4.623487in}{1.399510in}}%
\pgfpathmoveto{\pgfqpoint{4.618946in}{1.402460in}}%
\pgfpathlineto{\pgfqpoint{4.618946in}{1.402460in}}%
\pgfpathlineto{\pgfqpoint{4.618946in}{1.405409in}}%
\pgfpathlineto{\pgfqpoint{4.623487in}{1.405409in}}%
\pgfpathlineto{\pgfqpoint{4.623487in}{1.402460in}}%
\pgfpathmoveto{\pgfqpoint{4.623487in}{1.402460in}}%
\pgfpathlineto{\pgfqpoint{4.623487in}{1.402460in}}%
\pgfpathlineto{\pgfqpoint{4.623487in}{1.405409in}}%
\pgfpathlineto{\pgfqpoint{4.628028in}{1.405409in}}%
\pgfpathlineto{\pgfqpoint{4.628028in}{1.402460in}}%
\pgfpathmoveto{\pgfqpoint{4.623487in}{1.405409in}}%
\pgfpathlineto{\pgfqpoint{4.623487in}{1.405409in}}%
\pgfpathlineto{\pgfqpoint{4.623487in}{1.408358in}}%
\pgfpathlineto{\pgfqpoint{4.628028in}{1.408358in}}%
\pgfpathlineto{\pgfqpoint{4.628028in}{1.405409in}}%
\pgfpathmoveto{\pgfqpoint{4.623487in}{1.408358in}}%
\pgfpathlineto{\pgfqpoint{4.623487in}{1.408358in}}%
\pgfpathlineto{\pgfqpoint{4.623487in}{1.411307in}}%
\pgfpathlineto{\pgfqpoint{4.628028in}{1.411307in}}%
\pgfpathlineto{\pgfqpoint{4.628028in}{1.408358in}}%
\pgfpathmoveto{\pgfqpoint{4.628028in}{1.408358in}}%
\pgfpathlineto{\pgfqpoint{4.628028in}{1.408358in}}%
\pgfpathlineto{\pgfqpoint{4.628028in}{1.411307in}}%
\pgfpathlineto{\pgfqpoint{4.632569in}{1.411307in}}%
\pgfpathlineto{\pgfqpoint{4.632569in}{1.408358in}}%
\pgfpathmoveto{\pgfqpoint{4.628028in}{1.411307in}}%
\pgfpathlineto{\pgfqpoint{4.628028in}{1.411307in}}%
\pgfpathlineto{\pgfqpoint{4.628028in}{1.414256in}}%
\pgfpathlineto{\pgfqpoint{4.632569in}{1.414256in}}%
\pgfpathlineto{\pgfqpoint{4.632569in}{1.411307in}}%
\pgfpathmoveto{\pgfqpoint{4.632569in}{1.411307in}}%
\pgfpathlineto{\pgfqpoint{4.632569in}{1.411307in}}%
\pgfpathlineto{\pgfqpoint{4.632569in}{1.414256in}}%
\pgfpathlineto{\pgfqpoint{4.637110in}{1.414256in}}%
\pgfpathlineto{\pgfqpoint{4.637110in}{1.411307in}}%
\pgfpathmoveto{\pgfqpoint{4.632569in}{1.414256in}}%
\pgfpathlineto{\pgfqpoint{4.632569in}{1.414256in}}%
\pgfpathlineto{\pgfqpoint{4.632569in}{1.417205in}}%
\pgfpathlineto{\pgfqpoint{4.637110in}{1.417205in}}%
\pgfpathlineto{\pgfqpoint{4.637110in}{1.414256in}}%
\pgfpathmoveto{\pgfqpoint{4.637110in}{1.414256in}}%
\pgfpathlineto{\pgfqpoint{4.637110in}{1.414256in}}%
\pgfpathlineto{\pgfqpoint{4.637110in}{1.417205in}}%
\pgfpathlineto{\pgfqpoint{4.641651in}{1.417205in}}%
\pgfpathlineto{\pgfqpoint{4.641651in}{1.414256in}}%
\pgfpathmoveto{\pgfqpoint{4.637110in}{1.417205in}}%
\pgfpathlineto{\pgfqpoint{4.637110in}{1.417205in}}%
\pgfpathlineto{\pgfqpoint{4.637110in}{1.420154in}}%
\pgfpathlineto{\pgfqpoint{4.641651in}{1.420154in}}%
\pgfpathlineto{\pgfqpoint{4.641651in}{1.417205in}}%
\pgfpathmoveto{\pgfqpoint{4.641651in}{1.417205in}}%
\pgfpathlineto{\pgfqpoint{4.641651in}{1.417205in}}%
\pgfpathlineto{\pgfqpoint{4.641651in}{1.420154in}}%
\pgfpathlineto{\pgfqpoint{4.646191in}{1.420154in}}%
\pgfpathlineto{\pgfqpoint{4.646191in}{1.417205in}}%
\pgfpathmoveto{\pgfqpoint{4.641651in}{1.420154in}}%
\pgfpathlineto{\pgfqpoint{4.641651in}{1.420154in}}%
\pgfpathlineto{\pgfqpoint{4.641651in}{1.423103in}}%
\pgfpathlineto{\pgfqpoint{4.646191in}{1.423103in}}%
\pgfpathlineto{\pgfqpoint{4.646191in}{1.420154in}}%
\pgfpathmoveto{\pgfqpoint{4.646191in}{1.420154in}}%
\pgfpathlineto{\pgfqpoint{4.646191in}{1.420154in}}%
\pgfpathlineto{\pgfqpoint{4.646191in}{1.423103in}}%
\pgfpathlineto{\pgfqpoint{4.650732in}{1.423103in}}%
\pgfpathlineto{\pgfqpoint{4.650732in}{1.420154in}}%
\pgfpathmoveto{\pgfqpoint{4.646191in}{1.423103in}}%
\pgfpathlineto{\pgfqpoint{4.646191in}{1.423103in}}%
\pgfpathlineto{\pgfqpoint{4.646191in}{1.426053in}}%
\pgfpathlineto{\pgfqpoint{4.650732in}{1.426053in}}%
\pgfpathlineto{\pgfqpoint{4.650732in}{1.423103in}}%
\pgfpathmoveto{\pgfqpoint{4.646191in}{1.426053in}}%
\pgfpathlineto{\pgfqpoint{4.646191in}{1.426053in}}%
\pgfpathlineto{\pgfqpoint{4.646191in}{1.429002in}}%
\pgfpathlineto{\pgfqpoint{4.650732in}{1.429002in}}%
\pgfpathlineto{\pgfqpoint{4.650732in}{1.426053in}}%
\pgfpathmoveto{\pgfqpoint{4.650732in}{1.426053in}}%
\pgfpathlineto{\pgfqpoint{4.650732in}{1.426053in}}%
\pgfpathlineto{\pgfqpoint{4.650732in}{1.429002in}}%
\pgfpathlineto{\pgfqpoint{4.655273in}{1.429002in}}%
\pgfpathlineto{\pgfqpoint{4.655273in}{1.426053in}}%
\pgfpathmoveto{\pgfqpoint{4.650732in}{1.429002in}}%
\pgfpathlineto{\pgfqpoint{4.650732in}{1.429002in}}%
\pgfpathlineto{\pgfqpoint{4.650732in}{1.431951in}}%
\pgfpathlineto{\pgfqpoint{4.655273in}{1.431951in}}%
\pgfpathlineto{\pgfqpoint{4.655273in}{1.429002in}}%
\pgfpathmoveto{\pgfqpoint{4.655273in}{1.429002in}}%
\pgfpathlineto{\pgfqpoint{4.655273in}{1.429002in}}%
\pgfpathlineto{\pgfqpoint{4.655273in}{1.431951in}}%
\pgfpathlineto{\pgfqpoint{4.659814in}{1.431951in}}%
\pgfpathlineto{\pgfqpoint{4.659814in}{1.429002in}}%
\pgfpathmoveto{\pgfqpoint{4.655273in}{1.431951in}}%
\pgfpathlineto{\pgfqpoint{4.655273in}{1.431951in}}%
\pgfpathlineto{\pgfqpoint{4.655273in}{1.434900in}}%
\pgfpathlineto{\pgfqpoint{4.659814in}{1.434900in}}%
\pgfpathlineto{\pgfqpoint{4.659814in}{1.431951in}}%
\pgfpathmoveto{\pgfqpoint{4.659814in}{1.431951in}}%
\pgfpathlineto{\pgfqpoint{4.659814in}{1.431951in}}%
\pgfpathlineto{\pgfqpoint{4.659814in}{1.434900in}}%
\pgfpathlineto{\pgfqpoint{4.664355in}{1.434900in}}%
\pgfpathlineto{\pgfqpoint{4.664355in}{1.431951in}}%
\pgfpathmoveto{\pgfqpoint{4.659814in}{1.434900in}}%
\pgfpathlineto{\pgfqpoint{4.659814in}{1.434900in}}%
\pgfpathlineto{\pgfqpoint{4.659814in}{1.437849in}}%
\pgfpathlineto{\pgfqpoint{4.664355in}{1.437849in}}%
\pgfpathlineto{\pgfqpoint{4.664355in}{1.434900in}}%
\pgfpathmoveto{\pgfqpoint{4.664355in}{1.434900in}}%
\pgfpathlineto{\pgfqpoint{4.664355in}{1.434900in}}%
\pgfpathlineto{\pgfqpoint{4.664355in}{1.437849in}}%
\pgfpathlineto{\pgfqpoint{4.668896in}{1.437849in}}%
\pgfpathlineto{\pgfqpoint{4.668896in}{1.434900in}}%
\pgfpathmoveto{\pgfqpoint{4.664355in}{1.437849in}}%
\pgfpathlineto{\pgfqpoint{4.664355in}{1.437849in}}%
\pgfpathlineto{\pgfqpoint{4.664355in}{1.440798in}}%
\pgfpathlineto{\pgfqpoint{4.668896in}{1.440798in}}%
\pgfpathlineto{\pgfqpoint{4.668896in}{1.437849in}}%
\pgfpathmoveto{\pgfqpoint{4.668896in}{1.437849in}}%
\pgfpathlineto{\pgfqpoint{4.668896in}{1.437849in}}%
\pgfpathlineto{\pgfqpoint{4.668896in}{1.440798in}}%
\pgfpathlineto{\pgfqpoint{4.673437in}{1.440798in}}%
\pgfpathlineto{\pgfqpoint{4.673437in}{1.437849in}}%
\pgfpathmoveto{\pgfqpoint{4.668896in}{1.440798in}}%
\pgfpathlineto{\pgfqpoint{4.668896in}{1.440798in}}%
\pgfpathlineto{\pgfqpoint{4.668896in}{1.443747in}}%
\pgfpathlineto{\pgfqpoint{4.673437in}{1.443747in}}%
\pgfpathlineto{\pgfqpoint{4.673437in}{1.440798in}}%
\pgfpathmoveto{\pgfqpoint{4.668896in}{1.443747in}}%
\pgfpathlineto{\pgfqpoint{4.668896in}{1.443747in}}%
\pgfpathlineto{\pgfqpoint{4.668896in}{1.446697in}}%
\pgfpathlineto{\pgfqpoint{4.673437in}{1.446697in}}%
\pgfpathlineto{\pgfqpoint{4.673437in}{1.443747in}}%
\pgfpathmoveto{\pgfqpoint{4.673437in}{1.443747in}}%
\pgfpathlineto{\pgfqpoint{4.673437in}{1.443747in}}%
\pgfpathlineto{\pgfqpoint{4.673437in}{1.446697in}}%
\pgfpathlineto{\pgfqpoint{4.677978in}{1.446697in}}%
\pgfpathlineto{\pgfqpoint{4.677978in}{1.443747in}}%
\pgfpathmoveto{\pgfqpoint{4.673437in}{1.446697in}}%
\pgfpathlineto{\pgfqpoint{4.673437in}{1.446697in}}%
\pgfpathlineto{\pgfqpoint{4.673437in}{1.449646in}}%
\pgfpathlineto{\pgfqpoint{4.677978in}{1.449646in}}%
\pgfpathlineto{\pgfqpoint{4.677978in}{1.446697in}}%
\pgfpathmoveto{\pgfqpoint{4.677978in}{1.446697in}}%
\pgfpathlineto{\pgfqpoint{4.677978in}{1.446697in}}%
\pgfpathlineto{\pgfqpoint{4.677978in}{1.449646in}}%
\pgfpathlineto{\pgfqpoint{4.682519in}{1.449646in}}%
\pgfpathlineto{\pgfqpoint{4.682519in}{1.446697in}}%
\pgfpathmoveto{\pgfqpoint{4.677978in}{1.449646in}}%
\pgfpathlineto{\pgfqpoint{4.677978in}{1.449646in}}%
\pgfpathlineto{\pgfqpoint{4.677978in}{1.452595in}}%
\pgfpathlineto{\pgfqpoint{4.682519in}{1.452595in}}%
\pgfpathlineto{\pgfqpoint{4.682519in}{1.449646in}}%
\pgfpathmoveto{\pgfqpoint{4.682519in}{1.449646in}}%
\pgfpathlineto{\pgfqpoint{4.682519in}{1.449646in}}%
\pgfpathlineto{\pgfqpoint{4.682519in}{1.452595in}}%
\pgfpathlineto{\pgfqpoint{4.687060in}{1.452595in}}%
\pgfpathlineto{\pgfqpoint{4.687060in}{1.449646in}}%
\pgfpathmoveto{\pgfqpoint{4.682519in}{1.452595in}}%
\pgfpathlineto{\pgfqpoint{4.682519in}{1.452595in}}%
\pgfpathlineto{\pgfqpoint{4.682519in}{1.455544in}}%
\pgfpathlineto{\pgfqpoint{4.687060in}{1.455544in}}%
\pgfpathlineto{\pgfqpoint{4.687060in}{1.452595in}}%
\pgfpathmoveto{\pgfqpoint{4.687060in}{1.452595in}}%
\pgfpathlineto{\pgfqpoint{4.687060in}{1.452595in}}%
\pgfpathlineto{\pgfqpoint{4.687060in}{1.455544in}}%
\pgfpathlineto{\pgfqpoint{4.691601in}{1.455544in}}%
\pgfpathlineto{\pgfqpoint{4.691601in}{1.452595in}}%
\pgfpathmoveto{\pgfqpoint{4.687060in}{1.455544in}}%
\pgfpathlineto{\pgfqpoint{4.687060in}{1.455544in}}%
\pgfpathlineto{\pgfqpoint{4.687060in}{1.458494in}}%
\pgfpathlineto{\pgfqpoint{4.691601in}{1.458494in}}%
\pgfpathlineto{\pgfqpoint{4.691601in}{1.455544in}}%
\pgfpathmoveto{\pgfqpoint{4.691601in}{1.455544in}}%
\pgfpathlineto{\pgfqpoint{4.691601in}{1.455544in}}%
\pgfpathlineto{\pgfqpoint{4.691601in}{1.458494in}}%
\pgfpathlineto{\pgfqpoint{4.696142in}{1.458494in}}%
\pgfpathlineto{\pgfqpoint{4.696142in}{1.455544in}}%
\pgfpathmoveto{\pgfqpoint{4.691601in}{1.458494in}}%
\pgfpathlineto{\pgfqpoint{4.691601in}{1.458494in}}%
\pgfpathlineto{\pgfqpoint{4.691601in}{1.461443in}}%
\pgfpathlineto{\pgfqpoint{4.696142in}{1.461443in}}%
\pgfpathlineto{\pgfqpoint{4.696142in}{1.458494in}}%
\pgfpathmoveto{\pgfqpoint{4.691601in}{1.461443in}}%
\pgfpathlineto{\pgfqpoint{4.691601in}{1.461443in}}%
\pgfpathlineto{\pgfqpoint{4.691601in}{1.464392in}}%
\pgfpathlineto{\pgfqpoint{4.696142in}{1.464392in}}%
\pgfpathlineto{\pgfqpoint{4.696142in}{1.461443in}}%
\pgfpathmoveto{\pgfqpoint{4.696142in}{1.461443in}}%
\pgfpathlineto{\pgfqpoint{4.696142in}{1.461443in}}%
\pgfpathlineto{\pgfqpoint{4.696142in}{1.464392in}}%
\pgfpathlineto{\pgfqpoint{4.700682in}{1.464392in}}%
\pgfpathlineto{\pgfqpoint{4.700682in}{1.461443in}}%
\pgfpathmoveto{\pgfqpoint{4.696142in}{1.464392in}}%
\pgfpathlineto{\pgfqpoint{4.696142in}{1.464392in}}%
\pgfpathlineto{\pgfqpoint{4.696142in}{1.467341in}}%
\pgfpathlineto{\pgfqpoint{4.700682in}{1.467341in}}%
\pgfpathlineto{\pgfqpoint{4.700682in}{1.464392in}}%
\pgfpathmoveto{\pgfqpoint{4.700682in}{1.464392in}}%
\pgfpathlineto{\pgfqpoint{4.700682in}{1.464392in}}%
\pgfpathlineto{\pgfqpoint{4.700682in}{1.467341in}}%
\pgfpathlineto{\pgfqpoint{4.705223in}{1.467341in}}%
\pgfpathlineto{\pgfqpoint{4.705223in}{1.464392in}}%
\pgfpathmoveto{\pgfqpoint{4.700682in}{1.467341in}}%
\pgfpathlineto{\pgfqpoint{4.700682in}{1.467341in}}%
\pgfpathlineto{\pgfqpoint{4.700682in}{1.470291in}}%
\pgfpathlineto{\pgfqpoint{4.705223in}{1.470291in}}%
\pgfpathlineto{\pgfqpoint{4.705223in}{1.467341in}}%
\pgfpathmoveto{\pgfqpoint{4.705223in}{1.467341in}}%
\pgfpathlineto{\pgfqpoint{4.705223in}{1.467341in}}%
\pgfpathlineto{\pgfqpoint{4.705223in}{1.470291in}}%
\pgfpathlineto{\pgfqpoint{4.709764in}{1.470291in}}%
\pgfpathlineto{\pgfqpoint{4.709764in}{1.467341in}}%
\pgfpathmoveto{\pgfqpoint{4.705223in}{1.470291in}}%
\pgfpathlineto{\pgfqpoint{4.705223in}{1.470291in}}%
\pgfpathlineto{\pgfqpoint{4.705223in}{1.473240in}}%
\pgfpathlineto{\pgfqpoint{4.709764in}{1.473240in}}%
\pgfpathlineto{\pgfqpoint{4.709764in}{1.470291in}}%
\pgfpathmoveto{\pgfqpoint{4.709764in}{1.470291in}}%
\pgfpathlineto{\pgfqpoint{4.709764in}{1.470291in}}%
\pgfpathlineto{\pgfqpoint{4.709764in}{1.473240in}}%
\pgfpathlineto{\pgfqpoint{4.714305in}{1.473240in}}%
\pgfpathlineto{\pgfqpoint{4.714305in}{1.470291in}}%
\pgfpathmoveto{\pgfqpoint{4.709764in}{1.473240in}}%
\pgfpathlineto{\pgfqpoint{4.709764in}{1.473240in}}%
\pgfpathlineto{\pgfqpoint{4.709764in}{1.476189in}}%
\pgfpathlineto{\pgfqpoint{4.714305in}{1.476189in}}%
\pgfpathlineto{\pgfqpoint{4.714305in}{1.473240in}}%
\pgfpathmoveto{\pgfqpoint{4.714305in}{1.473240in}}%
\pgfpathlineto{\pgfqpoint{4.714305in}{1.473240in}}%
\pgfpathlineto{\pgfqpoint{4.714305in}{1.476189in}}%
\pgfpathlineto{\pgfqpoint{4.718846in}{1.476189in}}%
\pgfpathlineto{\pgfqpoint{4.718846in}{1.473240in}}%
\pgfpathmoveto{\pgfqpoint{4.714305in}{1.476189in}}%
\pgfpathlineto{\pgfqpoint{4.714305in}{1.476189in}}%
\pgfpathlineto{\pgfqpoint{4.714305in}{1.479138in}}%
\pgfpathlineto{\pgfqpoint{4.718846in}{1.479138in}}%
\pgfpathlineto{\pgfqpoint{4.718846in}{1.476189in}}%
\pgfpathmoveto{\pgfqpoint{4.714305in}{1.479138in}}%
\pgfpathlineto{\pgfqpoint{4.714305in}{1.479138in}}%
\pgfpathlineto{\pgfqpoint{4.714305in}{1.482088in}}%
\pgfpathlineto{\pgfqpoint{4.718846in}{1.482088in}}%
\pgfpathlineto{\pgfqpoint{4.718846in}{1.479138in}}%
\pgfpathmoveto{\pgfqpoint{4.718846in}{1.479138in}}%
\pgfpathlineto{\pgfqpoint{4.718846in}{1.479138in}}%
\pgfpathlineto{\pgfqpoint{4.718846in}{1.482088in}}%
\pgfpathlineto{\pgfqpoint{4.723387in}{1.482088in}}%
\pgfpathlineto{\pgfqpoint{4.723387in}{1.479138in}}%
\pgfpathmoveto{\pgfqpoint{4.718846in}{1.482088in}}%
\pgfpathlineto{\pgfqpoint{4.718846in}{1.482088in}}%
\pgfpathlineto{\pgfqpoint{4.718846in}{1.485037in}}%
\pgfpathlineto{\pgfqpoint{4.723387in}{1.485037in}}%
\pgfpathlineto{\pgfqpoint{4.723387in}{1.482088in}}%
\pgfpathmoveto{\pgfqpoint{4.723387in}{1.482088in}}%
\pgfpathlineto{\pgfqpoint{4.723387in}{1.482088in}}%
\pgfpathlineto{\pgfqpoint{4.723387in}{1.485037in}}%
\pgfpathlineto{\pgfqpoint{4.727928in}{1.485037in}}%
\pgfpathlineto{\pgfqpoint{4.727928in}{1.482088in}}%
\pgfpathmoveto{\pgfqpoint{4.723387in}{1.485037in}}%
\pgfpathlineto{\pgfqpoint{4.723387in}{1.485037in}}%
\pgfpathlineto{\pgfqpoint{4.723387in}{1.487986in}}%
\pgfpathlineto{\pgfqpoint{4.727928in}{1.487986in}}%
\pgfpathlineto{\pgfqpoint{4.727928in}{1.485037in}}%
\pgfpathmoveto{\pgfqpoint{4.727928in}{1.485037in}}%
\pgfpathlineto{\pgfqpoint{4.727928in}{1.485037in}}%
\pgfpathlineto{\pgfqpoint{4.727928in}{1.487986in}}%
\pgfpathlineto{\pgfqpoint{4.732469in}{1.487986in}}%
\pgfpathlineto{\pgfqpoint{4.732469in}{1.485037in}}%
\pgfpathmoveto{\pgfqpoint{4.727928in}{1.487986in}}%
\pgfpathlineto{\pgfqpoint{4.727928in}{1.487986in}}%
\pgfpathlineto{\pgfqpoint{4.727928in}{1.490936in}}%
\pgfpathlineto{\pgfqpoint{4.732469in}{1.490936in}}%
\pgfpathlineto{\pgfqpoint{4.732469in}{1.487986in}}%
\pgfpathmoveto{\pgfqpoint{4.732469in}{1.487986in}}%
\pgfpathlineto{\pgfqpoint{4.732469in}{1.487986in}}%
\pgfpathlineto{\pgfqpoint{4.732469in}{1.490936in}}%
\pgfpathlineto{\pgfqpoint{4.737010in}{1.490936in}}%
\pgfpathlineto{\pgfqpoint{4.737010in}{1.487986in}}%
\pgfpathmoveto{\pgfqpoint{4.732469in}{1.490936in}}%
\pgfpathlineto{\pgfqpoint{4.732469in}{1.490936in}}%
\pgfpathlineto{\pgfqpoint{4.732469in}{1.493885in}}%
\pgfpathlineto{\pgfqpoint{4.737010in}{1.493885in}}%
\pgfpathlineto{\pgfqpoint{4.737010in}{1.490936in}}%
\pgfpathmoveto{\pgfqpoint{4.737010in}{1.490936in}}%
\pgfpathlineto{\pgfqpoint{4.737010in}{1.490936in}}%
\pgfpathlineto{\pgfqpoint{4.737010in}{1.493885in}}%
\pgfpathlineto{\pgfqpoint{4.741551in}{1.493885in}}%
\pgfpathlineto{\pgfqpoint{4.741551in}{1.490936in}}%
\pgfpathmoveto{\pgfqpoint{4.737010in}{1.493885in}}%
\pgfpathlineto{\pgfqpoint{4.737010in}{1.493885in}}%
\pgfpathlineto{\pgfqpoint{4.737010in}{1.496834in}}%
\pgfpathlineto{\pgfqpoint{4.741551in}{1.496834in}}%
\pgfpathlineto{\pgfqpoint{4.741551in}{1.493885in}}%
\pgfpathmoveto{\pgfqpoint{4.737010in}{1.496834in}}%
\pgfpathlineto{\pgfqpoint{4.737010in}{1.496834in}}%
\pgfpathlineto{\pgfqpoint{4.737010in}{1.499783in}}%
\pgfpathlineto{\pgfqpoint{4.741551in}{1.499783in}}%
\pgfpathlineto{\pgfqpoint{4.741551in}{1.496834in}}%
\pgfpathmoveto{\pgfqpoint{4.741551in}{1.496834in}}%
\pgfpathlineto{\pgfqpoint{4.741551in}{1.496834in}}%
\pgfpathlineto{\pgfqpoint{4.741551in}{1.499783in}}%
\pgfpathlineto{\pgfqpoint{4.746092in}{1.499783in}}%
\pgfpathlineto{\pgfqpoint{4.746092in}{1.496834in}}%
\pgfpathmoveto{\pgfqpoint{4.741551in}{1.499783in}}%
\pgfpathlineto{\pgfqpoint{4.741551in}{1.499783in}}%
\pgfpathlineto{\pgfqpoint{4.741551in}{1.502733in}}%
\pgfpathlineto{\pgfqpoint{4.746092in}{1.502733in}}%
\pgfpathlineto{\pgfqpoint{4.746092in}{1.499783in}}%
\pgfpathmoveto{\pgfqpoint{4.746092in}{1.499783in}}%
\pgfpathlineto{\pgfqpoint{4.746092in}{1.499783in}}%
\pgfpathlineto{\pgfqpoint{4.746092in}{1.502733in}}%
\pgfpathlineto{\pgfqpoint{4.750633in}{1.502733in}}%
\pgfpathlineto{\pgfqpoint{4.750633in}{1.499783in}}%
\pgfpathmoveto{\pgfqpoint{4.746092in}{1.502733in}}%
\pgfpathlineto{\pgfqpoint{4.746092in}{1.502733in}}%
\pgfpathlineto{\pgfqpoint{4.746092in}{1.505682in}}%
\pgfpathlineto{\pgfqpoint{4.750633in}{1.505682in}}%
\pgfpathlineto{\pgfqpoint{4.750633in}{1.502733in}}%
\pgfpathmoveto{\pgfqpoint{4.750633in}{1.502733in}}%
\pgfpathlineto{\pgfqpoint{4.750633in}{1.502733in}}%
\pgfpathlineto{\pgfqpoint{4.750633in}{1.505682in}}%
\pgfpathlineto{\pgfqpoint{4.755174in}{1.505682in}}%
\pgfpathlineto{\pgfqpoint{4.755174in}{1.502733in}}%
\pgfpathmoveto{\pgfqpoint{4.750633in}{1.505682in}}%
\pgfpathlineto{\pgfqpoint{4.750633in}{1.505682in}}%
\pgfpathlineto{\pgfqpoint{4.750633in}{1.508631in}}%
\pgfpathlineto{\pgfqpoint{4.755174in}{1.508631in}}%
\pgfpathlineto{\pgfqpoint{4.755174in}{1.505682in}}%
\pgfpathmoveto{\pgfqpoint{4.755174in}{1.505682in}}%
\pgfpathlineto{\pgfqpoint{4.755174in}{1.505682in}}%
\pgfpathlineto{\pgfqpoint{4.755174in}{1.508631in}}%
\pgfpathlineto{\pgfqpoint{4.759715in}{1.508631in}}%
\pgfpathlineto{\pgfqpoint{4.759715in}{1.505682in}}%
\pgfpathmoveto{\pgfqpoint{4.755174in}{1.508631in}}%
\pgfpathlineto{\pgfqpoint{4.755174in}{1.508631in}}%
\pgfpathlineto{\pgfqpoint{4.755174in}{1.511580in}}%
\pgfpathlineto{\pgfqpoint{4.759715in}{1.511580in}}%
\pgfpathlineto{\pgfqpoint{4.759715in}{1.508631in}}%
\pgfpathmoveto{\pgfqpoint{4.759715in}{1.508631in}}%
\pgfpathlineto{\pgfqpoint{4.759715in}{1.508631in}}%
\pgfpathlineto{\pgfqpoint{4.759715in}{1.511580in}}%
\pgfpathlineto{\pgfqpoint{4.764256in}{1.511580in}}%
\pgfpathlineto{\pgfqpoint{4.764256in}{1.508631in}}%
\pgfpathmoveto{\pgfqpoint{4.759715in}{1.511580in}}%
\pgfpathlineto{\pgfqpoint{4.759715in}{1.511580in}}%
\pgfpathlineto{\pgfqpoint{4.759715in}{1.514530in}}%
\pgfpathlineto{\pgfqpoint{4.764256in}{1.514530in}}%
\pgfpathlineto{\pgfqpoint{4.764256in}{1.511580in}}%
\pgfpathmoveto{\pgfqpoint{4.759715in}{1.514530in}}%
\pgfpathlineto{\pgfqpoint{4.759715in}{1.514530in}}%
\pgfpathlineto{\pgfqpoint{4.759715in}{1.517479in}}%
\pgfpathlineto{\pgfqpoint{4.764256in}{1.517479in}}%
\pgfpathlineto{\pgfqpoint{4.764256in}{1.514530in}}%
\pgfpathmoveto{\pgfqpoint{4.764256in}{1.514530in}}%
\pgfpathlineto{\pgfqpoint{4.764256in}{1.514530in}}%
\pgfpathlineto{\pgfqpoint{4.764256in}{1.517479in}}%
\pgfpathlineto{\pgfqpoint{4.768797in}{1.517479in}}%
\pgfpathlineto{\pgfqpoint{4.768797in}{1.514530in}}%
\pgfpathmoveto{\pgfqpoint{4.764256in}{1.517479in}}%
\pgfpathlineto{\pgfqpoint{4.764256in}{1.517479in}}%
\pgfpathlineto{\pgfqpoint{4.764256in}{1.520428in}}%
\pgfpathlineto{\pgfqpoint{4.768797in}{1.520428in}}%
\pgfpathlineto{\pgfqpoint{4.768797in}{1.517479in}}%
\pgfpathmoveto{\pgfqpoint{4.768797in}{1.517479in}}%
\pgfpathlineto{\pgfqpoint{4.768797in}{1.517479in}}%
\pgfpathlineto{\pgfqpoint{4.768797in}{1.520428in}}%
\pgfpathlineto{\pgfqpoint{4.773338in}{1.520428in}}%
\pgfpathlineto{\pgfqpoint{4.773338in}{1.517479in}}%
\pgfpathmoveto{\pgfqpoint{4.768797in}{1.520428in}}%
\pgfpathlineto{\pgfqpoint{4.768797in}{1.520428in}}%
\pgfpathlineto{\pgfqpoint{4.768797in}{1.523377in}}%
\pgfpathlineto{\pgfqpoint{4.773338in}{1.523377in}}%
\pgfpathlineto{\pgfqpoint{4.773338in}{1.520428in}}%
\pgfpathmoveto{\pgfqpoint{4.773338in}{1.520428in}}%
\pgfpathlineto{\pgfqpoint{4.773338in}{1.520428in}}%
\pgfpathlineto{\pgfqpoint{4.773338in}{1.523377in}}%
\pgfpathlineto{\pgfqpoint{4.777879in}{1.523377in}}%
\pgfpathlineto{\pgfqpoint{4.777879in}{1.520428in}}%
\pgfpathmoveto{\pgfqpoint{4.773338in}{1.523377in}}%
\pgfpathlineto{\pgfqpoint{4.773338in}{1.523377in}}%
\pgfpathlineto{\pgfqpoint{4.773338in}{1.526327in}}%
\pgfpathlineto{\pgfqpoint{4.777879in}{1.526327in}}%
\pgfpathlineto{\pgfqpoint{4.777879in}{1.523377in}}%
\pgfpathmoveto{\pgfqpoint{4.777879in}{1.523377in}}%
\pgfpathlineto{\pgfqpoint{4.777879in}{1.523377in}}%
\pgfpathlineto{\pgfqpoint{4.777879in}{1.526327in}}%
\pgfpathlineto{\pgfqpoint{4.782420in}{1.526327in}}%
\pgfpathlineto{\pgfqpoint{4.782420in}{1.523377in}}%
\pgfpathmoveto{\pgfqpoint{4.777879in}{1.526327in}}%
\pgfpathlineto{\pgfqpoint{4.777879in}{1.526327in}}%
\pgfpathlineto{\pgfqpoint{4.777879in}{1.529276in}}%
\pgfpathlineto{\pgfqpoint{4.782420in}{1.529276in}}%
\pgfpathlineto{\pgfqpoint{4.782420in}{1.526327in}}%
\pgfpathmoveto{\pgfqpoint{4.782420in}{1.526327in}}%
\pgfpathlineto{\pgfqpoint{4.782420in}{1.526327in}}%
\pgfpathlineto{\pgfqpoint{4.782420in}{1.529276in}}%
\pgfpathlineto{\pgfqpoint{4.786960in}{1.529276in}}%
\pgfpathlineto{\pgfqpoint{4.786960in}{1.526327in}}%
\pgfpathmoveto{\pgfqpoint{4.782420in}{1.529276in}}%
\pgfpathlineto{\pgfqpoint{4.782420in}{1.529276in}}%
\pgfpathlineto{\pgfqpoint{4.782420in}{1.532225in}}%
\pgfpathlineto{\pgfqpoint{4.786960in}{1.532225in}}%
\pgfpathlineto{\pgfqpoint{4.786960in}{1.529276in}}%
\pgfpathmoveto{\pgfqpoint{4.786960in}{1.529276in}}%
\pgfpathlineto{\pgfqpoint{4.786960in}{1.529276in}}%
\pgfpathlineto{\pgfqpoint{4.786960in}{1.532225in}}%
\pgfpathlineto{\pgfqpoint{4.791501in}{1.532225in}}%
\pgfpathlineto{\pgfqpoint{4.791501in}{1.529276in}}%
\pgfpathmoveto{\pgfqpoint{4.786960in}{1.532225in}}%
\pgfpathlineto{\pgfqpoint{4.786960in}{1.532225in}}%
\pgfpathlineto{\pgfqpoint{4.786960in}{1.535174in}}%
\pgfpathlineto{\pgfqpoint{4.791501in}{1.535174in}}%
\pgfpathlineto{\pgfqpoint{4.791501in}{1.532225in}}%
\pgfpathmoveto{\pgfqpoint{4.786960in}{1.535174in}}%
\pgfpathlineto{\pgfqpoint{4.786960in}{1.535174in}}%
\pgfpathlineto{\pgfqpoint{4.786960in}{1.538124in}}%
\pgfpathlineto{\pgfqpoint{4.791501in}{1.538124in}}%
\pgfpathlineto{\pgfqpoint{4.791501in}{1.535174in}}%
\pgfpathmoveto{\pgfqpoint{4.791501in}{1.535174in}}%
\pgfpathlineto{\pgfqpoint{4.791501in}{1.535174in}}%
\pgfpathlineto{\pgfqpoint{4.791501in}{1.538124in}}%
\pgfpathlineto{\pgfqpoint{4.796042in}{1.538124in}}%
\pgfpathlineto{\pgfqpoint{4.796042in}{1.535174in}}%
\pgfpathmoveto{\pgfqpoint{4.791501in}{1.538124in}}%
\pgfpathlineto{\pgfqpoint{4.791501in}{1.538124in}}%
\pgfpathlineto{\pgfqpoint{4.791501in}{1.541073in}}%
\pgfpathlineto{\pgfqpoint{4.796042in}{1.541073in}}%
\pgfpathlineto{\pgfqpoint{4.796042in}{1.538124in}}%
\pgfpathmoveto{\pgfqpoint{4.796042in}{1.538124in}}%
\pgfpathlineto{\pgfqpoint{4.796042in}{1.538124in}}%
\pgfpathlineto{\pgfqpoint{4.796042in}{1.541073in}}%
\pgfpathlineto{\pgfqpoint{4.800583in}{1.541073in}}%
\pgfpathlineto{\pgfqpoint{4.800583in}{1.538124in}}%
\pgfpathmoveto{\pgfqpoint{4.796042in}{1.541073in}}%
\pgfpathlineto{\pgfqpoint{4.796042in}{1.541073in}}%
\pgfpathlineto{\pgfqpoint{4.796042in}{1.544022in}}%
\pgfpathlineto{\pgfqpoint{4.800583in}{1.544022in}}%
\pgfpathlineto{\pgfqpoint{4.800583in}{1.541073in}}%
\pgfpathmoveto{\pgfqpoint{4.800583in}{1.541073in}}%
\pgfpathlineto{\pgfqpoint{4.800583in}{1.541073in}}%
\pgfpathlineto{\pgfqpoint{4.800583in}{1.544022in}}%
\pgfpathlineto{\pgfqpoint{4.805124in}{1.544022in}}%
\pgfpathlineto{\pgfqpoint{4.805124in}{1.541073in}}%
\pgfpathmoveto{\pgfqpoint{4.800583in}{1.544022in}}%
\pgfpathlineto{\pgfqpoint{4.800583in}{1.544022in}}%
\pgfpathlineto{\pgfqpoint{4.800583in}{1.546971in}}%
\pgfpathlineto{\pgfqpoint{4.805124in}{1.546971in}}%
\pgfpathlineto{\pgfqpoint{4.805124in}{1.544022in}}%
\pgfpathmoveto{\pgfqpoint{4.805124in}{1.544022in}}%
\pgfpathlineto{\pgfqpoint{4.805124in}{1.544022in}}%
\pgfpathlineto{\pgfqpoint{4.805124in}{1.546971in}}%
\pgfpathlineto{\pgfqpoint{4.809665in}{1.546971in}}%
\pgfpathlineto{\pgfqpoint{4.809665in}{1.544022in}}%
\pgfpathmoveto{\pgfqpoint{4.805124in}{1.546971in}}%
\pgfpathlineto{\pgfqpoint{4.805124in}{1.546971in}}%
\pgfpathlineto{\pgfqpoint{4.805124in}{1.549920in}}%
\pgfpathlineto{\pgfqpoint{4.809665in}{1.549920in}}%
\pgfpathlineto{\pgfqpoint{4.809665in}{1.546971in}}%
\pgfpathmoveto{\pgfqpoint{4.809665in}{1.546971in}}%
\pgfpathlineto{\pgfqpoint{4.809665in}{1.546971in}}%
\pgfpathlineto{\pgfqpoint{4.809665in}{1.549920in}}%
\pgfpathlineto{\pgfqpoint{4.814206in}{1.549920in}}%
\pgfpathlineto{\pgfqpoint{4.814206in}{1.546971in}}%
\pgfpathmoveto{\pgfqpoint{4.809665in}{1.549920in}}%
\pgfpathlineto{\pgfqpoint{4.809665in}{1.549920in}}%
\pgfpathlineto{\pgfqpoint{4.809665in}{1.552870in}}%
\pgfpathlineto{\pgfqpoint{4.814206in}{1.552870in}}%
\pgfpathlineto{\pgfqpoint{4.814206in}{1.549920in}}%
\pgfpathmoveto{\pgfqpoint{4.809665in}{1.552870in}}%
\pgfpathlineto{\pgfqpoint{4.809665in}{1.552870in}}%
\pgfpathlineto{\pgfqpoint{4.809665in}{1.555819in}}%
\pgfpathlineto{\pgfqpoint{4.814206in}{1.555819in}}%
\pgfpathlineto{\pgfqpoint{4.814206in}{1.552870in}}%
\pgfpathmoveto{\pgfqpoint{4.814206in}{1.552870in}}%
\pgfpathlineto{\pgfqpoint{4.814206in}{1.552870in}}%
\pgfpathlineto{\pgfqpoint{4.814206in}{1.555819in}}%
\pgfpathlineto{\pgfqpoint{4.818747in}{1.555819in}}%
\pgfpathlineto{\pgfqpoint{4.818747in}{1.552870in}}%
\pgfpathmoveto{\pgfqpoint{4.814206in}{1.555819in}}%
\pgfpathlineto{\pgfqpoint{4.814206in}{1.555819in}}%
\pgfpathlineto{\pgfqpoint{4.814206in}{1.558768in}}%
\pgfpathlineto{\pgfqpoint{4.818747in}{1.558768in}}%
\pgfpathlineto{\pgfqpoint{4.818747in}{1.555819in}}%
\pgfpathmoveto{\pgfqpoint{4.818747in}{1.555819in}}%
\pgfpathlineto{\pgfqpoint{4.818747in}{1.555819in}}%
\pgfpathlineto{\pgfqpoint{4.818747in}{1.558768in}}%
\pgfpathlineto{\pgfqpoint{4.823288in}{1.558768in}}%
\pgfpathlineto{\pgfqpoint{4.823288in}{1.555819in}}%
\pgfpathmoveto{\pgfqpoint{4.818747in}{1.558768in}}%
\pgfpathlineto{\pgfqpoint{4.818747in}{1.558768in}}%
\pgfpathlineto{\pgfqpoint{4.818747in}{1.561717in}}%
\pgfpathlineto{\pgfqpoint{4.823288in}{1.561717in}}%
\pgfpathlineto{\pgfqpoint{4.823288in}{1.558768in}}%
\pgfpathmoveto{\pgfqpoint{4.823288in}{1.558768in}}%
\pgfpathlineto{\pgfqpoint{4.823288in}{1.558768in}}%
\pgfpathlineto{\pgfqpoint{4.823288in}{1.561717in}}%
\pgfpathlineto{\pgfqpoint{4.827829in}{1.561717in}}%
\pgfpathlineto{\pgfqpoint{4.827829in}{1.558768in}}%
\pgfpathmoveto{\pgfqpoint{4.823288in}{1.561717in}}%
\pgfpathlineto{\pgfqpoint{4.823288in}{1.561717in}}%
\pgfpathlineto{\pgfqpoint{4.823288in}{1.564666in}}%
\pgfpathlineto{\pgfqpoint{4.827829in}{1.564666in}}%
\pgfpathlineto{\pgfqpoint{4.827829in}{1.561717in}}%
\pgfpathmoveto{\pgfqpoint{4.827829in}{1.561717in}}%
\pgfpathlineto{\pgfqpoint{4.827829in}{1.561717in}}%
\pgfpathlineto{\pgfqpoint{4.827829in}{1.564666in}}%
\pgfpathlineto{\pgfqpoint{4.832370in}{1.564666in}}%
\pgfpathlineto{\pgfqpoint{4.832370in}{1.561717in}}%
\pgfpathmoveto{\pgfqpoint{4.827829in}{1.564666in}}%
\pgfpathlineto{\pgfqpoint{4.827829in}{1.564666in}}%
\pgfpathlineto{\pgfqpoint{4.827829in}{1.567616in}}%
\pgfpathlineto{\pgfqpoint{4.832370in}{1.567616in}}%
\pgfpathlineto{\pgfqpoint{4.832370in}{1.564666in}}%
\pgfpathmoveto{\pgfqpoint{4.832370in}{1.564666in}}%
\pgfpathlineto{\pgfqpoint{4.832370in}{1.564666in}}%
\pgfpathlineto{\pgfqpoint{4.832370in}{1.567616in}}%
\pgfpathlineto{\pgfqpoint{4.836911in}{1.567616in}}%
\pgfpathlineto{\pgfqpoint{4.836911in}{1.564666in}}%
\pgfpathmoveto{\pgfqpoint{4.832370in}{1.567616in}}%
\pgfpathlineto{\pgfqpoint{4.832370in}{1.567616in}}%
\pgfpathlineto{\pgfqpoint{4.832370in}{1.570565in}}%
\pgfpathlineto{\pgfqpoint{4.836911in}{1.570565in}}%
\pgfpathlineto{\pgfqpoint{4.836911in}{1.567616in}}%
\pgfpathmoveto{\pgfqpoint{4.832370in}{1.570565in}}%
\pgfpathlineto{\pgfqpoint{4.832370in}{1.570565in}}%
\pgfpathlineto{\pgfqpoint{4.832370in}{1.573514in}}%
\pgfpathlineto{\pgfqpoint{4.836911in}{1.573514in}}%
\pgfpathlineto{\pgfqpoint{4.836911in}{1.570565in}}%
\pgfpathmoveto{\pgfqpoint{4.836911in}{1.570565in}}%
\pgfpathlineto{\pgfqpoint{4.836911in}{1.570565in}}%
\pgfpathlineto{\pgfqpoint{4.836911in}{1.573514in}}%
\pgfpathlineto{\pgfqpoint{4.841452in}{1.573514in}}%
\pgfpathlineto{\pgfqpoint{4.841452in}{1.570565in}}%
\pgfpathmoveto{\pgfqpoint{4.836911in}{1.573514in}}%
\pgfpathlineto{\pgfqpoint{4.836911in}{1.573514in}}%
\pgfpathlineto{\pgfqpoint{4.836911in}{1.576463in}}%
\pgfpathlineto{\pgfqpoint{4.841452in}{1.576463in}}%
\pgfpathlineto{\pgfqpoint{4.841452in}{1.573514in}}%
\pgfpathmoveto{\pgfqpoint{4.841452in}{1.573514in}}%
\pgfpathlineto{\pgfqpoint{4.841452in}{1.573514in}}%
\pgfpathlineto{\pgfqpoint{4.841452in}{1.576463in}}%
\pgfpathlineto{\pgfqpoint{4.845993in}{1.576463in}}%
\pgfpathlineto{\pgfqpoint{4.845993in}{1.573514in}}%
\pgfpathmoveto{\pgfqpoint{4.841452in}{1.576463in}}%
\pgfpathlineto{\pgfqpoint{4.841452in}{1.576463in}}%
\pgfpathlineto{\pgfqpoint{4.841452in}{1.579412in}}%
\pgfpathlineto{\pgfqpoint{4.845993in}{1.579412in}}%
\pgfpathlineto{\pgfqpoint{4.845993in}{1.576463in}}%
\pgfpathmoveto{\pgfqpoint{4.845993in}{1.576463in}}%
\pgfpathlineto{\pgfqpoint{4.845993in}{1.576463in}}%
\pgfpathlineto{\pgfqpoint{4.845993in}{1.579412in}}%
\pgfpathlineto{\pgfqpoint{4.850534in}{1.579412in}}%
\pgfpathlineto{\pgfqpoint{4.850534in}{1.576463in}}%
\pgfpathmoveto{\pgfqpoint{4.845993in}{1.579412in}}%
\pgfpathlineto{\pgfqpoint{4.845993in}{1.579412in}}%
\pgfpathlineto{\pgfqpoint{4.845993in}{1.582361in}}%
\pgfpathlineto{\pgfqpoint{4.850534in}{1.582361in}}%
\pgfpathlineto{\pgfqpoint{4.850534in}{1.579412in}}%
\pgfpathmoveto{\pgfqpoint{4.850534in}{1.579412in}}%
\pgfpathlineto{\pgfqpoint{4.850534in}{1.579412in}}%
\pgfpathlineto{\pgfqpoint{4.850534in}{1.582361in}}%
\pgfpathlineto{\pgfqpoint{4.855075in}{1.582361in}}%
\pgfpathlineto{\pgfqpoint{4.855075in}{1.579412in}}%
\pgfpathmoveto{\pgfqpoint{4.850534in}{1.582361in}}%
\pgfpathlineto{\pgfqpoint{4.850534in}{1.582361in}}%
\pgfpathlineto{\pgfqpoint{4.850534in}{1.585311in}}%
\pgfpathlineto{\pgfqpoint{4.855075in}{1.585311in}}%
\pgfpathlineto{\pgfqpoint{4.855075in}{1.582361in}}%
\pgfpathmoveto{\pgfqpoint{4.855075in}{1.582361in}}%
\pgfpathlineto{\pgfqpoint{4.855075in}{1.582361in}}%
\pgfpathlineto{\pgfqpoint{4.855075in}{1.585311in}}%
\pgfpathlineto{\pgfqpoint{4.859616in}{1.585311in}}%
\pgfpathlineto{\pgfqpoint{4.859616in}{1.582361in}}%
\pgfpathmoveto{\pgfqpoint{4.855075in}{1.585311in}}%
\pgfpathlineto{\pgfqpoint{4.855075in}{1.585311in}}%
\pgfpathlineto{\pgfqpoint{4.855075in}{1.588260in}}%
\pgfpathlineto{\pgfqpoint{4.859616in}{1.588260in}}%
\pgfpathlineto{\pgfqpoint{4.859616in}{1.585311in}}%
\pgfpathmoveto{\pgfqpoint{4.855075in}{1.588260in}}%
\pgfpathlineto{\pgfqpoint{4.855075in}{1.588260in}}%
\pgfpathlineto{\pgfqpoint{4.855075in}{1.591209in}}%
\pgfpathlineto{\pgfqpoint{4.859616in}{1.591209in}}%
\pgfpathlineto{\pgfqpoint{4.859616in}{1.588260in}}%
\pgfpathmoveto{\pgfqpoint{4.859616in}{1.588260in}}%
\pgfpathlineto{\pgfqpoint{4.859616in}{1.588260in}}%
\pgfpathlineto{\pgfqpoint{4.859616in}{1.591209in}}%
\pgfpathlineto{\pgfqpoint{4.864157in}{1.591209in}}%
\pgfpathlineto{\pgfqpoint{4.864157in}{1.588260in}}%
\pgfpathmoveto{\pgfqpoint{4.859616in}{1.591209in}}%
\pgfpathlineto{\pgfqpoint{4.859616in}{1.591209in}}%
\pgfpathlineto{\pgfqpoint{4.859616in}{1.594158in}}%
\pgfpathlineto{\pgfqpoint{4.864157in}{1.594158in}}%
\pgfpathlineto{\pgfqpoint{4.864157in}{1.591209in}}%
\pgfpathmoveto{\pgfqpoint{4.864157in}{1.591209in}}%
\pgfpathlineto{\pgfqpoint{4.864157in}{1.591209in}}%
\pgfpathlineto{\pgfqpoint{4.864157in}{1.594158in}}%
\pgfpathlineto{\pgfqpoint{4.868698in}{1.594158in}}%
\pgfpathlineto{\pgfqpoint{4.868698in}{1.591209in}}%
\pgfpathmoveto{\pgfqpoint{4.864157in}{1.594158in}}%
\pgfpathlineto{\pgfqpoint{4.864157in}{1.594158in}}%
\pgfpathlineto{\pgfqpoint{4.864157in}{1.597107in}}%
\pgfpathlineto{\pgfqpoint{4.868698in}{1.597107in}}%
\pgfpathlineto{\pgfqpoint{4.868698in}{1.594158in}}%
\pgfpathmoveto{\pgfqpoint{4.868698in}{1.594158in}}%
\pgfpathlineto{\pgfqpoint{4.868698in}{1.594158in}}%
\pgfpathlineto{\pgfqpoint{4.868698in}{1.597107in}}%
\pgfpathlineto{\pgfqpoint{4.873239in}{1.597107in}}%
\pgfpathlineto{\pgfqpoint{4.873239in}{1.594158in}}%
\pgfpathmoveto{\pgfqpoint{4.868698in}{1.597107in}}%
\pgfpathlineto{\pgfqpoint{4.868698in}{1.597107in}}%
\pgfpathlineto{\pgfqpoint{4.868698in}{1.600056in}}%
\pgfpathlineto{\pgfqpoint{4.873239in}{1.600056in}}%
\pgfpathlineto{\pgfqpoint{4.873239in}{1.597107in}}%
\pgfpathmoveto{\pgfqpoint{4.873239in}{1.597107in}}%
\pgfpathlineto{\pgfqpoint{4.873239in}{1.597107in}}%
\pgfpathlineto{\pgfqpoint{4.873239in}{1.600056in}}%
\pgfpathlineto{\pgfqpoint{4.877780in}{1.600056in}}%
\pgfpathlineto{\pgfqpoint{4.877780in}{1.597107in}}%
\pgfpathmoveto{\pgfqpoint{4.873239in}{1.600056in}}%
\pgfpathlineto{\pgfqpoint{4.873239in}{1.600056in}}%
\pgfpathlineto{\pgfqpoint{4.873239in}{1.603006in}}%
\pgfpathlineto{\pgfqpoint{4.877780in}{1.603006in}}%
\pgfpathlineto{\pgfqpoint{4.877780in}{1.600056in}}%
\pgfpathmoveto{\pgfqpoint{4.873239in}{1.603006in}}%
\pgfpathlineto{\pgfqpoint{4.873239in}{1.603006in}}%
\pgfpathlineto{\pgfqpoint{4.873239in}{1.605955in}}%
\pgfpathlineto{\pgfqpoint{4.877780in}{1.605955in}}%
\pgfpathlineto{\pgfqpoint{4.877780in}{1.603006in}}%
\pgfpathmoveto{\pgfqpoint{4.877780in}{1.603006in}}%
\pgfpathlineto{\pgfqpoint{4.877780in}{1.603006in}}%
\pgfpathlineto{\pgfqpoint{4.877780in}{1.605955in}}%
\pgfpathlineto{\pgfqpoint{4.882321in}{1.605955in}}%
\pgfpathlineto{\pgfqpoint{4.882321in}{1.603006in}}%
\pgfpathmoveto{\pgfqpoint{4.877780in}{1.605955in}}%
\pgfpathlineto{\pgfqpoint{4.877780in}{1.605955in}}%
\pgfpathlineto{\pgfqpoint{4.877780in}{1.608904in}}%
\pgfpathlineto{\pgfqpoint{4.882321in}{1.608904in}}%
\pgfpathlineto{\pgfqpoint{4.882321in}{1.605955in}}%
\pgfpathmoveto{\pgfqpoint{4.882321in}{1.605955in}}%
\pgfpathlineto{\pgfqpoint{4.882321in}{1.605955in}}%
\pgfpathlineto{\pgfqpoint{4.882321in}{1.608904in}}%
\pgfpathlineto{\pgfqpoint{4.886863in}{1.608904in}}%
\pgfpathlineto{\pgfqpoint{4.886863in}{1.605955in}}%
\pgfpathmoveto{\pgfqpoint{4.882321in}{1.608904in}}%
\pgfpathlineto{\pgfqpoint{4.882321in}{1.608904in}}%
\pgfpathlineto{\pgfqpoint{4.882321in}{1.611853in}}%
\pgfpathlineto{\pgfqpoint{4.886863in}{1.611853in}}%
\pgfpathlineto{\pgfqpoint{4.886863in}{1.608904in}}%
\pgfpathmoveto{\pgfqpoint{4.886863in}{1.608904in}}%
\pgfpathlineto{\pgfqpoint{4.886863in}{1.608904in}}%
\pgfpathlineto{\pgfqpoint{4.886863in}{1.611853in}}%
\pgfpathlineto{\pgfqpoint{4.891404in}{1.611853in}}%
\pgfpathlineto{\pgfqpoint{4.891404in}{1.608904in}}%
\pgfpathmoveto{\pgfqpoint{4.886863in}{1.611853in}}%
\pgfpathlineto{\pgfqpoint{4.886863in}{1.611853in}}%
\pgfpathlineto{\pgfqpoint{4.886863in}{1.614802in}}%
\pgfpathlineto{\pgfqpoint{4.891404in}{1.614802in}}%
\pgfpathlineto{\pgfqpoint{4.891404in}{1.611853in}}%
\pgfpathmoveto{\pgfqpoint{4.891404in}{1.611853in}}%
\pgfpathlineto{\pgfqpoint{4.891404in}{1.611853in}}%
\pgfpathlineto{\pgfqpoint{4.891404in}{1.614802in}}%
\pgfpathlineto{\pgfqpoint{4.895945in}{1.614802in}}%
\pgfpathlineto{\pgfqpoint{4.895945in}{1.611853in}}%
\pgfpathmoveto{\pgfqpoint{4.891404in}{1.614802in}}%
\pgfpathlineto{\pgfqpoint{4.891404in}{1.614802in}}%
\pgfpathlineto{\pgfqpoint{4.891404in}{1.617752in}}%
\pgfpathlineto{\pgfqpoint{4.895945in}{1.617752in}}%
\pgfpathlineto{\pgfqpoint{4.895945in}{1.614802in}}%
\pgfpathmoveto{\pgfqpoint{4.895945in}{1.614802in}}%
\pgfpathlineto{\pgfqpoint{4.895945in}{1.614802in}}%
\pgfpathlineto{\pgfqpoint{4.895945in}{1.617752in}}%
\pgfpathlineto{\pgfqpoint{4.900486in}{1.617752in}}%
\pgfpathlineto{\pgfqpoint{4.900486in}{1.614802in}}%
\pgfpathmoveto{\pgfqpoint{4.895945in}{1.617752in}}%
\pgfpathlineto{\pgfqpoint{4.895945in}{1.617752in}}%
\pgfpathlineto{\pgfqpoint{4.895945in}{1.620701in}}%
\pgfpathlineto{\pgfqpoint{4.900486in}{1.620701in}}%
\pgfpathlineto{\pgfqpoint{4.900486in}{1.617752in}}%
\pgfpathmoveto{\pgfqpoint{4.895945in}{1.620701in}}%
\pgfpathlineto{\pgfqpoint{4.895945in}{1.620701in}}%
\pgfpathlineto{\pgfqpoint{4.895945in}{1.623650in}}%
\pgfpathlineto{\pgfqpoint{4.900486in}{1.623650in}}%
\pgfpathlineto{\pgfqpoint{4.900486in}{1.620701in}}%
\pgfpathmoveto{\pgfqpoint{4.900486in}{1.620701in}}%
\pgfpathlineto{\pgfqpoint{4.900486in}{1.620701in}}%
\pgfpathlineto{\pgfqpoint{4.900486in}{1.623650in}}%
\pgfpathlineto{\pgfqpoint{4.905027in}{1.623650in}}%
\pgfpathlineto{\pgfqpoint{4.905027in}{1.620701in}}%
\pgfpathmoveto{\pgfqpoint{4.900486in}{1.623650in}}%
\pgfpathlineto{\pgfqpoint{4.900486in}{1.623650in}}%
\pgfpathlineto{\pgfqpoint{4.900486in}{1.626599in}}%
\pgfpathlineto{\pgfqpoint{4.905027in}{1.626599in}}%
\pgfpathlineto{\pgfqpoint{4.905027in}{1.623650in}}%
\pgfpathmoveto{\pgfqpoint{4.905027in}{1.623650in}}%
\pgfpathlineto{\pgfqpoint{4.905027in}{1.623650in}}%
\pgfpathlineto{\pgfqpoint{4.905027in}{1.626599in}}%
\pgfpathlineto{\pgfqpoint{4.909568in}{1.626599in}}%
\pgfpathlineto{\pgfqpoint{4.909568in}{1.623650in}}%
\pgfpathmoveto{\pgfqpoint{4.905027in}{1.626599in}}%
\pgfpathlineto{\pgfqpoint{4.905027in}{1.626599in}}%
\pgfpathlineto{\pgfqpoint{4.905027in}{1.629548in}}%
\pgfpathlineto{\pgfqpoint{4.909568in}{1.629548in}}%
\pgfpathlineto{\pgfqpoint{4.909568in}{1.626599in}}%
\pgfpathmoveto{\pgfqpoint{4.909568in}{1.626599in}}%
\pgfpathlineto{\pgfqpoint{4.909568in}{1.626599in}}%
\pgfpathlineto{\pgfqpoint{4.909568in}{1.629548in}}%
\pgfpathlineto{\pgfqpoint{4.914109in}{1.629548in}}%
\pgfpathlineto{\pgfqpoint{4.914109in}{1.626599in}}%
\pgfpathmoveto{\pgfqpoint{4.909568in}{1.629548in}}%
\pgfpathlineto{\pgfqpoint{4.909568in}{1.629548in}}%
\pgfpathlineto{\pgfqpoint{4.909568in}{1.632497in}}%
\pgfpathlineto{\pgfqpoint{4.914109in}{1.632497in}}%
\pgfpathlineto{\pgfqpoint{4.914109in}{1.629548in}}%
\pgfpathmoveto{\pgfqpoint{4.914109in}{1.629548in}}%
\pgfpathlineto{\pgfqpoint{4.914109in}{1.629548in}}%
\pgfpathlineto{\pgfqpoint{4.914109in}{1.632497in}}%
\pgfpathlineto{\pgfqpoint{4.918650in}{1.632497in}}%
\pgfpathlineto{\pgfqpoint{4.918650in}{1.629548in}}%
\pgfpathmoveto{\pgfqpoint{4.914109in}{1.632497in}}%
\pgfpathlineto{\pgfqpoint{4.914109in}{1.632497in}}%
\pgfpathlineto{\pgfqpoint{4.914109in}{1.635447in}}%
\pgfpathlineto{\pgfqpoint{4.918650in}{1.635447in}}%
\pgfpathlineto{\pgfqpoint{4.918650in}{1.632497in}}%
\pgfpathmoveto{\pgfqpoint{4.918650in}{1.632497in}}%
\pgfpathlineto{\pgfqpoint{4.918650in}{1.632497in}}%
\pgfpathlineto{\pgfqpoint{4.918650in}{1.635447in}}%
\pgfpathlineto{\pgfqpoint{4.923191in}{1.635447in}}%
\pgfpathlineto{\pgfqpoint{4.923191in}{1.632497in}}%
\pgfpathmoveto{\pgfqpoint{4.918650in}{1.635447in}}%
\pgfpathlineto{\pgfqpoint{4.918650in}{1.635447in}}%
\pgfpathlineto{\pgfqpoint{4.918650in}{1.638396in}}%
\pgfpathlineto{\pgfqpoint{4.923191in}{1.638396in}}%
\pgfpathlineto{\pgfqpoint{4.923191in}{1.635447in}}%
\pgfpathmoveto{\pgfqpoint{4.918650in}{1.638396in}}%
\pgfpathlineto{\pgfqpoint{4.918650in}{1.638396in}}%
\pgfpathlineto{\pgfqpoint{4.918650in}{1.641345in}}%
\pgfpathlineto{\pgfqpoint{4.923191in}{1.641345in}}%
\pgfpathlineto{\pgfqpoint{4.923191in}{1.638396in}}%
\pgfpathmoveto{\pgfqpoint{4.923191in}{1.638396in}}%
\pgfpathlineto{\pgfqpoint{4.923191in}{1.638396in}}%
\pgfpathlineto{\pgfqpoint{4.923191in}{1.641345in}}%
\pgfpathlineto{\pgfqpoint{4.927732in}{1.641345in}}%
\pgfpathlineto{\pgfqpoint{4.927732in}{1.638396in}}%
\pgfpathmoveto{\pgfqpoint{4.923191in}{1.641345in}}%
\pgfpathlineto{\pgfqpoint{4.923191in}{1.641345in}}%
\pgfpathlineto{\pgfqpoint{4.923191in}{1.644294in}}%
\pgfpathlineto{\pgfqpoint{4.927732in}{1.644294in}}%
\pgfpathlineto{\pgfqpoint{4.927732in}{1.641345in}}%
\pgfpathmoveto{\pgfqpoint{4.927732in}{1.641345in}}%
\pgfpathlineto{\pgfqpoint{4.927732in}{1.641345in}}%
\pgfpathlineto{\pgfqpoint{4.927732in}{1.644294in}}%
\pgfpathlineto{\pgfqpoint{4.932273in}{1.644294in}}%
\pgfpathlineto{\pgfqpoint{4.932273in}{1.641345in}}%
\pgfpathmoveto{\pgfqpoint{4.927732in}{1.644294in}}%
\pgfpathlineto{\pgfqpoint{4.927732in}{1.644294in}}%
\pgfpathlineto{\pgfqpoint{4.927732in}{1.647244in}}%
\pgfpathlineto{\pgfqpoint{4.932273in}{1.647244in}}%
\pgfpathlineto{\pgfqpoint{4.932273in}{1.644294in}}%
\pgfpathmoveto{\pgfqpoint{4.932273in}{1.644294in}}%
\pgfpathlineto{\pgfqpoint{4.932273in}{1.644294in}}%
\pgfpathlineto{\pgfqpoint{4.932273in}{1.647244in}}%
\pgfpathlineto{\pgfqpoint{4.936814in}{1.647244in}}%
\pgfpathlineto{\pgfqpoint{4.936814in}{1.644294in}}%
\pgfpathmoveto{\pgfqpoint{4.932273in}{1.647244in}}%
\pgfpathlineto{\pgfqpoint{4.932273in}{1.647244in}}%
\pgfpathlineto{\pgfqpoint{4.932273in}{1.650193in}}%
\pgfpathlineto{\pgfqpoint{4.936814in}{1.650193in}}%
\pgfpathlineto{\pgfqpoint{4.936814in}{1.647244in}}%
\pgfpathmoveto{\pgfqpoint{4.936814in}{1.647244in}}%
\pgfpathlineto{\pgfqpoint{4.936814in}{1.647244in}}%
\pgfpathlineto{\pgfqpoint{4.936814in}{1.650193in}}%
\pgfpathlineto{\pgfqpoint{4.941355in}{1.650193in}}%
\pgfpathlineto{\pgfqpoint{4.941355in}{1.647244in}}%
\pgfpathmoveto{\pgfqpoint{4.936814in}{1.650193in}}%
\pgfpathlineto{\pgfqpoint{4.936814in}{1.650193in}}%
\pgfpathlineto{\pgfqpoint{4.936814in}{1.653142in}}%
\pgfpathlineto{\pgfqpoint{4.941355in}{1.653142in}}%
\pgfpathlineto{\pgfqpoint{4.941355in}{1.650193in}}%
\pgfpathmoveto{\pgfqpoint{4.941355in}{1.650193in}}%
\pgfpathlineto{\pgfqpoint{4.941355in}{1.650193in}}%
\pgfpathlineto{\pgfqpoint{4.941355in}{1.653142in}}%
\pgfpathlineto{\pgfqpoint{4.945896in}{1.653142in}}%
\pgfpathlineto{\pgfqpoint{4.945896in}{1.650193in}}%
\pgfpathmoveto{\pgfqpoint{4.941355in}{1.653142in}}%
\pgfpathlineto{\pgfqpoint{4.941355in}{1.653142in}}%
\pgfpathlineto{\pgfqpoint{4.941355in}{1.656091in}}%
\pgfpathlineto{\pgfqpoint{4.945896in}{1.656091in}}%
\pgfpathlineto{\pgfqpoint{4.945896in}{1.653142in}}%
\pgfpathmoveto{\pgfqpoint{4.941355in}{1.656091in}}%
\pgfpathlineto{\pgfqpoint{4.941355in}{1.656091in}}%
\pgfpathlineto{\pgfqpoint{4.941355in}{1.659040in}}%
\pgfpathlineto{\pgfqpoint{4.945896in}{1.659040in}}%
\pgfpathlineto{\pgfqpoint{4.945896in}{1.656091in}}%
\pgfpathmoveto{\pgfqpoint{4.945896in}{1.656091in}}%
\pgfpathlineto{\pgfqpoint{4.945896in}{1.656091in}}%
\pgfpathlineto{\pgfqpoint{4.945896in}{1.659040in}}%
\pgfpathlineto{\pgfqpoint{4.950437in}{1.659040in}}%
\pgfpathlineto{\pgfqpoint{4.950437in}{1.656091in}}%
\pgfpathmoveto{\pgfqpoint{4.945896in}{1.659040in}}%
\pgfpathlineto{\pgfqpoint{4.945896in}{1.659040in}}%
\pgfpathlineto{\pgfqpoint{4.945896in}{1.661990in}}%
\pgfpathlineto{\pgfqpoint{4.950437in}{1.661990in}}%
\pgfpathlineto{\pgfqpoint{4.950437in}{1.659040in}}%
\pgfpathmoveto{\pgfqpoint{4.950437in}{1.659040in}}%
\pgfpathlineto{\pgfqpoint{4.950437in}{1.659040in}}%
\pgfpathlineto{\pgfqpoint{4.950437in}{1.661990in}}%
\pgfpathlineto{\pgfqpoint{4.954978in}{1.661990in}}%
\pgfpathlineto{\pgfqpoint{4.954978in}{1.659040in}}%
\pgfpathmoveto{\pgfqpoint{4.950437in}{1.661990in}}%
\pgfpathlineto{\pgfqpoint{4.950437in}{1.661990in}}%
\pgfpathlineto{\pgfqpoint{4.950437in}{1.664939in}}%
\pgfpathlineto{\pgfqpoint{4.954978in}{1.664939in}}%
\pgfpathlineto{\pgfqpoint{4.954978in}{1.661990in}}%
\pgfpathmoveto{\pgfqpoint{4.954978in}{1.661990in}}%
\pgfpathlineto{\pgfqpoint{4.954978in}{1.661990in}}%
\pgfpathlineto{\pgfqpoint{4.954978in}{1.664939in}}%
\pgfpathlineto{\pgfqpoint{4.959519in}{1.664939in}}%
\pgfpathlineto{\pgfqpoint{4.959519in}{1.661990in}}%
\pgfpathmoveto{\pgfqpoint{4.954978in}{1.664939in}}%
\pgfpathlineto{\pgfqpoint{4.954978in}{1.664939in}}%
\pgfpathlineto{\pgfqpoint{4.954978in}{1.667888in}}%
\pgfpathlineto{\pgfqpoint{4.959519in}{1.667888in}}%
\pgfpathlineto{\pgfqpoint{4.959519in}{1.664939in}}%
\pgfpathmoveto{\pgfqpoint{4.959519in}{1.664939in}}%
\pgfpathlineto{\pgfqpoint{4.959519in}{1.664939in}}%
\pgfpathlineto{\pgfqpoint{4.959519in}{1.667888in}}%
\pgfpathlineto{\pgfqpoint{4.964060in}{1.667888in}}%
\pgfpathlineto{\pgfqpoint{4.964060in}{1.664939in}}%
\pgfpathmoveto{\pgfqpoint{4.959519in}{1.667888in}}%
\pgfpathlineto{\pgfqpoint{4.959519in}{1.667888in}}%
\pgfpathlineto{\pgfqpoint{4.959519in}{1.670837in}}%
\pgfpathlineto{\pgfqpoint{4.964060in}{1.670837in}}%
\pgfpathlineto{\pgfqpoint{4.964060in}{1.667888in}}%
\pgfpathmoveto{\pgfqpoint{4.964060in}{1.667888in}}%
\pgfpathlineto{\pgfqpoint{4.964060in}{1.667888in}}%
\pgfpathlineto{\pgfqpoint{4.964060in}{1.670837in}}%
\pgfpathlineto{\pgfqpoint{4.968601in}{1.670837in}}%
\pgfpathlineto{\pgfqpoint{4.968601in}{1.667888in}}%
\pgfpathmoveto{\pgfqpoint{4.964060in}{1.670837in}}%
\pgfpathlineto{\pgfqpoint{4.964060in}{1.670837in}}%
\pgfpathlineto{\pgfqpoint{4.964060in}{1.673787in}}%
\pgfpathlineto{\pgfqpoint{4.968601in}{1.673787in}}%
\pgfpathlineto{\pgfqpoint{4.968601in}{1.670837in}}%
\pgfpathmoveto{\pgfqpoint{4.964060in}{1.673787in}}%
\pgfpathlineto{\pgfqpoint{4.964060in}{1.673787in}}%
\pgfpathlineto{\pgfqpoint{4.964060in}{1.676736in}}%
\pgfpathlineto{\pgfqpoint{4.968601in}{1.676736in}}%
\pgfpathlineto{\pgfqpoint{4.968601in}{1.673787in}}%
\pgfpathmoveto{\pgfqpoint{4.968601in}{1.673787in}}%
\pgfpathlineto{\pgfqpoint{4.968601in}{1.673787in}}%
\pgfpathlineto{\pgfqpoint{4.968601in}{1.676736in}}%
\pgfpathlineto{\pgfqpoint{4.973142in}{1.676736in}}%
\pgfpathlineto{\pgfqpoint{4.973142in}{1.673787in}}%
\pgfpathmoveto{\pgfqpoint{4.968601in}{1.676736in}}%
\pgfpathlineto{\pgfqpoint{4.968601in}{1.676736in}}%
\pgfpathlineto{\pgfqpoint{4.968601in}{1.679685in}}%
\pgfpathlineto{\pgfqpoint{4.973142in}{1.679685in}}%
\pgfpathlineto{\pgfqpoint{4.973142in}{1.676736in}}%
\pgfpathmoveto{\pgfqpoint{4.973142in}{1.676736in}}%
\pgfpathlineto{\pgfqpoint{4.973142in}{1.676736in}}%
\pgfpathlineto{\pgfqpoint{4.973142in}{1.679685in}}%
\pgfpathlineto{\pgfqpoint{4.977683in}{1.679685in}}%
\pgfpathlineto{\pgfqpoint{4.977683in}{1.676736in}}%
\pgfpathmoveto{\pgfqpoint{4.973142in}{1.679685in}}%
\pgfpathlineto{\pgfqpoint{4.973142in}{1.679685in}}%
\pgfpathlineto{\pgfqpoint{4.973142in}{1.682634in}}%
\pgfpathlineto{\pgfqpoint{4.977683in}{1.682634in}}%
\pgfpathlineto{\pgfqpoint{4.977683in}{1.679685in}}%
\pgfpathmoveto{\pgfqpoint{4.977683in}{1.679685in}}%
\pgfpathlineto{\pgfqpoint{4.977683in}{1.679685in}}%
\pgfpathlineto{\pgfqpoint{4.977683in}{1.682634in}}%
\pgfpathlineto{\pgfqpoint{4.982224in}{1.682634in}}%
\pgfpathlineto{\pgfqpoint{4.982224in}{1.679685in}}%
\pgfpathmoveto{\pgfqpoint{4.977683in}{1.682634in}}%
\pgfpathlineto{\pgfqpoint{4.977683in}{1.682634in}}%
\pgfpathlineto{\pgfqpoint{4.977683in}{1.685583in}}%
\pgfpathlineto{\pgfqpoint{4.982224in}{1.685583in}}%
\pgfpathlineto{\pgfqpoint{4.982224in}{1.682634in}}%
\pgfpathmoveto{\pgfqpoint{4.982224in}{1.682634in}}%
\pgfpathlineto{\pgfqpoint{4.982224in}{1.682634in}}%
\pgfpathlineto{\pgfqpoint{4.982224in}{1.685583in}}%
\pgfpathlineto{\pgfqpoint{4.986765in}{1.685583in}}%
\pgfpathlineto{\pgfqpoint{4.986765in}{1.682634in}}%
\pgfpathmoveto{\pgfqpoint{4.982224in}{1.685583in}}%
\pgfpathlineto{\pgfqpoint{4.982224in}{1.685583in}}%
\pgfpathlineto{\pgfqpoint{4.982224in}{1.688533in}}%
\pgfpathlineto{\pgfqpoint{4.986765in}{1.688533in}}%
\pgfpathlineto{\pgfqpoint{4.986765in}{1.685583in}}%
\pgfpathmoveto{\pgfqpoint{4.986765in}{1.685583in}}%
\pgfpathlineto{\pgfqpoint{4.986765in}{1.685583in}}%
\pgfpathlineto{\pgfqpoint{4.986765in}{1.688533in}}%
\pgfpathlineto{\pgfqpoint{4.991306in}{1.688533in}}%
\pgfpathlineto{\pgfqpoint{4.991306in}{1.685583in}}%
\pgfpathmoveto{\pgfqpoint{4.986765in}{1.688533in}}%
\pgfpathlineto{\pgfqpoint{4.986765in}{1.688533in}}%
\pgfpathlineto{\pgfqpoint{4.986765in}{1.691482in}}%
\pgfpathlineto{\pgfqpoint{4.991306in}{1.691482in}}%
\pgfpathlineto{\pgfqpoint{4.991306in}{1.688533in}}%
\pgfpathmoveto{\pgfqpoint{4.986765in}{1.691482in}}%
\pgfpathlineto{\pgfqpoint{4.986765in}{1.691482in}}%
\pgfpathlineto{\pgfqpoint{4.986765in}{1.694431in}}%
\pgfpathlineto{\pgfqpoint{4.991306in}{1.694431in}}%
\pgfpathlineto{\pgfqpoint{4.991306in}{1.691482in}}%
\pgfpathmoveto{\pgfqpoint{4.991306in}{1.691482in}}%
\pgfpathlineto{\pgfqpoint{4.991306in}{1.691482in}}%
\pgfpathlineto{\pgfqpoint{4.991306in}{1.694431in}}%
\pgfpathlineto{\pgfqpoint{4.995847in}{1.694431in}}%
\pgfpathlineto{\pgfqpoint{4.995847in}{1.691482in}}%
\pgfpathmoveto{\pgfqpoint{4.991306in}{1.694431in}}%
\pgfpathlineto{\pgfqpoint{4.991306in}{1.694431in}}%
\pgfpathlineto{\pgfqpoint{4.991306in}{1.697380in}}%
\pgfpathlineto{\pgfqpoint{4.995847in}{1.697380in}}%
\pgfpathlineto{\pgfqpoint{4.995847in}{1.694431in}}%
\pgfpathmoveto{\pgfqpoint{4.995847in}{1.694431in}}%
\pgfpathlineto{\pgfqpoint{4.995847in}{1.694431in}}%
\pgfpathlineto{\pgfqpoint{4.995847in}{1.697380in}}%
\pgfpathlineto{\pgfqpoint{5.000388in}{1.697380in}}%
\pgfpathlineto{\pgfqpoint{5.000388in}{1.694431in}}%
\pgfpathmoveto{\pgfqpoint{4.995847in}{1.697380in}}%
\pgfpathlineto{\pgfqpoint{4.995847in}{1.697380in}}%
\pgfpathlineto{\pgfqpoint{4.995847in}{1.700330in}}%
\pgfpathlineto{\pgfqpoint{5.000388in}{1.700330in}}%
\pgfpathlineto{\pgfqpoint{5.000388in}{1.697380in}}%
\pgfpathmoveto{\pgfqpoint{5.000388in}{1.697380in}}%
\pgfpathlineto{\pgfqpoint{5.000388in}{1.697380in}}%
\pgfpathlineto{\pgfqpoint{5.000388in}{1.700330in}}%
\pgfpathlineto{\pgfqpoint{5.004929in}{1.700330in}}%
\pgfpathlineto{\pgfqpoint{5.004929in}{1.697380in}}%
\pgfpathmoveto{\pgfqpoint{5.000388in}{1.700330in}}%
\pgfpathlineto{\pgfqpoint{5.000388in}{1.700330in}}%
\pgfpathlineto{\pgfqpoint{5.000388in}{1.703279in}}%
\pgfpathlineto{\pgfqpoint{5.004929in}{1.703279in}}%
\pgfpathlineto{\pgfqpoint{5.004929in}{1.700330in}}%
\pgfpathmoveto{\pgfqpoint{5.004929in}{1.700330in}}%
\pgfpathlineto{\pgfqpoint{5.004929in}{1.700330in}}%
\pgfpathlineto{\pgfqpoint{5.004929in}{1.703279in}}%
\pgfpathlineto{\pgfqpoint{5.009470in}{1.703279in}}%
\pgfpathlineto{\pgfqpoint{5.009470in}{1.700330in}}%
\pgfpathmoveto{\pgfqpoint{5.004929in}{1.703279in}}%
\pgfpathlineto{\pgfqpoint{5.004929in}{1.703279in}}%
\pgfpathlineto{\pgfqpoint{5.004929in}{1.706228in}}%
\pgfpathlineto{\pgfqpoint{5.009470in}{1.706228in}}%
\pgfpathlineto{\pgfqpoint{5.009470in}{1.703279in}}%
\pgfpathmoveto{\pgfqpoint{5.009470in}{1.703279in}}%
\pgfpathlineto{\pgfqpoint{5.009470in}{1.703279in}}%
\pgfpathlineto{\pgfqpoint{5.009470in}{1.706228in}}%
\pgfpathlineto{\pgfqpoint{5.014012in}{1.706228in}}%
\pgfpathlineto{\pgfqpoint{5.014012in}{1.703279in}}%
\pgfpathmoveto{\pgfqpoint{5.009470in}{1.706228in}}%
\pgfpathlineto{\pgfqpoint{5.009470in}{1.706228in}}%
\pgfpathlineto{\pgfqpoint{5.009470in}{1.709177in}}%
\pgfpathlineto{\pgfqpoint{5.014012in}{1.709177in}}%
\pgfpathlineto{\pgfqpoint{5.014012in}{1.706228in}}%
\pgfpathmoveto{\pgfqpoint{5.009470in}{1.709177in}}%
\pgfpathlineto{\pgfqpoint{5.009470in}{1.709177in}}%
\pgfpathlineto{\pgfqpoint{5.009470in}{1.712126in}}%
\pgfpathlineto{\pgfqpoint{5.014012in}{1.712126in}}%
\pgfpathlineto{\pgfqpoint{5.014012in}{1.709177in}}%
\pgfpathmoveto{\pgfqpoint{5.014012in}{1.709177in}}%
\pgfpathlineto{\pgfqpoint{5.014012in}{1.709177in}}%
\pgfpathlineto{\pgfqpoint{5.014012in}{1.712126in}}%
\pgfpathlineto{\pgfqpoint{5.018553in}{1.712126in}}%
\pgfpathlineto{\pgfqpoint{5.018553in}{1.709177in}}%
\pgfpathmoveto{\pgfqpoint{5.014012in}{1.712126in}}%
\pgfpathlineto{\pgfqpoint{5.014012in}{1.712126in}}%
\pgfpathlineto{\pgfqpoint{5.014012in}{1.715076in}}%
\pgfpathlineto{\pgfqpoint{5.018553in}{1.715076in}}%
\pgfpathlineto{\pgfqpoint{5.018553in}{1.712126in}}%
\pgfpathmoveto{\pgfqpoint{5.018553in}{1.712126in}}%
\pgfpathlineto{\pgfqpoint{5.018553in}{1.712126in}}%
\pgfpathlineto{\pgfqpoint{5.018553in}{1.715076in}}%
\pgfpathlineto{\pgfqpoint{5.023094in}{1.715076in}}%
\pgfpathlineto{\pgfqpoint{5.023094in}{1.712126in}}%
\pgfpathmoveto{\pgfqpoint{5.018553in}{1.715076in}}%
\pgfpathlineto{\pgfqpoint{5.018553in}{1.715076in}}%
\pgfpathlineto{\pgfqpoint{5.018553in}{1.718025in}}%
\pgfpathlineto{\pgfqpoint{5.023094in}{1.718025in}}%
\pgfpathlineto{\pgfqpoint{5.023094in}{1.715076in}}%
\pgfpathmoveto{\pgfqpoint{5.023094in}{1.715076in}}%
\pgfpathlineto{\pgfqpoint{5.023094in}{1.715076in}}%
\pgfpathlineto{\pgfqpoint{5.023094in}{1.718025in}}%
\pgfpathlineto{\pgfqpoint{5.027635in}{1.718025in}}%
\pgfpathlineto{\pgfqpoint{5.027635in}{1.715076in}}%
\pgfpathmoveto{\pgfqpoint{5.023094in}{1.718025in}}%
\pgfpathlineto{\pgfqpoint{5.023094in}{1.718025in}}%
\pgfpathlineto{\pgfqpoint{5.023094in}{1.720974in}}%
\pgfpathlineto{\pgfqpoint{5.027635in}{1.720974in}}%
\pgfpathlineto{\pgfqpoint{5.027635in}{1.718025in}}%
\pgfpathmoveto{\pgfqpoint{5.027635in}{1.718025in}}%
\pgfpathlineto{\pgfqpoint{5.027635in}{1.718025in}}%
\pgfpathlineto{\pgfqpoint{5.027635in}{1.720974in}}%
\pgfpathlineto{\pgfqpoint{5.032176in}{1.720974in}}%
\pgfpathlineto{\pgfqpoint{5.032176in}{1.718025in}}%
\pgfpathmoveto{\pgfqpoint{5.027635in}{1.720974in}}%
\pgfpathlineto{\pgfqpoint{5.027635in}{1.720974in}}%
\pgfpathlineto{\pgfqpoint{5.027635in}{1.723923in}}%
\pgfpathlineto{\pgfqpoint{5.032176in}{1.723923in}}%
\pgfpathlineto{\pgfqpoint{5.032176in}{1.720974in}}%
\pgfpathmoveto{\pgfqpoint{5.032176in}{1.720974in}}%
\pgfpathlineto{\pgfqpoint{5.032176in}{1.720974in}}%
\pgfpathlineto{\pgfqpoint{5.032176in}{1.723923in}}%
\pgfpathlineto{\pgfqpoint{5.036717in}{1.723923in}}%
\pgfpathlineto{\pgfqpoint{5.036717in}{1.720974in}}%
\pgfpathmoveto{\pgfqpoint{5.032176in}{1.723923in}}%
\pgfpathlineto{\pgfqpoint{5.032176in}{1.723923in}}%
\pgfpathlineto{\pgfqpoint{5.032176in}{1.726873in}}%
\pgfpathlineto{\pgfqpoint{5.036717in}{1.726873in}}%
\pgfpathlineto{\pgfqpoint{5.036717in}{1.723923in}}%
\pgfpathmoveto{\pgfqpoint{5.032176in}{1.726873in}}%
\pgfpathlineto{\pgfqpoint{5.032176in}{1.726873in}}%
\pgfpathlineto{\pgfqpoint{5.032176in}{1.729822in}}%
\pgfpathlineto{\pgfqpoint{5.036717in}{1.729822in}}%
\pgfpathlineto{\pgfqpoint{5.036717in}{1.726873in}}%
\pgfpathmoveto{\pgfqpoint{5.036717in}{1.726873in}}%
\pgfpathlineto{\pgfqpoint{5.036717in}{1.726873in}}%
\pgfpathlineto{\pgfqpoint{5.036717in}{1.729822in}}%
\pgfpathlineto{\pgfqpoint{5.041258in}{1.729822in}}%
\pgfpathlineto{\pgfqpoint{5.041258in}{1.726873in}}%
\pgfpathmoveto{\pgfqpoint{5.036717in}{1.729822in}}%
\pgfpathlineto{\pgfqpoint{5.036717in}{1.729822in}}%
\pgfpathlineto{\pgfqpoint{5.036717in}{1.732771in}}%
\pgfpathlineto{\pgfqpoint{5.041258in}{1.732771in}}%
\pgfpathlineto{\pgfqpoint{5.041258in}{1.729822in}}%
\pgfpathmoveto{\pgfqpoint{5.041258in}{1.729822in}}%
\pgfpathlineto{\pgfqpoint{5.041258in}{1.729822in}}%
\pgfpathlineto{\pgfqpoint{5.041258in}{1.732771in}}%
\pgfpathlineto{\pgfqpoint{5.045799in}{1.732771in}}%
\pgfpathlineto{\pgfqpoint{5.045799in}{1.729822in}}%
\pgfpathmoveto{\pgfqpoint{5.041258in}{1.732771in}}%
\pgfpathlineto{\pgfqpoint{5.041258in}{1.732771in}}%
\pgfpathlineto{\pgfqpoint{5.041258in}{1.735720in}}%
\pgfpathlineto{\pgfqpoint{5.045799in}{1.735720in}}%
\pgfpathlineto{\pgfqpoint{5.045799in}{1.732771in}}%
\pgfpathmoveto{\pgfqpoint{5.045799in}{1.732771in}}%
\pgfpathlineto{\pgfqpoint{5.045799in}{1.732771in}}%
\pgfpathlineto{\pgfqpoint{5.045799in}{1.735720in}}%
\pgfpathlineto{\pgfqpoint{5.050340in}{1.735720in}}%
\pgfpathlineto{\pgfqpoint{5.050340in}{1.732771in}}%
\pgfpathmoveto{\pgfqpoint{5.045799in}{1.735720in}}%
\pgfpathlineto{\pgfqpoint{5.045799in}{1.735720in}}%
\pgfpathlineto{\pgfqpoint{5.045799in}{1.738670in}}%
\pgfpathlineto{\pgfqpoint{5.050340in}{1.738670in}}%
\pgfpathlineto{\pgfqpoint{5.050340in}{1.735720in}}%
\pgfpathmoveto{\pgfqpoint{5.050340in}{1.735720in}}%
\pgfpathlineto{\pgfqpoint{5.050340in}{1.735720in}}%
\pgfpathlineto{\pgfqpoint{5.050340in}{1.738670in}}%
\pgfpathlineto{\pgfqpoint{5.054881in}{1.738670in}}%
\pgfpathlineto{\pgfqpoint{5.054881in}{1.735720in}}%
\pgfpathmoveto{\pgfqpoint{5.050340in}{1.738670in}}%
\pgfpathlineto{\pgfqpoint{5.050340in}{1.738670in}}%
\pgfpathlineto{\pgfqpoint{5.050340in}{1.741619in}}%
\pgfpathlineto{\pgfqpoint{5.054881in}{1.741619in}}%
\pgfpathlineto{\pgfqpoint{5.054881in}{1.738670in}}%
\pgfpathmoveto{\pgfqpoint{5.054881in}{1.738670in}}%
\pgfpathlineto{\pgfqpoint{5.054881in}{1.738670in}}%
\pgfpathlineto{\pgfqpoint{5.054881in}{1.741619in}}%
\pgfpathlineto{\pgfqpoint{5.059422in}{1.741619in}}%
\pgfpathlineto{\pgfqpoint{5.059422in}{1.738670in}}%
\pgfpathmoveto{\pgfqpoint{5.054881in}{1.741619in}}%
\pgfpathlineto{\pgfqpoint{5.054881in}{1.741619in}}%
\pgfpathlineto{\pgfqpoint{5.054881in}{1.744568in}}%
\pgfpathlineto{\pgfqpoint{5.059422in}{1.744568in}}%
\pgfpathlineto{\pgfqpoint{5.059422in}{1.741619in}}%
\pgfpathmoveto{\pgfqpoint{5.054881in}{1.744568in}}%
\pgfpathlineto{\pgfqpoint{5.054881in}{1.744568in}}%
\pgfpathlineto{\pgfqpoint{5.054881in}{1.747518in}}%
\pgfpathlineto{\pgfqpoint{5.059422in}{1.747518in}}%
\pgfpathlineto{\pgfqpoint{5.059422in}{1.744568in}}%
\pgfpathmoveto{\pgfqpoint{5.059422in}{1.744568in}}%
\pgfpathlineto{\pgfqpoint{5.059422in}{1.744568in}}%
\pgfpathlineto{\pgfqpoint{5.059422in}{1.747518in}}%
\pgfpathlineto{\pgfqpoint{5.063963in}{1.747518in}}%
\pgfpathlineto{\pgfqpoint{5.063963in}{1.744568in}}%
\pgfpathmoveto{\pgfqpoint{5.059422in}{1.747518in}}%
\pgfpathlineto{\pgfqpoint{5.059422in}{1.747518in}}%
\pgfpathlineto{\pgfqpoint{5.059422in}{1.750467in}}%
\pgfpathlineto{\pgfqpoint{5.063963in}{1.750467in}}%
\pgfpathlineto{\pgfqpoint{5.063963in}{1.747518in}}%
\pgfpathmoveto{\pgfqpoint{5.063963in}{1.747518in}}%
\pgfpathlineto{\pgfqpoint{5.063963in}{1.747518in}}%
\pgfpathlineto{\pgfqpoint{5.063963in}{1.750467in}}%
\pgfpathlineto{\pgfqpoint{5.068504in}{1.750467in}}%
\pgfpathlineto{\pgfqpoint{5.068504in}{1.747518in}}%
\pgfpathmoveto{\pgfqpoint{5.063963in}{1.750467in}}%
\pgfpathlineto{\pgfqpoint{5.063963in}{1.750467in}}%
\pgfpathlineto{\pgfqpoint{5.063963in}{1.753416in}}%
\pgfpathlineto{\pgfqpoint{5.068504in}{1.753416in}}%
\pgfpathlineto{\pgfqpoint{5.068504in}{1.750467in}}%
\pgfpathmoveto{\pgfqpoint{5.068504in}{1.750467in}}%
\pgfpathlineto{\pgfqpoint{5.068504in}{1.750467in}}%
\pgfpathlineto{\pgfqpoint{5.068504in}{1.753416in}}%
\pgfpathlineto{\pgfqpoint{5.073045in}{1.753416in}}%
\pgfpathlineto{\pgfqpoint{5.073045in}{1.750467in}}%
\pgfpathmoveto{\pgfqpoint{5.068504in}{1.753416in}}%
\pgfpathlineto{\pgfqpoint{5.068504in}{1.753416in}}%
\pgfpathlineto{\pgfqpoint{5.068504in}{1.756365in}}%
\pgfpathlineto{\pgfqpoint{5.073045in}{1.756365in}}%
\pgfpathlineto{\pgfqpoint{5.073045in}{1.753416in}}%
\pgfpathmoveto{\pgfqpoint{5.073045in}{1.753416in}}%
\pgfpathlineto{\pgfqpoint{5.073045in}{1.753416in}}%
\pgfpathlineto{\pgfqpoint{5.073045in}{1.756365in}}%
\pgfpathlineto{\pgfqpoint{5.077586in}{1.756365in}}%
\pgfpathlineto{\pgfqpoint{5.077586in}{1.753416in}}%
\pgfpathmoveto{\pgfqpoint{5.073045in}{1.756365in}}%
\pgfpathlineto{\pgfqpoint{5.073045in}{1.756365in}}%
\pgfpathlineto{\pgfqpoint{5.073045in}{1.759315in}}%
\pgfpathlineto{\pgfqpoint{5.077586in}{1.759315in}}%
\pgfpathlineto{\pgfqpoint{5.077586in}{1.756365in}}%
\pgfpathmoveto{\pgfqpoint{5.077586in}{1.756365in}}%
\pgfpathlineto{\pgfqpoint{5.077586in}{1.756365in}}%
\pgfpathlineto{\pgfqpoint{5.077586in}{1.759315in}}%
\pgfpathlineto{\pgfqpoint{5.082127in}{1.759315in}}%
\pgfpathlineto{\pgfqpoint{5.082127in}{1.756365in}}%
\pgfpathmoveto{\pgfqpoint{5.077586in}{1.759315in}}%
\pgfpathlineto{\pgfqpoint{5.077586in}{1.759315in}}%
\pgfpathlineto{\pgfqpoint{5.077586in}{1.762264in}}%
\pgfpathlineto{\pgfqpoint{5.082127in}{1.762264in}}%
\pgfpathlineto{\pgfqpoint{5.082127in}{1.759315in}}%
\pgfpathmoveto{\pgfqpoint{5.077586in}{1.762264in}}%
\pgfpathlineto{\pgfqpoint{5.077586in}{1.762264in}}%
\pgfpathlineto{\pgfqpoint{5.077586in}{1.765213in}}%
\pgfpathlineto{\pgfqpoint{5.082127in}{1.765213in}}%
\pgfpathlineto{\pgfqpoint{5.082127in}{1.762264in}}%
\pgfpathmoveto{\pgfqpoint{5.082127in}{1.762264in}}%
\pgfpathlineto{\pgfqpoint{5.082127in}{1.762264in}}%
\pgfpathlineto{\pgfqpoint{5.082127in}{1.765213in}}%
\pgfpathlineto{\pgfqpoint{5.086668in}{1.765213in}}%
\pgfpathlineto{\pgfqpoint{5.086668in}{1.762264in}}%
\pgfpathmoveto{\pgfqpoint{5.082127in}{1.765213in}}%
\pgfpathlineto{\pgfqpoint{5.082127in}{1.765213in}}%
\pgfpathlineto{\pgfqpoint{5.082127in}{1.768163in}}%
\pgfpathlineto{\pgfqpoint{5.086668in}{1.768163in}}%
\pgfpathlineto{\pgfqpoint{5.086668in}{1.765213in}}%
\pgfpathmoveto{\pgfqpoint{5.086668in}{1.765213in}}%
\pgfpathlineto{\pgfqpoint{5.086668in}{1.765213in}}%
\pgfpathlineto{\pgfqpoint{5.086668in}{1.768163in}}%
\pgfpathlineto{\pgfqpoint{5.091209in}{1.768163in}}%
\pgfpathlineto{\pgfqpoint{5.091209in}{1.765213in}}%
\pgfpathmoveto{\pgfqpoint{5.086668in}{1.768163in}}%
\pgfpathlineto{\pgfqpoint{5.086668in}{1.768163in}}%
\pgfpathlineto{\pgfqpoint{5.086668in}{1.771112in}}%
\pgfpathlineto{\pgfqpoint{5.091209in}{1.771112in}}%
\pgfpathlineto{\pgfqpoint{5.091209in}{1.768163in}}%
\pgfpathmoveto{\pgfqpoint{5.091209in}{1.768163in}}%
\pgfpathlineto{\pgfqpoint{5.091209in}{1.768163in}}%
\pgfpathlineto{\pgfqpoint{5.091209in}{1.771112in}}%
\pgfpathlineto{\pgfqpoint{5.095750in}{1.771112in}}%
\pgfpathlineto{\pgfqpoint{5.095750in}{1.768163in}}%
\pgfpathmoveto{\pgfqpoint{5.091209in}{1.771112in}}%
\pgfpathlineto{\pgfqpoint{5.091209in}{1.771112in}}%
\pgfpathlineto{\pgfqpoint{5.091209in}{1.774061in}}%
\pgfpathlineto{\pgfqpoint{5.095750in}{1.774061in}}%
\pgfpathlineto{\pgfqpoint{5.095750in}{1.771112in}}%
\pgfpathmoveto{\pgfqpoint{5.095750in}{1.771112in}}%
\pgfpathlineto{\pgfqpoint{5.095750in}{1.771112in}}%
\pgfpathlineto{\pgfqpoint{5.095750in}{1.774061in}}%
\pgfpathlineto{\pgfqpoint{5.100291in}{1.774061in}}%
\pgfpathlineto{\pgfqpoint{5.100291in}{1.771112in}}%
\pgfpathmoveto{\pgfqpoint{5.095750in}{1.774061in}}%
\pgfpathlineto{\pgfqpoint{5.095750in}{1.774061in}}%
\pgfpathlineto{\pgfqpoint{5.095750in}{1.777010in}}%
\pgfpathlineto{\pgfqpoint{5.100291in}{1.777010in}}%
\pgfpathlineto{\pgfqpoint{5.100291in}{1.774061in}}%
\pgfpathmoveto{\pgfqpoint{5.100291in}{1.774061in}}%
\pgfpathlineto{\pgfqpoint{5.100291in}{1.774061in}}%
\pgfpathlineto{\pgfqpoint{5.100291in}{1.777010in}}%
\pgfpathlineto{\pgfqpoint{5.104832in}{1.777010in}}%
\pgfpathlineto{\pgfqpoint{5.104832in}{1.774061in}}%
\pgfpathmoveto{\pgfqpoint{5.100291in}{1.777010in}}%
\pgfpathlineto{\pgfqpoint{5.100291in}{1.777010in}}%
\pgfpathlineto{\pgfqpoint{5.100291in}{1.779960in}}%
\pgfpathlineto{\pgfqpoint{5.104832in}{1.779960in}}%
\pgfpathlineto{\pgfqpoint{5.104832in}{1.777010in}}%
\pgfpathmoveto{\pgfqpoint{5.104832in}{1.777010in}}%
\pgfpathlineto{\pgfqpoint{5.104832in}{1.777010in}}%
\pgfpathlineto{\pgfqpoint{5.104832in}{1.779960in}}%
\pgfpathlineto{\pgfqpoint{5.109374in}{1.779960in}}%
\pgfpathlineto{\pgfqpoint{5.109374in}{1.777010in}}%
\pgfpathmoveto{\pgfqpoint{5.104832in}{1.779960in}}%
\pgfpathlineto{\pgfqpoint{5.104832in}{1.779960in}}%
\pgfpathlineto{\pgfqpoint{5.104832in}{1.782909in}}%
\pgfpathlineto{\pgfqpoint{5.109374in}{1.782909in}}%
\pgfpathlineto{\pgfqpoint{5.109374in}{1.779960in}}%
\pgfpathmoveto{\pgfqpoint{5.104832in}{1.782909in}}%
\pgfpathlineto{\pgfqpoint{5.104832in}{1.782909in}}%
\pgfpathlineto{\pgfqpoint{5.104832in}{1.785858in}}%
\pgfpathlineto{\pgfqpoint{5.109374in}{1.785858in}}%
\pgfpathlineto{\pgfqpoint{5.109374in}{1.782909in}}%
\pgfpathmoveto{\pgfqpoint{5.109374in}{1.782909in}}%
\pgfpathlineto{\pgfqpoint{5.109374in}{1.782909in}}%
\pgfpathlineto{\pgfqpoint{5.109374in}{1.785858in}}%
\pgfpathlineto{\pgfqpoint{5.113914in}{1.785858in}}%
\pgfpathlineto{\pgfqpoint{5.113914in}{1.782909in}}%
\pgfpathmoveto{\pgfqpoint{5.109374in}{1.785858in}}%
\pgfpathlineto{\pgfqpoint{5.109374in}{1.785858in}}%
\pgfpathlineto{\pgfqpoint{5.109374in}{1.788808in}}%
\pgfpathlineto{\pgfqpoint{5.113914in}{1.788808in}}%
\pgfpathlineto{\pgfqpoint{5.113914in}{1.785858in}}%
\pgfpathmoveto{\pgfqpoint{5.113914in}{1.785858in}}%
\pgfpathlineto{\pgfqpoint{5.113914in}{1.785858in}}%
\pgfpathlineto{\pgfqpoint{5.113914in}{1.788808in}}%
\pgfpathlineto{\pgfqpoint{5.118455in}{1.788808in}}%
\pgfpathlineto{\pgfqpoint{5.118455in}{1.785858in}}%
\pgfpathmoveto{\pgfqpoint{5.113914in}{1.788808in}}%
\pgfpathlineto{\pgfqpoint{5.113914in}{1.788808in}}%
\pgfpathlineto{\pgfqpoint{5.113914in}{1.791757in}}%
\pgfpathlineto{\pgfqpoint{5.118455in}{1.791757in}}%
\pgfpathlineto{\pgfqpoint{5.118455in}{1.788808in}}%
\pgfpathmoveto{\pgfqpoint{5.118455in}{1.788808in}}%
\pgfpathlineto{\pgfqpoint{5.118455in}{1.788808in}}%
\pgfpathlineto{\pgfqpoint{5.118455in}{1.791757in}}%
\pgfpathlineto{\pgfqpoint{5.122996in}{1.791757in}}%
\pgfpathlineto{\pgfqpoint{5.122996in}{1.788808in}}%
\pgfpathmoveto{\pgfqpoint{5.118455in}{1.791757in}}%
\pgfpathlineto{\pgfqpoint{5.118455in}{1.791757in}}%
\pgfpathlineto{\pgfqpoint{5.118455in}{1.794706in}}%
\pgfpathlineto{\pgfqpoint{5.122996in}{1.794706in}}%
\pgfpathlineto{\pgfqpoint{5.122996in}{1.791757in}}%
\pgfpathmoveto{\pgfqpoint{5.122996in}{1.791757in}}%
\pgfpathlineto{\pgfqpoint{5.122996in}{1.791757in}}%
\pgfpathlineto{\pgfqpoint{5.122996in}{1.794706in}}%
\pgfpathlineto{\pgfqpoint{5.127537in}{1.794706in}}%
\pgfpathlineto{\pgfqpoint{5.127537in}{1.791757in}}%
\pgfpathmoveto{\pgfqpoint{5.122996in}{1.794706in}}%
\pgfpathlineto{\pgfqpoint{5.122996in}{1.794706in}}%
\pgfpathlineto{\pgfqpoint{5.122996in}{1.797655in}}%
\pgfpathlineto{\pgfqpoint{5.127537in}{1.797655in}}%
\pgfpathlineto{\pgfqpoint{5.127537in}{1.794706in}}%
\pgfpathmoveto{\pgfqpoint{5.127537in}{1.794706in}}%
\pgfpathlineto{\pgfqpoint{5.127537in}{1.794706in}}%
\pgfpathlineto{\pgfqpoint{5.127537in}{1.797655in}}%
\pgfpathlineto{\pgfqpoint{5.132078in}{1.797655in}}%
\pgfpathlineto{\pgfqpoint{5.132078in}{1.794706in}}%
\pgfpathmoveto{\pgfqpoint{5.127537in}{1.797655in}}%
\pgfpathlineto{\pgfqpoint{5.127537in}{1.797655in}}%
\pgfpathlineto{\pgfqpoint{5.127537in}{1.800605in}}%
\pgfpathlineto{\pgfqpoint{5.132078in}{1.800605in}}%
\pgfpathlineto{\pgfqpoint{5.132078in}{1.797655in}}%
\pgfpathmoveto{\pgfqpoint{5.127537in}{1.800605in}}%
\pgfpathlineto{\pgfqpoint{5.127537in}{1.800605in}}%
\pgfpathlineto{\pgfqpoint{5.127537in}{1.803554in}}%
\pgfpathlineto{\pgfqpoint{5.132078in}{1.803554in}}%
\pgfpathlineto{\pgfqpoint{5.132078in}{1.800605in}}%
\pgfpathmoveto{\pgfqpoint{5.132078in}{1.800605in}}%
\pgfpathlineto{\pgfqpoint{5.132078in}{1.800605in}}%
\pgfpathlineto{\pgfqpoint{5.132078in}{1.803554in}}%
\pgfpathlineto{\pgfqpoint{5.136619in}{1.803554in}}%
\pgfpathlineto{\pgfqpoint{5.136619in}{1.800605in}}%
\pgfpathmoveto{\pgfqpoint{5.132078in}{1.803554in}}%
\pgfpathlineto{\pgfqpoint{5.132078in}{1.803554in}}%
\pgfpathlineto{\pgfqpoint{5.132078in}{1.806503in}}%
\pgfpathlineto{\pgfqpoint{5.136619in}{1.806503in}}%
\pgfpathlineto{\pgfqpoint{5.136619in}{1.803554in}}%
\pgfpathmoveto{\pgfqpoint{5.136619in}{1.803554in}}%
\pgfpathlineto{\pgfqpoint{5.136619in}{1.803554in}}%
\pgfpathlineto{\pgfqpoint{5.136619in}{1.806503in}}%
\pgfpathlineto{\pgfqpoint{5.141160in}{1.806503in}}%
\pgfpathlineto{\pgfqpoint{5.141160in}{1.803554in}}%
\pgfpathmoveto{\pgfqpoint{5.136619in}{1.806503in}}%
\pgfpathlineto{\pgfqpoint{5.136619in}{1.806503in}}%
\pgfpathlineto{\pgfqpoint{5.136619in}{1.809453in}}%
\pgfpathlineto{\pgfqpoint{5.141160in}{1.809453in}}%
\pgfpathlineto{\pgfqpoint{5.141160in}{1.806503in}}%
\pgfpathmoveto{\pgfqpoint{5.141160in}{1.806503in}}%
\pgfpathlineto{\pgfqpoint{5.141160in}{1.806503in}}%
\pgfpathlineto{\pgfqpoint{5.141160in}{1.809453in}}%
\pgfpathlineto{\pgfqpoint{5.145701in}{1.809453in}}%
\pgfpathlineto{\pgfqpoint{5.145701in}{1.806503in}}%
\pgfpathmoveto{\pgfqpoint{5.141160in}{1.809453in}}%
\pgfpathlineto{\pgfqpoint{5.141160in}{1.809453in}}%
\pgfpathlineto{\pgfqpoint{5.141160in}{1.812402in}}%
\pgfpathlineto{\pgfqpoint{5.145701in}{1.812402in}}%
\pgfpathlineto{\pgfqpoint{5.145701in}{1.809453in}}%
\pgfpathmoveto{\pgfqpoint{5.145701in}{1.809453in}}%
\pgfpathlineto{\pgfqpoint{5.145701in}{1.809453in}}%
\pgfpathlineto{\pgfqpoint{5.145701in}{1.812402in}}%
\pgfpathlineto{\pgfqpoint{5.150242in}{1.812402in}}%
\pgfpathlineto{\pgfqpoint{5.150242in}{1.809453in}}%
\pgfpathmoveto{\pgfqpoint{5.145701in}{1.812402in}}%
\pgfpathlineto{\pgfqpoint{5.145701in}{1.812402in}}%
\pgfpathlineto{\pgfqpoint{5.145701in}{1.815351in}}%
\pgfpathlineto{\pgfqpoint{5.150242in}{1.815351in}}%
\pgfpathlineto{\pgfqpoint{5.150242in}{1.812402in}}%
\pgfpathmoveto{\pgfqpoint{5.150242in}{1.812402in}}%
\pgfpathlineto{\pgfqpoint{5.150242in}{1.812402in}}%
\pgfpathlineto{\pgfqpoint{5.150242in}{1.815351in}}%
\pgfpathlineto{\pgfqpoint{5.154783in}{1.815351in}}%
\pgfpathlineto{\pgfqpoint{5.154783in}{1.812402in}}%
\pgfpathmoveto{\pgfqpoint{5.150242in}{1.815351in}}%
\pgfpathlineto{\pgfqpoint{5.150242in}{1.815351in}}%
\pgfpathlineto{\pgfqpoint{5.150242in}{1.818301in}}%
\pgfpathlineto{\pgfqpoint{5.154783in}{1.818301in}}%
\pgfpathlineto{\pgfqpoint{5.154783in}{1.815351in}}%
\pgfpathmoveto{\pgfqpoint{5.150242in}{1.818301in}}%
\pgfpathlineto{\pgfqpoint{5.150242in}{1.818301in}}%
\pgfpathlineto{\pgfqpoint{5.150242in}{1.821250in}}%
\pgfpathlineto{\pgfqpoint{5.154783in}{1.821250in}}%
\pgfpathlineto{\pgfqpoint{5.154783in}{1.818301in}}%
\pgfpathmoveto{\pgfqpoint{5.154783in}{1.818301in}}%
\pgfpathlineto{\pgfqpoint{5.154783in}{1.818301in}}%
\pgfpathlineto{\pgfqpoint{5.154783in}{1.821250in}}%
\pgfpathlineto{\pgfqpoint{5.159324in}{1.821250in}}%
\pgfpathlineto{\pgfqpoint{5.159324in}{1.818301in}}%
\pgfpathmoveto{\pgfqpoint{5.154783in}{1.821250in}}%
\pgfpathlineto{\pgfqpoint{5.154783in}{1.821250in}}%
\pgfpathlineto{\pgfqpoint{5.154783in}{1.824199in}}%
\pgfpathlineto{\pgfqpoint{5.159324in}{1.824199in}}%
\pgfpathlineto{\pgfqpoint{5.159324in}{1.821250in}}%
\pgfpathmoveto{\pgfqpoint{5.159324in}{1.821250in}}%
\pgfpathlineto{\pgfqpoint{5.159324in}{1.821250in}}%
\pgfpathlineto{\pgfqpoint{5.159324in}{1.824199in}}%
\pgfpathlineto{\pgfqpoint{5.163865in}{1.824199in}}%
\pgfpathlineto{\pgfqpoint{5.163865in}{1.821250in}}%
\pgfpathmoveto{\pgfqpoint{5.159324in}{1.824199in}}%
\pgfpathlineto{\pgfqpoint{5.159324in}{1.824199in}}%
\pgfpathlineto{\pgfqpoint{5.159324in}{1.827148in}}%
\pgfpathlineto{\pgfqpoint{5.163865in}{1.827148in}}%
\pgfpathlineto{\pgfqpoint{5.163865in}{1.824199in}}%
\pgfpathmoveto{\pgfqpoint{5.163865in}{1.824199in}}%
\pgfpathlineto{\pgfqpoint{5.163865in}{1.824199in}}%
\pgfpathlineto{\pgfqpoint{5.163865in}{1.827148in}}%
\pgfpathlineto{\pgfqpoint{5.168406in}{1.827148in}}%
\pgfpathlineto{\pgfqpoint{5.168406in}{1.824199in}}%
\pgfpathmoveto{\pgfqpoint{5.163865in}{1.827148in}}%
\pgfpathlineto{\pgfqpoint{5.163865in}{1.827148in}}%
\pgfpathlineto{\pgfqpoint{5.163865in}{1.830097in}}%
\pgfpathlineto{\pgfqpoint{5.168406in}{1.830097in}}%
\pgfpathlineto{\pgfqpoint{5.168406in}{1.827148in}}%
\pgfpathmoveto{\pgfqpoint{5.168406in}{1.827148in}}%
\pgfpathlineto{\pgfqpoint{5.168406in}{1.827148in}}%
\pgfpathlineto{\pgfqpoint{5.168406in}{1.830097in}}%
\pgfpathlineto{\pgfqpoint{5.172947in}{1.830097in}}%
\pgfpathlineto{\pgfqpoint{5.172947in}{1.827148in}}%
\pgfpathmoveto{\pgfqpoint{5.168406in}{1.830097in}}%
\pgfpathlineto{\pgfqpoint{5.168406in}{1.830097in}}%
\pgfpathlineto{\pgfqpoint{5.168406in}{1.833047in}}%
\pgfpathlineto{\pgfqpoint{5.172947in}{1.833047in}}%
\pgfpathlineto{\pgfqpoint{5.172947in}{1.830097in}}%
\pgfpathmoveto{\pgfqpoint{5.172947in}{1.830097in}}%
\pgfpathlineto{\pgfqpoint{5.172947in}{1.830097in}}%
\pgfpathlineto{\pgfqpoint{5.172947in}{1.833047in}}%
\pgfpathlineto{\pgfqpoint{5.177488in}{1.833047in}}%
\pgfpathlineto{\pgfqpoint{5.177488in}{1.830097in}}%
\pgfpathmoveto{\pgfqpoint{5.172947in}{1.833047in}}%
\pgfpathlineto{\pgfqpoint{5.172947in}{1.833047in}}%
\pgfpathlineto{\pgfqpoint{5.172947in}{1.835996in}}%
\pgfpathlineto{\pgfqpoint{5.177488in}{1.835996in}}%
\pgfpathlineto{\pgfqpoint{5.177488in}{1.833047in}}%
\pgfpathmoveto{\pgfqpoint{5.172947in}{1.835996in}}%
\pgfpathlineto{\pgfqpoint{5.172947in}{1.835996in}}%
\pgfpathlineto{\pgfqpoint{5.172947in}{1.838945in}}%
\pgfpathlineto{\pgfqpoint{5.177488in}{1.838945in}}%
\pgfpathlineto{\pgfqpoint{5.177488in}{1.835996in}}%
\pgfpathmoveto{\pgfqpoint{5.177488in}{1.835996in}}%
\pgfpathlineto{\pgfqpoint{5.177488in}{1.835996in}}%
\pgfpathlineto{\pgfqpoint{5.177488in}{1.838945in}}%
\pgfpathlineto{\pgfqpoint{5.182029in}{1.838945in}}%
\pgfpathlineto{\pgfqpoint{5.182029in}{1.835996in}}%
\pgfpathmoveto{\pgfqpoint{5.177488in}{1.838945in}}%
\pgfpathlineto{\pgfqpoint{5.177488in}{1.838945in}}%
\pgfpathlineto{\pgfqpoint{5.177488in}{1.841894in}}%
\pgfpathlineto{\pgfqpoint{5.182029in}{1.841894in}}%
\pgfpathlineto{\pgfqpoint{5.182029in}{1.838945in}}%
\pgfpathmoveto{\pgfqpoint{5.182029in}{1.838945in}}%
\pgfpathlineto{\pgfqpoint{5.182029in}{1.838945in}}%
\pgfpathlineto{\pgfqpoint{5.182029in}{1.841894in}}%
\pgfpathlineto{\pgfqpoint{5.186570in}{1.841894in}}%
\pgfpathlineto{\pgfqpoint{5.186570in}{1.838945in}}%
\pgfpathmoveto{\pgfqpoint{5.182029in}{1.841894in}}%
\pgfpathlineto{\pgfqpoint{5.182029in}{1.841894in}}%
\pgfpathlineto{\pgfqpoint{5.182029in}{1.844843in}}%
\pgfpathlineto{\pgfqpoint{5.186570in}{1.844843in}}%
\pgfpathlineto{\pgfqpoint{5.186570in}{1.841894in}}%
\pgfpathmoveto{\pgfqpoint{5.186570in}{1.841894in}}%
\pgfpathlineto{\pgfqpoint{5.186570in}{1.841894in}}%
\pgfpathlineto{\pgfqpoint{5.186570in}{1.844843in}}%
\pgfpathlineto{\pgfqpoint{5.191111in}{1.844843in}}%
\pgfpathlineto{\pgfqpoint{5.191111in}{1.841894in}}%
\pgfpathmoveto{\pgfqpoint{5.186570in}{1.844843in}}%
\pgfpathlineto{\pgfqpoint{5.186570in}{1.844843in}}%
\pgfpathlineto{\pgfqpoint{5.186570in}{1.847793in}}%
\pgfpathlineto{\pgfqpoint{5.191111in}{1.847793in}}%
\pgfpathlineto{\pgfqpoint{5.191111in}{1.844843in}}%
\pgfpathmoveto{\pgfqpoint{5.191111in}{1.844843in}}%
\pgfpathlineto{\pgfqpoint{5.191111in}{1.844843in}}%
\pgfpathlineto{\pgfqpoint{5.191111in}{1.847793in}}%
\pgfpathlineto{\pgfqpoint{5.195652in}{1.847793in}}%
\pgfpathlineto{\pgfqpoint{5.195652in}{1.844843in}}%
\pgfpathmoveto{\pgfqpoint{5.191111in}{1.847793in}}%
\pgfpathlineto{\pgfqpoint{5.191111in}{1.847793in}}%
\pgfpathlineto{\pgfqpoint{5.191111in}{1.850742in}}%
\pgfpathlineto{\pgfqpoint{5.195652in}{1.850742in}}%
\pgfpathlineto{\pgfqpoint{5.195652in}{1.847793in}}%
\pgfpathmoveto{\pgfqpoint{5.195652in}{1.847793in}}%
\pgfpathlineto{\pgfqpoint{5.195652in}{1.847793in}}%
\pgfpathlineto{\pgfqpoint{5.195652in}{1.850742in}}%
\pgfpathlineto{\pgfqpoint{5.200193in}{1.850742in}}%
\pgfpathlineto{\pgfqpoint{5.200193in}{1.847793in}}%
\pgfpathmoveto{\pgfqpoint{5.195652in}{1.850742in}}%
\pgfpathlineto{\pgfqpoint{5.195652in}{1.850742in}}%
\pgfpathlineto{\pgfqpoint{5.195652in}{1.853691in}}%
\pgfpathlineto{\pgfqpoint{5.200193in}{1.853691in}}%
\pgfpathlineto{\pgfqpoint{5.200193in}{1.850742in}}%
\pgfpathmoveto{\pgfqpoint{5.195652in}{1.853691in}}%
\pgfpathlineto{\pgfqpoint{5.195652in}{1.853691in}}%
\pgfpathlineto{\pgfqpoint{5.195652in}{1.856640in}}%
\pgfpathlineto{\pgfqpoint{5.200193in}{1.856640in}}%
\pgfpathlineto{\pgfqpoint{5.200193in}{1.853691in}}%
\pgfpathmoveto{\pgfqpoint{5.200193in}{1.853691in}}%
\pgfpathlineto{\pgfqpoint{5.200193in}{1.853691in}}%
\pgfpathlineto{\pgfqpoint{5.200193in}{1.856640in}}%
\pgfpathlineto{\pgfqpoint{5.204734in}{1.856640in}}%
\pgfpathlineto{\pgfqpoint{5.204734in}{1.853691in}}%
\pgfpathmoveto{\pgfqpoint{5.200193in}{1.856640in}}%
\pgfpathlineto{\pgfqpoint{5.200193in}{1.856640in}}%
\pgfpathlineto{\pgfqpoint{5.200193in}{1.859589in}}%
\pgfpathlineto{\pgfqpoint{5.204734in}{1.859589in}}%
\pgfpathlineto{\pgfqpoint{5.204734in}{1.856640in}}%
\pgfpathmoveto{\pgfqpoint{5.204734in}{1.856640in}}%
\pgfpathlineto{\pgfqpoint{5.204734in}{1.856640in}}%
\pgfpathlineto{\pgfqpoint{5.204734in}{1.859589in}}%
\pgfpathlineto{\pgfqpoint{5.209275in}{1.859589in}}%
\pgfpathlineto{\pgfqpoint{5.209275in}{1.856640in}}%
\pgfpathmoveto{\pgfqpoint{5.204734in}{1.859589in}}%
\pgfpathlineto{\pgfqpoint{5.204734in}{1.859589in}}%
\pgfpathlineto{\pgfqpoint{5.204734in}{1.862539in}}%
\pgfpathlineto{\pgfqpoint{5.209275in}{1.862539in}}%
\pgfpathlineto{\pgfqpoint{5.209275in}{1.859589in}}%
\pgfpathmoveto{\pgfqpoint{5.209275in}{1.859589in}}%
\pgfpathlineto{\pgfqpoint{5.209275in}{1.859589in}}%
\pgfpathlineto{\pgfqpoint{5.209275in}{1.862539in}}%
\pgfpathlineto{\pgfqpoint{5.213816in}{1.862539in}}%
\pgfpathlineto{\pgfqpoint{5.213816in}{1.859589in}}%
\pgfpathmoveto{\pgfqpoint{5.209275in}{1.862539in}}%
\pgfpathlineto{\pgfqpoint{5.209275in}{1.862539in}}%
\pgfpathlineto{\pgfqpoint{5.209275in}{1.865488in}}%
\pgfpathlineto{\pgfqpoint{5.213816in}{1.865488in}}%
\pgfpathlineto{\pgfqpoint{5.213816in}{1.862539in}}%
\pgfpathmoveto{\pgfqpoint{5.213816in}{1.862539in}}%
\pgfpathlineto{\pgfqpoint{5.213816in}{1.862539in}}%
\pgfpathlineto{\pgfqpoint{5.213816in}{1.865488in}}%
\pgfpathlineto{\pgfqpoint{5.218357in}{1.865488in}}%
\pgfpathlineto{\pgfqpoint{5.218357in}{1.862539in}}%
\pgfpathmoveto{\pgfqpoint{5.213816in}{1.865488in}}%
\pgfpathlineto{\pgfqpoint{5.213816in}{1.865488in}}%
\pgfpathlineto{\pgfqpoint{5.213816in}{1.868437in}}%
\pgfpathlineto{\pgfqpoint{5.218357in}{1.868437in}}%
\pgfpathlineto{\pgfqpoint{5.218357in}{1.865488in}}%
\pgfpathmoveto{\pgfqpoint{5.218357in}{1.865488in}}%
\pgfpathlineto{\pgfqpoint{5.218357in}{1.865488in}}%
\pgfpathlineto{\pgfqpoint{5.218357in}{1.868437in}}%
\pgfpathlineto{\pgfqpoint{5.222898in}{1.868437in}}%
\pgfpathlineto{\pgfqpoint{5.222898in}{1.865488in}}%
\pgfpathmoveto{\pgfqpoint{5.218357in}{1.868437in}}%
\pgfpathlineto{\pgfqpoint{5.218357in}{1.868437in}}%
\pgfpathlineto{\pgfqpoint{5.218357in}{1.871386in}}%
\pgfpathlineto{\pgfqpoint{5.222898in}{1.871386in}}%
\pgfpathlineto{\pgfqpoint{5.222898in}{1.868437in}}%
\pgfpathmoveto{\pgfqpoint{5.218357in}{1.871386in}}%
\pgfpathlineto{\pgfqpoint{5.218357in}{1.871386in}}%
\pgfpathlineto{\pgfqpoint{5.218357in}{1.874335in}}%
\pgfpathlineto{\pgfqpoint{5.222898in}{1.874335in}}%
\pgfpathlineto{\pgfqpoint{5.222898in}{1.871386in}}%
\pgfpathmoveto{\pgfqpoint{5.222898in}{1.871386in}}%
\pgfpathlineto{\pgfqpoint{5.222898in}{1.871386in}}%
\pgfpathlineto{\pgfqpoint{5.222898in}{1.874335in}}%
\pgfpathlineto{\pgfqpoint{5.227439in}{1.874335in}}%
\pgfpathlineto{\pgfqpoint{5.227439in}{1.871386in}}%
\pgfpathmoveto{\pgfqpoint{5.222898in}{1.874335in}}%
\pgfpathlineto{\pgfqpoint{5.222898in}{1.874335in}}%
\pgfpathlineto{\pgfqpoint{5.222898in}{1.877285in}}%
\pgfpathlineto{\pgfqpoint{5.227439in}{1.877285in}}%
\pgfpathlineto{\pgfqpoint{5.227439in}{1.874335in}}%
\pgfpathmoveto{\pgfqpoint{5.227439in}{1.874335in}}%
\pgfpathlineto{\pgfqpoint{5.227439in}{1.874335in}}%
\pgfpathlineto{\pgfqpoint{5.227439in}{1.877285in}}%
\pgfpathlineto{\pgfqpoint{5.231979in}{1.877285in}}%
\pgfpathlineto{\pgfqpoint{5.231979in}{1.874335in}}%
\pgfpathmoveto{\pgfqpoint{5.227439in}{1.877285in}}%
\pgfpathlineto{\pgfqpoint{5.227439in}{1.877285in}}%
\pgfpathlineto{\pgfqpoint{5.227439in}{1.880234in}}%
\pgfpathlineto{\pgfqpoint{5.231979in}{1.880234in}}%
\pgfpathlineto{\pgfqpoint{5.231979in}{1.877285in}}%
\pgfpathmoveto{\pgfqpoint{5.231979in}{1.877285in}}%
\pgfpathlineto{\pgfqpoint{5.231979in}{1.877285in}}%
\pgfpathlineto{\pgfqpoint{5.231979in}{1.880234in}}%
\pgfpathlineto{\pgfqpoint{5.236520in}{1.880234in}}%
\pgfpathlineto{\pgfqpoint{5.236520in}{1.877285in}}%
\pgfpathmoveto{\pgfqpoint{5.231979in}{1.880234in}}%
\pgfpathlineto{\pgfqpoint{5.231979in}{1.880234in}}%
\pgfpathlineto{\pgfqpoint{5.231979in}{1.883183in}}%
\pgfpathlineto{\pgfqpoint{5.236520in}{1.883183in}}%
\pgfpathlineto{\pgfqpoint{5.236520in}{1.880234in}}%
\pgfpathmoveto{\pgfqpoint{5.236520in}{1.880234in}}%
\pgfpathlineto{\pgfqpoint{5.236520in}{1.880234in}}%
\pgfpathlineto{\pgfqpoint{5.236520in}{1.883183in}}%
\pgfpathlineto{\pgfqpoint{5.241061in}{1.883183in}}%
\pgfpathlineto{\pgfqpoint{5.241061in}{1.880234in}}%
\pgfpathmoveto{\pgfqpoint{5.236520in}{1.883183in}}%
\pgfpathlineto{\pgfqpoint{5.236520in}{1.883183in}}%
\pgfpathlineto{\pgfqpoint{5.236520in}{1.886132in}}%
\pgfpathlineto{\pgfqpoint{5.241061in}{1.886132in}}%
\pgfpathlineto{\pgfqpoint{5.241061in}{1.883183in}}%
\pgfpathmoveto{\pgfqpoint{5.241061in}{1.883183in}}%
\pgfpathlineto{\pgfqpoint{5.241061in}{1.883183in}}%
\pgfpathlineto{\pgfqpoint{5.241061in}{1.886132in}}%
\pgfpathlineto{\pgfqpoint{5.245602in}{1.886132in}}%
\pgfpathlineto{\pgfqpoint{5.245602in}{1.883183in}}%
\pgfpathmoveto{\pgfqpoint{5.241061in}{1.886132in}}%
\pgfpathlineto{\pgfqpoint{5.241061in}{1.886132in}}%
\pgfpathlineto{\pgfqpoint{5.241061in}{1.889081in}}%
\pgfpathlineto{\pgfqpoint{5.245602in}{1.889081in}}%
\pgfpathlineto{\pgfqpoint{5.245602in}{1.886132in}}%
\pgfpathmoveto{\pgfqpoint{5.241061in}{1.889081in}}%
\pgfpathlineto{\pgfqpoint{5.241061in}{1.889081in}}%
\pgfpathlineto{\pgfqpoint{5.241061in}{1.892031in}}%
\pgfpathlineto{\pgfqpoint{5.245602in}{1.892031in}}%
\pgfpathlineto{\pgfqpoint{5.245602in}{1.889081in}}%
\pgfpathmoveto{\pgfqpoint{5.245602in}{1.889081in}}%
\pgfpathlineto{\pgfqpoint{5.245602in}{1.889081in}}%
\pgfpathlineto{\pgfqpoint{5.245602in}{1.892031in}}%
\pgfpathlineto{\pgfqpoint{5.250143in}{1.892031in}}%
\pgfpathlineto{\pgfqpoint{5.250143in}{1.889081in}}%
\pgfpathmoveto{\pgfqpoint{5.245602in}{1.892031in}}%
\pgfpathlineto{\pgfqpoint{5.245602in}{1.892031in}}%
\pgfpathlineto{\pgfqpoint{5.245602in}{1.894980in}}%
\pgfpathlineto{\pgfqpoint{5.250143in}{1.894980in}}%
\pgfpathlineto{\pgfqpoint{5.250143in}{1.892031in}}%
\pgfpathmoveto{\pgfqpoint{5.250143in}{1.892031in}}%
\pgfpathlineto{\pgfqpoint{5.250143in}{1.892031in}}%
\pgfpathlineto{\pgfqpoint{5.250143in}{1.894980in}}%
\pgfpathlineto{\pgfqpoint{5.254684in}{1.894980in}}%
\pgfpathlineto{\pgfqpoint{5.254684in}{1.892031in}}%
\pgfpathmoveto{\pgfqpoint{5.250143in}{1.894980in}}%
\pgfpathlineto{\pgfqpoint{5.250143in}{1.894980in}}%
\pgfpathlineto{\pgfqpoint{5.250143in}{1.897929in}}%
\pgfpathlineto{\pgfqpoint{5.254684in}{1.897929in}}%
\pgfpathlineto{\pgfqpoint{5.254684in}{1.894980in}}%
\pgfpathmoveto{\pgfqpoint{5.254684in}{1.894980in}}%
\pgfpathlineto{\pgfqpoint{5.254684in}{1.894980in}}%
\pgfpathlineto{\pgfqpoint{5.254684in}{1.897929in}}%
\pgfpathlineto{\pgfqpoint{5.259225in}{1.897929in}}%
\pgfpathlineto{\pgfqpoint{5.259225in}{1.894980in}}%
\pgfpathmoveto{\pgfqpoint{5.254684in}{1.897929in}}%
\pgfpathlineto{\pgfqpoint{5.254684in}{1.897929in}}%
\pgfpathlineto{\pgfqpoint{5.254684in}{1.900878in}}%
\pgfpathlineto{\pgfqpoint{5.259225in}{1.900878in}}%
\pgfpathlineto{\pgfqpoint{5.259225in}{1.897929in}}%
\pgfpathmoveto{\pgfqpoint{5.259225in}{1.897929in}}%
\pgfpathlineto{\pgfqpoint{5.259225in}{1.897929in}}%
\pgfpathlineto{\pgfqpoint{5.259225in}{1.900878in}}%
\pgfpathlineto{\pgfqpoint{5.263767in}{1.900878in}}%
\pgfpathlineto{\pgfqpoint{5.263767in}{1.897929in}}%
\pgfpathmoveto{\pgfqpoint{5.259225in}{1.900878in}}%
\pgfpathlineto{\pgfqpoint{5.259225in}{1.900878in}}%
\pgfpathlineto{\pgfqpoint{5.259225in}{1.903827in}}%
\pgfpathlineto{\pgfqpoint{5.263767in}{1.903827in}}%
\pgfpathlineto{\pgfqpoint{5.263767in}{1.900878in}}%
\pgfpathmoveto{\pgfqpoint{5.263767in}{1.900878in}}%
\pgfpathlineto{\pgfqpoint{5.263767in}{1.900878in}}%
\pgfpathlineto{\pgfqpoint{5.263767in}{1.903827in}}%
\pgfpathlineto{\pgfqpoint{5.268308in}{1.903827in}}%
\pgfpathlineto{\pgfqpoint{5.268308in}{1.900878in}}%
\pgfpathmoveto{\pgfqpoint{5.263767in}{1.903827in}}%
\pgfpathlineto{\pgfqpoint{5.263767in}{1.903827in}}%
\pgfpathlineto{\pgfqpoint{5.263767in}{1.906777in}}%
\pgfpathlineto{\pgfqpoint{5.268308in}{1.906777in}}%
\pgfpathlineto{\pgfqpoint{5.268308in}{1.903827in}}%
\pgfpathmoveto{\pgfqpoint{5.263767in}{1.906777in}}%
\pgfpathlineto{\pgfqpoint{5.263767in}{1.906777in}}%
\pgfpathlineto{\pgfqpoint{5.263767in}{1.909726in}}%
\pgfpathlineto{\pgfqpoint{5.268308in}{1.909726in}}%
\pgfpathlineto{\pgfqpoint{5.268308in}{1.906777in}}%
\pgfpathmoveto{\pgfqpoint{5.268308in}{1.906777in}}%
\pgfpathlineto{\pgfqpoint{5.268308in}{1.906777in}}%
\pgfpathlineto{\pgfqpoint{5.268308in}{1.909726in}}%
\pgfpathlineto{\pgfqpoint{5.272849in}{1.909726in}}%
\pgfpathlineto{\pgfqpoint{5.272849in}{1.906777in}}%
\pgfpathmoveto{\pgfqpoint{5.268308in}{1.909726in}}%
\pgfpathlineto{\pgfqpoint{5.268308in}{1.909726in}}%
\pgfpathlineto{\pgfqpoint{5.268308in}{1.912675in}}%
\pgfpathlineto{\pgfqpoint{5.272849in}{1.912675in}}%
\pgfpathlineto{\pgfqpoint{5.272849in}{1.909726in}}%
\pgfpathmoveto{\pgfqpoint{5.272849in}{1.909726in}}%
\pgfpathlineto{\pgfqpoint{5.272849in}{1.909726in}}%
\pgfpathlineto{\pgfqpoint{5.272849in}{1.912675in}}%
\pgfpathlineto{\pgfqpoint{5.277390in}{1.912675in}}%
\pgfpathlineto{\pgfqpoint{5.277390in}{1.909726in}}%
\pgfpathmoveto{\pgfqpoint{5.272849in}{1.912675in}}%
\pgfpathlineto{\pgfqpoint{5.272849in}{1.912675in}}%
\pgfpathlineto{\pgfqpoint{5.272849in}{1.915624in}}%
\pgfpathlineto{\pgfqpoint{5.277390in}{1.915624in}}%
\pgfpathlineto{\pgfqpoint{5.277390in}{1.912675in}}%
\pgfpathmoveto{\pgfqpoint{5.277390in}{1.912675in}}%
\pgfpathlineto{\pgfqpoint{5.277390in}{1.912675in}}%
\pgfpathlineto{\pgfqpoint{5.277390in}{1.915624in}}%
\pgfpathlineto{\pgfqpoint{5.281932in}{1.915624in}}%
\pgfpathlineto{\pgfqpoint{5.281932in}{1.912675in}}%
\pgfpathmoveto{\pgfqpoint{5.277390in}{1.915624in}}%
\pgfpathlineto{\pgfqpoint{5.277390in}{1.915624in}}%
\pgfpathlineto{\pgfqpoint{5.277390in}{1.918574in}}%
\pgfpathlineto{\pgfqpoint{5.281932in}{1.918574in}}%
\pgfpathlineto{\pgfqpoint{5.281932in}{1.915624in}}%
\pgfpathmoveto{\pgfqpoint{5.281932in}{1.915624in}}%
\pgfpathlineto{\pgfqpoint{5.281932in}{1.915624in}}%
\pgfpathlineto{\pgfqpoint{5.281932in}{1.918574in}}%
\pgfpathlineto{\pgfqpoint{5.286473in}{1.918574in}}%
\pgfpathlineto{\pgfqpoint{5.286473in}{1.915624in}}%
\pgfpathmoveto{\pgfqpoint{5.281932in}{1.918574in}}%
\pgfpathlineto{\pgfqpoint{5.281932in}{1.918574in}}%
\pgfpathlineto{\pgfqpoint{5.281932in}{1.921523in}}%
\pgfpathlineto{\pgfqpoint{5.286473in}{1.921523in}}%
\pgfpathlineto{\pgfqpoint{5.286473in}{1.918574in}}%
\pgfpathmoveto{\pgfqpoint{5.286473in}{1.918574in}}%
\pgfpathlineto{\pgfqpoint{5.286473in}{1.918574in}}%
\pgfpathlineto{\pgfqpoint{5.286473in}{1.921523in}}%
\pgfpathlineto{\pgfqpoint{5.291014in}{1.921523in}}%
\pgfpathlineto{\pgfqpoint{5.291014in}{1.918574in}}%
\pgfpathmoveto{\pgfqpoint{5.286473in}{1.921523in}}%
\pgfpathlineto{\pgfqpoint{5.286473in}{1.921523in}}%
\pgfpathlineto{\pgfqpoint{5.286473in}{1.924472in}}%
\pgfpathlineto{\pgfqpoint{5.291014in}{1.924472in}}%
\pgfpathlineto{\pgfqpoint{5.291014in}{1.921523in}}%
\pgfpathmoveto{\pgfqpoint{5.286473in}{1.924472in}}%
\pgfpathlineto{\pgfqpoint{5.286473in}{1.924472in}}%
\pgfpathlineto{\pgfqpoint{5.286473in}{1.927421in}}%
\pgfpathlineto{\pgfqpoint{5.291014in}{1.927421in}}%
\pgfpathlineto{\pgfqpoint{5.291014in}{1.924472in}}%
\pgfpathmoveto{\pgfqpoint{5.291014in}{1.924472in}}%
\pgfpathlineto{\pgfqpoint{5.291014in}{1.924472in}}%
\pgfpathlineto{\pgfqpoint{5.291014in}{1.927421in}}%
\pgfpathlineto{\pgfqpoint{5.295555in}{1.927421in}}%
\pgfpathlineto{\pgfqpoint{5.295555in}{1.924472in}}%
\pgfpathmoveto{\pgfqpoint{5.291014in}{1.927421in}}%
\pgfpathlineto{\pgfqpoint{5.291014in}{1.927421in}}%
\pgfpathlineto{\pgfqpoint{5.291014in}{1.930371in}}%
\pgfpathlineto{\pgfqpoint{5.295555in}{1.930371in}}%
\pgfpathlineto{\pgfqpoint{5.295555in}{1.927421in}}%
\pgfpathmoveto{\pgfqpoint{5.295555in}{1.927421in}}%
\pgfpathlineto{\pgfqpoint{5.295555in}{1.927421in}}%
\pgfpathlineto{\pgfqpoint{5.295555in}{1.930371in}}%
\pgfpathlineto{\pgfqpoint{5.300097in}{1.930371in}}%
\pgfpathlineto{\pgfqpoint{5.300097in}{1.927421in}}%
\pgfpathmoveto{\pgfqpoint{5.295555in}{1.930371in}}%
\pgfpathlineto{\pgfqpoint{5.295555in}{1.930371in}}%
\pgfpathlineto{\pgfqpoint{5.295555in}{1.933320in}}%
\pgfpathlineto{\pgfqpoint{5.300097in}{1.933320in}}%
\pgfpathlineto{\pgfqpoint{5.300097in}{1.930371in}}%
\pgfpathmoveto{\pgfqpoint{5.300097in}{1.930371in}}%
\pgfpathlineto{\pgfqpoint{5.300097in}{1.930371in}}%
\pgfpathlineto{\pgfqpoint{5.300097in}{1.933320in}}%
\pgfpathlineto{\pgfqpoint{5.304638in}{1.933320in}}%
\pgfpathlineto{\pgfqpoint{5.304638in}{1.930371in}}%
\pgfpathmoveto{\pgfqpoint{5.300097in}{1.933320in}}%
\pgfpathlineto{\pgfqpoint{5.300097in}{1.933320in}}%
\pgfpathlineto{\pgfqpoint{5.300097in}{1.936269in}}%
\pgfpathlineto{\pgfqpoint{5.304638in}{1.936269in}}%
\pgfpathlineto{\pgfqpoint{5.304638in}{1.933320in}}%
\pgfpathmoveto{\pgfqpoint{5.304638in}{1.933320in}}%
\pgfpathlineto{\pgfqpoint{5.304638in}{1.933320in}}%
\pgfpathlineto{\pgfqpoint{5.304638in}{1.936269in}}%
\pgfpathlineto{\pgfqpoint{5.309179in}{1.936269in}}%
\pgfpathlineto{\pgfqpoint{5.309179in}{1.933320in}}%
\pgfpathmoveto{\pgfqpoint{5.304638in}{1.936269in}}%
\pgfpathlineto{\pgfqpoint{5.304638in}{1.936269in}}%
\pgfpathlineto{\pgfqpoint{5.304638in}{1.939219in}}%
\pgfpathlineto{\pgfqpoint{5.309179in}{1.939219in}}%
\pgfpathlineto{\pgfqpoint{5.309179in}{1.936269in}}%
\pgfpathmoveto{\pgfqpoint{5.309179in}{1.936269in}}%
\pgfpathlineto{\pgfqpoint{5.309179in}{1.936269in}}%
\pgfpathlineto{\pgfqpoint{5.309179in}{1.939219in}}%
\pgfpathlineto{\pgfqpoint{5.313720in}{1.939219in}}%
\pgfpathlineto{\pgfqpoint{5.313720in}{1.936269in}}%
\pgfpathmoveto{\pgfqpoint{5.309179in}{1.939219in}}%
\pgfpathlineto{\pgfqpoint{5.309179in}{1.939219in}}%
\pgfpathlineto{\pgfqpoint{5.309179in}{1.942168in}}%
\pgfpathlineto{\pgfqpoint{5.313720in}{1.942168in}}%
\pgfpathlineto{\pgfqpoint{5.313720in}{1.939219in}}%
\pgfpathmoveto{\pgfqpoint{5.309179in}{1.942168in}}%
\pgfpathlineto{\pgfqpoint{5.309179in}{1.942168in}}%
\pgfpathlineto{\pgfqpoint{5.309179in}{1.945117in}}%
\pgfpathlineto{\pgfqpoint{5.313720in}{1.945117in}}%
\pgfpathlineto{\pgfqpoint{5.313720in}{1.942168in}}%
\pgfpathmoveto{\pgfqpoint{5.313720in}{1.942168in}}%
\pgfpathlineto{\pgfqpoint{5.313720in}{1.942168in}}%
\pgfpathlineto{\pgfqpoint{5.313720in}{1.945117in}}%
\pgfpathlineto{\pgfqpoint{5.318261in}{1.945117in}}%
\pgfpathlineto{\pgfqpoint{5.318261in}{1.942168in}}%
\pgfpathmoveto{\pgfqpoint{5.313720in}{1.945117in}}%
\pgfpathlineto{\pgfqpoint{5.313720in}{1.945117in}}%
\pgfpathlineto{\pgfqpoint{5.313720in}{1.948067in}}%
\pgfpathlineto{\pgfqpoint{5.318261in}{1.948067in}}%
\pgfpathlineto{\pgfqpoint{5.318261in}{1.945117in}}%
\pgfpathmoveto{\pgfqpoint{5.318261in}{1.945117in}}%
\pgfpathlineto{\pgfqpoint{5.318261in}{1.945117in}}%
\pgfpathlineto{\pgfqpoint{5.318261in}{1.948067in}}%
\pgfpathlineto{\pgfqpoint{5.322803in}{1.948067in}}%
\pgfpathlineto{\pgfqpoint{5.322803in}{1.945117in}}%
\pgfpathmoveto{\pgfqpoint{5.318261in}{1.948067in}}%
\pgfpathlineto{\pgfqpoint{5.318261in}{1.948067in}}%
\pgfpathlineto{\pgfqpoint{5.318261in}{1.951016in}}%
\pgfpathlineto{\pgfqpoint{5.322803in}{1.951016in}}%
\pgfpathlineto{\pgfqpoint{5.322803in}{1.948067in}}%
\pgfpathmoveto{\pgfqpoint{5.322803in}{1.948067in}}%
\pgfpathlineto{\pgfqpoint{5.322803in}{1.948067in}}%
\pgfpathlineto{\pgfqpoint{5.322803in}{1.951016in}}%
\pgfpathlineto{\pgfqpoint{5.327344in}{1.951016in}}%
\pgfpathlineto{\pgfqpoint{5.327344in}{1.948067in}}%
\pgfpathmoveto{\pgfqpoint{5.322803in}{1.951016in}}%
\pgfpathlineto{\pgfqpoint{5.322803in}{1.951016in}}%
\pgfpathlineto{\pgfqpoint{5.322803in}{1.953965in}}%
\pgfpathlineto{\pgfqpoint{5.327344in}{1.953965in}}%
\pgfpathlineto{\pgfqpoint{5.327344in}{1.951016in}}%
\pgfpathmoveto{\pgfqpoint{5.327344in}{1.951016in}}%
\pgfpathlineto{\pgfqpoint{5.327344in}{1.951016in}}%
\pgfpathlineto{\pgfqpoint{5.327344in}{1.953965in}}%
\pgfpathlineto{\pgfqpoint{5.331885in}{1.953965in}}%
\pgfpathlineto{\pgfqpoint{5.331885in}{1.951016in}}%
\pgfpathmoveto{\pgfqpoint{5.327344in}{1.953965in}}%
\pgfpathlineto{\pgfqpoint{5.327344in}{1.953965in}}%
\pgfpathlineto{\pgfqpoint{5.327344in}{1.956914in}}%
\pgfpathlineto{\pgfqpoint{5.331885in}{1.956914in}}%
\pgfpathlineto{\pgfqpoint{5.331885in}{1.953965in}}%
\pgfpathmoveto{\pgfqpoint{5.331885in}{1.953965in}}%
\pgfpathlineto{\pgfqpoint{5.331885in}{1.953965in}}%
\pgfpathlineto{\pgfqpoint{5.331885in}{1.956914in}}%
\pgfpathlineto{\pgfqpoint{5.336426in}{1.956914in}}%
\pgfpathlineto{\pgfqpoint{5.336426in}{1.953965in}}%
\pgfpathmoveto{\pgfqpoint{5.331885in}{1.956914in}}%
\pgfpathlineto{\pgfqpoint{5.331885in}{1.956914in}}%
\pgfpathlineto{\pgfqpoint{5.331885in}{1.959864in}}%
\pgfpathlineto{\pgfqpoint{5.336426in}{1.959864in}}%
\pgfpathlineto{\pgfqpoint{5.336426in}{1.956914in}}%
\pgfpathmoveto{\pgfqpoint{5.331885in}{1.959864in}}%
\pgfpathlineto{\pgfqpoint{5.331885in}{1.959864in}}%
\pgfpathlineto{\pgfqpoint{5.331885in}{1.962813in}}%
\pgfpathlineto{\pgfqpoint{5.336426in}{1.962813in}}%
\pgfpathlineto{\pgfqpoint{5.336426in}{1.959864in}}%
\pgfpathmoveto{\pgfqpoint{5.336426in}{1.959864in}}%
\pgfpathlineto{\pgfqpoint{5.336426in}{1.959864in}}%
\pgfpathlineto{\pgfqpoint{5.336426in}{1.962813in}}%
\pgfpathlineto{\pgfqpoint{5.340968in}{1.962813in}}%
\pgfpathlineto{\pgfqpoint{5.340968in}{1.959864in}}%
\pgfpathmoveto{\pgfqpoint{5.336426in}{1.962813in}}%
\pgfpathlineto{\pgfqpoint{5.336426in}{1.962813in}}%
\pgfpathlineto{\pgfqpoint{5.336426in}{1.965762in}}%
\pgfpathlineto{\pgfqpoint{5.340968in}{1.965762in}}%
\pgfpathlineto{\pgfqpoint{5.340968in}{1.962813in}}%
\pgfpathmoveto{\pgfqpoint{5.340968in}{1.962813in}}%
\pgfpathlineto{\pgfqpoint{5.340968in}{1.962813in}}%
\pgfpathlineto{\pgfqpoint{5.340968in}{1.965762in}}%
\pgfpathlineto{\pgfqpoint{5.345509in}{1.965762in}}%
\pgfpathlineto{\pgfqpoint{5.345509in}{1.962813in}}%
\pgfpathmoveto{\pgfqpoint{5.340968in}{1.965762in}}%
\pgfpathlineto{\pgfqpoint{5.340968in}{1.965762in}}%
\pgfpathlineto{\pgfqpoint{5.340968in}{1.968712in}}%
\pgfpathlineto{\pgfqpoint{5.345509in}{1.968712in}}%
\pgfpathlineto{\pgfqpoint{5.345509in}{1.965762in}}%
\pgfpathmoveto{\pgfqpoint{5.345509in}{1.965762in}}%
\pgfpathlineto{\pgfqpoint{5.345509in}{1.965762in}}%
\pgfpathlineto{\pgfqpoint{5.345509in}{1.968712in}}%
\pgfpathlineto{\pgfqpoint{5.350050in}{1.968712in}}%
\pgfpathlineto{\pgfqpoint{5.350050in}{1.965762in}}%
\pgfpathmoveto{\pgfqpoint{5.345509in}{1.968712in}}%
\pgfpathlineto{\pgfqpoint{5.345509in}{1.968712in}}%
\pgfpathlineto{\pgfqpoint{5.345509in}{1.971661in}}%
\pgfpathlineto{\pgfqpoint{5.350050in}{1.971661in}}%
\pgfpathlineto{\pgfqpoint{5.350050in}{1.968712in}}%
\pgfpathmoveto{\pgfqpoint{5.350050in}{1.968712in}}%
\pgfpathlineto{\pgfqpoint{5.350050in}{1.968712in}}%
\pgfpathlineto{\pgfqpoint{5.350050in}{1.971661in}}%
\pgfpathlineto{\pgfqpoint{5.354591in}{1.971661in}}%
\pgfpathlineto{\pgfqpoint{5.354591in}{1.968712in}}%
\pgfpathmoveto{\pgfqpoint{5.350050in}{1.971661in}}%
\pgfpathlineto{\pgfqpoint{5.350050in}{1.971661in}}%
\pgfpathlineto{\pgfqpoint{5.350050in}{1.974610in}}%
\pgfpathlineto{\pgfqpoint{5.354591in}{1.974610in}}%
\pgfpathlineto{\pgfqpoint{5.354591in}{1.971661in}}%
\pgfpathmoveto{\pgfqpoint{5.350050in}{1.974610in}}%
\pgfpathlineto{\pgfqpoint{5.350050in}{1.974610in}}%
\pgfpathlineto{\pgfqpoint{5.350050in}{1.977560in}}%
\pgfpathlineto{\pgfqpoint{5.354591in}{1.977560in}}%
\pgfpathlineto{\pgfqpoint{5.354591in}{1.974610in}}%
\pgfpathmoveto{\pgfqpoint{5.354591in}{1.974610in}}%
\pgfpathlineto{\pgfqpoint{5.354591in}{1.974610in}}%
\pgfpathlineto{\pgfqpoint{5.354591in}{1.977560in}}%
\pgfpathlineto{\pgfqpoint{5.359132in}{1.977560in}}%
\pgfpathlineto{\pgfqpoint{5.359132in}{1.974610in}}%
\pgfpathmoveto{\pgfqpoint{5.354591in}{1.977560in}}%
\pgfpathlineto{\pgfqpoint{5.354591in}{1.977560in}}%
\pgfpathlineto{\pgfqpoint{5.354591in}{1.980509in}}%
\pgfpathlineto{\pgfqpoint{5.359132in}{1.980509in}}%
\pgfpathlineto{\pgfqpoint{5.359132in}{1.977560in}}%
\pgfpathmoveto{\pgfqpoint{5.359132in}{1.977560in}}%
\pgfpathlineto{\pgfqpoint{5.359132in}{1.977560in}}%
\pgfpathlineto{\pgfqpoint{5.359132in}{1.980509in}}%
\pgfpathlineto{\pgfqpoint{5.363674in}{1.980509in}}%
\pgfpathlineto{\pgfqpoint{5.363674in}{1.977560in}}%
\pgfpathmoveto{\pgfqpoint{5.359132in}{1.980509in}}%
\pgfpathlineto{\pgfqpoint{5.359132in}{1.980509in}}%
\pgfpathlineto{\pgfqpoint{5.359132in}{1.983458in}}%
\pgfpathlineto{\pgfqpoint{5.363674in}{1.983458in}}%
\pgfpathlineto{\pgfqpoint{5.363674in}{1.980509in}}%
\pgfpathmoveto{\pgfqpoint{5.363674in}{1.980509in}}%
\pgfpathlineto{\pgfqpoint{5.363674in}{1.980509in}}%
\pgfpathlineto{\pgfqpoint{5.363674in}{1.983458in}}%
\pgfpathlineto{\pgfqpoint{5.368215in}{1.983458in}}%
\pgfpathlineto{\pgfqpoint{5.368215in}{1.980509in}}%
\pgfpathmoveto{\pgfqpoint{5.363674in}{1.983458in}}%
\pgfpathlineto{\pgfqpoint{5.363674in}{1.983458in}}%
\pgfpathlineto{\pgfqpoint{5.363674in}{1.986407in}}%
\pgfpathlineto{\pgfqpoint{5.368215in}{1.986407in}}%
\pgfpathlineto{\pgfqpoint{5.368215in}{1.983458in}}%
\pgfpathmoveto{\pgfqpoint{5.368215in}{1.983458in}}%
\pgfpathlineto{\pgfqpoint{5.368215in}{1.983458in}}%
\pgfpathlineto{\pgfqpoint{5.368215in}{1.986407in}}%
\pgfpathlineto{\pgfqpoint{5.372756in}{1.986407in}}%
\pgfpathlineto{\pgfqpoint{5.372756in}{1.983458in}}%
\pgfpathmoveto{\pgfqpoint{5.368215in}{1.986407in}}%
\pgfpathlineto{\pgfqpoint{5.368215in}{1.986407in}}%
\pgfpathlineto{\pgfqpoint{5.368215in}{1.989357in}}%
\pgfpathlineto{\pgfqpoint{5.372756in}{1.989357in}}%
\pgfpathlineto{\pgfqpoint{5.372756in}{1.986407in}}%
\pgfpathmoveto{\pgfqpoint{5.372756in}{1.986407in}}%
\pgfpathlineto{\pgfqpoint{5.372756in}{1.986407in}}%
\pgfpathlineto{\pgfqpoint{5.372756in}{1.989357in}}%
\pgfpathlineto{\pgfqpoint{5.377297in}{1.989357in}}%
\pgfpathlineto{\pgfqpoint{5.377297in}{1.986407in}}%
\pgfpathmoveto{\pgfqpoint{5.372756in}{1.989357in}}%
\pgfpathlineto{\pgfqpoint{5.372756in}{1.989357in}}%
\pgfpathlineto{\pgfqpoint{5.372756in}{1.992306in}}%
\pgfpathlineto{\pgfqpoint{5.377297in}{1.992306in}}%
\pgfpathlineto{\pgfqpoint{5.377297in}{1.989357in}}%
\pgfpathmoveto{\pgfqpoint{5.372756in}{1.992306in}}%
\pgfpathlineto{\pgfqpoint{5.372756in}{1.992306in}}%
\pgfpathlineto{\pgfqpoint{5.372756in}{1.995255in}}%
\pgfpathlineto{\pgfqpoint{5.377297in}{1.995255in}}%
\pgfpathlineto{\pgfqpoint{5.377297in}{1.992306in}}%
\pgfpathmoveto{\pgfqpoint{5.377297in}{1.992306in}}%
\pgfpathlineto{\pgfqpoint{5.377297in}{1.992306in}}%
\pgfpathlineto{\pgfqpoint{5.377297in}{1.995255in}}%
\pgfpathlineto{\pgfqpoint{5.381839in}{1.995255in}}%
\pgfpathlineto{\pgfqpoint{5.381839in}{1.992306in}}%
\pgfpathmoveto{\pgfqpoint{5.377297in}{1.995255in}}%
\pgfpathlineto{\pgfqpoint{5.377297in}{1.995255in}}%
\pgfpathlineto{\pgfqpoint{5.377297in}{1.998205in}}%
\pgfpathlineto{\pgfqpoint{5.381839in}{1.998205in}}%
\pgfpathlineto{\pgfqpoint{5.381839in}{1.995255in}}%
\pgfpathmoveto{\pgfqpoint{5.381839in}{1.995255in}}%
\pgfpathlineto{\pgfqpoint{5.381839in}{1.995255in}}%
\pgfpathlineto{\pgfqpoint{5.381839in}{1.998205in}}%
\pgfpathlineto{\pgfqpoint{5.386380in}{1.998205in}}%
\pgfpathlineto{\pgfqpoint{5.386380in}{1.995255in}}%
\pgfpathmoveto{\pgfqpoint{5.381839in}{1.998205in}}%
\pgfpathlineto{\pgfqpoint{5.381839in}{1.998205in}}%
\pgfpathlineto{\pgfqpoint{5.381839in}{2.001154in}}%
\pgfpathlineto{\pgfqpoint{5.386380in}{2.001154in}}%
\pgfpathlineto{\pgfqpoint{5.386380in}{1.998205in}}%
\pgfpathmoveto{\pgfqpoint{5.386380in}{1.998205in}}%
\pgfpathlineto{\pgfqpoint{5.386380in}{1.998205in}}%
\pgfpathlineto{\pgfqpoint{5.386380in}{2.001154in}}%
\pgfpathlineto{\pgfqpoint{5.390921in}{2.001154in}}%
\pgfpathlineto{\pgfqpoint{5.390921in}{1.998205in}}%
\pgfpathmoveto{\pgfqpoint{5.386380in}{2.001154in}}%
\pgfpathlineto{\pgfqpoint{5.386380in}{2.001154in}}%
\pgfpathlineto{\pgfqpoint{5.386380in}{2.004103in}}%
\pgfpathlineto{\pgfqpoint{5.390921in}{2.004103in}}%
\pgfpathlineto{\pgfqpoint{5.390921in}{2.001154in}}%
\pgfpathmoveto{\pgfqpoint{5.390921in}{2.001154in}}%
\pgfpathlineto{\pgfqpoint{5.390921in}{2.001154in}}%
\pgfpathlineto{\pgfqpoint{5.390921in}{2.004103in}}%
\pgfpathlineto{\pgfqpoint{5.395462in}{2.004103in}}%
\pgfpathlineto{\pgfqpoint{5.395462in}{2.001154in}}%
\pgfpathmoveto{\pgfqpoint{5.390921in}{2.004103in}}%
\pgfpathlineto{\pgfqpoint{5.390921in}{2.004103in}}%
\pgfpathlineto{\pgfqpoint{5.390921in}{2.007053in}}%
\pgfpathlineto{\pgfqpoint{5.395462in}{2.007053in}}%
\pgfpathlineto{\pgfqpoint{5.395462in}{2.004103in}}%
\pgfpathmoveto{\pgfqpoint{5.395462in}{2.004103in}}%
\pgfpathlineto{\pgfqpoint{5.395462in}{2.004103in}}%
\pgfpathlineto{\pgfqpoint{5.395462in}{2.007053in}}%
\pgfpathlineto{\pgfqpoint{5.400003in}{2.007053in}}%
\pgfpathlineto{\pgfqpoint{5.400003in}{2.004103in}}%
\pgfpathmoveto{\pgfqpoint{5.395462in}{2.007053in}}%
\pgfpathlineto{\pgfqpoint{5.395462in}{2.007053in}}%
\pgfpathlineto{\pgfqpoint{5.395462in}{2.010002in}}%
\pgfpathlineto{\pgfqpoint{5.400003in}{2.010002in}}%
\pgfpathlineto{\pgfqpoint{5.400003in}{2.007053in}}%
\pgfpathmoveto{\pgfqpoint{5.395462in}{2.010002in}}%
\pgfpathlineto{\pgfqpoint{5.395462in}{2.010002in}}%
\pgfpathlineto{\pgfqpoint{5.395462in}{2.012951in}}%
\pgfpathlineto{\pgfqpoint{5.400003in}{2.012951in}}%
\pgfpathlineto{\pgfqpoint{5.400003in}{2.010002in}}%
\pgfpathclose%
\pgfusepath{fill}%
\end{pgfscope}%
\begin{pgfscope}%
\pgfpathrectangle{\pgfqpoint{0.750000in}{0.500000in}}{\pgfqpoint{4.650000in}{3.020000in}}%
\pgfusepath{clip}%
\pgfsetbuttcap%
\pgfsetmiterjoin%
\definecolor{currentfill}{rgb}{1.000000,0.000000,0.000000}%
\pgfsetfillcolor{currentfill}%
\pgfsetlinewidth{0.000000pt}%
\definecolor{currentstroke}{rgb}{0.000000,0.000000,0.000000}%
\pgfsetstrokecolor{currentstroke}%
\pgfsetstrokeopacity{0.000000}%
\pgfsetdash{}{0pt}%
\pgfpathmoveto{\pgfqpoint{1.326706in}{3.517054in}}%
\pgfpathlineto{\pgfqpoint{1.326706in}{3.520003in}}%
\pgfpathlineto{\pgfqpoint{1.331247in}{3.520003in}}%
\pgfpathlineto{\pgfqpoint{1.331247in}{3.517054in}}%
\pgfpathmoveto{\pgfqpoint{1.435694in}{3.422678in}}%
\pgfpathlineto{\pgfqpoint{1.435694in}{3.422678in}}%
\pgfpathlineto{\pgfqpoint{1.435694in}{3.425628in}}%
\pgfpathlineto{\pgfqpoint{1.440235in}{3.425628in}}%
\pgfpathlineto{\pgfqpoint{1.440235in}{3.422678in}}%
\pgfpathmoveto{\pgfqpoint{1.462941in}{3.399085in}}%
\pgfpathlineto{\pgfqpoint{1.462941in}{3.399085in}}%
\pgfpathlineto{\pgfqpoint{1.462941in}{3.402034in}}%
\pgfpathlineto{\pgfqpoint{1.467482in}{3.402034in}}%
\pgfpathlineto{\pgfqpoint{1.467482in}{3.399085in}}%
\pgfpathmoveto{\pgfqpoint{1.472024in}{3.393186in}}%
\pgfpathlineto{\pgfqpoint{1.472024in}{3.393186in}}%
\pgfpathlineto{\pgfqpoint{1.472024in}{3.396135in}}%
\pgfpathlineto{\pgfqpoint{1.476565in}{3.396135in}}%
\pgfpathlineto{\pgfqpoint{1.476565in}{3.393186in}}%
\pgfpathmoveto{\pgfqpoint{1.467482in}{3.396135in}}%
\pgfpathlineto{\pgfqpoint{1.467482in}{3.396135in}}%
\pgfpathlineto{\pgfqpoint{1.467482in}{3.399085in}}%
\pgfpathlineto{\pgfqpoint{1.472024in}{3.399085in}}%
\pgfpathlineto{\pgfqpoint{1.472024in}{3.396135in}}%
\pgfpathmoveto{\pgfqpoint{1.467482in}{3.399085in}}%
\pgfpathlineto{\pgfqpoint{1.467482in}{3.399085in}}%
\pgfpathlineto{\pgfqpoint{1.467482in}{3.402034in}}%
\pgfpathlineto{\pgfqpoint{1.472024in}{3.402034in}}%
\pgfpathlineto{\pgfqpoint{1.472024in}{3.399085in}}%
\pgfpathmoveto{\pgfqpoint{1.472024in}{3.396135in}}%
\pgfpathlineto{\pgfqpoint{1.472024in}{3.396135in}}%
\pgfpathlineto{\pgfqpoint{1.472024in}{3.399085in}}%
\pgfpathlineto{\pgfqpoint{1.476565in}{3.399085in}}%
\pgfpathlineto{\pgfqpoint{1.476565in}{3.396135in}}%
\pgfpathmoveto{\pgfqpoint{1.449318in}{3.410882in}}%
\pgfpathlineto{\pgfqpoint{1.449318in}{3.410882in}}%
\pgfpathlineto{\pgfqpoint{1.449318in}{3.413831in}}%
\pgfpathlineto{\pgfqpoint{1.453859in}{3.413831in}}%
\pgfpathlineto{\pgfqpoint{1.453859in}{3.410882in}}%
\pgfpathmoveto{\pgfqpoint{1.453859in}{3.407932in}}%
\pgfpathlineto{\pgfqpoint{1.453859in}{3.407932in}}%
\pgfpathlineto{\pgfqpoint{1.453859in}{3.410882in}}%
\pgfpathlineto{\pgfqpoint{1.458400in}{3.410882in}}%
\pgfpathlineto{\pgfqpoint{1.458400in}{3.407932in}}%
\pgfpathmoveto{\pgfqpoint{1.453859in}{3.410882in}}%
\pgfpathlineto{\pgfqpoint{1.453859in}{3.410882in}}%
\pgfpathlineto{\pgfqpoint{1.453859in}{3.413831in}}%
\pgfpathlineto{\pgfqpoint{1.458400in}{3.413831in}}%
\pgfpathlineto{\pgfqpoint{1.458400in}{3.410882in}}%
\pgfpathmoveto{\pgfqpoint{1.444777in}{3.416780in}}%
\pgfpathlineto{\pgfqpoint{1.444777in}{3.416780in}}%
\pgfpathlineto{\pgfqpoint{1.444777in}{3.419729in}}%
\pgfpathlineto{\pgfqpoint{1.449318in}{3.419729in}}%
\pgfpathlineto{\pgfqpoint{1.449318in}{3.416780in}}%
\pgfpathmoveto{\pgfqpoint{1.440235in}{3.419729in}}%
\pgfpathlineto{\pgfqpoint{1.440235in}{3.419729in}}%
\pgfpathlineto{\pgfqpoint{1.440235in}{3.422678in}}%
\pgfpathlineto{\pgfqpoint{1.444777in}{3.422678in}}%
\pgfpathlineto{\pgfqpoint{1.444777in}{3.419729in}}%
\pgfpathmoveto{\pgfqpoint{1.440235in}{3.422678in}}%
\pgfpathlineto{\pgfqpoint{1.440235in}{3.422678in}}%
\pgfpathlineto{\pgfqpoint{1.440235in}{3.425628in}}%
\pgfpathlineto{\pgfqpoint{1.444777in}{3.425628in}}%
\pgfpathlineto{\pgfqpoint{1.444777in}{3.422678in}}%
\pgfpathmoveto{\pgfqpoint{1.444777in}{3.419729in}}%
\pgfpathlineto{\pgfqpoint{1.444777in}{3.419729in}}%
\pgfpathlineto{\pgfqpoint{1.444777in}{3.422678in}}%
\pgfpathlineto{\pgfqpoint{1.449318in}{3.422678in}}%
\pgfpathlineto{\pgfqpoint{1.449318in}{3.419729in}}%
\pgfpathmoveto{\pgfqpoint{1.449318in}{3.413831in}}%
\pgfpathlineto{\pgfqpoint{1.449318in}{3.413831in}}%
\pgfpathlineto{\pgfqpoint{1.449318in}{3.416780in}}%
\pgfpathlineto{\pgfqpoint{1.453859in}{3.416780in}}%
\pgfpathlineto{\pgfqpoint{1.453859in}{3.413831in}}%
\pgfpathmoveto{\pgfqpoint{1.449318in}{3.416780in}}%
\pgfpathlineto{\pgfqpoint{1.449318in}{3.416780in}}%
\pgfpathlineto{\pgfqpoint{1.449318in}{3.419729in}}%
\pgfpathlineto{\pgfqpoint{1.453859in}{3.419729in}}%
\pgfpathlineto{\pgfqpoint{1.453859in}{3.416780in}}%
\pgfpathmoveto{\pgfqpoint{1.458400in}{3.404983in}}%
\pgfpathlineto{\pgfqpoint{1.458400in}{3.404983in}}%
\pgfpathlineto{\pgfqpoint{1.458400in}{3.407932in}}%
\pgfpathlineto{\pgfqpoint{1.462941in}{3.407932in}}%
\pgfpathlineto{\pgfqpoint{1.462941in}{3.404983in}}%
\pgfpathmoveto{\pgfqpoint{1.462941in}{3.402034in}}%
\pgfpathlineto{\pgfqpoint{1.462941in}{3.402034in}}%
\pgfpathlineto{\pgfqpoint{1.462941in}{3.404983in}}%
\pgfpathlineto{\pgfqpoint{1.467482in}{3.404983in}}%
\pgfpathlineto{\pgfqpoint{1.467482in}{3.402034in}}%
\pgfpathmoveto{\pgfqpoint{1.462941in}{3.404983in}}%
\pgfpathlineto{\pgfqpoint{1.462941in}{3.404983in}}%
\pgfpathlineto{\pgfqpoint{1.462941in}{3.407932in}}%
\pgfpathlineto{\pgfqpoint{1.467482in}{3.407932in}}%
\pgfpathlineto{\pgfqpoint{1.467482in}{3.404983in}}%
\pgfpathmoveto{\pgfqpoint{1.458400in}{3.407932in}}%
\pgfpathlineto{\pgfqpoint{1.458400in}{3.407932in}}%
\pgfpathlineto{\pgfqpoint{1.458400in}{3.410882in}}%
\pgfpathlineto{\pgfqpoint{1.462941in}{3.410882in}}%
\pgfpathlineto{\pgfqpoint{1.462941in}{3.407932in}}%
\pgfpathmoveto{\pgfqpoint{1.381200in}{3.469866in}}%
\pgfpathlineto{\pgfqpoint{1.381200in}{3.469866in}}%
\pgfpathlineto{\pgfqpoint{1.381200in}{3.472815in}}%
\pgfpathlineto{\pgfqpoint{1.385741in}{3.472815in}}%
\pgfpathlineto{\pgfqpoint{1.385741in}{3.469866in}}%
\pgfpathmoveto{\pgfqpoint{1.394824in}{3.458069in}}%
\pgfpathlineto{\pgfqpoint{1.394824in}{3.458069in}}%
\pgfpathlineto{\pgfqpoint{1.394824in}{3.461018in}}%
\pgfpathlineto{\pgfqpoint{1.399365in}{3.461018in}}%
\pgfpathlineto{\pgfqpoint{1.399365in}{3.458069in}}%
\pgfpathmoveto{\pgfqpoint{1.399365in}{3.455120in}}%
\pgfpathlineto{\pgfqpoint{1.399365in}{3.455120in}}%
\pgfpathlineto{\pgfqpoint{1.399365in}{3.458069in}}%
\pgfpathlineto{\pgfqpoint{1.403906in}{3.458069in}}%
\pgfpathlineto{\pgfqpoint{1.403906in}{3.455120in}}%
\pgfpathmoveto{\pgfqpoint{1.399365in}{3.458069in}}%
\pgfpathlineto{\pgfqpoint{1.399365in}{3.458069in}}%
\pgfpathlineto{\pgfqpoint{1.399365in}{3.461018in}}%
\pgfpathlineto{\pgfqpoint{1.403906in}{3.461018in}}%
\pgfpathlineto{\pgfqpoint{1.403906in}{3.458069in}}%
\pgfpathmoveto{\pgfqpoint{1.390283in}{3.463968in}}%
\pgfpathlineto{\pgfqpoint{1.390283in}{3.463968in}}%
\pgfpathlineto{\pgfqpoint{1.390283in}{3.466917in}}%
\pgfpathlineto{\pgfqpoint{1.394824in}{3.466917in}}%
\pgfpathlineto{\pgfqpoint{1.394824in}{3.463968in}}%
\pgfpathmoveto{\pgfqpoint{1.385741in}{3.466917in}}%
\pgfpathlineto{\pgfqpoint{1.385741in}{3.466917in}}%
\pgfpathlineto{\pgfqpoint{1.385741in}{3.469866in}}%
\pgfpathlineto{\pgfqpoint{1.390283in}{3.469866in}}%
\pgfpathlineto{\pgfqpoint{1.390283in}{3.466917in}}%
\pgfpathmoveto{\pgfqpoint{1.385741in}{3.469866in}}%
\pgfpathlineto{\pgfqpoint{1.385741in}{3.469866in}}%
\pgfpathlineto{\pgfqpoint{1.385741in}{3.472815in}}%
\pgfpathlineto{\pgfqpoint{1.390283in}{3.472815in}}%
\pgfpathlineto{\pgfqpoint{1.390283in}{3.469866in}}%
\pgfpathmoveto{\pgfqpoint{1.390283in}{3.466917in}}%
\pgfpathlineto{\pgfqpoint{1.390283in}{3.466917in}}%
\pgfpathlineto{\pgfqpoint{1.390283in}{3.469866in}}%
\pgfpathlineto{\pgfqpoint{1.394824in}{3.469866in}}%
\pgfpathlineto{\pgfqpoint{1.394824in}{3.466917in}}%
\pgfpathmoveto{\pgfqpoint{1.394824in}{3.461018in}}%
\pgfpathlineto{\pgfqpoint{1.394824in}{3.461018in}}%
\pgfpathlineto{\pgfqpoint{1.394824in}{3.463968in}}%
\pgfpathlineto{\pgfqpoint{1.399365in}{3.463968in}}%
\pgfpathlineto{\pgfqpoint{1.399365in}{3.461018in}}%
\pgfpathmoveto{\pgfqpoint{1.394824in}{3.463968in}}%
\pgfpathlineto{\pgfqpoint{1.394824in}{3.463968in}}%
\pgfpathlineto{\pgfqpoint{1.394824in}{3.466917in}}%
\pgfpathlineto{\pgfqpoint{1.399365in}{3.466917in}}%
\pgfpathlineto{\pgfqpoint{1.399365in}{3.463968in}}%
\pgfpathmoveto{\pgfqpoint{1.353953in}{3.493460in}}%
\pgfpathlineto{\pgfqpoint{1.353953in}{3.493460in}}%
\pgfpathlineto{\pgfqpoint{1.353953in}{3.496409in}}%
\pgfpathlineto{\pgfqpoint{1.358494in}{3.496409in}}%
\pgfpathlineto{\pgfqpoint{1.358494in}{3.493460in}}%
\pgfpathmoveto{\pgfqpoint{1.363036in}{3.487561in}}%
\pgfpathlineto{\pgfqpoint{1.363036in}{3.487561in}}%
\pgfpathlineto{\pgfqpoint{1.363036in}{3.490511in}}%
\pgfpathlineto{\pgfqpoint{1.367577in}{3.490511in}}%
\pgfpathlineto{\pgfqpoint{1.367577in}{3.487561in}}%
\pgfpathmoveto{\pgfqpoint{1.358494in}{3.490511in}}%
\pgfpathlineto{\pgfqpoint{1.358494in}{3.490511in}}%
\pgfpathlineto{\pgfqpoint{1.358494in}{3.493460in}}%
\pgfpathlineto{\pgfqpoint{1.363036in}{3.493460in}}%
\pgfpathlineto{\pgfqpoint{1.363036in}{3.490511in}}%
\pgfpathmoveto{\pgfqpoint{1.358494in}{3.493460in}}%
\pgfpathlineto{\pgfqpoint{1.358494in}{3.493460in}}%
\pgfpathlineto{\pgfqpoint{1.358494in}{3.496409in}}%
\pgfpathlineto{\pgfqpoint{1.363036in}{3.496409in}}%
\pgfpathlineto{\pgfqpoint{1.363036in}{3.493460in}}%
\pgfpathmoveto{\pgfqpoint{1.363036in}{3.490511in}}%
\pgfpathlineto{\pgfqpoint{1.363036in}{3.490511in}}%
\pgfpathlineto{\pgfqpoint{1.363036in}{3.493460in}}%
\pgfpathlineto{\pgfqpoint{1.367577in}{3.493460in}}%
\pgfpathlineto{\pgfqpoint{1.367577in}{3.490511in}}%
\pgfpathmoveto{\pgfqpoint{1.340330in}{3.505257in}}%
\pgfpathlineto{\pgfqpoint{1.340330in}{3.505257in}}%
\pgfpathlineto{\pgfqpoint{1.340330in}{3.508206in}}%
\pgfpathlineto{\pgfqpoint{1.344871in}{3.508206in}}%
\pgfpathlineto{\pgfqpoint{1.344871in}{3.505257in}}%
\pgfpathmoveto{\pgfqpoint{1.344871in}{3.502307in}}%
\pgfpathlineto{\pgfqpoint{1.344871in}{3.502307in}}%
\pgfpathlineto{\pgfqpoint{1.344871in}{3.505257in}}%
\pgfpathlineto{\pgfqpoint{1.349412in}{3.505257in}}%
\pgfpathlineto{\pgfqpoint{1.349412in}{3.502307in}}%
\pgfpathmoveto{\pgfqpoint{1.344871in}{3.505257in}}%
\pgfpathlineto{\pgfqpoint{1.344871in}{3.505257in}}%
\pgfpathlineto{\pgfqpoint{1.344871in}{3.508206in}}%
\pgfpathlineto{\pgfqpoint{1.349412in}{3.508206in}}%
\pgfpathlineto{\pgfqpoint{1.349412in}{3.505257in}}%
\pgfpathmoveto{\pgfqpoint{1.335789in}{3.511155in}}%
\pgfpathlineto{\pgfqpoint{1.335789in}{3.511155in}}%
\pgfpathlineto{\pgfqpoint{1.335789in}{3.514104in}}%
\pgfpathlineto{\pgfqpoint{1.340330in}{3.514104in}}%
\pgfpathlineto{\pgfqpoint{1.340330in}{3.511155in}}%
\pgfpathmoveto{\pgfqpoint{1.331247in}{3.514104in}}%
\pgfpathlineto{\pgfqpoint{1.331247in}{3.514104in}}%
\pgfpathlineto{\pgfqpoint{1.331247in}{3.517054in}}%
\pgfpathlineto{\pgfqpoint{1.335789in}{3.517054in}}%
\pgfpathlineto{\pgfqpoint{1.335789in}{3.514104in}}%
\pgfpathmoveto{\pgfqpoint{1.331247in}{3.517054in}}%
\pgfpathlineto{\pgfqpoint{1.331247in}{3.517054in}}%
\pgfpathlineto{\pgfqpoint{1.331247in}{3.520003in}}%
\pgfpathlineto{\pgfqpoint{1.335789in}{3.520003in}}%
\pgfpathlineto{\pgfqpoint{1.335789in}{3.517054in}}%
\pgfpathmoveto{\pgfqpoint{1.335789in}{3.514104in}}%
\pgfpathlineto{\pgfqpoint{1.335789in}{3.514104in}}%
\pgfpathlineto{\pgfqpoint{1.335789in}{3.517054in}}%
\pgfpathlineto{\pgfqpoint{1.340330in}{3.517054in}}%
\pgfpathlineto{\pgfqpoint{1.340330in}{3.514104in}}%
\pgfpathmoveto{\pgfqpoint{1.340330in}{3.508206in}}%
\pgfpathlineto{\pgfqpoint{1.340330in}{3.508206in}}%
\pgfpathlineto{\pgfqpoint{1.340330in}{3.511155in}}%
\pgfpathlineto{\pgfqpoint{1.344871in}{3.511155in}}%
\pgfpathlineto{\pgfqpoint{1.344871in}{3.508206in}}%
\pgfpathmoveto{\pgfqpoint{1.340330in}{3.511155in}}%
\pgfpathlineto{\pgfqpoint{1.340330in}{3.511155in}}%
\pgfpathlineto{\pgfqpoint{1.340330in}{3.514104in}}%
\pgfpathlineto{\pgfqpoint{1.344871in}{3.514104in}}%
\pgfpathlineto{\pgfqpoint{1.344871in}{3.511155in}}%
\pgfpathmoveto{\pgfqpoint{1.349412in}{3.499358in}}%
\pgfpathlineto{\pgfqpoint{1.349412in}{3.499358in}}%
\pgfpathlineto{\pgfqpoint{1.349412in}{3.502307in}}%
\pgfpathlineto{\pgfqpoint{1.353953in}{3.502307in}}%
\pgfpathlineto{\pgfqpoint{1.353953in}{3.499358in}}%
\pgfpathmoveto{\pgfqpoint{1.353953in}{3.496409in}}%
\pgfpathlineto{\pgfqpoint{1.353953in}{3.496409in}}%
\pgfpathlineto{\pgfqpoint{1.353953in}{3.499358in}}%
\pgfpathlineto{\pgfqpoint{1.358494in}{3.499358in}}%
\pgfpathlineto{\pgfqpoint{1.358494in}{3.496409in}}%
\pgfpathmoveto{\pgfqpoint{1.353953in}{3.499358in}}%
\pgfpathlineto{\pgfqpoint{1.353953in}{3.499358in}}%
\pgfpathlineto{\pgfqpoint{1.353953in}{3.502307in}}%
\pgfpathlineto{\pgfqpoint{1.358494in}{3.502307in}}%
\pgfpathlineto{\pgfqpoint{1.358494in}{3.499358in}}%
\pgfpathmoveto{\pgfqpoint{1.349412in}{3.502307in}}%
\pgfpathlineto{\pgfqpoint{1.349412in}{3.502307in}}%
\pgfpathlineto{\pgfqpoint{1.349412in}{3.505257in}}%
\pgfpathlineto{\pgfqpoint{1.353953in}{3.505257in}}%
\pgfpathlineto{\pgfqpoint{1.353953in}{3.502307in}}%
\pgfpathmoveto{\pgfqpoint{1.367577in}{3.481663in}}%
\pgfpathlineto{\pgfqpoint{1.367577in}{3.481663in}}%
\pgfpathlineto{\pgfqpoint{1.367577in}{3.484612in}}%
\pgfpathlineto{\pgfqpoint{1.372118in}{3.484612in}}%
\pgfpathlineto{\pgfqpoint{1.372118in}{3.481663in}}%
\pgfpathmoveto{\pgfqpoint{1.372118in}{3.478714in}}%
\pgfpathlineto{\pgfqpoint{1.372118in}{3.478714in}}%
\pgfpathlineto{\pgfqpoint{1.372118in}{3.481663in}}%
\pgfpathlineto{\pgfqpoint{1.376659in}{3.481663in}}%
\pgfpathlineto{\pgfqpoint{1.376659in}{3.478714in}}%
\pgfpathmoveto{\pgfqpoint{1.372118in}{3.481663in}}%
\pgfpathlineto{\pgfqpoint{1.372118in}{3.481663in}}%
\pgfpathlineto{\pgfqpoint{1.372118in}{3.484612in}}%
\pgfpathlineto{\pgfqpoint{1.376659in}{3.484612in}}%
\pgfpathlineto{\pgfqpoint{1.376659in}{3.481663in}}%
\pgfpathmoveto{\pgfqpoint{1.376659in}{3.475764in}}%
\pgfpathlineto{\pgfqpoint{1.376659in}{3.475764in}}%
\pgfpathlineto{\pgfqpoint{1.376659in}{3.478714in}}%
\pgfpathlineto{\pgfqpoint{1.381200in}{3.478714in}}%
\pgfpathlineto{\pgfqpoint{1.381200in}{3.475764in}}%
\pgfpathmoveto{\pgfqpoint{1.381200in}{3.472815in}}%
\pgfpathlineto{\pgfqpoint{1.381200in}{3.472815in}}%
\pgfpathlineto{\pgfqpoint{1.381200in}{3.475764in}}%
\pgfpathlineto{\pgfqpoint{1.385741in}{3.475764in}}%
\pgfpathlineto{\pgfqpoint{1.385741in}{3.472815in}}%
\pgfpathmoveto{\pgfqpoint{1.381200in}{3.475764in}}%
\pgfpathlineto{\pgfqpoint{1.381200in}{3.475764in}}%
\pgfpathlineto{\pgfqpoint{1.381200in}{3.478714in}}%
\pgfpathlineto{\pgfqpoint{1.385741in}{3.478714in}}%
\pgfpathlineto{\pgfqpoint{1.385741in}{3.475764in}}%
\pgfpathmoveto{\pgfqpoint{1.376659in}{3.478714in}}%
\pgfpathlineto{\pgfqpoint{1.376659in}{3.478714in}}%
\pgfpathlineto{\pgfqpoint{1.376659in}{3.481663in}}%
\pgfpathlineto{\pgfqpoint{1.381200in}{3.481663in}}%
\pgfpathlineto{\pgfqpoint{1.381200in}{3.478714in}}%
\pgfpathmoveto{\pgfqpoint{1.367577in}{3.484612in}}%
\pgfpathlineto{\pgfqpoint{1.367577in}{3.484612in}}%
\pgfpathlineto{\pgfqpoint{1.367577in}{3.487561in}}%
\pgfpathlineto{\pgfqpoint{1.372118in}{3.487561in}}%
\pgfpathlineto{\pgfqpoint{1.372118in}{3.484612in}}%
\pgfpathmoveto{\pgfqpoint{1.367577in}{3.487561in}}%
\pgfpathlineto{\pgfqpoint{1.367577in}{3.487561in}}%
\pgfpathlineto{\pgfqpoint{1.367577in}{3.490511in}}%
\pgfpathlineto{\pgfqpoint{1.372118in}{3.490511in}}%
\pgfpathlineto{\pgfqpoint{1.372118in}{3.487561in}}%
\pgfpathmoveto{\pgfqpoint{1.408447in}{3.446272in}}%
\pgfpathlineto{\pgfqpoint{1.408447in}{3.446272in}}%
\pgfpathlineto{\pgfqpoint{1.408447in}{3.449221in}}%
\pgfpathlineto{\pgfqpoint{1.412988in}{3.449221in}}%
\pgfpathlineto{\pgfqpoint{1.412988in}{3.446272in}}%
\pgfpathmoveto{\pgfqpoint{1.417530in}{3.440374in}}%
\pgfpathlineto{\pgfqpoint{1.417530in}{3.440374in}}%
\pgfpathlineto{\pgfqpoint{1.417530in}{3.443323in}}%
\pgfpathlineto{\pgfqpoint{1.422071in}{3.443323in}}%
\pgfpathlineto{\pgfqpoint{1.422071in}{3.440374in}}%
\pgfpathmoveto{\pgfqpoint{1.412988in}{3.443323in}}%
\pgfpathlineto{\pgfqpoint{1.412988in}{3.443323in}}%
\pgfpathlineto{\pgfqpoint{1.412988in}{3.446272in}}%
\pgfpathlineto{\pgfqpoint{1.417530in}{3.446272in}}%
\pgfpathlineto{\pgfqpoint{1.417530in}{3.443323in}}%
\pgfpathmoveto{\pgfqpoint{1.412988in}{3.446272in}}%
\pgfpathlineto{\pgfqpoint{1.412988in}{3.446272in}}%
\pgfpathlineto{\pgfqpoint{1.412988in}{3.449221in}}%
\pgfpathlineto{\pgfqpoint{1.417530in}{3.449221in}}%
\pgfpathlineto{\pgfqpoint{1.417530in}{3.446272in}}%
\pgfpathmoveto{\pgfqpoint{1.417530in}{3.443323in}}%
\pgfpathlineto{\pgfqpoint{1.417530in}{3.443323in}}%
\pgfpathlineto{\pgfqpoint{1.417530in}{3.446272in}}%
\pgfpathlineto{\pgfqpoint{1.422071in}{3.446272in}}%
\pgfpathlineto{\pgfqpoint{1.422071in}{3.443323in}}%
\pgfpathmoveto{\pgfqpoint{1.422071in}{3.434475in}}%
\pgfpathlineto{\pgfqpoint{1.422071in}{3.434475in}}%
\pgfpathlineto{\pgfqpoint{1.422071in}{3.437425in}}%
\pgfpathlineto{\pgfqpoint{1.426612in}{3.437425in}}%
\pgfpathlineto{\pgfqpoint{1.426612in}{3.434475in}}%
\pgfpathmoveto{\pgfqpoint{1.426612in}{3.431526in}}%
\pgfpathlineto{\pgfqpoint{1.426612in}{3.431526in}}%
\pgfpathlineto{\pgfqpoint{1.426612in}{3.434475in}}%
\pgfpathlineto{\pgfqpoint{1.431153in}{3.434475in}}%
\pgfpathlineto{\pgfqpoint{1.431153in}{3.431526in}}%
\pgfpathmoveto{\pgfqpoint{1.426612in}{3.434475in}}%
\pgfpathlineto{\pgfqpoint{1.426612in}{3.434475in}}%
\pgfpathlineto{\pgfqpoint{1.426612in}{3.437425in}}%
\pgfpathlineto{\pgfqpoint{1.431153in}{3.437425in}}%
\pgfpathlineto{\pgfqpoint{1.431153in}{3.434475in}}%
\pgfpathmoveto{\pgfqpoint{1.431153in}{3.428577in}}%
\pgfpathlineto{\pgfqpoint{1.431153in}{3.428577in}}%
\pgfpathlineto{\pgfqpoint{1.431153in}{3.431526in}}%
\pgfpathlineto{\pgfqpoint{1.435694in}{3.431526in}}%
\pgfpathlineto{\pgfqpoint{1.435694in}{3.428577in}}%
\pgfpathmoveto{\pgfqpoint{1.435694in}{3.425628in}}%
\pgfpathlineto{\pgfqpoint{1.435694in}{3.425628in}}%
\pgfpathlineto{\pgfqpoint{1.435694in}{3.428577in}}%
\pgfpathlineto{\pgfqpoint{1.440235in}{3.428577in}}%
\pgfpathlineto{\pgfqpoint{1.440235in}{3.425628in}}%
\pgfpathmoveto{\pgfqpoint{1.435694in}{3.428577in}}%
\pgfpathlineto{\pgfqpoint{1.435694in}{3.428577in}}%
\pgfpathlineto{\pgfqpoint{1.435694in}{3.431526in}}%
\pgfpathlineto{\pgfqpoint{1.440235in}{3.431526in}}%
\pgfpathlineto{\pgfqpoint{1.440235in}{3.428577in}}%
\pgfpathmoveto{\pgfqpoint{1.431153in}{3.431526in}}%
\pgfpathlineto{\pgfqpoint{1.431153in}{3.431526in}}%
\pgfpathlineto{\pgfqpoint{1.431153in}{3.434475in}}%
\pgfpathlineto{\pgfqpoint{1.435694in}{3.434475in}}%
\pgfpathlineto{\pgfqpoint{1.435694in}{3.431526in}}%
\pgfpathmoveto{\pgfqpoint{1.422071in}{3.437425in}}%
\pgfpathlineto{\pgfqpoint{1.422071in}{3.437425in}}%
\pgfpathlineto{\pgfqpoint{1.422071in}{3.440374in}}%
\pgfpathlineto{\pgfqpoint{1.426612in}{3.440374in}}%
\pgfpathlineto{\pgfqpoint{1.426612in}{3.437425in}}%
\pgfpathmoveto{\pgfqpoint{1.422071in}{3.440374in}}%
\pgfpathlineto{\pgfqpoint{1.422071in}{3.440374in}}%
\pgfpathlineto{\pgfqpoint{1.422071in}{3.443323in}}%
\pgfpathlineto{\pgfqpoint{1.426612in}{3.443323in}}%
\pgfpathlineto{\pgfqpoint{1.426612in}{3.440374in}}%
\pgfpathmoveto{\pgfqpoint{1.403906in}{3.452171in}}%
\pgfpathlineto{\pgfqpoint{1.403906in}{3.452171in}}%
\pgfpathlineto{\pgfqpoint{1.403906in}{3.455120in}}%
\pgfpathlineto{\pgfqpoint{1.408447in}{3.455120in}}%
\pgfpathlineto{\pgfqpoint{1.408447in}{3.452171in}}%
\pgfpathmoveto{\pgfqpoint{1.408447in}{3.449221in}}%
\pgfpathlineto{\pgfqpoint{1.408447in}{3.449221in}}%
\pgfpathlineto{\pgfqpoint{1.408447in}{3.452171in}}%
\pgfpathlineto{\pgfqpoint{1.412988in}{3.452171in}}%
\pgfpathlineto{\pgfqpoint{1.412988in}{3.449221in}}%
\pgfpathmoveto{\pgfqpoint{1.408447in}{3.452171in}}%
\pgfpathlineto{\pgfqpoint{1.408447in}{3.452171in}}%
\pgfpathlineto{\pgfqpoint{1.408447in}{3.455120in}}%
\pgfpathlineto{\pgfqpoint{1.412988in}{3.455120in}}%
\pgfpathlineto{\pgfqpoint{1.412988in}{3.452171in}}%
\pgfpathmoveto{\pgfqpoint{1.403906in}{3.455120in}}%
\pgfpathlineto{\pgfqpoint{1.403906in}{3.455120in}}%
\pgfpathlineto{\pgfqpoint{1.403906in}{3.458069in}}%
\pgfpathlineto{\pgfqpoint{1.408447in}{3.458069in}}%
\pgfpathlineto{\pgfqpoint{1.408447in}{3.455120in}}%
\pgfpathmoveto{\pgfqpoint{1.544679in}{3.328304in}}%
\pgfpathlineto{\pgfqpoint{1.544679in}{3.328304in}}%
\pgfpathlineto{\pgfqpoint{1.544679in}{3.331253in}}%
\pgfpathlineto{\pgfqpoint{1.549220in}{3.331253in}}%
\pgfpathlineto{\pgfqpoint{1.549220in}{3.328304in}}%
\pgfpathmoveto{\pgfqpoint{1.599170in}{3.281114in}}%
\pgfpathlineto{\pgfqpoint{1.599170in}{3.281114in}}%
\pgfpathlineto{\pgfqpoint{1.599170in}{3.284063in}}%
\pgfpathlineto{\pgfqpoint{1.603711in}{3.284063in}}%
\pgfpathlineto{\pgfqpoint{1.603711in}{3.281114in}}%
\pgfpathmoveto{\pgfqpoint{1.617333in}{3.266367in}}%
\pgfpathlineto{\pgfqpoint{1.617333in}{3.266367in}}%
\pgfpathlineto{\pgfqpoint{1.617333in}{3.269317in}}%
\pgfpathlineto{\pgfqpoint{1.621874in}{3.269317in}}%
\pgfpathlineto{\pgfqpoint{1.621874in}{3.266367in}}%
\pgfpathmoveto{\pgfqpoint{1.617333in}{3.269317in}}%
\pgfpathlineto{\pgfqpoint{1.617333in}{3.269317in}}%
\pgfpathlineto{\pgfqpoint{1.617333in}{3.272266in}}%
\pgfpathlineto{\pgfqpoint{1.621874in}{3.272266in}}%
\pgfpathlineto{\pgfqpoint{1.621874in}{3.269317in}}%
\pgfpathmoveto{\pgfqpoint{1.608252in}{3.275215in}}%
\pgfpathlineto{\pgfqpoint{1.608252in}{3.275215in}}%
\pgfpathlineto{\pgfqpoint{1.608252in}{3.278165in}}%
\pgfpathlineto{\pgfqpoint{1.612792in}{3.278165in}}%
\pgfpathlineto{\pgfqpoint{1.612792in}{3.275215in}}%
\pgfpathmoveto{\pgfqpoint{1.603711in}{3.278165in}}%
\pgfpathlineto{\pgfqpoint{1.603711in}{3.278165in}}%
\pgfpathlineto{\pgfqpoint{1.603711in}{3.281114in}}%
\pgfpathlineto{\pgfqpoint{1.608252in}{3.281114in}}%
\pgfpathlineto{\pgfqpoint{1.608252in}{3.278165in}}%
\pgfpathmoveto{\pgfqpoint{1.603711in}{3.281114in}}%
\pgfpathlineto{\pgfqpoint{1.603711in}{3.281114in}}%
\pgfpathlineto{\pgfqpoint{1.603711in}{3.284063in}}%
\pgfpathlineto{\pgfqpoint{1.608252in}{3.284063in}}%
\pgfpathlineto{\pgfqpoint{1.608252in}{3.281114in}}%
\pgfpathmoveto{\pgfqpoint{1.608252in}{3.278165in}}%
\pgfpathlineto{\pgfqpoint{1.608252in}{3.278165in}}%
\pgfpathlineto{\pgfqpoint{1.608252in}{3.281114in}}%
\pgfpathlineto{\pgfqpoint{1.612792in}{3.281114in}}%
\pgfpathlineto{\pgfqpoint{1.612792in}{3.278165in}}%
\pgfpathmoveto{\pgfqpoint{1.612792in}{3.272266in}}%
\pgfpathlineto{\pgfqpoint{1.612792in}{3.272266in}}%
\pgfpathlineto{\pgfqpoint{1.612792in}{3.275215in}}%
\pgfpathlineto{\pgfqpoint{1.617333in}{3.275215in}}%
\pgfpathlineto{\pgfqpoint{1.617333in}{3.272266in}}%
\pgfpathmoveto{\pgfqpoint{1.612792in}{3.275215in}}%
\pgfpathlineto{\pgfqpoint{1.612792in}{3.275215in}}%
\pgfpathlineto{\pgfqpoint{1.612792in}{3.278165in}}%
\pgfpathlineto{\pgfqpoint{1.617333in}{3.278165in}}%
\pgfpathlineto{\pgfqpoint{1.617333in}{3.275215in}}%
\pgfpathmoveto{\pgfqpoint{1.617333in}{3.272266in}}%
\pgfpathlineto{\pgfqpoint{1.617333in}{3.272266in}}%
\pgfpathlineto{\pgfqpoint{1.617333in}{3.275215in}}%
\pgfpathlineto{\pgfqpoint{1.621874in}{3.275215in}}%
\pgfpathlineto{\pgfqpoint{1.621874in}{3.272266in}}%
\pgfpathmoveto{\pgfqpoint{1.571924in}{3.304709in}}%
\pgfpathlineto{\pgfqpoint{1.571924in}{3.304709in}}%
\pgfpathlineto{\pgfqpoint{1.571924in}{3.307658in}}%
\pgfpathlineto{\pgfqpoint{1.576465in}{3.307658in}}%
\pgfpathlineto{\pgfqpoint{1.576465in}{3.304709in}}%
\pgfpathmoveto{\pgfqpoint{1.581006in}{3.298810in}}%
\pgfpathlineto{\pgfqpoint{1.581006in}{3.298810in}}%
\pgfpathlineto{\pgfqpoint{1.581006in}{3.301759in}}%
\pgfpathlineto{\pgfqpoint{1.585547in}{3.301759in}}%
\pgfpathlineto{\pgfqpoint{1.585547in}{3.298810in}}%
\pgfpathmoveto{\pgfqpoint{1.576465in}{3.301759in}}%
\pgfpathlineto{\pgfqpoint{1.576465in}{3.301759in}}%
\pgfpathlineto{\pgfqpoint{1.576465in}{3.304709in}}%
\pgfpathlineto{\pgfqpoint{1.581006in}{3.304709in}}%
\pgfpathlineto{\pgfqpoint{1.581006in}{3.301759in}}%
\pgfpathmoveto{\pgfqpoint{1.576465in}{3.304709in}}%
\pgfpathlineto{\pgfqpoint{1.576465in}{3.304709in}}%
\pgfpathlineto{\pgfqpoint{1.576465in}{3.307658in}}%
\pgfpathlineto{\pgfqpoint{1.581006in}{3.307658in}}%
\pgfpathlineto{\pgfqpoint{1.581006in}{3.304709in}}%
\pgfpathmoveto{\pgfqpoint{1.581006in}{3.301759in}}%
\pgfpathlineto{\pgfqpoint{1.581006in}{3.301759in}}%
\pgfpathlineto{\pgfqpoint{1.581006in}{3.304709in}}%
\pgfpathlineto{\pgfqpoint{1.585547in}{3.304709in}}%
\pgfpathlineto{\pgfqpoint{1.585547in}{3.301759in}}%
\pgfpathmoveto{\pgfqpoint{1.558301in}{3.316506in}}%
\pgfpathlineto{\pgfqpoint{1.558301in}{3.316506in}}%
\pgfpathlineto{\pgfqpoint{1.558301in}{3.319455in}}%
\pgfpathlineto{\pgfqpoint{1.562842in}{3.319455in}}%
\pgfpathlineto{\pgfqpoint{1.562842in}{3.316506in}}%
\pgfpathmoveto{\pgfqpoint{1.562842in}{3.313557in}}%
\pgfpathlineto{\pgfqpoint{1.562842in}{3.313557in}}%
\pgfpathlineto{\pgfqpoint{1.562842in}{3.316506in}}%
\pgfpathlineto{\pgfqpoint{1.567383in}{3.316506in}}%
\pgfpathlineto{\pgfqpoint{1.567383in}{3.313557in}}%
\pgfpathmoveto{\pgfqpoint{1.562842in}{3.316506in}}%
\pgfpathlineto{\pgfqpoint{1.562842in}{3.316506in}}%
\pgfpathlineto{\pgfqpoint{1.562842in}{3.319455in}}%
\pgfpathlineto{\pgfqpoint{1.567383in}{3.319455in}}%
\pgfpathlineto{\pgfqpoint{1.567383in}{3.316506in}}%
\pgfpathmoveto{\pgfqpoint{1.553760in}{3.322405in}}%
\pgfpathlineto{\pgfqpoint{1.553760in}{3.322405in}}%
\pgfpathlineto{\pgfqpoint{1.553760in}{3.325354in}}%
\pgfpathlineto{\pgfqpoint{1.558301in}{3.325354in}}%
\pgfpathlineto{\pgfqpoint{1.558301in}{3.322405in}}%
\pgfpathmoveto{\pgfqpoint{1.549220in}{3.325354in}}%
\pgfpathlineto{\pgfqpoint{1.549220in}{3.325354in}}%
\pgfpathlineto{\pgfqpoint{1.549220in}{3.328304in}}%
\pgfpathlineto{\pgfqpoint{1.553760in}{3.328304in}}%
\pgfpathlineto{\pgfqpoint{1.553760in}{3.325354in}}%
\pgfpathmoveto{\pgfqpoint{1.549220in}{3.328304in}}%
\pgfpathlineto{\pgfqpoint{1.549220in}{3.328304in}}%
\pgfpathlineto{\pgfqpoint{1.549220in}{3.331253in}}%
\pgfpathlineto{\pgfqpoint{1.553760in}{3.331253in}}%
\pgfpathlineto{\pgfqpoint{1.553760in}{3.328304in}}%
\pgfpathmoveto{\pgfqpoint{1.553760in}{3.325354in}}%
\pgfpathlineto{\pgfqpoint{1.553760in}{3.325354in}}%
\pgfpathlineto{\pgfqpoint{1.553760in}{3.328304in}}%
\pgfpathlineto{\pgfqpoint{1.558301in}{3.328304in}}%
\pgfpathlineto{\pgfqpoint{1.558301in}{3.325354in}}%
\pgfpathmoveto{\pgfqpoint{1.558301in}{3.319455in}}%
\pgfpathlineto{\pgfqpoint{1.558301in}{3.319455in}}%
\pgfpathlineto{\pgfqpoint{1.558301in}{3.322405in}}%
\pgfpathlineto{\pgfqpoint{1.562842in}{3.322405in}}%
\pgfpathlineto{\pgfqpoint{1.562842in}{3.319455in}}%
\pgfpathmoveto{\pgfqpoint{1.558301in}{3.322405in}}%
\pgfpathlineto{\pgfqpoint{1.558301in}{3.322405in}}%
\pgfpathlineto{\pgfqpoint{1.558301in}{3.325354in}}%
\pgfpathlineto{\pgfqpoint{1.562842in}{3.325354in}}%
\pgfpathlineto{\pgfqpoint{1.562842in}{3.322405in}}%
\pgfpathmoveto{\pgfqpoint{1.567383in}{3.310607in}}%
\pgfpathlineto{\pgfqpoint{1.567383in}{3.310607in}}%
\pgfpathlineto{\pgfqpoint{1.567383in}{3.313557in}}%
\pgfpathlineto{\pgfqpoint{1.571924in}{3.313557in}}%
\pgfpathlineto{\pgfqpoint{1.571924in}{3.310607in}}%
\pgfpathmoveto{\pgfqpoint{1.571924in}{3.307658in}}%
\pgfpathlineto{\pgfqpoint{1.571924in}{3.307658in}}%
\pgfpathlineto{\pgfqpoint{1.571924in}{3.310607in}}%
\pgfpathlineto{\pgfqpoint{1.576465in}{3.310607in}}%
\pgfpathlineto{\pgfqpoint{1.576465in}{3.307658in}}%
\pgfpathmoveto{\pgfqpoint{1.571924in}{3.310607in}}%
\pgfpathlineto{\pgfqpoint{1.571924in}{3.310607in}}%
\pgfpathlineto{\pgfqpoint{1.571924in}{3.313557in}}%
\pgfpathlineto{\pgfqpoint{1.576465in}{3.313557in}}%
\pgfpathlineto{\pgfqpoint{1.576465in}{3.310607in}}%
\pgfpathmoveto{\pgfqpoint{1.567383in}{3.313557in}}%
\pgfpathlineto{\pgfqpoint{1.567383in}{3.313557in}}%
\pgfpathlineto{\pgfqpoint{1.567383in}{3.316506in}}%
\pgfpathlineto{\pgfqpoint{1.571924in}{3.316506in}}%
\pgfpathlineto{\pgfqpoint{1.571924in}{3.313557in}}%
\pgfpathmoveto{\pgfqpoint{1.585547in}{3.292911in}}%
\pgfpathlineto{\pgfqpoint{1.585547in}{3.292911in}}%
\pgfpathlineto{\pgfqpoint{1.585547in}{3.295861in}}%
\pgfpathlineto{\pgfqpoint{1.590088in}{3.295861in}}%
\pgfpathlineto{\pgfqpoint{1.590088in}{3.292911in}}%
\pgfpathmoveto{\pgfqpoint{1.590088in}{3.289962in}}%
\pgfpathlineto{\pgfqpoint{1.590088in}{3.289962in}}%
\pgfpathlineto{\pgfqpoint{1.590088in}{3.292911in}}%
\pgfpathlineto{\pgfqpoint{1.594629in}{3.292911in}}%
\pgfpathlineto{\pgfqpoint{1.594629in}{3.289962in}}%
\pgfpathmoveto{\pgfqpoint{1.590088in}{3.292911in}}%
\pgfpathlineto{\pgfqpoint{1.590088in}{3.292911in}}%
\pgfpathlineto{\pgfqpoint{1.590088in}{3.295861in}}%
\pgfpathlineto{\pgfqpoint{1.594629in}{3.295861in}}%
\pgfpathlineto{\pgfqpoint{1.594629in}{3.292911in}}%
\pgfpathmoveto{\pgfqpoint{1.594629in}{3.287013in}}%
\pgfpathlineto{\pgfqpoint{1.594629in}{3.287013in}}%
\pgfpathlineto{\pgfqpoint{1.594629in}{3.289962in}}%
\pgfpathlineto{\pgfqpoint{1.599170in}{3.289962in}}%
\pgfpathlineto{\pgfqpoint{1.599170in}{3.287013in}}%
\pgfpathmoveto{\pgfqpoint{1.599170in}{3.284063in}}%
\pgfpathlineto{\pgfqpoint{1.599170in}{3.284063in}}%
\pgfpathlineto{\pgfqpoint{1.599170in}{3.287013in}}%
\pgfpathlineto{\pgfqpoint{1.603711in}{3.287013in}}%
\pgfpathlineto{\pgfqpoint{1.603711in}{3.284063in}}%
\pgfpathmoveto{\pgfqpoint{1.599170in}{3.287013in}}%
\pgfpathlineto{\pgfqpoint{1.599170in}{3.287013in}}%
\pgfpathlineto{\pgfqpoint{1.599170in}{3.289962in}}%
\pgfpathlineto{\pgfqpoint{1.603711in}{3.289962in}}%
\pgfpathlineto{\pgfqpoint{1.603711in}{3.287013in}}%
\pgfpathmoveto{\pgfqpoint{1.594629in}{3.289962in}}%
\pgfpathlineto{\pgfqpoint{1.594629in}{3.289962in}}%
\pgfpathlineto{\pgfqpoint{1.594629in}{3.292911in}}%
\pgfpathlineto{\pgfqpoint{1.599170in}{3.292911in}}%
\pgfpathlineto{\pgfqpoint{1.599170in}{3.289962in}}%
\pgfpathmoveto{\pgfqpoint{1.585547in}{3.295861in}}%
\pgfpathlineto{\pgfqpoint{1.585547in}{3.295861in}}%
\pgfpathlineto{\pgfqpoint{1.585547in}{3.298810in}}%
\pgfpathlineto{\pgfqpoint{1.590088in}{3.298810in}}%
\pgfpathlineto{\pgfqpoint{1.590088in}{3.295861in}}%
\pgfpathmoveto{\pgfqpoint{1.585547in}{3.298810in}}%
\pgfpathlineto{\pgfqpoint{1.585547in}{3.298810in}}%
\pgfpathlineto{\pgfqpoint{1.585547in}{3.301759in}}%
\pgfpathlineto{\pgfqpoint{1.590088in}{3.301759in}}%
\pgfpathlineto{\pgfqpoint{1.590088in}{3.298810in}}%
\pgfpathmoveto{\pgfqpoint{1.490187in}{3.375491in}}%
\pgfpathlineto{\pgfqpoint{1.490187in}{3.375491in}}%
\pgfpathlineto{\pgfqpoint{1.490187in}{3.378440in}}%
\pgfpathlineto{\pgfqpoint{1.494728in}{3.378440in}}%
\pgfpathlineto{\pgfqpoint{1.494728in}{3.375491in}}%
\pgfpathmoveto{\pgfqpoint{1.503810in}{3.363694in}}%
\pgfpathlineto{\pgfqpoint{1.503810in}{3.363694in}}%
\pgfpathlineto{\pgfqpoint{1.503810in}{3.366643in}}%
\pgfpathlineto{\pgfqpoint{1.508351in}{3.366643in}}%
\pgfpathlineto{\pgfqpoint{1.508351in}{3.363694in}}%
\pgfpathmoveto{\pgfqpoint{1.508351in}{3.360745in}}%
\pgfpathlineto{\pgfqpoint{1.508351in}{3.360745in}}%
\pgfpathlineto{\pgfqpoint{1.508351in}{3.363694in}}%
\pgfpathlineto{\pgfqpoint{1.512892in}{3.363694in}}%
\pgfpathlineto{\pgfqpoint{1.512892in}{3.360745in}}%
\pgfpathmoveto{\pgfqpoint{1.508351in}{3.363694in}}%
\pgfpathlineto{\pgfqpoint{1.508351in}{3.363694in}}%
\pgfpathlineto{\pgfqpoint{1.508351in}{3.366643in}}%
\pgfpathlineto{\pgfqpoint{1.512892in}{3.366643in}}%
\pgfpathlineto{\pgfqpoint{1.512892in}{3.363694in}}%
\pgfpathmoveto{\pgfqpoint{1.499269in}{3.369593in}}%
\pgfpathlineto{\pgfqpoint{1.499269in}{3.369593in}}%
\pgfpathlineto{\pgfqpoint{1.499269in}{3.372542in}}%
\pgfpathlineto{\pgfqpoint{1.503810in}{3.372542in}}%
\pgfpathlineto{\pgfqpoint{1.503810in}{3.369593in}}%
\pgfpathmoveto{\pgfqpoint{1.494728in}{3.372542in}}%
\pgfpathlineto{\pgfqpoint{1.494728in}{3.372542in}}%
\pgfpathlineto{\pgfqpoint{1.494728in}{3.375491in}}%
\pgfpathlineto{\pgfqpoint{1.499269in}{3.375491in}}%
\pgfpathlineto{\pgfqpoint{1.499269in}{3.372542in}}%
\pgfpathmoveto{\pgfqpoint{1.494728in}{3.375491in}}%
\pgfpathlineto{\pgfqpoint{1.494728in}{3.375491in}}%
\pgfpathlineto{\pgfqpoint{1.494728in}{3.378440in}}%
\pgfpathlineto{\pgfqpoint{1.499269in}{3.378440in}}%
\pgfpathlineto{\pgfqpoint{1.499269in}{3.375491in}}%
\pgfpathmoveto{\pgfqpoint{1.499269in}{3.372542in}}%
\pgfpathlineto{\pgfqpoint{1.499269in}{3.372542in}}%
\pgfpathlineto{\pgfqpoint{1.499269in}{3.375491in}}%
\pgfpathlineto{\pgfqpoint{1.503810in}{3.375491in}}%
\pgfpathlineto{\pgfqpoint{1.503810in}{3.372542in}}%
\pgfpathmoveto{\pgfqpoint{1.503810in}{3.366643in}}%
\pgfpathlineto{\pgfqpoint{1.503810in}{3.366643in}}%
\pgfpathlineto{\pgfqpoint{1.503810in}{3.369593in}}%
\pgfpathlineto{\pgfqpoint{1.508351in}{3.369593in}}%
\pgfpathlineto{\pgfqpoint{1.508351in}{3.366643in}}%
\pgfpathmoveto{\pgfqpoint{1.503810in}{3.369593in}}%
\pgfpathlineto{\pgfqpoint{1.503810in}{3.369593in}}%
\pgfpathlineto{\pgfqpoint{1.503810in}{3.372542in}}%
\pgfpathlineto{\pgfqpoint{1.508351in}{3.372542in}}%
\pgfpathlineto{\pgfqpoint{1.508351in}{3.369593in}}%
\pgfpathmoveto{\pgfqpoint{1.517433in}{3.351897in}}%
\pgfpathlineto{\pgfqpoint{1.517433in}{3.351897in}}%
\pgfpathlineto{\pgfqpoint{1.517433in}{3.354847in}}%
\pgfpathlineto{\pgfqpoint{1.521974in}{3.354847in}}%
\pgfpathlineto{\pgfqpoint{1.521974in}{3.351897in}}%
\pgfpathmoveto{\pgfqpoint{1.526515in}{3.345999in}}%
\pgfpathlineto{\pgfqpoint{1.526515in}{3.345999in}}%
\pgfpathlineto{\pgfqpoint{1.526515in}{3.348948in}}%
\pgfpathlineto{\pgfqpoint{1.531056in}{3.348948in}}%
\pgfpathlineto{\pgfqpoint{1.531056in}{3.345999in}}%
\pgfpathmoveto{\pgfqpoint{1.521974in}{3.348948in}}%
\pgfpathlineto{\pgfqpoint{1.521974in}{3.348948in}}%
\pgfpathlineto{\pgfqpoint{1.521974in}{3.351897in}}%
\pgfpathlineto{\pgfqpoint{1.526515in}{3.351897in}}%
\pgfpathlineto{\pgfqpoint{1.526515in}{3.348948in}}%
\pgfpathmoveto{\pgfqpoint{1.521974in}{3.351897in}}%
\pgfpathlineto{\pgfqpoint{1.521974in}{3.351897in}}%
\pgfpathlineto{\pgfqpoint{1.521974in}{3.354847in}}%
\pgfpathlineto{\pgfqpoint{1.526515in}{3.354847in}}%
\pgfpathlineto{\pgfqpoint{1.526515in}{3.351897in}}%
\pgfpathmoveto{\pgfqpoint{1.526515in}{3.348948in}}%
\pgfpathlineto{\pgfqpoint{1.526515in}{3.348948in}}%
\pgfpathlineto{\pgfqpoint{1.526515in}{3.351897in}}%
\pgfpathlineto{\pgfqpoint{1.531056in}{3.351897in}}%
\pgfpathlineto{\pgfqpoint{1.531056in}{3.348948in}}%
\pgfpathmoveto{\pgfqpoint{1.531056in}{3.340100in}}%
\pgfpathlineto{\pgfqpoint{1.531056in}{3.340100in}}%
\pgfpathlineto{\pgfqpoint{1.531056in}{3.343050in}}%
\pgfpathlineto{\pgfqpoint{1.535597in}{3.343050in}}%
\pgfpathlineto{\pgfqpoint{1.535597in}{3.340100in}}%
\pgfpathmoveto{\pgfqpoint{1.535597in}{3.337151in}}%
\pgfpathlineto{\pgfqpoint{1.535597in}{3.337151in}}%
\pgfpathlineto{\pgfqpoint{1.535597in}{3.340100in}}%
\pgfpathlineto{\pgfqpoint{1.540138in}{3.340100in}}%
\pgfpathlineto{\pgfqpoint{1.540138in}{3.337151in}}%
\pgfpathmoveto{\pgfqpoint{1.535597in}{3.340100in}}%
\pgfpathlineto{\pgfqpoint{1.535597in}{3.340100in}}%
\pgfpathlineto{\pgfqpoint{1.535597in}{3.343050in}}%
\pgfpathlineto{\pgfqpoint{1.540138in}{3.343050in}}%
\pgfpathlineto{\pgfqpoint{1.540138in}{3.340100in}}%
\pgfpathmoveto{\pgfqpoint{1.540138in}{3.334202in}}%
\pgfpathlineto{\pgfqpoint{1.540138in}{3.334202in}}%
\pgfpathlineto{\pgfqpoint{1.540138in}{3.337151in}}%
\pgfpathlineto{\pgfqpoint{1.544679in}{3.337151in}}%
\pgfpathlineto{\pgfqpoint{1.544679in}{3.334202in}}%
\pgfpathmoveto{\pgfqpoint{1.544679in}{3.331253in}}%
\pgfpathlineto{\pgfqpoint{1.544679in}{3.331253in}}%
\pgfpathlineto{\pgfqpoint{1.544679in}{3.334202in}}%
\pgfpathlineto{\pgfqpoint{1.549220in}{3.334202in}}%
\pgfpathlineto{\pgfqpoint{1.549220in}{3.331253in}}%
\pgfpathmoveto{\pgfqpoint{1.544679in}{3.334202in}}%
\pgfpathlineto{\pgfqpoint{1.544679in}{3.334202in}}%
\pgfpathlineto{\pgfqpoint{1.544679in}{3.337151in}}%
\pgfpathlineto{\pgfqpoint{1.549220in}{3.337151in}}%
\pgfpathlineto{\pgfqpoint{1.549220in}{3.334202in}}%
\pgfpathmoveto{\pgfqpoint{1.540138in}{3.337151in}}%
\pgfpathlineto{\pgfqpoint{1.540138in}{3.337151in}}%
\pgfpathlineto{\pgfqpoint{1.540138in}{3.340100in}}%
\pgfpathlineto{\pgfqpoint{1.544679in}{3.340100in}}%
\pgfpathlineto{\pgfqpoint{1.544679in}{3.337151in}}%
\pgfpathmoveto{\pgfqpoint{1.531056in}{3.343050in}}%
\pgfpathlineto{\pgfqpoint{1.531056in}{3.343050in}}%
\pgfpathlineto{\pgfqpoint{1.531056in}{3.345999in}}%
\pgfpathlineto{\pgfqpoint{1.535597in}{3.345999in}}%
\pgfpathlineto{\pgfqpoint{1.535597in}{3.343050in}}%
\pgfpathmoveto{\pgfqpoint{1.531056in}{3.345999in}}%
\pgfpathlineto{\pgfqpoint{1.531056in}{3.345999in}}%
\pgfpathlineto{\pgfqpoint{1.531056in}{3.348948in}}%
\pgfpathlineto{\pgfqpoint{1.535597in}{3.348948in}}%
\pgfpathlineto{\pgfqpoint{1.535597in}{3.345999in}}%
\pgfpathmoveto{\pgfqpoint{1.512892in}{3.357796in}}%
\pgfpathlineto{\pgfqpoint{1.512892in}{3.357796in}}%
\pgfpathlineto{\pgfqpoint{1.512892in}{3.360745in}}%
\pgfpathlineto{\pgfqpoint{1.517433in}{3.360745in}}%
\pgfpathlineto{\pgfqpoint{1.517433in}{3.357796in}}%
\pgfpathmoveto{\pgfqpoint{1.517433in}{3.354847in}}%
\pgfpathlineto{\pgfqpoint{1.517433in}{3.354847in}}%
\pgfpathlineto{\pgfqpoint{1.517433in}{3.357796in}}%
\pgfpathlineto{\pgfqpoint{1.521974in}{3.357796in}}%
\pgfpathlineto{\pgfqpoint{1.521974in}{3.354847in}}%
\pgfpathmoveto{\pgfqpoint{1.517433in}{3.357796in}}%
\pgfpathlineto{\pgfqpoint{1.517433in}{3.357796in}}%
\pgfpathlineto{\pgfqpoint{1.517433in}{3.360745in}}%
\pgfpathlineto{\pgfqpoint{1.521974in}{3.360745in}}%
\pgfpathlineto{\pgfqpoint{1.521974in}{3.357796in}}%
\pgfpathmoveto{\pgfqpoint{1.512892in}{3.360745in}}%
\pgfpathlineto{\pgfqpoint{1.512892in}{3.360745in}}%
\pgfpathlineto{\pgfqpoint{1.512892in}{3.363694in}}%
\pgfpathlineto{\pgfqpoint{1.517433in}{3.363694in}}%
\pgfpathlineto{\pgfqpoint{1.517433in}{3.360745in}}%
\pgfpathmoveto{\pgfqpoint{1.476565in}{3.387288in}}%
\pgfpathlineto{\pgfqpoint{1.476565in}{3.387288in}}%
\pgfpathlineto{\pgfqpoint{1.476565in}{3.390237in}}%
\pgfpathlineto{\pgfqpoint{1.481106in}{3.390237in}}%
\pgfpathlineto{\pgfqpoint{1.481106in}{3.387288in}}%
\pgfpathmoveto{\pgfqpoint{1.481106in}{3.384339in}}%
\pgfpathlineto{\pgfqpoint{1.481106in}{3.384339in}}%
\pgfpathlineto{\pgfqpoint{1.481106in}{3.387288in}}%
\pgfpathlineto{\pgfqpoint{1.485647in}{3.387288in}}%
\pgfpathlineto{\pgfqpoint{1.485647in}{3.384339in}}%
\pgfpathmoveto{\pgfqpoint{1.481106in}{3.387288in}}%
\pgfpathlineto{\pgfqpoint{1.481106in}{3.387288in}}%
\pgfpathlineto{\pgfqpoint{1.481106in}{3.390237in}}%
\pgfpathlineto{\pgfqpoint{1.485647in}{3.390237in}}%
\pgfpathlineto{\pgfqpoint{1.485647in}{3.387288in}}%
\pgfpathmoveto{\pgfqpoint{1.485647in}{3.381389in}}%
\pgfpathlineto{\pgfqpoint{1.485647in}{3.381389in}}%
\pgfpathlineto{\pgfqpoint{1.485647in}{3.384339in}}%
\pgfpathlineto{\pgfqpoint{1.490187in}{3.384339in}}%
\pgfpathlineto{\pgfqpoint{1.490187in}{3.381389in}}%
\pgfpathmoveto{\pgfqpoint{1.490187in}{3.378440in}}%
\pgfpathlineto{\pgfqpoint{1.490187in}{3.378440in}}%
\pgfpathlineto{\pgfqpoint{1.490187in}{3.381389in}}%
\pgfpathlineto{\pgfqpoint{1.494728in}{3.381389in}}%
\pgfpathlineto{\pgfqpoint{1.494728in}{3.378440in}}%
\pgfpathmoveto{\pgfqpoint{1.490187in}{3.381389in}}%
\pgfpathlineto{\pgfqpoint{1.490187in}{3.381389in}}%
\pgfpathlineto{\pgfqpoint{1.490187in}{3.384339in}}%
\pgfpathlineto{\pgfqpoint{1.494728in}{3.384339in}}%
\pgfpathlineto{\pgfqpoint{1.494728in}{3.381389in}}%
\pgfpathmoveto{\pgfqpoint{1.485647in}{3.384339in}}%
\pgfpathlineto{\pgfqpoint{1.485647in}{3.384339in}}%
\pgfpathlineto{\pgfqpoint{1.485647in}{3.387288in}}%
\pgfpathlineto{\pgfqpoint{1.490187in}{3.387288in}}%
\pgfpathlineto{\pgfqpoint{1.490187in}{3.384339in}}%
\pgfpathmoveto{\pgfqpoint{1.476565in}{3.390237in}}%
\pgfpathlineto{\pgfqpoint{1.476565in}{3.390237in}}%
\pgfpathlineto{\pgfqpoint{1.476565in}{3.393186in}}%
\pgfpathlineto{\pgfqpoint{1.481106in}{3.393186in}}%
\pgfpathlineto{\pgfqpoint{1.481106in}{3.390237in}}%
\pgfpathmoveto{\pgfqpoint{1.476565in}{3.393186in}}%
\pgfpathlineto{\pgfqpoint{1.476565in}{3.393186in}}%
\pgfpathlineto{\pgfqpoint{1.476565in}{3.396135in}}%
\pgfpathlineto{\pgfqpoint{1.481106in}{3.396135in}}%
\pgfpathlineto{\pgfqpoint{1.481106in}{3.393186in}}%
\pgfpathmoveto{\pgfqpoint{1.762650in}{3.139550in}}%
\pgfpathlineto{\pgfqpoint{1.762650in}{3.139550in}}%
\pgfpathlineto{\pgfqpoint{1.762650in}{3.142499in}}%
\pgfpathlineto{\pgfqpoint{1.767191in}{3.142499in}}%
\pgfpathlineto{\pgfqpoint{1.767191in}{3.139550in}}%
\pgfpathmoveto{\pgfqpoint{1.680909in}{3.210331in}}%
\pgfpathlineto{\pgfqpoint{1.680909in}{3.210331in}}%
\pgfpathlineto{\pgfqpoint{1.680909in}{3.213280in}}%
\pgfpathlineto{\pgfqpoint{1.685451in}{3.213280in}}%
\pgfpathlineto{\pgfqpoint{1.685451in}{3.210331in}}%
\pgfpathmoveto{\pgfqpoint{1.689992in}{3.204433in}}%
\pgfpathlineto{\pgfqpoint{1.689992in}{3.204433in}}%
\pgfpathlineto{\pgfqpoint{1.689992in}{3.207382in}}%
\pgfpathlineto{\pgfqpoint{1.694533in}{3.207382in}}%
\pgfpathlineto{\pgfqpoint{1.694533in}{3.204433in}}%
\pgfpathmoveto{\pgfqpoint{1.685451in}{3.207382in}}%
\pgfpathlineto{\pgfqpoint{1.685451in}{3.207382in}}%
\pgfpathlineto{\pgfqpoint{1.685451in}{3.210331in}}%
\pgfpathlineto{\pgfqpoint{1.689992in}{3.210331in}}%
\pgfpathlineto{\pgfqpoint{1.689992in}{3.207382in}}%
\pgfpathmoveto{\pgfqpoint{1.685451in}{3.210331in}}%
\pgfpathlineto{\pgfqpoint{1.685451in}{3.210331in}}%
\pgfpathlineto{\pgfqpoint{1.685451in}{3.213280in}}%
\pgfpathlineto{\pgfqpoint{1.689992in}{3.213280in}}%
\pgfpathlineto{\pgfqpoint{1.689992in}{3.210331in}}%
\pgfpathmoveto{\pgfqpoint{1.689992in}{3.207382in}}%
\pgfpathlineto{\pgfqpoint{1.689992in}{3.207382in}}%
\pgfpathlineto{\pgfqpoint{1.689992in}{3.210331in}}%
\pgfpathlineto{\pgfqpoint{1.694533in}{3.210331in}}%
\pgfpathlineto{\pgfqpoint{1.694533in}{3.207382in}}%
\pgfpathmoveto{\pgfqpoint{1.671827in}{3.219179in}}%
\pgfpathlineto{\pgfqpoint{1.671827in}{3.219179in}}%
\pgfpathlineto{\pgfqpoint{1.671827in}{3.222128in}}%
\pgfpathlineto{\pgfqpoint{1.676368in}{3.222128in}}%
\pgfpathlineto{\pgfqpoint{1.676368in}{3.219179in}}%
\pgfpathmoveto{\pgfqpoint{1.671827in}{3.222128in}}%
\pgfpathlineto{\pgfqpoint{1.671827in}{3.222128in}}%
\pgfpathlineto{\pgfqpoint{1.671827in}{3.225077in}}%
\pgfpathlineto{\pgfqpoint{1.676368in}{3.225077in}}%
\pgfpathlineto{\pgfqpoint{1.676368in}{3.222128in}}%
\pgfpathmoveto{\pgfqpoint{1.662745in}{3.228026in}}%
\pgfpathlineto{\pgfqpoint{1.662745in}{3.228026in}}%
\pgfpathlineto{\pgfqpoint{1.662745in}{3.230975in}}%
\pgfpathlineto{\pgfqpoint{1.667286in}{3.230975in}}%
\pgfpathlineto{\pgfqpoint{1.667286in}{3.228026in}}%
\pgfpathmoveto{\pgfqpoint{1.658204in}{3.230975in}}%
\pgfpathlineto{\pgfqpoint{1.658204in}{3.230975in}}%
\pgfpathlineto{\pgfqpoint{1.658204in}{3.233925in}}%
\pgfpathlineto{\pgfqpoint{1.662745in}{3.233925in}}%
\pgfpathlineto{\pgfqpoint{1.662745in}{3.230975in}}%
\pgfpathmoveto{\pgfqpoint{1.658204in}{3.233925in}}%
\pgfpathlineto{\pgfqpoint{1.658204in}{3.233925in}}%
\pgfpathlineto{\pgfqpoint{1.658204in}{3.236874in}}%
\pgfpathlineto{\pgfqpoint{1.662745in}{3.236874in}}%
\pgfpathlineto{\pgfqpoint{1.662745in}{3.233925in}}%
\pgfpathmoveto{\pgfqpoint{1.662745in}{3.230975in}}%
\pgfpathlineto{\pgfqpoint{1.662745in}{3.230975in}}%
\pgfpathlineto{\pgfqpoint{1.662745in}{3.233925in}}%
\pgfpathlineto{\pgfqpoint{1.667286in}{3.233925in}}%
\pgfpathlineto{\pgfqpoint{1.667286in}{3.230975in}}%
\pgfpathmoveto{\pgfqpoint{1.667286in}{3.225077in}}%
\pgfpathlineto{\pgfqpoint{1.667286in}{3.225077in}}%
\pgfpathlineto{\pgfqpoint{1.667286in}{3.228026in}}%
\pgfpathlineto{\pgfqpoint{1.671827in}{3.228026in}}%
\pgfpathlineto{\pgfqpoint{1.671827in}{3.225077in}}%
\pgfpathmoveto{\pgfqpoint{1.667286in}{3.228026in}}%
\pgfpathlineto{\pgfqpoint{1.667286in}{3.228026in}}%
\pgfpathlineto{\pgfqpoint{1.667286in}{3.230975in}}%
\pgfpathlineto{\pgfqpoint{1.671827in}{3.230975in}}%
\pgfpathlineto{\pgfqpoint{1.671827in}{3.228026in}}%
\pgfpathmoveto{\pgfqpoint{1.671827in}{3.225077in}}%
\pgfpathlineto{\pgfqpoint{1.671827in}{3.225077in}}%
\pgfpathlineto{\pgfqpoint{1.671827in}{3.228026in}}%
\pgfpathlineto{\pgfqpoint{1.676368in}{3.228026in}}%
\pgfpathlineto{\pgfqpoint{1.676368in}{3.225077in}}%
\pgfpathmoveto{\pgfqpoint{1.676368in}{3.216229in}}%
\pgfpathlineto{\pgfqpoint{1.676368in}{3.216229in}}%
\pgfpathlineto{\pgfqpoint{1.676368in}{3.219179in}}%
\pgfpathlineto{\pgfqpoint{1.680909in}{3.219179in}}%
\pgfpathlineto{\pgfqpoint{1.680909in}{3.216229in}}%
\pgfpathmoveto{\pgfqpoint{1.680909in}{3.213280in}}%
\pgfpathlineto{\pgfqpoint{1.680909in}{3.213280in}}%
\pgfpathlineto{\pgfqpoint{1.680909in}{3.216229in}}%
\pgfpathlineto{\pgfqpoint{1.685451in}{3.216229in}}%
\pgfpathlineto{\pgfqpoint{1.685451in}{3.213280in}}%
\pgfpathmoveto{\pgfqpoint{1.680909in}{3.216229in}}%
\pgfpathlineto{\pgfqpoint{1.680909in}{3.216229in}}%
\pgfpathlineto{\pgfqpoint{1.680909in}{3.219179in}}%
\pgfpathlineto{\pgfqpoint{1.685451in}{3.219179in}}%
\pgfpathlineto{\pgfqpoint{1.685451in}{3.216229in}}%
\pgfpathmoveto{\pgfqpoint{1.676368in}{3.219179in}}%
\pgfpathlineto{\pgfqpoint{1.676368in}{3.219179in}}%
\pgfpathlineto{\pgfqpoint{1.676368in}{3.222128in}}%
\pgfpathlineto{\pgfqpoint{1.680909in}{3.222128in}}%
\pgfpathlineto{\pgfqpoint{1.680909in}{3.219179in}}%
\pgfpathmoveto{\pgfqpoint{1.708156in}{3.186737in}}%
\pgfpathlineto{\pgfqpoint{1.708156in}{3.186737in}}%
\pgfpathlineto{\pgfqpoint{1.708156in}{3.189686in}}%
\pgfpathlineto{\pgfqpoint{1.712698in}{3.189686in}}%
\pgfpathlineto{\pgfqpoint{1.712698in}{3.186737in}}%
\pgfpathmoveto{\pgfqpoint{1.721780in}{3.174940in}}%
\pgfpathlineto{\pgfqpoint{1.721780in}{3.174940in}}%
\pgfpathlineto{\pgfqpoint{1.721780in}{3.177890in}}%
\pgfpathlineto{\pgfqpoint{1.726321in}{3.177890in}}%
\pgfpathlineto{\pgfqpoint{1.726321in}{3.174940in}}%
\pgfpathmoveto{\pgfqpoint{1.726321in}{3.171991in}}%
\pgfpathlineto{\pgfqpoint{1.726321in}{3.171991in}}%
\pgfpathlineto{\pgfqpoint{1.726321in}{3.174940in}}%
\pgfpathlineto{\pgfqpoint{1.730862in}{3.174940in}}%
\pgfpathlineto{\pgfqpoint{1.730862in}{3.171991in}}%
\pgfpathmoveto{\pgfqpoint{1.726321in}{3.174940in}}%
\pgfpathlineto{\pgfqpoint{1.726321in}{3.174940in}}%
\pgfpathlineto{\pgfqpoint{1.726321in}{3.177890in}}%
\pgfpathlineto{\pgfqpoint{1.730862in}{3.177890in}}%
\pgfpathlineto{\pgfqpoint{1.730862in}{3.174940in}}%
\pgfpathmoveto{\pgfqpoint{1.717239in}{3.180839in}}%
\pgfpathlineto{\pgfqpoint{1.717239in}{3.180839in}}%
\pgfpathlineto{\pgfqpoint{1.717239in}{3.183788in}}%
\pgfpathlineto{\pgfqpoint{1.721780in}{3.183788in}}%
\pgfpathlineto{\pgfqpoint{1.721780in}{3.180839in}}%
\pgfpathmoveto{\pgfqpoint{1.712698in}{3.183788in}}%
\pgfpathlineto{\pgfqpoint{1.712698in}{3.183788in}}%
\pgfpathlineto{\pgfqpoint{1.712698in}{3.186737in}}%
\pgfpathlineto{\pgfqpoint{1.717239in}{3.186737in}}%
\pgfpathlineto{\pgfqpoint{1.717239in}{3.183788in}}%
\pgfpathmoveto{\pgfqpoint{1.712698in}{3.186737in}}%
\pgfpathlineto{\pgfqpoint{1.712698in}{3.186737in}}%
\pgfpathlineto{\pgfqpoint{1.712698in}{3.189686in}}%
\pgfpathlineto{\pgfqpoint{1.717239in}{3.189686in}}%
\pgfpathlineto{\pgfqpoint{1.717239in}{3.186737in}}%
\pgfpathmoveto{\pgfqpoint{1.717239in}{3.183788in}}%
\pgfpathlineto{\pgfqpoint{1.717239in}{3.183788in}}%
\pgfpathlineto{\pgfqpoint{1.717239in}{3.186737in}}%
\pgfpathlineto{\pgfqpoint{1.721780in}{3.186737in}}%
\pgfpathlineto{\pgfqpoint{1.721780in}{3.183788in}}%
\pgfpathmoveto{\pgfqpoint{1.721780in}{3.177890in}}%
\pgfpathlineto{\pgfqpoint{1.721780in}{3.177890in}}%
\pgfpathlineto{\pgfqpoint{1.721780in}{3.180839in}}%
\pgfpathlineto{\pgfqpoint{1.726321in}{3.180839in}}%
\pgfpathlineto{\pgfqpoint{1.726321in}{3.177890in}}%
\pgfpathmoveto{\pgfqpoint{1.721780in}{3.180839in}}%
\pgfpathlineto{\pgfqpoint{1.721780in}{3.180839in}}%
\pgfpathlineto{\pgfqpoint{1.721780in}{3.183788in}}%
\pgfpathlineto{\pgfqpoint{1.726321in}{3.183788in}}%
\pgfpathlineto{\pgfqpoint{1.726321in}{3.180839in}}%
\pgfpathmoveto{\pgfqpoint{1.735403in}{3.163144in}}%
\pgfpathlineto{\pgfqpoint{1.735403in}{3.163144in}}%
\pgfpathlineto{\pgfqpoint{1.735403in}{3.166093in}}%
\pgfpathlineto{\pgfqpoint{1.739944in}{3.166093in}}%
\pgfpathlineto{\pgfqpoint{1.739944in}{3.163144in}}%
\pgfpathmoveto{\pgfqpoint{1.744486in}{3.157245in}}%
\pgfpathlineto{\pgfqpoint{1.744486in}{3.157245in}}%
\pgfpathlineto{\pgfqpoint{1.744486in}{3.160194in}}%
\pgfpathlineto{\pgfqpoint{1.749027in}{3.160194in}}%
\pgfpathlineto{\pgfqpoint{1.749027in}{3.157245in}}%
\pgfpathmoveto{\pgfqpoint{1.739944in}{3.160194in}}%
\pgfpathlineto{\pgfqpoint{1.739944in}{3.160194in}}%
\pgfpathlineto{\pgfqpoint{1.739944in}{3.163144in}}%
\pgfpathlineto{\pgfqpoint{1.744486in}{3.163144in}}%
\pgfpathlineto{\pgfqpoint{1.744486in}{3.160194in}}%
\pgfpathmoveto{\pgfqpoint{1.739944in}{3.163144in}}%
\pgfpathlineto{\pgfqpoint{1.739944in}{3.163144in}}%
\pgfpathlineto{\pgfqpoint{1.739944in}{3.166093in}}%
\pgfpathlineto{\pgfqpoint{1.744486in}{3.166093in}}%
\pgfpathlineto{\pgfqpoint{1.744486in}{3.163144in}}%
\pgfpathmoveto{\pgfqpoint{1.744486in}{3.160194in}}%
\pgfpathlineto{\pgfqpoint{1.744486in}{3.160194in}}%
\pgfpathlineto{\pgfqpoint{1.744486in}{3.163144in}}%
\pgfpathlineto{\pgfqpoint{1.749027in}{3.163144in}}%
\pgfpathlineto{\pgfqpoint{1.749027in}{3.160194in}}%
\pgfpathmoveto{\pgfqpoint{1.749027in}{3.151347in}}%
\pgfpathlineto{\pgfqpoint{1.749027in}{3.151347in}}%
\pgfpathlineto{\pgfqpoint{1.749027in}{3.154296in}}%
\pgfpathlineto{\pgfqpoint{1.753568in}{3.154296in}}%
\pgfpathlineto{\pgfqpoint{1.753568in}{3.151347in}}%
\pgfpathmoveto{\pgfqpoint{1.753568in}{3.148398in}}%
\pgfpathlineto{\pgfqpoint{1.753568in}{3.148398in}}%
\pgfpathlineto{\pgfqpoint{1.753568in}{3.151347in}}%
\pgfpathlineto{\pgfqpoint{1.758109in}{3.151347in}}%
\pgfpathlineto{\pgfqpoint{1.758109in}{3.148398in}}%
\pgfpathmoveto{\pgfqpoint{1.753568in}{3.151347in}}%
\pgfpathlineto{\pgfqpoint{1.753568in}{3.151347in}}%
\pgfpathlineto{\pgfqpoint{1.753568in}{3.154296in}}%
\pgfpathlineto{\pgfqpoint{1.758109in}{3.154296in}}%
\pgfpathlineto{\pgfqpoint{1.758109in}{3.151347in}}%
\pgfpathmoveto{\pgfqpoint{1.758109in}{3.145448in}}%
\pgfpathlineto{\pgfqpoint{1.758109in}{3.145448in}}%
\pgfpathlineto{\pgfqpoint{1.758109in}{3.148398in}}%
\pgfpathlineto{\pgfqpoint{1.762650in}{3.148398in}}%
\pgfpathlineto{\pgfqpoint{1.762650in}{3.145448in}}%
\pgfpathmoveto{\pgfqpoint{1.762650in}{3.142499in}}%
\pgfpathlineto{\pgfqpoint{1.762650in}{3.142499in}}%
\pgfpathlineto{\pgfqpoint{1.762650in}{3.145448in}}%
\pgfpathlineto{\pgfqpoint{1.767191in}{3.145448in}}%
\pgfpathlineto{\pgfqpoint{1.767191in}{3.142499in}}%
\pgfpathmoveto{\pgfqpoint{1.762650in}{3.145448in}}%
\pgfpathlineto{\pgfqpoint{1.762650in}{3.145448in}}%
\pgfpathlineto{\pgfqpoint{1.762650in}{3.148398in}}%
\pgfpathlineto{\pgfqpoint{1.767191in}{3.148398in}}%
\pgfpathlineto{\pgfqpoint{1.767191in}{3.145448in}}%
\pgfpathmoveto{\pgfqpoint{1.758109in}{3.148398in}}%
\pgfpathlineto{\pgfqpoint{1.758109in}{3.148398in}}%
\pgfpathlineto{\pgfqpoint{1.758109in}{3.151347in}}%
\pgfpathlineto{\pgfqpoint{1.762650in}{3.151347in}}%
\pgfpathlineto{\pgfqpoint{1.762650in}{3.148398in}}%
\pgfpathmoveto{\pgfqpoint{1.749027in}{3.154296in}}%
\pgfpathlineto{\pgfqpoint{1.749027in}{3.154296in}}%
\pgfpathlineto{\pgfqpoint{1.749027in}{3.157245in}}%
\pgfpathlineto{\pgfqpoint{1.753568in}{3.157245in}}%
\pgfpathlineto{\pgfqpoint{1.753568in}{3.154296in}}%
\pgfpathmoveto{\pgfqpoint{1.749027in}{3.157245in}}%
\pgfpathlineto{\pgfqpoint{1.749027in}{3.157245in}}%
\pgfpathlineto{\pgfqpoint{1.749027in}{3.160194in}}%
\pgfpathlineto{\pgfqpoint{1.753568in}{3.160194in}}%
\pgfpathlineto{\pgfqpoint{1.753568in}{3.157245in}}%
\pgfpathmoveto{\pgfqpoint{1.730862in}{3.169042in}}%
\pgfpathlineto{\pgfqpoint{1.730862in}{3.169042in}}%
\pgfpathlineto{\pgfqpoint{1.730862in}{3.171991in}}%
\pgfpathlineto{\pgfqpoint{1.735403in}{3.171991in}}%
\pgfpathlineto{\pgfqpoint{1.735403in}{3.169042in}}%
\pgfpathmoveto{\pgfqpoint{1.735403in}{3.166093in}}%
\pgfpathlineto{\pgfqpoint{1.735403in}{3.166093in}}%
\pgfpathlineto{\pgfqpoint{1.735403in}{3.169042in}}%
\pgfpathlineto{\pgfqpoint{1.739944in}{3.169042in}}%
\pgfpathlineto{\pgfqpoint{1.739944in}{3.166093in}}%
\pgfpathmoveto{\pgfqpoint{1.735403in}{3.169042in}}%
\pgfpathlineto{\pgfqpoint{1.735403in}{3.169042in}}%
\pgfpathlineto{\pgfqpoint{1.735403in}{3.171991in}}%
\pgfpathlineto{\pgfqpoint{1.739944in}{3.171991in}}%
\pgfpathlineto{\pgfqpoint{1.739944in}{3.169042in}}%
\pgfpathmoveto{\pgfqpoint{1.730862in}{3.171991in}}%
\pgfpathlineto{\pgfqpoint{1.730862in}{3.171991in}}%
\pgfpathlineto{\pgfqpoint{1.730862in}{3.174940in}}%
\pgfpathlineto{\pgfqpoint{1.735403in}{3.174940in}}%
\pgfpathlineto{\pgfqpoint{1.735403in}{3.171991in}}%
\pgfpathmoveto{\pgfqpoint{1.694533in}{3.198534in}}%
\pgfpathlineto{\pgfqpoint{1.694533in}{3.198534in}}%
\pgfpathlineto{\pgfqpoint{1.694533in}{3.201483in}}%
\pgfpathlineto{\pgfqpoint{1.699074in}{3.201483in}}%
\pgfpathlineto{\pgfqpoint{1.699074in}{3.198534in}}%
\pgfpathmoveto{\pgfqpoint{1.699074in}{3.195585in}}%
\pgfpathlineto{\pgfqpoint{1.699074in}{3.195585in}}%
\pgfpathlineto{\pgfqpoint{1.699074in}{3.198534in}}%
\pgfpathlineto{\pgfqpoint{1.703615in}{3.198534in}}%
\pgfpathlineto{\pgfqpoint{1.703615in}{3.195585in}}%
\pgfpathmoveto{\pgfqpoint{1.699074in}{3.198534in}}%
\pgfpathlineto{\pgfqpoint{1.699074in}{3.198534in}}%
\pgfpathlineto{\pgfqpoint{1.699074in}{3.201483in}}%
\pgfpathlineto{\pgfqpoint{1.703615in}{3.201483in}}%
\pgfpathlineto{\pgfqpoint{1.703615in}{3.198534in}}%
\pgfpathmoveto{\pgfqpoint{1.703615in}{3.192636in}}%
\pgfpathlineto{\pgfqpoint{1.703615in}{3.192636in}}%
\pgfpathlineto{\pgfqpoint{1.703615in}{3.195585in}}%
\pgfpathlineto{\pgfqpoint{1.708156in}{3.195585in}}%
\pgfpathlineto{\pgfqpoint{1.708156in}{3.192636in}}%
\pgfpathmoveto{\pgfqpoint{1.708156in}{3.189686in}}%
\pgfpathlineto{\pgfqpoint{1.708156in}{3.189686in}}%
\pgfpathlineto{\pgfqpoint{1.708156in}{3.192636in}}%
\pgfpathlineto{\pgfqpoint{1.712698in}{3.192636in}}%
\pgfpathlineto{\pgfqpoint{1.712698in}{3.189686in}}%
\pgfpathmoveto{\pgfqpoint{1.708156in}{3.192636in}}%
\pgfpathlineto{\pgfqpoint{1.708156in}{3.192636in}}%
\pgfpathlineto{\pgfqpoint{1.708156in}{3.195585in}}%
\pgfpathlineto{\pgfqpoint{1.712698in}{3.195585in}}%
\pgfpathlineto{\pgfqpoint{1.712698in}{3.192636in}}%
\pgfpathmoveto{\pgfqpoint{1.703615in}{3.195585in}}%
\pgfpathlineto{\pgfqpoint{1.703615in}{3.195585in}}%
\pgfpathlineto{\pgfqpoint{1.703615in}{3.198534in}}%
\pgfpathlineto{\pgfqpoint{1.708156in}{3.198534in}}%
\pgfpathlineto{\pgfqpoint{1.708156in}{3.195585in}}%
\pgfpathmoveto{\pgfqpoint{1.694533in}{3.201483in}}%
\pgfpathlineto{\pgfqpoint{1.694533in}{3.201483in}}%
\pgfpathlineto{\pgfqpoint{1.694533in}{3.204433in}}%
\pgfpathlineto{\pgfqpoint{1.699074in}{3.204433in}}%
\pgfpathlineto{\pgfqpoint{1.699074in}{3.201483in}}%
\pgfpathmoveto{\pgfqpoint{1.694533in}{3.204433in}}%
\pgfpathlineto{\pgfqpoint{1.694533in}{3.204433in}}%
\pgfpathlineto{\pgfqpoint{1.694533in}{3.207382in}}%
\pgfpathlineto{\pgfqpoint{1.699074in}{3.207382in}}%
\pgfpathlineto{\pgfqpoint{1.699074in}{3.204433in}}%
\pgfpathmoveto{\pgfqpoint{1.635498in}{3.251621in}}%
\pgfpathlineto{\pgfqpoint{1.635498in}{3.251621in}}%
\pgfpathlineto{\pgfqpoint{1.635498in}{3.254570in}}%
\pgfpathlineto{\pgfqpoint{1.640039in}{3.254570in}}%
\pgfpathlineto{\pgfqpoint{1.640039in}{3.251621in}}%
\pgfpathmoveto{\pgfqpoint{1.630957in}{3.254570in}}%
\pgfpathlineto{\pgfqpoint{1.630957in}{3.254570in}}%
\pgfpathlineto{\pgfqpoint{1.630957in}{3.257519in}}%
\pgfpathlineto{\pgfqpoint{1.635498in}{3.257519in}}%
\pgfpathlineto{\pgfqpoint{1.635498in}{3.254570in}}%
\pgfpathmoveto{\pgfqpoint{1.630957in}{3.257519in}}%
\pgfpathlineto{\pgfqpoint{1.630957in}{3.257519in}}%
\pgfpathlineto{\pgfqpoint{1.630957in}{3.260469in}}%
\pgfpathlineto{\pgfqpoint{1.635498in}{3.260469in}}%
\pgfpathlineto{\pgfqpoint{1.635498in}{3.257519in}}%
\pgfpathmoveto{\pgfqpoint{1.635498in}{3.254570in}}%
\pgfpathlineto{\pgfqpoint{1.635498in}{3.254570in}}%
\pgfpathlineto{\pgfqpoint{1.635498in}{3.257519in}}%
\pgfpathlineto{\pgfqpoint{1.640039in}{3.257519in}}%
\pgfpathlineto{\pgfqpoint{1.640039in}{3.254570in}}%
\pgfpathmoveto{\pgfqpoint{1.644580in}{3.242773in}}%
\pgfpathlineto{\pgfqpoint{1.644580in}{3.242773in}}%
\pgfpathlineto{\pgfqpoint{1.644580in}{3.245722in}}%
\pgfpathlineto{\pgfqpoint{1.649121in}{3.245722in}}%
\pgfpathlineto{\pgfqpoint{1.649121in}{3.242773in}}%
\pgfpathmoveto{\pgfqpoint{1.644580in}{3.245722in}}%
\pgfpathlineto{\pgfqpoint{1.644580in}{3.245722in}}%
\pgfpathlineto{\pgfqpoint{1.644580in}{3.248671in}}%
\pgfpathlineto{\pgfqpoint{1.649121in}{3.248671in}}%
\pgfpathlineto{\pgfqpoint{1.649121in}{3.245722in}}%
\pgfpathmoveto{\pgfqpoint{1.649121in}{3.239823in}}%
\pgfpathlineto{\pgfqpoint{1.649121in}{3.239823in}}%
\pgfpathlineto{\pgfqpoint{1.649121in}{3.242773in}}%
\pgfpathlineto{\pgfqpoint{1.653662in}{3.242773in}}%
\pgfpathlineto{\pgfqpoint{1.653662in}{3.239823in}}%
\pgfpathmoveto{\pgfqpoint{1.653662in}{3.236874in}}%
\pgfpathlineto{\pgfqpoint{1.653662in}{3.236874in}}%
\pgfpathlineto{\pgfqpoint{1.653662in}{3.239823in}}%
\pgfpathlineto{\pgfqpoint{1.658204in}{3.239823in}}%
\pgfpathlineto{\pgfqpoint{1.658204in}{3.236874in}}%
\pgfpathmoveto{\pgfqpoint{1.653662in}{3.239823in}}%
\pgfpathlineto{\pgfqpoint{1.653662in}{3.239823in}}%
\pgfpathlineto{\pgfqpoint{1.653662in}{3.242773in}}%
\pgfpathlineto{\pgfqpoint{1.658204in}{3.242773in}}%
\pgfpathlineto{\pgfqpoint{1.658204in}{3.239823in}}%
\pgfpathmoveto{\pgfqpoint{1.649121in}{3.242773in}}%
\pgfpathlineto{\pgfqpoint{1.649121in}{3.242773in}}%
\pgfpathlineto{\pgfqpoint{1.649121in}{3.245722in}}%
\pgfpathlineto{\pgfqpoint{1.653662in}{3.245722in}}%
\pgfpathlineto{\pgfqpoint{1.653662in}{3.242773in}}%
\pgfpathmoveto{\pgfqpoint{1.640039in}{3.248671in}}%
\pgfpathlineto{\pgfqpoint{1.640039in}{3.248671in}}%
\pgfpathlineto{\pgfqpoint{1.640039in}{3.251621in}}%
\pgfpathlineto{\pgfqpoint{1.644580in}{3.251621in}}%
\pgfpathlineto{\pgfqpoint{1.644580in}{3.248671in}}%
\pgfpathmoveto{\pgfqpoint{1.640039in}{3.251621in}}%
\pgfpathlineto{\pgfqpoint{1.640039in}{3.251621in}}%
\pgfpathlineto{\pgfqpoint{1.640039in}{3.254570in}}%
\pgfpathlineto{\pgfqpoint{1.644580in}{3.254570in}}%
\pgfpathlineto{\pgfqpoint{1.644580in}{3.251621in}}%
\pgfpathmoveto{\pgfqpoint{1.644580in}{3.248671in}}%
\pgfpathlineto{\pgfqpoint{1.644580in}{3.248671in}}%
\pgfpathlineto{\pgfqpoint{1.644580in}{3.251621in}}%
\pgfpathlineto{\pgfqpoint{1.649121in}{3.251621in}}%
\pgfpathlineto{\pgfqpoint{1.649121in}{3.248671in}}%
\pgfpathmoveto{\pgfqpoint{1.621874in}{3.263418in}}%
\pgfpathlineto{\pgfqpoint{1.621874in}{3.263418in}}%
\pgfpathlineto{\pgfqpoint{1.621874in}{3.266367in}}%
\pgfpathlineto{\pgfqpoint{1.626416in}{3.266367in}}%
\pgfpathlineto{\pgfqpoint{1.626416in}{3.263418in}}%
\pgfpathmoveto{\pgfqpoint{1.626416in}{3.260469in}}%
\pgfpathlineto{\pgfqpoint{1.626416in}{3.260469in}}%
\pgfpathlineto{\pgfqpoint{1.626416in}{3.263418in}}%
\pgfpathlineto{\pgfqpoint{1.630957in}{3.263418in}}%
\pgfpathlineto{\pgfqpoint{1.630957in}{3.260469in}}%
\pgfpathmoveto{\pgfqpoint{1.626416in}{3.263418in}}%
\pgfpathlineto{\pgfqpoint{1.626416in}{3.263418in}}%
\pgfpathlineto{\pgfqpoint{1.626416in}{3.266367in}}%
\pgfpathlineto{\pgfqpoint{1.630957in}{3.266367in}}%
\pgfpathlineto{\pgfqpoint{1.630957in}{3.263418in}}%
\pgfpathmoveto{\pgfqpoint{1.621874in}{3.266367in}}%
\pgfpathlineto{\pgfqpoint{1.621874in}{3.266367in}}%
\pgfpathlineto{\pgfqpoint{1.621874in}{3.269317in}}%
\pgfpathlineto{\pgfqpoint{1.626416in}{3.269317in}}%
\pgfpathlineto{\pgfqpoint{1.626416in}{3.266367in}}%
\pgfpathmoveto{\pgfqpoint{1.630957in}{3.260469in}}%
\pgfpathlineto{\pgfqpoint{1.630957in}{3.260469in}}%
\pgfpathlineto{\pgfqpoint{1.630957in}{3.263418in}}%
\pgfpathlineto{\pgfqpoint{1.635498in}{3.263418in}}%
\pgfpathlineto{\pgfqpoint{1.635498in}{3.260469in}}%
\pgfpathmoveto{\pgfqpoint{1.658204in}{3.236874in}}%
\pgfpathlineto{\pgfqpoint{1.658204in}{3.236874in}}%
\pgfpathlineto{\pgfqpoint{1.658204in}{3.239823in}}%
\pgfpathlineto{\pgfqpoint{1.662745in}{3.239823in}}%
\pgfpathlineto{\pgfqpoint{1.662745in}{3.236874in}}%
\pgfpathmoveto{\pgfqpoint{1.907955in}{3.015683in}}%
\pgfpathlineto{\pgfqpoint{1.907955in}{3.015683in}}%
\pgfpathlineto{\pgfqpoint{1.907955in}{3.018632in}}%
\pgfpathlineto{\pgfqpoint{1.912496in}{3.018632in}}%
\pgfpathlineto{\pgfqpoint{1.912496in}{3.015683in}}%
\pgfpathmoveto{\pgfqpoint{1.903414in}{3.018632in}}%
\pgfpathlineto{\pgfqpoint{1.903414in}{3.018632in}}%
\pgfpathlineto{\pgfqpoint{1.903414in}{3.021581in}}%
\pgfpathlineto{\pgfqpoint{1.907955in}{3.021581in}}%
\pgfpathlineto{\pgfqpoint{1.907955in}{3.018632in}}%
\pgfpathmoveto{\pgfqpoint{1.903414in}{3.021581in}}%
\pgfpathlineto{\pgfqpoint{1.903414in}{3.021581in}}%
\pgfpathlineto{\pgfqpoint{1.903414in}{3.024530in}}%
\pgfpathlineto{\pgfqpoint{1.907955in}{3.024530in}}%
\pgfpathlineto{\pgfqpoint{1.907955in}{3.021581in}}%
\pgfpathmoveto{\pgfqpoint{1.907955in}{3.018632in}}%
\pgfpathlineto{\pgfqpoint{1.907955in}{3.018632in}}%
\pgfpathlineto{\pgfqpoint{1.907955in}{3.021581in}}%
\pgfpathlineto{\pgfqpoint{1.912496in}{3.021581in}}%
\pgfpathlineto{\pgfqpoint{1.912496in}{3.018632in}}%
\pgfpathmoveto{\pgfqpoint{1.889792in}{3.030428in}}%
\pgfpathlineto{\pgfqpoint{1.889792in}{3.030428in}}%
\pgfpathlineto{\pgfqpoint{1.889792in}{3.033377in}}%
\pgfpathlineto{\pgfqpoint{1.894333in}{3.033377in}}%
\pgfpathlineto{\pgfqpoint{1.894333in}{3.030428in}}%
\pgfpathmoveto{\pgfqpoint{1.889792in}{3.033377in}}%
\pgfpathlineto{\pgfqpoint{1.889792in}{3.033377in}}%
\pgfpathlineto{\pgfqpoint{1.889792in}{3.036326in}}%
\pgfpathlineto{\pgfqpoint{1.894333in}{3.036326in}}%
\pgfpathlineto{\pgfqpoint{1.894333in}{3.033377in}}%
\pgfpathmoveto{\pgfqpoint{1.880710in}{3.039275in}}%
\pgfpathlineto{\pgfqpoint{1.880710in}{3.039275in}}%
\pgfpathlineto{\pgfqpoint{1.880710in}{3.042224in}}%
\pgfpathlineto{\pgfqpoint{1.885251in}{3.042224in}}%
\pgfpathlineto{\pgfqpoint{1.885251in}{3.039275in}}%
\pgfpathmoveto{\pgfqpoint{1.876170in}{3.042224in}}%
\pgfpathlineto{\pgfqpoint{1.876170in}{3.042224in}}%
\pgfpathlineto{\pgfqpoint{1.876170in}{3.045173in}}%
\pgfpathlineto{\pgfqpoint{1.880710in}{3.045173in}}%
\pgfpathlineto{\pgfqpoint{1.880710in}{3.042224in}}%
\pgfpathmoveto{\pgfqpoint{1.876170in}{3.045173in}}%
\pgfpathlineto{\pgfqpoint{1.876170in}{3.045173in}}%
\pgfpathlineto{\pgfqpoint{1.876170in}{3.048122in}}%
\pgfpathlineto{\pgfqpoint{1.880710in}{3.048122in}}%
\pgfpathlineto{\pgfqpoint{1.880710in}{3.045173in}}%
\pgfpathmoveto{\pgfqpoint{1.880710in}{3.042224in}}%
\pgfpathlineto{\pgfqpoint{1.880710in}{3.042224in}}%
\pgfpathlineto{\pgfqpoint{1.880710in}{3.045173in}}%
\pgfpathlineto{\pgfqpoint{1.885251in}{3.045173in}}%
\pgfpathlineto{\pgfqpoint{1.885251in}{3.042224in}}%
\pgfpathmoveto{\pgfqpoint{1.885251in}{3.036326in}}%
\pgfpathlineto{\pgfqpoint{1.885251in}{3.036326in}}%
\pgfpathlineto{\pgfqpoint{1.885251in}{3.039275in}}%
\pgfpathlineto{\pgfqpoint{1.889792in}{3.039275in}}%
\pgfpathlineto{\pgfqpoint{1.889792in}{3.036326in}}%
\pgfpathmoveto{\pgfqpoint{1.885251in}{3.039275in}}%
\pgfpathlineto{\pgfqpoint{1.885251in}{3.039275in}}%
\pgfpathlineto{\pgfqpoint{1.885251in}{3.042224in}}%
\pgfpathlineto{\pgfqpoint{1.889792in}{3.042224in}}%
\pgfpathlineto{\pgfqpoint{1.889792in}{3.039275in}}%
\pgfpathmoveto{\pgfqpoint{1.889792in}{3.036326in}}%
\pgfpathlineto{\pgfqpoint{1.889792in}{3.036326in}}%
\pgfpathlineto{\pgfqpoint{1.889792in}{3.039275in}}%
\pgfpathlineto{\pgfqpoint{1.894333in}{3.039275in}}%
\pgfpathlineto{\pgfqpoint{1.894333in}{3.036326in}}%
\pgfpathmoveto{\pgfqpoint{1.894333in}{3.027479in}}%
\pgfpathlineto{\pgfqpoint{1.894333in}{3.027479in}}%
\pgfpathlineto{\pgfqpoint{1.894333in}{3.030428in}}%
\pgfpathlineto{\pgfqpoint{1.898873in}{3.030428in}}%
\pgfpathlineto{\pgfqpoint{1.898873in}{3.027479in}}%
\pgfpathmoveto{\pgfqpoint{1.898873in}{3.024530in}}%
\pgfpathlineto{\pgfqpoint{1.898873in}{3.024530in}}%
\pgfpathlineto{\pgfqpoint{1.898873in}{3.027479in}}%
\pgfpathlineto{\pgfqpoint{1.903414in}{3.027479in}}%
\pgfpathlineto{\pgfqpoint{1.903414in}{3.024530in}}%
\pgfpathmoveto{\pgfqpoint{1.898873in}{3.027479in}}%
\pgfpathlineto{\pgfqpoint{1.898873in}{3.027479in}}%
\pgfpathlineto{\pgfqpoint{1.898873in}{3.030428in}}%
\pgfpathlineto{\pgfqpoint{1.903414in}{3.030428in}}%
\pgfpathlineto{\pgfqpoint{1.903414in}{3.027479in}}%
\pgfpathmoveto{\pgfqpoint{1.894333in}{3.030428in}}%
\pgfpathlineto{\pgfqpoint{1.894333in}{3.030428in}}%
\pgfpathlineto{\pgfqpoint{1.894333in}{3.033377in}}%
\pgfpathlineto{\pgfqpoint{1.898873in}{3.033377in}}%
\pgfpathlineto{\pgfqpoint{1.898873in}{3.030428in}}%
\pgfpathmoveto{\pgfqpoint{1.903414in}{3.024530in}}%
\pgfpathlineto{\pgfqpoint{1.903414in}{3.024530in}}%
\pgfpathlineto{\pgfqpoint{1.903414in}{3.027479in}}%
\pgfpathlineto{\pgfqpoint{1.907955in}{3.027479in}}%
\pgfpathlineto{\pgfqpoint{1.907955in}{3.024530in}}%
\pgfpathmoveto{\pgfqpoint{1.835303in}{3.077615in}}%
\pgfpathlineto{\pgfqpoint{1.835303in}{3.077615in}}%
\pgfpathlineto{\pgfqpoint{1.835303in}{3.080564in}}%
\pgfpathlineto{\pgfqpoint{1.839844in}{3.080564in}}%
\pgfpathlineto{\pgfqpoint{1.839844in}{3.077615in}}%
\pgfpathmoveto{\pgfqpoint{1.835303in}{3.080564in}}%
\pgfpathlineto{\pgfqpoint{1.835303in}{3.080564in}}%
\pgfpathlineto{\pgfqpoint{1.835303in}{3.083513in}}%
\pgfpathlineto{\pgfqpoint{1.839844in}{3.083513in}}%
\pgfpathlineto{\pgfqpoint{1.839844in}{3.080564in}}%
\pgfpathmoveto{\pgfqpoint{1.826221in}{3.086463in}}%
\pgfpathlineto{\pgfqpoint{1.826221in}{3.086463in}}%
\pgfpathlineto{\pgfqpoint{1.826221in}{3.089412in}}%
\pgfpathlineto{\pgfqpoint{1.830762in}{3.089412in}}%
\pgfpathlineto{\pgfqpoint{1.830762in}{3.086463in}}%
\pgfpathmoveto{\pgfqpoint{1.821681in}{3.089412in}}%
\pgfpathlineto{\pgfqpoint{1.821681in}{3.089412in}}%
\pgfpathlineto{\pgfqpoint{1.821681in}{3.092361in}}%
\pgfpathlineto{\pgfqpoint{1.826221in}{3.092361in}}%
\pgfpathlineto{\pgfqpoint{1.826221in}{3.089412in}}%
\pgfpathmoveto{\pgfqpoint{1.821681in}{3.092361in}}%
\pgfpathlineto{\pgfqpoint{1.821681in}{3.092361in}}%
\pgfpathlineto{\pgfqpoint{1.821681in}{3.095311in}}%
\pgfpathlineto{\pgfqpoint{1.826221in}{3.095311in}}%
\pgfpathlineto{\pgfqpoint{1.826221in}{3.092361in}}%
\pgfpathmoveto{\pgfqpoint{1.826221in}{3.089412in}}%
\pgfpathlineto{\pgfqpoint{1.826221in}{3.089412in}}%
\pgfpathlineto{\pgfqpoint{1.826221in}{3.092361in}}%
\pgfpathlineto{\pgfqpoint{1.830762in}{3.092361in}}%
\pgfpathlineto{\pgfqpoint{1.830762in}{3.089412in}}%
\pgfpathmoveto{\pgfqpoint{1.830762in}{3.083513in}}%
\pgfpathlineto{\pgfqpoint{1.830762in}{3.083513in}}%
\pgfpathlineto{\pgfqpoint{1.830762in}{3.086463in}}%
\pgfpathlineto{\pgfqpoint{1.835303in}{3.086463in}}%
\pgfpathlineto{\pgfqpoint{1.835303in}{3.083513in}}%
\pgfpathmoveto{\pgfqpoint{1.830762in}{3.086463in}}%
\pgfpathlineto{\pgfqpoint{1.830762in}{3.086463in}}%
\pgfpathlineto{\pgfqpoint{1.830762in}{3.089412in}}%
\pgfpathlineto{\pgfqpoint{1.835303in}{3.089412in}}%
\pgfpathlineto{\pgfqpoint{1.835303in}{3.086463in}}%
\pgfpathmoveto{\pgfqpoint{1.835303in}{3.083513in}}%
\pgfpathlineto{\pgfqpoint{1.835303in}{3.083513in}}%
\pgfpathlineto{\pgfqpoint{1.835303in}{3.086463in}}%
\pgfpathlineto{\pgfqpoint{1.839844in}{3.086463in}}%
\pgfpathlineto{\pgfqpoint{1.839844in}{3.083513in}}%
\pgfpathmoveto{\pgfqpoint{1.789895in}{3.115956in}}%
\pgfpathlineto{\pgfqpoint{1.789895in}{3.115956in}}%
\pgfpathlineto{\pgfqpoint{1.789895in}{3.118905in}}%
\pgfpathlineto{\pgfqpoint{1.794436in}{3.118905in}}%
\pgfpathlineto{\pgfqpoint{1.794436in}{3.115956in}}%
\pgfpathmoveto{\pgfqpoint{1.798977in}{3.110057in}}%
\pgfpathlineto{\pgfqpoint{1.798977in}{3.110057in}}%
\pgfpathlineto{\pgfqpoint{1.798977in}{3.113006in}}%
\pgfpathlineto{\pgfqpoint{1.803518in}{3.113006in}}%
\pgfpathlineto{\pgfqpoint{1.803518in}{3.110057in}}%
\pgfpathmoveto{\pgfqpoint{1.794436in}{3.113006in}}%
\pgfpathlineto{\pgfqpoint{1.794436in}{3.113006in}}%
\pgfpathlineto{\pgfqpoint{1.794436in}{3.115956in}}%
\pgfpathlineto{\pgfqpoint{1.798977in}{3.115956in}}%
\pgfpathlineto{\pgfqpoint{1.798977in}{3.113006in}}%
\pgfpathmoveto{\pgfqpoint{1.794436in}{3.115956in}}%
\pgfpathlineto{\pgfqpoint{1.794436in}{3.115956in}}%
\pgfpathlineto{\pgfqpoint{1.794436in}{3.118905in}}%
\pgfpathlineto{\pgfqpoint{1.798977in}{3.118905in}}%
\pgfpathlineto{\pgfqpoint{1.798977in}{3.115956in}}%
\pgfpathmoveto{\pgfqpoint{1.798977in}{3.113006in}}%
\pgfpathlineto{\pgfqpoint{1.798977in}{3.113006in}}%
\pgfpathlineto{\pgfqpoint{1.798977in}{3.115956in}}%
\pgfpathlineto{\pgfqpoint{1.803518in}{3.115956in}}%
\pgfpathlineto{\pgfqpoint{1.803518in}{3.113006in}}%
\pgfpathmoveto{\pgfqpoint{1.776273in}{3.127753in}}%
\pgfpathlineto{\pgfqpoint{1.776273in}{3.127753in}}%
\pgfpathlineto{\pgfqpoint{1.776273in}{3.130702in}}%
\pgfpathlineto{\pgfqpoint{1.780814in}{3.130702in}}%
\pgfpathlineto{\pgfqpoint{1.780814in}{3.127753in}}%
\pgfpathmoveto{\pgfqpoint{1.780814in}{3.124803in}}%
\pgfpathlineto{\pgfqpoint{1.780814in}{3.124803in}}%
\pgfpathlineto{\pgfqpoint{1.780814in}{3.127753in}}%
\pgfpathlineto{\pgfqpoint{1.785354in}{3.127753in}}%
\pgfpathlineto{\pgfqpoint{1.785354in}{3.124803in}}%
\pgfpathmoveto{\pgfqpoint{1.780814in}{3.127753in}}%
\pgfpathlineto{\pgfqpoint{1.780814in}{3.127753in}}%
\pgfpathlineto{\pgfqpoint{1.780814in}{3.130702in}}%
\pgfpathlineto{\pgfqpoint{1.785354in}{3.130702in}}%
\pgfpathlineto{\pgfqpoint{1.785354in}{3.127753in}}%
\pgfpathmoveto{\pgfqpoint{1.771732in}{3.133651in}}%
\pgfpathlineto{\pgfqpoint{1.771732in}{3.133651in}}%
\pgfpathlineto{\pgfqpoint{1.771732in}{3.136601in}}%
\pgfpathlineto{\pgfqpoint{1.776273in}{3.136601in}}%
\pgfpathlineto{\pgfqpoint{1.776273in}{3.133651in}}%
\pgfpathmoveto{\pgfqpoint{1.767191in}{3.136601in}}%
\pgfpathlineto{\pgfqpoint{1.767191in}{3.136601in}}%
\pgfpathlineto{\pgfqpoint{1.767191in}{3.139550in}}%
\pgfpathlineto{\pgfqpoint{1.771732in}{3.139550in}}%
\pgfpathlineto{\pgfqpoint{1.771732in}{3.136601in}}%
\pgfpathmoveto{\pgfqpoint{1.767191in}{3.139550in}}%
\pgfpathlineto{\pgfqpoint{1.767191in}{3.139550in}}%
\pgfpathlineto{\pgfqpoint{1.767191in}{3.142499in}}%
\pgfpathlineto{\pgfqpoint{1.771732in}{3.142499in}}%
\pgfpathlineto{\pgfqpoint{1.771732in}{3.139550in}}%
\pgfpathmoveto{\pgfqpoint{1.771732in}{3.136601in}}%
\pgfpathlineto{\pgfqpoint{1.771732in}{3.136601in}}%
\pgfpathlineto{\pgfqpoint{1.771732in}{3.139550in}}%
\pgfpathlineto{\pgfqpoint{1.776273in}{3.139550in}}%
\pgfpathlineto{\pgfqpoint{1.776273in}{3.136601in}}%
\pgfpathmoveto{\pgfqpoint{1.776273in}{3.130702in}}%
\pgfpathlineto{\pgfqpoint{1.776273in}{3.130702in}}%
\pgfpathlineto{\pgfqpoint{1.776273in}{3.133651in}}%
\pgfpathlineto{\pgfqpoint{1.780814in}{3.133651in}}%
\pgfpathlineto{\pgfqpoint{1.780814in}{3.130702in}}%
\pgfpathmoveto{\pgfqpoint{1.776273in}{3.133651in}}%
\pgfpathlineto{\pgfqpoint{1.776273in}{3.133651in}}%
\pgfpathlineto{\pgfqpoint{1.776273in}{3.136601in}}%
\pgfpathlineto{\pgfqpoint{1.780814in}{3.136601in}}%
\pgfpathlineto{\pgfqpoint{1.780814in}{3.133651in}}%
\pgfpathmoveto{\pgfqpoint{1.785354in}{3.121854in}}%
\pgfpathlineto{\pgfqpoint{1.785354in}{3.121854in}}%
\pgfpathlineto{\pgfqpoint{1.785354in}{3.124803in}}%
\pgfpathlineto{\pgfqpoint{1.789895in}{3.124803in}}%
\pgfpathlineto{\pgfqpoint{1.789895in}{3.121854in}}%
\pgfpathmoveto{\pgfqpoint{1.789895in}{3.118905in}}%
\pgfpathlineto{\pgfqpoint{1.789895in}{3.118905in}}%
\pgfpathlineto{\pgfqpoint{1.789895in}{3.121854in}}%
\pgfpathlineto{\pgfqpoint{1.794436in}{3.121854in}}%
\pgfpathlineto{\pgfqpoint{1.794436in}{3.118905in}}%
\pgfpathmoveto{\pgfqpoint{1.789895in}{3.121854in}}%
\pgfpathlineto{\pgfqpoint{1.789895in}{3.121854in}}%
\pgfpathlineto{\pgfqpoint{1.789895in}{3.124803in}}%
\pgfpathlineto{\pgfqpoint{1.794436in}{3.124803in}}%
\pgfpathlineto{\pgfqpoint{1.794436in}{3.121854in}}%
\pgfpathmoveto{\pgfqpoint{1.785354in}{3.124803in}}%
\pgfpathlineto{\pgfqpoint{1.785354in}{3.124803in}}%
\pgfpathlineto{\pgfqpoint{1.785354in}{3.127753in}}%
\pgfpathlineto{\pgfqpoint{1.789895in}{3.127753in}}%
\pgfpathlineto{\pgfqpoint{1.789895in}{3.124803in}}%
\pgfpathmoveto{\pgfqpoint{1.808058in}{3.101209in}}%
\pgfpathlineto{\pgfqpoint{1.808058in}{3.101209in}}%
\pgfpathlineto{\pgfqpoint{1.808058in}{3.104158in}}%
\pgfpathlineto{\pgfqpoint{1.812599in}{3.104158in}}%
\pgfpathlineto{\pgfqpoint{1.812599in}{3.101209in}}%
\pgfpathmoveto{\pgfqpoint{1.808058in}{3.104158in}}%
\pgfpathlineto{\pgfqpoint{1.808058in}{3.104158in}}%
\pgfpathlineto{\pgfqpoint{1.808058in}{3.107108in}}%
\pgfpathlineto{\pgfqpoint{1.812599in}{3.107108in}}%
\pgfpathlineto{\pgfqpoint{1.812599in}{3.104158in}}%
\pgfpathmoveto{\pgfqpoint{1.812599in}{3.098260in}}%
\pgfpathlineto{\pgfqpoint{1.812599in}{3.098260in}}%
\pgfpathlineto{\pgfqpoint{1.812599in}{3.101209in}}%
\pgfpathlineto{\pgfqpoint{1.817140in}{3.101209in}}%
\pgfpathlineto{\pgfqpoint{1.817140in}{3.098260in}}%
\pgfpathmoveto{\pgfqpoint{1.817140in}{3.095311in}}%
\pgfpathlineto{\pgfqpoint{1.817140in}{3.095311in}}%
\pgfpathlineto{\pgfqpoint{1.817140in}{3.098260in}}%
\pgfpathlineto{\pgfqpoint{1.821681in}{3.098260in}}%
\pgfpathlineto{\pgfqpoint{1.821681in}{3.095311in}}%
\pgfpathmoveto{\pgfqpoint{1.817140in}{3.098260in}}%
\pgfpathlineto{\pgfqpoint{1.817140in}{3.098260in}}%
\pgfpathlineto{\pgfqpoint{1.817140in}{3.101209in}}%
\pgfpathlineto{\pgfqpoint{1.821681in}{3.101209in}}%
\pgfpathlineto{\pgfqpoint{1.821681in}{3.098260in}}%
\pgfpathmoveto{\pgfqpoint{1.812599in}{3.101209in}}%
\pgfpathlineto{\pgfqpoint{1.812599in}{3.101209in}}%
\pgfpathlineto{\pgfqpoint{1.812599in}{3.104158in}}%
\pgfpathlineto{\pgfqpoint{1.817140in}{3.104158in}}%
\pgfpathlineto{\pgfqpoint{1.817140in}{3.101209in}}%
\pgfpathmoveto{\pgfqpoint{1.803518in}{3.107108in}}%
\pgfpathlineto{\pgfqpoint{1.803518in}{3.107108in}}%
\pgfpathlineto{\pgfqpoint{1.803518in}{3.110057in}}%
\pgfpathlineto{\pgfqpoint{1.808058in}{3.110057in}}%
\pgfpathlineto{\pgfqpoint{1.808058in}{3.107108in}}%
\pgfpathmoveto{\pgfqpoint{1.803518in}{3.110057in}}%
\pgfpathlineto{\pgfqpoint{1.803518in}{3.110057in}}%
\pgfpathlineto{\pgfqpoint{1.803518in}{3.113006in}}%
\pgfpathlineto{\pgfqpoint{1.808058in}{3.113006in}}%
\pgfpathlineto{\pgfqpoint{1.808058in}{3.110057in}}%
\pgfpathmoveto{\pgfqpoint{1.808058in}{3.107108in}}%
\pgfpathlineto{\pgfqpoint{1.808058in}{3.107108in}}%
\pgfpathlineto{\pgfqpoint{1.808058in}{3.110057in}}%
\pgfpathlineto{\pgfqpoint{1.812599in}{3.110057in}}%
\pgfpathlineto{\pgfqpoint{1.812599in}{3.107108in}}%
\pgfpathmoveto{\pgfqpoint{1.821681in}{3.095311in}}%
\pgfpathlineto{\pgfqpoint{1.821681in}{3.095311in}}%
\pgfpathlineto{\pgfqpoint{1.821681in}{3.098260in}}%
\pgfpathlineto{\pgfqpoint{1.826221in}{3.098260in}}%
\pgfpathlineto{\pgfqpoint{1.826221in}{3.095311in}}%
\pgfpathmoveto{\pgfqpoint{1.853466in}{3.062869in}}%
\pgfpathlineto{\pgfqpoint{1.853466in}{3.062869in}}%
\pgfpathlineto{\pgfqpoint{1.853466in}{3.065818in}}%
\pgfpathlineto{\pgfqpoint{1.858007in}{3.065818in}}%
\pgfpathlineto{\pgfqpoint{1.858007in}{3.062869in}}%
\pgfpathmoveto{\pgfqpoint{1.848925in}{3.065818in}}%
\pgfpathlineto{\pgfqpoint{1.848925in}{3.065818in}}%
\pgfpathlineto{\pgfqpoint{1.848925in}{3.068767in}}%
\pgfpathlineto{\pgfqpoint{1.853466in}{3.068767in}}%
\pgfpathlineto{\pgfqpoint{1.853466in}{3.065818in}}%
\pgfpathmoveto{\pgfqpoint{1.848925in}{3.068767in}}%
\pgfpathlineto{\pgfqpoint{1.848925in}{3.068767in}}%
\pgfpathlineto{\pgfqpoint{1.848925in}{3.071716in}}%
\pgfpathlineto{\pgfqpoint{1.853466in}{3.071716in}}%
\pgfpathlineto{\pgfqpoint{1.853466in}{3.068767in}}%
\pgfpathmoveto{\pgfqpoint{1.853466in}{3.065818in}}%
\pgfpathlineto{\pgfqpoint{1.853466in}{3.065818in}}%
\pgfpathlineto{\pgfqpoint{1.853466in}{3.068767in}}%
\pgfpathlineto{\pgfqpoint{1.858007in}{3.068767in}}%
\pgfpathlineto{\pgfqpoint{1.858007in}{3.065818in}}%
\pgfpathmoveto{\pgfqpoint{1.862547in}{3.054021in}}%
\pgfpathlineto{\pgfqpoint{1.862547in}{3.054021in}}%
\pgfpathlineto{\pgfqpoint{1.862547in}{3.056970in}}%
\pgfpathlineto{\pgfqpoint{1.867088in}{3.056970in}}%
\pgfpathlineto{\pgfqpoint{1.867088in}{3.054021in}}%
\pgfpathmoveto{\pgfqpoint{1.862547in}{3.056970in}}%
\pgfpathlineto{\pgfqpoint{1.862547in}{3.056970in}}%
\pgfpathlineto{\pgfqpoint{1.862547in}{3.059919in}}%
\pgfpathlineto{\pgfqpoint{1.867088in}{3.059919in}}%
\pgfpathlineto{\pgfqpoint{1.867088in}{3.056970in}}%
\pgfpathmoveto{\pgfqpoint{1.867088in}{3.051071in}}%
\pgfpathlineto{\pgfqpoint{1.867088in}{3.051071in}}%
\pgfpathlineto{\pgfqpoint{1.867088in}{3.054021in}}%
\pgfpathlineto{\pgfqpoint{1.871629in}{3.054021in}}%
\pgfpathlineto{\pgfqpoint{1.871629in}{3.051071in}}%
\pgfpathmoveto{\pgfqpoint{1.871629in}{3.048122in}}%
\pgfpathlineto{\pgfqpoint{1.871629in}{3.048122in}}%
\pgfpathlineto{\pgfqpoint{1.871629in}{3.051071in}}%
\pgfpathlineto{\pgfqpoint{1.876170in}{3.051071in}}%
\pgfpathlineto{\pgfqpoint{1.876170in}{3.048122in}}%
\pgfpathmoveto{\pgfqpoint{1.871629in}{3.051071in}}%
\pgfpathlineto{\pgfqpoint{1.871629in}{3.051071in}}%
\pgfpathlineto{\pgfqpoint{1.871629in}{3.054021in}}%
\pgfpathlineto{\pgfqpoint{1.876170in}{3.054021in}}%
\pgfpathlineto{\pgfqpoint{1.876170in}{3.051071in}}%
\pgfpathmoveto{\pgfqpoint{1.867088in}{3.054021in}}%
\pgfpathlineto{\pgfqpoint{1.867088in}{3.054021in}}%
\pgfpathlineto{\pgfqpoint{1.867088in}{3.056970in}}%
\pgfpathlineto{\pgfqpoint{1.871629in}{3.056970in}}%
\pgfpathlineto{\pgfqpoint{1.871629in}{3.054021in}}%
\pgfpathmoveto{\pgfqpoint{1.858007in}{3.059919in}}%
\pgfpathlineto{\pgfqpoint{1.858007in}{3.059919in}}%
\pgfpathlineto{\pgfqpoint{1.858007in}{3.062869in}}%
\pgfpathlineto{\pgfqpoint{1.862547in}{3.062869in}}%
\pgfpathlineto{\pgfqpoint{1.862547in}{3.059919in}}%
\pgfpathmoveto{\pgfqpoint{1.858007in}{3.062869in}}%
\pgfpathlineto{\pgfqpoint{1.858007in}{3.062869in}}%
\pgfpathlineto{\pgfqpoint{1.858007in}{3.065818in}}%
\pgfpathlineto{\pgfqpoint{1.862547in}{3.065818in}}%
\pgfpathlineto{\pgfqpoint{1.862547in}{3.062869in}}%
\pgfpathmoveto{\pgfqpoint{1.862547in}{3.059919in}}%
\pgfpathlineto{\pgfqpoint{1.862547in}{3.059919in}}%
\pgfpathlineto{\pgfqpoint{1.862547in}{3.062869in}}%
\pgfpathlineto{\pgfqpoint{1.867088in}{3.062869in}}%
\pgfpathlineto{\pgfqpoint{1.867088in}{3.059919in}}%
\pgfpathmoveto{\pgfqpoint{1.839844in}{3.074666in}}%
\pgfpathlineto{\pgfqpoint{1.839844in}{3.074666in}}%
\pgfpathlineto{\pgfqpoint{1.839844in}{3.077615in}}%
\pgfpathlineto{\pgfqpoint{1.844384in}{3.077615in}}%
\pgfpathlineto{\pgfqpoint{1.844384in}{3.074666in}}%
\pgfpathmoveto{\pgfqpoint{1.844384in}{3.071716in}}%
\pgfpathlineto{\pgfqpoint{1.844384in}{3.071716in}}%
\pgfpathlineto{\pgfqpoint{1.844384in}{3.074666in}}%
\pgfpathlineto{\pgfqpoint{1.848925in}{3.074666in}}%
\pgfpathlineto{\pgfqpoint{1.848925in}{3.071716in}}%
\pgfpathmoveto{\pgfqpoint{1.844384in}{3.074666in}}%
\pgfpathlineto{\pgfqpoint{1.844384in}{3.074666in}}%
\pgfpathlineto{\pgfqpoint{1.844384in}{3.077615in}}%
\pgfpathlineto{\pgfqpoint{1.848925in}{3.077615in}}%
\pgfpathlineto{\pgfqpoint{1.848925in}{3.074666in}}%
\pgfpathmoveto{\pgfqpoint{1.839844in}{3.077615in}}%
\pgfpathlineto{\pgfqpoint{1.839844in}{3.077615in}}%
\pgfpathlineto{\pgfqpoint{1.839844in}{3.080564in}}%
\pgfpathlineto{\pgfqpoint{1.844384in}{3.080564in}}%
\pgfpathlineto{\pgfqpoint{1.844384in}{3.077615in}}%
\pgfpathmoveto{\pgfqpoint{1.848925in}{3.071716in}}%
\pgfpathlineto{\pgfqpoint{1.848925in}{3.071716in}}%
\pgfpathlineto{\pgfqpoint{1.848925in}{3.074666in}}%
\pgfpathlineto{\pgfqpoint{1.853466in}{3.074666in}}%
\pgfpathlineto{\pgfqpoint{1.853466in}{3.071716in}}%
\pgfpathmoveto{\pgfqpoint{1.876170in}{3.048122in}}%
\pgfpathlineto{\pgfqpoint{1.876170in}{3.048122in}}%
\pgfpathlineto{\pgfqpoint{1.876170in}{3.051071in}}%
\pgfpathlineto{\pgfqpoint{1.880710in}{3.051071in}}%
\pgfpathlineto{\pgfqpoint{1.880710in}{3.048122in}}%
\pgfpathmoveto{\pgfqpoint{1.980615in}{2.950803in}}%
\pgfpathlineto{\pgfqpoint{1.980615in}{2.950803in}}%
\pgfpathlineto{\pgfqpoint{1.980615in}{2.953753in}}%
\pgfpathlineto{\pgfqpoint{1.985156in}{2.953753in}}%
\pgfpathlineto{\pgfqpoint{1.985156in}{2.950803in}}%
\pgfpathmoveto{\pgfqpoint{2.035110in}{2.903615in}}%
\pgfpathlineto{\pgfqpoint{2.035110in}{2.903615in}}%
\pgfpathlineto{\pgfqpoint{2.035110in}{2.906565in}}%
\pgfpathlineto{\pgfqpoint{2.039652in}{2.906565in}}%
\pgfpathlineto{\pgfqpoint{2.039652in}{2.903615in}}%
\pgfpathmoveto{\pgfqpoint{2.048734in}{2.891818in}}%
\pgfpathlineto{\pgfqpoint{2.048734in}{2.891818in}}%
\pgfpathlineto{\pgfqpoint{2.048734in}{2.894768in}}%
\pgfpathlineto{\pgfqpoint{2.053276in}{2.894768in}}%
\pgfpathlineto{\pgfqpoint{2.053276in}{2.891818in}}%
\pgfpathmoveto{\pgfqpoint{2.053276in}{2.888869in}}%
\pgfpathlineto{\pgfqpoint{2.053276in}{2.888869in}}%
\pgfpathlineto{\pgfqpoint{2.053276in}{2.891818in}}%
\pgfpathlineto{\pgfqpoint{2.057817in}{2.891818in}}%
\pgfpathlineto{\pgfqpoint{2.057817in}{2.888869in}}%
\pgfpathmoveto{\pgfqpoint{2.053276in}{2.891818in}}%
\pgfpathlineto{\pgfqpoint{2.053276in}{2.891818in}}%
\pgfpathlineto{\pgfqpoint{2.053276in}{2.894768in}}%
\pgfpathlineto{\pgfqpoint{2.057817in}{2.894768in}}%
\pgfpathlineto{\pgfqpoint{2.057817in}{2.891818in}}%
\pgfpathmoveto{\pgfqpoint{2.044193in}{2.897717in}}%
\pgfpathlineto{\pgfqpoint{2.044193in}{2.897717in}}%
\pgfpathlineto{\pgfqpoint{2.044193in}{2.900666in}}%
\pgfpathlineto{\pgfqpoint{2.048734in}{2.900666in}}%
\pgfpathlineto{\pgfqpoint{2.048734in}{2.897717in}}%
\pgfpathmoveto{\pgfqpoint{2.039652in}{2.900666in}}%
\pgfpathlineto{\pgfqpoint{2.039652in}{2.900666in}}%
\pgfpathlineto{\pgfqpoint{2.039652in}{2.903615in}}%
\pgfpathlineto{\pgfqpoint{2.044193in}{2.903615in}}%
\pgfpathlineto{\pgfqpoint{2.044193in}{2.900666in}}%
\pgfpathmoveto{\pgfqpoint{2.039652in}{2.903615in}}%
\pgfpathlineto{\pgfqpoint{2.039652in}{2.903615in}}%
\pgfpathlineto{\pgfqpoint{2.039652in}{2.906565in}}%
\pgfpathlineto{\pgfqpoint{2.044193in}{2.906565in}}%
\pgfpathlineto{\pgfqpoint{2.044193in}{2.903615in}}%
\pgfpathmoveto{\pgfqpoint{2.044193in}{2.900666in}}%
\pgfpathlineto{\pgfqpoint{2.044193in}{2.900666in}}%
\pgfpathlineto{\pgfqpoint{2.044193in}{2.903615in}}%
\pgfpathlineto{\pgfqpoint{2.048734in}{2.903615in}}%
\pgfpathlineto{\pgfqpoint{2.048734in}{2.900666in}}%
\pgfpathmoveto{\pgfqpoint{2.048734in}{2.894768in}}%
\pgfpathlineto{\pgfqpoint{2.048734in}{2.894768in}}%
\pgfpathlineto{\pgfqpoint{2.048734in}{2.897717in}}%
\pgfpathlineto{\pgfqpoint{2.053276in}{2.897717in}}%
\pgfpathlineto{\pgfqpoint{2.053276in}{2.894768in}}%
\pgfpathmoveto{\pgfqpoint{2.048734in}{2.897717in}}%
\pgfpathlineto{\pgfqpoint{2.048734in}{2.897717in}}%
\pgfpathlineto{\pgfqpoint{2.048734in}{2.900666in}}%
\pgfpathlineto{\pgfqpoint{2.053276in}{2.900666in}}%
\pgfpathlineto{\pgfqpoint{2.053276in}{2.897717in}}%
\pgfpathmoveto{\pgfqpoint{2.007863in}{2.927209in}}%
\pgfpathlineto{\pgfqpoint{2.007863in}{2.927209in}}%
\pgfpathlineto{\pgfqpoint{2.007863in}{2.930159in}}%
\pgfpathlineto{\pgfqpoint{2.012404in}{2.930159in}}%
\pgfpathlineto{\pgfqpoint{2.012404in}{2.927209in}}%
\pgfpathmoveto{\pgfqpoint{2.016945in}{2.921311in}}%
\pgfpathlineto{\pgfqpoint{2.016945in}{2.921311in}}%
\pgfpathlineto{\pgfqpoint{2.016945in}{2.924260in}}%
\pgfpathlineto{\pgfqpoint{2.021487in}{2.924260in}}%
\pgfpathlineto{\pgfqpoint{2.021487in}{2.921311in}}%
\pgfpathmoveto{\pgfqpoint{2.012404in}{2.924260in}}%
\pgfpathlineto{\pgfqpoint{2.012404in}{2.924260in}}%
\pgfpathlineto{\pgfqpoint{2.012404in}{2.927209in}}%
\pgfpathlineto{\pgfqpoint{2.016945in}{2.927209in}}%
\pgfpathlineto{\pgfqpoint{2.016945in}{2.924260in}}%
\pgfpathmoveto{\pgfqpoint{2.012404in}{2.927209in}}%
\pgfpathlineto{\pgfqpoint{2.012404in}{2.927209in}}%
\pgfpathlineto{\pgfqpoint{2.012404in}{2.930159in}}%
\pgfpathlineto{\pgfqpoint{2.016945in}{2.930159in}}%
\pgfpathlineto{\pgfqpoint{2.016945in}{2.927209in}}%
\pgfpathmoveto{\pgfqpoint{2.016945in}{2.924260in}}%
\pgfpathlineto{\pgfqpoint{2.016945in}{2.924260in}}%
\pgfpathlineto{\pgfqpoint{2.016945in}{2.927209in}}%
\pgfpathlineto{\pgfqpoint{2.021487in}{2.927209in}}%
\pgfpathlineto{\pgfqpoint{2.021487in}{2.924260in}}%
\pgfpathmoveto{\pgfqpoint{1.994239in}{2.939006in}}%
\pgfpathlineto{\pgfqpoint{1.994239in}{2.939006in}}%
\pgfpathlineto{\pgfqpoint{1.994239in}{2.941956in}}%
\pgfpathlineto{\pgfqpoint{1.998780in}{2.941956in}}%
\pgfpathlineto{\pgfqpoint{1.998780in}{2.939006in}}%
\pgfpathmoveto{\pgfqpoint{1.998780in}{2.936057in}}%
\pgfpathlineto{\pgfqpoint{1.998780in}{2.936057in}}%
\pgfpathlineto{\pgfqpoint{1.998780in}{2.939006in}}%
\pgfpathlineto{\pgfqpoint{2.003321in}{2.939006in}}%
\pgfpathlineto{\pgfqpoint{2.003321in}{2.936057in}}%
\pgfpathmoveto{\pgfqpoint{1.998780in}{2.939006in}}%
\pgfpathlineto{\pgfqpoint{1.998780in}{2.939006in}}%
\pgfpathlineto{\pgfqpoint{1.998780in}{2.941956in}}%
\pgfpathlineto{\pgfqpoint{2.003321in}{2.941956in}}%
\pgfpathlineto{\pgfqpoint{2.003321in}{2.939006in}}%
\pgfpathmoveto{\pgfqpoint{1.989698in}{2.944905in}}%
\pgfpathlineto{\pgfqpoint{1.989698in}{2.944905in}}%
\pgfpathlineto{\pgfqpoint{1.989698in}{2.947854in}}%
\pgfpathlineto{\pgfqpoint{1.994239in}{2.947854in}}%
\pgfpathlineto{\pgfqpoint{1.994239in}{2.944905in}}%
\pgfpathmoveto{\pgfqpoint{1.985156in}{2.947854in}}%
\pgfpathlineto{\pgfqpoint{1.985156in}{2.947854in}}%
\pgfpathlineto{\pgfqpoint{1.985156in}{2.950803in}}%
\pgfpathlineto{\pgfqpoint{1.989698in}{2.950803in}}%
\pgfpathlineto{\pgfqpoint{1.989698in}{2.947854in}}%
\pgfpathmoveto{\pgfqpoint{1.985156in}{2.950803in}}%
\pgfpathlineto{\pgfqpoint{1.985156in}{2.950803in}}%
\pgfpathlineto{\pgfqpoint{1.985156in}{2.953753in}}%
\pgfpathlineto{\pgfqpoint{1.989698in}{2.953753in}}%
\pgfpathlineto{\pgfqpoint{1.989698in}{2.950803in}}%
\pgfpathmoveto{\pgfqpoint{1.989698in}{2.947854in}}%
\pgfpathlineto{\pgfqpoint{1.989698in}{2.947854in}}%
\pgfpathlineto{\pgfqpoint{1.989698in}{2.950803in}}%
\pgfpathlineto{\pgfqpoint{1.994239in}{2.950803in}}%
\pgfpathlineto{\pgfqpoint{1.994239in}{2.947854in}}%
\pgfpathmoveto{\pgfqpoint{1.994239in}{2.941956in}}%
\pgfpathlineto{\pgfqpoint{1.994239in}{2.941956in}}%
\pgfpathlineto{\pgfqpoint{1.994239in}{2.944905in}}%
\pgfpathlineto{\pgfqpoint{1.998780in}{2.944905in}}%
\pgfpathlineto{\pgfqpoint{1.998780in}{2.941956in}}%
\pgfpathmoveto{\pgfqpoint{1.994239in}{2.944905in}}%
\pgfpathlineto{\pgfqpoint{1.994239in}{2.944905in}}%
\pgfpathlineto{\pgfqpoint{1.994239in}{2.947854in}}%
\pgfpathlineto{\pgfqpoint{1.998780in}{2.947854in}}%
\pgfpathlineto{\pgfqpoint{1.998780in}{2.944905in}}%
\pgfpathmoveto{\pgfqpoint{2.003321in}{2.933108in}}%
\pgfpathlineto{\pgfqpoint{2.003321in}{2.933108in}}%
\pgfpathlineto{\pgfqpoint{2.003321in}{2.936057in}}%
\pgfpathlineto{\pgfqpoint{2.007863in}{2.936057in}}%
\pgfpathlineto{\pgfqpoint{2.007863in}{2.933108in}}%
\pgfpathmoveto{\pgfqpoint{2.007863in}{2.930159in}}%
\pgfpathlineto{\pgfqpoint{2.007863in}{2.930159in}}%
\pgfpathlineto{\pgfqpoint{2.007863in}{2.933108in}}%
\pgfpathlineto{\pgfqpoint{2.012404in}{2.933108in}}%
\pgfpathlineto{\pgfqpoint{2.012404in}{2.930159in}}%
\pgfpathmoveto{\pgfqpoint{2.007863in}{2.933108in}}%
\pgfpathlineto{\pgfqpoint{2.007863in}{2.933108in}}%
\pgfpathlineto{\pgfqpoint{2.007863in}{2.936057in}}%
\pgfpathlineto{\pgfqpoint{2.012404in}{2.936057in}}%
\pgfpathlineto{\pgfqpoint{2.012404in}{2.933108in}}%
\pgfpathmoveto{\pgfqpoint{2.003321in}{2.936057in}}%
\pgfpathlineto{\pgfqpoint{2.003321in}{2.936057in}}%
\pgfpathlineto{\pgfqpoint{2.003321in}{2.939006in}}%
\pgfpathlineto{\pgfqpoint{2.007863in}{2.939006in}}%
\pgfpathlineto{\pgfqpoint{2.007863in}{2.936057in}}%
\pgfpathmoveto{\pgfqpoint{2.021487in}{2.915412in}}%
\pgfpathlineto{\pgfqpoint{2.021487in}{2.915412in}}%
\pgfpathlineto{\pgfqpoint{2.021487in}{2.918362in}}%
\pgfpathlineto{\pgfqpoint{2.026028in}{2.918362in}}%
\pgfpathlineto{\pgfqpoint{2.026028in}{2.915412in}}%
\pgfpathmoveto{\pgfqpoint{2.026028in}{2.912463in}}%
\pgfpathlineto{\pgfqpoint{2.026028in}{2.912463in}}%
\pgfpathlineto{\pgfqpoint{2.026028in}{2.915412in}}%
\pgfpathlineto{\pgfqpoint{2.030569in}{2.915412in}}%
\pgfpathlineto{\pgfqpoint{2.030569in}{2.912463in}}%
\pgfpathmoveto{\pgfqpoint{2.026028in}{2.915412in}}%
\pgfpathlineto{\pgfqpoint{2.026028in}{2.915412in}}%
\pgfpathlineto{\pgfqpoint{2.026028in}{2.918362in}}%
\pgfpathlineto{\pgfqpoint{2.030569in}{2.918362in}}%
\pgfpathlineto{\pgfqpoint{2.030569in}{2.915412in}}%
\pgfpathmoveto{\pgfqpoint{2.030569in}{2.909514in}}%
\pgfpathlineto{\pgfqpoint{2.030569in}{2.909514in}}%
\pgfpathlineto{\pgfqpoint{2.030569in}{2.912463in}}%
\pgfpathlineto{\pgfqpoint{2.035110in}{2.912463in}}%
\pgfpathlineto{\pgfqpoint{2.035110in}{2.909514in}}%
\pgfpathmoveto{\pgfqpoint{2.035110in}{2.906565in}}%
\pgfpathlineto{\pgfqpoint{2.035110in}{2.906565in}}%
\pgfpathlineto{\pgfqpoint{2.035110in}{2.909514in}}%
\pgfpathlineto{\pgfqpoint{2.039652in}{2.909514in}}%
\pgfpathlineto{\pgfqpoint{2.039652in}{2.906565in}}%
\pgfpathmoveto{\pgfqpoint{2.035110in}{2.909514in}}%
\pgfpathlineto{\pgfqpoint{2.035110in}{2.909514in}}%
\pgfpathlineto{\pgfqpoint{2.035110in}{2.912463in}}%
\pgfpathlineto{\pgfqpoint{2.039652in}{2.912463in}}%
\pgfpathlineto{\pgfqpoint{2.039652in}{2.909514in}}%
\pgfpathmoveto{\pgfqpoint{2.030569in}{2.912463in}}%
\pgfpathlineto{\pgfqpoint{2.030569in}{2.912463in}}%
\pgfpathlineto{\pgfqpoint{2.030569in}{2.915412in}}%
\pgfpathlineto{\pgfqpoint{2.035110in}{2.915412in}}%
\pgfpathlineto{\pgfqpoint{2.035110in}{2.912463in}}%
\pgfpathmoveto{\pgfqpoint{2.021487in}{2.918362in}}%
\pgfpathlineto{\pgfqpoint{2.021487in}{2.918362in}}%
\pgfpathlineto{\pgfqpoint{2.021487in}{2.921311in}}%
\pgfpathlineto{\pgfqpoint{2.026028in}{2.921311in}}%
\pgfpathlineto{\pgfqpoint{2.026028in}{2.918362in}}%
\pgfpathmoveto{\pgfqpoint{2.021487in}{2.921311in}}%
\pgfpathlineto{\pgfqpoint{2.021487in}{2.921311in}}%
\pgfpathlineto{\pgfqpoint{2.021487in}{2.924260in}}%
\pgfpathlineto{\pgfqpoint{2.026028in}{2.924260in}}%
\pgfpathlineto{\pgfqpoint{2.026028in}{2.921311in}}%
\pgfpathmoveto{\pgfqpoint{1.944285in}{2.983243in}}%
\pgfpathlineto{\pgfqpoint{1.944285in}{2.983243in}}%
\pgfpathlineto{\pgfqpoint{1.944285in}{2.986192in}}%
\pgfpathlineto{\pgfqpoint{1.948826in}{2.986192in}}%
\pgfpathlineto{\pgfqpoint{1.948826in}{2.983243in}}%
\pgfpathmoveto{\pgfqpoint{1.944285in}{2.986192in}}%
\pgfpathlineto{\pgfqpoint{1.944285in}{2.986192in}}%
\pgfpathlineto{\pgfqpoint{1.944285in}{2.989141in}}%
\pgfpathlineto{\pgfqpoint{1.948826in}{2.989141in}}%
\pgfpathlineto{\pgfqpoint{1.948826in}{2.986192in}}%
\pgfpathmoveto{\pgfqpoint{1.935202in}{2.992090in}}%
\pgfpathlineto{\pgfqpoint{1.935202in}{2.992090in}}%
\pgfpathlineto{\pgfqpoint{1.935202in}{2.995039in}}%
\pgfpathlineto{\pgfqpoint{1.939743in}{2.995039in}}%
\pgfpathlineto{\pgfqpoint{1.939743in}{2.992090in}}%
\pgfpathmoveto{\pgfqpoint{1.930661in}{2.995039in}}%
\pgfpathlineto{\pgfqpoint{1.930661in}{2.995039in}}%
\pgfpathlineto{\pgfqpoint{1.930661in}{2.997988in}}%
\pgfpathlineto{\pgfqpoint{1.935202in}{2.997988in}}%
\pgfpathlineto{\pgfqpoint{1.935202in}{2.995039in}}%
\pgfpathmoveto{\pgfqpoint{1.930661in}{2.997988in}}%
\pgfpathlineto{\pgfqpoint{1.930661in}{2.997988in}}%
\pgfpathlineto{\pgfqpoint{1.930661in}{3.000937in}}%
\pgfpathlineto{\pgfqpoint{1.935202in}{3.000937in}}%
\pgfpathlineto{\pgfqpoint{1.935202in}{2.997988in}}%
\pgfpathmoveto{\pgfqpoint{1.935202in}{2.995039in}}%
\pgfpathlineto{\pgfqpoint{1.935202in}{2.995039in}}%
\pgfpathlineto{\pgfqpoint{1.935202in}{2.997988in}}%
\pgfpathlineto{\pgfqpoint{1.939743in}{2.997988in}}%
\pgfpathlineto{\pgfqpoint{1.939743in}{2.995039in}}%
\pgfpathmoveto{\pgfqpoint{1.939743in}{2.989141in}}%
\pgfpathlineto{\pgfqpoint{1.939743in}{2.989141in}}%
\pgfpathlineto{\pgfqpoint{1.939743in}{2.992090in}}%
\pgfpathlineto{\pgfqpoint{1.944285in}{2.992090in}}%
\pgfpathlineto{\pgfqpoint{1.944285in}{2.989141in}}%
\pgfpathmoveto{\pgfqpoint{1.939743in}{2.992090in}}%
\pgfpathlineto{\pgfqpoint{1.939743in}{2.992090in}}%
\pgfpathlineto{\pgfqpoint{1.939743in}{2.995039in}}%
\pgfpathlineto{\pgfqpoint{1.944285in}{2.995039in}}%
\pgfpathlineto{\pgfqpoint{1.944285in}{2.992090in}}%
\pgfpathmoveto{\pgfqpoint{1.944285in}{2.989141in}}%
\pgfpathlineto{\pgfqpoint{1.944285in}{2.989141in}}%
\pgfpathlineto{\pgfqpoint{1.944285in}{2.992090in}}%
\pgfpathlineto{\pgfqpoint{1.948826in}{2.992090in}}%
\pgfpathlineto{\pgfqpoint{1.948826in}{2.989141in}}%
\pgfpathmoveto{\pgfqpoint{1.962450in}{2.968498in}}%
\pgfpathlineto{\pgfqpoint{1.962450in}{2.968498in}}%
\pgfpathlineto{\pgfqpoint{1.962450in}{2.971447in}}%
\pgfpathlineto{\pgfqpoint{1.966991in}{2.971447in}}%
\pgfpathlineto{\pgfqpoint{1.966991in}{2.968498in}}%
\pgfpathmoveto{\pgfqpoint{1.957909in}{2.971447in}}%
\pgfpathlineto{\pgfqpoint{1.957909in}{2.971447in}}%
\pgfpathlineto{\pgfqpoint{1.957909in}{2.974396in}}%
\pgfpathlineto{\pgfqpoint{1.962450in}{2.974396in}}%
\pgfpathlineto{\pgfqpoint{1.962450in}{2.971447in}}%
\pgfpathmoveto{\pgfqpoint{1.957909in}{2.974396in}}%
\pgfpathlineto{\pgfqpoint{1.957909in}{2.974396in}}%
\pgfpathlineto{\pgfqpoint{1.957909in}{2.977345in}}%
\pgfpathlineto{\pgfqpoint{1.962450in}{2.977345in}}%
\pgfpathlineto{\pgfqpoint{1.962450in}{2.974396in}}%
\pgfpathmoveto{\pgfqpoint{1.962450in}{2.971447in}}%
\pgfpathlineto{\pgfqpoint{1.962450in}{2.971447in}}%
\pgfpathlineto{\pgfqpoint{1.962450in}{2.974396in}}%
\pgfpathlineto{\pgfqpoint{1.966991in}{2.974396in}}%
\pgfpathlineto{\pgfqpoint{1.966991in}{2.971447in}}%
\pgfpathmoveto{\pgfqpoint{1.966991in}{2.962600in}}%
\pgfpathlineto{\pgfqpoint{1.966991in}{2.962600in}}%
\pgfpathlineto{\pgfqpoint{1.966991in}{2.965549in}}%
\pgfpathlineto{\pgfqpoint{1.971532in}{2.965549in}}%
\pgfpathlineto{\pgfqpoint{1.971532in}{2.962600in}}%
\pgfpathmoveto{\pgfqpoint{1.971532in}{2.959651in}}%
\pgfpathlineto{\pgfqpoint{1.971532in}{2.959651in}}%
\pgfpathlineto{\pgfqpoint{1.971532in}{2.962600in}}%
\pgfpathlineto{\pgfqpoint{1.976074in}{2.962600in}}%
\pgfpathlineto{\pgfqpoint{1.976074in}{2.959651in}}%
\pgfpathmoveto{\pgfqpoint{1.971532in}{2.962600in}}%
\pgfpathlineto{\pgfqpoint{1.971532in}{2.962600in}}%
\pgfpathlineto{\pgfqpoint{1.971532in}{2.965549in}}%
\pgfpathlineto{\pgfqpoint{1.976074in}{2.965549in}}%
\pgfpathlineto{\pgfqpoint{1.976074in}{2.962600in}}%
\pgfpathmoveto{\pgfqpoint{1.976074in}{2.956702in}}%
\pgfpathlineto{\pgfqpoint{1.976074in}{2.956702in}}%
\pgfpathlineto{\pgfqpoint{1.976074in}{2.959651in}}%
\pgfpathlineto{\pgfqpoint{1.980615in}{2.959651in}}%
\pgfpathlineto{\pgfqpoint{1.980615in}{2.956702in}}%
\pgfpathmoveto{\pgfqpoint{1.980615in}{2.953753in}}%
\pgfpathlineto{\pgfqpoint{1.980615in}{2.953753in}}%
\pgfpathlineto{\pgfqpoint{1.980615in}{2.956702in}}%
\pgfpathlineto{\pgfqpoint{1.985156in}{2.956702in}}%
\pgfpathlineto{\pgfqpoint{1.985156in}{2.953753in}}%
\pgfpathmoveto{\pgfqpoint{1.980615in}{2.956702in}}%
\pgfpathlineto{\pgfqpoint{1.980615in}{2.956702in}}%
\pgfpathlineto{\pgfqpoint{1.980615in}{2.959651in}}%
\pgfpathlineto{\pgfqpoint{1.985156in}{2.959651in}}%
\pgfpathlineto{\pgfqpoint{1.985156in}{2.956702in}}%
\pgfpathmoveto{\pgfqpoint{1.976074in}{2.959651in}}%
\pgfpathlineto{\pgfqpoint{1.976074in}{2.959651in}}%
\pgfpathlineto{\pgfqpoint{1.976074in}{2.962600in}}%
\pgfpathlineto{\pgfqpoint{1.980615in}{2.962600in}}%
\pgfpathlineto{\pgfqpoint{1.980615in}{2.959651in}}%
\pgfpathmoveto{\pgfqpoint{1.966991in}{2.965549in}}%
\pgfpathlineto{\pgfqpoint{1.966991in}{2.965549in}}%
\pgfpathlineto{\pgfqpoint{1.966991in}{2.968498in}}%
\pgfpathlineto{\pgfqpoint{1.971532in}{2.968498in}}%
\pgfpathlineto{\pgfqpoint{1.971532in}{2.965549in}}%
\pgfpathmoveto{\pgfqpoint{1.966991in}{2.968498in}}%
\pgfpathlineto{\pgfqpoint{1.966991in}{2.968498in}}%
\pgfpathlineto{\pgfqpoint{1.966991in}{2.971447in}}%
\pgfpathlineto{\pgfqpoint{1.971532in}{2.971447in}}%
\pgfpathlineto{\pgfqpoint{1.971532in}{2.968498in}}%
\pgfpathmoveto{\pgfqpoint{1.948826in}{2.980294in}}%
\pgfpathlineto{\pgfqpoint{1.948826in}{2.980294in}}%
\pgfpathlineto{\pgfqpoint{1.948826in}{2.983243in}}%
\pgfpathlineto{\pgfqpoint{1.953367in}{2.983243in}}%
\pgfpathlineto{\pgfqpoint{1.953367in}{2.980294in}}%
\pgfpathmoveto{\pgfqpoint{1.953367in}{2.977345in}}%
\pgfpathlineto{\pgfqpoint{1.953367in}{2.977345in}}%
\pgfpathlineto{\pgfqpoint{1.953367in}{2.980294in}}%
\pgfpathlineto{\pgfqpoint{1.957909in}{2.980294in}}%
\pgfpathlineto{\pgfqpoint{1.957909in}{2.977345in}}%
\pgfpathmoveto{\pgfqpoint{1.953367in}{2.980294in}}%
\pgfpathlineto{\pgfqpoint{1.953367in}{2.980294in}}%
\pgfpathlineto{\pgfqpoint{1.953367in}{2.983243in}}%
\pgfpathlineto{\pgfqpoint{1.957909in}{2.983243in}}%
\pgfpathlineto{\pgfqpoint{1.957909in}{2.980294in}}%
\pgfpathmoveto{\pgfqpoint{1.948826in}{2.983243in}}%
\pgfpathlineto{\pgfqpoint{1.948826in}{2.983243in}}%
\pgfpathlineto{\pgfqpoint{1.948826in}{2.986192in}}%
\pgfpathlineto{\pgfqpoint{1.953367in}{2.986192in}}%
\pgfpathlineto{\pgfqpoint{1.953367in}{2.983243in}}%
\pgfpathmoveto{\pgfqpoint{1.957909in}{2.977345in}}%
\pgfpathlineto{\pgfqpoint{1.957909in}{2.977345in}}%
\pgfpathlineto{\pgfqpoint{1.957909in}{2.980294in}}%
\pgfpathlineto{\pgfqpoint{1.962450in}{2.980294in}}%
\pgfpathlineto{\pgfqpoint{1.962450in}{2.977345in}}%
\pgfpathmoveto{\pgfqpoint{1.917037in}{3.006835in}}%
\pgfpathlineto{\pgfqpoint{1.917037in}{3.006835in}}%
\pgfpathlineto{\pgfqpoint{1.917037in}{3.009784in}}%
\pgfpathlineto{\pgfqpoint{1.921578in}{3.009784in}}%
\pgfpathlineto{\pgfqpoint{1.921578in}{3.006835in}}%
\pgfpathmoveto{\pgfqpoint{1.917037in}{3.009784in}}%
\pgfpathlineto{\pgfqpoint{1.917037in}{3.009784in}}%
\pgfpathlineto{\pgfqpoint{1.917037in}{3.012734in}}%
\pgfpathlineto{\pgfqpoint{1.921578in}{3.012734in}}%
\pgfpathlineto{\pgfqpoint{1.921578in}{3.009784in}}%
\pgfpathmoveto{\pgfqpoint{1.921578in}{3.003886in}}%
\pgfpathlineto{\pgfqpoint{1.921578in}{3.003886in}}%
\pgfpathlineto{\pgfqpoint{1.921578in}{3.006835in}}%
\pgfpathlineto{\pgfqpoint{1.926120in}{3.006835in}}%
\pgfpathlineto{\pgfqpoint{1.926120in}{3.003886in}}%
\pgfpathmoveto{\pgfqpoint{1.926120in}{3.000937in}}%
\pgfpathlineto{\pgfqpoint{1.926120in}{3.000937in}}%
\pgfpathlineto{\pgfqpoint{1.926120in}{3.003886in}}%
\pgfpathlineto{\pgfqpoint{1.930661in}{3.003886in}}%
\pgfpathlineto{\pgfqpoint{1.930661in}{3.000937in}}%
\pgfpathmoveto{\pgfqpoint{1.926120in}{3.003886in}}%
\pgfpathlineto{\pgfqpoint{1.926120in}{3.003886in}}%
\pgfpathlineto{\pgfqpoint{1.926120in}{3.006835in}}%
\pgfpathlineto{\pgfqpoint{1.930661in}{3.006835in}}%
\pgfpathlineto{\pgfqpoint{1.930661in}{3.003886in}}%
\pgfpathmoveto{\pgfqpoint{1.921578in}{3.006835in}}%
\pgfpathlineto{\pgfqpoint{1.921578in}{3.006835in}}%
\pgfpathlineto{\pgfqpoint{1.921578in}{3.009784in}}%
\pgfpathlineto{\pgfqpoint{1.926120in}{3.009784in}}%
\pgfpathlineto{\pgfqpoint{1.926120in}{3.006835in}}%
\pgfpathmoveto{\pgfqpoint{1.912496in}{3.012734in}}%
\pgfpathlineto{\pgfqpoint{1.912496in}{3.012734in}}%
\pgfpathlineto{\pgfqpoint{1.912496in}{3.015683in}}%
\pgfpathlineto{\pgfqpoint{1.917037in}{3.015683in}}%
\pgfpathlineto{\pgfqpoint{1.917037in}{3.012734in}}%
\pgfpathmoveto{\pgfqpoint{1.912496in}{3.015683in}}%
\pgfpathlineto{\pgfqpoint{1.912496in}{3.015683in}}%
\pgfpathlineto{\pgfqpoint{1.912496in}{3.018632in}}%
\pgfpathlineto{\pgfqpoint{1.917037in}{3.018632in}}%
\pgfpathlineto{\pgfqpoint{1.917037in}{3.015683in}}%
\pgfpathmoveto{\pgfqpoint{1.917037in}{3.012734in}}%
\pgfpathlineto{\pgfqpoint{1.917037in}{3.012734in}}%
\pgfpathlineto{\pgfqpoint{1.917037in}{3.015683in}}%
\pgfpathlineto{\pgfqpoint{1.921578in}{3.015683in}}%
\pgfpathlineto{\pgfqpoint{1.921578in}{3.012734in}}%
\pgfpathmoveto{\pgfqpoint{1.930661in}{3.000937in}}%
\pgfpathlineto{\pgfqpoint{1.930661in}{3.000937in}}%
\pgfpathlineto{\pgfqpoint{1.930661in}{3.003886in}}%
\pgfpathlineto{\pgfqpoint{1.935202in}{3.003886in}}%
\pgfpathlineto{\pgfqpoint{1.935202in}{3.000937in}}%
\pgfpathmoveto{\pgfqpoint{2.198585in}{2.762052in}}%
\pgfpathlineto{\pgfqpoint{2.198585in}{2.762052in}}%
\pgfpathlineto{\pgfqpoint{2.198585in}{2.765001in}}%
\pgfpathlineto{\pgfqpoint{2.203126in}{2.765001in}}%
\pgfpathlineto{\pgfqpoint{2.203126in}{2.762052in}}%
\pgfpathmoveto{\pgfqpoint{2.089603in}{2.856427in}}%
\pgfpathlineto{\pgfqpoint{2.089603in}{2.856427in}}%
\pgfpathlineto{\pgfqpoint{2.089603in}{2.859376in}}%
\pgfpathlineto{\pgfqpoint{2.094144in}{2.859376in}}%
\pgfpathlineto{\pgfqpoint{2.094144in}{2.856427in}}%
\pgfpathmoveto{\pgfqpoint{2.116849in}{2.832833in}}%
\pgfpathlineto{\pgfqpoint{2.116849in}{2.832833in}}%
\pgfpathlineto{\pgfqpoint{2.116849in}{2.835783in}}%
\pgfpathlineto{\pgfqpoint{2.121390in}{2.835783in}}%
\pgfpathlineto{\pgfqpoint{2.121390in}{2.832833in}}%
\pgfpathmoveto{\pgfqpoint{2.125930in}{2.826935in}}%
\pgfpathlineto{\pgfqpoint{2.125930in}{2.826935in}}%
\pgfpathlineto{\pgfqpoint{2.125930in}{2.829884in}}%
\pgfpathlineto{\pgfqpoint{2.130471in}{2.829884in}}%
\pgfpathlineto{\pgfqpoint{2.130471in}{2.826935in}}%
\pgfpathmoveto{\pgfqpoint{2.121390in}{2.829884in}}%
\pgfpathlineto{\pgfqpoint{2.121390in}{2.829884in}}%
\pgfpathlineto{\pgfqpoint{2.121390in}{2.832833in}}%
\pgfpathlineto{\pgfqpoint{2.125930in}{2.832833in}}%
\pgfpathlineto{\pgfqpoint{2.125930in}{2.829884in}}%
\pgfpathmoveto{\pgfqpoint{2.121390in}{2.832833in}}%
\pgfpathlineto{\pgfqpoint{2.121390in}{2.832833in}}%
\pgfpathlineto{\pgfqpoint{2.121390in}{2.835783in}}%
\pgfpathlineto{\pgfqpoint{2.125930in}{2.835783in}}%
\pgfpathlineto{\pgfqpoint{2.125930in}{2.832833in}}%
\pgfpathmoveto{\pgfqpoint{2.125930in}{2.829884in}}%
\pgfpathlineto{\pgfqpoint{2.125930in}{2.829884in}}%
\pgfpathlineto{\pgfqpoint{2.125930in}{2.832833in}}%
\pgfpathlineto{\pgfqpoint{2.130471in}{2.832833in}}%
\pgfpathlineto{\pgfqpoint{2.130471in}{2.829884in}}%
\pgfpathmoveto{\pgfqpoint{2.103226in}{2.844630in}}%
\pgfpathlineto{\pgfqpoint{2.103226in}{2.844630in}}%
\pgfpathlineto{\pgfqpoint{2.103226in}{2.847580in}}%
\pgfpathlineto{\pgfqpoint{2.107767in}{2.847580in}}%
\pgfpathlineto{\pgfqpoint{2.107767in}{2.844630in}}%
\pgfpathmoveto{\pgfqpoint{2.107767in}{2.841681in}}%
\pgfpathlineto{\pgfqpoint{2.107767in}{2.841681in}}%
\pgfpathlineto{\pgfqpoint{2.107767in}{2.844630in}}%
\pgfpathlineto{\pgfqpoint{2.112308in}{2.844630in}}%
\pgfpathlineto{\pgfqpoint{2.112308in}{2.841681in}}%
\pgfpathmoveto{\pgfqpoint{2.107767in}{2.844630in}}%
\pgfpathlineto{\pgfqpoint{2.107767in}{2.844630in}}%
\pgfpathlineto{\pgfqpoint{2.107767in}{2.847580in}}%
\pgfpathlineto{\pgfqpoint{2.112308in}{2.847580in}}%
\pgfpathlineto{\pgfqpoint{2.112308in}{2.844630in}}%
\pgfpathmoveto{\pgfqpoint{2.098685in}{2.850529in}}%
\pgfpathlineto{\pgfqpoint{2.098685in}{2.850529in}}%
\pgfpathlineto{\pgfqpoint{2.098685in}{2.853478in}}%
\pgfpathlineto{\pgfqpoint{2.103226in}{2.853478in}}%
\pgfpathlineto{\pgfqpoint{2.103226in}{2.850529in}}%
\pgfpathmoveto{\pgfqpoint{2.094144in}{2.853478in}}%
\pgfpathlineto{\pgfqpoint{2.094144in}{2.853478in}}%
\pgfpathlineto{\pgfqpoint{2.094144in}{2.856427in}}%
\pgfpathlineto{\pgfqpoint{2.098685in}{2.856427in}}%
\pgfpathlineto{\pgfqpoint{2.098685in}{2.853478in}}%
\pgfpathmoveto{\pgfqpoint{2.094144in}{2.856427in}}%
\pgfpathlineto{\pgfqpoint{2.094144in}{2.856427in}}%
\pgfpathlineto{\pgfqpoint{2.094144in}{2.859376in}}%
\pgfpathlineto{\pgfqpoint{2.098685in}{2.859376in}}%
\pgfpathlineto{\pgfqpoint{2.098685in}{2.856427in}}%
\pgfpathmoveto{\pgfqpoint{2.098685in}{2.853478in}}%
\pgfpathlineto{\pgfqpoint{2.098685in}{2.853478in}}%
\pgfpathlineto{\pgfqpoint{2.098685in}{2.856427in}}%
\pgfpathlineto{\pgfqpoint{2.103226in}{2.856427in}}%
\pgfpathlineto{\pgfqpoint{2.103226in}{2.853478in}}%
\pgfpathmoveto{\pgfqpoint{2.103226in}{2.847580in}}%
\pgfpathlineto{\pgfqpoint{2.103226in}{2.847580in}}%
\pgfpathlineto{\pgfqpoint{2.103226in}{2.850529in}}%
\pgfpathlineto{\pgfqpoint{2.107767in}{2.850529in}}%
\pgfpathlineto{\pgfqpoint{2.107767in}{2.847580in}}%
\pgfpathmoveto{\pgfqpoint{2.103226in}{2.850529in}}%
\pgfpathlineto{\pgfqpoint{2.103226in}{2.850529in}}%
\pgfpathlineto{\pgfqpoint{2.103226in}{2.853478in}}%
\pgfpathlineto{\pgfqpoint{2.107767in}{2.853478in}}%
\pgfpathlineto{\pgfqpoint{2.107767in}{2.850529in}}%
\pgfpathmoveto{\pgfqpoint{2.112308in}{2.838732in}}%
\pgfpathlineto{\pgfqpoint{2.112308in}{2.838732in}}%
\pgfpathlineto{\pgfqpoint{2.112308in}{2.841681in}}%
\pgfpathlineto{\pgfqpoint{2.116849in}{2.841681in}}%
\pgfpathlineto{\pgfqpoint{2.116849in}{2.838732in}}%
\pgfpathmoveto{\pgfqpoint{2.116849in}{2.835783in}}%
\pgfpathlineto{\pgfqpoint{2.116849in}{2.835783in}}%
\pgfpathlineto{\pgfqpoint{2.116849in}{2.838732in}}%
\pgfpathlineto{\pgfqpoint{2.121390in}{2.838732in}}%
\pgfpathlineto{\pgfqpoint{2.121390in}{2.835783in}}%
\pgfpathmoveto{\pgfqpoint{2.116849in}{2.838732in}}%
\pgfpathlineto{\pgfqpoint{2.116849in}{2.838732in}}%
\pgfpathlineto{\pgfqpoint{2.116849in}{2.841681in}}%
\pgfpathlineto{\pgfqpoint{2.121390in}{2.841681in}}%
\pgfpathlineto{\pgfqpoint{2.121390in}{2.838732in}}%
\pgfpathmoveto{\pgfqpoint{2.112308in}{2.841681in}}%
\pgfpathlineto{\pgfqpoint{2.112308in}{2.841681in}}%
\pgfpathlineto{\pgfqpoint{2.112308in}{2.844630in}}%
\pgfpathlineto{\pgfqpoint{2.116849in}{2.844630in}}%
\pgfpathlineto{\pgfqpoint{2.116849in}{2.841681in}}%
\pgfpathmoveto{\pgfqpoint{2.144094in}{2.809239in}}%
\pgfpathlineto{\pgfqpoint{2.144094in}{2.809239in}}%
\pgfpathlineto{\pgfqpoint{2.144094in}{2.812189in}}%
\pgfpathlineto{\pgfqpoint{2.148635in}{2.812189in}}%
\pgfpathlineto{\pgfqpoint{2.148635in}{2.809239in}}%
\pgfpathmoveto{\pgfqpoint{2.157717in}{2.797442in}}%
\pgfpathlineto{\pgfqpoint{2.157717in}{2.797442in}}%
\pgfpathlineto{\pgfqpoint{2.157717in}{2.800392in}}%
\pgfpathlineto{\pgfqpoint{2.162258in}{2.800392in}}%
\pgfpathlineto{\pgfqpoint{2.162258in}{2.797442in}}%
\pgfpathmoveto{\pgfqpoint{2.162258in}{2.794493in}}%
\pgfpathlineto{\pgfqpoint{2.162258in}{2.794493in}}%
\pgfpathlineto{\pgfqpoint{2.162258in}{2.797442in}}%
\pgfpathlineto{\pgfqpoint{2.166799in}{2.797442in}}%
\pgfpathlineto{\pgfqpoint{2.166799in}{2.794493in}}%
\pgfpathmoveto{\pgfqpoint{2.162258in}{2.797442in}}%
\pgfpathlineto{\pgfqpoint{2.162258in}{2.797442in}}%
\pgfpathlineto{\pgfqpoint{2.162258in}{2.800392in}}%
\pgfpathlineto{\pgfqpoint{2.166799in}{2.800392in}}%
\pgfpathlineto{\pgfqpoint{2.166799in}{2.797442in}}%
\pgfpathmoveto{\pgfqpoint{2.153176in}{2.803341in}}%
\pgfpathlineto{\pgfqpoint{2.153176in}{2.803341in}}%
\pgfpathlineto{\pgfqpoint{2.153176in}{2.806290in}}%
\pgfpathlineto{\pgfqpoint{2.157717in}{2.806290in}}%
\pgfpathlineto{\pgfqpoint{2.157717in}{2.803341in}}%
\pgfpathmoveto{\pgfqpoint{2.148635in}{2.806290in}}%
\pgfpathlineto{\pgfqpoint{2.148635in}{2.806290in}}%
\pgfpathlineto{\pgfqpoint{2.148635in}{2.809239in}}%
\pgfpathlineto{\pgfqpoint{2.153176in}{2.809239in}}%
\pgfpathlineto{\pgfqpoint{2.153176in}{2.806290in}}%
\pgfpathmoveto{\pgfqpoint{2.148635in}{2.809239in}}%
\pgfpathlineto{\pgfqpoint{2.148635in}{2.809239in}}%
\pgfpathlineto{\pgfqpoint{2.148635in}{2.812189in}}%
\pgfpathlineto{\pgfqpoint{2.153176in}{2.812189in}}%
\pgfpathlineto{\pgfqpoint{2.153176in}{2.809239in}}%
\pgfpathmoveto{\pgfqpoint{2.153176in}{2.806290in}}%
\pgfpathlineto{\pgfqpoint{2.153176in}{2.806290in}}%
\pgfpathlineto{\pgfqpoint{2.153176in}{2.809239in}}%
\pgfpathlineto{\pgfqpoint{2.157717in}{2.809239in}}%
\pgfpathlineto{\pgfqpoint{2.157717in}{2.806290in}}%
\pgfpathmoveto{\pgfqpoint{2.157717in}{2.800392in}}%
\pgfpathlineto{\pgfqpoint{2.157717in}{2.800392in}}%
\pgfpathlineto{\pgfqpoint{2.157717in}{2.803341in}}%
\pgfpathlineto{\pgfqpoint{2.162258in}{2.803341in}}%
\pgfpathlineto{\pgfqpoint{2.162258in}{2.800392in}}%
\pgfpathmoveto{\pgfqpoint{2.157717in}{2.803341in}}%
\pgfpathlineto{\pgfqpoint{2.157717in}{2.803341in}}%
\pgfpathlineto{\pgfqpoint{2.157717in}{2.806290in}}%
\pgfpathlineto{\pgfqpoint{2.162258in}{2.806290in}}%
\pgfpathlineto{\pgfqpoint{2.162258in}{2.803341in}}%
\pgfpathmoveto{\pgfqpoint{2.171339in}{2.785646in}}%
\pgfpathlineto{\pgfqpoint{2.171339in}{2.785646in}}%
\pgfpathlineto{\pgfqpoint{2.171339in}{2.788595in}}%
\pgfpathlineto{\pgfqpoint{2.175880in}{2.788595in}}%
\pgfpathlineto{\pgfqpoint{2.175880in}{2.785646in}}%
\pgfpathmoveto{\pgfqpoint{2.180421in}{2.779747in}}%
\pgfpathlineto{\pgfqpoint{2.180421in}{2.779747in}}%
\pgfpathlineto{\pgfqpoint{2.180421in}{2.782696in}}%
\pgfpathlineto{\pgfqpoint{2.184962in}{2.782696in}}%
\pgfpathlineto{\pgfqpoint{2.184962in}{2.779747in}}%
\pgfpathmoveto{\pgfqpoint{2.175880in}{2.782696in}}%
\pgfpathlineto{\pgfqpoint{2.175880in}{2.782696in}}%
\pgfpathlineto{\pgfqpoint{2.175880in}{2.785646in}}%
\pgfpathlineto{\pgfqpoint{2.180421in}{2.785646in}}%
\pgfpathlineto{\pgfqpoint{2.180421in}{2.782696in}}%
\pgfpathmoveto{\pgfqpoint{2.175880in}{2.785646in}}%
\pgfpathlineto{\pgfqpoint{2.175880in}{2.785646in}}%
\pgfpathlineto{\pgfqpoint{2.175880in}{2.788595in}}%
\pgfpathlineto{\pgfqpoint{2.180421in}{2.788595in}}%
\pgfpathlineto{\pgfqpoint{2.180421in}{2.785646in}}%
\pgfpathmoveto{\pgfqpoint{2.180421in}{2.782696in}}%
\pgfpathlineto{\pgfqpoint{2.180421in}{2.782696in}}%
\pgfpathlineto{\pgfqpoint{2.180421in}{2.785646in}}%
\pgfpathlineto{\pgfqpoint{2.184962in}{2.785646in}}%
\pgfpathlineto{\pgfqpoint{2.184962in}{2.782696in}}%
\pgfpathmoveto{\pgfqpoint{2.184962in}{2.773849in}}%
\pgfpathlineto{\pgfqpoint{2.184962in}{2.773849in}}%
\pgfpathlineto{\pgfqpoint{2.184962in}{2.776798in}}%
\pgfpathlineto{\pgfqpoint{2.189503in}{2.776798in}}%
\pgfpathlineto{\pgfqpoint{2.189503in}{2.773849in}}%
\pgfpathmoveto{\pgfqpoint{2.189503in}{2.770899in}}%
\pgfpathlineto{\pgfqpoint{2.189503in}{2.770899in}}%
\pgfpathlineto{\pgfqpoint{2.189503in}{2.773849in}}%
\pgfpathlineto{\pgfqpoint{2.194044in}{2.773849in}}%
\pgfpathlineto{\pgfqpoint{2.194044in}{2.770899in}}%
\pgfpathmoveto{\pgfqpoint{2.189503in}{2.773849in}}%
\pgfpathlineto{\pgfqpoint{2.189503in}{2.773849in}}%
\pgfpathlineto{\pgfqpoint{2.189503in}{2.776798in}}%
\pgfpathlineto{\pgfqpoint{2.194044in}{2.776798in}}%
\pgfpathlineto{\pgfqpoint{2.194044in}{2.773849in}}%
\pgfpathmoveto{\pgfqpoint{2.194044in}{2.767950in}}%
\pgfpathlineto{\pgfqpoint{2.194044in}{2.767950in}}%
\pgfpathlineto{\pgfqpoint{2.194044in}{2.770899in}}%
\pgfpathlineto{\pgfqpoint{2.198585in}{2.770899in}}%
\pgfpathlineto{\pgfqpoint{2.198585in}{2.767950in}}%
\pgfpathmoveto{\pgfqpoint{2.198585in}{2.765001in}}%
\pgfpathlineto{\pgfqpoint{2.198585in}{2.765001in}}%
\pgfpathlineto{\pgfqpoint{2.198585in}{2.767950in}}%
\pgfpathlineto{\pgfqpoint{2.203126in}{2.767950in}}%
\pgfpathlineto{\pgfqpoint{2.203126in}{2.765001in}}%
\pgfpathmoveto{\pgfqpoint{2.198585in}{2.767950in}}%
\pgfpathlineto{\pgfqpoint{2.198585in}{2.767950in}}%
\pgfpathlineto{\pgfqpoint{2.198585in}{2.770899in}}%
\pgfpathlineto{\pgfqpoint{2.203126in}{2.770899in}}%
\pgfpathlineto{\pgfqpoint{2.203126in}{2.767950in}}%
\pgfpathmoveto{\pgfqpoint{2.194044in}{2.770899in}}%
\pgfpathlineto{\pgfqpoint{2.194044in}{2.770899in}}%
\pgfpathlineto{\pgfqpoint{2.194044in}{2.773849in}}%
\pgfpathlineto{\pgfqpoint{2.198585in}{2.773849in}}%
\pgfpathlineto{\pgfqpoint{2.198585in}{2.770899in}}%
\pgfpathmoveto{\pgfqpoint{2.184962in}{2.776798in}}%
\pgfpathlineto{\pgfqpoint{2.184962in}{2.776798in}}%
\pgfpathlineto{\pgfqpoint{2.184962in}{2.779747in}}%
\pgfpathlineto{\pgfqpoint{2.189503in}{2.779747in}}%
\pgfpathlineto{\pgfqpoint{2.189503in}{2.776798in}}%
\pgfpathmoveto{\pgfqpoint{2.184962in}{2.779747in}}%
\pgfpathlineto{\pgfqpoint{2.184962in}{2.779747in}}%
\pgfpathlineto{\pgfqpoint{2.184962in}{2.782696in}}%
\pgfpathlineto{\pgfqpoint{2.189503in}{2.782696in}}%
\pgfpathlineto{\pgfqpoint{2.189503in}{2.779747in}}%
\pgfpathmoveto{\pgfqpoint{2.166799in}{2.791544in}}%
\pgfpathlineto{\pgfqpoint{2.166799in}{2.791544in}}%
\pgfpathlineto{\pgfqpoint{2.166799in}{2.794493in}}%
\pgfpathlineto{\pgfqpoint{2.171339in}{2.794493in}}%
\pgfpathlineto{\pgfqpoint{2.171339in}{2.791544in}}%
\pgfpathmoveto{\pgfqpoint{2.171339in}{2.788595in}}%
\pgfpathlineto{\pgfqpoint{2.171339in}{2.788595in}}%
\pgfpathlineto{\pgfqpoint{2.171339in}{2.791544in}}%
\pgfpathlineto{\pgfqpoint{2.175880in}{2.791544in}}%
\pgfpathlineto{\pgfqpoint{2.175880in}{2.788595in}}%
\pgfpathmoveto{\pgfqpoint{2.171339in}{2.791544in}}%
\pgfpathlineto{\pgfqpoint{2.171339in}{2.791544in}}%
\pgfpathlineto{\pgfqpoint{2.171339in}{2.794493in}}%
\pgfpathlineto{\pgfqpoint{2.175880in}{2.794493in}}%
\pgfpathlineto{\pgfqpoint{2.175880in}{2.791544in}}%
\pgfpathmoveto{\pgfqpoint{2.166799in}{2.794493in}}%
\pgfpathlineto{\pgfqpoint{2.166799in}{2.794493in}}%
\pgfpathlineto{\pgfqpoint{2.166799in}{2.797442in}}%
\pgfpathlineto{\pgfqpoint{2.171339in}{2.797442in}}%
\pgfpathlineto{\pgfqpoint{2.171339in}{2.794493in}}%
\pgfpathmoveto{\pgfqpoint{2.130471in}{2.821036in}}%
\pgfpathlineto{\pgfqpoint{2.130471in}{2.821036in}}%
\pgfpathlineto{\pgfqpoint{2.130471in}{2.823986in}}%
\pgfpathlineto{\pgfqpoint{2.135012in}{2.823986in}}%
\pgfpathlineto{\pgfqpoint{2.135012in}{2.821036in}}%
\pgfpathmoveto{\pgfqpoint{2.135012in}{2.818087in}}%
\pgfpathlineto{\pgfqpoint{2.135012in}{2.818087in}}%
\pgfpathlineto{\pgfqpoint{2.135012in}{2.821036in}}%
\pgfpathlineto{\pgfqpoint{2.139553in}{2.821036in}}%
\pgfpathlineto{\pgfqpoint{2.139553in}{2.818087in}}%
\pgfpathmoveto{\pgfqpoint{2.135012in}{2.821036in}}%
\pgfpathlineto{\pgfqpoint{2.135012in}{2.821036in}}%
\pgfpathlineto{\pgfqpoint{2.135012in}{2.823986in}}%
\pgfpathlineto{\pgfqpoint{2.139553in}{2.823986in}}%
\pgfpathlineto{\pgfqpoint{2.139553in}{2.821036in}}%
\pgfpathmoveto{\pgfqpoint{2.139553in}{2.815138in}}%
\pgfpathlineto{\pgfqpoint{2.139553in}{2.815138in}}%
\pgfpathlineto{\pgfqpoint{2.139553in}{2.818087in}}%
\pgfpathlineto{\pgfqpoint{2.144094in}{2.818087in}}%
\pgfpathlineto{\pgfqpoint{2.144094in}{2.815138in}}%
\pgfpathmoveto{\pgfqpoint{2.144094in}{2.812189in}}%
\pgfpathlineto{\pgfqpoint{2.144094in}{2.812189in}}%
\pgfpathlineto{\pgfqpoint{2.144094in}{2.815138in}}%
\pgfpathlineto{\pgfqpoint{2.148635in}{2.815138in}}%
\pgfpathlineto{\pgfqpoint{2.148635in}{2.812189in}}%
\pgfpathmoveto{\pgfqpoint{2.144094in}{2.815138in}}%
\pgfpathlineto{\pgfqpoint{2.144094in}{2.815138in}}%
\pgfpathlineto{\pgfqpoint{2.144094in}{2.818087in}}%
\pgfpathlineto{\pgfqpoint{2.148635in}{2.818087in}}%
\pgfpathlineto{\pgfqpoint{2.148635in}{2.815138in}}%
\pgfpathmoveto{\pgfqpoint{2.139553in}{2.818087in}}%
\pgfpathlineto{\pgfqpoint{2.139553in}{2.818087in}}%
\pgfpathlineto{\pgfqpoint{2.139553in}{2.821036in}}%
\pgfpathlineto{\pgfqpoint{2.144094in}{2.821036in}}%
\pgfpathlineto{\pgfqpoint{2.144094in}{2.818087in}}%
\pgfpathmoveto{\pgfqpoint{2.130471in}{2.823986in}}%
\pgfpathlineto{\pgfqpoint{2.130471in}{2.823986in}}%
\pgfpathlineto{\pgfqpoint{2.130471in}{2.826935in}}%
\pgfpathlineto{\pgfqpoint{2.135012in}{2.826935in}}%
\pgfpathlineto{\pgfqpoint{2.135012in}{2.823986in}}%
\pgfpathmoveto{\pgfqpoint{2.130471in}{2.826935in}}%
\pgfpathlineto{\pgfqpoint{2.130471in}{2.826935in}}%
\pgfpathlineto{\pgfqpoint{2.130471in}{2.829884in}}%
\pgfpathlineto{\pgfqpoint{2.135012in}{2.829884in}}%
\pgfpathlineto{\pgfqpoint{2.135012in}{2.826935in}}%
\pgfpathmoveto{\pgfqpoint{2.062358in}{2.880021in}}%
\pgfpathlineto{\pgfqpoint{2.062358in}{2.880021in}}%
\pgfpathlineto{\pgfqpoint{2.062358in}{2.882970in}}%
\pgfpathlineto{\pgfqpoint{2.066899in}{2.882970in}}%
\pgfpathlineto{\pgfqpoint{2.066899in}{2.880021in}}%
\pgfpathmoveto{\pgfqpoint{2.071440in}{2.874123in}}%
\pgfpathlineto{\pgfqpoint{2.071440in}{2.874123in}}%
\pgfpathlineto{\pgfqpoint{2.071440in}{2.877072in}}%
\pgfpathlineto{\pgfqpoint{2.075980in}{2.877072in}}%
\pgfpathlineto{\pgfqpoint{2.075980in}{2.874123in}}%
\pgfpathmoveto{\pgfqpoint{2.066899in}{2.877072in}}%
\pgfpathlineto{\pgfqpoint{2.066899in}{2.877072in}}%
\pgfpathlineto{\pgfqpoint{2.066899in}{2.880021in}}%
\pgfpathlineto{\pgfqpoint{2.071440in}{2.880021in}}%
\pgfpathlineto{\pgfqpoint{2.071440in}{2.877072in}}%
\pgfpathmoveto{\pgfqpoint{2.066899in}{2.880021in}}%
\pgfpathlineto{\pgfqpoint{2.066899in}{2.880021in}}%
\pgfpathlineto{\pgfqpoint{2.066899in}{2.882970in}}%
\pgfpathlineto{\pgfqpoint{2.071440in}{2.882970in}}%
\pgfpathlineto{\pgfqpoint{2.071440in}{2.880021in}}%
\pgfpathmoveto{\pgfqpoint{2.071440in}{2.877072in}}%
\pgfpathlineto{\pgfqpoint{2.071440in}{2.877072in}}%
\pgfpathlineto{\pgfqpoint{2.071440in}{2.880021in}}%
\pgfpathlineto{\pgfqpoint{2.075980in}{2.880021in}}%
\pgfpathlineto{\pgfqpoint{2.075980in}{2.877072in}}%
\pgfpathmoveto{\pgfqpoint{2.075980in}{2.868224in}}%
\pgfpathlineto{\pgfqpoint{2.075980in}{2.868224in}}%
\pgfpathlineto{\pgfqpoint{2.075980in}{2.871173in}}%
\pgfpathlineto{\pgfqpoint{2.080521in}{2.871173in}}%
\pgfpathlineto{\pgfqpoint{2.080521in}{2.868224in}}%
\pgfpathmoveto{\pgfqpoint{2.080521in}{2.865275in}}%
\pgfpathlineto{\pgfqpoint{2.080521in}{2.865275in}}%
\pgfpathlineto{\pgfqpoint{2.080521in}{2.868224in}}%
\pgfpathlineto{\pgfqpoint{2.085062in}{2.868224in}}%
\pgfpathlineto{\pgfqpoint{2.085062in}{2.865275in}}%
\pgfpathmoveto{\pgfqpoint{2.080521in}{2.868224in}}%
\pgfpathlineto{\pgfqpoint{2.080521in}{2.868224in}}%
\pgfpathlineto{\pgfqpoint{2.080521in}{2.871173in}}%
\pgfpathlineto{\pgfqpoint{2.085062in}{2.871173in}}%
\pgfpathlineto{\pgfqpoint{2.085062in}{2.868224in}}%
\pgfpathmoveto{\pgfqpoint{2.085062in}{2.862326in}}%
\pgfpathlineto{\pgfqpoint{2.085062in}{2.862326in}}%
\pgfpathlineto{\pgfqpoint{2.085062in}{2.865275in}}%
\pgfpathlineto{\pgfqpoint{2.089603in}{2.865275in}}%
\pgfpathlineto{\pgfqpoint{2.089603in}{2.862326in}}%
\pgfpathmoveto{\pgfqpoint{2.089603in}{2.859376in}}%
\pgfpathlineto{\pgfqpoint{2.089603in}{2.859376in}}%
\pgfpathlineto{\pgfqpoint{2.089603in}{2.862326in}}%
\pgfpathlineto{\pgfqpoint{2.094144in}{2.862326in}}%
\pgfpathlineto{\pgfqpoint{2.094144in}{2.859376in}}%
\pgfpathmoveto{\pgfqpoint{2.089603in}{2.862326in}}%
\pgfpathlineto{\pgfqpoint{2.089603in}{2.862326in}}%
\pgfpathlineto{\pgfqpoint{2.089603in}{2.865275in}}%
\pgfpathlineto{\pgfqpoint{2.094144in}{2.865275in}}%
\pgfpathlineto{\pgfqpoint{2.094144in}{2.862326in}}%
\pgfpathmoveto{\pgfqpoint{2.085062in}{2.865275in}}%
\pgfpathlineto{\pgfqpoint{2.085062in}{2.865275in}}%
\pgfpathlineto{\pgfqpoint{2.085062in}{2.868224in}}%
\pgfpathlineto{\pgfqpoint{2.089603in}{2.868224in}}%
\pgfpathlineto{\pgfqpoint{2.089603in}{2.865275in}}%
\pgfpathmoveto{\pgfqpoint{2.075980in}{2.871173in}}%
\pgfpathlineto{\pgfqpoint{2.075980in}{2.871173in}}%
\pgfpathlineto{\pgfqpoint{2.075980in}{2.874123in}}%
\pgfpathlineto{\pgfqpoint{2.080521in}{2.874123in}}%
\pgfpathlineto{\pgfqpoint{2.080521in}{2.871173in}}%
\pgfpathmoveto{\pgfqpoint{2.075980in}{2.874123in}}%
\pgfpathlineto{\pgfqpoint{2.075980in}{2.874123in}}%
\pgfpathlineto{\pgfqpoint{2.075980in}{2.877072in}}%
\pgfpathlineto{\pgfqpoint{2.080521in}{2.877072in}}%
\pgfpathlineto{\pgfqpoint{2.080521in}{2.874123in}}%
\pgfpathmoveto{\pgfqpoint{2.057817in}{2.885920in}}%
\pgfpathlineto{\pgfqpoint{2.057817in}{2.885920in}}%
\pgfpathlineto{\pgfqpoint{2.057817in}{2.888869in}}%
\pgfpathlineto{\pgfqpoint{2.062358in}{2.888869in}}%
\pgfpathlineto{\pgfqpoint{2.062358in}{2.885920in}}%
\pgfpathmoveto{\pgfqpoint{2.062358in}{2.882970in}}%
\pgfpathlineto{\pgfqpoint{2.062358in}{2.882970in}}%
\pgfpathlineto{\pgfqpoint{2.062358in}{2.885920in}}%
\pgfpathlineto{\pgfqpoint{2.066899in}{2.885920in}}%
\pgfpathlineto{\pgfqpoint{2.066899in}{2.882970in}}%
\pgfpathmoveto{\pgfqpoint{2.062358in}{2.885920in}}%
\pgfpathlineto{\pgfqpoint{2.062358in}{2.885920in}}%
\pgfpathlineto{\pgfqpoint{2.062358in}{2.888869in}}%
\pgfpathlineto{\pgfqpoint{2.066899in}{2.888869in}}%
\pgfpathlineto{\pgfqpoint{2.066899in}{2.885920in}}%
\pgfpathmoveto{\pgfqpoint{2.057817in}{2.888869in}}%
\pgfpathlineto{\pgfqpoint{2.057817in}{2.888869in}}%
\pgfpathlineto{\pgfqpoint{2.057817in}{2.891818in}}%
\pgfpathlineto{\pgfqpoint{2.062358in}{2.891818in}}%
\pgfpathlineto{\pgfqpoint{2.062358in}{2.888869in}}%
\pgfpathmoveto{\pgfqpoint{2.307566in}{2.667678in}}%
\pgfpathlineto{\pgfqpoint{2.307566in}{2.667678in}}%
\pgfpathlineto{\pgfqpoint{2.307566in}{2.670628in}}%
\pgfpathlineto{\pgfqpoint{2.312107in}{2.670628in}}%
\pgfpathlineto{\pgfqpoint{2.312107in}{2.667678in}}%
\pgfpathmoveto{\pgfqpoint{2.343894in}{2.638186in}}%
\pgfpathlineto{\pgfqpoint{2.343894in}{2.638186in}}%
\pgfpathlineto{\pgfqpoint{2.343894in}{2.641135in}}%
\pgfpathlineto{\pgfqpoint{2.348434in}{2.641135in}}%
\pgfpathlineto{\pgfqpoint{2.348434in}{2.638186in}}%
\pgfpathmoveto{\pgfqpoint{2.339353in}{2.641135in}}%
\pgfpathlineto{\pgfqpoint{2.339353in}{2.641135in}}%
\pgfpathlineto{\pgfqpoint{2.339353in}{2.644084in}}%
\pgfpathlineto{\pgfqpoint{2.343894in}{2.644084in}}%
\pgfpathlineto{\pgfqpoint{2.343894in}{2.641135in}}%
\pgfpathmoveto{\pgfqpoint{2.339353in}{2.644084in}}%
\pgfpathlineto{\pgfqpoint{2.339353in}{2.644084in}}%
\pgfpathlineto{\pgfqpoint{2.339353in}{2.647033in}}%
\pgfpathlineto{\pgfqpoint{2.343894in}{2.647033in}}%
\pgfpathlineto{\pgfqpoint{2.343894in}{2.644084in}}%
\pgfpathmoveto{\pgfqpoint{2.343894in}{2.641135in}}%
\pgfpathlineto{\pgfqpoint{2.343894in}{2.641135in}}%
\pgfpathlineto{\pgfqpoint{2.343894in}{2.644084in}}%
\pgfpathlineto{\pgfqpoint{2.348434in}{2.644084in}}%
\pgfpathlineto{\pgfqpoint{2.348434in}{2.641135in}}%
\pgfpathmoveto{\pgfqpoint{2.321189in}{2.655881in}}%
\pgfpathlineto{\pgfqpoint{2.321189in}{2.655881in}}%
\pgfpathlineto{\pgfqpoint{2.321189in}{2.658831in}}%
\pgfpathlineto{\pgfqpoint{2.325730in}{2.658831in}}%
\pgfpathlineto{\pgfqpoint{2.325730in}{2.655881in}}%
\pgfpathmoveto{\pgfqpoint{2.325730in}{2.652932in}}%
\pgfpathlineto{\pgfqpoint{2.325730in}{2.652932in}}%
\pgfpathlineto{\pgfqpoint{2.325730in}{2.655881in}}%
\pgfpathlineto{\pgfqpoint{2.330271in}{2.655881in}}%
\pgfpathlineto{\pgfqpoint{2.330271in}{2.652932in}}%
\pgfpathmoveto{\pgfqpoint{2.325730in}{2.655881in}}%
\pgfpathlineto{\pgfqpoint{2.325730in}{2.655881in}}%
\pgfpathlineto{\pgfqpoint{2.325730in}{2.658831in}}%
\pgfpathlineto{\pgfqpoint{2.330271in}{2.658831in}}%
\pgfpathlineto{\pgfqpoint{2.330271in}{2.655881in}}%
\pgfpathmoveto{\pgfqpoint{2.316648in}{2.661780in}}%
\pgfpathlineto{\pgfqpoint{2.316648in}{2.661780in}}%
\pgfpathlineto{\pgfqpoint{2.316648in}{2.664729in}}%
\pgfpathlineto{\pgfqpoint{2.321189in}{2.664729in}}%
\pgfpathlineto{\pgfqpoint{2.321189in}{2.661780in}}%
\pgfpathmoveto{\pgfqpoint{2.312107in}{2.664729in}}%
\pgfpathlineto{\pgfqpoint{2.312107in}{2.664729in}}%
\pgfpathlineto{\pgfqpoint{2.312107in}{2.667678in}}%
\pgfpathlineto{\pgfqpoint{2.316648in}{2.667678in}}%
\pgfpathlineto{\pgfqpoint{2.316648in}{2.664729in}}%
\pgfpathmoveto{\pgfqpoint{2.312107in}{2.667678in}}%
\pgfpathlineto{\pgfqpoint{2.312107in}{2.667678in}}%
\pgfpathlineto{\pgfqpoint{2.312107in}{2.670628in}}%
\pgfpathlineto{\pgfqpoint{2.316648in}{2.670628in}}%
\pgfpathlineto{\pgfqpoint{2.316648in}{2.667678in}}%
\pgfpathmoveto{\pgfqpoint{2.316648in}{2.664729in}}%
\pgfpathlineto{\pgfqpoint{2.316648in}{2.664729in}}%
\pgfpathlineto{\pgfqpoint{2.316648in}{2.667678in}}%
\pgfpathlineto{\pgfqpoint{2.321189in}{2.667678in}}%
\pgfpathlineto{\pgfqpoint{2.321189in}{2.664729in}}%
\pgfpathmoveto{\pgfqpoint{2.321189in}{2.658831in}}%
\pgfpathlineto{\pgfqpoint{2.321189in}{2.658831in}}%
\pgfpathlineto{\pgfqpoint{2.321189in}{2.661780in}}%
\pgfpathlineto{\pgfqpoint{2.325730in}{2.661780in}}%
\pgfpathlineto{\pgfqpoint{2.325730in}{2.658831in}}%
\pgfpathmoveto{\pgfqpoint{2.321189in}{2.661780in}}%
\pgfpathlineto{\pgfqpoint{2.321189in}{2.661780in}}%
\pgfpathlineto{\pgfqpoint{2.321189in}{2.664729in}}%
\pgfpathlineto{\pgfqpoint{2.325730in}{2.664729in}}%
\pgfpathlineto{\pgfqpoint{2.325730in}{2.661780in}}%
\pgfpathmoveto{\pgfqpoint{2.330271in}{2.649983in}}%
\pgfpathlineto{\pgfqpoint{2.330271in}{2.649983in}}%
\pgfpathlineto{\pgfqpoint{2.330271in}{2.652932in}}%
\pgfpathlineto{\pgfqpoint{2.334812in}{2.652932in}}%
\pgfpathlineto{\pgfqpoint{2.334812in}{2.649983in}}%
\pgfpathmoveto{\pgfqpoint{2.334812in}{2.647033in}}%
\pgfpathlineto{\pgfqpoint{2.334812in}{2.647033in}}%
\pgfpathlineto{\pgfqpoint{2.334812in}{2.649983in}}%
\pgfpathlineto{\pgfqpoint{2.339353in}{2.649983in}}%
\pgfpathlineto{\pgfqpoint{2.339353in}{2.647033in}}%
\pgfpathmoveto{\pgfqpoint{2.334812in}{2.649983in}}%
\pgfpathlineto{\pgfqpoint{2.334812in}{2.649983in}}%
\pgfpathlineto{\pgfqpoint{2.334812in}{2.652932in}}%
\pgfpathlineto{\pgfqpoint{2.339353in}{2.652932in}}%
\pgfpathlineto{\pgfqpoint{2.339353in}{2.649983in}}%
\pgfpathmoveto{\pgfqpoint{2.330271in}{2.652932in}}%
\pgfpathlineto{\pgfqpoint{2.330271in}{2.652932in}}%
\pgfpathlineto{\pgfqpoint{2.330271in}{2.655881in}}%
\pgfpathlineto{\pgfqpoint{2.334812in}{2.655881in}}%
\pgfpathlineto{\pgfqpoint{2.334812in}{2.652932in}}%
\pgfpathmoveto{\pgfqpoint{2.339353in}{2.647033in}}%
\pgfpathlineto{\pgfqpoint{2.339353in}{2.647033in}}%
\pgfpathlineto{\pgfqpoint{2.339353in}{2.649983in}}%
\pgfpathlineto{\pgfqpoint{2.343894in}{2.649983in}}%
\pgfpathlineto{\pgfqpoint{2.343894in}{2.647033in}}%
\pgfpathmoveto{\pgfqpoint{2.253076in}{2.714865in}}%
\pgfpathlineto{\pgfqpoint{2.253076in}{2.714865in}}%
\pgfpathlineto{\pgfqpoint{2.253076in}{2.717814in}}%
\pgfpathlineto{\pgfqpoint{2.257617in}{2.717814in}}%
\pgfpathlineto{\pgfqpoint{2.257617in}{2.714865in}}%
\pgfpathmoveto{\pgfqpoint{2.266698in}{2.703069in}}%
\pgfpathlineto{\pgfqpoint{2.266698in}{2.703069in}}%
\pgfpathlineto{\pgfqpoint{2.266698in}{2.706018in}}%
\pgfpathlineto{\pgfqpoint{2.271239in}{2.706018in}}%
\pgfpathlineto{\pgfqpoint{2.271239in}{2.703069in}}%
\pgfpathmoveto{\pgfqpoint{2.271239in}{2.700119in}}%
\pgfpathlineto{\pgfqpoint{2.271239in}{2.700119in}}%
\pgfpathlineto{\pgfqpoint{2.271239in}{2.703069in}}%
\pgfpathlineto{\pgfqpoint{2.275780in}{2.703069in}}%
\pgfpathlineto{\pgfqpoint{2.275780in}{2.700119in}}%
\pgfpathmoveto{\pgfqpoint{2.271239in}{2.703069in}}%
\pgfpathlineto{\pgfqpoint{2.271239in}{2.703069in}}%
\pgfpathlineto{\pgfqpoint{2.271239in}{2.706018in}}%
\pgfpathlineto{\pgfqpoint{2.275780in}{2.706018in}}%
\pgfpathlineto{\pgfqpoint{2.275780in}{2.703069in}}%
\pgfpathmoveto{\pgfqpoint{2.262157in}{2.708967in}}%
\pgfpathlineto{\pgfqpoint{2.262157in}{2.708967in}}%
\pgfpathlineto{\pgfqpoint{2.262157in}{2.711916in}}%
\pgfpathlineto{\pgfqpoint{2.266698in}{2.711916in}}%
\pgfpathlineto{\pgfqpoint{2.266698in}{2.708967in}}%
\pgfpathmoveto{\pgfqpoint{2.257617in}{2.711916in}}%
\pgfpathlineto{\pgfqpoint{2.257617in}{2.711916in}}%
\pgfpathlineto{\pgfqpoint{2.257617in}{2.714865in}}%
\pgfpathlineto{\pgfqpoint{2.262157in}{2.714865in}}%
\pgfpathlineto{\pgfqpoint{2.262157in}{2.711916in}}%
\pgfpathmoveto{\pgfqpoint{2.257617in}{2.714865in}}%
\pgfpathlineto{\pgfqpoint{2.257617in}{2.714865in}}%
\pgfpathlineto{\pgfqpoint{2.257617in}{2.717814in}}%
\pgfpathlineto{\pgfqpoint{2.262157in}{2.717814in}}%
\pgfpathlineto{\pgfqpoint{2.262157in}{2.714865in}}%
\pgfpathmoveto{\pgfqpoint{2.262157in}{2.711916in}}%
\pgfpathlineto{\pgfqpoint{2.262157in}{2.711916in}}%
\pgfpathlineto{\pgfqpoint{2.262157in}{2.714865in}}%
\pgfpathlineto{\pgfqpoint{2.266698in}{2.714865in}}%
\pgfpathlineto{\pgfqpoint{2.266698in}{2.711916in}}%
\pgfpathmoveto{\pgfqpoint{2.266698in}{2.706018in}}%
\pgfpathlineto{\pgfqpoint{2.266698in}{2.706018in}}%
\pgfpathlineto{\pgfqpoint{2.266698in}{2.708967in}}%
\pgfpathlineto{\pgfqpoint{2.271239in}{2.708967in}}%
\pgfpathlineto{\pgfqpoint{2.271239in}{2.706018in}}%
\pgfpathmoveto{\pgfqpoint{2.266698in}{2.708967in}}%
\pgfpathlineto{\pgfqpoint{2.266698in}{2.708967in}}%
\pgfpathlineto{\pgfqpoint{2.266698in}{2.711916in}}%
\pgfpathlineto{\pgfqpoint{2.271239in}{2.711916in}}%
\pgfpathlineto{\pgfqpoint{2.271239in}{2.708967in}}%
\pgfpathmoveto{\pgfqpoint{2.225830in}{2.738458in}}%
\pgfpathlineto{\pgfqpoint{2.225830in}{2.738458in}}%
\pgfpathlineto{\pgfqpoint{2.225830in}{2.741408in}}%
\pgfpathlineto{\pgfqpoint{2.230371in}{2.741408in}}%
\pgfpathlineto{\pgfqpoint{2.230371in}{2.738458in}}%
\pgfpathmoveto{\pgfqpoint{2.234912in}{2.732560in}}%
\pgfpathlineto{\pgfqpoint{2.234912in}{2.732560in}}%
\pgfpathlineto{\pgfqpoint{2.234912in}{2.735509in}}%
\pgfpathlineto{\pgfqpoint{2.239453in}{2.735509in}}%
\pgfpathlineto{\pgfqpoint{2.239453in}{2.732560in}}%
\pgfpathmoveto{\pgfqpoint{2.230371in}{2.735509in}}%
\pgfpathlineto{\pgfqpoint{2.230371in}{2.735509in}}%
\pgfpathlineto{\pgfqpoint{2.230371in}{2.738458in}}%
\pgfpathlineto{\pgfqpoint{2.234912in}{2.738458in}}%
\pgfpathlineto{\pgfqpoint{2.234912in}{2.735509in}}%
\pgfpathmoveto{\pgfqpoint{2.230371in}{2.738458in}}%
\pgfpathlineto{\pgfqpoint{2.230371in}{2.738458in}}%
\pgfpathlineto{\pgfqpoint{2.230371in}{2.741408in}}%
\pgfpathlineto{\pgfqpoint{2.234912in}{2.741408in}}%
\pgfpathlineto{\pgfqpoint{2.234912in}{2.738458in}}%
\pgfpathmoveto{\pgfqpoint{2.234912in}{2.735509in}}%
\pgfpathlineto{\pgfqpoint{2.234912in}{2.735509in}}%
\pgfpathlineto{\pgfqpoint{2.234912in}{2.738458in}}%
\pgfpathlineto{\pgfqpoint{2.239453in}{2.738458in}}%
\pgfpathlineto{\pgfqpoint{2.239453in}{2.735509in}}%
\pgfpathmoveto{\pgfqpoint{2.212208in}{2.750255in}}%
\pgfpathlineto{\pgfqpoint{2.212208in}{2.750255in}}%
\pgfpathlineto{\pgfqpoint{2.212208in}{2.753204in}}%
\pgfpathlineto{\pgfqpoint{2.216749in}{2.753204in}}%
\pgfpathlineto{\pgfqpoint{2.216749in}{2.750255in}}%
\pgfpathmoveto{\pgfqpoint{2.216749in}{2.747306in}}%
\pgfpathlineto{\pgfqpoint{2.216749in}{2.747306in}}%
\pgfpathlineto{\pgfqpoint{2.216749in}{2.750255in}}%
\pgfpathlineto{\pgfqpoint{2.221289in}{2.750255in}}%
\pgfpathlineto{\pgfqpoint{2.221289in}{2.747306in}}%
\pgfpathmoveto{\pgfqpoint{2.216749in}{2.750255in}}%
\pgfpathlineto{\pgfqpoint{2.216749in}{2.750255in}}%
\pgfpathlineto{\pgfqpoint{2.216749in}{2.753204in}}%
\pgfpathlineto{\pgfqpoint{2.221289in}{2.753204in}}%
\pgfpathlineto{\pgfqpoint{2.221289in}{2.750255in}}%
\pgfpathmoveto{\pgfqpoint{2.207667in}{2.756153in}}%
\pgfpathlineto{\pgfqpoint{2.207667in}{2.756153in}}%
\pgfpathlineto{\pgfqpoint{2.207667in}{2.759103in}}%
\pgfpathlineto{\pgfqpoint{2.212208in}{2.759103in}}%
\pgfpathlineto{\pgfqpoint{2.212208in}{2.756153in}}%
\pgfpathmoveto{\pgfqpoint{2.203126in}{2.759103in}}%
\pgfpathlineto{\pgfqpoint{2.203126in}{2.759103in}}%
\pgfpathlineto{\pgfqpoint{2.203126in}{2.762052in}}%
\pgfpathlineto{\pgfqpoint{2.207667in}{2.762052in}}%
\pgfpathlineto{\pgfqpoint{2.207667in}{2.759103in}}%
\pgfpathmoveto{\pgfqpoint{2.203126in}{2.762052in}}%
\pgfpathlineto{\pgfqpoint{2.203126in}{2.762052in}}%
\pgfpathlineto{\pgfqpoint{2.203126in}{2.765001in}}%
\pgfpathlineto{\pgfqpoint{2.207667in}{2.765001in}}%
\pgfpathlineto{\pgfqpoint{2.207667in}{2.762052in}}%
\pgfpathmoveto{\pgfqpoint{2.207667in}{2.759103in}}%
\pgfpathlineto{\pgfqpoint{2.207667in}{2.759103in}}%
\pgfpathlineto{\pgfqpoint{2.207667in}{2.762052in}}%
\pgfpathlineto{\pgfqpoint{2.212208in}{2.762052in}}%
\pgfpathlineto{\pgfqpoint{2.212208in}{2.759103in}}%
\pgfpathmoveto{\pgfqpoint{2.212208in}{2.753204in}}%
\pgfpathlineto{\pgfqpoint{2.212208in}{2.753204in}}%
\pgfpathlineto{\pgfqpoint{2.212208in}{2.756153in}}%
\pgfpathlineto{\pgfqpoint{2.216749in}{2.756153in}}%
\pgfpathlineto{\pgfqpoint{2.216749in}{2.753204in}}%
\pgfpathmoveto{\pgfqpoint{2.212208in}{2.756153in}}%
\pgfpathlineto{\pgfqpoint{2.212208in}{2.756153in}}%
\pgfpathlineto{\pgfqpoint{2.212208in}{2.759103in}}%
\pgfpathlineto{\pgfqpoint{2.216749in}{2.759103in}}%
\pgfpathlineto{\pgfqpoint{2.216749in}{2.756153in}}%
\pgfpathmoveto{\pgfqpoint{2.221289in}{2.744357in}}%
\pgfpathlineto{\pgfqpoint{2.221289in}{2.744357in}}%
\pgfpathlineto{\pgfqpoint{2.221289in}{2.747306in}}%
\pgfpathlineto{\pgfqpoint{2.225830in}{2.747306in}}%
\pgfpathlineto{\pgfqpoint{2.225830in}{2.744357in}}%
\pgfpathmoveto{\pgfqpoint{2.225830in}{2.741408in}}%
\pgfpathlineto{\pgfqpoint{2.225830in}{2.741408in}}%
\pgfpathlineto{\pgfqpoint{2.225830in}{2.744357in}}%
\pgfpathlineto{\pgfqpoint{2.230371in}{2.744357in}}%
\pgfpathlineto{\pgfqpoint{2.230371in}{2.741408in}}%
\pgfpathmoveto{\pgfqpoint{2.225830in}{2.744357in}}%
\pgfpathlineto{\pgfqpoint{2.225830in}{2.744357in}}%
\pgfpathlineto{\pgfqpoint{2.225830in}{2.747306in}}%
\pgfpathlineto{\pgfqpoint{2.230371in}{2.747306in}}%
\pgfpathlineto{\pgfqpoint{2.230371in}{2.744357in}}%
\pgfpathmoveto{\pgfqpoint{2.221289in}{2.747306in}}%
\pgfpathlineto{\pgfqpoint{2.221289in}{2.747306in}}%
\pgfpathlineto{\pgfqpoint{2.221289in}{2.750255in}}%
\pgfpathlineto{\pgfqpoint{2.225830in}{2.750255in}}%
\pgfpathlineto{\pgfqpoint{2.225830in}{2.747306in}}%
\pgfpathmoveto{\pgfqpoint{2.239453in}{2.726662in}}%
\pgfpathlineto{\pgfqpoint{2.239453in}{2.726662in}}%
\pgfpathlineto{\pgfqpoint{2.239453in}{2.729611in}}%
\pgfpathlineto{\pgfqpoint{2.243994in}{2.729611in}}%
\pgfpathlineto{\pgfqpoint{2.243994in}{2.726662in}}%
\pgfpathmoveto{\pgfqpoint{2.243994in}{2.723713in}}%
\pgfpathlineto{\pgfqpoint{2.243994in}{2.723713in}}%
\pgfpathlineto{\pgfqpoint{2.243994in}{2.726662in}}%
\pgfpathlineto{\pgfqpoint{2.248535in}{2.726662in}}%
\pgfpathlineto{\pgfqpoint{2.248535in}{2.723713in}}%
\pgfpathmoveto{\pgfqpoint{2.243994in}{2.726662in}}%
\pgfpathlineto{\pgfqpoint{2.243994in}{2.726662in}}%
\pgfpathlineto{\pgfqpoint{2.243994in}{2.729611in}}%
\pgfpathlineto{\pgfqpoint{2.248535in}{2.729611in}}%
\pgfpathlineto{\pgfqpoint{2.248535in}{2.726662in}}%
\pgfpathmoveto{\pgfqpoint{2.248535in}{2.720763in}}%
\pgfpathlineto{\pgfqpoint{2.248535in}{2.720763in}}%
\pgfpathlineto{\pgfqpoint{2.248535in}{2.723713in}}%
\pgfpathlineto{\pgfqpoint{2.253076in}{2.723713in}}%
\pgfpathlineto{\pgfqpoint{2.253076in}{2.720763in}}%
\pgfpathmoveto{\pgfqpoint{2.253076in}{2.717814in}}%
\pgfpathlineto{\pgfqpoint{2.253076in}{2.717814in}}%
\pgfpathlineto{\pgfqpoint{2.253076in}{2.720763in}}%
\pgfpathlineto{\pgfqpoint{2.257617in}{2.720763in}}%
\pgfpathlineto{\pgfqpoint{2.257617in}{2.717814in}}%
\pgfpathmoveto{\pgfqpoint{2.253076in}{2.720763in}}%
\pgfpathlineto{\pgfqpoint{2.253076in}{2.720763in}}%
\pgfpathlineto{\pgfqpoint{2.253076in}{2.723713in}}%
\pgfpathlineto{\pgfqpoint{2.257617in}{2.723713in}}%
\pgfpathlineto{\pgfqpoint{2.257617in}{2.720763in}}%
\pgfpathmoveto{\pgfqpoint{2.248535in}{2.723713in}}%
\pgfpathlineto{\pgfqpoint{2.248535in}{2.723713in}}%
\pgfpathlineto{\pgfqpoint{2.248535in}{2.726662in}}%
\pgfpathlineto{\pgfqpoint{2.253076in}{2.726662in}}%
\pgfpathlineto{\pgfqpoint{2.253076in}{2.723713in}}%
\pgfpathmoveto{\pgfqpoint{2.239453in}{2.729611in}}%
\pgfpathlineto{\pgfqpoint{2.239453in}{2.729611in}}%
\pgfpathlineto{\pgfqpoint{2.239453in}{2.732560in}}%
\pgfpathlineto{\pgfqpoint{2.243994in}{2.732560in}}%
\pgfpathlineto{\pgfqpoint{2.243994in}{2.729611in}}%
\pgfpathmoveto{\pgfqpoint{2.239453in}{2.732560in}}%
\pgfpathlineto{\pgfqpoint{2.239453in}{2.732560in}}%
\pgfpathlineto{\pgfqpoint{2.239453in}{2.735509in}}%
\pgfpathlineto{\pgfqpoint{2.243994in}{2.735509in}}%
\pgfpathlineto{\pgfqpoint{2.243994in}{2.732560in}}%
\pgfpathmoveto{\pgfqpoint{2.280321in}{2.691272in}}%
\pgfpathlineto{\pgfqpoint{2.280321in}{2.691272in}}%
\pgfpathlineto{\pgfqpoint{2.280321in}{2.694221in}}%
\pgfpathlineto{\pgfqpoint{2.284862in}{2.694221in}}%
\pgfpathlineto{\pgfqpoint{2.284862in}{2.691272in}}%
\pgfpathmoveto{\pgfqpoint{2.289403in}{2.685374in}}%
\pgfpathlineto{\pgfqpoint{2.289403in}{2.685374in}}%
\pgfpathlineto{\pgfqpoint{2.289403in}{2.688323in}}%
\pgfpathlineto{\pgfqpoint{2.293944in}{2.688323in}}%
\pgfpathlineto{\pgfqpoint{2.293944in}{2.685374in}}%
\pgfpathmoveto{\pgfqpoint{2.284862in}{2.688323in}}%
\pgfpathlineto{\pgfqpoint{2.284862in}{2.688323in}}%
\pgfpathlineto{\pgfqpoint{2.284862in}{2.691272in}}%
\pgfpathlineto{\pgfqpoint{2.289403in}{2.691272in}}%
\pgfpathlineto{\pgfqpoint{2.289403in}{2.688323in}}%
\pgfpathmoveto{\pgfqpoint{2.284862in}{2.691272in}}%
\pgfpathlineto{\pgfqpoint{2.284862in}{2.691272in}}%
\pgfpathlineto{\pgfqpoint{2.284862in}{2.694221in}}%
\pgfpathlineto{\pgfqpoint{2.289403in}{2.694221in}}%
\pgfpathlineto{\pgfqpoint{2.289403in}{2.691272in}}%
\pgfpathmoveto{\pgfqpoint{2.289403in}{2.688323in}}%
\pgfpathlineto{\pgfqpoint{2.289403in}{2.688323in}}%
\pgfpathlineto{\pgfqpoint{2.289403in}{2.691272in}}%
\pgfpathlineto{\pgfqpoint{2.293944in}{2.691272in}}%
\pgfpathlineto{\pgfqpoint{2.293944in}{2.688323in}}%
\pgfpathmoveto{\pgfqpoint{2.293944in}{2.679475in}}%
\pgfpathlineto{\pgfqpoint{2.293944in}{2.679475in}}%
\pgfpathlineto{\pgfqpoint{2.293944in}{2.682424in}}%
\pgfpathlineto{\pgfqpoint{2.298485in}{2.682424in}}%
\pgfpathlineto{\pgfqpoint{2.298485in}{2.679475in}}%
\pgfpathmoveto{\pgfqpoint{2.298485in}{2.676526in}}%
\pgfpathlineto{\pgfqpoint{2.298485in}{2.676526in}}%
\pgfpathlineto{\pgfqpoint{2.298485in}{2.679475in}}%
\pgfpathlineto{\pgfqpoint{2.303026in}{2.679475in}}%
\pgfpathlineto{\pgfqpoint{2.303026in}{2.676526in}}%
\pgfpathmoveto{\pgfqpoint{2.298485in}{2.679475in}}%
\pgfpathlineto{\pgfqpoint{2.298485in}{2.679475in}}%
\pgfpathlineto{\pgfqpoint{2.298485in}{2.682424in}}%
\pgfpathlineto{\pgfqpoint{2.303026in}{2.682424in}}%
\pgfpathlineto{\pgfqpoint{2.303026in}{2.679475in}}%
\pgfpathmoveto{\pgfqpoint{2.303026in}{2.673577in}}%
\pgfpathlineto{\pgfqpoint{2.303026in}{2.673577in}}%
\pgfpathlineto{\pgfqpoint{2.303026in}{2.676526in}}%
\pgfpathlineto{\pgfqpoint{2.307566in}{2.676526in}}%
\pgfpathlineto{\pgfqpoint{2.307566in}{2.673577in}}%
\pgfpathmoveto{\pgfqpoint{2.307566in}{2.670628in}}%
\pgfpathlineto{\pgfqpoint{2.307566in}{2.670628in}}%
\pgfpathlineto{\pgfqpoint{2.307566in}{2.673577in}}%
\pgfpathlineto{\pgfqpoint{2.312107in}{2.673577in}}%
\pgfpathlineto{\pgfqpoint{2.312107in}{2.670628in}}%
\pgfpathmoveto{\pgfqpoint{2.307566in}{2.673577in}}%
\pgfpathlineto{\pgfqpoint{2.307566in}{2.673577in}}%
\pgfpathlineto{\pgfqpoint{2.307566in}{2.676526in}}%
\pgfpathlineto{\pgfqpoint{2.312107in}{2.676526in}}%
\pgfpathlineto{\pgfqpoint{2.312107in}{2.673577in}}%
\pgfpathmoveto{\pgfqpoint{2.303026in}{2.676526in}}%
\pgfpathlineto{\pgfqpoint{2.303026in}{2.676526in}}%
\pgfpathlineto{\pgfqpoint{2.303026in}{2.679475in}}%
\pgfpathlineto{\pgfqpoint{2.307566in}{2.679475in}}%
\pgfpathlineto{\pgfqpoint{2.307566in}{2.676526in}}%
\pgfpathmoveto{\pgfqpoint{2.293944in}{2.682424in}}%
\pgfpathlineto{\pgfqpoint{2.293944in}{2.682424in}}%
\pgfpathlineto{\pgfqpoint{2.293944in}{2.685374in}}%
\pgfpathlineto{\pgfqpoint{2.298485in}{2.685374in}}%
\pgfpathlineto{\pgfqpoint{2.298485in}{2.682424in}}%
\pgfpathmoveto{\pgfqpoint{2.293944in}{2.685374in}}%
\pgfpathlineto{\pgfqpoint{2.293944in}{2.685374in}}%
\pgfpathlineto{\pgfqpoint{2.293944in}{2.688323in}}%
\pgfpathlineto{\pgfqpoint{2.298485in}{2.688323in}}%
\pgfpathlineto{\pgfqpoint{2.298485in}{2.685374in}}%
\pgfpathmoveto{\pgfqpoint{2.275780in}{2.697170in}}%
\pgfpathlineto{\pgfqpoint{2.275780in}{2.697170in}}%
\pgfpathlineto{\pgfqpoint{2.275780in}{2.700119in}}%
\pgfpathlineto{\pgfqpoint{2.280321in}{2.700119in}}%
\pgfpathlineto{\pgfqpoint{2.280321in}{2.697170in}}%
\pgfpathmoveto{\pgfqpoint{2.280321in}{2.694221in}}%
\pgfpathlineto{\pgfqpoint{2.280321in}{2.694221in}}%
\pgfpathlineto{\pgfqpoint{2.280321in}{2.697170in}}%
\pgfpathlineto{\pgfqpoint{2.284862in}{2.697170in}}%
\pgfpathlineto{\pgfqpoint{2.284862in}{2.694221in}}%
\pgfpathmoveto{\pgfqpoint{2.280321in}{2.697170in}}%
\pgfpathlineto{\pgfqpoint{2.280321in}{2.697170in}}%
\pgfpathlineto{\pgfqpoint{2.280321in}{2.700119in}}%
\pgfpathlineto{\pgfqpoint{2.284862in}{2.700119in}}%
\pgfpathlineto{\pgfqpoint{2.284862in}{2.697170in}}%
\pgfpathmoveto{\pgfqpoint{2.275780in}{2.700119in}}%
\pgfpathlineto{\pgfqpoint{2.275780in}{2.700119in}}%
\pgfpathlineto{\pgfqpoint{2.275780in}{2.703069in}}%
\pgfpathlineto{\pgfqpoint{2.280321in}{2.703069in}}%
\pgfpathlineto{\pgfqpoint{2.280321in}{2.700119in}}%
\pgfpathmoveto{\pgfqpoint{2.489209in}{2.511366in}}%
\pgfpathlineto{\pgfqpoint{2.489209in}{2.511366in}}%
\pgfpathlineto{\pgfqpoint{2.489209in}{2.514315in}}%
\pgfpathlineto{\pgfqpoint{2.493750in}{2.514315in}}%
\pgfpathlineto{\pgfqpoint{2.493750in}{2.511366in}}%
\pgfpathmoveto{\pgfqpoint{2.489209in}{2.514315in}}%
\pgfpathlineto{\pgfqpoint{2.489209in}{2.514315in}}%
\pgfpathlineto{\pgfqpoint{2.489209in}{2.517264in}}%
\pgfpathlineto{\pgfqpoint{2.493750in}{2.517264in}}%
\pgfpathlineto{\pgfqpoint{2.493750in}{2.514315in}}%
\pgfpathmoveto{\pgfqpoint{2.480126in}{2.520214in}}%
\pgfpathlineto{\pgfqpoint{2.480126in}{2.520214in}}%
\pgfpathlineto{\pgfqpoint{2.480126in}{2.523163in}}%
\pgfpathlineto{\pgfqpoint{2.484667in}{2.523163in}}%
\pgfpathlineto{\pgfqpoint{2.484667in}{2.520214in}}%
\pgfpathmoveto{\pgfqpoint{2.475585in}{2.523163in}}%
\pgfpathlineto{\pgfqpoint{2.475585in}{2.523163in}}%
\pgfpathlineto{\pgfqpoint{2.475585in}{2.526112in}}%
\pgfpathlineto{\pgfqpoint{2.480126in}{2.526112in}}%
\pgfpathlineto{\pgfqpoint{2.480126in}{2.523163in}}%
\pgfpathmoveto{\pgfqpoint{2.475585in}{2.526112in}}%
\pgfpathlineto{\pgfqpoint{2.475585in}{2.526112in}}%
\pgfpathlineto{\pgfqpoint{2.475585in}{2.529062in}}%
\pgfpathlineto{\pgfqpoint{2.480126in}{2.529062in}}%
\pgfpathlineto{\pgfqpoint{2.480126in}{2.526112in}}%
\pgfpathmoveto{\pgfqpoint{2.480126in}{2.523163in}}%
\pgfpathlineto{\pgfqpoint{2.480126in}{2.523163in}}%
\pgfpathlineto{\pgfqpoint{2.480126in}{2.526112in}}%
\pgfpathlineto{\pgfqpoint{2.484667in}{2.526112in}}%
\pgfpathlineto{\pgfqpoint{2.484667in}{2.523163in}}%
\pgfpathmoveto{\pgfqpoint{2.484667in}{2.517264in}}%
\pgfpathlineto{\pgfqpoint{2.484667in}{2.517264in}}%
\pgfpathlineto{\pgfqpoint{2.484667in}{2.520214in}}%
\pgfpathlineto{\pgfqpoint{2.489209in}{2.520214in}}%
\pgfpathlineto{\pgfqpoint{2.489209in}{2.517264in}}%
\pgfpathmoveto{\pgfqpoint{2.484667in}{2.520214in}}%
\pgfpathlineto{\pgfqpoint{2.484667in}{2.520214in}}%
\pgfpathlineto{\pgfqpoint{2.484667in}{2.523163in}}%
\pgfpathlineto{\pgfqpoint{2.489209in}{2.523163in}}%
\pgfpathlineto{\pgfqpoint{2.489209in}{2.520214in}}%
\pgfpathmoveto{\pgfqpoint{2.489209in}{2.517264in}}%
\pgfpathlineto{\pgfqpoint{2.489209in}{2.517264in}}%
\pgfpathlineto{\pgfqpoint{2.489209in}{2.520214in}}%
\pgfpathlineto{\pgfqpoint{2.493750in}{2.520214in}}%
\pgfpathlineto{\pgfqpoint{2.493750in}{2.517264in}}%
\pgfpathmoveto{\pgfqpoint{2.452880in}{2.543808in}}%
\pgfpathlineto{\pgfqpoint{2.452880in}{2.543808in}}%
\pgfpathlineto{\pgfqpoint{2.452880in}{2.546758in}}%
\pgfpathlineto{\pgfqpoint{2.457421in}{2.546758in}}%
\pgfpathlineto{\pgfqpoint{2.457421in}{2.543808in}}%
\pgfpathmoveto{\pgfqpoint{2.448339in}{2.546758in}}%
\pgfpathlineto{\pgfqpoint{2.448339in}{2.546758in}}%
\pgfpathlineto{\pgfqpoint{2.448339in}{2.549707in}}%
\pgfpathlineto{\pgfqpoint{2.452880in}{2.549707in}}%
\pgfpathlineto{\pgfqpoint{2.452880in}{2.546758in}}%
\pgfpathmoveto{\pgfqpoint{2.448339in}{2.549707in}}%
\pgfpathlineto{\pgfqpoint{2.448339in}{2.549707in}}%
\pgfpathlineto{\pgfqpoint{2.448339in}{2.552656in}}%
\pgfpathlineto{\pgfqpoint{2.452880in}{2.552656in}}%
\pgfpathlineto{\pgfqpoint{2.452880in}{2.549707in}}%
\pgfpathmoveto{\pgfqpoint{2.452880in}{2.546758in}}%
\pgfpathlineto{\pgfqpoint{2.452880in}{2.546758in}}%
\pgfpathlineto{\pgfqpoint{2.452880in}{2.549707in}}%
\pgfpathlineto{\pgfqpoint{2.457421in}{2.549707in}}%
\pgfpathlineto{\pgfqpoint{2.457421in}{2.546758in}}%
\pgfpathmoveto{\pgfqpoint{2.434715in}{2.558555in}}%
\pgfpathlineto{\pgfqpoint{2.434715in}{2.558555in}}%
\pgfpathlineto{\pgfqpoint{2.434715in}{2.561504in}}%
\pgfpathlineto{\pgfqpoint{2.439256in}{2.561504in}}%
\pgfpathlineto{\pgfqpoint{2.439256in}{2.558555in}}%
\pgfpathmoveto{\pgfqpoint{2.434715in}{2.561504in}}%
\pgfpathlineto{\pgfqpoint{2.434715in}{2.561504in}}%
\pgfpathlineto{\pgfqpoint{2.434715in}{2.564453in}}%
\pgfpathlineto{\pgfqpoint{2.439256in}{2.564453in}}%
\pgfpathlineto{\pgfqpoint{2.439256in}{2.561504in}}%
\pgfpathmoveto{\pgfqpoint{2.425633in}{2.567403in}}%
\pgfpathlineto{\pgfqpoint{2.425633in}{2.567403in}}%
\pgfpathlineto{\pgfqpoint{2.425633in}{2.570352in}}%
\pgfpathlineto{\pgfqpoint{2.430174in}{2.570352in}}%
\pgfpathlineto{\pgfqpoint{2.430174in}{2.567403in}}%
\pgfpathmoveto{\pgfqpoint{2.421092in}{2.570352in}}%
\pgfpathlineto{\pgfqpoint{2.421092in}{2.570352in}}%
\pgfpathlineto{\pgfqpoint{2.421092in}{2.573301in}}%
\pgfpathlineto{\pgfqpoint{2.425633in}{2.573301in}}%
\pgfpathlineto{\pgfqpoint{2.425633in}{2.570352in}}%
\pgfpathmoveto{\pgfqpoint{2.421092in}{2.573301in}}%
\pgfpathlineto{\pgfqpoint{2.421092in}{2.573301in}}%
\pgfpathlineto{\pgfqpoint{2.421092in}{2.576251in}}%
\pgfpathlineto{\pgfqpoint{2.425633in}{2.576251in}}%
\pgfpathlineto{\pgfqpoint{2.425633in}{2.573301in}}%
\pgfpathmoveto{\pgfqpoint{2.425633in}{2.570352in}}%
\pgfpathlineto{\pgfqpoint{2.425633in}{2.570352in}}%
\pgfpathlineto{\pgfqpoint{2.425633in}{2.573301in}}%
\pgfpathlineto{\pgfqpoint{2.430174in}{2.573301in}}%
\pgfpathlineto{\pgfqpoint{2.430174in}{2.570352in}}%
\pgfpathmoveto{\pgfqpoint{2.430174in}{2.564453in}}%
\pgfpathlineto{\pgfqpoint{2.430174in}{2.564453in}}%
\pgfpathlineto{\pgfqpoint{2.430174in}{2.567403in}}%
\pgfpathlineto{\pgfqpoint{2.434715in}{2.567403in}}%
\pgfpathlineto{\pgfqpoint{2.434715in}{2.564453in}}%
\pgfpathmoveto{\pgfqpoint{2.430174in}{2.567403in}}%
\pgfpathlineto{\pgfqpoint{2.430174in}{2.567403in}}%
\pgfpathlineto{\pgfqpoint{2.430174in}{2.570352in}}%
\pgfpathlineto{\pgfqpoint{2.434715in}{2.570352in}}%
\pgfpathlineto{\pgfqpoint{2.434715in}{2.567403in}}%
\pgfpathmoveto{\pgfqpoint{2.434715in}{2.564453in}}%
\pgfpathlineto{\pgfqpoint{2.434715in}{2.564453in}}%
\pgfpathlineto{\pgfqpoint{2.434715in}{2.567403in}}%
\pgfpathlineto{\pgfqpoint{2.439256in}{2.567403in}}%
\pgfpathlineto{\pgfqpoint{2.439256in}{2.564453in}}%
\pgfpathmoveto{\pgfqpoint{2.439256in}{2.555605in}}%
\pgfpathlineto{\pgfqpoint{2.439256in}{2.555605in}}%
\pgfpathlineto{\pgfqpoint{2.439256in}{2.558555in}}%
\pgfpathlineto{\pgfqpoint{2.443798in}{2.558555in}}%
\pgfpathlineto{\pgfqpoint{2.443798in}{2.555605in}}%
\pgfpathmoveto{\pgfqpoint{2.443798in}{2.552656in}}%
\pgfpathlineto{\pgfqpoint{2.443798in}{2.552656in}}%
\pgfpathlineto{\pgfqpoint{2.443798in}{2.555605in}}%
\pgfpathlineto{\pgfqpoint{2.448339in}{2.555605in}}%
\pgfpathlineto{\pgfqpoint{2.448339in}{2.552656in}}%
\pgfpathmoveto{\pgfqpoint{2.443798in}{2.555605in}}%
\pgfpathlineto{\pgfqpoint{2.443798in}{2.555605in}}%
\pgfpathlineto{\pgfqpoint{2.443798in}{2.558555in}}%
\pgfpathlineto{\pgfqpoint{2.448339in}{2.558555in}}%
\pgfpathlineto{\pgfqpoint{2.448339in}{2.555605in}}%
\pgfpathmoveto{\pgfqpoint{2.439256in}{2.558555in}}%
\pgfpathlineto{\pgfqpoint{2.439256in}{2.558555in}}%
\pgfpathlineto{\pgfqpoint{2.439256in}{2.561504in}}%
\pgfpathlineto{\pgfqpoint{2.443798in}{2.561504in}}%
\pgfpathlineto{\pgfqpoint{2.443798in}{2.558555in}}%
\pgfpathmoveto{\pgfqpoint{2.448339in}{2.552656in}}%
\pgfpathlineto{\pgfqpoint{2.448339in}{2.552656in}}%
\pgfpathlineto{\pgfqpoint{2.448339in}{2.555605in}}%
\pgfpathlineto{\pgfqpoint{2.452880in}{2.555605in}}%
\pgfpathlineto{\pgfqpoint{2.452880in}{2.552656in}}%
\pgfpathmoveto{\pgfqpoint{2.461962in}{2.534960in}}%
\pgfpathlineto{\pgfqpoint{2.461962in}{2.534960in}}%
\pgfpathlineto{\pgfqpoint{2.461962in}{2.537910in}}%
\pgfpathlineto{\pgfqpoint{2.466503in}{2.537910in}}%
\pgfpathlineto{\pgfqpoint{2.466503in}{2.534960in}}%
\pgfpathmoveto{\pgfqpoint{2.461962in}{2.537910in}}%
\pgfpathlineto{\pgfqpoint{2.461962in}{2.537910in}}%
\pgfpathlineto{\pgfqpoint{2.461962in}{2.540859in}}%
\pgfpathlineto{\pgfqpoint{2.466503in}{2.540859in}}%
\pgfpathlineto{\pgfqpoint{2.466503in}{2.537910in}}%
\pgfpathmoveto{\pgfqpoint{2.466503in}{2.532011in}}%
\pgfpathlineto{\pgfqpoint{2.466503in}{2.532011in}}%
\pgfpathlineto{\pgfqpoint{2.466503in}{2.534960in}}%
\pgfpathlineto{\pgfqpoint{2.471044in}{2.534960in}}%
\pgfpathlineto{\pgfqpoint{2.471044in}{2.532011in}}%
\pgfpathmoveto{\pgfqpoint{2.471044in}{2.529062in}}%
\pgfpathlineto{\pgfqpoint{2.471044in}{2.529062in}}%
\pgfpathlineto{\pgfqpoint{2.471044in}{2.532011in}}%
\pgfpathlineto{\pgfqpoint{2.475585in}{2.532011in}}%
\pgfpathlineto{\pgfqpoint{2.475585in}{2.529062in}}%
\pgfpathmoveto{\pgfqpoint{2.471044in}{2.532011in}}%
\pgfpathlineto{\pgfqpoint{2.471044in}{2.532011in}}%
\pgfpathlineto{\pgfqpoint{2.471044in}{2.534960in}}%
\pgfpathlineto{\pgfqpoint{2.475585in}{2.534960in}}%
\pgfpathlineto{\pgfqpoint{2.475585in}{2.532011in}}%
\pgfpathmoveto{\pgfqpoint{2.466503in}{2.534960in}}%
\pgfpathlineto{\pgfqpoint{2.466503in}{2.534960in}}%
\pgfpathlineto{\pgfqpoint{2.466503in}{2.537910in}}%
\pgfpathlineto{\pgfqpoint{2.471044in}{2.537910in}}%
\pgfpathlineto{\pgfqpoint{2.471044in}{2.534960in}}%
\pgfpathmoveto{\pgfqpoint{2.457421in}{2.540859in}}%
\pgfpathlineto{\pgfqpoint{2.457421in}{2.540859in}}%
\pgfpathlineto{\pgfqpoint{2.457421in}{2.543808in}}%
\pgfpathlineto{\pgfqpoint{2.461962in}{2.543808in}}%
\pgfpathlineto{\pgfqpoint{2.461962in}{2.540859in}}%
\pgfpathmoveto{\pgfqpoint{2.457421in}{2.543808in}}%
\pgfpathlineto{\pgfqpoint{2.457421in}{2.543808in}}%
\pgfpathlineto{\pgfqpoint{2.457421in}{2.546758in}}%
\pgfpathlineto{\pgfqpoint{2.461962in}{2.546758in}}%
\pgfpathlineto{\pgfqpoint{2.461962in}{2.543808in}}%
\pgfpathmoveto{\pgfqpoint{2.461962in}{2.540859in}}%
\pgfpathlineto{\pgfqpoint{2.461962in}{2.540859in}}%
\pgfpathlineto{\pgfqpoint{2.461962in}{2.543808in}}%
\pgfpathlineto{\pgfqpoint{2.466503in}{2.543808in}}%
\pgfpathlineto{\pgfqpoint{2.466503in}{2.540859in}}%
\pgfpathmoveto{\pgfqpoint{2.475585in}{2.529062in}}%
\pgfpathlineto{\pgfqpoint{2.475585in}{2.529062in}}%
\pgfpathlineto{\pgfqpoint{2.475585in}{2.532011in}}%
\pgfpathlineto{\pgfqpoint{2.480126in}{2.532011in}}%
\pgfpathlineto{\pgfqpoint{2.480126in}{2.529062in}}%
\pgfpathmoveto{\pgfqpoint{2.380222in}{2.605743in}}%
\pgfpathlineto{\pgfqpoint{2.380222in}{2.605743in}}%
\pgfpathlineto{\pgfqpoint{2.380222in}{2.608693in}}%
\pgfpathlineto{\pgfqpoint{2.384763in}{2.608693in}}%
\pgfpathlineto{\pgfqpoint{2.384763in}{2.605743in}}%
\pgfpathmoveto{\pgfqpoint{2.380222in}{2.608693in}}%
\pgfpathlineto{\pgfqpoint{2.380222in}{2.608693in}}%
\pgfpathlineto{\pgfqpoint{2.380222in}{2.611642in}}%
\pgfpathlineto{\pgfqpoint{2.384763in}{2.611642in}}%
\pgfpathlineto{\pgfqpoint{2.384763in}{2.608693in}}%
\pgfpathmoveto{\pgfqpoint{2.371140in}{2.614591in}}%
\pgfpathlineto{\pgfqpoint{2.371140in}{2.614591in}}%
\pgfpathlineto{\pgfqpoint{2.371140in}{2.617541in}}%
\pgfpathlineto{\pgfqpoint{2.375681in}{2.617541in}}%
\pgfpathlineto{\pgfqpoint{2.375681in}{2.614591in}}%
\pgfpathmoveto{\pgfqpoint{2.366599in}{2.617541in}}%
\pgfpathlineto{\pgfqpoint{2.366599in}{2.617541in}}%
\pgfpathlineto{\pgfqpoint{2.366599in}{2.620490in}}%
\pgfpathlineto{\pgfqpoint{2.371140in}{2.620490in}}%
\pgfpathlineto{\pgfqpoint{2.371140in}{2.617541in}}%
\pgfpathmoveto{\pgfqpoint{2.366599in}{2.620490in}}%
\pgfpathlineto{\pgfqpoint{2.366599in}{2.620490in}}%
\pgfpathlineto{\pgfqpoint{2.366599in}{2.623439in}}%
\pgfpathlineto{\pgfqpoint{2.371140in}{2.623439in}}%
\pgfpathlineto{\pgfqpoint{2.371140in}{2.620490in}}%
\pgfpathmoveto{\pgfqpoint{2.371140in}{2.617541in}}%
\pgfpathlineto{\pgfqpoint{2.371140in}{2.617541in}}%
\pgfpathlineto{\pgfqpoint{2.371140in}{2.620490in}}%
\pgfpathlineto{\pgfqpoint{2.375681in}{2.620490in}}%
\pgfpathlineto{\pgfqpoint{2.375681in}{2.617541in}}%
\pgfpathmoveto{\pgfqpoint{2.375681in}{2.611642in}}%
\pgfpathlineto{\pgfqpoint{2.375681in}{2.611642in}}%
\pgfpathlineto{\pgfqpoint{2.375681in}{2.614591in}}%
\pgfpathlineto{\pgfqpoint{2.380222in}{2.614591in}}%
\pgfpathlineto{\pgfqpoint{2.380222in}{2.611642in}}%
\pgfpathmoveto{\pgfqpoint{2.375681in}{2.614591in}}%
\pgfpathlineto{\pgfqpoint{2.375681in}{2.614591in}}%
\pgfpathlineto{\pgfqpoint{2.375681in}{2.617541in}}%
\pgfpathlineto{\pgfqpoint{2.380222in}{2.617541in}}%
\pgfpathlineto{\pgfqpoint{2.380222in}{2.614591in}}%
\pgfpathmoveto{\pgfqpoint{2.380222in}{2.611642in}}%
\pgfpathlineto{\pgfqpoint{2.380222in}{2.611642in}}%
\pgfpathlineto{\pgfqpoint{2.380222in}{2.614591in}}%
\pgfpathlineto{\pgfqpoint{2.384763in}{2.614591in}}%
\pgfpathlineto{\pgfqpoint{2.384763in}{2.611642in}}%
\pgfpathmoveto{\pgfqpoint{2.398387in}{2.590997in}}%
\pgfpathlineto{\pgfqpoint{2.398387in}{2.590997in}}%
\pgfpathlineto{\pgfqpoint{2.398387in}{2.593946in}}%
\pgfpathlineto{\pgfqpoint{2.402928in}{2.593946in}}%
\pgfpathlineto{\pgfqpoint{2.402928in}{2.590997in}}%
\pgfpathmoveto{\pgfqpoint{2.393845in}{2.593946in}}%
\pgfpathlineto{\pgfqpoint{2.393845in}{2.593946in}}%
\pgfpathlineto{\pgfqpoint{2.393845in}{2.596896in}}%
\pgfpathlineto{\pgfqpoint{2.398387in}{2.596896in}}%
\pgfpathlineto{\pgfqpoint{2.398387in}{2.593946in}}%
\pgfpathmoveto{\pgfqpoint{2.393845in}{2.596896in}}%
\pgfpathlineto{\pgfqpoint{2.393845in}{2.596896in}}%
\pgfpathlineto{\pgfqpoint{2.393845in}{2.599845in}}%
\pgfpathlineto{\pgfqpoint{2.398387in}{2.599845in}}%
\pgfpathlineto{\pgfqpoint{2.398387in}{2.596896in}}%
\pgfpathmoveto{\pgfqpoint{2.398387in}{2.593946in}}%
\pgfpathlineto{\pgfqpoint{2.398387in}{2.593946in}}%
\pgfpathlineto{\pgfqpoint{2.398387in}{2.596896in}}%
\pgfpathlineto{\pgfqpoint{2.402928in}{2.596896in}}%
\pgfpathlineto{\pgfqpoint{2.402928in}{2.593946in}}%
\pgfpathmoveto{\pgfqpoint{2.407469in}{2.582149in}}%
\pgfpathlineto{\pgfqpoint{2.407469in}{2.582149in}}%
\pgfpathlineto{\pgfqpoint{2.407469in}{2.585098in}}%
\pgfpathlineto{\pgfqpoint{2.412010in}{2.585098in}}%
\pgfpathlineto{\pgfqpoint{2.412010in}{2.582149in}}%
\pgfpathmoveto{\pgfqpoint{2.407469in}{2.585098in}}%
\pgfpathlineto{\pgfqpoint{2.407469in}{2.585098in}}%
\pgfpathlineto{\pgfqpoint{2.407469in}{2.588048in}}%
\pgfpathlineto{\pgfqpoint{2.412010in}{2.588048in}}%
\pgfpathlineto{\pgfqpoint{2.412010in}{2.585098in}}%
\pgfpathmoveto{\pgfqpoint{2.412010in}{2.579200in}}%
\pgfpathlineto{\pgfqpoint{2.412010in}{2.579200in}}%
\pgfpathlineto{\pgfqpoint{2.412010in}{2.582149in}}%
\pgfpathlineto{\pgfqpoint{2.416551in}{2.582149in}}%
\pgfpathlineto{\pgfqpoint{2.416551in}{2.579200in}}%
\pgfpathmoveto{\pgfqpoint{2.416551in}{2.576251in}}%
\pgfpathlineto{\pgfqpoint{2.416551in}{2.576251in}}%
\pgfpathlineto{\pgfqpoint{2.416551in}{2.579200in}}%
\pgfpathlineto{\pgfqpoint{2.421092in}{2.579200in}}%
\pgfpathlineto{\pgfqpoint{2.421092in}{2.576251in}}%
\pgfpathmoveto{\pgfqpoint{2.416551in}{2.579200in}}%
\pgfpathlineto{\pgfqpoint{2.416551in}{2.579200in}}%
\pgfpathlineto{\pgfqpoint{2.416551in}{2.582149in}}%
\pgfpathlineto{\pgfqpoint{2.421092in}{2.582149in}}%
\pgfpathlineto{\pgfqpoint{2.421092in}{2.579200in}}%
\pgfpathmoveto{\pgfqpoint{2.412010in}{2.582149in}}%
\pgfpathlineto{\pgfqpoint{2.412010in}{2.582149in}}%
\pgfpathlineto{\pgfqpoint{2.412010in}{2.585098in}}%
\pgfpathlineto{\pgfqpoint{2.416551in}{2.585098in}}%
\pgfpathlineto{\pgfqpoint{2.416551in}{2.582149in}}%
\pgfpathmoveto{\pgfqpoint{2.402928in}{2.588048in}}%
\pgfpathlineto{\pgfqpoint{2.402928in}{2.588048in}}%
\pgfpathlineto{\pgfqpoint{2.402928in}{2.590997in}}%
\pgfpathlineto{\pgfqpoint{2.407469in}{2.590997in}}%
\pgfpathlineto{\pgfqpoint{2.407469in}{2.588048in}}%
\pgfpathmoveto{\pgfqpoint{2.402928in}{2.590997in}}%
\pgfpathlineto{\pgfqpoint{2.402928in}{2.590997in}}%
\pgfpathlineto{\pgfqpoint{2.402928in}{2.593946in}}%
\pgfpathlineto{\pgfqpoint{2.407469in}{2.593946in}}%
\pgfpathlineto{\pgfqpoint{2.407469in}{2.590997in}}%
\pgfpathmoveto{\pgfqpoint{2.407469in}{2.588048in}}%
\pgfpathlineto{\pgfqpoint{2.407469in}{2.588048in}}%
\pgfpathlineto{\pgfqpoint{2.407469in}{2.590997in}}%
\pgfpathlineto{\pgfqpoint{2.412010in}{2.590997in}}%
\pgfpathlineto{\pgfqpoint{2.412010in}{2.588048in}}%
\pgfpathmoveto{\pgfqpoint{2.384763in}{2.602794in}}%
\pgfpathlineto{\pgfqpoint{2.384763in}{2.602794in}}%
\pgfpathlineto{\pgfqpoint{2.384763in}{2.605743in}}%
\pgfpathlineto{\pgfqpoint{2.389304in}{2.605743in}}%
\pgfpathlineto{\pgfqpoint{2.389304in}{2.602794in}}%
\pgfpathmoveto{\pgfqpoint{2.389304in}{2.599845in}}%
\pgfpathlineto{\pgfqpoint{2.389304in}{2.599845in}}%
\pgfpathlineto{\pgfqpoint{2.389304in}{2.602794in}}%
\pgfpathlineto{\pgfqpoint{2.393845in}{2.602794in}}%
\pgfpathlineto{\pgfqpoint{2.393845in}{2.599845in}}%
\pgfpathmoveto{\pgfqpoint{2.389304in}{2.602794in}}%
\pgfpathlineto{\pgfqpoint{2.389304in}{2.602794in}}%
\pgfpathlineto{\pgfqpoint{2.389304in}{2.605743in}}%
\pgfpathlineto{\pgfqpoint{2.393845in}{2.605743in}}%
\pgfpathlineto{\pgfqpoint{2.393845in}{2.602794in}}%
\pgfpathmoveto{\pgfqpoint{2.384763in}{2.605743in}}%
\pgfpathlineto{\pgfqpoint{2.384763in}{2.605743in}}%
\pgfpathlineto{\pgfqpoint{2.384763in}{2.608693in}}%
\pgfpathlineto{\pgfqpoint{2.389304in}{2.608693in}}%
\pgfpathlineto{\pgfqpoint{2.389304in}{2.605743in}}%
\pgfpathmoveto{\pgfqpoint{2.393845in}{2.599845in}}%
\pgfpathlineto{\pgfqpoint{2.393845in}{2.599845in}}%
\pgfpathlineto{\pgfqpoint{2.393845in}{2.602794in}}%
\pgfpathlineto{\pgfqpoint{2.398387in}{2.602794in}}%
\pgfpathlineto{\pgfqpoint{2.398387in}{2.599845in}}%
\pgfpathmoveto{\pgfqpoint{2.352976in}{2.629338in}}%
\pgfpathlineto{\pgfqpoint{2.352976in}{2.629338in}}%
\pgfpathlineto{\pgfqpoint{2.352976in}{2.632287in}}%
\pgfpathlineto{\pgfqpoint{2.357517in}{2.632287in}}%
\pgfpathlineto{\pgfqpoint{2.357517in}{2.629338in}}%
\pgfpathmoveto{\pgfqpoint{2.352976in}{2.632287in}}%
\pgfpathlineto{\pgfqpoint{2.352976in}{2.632287in}}%
\pgfpathlineto{\pgfqpoint{2.352976in}{2.635236in}}%
\pgfpathlineto{\pgfqpoint{2.357517in}{2.635236in}}%
\pgfpathlineto{\pgfqpoint{2.357517in}{2.632287in}}%
\pgfpathmoveto{\pgfqpoint{2.357517in}{2.626388in}}%
\pgfpathlineto{\pgfqpoint{2.357517in}{2.626388in}}%
\pgfpathlineto{\pgfqpoint{2.357517in}{2.629338in}}%
\pgfpathlineto{\pgfqpoint{2.362058in}{2.629338in}}%
\pgfpathlineto{\pgfqpoint{2.362058in}{2.626388in}}%
\pgfpathmoveto{\pgfqpoint{2.362058in}{2.623439in}}%
\pgfpathlineto{\pgfqpoint{2.362058in}{2.623439in}}%
\pgfpathlineto{\pgfqpoint{2.362058in}{2.626388in}}%
\pgfpathlineto{\pgfqpoint{2.366599in}{2.626388in}}%
\pgfpathlineto{\pgfqpoint{2.366599in}{2.623439in}}%
\pgfpathmoveto{\pgfqpoint{2.362058in}{2.626388in}}%
\pgfpathlineto{\pgfqpoint{2.362058in}{2.626388in}}%
\pgfpathlineto{\pgfqpoint{2.362058in}{2.629338in}}%
\pgfpathlineto{\pgfqpoint{2.366599in}{2.629338in}}%
\pgfpathlineto{\pgfqpoint{2.366599in}{2.626388in}}%
\pgfpathmoveto{\pgfqpoint{2.357517in}{2.629338in}}%
\pgfpathlineto{\pgfqpoint{2.357517in}{2.629338in}}%
\pgfpathlineto{\pgfqpoint{2.357517in}{2.632287in}}%
\pgfpathlineto{\pgfqpoint{2.362058in}{2.632287in}}%
\pgfpathlineto{\pgfqpoint{2.362058in}{2.629338in}}%
\pgfpathmoveto{\pgfqpoint{2.348434in}{2.635236in}}%
\pgfpathlineto{\pgfqpoint{2.348434in}{2.635236in}}%
\pgfpathlineto{\pgfqpoint{2.348434in}{2.638186in}}%
\pgfpathlineto{\pgfqpoint{2.352976in}{2.638186in}}%
\pgfpathlineto{\pgfqpoint{2.352976in}{2.635236in}}%
\pgfpathmoveto{\pgfqpoint{2.348434in}{2.638186in}}%
\pgfpathlineto{\pgfqpoint{2.348434in}{2.638186in}}%
\pgfpathlineto{\pgfqpoint{2.348434in}{2.641135in}}%
\pgfpathlineto{\pgfqpoint{2.352976in}{2.641135in}}%
\pgfpathlineto{\pgfqpoint{2.352976in}{2.638186in}}%
\pgfpathmoveto{\pgfqpoint{2.352976in}{2.635236in}}%
\pgfpathlineto{\pgfqpoint{2.352976in}{2.635236in}}%
\pgfpathlineto{\pgfqpoint{2.352976in}{2.638186in}}%
\pgfpathlineto{\pgfqpoint{2.357517in}{2.638186in}}%
\pgfpathlineto{\pgfqpoint{2.357517in}{2.635236in}}%
\pgfpathmoveto{\pgfqpoint{2.366599in}{2.623439in}}%
\pgfpathlineto{\pgfqpoint{2.366599in}{2.623439in}}%
\pgfpathlineto{\pgfqpoint{2.366599in}{2.626388in}}%
\pgfpathlineto{\pgfqpoint{2.371140in}{2.626388in}}%
\pgfpathlineto{\pgfqpoint{2.371140in}{2.623439in}}%
\pgfpathmoveto{\pgfqpoint{2.421092in}{2.576251in}}%
\pgfpathlineto{\pgfqpoint{2.421092in}{2.576251in}}%
\pgfpathlineto{\pgfqpoint{2.421092in}{2.579200in}}%
\pgfpathlineto{\pgfqpoint{2.425633in}{2.579200in}}%
\pgfpathlineto{\pgfqpoint{2.425633in}{2.576251in}}%
\pgfpathmoveto{\pgfqpoint{2.634522in}{2.384552in}}%
\pgfpathlineto{\pgfqpoint{2.634522in}{2.384552in}}%
\pgfpathlineto{\pgfqpoint{2.634522in}{2.387501in}}%
\pgfpathlineto{\pgfqpoint{2.639063in}{2.387501in}}%
\pgfpathlineto{\pgfqpoint{2.639063in}{2.384552in}}%
\pgfpathmoveto{\pgfqpoint{2.561865in}{2.449432in}}%
\pgfpathlineto{\pgfqpoint{2.561865in}{2.449432in}}%
\pgfpathlineto{\pgfqpoint{2.561865in}{2.452382in}}%
\pgfpathlineto{\pgfqpoint{2.566406in}{2.452382in}}%
\pgfpathlineto{\pgfqpoint{2.566406in}{2.449432in}}%
\pgfpathmoveto{\pgfqpoint{2.557324in}{2.452382in}}%
\pgfpathlineto{\pgfqpoint{2.557324in}{2.452382in}}%
\pgfpathlineto{\pgfqpoint{2.557324in}{2.455331in}}%
\pgfpathlineto{\pgfqpoint{2.561865in}{2.455331in}}%
\pgfpathlineto{\pgfqpoint{2.561865in}{2.452382in}}%
\pgfpathmoveto{\pgfqpoint{2.557324in}{2.455331in}}%
\pgfpathlineto{\pgfqpoint{2.557324in}{2.455331in}}%
\pgfpathlineto{\pgfqpoint{2.557324in}{2.458280in}}%
\pgfpathlineto{\pgfqpoint{2.561865in}{2.458280in}}%
\pgfpathlineto{\pgfqpoint{2.561865in}{2.455331in}}%
\pgfpathmoveto{\pgfqpoint{2.561865in}{2.452382in}}%
\pgfpathlineto{\pgfqpoint{2.561865in}{2.452382in}}%
\pgfpathlineto{\pgfqpoint{2.561865in}{2.455331in}}%
\pgfpathlineto{\pgfqpoint{2.566406in}{2.455331in}}%
\pgfpathlineto{\pgfqpoint{2.566406in}{2.452382in}}%
\pgfpathmoveto{\pgfqpoint{2.543701in}{2.464178in}}%
\pgfpathlineto{\pgfqpoint{2.543701in}{2.464178in}}%
\pgfpathlineto{\pgfqpoint{2.543701in}{2.467127in}}%
\pgfpathlineto{\pgfqpoint{2.548242in}{2.467127in}}%
\pgfpathlineto{\pgfqpoint{2.548242in}{2.464178in}}%
\pgfpathmoveto{\pgfqpoint{2.543701in}{2.467127in}}%
\pgfpathlineto{\pgfqpoint{2.543701in}{2.467127in}}%
\pgfpathlineto{\pgfqpoint{2.543701in}{2.470076in}}%
\pgfpathlineto{\pgfqpoint{2.548242in}{2.470076in}}%
\pgfpathlineto{\pgfqpoint{2.548242in}{2.467127in}}%
\pgfpathmoveto{\pgfqpoint{2.534619in}{2.473025in}}%
\pgfpathlineto{\pgfqpoint{2.534619in}{2.473025in}}%
\pgfpathlineto{\pgfqpoint{2.534619in}{2.475974in}}%
\pgfpathlineto{\pgfqpoint{2.539160in}{2.475974in}}%
\pgfpathlineto{\pgfqpoint{2.539160in}{2.473025in}}%
\pgfpathmoveto{\pgfqpoint{2.530078in}{2.475974in}}%
\pgfpathlineto{\pgfqpoint{2.530078in}{2.475974in}}%
\pgfpathlineto{\pgfqpoint{2.530078in}{2.478924in}}%
\pgfpathlineto{\pgfqpoint{2.534619in}{2.478924in}}%
\pgfpathlineto{\pgfqpoint{2.534619in}{2.475974in}}%
\pgfpathmoveto{\pgfqpoint{2.530078in}{2.478924in}}%
\pgfpathlineto{\pgfqpoint{2.530078in}{2.478924in}}%
\pgfpathlineto{\pgfqpoint{2.530078in}{2.481873in}}%
\pgfpathlineto{\pgfqpoint{2.534619in}{2.481873in}}%
\pgfpathlineto{\pgfqpoint{2.534619in}{2.478924in}}%
\pgfpathmoveto{\pgfqpoint{2.534619in}{2.475974in}}%
\pgfpathlineto{\pgfqpoint{2.534619in}{2.475974in}}%
\pgfpathlineto{\pgfqpoint{2.534619in}{2.478924in}}%
\pgfpathlineto{\pgfqpoint{2.539160in}{2.478924in}}%
\pgfpathlineto{\pgfqpoint{2.539160in}{2.475974in}}%
\pgfpathmoveto{\pgfqpoint{2.539160in}{2.470076in}}%
\pgfpathlineto{\pgfqpoint{2.539160in}{2.470076in}}%
\pgfpathlineto{\pgfqpoint{2.539160in}{2.473025in}}%
\pgfpathlineto{\pgfqpoint{2.543701in}{2.473025in}}%
\pgfpathlineto{\pgfqpoint{2.543701in}{2.470076in}}%
\pgfpathmoveto{\pgfqpoint{2.539160in}{2.473025in}}%
\pgfpathlineto{\pgfqpoint{2.539160in}{2.473025in}}%
\pgfpathlineto{\pgfqpoint{2.539160in}{2.475974in}}%
\pgfpathlineto{\pgfqpoint{2.543701in}{2.475974in}}%
\pgfpathlineto{\pgfqpoint{2.543701in}{2.473025in}}%
\pgfpathmoveto{\pgfqpoint{2.543701in}{2.470076in}}%
\pgfpathlineto{\pgfqpoint{2.543701in}{2.470076in}}%
\pgfpathlineto{\pgfqpoint{2.543701in}{2.473025in}}%
\pgfpathlineto{\pgfqpoint{2.548242in}{2.473025in}}%
\pgfpathlineto{\pgfqpoint{2.548242in}{2.470076in}}%
\pgfpathmoveto{\pgfqpoint{2.548242in}{2.461229in}}%
\pgfpathlineto{\pgfqpoint{2.548242in}{2.461229in}}%
\pgfpathlineto{\pgfqpoint{2.548242in}{2.464178in}}%
\pgfpathlineto{\pgfqpoint{2.552783in}{2.464178in}}%
\pgfpathlineto{\pgfqpoint{2.552783in}{2.461229in}}%
\pgfpathmoveto{\pgfqpoint{2.552783in}{2.458280in}}%
\pgfpathlineto{\pgfqpoint{2.552783in}{2.458280in}}%
\pgfpathlineto{\pgfqpoint{2.552783in}{2.461229in}}%
\pgfpathlineto{\pgfqpoint{2.557324in}{2.461229in}}%
\pgfpathlineto{\pgfqpoint{2.557324in}{2.458280in}}%
\pgfpathmoveto{\pgfqpoint{2.552783in}{2.461229in}}%
\pgfpathlineto{\pgfqpoint{2.552783in}{2.461229in}}%
\pgfpathlineto{\pgfqpoint{2.552783in}{2.464178in}}%
\pgfpathlineto{\pgfqpoint{2.557324in}{2.464178in}}%
\pgfpathlineto{\pgfqpoint{2.557324in}{2.461229in}}%
\pgfpathmoveto{\pgfqpoint{2.548242in}{2.464178in}}%
\pgfpathlineto{\pgfqpoint{2.548242in}{2.464178in}}%
\pgfpathlineto{\pgfqpoint{2.548242in}{2.467127in}}%
\pgfpathlineto{\pgfqpoint{2.552783in}{2.467127in}}%
\pgfpathlineto{\pgfqpoint{2.552783in}{2.464178in}}%
\pgfpathmoveto{\pgfqpoint{2.557324in}{2.458280in}}%
\pgfpathlineto{\pgfqpoint{2.557324in}{2.458280in}}%
\pgfpathlineto{\pgfqpoint{2.557324in}{2.461229in}}%
\pgfpathlineto{\pgfqpoint{2.561865in}{2.461229in}}%
\pgfpathlineto{\pgfqpoint{2.561865in}{2.458280in}}%
\pgfpathmoveto{\pgfqpoint{2.598194in}{2.416992in}}%
\pgfpathlineto{\pgfqpoint{2.598194in}{2.416992in}}%
\pgfpathlineto{\pgfqpoint{2.598194in}{2.419941in}}%
\pgfpathlineto{\pgfqpoint{2.602735in}{2.419941in}}%
\pgfpathlineto{\pgfqpoint{2.602735in}{2.416992in}}%
\pgfpathmoveto{\pgfqpoint{2.598194in}{2.419941in}}%
\pgfpathlineto{\pgfqpoint{2.598194in}{2.419941in}}%
\pgfpathlineto{\pgfqpoint{2.598194in}{2.422890in}}%
\pgfpathlineto{\pgfqpoint{2.602735in}{2.422890in}}%
\pgfpathlineto{\pgfqpoint{2.602735in}{2.419941in}}%
\pgfpathmoveto{\pgfqpoint{2.589111in}{2.425840in}}%
\pgfpathlineto{\pgfqpoint{2.589111in}{2.425840in}}%
\pgfpathlineto{\pgfqpoint{2.589111in}{2.428789in}}%
\pgfpathlineto{\pgfqpoint{2.593652in}{2.428789in}}%
\pgfpathlineto{\pgfqpoint{2.593652in}{2.425840in}}%
\pgfpathmoveto{\pgfqpoint{2.584570in}{2.428789in}}%
\pgfpathlineto{\pgfqpoint{2.584570in}{2.428789in}}%
\pgfpathlineto{\pgfqpoint{2.584570in}{2.431738in}}%
\pgfpathlineto{\pgfqpoint{2.589111in}{2.431738in}}%
\pgfpathlineto{\pgfqpoint{2.589111in}{2.428789in}}%
\pgfpathmoveto{\pgfqpoint{2.584570in}{2.431738in}}%
\pgfpathlineto{\pgfqpoint{2.584570in}{2.431738in}}%
\pgfpathlineto{\pgfqpoint{2.584570in}{2.434687in}}%
\pgfpathlineto{\pgfqpoint{2.589111in}{2.434687in}}%
\pgfpathlineto{\pgfqpoint{2.589111in}{2.431738in}}%
\pgfpathmoveto{\pgfqpoint{2.589111in}{2.428789in}}%
\pgfpathlineto{\pgfqpoint{2.589111in}{2.428789in}}%
\pgfpathlineto{\pgfqpoint{2.589111in}{2.431738in}}%
\pgfpathlineto{\pgfqpoint{2.593652in}{2.431738in}}%
\pgfpathlineto{\pgfqpoint{2.593652in}{2.428789in}}%
\pgfpathmoveto{\pgfqpoint{2.593652in}{2.422890in}}%
\pgfpathlineto{\pgfqpoint{2.593652in}{2.422890in}}%
\pgfpathlineto{\pgfqpoint{2.593652in}{2.425840in}}%
\pgfpathlineto{\pgfqpoint{2.598194in}{2.425840in}}%
\pgfpathlineto{\pgfqpoint{2.598194in}{2.422890in}}%
\pgfpathmoveto{\pgfqpoint{2.593652in}{2.425840in}}%
\pgfpathlineto{\pgfqpoint{2.593652in}{2.425840in}}%
\pgfpathlineto{\pgfqpoint{2.593652in}{2.428789in}}%
\pgfpathlineto{\pgfqpoint{2.598194in}{2.428789in}}%
\pgfpathlineto{\pgfqpoint{2.598194in}{2.425840in}}%
\pgfpathmoveto{\pgfqpoint{2.598194in}{2.422890in}}%
\pgfpathlineto{\pgfqpoint{2.598194in}{2.422890in}}%
\pgfpathlineto{\pgfqpoint{2.598194in}{2.425840in}}%
\pgfpathlineto{\pgfqpoint{2.602735in}{2.425840in}}%
\pgfpathlineto{\pgfqpoint{2.602735in}{2.422890in}}%
\pgfpathmoveto{\pgfqpoint{2.607276in}{2.408145in}}%
\pgfpathlineto{\pgfqpoint{2.607276in}{2.408145in}}%
\pgfpathlineto{\pgfqpoint{2.607276in}{2.411094in}}%
\pgfpathlineto{\pgfqpoint{2.611817in}{2.411094in}}%
\pgfpathlineto{\pgfqpoint{2.611817in}{2.408145in}}%
\pgfpathmoveto{\pgfqpoint{2.616358in}{2.402247in}}%
\pgfpathlineto{\pgfqpoint{2.616358in}{2.402247in}}%
\pgfpathlineto{\pgfqpoint{2.616358in}{2.405196in}}%
\pgfpathlineto{\pgfqpoint{2.620899in}{2.405196in}}%
\pgfpathlineto{\pgfqpoint{2.620899in}{2.402247in}}%
\pgfpathmoveto{\pgfqpoint{2.611817in}{2.405196in}}%
\pgfpathlineto{\pgfqpoint{2.611817in}{2.405196in}}%
\pgfpathlineto{\pgfqpoint{2.611817in}{2.408145in}}%
\pgfpathlineto{\pgfqpoint{2.616358in}{2.408145in}}%
\pgfpathlineto{\pgfqpoint{2.616358in}{2.405196in}}%
\pgfpathmoveto{\pgfqpoint{2.611817in}{2.408145in}}%
\pgfpathlineto{\pgfqpoint{2.611817in}{2.408145in}}%
\pgfpathlineto{\pgfqpoint{2.611817in}{2.411094in}}%
\pgfpathlineto{\pgfqpoint{2.616358in}{2.411094in}}%
\pgfpathlineto{\pgfqpoint{2.616358in}{2.408145in}}%
\pgfpathmoveto{\pgfqpoint{2.616358in}{2.405196in}}%
\pgfpathlineto{\pgfqpoint{2.616358in}{2.405196in}}%
\pgfpathlineto{\pgfqpoint{2.616358in}{2.408145in}}%
\pgfpathlineto{\pgfqpoint{2.620899in}{2.408145in}}%
\pgfpathlineto{\pgfqpoint{2.620899in}{2.405196in}}%
\pgfpathmoveto{\pgfqpoint{2.620899in}{2.396348in}}%
\pgfpathlineto{\pgfqpoint{2.620899in}{2.396348in}}%
\pgfpathlineto{\pgfqpoint{2.620899in}{2.399298in}}%
\pgfpathlineto{\pgfqpoint{2.625440in}{2.399298in}}%
\pgfpathlineto{\pgfqpoint{2.625440in}{2.396348in}}%
\pgfpathmoveto{\pgfqpoint{2.625440in}{2.393399in}}%
\pgfpathlineto{\pgfqpoint{2.625440in}{2.393399in}}%
\pgfpathlineto{\pgfqpoint{2.625440in}{2.396348in}}%
\pgfpathlineto{\pgfqpoint{2.629981in}{2.396348in}}%
\pgfpathlineto{\pgfqpoint{2.629981in}{2.393399in}}%
\pgfpathmoveto{\pgfqpoint{2.625440in}{2.396348in}}%
\pgfpathlineto{\pgfqpoint{2.625440in}{2.396348in}}%
\pgfpathlineto{\pgfqpoint{2.625440in}{2.399298in}}%
\pgfpathlineto{\pgfqpoint{2.629981in}{2.399298in}}%
\pgfpathlineto{\pgfqpoint{2.629981in}{2.396348in}}%
\pgfpathmoveto{\pgfqpoint{2.629981in}{2.390450in}}%
\pgfpathlineto{\pgfqpoint{2.629981in}{2.390450in}}%
\pgfpathlineto{\pgfqpoint{2.629981in}{2.393399in}}%
\pgfpathlineto{\pgfqpoint{2.634522in}{2.393399in}}%
\pgfpathlineto{\pgfqpoint{2.634522in}{2.390450in}}%
\pgfpathmoveto{\pgfqpoint{2.634522in}{2.387501in}}%
\pgfpathlineto{\pgfqpoint{2.634522in}{2.387501in}}%
\pgfpathlineto{\pgfqpoint{2.634522in}{2.390450in}}%
\pgfpathlineto{\pgfqpoint{2.639063in}{2.390450in}}%
\pgfpathlineto{\pgfqpoint{2.639063in}{2.387501in}}%
\pgfpathmoveto{\pgfqpoint{2.634522in}{2.390450in}}%
\pgfpathlineto{\pgfqpoint{2.634522in}{2.390450in}}%
\pgfpathlineto{\pgfqpoint{2.634522in}{2.393399in}}%
\pgfpathlineto{\pgfqpoint{2.639063in}{2.393399in}}%
\pgfpathlineto{\pgfqpoint{2.639063in}{2.390450in}}%
\pgfpathmoveto{\pgfqpoint{2.629981in}{2.393399in}}%
\pgfpathlineto{\pgfqpoint{2.629981in}{2.393399in}}%
\pgfpathlineto{\pgfqpoint{2.629981in}{2.396348in}}%
\pgfpathlineto{\pgfqpoint{2.634522in}{2.396348in}}%
\pgfpathlineto{\pgfqpoint{2.634522in}{2.393399in}}%
\pgfpathmoveto{\pgfqpoint{2.620899in}{2.399298in}}%
\pgfpathlineto{\pgfqpoint{2.620899in}{2.399298in}}%
\pgfpathlineto{\pgfqpoint{2.620899in}{2.402247in}}%
\pgfpathlineto{\pgfqpoint{2.625440in}{2.402247in}}%
\pgfpathlineto{\pgfqpoint{2.625440in}{2.399298in}}%
\pgfpathmoveto{\pgfqpoint{2.620899in}{2.402247in}}%
\pgfpathlineto{\pgfqpoint{2.620899in}{2.402247in}}%
\pgfpathlineto{\pgfqpoint{2.620899in}{2.405196in}}%
\pgfpathlineto{\pgfqpoint{2.625440in}{2.405196in}}%
\pgfpathlineto{\pgfqpoint{2.625440in}{2.402247in}}%
\pgfpathmoveto{\pgfqpoint{2.602735in}{2.414043in}}%
\pgfpathlineto{\pgfqpoint{2.602735in}{2.414043in}}%
\pgfpathlineto{\pgfqpoint{2.602735in}{2.416992in}}%
\pgfpathlineto{\pgfqpoint{2.607276in}{2.416992in}}%
\pgfpathlineto{\pgfqpoint{2.607276in}{2.414043in}}%
\pgfpathmoveto{\pgfqpoint{2.607276in}{2.411094in}}%
\pgfpathlineto{\pgfqpoint{2.607276in}{2.411094in}}%
\pgfpathlineto{\pgfqpoint{2.607276in}{2.414043in}}%
\pgfpathlineto{\pgfqpoint{2.611817in}{2.414043in}}%
\pgfpathlineto{\pgfqpoint{2.611817in}{2.411094in}}%
\pgfpathmoveto{\pgfqpoint{2.607276in}{2.414043in}}%
\pgfpathlineto{\pgfqpoint{2.607276in}{2.414043in}}%
\pgfpathlineto{\pgfqpoint{2.607276in}{2.416992in}}%
\pgfpathlineto{\pgfqpoint{2.611817in}{2.416992in}}%
\pgfpathlineto{\pgfqpoint{2.611817in}{2.414043in}}%
\pgfpathmoveto{\pgfqpoint{2.602735in}{2.416992in}}%
\pgfpathlineto{\pgfqpoint{2.602735in}{2.416992in}}%
\pgfpathlineto{\pgfqpoint{2.602735in}{2.419941in}}%
\pgfpathlineto{\pgfqpoint{2.607276in}{2.419941in}}%
\pgfpathlineto{\pgfqpoint{2.607276in}{2.416992in}}%
\pgfpathmoveto{\pgfqpoint{2.570947in}{2.440585in}}%
\pgfpathlineto{\pgfqpoint{2.570947in}{2.440585in}}%
\pgfpathlineto{\pgfqpoint{2.570947in}{2.443534in}}%
\pgfpathlineto{\pgfqpoint{2.575488in}{2.443534in}}%
\pgfpathlineto{\pgfqpoint{2.575488in}{2.440585in}}%
\pgfpathmoveto{\pgfqpoint{2.570947in}{2.443534in}}%
\pgfpathlineto{\pgfqpoint{2.570947in}{2.443534in}}%
\pgfpathlineto{\pgfqpoint{2.570947in}{2.446483in}}%
\pgfpathlineto{\pgfqpoint{2.575488in}{2.446483in}}%
\pgfpathlineto{\pgfqpoint{2.575488in}{2.443534in}}%
\pgfpathmoveto{\pgfqpoint{2.575488in}{2.437636in}}%
\pgfpathlineto{\pgfqpoint{2.575488in}{2.437636in}}%
\pgfpathlineto{\pgfqpoint{2.575488in}{2.440585in}}%
\pgfpathlineto{\pgfqpoint{2.580029in}{2.440585in}}%
\pgfpathlineto{\pgfqpoint{2.580029in}{2.437636in}}%
\pgfpathmoveto{\pgfqpoint{2.580029in}{2.434687in}}%
\pgfpathlineto{\pgfqpoint{2.580029in}{2.434687in}}%
\pgfpathlineto{\pgfqpoint{2.580029in}{2.437636in}}%
\pgfpathlineto{\pgfqpoint{2.584570in}{2.437636in}}%
\pgfpathlineto{\pgfqpoint{2.584570in}{2.434687in}}%
\pgfpathmoveto{\pgfqpoint{2.580029in}{2.437636in}}%
\pgfpathlineto{\pgfqpoint{2.580029in}{2.437636in}}%
\pgfpathlineto{\pgfqpoint{2.580029in}{2.440585in}}%
\pgfpathlineto{\pgfqpoint{2.584570in}{2.440585in}}%
\pgfpathlineto{\pgfqpoint{2.584570in}{2.437636in}}%
\pgfpathmoveto{\pgfqpoint{2.575488in}{2.440585in}}%
\pgfpathlineto{\pgfqpoint{2.575488in}{2.440585in}}%
\pgfpathlineto{\pgfqpoint{2.575488in}{2.443534in}}%
\pgfpathlineto{\pgfqpoint{2.580029in}{2.443534in}}%
\pgfpathlineto{\pgfqpoint{2.580029in}{2.440585in}}%
\pgfpathmoveto{\pgfqpoint{2.566406in}{2.446483in}}%
\pgfpathlineto{\pgfqpoint{2.566406in}{2.446483in}}%
\pgfpathlineto{\pgfqpoint{2.566406in}{2.449432in}}%
\pgfpathlineto{\pgfqpoint{2.570947in}{2.449432in}}%
\pgfpathlineto{\pgfqpoint{2.570947in}{2.446483in}}%
\pgfpathmoveto{\pgfqpoint{2.566406in}{2.449432in}}%
\pgfpathlineto{\pgfqpoint{2.566406in}{2.449432in}}%
\pgfpathlineto{\pgfqpoint{2.566406in}{2.452382in}}%
\pgfpathlineto{\pgfqpoint{2.570947in}{2.452382in}}%
\pgfpathlineto{\pgfqpoint{2.570947in}{2.449432in}}%
\pgfpathmoveto{\pgfqpoint{2.570947in}{2.446483in}}%
\pgfpathlineto{\pgfqpoint{2.570947in}{2.446483in}}%
\pgfpathlineto{\pgfqpoint{2.570947in}{2.449432in}}%
\pgfpathlineto{\pgfqpoint{2.575488in}{2.449432in}}%
\pgfpathlineto{\pgfqpoint{2.575488in}{2.446483in}}%
\pgfpathmoveto{\pgfqpoint{2.584570in}{2.434687in}}%
\pgfpathlineto{\pgfqpoint{2.584570in}{2.434687in}}%
\pgfpathlineto{\pgfqpoint{2.584570in}{2.437636in}}%
\pgfpathlineto{\pgfqpoint{2.589111in}{2.437636in}}%
\pgfpathlineto{\pgfqpoint{2.589111in}{2.434687in}}%
\pgfpathmoveto{\pgfqpoint{2.507373in}{2.496619in}}%
\pgfpathlineto{\pgfqpoint{2.507373in}{2.496619in}}%
\pgfpathlineto{\pgfqpoint{2.507373in}{2.499569in}}%
\pgfpathlineto{\pgfqpoint{2.511914in}{2.499569in}}%
\pgfpathlineto{\pgfqpoint{2.511914in}{2.496619in}}%
\pgfpathmoveto{\pgfqpoint{2.502832in}{2.499569in}}%
\pgfpathlineto{\pgfqpoint{2.502832in}{2.499569in}}%
\pgfpathlineto{\pgfqpoint{2.502832in}{2.502518in}}%
\pgfpathlineto{\pgfqpoint{2.507373in}{2.502518in}}%
\pgfpathlineto{\pgfqpoint{2.507373in}{2.499569in}}%
\pgfpathmoveto{\pgfqpoint{2.502832in}{2.502518in}}%
\pgfpathlineto{\pgfqpoint{2.502832in}{2.502518in}}%
\pgfpathlineto{\pgfqpoint{2.502832in}{2.505467in}}%
\pgfpathlineto{\pgfqpoint{2.507373in}{2.505467in}}%
\pgfpathlineto{\pgfqpoint{2.507373in}{2.502518in}}%
\pgfpathmoveto{\pgfqpoint{2.507373in}{2.499569in}}%
\pgfpathlineto{\pgfqpoint{2.507373in}{2.499569in}}%
\pgfpathlineto{\pgfqpoint{2.507373in}{2.502518in}}%
\pgfpathlineto{\pgfqpoint{2.511914in}{2.502518in}}%
\pgfpathlineto{\pgfqpoint{2.511914in}{2.499569in}}%
\pgfpathmoveto{\pgfqpoint{2.516455in}{2.487771in}}%
\pgfpathlineto{\pgfqpoint{2.516455in}{2.487771in}}%
\pgfpathlineto{\pgfqpoint{2.516455in}{2.490721in}}%
\pgfpathlineto{\pgfqpoint{2.520996in}{2.490721in}}%
\pgfpathlineto{\pgfqpoint{2.520996in}{2.487771in}}%
\pgfpathmoveto{\pgfqpoint{2.516455in}{2.490721in}}%
\pgfpathlineto{\pgfqpoint{2.516455in}{2.490721in}}%
\pgfpathlineto{\pgfqpoint{2.516455in}{2.493670in}}%
\pgfpathlineto{\pgfqpoint{2.520996in}{2.493670in}}%
\pgfpathlineto{\pgfqpoint{2.520996in}{2.490721in}}%
\pgfpathmoveto{\pgfqpoint{2.520996in}{2.484822in}}%
\pgfpathlineto{\pgfqpoint{2.520996in}{2.484822in}}%
\pgfpathlineto{\pgfqpoint{2.520996in}{2.487771in}}%
\pgfpathlineto{\pgfqpoint{2.525537in}{2.487771in}}%
\pgfpathlineto{\pgfqpoint{2.525537in}{2.484822in}}%
\pgfpathmoveto{\pgfqpoint{2.525537in}{2.481873in}}%
\pgfpathlineto{\pgfqpoint{2.525537in}{2.481873in}}%
\pgfpathlineto{\pgfqpoint{2.525537in}{2.484822in}}%
\pgfpathlineto{\pgfqpoint{2.530078in}{2.484822in}}%
\pgfpathlineto{\pgfqpoint{2.530078in}{2.481873in}}%
\pgfpathmoveto{\pgfqpoint{2.525537in}{2.484822in}}%
\pgfpathlineto{\pgfqpoint{2.525537in}{2.484822in}}%
\pgfpathlineto{\pgfqpoint{2.525537in}{2.487771in}}%
\pgfpathlineto{\pgfqpoint{2.530078in}{2.487771in}}%
\pgfpathlineto{\pgfqpoint{2.530078in}{2.484822in}}%
\pgfpathmoveto{\pgfqpoint{2.520996in}{2.487771in}}%
\pgfpathlineto{\pgfqpoint{2.520996in}{2.487771in}}%
\pgfpathlineto{\pgfqpoint{2.520996in}{2.490721in}}%
\pgfpathlineto{\pgfqpoint{2.525537in}{2.490721in}}%
\pgfpathlineto{\pgfqpoint{2.525537in}{2.487771in}}%
\pgfpathmoveto{\pgfqpoint{2.511914in}{2.493670in}}%
\pgfpathlineto{\pgfqpoint{2.511914in}{2.493670in}}%
\pgfpathlineto{\pgfqpoint{2.511914in}{2.496619in}}%
\pgfpathlineto{\pgfqpoint{2.516455in}{2.496619in}}%
\pgfpathlineto{\pgfqpoint{2.516455in}{2.493670in}}%
\pgfpathmoveto{\pgfqpoint{2.511914in}{2.496619in}}%
\pgfpathlineto{\pgfqpoint{2.511914in}{2.496619in}}%
\pgfpathlineto{\pgfqpoint{2.511914in}{2.499569in}}%
\pgfpathlineto{\pgfqpoint{2.516455in}{2.499569in}}%
\pgfpathlineto{\pgfqpoint{2.516455in}{2.496619in}}%
\pgfpathmoveto{\pgfqpoint{2.516455in}{2.493670in}}%
\pgfpathlineto{\pgfqpoint{2.516455in}{2.493670in}}%
\pgfpathlineto{\pgfqpoint{2.516455in}{2.496619in}}%
\pgfpathlineto{\pgfqpoint{2.520996in}{2.496619in}}%
\pgfpathlineto{\pgfqpoint{2.520996in}{2.493670in}}%
\pgfpathmoveto{\pgfqpoint{2.493750in}{2.508417in}}%
\pgfpathlineto{\pgfqpoint{2.493750in}{2.508417in}}%
\pgfpathlineto{\pgfqpoint{2.493750in}{2.511366in}}%
\pgfpathlineto{\pgfqpoint{2.498291in}{2.511366in}}%
\pgfpathlineto{\pgfqpoint{2.498291in}{2.508417in}}%
\pgfpathmoveto{\pgfqpoint{2.498291in}{2.505467in}}%
\pgfpathlineto{\pgfqpoint{2.498291in}{2.505467in}}%
\pgfpathlineto{\pgfqpoint{2.498291in}{2.508417in}}%
\pgfpathlineto{\pgfqpoint{2.502832in}{2.508417in}}%
\pgfpathlineto{\pgfqpoint{2.502832in}{2.505467in}}%
\pgfpathmoveto{\pgfqpoint{2.498291in}{2.508417in}}%
\pgfpathlineto{\pgfqpoint{2.498291in}{2.508417in}}%
\pgfpathlineto{\pgfqpoint{2.498291in}{2.511366in}}%
\pgfpathlineto{\pgfqpoint{2.502832in}{2.511366in}}%
\pgfpathlineto{\pgfqpoint{2.502832in}{2.508417in}}%
\pgfpathmoveto{\pgfqpoint{2.493750in}{2.511366in}}%
\pgfpathlineto{\pgfqpoint{2.493750in}{2.511366in}}%
\pgfpathlineto{\pgfqpoint{2.493750in}{2.514315in}}%
\pgfpathlineto{\pgfqpoint{2.498291in}{2.514315in}}%
\pgfpathlineto{\pgfqpoint{2.498291in}{2.511366in}}%
\pgfpathmoveto{\pgfqpoint{2.502832in}{2.505467in}}%
\pgfpathlineto{\pgfqpoint{2.502832in}{2.505467in}}%
\pgfpathlineto{\pgfqpoint{2.502832in}{2.508417in}}%
\pgfpathlineto{\pgfqpoint{2.507373in}{2.508417in}}%
\pgfpathlineto{\pgfqpoint{2.507373in}{2.505467in}}%
\pgfpathmoveto{\pgfqpoint{2.530078in}{2.481873in}}%
\pgfpathlineto{\pgfqpoint{2.530078in}{2.481873in}}%
\pgfpathlineto{\pgfqpoint{2.530078in}{2.484822in}}%
\pgfpathlineto{\pgfqpoint{2.534619in}{2.484822in}}%
\pgfpathlineto{\pgfqpoint{2.534619in}{2.481873in}}%
\pgfpathmoveto{\pgfqpoint{2.743505in}{2.290178in}}%
\pgfpathlineto{\pgfqpoint{2.743505in}{2.290178in}}%
\pgfpathlineto{\pgfqpoint{2.743505in}{2.293128in}}%
\pgfpathlineto{\pgfqpoint{2.748046in}{2.293128in}}%
\pgfpathlineto{\pgfqpoint{2.748046in}{2.290178in}}%
\pgfpathmoveto{\pgfqpoint{2.770751in}{2.266584in}}%
\pgfpathlineto{\pgfqpoint{2.770751in}{2.266584in}}%
\pgfpathlineto{\pgfqpoint{2.770751in}{2.269533in}}%
\pgfpathlineto{\pgfqpoint{2.775292in}{2.269533in}}%
\pgfpathlineto{\pgfqpoint{2.775292in}{2.266584in}}%
\pgfpathmoveto{\pgfqpoint{2.779833in}{2.260685in}}%
\pgfpathlineto{\pgfqpoint{2.779833in}{2.260685in}}%
\pgfpathlineto{\pgfqpoint{2.779833in}{2.263634in}}%
\pgfpathlineto{\pgfqpoint{2.784374in}{2.263634in}}%
\pgfpathlineto{\pgfqpoint{2.784374in}{2.260685in}}%
\pgfpathmoveto{\pgfqpoint{2.775292in}{2.263634in}}%
\pgfpathlineto{\pgfqpoint{2.775292in}{2.263634in}}%
\pgfpathlineto{\pgfqpoint{2.775292in}{2.266584in}}%
\pgfpathlineto{\pgfqpoint{2.779833in}{2.266584in}}%
\pgfpathlineto{\pgfqpoint{2.779833in}{2.263634in}}%
\pgfpathmoveto{\pgfqpoint{2.775292in}{2.266584in}}%
\pgfpathlineto{\pgfqpoint{2.775292in}{2.266584in}}%
\pgfpathlineto{\pgfqpoint{2.775292in}{2.269533in}}%
\pgfpathlineto{\pgfqpoint{2.779833in}{2.269533in}}%
\pgfpathlineto{\pgfqpoint{2.779833in}{2.266584in}}%
\pgfpathmoveto{\pgfqpoint{2.779833in}{2.263634in}}%
\pgfpathlineto{\pgfqpoint{2.779833in}{2.263634in}}%
\pgfpathlineto{\pgfqpoint{2.779833in}{2.266584in}}%
\pgfpathlineto{\pgfqpoint{2.784374in}{2.266584in}}%
\pgfpathlineto{\pgfqpoint{2.784374in}{2.263634in}}%
\pgfpathmoveto{\pgfqpoint{2.757128in}{2.278381in}}%
\pgfpathlineto{\pgfqpoint{2.757128in}{2.278381in}}%
\pgfpathlineto{\pgfqpoint{2.757128in}{2.281330in}}%
\pgfpathlineto{\pgfqpoint{2.761669in}{2.281330in}}%
\pgfpathlineto{\pgfqpoint{2.761669in}{2.278381in}}%
\pgfpathmoveto{\pgfqpoint{2.761669in}{2.275432in}}%
\pgfpathlineto{\pgfqpoint{2.761669in}{2.275432in}}%
\pgfpathlineto{\pgfqpoint{2.761669in}{2.278381in}}%
\pgfpathlineto{\pgfqpoint{2.766210in}{2.278381in}}%
\pgfpathlineto{\pgfqpoint{2.766210in}{2.275432in}}%
\pgfpathmoveto{\pgfqpoint{2.761669in}{2.278381in}}%
\pgfpathlineto{\pgfqpoint{2.761669in}{2.278381in}}%
\pgfpathlineto{\pgfqpoint{2.761669in}{2.281330in}}%
\pgfpathlineto{\pgfqpoint{2.766210in}{2.281330in}}%
\pgfpathlineto{\pgfqpoint{2.766210in}{2.278381in}}%
\pgfpathmoveto{\pgfqpoint{2.752587in}{2.284280in}}%
\pgfpathlineto{\pgfqpoint{2.752587in}{2.284280in}}%
\pgfpathlineto{\pgfqpoint{2.752587in}{2.287229in}}%
\pgfpathlineto{\pgfqpoint{2.757128in}{2.287229in}}%
\pgfpathlineto{\pgfqpoint{2.757128in}{2.284280in}}%
\pgfpathmoveto{\pgfqpoint{2.748046in}{2.287229in}}%
\pgfpathlineto{\pgfqpoint{2.748046in}{2.287229in}}%
\pgfpathlineto{\pgfqpoint{2.748046in}{2.290178in}}%
\pgfpathlineto{\pgfqpoint{2.752587in}{2.290178in}}%
\pgfpathlineto{\pgfqpoint{2.752587in}{2.287229in}}%
\pgfpathmoveto{\pgfqpoint{2.748046in}{2.290178in}}%
\pgfpathlineto{\pgfqpoint{2.748046in}{2.290178in}}%
\pgfpathlineto{\pgfqpoint{2.748046in}{2.293128in}}%
\pgfpathlineto{\pgfqpoint{2.752587in}{2.293128in}}%
\pgfpathlineto{\pgfqpoint{2.752587in}{2.290178in}}%
\pgfpathmoveto{\pgfqpoint{2.752587in}{2.287229in}}%
\pgfpathlineto{\pgfqpoint{2.752587in}{2.287229in}}%
\pgfpathlineto{\pgfqpoint{2.752587in}{2.290178in}}%
\pgfpathlineto{\pgfqpoint{2.757128in}{2.290178in}}%
\pgfpathlineto{\pgfqpoint{2.757128in}{2.287229in}}%
\pgfpathmoveto{\pgfqpoint{2.757128in}{2.281330in}}%
\pgfpathlineto{\pgfqpoint{2.757128in}{2.281330in}}%
\pgfpathlineto{\pgfqpoint{2.757128in}{2.284280in}}%
\pgfpathlineto{\pgfqpoint{2.761669in}{2.284280in}}%
\pgfpathlineto{\pgfqpoint{2.761669in}{2.281330in}}%
\pgfpathmoveto{\pgfqpoint{2.757128in}{2.284280in}}%
\pgfpathlineto{\pgfqpoint{2.757128in}{2.284280in}}%
\pgfpathlineto{\pgfqpoint{2.757128in}{2.287229in}}%
\pgfpathlineto{\pgfqpoint{2.761669in}{2.287229in}}%
\pgfpathlineto{\pgfqpoint{2.761669in}{2.284280in}}%
\pgfpathmoveto{\pgfqpoint{2.766210in}{2.272482in}}%
\pgfpathlineto{\pgfqpoint{2.766210in}{2.272482in}}%
\pgfpathlineto{\pgfqpoint{2.766210in}{2.275432in}}%
\pgfpathlineto{\pgfqpoint{2.770751in}{2.275432in}}%
\pgfpathlineto{\pgfqpoint{2.770751in}{2.272482in}}%
\pgfpathmoveto{\pgfqpoint{2.770751in}{2.269533in}}%
\pgfpathlineto{\pgfqpoint{2.770751in}{2.269533in}}%
\pgfpathlineto{\pgfqpoint{2.770751in}{2.272482in}}%
\pgfpathlineto{\pgfqpoint{2.775292in}{2.272482in}}%
\pgfpathlineto{\pgfqpoint{2.775292in}{2.269533in}}%
\pgfpathmoveto{\pgfqpoint{2.770751in}{2.272482in}}%
\pgfpathlineto{\pgfqpoint{2.770751in}{2.272482in}}%
\pgfpathlineto{\pgfqpoint{2.770751in}{2.275432in}}%
\pgfpathlineto{\pgfqpoint{2.775292in}{2.275432in}}%
\pgfpathlineto{\pgfqpoint{2.775292in}{2.272482in}}%
\pgfpathmoveto{\pgfqpoint{2.766210in}{2.275432in}}%
\pgfpathlineto{\pgfqpoint{2.766210in}{2.275432in}}%
\pgfpathlineto{\pgfqpoint{2.766210in}{2.278381in}}%
\pgfpathlineto{\pgfqpoint{2.770751in}{2.278381in}}%
\pgfpathlineto{\pgfqpoint{2.770751in}{2.275432in}}%
\pgfpathmoveto{\pgfqpoint{2.689014in}{2.337365in}}%
\pgfpathlineto{\pgfqpoint{2.689014in}{2.337365in}}%
\pgfpathlineto{\pgfqpoint{2.689014in}{2.340314in}}%
\pgfpathlineto{\pgfqpoint{2.693555in}{2.340314in}}%
\pgfpathlineto{\pgfqpoint{2.693555in}{2.337365in}}%
\pgfpathmoveto{\pgfqpoint{2.702637in}{2.325569in}}%
\pgfpathlineto{\pgfqpoint{2.702637in}{2.325569in}}%
\pgfpathlineto{\pgfqpoint{2.702637in}{2.328518in}}%
\pgfpathlineto{\pgfqpoint{2.707178in}{2.328518in}}%
\pgfpathlineto{\pgfqpoint{2.707178in}{2.325569in}}%
\pgfpathmoveto{\pgfqpoint{2.707178in}{2.322619in}}%
\pgfpathlineto{\pgfqpoint{2.707178in}{2.322619in}}%
\pgfpathlineto{\pgfqpoint{2.707178in}{2.325569in}}%
\pgfpathlineto{\pgfqpoint{2.711719in}{2.325569in}}%
\pgfpathlineto{\pgfqpoint{2.711719in}{2.322619in}}%
\pgfpathmoveto{\pgfqpoint{2.707178in}{2.325569in}}%
\pgfpathlineto{\pgfqpoint{2.707178in}{2.325569in}}%
\pgfpathlineto{\pgfqpoint{2.707178in}{2.328518in}}%
\pgfpathlineto{\pgfqpoint{2.711719in}{2.328518in}}%
\pgfpathlineto{\pgfqpoint{2.711719in}{2.325569in}}%
\pgfpathmoveto{\pgfqpoint{2.698096in}{2.331467in}}%
\pgfpathlineto{\pgfqpoint{2.698096in}{2.331467in}}%
\pgfpathlineto{\pgfqpoint{2.698096in}{2.334416in}}%
\pgfpathlineto{\pgfqpoint{2.702637in}{2.334416in}}%
\pgfpathlineto{\pgfqpoint{2.702637in}{2.331467in}}%
\pgfpathmoveto{\pgfqpoint{2.693555in}{2.334416in}}%
\pgfpathlineto{\pgfqpoint{2.693555in}{2.334416in}}%
\pgfpathlineto{\pgfqpoint{2.693555in}{2.337365in}}%
\pgfpathlineto{\pgfqpoint{2.698096in}{2.337365in}}%
\pgfpathlineto{\pgfqpoint{2.698096in}{2.334416in}}%
\pgfpathmoveto{\pgfqpoint{2.693555in}{2.337365in}}%
\pgfpathlineto{\pgfqpoint{2.693555in}{2.337365in}}%
\pgfpathlineto{\pgfqpoint{2.693555in}{2.340314in}}%
\pgfpathlineto{\pgfqpoint{2.698096in}{2.340314in}}%
\pgfpathlineto{\pgfqpoint{2.698096in}{2.337365in}}%
\pgfpathmoveto{\pgfqpoint{2.698096in}{2.334416in}}%
\pgfpathlineto{\pgfqpoint{2.698096in}{2.334416in}}%
\pgfpathlineto{\pgfqpoint{2.698096in}{2.337365in}}%
\pgfpathlineto{\pgfqpoint{2.702637in}{2.337365in}}%
\pgfpathlineto{\pgfqpoint{2.702637in}{2.334416in}}%
\pgfpathmoveto{\pgfqpoint{2.702637in}{2.328518in}}%
\pgfpathlineto{\pgfqpoint{2.702637in}{2.328518in}}%
\pgfpathlineto{\pgfqpoint{2.702637in}{2.331467in}}%
\pgfpathlineto{\pgfqpoint{2.707178in}{2.331467in}}%
\pgfpathlineto{\pgfqpoint{2.707178in}{2.328518in}}%
\pgfpathmoveto{\pgfqpoint{2.702637in}{2.331467in}}%
\pgfpathlineto{\pgfqpoint{2.702637in}{2.331467in}}%
\pgfpathlineto{\pgfqpoint{2.702637in}{2.334416in}}%
\pgfpathlineto{\pgfqpoint{2.707178in}{2.334416in}}%
\pgfpathlineto{\pgfqpoint{2.707178in}{2.331467in}}%
\pgfpathmoveto{\pgfqpoint{2.661768in}{2.360959in}}%
\pgfpathlineto{\pgfqpoint{2.661768in}{2.360959in}}%
\pgfpathlineto{\pgfqpoint{2.661768in}{2.363908in}}%
\pgfpathlineto{\pgfqpoint{2.666309in}{2.363908in}}%
\pgfpathlineto{\pgfqpoint{2.666309in}{2.360959in}}%
\pgfpathmoveto{\pgfqpoint{2.670850in}{2.355060in}}%
\pgfpathlineto{\pgfqpoint{2.670850in}{2.355060in}}%
\pgfpathlineto{\pgfqpoint{2.670850in}{2.358009in}}%
\pgfpathlineto{\pgfqpoint{2.675391in}{2.358009in}}%
\pgfpathlineto{\pgfqpoint{2.675391in}{2.355060in}}%
\pgfpathmoveto{\pgfqpoint{2.666309in}{2.358009in}}%
\pgfpathlineto{\pgfqpoint{2.666309in}{2.358009in}}%
\pgfpathlineto{\pgfqpoint{2.666309in}{2.360959in}}%
\pgfpathlineto{\pgfqpoint{2.670850in}{2.360959in}}%
\pgfpathlineto{\pgfqpoint{2.670850in}{2.358009in}}%
\pgfpathmoveto{\pgfqpoint{2.666309in}{2.360959in}}%
\pgfpathlineto{\pgfqpoint{2.666309in}{2.360959in}}%
\pgfpathlineto{\pgfqpoint{2.666309in}{2.363908in}}%
\pgfpathlineto{\pgfqpoint{2.670850in}{2.363908in}}%
\pgfpathlineto{\pgfqpoint{2.670850in}{2.360959in}}%
\pgfpathmoveto{\pgfqpoint{2.670850in}{2.358009in}}%
\pgfpathlineto{\pgfqpoint{2.670850in}{2.358009in}}%
\pgfpathlineto{\pgfqpoint{2.670850in}{2.360959in}}%
\pgfpathlineto{\pgfqpoint{2.675391in}{2.360959in}}%
\pgfpathlineto{\pgfqpoint{2.675391in}{2.358009in}}%
\pgfpathmoveto{\pgfqpoint{2.648145in}{2.372755in}}%
\pgfpathlineto{\pgfqpoint{2.648145in}{2.372755in}}%
\pgfpathlineto{\pgfqpoint{2.648145in}{2.375704in}}%
\pgfpathlineto{\pgfqpoint{2.652686in}{2.375704in}}%
\pgfpathlineto{\pgfqpoint{2.652686in}{2.372755in}}%
\pgfpathmoveto{\pgfqpoint{2.652686in}{2.369806in}}%
\pgfpathlineto{\pgfqpoint{2.652686in}{2.369806in}}%
\pgfpathlineto{\pgfqpoint{2.652686in}{2.372755in}}%
\pgfpathlineto{\pgfqpoint{2.657227in}{2.372755in}}%
\pgfpathlineto{\pgfqpoint{2.657227in}{2.369806in}}%
\pgfpathmoveto{\pgfqpoint{2.652686in}{2.372755in}}%
\pgfpathlineto{\pgfqpoint{2.652686in}{2.372755in}}%
\pgfpathlineto{\pgfqpoint{2.652686in}{2.375704in}}%
\pgfpathlineto{\pgfqpoint{2.657227in}{2.375704in}}%
\pgfpathlineto{\pgfqpoint{2.657227in}{2.372755in}}%
\pgfpathmoveto{\pgfqpoint{2.643604in}{2.378654in}}%
\pgfpathlineto{\pgfqpoint{2.643604in}{2.378654in}}%
\pgfpathlineto{\pgfqpoint{2.643604in}{2.381603in}}%
\pgfpathlineto{\pgfqpoint{2.648145in}{2.381603in}}%
\pgfpathlineto{\pgfqpoint{2.648145in}{2.378654in}}%
\pgfpathmoveto{\pgfqpoint{2.639063in}{2.381603in}}%
\pgfpathlineto{\pgfqpoint{2.639063in}{2.381603in}}%
\pgfpathlineto{\pgfqpoint{2.639063in}{2.384552in}}%
\pgfpathlineto{\pgfqpoint{2.643604in}{2.384552in}}%
\pgfpathlineto{\pgfqpoint{2.643604in}{2.381603in}}%
\pgfpathmoveto{\pgfqpoint{2.639063in}{2.384552in}}%
\pgfpathlineto{\pgfqpoint{2.639063in}{2.384552in}}%
\pgfpathlineto{\pgfqpoint{2.639063in}{2.387501in}}%
\pgfpathlineto{\pgfqpoint{2.643604in}{2.387501in}}%
\pgfpathlineto{\pgfqpoint{2.643604in}{2.384552in}}%
\pgfpathmoveto{\pgfqpoint{2.643604in}{2.381603in}}%
\pgfpathlineto{\pgfqpoint{2.643604in}{2.381603in}}%
\pgfpathlineto{\pgfqpoint{2.643604in}{2.384552in}}%
\pgfpathlineto{\pgfqpoint{2.648145in}{2.384552in}}%
\pgfpathlineto{\pgfqpoint{2.648145in}{2.381603in}}%
\pgfpathmoveto{\pgfqpoint{2.648145in}{2.375704in}}%
\pgfpathlineto{\pgfqpoint{2.648145in}{2.375704in}}%
\pgfpathlineto{\pgfqpoint{2.648145in}{2.378654in}}%
\pgfpathlineto{\pgfqpoint{2.652686in}{2.378654in}}%
\pgfpathlineto{\pgfqpoint{2.652686in}{2.375704in}}%
\pgfpathmoveto{\pgfqpoint{2.648145in}{2.378654in}}%
\pgfpathlineto{\pgfqpoint{2.648145in}{2.378654in}}%
\pgfpathlineto{\pgfqpoint{2.648145in}{2.381603in}}%
\pgfpathlineto{\pgfqpoint{2.652686in}{2.381603in}}%
\pgfpathlineto{\pgfqpoint{2.652686in}{2.378654in}}%
\pgfpathmoveto{\pgfqpoint{2.657227in}{2.366857in}}%
\pgfpathlineto{\pgfqpoint{2.657227in}{2.366857in}}%
\pgfpathlineto{\pgfqpoint{2.657227in}{2.369806in}}%
\pgfpathlineto{\pgfqpoint{2.661768in}{2.369806in}}%
\pgfpathlineto{\pgfqpoint{2.661768in}{2.366857in}}%
\pgfpathmoveto{\pgfqpoint{2.661768in}{2.363908in}}%
\pgfpathlineto{\pgfqpoint{2.661768in}{2.363908in}}%
\pgfpathlineto{\pgfqpoint{2.661768in}{2.366857in}}%
\pgfpathlineto{\pgfqpoint{2.666309in}{2.366857in}}%
\pgfpathlineto{\pgfqpoint{2.666309in}{2.363908in}}%
\pgfpathmoveto{\pgfqpoint{2.661768in}{2.366857in}}%
\pgfpathlineto{\pgfqpoint{2.661768in}{2.366857in}}%
\pgfpathlineto{\pgfqpoint{2.661768in}{2.369806in}}%
\pgfpathlineto{\pgfqpoint{2.666309in}{2.369806in}}%
\pgfpathlineto{\pgfqpoint{2.666309in}{2.366857in}}%
\pgfpathmoveto{\pgfqpoint{2.657227in}{2.369806in}}%
\pgfpathlineto{\pgfqpoint{2.657227in}{2.369806in}}%
\pgfpathlineto{\pgfqpoint{2.657227in}{2.372755in}}%
\pgfpathlineto{\pgfqpoint{2.661768in}{2.372755in}}%
\pgfpathlineto{\pgfqpoint{2.661768in}{2.369806in}}%
\pgfpathmoveto{\pgfqpoint{2.675391in}{2.349162in}}%
\pgfpathlineto{\pgfqpoint{2.675391in}{2.349162in}}%
\pgfpathlineto{\pgfqpoint{2.675391in}{2.352111in}}%
\pgfpathlineto{\pgfqpoint{2.679932in}{2.352111in}}%
\pgfpathlineto{\pgfqpoint{2.679932in}{2.349162in}}%
\pgfpathmoveto{\pgfqpoint{2.679932in}{2.346213in}}%
\pgfpathlineto{\pgfqpoint{2.679932in}{2.346213in}}%
\pgfpathlineto{\pgfqpoint{2.679932in}{2.349162in}}%
\pgfpathlineto{\pgfqpoint{2.684473in}{2.349162in}}%
\pgfpathlineto{\pgfqpoint{2.684473in}{2.346213in}}%
\pgfpathmoveto{\pgfqpoint{2.679932in}{2.349162in}}%
\pgfpathlineto{\pgfqpoint{2.679932in}{2.349162in}}%
\pgfpathlineto{\pgfqpoint{2.679932in}{2.352111in}}%
\pgfpathlineto{\pgfqpoint{2.684473in}{2.352111in}}%
\pgfpathlineto{\pgfqpoint{2.684473in}{2.349162in}}%
\pgfpathmoveto{\pgfqpoint{2.684473in}{2.343264in}}%
\pgfpathlineto{\pgfqpoint{2.684473in}{2.343264in}}%
\pgfpathlineto{\pgfqpoint{2.684473in}{2.346213in}}%
\pgfpathlineto{\pgfqpoint{2.689014in}{2.346213in}}%
\pgfpathlineto{\pgfqpoint{2.689014in}{2.343264in}}%
\pgfpathmoveto{\pgfqpoint{2.689014in}{2.340314in}}%
\pgfpathlineto{\pgfqpoint{2.689014in}{2.340314in}}%
\pgfpathlineto{\pgfqpoint{2.689014in}{2.343264in}}%
\pgfpathlineto{\pgfqpoint{2.693555in}{2.343264in}}%
\pgfpathlineto{\pgfqpoint{2.693555in}{2.340314in}}%
\pgfpathmoveto{\pgfqpoint{2.689014in}{2.343264in}}%
\pgfpathlineto{\pgfqpoint{2.689014in}{2.343264in}}%
\pgfpathlineto{\pgfqpoint{2.689014in}{2.346213in}}%
\pgfpathlineto{\pgfqpoint{2.693555in}{2.346213in}}%
\pgfpathlineto{\pgfqpoint{2.693555in}{2.343264in}}%
\pgfpathmoveto{\pgfqpoint{2.684473in}{2.346213in}}%
\pgfpathlineto{\pgfqpoint{2.684473in}{2.346213in}}%
\pgfpathlineto{\pgfqpoint{2.684473in}{2.349162in}}%
\pgfpathlineto{\pgfqpoint{2.689014in}{2.349162in}}%
\pgfpathlineto{\pgfqpoint{2.689014in}{2.346213in}}%
\pgfpathmoveto{\pgfqpoint{2.675391in}{2.352111in}}%
\pgfpathlineto{\pgfqpoint{2.675391in}{2.352111in}}%
\pgfpathlineto{\pgfqpoint{2.675391in}{2.355060in}}%
\pgfpathlineto{\pgfqpoint{2.679932in}{2.355060in}}%
\pgfpathlineto{\pgfqpoint{2.679932in}{2.352111in}}%
\pgfpathmoveto{\pgfqpoint{2.675391in}{2.355060in}}%
\pgfpathlineto{\pgfqpoint{2.675391in}{2.355060in}}%
\pgfpathlineto{\pgfqpoint{2.675391in}{2.358009in}}%
\pgfpathlineto{\pgfqpoint{2.679932in}{2.358009in}}%
\pgfpathlineto{\pgfqpoint{2.679932in}{2.355060in}}%
\pgfpathmoveto{\pgfqpoint{2.716260in}{2.313772in}}%
\pgfpathlineto{\pgfqpoint{2.716260in}{2.313772in}}%
\pgfpathlineto{\pgfqpoint{2.716260in}{2.316721in}}%
\pgfpathlineto{\pgfqpoint{2.720801in}{2.316721in}}%
\pgfpathlineto{\pgfqpoint{2.720801in}{2.313772in}}%
\pgfpathmoveto{\pgfqpoint{2.725342in}{2.307874in}}%
\pgfpathlineto{\pgfqpoint{2.725342in}{2.307874in}}%
\pgfpathlineto{\pgfqpoint{2.725342in}{2.310823in}}%
\pgfpathlineto{\pgfqpoint{2.729883in}{2.310823in}}%
\pgfpathlineto{\pgfqpoint{2.729883in}{2.307874in}}%
\pgfpathmoveto{\pgfqpoint{2.720801in}{2.310823in}}%
\pgfpathlineto{\pgfqpoint{2.720801in}{2.310823in}}%
\pgfpathlineto{\pgfqpoint{2.720801in}{2.313772in}}%
\pgfpathlineto{\pgfqpoint{2.725342in}{2.313772in}}%
\pgfpathlineto{\pgfqpoint{2.725342in}{2.310823in}}%
\pgfpathmoveto{\pgfqpoint{2.720801in}{2.313772in}}%
\pgfpathlineto{\pgfqpoint{2.720801in}{2.313772in}}%
\pgfpathlineto{\pgfqpoint{2.720801in}{2.316721in}}%
\pgfpathlineto{\pgfqpoint{2.725342in}{2.316721in}}%
\pgfpathlineto{\pgfqpoint{2.725342in}{2.313772in}}%
\pgfpathmoveto{\pgfqpoint{2.725342in}{2.310823in}}%
\pgfpathlineto{\pgfqpoint{2.725342in}{2.310823in}}%
\pgfpathlineto{\pgfqpoint{2.725342in}{2.313772in}}%
\pgfpathlineto{\pgfqpoint{2.729883in}{2.313772in}}%
\pgfpathlineto{\pgfqpoint{2.729883in}{2.310823in}}%
\pgfpathmoveto{\pgfqpoint{2.729883in}{2.301975in}}%
\pgfpathlineto{\pgfqpoint{2.729883in}{2.301975in}}%
\pgfpathlineto{\pgfqpoint{2.729883in}{2.304924in}}%
\pgfpathlineto{\pgfqpoint{2.734423in}{2.304924in}}%
\pgfpathlineto{\pgfqpoint{2.734423in}{2.301975in}}%
\pgfpathmoveto{\pgfqpoint{2.734423in}{2.299026in}}%
\pgfpathlineto{\pgfqpoint{2.734423in}{2.299026in}}%
\pgfpathlineto{\pgfqpoint{2.734423in}{2.301975in}}%
\pgfpathlineto{\pgfqpoint{2.738964in}{2.301975in}}%
\pgfpathlineto{\pgfqpoint{2.738964in}{2.299026in}}%
\pgfpathmoveto{\pgfqpoint{2.734423in}{2.301975in}}%
\pgfpathlineto{\pgfqpoint{2.734423in}{2.301975in}}%
\pgfpathlineto{\pgfqpoint{2.734423in}{2.304924in}}%
\pgfpathlineto{\pgfqpoint{2.738964in}{2.304924in}}%
\pgfpathlineto{\pgfqpoint{2.738964in}{2.301975in}}%
\pgfpathmoveto{\pgfqpoint{2.738964in}{2.296077in}}%
\pgfpathlineto{\pgfqpoint{2.738964in}{2.296077in}}%
\pgfpathlineto{\pgfqpoint{2.738964in}{2.299026in}}%
\pgfpathlineto{\pgfqpoint{2.743505in}{2.299026in}}%
\pgfpathlineto{\pgfqpoint{2.743505in}{2.296077in}}%
\pgfpathmoveto{\pgfqpoint{2.743505in}{2.293128in}}%
\pgfpathlineto{\pgfqpoint{2.743505in}{2.293128in}}%
\pgfpathlineto{\pgfqpoint{2.743505in}{2.296077in}}%
\pgfpathlineto{\pgfqpoint{2.748046in}{2.296077in}}%
\pgfpathlineto{\pgfqpoint{2.748046in}{2.293128in}}%
\pgfpathmoveto{\pgfqpoint{2.743505in}{2.296077in}}%
\pgfpathlineto{\pgfqpoint{2.743505in}{2.296077in}}%
\pgfpathlineto{\pgfqpoint{2.743505in}{2.299026in}}%
\pgfpathlineto{\pgfqpoint{2.748046in}{2.299026in}}%
\pgfpathlineto{\pgfqpoint{2.748046in}{2.296077in}}%
\pgfpathmoveto{\pgfqpoint{2.738964in}{2.299026in}}%
\pgfpathlineto{\pgfqpoint{2.738964in}{2.299026in}}%
\pgfpathlineto{\pgfqpoint{2.738964in}{2.301975in}}%
\pgfpathlineto{\pgfqpoint{2.743505in}{2.301975in}}%
\pgfpathlineto{\pgfqpoint{2.743505in}{2.299026in}}%
\pgfpathmoveto{\pgfqpoint{2.729883in}{2.304924in}}%
\pgfpathlineto{\pgfqpoint{2.729883in}{2.304924in}}%
\pgfpathlineto{\pgfqpoint{2.729883in}{2.307874in}}%
\pgfpathlineto{\pgfqpoint{2.734423in}{2.307874in}}%
\pgfpathlineto{\pgfqpoint{2.734423in}{2.304924in}}%
\pgfpathmoveto{\pgfqpoint{2.729883in}{2.307874in}}%
\pgfpathlineto{\pgfqpoint{2.729883in}{2.307874in}}%
\pgfpathlineto{\pgfqpoint{2.729883in}{2.310823in}}%
\pgfpathlineto{\pgfqpoint{2.734423in}{2.310823in}}%
\pgfpathlineto{\pgfqpoint{2.734423in}{2.307874in}}%
\pgfpathmoveto{\pgfqpoint{2.711719in}{2.319670in}}%
\pgfpathlineto{\pgfqpoint{2.711719in}{2.319670in}}%
\pgfpathlineto{\pgfqpoint{2.711719in}{2.322619in}}%
\pgfpathlineto{\pgfqpoint{2.716260in}{2.322619in}}%
\pgfpathlineto{\pgfqpoint{2.716260in}{2.319670in}}%
\pgfpathmoveto{\pgfqpoint{2.716260in}{2.316721in}}%
\pgfpathlineto{\pgfqpoint{2.716260in}{2.316721in}}%
\pgfpathlineto{\pgfqpoint{2.716260in}{2.319670in}}%
\pgfpathlineto{\pgfqpoint{2.720801in}{2.319670in}}%
\pgfpathlineto{\pgfqpoint{2.720801in}{2.316721in}}%
\pgfpathmoveto{\pgfqpoint{2.716260in}{2.319670in}}%
\pgfpathlineto{\pgfqpoint{2.716260in}{2.319670in}}%
\pgfpathlineto{\pgfqpoint{2.716260in}{2.322619in}}%
\pgfpathlineto{\pgfqpoint{2.720801in}{2.322619in}}%
\pgfpathlineto{\pgfqpoint{2.720801in}{2.319670in}}%
\pgfpathmoveto{\pgfqpoint{2.711719in}{2.322619in}}%
\pgfpathlineto{\pgfqpoint{2.711719in}{2.322619in}}%
\pgfpathlineto{\pgfqpoint{2.711719in}{2.325569in}}%
\pgfpathlineto{\pgfqpoint{2.716260in}{2.325569in}}%
\pgfpathlineto{\pgfqpoint{2.716260in}{2.322619in}}%
\pgfpathmoveto{\pgfqpoint{2.920606in}{2.136816in}}%
\pgfpathlineto{\pgfqpoint{2.920606in}{2.136816in}}%
\pgfpathlineto{\pgfqpoint{2.920606in}{2.139766in}}%
\pgfpathlineto{\pgfqpoint{2.925147in}{2.139766in}}%
\pgfpathlineto{\pgfqpoint{2.925147in}{2.136816in}}%
\pgfpathmoveto{\pgfqpoint{2.925147in}{2.133867in}}%
\pgfpathlineto{\pgfqpoint{2.925147in}{2.133867in}}%
\pgfpathlineto{\pgfqpoint{2.925147in}{2.136816in}}%
\pgfpathlineto{\pgfqpoint{2.929688in}{2.136816in}}%
\pgfpathlineto{\pgfqpoint{2.929688in}{2.133867in}}%
\pgfpathmoveto{\pgfqpoint{2.925147in}{2.136816in}}%
\pgfpathlineto{\pgfqpoint{2.925147in}{2.136816in}}%
\pgfpathlineto{\pgfqpoint{2.925147in}{2.139766in}}%
\pgfpathlineto{\pgfqpoint{2.929688in}{2.139766in}}%
\pgfpathlineto{\pgfqpoint{2.929688in}{2.136816in}}%
\pgfpathmoveto{\pgfqpoint{2.916065in}{2.142715in}}%
\pgfpathlineto{\pgfqpoint{2.916065in}{2.142715in}}%
\pgfpathlineto{\pgfqpoint{2.916065in}{2.145664in}}%
\pgfpathlineto{\pgfqpoint{2.920606in}{2.145664in}}%
\pgfpathlineto{\pgfqpoint{2.920606in}{2.142715in}}%
\pgfpathmoveto{\pgfqpoint{2.911524in}{2.145664in}}%
\pgfpathlineto{\pgfqpoint{2.911524in}{2.145664in}}%
\pgfpathlineto{\pgfqpoint{2.911524in}{2.148613in}}%
\pgfpathlineto{\pgfqpoint{2.916065in}{2.148613in}}%
\pgfpathlineto{\pgfqpoint{2.916065in}{2.145664in}}%
\pgfpathmoveto{\pgfqpoint{2.911524in}{2.148613in}}%
\pgfpathlineto{\pgfqpoint{2.911524in}{2.148613in}}%
\pgfpathlineto{\pgfqpoint{2.911524in}{2.151562in}}%
\pgfpathlineto{\pgfqpoint{2.916065in}{2.151562in}}%
\pgfpathlineto{\pgfqpoint{2.916065in}{2.148613in}}%
\pgfpathmoveto{\pgfqpoint{2.916065in}{2.145664in}}%
\pgfpathlineto{\pgfqpoint{2.916065in}{2.145664in}}%
\pgfpathlineto{\pgfqpoint{2.916065in}{2.148613in}}%
\pgfpathlineto{\pgfqpoint{2.920606in}{2.148613in}}%
\pgfpathlineto{\pgfqpoint{2.920606in}{2.145664in}}%
\pgfpathmoveto{\pgfqpoint{2.920606in}{2.139766in}}%
\pgfpathlineto{\pgfqpoint{2.920606in}{2.139766in}}%
\pgfpathlineto{\pgfqpoint{2.920606in}{2.142715in}}%
\pgfpathlineto{\pgfqpoint{2.925147in}{2.142715in}}%
\pgfpathlineto{\pgfqpoint{2.925147in}{2.139766in}}%
\pgfpathmoveto{\pgfqpoint{2.920606in}{2.142715in}}%
\pgfpathlineto{\pgfqpoint{2.920606in}{2.142715in}}%
\pgfpathlineto{\pgfqpoint{2.920606in}{2.145664in}}%
\pgfpathlineto{\pgfqpoint{2.925147in}{2.145664in}}%
\pgfpathlineto{\pgfqpoint{2.925147in}{2.142715in}}%
\pgfpathmoveto{\pgfqpoint{2.888818in}{2.166308in}}%
\pgfpathlineto{\pgfqpoint{2.888818in}{2.166308in}}%
\pgfpathlineto{\pgfqpoint{2.888818in}{2.169257in}}%
\pgfpathlineto{\pgfqpoint{2.893359in}{2.169257in}}%
\pgfpathlineto{\pgfqpoint{2.893359in}{2.166308in}}%
\pgfpathmoveto{\pgfqpoint{2.884277in}{2.169257in}}%
\pgfpathlineto{\pgfqpoint{2.884277in}{2.169257in}}%
\pgfpathlineto{\pgfqpoint{2.884277in}{2.172206in}}%
\pgfpathlineto{\pgfqpoint{2.888818in}{2.172206in}}%
\pgfpathlineto{\pgfqpoint{2.888818in}{2.169257in}}%
\pgfpathmoveto{\pgfqpoint{2.884277in}{2.172206in}}%
\pgfpathlineto{\pgfqpoint{2.884277in}{2.172206in}}%
\pgfpathlineto{\pgfqpoint{2.884277in}{2.175155in}}%
\pgfpathlineto{\pgfqpoint{2.888818in}{2.175155in}}%
\pgfpathlineto{\pgfqpoint{2.888818in}{2.172206in}}%
\pgfpathmoveto{\pgfqpoint{2.888818in}{2.169257in}}%
\pgfpathlineto{\pgfqpoint{2.888818in}{2.169257in}}%
\pgfpathlineto{\pgfqpoint{2.888818in}{2.172206in}}%
\pgfpathlineto{\pgfqpoint{2.893359in}{2.172206in}}%
\pgfpathlineto{\pgfqpoint{2.893359in}{2.169257in}}%
\pgfpathmoveto{\pgfqpoint{2.870654in}{2.181054in}}%
\pgfpathlineto{\pgfqpoint{2.870654in}{2.181054in}}%
\pgfpathlineto{\pgfqpoint{2.870654in}{2.184003in}}%
\pgfpathlineto{\pgfqpoint{2.875195in}{2.184003in}}%
\pgfpathlineto{\pgfqpoint{2.875195in}{2.181054in}}%
\pgfpathmoveto{\pgfqpoint{2.870654in}{2.184003in}}%
\pgfpathlineto{\pgfqpoint{2.870654in}{2.184003in}}%
\pgfpathlineto{\pgfqpoint{2.870654in}{2.186952in}}%
\pgfpathlineto{\pgfqpoint{2.875195in}{2.186952in}}%
\pgfpathlineto{\pgfqpoint{2.875195in}{2.184003in}}%
\pgfpathmoveto{\pgfqpoint{2.861572in}{2.189901in}}%
\pgfpathlineto{\pgfqpoint{2.861572in}{2.189901in}}%
\pgfpathlineto{\pgfqpoint{2.861572in}{2.192850in}}%
\pgfpathlineto{\pgfqpoint{2.866113in}{2.192850in}}%
\pgfpathlineto{\pgfqpoint{2.866113in}{2.189901in}}%
\pgfpathmoveto{\pgfqpoint{2.857031in}{2.192850in}}%
\pgfpathlineto{\pgfqpoint{2.857031in}{2.192850in}}%
\pgfpathlineto{\pgfqpoint{2.857031in}{2.195799in}}%
\pgfpathlineto{\pgfqpoint{2.861572in}{2.195799in}}%
\pgfpathlineto{\pgfqpoint{2.861572in}{2.192850in}}%
\pgfpathmoveto{\pgfqpoint{2.857031in}{2.195799in}}%
\pgfpathlineto{\pgfqpoint{2.857031in}{2.195799in}}%
\pgfpathlineto{\pgfqpoint{2.857031in}{2.198748in}}%
\pgfpathlineto{\pgfqpoint{2.861572in}{2.198748in}}%
\pgfpathlineto{\pgfqpoint{2.861572in}{2.195799in}}%
\pgfpathmoveto{\pgfqpoint{2.861572in}{2.192850in}}%
\pgfpathlineto{\pgfqpoint{2.861572in}{2.192850in}}%
\pgfpathlineto{\pgfqpoint{2.861572in}{2.195799in}}%
\pgfpathlineto{\pgfqpoint{2.866113in}{2.195799in}}%
\pgfpathlineto{\pgfqpoint{2.866113in}{2.192850in}}%
\pgfpathmoveto{\pgfqpoint{2.866113in}{2.186952in}}%
\pgfpathlineto{\pgfqpoint{2.866113in}{2.186952in}}%
\pgfpathlineto{\pgfqpoint{2.866113in}{2.189901in}}%
\pgfpathlineto{\pgfqpoint{2.870654in}{2.189901in}}%
\pgfpathlineto{\pgfqpoint{2.870654in}{2.186952in}}%
\pgfpathmoveto{\pgfqpoint{2.866113in}{2.189901in}}%
\pgfpathlineto{\pgfqpoint{2.866113in}{2.189901in}}%
\pgfpathlineto{\pgfqpoint{2.866113in}{2.192850in}}%
\pgfpathlineto{\pgfqpoint{2.870654in}{2.192850in}}%
\pgfpathlineto{\pgfqpoint{2.870654in}{2.189901in}}%
\pgfpathmoveto{\pgfqpoint{2.870654in}{2.186952in}}%
\pgfpathlineto{\pgfqpoint{2.870654in}{2.186952in}}%
\pgfpathlineto{\pgfqpoint{2.870654in}{2.189901in}}%
\pgfpathlineto{\pgfqpoint{2.875195in}{2.189901in}}%
\pgfpathlineto{\pgfqpoint{2.875195in}{2.186952in}}%
\pgfpathmoveto{\pgfqpoint{2.875195in}{2.178104in}}%
\pgfpathlineto{\pgfqpoint{2.875195in}{2.178104in}}%
\pgfpathlineto{\pgfqpoint{2.875195in}{2.181054in}}%
\pgfpathlineto{\pgfqpoint{2.879736in}{2.181054in}}%
\pgfpathlineto{\pgfqpoint{2.879736in}{2.178104in}}%
\pgfpathmoveto{\pgfqpoint{2.879736in}{2.175155in}}%
\pgfpathlineto{\pgfqpoint{2.879736in}{2.175155in}}%
\pgfpathlineto{\pgfqpoint{2.879736in}{2.178104in}}%
\pgfpathlineto{\pgfqpoint{2.884277in}{2.178104in}}%
\pgfpathlineto{\pgfqpoint{2.884277in}{2.175155in}}%
\pgfpathmoveto{\pgfqpoint{2.879736in}{2.178104in}}%
\pgfpathlineto{\pgfqpoint{2.879736in}{2.178104in}}%
\pgfpathlineto{\pgfqpoint{2.879736in}{2.181054in}}%
\pgfpathlineto{\pgfqpoint{2.884277in}{2.181054in}}%
\pgfpathlineto{\pgfqpoint{2.884277in}{2.178104in}}%
\pgfpathmoveto{\pgfqpoint{2.875195in}{2.181054in}}%
\pgfpathlineto{\pgfqpoint{2.875195in}{2.181054in}}%
\pgfpathlineto{\pgfqpoint{2.875195in}{2.184003in}}%
\pgfpathlineto{\pgfqpoint{2.879736in}{2.184003in}}%
\pgfpathlineto{\pgfqpoint{2.879736in}{2.181054in}}%
\pgfpathmoveto{\pgfqpoint{2.884277in}{2.175155in}}%
\pgfpathlineto{\pgfqpoint{2.884277in}{2.175155in}}%
\pgfpathlineto{\pgfqpoint{2.884277in}{2.178104in}}%
\pgfpathlineto{\pgfqpoint{2.888818in}{2.178104in}}%
\pgfpathlineto{\pgfqpoint{2.888818in}{2.175155in}}%
\pgfpathmoveto{\pgfqpoint{2.897900in}{2.157460in}}%
\pgfpathlineto{\pgfqpoint{2.897900in}{2.157460in}}%
\pgfpathlineto{\pgfqpoint{2.897900in}{2.160410in}}%
\pgfpathlineto{\pgfqpoint{2.902441in}{2.160410in}}%
\pgfpathlineto{\pgfqpoint{2.902441in}{2.157460in}}%
\pgfpathmoveto{\pgfqpoint{2.897900in}{2.160410in}}%
\pgfpathlineto{\pgfqpoint{2.897900in}{2.160410in}}%
\pgfpathlineto{\pgfqpoint{2.897900in}{2.163359in}}%
\pgfpathlineto{\pgfqpoint{2.902441in}{2.163359in}}%
\pgfpathlineto{\pgfqpoint{2.902441in}{2.160410in}}%
\pgfpathmoveto{\pgfqpoint{2.902441in}{2.154511in}}%
\pgfpathlineto{\pgfqpoint{2.902441in}{2.154511in}}%
\pgfpathlineto{\pgfqpoint{2.902441in}{2.157460in}}%
\pgfpathlineto{\pgfqpoint{2.906983in}{2.157460in}}%
\pgfpathlineto{\pgfqpoint{2.906983in}{2.154511in}}%
\pgfpathmoveto{\pgfqpoint{2.906983in}{2.151562in}}%
\pgfpathlineto{\pgfqpoint{2.906983in}{2.151562in}}%
\pgfpathlineto{\pgfqpoint{2.906983in}{2.154511in}}%
\pgfpathlineto{\pgfqpoint{2.911524in}{2.154511in}}%
\pgfpathlineto{\pgfqpoint{2.911524in}{2.151562in}}%
\pgfpathmoveto{\pgfqpoint{2.906983in}{2.154511in}}%
\pgfpathlineto{\pgfqpoint{2.906983in}{2.154511in}}%
\pgfpathlineto{\pgfqpoint{2.906983in}{2.157460in}}%
\pgfpathlineto{\pgfqpoint{2.911524in}{2.157460in}}%
\pgfpathlineto{\pgfqpoint{2.911524in}{2.154511in}}%
\pgfpathmoveto{\pgfqpoint{2.902441in}{2.157460in}}%
\pgfpathlineto{\pgfqpoint{2.902441in}{2.157460in}}%
\pgfpathlineto{\pgfqpoint{2.902441in}{2.160410in}}%
\pgfpathlineto{\pgfqpoint{2.906983in}{2.160410in}}%
\pgfpathlineto{\pgfqpoint{2.906983in}{2.157460in}}%
\pgfpathmoveto{\pgfqpoint{2.893359in}{2.163359in}}%
\pgfpathlineto{\pgfqpoint{2.893359in}{2.163359in}}%
\pgfpathlineto{\pgfqpoint{2.893359in}{2.166308in}}%
\pgfpathlineto{\pgfqpoint{2.897900in}{2.166308in}}%
\pgfpathlineto{\pgfqpoint{2.897900in}{2.163359in}}%
\pgfpathmoveto{\pgfqpoint{2.893359in}{2.166308in}}%
\pgfpathlineto{\pgfqpoint{2.893359in}{2.166308in}}%
\pgfpathlineto{\pgfqpoint{2.893359in}{2.169257in}}%
\pgfpathlineto{\pgfqpoint{2.897900in}{2.169257in}}%
\pgfpathlineto{\pgfqpoint{2.897900in}{2.166308in}}%
\pgfpathmoveto{\pgfqpoint{2.897900in}{2.163359in}}%
\pgfpathlineto{\pgfqpoint{2.897900in}{2.163359in}}%
\pgfpathlineto{\pgfqpoint{2.897900in}{2.166308in}}%
\pgfpathlineto{\pgfqpoint{2.902441in}{2.166308in}}%
\pgfpathlineto{\pgfqpoint{2.902441in}{2.163359in}}%
\pgfpathmoveto{\pgfqpoint{2.911524in}{2.151562in}}%
\pgfpathlineto{\pgfqpoint{2.911524in}{2.151562in}}%
\pgfpathlineto{\pgfqpoint{2.911524in}{2.154511in}}%
\pgfpathlineto{\pgfqpoint{2.916065in}{2.154511in}}%
\pgfpathlineto{\pgfqpoint{2.916065in}{2.151562in}}%
\pgfpathmoveto{\pgfqpoint{2.797997in}{2.242989in}}%
\pgfpathlineto{\pgfqpoint{2.797997in}{2.242989in}}%
\pgfpathlineto{\pgfqpoint{2.797997in}{2.245938in}}%
\pgfpathlineto{\pgfqpoint{2.802538in}{2.245938in}}%
\pgfpathlineto{\pgfqpoint{2.802538in}{2.242989in}}%
\pgfpathmoveto{\pgfqpoint{2.816162in}{2.228242in}}%
\pgfpathlineto{\pgfqpoint{2.816162in}{2.228242in}}%
\pgfpathlineto{\pgfqpoint{2.816162in}{2.231191in}}%
\pgfpathlineto{\pgfqpoint{2.820703in}{2.231191in}}%
\pgfpathlineto{\pgfqpoint{2.820703in}{2.228242in}}%
\pgfpathmoveto{\pgfqpoint{2.816162in}{2.231191in}}%
\pgfpathlineto{\pgfqpoint{2.816162in}{2.231191in}}%
\pgfpathlineto{\pgfqpoint{2.816162in}{2.234141in}}%
\pgfpathlineto{\pgfqpoint{2.820703in}{2.234141in}}%
\pgfpathlineto{\pgfqpoint{2.820703in}{2.231191in}}%
\pgfpathmoveto{\pgfqpoint{2.807080in}{2.237090in}}%
\pgfpathlineto{\pgfqpoint{2.807080in}{2.237090in}}%
\pgfpathlineto{\pgfqpoint{2.807080in}{2.240039in}}%
\pgfpathlineto{\pgfqpoint{2.811621in}{2.240039in}}%
\pgfpathlineto{\pgfqpoint{2.811621in}{2.237090in}}%
\pgfpathmoveto{\pgfqpoint{2.802538in}{2.240039in}}%
\pgfpathlineto{\pgfqpoint{2.802538in}{2.240039in}}%
\pgfpathlineto{\pgfqpoint{2.802538in}{2.242989in}}%
\pgfpathlineto{\pgfqpoint{2.807080in}{2.242989in}}%
\pgfpathlineto{\pgfqpoint{2.807080in}{2.240039in}}%
\pgfpathmoveto{\pgfqpoint{2.802538in}{2.242989in}}%
\pgfpathlineto{\pgfqpoint{2.802538in}{2.242989in}}%
\pgfpathlineto{\pgfqpoint{2.802538in}{2.245938in}}%
\pgfpathlineto{\pgfqpoint{2.807080in}{2.245938in}}%
\pgfpathlineto{\pgfqpoint{2.807080in}{2.242989in}}%
\pgfpathmoveto{\pgfqpoint{2.807080in}{2.240039in}}%
\pgfpathlineto{\pgfqpoint{2.807080in}{2.240039in}}%
\pgfpathlineto{\pgfqpoint{2.807080in}{2.242989in}}%
\pgfpathlineto{\pgfqpoint{2.811621in}{2.242989in}}%
\pgfpathlineto{\pgfqpoint{2.811621in}{2.240039in}}%
\pgfpathmoveto{\pgfqpoint{2.811621in}{2.234141in}}%
\pgfpathlineto{\pgfqpoint{2.811621in}{2.234141in}}%
\pgfpathlineto{\pgfqpoint{2.811621in}{2.237090in}}%
\pgfpathlineto{\pgfqpoint{2.816162in}{2.237090in}}%
\pgfpathlineto{\pgfqpoint{2.816162in}{2.234141in}}%
\pgfpathmoveto{\pgfqpoint{2.811621in}{2.237090in}}%
\pgfpathlineto{\pgfqpoint{2.811621in}{2.237090in}}%
\pgfpathlineto{\pgfqpoint{2.811621in}{2.240039in}}%
\pgfpathlineto{\pgfqpoint{2.816162in}{2.240039in}}%
\pgfpathlineto{\pgfqpoint{2.816162in}{2.237090in}}%
\pgfpathmoveto{\pgfqpoint{2.816162in}{2.234141in}}%
\pgfpathlineto{\pgfqpoint{2.816162in}{2.234141in}}%
\pgfpathlineto{\pgfqpoint{2.816162in}{2.237090in}}%
\pgfpathlineto{\pgfqpoint{2.820703in}{2.237090in}}%
\pgfpathlineto{\pgfqpoint{2.820703in}{2.234141in}}%
\pgfpathmoveto{\pgfqpoint{2.834326in}{2.213495in}}%
\pgfpathlineto{\pgfqpoint{2.834326in}{2.213495in}}%
\pgfpathlineto{\pgfqpoint{2.834326in}{2.216444in}}%
\pgfpathlineto{\pgfqpoint{2.838867in}{2.216444in}}%
\pgfpathlineto{\pgfqpoint{2.838867in}{2.213495in}}%
\pgfpathmoveto{\pgfqpoint{2.829785in}{2.216444in}}%
\pgfpathlineto{\pgfqpoint{2.829785in}{2.216444in}}%
\pgfpathlineto{\pgfqpoint{2.829785in}{2.219394in}}%
\pgfpathlineto{\pgfqpoint{2.834326in}{2.219394in}}%
\pgfpathlineto{\pgfqpoint{2.834326in}{2.216444in}}%
\pgfpathmoveto{\pgfqpoint{2.829785in}{2.219394in}}%
\pgfpathlineto{\pgfqpoint{2.829785in}{2.219394in}}%
\pgfpathlineto{\pgfqpoint{2.829785in}{2.222343in}}%
\pgfpathlineto{\pgfqpoint{2.834326in}{2.222343in}}%
\pgfpathlineto{\pgfqpoint{2.834326in}{2.219394in}}%
\pgfpathmoveto{\pgfqpoint{2.834326in}{2.216444in}}%
\pgfpathlineto{\pgfqpoint{2.834326in}{2.216444in}}%
\pgfpathlineto{\pgfqpoint{2.834326in}{2.219394in}}%
\pgfpathlineto{\pgfqpoint{2.838867in}{2.219394in}}%
\pgfpathlineto{\pgfqpoint{2.838867in}{2.216444in}}%
\pgfpathmoveto{\pgfqpoint{2.843408in}{2.204647in}}%
\pgfpathlineto{\pgfqpoint{2.843408in}{2.204647in}}%
\pgfpathlineto{\pgfqpoint{2.843408in}{2.207596in}}%
\pgfpathlineto{\pgfqpoint{2.847949in}{2.207596in}}%
\pgfpathlineto{\pgfqpoint{2.847949in}{2.204647in}}%
\pgfpathmoveto{\pgfqpoint{2.843408in}{2.207596in}}%
\pgfpathlineto{\pgfqpoint{2.843408in}{2.207596in}}%
\pgfpathlineto{\pgfqpoint{2.843408in}{2.210546in}}%
\pgfpathlineto{\pgfqpoint{2.847949in}{2.210546in}}%
\pgfpathlineto{\pgfqpoint{2.847949in}{2.207596in}}%
\pgfpathmoveto{\pgfqpoint{2.847949in}{2.201698in}}%
\pgfpathlineto{\pgfqpoint{2.847949in}{2.201698in}}%
\pgfpathlineto{\pgfqpoint{2.847949in}{2.204647in}}%
\pgfpathlineto{\pgfqpoint{2.852490in}{2.204647in}}%
\pgfpathlineto{\pgfqpoint{2.852490in}{2.201698in}}%
\pgfpathmoveto{\pgfqpoint{2.852490in}{2.198748in}}%
\pgfpathlineto{\pgfqpoint{2.852490in}{2.198748in}}%
\pgfpathlineto{\pgfqpoint{2.852490in}{2.201698in}}%
\pgfpathlineto{\pgfqpoint{2.857031in}{2.201698in}}%
\pgfpathlineto{\pgfqpoint{2.857031in}{2.198748in}}%
\pgfpathmoveto{\pgfqpoint{2.852490in}{2.201698in}}%
\pgfpathlineto{\pgfqpoint{2.852490in}{2.201698in}}%
\pgfpathlineto{\pgfqpoint{2.852490in}{2.204647in}}%
\pgfpathlineto{\pgfqpoint{2.857031in}{2.204647in}}%
\pgfpathlineto{\pgfqpoint{2.857031in}{2.201698in}}%
\pgfpathmoveto{\pgfqpoint{2.847949in}{2.204647in}}%
\pgfpathlineto{\pgfqpoint{2.847949in}{2.204647in}}%
\pgfpathlineto{\pgfqpoint{2.847949in}{2.207596in}}%
\pgfpathlineto{\pgfqpoint{2.852490in}{2.207596in}}%
\pgfpathlineto{\pgfqpoint{2.852490in}{2.204647in}}%
\pgfpathmoveto{\pgfqpoint{2.838867in}{2.210546in}}%
\pgfpathlineto{\pgfqpoint{2.838867in}{2.210546in}}%
\pgfpathlineto{\pgfqpoint{2.838867in}{2.213495in}}%
\pgfpathlineto{\pgfqpoint{2.843408in}{2.213495in}}%
\pgfpathlineto{\pgfqpoint{2.843408in}{2.210546in}}%
\pgfpathmoveto{\pgfqpoint{2.838867in}{2.213495in}}%
\pgfpathlineto{\pgfqpoint{2.838867in}{2.213495in}}%
\pgfpathlineto{\pgfqpoint{2.838867in}{2.216444in}}%
\pgfpathlineto{\pgfqpoint{2.843408in}{2.216444in}}%
\pgfpathlineto{\pgfqpoint{2.843408in}{2.213495in}}%
\pgfpathmoveto{\pgfqpoint{2.843408in}{2.210546in}}%
\pgfpathlineto{\pgfqpoint{2.843408in}{2.210546in}}%
\pgfpathlineto{\pgfqpoint{2.843408in}{2.213495in}}%
\pgfpathlineto{\pgfqpoint{2.847949in}{2.213495in}}%
\pgfpathlineto{\pgfqpoint{2.847949in}{2.210546in}}%
\pgfpathmoveto{\pgfqpoint{2.820703in}{2.225293in}}%
\pgfpathlineto{\pgfqpoint{2.820703in}{2.225293in}}%
\pgfpathlineto{\pgfqpoint{2.820703in}{2.228242in}}%
\pgfpathlineto{\pgfqpoint{2.825244in}{2.228242in}}%
\pgfpathlineto{\pgfqpoint{2.825244in}{2.225293in}}%
\pgfpathmoveto{\pgfqpoint{2.825244in}{2.222343in}}%
\pgfpathlineto{\pgfqpoint{2.825244in}{2.222343in}}%
\pgfpathlineto{\pgfqpoint{2.825244in}{2.225293in}}%
\pgfpathlineto{\pgfqpoint{2.829785in}{2.225293in}}%
\pgfpathlineto{\pgfqpoint{2.829785in}{2.222343in}}%
\pgfpathmoveto{\pgfqpoint{2.825244in}{2.225293in}}%
\pgfpathlineto{\pgfqpoint{2.825244in}{2.225293in}}%
\pgfpathlineto{\pgfqpoint{2.825244in}{2.228242in}}%
\pgfpathlineto{\pgfqpoint{2.829785in}{2.228242in}}%
\pgfpathlineto{\pgfqpoint{2.829785in}{2.225293in}}%
\pgfpathmoveto{\pgfqpoint{2.820703in}{2.228242in}}%
\pgfpathlineto{\pgfqpoint{2.820703in}{2.228242in}}%
\pgfpathlineto{\pgfqpoint{2.820703in}{2.231191in}}%
\pgfpathlineto{\pgfqpoint{2.825244in}{2.231191in}}%
\pgfpathlineto{\pgfqpoint{2.825244in}{2.228242in}}%
\pgfpathmoveto{\pgfqpoint{2.829785in}{2.222343in}}%
\pgfpathlineto{\pgfqpoint{2.829785in}{2.222343in}}%
\pgfpathlineto{\pgfqpoint{2.829785in}{2.225293in}}%
\pgfpathlineto{\pgfqpoint{2.834326in}{2.225293in}}%
\pgfpathlineto{\pgfqpoint{2.834326in}{2.222343in}}%
\pgfpathmoveto{\pgfqpoint{2.784374in}{2.254786in}}%
\pgfpathlineto{\pgfqpoint{2.784374in}{2.254786in}}%
\pgfpathlineto{\pgfqpoint{2.784374in}{2.257736in}}%
\pgfpathlineto{\pgfqpoint{2.788915in}{2.257736in}}%
\pgfpathlineto{\pgfqpoint{2.788915in}{2.254786in}}%
\pgfpathmoveto{\pgfqpoint{2.788915in}{2.251837in}}%
\pgfpathlineto{\pgfqpoint{2.788915in}{2.251837in}}%
\pgfpathlineto{\pgfqpoint{2.788915in}{2.254786in}}%
\pgfpathlineto{\pgfqpoint{2.793456in}{2.254786in}}%
\pgfpathlineto{\pgfqpoint{2.793456in}{2.251837in}}%
\pgfpathmoveto{\pgfqpoint{2.788915in}{2.254786in}}%
\pgfpathlineto{\pgfqpoint{2.788915in}{2.254786in}}%
\pgfpathlineto{\pgfqpoint{2.788915in}{2.257736in}}%
\pgfpathlineto{\pgfqpoint{2.793456in}{2.257736in}}%
\pgfpathlineto{\pgfqpoint{2.793456in}{2.254786in}}%
\pgfpathmoveto{\pgfqpoint{2.793456in}{2.248887in}}%
\pgfpathlineto{\pgfqpoint{2.793456in}{2.248887in}}%
\pgfpathlineto{\pgfqpoint{2.793456in}{2.251837in}}%
\pgfpathlineto{\pgfqpoint{2.797997in}{2.251837in}}%
\pgfpathlineto{\pgfqpoint{2.797997in}{2.248887in}}%
\pgfpathmoveto{\pgfqpoint{2.797997in}{2.245938in}}%
\pgfpathlineto{\pgfqpoint{2.797997in}{2.245938in}}%
\pgfpathlineto{\pgfqpoint{2.797997in}{2.248887in}}%
\pgfpathlineto{\pgfqpoint{2.802538in}{2.248887in}}%
\pgfpathlineto{\pgfqpoint{2.802538in}{2.245938in}}%
\pgfpathmoveto{\pgfqpoint{2.797997in}{2.248887in}}%
\pgfpathlineto{\pgfqpoint{2.797997in}{2.248887in}}%
\pgfpathlineto{\pgfqpoint{2.797997in}{2.251837in}}%
\pgfpathlineto{\pgfqpoint{2.802538in}{2.251837in}}%
\pgfpathlineto{\pgfqpoint{2.802538in}{2.248887in}}%
\pgfpathmoveto{\pgfqpoint{2.793456in}{2.251837in}}%
\pgfpathlineto{\pgfqpoint{2.793456in}{2.251837in}}%
\pgfpathlineto{\pgfqpoint{2.793456in}{2.254786in}}%
\pgfpathlineto{\pgfqpoint{2.797997in}{2.254786in}}%
\pgfpathlineto{\pgfqpoint{2.797997in}{2.251837in}}%
\pgfpathmoveto{\pgfqpoint{2.784374in}{2.257736in}}%
\pgfpathlineto{\pgfqpoint{2.784374in}{2.257736in}}%
\pgfpathlineto{\pgfqpoint{2.784374in}{2.260685in}}%
\pgfpathlineto{\pgfqpoint{2.788915in}{2.260685in}}%
\pgfpathlineto{\pgfqpoint{2.788915in}{2.257736in}}%
\pgfpathmoveto{\pgfqpoint{2.784374in}{2.260685in}}%
\pgfpathlineto{\pgfqpoint{2.784374in}{2.260685in}}%
\pgfpathlineto{\pgfqpoint{2.784374in}{2.263634in}}%
\pgfpathlineto{\pgfqpoint{2.788915in}{2.263634in}}%
\pgfpathlineto{\pgfqpoint{2.788915in}{2.260685in}}%
\pgfpathmoveto{\pgfqpoint{2.857031in}{2.198748in}}%
\pgfpathlineto{\pgfqpoint{2.857031in}{2.198748in}}%
\pgfpathlineto{\pgfqpoint{2.857031in}{2.201698in}}%
\pgfpathlineto{\pgfqpoint{2.861572in}{2.201698in}}%
\pgfpathlineto{\pgfqpoint{2.861572in}{2.198748in}}%
\pgfpathmoveto{\pgfqpoint{3.070461in}{2.007052in}}%
\pgfpathlineto{\pgfqpoint{3.070461in}{2.007052in}}%
\pgfpathlineto{\pgfqpoint{3.070461in}{2.010001in}}%
\pgfpathlineto{\pgfqpoint{3.075002in}{2.010001in}}%
\pgfpathlineto{\pgfqpoint{3.075002in}{2.007052in}}%
\pgfpathmoveto{\pgfqpoint{2.961475in}{2.101427in}}%
\pgfpathlineto{\pgfqpoint{2.961475in}{2.101427in}}%
\pgfpathlineto{\pgfqpoint{2.961475in}{2.104376in}}%
\pgfpathlineto{\pgfqpoint{2.966016in}{2.104376in}}%
\pgfpathlineto{\pgfqpoint{2.966016in}{2.101427in}}%
\pgfpathmoveto{\pgfqpoint{2.988722in}{2.077833in}}%
\pgfpathlineto{\pgfqpoint{2.988722in}{2.077833in}}%
\pgfpathlineto{\pgfqpoint{2.988722in}{2.080782in}}%
\pgfpathlineto{\pgfqpoint{2.993263in}{2.080782in}}%
\pgfpathlineto{\pgfqpoint{2.993263in}{2.077833in}}%
\pgfpathmoveto{\pgfqpoint{2.997804in}{2.071935in}}%
\pgfpathlineto{\pgfqpoint{2.997804in}{2.071935in}}%
\pgfpathlineto{\pgfqpoint{2.997804in}{2.074884in}}%
\pgfpathlineto{\pgfqpoint{3.002345in}{2.074884in}}%
\pgfpathlineto{\pgfqpoint{3.002345in}{2.071935in}}%
\pgfpathmoveto{\pgfqpoint{2.993263in}{2.074884in}}%
\pgfpathlineto{\pgfqpoint{2.993263in}{2.074884in}}%
\pgfpathlineto{\pgfqpoint{2.993263in}{2.077833in}}%
\pgfpathlineto{\pgfqpoint{2.997804in}{2.077833in}}%
\pgfpathlineto{\pgfqpoint{2.997804in}{2.074884in}}%
\pgfpathmoveto{\pgfqpoint{2.993263in}{2.077833in}}%
\pgfpathlineto{\pgfqpoint{2.993263in}{2.077833in}}%
\pgfpathlineto{\pgfqpoint{2.993263in}{2.080782in}}%
\pgfpathlineto{\pgfqpoint{2.997804in}{2.080782in}}%
\pgfpathlineto{\pgfqpoint{2.997804in}{2.077833in}}%
\pgfpathmoveto{\pgfqpoint{2.997804in}{2.074884in}}%
\pgfpathlineto{\pgfqpoint{2.997804in}{2.074884in}}%
\pgfpathlineto{\pgfqpoint{2.997804in}{2.077833in}}%
\pgfpathlineto{\pgfqpoint{3.002345in}{2.077833in}}%
\pgfpathlineto{\pgfqpoint{3.002345in}{2.074884in}}%
\pgfpathmoveto{\pgfqpoint{2.975098in}{2.089630in}}%
\pgfpathlineto{\pgfqpoint{2.975098in}{2.089630in}}%
\pgfpathlineto{\pgfqpoint{2.975098in}{2.092579in}}%
\pgfpathlineto{\pgfqpoint{2.979639in}{2.092579in}}%
\pgfpathlineto{\pgfqpoint{2.979639in}{2.089630in}}%
\pgfpathmoveto{\pgfqpoint{2.979639in}{2.086681in}}%
\pgfpathlineto{\pgfqpoint{2.979639in}{2.086681in}}%
\pgfpathlineto{\pgfqpoint{2.979639in}{2.089630in}}%
\pgfpathlineto{\pgfqpoint{2.984181in}{2.089630in}}%
\pgfpathlineto{\pgfqpoint{2.984181in}{2.086681in}}%
\pgfpathmoveto{\pgfqpoint{2.979639in}{2.089630in}}%
\pgfpathlineto{\pgfqpoint{2.979639in}{2.089630in}}%
\pgfpathlineto{\pgfqpoint{2.979639in}{2.092579in}}%
\pgfpathlineto{\pgfqpoint{2.984181in}{2.092579in}}%
\pgfpathlineto{\pgfqpoint{2.984181in}{2.089630in}}%
\pgfpathmoveto{\pgfqpoint{2.970557in}{2.095528in}}%
\pgfpathlineto{\pgfqpoint{2.970557in}{2.095528in}}%
\pgfpathlineto{\pgfqpoint{2.970557in}{2.098478in}}%
\pgfpathlineto{\pgfqpoint{2.975098in}{2.098478in}}%
\pgfpathlineto{\pgfqpoint{2.975098in}{2.095528in}}%
\pgfpathmoveto{\pgfqpoint{2.966016in}{2.098478in}}%
\pgfpathlineto{\pgfqpoint{2.966016in}{2.098478in}}%
\pgfpathlineto{\pgfqpoint{2.966016in}{2.101427in}}%
\pgfpathlineto{\pgfqpoint{2.970557in}{2.101427in}}%
\pgfpathlineto{\pgfqpoint{2.970557in}{2.098478in}}%
\pgfpathmoveto{\pgfqpoint{2.966016in}{2.101427in}}%
\pgfpathlineto{\pgfqpoint{2.966016in}{2.101427in}}%
\pgfpathlineto{\pgfqpoint{2.966016in}{2.104376in}}%
\pgfpathlineto{\pgfqpoint{2.970557in}{2.104376in}}%
\pgfpathlineto{\pgfqpoint{2.970557in}{2.101427in}}%
\pgfpathmoveto{\pgfqpoint{2.970557in}{2.098478in}}%
\pgfpathlineto{\pgfqpoint{2.970557in}{2.098478in}}%
\pgfpathlineto{\pgfqpoint{2.970557in}{2.101427in}}%
\pgfpathlineto{\pgfqpoint{2.975098in}{2.101427in}}%
\pgfpathlineto{\pgfqpoint{2.975098in}{2.098478in}}%
\pgfpathmoveto{\pgfqpoint{2.975098in}{2.092579in}}%
\pgfpathlineto{\pgfqpoint{2.975098in}{2.092579in}}%
\pgfpathlineto{\pgfqpoint{2.975098in}{2.095528in}}%
\pgfpathlineto{\pgfqpoint{2.979639in}{2.095528in}}%
\pgfpathlineto{\pgfqpoint{2.979639in}{2.092579in}}%
\pgfpathmoveto{\pgfqpoint{2.975098in}{2.095528in}}%
\pgfpathlineto{\pgfqpoint{2.975098in}{2.095528in}}%
\pgfpathlineto{\pgfqpoint{2.975098in}{2.098478in}}%
\pgfpathlineto{\pgfqpoint{2.979639in}{2.098478in}}%
\pgfpathlineto{\pgfqpoint{2.979639in}{2.095528in}}%
\pgfpathmoveto{\pgfqpoint{2.984181in}{2.083731in}}%
\pgfpathlineto{\pgfqpoint{2.984181in}{2.083731in}}%
\pgfpathlineto{\pgfqpoint{2.984181in}{2.086681in}}%
\pgfpathlineto{\pgfqpoint{2.988722in}{2.086681in}}%
\pgfpathlineto{\pgfqpoint{2.988722in}{2.083731in}}%
\pgfpathmoveto{\pgfqpoint{2.988722in}{2.080782in}}%
\pgfpathlineto{\pgfqpoint{2.988722in}{2.080782in}}%
\pgfpathlineto{\pgfqpoint{2.988722in}{2.083731in}}%
\pgfpathlineto{\pgfqpoint{2.993263in}{2.083731in}}%
\pgfpathlineto{\pgfqpoint{2.993263in}{2.080782in}}%
\pgfpathmoveto{\pgfqpoint{2.988722in}{2.083731in}}%
\pgfpathlineto{\pgfqpoint{2.988722in}{2.083731in}}%
\pgfpathlineto{\pgfqpoint{2.988722in}{2.086681in}}%
\pgfpathlineto{\pgfqpoint{2.993263in}{2.086681in}}%
\pgfpathlineto{\pgfqpoint{2.993263in}{2.083731in}}%
\pgfpathmoveto{\pgfqpoint{2.984181in}{2.086681in}}%
\pgfpathlineto{\pgfqpoint{2.984181in}{2.086681in}}%
\pgfpathlineto{\pgfqpoint{2.984181in}{2.089630in}}%
\pgfpathlineto{\pgfqpoint{2.988722in}{2.089630in}}%
\pgfpathlineto{\pgfqpoint{2.988722in}{2.086681in}}%
\pgfpathmoveto{\pgfqpoint{3.015968in}{2.054239in}}%
\pgfpathlineto{\pgfqpoint{3.015968in}{2.054239in}}%
\pgfpathlineto{\pgfqpoint{3.015968in}{2.057189in}}%
\pgfpathlineto{\pgfqpoint{3.020509in}{2.057189in}}%
\pgfpathlineto{\pgfqpoint{3.020509in}{2.054239in}}%
\pgfpathmoveto{\pgfqpoint{3.029591in}{2.042442in}}%
\pgfpathlineto{\pgfqpoint{3.029591in}{2.042442in}}%
\pgfpathlineto{\pgfqpoint{3.029591in}{2.045392in}}%
\pgfpathlineto{\pgfqpoint{3.034132in}{2.045392in}}%
\pgfpathlineto{\pgfqpoint{3.034132in}{2.042442in}}%
\pgfpathmoveto{\pgfqpoint{3.034132in}{2.039493in}}%
\pgfpathlineto{\pgfqpoint{3.034132in}{2.039493in}}%
\pgfpathlineto{\pgfqpoint{3.034132in}{2.042442in}}%
\pgfpathlineto{\pgfqpoint{3.038673in}{2.042442in}}%
\pgfpathlineto{\pgfqpoint{3.038673in}{2.039493in}}%
\pgfpathmoveto{\pgfqpoint{3.034132in}{2.042442in}}%
\pgfpathlineto{\pgfqpoint{3.034132in}{2.042442in}}%
\pgfpathlineto{\pgfqpoint{3.034132in}{2.045392in}}%
\pgfpathlineto{\pgfqpoint{3.038673in}{2.045392in}}%
\pgfpathlineto{\pgfqpoint{3.038673in}{2.042442in}}%
\pgfpathmoveto{\pgfqpoint{3.025050in}{2.048341in}}%
\pgfpathlineto{\pgfqpoint{3.025050in}{2.048341in}}%
\pgfpathlineto{\pgfqpoint{3.025050in}{2.051290in}}%
\pgfpathlineto{\pgfqpoint{3.029591in}{2.051290in}}%
\pgfpathlineto{\pgfqpoint{3.029591in}{2.048341in}}%
\pgfpathmoveto{\pgfqpoint{3.020509in}{2.051290in}}%
\pgfpathlineto{\pgfqpoint{3.020509in}{2.051290in}}%
\pgfpathlineto{\pgfqpoint{3.020509in}{2.054239in}}%
\pgfpathlineto{\pgfqpoint{3.025050in}{2.054239in}}%
\pgfpathlineto{\pgfqpoint{3.025050in}{2.051290in}}%
\pgfpathmoveto{\pgfqpoint{3.020509in}{2.054239in}}%
\pgfpathlineto{\pgfqpoint{3.020509in}{2.054239in}}%
\pgfpathlineto{\pgfqpoint{3.020509in}{2.057189in}}%
\pgfpathlineto{\pgfqpoint{3.025050in}{2.057189in}}%
\pgfpathlineto{\pgfqpoint{3.025050in}{2.054239in}}%
\pgfpathmoveto{\pgfqpoint{3.025050in}{2.051290in}}%
\pgfpathlineto{\pgfqpoint{3.025050in}{2.051290in}}%
\pgfpathlineto{\pgfqpoint{3.025050in}{2.054239in}}%
\pgfpathlineto{\pgfqpoint{3.029591in}{2.054239in}}%
\pgfpathlineto{\pgfqpoint{3.029591in}{2.051290in}}%
\pgfpathmoveto{\pgfqpoint{3.029591in}{2.045392in}}%
\pgfpathlineto{\pgfqpoint{3.029591in}{2.045392in}}%
\pgfpathlineto{\pgfqpoint{3.029591in}{2.048341in}}%
\pgfpathlineto{\pgfqpoint{3.034132in}{2.048341in}}%
\pgfpathlineto{\pgfqpoint{3.034132in}{2.045392in}}%
\pgfpathmoveto{\pgfqpoint{3.029591in}{2.048341in}}%
\pgfpathlineto{\pgfqpoint{3.029591in}{2.048341in}}%
\pgfpathlineto{\pgfqpoint{3.029591in}{2.051290in}}%
\pgfpathlineto{\pgfqpoint{3.034132in}{2.051290in}}%
\pgfpathlineto{\pgfqpoint{3.034132in}{2.048341in}}%
\pgfpathmoveto{\pgfqpoint{3.043214in}{2.030646in}}%
\pgfpathlineto{\pgfqpoint{3.043214in}{2.030646in}}%
\pgfpathlineto{\pgfqpoint{3.043214in}{2.033595in}}%
\pgfpathlineto{\pgfqpoint{3.047755in}{2.033595in}}%
\pgfpathlineto{\pgfqpoint{3.047755in}{2.030646in}}%
\pgfpathmoveto{\pgfqpoint{3.052297in}{2.024747in}}%
\pgfpathlineto{\pgfqpoint{3.052297in}{2.024747in}}%
\pgfpathlineto{\pgfqpoint{3.052297in}{2.027696in}}%
\pgfpathlineto{\pgfqpoint{3.056838in}{2.027696in}}%
\pgfpathlineto{\pgfqpoint{3.056838in}{2.024747in}}%
\pgfpathmoveto{\pgfqpoint{3.047755in}{2.027696in}}%
\pgfpathlineto{\pgfqpoint{3.047755in}{2.027696in}}%
\pgfpathlineto{\pgfqpoint{3.047755in}{2.030646in}}%
\pgfpathlineto{\pgfqpoint{3.052297in}{2.030646in}}%
\pgfpathlineto{\pgfqpoint{3.052297in}{2.027696in}}%
\pgfpathmoveto{\pgfqpoint{3.047755in}{2.030646in}}%
\pgfpathlineto{\pgfqpoint{3.047755in}{2.030646in}}%
\pgfpathlineto{\pgfqpoint{3.047755in}{2.033595in}}%
\pgfpathlineto{\pgfqpoint{3.052297in}{2.033595in}}%
\pgfpathlineto{\pgfqpoint{3.052297in}{2.030646in}}%
\pgfpathmoveto{\pgfqpoint{3.052297in}{2.027696in}}%
\pgfpathlineto{\pgfqpoint{3.052297in}{2.027696in}}%
\pgfpathlineto{\pgfqpoint{3.052297in}{2.030646in}}%
\pgfpathlineto{\pgfqpoint{3.056838in}{2.030646in}}%
\pgfpathlineto{\pgfqpoint{3.056838in}{2.027696in}}%
\pgfpathmoveto{\pgfqpoint{3.056838in}{2.018849in}}%
\pgfpathlineto{\pgfqpoint{3.056838in}{2.018849in}}%
\pgfpathlineto{\pgfqpoint{3.056838in}{2.021798in}}%
\pgfpathlineto{\pgfqpoint{3.061379in}{2.021798in}}%
\pgfpathlineto{\pgfqpoint{3.061379in}{2.018849in}}%
\pgfpathmoveto{\pgfqpoint{3.061379in}{2.015900in}}%
\pgfpathlineto{\pgfqpoint{3.061379in}{2.015900in}}%
\pgfpathlineto{\pgfqpoint{3.061379in}{2.018849in}}%
\pgfpathlineto{\pgfqpoint{3.065920in}{2.018849in}}%
\pgfpathlineto{\pgfqpoint{3.065920in}{2.015900in}}%
\pgfpathmoveto{\pgfqpoint{3.061379in}{2.018849in}}%
\pgfpathlineto{\pgfqpoint{3.061379in}{2.018849in}}%
\pgfpathlineto{\pgfqpoint{3.061379in}{2.021798in}}%
\pgfpathlineto{\pgfqpoint{3.065920in}{2.021798in}}%
\pgfpathlineto{\pgfqpoint{3.065920in}{2.018849in}}%
\pgfpathmoveto{\pgfqpoint{3.065920in}{2.012950in}}%
\pgfpathlineto{\pgfqpoint{3.065920in}{2.012950in}}%
\pgfpathlineto{\pgfqpoint{3.065920in}{2.015900in}}%
\pgfpathlineto{\pgfqpoint{3.070461in}{2.015900in}}%
\pgfpathlineto{\pgfqpoint{3.070461in}{2.012950in}}%
\pgfpathmoveto{\pgfqpoint{3.070461in}{2.010001in}}%
\pgfpathlineto{\pgfqpoint{3.070461in}{2.010001in}}%
\pgfpathlineto{\pgfqpoint{3.070461in}{2.012950in}}%
\pgfpathlineto{\pgfqpoint{3.075002in}{2.012950in}}%
\pgfpathlineto{\pgfqpoint{3.075002in}{2.010001in}}%
\pgfpathmoveto{\pgfqpoint{3.070461in}{2.012950in}}%
\pgfpathlineto{\pgfqpoint{3.070461in}{2.012950in}}%
\pgfpathlineto{\pgfqpoint{3.070461in}{2.015900in}}%
\pgfpathlineto{\pgfqpoint{3.075002in}{2.015900in}}%
\pgfpathlineto{\pgfqpoint{3.075002in}{2.012950in}}%
\pgfpathmoveto{\pgfqpoint{3.065920in}{2.015900in}}%
\pgfpathlineto{\pgfqpoint{3.065920in}{2.015900in}}%
\pgfpathlineto{\pgfqpoint{3.065920in}{2.018849in}}%
\pgfpathlineto{\pgfqpoint{3.070461in}{2.018849in}}%
\pgfpathlineto{\pgfqpoint{3.070461in}{2.015900in}}%
\pgfpathmoveto{\pgfqpoint{3.056838in}{2.021798in}}%
\pgfpathlineto{\pgfqpoint{3.056838in}{2.021798in}}%
\pgfpathlineto{\pgfqpoint{3.056838in}{2.024747in}}%
\pgfpathlineto{\pgfqpoint{3.061379in}{2.024747in}}%
\pgfpathlineto{\pgfqpoint{3.061379in}{2.021798in}}%
\pgfpathmoveto{\pgfqpoint{3.056838in}{2.024747in}}%
\pgfpathlineto{\pgfqpoint{3.056838in}{2.024747in}}%
\pgfpathlineto{\pgfqpoint{3.056838in}{2.027696in}}%
\pgfpathlineto{\pgfqpoint{3.061379in}{2.027696in}}%
\pgfpathlineto{\pgfqpoint{3.061379in}{2.024747in}}%
\pgfpathmoveto{\pgfqpoint{3.038673in}{2.036544in}}%
\pgfpathlineto{\pgfqpoint{3.038673in}{2.036544in}}%
\pgfpathlineto{\pgfqpoint{3.038673in}{2.039493in}}%
\pgfpathlineto{\pgfqpoint{3.043214in}{2.039493in}}%
\pgfpathlineto{\pgfqpoint{3.043214in}{2.036544in}}%
\pgfpathmoveto{\pgfqpoint{3.043214in}{2.033595in}}%
\pgfpathlineto{\pgfqpoint{3.043214in}{2.033595in}}%
\pgfpathlineto{\pgfqpoint{3.043214in}{2.036544in}}%
\pgfpathlineto{\pgfqpoint{3.047755in}{2.036544in}}%
\pgfpathlineto{\pgfqpoint{3.047755in}{2.033595in}}%
\pgfpathmoveto{\pgfqpoint{3.043214in}{2.036544in}}%
\pgfpathlineto{\pgfqpoint{3.043214in}{2.036544in}}%
\pgfpathlineto{\pgfqpoint{3.043214in}{2.039493in}}%
\pgfpathlineto{\pgfqpoint{3.047755in}{2.039493in}}%
\pgfpathlineto{\pgfqpoint{3.047755in}{2.036544in}}%
\pgfpathmoveto{\pgfqpoint{3.038673in}{2.039493in}}%
\pgfpathlineto{\pgfqpoint{3.038673in}{2.039493in}}%
\pgfpathlineto{\pgfqpoint{3.038673in}{2.042442in}}%
\pgfpathlineto{\pgfqpoint{3.043214in}{2.042442in}}%
\pgfpathlineto{\pgfqpoint{3.043214in}{2.039493in}}%
\pgfpathmoveto{\pgfqpoint{3.002345in}{2.066036in}}%
\pgfpathlineto{\pgfqpoint{3.002345in}{2.066036in}}%
\pgfpathlineto{\pgfqpoint{3.002345in}{2.068985in}}%
\pgfpathlineto{\pgfqpoint{3.006886in}{2.068985in}}%
\pgfpathlineto{\pgfqpoint{3.006886in}{2.066036in}}%
\pgfpathmoveto{\pgfqpoint{3.006886in}{2.063087in}}%
\pgfpathlineto{\pgfqpoint{3.006886in}{2.063087in}}%
\pgfpathlineto{\pgfqpoint{3.006886in}{2.066036in}}%
\pgfpathlineto{\pgfqpoint{3.011427in}{2.066036in}}%
\pgfpathlineto{\pgfqpoint{3.011427in}{2.063087in}}%
\pgfpathmoveto{\pgfqpoint{3.006886in}{2.066036in}}%
\pgfpathlineto{\pgfqpoint{3.006886in}{2.066036in}}%
\pgfpathlineto{\pgfqpoint{3.006886in}{2.068985in}}%
\pgfpathlineto{\pgfqpoint{3.011427in}{2.068985in}}%
\pgfpathlineto{\pgfqpoint{3.011427in}{2.066036in}}%
\pgfpathmoveto{\pgfqpoint{3.011427in}{2.060138in}}%
\pgfpathlineto{\pgfqpoint{3.011427in}{2.060138in}}%
\pgfpathlineto{\pgfqpoint{3.011427in}{2.063087in}}%
\pgfpathlineto{\pgfqpoint{3.015968in}{2.063087in}}%
\pgfpathlineto{\pgfqpoint{3.015968in}{2.060138in}}%
\pgfpathmoveto{\pgfqpoint{3.015968in}{2.057189in}}%
\pgfpathlineto{\pgfqpoint{3.015968in}{2.057189in}}%
\pgfpathlineto{\pgfqpoint{3.015968in}{2.060138in}}%
\pgfpathlineto{\pgfqpoint{3.020509in}{2.060138in}}%
\pgfpathlineto{\pgfqpoint{3.020509in}{2.057189in}}%
\pgfpathmoveto{\pgfqpoint{3.015968in}{2.060138in}}%
\pgfpathlineto{\pgfqpoint{3.015968in}{2.060138in}}%
\pgfpathlineto{\pgfqpoint{3.015968in}{2.063087in}}%
\pgfpathlineto{\pgfqpoint{3.020509in}{2.063087in}}%
\pgfpathlineto{\pgfqpoint{3.020509in}{2.060138in}}%
\pgfpathmoveto{\pgfqpoint{3.011427in}{2.063087in}}%
\pgfpathlineto{\pgfqpoint{3.011427in}{2.063087in}}%
\pgfpathlineto{\pgfqpoint{3.011427in}{2.066036in}}%
\pgfpathlineto{\pgfqpoint{3.015968in}{2.066036in}}%
\pgfpathlineto{\pgfqpoint{3.015968in}{2.063087in}}%
\pgfpathmoveto{\pgfqpoint{3.002345in}{2.068985in}}%
\pgfpathlineto{\pgfqpoint{3.002345in}{2.068985in}}%
\pgfpathlineto{\pgfqpoint{3.002345in}{2.071935in}}%
\pgfpathlineto{\pgfqpoint{3.006886in}{2.071935in}}%
\pgfpathlineto{\pgfqpoint{3.006886in}{2.068985in}}%
\pgfpathmoveto{\pgfqpoint{3.002345in}{2.071935in}}%
\pgfpathlineto{\pgfqpoint{3.002345in}{2.071935in}}%
\pgfpathlineto{\pgfqpoint{3.002345in}{2.074884in}}%
\pgfpathlineto{\pgfqpoint{3.006886in}{2.074884in}}%
\pgfpathlineto{\pgfqpoint{3.006886in}{2.071935in}}%
\pgfpathmoveto{\pgfqpoint{2.934229in}{2.125020in}}%
\pgfpathlineto{\pgfqpoint{2.934229in}{2.125020in}}%
\pgfpathlineto{\pgfqpoint{2.934229in}{2.127969in}}%
\pgfpathlineto{\pgfqpoint{2.938770in}{2.127969in}}%
\pgfpathlineto{\pgfqpoint{2.938770in}{2.125020in}}%
\pgfpathmoveto{\pgfqpoint{2.943311in}{2.119122in}}%
\pgfpathlineto{\pgfqpoint{2.943311in}{2.119122in}}%
\pgfpathlineto{\pgfqpoint{2.943311in}{2.122071in}}%
\pgfpathlineto{\pgfqpoint{2.947852in}{2.122071in}}%
\pgfpathlineto{\pgfqpoint{2.947852in}{2.119122in}}%
\pgfpathmoveto{\pgfqpoint{2.938770in}{2.122071in}}%
\pgfpathlineto{\pgfqpoint{2.938770in}{2.122071in}}%
\pgfpathlineto{\pgfqpoint{2.938770in}{2.125020in}}%
\pgfpathlineto{\pgfqpoint{2.943311in}{2.125020in}}%
\pgfpathlineto{\pgfqpoint{2.943311in}{2.122071in}}%
\pgfpathmoveto{\pgfqpoint{2.938770in}{2.125020in}}%
\pgfpathlineto{\pgfqpoint{2.938770in}{2.125020in}}%
\pgfpathlineto{\pgfqpoint{2.938770in}{2.127969in}}%
\pgfpathlineto{\pgfqpoint{2.943311in}{2.127969in}}%
\pgfpathlineto{\pgfqpoint{2.943311in}{2.125020in}}%
\pgfpathmoveto{\pgfqpoint{2.943311in}{2.122071in}}%
\pgfpathlineto{\pgfqpoint{2.943311in}{2.122071in}}%
\pgfpathlineto{\pgfqpoint{2.943311in}{2.125020in}}%
\pgfpathlineto{\pgfqpoint{2.947852in}{2.125020in}}%
\pgfpathlineto{\pgfqpoint{2.947852in}{2.122071in}}%
\pgfpathmoveto{\pgfqpoint{2.947852in}{2.113223in}}%
\pgfpathlineto{\pgfqpoint{2.947852in}{2.113223in}}%
\pgfpathlineto{\pgfqpoint{2.947852in}{2.116173in}}%
\pgfpathlineto{\pgfqpoint{2.952393in}{2.116173in}}%
\pgfpathlineto{\pgfqpoint{2.952393in}{2.113223in}}%
\pgfpathmoveto{\pgfqpoint{2.952393in}{2.110274in}}%
\pgfpathlineto{\pgfqpoint{2.952393in}{2.110274in}}%
\pgfpathlineto{\pgfqpoint{2.952393in}{2.113223in}}%
\pgfpathlineto{\pgfqpoint{2.956934in}{2.113223in}}%
\pgfpathlineto{\pgfqpoint{2.956934in}{2.110274in}}%
\pgfpathmoveto{\pgfqpoint{2.952393in}{2.113223in}}%
\pgfpathlineto{\pgfqpoint{2.952393in}{2.113223in}}%
\pgfpathlineto{\pgfqpoint{2.952393in}{2.116173in}}%
\pgfpathlineto{\pgfqpoint{2.956934in}{2.116173in}}%
\pgfpathlineto{\pgfqpoint{2.956934in}{2.113223in}}%
\pgfpathmoveto{\pgfqpoint{2.956934in}{2.107325in}}%
\pgfpathlineto{\pgfqpoint{2.956934in}{2.107325in}}%
\pgfpathlineto{\pgfqpoint{2.956934in}{2.110274in}}%
\pgfpathlineto{\pgfqpoint{2.961475in}{2.110274in}}%
\pgfpathlineto{\pgfqpoint{2.961475in}{2.107325in}}%
\pgfpathmoveto{\pgfqpoint{2.961475in}{2.104376in}}%
\pgfpathlineto{\pgfqpoint{2.961475in}{2.104376in}}%
\pgfpathlineto{\pgfqpoint{2.961475in}{2.107325in}}%
\pgfpathlineto{\pgfqpoint{2.966016in}{2.107325in}}%
\pgfpathlineto{\pgfqpoint{2.966016in}{2.104376in}}%
\pgfpathmoveto{\pgfqpoint{2.961475in}{2.107325in}}%
\pgfpathlineto{\pgfqpoint{2.961475in}{2.107325in}}%
\pgfpathlineto{\pgfqpoint{2.961475in}{2.110274in}}%
\pgfpathlineto{\pgfqpoint{2.966016in}{2.110274in}}%
\pgfpathlineto{\pgfqpoint{2.966016in}{2.107325in}}%
\pgfpathmoveto{\pgfqpoint{2.956934in}{2.110274in}}%
\pgfpathlineto{\pgfqpoint{2.956934in}{2.110274in}}%
\pgfpathlineto{\pgfqpoint{2.956934in}{2.113223in}}%
\pgfpathlineto{\pgfqpoint{2.961475in}{2.113223in}}%
\pgfpathlineto{\pgfqpoint{2.961475in}{2.110274in}}%
\pgfpathmoveto{\pgfqpoint{2.947852in}{2.116173in}}%
\pgfpathlineto{\pgfqpoint{2.947852in}{2.116173in}}%
\pgfpathlineto{\pgfqpoint{2.947852in}{2.119122in}}%
\pgfpathlineto{\pgfqpoint{2.952393in}{2.119122in}}%
\pgfpathlineto{\pgfqpoint{2.952393in}{2.116173in}}%
\pgfpathmoveto{\pgfqpoint{2.947852in}{2.119122in}}%
\pgfpathlineto{\pgfqpoint{2.947852in}{2.119122in}}%
\pgfpathlineto{\pgfqpoint{2.947852in}{2.122071in}}%
\pgfpathlineto{\pgfqpoint{2.952393in}{2.122071in}}%
\pgfpathlineto{\pgfqpoint{2.952393in}{2.119122in}}%
\pgfpathmoveto{\pgfqpoint{2.929688in}{2.130918in}}%
\pgfpathlineto{\pgfqpoint{2.929688in}{2.130918in}}%
\pgfpathlineto{\pgfqpoint{2.929688in}{2.133867in}}%
\pgfpathlineto{\pgfqpoint{2.934229in}{2.133867in}}%
\pgfpathlineto{\pgfqpoint{2.934229in}{2.130918in}}%
\pgfpathmoveto{\pgfqpoint{2.934229in}{2.127969in}}%
\pgfpathlineto{\pgfqpoint{2.934229in}{2.127969in}}%
\pgfpathlineto{\pgfqpoint{2.934229in}{2.130918in}}%
\pgfpathlineto{\pgfqpoint{2.938770in}{2.130918in}}%
\pgfpathlineto{\pgfqpoint{2.938770in}{2.127969in}}%
\pgfpathmoveto{\pgfqpoint{2.934229in}{2.130918in}}%
\pgfpathlineto{\pgfqpoint{2.934229in}{2.130918in}}%
\pgfpathlineto{\pgfqpoint{2.934229in}{2.133867in}}%
\pgfpathlineto{\pgfqpoint{2.938770in}{2.133867in}}%
\pgfpathlineto{\pgfqpoint{2.938770in}{2.130918in}}%
\pgfpathmoveto{\pgfqpoint{2.929688in}{2.133867in}}%
\pgfpathlineto{\pgfqpoint{2.929688in}{2.133867in}}%
\pgfpathlineto{\pgfqpoint{2.929688in}{2.136816in}}%
\pgfpathlineto{\pgfqpoint{2.934229in}{2.136816in}}%
\pgfpathlineto{\pgfqpoint{2.934229in}{2.133867in}}%
\pgfpathmoveto{\pgfqpoint{3.179447in}{1.912673in}}%
\pgfpathlineto{\pgfqpoint{3.179447in}{1.912673in}}%
\pgfpathlineto{\pgfqpoint{3.179447in}{1.915622in}}%
\pgfpathlineto{\pgfqpoint{3.183988in}{1.915622in}}%
\pgfpathlineto{\pgfqpoint{3.183988in}{1.912673in}}%
\pgfpathmoveto{\pgfqpoint{3.206694in}{1.889080in}}%
\pgfpathlineto{\pgfqpoint{3.206694in}{1.889080in}}%
\pgfpathlineto{\pgfqpoint{3.206694in}{1.892030in}}%
\pgfpathlineto{\pgfqpoint{3.211235in}{1.892030in}}%
\pgfpathlineto{\pgfqpoint{3.211235in}{1.889080in}}%
\pgfpathmoveto{\pgfqpoint{3.215776in}{1.883182in}}%
\pgfpathlineto{\pgfqpoint{3.215776in}{1.883182in}}%
\pgfpathlineto{\pgfqpoint{3.215776in}{1.886131in}}%
\pgfpathlineto{\pgfqpoint{3.220317in}{1.886131in}}%
\pgfpathlineto{\pgfqpoint{3.220317in}{1.883182in}}%
\pgfpathmoveto{\pgfqpoint{3.211235in}{1.886131in}}%
\pgfpathlineto{\pgfqpoint{3.211235in}{1.886131in}}%
\pgfpathlineto{\pgfqpoint{3.211235in}{1.889080in}}%
\pgfpathlineto{\pgfqpoint{3.215776in}{1.889080in}}%
\pgfpathlineto{\pgfqpoint{3.215776in}{1.886131in}}%
\pgfpathmoveto{\pgfqpoint{3.211235in}{1.889080in}}%
\pgfpathlineto{\pgfqpoint{3.211235in}{1.889080in}}%
\pgfpathlineto{\pgfqpoint{3.211235in}{1.892030in}}%
\pgfpathlineto{\pgfqpoint{3.215776in}{1.892030in}}%
\pgfpathlineto{\pgfqpoint{3.215776in}{1.889080in}}%
\pgfpathmoveto{\pgfqpoint{3.215776in}{1.886131in}}%
\pgfpathlineto{\pgfqpoint{3.215776in}{1.886131in}}%
\pgfpathlineto{\pgfqpoint{3.215776in}{1.889080in}}%
\pgfpathlineto{\pgfqpoint{3.220317in}{1.889080in}}%
\pgfpathlineto{\pgfqpoint{3.220317in}{1.886131in}}%
\pgfpathmoveto{\pgfqpoint{3.193070in}{1.900877in}}%
\pgfpathlineto{\pgfqpoint{3.193070in}{1.900877in}}%
\pgfpathlineto{\pgfqpoint{3.193070in}{1.903826in}}%
\pgfpathlineto{\pgfqpoint{3.197611in}{1.903826in}}%
\pgfpathlineto{\pgfqpoint{3.197611in}{1.900877in}}%
\pgfpathmoveto{\pgfqpoint{3.197611in}{1.897928in}}%
\pgfpathlineto{\pgfqpoint{3.197611in}{1.897928in}}%
\pgfpathlineto{\pgfqpoint{3.197611in}{1.900877in}}%
\pgfpathlineto{\pgfqpoint{3.202153in}{1.900877in}}%
\pgfpathlineto{\pgfqpoint{3.202153in}{1.897928in}}%
\pgfpathmoveto{\pgfqpoint{3.197611in}{1.900877in}}%
\pgfpathlineto{\pgfqpoint{3.197611in}{1.900877in}}%
\pgfpathlineto{\pgfqpoint{3.197611in}{1.903826in}}%
\pgfpathlineto{\pgfqpoint{3.202153in}{1.903826in}}%
\pgfpathlineto{\pgfqpoint{3.202153in}{1.900877in}}%
\pgfpathmoveto{\pgfqpoint{3.188529in}{1.906775in}}%
\pgfpathlineto{\pgfqpoint{3.188529in}{1.906775in}}%
\pgfpathlineto{\pgfqpoint{3.188529in}{1.909724in}}%
\pgfpathlineto{\pgfqpoint{3.193070in}{1.909724in}}%
\pgfpathlineto{\pgfqpoint{3.193070in}{1.906775in}}%
\pgfpathmoveto{\pgfqpoint{3.183988in}{1.909724in}}%
\pgfpathlineto{\pgfqpoint{3.183988in}{1.909724in}}%
\pgfpathlineto{\pgfqpoint{3.183988in}{1.912673in}}%
\pgfpathlineto{\pgfqpoint{3.188529in}{1.912673in}}%
\pgfpathlineto{\pgfqpoint{3.188529in}{1.909724in}}%
\pgfpathmoveto{\pgfqpoint{3.183988in}{1.912673in}}%
\pgfpathlineto{\pgfqpoint{3.183988in}{1.912673in}}%
\pgfpathlineto{\pgfqpoint{3.183988in}{1.915622in}}%
\pgfpathlineto{\pgfqpoint{3.188529in}{1.915622in}}%
\pgfpathlineto{\pgfqpoint{3.188529in}{1.912673in}}%
\pgfpathmoveto{\pgfqpoint{3.188529in}{1.909724in}}%
\pgfpathlineto{\pgfqpoint{3.188529in}{1.909724in}}%
\pgfpathlineto{\pgfqpoint{3.188529in}{1.912673in}}%
\pgfpathlineto{\pgfqpoint{3.193070in}{1.912673in}}%
\pgfpathlineto{\pgfqpoint{3.193070in}{1.909724in}}%
\pgfpathmoveto{\pgfqpoint{3.193070in}{1.903826in}}%
\pgfpathlineto{\pgfqpoint{3.193070in}{1.903826in}}%
\pgfpathlineto{\pgfqpoint{3.193070in}{1.906775in}}%
\pgfpathlineto{\pgfqpoint{3.197611in}{1.906775in}}%
\pgfpathlineto{\pgfqpoint{3.197611in}{1.903826in}}%
\pgfpathmoveto{\pgfqpoint{3.193070in}{1.906775in}}%
\pgfpathlineto{\pgfqpoint{3.193070in}{1.906775in}}%
\pgfpathlineto{\pgfqpoint{3.193070in}{1.909724in}}%
\pgfpathlineto{\pgfqpoint{3.197611in}{1.909724in}}%
\pgfpathlineto{\pgfqpoint{3.197611in}{1.906775in}}%
\pgfpathmoveto{\pgfqpoint{3.202153in}{1.894979in}}%
\pgfpathlineto{\pgfqpoint{3.202153in}{1.894979in}}%
\pgfpathlineto{\pgfqpoint{3.202153in}{1.897928in}}%
\pgfpathlineto{\pgfqpoint{3.206694in}{1.897928in}}%
\pgfpathlineto{\pgfqpoint{3.206694in}{1.894979in}}%
\pgfpathmoveto{\pgfqpoint{3.206694in}{1.892030in}}%
\pgfpathlineto{\pgfqpoint{3.206694in}{1.892030in}}%
\pgfpathlineto{\pgfqpoint{3.206694in}{1.894979in}}%
\pgfpathlineto{\pgfqpoint{3.211235in}{1.894979in}}%
\pgfpathlineto{\pgfqpoint{3.211235in}{1.892030in}}%
\pgfpathmoveto{\pgfqpoint{3.206694in}{1.894979in}}%
\pgfpathlineto{\pgfqpoint{3.206694in}{1.894979in}}%
\pgfpathlineto{\pgfqpoint{3.206694in}{1.897928in}}%
\pgfpathlineto{\pgfqpoint{3.211235in}{1.897928in}}%
\pgfpathlineto{\pgfqpoint{3.211235in}{1.894979in}}%
\pgfpathmoveto{\pgfqpoint{3.202153in}{1.897928in}}%
\pgfpathlineto{\pgfqpoint{3.202153in}{1.897928in}}%
\pgfpathlineto{\pgfqpoint{3.202153in}{1.900877in}}%
\pgfpathlineto{\pgfqpoint{3.206694in}{1.900877in}}%
\pgfpathlineto{\pgfqpoint{3.206694in}{1.897928in}}%
\pgfpathmoveto{\pgfqpoint{3.124954in}{1.959862in}}%
\pgfpathlineto{\pgfqpoint{3.124954in}{1.959862in}}%
\pgfpathlineto{\pgfqpoint{3.124954in}{1.962812in}}%
\pgfpathlineto{\pgfqpoint{3.129495in}{1.962812in}}%
\pgfpathlineto{\pgfqpoint{3.129495in}{1.959862in}}%
\pgfpathmoveto{\pgfqpoint{3.138577in}{1.948065in}}%
\pgfpathlineto{\pgfqpoint{3.138577in}{1.948065in}}%
\pgfpathlineto{\pgfqpoint{3.138577in}{1.951014in}}%
\pgfpathlineto{\pgfqpoint{3.143118in}{1.951014in}}%
\pgfpathlineto{\pgfqpoint{3.143118in}{1.948065in}}%
\pgfpathmoveto{\pgfqpoint{3.143118in}{1.945116in}}%
\pgfpathlineto{\pgfqpoint{3.143118in}{1.945116in}}%
\pgfpathlineto{\pgfqpoint{3.143118in}{1.948065in}}%
\pgfpathlineto{\pgfqpoint{3.147659in}{1.948065in}}%
\pgfpathlineto{\pgfqpoint{3.147659in}{1.945116in}}%
\pgfpathmoveto{\pgfqpoint{3.143118in}{1.948065in}}%
\pgfpathlineto{\pgfqpoint{3.143118in}{1.948065in}}%
\pgfpathlineto{\pgfqpoint{3.143118in}{1.951014in}}%
\pgfpathlineto{\pgfqpoint{3.147659in}{1.951014in}}%
\pgfpathlineto{\pgfqpoint{3.147659in}{1.948065in}}%
\pgfpathmoveto{\pgfqpoint{3.134036in}{1.953964in}}%
\pgfpathlineto{\pgfqpoint{3.134036in}{1.953964in}}%
\pgfpathlineto{\pgfqpoint{3.134036in}{1.956913in}}%
\pgfpathlineto{\pgfqpoint{3.138577in}{1.956913in}}%
\pgfpathlineto{\pgfqpoint{3.138577in}{1.953964in}}%
\pgfpathmoveto{\pgfqpoint{3.129495in}{1.956913in}}%
\pgfpathlineto{\pgfqpoint{3.129495in}{1.956913in}}%
\pgfpathlineto{\pgfqpoint{3.129495in}{1.959862in}}%
\pgfpathlineto{\pgfqpoint{3.134036in}{1.959862in}}%
\pgfpathlineto{\pgfqpoint{3.134036in}{1.956913in}}%
\pgfpathmoveto{\pgfqpoint{3.129495in}{1.959862in}}%
\pgfpathlineto{\pgfqpoint{3.129495in}{1.959862in}}%
\pgfpathlineto{\pgfqpoint{3.129495in}{1.962812in}}%
\pgfpathlineto{\pgfqpoint{3.134036in}{1.962812in}}%
\pgfpathlineto{\pgfqpoint{3.134036in}{1.959862in}}%
\pgfpathmoveto{\pgfqpoint{3.134036in}{1.956913in}}%
\pgfpathlineto{\pgfqpoint{3.134036in}{1.956913in}}%
\pgfpathlineto{\pgfqpoint{3.134036in}{1.959862in}}%
\pgfpathlineto{\pgfqpoint{3.138577in}{1.959862in}}%
\pgfpathlineto{\pgfqpoint{3.138577in}{1.956913in}}%
\pgfpathmoveto{\pgfqpoint{3.138577in}{1.951014in}}%
\pgfpathlineto{\pgfqpoint{3.138577in}{1.951014in}}%
\pgfpathlineto{\pgfqpoint{3.138577in}{1.953964in}}%
\pgfpathlineto{\pgfqpoint{3.143118in}{1.953964in}}%
\pgfpathlineto{\pgfqpoint{3.143118in}{1.951014in}}%
\pgfpathmoveto{\pgfqpoint{3.138577in}{1.953964in}}%
\pgfpathlineto{\pgfqpoint{3.138577in}{1.953964in}}%
\pgfpathlineto{\pgfqpoint{3.138577in}{1.956913in}}%
\pgfpathlineto{\pgfqpoint{3.143118in}{1.956913in}}%
\pgfpathlineto{\pgfqpoint{3.143118in}{1.953964in}}%
\pgfpathmoveto{\pgfqpoint{3.097707in}{1.983457in}}%
\pgfpathlineto{\pgfqpoint{3.097707in}{1.983457in}}%
\pgfpathlineto{\pgfqpoint{3.097707in}{1.986406in}}%
\pgfpathlineto{\pgfqpoint{3.102248in}{1.986406in}}%
\pgfpathlineto{\pgfqpoint{3.102248in}{1.983457in}}%
\pgfpathmoveto{\pgfqpoint{3.106790in}{1.977558in}}%
\pgfpathlineto{\pgfqpoint{3.106790in}{1.977558in}}%
\pgfpathlineto{\pgfqpoint{3.106790in}{1.980508in}}%
\pgfpathlineto{\pgfqpoint{3.111331in}{1.980508in}}%
\pgfpathlineto{\pgfqpoint{3.111331in}{1.977558in}}%
\pgfpathmoveto{\pgfqpoint{3.102248in}{1.980508in}}%
\pgfpathlineto{\pgfqpoint{3.102248in}{1.980508in}}%
\pgfpathlineto{\pgfqpoint{3.102248in}{1.983457in}}%
\pgfpathlineto{\pgfqpoint{3.106790in}{1.983457in}}%
\pgfpathlineto{\pgfqpoint{3.106790in}{1.980508in}}%
\pgfpathmoveto{\pgfqpoint{3.102248in}{1.983457in}}%
\pgfpathlineto{\pgfqpoint{3.102248in}{1.983457in}}%
\pgfpathlineto{\pgfqpoint{3.102248in}{1.986406in}}%
\pgfpathlineto{\pgfqpoint{3.106790in}{1.986406in}}%
\pgfpathlineto{\pgfqpoint{3.106790in}{1.983457in}}%
\pgfpathmoveto{\pgfqpoint{3.106790in}{1.980508in}}%
\pgfpathlineto{\pgfqpoint{3.106790in}{1.980508in}}%
\pgfpathlineto{\pgfqpoint{3.106790in}{1.983457in}}%
\pgfpathlineto{\pgfqpoint{3.111331in}{1.983457in}}%
\pgfpathlineto{\pgfqpoint{3.111331in}{1.980508in}}%
\pgfpathmoveto{\pgfqpoint{3.084084in}{1.995254in}}%
\pgfpathlineto{\pgfqpoint{3.084084in}{1.995254in}}%
\pgfpathlineto{\pgfqpoint{3.084084in}{1.998204in}}%
\pgfpathlineto{\pgfqpoint{3.088625in}{1.998204in}}%
\pgfpathlineto{\pgfqpoint{3.088625in}{1.995254in}}%
\pgfpathmoveto{\pgfqpoint{3.088625in}{1.992305in}}%
\pgfpathlineto{\pgfqpoint{3.088625in}{1.992305in}}%
\pgfpathlineto{\pgfqpoint{3.088625in}{1.995254in}}%
\pgfpathlineto{\pgfqpoint{3.093166in}{1.995254in}}%
\pgfpathlineto{\pgfqpoint{3.093166in}{1.992305in}}%
\pgfpathmoveto{\pgfqpoint{3.088625in}{1.995254in}}%
\pgfpathlineto{\pgfqpoint{3.088625in}{1.995254in}}%
\pgfpathlineto{\pgfqpoint{3.088625in}{1.998204in}}%
\pgfpathlineto{\pgfqpoint{3.093166in}{1.998204in}}%
\pgfpathlineto{\pgfqpoint{3.093166in}{1.995254in}}%
\pgfpathmoveto{\pgfqpoint{3.079543in}{2.001153in}}%
\pgfpathlineto{\pgfqpoint{3.079543in}{2.001153in}}%
\pgfpathlineto{\pgfqpoint{3.079543in}{2.004102in}}%
\pgfpathlineto{\pgfqpoint{3.084084in}{2.004102in}}%
\pgfpathlineto{\pgfqpoint{3.084084in}{2.001153in}}%
\pgfpathmoveto{\pgfqpoint{3.075002in}{2.004102in}}%
\pgfpathlineto{\pgfqpoint{3.075002in}{2.004102in}}%
\pgfpathlineto{\pgfqpoint{3.075002in}{2.007052in}}%
\pgfpathlineto{\pgfqpoint{3.079543in}{2.007052in}}%
\pgfpathlineto{\pgfqpoint{3.079543in}{2.004102in}}%
\pgfpathmoveto{\pgfqpoint{3.075002in}{2.007052in}}%
\pgfpathlineto{\pgfqpoint{3.075002in}{2.007052in}}%
\pgfpathlineto{\pgfqpoint{3.075002in}{2.010001in}}%
\pgfpathlineto{\pgfqpoint{3.079543in}{2.010001in}}%
\pgfpathlineto{\pgfqpoint{3.079543in}{2.007052in}}%
\pgfpathmoveto{\pgfqpoint{3.079543in}{2.004102in}}%
\pgfpathlineto{\pgfqpoint{3.079543in}{2.004102in}}%
\pgfpathlineto{\pgfqpoint{3.079543in}{2.007052in}}%
\pgfpathlineto{\pgfqpoint{3.084084in}{2.007052in}}%
\pgfpathlineto{\pgfqpoint{3.084084in}{2.004102in}}%
\pgfpathmoveto{\pgfqpoint{3.084084in}{1.998204in}}%
\pgfpathlineto{\pgfqpoint{3.084084in}{1.998204in}}%
\pgfpathlineto{\pgfqpoint{3.084084in}{2.001153in}}%
\pgfpathlineto{\pgfqpoint{3.088625in}{2.001153in}}%
\pgfpathlineto{\pgfqpoint{3.088625in}{1.998204in}}%
\pgfpathmoveto{\pgfqpoint{3.084084in}{2.001153in}}%
\pgfpathlineto{\pgfqpoint{3.084084in}{2.001153in}}%
\pgfpathlineto{\pgfqpoint{3.084084in}{2.004102in}}%
\pgfpathlineto{\pgfqpoint{3.088625in}{2.004102in}}%
\pgfpathlineto{\pgfqpoint{3.088625in}{2.001153in}}%
\pgfpathmoveto{\pgfqpoint{3.093166in}{1.989356in}}%
\pgfpathlineto{\pgfqpoint{3.093166in}{1.989356in}}%
\pgfpathlineto{\pgfqpoint{3.093166in}{1.992305in}}%
\pgfpathlineto{\pgfqpoint{3.097707in}{1.992305in}}%
\pgfpathlineto{\pgfqpoint{3.097707in}{1.989356in}}%
\pgfpathmoveto{\pgfqpoint{3.097707in}{1.986406in}}%
\pgfpathlineto{\pgfqpoint{3.097707in}{1.986406in}}%
\pgfpathlineto{\pgfqpoint{3.097707in}{1.989356in}}%
\pgfpathlineto{\pgfqpoint{3.102248in}{1.989356in}}%
\pgfpathlineto{\pgfqpoint{3.102248in}{1.986406in}}%
\pgfpathmoveto{\pgfqpoint{3.097707in}{1.989356in}}%
\pgfpathlineto{\pgfqpoint{3.097707in}{1.989356in}}%
\pgfpathlineto{\pgfqpoint{3.097707in}{1.992305in}}%
\pgfpathlineto{\pgfqpoint{3.102248in}{1.992305in}}%
\pgfpathlineto{\pgfqpoint{3.102248in}{1.989356in}}%
\pgfpathmoveto{\pgfqpoint{3.093166in}{1.992305in}}%
\pgfpathlineto{\pgfqpoint{3.093166in}{1.992305in}}%
\pgfpathlineto{\pgfqpoint{3.093166in}{1.995254in}}%
\pgfpathlineto{\pgfqpoint{3.097707in}{1.995254in}}%
\pgfpathlineto{\pgfqpoint{3.097707in}{1.992305in}}%
\pgfpathmoveto{\pgfqpoint{3.111331in}{1.971660in}}%
\pgfpathlineto{\pgfqpoint{3.111331in}{1.971660in}}%
\pgfpathlineto{\pgfqpoint{3.111331in}{1.974609in}}%
\pgfpathlineto{\pgfqpoint{3.115872in}{1.974609in}}%
\pgfpathlineto{\pgfqpoint{3.115872in}{1.971660in}}%
\pgfpathmoveto{\pgfqpoint{3.115872in}{1.968710in}}%
\pgfpathlineto{\pgfqpoint{3.115872in}{1.968710in}}%
\pgfpathlineto{\pgfqpoint{3.115872in}{1.971660in}}%
\pgfpathlineto{\pgfqpoint{3.120413in}{1.971660in}}%
\pgfpathlineto{\pgfqpoint{3.120413in}{1.968710in}}%
\pgfpathmoveto{\pgfqpoint{3.115872in}{1.971660in}}%
\pgfpathlineto{\pgfqpoint{3.115872in}{1.971660in}}%
\pgfpathlineto{\pgfqpoint{3.115872in}{1.974609in}}%
\pgfpathlineto{\pgfqpoint{3.120413in}{1.974609in}}%
\pgfpathlineto{\pgfqpoint{3.120413in}{1.971660in}}%
\pgfpathmoveto{\pgfqpoint{3.120413in}{1.965761in}}%
\pgfpathlineto{\pgfqpoint{3.120413in}{1.965761in}}%
\pgfpathlineto{\pgfqpoint{3.120413in}{1.968710in}}%
\pgfpathlineto{\pgfqpoint{3.124954in}{1.968710in}}%
\pgfpathlineto{\pgfqpoint{3.124954in}{1.965761in}}%
\pgfpathmoveto{\pgfqpoint{3.124954in}{1.962812in}}%
\pgfpathlineto{\pgfqpoint{3.124954in}{1.962812in}}%
\pgfpathlineto{\pgfqpoint{3.124954in}{1.965761in}}%
\pgfpathlineto{\pgfqpoint{3.129495in}{1.965761in}}%
\pgfpathlineto{\pgfqpoint{3.129495in}{1.962812in}}%
\pgfpathmoveto{\pgfqpoint{3.124954in}{1.965761in}}%
\pgfpathlineto{\pgfqpoint{3.124954in}{1.965761in}}%
\pgfpathlineto{\pgfqpoint{3.124954in}{1.968710in}}%
\pgfpathlineto{\pgfqpoint{3.129495in}{1.968710in}}%
\pgfpathlineto{\pgfqpoint{3.129495in}{1.965761in}}%
\pgfpathmoveto{\pgfqpoint{3.120413in}{1.968710in}}%
\pgfpathlineto{\pgfqpoint{3.120413in}{1.968710in}}%
\pgfpathlineto{\pgfqpoint{3.120413in}{1.971660in}}%
\pgfpathlineto{\pgfqpoint{3.124954in}{1.971660in}}%
\pgfpathlineto{\pgfqpoint{3.124954in}{1.968710in}}%
\pgfpathmoveto{\pgfqpoint{3.111331in}{1.974609in}}%
\pgfpathlineto{\pgfqpoint{3.111331in}{1.974609in}}%
\pgfpathlineto{\pgfqpoint{3.111331in}{1.977558in}}%
\pgfpathlineto{\pgfqpoint{3.115872in}{1.977558in}}%
\pgfpathlineto{\pgfqpoint{3.115872in}{1.974609in}}%
\pgfpathmoveto{\pgfqpoint{3.111331in}{1.977558in}}%
\pgfpathlineto{\pgfqpoint{3.111331in}{1.977558in}}%
\pgfpathlineto{\pgfqpoint{3.111331in}{1.980508in}}%
\pgfpathlineto{\pgfqpoint{3.115872in}{1.980508in}}%
\pgfpathlineto{\pgfqpoint{3.115872in}{1.977558in}}%
\pgfpathmoveto{\pgfqpoint{3.152200in}{1.936268in}}%
\pgfpathlineto{\pgfqpoint{3.152200in}{1.936268in}}%
\pgfpathlineto{\pgfqpoint{3.152200in}{1.939217in}}%
\pgfpathlineto{\pgfqpoint{3.156742in}{1.939217in}}%
\pgfpathlineto{\pgfqpoint{3.156742in}{1.936268in}}%
\pgfpathmoveto{\pgfqpoint{3.161283in}{1.930369in}}%
\pgfpathlineto{\pgfqpoint{3.161283in}{1.930369in}}%
\pgfpathlineto{\pgfqpoint{3.161283in}{1.933318in}}%
\pgfpathlineto{\pgfqpoint{3.165824in}{1.933318in}}%
\pgfpathlineto{\pgfqpoint{3.165824in}{1.930369in}}%
\pgfpathmoveto{\pgfqpoint{3.156742in}{1.933318in}}%
\pgfpathlineto{\pgfqpoint{3.156742in}{1.933318in}}%
\pgfpathlineto{\pgfqpoint{3.156742in}{1.936268in}}%
\pgfpathlineto{\pgfqpoint{3.161283in}{1.936268in}}%
\pgfpathlineto{\pgfqpoint{3.161283in}{1.933318in}}%
\pgfpathmoveto{\pgfqpoint{3.156742in}{1.936268in}}%
\pgfpathlineto{\pgfqpoint{3.156742in}{1.936268in}}%
\pgfpathlineto{\pgfqpoint{3.156742in}{1.939217in}}%
\pgfpathlineto{\pgfqpoint{3.161283in}{1.939217in}}%
\pgfpathlineto{\pgfqpoint{3.161283in}{1.936268in}}%
\pgfpathmoveto{\pgfqpoint{3.161283in}{1.933318in}}%
\pgfpathlineto{\pgfqpoint{3.161283in}{1.933318in}}%
\pgfpathlineto{\pgfqpoint{3.161283in}{1.936268in}}%
\pgfpathlineto{\pgfqpoint{3.165824in}{1.936268in}}%
\pgfpathlineto{\pgfqpoint{3.165824in}{1.933318in}}%
\pgfpathmoveto{\pgfqpoint{3.165824in}{1.924470in}}%
\pgfpathlineto{\pgfqpoint{3.165824in}{1.924470in}}%
\pgfpathlineto{\pgfqpoint{3.165824in}{1.927420in}}%
\pgfpathlineto{\pgfqpoint{3.170365in}{1.927420in}}%
\pgfpathlineto{\pgfqpoint{3.170365in}{1.924470in}}%
\pgfpathmoveto{\pgfqpoint{3.170365in}{1.921521in}}%
\pgfpathlineto{\pgfqpoint{3.170365in}{1.921521in}}%
\pgfpathlineto{\pgfqpoint{3.170365in}{1.924470in}}%
\pgfpathlineto{\pgfqpoint{3.174906in}{1.924470in}}%
\pgfpathlineto{\pgfqpoint{3.174906in}{1.921521in}}%
\pgfpathmoveto{\pgfqpoint{3.170365in}{1.924470in}}%
\pgfpathlineto{\pgfqpoint{3.170365in}{1.924470in}}%
\pgfpathlineto{\pgfqpoint{3.170365in}{1.927420in}}%
\pgfpathlineto{\pgfqpoint{3.174906in}{1.927420in}}%
\pgfpathlineto{\pgfqpoint{3.174906in}{1.924470in}}%
\pgfpathmoveto{\pgfqpoint{3.174906in}{1.918572in}}%
\pgfpathlineto{\pgfqpoint{3.174906in}{1.918572in}}%
\pgfpathlineto{\pgfqpoint{3.174906in}{1.921521in}}%
\pgfpathlineto{\pgfqpoint{3.179447in}{1.921521in}}%
\pgfpathlineto{\pgfqpoint{3.179447in}{1.918572in}}%
\pgfpathmoveto{\pgfqpoint{3.179447in}{1.915622in}}%
\pgfpathlineto{\pgfqpoint{3.179447in}{1.915622in}}%
\pgfpathlineto{\pgfqpoint{3.179447in}{1.918572in}}%
\pgfpathlineto{\pgfqpoint{3.183988in}{1.918572in}}%
\pgfpathlineto{\pgfqpoint{3.183988in}{1.915622in}}%
\pgfpathmoveto{\pgfqpoint{3.179447in}{1.918572in}}%
\pgfpathlineto{\pgfqpoint{3.179447in}{1.918572in}}%
\pgfpathlineto{\pgfqpoint{3.179447in}{1.921521in}}%
\pgfpathlineto{\pgfqpoint{3.183988in}{1.921521in}}%
\pgfpathlineto{\pgfqpoint{3.183988in}{1.918572in}}%
\pgfpathmoveto{\pgfqpoint{3.174906in}{1.921521in}}%
\pgfpathlineto{\pgfqpoint{3.174906in}{1.921521in}}%
\pgfpathlineto{\pgfqpoint{3.174906in}{1.924470in}}%
\pgfpathlineto{\pgfqpoint{3.179447in}{1.924470in}}%
\pgfpathlineto{\pgfqpoint{3.179447in}{1.921521in}}%
\pgfpathmoveto{\pgfqpoint{3.165824in}{1.927420in}}%
\pgfpathlineto{\pgfqpoint{3.165824in}{1.927420in}}%
\pgfpathlineto{\pgfqpoint{3.165824in}{1.930369in}}%
\pgfpathlineto{\pgfqpoint{3.170365in}{1.930369in}}%
\pgfpathlineto{\pgfqpoint{3.170365in}{1.927420in}}%
\pgfpathmoveto{\pgfqpoint{3.165824in}{1.930369in}}%
\pgfpathlineto{\pgfqpoint{3.165824in}{1.930369in}}%
\pgfpathlineto{\pgfqpoint{3.165824in}{1.933318in}}%
\pgfpathlineto{\pgfqpoint{3.170365in}{1.933318in}}%
\pgfpathlineto{\pgfqpoint{3.170365in}{1.930369in}}%
\pgfpathmoveto{\pgfqpoint{3.147659in}{1.942166in}}%
\pgfpathlineto{\pgfqpoint{3.147659in}{1.942166in}}%
\pgfpathlineto{\pgfqpoint{3.147659in}{1.945116in}}%
\pgfpathlineto{\pgfqpoint{3.152200in}{1.945116in}}%
\pgfpathlineto{\pgfqpoint{3.152200in}{1.942166in}}%
\pgfpathmoveto{\pgfqpoint{3.152200in}{1.939217in}}%
\pgfpathlineto{\pgfqpoint{3.152200in}{1.939217in}}%
\pgfpathlineto{\pgfqpoint{3.152200in}{1.942166in}}%
\pgfpathlineto{\pgfqpoint{3.156742in}{1.942166in}}%
\pgfpathlineto{\pgfqpoint{3.156742in}{1.939217in}}%
\pgfpathmoveto{\pgfqpoint{3.152200in}{1.942166in}}%
\pgfpathlineto{\pgfqpoint{3.152200in}{1.942166in}}%
\pgfpathlineto{\pgfqpoint{3.152200in}{1.945116in}}%
\pgfpathlineto{\pgfqpoint{3.156742in}{1.945116in}}%
\pgfpathlineto{\pgfqpoint{3.156742in}{1.942166in}}%
\pgfpathmoveto{\pgfqpoint{3.147659in}{1.945116in}}%
\pgfpathlineto{\pgfqpoint{3.147659in}{1.945116in}}%
\pgfpathlineto{\pgfqpoint{3.147659in}{1.948065in}}%
\pgfpathlineto{\pgfqpoint{3.152200in}{1.948065in}}%
\pgfpathlineto{\pgfqpoint{3.152200in}{1.945116in}}%
\pgfpathmoveto{\pgfqpoint{3.288430in}{1.818302in}}%
\pgfpathlineto{\pgfqpoint{3.288430in}{1.818302in}}%
\pgfpathlineto{\pgfqpoint{3.288430in}{1.821251in}}%
\pgfpathlineto{\pgfqpoint{3.292971in}{1.821251in}}%
\pgfpathlineto{\pgfqpoint{3.292971in}{1.818302in}}%
\pgfpathmoveto{\pgfqpoint{3.342921in}{1.771112in}}%
\pgfpathlineto{\pgfqpoint{3.342921in}{1.771112in}}%
\pgfpathlineto{\pgfqpoint{3.342921in}{1.774062in}}%
\pgfpathlineto{\pgfqpoint{3.347462in}{1.774062in}}%
\pgfpathlineto{\pgfqpoint{3.347462in}{1.771112in}}%
\pgfpathmoveto{\pgfqpoint{3.361085in}{1.756366in}}%
\pgfpathlineto{\pgfqpoint{3.361085in}{1.756366in}}%
\pgfpathlineto{\pgfqpoint{3.361085in}{1.759315in}}%
\pgfpathlineto{\pgfqpoint{3.365626in}{1.759315in}}%
\pgfpathlineto{\pgfqpoint{3.365626in}{1.756366in}}%
\pgfpathmoveto{\pgfqpoint{3.361085in}{1.759315in}}%
\pgfpathlineto{\pgfqpoint{3.361085in}{1.759315in}}%
\pgfpathlineto{\pgfqpoint{3.361085in}{1.762264in}}%
\pgfpathlineto{\pgfqpoint{3.365626in}{1.762264in}}%
\pgfpathlineto{\pgfqpoint{3.365626in}{1.759315in}}%
\pgfpathmoveto{\pgfqpoint{3.352003in}{1.765214in}}%
\pgfpathlineto{\pgfqpoint{3.352003in}{1.765214in}}%
\pgfpathlineto{\pgfqpoint{3.352003in}{1.768163in}}%
\pgfpathlineto{\pgfqpoint{3.356544in}{1.768163in}}%
\pgfpathlineto{\pgfqpoint{3.356544in}{1.765214in}}%
\pgfpathmoveto{\pgfqpoint{3.347462in}{1.768163in}}%
\pgfpathlineto{\pgfqpoint{3.347462in}{1.768163in}}%
\pgfpathlineto{\pgfqpoint{3.347462in}{1.771112in}}%
\pgfpathlineto{\pgfqpoint{3.352003in}{1.771112in}}%
\pgfpathlineto{\pgfqpoint{3.352003in}{1.768163in}}%
\pgfpathmoveto{\pgfqpoint{3.347462in}{1.771112in}}%
\pgfpathlineto{\pgfqpoint{3.347462in}{1.771112in}}%
\pgfpathlineto{\pgfqpoint{3.347462in}{1.774062in}}%
\pgfpathlineto{\pgfqpoint{3.352003in}{1.774062in}}%
\pgfpathlineto{\pgfqpoint{3.352003in}{1.771112in}}%
\pgfpathmoveto{\pgfqpoint{3.352003in}{1.768163in}}%
\pgfpathlineto{\pgfqpoint{3.352003in}{1.768163in}}%
\pgfpathlineto{\pgfqpoint{3.352003in}{1.771112in}}%
\pgfpathlineto{\pgfqpoint{3.356544in}{1.771112in}}%
\pgfpathlineto{\pgfqpoint{3.356544in}{1.768163in}}%
\pgfpathmoveto{\pgfqpoint{3.356544in}{1.762264in}}%
\pgfpathlineto{\pgfqpoint{3.356544in}{1.762264in}}%
\pgfpathlineto{\pgfqpoint{3.356544in}{1.765214in}}%
\pgfpathlineto{\pgfqpoint{3.361085in}{1.765214in}}%
\pgfpathlineto{\pgfqpoint{3.361085in}{1.762264in}}%
\pgfpathmoveto{\pgfqpoint{3.356544in}{1.765214in}}%
\pgfpathlineto{\pgfqpoint{3.356544in}{1.765214in}}%
\pgfpathlineto{\pgfqpoint{3.356544in}{1.768163in}}%
\pgfpathlineto{\pgfqpoint{3.361085in}{1.768163in}}%
\pgfpathlineto{\pgfqpoint{3.361085in}{1.765214in}}%
\pgfpathmoveto{\pgfqpoint{3.361085in}{1.762264in}}%
\pgfpathlineto{\pgfqpoint{3.361085in}{1.762264in}}%
\pgfpathlineto{\pgfqpoint{3.361085in}{1.765214in}}%
\pgfpathlineto{\pgfqpoint{3.365626in}{1.765214in}}%
\pgfpathlineto{\pgfqpoint{3.365626in}{1.762264in}}%
\pgfpathmoveto{\pgfqpoint{3.315676in}{1.794707in}}%
\pgfpathlineto{\pgfqpoint{3.315676in}{1.794707in}}%
\pgfpathlineto{\pgfqpoint{3.315676in}{1.797656in}}%
\pgfpathlineto{\pgfqpoint{3.320217in}{1.797656in}}%
\pgfpathlineto{\pgfqpoint{3.320217in}{1.794707in}}%
\pgfpathmoveto{\pgfqpoint{3.324758in}{1.788808in}}%
\pgfpathlineto{\pgfqpoint{3.324758in}{1.788808in}}%
\pgfpathlineto{\pgfqpoint{3.324758in}{1.791758in}}%
\pgfpathlineto{\pgfqpoint{3.329299in}{1.791758in}}%
\pgfpathlineto{\pgfqpoint{3.329299in}{1.788808in}}%
\pgfpathmoveto{\pgfqpoint{3.320217in}{1.791758in}}%
\pgfpathlineto{\pgfqpoint{3.320217in}{1.791758in}}%
\pgfpathlineto{\pgfqpoint{3.320217in}{1.794707in}}%
\pgfpathlineto{\pgfqpoint{3.324758in}{1.794707in}}%
\pgfpathlineto{\pgfqpoint{3.324758in}{1.791758in}}%
\pgfpathmoveto{\pgfqpoint{3.320217in}{1.794707in}}%
\pgfpathlineto{\pgfqpoint{3.320217in}{1.794707in}}%
\pgfpathlineto{\pgfqpoint{3.320217in}{1.797656in}}%
\pgfpathlineto{\pgfqpoint{3.324758in}{1.797656in}}%
\pgfpathlineto{\pgfqpoint{3.324758in}{1.794707in}}%
\pgfpathmoveto{\pgfqpoint{3.324758in}{1.791758in}}%
\pgfpathlineto{\pgfqpoint{3.324758in}{1.791758in}}%
\pgfpathlineto{\pgfqpoint{3.324758in}{1.794707in}}%
\pgfpathlineto{\pgfqpoint{3.329299in}{1.794707in}}%
\pgfpathlineto{\pgfqpoint{3.329299in}{1.791758in}}%
\pgfpathmoveto{\pgfqpoint{3.302053in}{1.806504in}}%
\pgfpathlineto{\pgfqpoint{3.302053in}{1.806504in}}%
\pgfpathlineto{\pgfqpoint{3.302053in}{1.809454in}}%
\pgfpathlineto{\pgfqpoint{3.306594in}{1.809454in}}%
\pgfpathlineto{\pgfqpoint{3.306594in}{1.806504in}}%
\pgfpathmoveto{\pgfqpoint{3.306594in}{1.803555in}}%
\pgfpathlineto{\pgfqpoint{3.306594in}{1.803555in}}%
\pgfpathlineto{\pgfqpoint{3.306594in}{1.806504in}}%
\pgfpathlineto{\pgfqpoint{3.311135in}{1.806504in}}%
\pgfpathlineto{\pgfqpoint{3.311135in}{1.803555in}}%
\pgfpathmoveto{\pgfqpoint{3.306594in}{1.806504in}}%
\pgfpathlineto{\pgfqpoint{3.306594in}{1.806504in}}%
\pgfpathlineto{\pgfqpoint{3.306594in}{1.809454in}}%
\pgfpathlineto{\pgfqpoint{3.311135in}{1.809454in}}%
\pgfpathlineto{\pgfqpoint{3.311135in}{1.806504in}}%
\pgfpathmoveto{\pgfqpoint{3.297512in}{1.812403in}}%
\pgfpathlineto{\pgfqpoint{3.297512in}{1.812403in}}%
\pgfpathlineto{\pgfqpoint{3.297512in}{1.815352in}}%
\pgfpathlineto{\pgfqpoint{3.302053in}{1.815352in}}%
\pgfpathlineto{\pgfqpoint{3.302053in}{1.812403in}}%
\pgfpathmoveto{\pgfqpoint{3.292971in}{1.815352in}}%
\pgfpathlineto{\pgfqpoint{3.292971in}{1.815352in}}%
\pgfpathlineto{\pgfqpoint{3.292971in}{1.818302in}}%
\pgfpathlineto{\pgfqpoint{3.297512in}{1.818302in}}%
\pgfpathlineto{\pgfqpoint{3.297512in}{1.815352in}}%
\pgfpathmoveto{\pgfqpoint{3.292971in}{1.818302in}}%
\pgfpathlineto{\pgfqpoint{3.292971in}{1.818302in}}%
\pgfpathlineto{\pgfqpoint{3.292971in}{1.821251in}}%
\pgfpathlineto{\pgfqpoint{3.297512in}{1.821251in}}%
\pgfpathlineto{\pgfqpoint{3.297512in}{1.818302in}}%
\pgfpathmoveto{\pgfqpoint{3.297512in}{1.815352in}}%
\pgfpathlineto{\pgfqpoint{3.297512in}{1.815352in}}%
\pgfpathlineto{\pgfqpoint{3.297512in}{1.818302in}}%
\pgfpathlineto{\pgfqpoint{3.302053in}{1.818302in}}%
\pgfpathlineto{\pgfqpoint{3.302053in}{1.815352in}}%
\pgfpathmoveto{\pgfqpoint{3.302053in}{1.809454in}}%
\pgfpathlineto{\pgfqpoint{3.302053in}{1.809454in}}%
\pgfpathlineto{\pgfqpoint{3.302053in}{1.812403in}}%
\pgfpathlineto{\pgfqpoint{3.306594in}{1.812403in}}%
\pgfpathlineto{\pgfqpoint{3.306594in}{1.809454in}}%
\pgfpathmoveto{\pgfqpoint{3.302053in}{1.812403in}}%
\pgfpathlineto{\pgfqpoint{3.302053in}{1.812403in}}%
\pgfpathlineto{\pgfqpoint{3.302053in}{1.815352in}}%
\pgfpathlineto{\pgfqpoint{3.306594in}{1.815352in}}%
\pgfpathlineto{\pgfqpoint{3.306594in}{1.812403in}}%
\pgfpathmoveto{\pgfqpoint{3.311135in}{1.800606in}}%
\pgfpathlineto{\pgfqpoint{3.311135in}{1.800606in}}%
\pgfpathlineto{\pgfqpoint{3.311135in}{1.803555in}}%
\pgfpathlineto{\pgfqpoint{3.315676in}{1.803555in}}%
\pgfpathlineto{\pgfqpoint{3.315676in}{1.800606in}}%
\pgfpathmoveto{\pgfqpoint{3.315676in}{1.797656in}}%
\pgfpathlineto{\pgfqpoint{3.315676in}{1.797656in}}%
\pgfpathlineto{\pgfqpoint{3.315676in}{1.800606in}}%
\pgfpathlineto{\pgfqpoint{3.320217in}{1.800606in}}%
\pgfpathlineto{\pgfqpoint{3.320217in}{1.797656in}}%
\pgfpathmoveto{\pgfqpoint{3.315676in}{1.800606in}}%
\pgfpathlineto{\pgfqpoint{3.315676in}{1.800606in}}%
\pgfpathlineto{\pgfqpoint{3.315676in}{1.803555in}}%
\pgfpathlineto{\pgfqpoint{3.320217in}{1.803555in}}%
\pgfpathlineto{\pgfqpoint{3.320217in}{1.800606in}}%
\pgfpathmoveto{\pgfqpoint{3.311135in}{1.803555in}}%
\pgfpathlineto{\pgfqpoint{3.311135in}{1.803555in}}%
\pgfpathlineto{\pgfqpoint{3.311135in}{1.806504in}}%
\pgfpathlineto{\pgfqpoint{3.315676in}{1.806504in}}%
\pgfpathlineto{\pgfqpoint{3.315676in}{1.803555in}}%
\pgfpathmoveto{\pgfqpoint{3.329299in}{1.782910in}}%
\pgfpathlineto{\pgfqpoint{3.329299in}{1.782910in}}%
\pgfpathlineto{\pgfqpoint{3.329299in}{1.785859in}}%
\pgfpathlineto{\pgfqpoint{3.333840in}{1.785859in}}%
\pgfpathlineto{\pgfqpoint{3.333840in}{1.782910in}}%
\pgfpathmoveto{\pgfqpoint{3.333840in}{1.779960in}}%
\pgfpathlineto{\pgfqpoint{3.333840in}{1.779960in}}%
\pgfpathlineto{\pgfqpoint{3.333840in}{1.782910in}}%
\pgfpathlineto{\pgfqpoint{3.338380in}{1.782910in}}%
\pgfpathlineto{\pgfqpoint{3.338380in}{1.779960in}}%
\pgfpathmoveto{\pgfqpoint{3.333840in}{1.782910in}}%
\pgfpathlineto{\pgfqpoint{3.333840in}{1.782910in}}%
\pgfpathlineto{\pgfqpoint{3.333840in}{1.785859in}}%
\pgfpathlineto{\pgfqpoint{3.338380in}{1.785859in}}%
\pgfpathlineto{\pgfqpoint{3.338380in}{1.782910in}}%
\pgfpathmoveto{\pgfqpoint{3.338380in}{1.777011in}}%
\pgfpathlineto{\pgfqpoint{3.338380in}{1.777011in}}%
\pgfpathlineto{\pgfqpoint{3.338380in}{1.779960in}}%
\pgfpathlineto{\pgfqpoint{3.342921in}{1.779960in}}%
\pgfpathlineto{\pgfqpoint{3.342921in}{1.777011in}}%
\pgfpathmoveto{\pgfqpoint{3.342921in}{1.774062in}}%
\pgfpathlineto{\pgfqpoint{3.342921in}{1.774062in}}%
\pgfpathlineto{\pgfqpoint{3.342921in}{1.777011in}}%
\pgfpathlineto{\pgfqpoint{3.347462in}{1.777011in}}%
\pgfpathlineto{\pgfqpoint{3.347462in}{1.774062in}}%
\pgfpathmoveto{\pgfqpoint{3.342921in}{1.777011in}}%
\pgfpathlineto{\pgfqpoint{3.342921in}{1.777011in}}%
\pgfpathlineto{\pgfqpoint{3.342921in}{1.779960in}}%
\pgfpathlineto{\pgfqpoint{3.347462in}{1.779960in}}%
\pgfpathlineto{\pgfqpoint{3.347462in}{1.777011in}}%
\pgfpathmoveto{\pgfqpoint{3.338380in}{1.779960in}}%
\pgfpathlineto{\pgfqpoint{3.338380in}{1.779960in}}%
\pgfpathlineto{\pgfqpoint{3.338380in}{1.782910in}}%
\pgfpathlineto{\pgfqpoint{3.342921in}{1.782910in}}%
\pgfpathlineto{\pgfqpoint{3.342921in}{1.779960in}}%
\pgfpathmoveto{\pgfqpoint{3.329299in}{1.785859in}}%
\pgfpathlineto{\pgfqpoint{3.329299in}{1.785859in}}%
\pgfpathlineto{\pgfqpoint{3.329299in}{1.788808in}}%
\pgfpathlineto{\pgfqpoint{3.333840in}{1.788808in}}%
\pgfpathlineto{\pgfqpoint{3.333840in}{1.785859in}}%
\pgfpathmoveto{\pgfqpoint{3.329299in}{1.788808in}}%
\pgfpathlineto{\pgfqpoint{3.329299in}{1.788808in}}%
\pgfpathlineto{\pgfqpoint{3.329299in}{1.791758in}}%
\pgfpathlineto{\pgfqpoint{3.333840in}{1.791758in}}%
\pgfpathlineto{\pgfqpoint{3.333840in}{1.788808in}}%
\pgfpathmoveto{\pgfqpoint{3.233940in}{1.865488in}}%
\pgfpathlineto{\pgfqpoint{3.233940in}{1.865488in}}%
\pgfpathlineto{\pgfqpoint{3.233940in}{1.868437in}}%
\pgfpathlineto{\pgfqpoint{3.238481in}{1.868437in}}%
\pgfpathlineto{\pgfqpoint{3.238481in}{1.865488in}}%
\pgfpathmoveto{\pgfqpoint{3.247562in}{1.853691in}}%
\pgfpathlineto{\pgfqpoint{3.247562in}{1.853691in}}%
\pgfpathlineto{\pgfqpoint{3.247562in}{1.856640in}}%
\pgfpathlineto{\pgfqpoint{3.252103in}{1.856640in}}%
\pgfpathlineto{\pgfqpoint{3.252103in}{1.853691in}}%
\pgfpathmoveto{\pgfqpoint{3.252103in}{1.850742in}}%
\pgfpathlineto{\pgfqpoint{3.252103in}{1.850742in}}%
\pgfpathlineto{\pgfqpoint{3.252103in}{1.853691in}}%
\pgfpathlineto{\pgfqpoint{3.256644in}{1.853691in}}%
\pgfpathlineto{\pgfqpoint{3.256644in}{1.850742in}}%
\pgfpathmoveto{\pgfqpoint{3.252103in}{1.853691in}}%
\pgfpathlineto{\pgfqpoint{3.252103in}{1.853691in}}%
\pgfpathlineto{\pgfqpoint{3.252103in}{1.856640in}}%
\pgfpathlineto{\pgfqpoint{3.256644in}{1.856640in}}%
\pgfpathlineto{\pgfqpoint{3.256644in}{1.853691in}}%
\pgfpathmoveto{\pgfqpoint{3.243021in}{1.859589in}}%
\pgfpathlineto{\pgfqpoint{3.243021in}{1.859589in}}%
\pgfpathlineto{\pgfqpoint{3.243021in}{1.862538in}}%
\pgfpathlineto{\pgfqpoint{3.247562in}{1.862538in}}%
\pgfpathlineto{\pgfqpoint{3.247562in}{1.859589in}}%
\pgfpathmoveto{\pgfqpoint{3.238481in}{1.862538in}}%
\pgfpathlineto{\pgfqpoint{3.238481in}{1.862538in}}%
\pgfpathlineto{\pgfqpoint{3.238481in}{1.865488in}}%
\pgfpathlineto{\pgfqpoint{3.243021in}{1.865488in}}%
\pgfpathlineto{\pgfqpoint{3.243021in}{1.862538in}}%
\pgfpathmoveto{\pgfqpoint{3.238481in}{1.865488in}}%
\pgfpathlineto{\pgfqpoint{3.238481in}{1.865488in}}%
\pgfpathlineto{\pgfqpoint{3.238481in}{1.868437in}}%
\pgfpathlineto{\pgfqpoint{3.243021in}{1.868437in}}%
\pgfpathlineto{\pgfqpoint{3.243021in}{1.865488in}}%
\pgfpathmoveto{\pgfqpoint{3.243021in}{1.862538in}}%
\pgfpathlineto{\pgfqpoint{3.243021in}{1.862538in}}%
\pgfpathlineto{\pgfqpoint{3.243021in}{1.865488in}}%
\pgfpathlineto{\pgfqpoint{3.247562in}{1.865488in}}%
\pgfpathlineto{\pgfqpoint{3.247562in}{1.862538in}}%
\pgfpathmoveto{\pgfqpoint{3.247562in}{1.856640in}}%
\pgfpathlineto{\pgfqpoint{3.247562in}{1.856640in}}%
\pgfpathlineto{\pgfqpoint{3.247562in}{1.859589in}}%
\pgfpathlineto{\pgfqpoint{3.252103in}{1.859589in}}%
\pgfpathlineto{\pgfqpoint{3.252103in}{1.856640in}}%
\pgfpathmoveto{\pgfqpoint{3.247562in}{1.859589in}}%
\pgfpathlineto{\pgfqpoint{3.247562in}{1.859589in}}%
\pgfpathlineto{\pgfqpoint{3.247562in}{1.862538in}}%
\pgfpathlineto{\pgfqpoint{3.252103in}{1.862538in}}%
\pgfpathlineto{\pgfqpoint{3.252103in}{1.859589in}}%
\pgfpathmoveto{\pgfqpoint{3.261185in}{1.841895in}}%
\pgfpathlineto{\pgfqpoint{3.261185in}{1.841895in}}%
\pgfpathlineto{\pgfqpoint{3.261185in}{1.844844in}}%
\pgfpathlineto{\pgfqpoint{3.265726in}{1.844844in}}%
\pgfpathlineto{\pgfqpoint{3.265726in}{1.841895in}}%
\pgfpathmoveto{\pgfqpoint{3.270267in}{1.835997in}}%
\pgfpathlineto{\pgfqpoint{3.270267in}{1.835997in}}%
\pgfpathlineto{\pgfqpoint{3.270267in}{1.838946in}}%
\pgfpathlineto{\pgfqpoint{3.274808in}{1.838946in}}%
\pgfpathlineto{\pgfqpoint{3.274808in}{1.835997in}}%
\pgfpathmoveto{\pgfqpoint{3.265726in}{1.838946in}}%
\pgfpathlineto{\pgfqpoint{3.265726in}{1.838946in}}%
\pgfpathlineto{\pgfqpoint{3.265726in}{1.841895in}}%
\pgfpathlineto{\pgfqpoint{3.270267in}{1.841895in}}%
\pgfpathlineto{\pgfqpoint{3.270267in}{1.838946in}}%
\pgfpathmoveto{\pgfqpoint{3.265726in}{1.841895in}}%
\pgfpathlineto{\pgfqpoint{3.265726in}{1.841895in}}%
\pgfpathlineto{\pgfqpoint{3.265726in}{1.844844in}}%
\pgfpathlineto{\pgfqpoint{3.270267in}{1.844844in}}%
\pgfpathlineto{\pgfqpoint{3.270267in}{1.841895in}}%
\pgfpathmoveto{\pgfqpoint{3.270267in}{1.838946in}}%
\pgfpathlineto{\pgfqpoint{3.270267in}{1.838946in}}%
\pgfpathlineto{\pgfqpoint{3.270267in}{1.841895in}}%
\pgfpathlineto{\pgfqpoint{3.274808in}{1.841895in}}%
\pgfpathlineto{\pgfqpoint{3.274808in}{1.838946in}}%
\pgfpathmoveto{\pgfqpoint{3.274808in}{1.830098in}}%
\pgfpathlineto{\pgfqpoint{3.274808in}{1.830098in}}%
\pgfpathlineto{\pgfqpoint{3.274808in}{1.833047in}}%
\pgfpathlineto{\pgfqpoint{3.279349in}{1.833047in}}%
\pgfpathlineto{\pgfqpoint{3.279349in}{1.830098in}}%
\pgfpathmoveto{\pgfqpoint{3.279349in}{1.827149in}}%
\pgfpathlineto{\pgfqpoint{3.279349in}{1.827149in}}%
\pgfpathlineto{\pgfqpoint{3.279349in}{1.830098in}}%
\pgfpathlineto{\pgfqpoint{3.283890in}{1.830098in}}%
\pgfpathlineto{\pgfqpoint{3.283890in}{1.827149in}}%
\pgfpathmoveto{\pgfqpoint{3.279349in}{1.830098in}}%
\pgfpathlineto{\pgfqpoint{3.279349in}{1.830098in}}%
\pgfpathlineto{\pgfqpoint{3.279349in}{1.833047in}}%
\pgfpathlineto{\pgfqpoint{3.283890in}{1.833047in}}%
\pgfpathlineto{\pgfqpoint{3.283890in}{1.830098in}}%
\pgfpathmoveto{\pgfqpoint{3.283890in}{1.824200in}}%
\pgfpathlineto{\pgfqpoint{3.283890in}{1.824200in}}%
\pgfpathlineto{\pgfqpoint{3.283890in}{1.827149in}}%
\pgfpathlineto{\pgfqpoint{3.288430in}{1.827149in}}%
\pgfpathlineto{\pgfqpoint{3.288430in}{1.824200in}}%
\pgfpathmoveto{\pgfqpoint{3.288430in}{1.821251in}}%
\pgfpathlineto{\pgfqpoint{3.288430in}{1.821251in}}%
\pgfpathlineto{\pgfqpoint{3.288430in}{1.824200in}}%
\pgfpathlineto{\pgfqpoint{3.292971in}{1.824200in}}%
\pgfpathlineto{\pgfqpoint{3.292971in}{1.821251in}}%
\pgfpathmoveto{\pgfqpoint{3.288430in}{1.824200in}}%
\pgfpathlineto{\pgfqpoint{3.288430in}{1.824200in}}%
\pgfpathlineto{\pgfqpoint{3.288430in}{1.827149in}}%
\pgfpathlineto{\pgfqpoint{3.292971in}{1.827149in}}%
\pgfpathlineto{\pgfqpoint{3.292971in}{1.824200in}}%
\pgfpathmoveto{\pgfqpoint{3.283890in}{1.827149in}}%
\pgfpathlineto{\pgfqpoint{3.283890in}{1.827149in}}%
\pgfpathlineto{\pgfqpoint{3.283890in}{1.830098in}}%
\pgfpathlineto{\pgfqpoint{3.288430in}{1.830098in}}%
\pgfpathlineto{\pgfqpoint{3.288430in}{1.827149in}}%
\pgfpathmoveto{\pgfqpoint{3.274808in}{1.833047in}}%
\pgfpathlineto{\pgfqpoint{3.274808in}{1.833047in}}%
\pgfpathlineto{\pgfqpoint{3.274808in}{1.835997in}}%
\pgfpathlineto{\pgfqpoint{3.279349in}{1.835997in}}%
\pgfpathlineto{\pgfqpoint{3.279349in}{1.833047in}}%
\pgfpathmoveto{\pgfqpoint{3.274808in}{1.835997in}}%
\pgfpathlineto{\pgfqpoint{3.274808in}{1.835997in}}%
\pgfpathlineto{\pgfqpoint{3.274808in}{1.838946in}}%
\pgfpathlineto{\pgfqpoint{3.279349in}{1.838946in}}%
\pgfpathlineto{\pgfqpoint{3.279349in}{1.835997in}}%
\pgfpathmoveto{\pgfqpoint{3.256644in}{1.847793in}}%
\pgfpathlineto{\pgfqpoint{3.256644in}{1.847793in}}%
\pgfpathlineto{\pgfqpoint{3.256644in}{1.850742in}}%
\pgfpathlineto{\pgfqpoint{3.261185in}{1.850742in}}%
\pgfpathlineto{\pgfqpoint{3.261185in}{1.847793in}}%
\pgfpathmoveto{\pgfqpoint{3.261185in}{1.844844in}}%
\pgfpathlineto{\pgfqpoint{3.261185in}{1.844844in}}%
\pgfpathlineto{\pgfqpoint{3.261185in}{1.847793in}}%
\pgfpathlineto{\pgfqpoint{3.265726in}{1.847793in}}%
\pgfpathlineto{\pgfqpoint{3.265726in}{1.844844in}}%
\pgfpathmoveto{\pgfqpoint{3.261185in}{1.847793in}}%
\pgfpathlineto{\pgfqpoint{3.261185in}{1.847793in}}%
\pgfpathlineto{\pgfqpoint{3.261185in}{1.850742in}}%
\pgfpathlineto{\pgfqpoint{3.265726in}{1.850742in}}%
\pgfpathlineto{\pgfqpoint{3.265726in}{1.847793in}}%
\pgfpathmoveto{\pgfqpoint{3.256644in}{1.850742in}}%
\pgfpathlineto{\pgfqpoint{3.256644in}{1.850742in}}%
\pgfpathlineto{\pgfqpoint{3.256644in}{1.853691in}}%
\pgfpathlineto{\pgfqpoint{3.261185in}{1.853691in}}%
\pgfpathlineto{\pgfqpoint{3.261185in}{1.850742in}}%
\pgfpathmoveto{\pgfqpoint{3.220317in}{1.877284in}}%
\pgfpathlineto{\pgfqpoint{3.220317in}{1.877284in}}%
\pgfpathlineto{\pgfqpoint{3.220317in}{1.880233in}}%
\pgfpathlineto{\pgfqpoint{3.224858in}{1.880233in}}%
\pgfpathlineto{\pgfqpoint{3.224858in}{1.877284in}}%
\pgfpathmoveto{\pgfqpoint{3.224858in}{1.874335in}}%
\pgfpathlineto{\pgfqpoint{3.224858in}{1.874335in}}%
\pgfpathlineto{\pgfqpoint{3.224858in}{1.877284in}}%
\pgfpathlineto{\pgfqpoint{3.229399in}{1.877284in}}%
\pgfpathlineto{\pgfqpoint{3.229399in}{1.874335in}}%
\pgfpathmoveto{\pgfqpoint{3.224858in}{1.877284in}}%
\pgfpathlineto{\pgfqpoint{3.224858in}{1.877284in}}%
\pgfpathlineto{\pgfqpoint{3.224858in}{1.880233in}}%
\pgfpathlineto{\pgfqpoint{3.229399in}{1.880233in}}%
\pgfpathlineto{\pgfqpoint{3.229399in}{1.877284in}}%
\pgfpathmoveto{\pgfqpoint{3.229399in}{1.871386in}}%
\pgfpathlineto{\pgfqpoint{3.229399in}{1.871386in}}%
\pgfpathlineto{\pgfqpoint{3.229399in}{1.874335in}}%
\pgfpathlineto{\pgfqpoint{3.233940in}{1.874335in}}%
\pgfpathlineto{\pgfqpoint{3.233940in}{1.871386in}}%
\pgfpathmoveto{\pgfqpoint{3.233940in}{1.868437in}}%
\pgfpathlineto{\pgfqpoint{3.233940in}{1.868437in}}%
\pgfpathlineto{\pgfqpoint{3.233940in}{1.871386in}}%
\pgfpathlineto{\pgfqpoint{3.238481in}{1.871386in}}%
\pgfpathlineto{\pgfqpoint{3.238481in}{1.868437in}}%
\pgfpathmoveto{\pgfqpoint{3.233940in}{1.871386in}}%
\pgfpathlineto{\pgfqpoint{3.233940in}{1.871386in}}%
\pgfpathlineto{\pgfqpoint{3.233940in}{1.874335in}}%
\pgfpathlineto{\pgfqpoint{3.238481in}{1.874335in}}%
\pgfpathlineto{\pgfqpoint{3.238481in}{1.871386in}}%
\pgfpathmoveto{\pgfqpoint{3.229399in}{1.874335in}}%
\pgfpathlineto{\pgfqpoint{3.229399in}{1.874335in}}%
\pgfpathlineto{\pgfqpoint{3.229399in}{1.877284in}}%
\pgfpathlineto{\pgfqpoint{3.233940in}{1.877284in}}%
\pgfpathlineto{\pgfqpoint{3.233940in}{1.874335in}}%
\pgfpathmoveto{\pgfqpoint{3.220317in}{1.880233in}}%
\pgfpathlineto{\pgfqpoint{3.220317in}{1.880233in}}%
\pgfpathlineto{\pgfqpoint{3.220317in}{1.883182in}}%
\pgfpathlineto{\pgfqpoint{3.224858in}{1.883182in}}%
\pgfpathlineto{\pgfqpoint{3.224858in}{1.880233in}}%
\pgfpathmoveto{\pgfqpoint{3.220317in}{1.883182in}}%
\pgfpathlineto{\pgfqpoint{3.220317in}{1.883182in}}%
\pgfpathlineto{\pgfqpoint{3.220317in}{1.886131in}}%
\pgfpathlineto{\pgfqpoint{3.224858in}{1.886131in}}%
\pgfpathlineto{\pgfqpoint{3.224858in}{1.883182in}}%
\pgfpathmoveto{\pgfqpoint{3.433739in}{1.694433in}}%
\pgfpathlineto{\pgfqpoint{3.433739in}{1.694433in}}%
\pgfpathlineto{\pgfqpoint{3.433739in}{1.697382in}}%
\pgfpathlineto{\pgfqpoint{3.438280in}{1.697382in}}%
\pgfpathlineto{\pgfqpoint{3.438280in}{1.694433in}}%
\pgfpathmoveto{\pgfqpoint{3.429198in}{1.697382in}}%
\pgfpathlineto{\pgfqpoint{3.429198in}{1.697382in}}%
\pgfpathlineto{\pgfqpoint{3.429198in}{1.700331in}}%
\pgfpathlineto{\pgfqpoint{3.433739in}{1.700331in}}%
\pgfpathlineto{\pgfqpoint{3.433739in}{1.697382in}}%
\pgfpathmoveto{\pgfqpoint{3.429198in}{1.700331in}}%
\pgfpathlineto{\pgfqpoint{3.429198in}{1.700331in}}%
\pgfpathlineto{\pgfqpoint{3.429198in}{1.703280in}}%
\pgfpathlineto{\pgfqpoint{3.433739in}{1.703280in}}%
\pgfpathlineto{\pgfqpoint{3.433739in}{1.700331in}}%
\pgfpathmoveto{\pgfqpoint{3.433739in}{1.697382in}}%
\pgfpathlineto{\pgfqpoint{3.433739in}{1.697382in}}%
\pgfpathlineto{\pgfqpoint{3.433739in}{1.700331in}}%
\pgfpathlineto{\pgfqpoint{3.438280in}{1.700331in}}%
\pgfpathlineto{\pgfqpoint{3.438280in}{1.697382in}}%
\pgfpathmoveto{\pgfqpoint{3.415576in}{1.709178in}}%
\pgfpathlineto{\pgfqpoint{3.415576in}{1.709178in}}%
\pgfpathlineto{\pgfqpoint{3.415576in}{1.712127in}}%
\pgfpathlineto{\pgfqpoint{3.420117in}{1.712127in}}%
\pgfpathlineto{\pgfqpoint{3.420117in}{1.709178in}}%
\pgfpathmoveto{\pgfqpoint{3.415576in}{1.712127in}}%
\pgfpathlineto{\pgfqpoint{3.415576in}{1.712127in}}%
\pgfpathlineto{\pgfqpoint{3.415576in}{1.715076in}}%
\pgfpathlineto{\pgfqpoint{3.420117in}{1.715076in}}%
\pgfpathlineto{\pgfqpoint{3.420117in}{1.712127in}}%
\pgfpathmoveto{\pgfqpoint{3.406494in}{1.718025in}}%
\pgfpathlineto{\pgfqpoint{3.406494in}{1.718025in}}%
\pgfpathlineto{\pgfqpoint{3.406494in}{1.720974in}}%
\pgfpathlineto{\pgfqpoint{3.411035in}{1.720974in}}%
\pgfpathlineto{\pgfqpoint{3.411035in}{1.718025in}}%
\pgfpathmoveto{\pgfqpoint{3.401953in}{1.720974in}}%
\pgfpathlineto{\pgfqpoint{3.401953in}{1.720974in}}%
\pgfpathlineto{\pgfqpoint{3.401953in}{1.723923in}}%
\pgfpathlineto{\pgfqpoint{3.406494in}{1.723923in}}%
\pgfpathlineto{\pgfqpoint{3.406494in}{1.720974in}}%
\pgfpathmoveto{\pgfqpoint{3.401953in}{1.723923in}}%
\pgfpathlineto{\pgfqpoint{3.401953in}{1.723923in}}%
\pgfpathlineto{\pgfqpoint{3.401953in}{1.726872in}}%
\pgfpathlineto{\pgfqpoint{3.406494in}{1.726872in}}%
\pgfpathlineto{\pgfqpoint{3.406494in}{1.723923in}}%
\pgfpathmoveto{\pgfqpoint{3.406494in}{1.720974in}}%
\pgfpathlineto{\pgfqpoint{3.406494in}{1.720974in}}%
\pgfpathlineto{\pgfqpoint{3.406494in}{1.723923in}}%
\pgfpathlineto{\pgfqpoint{3.411035in}{1.723923in}}%
\pgfpathlineto{\pgfqpoint{3.411035in}{1.720974in}}%
\pgfpathmoveto{\pgfqpoint{3.411035in}{1.715076in}}%
\pgfpathlineto{\pgfqpoint{3.411035in}{1.715076in}}%
\pgfpathlineto{\pgfqpoint{3.411035in}{1.718025in}}%
\pgfpathlineto{\pgfqpoint{3.415576in}{1.718025in}}%
\pgfpathlineto{\pgfqpoint{3.415576in}{1.715076in}}%
\pgfpathmoveto{\pgfqpoint{3.411035in}{1.718025in}}%
\pgfpathlineto{\pgfqpoint{3.411035in}{1.718025in}}%
\pgfpathlineto{\pgfqpoint{3.411035in}{1.720974in}}%
\pgfpathlineto{\pgfqpoint{3.415576in}{1.720974in}}%
\pgfpathlineto{\pgfqpoint{3.415576in}{1.718025in}}%
\pgfpathmoveto{\pgfqpoint{3.415576in}{1.715076in}}%
\pgfpathlineto{\pgfqpoint{3.415576in}{1.715076in}}%
\pgfpathlineto{\pgfqpoint{3.415576in}{1.718025in}}%
\pgfpathlineto{\pgfqpoint{3.420117in}{1.718025in}}%
\pgfpathlineto{\pgfqpoint{3.420117in}{1.715076in}}%
\pgfpathmoveto{\pgfqpoint{3.420117in}{1.706229in}}%
\pgfpathlineto{\pgfqpoint{3.420117in}{1.706229in}}%
\pgfpathlineto{\pgfqpoint{3.420117in}{1.709178in}}%
\pgfpathlineto{\pgfqpoint{3.424657in}{1.709178in}}%
\pgfpathlineto{\pgfqpoint{3.424657in}{1.706229in}}%
\pgfpathmoveto{\pgfqpoint{3.424657in}{1.703280in}}%
\pgfpathlineto{\pgfqpoint{3.424657in}{1.703280in}}%
\pgfpathlineto{\pgfqpoint{3.424657in}{1.706229in}}%
\pgfpathlineto{\pgfqpoint{3.429198in}{1.706229in}}%
\pgfpathlineto{\pgfqpoint{3.429198in}{1.703280in}}%
\pgfpathmoveto{\pgfqpoint{3.424657in}{1.706229in}}%
\pgfpathlineto{\pgfqpoint{3.424657in}{1.706229in}}%
\pgfpathlineto{\pgfqpoint{3.424657in}{1.709178in}}%
\pgfpathlineto{\pgfqpoint{3.429198in}{1.709178in}}%
\pgfpathlineto{\pgfqpoint{3.429198in}{1.706229in}}%
\pgfpathmoveto{\pgfqpoint{3.420117in}{1.709178in}}%
\pgfpathlineto{\pgfqpoint{3.420117in}{1.709178in}}%
\pgfpathlineto{\pgfqpoint{3.420117in}{1.712127in}}%
\pgfpathlineto{\pgfqpoint{3.424657in}{1.712127in}}%
\pgfpathlineto{\pgfqpoint{3.424657in}{1.709178in}}%
\pgfpathmoveto{\pgfqpoint{3.429198in}{1.703280in}}%
\pgfpathlineto{\pgfqpoint{3.429198in}{1.703280in}}%
\pgfpathlineto{\pgfqpoint{3.429198in}{1.706229in}}%
\pgfpathlineto{\pgfqpoint{3.433739in}{1.706229in}}%
\pgfpathlineto{\pgfqpoint{3.433739in}{1.703280in}}%
\pgfpathmoveto{\pgfqpoint{3.470066in}{1.661993in}}%
\pgfpathlineto{\pgfqpoint{3.470066in}{1.661993in}}%
\pgfpathlineto{\pgfqpoint{3.470066in}{1.664942in}}%
\pgfpathlineto{\pgfqpoint{3.474607in}{1.664942in}}%
\pgfpathlineto{\pgfqpoint{3.474607in}{1.661993in}}%
\pgfpathmoveto{\pgfqpoint{3.470066in}{1.664942in}}%
\pgfpathlineto{\pgfqpoint{3.470066in}{1.664942in}}%
\pgfpathlineto{\pgfqpoint{3.470066in}{1.667891in}}%
\pgfpathlineto{\pgfqpoint{3.474607in}{1.667891in}}%
\pgfpathlineto{\pgfqpoint{3.474607in}{1.664942in}}%
\pgfpathmoveto{\pgfqpoint{3.460985in}{1.670840in}}%
\pgfpathlineto{\pgfqpoint{3.460985in}{1.670840in}}%
\pgfpathlineto{\pgfqpoint{3.460985in}{1.673789in}}%
\pgfpathlineto{\pgfqpoint{3.465525in}{1.673789in}}%
\pgfpathlineto{\pgfqpoint{3.465525in}{1.670840in}}%
\pgfpathmoveto{\pgfqpoint{3.456444in}{1.673789in}}%
\pgfpathlineto{\pgfqpoint{3.456444in}{1.673789in}}%
\pgfpathlineto{\pgfqpoint{3.456444in}{1.676738in}}%
\pgfpathlineto{\pgfqpoint{3.460985in}{1.676738in}}%
\pgfpathlineto{\pgfqpoint{3.460985in}{1.673789in}}%
\pgfpathmoveto{\pgfqpoint{3.456444in}{1.676738in}}%
\pgfpathlineto{\pgfqpoint{3.456444in}{1.676738in}}%
\pgfpathlineto{\pgfqpoint{3.456444in}{1.679687in}}%
\pgfpathlineto{\pgfqpoint{3.460985in}{1.679687in}}%
\pgfpathlineto{\pgfqpoint{3.460985in}{1.676738in}}%
\pgfpathmoveto{\pgfqpoint{3.460985in}{1.673789in}}%
\pgfpathlineto{\pgfqpoint{3.460985in}{1.673789in}}%
\pgfpathlineto{\pgfqpoint{3.460985in}{1.676738in}}%
\pgfpathlineto{\pgfqpoint{3.465525in}{1.676738in}}%
\pgfpathlineto{\pgfqpoint{3.465525in}{1.673789in}}%
\pgfpathmoveto{\pgfqpoint{3.465525in}{1.667891in}}%
\pgfpathlineto{\pgfqpoint{3.465525in}{1.667891in}}%
\pgfpathlineto{\pgfqpoint{3.465525in}{1.670840in}}%
\pgfpathlineto{\pgfqpoint{3.470066in}{1.670840in}}%
\pgfpathlineto{\pgfqpoint{3.470066in}{1.667891in}}%
\pgfpathmoveto{\pgfqpoint{3.465525in}{1.670840in}}%
\pgfpathlineto{\pgfqpoint{3.465525in}{1.670840in}}%
\pgfpathlineto{\pgfqpoint{3.465525in}{1.673789in}}%
\pgfpathlineto{\pgfqpoint{3.470066in}{1.673789in}}%
\pgfpathlineto{\pgfqpoint{3.470066in}{1.670840in}}%
\pgfpathmoveto{\pgfqpoint{3.470066in}{1.667891in}}%
\pgfpathlineto{\pgfqpoint{3.470066in}{1.667891in}}%
\pgfpathlineto{\pgfqpoint{3.470066in}{1.670840in}}%
\pgfpathlineto{\pgfqpoint{3.474607in}{1.670840in}}%
\pgfpathlineto{\pgfqpoint{3.474607in}{1.667891in}}%
\pgfpathmoveto{\pgfqpoint{3.488230in}{1.647248in}}%
\pgfpathlineto{\pgfqpoint{3.488230in}{1.647248in}}%
\pgfpathlineto{\pgfqpoint{3.488230in}{1.650197in}}%
\pgfpathlineto{\pgfqpoint{3.492771in}{1.650197in}}%
\pgfpathlineto{\pgfqpoint{3.492771in}{1.647248in}}%
\pgfpathmoveto{\pgfqpoint{3.483689in}{1.650197in}}%
\pgfpathlineto{\pgfqpoint{3.483689in}{1.650197in}}%
\pgfpathlineto{\pgfqpoint{3.483689in}{1.653146in}}%
\pgfpathlineto{\pgfqpoint{3.488230in}{1.653146in}}%
\pgfpathlineto{\pgfqpoint{3.488230in}{1.650197in}}%
\pgfpathmoveto{\pgfqpoint{3.483689in}{1.653146in}}%
\pgfpathlineto{\pgfqpoint{3.483689in}{1.653146in}}%
\pgfpathlineto{\pgfqpoint{3.483689in}{1.656095in}}%
\pgfpathlineto{\pgfqpoint{3.488230in}{1.656095in}}%
\pgfpathlineto{\pgfqpoint{3.488230in}{1.653146in}}%
\pgfpathmoveto{\pgfqpoint{3.488230in}{1.650197in}}%
\pgfpathlineto{\pgfqpoint{3.488230in}{1.650197in}}%
\pgfpathlineto{\pgfqpoint{3.488230in}{1.653146in}}%
\pgfpathlineto{\pgfqpoint{3.492771in}{1.653146in}}%
\pgfpathlineto{\pgfqpoint{3.492771in}{1.650197in}}%
\pgfpathmoveto{\pgfqpoint{3.497312in}{1.638401in}}%
\pgfpathlineto{\pgfqpoint{3.497312in}{1.638401in}}%
\pgfpathlineto{\pgfqpoint{3.497312in}{1.641350in}}%
\pgfpathlineto{\pgfqpoint{3.501853in}{1.641350in}}%
\pgfpathlineto{\pgfqpoint{3.501853in}{1.638401in}}%
\pgfpathmoveto{\pgfqpoint{3.497312in}{1.641350in}}%
\pgfpathlineto{\pgfqpoint{3.497312in}{1.641350in}}%
\pgfpathlineto{\pgfqpoint{3.497312in}{1.644299in}}%
\pgfpathlineto{\pgfqpoint{3.501853in}{1.644299in}}%
\pgfpathlineto{\pgfqpoint{3.501853in}{1.641350in}}%
\pgfpathmoveto{\pgfqpoint{3.501853in}{1.635452in}}%
\pgfpathlineto{\pgfqpoint{3.501853in}{1.635452in}}%
\pgfpathlineto{\pgfqpoint{3.501853in}{1.638401in}}%
\pgfpathlineto{\pgfqpoint{3.506393in}{1.638401in}}%
\pgfpathlineto{\pgfqpoint{3.506393in}{1.635452in}}%
\pgfpathmoveto{\pgfqpoint{3.506393in}{1.632502in}}%
\pgfpathlineto{\pgfqpoint{3.506393in}{1.632502in}}%
\pgfpathlineto{\pgfqpoint{3.506393in}{1.635452in}}%
\pgfpathlineto{\pgfqpoint{3.510934in}{1.635452in}}%
\pgfpathlineto{\pgfqpoint{3.510934in}{1.632502in}}%
\pgfpathmoveto{\pgfqpoint{3.506393in}{1.635452in}}%
\pgfpathlineto{\pgfqpoint{3.506393in}{1.635452in}}%
\pgfpathlineto{\pgfqpoint{3.506393in}{1.638401in}}%
\pgfpathlineto{\pgfqpoint{3.510934in}{1.638401in}}%
\pgfpathlineto{\pgfqpoint{3.510934in}{1.635452in}}%
\pgfpathmoveto{\pgfqpoint{3.501853in}{1.638401in}}%
\pgfpathlineto{\pgfqpoint{3.501853in}{1.638401in}}%
\pgfpathlineto{\pgfqpoint{3.501853in}{1.641350in}}%
\pgfpathlineto{\pgfqpoint{3.506393in}{1.641350in}}%
\pgfpathlineto{\pgfqpoint{3.506393in}{1.638401in}}%
\pgfpathmoveto{\pgfqpoint{3.492771in}{1.644299in}}%
\pgfpathlineto{\pgfqpoint{3.492771in}{1.644299in}}%
\pgfpathlineto{\pgfqpoint{3.492771in}{1.647248in}}%
\pgfpathlineto{\pgfqpoint{3.497312in}{1.647248in}}%
\pgfpathlineto{\pgfqpoint{3.497312in}{1.644299in}}%
\pgfpathmoveto{\pgfqpoint{3.492771in}{1.647248in}}%
\pgfpathlineto{\pgfqpoint{3.492771in}{1.647248in}}%
\pgfpathlineto{\pgfqpoint{3.492771in}{1.650197in}}%
\pgfpathlineto{\pgfqpoint{3.497312in}{1.650197in}}%
\pgfpathlineto{\pgfqpoint{3.497312in}{1.647248in}}%
\pgfpathmoveto{\pgfqpoint{3.497312in}{1.644299in}}%
\pgfpathlineto{\pgfqpoint{3.497312in}{1.644299in}}%
\pgfpathlineto{\pgfqpoint{3.497312in}{1.647248in}}%
\pgfpathlineto{\pgfqpoint{3.501853in}{1.647248in}}%
\pgfpathlineto{\pgfqpoint{3.501853in}{1.644299in}}%
\pgfpathmoveto{\pgfqpoint{3.474607in}{1.659044in}}%
\pgfpathlineto{\pgfqpoint{3.474607in}{1.659044in}}%
\pgfpathlineto{\pgfqpoint{3.474607in}{1.661993in}}%
\pgfpathlineto{\pgfqpoint{3.479148in}{1.661993in}}%
\pgfpathlineto{\pgfqpoint{3.479148in}{1.659044in}}%
\pgfpathmoveto{\pgfqpoint{3.479148in}{1.656095in}}%
\pgfpathlineto{\pgfqpoint{3.479148in}{1.656095in}}%
\pgfpathlineto{\pgfqpoint{3.479148in}{1.659044in}}%
\pgfpathlineto{\pgfqpoint{3.483689in}{1.659044in}}%
\pgfpathlineto{\pgfqpoint{3.483689in}{1.656095in}}%
\pgfpathmoveto{\pgfqpoint{3.479148in}{1.659044in}}%
\pgfpathlineto{\pgfqpoint{3.479148in}{1.659044in}}%
\pgfpathlineto{\pgfqpoint{3.479148in}{1.661993in}}%
\pgfpathlineto{\pgfqpoint{3.483689in}{1.661993in}}%
\pgfpathlineto{\pgfqpoint{3.483689in}{1.659044in}}%
\pgfpathmoveto{\pgfqpoint{3.474607in}{1.661993in}}%
\pgfpathlineto{\pgfqpoint{3.474607in}{1.661993in}}%
\pgfpathlineto{\pgfqpoint{3.474607in}{1.664942in}}%
\pgfpathlineto{\pgfqpoint{3.479148in}{1.664942in}}%
\pgfpathlineto{\pgfqpoint{3.479148in}{1.661993in}}%
\pgfpathmoveto{\pgfqpoint{3.483689in}{1.656095in}}%
\pgfpathlineto{\pgfqpoint{3.483689in}{1.656095in}}%
\pgfpathlineto{\pgfqpoint{3.483689in}{1.659044in}}%
\pgfpathlineto{\pgfqpoint{3.488230in}{1.659044in}}%
\pgfpathlineto{\pgfqpoint{3.488230in}{1.656095in}}%
\pgfpathmoveto{\pgfqpoint{3.442821in}{1.685585in}}%
\pgfpathlineto{\pgfqpoint{3.442821in}{1.685585in}}%
\pgfpathlineto{\pgfqpoint{3.442821in}{1.688534in}}%
\pgfpathlineto{\pgfqpoint{3.447362in}{1.688534in}}%
\pgfpathlineto{\pgfqpoint{3.447362in}{1.685585in}}%
\pgfpathmoveto{\pgfqpoint{3.442821in}{1.688534in}}%
\pgfpathlineto{\pgfqpoint{3.442821in}{1.688534in}}%
\pgfpathlineto{\pgfqpoint{3.442821in}{1.691484in}}%
\pgfpathlineto{\pgfqpoint{3.447362in}{1.691484in}}%
\pgfpathlineto{\pgfqpoint{3.447362in}{1.688534in}}%
\pgfpathmoveto{\pgfqpoint{3.447362in}{1.682636in}}%
\pgfpathlineto{\pgfqpoint{3.447362in}{1.682636in}}%
\pgfpathlineto{\pgfqpoint{3.447362in}{1.685585in}}%
\pgfpathlineto{\pgfqpoint{3.451903in}{1.685585in}}%
\pgfpathlineto{\pgfqpoint{3.451903in}{1.682636in}}%
\pgfpathmoveto{\pgfqpoint{3.451903in}{1.679687in}}%
\pgfpathlineto{\pgfqpoint{3.451903in}{1.679687in}}%
\pgfpathlineto{\pgfqpoint{3.451903in}{1.682636in}}%
\pgfpathlineto{\pgfqpoint{3.456444in}{1.682636in}}%
\pgfpathlineto{\pgfqpoint{3.456444in}{1.679687in}}%
\pgfpathmoveto{\pgfqpoint{3.451903in}{1.682636in}}%
\pgfpathlineto{\pgfqpoint{3.451903in}{1.682636in}}%
\pgfpathlineto{\pgfqpoint{3.451903in}{1.685585in}}%
\pgfpathlineto{\pgfqpoint{3.456444in}{1.685585in}}%
\pgfpathlineto{\pgfqpoint{3.456444in}{1.682636in}}%
\pgfpathmoveto{\pgfqpoint{3.447362in}{1.685585in}}%
\pgfpathlineto{\pgfqpoint{3.447362in}{1.685585in}}%
\pgfpathlineto{\pgfqpoint{3.447362in}{1.688534in}}%
\pgfpathlineto{\pgfqpoint{3.451903in}{1.688534in}}%
\pgfpathlineto{\pgfqpoint{3.451903in}{1.685585in}}%
\pgfpathmoveto{\pgfqpoint{3.438280in}{1.691484in}}%
\pgfpathlineto{\pgfqpoint{3.438280in}{1.691484in}}%
\pgfpathlineto{\pgfqpoint{3.438280in}{1.694433in}}%
\pgfpathlineto{\pgfqpoint{3.442821in}{1.694433in}}%
\pgfpathlineto{\pgfqpoint{3.442821in}{1.691484in}}%
\pgfpathmoveto{\pgfqpoint{3.438280in}{1.694433in}}%
\pgfpathlineto{\pgfqpoint{3.438280in}{1.694433in}}%
\pgfpathlineto{\pgfqpoint{3.438280in}{1.697382in}}%
\pgfpathlineto{\pgfqpoint{3.442821in}{1.697382in}}%
\pgfpathlineto{\pgfqpoint{3.442821in}{1.694433in}}%
\pgfpathmoveto{\pgfqpoint{3.442821in}{1.691484in}}%
\pgfpathlineto{\pgfqpoint{3.442821in}{1.691484in}}%
\pgfpathlineto{\pgfqpoint{3.442821in}{1.694433in}}%
\pgfpathlineto{\pgfqpoint{3.447362in}{1.694433in}}%
\pgfpathlineto{\pgfqpoint{3.447362in}{1.691484in}}%
\pgfpathmoveto{\pgfqpoint{3.456444in}{1.679687in}}%
\pgfpathlineto{\pgfqpoint{3.456444in}{1.679687in}}%
\pgfpathlineto{\pgfqpoint{3.456444in}{1.682636in}}%
\pgfpathlineto{\pgfqpoint{3.460985in}{1.682636in}}%
\pgfpathlineto{\pgfqpoint{3.460985in}{1.679687in}}%
\pgfpathmoveto{\pgfqpoint{3.379249in}{1.741619in}}%
\pgfpathlineto{\pgfqpoint{3.379249in}{1.741619in}}%
\pgfpathlineto{\pgfqpoint{3.379249in}{1.744568in}}%
\pgfpathlineto{\pgfqpoint{3.383789in}{1.744568in}}%
\pgfpathlineto{\pgfqpoint{3.383789in}{1.741619in}}%
\pgfpathmoveto{\pgfqpoint{3.374708in}{1.744568in}}%
\pgfpathlineto{\pgfqpoint{3.374708in}{1.744568in}}%
\pgfpathlineto{\pgfqpoint{3.374708in}{1.747518in}}%
\pgfpathlineto{\pgfqpoint{3.379249in}{1.747518in}}%
\pgfpathlineto{\pgfqpoint{3.379249in}{1.744568in}}%
\pgfpathmoveto{\pgfqpoint{3.374708in}{1.747518in}}%
\pgfpathlineto{\pgfqpoint{3.374708in}{1.747518in}}%
\pgfpathlineto{\pgfqpoint{3.374708in}{1.750467in}}%
\pgfpathlineto{\pgfqpoint{3.379249in}{1.750467in}}%
\pgfpathlineto{\pgfqpoint{3.379249in}{1.747518in}}%
\pgfpathmoveto{\pgfqpoint{3.379249in}{1.744568in}}%
\pgfpathlineto{\pgfqpoint{3.379249in}{1.744568in}}%
\pgfpathlineto{\pgfqpoint{3.379249in}{1.747518in}}%
\pgfpathlineto{\pgfqpoint{3.383789in}{1.747518in}}%
\pgfpathlineto{\pgfqpoint{3.383789in}{1.744568in}}%
\pgfpathmoveto{\pgfqpoint{3.388330in}{1.732771in}}%
\pgfpathlineto{\pgfqpoint{3.388330in}{1.732771in}}%
\pgfpathlineto{\pgfqpoint{3.388330in}{1.735720in}}%
\pgfpathlineto{\pgfqpoint{3.392871in}{1.735720in}}%
\pgfpathlineto{\pgfqpoint{3.392871in}{1.732771in}}%
\pgfpathmoveto{\pgfqpoint{3.388330in}{1.735720in}}%
\pgfpathlineto{\pgfqpoint{3.388330in}{1.735720in}}%
\pgfpathlineto{\pgfqpoint{3.388330in}{1.738670in}}%
\pgfpathlineto{\pgfqpoint{3.392871in}{1.738670in}}%
\pgfpathlineto{\pgfqpoint{3.392871in}{1.735720in}}%
\pgfpathmoveto{\pgfqpoint{3.392871in}{1.729822in}}%
\pgfpathlineto{\pgfqpoint{3.392871in}{1.729822in}}%
\pgfpathlineto{\pgfqpoint{3.392871in}{1.732771in}}%
\pgfpathlineto{\pgfqpoint{3.397412in}{1.732771in}}%
\pgfpathlineto{\pgfqpoint{3.397412in}{1.729822in}}%
\pgfpathmoveto{\pgfqpoint{3.397412in}{1.726872in}}%
\pgfpathlineto{\pgfqpoint{3.397412in}{1.726872in}}%
\pgfpathlineto{\pgfqpoint{3.397412in}{1.729822in}}%
\pgfpathlineto{\pgfqpoint{3.401953in}{1.729822in}}%
\pgfpathlineto{\pgfqpoint{3.401953in}{1.726872in}}%
\pgfpathmoveto{\pgfqpoint{3.397412in}{1.729822in}}%
\pgfpathlineto{\pgfqpoint{3.397412in}{1.729822in}}%
\pgfpathlineto{\pgfqpoint{3.397412in}{1.732771in}}%
\pgfpathlineto{\pgfqpoint{3.401953in}{1.732771in}}%
\pgfpathlineto{\pgfqpoint{3.401953in}{1.729822in}}%
\pgfpathmoveto{\pgfqpoint{3.392871in}{1.732771in}}%
\pgfpathlineto{\pgfqpoint{3.392871in}{1.732771in}}%
\pgfpathlineto{\pgfqpoint{3.392871in}{1.735720in}}%
\pgfpathlineto{\pgfqpoint{3.397412in}{1.735720in}}%
\pgfpathlineto{\pgfqpoint{3.397412in}{1.732771in}}%
\pgfpathmoveto{\pgfqpoint{3.383789in}{1.738670in}}%
\pgfpathlineto{\pgfqpoint{3.383789in}{1.738670in}}%
\pgfpathlineto{\pgfqpoint{3.383789in}{1.741619in}}%
\pgfpathlineto{\pgfqpoint{3.388330in}{1.741619in}}%
\pgfpathlineto{\pgfqpoint{3.388330in}{1.738670in}}%
\pgfpathmoveto{\pgfqpoint{3.383789in}{1.741619in}}%
\pgfpathlineto{\pgfqpoint{3.383789in}{1.741619in}}%
\pgfpathlineto{\pgfqpoint{3.383789in}{1.744568in}}%
\pgfpathlineto{\pgfqpoint{3.388330in}{1.744568in}}%
\pgfpathlineto{\pgfqpoint{3.388330in}{1.741619in}}%
\pgfpathmoveto{\pgfqpoint{3.388330in}{1.738670in}}%
\pgfpathlineto{\pgfqpoint{3.388330in}{1.738670in}}%
\pgfpathlineto{\pgfqpoint{3.388330in}{1.741619in}}%
\pgfpathlineto{\pgfqpoint{3.392871in}{1.741619in}}%
\pgfpathlineto{\pgfqpoint{3.392871in}{1.738670in}}%
\pgfpathmoveto{\pgfqpoint{3.365626in}{1.753416in}}%
\pgfpathlineto{\pgfqpoint{3.365626in}{1.753416in}}%
\pgfpathlineto{\pgfqpoint{3.365626in}{1.756366in}}%
\pgfpathlineto{\pgfqpoint{3.370167in}{1.756366in}}%
\pgfpathlineto{\pgfqpoint{3.370167in}{1.753416in}}%
\pgfpathmoveto{\pgfqpoint{3.370167in}{1.750467in}}%
\pgfpathlineto{\pgfqpoint{3.370167in}{1.750467in}}%
\pgfpathlineto{\pgfqpoint{3.370167in}{1.753416in}}%
\pgfpathlineto{\pgfqpoint{3.374708in}{1.753416in}}%
\pgfpathlineto{\pgfqpoint{3.374708in}{1.750467in}}%
\pgfpathmoveto{\pgfqpoint{3.370167in}{1.753416in}}%
\pgfpathlineto{\pgfqpoint{3.370167in}{1.753416in}}%
\pgfpathlineto{\pgfqpoint{3.370167in}{1.756366in}}%
\pgfpathlineto{\pgfqpoint{3.374708in}{1.756366in}}%
\pgfpathlineto{\pgfqpoint{3.374708in}{1.753416in}}%
\pgfpathmoveto{\pgfqpoint{3.365626in}{1.756366in}}%
\pgfpathlineto{\pgfqpoint{3.365626in}{1.756366in}}%
\pgfpathlineto{\pgfqpoint{3.365626in}{1.759315in}}%
\pgfpathlineto{\pgfqpoint{3.370167in}{1.759315in}}%
\pgfpathlineto{\pgfqpoint{3.370167in}{1.756366in}}%
\pgfpathmoveto{\pgfqpoint{3.374708in}{1.750467in}}%
\pgfpathlineto{\pgfqpoint{3.374708in}{1.750467in}}%
\pgfpathlineto{\pgfqpoint{3.374708in}{1.753416in}}%
\pgfpathlineto{\pgfqpoint{3.379249in}{1.753416in}}%
\pgfpathlineto{\pgfqpoint{3.379249in}{1.750467in}}%
\pgfpathmoveto{\pgfqpoint{3.401953in}{1.726872in}}%
\pgfpathlineto{\pgfqpoint{3.401953in}{1.726872in}}%
\pgfpathlineto{\pgfqpoint{3.401953in}{1.729822in}}%
\pgfpathlineto{\pgfqpoint{3.406494in}{1.729822in}}%
\pgfpathlineto{\pgfqpoint{3.406494in}{1.726872in}}%
\pgfpathmoveto{\pgfqpoint{3.651709in}{1.505683in}}%
\pgfpathlineto{\pgfqpoint{3.651709in}{1.505683in}}%
\pgfpathlineto{\pgfqpoint{3.651709in}{1.508632in}}%
\pgfpathlineto{\pgfqpoint{3.656251in}{1.508632in}}%
\pgfpathlineto{\pgfqpoint{3.656251in}{1.505683in}}%
\pgfpathmoveto{\pgfqpoint{3.647168in}{1.508632in}}%
\pgfpathlineto{\pgfqpoint{3.647168in}{1.508632in}}%
\pgfpathlineto{\pgfqpoint{3.647168in}{1.511581in}}%
\pgfpathlineto{\pgfqpoint{3.651709in}{1.511581in}}%
\pgfpathlineto{\pgfqpoint{3.651709in}{1.508632in}}%
\pgfpathmoveto{\pgfqpoint{3.647168in}{1.511581in}}%
\pgfpathlineto{\pgfqpoint{3.647168in}{1.511581in}}%
\pgfpathlineto{\pgfqpoint{3.647168in}{1.514531in}}%
\pgfpathlineto{\pgfqpoint{3.651709in}{1.514531in}}%
\pgfpathlineto{\pgfqpoint{3.651709in}{1.511581in}}%
\pgfpathmoveto{\pgfqpoint{3.651709in}{1.508632in}}%
\pgfpathlineto{\pgfqpoint{3.651709in}{1.508632in}}%
\pgfpathlineto{\pgfqpoint{3.651709in}{1.511581in}}%
\pgfpathlineto{\pgfqpoint{3.656251in}{1.511581in}}%
\pgfpathlineto{\pgfqpoint{3.656251in}{1.508632in}}%
\pgfpathmoveto{\pgfqpoint{3.633545in}{1.520429in}}%
\pgfpathlineto{\pgfqpoint{3.633545in}{1.520429in}}%
\pgfpathlineto{\pgfqpoint{3.633545in}{1.523378in}}%
\pgfpathlineto{\pgfqpoint{3.638086in}{1.523378in}}%
\pgfpathlineto{\pgfqpoint{3.638086in}{1.520429in}}%
\pgfpathmoveto{\pgfqpoint{3.633545in}{1.523378in}}%
\pgfpathlineto{\pgfqpoint{3.633545in}{1.523378in}}%
\pgfpathlineto{\pgfqpoint{3.633545in}{1.526328in}}%
\pgfpathlineto{\pgfqpoint{3.638086in}{1.526328in}}%
\pgfpathlineto{\pgfqpoint{3.638086in}{1.523378in}}%
\pgfpathmoveto{\pgfqpoint{3.624463in}{1.529277in}}%
\pgfpathlineto{\pgfqpoint{3.624463in}{1.529277in}}%
\pgfpathlineto{\pgfqpoint{3.624463in}{1.532226in}}%
\pgfpathlineto{\pgfqpoint{3.629004in}{1.532226in}}%
\pgfpathlineto{\pgfqpoint{3.629004in}{1.529277in}}%
\pgfpathmoveto{\pgfqpoint{3.619922in}{1.532226in}}%
\pgfpathlineto{\pgfqpoint{3.619922in}{1.532226in}}%
\pgfpathlineto{\pgfqpoint{3.619922in}{1.535175in}}%
\pgfpathlineto{\pgfqpoint{3.624463in}{1.535175in}}%
\pgfpathlineto{\pgfqpoint{3.624463in}{1.532226in}}%
\pgfpathmoveto{\pgfqpoint{3.619922in}{1.535175in}}%
\pgfpathlineto{\pgfqpoint{3.619922in}{1.535175in}}%
\pgfpathlineto{\pgfqpoint{3.619922in}{1.538124in}}%
\pgfpathlineto{\pgfqpoint{3.624463in}{1.538124in}}%
\pgfpathlineto{\pgfqpoint{3.624463in}{1.535175in}}%
\pgfpathmoveto{\pgfqpoint{3.624463in}{1.532226in}}%
\pgfpathlineto{\pgfqpoint{3.624463in}{1.532226in}}%
\pgfpathlineto{\pgfqpoint{3.624463in}{1.535175in}}%
\pgfpathlineto{\pgfqpoint{3.629004in}{1.535175in}}%
\pgfpathlineto{\pgfqpoint{3.629004in}{1.532226in}}%
\pgfpathmoveto{\pgfqpoint{3.629004in}{1.526328in}}%
\pgfpathlineto{\pgfqpoint{3.629004in}{1.526328in}}%
\pgfpathlineto{\pgfqpoint{3.629004in}{1.529277in}}%
\pgfpathlineto{\pgfqpoint{3.633545in}{1.529277in}}%
\pgfpathlineto{\pgfqpoint{3.633545in}{1.526328in}}%
\pgfpathmoveto{\pgfqpoint{3.629004in}{1.529277in}}%
\pgfpathlineto{\pgfqpoint{3.629004in}{1.529277in}}%
\pgfpathlineto{\pgfqpoint{3.629004in}{1.532226in}}%
\pgfpathlineto{\pgfqpoint{3.633545in}{1.532226in}}%
\pgfpathlineto{\pgfqpoint{3.633545in}{1.529277in}}%
\pgfpathmoveto{\pgfqpoint{3.633545in}{1.526328in}}%
\pgfpathlineto{\pgfqpoint{3.633545in}{1.526328in}}%
\pgfpathlineto{\pgfqpoint{3.633545in}{1.529277in}}%
\pgfpathlineto{\pgfqpoint{3.638086in}{1.529277in}}%
\pgfpathlineto{\pgfqpoint{3.638086in}{1.526328in}}%
\pgfpathmoveto{\pgfqpoint{3.638086in}{1.517480in}}%
\pgfpathlineto{\pgfqpoint{3.638086in}{1.517480in}}%
\pgfpathlineto{\pgfqpoint{3.638086in}{1.520429in}}%
\pgfpathlineto{\pgfqpoint{3.642627in}{1.520429in}}%
\pgfpathlineto{\pgfqpoint{3.642627in}{1.517480in}}%
\pgfpathmoveto{\pgfqpoint{3.642627in}{1.514531in}}%
\pgfpathlineto{\pgfqpoint{3.642627in}{1.514531in}}%
\pgfpathlineto{\pgfqpoint{3.642627in}{1.517480in}}%
\pgfpathlineto{\pgfqpoint{3.647168in}{1.517480in}}%
\pgfpathlineto{\pgfqpoint{3.647168in}{1.514531in}}%
\pgfpathmoveto{\pgfqpoint{3.642627in}{1.517480in}}%
\pgfpathlineto{\pgfqpoint{3.642627in}{1.517480in}}%
\pgfpathlineto{\pgfqpoint{3.642627in}{1.520429in}}%
\pgfpathlineto{\pgfqpoint{3.647168in}{1.520429in}}%
\pgfpathlineto{\pgfqpoint{3.647168in}{1.517480in}}%
\pgfpathmoveto{\pgfqpoint{3.638086in}{1.520429in}}%
\pgfpathlineto{\pgfqpoint{3.638086in}{1.520429in}}%
\pgfpathlineto{\pgfqpoint{3.638086in}{1.523378in}}%
\pgfpathlineto{\pgfqpoint{3.642627in}{1.523378in}}%
\pgfpathlineto{\pgfqpoint{3.642627in}{1.520429in}}%
\pgfpathmoveto{\pgfqpoint{3.647168in}{1.514531in}}%
\pgfpathlineto{\pgfqpoint{3.647168in}{1.514531in}}%
\pgfpathlineto{\pgfqpoint{3.647168in}{1.517480in}}%
\pgfpathlineto{\pgfqpoint{3.651709in}{1.517480in}}%
\pgfpathlineto{\pgfqpoint{3.651709in}{1.514531in}}%
\pgfpathmoveto{\pgfqpoint{3.579051in}{1.567618in}}%
\pgfpathlineto{\pgfqpoint{3.579051in}{1.567618in}}%
\pgfpathlineto{\pgfqpoint{3.579051in}{1.570567in}}%
\pgfpathlineto{\pgfqpoint{3.583592in}{1.570567in}}%
\pgfpathlineto{\pgfqpoint{3.583592in}{1.567618in}}%
\pgfpathmoveto{\pgfqpoint{3.579051in}{1.570567in}}%
\pgfpathlineto{\pgfqpoint{3.579051in}{1.570567in}}%
\pgfpathlineto{\pgfqpoint{3.579051in}{1.573516in}}%
\pgfpathlineto{\pgfqpoint{3.583592in}{1.573516in}}%
\pgfpathlineto{\pgfqpoint{3.583592in}{1.570567in}}%
\pgfpathmoveto{\pgfqpoint{3.569969in}{1.576466in}}%
\pgfpathlineto{\pgfqpoint{3.569969in}{1.576466in}}%
\pgfpathlineto{\pgfqpoint{3.569969in}{1.579415in}}%
\pgfpathlineto{\pgfqpoint{3.574510in}{1.579415in}}%
\pgfpathlineto{\pgfqpoint{3.574510in}{1.576466in}}%
\pgfpathmoveto{\pgfqpoint{3.565428in}{1.579415in}}%
\pgfpathlineto{\pgfqpoint{3.565428in}{1.579415in}}%
\pgfpathlineto{\pgfqpoint{3.565428in}{1.582364in}}%
\pgfpathlineto{\pgfqpoint{3.569969in}{1.582364in}}%
\pgfpathlineto{\pgfqpoint{3.569969in}{1.579415in}}%
\pgfpathmoveto{\pgfqpoint{3.565428in}{1.582364in}}%
\pgfpathlineto{\pgfqpoint{3.565428in}{1.582364in}}%
\pgfpathlineto{\pgfqpoint{3.565428in}{1.585313in}}%
\pgfpathlineto{\pgfqpoint{3.569969in}{1.585313in}}%
\pgfpathlineto{\pgfqpoint{3.569969in}{1.582364in}}%
\pgfpathmoveto{\pgfqpoint{3.569969in}{1.579415in}}%
\pgfpathlineto{\pgfqpoint{3.569969in}{1.579415in}}%
\pgfpathlineto{\pgfqpoint{3.569969in}{1.582364in}}%
\pgfpathlineto{\pgfqpoint{3.574510in}{1.582364in}}%
\pgfpathlineto{\pgfqpoint{3.574510in}{1.579415in}}%
\pgfpathmoveto{\pgfqpoint{3.574510in}{1.573516in}}%
\pgfpathlineto{\pgfqpoint{3.574510in}{1.573516in}}%
\pgfpathlineto{\pgfqpoint{3.574510in}{1.576466in}}%
\pgfpathlineto{\pgfqpoint{3.579051in}{1.576466in}}%
\pgfpathlineto{\pgfqpoint{3.579051in}{1.573516in}}%
\pgfpathmoveto{\pgfqpoint{3.574510in}{1.576466in}}%
\pgfpathlineto{\pgfqpoint{3.574510in}{1.576466in}}%
\pgfpathlineto{\pgfqpoint{3.574510in}{1.579415in}}%
\pgfpathlineto{\pgfqpoint{3.579051in}{1.579415in}}%
\pgfpathlineto{\pgfqpoint{3.579051in}{1.576466in}}%
\pgfpathmoveto{\pgfqpoint{3.579051in}{1.573516in}}%
\pgfpathlineto{\pgfqpoint{3.579051in}{1.573516in}}%
\pgfpathlineto{\pgfqpoint{3.579051in}{1.576466in}}%
\pgfpathlineto{\pgfqpoint{3.583592in}{1.576466in}}%
\pgfpathlineto{\pgfqpoint{3.583592in}{1.573516in}}%
\pgfpathmoveto{\pgfqpoint{3.542722in}{1.600060in}}%
\pgfpathlineto{\pgfqpoint{3.542722in}{1.600060in}}%
\pgfpathlineto{\pgfqpoint{3.542722in}{1.603009in}}%
\pgfpathlineto{\pgfqpoint{3.547263in}{1.603009in}}%
\pgfpathlineto{\pgfqpoint{3.547263in}{1.600060in}}%
\pgfpathmoveto{\pgfqpoint{3.538181in}{1.603009in}}%
\pgfpathlineto{\pgfqpoint{3.538181in}{1.603009in}}%
\pgfpathlineto{\pgfqpoint{3.538181in}{1.605959in}}%
\pgfpathlineto{\pgfqpoint{3.542722in}{1.605959in}}%
\pgfpathlineto{\pgfqpoint{3.542722in}{1.603009in}}%
\pgfpathmoveto{\pgfqpoint{3.538181in}{1.605959in}}%
\pgfpathlineto{\pgfqpoint{3.538181in}{1.605959in}}%
\pgfpathlineto{\pgfqpoint{3.538181in}{1.608908in}}%
\pgfpathlineto{\pgfqpoint{3.542722in}{1.608908in}}%
\pgfpathlineto{\pgfqpoint{3.542722in}{1.605959in}}%
\pgfpathmoveto{\pgfqpoint{3.542722in}{1.603009in}}%
\pgfpathlineto{\pgfqpoint{3.542722in}{1.603009in}}%
\pgfpathlineto{\pgfqpoint{3.542722in}{1.605959in}}%
\pgfpathlineto{\pgfqpoint{3.547263in}{1.605959in}}%
\pgfpathlineto{\pgfqpoint{3.547263in}{1.603009in}}%
\pgfpathmoveto{\pgfqpoint{3.524558in}{1.614807in}}%
\pgfpathlineto{\pgfqpoint{3.524558in}{1.614807in}}%
\pgfpathlineto{\pgfqpoint{3.524558in}{1.617756in}}%
\pgfpathlineto{\pgfqpoint{3.529099in}{1.617756in}}%
\pgfpathlineto{\pgfqpoint{3.529099in}{1.614807in}}%
\pgfpathmoveto{\pgfqpoint{3.524558in}{1.617756in}}%
\pgfpathlineto{\pgfqpoint{3.524558in}{1.617756in}}%
\pgfpathlineto{\pgfqpoint{3.524558in}{1.620705in}}%
\pgfpathlineto{\pgfqpoint{3.529099in}{1.620705in}}%
\pgfpathlineto{\pgfqpoint{3.529099in}{1.617756in}}%
\pgfpathmoveto{\pgfqpoint{3.515476in}{1.623655in}}%
\pgfpathlineto{\pgfqpoint{3.515476in}{1.623655in}}%
\pgfpathlineto{\pgfqpoint{3.515476in}{1.626604in}}%
\pgfpathlineto{\pgfqpoint{3.520017in}{1.626604in}}%
\pgfpathlineto{\pgfqpoint{3.520017in}{1.623655in}}%
\pgfpathmoveto{\pgfqpoint{3.510934in}{1.626604in}}%
\pgfpathlineto{\pgfqpoint{3.510934in}{1.626604in}}%
\pgfpathlineto{\pgfqpoint{3.510934in}{1.629553in}}%
\pgfpathlineto{\pgfqpoint{3.515476in}{1.629553in}}%
\pgfpathlineto{\pgfqpoint{3.515476in}{1.626604in}}%
\pgfpathmoveto{\pgfqpoint{3.510934in}{1.629553in}}%
\pgfpathlineto{\pgfqpoint{3.510934in}{1.629553in}}%
\pgfpathlineto{\pgfqpoint{3.510934in}{1.632502in}}%
\pgfpathlineto{\pgfqpoint{3.515476in}{1.632502in}}%
\pgfpathlineto{\pgfqpoint{3.515476in}{1.629553in}}%
\pgfpathmoveto{\pgfqpoint{3.515476in}{1.626604in}}%
\pgfpathlineto{\pgfqpoint{3.515476in}{1.626604in}}%
\pgfpathlineto{\pgfqpoint{3.515476in}{1.629553in}}%
\pgfpathlineto{\pgfqpoint{3.520017in}{1.629553in}}%
\pgfpathlineto{\pgfqpoint{3.520017in}{1.626604in}}%
\pgfpathmoveto{\pgfqpoint{3.520017in}{1.620705in}}%
\pgfpathlineto{\pgfqpoint{3.520017in}{1.620705in}}%
\pgfpathlineto{\pgfqpoint{3.520017in}{1.623655in}}%
\pgfpathlineto{\pgfqpoint{3.524558in}{1.623655in}}%
\pgfpathlineto{\pgfqpoint{3.524558in}{1.620705in}}%
\pgfpathmoveto{\pgfqpoint{3.520017in}{1.623655in}}%
\pgfpathlineto{\pgfqpoint{3.520017in}{1.623655in}}%
\pgfpathlineto{\pgfqpoint{3.520017in}{1.626604in}}%
\pgfpathlineto{\pgfqpoint{3.524558in}{1.626604in}}%
\pgfpathlineto{\pgfqpoint{3.524558in}{1.623655in}}%
\pgfpathmoveto{\pgfqpoint{3.524558in}{1.620705in}}%
\pgfpathlineto{\pgfqpoint{3.524558in}{1.620705in}}%
\pgfpathlineto{\pgfqpoint{3.524558in}{1.623655in}}%
\pgfpathlineto{\pgfqpoint{3.529099in}{1.623655in}}%
\pgfpathlineto{\pgfqpoint{3.529099in}{1.620705in}}%
\pgfpathmoveto{\pgfqpoint{3.529099in}{1.611857in}}%
\pgfpathlineto{\pgfqpoint{3.529099in}{1.611857in}}%
\pgfpathlineto{\pgfqpoint{3.529099in}{1.614807in}}%
\pgfpathlineto{\pgfqpoint{3.533640in}{1.614807in}}%
\pgfpathlineto{\pgfqpoint{3.533640in}{1.611857in}}%
\pgfpathmoveto{\pgfqpoint{3.533640in}{1.608908in}}%
\pgfpathlineto{\pgfqpoint{3.533640in}{1.608908in}}%
\pgfpathlineto{\pgfqpoint{3.533640in}{1.611857in}}%
\pgfpathlineto{\pgfqpoint{3.538181in}{1.611857in}}%
\pgfpathlineto{\pgfqpoint{3.538181in}{1.608908in}}%
\pgfpathmoveto{\pgfqpoint{3.533640in}{1.611857in}}%
\pgfpathlineto{\pgfqpoint{3.533640in}{1.611857in}}%
\pgfpathlineto{\pgfqpoint{3.533640in}{1.614807in}}%
\pgfpathlineto{\pgfqpoint{3.538181in}{1.614807in}}%
\pgfpathlineto{\pgfqpoint{3.538181in}{1.611857in}}%
\pgfpathmoveto{\pgfqpoint{3.529099in}{1.614807in}}%
\pgfpathlineto{\pgfqpoint{3.529099in}{1.614807in}}%
\pgfpathlineto{\pgfqpoint{3.529099in}{1.617756in}}%
\pgfpathlineto{\pgfqpoint{3.533640in}{1.617756in}}%
\pgfpathlineto{\pgfqpoint{3.533640in}{1.614807in}}%
\pgfpathmoveto{\pgfqpoint{3.538181in}{1.608908in}}%
\pgfpathlineto{\pgfqpoint{3.538181in}{1.608908in}}%
\pgfpathlineto{\pgfqpoint{3.538181in}{1.611857in}}%
\pgfpathlineto{\pgfqpoint{3.542722in}{1.611857in}}%
\pgfpathlineto{\pgfqpoint{3.542722in}{1.608908in}}%
\pgfpathmoveto{\pgfqpoint{3.551805in}{1.591212in}}%
\pgfpathlineto{\pgfqpoint{3.551805in}{1.591212in}}%
\pgfpathlineto{\pgfqpoint{3.551805in}{1.594161in}}%
\pgfpathlineto{\pgfqpoint{3.556346in}{1.594161in}}%
\pgfpathlineto{\pgfqpoint{3.556346in}{1.591212in}}%
\pgfpathmoveto{\pgfqpoint{3.551805in}{1.594161in}}%
\pgfpathlineto{\pgfqpoint{3.551805in}{1.594161in}}%
\pgfpathlineto{\pgfqpoint{3.551805in}{1.597111in}}%
\pgfpathlineto{\pgfqpoint{3.556346in}{1.597111in}}%
\pgfpathlineto{\pgfqpoint{3.556346in}{1.594161in}}%
\pgfpathmoveto{\pgfqpoint{3.556346in}{1.588263in}}%
\pgfpathlineto{\pgfqpoint{3.556346in}{1.588263in}}%
\pgfpathlineto{\pgfqpoint{3.556346in}{1.591212in}}%
\pgfpathlineto{\pgfqpoint{3.560887in}{1.591212in}}%
\pgfpathlineto{\pgfqpoint{3.560887in}{1.588263in}}%
\pgfpathmoveto{\pgfqpoint{3.560887in}{1.585313in}}%
\pgfpathlineto{\pgfqpoint{3.560887in}{1.585313in}}%
\pgfpathlineto{\pgfqpoint{3.560887in}{1.588263in}}%
\pgfpathlineto{\pgfqpoint{3.565428in}{1.588263in}}%
\pgfpathlineto{\pgfqpoint{3.565428in}{1.585313in}}%
\pgfpathmoveto{\pgfqpoint{3.560887in}{1.588263in}}%
\pgfpathlineto{\pgfqpoint{3.560887in}{1.588263in}}%
\pgfpathlineto{\pgfqpoint{3.560887in}{1.591212in}}%
\pgfpathlineto{\pgfqpoint{3.565428in}{1.591212in}}%
\pgfpathlineto{\pgfqpoint{3.565428in}{1.588263in}}%
\pgfpathmoveto{\pgfqpoint{3.556346in}{1.591212in}}%
\pgfpathlineto{\pgfqpoint{3.556346in}{1.591212in}}%
\pgfpathlineto{\pgfqpoint{3.556346in}{1.594161in}}%
\pgfpathlineto{\pgfqpoint{3.560887in}{1.594161in}}%
\pgfpathlineto{\pgfqpoint{3.560887in}{1.591212in}}%
\pgfpathmoveto{\pgfqpoint{3.547263in}{1.597111in}}%
\pgfpathlineto{\pgfqpoint{3.547263in}{1.597111in}}%
\pgfpathlineto{\pgfqpoint{3.547263in}{1.600060in}}%
\pgfpathlineto{\pgfqpoint{3.551805in}{1.600060in}}%
\pgfpathlineto{\pgfqpoint{3.551805in}{1.597111in}}%
\pgfpathmoveto{\pgfqpoint{3.547263in}{1.600060in}}%
\pgfpathlineto{\pgfqpoint{3.547263in}{1.600060in}}%
\pgfpathlineto{\pgfqpoint{3.547263in}{1.603009in}}%
\pgfpathlineto{\pgfqpoint{3.551805in}{1.603009in}}%
\pgfpathlineto{\pgfqpoint{3.551805in}{1.600060in}}%
\pgfpathmoveto{\pgfqpoint{3.551805in}{1.597111in}}%
\pgfpathlineto{\pgfqpoint{3.551805in}{1.597111in}}%
\pgfpathlineto{\pgfqpoint{3.551805in}{1.600060in}}%
\pgfpathlineto{\pgfqpoint{3.556346in}{1.600060in}}%
\pgfpathlineto{\pgfqpoint{3.556346in}{1.597111in}}%
\pgfpathmoveto{\pgfqpoint{3.565428in}{1.585313in}}%
\pgfpathlineto{\pgfqpoint{3.565428in}{1.585313in}}%
\pgfpathlineto{\pgfqpoint{3.565428in}{1.588263in}}%
\pgfpathlineto{\pgfqpoint{3.569969in}{1.588263in}}%
\pgfpathlineto{\pgfqpoint{3.569969in}{1.585313in}}%
\pgfpathmoveto{\pgfqpoint{3.597216in}{1.552871in}}%
\pgfpathlineto{\pgfqpoint{3.597216in}{1.552871in}}%
\pgfpathlineto{\pgfqpoint{3.597216in}{1.555820in}}%
\pgfpathlineto{\pgfqpoint{3.601757in}{1.555820in}}%
\pgfpathlineto{\pgfqpoint{3.601757in}{1.552871in}}%
\pgfpathmoveto{\pgfqpoint{3.592675in}{1.555820in}}%
\pgfpathlineto{\pgfqpoint{3.592675in}{1.555820in}}%
\pgfpathlineto{\pgfqpoint{3.592675in}{1.558770in}}%
\pgfpathlineto{\pgfqpoint{3.597216in}{1.558770in}}%
\pgfpathlineto{\pgfqpoint{3.597216in}{1.555820in}}%
\pgfpathmoveto{\pgfqpoint{3.592675in}{1.558770in}}%
\pgfpathlineto{\pgfqpoint{3.592675in}{1.558770in}}%
\pgfpathlineto{\pgfqpoint{3.592675in}{1.561719in}}%
\pgfpathlineto{\pgfqpoint{3.597216in}{1.561719in}}%
\pgfpathlineto{\pgfqpoint{3.597216in}{1.558770in}}%
\pgfpathmoveto{\pgfqpoint{3.597216in}{1.555820in}}%
\pgfpathlineto{\pgfqpoint{3.597216in}{1.555820in}}%
\pgfpathlineto{\pgfqpoint{3.597216in}{1.558770in}}%
\pgfpathlineto{\pgfqpoint{3.601757in}{1.558770in}}%
\pgfpathlineto{\pgfqpoint{3.601757in}{1.555820in}}%
\pgfpathmoveto{\pgfqpoint{3.606298in}{1.544023in}}%
\pgfpathlineto{\pgfqpoint{3.606298in}{1.544023in}}%
\pgfpathlineto{\pgfqpoint{3.606298in}{1.546972in}}%
\pgfpathlineto{\pgfqpoint{3.610839in}{1.546972in}}%
\pgfpathlineto{\pgfqpoint{3.610839in}{1.544023in}}%
\pgfpathmoveto{\pgfqpoint{3.606298in}{1.546972in}}%
\pgfpathlineto{\pgfqpoint{3.606298in}{1.546972in}}%
\pgfpathlineto{\pgfqpoint{3.606298in}{1.549922in}}%
\pgfpathlineto{\pgfqpoint{3.610839in}{1.549922in}}%
\pgfpathlineto{\pgfqpoint{3.610839in}{1.546972in}}%
\pgfpathmoveto{\pgfqpoint{3.610839in}{1.541074in}}%
\pgfpathlineto{\pgfqpoint{3.610839in}{1.541074in}}%
\pgfpathlineto{\pgfqpoint{3.610839in}{1.544023in}}%
\pgfpathlineto{\pgfqpoint{3.615380in}{1.544023in}}%
\pgfpathlineto{\pgfqpoint{3.615380in}{1.541074in}}%
\pgfpathmoveto{\pgfqpoint{3.615380in}{1.538124in}}%
\pgfpathlineto{\pgfqpoint{3.615380in}{1.538124in}}%
\pgfpathlineto{\pgfqpoint{3.615380in}{1.541074in}}%
\pgfpathlineto{\pgfqpoint{3.619922in}{1.541074in}}%
\pgfpathlineto{\pgfqpoint{3.619922in}{1.538124in}}%
\pgfpathmoveto{\pgfqpoint{3.615380in}{1.541074in}}%
\pgfpathlineto{\pgfqpoint{3.615380in}{1.541074in}}%
\pgfpathlineto{\pgfqpoint{3.615380in}{1.544023in}}%
\pgfpathlineto{\pgfqpoint{3.619922in}{1.544023in}}%
\pgfpathlineto{\pgfqpoint{3.619922in}{1.541074in}}%
\pgfpathmoveto{\pgfqpoint{3.610839in}{1.544023in}}%
\pgfpathlineto{\pgfqpoint{3.610839in}{1.544023in}}%
\pgfpathlineto{\pgfqpoint{3.610839in}{1.546972in}}%
\pgfpathlineto{\pgfqpoint{3.615380in}{1.546972in}}%
\pgfpathlineto{\pgfqpoint{3.615380in}{1.544023in}}%
\pgfpathmoveto{\pgfqpoint{3.601757in}{1.549922in}}%
\pgfpathlineto{\pgfqpoint{3.601757in}{1.549922in}}%
\pgfpathlineto{\pgfqpoint{3.601757in}{1.552871in}}%
\pgfpathlineto{\pgfqpoint{3.606298in}{1.552871in}}%
\pgfpathlineto{\pgfqpoint{3.606298in}{1.549922in}}%
\pgfpathmoveto{\pgfqpoint{3.601757in}{1.552871in}}%
\pgfpathlineto{\pgfqpoint{3.601757in}{1.552871in}}%
\pgfpathlineto{\pgfqpoint{3.601757in}{1.555820in}}%
\pgfpathlineto{\pgfqpoint{3.606298in}{1.555820in}}%
\pgfpathlineto{\pgfqpoint{3.606298in}{1.552871in}}%
\pgfpathmoveto{\pgfqpoint{3.606298in}{1.549922in}}%
\pgfpathlineto{\pgfqpoint{3.606298in}{1.549922in}}%
\pgfpathlineto{\pgfqpoint{3.606298in}{1.552871in}}%
\pgfpathlineto{\pgfqpoint{3.610839in}{1.552871in}}%
\pgfpathlineto{\pgfqpoint{3.610839in}{1.549922in}}%
\pgfpathmoveto{\pgfqpoint{3.583592in}{1.564668in}}%
\pgfpathlineto{\pgfqpoint{3.583592in}{1.564668in}}%
\pgfpathlineto{\pgfqpoint{3.583592in}{1.567618in}}%
\pgfpathlineto{\pgfqpoint{3.588134in}{1.567618in}}%
\pgfpathlineto{\pgfqpoint{3.588134in}{1.564668in}}%
\pgfpathmoveto{\pgfqpoint{3.588134in}{1.561719in}}%
\pgfpathlineto{\pgfqpoint{3.588134in}{1.561719in}}%
\pgfpathlineto{\pgfqpoint{3.588134in}{1.564668in}}%
\pgfpathlineto{\pgfqpoint{3.592675in}{1.564668in}}%
\pgfpathlineto{\pgfqpoint{3.592675in}{1.561719in}}%
\pgfpathmoveto{\pgfqpoint{3.588134in}{1.564668in}}%
\pgfpathlineto{\pgfqpoint{3.588134in}{1.564668in}}%
\pgfpathlineto{\pgfqpoint{3.588134in}{1.567618in}}%
\pgfpathlineto{\pgfqpoint{3.592675in}{1.567618in}}%
\pgfpathlineto{\pgfqpoint{3.592675in}{1.564668in}}%
\pgfpathmoveto{\pgfqpoint{3.583592in}{1.567618in}}%
\pgfpathlineto{\pgfqpoint{3.583592in}{1.567618in}}%
\pgfpathlineto{\pgfqpoint{3.583592in}{1.570567in}}%
\pgfpathlineto{\pgfqpoint{3.588134in}{1.570567in}}%
\pgfpathlineto{\pgfqpoint{3.588134in}{1.567618in}}%
\pgfpathmoveto{\pgfqpoint{3.592675in}{1.561719in}}%
\pgfpathlineto{\pgfqpoint{3.592675in}{1.561719in}}%
\pgfpathlineto{\pgfqpoint{3.592675in}{1.564668in}}%
\pgfpathlineto{\pgfqpoint{3.597216in}{1.564668in}}%
\pgfpathlineto{\pgfqpoint{3.597216in}{1.561719in}}%
\pgfpathmoveto{\pgfqpoint{3.619922in}{1.538124in}}%
\pgfpathlineto{\pgfqpoint{3.619922in}{1.538124in}}%
\pgfpathlineto{\pgfqpoint{3.619922in}{1.541074in}}%
\pgfpathlineto{\pgfqpoint{3.624463in}{1.541074in}}%
\pgfpathlineto{\pgfqpoint{3.624463in}{1.538124in}}%
\pgfpathmoveto{\pgfqpoint{3.510934in}{1.632502in}}%
\pgfpathlineto{\pgfqpoint{3.510934in}{1.632502in}}%
\pgfpathlineto{\pgfqpoint{3.510934in}{1.635452in}}%
\pgfpathlineto{\pgfqpoint{3.515476in}{1.635452in}}%
\pgfpathlineto{\pgfqpoint{3.515476in}{1.632502in}}%
\pgfpathmoveto{\pgfqpoint{3.724367in}{1.440800in}}%
\pgfpathlineto{\pgfqpoint{3.724367in}{1.440800in}}%
\pgfpathlineto{\pgfqpoint{3.724367in}{1.443749in}}%
\pgfpathlineto{\pgfqpoint{3.728908in}{1.443749in}}%
\pgfpathlineto{\pgfqpoint{3.728908in}{1.440800in}}%
\pgfpathmoveto{\pgfqpoint{3.778860in}{1.393613in}}%
\pgfpathlineto{\pgfqpoint{3.778860in}{1.393613in}}%
\pgfpathlineto{\pgfqpoint{3.778860in}{1.396562in}}%
\pgfpathlineto{\pgfqpoint{3.783401in}{1.396562in}}%
\pgfpathlineto{\pgfqpoint{3.783401in}{1.393613in}}%
\pgfpathmoveto{\pgfqpoint{3.792484in}{1.381817in}}%
\pgfpathlineto{\pgfqpoint{3.792484in}{1.381817in}}%
\pgfpathlineto{\pgfqpoint{3.792484in}{1.384766in}}%
\pgfpathlineto{\pgfqpoint{3.797025in}{1.384766in}}%
\pgfpathlineto{\pgfqpoint{3.797025in}{1.381817in}}%
\pgfpathmoveto{\pgfqpoint{3.797025in}{1.378867in}}%
\pgfpathlineto{\pgfqpoint{3.797025in}{1.378867in}}%
\pgfpathlineto{\pgfqpoint{3.797025in}{1.381817in}}%
\pgfpathlineto{\pgfqpoint{3.801566in}{1.381817in}}%
\pgfpathlineto{\pgfqpoint{3.801566in}{1.378867in}}%
\pgfpathmoveto{\pgfqpoint{3.797025in}{1.381817in}}%
\pgfpathlineto{\pgfqpoint{3.797025in}{1.381817in}}%
\pgfpathlineto{\pgfqpoint{3.797025in}{1.384766in}}%
\pgfpathlineto{\pgfqpoint{3.801566in}{1.384766in}}%
\pgfpathlineto{\pgfqpoint{3.801566in}{1.381817in}}%
\pgfpathmoveto{\pgfqpoint{3.787943in}{1.387715in}}%
\pgfpathlineto{\pgfqpoint{3.787943in}{1.387715in}}%
\pgfpathlineto{\pgfqpoint{3.787943in}{1.390664in}}%
\pgfpathlineto{\pgfqpoint{3.792484in}{1.390664in}}%
\pgfpathlineto{\pgfqpoint{3.792484in}{1.387715in}}%
\pgfpathmoveto{\pgfqpoint{3.783401in}{1.390664in}}%
\pgfpathlineto{\pgfqpoint{3.783401in}{1.390664in}}%
\pgfpathlineto{\pgfqpoint{3.783401in}{1.393613in}}%
\pgfpathlineto{\pgfqpoint{3.787943in}{1.393613in}}%
\pgfpathlineto{\pgfqpoint{3.787943in}{1.390664in}}%
\pgfpathmoveto{\pgfqpoint{3.783401in}{1.393613in}}%
\pgfpathlineto{\pgfqpoint{3.783401in}{1.393613in}}%
\pgfpathlineto{\pgfqpoint{3.783401in}{1.396562in}}%
\pgfpathlineto{\pgfqpoint{3.787943in}{1.396562in}}%
\pgfpathlineto{\pgfqpoint{3.787943in}{1.393613in}}%
\pgfpathmoveto{\pgfqpoint{3.787943in}{1.390664in}}%
\pgfpathlineto{\pgfqpoint{3.787943in}{1.390664in}}%
\pgfpathlineto{\pgfqpoint{3.787943in}{1.393613in}}%
\pgfpathlineto{\pgfqpoint{3.792484in}{1.393613in}}%
\pgfpathlineto{\pgfqpoint{3.792484in}{1.390664in}}%
\pgfpathmoveto{\pgfqpoint{3.792484in}{1.384766in}}%
\pgfpathlineto{\pgfqpoint{3.792484in}{1.384766in}}%
\pgfpathlineto{\pgfqpoint{3.792484in}{1.387715in}}%
\pgfpathlineto{\pgfqpoint{3.797025in}{1.387715in}}%
\pgfpathlineto{\pgfqpoint{3.797025in}{1.384766in}}%
\pgfpathmoveto{\pgfqpoint{3.792484in}{1.387715in}}%
\pgfpathlineto{\pgfqpoint{3.792484in}{1.387715in}}%
\pgfpathlineto{\pgfqpoint{3.792484in}{1.390664in}}%
\pgfpathlineto{\pgfqpoint{3.797025in}{1.390664in}}%
\pgfpathlineto{\pgfqpoint{3.797025in}{1.387715in}}%
\pgfpathmoveto{\pgfqpoint{3.751614in}{1.417207in}}%
\pgfpathlineto{\pgfqpoint{3.751614in}{1.417207in}}%
\pgfpathlineto{\pgfqpoint{3.751614in}{1.420156in}}%
\pgfpathlineto{\pgfqpoint{3.756155in}{1.420156in}}%
\pgfpathlineto{\pgfqpoint{3.756155in}{1.417207in}}%
\pgfpathmoveto{\pgfqpoint{3.760696in}{1.411308in}}%
\pgfpathlineto{\pgfqpoint{3.760696in}{1.411308in}}%
\pgfpathlineto{\pgfqpoint{3.760696in}{1.414257in}}%
\pgfpathlineto{\pgfqpoint{3.765237in}{1.414257in}}%
\pgfpathlineto{\pgfqpoint{3.765237in}{1.411308in}}%
\pgfpathmoveto{\pgfqpoint{3.756155in}{1.414257in}}%
\pgfpathlineto{\pgfqpoint{3.756155in}{1.414257in}}%
\pgfpathlineto{\pgfqpoint{3.756155in}{1.417207in}}%
\pgfpathlineto{\pgfqpoint{3.760696in}{1.417207in}}%
\pgfpathlineto{\pgfqpoint{3.760696in}{1.414257in}}%
\pgfpathmoveto{\pgfqpoint{3.756155in}{1.417207in}}%
\pgfpathlineto{\pgfqpoint{3.756155in}{1.417207in}}%
\pgfpathlineto{\pgfqpoint{3.756155in}{1.420156in}}%
\pgfpathlineto{\pgfqpoint{3.760696in}{1.420156in}}%
\pgfpathlineto{\pgfqpoint{3.760696in}{1.417207in}}%
\pgfpathmoveto{\pgfqpoint{3.760696in}{1.414257in}}%
\pgfpathlineto{\pgfqpoint{3.760696in}{1.414257in}}%
\pgfpathlineto{\pgfqpoint{3.760696in}{1.417207in}}%
\pgfpathlineto{\pgfqpoint{3.765237in}{1.417207in}}%
\pgfpathlineto{\pgfqpoint{3.765237in}{1.414257in}}%
\pgfpathmoveto{\pgfqpoint{3.737990in}{1.429003in}}%
\pgfpathlineto{\pgfqpoint{3.737990in}{1.429003in}}%
\pgfpathlineto{\pgfqpoint{3.737990in}{1.431952in}}%
\pgfpathlineto{\pgfqpoint{3.742531in}{1.431952in}}%
\pgfpathlineto{\pgfqpoint{3.742531in}{1.429003in}}%
\pgfpathmoveto{\pgfqpoint{3.742531in}{1.426054in}}%
\pgfpathlineto{\pgfqpoint{3.742531in}{1.426054in}}%
\pgfpathlineto{\pgfqpoint{3.742531in}{1.429003in}}%
\pgfpathlineto{\pgfqpoint{3.747073in}{1.429003in}}%
\pgfpathlineto{\pgfqpoint{3.747073in}{1.426054in}}%
\pgfpathmoveto{\pgfqpoint{3.742531in}{1.429003in}}%
\pgfpathlineto{\pgfqpoint{3.742531in}{1.429003in}}%
\pgfpathlineto{\pgfqpoint{3.742531in}{1.431952in}}%
\pgfpathlineto{\pgfqpoint{3.747073in}{1.431952in}}%
\pgfpathlineto{\pgfqpoint{3.747073in}{1.429003in}}%
\pgfpathmoveto{\pgfqpoint{3.733449in}{1.434901in}}%
\pgfpathlineto{\pgfqpoint{3.733449in}{1.434901in}}%
\pgfpathlineto{\pgfqpoint{3.733449in}{1.437851in}}%
\pgfpathlineto{\pgfqpoint{3.737990in}{1.437851in}}%
\pgfpathlineto{\pgfqpoint{3.737990in}{1.434901in}}%
\pgfpathmoveto{\pgfqpoint{3.728908in}{1.437851in}}%
\pgfpathlineto{\pgfqpoint{3.728908in}{1.437851in}}%
\pgfpathlineto{\pgfqpoint{3.728908in}{1.440800in}}%
\pgfpathlineto{\pgfqpoint{3.733449in}{1.440800in}}%
\pgfpathlineto{\pgfqpoint{3.733449in}{1.437851in}}%
\pgfpathmoveto{\pgfqpoint{3.728908in}{1.440800in}}%
\pgfpathlineto{\pgfqpoint{3.728908in}{1.440800in}}%
\pgfpathlineto{\pgfqpoint{3.728908in}{1.443749in}}%
\pgfpathlineto{\pgfqpoint{3.733449in}{1.443749in}}%
\pgfpathlineto{\pgfqpoint{3.733449in}{1.440800in}}%
\pgfpathmoveto{\pgfqpoint{3.733449in}{1.437851in}}%
\pgfpathlineto{\pgfqpoint{3.733449in}{1.437851in}}%
\pgfpathlineto{\pgfqpoint{3.733449in}{1.440800in}}%
\pgfpathlineto{\pgfqpoint{3.737990in}{1.440800in}}%
\pgfpathlineto{\pgfqpoint{3.737990in}{1.437851in}}%
\pgfpathmoveto{\pgfqpoint{3.737990in}{1.431952in}}%
\pgfpathlineto{\pgfqpoint{3.737990in}{1.431952in}}%
\pgfpathlineto{\pgfqpoint{3.737990in}{1.434901in}}%
\pgfpathlineto{\pgfqpoint{3.742531in}{1.434901in}}%
\pgfpathlineto{\pgfqpoint{3.742531in}{1.431952in}}%
\pgfpathmoveto{\pgfqpoint{3.737990in}{1.434901in}}%
\pgfpathlineto{\pgfqpoint{3.737990in}{1.434901in}}%
\pgfpathlineto{\pgfqpoint{3.737990in}{1.437851in}}%
\pgfpathlineto{\pgfqpoint{3.742531in}{1.437851in}}%
\pgfpathlineto{\pgfqpoint{3.742531in}{1.434901in}}%
\pgfpathmoveto{\pgfqpoint{3.747073in}{1.423105in}}%
\pgfpathlineto{\pgfqpoint{3.747073in}{1.423105in}}%
\pgfpathlineto{\pgfqpoint{3.747073in}{1.426054in}}%
\pgfpathlineto{\pgfqpoint{3.751614in}{1.426054in}}%
\pgfpathlineto{\pgfqpoint{3.751614in}{1.423105in}}%
\pgfpathmoveto{\pgfqpoint{3.751614in}{1.420156in}}%
\pgfpathlineto{\pgfqpoint{3.751614in}{1.420156in}}%
\pgfpathlineto{\pgfqpoint{3.751614in}{1.423105in}}%
\pgfpathlineto{\pgfqpoint{3.756155in}{1.423105in}}%
\pgfpathlineto{\pgfqpoint{3.756155in}{1.420156in}}%
\pgfpathmoveto{\pgfqpoint{3.751614in}{1.423105in}}%
\pgfpathlineto{\pgfqpoint{3.751614in}{1.423105in}}%
\pgfpathlineto{\pgfqpoint{3.751614in}{1.426054in}}%
\pgfpathlineto{\pgfqpoint{3.756155in}{1.426054in}}%
\pgfpathlineto{\pgfqpoint{3.756155in}{1.423105in}}%
\pgfpathmoveto{\pgfqpoint{3.747073in}{1.426054in}}%
\pgfpathlineto{\pgfqpoint{3.747073in}{1.426054in}}%
\pgfpathlineto{\pgfqpoint{3.747073in}{1.429003in}}%
\pgfpathlineto{\pgfqpoint{3.751614in}{1.429003in}}%
\pgfpathlineto{\pgfqpoint{3.751614in}{1.426054in}}%
\pgfpathmoveto{\pgfqpoint{3.765237in}{1.405410in}}%
\pgfpathlineto{\pgfqpoint{3.765237in}{1.405410in}}%
\pgfpathlineto{\pgfqpoint{3.765237in}{1.408359in}}%
\pgfpathlineto{\pgfqpoint{3.769778in}{1.408359in}}%
\pgfpathlineto{\pgfqpoint{3.769778in}{1.405410in}}%
\pgfpathmoveto{\pgfqpoint{3.769778in}{1.402461in}}%
\pgfpathlineto{\pgfqpoint{3.769778in}{1.402461in}}%
\pgfpathlineto{\pgfqpoint{3.769778in}{1.405410in}}%
\pgfpathlineto{\pgfqpoint{3.774319in}{1.405410in}}%
\pgfpathlineto{\pgfqpoint{3.774319in}{1.402461in}}%
\pgfpathmoveto{\pgfqpoint{3.769778in}{1.405410in}}%
\pgfpathlineto{\pgfqpoint{3.769778in}{1.405410in}}%
\pgfpathlineto{\pgfqpoint{3.769778in}{1.408359in}}%
\pgfpathlineto{\pgfqpoint{3.774319in}{1.408359in}}%
\pgfpathlineto{\pgfqpoint{3.774319in}{1.405410in}}%
\pgfpathmoveto{\pgfqpoint{3.774319in}{1.399512in}}%
\pgfpathlineto{\pgfqpoint{3.774319in}{1.399512in}}%
\pgfpathlineto{\pgfqpoint{3.774319in}{1.402461in}}%
\pgfpathlineto{\pgfqpoint{3.778860in}{1.402461in}}%
\pgfpathlineto{\pgfqpoint{3.778860in}{1.399512in}}%
\pgfpathmoveto{\pgfqpoint{3.778860in}{1.396562in}}%
\pgfpathlineto{\pgfqpoint{3.778860in}{1.396562in}}%
\pgfpathlineto{\pgfqpoint{3.778860in}{1.399512in}}%
\pgfpathlineto{\pgfqpoint{3.783401in}{1.399512in}}%
\pgfpathlineto{\pgfqpoint{3.783401in}{1.396562in}}%
\pgfpathmoveto{\pgfqpoint{3.778860in}{1.399512in}}%
\pgfpathlineto{\pgfqpoint{3.778860in}{1.399512in}}%
\pgfpathlineto{\pgfqpoint{3.778860in}{1.402461in}}%
\pgfpathlineto{\pgfqpoint{3.783401in}{1.402461in}}%
\pgfpathlineto{\pgfqpoint{3.783401in}{1.399512in}}%
\pgfpathmoveto{\pgfqpoint{3.774319in}{1.402461in}}%
\pgfpathlineto{\pgfqpoint{3.774319in}{1.402461in}}%
\pgfpathlineto{\pgfqpoint{3.774319in}{1.405410in}}%
\pgfpathlineto{\pgfqpoint{3.778860in}{1.405410in}}%
\pgfpathlineto{\pgfqpoint{3.778860in}{1.402461in}}%
\pgfpathmoveto{\pgfqpoint{3.765237in}{1.408359in}}%
\pgfpathlineto{\pgfqpoint{3.765237in}{1.408359in}}%
\pgfpathlineto{\pgfqpoint{3.765237in}{1.411308in}}%
\pgfpathlineto{\pgfqpoint{3.769778in}{1.411308in}}%
\pgfpathlineto{\pgfqpoint{3.769778in}{1.408359in}}%
\pgfpathmoveto{\pgfqpoint{3.765237in}{1.411308in}}%
\pgfpathlineto{\pgfqpoint{3.765237in}{1.411308in}}%
\pgfpathlineto{\pgfqpoint{3.765237in}{1.414257in}}%
\pgfpathlineto{\pgfqpoint{3.769778in}{1.414257in}}%
\pgfpathlineto{\pgfqpoint{3.769778in}{1.411308in}}%
\pgfpathmoveto{\pgfqpoint{3.683497in}{1.476191in}}%
\pgfpathlineto{\pgfqpoint{3.683497in}{1.476191in}}%
\pgfpathlineto{\pgfqpoint{3.683497in}{1.479140in}}%
\pgfpathlineto{\pgfqpoint{3.688038in}{1.479140in}}%
\pgfpathlineto{\pgfqpoint{3.688038in}{1.476191in}}%
\pgfpathmoveto{\pgfqpoint{3.688038in}{1.473241in}}%
\pgfpathlineto{\pgfqpoint{3.688038in}{1.473241in}}%
\pgfpathlineto{\pgfqpoint{3.688038in}{1.476191in}}%
\pgfpathlineto{\pgfqpoint{3.692579in}{1.476191in}}%
\pgfpathlineto{\pgfqpoint{3.692579in}{1.473241in}}%
\pgfpathmoveto{\pgfqpoint{3.688038in}{1.476191in}}%
\pgfpathlineto{\pgfqpoint{3.688038in}{1.476191in}}%
\pgfpathlineto{\pgfqpoint{3.688038in}{1.479140in}}%
\pgfpathlineto{\pgfqpoint{3.692579in}{1.479140in}}%
\pgfpathlineto{\pgfqpoint{3.692579in}{1.476191in}}%
\pgfpathmoveto{\pgfqpoint{3.678956in}{1.482089in}}%
\pgfpathlineto{\pgfqpoint{3.678956in}{1.482089in}}%
\pgfpathlineto{\pgfqpoint{3.678956in}{1.485038in}}%
\pgfpathlineto{\pgfqpoint{3.683497in}{1.485038in}}%
\pgfpathlineto{\pgfqpoint{3.683497in}{1.482089in}}%
\pgfpathmoveto{\pgfqpoint{3.674415in}{1.485038in}}%
\pgfpathlineto{\pgfqpoint{3.674415in}{1.485038in}}%
\pgfpathlineto{\pgfqpoint{3.674415in}{1.487987in}}%
\pgfpathlineto{\pgfqpoint{3.678956in}{1.487987in}}%
\pgfpathlineto{\pgfqpoint{3.678956in}{1.485038in}}%
\pgfpathmoveto{\pgfqpoint{3.674415in}{1.487987in}}%
\pgfpathlineto{\pgfqpoint{3.674415in}{1.487987in}}%
\pgfpathlineto{\pgfqpoint{3.674415in}{1.490937in}}%
\pgfpathlineto{\pgfqpoint{3.678956in}{1.490937in}}%
\pgfpathlineto{\pgfqpoint{3.678956in}{1.487987in}}%
\pgfpathmoveto{\pgfqpoint{3.678956in}{1.485038in}}%
\pgfpathlineto{\pgfqpoint{3.678956in}{1.485038in}}%
\pgfpathlineto{\pgfqpoint{3.678956in}{1.487987in}}%
\pgfpathlineto{\pgfqpoint{3.683497in}{1.487987in}}%
\pgfpathlineto{\pgfqpoint{3.683497in}{1.485038in}}%
\pgfpathmoveto{\pgfqpoint{3.683497in}{1.479140in}}%
\pgfpathlineto{\pgfqpoint{3.683497in}{1.479140in}}%
\pgfpathlineto{\pgfqpoint{3.683497in}{1.482089in}}%
\pgfpathlineto{\pgfqpoint{3.688038in}{1.482089in}}%
\pgfpathlineto{\pgfqpoint{3.688038in}{1.479140in}}%
\pgfpathmoveto{\pgfqpoint{3.683497in}{1.482089in}}%
\pgfpathlineto{\pgfqpoint{3.683497in}{1.482089in}}%
\pgfpathlineto{\pgfqpoint{3.683497in}{1.485038in}}%
\pgfpathlineto{\pgfqpoint{3.688038in}{1.485038in}}%
\pgfpathlineto{\pgfqpoint{3.688038in}{1.482089in}}%
\pgfpathmoveto{\pgfqpoint{3.697120in}{1.464394in}}%
\pgfpathlineto{\pgfqpoint{3.697120in}{1.464394in}}%
\pgfpathlineto{\pgfqpoint{3.697120in}{1.467343in}}%
\pgfpathlineto{\pgfqpoint{3.701662in}{1.467343in}}%
\pgfpathlineto{\pgfqpoint{3.701662in}{1.464394in}}%
\pgfpathmoveto{\pgfqpoint{3.706203in}{1.458495in}}%
\pgfpathlineto{\pgfqpoint{3.706203in}{1.458495in}}%
\pgfpathlineto{\pgfqpoint{3.706203in}{1.461444in}}%
\pgfpathlineto{\pgfqpoint{3.710744in}{1.461444in}}%
\pgfpathlineto{\pgfqpoint{3.710744in}{1.458495in}}%
\pgfpathmoveto{\pgfqpoint{3.701662in}{1.461444in}}%
\pgfpathlineto{\pgfqpoint{3.701662in}{1.461444in}}%
\pgfpathlineto{\pgfqpoint{3.701662in}{1.464394in}}%
\pgfpathlineto{\pgfqpoint{3.706203in}{1.464394in}}%
\pgfpathlineto{\pgfqpoint{3.706203in}{1.461444in}}%
\pgfpathmoveto{\pgfqpoint{3.701662in}{1.464394in}}%
\pgfpathlineto{\pgfqpoint{3.701662in}{1.464394in}}%
\pgfpathlineto{\pgfqpoint{3.701662in}{1.467343in}}%
\pgfpathlineto{\pgfqpoint{3.706203in}{1.467343in}}%
\pgfpathlineto{\pgfqpoint{3.706203in}{1.464394in}}%
\pgfpathmoveto{\pgfqpoint{3.706203in}{1.461444in}}%
\pgfpathlineto{\pgfqpoint{3.706203in}{1.461444in}}%
\pgfpathlineto{\pgfqpoint{3.706203in}{1.464394in}}%
\pgfpathlineto{\pgfqpoint{3.710744in}{1.464394in}}%
\pgfpathlineto{\pgfqpoint{3.710744in}{1.461444in}}%
\pgfpathmoveto{\pgfqpoint{3.710744in}{1.452597in}}%
\pgfpathlineto{\pgfqpoint{3.710744in}{1.452597in}}%
\pgfpathlineto{\pgfqpoint{3.710744in}{1.455546in}}%
\pgfpathlineto{\pgfqpoint{3.715285in}{1.455546in}}%
\pgfpathlineto{\pgfqpoint{3.715285in}{1.452597in}}%
\pgfpathmoveto{\pgfqpoint{3.715285in}{1.449647in}}%
\pgfpathlineto{\pgfqpoint{3.715285in}{1.449647in}}%
\pgfpathlineto{\pgfqpoint{3.715285in}{1.452597in}}%
\pgfpathlineto{\pgfqpoint{3.719826in}{1.452597in}}%
\pgfpathlineto{\pgfqpoint{3.719826in}{1.449647in}}%
\pgfpathmoveto{\pgfqpoint{3.715285in}{1.452597in}}%
\pgfpathlineto{\pgfqpoint{3.715285in}{1.452597in}}%
\pgfpathlineto{\pgfqpoint{3.715285in}{1.455546in}}%
\pgfpathlineto{\pgfqpoint{3.719826in}{1.455546in}}%
\pgfpathlineto{\pgfqpoint{3.719826in}{1.452597in}}%
\pgfpathmoveto{\pgfqpoint{3.719826in}{1.446698in}}%
\pgfpathlineto{\pgfqpoint{3.719826in}{1.446698in}}%
\pgfpathlineto{\pgfqpoint{3.719826in}{1.449647in}}%
\pgfpathlineto{\pgfqpoint{3.724367in}{1.449647in}}%
\pgfpathlineto{\pgfqpoint{3.724367in}{1.446698in}}%
\pgfpathmoveto{\pgfqpoint{3.724367in}{1.443749in}}%
\pgfpathlineto{\pgfqpoint{3.724367in}{1.443749in}}%
\pgfpathlineto{\pgfqpoint{3.724367in}{1.446698in}}%
\pgfpathlineto{\pgfqpoint{3.728908in}{1.446698in}}%
\pgfpathlineto{\pgfqpoint{3.728908in}{1.443749in}}%
\pgfpathmoveto{\pgfqpoint{3.724367in}{1.446698in}}%
\pgfpathlineto{\pgfqpoint{3.724367in}{1.446698in}}%
\pgfpathlineto{\pgfqpoint{3.724367in}{1.449647in}}%
\pgfpathlineto{\pgfqpoint{3.728908in}{1.449647in}}%
\pgfpathlineto{\pgfqpoint{3.728908in}{1.446698in}}%
\pgfpathmoveto{\pgfqpoint{3.719826in}{1.449647in}}%
\pgfpathlineto{\pgfqpoint{3.719826in}{1.449647in}}%
\pgfpathlineto{\pgfqpoint{3.719826in}{1.452597in}}%
\pgfpathlineto{\pgfqpoint{3.724367in}{1.452597in}}%
\pgfpathlineto{\pgfqpoint{3.724367in}{1.449647in}}%
\pgfpathmoveto{\pgfqpoint{3.710744in}{1.455546in}}%
\pgfpathlineto{\pgfqpoint{3.710744in}{1.455546in}}%
\pgfpathlineto{\pgfqpoint{3.710744in}{1.458495in}}%
\pgfpathlineto{\pgfqpoint{3.715285in}{1.458495in}}%
\pgfpathlineto{\pgfqpoint{3.715285in}{1.455546in}}%
\pgfpathmoveto{\pgfqpoint{3.710744in}{1.458495in}}%
\pgfpathlineto{\pgfqpoint{3.710744in}{1.458495in}}%
\pgfpathlineto{\pgfqpoint{3.710744in}{1.461444in}}%
\pgfpathlineto{\pgfqpoint{3.715285in}{1.461444in}}%
\pgfpathlineto{\pgfqpoint{3.715285in}{1.458495in}}%
\pgfpathmoveto{\pgfqpoint{3.692579in}{1.470292in}}%
\pgfpathlineto{\pgfqpoint{3.692579in}{1.470292in}}%
\pgfpathlineto{\pgfqpoint{3.692579in}{1.473241in}}%
\pgfpathlineto{\pgfqpoint{3.697120in}{1.473241in}}%
\pgfpathlineto{\pgfqpoint{3.697120in}{1.470292in}}%
\pgfpathmoveto{\pgfqpoint{3.697120in}{1.467343in}}%
\pgfpathlineto{\pgfqpoint{3.697120in}{1.467343in}}%
\pgfpathlineto{\pgfqpoint{3.697120in}{1.470292in}}%
\pgfpathlineto{\pgfqpoint{3.701662in}{1.470292in}}%
\pgfpathlineto{\pgfqpoint{3.701662in}{1.467343in}}%
\pgfpathmoveto{\pgfqpoint{3.697120in}{1.470292in}}%
\pgfpathlineto{\pgfqpoint{3.697120in}{1.470292in}}%
\pgfpathlineto{\pgfqpoint{3.697120in}{1.473241in}}%
\pgfpathlineto{\pgfqpoint{3.701662in}{1.473241in}}%
\pgfpathlineto{\pgfqpoint{3.701662in}{1.470292in}}%
\pgfpathmoveto{\pgfqpoint{3.692579in}{1.473241in}}%
\pgfpathlineto{\pgfqpoint{3.692579in}{1.473241in}}%
\pgfpathlineto{\pgfqpoint{3.692579in}{1.476191in}}%
\pgfpathlineto{\pgfqpoint{3.697120in}{1.476191in}}%
\pgfpathlineto{\pgfqpoint{3.697120in}{1.473241in}}%
\pgfpathmoveto{\pgfqpoint{3.660792in}{1.496835in}}%
\pgfpathlineto{\pgfqpoint{3.660792in}{1.496835in}}%
\pgfpathlineto{\pgfqpoint{3.660792in}{1.499784in}}%
\pgfpathlineto{\pgfqpoint{3.665333in}{1.499784in}}%
\pgfpathlineto{\pgfqpoint{3.665333in}{1.496835in}}%
\pgfpathmoveto{\pgfqpoint{3.660792in}{1.499784in}}%
\pgfpathlineto{\pgfqpoint{3.660792in}{1.499784in}}%
\pgfpathlineto{\pgfqpoint{3.660792in}{1.502734in}}%
\pgfpathlineto{\pgfqpoint{3.665333in}{1.502734in}}%
\pgfpathlineto{\pgfqpoint{3.665333in}{1.499784in}}%
\pgfpathmoveto{\pgfqpoint{3.665333in}{1.493886in}}%
\pgfpathlineto{\pgfqpoint{3.665333in}{1.493886in}}%
\pgfpathlineto{\pgfqpoint{3.665333in}{1.496835in}}%
\pgfpathlineto{\pgfqpoint{3.669874in}{1.496835in}}%
\pgfpathlineto{\pgfqpoint{3.669874in}{1.493886in}}%
\pgfpathmoveto{\pgfqpoint{3.669874in}{1.490937in}}%
\pgfpathlineto{\pgfqpoint{3.669874in}{1.490937in}}%
\pgfpathlineto{\pgfqpoint{3.669874in}{1.493886in}}%
\pgfpathlineto{\pgfqpoint{3.674415in}{1.493886in}}%
\pgfpathlineto{\pgfqpoint{3.674415in}{1.490937in}}%
\pgfpathmoveto{\pgfqpoint{3.669874in}{1.493886in}}%
\pgfpathlineto{\pgfqpoint{3.669874in}{1.493886in}}%
\pgfpathlineto{\pgfqpoint{3.669874in}{1.496835in}}%
\pgfpathlineto{\pgfqpoint{3.674415in}{1.496835in}}%
\pgfpathlineto{\pgfqpoint{3.674415in}{1.493886in}}%
\pgfpathmoveto{\pgfqpoint{3.665333in}{1.496835in}}%
\pgfpathlineto{\pgfqpoint{3.665333in}{1.496835in}}%
\pgfpathlineto{\pgfqpoint{3.665333in}{1.499784in}}%
\pgfpathlineto{\pgfqpoint{3.669874in}{1.499784in}}%
\pgfpathlineto{\pgfqpoint{3.669874in}{1.496835in}}%
\pgfpathmoveto{\pgfqpoint{3.656251in}{1.502734in}}%
\pgfpathlineto{\pgfqpoint{3.656251in}{1.502734in}}%
\pgfpathlineto{\pgfqpoint{3.656251in}{1.505683in}}%
\pgfpathlineto{\pgfqpoint{3.660792in}{1.505683in}}%
\pgfpathlineto{\pgfqpoint{3.660792in}{1.502734in}}%
\pgfpathmoveto{\pgfqpoint{3.656251in}{1.505683in}}%
\pgfpathlineto{\pgfqpoint{3.656251in}{1.505683in}}%
\pgfpathlineto{\pgfqpoint{3.656251in}{1.508632in}}%
\pgfpathlineto{\pgfqpoint{3.660792in}{1.508632in}}%
\pgfpathlineto{\pgfqpoint{3.660792in}{1.505683in}}%
\pgfpathmoveto{\pgfqpoint{3.660792in}{1.502734in}}%
\pgfpathlineto{\pgfqpoint{3.660792in}{1.502734in}}%
\pgfpathlineto{\pgfqpoint{3.660792in}{1.505683in}}%
\pgfpathlineto{\pgfqpoint{3.665333in}{1.505683in}}%
\pgfpathlineto{\pgfqpoint{3.665333in}{1.502734in}}%
\pgfpathmoveto{\pgfqpoint{3.674415in}{1.490937in}}%
\pgfpathlineto{\pgfqpoint{3.674415in}{1.490937in}}%
\pgfpathlineto{\pgfqpoint{3.674415in}{1.493886in}}%
\pgfpathlineto{\pgfqpoint{3.678956in}{1.493886in}}%
\pgfpathlineto{\pgfqpoint{3.678956in}{1.490937in}}%
\pgfpathmoveto{\pgfqpoint{3.833351in}{1.346427in}}%
\pgfpathlineto{\pgfqpoint{3.833351in}{1.346427in}}%
\pgfpathlineto{\pgfqpoint{3.833351in}{1.349376in}}%
\pgfpathlineto{\pgfqpoint{3.837892in}{1.349376in}}%
\pgfpathlineto{\pgfqpoint{3.837892in}{1.346427in}}%
\pgfpathmoveto{\pgfqpoint{3.860596in}{1.322833in}}%
\pgfpathlineto{\pgfqpoint{3.860596in}{1.322833in}}%
\pgfpathlineto{\pgfqpoint{3.860596in}{1.325782in}}%
\pgfpathlineto{\pgfqpoint{3.865137in}{1.325782in}}%
\pgfpathlineto{\pgfqpoint{3.865137in}{1.322833in}}%
\pgfpathmoveto{\pgfqpoint{3.869678in}{1.316935in}}%
\pgfpathlineto{\pgfqpoint{3.869678in}{1.316935in}}%
\pgfpathlineto{\pgfqpoint{3.869678in}{1.319884in}}%
\pgfpathlineto{\pgfqpoint{3.874219in}{1.319884in}}%
\pgfpathlineto{\pgfqpoint{3.874219in}{1.316935in}}%
\pgfpathmoveto{\pgfqpoint{3.865137in}{1.319884in}}%
\pgfpathlineto{\pgfqpoint{3.865137in}{1.319884in}}%
\pgfpathlineto{\pgfqpoint{3.865137in}{1.322833in}}%
\pgfpathlineto{\pgfqpoint{3.869678in}{1.322833in}}%
\pgfpathlineto{\pgfqpoint{3.869678in}{1.319884in}}%
\pgfpathmoveto{\pgfqpoint{3.865137in}{1.322833in}}%
\pgfpathlineto{\pgfqpoint{3.865137in}{1.322833in}}%
\pgfpathlineto{\pgfqpoint{3.865137in}{1.325782in}}%
\pgfpathlineto{\pgfqpoint{3.869678in}{1.325782in}}%
\pgfpathlineto{\pgfqpoint{3.869678in}{1.322833in}}%
\pgfpathmoveto{\pgfqpoint{3.869678in}{1.319884in}}%
\pgfpathlineto{\pgfqpoint{3.869678in}{1.319884in}}%
\pgfpathlineto{\pgfqpoint{3.869678in}{1.322833in}}%
\pgfpathlineto{\pgfqpoint{3.874219in}{1.322833in}}%
\pgfpathlineto{\pgfqpoint{3.874219in}{1.319884in}}%
\pgfpathmoveto{\pgfqpoint{3.846974in}{1.334630in}}%
\pgfpathlineto{\pgfqpoint{3.846974in}{1.334630in}}%
\pgfpathlineto{\pgfqpoint{3.846974in}{1.337579in}}%
\pgfpathlineto{\pgfqpoint{3.851515in}{1.337579in}}%
\pgfpathlineto{\pgfqpoint{3.851515in}{1.334630in}}%
\pgfpathmoveto{\pgfqpoint{3.851515in}{1.331681in}}%
\pgfpathlineto{\pgfqpoint{3.851515in}{1.331681in}}%
\pgfpathlineto{\pgfqpoint{3.851515in}{1.334630in}}%
\pgfpathlineto{\pgfqpoint{3.856056in}{1.334630in}}%
\pgfpathlineto{\pgfqpoint{3.856056in}{1.331681in}}%
\pgfpathmoveto{\pgfqpoint{3.851515in}{1.334630in}}%
\pgfpathlineto{\pgfqpoint{3.851515in}{1.334630in}}%
\pgfpathlineto{\pgfqpoint{3.851515in}{1.337579in}}%
\pgfpathlineto{\pgfqpoint{3.856056in}{1.337579in}}%
\pgfpathlineto{\pgfqpoint{3.856056in}{1.334630in}}%
\pgfpathmoveto{\pgfqpoint{3.842433in}{1.340528in}}%
\pgfpathlineto{\pgfqpoint{3.842433in}{1.340528in}}%
\pgfpathlineto{\pgfqpoint{3.842433in}{1.343478in}}%
\pgfpathlineto{\pgfqpoint{3.846974in}{1.343478in}}%
\pgfpathlineto{\pgfqpoint{3.846974in}{1.340528in}}%
\pgfpathmoveto{\pgfqpoint{3.837892in}{1.343478in}}%
\pgfpathlineto{\pgfqpoint{3.837892in}{1.343478in}}%
\pgfpathlineto{\pgfqpoint{3.837892in}{1.346427in}}%
\pgfpathlineto{\pgfqpoint{3.842433in}{1.346427in}}%
\pgfpathlineto{\pgfqpoint{3.842433in}{1.343478in}}%
\pgfpathmoveto{\pgfqpoint{3.837892in}{1.346427in}}%
\pgfpathlineto{\pgfqpoint{3.837892in}{1.346427in}}%
\pgfpathlineto{\pgfqpoint{3.837892in}{1.349376in}}%
\pgfpathlineto{\pgfqpoint{3.842433in}{1.349376in}}%
\pgfpathlineto{\pgfqpoint{3.842433in}{1.346427in}}%
\pgfpathmoveto{\pgfqpoint{3.842433in}{1.343478in}}%
\pgfpathlineto{\pgfqpoint{3.842433in}{1.343478in}}%
\pgfpathlineto{\pgfqpoint{3.842433in}{1.346427in}}%
\pgfpathlineto{\pgfqpoint{3.846974in}{1.346427in}}%
\pgfpathlineto{\pgfqpoint{3.846974in}{1.343478in}}%
\pgfpathmoveto{\pgfqpoint{3.846974in}{1.337579in}}%
\pgfpathlineto{\pgfqpoint{3.846974in}{1.337579in}}%
\pgfpathlineto{\pgfqpoint{3.846974in}{1.340528in}}%
\pgfpathlineto{\pgfqpoint{3.851515in}{1.340528in}}%
\pgfpathlineto{\pgfqpoint{3.851515in}{1.337579in}}%
\pgfpathmoveto{\pgfqpoint{3.846974in}{1.340528in}}%
\pgfpathlineto{\pgfqpoint{3.846974in}{1.340528in}}%
\pgfpathlineto{\pgfqpoint{3.846974in}{1.343478in}}%
\pgfpathlineto{\pgfqpoint{3.851515in}{1.343478in}}%
\pgfpathlineto{\pgfqpoint{3.851515in}{1.340528in}}%
\pgfpathmoveto{\pgfqpoint{3.856056in}{1.328732in}}%
\pgfpathlineto{\pgfqpoint{3.856056in}{1.328732in}}%
\pgfpathlineto{\pgfqpoint{3.856056in}{1.331681in}}%
\pgfpathlineto{\pgfqpoint{3.860596in}{1.331681in}}%
\pgfpathlineto{\pgfqpoint{3.860596in}{1.328732in}}%
\pgfpathmoveto{\pgfqpoint{3.860596in}{1.325782in}}%
\pgfpathlineto{\pgfqpoint{3.860596in}{1.325782in}}%
\pgfpathlineto{\pgfqpoint{3.860596in}{1.328732in}}%
\pgfpathlineto{\pgfqpoint{3.865137in}{1.328732in}}%
\pgfpathlineto{\pgfqpoint{3.865137in}{1.325782in}}%
\pgfpathmoveto{\pgfqpoint{3.860596in}{1.328732in}}%
\pgfpathlineto{\pgfqpoint{3.860596in}{1.328732in}}%
\pgfpathlineto{\pgfqpoint{3.860596in}{1.331681in}}%
\pgfpathlineto{\pgfqpoint{3.865137in}{1.331681in}}%
\pgfpathlineto{\pgfqpoint{3.865137in}{1.328732in}}%
\pgfpathmoveto{\pgfqpoint{3.856056in}{1.331681in}}%
\pgfpathlineto{\pgfqpoint{3.856056in}{1.331681in}}%
\pgfpathlineto{\pgfqpoint{3.856056in}{1.334630in}}%
\pgfpathlineto{\pgfqpoint{3.860596in}{1.334630in}}%
\pgfpathlineto{\pgfqpoint{3.860596in}{1.331681in}}%
\pgfpathmoveto{\pgfqpoint{3.887841in}{1.299240in}}%
\pgfpathlineto{\pgfqpoint{3.887841in}{1.299240in}}%
\pgfpathlineto{\pgfqpoint{3.887841in}{1.302189in}}%
\pgfpathlineto{\pgfqpoint{3.892382in}{1.302189in}}%
\pgfpathlineto{\pgfqpoint{3.892382in}{1.299240in}}%
\pgfpathmoveto{\pgfqpoint{3.901464in}{1.287443in}}%
\pgfpathlineto{\pgfqpoint{3.901464in}{1.287443in}}%
\pgfpathlineto{\pgfqpoint{3.901464in}{1.290392in}}%
\pgfpathlineto{\pgfqpoint{3.906004in}{1.290392in}}%
\pgfpathlineto{\pgfqpoint{3.906004in}{1.287443in}}%
\pgfpathmoveto{\pgfqpoint{3.906004in}{1.284494in}}%
\pgfpathlineto{\pgfqpoint{3.906004in}{1.284494in}}%
\pgfpathlineto{\pgfqpoint{3.906004in}{1.287443in}}%
\pgfpathlineto{\pgfqpoint{3.910545in}{1.287443in}}%
\pgfpathlineto{\pgfqpoint{3.910545in}{1.284494in}}%
\pgfpathmoveto{\pgfqpoint{3.906004in}{1.287443in}}%
\pgfpathlineto{\pgfqpoint{3.906004in}{1.287443in}}%
\pgfpathlineto{\pgfqpoint{3.906004in}{1.290392in}}%
\pgfpathlineto{\pgfqpoint{3.910545in}{1.290392in}}%
\pgfpathlineto{\pgfqpoint{3.910545in}{1.287443in}}%
\pgfpathmoveto{\pgfqpoint{3.896923in}{1.293341in}}%
\pgfpathlineto{\pgfqpoint{3.896923in}{1.293341in}}%
\pgfpathlineto{\pgfqpoint{3.896923in}{1.296291in}}%
\pgfpathlineto{\pgfqpoint{3.901464in}{1.296291in}}%
\pgfpathlineto{\pgfqpoint{3.901464in}{1.293341in}}%
\pgfpathmoveto{\pgfqpoint{3.892382in}{1.296291in}}%
\pgfpathlineto{\pgfqpoint{3.892382in}{1.296291in}}%
\pgfpathlineto{\pgfqpoint{3.892382in}{1.299240in}}%
\pgfpathlineto{\pgfqpoint{3.896923in}{1.299240in}}%
\pgfpathlineto{\pgfqpoint{3.896923in}{1.296291in}}%
\pgfpathmoveto{\pgfqpoint{3.892382in}{1.299240in}}%
\pgfpathlineto{\pgfqpoint{3.892382in}{1.299240in}}%
\pgfpathlineto{\pgfqpoint{3.892382in}{1.302189in}}%
\pgfpathlineto{\pgfqpoint{3.896923in}{1.302189in}}%
\pgfpathlineto{\pgfqpoint{3.896923in}{1.299240in}}%
\pgfpathmoveto{\pgfqpoint{3.896923in}{1.296291in}}%
\pgfpathlineto{\pgfqpoint{3.896923in}{1.296291in}}%
\pgfpathlineto{\pgfqpoint{3.896923in}{1.299240in}}%
\pgfpathlineto{\pgfqpoint{3.901464in}{1.299240in}}%
\pgfpathlineto{\pgfqpoint{3.901464in}{1.296291in}}%
\pgfpathmoveto{\pgfqpoint{3.901464in}{1.290392in}}%
\pgfpathlineto{\pgfqpoint{3.901464in}{1.290392in}}%
\pgfpathlineto{\pgfqpoint{3.901464in}{1.293341in}}%
\pgfpathlineto{\pgfqpoint{3.906004in}{1.293341in}}%
\pgfpathlineto{\pgfqpoint{3.906004in}{1.290392in}}%
\pgfpathmoveto{\pgfqpoint{3.901464in}{1.293341in}}%
\pgfpathlineto{\pgfqpoint{3.901464in}{1.293341in}}%
\pgfpathlineto{\pgfqpoint{3.901464in}{1.296291in}}%
\pgfpathlineto{\pgfqpoint{3.906004in}{1.296291in}}%
\pgfpathlineto{\pgfqpoint{3.906004in}{1.293341in}}%
\pgfpathmoveto{\pgfqpoint{3.915086in}{1.275646in}}%
\pgfpathlineto{\pgfqpoint{3.915086in}{1.275646in}}%
\pgfpathlineto{\pgfqpoint{3.915086in}{1.278596in}}%
\pgfpathlineto{\pgfqpoint{3.919627in}{1.278596in}}%
\pgfpathlineto{\pgfqpoint{3.919627in}{1.275646in}}%
\pgfpathmoveto{\pgfqpoint{3.924168in}{1.269748in}}%
\pgfpathlineto{\pgfqpoint{3.924168in}{1.269748in}}%
\pgfpathlineto{\pgfqpoint{3.924168in}{1.272697in}}%
\pgfpathlineto{\pgfqpoint{3.928709in}{1.272697in}}%
\pgfpathlineto{\pgfqpoint{3.928709in}{1.269748in}}%
\pgfpathmoveto{\pgfqpoint{3.919627in}{1.272697in}}%
\pgfpathlineto{\pgfqpoint{3.919627in}{1.272697in}}%
\pgfpathlineto{\pgfqpoint{3.919627in}{1.275646in}}%
\pgfpathlineto{\pgfqpoint{3.924168in}{1.275646in}}%
\pgfpathlineto{\pgfqpoint{3.924168in}{1.272697in}}%
\pgfpathmoveto{\pgfqpoint{3.919627in}{1.275646in}}%
\pgfpathlineto{\pgfqpoint{3.919627in}{1.275646in}}%
\pgfpathlineto{\pgfqpoint{3.919627in}{1.278596in}}%
\pgfpathlineto{\pgfqpoint{3.924168in}{1.278596in}}%
\pgfpathlineto{\pgfqpoint{3.924168in}{1.275646in}}%
\pgfpathmoveto{\pgfqpoint{3.924168in}{1.272697in}}%
\pgfpathlineto{\pgfqpoint{3.924168in}{1.272697in}}%
\pgfpathlineto{\pgfqpoint{3.924168in}{1.275646in}}%
\pgfpathlineto{\pgfqpoint{3.928709in}{1.275646in}}%
\pgfpathlineto{\pgfqpoint{3.928709in}{1.272697in}}%
\pgfpathmoveto{\pgfqpoint{3.933249in}{1.260900in}}%
\pgfpathlineto{\pgfqpoint{3.933249in}{1.260900in}}%
\pgfpathlineto{\pgfqpoint{3.933249in}{1.263850in}}%
\pgfpathlineto{\pgfqpoint{3.937790in}{1.263850in}}%
\pgfpathlineto{\pgfqpoint{3.937790in}{1.260900in}}%
\pgfpathmoveto{\pgfqpoint{3.933249in}{1.263850in}}%
\pgfpathlineto{\pgfqpoint{3.933249in}{1.263850in}}%
\pgfpathlineto{\pgfqpoint{3.933249in}{1.266799in}}%
\pgfpathlineto{\pgfqpoint{3.937790in}{1.266799in}}%
\pgfpathlineto{\pgfqpoint{3.937790in}{1.263850in}}%
\pgfpathmoveto{\pgfqpoint{3.937790in}{1.257951in}}%
\pgfpathlineto{\pgfqpoint{3.937790in}{1.257951in}}%
\pgfpathlineto{\pgfqpoint{3.937790in}{1.260900in}}%
\pgfpathlineto{\pgfqpoint{3.942331in}{1.260900in}}%
\pgfpathlineto{\pgfqpoint{3.942331in}{1.257951in}}%
\pgfpathmoveto{\pgfqpoint{3.942331in}{1.255002in}}%
\pgfpathlineto{\pgfqpoint{3.942331in}{1.255002in}}%
\pgfpathlineto{\pgfqpoint{3.942331in}{1.257951in}}%
\pgfpathlineto{\pgfqpoint{3.946872in}{1.257951in}}%
\pgfpathlineto{\pgfqpoint{3.946872in}{1.255002in}}%
\pgfpathmoveto{\pgfqpoint{3.942331in}{1.257951in}}%
\pgfpathlineto{\pgfqpoint{3.942331in}{1.257951in}}%
\pgfpathlineto{\pgfqpoint{3.942331in}{1.260900in}}%
\pgfpathlineto{\pgfqpoint{3.946872in}{1.260900in}}%
\pgfpathlineto{\pgfqpoint{3.946872in}{1.257951in}}%
\pgfpathmoveto{\pgfqpoint{3.937790in}{1.260900in}}%
\pgfpathlineto{\pgfqpoint{3.937790in}{1.260900in}}%
\pgfpathlineto{\pgfqpoint{3.937790in}{1.263850in}}%
\pgfpathlineto{\pgfqpoint{3.942331in}{1.263850in}}%
\pgfpathlineto{\pgfqpoint{3.942331in}{1.260900in}}%
\pgfpathmoveto{\pgfqpoint{3.928709in}{1.266799in}}%
\pgfpathlineto{\pgfqpoint{3.928709in}{1.266799in}}%
\pgfpathlineto{\pgfqpoint{3.928709in}{1.269748in}}%
\pgfpathlineto{\pgfqpoint{3.933249in}{1.269748in}}%
\pgfpathlineto{\pgfqpoint{3.933249in}{1.266799in}}%
\pgfpathmoveto{\pgfqpoint{3.928709in}{1.269748in}}%
\pgfpathlineto{\pgfqpoint{3.928709in}{1.269748in}}%
\pgfpathlineto{\pgfqpoint{3.928709in}{1.272697in}}%
\pgfpathlineto{\pgfqpoint{3.933249in}{1.272697in}}%
\pgfpathlineto{\pgfqpoint{3.933249in}{1.269748in}}%
\pgfpathmoveto{\pgfqpoint{3.933249in}{1.266799in}}%
\pgfpathlineto{\pgfqpoint{3.933249in}{1.266799in}}%
\pgfpathlineto{\pgfqpoint{3.933249in}{1.269748in}}%
\pgfpathlineto{\pgfqpoint{3.937790in}{1.269748in}}%
\pgfpathlineto{\pgfqpoint{3.937790in}{1.266799in}}%
\pgfpathmoveto{\pgfqpoint{3.910545in}{1.281545in}}%
\pgfpathlineto{\pgfqpoint{3.910545in}{1.281545in}}%
\pgfpathlineto{\pgfqpoint{3.910545in}{1.284494in}}%
\pgfpathlineto{\pgfqpoint{3.915086in}{1.284494in}}%
\pgfpathlineto{\pgfqpoint{3.915086in}{1.281545in}}%
\pgfpathmoveto{\pgfqpoint{3.915086in}{1.278596in}}%
\pgfpathlineto{\pgfqpoint{3.915086in}{1.278596in}}%
\pgfpathlineto{\pgfqpoint{3.915086in}{1.281545in}}%
\pgfpathlineto{\pgfqpoint{3.919627in}{1.281545in}}%
\pgfpathlineto{\pgfqpoint{3.919627in}{1.278596in}}%
\pgfpathmoveto{\pgfqpoint{3.915086in}{1.281545in}}%
\pgfpathlineto{\pgfqpoint{3.915086in}{1.281545in}}%
\pgfpathlineto{\pgfqpoint{3.915086in}{1.284494in}}%
\pgfpathlineto{\pgfqpoint{3.919627in}{1.284494in}}%
\pgfpathlineto{\pgfqpoint{3.919627in}{1.281545in}}%
\pgfpathmoveto{\pgfqpoint{3.910545in}{1.284494in}}%
\pgfpathlineto{\pgfqpoint{3.910545in}{1.284494in}}%
\pgfpathlineto{\pgfqpoint{3.910545in}{1.287443in}}%
\pgfpathlineto{\pgfqpoint{3.915086in}{1.287443in}}%
\pgfpathlineto{\pgfqpoint{3.915086in}{1.284494in}}%
\pgfpathmoveto{\pgfqpoint{3.874219in}{1.311037in}}%
\pgfpathlineto{\pgfqpoint{3.874219in}{1.311037in}}%
\pgfpathlineto{\pgfqpoint{3.874219in}{1.313986in}}%
\pgfpathlineto{\pgfqpoint{3.878760in}{1.313986in}}%
\pgfpathlineto{\pgfqpoint{3.878760in}{1.311037in}}%
\pgfpathmoveto{\pgfqpoint{3.878760in}{1.308087in}}%
\pgfpathlineto{\pgfqpoint{3.878760in}{1.308087in}}%
\pgfpathlineto{\pgfqpoint{3.878760in}{1.311037in}}%
\pgfpathlineto{\pgfqpoint{3.883300in}{1.311037in}}%
\pgfpathlineto{\pgfqpoint{3.883300in}{1.308087in}}%
\pgfpathmoveto{\pgfqpoint{3.878760in}{1.311037in}}%
\pgfpathlineto{\pgfqpoint{3.878760in}{1.311037in}}%
\pgfpathlineto{\pgfqpoint{3.878760in}{1.313986in}}%
\pgfpathlineto{\pgfqpoint{3.883300in}{1.313986in}}%
\pgfpathlineto{\pgfqpoint{3.883300in}{1.311037in}}%
\pgfpathmoveto{\pgfqpoint{3.883300in}{1.305138in}}%
\pgfpathlineto{\pgfqpoint{3.883300in}{1.305138in}}%
\pgfpathlineto{\pgfqpoint{3.883300in}{1.308087in}}%
\pgfpathlineto{\pgfqpoint{3.887841in}{1.308087in}}%
\pgfpathlineto{\pgfqpoint{3.887841in}{1.305138in}}%
\pgfpathmoveto{\pgfqpoint{3.887841in}{1.302189in}}%
\pgfpathlineto{\pgfqpoint{3.887841in}{1.302189in}}%
\pgfpathlineto{\pgfqpoint{3.887841in}{1.305138in}}%
\pgfpathlineto{\pgfqpoint{3.892382in}{1.305138in}}%
\pgfpathlineto{\pgfqpoint{3.892382in}{1.302189in}}%
\pgfpathmoveto{\pgfqpoint{3.887841in}{1.305138in}}%
\pgfpathlineto{\pgfqpoint{3.887841in}{1.305138in}}%
\pgfpathlineto{\pgfqpoint{3.887841in}{1.308087in}}%
\pgfpathlineto{\pgfqpoint{3.892382in}{1.308087in}}%
\pgfpathlineto{\pgfqpoint{3.892382in}{1.305138in}}%
\pgfpathmoveto{\pgfqpoint{3.883300in}{1.308087in}}%
\pgfpathlineto{\pgfqpoint{3.883300in}{1.308087in}}%
\pgfpathlineto{\pgfqpoint{3.883300in}{1.311037in}}%
\pgfpathlineto{\pgfqpoint{3.887841in}{1.311037in}}%
\pgfpathlineto{\pgfqpoint{3.887841in}{1.308087in}}%
\pgfpathmoveto{\pgfqpoint{3.874219in}{1.313986in}}%
\pgfpathlineto{\pgfqpoint{3.874219in}{1.313986in}}%
\pgfpathlineto{\pgfqpoint{3.874219in}{1.316935in}}%
\pgfpathlineto{\pgfqpoint{3.878760in}{1.316935in}}%
\pgfpathlineto{\pgfqpoint{3.878760in}{1.313986in}}%
\pgfpathmoveto{\pgfqpoint{3.874219in}{1.316935in}}%
\pgfpathlineto{\pgfqpoint{3.874219in}{1.316935in}}%
\pgfpathlineto{\pgfqpoint{3.874219in}{1.319884in}}%
\pgfpathlineto{\pgfqpoint{3.878760in}{1.319884in}}%
\pgfpathlineto{\pgfqpoint{3.878760in}{1.316935in}}%
\pgfpathmoveto{\pgfqpoint{3.806107in}{1.370020in}}%
\pgfpathlineto{\pgfqpoint{3.806107in}{1.370020in}}%
\pgfpathlineto{\pgfqpoint{3.806107in}{1.372969in}}%
\pgfpathlineto{\pgfqpoint{3.810647in}{1.372969in}}%
\pgfpathlineto{\pgfqpoint{3.810647in}{1.370020in}}%
\pgfpathmoveto{\pgfqpoint{3.815188in}{1.364122in}}%
\pgfpathlineto{\pgfqpoint{3.815188in}{1.364122in}}%
\pgfpathlineto{\pgfqpoint{3.815188in}{1.367071in}}%
\pgfpathlineto{\pgfqpoint{3.819729in}{1.367071in}}%
\pgfpathlineto{\pgfqpoint{3.819729in}{1.364122in}}%
\pgfpathmoveto{\pgfqpoint{3.810647in}{1.367071in}}%
\pgfpathlineto{\pgfqpoint{3.810647in}{1.367071in}}%
\pgfpathlineto{\pgfqpoint{3.810647in}{1.370020in}}%
\pgfpathlineto{\pgfqpoint{3.815188in}{1.370020in}}%
\pgfpathlineto{\pgfqpoint{3.815188in}{1.367071in}}%
\pgfpathmoveto{\pgfqpoint{3.810647in}{1.370020in}}%
\pgfpathlineto{\pgfqpoint{3.810647in}{1.370020in}}%
\pgfpathlineto{\pgfqpoint{3.810647in}{1.372969in}}%
\pgfpathlineto{\pgfqpoint{3.815188in}{1.372969in}}%
\pgfpathlineto{\pgfqpoint{3.815188in}{1.370020in}}%
\pgfpathmoveto{\pgfqpoint{3.815188in}{1.367071in}}%
\pgfpathlineto{\pgfqpoint{3.815188in}{1.367071in}}%
\pgfpathlineto{\pgfqpoint{3.815188in}{1.370020in}}%
\pgfpathlineto{\pgfqpoint{3.819729in}{1.370020in}}%
\pgfpathlineto{\pgfqpoint{3.819729in}{1.367071in}}%
\pgfpathmoveto{\pgfqpoint{3.819729in}{1.358223in}}%
\pgfpathlineto{\pgfqpoint{3.819729in}{1.358223in}}%
\pgfpathlineto{\pgfqpoint{3.819729in}{1.361173in}}%
\pgfpathlineto{\pgfqpoint{3.824270in}{1.361173in}}%
\pgfpathlineto{\pgfqpoint{3.824270in}{1.358223in}}%
\pgfpathmoveto{\pgfqpoint{3.824270in}{1.355274in}}%
\pgfpathlineto{\pgfqpoint{3.824270in}{1.355274in}}%
\pgfpathlineto{\pgfqpoint{3.824270in}{1.358223in}}%
\pgfpathlineto{\pgfqpoint{3.828811in}{1.358223in}}%
\pgfpathlineto{\pgfqpoint{3.828811in}{1.355274in}}%
\pgfpathmoveto{\pgfqpoint{3.824270in}{1.358223in}}%
\pgfpathlineto{\pgfqpoint{3.824270in}{1.358223in}}%
\pgfpathlineto{\pgfqpoint{3.824270in}{1.361173in}}%
\pgfpathlineto{\pgfqpoint{3.828811in}{1.361173in}}%
\pgfpathlineto{\pgfqpoint{3.828811in}{1.358223in}}%
\pgfpathmoveto{\pgfqpoint{3.828811in}{1.352325in}}%
\pgfpathlineto{\pgfqpoint{3.828811in}{1.352325in}}%
\pgfpathlineto{\pgfqpoint{3.828811in}{1.355274in}}%
\pgfpathlineto{\pgfqpoint{3.833351in}{1.355274in}}%
\pgfpathlineto{\pgfqpoint{3.833351in}{1.352325in}}%
\pgfpathmoveto{\pgfqpoint{3.833351in}{1.349376in}}%
\pgfpathlineto{\pgfqpoint{3.833351in}{1.349376in}}%
\pgfpathlineto{\pgfqpoint{3.833351in}{1.352325in}}%
\pgfpathlineto{\pgfqpoint{3.837892in}{1.352325in}}%
\pgfpathlineto{\pgfqpoint{3.837892in}{1.349376in}}%
\pgfpathmoveto{\pgfqpoint{3.833351in}{1.352325in}}%
\pgfpathlineto{\pgfqpoint{3.833351in}{1.352325in}}%
\pgfpathlineto{\pgfqpoint{3.833351in}{1.355274in}}%
\pgfpathlineto{\pgfqpoint{3.837892in}{1.355274in}}%
\pgfpathlineto{\pgfqpoint{3.837892in}{1.352325in}}%
\pgfpathmoveto{\pgfqpoint{3.828811in}{1.355274in}}%
\pgfpathlineto{\pgfqpoint{3.828811in}{1.355274in}}%
\pgfpathlineto{\pgfqpoint{3.828811in}{1.358223in}}%
\pgfpathlineto{\pgfqpoint{3.833351in}{1.358223in}}%
\pgfpathlineto{\pgfqpoint{3.833351in}{1.355274in}}%
\pgfpathmoveto{\pgfqpoint{3.819729in}{1.361173in}}%
\pgfpathlineto{\pgfqpoint{3.819729in}{1.361173in}}%
\pgfpathlineto{\pgfqpoint{3.819729in}{1.364122in}}%
\pgfpathlineto{\pgfqpoint{3.824270in}{1.364122in}}%
\pgfpathlineto{\pgfqpoint{3.824270in}{1.361173in}}%
\pgfpathmoveto{\pgfqpoint{3.819729in}{1.364122in}}%
\pgfpathlineto{\pgfqpoint{3.819729in}{1.364122in}}%
\pgfpathlineto{\pgfqpoint{3.819729in}{1.367071in}}%
\pgfpathlineto{\pgfqpoint{3.824270in}{1.367071in}}%
\pgfpathlineto{\pgfqpoint{3.824270in}{1.364122in}}%
\pgfpathmoveto{\pgfqpoint{3.801566in}{1.375918in}}%
\pgfpathlineto{\pgfqpoint{3.801566in}{1.375918in}}%
\pgfpathlineto{\pgfqpoint{3.801566in}{1.378867in}}%
\pgfpathlineto{\pgfqpoint{3.806107in}{1.378867in}}%
\pgfpathlineto{\pgfqpoint{3.806107in}{1.375918in}}%
\pgfpathmoveto{\pgfqpoint{3.806107in}{1.372969in}}%
\pgfpathlineto{\pgfqpoint{3.806107in}{1.372969in}}%
\pgfpathlineto{\pgfqpoint{3.806107in}{1.375918in}}%
\pgfpathlineto{\pgfqpoint{3.810647in}{1.375918in}}%
\pgfpathlineto{\pgfqpoint{3.810647in}{1.372969in}}%
\pgfpathmoveto{\pgfqpoint{3.806107in}{1.375918in}}%
\pgfpathlineto{\pgfqpoint{3.806107in}{1.375918in}}%
\pgfpathlineto{\pgfqpoint{3.806107in}{1.378867in}}%
\pgfpathlineto{\pgfqpoint{3.810647in}{1.378867in}}%
\pgfpathlineto{\pgfqpoint{3.810647in}{1.375918in}}%
\pgfpathmoveto{\pgfqpoint{3.801566in}{1.378867in}}%
\pgfpathlineto{\pgfqpoint{3.801566in}{1.378867in}}%
\pgfpathlineto{\pgfqpoint{3.801566in}{1.381817in}}%
\pgfpathlineto{\pgfqpoint{3.806107in}{1.381817in}}%
\pgfpathlineto{\pgfqpoint{3.806107in}{1.378867in}}%
\pgfpathmoveto{\pgfqpoint{4.051320in}{1.157675in}}%
\pgfpathlineto{\pgfqpoint{4.051320in}{1.157675in}}%
\pgfpathlineto{\pgfqpoint{4.051320in}{1.160624in}}%
\pgfpathlineto{\pgfqpoint{4.055861in}{1.160624in}}%
\pgfpathlineto{\pgfqpoint{4.055861in}{1.157675in}}%
\pgfpathmoveto{\pgfqpoint{4.078567in}{1.134081in}}%
\pgfpathlineto{\pgfqpoint{4.078567in}{1.134081in}}%
\pgfpathlineto{\pgfqpoint{4.078567in}{1.137030in}}%
\pgfpathlineto{\pgfqpoint{4.083108in}{1.137030in}}%
\pgfpathlineto{\pgfqpoint{4.083108in}{1.134081in}}%
\pgfpathmoveto{\pgfqpoint{4.087650in}{1.128182in}}%
\pgfpathlineto{\pgfqpoint{4.087650in}{1.128182in}}%
\pgfpathlineto{\pgfqpoint{4.087650in}{1.131131in}}%
\pgfpathlineto{\pgfqpoint{4.092191in}{1.131131in}}%
\pgfpathlineto{\pgfqpoint{4.092191in}{1.128182in}}%
\pgfpathmoveto{\pgfqpoint{4.083108in}{1.131131in}}%
\pgfpathlineto{\pgfqpoint{4.083108in}{1.131131in}}%
\pgfpathlineto{\pgfqpoint{4.083108in}{1.134081in}}%
\pgfpathlineto{\pgfqpoint{4.087650in}{1.134081in}}%
\pgfpathlineto{\pgfqpoint{4.087650in}{1.131131in}}%
\pgfpathmoveto{\pgfqpoint{4.083108in}{1.134081in}}%
\pgfpathlineto{\pgfqpoint{4.083108in}{1.134081in}}%
\pgfpathlineto{\pgfqpoint{4.083108in}{1.137030in}}%
\pgfpathlineto{\pgfqpoint{4.087650in}{1.137030in}}%
\pgfpathlineto{\pgfqpoint{4.087650in}{1.134081in}}%
\pgfpathmoveto{\pgfqpoint{4.087650in}{1.131131in}}%
\pgfpathlineto{\pgfqpoint{4.087650in}{1.131131in}}%
\pgfpathlineto{\pgfqpoint{4.087650in}{1.134081in}}%
\pgfpathlineto{\pgfqpoint{4.092191in}{1.134081in}}%
\pgfpathlineto{\pgfqpoint{4.092191in}{1.131131in}}%
\pgfpathmoveto{\pgfqpoint{4.064944in}{1.145878in}}%
\pgfpathlineto{\pgfqpoint{4.064944in}{1.145878in}}%
\pgfpathlineto{\pgfqpoint{4.064944in}{1.148827in}}%
\pgfpathlineto{\pgfqpoint{4.069485in}{1.148827in}}%
\pgfpathlineto{\pgfqpoint{4.069485in}{1.145878in}}%
\pgfpathmoveto{\pgfqpoint{4.069485in}{1.142929in}}%
\pgfpathlineto{\pgfqpoint{4.069485in}{1.142929in}}%
\pgfpathlineto{\pgfqpoint{4.069485in}{1.145878in}}%
\pgfpathlineto{\pgfqpoint{4.074026in}{1.145878in}}%
\pgfpathlineto{\pgfqpoint{4.074026in}{1.142929in}}%
\pgfpathmoveto{\pgfqpoint{4.069485in}{1.145878in}}%
\pgfpathlineto{\pgfqpoint{4.069485in}{1.145878in}}%
\pgfpathlineto{\pgfqpoint{4.069485in}{1.148827in}}%
\pgfpathlineto{\pgfqpoint{4.074026in}{1.148827in}}%
\pgfpathlineto{\pgfqpoint{4.074026in}{1.145878in}}%
\pgfpathmoveto{\pgfqpoint{4.060402in}{1.151776in}}%
\pgfpathlineto{\pgfqpoint{4.060402in}{1.151776in}}%
\pgfpathlineto{\pgfqpoint{4.060402in}{1.154726in}}%
\pgfpathlineto{\pgfqpoint{4.064944in}{1.154726in}}%
\pgfpathlineto{\pgfqpoint{4.064944in}{1.151776in}}%
\pgfpathmoveto{\pgfqpoint{4.055861in}{1.154726in}}%
\pgfpathlineto{\pgfqpoint{4.055861in}{1.154726in}}%
\pgfpathlineto{\pgfqpoint{4.055861in}{1.157675in}}%
\pgfpathlineto{\pgfqpoint{4.060402in}{1.157675in}}%
\pgfpathlineto{\pgfqpoint{4.060402in}{1.154726in}}%
\pgfpathmoveto{\pgfqpoint{4.055861in}{1.157675in}}%
\pgfpathlineto{\pgfqpoint{4.055861in}{1.157675in}}%
\pgfpathlineto{\pgfqpoint{4.055861in}{1.160624in}}%
\pgfpathlineto{\pgfqpoint{4.060402in}{1.160624in}}%
\pgfpathlineto{\pgfqpoint{4.060402in}{1.157675in}}%
\pgfpathmoveto{\pgfqpoint{4.060402in}{1.154726in}}%
\pgfpathlineto{\pgfqpoint{4.060402in}{1.154726in}}%
\pgfpathlineto{\pgfqpoint{4.060402in}{1.157675in}}%
\pgfpathlineto{\pgfqpoint{4.064944in}{1.157675in}}%
\pgfpathlineto{\pgfqpoint{4.064944in}{1.154726in}}%
\pgfpathmoveto{\pgfqpoint{4.064944in}{1.148827in}}%
\pgfpathlineto{\pgfqpoint{4.064944in}{1.148827in}}%
\pgfpathlineto{\pgfqpoint{4.064944in}{1.151776in}}%
\pgfpathlineto{\pgfqpoint{4.069485in}{1.151776in}}%
\pgfpathlineto{\pgfqpoint{4.069485in}{1.148827in}}%
\pgfpathmoveto{\pgfqpoint{4.064944in}{1.151776in}}%
\pgfpathlineto{\pgfqpoint{4.064944in}{1.151776in}}%
\pgfpathlineto{\pgfqpoint{4.064944in}{1.154726in}}%
\pgfpathlineto{\pgfqpoint{4.069485in}{1.154726in}}%
\pgfpathlineto{\pgfqpoint{4.069485in}{1.151776in}}%
\pgfpathmoveto{\pgfqpoint{4.074026in}{1.139979in}}%
\pgfpathlineto{\pgfqpoint{4.074026in}{1.139979in}}%
\pgfpathlineto{\pgfqpoint{4.074026in}{1.142929in}}%
\pgfpathlineto{\pgfqpoint{4.078567in}{1.142929in}}%
\pgfpathlineto{\pgfqpoint{4.078567in}{1.139979in}}%
\pgfpathmoveto{\pgfqpoint{4.078567in}{1.137030in}}%
\pgfpathlineto{\pgfqpoint{4.078567in}{1.137030in}}%
\pgfpathlineto{\pgfqpoint{4.078567in}{1.139979in}}%
\pgfpathlineto{\pgfqpoint{4.083108in}{1.139979in}}%
\pgfpathlineto{\pgfqpoint{4.083108in}{1.137030in}}%
\pgfpathmoveto{\pgfqpoint{4.078567in}{1.139979in}}%
\pgfpathlineto{\pgfqpoint{4.078567in}{1.139979in}}%
\pgfpathlineto{\pgfqpoint{4.078567in}{1.142929in}}%
\pgfpathlineto{\pgfqpoint{4.083108in}{1.142929in}}%
\pgfpathlineto{\pgfqpoint{4.083108in}{1.139979in}}%
\pgfpathmoveto{\pgfqpoint{4.074026in}{1.142929in}}%
\pgfpathlineto{\pgfqpoint{4.074026in}{1.142929in}}%
\pgfpathlineto{\pgfqpoint{4.074026in}{1.145878in}}%
\pgfpathlineto{\pgfqpoint{4.078567in}{1.145878in}}%
\pgfpathlineto{\pgfqpoint{4.078567in}{1.142929in}}%
\pgfpathmoveto{\pgfqpoint{3.996825in}{1.204864in}}%
\pgfpathlineto{\pgfqpoint{3.996825in}{1.204864in}}%
\pgfpathlineto{\pgfqpoint{3.996825in}{1.207813in}}%
\pgfpathlineto{\pgfqpoint{4.001366in}{1.207813in}}%
\pgfpathlineto{\pgfqpoint{4.001366in}{1.204864in}}%
\pgfpathmoveto{\pgfqpoint{4.010449in}{1.193067in}}%
\pgfpathlineto{\pgfqpoint{4.010449in}{1.193067in}}%
\pgfpathlineto{\pgfqpoint{4.010449in}{1.196016in}}%
\pgfpathlineto{\pgfqpoint{4.014990in}{1.196016in}}%
\pgfpathlineto{\pgfqpoint{4.014990in}{1.193067in}}%
\pgfpathmoveto{\pgfqpoint{4.014990in}{1.190117in}}%
\pgfpathlineto{\pgfqpoint{4.014990in}{1.190117in}}%
\pgfpathlineto{\pgfqpoint{4.014990in}{1.193067in}}%
\pgfpathlineto{\pgfqpoint{4.019531in}{1.193067in}}%
\pgfpathlineto{\pgfqpoint{4.019531in}{1.190117in}}%
\pgfpathmoveto{\pgfqpoint{4.014990in}{1.193067in}}%
\pgfpathlineto{\pgfqpoint{4.014990in}{1.193067in}}%
\pgfpathlineto{\pgfqpoint{4.014990in}{1.196016in}}%
\pgfpathlineto{\pgfqpoint{4.019531in}{1.196016in}}%
\pgfpathlineto{\pgfqpoint{4.019531in}{1.193067in}}%
\pgfpathmoveto{\pgfqpoint{4.005908in}{1.198965in}}%
\pgfpathlineto{\pgfqpoint{4.005908in}{1.198965in}}%
\pgfpathlineto{\pgfqpoint{4.005908in}{1.201915in}}%
\pgfpathlineto{\pgfqpoint{4.010449in}{1.201915in}}%
\pgfpathlineto{\pgfqpoint{4.010449in}{1.198965in}}%
\pgfpathmoveto{\pgfqpoint{4.001366in}{1.201915in}}%
\pgfpathlineto{\pgfqpoint{4.001366in}{1.201915in}}%
\pgfpathlineto{\pgfqpoint{4.001366in}{1.204864in}}%
\pgfpathlineto{\pgfqpoint{4.005908in}{1.204864in}}%
\pgfpathlineto{\pgfqpoint{4.005908in}{1.201915in}}%
\pgfpathmoveto{\pgfqpoint{4.001366in}{1.204864in}}%
\pgfpathlineto{\pgfqpoint{4.001366in}{1.204864in}}%
\pgfpathlineto{\pgfqpoint{4.001366in}{1.207813in}}%
\pgfpathlineto{\pgfqpoint{4.005908in}{1.207813in}}%
\pgfpathlineto{\pgfqpoint{4.005908in}{1.204864in}}%
\pgfpathmoveto{\pgfqpoint{4.005908in}{1.201915in}}%
\pgfpathlineto{\pgfqpoint{4.005908in}{1.201915in}}%
\pgfpathlineto{\pgfqpoint{4.005908in}{1.204864in}}%
\pgfpathlineto{\pgfqpoint{4.010449in}{1.204864in}}%
\pgfpathlineto{\pgfqpoint{4.010449in}{1.201915in}}%
\pgfpathmoveto{\pgfqpoint{4.010449in}{1.196016in}}%
\pgfpathlineto{\pgfqpoint{4.010449in}{1.196016in}}%
\pgfpathlineto{\pgfqpoint{4.010449in}{1.198965in}}%
\pgfpathlineto{\pgfqpoint{4.014990in}{1.198965in}}%
\pgfpathlineto{\pgfqpoint{4.014990in}{1.196016in}}%
\pgfpathmoveto{\pgfqpoint{4.010449in}{1.198965in}}%
\pgfpathlineto{\pgfqpoint{4.010449in}{1.198965in}}%
\pgfpathlineto{\pgfqpoint{4.010449in}{1.201915in}}%
\pgfpathlineto{\pgfqpoint{4.014990in}{1.201915in}}%
\pgfpathlineto{\pgfqpoint{4.014990in}{1.198965in}}%
\pgfpathmoveto{\pgfqpoint{3.978660in}{1.222560in}}%
\pgfpathlineto{\pgfqpoint{3.978660in}{1.222560in}}%
\pgfpathlineto{\pgfqpoint{3.978660in}{1.225509in}}%
\pgfpathlineto{\pgfqpoint{3.983202in}{1.225509in}}%
\pgfpathlineto{\pgfqpoint{3.983202in}{1.222560in}}%
\pgfpathmoveto{\pgfqpoint{3.974119in}{1.225509in}}%
\pgfpathlineto{\pgfqpoint{3.974119in}{1.225509in}}%
\pgfpathlineto{\pgfqpoint{3.974119in}{1.228458in}}%
\pgfpathlineto{\pgfqpoint{3.978660in}{1.228458in}}%
\pgfpathlineto{\pgfqpoint{3.978660in}{1.225509in}}%
\pgfpathmoveto{\pgfqpoint{3.974119in}{1.228458in}}%
\pgfpathlineto{\pgfqpoint{3.974119in}{1.228458in}}%
\pgfpathlineto{\pgfqpoint{3.974119in}{1.231408in}}%
\pgfpathlineto{\pgfqpoint{3.978660in}{1.231408in}}%
\pgfpathlineto{\pgfqpoint{3.978660in}{1.228458in}}%
\pgfpathmoveto{\pgfqpoint{3.978660in}{1.225509in}}%
\pgfpathlineto{\pgfqpoint{3.978660in}{1.225509in}}%
\pgfpathlineto{\pgfqpoint{3.978660in}{1.228458in}}%
\pgfpathlineto{\pgfqpoint{3.983202in}{1.228458in}}%
\pgfpathlineto{\pgfqpoint{3.983202in}{1.225509in}}%
\pgfpathmoveto{\pgfqpoint{3.960495in}{1.237306in}}%
\pgfpathlineto{\pgfqpoint{3.960495in}{1.237306in}}%
\pgfpathlineto{\pgfqpoint{3.960495in}{1.240256in}}%
\pgfpathlineto{\pgfqpoint{3.965037in}{1.240256in}}%
\pgfpathlineto{\pgfqpoint{3.965037in}{1.237306in}}%
\pgfpathmoveto{\pgfqpoint{3.960495in}{1.240256in}}%
\pgfpathlineto{\pgfqpoint{3.960495in}{1.240256in}}%
\pgfpathlineto{\pgfqpoint{3.960495in}{1.243205in}}%
\pgfpathlineto{\pgfqpoint{3.965037in}{1.243205in}}%
\pgfpathlineto{\pgfqpoint{3.965037in}{1.240256in}}%
\pgfpathmoveto{\pgfqpoint{3.951413in}{1.246154in}}%
\pgfpathlineto{\pgfqpoint{3.951413in}{1.246154in}}%
\pgfpathlineto{\pgfqpoint{3.951413in}{1.249103in}}%
\pgfpathlineto{\pgfqpoint{3.955954in}{1.249103in}}%
\pgfpathlineto{\pgfqpoint{3.955954in}{1.246154in}}%
\pgfpathmoveto{\pgfqpoint{3.946872in}{1.249103in}}%
\pgfpathlineto{\pgfqpoint{3.946872in}{1.249103in}}%
\pgfpathlineto{\pgfqpoint{3.946872in}{1.252053in}}%
\pgfpathlineto{\pgfqpoint{3.951413in}{1.252053in}}%
\pgfpathlineto{\pgfqpoint{3.951413in}{1.249103in}}%
\pgfpathmoveto{\pgfqpoint{3.946872in}{1.252053in}}%
\pgfpathlineto{\pgfqpoint{3.946872in}{1.252053in}}%
\pgfpathlineto{\pgfqpoint{3.946872in}{1.255002in}}%
\pgfpathlineto{\pgfqpoint{3.951413in}{1.255002in}}%
\pgfpathlineto{\pgfqpoint{3.951413in}{1.252053in}}%
\pgfpathmoveto{\pgfqpoint{3.951413in}{1.249103in}}%
\pgfpathlineto{\pgfqpoint{3.951413in}{1.249103in}}%
\pgfpathlineto{\pgfqpoint{3.951413in}{1.252053in}}%
\pgfpathlineto{\pgfqpoint{3.955954in}{1.252053in}}%
\pgfpathlineto{\pgfqpoint{3.955954in}{1.249103in}}%
\pgfpathmoveto{\pgfqpoint{3.955954in}{1.243205in}}%
\pgfpathlineto{\pgfqpoint{3.955954in}{1.243205in}}%
\pgfpathlineto{\pgfqpoint{3.955954in}{1.246154in}}%
\pgfpathlineto{\pgfqpoint{3.960495in}{1.246154in}}%
\pgfpathlineto{\pgfqpoint{3.960495in}{1.243205in}}%
\pgfpathmoveto{\pgfqpoint{3.955954in}{1.246154in}}%
\pgfpathlineto{\pgfqpoint{3.955954in}{1.246154in}}%
\pgfpathlineto{\pgfqpoint{3.955954in}{1.249103in}}%
\pgfpathlineto{\pgfqpoint{3.960495in}{1.249103in}}%
\pgfpathlineto{\pgfqpoint{3.960495in}{1.246154in}}%
\pgfpathmoveto{\pgfqpoint{3.960495in}{1.243205in}}%
\pgfpathlineto{\pgfqpoint{3.960495in}{1.243205in}}%
\pgfpathlineto{\pgfqpoint{3.960495in}{1.246154in}}%
\pgfpathlineto{\pgfqpoint{3.965037in}{1.246154in}}%
\pgfpathlineto{\pgfqpoint{3.965037in}{1.243205in}}%
\pgfpathmoveto{\pgfqpoint{3.965037in}{1.234357in}}%
\pgfpathlineto{\pgfqpoint{3.965037in}{1.234357in}}%
\pgfpathlineto{\pgfqpoint{3.965037in}{1.237306in}}%
\pgfpathlineto{\pgfqpoint{3.969578in}{1.237306in}}%
\pgfpathlineto{\pgfqpoint{3.969578in}{1.234357in}}%
\pgfpathmoveto{\pgfqpoint{3.969578in}{1.231408in}}%
\pgfpathlineto{\pgfqpoint{3.969578in}{1.231408in}}%
\pgfpathlineto{\pgfqpoint{3.969578in}{1.234357in}}%
\pgfpathlineto{\pgfqpoint{3.974119in}{1.234357in}}%
\pgfpathlineto{\pgfqpoint{3.974119in}{1.231408in}}%
\pgfpathmoveto{\pgfqpoint{3.969578in}{1.234357in}}%
\pgfpathlineto{\pgfqpoint{3.969578in}{1.234357in}}%
\pgfpathlineto{\pgfqpoint{3.969578in}{1.237306in}}%
\pgfpathlineto{\pgfqpoint{3.974119in}{1.237306in}}%
\pgfpathlineto{\pgfqpoint{3.974119in}{1.234357in}}%
\pgfpathmoveto{\pgfqpoint{3.965037in}{1.237306in}}%
\pgfpathlineto{\pgfqpoint{3.965037in}{1.237306in}}%
\pgfpathlineto{\pgfqpoint{3.965037in}{1.240256in}}%
\pgfpathlineto{\pgfqpoint{3.969578in}{1.240256in}}%
\pgfpathlineto{\pgfqpoint{3.969578in}{1.237306in}}%
\pgfpathmoveto{\pgfqpoint{3.974119in}{1.231408in}}%
\pgfpathlineto{\pgfqpoint{3.974119in}{1.231408in}}%
\pgfpathlineto{\pgfqpoint{3.974119in}{1.234357in}}%
\pgfpathlineto{\pgfqpoint{3.978660in}{1.234357in}}%
\pgfpathlineto{\pgfqpoint{3.978660in}{1.231408in}}%
\pgfpathmoveto{\pgfqpoint{3.987743in}{1.213712in}}%
\pgfpathlineto{\pgfqpoint{3.987743in}{1.213712in}}%
\pgfpathlineto{\pgfqpoint{3.987743in}{1.216661in}}%
\pgfpathlineto{\pgfqpoint{3.992284in}{1.216661in}}%
\pgfpathlineto{\pgfqpoint{3.992284in}{1.213712in}}%
\pgfpathmoveto{\pgfqpoint{3.987743in}{1.216661in}}%
\pgfpathlineto{\pgfqpoint{3.987743in}{1.216661in}}%
\pgfpathlineto{\pgfqpoint{3.987743in}{1.219610in}}%
\pgfpathlineto{\pgfqpoint{3.992284in}{1.219610in}}%
\pgfpathlineto{\pgfqpoint{3.992284in}{1.216661in}}%
\pgfpathmoveto{\pgfqpoint{3.992284in}{1.210762in}}%
\pgfpathlineto{\pgfqpoint{3.992284in}{1.210762in}}%
\pgfpathlineto{\pgfqpoint{3.992284in}{1.213712in}}%
\pgfpathlineto{\pgfqpoint{3.996825in}{1.213712in}}%
\pgfpathlineto{\pgfqpoint{3.996825in}{1.210762in}}%
\pgfpathmoveto{\pgfqpoint{3.996825in}{1.207813in}}%
\pgfpathlineto{\pgfqpoint{3.996825in}{1.207813in}}%
\pgfpathlineto{\pgfqpoint{3.996825in}{1.210762in}}%
\pgfpathlineto{\pgfqpoint{4.001366in}{1.210762in}}%
\pgfpathlineto{\pgfqpoint{4.001366in}{1.207813in}}%
\pgfpathmoveto{\pgfqpoint{3.996825in}{1.210762in}}%
\pgfpathlineto{\pgfqpoint{3.996825in}{1.210762in}}%
\pgfpathlineto{\pgfqpoint{3.996825in}{1.213712in}}%
\pgfpathlineto{\pgfqpoint{4.001366in}{1.213712in}}%
\pgfpathlineto{\pgfqpoint{4.001366in}{1.210762in}}%
\pgfpathmoveto{\pgfqpoint{3.992284in}{1.213712in}}%
\pgfpathlineto{\pgfqpoint{3.992284in}{1.213712in}}%
\pgfpathlineto{\pgfqpoint{3.992284in}{1.216661in}}%
\pgfpathlineto{\pgfqpoint{3.996825in}{1.216661in}}%
\pgfpathlineto{\pgfqpoint{3.996825in}{1.213712in}}%
\pgfpathmoveto{\pgfqpoint{3.983202in}{1.219610in}}%
\pgfpathlineto{\pgfqpoint{3.983202in}{1.219610in}}%
\pgfpathlineto{\pgfqpoint{3.983202in}{1.222560in}}%
\pgfpathlineto{\pgfqpoint{3.987743in}{1.222560in}}%
\pgfpathlineto{\pgfqpoint{3.987743in}{1.219610in}}%
\pgfpathmoveto{\pgfqpoint{3.983202in}{1.222560in}}%
\pgfpathlineto{\pgfqpoint{3.983202in}{1.222560in}}%
\pgfpathlineto{\pgfqpoint{3.983202in}{1.225509in}}%
\pgfpathlineto{\pgfqpoint{3.987743in}{1.225509in}}%
\pgfpathlineto{\pgfqpoint{3.987743in}{1.222560in}}%
\pgfpathmoveto{\pgfqpoint{3.987743in}{1.219610in}}%
\pgfpathlineto{\pgfqpoint{3.987743in}{1.219610in}}%
\pgfpathlineto{\pgfqpoint{3.987743in}{1.222560in}}%
\pgfpathlineto{\pgfqpoint{3.992284in}{1.222560in}}%
\pgfpathlineto{\pgfqpoint{3.992284in}{1.219610in}}%
\pgfpathmoveto{\pgfqpoint{4.024073in}{1.181269in}}%
\pgfpathlineto{\pgfqpoint{4.024073in}{1.181269in}}%
\pgfpathlineto{\pgfqpoint{4.024073in}{1.184219in}}%
\pgfpathlineto{\pgfqpoint{4.028614in}{1.184219in}}%
\pgfpathlineto{\pgfqpoint{4.028614in}{1.181269in}}%
\pgfpathmoveto{\pgfqpoint{4.033155in}{1.175371in}}%
\pgfpathlineto{\pgfqpoint{4.033155in}{1.175371in}}%
\pgfpathlineto{\pgfqpoint{4.033155in}{1.178320in}}%
\pgfpathlineto{\pgfqpoint{4.037696in}{1.178320in}}%
\pgfpathlineto{\pgfqpoint{4.037696in}{1.175371in}}%
\pgfpathmoveto{\pgfqpoint{4.028614in}{1.178320in}}%
\pgfpathlineto{\pgfqpoint{4.028614in}{1.178320in}}%
\pgfpathlineto{\pgfqpoint{4.028614in}{1.181269in}}%
\pgfpathlineto{\pgfqpoint{4.033155in}{1.181269in}}%
\pgfpathlineto{\pgfqpoint{4.033155in}{1.178320in}}%
\pgfpathmoveto{\pgfqpoint{4.028614in}{1.181269in}}%
\pgfpathlineto{\pgfqpoint{4.028614in}{1.181269in}}%
\pgfpathlineto{\pgfqpoint{4.028614in}{1.184219in}}%
\pgfpathlineto{\pgfqpoint{4.033155in}{1.184219in}}%
\pgfpathlineto{\pgfqpoint{4.033155in}{1.181269in}}%
\pgfpathmoveto{\pgfqpoint{4.033155in}{1.178320in}}%
\pgfpathlineto{\pgfqpoint{4.033155in}{1.178320in}}%
\pgfpathlineto{\pgfqpoint{4.033155in}{1.181269in}}%
\pgfpathlineto{\pgfqpoint{4.037696in}{1.181269in}}%
\pgfpathlineto{\pgfqpoint{4.037696in}{1.178320in}}%
\pgfpathmoveto{\pgfqpoint{4.037696in}{1.169472in}}%
\pgfpathlineto{\pgfqpoint{4.037696in}{1.169472in}}%
\pgfpathlineto{\pgfqpoint{4.037696in}{1.172421in}}%
\pgfpathlineto{\pgfqpoint{4.042237in}{1.172421in}}%
\pgfpathlineto{\pgfqpoint{4.042237in}{1.169472in}}%
\pgfpathmoveto{\pgfqpoint{4.042237in}{1.166523in}}%
\pgfpathlineto{\pgfqpoint{4.042237in}{1.166523in}}%
\pgfpathlineto{\pgfqpoint{4.042237in}{1.169472in}}%
\pgfpathlineto{\pgfqpoint{4.046779in}{1.169472in}}%
\pgfpathlineto{\pgfqpoint{4.046779in}{1.166523in}}%
\pgfpathmoveto{\pgfqpoint{4.042237in}{1.169472in}}%
\pgfpathlineto{\pgfqpoint{4.042237in}{1.169472in}}%
\pgfpathlineto{\pgfqpoint{4.042237in}{1.172421in}}%
\pgfpathlineto{\pgfqpoint{4.046779in}{1.172421in}}%
\pgfpathlineto{\pgfqpoint{4.046779in}{1.169472in}}%
\pgfpathmoveto{\pgfqpoint{4.046779in}{1.163574in}}%
\pgfpathlineto{\pgfqpoint{4.046779in}{1.163574in}}%
\pgfpathlineto{\pgfqpoint{4.046779in}{1.166523in}}%
\pgfpathlineto{\pgfqpoint{4.051320in}{1.166523in}}%
\pgfpathlineto{\pgfqpoint{4.051320in}{1.163574in}}%
\pgfpathmoveto{\pgfqpoint{4.051320in}{1.160624in}}%
\pgfpathlineto{\pgfqpoint{4.051320in}{1.160624in}}%
\pgfpathlineto{\pgfqpoint{4.051320in}{1.163574in}}%
\pgfpathlineto{\pgfqpoint{4.055861in}{1.163574in}}%
\pgfpathlineto{\pgfqpoint{4.055861in}{1.160624in}}%
\pgfpathmoveto{\pgfqpoint{4.051320in}{1.163574in}}%
\pgfpathlineto{\pgfqpoint{4.051320in}{1.163574in}}%
\pgfpathlineto{\pgfqpoint{4.051320in}{1.166523in}}%
\pgfpathlineto{\pgfqpoint{4.055861in}{1.166523in}}%
\pgfpathlineto{\pgfqpoint{4.055861in}{1.163574in}}%
\pgfpathmoveto{\pgfqpoint{4.046779in}{1.166523in}}%
\pgfpathlineto{\pgfqpoint{4.046779in}{1.166523in}}%
\pgfpathlineto{\pgfqpoint{4.046779in}{1.169472in}}%
\pgfpathlineto{\pgfqpoint{4.051320in}{1.169472in}}%
\pgfpathlineto{\pgfqpoint{4.051320in}{1.166523in}}%
\pgfpathmoveto{\pgfqpoint{4.037696in}{1.172421in}}%
\pgfpathlineto{\pgfqpoint{4.037696in}{1.172421in}}%
\pgfpathlineto{\pgfqpoint{4.037696in}{1.175371in}}%
\pgfpathlineto{\pgfqpoint{4.042237in}{1.175371in}}%
\pgfpathlineto{\pgfqpoint{4.042237in}{1.172421in}}%
\pgfpathmoveto{\pgfqpoint{4.037696in}{1.175371in}}%
\pgfpathlineto{\pgfqpoint{4.037696in}{1.175371in}}%
\pgfpathlineto{\pgfqpoint{4.037696in}{1.178320in}}%
\pgfpathlineto{\pgfqpoint{4.042237in}{1.178320in}}%
\pgfpathlineto{\pgfqpoint{4.042237in}{1.175371in}}%
\pgfpathmoveto{\pgfqpoint{4.019531in}{1.187168in}}%
\pgfpathlineto{\pgfqpoint{4.019531in}{1.187168in}}%
\pgfpathlineto{\pgfqpoint{4.019531in}{1.190117in}}%
\pgfpathlineto{\pgfqpoint{4.024073in}{1.190117in}}%
\pgfpathlineto{\pgfqpoint{4.024073in}{1.187168in}}%
\pgfpathmoveto{\pgfqpoint{4.024073in}{1.184219in}}%
\pgfpathlineto{\pgfqpoint{4.024073in}{1.184219in}}%
\pgfpathlineto{\pgfqpoint{4.024073in}{1.187168in}}%
\pgfpathlineto{\pgfqpoint{4.028614in}{1.187168in}}%
\pgfpathlineto{\pgfqpoint{4.028614in}{1.184219in}}%
\pgfpathmoveto{\pgfqpoint{4.024073in}{1.187168in}}%
\pgfpathlineto{\pgfqpoint{4.024073in}{1.187168in}}%
\pgfpathlineto{\pgfqpoint{4.024073in}{1.190117in}}%
\pgfpathlineto{\pgfqpoint{4.028614in}{1.190117in}}%
\pgfpathlineto{\pgfqpoint{4.028614in}{1.187168in}}%
\pgfpathmoveto{\pgfqpoint{4.019531in}{1.190117in}}%
\pgfpathlineto{\pgfqpoint{4.019531in}{1.190117in}}%
\pgfpathlineto{\pgfqpoint{4.019531in}{1.193067in}}%
\pgfpathlineto{\pgfqpoint{4.024073in}{1.193067in}}%
\pgfpathlineto{\pgfqpoint{4.024073in}{1.190117in}}%
\pgfpathmoveto{\pgfqpoint{3.946872in}{1.255002in}}%
\pgfpathlineto{\pgfqpoint{3.946872in}{1.255002in}}%
\pgfpathlineto{\pgfqpoint{3.946872in}{1.257951in}}%
\pgfpathlineto{\pgfqpoint{3.951413in}{1.257951in}}%
\pgfpathlineto{\pgfqpoint{3.951413in}{1.255002in}}%
\pgfpathmoveto{\pgfqpoint{4.232958in}{1.001368in}}%
\pgfpathlineto{\pgfqpoint{4.232958in}{1.001368in}}%
\pgfpathlineto{\pgfqpoint{4.232958in}{1.004317in}}%
\pgfpathlineto{\pgfqpoint{4.237498in}{1.004317in}}%
\pgfpathlineto{\pgfqpoint{4.237498in}{1.001368in}}%
\pgfpathmoveto{\pgfqpoint{4.232958in}{1.004317in}}%
\pgfpathlineto{\pgfqpoint{4.232958in}{1.004317in}}%
\pgfpathlineto{\pgfqpoint{4.232958in}{1.007266in}}%
\pgfpathlineto{\pgfqpoint{4.237498in}{1.007266in}}%
\pgfpathlineto{\pgfqpoint{4.237498in}{1.004317in}}%
\pgfpathmoveto{\pgfqpoint{4.223876in}{1.010215in}}%
\pgfpathlineto{\pgfqpoint{4.223876in}{1.010215in}}%
\pgfpathlineto{\pgfqpoint{4.223876in}{1.013164in}}%
\pgfpathlineto{\pgfqpoint{4.228417in}{1.013164in}}%
\pgfpathlineto{\pgfqpoint{4.228417in}{1.010215in}}%
\pgfpathmoveto{\pgfqpoint{4.219335in}{1.013164in}}%
\pgfpathlineto{\pgfqpoint{4.219335in}{1.013164in}}%
\pgfpathlineto{\pgfqpoint{4.219335in}{1.016113in}}%
\pgfpathlineto{\pgfqpoint{4.223876in}{1.016113in}}%
\pgfpathlineto{\pgfqpoint{4.223876in}{1.013164in}}%
\pgfpathmoveto{\pgfqpoint{4.219335in}{1.016113in}}%
\pgfpathlineto{\pgfqpoint{4.219335in}{1.016113in}}%
\pgfpathlineto{\pgfqpoint{4.219335in}{1.019062in}}%
\pgfpathlineto{\pgfqpoint{4.223876in}{1.019062in}}%
\pgfpathlineto{\pgfqpoint{4.223876in}{1.016113in}}%
\pgfpathmoveto{\pgfqpoint{4.223876in}{1.013164in}}%
\pgfpathlineto{\pgfqpoint{4.223876in}{1.013164in}}%
\pgfpathlineto{\pgfqpoint{4.223876in}{1.016113in}}%
\pgfpathlineto{\pgfqpoint{4.228417in}{1.016113in}}%
\pgfpathlineto{\pgfqpoint{4.228417in}{1.013164in}}%
\pgfpathmoveto{\pgfqpoint{4.228417in}{1.007266in}}%
\pgfpathlineto{\pgfqpoint{4.228417in}{1.007266in}}%
\pgfpathlineto{\pgfqpoint{4.228417in}{1.010215in}}%
\pgfpathlineto{\pgfqpoint{4.232958in}{1.010215in}}%
\pgfpathlineto{\pgfqpoint{4.232958in}{1.007266in}}%
\pgfpathmoveto{\pgfqpoint{4.228417in}{1.010215in}}%
\pgfpathlineto{\pgfqpoint{4.228417in}{1.010215in}}%
\pgfpathlineto{\pgfqpoint{4.228417in}{1.013164in}}%
\pgfpathlineto{\pgfqpoint{4.232958in}{1.013164in}}%
\pgfpathlineto{\pgfqpoint{4.232958in}{1.010215in}}%
\pgfpathmoveto{\pgfqpoint{4.232958in}{1.007266in}}%
\pgfpathlineto{\pgfqpoint{4.232958in}{1.007266in}}%
\pgfpathlineto{\pgfqpoint{4.232958in}{1.010215in}}%
\pgfpathlineto{\pgfqpoint{4.237498in}{1.010215in}}%
\pgfpathlineto{\pgfqpoint{4.237498in}{1.007266in}}%
\pgfpathmoveto{\pgfqpoint{4.196631in}{1.033808in}}%
\pgfpathlineto{\pgfqpoint{4.196631in}{1.033808in}}%
\pgfpathlineto{\pgfqpoint{4.196631in}{1.036757in}}%
\pgfpathlineto{\pgfqpoint{4.201172in}{1.036757in}}%
\pgfpathlineto{\pgfqpoint{4.201172in}{1.033808in}}%
\pgfpathmoveto{\pgfqpoint{4.192090in}{1.036757in}}%
\pgfpathlineto{\pgfqpoint{4.192090in}{1.036757in}}%
\pgfpathlineto{\pgfqpoint{4.192090in}{1.039706in}}%
\pgfpathlineto{\pgfqpoint{4.196631in}{1.039706in}}%
\pgfpathlineto{\pgfqpoint{4.196631in}{1.036757in}}%
\pgfpathmoveto{\pgfqpoint{4.192090in}{1.039706in}}%
\pgfpathlineto{\pgfqpoint{4.192090in}{1.039706in}}%
\pgfpathlineto{\pgfqpoint{4.192090in}{1.042655in}}%
\pgfpathlineto{\pgfqpoint{4.196631in}{1.042655in}}%
\pgfpathlineto{\pgfqpoint{4.196631in}{1.039706in}}%
\pgfpathmoveto{\pgfqpoint{4.196631in}{1.036757in}}%
\pgfpathlineto{\pgfqpoint{4.196631in}{1.036757in}}%
\pgfpathlineto{\pgfqpoint{4.196631in}{1.039706in}}%
\pgfpathlineto{\pgfqpoint{4.201172in}{1.039706in}}%
\pgfpathlineto{\pgfqpoint{4.201172in}{1.036757in}}%
\pgfpathmoveto{\pgfqpoint{4.178467in}{1.048553in}}%
\pgfpathlineto{\pgfqpoint{4.178467in}{1.048553in}}%
\pgfpathlineto{\pgfqpoint{4.178467in}{1.051502in}}%
\pgfpathlineto{\pgfqpoint{4.183008in}{1.051502in}}%
\pgfpathlineto{\pgfqpoint{4.183008in}{1.048553in}}%
\pgfpathmoveto{\pgfqpoint{4.178467in}{1.051502in}}%
\pgfpathlineto{\pgfqpoint{4.178467in}{1.051502in}}%
\pgfpathlineto{\pgfqpoint{4.178467in}{1.054451in}}%
\pgfpathlineto{\pgfqpoint{4.183008in}{1.054451in}}%
\pgfpathlineto{\pgfqpoint{4.183008in}{1.051502in}}%
\pgfpathmoveto{\pgfqpoint{4.169386in}{1.057400in}}%
\pgfpathlineto{\pgfqpoint{4.169386in}{1.057400in}}%
\pgfpathlineto{\pgfqpoint{4.169386in}{1.060349in}}%
\pgfpathlineto{\pgfqpoint{4.173926in}{1.060349in}}%
\pgfpathlineto{\pgfqpoint{4.173926in}{1.057400in}}%
\pgfpathmoveto{\pgfqpoint{4.164845in}{1.060349in}}%
\pgfpathlineto{\pgfqpoint{4.164845in}{1.060349in}}%
\pgfpathlineto{\pgfqpoint{4.164845in}{1.063298in}}%
\pgfpathlineto{\pgfqpoint{4.169386in}{1.063298in}}%
\pgfpathlineto{\pgfqpoint{4.169386in}{1.060349in}}%
\pgfpathmoveto{\pgfqpoint{4.164845in}{1.063298in}}%
\pgfpathlineto{\pgfqpoint{4.164845in}{1.063298in}}%
\pgfpathlineto{\pgfqpoint{4.164845in}{1.066247in}}%
\pgfpathlineto{\pgfqpoint{4.169386in}{1.066247in}}%
\pgfpathlineto{\pgfqpoint{4.169386in}{1.063298in}}%
\pgfpathmoveto{\pgfqpoint{4.169386in}{1.060349in}}%
\pgfpathlineto{\pgfqpoint{4.169386in}{1.060349in}}%
\pgfpathlineto{\pgfqpoint{4.169386in}{1.063298in}}%
\pgfpathlineto{\pgfqpoint{4.173926in}{1.063298in}}%
\pgfpathlineto{\pgfqpoint{4.173926in}{1.060349in}}%
\pgfpathmoveto{\pgfqpoint{4.173926in}{1.054451in}}%
\pgfpathlineto{\pgfqpoint{4.173926in}{1.054451in}}%
\pgfpathlineto{\pgfqpoint{4.173926in}{1.057400in}}%
\pgfpathlineto{\pgfqpoint{4.178467in}{1.057400in}}%
\pgfpathlineto{\pgfqpoint{4.178467in}{1.054451in}}%
\pgfpathmoveto{\pgfqpoint{4.173926in}{1.057400in}}%
\pgfpathlineto{\pgfqpoint{4.173926in}{1.057400in}}%
\pgfpathlineto{\pgfqpoint{4.173926in}{1.060349in}}%
\pgfpathlineto{\pgfqpoint{4.178467in}{1.060349in}}%
\pgfpathlineto{\pgfqpoint{4.178467in}{1.057400in}}%
\pgfpathmoveto{\pgfqpoint{4.178467in}{1.054451in}}%
\pgfpathlineto{\pgfqpoint{4.178467in}{1.054451in}}%
\pgfpathlineto{\pgfqpoint{4.178467in}{1.057400in}}%
\pgfpathlineto{\pgfqpoint{4.183008in}{1.057400in}}%
\pgfpathlineto{\pgfqpoint{4.183008in}{1.054451in}}%
\pgfpathmoveto{\pgfqpoint{4.183008in}{1.045604in}}%
\pgfpathlineto{\pgfqpoint{4.183008in}{1.045604in}}%
\pgfpathlineto{\pgfqpoint{4.183008in}{1.048553in}}%
\pgfpathlineto{\pgfqpoint{4.187549in}{1.048553in}}%
\pgfpathlineto{\pgfqpoint{4.187549in}{1.045604in}}%
\pgfpathmoveto{\pgfqpoint{4.187549in}{1.042655in}}%
\pgfpathlineto{\pgfqpoint{4.187549in}{1.042655in}}%
\pgfpathlineto{\pgfqpoint{4.187549in}{1.045604in}}%
\pgfpathlineto{\pgfqpoint{4.192090in}{1.045604in}}%
\pgfpathlineto{\pgfqpoint{4.192090in}{1.042655in}}%
\pgfpathmoveto{\pgfqpoint{4.187549in}{1.045604in}}%
\pgfpathlineto{\pgfqpoint{4.187549in}{1.045604in}}%
\pgfpathlineto{\pgfqpoint{4.187549in}{1.048553in}}%
\pgfpathlineto{\pgfqpoint{4.192090in}{1.048553in}}%
\pgfpathlineto{\pgfqpoint{4.192090in}{1.045604in}}%
\pgfpathmoveto{\pgfqpoint{4.183008in}{1.048553in}}%
\pgfpathlineto{\pgfqpoint{4.183008in}{1.048553in}}%
\pgfpathlineto{\pgfqpoint{4.183008in}{1.051502in}}%
\pgfpathlineto{\pgfqpoint{4.187549in}{1.051502in}}%
\pgfpathlineto{\pgfqpoint{4.187549in}{1.048553in}}%
\pgfpathmoveto{\pgfqpoint{4.192090in}{1.042655in}}%
\pgfpathlineto{\pgfqpoint{4.192090in}{1.042655in}}%
\pgfpathlineto{\pgfqpoint{4.192090in}{1.045604in}}%
\pgfpathlineto{\pgfqpoint{4.196631in}{1.045604in}}%
\pgfpathlineto{\pgfqpoint{4.196631in}{1.042655in}}%
\pgfpathmoveto{\pgfqpoint{4.205712in}{1.024961in}}%
\pgfpathlineto{\pgfqpoint{4.205712in}{1.024961in}}%
\pgfpathlineto{\pgfqpoint{4.205712in}{1.027910in}}%
\pgfpathlineto{\pgfqpoint{4.210253in}{1.027910in}}%
\pgfpathlineto{\pgfqpoint{4.210253in}{1.024961in}}%
\pgfpathmoveto{\pgfqpoint{4.205712in}{1.027910in}}%
\pgfpathlineto{\pgfqpoint{4.205712in}{1.027910in}}%
\pgfpathlineto{\pgfqpoint{4.205712in}{1.030859in}}%
\pgfpathlineto{\pgfqpoint{4.210253in}{1.030859in}}%
\pgfpathlineto{\pgfqpoint{4.210253in}{1.027910in}}%
\pgfpathmoveto{\pgfqpoint{4.210253in}{1.022011in}}%
\pgfpathlineto{\pgfqpoint{4.210253in}{1.022011in}}%
\pgfpathlineto{\pgfqpoint{4.210253in}{1.024961in}}%
\pgfpathlineto{\pgfqpoint{4.214794in}{1.024961in}}%
\pgfpathlineto{\pgfqpoint{4.214794in}{1.022011in}}%
\pgfpathmoveto{\pgfqpoint{4.214794in}{1.019062in}}%
\pgfpathlineto{\pgfqpoint{4.214794in}{1.019062in}}%
\pgfpathlineto{\pgfqpoint{4.214794in}{1.022011in}}%
\pgfpathlineto{\pgfqpoint{4.219335in}{1.022011in}}%
\pgfpathlineto{\pgfqpoint{4.219335in}{1.019062in}}%
\pgfpathmoveto{\pgfqpoint{4.214794in}{1.022011in}}%
\pgfpathlineto{\pgfqpoint{4.214794in}{1.022011in}}%
\pgfpathlineto{\pgfqpoint{4.214794in}{1.024961in}}%
\pgfpathlineto{\pgfqpoint{4.219335in}{1.024961in}}%
\pgfpathlineto{\pgfqpoint{4.219335in}{1.022011in}}%
\pgfpathmoveto{\pgfqpoint{4.210253in}{1.024961in}}%
\pgfpathlineto{\pgfqpoint{4.210253in}{1.024961in}}%
\pgfpathlineto{\pgfqpoint{4.210253in}{1.027910in}}%
\pgfpathlineto{\pgfqpoint{4.214794in}{1.027910in}}%
\pgfpathlineto{\pgfqpoint{4.214794in}{1.024961in}}%
\pgfpathmoveto{\pgfqpoint{4.201172in}{1.030859in}}%
\pgfpathlineto{\pgfqpoint{4.201172in}{1.030859in}}%
\pgfpathlineto{\pgfqpoint{4.201172in}{1.033808in}}%
\pgfpathlineto{\pgfqpoint{4.205712in}{1.033808in}}%
\pgfpathlineto{\pgfqpoint{4.205712in}{1.030859in}}%
\pgfpathmoveto{\pgfqpoint{4.201172in}{1.033808in}}%
\pgfpathlineto{\pgfqpoint{4.201172in}{1.033808in}}%
\pgfpathlineto{\pgfqpoint{4.201172in}{1.036757in}}%
\pgfpathlineto{\pgfqpoint{4.205712in}{1.036757in}}%
\pgfpathlineto{\pgfqpoint{4.205712in}{1.033808in}}%
\pgfpathmoveto{\pgfqpoint{4.205712in}{1.030859in}}%
\pgfpathlineto{\pgfqpoint{4.205712in}{1.030859in}}%
\pgfpathlineto{\pgfqpoint{4.205712in}{1.033808in}}%
\pgfpathlineto{\pgfqpoint{4.210253in}{1.033808in}}%
\pgfpathlineto{\pgfqpoint{4.210253in}{1.030859in}}%
\pgfpathmoveto{\pgfqpoint{4.219335in}{1.019062in}}%
\pgfpathlineto{\pgfqpoint{4.219335in}{1.019062in}}%
\pgfpathlineto{\pgfqpoint{4.219335in}{1.022011in}}%
\pgfpathlineto{\pgfqpoint{4.223876in}{1.022011in}}%
\pgfpathlineto{\pgfqpoint{4.223876in}{1.019062in}}%
\pgfpathmoveto{\pgfqpoint{4.105814in}{1.110486in}}%
\pgfpathlineto{\pgfqpoint{4.105814in}{1.110486in}}%
\pgfpathlineto{\pgfqpoint{4.105814in}{1.113436in}}%
\pgfpathlineto{\pgfqpoint{4.110354in}{1.113436in}}%
\pgfpathlineto{\pgfqpoint{4.110354in}{1.110486in}}%
\pgfpathmoveto{\pgfqpoint{4.123977in}{1.095740in}}%
\pgfpathlineto{\pgfqpoint{4.123977in}{1.095740in}}%
\pgfpathlineto{\pgfqpoint{4.123977in}{1.098689in}}%
\pgfpathlineto{\pgfqpoint{4.128518in}{1.098689in}}%
\pgfpathlineto{\pgfqpoint{4.128518in}{1.095740in}}%
\pgfpathmoveto{\pgfqpoint{4.123977in}{1.098689in}}%
\pgfpathlineto{\pgfqpoint{4.123977in}{1.098689in}}%
\pgfpathlineto{\pgfqpoint{4.123977in}{1.101639in}}%
\pgfpathlineto{\pgfqpoint{4.128518in}{1.101639in}}%
\pgfpathlineto{\pgfqpoint{4.128518in}{1.098689in}}%
\pgfpathmoveto{\pgfqpoint{4.114895in}{1.104588in}}%
\pgfpathlineto{\pgfqpoint{4.114895in}{1.104588in}}%
\pgfpathlineto{\pgfqpoint{4.114895in}{1.107537in}}%
\pgfpathlineto{\pgfqpoint{4.119436in}{1.107537in}}%
\pgfpathlineto{\pgfqpoint{4.119436in}{1.104588in}}%
\pgfpathmoveto{\pgfqpoint{4.110354in}{1.107537in}}%
\pgfpathlineto{\pgfqpoint{4.110354in}{1.107537in}}%
\pgfpathlineto{\pgfqpoint{4.110354in}{1.110486in}}%
\pgfpathlineto{\pgfqpoint{4.114895in}{1.110486in}}%
\pgfpathlineto{\pgfqpoint{4.114895in}{1.107537in}}%
\pgfpathmoveto{\pgfqpoint{4.110354in}{1.110486in}}%
\pgfpathlineto{\pgfqpoint{4.110354in}{1.110486in}}%
\pgfpathlineto{\pgfqpoint{4.110354in}{1.113436in}}%
\pgfpathlineto{\pgfqpoint{4.114895in}{1.113436in}}%
\pgfpathlineto{\pgfqpoint{4.114895in}{1.110486in}}%
\pgfpathmoveto{\pgfqpoint{4.114895in}{1.107537in}}%
\pgfpathlineto{\pgfqpoint{4.114895in}{1.107537in}}%
\pgfpathlineto{\pgfqpoint{4.114895in}{1.110486in}}%
\pgfpathlineto{\pgfqpoint{4.119436in}{1.110486in}}%
\pgfpathlineto{\pgfqpoint{4.119436in}{1.107537in}}%
\pgfpathmoveto{\pgfqpoint{4.119436in}{1.101639in}}%
\pgfpathlineto{\pgfqpoint{4.119436in}{1.101639in}}%
\pgfpathlineto{\pgfqpoint{4.119436in}{1.104588in}}%
\pgfpathlineto{\pgfqpoint{4.123977in}{1.104588in}}%
\pgfpathlineto{\pgfqpoint{4.123977in}{1.101639in}}%
\pgfpathmoveto{\pgfqpoint{4.119436in}{1.104588in}}%
\pgfpathlineto{\pgfqpoint{4.119436in}{1.104588in}}%
\pgfpathlineto{\pgfqpoint{4.119436in}{1.107537in}}%
\pgfpathlineto{\pgfqpoint{4.123977in}{1.107537in}}%
\pgfpathlineto{\pgfqpoint{4.123977in}{1.104588in}}%
\pgfpathmoveto{\pgfqpoint{4.123977in}{1.101639in}}%
\pgfpathlineto{\pgfqpoint{4.123977in}{1.101639in}}%
\pgfpathlineto{\pgfqpoint{4.123977in}{1.104588in}}%
\pgfpathlineto{\pgfqpoint{4.128518in}{1.104588in}}%
\pgfpathlineto{\pgfqpoint{4.128518in}{1.101639in}}%
\pgfpathmoveto{\pgfqpoint{4.142140in}{1.080994in}}%
\pgfpathlineto{\pgfqpoint{4.142140in}{1.080994in}}%
\pgfpathlineto{\pgfqpoint{4.142140in}{1.083943in}}%
\pgfpathlineto{\pgfqpoint{4.146681in}{1.083943in}}%
\pgfpathlineto{\pgfqpoint{4.146681in}{1.080994in}}%
\pgfpathmoveto{\pgfqpoint{4.137600in}{1.083943in}}%
\pgfpathlineto{\pgfqpoint{4.137600in}{1.083943in}}%
\pgfpathlineto{\pgfqpoint{4.137600in}{1.086892in}}%
\pgfpathlineto{\pgfqpoint{4.142140in}{1.086892in}}%
\pgfpathlineto{\pgfqpoint{4.142140in}{1.083943in}}%
\pgfpathmoveto{\pgfqpoint{4.137600in}{1.086892in}}%
\pgfpathlineto{\pgfqpoint{4.137600in}{1.086892in}}%
\pgfpathlineto{\pgfqpoint{4.137600in}{1.089841in}}%
\pgfpathlineto{\pgfqpoint{4.142140in}{1.089841in}}%
\pgfpathlineto{\pgfqpoint{4.142140in}{1.086892in}}%
\pgfpathmoveto{\pgfqpoint{4.142140in}{1.083943in}}%
\pgfpathlineto{\pgfqpoint{4.142140in}{1.083943in}}%
\pgfpathlineto{\pgfqpoint{4.142140in}{1.086892in}}%
\pgfpathlineto{\pgfqpoint{4.146681in}{1.086892in}}%
\pgfpathlineto{\pgfqpoint{4.146681in}{1.083943in}}%
\pgfpathmoveto{\pgfqpoint{4.151222in}{1.072146in}}%
\pgfpathlineto{\pgfqpoint{4.151222in}{1.072146in}}%
\pgfpathlineto{\pgfqpoint{4.151222in}{1.075095in}}%
\pgfpathlineto{\pgfqpoint{4.155763in}{1.075095in}}%
\pgfpathlineto{\pgfqpoint{4.155763in}{1.072146in}}%
\pgfpathmoveto{\pgfqpoint{4.151222in}{1.075095in}}%
\pgfpathlineto{\pgfqpoint{4.151222in}{1.075095in}}%
\pgfpathlineto{\pgfqpoint{4.151222in}{1.078044in}}%
\pgfpathlineto{\pgfqpoint{4.155763in}{1.078044in}}%
\pgfpathlineto{\pgfqpoint{4.155763in}{1.075095in}}%
\pgfpathmoveto{\pgfqpoint{4.155763in}{1.069196in}}%
\pgfpathlineto{\pgfqpoint{4.155763in}{1.069196in}}%
\pgfpathlineto{\pgfqpoint{4.155763in}{1.072146in}}%
\pgfpathlineto{\pgfqpoint{4.160304in}{1.072146in}}%
\pgfpathlineto{\pgfqpoint{4.160304in}{1.069196in}}%
\pgfpathmoveto{\pgfqpoint{4.160304in}{1.066247in}}%
\pgfpathlineto{\pgfqpoint{4.160304in}{1.066247in}}%
\pgfpathlineto{\pgfqpoint{4.160304in}{1.069196in}}%
\pgfpathlineto{\pgfqpoint{4.164845in}{1.069196in}}%
\pgfpathlineto{\pgfqpoint{4.164845in}{1.066247in}}%
\pgfpathmoveto{\pgfqpoint{4.160304in}{1.069196in}}%
\pgfpathlineto{\pgfqpoint{4.160304in}{1.069196in}}%
\pgfpathlineto{\pgfqpoint{4.160304in}{1.072146in}}%
\pgfpathlineto{\pgfqpoint{4.164845in}{1.072146in}}%
\pgfpathlineto{\pgfqpoint{4.164845in}{1.069196in}}%
\pgfpathmoveto{\pgfqpoint{4.155763in}{1.072146in}}%
\pgfpathlineto{\pgfqpoint{4.155763in}{1.072146in}}%
\pgfpathlineto{\pgfqpoint{4.155763in}{1.075095in}}%
\pgfpathlineto{\pgfqpoint{4.160304in}{1.075095in}}%
\pgfpathlineto{\pgfqpoint{4.160304in}{1.072146in}}%
\pgfpathmoveto{\pgfqpoint{4.146681in}{1.078044in}}%
\pgfpathlineto{\pgfqpoint{4.146681in}{1.078044in}}%
\pgfpathlineto{\pgfqpoint{4.146681in}{1.080994in}}%
\pgfpathlineto{\pgfqpoint{4.151222in}{1.080994in}}%
\pgfpathlineto{\pgfqpoint{4.151222in}{1.078044in}}%
\pgfpathmoveto{\pgfqpoint{4.146681in}{1.080994in}}%
\pgfpathlineto{\pgfqpoint{4.146681in}{1.080994in}}%
\pgfpathlineto{\pgfqpoint{4.146681in}{1.083943in}}%
\pgfpathlineto{\pgfqpoint{4.151222in}{1.083943in}}%
\pgfpathlineto{\pgfqpoint{4.151222in}{1.080994in}}%
\pgfpathmoveto{\pgfqpoint{4.151222in}{1.078044in}}%
\pgfpathlineto{\pgfqpoint{4.151222in}{1.078044in}}%
\pgfpathlineto{\pgfqpoint{4.151222in}{1.080994in}}%
\pgfpathlineto{\pgfqpoint{4.155763in}{1.080994in}}%
\pgfpathlineto{\pgfqpoint{4.155763in}{1.078044in}}%
\pgfpathmoveto{\pgfqpoint{4.128518in}{1.092791in}}%
\pgfpathlineto{\pgfqpoint{4.128518in}{1.092791in}}%
\pgfpathlineto{\pgfqpoint{4.128518in}{1.095740in}}%
\pgfpathlineto{\pgfqpoint{4.133059in}{1.095740in}}%
\pgfpathlineto{\pgfqpoint{4.133059in}{1.092791in}}%
\pgfpathmoveto{\pgfqpoint{4.133059in}{1.089841in}}%
\pgfpathlineto{\pgfqpoint{4.133059in}{1.089841in}}%
\pgfpathlineto{\pgfqpoint{4.133059in}{1.092791in}}%
\pgfpathlineto{\pgfqpoint{4.137600in}{1.092791in}}%
\pgfpathlineto{\pgfqpoint{4.137600in}{1.089841in}}%
\pgfpathmoveto{\pgfqpoint{4.133059in}{1.092791in}}%
\pgfpathlineto{\pgfqpoint{4.133059in}{1.092791in}}%
\pgfpathlineto{\pgfqpoint{4.133059in}{1.095740in}}%
\pgfpathlineto{\pgfqpoint{4.137600in}{1.095740in}}%
\pgfpathlineto{\pgfqpoint{4.137600in}{1.092791in}}%
\pgfpathmoveto{\pgfqpoint{4.128518in}{1.095740in}}%
\pgfpathlineto{\pgfqpoint{4.128518in}{1.095740in}}%
\pgfpathlineto{\pgfqpoint{4.128518in}{1.098689in}}%
\pgfpathlineto{\pgfqpoint{4.133059in}{1.098689in}}%
\pgfpathlineto{\pgfqpoint{4.133059in}{1.095740in}}%
\pgfpathmoveto{\pgfqpoint{4.137600in}{1.089841in}}%
\pgfpathlineto{\pgfqpoint{4.137600in}{1.089841in}}%
\pgfpathlineto{\pgfqpoint{4.137600in}{1.092791in}}%
\pgfpathlineto{\pgfqpoint{4.142140in}{1.092791in}}%
\pgfpathlineto{\pgfqpoint{4.142140in}{1.089841in}}%
\pgfpathmoveto{\pgfqpoint{4.092191in}{1.122284in}}%
\pgfpathlineto{\pgfqpoint{4.092191in}{1.122284in}}%
\pgfpathlineto{\pgfqpoint{4.092191in}{1.125233in}}%
\pgfpathlineto{\pgfqpoint{4.096732in}{1.125233in}}%
\pgfpathlineto{\pgfqpoint{4.096732in}{1.122284in}}%
\pgfpathmoveto{\pgfqpoint{4.096732in}{1.119334in}}%
\pgfpathlineto{\pgfqpoint{4.096732in}{1.119334in}}%
\pgfpathlineto{\pgfqpoint{4.096732in}{1.122284in}}%
\pgfpathlineto{\pgfqpoint{4.101273in}{1.122284in}}%
\pgfpathlineto{\pgfqpoint{4.101273in}{1.119334in}}%
\pgfpathmoveto{\pgfqpoint{4.096732in}{1.122284in}}%
\pgfpathlineto{\pgfqpoint{4.096732in}{1.122284in}}%
\pgfpathlineto{\pgfqpoint{4.096732in}{1.125233in}}%
\pgfpathlineto{\pgfqpoint{4.101273in}{1.125233in}}%
\pgfpathlineto{\pgfqpoint{4.101273in}{1.122284in}}%
\pgfpathmoveto{\pgfqpoint{4.101273in}{1.116385in}}%
\pgfpathlineto{\pgfqpoint{4.101273in}{1.116385in}}%
\pgfpathlineto{\pgfqpoint{4.101273in}{1.119334in}}%
\pgfpathlineto{\pgfqpoint{4.105814in}{1.119334in}}%
\pgfpathlineto{\pgfqpoint{4.105814in}{1.116385in}}%
\pgfpathmoveto{\pgfqpoint{4.105814in}{1.113436in}}%
\pgfpathlineto{\pgfqpoint{4.105814in}{1.113436in}}%
\pgfpathlineto{\pgfqpoint{4.105814in}{1.116385in}}%
\pgfpathlineto{\pgfqpoint{4.110354in}{1.116385in}}%
\pgfpathlineto{\pgfqpoint{4.110354in}{1.113436in}}%
\pgfpathmoveto{\pgfqpoint{4.105814in}{1.116385in}}%
\pgfpathlineto{\pgfqpoint{4.105814in}{1.116385in}}%
\pgfpathlineto{\pgfqpoint{4.105814in}{1.119334in}}%
\pgfpathlineto{\pgfqpoint{4.110354in}{1.119334in}}%
\pgfpathlineto{\pgfqpoint{4.110354in}{1.116385in}}%
\pgfpathmoveto{\pgfqpoint{4.101273in}{1.119334in}}%
\pgfpathlineto{\pgfqpoint{4.101273in}{1.119334in}}%
\pgfpathlineto{\pgfqpoint{4.101273in}{1.122284in}}%
\pgfpathlineto{\pgfqpoint{4.105814in}{1.122284in}}%
\pgfpathlineto{\pgfqpoint{4.105814in}{1.119334in}}%
\pgfpathmoveto{\pgfqpoint{4.092191in}{1.125233in}}%
\pgfpathlineto{\pgfqpoint{4.092191in}{1.125233in}}%
\pgfpathlineto{\pgfqpoint{4.092191in}{1.128182in}}%
\pgfpathlineto{\pgfqpoint{4.096732in}{1.128182in}}%
\pgfpathlineto{\pgfqpoint{4.096732in}{1.125233in}}%
\pgfpathmoveto{\pgfqpoint{4.092191in}{1.128182in}}%
\pgfpathlineto{\pgfqpoint{4.092191in}{1.128182in}}%
\pgfpathlineto{\pgfqpoint{4.092191in}{1.131131in}}%
\pgfpathlineto{\pgfqpoint{4.096732in}{1.131131in}}%
\pgfpathlineto{\pgfqpoint{4.096732in}{1.128182in}}%
\pgfpathmoveto{\pgfqpoint{4.164845in}{1.066247in}}%
\pgfpathlineto{\pgfqpoint{4.164845in}{1.066247in}}%
\pgfpathlineto{\pgfqpoint{4.164845in}{1.069196in}}%
\pgfpathlineto{\pgfqpoint{4.169386in}{1.069196in}}%
\pgfpathlineto{\pgfqpoint{4.169386in}{1.066247in}}%
\pgfpathmoveto{\pgfqpoint{4.269286in}{0.968928in}}%
\pgfpathlineto{\pgfqpoint{4.269286in}{0.968928in}}%
\pgfpathlineto{\pgfqpoint{4.269286in}{0.971878in}}%
\pgfpathlineto{\pgfqpoint{4.273827in}{0.971878in}}%
\pgfpathlineto{\pgfqpoint{4.273827in}{0.968928in}}%
\pgfpathmoveto{\pgfqpoint{4.296532in}{0.945334in}}%
\pgfpathlineto{\pgfqpoint{4.296532in}{0.945334in}}%
\pgfpathlineto{\pgfqpoint{4.296532in}{0.948283in}}%
\pgfpathlineto{\pgfqpoint{4.301073in}{0.948283in}}%
\pgfpathlineto{\pgfqpoint{4.301073in}{0.945334in}}%
\pgfpathmoveto{\pgfqpoint{4.305614in}{0.939435in}}%
\pgfpathlineto{\pgfqpoint{4.305614in}{0.939435in}}%
\pgfpathlineto{\pgfqpoint{4.305614in}{0.942384in}}%
\pgfpathlineto{\pgfqpoint{4.310155in}{0.942384in}}%
\pgfpathlineto{\pgfqpoint{4.310155in}{0.939435in}}%
\pgfpathmoveto{\pgfqpoint{4.301073in}{0.942384in}}%
\pgfpathlineto{\pgfqpoint{4.301073in}{0.942384in}}%
\pgfpathlineto{\pgfqpoint{4.301073in}{0.945334in}}%
\pgfpathlineto{\pgfqpoint{4.305614in}{0.945334in}}%
\pgfpathlineto{\pgfqpoint{4.305614in}{0.942384in}}%
\pgfpathmoveto{\pgfqpoint{4.301073in}{0.945334in}}%
\pgfpathlineto{\pgfqpoint{4.301073in}{0.945334in}}%
\pgfpathlineto{\pgfqpoint{4.301073in}{0.948283in}}%
\pgfpathlineto{\pgfqpoint{4.305614in}{0.948283in}}%
\pgfpathlineto{\pgfqpoint{4.305614in}{0.945334in}}%
\pgfpathmoveto{\pgfqpoint{4.305614in}{0.942384in}}%
\pgfpathlineto{\pgfqpoint{4.305614in}{0.942384in}}%
\pgfpathlineto{\pgfqpoint{4.305614in}{0.945334in}}%
\pgfpathlineto{\pgfqpoint{4.310155in}{0.945334in}}%
\pgfpathlineto{\pgfqpoint{4.310155in}{0.942384in}}%
\pgfpathmoveto{\pgfqpoint{4.282909in}{0.957131in}}%
\pgfpathlineto{\pgfqpoint{4.282909in}{0.957131in}}%
\pgfpathlineto{\pgfqpoint{4.282909in}{0.960080in}}%
\pgfpathlineto{\pgfqpoint{4.287450in}{0.960080in}}%
\pgfpathlineto{\pgfqpoint{4.287450in}{0.957131in}}%
\pgfpathmoveto{\pgfqpoint{4.287450in}{0.954182in}}%
\pgfpathlineto{\pgfqpoint{4.287450in}{0.954182in}}%
\pgfpathlineto{\pgfqpoint{4.287450in}{0.957131in}}%
\pgfpathlineto{\pgfqpoint{4.291991in}{0.957131in}}%
\pgfpathlineto{\pgfqpoint{4.291991in}{0.954182in}}%
\pgfpathmoveto{\pgfqpoint{4.287450in}{0.957131in}}%
\pgfpathlineto{\pgfqpoint{4.287450in}{0.957131in}}%
\pgfpathlineto{\pgfqpoint{4.287450in}{0.960080in}}%
\pgfpathlineto{\pgfqpoint{4.291991in}{0.960080in}}%
\pgfpathlineto{\pgfqpoint{4.291991in}{0.957131in}}%
\pgfpathmoveto{\pgfqpoint{4.278368in}{0.963030in}}%
\pgfpathlineto{\pgfqpoint{4.278368in}{0.963030in}}%
\pgfpathlineto{\pgfqpoint{4.278368in}{0.965979in}}%
\pgfpathlineto{\pgfqpoint{4.282909in}{0.965979in}}%
\pgfpathlineto{\pgfqpoint{4.282909in}{0.963030in}}%
\pgfpathmoveto{\pgfqpoint{4.273827in}{0.965979in}}%
\pgfpathlineto{\pgfqpoint{4.273827in}{0.965979in}}%
\pgfpathlineto{\pgfqpoint{4.273827in}{0.968928in}}%
\pgfpathlineto{\pgfqpoint{4.278368in}{0.968928in}}%
\pgfpathlineto{\pgfqpoint{4.278368in}{0.965979in}}%
\pgfpathmoveto{\pgfqpoint{4.273827in}{0.968928in}}%
\pgfpathlineto{\pgfqpoint{4.273827in}{0.968928in}}%
\pgfpathlineto{\pgfqpoint{4.273827in}{0.971878in}}%
\pgfpathlineto{\pgfqpoint{4.278368in}{0.971878in}}%
\pgfpathlineto{\pgfqpoint{4.278368in}{0.968928in}}%
\pgfpathmoveto{\pgfqpoint{4.278368in}{0.965979in}}%
\pgfpathlineto{\pgfqpoint{4.278368in}{0.965979in}}%
\pgfpathlineto{\pgfqpoint{4.278368in}{0.968928in}}%
\pgfpathlineto{\pgfqpoint{4.282909in}{0.968928in}}%
\pgfpathlineto{\pgfqpoint{4.282909in}{0.965979in}}%
\pgfpathmoveto{\pgfqpoint{4.282909in}{0.960080in}}%
\pgfpathlineto{\pgfqpoint{4.282909in}{0.960080in}}%
\pgfpathlineto{\pgfqpoint{4.282909in}{0.963030in}}%
\pgfpathlineto{\pgfqpoint{4.287450in}{0.963030in}}%
\pgfpathlineto{\pgfqpoint{4.287450in}{0.960080in}}%
\pgfpathmoveto{\pgfqpoint{4.282909in}{0.963030in}}%
\pgfpathlineto{\pgfqpoint{4.282909in}{0.963030in}}%
\pgfpathlineto{\pgfqpoint{4.282909in}{0.965979in}}%
\pgfpathlineto{\pgfqpoint{4.287450in}{0.965979in}}%
\pgfpathlineto{\pgfqpoint{4.287450in}{0.963030in}}%
\pgfpathmoveto{\pgfqpoint{4.291991in}{0.951232in}}%
\pgfpathlineto{\pgfqpoint{4.291991in}{0.951232in}}%
\pgfpathlineto{\pgfqpoint{4.291991in}{0.954182in}}%
\pgfpathlineto{\pgfqpoint{4.296532in}{0.954182in}}%
\pgfpathlineto{\pgfqpoint{4.296532in}{0.951232in}}%
\pgfpathmoveto{\pgfqpoint{4.296532in}{0.948283in}}%
\pgfpathlineto{\pgfqpoint{4.296532in}{0.948283in}}%
\pgfpathlineto{\pgfqpoint{4.296532in}{0.951232in}}%
\pgfpathlineto{\pgfqpoint{4.301073in}{0.951232in}}%
\pgfpathlineto{\pgfqpoint{4.301073in}{0.948283in}}%
\pgfpathmoveto{\pgfqpoint{4.296532in}{0.951232in}}%
\pgfpathlineto{\pgfqpoint{4.296532in}{0.951232in}}%
\pgfpathlineto{\pgfqpoint{4.296532in}{0.954182in}}%
\pgfpathlineto{\pgfqpoint{4.301073in}{0.954182in}}%
\pgfpathlineto{\pgfqpoint{4.301073in}{0.951232in}}%
\pgfpathmoveto{\pgfqpoint{4.291991in}{0.954182in}}%
\pgfpathlineto{\pgfqpoint{4.291991in}{0.954182in}}%
\pgfpathlineto{\pgfqpoint{4.291991in}{0.957131in}}%
\pgfpathlineto{\pgfqpoint{4.296532in}{0.957131in}}%
\pgfpathlineto{\pgfqpoint{4.296532in}{0.954182in}}%
\pgfpathmoveto{\pgfqpoint{4.341942in}{0.906992in}}%
\pgfpathlineto{\pgfqpoint{4.341942in}{0.906992in}}%
\pgfpathlineto{\pgfqpoint{4.341942in}{0.909942in}}%
\pgfpathlineto{\pgfqpoint{4.346483in}{0.909942in}}%
\pgfpathlineto{\pgfqpoint{4.346483in}{0.906992in}}%
\pgfpathmoveto{\pgfqpoint{4.341942in}{0.909942in}}%
\pgfpathlineto{\pgfqpoint{4.341942in}{0.909942in}}%
\pgfpathlineto{\pgfqpoint{4.341942in}{0.912891in}}%
\pgfpathlineto{\pgfqpoint{4.346483in}{0.912891in}}%
\pgfpathlineto{\pgfqpoint{4.346483in}{0.909942in}}%
\pgfpathmoveto{\pgfqpoint{4.332860in}{0.915840in}}%
\pgfpathlineto{\pgfqpoint{4.332860in}{0.915840in}}%
\pgfpathlineto{\pgfqpoint{4.332860in}{0.918790in}}%
\pgfpathlineto{\pgfqpoint{4.337401in}{0.918790in}}%
\pgfpathlineto{\pgfqpoint{4.337401in}{0.915840in}}%
\pgfpathmoveto{\pgfqpoint{4.328319in}{0.918790in}}%
\pgfpathlineto{\pgfqpoint{4.328319in}{0.918790in}}%
\pgfpathlineto{\pgfqpoint{4.328319in}{0.921739in}}%
\pgfpathlineto{\pgfqpoint{4.332860in}{0.921739in}}%
\pgfpathlineto{\pgfqpoint{4.332860in}{0.918790in}}%
\pgfpathmoveto{\pgfqpoint{4.328319in}{0.921739in}}%
\pgfpathlineto{\pgfqpoint{4.328319in}{0.921739in}}%
\pgfpathlineto{\pgfqpoint{4.328319in}{0.924688in}}%
\pgfpathlineto{\pgfqpoint{4.332860in}{0.924688in}}%
\pgfpathlineto{\pgfqpoint{4.332860in}{0.921739in}}%
\pgfpathmoveto{\pgfqpoint{4.332860in}{0.918790in}}%
\pgfpathlineto{\pgfqpoint{4.332860in}{0.918790in}}%
\pgfpathlineto{\pgfqpoint{4.332860in}{0.921739in}}%
\pgfpathlineto{\pgfqpoint{4.337401in}{0.921739in}}%
\pgfpathlineto{\pgfqpoint{4.337401in}{0.918790in}}%
\pgfpathmoveto{\pgfqpoint{4.337401in}{0.912891in}}%
\pgfpathlineto{\pgfqpoint{4.337401in}{0.912891in}}%
\pgfpathlineto{\pgfqpoint{4.337401in}{0.915840in}}%
\pgfpathlineto{\pgfqpoint{4.341942in}{0.915840in}}%
\pgfpathlineto{\pgfqpoint{4.341942in}{0.912891in}}%
\pgfpathmoveto{\pgfqpoint{4.337401in}{0.915840in}}%
\pgfpathlineto{\pgfqpoint{4.337401in}{0.915840in}}%
\pgfpathlineto{\pgfqpoint{4.337401in}{0.918790in}}%
\pgfpathlineto{\pgfqpoint{4.341942in}{0.918790in}}%
\pgfpathlineto{\pgfqpoint{4.341942in}{0.915840in}}%
\pgfpathmoveto{\pgfqpoint{4.341942in}{0.912891in}}%
\pgfpathlineto{\pgfqpoint{4.341942in}{0.912891in}}%
\pgfpathlineto{\pgfqpoint{4.341942in}{0.915840in}}%
\pgfpathlineto{\pgfqpoint{4.346483in}{0.915840in}}%
\pgfpathlineto{\pgfqpoint{4.346483in}{0.912891in}}%
\pgfpathmoveto{\pgfqpoint{4.360106in}{0.892246in}}%
\pgfpathlineto{\pgfqpoint{4.360106in}{0.892246in}}%
\pgfpathlineto{\pgfqpoint{4.360106in}{0.895195in}}%
\pgfpathlineto{\pgfqpoint{4.364647in}{0.895195in}}%
\pgfpathlineto{\pgfqpoint{4.364647in}{0.892246in}}%
\pgfpathmoveto{\pgfqpoint{4.355565in}{0.895195in}}%
\pgfpathlineto{\pgfqpoint{4.355565in}{0.895195in}}%
\pgfpathlineto{\pgfqpoint{4.355565in}{0.898144in}}%
\pgfpathlineto{\pgfqpoint{4.360106in}{0.898144in}}%
\pgfpathlineto{\pgfqpoint{4.360106in}{0.895195in}}%
\pgfpathmoveto{\pgfqpoint{4.355565in}{0.898144in}}%
\pgfpathlineto{\pgfqpoint{4.355565in}{0.898144in}}%
\pgfpathlineto{\pgfqpoint{4.355565in}{0.901094in}}%
\pgfpathlineto{\pgfqpoint{4.360106in}{0.901094in}}%
\pgfpathlineto{\pgfqpoint{4.360106in}{0.898144in}}%
\pgfpathmoveto{\pgfqpoint{4.360106in}{0.895195in}}%
\pgfpathlineto{\pgfqpoint{4.360106in}{0.895195in}}%
\pgfpathlineto{\pgfqpoint{4.360106in}{0.898144in}}%
\pgfpathlineto{\pgfqpoint{4.364647in}{0.898144in}}%
\pgfpathlineto{\pgfqpoint{4.364647in}{0.895195in}}%
\pgfpathmoveto{\pgfqpoint{4.369189in}{0.883398in}}%
\pgfpathlineto{\pgfqpoint{4.369189in}{0.883398in}}%
\pgfpathlineto{\pgfqpoint{4.369189in}{0.886347in}}%
\pgfpathlineto{\pgfqpoint{4.373730in}{0.886347in}}%
\pgfpathlineto{\pgfqpoint{4.373730in}{0.883398in}}%
\pgfpathmoveto{\pgfqpoint{4.369189in}{0.886347in}}%
\pgfpathlineto{\pgfqpoint{4.369189in}{0.886347in}}%
\pgfpathlineto{\pgfqpoint{4.369189in}{0.889296in}}%
\pgfpathlineto{\pgfqpoint{4.373730in}{0.889296in}}%
\pgfpathlineto{\pgfqpoint{4.373730in}{0.886347in}}%
\pgfpathmoveto{\pgfqpoint{4.373730in}{0.880449in}}%
\pgfpathlineto{\pgfqpoint{4.373730in}{0.880449in}}%
\pgfpathlineto{\pgfqpoint{4.373730in}{0.883398in}}%
\pgfpathlineto{\pgfqpoint{4.378271in}{0.883398in}}%
\pgfpathlineto{\pgfqpoint{4.378271in}{0.880449in}}%
\pgfpathmoveto{\pgfqpoint{4.378271in}{0.877499in}}%
\pgfpathlineto{\pgfqpoint{4.378271in}{0.877499in}}%
\pgfpathlineto{\pgfqpoint{4.378271in}{0.880449in}}%
\pgfpathlineto{\pgfqpoint{4.382812in}{0.880449in}}%
\pgfpathlineto{\pgfqpoint{4.382812in}{0.877499in}}%
\pgfpathmoveto{\pgfqpoint{4.378271in}{0.880449in}}%
\pgfpathlineto{\pgfqpoint{4.378271in}{0.880449in}}%
\pgfpathlineto{\pgfqpoint{4.378271in}{0.883398in}}%
\pgfpathlineto{\pgfqpoint{4.382812in}{0.883398in}}%
\pgfpathlineto{\pgfqpoint{4.382812in}{0.880449in}}%
\pgfpathmoveto{\pgfqpoint{4.373730in}{0.883398in}}%
\pgfpathlineto{\pgfqpoint{4.373730in}{0.883398in}}%
\pgfpathlineto{\pgfqpoint{4.373730in}{0.886347in}}%
\pgfpathlineto{\pgfqpoint{4.378271in}{0.886347in}}%
\pgfpathlineto{\pgfqpoint{4.378271in}{0.883398in}}%
\pgfpathmoveto{\pgfqpoint{4.364647in}{0.889296in}}%
\pgfpathlineto{\pgfqpoint{4.364647in}{0.889296in}}%
\pgfpathlineto{\pgfqpoint{4.364647in}{0.892246in}}%
\pgfpathlineto{\pgfqpoint{4.369189in}{0.892246in}}%
\pgfpathlineto{\pgfqpoint{4.369189in}{0.889296in}}%
\pgfpathmoveto{\pgfqpoint{4.364647in}{0.892246in}}%
\pgfpathlineto{\pgfqpoint{4.364647in}{0.892246in}}%
\pgfpathlineto{\pgfqpoint{4.364647in}{0.895195in}}%
\pgfpathlineto{\pgfqpoint{4.369189in}{0.895195in}}%
\pgfpathlineto{\pgfqpoint{4.369189in}{0.892246in}}%
\pgfpathmoveto{\pgfqpoint{4.369189in}{0.889296in}}%
\pgfpathlineto{\pgfqpoint{4.369189in}{0.889296in}}%
\pgfpathlineto{\pgfqpoint{4.369189in}{0.892246in}}%
\pgfpathlineto{\pgfqpoint{4.373730in}{0.892246in}}%
\pgfpathlineto{\pgfqpoint{4.373730in}{0.889296in}}%
\pgfpathmoveto{\pgfqpoint{4.346483in}{0.904043in}}%
\pgfpathlineto{\pgfqpoint{4.346483in}{0.904043in}}%
\pgfpathlineto{\pgfqpoint{4.346483in}{0.906992in}}%
\pgfpathlineto{\pgfqpoint{4.351024in}{0.906992in}}%
\pgfpathlineto{\pgfqpoint{4.351024in}{0.904043in}}%
\pgfpathmoveto{\pgfqpoint{4.351024in}{0.901094in}}%
\pgfpathlineto{\pgfqpoint{4.351024in}{0.901094in}}%
\pgfpathlineto{\pgfqpoint{4.351024in}{0.904043in}}%
\pgfpathlineto{\pgfqpoint{4.355565in}{0.904043in}}%
\pgfpathlineto{\pgfqpoint{4.355565in}{0.901094in}}%
\pgfpathmoveto{\pgfqpoint{4.351024in}{0.904043in}}%
\pgfpathlineto{\pgfqpoint{4.351024in}{0.904043in}}%
\pgfpathlineto{\pgfqpoint{4.351024in}{0.906992in}}%
\pgfpathlineto{\pgfqpoint{4.355565in}{0.906992in}}%
\pgfpathlineto{\pgfqpoint{4.355565in}{0.904043in}}%
\pgfpathmoveto{\pgfqpoint{4.346483in}{0.906992in}}%
\pgfpathlineto{\pgfqpoint{4.346483in}{0.906992in}}%
\pgfpathlineto{\pgfqpoint{4.346483in}{0.909942in}}%
\pgfpathlineto{\pgfqpoint{4.351024in}{0.909942in}}%
\pgfpathlineto{\pgfqpoint{4.351024in}{0.906992in}}%
\pgfpathmoveto{\pgfqpoint{4.355565in}{0.901094in}}%
\pgfpathlineto{\pgfqpoint{4.355565in}{0.901094in}}%
\pgfpathlineto{\pgfqpoint{4.355565in}{0.904043in}}%
\pgfpathlineto{\pgfqpoint{4.360106in}{0.904043in}}%
\pgfpathlineto{\pgfqpoint{4.360106in}{0.901094in}}%
\pgfpathmoveto{\pgfqpoint{4.310155in}{0.933536in}}%
\pgfpathlineto{\pgfqpoint{4.310155in}{0.933536in}}%
\pgfpathlineto{\pgfqpoint{4.310155in}{0.936486in}}%
\pgfpathlineto{\pgfqpoint{4.314696in}{0.936486in}}%
\pgfpathlineto{\pgfqpoint{4.314696in}{0.933536in}}%
\pgfpathmoveto{\pgfqpoint{4.314696in}{0.930587in}}%
\pgfpathlineto{\pgfqpoint{4.314696in}{0.930587in}}%
\pgfpathlineto{\pgfqpoint{4.314696in}{0.933536in}}%
\pgfpathlineto{\pgfqpoint{4.319237in}{0.933536in}}%
\pgfpathlineto{\pgfqpoint{4.319237in}{0.930587in}}%
\pgfpathmoveto{\pgfqpoint{4.314696in}{0.933536in}}%
\pgfpathlineto{\pgfqpoint{4.314696in}{0.933536in}}%
\pgfpathlineto{\pgfqpoint{4.314696in}{0.936486in}}%
\pgfpathlineto{\pgfqpoint{4.319237in}{0.936486in}}%
\pgfpathlineto{\pgfqpoint{4.319237in}{0.933536in}}%
\pgfpathmoveto{\pgfqpoint{4.319237in}{0.927638in}}%
\pgfpathlineto{\pgfqpoint{4.319237in}{0.927638in}}%
\pgfpathlineto{\pgfqpoint{4.319237in}{0.930587in}}%
\pgfpathlineto{\pgfqpoint{4.323778in}{0.930587in}}%
\pgfpathlineto{\pgfqpoint{4.323778in}{0.927638in}}%
\pgfpathmoveto{\pgfqpoint{4.323778in}{0.924688in}}%
\pgfpathlineto{\pgfqpoint{4.323778in}{0.924688in}}%
\pgfpathlineto{\pgfqpoint{4.323778in}{0.927638in}}%
\pgfpathlineto{\pgfqpoint{4.328319in}{0.927638in}}%
\pgfpathlineto{\pgfqpoint{4.328319in}{0.924688in}}%
\pgfpathmoveto{\pgfqpoint{4.323778in}{0.927638in}}%
\pgfpathlineto{\pgfqpoint{4.323778in}{0.927638in}}%
\pgfpathlineto{\pgfqpoint{4.323778in}{0.930587in}}%
\pgfpathlineto{\pgfqpoint{4.328319in}{0.930587in}}%
\pgfpathlineto{\pgfqpoint{4.328319in}{0.927638in}}%
\pgfpathmoveto{\pgfqpoint{4.319237in}{0.930587in}}%
\pgfpathlineto{\pgfqpoint{4.319237in}{0.930587in}}%
\pgfpathlineto{\pgfqpoint{4.319237in}{0.933536in}}%
\pgfpathlineto{\pgfqpoint{4.323778in}{0.933536in}}%
\pgfpathlineto{\pgfqpoint{4.323778in}{0.930587in}}%
\pgfpathmoveto{\pgfqpoint{4.310155in}{0.936486in}}%
\pgfpathlineto{\pgfqpoint{4.310155in}{0.936486in}}%
\pgfpathlineto{\pgfqpoint{4.310155in}{0.939435in}}%
\pgfpathlineto{\pgfqpoint{4.314696in}{0.939435in}}%
\pgfpathlineto{\pgfqpoint{4.314696in}{0.936486in}}%
\pgfpathmoveto{\pgfqpoint{4.310155in}{0.939435in}}%
\pgfpathlineto{\pgfqpoint{4.310155in}{0.939435in}}%
\pgfpathlineto{\pgfqpoint{4.310155in}{0.942384in}}%
\pgfpathlineto{\pgfqpoint{4.314696in}{0.942384in}}%
\pgfpathlineto{\pgfqpoint{4.314696in}{0.939435in}}%
\pgfpathmoveto{\pgfqpoint{4.328319in}{0.924688in}}%
\pgfpathlineto{\pgfqpoint{4.328319in}{0.924688in}}%
\pgfpathlineto{\pgfqpoint{4.328319in}{0.927638in}}%
\pgfpathlineto{\pgfqpoint{4.332860in}{0.927638in}}%
\pgfpathlineto{\pgfqpoint{4.332860in}{0.924688in}}%
\pgfpathmoveto{\pgfqpoint{4.251121in}{0.986623in}}%
\pgfpathlineto{\pgfqpoint{4.251121in}{0.986623in}}%
\pgfpathlineto{\pgfqpoint{4.251121in}{0.989572in}}%
\pgfpathlineto{\pgfqpoint{4.255663in}{0.989572in}}%
\pgfpathlineto{\pgfqpoint{4.255663in}{0.986623in}}%
\pgfpathmoveto{\pgfqpoint{4.246580in}{0.989572in}}%
\pgfpathlineto{\pgfqpoint{4.246580in}{0.989572in}}%
\pgfpathlineto{\pgfqpoint{4.246580in}{0.992521in}}%
\pgfpathlineto{\pgfqpoint{4.251121in}{0.992521in}}%
\pgfpathlineto{\pgfqpoint{4.251121in}{0.989572in}}%
\pgfpathmoveto{\pgfqpoint{4.246580in}{0.992521in}}%
\pgfpathlineto{\pgfqpoint{4.246580in}{0.992521in}}%
\pgfpathlineto{\pgfqpoint{4.246580in}{0.995470in}}%
\pgfpathlineto{\pgfqpoint{4.251121in}{0.995470in}}%
\pgfpathlineto{\pgfqpoint{4.251121in}{0.992521in}}%
\pgfpathmoveto{\pgfqpoint{4.251121in}{0.989572in}}%
\pgfpathlineto{\pgfqpoint{4.251121in}{0.989572in}}%
\pgfpathlineto{\pgfqpoint{4.251121in}{0.992521in}}%
\pgfpathlineto{\pgfqpoint{4.255663in}{0.992521in}}%
\pgfpathlineto{\pgfqpoint{4.255663in}{0.989572in}}%
\pgfpathmoveto{\pgfqpoint{4.255663in}{0.980725in}}%
\pgfpathlineto{\pgfqpoint{4.255663in}{0.980725in}}%
\pgfpathlineto{\pgfqpoint{4.255663in}{0.983674in}}%
\pgfpathlineto{\pgfqpoint{4.260204in}{0.983674in}}%
\pgfpathlineto{\pgfqpoint{4.260204in}{0.980725in}}%
\pgfpathmoveto{\pgfqpoint{4.260204in}{0.977776in}}%
\pgfpathlineto{\pgfqpoint{4.260204in}{0.977776in}}%
\pgfpathlineto{\pgfqpoint{4.260204in}{0.980725in}}%
\pgfpathlineto{\pgfqpoint{4.264745in}{0.980725in}}%
\pgfpathlineto{\pgfqpoint{4.264745in}{0.977776in}}%
\pgfpathmoveto{\pgfqpoint{4.260204in}{0.980725in}}%
\pgfpathlineto{\pgfqpoint{4.260204in}{0.980725in}}%
\pgfpathlineto{\pgfqpoint{4.260204in}{0.983674in}}%
\pgfpathlineto{\pgfqpoint{4.264745in}{0.983674in}}%
\pgfpathlineto{\pgfqpoint{4.264745in}{0.980725in}}%
\pgfpathmoveto{\pgfqpoint{4.264745in}{0.974827in}}%
\pgfpathlineto{\pgfqpoint{4.264745in}{0.974827in}}%
\pgfpathlineto{\pgfqpoint{4.264745in}{0.977776in}}%
\pgfpathlineto{\pgfqpoint{4.269286in}{0.977776in}}%
\pgfpathlineto{\pgfqpoint{4.269286in}{0.974827in}}%
\pgfpathmoveto{\pgfqpoint{4.269286in}{0.971878in}}%
\pgfpathlineto{\pgfqpoint{4.269286in}{0.971878in}}%
\pgfpathlineto{\pgfqpoint{4.269286in}{0.974827in}}%
\pgfpathlineto{\pgfqpoint{4.273827in}{0.974827in}}%
\pgfpathlineto{\pgfqpoint{4.273827in}{0.971878in}}%
\pgfpathmoveto{\pgfqpoint{4.269286in}{0.974827in}}%
\pgfpathlineto{\pgfqpoint{4.269286in}{0.974827in}}%
\pgfpathlineto{\pgfqpoint{4.269286in}{0.977776in}}%
\pgfpathlineto{\pgfqpoint{4.273827in}{0.977776in}}%
\pgfpathlineto{\pgfqpoint{4.273827in}{0.974827in}}%
\pgfpathmoveto{\pgfqpoint{4.264745in}{0.977776in}}%
\pgfpathlineto{\pgfqpoint{4.264745in}{0.977776in}}%
\pgfpathlineto{\pgfqpoint{4.264745in}{0.980725in}}%
\pgfpathlineto{\pgfqpoint{4.269286in}{0.980725in}}%
\pgfpathlineto{\pgfqpoint{4.269286in}{0.977776in}}%
\pgfpathmoveto{\pgfqpoint{4.255663in}{0.983674in}}%
\pgfpathlineto{\pgfqpoint{4.255663in}{0.983674in}}%
\pgfpathlineto{\pgfqpoint{4.255663in}{0.986623in}}%
\pgfpathlineto{\pgfqpoint{4.260204in}{0.986623in}}%
\pgfpathlineto{\pgfqpoint{4.260204in}{0.983674in}}%
\pgfpathmoveto{\pgfqpoint{4.255663in}{0.986623in}}%
\pgfpathlineto{\pgfqpoint{4.255663in}{0.986623in}}%
\pgfpathlineto{\pgfqpoint{4.255663in}{0.989572in}}%
\pgfpathlineto{\pgfqpoint{4.260204in}{0.989572in}}%
\pgfpathlineto{\pgfqpoint{4.260204in}{0.986623in}}%
\pgfpathmoveto{\pgfqpoint{4.237498in}{0.998419in}}%
\pgfpathlineto{\pgfqpoint{4.237498in}{0.998419in}}%
\pgfpathlineto{\pgfqpoint{4.237498in}{1.001368in}}%
\pgfpathlineto{\pgfqpoint{4.242039in}{1.001368in}}%
\pgfpathlineto{\pgfqpoint{4.242039in}{0.998419in}}%
\pgfpathmoveto{\pgfqpoint{4.242039in}{0.995470in}}%
\pgfpathlineto{\pgfqpoint{4.242039in}{0.995470in}}%
\pgfpathlineto{\pgfqpoint{4.242039in}{0.998419in}}%
\pgfpathlineto{\pgfqpoint{4.246580in}{0.998419in}}%
\pgfpathlineto{\pgfqpoint{4.246580in}{0.995470in}}%
\pgfpathmoveto{\pgfqpoint{4.242039in}{0.998419in}}%
\pgfpathlineto{\pgfqpoint{4.242039in}{0.998419in}}%
\pgfpathlineto{\pgfqpoint{4.242039in}{1.001368in}}%
\pgfpathlineto{\pgfqpoint{4.246580in}{1.001368in}}%
\pgfpathlineto{\pgfqpoint{4.246580in}{0.998419in}}%
\pgfpathmoveto{\pgfqpoint{4.237498in}{1.001368in}}%
\pgfpathlineto{\pgfqpoint{4.237498in}{1.001368in}}%
\pgfpathlineto{\pgfqpoint{4.237498in}{1.004317in}}%
\pgfpathlineto{\pgfqpoint{4.242039in}{1.004317in}}%
\pgfpathlineto{\pgfqpoint{4.242039in}{1.001368in}}%
\pgfpathmoveto{\pgfqpoint{4.246580in}{0.995470in}}%
\pgfpathlineto{\pgfqpoint{4.246580in}{0.995470in}}%
\pgfpathlineto{\pgfqpoint{4.246580in}{0.998419in}}%
\pgfpathlineto{\pgfqpoint{4.251121in}{0.998419in}}%
\pgfpathlineto{\pgfqpoint{4.251121in}{0.995470in}}%
\pgfpathmoveto{\pgfqpoint{4.487259in}{0.780175in}}%
\pgfpathlineto{\pgfqpoint{4.487259in}{0.780175in}}%
\pgfpathlineto{\pgfqpoint{4.487259in}{0.783124in}}%
\pgfpathlineto{\pgfqpoint{4.491800in}{0.783124in}}%
\pgfpathlineto{\pgfqpoint{4.491800in}{0.780175in}}%
\pgfpathmoveto{\pgfqpoint{4.514506in}{0.756582in}}%
\pgfpathlineto{\pgfqpoint{4.514506in}{0.756582in}}%
\pgfpathlineto{\pgfqpoint{4.514506in}{0.759531in}}%
\pgfpathlineto{\pgfqpoint{4.519047in}{0.759531in}}%
\pgfpathlineto{\pgfqpoint{4.519047in}{0.756582in}}%
\pgfpathmoveto{\pgfqpoint{4.523588in}{0.750684in}}%
\pgfpathlineto{\pgfqpoint{4.523588in}{0.750684in}}%
\pgfpathlineto{\pgfqpoint{4.523588in}{0.753633in}}%
\pgfpathlineto{\pgfqpoint{4.528129in}{0.753633in}}%
\pgfpathlineto{\pgfqpoint{4.528129in}{0.750684in}}%
\pgfpathmoveto{\pgfqpoint{4.519047in}{0.753633in}}%
\pgfpathlineto{\pgfqpoint{4.519047in}{0.753633in}}%
\pgfpathlineto{\pgfqpoint{4.519047in}{0.756582in}}%
\pgfpathlineto{\pgfqpoint{4.523588in}{0.756582in}}%
\pgfpathlineto{\pgfqpoint{4.523588in}{0.753633in}}%
\pgfpathmoveto{\pgfqpoint{4.519047in}{0.756582in}}%
\pgfpathlineto{\pgfqpoint{4.519047in}{0.756582in}}%
\pgfpathlineto{\pgfqpoint{4.519047in}{0.759531in}}%
\pgfpathlineto{\pgfqpoint{4.523588in}{0.759531in}}%
\pgfpathlineto{\pgfqpoint{4.523588in}{0.756582in}}%
\pgfpathmoveto{\pgfqpoint{4.523588in}{0.753633in}}%
\pgfpathlineto{\pgfqpoint{4.523588in}{0.753633in}}%
\pgfpathlineto{\pgfqpoint{4.523588in}{0.756582in}}%
\pgfpathlineto{\pgfqpoint{4.528129in}{0.756582in}}%
\pgfpathlineto{\pgfqpoint{4.528129in}{0.753633in}}%
\pgfpathmoveto{\pgfqpoint{4.500882in}{0.768379in}}%
\pgfpathlineto{\pgfqpoint{4.500882in}{0.768379in}}%
\pgfpathlineto{\pgfqpoint{4.500882in}{0.771328in}}%
\pgfpathlineto{\pgfqpoint{4.505423in}{0.771328in}}%
\pgfpathlineto{\pgfqpoint{4.505423in}{0.768379in}}%
\pgfpathmoveto{\pgfqpoint{4.505423in}{0.765429in}}%
\pgfpathlineto{\pgfqpoint{4.505423in}{0.765429in}}%
\pgfpathlineto{\pgfqpoint{4.505423in}{0.768379in}}%
\pgfpathlineto{\pgfqpoint{4.509965in}{0.768379in}}%
\pgfpathlineto{\pgfqpoint{4.509965in}{0.765429in}}%
\pgfpathmoveto{\pgfqpoint{4.505423in}{0.768379in}}%
\pgfpathlineto{\pgfqpoint{4.505423in}{0.768379in}}%
\pgfpathlineto{\pgfqpoint{4.505423in}{0.771328in}}%
\pgfpathlineto{\pgfqpoint{4.509965in}{0.771328in}}%
\pgfpathlineto{\pgfqpoint{4.509965in}{0.768379in}}%
\pgfpathmoveto{\pgfqpoint{4.496341in}{0.774277in}}%
\pgfpathlineto{\pgfqpoint{4.496341in}{0.774277in}}%
\pgfpathlineto{\pgfqpoint{4.496341in}{0.777226in}}%
\pgfpathlineto{\pgfqpoint{4.500882in}{0.777226in}}%
\pgfpathlineto{\pgfqpoint{4.500882in}{0.774277in}}%
\pgfpathmoveto{\pgfqpoint{4.491800in}{0.777226in}}%
\pgfpathlineto{\pgfqpoint{4.491800in}{0.777226in}}%
\pgfpathlineto{\pgfqpoint{4.491800in}{0.780175in}}%
\pgfpathlineto{\pgfqpoint{4.496341in}{0.780175in}}%
\pgfpathlineto{\pgfqpoint{4.496341in}{0.777226in}}%
\pgfpathmoveto{\pgfqpoint{4.491800in}{0.780175in}}%
\pgfpathlineto{\pgfqpoint{4.491800in}{0.780175in}}%
\pgfpathlineto{\pgfqpoint{4.491800in}{0.783124in}}%
\pgfpathlineto{\pgfqpoint{4.496341in}{0.783124in}}%
\pgfpathlineto{\pgfqpoint{4.496341in}{0.780175in}}%
\pgfpathmoveto{\pgfqpoint{4.496341in}{0.777226in}}%
\pgfpathlineto{\pgfqpoint{4.496341in}{0.777226in}}%
\pgfpathlineto{\pgfqpoint{4.496341in}{0.780175in}}%
\pgfpathlineto{\pgfqpoint{4.500882in}{0.780175in}}%
\pgfpathlineto{\pgfqpoint{4.500882in}{0.777226in}}%
\pgfpathmoveto{\pgfqpoint{4.500882in}{0.771328in}}%
\pgfpathlineto{\pgfqpoint{4.500882in}{0.771328in}}%
\pgfpathlineto{\pgfqpoint{4.500882in}{0.774277in}}%
\pgfpathlineto{\pgfqpoint{4.505423in}{0.774277in}}%
\pgfpathlineto{\pgfqpoint{4.505423in}{0.771328in}}%
\pgfpathmoveto{\pgfqpoint{4.500882in}{0.774277in}}%
\pgfpathlineto{\pgfqpoint{4.500882in}{0.774277in}}%
\pgfpathlineto{\pgfqpoint{4.500882in}{0.777226in}}%
\pgfpathlineto{\pgfqpoint{4.505423in}{0.777226in}}%
\pgfpathlineto{\pgfqpoint{4.505423in}{0.774277in}}%
\pgfpathmoveto{\pgfqpoint{4.509965in}{0.762480in}}%
\pgfpathlineto{\pgfqpoint{4.509965in}{0.762480in}}%
\pgfpathlineto{\pgfqpoint{4.509965in}{0.765429in}}%
\pgfpathlineto{\pgfqpoint{4.514506in}{0.765429in}}%
\pgfpathlineto{\pgfqpoint{4.514506in}{0.762480in}}%
\pgfpathmoveto{\pgfqpoint{4.514506in}{0.759531in}}%
\pgfpathlineto{\pgfqpoint{4.514506in}{0.759531in}}%
\pgfpathlineto{\pgfqpoint{4.514506in}{0.762480in}}%
\pgfpathlineto{\pgfqpoint{4.519047in}{0.762480in}}%
\pgfpathlineto{\pgfqpoint{4.519047in}{0.759531in}}%
\pgfpathmoveto{\pgfqpoint{4.514506in}{0.762480in}}%
\pgfpathlineto{\pgfqpoint{4.514506in}{0.762480in}}%
\pgfpathlineto{\pgfqpoint{4.514506in}{0.765429in}}%
\pgfpathlineto{\pgfqpoint{4.519047in}{0.765429in}}%
\pgfpathlineto{\pgfqpoint{4.519047in}{0.762480in}}%
\pgfpathmoveto{\pgfqpoint{4.509965in}{0.765429in}}%
\pgfpathlineto{\pgfqpoint{4.509965in}{0.765429in}}%
\pgfpathlineto{\pgfqpoint{4.509965in}{0.768379in}}%
\pgfpathlineto{\pgfqpoint{4.514506in}{0.768379in}}%
\pgfpathlineto{\pgfqpoint{4.514506in}{0.765429in}}%
\pgfpathmoveto{\pgfqpoint{4.432765in}{0.827363in}}%
\pgfpathlineto{\pgfqpoint{4.432765in}{0.827363in}}%
\pgfpathlineto{\pgfqpoint{4.432765in}{0.830312in}}%
\pgfpathlineto{\pgfqpoint{4.437306in}{0.830312in}}%
\pgfpathlineto{\pgfqpoint{4.437306in}{0.827363in}}%
\pgfpathmoveto{\pgfqpoint{4.446388in}{0.815566in}}%
\pgfpathlineto{\pgfqpoint{4.446388in}{0.815566in}}%
\pgfpathlineto{\pgfqpoint{4.446388in}{0.818515in}}%
\pgfpathlineto{\pgfqpoint{4.450929in}{0.818515in}}%
\pgfpathlineto{\pgfqpoint{4.450929in}{0.815566in}}%
\pgfpathmoveto{\pgfqpoint{4.450929in}{0.812616in}}%
\pgfpathlineto{\pgfqpoint{4.450929in}{0.812616in}}%
\pgfpathlineto{\pgfqpoint{4.450929in}{0.815566in}}%
\pgfpathlineto{\pgfqpoint{4.455470in}{0.815566in}}%
\pgfpathlineto{\pgfqpoint{4.455470in}{0.812616in}}%
\pgfpathmoveto{\pgfqpoint{4.450929in}{0.815566in}}%
\pgfpathlineto{\pgfqpoint{4.450929in}{0.815566in}}%
\pgfpathlineto{\pgfqpoint{4.450929in}{0.818515in}}%
\pgfpathlineto{\pgfqpoint{4.455470in}{0.818515in}}%
\pgfpathlineto{\pgfqpoint{4.455470in}{0.815566in}}%
\pgfpathmoveto{\pgfqpoint{4.441847in}{0.821464in}}%
\pgfpathlineto{\pgfqpoint{4.441847in}{0.821464in}}%
\pgfpathlineto{\pgfqpoint{4.441847in}{0.824413in}}%
\pgfpathlineto{\pgfqpoint{4.446388in}{0.824413in}}%
\pgfpathlineto{\pgfqpoint{4.446388in}{0.821464in}}%
\pgfpathmoveto{\pgfqpoint{4.437306in}{0.824413in}}%
\pgfpathlineto{\pgfqpoint{4.437306in}{0.824413in}}%
\pgfpathlineto{\pgfqpoint{4.437306in}{0.827363in}}%
\pgfpathlineto{\pgfqpoint{4.441847in}{0.827363in}}%
\pgfpathlineto{\pgfqpoint{4.441847in}{0.824413in}}%
\pgfpathmoveto{\pgfqpoint{4.437306in}{0.827363in}}%
\pgfpathlineto{\pgfqpoint{4.437306in}{0.827363in}}%
\pgfpathlineto{\pgfqpoint{4.437306in}{0.830312in}}%
\pgfpathlineto{\pgfqpoint{4.441847in}{0.830312in}}%
\pgfpathlineto{\pgfqpoint{4.441847in}{0.827363in}}%
\pgfpathmoveto{\pgfqpoint{4.441847in}{0.824413in}}%
\pgfpathlineto{\pgfqpoint{4.441847in}{0.824413in}}%
\pgfpathlineto{\pgfqpoint{4.441847in}{0.827363in}}%
\pgfpathlineto{\pgfqpoint{4.446388in}{0.827363in}}%
\pgfpathlineto{\pgfqpoint{4.446388in}{0.824413in}}%
\pgfpathmoveto{\pgfqpoint{4.446388in}{0.818515in}}%
\pgfpathlineto{\pgfqpoint{4.446388in}{0.818515in}}%
\pgfpathlineto{\pgfqpoint{4.446388in}{0.821464in}}%
\pgfpathlineto{\pgfqpoint{4.450929in}{0.821464in}}%
\pgfpathlineto{\pgfqpoint{4.450929in}{0.818515in}}%
\pgfpathmoveto{\pgfqpoint{4.446388in}{0.821464in}}%
\pgfpathlineto{\pgfqpoint{4.446388in}{0.821464in}}%
\pgfpathlineto{\pgfqpoint{4.446388in}{0.824413in}}%
\pgfpathlineto{\pgfqpoint{4.450929in}{0.824413in}}%
\pgfpathlineto{\pgfqpoint{4.450929in}{0.821464in}}%
\pgfpathmoveto{\pgfqpoint{4.414600in}{0.845058in}}%
\pgfpathlineto{\pgfqpoint{4.414600in}{0.845058in}}%
\pgfpathlineto{\pgfqpoint{4.414600in}{0.848007in}}%
\pgfpathlineto{\pgfqpoint{4.419141in}{0.848007in}}%
\pgfpathlineto{\pgfqpoint{4.419141in}{0.845058in}}%
\pgfpathmoveto{\pgfqpoint{4.410059in}{0.848007in}}%
\pgfpathlineto{\pgfqpoint{4.410059in}{0.848007in}}%
\pgfpathlineto{\pgfqpoint{4.410059in}{0.850956in}}%
\pgfpathlineto{\pgfqpoint{4.414600in}{0.850956in}}%
\pgfpathlineto{\pgfqpoint{4.414600in}{0.848007in}}%
\pgfpathmoveto{\pgfqpoint{4.410059in}{0.850956in}}%
\pgfpathlineto{\pgfqpoint{4.410059in}{0.850956in}}%
\pgfpathlineto{\pgfqpoint{4.410059in}{0.853905in}}%
\pgfpathlineto{\pgfqpoint{4.414600in}{0.853905in}}%
\pgfpathlineto{\pgfqpoint{4.414600in}{0.850956in}}%
\pgfpathmoveto{\pgfqpoint{4.414600in}{0.848007in}}%
\pgfpathlineto{\pgfqpoint{4.414600in}{0.848007in}}%
\pgfpathlineto{\pgfqpoint{4.414600in}{0.850956in}}%
\pgfpathlineto{\pgfqpoint{4.419141in}{0.850956in}}%
\pgfpathlineto{\pgfqpoint{4.419141in}{0.848007in}}%
\pgfpathmoveto{\pgfqpoint{4.396435in}{0.859804in}}%
\pgfpathlineto{\pgfqpoint{4.396435in}{0.859804in}}%
\pgfpathlineto{\pgfqpoint{4.396435in}{0.862753in}}%
\pgfpathlineto{\pgfqpoint{4.400976in}{0.862753in}}%
\pgfpathlineto{\pgfqpoint{4.400976in}{0.859804in}}%
\pgfpathmoveto{\pgfqpoint{4.396435in}{0.862753in}}%
\pgfpathlineto{\pgfqpoint{4.396435in}{0.862753in}}%
\pgfpathlineto{\pgfqpoint{4.396435in}{0.865702in}}%
\pgfpathlineto{\pgfqpoint{4.400976in}{0.865702in}}%
\pgfpathlineto{\pgfqpoint{4.400976in}{0.862753in}}%
\pgfpathmoveto{\pgfqpoint{4.387353in}{0.868652in}}%
\pgfpathlineto{\pgfqpoint{4.387353in}{0.868652in}}%
\pgfpathlineto{\pgfqpoint{4.387353in}{0.871601in}}%
\pgfpathlineto{\pgfqpoint{4.391894in}{0.871601in}}%
\pgfpathlineto{\pgfqpoint{4.391894in}{0.868652in}}%
\pgfpathmoveto{\pgfqpoint{4.382812in}{0.871601in}}%
\pgfpathlineto{\pgfqpoint{4.382812in}{0.871601in}}%
\pgfpathlineto{\pgfqpoint{4.382812in}{0.874550in}}%
\pgfpathlineto{\pgfqpoint{4.387353in}{0.874550in}}%
\pgfpathlineto{\pgfqpoint{4.387353in}{0.871601in}}%
\pgfpathmoveto{\pgfqpoint{4.382812in}{0.874550in}}%
\pgfpathlineto{\pgfqpoint{4.382812in}{0.874550in}}%
\pgfpathlineto{\pgfqpoint{4.382812in}{0.877499in}}%
\pgfpathlineto{\pgfqpoint{4.387353in}{0.877499in}}%
\pgfpathlineto{\pgfqpoint{4.387353in}{0.874550in}}%
\pgfpathmoveto{\pgfqpoint{4.387353in}{0.871601in}}%
\pgfpathlineto{\pgfqpoint{4.387353in}{0.871601in}}%
\pgfpathlineto{\pgfqpoint{4.387353in}{0.874550in}}%
\pgfpathlineto{\pgfqpoint{4.391894in}{0.874550in}}%
\pgfpathlineto{\pgfqpoint{4.391894in}{0.871601in}}%
\pgfpathmoveto{\pgfqpoint{4.391894in}{0.865702in}}%
\pgfpathlineto{\pgfqpoint{4.391894in}{0.865702in}}%
\pgfpathlineto{\pgfqpoint{4.391894in}{0.868652in}}%
\pgfpathlineto{\pgfqpoint{4.396435in}{0.868652in}}%
\pgfpathlineto{\pgfqpoint{4.396435in}{0.865702in}}%
\pgfpathmoveto{\pgfqpoint{4.391894in}{0.868652in}}%
\pgfpathlineto{\pgfqpoint{4.391894in}{0.868652in}}%
\pgfpathlineto{\pgfqpoint{4.391894in}{0.871601in}}%
\pgfpathlineto{\pgfqpoint{4.396435in}{0.871601in}}%
\pgfpathlineto{\pgfqpoint{4.396435in}{0.868652in}}%
\pgfpathmoveto{\pgfqpoint{4.396435in}{0.865702in}}%
\pgfpathlineto{\pgfqpoint{4.396435in}{0.865702in}}%
\pgfpathlineto{\pgfqpoint{4.396435in}{0.868652in}}%
\pgfpathlineto{\pgfqpoint{4.400976in}{0.868652in}}%
\pgfpathlineto{\pgfqpoint{4.400976in}{0.865702in}}%
\pgfpathmoveto{\pgfqpoint{4.400976in}{0.856855in}}%
\pgfpathlineto{\pgfqpoint{4.400976in}{0.856855in}}%
\pgfpathlineto{\pgfqpoint{4.400976in}{0.859804in}}%
\pgfpathlineto{\pgfqpoint{4.405518in}{0.859804in}}%
\pgfpathlineto{\pgfqpoint{4.405518in}{0.856855in}}%
\pgfpathmoveto{\pgfqpoint{4.405518in}{0.853905in}}%
\pgfpathlineto{\pgfqpoint{4.405518in}{0.853905in}}%
\pgfpathlineto{\pgfqpoint{4.405518in}{0.856855in}}%
\pgfpathlineto{\pgfqpoint{4.410059in}{0.856855in}}%
\pgfpathlineto{\pgfqpoint{4.410059in}{0.853905in}}%
\pgfpathmoveto{\pgfqpoint{4.405518in}{0.856855in}}%
\pgfpathlineto{\pgfqpoint{4.405518in}{0.856855in}}%
\pgfpathlineto{\pgfqpoint{4.405518in}{0.859804in}}%
\pgfpathlineto{\pgfqpoint{4.410059in}{0.859804in}}%
\pgfpathlineto{\pgfqpoint{4.410059in}{0.856855in}}%
\pgfpathmoveto{\pgfqpoint{4.400976in}{0.859804in}}%
\pgfpathlineto{\pgfqpoint{4.400976in}{0.859804in}}%
\pgfpathlineto{\pgfqpoint{4.400976in}{0.862753in}}%
\pgfpathlineto{\pgfqpoint{4.405518in}{0.862753in}}%
\pgfpathlineto{\pgfqpoint{4.405518in}{0.859804in}}%
\pgfpathmoveto{\pgfqpoint{4.410059in}{0.853905in}}%
\pgfpathlineto{\pgfqpoint{4.410059in}{0.853905in}}%
\pgfpathlineto{\pgfqpoint{4.410059in}{0.856855in}}%
\pgfpathlineto{\pgfqpoint{4.414600in}{0.856855in}}%
\pgfpathlineto{\pgfqpoint{4.414600in}{0.853905in}}%
\pgfpathmoveto{\pgfqpoint{4.423682in}{0.836210in}}%
\pgfpathlineto{\pgfqpoint{4.423682in}{0.836210in}}%
\pgfpathlineto{\pgfqpoint{4.423682in}{0.839159in}}%
\pgfpathlineto{\pgfqpoint{4.428223in}{0.839159in}}%
\pgfpathlineto{\pgfqpoint{4.428223in}{0.836210in}}%
\pgfpathmoveto{\pgfqpoint{4.423682in}{0.839159in}}%
\pgfpathlineto{\pgfqpoint{4.423682in}{0.839159in}}%
\pgfpathlineto{\pgfqpoint{4.423682in}{0.842109in}}%
\pgfpathlineto{\pgfqpoint{4.428223in}{0.842109in}}%
\pgfpathlineto{\pgfqpoint{4.428223in}{0.839159in}}%
\pgfpathmoveto{\pgfqpoint{4.428223in}{0.833261in}}%
\pgfpathlineto{\pgfqpoint{4.428223in}{0.833261in}}%
\pgfpathlineto{\pgfqpoint{4.428223in}{0.836210in}}%
\pgfpathlineto{\pgfqpoint{4.432765in}{0.836210in}}%
\pgfpathlineto{\pgfqpoint{4.432765in}{0.833261in}}%
\pgfpathmoveto{\pgfqpoint{4.432765in}{0.830312in}}%
\pgfpathlineto{\pgfqpoint{4.432765in}{0.830312in}}%
\pgfpathlineto{\pgfqpoint{4.432765in}{0.833261in}}%
\pgfpathlineto{\pgfqpoint{4.437306in}{0.833261in}}%
\pgfpathlineto{\pgfqpoint{4.437306in}{0.830312in}}%
\pgfpathmoveto{\pgfqpoint{4.432765in}{0.833261in}}%
\pgfpathlineto{\pgfqpoint{4.432765in}{0.833261in}}%
\pgfpathlineto{\pgfqpoint{4.432765in}{0.836210in}}%
\pgfpathlineto{\pgfqpoint{4.437306in}{0.836210in}}%
\pgfpathlineto{\pgfqpoint{4.437306in}{0.833261in}}%
\pgfpathmoveto{\pgfqpoint{4.428223in}{0.836210in}}%
\pgfpathlineto{\pgfqpoint{4.428223in}{0.836210in}}%
\pgfpathlineto{\pgfqpoint{4.428223in}{0.839159in}}%
\pgfpathlineto{\pgfqpoint{4.432765in}{0.839159in}}%
\pgfpathlineto{\pgfqpoint{4.432765in}{0.836210in}}%
\pgfpathmoveto{\pgfqpoint{4.419141in}{0.842109in}}%
\pgfpathlineto{\pgfqpoint{4.419141in}{0.842109in}}%
\pgfpathlineto{\pgfqpoint{4.419141in}{0.845058in}}%
\pgfpathlineto{\pgfqpoint{4.423682in}{0.845058in}}%
\pgfpathlineto{\pgfqpoint{4.423682in}{0.842109in}}%
\pgfpathmoveto{\pgfqpoint{4.419141in}{0.845058in}}%
\pgfpathlineto{\pgfqpoint{4.419141in}{0.845058in}}%
\pgfpathlineto{\pgfqpoint{4.419141in}{0.848007in}}%
\pgfpathlineto{\pgfqpoint{4.423682in}{0.848007in}}%
\pgfpathlineto{\pgfqpoint{4.423682in}{0.845058in}}%
\pgfpathmoveto{\pgfqpoint{4.423682in}{0.842109in}}%
\pgfpathlineto{\pgfqpoint{4.423682in}{0.842109in}}%
\pgfpathlineto{\pgfqpoint{4.423682in}{0.845058in}}%
\pgfpathlineto{\pgfqpoint{4.428223in}{0.845058in}}%
\pgfpathlineto{\pgfqpoint{4.428223in}{0.842109in}}%
\pgfpathmoveto{\pgfqpoint{4.460012in}{0.803769in}}%
\pgfpathlineto{\pgfqpoint{4.460012in}{0.803769in}}%
\pgfpathlineto{\pgfqpoint{4.460012in}{0.806718in}}%
\pgfpathlineto{\pgfqpoint{4.464553in}{0.806718in}}%
\pgfpathlineto{\pgfqpoint{4.464553in}{0.803769in}}%
\pgfpathmoveto{\pgfqpoint{4.469094in}{0.797870in}}%
\pgfpathlineto{\pgfqpoint{4.469094in}{0.797870in}}%
\pgfpathlineto{\pgfqpoint{4.469094in}{0.800820in}}%
\pgfpathlineto{\pgfqpoint{4.473635in}{0.800820in}}%
\pgfpathlineto{\pgfqpoint{4.473635in}{0.797870in}}%
\pgfpathmoveto{\pgfqpoint{4.464553in}{0.800820in}}%
\pgfpathlineto{\pgfqpoint{4.464553in}{0.800820in}}%
\pgfpathlineto{\pgfqpoint{4.464553in}{0.803769in}}%
\pgfpathlineto{\pgfqpoint{4.469094in}{0.803769in}}%
\pgfpathlineto{\pgfqpoint{4.469094in}{0.800820in}}%
\pgfpathmoveto{\pgfqpoint{4.464553in}{0.803769in}}%
\pgfpathlineto{\pgfqpoint{4.464553in}{0.803769in}}%
\pgfpathlineto{\pgfqpoint{4.464553in}{0.806718in}}%
\pgfpathlineto{\pgfqpoint{4.469094in}{0.806718in}}%
\pgfpathlineto{\pgfqpoint{4.469094in}{0.803769in}}%
\pgfpathmoveto{\pgfqpoint{4.469094in}{0.800820in}}%
\pgfpathlineto{\pgfqpoint{4.469094in}{0.800820in}}%
\pgfpathlineto{\pgfqpoint{4.469094in}{0.803769in}}%
\pgfpathlineto{\pgfqpoint{4.473635in}{0.803769in}}%
\pgfpathlineto{\pgfqpoint{4.473635in}{0.800820in}}%
\pgfpathmoveto{\pgfqpoint{4.473635in}{0.791972in}}%
\pgfpathlineto{\pgfqpoint{4.473635in}{0.791972in}}%
\pgfpathlineto{\pgfqpoint{4.473635in}{0.794921in}}%
\pgfpathlineto{\pgfqpoint{4.478176in}{0.794921in}}%
\pgfpathlineto{\pgfqpoint{4.478176in}{0.791972in}}%
\pgfpathmoveto{\pgfqpoint{4.478176in}{0.789023in}}%
\pgfpathlineto{\pgfqpoint{4.478176in}{0.789023in}}%
\pgfpathlineto{\pgfqpoint{4.478176in}{0.791972in}}%
\pgfpathlineto{\pgfqpoint{4.482718in}{0.791972in}}%
\pgfpathlineto{\pgfqpoint{4.482718in}{0.789023in}}%
\pgfpathmoveto{\pgfqpoint{4.478176in}{0.791972in}}%
\pgfpathlineto{\pgfqpoint{4.478176in}{0.791972in}}%
\pgfpathlineto{\pgfqpoint{4.478176in}{0.794921in}}%
\pgfpathlineto{\pgfqpoint{4.482718in}{0.794921in}}%
\pgfpathlineto{\pgfqpoint{4.482718in}{0.791972in}}%
\pgfpathmoveto{\pgfqpoint{4.482718in}{0.786074in}}%
\pgfpathlineto{\pgfqpoint{4.482718in}{0.786074in}}%
\pgfpathlineto{\pgfqpoint{4.482718in}{0.789023in}}%
\pgfpathlineto{\pgfqpoint{4.487259in}{0.789023in}}%
\pgfpathlineto{\pgfqpoint{4.487259in}{0.786074in}}%
\pgfpathmoveto{\pgfqpoint{4.487259in}{0.783124in}}%
\pgfpathlineto{\pgfqpoint{4.487259in}{0.783124in}}%
\pgfpathlineto{\pgfqpoint{4.487259in}{0.786074in}}%
\pgfpathlineto{\pgfqpoint{4.491800in}{0.786074in}}%
\pgfpathlineto{\pgfqpoint{4.491800in}{0.783124in}}%
\pgfpathmoveto{\pgfqpoint{4.487259in}{0.786074in}}%
\pgfpathlineto{\pgfqpoint{4.487259in}{0.786074in}}%
\pgfpathlineto{\pgfqpoint{4.487259in}{0.789023in}}%
\pgfpathlineto{\pgfqpoint{4.491800in}{0.789023in}}%
\pgfpathlineto{\pgfqpoint{4.491800in}{0.786074in}}%
\pgfpathmoveto{\pgfqpoint{4.482718in}{0.789023in}}%
\pgfpathlineto{\pgfqpoint{4.482718in}{0.789023in}}%
\pgfpathlineto{\pgfqpoint{4.482718in}{0.791972in}}%
\pgfpathlineto{\pgfqpoint{4.487259in}{0.791972in}}%
\pgfpathlineto{\pgfqpoint{4.487259in}{0.789023in}}%
\pgfpathmoveto{\pgfqpoint{4.473635in}{0.794921in}}%
\pgfpathlineto{\pgfqpoint{4.473635in}{0.794921in}}%
\pgfpathlineto{\pgfqpoint{4.473635in}{0.797870in}}%
\pgfpathlineto{\pgfqpoint{4.478176in}{0.797870in}}%
\pgfpathlineto{\pgfqpoint{4.478176in}{0.794921in}}%
\pgfpathmoveto{\pgfqpoint{4.473635in}{0.797870in}}%
\pgfpathlineto{\pgfqpoint{4.473635in}{0.797870in}}%
\pgfpathlineto{\pgfqpoint{4.473635in}{0.800820in}}%
\pgfpathlineto{\pgfqpoint{4.478176in}{0.800820in}}%
\pgfpathlineto{\pgfqpoint{4.478176in}{0.797870in}}%
\pgfpathmoveto{\pgfqpoint{4.455470in}{0.809667in}}%
\pgfpathlineto{\pgfqpoint{4.455470in}{0.809667in}}%
\pgfpathlineto{\pgfqpoint{4.455470in}{0.812616in}}%
\pgfpathlineto{\pgfqpoint{4.460012in}{0.812616in}}%
\pgfpathlineto{\pgfqpoint{4.460012in}{0.809667in}}%
\pgfpathmoveto{\pgfqpoint{4.460012in}{0.806718in}}%
\pgfpathlineto{\pgfqpoint{4.460012in}{0.806718in}}%
\pgfpathlineto{\pgfqpoint{4.460012in}{0.809667in}}%
\pgfpathlineto{\pgfqpoint{4.464553in}{0.809667in}}%
\pgfpathlineto{\pgfqpoint{4.464553in}{0.806718in}}%
\pgfpathmoveto{\pgfqpoint{4.460012in}{0.809667in}}%
\pgfpathlineto{\pgfqpoint{4.460012in}{0.809667in}}%
\pgfpathlineto{\pgfqpoint{4.460012in}{0.812616in}}%
\pgfpathlineto{\pgfqpoint{4.464553in}{0.812616in}}%
\pgfpathlineto{\pgfqpoint{4.464553in}{0.809667in}}%
\pgfpathmoveto{\pgfqpoint{4.455470in}{0.812616in}}%
\pgfpathlineto{\pgfqpoint{4.455470in}{0.812616in}}%
\pgfpathlineto{\pgfqpoint{4.455470in}{0.815566in}}%
\pgfpathlineto{\pgfqpoint{4.460012in}{0.815566in}}%
\pgfpathlineto{\pgfqpoint{4.460012in}{0.812616in}}%
\pgfpathmoveto{\pgfqpoint{4.382812in}{0.877499in}}%
\pgfpathlineto{\pgfqpoint{4.382812in}{0.877499in}}%
\pgfpathlineto{\pgfqpoint{4.382812in}{0.880449in}}%
\pgfpathlineto{\pgfqpoint{4.387353in}{0.880449in}}%
\pgfpathlineto{\pgfqpoint{4.387353in}{0.877499in}}%
\pgfpathmoveto{\pgfqpoint{4.596244in}{0.685802in}}%
\pgfpathlineto{\pgfqpoint{4.596244in}{0.685802in}}%
\pgfpathlineto{\pgfqpoint{4.596244in}{0.688752in}}%
\pgfpathlineto{\pgfqpoint{4.600785in}{0.688752in}}%
\pgfpathlineto{\pgfqpoint{4.600785in}{0.685802in}}%
\pgfpathmoveto{\pgfqpoint{4.650736in}{0.638614in}}%
\pgfpathlineto{\pgfqpoint{4.650736in}{0.638614in}}%
\pgfpathlineto{\pgfqpoint{4.650736in}{0.641564in}}%
\pgfpathlineto{\pgfqpoint{4.655277in}{0.641564in}}%
\pgfpathlineto{\pgfqpoint{4.655277in}{0.638614in}}%
\pgfpathmoveto{\pgfqpoint{4.664359in}{0.626817in}}%
\pgfpathlineto{\pgfqpoint{4.664359in}{0.626817in}}%
\pgfpathlineto{\pgfqpoint{4.664359in}{0.629766in}}%
\pgfpathlineto{\pgfqpoint{4.668900in}{0.629766in}}%
\pgfpathlineto{\pgfqpoint{4.668900in}{0.626817in}}%
\pgfpathmoveto{\pgfqpoint{4.668900in}{0.623868in}}%
\pgfpathlineto{\pgfqpoint{4.668900in}{0.623868in}}%
\pgfpathlineto{\pgfqpoint{4.668900in}{0.626817in}}%
\pgfpathlineto{\pgfqpoint{4.673441in}{0.626817in}}%
\pgfpathlineto{\pgfqpoint{4.673441in}{0.623868in}}%
\pgfpathmoveto{\pgfqpoint{4.668900in}{0.626817in}}%
\pgfpathlineto{\pgfqpoint{4.668900in}{0.626817in}}%
\pgfpathlineto{\pgfqpoint{4.668900in}{0.629766in}}%
\pgfpathlineto{\pgfqpoint{4.673441in}{0.629766in}}%
\pgfpathlineto{\pgfqpoint{4.673441in}{0.626817in}}%
\pgfpathmoveto{\pgfqpoint{4.659818in}{0.632716in}}%
\pgfpathlineto{\pgfqpoint{4.659818in}{0.632716in}}%
\pgfpathlineto{\pgfqpoint{4.659818in}{0.635665in}}%
\pgfpathlineto{\pgfqpoint{4.664359in}{0.635665in}}%
\pgfpathlineto{\pgfqpoint{4.664359in}{0.632716in}}%
\pgfpathmoveto{\pgfqpoint{4.655277in}{0.635665in}}%
\pgfpathlineto{\pgfqpoint{4.655277in}{0.635665in}}%
\pgfpathlineto{\pgfqpoint{4.655277in}{0.638614in}}%
\pgfpathlineto{\pgfqpoint{4.659818in}{0.638614in}}%
\pgfpathlineto{\pgfqpoint{4.659818in}{0.635665in}}%
\pgfpathmoveto{\pgfqpoint{4.655277in}{0.638614in}}%
\pgfpathlineto{\pgfqpoint{4.655277in}{0.638614in}}%
\pgfpathlineto{\pgfqpoint{4.655277in}{0.641564in}}%
\pgfpathlineto{\pgfqpoint{4.659818in}{0.641564in}}%
\pgfpathlineto{\pgfqpoint{4.659818in}{0.638614in}}%
\pgfpathmoveto{\pgfqpoint{4.659818in}{0.635665in}}%
\pgfpathlineto{\pgfqpoint{4.659818in}{0.635665in}}%
\pgfpathlineto{\pgfqpoint{4.659818in}{0.638614in}}%
\pgfpathlineto{\pgfqpoint{4.664359in}{0.638614in}}%
\pgfpathlineto{\pgfqpoint{4.664359in}{0.635665in}}%
\pgfpathmoveto{\pgfqpoint{4.664359in}{0.629766in}}%
\pgfpathlineto{\pgfqpoint{4.664359in}{0.629766in}}%
\pgfpathlineto{\pgfqpoint{4.664359in}{0.632716in}}%
\pgfpathlineto{\pgfqpoint{4.668900in}{0.632716in}}%
\pgfpathlineto{\pgfqpoint{4.668900in}{0.629766in}}%
\pgfpathmoveto{\pgfqpoint{4.664359in}{0.632716in}}%
\pgfpathlineto{\pgfqpoint{4.664359in}{0.632716in}}%
\pgfpathlineto{\pgfqpoint{4.664359in}{0.635665in}}%
\pgfpathlineto{\pgfqpoint{4.668900in}{0.635665in}}%
\pgfpathlineto{\pgfqpoint{4.668900in}{0.632716in}}%
\pgfpathmoveto{\pgfqpoint{4.623490in}{0.662208in}}%
\pgfpathlineto{\pgfqpoint{4.623490in}{0.662208in}}%
\pgfpathlineto{\pgfqpoint{4.623490in}{0.665158in}}%
\pgfpathlineto{\pgfqpoint{4.628031in}{0.665158in}}%
\pgfpathlineto{\pgfqpoint{4.628031in}{0.662208in}}%
\pgfpathmoveto{\pgfqpoint{4.632572in}{0.656310in}}%
\pgfpathlineto{\pgfqpoint{4.632572in}{0.656310in}}%
\pgfpathlineto{\pgfqpoint{4.632572in}{0.659259in}}%
\pgfpathlineto{\pgfqpoint{4.637113in}{0.659259in}}%
\pgfpathlineto{\pgfqpoint{4.637113in}{0.656310in}}%
\pgfpathmoveto{\pgfqpoint{4.628031in}{0.659259in}}%
\pgfpathlineto{\pgfqpoint{4.628031in}{0.659259in}}%
\pgfpathlineto{\pgfqpoint{4.628031in}{0.662208in}}%
\pgfpathlineto{\pgfqpoint{4.632572in}{0.662208in}}%
\pgfpathlineto{\pgfqpoint{4.632572in}{0.659259in}}%
\pgfpathmoveto{\pgfqpoint{4.628031in}{0.662208in}}%
\pgfpathlineto{\pgfqpoint{4.628031in}{0.662208in}}%
\pgfpathlineto{\pgfqpoint{4.628031in}{0.665158in}}%
\pgfpathlineto{\pgfqpoint{4.632572in}{0.665158in}}%
\pgfpathlineto{\pgfqpoint{4.632572in}{0.662208in}}%
\pgfpathmoveto{\pgfqpoint{4.632572in}{0.659259in}}%
\pgfpathlineto{\pgfqpoint{4.632572in}{0.659259in}}%
\pgfpathlineto{\pgfqpoint{4.632572in}{0.662208in}}%
\pgfpathlineto{\pgfqpoint{4.637113in}{0.662208in}}%
\pgfpathlineto{\pgfqpoint{4.637113in}{0.659259in}}%
\pgfpathmoveto{\pgfqpoint{4.609867in}{0.674005in}}%
\pgfpathlineto{\pgfqpoint{4.609867in}{0.674005in}}%
\pgfpathlineto{\pgfqpoint{4.609867in}{0.676955in}}%
\pgfpathlineto{\pgfqpoint{4.614408in}{0.676955in}}%
\pgfpathlineto{\pgfqpoint{4.614408in}{0.674005in}}%
\pgfpathmoveto{\pgfqpoint{4.614408in}{0.671056in}}%
\pgfpathlineto{\pgfqpoint{4.614408in}{0.671056in}}%
\pgfpathlineto{\pgfqpoint{4.614408in}{0.674005in}}%
\pgfpathlineto{\pgfqpoint{4.618949in}{0.674005in}}%
\pgfpathlineto{\pgfqpoint{4.618949in}{0.671056in}}%
\pgfpathmoveto{\pgfqpoint{4.614408in}{0.674005in}}%
\pgfpathlineto{\pgfqpoint{4.614408in}{0.674005in}}%
\pgfpathlineto{\pgfqpoint{4.614408in}{0.676955in}}%
\pgfpathlineto{\pgfqpoint{4.618949in}{0.676955in}}%
\pgfpathlineto{\pgfqpoint{4.618949in}{0.674005in}}%
\pgfpathmoveto{\pgfqpoint{4.605326in}{0.679904in}}%
\pgfpathlineto{\pgfqpoint{4.605326in}{0.679904in}}%
\pgfpathlineto{\pgfqpoint{4.605326in}{0.682853in}}%
\pgfpathlineto{\pgfqpoint{4.609867in}{0.682853in}}%
\pgfpathlineto{\pgfqpoint{4.609867in}{0.679904in}}%
\pgfpathmoveto{\pgfqpoint{4.600785in}{0.682853in}}%
\pgfpathlineto{\pgfqpoint{4.600785in}{0.682853in}}%
\pgfpathlineto{\pgfqpoint{4.600785in}{0.685802in}}%
\pgfpathlineto{\pgfqpoint{4.605326in}{0.685802in}}%
\pgfpathlineto{\pgfqpoint{4.605326in}{0.682853in}}%
\pgfpathmoveto{\pgfqpoint{4.600785in}{0.685802in}}%
\pgfpathlineto{\pgfqpoint{4.600785in}{0.685802in}}%
\pgfpathlineto{\pgfqpoint{4.600785in}{0.688752in}}%
\pgfpathlineto{\pgfqpoint{4.605326in}{0.688752in}}%
\pgfpathlineto{\pgfqpoint{4.605326in}{0.685802in}}%
\pgfpathmoveto{\pgfqpoint{4.605326in}{0.682853in}}%
\pgfpathlineto{\pgfqpoint{4.605326in}{0.682853in}}%
\pgfpathlineto{\pgfqpoint{4.605326in}{0.685802in}}%
\pgfpathlineto{\pgfqpoint{4.609867in}{0.685802in}}%
\pgfpathlineto{\pgfqpoint{4.609867in}{0.682853in}}%
\pgfpathmoveto{\pgfqpoint{4.609867in}{0.676955in}}%
\pgfpathlineto{\pgfqpoint{4.609867in}{0.676955in}}%
\pgfpathlineto{\pgfqpoint{4.609867in}{0.679904in}}%
\pgfpathlineto{\pgfqpoint{4.614408in}{0.679904in}}%
\pgfpathlineto{\pgfqpoint{4.614408in}{0.676955in}}%
\pgfpathmoveto{\pgfqpoint{4.609867in}{0.679904in}}%
\pgfpathlineto{\pgfqpoint{4.609867in}{0.679904in}}%
\pgfpathlineto{\pgfqpoint{4.609867in}{0.682853in}}%
\pgfpathlineto{\pgfqpoint{4.614408in}{0.682853in}}%
\pgfpathlineto{\pgfqpoint{4.614408in}{0.679904in}}%
\pgfpathmoveto{\pgfqpoint{4.618949in}{0.668107in}}%
\pgfpathlineto{\pgfqpoint{4.618949in}{0.668107in}}%
\pgfpathlineto{\pgfqpoint{4.618949in}{0.671056in}}%
\pgfpathlineto{\pgfqpoint{4.623490in}{0.671056in}}%
\pgfpathlineto{\pgfqpoint{4.623490in}{0.668107in}}%
\pgfpathmoveto{\pgfqpoint{4.623490in}{0.665158in}}%
\pgfpathlineto{\pgfqpoint{4.623490in}{0.665158in}}%
\pgfpathlineto{\pgfqpoint{4.623490in}{0.668107in}}%
\pgfpathlineto{\pgfqpoint{4.628031in}{0.668107in}}%
\pgfpathlineto{\pgfqpoint{4.628031in}{0.665158in}}%
\pgfpathmoveto{\pgfqpoint{4.623490in}{0.668107in}}%
\pgfpathlineto{\pgfqpoint{4.623490in}{0.668107in}}%
\pgfpathlineto{\pgfqpoint{4.623490in}{0.671056in}}%
\pgfpathlineto{\pgfqpoint{4.628031in}{0.671056in}}%
\pgfpathlineto{\pgfqpoint{4.628031in}{0.668107in}}%
\pgfpathmoveto{\pgfqpoint{4.618949in}{0.671056in}}%
\pgfpathlineto{\pgfqpoint{4.618949in}{0.671056in}}%
\pgfpathlineto{\pgfqpoint{4.618949in}{0.674005in}}%
\pgfpathlineto{\pgfqpoint{4.623490in}{0.674005in}}%
\pgfpathlineto{\pgfqpoint{4.623490in}{0.671056in}}%
\pgfpathmoveto{\pgfqpoint{4.637113in}{0.650411in}}%
\pgfpathlineto{\pgfqpoint{4.637113in}{0.650411in}}%
\pgfpathlineto{\pgfqpoint{4.637113in}{0.653361in}}%
\pgfpathlineto{\pgfqpoint{4.641654in}{0.653361in}}%
\pgfpathlineto{\pgfqpoint{4.641654in}{0.650411in}}%
\pgfpathmoveto{\pgfqpoint{4.641654in}{0.647462in}}%
\pgfpathlineto{\pgfqpoint{4.641654in}{0.647462in}}%
\pgfpathlineto{\pgfqpoint{4.641654in}{0.650411in}}%
\pgfpathlineto{\pgfqpoint{4.646195in}{0.650411in}}%
\pgfpathlineto{\pgfqpoint{4.646195in}{0.647462in}}%
\pgfpathmoveto{\pgfqpoint{4.641654in}{0.650411in}}%
\pgfpathlineto{\pgfqpoint{4.641654in}{0.650411in}}%
\pgfpathlineto{\pgfqpoint{4.641654in}{0.653361in}}%
\pgfpathlineto{\pgfqpoint{4.646195in}{0.653361in}}%
\pgfpathlineto{\pgfqpoint{4.646195in}{0.650411in}}%
\pgfpathmoveto{\pgfqpoint{4.646195in}{0.644513in}}%
\pgfpathlineto{\pgfqpoint{4.646195in}{0.644513in}}%
\pgfpathlineto{\pgfqpoint{4.646195in}{0.647462in}}%
\pgfpathlineto{\pgfqpoint{4.650736in}{0.647462in}}%
\pgfpathlineto{\pgfqpoint{4.650736in}{0.644513in}}%
\pgfpathmoveto{\pgfqpoint{4.650736in}{0.641564in}}%
\pgfpathlineto{\pgfqpoint{4.650736in}{0.641564in}}%
\pgfpathlineto{\pgfqpoint{4.650736in}{0.644513in}}%
\pgfpathlineto{\pgfqpoint{4.655277in}{0.644513in}}%
\pgfpathlineto{\pgfqpoint{4.655277in}{0.641564in}}%
\pgfpathmoveto{\pgfqpoint{4.650736in}{0.644513in}}%
\pgfpathlineto{\pgfqpoint{4.650736in}{0.644513in}}%
\pgfpathlineto{\pgfqpoint{4.650736in}{0.647462in}}%
\pgfpathlineto{\pgfqpoint{4.655277in}{0.647462in}}%
\pgfpathlineto{\pgfqpoint{4.655277in}{0.644513in}}%
\pgfpathmoveto{\pgfqpoint{4.646195in}{0.647462in}}%
\pgfpathlineto{\pgfqpoint{4.646195in}{0.647462in}}%
\pgfpathlineto{\pgfqpoint{4.646195in}{0.650411in}}%
\pgfpathlineto{\pgfqpoint{4.650736in}{0.650411in}}%
\pgfpathlineto{\pgfqpoint{4.650736in}{0.647462in}}%
\pgfpathmoveto{\pgfqpoint{4.637113in}{0.653361in}}%
\pgfpathlineto{\pgfqpoint{4.637113in}{0.653361in}}%
\pgfpathlineto{\pgfqpoint{4.637113in}{0.656310in}}%
\pgfpathlineto{\pgfqpoint{4.641654in}{0.656310in}}%
\pgfpathlineto{\pgfqpoint{4.641654in}{0.653361in}}%
\pgfpathmoveto{\pgfqpoint{4.637113in}{0.656310in}}%
\pgfpathlineto{\pgfqpoint{4.637113in}{0.656310in}}%
\pgfpathlineto{\pgfqpoint{4.637113in}{0.659259in}}%
\pgfpathlineto{\pgfqpoint{4.641654in}{0.659259in}}%
\pgfpathlineto{\pgfqpoint{4.641654in}{0.656310in}}%
\pgfpathmoveto{\pgfqpoint{4.541752in}{0.732989in}}%
\pgfpathlineto{\pgfqpoint{4.541752in}{0.732989in}}%
\pgfpathlineto{\pgfqpoint{4.541752in}{0.735938in}}%
\pgfpathlineto{\pgfqpoint{4.546293in}{0.735938in}}%
\pgfpathlineto{\pgfqpoint{4.546293in}{0.732989in}}%
\pgfpathmoveto{\pgfqpoint{4.555375in}{0.721192in}}%
\pgfpathlineto{\pgfqpoint{4.555375in}{0.721192in}}%
\pgfpathlineto{\pgfqpoint{4.555375in}{0.724141in}}%
\pgfpathlineto{\pgfqpoint{4.559916in}{0.724141in}}%
\pgfpathlineto{\pgfqpoint{4.559916in}{0.721192in}}%
\pgfpathmoveto{\pgfqpoint{4.559916in}{0.718243in}}%
\pgfpathlineto{\pgfqpoint{4.559916in}{0.718243in}}%
\pgfpathlineto{\pgfqpoint{4.559916in}{0.721192in}}%
\pgfpathlineto{\pgfqpoint{4.564457in}{0.721192in}}%
\pgfpathlineto{\pgfqpoint{4.564457in}{0.718243in}}%
\pgfpathmoveto{\pgfqpoint{4.559916in}{0.721192in}}%
\pgfpathlineto{\pgfqpoint{4.559916in}{0.721192in}}%
\pgfpathlineto{\pgfqpoint{4.559916in}{0.724141in}}%
\pgfpathlineto{\pgfqpoint{4.564457in}{0.724141in}}%
\pgfpathlineto{\pgfqpoint{4.564457in}{0.721192in}}%
\pgfpathmoveto{\pgfqpoint{4.550834in}{0.727091in}}%
\pgfpathlineto{\pgfqpoint{4.550834in}{0.727091in}}%
\pgfpathlineto{\pgfqpoint{4.550834in}{0.730040in}}%
\pgfpathlineto{\pgfqpoint{4.555375in}{0.730040in}}%
\pgfpathlineto{\pgfqpoint{4.555375in}{0.727091in}}%
\pgfpathmoveto{\pgfqpoint{4.546293in}{0.730040in}}%
\pgfpathlineto{\pgfqpoint{4.546293in}{0.730040in}}%
\pgfpathlineto{\pgfqpoint{4.546293in}{0.732989in}}%
\pgfpathlineto{\pgfqpoint{4.550834in}{0.732989in}}%
\pgfpathlineto{\pgfqpoint{4.550834in}{0.730040in}}%
\pgfpathmoveto{\pgfqpoint{4.546293in}{0.732989in}}%
\pgfpathlineto{\pgfqpoint{4.546293in}{0.732989in}}%
\pgfpathlineto{\pgfqpoint{4.546293in}{0.735938in}}%
\pgfpathlineto{\pgfqpoint{4.550834in}{0.735938in}}%
\pgfpathlineto{\pgfqpoint{4.550834in}{0.732989in}}%
\pgfpathmoveto{\pgfqpoint{4.550834in}{0.730040in}}%
\pgfpathlineto{\pgfqpoint{4.550834in}{0.730040in}}%
\pgfpathlineto{\pgfqpoint{4.550834in}{0.732989in}}%
\pgfpathlineto{\pgfqpoint{4.555375in}{0.732989in}}%
\pgfpathlineto{\pgfqpoint{4.555375in}{0.730040in}}%
\pgfpathmoveto{\pgfqpoint{4.555375in}{0.724141in}}%
\pgfpathlineto{\pgfqpoint{4.555375in}{0.724141in}}%
\pgfpathlineto{\pgfqpoint{4.555375in}{0.727091in}}%
\pgfpathlineto{\pgfqpoint{4.559916in}{0.727091in}}%
\pgfpathlineto{\pgfqpoint{4.559916in}{0.724141in}}%
\pgfpathmoveto{\pgfqpoint{4.555375in}{0.727091in}}%
\pgfpathlineto{\pgfqpoint{4.555375in}{0.727091in}}%
\pgfpathlineto{\pgfqpoint{4.555375in}{0.730040in}}%
\pgfpathlineto{\pgfqpoint{4.559916in}{0.730040in}}%
\pgfpathlineto{\pgfqpoint{4.559916in}{0.727091in}}%
\pgfpathmoveto{\pgfqpoint{4.568998in}{0.709396in}}%
\pgfpathlineto{\pgfqpoint{4.568998in}{0.709396in}}%
\pgfpathlineto{\pgfqpoint{4.568998in}{0.712345in}}%
\pgfpathlineto{\pgfqpoint{4.573539in}{0.712345in}}%
\pgfpathlineto{\pgfqpoint{4.573539in}{0.709396in}}%
\pgfpathmoveto{\pgfqpoint{4.578080in}{0.703497in}}%
\pgfpathlineto{\pgfqpoint{4.578080in}{0.703497in}}%
\pgfpathlineto{\pgfqpoint{4.578080in}{0.706447in}}%
\pgfpathlineto{\pgfqpoint{4.582621in}{0.706447in}}%
\pgfpathlineto{\pgfqpoint{4.582621in}{0.703497in}}%
\pgfpathmoveto{\pgfqpoint{4.573539in}{0.706447in}}%
\pgfpathlineto{\pgfqpoint{4.573539in}{0.706447in}}%
\pgfpathlineto{\pgfqpoint{4.573539in}{0.709396in}}%
\pgfpathlineto{\pgfqpoint{4.578080in}{0.709396in}}%
\pgfpathlineto{\pgfqpoint{4.578080in}{0.706447in}}%
\pgfpathmoveto{\pgfqpoint{4.573539in}{0.709396in}}%
\pgfpathlineto{\pgfqpoint{4.573539in}{0.709396in}}%
\pgfpathlineto{\pgfqpoint{4.573539in}{0.712345in}}%
\pgfpathlineto{\pgfqpoint{4.578080in}{0.712345in}}%
\pgfpathlineto{\pgfqpoint{4.578080in}{0.709396in}}%
\pgfpathmoveto{\pgfqpoint{4.578080in}{0.706447in}}%
\pgfpathlineto{\pgfqpoint{4.578080in}{0.706447in}}%
\pgfpathlineto{\pgfqpoint{4.578080in}{0.709396in}}%
\pgfpathlineto{\pgfqpoint{4.582621in}{0.709396in}}%
\pgfpathlineto{\pgfqpoint{4.582621in}{0.706447in}}%
\pgfpathmoveto{\pgfqpoint{4.582621in}{0.697599in}}%
\pgfpathlineto{\pgfqpoint{4.582621in}{0.697599in}}%
\pgfpathlineto{\pgfqpoint{4.582621in}{0.700548in}}%
\pgfpathlineto{\pgfqpoint{4.587162in}{0.700548in}}%
\pgfpathlineto{\pgfqpoint{4.587162in}{0.697599in}}%
\pgfpathmoveto{\pgfqpoint{4.587162in}{0.694650in}}%
\pgfpathlineto{\pgfqpoint{4.587162in}{0.694650in}}%
\pgfpathlineto{\pgfqpoint{4.587162in}{0.697599in}}%
\pgfpathlineto{\pgfqpoint{4.591703in}{0.697599in}}%
\pgfpathlineto{\pgfqpoint{4.591703in}{0.694650in}}%
\pgfpathmoveto{\pgfqpoint{4.587162in}{0.697599in}}%
\pgfpathlineto{\pgfqpoint{4.587162in}{0.697599in}}%
\pgfpathlineto{\pgfqpoint{4.587162in}{0.700548in}}%
\pgfpathlineto{\pgfqpoint{4.591703in}{0.700548in}}%
\pgfpathlineto{\pgfqpoint{4.591703in}{0.697599in}}%
\pgfpathmoveto{\pgfqpoint{4.591703in}{0.691701in}}%
\pgfpathlineto{\pgfqpoint{4.591703in}{0.691701in}}%
\pgfpathlineto{\pgfqpoint{4.591703in}{0.694650in}}%
\pgfpathlineto{\pgfqpoint{4.596244in}{0.694650in}}%
\pgfpathlineto{\pgfqpoint{4.596244in}{0.691701in}}%
\pgfpathmoveto{\pgfqpoint{4.596244in}{0.688752in}}%
\pgfpathlineto{\pgfqpoint{4.596244in}{0.688752in}}%
\pgfpathlineto{\pgfqpoint{4.596244in}{0.691701in}}%
\pgfpathlineto{\pgfqpoint{4.600785in}{0.691701in}}%
\pgfpathlineto{\pgfqpoint{4.600785in}{0.688752in}}%
\pgfpathmoveto{\pgfqpoint{4.596244in}{0.691701in}}%
\pgfpathlineto{\pgfqpoint{4.596244in}{0.691701in}}%
\pgfpathlineto{\pgfqpoint{4.596244in}{0.694650in}}%
\pgfpathlineto{\pgfqpoint{4.600785in}{0.694650in}}%
\pgfpathlineto{\pgfqpoint{4.600785in}{0.691701in}}%
\pgfpathmoveto{\pgfqpoint{4.591703in}{0.694650in}}%
\pgfpathlineto{\pgfqpoint{4.591703in}{0.694650in}}%
\pgfpathlineto{\pgfqpoint{4.591703in}{0.697599in}}%
\pgfpathlineto{\pgfqpoint{4.596244in}{0.697599in}}%
\pgfpathlineto{\pgfqpoint{4.596244in}{0.694650in}}%
\pgfpathmoveto{\pgfqpoint{4.582621in}{0.700548in}}%
\pgfpathlineto{\pgfqpoint{4.582621in}{0.700548in}}%
\pgfpathlineto{\pgfqpoint{4.582621in}{0.703497in}}%
\pgfpathlineto{\pgfqpoint{4.587162in}{0.703497in}}%
\pgfpathlineto{\pgfqpoint{4.587162in}{0.700548in}}%
\pgfpathmoveto{\pgfqpoint{4.582621in}{0.703497in}}%
\pgfpathlineto{\pgfqpoint{4.582621in}{0.703497in}}%
\pgfpathlineto{\pgfqpoint{4.582621in}{0.706447in}}%
\pgfpathlineto{\pgfqpoint{4.587162in}{0.706447in}}%
\pgfpathlineto{\pgfqpoint{4.587162in}{0.703497in}}%
\pgfpathmoveto{\pgfqpoint{4.564457in}{0.715294in}}%
\pgfpathlineto{\pgfqpoint{4.564457in}{0.715294in}}%
\pgfpathlineto{\pgfqpoint{4.564457in}{0.718243in}}%
\pgfpathlineto{\pgfqpoint{4.568998in}{0.718243in}}%
\pgfpathlineto{\pgfqpoint{4.568998in}{0.715294in}}%
\pgfpathmoveto{\pgfqpoint{4.568998in}{0.712345in}}%
\pgfpathlineto{\pgfqpoint{4.568998in}{0.712345in}}%
\pgfpathlineto{\pgfqpoint{4.568998in}{0.715294in}}%
\pgfpathlineto{\pgfqpoint{4.573539in}{0.715294in}}%
\pgfpathlineto{\pgfqpoint{4.573539in}{0.712345in}}%
\pgfpathmoveto{\pgfqpoint{4.568998in}{0.715294in}}%
\pgfpathlineto{\pgfqpoint{4.568998in}{0.715294in}}%
\pgfpathlineto{\pgfqpoint{4.568998in}{0.718243in}}%
\pgfpathlineto{\pgfqpoint{4.573539in}{0.718243in}}%
\pgfpathlineto{\pgfqpoint{4.573539in}{0.715294in}}%
\pgfpathmoveto{\pgfqpoint{4.564457in}{0.718243in}}%
\pgfpathlineto{\pgfqpoint{4.564457in}{0.718243in}}%
\pgfpathlineto{\pgfqpoint{4.564457in}{0.721192in}}%
\pgfpathlineto{\pgfqpoint{4.568998in}{0.721192in}}%
\pgfpathlineto{\pgfqpoint{4.568998in}{0.718243in}}%
\pgfpathmoveto{\pgfqpoint{4.528129in}{0.744785in}}%
\pgfpathlineto{\pgfqpoint{4.528129in}{0.744785in}}%
\pgfpathlineto{\pgfqpoint{4.528129in}{0.747735in}}%
\pgfpathlineto{\pgfqpoint{4.532670in}{0.747735in}}%
\pgfpathlineto{\pgfqpoint{4.532670in}{0.744785in}}%
\pgfpathmoveto{\pgfqpoint{4.532670in}{0.741836in}}%
\pgfpathlineto{\pgfqpoint{4.532670in}{0.741836in}}%
\pgfpathlineto{\pgfqpoint{4.532670in}{0.744785in}}%
\pgfpathlineto{\pgfqpoint{4.537211in}{0.744785in}}%
\pgfpathlineto{\pgfqpoint{4.537211in}{0.741836in}}%
\pgfpathmoveto{\pgfqpoint{4.532670in}{0.744785in}}%
\pgfpathlineto{\pgfqpoint{4.532670in}{0.744785in}}%
\pgfpathlineto{\pgfqpoint{4.532670in}{0.747735in}}%
\pgfpathlineto{\pgfqpoint{4.537211in}{0.747735in}}%
\pgfpathlineto{\pgfqpoint{4.537211in}{0.744785in}}%
\pgfpathmoveto{\pgfqpoint{4.537211in}{0.738887in}}%
\pgfpathlineto{\pgfqpoint{4.537211in}{0.738887in}}%
\pgfpathlineto{\pgfqpoint{4.537211in}{0.741836in}}%
\pgfpathlineto{\pgfqpoint{4.541752in}{0.741836in}}%
\pgfpathlineto{\pgfqpoint{4.541752in}{0.738887in}}%
\pgfpathmoveto{\pgfqpoint{4.541752in}{0.735938in}}%
\pgfpathlineto{\pgfqpoint{4.541752in}{0.735938in}}%
\pgfpathlineto{\pgfqpoint{4.541752in}{0.738887in}}%
\pgfpathlineto{\pgfqpoint{4.546293in}{0.738887in}}%
\pgfpathlineto{\pgfqpoint{4.546293in}{0.735938in}}%
\pgfpathmoveto{\pgfqpoint{4.541752in}{0.738887in}}%
\pgfpathlineto{\pgfqpoint{4.541752in}{0.738887in}}%
\pgfpathlineto{\pgfqpoint{4.541752in}{0.741836in}}%
\pgfpathlineto{\pgfqpoint{4.546293in}{0.741836in}}%
\pgfpathlineto{\pgfqpoint{4.546293in}{0.738887in}}%
\pgfpathmoveto{\pgfqpoint{4.537211in}{0.741836in}}%
\pgfpathlineto{\pgfqpoint{4.537211in}{0.741836in}}%
\pgfpathlineto{\pgfqpoint{4.537211in}{0.744785in}}%
\pgfpathlineto{\pgfqpoint{4.541752in}{0.744785in}}%
\pgfpathlineto{\pgfqpoint{4.541752in}{0.741836in}}%
\pgfpathmoveto{\pgfqpoint{4.528129in}{0.747735in}}%
\pgfpathlineto{\pgfqpoint{4.528129in}{0.747735in}}%
\pgfpathlineto{\pgfqpoint{4.528129in}{0.750684in}}%
\pgfpathlineto{\pgfqpoint{4.532670in}{0.750684in}}%
\pgfpathlineto{\pgfqpoint{4.532670in}{0.747735in}}%
\pgfpathmoveto{\pgfqpoint{4.528129in}{0.750684in}}%
\pgfpathlineto{\pgfqpoint{4.528129in}{0.750684in}}%
\pgfpathlineto{\pgfqpoint{4.528129in}{0.753633in}}%
\pgfpathlineto{\pgfqpoint{4.532670in}{0.753633in}}%
\pgfpathlineto{\pgfqpoint{4.532670in}{0.750684in}}%
\pgfpathmoveto{\pgfqpoint{4.705227in}{0.591426in}}%
\pgfpathlineto{\pgfqpoint{4.705227in}{0.591426in}}%
\pgfpathlineto{\pgfqpoint{4.705227in}{0.594375in}}%
\pgfpathlineto{\pgfqpoint{4.709768in}{0.594375in}}%
\pgfpathlineto{\pgfqpoint{4.709768in}{0.591426in}}%
\pgfpathmoveto{\pgfqpoint{4.732472in}{0.567833in}}%
\pgfpathlineto{\pgfqpoint{4.732472in}{0.567833in}}%
\pgfpathlineto{\pgfqpoint{4.732472in}{0.570782in}}%
\pgfpathlineto{\pgfqpoint{4.737013in}{0.570782in}}%
\pgfpathlineto{\pgfqpoint{4.737013in}{0.567833in}}%
\pgfpathmoveto{\pgfqpoint{4.741554in}{0.561934in}}%
\pgfpathlineto{\pgfqpoint{4.741554in}{0.561934in}}%
\pgfpathlineto{\pgfqpoint{4.741554in}{0.564883in}}%
\pgfpathlineto{\pgfqpoint{4.746095in}{0.564883in}}%
\pgfpathlineto{\pgfqpoint{4.746095in}{0.561934in}}%
\pgfpathmoveto{\pgfqpoint{4.737013in}{0.564883in}}%
\pgfpathlineto{\pgfqpoint{4.737013in}{0.564883in}}%
\pgfpathlineto{\pgfqpoint{4.737013in}{0.567833in}}%
\pgfpathlineto{\pgfqpoint{4.741554in}{0.567833in}}%
\pgfpathlineto{\pgfqpoint{4.741554in}{0.564883in}}%
\pgfpathmoveto{\pgfqpoint{4.737013in}{0.567833in}}%
\pgfpathlineto{\pgfqpoint{4.737013in}{0.567833in}}%
\pgfpathlineto{\pgfqpoint{4.737013in}{0.570782in}}%
\pgfpathlineto{\pgfqpoint{4.741554in}{0.570782in}}%
\pgfpathlineto{\pgfqpoint{4.741554in}{0.567833in}}%
\pgfpathmoveto{\pgfqpoint{4.741554in}{0.564883in}}%
\pgfpathlineto{\pgfqpoint{4.741554in}{0.564883in}}%
\pgfpathlineto{\pgfqpoint{4.741554in}{0.567833in}}%
\pgfpathlineto{\pgfqpoint{4.746095in}{0.567833in}}%
\pgfpathlineto{\pgfqpoint{4.746095in}{0.564883in}}%
\pgfpathmoveto{\pgfqpoint{4.718850in}{0.579629in}}%
\pgfpathlineto{\pgfqpoint{4.718850in}{0.579629in}}%
\pgfpathlineto{\pgfqpoint{4.718850in}{0.582579in}}%
\pgfpathlineto{\pgfqpoint{4.723390in}{0.582579in}}%
\pgfpathlineto{\pgfqpoint{4.723390in}{0.579629in}}%
\pgfpathmoveto{\pgfqpoint{4.723390in}{0.576680in}}%
\pgfpathlineto{\pgfqpoint{4.723390in}{0.576680in}}%
\pgfpathlineto{\pgfqpoint{4.723390in}{0.579629in}}%
\pgfpathlineto{\pgfqpoint{4.727931in}{0.579629in}}%
\pgfpathlineto{\pgfqpoint{4.727931in}{0.576680in}}%
\pgfpathmoveto{\pgfqpoint{4.723390in}{0.579629in}}%
\pgfpathlineto{\pgfqpoint{4.723390in}{0.579629in}}%
\pgfpathlineto{\pgfqpoint{4.723390in}{0.582579in}}%
\pgfpathlineto{\pgfqpoint{4.727931in}{0.582579in}}%
\pgfpathlineto{\pgfqpoint{4.727931in}{0.579629in}}%
\pgfpathmoveto{\pgfqpoint{4.714309in}{0.585528in}}%
\pgfpathlineto{\pgfqpoint{4.714309in}{0.585528in}}%
\pgfpathlineto{\pgfqpoint{4.714309in}{0.588477in}}%
\pgfpathlineto{\pgfqpoint{4.718850in}{0.588477in}}%
\pgfpathlineto{\pgfqpoint{4.718850in}{0.585528in}}%
\pgfpathmoveto{\pgfqpoint{4.709768in}{0.588477in}}%
\pgfpathlineto{\pgfqpoint{4.709768in}{0.588477in}}%
\pgfpathlineto{\pgfqpoint{4.709768in}{0.591426in}}%
\pgfpathlineto{\pgfqpoint{4.714309in}{0.591426in}}%
\pgfpathlineto{\pgfqpoint{4.714309in}{0.588477in}}%
\pgfpathmoveto{\pgfqpoint{4.709768in}{0.591426in}}%
\pgfpathlineto{\pgfqpoint{4.709768in}{0.591426in}}%
\pgfpathlineto{\pgfqpoint{4.709768in}{0.594375in}}%
\pgfpathlineto{\pgfqpoint{4.714309in}{0.594375in}}%
\pgfpathlineto{\pgfqpoint{4.714309in}{0.591426in}}%
\pgfpathmoveto{\pgfqpoint{4.714309in}{0.588477in}}%
\pgfpathlineto{\pgfqpoint{4.714309in}{0.588477in}}%
\pgfpathlineto{\pgfqpoint{4.714309in}{0.591426in}}%
\pgfpathlineto{\pgfqpoint{4.718850in}{0.591426in}}%
\pgfpathlineto{\pgfqpoint{4.718850in}{0.588477in}}%
\pgfpathmoveto{\pgfqpoint{4.718850in}{0.582579in}}%
\pgfpathlineto{\pgfqpoint{4.718850in}{0.582579in}}%
\pgfpathlineto{\pgfqpoint{4.718850in}{0.585528in}}%
\pgfpathlineto{\pgfqpoint{4.723390in}{0.585528in}}%
\pgfpathlineto{\pgfqpoint{4.723390in}{0.582579in}}%
\pgfpathmoveto{\pgfqpoint{4.718850in}{0.585528in}}%
\pgfpathlineto{\pgfqpoint{4.718850in}{0.585528in}}%
\pgfpathlineto{\pgfqpoint{4.718850in}{0.588477in}}%
\pgfpathlineto{\pgfqpoint{4.723390in}{0.588477in}}%
\pgfpathlineto{\pgfqpoint{4.723390in}{0.585528in}}%
\pgfpathmoveto{\pgfqpoint{4.727931in}{0.573731in}}%
\pgfpathlineto{\pgfqpoint{4.727931in}{0.573731in}}%
\pgfpathlineto{\pgfqpoint{4.727931in}{0.576680in}}%
\pgfpathlineto{\pgfqpoint{4.732472in}{0.576680in}}%
\pgfpathlineto{\pgfqpoint{4.732472in}{0.573731in}}%
\pgfpathmoveto{\pgfqpoint{4.732472in}{0.570782in}}%
\pgfpathlineto{\pgfqpoint{4.732472in}{0.570782in}}%
\pgfpathlineto{\pgfqpoint{4.732472in}{0.573731in}}%
\pgfpathlineto{\pgfqpoint{4.737013in}{0.573731in}}%
\pgfpathlineto{\pgfqpoint{4.737013in}{0.570782in}}%
\pgfpathmoveto{\pgfqpoint{4.732472in}{0.573731in}}%
\pgfpathlineto{\pgfqpoint{4.732472in}{0.573731in}}%
\pgfpathlineto{\pgfqpoint{4.732472in}{0.576680in}}%
\pgfpathlineto{\pgfqpoint{4.737013in}{0.576680in}}%
\pgfpathlineto{\pgfqpoint{4.737013in}{0.573731in}}%
\pgfpathmoveto{\pgfqpoint{4.727931in}{0.576680in}}%
\pgfpathlineto{\pgfqpoint{4.727931in}{0.576680in}}%
\pgfpathlineto{\pgfqpoint{4.727931in}{0.579629in}}%
\pgfpathlineto{\pgfqpoint{4.732472in}{0.579629in}}%
\pgfpathlineto{\pgfqpoint{4.732472in}{0.576680in}}%
\pgfpathmoveto{\pgfqpoint{4.759718in}{0.544239in}}%
\pgfpathlineto{\pgfqpoint{4.759718in}{0.544239in}}%
\pgfpathlineto{\pgfqpoint{4.759718in}{0.547188in}}%
\pgfpathlineto{\pgfqpoint{4.764259in}{0.547188in}}%
\pgfpathlineto{\pgfqpoint{4.764259in}{0.544239in}}%
\pgfpathmoveto{\pgfqpoint{4.773340in}{0.532442in}}%
\pgfpathlineto{\pgfqpoint{4.773340in}{0.532442in}}%
\pgfpathlineto{\pgfqpoint{4.773340in}{0.535391in}}%
\pgfpathlineto{\pgfqpoint{4.777881in}{0.535391in}}%
\pgfpathlineto{\pgfqpoint{4.777881in}{0.532442in}}%
\pgfpathmoveto{\pgfqpoint{4.777881in}{0.529493in}}%
\pgfpathlineto{\pgfqpoint{4.777881in}{0.529493in}}%
\pgfpathlineto{\pgfqpoint{4.777881in}{0.532442in}}%
\pgfpathlineto{\pgfqpoint{4.782422in}{0.532442in}}%
\pgfpathlineto{\pgfqpoint{4.782422in}{0.529493in}}%
\pgfpathmoveto{\pgfqpoint{4.777881in}{0.532442in}}%
\pgfpathlineto{\pgfqpoint{4.777881in}{0.532442in}}%
\pgfpathlineto{\pgfqpoint{4.777881in}{0.535391in}}%
\pgfpathlineto{\pgfqpoint{4.782422in}{0.535391in}}%
\pgfpathlineto{\pgfqpoint{4.782422in}{0.532442in}}%
\pgfpathmoveto{\pgfqpoint{4.768799in}{0.538341in}}%
\pgfpathlineto{\pgfqpoint{4.768799in}{0.538341in}}%
\pgfpathlineto{\pgfqpoint{4.768799in}{0.541290in}}%
\pgfpathlineto{\pgfqpoint{4.773340in}{0.541290in}}%
\pgfpathlineto{\pgfqpoint{4.773340in}{0.538341in}}%
\pgfpathmoveto{\pgfqpoint{4.764259in}{0.541290in}}%
\pgfpathlineto{\pgfqpoint{4.764259in}{0.541290in}}%
\pgfpathlineto{\pgfqpoint{4.764259in}{0.544239in}}%
\pgfpathlineto{\pgfqpoint{4.768799in}{0.544239in}}%
\pgfpathlineto{\pgfqpoint{4.768799in}{0.541290in}}%
\pgfpathmoveto{\pgfqpoint{4.764259in}{0.544239in}}%
\pgfpathlineto{\pgfqpoint{4.764259in}{0.544239in}}%
\pgfpathlineto{\pgfqpoint{4.764259in}{0.547188in}}%
\pgfpathlineto{\pgfqpoint{4.768799in}{0.547188in}}%
\pgfpathlineto{\pgfqpoint{4.768799in}{0.544239in}}%
\pgfpathmoveto{\pgfqpoint{4.768799in}{0.541290in}}%
\pgfpathlineto{\pgfqpoint{4.768799in}{0.541290in}}%
\pgfpathlineto{\pgfqpoint{4.768799in}{0.544239in}}%
\pgfpathlineto{\pgfqpoint{4.773340in}{0.544239in}}%
\pgfpathlineto{\pgfqpoint{4.773340in}{0.541290in}}%
\pgfpathmoveto{\pgfqpoint{4.773340in}{0.535391in}}%
\pgfpathlineto{\pgfqpoint{4.773340in}{0.535391in}}%
\pgfpathlineto{\pgfqpoint{4.773340in}{0.538341in}}%
\pgfpathlineto{\pgfqpoint{4.777881in}{0.538341in}}%
\pgfpathlineto{\pgfqpoint{4.777881in}{0.535391in}}%
\pgfpathmoveto{\pgfqpoint{4.773340in}{0.538341in}}%
\pgfpathlineto{\pgfqpoint{4.773340in}{0.538341in}}%
\pgfpathlineto{\pgfqpoint{4.773340in}{0.541290in}}%
\pgfpathlineto{\pgfqpoint{4.777881in}{0.541290in}}%
\pgfpathlineto{\pgfqpoint{4.777881in}{0.538341in}}%
\pgfpathmoveto{\pgfqpoint{4.786963in}{0.520645in}}%
\pgfpathlineto{\pgfqpoint{4.786963in}{0.520645in}}%
\pgfpathlineto{\pgfqpoint{4.786963in}{0.523595in}}%
\pgfpathlineto{\pgfqpoint{4.791504in}{0.523595in}}%
\pgfpathlineto{\pgfqpoint{4.791504in}{0.520645in}}%
\pgfpathmoveto{\pgfqpoint{4.796045in}{0.514747in}}%
\pgfpathlineto{\pgfqpoint{4.796045in}{0.514747in}}%
\pgfpathlineto{\pgfqpoint{4.796045in}{0.517696in}}%
\pgfpathlineto{\pgfqpoint{4.800586in}{0.517696in}}%
\pgfpathlineto{\pgfqpoint{4.800586in}{0.514747in}}%
\pgfpathmoveto{\pgfqpoint{4.791504in}{0.517696in}}%
\pgfpathlineto{\pgfqpoint{4.791504in}{0.517696in}}%
\pgfpathlineto{\pgfqpoint{4.791504in}{0.520645in}}%
\pgfpathlineto{\pgfqpoint{4.796045in}{0.520645in}}%
\pgfpathlineto{\pgfqpoint{4.796045in}{0.517696in}}%
\pgfpathmoveto{\pgfqpoint{4.791504in}{0.520645in}}%
\pgfpathlineto{\pgfqpoint{4.791504in}{0.520645in}}%
\pgfpathlineto{\pgfqpoint{4.791504in}{0.523595in}}%
\pgfpathlineto{\pgfqpoint{4.796045in}{0.523595in}}%
\pgfpathlineto{\pgfqpoint{4.796045in}{0.520645in}}%
\pgfpathmoveto{\pgfqpoint{4.796045in}{0.517696in}}%
\pgfpathlineto{\pgfqpoint{4.796045in}{0.517696in}}%
\pgfpathlineto{\pgfqpoint{4.796045in}{0.520645in}}%
\pgfpathlineto{\pgfqpoint{4.800586in}{0.520645in}}%
\pgfpathlineto{\pgfqpoint{4.800586in}{0.517696in}}%
\pgfpathmoveto{\pgfqpoint{4.800586in}{0.508849in}}%
\pgfpathlineto{\pgfqpoint{4.800586in}{0.508849in}}%
\pgfpathlineto{\pgfqpoint{4.800586in}{0.511798in}}%
\pgfpathlineto{\pgfqpoint{4.805127in}{0.511798in}}%
\pgfpathlineto{\pgfqpoint{4.805127in}{0.508849in}}%
\pgfpathmoveto{\pgfqpoint{4.805127in}{0.505899in}}%
\pgfpathlineto{\pgfqpoint{4.805127in}{0.505899in}}%
\pgfpathlineto{\pgfqpoint{4.805127in}{0.508849in}}%
\pgfpathlineto{\pgfqpoint{4.809668in}{0.508849in}}%
\pgfpathlineto{\pgfqpoint{4.809668in}{0.505899in}}%
\pgfpathmoveto{\pgfqpoint{4.805127in}{0.508849in}}%
\pgfpathlineto{\pgfqpoint{4.805127in}{0.508849in}}%
\pgfpathlineto{\pgfqpoint{4.805127in}{0.511798in}}%
\pgfpathlineto{\pgfqpoint{4.809668in}{0.511798in}}%
\pgfpathlineto{\pgfqpoint{4.809668in}{0.508849in}}%
\pgfpathmoveto{\pgfqpoint{4.809668in}{0.502950in}}%
\pgfpathlineto{\pgfqpoint{4.809668in}{0.502950in}}%
\pgfpathlineto{\pgfqpoint{4.809668in}{0.505899in}}%
\pgfpathlineto{\pgfqpoint{4.814208in}{0.505899in}}%
\pgfpathlineto{\pgfqpoint{4.814208in}{0.502950in}}%
\pgfpathmoveto{\pgfqpoint{4.814208in}{0.500001in}}%
\pgfpathlineto{\pgfqpoint{4.814208in}{0.500001in}}%
\pgfpathlineto{\pgfqpoint{4.814208in}{0.502950in}}%
\pgfpathlineto{\pgfqpoint{4.818749in}{0.502950in}}%
\pgfpathlineto{\pgfqpoint{4.818749in}{0.500001in}}%
\pgfpathmoveto{\pgfqpoint{4.814208in}{0.502950in}}%
\pgfpathlineto{\pgfqpoint{4.814208in}{0.502950in}}%
\pgfpathlineto{\pgfqpoint{4.814208in}{0.505899in}}%
\pgfpathlineto{\pgfqpoint{4.818749in}{0.505899in}}%
\pgfpathlineto{\pgfqpoint{4.818749in}{0.502950in}}%
\pgfpathmoveto{\pgfqpoint{4.809668in}{0.505899in}}%
\pgfpathlineto{\pgfqpoint{4.809668in}{0.505899in}}%
\pgfpathlineto{\pgfqpoint{4.809668in}{0.508849in}}%
\pgfpathlineto{\pgfqpoint{4.814208in}{0.508849in}}%
\pgfpathlineto{\pgfqpoint{4.814208in}{0.505899in}}%
\pgfpathmoveto{\pgfqpoint{4.800586in}{0.511798in}}%
\pgfpathlineto{\pgfqpoint{4.800586in}{0.511798in}}%
\pgfpathlineto{\pgfqpoint{4.800586in}{0.514747in}}%
\pgfpathlineto{\pgfqpoint{4.805127in}{0.514747in}}%
\pgfpathlineto{\pgfqpoint{4.805127in}{0.511798in}}%
\pgfpathmoveto{\pgfqpoint{4.800586in}{0.514747in}}%
\pgfpathlineto{\pgfqpoint{4.800586in}{0.514747in}}%
\pgfpathlineto{\pgfqpoint{4.800586in}{0.517696in}}%
\pgfpathlineto{\pgfqpoint{4.805127in}{0.517696in}}%
\pgfpathlineto{\pgfqpoint{4.805127in}{0.514747in}}%
\pgfpathmoveto{\pgfqpoint{4.782422in}{0.526544in}}%
\pgfpathlineto{\pgfqpoint{4.782422in}{0.526544in}}%
\pgfpathlineto{\pgfqpoint{4.782422in}{0.529493in}}%
\pgfpathlineto{\pgfqpoint{4.786963in}{0.529493in}}%
\pgfpathlineto{\pgfqpoint{4.786963in}{0.526544in}}%
\pgfpathmoveto{\pgfqpoint{4.786963in}{0.523595in}}%
\pgfpathlineto{\pgfqpoint{4.786963in}{0.523595in}}%
\pgfpathlineto{\pgfqpoint{4.786963in}{0.526544in}}%
\pgfpathlineto{\pgfqpoint{4.791504in}{0.526544in}}%
\pgfpathlineto{\pgfqpoint{4.791504in}{0.523595in}}%
\pgfpathmoveto{\pgfqpoint{4.786963in}{0.526544in}}%
\pgfpathlineto{\pgfqpoint{4.786963in}{0.526544in}}%
\pgfpathlineto{\pgfqpoint{4.786963in}{0.529493in}}%
\pgfpathlineto{\pgfqpoint{4.791504in}{0.529493in}}%
\pgfpathlineto{\pgfqpoint{4.791504in}{0.526544in}}%
\pgfpathmoveto{\pgfqpoint{4.782422in}{0.529493in}}%
\pgfpathlineto{\pgfqpoint{4.782422in}{0.529493in}}%
\pgfpathlineto{\pgfqpoint{4.782422in}{0.532442in}}%
\pgfpathlineto{\pgfqpoint{4.786963in}{0.532442in}}%
\pgfpathlineto{\pgfqpoint{4.786963in}{0.529493in}}%
\pgfpathmoveto{\pgfqpoint{4.746095in}{0.556036in}}%
\pgfpathlineto{\pgfqpoint{4.746095in}{0.556036in}}%
\pgfpathlineto{\pgfqpoint{4.746095in}{0.558985in}}%
\pgfpathlineto{\pgfqpoint{4.750636in}{0.558985in}}%
\pgfpathlineto{\pgfqpoint{4.750636in}{0.556036in}}%
\pgfpathmoveto{\pgfqpoint{4.750636in}{0.553087in}}%
\pgfpathlineto{\pgfqpoint{4.750636in}{0.553087in}}%
\pgfpathlineto{\pgfqpoint{4.750636in}{0.556036in}}%
\pgfpathlineto{\pgfqpoint{4.755177in}{0.556036in}}%
\pgfpathlineto{\pgfqpoint{4.755177in}{0.553087in}}%
\pgfpathmoveto{\pgfqpoint{4.750636in}{0.556036in}}%
\pgfpathlineto{\pgfqpoint{4.750636in}{0.556036in}}%
\pgfpathlineto{\pgfqpoint{4.750636in}{0.558985in}}%
\pgfpathlineto{\pgfqpoint{4.755177in}{0.558985in}}%
\pgfpathlineto{\pgfqpoint{4.755177in}{0.556036in}}%
\pgfpathmoveto{\pgfqpoint{4.755177in}{0.550137in}}%
\pgfpathlineto{\pgfqpoint{4.755177in}{0.550137in}}%
\pgfpathlineto{\pgfqpoint{4.755177in}{0.553087in}}%
\pgfpathlineto{\pgfqpoint{4.759718in}{0.553087in}}%
\pgfpathlineto{\pgfqpoint{4.759718in}{0.550137in}}%
\pgfpathmoveto{\pgfqpoint{4.759718in}{0.547188in}}%
\pgfpathlineto{\pgfqpoint{4.759718in}{0.547188in}}%
\pgfpathlineto{\pgfqpoint{4.759718in}{0.550137in}}%
\pgfpathlineto{\pgfqpoint{4.764259in}{0.550137in}}%
\pgfpathlineto{\pgfqpoint{4.764259in}{0.547188in}}%
\pgfpathmoveto{\pgfqpoint{4.759718in}{0.550137in}}%
\pgfpathlineto{\pgfqpoint{4.759718in}{0.550137in}}%
\pgfpathlineto{\pgfqpoint{4.759718in}{0.553087in}}%
\pgfpathlineto{\pgfqpoint{4.764259in}{0.553087in}}%
\pgfpathlineto{\pgfqpoint{4.764259in}{0.550137in}}%
\pgfpathmoveto{\pgfqpoint{4.755177in}{0.553087in}}%
\pgfpathlineto{\pgfqpoint{4.755177in}{0.553087in}}%
\pgfpathlineto{\pgfqpoint{4.755177in}{0.556036in}}%
\pgfpathlineto{\pgfqpoint{4.759718in}{0.556036in}}%
\pgfpathlineto{\pgfqpoint{4.759718in}{0.553087in}}%
\pgfpathmoveto{\pgfqpoint{4.746095in}{0.558985in}}%
\pgfpathlineto{\pgfqpoint{4.746095in}{0.558985in}}%
\pgfpathlineto{\pgfqpoint{4.746095in}{0.561934in}}%
\pgfpathlineto{\pgfqpoint{4.750636in}{0.561934in}}%
\pgfpathlineto{\pgfqpoint{4.750636in}{0.558985in}}%
\pgfpathmoveto{\pgfqpoint{4.746095in}{0.561934in}}%
\pgfpathlineto{\pgfqpoint{4.746095in}{0.561934in}}%
\pgfpathlineto{\pgfqpoint{4.746095in}{0.564883in}}%
\pgfpathlineto{\pgfqpoint{4.750636in}{0.564883in}}%
\pgfpathlineto{\pgfqpoint{4.750636in}{0.561934in}}%
\pgfpathmoveto{\pgfqpoint{4.677981in}{0.615020in}}%
\pgfpathlineto{\pgfqpoint{4.677981in}{0.615020in}}%
\pgfpathlineto{\pgfqpoint{4.677981in}{0.617969in}}%
\pgfpathlineto{\pgfqpoint{4.682522in}{0.617969in}}%
\pgfpathlineto{\pgfqpoint{4.682522in}{0.615020in}}%
\pgfpathmoveto{\pgfqpoint{4.687063in}{0.609122in}}%
\pgfpathlineto{\pgfqpoint{4.687063in}{0.609122in}}%
\pgfpathlineto{\pgfqpoint{4.687063in}{0.612071in}}%
\pgfpathlineto{\pgfqpoint{4.691604in}{0.612071in}}%
\pgfpathlineto{\pgfqpoint{4.691604in}{0.609122in}}%
\pgfpathmoveto{\pgfqpoint{4.682522in}{0.612071in}}%
\pgfpathlineto{\pgfqpoint{4.682522in}{0.612071in}}%
\pgfpathlineto{\pgfqpoint{4.682522in}{0.615020in}}%
\pgfpathlineto{\pgfqpoint{4.687063in}{0.615020in}}%
\pgfpathlineto{\pgfqpoint{4.687063in}{0.612071in}}%
\pgfpathmoveto{\pgfqpoint{4.682522in}{0.615020in}}%
\pgfpathlineto{\pgfqpoint{4.682522in}{0.615020in}}%
\pgfpathlineto{\pgfqpoint{4.682522in}{0.617969in}}%
\pgfpathlineto{\pgfqpoint{4.687063in}{0.617969in}}%
\pgfpathlineto{\pgfqpoint{4.687063in}{0.615020in}}%
\pgfpathmoveto{\pgfqpoint{4.687063in}{0.612071in}}%
\pgfpathlineto{\pgfqpoint{4.687063in}{0.612071in}}%
\pgfpathlineto{\pgfqpoint{4.687063in}{0.615020in}}%
\pgfpathlineto{\pgfqpoint{4.691604in}{0.615020in}}%
\pgfpathlineto{\pgfqpoint{4.691604in}{0.612071in}}%
\pgfpathmoveto{\pgfqpoint{4.691604in}{0.603223in}}%
\pgfpathlineto{\pgfqpoint{4.691604in}{0.603223in}}%
\pgfpathlineto{\pgfqpoint{4.691604in}{0.606172in}}%
\pgfpathlineto{\pgfqpoint{4.696145in}{0.606172in}}%
\pgfpathlineto{\pgfqpoint{4.696145in}{0.603223in}}%
\pgfpathmoveto{\pgfqpoint{4.696145in}{0.600274in}}%
\pgfpathlineto{\pgfqpoint{4.696145in}{0.600274in}}%
\pgfpathlineto{\pgfqpoint{4.696145in}{0.603223in}}%
\pgfpathlineto{\pgfqpoint{4.700686in}{0.603223in}}%
\pgfpathlineto{\pgfqpoint{4.700686in}{0.600274in}}%
\pgfpathmoveto{\pgfqpoint{4.696145in}{0.603223in}}%
\pgfpathlineto{\pgfqpoint{4.696145in}{0.603223in}}%
\pgfpathlineto{\pgfqpoint{4.696145in}{0.606172in}}%
\pgfpathlineto{\pgfqpoint{4.700686in}{0.606172in}}%
\pgfpathlineto{\pgfqpoint{4.700686in}{0.603223in}}%
\pgfpathmoveto{\pgfqpoint{4.700686in}{0.597325in}}%
\pgfpathlineto{\pgfqpoint{4.700686in}{0.597325in}}%
\pgfpathlineto{\pgfqpoint{4.700686in}{0.600274in}}%
\pgfpathlineto{\pgfqpoint{4.705227in}{0.600274in}}%
\pgfpathlineto{\pgfqpoint{4.705227in}{0.597325in}}%
\pgfpathmoveto{\pgfqpoint{4.705227in}{0.594375in}}%
\pgfpathlineto{\pgfqpoint{4.705227in}{0.594375in}}%
\pgfpathlineto{\pgfqpoint{4.705227in}{0.597325in}}%
\pgfpathlineto{\pgfqpoint{4.709768in}{0.597325in}}%
\pgfpathlineto{\pgfqpoint{4.709768in}{0.594375in}}%
\pgfpathmoveto{\pgfqpoint{4.705227in}{0.597325in}}%
\pgfpathlineto{\pgfqpoint{4.705227in}{0.597325in}}%
\pgfpathlineto{\pgfqpoint{4.705227in}{0.600274in}}%
\pgfpathlineto{\pgfqpoint{4.709768in}{0.600274in}}%
\pgfpathlineto{\pgfqpoint{4.709768in}{0.597325in}}%
\pgfpathmoveto{\pgfqpoint{4.700686in}{0.600274in}}%
\pgfpathlineto{\pgfqpoint{4.700686in}{0.600274in}}%
\pgfpathlineto{\pgfqpoint{4.700686in}{0.603223in}}%
\pgfpathlineto{\pgfqpoint{4.705227in}{0.603223in}}%
\pgfpathlineto{\pgfqpoint{4.705227in}{0.600274in}}%
\pgfpathmoveto{\pgfqpoint{4.691604in}{0.606172in}}%
\pgfpathlineto{\pgfqpoint{4.691604in}{0.606172in}}%
\pgfpathlineto{\pgfqpoint{4.691604in}{0.609122in}}%
\pgfpathlineto{\pgfqpoint{4.696145in}{0.609122in}}%
\pgfpathlineto{\pgfqpoint{4.696145in}{0.606172in}}%
\pgfpathmoveto{\pgfqpoint{4.691604in}{0.609122in}}%
\pgfpathlineto{\pgfqpoint{4.691604in}{0.609122in}}%
\pgfpathlineto{\pgfqpoint{4.691604in}{0.612071in}}%
\pgfpathlineto{\pgfqpoint{4.696145in}{0.612071in}}%
\pgfpathlineto{\pgfqpoint{4.696145in}{0.609122in}}%
\pgfpathmoveto{\pgfqpoint{4.673441in}{0.620919in}}%
\pgfpathlineto{\pgfqpoint{4.673441in}{0.620919in}}%
\pgfpathlineto{\pgfqpoint{4.673441in}{0.623868in}}%
\pgfpathlineto{\pgfqpoint{4.677981in}{0.623868in}}%
\pgfpathlineto{\pgfqpoint{4.677981in}{0.620919in}}%
\pgfpathmoveto{\pgfqpoint{4.677981in}{0.617969in}}%
\pgfpathlineto{\pgfqpoint{4.677981in}{0.617969in}}%
\pgfpathlineto{\pgfqpoint{4.677981in}{0.620919in}}%
\pgfpathlineto{\pgfqpoint{4.682522in}{0.620919in}}%
\pgfpathlineto{\pgfqpoint{4.682522in}{0.617969in}}%
\pgfpathmoveto{\pgfqpoint{4.677981in}{0.620919in}}%
\pgfpathlineto{\pgfqpoint{4.677981in}{0.620919in}}%
\pgfpathlineto{\pgfqpoint{4.677981in}{0.623868in}}%
\pgfpathlineto{\pgfqpoint{4.682522in}{0.623868in}}%
\pgfpathlineto{\pgfqpoint{4.682522in}{0.620919in}}%
\pgfpathmoveto{\pgfqpoint{4.673441in}{0.623868in}}%
\pgfpathlineto{\pgfqpoint{4.673441in}{0.623868in}}%
\pgfpathlineto{\pgfqpoint{4.673441in}{0.626817in}}%
\pgfpathlineto{\pgfqpoint{4.677981in}{0.626817in}}%
\pgfpathlineto{\pgfqpoint{4.677981in}{0.623868in}}%
\pgfpathclose%
\pgfusepath{fill}%
\end{pgfscope}%
\begin{pgfscope}%
\pgfsetbuttcap%
\pgfsetroundjoin%
\definecolor{currentfill}{rgb}{0.000000,0.000000,0.000000}%
\pgfsetfillcolor{currentfill}%
\pgfsetlinewidth{0.803000pt}%
\definecolor{currentstroke}{rgb}{0.000000,0.000000,0.000000}%
\pgfsetstrokecolor{currentstroke}%
\pgfsetdash{}{0pt}%
\pgfsys@defobject{currentmarker}{\pgfqpoint{0.000000in}{-0.048611in}}{\pgfqpoint{0.000000in}{0.000000in}}{%
\pgfpathmoveto{\pgfqpoint{0.000000in}{0.000000in}}%
\pgfpathlineto{\pgfqpoint{0.000000in}{-0.048611in}}%
\pgfusepath{stroke,fill}%
}%
\begin{pgfscope}%
\pgfsys@transformshift{1.215000in}{2.010000in}%
\pgfsys@useobject{currentmarker}{}%
\end{pgfscope}%
\end{pgfscope}%
\begin{pgfscope}%
\definecolor{textcolor}{rgb}{0.000000,0.000000,0.000000}%
\pgfsetstrokecolor{textcolor}%
\pgfsetfillcolor{textcolor}%
\pgftext[x=1.215000in,y=1.912778in,,top]{\color{textcolor}\sffamily\fontsize{10.000000}{12.000000}\selectfont −4}%
\end{pgfscope}%
\begin{pgfscope}%
\pgfsetbuttcap%
\pgfsetroundjoin%
\definecolor{currentfill}{rgb}{0.000000,0.000000,0.000000}%
\pgfsetfillcolor{currentfill}%
\pgfsetlinewidth{0.803000pt}%
\definecolor{currentstroke}{rgb}{0.000000,0.000000,0.000000}%
\pgfsetstrokecolor{currentstroke}%
\pgfsetdash{}{0pt}%
\pgfsys@defobject{currentmarker}{\pgfqpoint{0.000000in}{-0.048611in}}{\pgfqpoint{0.000000in}{0.000000in}}{%
\pgfpathmoveto{\pgfqpoint{0.000000in}{0.000000in}}%
\pgfpathlineto{\pgfqpoint{0.000000in}{-0.048611in}}%
\pgfusepath{stroke,fill}%
}%
\begin{pgfscope}%
\pgfsys@transformshift{2.145000in}{2.010000in}%
\pgfsys@useobject{currentmarker}{}%
\end{pgfscope}%
\end{pgfscope}%
\begin{pgfscope}%
\definecolor{textcolor}{rgb}{0.000000,0.000000,0.000000}%
\pgfsetstrokecolor{textcolor}%
\pgfsetfillcolor{textcolor}%
\pgftext[x=2.145000in,y=1.912778in,,top]{\color{textcolor}\sffamily\fontsize{10.000000}{12.000000}\selectfont −2}%
\end{pgfscope}%
\begin{pgfscope}%
\pgfsetbuttcap%
\pgfsetroundjoin%
\definecolor{currentfill}{rgb}{0.000000,0.000000,0.000000}%
\pgfsetfillcolor{currentfill}%
\pgfsetlinewidth{0.803000pt}%
\definecolor{currentstroke}{rgb}{0.000000,0.000000,0.000000}%
\pgfsetstrokecolor{currentstroke}%
\pgfsetdash{}{0pt}%
\pgfsys@defobject{currentmarker}{\pgfqpoint{0.000000in}{-0.048611in}}{\pgfqpoint{0.000000in}{0.000000in}}{%
\pgfpathmoveto{\pgfqpoint{0.000000in}{0.000000in}}%
\pgfpathlineto{\pgfqpoint{0.000000in}{-0.048611in}}%
\pgfusepath{stroke,fill}%
}%
\begin{pgfscope}%
\pgfsys@transformshift{3.075000in}{2.010000in}%
\pgfsys@useobject{currentmarker}{}%
\end{pgfscope}%
\end{pgfscope}%
\begin{pgfscope}%
\definecolor{textcolor}{rgb}{0.000000,0.000000,0.000000}%
\pgfsetstrokecolor{textcolor}%
\pgfsetfillcolor{textcolor}%
\pgftext[x=3.075000in,y=1.912778in,,top]{\color{textcolor}\sffamily\fontsize{10.000000}{12.000000}\selectfont 0}%
\end{pgfscope}%
\begin{pgfscope}%
\pgfsetbuttcap%
\pgfsetroundjoin%
\definecolor{currentfill}{rgb}{0.000000,0.000000,0.000000}%
\pgfsetfillcolor{currentfill}%
\pgfsetlinewidth{0.803000pt}%
\definecolor{currentstroke}{rgb}{0.000000,0.000000,0.000000}%
\pgfsetstrokecolor{currentstroke}%
\pgfsetdash{}{0pt}%
\pgfsys@defobject{currentmarker}{\pgfqpoint{0.000000in}{-0.048611in}}{\pgfqpoint{0.000000in}{0.000000in}}{%
\pgfpathmoveto{\pgfqpoint{0.000000in}{0.000000in}}%
\pgfpathlineto{\pgfqpoint{0.000000in}{-0.048611in}}%
\pgfusepath{stroke,fill}%
}%
\begin{pgfscope}%
\pgfsys@transformshift{4.005000in}{2.010000in}%
\pgfsys@useobject{currentmarker}{}%
\end{pgfscope}%
\end{pgfscope}%
\begin{pgfscope}%
\definecolor{textcolor}{rgb}{0.000000,0.000000,0.000000}%
\pgfsetstrokecolor{textcolor}%
\pgfsetfillcolor{textcolor}%
\pgftext[x=4.005000in,y=1.912778in,,top]{\color{textcolor}\sffamily\fontsize{10.000000}{12.000000}\selectfont 2}%
\end{pgfscope}%
\begin{pgfscope}%
\pgfsetbuttcap%
\pgfsetroundjoin%
\definecolor{currentfill}{rgb}{0.000000,0.000000,0.000000}%
\pgfsetfillcolor{currentfill}%
\pgfsetlinewidth{0.803000pt}%
\definecolor{currentstroke}{rgb}{0.000000,0.000000,0.000000}%
\pgfsetstrokecolor{currentstroke}%
\pgfsetdash{}{0pt}%
\pgfsys@defobject{currentmarker}{\pgfqpoint{0.000000in}{-0.048611in}}{\pgfqpoint{0.000000in}{0.000000in}}{%
\pgfpathmoveto{\pgfqpoint{0.000000in}{0.000000in}}%
\pgfpathlineto{\pgfqpoint{0.000000in}{-0.048611in}}%
\pgfusepath{stroke,fill}%
}%
\begin{pgfscope}%
\pgfsys@transformshift{4.935000in}{2.010000in}%
\pgfsys@useobject{currentmarker}{}%
\end{pgfscope}%
\end{pgfscope}%
\begin{pgfscope}%
\definecolor{textcolor}{rgb}{0.000000,0.000000,0.000000}%
\pgfsetstrokecolor{textcolor}%
\pgfsetfillcolor{textcolor}%
\pgftext[x=4.935000in,y=1.912778in,,top]{\color{textcolor}\sffamily\fontsize{10.000000}{12.000000}\selectfont 4}%
\end{pgfscope}%
\begin{pgfscope}%
\definecolor{textcolor}{rgb}{0.000000,0.000000,0.000000}%
\pgfsetstrokecolor{textcolor}%
\pgfsetfillcolor{textcolor}%
\pgftext[x=5.400000in,y=1.722809in,,top]{\color{textcolor}\sffamily\fontsize{10.000000}{12.000000}\selectfont x}%
\end{pgfscope}%
\begin{pgfscope}%
\pgfsetbuttcap%
\pgfsetroundjoin%
\definecolor{currentfill}{rgb}{0.000000,0.000000,0.000000}%
\pgfsetfillcolor{currentfill}%
\pgfsetlinewidth{0.803000pt}%
\definecolor{currentstroke}{rgb}{0.000000,0.000000,0.000000}%
\pgfsetstrokecolor{currentstroke}%
\pgfsetdash{}{0pt}%
\pgfsys@defobject{currentmarker}{\pgfqpoint{-0.048611in}{0.000000in}}{\pgfqpoint{-0.000000in}{0.000000in}}{%
\pgfpathmoveto{\pgfqpoint{-0.000000in}{0.000000in}}%
\pgfpathlineto{\pgfqpoint{-0.048611in}{0.000000in}}%
\pgfusepath{stroke,fill}%
}%
\begin{pgfscope}%
\pgfsys@transformshift{3.075000in}{0.802000in}%
\pgfsys@useobject{currentmarker}{}%
\end{pgfscope}%
\end{pgfscope}%
\begin{pgfscope}%
\definecolor{textcolor}{rgb}{0.000000,0.000000,0.000000}%
\pgfsetstrokecolor{textcolor}%
\pgfsetfillcolor{textcolor}%
\pgftext[x=2.773039in, y=0.749238in, left, base]{\color{textcolor}\sffamily\fontsize{10.000000}{12.000000}\selectfont −4}%
\end{pgfscope}%
\begin{pgfscope}%
\pgfsetbuttcap%
\pgfsetroundjoin%
\definecolor{currentfill}{rgb}{0.000000,0.000000,0.000000}%
\pgfsetfillcolor{currentfill}%
\pgfsetlinewidth{0.803000pt}%
\definecolor{currentstroke}{rgb}{0.000000,0.000000,0.000000}%
\pgfsetstrokecolor{currentstroke}%
\pgfsetdash{}{0pt}%
\pgfsys@defobject{currentmarker}{\pgfqpoint{-0.048611in}{0.000000in}}{\pgfqpoint{-0.000000in}{0.000000in}}{%
\pgfpathmoveto{\pgfqpoint{-0.000000in}{0.000000in}}%
\pgfpathlineto{\pgfqpoint{-0.048611in}{0.000000in}}%
\pgfusepath{stroke,fill}%
}%
\begin{pgfscope}%
\pgfsys@transformshift{3.075000in}{1.406000in}%
\pgfsys@useobject{currentmarker}{}%
\end{pgfscope}%
\end{pgfscope}%
\begin{pgfscope}%
\definecolor{textcolor}{rgb}{0.000000,0.000000,0.000000}%
\pgfsetstrokecolor{textcolor}%
\pgfsetfillcolor{textcolor}%
\pgftext[x=2.773039in, y=1.353238in, left, base]{\color{textcolor}\sffamily\fontsize{10.000000}{12.000000}\selectfont −2}%
\end{pgfscope}%
\begin{pgfscope}%
\pgfsetbuttcap%
\pgfsetroundjoin%
\definecolor{currentfill}{rgb}{0.000000,0.000000,0.000000}%
\pgfsetfillcolor{currentfill}%
\pgfsetlinewidth{0.803000pt}%
\definecolor{currentstroke}{rgb}{0.000000,0.000000,0.000000}%
\pgfsetstrokecolor{currentstroke}%
\pgfsetdash{}{0pt}%
\pgfsys@defobject{currentmarker}{\pgfqpoint{-0.048611in}{0.000000in}}{\pgfqpoint{-0.000000in}{0.000000in}}{%
\pgfpathmoveto{\pgfqpoint{-0.000000in}{0.000000in}}%
\pgfpathlineto{\pgfqpoint{-0.048611in}{0.000000in}}%
\pgfusepath{stroke,fill}%
}%
\begin{pgfscope}%
\pgfsys@transformshift{3.075000in}{2.010000in}%
\pgfsys@useobject{currentmarker}{}%
\end{pgfscope}%
\end{pgfscope}%
\begin{pgfscope}%
\definecolor{textcolor}{rgb}{0.000000,0.000000,0.000000}%
\pgfsetstrokecolor{textcolor}%
\pgfsetfillcolor{textcolor}%
\pgftext[x=2.889413in, y=1.957238in, left, base]{\color{textcolor}\sffamily\fontsize{10.000000}{12.000000}\selectfont 0}%
\end{pgfscope}%
\begin{pgfscope}%
\pgfsetbuttcap%
\pgfsetroundjoin%
\definecolor{currentfill}{rgb}{0.000000,0.000000,0.000000}%
\pgfsetfillcolor{currentfill}%
\pgfsetlinewidth{0.803000pt}%
\definecolor{currentstroke}{rgb}{0.000000,0.000000,0.000000}%
\pgfsetstrokecolor{currentstroke}%
\pgfsetdash{}{0pt}%
\pgfsys@defobject{currentmarker}{\pgfqpoint{-0.048611in}{0.000000in}}{\pgfqpoint{-0.000000in}{0.000000in}}{%
\pgfpathmoveto{\pgfqpoint{-0.000000in}{0.000000in}}%
\pgfpathlineto{\pgfqpoint{-0.048611in}{0.000000in}}%
\pgfusepath{stroke,fill}%
}%
\begin{pgfscope}%
\pgfsys@transformshift{3.075000in}{2.614000in}%
\pgfsys@useobject{currentmarker}{}%
\end{pgfscope}%
\end{pgfscope}%
\begin{pgfscope}%
\definecolor{textcolor}{rgb}{0.000000,0.000000,0.000000}%
\pgfsetstrokecolor{textcolor}%
\pgfsetfillcolor{textcolor}%
\pgftext[x=2.889413in, y=2.561238in, left, base]{\color{textcolor}\sffamily\fontsize{10.000000}{12.000000}\selectfont 2}%
\end{pgfscope}%
\begin{pgfscope}%
\pgfsetbuttcap%
\pgfsetroundjoin%
\definecolor{currentfill}{rgb}{0.000000,0.000000,0.000000}%
\pgfsetfillcolor{currentfill}%
\pgfsetlinewidth{0.803000pt}%
\definecolor{currentstroke}{rgb}{0.000000,0.000000,0.000000}%
\pgfsetstrokecolor{currentstroke}%
\pgfsetdash{}{0pt}%
\pgfsys@defobject{currentmarker}{\pgfqpoint{-0.048611in}{0.000000in}}{\pgfqpoint{-0.000000in}{0.000000in}}{%
\pgfpathmoveto{\pgfqpoint{-0.000000in}{0.000000in}}%
\pgfpathlineto{\pgfqpoint{-0.048611in}{0.000000in}}%
\pgfusepath{stroke,fill}%
}%
\begin{pgfscope}%
\pgfsys@transformshift{3.075000in}{3.218000in}%
\pgfsys@useobject{currentmarker}{}%
\end{pgfscope}%
\end{pgfscope}%
\begin{pgfscope}%
\definecolor{textcolor}{rgb}{0.000000,0.000000,0.000000}%
\pgfsetstrokecolor{textcolor}%
\pgfsetfillcolor{textcolor}%
\pgftext[x=2.889413in, y=3.165238in, left, base]{\color{textcolor}\sffamily\fontsize{10.000000}{12.000000}\selectfont 4}%
\end{pgfscope}%
\begin{pgfscope}%
\definecolor{textcolor}{rgb}{0.000000,0.000000,0.000000}%
\pgfsetstrokecolor{textcolor}%
\pgfsetfillcolor{textcolor}%
\pgftext[x=2.717483in,y=3.520000in,,bottom,rotate=90.000000]{\color{textcolor}\sffamily\fontsize{10.000000}{12.000000}\selectfont y}%
\end{pgfscope}%
\begin{pgfscope}%
\pgfsetrectcap%
\pgfsetmiterjoin%
\pgfsetlinewidth{0.803000pt}%
\definecolor{currentstroke}{rgb}{0.000000,0.000000,0.000000}%
\pgfsetstrokecolor{currentstroke}%
\pgfsetdash{}{0pt}%
\pgfpathmoveto{\pgfqpoint{3.075000in}{0.500000in}}%
\pgfpathlineto{\pgfqpoint{3.075000in}{3.520000in}}%
\pgfusepath{stroke}%
\end{pgfscope}%
\begin{pgfscope}%
\pgfsetrectcap%
\pgfsetmiterjoin%
\pgfsetlinewidth{0.000000pt}%
\definecolor{currentstroke}{rgb}{0.000000,0.000000,0.000000}%
\pgfsetstrokecolor{currentstroke}%
\pgfsetstrokeopacity{0.000000}%
\pgfsetdash{}{0pt}%
\pgfpathmoveto{\pgfqpoint{5.400000in}{0.500000in}}%
\pgfpathlineto{\pgfqpoint{5.400000in}{3.520000in}}%
\pgfusepath{}%
\end{pgfscope}%
\begin{pgfscope}%
\pgfsetrectcap%
\pgfsetmiterjoin%
\pgfsetlinewidth{0.803000pt}%
\definecolor{currentstroke}{rgb}{0.000000,0.000000,0.000000}%
\pgfsetstrokecolor{currentstroke}%
\pgfsetdash{}{0pt}%
\pgfpathmoveto{\pgfqpoint{0.750000in}{2.010000in}}%
\pgfpathlineto{\pgfqpoint{5.400000in}{2.010000in}}%
\pgfusepath{stroke}%
\end{pgfscope}%
\begin{pgfscope}%
\pgfsetrectcap%
\pgfsetmiterjoin%
\pgfsetlinewidth{0.000000pt}%
\definecolor{currentstroke}{rgb}{0.000000,0.000000,0.000000}%
\pgfsetstrokecolor{currentstroke}%
\pgfsetstrokeopacity{0.000000}%
\pgfsetdash{}{0pt}%
\pgfpathmoveto{\pgfqpoint{0.750000in}{3.520000in}}%
\pgfpathlineto{\pgfqpoint{5.400000in}{3.520000in}}%
\pgfusepath{}%
\end{pgfscope}%
\end{pgfpicture}%
\makeatother%
\endgroup%
} \end{solution}
\part[] $\left\{ {\begin{matrix}
   {x + y \leqslant 5}  \\ 
   {x + 3y \geqslant 9}  \\ 
   {y \geqslant 0}  \\ 
   {x \geqslant 0}  \\ 
   {y \geqslant 0}  \\ 

 \end{matrix} } \right.$
\begin{solution} \scalebox{.6}{%% Creator: Matplotlib, PGF backend
%%
%% To include the figure in your LaTeX document, write
%%   \input{<filename>.pgf}
%%
%% Make sure the required packages are loaded in your preamble
%%   \usepackage{pgf}
%%
%% and, on pdftex
%%   \usepackage[utf8]{inputenc}\DeclareUnicodeCharacter{2212}{-}
%%
%% or, on luatex and xetex
%%   \usepackage{unicode-math}
%%
%% Figures using additional raster images can only be included by \input if
%% they are in the same directory as the main LaTeX file. For loading figures
%% from other directories you can use the `import` package
%%   \usepackage{import}
%%
%% and then include the figures with
%%   \import{<path to file>}{<filename>.pgf}
%%
%% Matplotlib used the following preamble
%%   \usepackage{fontspec}
%%   \setmainfont{DejaVuSerif.ttf}[Path=/home/hp/Mis_aplicaciones/anaconda3/lib/python3.6/site-packages/matplotlib/mpl-data/fonts/ttf/]
%%   \setsansfont{DejaVuSans.ttf}[Path=/home/hp/Mis_aplicaciones/anaconda3/lib/python3.6/site-packages/matplotlib/mpl-data/fonts/ttf/]
%%   \setmonofont{DejaVuSansMono.ttf}[Path=/home/hp/Mis_aplicaciones/anaconda3/lib/python3.6/site-packages/matplotlib/mpl-data/fonts/ttf/]
%%
\begingroup%
\makeatletter%
\begin{pgfpicture}%
\pgfpathrectangle{\pgfpointorigin}{\pgfqpoint{6.000000in}{4.000000in}}%
\pgfusepath{use as bounding box, clip}%
\begin{pgfscope}%
\pgfsetbuttcap%
\pgfsetmiterjoin%
\pgfsetlinewidth{0.000000pt}%
\definecolor{currentstroke}{rgb}{1.000000,1.000000,1.000000}%
\pgfsetstrokecolor{currentstroke}%
\pgfsetstrokeopacity{0.000000}%
\pgfsetdash{}{0pt}%
\pgfpathmoveto{\pgfqpoint{0.000000in}{0.000000in}}%
\pgfpathlineto{\pgfqpoint{6.000000in}{0.000000in}}%
\pgfpathlineto{\pgfqpoint{6.000000in}{4.000000in}}%
\pgfpathlineto{\pgfqpoint{0.000000in}{4.000000in}}%
\pgfpathclose%
\pgfusepath{}%
\end{pgfscope}%
\begin{pgfscope}%
\pgfsetbuttcap%
\pgfsetmiterjoin%
\definecolor{currentfill}{rgb}{1.000000,1.000000,1.000000}%
\pgfsetfillcolor{currentfill}%
\pgfsetlinewidth{0.000000pt}%
\definecolor{currentstroke}{rgb}{0.000000,0.000000,0.000000}%
\pgfsetstrokecolor{currentstroke}%
\pgfsetstrokeopacity{0.000000}%
\pgfsetdash{}{0pt}%
\pgfpathmoveto{\pgfqpoint{0.750000in}{0.500000in}}%
\pgfpathlineto{\pgfqpoint{5.400000in}{0.500000in}}%
\pgfpathlineto{\pgfqpoint{5.400000in}{3.520000in}}%
\pgfpathlineto{\pgfqpoint{0.750000in}{3.520000in}}%
\pgfpathclose%
\pgfusepath{fill}%
\end{pgfscope}%
\begin{pgfscope}%
\pgfpathrectangle{\pgfqpoint{0.750000in}{0.500000in}}{\pgfqpoint{4.650000in}{3.020000in}}%
\pgfusepath{clip}%
\pgfsetbuttcap%
\pgfsetmiterjoin%
\definecolor{currentfill}{rgb}{0.000000,0.000000,1.000000}%
\pgfsetfillcolor{currentfill}%
\pgfsetlinewidth{0.000000pt}%
\definecolor{currentstroke}{rgb}{0.000000,0.000000,0.000000}%
\pgfsetstrokecolor{currentstroke}%
\pgfsetstrokeopacity{0.000000}%
\pgfsetdash{}{0pt}%
\pgfpathmoveto{\pgfqpoint{3.075001in}{2.953752in}}%
\pgfpathlineto{\pgfqpoint{3.075001in}{3.048125in}}%
\pgfpathlineto{\pgfqpoint{3.220314in}{3.048125in}}%
\pgfpathlineto{\pgfqpoint{3.220314in}{2.953752in}}%
\pgfpathmoveto{\pgfqpoint{3.075001in}{3.048125in}}%
\pgfpathlineto{\pgfqpoint{3.075001in}{3.048125in}}%
\pgfpathlineto{\pgfqpoint{3.075001in}{3.142499in}}%
\pgfpathlineto{\pgfqpoint{3.220314in}{3.142499in}}%
\pgfpathlineto{\pgfqpoint{3.220314in}{3.048125in}}%
\pgfpathmoveto{\pgfqpoint{3.075001in}{3.142499in}}%
\pgfpathlineto{\pgfqpoint{3.075001in}{3.142499in}}%
\pgfpathlineto{\pgfqpoint{3.075001in}{3.236874in}}%
\pgfpathlineto{\pgfqpoint{3.220314in}{3.236874in}}%
\pgfpathlineto{\pgfqpoint{3.220314in}{3.142499in}}%
\pgfpathmoveto{\pgfqpoint{3.075001in}{3.236874in}}%
\pgfpathlineto{\pgfqpoint{3.075001in}{3.236874in}}%
\pgfpathlineto{\pgfqpoint{3.075001in}{3.331248in}}%
\pgfpathlineto{\pgfqpoint{3.220314in}{3.331248in}}%
\pgfpathlineto{\pgfqpoint{3.220314in}{3.236874in}}%
\pgfpathmoveto{\pgfqpoint{3.220314in}{2.953752in}}%
\pgfpathlineto{\pgfqpoint{3.220314in}{2.953752in}}%
\pgfpathlineto{\pgfqpoint{3.220314in}{3.048125in}}%
\pgfpathlineto{\pgfqpoint{3.365621in}{3.048125in}}%
\pgfpathlineto{\pgfqpoint{3.365621in}{2.953752in}}%
\pgfpathmoveto{\pgfqpoint{3.220314in}{3.048125in}}%
\pgfpathlineto{\pgfqpoint{3.220314in}{3.048125in}}%
\pgfpathlineto{\pgfqpoint{3.220314in}{3.142499in}}%
\pgfpathlineto{\pgfqpoint{3.365621in}{3.142499in}}%
\pgfpathlineto{\pgfqpoint{3.365621in}{3.048125in}}%
\pgfpathmoveto{\pgfqpoint{3.220314in}{3.142499in}}%
\pgfpathlineto{\pgfqpoint{3.220314in}{3.142499in}}%
\pgfpathlineto{\pgfqpoint{3.220314in}{3.236874in}}%
\pgfpathlineto{\pgfqpoint{3.365621in}{3.236874in}}%
\pgfpathlineto{\pgfqpoint{3.365621in}{3.142499in}}%
\pgfpathmoveto{\pgfqpoint{3.220314in}{3.236874in}}%
\pgfpathlineto{\pgfqpoint{3.220314in}{3.236874in}}%
\pgfpathlineto{\pgfqpoint{3.220314in}{3.331248in}}%
\pgfpathlineto{\pgfqpoint{3.365621in}{3.331248in}}%
\pgfpathlineto{\pgfqpoint{3.365621in}{3.236874in}}%
\pgfpathmoveto{\pgfqpoint{3.365621in}{2.859375in}}%
\pgfpathlineto{\pgfqpoint{3.365621in}{2.859375in}}%
\pgfpathlineto{\pgfqpoint{3.365621in}{2.953752in}}%
\pgfpathlineto{\pgfqpoint{3.510942in}{2.953752in}}%
\pgfpathlineto{\pgfqpoint{3.510942in}{2.859375in}}%
\pgfpathmoveto{\pgfqpoint{3.365621in}{2.953752in}}%
\pgfpathlineto{\pgfqpoint{3.365621in}{2.953752in}}%
\pgfpathlineto{\pgfqpoint{3.365621in}{3.048125in}}%
\pgfpathlineto{\pgfqpoint{3.510942in}{3.048125in}}%
\pgfpathlineto{\pgfqpoint{3.510942in}{2.953752in}}%
\pgfpathmoveto{\pgfqpoint{3.365621in}{3.048125in}}%
\pgfpathlineto{\pgfqpoint{3.365621in}{3.048125in}}%
\pgfpathlineto{\pgfqpoint{3.365621in}{3.142499in}}%
\pgfpathlineto{\pgfqpoint{3.510942in}{3.142499in}}%
\pgfpathlineto{\pgfqpoint{3.510942in}{3.048125in}}%
\pgfpathmoveto{\pgfqpoint{3.510942in}{2.859375in}}%
\pgfpathlineto{\pgfqpoint{3.510942in}{2.859375in}}%
\pgfpathlineto{\pgfqpoint{3.510942in}{2.953752in}}%
\pgfpathlineto{\pgfqpoint{3.656252in}{2.953752in}}%
\pgfpathlineto{\pgfqpoint{3.656252in}{2.859375in}}%
\pgfpathmoveto{\pgfqpoint{3.510942in}{2.953752in}}%
\pgfpathlineto{\pgfqpoint{3.510942in}{2.953752in}}%
\pgfpathlineto{\pgfqpoint{3.510942in}{3.048125in}}%
\pgfpathlineto{\pgfqpoint{3.656252in}{3.048125in}}%
\pgfpathlineto{\pgfqpoint{3.656252in}{2.953752in}}%
\pgfpathmoveto{\pgfqpoint{3.656252in}{2.859375in}}%
\pgfpathlineto{\pgfqpoint{3.656252in}{2.859375in}}%
\pgfpathlineto{\pgfqpoint{3.656252in}{2.953752in}}%
\pgfpathlineto{\pgfqpoint{3.801560in}{2.953752in}}%
\pgfpathlineto{\pgfqpoint{3.801560in}{2.859375in}}%
\pgfpathmoveto{\pgfqpoint{3.656252in}{2.953752in}}%
\pgfpathlineto{\pgfqpoint{3.656252in}{2.953752in}}%
\pgfpathlineto{\pgfqpoint{3.656252in}{3.048125in}}%
\pgfpathlineto{\pgfqpoint{3.801560in}{3.048125in}}%
\pgfpathlineto{\pgfqpoint{3.801560in}{2.953752in}}%
\pgfpathmoveto{\pgfqpoint{3.801560in}{2.764998in}}%
\pgfpathlineto{\pgfqpoint{3.801560in}{2.764998in}}%
\pgfpathlineto{\pgfqpoint{3.801560in}{2.859375in}}%
\pgfpathlineto{\pgfqpoint{3.946873in}{2.859375in}}%
\pgfpathlineto{\pgfqpoint{3.946873in}{2.764998in}}%
\pgfpathmoveto{\pgfqpoint{3.147657in}{2.906564in}}%
\pgfpathlineto{\pgfqpoint{3.147657in}{2.906564in}}%
\pgfpathlineto{\pgfqpoint{3.147657in}{2.953752in}}%
\pgfpathlineto{\pgfqpoint{3.220314in}{2.953752in}}%
\pgfpathlineto{\pgfqpoint{3.220314in}{2.906564in}}%
\pgfpathmoveto{\pgfqpoint{3.075001in}{3.331248in}}%
\pgfpathlineto{\pgfqpoint{3.075001in}{3.331248in}}%
\pgfpathlineto{\pgfqpoint{3.075001in}{3.378436in}}%
\pgfpathlineto{\pgfqpoint{3.147657in}{3.378436in}}%
\pgfpathlineto{\pgfqpoint{3.147657in}{3.331248in}}%
\pgfpathmoveto{\pgfqpoint{3.075001in}{3.378436in}}%
\pgfpathlineto{\pgfqpoint{3.075001in}{3.378436in}}%
\pgfpathlineto{\pgfqpoint{3.075001in}{3.425625in}}%
\pgfpathlineto{\pgfqpoint{3.147657in}{3.425625in}}%
\pgfpathlineto{\pgfqpoint{3.147657in}{3.378436in}}%
\pgfpathmoveto{\pgfqpoint{3.147657in}{3.331248in}}%
\pgfpathlineto{\pgfqpoint{3.147657in}{3.331248in}}%
\pgfpathlineto{\pgfqpoint{3.147657in}{3.378436in}}%
\pgfpathlineto{\pgfqpoint{3.220314in}{3.378436in}}%
\pgfpathlineto{\pgfqpoint{3.220314in}{3.331248in}}%
\pgfpathmoveto{\pgfqpoint{3.220314in}{2.906564in}}%
\pgfpathlineto{\pgfqpoint{3.220314in}{2.906564in}}%
\pgfpathlineto{\pgfqpoint{3.220314in}{2.953752in}}%
\pgfpathlineto{\pgfqpoint{3.292968in}{2.953752in}}%
\pgfpathlineto{\pgfqpoint{3.292968in}{2.906564in}}%
\pgfpathmoveto{\pgfqpoint{3.292968in}{2.906564in}}%
\pgfpathlineto{\pgfqpoint{3.292968in}{2.906564in}}%
\pgfpathlineto{\pgfqpoint{3.292968in}{2.953752in}}%
\pgfpathlineto{\pgfqpoint{3.365621in}{2.953752in}}%
\pgfpathlineto{\pgfqpoint{3.365621in}{2.906564in}}%
\pgfpathmoveto{\pgfqpoint{3.220314in}{3.331248in}}%
\pgfpathlineto{\pgfqpoint{3.220314in}{3.331248in}}%
\pgfpathlineto{\pgfqpoint{3.220314in}{3.378436in}}%
\pgfpathlineto{\pgfqpoint{3.292968in}{3.378436in}}%
\pgfpathlineto{\pgfqpoint{3.292968in}{3.331248in}}%
\pgfpathmoveto{\pgfqpoint{3.365621in}{3.142499in}}%
\pgfpathlineto{\pgfqpoint{3.365621in}{3.142499in}}%
\pgfpathlineto{\pgfqpoint{3.365621in}{3.189687in}}%
\pgfpathlineto{\pgfqpoint{3.438281in}{3.189687in}}%
\pgfpathlineto{\pgfqpoint{3.438281in}{3.142499in}}%
\pgfpathmoveto{\pgfqpoint{3.365621in}{3.189687in}}%
\pgfpathlineto{\pgfqpoint{3.365621in}{3.189687in}}%
\pgfpathlineto{\pgfqpoint{3.365621in}{3.236874in}}%
\pgfpathlineto{\pgfqpoint{3.438281in}{3.236874in}}%
\pgfpathlineto{\pgfqpoint{3.438281in}{3.189687in}}%
\pgfpathmoveto{\pgfqpoint{3.438281in}{3.142499in}}%
\pgfpathlineto{\pgfqpoint{3.438281in}{3.142499in}}%
\pgfpathlineto{\pgfqpoint{3.438281in}{3.189687in}}%
\pgfpathlineto{\pgfqpoint{3.510942in}{3.189687in}}%
\pgfpathlineto{\pgfqpoint{3.510942in}{3.142499in}}%
\pgfpathmoveto{\pgfqpoint{3.365621in}{3.236874in}}%
\pgfpathlineto{\pgfqpoint{3.365621in}{3.236874in}}%
\pgfpathlineto{\pgfqpoint{3.365621in}{3.284061in}}%
\pgfpathlineto{\pgfqpoint{3.438281in}{3.284061in}}%
\pgfpathlineto{\pgfqpoint{3.438281in}{3.236874in}}%
\pgfpathmoveto{\pgfqpoint{3.583597in}{2.812187in}}%
\pgfpathlineto{\pgfqpoint{3.583597in}{2.812187in}}%
\pgfpathlineto{\pgfqpoint{3.583597in}{2.859375in}}%
\pgfpathlineto{\pgfqpoint{3.656252in}{2.859375in}}%
\pgfpathlineto{\pgfqpoint{3.656252in}{2.812187in}}%
\pgfpathmoveto{\pgfqpoint{3.510942in}{3.048125in}}%
\pgfpathlineto{\pgfqpoint{3.510942in}{3.048125in}}%
\pgfpathlineto{\pgfqpoint{3.510942in}{3.095312in}}%
\pgfpathlineto{\pgfqpoint{3.583597in}{3.095312in}}%
\pgfpathlineto{\pgfqpoint{3.583597in}{3.048125in}}%
\pgfpathmoveto{\pgfqpoint{3.510942in}{3.095312in}}%
\pgfpathlineto{\pgfqpoint{3.510942in}{3.095312in}}%
\pgfpathlineto{\pgfqpoint{3.510942in}{3.142499in}}%
\pgfpathlineto{\pgfqpoint{3.583597in}{3.142499in}}%
\pgfpathlineto{\pgfqpoint{3.583597in}{3.095312in}}%
\pgfpathmoveto{\pgfqpoint{3.583597in}{3.048125in}}%
\pgfpathlineto{\pgfqpoint{3.583597in}{3.048125in}}%
\pgfpathlineto{\pgfqpoint{3.583597in}{3.095312in}}%
\pgfpathlineto{\pgfqpoint{3.656252in}{3.095312in}}%
\pgfpathlineto{\pgfqpoint{3.656252in}{3.048125in}}%
\pgfpathmoveto{\pgfqpoint{3.656252in}{2.812187in}}%
\pgfpathlineto{\pgfqpoint{3.656252in}{2.812187in}}%
\pgfpathlineto{\pgfqpoint{3.656252in}{2.859375in}}%
\pgfpathlineto{\pgfqpoint{3.728906in}{2.859375in}}%
\pgfpathlineto{\pgfqpoint{3.728906in}{2.812187in}}%
\pgfpathmoveto{\pgfqpoint{3.728906in}{2.812187in}}%
\pgfpathlineto{\pgfqpoint{3.728906in}{2.812187in}}%
\pgfpathlineto{\pgfqpoint{3.728906in}{2.859375in}}%
\pgfpathlineto{\pgfqpoint{3.801560in}{2.859375in}}%
\pgfpathlineto{\pgfqpoint{3.801560in}{2.812187in}}%
\pgfpathmoveto{\pgfqpoint{3.656252in}{3.048125in}}%
\pgfpathlineto{\pgfqpoint{3.656252in}{3.048125in}}%
\pgfpathlineto{\pgfqpoint{3.656252in}{3.095312in}}%
\pgfpathlineto{\pgfqpoint{3.728906in}{3.095312in}}%
\pgfpathlineto{\pgfqpoint{3.728906in}{3.048125in}}%
\pgfpathmoveto{\pgfqpoint{3.801560in}{2.859375in}}%
\pgfpathlineto{\pgfqpoint{3.801560in}{2.859375in}}%
\pgfpathlineto{\pgfqpoint{3.801560in}{2.906564in}}%
\pgfpathlineto{\pgfqpoint{3.874216in}{2.906564in}}%
\pgfpathlineto{\pgfqpoint{3.874216in}{2.859375in}}%
\pgfpathmoveto{\pgfqpoint{3.801560in}{2.906564in}}%
\pgfpathlineto{\pgfqpoint{3.801560in}{2.906564in}}%
\pgfpathlineto{\pgfqpoint{3.801560in}{2.953752in}}%
\pgfpathlineto{\pgfqpoint{3.874216in}{2.953752in}}%
\pgfpathlineto{\pgfqpoint{3.874216in}{2.906564in}}%
\pgfpathmoveto{\pgfqpoint{3.874216in}{2.859375in}}%
\pgfpathlineto{\pgfqpoint{3.874216in}{2.859375in}}%
\pgfpathlineto{\pgfqpoint{3.874216in}{2.906564in}}%
\pgfpathlineto{\pgfqpoint{3.946873in}{2.906564in}}%
\pgfpathlineto{\pgfqpoint{3.946873in}{2.859375in}}%
\pgfpathmoveto{\pgfqpoint{3.801560in}{2.953752in}}%
\pgfpathlineto{\pgfqpoint{3.801560in}{2.953752in}}%
\pgfpathlineto{\pgfqpoint{3.801560in}{3.000938in}}%
\pgfpathlineto{\pgfqpoint{3.874216in}{3.000938in}}%
\pgfpathlineto{\pgfqpoint{3.874216in}{2.953752in}}%
\pgfpathmoveto{\pgfqpoint{4.019531in}{2.717811in}}%
\pgfpathlineto{\pgfqpoint{4.019531in}{2.717811in}}%
\pgfpathlineto{\pgfqpoint{4.019531in}{2.764998in}}%
\pgfpathlineto{\pgfqpoint{4.092189in}{2.764998in}}%
\pgfpathlineto{\pgfqpoint{4.092189in}{2.717811in}}%
\pgfpathmoveto{\pgfqpoint{3.946873in}{2.764998in}}%
\pgfpathlineto{\pgfqpoint{3.946873in}{2.764998in}}%
\pgfpathlineto{\pgfqpoint{3.946873in}{2.812187in}}%
\pgfpathlineto{\pgfqpoint{4.019531in}{2.812187in}}%
\pgfpathlineto{\pgfqpoint{4.019531in}{2.764998in}}%
\pgfpathmoveto{\pgfqpoint{3.946873in}{2.812187in}}%
\pgfpathlineto{\pgfqpoint{3.946873in}{2.812187in}}%
\pgfpathlineto{\pgfqpoint{3.946873in}{2.859375in}}%
\pgfpathlineto{\pgfqpoint{4.019531in}{2.859375in}}%
\pgfpathlineto{\pgfqpoint{4.019531in}{2.812187in}}%
\pgfpathmoveto{\pgfqpoint{4.019531in}{2.764998in}}%
\pgfpathlineto{\pgfqpoint{4.019531in}{2.764998in}}%
\pgfpathlineto{\pgfqpoint{4.019531in}{2.812187in}}%
\pgfpathlineto{\pgfqpoint{4.092189in}{2.812187in}}%
\pgfpathlineto{\pgfqpoint{4.092189in}{2.764998in}}%
\pgfpathmoveto{\pgfqpoint{4.092189in}{2.717811in}}%
\pgfpathlineto{\pgfqpoint{4.092189in}{2.717811in}}%
\pgfpathlineto{\pgfqpoint{4.092189in}{2.764998in}}%
\pgfpathlineto{\pgfqpoint{4.164846in}{2.764998in}}%
\pgfpathlineto{\pgfqpoint{4.164846in}{2.717811in}}%
\pgfpathmoveto{\pgfqpoint{4.237503in}{2.670624in}}%
\pgfpathlineto{\pgfqpoint{4.237503in}{2.670624in}}%
\pgfpathlineto{\pgfqpoint{4.237503in}{2.717811in}}%
\pgfpathlineto{\pgfqpoint{4.310157in}{2.717811in}}%
\pgfpathlineto{\pgfqpoint{4.310157in}{2.670624in}}%
\pgfpathmoveto{\pgfqpoint{3.075001in}{2.930158in}}%
\pgfpathlineto{\pgfqpoint{3.075001in}{2.930158in}}%
\pgfpathlineto{\pgfqpoint{3.075001in}{2.953752in}}%
\pgfpathlineto{\pgfqpoint{3.111329in}{2.953752in}}%
\pgfpathlineto{\pgfqpoint{3.111329in}{2.930158in}}%
\pgfpathmoveto{\pgfqpoint{3.111329in}{2.930158in}}%
\pgfpathlineto{\pgfqpoint{3.111329in}{2.930158in}}%
\pgfpathlineto{\pgfqpoint{3.111329in}{2.953752in}}%
\pgfpathlineto{\pgfqpoint{3.147657in}{2.953752in}}%
\pgfpathlineto{\pgfqpoint{3.147657in}{2.930158in}}%
\pgfpathmoveto{\pgfqpoint{3.147657in}{3.378436in}}%
\pgfpathlineto{\pgfqpoint{3.147657in}{3.378436in}}%
\pgfpathlineto{\pgfqpoint{3.147657in}{3.402031in}}%
\pgfpathlineto{\pgfqpoint{3.183986in}{3.402031in}}%
\pgfpathlineto{\pgfqpoint{3.183986in}{3.378436in}}%
\pgfpathmoveto{\pgfqpoint{3.147657in}{3.402031in}}%
\pgfpathlineto{\pgfqpoint{3.147657in}{3.402031in}}%
\pgfpathlineto{\pgfqpoint{3.147657in}{3.425625in}}%
\pgfpathlineto{\pgfqpoint{3.183986in}{3.425625in}}%
\pgfpathlineto{\pgfqpoint{3.183986in}{3.402031in}}%
\pgfpathmoveto{\pgfqpoint{3.183986in}{3.378436in}}%
\pgfpathlineto{\pgfqpoint{3.183986in}{3.378436in}}%
\pgfpathlineto{\pgfqpoint{3.183986in}{3.402031in}}%
\pgfpathlineto{\pgfqpoint{3.220314in}{3.402031in}}%
\pgfpathlineto{\pgfqpoint{3.220314in}{3.378436in}}%
\pgfpathmoveto{\pgfqpoint{3.075001in}{3.425625in}}%
\pgfpathlineto{\pgfqpoint{3.075001in}{3.425625in}}%
\pgfpathlineto{\pgfqpoint{3.075001in}{3.449219in}}%
\pgfpathlineto{\pgfqpoint{3.111329in}{3.449219in}}%
\pgfpathlineto{\pgfqpoint{3.111329in}{3.425625in}}%
\pgfpathmoveto{\pgfqpoint{3.075001in}{3.449219in}}%
\pgfpathlineto{\pgfqpoint{3.075001in}{3.449219in}}%
\pgfpathlineto{\pgfqpoint{3.075001in}{3.472813in}}%
\pgfpathlineto{\pgfqpoint{3.111329in}{3.472813in}}%
\pgfpathlineto{\pgfqpoint{3.111329in}{3.449219in}}%
\pgfpathmoveto{\pgfqpoint{3.111329in}{3.425625in}}%
\pgfpathlineto{\pgfqpoint{3.111329in}{3.425625in}}%
\pgfpathlineto{\pgfqpoint{3.111329in}{3.449219in}}%
\pgfpathlineto{\pgfqpoint{3.147657in}{3.449219in}}%
\pgfpathlineto{\pgfqpoint{3.147657in}{3.425625in}}%
\pgfpathmoveto{\pgfqpoint{3.256641in}{2.882970in}}%
\pgfpathlineto{\pgfqpoint{3.256641in}{2.882970in}}%
\pgfpathlineto{\pgfqpoint{3.256641in}{2.906564in}}%
\pgfpathlineto{\pgfqpoint{3.292968in}{2.906564in}}%
\pgfpathlineto{\pgfqpoint{3.292968in}{2.882970in}}%
\pgfpathmoveto{\pgfqpoint{3.292968in}{2.882970in}}%
\pgfpathlineto{\pgfqpoint{3.292968in}{2.882970in}}%
\pgfpathlineto{\pgfqpoint{3.292968in}{2.906564in}}%
\pgfpathlineto{\pgfqpoint{3.329294in}{2.906564in}}%
\pgfpathlineto{\pgfqpoint{3.329294in}{2.882970in}}%
\pgfpathmoveto{\pgfqpoint{3.329294in}{2.882970in}}%
\pgfpathlineto{\pgfqpoint{3.329294in}{2.882970in}}%
\pgfpathlineto{\pgfqpoint{3.329294in}{2.906564in}}%
\pgfpathlineto{\pgfqpoint{3.365621in}{2.906564in}}%
\pgfpathlineto{\pgfqpoint{3.365621in}{2.882970in}}%
\pgfpathmoveto{\pgfqpoint{3.220314in}{3.378436in}}%
\pgfpathlineto{\pgfqpoint{3.220314in}{3.378436in}}%
\pgfpathlineto{\pgfqpoint{3.220314in}{3.402031in}}%
\pgfpathlineto{\pgfqpoint{3.256641in}{3.402031in}}%
\pgfpathlineto{\pgfqpoint{3.256641in}{3.378436in}}%
\pgfpathmoveto{\pgfqpoint{3.292968in}{3.331248in}}%
\pgfpathlineto{\pgfqpoint{3.292968in}{3.331248in}}%
\pgfpathlineto{\pgfqpoint{3.292968in}{3.354842in}}%
\pgfpathlineto{\pgfqpoint{3.329294in}{3.354842in}}%
\pgfpathlineto{\pgfqpoint{3.329294in}{3.331248in}}%
\pgfpathmoveto{\pgfqpoint{3.474612in}{2.835781in}}%
\pgfpathlineto{\pgfqpoint{3.474612in}{2.835781in}}%
\pgfpathlineto{\pgfqpoint{3.474612in}{2.859375in}}%
\pgfpathlineto{\pgfqpoint{3.510942in}{2.859375in}}%
\pgfpathlineto{\pgfqpoint{3.510942in}{2.835781in}}%
\pgfpathmoveto{\pgfqpoint{3.438281in}{3.189687in}}%
\pgfpathlineto{\pgfqpoint{3.438281in}{3.189687in}}%
\pgfpathlineto{\pgfqpoint{3.438281in}{3.213281in}}%
\pgfpathlineto{\pgfqpoint{3.474612in}{3.213281in}}%
\pgfpathlineto{\pgfqpoint{3.474612in}{3.189687in}}%
\pgfpathmoveto{\pgfqpoint{3.438281in}{3.213281in}}%
\pgfpathlineto{\pgfqpoint{3.438281in}{3.213281in}}%
\pgfpathlineto{\pgfqpoint{3.438281in}{3.236874in}}%
\pgfpathlineto{\pgfqpoint{3.474612in}{3.236874in}}%
\pgfpathlineto{\pgfqpoint{3.474612in}{3.213281in}}%
\pgfpathmoveto{\pgfqpoint{3.474612in}{3.189687in}}%
\pgfpathlineto{\pgfqpoint{3.474612in}{3.189687in}}%
\pgfpathlineto{\pgfqpoint{3.474612in}{3.213281in}}%
\pgfpathlineto{\pgfqpoint{3.510942in}{3.213281in}}%
\pgfpathlineto{\pgfqpoint{3.510942in}{3.189687in}}%
\pgfpathmoveto{\pgfqpoint{3.365621in}{3.284061in}}%
\pgfpathlineto{\pgfqpoint{3.365621in}{3.284061in}}%
\pgfpathlineto{\pgfqpoint{3.365621in}{3.307654in}}%
\pgfpathlineto{\pgfqpoint{3.401951in}{3.307654in}}%
\pgfpathlineto{\pgfqpoint{3.401951in}{3.284061in}}%
\pgfpathmoveto{\pgfqpoint{3.510942in}{2.835781in}}%
\pgfpathlineto{\pgfqpoint{3.510942in}{2.835781in}}%
\pgfpathlineto{\pgfqpoint{3.510942in}{2.859375in}}%
\pgfpathlineto{\pgfqpoint{3.547269in}{2.859375in}}%
\pgfpathlineto{\pgfqpoint{3.547269in}{2.835781in}}%
\pgfpathmoveto{\pgfqpoint{3.547269in}{2.835781in}}%
\pgfpathlineto{\pgfqpoint{3.547269in}{2.835781in}}%
\pgfpathlineto{\pgfqpoint{3.547269in}{2.859375in}}%
\pgfpathlineto{\pgfqpoint{3.583597in}{2.859375in}}%
\pgfpathlineto{\pgfqpoint{3.583597in}{2.835781in}}%
\pgfpathmoveto{\pgfqpoint{3.583597in}{3.095312in}}%
\pgfpathlineto{\pgfqpoint{3.583597in}{3.095312in}}%
\pgfpathlineto{\pgfqpoint{3.583597in}{3.118906in}}%
\pgfpathlineto{\pgfqpoint{3.619925in}{3.118906in}}%
\pgfpathlineto{\pgfqpoint{3.619925in}{3.095312in}}%
\pgfpathmoveto{\pgfqpoint{3.583597in}{3.118906in}}%
\pgfpathlineto{\pgfqpoint{3.583597in}{3.118906in}}%
\pgfpathlineto{\pgfqpoint{3.583597in}{3.142499in}}%
\pgfpathlineto{\pgfqpoint{3.619925in}{3.142499in}}%
\pgfpathlineto{\pgfqpoint{3.619925in}{3.118906in}}%
\pgfpathmoveto{\pgfqpoint{3.619925in}{3.095312in}}%
\pgfpathlineto{\pgfqpoint{3.619925in}{3.095312in}}%
\pgfpathlineto{\pgfqpoint{3.619925in}{3.118906in}}%
\pgfpathlineto{\pgfqpoint{3.656252in}{3.118906in}}%
\pgfpathlineto{\pgfqpoint{3.656252in}{3.095312in}}%
\pgfpathmoveto{\pgfqpoint{3.510942in}{3.142499in}}%
\pgfpathlineto{\pgfqpoint{3.510942in}{3.142499in}}%
\pgfpathlineto{\pgfqpoint{3.510942in}{3.166093in}}%
\pgfpathlineto{\pgfqpoint{3.547269in}{3.166093in}}%
\pgfpathlineto{\pgfqpoint{3.547269in}{3.142499in}}%
\pgfpathmoveto{\pgfqpoint{3.510942in}{3.166093in}}%
\pgfpathlineto{\pgfqpoint{3.510942in}{3.166093in}}%
\pgfpathlineto{\pgfqpoint{3.510942in}{3.189687in}}%
\pgfpathlineto{\pgfqpoint{3.547269in}{3.189687in}}%
\pgfpathlineto{\pgfqpoint{3.547269in}{3.166093in}}%
\pgfpathmoveto{\pgfqpoint{3.547269in}{3.142499in}}%
\pgfpathlineto{\pgfqpoint{3.547269in}{3.142499in}}%
\pgfpathlineto{\pgfqpoint{3.547269in}{3.166093in}}%
\pgfpathlineto{\pgfqpoint{3.583597in}{3.166093in}}%
\pgfpathlineto{\pgfqpoint{3.583597in}{3.142499in}}%
\pgfpathmoveto{\pgfqpoint{3.692579in}{2.788592in}}%
\pgfpathlineto{\pgfqpoint{3.692579in}{2.788592in}}%
\pgfpathlineto{\pgfqpoint{3.692579in}{2.812187in}}%
\pgfpathlineto{\pgfqpoint{3.728906in}{2.812187in}}%
\pgfpathlineto{\pgfqpoint{3.728906in}{2.788592in}}%
\pgfpathmoveto{\pgfqpoint{3.728906in}{2.788592in}}%
\pgfpathlineto{\pgfqpoint{3.728906in}{2.788592in}}%
\pgfpathlineto{\pgfqpoint{3.728906in}{2.812187in}}%
\pgfpathlineto{\pgfqpoint{3.765233in}{2.812187in}}%
\pgfpathlineto{\pgfqpoint{3.765233in}{2.788592in}}%
\pgfpathmoveto{\pgfqpoint{3.765233in}{2.788592in}}%
\pgfpathlineto{\pgfqpoint{3.765233in}{2.788592in}}%
\pgfpathlineto{\pgfqpoint{3.765233in}{2.812187in}}%
\pgfpathlineto{\pgfqpoint{3.801560in}{2.812187in}}%
\pgfpathlineto{\pgfqpoint{3.801560in}{2.788592in}}%
\pgfpathmoveto{\pgfqpoint{3.728906in}{3.048125in}}%
\pgfpathlineto{\pgfqpoint{3.728906in}{3.048125in}}%
\pgfpathlineto{\pgfqpoint{3.728906in}{3.071718in}}%
\pgfpathlineto{\pgfqpoint{3.765233in}{3.071718in}}%
\pgfpathlineto{\pgfqpoint{3.765233in}{3.048125in}}%
\pgfpathmoveto{\pgfqpoint{3.910544in}{2.741405in}}%
\pgfpathlineto{\pgfqpoint{3.910544in}{2.741405in}}%
\pgfpathlineto{\pgfqpoint{3.910544in}{2.764998in}}%
\pgfpathlineto{\pgfqpoint{3.946873in}{2.764998in}}%
\pgfpathlineto{\pgfqpoint{3.946873in}{2.741405in}}%
\pgfpathmoveto{\pgfqpoint{3.874216in}{2.906564in}}%
\pgfpathlineto{\pgfqpoint{3.874216in}{2.906564in}}%
\pgfpathlineto{\pgfqpoint{3.874216in}{2.930158in}}%
\pgfpathlineto{\pgfqpoint{3.910544in}{2.930158in}}%
\pgfpathlineto{\pgfqpoint{3.910544in}{2.906564in}}%
\pgfpathmoveto{\pgfqpoint{3.874216in}{2.930158in}}%
\pgfpathlineto{\pgfqpoint{3.874216in}{2.930158in}}%
\pgfpathlineto{\pgfqpoint{3.874216in}{2.953752in}}%
\pgfpathlineto{\pgfqpoint{3.910544in}{2.953752in}}%
\pgfpathlineto{\pgfqpoint{3.910544in}{2.930158in}}%
\pgfpathmoveto{\pgfqpoint{3.910544in}{2.906564in}}%
\pgfpathlineto{\pgfqpoint{3.910544in}{2.906564in}}%
\pgfpathlineto{\pgfqpoint{3.910544in}{2.930158in}}%
\pgfpathlineto{\pgfqpoint{3.946873in}{2.930158in}}%
\pgfpathlineto{\pgfqpoint{3.946873in}{2.906564in}}%
\pgfpathmoveto{\pgfqpoint{3.801560in}{3.000938in}}%
\pgfpathlineto{\pgfqpoint{3.801560in}{3.000938in}}%
\pgfpathlineto{\pgfqpoint{3.801560in}{3.024531in}}%
\pgfpathlineto{\pgfqpoint{3.837888in}{3.024531in}}%
\pgfpathlineto{\pgfqpoint{3.837888in}{3.000938in}}%
\pgfpathmoveto{\pgfqpoint{3.874216in}{2.953752in}}%
\pgfpathlineto{\pgfqpoint{3.874216in}{2.953752in}}%
\pgfpathlineto{\pgfqpoint{3.874216in}{2.977345in}}%
\pgfpathlineto{\pgfqpoint{3.910544in}{2.977345in}}%
\pgfpathlineto{\pgfqpoint{3.910544in}{2.953752in}}%
\pgfpathmoveto{\pgfqpoint{3.946873in}{2.741405in}}%
\pgfpathlineto{\pgfqpoint{3.946873in}{2.741405in}}%
\pgfpathlineto{\pgfqpoint{3.946873in}{2.764998in}}%
\pgfpathlineto{\pgfqpoint{3.983202in}{2.764998in}}%
\pgfpathlineto{\pgfqpoint{3.983202in}{2.741405in}}%
\pgfpathmoveto{\pgfqpoint{3.983202in}{2.741405in}}%
\pgfpathlineto{\pgfqpoint{3.983202in}{2.741405in}}%
\pgfpathlineto{\pgfqpoint{3.983202in}{2.764998in}}%
\pgfpathlineto{\pgfqpoint{4.019531in}{2.764998in}}%
\pgfpathlineto{\pgfqpoint{4.019531in}{2.741405in}}%
\pgfpathmoveto{\pgfqpoint{4.019531in}{2.812187in}}%
\pgfpathlineto{\pgfqpoint{4.019531in}{2.812187in}}%
\pgfpathlineto{\pgfqpoint{4.019531in}{2.835781in}}%
\pgfpathlineto{\pgfqpoint{4.055860in}{2.835781in}}%
\pgfpathlineto{\pgfqpoint{4.055860in}{2.812187in}}%
\pgfpathmoveto{\pgfqpoint{4.019531in}{2.835781in}}%
\pgfpathlineto{\pgfqpoint{4.019531in}{2.835781in}}%
\pgfpathlineto{\pgfqpoint{4.019531in}{2.859375in}}%
\pgfpathlineto{\pgfqpoint{4.055860in}{2.859375in}}%
\pgfpathlineto{\pgfqpoint{4.055860in}{2.835781in}}%
\pgfpathmoveto{\pgfqpoint{4.055860in}{2.812187in}}%
\pgfpathlineto{\pgfqpoint{4.055860in}{2.812187in}}%
\pgfpathlineto{\pgfqpoint{4.055860in}{2.835781in}}%
\pgfpathlineto{\pgfqpoint{4.092189in}{2.835781in}}%
\pgfpathlineto{\pgfqpoint{4.092189in}{2.812187in}}%
\pgfpathmoveto{\pgfqpoint{3.946873in}{2.859375in}}%
\pgfpathlineto{\pgfqpoint{3.946873in}{2.859375in}}%
\pgfpathlineto{\pgfqpoint{3.946873in}{2.882970in}}%
\pgfpathlineto{\pgfqpoint{3.983202in}{2.882970in}}%
\pgfpathlineto{\pgfqpoint{3.983202in}{2.859375in}}%
\pgfpathmoveto{\pgfqpoint{3.946873in}{2.882970in}}%
\pgfpathlineto{\pgfqpoint{3.946873in}{2.882970in}}%
\pgfpathlineto{\pgfqpoint{3.946873in}{2.906564in}}%
\pgfpathlineto{\pgfqpoint{3.983202in}{2.906564in}}%
\pgfpathlineto{\pgfqpoint{3.983202in}{2.882970in}}%
\pgfpathmoveto{\pgfqpoint{3.983202in}{2.859375in}}%
\pgfpathlineto{\pgfqpoint{3.983202in}{2.859375in}}%
\pgfpathlineto{\pgfqpoint{3.983202in}{2.882970in}}%
\pgfpathlineto{\pgfqpoint{4.019531in}{2.882970in}}%
\pgfpathlineto{\pgfqpoint{4.019531in}{2.859375in}}%
\pgfpathmoveto{\pgfqpoint{4.128517in}{2.694217in}}%
\pgfpathlineto{\pgfqpoint{4.128517in}{2.694217in}}%
\pgfpathlineto{\pgfqpoint{4.128517in}{2.717811in}}%
\pgfpathlineto{\pgfqpoint{4.164846in}{2.717811in}}%
\pgfpathlineto{\pgfqpoint{4.164846in}{2.694217in}}%
\pgfpathmoveto{\pgfqpoint{4.164846in}{2.694217in}}%
\pgfpathlineto{\pgfqpoint{4.164846in}{2.694217in}}%
\pgfpathlineto{\pgfqpoint{4.164846in}{2.717811in}}%
\pgfpathlineto{\pgfqpoint{4.201175in}{2.717811in}}%
\pgfpathlineto{\pgfqpoint{4.201175in}{2.694217in}}%
\pgfpathmoveto{\pgfqpoint{4.201175in}{2.694217in}}%
\pgfpathlineto{\pgfqpoint{4.201175in}{2.694217in}}%
\pgfpathlineto{\pgfqpoint{4.201175in}{2.717811in}}%
\pgfpathlineto{\pgfqpoint{4.237503in}{2.717811in}}%
\pgfpathlineto{\pgfqpoint{4.237503in}{2.694217in}}%
\pgfpathmoveto{\pgfqpoint{4.164846in}{2.717811in}}%
\pgfpathlineto{\pgfqpoint{4.164846in}{2.717811in}}%
\pgfpathlineto{\pgfqpoint{4.164846in}{2.741405in}}%
\pgfpathlineto{\pgfqpoint{4.201175in}{2.741405in}}%
\pgfpathlineto{\pgfqpoint{4.201175in}{2.717811in}}%
\pgfpathmoveto{\pgfqpoint{4.164846in}{2.741405in}}%
\pgfpathlineto{\pgfqpoint{4.164846in}{2.741405in}}%
\pgfpathlineto{\pgfqpoint{4.164846in}{2.764998in}}%
\pgfpathlineto{\pgfqpoint{4.201175in}{2.764998in}}%
\pgfpathlineto{\pgfqpoint{4.201175in}{2.741405in}}%
\pgfpathmoveto{\pgfqpoint{4.201175in}{2.717811in}}%
\pgfpathlineto{\pgfqpoint{4.201175in}{2.717811in}}%
\pgfpathlineto{\pgfqpoint{4.201175in}{2.741405in}}%
\pgfpathlineto{\pgfqpoint{4.237503in}{2.741405in}}%
\pgfpathlineto{\pgfqpoint{4.237503in}{2.717811in}}%
\pgfpathmoveto{\pgfqpoint{4.092189in}{2.764998in}}%
\pgfpathlineto{\pgfqpoint{4.092189in}{2.764998in}}%
\pgfpathlineto{\pgfqpoint{4.092189in}{2.788592in}}%
\pgfpathlineto{\pgfqpoint{4.128517in}{2.788592in}}%
\pgfpathlineto{\pgfqpoint{4.128517in}{2.764998in}}%
\pgfpathmoveto{\pgfqpoint{4.092189in}{2.788592in}}%
\pgfpathlineto{\pgfqpoint{4.092189in}{2.788592in}}%
\pgfpathlineto{\pgfqpoint{4.092189in}{2.812187in}}%
\pgfpathlineto{\pgfqpoint{4.128517in}{2.812187in}}%
\pgfpathlineto{\pgfqpoint{4.128517in}{2.788592in}}%
\pgfpathmoveto{\pgfqpoint{4.128517in}{2.764998in}}%
\pgfpathlineto{\pgfqpoint{4.128517in}{2.764998in}}%
\pgfpathlineto{\pgfqpoint{4.128517in}{2.788592in}}%
\pgfpathlineto{\pgfqpoint{4.164846in}{2.788592in}}%
\pgfpathlineto{\pgfqpoint{4.164846in}{2.764998in}}%
\pgfpathmoveto{\pgfqpoint{4.346484in}{2.647030in}}%
\pgfpathlineto{\pgfqpoint{4.346484in}{2.647030in}}%
\pgfpathlineto{\pgfqpoint{4.346484in}{2.670624in}}%
\pgfpathlineto{\pgfqpoint{4.382811in}{2.670624in}}%
\pgfpathlineto{\pgfqpoint{4.382811in}{2.647030in}}%
\pgfpathmoveto{\pgfqpoint{4.237503in}{2.717811in}}%
\pgfpathlineto{\pgfqpoint{4.237503in}{2.717811in}}%
\pgfpathlineto{\pgfqpoint{4.237503in}{2.741405in}}%
\pgfpathlineto{\pgfqpoint{4.273830in}{2.741405in}}%
\pgfpathlineto{\pgfqpoint{4.273830in}{2.717811in}}%
\pgfpathmoveto{\pgfqpoint{4.310157in}{2.670624in}}%
\pgfpathlineto{\pgfqpoint{4.310157in}{2.670624in}}%
\pgfpathlineto{\pgfqpoint{4.310157in}{2.694217in}}%
\pgfpathlineto{\pgfqpoint{4.346484in}{2.694217in}}%
\pgfpathlineto{\pgfqpoint{4.346484in}{2.670624in}}%
\pgfpathmoveto{\pgfqpoint{3.075001in}{2.918361in}}%
\pgfpathlineto{\pgfqpoint{3.075001in}{2.918361in}}%
\pgfpathlineto{\pgfqpoint{3.075001in}{2.930158in}}%
\pgfpathlineto{\pgfqpoint{3.093165in}{2.930158in}}%
\pgfpathlineto{\pgfqpoint{3.093165in}{2.918361in}}%
\pgfpathmoveto{\pgfqpoint{3.093165in}{2.918361in}}%
\pgfpathlineto{\pgfqpoint{3.093165in}{2.918361in}}%
\pgfpathlineto{\pgfqpoint{3.093165in}{2.930158in}}%
\pgfpathlineto{\pgfqpoint{3.111329in}{2.930158in}}%
\pgfpathlineto{\pgfqpoint{3.111329in}{2.918361in}}%
\pgfpathmoveto{\pgfqpoint{3.111329in}{2.918361in}}%
\pgfpathlineto{\pgfqpoint{3.111329in}{2.918361in}}%
\pgfpathlineto{\pgfqpoint{3.111329in}{2.930158in}}%
\pgfpathlineto{\pgfqpoint{3.129493in}{2.930158in}}%
\pgfpathlineto{\pgfqpoint{3.129493in}{2.918361in}}%
\pgfpathmoveto{\pgfqpoint{3.129493in}{2.906564in}}%
\pgfpathlineto{\pgfqpoint{3.129493in}{2.906564in}}%
\pgfpathlineto{\pgfqpoint{3.129493in}{2.918361in}}%
\pgfpathlineto{\pgfqpoint{3.147657in}{2.918361in}}%
\pgfpathlineto{\pgfqpoint{3.147657in}{2.906564in}}%
\pgfpathmoveto{\pgfqpoint{3.129493in}{2.918361in}}%
\pgfpathlineto{\pgfqpoint{3.129493in}{2.918361in}}%
\pgfpathlineto{\pgfqpoint{3.129493in}{2.930158in}}%
\pgfpathlineto{\pgfqpoint{3.147657in}{2.930158in}}%
\pgfpathlineto{\pgfqpoint{3.147657in}{2.918361in}}%
\pgfpathmoveto{\pgfqpoint{3.183986in}{2.894767in}}%
\pgfpathlineto{\pgfqpoint{3.183986in}{2.894767in}}%
\pgfpathlineto{\pgfqpoint{3.183986in}{2.906564in}}%
\pgfpathlineto{\pgfqpoint{3.202150in}{2.906564in}}%
\pgfpathlineto{\pgfqpoint{3.202150in}{2.894767in}}%
\pgfpathmoveto{\pgfqpoint{3.202150in}{2.894767in}}%
\pgfpathlineto{\pgfqpoint{3.202150in}{2.894767in}}%
\pgfpathlineto{\pgfqpoint{3.202150in}{2.906564in}}%
\pgfpathlineto{\pgfqpoint{3.220314in}{2.906564in}}%
\pgfpathlineto{\pgfqpoint{3.220314in}{2.894767in}}%
\pgfpathmoveto{\pgfqpoint{3.183986in}{3.402031in}}%
\pgfpathlineto{\pgfqpoint{3.183986in}{3.402031in}}%
\pgfpathlineto{\pgfqpoint{3.183986in}{3.413828in}}%
\pgfpathlineto{\pgfqpoint{3.202150in}{3.413828in}}%
\pgfpathlineto{\pgfqpoint{3.202150in}{3.402031in}}%
\pgfpathmoveto{\pgfqpoint{3.183986in}{3.413828in}}%
\pgfpathlineto{\pgfqpoint{3.183986in}{3.413828in}}%
\pgfpathlineto{\pgfqpoint{3.183986in}{3.425625in}}%
\pgfpathlineto{\pgfqpoint{3.202150in}{3.425625in}}%
\pgfpathlineto{\pgfqpoint{3.202150in}{3.413828in}}%
\pgfpathmoveto{\pgfqpoint{3.202150in}{3.402031in}}%
\pgfpathlineto{\pgfqpoint{3.202150in}{3.402031in}}%
\pgfpathlineto{\pgfqpoint{3.202150in}{3.413828in}}%
\pgfpathlineto{\pgfqpoint{3.220314in}{3.413828in}}%
\pgfpathlineto{\pgfqpoint{3.220314in}{3.402031in}}%
\pgfpathmoveto{\pgfqpoint{3.111329in}{3.449219in}}%
\pgfpathlineto{\pgfqpoint{3.111329in}{3.449219in}}%
\pgfpathlineto{\pgfqpoint{3.111329in}{3.461016in}}%
\pgfpathlineto{\pgfqpoint{3.129493in}{3.461016in}}%
\pgfpathlineto{\pgfqpoint{3.129493in}{3.449219in}}%
\pgfpathmoveto{\pgfqpoint{3.111329in}{3.461016in}}%
\pgfpathlineto{\pgfqpoint{3.111329in}{3.461016in}}%
\pgfpathlineto{\pgfqpoint{3.111329in}{3.472813in}}%
\pgfpathlineto{\pgfqpoint{3.129493in}{3.472813in}}%
\pgfpathlineto{\pgfqpoint{3.129493in}{3.461016in}}%
\pgfpathmoveto{\pgfqpoint{3.129493in}{3.449219in}}%
\pgfpathlineto{\pgfqpoint{3.129493in}{3.449219in}}%
\pgfpathlineto{\pgfqpoint{3.129493in}{3.461016in}}%
\pgfpathlineto{\pgfqpoint{3.147657in}{3.461016in}}%
\pgfpathlineto{\pgfqpoint{3.147657in}{3.449219in}}%
\pgfpathmoveto{\pgfqpoint{3.075001in}{3.472813in}}%
\pgfpathlineto{\pgfqpoint{3.075001in}{3.472813in}}%
\pgfpathlineto{\pgfqpoint{3.075001in}{3.484610in}}%
\pgfpathlineto{\pgfqpoint{3.093165in}{3.484610in}}%
\pgfpathlineto{\pgfqpoint{3.093165in}{3.472813in}}%
\pgfpathmoveto{\pgfqpoint{3.075001in}{3.484610in}}%
\pgfpathlineto{\pgfqpoint{3.075001in}{3.484610in}}%
\pgfpathlineto{\pgfqpoint{3.075001in}{3.496408in}}%
\pgfpathlineto{\pgfqpoint{3.093165in}{3.496408in}}%
\pgfpathlineto{\pgfqpoint{3.093165in}{3.484610in}}%
\pgfpathmoveto{\pgfqpoint{3.093165in}{3.472813in}}%
\pgfpathlineto{\pgfqpoint{3.093165in}{3.472813in}}%
\pgfpathlineto{\pgfqpoint{3.093165in}{3.484610in}}%
\pgfpathlineto{\pgfqpoint{3.111329in}{3.484610in}}%
\pgfpathlineto{\pgfqpoint{3.111329in}{3.472813in}}%
\pgfpathmoveto{\pgfqpoint{3.147657in}{3.425625in}}%
\pgfpathlineto{\pgfqpoint{3.147657in}{3.425625in}}%
\pgfpathlineto{\pgfqpoint{3.147657in}{3.437422in}}%
\pgfpathlineto{\pgfqpoint{3.165822in}{3.437422in}}%
\pgfpathlineto{\pgfqpoint{3.165822in}{3.425625in}}%
\pgfpathmoveto{\pgfqpoint{3.147657in}{3.437422in}}%
\pgfpathlineto{\pgfqpoint{3.147657in}{3.437422in}}%
\pgfpathlineto{\pgfqpoint{3.147657in}{3.449219in}}%
\pgfpathlineto{\pgfqpoint{3.165822in}{3.449219in}}%
\pgfpathlineto{\pgfqpoint{3.165822in}{3.437422in}}%
\pgfpathmoveto{\pgfqpoint{3.165822in}{3.425625in}}%
\pgfpathlineto{\pgfqpoint{3.165822in}{3.425625in}}%
\pgfpathlineto{\pgfqpoint{3.165822in}{3.437422in}}%
\pgfpathlineto{\pgfqpoint{3.183986in}{3.437422in}}%
\pgfpathlineto{\pgfqpoint{3.183986in}{3.425625in}}%
\pgfpathmoveto{\pgfqpoint{3.220314in}{2.894767in}}%
\pgfpathlineto{\pgfqpoint{3.220314in}{2.894767in}}%
\pgfpathlineto{\pgfqpoint{3.220314in}{2.906564in}}%
\pgfpathlineto{\pgfqpoint{3.238478in}{2.906564in}}%
\pgfpathlineto{\pgfqpoint{3.238478in}{2.894767in}}%
\pgfpathmoveto{\pgfqpoint{3.238478in}{2.882970in}}%
\pgfpathlineto{\pgfqpoint{3.238478in}{2.882970in}}%
\pgfpathlineto{\pgfqpoint{3.238478in}{2.894767in}}%
\pgfpathlineto{\pgfqpoint{3.256641in}{2.894767in}}%
\pgfpathlineto{\pgfqpoint{3.256641in}{2.882970in}}%
\pgfpathmoveto{\pgfqpoint{3.238478in}{2.894767in}}%
\pgfpathlineto{\pgfqpoint{3.238478in}{2.894767in}}%
\pgfpathlineto{\pgfqpoint{3.238478in}{2.906564in}}%
\pgfpathlineto{\pgfqpoint{3.256641in}{2.906564in}}%
\pgfpathlineto{\pgfqpoint{3.256641in}{2.894767in}}%
\pgfpathmoveto{\pgfqpoint{3.292968in}{2.871173in}}%
\pgfpathlineto{\pgfqpoint{3.292968in}{2.871173in}}%
\pgfpathlineto{\pgfqpoint{3.292968in}{2.882970in}}%
\pgfpathlineto{\pgfqpoint{3.311131in}{2.882970in}}%
\pgfpathlineto{\pgfqpoint{3.311131in}{2.871173in}}%
\pgfpathmoveto{\pgfqpoint{3.311131in}{2.871173in}}%
\pgfpathlineto{\pgfqpoint{3.311131in}{2.871173in}}%
\pgfpathlineto{\pgfqpoint{3.311131in}{2.882970in}}%
\pgfpathlineto{\pgfqpoint{3.329294in}{2.882970in}}%
\pgfpathlineto{\pgfqpoint{3.329294in}{2.871173in}}%
\pgfpathmoveto{\pgfqpoint{3.329294in}{2.871173in}}%
\pgfpathlineto{\pgfqpoint{3.329294in}{2.871173in}}%
\pgfpathlineto{\pgfqpoint{3.329294in}{2.882970in}}%
\pgfpathlineto{\pgfqpoint{3.347458in}{2.882970in}}%
\pgfpathlineto{\pgfqpoint{3.347458in}{2.871173in}}%
\pgfpathmoveto{\pgfqpoint{3.347458in}{2.859375in}}%
\pgfpathlineto{\pgfqpoint{3.347458in}{2.859375in}}%
\pgfpathlineto{\pgfqpoint{3.347458in}{2.871173in}}%
\pgfpathlineto{\pgfqpoint{3.365621in}{2.871173in}}%
\pgfpathlineto{\pgfqpoint{3.365621in}{2.859375in}}%
\pgfpathmoveto{\pgfqpoint{3.347458in}{2.871173in}}%
\pgfpathlineto{\pgfqpoint{3.347458in}{2.871173in}}%
\pgfpathlineto{\pgfqpoint{3.347458in}{2.882970in}}%
\pgfpathlineto{\pgfqpoint{3.365621in}{2.882970in}}%
\pgfpathlineto{\pgfqpoint{3.365621in}{2.871173in}}%
\pgfpathmoveto{\pgfqpoint{3.256641in}{3.378436in}}%
\pgfpathlineto{\pgfqpoint{3.256641in}{3.378436in}}%
\pgfpathlineto{\pgfqpoint{3.256641in}{3.390233in}}%
\pgfpathlineto{\pgfqpoint{3.274804in}{3.390233in}}%
\pgfpathlineto{\pgfqpoint{3.274804in}{3.378436in}}%
\pgfpathmoveto{\pgfqpoint{3.292968in}{3.354842in}}%
\pgfpathlineto{\pgfqpoint{3.292968in}{3.354842in}}%
\pgfpathlineto{\pgfqpoint{3.292968in}{3.366639in}}%
\pgfpathlineto{\pgfqpoint{3.311131in}{3.366639in}}%
\pgfpathlineto{\pgfqpoint{3.311131in}{3.354842in}}%
\pgfpathmoveto{\pgfqpoint{3.329294in}{3.331248in}}%
\pgfpathlineto{\pgfqpoint{3.329294in}{3.331248in}}%
\pgfpathlineto{\pgfqpoint{3.329294in}{3.343045in}}%
\pgfpathlineto{\pgfqpoint{3.347458in}{3.343045in}}%
\pgfpathlineto{\pgfqpoint{3.347458in}{3.331248in}}%
\pgfpathmoveto{\pgfqpoint{3.401951in}{2.847578in}}%
\pgfpathlineto{\pgfqpoint{3.401951in}{2.847578in}}%
\pgfpathlineto{\pgfqpoint{3.401951in}{2.859375in}}%
\pgfpathlineto{\pgfqpoint{3.420116in}{2.859375in}}%
\pgfpathlineto{\pgfqpoint{3.420116in}{2.847578in}}%
\pgfpathmoveto{\pgfqpoint{3.420116in}{2.847578in}}%
\pgfpathlineto{\pgfqpoint{3.420116in}{2.847578in}}%
\pgfpathlineto{\pgfqpoint{3.420116in}{2.859375in}}%
\pgfpathlineto{\pgfqpoint{3.438281in}{2.859375in}}%
\pgfpathlineto{\pgfqpoint{3.438281in}{2.847578in}}%
\pgfpathmoveto{\pgfqpoint{3.438281in}{2.847578in}}%
\pgfpathlineto{\pgfqpoint{3.438281in}{2.847578in}}%
\pgfpathlineto{\pgfqpoint{3.438281in}{2.859375in}}%
\pgfpathlineto{\pgfqpoint{3.456447in}{2.859375in}}%
\pgfpathlineto{\pgfqpoint{3.456447in}{2.847578in}}%
\pgfpathmoveto{\pgfqpoint{3.456447in}{2.835781in}}%
\pgfpathlineto{\pgfqpoint{3.456447in}{2.835781in}}%
\pgfpathlineto{\pgfqpoint{3.456447in}{2.847578in}}%
\pgfpathlineto{\pgfqpoint{3.474612in}{2.847578in}}%
\pgfpathlineto{\pgfqpoint{3.474612in}{2.835781in}}%
\pgfpathmoveto{\pgfqpoint{3.456447in}{2.847578in}}%
\pgfpathlineto{\pgfqpoint{3.456447in}{2.847578in}}%
\pgfpathlineto{\pgfqpoint{3.456447in}{2.859375in}}%
\pgfpathlineto{\pgfqpoint{3.474612in}{2.859375in}}%
\pgfpathlineto{\pgfqpoint{3.474612in}{2.847578in}}%
\pgfpathmoveto{\pgfqpoint{3.474612in}{3.213281in}}%
\pgfpathlineto{\pgfqpoint{3.474612in}{3.213281in}}%
\pgfpathlineto{\pgfqpoint{3.474612in}{3.225077in}}%
\pgfpathlineto{\pgfqpoint{3.492777in}{3.225077in}}%
\pgfpathlineto{\pgfqpoint{3.492777in}{3.213281in}}%
\pgfpathmoveto{\pgfqpoint{3.474612in}{3.225077in}}%
\pgfpathlineto{\pgfqpoint{3.474612in}{3.225077in}}%
\pgfpathlineto{\pgfqpoint{3.474612in}{3.236874in}}%
\pgfpathlineto{\pgfqpoint{3.492777in}{3.236874in}}%
\pgfpathlineto{\pgfqpoint{3.492777in}{3.225077in}}%
\pgfpathmoveto{\pgfqpoint{3.492777in}{3.213281in}}%
\pgfpathlineto{\pgfqpoint{3.492777in}{3.213281in}}%
\pgfpathlineto{\pgfqpoint{3.492777in}{3.225077in}}%
\pgfpathlineto{\pgfqpoint{3.510942in}{3.225077in}}%
\pgfpathlineto{\pgfqpoint{3.510942in}{3.213281in}}%
\pgfpathmoveto{\pgfqpoint{3.365621in}{3.307654in}}%
\pgfpathlineto{\pgfqpoint{3.365621in}{3.307654in}}%
\pgfpathlineto{\pgfqpoint{3.365621in}{3.319451in}}%
\pgfpathlineto{\pgfqpoint{3.383786in}{3.319451in}}%
\pgfpathlineto{\pgfqpoint{3.383786in}{3.307654in}}%
\pgfpathmoveto{\pgfqpoint{3.401951in}{3.284061in}}%
\pgfpathlineto{\pgfqpoint{3.401951in}{3.284061in}}%
\pgfpathlineto{\pgfqpoint{3.401951in}{3.295858in}}%
\pgfpathlineto{\pgfqpoint{3.420116in}{3.295858in}}%
\pgfpathlineto{\pgfqpoint{3.420116in}{3.284061in}}%
\pgfpathmoveto{\pgfqpoint{3.438281in}{3.236874in}}%
\pgfpathlineto{\pgfqpoint{3.438281in}{3.236874in}}%
\pgfpathlineto{\pgfqpoint{3.438281in}{3.248671in}}%
\pgfpathlineto{\pgfqpoint{3.456447in}{3.248671in}}%
\pgfpathlineto{\pgfqpoint{3.456447in}{3.236874in}}%
\pgfpathmoveto{\pgfqpoint{3.438281in}{3.248671in}}%
\pgfpathlineto{\pgfqpoint{3.438281in}{3.248671in}}%
\pgfpathlineto{\pgfqpoint{3.438281in}{3.260468in}}%
\pgfpathlineto{\pgfqpoint{3.456447in}{3.260468in}}%
\pgfpathlineto{\pgfqpoint{3.456447in}{3.248671in}}%
\pgfpathmoveto{\pgfqpoint{3.456447in}{3.236874in}}%
\pgfpathlineto{\pgfqpoint{3.456447in}{3.236874in}}%
\pgfpathlineto{\pgfqpoint{3.456447in}{3.248671in}}%
\pgfpathlineto{\pgfqpoint{3.474612in}{3.248671in}}%
\pgfpathlineto{\pgfqpoint{3.474612in}{3.236874in}}%
\pgfpathmoveto{\pgfqpoint{3.438281in}{3.260468in}}%
\pgfpathlineto{\pgfqpoint{3.438281in}{3.260468in}}%
\pgfpathlineto{\pgfqpoint{3.438281in}{3.272264in}}%
\pgfpathlineto{\pgfqpoint{3.456447in}{3.272264in}}%
\pgfpathlineto{\pgfqpoint{3.456447in}{3.260468in}}%
\pgfpathmoveto{\pgfqpoint{3.510942in}{2.823984in}}%
\pgfpathlineto{\pgfqpoint{3.510942in}{2.823984in}}%
\pgfpathlineto{\pgfqpoint{3.510942in}{2.835781in}}%
\pgfpathlineto{\pgfqpoint{3.529106in}{2.835781in}}%
\pgfpathlineto{\pgfqpoint{3.529106in}{2.823984in}}%
\pgfpathmoveto{\pgfqpoint{3.529106in}{2.823984in}}%
\pgfpathlineto{\pgfqpoint{3.529106in}{2.823984in}}%
\pgfpathlineto{\pgfqpoint{3.529106in}{2.835781in}}%
\pgfpathlineto{\pgfqpoint{3.547269in}{2.835781in}}%
\pgfpathlineto{\pgfqpoint{3.547269in}{2.823984in}}%
\pgfpathmoveto{\pgfqpoint{3.547269in}{2.823984in}}%
\pgfpathlineto{\pgfqpoint{3.547269in}{2.823984in}}%
\pgfpathlineto{\pgfqpoint{3.547269in}{2.835781in}}%
\pgfpathlineto{\pgfqpoint{3.565433in}{2.835781in}}%
\pgfpathlineto{\pgfqpoint{3.565433in}{2.823984in}}%
\pgfpathmoveto{\pgfqpoint{3.565433in}{2.812187in}}%
\pgfpathlineto{\pgfqpoint{3.565433in}{2.812187in}}%
\pgfpathlineto{\pgfqpoint{3.565433in}{2.823984in}}%
\pgfpathlineto{\pgfqpoint{3.583597in}{2.823984in}}%
\pgfpathlineto{\pgfqpoint{3.583597in}{2.812187in}}%
\pgfpathmoveto{\pgfqpoint{3.565433in}{2.823984in}}%
\pgfpathlineto{\pgfqpoint{3.565433in}{2.823984in}}%
\pgfpathlineto{\pgfqpoint{3.565433in}{2.835781in}}%
\pgfpathlineto{\pgfqpoint{3.583597in}{2.835781in}}%
\pgfpathlineto{\pgfqpoint{3.583597in}{2.823984in}}%
\pgfpathmoveto{\pgfqpoint{3.619925in}{2.800390in}}%
\pgfpathlineto{\pgfqpoint{3.619925in}{2.800390in}}%
\pgfpathlineto{\pgfqpoint{3.619925in}{2.812187in}}%
\pgfpathlineto{\pgfqpoint{3.638089in}{2.812187in}}%
\pgfpathlineto{\pgfqpoint{3.638089in}{2.800390in}}%
\pgfpathmoveto{\pgfqpoint{3.638089in}{2.800390in}}%
\pgfpathlineto{\pgfqpoint{3.638089in}{2.800390in}}%
\pgfpathlineto{\pgfqpoint{3.638089in}{2.812187in}}%
\pgfpathlineto{\pgfqpoint{3.656252in}{2.812187in}}%
\pgfpathlineto{\pgfqpoint{3.656252in}{2.800390in}}%
\pgfpathmoveto{\pgfqpoint{3.619925in}{3.118906in}}%
\pgfpathlineto{\pgfqpoint{3.619925in}{3.118906in}}%
\pgfpathlineto{\pgfqpoint{3.619925in}{3.130703in}}%
\pgfpathlineto{\pgfqpoint{3.638089in}{3.130703in}}%
\pgfpathlineto{\pgfqpoint{3.638089in}{3.118906in}}%
\pgfpathmoveto{\pgfqpoint{3.619925in}{3.130703in}}%
\pgfpathlineto{\pgfqpoint{3.619925in}{3.130703in}}%
\pgfpathlineto{\pgfqpoint{3.619925in}{3.142499in}}%
\pgfpathlineto{\pgfqpoint{3.638089in}{3.142499in}}%
\pgfpathlineto{\pgfqpoint{3.638089in}{3.130703in}}%
\pgfpathmoveto{\pgfqpoint{3.638089in}{3.118906in}}%
\pgfpathlineto{\pgfqpoint{3.638089in}{3.118906in}}%
\pgfpathlineto{\pgfqpoint{3.638089in}{3.130703in}}%
\pgfpathlineto{\pgfqpoint{3.656252in}{3.130703in}}%
\pgfpathlineto{\pgfqpoint{3.656252in}{3.118906in}}%
\pgfpathmoveto{\pgfqpoint{3.547269in}{3.166093in}}%
\pgfpathlineto{\pgfqpoint{3.547269in}{3.166093in}}%
\pgfpathlineto{\pgfqpoint{3.547269in}{3.177890in}}%
\pgfpathlineto{\pgfqpoint{3.565433in}{3.177890in}}%
\pgfpathlineto{\pgfqpoint{3.565433in}{3.166093in}}%
\pgfpathmoveto{\pgfqpoint{3.547269in}{3.177890in}}%
\pgfpathlineto{\pgfqpoint{3.547269in}{3.177890in}}%
\pgfpathlineto{\pgfqpoint{3.547269in}{3.189687in}}%
\pgfpathlineto{\pgfqpoint{3.565433in}{3.189687in}}%
\pgfpathlineto{\pgfqpoint{3.565433in}{3.177890in}}%
\pgfpathmoveto{\pgfqpoint{3.565433in}{3.166093in}}%
\pgfpathlineto{\pgfqpoint{3.565433in}{3.166093in}}%
\pgfpathlineto{\pgfqpoint{3.565433in}{3.177890in}}%
\pgfpathlineto{\pgfqpoint{3.583597in}{3.177890in}}%
\pgfpathlineto{\pgfqpoint{3.583597in}{3.166093in}}%
\pgfpathmoveto{\pgfqpoint{3.510942in}{3.189687in}}%
\pgfpathlineto{\pgfqpoint{3.510942in}{3.189687in}}%
\pgfpathlineto{\pgfqpoint{3.510942in}{3.201484in}}%
\pgfpathlineto{\pgfqpoint{3.529106in}{3.201484in}}%
\pgfpathlineto{\pgfqpoint{3.529106in}{3.189687in}}%
\pgfpathmoveto{\pgfqpoint{3.510942in}{3.201484in}}%
\pgfpathlineto{\pgfqpoint{3.510942in}{3.201484in}}%
\pgfpathlineto{\pgfqpoint{3.510942in}{3.213281in}}%
\pgfpathlineto{\pgfqpoint{3.529106in}{3.213281in}}%
\pgfpathlineto{\pgfqpoint{3.529106in}{3.201484in}}%
\pgfpathmoveto{\pgfqpoint{3.529106in}{3.189687in}}%
\pgfpathlineto{\pgfqpoint{3.529106in}{3.189687in}}%
\pgfpathlineto{\pgfqpoint{3.529106in}{3.201484in}}%
\pgfpathlineto{\pgfqpoint{3.547269in}{3.201484in}}%
\pgfpathlineto{\pgfqpoint{3.547269in}{3.189687in}}%
\pgfpathmoveto{\pgfqpoint{3.583597in}{3.142499in}}%
\pgfpathlineto{\pgfqpoint{3.583597in}{3.142499in}}%
\pgfpathlineto{\pgfqpoint{3.583597in}{3.154296in}}%
\pgfpathlineto{\pgfqpoint{3.601761in}{3.154296in}}%
\pgfpathlineto{\pgfqpoint{3.601761in}{3.142499in}}%
\pgfpathmoveto{\pgfqpoint{3.583597in}{3.154296in}}%
\pgfpathlineto{\pgfqpoint{3.583597in}{3.154296in}}%
\pgfpathlineto{\pgfqpoint{3.583597in}{3.166093in}}%
\pgfpathlineto{\pgfqpoint{3.601761in}{3.166093in}}%
\pgfpathlineto{\pgfqpoint{3.601761in}{3.154296in}}%
\pgfpathmoveto{\pgfqpoint{3.601761in}{3.142499in}}%
\pgfpathlineto{\pgfqpoint{3.601761in}{3.142499in}}%
\pgfpathlineto{\pgfqpoint{3.601761in}{3.154296in}}%
\pgfpathlineto{\pgfqpoint{3.619925in}{3.154296in}}%
\pgfpathlineto{\pgfqpoint{3.619925in}{3.142499in}}%
\pgfpathmoveto{\pgfqpoint{3.656252in}{2.800390in}}%
\pgfpathlineto{\pgfqpoint{3.656252in}{2.800390in}}%
\pgfpathlineto{\pgfqpoint{3.656252in}{2.812187in}}%
\pgfpathlineto{\pgfqpoint{3.674416in}{2.812187in}}%
\pgfpathlineto{\pgfqpoint{3.674416in}{2.800390in}}%
\pgfpathmoveto{\pgfqpoint{3.674416in}{2.788592in}}%
\pgfpathlineto{\pgfqpoint{3.674416in}{2.788592in}}%
\pgfpathlineto{\pgfqpoint{3.674416in}{2.800390in}}%
\pgfpathlineto{\pgfqpoint{3.692579in}{2.800390in}}%
\pgfpathlineto{\pgfqpoint{3.692579in}{2.788592in}}%
\pgfpathmoveto{\pgfqpoint{3.674416in}{2.800390in}}%
\pgfpathlineto{\pgfqpoint{3.674416in}{2.800390in}}%
\pgfpathlineto{\pgfqpoint{3.674416in}{2.812187in}}%
\pgfpathlineto{\pgfqpoint{3.692579in}{2.812187in}}%
\pgfpathlineto{\pgfqpoint{3.692579in}{2.800390in}}%
\pgfpathmoveto{\pgfqpoint{3.728906in}{2.776795in}}%
\pgfpathlineto{\pgfqpoint{3.728906in}{2.776795in}}%
\pgfpathlineto{\pgfqpoint{3.728906in}{2.788592in}}%
\pgfpathlineto{\pgfqpoint{3.747069in}{2.788592in}}%
\pgfpathlineto{\pgfqpoint{3.747069in}{2.776795in}}%
\pgfpathmoveto{\pgfqpoint{3.747069in}{2.776795in}}%
\pgfpathlineto{\pgfqpoint{3.747069in}{2.776795in}}%
\pgfpathlineto{\pgfqpoint{3.747069in}{2.788592in}}%
\pgfpathlineto{\pgfqpoint{3.765233in}{2.788592in}}%
\pgfpathlineto{\pgfqpoint{3.765233in}{2.776795in}}%
\pgfpathmoveto{\pgfqpoint{3.765233in}{2.776795in}}%
\pgfpathlineto{\pgfqpoint{3.765233in}{2.776795in}}%
\pgfpathlineto{\pgfqpoint{3.765233in}{2.788592in}}%
\pgfpathlineto{\pgfqpoint{3.783396in}{2.788592in}}%
\pgfpathlineto{\pgfqpoint{3.783396in}{2.776795in}}%
\pgfpathmoveto{\pgfqpoint{3.783396in}{2.764998in}}%
\pgfpathlineto{\pgfqpoint{3.783396in}{2.764998in}}%
\pgfpathlineto{\pgfqpoint{3.783396in}{2.776795in}}%
\pgfpathlineto{\pgfqpoint{3.801560in}{2.776795in}}%
\pgfpathlineto{\pgfqpoint{3.801560in}{2.764998in}}%
\pgfpathmoveto{\pgfqpoint{3.783396in}{2.776795in}}%
\pgfpathlineto{\pgfqpoint{3.783396in}{2.776795in}}%
\pgfpathlineto{\pgfqpoint{3.783396in}{2.788592in}}%
\pgfpathlineto{\pgfqpoint{3.801560in}{2.788592in}}%
\pgfpathlineto{\pgfqpoint{3.801560in}{2.776795in}}%
\pgfpathmoveto{\pgfqpoint{3.656252in}{3.095312in}}%
\pgfpathlineto{\pgfqpoint{3.656252in}{3.095312in}}%
\pgfpathlineto{\pgfqpoint{3.656252in}{3.107109in}}%
\pgfpathlineto{\pgfqpoint{3.674416in}{3.107109in}}%
\pgfpathlineto{\pgfqpoint{3.674416in}{3.095312in}}%
\pgfpathmoveto{\pgfqpoint{3.656252in}{3.107109in}}%
\pgfpathlineto{\pgfqpoint{3.656252in}{3.107109in}}%
\pgfpathlineto{\pgfqpoint{3.656252in}{3.118906in}}%
\pgfpathlineto{\pgfqpoint{3.674416in}{3.118906in}}%
\pgfpathlineto{\pgfqpoint{3.674416in}{3.107109in}}%
\pgfpathmoveto{\pgfqpoint{3.674416in}{3.095312in}}%
\pgfpathlineto{\pgfqpoint{3.674416in}{3.095312in}}%
\pgfpathlineto{\pgfqpoint{3.674416in}{3.107109in}}%
\pgfpathlineto{\pgfqpoint{3.692579in}{3.107109in}}%
\pgfpathlineto{\pgfqpoint{3.692579in}{3.095312in}}%
\pgfpathmoveto{\pgfqpoint{3.692579in}{3.095312in}}%
\pgfpathlineto{\pgfqpoint{3.692579in}{3.095312in}}%
\pgfpathlineto{\pgfqpoint{3.692579in}{3.107109in}}%
\pgfpathlineto{\pgfqpoint{3.710743in}{3.107109in}}%
\pgfpathlineto{\pgfqpoint{3.710743in}{3.095312in}}%
\pgfpathmoveto{\pgfqpoint{3.728906in}{3.071718in}}%
\pgfpathlineto{\pgfqpoint{3.728906in}{3.071718in}}%
\pgfpathlineto{\pgfqpoint{3.728906in}{3.083515in}}%
\pgfpathlineto{\pgfqpoint{3.747069in}{3.083515in}}%
\pgfpathlineto{\pgfqpoint{3.747069in}{3.071718in}}%
\pgfpathmoveto{\pgfqpoint{3.765233in}{3.048125in}}%
\pgfpathlineto{\pgfqpoint{3.765233in}{3.048125in}}%
\pgfpathlineto{\pgfqpoint{3.765233in}{3.059922in}}%
\pgfpathlineto{\pgfqpoint{3.783396in}{3.059922in}}%
\pgfpathlineto{\pgfqpoint{3.783396in}{3.048125in}}%
\pgfpathmoveto{\pgfqpoint{3.837888in}{2.753201in}}%
\pgfpathlineto{\pgfqpoint{3.837888in}{2.753201in}}%
\pgfpathlineto{\pgfqpoint{3.837888in}{2.764998in}}%
\pgfpathlineto{\pgfqpoint{3.856052in}{2.764998in}}%
\pgfpathlineto{\pgfqpoint{3.856052in}{2.753201in}}%
\pgfpathmoveto{\pgfqpoint{3.856052in}{2.753201in}}%
\pgfpathlineto{\pgfqpoint{3.856052in}{2.753201in}}%
\pgfpathlineto{\pgfqpoint{3.856052in}{2.764998in}}%
\pgfpathlineto{\pgfqpoint{3.874216in}{2.764998in}}%
\pgfpathlineto{\pgfqpoint{3.874216in}{2.753201in}}%
\pgfpathmoveto{\pgfqpoint{3.874216in}{2.753201in}}%
\pgfpathlineto{\pgfqpoint{3.874216in}{2.753201in}}%
\pgfpathlineto{\pgfqpoint{3.874216in}{2.764998in}}%
\pgfpathlineto{\pgfqpoint{3.892380in}{2.764998in}}%
\pgfpathlineto{\pgfqpoint{3.892380in}{2.753201in}}%
\pgfpathmoveto{\pgfqpoint{3.892380in}{2.741405in}}%
\pgfpathlineto{\pgfqpoint{3.892380in}{2.741405in}}%
\pgfpathlineto{\pgfqpoint{3.892380in}{2.753201in}}%
\pgfpathlineto{\pgfqpoint{3.910544in}{2.753201in}}%
\pgfpathlineto{\pgfqpoint{3.910544in}{2.741405in}}%
\pgfpathmoveto{\pgfqpoint{3.892380in}{2.753201in}}%
\pgfpathlineto{\pgfqpoint{3.892380in}{2.753201in}}%
\pgfpathlineto{\pgfqpoint{3.892380in}{2.764998in}}%
\pgfpathlineto{\pgfqpoint{3.910544in}{2.764998in}}%
\pgfpathlineto{\pgfqpoint{3.910544in}{2.753201in}}%
\pgfpathmoveto{\pgfqpoint{3.910544in}{2.930158in}}%
\pgfpathlineto{\pgfqpoint{3.910544in}{2.930158in}}%
\pgfpathlineto{\pgfqpoint{3.910544in}{2.941955in}}%
\pgfpathlineto{\pgfqpoint{3.928709in}{2.941955in}}%
\pgfpathlineto{\pgfqpoint{3.928709in}{2.930158in}}%
\pgfpathmoveto{\pgfqpoint{3.910544in}{2.941955in}}%
\pgfpathlineto{\pgfqpoint{3.910544in}{2.941955in}}%
\pgfpathlineto{\pgfqpoint{3.910544in}{2.953752in}}%
\pgfpathlineto{\pgfqpoint{3.928709in}{2.953752in}}%
\pgfpathlineto{\pgfqpoint{3.928709in}{2.941955in}}%
\pgfpathmoveto{\pgfqpoint{3.928709in}{2.930158in}}%
\pgfpathlineto{\pgfqpoint{3.928709in}{2.930158in}}%
\pgfpathlineto{\pgfqpoint{3.928709in}{2.941955in}}%
\pgfpathlineto{\pgfqpoint{3.946873in}{2.941955in}}%
\pgfpathlineto{\pgfqpoint{3.946873in}{2.930158in}}%
\pgfpathmoveto{\pgfqpoint{3.801560in}{3.024531in}}%
\pgfpathlineto{\pgfqpoint{3.801560in}{3.024531in}}%
\pgfpathlineto{\pgfqpoint{3.801560in}{3.036328in}}%
\pgfpathlineto{\pgfqpoint{3.819724in}{3.036328in}}%
\pgfpathlineto{\pgfqpoint{3.819724in}{3.024531in}}%
\pgfpathmoveto{\pgfqpoint{3.837888in}{3.000938in}}%
\pgfpathlineto{\pgfqpoint{3.837888in}{3.000938in}}%
\pgfpathlineto{\pgfqpoint{3.837888in}{3.012735in}}%
\pgfpathlineto{\pgfqpoint{3.856052in}{3.012735in}}%
\pgfpathlineto{\pgfqpoint{3.856052in}{3.000938in}}%
\pgfpathmoveto{\pgfqpoint{3.874216in}{2.977345in}}%
\pgfpathlineto{\pgfqpoint{3.874216in}{2.977345in}}%
\pgfpathlineto{\pgfqpoint{3.874216in}{2.989142in}}%
\pgfpathlineto{\pgfqpoint{3.892380in}{2.989142in}}%
\pgfpathlineto{\pgfqpoint{3.892380in}{2.977345in}}%
\pgfpathmoveto{\pgfqpoint{3.946873in}{2.729608in}}%
\pgfpathlineto{\pgfqpoint{3.946873in}{2.729608in}}%
\pgfpathlineto{\pgfqpoint{3.946873in}{2.741405in}}%
\pgfpathlineto{\pgfqpoint{3.965037in}{2.741405in}}%
\pgfpathlineto{\pgfqpoint{3.965037in}{2.729608in}}%
\pgfpathmoveto{\pgfqpoint{3.965037in}{2.729608in}}%
\pgfpathlineto{\pgfqpoint{3.965037in}{2.729608in}}%
\pgfpathlineto{\pgfqpoint{3.965037in}{2.741405in}}%
\pgfpathlineto{\pgfqpoint{3.983202in}{2.741405in}}%
\pgfpathlineto{\pgfqpoint{3.983202in}{2.729608in}}%
\pgfpathmoveto{\pgfqpoint{3.983202in}{2.729608in}}%
\pgfpathlineto{\pgfqpoint{3.983202in}{2.729608in}}%
\pgfpathlineto{\pgfqpoint{3.983202in}{2.741405in}}%
\pgfpathlineto{\pgfqpoint{4.001366in}{2.741405in}}%
\pgfpathlineto{\pgfqpoint{4.001366in}{2.729608in}}%
\pgfpathmoveto{\pgfqpoint{4.001366in}{2.717811in}}%
\pgfpathlineto{\pgfqpoint{4.001366in}{2.717811in}}%
\pgfpathlineto{\pgfqpoint{4.001366in}{2.729608in}}%
\pgfpathlineto{\pgfqpoint{4.019531in}{2.729608in}}%
\pgfpathlineto{\pgfqpoint{4.019531in}{2.717811in}}%
\pgfpathmoveto{\pgfqpoint{4.001366in}{2.729608in}}%
\pgfpathlineto{\pgfqpoint{4.001366in}{2.729608in}}%
\pgfpathlineto{\pgfqpoint{4.001366in}{2.741405in}}%
\pgfpathlineto{\pgfqpoint{4.019531in}{2.741405in}}%
\pgfpathlineto{\pgfqpoint{4.019531in}{2.729608in}}%
\pgfpathmoveto{\pgfqpoint{4.055860in}{2.706014in}}%
\pgfpathlineto{\pgfqpoint{4.055860in}{2.706014in}}%
\pgfpathlineto{\pgfqpoint{4.055860in}{2.717811in}}%
\pgfpathlineto{\pgfqpoint{4.074024in}{2.717811in}}%
\pgfpathlineto{\pgfqpoint{4.074024in}{2.706014in}}%
\pgfpathmoveto{\pgfqpoint{4.074024in}{2.706014in}}%
\pgfpathlineto{\pgfqpoint{4.074024in}{2.706014in}}%
\pgfpathlineto{\pgfqpoint{4.074024in}{2.717811in}}%
\pgfpathlineto{\pgfqpoint{4.092189in}{2.717811in}}%
\pgfpathlineto{\pgfqpoint{4.092189in}{2.706014in}}%
\pgfpathmoveto{\pgfqpoint{4.055860in}{2.835781in}}%
\pgfpathlineto{\pgfqpoint{4.055860in}{2.835781in}}%
\pgfpathlineto{\pgfqpoint{4.055860in}{2.847578in}}%
\pgfpathlineto{\pgfqpoint{4.074024in}{2.847578in}}%
\pgfpathlineto{\pgfqpoint{4.074024in}{2.835781in}}%
\pgfpathmoveto{\pgfqpoint{4.055860in}{2.847578in}}%
\pgfpathlineto{\pgfqpoint{4.055860in}{2.847578in}}%
\pgfpathlineto{\pgfqpoint{4.055860in}{2.859375in}}%
\pgfpathlineto{\pgfqpoint{4.074024in}{2.859375in}}%
\pgfpathlineto{\pgfqpoint{4.074024in}{2.847578in}}%
\pgfpathmoveto{\pgfqpoint{4.074024in}{2.835781in}}%
\pgfpathlineto{\pgfqpoint{4.074024in}{2.835781in}}%
\pgfpathlineto{\pgfqpoint{4.074024in}{2.847578in}}%
\pgfpathlineto{\pgfqpoint{4.092189in}{2.847578in}}%
\pgfpathlineto{\pgfqpoint{4.092189in}{2.835781in}}%
\pgfpathmoveto{\pgfqpoint{3.983202in}{2.882970in}}%
\pgfpathlineto{\pgfqpoint{3.983202in}{2.882970in}}%
\pgfpathlineto{\pgfqpoint{3.983202in}{2.894767in}}%
\pgfpathlineto{\pgfqpoint{4.001366in}{2.894767in}}%
\pgfpathlineto{\pgfqpoint{4.001366in}{2.882970in}}%
\pgfpathmoveto{\pgfqpoint{3.983202in}{2.894767in}}%
\pgfpathlineto{\pgfqpoint{3.983202in}{2.894767in}}%
\pgfpathlineto{\pgfqpoint{3.983202in}{2.906564in}}%
\pgfpathlineto{\pgfqpoint{4.001366in}{2.906564in}}%
\pgfpathlineto{\pgfqpoint{4.001366in}{2.894767in}}%
\pgfpathmoveto{\pgfqpoint{4.001366in}{2.882970in}}%
\pgfpathlineto{\pgfqpoint{4.001366in}{2.882970in}}%
\pgfpathlineto{\pgfqpoint{4.001366in}{2.894767in}}%
\pgfpathlineto{\pgfqpoint{4.019531in}{2.894767in}}%
\pgfpathlineto{\pgfqpoint{4.019531in}{2.882970in}}%
\pgfpathmoveto{\pgfqpoint{3.946873in}{2.906564in}}%
\pgfpathlineto{\pgfqpoint{3.946873in}{2.906564in}}%
\pgfpathlineto{\pgfqpoint{3.946873in}{2.918361in}}%
\pgfpathlineto{\pgfqpoint{3.965037in}{2.918361in}}%
\pgfpathlineto{\pgfqpoint{3.965037in}{2.906564in}}%
\pgfpathmoveto{\pgfqpoint{3.946873in}{2.918361in}}%
\pgfpathlineto{\pgfqpoint{3.946873in}{2.918361in}}%
\pgfpathlineto{\pgfqpoint{3.946873in}{2.930158in}}%
\pgfpathlineto{\pgfqpoint{3.965037in}{2.930158in}}%
\pgfpathlineto{\pgfqpoint{3.965037in}{2.918361in}}%
\pgfpathmoveto{\pgfqpoint{3.965037in}{2.906564in}}%
\pgfpathlineto{\pgfqpoint{3.965037in}{2.906564in}}%
\pgfpathlineto{\pgfqpoint{3.965037in}{2.918361in}}%
\pgfpathlineto{\pgfqpoint{3.983202in}{2.918361in}}%
\pgfpathlineto{\pgfqpoint{3.983202in}{2.906564in}}%
\pgfpathmoveto{\pgfqpoint{4.019531in}{2.859375in}}%
\pgfpathlineto{\pgfqpoint{4.019531in}{2.859375in}}%
\pgfpathlineto{\pgfqpoint{4.019531in}{2.871173in}}%
\pgfpathlineto{\pgfqpoint{4.037695in}{2.871173in}}%
\pgfpathlineto{\pgfqpoint{4.037695in}{2.859375in}}%
\pgfpathmoveto{\pgfqpoint{4.019531in}{2.871173in}}%
\pgfpathlineto{\pgfqpoint{4.019531in}{2.871173in}}%
\pgfpathlineto{\pgfqpoint{4.019531in}{2.882970in}}%
\pgfpathlineto{\pgfqpoint{4.037695in}{2.882970in}}%
\pgfpathlineto{\pgfqpoint{4.037695in}{2.871173in}}%
\pgfpathmoveto{\pgfqpoint{4.037695in}{2.859375in}}%
\pgfpathlineto{\pgfqpoint{4.037695in}{2.859375in}}%
\pgfpathlineto{\pgfqpoint{4.037695in}{2.871173in}}%
\pgfpathlineto{\pgfqpoint{4.055860in}{2.871173in}}%
\pgfpathlineto{\pgfqpoint{4.055860in}{2.859375in}}%
\pgfpathmoveto{\pgfqpoint{4.092189in}{2.706014in}}%
\pgfpathlineto{\pgfqpoint{4.092189in}{2.706014in}}%
\pgfpathlineto{\pgfqpoint{4.092189in}{2.717811in}}%
\pgfpathlineto{\pgfqpoint{4.110353in}{2.717811in}}%
\pgfpathlineto{\pgfqpoint{4.110353in}{2.706014in}}%
\pgfpathmoveto{\pgfqpoint{4.110353in}{2.694217in}}%
\pgfpathlineto{\pgfqpoint{4.110353in}{2.694217in}}%
\pgfpathlineto{\pgfqpoint{4.110353in}{2.706014in}}%
\pgfpathlineto{\pgfqpoint{4.128517in}{2.706014in}}%
\pgfpathlineto{\pgfqpoint{4.128517in}{2.694217in}}%
\pgfpathmoveto{\pgfqpoint{4.110353in}{2.706014in}}%
\pgfpathlineto{\pgfqpoint{4.110353in}{2.706014in}}%
\pgfpathlineto{\pgfqpoint{4.110353in}{2.717811in}}%
\pgfpathlineto{\pgfqpoint{4.128517in}{2.717811in}}%
\pgfpathlineto{\pgfqpoint{4.128517in}{2.706014in}}%
\pgfpathmoveto{\pgfqpoint{4.164846in}{2.682421in}}%
\pgfpathlineto{\pgfqpoint{4.164846in}{2.682421in}}%
\pgfpathlineto{\pgfqpoint{4.164846in}{2.694217in}}%
\pgfpathlineto{\pgfqpoint{4.183010in}{2.694217in}}%
\pgfpathlineto{\pgfqpoint{4.183010in}{2.682421in}}%
\pgfpathmoveto{\pgfqpoint{4.183010in}{2.682421in}}%
\pgfpathlineto{\pgfqpoint{4.183010in}{2.682421in}}%
\pgfpathlineto{\pgfqpoint{4.183010in}{2.694217in}}%
\pgfpathlineto{\pgfqpoint{4.201175in}{2.694217in}}%
\pgfpathlineto{\pgfqpoint{4.201175in}{2.682421in}}%
\pgfpathmoveto{\pgfqpoint{4.201175in}{2.682421in}}%
\pgfpathlineto{\pgfqpoint{4.201175in}{2.682421in}}%
\pgfpathlineto{\pgfqpoint{4.201175in}{2.694217in}}%
\pgfpathlineto{\pgfqpoint{4.219339in}{2.694217in}}%
\pgfpathlineto{\pgfqpoint{4.219339in}{2.682421in}}%
\pgfpathmoveto{\pgfqpoint{4.219339in}{2.670624in}}%
\pgfpathlineto{\pgfqpoint{4.219339in}{2.670624in}}%
\pgfpathlineto{\pgfqpoint{4.219339in}{2.682421in}}%
\pgfpathlineto{\pgfqpoint{4.237503in}{2.682421in}}%
\pgfpathlineto{\pgfqpoint{4.237503in}{2.670624in}}%
\pgfpathmoveto{\pgfqpoint{4.219339in}{2.682421in}}%
\pgfpathlineto{\pgfqpoint{4.219339in}{2.682421in}}%
\pgfpathlineto{\pgfqpoint{4.219339in}{2.694217in}}%
\pgfpathlineto{\pgfqpoint{4.237503in}{2.694217in}}%
\pgfpathlineto{\pgfqpoint{4.237503in}{2.682421in}}%
\pgfpathmoveto{\pgfqpoint{4.201175in}{2.741405in}}%
\pgfpathlineto{\pgfqpoint{4.201175in}{2.741405in}}%
\pgfpathlineto{\pgfqpoint{4.201175in}{2.753201in}}%
\pgfpathlineto{\pgfqpoint{4.219339in}{2.753201in}}%
\pgfpathlineto{\pgfqpoint{4.219339in}{2.741405in}}%
\pgfpathmoveto{\pgfqpoint{4.201175in}{2.753201in}}%
\pgfpathlineto{\pgfqpoint{4.201175in}{2.753201in}}%
\pgfpathlineto{\pgfqpoint{4.201175in}{2.764998in}}%
\pgfpathlineto{\pgfqpoint{4.219339in}{2.764998in}}%
\pgfpathlineto{\pgfqpoint{4.219339in}{2.753201in}}%
\pgfpathmoveto{\pgfqpoint{4.219339in}{2.741405in}}%
\pgfpathlineto{\pgfqpoint{4.219339in}{2.741405in}}%
\pgfpathlineto{\pgfqpoint{4.219339in}{2.753201in}}%
\pgfpathlineto{\pgfqpoint{4.237503in}{2.753201in}}%
\pgfpathlineto{\pgfqpoint{4.237503in}{2.741405in}}%
\pgfpathmoveto{\pgfqpoint{4.128517in}{2.788592in}}%
\pgfpathlineto{\pgfqpoint{4.128517in}{2.788592in}}%
\pgfpathlineto{\pgfqpoint{4.128517in}{2.800390in}}%
\pgfpathlineto{\pgfqpoint{4.146682in}{2.800390in}}%
\pgfpathlineto{\pgfqpoint{4.146682in}{2.788592in}}%
\pgfpathmoveto{\pgfqpoint{4.128517in}{2.800390in}}%
\pgfpathlineto{\pgfqpoint{4.128517in}{2.800390in}}%
\pgfpathlineto{\pgfqpoint{4.128517in}{2.812187in}}%
\pgfpathlineto{\pgfqpoint{4.146682in}{2.812187in}}%
\pgfpathlineto{\pgfqpoint{4.146682in}{2.800390in}}%
\pgfpathmoveto{\pgfqpoint{4.146682in}{2.788592in}}%
\pgfpathlineto{\pgfqpoint{4.146682in}{2.788592in}}%
\pgfpathlineto{\pgfqpoint{4.146682in}{2.800390in}}%
\pgfpathlineto{\pgfqpoint{4.164846in}{2.800390in}}%
\pgfpathlineto{\pgfqpoint{4.164846in}{2.788592in}}%
\pgfpathmoveto{\pgfqpoint{4.092189in}{2.812187in}}%
\pgfpathlineto{\pgfqpoint{4.092189in}{2.812187in}}%
\pgfpathlineto{\pgfqpoint{4.092189in}{2.823984in}}%
\pgfpathlineto{\pgfqpoint{4.110353in}{2.823984in}}%
\pgfpathlineto{\pgfqpoint{4.110353in}{2.812187in}}%
\pgfpathmoveto{\pgfqpoint{4.092189in}{2.823984in}}%
\pgfpathlineto{\pgfqpoint{4.092189in}{2.823984in}}%
\pgfpathlineto{\pgfqpoint{4.092189in}{2.835781in}}%
\pgfpathlineto{\pgfqpoint{4.110353in}{2.835781in}}%
\pgfpathlineto{\pgfqpoint{4.110353in}{2.823984in}}%
\pgfpathmoveto{\pgfqpoint{4.110353in}{2.812187in}}%
\pgfpathlineto{\pgfqpoint{4.110353in}{2.812187in}}%
\pgfpathlineto{\pgfqpoint{4.110353in}{2.823984in}}%
\pgfpathlineto{\pgfqpoint{4.128517in}{2.823984in}}%
\pgfpathlineto{\pgfqpoint{4.128517in}{2.812187in}}%
\pgfpathmoveto{\pgfqpoint{4.164846in}{2.764998in}}%
\pgfpathlineto{\pgfqpoint{4.164846in}{2.764998in}}%
\pgfpathlineto{\pgfqpoint{4.164846in}{2.776795in}}%
\pgfpathlineto{\pgfqpoint{4.183010in}{2.776795in}}%
\pgfpathlineto{\pgfqpoint{4.183010in}{2.764998in}}%
\pgfpathmoveto{\pgfqpoint{4.164846in}{2.776795in}}%
\pgfpathlineto{\pgfqpoint{4.164846in}{2.776795in}}%
\pgfpathlineto{\pgfqpoint{4.164846in}{2.788592in}}%
\pgfpathlineto{\pgfqpoint{4.183010in}{2.788592in}}%
\pgfpathlineto{\pgfqpoint{4.183010in}{2.776795in}}%
\pgfpathmoveto{\pgfqpoint{4.183010in}{2.764998in}}%
\pgfpathlineto{\pgfqpoint{4.183010in}{2.764998in}}%
\pgfpathlineto{\pgfqpoint{4.183010in}{2.776795in}}%
\pgfpathlineto{\pgfqpoint{4.201175in}{2.776795in}}%
\pgfpathlineto{\pgfqpoint{4.201175in}{2.764998in}}%
\pgfpathmoveto{\pgfqpoint{4.273830in}{2.658827in}}%
\pgfpathlineto{\pgfqpoint{4.273830in}{2.658827in}}%
\pgfpathlineto{\pgfqpoint{4.273830in}{2.670624in}}%
\pgfpathlineto{\pgfqpoint{4.291994in}{2.670624in}}%
\pgfpathlineto{\pgfqpoint{4.291994in}{2.658827in}}%
\pgfpathmoveto{\pgfqpoint{4.291994in}{2.658827in}}%
\pgfpathlineto{\pgfqpoint{4.291994in}{2.658827in}}%
\pgfpathlineto{\pgfqpoint{4.291994in}{2.670624in}}%
\pgfpathlineto{\pgfqpoint{4.310157in}{2.670624in}}%
\pgfpathlineto{\pgfqpoint{4.310157in}{2.658827in}}%
\pgfpathmoveto{\pgfqpoint{4.310157in}{2.658827in}}%
\pgfpathlineto{\pgfqpoint{4.310157in}{2.658827in}}%
\pgfpathlineto{\pgfqpoint{4.310157in}{2.670624in}}%
\pgfpathlineto{\pgfqpoint{4.328321in}{2.670624in}}%
\pgfpathlineto{\pgfqpoint{4.328321in}{2.658827in}}%
\pgfpathmoveto{\pgfqpoint{4.328321in}{2.647030in}}%
\pgfpathlineto{\pgfqpoint{4.328321in}{2.647030in}}%
\pgfpathlineto{\pgfqpoint{4.328321in}{2.658827in}}%
\pgfpathlineto{\pgfqpoint{4.346484in}{2.658827in}}%
\pgfpathlineto{\pgfqpoint{4.346484in}{2.647030in}}%
\pgfpathmoveto{\pgfqpoint{4.328321in}{2.658827in}}%
\pgfpathlineto{\pgfqpoint{4.328321in}{2.658827in}}%
\pgfpathlineto{\pgfqpoint{4.328321in}{2.670624in}}%
\pgfpathlineto{\pgfqpoint{4.346484in}{2.670624in}}%
\pgfpathlineto{\pgfqpoint{4.346484in}{2.658827in}}%
\pgfpathmoveto{\pgfqpoint{4.237503in}{2.741405in}}%
\pgfpathlineto{\pgfqpoint{4.237503in}{2.741405in}}%
\pgfpathlineto{\pgfqpoint{4.237503in}{2.753201in}}%
\pgfpathlineto{\pgfqpoint{4.255667in}{2.753201in}}%
\pgfpathlineto{\pgfqpoint{4.255667in}{2.741405in}}%
\pgfpathmoveto{\pgfqpoint{4.273830in}{2.717811in}}%
\pgfpathlineto{\pgfqpoint{4.273830in}{2.717811in}}%
\pgfpathlineto{\pgfqpoint{4.273830in}{2.729608in}}%
\pgfpathlineto{\pgfqpoint{4.291994in}{2.729608in}}%
\pgfpathlineto{\pgfqpoint{4.291994in}{2.717811in}}%
\pgfpathmoveto{\pgfqpoint{4.310157in}{2.694217in}}%
\pgfpathlineto{\pgfqpoint{4.310157in}{2.694217in}}%
\pgfpathlineto{\pgfqpoint{4.310157in}{2.706014in}}%
\pgfpathlineto{\pgfqpoint{4.328321in}{2.706014in}}%
\pgfpathlineto{\pgfqpoint{4.328321in}{2.694217in}}%
\pgfpathmoveto{\pgfqpoint{4.346484in}{2.670624in}}%
\pgfpathlineto{\pgfqpoint{4.346484in}{2.670624in}}%
\pgfpathlineto{\pgfqpoint{4.346484in}{2.682421in}}%
\pgfpathlineto{\pgfqpoint{4.364648in}{2.682421in}}%
\pgfpathlineto{\pgfqpoint{4.364648in}{2.670624in}}%
\pgfpathmoveto{\pgfqpoint{4.382811in}{2.635233in}}%
\pgfpathlineto{\pgfqpoint{4.382811in}{2.635233in}}%
\pgfpathlineto{\pgfqpoint{4.382811in}{2.647030in}}%
\pgfpathlineto{\pgfqpoint{4.400976in}{2.647030in}}%
\pgfpathlineto{\pgfqpoint{4.400976in}{2.635233in}}%
\pgfpathmoveto{\pgfqpoint{4.400976in}{2.635233in}}%
\pgfpathlineto{\pgfqpoint{4.400976in}{2.635233in}}%
\pgfpathlineto{\pgfqpoint{4.400976in}{2.647030in}}%
\pgfpathlineto{\pgfqpoint{4.419140in}{2.647030in}}%
\pgfpathlineto{\pgfqpoint{4.419140in}{2.635233in}}%
\pgfpathmoveto{\pgfqpoint{4.382811in}{2.647030in}}%
\pgfpathlineto{\pgfqpoint{4.382811in}{2.647030in}}%
\pgfpathlineto{\pgfqpoint{4.382811in}{2.658827in}}%
\pgfpathlineto{\pgfqpoint{4.400976in}{2.658827in}}%
\pgfpathlineto{\pgfqpoint{4.400976in}{2.647030in}}%
\pgfpathmoveto{\pgfqpoint{3.093165in}{2.912462in}}%
\pgfpathlineto{\pgfqpoint{3.093165in}{2.912462in}}%
\pgfpathlineto{\pgfqpoint{3.093165in}{2.918361in}}%
\pgfpathlineto{\pgfqpoint{3.102247in}{2.918361in}}%
\pgfpathlineto{\pgfqpoint{3.102247in}{2.912462in}}%
\pgfpathmoveto{\pgfqpoint{3.102247in}{2.912462in}}%
\pgfpathlineto{\pgfqpoint{3.102247in}{2.912462in}}%
\pgfpathlineto{\pgfqpoint{3.102247in}{2.918361in}}%
\pgfpathlineto{\pgfqpoint{3.111329in}{2.918361in}}%
\pgfpathlineto{\pgfqpoint{3.111329in}{2.912462in}}%
\pgfpathmoveto{\pgfqpoint{3.111329in}{2.912462in}}%
\pgfpathlineto{\pgfqpoint{3.111329in}{2.912462in}}%
\pgfpathlineto{\pgfqpoint{3.111329in}{2.918361in}}%
\pgfpathlineto{\pgfqpoint{3.120411in}{2.918361in}}%
\pgfpathlineto{\pgfqpoint{3.120411in}{2.912462in}}%
\pgfpathmoveto{\pgfqpoint{3.120411in}{2.906564in}}%
\pgfpathlineto{\pgfqpoint{3.120411in}{2.906564in}}%
\pgfpathlineto{\pgfqpoint{3.120411in}{2.912462in}}%
\pgfpathlineto{\pgfqpoint{3.129493in}{2.912462in}}%
\pgfpathlineto{\pgfqpoint{3.129493in}{2.906564in}}%
\pgfpathmoveto{\pgfqpoint{3.120411in}{2.912462in}}%
\pgfpathlineto{\pgfqpoint{3.120411in}{2.912462in}}%
\pgfpathlineto{\pgfqpoint{3.120411in}{2.918361in}}%
\pgfpathlineto{\pgfqpoint{3.129493in}{2.918361in}}%
\pgfpathlineto{\pgfqpoint{3.129493in}{2.912462in}}%
\pgfpathmoveto{\pgfqpoint{3.147657in}{2.900665in}}%
\pgfpathlineto{\pgfqpoint{3.147657in}{2.900665in}}%
\pgfpathlineto{\pgfqpoint{3.147657in}{2.906564in}}%
\pgfpathlineto{\pgfqpoint{3.156740in}{2.906564in}}%
\pgfpathlineto{\pgfqpoint{3.156740in}{2.900665in}}%
\pgfpathmoveto{\pgfqpoint{3.156740in}{2.900665in}}%
\pgfpathlineto{\pgfqpoint{3.156740in}{2.900665in}}%
\pgfpathlineto{\pgfqpoint{3.156740in}{2.906564in}}%
\pgfpathlineto{\pgfqpoint{3.165822in}{2.906564in}}%
\pgfpathlineto{\pgfqpoint{3.165822in}{2.900665in}}%
\pgfpathmoveto{\pgfqpoint{3.165822in}{2.900665in}}%
\pgfpathlineto{\pgfqpoint{3.165822in}{2.900665in}}%
\pgfpathlineto{\pgfqpoint{3.165822in}{2.906564in}}%
\pgfpathlineto{\pgfqpoint{3.174904in}{2.906564in}}%
\pgfpathlineto{\pgfqpoint{3.174904in}{2.900665in}}%
\pgfpathmoveto{\pgfqpoint{3.174904in}{2.894767in}}%
\pgfpathlineto{\pgfqpoint{3.174904in}{2.894767in}}%
\pgfpathlineto{\pgfqpoint{3.174904in}{2.900665in}}%
\pgfpathlineto{\pgfqpoint{3.183986in}{2.900665in}}%
\pgfpathlineto{\pgfqpoint{3.183986in}{2.894767in}}%
\pgfpathmoveto{\pgfqpoint{3.174904in}{2.900665in}}%
\pgfpathlineto{\pgfqpoint{3.174904in}{2.900665in}}%
\pgfpathlineto{\pgfqpoint{3.174904in}{2.906564in}}%
\pgfpathlineto{\pgfqpoint{3.183986in}{2.906564in}}%
\pgfpathlineto{\pgfqpoint{3.183986in}{2.900665in}}%
\pgfpathmoveto{\pgfqpoint{3.202150in}{2.888868in}}%
\pgfpathlineto{\pgfqpoint{3.202150in}{2.888868in}}%
\pgfpathlineto{\pgfqpoint{3.202150in}{2.894767in}}%
\pgfpathlineto{\pgfqpoint{3.211232in}{2.894767in}}%
\pgfpathlineto{\pgfqpoint{3.211232in}{2.888868in}}%
\pgfpathmoveto{\pgfqpoint{3.211232in}{2.888868in}}%
\pgfpathlineto{\pgfqpoint{3.211232in}{2.888868in}}%
\pgfpathlineto{\pgfqpoint{3.211232in}{2.894767in}}%
\pgfpathlineto{\pgfqpoint{3.220314in}{2.894767in}}%
\pgfpathlineto{\pgfqpoint{3.220314in}{2.888868in}}%
\pgfpathmoveto{\pgfqpoint{3.202150in}{3.413828in}}%
\pgfpathlineto{\pgfqpoint{3.202150in}{3.413828in}}%
\pgfpathlineto{\pgfqpoint{3.202150in}{3.419726in}}%
\pgfpathlineto{\pgfqpoint{3.211232in}{3.419726in}}%
\pgfpathlineto{\pgfqpoint{3.211232in}{3.413828in}}%
\pgfpathmoveto{\pgfqpoint{3.202150in}{3.419726in}}%
\pgfpathlineto{\pgfqpoint{3.202150in}{3.419726in}}%
\pgfpathlineto{\pgfqpoint{3.202150in}{3.425625in}}%
\pgfpathlineto{\pgfqpoint{3.211232in}{3.425625in}}%
\pgfpathlineto{\pgfqpoint{3.211232in}{3.419726in}}%
\pgfpathmoveto{\pgfqpoint{3.211232in}{3.413828in}}%
\pgfpathlineto{\pgfqpoint{3.211232in}{3.413828in}}%
\pgfpathlineto{\pgfqpoint{3.211232in}{3.419726in}}%
\pgfpathlineto{\pgfqpoint{3.220314in}{3.419726in}}%
\pgfpathlineto{\pgfqpoint{3.220314in}{3.413828in}}%
\pgfpathmoveto{\pgfqpoint{3.129493in}{3.461016in}}%
\pgfpathlineto{\pgfqpoint{3.129493in}{3.461016in}}%
\pgfpathlineto{\pgfqpoint{3.129493in}{3.466915in}}%
\pgfpathlineto{\pgfqpoint{3.138575in}{3.466915in}}%
\pgfpathlineto{\pgfqpoint{3.138575in}{3.461016in}}%
\pgfpathmoveto{\pgfqpoint{3.129493in}{3.466915in}}%
\pgfpathlineto{\pgfqpoint{3.129493in}{3.466915in}}%
\pgfpathlineto{\pgfqpoint{3.129493in}{3.472813in}}%
\pgfpathlineto{\pgfqpoint{3.138575in}{3.472813in}}%
\pgfpathlineto{\pgfqpoint{3.138575in}{3.466915in}}%
\pgfpathmoveto{\pgfqpoint{3.138575in}{3.461016in}}%
\pgfpathlineto{\pgfqpoint{3.138575in}{3.461016in}}%
\pgfpathlineto{\pgfqpoint{3.138575in}{3.466915in}}%
\pgfpathlineto{\pgfqpoint{3.147657in}{3.466915in}}%
\pgfpathlineto{\pgfqpoint{3.147657in}{3.461016in}}%
\pgfpathmoveto{\pgfqpoint{3.093165in}{3.484610in}}%
\pgfpathlineto{\pgfqpoint{3.093165in}{3.484610in}}%
\pgfpathlineto{\pgfqpoint{3.093165in}{3.490509in}}%
\pgfpathlineto{\pgfqpoint{3.102247in}{3.490509in}}%
\pgfpathlineto{\pgfqpoint{3.102247in}{3.484610in}}%
\pgfpathmoveto{\pgfqpoint{3.093165in}{3.490509in}}%
\pgfpathlineto{\pgfqpoint{3.093165in}{3.490509in}}%
\pgfpathlineto{\pgfqpoint{3.093165in}{3.496408in}}%
\pgfpathlineto{\pgfqpoint{3.102247in}{3.496408in}}%
\pgfpathlineto{\pgfqpoint{3.102247in}{3.490509in}}%
\pgfpathmoveto{\pgfqpoint{3.102247in}{3.484610in}}%
\pgfpathlineto{\pgfqpoint{3.102247in}{3.484610in}}%
\pgfpathlineto{\pgfqpoint{3.102247in}{3.490509in}}%
\pgfpathlineto{\pgfqpoint{3.111329in}{3.490509in}}%
\pgfpathlineto{\pgfqpoint{3.111329in}{3.484610in}}%
\pgfpathmoveto{\pgfqpoint{3.075001in}{3.496408in}}%
\pgfpathlineto{\pgfqpoint{3.075001in}{3.496408in}}%
\pgfpathlineto{\pgfqpoint{3.075001in}{3.502306in}}%
\pgfpathlineto{\pgfqpoint{3.084083in}{3.502306in}}%
\pgfpathlineto{\pgfqpoint{3.084083in}{3.496408in}}%
\pgfpathmoveto{\pgfqpoint{3.075001in}{3.502306in}}%
\pgfpathlineto{\pgfqpoint{3.075001in}{3.502306in}}%
\pgfpathlineto{\pgfqpoint{3.075001in}{3.508205in}}%
\pgfpathlineto{\pgfqpoint{3.084083in}{3.508205in}}%
\pgfpathlineto{\pgfqpoint{3.084083in}{3.502306in}}%
\pgfpathmoveto{\pgfqpoint{3.084083in}{3.496408in}}%
\pgfpathlineto{\pgfqpoint{3.084083in}{3.496408in}}%
\pgfpathlineto{\pgfqpoint{3.084083in}{3.502306in}}%
\pgfpathlineto{\pgfqpoint{3.093165in}{3.502306in}}%
\pgfpathlineto{\pgfqpoint{3.093165in}{3.496408in}}%
\pgfpathmoveto{\pgfqpoint{3.111329in}{3.472813in}}%
\pgfpathlineto{\pgfqpoint{3.111329in}{3.472813in}}%
\pgfpathlineto{\pgfqpoint{3.111329in}{3.478712in}}%
\pgfpathlineto{\pgfqpoint{3.120411in}{3.478712in}}%
\pgfpathlineto{\pgfqpoint{3.120411in}{3.472813in}}%
\pgfpathmoveto{\pgfqpoint{3.111329in}{3.478712in}}%
\pgfpathlineto{\pgfqpoint{3.111329in}{3.478712in}}%
\pgfpathlineto{\pgfqpoint{3.111329in}{3.484610in}}%
\pgfpathlineto{\pgfqpoint{3.120411in}{3.484610in}}%
\pgfpathlineto{\pgfqpoint{3.120411in}{3.478712in}}%
\pgfpathmoveto{\pgfqpoint{3.120411in}{3.472813in}}%
\pgfpathlineto{\pgfqpoint{3.120411in}{3.472813in}}%
\pgfpathlineto{\pgfqpoint{3.120411in}{3.478712in}}%
\pgfpathlineto{\pgfqpoint{3.129493in}{3.478712in}}%
\pgfpathlineto{\pgfqpoint{3.129493in}{3.472813in}}%
\pgfpathmoveto{\pgfqpoint{3.165822in}{3.437422in}}%
\pgfpathlineto{\pgfqpoint{3.165822in}{3.437422in}}%
\pgfpathlineto{\pgfqpoint{3.165822in}{3.443321in}}%
\pgfpathlineto{\pgfqpoint{3.174904in}{3.443321in}}%
\pgfpathlineto{\pgfqpoint{3.174904in}{3.437422in}}%
\pgfpathmoveto{\pgfqpoint{3.165822in}{3.443321in}}%
\pgfpathlineto{\pgfqpoint{3.165822in}{3.443321in}}%
\pgfpathlineto{\pgfqpoint{3.165822in}{3.449219in}}%
\pgfpathlineto{\pgfqpoint{3.174904in}{3.449219in}}%
\pgfpathlineto{\pgfqpoint{3.174904in}{3.443321in}}%
\pgfpathmoveto{\pgfqpoint{3.174904in}{3.437422in}}%
\pgfpathlineto{\pgfqpoint{3.174904in}{3.437422in}}%
\pgfpathlineto{\pgfqpoint{3.174904in}{3.443321in}}%
\pgfpathlineto{\pgfqpoint{3.183986in}{3.443321in}}%
\pgfpathlineto{\pgfqpoint{3.183986in}{3.437422in}}%
\pgfpathmoveto{\pgfqpoint{3.147657in}{3.449219in}}%
\pgfpathlineto{\pgfqpoint{3.147657in}{3.449219in}}%
\pgfpathlineto{\pgfqpoint{3.147657in}{3.455118in}}%
\pgfpathlineto{\pgfqpoint{3.156740in}{3.455118in}}%
\pgfpathlineto{\pgfqpoint{3.156740in}{3.449219in}}%
\pgfpathmoveto{\pgfqpoint{3.147657in}{3.455118in}}%
\pgfpathlineto{\pgfqpoint{3.147657in}{3.455118in}}%
\pgfpathlineto{\pgfqpoint{3.147657in}{3.461016in}}%
\pgfpathlineto{\pgfqpoint{3.156740in}{3.461016in}}%
\pgfpathlineto{\pgfqpoint{3.156740in}{3.455118in}}%
\pgfpathmoveto{\pgfqpoint{3.156740in}{3.449219in}}%
\pgfpathlineto{\pgfqpoint{3.156740in}{3.449219in}}%
\pgfpathlineto{\pgfqpoint{3.156740in}{3.455118in}}%
\pgfpathlineto{\pgfqpoint{3.165822in}{3.455118in}}%
\pgfpathlineto{\pgfqpoint{3.165822in}{3.449219in}}%
\pgfpathmoveto{\pgfqpoint{3.183986in}{3.425625in}}%
\pgfpathlineto{\pgfqpoint{3.183986in}{3.425625in}}%
\pgfpathlineto{\pgfqpoint{3.183986in}{3.431524in}}%
\pgfpathlineto{\pgfqpoint{3.193068in}{3.431524in}}%
\pgfpathlineto{\pgfqpoint{3.193068in}{3.425625in}}%
\pgfpathmoveto{\pgfqpoint{3.183986in}{3.431524in}}%
\pgfpathlineto{\pgfqpoint{3.183986in}{3.431524in}}%
\pgfpathlineto{\pgfqpoint{3.183986in}{3.437422in}}%
\pgfpathlineto{\pgfqpoint{3.193068in}{3.437422in}}%
\pgfpathlineto{\pgfqpoint{3.193068in}{3.431524in}}%
\pgfpathmoveto{\pgfqpoint{3.193068in}{3.425625in}}%
\pgfpathlineto{\pgfqpoint{3.193068in}{3.425625in}}%
\pgfpathlineto{\pgfqpoint{3.193068in}{3.431524in}}%
\pgfpathlineto{\pgfqpoint{3.202150in}{3.431524in}}%
\pgfpathlineto{\pgfqpoint{3.202150in}{3.425625in}}%
\pgfpathmoveto{\pgfqpoint{3.220314in}{2.888868in}}%
\pgfpathlineto{\pgfqpoint{3.220314in}{2.888868in}}%
\pgfpathlineto{\pgfqpoint{3.220314in}{2.894767in}}%
\pgfpathlineto{\pgfqpoint{3.229396in}{2.894767in}}%
\pgfpathlineto{\pgfqpoint{3.229396in}{2.888868in}}%
\pgfpathmoveto{\pgfqpoint{3.229396in}{2.882970in}}%
\pgfpathlineto{\pgfqpoint{3.229396in}{2.882970in}}%
\pgfpathlineto{\pgfqpoint{3.229396in}{2.888868in}}%
\pgfpathlineto{\pgfqpoint{3.238478in}{2.888868in}}%
\pgfpathlineto{\pgfqpoint{3.238478in}{2.882970in}}%
\pgfpathmoveto{\pgfqpoint{3.229396in}{2.888868in}}%
\pgfpathlineto{\pgfqpoint{3.229396in}{2.888868in}}%
\pgfpathlineto{\pgfqpoint{3.229396in}{2.894767in}}%
\pgfpathlineto{\pgfqpoint{3.238478in}{2.894767in}}%
\pgfpathlineto{\pgfqpoint{3.238478in}{2.888868in}}%
\pgfpathmoveto{\pgfqpoint{3.256641in}{2.877071in}}%
\pgfpathlineto{\pgfqpoint{3.256641in}{2.877071in}}%
\pgfpathlineto{\pgfqpoint{3.256641in}{2.882970in}}%
\pgfpathlineto{\pgfqpoint{3.265723in}{2.882970in}}%
\pgfpathlineto{\pgfqpoint{3.265723in}{2.877071in}}%
\pgfpathmoveto{\pgfqpoint{3.265723in}{2.877071in}}%
\pgfpathlineto{\pgfqpoint{3.265723in}{2.877071in}}%
\pgfpathlineto{\pgfqpoint{3.265723in}{2.882970in}}%
\pgfpathlineto{\pgfqpoint{3.274804in}{2.882970in}}%
\pgfpathlineto{\pgfqpoint{3.274804in}{2.877071in}}%
\pgfpathmoveto{\pgfqpoint{3.274804in}{2.877071in}}%
\pgfpathlineto{\pgfqpoint{3.274804in}{2.877071in}}%
\pgfpathlineto{\pgfqpoint{3.274804in}{2.882970in}}%
\pgfpathlineto{\pgfqpoint{3.283886in}{2.882970in}}%
\pgfpathlineto{\pgfqpoint{3.283886in}{2.877071in}}%
\pgfpathmoveto{\pgfqpoint{3.283886in}{2.871173in}}%
\pgfpathlineto{\pgfqpoint{3.283886in}{2.871173in}}%
\pgfpathlineto{\pgfqpoint{3.283886in}{2.877071in}}%
\pgfpathlineto{\pgfqpoint{3.292968in}{2.877071in}}%
\pgfpathlineto{\pgfqpoint{3.292968in}{2.871173in}}%
\pgfpathmoveto{\pgfqpoint{3.283886in}{2.877071in}}%
\pgfpathlineto{\pgfqpoint{3.283886in}{2.877071in}}%
\pgfpathlineto{\pgfqpoint{3.283886in}{2.882970in}}%
\pgfpathlineto{\pgfqpoint{3.292968in}{2.882970in}}%
\pgfpathlineto{\pgfqpoint{3.292968in}{2.877071in}}%
\pgfpathmoveto{\pgfqpoint{3.311131in}{2.865274in}}%
\pgfpathlineto{\pgfqpoint{3.311131in}{2.865274in}}%
\pgfpathlineto{\pgfqpoint{3.311131in}{2.871173in}}%
\pgfpathlineto{\pgfqpoint{3.320213in}{2.871173in}}%
\pgfpathlineto{\pgfqpoint{3.320213in}{2.865274in}}%
\pgfpathmoveto{\pgfqpoint{3.320213in}{2.865274in}}%
\pgfpathlineto{\pgfqpoint{3.320213in}{2.865274in}}%
\pgfpathlineto{\pgfqpoint{3.320213in}{2.871173in}}%
\pgfpathlineto{\pgfqpoint{3.329294in}{2.871173in}}%
\pgfpathlineto{\pgfqpoint{3.329294in}{2.865274in}}%
\pgfpathmoveto{\pgfqpoint{3.329294in}{2.865274in}}%
\pgfpathlineto{\pgfqpoint{3.329294in}{2.865274in}}%
\pgfpathlineto{\pgfqpoint{3.329294in}{2.871173in}}%
\pgfpathlineto{\pgfqpoint{3.338376in}{2.871173in}}%
\pgfpathlineto{\pgfqpoint{3.338376in}{2.865274in}}%
\pgfpathmoveto{\pgfqpoint{3.338376in}{2.859375in}}%
\pgfpathlineto{\pgfqpoint{3.338376in}{2.859375in}}%
\pgfpathlineto{\pgfqpoint{3.338376in}{2.865274in}}%
\pgfpathlineto{\pgfqpoint{3.347458in}{2.865274in}}%
\pgfpathlineto{\pgfqpoint{3.347458in}{2.859375in}}%
\pgfpathmoveto{\pgfqpoint{3.338376in}{2.865274in}}%
\pgfpathlineto{\pgfqpoint{3.338376in}{2.865274in}}%
\pgfpathlineto{\pgfqpoint{3.338376in}{2.871173in}}%
\pgfpathlineto{\pgfqpoint{3.347458in}{2.871173in}}%
\pgfpathlineto{\pgfqpoint{3.347458in}{2.865274in}}%
\pgfpathmoveto{\pgfqpoint{3.220314in}{3.402031in}}%
\pgfpathlineto{\pgfqpoint{3.220314in}{3.402031in}}%
\pgfpathlineto{\pgfqpoint{3.220314in}{3.407929in}}%
\pgfpathlineto{\pgfqpoint{3.229396in}{3.407929in}}%
\pgfpathlineto{\pgfqpoint{3.229396in}{3.402031in}}%
\pgfpathmoveto{\pgfqpoint{3.220314in}{3.407929in}}%
\pgfpathlineto{\pgfqpoint{3.220314in}{3.407929in}}%
\pgfpathlineto{\pgfqpoint{3.220314in}{3.413828in}}%
\pgfpathlineto{\pgfqpoint{3.229396in}{3.413828in}}%
\pgfpathlineto{\pgfqpoint{3.229396in}{3.407929in}}%
\pgfpathmoveto{\pgfqpoint{3.229396in}{3.402031in}}%
\pgfpathlineto{\pgfqpoint{3.229396in}{3.402031in}}%
\pgfpathlineto{\pgfqpoint{3.229396in}{3.407929in}}%
\pgfpathlineto{\pgfqpoint{3.238478in}{3.407929in}}%
\pgfpathlineto{\pgfqpoint{3.238478in}{3.402031in}}%
\pgfpathmoveto{\pgfqpoint{3.256641in}{3.390233in}}%
\pgfpathlineto{\pgfqpoint{3.256641in}{3.390233in}}%
\pgfpathlineto{\pgfqpoint{3.256641in}{3.396132in}}%
\pgfpathlineto{\pgfqpoint{3.265723in}{3.396132in}}%
\pgfpathlineto{\pgfqpoint{3.265723in}{3.390233in}}%
\pgfpathmoveto{\pgfqpoint{3.274804in}{3.378436in}}%
\pgfpathlineto{\pgfqpoint{3.274804in}{3.378436in}}%
\pgfpathlineto{\pgfqpoint{3.274804in}{3.384335in}}%
\pgfpathlineto{\pgfqpoint{3.283886in}{3.384335in}}%
\pgfpathlineto{\pgfqpoint{3.283886in}{3.378436in}}%
\pgfpathmoveto{\pgfqpoint{3.292968in}{3.366639in}}%
\pgfpathlineto{\pgfqpoint{3.292968in}{3.366639in}}%
\pgfpathlineto{\pgfqpoint{3.292968in}{3.372538in}}%
\pgfpathlineto{\pgfqpoint{3.302049in}{3.372538in}}%
\pgfpathlineto{\pgfqpoint{3.302049in}{3.366639in}}%
\pgfpathmoveto{\pgfqpoint{3.311131in}{3.354842in}}%
\pgfpathlineto{\pgfqpoint{3.311131in}{3.354842in}}%
\pgfpathlineto{\pgfqpoint{3.311131in}{3.360741in}}%
\pgfpathlineto{\pgfqpoint{3.320213in}{3.360741in}}%
\pgfpathlineto{\pgfqpoint{3.320213in}{3.354842in}}%
\pgfpathmoveto{\pgfqpoint{3.329294in}{3.343045in}}%
\pgfpathlineto{\pgfqpoint{3.329294in}{3.343045in}}%
\pgfpathlineto{\pgfqpoint{3.329294in}{3.348943in}}%
\pgfpathlineto{\pgfqpoint{3.338376in}{3.348943in}}%
\pgfpathlineto{\pgfqpoint{3.338376in}{3.343045in}}%
\pgfpathmoveto{\pgfqpoint{3.347458in}{3.331248in}}%
\pgfpathlineto{\pgfqpoint{3.347458in}{3.331248in}}%
\pgfpathlineto{\pgfqpoint{3.347458in}{3.337146in}}%
\pgfpathlineto{\pgfqpoint{3.356539in}{3.337146in}}%
\pgfpathlineto{\pgfqpoint{3.356539in}{3.331248in}}%
\pgfpathmoveto{\pgfqpoint{3.365621in}{2.853477in}}%
\pgfpathlineto{\pgfqpoint{3.365621in}{2.853477in}}%
\pgfpathlineto{\pgfqpoint{3.365621in}{2.859375in}}%
\pgfpathlineto{\pgfqpoint{3.374704in}{2.859375in}}%
\pgfpathlineto{\pgfqpoint{3.374704in}{2.853477in}}%
\pgfpathmoveto{\pgfqpoint{3.374704in}{2.853477in}}%
\pgfpathlineto{\pgfqpoint{3.374704in}{2.853477in}}%
\pgfpathlineto{\pgfqpoint{3.374704in}{2.859375in}}%
\pgfpathlineto{\pgfqpoint{3.383786in}{2.859375in}}%
\pgfpathlineto{\pgfqpoint{3.383786in}{2.853477in}}%
\pgfpathmoveto{\pgfqpoint{3.383786in}{2.853477in}}%
\pgfpathlineto{\pgfqpoint{3.383786in}{2.853477in}}%
\pgfpathlineto{\pgfqpoint{3.383786in}{2.859375in}}%
\pgfpathlineto{\pgfqpoint{3.392869in}{2.859375in}}%
\pgfpathlineto{\pgfqpoint{3.392869in}{2.853477in}}%
\pgfpathmoveto{\pgfqpoint{3.392869in}{2.847578in}}%
\pgfpathlineto{\pgfqpoint{3.392869in}{2.847578in}}%
\pgfpathlineto{\pgfqpoint{3.392869in}{2.853477in}}%
\pgfpathlineto{\pgfqpoint{3.401951in}{2.853477in}}%
\pgfpathlineto{\pgfqpoint{3.401951in}{2.847578in}}%
\pgfpathmoveto{\pgfqpoint{3.392869in}{2.853477in}}%
\pgfpathlineto{\pgfqpoint{3.392869in}{2.853477in}}%
\pgfpathlineto{\pgfqpoint{3.392869in}{2.859375in}}%
\pgfpathlineto{\pgfqpoint{3.401951in}{2.859375in}}%
\pgfpathlineto{\pgfqpoint{3.401951in}{2.853477in}}%
\pgfpathmoveto{\pgfqpoint{3.420116in}{2.841680in}}%
\pgfpathlineto{\pgfqpoint{3.420116in}{2.841680in}}%
\pgfpathlineto{\pgfqpoint{3.420116in}{2.847578in}}%
\pgfpathlineto{\pgfqpoint{3.429199in}{2.847578in}}%
\pgfpathlineto{\pgfqpoint{3.429199in}{2.841680in}}%
\pgfpathmoveto{\pgfqpoint{3.429199in}{2.841680in}}%
\pgfpathlineto{\pgfqpoint{3.429199in}{2.841680in}}%
\pgfpathlineto{\pgfqpoint{3.429199in}{2.847578in}}%
\pgfpathlineto{\pgfqpoint{3.438281in}{2.847578in}}%
\pgfpathlineto{\pgfqpoint{3.438281in}{2.841680in}}%
\pgfpathmoveto{\pgfqpoint{3.438281in}{2.841680in}}%
\pgfpathlineto{\pgfqpoint{3.438281in}{2.841680in}}%
\pgfpathlineto{\pgfqpoint{3.438281in}{2.847578in}}%
\pgfpathlineto{\pgfqpoint{3.447364in}{2.847578in}}%
\pgfpathlineto{\pgfqpoint{3.447364in}{2.841680in}}%
\pgfpathmoveto{\pgfqpoint{3.447364in}{2.835781in}}%
\pgfpathlineto{\pgfqpoint{3.447364in}{2.835781in}}%
\pgfpathlineto{\pgfqpoint{3.447364in}{2.841680in}}%
\pgfpathlineto{\pgfqpoint{3.456447in}{2.841680in}}%
\pgfpathlineto{\pgfqpoint{3.456447in}{2.835781in}}%
\pgfpathmoveto{\pgfqpoint{3.447364in}{2.841680in}}%
\pgfpathlineto{\pgfqpoint{3.447364in}{2.841680in}}%
\pgfpathlineto{\pgfqpoint{3.447364in}{2.847578in}}%
\pgfpathlineto{\pgfqpoint{3.456447in}{2.847578in}}%
\pgfpathlineto{\pgfqpoint{3.456447in}{2.841680in}}%
\pgfpathmoveto{\pgfqpoint{3.474612in}{2.829883in}}%
\pgfpathlineto{\pgfqpoint{3.474612in}{2.829883in}}%
\pgfpathlineto{\pgfqpoint{3.474612in}{2.835781in}}%
\pgfpathlineto{\pgfqpoint{3.483694in}{2.835781in}}%
\pgfpathlineto{\pgfqpoint{3.483694in}{2.829883in}}%
\pgfpathmoveto{\pgfqpoint{3.483694in}{2.829883in}}%
\pgfpathlineto{\pgfqpoint{3.483694in}{2.829883in}}%
\pgfpathlineto{\pgfqpoint{3.483694in}{2.835781in}}%
\pgfpathlineto{\pgfqpoint{3.492777in}{2.835781in}}%
\pgfpathlineto{\pgfqpoint{3.492777in}{2.829883in}}%
\pgfpathmoveto{\pgfqpoint{3.492777in}{2.829883in}}%
\pgfpathlineto{\pgfqpoint{3.492777in}{2.829883in}}%
\pgfpathlineto{\pgfqpoint{3.492777in}{2.835781in}}%
\pgfpathlineto{\pgfqpoint{3.501859in}{2.835781in}}%
\pgfpathlineto{\pgfqpoint{3.501859in}{2.829883in}}%
\pgfpathmoveto{\pgfqpoint{3.501859in}{2.823984in}}%
\pgfpathlineto{\pgfqpoint{3.501859in}{2.823984in}}%
\pgfpathlineto{\pgfqpoint{3.501859in}{2.829883in}}%
\pgfpathlineto{\pgfqpoint{3.510942in}{2.829883in}}%
\pgfpathlineto{\pgfqpoint{3.510942in}{2.823984in}}%
\pgfpathmoveto{\pgfqpoint{3.501859in}{2.829883in}}%
\pgfpathlineto{\pgfqpoint{3.501859in}{2.829883in}}%
\pgfpathlineto{\pgfqpoint{3.501859in}{2.835781in}}%
\pgfpathlineto{\pgfqpoint{3.510942in}{2.835781in}}%
\pgfpathlineto{\pgfqpoint{3.510942in}{2.829883in}}%
\pgfpathmoveto{\pgfqpoint{3.492777in}{3.225077in}}%
\pgfpathlineto{\pgfqpoint{3.492777in}{3.225077in}}%
\pgfpathlineto{\pgfqpoint{3.492777in}{3.230976in}}%
\pgfpathlineto{\pgfqpoint{3.501859in}{3.230976in}}%
\pgfpathlineto{\pgfqpoint{3.501859in}{3.225077in}}%
\pgfpathmoveto{\pgfqpoint{3.492777in}{3.230976in}}%
\pgfpathlineto{\pgfqpoint{3.492777in}{3.230976in}}%
\pgfpathlineto{\pgfqpoint{3.492777in}{3.236874in}}%
\pgfpathlineto{\pgfqpoint{3.501859in}{3.236874in}}%
\pgfpathlineto{\pgfqpoint{3.501859in}{3.230976in}}%
\pgfpathmoveto{\pgfqpoint{3.501859in}{3.225077in}}%
\pgfpathlineto{\pgfqpoint{3.501859in}{3.225077in}}%
\pgfpathlineto{\pgfqpoint{3.501859in}{3.230976in}}%
\pgfpathlineto{\pgfqpoint{3.510942in}{3.230976in}}%
\pgfpathlineto{\pgfqpoint{3.510942in}{3.225077in}}%
\pgfpathmoveto{\pgfqpoint{3.365621in}{3.319451in}}%
\pgfpathlineto{\pgfqpoint{3.365621in}{3.319451in}}%
\pgfpathlineto{\pgfqpoint{3.365621in}{3.325349in}}%
\pgfpathlineto{\pgfqpoint{3.374704in}{3.325349in}}%
\pgfpathlineto{\pgfqpoint{3.374704in}{3.319451in}}%
\pgfpathmoveto{\pgfqpoint{3.383786in}{3.307654in}}%
\pgfpathlineto{\pgfqpoint{3.383786in}{3.307654in}}%
\pgfpathlineto{\pgfqpoint{3.383786in}{3.313553in}}%
\pgfpathlineto{\pgfqpoint{3.392869in}{3.313553in}}%
\pgfpathlineto{\pgfqpoint{3.392869in}{3.307654in}}%
\pgfpathmoveto{\pgfqpoint{3.401951in}{3.295858in}}%
\pgfpathlineto{\pgfqpoint{3.401951in}{3.295858in}}%
\pgfpathlineto{\pgfqpoint{3.401951in}{3.301756in}}%
\pgfpathlineto{\pgfqpoint{3.411034in}{3.301756in}}%
\pgfpathlineto{\pgfqpoint{3.411034in}{3.295858in}}%
\pgfpathmoveto{\pgfqpoint{3.420116in}{3.284061in}}%
\pgfpathlineto{\pgfqpoint{3.420116in}{3.284061in}}%
\pgfpathlineto{\pgfqpoint{3.420116in}{3.289959in}}%
\pgfpathlineto{\pgfqpoint{3.429199in}{3.289959in}}%
\pgfpathlineto{\pgfqpoint{3.429199in}{3.284061in}}%
\pgfpathmoveto{\pgfqpoint{3.456447in}{3.248671in}}%
\pgfpathlineto{\pgfqpoint{3.456447in}{3.248671in}}%
\pgfpathlineto{\pgfqpoint{3.456447in}{3.254569in}}%
\pgfpathlineto{\pgfqpoint{3.465529in}{3.254569in}}%
\pgfpathlineto{\pgfqpoint{3.465529in}{3.248671in}}%
\pgfpathmoveto{\pgfqpoint{3.456447in}{3.254569in}}%
\pgfpathlineto{\pgfqpoint{3.456447in}{3.254569in}}%
\pgfpathlineto{\pgfqpoint{3.456447in}{3.260468in}}%
\pgfpathlineto{\pgfqpoint{3.465529in}{3.260468in}}%
\pgfpathlineto{\pgfqpoint{3.465529in}{3.254569in}}%
\pgfpathmoveto{\pgfqpoint{3.465529in}{3.248671in}}%
\pgfpathlineto{\pgfqpoint{3.465529in}{3.248671in}}%
\pgfpathlineto{\pgfqpoint{3.465529in}{3.254569in}}%
\pgfpathlineto{\pgfqpoint{3.474612in}{3.254569in}}%
\pgfpathlineto{\pgfqpoint{3.474612in}{3.248671in}}%
\pgfpathmoveto{\pgfqpoint{3.438281in}{3.272264in}}%
\pgfpathlineto{\pgfqpoint{3.438281in}{3.272264in}}%
\pgfpathlineto{\pgfqpoint{3.438281in}{3.278163in}}%
\pgfpathlineto{\pgfqpoint{3.447364in}{3.278163in}}%
\pgfpathlineto{\pgfqpoint{3.447364in}{3.272264in}}%
\pgfpathmoveto{\pgfqpoint{3.456447in}{3.260468in}}%
\pgfpathlineto{\pgfqpoint{3.456447in}{3.260468in}}%
\pgfpathlineto{\pgfqpoint{3.456447in}{3.266366in}}%
\pgfpathlineto{\pgfqpoint{3.465529in}{3.266366in}}%
\pgfpathlineto{\pgfqpoint{3.465529in}{3.260468in}}%
\pgfpathmoveto{\pgfqpoint{3.474612in}{3.236874in}}%
\pgfpathlineto{\pgfqpoint{3.474612in}{3.236874in}}%
\pgfpathlineto{\pgfqpoint{3.474612in}{3.242773in}}%
\pgfpathlineto{\pgfqpoint{3.483694in}{3.242773in}}%
\pgfpathlineto{\pgfqpoint{3.483694in}{3.236874in}}%
\pgfpathmoveto{\pgfqpoint{3.474612in}{3.242773in}}%
\pgfpathlineto{\pgfqpoint{3.474612in}{3.242773in}}%
\pgfpathlineto{\pgfqpoint{3.474612in}{3.248671in}}%
\pgfpathlineto{\pgfqpoint{3.483694in}{3.248671in}}%
\pgfpathlineto{\pgfqpoint{3.483694in}{3.242773in}}%
\pgfpathmoveto{\pgfqpoint{3.483694in}{3.236874in}}%
\pgfpathlineto{\pgfqpoint{3.483694in}{3.236874in}}%
\pgfpathlineto{\pgfqpoint{3.483694in}{3.242773in}}%
\pgfpathlineto{\pgfqpoint{3.492777in}{3.242773in}}%
\pgfpathlineto{\pgfqpoint{3.492777in}{3.236874in}}%
\pgfpathmoveto{\pgfqpoint{3.529106in}{2.818085in}}%
\pgfpathlineto{\pgfqpoint{3.529106in}{2.818085in}}%
\pgfpathlineto{\pgfqpoint{3.529106in}{2.823984in}}%
\pgfpathlineto{\pgfqpoint{3.538188in}{2.823984in}}%
\pgfpathlineto{\pgfqpoint{3.538188in}{2.818085in}}%
\pgfpathmoveto{\pgfqpoint{3.538188in}{2.818085in}}%
\pgfpathlineto{\pgfqpoint{3.538188in}{2.818085in}}%
\pgfpathlineto{\pgfqpoint{3.538188in}{2.823984in}}%
\pgfpathlineto{\pgfqpoint{3.547269in}{2.823984in}}%
\pgfpathlineto{\pgfqpoint{3.547269in}{2.818085in}}%
\pgfpathmoveto{\pgfqpoint{3.547269in}{2.818085in}}%
\pgfpathlineto{\pgfqpoint{3.547269in}{2.818085in}}%
\pgfpathlineto{\pgfqpoint{3.547269in}{2.823984in}}%
\pgfpathlineto{\pgfqpoint{3.556351in}{2.823984in}}%
\pgfpathlineto{\pgfqpoint{3.556351in}{2.818085in}}%
\pgfpathmoveto{\pgfqpoint{3.556351in}{2.812187in}}%
\pgfpathlineto{\pgfqpoint{3.556351in}{2.812187in}}%
\pgfpathlineto{\pgfqpoint{3.556351in}{2.818085in}}%
\pgfpathlineto{\pgfqpoint{3.565433in}{2.818085in}}%
\pgfpathlineto{\pgfqpoint{3.565433in}{2.812187in}}%
\pgfpathmoveto{\pgfqpoint{3.556351in}{2.818085in}}%
\pgfpathlineto{\pgfqpoint{3.556351in}{2.818085in}}%
\pgfpathlineto{\pgfqpoint{3.556351in}{2.823984in}}%
\pgfpathlineto{\pgfqpoint{3.565433in}{2.823984in}}%
\pgfpathlineto{\pgfqpoint{3.565433in}{2.818085in}}%
\pgfpathmoveto{\pgfqpoint{3.583597in}{2.806288in}}%
\pgfpathlineto{\pgfqpoint{3.583597in}{2.806288in}}%
\pgfpathlineto{\pgfqpoint{3.583597in}{2.812187in}}%
\pgfpathlineto{\pgfqpoint{3.592679in}{2.812187in}}%
\pgfpathlineto{\pgfqpoint{3.592679in}{2.806288in}}%
\pgfpathmoveto{\pgfqpoint{3.592679in}{2.806288in}}%
\pgfpathlineto{\pgfqpoint{3.592679in}{2.806288in}}%
\pgfpathlineto{\pgfqpoint{3.592679in}{2.812187in}}%
\pgfpathlineto{\pgfqpoint{3.601761in}{2.812187in}}%
\pgfpathlineto{\pgfqpoint{3.601761in}{2.806288in}}%
\pgfpathmoveto{\pgfqpoint{3.601761in}{2.806288in}}%
\pgfpathlineto{\pgfqpoint{3.601761in}{2.806288in}}%
\pgfpathlineto{\pgfqpoint{3.601761in}{2.812187in}}%
\pgfpathlineto{\pgfqpoint{3.610843in}{2.812187in}}%
\pgfpathlineto{\pgfqpoint{3.610843in}{2.806288in}}%
\pgfpathmoveto{\pgfqpoint{3.610843in}{2.800390in}}%
\pgfpathlineto{\pgfqpoint{3.610843in}{2.800390in}}%
\pgfpathlineto{\pgfqpoint{3.610843in}{2.806288in}}%
\pgfpathlineto{\pgfqpoint{3.619925in}{2.806288in}}%
\pgfpathlineto{\pgfqpoint{3.619925in}{2.800390in}}%
\pgfpathmoveto{\pgfqpoint{3.610843in}{2.806288in}}%
\pgfpathlineto{\pgfqpoint{3.610843in}{2.806288in}}%
\pgfpathlineto{\pgfqpoint{3.610843in}{2.812187in}}%
\pgfpathlineto{\pgfqpoint{3.619925in}{2.812187in}}%
\pgfpathlineto{\pgfqpoint{3.619925in}{2.806288in}}%
\pgfpathmoveto{\pgfqpoint{3.638089in}{2.794491in}}%
\pgfpathlineto{\pgfqpoint{3.638089in}{2.794491in}}%
\pgfpathlineto{\pgfqpoint{3.638089in}{2.800390in}}%
\pgfpathlineto{\pgfqpoint{3.647171in}{2.800390in}}%
\pgfpathlineto{\pgfqpoint{3.647171in}{2.794491in}}%
\pgfpathmoveto{\pgfqpoint{3.647171in}{2.794491in}}%
\pgfpathlineto{\pgfqpoint{3.647171in}{2.794491in}}%
\pgfpathlineto{\pgfqpoint{3.647171in}{2.800390in}}%
\pgfpathlineto{\pgfqpoint{3.656252in}{2.800390in}}%
\pgfpathlineto{\pgfqpoint{3.656252in}{2.794491in}}%
\pgfpathmoveto{\pgfqpoint{3.638089in}{3.130703in}}%
\pgfpathlineto{\pgfqpoint{3.638089in}{3.130703in}}%
\pgfpathlineto{\pgfqpoint{3.638089in}{3.136601in}}%
\pgfpathlineto{\pgfqpoint{3.647171in}{3.136601in}}%
\pgfpathlineto{\pgfqpoint{3.647171in}{3.130703in}}%
\pgfpathmoveto{\pgfqpoint{3.638089in}{3.136601in}}%
\pgfpathlineto{\pgfqpoint{3.638089in}{3.136601in}}%
\pgfpathlineto{\pgfqpoint{3.638089in}{3.142499in}}%
\pgfpathlineto{\pgfqpoint{3.647171in}{3.142499in}}%
\pgfpathlineto{\pgfqpoint{3.647171in}{3.136601in}}%
\pgfpathmoveto{\pgfqpoint{3.647171in}{3.130703in}}%
\pgfpathlineto{\pgfqpoint{3.647171in}{3.130703in}}%
\pgfpathlineto{\pgfqpoint{3.647171in}{3.136601in}}%
\pgfpathlineto{\pgfqpoint{3.656252in}{3.136601in}}%
\pgfpathlineto{\pgfqpoint{3.656252in}{3.130703in}}%
\pgfpathmoveto{\pgfqpoint{3.565433in}{3.177890in}}%
\pgfpathlineto{\pgfqpoint{3.565433in}{3.177890in}}%
\pgfpathlineto{\pgfqpoint{3.565433in}{3.183788in}}%
\pgfpathlineto{\pgfqpoint{3.574515in}{3.183788in}}%
\pgfpathlineto{\pgfqpoint{3.574515in}{3.177890in}}%
\pgfpathmoveto{\pgfqpoint{3.565433in}{3.183788in}}%
\pgfpathlineto{\pgfqpoint{3.565433in}{3.183788in}}%
\pgfpathlineto{\pgfqpoint{3.565433in}{3.189687in}}%
\pgfpathlineto{\pgfqpoint{3.574515in}{3.189687in}}%
\pgfpathlineto{\pgfqpoint{3.574515in}{3.183788in}}%
\pgfpathmoveto{\pgfqpoint{3.574515in}{3.177890in}}%
\pgfpathlineto{\pgfqpoint{3.574515in}{3.177890in}}%
\pgfpathlineto{\pgfqpoint{3.574515in}{3.183788in}}%
\pgfpathlineto{\pgfqpoint{3.583597in}{3.183788in}}%
\pgfpathlineto{\pgfqpoint{3.583597in}{3.177890in}}%
\pgfpathmoveto{\pgfqpoint{3.529106in}{3.201484in}}%
\pgfpathlineto{\pgfqpoint{3.529106in}{3.201484in}}%
\pgfpathlineto{\pgfqpoint{3.529106in}{3.207382in}}%
\pgfpathlineto{\pgfqpoint{3.538188in}{3.207382in}}%
\pgfpathlineto{\pgfqpoint{3.538188in}{3.201484in}}%
\pgfpathmoveto{\pgfqpoint{3.529106in}{3.207382in}}%
\pgfpathlineto{\pgfqpoint{3.529106in}{3.207382in}}%
\pgfpathlineto{\pgfqpoint{3.529106in}{3.213281in}}%
\pgfpathlineto{\pgfqpoint{3.538188in}{3.213281in}}%
\pgfpathlineto{\pgfqpoint{3.538188in}{3.207382in}}%
\pgfpathmoveto{\pgfqpoint{3.538188in}{3.201484in}}%
\pgfpathlineto{\pgfqpoint{3.538188in}{3.201484in}}%
\pgfpathlineto{\pgfqpoint{3.538188in}{3.207382in}}%
\pgfpathlineto{\pgfqpoint{3.547269in}{3.207382in}}%
\pgfpathlineto{\pgfqpoint{3.547269in}{3.201484in}}%
\pgfpathmoveto{\pgfqpoint{3.510942in}{3.213281in}}%
\pgfpathlineto{\pgfqpoint{3.510942in}{3.213281in}}%
\pgfpathlineto{\pgfqpoint{3.510942in}{3.219179in}}%
\pgfpathlineto{\pgfqpoint{3.520024in}{3.219179in}}%
\pgfpathlineto{\pgfqpoint{3.520024in}{3.213281in}}%
\pgfpathmoveto{\pgfqpoint{3.510942in}{3.219179in}}%
\pgfpathlineto{\pgfqpoint{3.510942in}{3.219179in}}%
\pgfpathlineto{\pgfqpoint{3.510942in}{3.225077in}}%
\pgfpathlineto{\pgfqpoint{3.520024in}{3.225077in}}%
\pgfpathlineto{\pgfqpoint{3.520024in}{3.219179in}}%
\pgfpathmoveto{\pgfqpoint{3.520024in}{3.213281in}}%
\pgfpathlineto{\pgfqpoint{3.520024in}{3.213281in}}%
\pgfpathlineto{\pgfqpoint{3.520024in}{3.219179in}}%
\pgfpathlineto{\pgfqpoint{3.529106in}{3.219179in}}%
\pgfpathlineto{\pgfqpoint{3.529106in}{3.213281in}}%
\pgfpathmoveto{\pgfqpoint{3.547269in}{3.189687in}}%
\pgfpathlineto{\pgfqpoint{3.547269in}{3.189687in}}%
\pgfpathlineto{\pgfqpoint{3.547269in}{3.195585in}}%
\pgfpathlineto{\pgfqpoint{3.556351in}{3.195585in}}%
\pgfpathlineto{\pgfqpoint{3.556351in}{3.189687in}}%
\pgfpathmoveto{\pgfqpoint{3.547269in}{3.195585in}}%
\pgfpathlineto{\pgfqpoint{3.547269in}{3.195585in}}%
\pgfpathlineto{\pgfqpoint{3.547269in}{3.201484in}}%
\pgfpathlineto{\pgfqpoint{3.556351in}{3.201484in}}%
\pgfpathlineto{\pgfqpoint{3.556351in}{3.195585in}}%
\pgfpathmoveto{\pgfqpoint{3.556351in}{3.189687in}}%
\pgfpathlineto{\pgfqpoint{3.556351in}{3.189687in}}%
\pgfpathlineto{\pgfqpoint{3.556351in}{3.195585in}}%
\pgfpathlineto{\pgfqpoint{3.565433in}{3.195585in}}%
\pgfpathlineto{\pgfqpoint{3.565433in}{3.189687in}}%
\pgfpathmoveto{\pgfqpoint{3.601761in}{3.154296in}}%
\pgfpathlineto{\pgfqpoint{3.601761in}{3.154296in}}%
\pgfpathlineto{\pgfqpoint{3.601761in}{3.160195in}}%
\pgfpathlineto{\pgfqpoint{3.610843in}{3.160195in}}%
\pgfpathlineto{\pgfqpoint{3.610843in}{3.154296in}}%
\pgfpathmoveto{\pgfqpoint{3.601761in}{3.160195in}}%
\pgfpathlineto{\pgfqpoint{3.601761in}{3.160195in}}%
\pgfpathlineto{\pgfqpoint{3.601761in}{3.166093in}}%
\pgfpathlineto{\pgfqpoint{3.610843in}{3.166093in}}%
\pgfpathlineto{\pgfqpoint{3.610843in}{3.160195in}}%
\pgfpathmoveto{\pgfqpoint{3.610843in}{3.154296in}}%
\pgfpathlineto{\pgfqpoint{3.610843in}{3.154296in}}%
\pgfpathlineto{\pgfqpoint{3.610843in}{3.160195in}}%
\pgfpathlineto{\pgfqpoint{3.619925in}{3.160195in}}%
\pgfpathlineto{\pgfqpoint{3.619925in}{3.154296in}}%
\pgfpathmoveto{\pgfqpoint{3.583597in}{3.166093in}}%
\pgfpathlineto{\pgfqpoint{3.583597in}{3.166093in}}%
\pgfpathlineto{\pgfqpoint{3.583597in}{3.171992in}}%
\pgfpathlineto{\pgfqpoint{3.592679in}{3.171992in}}%
\pgfpathlineto{\pgfqpoint{3.592679in}{3.166093in}}%
\pgfpathmoveto{\pgfqpoint{3.583597in}{3.171992in}}%
\pgfpathlineto{\pgfqpoint{3.583597in}{3.171992in}}%
\pgfpathlineto{\pgfqpoint{3.583597in}{3.177890in}}%
\pgfpathlineto{\pgfqpoint{3.592679in}{3.177890in}}%
\pgfpathlineto{\pgfqpoint{3.592679in}{3.171992in}}%
\pgfpathmoveto{\pgfqpoint{3.592679in}{3.166093in}}%
\pgfpathlineto{\pgfqpoint{3.592679in}{3.166093in}}%
\pgfpathlineto{\pgfqpoint{3.592679in}{3.171992in}}%
\pgfpathlineto{\pgfqpoint{3.601761in}{3.171992in}}%
\pgfpathlineto{\pgfqpoint{3.601761in}{3.166093in}}%
\pgfpathmoveto{\pgfqpoint{3.619925in}{3.142499in}}%
\pgfpathlineto{\pgfqpoint{3.619925in}{3.142499in}}%
\pgfpathlineto{\pgfqpoint{3.619925in}{3.148398in}}%
\pgfpathlineto{\pgfqpoint{3.629007in}{3.148398in}}%
\pgfpathlineto{\pgfqpoint{3.629007in}{3.142499in}}%
\pgfpathmoveto{\pgfqpoint{3.619925in}{3.148398in}}%
\pgfpathlineto{\pgfqpoint{3.619925in}{3.148398in}}%
\pgfpathlineto{\pgfqpoint{3.619925in}{3.154296in}}%
\pgfpathlineto{\pgfqpoint{3.629007in}{3.154296in}}%
\pgfpathlineto{\pgfqpoint{3.629007in}{3.148398in}}%
\pgfpathmoveto{\pgfqpoint{3.629007in}{3.142499in}}%
\pgfpathlineto{\pgfqpoint{3.629007in}{3.142499in}}%
\pgfpathlineto{\pgfqpoint{3.629007in}{3.148398in}}%
\pgfpathlineto{\pgfqpoint{3.638089in}{3.148398in}}%
\pgfpathlineto{\pgfqpoint{3.638089in}{3.142499in}}%
\pgfpathmoveto{\pgfqpoint{3.656252in}{2.794491in}}%
\pgfpathlineto{\pgfqpoint{3.656252in}{2.794491in}}%
\pgfpathlineto{\pgfqpoint{3.656252in}{2.800390in}}%
\pgfpathlineto{\pgfqpoint{3.665334in}{2.800390in}}%
\pgfpathlineto{\pgfqpoint{3.665334in}{2.794491in}}%
\pgfpathmoveto{\pgfqpoint{3.665334in}{2.788592in}}%
\pgfpathlineto{\pgfqpoint{3.665334in}{2.788592in}}%
\pgfpathlineto{\pgfqpoint{3.665334in}{2.794491in}}%
\pgfpathlineto{\pgfqpoint{3.674416in}{2.794491in}}%
\pgfpathlineto{\pgfqpoint{3.674416in}{2.788592in}}%
\pgfpathmoveto{\pgfqpoint{3.665334in}{2.794491in}}%
\pgfpathlineto{\pgfqpoint{3.665334in}{2.794491in}}%
\pgfpathlineto{\pgfqpoint{3.665334in}{2.800390in}}%
\pgfpathlineto{\pgfqpoint{3.674416in}{2.800390in}}%
\pgfpathlineto{\pgfqpoint{3.674416in}{2.794491in}}%
\pgfpathmoveto{\pgfqpoint{3.692579in}{2.782694in}}%
\pgfpathlineto{\pgfqpoint{3.692579in}{2.782694in}}%
\pgfpathlineto{\pgfqpoint{3.692579in}{2.788592in}}%
\pgfpathlineto{\pgfqpoint{3.701661in}{2.788592in}}%
\pgfpathlineto{\pgfqpoint{3.701661in}{2.782694in}}%
\pgfpathmoveto{\pgfqpoint{3.701661in}{2.782694in}}%
\pgfpathlineto{\pgfqpoint{3.701661in}{2.782694in}}%
\pgfpathlineto{\pgfqpoint{3.701661in}{2.788592in}}%
\pgfpathlineto{\pgfqpoint{3.710743in}{2.788592in}}%
\pgfpathlineto{\pgfqpoint{3.710743in}{2.782694in}}%
\pgfpathmoveto{\pgfqpoint{3.710743in}{2.782694in}}%
\pgfpathlineto{\pgfqpoint{3.710743in}{2.782694in}}%
\pgfpathlineto{\pgfqpoint{3.710743in}{2.788592in}}%
\pgfpathlineto{\pgfqpoint{3.719824in}{2.788592in}}%
\pgfpathlineto{\pgfqpoint{3.719824in}{2.782694in}}%
\pgfpathmoveto{\pgfqpoint{3.719824in}{2.776795in}}%
\pgfpathlineto{\pgfqpoint{3.719824in}{2.776795in}}%
\pgfpathlineto{\pgfqpoint{3.719824in}{2.782694in}}%
\pgfpathlineto{\pgfqpoint{3.728906in}{2.782694in}}%
\pgfpathlineto{\pgfqpoint{3.728906in}{2.776795in}}%
\pgfpathmoveto{\pgfqpoint{3.719824in}{2.782694in}}%
\pgfpathlineto{\pgfqpoint{3.719824in}{2.782694in}}%
\pgfpathlineto{\pgfqpoint{3.719824in}{2.788592in}}%
\pgfpathlineto{\pgfqpoint{3.728906in}{2.788592in}}%
\pgfpathlineto{\pgfqpoint{3.728906in}{2.782694in}}%
\pgfpathmoveto{\pgfqpoint{3.747069in}{2.770897in}}%
\pgfpathlineto{\pgfqpoint{3.747069in}{2.770897in}}%
\pgfpathlineto{\pgfqpoint{3.747069in}{2.776795in}}%
\pgfpathlineto{\pgfqpoint{3.756151in}{2.776795in}}%
\pgfpathlineto{\pgfqpoint{3.756151in}{2.770897in}}%
\pgfpathmoveto{\pgfqpoint{3.756151in}{2.770897in}}%
\pgfpathlineto{\pgfqpoint{3.756151in}{2.770897in}}%
\pgfpathlineto{\pgfqpoint{3.756151in}{2.776795in}}%
\pgfpathlineto{\pgfqpoint{3.765233in}{2.776795in}}%
\pgfpathlineto{\pgfqpoint{3.765233in}{2.770897in}}%
\pgfpathmoveto{\pgfqpoint{3.765233in}{2.770897in}}%
\pgfpathlineto{\pgfqpoint{3.765233in}{2.770897in}}%
\pgfpathlineto{\pgfqpoint{3.765233in}{2.776795in}}%
\pgfpathlineto{\pgfqpoint{3.774315in}{2.776795in}}%
\pgfpathlineto{\pgfqpoint{3.774315in}{2.770897in}}%
\pgfpathmoveto{\pgfqpoint{3.774315in}{2.764998in}}%
\pgfpathlineto{\pgfqpoint{3.774315in}{2.764998in}}%
\pgfpathlineto{\pgfqpoint{3.774315in}{2.770897in}}%
\pgfpathlineto{\pgfqpoint{3.783396in}{2.770897in}}%
\pgfpathlineto{\pgfqpoint{3.783396in}{2.764998in}}%
\pgfpathmoveto{\pgfqpoint{3.774315in}{2.770897in}}%
\pgfpathlineto{\pgfqpoint{3.774315in}{2.770897in}}%
\pgfpathlineto{\pgfqpoint{3.774315in}{2.776795in}}%
\pgfpathlineto{\pgfqpoint{3.783396in}{2.776795in}}%
\pgfpathlineto{\pgfqpoint{3.783396in}{2.770897in}}%
\pgfpathmoveto{\pgfqpoint{3.674416in}{3.107109in}}%
\pgfpathlineto{\pgfqpoint{3.674416in}{3.107109in}}%
\pgfpathlineto{\pgfqpoint{3.674416in}{3.113007in}}%
\pgfpathlineto{\pgfqpoint{3.683498in}{3.113007in}}%
\pgfpathlineto{\pgfqpoint{3.683498in}{3.107109in}}%
\pgfpathmoveto{\pgfqpoint{3.674416in}{3.113007in}}%
\pgfpathlineto{\pgfqpoint{3.674416in}{3.113007in}}%
\pgfpathlineto{\pgfqpoint{3.674416in}{3.118906in}}%
\pgfpathlineto{\pgfqpoint{3.683498in}{3.118906in}}%
\pgfpathlineto{\pgfqpoint{3.683498in}{3.113007in}}%
\pgfpathmoveto{\pgfqpoint{3.683498in}{3.107109in}}%
\pgfpathlineto{\pgfqpoint{3.683498in}{3.107109in}}%
\pgfpathlineto{\pgfqpoint{3.683498in}{3.113007in}}%
\pgfpathlineto{\pgfqpoint{3.692579in}{3.113007in}}%
\pgfpathlineto{\pgfqpoint{3.692579in}{3.107109in}}%
\pgfpathmoveto{\pgfqpoint{3.656252in}{3.118906in}}%
\pgfpathlineto{\pgfqpoint{3.656252in}{3.118906in}}%
\pgfpathlineto{\pgfqpoint{3.656252in}{3.124804in}}%
\pgfpathlineto{\pgfqpoint{3.665334in}{3.124804in}}%
\pgfpathlineto{\pgfqpoint{3.665334in}{3.118906in}}%
\pgfpathmoveto{\pgfqpoint{3.656252in}{3.124804in}}%
\pgfpathlineto{\pgfqpoint{3.656252in}{3.124804in}}%
\pgfpathlineto{\pgfqpoint{3.656252in}{3.130703in}}%
\pgfpathlineto{\pgfqpoint{3.665334in}{3.130703in}}%
\pgfpathlineto{\pgfqpoint{3.665334in}{3.124804in}}%
\pgfpathmoveto{\pgfqpoint{3.665334in}{3.118906in}}%
\pgfpathlineto{\pgfqpoint{3.665334in}{3.118906in}}%
\pgfpathlineto{\pgfqpoint{3.665334in}{3.124804in}}%
\pgfpathlineto{\pgfqpoint{3.674416in}{3.124804in}}%
\pgfpathlineto{\pgfqpoint{3.674416in}{3.118906in}}%
\pgfpathmoveto{\pgfqpoint{3.710743in}{3.095312in}}%
\pgfpathlineto{\pgfqpoint{3.710743in}{3.095312in}}%
\pgfpathlineto{\pgfqpoint{3.710743in}{3.101210in}}%
\pgfpathlineto{\pgfqpoint{3.719824in}{3.101210in}}%
\pgfpathlineto{\pgfqpoint{3.719824in}{3.095312in}}%
\pgfpathmoveto{\pgfqpoint{3.728906in}{3.083515in}}%
\pgfpathlineto{\pgfqpoint{3.728906in}{3.083515in}}%
\pgfpathlineto{\pgfqpoint{3.728906in}{3.089414in}}%
\pgfpathlineto{\pgfqpoint{3.737988in}{3.089414in}}%
\pgfpathlineto{\pgfqpoint{3.737988in}{3.083515in}}%
\pgfpathmoveto{\pgfqpoint{3.747069in}{3.071718in}}%
\pgfpathlineto{\pgfqpoint{3.747069in}{3.071718in}}%
\pgfpathlineto{\pgfqpoint{3.747069in}{3.077617in}}%
\pgfpathlineto{\pgfqpoint{3.756151in}{3.077617in}}%
\pgfpathlineto{\pgfqpoint{3.756151in}{3.071718in}}%
\pgfpathmoveto{\pgfqpoint{3.765233in}{3.059922in}}%
\pgfpathlineto{\pgfqpoint{3.765233in}{3.059922in}}%
\pgfpathlineto{\pgfqpoint{3.765233in}{3.065820in}}%
\pgfpathlineto{\pgfqpoint{3.774315in}{3.065820in}}%
\pgfpathlineto{\pgfqpoint{3.774315in}{3.059922in}}%
\pgfpathmoveto{\pgfqpoint{3.783396in}{3.048125in}}%
\pgfpathlineto{\pgfqpoint{3.783396in}{3.048125in}}%
\pgfpathlineto{\pgfqpoint{3.783396in}{3.054023in}}%
\pgfpathlineto{\pgfqpoint{3.792478in}{3.054023in}}%
\pgfpathlineto{\pgfqpoint{3.792478in}{3.048125in}}%
\pgfpathmoveto{\pgfqpoint{3.801560in}{2.759100in}}%
\pgfpathlineto{\pgfqpoint{3.801560in}{2.759100in}}%
\pgfpathlineto{\pgfqpoint{3.801560in}{2.764998in}}%
\pgfpathlineto{\pgfqpoint{3.810642in}{2.764998in}}%
\pgfpathlineto{\pgfqpoint{3.810642in}{2.759100in}}%
\pgfpathmoveto{\pgfqpoint{3.810642in}{2.759100in}}%
\pgfpathlineto{\pgfqpoint{3.810642in}{2.759100in}}%
\pgfpathlineto{\pgfqpoint{3.810642in}{2.764998in}}%
\pgfpathlineto{\pgfqpoint{3.819724in}{2.764998in}}%
\pgfpathlineto{\pgfqpoint{3.819724in}{2.759100in}}%
\pgfpathmoveto{\pgfqpoint{3.819724in}{2.759100in}}%
\pgfpathlineto{\pgfqpoint{3.819724in}{2.759100in}}%
\pgfpathlineto{\pgfqpoint{3.819724in}{2.764998in}}%
\pgfpathlineto{\pgfqpoint{3.828806in}{2.764998in}}%
\pgfpathlineto{\pgfqpoint{3.828806in}{2.759100in}}%
\pgfpathmoveto{\pgfqpoint{3.828806in}{2.753201in}}%
\pgfpathlineto{\pgfqpoint{3.828806in}{2.753201in}}%
\pgfpathlineto{\pgfqpoint{3.828806in}{2.759100in}}%
\pgfpathlineto{\pgfqpoint{3.837888in}{2.759100in}}%
\pgfpathlineto{\pgfqpoint{3.837888in}{2.753201in}}%
\pgfpathmoveto{\pgfqpoint{3.828806in}{2.759100in}}%
\pgfpathlineto{\pgfqpoint{3.828806in}{2.759100in}}%
\pgfpathlineto{\pgfqpoint{3.828806in}{2.764998in}}%
\pgfpathlineto{\pgfqpoint{3.837888in}{2.764998in}}%
\pgfpathlineto{\pgfqpoint{3.837888in}{2.759100in}}%
\pgfpathmoveto{\pgfqpoint{3.856052in}{2.747303in}}%
\pgfpathlineto{\pgfqpoint{3.856052in}{2.747303in}}%
\pgfpathlineto{\pgfqpoint{3.856052in}{2.753201in}}%
\pgfpathlineto{\pgfqpoint{3.865134in}{2.753201in}}%
\pgfpathlineto{\pgfqpoint{3.865134in}{2.747303in}}%
\pgfpathmoveto{\pgfqpoint{3.865134in}{2.747303in}}%
\pgfpathlineto{\pgfqpoint{3.865134in}{2.747303in}}%
\pgfpathlineto{\pgfqpoint{3.865134in}{2.753201in}}%
\pgfpathlineto{\pgfqpoint{3.874216in}{2.753201in}}%
\pgfpathlineto{\pgfqpoint{3.874216in}{2.747303in}}%
\pgfpathmoveto{\pgfqpoint{3.874216in}{2.747303in}}%
\pgfpathlineto{\pgfqpoint{3.874216in}{2.747303in}}%
\pgfpathlineto{\pgfqpoint{3.874216in}{2.753201in}}%
\pgfpathlineto{\pgfqpoint{3.883298in}{2.753201in}}%
\pgfpathlineto{\pgfqpoint{3.883298in}{2.747303in}}%
\pgfpathmoveto{\pgfqpoint{3.883298in}{2.741405in}}%
\pgfpathlineto{\pgfqpoint{3.883298in}{2.741405in}}%
\pgfpathlineto{\pgfqpoint{3.883298in}{2.747303in}}%
\pgfpathlineto{\pgfqpoint{3.892380in}{2.747303in}}%
\pgfpathlineto{\pgfqpoint{3.892380in}{2.741405in}}%
\pgfpathmoveto{\pgfqpoint{3.883298in}{2.747303in}}%
\pgfpathlineto{\pgfqpoint{3.883298in}{2.747303in}}%
\pgfpathlineto{\pgfqpoint{3.883298in}{2.753201in}}%
\pgfpathlineto{\pgfqpoint{3.892380in}{2.753201in}}%
\pgfpathlineto{\pgfqpoint{3.892380in}{2.747303in}}%
\pgfpathmoveto{\pgfqpoint{3.910544in}{2.735506in}}%
\pgfpathlineto{\pgfqpoint{3.910544in}{2.735506in}}%
\pgfpathlineto{\pgfqpoint{3.910544in}{2.741405in}}%
\pgfpathlineto{\pgfqpoint{3.919627in}{2.741405in}}%
\pgfpathlineto{\pgfqpoint{3.919627in}{2.735506in}}%
\pgfpathmoveto{\pgfqpoint{3.919627in}{2.735506in}}%
\pgfpathlineto{\pgfqpoint{3.919627in}{2.735506in}}%
\pgfpathlineto{\pgfqpoint{3.919627in}{2.741405in}}%
\pgfpathlineto{\pgfqpoint{3.928709in}{2.741405in}}%
\pgfpathlineto{\pgfqpoint{3.928709in}{2.735506in}}%
\pgfpathmoveto{\pgfqpoint{3.928709in}{2.735506in}}%
\pgfpathlineto{\pgfqpoint{3.928709in}{2.735506in}}%
\pgfpathlineto{\pgfqpoint{3.928709in}{2.741405in}}%
\pgfpathlineto{\pgfqpoint{3.937791in}{2.741405in}}%
\pgfpathlineto{\pgfqpoint{3.937791in}{2.735506in}}%
\pgfpathmoveto{\pgfqpoint{3.937791in}{2.729608in}}%
\pgfpathlineto{\pgfqpoint{3.937791in}{2.729608in}}%
\pgfpathlineto{\pgfqpoint{3.937791in}{2.735506in}}%
\pgfpathlineto{\pgfqpoint{3.946873in}{2.735506in}}%
\pgfpathlineto{\pgfqpoint{3.946873in}{2.729608in}}%
\pgfpathmoveto{\pgfqpoint{3.937791in}{2.735506in}}%
\pgfpathlineto{\pgfqpoint{3.937791in}{2.735506in}}%
\pgfpathlineto{\pgfqpoint{3.937791in}{2.741405in}}%
\pgfpathlineto{\pgfqpoint{3.946873in}{2.741405in}}%
\pgfpathlineto{\pgfqpoint{3.946873in}{2.735506in}}%
\pgfpathmoveto{\pgfqpoint{3.928709in}{2.941955in}}%
\pgfpathlineto{\pgfqpoint{3.928709in}{2.941955in}}%
\pgfpathlineto{\pgfqpoint{3.928709in}{2.947853in}}%
\pgfpathlineto{\pgfqpoint{3.937791in}{2.947853in}}%
\pgfpathlineto{\pgfqpoint{3.937791in}{2.941955in}}%
\pgfpathmoveto{\pgfqpoint{3.928709in}{2.947853in}}%
\pgfpathlineto{\pgfqpoint{3.928709in}{2.947853in}}%
\pgfpathlineto{\pgfqpoint{3.928709in}{2.953752in}}%
\pgfpathlineto{\pgfqpoint{3.937791in}{2.953752in}}%
\pgfpathlineto{\pgfqpoint{3.937791in}{2.947853in}}%
\pgfpathmoveto{\pgfqpoint{3.937791in}{2.941955in}}%
\pgfpathlineto{\pgfqpoint{3.937791in}{2.941955in}}%
\pgfpathlineto{\pgfqpoint{3.937791in}{2.947853in}}%
\pgfpathlineto{\pgfqpoint{3.946873in}{2.947853in}}%
\pgfpathlineto{\pgfqpoint{3.946873in}{2.941955in}}%
\pgfpathmoveto{\pgfqpoint{3.801560in}{3.036328in}}%
\pgfpathlineto{\pgfqpoint{3.801560in}{3.036328in}}%
\pgfpathlineto{\pgfqpoint{3.801560in}{3.042226in}}%
\pgfpathlineto{\pgfqpoint{3.810642in}{3.042226in}}%
\pgfpathlineto{\pgfqpoint{3.810642in}{3.036328in}}%
\pgfpathmoveto{\pgfqpoint{3.819724in}{3.024531in}}%
\pgfpathlineto{\pgfqpoint{3.819724in}{3.024531in}}%
\pgfpathlineto{\pgfqpoint{3.819724in}{3.030430in}}%
\pgfpathlineto{\pgfqpoint{3.828806in}{3.030430in}}%
\pgfpathlineto{\pgfqpoint{3.828806in}{3.024531in}}%
\pgfpathmoveto{\pgfqpoint{3.837888in}{3.012735in}}%
\pgfpathlineto{\pgfqpoint{3.837888in}{3.012735in}}%
\pgfpathlineto{\pgfqpoint{3.837888in}{3.018633in}}%
\pgfpathlineto{\pgfqpoint{3.846970in}{3.018633in}}%
\pgfpathlineto{\pgfqpoint{3.846970in}{3.012735in}}%
\pgfpathmoveto{\pgfqpoint{3.856052in}{3.000938in}}%
\pgfpathlineto{\pgfqpoint{3.856052in}{3.000938in}}%
\pgfpathlineto{\pgfqpoint{3.856052in}{3.006837in}}%
\pgfpathlineto{\pgfqpoint{3.865134in}{3.006837in}}%
\pgfpathlineto{\pgfqpoint{3.865134in}{3.000938in}}%
\pgfpathmoveto{\pgfqpoint{3.874216in}{2.989142in}}%
\pgfpathlineto{\pgfqpoint{3.874216in}{2.989142in}}%
\pgfpathlineto{\pgfqpoint{3.874216in}{2.995040in}}%
\pgfpathlineto{\pgfqpoint{3.883298in}{2.995040in}}%
\pgfpathlineto{\pgfqpoint{3.883298in}{2.989142in}}%
\pgfpathmoveto{\pgfqpoint{3.892380in}{2.977345in}}%
\pgfpathlineto{\pgfqpoint{3.892380in}{2.977345in}}%
\pgfpathlineto{\pgfqpoint{3.892380in}{2.983243in}}%
\pgfpathlineto{\pgfqpoint{3.901462in}{2.983243in}}%
\pgfpathlineto{\pgfqpoint{3.901462in}{2.977345in}}%
\pgfpathmoveto{\pgfqpoint{3.910544in}{2.953752in}}%
\pgfpathlineto{\pgfqpoint{3.910544in}{2.953752in}}%
\pgfpathlineto{\pgfqpoint{3.910544in}{2.959650in}}%
\pgfpathlineto{\pgfqpoint{3.919627in}{2.959650in}}%
\pgfpathlineto{\pgfqpoint{3.919627in}{2.953752in}}%
\pgfpathmoveto{\pgfqpoint{3.910544in}{2.959650in}}%
\pgfpathlineto{\pgfqpoint{3.910544in}{2.959650in}}%
\pgfpathlineto{\pgfqpoint{3.910544in}{2.965548in}}%
\pgfpathlineto{\pgfqpoint{3.919627in}{2.965548in}}%
\pgfpathlineto{\pgfqpoint{3.919627in}{2.959650in}}%
\pgfpathmoveto{\pgfqpoint{3.919627in}{2.953752in}}%
\pgfpathlineto{\pgfqpoint{3.919627in}{2.953752in}}%
\pgfpathlineto{\pgfqpoint{3.919627in}{2.959650in}}%
\pgfpathlineto{\pgfqpoint{3.928709in}{2.959650in}}%
\pgfpathlineto{\pgfqpoint{3.928709in}{2.953752in}}%
\pgfpathmoveto{\pgfqpoint{3.910544in}{2.965548in}}%
\pgfpathlineto{\pgfqpoint{3.910544in}{2.965548in}}%
\pgfpathlineto{\pgfqpoint{3.910544in}{2.971447in}}%
\pgfpathlineto{\pgfqpoint{3.919627in}{2.971447in}}%
\pgfpathlineto{\pgfqpoint{3.919627in}{2.965548in}}%
\pgfpathmoveto{\pgfqpoint{3.965037in}{2.723709in}}%
\pgfpathlineto{\pgfqpoint{3.965037in}{2.723709in}}%
\pgfpathlineto{\pgfqpoint{3.965037in}{2.729608in}}%
\pgfpathlineto{\pgfqpoint{3.974119in}{2.729608in}}%
\pgfpathlineto{\pgfqpoint{3.974119in}{2.723709in}}%
\pgfpathmoveto{\pgfqpoint{3.974119in}{2.723709in}}%
\pgfpathlineto{\pgfqpoint{3.974119in}{2.723709in}}%
\pgfpathlineto{\pgfqpoint{3.974119in}{2.729608in}}%
\pgfpathlineto{\pgfqpoint{3.983202in}{2.729608in}}%
\pgfpathlineto{\pgfqpoint{3.983202in}{2.723709in}}%
\pgfpathmoveto{\pgfqpoint{3.983202in}{2.723709in}}%
\pgfpathlineto{\pgfqpoint{3.983202in}{2.723709in}}%
\pgfpathlineto{\pgfqpoint{3.983202in}{2.729608in}}%
\pgfpathlineto{\pgfqpoint{3.992284in}{2.729608in}}%
\pgfpathlineto{\pgfqpoint{3.992284in}{2.723709in}}%
\pgfpathmoveto{\pgfqpoint{3.992284in}{2.717811in}}%
\pgfpathlineto{\pgfqpoint{3.992284in}{2.717811in}}%
\pgfpathlineto{\pgfqpoint{3.992284in}{2.723709in}}%
\pgfpathlineto{\pgfqpoint{4.001366in}{2.723709in}}%
\pgfpathlineto{\pgfqpoint{4.001366in}{2.717811in}}%
\pgfpathmoveto{\pgfqpoint{3.992284in}{2.723709in}}%
\pgfpathlineto{\pgfqpoint{3.992284in}{2.723709in}}%
\pgfpathlineto{\pgfqpoint{3.992284in}{2.729608in}}%
\pgfpathlineto{\pgfqpoint{4.001366in}{2.729608in}}%
\pgfpathlineto{\pgfqpoint{4.001366in}{2.723709in}}%
\pgfpathmoveto{\pgfqpoint{4.019531in}{2.711913in}}%
\pgfpathlineto{\pgfqpoint{4.019531in}{2.711913in}}%
\pgfpathlineto{\pgfqpoint{4.019531in}{2.717811in}}%
\pgfpathlineto{\pgfqpoint{4.028613in}{2.717811in}}%
\pgfpathlineto{\pgfqpoint{4.028613in}{2.711913in}}%
\pgfpathmoveto{\pgfqpoint{4.028613in}{2.711913in}}%
\pgfpathlineto{\pgfqpoint{4.028613in}{2.711913in}}%
\pgfpathlineto{\pgfqpoint{4.028613in}{2.717811in}}%
\pgfpathlineto{\pgfqpoint{4.037695in}{2.717811in}}%
\pgfpathlineto{\pgfqpoint{4.037695in}{2.711913in}}%
\pgfpathmoveto{\pgfqpoint{4.037695in}{2.711913in}}%
\pgfpathlineto{\pgfqpoint{4.037695in}{2.711913in}}%
\pgfpathlineto{\pgfqpoint{4.037695in}{2.717811in}}%
\pgfpathlineto{\pgfqpoint{4.046777in}{2.717811in}}%
\pgfpathlineto{\pgfqpoint{4.046777in}{2.711913in}}%
\pgfpathmoveto{\pgfqpoint{4.046777in}{2.706014in}}%
\pgfpathlineto{\pgfqpoint{4.046777in}{2.706014in}}%
\pgfpathlineto{\pgfqpoint{4.046777in}{2.711913in}}%
\pgfpathlineto{\pgfqpoint{4.055860in}{2.711913in}}%
\pgfpathlineto{\pgfqpoint{4.055860in}{2.706014in}}%
\pgfpathmoveto{\pgfqpoint{4.046777in}{2.711913in}}%
\pgfpathlineto{\pgfqpoint{4.046777in}{2.711913in}}%
\pgfpathlineto{\pgfqpoint{4.046777in}{2.717811in}}%
\pgfpathlineto{\pgfqpoint{4.055860in}{2.717811in}}%
\pgfpathlineto{\pgfqpoint{4.055860in}{2.711913in}}%
\pgfpathmoveto{\pgfqpoint{4.074024in}{2.700116in}}%
\pgfpathlineto{\pgfqpoint{4.074024in}{2.700116in}}%
\pgfpathlineto{\pgfqpoint{4.074024in}{2.706014in}}%
\pgfpathlineto{\pgfqpoint{4.083106in}{2.706014in}}%
\pgfpathlineto{\pgfqpoint{4.083106in}{2.700116in}}%
\pgfpathmoveto{\pgfqpoint{4.083106in}{2.700116in}}%
\pgfpathlineto{\pgfqpoint{4.083106in}{2.700116in}}%
\pgfpathlineto{\pgfqpoint{4.083106in}{2.706014in}}%
\pgfpathlineto{\pgfqpoint{4.092189in}{2.706014in}}%
\pgfpathlineto{\pgfqpoint{4.092189in}{2.700116in}}%
\pgfpathmoveto{\pgfqpoint{4.074024in}{2.847578in}}%
\pgfpathlineto{\pgfqpoint{4.074024in}{2.847578in}}%
\pgfpathlineto{\pgfqpoint{4.074024in}{2.853477in}}%
\pgfpathlineto{\pgfqpoint{4.083106in}{2.853477in}}%
\pgfpathlineto{\pgfqpoint{4.083106in}{2.847578in}}%
\pgfpathmoveto{\pgfqpoint{4.074024in}{2.853477in}}%
\pgfpathlineto{\pgfqpoint{4.074024in}{2.853477in}}%
\pgfpathlineto{\pgfqpoint{4.074024in}{2.859375in}}%
\pgfpathlineto{\pgfqpoint{4.083106in}{2.859375in}}%
\pgfpathlineto{\pgfqpoint{4.083106in}{2.853477in}}%
\pgfpathmoveto{\pgfqpoint{4.083106in}{2.847578in}}%
\pgfpathlineto{\pgfqpoint{4.083106in}{2.847578in}}%
\pgfpathlineto{\pgfqpoint{4.083106in}{2.853477in}}%
\pgfpathlineto{\pgfqpoint{4.092189in}{2.853477in}}%
\pgfpathlineto{\pgfqpoint{4.092189in}{2.847578in}}%
\pgfpathmoveto{\pgfqpoint{4.001366in}{2.894767in}}%
\pgfpathlineto{\pgfqpoint{4.001366in}{2.894767in}}%
\pgfpathlineto{\pgfqpoint{4.001366in}{2.900665in}}%
\pgfpathlineto{\pgfqpoint{4.010448in}{2.900665in}}%
\pgfpathlineto{\pgfqpoint{4.010448in}{2.894767in}}%
\pgfpathmoveto{\pgfqpoint{4.001366in}{2.900665in}}%
\pgfpathlineto{\pgfqpoint{4.001366in}{2.900665in}}%
\pgfpathlineto{\pgfqpoint{4.001366in}{2.906564in}}%
\pgfpathlineto{\pgfqpoint{4.010448in}{2.906564in}}%
\pgfpathlineto{\pgfqpoint{4.010448in}{2.900665in}}%
\pgfpathmoveto{\pgfqpoint{4.010448in}{2.894767in}}%
\pgfpathlineto{\pgfqpoint{4.010448in}{2.894767in}}%
\pgfpathlineto{\pgfqpoint{4.010448in}{2.900665in}}%
\pgfpathlineto{\pgfqpoint{4.019531in}{2.900665in}}%
\pgfpathlineto{\pgfqpoint{4.019531in}{2.894767in}}%
\pgfpathmoveto{\pgfqpoint{3.965037in}{2.918361in}}%
\pgfpathlineto{\pgfqpoint{3.965037in}{2.918361in}}%
\pgfpathlineto{\pgfqpoint{3.965037in}{2.924259in}}%
\pgfpathlineto{\pgfqpoint{3.974119in}{2.924259in}}%
\pgfpathlineto{\pgfqpoint{3.974119in}{2.918361in}}%
\pgfpathmoveto{\pgfqpoint{3.965037in}{2.924259in}}%
\pgfpathlineto{\pgfqpoint{3.965037in}{2.924259in}}%
\pgfpathlineto{\pgfqpoint{3.965037in}{2.930158in}}%
\pgfpathlineto{\pgfqpoint{3.974119in}{2.930158in}}%
\pgfpathlineto{\pgfqpoint{3.974119in}{2.924259in}}%
\pgfpathmoveto{\pgfqpoint{3.974119in}{2.918361in}}%
\pgfpathlineto{\pgfqpoint{3.974119in}{2.918361in}}%
\pgfpathlineto{\pgfqpoint{3.974119in}{2.924259in}}%
\pgfpathlineto{\pgfqpoint{3.983202in}{2.924259in}}%
\pgfpathlineto{\pgfqpoint{3.983202in}{2.918361in}}%
\pgfpathmoveto{\pgfqpoint{3.946873in}{2.930158in}}%
\pgfpathlineto{\pgfqpoint{3.946873in}{2.930158in}}%
\pgfpathlineto{\pgfqpoint{3.946873in}{2.936056in}}%
\pgfpathlineto{\pgfqpoint{3.955955in}{2.936056in}}%
\pgfpathlineto{\pgfqpoint{3.955955in}{2.930158in}}%
\pgfpathmoveto{\pgfqpoint{3.946873in}{2.936056in}}%
\pgfpathlineto{\pgfqpoint{3.946873in}{2.936056in}}%
\pgfpathlineto{\pgfqpoint{3.946873in}{2.941955in}}%
\pgfpathlineto{\pgfqpoint{3.955955in}{2.941955in}}%
\pgfpathlineto{\pgfqpoint{3.955955in}{2.936056in}}%
\pgfpathmoveto{\pgfqpoint{3.955955in}{2.930158in}}%
\pgfpathlineto{\pgfqpoint{3.955955in}{2.930158in}}%
\pgfpathlineto{\pgfqpoint{3.955955in}{2.936056in}}%
\pgfpathlineto{\pgfqpoint{3.965037in}{2.936056in}}%
\pgfpathlineto{\pgfqpoint{3.965037in}{2.930158in}}%
\pgfpathmoveto{\pgfqpoint{3.983202in}{2.906564in}}%
\pgfpathlineto{\pgfqpoint{3.983202in}{2.906564in}}%
\pgfpathlineto{\pgfqpoint{3.983202in}{2.912462in}}%
\pgfpathlineto{\pgfqpoint{3.992284in}{2.912462in}}%
\pgfpathlineto{\pgfqpoint{3.992284in}{2.906564in}}%
\pgfpathmoveto{\pgfqpoint{3.983202in}{2.912462in}}%
\pgfpathlineto{\pgfqpoint{3.983202in}{2.912462in}}%
\pgfpathlineto{\pgfqpoint{3.983202in}{2.918361in}}%
\pgfpathlineto{\pgfqpoint{3.992284in}{2.918361in}}%
\pgfpathlineto{\pgfqpoint{3.992284in}{2.912462in}}%
\pgfpathmoveto{\pgfqpoint{3.992284in}{2.906564in}}%
\pgfpathlineto{\pgfqpoint{3.992284in}{2.906564in}}%
\pgfpathlineto{\pgfqpoint{3.992284in}{2.912462in}}%
\pgfpathlineto{\pgfqpoint{4.001366in}{2.912462in}}%
\pgfpathlineto{\pgfqpoint{4.001366in}{2.906564in}}%
\pgfpathmoveto{\pgfqpoint{4.037695in}{2.871173in}}%
\pgfpathlineto{\pgfqpoint{4.037695in}{2.871173in}}%
\pgfpathlineto{\pgfqpoint{4.037695in}{2.877071in}}%
\pgfpathlineto{\pgfqpoint{4.046777in}{2.877071in}}%
\pgfpathlineto{\pgfqpoint{4.046777in}{2.871173in}}%
\pgfpathmoveto{\pgfqpoint{4.037695in}{2.877071in}}%
\pgfpathlineto{\pgfqpoint{4.037695in}{2.877071in}}%
\pgfpathlineto{\pgfqpoint{4.037695in}{2.882970in}}%
\pgfpathlineto{\pgfqpoint{4.046777in}{2.882970in}}%
\pgfpathlineto{\pgfqpoint{4.046777in}{2.877071in}}%
\pgfpathmoveto{\pgfqpoint{4.046777in}{2.871173in}}%
\pgfpathlineto{\pgfqpoint{4.046777in}{2.871173in}}%
\pgfpathlineto{\pgfqpoint{4.046777in}{2.877071in}}%
\pgfpathlineto{\pgfqpoint{4.055860in}{2.877071in}}%
\pgfpathlineto{\pgfqpoint{4.055860in}{2.871173in}}%
\pgfpathmoveto{\pgfqpoint{4.019531in}{2.882970in}}%
\pgfpathlineto{\pgfqpoint{4.019531in}{2.882970in}}%
\pgfpathlineto{\pgfqpoint{4.019531in}{2.888868in}}%
\pgfpathlineto{\pgfqpoint{4.028613in}{2.888868in}}%
\pgfpathlineto{\pgfqpoint{4.028613in}{2.882970in}}%
\pgfpathmoveto{\pgfqpoint{4.019531in}{2.888868in}}%
\pgfpathlineto{\pgfqpoint{4.019531in}{2.888868in}}%
\pgfpathlineto{\pgfqpoint{4.019531in}{2.894767in}}%
\pgfpathlineto{\pgfqpoint{4.028613in}{2.894767in}}%
\pgfpathlineto{\pgfqpoint{4.028613in}{2.888868in}}%
\pgfpathmoveto{\pgfqpoint{4.028613in}{2.882970in}}%
\pgfpathlineto{\pgfqpoint{4.028613in}{2.882970in}}%
\pgfpathlineto{\pgfqpoint{4.028613in}{2.888868in}}%
\pgfpathlineto{\pgfqpoint{4.037695in}{2.888868in}}%
\pgfpathlineto{\pgfqpoint{4.037695in}{2.882970in}}%
\pgfpathmoveto{\pgfqpoint{4.055860in}{2.859375in}}%
\pgfpathlineto{\pgfqpoint{4.055860in}{2.859375in}}%
\pgfpathlineto{\pgfqpoint{4.055860in}{2.865274in}}%
\pgfpathlineto{\pgfqpoint{4.064942in}{2.865274in}}%
\pgfpathlineto{\pgfqpoint{4.064942in}{2.859375in}}%
\pgfpathmoveto{\pgfqpoint{4.055860in}{2.865274in}}%
\pgfpathlineto{\pgfqpoint{4.055860in}{2.865274in}}%
\pgfpathlineto{\pgfqpoint{4.055860in}{2.871173in}}%
\pgfpathlineto{\pgfqpoint{4.064942in}{2.871173in}}%
\pgfpathlineto{\pgfqpoint{4.064942in}{2.865274in}}%
\pgfpathmoveto{\pgfqpoint{4.064942in}{2.859375in}}%
\pgfpathlineto{\pgfqpoint{4.064942in}{2.859375in}}%
\pgfpathlineto{\pgfqpoint{4.064942in}{2.865274in}}%
\pgfpathlineto{\pgfqpoint{4.074024in}{2.865274in}}%
\pgfpathlineto{\pgfqpoint{4.074024in}{2.859375in}}%
\pgfpathmoveto{\pgfqpoint{4.092189in}{2.700116in}}%
\pgfpathlineto{\pgfqpoint{4.092189in}{2.700116in}}%
\pgfpathlineto{\pgfqpoint{4.092189in}{2.706014in}}%
\pgfpathlineto{\pgfqpoint{4.101271in}{2.706014in}}%
\pgfpathlineto{\pgfqpoint{4.101271in}{2.700116in}}%
\pgfpathmoveto{\pgfqpoint{4.101271in}{2.694217in}}%
\pgfpathlineto{\pgfqpoint{4.101271in}{2.694217in}}%
\pgfpathlineto{\pgfqpoint{4.101271in}{2.700116in}}%
\pgfpathlineto{\pgfqpoint{4.110353in}{2.700116in}}%
\pgfpathlineto{\pgfqpoint{4.110353in}{2.694217in}}%
\pgfpathmoveto{\pgfqpoint{4.101271in}{2.700116in}}%
\pgfpathlineto{\pgfqpoint{4.101271in}{2.700116in}}%
\pgfpathlineto{\pgfqpoint{4.101271in}{2.706014in}}%
\pgfpathlineto{\pgfqpoint{4.110353in}{2.706014in}}%
\pgfpathlineto{\pgfqpoint{4.110353in}{2.700116in}}%
\pgfpathmoveto{\pgfqpoint{4.128517in}{2.688319in}}%
\pgfpathlineto{\pgfqpoint{4.128517in}{2.688319in}}%
\pgfpathlineto{\pgfqpoint{4.128517in}{2.694217in}}%
\pgfpathlineto{\pgfqpoint{4.137599in}{2.694217in}}%
\pgfpathlineto{\pgfqpoint{4.137599in}{2.688319in}}%
\pgfpathmoveto{\pgfqpoint{4.137599in}{2.688319in}}%
\pgfpathlineto{\pgfqpoint{4.137599in}{2.688319in}}%
\pgfpathlineto{\pgfqpoint{4.137599in}{2.694217in}}%
\pgfpathlineto{\pgfqpoint{4.146682in}{2.694217in}}%
\pgfpathlineto{\pgfqpoint{4.146682in}{2.688319in}}%
\pgfpathmoveto{\pgfqpoint{4.146682in}{2.688319in}}%
\pgfpathlineto{\pgfqpoint{4.146682in}{2.688319in}}%
\pgfpathlineto{\pgfqpoint{4.146682in}{2.694217in}}%
\pgfpathlineto{\pgfqpoint{4.155764in}{2.694217in}}%
\pgfpathlineto{\pgfqpoint{4.155764in}{2.688319in}}%
\pgfpathmoveto{\pgfqpoint{4.155764in}{2.682421in}}%
\pgfpathlineto{\pgfqpoint{4.155764in}{2.682421in}}%
\pgfpathlineto{\pgfqpoint{4.155764in}{2.688319in}}%
\pgfpathlineto{\pgfqpoint{4.164846in}{2.688319in}}%
\pgfpathlineto{\pgfqpoint{4.164846in}{2.682421in}}%
\pgfpathmoveto{\pgfqpoint{4.155764in}{2.688319in}}%
\pgfpathlineto{\pgfqpoint{4.155764in}{2.688319in}}%
\pgfpathlineto{\pgfqpoint{4.155764in}{2.694217in}}%
\pgfpathlineto{\pgfqpoint{4.164846in}{2.694217in}}%
\pgfpathlineto{\pgfqpoint{4.164846in}{2.688319in}}%
\pgfpathmoveto{\pgfqpoint{4.183010in}{2.676522in}}%
\pgfpathlineto{\pgfqpoint{4.183010in}{2.676522in}}%
\pgfpathlineto{\pgfqpoint{4.183010in}{2.682421in}}%
\pgfpathlineto{\pgfqpoint{4.192092in}{2.682421in}}%
\pgfpathlineto{\pgfqpoint{4.192092in}{2.676522in}}%
\pgfpathmoveto{\pgfqpoint{4.192092in}{2.676522in}}%
\pgfpathlineto{\pgfqpoint{4.192092in}{2.676522in}}%
\pgfpathlineto{\pgfqpoint{4.192092in}{2.682421in}}%
\pgfpathlineto{\pgfqpoint{4.201175in}{2.682421in}}%
\pgfpathlineto{\pgfqpoint{4.201175in}{2.676522in}}%
\pgfpathmoveto{\pgfqpoint{4.201175in}{2.676522in}}%
\pgfpathlineto{\pgfqpoint{4.201175in}{2.676522in}}%
\pgfpathlineto{\pgfqpoint{4.201175in}{2.682421in}}%
\pgfpathlineto{\pgfqpoint{4.210257in}{2.682421in}}%
\pgfpathlineto{\pgfqpoint{4.210257in}{2.676522in}}%
\pgfpathmoveto{\pgfqpoint{4.210257in}{2.670624in}}%
\pgfpathlineto{\pgfqpoint{4.210257in}{2.670624in}}%
\pgfpathlineto{\pgfqpoint{4.210257in}{2.676522in}}%
\pgfpathlineto{\pgfqpoint{4.219339in}{2.676522in}}%
\pgfpathlineto{\pgfqpoint{4.219339in}{2.670624in}}%
\pgfpathmoveto{\pgfqpoint{4.210257in}{2.676522in}}%
\pgfpathlineto{\pgfqpoint{4.210257in}{2.676522in}}%
\pgfpathlineto{\pgfqpoint{4.210257in}{2.682421in}}%
\pgfpathlineto{\pgfqpoint{4.219339in}{2.682421in}}%
\pgfpathlineto{\pgfqpoint{4.219339in}{2.676522in}}%
\pgfpathmoveto{\pgfqpoint{4.219339in}{2.753201in}}%
\pgfpathlineto{\pgfqpoint{4.219339in}{2.753201in}}%
\pgfpathlineto{\pgfqpoint{4.219339in}{2.759100in}}%
\pgfpathlineto{\pgfqpoint{4.228421in}{2.759100in}}%
\pgfpathlineto{\pgfqpoint{4.228421in}{2.753201in}}%
\pgfpathmoveto{\pgfqpoint{4.219339in}{2.759100in}}%
\pgfpathlineto{\pgfqpoint{4.219339in}{2.759100in}}%
\pgfpathlineto{\pgfqpoint{4.219339in}{2.764998in}}%
\pgfpathlineto{\pgfqpoint{4.228421in}{2.764998in}}%
\pgfpathlineto{\pgfqpoint{4.228421in}{2.759100in}}%
\pgfpathmoveto{\pgfqpoint{4.228421in}{2.753201in}}%
\pgfpathlineto{\pgfqpoint{4.228421in}{2.753201in}}%
\pgfpathlineto{\pgfqpoint{4.228421in}{2.759100in}}%
\pgfpathlineto{\pgfqpoint{4.237503in}{2.759100in}}%
\pgfpathlineto{\pgfqpoint{4.237503in}{2.753201in}}%
\pgfpathmoveto{\pgfqpoint{4.146682in}{2.800390in}}%
\pgfpathlineto{\pgfqpoint{4.146682in}{2.800390in}}%
\pgfpathlineto{\pgfqpoint{4.146682in}{2.806288in}}%
\pgfpathlineto{\pgfqpoint{4.155764in}{2.806288in}}%
\pgfpathlineto{\pgfqpoint{4.155764in}{2.800390in}}%
\pgfpathmoveto{\pgfqpoint{4.146682in}{2.806288in}}%
\pgfpathlineto{\pgfqpoint{4.146682in}{2.806288in}}%
\pgfpathlineto{\pgfqpoint{4.146682in}{2.812187in}}%
\pgfpathlineto{\pgfqpoint{4.155764in}{2.812187in}}%
\pgfpathlineto{\pgfqpoint{4.155764in}{2.806288in}}%
\pgfpathmoveto{\pgfqpoint{4.155764in}{2.800390in}}%
\pgfpathlineto{\pgfqpoint{4.155764in}{2.800390in}}%
\pgfpathlineto{\pgfqpoint{4.155764in}{2.806288in}}%
\pgfpathlineto{\pgfqpoint{4.164846in}{2.806288in}}%
\pgfpathlineto{\pgfqpoint{4.164846in}{2.800390in}}%
\pgfpathmoveto{\pgfqpoint{4.110353in}{2.823984in}}%
\pgfpathlineto{\pgfqpoint{4.110353in}{2.823984in}}%
\pgfpathlineto{\pgfqpoint{4.110353in}{2.829883in}}%
\pgfpathlineto{\pgfqpoint{4.119435in}{2.829883in}}%
\pgfpathlineto{\pgfqpoint{4.119435in}{2.823984in}}%
\pgfpathmoveto{\pgfqpoint{4.110353in}{2.829883in}}%
\pgfpathlineto{\pgfqpoint{4.110353in}{2.829883in}}%
\pgfpathlineto{\pgfqpoint{4.110353in}{2.835781in}}%
\pgfpathlineto{\pgfqpoint{4.119435in}{2.835781in}}%
\pgfpathlineto{\pgfqpoint{4.119435in}{2.829883in}}%
\pgfpathmoveto{\pgfqpoint{4.119435in}{2.823984in}}%
\pgfpathlineto{\pgfqpoint{4.119435in}{2.823984in}}%
\pgfpathlineto{\pgfqpoint{4.119435in}{2.829883in}}%
\pgfpathlineto{\pgfqpoint{4.128517in}{2.829883in}}%
\pgfpathlineto{\pgfqpoint{4.128517in}{2.823984in}}%
\pgfpathmoveto{\pgfqpoint{4.092189in}{2.835781in}}%
\pgfpathlineto{\pgfqpoint{4.092189in}{2.835781in}}%
\pgfpathlineto{\pgfqpoint{4.092189in}{2.841680in}}%
\pgfpathlineto{\pgfqpoint{4.101271in}{2.841680in}}%
\pgfpathlineto{\pgfqpoint{4.101271in}{2.835781in}}%
\pgfpathmoveto{\pgfqpoint{4.092189in}{2.841680in}}%
\pgfpathlineto{\pgfqpoint{4.092189in}{2.841680in}}%
\pgfpathlineto{\pgfqpoint{4.092189in}{2.847578in}}%
\pgfpathlineto{\pgfqpoint{4.101271in}{2.847578in}}%
\pgfpathlineto{\pgfqpoint{4.101271in}{2.841680in}}%
\pgfpathmoveto{\pgfqpoint{4.101271in}{2.835781in}}%
\pgfpathlineto{\pgfqpoint{4.101271in}{2.835781in}}%
\pgfpathlineto{\pgfqpoint{4.101271in}{2.841680in}}%
\pgfpathlineto{\pgfqpoint{4.110353in}{2.841680in}}%
\pgfpathlineto{\pgfqpoint{4.110353in}{2.835781in}}%
\pgfpathmoveto{\pgfqpoint{4.128517in}{2.812187in}}%
\pgfpathlineto{\pgfqpoint{4.128517in}{2.812187in}}%
\pgfpathlineto{\pgfqpoint{4.128517in}{2.818085in}}%
\pgfpathlineto{\pgfqpoint{4.137599in}{2.818085in}}%
\pgfpathlineto{\pgfqpoint{4.137599in}{2.812187in}}%
\pgfpathmoveto{\pgfqpoint{4.128517in}{2.818085in}}%
\pgfpathlineto{\pgfqpoint{4.128517in}{2.818085in}}%
\pgfpathlineto{\pgfqpoint{4.128517in}{2.823984in}}%
\pgfpathlineto{\pgfqpoint{4.137599in}{2.823984in}}%
\pgfpathlineto{\pgfqpoint{4.137599in}{2.818085in}}%
\pgfpathmoveto{\pgfqpoint{4.137599in}{2.812187in}}%
\pgfpathlineto{\pgfqpoint{4.137599in}{2.812187in}}%
\pgfpathlineto{\pgfqpoint{4.137599in}{2.818085in}}%
\pgfpathlineto{\pgfqpoint{4.146682in}{2.818085in}}%
\pgfpathlineto{\pgfqpoint{4.146682in}{2.812187in}}%
\pgfpathmoveto{\pgfqpoint{4.183010in}{2.776795in}}%
\pgfpathlineto{\pgfqpoint{4.183010in}{2.776795in}}%
\pgfpathlineto{\pgfqpoint{4.183010in}{2.782694in}}%
\pgfpathlineto{\pgfqpoint{4.192092in}{2.782694in}}%
\pgfpathlineto{\pgfqpoint{4.192092in}{2.776795in}}%
\pgfpathmoveto{\pgfqpoint{4.183010in}{2.782694in}}%
\pgfpathlineto{\pgfqpoint{4.183010in}{2.782694in}}%
\pgfpathlineto{\pgfqpoint{4.183010in}{2.788592in}}%
\pgfpathlineto{\pgfqpoint{4.192092in}{2.788592in}}%
\pgfpathlineto{\pgfqpoint{4.192092in}{2.782694in}}%
\pgfpathmoveto{\pgfqpoint{4.192092in}{2.776795in}}%
\pgfpathlineto{\pgfqpoint{4.192092in}{2.776795in}}%
\pgfpathlineto{\pgfqpoint{4.192092in}{2.782694in}}%
\pgfpathlineto{\pgfqpoint{4.201175in}{2.782694in}}%
\pgfpathlineto{\pgfqpoint{4.201175in}{2.776795in}}%
\pgfpathmoveto{\pgfqpoint{4.164846in}{2.788592in}}%
\pgfpathlineto{\pgfqpoint{4.164846in}{2.788592in}}%
\pgfpathlineto{\pgfqpoint{4.164846in}{2.794491in}}%
\pgfpathlineto{\pgfqpoint{4.173928in}{2.794491in}}%
\pgfpathlineto{\pgfqpoint{4.173928in}{2.788592in}}%
\pgfpathmoveto{\pgfqpoint{4.164846in}{2.794491in}}%
\pgfpathlineto{\pgfqpoint{4.164846in}{2.794491in}}%
\pgfpathlineto{\pgfqpoint{4.164846in}{2.800390in}}%
\pgfpathlineto{\pgfqpoint{4.173928in}{2.800390in}}%
\pgfpathlineto{\pgfqpoint{4.173928in}{2.794491in}}%
\pgfpathmoveto{\pgfqpoint{4.173928in}{2.788592in}}%
\pgfpathlineto{\pgfqpoint{4.173928in}{2.788592in}}%
\pgfpathlineto{\pgfqpoint{4.173928in}{2.794491in}}%
\pgfpathlineto{\pgfqpoint{4.183010in}{2.794491in}}%
\pgfpathlineto{\pgfqpoint{4.183010in}{2.788592in}}%
\pgfpathmoveto{\pgfqpoint{4.201175in}{2.764998in}}%
\pgfpathlineto{\pgfqpoint{4.201175in}{2.764998in}}%
\pgfpathlineto{\pgfqpoint{4.201175in}{2.770897in}}%
\pgfpathlineto{\pgfqpoint{4.210257in}{2.770897in}}%
\pgfpathlineto{\pgfqpoint{4.210257in}{2.764998in}}%
\pgfpathmoveto{\pgfqpoint{4.201175in}{2.770897in}}%
\pgfpathlineto{\pgfqpoint{4.201175in}{2.770897in}}%
\pgfpathlineto{\pgfqpoint{4.201175in}{2.776795in}}%
\pgfpathlineto{\pgfqpoint{4.210257in}{2.776795in}}%
\pgfpathlineto{\pgfqpoint{4.210257in}{2.770897in}}%
\pgfpathmoveto{\pgfqpoint{4.210257in}{2.764998in}}%
\pgfpathlineto{\pgfqpoint{4.210257in}{2.764998in}}%
\pgfpathlineto{\pgfqpoint{4.210257in}{2.770897in}}%
\pgfpathlineto{\pgfqpoint{4.219339in}{2.770897in}}%
\pgfpathlineto{\pgfqpoint{4.219339in}{2.764998in}}%
\pgfpathmoveto{\pgfqpoint{4.237503in}{2.664725in}}%
\pgfpathlineto{\pgfqpoint{4.237503in}{2.664725in}}%
\pgfpathlineto{\pgfqpoint{4.237503in}{2.670624in}}%
\pgfpathlineto{\pgfqpoint{4.246585in}{2.670624in}}%
\pgfpathlineto{\pgfqpoint{4.246585in}{2.664725in}}%
\pgfpathmoveto{\pgfqpoint{4.246585in}{2.664725in}}%
\pgfpathlineto{\pgfqpoint{4.246585in}{2.664725in}}%
\pgfpathlineto{\pgfqpoint{4.246585in}{2.670624in}}%
\pgfpathlineto{\pgfqpoint{4.255667in}{2.670624in}}%
\pgfpathlineto{\pgfqpoint{4.255667in}{2.664725in}}%
\pgfpathmoveto{\pgfqpoint{4.255667in}{2.664725in}}%
\pgfpathlineto{\pgfqpoint{4.255667in}{2.664725in}}%
\pgfpathlineto{\pgfqpoint{4.255667in}{2.670624in}}%
\pgfpathlineto{\pgfqpoint{4.264748in}{2.670624in}}%
\pgfpathlineto{\pgfqpoint{4.264748in}{2.664725in}}%
\pgfpathmoveto{\pgfqpoint{4.264748in}{2.658827in}}%
\pgfpathlineto{\pgfqpoint{4.264748in}{2.658827in}}%
\pgfpathlineto{\pgfqpoint{4.264748in}{2.664725in}}%
\pgfpathlineto{\pgfqpoint{4.273830in}{2.664725in}}%
\pgfpathlineto{\pgfqpoint{4.273830in}{2.658827in}}%
\pgfpathmoveto{\pgfqpoint{4.264748in}{2.664725in}}%
\pgfpathlineto{\pgfqpoint{4.264748in}{2.664725in}}%
\pgfpathlineto{\pgfqpoint{4.264748in}{2.670624in}}%
\pgfpathlineto{\pgfqpoint{4.273830in}{2.670624in}}%
\pgfpathlineto{\pgfqpoint{4.273830in}{2.664725in}}%
\pgfpathmoveto{\pgfqpoint{4.291994in}{2.652928in}}%
\pgfpathlineto{\pgfqpoint{4.291994in}{2.652928in}}%
\pgfpathlineto{\pgfqpoint{4.291994in}{2.658827in}}%
\pgfpathlineto{\pgfqpoint{4.301075in}{2.658827in}}%
\pgfpathlineto{\pgfqpoint{4.301075in}{2.652928in}}%
\pgfpathmoveto{\pgfqpoint{4.301075in}{2.652928in}}%
\pgfpathlineto{\pgfqpoint{4.301075in}{2.652928in}}%
\pgfpathlineto{\pgfqpoint{4.301075in}{2.658827in}}%
\pgfpathlineto{\pgfqpoint{4.310157in}{2.658827in}}%
\pgfpathlineto{\pgfqpoint{4.310157in}{2.652928in}}%
\pgfpathmoveto{\pgfqpoint{4.310157in}{2.652928in}}%
\pgfpathlineto{\pgfqpoint{4.310157in}{2.652928in}}%
\pgfpathlineto{\pgfqpoint{4.310157in}{2.658827in}}%
\pgfpathlineto{\pgfqpoint{4.319239in}{2.658827in}}%
\pgfpathlineto{\pgfqpoint{4.319239in}{2.652928in}}%
\pgfpathmoveto{\pgfqpoint{4.319239in}{2.647030in}}%
\pgfpathlineto{\pgfqpoint{4.319239in}{2.647030in}}%
\pgfpathlineto{\pgfqpoint{4.319239in}{2.652928in}}%
\pgfpathlineto{\pgfqpoint{4.328321in}{2.652928in}}%
\pgfpathlineto{\pgfqpoint{4.328321in}{2.647030in}}%
\pgfpathmoveto{\pgfqpoint{4.319239in}{2.652928in}}%
\pgfpathlineto{\pgfqpoint{4.319239in}{2.652928in}}%
\pgfpathlineto{\pgfqpoint{4.319239in}{2.658827in}}%
\pgfpathlineto{\pgfqpoint{4.328321in}{2.658827in}}%
\pgfpathlineto{\pgfqpoint{4.328321in}{2.652928in}}%
\pgfpathmoveto{\pgfqpoint{4.346484in}{2.641131in}}%
\pgfpathlineto{\pgfqpoint{4.346484in}{2.641131in}}%
\pgfpathlineto{\pgfqpoint{4.346484in}{2.647030in}}%
\pgfpathlineto{\pgfqpoint{4.355566in}{2.647030in}}%
\pgfpathlineto{\pgfqpoint{4.355566in}{2.641131in}}%
\pgfpathmoveto{\pgfqpoint{4.355566in}{2.641131in}}%
\pgfpathlineto{\pgfqpoint{4.355566in}{2.641131in}}%
\pgfpathlineto{\pgfqpoint{4.355566in}{2.647030in}}%
\pgfpathlineto{\pgfqpoint{4.364648in}{2.647030in}}%
\pgfpathlineto{\pgfqpoint{4.364648in}{2.641131in}}%
\pgfpathmoveto{\pgfqpoint{4.364648in}{2.641131in}}%
\pgfpathlineto{\pgfqpoint{4.364648in}{2.641131in}}%
\pgfpathlineto{\pgfqpoint{4.364648in}{2.647030in}}%
\pgfpathlineto{\pgfqpoint{4.373730in}{2.647030in}}%
\pgfpathlineto{\pgfqpoint{4.373730in}{2.641131in}}%
\pgfpathmoveto{\pgfqpoint{4.373730in}{2.635233in}}%
\pgfpathlineto{\pgfqpoint{4.373730in}{2.635233in}}%
\pgfpathlineto{\pgfqpoint{4.373730in}{2.641131in}}%
\pgfpathlineto{\pgfqpoint{4.382811in}{2.641131in}}%
\pgfpathlineto{\pgfqpoint{4.382811in}{2.635233in}}%
\pgfpathmoveto{\pgfqpoint{4.373730in}{2.641131in}}%
\pgfpathlineto{\pgfqpoint{4.373730in}{2.641131in}}%
\pgfpathlineto{\pgfqpoint{4.373730in}{2.647030in}}%
\pgfpathlineto{\pgfqpoint{4.382811in}{2.647030in}}%
\pgfpathlineto{\pgfqpoint{4.382811in}{2.641131in}}%
\pgfpathmoveto{\pgfqpoint{4.255667in}{2.741405in}}%
\pgfpathlineto{\pgfqpoint{4.255667in}{2.741405in}}%
\pgfpathlineto{\pgfqpoint{4.255667in}{2.747303in}}%
\pgfpathlineto{\pgfqpoint{4.264748in}{2.747303in}}%
\pgfpathlineto{\pgfqpoint{4.264748in}{2.741405in}}%
\pgfpathmoveto{\pgfqpoint{4.273830in}{2.729608in}}%
\pgfpathlineto{\pgfqpoint{4.273830in}{2.729608in}}%
\pgfpathlineto{\pgfqpoint{4.273830in}{2.735506in}}%
\pgfpathlineto{\pgfqpoint{4.282912in}{2.735506in}}%
\pgfpathlineto{\pgfqpoint{4.282912in}{2.729608in}}%
\pgfpathmoveto{\pgfqpoint{4.291994in}{2.717811in}}%
\pgfpathlineto{\pgfqpoint{4.291994in}{2.717811in}}%
\pgfpathlineto{\pgfqpoint{4.291994in}{2.723709in}}%
\pgfpathlineto{\pgfqpoint{4.301075in}{2.723709in}}%
\pgfpathlineto{\pgfqpoint{4.301075in}{2.717811in}}%
\pgfpathmoveto{\pgfqpoint{4.310157in}{2.706014in}}%
\pgfpathlineto{\pgfqpoint{4.310157in}{2.706014in}}%
\pgfpathlineto{\pgfqpoint{4.310157in}{2.711913in}}%
\pgfpathlineto{\pgfqpoint{4.319239in}{2.711913in}}%
\pgfpathlineto{\pgfqpoint{4.319239in}{2.706014in}}%
\pgfpathmoveto{\pgfqpoint{4.328321in}{2.694217in}}%
\pgfpathlineto{\pgfqpoint{4.328321in}{2.694217in}}%
\pgfpathlineto{\pgfqpoint{4.328321in}{2.700116in}}%
\pgfpathlineto{\pgfqpoint{4.337403in}{2.700116in}}%
\pgfpathlineto{\pgfqpoint{4.337403in}{2.694217in}}%
\pgfpathmoveto{\pgfqpoint{4.346484in}{2.682421in}}%
\pgfpathlineto{\pgfqpoint{4.346484in}{2.682421in}}%
\pgfpathlineto{\pgfqpoint{4.346484in}{2.688319in}}%
\pgfpathlineto{\pgfqpoint{4.355566in}{2.688319in}}%
\pgfpathlineto{\pgfqpoint{4.355566in}{2.682421in}}%
\pgfpathmoveto{\pgfqpoint{4.364648in}{2.670624in}}%
\pgfpathlineto{\pgfqpoint{4.364648in}{2.670624in}}%
\pgfpathlineto{\pgfqpoint{4.364648in}{2.676522in}}%
\pgfpathlineto{\pgfqpoint{4.373730in}{2.676522in}}%
\pgfpathlineto{\pgfqpoint{4.373730in}{2.670624in}}%
\pgfpathmoveto{\pgfqpoint{4.400976in}{2.629334in}}%
\pgfpathlineto{\pgfqpoint{4.400976in}{2.629334in}}%
\pgfpathlineto{\pgfqpoint{4.400976in}{2.635233in}}%
\pgfpathlineto{\pgfqpoint{4.410058in}{2.635233in}}%
\pgfpathlineto{\pgfqpoint{4.410058in}{2.629334in}}%
\pgfpathmoveto{\pgfqpoint{4.410058in}{2.629334in}}%
\pgfpathlineto{\pgfqpoint{4.410058in}{2.629334in}}%
\pgfpathlineto{\pgfqpoint{4.410058in}{2.635233in}}%
\pgfpathlineto{\pgfqpoint{4.419140in}{2.635233in}}%
\pgfpathlineto{\pgfqpoint{4.419140in}{2.629334in}}%
\pgfpathmoveto{\pgfqpoint{4.382811in}{2.658827in}}%
\pgfpathlineto{\pgfqpoint{4.382811in}{2.658827in}}%
\pgfpathlineto{\pgfqpoint{4.382811in}{2.664725in}}%
\pgfpathlineto{\pgfqpoint{4.391893in}{2.664725in}}%
\pgfpathlineto{\pgfqpoint{4.391893in}{2.658827in}}%
\pgfpathmoveto{\pgfqpoint{4.400976in}{2.647030in}}%
\pgfpathlineto{\pgfqpoint{4.400976in}{2.647030in}}%
\pgfpathlineto{\pgfqpoint{4.400976in}{2.652928in}}%
\pgfpathlineto{\pgfqpoint{4.410058in}{2.652928in}}%
\pgfpathlineto{\pgfqpoint{4.410058in}{2.647030in}}%
\pgfpathmoveto{\pgfqpoint{4.419140in}{2.629334in}}%
\pgfpathlineto{\pgfqpoint{4.419140in}{2.629334in}}%
\pgfpathlineto{\pgfqpoint{4.419140in}{2.635233in}}%
\pgfpathlineto{\pgfqpoint{4.428222in}{2.635233in}}%
\pgfpathlineto{\pgfqpoint{4.428222in}{2.629334in}}%
\pgfpathmoveto{\pgfqpoint{4.428222in}{2.623436in}}%
\pgfpathlineto{\pgfqpoint{4.428222in}{2.623436in}}%
\pgfpathlineto{\pgfqpoint{4.428222in}{2.629334in}}%
\pgfpathlineto{\pgfqpoint{4.437304in}{2.629334in}}%
\pgfpathlineto{\pgfqpoint{4.437304in}{2.623436in}}%
\pgfpathmoveto{\pgfqpoint{4.428222in}{2.629334in}}%
\pgfpathlineto{\pgfqpoint{4.428222in}{2.629334in}}%
\pgfpathlineto{\pgfqpoint{4.428222in}{2.635233in}}%
\pgfpathlineto{\pgfqpoint{4.437304in}{2.635233in}}%
\pgfpathlineto{\pgfqpoint{4.437304in}{2.629334in}}%
\pgfpathmoveto{\pgfqpoint{4.419140in}{2.635233in}}%
\pgfpathlineto{\pgfqpoint{4.419140in}{2.635233in}}%
\pgfpathlineto{\pgfqpoint{4.419140in}{2.641131in}}%
\pgfpathlineto{\pgfqpoint{4.428222in}{2.641131in}}%
\pgfpathlineto{\pgfqpoint{4.428222in}{2.635233in}}%
\pgfpathmoveto{\pgfqpoint{4.437304in}{2.623436in}}%
\pgfpathlineto{\pgfqpoint{4.437304in}{2.623436in}}%
\pgfpathlineto{\pgfqpoint{4.437304in}{2.629334in}}%
\pgfpathlineto{\pgfqpoint{4.446386in}{2.629334in}}%
\pgfpathlineto{\pgfqpoint{4.446386in}{2.623436in}}%
\pgfpathmoveto{\pgfqpoint{3.070460in}{2.915411in}}%
\pgfpathlineto{\pgfqpoint{3.070460in}{2.915411in}}%
\pgfpathlineto{\pgfqpoint{3.070460in}{2.918361in}}%
\pgfpathlineto{\pgfqpoint{3.075001in}{2.918361in}}%
\pgfpathlineto{\pgfqpoint{3.075001in}{2.915411in}}%
\pgfpathmoveto{\pgfqpoint{3.070460in}{2.918361in}}%
\pgfpathlineto{\pgfqpoint{3.070460in}{2.918361in}}%
\pgfpathlineto{\pgfqpoint{3.070460in}{2.921310in}}%
\pgfpathlineto{\pgfqpoint{3.075001in}{2.921310in}}%
\pgfpathlineto{\pgfqpoint{3.075001in}{2.918361in}}%
\pgfpathmoveto{\pgfqpoint{3.070460in}{2.921310in}}%
\pgfpathlineto{\pgfqpoint{3.070460in}{2.921310in}}%
\pgfpathlineto{\pgfqpoint{3.070460in}{2.924259in}}%
\pgfpathlineto{\pgfqpoint{3.075001in}{2.924259in}}%
\pgfpathlineto{\pgfqpoint{3.075001in}{2.921310in}}%
\pgfpathmoveto{\pgfqpoint{3.070460in}{2.924259in}}%
\pgfpathlineto{\pgfqpoint{3.070460in}{2.924259in}}%
\pgfpathlineto{\pgfqpoint{3.070460in}{2.927209in}}%
\pgfpathlineto{\pgfqpoint{3.075001in}{2.927209in}}%
\pgfpathlineto{\pgfqpoint{3.075001in}{2.924259in}}%
\pgfpathmoveto{\pgfqpoint{3.070460in}{2.927209in}}%
\pgfpathlineto{\pgfqpoint{3.070460in}{2.927209in}}%
\pgfpathlineto{\pgfqpoint{3.070460in}{2.930158in}}%
\pgfpathlineto{\pgfqpoint{3.075001in}{2.930158in}}%
\pgfpathlineto{\pgfqpoint{3.075001in}{2.927209in}}%
\pgfpathmoveto{\pgfqpoint{3.070460in}{2.930158in}}%
\pgfpathlineto{\pgfqpoint{3.070460in}{2.930158in}}%
\pgfpathlineto{\pgfqpoint{3.070460in}{2.933107in}}%
\pgfpathlineto{\pgfqpoint{3.075001in}{2.933107in}}%
\pgfpathlineto{\pgfqpoint{3.075001in}{2.930158in}}%
\pgfpathmoveto{\pgfqpoint{3.070460in}{2.933107in}}%
\pgfpathlineto{\pgfqpoint{3.070460in}{2.933107in}}%
\pgfpathlineto{\pgfqpoint{3.070460in}{2.936056in}}%
\pgfpathlineto{\pgfqpoint{3.075001in}{2.936056in}}%
\pgfpathlineto{\pgfqpoint{3.075001in}{2.933107in}}%
\pgfpathmoveto{\pgfqpoint{3.070460in}{2.936056in}}%
\pgfpathlineto{\pgfqpoint{3.070460in}{2.936056in}}%
\pgfpathlineto{\pgfqpoint{3.070460in}{2.939006in}}%
\pgfpathlineto{\pgfqpoint{3.075001in}{2.939006in}}%
\pgfpathlineto{\pgfqpoint{3.075001in}{2.936056in}}%
\pgfpathmoveto{\pgfqpoint{3.070460in}{2.939006in}}%
\pgfpathlineto{\pgfqpoint{3.070460in}{2.939006in}}%
\pgfpathlineto{\pgfqpoint{3.070460in}{2.941955in}}%
\pgfpathlineto{\pgfqpoint{3.075001in}{2.941955in}}%
\pgfpathlineto{\pgfqpoint{3.075001in}{2.939006in}}%
\pgfpathmoveto{\pgfqpoint{3.070460in}{2.941955in}}%
\pgfpathlineto{\pgfqpoint{3.070460in}{2.941955in}}%
\pgfpathlineto{\pgfqpoint{3.070460in}{2.944904in}}%
\pgfpathlineto{\pgfqpoint{3.075001in}{2.944904in}}%
\pgfpathlineto{\pgfqpoint{3.075001in}{2.941955in}}%
\pgfpathmoveto{\pgfqpoint{3.070460in}{2.944904in}}%
\pgfpathlineto{\pgfqpoint{3.070460in}{2.944904in}}%
\pgfpathlineto{\pgfqpoint{3.070460in}{2.947853in}}%
\pgfpathlineto{\pgfqpoint{3.075001in}{2.947853in}}%
\pgfpathlineto{\pgfqpoint{3.075001in}{2.944904in}}%
\pgfpathmoveto{\pgfqpoint{3.070460in}{2.947853in}}%
\pgfpathlineto{\pgfqpoint{3.070460in}{2.947853in}}%
\pgfpathlineto{\pgfqpoint{3.070460in}{2.950803in}}%
\pgfpathlineto{\pgfqpoint{3.075001in}{2.950803in}}%
\pgfpathlineto{\pgfqpoint{3.075001in}{2.947853in}}%
\pgfpathmoveto{\pgfqpoint{3.070460in}{2.950803in}}%
\pgfpathlineto{\pgfqpoint{3.070460in}{2.950803in}}%
\pgfpathlineto{\pgfqpoint{3.070460in}{2.953752in}}%
\pgfpathlineto{\pgfqpoint{3.075001in}{2.953752in}}%
\pgfpathlineto{\pgfqpoint{3.075001in}{2.950803in}}%
\pgfpathmoveto{\pgfqpoint{3.070460in}{2.953752in}}%
\pgfpathlineto{\pgfqpoint{3.070460in}{2.953752in}}%
\pgfpathlineto{\pgfqpoint{3.070460in}{2.956701in}}%
\pgfpathlineto{\pgfqpoint{3.075001in}{2.956701in}}%
\pgfpathlineto{\pgfqpoint{3.075001in}{2.953752in}}%
\pgfpathmoveto{\pgfqpoint{3.070460in}{2.956701in}}%
\pgfpathlineto{\pgfqpoint{3.070460in}{2.956701in}}%
\pgfpathlineto{\pgfqpoint{3.070460in}{2.959650in}}%
\pgfpathlineto{\pgfqpoint{3.075001in}{2.959650in}}%
\pgfpathlineto{\pgfqpoint{3.075001in}{2.956701in}}%
\pgfpathmoveto{\pgfqpoint{3.070460in}{2.959650in}}%
\pgfpathlineto{\pgfqpoint{3.070460in}{2.959650in}}%
\pgfpathlineto{\pgfqpoint{3.070460in}{2.962599in}}%
\pgfpathlineto{\pgfqpoint{3.075001in}{2.962599in}}%
\pgfpathlineto{\pgfqpoint{3.075001in}{2.959650in}}%
\pgfpathmoveto{\pgfqpoint{3.070460in}{2.962599in}}%
\pgfpathlineto{\pgfqpoint{3.070460in}{2.962599in}}%
\pgfpathlineto{\pgfqpoint{3.070460in}{2.965548in}}%
\pgfpathlineto{\pgfqpoint{3.075001in}{2.965548in}}%
\pgfpathlineto{\pgfqpoint{3.075001in}{2.962599in}}%
\pgfpathmoveto{\pgfqpoint{3.070460in}{2.965548in}}%
\pgfpathlineto{\pgfqpoint{3.070460in}{2.965548in}}%
\pgfpathlineto{\pgfqpoint{3.070460in}{2.968498in}}%
\pgfpathlineto{\pgfqpoint{3.075001in}{2.968498in}}%
\pgfpathlineto{\pgfqpoint{3.075001in}{2.965548in}}%
\pgfpathmoveto{\pgfqpoint{3.070460in}{2.968498in}}%
\pgfpathlineto{\pgfqpoint{3.070460in}{2.968498in}}%
\pgfpathlineto{\pgfqpoint{3.070460in}{2.971447in}}%
\pgfpathlineto{\pgfqpoint{3.075001in}{2.971447in}}%
\pgfpathlineto{\pgfqpoint{3.075001in}{2.968498in}}%
\pgfpathmoveto{\pgfqpoint{3.070460in}{2.971447in}}%
\pgfpathlineto{\pgfqpoint{3.070460in}{2.971447in}}%
\pgfpathlineto{\pgfqpoint{3.070460in}{2.974396in}}%
\pgfpathlineto{\pgfqpoint{3.075001in}{2.974396in}}%
\pgfpathlineto{\pgfqpoint{3.075001in}{2.971447in}}%
\pgfpathmoveto{\pgfqpoint{3.070460in}{2.974396in}}%
\pgfpathlineto{\pgfqpoint{3.070460in}{2.974396in}}%
\pgfpathlineto{\pgfqpoint{3.070460in}{2.977345in}}%
\pgfpathlineto{\pgfqpoint{3.075001in}{2.977345in}}%
\pgfpathlineto{\pgfqpoint{3.075001in}{2.974396in}}%
\pgfpathmoveto{\pgfqpoint{3.070460in}{2.977345in}}%
\pgfpathlineto{\pgfqpoint{3.070460in}{2.977345in}}%
\pgfpathlineto{\pgfqpoint{3.070460in}{2.980294in}}%
\pgfpathlineto{\pgfqpoint{3.075001in}{2.980294in}}%
\pgfpathlineto{\pgfqpoint{3.075001in}{2.977345in}}%
\pgfpathmoveto{\pgfqpoint{3.070460in}{2.980294in}}%
\pgfpathlineto{\pgfqpoint{3.070460in}{2.980294in}}%
\pgfpathlineto{\pgfqpoint{3.070460in}{2.983243in}}%
\pgfpathlineto{\pgfqpoint{3.075001in}{2.983243in}}%
\pgfpathlineto{\pgfqpoint{3.075001in}{2.980294in}}%
\pgfpathmoveto{\pgfqpoint{3.070460in}{2.983243in}}%
\pgfpathlineto{\pgfqpoint{3.070460in}{2.983243in}}%
\pgfpathlineto{\pgfqpoint{3.070460in}{2.986193in}}%
\pgfpathlineto{\pgfqpoint{3.075001in}{2.986193in}}%
\pgfpathlineto{\pgfqpoint{3.075001in}{2.983243in}}%
\pgfpathmoveto{\pgfqpoint{3.070460in}{2.986193in}}%
\pgfpathlineto{\pgfqpoint{3.070460in}{2.986193in}}%
\pgfpathlineto{\pgfqpoint{3.070460in}{2.989142in}}%
\pgfpathlineto{\pgfqpoint{3.075001in}{2.989142in}}%
\pgfpathlineto{\pgfqpoint{3.075001in}{2.986193in}}%
\pgfpathmoveto{\pgfqpoint{3.070460in}{2.989142in}}%
\pgfpathlineto{\pgfqpoint{3.070460in}{2.989142in}}%
\pgfpathlineto{\pgfqpoint{3.070460in}{2.992091in}}%
\pgfpathlineto{\pgfqpoint{3.075001in}{2.992091in}}%
\pgfpathlineto{\pgfqpoint{3.075001in}{2.989142in}}%
\pgfpathmoveto{\pgfqpoint{3.070460in}{2.992091in}}%
\pgfpathlineto{\pgfqpoint{3.070460in}{2.992091in}}%
\pgfpathlineto{\pgfqpoint{3.070460in}{2.995040in}}%
\pgfpathlineto{\pgfqpoint{3.075001in}{2.995040in}}%
\pgfpathlineto{\pgfqpoint{3.075001in}{2.992091in}}%
\pgfpathmoveto{\pgfqpoint{3.070460in}{2.995040in}}%
\pgfpathlineto{\pgfqpoint{3.070460in}{2.995040in}}%
\pgfpathlineto{\pgfqpoint{3.070460in}{2.997989in}}%
\pgfpathlineto{\pgfqpoint{3.075001in}{2.997989in}}%
\pgfpathlineto{\pgfqpoint{3.075001in}{2.995040in}}%
\pgfpathmoveto{\pgfqpoint{3.070460in}{2.997989in}}%
\pgfpathlineto{\pgfqpoint{3.070460in}{2.997989in}}%
\pgfpathlineto{\pgfqpoint{3.070460in}{3.000938in}}%
\pgfpathlineto{\pgfqpoint{3.075001in}{3.000938in}}%
\pgfpathlineto{\pgfqpoint{3.075001in}{2.997989in}}%
\pgfpathmoveto{\pgfqpoint{3.070460in}{3.000938in}}%
\pgfpathlineto{\pgfqpoint{3.070460in}{3.000938in}}%
\pgfpathlineto{\pgfqpoint{3.070460in}{3.003887in}}%
\pgfpathlineto{\pgfqpoint{3.075001in}{3.003887in}}%
\pgfpathlineto{\pgfqpoint{3.075001in}{3.000938in}}%
\pgfpathmoveto{\pgfqpoint{3.070460in}{3.003887in}}%
\pgfpathlineto{\pgfqpoint{3.070460in}{3.003887in}}%
\pgfpathlineto{\pgfqpoint{3.070460in}{3.006837in}}%
\pgfpathlineto{\pgfqpoint{3.075001in}{3.006837in}}%
\pgfpathlineto{\pgfqpoint{3.075001in}{3.003887in}}%
\pgfpathmoveto{\pgfqpoint{3.070460in}{3.006837in}}%
\pgfpathlineto{\pgfqpoint{3.070460in}{3.006837in}}%
\pgfpathlineto{\pgfqpoint{3.070460in}{3.009786in}}%
\pgfpathlineto{\pgfqpoint{3.075001in}{3.009786in}}%
\pgfpathlineto{\pgfqpoint{3.075001in}{3.006837in}}%
\pgfpathmoveto{\pgfqpoint{3.070460in}{3.009786in}}%
\pgfpathlineto{\pgfqpoint{3.070460in}{3.009786in}}%
\pgfpathlineto{\pgfqpoint{3.070460in}{3.012735in}}%
\pgfpathlineto{\pgfqpoint{3.075001in}{3.012735in}}%
\pgfpathlineto{\pgfqpoint{3.075001in}{3.009786in}}%
\pgfpathmoveto{\pgfqpoint{3.070460in}{3.012735in}}%
\pgfpathlineto{\pgfqpoint{3.070460in}{3.012735in}}%
\pgfpathlineto{\pgfqpoint{3.070460in}{3.015684in}}%
\pgfpathlineto{\pgfqpoint{3.075001in}{3.015684in}}%
\pgfpathlineto{\pgfqpoint{3.075001in}{3.012735in}}%
\pgfpathmoveto{\pgfqpoint{3.070460in}{3.015684in}}%
\pgfpathlineto{\pgfqpoint{3.070460in}{3.015684in}}%
\pgfpathlineto{\pgfqpoint{3.070460in}{3.018633in}}%
\pgfpathlineto{\pgfqpoint{3.075001in}{3.018633in}}%
\pgfpathlineto{\pgfqpoint{3.075001in}{3.015684in}}%
\pgfpathmoveto{\pgfqpoint{3.070460in}{3.018633in}}%
\pgfpathlineto{\pgfqpoint{3.070460in}{3.018633in}}%
\pgfpathlineto{\pgfqpoint{3.070460in}{3.021582in}}%
\pgfpathlineto{\pgfqpoint{3.075001in}{3.021582in}}%
\pgfpathlineto{\pgfqpoint{3.075001in}{3.018633in}}%
\pgfpathmoveto{\pgfqpoint{3.070460in}{3.021582in}}%
\pgfpathlineto{\pgfqpoint{3.070460in}{3.021582in}}%
\pgfpathlineto{\pgfqpoint{3.070460in}{3.024531in}}%
\pgfpathlineto{\pgfqpoint{3.075001in}{3.024531in}}%
\pgfpathlineto{\pgfqpoint{3.075001in}{3.021582in}}%
\pgfpathmoveto{\pgfqpoint{3.070460in}{3.024531in}}%
\pgfpathlineto{\pgfqpoint{3.070460in}{3.024531in}}%
\pgfpathlineto{\pgfqpoint{3.070460in}{3.027481in}}%
\pgfpathlineto{\pgfqpoint{3.075001in}{3.027481in}}%
\pgfpathlineto{\pgfqpoint{3.075001in}{3.024531in}}%
\pgfpathmoveto{\pgfqpoint{3.070460in}{3.027481in}}%
\pgfpathlineto{\pgfqpoint{3.070460in}{3.027481in}}%
\pgfpathlineto{\pgfqpoint{3.070460in}{3.030430in}}%
\pgfpathlineto{\pgfqpoint{3.075001in}{3.030430in}}%
\pgfpathlineto{\pgfqpoint{3.075001in}{3.027481in}}%
\pgfpathmoveto{\pgfqpoint{3.070460in}{3.030430in}}%
\pgfpathlineto{\pgfqpoint{3.070460in}{3.030430in}}%
\pgfpathlineto{\pgfqpoint{3.070460in}{3.033379in}}%
\pgfpathlineto{\pgfqpoint{3.075001in}{3.033379in}}%
\pgfpathlineto{\pgfqpoint{3.075001in}{3.030430in}}%
\pgfpathmoveto{\pgfqpoint{3.070460in}{3.033379in}}%
\pgfpathlineto{\pgfqpoint{3.070460in}{3.033379in}}%
\pgfpathlineto{\pgfqpoint{3.070460in}{3.036328in}}%
\pgfpathlineto{\pgfqpoint{3.075001in}{3.036328in}}%
\pgfpathlineto{\pgfqpoint{3.075001in}{3.033379in}}%
\pgfpathmoveto{\pgfqpoint{3.070460in}{3.036328in}}%
\pgfpathlineto{\pgfqpoint{3.070460in}{3.036328in}}%
\pgfpathlineto{\pgfqpoint{3.070460in}{3.039277in}}%
\pgfpathlineto{\pgfqpoint{3.075001in}{3.039277in}}%
\pgfpathlineto{\pgfqpoint{3.075001in}{3.036328in}}%
\pgfpathmoveto{\pgfqpoint{3.070460in}{3.039277in}}%
\pgfpathlineto{\pgfqpoint{3.070460in}{3.039277in}}%
\pgfpathlineto{\pgfqpoint{3.070460in}{3.042226in}}%
\pgfpathlineto{\pgfqpoint{3.075001in}{3.042226in}}%
\pgfpathlineto{\pgfqpoint{3.075001in}{3.039277in}}%
\pgfpathmoveto{\pgfqpoint{3.070460in}{3.042226in}}%
\pgfpathlineto{\pgfqpoint{3.070460in}{3.042226in}}%
\pgfpathlineto{\pgfqpoint{3.070460in}{3.045176in}}%
\pgfpathlineto{\pgfqpoint{3.075001in}{3.045176in}}%
\pgfpathlineto{\pgfqpoint{3.075001in}{3.042226in}}%
\pgfpathmoveto{\pgfqpoint{3.070460in}{3.045176in}}%
\pgfpathlineto{\pgfqpoint{3.070460in}{3.045176in}}%
\pgfpathlineto{\pgfqpoint{3.070460in}{3.048125in}}%
\pgfpathlineto{\pgfqpoint{3.075001in}{3.048125in}}%
\pgfpathlineto{\pgfqpoint{3.075001in}{3.045176in}}%
\pgfpathmoveto{\pgfqpoint{3.070460in}{3.048125in}}%
\pgfpathlineto{\pgfqpoint{3.070460in}{3.048125in}}%
\pgfpathlineto{\pgfqpoint{3.070460in}{3.051074in}}%
\pgfpathlineto{\pgfqpoint{3.075001in}{3.051074in}}%
\pgfpathlineto{\pgfqpoint{3.075001in}{3.048125in}}%
\pgfpathmoveto{\pgfqpoint{3.070460in}{3.051074in}}%
\pgfpathlineto{\pgfqpoint{3.070460in}{3.051074in}}%
\pgfpathlineto{\pgfqpoint{3.070460in}{3.054023in}}%
\pgfpathlineto{\pgfqpoint{3.075001in}{3.054023in}}%
\pgfpathlineto{\pgfqpoint{3.075001in}{3.051074in}}%
\pgfpathmoveto{\pgfqpoint{3.070460in}{3.054023in}}%
\pgfpathlineto{\pgfqpoint{3.070460in}{3.054023in}}%
\pgfpathlineto{\pgfqpoint{3.070460in}{3.056972in}}%
\pgfpathlineto{\pgfqpoint{3.075001in}{3.056972in}}%
\pgfpathlineto{\pgfqpoint{3.075001in}{3.054023in}}%
\pgfpathmoveto{\pgfqpoint{3.070460in}{3.056972in}}%
\pgfpathlineto{\pgfqpoint{3.070460in}{3.056972in}}%
\pgfpathlineto{\pgfqpoint{3.070460in}{3.059922in}}%
\pgfpathlineto{\pgfqpoint{3.075001in}{3.059922in}}%
\pgfpathlineto{\pgfqpoint{3.075001in}{3.056972in}}%
\pgfpathmoveto{\pgfqpoint{3.070460in}{3.059922in}}%
\pgfpathlineto{\pgfqpoint{3.070460in}{3.059922in}}%
\pgfpathlineto{\pgfqpoint{3.070460in}{3.062871in}}%
\pgfpathlineto{\pgfqpoint{3.075001in}{3.062871in}}%
\pgfpathlineto{\pgfqpoint{3.075001in}{3.059922in}}%
\pgfpathmoveto{\pgfqpoint{3.070460in}{3.062871in}}%
\pgfpathlineto{\pgfqpoint{3.070460in}{3.062871in}}%
\pgfpathlineto{\pgfqpoint{3.070460in}{3.065820in}}%
\pgfpathlineto{\pgfqpoint{3.075001in}{3.065820in}}%
\pgfpathlineto{\pgfqpoint{3.075001in}{3.062871in}}%
\pgfpathmoveto{\pgfqpoint{3.070460in}{3.065820in}}%
\pgfpathlineto{\pgfqpoint{3.070460in}{3.065820in}}%
\pgfpathlineto{\pgfqpoint{3.070460in}{3.068769in}}%
\pgfpathlineto{\pgfqpoint{3.075001in}{3.068769in}}%
\pgfpathlineto{\pgfqpoint{3.075001in}{3.065820in}}%
\pgfpathmoveto{\pgfqpoint{3.070460in}{3.068769in}}%
\pgfpathlineto{\pgfqpoint{3.070460in}{3.068769in}}%
\pgfpathlineto{\pgfqpoint{3.070460in}{3.071718in}}%
\pgfpathlineto{\pgfqpoint{3.075001in}{3.071718in}}%
\pgfpathlineto{\pgfqpoint{3.075001in}{3.068769in}}%
\pgfpathmoveto{\pgfqpoint{3.070460in}{3.071718in}}%
\pgfpathlineto{\pgfqpoint{3.070460in}{3.071718in}}%
\pgfpathlineto{\pgfqpoint{3.070460in}{3.074668in}}%
\pgfpathlineto{\pgfqpoint{3.075001in}{3.074668in}}%
\pgfpathlineto{\pgfqpoint{3.075001in}{3.071718in}}%
\pgfpathmoveto{\pgfqpoint{3.070460in}{3.074668in}}%
\pgfpathlineto{\pgfqpoint{3.070460in}{3.074668in}}%
\pgfpathlineto{\pgfqpoint{3.070460in}{3.077617in}}%
\pgfpathlineto{\pgfqpoint{3.075001in}{3.077617in}}%
\pgfpathlineto{\pgfqpoint{3.075001in}{3.074668in}}%
\pgfpathmoveto{\pgfqpoint{3.070460in}{3.077617in}}%
\pgfpathlineto{\pgfqpoint{3.070460in}{3.077617in}}%
\pgfpathlineto{\pgfqpoint{3.070460in}{3.080566in}}%
\pgfpathlineto{\pgfqpoint{3.075001in}{3.080566in}}%
\pgfpathlineto{\pgfqpoint{3.075001in}{3.077617in}}%
\pgfpathmoveto{\pgfqpoint{3.070460in}{3.080566in}}%
\pgfpathlineto{\pgfqpoint{3.070460in}{3.080566in}}%
\pgfpathlineto{\pgfqpoint{3.070460in}{3.083515in}}%
\pgfpathlineto{\pgfqpoint{3.075001in}{3.083515in}}%
\pgfpathlineto{\pgfqpoint{3.075001in}{3.080566in}}%
\pgfpathmoveto{\pgfqpoint{3.070460in}{3.083515in}}%
\pgfpathlineto{\pgfqpoint{3.070460in}{3.083515in}}%
\pgfpathlineto{\pgfqpoint{3.070460in}{3.086464in}}%
\pgfpathlineto{\pgfqpoint{3.075001in}{3.086464in}}%
\pgfpathlineto{\pgfqpoint{3.075001in}{3.083515in}}%
\pgfpathmoveto{\pgfqpoint{3.070460in}{3.086464in}}%
\pgfpathlineto{\pgfqpoint{3.070460in}{3.086464in}}%
\pgfpathlineto{\pgfqpoint{3.070460in}{3.089414in}}%
\pgfpathlineto{\pgfqpoint{3.075001in}{3.089414in}}%
\pgfpathlineto{\pgfqpoint{3.075001in}{3.086464in}}%
\pgfpathmoveto{\pgfqpoint{3.070460in}{3.089414in}}%
\pgfpathlineto{\pgfqpoint{3.070460in}{3.089414in}}%
\pgfpathlineto{\pgfqpoint{3.070460in}{3.092363in}}%
\pgfpathlineto{\pgfqpoint{3.075001in}{3.092363in}}%
\pgfpathlineto{\pgfqpoint{3.075001in}{3.089414in}}%
\pgfpathmoveto{\pgfqpoint{3.070460in}{3.092363in}}%
\pgfpathlineto{\pgfqpoint{3.070460in}{3.092363in}}%
\pgfpathlineto{\pgfqpoint{3.070460in}{3.095312in}}%
\pgfpathlineto{\pgfqpoint{3.075001in}{3.095312in}}%
\pgfpathlineto{\pgfqpoint{3.075001in}{3.092363in}}%
\pgfpathmoveto{\pgfqpoint{3.070460in}{3.095312in}}%
\pgfpathlineto{\pgfqpoint{3.070460in}{3.095312in}}%
\pgfpathlineto{\pgfqpoint{3.070460in}{3.098261in}}%
\pgfpathlineto{\pgfqpoint{3.075001in}{3.098261in}}%
\pgfpathlineto{\pgfqpoint{3.075001in}{3.095312in}}%
\pgfpathmoveto{\pgfqpoint{3.070460in}{3.098261in}}%
\pgfpathlineto{\pgfqpoint{3.070460in}{3.098261in}}%
\pgfpathlineto{\pgfqpoint{3.070460in}{3.101210in}}%
\pgfpathlineto{\pgfqpoint{3.075001in}{3.101210in}}%
\pgfpathlineto{\pgfqpoint{3.075001in}{3.098261in}}%
\pgfpathmoveto{\pgfqpoint{3.070460in}{3.101210in}}%
\pgfpathlineto{\pgfqpoint{3.070460in}{3.101210in}}%
\pgfpathlineto{\pgfqpoint{3.070460in}{3.104160in}}%
\pgfpathlineto{\pgfqpoint{3.075001in}{3.104160in}}%
\pgfpathlineto{\pgfqpoint{3.075001in}{3.101210in}}%
\pgfpathmoveto{\pgfqpoint{3.070460in}{3.104160in}}%
\pgfpathlineto{\pgfqpoint{3.070460in}{3.104160in}}%
\pgfpathlineto{\pgfqpoint{3.070460in}{3.107109in}}%
\pgfpathlineto{\pgfqpoint{3.075001in}{3.107109in}}%
\pgfpathlineto{\pgfqpoint{3.075001in}{3.104160in}}%
\pgfpathmoveto{\pgfqpoint{3.070460in}{3.107109in}}%
\pgfpathlineto{\pgfqpoint{3.070460in}{3.107109in}}%
\pgfpathlineto{\pgfqpoint{3.070460in}{3.110058in}}%
\pgfpathlineto{\pgfqpoint{3.075001in}{3.110058in}}%
\pgfpathlineto{\pgfqpoint{3.075001in}{3.107109in}}%
\pgfpathmoveto{\pgfqpoint{3.070460in}{3.110058in}}%
\pgfpathlineto{\pgfqpoint{3.070460in}{3.110058in}}%
\pgfpathlineto{\pgfqpoint{3.070460in}{3.113007in}}%
\pgfpathlineto{\pgfqpoint{3.075001in}{3.113007in}}%
\pgfpathlineto{\pgfqpoint{3.075001in}{3.110058in}}%
\pgfpathmoveto{\pgfqpoint{3.070460in}{3.113007in}}%
\pgfpathlineto{\pgfqpoint{3.070460in}{3.113007in}}%
\pgfpathlineto{\pgfqpoint{3.070460in}{3.115957in}}%
\pgfpathlineto{\pgfqpoint{3.075001in}{3.115957in}}%
\pgfpathlineto{\pgfqpoint{3.075001in}{3.113007in}}%
\pgfpathmoveto{\pgfqpoint{3.070460in}{3.115957in}}%
\pgfpathlineto{\pgfqpoint{3.070460in}{3.115957in}}%
\pgfpathlineto{\pgfqpoint{3.070460in}{3.118906in}}%
\pgfpathlineto{\pgfqpoint{3.075001in}{3.118906in}}%
\pgfpathlineto{\pgfqpoint{3.075001in}{3.115957in}}%
\pgfpathmoveto{\pgfqpoint{3.070460in}{3.118906in}}%
\pgfpathlineto{\pgfqpoint{3.070460in}{3.118906in}}%
\pgfpathlineto{\pgfqpoint{3.070460in}{3.121855in}}%
\pgfpathlineto{\pgfqpoint{3.075001in}{3.121855in}}%
\pgfpathlineto{\pgfqpoint{3.075001in}{3.118906in}}%
\pgfpathmoveto{\pgfqpoint{3.070460in}{3.121855in}}%
\pgfpathlineto{\pgfqpoint{3.070460in}{3.121855in}}%
\pgfpathlineto{\pgfqpoint{3.070460in}{3.124804in}}%
\pgfpathlineto{\pgfqpoint{3.075001in}{3.124804in}}%
\pgfpathlineto{\pgfqpoint{3.075001in}{3.121855in}}%
\pgfpathmoveto{\pgfqpoint{3.070460in}{3.124804in}}%
\pgfpathlineto{\pgfqpoint{3.070460in}{3.124804in}}%
\pgfpathlineto{\pgfqpoint{3.070460in}{3.127753in}}%
\pgfpathlineto{\pgfqpoint{3.075001in}{3.127753in}}%
\pgfpathlineto{\pgfqpoint{3.075001in}{3.124804in}}%
\pgfpathmoveto{\pgfqpoint{3.070460in}{3.127753in}}%
\pgfpathlineto{\pgfqpoint{3.070460in}{3.127753in}}%
\pgfpathlineto{\pgfqpoint{3.070460in}{3.130703in}}%
\pgfpathlineto{\pgfqpoint{3.075001in}{3.130703in}}%
\pgfpathlineto{\pgfqpoint{3.075001in}{3.127753in}}%
\pgfpathmoveto{\pgfqpoint{3.070460in}{3.130703in}}%
\pgfpathlineto{\pgfqpoint{3.070460in}{3.130703in}}%
\pgfpathlineto{\pgfqpoint{3.070460in}{3.133652in}}%
\pgfpathlineto{\pgfqpoint{3.075001in}{3.133652in}}%
\pgfpathlineto{\pgfqpoint{3.075001in}{3.130703in}}%
\pgfpathmoveto{\pgfqpoint{3.070460in}{3.133652in}}%
\pgfpathlineto{\pgfqpoint{3.070460in}{3.133652in}}%
\pgfpathlineto{\pgfqpoint{3.070460in}{3.136601in}}%
\pgfpathlineto{\pgfqpoint{3.075001in}{3.136601in}}%
\pgfpathlineto{\pgfqpoint{3.075001in}{3.133652in}}%
\pgfpathmoveto{\pgfqpoint{3.070460in}{3.136601in}}%
\pgfpathlineto{\pgfqpoint{3.070460in}{3.136601in}}%
\pgfpathlineto{\pgfqpoint{3.070460in}{3.139550in}}%
\pgfpathlineto{\pgfqpoint{3.075001in}{3.139550in}}%
\pgfpathlineto{\pgfqpoint{3.075001in}{3.136601in}}%
\pgfpathmoveto{\pgfqpoint{3.070460in}{3.139550in}}%
\pgfpathlineto{\pgfqpoint{3.070460in}{3.139550in}}%
\pgfpathlineto{\pgfqpoint{3.070460in}{3.142499in}}%
\pgfpathlineto{\pgfqpoint{3.075001in}{3.142499in}}%
\pgfpathlineto{\pgfqpoint{3.075001in}{3.139550in}}%
\pgfpathmoveto{\pgfqpoint{3.070460in}{3.142499in}}%
\pgfpathlineto{\pgfqpoint{3.070460in}{3.142499in}}%
\pgfpathlineto{\pgfqpoint{3.070460in}{3.145449in}}%
\pgfpathlineto{\pgfqpoint{3.075001in}{3.145449in}}%
\pgfpathlineto{\pgfqpoint{3.075001in}{3.142499in}}%
\pgfpathmoveto{\pgfqpoint{3.070460in}{3.145449in}}%
\pgfpathlineto{\pgfqpoint{3.070460in}{3.145449in}}%
\pgfpathlineto{\pgfqpoint{3.070460in}{3.148398in}}%
\pgfpathlineto{\pgfqpoint{3.075001in}{3.148398in}}%
\pgfpathlineto{\pgfqpoint{3.075001in}{3.145449in}}%
\pgfpathmoveto{\pgfqpoint{3.070460in}{3.148398in}}%
\pgfpathlineto{\pgfqpoint{3.070460in}{3.148398in}}%
\pgfpathlineto{\pgfqpoint{3.070460in}{3.151347in}}%
\pgfpathlineto{\pgfqpoint{3.075001in}{3.151347in}}%
\pgfpathlineto{\pgfqpoint{3.075001in}{3.148398in}}%
\pgfpathmoveto{\pgfqpoint{3.070460in}{3.151347in}}%
\pgfpathlineto{\pgfqpoint{3.070460in}{3.151347in}}%
\pgfpathlineto{\pgfqpoint{3.070460in}{3.154296in}}%
\pgfpathlineto{\pgfqpoint{3.075001in}{3.154296in}}%
\pgfpathlineto{\pgfqpoint{3.075001in}{3.151347in}}%
\pgfpathmoveto{\pgfqpoint{3.070460in}{3.154296in}}%
\pgfpathlineto{\pgfqpoint{3.070460in}{3.154296in}}%
\pgfpathlineto{\pgfqpoint{3.070460in}{3.157246in}}%
\pgfpathlineto{\pgfqpoint{3.075001in}{3.157246in}}%
\pgfpathlineto{\pgfqpoint{3.075001in}{3.154296in}}%
\pgfpathmoveto{\pgfqpoint{3.070460in}{3.157246in}}%
\pgfpathlineto{\pgfqpoint{3.070460in}{3.157246in}}%
\pgfpathlineto{\pgfqpoint{3.070460in}{3.160195in}}%
\pgfpathlineto{\pgfqpoint{3.075001in}{3.160195in}}%
\pgfpathlineto{\pgfqpoint{3.075001in}{3.157246in}}%
\pgfpathmoveto{\pgfqpoint{3.070460in}{3.160195in}}%
\pgfpathlineto{\pgfqpoint{3.070460in}{3.160195in}}%
\pgfpathlineto{\pgfqpoint{3.070460in}{3.163144in}}%
\pgfpathlineto{\pgfqpoint{3.075001in}{3.163144in}}%
\pgfpathlineto{\pgfqpoint{3.075001in}{3.160195in}}%
\pgfpathmoveto{\pgfqpoint{3.070460in}{3.163144in}}%
\pgfpathlineto{\pgfqpoint{3.070460in}{3.163144in}}%
\pgfpathlineto{\pgfqpoint{3.070460in}{3.166093in}}%
\pgfpathlineto{\pgfqpoint{3.075001in}{3.166093in}}%
\pgfpathlineto{\pgfqpoint{3.075001in}{3.163144in}}%
\pgfpathmoveto{\pgfqpoint{3.070460in}{3.166093in}}%
\pgfpathlineto{\pgfqpoint{3.070460in}{3.166093in}}%
\pgfpathlineto{\pgfqpoint{3.070460in}{3.169042in}}%
\pgfpathlineto{\pgfqpoint{3.075001in}{3.169042in}}%
\pgfpathlineto{\pgfqpoint{3.075001in}{3.166093in}}%
\pgfpathmoveto{\pgfqpoint{3.070460in}{3.169042in}}%
\pgfpathlineto{\pgfqpoint{3.070460in}{3.169042in}}%
\pgfpathlineto{\pgfqpoint{3.070460in}{3.171992in}}%
\pgfpathlineto{\pgfqpoint{3.075001in}{3.171992in}}%
\pgfpathlineto{\pgfqpoint{3.075001in}{3.169042in}}%
\pgfpathmoveto{\pgfqpoint{3.070460in}{3.171992in}}%
\pgfpathlineto{\pgfqpoint{3.070460in}{3.171992in}}%
\pgfpathlineto{\pgfqpoint{3.070460in}{3.174941in}}%
\pgfpathlineto{\pgfqpoint{3.075001in}{3.174941in}}%
\pgfpathlineto{\pgfqpoint{3.075001in}{3.171992in}}%
\pgfpathmoveto{\pgfqpoint{3.070460in}{3.174941in}}%
\pgfpathlineto{\pgfqpoint{3.070460in}{3.174941in}}%
\pgfpathlineto{\pgfqpoint{3.070460in}{3.177890in}}%
\pgfpathlineto{\pgfqpoint{3.075001in}{3.177890in}}%
\pgfpathlineto{\pgfqpoint{3.075001in}{3.174941in}}%
\pgfpathmoveto{\pgfqpoint{3.070460in}{3.177890in}}%
\pgfpathlineto{\pgfqpoint{3.070460in}{3.177890in}}%
\pgfpathlineto{\pgfqpoint{3.070460in}{3.180839in}}%
\pgfpathlineto{\pgfqpoint{3.075001in}{3.180839in}}%
\pgfpathlineto{\pgfqpoint{3.075001in}{3.177890in}}%
\pgfpathmoveto{\pgfqpoint{3.070460in}{3.180839in}}%
\pgfpathlineto{\pgfqpoint{3.070460in}{3.180839in}}%
\pgfpathlineto{\pgfqpoint{3.070460in}{3.183788in}}%
\pgfpathlineto{\pgfqpoint{3.075001in}{3.183788in}}%
\pgfpathlineto{\pgfqpoint{3.075001in}{3.180839in}}%
\pgfpathmoveto{\pgfqpoint{3.070460in}{3.183788in}}%
\pgfpathlineto{\pgfqpoint{3.070460in}{3.183788in}}%
\pgfpathlineto{\pgfqpoint{3.070460in}{3.186738in}}%
\pgfpathlineto{\pgfqpoint{3.075001in}{3.186738in}}%
\pgfpathlineto{\pgfqpoint{3.075001in}{3.183788in}}%
\pgfpathmoveto{\pgfqpoint{3.070460in}{3.186738in}}%
\pgfpathlineto{\pgfqpoint{3.070460in}{3.186738in}}%
\pgfpathlineto{\pgfqpoint{3.070460in}{3.189687in}}%
\pgfpathlineto{\pgfqpoint{3.075001in}{3.189687in}}%
\pgfpathlineto{\pgfqpoint{3.075001in}{3.186738in}}%
\pgfpathmoveto{\pgfqpoint{3.070460in}{3.189687in}}%
\pgfpathlineto{\pgfqpoint{3.070460in}{3.189687in}}%
\pgfpathlineto{\pgfqpoint{3.070460in}{3.192636in}}%
\pgfpathlineto{\pgfqpoint{3.075001in}{3.192636in}}%
\pgfpathlineto{\pgfqpoint{3.075001in}{3.189687in}}%
\pgfpathmoveto{\pgfqpoint{3.070460in}{3.192636in}}%
\pgfpathlineto{\pgfqpoint{3.070460in}{3.192636in}}%
\pgfpathlineto{\pgfqpoint{3.070460in}{3.195585in}}%
\pgfpathlineto{\pgfqpoint{3.075001in}{3.195585in}}%
\pgfpathlineto{\pgfqpoint{3.075001in}{3.192636in}}%
\pgfpathmoveto{\pgfqpoint{3.070460in}{3.195585in}}%
\pgfpathlineto{\pgfqpoint{3.070460in}{3.195585in}}%
\pgfpathlineto{\pgfqpoint{3.070460in}{3.198535in}}%
\pgfpathlineto{\pgfqpoint{3.075001in}{3.198535in}}%
\pgfpathlineto{\pgfqpoint{3.075001in}{3.195585in}}%
\pgfpathmoveto{\pgfqpoint{3.070460in}{3.198535in}}%
\pgfpathlineto{\pgfqpoint{3.070460in}{3.198535in}}%
\pgfpathlineto{\pgfqpoint{3.070460in}{3.201484in}}%
\pgfpathlineto{\pgfqpoint{3.075001in}{3.201484in}}%
\pgfpathlineto{\pgfqpoint{3.075001in}{3.198535in}}%
\pgfpathmoveto{\pgfqpoint{3.070460in}{3.201484in}}%
\pgfpathlineto{\pgfqpoint{3.070460in}{3.201484in}}%
\pgfpathlineto{\pgfqpoint{3.070460in}{3.204433in}}%
\pgfpathlineto{\pgfqpoint{3.075001in}{3.204433in}}%
\pgfpathlineto{\pgfqpoint{3.075001in}{3.201484in}}%
\pgfpathmoveto{\pgfqpoint{3.070460in}{3.204433in}}%
\pgfpathlineto{\pgfqpoint{3.070460in}{3.204433in}}%
\pgfpathlineto{\pgfqpoint{3.070460in}{3.207382in}}%
\pgfpathlineto{\pgfqpoint{3.075001in}{3.207382in}}%
\pgfpathlineto{\pgfqpoint{3.075001in}{3.204433in}}%
\pgfpathmoveto{\pgfqpoint{3.070460in}{3.207382in}}%
\pgfpathlineto{\pgfqpoint{3.070460in}{3.207382in}}%
\pgfpathlineto{\pgfqpoint{3.070460in}{3.210331in}}%
\pgfpathlineto{\pgfqpoint{3.075001in}{3.210331in}}%
\pgfpathlineto{\pgfqpoint{3.075001in}{3.207382in}}%
\pgfpathmoveto{\pgfqpoint{3.070460in}{3.210331in}}%
\pgfpathlineto{\pgfqpoint{3.070460in}{3.210331in}}%
\pgfpathlineto{\pgfqpoint{3.070460in}{3.213281in}}%
\pgfpathlineto{\pgfqpoint{3.075001in}{3.213281in}}%
\pgfpathlineto{\pgfqpoint{3.075001in}{3.210331in}}%
\pgfpathmoveto{\pgfqpoint{3.070460in}{3.213281in}}%
\pgfpathlineto{\pgfqpoint{3.070460in}{3.213281in}}%
\pgfpathlineto{\pgfqpoint{3.070460in}{3.216230in}}%
\pgfpathlineto{\pgfqpoint{3.075001in}{3.216230in}}%
\pgfpathlineto{\pgfqpoint{3.075001in}{3.213281in}}%
\pgfpathmoveto{\pgfqpoint{3.070460in}{3.216230in}}%
\pgfpathlineto{\pgfqpoint{3.070460in}{3.216230in}}%
\pgfpathlineto{\pgfqpoint{3.070460in}{3.219179in}}%
\pgfpathlineto{\pgfqpoint{3.075001in}{3.219179in}}%
\pgfpathlineto{\pgfqpoint{3.075001in}{3.216230in}}%
\pgfpathmoveto{\pgfqpoint{3.070460in}{3.219179in}}%
\pgfpathlineto{\pgfqpoint{3.070460in}{3.219179in}}%
\pgfpathlineto{\pgfqpoint{3.070460in}{3.222128in}}%
\pgfpathlineto{\pgfqpoint{3.075001in}{3.222128in}}%
\pgfpathlineto{\pgfqpoint{3.075001in}{3.219179in}}%
\pgfpathmoveto{\pgfqpoint{3.070460in}{3.222128in}}%
\pgfpathlineto{\pgfqpoint{3.070460in}{3.222128in}}%
\pgfpathlineto{\pgfqpoint{3.070460in}{3.225077in}}%
\pgfpathlineto{\pgfqpoint{3.075001in}{3.225077in}}%
\pgfpathlineto{\pgfqpoint{3.075001in}{3.222128in}}%
\pgfpathmoveto{\pgfqpoint{3.070460in}{3.225077in}}%
\pgfpathlineto{\pgfqpoint{3.070460in}{3.225077in}}%
\pgfpathlineto{\pgfqpoint{3.070460in}{3.228027in}}%
\pgfpathlineto{\pgfqpoint{3.075001in}{3.228027in}}%
\pgfpathlineto{\pgfqpoint{3.075001in}{3.225077in}}%
\pgfpathmoveto{\pgfqpoint{3.070460in}{3.228027in}}%
\pgfpathlineto{\pgfqpoint{3.070460in}{3.228027in}}%
\pgfpathlineto{\pgfqpoint{3.070460in}{3.230976in}}%
\pgfpathlineto{\pgfqpoint{3.075001in}{3.230976in}}%
\pgfpathlineto{\pgfqpoint{3.075001in}{3.228027in}}%
\pgfpathmoveto{\pgfqpoint{3.070460in}{3.230976in}}%
\pgfpathlineto{\pgfqpoint{3.070460in}{3.230976in}}%
\pgfpathlineto{\pgfqpoint{3.070460in}{3.233925in}}%
\pgfpathlineto{\pgfqpoint{3.075001in}{3.233925in}}%
\pgfpathlineto{\pgfqpoint{3.075001in}{3.230976in}}%
\pgfpathmoveto{\pgfqpoint{3.070460in}{3.233925in}}%
\pgfpathlineto{\pgfqpoint{3.070460in}{3.233925in}}%
\pgfpathlineto{\pgfqpoint{3.070460in}{3.236874in}}%
\pgfpathlineto{\pgfqpoint{3.075001in}{3.236874in}}%
\pgfpathlineto{\pgfqpoint{3.075001in}{3.233925in}}%
\pgfpathmoveto{\pgfqpoint{3.070460in}{3.236874in}}%
\pgfpathlineto{\pgfqpoint{3.070460in}{3.236874in}}%
\pgfpathlineto{\pgfqpoint{3.070460in}{3.239823in}}%
\pgfpathlineto{\pgfqpoint{3.075001in}{3.239823in}}%
\pgfpathlineto{\pgfqpoint{3.075001in}{3.236874in}}%
\pgfpathmoveto{\pgfqpoint{3.070460in}{3.239823in}}%
\pgfpathlineto{\pgfqpoint{3.070460in}{3.239823in}}%
\pgfpathlineto{\pgfqpoint{3.070460in}{3.242773in}}%
\pgfpathlineto{\pgfqpoint{3.075001in}{3.242773in}}%
\pgfpathlineto{\pgfqpoint{3.075001in}{3.239823in}}%
\pgfpathmoveto{\pgfqpoint{3.070460in}{3.242773in}}%
\pgfpathlineto{\pgfqpoint{3.070460in}{3.242773in}}%
\pgfpathlineto{\pgfqpoint{3.070460in}{3.245722in}}%
\pgfpathlineto{\pgfqpoint{3.075001in}{3.245722in}}%
\pgfpathlineto{\pgfqpoint{3.075001in}{3.242773in}}%
\pgfpathmoveto{\pgfqpoint{3.070460in}{3.245722in}}%
\pgfpathlineto{\pgfqpoint{3.070460in}{3.245722in}}%
\pgfpathlineto{\pgfqpoint{3.070460in}{3.248671in}}%
\pgfpathlineto{\pgfqpoint{3.075001in}{3.248671in}}%
\pgfpathlineto{\pgfqpoint{3.075001in}{3.245722in}}%
\pgfpathmoveto{\pgfqpoint{3.070460in}{3.248671in}}%
\pgfpathlineto{\pgfqpoint{3.070460in}{3.248671in}}%
\pgfpathlineto{\pgfqpoint{3.070460in}{3.251620in}}%
\pgfpathlineto{\pgfqpoint{3.075001in}{3.251620in}}%
\pgfpathlineto{\pgfqpoint{3.075001in}{3.248671in}}%
\pgfpathmoveto{\pgfqpoint{3.070460in}{3.251620in}}%
\pgfpathlineto{\pgfqpoint{3.070460in}{3.251620in}}%
\pgfpathlineto{\pgfqpoint{3.070460in}{3.254569in}}%
\pgfpathlineto{\pgfqpoint{3.075001in}{3.254569in}}%
\pgfpathlineto{\pgfqpoint{3.075001in}{3.251620in}}%
\pgfpathmoveto{\pgfqpoint{3.070460in}{3.254569in}}%
\pgfpathlineto{\pgfqpoint{3.070460in}{3.254569in}}%
\pgfpathlineto{\pgfqpoint{3.070460in}{3.257518in}}%
\pgfpathlineto{\pgfqpoint{3.075001in}{3.257518in}}%
\pgfpathlineto{\pgfqpoint{3.075001in}{3.254569in}}%
\pgfpathmoveto{\pgfqpoint{3.070460in}{3.257518in}}%
\pgfpathlineto{\pgfqpoint{3.070460in}{3.257518in}}%
\pgfpathlineto{\pgfqpoint{3.070460in}{3.260468in}}%
\pgfpathlineto{\pgfqpoint{3.075001in}{3.260468in}}%
\pgfpathlineto{\pgfqpoint{3.075001in}{3.257518in}}%
\pgfpathmoveto{\pgfqpoint{3.070460in}{3.260468in}}%
\pgfpathlineto{\pgfqpoint{3.070460in}{3.260468in}}%
\pgfpathlineto{\pgfqpoint{3.070460in}{3.263417in}}%
\pgfpathlineto{\pgfqpoint{3.075001in}{3.263417in}}%
\pgfpathlineto{\pgfqpoint{3.075001in}{3.260468in}}%
\pgfpathmoveto{\pgfqpoint{3.070460in}{3.263417in}}%
\pgfpathlineto{\pgfqpoint{3.070460in}{3.263417in}}%
\pgfpathlineto{\pgfqpoint{3.070460in}{3.266366in}}%
\pgfpathlineto{\pgfqpoint{3.075001in}{3.266366in}}%
\pgfpathlineto{\pgfqpoint{3.075001in}{3.263417in}}%
\pgfpathmoveto{\pgfqpoint{3.070460in}{3.266366in}}%
\pgfpathlineto{\pgfqpoint{3.070460in}{3.266366in}}%
\pgfpathlineto{\pgfqpoint{3.070460in}{3.269315in}}%
\pgfpathlineto{\pgfqpoint{3.075001in}{3.269315in}}%
\pgfpathlineto{\pgfqpoint{3.075001in}{3.266366in}}%
\pgfpathmoveto{\pgfqpoint{3.070460in}{3.269315in}}%
\pgfpathlineto{\pgfqpoint{3.070460in}{3.269315in}}%
\pgfpathlineto{\pgfqpoint{3.070460in}{3.272264in}}%
\pgfpathlineto{\pgfqpoint{3.075001in}{3.272264in}}%
\pgfpathlineto{\pgfqpoint{3.075001in}{3.269315in}}%
\pgfpathmoveto{\pgfqpoint{3.070460in}{3.272264in}}%
\pgfpathlineto{\pgfqpoint{3.070460in}{3.272264in}}%
\pgfpathlineto{\pgfqpoint{3.070460in}{3.275213in}}%
\pgfpathlineto{\pgfqpoint{3.075001in}{3.275213in}}%
\pgfpathlineto{\pgfqpoint{3.075001in}{3.272264in}}%
\pgfpathmoveto{\pgfqpoint{3.070460in}{3.275213in}}%
\pgfpathlineto{\pgfqpoint{3.070460in}{3.275213in}}%
\pgfpathlineto{\pgfqpoint{3.070460in}{3.278163in}}%
\pgfpathlineto{\pgfqpoint{3.075001in}{3.278163in}}%
\pgfpathlineto{\pgfqpoint{3.075001in}{3.275213in}}%
\pgfpathmoveto{\pgfqpoint{3.070460in}{3.278163in}}%
\pgfpathlineto{\pgfqpoint{3.070460in}{3.278163in}}%
\pgfpathlineto{\pgfqpoint{3.070460in}{3.281112in}}%
\pgfpathlineto{\pgfqpoint{3.075001in}{3.281112in}}%
\pgfpathlineto{\pgfqpoint{3.075001in}{3.278163in}}%
\pgfpathmoveto{\pgfqpoint{3.070460in}{3.281112in}}%
\pgfpathlineto{\pgfqpoint{3.070460in}{3.281112in}}%
\pgfpathlineto{\pgfqpoint{3.070460in}{3.284061in}}%
\pgfpathlineto{\pgfqpoint{3.075001in}{3.284061in}}%
\pgfpathlineto{\pgfqpoint{3.075001in}{3.281112in}}%
\pgfpathmoveto{\pgfqpoint{3.070460in}{3.284061in}}%
\pgfpathlineto{\pgfqpoint{3.070460in}{3.284061in}}%
\pgfpathlineto{\pgfqpoint{3.070460in}{3.287010in}}%
\pgfpathlineto{\pgfqpoint{3.075001in}{3.287010in}}%
\pgfpathlineto{\pgfqpoint{3.075001in}{3.284061in}}%
\pgfpathmoveto{\pgfqpoint{3.070460in}{3.287010in}}%
\pgfpathlineto{\pgfqpoint{3.070460in}{3.287010in}}%
\pgfpathlineto{\pgfqpoint{3.070460in}{3.289959in}}%
\pgfpathlineto{\pgfqpoint{3.075001in}{3.289959in}}%
\pgfpathlineto{\pgfqpoint{3.075001in}{3.287010in}}%
\pgfpathmoveto{\pgfqpoint{3.070460in}{3.289959in}}%
\pgfpathlineto{\pgfqpoint{3.070460in}{3.289959in}}%
\pgfpathlineto{\pgfqpoint{3.070460in}{3.292908in}}%
\pgfpathlineto{\pgfqpoint{3.075001in}{3.292908in}}%
\pgfpathlineto{\pgfqpoint{3.075001in}{3.289959in}}%
\pgfpathmoveto{\pgfqpoint{3.070460in}{3.292908in}}%
\pgfpathlineto{\pgfqpoint{3.070460in}{3.292908in}}%
\pgfpathlineto{\pgfqpoint{3.070460in}{3.295858in}}%
\pgfpathlineto{\pgfqpoint{3.075001in}{3.295858in}}%
\pgfpathlineto{\pgfqpoint{3.075001in}{3.292908in}}%
\pgfpathmoveto{\pgfqpoint{3.070460in}{3.295858in}}%
\pgfpathlineto{\pgfqpoint{3.070460in}{3.295858in}}%
\pgfpathlineto{\pgfqpoint{3.070460in}{3.298807in}}%
\pgfpathlineto{\pgfqpoint{3.075001in}{3.298807in}}%
\pgfpathlineto{\pgfqpoint{3.075001in}{3.295858in}}%
\pgfpathmoveto{\pgfqpoint{3.070460in}{3.298807in}}%
\pgfpathlineto{\pgfqpoint{3.070460in}{3.298807in}}%
\pgfpathlineto{\pgfqpoint{3.070460in}{3.301756in}}%
\pgfpathlineto{\pgfqpoint{3.075001in}{3.301756in}}%
\pgfpathlineto{\pgfqpoint{3.075001in}{3.298807in}}%
\pgfpathmoveto{\pgfqpoint{3.070460in}{3.301756in}}%
\pgfpathlineto{\pgfqpoint{3.070460in}{3.301756in}}%
\pgfpathlineto{\pgfqpoint{3.070460in}{3.304705in}}%
\pgfpathlineto{\pgfqpoint{3.075001in}{3.304705in}}%
\pgfpathlineto{\pgfqpoint{3.075001in}{3.301756in}}%
\pgfpathmoveto{\pgfqpoint{3.070460in}{3.304705in}}%
\pgfpathlineto{\pgfqpoint{3.070460in}{3.304705in}}%
\pgfpathlineto{\pgfqpoint{3.070460in}{3.307654in}}%
\pgfpathlineto{\pgfqpoint{3.075001in}{3.307654in}}%
\pgfpathlineto{\pgfqpoint{3.075001in}{3.304705in}}%
\pgfpathmoveto{\pgfqpoint{3.070460in}{3.307654in}}%
\pgfpathlineto{\pgfqpoint{3.070460in}{3.307654in}}%
\pgfpathlineto{\pgfqpoint{3.070460in}{3.310604in}}%
\pgfpathlineto{\pgfqpoint{3.075001in}{3.310604in}}%
\pgfpathlineto{\pgfqpoint{3.075001in}{3.307654in}}%
\pgfpathmoveto{\pgfqpoint{3.070460in}{3.310604in}}%
\pgfpathlineto{\pgfqpoint{3.070460in}{3.310604in}}%
\pgfpathlineto{\pgfqpoint{3.070460in}{3.313553in}}%
\pgfpathlineto{\pgfqpoint{3.075001in}{3.313553in}}%
\pgfpathlineto{\pgfqpoint{3.075001in}{3.310604in}}%
\pgfpathmoveto{\pgfqpoint{3.070460in}{3.313553in}}%
\pgfpathlineto{\pgfqpoint{3.070460in}{3.313553in}}%
\pgfpathlineto{\pgfqpoint{3.070460in}{3.316502in}}%
\pgfpathlineto{\pgfqpoint{3.075001in}{3.316502in}}%
\pgfpathlineto{\pgfqpoint{3.075001in}{3.313553in}}%
\pgfpathmoveto{\pgfqpoint{3.070460in}{3.316502in}}%
\pgfpathlineto{\pgfqpoint{3.070460in}{3.316502in}}%
\pgfpathlineto{\pgfqpoint{3.070460in}{3.319451in}}%
\pgfpathlineto{\pgfqpoint{3.075001in}{3.319451in}}%
\pgfpathlineto{\pgfqpoint{3.075001in}{3.316502in}}%
\pgfpathmoveto{\pgfqpoint{3.070460in}{3.319451in}}%
\pgfpathlineto{\pgfqpoint{3.070460in}{3.319451in}}%
\pgfpathlineto{\pgfqpoint{3.070460in}{3.322400in}}%
\pgfpathlineto{\pgfqpoint{3.075001in}{3.322400in}}%
\pgfpathlineto{\pgfqpoint{3.075001in}{3.319451in}}%
\pgfpathmoveto{\pgfqpoint{3.070460in}{3.322400in}}%
\pgfpathlineto{\pgfqpoint{3.070460in}{3.322400in}}%
\pgfpathlineto{\pgfqpoint{3.070460in}{3.325349in}}%
\pgfpathlineto{\pgfqpoint{3.075001in}{3.325349in}}%
\pgfpathlineto{\pgfqpoint{3.075001in}{3.322400in}}%
\pgfpathmoveto{\pgfqpoint{3.070460in}{3.325349in}}%
\pgfpathlineto{\pgfqpoint{3.070460in}{3.325349in}}%
\pgfpathlineto{\pgfqpoint{3.070460in}{3.328299in}}%
\pgfpathlineto{\pgfqpoint{3.075001in}{3.328299in}}%
\pgfpathlineto{\pgfqpoint{3.075001in}{3.325349in}}%
\pgfpathmoveto{\pgfqpoint{3.070460in}{3.328299in}}%
\pgfpathlineto{\pgfqpoint{3.070460in}{3.328299in}}%
\pgfpathlineto{\pgfqpoint{3.070460in}{3.331248in}}%
\pgfpathlineto{\pgfqpoint{3.075001in}{3.331248in}}%
\pgfpathlineto{\pgfqpoint{3.075001in}{3.328299in}}%
\pgfpathmoveto{\pgfqpoint{3.070460in}{3.331248in}}%
\pgfpathlineto{\pgfqpoint{3.070460in}{3.331248in}}%
\pgfpathlineto{\pgfqpoint{3.070460in}{3.334197in}}%
\pgfpathlineto{\pgfqpoint{3.075001in}{3.334197in}}%
\pgfpathlineto{\pgfqpoint{3.075001in}{3.331248in}}%
\pgfpathmoveto{\pgfqpoint{3.070460in}{3.334197in}}%
\pgfpathlineto{\pgfqpoint{3.070460in}{3.334197in}}%
\pgfpathlineto{\pgfqpoint{3.070460in}{3.337146in}}%
\pgfpathlineto{\pgfqpoint{3.075001in}{3.337146in}}%
\pgfpathlineto{\pgfqpoint{3.075001in}{3.334197in}}%
\pgfpathmoveto{\pgfqpoint{3.070460in}{3.337146in}}%
\pgfpathlineto{\pgfqpoint{3.070460in}{3.337146in}}%
\pgfpathlineto{\pgfqpoint{3.070460in}{3.340096in}}%
\pgfpathlineto{\pgfqpoint{3.075001in}{3.340096in}}%
\pgfpathlineto{\pgfqpoint{3.075001in}{3.337146in}}%
\pgfpathmoveto{\pgfqpoint{3.070460in}{3.340096in}}%
\pgfpathlineto{\pgfqpoint{3.070460in}{3.340096in}}%
\pgfpathlineto{\pgfqpoint{3.070460in}{3.343045in}}%
\pgfpathlineto{\pgfqpoint{3.075001in}{3.343045in}}%
\pgfpathlineto{\pgfqpoint{3.075001in}{3.340096in}}%
\pgfpathmoveto{\pgfqpoint{3.070460in}{3.343045in}}%
\pgfpathlineto{\pgfqpoint{3.070460in}{3.343045in}}%
\pgfpathlineto{\pgfqpoint{3.070460in}{3.345994in}}%
\pgfpathlineto{\pgfqpoint{3.075001in}{3.345994in}}%
\pgfpathlineto{\pgfqpoint{3.075001in}{3.343045in}}%
\pgfpathmoveto{\pgfqpoint{3.070460in}{3.345994in}}%
\pgfpathlineto{\pgfqpoint{3.070460in}{3.345994in}}%
\pgfpathlineto{\pgfqpoint{3.070460in}{3.348943in}}%
\pgfpathlineto{\pgfqpoint{3.075001in}{3.348943in}}%
\pgfpathlineto{\pgfqpoint{3.075001in}{3.345994in}}%
\pgfpathmoveto{\pgfqpoint{3.070460in}{3.348943in}}%
\pgfpathlineto{\pgfqpoint{3.070460in}{3.348943in}}%
\pgfpathlineto{\pgfqpoint{3.070460in}{3.351893in}}%
\pgfpathlineto{\pgfqpoint{3.075001in}{3.351893in}}%
\pgfpathlineto{\pgfqpoint{3.075001in}{3.348943in}}%
\pgfpathmoveto{\pgfqpoint{3.070460in}{3.351893in}}%
\pgfpathlineto{\pgfqpoint{3.070460in}{3.351893in}}%
\pgfpathlineto{\pgfqpoint{3.070460in}{3.354842in}}%
\pgfpathlineto{\pgfqpoint{3.075001in}{3.354842in}}%
\pgfpathlineto{\pgfqpoint{3.075001in}{3.351893in}}%
\pgfpathmoveto{\pgfqpoint{3.070460in}{3.354842in}}%
\pgfpathlineto{\pgfqpoint{3.070460in}{3.354842in}}%
\pgfpathlineto{\pgfqpoint{3.070460in}{3.357791in}}%
\pgfpathlineto{\pgfqpoint{3.075001in}{3.357791in}}%
\pgfpathlineto{\pgfqpoint{3.075001in}{3.354842in}}%
\pgfpathmoveto{\pgfqpoint{3.070460in}{3.357791in}}%
\pgfpathlineto{\pgfqpoint{3.070460in}{3.357791in}}%
\pgfpathlineto{\pgfqpoint{3.070460in}{3.360741in}}%
\pgfpathlineto{\pgfqpoint{3.075001in}{3.360741in}}%
\pgfpathlineto{\pgfqpoint{3.075001in}{3.357791in}}%
\pgfpathmoveto{\pgfqpoint{3.070460in}{3.360741in}}%
\pgfpathlineto{\pgfqpoint{3.070460in}{3.360741in}}%
\pgfpathlineto{\pgfqpoint{3.070460in}{3.363690in}}%
\pgfpathlineto{\pgfqpoint{3.075001in}{3.363690in}}%
\pgfpathlineto{\pgfqpoint{3.075001in}{3.360741in}}%
\pgfpathmoveto{\pgfqpoint{3.070460in}{3.363690in}}%
\pgfpathlineto{\pgfqpoint{3.070460in}{3.363690in}}%
\pgfpathlineto{\pgfqpoint{3.070460in}{3.366639in}}%
\pgfpathlineto{\pgfqpoint{3.075001in}{3.366639in}}%
\pgfpathlineto{\pgfqpoint{3.075001in}{3.363690in}}%
\pgfpathmoveto{\pgfqpoint{3.070460in}{3.366639in}}%
\pgfpathlineto{\pgfqpoint{3.070460in}{3.366639in}}%
\pgfpathlineto{\pgfqpoint{3.070460in}{3.369588in}}%
\pgfpathlineto{\pgfqpoint{3.075001in}{3.369588in}}%
\pgfpathlineto{\pgfqpoint{3.075001in}{3.366639in}}%
\pgfpathmoveto{\pgfqpoint{3.070460in}{3.369588in}}%
\pgfpathlineto{\pgfqpoint{3.070460in}{3.369588in}}%
\pgfpathlineto{\pgfqpoint{3.070460in}{3.372538in}}%
\pgfpathlineto{\pgfqpoint{3.075001in}{3.372538in}}%
\pgfpathlineto{\pgfqpoint{3.075001in}{3.369588in}}%
\pgfpathmoveto{\pgfqpoint{3.070460in}{3.372538in}}%
\pgfpathlineto{\pgfqpoint{3.070460in}{3.372538in}}%
\pgfpathlineto{\pgfqpoint{3.070460in}{3.375487in}}%
\pgfpathlineto{\pgfqpoint{3.075001in}{3.375487in}}%
\pgfpathlineto{\pgfqpoint{3.075001in}{3.372538in}}%
\pgfpathmoveto{\pgfqpoint{3.070460in}{3.375487in}}%
\pgfpathlineto{\pgfqpoint{3.070460in}{3.375487in}}%
\pgfpathlineto{\pgfqpoint{3.070460in}{3.378436in}}%
\pgfpathlineto{\pgfqpoint{3.075001in}{3.378436in}}%
\pgfpathlineto{\pgfqpoint{3.075001in}{3.375487in}}%
\pgfpathmoveto{\pgfqpoint{3.070460in}{3.378436in}}%
\pgfpathlineto{\pgfqpoint{3.070460in}{3.378436in}}%
\pgfpathlineto{\pgfqpoint{3.070460in}{3.381386in}}%
\pgfpathlineto{\pgfqpoint{3.075001in}{3.381386in}}%
\pgfpathlineto{\pgfqpoint{3.075001in}{3.378436in}}%
\pgfpathmoveto{\pgfqpoint{3.070460in}{3.381386in}}%
\pgfpathlineto{\pgfqpoint{3.070460in}{3.381386in}}%
\pgfpathlineto{\pgfqpoint{3.070460in}{3.384335in}}%
\pgfpathlineto{\pgfqpoint{3.075001in}{3.384335in}}%
\pgfpathlineto{\pgfqpoint{3.075001in}{3.381386in}}%
\pgfpathmoveto{\pgfqpoint{3.070460in}{3.384335in}}%
\pgfpathlineto{\pgfqpoint{3.070460in}{3.384335in}}%
\pgfpathlineto{\pgfqpoint{3.070460in}{3.387284in}}%
\pgfpathlineto{\pgfqpoint{3.075001in}{3.387284in}}%
\pgfpathlineto{\pgfqpoint{3.075001in}{3.384335in}}%
\pgfpathmoveto{\pgfqpoint{3.070460in}{3.387284in}}%
\pgfpathlineto{\pgfqpoint{3.070460in}{3.387284in}}%
\pgfpathlineto{\pgfqpoint{3.070460in}{3.390233in}}%
\pgfpathlineto{\pgfqpoint{3.075001in}{3.390233in}}%
\pgfpathlineto{\pgfqpoint{3.075001in}{3.387284in}}%
\pgfpathmoveto{\pgfqpoint{3.070460in}{3.390233in}}%
\pgfpathlineto{\pgfqpoint{3.070460in}{3.390233in}}%
\pgfpathlineto{\pgfqpoint{3.070460in}{3.393183in}}%
\pgfpathlineto{\pgfqpoint{3.075001in}{3.393183in}}%
\pgfpathlineto{\pgfqpoint{3.075001in}{3.390233in}}%
\pgfpathmoveto{\pgfqpoint{3.070460in}{3.393183in}}%
\pgfpathlineto{\pgfqpoint{3.070460in}{3.393183in}}%
\pgfpathlineto{\pgfqpoint{3.070460in}{3.396132in}}%
\pgfpathlineto{\pgfqpoint{3.075001in}{3.396132in}}%
\pgfpathlineto{\pgfqpoint{3.075001in}{3.393183in}}%
\pgfpathmoveto{\pgfqpoint{3.070460in}{3.396132in}}%
\pgfpathlineto{\pgfqpoint{3.070460in}{3.396132in}}%
\pgfpathlineto{\pgfqpoint{3.070460in}{3.399081in}}%
\pgfpathlineto{\pgfqpoint{3.075001in}{3.399081in}}%
\pgfpathlineto{\pgfqpoint{3.075001in}{3.396132in}}%
\pgfpathmoveto{\pgfqpoint{3.070460in}{3.399081in}}%
\pgfpathlineto{\pgfqpoint{3.070460in}{3.399081in}}%
\pgfpathlineto{\pgfqpoint{3.070460in}{3.402031in}}%
\pgfpathlineto{\pgfqpoint{3.075001in}{3.402031in}}%
\pgfpathlineto{\pgfqpoint{3.075001in}{3.399081in}}%
\pgfpathmoveto{\pgfqpoint{3.070460in}{3.402031in}}%
\pgfpathlineto{\pgfqpoint{3.070460in}{3.402031in}}%
\pgfpathlineto{\pgfqpoint{3.070460in}{3.404980in}}%
\pgfpathlineto{\pgfqpoint{3.075001in}{3.404980in}}%
\pgfpathlineto{\pgfqpoint{3.075001in}{3.402031in}}%
\pgfpathmoveto{\pgfqpoint{3.070460in}{3.404980in}}%
\pgfpathlineto{\pgfqpoint{3.070460in}{3.404980in}}%
\pgfpathlineto{\pgfqpoint{3.070460in}{3.407929in}}%
\pgfpathlineto{\pgfqpoint{3.075001in}{3.407929in}}%
\pgfpathlineto{\pgfqpoint{3.075001in}{3.404980in}}%
\pgfpathmoveto{\pgfqpoint{3.070460in}{3.407929in}}%
\pgfpathlineto{\pgfqpoint{3.070460in}{3.407929in}}%
\pgfpathlineto{\pgfqpoint{3.070460in}{3.410879in}}%
\pgfpathlineto{\pgfqpoint{3.075001in}{3.410879in}}%
\pgfpathlineto{\pgfqpoint{3.075001in}{3.407929in}}%
\pgfpathmoveto{\pgfqpoint{3.070460in}{3.410879in}}%
\pgfpathlineto{\pgfqpoint{3.070460in}{3.410879in}}%
\pgfpathlineto{\pgfqpoint{3.070460in}{3.413828in}}%
\pgfpathlineto{\pgfqpoint{3.075001in}{3.413828in}}%
\pgfpathlineto{\pgfqpoint{3.075001in}{3.410879in}}%
\pgfpathmoveto{\pgfqpoint{3.070460in}{3.413828in}}%
\pgfpathlineto{\pgfqpoint{3.070460in}{3.413828in}}%
\pgfpathlineto{\pgfqpoint{3.070460in}{3.416777in}}%
\pgfpathlineto{\pgfqpoint{3.075001in}{3.416777in}}%
\pgfpathlineto{\pgfqpoint{3.075001in}{3.413828in}}%
\pgfpathmoveto{\pgfqpoint{3.070460in}{3.416777in}}%
\pgfpathlineto{\pgfqpoint{3.070460in}{3.416777in}}%
\pgfpathlineto{\pgfqpoint{3.070460in}{3.419726in}}%
\pgfpathlineto{\pgfqpoint{3.075001in}{3.419726in}}%
\pgfpathlineto{\pgfqpoint{3.075001in}{3.416777in}}%
\pgfpathmoveto{\pgfqpoint{3.070460in}{3.419726in}}%
\pgfpathlineto{\pgfqpoint{3.070460in}{3.419726in}}%
\pgfpathlineto{\pgfqpoint{3.070460in}{3.422676in}}%
\pgfpathlineto{\pgfqpoint{3.075001in}{3.422676in}}%
\pgfpathlineto{\pgfqpoint{3.075001in}{3.419726in}}%
\pgfpathmoveto{\pgfqpoint{3.070460in}{3.422676in}}%
\pgfpathlineto{\pgfqpoint{3.070460in}{3.422676in}}%
\pgfpathlineto{\pgfqpoint{3.070460in}{3.425625in}}%
\pgfpathlineto{\pgfqpoint{3.075001in}{3.425625in}}%
\pgfpathlineto{\pgfqpoint{3.075001in}{3.422676in}}%
\pgfpathmoveto{\pgfqpoint{3.070460in}{3.425625in}}%
\pgfpathlineto{\pgfqpoint{3.070460in}{3.425625in}}%
\pgfpathlineto{\pgfqpoint{3.070460in}{3.428574in}}%
\pgfpathlineto{\pgfqpoint{3.075001in}{3.428574in}}%
\pgfpathlineto{\pgfqpoint{3.075001in}{3.425625in}}%
\pgfpathmoveto{\pgfqpoint{3.070460in}{3.428574in}}%
\pgfpathlineto{\pgfqpoint{3.070460in}{3.428574in}}%
\pgfpathlineto{\pgfqpoint{3.070460in}{3.431524in}}%
\pgfpathlineto{\pgfqpoint{3.075001in}{3.431524in}}%
\pgfpathlineto{\pgfqpoint{3.075001in}{3.428574in}}%
\pgfpathmoveto{\pgfqpoint{3.070460in}{3.431524in}}%
\pgfpathlineto{\pgfqpoint{3.070460in}{3.431524in}}%
\pgfpathlineto{\pgfqpoint{3.070460in}{3.434473in}}%
\pgfpathlineto{\pgfqpoint{3.075001in}{3.434473in}}%
\pgfpathlineto{\pgfqpoint{3.075001in}{3.431524in}}%
\pgfpathmoveto{\pgfqpoint{3.070460in}{3.434473in}}%
\pgfpathlineto{\pgfqpoint{3.070460in}{3.434473in}}%
\pgfpathlineto{\pgfqpoint{3.070460in}{3.437422in}}%
\pgfpathlineto{\pgfqpoint{3.075001in}{3.437422in}}%
\pgfpathlineto{\pgfqpoint{3.075001in}{3.434473in}}%
\pgfpathmoveto{\pgfqpoint{3.070460in}{3.437422in}}%
\pgfpathlineto{\pgfqpoint{3.070460in}{3.437422in}}%
\pgfpathlineto{\pgfqpoint{3.070460in}{3.440371in}}%
\pgfpathlineto{\pgfqpoint{3.075001in}{3.440371in}}%
\pgfpathlineto{\pgfqpoint{3.075001in}{3.437422in}}%
\pgfpathmoveto{\pgfqpoint{3.070460in}{3.440371in}}%
\pgfpathlineto{\pgfqpoint{3.070460in}{3.440371in}}%
\pgfpathlineto{\pgfqpoint{3.070460in}{3.443321in}}%
\pgfpathlineto{\pgfqpoint{3.075001in}{3.443321in}}%
\pgfpathlineto{\pgfqpoint{3.075001in}{3.440371in}}%
\pgfpathmoveto{\pgfqpoint{3.070460in}{3.443321in}}%
\pgfpathlineto{\pgfqpoint{3.070460in}{3.443321in}}%
\pgfpathlineto{\pgfqpoint{3.070460in}{3.446270in}}%
\pgfpathlineto{\pgfqpoint{3.075001in}{3.446270in}}%
\pgfpathlineto{\pgfqpoint{3.075001in}{3.443321in}}%
\pgfpathmoveto{\pgfqpoint{3.070460in}{3.446270in}}%
\pgfpathlineto{\pgfqpoint{3.070460in}{3.446270in}}%
\pgfpathlineto{\pgfqpoint{3.070460in}{3.449219in}}%
\pgfpathlineto{\pgfqpoint{3.075001in}{3.449219in}}%
\pgfpathlineto{\pgfqpoint{3.075001in}{3.446270in}}%
\pgfpathmoveto{\pgfqpoint{3.070460in}{3.449219in}}%
\pgfpathlineto{\pgfqpoint{3.070460in}{3.449219in}}%
\pgfpathlineto{\pgfqpoint{3.070460in}{3.452168in}}%
\pgfpathlineto{\pgfqpoint{3.075001in}{3.452168in}}%
\pgfpathlineto{\pgfqpoint{3.075001in}{3.449219in}}%
\pgfpathmoveto{\pgfqpoint{3.070460in}{3.452168in}}%
\pgfpathlineto{\pgfqpoint{3.070460in}{3.452168in}}%
\pgfpathlineto{\pgfqpoint{3.070460in}{3.455118in}}%
\pgfpathlineto{\pgfqpoint{3.075001in}{3.455118in}}%
\pgfpathlineto{\pgfqpoint{3.075001in}{3.452168in}}%
\pgfpathmoveto{\pgfqpoint{3.070460in}{3.455118in}}%
\pgfpathlineto{\pgfqpoint{3.070460in}{3.455118in}}%
\pgfpathlineto{\pgfqpoint{3.070460in}{3.458067in}}%
\pgfpathlineto{\pgfqpoint{3.075001in}{3.458067in}}%
\pgfpathlineto{\pgfqpoint{3.075001in}{3.455118in}}%
\pgfpathmoveto{\pgfqpoint{3.070460in}{3.458067in}}%
\pgfpathlineto{\pgfqpoint{3.070460in}{3.458067in}}%
\pgfpathlineto{\pgfqpoint{3.070460in}{3.461016in}}%
\pgfpathlineto{\pgfqpoint{3.075001in}{3.461016in}}%
\pgfpathlineto{\pgfqpoint{3.075001in}{3.458067in}}%
\pgfpathmoveto{\pgfqpoint{3.070460in}{3.461016in}}%
\pgfpathlineto{\pgfqpoint{3.070460in}{3.461016in}}%
\pgfpathlineto{\pgfqpoint{3.070460in}{3.463966in}}%
\pgfpathlineto{\pgfqpoint{3.075001in}{3.463966in}}%
\pgfpathlineto{\pgfqpoint{3.075001in}{3.461016in}}%
\pgfpathmoveto{\pgfqpoint{3.070460in}{3.463966in}}%
\pgfpathlineto{\pgfqpoint{3.070460in}{3.463966in}}%
\pgfpathlineto{\pgfqpoint{3.070460in}{3.466915in}}%
\pgfpathlineto{\pgfqpoint{3.075001in}{3.466915in}}%
\pgfpathlineto{\pgfqpoint{3.075001in}{3.463966in}}%
\pgfpathmoveto{\pgfqpoint{3.070460in}{3.466915in}}%
\pgfpathlineto{\pgfqpoint{3.070460in}{3.466915in}}%
\pgfpathlineto{\pgfqpoint{3.070460in}{3.469864in}}%
\pgfpathlineto{\pgfqpoint{3.075001in}{3.469864in}}%
\pgfpathlineto{\pgfqpoint{3.075001in}{3.466915in}}%
\pgfpathmoveto{\pgfqpoint{3.070460in}{3.469864in}}%
\pgfpathlineto{\pgfqpoint{3.070460in}{3.469864in}}%
\pgfpathlineto{\pgfqpoint{3.070460in}{3.472813in}}%
\pgfpathlineto{\pgfqpoint{3.075001in}{3.472813in}}%
\pgfpathlineto{\pgfqpoint{3.075001in}{3.469864in}}%
\pgfpathmoveto{\pgfqpoint{3.070460in}{3.472813in}}%
\pgfpathlineto{\pgfqpoint{3.070460in}{3.472813in}}%
\pgfpathlineto{\pgfqpoint{3.070460in}{3.475763in}}%
\pgfpathlineto{\pgfqpoint{3.075001in}{3.475763in}}%
\pgfpathlineto{\pgfqpoint{3.075001in}{3.472813in}}%
\pgfpathmoveto{\pgfqpoint{3.070460in}{3.475763in}}%
\pgfpathlineto{\pgfqpoint{3.070460in}{3.475763in}}%
\pgfpathlineto{\pgfqpoint{3.070460in}{3.478712in}}%
\pgfpathlineto{\pgfqpoint{3.075001in}{3.478712in}}%
\pgfpathlineto{\pgfqpoint{3.075001in}{3.475763in}}%
\pgfpathmoveto{\pgfqpoint{3.070460in}{3.478712in}}%
\pgfpathlineto{\pgfqpoint{3.070460in}{3.478712in}}%
\pgfpathlineto{\pgfqpoint{3.070460in}{3.481661in}}%
\pgfpathlineto{\pgfqpoint{3.075001in}{3.481661in}}%
\pgfpathlineto{\pgfqpoint{3.075001in}{3.478712in}}%
\pgfpathmoveto{\pgfqpoint{3.070460in}{3.481661in}}%
\pgfpathlineto{\pgfqpoint{3.070460in}{3.481661in}}%
\pgfpathlineto{\pgfqpoint{3.070460in}{3.484610in}}%
\pgfpathlineto{\pgfqpoint{3.075001in}{3.484610in}}%
\pgfpathlineto{\pgfqpoint{3.075001in}{3.481661in}}%
\pgfpathmoveto{\pgfqpoint{3.070460in}{3.484610in}}%
\pgfpathlineto{\pgfqpoint{3.070460in}{3.484610in}}%
\pgfpathlineto{\pgfqpoint{3.070460in}{3.487560in}}%
\pgfpathlineto{\pgfqpoint{3.075001in}{3.487560in}}%
\pgfpathlineto{\pgfqpoint{3.075001in}{3.484610in}}%
\pgfpathmoveto{\pgfqpoint{3.070460in}{3.487560in}}%
\pgfpathlineto{\pgfqpoint{3.070460in}{3.487560in}}%
\pgfpathlineto{\pgfqpoint{3.070460in}{3.490509in}}%
\pgfpathlineto{\pgfqpoint{3.075001in}{3.490509in}}%
\pgfpathlineto{\pgfqpoint{3.075001in}{3.487560in}}%
\pgfpathmoveto{\pgfqpoint{3.070460in}{3.490509in}}%
\pgfpathlineto{\pgfqpoint{3.070460in}{3.490509in}}%
\pgfpathlineto{\pgfqpoint{3.070460in}{3.493458in}}%
\pgfpathlineto{\pgfqpoint{3.075001in}{3.493458in}}%
\pgfpathlineto{\pgfqpoint{3.075001in}{3.490509in}}%
\pgfpathmoveto{\pgfqpoint{3.070460in}{3.493458in}}%
\pgfpathlineto{\pgfqpoint{3.070460in}{3.493458in}}%
\pgfpathlineto{\pgfqpoint{3.070460in}{3.496408in}}%
\pgfpathlineto{\pgfqpoint{3.075001in}{3.496408in}}%
\pgfpathlineto{\pgfqpoint{3.075001in}{3.493458in}}%
\pgfpathmoveto{\pgfqpoint{3.070460in}{3.496408in}}%
\pgfpathlineto{\pgfqpoint{3.070460in}{3.496408in}}%
\pgfpathlineto{\pgfqpoint{3.070460in}{3.499357in}}%
\pgfpathlineto{\pgfqpoint{3.075001in}{3.499357in}}%
\pgfpathlineto{\pgfqpoint{3.075001in}{3.496408in}}%
\pgfpathmoveto{\pgfqpoint{3.070460in}{3.499357in}}%
\pgfpathlineto{\pgfqpoint{3.070460in}{3.499357in}}%
\pgfpathlineto{\pgfqpoint{3.070460in}{3.502306in}}%
\pgfpathlineto{\pgfqpoint{3.075001in}{3.502306in}}%
\pgfpathlineto{\pgfqpoint{3.075001in}{3.499357in}}%
\pgfpathmoveto{\pgfqpoint{3.070460in}{3.502306in}}%
\pgfpathlineto{\pgfqpoint{3.070460in}{3.502306in}}%
\pgfpathlineto{\pgfqpoint{3.070460in}{3.505255in}}%
\pgfpathlineto{\pgfqpoint{3.075001in}{3.505255in}}%
\pgfpathlineto{\pgfqpoint{3.075001in}{3.502306in}}%
\pgfpathmoveto{\pgfqpoint{3.070460in}{3.505255in}}%
\pgfpathlineto{\pgfqpoint{3.070460in}{3.505255in}}%
\pgfpathlineto{\pgfqpoint{3.070460in}{3.508205in}}%
\pgfpathlineto{\pgfqpoint{3.075001in}{3.508205in}}%
\pgfpathlineto{\pgfqpoint{3.075001in}{3.505255in}}%
\pgfpathmoveto{\pgfqpoint{3.070460in}{3.508205in}}%
\pgfpathlineto{\pgfqpoint{3.070460in}{3.508205in}}%
\pgfpathlineto{\pgfqpoint{3.070460in}{3.511154in}}%
\pgfpathlineto{\pgfqpoint{3.075001in}{3.511154in}}%
\pgfpathlineto{\pgfqpoint{3.075001in}{3.508205in}}%
\pgfpathmoveto{\pgfqpoint{3.070460in}{3.511154in}}%
\pgfpathlineto{\pgfqpoint{3.070460in}{3.511154in}}%
\pgfpathlineto{\pgfqpoint{3.070460in}{3.514103in}}%
\pgfpathlineto{\pgfqpoint{3.075001in}{3.514103in}}%
\pgfpathlineto{\pgfqpoint{3.075001in}{3.511154in}}%
\pgfpathmoveto{\pgfqpoint{3.070460in}{3.514103in}}%
\pgfpathlineto{\pgfqpoint{3.070460in}{3.514103in}}%
\pgfpathlineto{\pgfqpoint{3.070460in}{3.517052in}}%
\pgfpathlineto{\pgfqpoint{3.075001in}{3.517052in}}%
\pgfpathlineto{\pgfqpoint{3.075001in}{3.514103in}}%
\pgfpathmoveto{\pgfqpoint{3.070460in}{3.517052in}}%
\pgfpathlineto{\pgfqpoint{3.070460in}{3.517052in}}%
\pgfpathlineto{\pgfqpoint{3.070460in}{3.520002in}}%
\pgfpathlineto{\pgfqpoint{3.075001in}{3.520002in}}%
\pgfpathlineto{\pgfqpoint{3.075001in}{3.517052in}}%
\pgfpathmoveto{\pgfqpoint{3.115870in}{2.903614in}}%
\pgfpathlineto{\pgfqpoint{3.115870in}{2.903614in}}%
\pgfpathlineto{\pgfqpoint{3.115870in}{2.906564in}}%
\pgfpathlineto{\pgfqpoint{3.120411in}{2.906564in}}%
\pgfpathlineto{\pgfqpoint{3.120411in}{2.903614in}}%
\pgfpathmoveto{\pgfqpoint{3.120411in}{2.903614in}}%
\pgfpathlineto{\pgfqpoint{3.120411in}{2.903614in}}%
\pgfpathlineto{\pgfqpoint{3.120411in}{2.906564in}}%
\pgfpathlineto{\pgfqpoint{3.124952in}{2.906564in}}%
\pgfpathlineto{\pgfqpoint{3.124952in}{2.903614in}}%
\pgfpathmoveto{\pgfqpoint{3.124952in}{2.903614in}}%
\pgfpathlineto{\pgfqpoint{3.124952in}{2.903614in}}%
\pgfpathlineto{\pgfqpoint{3.124952in}{2.906564in}}%
\pgfpathlineto{\pgfqpoint{3.129493in}{2.906564in}}%
\pgfpathlineto{\pgfqpoint{3.129493in}{2.903614in}}%
\pgfpathmoveto{\pgfqpoint{3.129493in}{2.900665in}}%
\pgfpathlineto{\pgfqpoint{3.129493in}{2.900665in}}%
\pgfpathlineto{\pgfqpoint{3.129493in}{2.903614in}}%
\pgfpathlineto{\pgfqpoint{3.134034in}{2.903614in}}%
\pgfpathlineto{\pgfqpoint{3.134034in}{2.900665in}}%
\pgfpathmoveto{\pgfqpoint{3.129493in}{2.903614in}}%
\pgfpathlineto{\pgfqpoint{3.129493in}{2.903614in}}%
\pgfpathlineto{\pgfqpoint{3.129493in}{2.906564in}}%
\pgfpathlineto{\pgfqpoint{3.134034in}{2.906564in}}%
\pgfpathlineto{\pgfqpoint{3.134034in}{2.903614in}}%
\pgfpathmoveto{\pgfqpoint{3.134034in}{2.900665in}}%
\pgfpathlineto{\pgfqpoint{3.134034in}{2.900665in}}%
\pgfpathlineto{\pgfqpoint{3.134034in}{2.903614in}}%
\pgfpathlineto{\pgfqpoint{3.138575in}{2.903614in}}%
\pgfpathlineto{\pgfqpoint{3.138575in}{2.900665in}}%
\pgfpathmoveto{\pgfqpoint{3.134034in}{2.903614in}}%
\pgfpathlineto{\pgfqpoint{3.134034in}{2.903614in}}%
\pgfpathlineto{\pgfqpoint{3.134034in}{2.906564in}}%
\pgfpathlineto{\pgfqpoint{3.138575in}{2.906564in}}%
\pgfpathlineto{\pgfqpoint{3.138575in}{2.903614in}}%
\pgfpathmoveto{\pgfqpoint{3.143116in}{2.897716in}}%
\pgfpathlineto{\pgfqpoint{3.143116in}{2.897716in}}%
\pgfpathlineto{\pgfqpoint{3.143116in}{2.900665in}}%
\pgfpathlineto{\pgfqpoint{3.147657in}{2.900665in}}%
\pgfpathlineto{\pgfqpoint{3.147657in}{2.897716in}}%
\pgfpathmoveto{\pgfqpoint{3.138575in}{2.900665in}}%
\pgfpathlineto{\pgfqpoint{3.138575in}{2.900665in}}%
\pgfpathlineto{\pgfqpoint{3.138575in}{2.903614in}}%
\pgfpathlineto{\pgfqpoint{3.143116in}{2.903614in}}%
\pgfpathlineto{\pgfqpoint{3.143116in}{2.900665in}}%
\pgfpathmoveto{\pgfqpoint{3.138575in}{2.903614in}}%
\pgfpathlineto{\pgfqpoint{3.138575in}{2.903614in}}%
\pgfpathlineto{\pgfqpoint{3.138575in}{2.906564in}}%
\pgfpathlineto{\pgfqpoint{3.143116in}{2.906564in}}%
\pgfpathlineto{\pgfqpoint{3.143116in}{2.903614in}}%
\pgfpathmoveto{\pgfqpoint{3.143116in}{2.900665in}}%
\pgfpathlineto{\pgfqpoint{3.143116in}{2.900665in}}%
\pgfpathlineto{\pgfqpoint{3.143116in}{2.903614in}}%
\pgfpathlineto{\pgfqpoint{3.147657in}{2.903614in}}%
\pgfpathlineto{\pgfqpoint{3.147657in}{2.900665in}}%
\pgfpathmoveto{\pgfqpoint{3.143116in}{2.903614in}}%
\pgfpathlineto{\pgfqpoint{3.143116in}{2.903614in}}%
\pgfpathlineto{\pgfqpoint{3.143116in}{2.906564in}}%
\pgfpathlineto{\pgfqpoint{3.147657in}{2.906564in}}%
\pgfpathlineto{\pgfqpoint{3.147657in}{2.903614in}}%
\pgfpathmoveto{\pgfqpoint{3.075001in}{2.912462in}}%
\pgfpathlineto{\pgfqpoint{3.075001in}{2.912462in}}%
\pgfpathlineto{\pgfqpoint{3.075001in}{2.915411in}}%
\pgfpathlineto{\pgfqpoint{3.079542in}{2.915411in}}%
\pgfpathlineto{\pgfqpoint{3.079542in}{2.912462in}}%
\pgfpathmoveto{\pgfqpoint{3.075001in}{2.915411in}}%
\pgfpathlineto{\pgfqpoint{3.075001in}{2.915411in}}%
\pgfpathlineto{\pgfqpoint{3.075001in}{2.918361in}}%
\pgfpathlineto{\pgfqpoint{3.079542in}{2.918361in}}%
\pgfpathlineto{\pgfqpoint{3.079542in}{2.915411in}}%
\pgfpathmoveto{\pgfqpoint{3.079542in}{2.912462in}}%
\pgfpathlineto{\pgfqpoint{3.079542in}{2.912462in}}%
\pgfpathlineto{\pgfqpoint{3.079542in}{2.915411in}}%
\pgfpathlineto{\pgfqpoint{3.084083in}{2.915411in}}%
\pgfpathlineto{\pgfqpoint{3.084083in}{2.912462in}}%
\pgfpathmoveto{\pgfqpoint{3.079542in}{2.915411in}}%
\pgfpathlineto{\pgfqpoint{3.079542in}{2.915411in}}%
\pgfpathlineto{\pgfqpoint{3.079542in}{2.918361in}}%
\pgfpathlineto{\pgfqpoint{3.084083in}{2.918361in}}%
\pgfpathlineto{\pgfqpoint{3.084083in}{2.915411in}}%
\pgfpathmoveto{\pgfqpoint{3.088624in}{2.909513in}}%
\pgfpathlineto{\pgfqpoint{3.088624in}{2.909513in}}%
\pgfpathlineto{\pgfqpoint{3.088624in}{2.912462in}}%
\pgfpathlineto{\pgfqpoint{3.093165in}{2.912462in}}%
\pgfpathlineto{\pgfqpoint{3.093165in}{2.909513in}}%
\pgfpathmoveto{\pgfqpoint{3.084083in}{2.912462in}}%
\pgfpathlineto{\pgfqpoint{3.084083in}{2.912462in}}%
\pgfpathlineto{\pgfqpoint{3.084083in}{2.915411in}}%
\pgfpathlineto{\pgfqpoint{3.088624in}{2.915411in}}%
\pgfpathlineto{\pgfqpoint{3.088624in}{2.912462in}}%
\pgfpathmoveto{\pgfqpoint{3.084083in}{2.915411in}}%
\pgfpathlineto{\pgfqpoint{3.084083in}{2.915411in}}%
\pgfpathlineto{\pgfqpoint{3.084083in}{2.918361in}}%
\pgfpathlineto{\pgfqpoint{3.088624in}{2.918361in}}%
\pgfpathlineto{\pgfqpoint{3.088624in}{2.915411in}}%
\pgfpathmoveto{\pgfqpoint{3.088624in}{2.912462in}}%
\pgfpathlineto{\pgfqpoint{3.088624in}{2.912462in}}%
\pgfpathlineto{\pgfqpoint{3.088624in}{2.915411in}}%
\pgfpathlineto{\pgfqpoint{3.093165in}{2.915411in}}%
\pgfpathlineto{\pgfqpoint{3.093165in}{2.912462in}}%
\pgfpathmoveto{\pgfqpoint{3.088624in}{2.915411in}}%
\pgfpathlineto{\pgfqpoint{3.088624in}{2.915411in}}%
\pgfpathlineto{\pgfqpoint{3.088624in}{2.918361in}}%
\pgfpathlineto{\pgfqpoint{3.093165in}{2.918361in}}%
\pgfpathlineto{\pgfqpoint{3.093165in}{2.915411in}}%
\pgfpathmoveto{\pgfqpoint{3.093165in}{2.909513in}}%
\pgfpathlineto{\pgfqpoint{3.093165in}{2.909513in}}%
\pgfpathlineto{\pgfqpoint{3.093165in}{2.912462in}}%
\pgfpathlineto{\pgfqpoint{3.097706in}{2.912462in}}%
\pgfpathlineto{\pgfqpoint{3.097706in}{2.909513in}}%
\pgfpathmoveto{\pgfqpoint{3.097706in}{2.909513in}}%
\pgfpathlineto{\pgfqpoint{3.097706in}{2.909513in}}%
\pgfpathlineto{\pgfqpoint{3.097706in}{2.912462in}}%
\pgfpathlineto{\pgfqpoint{3.102247in}{2.912462in}}%
\pgfpathlineto{\pgfqpoint{3.102247in}{2.909513in}}%
\pgfpathmoveto{\pgfqpoint{3.102247in}{2.906564in}}%
\pgfpathlineto{\pgfqpoint{3.102247in}{2.906564in}}%
\pgfpathlineto{\pgfqpoint{3.102247in}{2.909513in}}%
\pgfpathlineto{\pgfqpoint{3.106788in}{2.909513in}}%
\pgfpathlineto{\pgfqpoint{3.106788in}{2.906564in}}%
\pgfpathmoveto{\pgfqpoint{3.102247in}{2.909513in}}%
\pgfpathlineto{\pgfqpoint{3.102247in}{2.909513in}}%
\pgfpathlineto{\pgfqpoint{3.102247in}{2.912462in}}%
\pgfpathlineto{\pgfqpoint{3.106788in}{2.912462in}}%
\pgfpathlineto{\pgfqpoint{3.106788in}{2.909513in}}%
\pgfpathmoveto{\pgfqpoint{3.106788in}{2.906564in}}%
\pgfpathlineto{\pgfqpoint{3.106788in}{2.906564in}}%
\pgfpathlineto{\pgfqpoint{3.106788in}{2.909513in}}%
\pgfpathlineto{\pgfqpoint{3.111329in}{2.909513in}}%
\pgfpathlineto{\pgfqpoint{3.111329in}{2.906564in}}%
\pgfpathmoveto{\pgfqpoint{3.106788in}{2.909513in}}%
\pgfpathlineto{\pgfqpoint{3.106788in}{2.909513in}}%
\pgfpathlineto{\pgfqpoint{3.106788in}{2.912462in}}%
\pgfpathlineto{\pgfqpoint{3.111329in}{2.912462in}}%
\pgfpathlineto{\pgfqpoint{3.111329in}{2.909513in}}%
\pgfpathmoveto{\pgfqpoint{3.111329in}{2.906564in}}%
\pgfpathlineto{\pgfqpoint{3.111329in}{2.906564in}}%
\pgfpathlineto{\pgfqpoint{3.111329in}{2.909513in}}%
\pgfpathlineto{\pgfqpoint{3.115870in}{2.909513in}}%
\pgfpathlineto{\pgfqpoint{3.115870in}{2.906564in}}%
\pgfpathmoveto{\pgfqpoint{3.111329in}{2.909513in}}%
\pgfpathlineto{\pgfqpoint{3.111329in}{2.909513in}}%
\pgfpathlineto{\pgfqpoint{3.111329in}{2.912462in}}%
\pgfpathlineto{\pgfqpoint{3.115870in}{2.912462in}}%
\pgfpathlineto{\pgfqpoint{3.115870in}{2.909513in}}%
\pgfpathmoveto{\pgfqpoint{3.115870in}{2.906564in}}%
\pgfpathlineto{\pgfqpoint{3.115870in}{2.906564in}}%
\pgfpathlineto{\pgfqpoint{3.115870in}{2.909513in}}%
\pgfpathlineto{\pgfqpoint{3.120411in}{2.909513in}}%
\pgfpathlineto{\pgfqpoint{3.120411in}{2.906564in}}%
\pgfpathmoveto{\pgfqpoint{3.115870in}{2.909513in}}%
\pgfpathlineto{\pgfqpoint{3.115870in}{2.909513in}}%
\pgfpathlineto{\pgfqpoint{3.115870in}{2.912462in}}%
\pgfpathlineto{\pgfqpoint{3.120411in}{2.912462in}}%
\pgfpathlineto{\pgfqpoint{3.120411in}{2.909513in}}%
\pgfpathmoveto{\pgfqpoint{3.147657in}{2.897716in}}%
\pgfpathlineto{\pgfqpoint{3.147657in}{2.897716in}}%
\pgfpathlineto{\pgfqpoint{3.147657in}{2.900665in}}%
\pgfpathlineto{\pgfqpoint{3.152199in}{2.900665in}}%
\pgfpathlineto{\pgfqpoint{3.152199in}{2.897716in}}%
\pgfpathmoveto{\pgfqpoint{3.152199in}{2.897716in}}%
\pgfpathlineto{\pgfqpoint{3.152199in}{2.897716in}}%
\pgfpathlineto{\pgfqpoint{3.152199in}{2.900665in}}%
\pgfpathlineto{\pgfqpoint{3.156740in}{2.900665in}}%
\pgfpathlineto{\pgfqpoint{3.156740in}{2.897716in}}%
\pgfpathmoveto{\pgfqpoint{3.156740in}{2.894767in}}%
\pgfpathlineto{\pgfqpoint{3.156740in}{2.894767in}}%
\pgfpathlineto{\pgfqpoint{3.156740in}{2.897716in}}%
\pgfpathlineto{\pgfqpoint{3.161281in}{2.897716in}}%
\pgfpathlineto{\pgfqpoint{3.161281in}{2.894767in}}%
\pgfpathmoveto{\pgfqpoint{3.156740in}{2.897716in}}%
\pgfpathlineto{\pgfqpoint{3.156740in}{2.897716in}}%
\pgfpathlineto{\pgfqpoint{3.156740in}{2.900665in}}%
\pgfpathlineto{\pgfqpoint{3.161281in}{2.900665in}}%
\pgfpathlineto{\pgfqpoint{3.161281in}{2.897716in}}%
\pgfpathmoveto{\pgfqpoint{3.161281in}{2.894767in}}%
\pgfpathlineto{\pgfqpoint{3.161281in}{2.894767in}}%
\pgfpathlineto{\pgfqpoint{3.161281in}{2.897716in}}%
\pgfpathlineto{\pgfqpoint{3.165822in}{2.897716in}}%
\pgfpathlineto{\pgfqpoint{3.165822in}{2.894767in}}%
\pgfpathmoveto{\pgfqpoint{3.161281in}{2.897716in}}%
\pgfpathlineto{\pgfqpoint{3.161281in}{2.897716in}}%
\pgfpathlineto{\pgfqpoint{3.161281in}{2.900665in}}%
\pgfpathlineto{\pgfqpoint{3.165822in}{2.900665in}}%
\pgfpathlineto{\pgfqpoint{3.165822in}{2.897716in}}%
\pgfpathmoveto{\pgfqpoint{3.170363in}{2.891817in}}%
\pgfpathlineto{\pgfqpoint{3.170363in}{2.891817in}}%
\pgfpathlineto{\pgfqpoint{3.170363in}{2.894767in}}%
\pgfpathlineto{\pgfqpoint{3.174904in}{2.894767in}}%
\pgfpathlineto{\pgfqpoint{3.174904in}{2.891817in}}%
\pgfpathmoveto{\pgfqpoint{3.174904in}{2.891817in}}%
\pgfpathlineto{\pgfqpoint{3.174904in}{2.891817in}}%
\pgfpathlineto{\pgfqpoint{3.174904in}{2.894767in}}%
\pgfpathlineto{\pgfqpoint{3.179445in}{2.894767in}}%
\pgfpathlineto{\pgfqpoint{3.179445in}{2.891817in}}%
\pgfpathmoveto{\pgfqpoint{3.179445in}{2.891817in}}%
\pgfpathlineto{\pgfqpoint{3.179445in}{2.891817in}}%
\pgfpathlineto{\pgfqpoint{3.179445in}{2.894767in}}%
\pgfpathlineto{\pgfqpoint{3.183986in}{2.894767in}}%
\pgfpathlineto{\pgfqpoint{3.183986in}{2.891817in}}%
\pgfpathmoveto{\pgfqpoint{3.165822in}{2.894767in}}%
\pgfpathlineto{\pgfqpoint{3.165822in}{2.894767in}}%
\pgfpathlineto{\pgfqpoint{3.165822in}{2.897716in}}%
\pgfpathlineto{\pgfqpoint{3.170363in}{2.897716in}}%
\pgfpathlineto{\pgfqpoint{3.170363in}{2.894767in}}%
\pgfpathmoveto{\pgfqpoint{3.165822in}{2.897716in}}%
\pgfpathlineto{\pgfqpoint{3.165822in}{2.897716in}}%
\pgfpathlineto{\pgfqpoint{3.165822in}{2.900665in}}%
\pgfpathlineto{\pgfqpoint{3.170363in}{2.900665in}}%
\pgfpathlineto{\pgfqpoint{3.170363in}{2.897716in}}%
\pgfpathmoveto{\pgfqpoint{3.170363in}{2.894767in}}%
\pgfpathlineto{\pgfqpoint{3.170363in}{2.894767in}}%
\pgfpathlineto{\pgfqpoint{3.170363in}{2.897716in}}%
\pgfpathlineto{\pgfqpoint{3.174904in}{2.897716in}}%
\pgfpathlineto{\pgfqpoint{3.174904in}{2.894767in}}%
\pgfpathmoveto{\pgfqpoint{3.170363in}{2.897716in}}%
\pgfpathlineto{\pgfqpoint{3.170363in}{2.897716in}}%
\pgfpathlineto{\pgfqpoint{3.170363in}{2.900665in}}%
\pgfpathlineto{\pgfqpoint{3.174904in}{2.900665in}}%
\pgfpathlineto{\pgfqpoint{3.174904in}{2.897716in}}%
\pgfpathmoveto{\pgfqpoint{3.183986in}{2.888868in}}%
\pgfpathlineto{\pgfqpoint{3.183986in}{2.888868in}}%
\pgfpathlineto{\pgfqpoint{3.183986in}{2.891817in}}%
\pgfpathlineto{\pgfqpoint{3.188527in}{2.891817in}}%
\pgfpathlineto{\pgfqpoint{3.188527in}{2.888868in}}%
\pgfpathmoveto{\pgfqpoint{3.183986in}{2.891817in}}%
\pgfpathlineto{\pgfqpoint{3.183986in}{2.891817in}}%
\pgfpathlineto{\pgfqpoint{3.183986in}{2.894767in}}%
\pgfpathlineto{\pgfqpoint{3.188527in}{2.894767in}}%
\pgfpathlineto{\pgfqpoint{3.188527in}{2.891817in}}%
\pgfpathmoveto{\pgfqpoint{3.188527in}{2.888868in}}%
\pgfpathlineto{\pgfqpoint{3.188527in}{2.888868in}}%
\pgfpathlineto{\pgfqpoint{3.188527in}{2.891817in}}%
\pgfpathlineto{\pgfqpoint{3.193068in}{2.891817in}}%
\pgfpathlineto{\pgfqpoint{3.193068in}{2.888868in}}%
\pgfpathmoveto{\pgfqpoint{3.188527in}{2.891817in}}%
\pgfpathlineto{\pgfqpoint{3.188527in}{2.891817in}}%
\pgfpathlineto{\pgfqpoint{3.188527in}{2.894767in}}%
\pgfpathlineto{\pgfqpoint{3.193068in}{2.894767in}}%
\pgfpathlineto{\pgfqpoint{3.193068in}{2.891817in}}%
\pgfpathmoveto{\pgfqpoint{3.197609in}{2.885919in}}%
\pgfpathlineto{\pgfqpoint{3.197609in}{2.885919in}}%
\pgfpathlineto{\pgfqpoint{3.197609in}{2.888868in}}%
\pgfpathlineto{\pgfqpoint{3.202150in}{2.888868in}}%
\pgfpathlineto{\pgfqpoint{3.202150in}{2.885919in}}%
\pgfpathmoveto{\pgfqpoint{3.193068in}{2.888868in}}%
\pgfpathlineto{\pgfqpoint{3.193068in}{2.888868in}}%
\pgfpathlineto{\pgfqpoint{3.193068in}{2.891817in}}%
\pgfpathlineto{\pgfqpoint{3.197609in}{2.891817in}}%
\pgfpathlineto{\pgfqpoint{3.197609in}{2.888868in}}%
\pgfpathmoveto{\pgfqpoint{3.193068in}{2.891817in}}%
\pgfpathlineto{\pgfqpoint{3.193068in}{2.891817in}}%
\pgfpathlineto{\pgfqpoint{3.193068in}{2.894767in}}%
\pgfpathlineto{\pgfqpoint{3.197609in}{2.894767in}}%
\pgfpathlineto{\pgfqpoint{3.197609in}{2.891817in}}%
\pgfpathmoveto{\pgfqpoint{3.197609in}{2.888868in}}%
\pgfpathlineto{\pgfqpoint{3.197609in}{2.888868in}}%
\pgfpathlineto{\pgfqpoint{3.197609in}{2.891817in}}%
\pgfpathlineto{\pgfqpoint{3.202150in}{2.891817in}}%
\pgfpathlineto{\pgfqpoint{3.202150in}{2.888868in}}%
\pgfpathmoveto{\pgfqpoint{3.197609in}{2.891817in}}%
\pgfpathlineto{\pgfqpoint{3.197609in}{2.891817in}}%
\pgfpathlineto{\pgfqpoint{3.197609in}{2.894767in}}%
\pgfpathlineto{\pgfqpoint{3.202150in}{2.894767in}}%
\pgfpathlineto{\pgfqpoint{3.202150in}{2.891817in}}%
\pgfpathmoveto{\pgfqpoint{3.202150in}{2.885919in}}%
\pgfpathlineto{\pgfqpoint{3.202150in}{2.885919in}}%
\pgfpathlineto{\pgfqpoint{3.202150in}{2.888868in}}%
\pgfpathlineto{\pgfqpoint{3.206691in}{2.888868in}}%
\pgfpathlineto{\pgfqpoint{3.206691in}{2.885919in}}%
\pgfpathmoveto{\pgfqpoint{3.206691in}{2.885919in}}%
\pgfpathlineto{\pgfqpoint{3.206691in}{2.885919in}}%
\pgfpathlineto{\pgfqpoint{3.206691in}{2.888868in}}%
\pgfpathlineto{\pgfqpoint{3.211232in}{2.888868in}}%
\pgfpathlineto{\pgfqpoint{3.211232in}{2.885919in}}%
\pgfpathmoveto{\pgfqpoint{3.211232in}{2.882970in}}%
\pgfpathlineto{\pgfqpoint{3.211232in}{2.882970in}}%
\pgfpathlineto{\pgfqpoint{3.211232in}{2.885919in}}%
\pgfpathlineto{\pgfqpoint{3.215773in}{2.885919in}}%
\pgfpathlineto{\pgfqpoint{3.215773in}{2.882970in}}%
\pgfpathmoveto{\pgfqpoint{3.211232in}{2.885919in}}%
\pgfpathlineto{\pgfqpoint{3.211232in}{2.885919in}}%
\pgfpathlineto{\pgfqpoint{3.211232in}{2.888868in}}%
\pgfpathlineto{\pgfqpoint{3.215773in}{2.888868in}}%
\pgfpathlineto{\pgfqpoint{3.215773in}{2.885919in}}%
\pgfpathmoveto{\pgfqpoint{3.215773in}{2.882970in}}%
\pgfpathlineto{\pgfqpoint{3.215773in}{2.882970in}}%
\pgfpathlineto{\pgfqpoint{3.215773in}{2.885919in}}%
\pgfpathlineto{\pgfqpoint{3.220314in}{2.885919in}}%
\pgfpathlineto{\pgfqpoint{3.220314in}{2.882970in}}%
\pgfpathmoveto{\pgfqpoint{3.215773in}{2.885919in}}%
\pgfpathlineto{\pgfqpoint{3.215773in}{2.885919in}}%
\pgfpathlineto{\pgfqpoint{3.215773in}{2.888868in}}%
\pgfpathlineto{\pgfqpoint{3.220314in}{2.888868in}}%
\pgfpathlineto{\pgfqpoint{3.220314in}{2.885919in}}%
\pgfpathmoveto{\pgfqpoint{3.211232in}{3.419726in}}%
\pgfpathlineto{\pgfqpoint{3.211232in}{3.419726in}}%
\pgfpathlineto{\pgfqpoint{3.211232in}{3.422676in}}%
\pgfpathlineto{\pgfqpoint{3.215773in}{3.422676in}}%
\pgfpathlineto{\pgfqpoint{3.215773in}{3.419726in}}%
\pgfpathmoveto{\pgfqpoint{3.211232in}{3.422676in}}%
\pgfpathlineto{\pgfqpoint{3.211232in}{3.422676in}}%
\pgfpathlineto{\pgfqpoint{3.211232in}{3.425625in}}%
\pgfpathlineto{\pgfqpoint{3.215773in}{3.425625in}}%
\pgfpathlineto{\pgfqpoint{3.215773in}{3.422676in}}%
\pgfpathmoveto{\pgfqpoint{3.215773in}{3.419726in}}%
\pgfpathlineto{\pgfqpoint{3.215773in}{3.419726in}}%
\pgfpathlineto{\pgfqpoint{3.215773in}{3.422676in}}%
\pgfpathlineto{\pgfqpoint{3.220314in}{3.422676in}}%
\pgfpathlineto{\pgfqpoint{3.220314in}{3.419726in}}%
\pgfpathmoveto{\pgfqpoint{3.215773in}{3.422676in}}%
\pgfpathlineto{\pgfqpoint{3.215773in}{3.422676in}}%
\pgfpathlineto{\pgfqpoint{3.215773in}{3.425625in}}%
\pgfpathlineto{\pgfqpoint{3.220314in}{3.425625in}}%
\pgfpathlineto{\pgfqpoint{3.220314in}{3.422676in}}%
\pgfpathmoveto{\pgfqpoint{3.138575in}{3.466915in}}%
\pgfpathlineto{\pgfqpoint{3.138575in}{3.466915in}}%
\pgfpathlineto{\pgfqpoint{3.138575in}{3.469864in}}%
\pgfpathlineto{\pgfqpoint{3.143116in}{3.469864in}}%
\pgfpathlineto{\pgfqpoint{3.143116in}{3.466915in}}%
\pgfpathmoveto{\pgfqpoint{3.138575in}{3.469864in}}%
\pgfpathlineto{\pgfqpoint{3.138575in}{3.469864in}}%
\pgfpathlineto{\pgfqpoint{3.138575in}{3.472813in}}%
\pgfpathlineto{\pgfqpoint{3.143116in}{3.472813in}}%
\pgfpathlineto{\pgfqpoint{3.143116in}{3.469864in}}%
\pgfpathmoveto{\pgfqpoint{3.143116in}{3.466915in}}%
\pgfpathlineto{\pgfqpoint{3.143116in}{3.466915in}}%
\pgfpathlineto{\pgfqpoint{3.143116in}{3.469864in}}%
\pgfpathlineto{\pgfqpoint{3.147657in}{3.469864in}}%
\pgfpathlineto{\pgfqpoint{3.147657in}{3.466915in}}%
\pgfpathmoveto{\pgfqpoint{3.143116in}{3.469864in}}%
\pgfpathlineto{\pgfqpoint{3.143116in}{3.469864in}}%
\pgfpathlineto{\pgfqpoint{3.143116in}{3.472813in}}%
\pgfpathlineto{\pgfqpoint{3.147657in}{3.472813in}}%
\pgfpathlineto{\pgfqpoint{3.147657in}{3.469864in}}%
\pgfpathmoveto{\pgfqpoint{3.102247in}{3.490509in}}%
\pgfpathlineto{\pgfqpoint{3.102247in}{3.490509in}}%
\pgfpathlineto{\pgfqpoint{3.102247in}{3.493458in}}%
\pgfpathlineto{\pgfqpoint{3.106788in}{3.493458in}}%
\pgfpathlineto{\pgfqpoint{3.106788in}{3.490509in}}%
\pgfpathmoveto{\pgfqpoint{3.102247in}{3.493458in}}%
\pgfpathlineto{\pgfqpoint{3.102247in}{3.493458in}}%
\pgfpathlineto{\pgfqpoint{3.102247in}{3.496408in}}%
\pgfpathlineto{\pgfqpoint{3.106788in}{3.496408in}}%
\pgfpathlineto{\pgfqpoint{3.106788in}{3.493458in}}%
\pgfpathmoveto{\pgfqpoint{3.106788in}{3.490509in}}%
\pgfpathlineto{\pgfqpoint{3.106788in}{3.490509in}}%
\pgfpathlineto{\pgfqpoint{3.106788in}{3.493458in}}%
\pgfpathlineto{\pgfqpoint{3.111329in}{3.493458in}}%
\pgfpathlineto{\pgfqpoint{3.111329in}{3.490509in}}%
\pgfpathmoveto{\pgfqpoint{3.106788in}{3.493458in}}%
\pgfpathlineto{\pgfqpoint{3.106788in}{3.493458in}}%
\pgfpathlineto{\pgfqpoint{3.106788in}{3.496408in}}%
\pgfpathlineto{\pgfqpoint{3.111329in}{3.496408in}}%
\pgfpathlineto{\pgfqpoint{3.111329in}{3.493458in}}%
\pgfpathmoveto{\pgfqpoint{3.084083in}{3.502306in}}%
\pgfpathlineto{\pgfqpoint{3.084083in}{3.502306in}}%
\pgfpathlineto{\pgfqpoint{3.084083in}{3.505255in}}%
\pgfpathlineto{\pgfqpoint{3.088624in}{3.505255in}}%
\pgfpathlineto{\pgfqpoint{3.088624in}{3.502306in}}%
\pgfpathmoveto{\pgfqpoint{3.084083in}{3.505255in}}%
\pgfpathlineto{\pgfqpoint{3.084083in}{3.505255in}}%
\pgfpathlineto{\pgfqpoint{3.084083in}{3.508205in}}%
\pgfpathlineto{\pgfqpoint{3.088624in}{3.508205in}}%
\pgfpathlineto{\pgfqpoint{3.088624in}{3.505255in}}%
\pgfpathmoveto{\pgfqpoint{3.088624in}{3.502306in}}%
\pgfpathlineto{\pgfqpoint{3.088624in}{3.502306in}}%
\pgfpathlineto{\pgfqpoint{3.088624in}{3.505255in}}%
\pgfpathlineto{\pgfqpoint{3.093165in}{3.505255in}}%
\pgfpathlineto{\pgfqpoint{3.093165in}{3.502306in}}%
\pgfpathmoveto{\pgfqpoint{3.088624in}{3.505255in}}%
\pgfpathlineto{\pgfqpoint{3.088624in}{3.505255in}}%
\pgfpathlineto{\pgfqpoint{3.088624in}{3.508205in}}%
\pgfpathlineto{\pgfqpoint{3.093165in}{3.508205in}}%
\pgfpathlineto{\pgfqpoint{3.093165in}{3.505255in}}%
\pgfpathmoveto{\pgfqpoint{3.075001in}{3.508205in}}%
\pgfpathlineto{\pgfqpoint{3.075001in}{3.508205in}}%
\pgfpathlineto{\pgfqpoint{3.075001in}{3.511154in}}%
\pgfpathlineto{\pgfqpoint{3.079542in}{3.511154in}}%
\pgfpathlineto{\pgfqpoint{3.079542in}{3.508205in}}%
\pgfpathmoveto{\pgfqpoint{3.075001in}{3.511154in}}%
\pgfpathlineto{\pgfqpoint{3.075001in}{3.511154in}}%
\pgfpathlineto{\pgfqpoint{3.075001in}{3.514103in}}%
\pgfpathlineto{\pgfqpoint{3.079542in}{3.514103in}}%
\pgfpathlineto{\pgfqpoint{3.079542in}{3.511154in}}%
\pgfpathmoveto{\pgfqpoint{3.079542in}{3.508205in}}%
\pgfpathlineto{\pgfqpoint{3.079542in}{3.508205in}}%
\pgfpathlineto{\pgfqpoint{3.079542in}{3.511154in}}%
\pgfpathlineto{\pgfqpoint{3.084083in}{3.511154in}}%
\pgfpathlineto{\pgfqpoint{3.084083in}{3.508205in}}%
\pgfpathmoveto{\pgfqpoint{3.079542in}{3.511154in}}%
\pgfpathlineto{\pgfqpoint{3.079542in}{3.511154in}}%
\pgfpathlineto{\pgfqpoint{3.079542in}{3.514103in}}%
\pgfpathlineto{\pgfqpoint{3.084083in}{3.514103in}}%
\pgfpathlineto{\pgfqpoint{3.084083in}{3.511154in}}%
\pgfpathmoveto{\pgfqpoint{3.075001in}{3.514103in}}%
\pgfpathlineto{\pgfqpoint{3.075001in}{3.514103in}}%
\pgfpathlineto{\pgfqpoint{3.075001in}{3.517052in}}%
\pgfpathlineto{\pgfqpoint{3.079542in}{3.517052in}}%
\pgfpathlineto{\pgfqpoint{3.079542in}{3.514103in}}%
\pgfpathmoveto{\pgfqpoint{3.075001in}{3.517052in}}%
\pgfpathlineto{\pgfqpoint{3.075001in}{3.517052in}}%
\pgfpathlineto{\pgfqpoint{3.075001in}{3.520002in}}%
\pgfpathlineto{\pgfqpoint{3.079542in}{3.520002in}}%
\pgfpathlineto{\pgfqpoint{3.079542in}{3.517052in}}%
\pgfpathmoveto{\pgfqpoint{3.079542in}{3.514103in}}%
\pgfpathlineto{\pgfqpoint{3.079542in}{3.514103in}}%
\pgfpathlineto{\pgfqpoint{3.079542in}{3.517052in}}%
\pgfpathlineto{\pgfqpoint{3.084083in}{3.517052in}}%
\pgfpathlineto{\pgfqpoint{3.084083in}{3.514103in}}%
\pgfpathmoveto{\pgfqpoint{3.084083in}{3.508205in}}%
\pgfpathlineto{\pgfqpoint{3.084083in}{3.508205in}}%
\pgfpathlineto{\pgfqpoint{3.084083in}{3.511154in}}%
\pgfpathlineto{\pgfqpoint{3.088624in}{3.511154in}}%
\pgfpathlineto{\pgfqpoint{3.088624in}{3.508205in}}%
\pgfpathmoveto{\pgfqpoint{3.084083in}{3.511154in}}%
\pgfpathlineto{\pgfqpoint{3.084083in}{3.511154in}}%
\pgfpathlineto{\pgfqpoint{3.084083in}{3.514103in}}%
\pgfpathlineto{\pgfqpoint{3.088624in}{3.514103in}}%
\pgfpathlineto{\pgfqpoint{3.088624in}{3.511154in}}%
\pgfpathmoveto{\pgfqpoint{3.088624in}{3.508205in}}%
\pgfpathlineto{\pgfqpoint{3.088624in}{3.508205in}}%
\pgfpathlineto{\pgfqpoint{3.088624in}{3.511154in}}%
\pgfpathlineto{\pgfqpoint{3.093165in}{3.511154in}}%
\pgfpathlineto{\pgfqpoint{3.093165in}{3.508205in}}%
\pgfpathmoveto{\pgfqpoint{3.093165in}{3.496408in}}%
\pgfpathlineto{\pgfqpoint{3.093165in}{3.496408in}}%
\pgfpathlineto{\pgfqpoint{3.093165in}{3.499357in}}%
\pgfpathlineto{\pgfqpoint{3.097706in}{3.499357in}}%
\pgfpathlineto{\pgfqpoint{3.097706in}{3.496408in}}%
\pgfpathmoveto{\pgfqpoint{3.093165in}{3.499357in}}%
\pgfpathlineto{\pgfqpoint{3.093165in}{3.499357in}}%
\pgfpathlineto{\pgfqpoint{3.093165in}{3.502306in}}%
\pgfpathlineto{\pgfqpoint{3.097706in}{3.502306in}}%
\pgfpathlineto{\pgfqpoint{3.097706in}{3.499357in}}%
\pgfpathmoveto{\pgfqpoint{3.097706in}{3.496408in}}%
\pgfpathlineto{\pgfqpoint{3.097706in}{3.496408in}}%
\pgfpathlineto{\pgfqpoint{3.097706in}{3.499357in}}%
\pgfpathlineto{\pgfqpoint{3.102247in}{3.499357in}}%
\pgfpathlineto{\pgfqpoint{3.102247in}{3.496408in}}%
\pgfpathmoveto{\pgfqpoint{3.097706in}{3.499357in}}%
\pgfpathlineto{\pgfqpoint{3.097706in}{3.499357in}}%
\pgfpathlineto{\pgfqpoint{3.097706in}{3.502306in}}%
\pgfpathlineto{\pgfqpoint{3.102247in}{3.502306in}}%
\pgfpathlineto{\pgfqpoint{3.102247in}{3.499357in}}%
\pgfpathmoveto{\pgfqpoint{3.093165in}{3.502306in}}%
\pgfpathlineto{\pgfqpoint{3.093165in}{3.502306in}}%
\pgfpathlineto{\pgfqpoint{3.093165in}{3.505255in}}%
\pgfpathlineto{\pgfqpoint{3.097706in}{3.505255in}}%
\pgfpathlineto{\pgfqpoint{3.097706in}{3.502306in}}%
\pgfpathmoveto{\pgfqpoint{3.093165in}{3.505255in}}%
\pgfpathlineto{\pgfqpoint{3.093165in}{3.505255in}}%
\pgfpathlineto{\pgfqpoint{3.093165in}{3.508205in}}%
\pgfpathlineto{\pgfqpoint{3.097706in}{3.508205in}}%
\pgfpathlineto{\pgfqpoint{3.097706in}{3.505255in}}%
\pgfpathmoveto{\pgfqpoint{3.097706in}{3.502306in}}%
\pgfpathlineto{\pgfqpoint{3.097706in}{3.502306in}}%
\pgfpathlineto{\pgfqpoint{3.097706in}{3.505255in}}%
\pgfpathlineto{\pgfqpoint{3.102247in}{3.505255in}}%
\pgfpathlineto{\pgfqpoint{3.102247in}{3.502306in}}%
\pgfpathmoveto{\pgfqpoint{3.102247in}{3.496408in}}%
\pgfpathlineto{\pgfqpoint{3.102247in}{3.496408in}}%
\pgfpathlineto{\pgfqpoint{3.102247in}{3.499357in}}%
\pgfpathlineto{\pgfqpoint{3.106788in}{3.499357in}}%
\pgfpathlineto{\pgfqpoint{3.106788in}{3.496408in}}%
\pgfpathmoveto{\pgfqpoint{3.102247in}{3.499357in}}%
\pgfpathlineto{\pgfqpoint{3.102247in}{3.499357in}}%
\pgfpathlineto{\pgfqpoint{3.102247in}{3.502306in}}%
\pgfpathlineto{\pgfqpoint{3.106788in}{3.502306in}}%
\pgfpathlineto{\pgfqpoint{3.106788in}{3.499357in}}%
\pgfpathmoveto{\pgfqpoint{3.106788in}{3.496408in}}%
\pgfpathlineto{\pgfqpoint{3.106788in}{3.496408in}}%
\pgfpathlineto{\pgfqpoint{3.106788in}{3.499357in}}%
\pgfpathlineto{\pgfqpoint{3.111329in}{3.499357in}}%
\pgfpathlineto{\pgfqpoint{3.111329in}{3.496408in}}%
\pgfpathmoveto{\pgfqpoint{3.120411in}{3.478712in}}%
\pgfpathlineto{\pgfqpoint{3.120411in}{3.478712in}}%
\pgfpathlineto{\pgfqpoint{3.120411in}{3.481661in}}%
\pgfpathlineto{\pgfqpoint{3.124952in}{3.481661in}}%
\pgfpathlineto{\pgfqpoint{3.124952in}{3.478712in}}%
\pgfpathmoveto{\pgfqpoint{3.120411in}{3.481661in}}%
\pgfpathlineto{\pgfqpoint{3.120411in}{3.481661in}}%
\pgfpathlineto{\pgfqpoint{3.120411in}{3.484610in}}%
\pgfpathlineto{\pgfqpoint{3.124952in}{3.484610in}}%
\pgfpathlineto{\pgfqpoint{3.124952in}{3.481661in}}%
\pgfpathmoveto{\pgfqpoint{3.124952in}{3.478712in}}%
\pgfpathlineto{\pgfqpoint{3.124952in}{3.478712in}}%
\pgfpathlineto{\pgfqpoint{3.124952in}{3.481661in}}%
\pgfpathlineto{\pgfqpoint{3.129493in}{3.481661in}}%
\pgfpathlineto{\pgfqpoint{3.129493in}{3.478712in}}%
\pgfpathmoveto{\pgfqpoint{3.124952in}{3.481661in}}%
\pgfpathlineto{\pgfqpoint{3.124952in}{3.481661in}}%
\pgfpathlineto{\pgfqpoint{3.124952in}{3.484610in}}%
\pgfpathlineto{\pgfqpoint{3.129493in}{3.484610in}}%
\pgfpathlineto{\pgfqpoint{3.129493in}{3.481661in}}%
\pgfpathmoveto{\pgfqpoint{3.111329in}{3.484610in}}%
\pgfpathlineto{\pgfqpoint{3.111329in}{3.484610in}}%
\pgfpathlineto{\pgfqpoint{3.111329in}{3.487560in}}%
\pgfpathlineto{\pgfqpoint{3.115870in}{3.487560in}}%
\pgfpathlineto{\pgfqpoint{3.115870in}{3.484610in}}%
\pgfpathmoveto{\pgfqpoint{3.111329in}{3.487560in}}%
\pgfpathlineto{\pgfqpoint{3.111329in}{3.487560in}}%
\pgfpathlineto{\pgfqpoint{3.111329in}{3.490509in}}%
\pgfpathlineto{\pgfqpoint{3.115870in}{3.490509in}}%
\pgfpathlineto{\pgfqpoint{3.115870in}{3.487560in}}%
\pgfpathmoveto{\pgfqpoint{3.115870in}{3.484610in}}%
\pgfpathlineto{\pgfqpoint{3.115870in}{3.484610in}}%
\pgfpathlineto{\pgfqpoint{3.115870in}{3.487560in}}%
\pgfpathlineto{\pgfqpoint{3.120411in}{3.487560in}}%
\pgfpathlineto{\pgfqpoint{3.120411in}{3.484610in}}%
\pgfpathmoveto{\pgfqpoint{3.115870in}{3.487560in}}%
\pgfpathlineto{\pgfqpoint{3.115870in}{3.487560in}}%
\pgfpathlineto{\pgfqpoint{3.115870in}{3.490509in}}%
\pgfpathlineto{\pgfqpoint{3.120411in}{3.490509in}}%
\pgfpathlineto{\pgfqpoint{3.120411in}{3.487560in}}%
\pgfpathmoveto{\pgfqpoint{3.111329in}{3.490509in}}%
\pgfpathlineto{\pgfqpoint{3.111329in}{3.490509in}}%
\pgfpathlineto{\pgfqpoint{3.111329in}{3.493458in}}%
\pgfpathlineto{\pgfqpoint{3.115870in}{3.493458in}}%
\pgfpathlineto{\pgfqpoint{3.115870in}{3.490509in}}%
\pgfpathmoveto{\pgfqpoint{3.111329in}{3.493458in}}%
\pgfpathlineto{\pgfqpoint{3.111329in}{3.493458in}}%
\pgfpathlineto{\pgfqpoint{3.111329in}{3.496408in}}%
\pgfpathlineto{\pgfqpoint{3.115870in}{3.496408in}}%
\pgfpathlineto{\pgfqpoint{3.115870in}{3.493458in}}%
\pgfpathmoveto{\pgfqpoint{3.115870in}{3.490509in}}%
\pgfpathlineto{\pgfqpoint{3.115870in}{3.490509in}}%
\pgfpathlineto{\pgfqpoint{3.115870in}{3.493458in}}%
\pgfpathlineto{\pgfqpoint{3.120411in}{3.493458in}}%
\pgfpathlineto{\pgfqpoint{3.120411in}{3.490509in}}%
\pgfpathmoveto{\pgfqpoint{3.120411in}{3.484610in}}%
\pgfpathlineto{\pgfqpoint{3.120411in}{3.484610in}}%
\pgfpathlineto{\pgfqpoint{3.120411in}{3.487560in}}%
\pgfpathlineto{\pgfqpoint{3.124952in}{3.487560in}}%
\pgfpathlineto{\pgfqpoint{3.124952in}{3.484610in}}%
\pgfpathmoveto{\pgfqpoint{3.120411in}{3.487560in}}%
\pgfpathlineto{\pgfqpoint{3.120411in}{3.487560in}}%
\pgfpathlineto{\pgfqpoint{3.120411in}{3.490509in}}%
\pgfpathlineto{\pgfqpoint{3.124952in}{3.490509in}}%
\pgfpathlineto{\pgfqpoint{3.124952in}{3.487560in}}%
\pgfpathmoveto{\pgfqpoint{3.124952in}{3.484610in}}%
\pgfpathlineto{\pgfqpoint{3.124952in}{3.484610in}}%
\pgfpathlineto{\pgfqpoint{3.124952in}{3.487560in}}%
\pgfpathlineto{\pgfqpoint{3.129493in}{3.487560in}}%
\pgfpathlineto{\pgfqpoint{3.129493in}{3.484610in}}%
\pgfpathmoveto{\pgfqpoint{3.129493in}{3.472813in}}%
\pgfpathlineto{\pgfqpoint{3.129493in}{3.472813in}}%
\pgfpathlineto{\pgfqpoint{3.129493in}{3.475763in}}%
\pgfpathlineto{\pgfqpoint{3.134034in}{3.475763in}}%
\pgfpathlineto{\pgfqpoint{3.134034in}{3.472813in}}%
\pgfpathmoveto{\pgfqpoint{3.129493in}{3.475763in}}%
\pgfpathlineto{\pgfqpoint{3.129493in}{3.475763in}}%
\pgfpathlineto{\pgfqpoint{3.129493in}{3.478712in}}%
\pgfpathlineto{\pgfqpoint{3.134034in}{3.478712in}}%
\pgfpathlineto{\pgfqpoint{3.134034in}{3.475763in}}%
\pgfpathmoveto{\pgfqpoint{3.134034in}{3.472813in}}%
\pgfpathlineto{\pgfqpoint{3.134034in}{3.472813in}}%
\pgfpathlineto{\pgfqpoint{3.134034in}{3.475763in}}%
\pgfpathlineto{\pgfqpoint{3.138575in}{3.475763in}}%
\pgfpathlineto{\pgfqpoint{3.138575in}{3.472813in}}%
\pgfpathmoveto{\pgfqpoint{3.134034in}{3.475763in}}%
\pgfpathlineto{\pgfqpoint{3.134034in}{3.475763in}}%
\pgfpathlineto{\pgfqpoint{3.134034in}{3.478712in}}%
\pgfpathlineto{\pgfqpoint{3.138575in}{3.478712in}}%
\pgfpathlineto{\pgfqpoint{3.138575in}{3.475763in}}%
\pgfpathmoveto{\pgfqpoint{3.129493in}{3.478712in}}%
\pgfpathlineto{\pgfqpoint{3.129493in}{3.478712in}}%
\pgfpathlineto{\pgfqpoint{3.129493in}{3.481661in}}%
\pgfpathlineto{\pgfqpoint{3.134034in}{3.481661in}}%
\pgfpathlineto{\pgfqpoint{3.134034in}{3.478712in}}%
\pgfpathmoveto{\pgfqpoint{3.129493in}{3.481661in}}%
\pgfpathlineto{\pgfqpoint{3.129493in}{3.481661in}}%
\pgfpathlineto{\pgfqpoint{3.129493in}{3.484610in}}%
\pgfpathlineto{\pgfqpoint{3.134034in}{3.484610in}}%
\pgfpathlineto{\pgfqpoint{3.134034in}{3.481661in}}%
\pgfpathmoveto{\pgfqpoint{3.134034in}{3.478712in}}%
\pgfpathlineto{\pgfqpoint{3.134034in}{3.478712in}}%
\pgfpathlineto{\pgfqpoint{3.134034in}{3.481661in}}%
\pgfpathlineto{\pgfqpoint{3.138575in}{3.481661in}}%
\pgfpathlineto{\pgfqpoint{3.138575in}{3.478712in}}%
\pgfpathmoveto{\pgfqpoint{3.138575in}{3.472813in}}%
\pgfpathlineto{\pgfqpoint{3.138575in}{3.472813in}}%
\pgfpathlineto{\pgfqpoint{3.138575in}{3.475763in}}%
\pgfpathlineto{\pgfqpoint{3.143116in}{3.475763in}}%
\pgfpathlineto{\pgfqpoint{3.143116in}{3.472813in}}%
\pgfpathmoveto{\pgfqpoint{3.138575in}{3.475763in}}%
\pgfpathlineto{\pgfqpoint{3.138575in}{3.475763in}}%
\pgfpathlineto{\pgfqpoint{3.138575in}{3.478712in}}%
\pgfpathlineto{\pgfqpoint{3.143116in}{3.478712in}}%
\pgfpathlineto{\pgfqpoint{3.143116in}{3.475763in}}%
\pgfpathmoveto{\pgfqpoint{3.143116in}{3.472813in}}%
\pgfpathlineto{\pgfqpoint{3.143116in}{3.472813in}}%
\pgfpathlineto{\pgfqpoint{3.143116in}{3.475763in}}%
\pgfpathlineto{\pgfqpoint{3.147657in}{3.475763in}}%
\pgfpathlineto{\pgfqpoint{3.147657in}{3.472813in}}%
\pgfpathmoveto{\pgfqpoint{3.174904in}{3.443321in}}%
\pgfpathlineto{\pgfqpoint{3.174904in}{3.443321in}}%
\pgfpathlineto{\pgfqpoint{3.174904in}{3.446270in}}%
\pgfpathlineto{\pgfqpoint{3.179445in}{3.446270in}}%
\pgfpathlineto{\pgfqpoint{3.179445in}{3.443321in}}%
\pgfpathmoveto{\pgfqpoint{3.174904in}{3.446270in}}%
\pgfpathlineto{\pgfqpoint{3.174904in}{3.446270in}}%
\pgfpathlineto{\pgfqpoint{3.174904in}{3.449219in}}%
\pgfpathlineto{\pgfqpoint{3.179445in}{3.449219in}}%
\pgfpathlineto{\pgfqpoint{3.179445in}{3.446270in}}%
\pgfpathmoveto{\pgfqpoint{3.179445in}{3.443321in}}%
\pgfpathlineto{\pgfqpoint{3.179445in}{3.443321in}}%
\pgfpathlineto{\pgfqpoint{3.179445in}{3.446270in}}%
\pgfpathlineto{\pgfqpoint{3.183986in}{3.446270in}}%
\pgfpathlineto{\pgfqpoint{3.183986in}{3.443321in}}%
\pgfpathmoveto{\pgfqpoint{3.179445in}{3.446270in}}%
\pgfpathlineto{\pgfqpoint{3.179445in}{3.446270in}}%
\pgfpathlineto{\pgfqpoint{3.179445in}{3.449219in}}%
\pgfpathlineto{\pgfqpoint{3.183986in}{3.449219in}}%
\pgfpathlineto{\pgfqpoint{3.183986in}{3.446270in}}%
\pgfpathmoveto{\pgfqpoint{3.156740in}{3.455118in}}%
\pgfpathlineto{\pgfqpoint{3.156740in}{3.455118in}}%
\pgfpathlineto{\pgfqpoint{3.156740in}{3.458067in}}%
\pgfpathlineto{\pgfqpoint{3.161281in}{3.458067in}}%
\pgfpathlineto{\pgfqpoint{3.161281in}{3.455118in}}%
\pgfpathmoveto{\pgfqpoint{3.156740in}{3.458067in}}%
\pgfpathlineto{\pgfqpoint{3.156740in}{3.458067in}}%
\pgfpathlineto{\pgfqpoint{3.156740in}{3.461016in}}%
\pgfpathlineto{\pgfqpoint{3.161281in}{3.461016in}}%
\pgfpathlineto{\pgfqpoint{3.161281in}{3.458067in}}%
\pgfpathmoveto{\pgfqpoint{3.161281in}{3.455118in}}%
\pgfpathlineto{\pgfqpoint{3.161281in}{3.455118in}}%
\pgfpathlineto{\pgfqpoint{3.161281in}{3.458067in}}%
\pgfpathlineto{\pgfqpoint{3.165822in}{3.458067in}}%
\pgfpathlineto{\pgfqpoint{3.165822in}{3.455118in}}%
\pgfpathmoveto{\pgfqpoint{3.161281in}{3.458067in}}%
\pgfpathlineto{\pgfqpoint{3.161281in}{3.458067in}}%
\pgfpathlineto{\pgfqpoint{3.161281in}{3.461016in}}%
\pgfpathlineto{\pgfqpoint{3.165822in}{3.461016in}}%
\pgfpathlineto{\pgfqpoint{3.165822in}{3.458067in}}%
\pgfpathmoveto{\pgfqpoint{3.147657in}{3.461016in}}%
\pgfpathlineto{\pgfqpoint{3.147657in}{3.461016in}}%
\pgfpathlineto{\pgfqpoint{3.147657in}{3.463966in}}%
\pgfpathlineto{\pgfqpoint{3.152199in}{3.463966in}}%
\pgfpathlineto{\pgfqpoint{3.152199in}{3.461016in}}%
\pgfpathmoveto{\pgfqpoint{3.147657in}{3.463966in}}%
\pgfpathlineto{\pgfqpoint{3.147657in}{3.463966in}}%
\pgfpathlineto{\pgfqpoint{3.147657in}{3.466915in}}%
\pgfpathlineto{\pgfqpoint{3.152199in}{3.466915in}}%
\pgfpathlineto{\pgfqpoint{3.152199in}{3.463966in}}%
\pgfpathmoveto{\pgfqpoint{3.152199in}{3.461016in}}%
\pgfpathlineto{\pgfqpoint{3.152199in}{3.461016in}}%
\pgfpathlineto{\pgfqpoint{3.152199in}{3.463966in}}%
\pgfpathlineto{\pgfqpoint{3.156740in}{3.463966in}}%
\pgfpathlineto{\pgfqpoint{3.156740in}{3.461016in}}%
\pgfpathmoveto{\pgfqpoint{3.152199in}{3.463966in}}%
\pgfpathlineto{\pgfqpoint{3.152199in}{3.463966in}}%
\pgfpathlineto{\pgfqpoint{3.152199in}{3.466915in}}%
\pgfpathlineto{\pgfqpoint{3.156740in}{3.466915in}}%
\pgfpathlineto{\pgfqpoint{3.156740in}{3.463966in}}%
\pgfpathmoveto{\pgfqpoint{3.147657in}{3.466915in}}%
\pgfpathlineto{\pgfqpoint{3.147657in}{3.466915in}}%
\pgfpathlineto{\pgfqpoint{3.147657in}{3.469864in}}%
\pgfpathlineto{\pgfqpoint{3.152199in}{3.469864in}}%
\pgfpathlineto{\pgfqpoint{3.152199in}{3.466915in}}%
\pgfpathmoveto{\pgfqpoint{3.147657in}{3.469864in}}%
\pgfpathlineto{\pgfqpoint{3.147657in}{3.469864in}}%
\pgfpathlineto{\pgfqpoint{3.147657in}{3.472813in}}%
\pgfpathlineto{\pgfqpoint{3.152199in}{3.472813in}}%
\pgfpathlineto{\pgfqpoint{3.152199in}{3.469864in}}%
\pgfpathmoveto{\pgfqpoint{3.152199in}{3.466915in}}%
\pgfpathlineto{\pgfqpoint{3.152199in}{3.466915in}}%
\pgfpathlineto{\pgfqpoint{3.152199in}{3.469864in}}%
\pgfpathlineto{\pgfqpoint{3.156740in}{3.469864in}}%
\pgfpathlineto{\pgfqpoint{3.156740in}{3.466915in}}%
\pgfpathmoveto{\pgfqpoint{3.156740in}{3.461016in}}%
\pgfpathlineto{\pgfqpoint{3.156740in}{3.461016in}}%
\pgfpathlineto{\pgfqpoint{3.156740in}{3.463966in}}%
\pgfpathlineto{\pgfqpoint{3.161281in}{3.463966in}}%
\pgfpathlineto{\pgfqpoint{3.161281in}{3.461016in}}%
\pgfpathmoveto{\pgfqpoint{3.156740in}{3.463966in}}%
\pgfpathlineto{\pgfqpoint{3.156740in}{3.463966in}}%
\pgfpathlineto{\pgfqpoint{3.156740in}{3.466915in}}%
\pgfpathlineto{\pgfqpoint{3.161281in}{3.466915in}}%
\pgfpathlineto{\pgfqpoint{3.161281in}{3.463966in}}%
\pgfpathmoveto{\pgfqpoint{3.161281in}{3.461016in}}%
\pgfpathlineto{\pgfqpoint{3.161281in}{3.461016in}}%
\pgfpathlineto{\pgfqpoint{3.161281in}{3.463966in}}%
\pgfpathlineto{\pgfqpoint{3.165822in}{3.463966in}}%
\pgfpathlineto{\pgfqpoint{3.165822in}{3.461016in}}%
\pgfpathmoveto{\pgfqpoint{3.165822in}{3.449219in}}%
\pgfpathlineto{\pgfqpoint{3.165822in}{3.449219in}}%
\pgfpathlineto{\pgfqpoint{3.165822in}{3.452168in}}%
\pgfpathlineto{\pgfqpoint{3.170363in}{3.452168in}}%
\pgfpathlineto{\pgfqpoint{3.170363in}{3.449219in}}%
\pgfpathmoveto{\pgfqpoint{3.165822in}{3.452168in}}%
\pgfpathlineto{\pgfqpoint{3.165822in}{3.452168in}}%
\pgfpathlineto{\pgfqpoint{3.165822in}{3.455118in}}%
\pgfpathlineto{\pgfqpoint{3.170363in}{3.455118in}}%
\pgfpathlineto{\pgfqpoint{3.170363in}{3.452168in}}%
\pgfpathmoveto{\pgfqpoint{3.170363in}{3.449219in}}%
\pgfpathlineto{\pgfqpoint{3.170363in}{3.449219in}}%
\pgfpathlineto{\pgfqpoint{3.170363in}{3.452168in}}%
\pgfpathlineto{\pgfqpoint{3.174904in}{3.452168in}}%
\pgfpathlineto{\pgfqpoint{3.174904in}{3.449219in}}%
\pgfpathmoveto{\pgfqpoint{3.170363in}{3.452168in}}%
\pgfpathlineto{\pgfqpoint{3.170363in}{3.452168in}}%
\pgfpathlineto{\pgfqpoint{3.170363in}{3.455118in}}%
\pgfpathlineto{\pgfqpoint{3.174904in}{3.455118in}}%
\pgfpathlineto{\pgfqpoint{3.174904in}{3.452168in}}%
\pgfpathmoveto{\pgfqpoint{3.165822in}{3.455118in}}%
\pgfpathlineto{\pgfqpoint{3.165822in}{3.455118in}}%
\pgfpathlineto{\pgfqpoint{3.165822in}{3.458067in}}%
\pgfpathlineto{\pgfqpoint{3.170363in}{3.458067in}}%
\pgfpathlineto{\pgfqpoint{3.170363in}{3.455118in}}%
\pgfpathmoveto{\pgfqpoint{3.165822in}{3.458067in}}%
\pgfpathlineto{\pgfqpoint{3.165822in}{3.458067in}}%
\pgfpathlineto{\pgfqpoint{3.165822in}{3.461016in}}%
\pgfpathlineto{\pgfqpoint{3.170363in}{3.461016in}}%
\pgfpathlineto{\pgfqpoint{3.170363in}{3.458067in}}%
\pgfpathmoveto{\pgfqpoint{3.170363in}{3.455118in}}%
\pgfpathlineto{\pgfqpoint{3.170363in}{3.455118in}}%
\pgfpathlineto{\pgfqpoint{3.170363in}{3.458067in}}%
\pgfpathlineto{\pgfqpoint{3.174904in}{3.458067in}}%
\pgfpathlineto{\pgfqpoint{3.174904in}{3.455118in}}%
\pgfpathmoveto{\pgfqpoint{3.174904in}{3.449219in}}%
\pgfpathlineto{\pgfqpoint{3.174904in}{3.449219in}}%
\pgfpathlineto{\pgfqpoint{3.174904in}{3.452168in}}%
\pgfpathlineto{\pgfqpoint{3.179445in}{3.452168in}}%
\pgfpathlineto{\pgfqpoint{3.179445in}{3.449219in}}%
\pgfpathmoveto{\pgfqpoint{3.174904in}{3.452168in}}%
\pgfpathlineto{\pgfqpoint{3.174904in}{3.452168in}}%
\pgfpathlineto{\pgfqpoint{3.174904in}{3.455118in}}%
\pgfpathlineto{\pgfqpoint{3.179445in}{3.455118in}}%
\pgfpathlineto{\pgfqpoint{3.179445in}{3.452168in}}%
\pgfpathmoveto{\pgfqpoint{3.179445in}{3.449219in}}%
\pgfpathlineto{\pgfqpoint{3.179445in}{3.449219in}}%
\pgfpathlineto{\pgfqpoint{3.179445in}{3.452168in}}%
\pgfpathlineto{\pgfqpoint{3.183986in}{3.452168in}}%
\pgfpathlineto{\pgfqpoint{3.183986in}{3.449219in}}%
\pgfpathmoveto{\pgfqpoint{3.193068in}{3.431524in}}%
\pgfpathlineto{\pgfqpoint{3.193068in}{3.431524in}}%
\pgfpathlineto{\pgfqpoint{3.193068in}{3.434473in}}%
\pgfpathlineto{\pgfqpoint{3.197609in}{3.434473in}}%
\pgfpathlineto{\pgfqpoint{3.197609in}{3.431524in}}%
\pgfpathmoveto{\pgfqpoint{3.193068in}{3.434473in}}%
\pgfpathlineto{\pgfqpoint{3.193068in}{3.434473in}}%
\pgfpathlineto{\pgfqpoint{3.193068in}{3.437422in}}%
\pgfpathlineto{\pgfqpoint{3.197609in}{3.437422in}}%
\pgfpathlineto{\pgfqpoint{3.197609in}{3.434473in}}%
\pgfpathmoveto{\pgfqpoint{3.197609in}{3.431524in}}%
\pgfpathlineto{\pgfqpoint{3.197609in}{3.431524in}}%
\pgfpathlineto{\pgfqpoint{3.197609in}{3.434473in}}%
\pgfpathlineto{\pgfqpoint{3.202150in}{3.434473in}}%
\pgfpathlineto{\pgfqpoint{3.202150in}{3.431524in}}%
\pgfpathmoveto{\pgfqpoint{3.197609in}{3.434473in}}%
\pgfpathlineto{\pgfqpoint{3.197609in}{3.434473in}}%
\pgfpathlineto{\pgfqpoint{3.197609in}{3.437422in}}%
\pgfpathlineto{\pgfqpoint{3.202150in}{3.437422in}}%
\pgfpathlineto{\pgfqpoint{3.202150in}{3.434473in}}%
\pgfpathmoveto{\pgfqpoint{3.183986in}{3.437422in}}%
\pgfpathlineto{\pgfqpoint{3.183986in}{3.437422in}}%
\pgfpathlineto{\pgfqpoint{3.183986in}{3.440371in}}%
\pgfpathlineto{\pgfqpoint{3.188527in}{3.440371in}}%
\pgfpathlineto{\pgfqpoint{3.188527in}{3.437422in}}%
\pgfpathmoveto{\pgfqpoint{3.183986in}{3.440371in}}%
\pgfpathlineto{\pgfqpoint{3.183986in}{3.440371in}}%
\pgfpathlineto{\pgfqpoint{3.183986in}{3.443321in}}%
\pgfpathlineto{\pgfqpoint{3.188527in}{3.443321in}}%
\pgfpathlineto{\pgfqpoint{3.188527in}{3.440371in}}%
\pgfpathmoveto{\pgfqpoint{3.188527in}{3.437422in}}%
\pgfpathlineto{\pgfqpoint{3.188527in}{3.437422in}}%
\pgfpathlineto{\pgfqpoint{3.188527in}{3.440371in}}%
\pgfpathlineto{\pgfqpoint{3.193068in}{3.440371in}}%
\pgfpathlineto{\pgfqpoint{3.193068in}{3.437422in}}%
\pgfpathmoveto{\pgfqpoint{3.188527in}{3.440371in}}%
\pgfpathlineto{\pgfqpoint{3.188527in}{3.440371in}}%
\pgfpathlineto{\pgfqpoint{3.188527in}{3.443321in}}%
\pgfpathlineto{\pgfqpoint{3.193068in}{3.443321in}}%
\pgfpathlineto{\pgfqpoint{3.193068in}{3.440371in}}%
\pgfpathmoveto{\pgfqpoint{3.183986in}{3.443321in}}%
\pgfpathlineto{\pgfqpoint{3.183986in}{3.443321in}}%
\pgfpathlineto{\pgfqpoint{3.183986in}{3.446270in}}%
\pgfpathlineto{\pgfqpoint{3.188527in}{3.446270in}}%
\pgfpathlineto{\pgfqpoint{3.188527in}{3.443321in}}%
\pgfpathmoveto{\pgfqpoint{3.183986in}{3.446270in}}%
\pgfpathlineto{\pgfqpoint{3.183986in}{3.446270in}}%
\pgfpathlineto{\pgfqpoint{3.183986in}{3.449219in}}%
\pgfpathlineto{\pgfqpoint{3.188527in}{3.449219in}}%
\pgfpathlineto{\pgfqpoint{3.188527in}{3.446270in}}%
\pgfpathmoveto{\pgfqpoint{3.188527in}{3.443321in}}%
\pgfpathlineto{\pgfqpoint{3.188527in}{3.443321in}}%
\pgfpathlineto{\pgfqpoint{3.188527in}{3.446270in}}%
\pgfpathlineto{\pgfqpoint{3.193068in}{3.446270in}}%
\pgfpathlineto{\pgfqpoint{3.193068in}{3.443321in}}%
\pgfpathmoveto{\pgfqpoint{3.193068in}{3.437422in}}%
\pgfpathlineto{\pgfqpoint{3.193068in}{3.437422in}}%
\pgfpathlineto{\pgfqpoint{3.193068in}{3.440371in}}%
\pgfpathlineto{\pgfqpoint{3.197609in}{3.440371in}}%
\pgfpathlineto{\pgfqpoint{3.197609in}{3.437422in}}%
\pgfpathmoveto{\pgfqpoint{3.193068in}{3.440371in}}%
\pgfpathlineto{\pgfqpoint{3.193068in}{3.440371in}}%
\pgfpathlineto{\pgfqpoint{3.193068in}{3.443321in}}%
\pgfpathlineto{\pgfqpoint{3.197609in}{3.443321in}}%
\pgfpathlineto{\pgfqpoint{3.197609in}{3.440371in}}%
\pgfpathmoveto{\pgfqpoint{3.197609in}{3.437422in}}%
\pgfpathlineto{\pgfqpoint{3.197609in}{3.437422in}}%
\pgfpathlineto{\pgfqpoint{3.197609in}{3.440371in}}%
\pgfpathlineto{\pgfqpoint{3.202150in}{3.440371in}}%
\pgfpathlineto{\pgfqpoint{3.202150in}{3.437422in}}%
\pgfpathmoveto{\pgfqpoint{3.202150in}{3.425625in}}%
\pgfpathlineto{\pgfqpoint{3.202150in}{3.425625in}}%
\pgfpathlineto{\pgfqpoint{3.202150in}{3.428574in}}%
\pgfpathlineto{\pgfqpoint{3.206691in}{3.428574in}}%
\pgfpathlineto{\pgfqpoint{3.206691in}{3.425625in}}%
\pgfpathmoveto{\pgfqpoint{3.202150in}{3.428574in}}%
\pgfpathlineto{\pgfqpoint{3.202150in}{3.428574in}}%
\pgfpathlineto{\pgfqpoint{3.202150in}{3.431524in}}%
\pgfpathlineto{\pgfqpoint{3.206691in}{3.431524in}}%
\pgfpathlineto{\pgfqpoint{3.206691in}{3.428574in}}%
\pgfpathmoveto{\pgfqpoint{3.206691in}{3.425625in}}%
\pgfpathlineto{\pgfqpoint{3.206691in}{3.425625in}}%
\pgfpathlineto{\pgfqpoint{3.206691in}{3.428574in}}%
\pgfpathlineto{\pgfqpoint{3.211232in}{3.428574in}}%
\pgfpathlineto{\pgfqpoint{3.211232in}{3.425625in}}%
\pgfpathmoveto{\pgfqpoint{3.206691in}{3.428574in}}%
\pgfpathlineto{\pgfqpoint{3.206691in}{3.428574in}}%
\pgfpathlineto{\pgfqpoint{3.206691in}{3.431524in}}%
\pgfpathlineto{\pgfqpoint{3.211232in}{3.431524in}}%
\pgfpathlineto{\pgfqpoint{3.211232in}{3.428574in}}%
\pgfpathmoveto{\pgfqpoint{3.202150in}{3.431524in}}%
\pgfpathlineto{\pgfqpoint{3.202150in}{3.431524in}}%
\pgfpathlineto{\pgfqpoint{3.202150in}{3.434473in}}%
\pgfpathlineto{\pgfqpoint{3.206691in}{3.434473in}}%
\pgfpathlineto{\pgfqpoint{3.206691in}{3.431524in}}%
\pgfpathmoveto{\pgfqpoint{3.202150in}{3.434473in}}%
\pgfpathlineto{\pgfqpoint{3.202150in}{3.434473in}}%
\pgfpathlineto{\pgfqpoint{3.202150in}{3.437422in}}%
\pgfpathlineto{\pgfqpoint{3.206691in}{3.437422in}}%
\pgfpathlineto{\pgfqpoint{3.206691in}{3.434473in}}%
\pgfpathmoveto{\pgfqpoint{3.206691in}{3.431524in}}%
\pgfpathlineto{\pgfqpoint{3.206691in}{3.431524in}}%
\pgfpathlineto{\pgfqpoint{3.206691in}{3.434473in}}%
\pgfpathlineto{\pgfqpoint{3.211232in}{3.434473in}}%
\pgfpathlineto{\pgfqpoint{3.211232in}{3.431524in}}%
\pgfpathmoveto{\pgfqpoint{3.211232in}{3.425625in}}%
\pgfpathlineto{\pgfqpoint{3.211232in}{3.425625in}}%
\pgfpathlineto{\pgfqpoint{3.211232in}{3.428574in}}%
\pgfpathlineto{\pgfqpoint{3.215773in}{3.428574in}}%
\pgfpathlineto{\pgfqpoint{3.215773in}{3.425625in}}%
\pgfpathmoveto{\pgfqpoint{3.211232in}{3.428574in}}%
\pgfpathlineto{\pgfqpoint{3.211232in}{3.428574in}}%
\pgfpathlineto{\pgfqpoint{3.211232in}{3.431524in}}%
\pgfpathlineto{\pgfqpoint{3.215773in}{3.431524in}}%
\pgfpathlineto{\pgfqpoint{3.215773in}{3.428574in}}%
\pgfpathmoveto{\pgfqpoint{3.215773in}{3.425625in}}%
\pgfpathlineto{\pgfqpoint{3.215773in}{3.425625in}}%
\pgfpathlineto{\pgfqpoint{3.215773in}{3.428574in}}%
\pgfpathlineto{\pgfqpoint{3.220314in}{3.428574in}}%
\pgfpathlineto{\pgfqpoint{3.220314in}{3.425625in}}%
\pgfpathmoveto{\pgfqpoint{3.333835in}{2.856426in}}%
\pgfpathlineto{\pgfqpoint{3.333835in}{2.856426in}}%
\pgfpathlineto{\pgfqpoint{3.333835in}{2.859375in}}%
\pgfpathlineto{\pgfqpoint{3.338376in}{2.859375in}}%
\pgfpathlineto{\pgfqpoint{3.338376in}{2.856426in}}%
\pgfpathmoveto{\pgfqpoint{3.338376in}{2.856426in}}%
\pgfpathlineto{\pgfqpoint{3.338376in}{2.856426in}}%
\pgfpathlineto{\pgfqpoint{3.338376in}{2.859375in}}%
\pgfpathlineto{\pgfqpoint{3.342917in}{2.859375in}}%
\pgfpathlineto{\pgfqpoint{3.342917in}{2.856426in}}%
\pgfpathmoveto{\pgfqpoint{3.342917in}{2.856426in}}%
\pgfpathlineto{\pgfqpoint{3.342917in}{2.856426in}}%
\pgfpathlineto{\pgfqpoint{3.342917in}{2.859375in}}%
\pgfpathlineto{\pgfqpoint{3.347458in}{2.859375in}}%
\pgfpathlineto{\pgfqpoint{3.347458in}{2.856426in}}%
\pgfpathmoveto{\pgfqpoint{3.347458in}{2.853477in}}%
\pgfpathlineto{\pgfqpoint{3.347458in}{2.853477in}}%
\pgfpathlineto{\pgfqpoint{3.347458in}{2.856426in}}%
\pgfpathlineto{\pgfqpoint{3.351999in}{2.856426in}}%
\pgfpathlineto{\pgfqpoint{3.351999in}{2.853477in}}%
\pgfpathmoveto{\pgfqpoint{3.347458in}{2.856426in}}%
\pgfpathlineto{\pgfqpoint{3.347458in}{2.856426in}}%
\pgfpathlineto{\pgfqpoint{3.347458in}{2.859375in}}%
\pgfpathlineto{\pgfqpoint{3.351999in}{2.859375in}}%
\pgfpathlineto{\pgfqpoint{3.351999in}{2.856426in}}%
\pgfpathmoveto{\pgfqpoint{3.351999in}{2.853477in}}%
\pgfpathlineto{\pgfqpoint{3.351999in}{2.853477in}}%
\pgfpathlineto{\pgfqpoint{3.351999in}{2.856426in}}%
\pgfpathlineto{\pgfqpoint{3.356539in}{2.856426in}}%
\pgfpathlineto{\pgfqpoint{3.356539in}{2.853477in}}%
\pgfpathmoveto{\pgfqpoint{3.351999in}{2.856426in}}%
\pgfpathlineto{\pgfqpoint{3.351999in}{2.856426in}}%
\pgfpathlineto{\pgfqpoint{3.351999in}{2.859375in}}%
\pgfpathlineto{\pgfqpoint{3.356539in}{2.859375in}}%
\pgfpathlineto{\pgfqpoint{3.356539in}{2.856426in}}%
\pgfpathmoveto{\pgfqpoint{3.361080in}{2.850528in}}%
\pgfpathlineto{\pgfqpoint{3.361080in}{2.850528in}}%
\pgfpathlineto{\pgfqpoint{3.361080in}{2.853477in}}%
\pgfpathlineto{\pgfqpoint{3.365621in}{2.853477in}}%
\pgfpathlineto{\pgfqpoint{3.365621in}{2.850528in}}%
\pgfpathmoveto{\pgfqpoint{3.356539in}{2.853477in}}%
\pgfpathlineto{\pgfqpoint{3.356539in}{2.853477in}}%
\pgfpathlineto{\pgfqpoint{3.356539in}{2.856426in}}%
\pgfpathlineto{\pgfqpoint{3.361080in}{2.856426in}}%
\pgfpathlineto{\pgfqpoint{3.361080in}{2.853477in}}%
\pgfpathmoveto{\pgfqpoint{3.356539in}{2.856426in}}%
\pgfpathlineto{\pgfqpoint{3.356539in}{2.856426in}}%
\pgfpathlineto{\pgfqpoint{3.356539in}{2.859375in}}%
\pgfpathlineto{\pgfqpoint{3.361080in}{2.859375in}}%
\pgfpathlineto{\pgfqpoint{3.361080in}{2.856426in}}%
\pgfpathmoveto{\pgfqpoint{3.361080in}{2.853477in}}%
\pgfpathlineto{\pgfqpoint{3.361080in}{2.853477in}}%
\pgfpathlineto{\pgfqpoint{3.361080in}{2.856426in}}%
\pgfpathlineto{\pgfqpoint{3.365621in}{2.856426in}}%
\pgfpathlineto{\pgfqpoint{3.365621in}{2.853477in}}%
\pgfpathmoveto{\pgfqpoint{3.361080in}{2.856426in}}%
\pgfpathlineto{\pgfqpoint{3.361080in}{2.856426in}}%
\pgfpathlineto{\pgfqpoint{3.361080in}{2.859375in}}%
\pgfpathlineto{\pgfqpoint{3.365621in}{2.859375in}}%
\pgfpathlineto{\pgfqpoint{3.365621in}{2.856426in}}%
\pgfpathmoveto{\pgfqpoint{3.224855in}{2.880020in}}%
\pgfpathlineto{\pgfqpoint{3.224855in}{2.880020in}}%
\pgfpathlineto{\pgfqpoint{3.224855in}{2.882970in}}%
\pgfpathlineto{\pgfqpoint{3.229396in}{2.882970in}}%
\pgfpathlineto{\pgfqpoint{3.229396in}{2.880020in}}%
\pgfpathmoveto{\pgfqpoint{3.229396in}{2.880020in}}%
\pgfpathlineto{\pgfqpoint{3.229396in}{2.880020in}}%
\pgfpathlineto{\pgfqpoint{3.229396in}{2.882970in}}%
\pgfpathlineto{\pgfqpoint{3.233937in}{2.882970in}}%
\pgfpathlineto{\pgfqpoint{3.233937in}{2.880020in}}%
\pgfpathmoveto{\pgfqpoint{3.233937in}{2.880020in}}%
\pgfpathlineto{\pgfqpoint{3.233937in}{2.880020in}}%
\pgfpathlineto{\pgfqpoint{3.233937in}{2.882970in}}%
\pgfpathlineto{\pgfqpoint{3.238478in}{2.882970in}}%
\pgfpathlineto{\pgfqpoint{3.238478in}{2.880020in}}%
\pgfpathmoveto{\pgfqpoint{3.238478in}{2.877071in}}%
\pgfpathlineto{\pgfqpoint{3.238478in}{2.877071in}}%
\pgfpathlineto{\pgfqpoint{3.238478in}{2.880020in}}%
\pgfpathlineto{\pgfqpoint{3.243018in}{2.880020in}}%
\pgfpathlineto{\pgfqpoint{3.243018in}{2.877071in}}%
\pgfpathmoveto{\pgfqpoint{3.238478in}{2.880020in}}%
\pgfpathlineto{\pgfqpoint{3.238478in}{2.880020in}}%
\pgfpathlineto{\pgfqpoint{3.238478in}{2.882970in}}%
\pgfpathlineto{\pgfqpoint{3.243018in}{2.882970in}}%
\pgfpathlineto{\pgfqpoint{3.243018in}{2.880020in}}%
\pgfpathmoveto{\pgfqpoint{3.243018in}{2.877071in}}%
\pgfpathlineto{\pgfqpoint{3.243018in}{2.877071in}}%
\pgfpathlineto{\pgfqpoint{3.243018in}{2.880020in}}%
\pgfpathlineto{\pgfqpoint{3.247559in}{2.880020in}}%
\pgfpathlineto{\pgfqpoint{3.247559in}{2.877071in}}%
\pgfpathmoveto{\pgfqpoint{3.243018in}{2.880020in}}%
\pgfpathlineto{\pgfqpoint{3.243018in}{2.880020in}}%
\pgfpathlineto{\pgfqpoint{3.243018in}{2.882970in}}%
\pgfpathlineto{\pgfqpoint{3.247559in}{2.882970in}}%
\pgfpathlineto{\pgfqpoint{3.247559in}{2.880020in}}%
\pgfpathmoveto{\pgfqpoint{3.252100in}{2.874122in}}%
\pgfpathlineto{\pgfqpoint{3.252100in}{2.874122in}}%
\pgfpathlineto{\pgfqpoint{3.252100in}{2.877071in}}%
\pgfpathlineto{\pgfqpoint{3.256641in}{2.877071in}}%
\pgfpathlineto{\pgfqpoint{3.256641in}{2.874122in}}%
\pgfpathmoveto{\pgfqpoint{3.247559in}{2.877071in}}%
\pgfpathlineto{\pgfqpoint{3.247559in}{2.877071in}}%
\pgfpathlineto{\pgfqpoint{3.247559in}{2.880020in}}%
\pgfpathlineto{\pgfqpoint{3.252100in}{2.880020in}}%
\pgfpathlineto{\pgfqpoint{3.252100in}{2.877071in}}%
\pgfpathmoveto{\pgfqpoint{3.247559in}{2.880020in}}%
\pgfpathlineto{\pgfqpoint{3.247559in}{2.880020in}}%
\pgfpathlineto{\pgfqpoint{3.247559in}{2.882970in}}%
\pgfpathlineto{\pgfqpoint{3.252100in}{2.882970in}}%
\pgfpathlineto{\pgfqpoint{3.252100in}{2.880020in}}%
\pgfpathmoveto{\pgfqpoint{3.252100in}{2.877071in}}%
\pgfpathlineto{\pgfqpoint{3.252100in}{2.877071in}}%
\pgfpathlineto{\pgfqpoint{3.252100in}{2.880020in}}%
\pgfpathlineto{\pgfqpoint{3.256641in}{2.880020in}}%
\pgfpathlineto{\pgfqpoint{3.256641in}{2.877071in}}%
\pgfpathmoveto{\pgfqpoint{3.252100in}{2.880020in}}%
\pgfpathlineto{\pgfqpoint{3.252100in}{2.880020in}}%
\pgfpathlineto{\pgfqpoint{3.252100in}{2.882970in}}%
\pgfpathlineto{\pgfqpoint{3.256641in}{2.882970in}}%
\pgfpathlineto{\pgfqpoint{3.256641in}{2.880020in}}%
\pgfpathmoveto{\pgfqpoint{3.220314in}{2.882970in}}%
\pgfpathlineto{\pgfqpoint{3.220314in}{2.882970in}}%
\pgfpathlineto{\pgfqpoint{3.220314in}{2.885919in}}%
\pgfpathlineto{\pgfqpoint{3.224855in}{2.885919in}}%
\pgfpathlineto{\pgfqpoint{3.224855in}{2.882970in}}%
\pgfpathmoveto{\pgfqpoint{3.220314in}{2.885919in}}%
\pgfpathlineto{\pgfqpoint{3.220314in}{2.885919in}}%
\pgfpathlineto{\pgfqpoint{3.220314in}{2.888868in}}%
\pgfpathlineto{\pgfqpoint{3.224855in}{2.888868in}}%
\pgfpathlineto{\pgfqpoint{3.224855in}{2.885919in}}%
\pgfpathmoveto{\pgfqpoint{3.224855in}{2.882970in}}%
\pgfpathlineto{\pgfqpoint{3.224855in}{2.882970in}}%
\pgfpathlineto{\pgfqpoint{3.224855in}{2.885919in}}%
\pgfpathlineto{\pgfqpoint{3.229396in}{2.885919in}}%
\pgfpathlineto{\pgfqpoint{3.229396in}{2.882970in}}%
\pgfpathmoveto{\pgfqpoint{3.224855in}{2.885919in}}%
\pgfpathlineto{\pgfqpoint{3.224855in}{2.885919in}}%
\pgfpathlineto{\pgfqpoint{3.224855in}{2.888868in}}%
\pgfpathlineto{\pgfqpoint{3.229396in}{2.888868in}}%
\pgfpathlineto{\pgfqpoint{3.229396in}{2.885919in}}%
\pgfpathmoveto{\pgfqpoint{3.256641in}{2.874122in}}%
\pgfpathlineto{\pgfqpoint{3.256641in}{2.874122in}}%
\pgfpathlineto{\pgfqpoint{3.256641in}{2.877071in}}%
\pgfpathlineto{\pgfqpoint{3.261182in}{2.877071in}}%
\pgfpathlineto{\pgfqpoint{3.261182in}{2.874122in}}%
\pgfpathmoveto{\pgfqpoint{3.261182in}{2.874122in}}%
\pgfpathlineto{\pgfqpoint{3.261182in}{2.874122in}}%
\pgfpathlineto{\pgfqpoint{3.261182in}{2.877071in}}%
\pgfpathlineto{\pgfqpoint{3.265723in}{2.877071in}}%
\pgfpathlineto{\pgfqpoint{3.265723in}{2.874122in}}%
\pgfpathmoveto{\pgfqpoint{3.265723in}{2.871173in}}%
\pgfpathlineto{\pgfqpoint{3.265723in}{2.871173in}}%
\pgfpathlineto{\pgfqpoint{3.265723in}{2.874122in}}%
\pgfpathlineto{\pgfqpoint{3.270263in}{2.874122in}}%
\pgfpathlineto{\pgfqpoint{3.270263in}{2.871173in}}%
\pgfpathmoveto{\pgfqpoint{3.265723in}{2.874122in}}%
\pgfpathlineto{\pgfqpoint{3.265723in}{2.874122in}}%
\pgfpathlineto{\pgfqpoint{3.265723in}{2.877071in}}%
\pgfpathlineto{\pgfqpoint{3.270263in}{2.877071in}}%
\pgfpathlineto{\pgfqpoint{3.270263in}{2.874122in}}%
\pgfpathmoveto{\pgfqpoint{3.270263in}{2.871173in}}%
\pgfpathlineto{\pgfqpoint{3.270263in}{2.871173in}}%
\pgfpathlineto{\pgfqpoint{3.270263in}{2.874122in}}%
\pgfpathlineto{\pgfqpoint{3.274804in}{2.874122in}}%
\pgfpathlineto{\pgfqpoint{3.274804in}{2.871173in}}%
\pgfpathmoveto{\pgfqpoint{3.270263in}{2.874122in}}%
\pgfpathlineto{\pgfqpoint{3.270263in}{2.874122in}}%
\pgfpathlineto{\pgfqpoint{3.270263in}{2.877071in}}%
\pgfpathlineto{\pgfqpoint{3.274804in}{2.877071in}}%
\pgfpathlineto{\pgfqpoint{3.274804in}{2.874122in}}%
\pgfpathmoveto{\pgfqpoint{3.279345in}{2.868223in}}%
\pgfpathlineto{\pgfqpoint{3.279345in}{2.868223in}}%
\pgfpathlineto{\pgfqpoint{3.279345in}{2.871173in}}%
\pgfpathlineto{\pgfqpoint{3.283886in}{2.871173in}}%
\pgfpathlineto{\pgfqpoint{3.283886in}{2.868223in}}%
\pgfpathmoveto{\pgfqpoint{3.283886in}{2.868223in}}%
\pgfpathlineto{\pgfqpoint{3.283886in}{2.868223in}}%
\pgfpathlineto{\pgfqpoint{3.283886in}{2.871173in}}%
\pgfpathlineto{\pgfqpoint{3.288427in}{2.871173in}}%
\pgfpathlineto{\pgfqpoint{3.288427in}{2.868223in}}%
\pgfpathmoveto{\pgfqpoint{3.288427in}{2.868223in}}%
\pgfpathlineto{\pgfqpoint{3.288427in}{2.868223in}}%
\pgfpathlineto{\pgfqpoint{3.288427in}{2.871173in}}%
\pgfpathlineto{\pgfqpoint{3.292968in}{2.871173in}}%
\pgfpathlineto{\pgfqpoint{3.292968in}{2.868223in}}%
\pgfpathmoveto{\pgfqpoint{3.274804in}{2.871173in}}%
\pgfpathlineto{\pgfqpoint{3.274804in}{2.871173in}}%
\pgfpathlineto{\pgfqpoint{3.274804in}{2.874122in}}%
\pgfpathlineto{\pgfqpoint{3.279345in}{2.874122in}}%
\pgfpathlineto{\pgfqpoint{3.279345in}{2.871173in}}%
\pgfpathmoveto{\pgfqpoint{3.274804in}{2.874122in}}%
\pgfpathlineto{\pgfqpoint{3.274804in}{2.874122in}}%
\pgfpathlineto{\pgfqpoint{3.274804in}{2.877071in}}%
\pgfpathlineto{\pgfqpoint{3.279345in}{2.877071in}}%
\pgfpathlineto{\pgfqpoint{3.279345in}{2.874122in}}%
\pgfpathmoveto{\pgfqpoint{3.279345in}{2.871173in}}%
\pgfpathlineto{\pgfqpoint{3.279345in}{2.871173in}}%
\pgfpathlineto{\pgfqpoint{3.279345in}{2.874122in}}%
\pgfpathlineto{\pgfqpoint{3.283886in}{2.874122in}}%
\pgfpathlineto{\pgfqpoint{3.283886in}{2.871173in}}%
\pgfpathmoveto{\pgfqpoint{3.279345in}{2.874122in}}%
\pgfpathlineto{\pgfqpoint{3.279345in}{2.874122in}}%
\pgfpathlineto{\pgfqpoint{3.279345in}{2.877071in}}%
\pgfpathlineto{\pgfqpoint{3.283886in}{2.877071in}}%
\pgfpathlineto{\pgfqpoint{3.283886in}{2.874122in}}%
\pgfpathmoveto{\pgfqpoint{3.292968in}{2.865274in}}%
\pgfpathlineto{\pgfqpoint{3.292968in}{2.865274in}}%
\pgfpathlineto{\pgfqpoint{3.292968in}{2.868223in}}%
\pgfpathlineto{\pgfqpoint{3.297509in}{2.868223in}}%
\pgfpathlineto{\pgfqpoint{3.297509in}{2.865274in}}%
\pgfpathmoveto{\pgfqpoint{3.292968in}{2.868223in}}%
\pgfpathlineto{\pgfqpoint{3.292968in}{2.868223in}}%
\pgfpathlineto{\pgfqpoint{3.292968in}{2.871173in}}%
\pgfpathlineto{\pgfqpoint{3.297509in}{2.871173in}}%
\pgfpathlineto{\pgfqpoint{3.297509in}{2.868223in}}%
\pgfpathmoveto{\pgfqpoint{3.297509in}{2.865274in}}%
\pgfpathlineto{\pgfqpoint{3.297509in}{2.865274in}}%
\pgfpathlineto{\pgfqpoint{3.297509in}{2.868223in}}%
\pgfpathlineto{\pgfqpoint{3.302049in}{2.868223in}}%
\pgfpathlineto{\pgfqpoint{3.302049in}{2.865274in}}%
\pgfpathmoveto{\pgfqpoint{3.297509in}{2.868223in}}%
\pgfpathlineto{\pgfqpoint{3.297509in}{2.868223in}}%
\pgfpathlineto{\pgfqpoint{3.297509in}{2.871173in}}%
\pgfpathlineto{\pgfqpoint{3.302049in}{2.871173in}}%
\pgfpathlineto{\pgfqpoint{3.302049in}{2.868223in}}%
\pgfpathmoveto{\pgfqpoint{3.306590in}{2.862325in}}%
\pgfpathlineto{\pgfqpoint{3.306590in}{2.862325in}}%
\pgfpathlineto{\pgfqpoint{3.306590in}{2.865274in}}%
\pgfpathlineto{\pgfqpoint{3.311131in}{2.865274in}}%
\pgfpathlineto{\pgfqpoint{3.311131in}{2.862325in}}%
\pgfpathmoveto{\pgfqpoint{3.302049in}{2.865274in}}%
\pgfpathlineto{\pgfqpoint{3.302049in}{2.865274in}}%
\pgfpathlineto{\pgfqpoint{3.302049in}{2.868223in}}%
\pgfpathlineto{\pgfqpoint{3.306590in}{2.868223in}}%
\pgfpathlineto{\pgfqpoint{3.306590in}{2.865274in}}%
\pgfpathmoveto{\pgfqpoint{3.302049in}{2.868223in}}%
\pgfpathlineto{\pgfqpoint{3.302049in}{2.868223in}}%
\pgfpathlineto{\pgfqpoint{3.302049in}{2.871173in}}%
\pgfpathlineto{\pgfqpoint{3.306590in}{2.871173in}}%
\pgfpathlineto{\pgfqpoint{3.306590in}{2.868223in}}%
\pgfpathmoveto{\pgfqpoint{3.306590in}{2.865274in}}%
\pgfpathlineto{\pgfqpoint{3.306590in}{2.865274in}}%
\pgfpathlineto{\pgfqpoint{3.306590in}{2.868223in}}%
\pgfpathlineto{\pgfqpoint{3.311131in}{2.868223in}}%
\pgfpathlineto{\pgfqpoint{3.311131in}{2.865274in}}%
\pgfpathmoveto{\pgfqpoint{3.306590in}{2.868223in}}%
\pgfpathlineto{\pgfqpoint{3.306590in}{2.868223in}}%
\pgfpathlineto{\pgfqpoint{3.306590in}{2.871173in}}%
\pgfpathlineto{\pgfqpoint{3.311131in}{2.871173in}}%
\pgfpathlineto{\pgfqpoint{3.311131in}{2.868223in}}%
\pgfpathmoveto{\pgfqpoint{3.311131in}{2.862325in}}%
\pgfpathlineto{\pgfqpoint{3.311131in}{2.862325in}}%
\pgfpathlineto{\pgfqpoint{3.311131in}{2.865274in}}%
\pgfpathlineto{\pgfqpoint{3.315672in}{2.865274in}}%
\pgfpathlineto{\pgfqpoint{3.315672in}{2.862325in}}%
\pgfpathmoveto{\pgfqpoint{3.315672in}{2.862325in}}%
\pgfpathlineto{\pgfqpoint{3.315672in}{2.862325in}}%
\pgfpathlineto{\pgfqpoint{3.315672in}{2.865274in}}%
\pgfpathlineto{\pgfqpoint{3.320213in}{2.865274in}}%
\pgfpathlineto{\pgfqpoint{3.320213in}{2.862325in}}%
\pgfpathmoveto{\pgfqpoint{3.320213in}{2.859375in}}%
\pgfpathlineto{\pgfqpoint{3.320213in}{2.859375in}}%
\pgfpathlineto{\pgfqpoint{3.320213in}{2.862325in}}%
\pgfpathlineto{\pgfqpoint{3.324754in}{2.862325in}}%
\pgfpathlineto{\pgfqpoint{3.324754in}{2.859375in}}%
\pgfpathmoveto{\pgfqpoint{3.320213in}{2.862325in}}%
\pgfpathlineto{\pgfqpoint{3.320213in}{2.862325in}}%
\pgfpathlineto{\pgfqpoint{3.320213in}{2.865274in}}%
\pgfpathlineto{\pgfqpoint{3.324754in}{2.865274in}}%
\pgfpathlineto{\pgfqpoint{3.324754in}{2.862325in}}%
\pgfpathmoveto{\pgfqpoint{3.324754in}{2.859375in}}%
\pgfpathlineto{\pgfqpoint{3.324754in}{2.859375in}}%
\pgfpathlineto{\pgfqpoint{3.324754in}{2.862325in}}%
\pgfpathlineto{\pgfqpoint{3.329294in}{2.862325in}}%
\pgfpathlineto{\pgfqpoint{3.329294in}{2.859375in}}%
\pgfpathmoveto{\pgfqpoint{3.324754in}{2.862325in}}%
\pgfpathlineto{\pgfqpoint{3.324754in}{2.862325in}}%
\pgfpathlineto{\pgfqpoint{3.324754in}{2.865274in}}%
\pgfpathlineto{\pgfqpoint{3.329294in}{2.865274in}}%
\pgfpathlineto{\pgfqpoint{3.329294in}{2.862325in}}%
\pgfpathmoveto{\pgfqpoint{3.329294in}{2.859375in}}%
\pgfpathlineto{\pgfqpoint{3.329294in}{2.859375in}}%
\pgfpathlineto{\pgfqpoint{3.329294in}{2.862325in}}%
\pgfpathlineto{\pgfqpoint{3.333835in}{2.862325in}}%
\pgfpathlineto{\pgfqpoint{3.333835in}{2.859375in}}%
\pgfpathmoveto{\pgfqpoint{3.329294in}{2.862325in}}%
\pgfpathlineto{\pgfqpoint{3.329294in}{2.862325in}}%
\pgfpathlineto{\pgfqpoint{3.329294in}{2.865274in}}%
\pgfpathlineto{\pgfqpoint{3.333835in}{2.865274in}}%
\pgfpathlineto{\pgfqpoint{3.333835in}{2.862325in}}%
\pgfpathmoveto{\pgfqpoint{3.333835in}{2.859375in}}%
\pgfpathlineto{\pgfqpoint{3.333835in}{2.859375in}}%
\pgfpathlineto{\pgfqpoint{3.333835in}{2.862325in}}%
\pgfpathlineto{\pgfqpoint{3.338376in}{2.862325in}}%
\pgfpathlineto{\pgfqpoint{3.338376in}{2.859375in}}%
\pgfpathmoveto{\pgfqpoint{3.333835in}{2.862325in}}%
\pgfpathlineto{\pgfqpoint{3.333835in}{2.862325in}}%
\pgfpathlineto{\pgfqpoint{3.333835in}{2.865274in}}%
\pgfpathlineto{\pgfqpoint{3.338376in}{2.865274in}}%
\pgfpathlineto{\pgfqpoint{3.338376in}{2.862325in}}%
\pgfpathmoveto{\pgfqpoint{3.229396in}{3.407929in}}%
\pgfpathlineto{\pgfqpoint{3.229396in}{3.407929in}}%
\pgfpathlineto{\pgfqpoint{3.229396in}{3.410879in}}%
\pgfpathlineto{\pgfqpoint{3.233937in}{3.410879in}}%
\pgfpathlineto{\pgfqpoint{3.233937in}{3.407929in}}%
\pgfpathmoveto{\pgfqpoint{3.229396in}{3.410879in}}%
\pgfpathlineto{\pgfqpoint{3.229396in}{3.410879in}}%
\pgfpathlineto{\pgfqpoint{3.229396in}{3.413828in}}%
\pgfpathlineto{\pgfqpoint{3.233937in}{3.413828in}}%
\pgfpathlineto{\pgfqpoint{3.233937in}{3.410879in}}%
\pgfpathmoveto{\pgfqpoint{3.233937in}{3.407929in}}%
\pgfpathlineto{\pgfqpoint{3.233937in}{3.407929in}}%
\pgfpathlineto{\pgfqpoint{3.233937in}{3.410879in}}%
\pgfpathlineto{\pgfqpoint{3.238478in}{3.410879in}}%
\pgfpathlineto{\pgfqpoint{3.238478in}{3.407929in}}%
\pgfpathmoveto{\pgfqpoint{3.233937in}{3.410879in}}%
\pgfpathlineto{\pgfqpoint{3.233937in}{3.410879in}}%
\pgfpathlineto{\pgfqpoint{3.233937in}{3.413828in}}%
\pgfpathlineto{\pgfqpoint{3.238478in}{3.413828in}}%
\pgfpathlineto{\pgfqpoint{3.238478in}{3.410879in}}%
\pgfpathmoveto{\pgfqpoint{3.220314in}{3.413828in}}%
\pgfpathlineto{\pgfqpoint{3.220314in}{3.413828in}}%
\pgfpathlineto{\pgfqpoint{3.220314in}{3.416777in}}%
\pgfpathlineto{\pgfqpoint{3.224855in}{3.416777in}}%
\pgfpathlineto{\pgfqpoint{3.224855in}{3.413828in}}%
\pgfpathmoveto{\pgfqpoint{3.220314in}{3.416777in}}%
\pgfpathlineto{\pgfqpoint{3.220314in}{3.416777in}}%
\pgfpathlineto{\pgfqpoint{3.220314in}{3.419726in}}%
\pgfpathlineto{\pgfqpoint{3.224855in}{3.419726in}}%
\pgfpathlineto{\pgfqpoint{3.224855in}{3.416777in}}%
\pgfpathmoveto{\pgfqpoint{3.224855in}{3.413828in}}%
\pgfpathlineto{\pgfqpoint{3.224855in}{3.413828in}}%
\pgfpathlineto{\pgfqpoint{3.224855in}{3.416777in}}%
\pgfpathlineto{\pgfqpoint{3.229396in}{3.416777in}}%
\pgfpathlineto{\pgfqpoint{3.229396in}{3.413828in}}%
\pgfpathmoveto{\pgfqpoint{3.224855in}{3.416777in}}%
\pgfpathlineto{\pgfqpoint{3.224855in}{3.416777in}}%
\pgfpathlineto{\pgfqpoint{3.224855in}{3.419726in}}%
\pgfpathlineto{\pgfqpoint{3.229396in}{3.419726in}}%
\pgfpathlineto{\pgfqpoint{3.229396in}{3.416777in}}%
\pgfpathmoveto{\pgfqpoint{3.220314in}{3.419726in}}%
\pgfpathlineto{\pgfqpoint{3.220314in}{3.419726in}}%
\pgfpathlineto{\pgfqpoint{3.220314in}{3.422676in}}%
\pgfpathlineto{\pgfqpoint{3.224855in}{3.422676in}}%
\pgfpathlineto{\pgfqpoint{3.224855in}{3.419726in}}%
\pgfpathmoveto{\pgfqpoint{3.220314in}{3.422676in}}%
\pgfpathlineto{\pgfqpoint{3.220314in}{3.422676in}}%
\pgfpathlineto{\pgfqpoint{3.220314in}{3.425625in}}%
\pgfpathlineto{\pgfqpoint{3.224855in}{3.425625in}}%
\pgfpathlineto{\pgfqpoint{3.224855in}{3.422676in}}%
\pgfpathmoveto{\pgfqpoint{3.224855in}{3.419726in}}%
\pgfpathlineto{\pgfqpoint{3.224855in}{3.419726in}}%
\pgfpathlineto{\pgfqpoint{3.224855in}{3.422676in}}%
\pgfpathlineto{\pgfqpoint{3.229396in}{3.422676in}}%
\pgfpathlineto{\pgfqpoint{3.229396in}{3.419726in}}%
\pgfpathmoveto{\pgfqpoint{3.229396in}{3.413828in}}%
\pgfpathlineto{\pgfqpoint{3.229396in}{3.413828in}}%
\pgfpathlineto{\pgfqpoint{3.229396in}{3.416777in}}%
\pgfpathlineto{\pgfqpoint{3.233937in}{3.416777in}}%
\pgfpathlineto{\pgfqpoint{3.233937in}{3.413828in}}%
\pgfpathmoveto{\pgfqpoint{3.229396in}{3.416777in}}%
\pgfpathlineto{\pgfqpoint{3.229396in}{3.416777in}}%
\pgfpathlineto{\pgfqpoint{3.229396in}{3.419726in}}%
\pgfpathlineto{\pgfqpoint{3.233937in}{3.419726in}}%
\pgfpathlineto{\pgfqpoint{3.233937in}{3.416777in}}%
\pgfpathmoveto{\pgfqpoint{3.233937in}{3.413828in}}%
\pgfpathlineto{\pgfqpoint{3.233937in}{3.413828in}}%
\pgfpathlineto{\pgfqpoint{3.233937in}{3.416777in}}%
\pgfpathlineto{\pgfqpoint{3.238478in}{3.416777in}}%
\pgfpathlineto{\pgfqpoint{3.238478in}{3.413828in}}%
\pgfpathmoveto{\pgfqpoint{3.238478in}{3.402031in}}%
\pgfpathlineto{\pgfqpoint{3.238478in}{3.402031in}}%
\pgfpathlineto{\pgfqpoint{3.238478in}{3.404980in}}%
\pgfpathlineto{\pgfqpoint{3.243018in}{3.404980in}}%
\pgfpathlineto{\pgfqpoint{3.243018in}{3.402031in}}%
\pgfpathmoveto{\pgfqpoint{3.238478in}{3.404980in}}%
\pgfpathlineto{\pgfqpoint{3.238478in}{3.404980in}}%
\pgfpathlineto{\pgfqpoint{3.238478in}{3.407929in}}%
\pgfpathlineto{\pgfqpoint{3.243018in}{3.407929in}}%
\pgfpathlineto{\pgfqpoint{3.243018in}{3.404980in}}%
\pgfpathmoveto{\pgfqpoint{3.243018in}{3.402031in}}%
\pgfpathlineto{\pgfqpoint{3.243018in}{3.402031in}}%
\pgfpathlineto{\pgfqpoint{3.243018in}{3.404980in}}%
\pgfpathlineto{\pgfqpoint{3.247559in}{3.404980in}}%
\pgfpathlineto{\pgfqpoint{3.247559in}{3.402031in}}%
\pgfpathmoveto{\pgfqpoint{3.243018in}{3.404980in}}%
\pgfpathlineto{\pgfqpoint{3.243018in}{3.404980in}}%
\pgfpathlineto{\pgfqpoint{3.243018in}{3.407929in}}%
\pgfpathlineto{\pgfqpoint{3.247559in}{3.407929in}}%
\pgfpathlineto{\pgfqpoint{3.247559in}{3.404980in}}%
\pgfpathmoveto{\pgfqpoint{3.238478in}{3.407929in}}%
\pgfpathlineto{\pgfqpoint{3.238478in}{3.407929in}}%
\pgfpathlineto{\pgfqpoint{3.238478in}{3.410879in}}%
\pgfpathlineto{\pgfqpoint{3.243018in}{3.410879in}}%
\pgfpathlineto{\pgfqpoint{3.243018in}{3.407929in}}%
\pgfpathmoveto{\pgfqpoint{3.238478in}{3.410879in}}%
\pgfpathlineto{\pgfqpoint{3.238478in}{3.410879in}}%
\pgfpathlineto{\pgfqpoint{3.238478in}{3.413828in}}%
\pgfpathlineto{\pgfqpoint{3.243018in}{3.413828in}}%
\pgfpathlineto{\pgfqpoint{3.243018in}{3.410879in}}%
\pgfpathmoveto{\pgfqpoint{3.243018in}{3.407929in}}%
\pgfpathlineto{\pgfqpoint{3.243018in}{3.407929in}}%
\pgfpathlineto{\pgfqpoint{3.243018in}{3.410879in}}%
\pgfpathlineto{\pgfqpoint{3.247559in}{3.410879in}}%
\pgfpathlineto{\pgfqpoint{3.247559in}{3.407929in}}%
\pgfpathmoveto{\pgfqpoint{3.247559in}{3.402031in}}%
\pgfpathlineto{\pgfqpoint{3.247559in}{3.402031in}}%
\pgfpathlineto{\pgfqpoint{3.247559in}{3.404980in}}%
\pgfpathlineto{\pgfqpoint{3.252100in}{3.404980in}}%
\pgfpathlineto{\pgfqpoint{3.252100in}{3.402031in}}%
\pgfpathmoveto{\pgfqpoint{3.247559in}{3.404980in}}%
\pgfpathlineto{\pgfqpoint{3.247559in}{3.404980in}}%
\pgfpathlineto{\pgfqpoint{3.247559in}{3.407929in}}%
\pgfpathlineto{\pgfqpoint{3.252100in}{3.407929in}}%
\pgfpathlineto{\pgfqpoint{3.252100in}{3.404980in}}%
\pgfpathmoveto{\pgfqpoint{3.252100in}{3.402031in}}%
\pgfpathlineto{\pgfqpoint{3.252100in}{3.402031in}}%
\pgfpathlineto{\pgfqpoint{3.252100in}{3.404980in}}%
\pgfpathlineto{\pgfqpoint{3.256641in}{3.404980in}}%
\pgfpathlineto{\pgfqpoint{3.256641in}{3.402031in}}%
\pgfpathmoveto{\pgfqpoint{3.252100in}{3.404980in}}%
\pgfpathlineto{\pgfqpoint{3.252100in}{3.404980in}}%
\pgfpathlineto{\pgfqpoint{3.252100in}{3.407929in}}%
\pgfpathlineto{\pgfqpoint{3.256641in}{3.407929in}}%
\pgfpathlineto{\pgfqpoint{3.256641in}{3.404980in}}%
\pgfpathmoveto{\pgfqpoint{3.256641in}{3.396132in}}%
\pgfpathlineto{\pgfqpoint{3.256641in}{3.396132in}}%
\pgfpathlineto{\pgfqpoint{3.256641in}{3.399081in}}%
\pgfpathlineto{\pgfqpoint{3.261182in}{3.399081in}}%
\pgfpathlineto{\pgfqpoint{3.261182in}{3.396132in}}%
\pgfpathmoveto{\pgfqpoint{3.256641in}{3.399081in}}%
\pgfpathlineto{\pgfqpoint{3.256641in}{3.399081in}}%
\pgfpathlineto{\pgfqpoint{3.256641in}{3.402031in}}%
\pgfpathlineto{\pgfqpoint{3.261182in}{3.402031in}}%
\pgfpathlineto{\pgfqpoint{3.261182in}{3.399081in}}%
\pgfpathmoveto{\pgfqpoint{3.261182in}{3.396132in}}%
\pgfpathlineto{\pgfqpoint{3.261182in}{3.396132in}}%
\pgfpathlineto{\pgfqpoint{3.261182in}{3.399081in}}%
\pgfpathlineto{\pgfqpoint{3.265723in}{3.399081in}}%
\pgfpathlineto{\pgfqpoint{3.265723in}{3.396132in}}%
\pgfpathmoveto{\pgfqpoint{3.261182in}{3.399081in}}%
\pgfpathlineto{\pgfqpoint{3.261182in}{3.399081in}}%
\pgfpathlineto{\pgfqpoint{3.261182in}{3.402031in}}%
\pgfpathlineto{\pgfqpoint{3.265723in}{3.402031in}}%
\pgfpathlineto{\pgfqpoint{3.265723in}{3.399081in}}%
\pgfpathmoveto{\pgfqpoint{3.265723in}{3.390233in}}%
\pgfpathlineto{\pgfqpoint{3.265723in}{3.390233in}}%
\pgfpathlineto{\pgfqpoint{3.265723in}{3.393183in}}%
\pgfpathlineto{\pgfqpoint{3.270263in}{3.393183in}}%
\pgfpathlineto{\pgfqpoint{3.270263in}{3.390233in}}%
\pgfpathmoveto{\pgfqpoint{3.265723in}{3.393183in}}%
\pgfpathlineto{\pgfqpoint{3.265723in}{3.393183in}}%
\pgfpathlineto{\pgfqpoint{3.265723in}{3.396132in}}%
\pgfpathlineto{\pgfqpoint{3.270263in}{3.396132in}}%
\pgfpathlineto{\pgfqpoint{3.270263in}{3.393183in}}%
\pgfpathmoveto{\pgfqpoint{3.270263in}{3.390233in}}%
\pgfpathlineto{\pgfqpoint{3.270263in}{3.390233in}}%
\pgfpathlineto{\pgfqpoint{3.270263in}{3.393183in}}%
\pgfpathlineto{\pgfqpoint{3.274804in}{3.393183in}}%
\pgfpathlineto{\pgfqpoint{3.274804in}{3.390233in}}%
\pgfpathmoveto{\pgfqpoint{3.270263in}{3.393183in}}%
\pgfpathlineto{\pgfqpoint{3.270263in}{3.393183in}}%
\pgfpathlineto{\pgfqpoint{3.270263in}{3.396132in}}%
\pgfpathlineto{\pgfqpoint{3.274804in}{3.396132in}}%
\pgfpathlineto{\pgfqpoint{3.274804in}{3.393183in}}%
\pgfpathmoveto{\pgfqpoint{3.265723in}{3.396132in}}%
\pgfpathlineto{\pgfqpoint{3.265723in}{3.396132in}}%
\pgfpathlineto{\pgfqpoint{3.265723in}{3.399081in}}%
\pgfpathlineto{\pgfqpoint{3.270263in}{3.399081in}}%
\pgfpathlineto{\pgfqpoint{3.270263in}{3.396132in}}%
\pgfpathmoveto{\pgfqpoint{3.274804in}{3.384335in}}%
\pgfpathlineto{\pgfqpoint{3.274804in}{3.384335in}}%
\pgfpathlineto{\pgfqpoint{3.274804in}{3.387284in}}%
\pgfpathlineto{\pgfqpoint{3.279345in}{3.387284in}}%
\pgfpathlineto{\pgfqpoint{3.279345in}{3.384335in}}%
\pgfpathmoveto{\pgfqpoint{3.274804in}{3.387284in}}%
\pgfpathlineto{\pgfqpoint{3.274804in}{3.387284in}}%
\pgfpathlineto{\pgfqpoint{3.274804in}{3.390233in}}%
\pgfpathlineto{\pgfqpoint{3.279345in}{3.390233in}}%
\pgfpathlineto{\pgfqpoint{3.279345in}{3.387284in}}%
\pgfpathmoveto{\pgfqpoint{3.279345in}{3.384335in}}%
\pgfpathlineto{\pgfqpoint{3.279345in}{3.384335in}}%
\pgfpathlineto{\pgfqpoint{3.279345in}{3.387284in}}%
\pgfpathlineto{\pgfqpoint{3.283886in}{3.387284in}}%
\pgfpathlineto{\pgfqpoint{3.283886in}{3.384335in}}%
\pgfpathmoveto{\pgfqpoint{3.279345in}{3.387284in}}%
\pgfpathlineto{\pgfqpoint{3.279345in}{3.387284in}}%
\pgfpathlineto{\pgfqpoint{3.279345in}{3.390233in}}%
\pgfpathlineto{\pgfqpoint{3.283886in}{3.390233in}}%
\pgfpathlineto{\pgfqpoint{3.283886in}{3.387284in}}%
\pgfpathmoveto{\pgfqpoint{3.283886in}{3.378436in}}%
\pgfpathlineto{\pgfqpoint{3.283886in}{3.378436in}}%
\pgfpathlineto{\pgfqpoint{3.283886in}{3.381386in}}%
\pgfpathlineto{\pgfqpoint{3.288427in}{3.381386in}}%
\pgfpathlineto{\pgfqpoint{3.288427in}{3.378436in}}%
\pgfpathmoveto{\pgfqpoint{3.283886in}{3.381386in}}%
\pgfpathlineto{\pgfqpoint{3.283886in}{3.381386in}}%
\pgfpathlineto{\pgfqpoint{3.283886in}{3.384335in}}%
\pgfpathlineto{\pgfqpoint{3.288427in}{3.384335in}}%
\pgfpathlineto{\pgfqpoint{3.288427in}{3.381386in}}%
\pgfpathmoveto{\pgfqpoint{3.288427in}{3.378436in}}%
\pgfpathlineto{\pgfqpoint{3.288427in}{3.378436in}}%
\pgfpathlineto{\pgfqpoint{3.288427in}{3.381386in}}%
\pgfpathlineto{\pgfqpoint{3.292968in}{3.381386in}}%
\pgfpathlineto{\pgfqpoint{3.292968in}{3.378436in}}%
\pgfpathmoveto{\pgfqpoint{3.288427in}{3.381386in}}%
\pgfpathlineto{\pgfqpoint{3.288427in}{3.381386in}}%
\pgfpathlineto{\pgfqpoint{3.288427in}{3.384335in}}%
\pgfpathlineto{\pgfqpoint{3.292968in}{3.384335in}}%
\pgfpathlineto{\pgfqpoint{3.292968in}{3.381386in}}%
\pgfpathmoveto{\pgfqpoint{3.283886in}{3.384335in}}%
\pgfpathlineto{\pgfqpoint{3.283886in}{3.384335in}}%
\pgfpathlineto{\pgfqpoint{3.283886in}{3.387284in}}%
\pgfpathlineto{\pgfqpoint{3.288427in}{3.387284in}}%
\pgfpathlineto{\pgfqpoint{3.288427in}{3.384335in}}%
\pgfpathmoveto{\pgfqpoint{3.274804in}{3.390233in}}%
\pgfpathlineto{\pgfqpoint{3.274804in}{3.390233in}}%
\pgfpathlineto{\pgfqpoint{3.274804in}{3.393183in}}%
\pgfpathlineto{\pgfqpoint{3.279345in}{3.393183in}}%
\pgfpathlineto{\pgfqpoint{3.279345in}{3.390233in}}%
\pgfpathmoveto{\pgfqpoint{3.256641in}{3.402031in}}%
\pgfpathlineto{\pgfqpoint{3.256641in}{3.402031in}}%
\pgfpathlineto{\pgfqpoint{3.256641in}{3.404980in}}%
\pgfpathlineto{\pgfqpoint{3.261182in}{3.404980in}}%
\pgfpathlineto{\pgfqpoint{3.261182in}{3.402031in}}%
\pgfpathmoveto{\pgfqpoint{3.292968in}{3.372538in}}%
\pgfpathlineto{\pgfqpoint{3.292968in}{3.372538in}}%
\pgfpathlineto{\pgfqpoint{3.292968in}{3.375487in}}%
\pgfpathlineto{\pgfqpoint{3.297509in}{3.375487in}}%
\pgfpathlineto{\pgfqpoint{3.297509in}{3.372538in}}%
\pgfpathmoveto{\pgfqpoint{3.292968in}{3.375487in}}%
\pgfpathlineto{\pgfqpoint{3.292968in}{3.375487in}}%
\pgfpathlineto{\pgfqpoint{3.292968in}{3.378436in}}%
\pgfpathlineto{\pgfqpoint{3.297509in}{3.378436in}}%
\pgfpathlineto{\pgfqpoint{3.297509in}{3.375487in}}%
\pgfpathmoveto{\pgfqpoint{3.297509in}{3.372538in}}%
\pgfpathlineto{\pgfqpoint{3.297509in}{3.372538in}}%
\pgfpathlineto{\pgfqpoint{3.297509in}{3.375487in}}%
\pgfpathlineto{\pgfqpoint{3.302049in}{3.375487in}}%
\pgfpathlineto{\pgfqpoint{3.302049in}{3.372538in}}%
\pgfpathmoveto{\pgfqpoint{3.297509in}{3.375487in}}%
\pgfpathlineto{\pgfqpoint{3.297509in}{3.375487in}}%
\pgfpathlineto{\pgfqpoint{3.297509in}{3.378436in}}%
\pgfpathlineto{\pgfqpoint{3.302049in}{3.378436in}}%
\pgfpathlineto{\pgfqpoint{3.302049in}{3.375487in}}%
\pgfpathmoveto{\pgfqpoint{3.302049in}{3.366639in}}%
\pgfpathlineto{\pgfqpoint{3.302049in}{3.366639in}}%
\pgfpathlineto{\pgfqpoint{3.302049in}{3.369588in}}%
\pgfpathlineto{\pgfqpoint{3.306590in}{3.369588in}}%
\pgfpathlineto{\pgfqpoint{3.306590in}{3.366639in}}%
\pgfpathmoveto{\pgfqpoint{3.302049in}{3.369588in}}%
\pgfpathlineto{\pgfqpoint{3.302049in}{3.369588in}}%
\pgfpathlineto{\pgfqpoint{3.302049in}{3.372538in}}%
\pgfpathlineto{\pgfqpoint{3.306590in}{3.372538in}}%
\pgfpathlineto{\pgfqpoint{3.306590in}{3.369588in}}%
\pgfpathmoveto{\pgfqpoint{3.306590in}{3.366639in}}%
\pgfpathlineto{\pgfqpoint{3.306590in}{3.366639in}}%
\pgfpathlineto{\pgfqpoint{3.306590in}{3.369588in}}%
\pgfpathlineto{\pgfqpoint{3.311131in}{3.369588in}}%
\pgfpathlineto{\pgfqpoint{3.311131in}{3.366639in}}%
\pgfpathmoveto{\pgfqpoint{3.306590in}{3.369588in}}%
\pgfpathlineto{\pgfqpoint{3.306590in}{3.369588in}}%
\pgfpathlineto{\pgfqpoint{3.306590in}{3.372538in}}%
\pgfpathlineto{\pgfqpoint{3.311131in}{3.372538in}}%
\pgfpathlineto{\pgfqpoint{3.311131in}{3.369588in}}%
\pgfpathmoveto{\pgfqpoint{3.302049in}{3.372538in}}%
\pgfpathlineto{\pgfqpoint{3.302049in}{3.372538in}}%
\pgfpathlineto{\pgfqpoint{3.302049in}{3.375487in}}%
\pgfpathlineto{\pgfqpoint{3.306590in}{3.375487in}}%
\pgfpathlineto{\pgfqpoint{3.306590in}{3.372538in}}%
\pgfpathmoveto{\pgfqpoint{3.311131in}{3.360741in}}%
\pgfpathlineto{\pgfqpoint{3.311131in}{3.360741in}}%
\pgfpathlineto{\pgfqpoint{3.311131in}{3.363690in}}%
\pgfpathlineto{\pgfqpoint{3.315672in}{3.363690in}}%
\pgfpathlineto{\pgfqpoint{3.315672in}{3.360741in}}%
\pgfpathmoveto{\pgfqpoint{3.311131in}{3.363690in}}%
\pgfpathlineto{\pgfqpoint{3.311131in}{3.363690in}}%
\pgfpathlineto{\pgfqpoint{3.311131in}{3.366639in}}%
\pgfpathlineto{\pgfqpoint{3.315672in}{3.366639in}}%
\pgfpathlineto{\pgfqpoint{3.315672in}{3.363690in}}%
\pgfpathmoveto{\pgfqpoint{3.315672in}{3.360741in}}%
\pgfpathlineto{\pgfqpoint{3.315672in}{3.360741in}}%
\pgfpathlineto{\pgfqpoint{3.315672in}{3.363690in}}%
\pgfpathlineto{\pgfqpoint{3.320213in}{3.363690in}}%
\pgfpathlineto{\pgfqpoint{3.320213in}{3.360741in}}%
\pgfpathmoveto{\pgfqpoint{3.315672in}{3.363690in}}%
\pgfpathlineto{\pgfqpoint{3.315672in}{3.363690in}}%
\pgfpathlineto{\pgfqpoint{3.315672in}{3.366639in}}%
\pgfpathlineto{\pgfqpoint{3.320213in}{3.366639in}}%
\pgfpathlineto{\pgfqpoint{3.320213in}{3.363690in}}%
\pgfpathmoveto{\pgfqpoint{3.320213in}{3.354842in}}%
\pgfpathlineto{\pgfqpoint{3.320213in}{3.354842in}}%
\pgfpathlineto{\pgfqpoint{3.320213in}{3.357791in}}%
\pgfpathlineto{\pgfqpoint{3.324754in}{3.357791in}}%
\pgfpathlineto{\pgfqpoint{3.324754in}{3.354842in}}%
\pgfpathmoveto{\pgfqpoint{3.320213in}{3.357791in}}%
\pgfpathlineto{\pgfqpoint{3.320213in}{3.357791in}}%
\pgfpathlineto{\pgfqpoint{3.320213in}{3.360741in}}%
\pgfpathlineto{\pgfqpoint{3.324754in}{3.360741in}}%
\pgfpathlineto{\pgfqpoint{3.324754in}{3.357791in}}%
\pgfpathmoveto{\pgfqpoint{3.324754in}{3.354842in}}%
\pgfpathlineto{\pgfqpoint{3.324754in}{3.354842in}}%
\pgfpathlineto{\pgfqpoint{3.324754in}{3.357791in}}%
\pgfpathlineto{\pgfqpoint{3.329294in}{3.357791in}}%
\pgfpathlineto{\pgfqpoint{3.329294in}{3.354842in}}%
\pgfpathmoveto{\pgfqpoint{3.324754in}{3.357791in}}%
\pgfpathlineto{\pgfqpoint{3.324754in}{3.357791in}}%
\pgfpathlineto{\pgfqpoint{3.324754in}{3.360741in}}%
\pgfpathlineto{\pgfqpoint{3.329294in}{3.360741in}}%
\pgfpathlineto{\pgfqpoint{3.329294in}{3.357791in}}%
\pgfpathmoveto{\pgfqpoint{3.320213in}{3.360741in}}%
\pgfpathlineto{\pgfqpoint{3.320213in}{3.360741in}}%
\pgfpathlineto{\pgfqpoint{3.320213in}{3.363690in}}%
\pgfpathlineto{\pgfqpoint{3.324754in}{3.363690in}}%
\pgfpathlineto{\pgfqpoint{3.324754in}{3.360741in}}%
\pgfpathmoveto{\pgfqpoint{3.311131in}{3.366639in}}%
\pgfpathlineto{\pgfqpoint{3.311131in}{3.366639in}}%
\pgfpathlineto{\pgfqpoint{3.311131in}{3.369588in}}%
\pgfpathlineto{\pgfqpoint{3.315672in}{3.369588in}}%
\pgfpathlineto{\pgfqpoint{3.315672in}{3.366639in}}%
\pgfpathmoveto{\pgfqpoint{3.329294in}{3.348943in}}%
\pgfpathlineto{\pgfqpoint{3.329294in}{3.348943in}}%
\pgfpathlineto{\pgfqpoint{3.329294in}{3.351893in}}%
\pgfpathlineto{\pgfqpoint{3.333835in}{3.351893in}}%
\pgfpathlineto{\pgfqpoint{3.333835in}{3.348943in}}%
\pgfpathmoveto{\pgfqpoint{3.329294in}{3.351893in}}%
\pgfpathlineto{\pgfqpoint{3.329294in}{3.351893in}}%
\pgfpathlineto{\pgfqpoint{3.329294in}{3.354842in}}%
\pgfpathlineto{\pgfqpoint{3.333835in}{3.354842in}}%
\pgfpathlineto{\pgfqpoint{3.333835in}{3.351893in}}%
\pgfpathmoveto{\pgfqpoint{3.333835in}{3.348943in}}%
\pgfpathlineto{\pgfqpoint{3.333835in}{3.348943in}}%
\pgfpathlineto{\pgfqpoint{3.333835in}{3.351893in}}%
\pgfpathlineto{\pgfqpoint{3.338376in}{3.351893in}}%
\pgfpathlineto{\pgfqpoint{3.338376in}{3.348943in}}%
\pgfpathmoveto{\pgfqpoint{3.333835in}{3.351893in}}%
\pgfpathlineto{\pgfqpoint{3.333835in}{3.351893in}}%
\pgfpathlineto{\pgfqpoint{3.333835in}{3.354842in}}%
\pgfpathlineto{\pgfqpoint{3.338376in}{3.354842in}}%
\pgfpathlineto{\pgfqpoint{3.338376in}{3.351893in}}%
\pgfpathmoveto{\pgfqpoint{3.338376in}{3.343045in}}%
\pgfpathlineto{\pgfqpoint{3.338376in}{3.343045in}}%
\pgfpathlineto{\pgfqpoint{3.338376in}{3.345994in}}%
\pgfpathlineto{\pgfqpoint{3.342917in}{3.345994in}}%
\pgfpathlineto{\pgfqpoint{3.342917in}{3.343045in}}%
\pgfpathmoveto{\pgfqpoint{3.338376in}{3.345994in}}%
\pgfpathlineto{\pgfqpoint{3.338376in}{3.345994in}}%
\pgfpathlineto{\pgfqpoint{3.338376in}{3.348943in}}%
\pgfpathlineto{\pgfqpoint{3.342917in}{3.348943in}}%
\pgfpathlineto{\pgfqpoint{3.342917in}{3.345994in}}%
\pgfpathmoveto{\pgfqpoint{3.342917in}{3.343045in}}%
\pgfpathlineto{\pgfqpoint{3.342917in}{3.343045in}}%
\pgfpathlineto{\pgfqpoint{3.342917in}{3.345994in}}%
\pgfpathlineto{\pgfqpoint{3.347458in}{3.345994in}}%
\pgfpathlineto{\pgfqpoint{3.347458in}{3.343045in}}%
\pgfpathmoveto{\pgfqpoint{3.342917in}{3.345994in}}%
\pgfpathlineto{\pgfqpoint{3.342917in}{3.345994in}}%
\pgfpathlineto{\pgfqpoint{3.342917in}{3.348943in}}%
\pgfpathlineto{\pgfqpoint{3.347458in}{3.348943in}}%
\pgfpathlineto{\pgfqpoint{3.347458in}{3.345994in}}%
\pgfpathmoveto{\pgfqpoint{3.338376in}{3.348943in}}%
\pgfpathlineto{\pgfqpoint{3.338376in}{3.348943in}}%
\pgfpathlineto{\pgfqpoint{3.338376in}{3.351893in}}%
\pgfpathlineto{\pgfqpoint{3.342917in}{3.351893in}}%
\pgfpathlineto{\pgfqpoint{3.342917in}{3.348943in}}%
\pgfpathmoveto{\pgfqpoint{3.347458in}{3.337146in}}%
\pgfpathlineto{\pgfqpoint{3.347458in}{3.337146in}}%
\pgfpathlineto{\pgfqpoint{3.347458in}{3.340096in}}%
\pgfpathlineto{\pgfqpoint{3.351999in}{3.340096in}}%
\pgfpathlineto{\pgfqpoint{3.351999in}{3.337146in}}%
\pgfpathmoveto{\pgfqpoint{3.347458in}{3.340096in}}%
\pgfpathlineto{\pgfqpoint{3.347458in}{3.340096in}}%
\pgfpathlineto{\pgfqpoint{3.347458in}{3.343045in}}%
\pgfpathlineto{\pgfqpoint{3.351999in}{3.343045in}}%
\pgfpathlineto{\pgfqpoint{3.351999in}{3.340096in}}%
\pgfpathmoveto{\pgfqpoint{3.351999in}{3.337146in}}%
\pgfpathlineto{\pgfqpoint{3.351999in}{3.337146in}}%
\pgfpathlineto{\pgfqpoint{3.351999in}{3.340096in}}%
\pgfpathlineto{\pgfqpoint{3.356539in}{3.340096in}}%
\pgfpathlineto{\pgfqpoint{3.356539in}{3.337146in}}%
\pgfpathmoveto{\pgfqpoint{3.351999in}{3.340096in}}%
\pgfpathlineto{\pgfqpoint{3.351999in}{3.340096in}}%
\pgfpathlineto{\pgfqpoint{3.351999in}{3.343045in}}%
\pgfpathlineto{\pgfqpoint{3.356539in}{3.343045in}}%
\pgfpathlineto{\pgfqpoint{3.356539in}{3.340096in}}%
\pgfpathmoveto{\pgfqpoint{3.356539in}{3.331248in}}%
\pgfpathlineto{\pgfqpoint{3.356539in}{3.331248in}}%
\pgfpathlineto{\pgfqpoint{3.356539in}{3.334197in}}%
\pgfpathlineto{\pgfqpoint{3.361080in}{3.334197in}}%
\pgfpathlineto{\pgfqpoint{3.361080in}{3.331248in}}%
\pgfpathmoveto{\pgfqpoint{3.356539in}{3.334197in}}%
\pgfpathlineto{\pgfqpoint{3.356539in}{3.334197in}}%
\pgfpathlineto{\pgfqpoint{3.356539in}{3.337146in}}%
\pgfpathlineto{\pgfqpoint{3.361080in}{3.337146in}}%
\pgfpathlineto{\pgfqpoint{3.361080in}{3.334197in}}%
\pgfpathmoveto{\pgfqpoint{3.361080in}{3.331248in}}%
\pgfpathlineto{\pgfqpoint{3.361080in}{3.331248in}}%
\pgfpathlineto{\pgfqpoint{3.361080in}{3.334197in}}%
\pgfpathlineto{\pgfqpoint{3.365621in}{3.334197in}}%
\pgfpathlineto{\pgfqpoint{3.365621in}{3.331248in}}%
\pgfpathmoveto{\pgfqpoint{3.361080in}{3.334197in}}%
\pgfpathlineto{\pgfqpoint{3.361080in}{3.334197in}}%
\pgfpathlineto{\pgfqpoint{3.361080in}{3.337146in}}%
\pgfpathlineto{\pgfqpoint{3.365621in}{3.337146in}}%
\pgfpathlineto{\pgfqpoint{3.365621in}{3.334197in}}%
\pgfpathmoveto{\pgfqpoint{3.356539in}{3.337146in}}%
\pgfpathlineto{\pgfqpoint{3.356539in}{3.337146in}}%
\pgfpathlineto{\pgfqpoint{3.356539in}{3.340096in}}%
\pgfpathlineto{\pgfqpoint{3.361080in}{3.340096in}}%
\pgfpathlineto{\pgfqpoint{3.361080in}{3.337146in}}%
\pgfpathmoveto{\pgfqpoint{3.347458in}{3.343045in}}%
\pgfpathlineto{\pgfqpoint{3.347458in}{3.343045in}}%
\pgfpathlineto{\pgfqpoint{3.347458in}{3.345994in}}%
\pgfpathlineto{\pgfqpoint{3.351999in}{3.345994in}}%
\pgfpathlineto{\pgfqpoint{3.351999in}{3.343045in}}%
\pgfpathmoveto{\pgfqpoint{3.329294in}{3.354842in}}%
\pgfpathlineto{\pgfqpoint{3.329294in}{3.354842in}}%
\pgfpathlineto{\pgfqpoint{3.329294in}{3.357791in}}%
\pgfpathlineto{\pgfqpoint{3.333835in}{3.357791in}}%
\pgfpathlineto{\pgfqpoint{3.333835in}{3.354842in}}%
\pgfpathmoveto{\pgfqpoint{3.292968in}{3.378436in}}%
\pgfpathlineto{\pgfqpoint{3.292968in}{3.378436in}}%
\pgfpathlineto{\pgfqpoint{3.292968in}{3.381386in}}%
\pgfpathlineto{\pgfqpoint{3.297509in}{3.381386in}}%
\pgfpathlineto{\pgfqpoint{3.297509in}{3.378436in}}%
\pgfpathmoveto{\pgfqpoint{3.365621in}{2.850528in}}%
\pgfpathlineto{\pgfqpoint{3.365621in}{2.850528in}}%
\pgfpathlineto{\pgfqpoint{3.365621in}{2.853477in}}%
\pgfpathlineto{\pgfqpoint{3.370162in}{2.853477in}}%
\pgfpathlineto{\pgfqpoint{3.370162in}{2.850528in}}%
\pgfpathmoveto{\pgfqpoint{3.370162in}{2.850528in}}%
\pgfpathlineto{\pgfqpoint{3.370162in}{2.850528in}}%
\pgfpathlineto{\pgfqpoint{3.370162in}{2.853477in}}%
\pgfpathlineto{\pgfqpoint{3.374704in}{2.853477in}}%
\pgfpathlineto{\pgfqpoint{3.374704in}{2.850528in}}%
\pgfpathmoveto{\pgfqpoint{3.374704in}{2.847578in}}%
\pgfpathlineto{\pgfqpoint{3.374704in}{2.847578in}}%
\pgfpathlineto{\pgfqpoint{3.374704in}{2.850528in}}%
\pgfpathlineto{\pgfqpoint{3.379245in}{2.850528in}}%
\pgfpathlineto{\pgfqpoint{3.379245in}{2.847578in}}%
\pgfpathmoveto{\pgfqpoint{3.374704in}{2.850528in}}%
\pgfpathlineto{\pgfqpoint{3.374704in}{2.850528in}}%
\pgfpathlineto{\pgfqpoint{3.374704in}{2.853477in}}%
\pgfpathlineto{\pgfqpoint{3.379245in}{2.853477in}}%
\pgfpathlineto{\pgfqpoint{3.379245in}{2.850528in}}%
\pgfpathmoveto{\pgfqpoint{3.379245in}{2.847578in}}%
\pgfpathlineto{\pgfqpoint{3.379245in}{2.847578in}}%
\pgfpathlineto{\pgfqpoint{3.379245in}{2.850528in}}%
\pgfpathlineto{\pgfqpoint{3.383786in}{2.850528in}}%
\pgfpathlineto{\pgfqpoint{3.383786in}{2.847578in}}%
\pgfpathmoveto{\pgfqpoint{3.379245in}{2.850528in}}%
\pgfpathlineto{\pgfqpoint{3.379245in}{2.850528in}}%
\pgfpathlineto{\pgfqpoint{3.379245in}{2.853477in}}%
\pgfpathlineto{\pgfqpoint{3.383786in}{2.853477in}}%
\pgfpathlineto{\pgfqpoint{3.383786in}{2.850528in}}%
\pgfpathmoveto{\pgfqpoint{3.388327in}{2.844629in}}%
\pgfpathlineto{\pgfqpoint{3.388327in}{2.844629in}}%
\pgfpathlineto{\pgfqpoint{3.388327in}{2.847578in}}%
\pgfpathlineto{\pgfqpoint{3.392869in}{2.847578in}}%
\pgfpathlineto{\pgfqpoint{3.392869in}{2.844629in}}%
\pgfpathmoveto{\pgfqpoint{3.392869in}{2.844629in}}%
\pgfpathlineto{\pgfqpoint{3.392869in}{2.844629in}}%
\pgfpathlineto{\pgfqpoint{3.392869in}{2.847578in}}%
\pgfpathlineto{\pgfqpoint{3.397410in}{2.847578in}}%
\pgfpathlineto{\pgfqpoint{3.397410in}{2.844629in}}%
\pgfpathmoveto{\pgfqpoint{3.397410in}{2.844629in}}%
\pgfpathlineto{\pgfqpoint{3.397410in}{2.844629in}}%
\pgfpathlineto{\pgfqpoint{3.397410in}{2.847578in}}%
\pgfpathlineto{\pgfqpoint{3.401951in}{2.847578in}}%
\pgfpathlineto{\pgfqpoint{3.401951in}{2.844629in}}%
\pgfpathmoveto{\pgfqpoint{3.383786in}{2.847578in}}%
\pgfpathlineto{\pgfqpoint{3.383786in}{2.847578in}}%
\pgfpathlineto{\pgfqpoint{3.383786in}{2.850528in}}%
\pgfpathlineto{\pgfqpoint{3.388327in}{2.850528in}}%
\pgfpathlineto{\pgfqpoint{3.388327in}{2.847578in}}%
\pgfpathmoveto{\pgfqpoint{3.383786in}{2.850528in}}%
\pgfpathlineto{\pgfqpoint{3.383786in}{2.850528in}}%
\pgfpathlineto{\pgfqpoint{3.383786in}{2.853477in}}%
\pgfpathlineto{\pgfqpoint{3.388327in}{2.853477in}}%
\pgfpathlineto{\pgfqpoint{3.388327in}{2.850528in}}%
\pgfpathmoveto{\pgfqpoint{3.388327in}{2.847578in}}%
\pgfpathlineto{\pgfqpoint{3.388327in}{2.847578in}}%
\pgfpathlineto{\pgfqpoint{3.388327in}{2.850528in}}%
\pgfpathlineto{\pgfqpoint{3.392869in}{2.850528in}}%
\pgfpathlineto{\pgfqpoint{3.392869in}{2.847578in}}%
\pgfpathmoveto{\pgfqpoint{3.388327in}{2.850528in}}%
\pgfpathlineto{\pgfqpoint{3.388327in}{2.850528in}}%
\pgfpathlineto{\pgfqpoint{3.388327in}{2.853477in}}%
\pgfpathlineto{\pgfqpoint{3.392869in}{2.853477in}}%
\pgfpathlineto{\pgfqpoint{3.392869in}{2.850528in}}%
\pgfpathmoveto{\pgfqpoint{3.401951in}{2.841680in}}%
\pgfpathlineto{\pgfqpoint{3.401951in}{2.841680in}}%
\pgfpathlineto{\pgfqpoint{3.401951in}{2.844629in}}%
\pgfpathlineto{\pgfqpoint{3.406493in}{2.844629in}}%
\pgfpathlineto{\pgfqpoint{3.406493in}{2.841680in}}%
\pgfpathmoveto{\pgfqpoint{3.401951in}{2.844629in}}%
\pgfpathlineto{\pgfqpoint{3.401951in}{2.844629in}}%
\pgfpathlineto{\pgfqpoint{3.401951in}{2.847578in}}%
\pgfpathlineto{\pgfqpoint{3.406493in}{2.847578in}}%
\pgfpathlineto{\pgfqpoint{3.406493in}{2.844629in}}%
\pgfpathmoveto{\pgfqpoint{3.406493in}{2.841680in}}%
\pgfpathlineto{\pgfqpoint{3.406493in}{2.841680in}}%
\pgfpathlineto{\pgfqpoint{3.406493in}{2.844629in}}%
\pgfpathlineto{\pgfqpoint{3.411034in}{2.844629in}}%
\pgfpathlineto{\pgfqpoint{3.411034in}{2.841680in}}%
\pgfpathmoveto{\pgfqpoint{3.406493in}{2.844629in}}%
\pgfpathlineto{\pgfqpoint{3.406493in}{2.844629in}}%
\pgfpathlineto{\pgfqpoint{3.406493in}{2.847578in}}%
\pgfpathlineto{\pgfqpoint{3.411034in}{2.847578in}}%
\pgfpathlineto{\pgfqpoint{3.411034in}{2.844629in}}%
\pgfpathmoveto{\pgfqpoint{3.415575in}{2.838730in}}%
\pgfpathlineto{\pgfqpoint{3.415575in}{2.838730in}}%
\pgfpathlineto{\pgfqpoint{3.415575in}{2.841680in}}%
\pgfpathlineto{\pgfqpoint{3.420116in}{2.841680in}}%
\pgfpathlineto{\pgfqpoint{3.420116in}{2.838730in}}%
\pgfpathmoveto{\pgfqpoint{3.411034in}{2.841680in}}%
\pgfpathlineto{\pgfqpoint{3.411034in}{2.841680in}}%
\pgfpathlineto{\pgfqpoint{3.411034in}{2.844629in}}%
\pgfpathlineto{\pgfqpoint{3.415575in}{2.844629in}}%
\pgfpathlineto{\pgfqpoint{3.415575in}{2.841680in}}%
\pgfpathmoveto{\pgfqpoint{3.411034in}{2.844629in}}%
\pgfpathlineto{\pgfqpoint{3.411034in}{2.844629in}}%
\pgfpathlineto{\pgfqpoint{3.411034in}{2.847578in}}%
\pgfpathlineto{\pgfqpoint{3.415575in}{2.847578in}}%
\pgfpathlineto{\pgfqpoint{3.415575in}{2.844629in}}%
\pgfpathmoveto{\pgfqpoint{3.415575in}{2.841680in}}%
\pgfpathlineto{\pgfqpoint{3.415575in}{2.841680in}}%
\pgfpathlineto{\pgfqpoint{3.415575in}{2.844629in}}%
\pgfpathlineto{\pgfqpoint{3.420116in}{2.844629in}}%
\pgfpathlineto{\pgfqpoint{3.420116in}{2.841680in}}%
\pgfpathmoveto{\pgfqpoint{3.415575in}{2.844629in}}%
\pgfpathlineto{\pgfqpoint{3.415575in}{2.844629in}}%
\pgfpathlineto{\pgfqpoint{3.415575in}{2.847578in}}%
\pgfpathlineto{\pgfqpoint{3.420116in}{2.847578in}}%
\pgfpathlineto{\pgfqpoint{3.420116in}{2.844629in}}%
\pgfpathmoveto{\pgfqpoint{3.420116in}{2.838730in}}%
\pgfpathlineto{\pgfqpoint{3.420116in}{2.838730in}}%
\pgfpathlineto{\pgfqpoint{3.420116in}{2.841680in}}%
\pgfpathlineto{\pgfqpoint{3.424658in}{2.841680in}}%
\pgfpathlineto{\pgfqpoint{3.424658in}{2.838730in}}%
\pgfpathmoveto{\pgfqpoint{3.424658in}{2.838730in}}%
\pgfpathlineto{\pgfqpoint{3.424658in}{2.838730in}}%
\pgfpathlineto{\pgfqpoint{3.424658in}{2.841680in}}%
\pgfpathlineto{\pgfqpoint{3.429199in}{2.841680in}}%
\pgfpathlineto{\pgfqpoint{3.429199in}{2.838730in}}%
\pgfpathmoveto{\pgfqpoint{3.429199in}{2.835781in}}%
\pgfpathlineto{\pgfqpoint{3.429199in}{2.835781in}}%
\pgfpathlineto{\pgfqpoint{3.429199in}{2.838730in}}%
\pgfpathlineto{\pgfqpoint{3.433740in}{2.838730in}}%
\pgfpathlineto{\pgfqpoint{3.433740in}{2.835781in}}%
\pgfpathmoveto{\pgfqpoint{3.429199in}{2.838730in}}%
\pgfpathlineto{\pgfqpoint{3.429199in}{2.838730in}}%
\pgfpathlineto{\pgfqpoint{3.429199in}{2.841680in}}%
\pgfpathlineto{\pgfqpoint{3.433740in}{2.841680in}}%
\pgfpathlineto{\pgfqpoint{3.433740in}{2.838730in}}%
\pgfpathmoveto{\pgfqpoint{3.433740in}{2.835781in}}%
\pgfpathlineto{\pgfqpoint{3.433740in}{2.835781in}}%
\pgfpathlineto{\pgfqpoint{3.433740in}{2.838730in}}%
\pgfpathlineto{\pgfqpoint{3.438281in}{2.838730in}}%
\pgfpathlineto{\pgfqpoint{3.438281in}{2.835781in}}%
\pgfpathmoveto{\pgfqpoint{3.433740in}{2.838730in}}%
\pgfpathlineto{\pgfqpoint{3.433740in}{2.838730in}}%
\pgfpathlineto{\pgfqpoint{3.433740in}{2.841680in}}%
\pgfpathlineto{\pgfqpoint{3.438281in}{2.841680in}}%
\pgfpathlineto{\pgfqpoint{3.438281in}{2.838730in}}%
\pgfpathmoveto{\pgfqpoint{3.442823in}{2.832832in}}%
\pgfpathlineto{\pgfqpoint{3.442823in}{2.832832in}}%
\pgfpathlineto{\pgfqpoint{3.442823in}{2.835781in}}%
\pgfpathlineto{\pgfqpoint{3.447364in}{2.835781in}}%
\pgfpathlineto{\pgfqpoint{3.447364in}{2.832832in}}%
\pgfpathmoveto{\pgfqpoint{3.447364in}{2.832832in}}%
\pgfpathlineto{\pgfqpoint{3.447364in}{2.832832in}}%
\pgfpathlineto{\pgfqpoint{3.447364in}{2.835781in}}%
\pgfpathlineto{\pgfqpoint{3.451905in}{2.835781in}}%
\pgfpathlineto{\pgfqpoint{3.451905in}{2.832832in}}%
\pgfpathmoveto{\pgfqpoint{3.451905in}{2.832832in}}%
\pgfpathlineto{\pgfqpoint{3.451905in}{2.832832in}}%
\pgfpathlineto{\pgfqpoint{3.451905in}{2.835781in}}%
\pgfpathlineto{\pgfqpoint{3.456447in}{2.835781in}}%
\pgfpathlineto{\pgfqpoint{3.456447in}{2.832832in}}%
\pgfpathmoveto{\pgfqpoint{3.456447in}{2.829883in}}%
\pgfpathlineto{\pgfqpoint{3.456447in}{2.829883in}}%
\pgfpathlineto{\pgfqpoint{3.456447in}{2.832832in}}%
\pgfpathlineto{\pgfqpoint{3.460988in}{2.832832in}}%
\pgfpathlineto{\pgfqpoint{3.460988in}{2.829883in}}%
\pgfpathmoveto{\pgfqpoint{3.456447in}{2.832832in}}%
\pgfpathlineto{\pgfqpoint{3.456447in}{2.832832in}}%
\pgfpathlineto{\pgfqpoint{3.456447in}{2.835781in}}%
\pgfpathlineto{\pgfqpoint{3.460988in}{2.835781in}}%
\pgfpathlineto{\pgfqpoint{3.460988in}{2.832832in}}%
\pgfpathmoveto{\pgfqpoint{3.460988in}{2.829883in}}%
\pgfpathlineto{\pgfqpoint{3.460988in}{2.829883in}}%
\pgfpathlineto{\pgfqpoint{3.460988in}{2.832832in}}%
\pgfpathlineto{\pgfqpoint{3.465529in}{2.832832in}}%
\pgfpathlineto{\pgfqpoint{3.465529in}{2.829883in}}%
\pgfpathmoveto{\pgfqpoint{3.460988in}{2.832832in}}%
\pgfpathlineto{\pgfqpoint{3.460988in}{2.832832in}}%
\pgfpathlineto{\pgfqpoint{3.460988in}{2.835781in}}%
\pgfpathlineto{\pgfqpoint{3.465529in}{2.835781in}}%
\pgfpathlineto{\pgfqpoint{3.465529in}{2.832832in}}%
\pgfpathmoveto{\pgfqpoint{3.470070in}{2.826933in}}%
\pgfpathlineto{\pgfqpoint{3.470070in}{2.826933in}}%
\pgfpathlineto{\pgfqpoint{3.470070in}{2.829883in}}%
\pgfpathlineto{\pgfqpoint{3.474612in}{2.829883in}}%
\pgfpathlineto{\pgfqpoint{3.474612in}{2.826933in}}%
\pgfpathmoveto{\pgfqpoint{3.465529in}{2.829883in}}%
\pgfpathlineto{\pgfqpoint{3.465529in}{2.829883in}}%
\pgfpathlineto{\pgfqpoint{3.465529in}{2.832832in}}%
\pgfpathlineto{\pgfqpoint{3.470070in}{2.832832in}}%
\pgfpathlineto{\pgfqpoint{3.470070in}{2.829883in}}%
\pgfpathmoveto{\pgfqpoint{3.465529in}{2.832832in}}%
\pgfpathlineto{\pgfqpoint{3.465529in}{2.832832in}}%
\pgfpathlineto{\pgfqpoint{3.465529in}{2.835781in}}%
\pgfpathlineto{\pgfqpoint{3.470070in}{2.835781in}}%
\pgfpathlineto{\pgfqpoint{3.470070in}{2.832832in}}%
\pgfpathmoveto{\pgfqpoint{3.470070in}{2.829883in}}%
\pgfpathlineto{\pgfqpoint{3.470070in}{2.829883in}}%
\pgfpathlineto{\pgfqpoint{3.470070in}{2.832832in}}%
\pgfpathlineto{\pgfqpoint{3.474612in}{2.832832in}}%
\pgfpathlineto{\pgfqpoint{3.474612in}{2.829883in}}%
\pgfpathmoveto{\pgfqpoint{3.470070in}{2.832832in}}%
\pgfpathlineto{\pgfqpoint{3.470070in}{2.832832in}}%
\pgfpathlineto{\pgfqpoint{3.470070in}{2.835781in}}%
\pgfpathlineto{\pgfqpoint{3.474612in}{2.835781in}}%
\pgfpathlineto{\pgfqpoint{3.474612in}{2.832832in}}%
\pgfpathmoveto{\pgfqpoint{3.438281in}{2.835781in}}%
\pgfpathlineto{\pgfqpoint{3.438281in}{2.835781in}}%
\pgfpathlineto{\pgfqpoint{3.438281in}{2.838730in}}%
\pgfpathlineto{\pgfqpoint{3.442823in}{2.838730in}}%
\pgfpathlineto{\pgfqpoint{3.442823in}{2.835781in}}%
\pgfpathmoveto{\pgfqpoint{3.438281in}{2.838730in}}%
\pgfpathlineto{\pgfqpoint{3.438281in}{2.838730in}}%
\pgfpathlineto{\pgfqpoint{3.438281in}{2.841680in}}%
\pgfpathlineto{\pgfqpoint{3.442823in}{2.841680in}}%
\pgfpathlineto{\pgfqpoint{3.442823in}{2.838730in}}%
\pgfpathmoveto{\pgfqpoint{3.442823in}{2.835781in}}%
\pgfpathlineto{\pgfqpoint{3.442823in}{2.835781in}}%
\pgfpathlineto{\pgfqpoint{3.442823in}{2.838730in}}%
\pgfpathlineto{\pgfqpoint{3.447364in}{2.838730in}}%
\pgfpathlineto{\pgfqpoint{3.447364in}{2.835781in}}%
\pgfpathmoveto{\pgfqpoint{3.442823in}{2.838730in}}%
\pgfpathlineto{\pgfqpoint{3.442823in}{2.838730in}}%
\pgfpathlineto{\pgfqpoint{3.442823in}{2.841680in}}%
\pgfpathlineto{\pgfqpoint{3.447364in}{2.841680in}}%
\pgfpathlineto{\pgfqpoint{3.447364in}{2.838730in}}%
\pgfpathmoveto{\pgfqpoint{3.474612in}{2.826933in}}%
\pgfpathlineto{\pgfqpoint{3.474612in}{2.826933in}}%
\pgfpathlineto{\pgfqpoint{3.474612in}{2.829883in}}%
\pgfpathlineto{\pgfqpoint{3.479153in}{2.829883in}}%
\pgfpathlineto{\pgfqpoint{3.479153in}{2.826933in}}%
\pgfpathmoveto{\pgfqpoint{3.479153in}{2.826933in}}%
\pgfpathlineto{\pgfqpoint{3.479153in}{2.826933in}}%
\pgfpathlineto{\pgfqpoint{3.479153in}{2.829883in}}%
\pgfpathlineto{\pgfqpoint{3.483694in}{2.829883in}}%
\pgfpathlineto{\pgfqpoint{3.483694in}{2.826933in}}%
\pgfpathmoveto{\pgfqpoint{3.483694in}{2.823984in}}%
\pgfpathlineto{\pgfqpoint{3.483694in}{2.823984in}}%
\pgfpathlineto{\pgfqpoint{3.483694in}{2.826933in}}%
\pgfpathlineto{\pgfqpoint{3.488235in}{2.826933in}}%
\pgfpathlineto{\pgfqpoint{3.488235in}{2.823984in}}%
\pgfpathmoveto{\pgfqpoint{3.483694in}{2.826933in}}%
\pgfpathlineto{\pgfqpoint{3.483694in}{2.826933in}}%
\pgfpathlineto{\pgfqpoint{3.483694in}{2.829883in}}%
\pgfpathlineto{\pgfqpoint{3.488235in}{2.829883in}}%
\pgfpathlineto{\pgfqpoint{3.488235in}{2.826933in}}%
\pgfpathmoveto{\pgfqpoint{3.488235in}{2.823984in}}%
\pgfpathlineto{\pgfqpoint{3.488235in}{2.823984in}}%
\pgfpathlineto{\pgfqpoint{3.488235in}{2.826933in}}%
\pgfpathlineto{\pgfqpoint{3.492777in}{2.826933in}}%
\pgfpathlineto{\pgfqpoint{3.492777in}{2.823984in}}%
\pgfpathmoveto{\pgfqpoint{3.488235in}{2.826933in}}%
\pgfpathlineto{\pgfqpoint{3.488235in}{2.826933in}}%
\pgfpathlineto{\pgfqpoint{3.488235in}{2.829883in}}%
\pgfpathlineto{\pgfqpoint{3.492777in}{2.829883in}}%
\pgfpathlineto{\pgfqpoint{3.492777in}{2.826933in}}%
\pgfpathmoveto{\pgfqpoint{3.497318in}{2.821035in}}%
\pgfpathlineto{\pgfqpoint{3.497318in}{2.821035in}}%
\pgfpathlineto{\pgfqpoint{3.497318in}{2.823984in}}%
\pgfpathlineto{\pgfqpoint{3.501859in}{2.823984in}}%
\pgfpathlineto{\pgfqpoint{3.501859in}{2.821035in}}%
\pgfpathmoveto{\pgfqpoint{3.501859in}{2.821035in}}%
\pgfpathlineto{\pgfqpoint{3.501859in}{2.821035in}}%
\pgfpathlineto{\pgfqpoint{3.501859in}{2.823984in}}%
\pgfpathlineto{\pgfqpoint{3.506401in}{2.823984in}}%
\pgfpathlineto{\pgfqpoint{3.506401in}{2.821035in}}%
\pgfpathmoveto{\pgfqpoint{3.506401in}{2.821035in}}%
\pgfpathlineto{\pgfqpoint{3.506401in}{2.821035in}}%
\pgfpathlineto{\pgfqpoint{3.506401in}{2.823984in}}%
\pgfpathlineto{\pgfqpoint{3.510942in}{2.823984in}}%
\pgfpathlineto{\pgfqpoint{3.510942in}{2.821035in}}%
\pgfpathmoveto{\pgfqpoint{3.492777in}{2.823984in}}%
\pgfpathlineto{\pgfqpoint{3.492777in}{2.823984in}}%
\pgfpathlineto{\pgfqpoint{3.492777in}{2.826933in}}%
\pgfpathlineto{\pgfqpoint{3.497318in}{2.826933in}}%
\pgfpathlineto{\pgfqpoint{3.497318in}{2.823984in}}%
\pgfpathmoveto{\pgfqpoint{3.492777in}{2.826933in}}%
\pgfpathlineto{\pgfqpoint{3.492777in}{2.826933in}}%
\pgfpathlineto{\pgfqpoint{3.492777in}{2.829883in}}%
\pgfpathlineto{\pgfqpoint{3.497318in}{2.829883in}}%
\pgfpathlineto{\pgfqpoint{3.497318in}{2.826933in}}%
\pgfpathmoveto{\pgfqpoint{3.497318in}{2.823984in}}%
\pgfpathlineto{\pgfqpoint{3.497318in}{2.823984in}}%
\pgfpathlineto{\pgfqpoint{3.497318in}{2.826933in}}%
\pgfpathlineto{\pgfqpoint{3.501859in}{2.826933in}}%
\pgfpathlineto{\pgfqpoint{3.501859in}{2.823984in}}%
\pgfpathmoveto{\pgfqpoint{3.497318in}{2.826933in}}%
\pgfpathlineto{\pgfqpoint{3.497318in}{2.826933in}}%
\pgfpathlineto{\pgfqpoint{3.497318in}{2.829883in}}%
\pgfpathlineto{\pgfqpoint{3.501859in}{2.829883in}}%
\pgfpathlineto{\pgfqpoint{3.501859in}{2.826933in}}%
\pgfpathmoveto{\pgfqpoint{3.501859in}{3.230976in}}%
\pgfpathlineto{\pgfqpoint{3.501859in}{3.230976in}}%
\pgfpathlineto{\pgfqpoint{3.501859in}{3.233925in}}%
\pgfpathlineto{\pgfqpoint{3.506401in}{3.233925in}}%
\pgfpathlineto{\pgfqpoint{3.506401in}{3.230976in}}%
\pgfpathmoveto{\pgfqpoint{3.501859in}{3.233925in}}%
\pgfpathlineto{\pgfqpoint{3.501859in}{3.233925in}}%
\pgfpathlineto{\pgfqpoint{3.501859in}{3.236874in}}%
\pgfpathlineto{\pgfqpoint{3.506401in}{3.236874in}}%
\pgfpathlineto{\pgfqpoint{3.506401in}{3.233925in}}%
\pgfpathmoveto{\pgfqpoint{3.506401in}{3.230976in}}%
\pgfpathlineto{\pgfqpoint{3.506401in}{3.230976in}}%
\pgfpathlineto{\pgfqpoint{3.506401in}{3.233925in}}%
\pgfpathlineto{\pgfqpoint{3.510942in}{3.233925in}}%
\pgfpathlineto{\pgfqpoint{3.510942in}{3.230976in}}%
\pgfpathmoveto{\pgfqpoint{3.506401in}{3.233925in}}%
\pgfpathlineto{\pgfqpoint{3.506401in}{3.233925in}}%
\pgfpathlineto{\pgfqpoint{3.506401in}{3.236874in}}%
\pgfpathlineto{\pgfqpoint{3.510942in}{3.236874in}}%
\pgfpathlineto{\pgfqpoint{3.510942in}{3.233925in}}%
\pgfpathmoveto{\pgfqpoint{3.365621in}{3.325349in}}%
\pgfpathlineto{\pgfqpoint{3.365621in}{3.325349in}}%
\pgfpathlineto{\pgfqpoint{3.365621in}{3.328299in}}%
\pgfpathlineto{\pgfqpoint{3.370162in}{3.328299in}}%
\pgfpathlineto{\pgfqpoint{3.370162in}{3.325349in}}%
\pgfpathmoveto{\pgfqpoint{3.365621in}{3.328299in}}%
\pgfpathlineto{\pgfqpoint{3.365621in}{3.328299in}}%
\pgfpathlineto{\pgfqpoint{3.365621in}{3.331248in}}%
\pgfpathlineto{\pgfqpoint{3.370162in}{3.331248in}}%
\pgfpathlineto{\pgfqpoint{3.370162in}{3.328299in}}%
\pgfpathmoveto{\pgfqpoint{3.370162in}{3.325349in}}%
\pgfpathlineto{\pgfqpoint{3.370162in}{3.325349in}}%
\pgfpathlineto{\pgfqpoint{3.370162in}{3.328299in}}%
\pgfpathlineto{\pgfqpoint{3.374704in}{3.328299in}}%
\pgfpathlineto{\pgfqpoint{3.374704in}{3.325349in}}%
\pgfpathmoveto{\pgfqpoint{3.370162in}{3.328299in}}%
\pgfpathlineto{\pgfqpoint{3.370162in}{3.328299in}}%
\pgfpathlineto{\pgfqpoint{3.370162in}{3.331248in}}%
\pgfpathlineto{\pgfqpoint{3.374704in}{3.331248in}}%
\pgfpathlineto{\pgfqpoint{3.374704in}{3.328299in}}%
\pgfpathmoveto{\pgfqpoint{3.374704in}{3.319451in}}%
\pgfpathlineto{\pgfqpoint{3.374704in}{3.319451in}}%
\pgfpathlineto{\pgfqpoint{3.374704in}{3.322400in}}%
\pgfpathlineto{\pgfqpoint{3.379245in}{3.322400in}}%
\pgfpathlineto{\pgfqpoint{3.379245in}{3.319451in}}%
\pgfpathmoveto{\pgfqpoint{3.374704in}{3.322400in}}%
\pgfpathlineto{\pgfqpoint{3.374704in}{3.322400in}}%
\pgfpathlineto{\pgfqpoint{3.374704in}{3.325349in}}%
\pgfpathlineto{\pgfqpoint{3.379245in}{3.325349in}}%
\pgfpathlineto{\pgfqpoint{3.379245in}{3.322400in}}%
\pgfpathmoveto{\pgfqpoint{3.379245in}{3.319451in}}%
\pgfpathlineto{\pgfqpoint{3.379245in}{3.319451in}}%
\pgfpathlineto{\pgfqpoint{3.379245in}{3.322400in}}%
\pgfpathlineto{\pgfqpoint{3.383786in}{3.322400in}}%
\pgfpathlineto{\pgfqpoint{3.383786in}{3.319451in}}%
\pgfpathmoveto{\pgfqpoint{3.379245in}{3.322400in}}%
\pgfpathlineto{\pgfqpoint{3.379245in}{3.322400in}}%
\pgfpathlineto{\pgfqpoint{3.379245in}{3.325349in}}%
\pgfpathlineto{\pgfqpoint{3.383786in}{3.325349in}}%
\pgfpathlineto{\pgfqpoint{3.383786in}{3.322400in}}%
\pgfpathmoveto{\pgfqpoint{3.374704in}{3.325349in}}%
\pgfpathlineto{\pgfqpoint{3.374704in}{3.325349in}}%
\pgfpathlineto{\pgfqpoint{3.374704in}{3.328299in}}%
\pgfpathlineto{\pgfqpoint{3.379245in}{3.328299in}}%
\pgfpathlineto{\pgfqpoint{3.379245in}{3.325349in}}%
\pgfpathmoveto{\pgfqpoint{3.383786in}{3.313553in}}%
\pgfpathlineto{\pgfqpoint{3.383786in}{3.313553in}}%
\pgfpathlineto{\pgfqpoint{3.383786in}{3.316502in}}%
\pgfpathlineto{\pgfqpoint{3.388327in}{3.316502in}}%
\pgfpathlineto{\pgfqpoint{3.388327in}{3.313553in}}%
\pgfpathmoveto{\pgfqpoint{3.383786in}{3.316502in}}%
\pgfpathlineto{\pgfqpoint{3.383786in}{3.316502in}}%
\pgfpathlineto{\pgfqpoint{3.383786in}{3.319451in}}%
\pgfpathlineto{\pgfqpoint{3.388327in}{3.319451in}}%
\pgfpathlineto{\pgfqpoint{3.388327in}{3.316502in}}%
\pgfpathmoveto{\pgfqpoint{3.388327in}{3.313553in}}%
\pgfpathlineto{\pgfqpoint{3.388327in}{3.313553in}}%
\pgfpathlineto{\pgfqpoint{3.388327in}{3.316502in}}%
\pgfpathlineto{\pgfqpoint{3.392869in}{3.316502in}}%
\pgfpathlineto{\pgfqpoint{3.392869in}{3.313553in}}%
\pgfpathmoveto{\pgfqpoint{3.388327in}{3.316502in}}%
\pgfpathlineto{\pgfqpoint{3.388327in}{3.316502in}}%
\pgfpathlineto{\pgfqpoint{3.388327in}{3.319451in}}%
\pgfpathlineto{\pgfqpoint{3.392869in}{3.319451in}}%
\pgfpathlineto{\pgfqpoint{3.392869in}{3.316502in}}%
\pgfpathmoveto{\pgfqpoint{3.392869in}{3.307654in}}%
\pgfpathlineto{\pgfqpoint{3.392869in}{3.307654in}}%
\pgfpathlineto{\pgfqpoint{3.392869in}{3.310604in}}%
\pgfpathlineto{\pgfqpoint{3.397410in}{3.310604in}}%
\pgfpathlineto{\pgfqpoint{3.397410in}{3.307654in}}%
\pgfpathmoveto{\pgfqpoint{3.392869in}{3.310604in}}%
\pgfpathlineto{\pgfqpoint{3.392869in}{3.310604in}}%
\pgfpathlineto{\pgfqpoint{3.392869in}{3.313553in}}%
\pgfpathlineto{\pgfqpoint{3.397410in}{3.313553in}}%
\pgfpathlineto{\pgfqpoint{3.397410in}{3.310604in}}%
\pgfpathmoveto{\pgfqpoint{3.397410in}{3.307654in}}%
\pgfpathlineto{\pgfqpoint{3.397410in}{3.307654in}}%
\pgfpathlineto{\pgfqpoint{3.397410in}{3.310604in}}%
\pgfpathlineto{\pgfqpoint{3.401951in}{3.310604in}}%
\pgfpathlineto{\pgfqpoint{3.401951in}{3.307654in}}%
\pgfpathmoveto{\pgfqpoint{3.397410in}{3.310604in}}%
\pgfpathlineto{\pgfqpoint{3.397410in}{3.310604in}}%
\pgfpathlineto{\pgfqpoint{3.397410in}{3.313553in}}%
\pgfpathlineto{\pgfqpoint{3.401951in}{3.313553in}}%
\pgfpathlineto{\pgfqpoint{3.401951in}{3.310604in}}%
\pgfpathmoveto{\pgfqpoint{3.392869in}{3.313553in}}%
\pgfpathlineto{\pgfqpoint{3.392869in}{3.313553in}}%
\pgfpathlineto{\pgfqpoint{3.392869in}{3.316502in}}%
\pgfpathlineto{\pgfqpoint{3.397410in}{3.316502in}}%
\pgfpathlineto{\pgfqpoint{3.397410in}{3.313553in}}%
\pgfpathmoveto{\pgfqpoint{3.383786in}{3.319451in}}%
\pgfpathlineto{\pgfqpoint{3.383786in}{3.319451in}}%
\pgfpathlineto{\pgfqpoint{3.383786in}{3.322400in}}%
\pgfpathlineto{\pgfqpoint{3.388327in}{3.322400in}}%
\pgfpathlineto{\pgfqpoint{3.388327in}{3.319451in}}%
\pgfpathmoveto{\pgfqpoint{3.401951in}{3.301756in}}%
\pgfpathlineto{\pgfqpoint{3.401951in}{3.301756in}}%
\pgfpathlineto{\pgfqpoint{3.401951in}{3.304705in}}%
\pgfpathlineto{\pgfqpoint{3.406493in}{3.304705in}}%
\pgfpathlineto{\pgfqpoint{3.406493in}{3.301756in}}%
\pgfpathmoveto{\pgfqpoint{3.401951in}{3.304705in}}%
\pgfpathlineto{\pgfqpoint{3.401951in}{3.304705in}}%
\pgfpathlineto{\pgfqpoint{3.401951in}{3.307654in}}%
\pgfpathlineto{\pgfqpoint{3.406493in}{3.307654in}}%
\pgfpathlineto{\pgfqpoint{3.406493in}{3.304705in}}%
\pgfpathmoveto{\pgfqpoint{3.406493in}{3.301756in}}%
\pgfpathlineto{\pgfqpoint{3.406493in}{3.301756in}}%
\pgfpathlineto{\pgfqpoint{3.406493in}{3.304705in}}%
\pgfpathlineto{\pgfqpoint{3.411034in}{3.304705in}}%
\pgfpathlineto{\pgfqpoint{3.411034in}{3.301756in}}%
\pgfpathmoveto{\pgfqpoint{3.406493in}{3.304705in}}%
\pgfpathlineto{\pgfqpoint{3.406493in}{3.304705in}}%
\pgfpathlineto{\pgfqpoint{3.406493in}{3.307654in}}%
\pgfpathlineto{\pgfqpoint{3.411034in}{3.307654in}}%
\pgfpathlineto{\pgfqpoint{3.411034in}{3.304705in}}%
\pgfpathmoveto{\pgfqpoint{3.411034in}{3.295858in}}%
\pgfpathlineto{\pgfqpoint{3.411034in}{3.295858in}}%
\pgfpathlineto{\pgfqpoint{3.411034in}{3.298807in}}%
\pgfpathlineto{\pgfqpoint{3.415575in}{3.298807in}}%
\pgfpathlineto{\pgfqpoint{3.415575in}{3.295858in}}%
\pgfpathmoveto{\pgfqpoint{3.411034in}{3.298807in}}%
\pgfpathlineto{\pgfqpoint{3.411034in}{3.298807in}}%
\pgfpathlineto{\pgfqpoint{3.411034in}{3.301756in}}%
\pgfpathlineto{\pgfqpoint{3.415575in}{3.301756in}}%
\pgfpathlineto{\pgfqpoint{3.415575in}{3.298807in}}%
\pgfpathmoveto{\pgfqpoint{3.415575in}{3.295858in}}%
\pgfpathlineto{\pgfqpoint{3.415575in}{3.295858in}}%
\pgfpathlineto{\pgfqpoint{3.415575in}{3.298807in}}%
\pgfpathlineto{\pgfqpoint{3.420116in}{3.298807in}}%
\pgfpathlineto{\pgfqpoint{3.420116in}{3.295858in}}%
\pgfpathmoveto{\pgfqpoint{3.415575in}{3.298807in}}%
\pgfpathlineto{\pgfqpoint{3.415575in}{3.298807in}}%
\pgfpathlineto{\pgfqpoint{3.415575in}{3.301756in}}%
\pgfpathlineto{\pgfqpoint{3.420116in}{3.301756in}}%
\pgfpathlineto{\pgfqpoint{3.420116in}{3.298807in}}%
\pgfpathmoveto{\pgfqpoint{3.411034in}{3.301756in}}%
\pgfpathlineto{\pgfqpoint{3.411034in}{3.301756in}}%
\pgfpathlineto{\pgfqpoint{3.411034in}{3.304705in}}%
\pgfpathlineto{\pgfqpoint{3.415575in}{3.304705in}}%
\pgfpathlineto{\pgfqpoint{3.415575in}{3.301756in}}%
\pgfpathmoveto{\pgfqpoint{3.420116in}{3.289959in}}%
\pgfpathlineto{\pgfqpoint{3.420116in}{3.289959in}}%
\pgfpathlineto{\pgfqpoint{3.420116in}{3.292908in}}%
\pgfpathlineto{\pgfqpoint{3.424658in}{3.292908in}}%
\pgfpathlineto{\pgfqpoint{3.424658in}{3.289959in}}%
\pgfpathmoveto{\pgfqpoint{3.420116in}{3.292908in}}%
\pgfpathlineto{\pgfqpoint{3.420116in}{3.292908in}}%
\pgfpathlineto{\pgfqpoint{3.420116in}{3.295858in}}%
\pgfpathlineto{\pgfqpoint{3.424658in}{3.295858in}}%
\pgfpathlineto{\pgfqpoint{3.424658in}{3.292908in}}%
\pgfpathmoveto{\pgfqpoint{3.424658in}{3.289959in}}%
\pgfpathlineto{\pgfqpoint{3.424658in}{3.289959in}}%
\pgfpathlineto{\pgfqpoint{3.424658in}{3.292908in}}%
\pgfpathlineto{\pgfqpoint{3.429199in}{3.292908in}}%
\pgfpathlineto{\pgfqpoint{3.429199in}{3.289959in}}%
\pgfpathmoveto{\pgfqpoint{3.424658in}{3.292908in}}%
\pgfpathlineto{\pgfqpoint{3.424658in}{3.292908in}}%
\pgfpathlineto{\pgfqpoint{3.424658in}{3.295858in}}%
\pgfpathlineto{\pgfqpoint{3.429199in}{3.295858in}}%
\pgfpathlineto{\pgfqpoint{3.429199in}{3.292908in}}%
\pgfpathmoveto{\pgfqpoint{3.429199in}{3.284061in}}%
\pgfpathlineto{\pgfqpoint{3.429199in}{3.284061in}}%
\pgfpathlineto{\pgfqpoint{3.429199in}{3.287010in}}%
\pgfpathlineto{\pgfqpoint{3.433740in}{3.287010in}}%
\pgfpathlineto{\pgfqpoint{3.433740in}{3.284061in}}%
\pgfpathmoveto{\pgfqpoint{3.429199in}{3.287010in}}%
\pgfpathlineto{\pgfqpoint{3.429199in}{3.287010in}}%
\pgfpathlineto{\pgfqpoint{3.429199in}{3.289959in}}%
\pgfpathlineto{\pgfqpoint{3.433740in}{3.289959in}}%
\pgfpathlineto{\pgfqpoint{3.433740in}{3.287010in}}%
\pgfpathmoveto{\pgfqpoint{3.433740in}{3.284061in}}%
\pgfpathlineto{\pgfqpoint{3.433740in}{3.284061in}}%
\pgfpathlineto{\pgfqpoint{3.433740in}{3.287010in}}%
\pgfpathlineto{\pgfqpoint{3.438281in}{3.287010in}}%
\pgfpathlineto{\pgfqpoint{3.438281in}{3.284061in}}%
\pgfpathmoveto{\pgfqpoint{3.433740in}{3.287010in}}%
\pgfpathlineto{\pgfqpoint{3.433740in}{3.287010in}}%
\pgfpathlineto{\pgfqpoint{3.433740in}{3.289959in}}%
\pgfpathlineto{\pgfqpoint{3.438281in}{3.289959in}}%
\pgfpathlineto{\pgfqpoint{3.438281in}{3.287010in}}%
\pgfpathmoveto{\pgfqpoint{3.429199in}{3.289959in}}%
\pgfpathlineto{\pgfqpoint{3.429199in}{3.289959in}}%
\pgfpathlineto{\pgfqpoint{3.429199in}{3.292908in}}%
\pgfpathlineto{\pgfqpoint{3.433740in}{3.292908in}}%
\pgfpathlineto{\pgfqpoint{3.433740in}{3.289959in}}%
\pgfpathmoveto{\pgfqpoint{3.420116in}{3.295858in}}%
\pgfpathlineto{\pgfqpoint{3.420116in}{3.295858in}}%
\pgfpathlineto{\pgfqpoint{3.420116in}{3.298807in}}%
\pgfpathlineto{\pgfqpoint{3.424658in}{3.298807in}}%
\pgfpathlineto{\pgfqpoint{3.424658in}{3.295858in}}%
\pgfpathmoveto{\pgfqpoint{3.401951in}{3.307654in}}%
\pgfpathlineto{\pgfqpoint{3.401951in}{3.307654in}}%
\pgfpathlineto{\pgfqpoint{3.401951in}{3.310604in}}%
\pgfpathlineto{\pgfqpoint{3.406493in}{3.310604in}}%
\pgfpathlineto{\pgfqpoint{3.406493in}{3.307654in}}%
\pgfpathmoveto{\pgfqpoint{3.465529in}{3.254569in}}%
\pgfpathlineto{\pgfqpoint{3.465529in}{3.254569in}}%
\pgfpathlineto{\pgfqpoint{3.465529in}{3.257518in}}%
\pgfpathlineto{\pgfqpoint{3.470070in}{3.257518in}}%
\pgfpathlineto{\pgfqpoint{3.470070in}{3.254569in}}%
\pgfpathmoveto{\pgfqpoint{3.465529in}{3.257518in}}%
\pgfpathlineto{\pgfqpoint{3.465529in}{3.257518in}}%
\pgfpathlineto{\pgfqpoint{3.465529in}{3.260468in}}%
\pgfpathlineto{\pgfqpoint{3.470070in}{3.260468in}}%
\pgfpathlineto{\pgfqpoint{3.470070in}{3.257518in}}%
\pgfpathmoveto{\pgfqpoint{3.470070in}{3.254569in}}%
\pgfpathlineto{\pgfqpoint{3.470070in}{3.254569in}}%
\pgfpathlineto{\pgfqpoint{3.470070in}{3.257518in}}%
\pgfpathlineto{\pgfqpoint{3.474612in}{3.257518in}}%
\pgfpathlineto{\pgfqpoint{3.474612in}{3.254569in}}%
\pgfpathmoveto{\pgfqpoint{3.470070in}{3.257518in}}%
\pgfpathlineto{\pgfqpoint{3.470070in}{3.257518in}}%
\pgfpathlineto{\pgfqpoint{3.470070in}{3.260468in}}%
\pgfpathlineto{\pgfqpoint{3.474612in}{3.260468in}}%
\pgfpathlineto{\pgfqpoint{3.474612in}{3.257518in}}%
\pgfpathmoveto{\pgfqpoint{3.438281in}{3.278163in}}%
\pgfpathlineto{\pgfqpoint{3.438281in}{3.278163in}}%
\pgfpathlineto{\pgfqpoint{3.438281in}{3.281112in}}%
\pgfpathlineto{\pgfqpoint{3.442823in}{3.281112in}}%
\pgfpathlineto{\pgfqpoint{3.442823in}{3.278163in}}%
\pgfpathmoveto{\pgfqpoint{3.438281in}{3.281112in}}%
\pgfpathlineto{\pgfqpoint{3.438281in}{3.281112in}}%
\pgfpathlineto{\pgfqpoint{3.438281in}{3.284061in}}%
\pgfpathlineto{\pgfqpoint{3.442823in}{3.284061in}}%
\pgfpathlineto{\pgfqpoint{3.442823in}{3.281112in}}%
\pgfpathmoveto{\pgfqpoint{3.442823in}{3.278163in}}%
\pgfpathlineto{\pgfqpoint{3.442823in}{3.278163in}}%
\pgfpathlineto{\pgfqpoint{3.442823in}{3.281112in}}%
\pgfpathlineto{\pgfqpoint{3.447364in}{3.281112in}}%
\pgfpathlineto{\pgfqpoint{3.447364in}{3.278163in}}%
\pgfpathmoveto{\pgfqpoint{3.442823in}{3.281112in}}%
\pgfpathlineto{\pgfqpoint{3.442823in}{3.281112in}}%
\pgfpathlineto{\pgfqpoint{3.442823in}{3.284061in}}%
\pgfpathlineto{\pgfqpoint{3.447364in}{3.284061in}}%
\pgfpathlineto{\pgfqpoint{3.447364in}{3.281112in}}%
\pgfpathmoveto{\pgfqpoint{3.447364in}{3.272264in}}%
\pgfpathlineto{\pgfqpoint{3.447364in}{3.272264in}}%
\pgfpathlineto{\pgfqpoint{3.447364in}{3.275213in}}%
\pgfpathlineto{\pgfqpoint{3.451905in}{3.275213in}}%
\pgfpathlineto{\pgfqpoint{3.451905in}{3.272264in}}%
\pgfpathmoveto{\pgfqpoint{3.447364in}{3.275213in}}%
\pgfpathlineto{\pgfqpoint{3.447364in}{3.275213in}}%
\pgfpathlineto{\pgfqpoint{3.447364in}{3.278163in}}%
\pgfpathlineto{\pgfqpoint{3.451905in}{3.278163in}}%
\pgfpathlineto{\pgfqpoint{3.451905in}{3.275213in}}%
\pgfpathmoveto{\pgfqpoint{3.451905in}{3.272264in}}%
\pgfpathlineto{\pgfqpoint{3.451905in}{3.272264in}}%
\pgfpathlineto{\pgfqpoint{3.451905in}{3.275213in}}%
\pgfpathlineto{\pgfqpoint{3.456447in}{3.275213in}}%
\pgfpathlineto{\pgfqpoint{3.456447in}{3.272264in}}%
\pgfpathmoveto{\pgfqpoint{3.451905in}{3.275213in}}%
\pgfpathlineto{\pgfqpoint{3.451905in}{3.275213in}}%
\pgfpathlineto{\pgfqpoint{3.451905in}{3.278163in}}%
\pgfpathlineto{\pgfqpoint{3.456447in}{3.278163in}}%
\pgfpathlineto{\pgfqpoint{3.456447in}{3.275213in}}%
\pgfpathmoveto{\pgfqpoint{3.447364in}{3.278163in}}%
\pgfpathlineto{\pgfqpoint{3.447364in}{3.278163in}}%
\pgfpathlineto{\pgfqpoint{3.447364in}{3.281112in}}%
\pgfpathlineto{\pgfqpoint{3.451905in}{3.281112in}}%
\pgfpathlineto{\pgfqpoint{3.451905in}{3.278163in}}%
\pgfpathmoveto{\pgfqpoint{3.456447in}{3.266366in}}%
\pgfpathlineto{\pgfqpoint{3.456447in}{3.266366in}}%
\pgfpathlineto{\pgfqpoint{3.456447in}{3.269315in}}%
\pgfpathlineto{\pgfqpoint{3.460988in}{3.269315in}}%
\pgfpathlineto{\pgfqpoint{3.460988in}{3.266366in}}%
\pgfpathmoveto{\pgfqpoint{3.456447in}{3.269315in}}%
\pgfpathlineto{\pgfqpoint{3.456447in}{3.269315in}}%
\pgfpathlineto{\pgfqpoint{3.456447in}{3.272264in}}%
\pgfpathlineto{\pgfqpoint{3.460988in}{3.272264in}}%
\pgfpathlineto{\pgfqpoint{3.460988in}{3.269315in}}%
\pgfpathmoveto{\pgfqpoint{3.460988in}{3.266366in}}%
\pgfpathlineto{\pgfqpoint{3.460988in}{3.266366in}}%
\pgfpathlineto{\pgfqpoint{3.460988in}{3.269315in}}%
\pgfpathlineto{\pgfqpoint{3.465529in}{3.269315in}}%
\pgfpathlineto{\pgfqpoint{3.465529in}{3.266366in}}%
\pgfpathmoveto{\pgfqpoint{3.460988in}{3.269315in}}%
\pgfpathlineto{\pgfqpoint{3.460988in}{3.269315in}}%
\pgfpathlineto{\pgfqpoint{3.460988in}{3.272264in}}%
\pgfpathlineto{\pgfqpoint{3.465529in}{3.272264in}}%
\pgfpathlineto{\pgfqpoint{3.465529in}{3.269315in}}%
\pgfpathmoveto{\pgfqpoint{3.465529in}{3.260468in}}%
\pgfpathlineto{\pgfqpoint{3.465529in}{3.260468in}}%
\pgfpathlineto{\pgfqpoint{3.465529in}{3.263417in}}%
\pgfpathlineto{\pgfqpoint{3.470070in}{3.263417in}}%
\pgfpathlineto{\pgfqpoint{3.470070in}{3.260468in}}%
\pgfpathmoveto{\pgfqpoint{3.465529in}{3.263417in}}%
\pgfpathlineto{\pgfqpoint{3.465529in}{3.263417in}}%
\pgfpathlineto{\pgfqpoint{3.465529in}{3.266366in}}%
\pgfpathlineto{\pgfqpoint{3.470070in}{3.266366in}}%
\pgfpathlineto{\pgfqpoint{3.470070in}{3.263417in}}%
\pgfpathmoveto{\pgfqpoint{3.470070in}{3.260468in}}%
\pgfpathlineto{\pgfqpoint{3.470070in}{3.260468in}}%
\pgfpathlineto{\pgfqpoint{3.470070in}{3.263417in}}%
\pgfpathlineto{\pgfqpoint{3.474612in}{3.263417in}}%
\pgfpathlineto{\pgfqpoint{3.474612in}{3.260468in}}%
\pgfpathmoveto{\pgfqpoint{3.465529in}{3.266366in}}%
\pgfpathlineto{\pgfqpoint{3.465529in}{3.266366in}}%
\pgfpathlineto{\pgfqpoint{3.465529in}{3.269315in}}%
\pgfpathlineto{\pgfqpoint{3.470070in}{3.269315in}}%
\pgfpathlineto{\pgfqpoint{3.470070in}{3.266366in}}%
\pgfpathmoveto{\pgfqpoint{3.456447in}{3.272264in}}%
\pgfpathlineto{\pgfqpoint{3.456447in}{3.272264in}}%
\pgfpathlineto{\pgfqpoint{3.456447in}{3.275213in}}%
\pgfpathlineto{\pgfqpoint{3.460988in}{3.275213in}}%
\pgfpathlineto{\pgfqpoint{3.460988in}{3.272264in}}%
\pgfpathmoveto{\pgfqpoint{3.483694in}{3.242773in}}%
\pgfpathlineto{\pgfqpoint{3.483694in}{3.242773in}}%
\pgfpathlineto{\pgfqpoint{3.483694in}{3.245722in}}%
\pgfpathlineto{\pgfqpoint{3.488235in}{3.245722in}}%
\pgfpathlineto{\pgfqpoint{3.488235in}{3.242773in}}%
\pgfpathmoveto{\pgfqpoint{3.483694in}{3.245722in}}%
\pgfpathlineto{\pgfqpoint{3.483694in}{3.245722in}}%
\pgfpathlineto{\pgfqpoint{3.483694in}{3.248671in}}%
\pgfpathlineto{\pgfqpoint{3.488235in}{3.248671in}}%
\pgfpathlineto{\pgfqpoint{3.488235in}{3.245722in}}%
\pgfpathmoveto{\pgfqpoint{3.488235in}{3.242773in}}%
\pgfpathlineto{\pgfqpoint{3.488235in}{3.242773in}}%
\pgfpathlineto{\pgfqpoint{3.488235in}{3.245722in}}%
\pgfpathlineto{\pgfqpoint{3.492777in}{3.245722in}}%
\pgfpathlineto{\pgfqpoint{3.492777in}{3.242773in}}%
\pgfpathmoveto{\pgfqpoint{3.488235in}{3.245722in}}%
\pgfpathlineto{\pgfqpoint{3.488235in}{3.245722in}}%
\pgfpathlineto{\pgfqpoint{3.488235in}{3.248671in}}%
\pgfpathlineto{\pgfqpoint{3.492777in}{3.248671in}}%
\pgfpathlineto{\pgfqpoint{3.492777in}{3.245722in}}%
\pgfpathmoveto{\pgfqpoint{3.474612in}{3.248671in}}%
\pgfpathlineto{\pgfqpoint{3.474612in}{3.248671in}}%
\pgfpathlineto{\pgfqpoint{3.474612in}{3.251620in}}%
\pgfpathlineto{\pgfqpoint{3.479153in}{3.251620in}}%
\pgfpathlineto{\pgfqpoint{3.479153in}{3.248671in}}%
\pgfpathmoveto{\pgfqpoint{3.474612in}{3.251620in}}%
\pgfpathlineto{\pgfqpoint{3.474612in}{3.251620in}}%
\pgfpathlineto{\pgfqpoint{3.474612in}{3.254569in}}%
\pgfpathlineto{\pgfqpoint{3.479153in}{3.254569in}}%
\pgfpathlineto{\pgfqpoint{3.479153in}{3.251620in}}%
\pgfpathmoveto{\pgfqpoint{3.479153in}{3.248671in}}%
\pgfpathlineto{\pgfqpoint{3.479153in}{3.248671in}}%
\pgfpathlineto{\pgfqpoint{3.479153in}{3.251620in}}%
\pgfpathlineto{\pgfqpoint{3.483694in}{3.251620in}}%
\pgfpathlineto{\pgfqpoint{3.483694in}{3.248671in}}%
\pgfpathmoveto{\pgfqpoint{3.479153in}{3.251620in}}%
\pgfpathlineto{\pgfqpoint{3.479153in}{3.251620in}}%
\pgfpathlineto{\pgfqpoint{3.479153in}{3.254569in}}%
\pgfpathlineto{\pgfqpoint{3.483694in}{3.254569in}}%
\pgfpathlineto{\pgfqpoint{3.483694in}{3.251620in}}%
\pgfpathmoveto{\pgfqpoint{3.474612in}{3.254569in}}%
\pgfpathlineto{\pgfqpoint{3.474612in}{3.254569in}}%
\pgfpathlineto{\pgfqpoint{3.474612in}{3.257518in}}%
\pgfpathlineto{\pgfqpoint{3.479153in}{3.257518in}}%
\pgfpathlineto{\pgfqpoint{3.479153in}{3.254569in}}%
\pgfpathmoveto{\pgfqpoint{3.474612in}{3.257518in}}%
\pgfpathlineto{\pgfqpoint{3.474612in}{3.257518in}}%
\pgfpathlineto{\pgfqpoint{3.474612in}{3.260468in}}%
\pgfpathlineto{\pgfqpoint{3.479153in}{3.260468in}}%
\pgfpathlineto{\pgfqpoint{3.479153in}{3.257518in}}%
\pgfpathmoveto{\pgfqpoint{3.479153in}{3.254569in}}%
\pgfpathlineto{\pgfqpoint{3.479153in}{3.254569in}}%
\pgfpathlineto{\pgfqpoint{3.479153in}{3.257518in}}%
\pgfpathlineto{\pgfqpoint{3.483694in}{3.257518in}}%
\pgfpathlineto{\pgfqpoint{3.483694in}{3.254569in}}%
\pgfpathmoveto{\pgfqpoint{3.483694in}{3.248671in}}%
\pgfpathlineto{\pgfqpoint{3.483694in}{3.248671in}}%
\pgfpathlineto{\pgfqpoint{3.483694in}{3.251620in}}%
\pgfpathlineto{\pgfqpoint{3.488235in}{3.251620in}}%
\pgfpathlineto{\pgfqpoint{3.488235in}{3.248671in}}%
\pgfpathmoveto{\pgfqpoint{3.483694in}{3.251620in}}%
\pgfpathlineto{\pgfqpoint{3.483694in}{3.251620in}}%
\pgfpathlineto{\pgfqpoint{3.483694in}{3.254569in}}%
\pgfpathlineto{\pgfqpoint{3.488235in}{3.254569in}}%
\pgfpathlineto{\pgfqpoint{3.488235in}{3.251620in}}%
\pgfpathmoveto{\pgfqpoint{3.488235in}{3.248671in}}%
\pgfpathlineto{\pgfqpoint{3.488235in}{3.248671in}}%
\pgfpathlineto{\pgfqpoint{3.488235in}{3.251620in}}%
\pgfpathlineto{\pgfqpoint{3.492777in}{3.251620in}}%
\pgfpathlineto{\pgfqpoint{3.492777in}{3.248671in}}%
\pgfpathmoveto{\pgfqpoint{3.492777in}{3.236874in}}%
\pgfpathlineto{\pgfqpoint{3.492777in}{3.236874in}}%
\pgfpathlineto{\pgfqpoint{3.492777in}{3.239823in}}%
\pgfpathlineto{\pgfqpoint{3.497318in}{3.239823in}}%
\pgfpathlineto{\pgfqpoint{3.497318in}{3.236874in}}%
\pgfpathmoveto{\pgfqpoint{3.492777in}{3.239823in}}%
\pgfpathlineto{\pgfqpoint{3.492777in}{3.239823in}}%
\pgfpathlineto{\pgfqpoint{3.492777in}{3.242773in}}%
\pgfpathlineto{\pgfqpoint{3.497318in}{3.242773in}}%
\pgfpathlineto{\pgfqpoint{3.497318in}{3.239823in}}%
\pgfpathmoveto{\pgfqpoint{3.497318in}{3.236874in}}%
\pgfpathlineto{\pgfqpoint{3.497318in}{3.236874in}}%
\pgfpathlineto{\pgfqpoint{3.497318in}{3.239823in}}%
\pgfpathlineto{\pgfqpoint{3.501859in}{3.239823in}}%
\pgfpathlineto{\pgfqpoint{3.501859in}{3.236874in}}%
\pgfpathmoveto{\pgfqpoint{3.497318in}{3.239823in}}%
\pgfpathlineto{\pgfqpoint{3.497318in}{3.239823in}}%
\pgfpathlineto{\pgfqpoint{3.497318in}{3.242773in}}%
\pgfpathlineto{\pgfqpoint{3.501859in}{3.242773in}}%
\pgfpathlineto{\pgfqpoint{3.501859in}{3.239823in}}%
\pgfpathmoveto{\pgfqpoint{3.492777in}{3.242773in}}%
\pgfpathlineto{\pgfqpoint{3.492777in}{3.242773in}}%
\pgfpathlineto{\pgfqpoint{3.492777in}{3.245722in}}%
\pgfpathlineto{\pgfqpoint{3.497318in}{3.245722in}}%
\pgfpathlineto{\pgfqpoint{3.497318in}{3.242773in}}%
\pgfpathmoveto{\pgfqpoint{3.492777in}{3.245722in}}%
\pgfpathlineto{\pgfqpoint{3.492777in}{3.245722in}}%
\pgfpathlineto{\pgfqpoint{3.492777in}{3.248671in}}%
\pgfpathlineto{\pgfqpoint{3.497318in}{3.248671in}}%
\pgfpathlineto{\pgfqpoint{3.497318in}{3.245722in}}%
\pgfpathmoveto{\pgfqpoint{3.497318in}{3.242773in}}%
\pgfpathlineto{\pgfqpoint{3.497318in}{3.242773in}}%
\pgfpathlineto{\pgfqpoint{3.497318in}{3.245722in}}%
\pgfpathlineto{\pgfqpoint{3.501859in}{3.245722in}}%
\pgfpathlineto{\pgfqpoint{3.501859in}{3.242773in}}%
\pgfpathmoveto{\pgfqpoint{3.501859in}{3.236874in}}%
\pgfpathlineto{\pgfqpoint{3.501859in}{3.236874in}}%
\pgfpathlineto{\pgfqpoint{3.501859in}{3.239823in}}%
\pgfpathlineto{\pgfqpoint{3.506401in}{3.239823in}}%
\pgfpathlineto{\pgfqpoint{3.506401in}{3.236874in}}%
\pgfpathmoveto{\pgfqpoint{3.501859in}{3.239823in}}%
\pgfpathlineto{\pgfqpoint{3.501859in}{3.239823in}}%
\pgfpathlineto{\pgfqpoint{3.501859in}{3.242773in}}%
\pgfpathlineto{\pgfqpoint{3.506401in}{3.242773in}}%
\pgfpathlineto{\pgfqpoint{3.506401in}{3.239823in}}%
\pgfpathmoveto{\pgfqpoint{3.506401in}{3.236874in}}%
\pgfpathlineto{\pgfqpoint{3.506401in}{3.236874in}}%
\pgfpathlineto{\pgfqpoint{3.506401in}{3.239823in}}%
\pgfpathlineto{\pgfqpoint{3.510942in}{3.239823in}}%
\pgfpathlineto{\pgfqpoint{3.510942in}{3.236874in}}%
\pgfpathmoveto{\pgfqpoint{3.438281in}{3.284061in}}%
\pgfpathlineto{\pgfqpoint{3.438281in}{3.284061in}}%
\pgfpathlineto{\pgfqpoint{3.438281in}{3.287010in}}%
\pgfpathlineto{\pgfqpoint{3.442823in}{3.287010in}}%
\pgfpathlineto{\pgfqpoint{3.442823in}{3.284061in}}%
\pgfpathmoveto{\pgfqpoint{3.365621in}{3.331248in}}%
\pgfpathlineto{\pgfqpoint{3.365621in}{3.331248in}}%
\pgfpathlineto{\pgfqpoint{3.365621in}{3.334197in}}%
\pgfpathlineto{\pgfqpoint{3.370162in}{3.334197in}}%
\pgfpathlineto{\pgfqpoint{3.370162in}{3.331248in}}%
\pgfpathmoveto{\pgfqpoint{3.551810in}{2.809238in}}%
\pgfpathlineto{\pgfqpoint{3.551810in}{2.809238in}}%
\pgfpathlineto{\pgfqpoint{3.551810in}{2.812187in}}%
\pgfpathlineto{\pgfqpoint{3.556351in}{2.812187in}}%
\pgfpathlineto{\pgfqpoint{3.556351in}{2.809238in}}%
\pgfpathmoveto{\pgfqpoint{3.556351in}{2.809238in}}%
\pgfpathlineto{\pgfqpoint{3.556351in}{2.809238in}}%
\pgfpathlineto{\pgfqpoint{3.556351in}{2.812187in}}%
\pgfpathlineto{\pgfqpoint{3.560892in}{2.812187in}}%
\pgfpathlineto{\pgfqpoint{3.560892in}{2.809238in}}%
\pgfpathmoveto{\pgfqpoint{3.560892in}{2.809238in}}%
\pgfpathlineto{\pgfqpoint{3.560892in}{2.809238in}}%
\pgfpathlineto{\pgfqpoint{3.560892in}{2.812187in}}%
\pgfpathlineto{\pgfqpoint{3.565433in}{2.812187in}}%
\pgfpathlineto{\pgfqpoint{3.565433in}{2.809238in}}%
\pgfpathmoveto{\pgfqpoint{3.565433in}{2.806288in}}%
\pgfpathlineto{\pgfqpoint{3.565433in}{2.806288in}}%
\pgfpathlineto{\pgfqpoint{3.565433in}{2.809238in}}%
\pgfpathlineto{\pgfqpoint{3.569974in}{2.809238in}}%
\pgfpathlineto{\pgfqpoint{3.569974in}{2.806288in}}%
\pgfpathmoveto{\pgfqpoint{3.565433in}{2.809238in}}%
\pgfpathlineto{\pgfqpoint{3.565433in}{2.809238in}}%
\pgfpathlineto{\pgfqpoint{3.565433in}{2.812187in}}%
\pgfpathlineto{\pgfqpoint{3.569974in}{2.812187in}}%
\pgfpathlineto{\pgfqpoint{3.569974in}{2.809238in}}%
\pgfpathmoveto{\pgfqpoint{3.569974in}{2.806288in}}%
\pgfpathlineto{\pgfqpoint{3.569974in}{2.806288in}}%
\pgfpathlineto{\pgfqpoint{3.569974in}{2.809238in}}%
\pgfpathlineto{\pgfqpoint{3.574515in}{2.809238in}}%
\pgfpathlineto{\pgfqpoint{3.574515in}{2.806288in}}%
\pgfpathmoveto{\pgfqpoint{3.569974in}{2.809238in}}%
\pgfpathlineto{\pgfqpoint{3.569974in}{2.809238in}}%
\pgfpathlineto{\pgfqpoint{3.569974in}{2.812187in}}%
\pgfpathlineto{\pgfqpoint{3.574515in}{2.812187in}}%
\pgfpathlineto{\pgfqpoint{3.574515in}{2.809238in}}%
\pgfpathmoveto{\pgfqpoint{3.579056in}{2.803339in}}%
\pgfpathlineto{\pgfqpoint{3.579056in}{2.803339in}}%
\pgfpathlineto{\pgfqpoint{3.579056in}{2.806288in}}%
\pgfpathlineto{\pgfqpoint{3.583597in}{2.806288in}}%
\pgfpathlineto{\pgfqpoint{3.583597in}{2.803339in}}%
\pgfpathmoveto{\pgfqpoint{3.574515in}{2.806288in}}%
\pgfpathlineto{\pgfqpoint{3.574515in}{2.806288in}}%
\pgfpathlineto{\pgfqpoint{3.574515in}{2.809238in}}%
\pgfpathlineto{\pgfqpoint{3.579056in}{2.809238in}}%
\pgfpathlineto{\pgfqpoint{3.579056in}{2.806288in}}%
\pgfpathmoveto{\pgfqpoint{3.574515in}{2.809238in}}%
\pgfpathlineto{\pgfqpoint{3.574515in}{2.809238in}}%
\pgfpathlineto{\pgfqpoint{3.574515in}{2.812187in}}%
\pgfpathlineto{\pgfqpoint{3.579056in}{2.812187in}}%
\pgfpathlineto{\pgfqpoint{3.579056in}{2.809238in}}%
\pgfpathmoveto{\pgfqpoint{3.579056in}{2.806288in}}%
\pgfpathlineto{\pgfqpoint{3.579056in}{2.806288in}}%
\pgfpathlineto{\pgfqpoint{3.579056in}{2.809238in}}%
\pgfpathlineto{\pgfqpoint{3.583597in}{2.809238in}}%
\pgfpathlineto{\pgfqpoint{3.583597in}{2.806288in}}%
\pgfpathmoveto{\pgfqpoint{3.579056in}{2.809238in}}%
\pgfpathlineto{\pgfqpoint{3.579056in}{2.809238in}}%
\pgfpathlineto{\pgfqpoint{3.579056in}{2.812187in}}%
\pgfpathlineto{\pgfqpoint{3.583597in}{2.812187in}}%
\pgfpathlineto{\pgfqpoint{3.583597in}{2.809238in}}%
\pgfpathmoveto{\pgfqpoint{3.510942in}{2.818085in}}%
\pgfpathlineto{\pgfqpoint{3.510942in}{2.818085in}}%
\pgfpathlineto{\pgfqpoint{3.510942in}{2.821035in}}%
\pgfpathlineto{\pgfqpoint{3.515483in}{2.821035in}}%
\pgfpathlineto{\pgfqpoint{3.515483in}{2.818085in}}%
\pgfpathmoveto{\pgfqpoint{3.510942in}{2.821035in}}%
\pgfpathlineto{\pgfqpoint{3.510942in}{2.821035in}}%
\pgfpathlineto{\pgfqpoint{3.510942in}{2.823984in}}%
\pgfpathlineto{\pgfqpoint{3.515483in}{2.823984in}}%
\pgfpathlineto{\pgfqpoint{3.515483in}{2.821035in}}%
\pgfpathmoveto{\pgfqpoint{3.515483in}{2.818085in}}%
\pgfpathlineto{\pgfqpoint{3.515483in}{2.818085in}}%
\pgfpathlineto{\pgfqpoint{3.515483in}{2.821035in}}%
\pgfpathlineto{\pgfqpoint{3.520024in}{2.821035in}}%
\pgfpathlineto{\pgfqpoint{3.520024in}{2.818085in}}%
\pgfpathmoveto{\pgfqpoint{3.515483in}{2.821035in}}%
\pgfpathlineto{\pgfqpoint{3.515483in}{2.821035in}}%
\pgfpathlineto{\pgfqpoint{3.515483in}{2.823984in}}%
\pgfpathlineto{\pgfqpoint{3.520024in}{2.823984in}}%
\pgfpathlineto{\pgfqpoint{3.520024in}{2.821035in}}%
\pgfpathmoveto{\pgfqpoint{3.524565in}{2.815136in}}%
\pgfpathlineto{\pgfqpoint{3.524565in}{2.815136in}}%
\pgfpathlineto{\pgfqpoint{3.524565in}{2.818085in}}%
\pgfpathlineto{\pgfqpoint{3.529106in}{2.818085in}}%
\pgfpathlineto{\pgfqpoint{3.529106in}{2.815136in}}%
\pgfpathmoveto{\pgfqpoint{3.520024in}{2.818085in}}%
\pgfpathlineto{\pgfqpoint{3.520024in}{2.818085in}}%
\pgfpathlineto{\pgfqpoint{3.520024in}{2.821035in}}%
\pgfpathlineto{\pgfqpoint{3.524565in}{2.821035in}}%
\pgfpathlineto{\pgfqpoint{3.524565in}{2.818085in}}%
\pgfpathmoveto{\pgfqpoint{3.520024in}{2.821035in}}%
\pgfpathlineto{\pgfqpoint{3.520024in}{2.821035in}}%
\pgfpathlineto{\pgfqpoint{3.520024in}{2.823984in}}%
\pgfpathlineto{\pgfqpoint{3.524565in}{2.823984in}}%
\pgfpathlineto{\pgfqpoint{3.524565in}{2.821035in}}%
\pgfpathmoveto{\pgfqpoint{3.524565in}{2.818085in}}%
\pgfpathlineto{\pgfqpoint{3.524565in}{2.818085in}}%
\pgfpathlineto{\pgfqpoint{3.524565in}{2.821035in}}%
\pgfpathlineto{\pgfqpoint{3.529106in}{2.821035in}}%
\pgfpathlineto{\pgfqpoint{3.529106in}{2.818085in}}%
\pgfpathmoveto{\pgfqpoint{3.524565in}{2.821035in}}%
\pgfpathlineto{\pgfqpoint{3.524565in}{2.821035in}}%
\pgfpathlineto{\pgfqpoint{3.524565in}{2.823984in}}%
\pgfpathlineto{\pgfqpoint{3.529106in}{2.823984in}}%
\pgfpathlineto{\pgfqpoint{3.529106in}{2.821035in}}%
\pgfpathmoveto{\pgfqpoint{3.529106in}{2.815136in}}%
\pgfpathlineto{\pgfqpoint{3.529106in}{2.815136in}}%
\pgfpathlineto{\pgfqpoint{3.529106in}{2.818085in}}%
\pgfpathlineto{\pgfqpoint{3.533647in}{2.818085in}}%
\pgfpathlineto{\pgfqpoint{3.533647in}{2.815136in}}%
\pgfpathmoveto{\pgfqpoint{3.533647in}{2.815136in}}%
\pgfpathlineto{\pgfqpoint{3.533647in}{2.815136in}}%
\pgfpathlineto{\pgfqpoint{3.533647in}{2.818085in}}%
\pgfpathlineto{\pgfqpoint{3.538188in}{2.818085in}}%
\pgfpathlineto{\pgfqpoint{3.538188in}{2.815136in}}%
\pgfpathmoveto{\pgfqpoint{3.538188in}{2.812187in}}%
\pgfpathlineto{\pgfqpoint{3.538188in}{2.812187in}}%
\pgfpathlineto{\pgfqpoint{3.538188in}{2.815136in}}%
\pgfpathlineto{\pgfqpoint{3.542729in}{2.815136in}}%
\pgfpathlineto{\pgfqpoint{3.542729in}{2.812187in}}%
\pgfpathmoveto{\pgfqpoint{3.538188in}{2.815136in}}%
\pgfpathlineto{\pgfqpoint{3.538188in}{2.815136in}}%
\pgfpathlineto{\pgfqpoint{3.538188in}{2.818085in}}%
\pgfpathlineto{\pgfqpoint{3.542729in}{2.818085in}}%
\pgfpathlineto{\pgfqpoint{3.542729in}{2.815136in}}%
\pgfpathmoveto{\pgfqpoint{3.542729in}{2.812187in}}%
\pgfpathlineto{\pgfqpoint{3.542729in}{2.812187in}}%
\pgfpathlineto{\pgfqpoint{3.542729in}{2.815136in}}%
\pgfpathlineto{\pgfqpoint{3.547269in}{2.815136in}}%
\pgfpathlineto{\pgfqpoint{3.547269in}{2.812187in}}%
\pgfpathmoveto{\pgfqpoint{3.542729in}{2.815136in}}%
\pgfpathlineto{\pgfqpoint{3.542729in}{2.815136in}}%
\pgfpathlineto{\pgfqpoint{3.542729in}{2.818085in}}%
\pgfpathlineto{\pgfqpoint{3.547269in}{2.818085in}}%
\pgfpathlineto{\pgfqpoint{3.547269in}{2.815136in}}%
\pgfpathmoveto{\pgfqpoint{3.547269in}{2.812187in}}%
\pgfpathlineto{\pgfqpoint{3.547269in}{2.812187in}}%
\pgfpathlineto{\pgfqpoint{3.547269in}{2.815136in}}%
\pgfpathlineto{\pgfqpoint{3.551810in}{2.815136in}}%
\pgfpathlineto{\pgfqpoint{3.551810in}{2.812187in}}%
\pgfpathmoveto{\pgfqpoint{3.547269in}{2.815136in}}%
\pgfpathlineto{\pgfqpoint{3.547269in}{2.815136in}}%
\pgfpathlineto{\pgfqpoint{3.547269in}{2.818085in}}%
\pgfpathlineto{\pgfqpoint{3.551810in}{2.818085in}}%
\pgfpathlineto{\pgfqpoint{3.551810in}{2.815136in}}%
\pgfpathmoveto{\pgfqpoint{3.551810in}{2.812187in}}%
\pgfpathlineto{\pgfqpoint{3.551810in}{2.812187in}}%
\pgfpathlineto{\pgfqpoint{3.551810in}{2.815136in}}%
\pgfpathlineto{\pgfqpoint{3.556351in}{2.815136in}}%
\pgfpathlineto{\pgfqpoint{3.556351in}{2.812187in}}%
\pgfpathmoveto{\pgfqpoint{3.551810in}{2.815136in}}%
\pgfpathlineto{\pgfqpoint{3.551810in}{2.815136in}}%
\pgfpathlineto{\pgfqpoint{3.551810in}{2.818085in}}%
\pgfpathlineto{\pgfqpoint{3.556351in}{2.818085in}}%
\pgfpathlineto{\pgfqpoint{3.556351in}{2.815136in}}%
\pgfpathmoveto{\pgfqpoint{3.583597in}{2.803339in}}%
\pgfpathlineto{\pgfqpoint{3.583597in}{2.803339in}}%
\pgfpathlineto{\pgfqpoint{3.583597in}{2.806288in}}%
\pgfpathlineto{\pgfqpoint{3.588138in}{2.806288in}}%
\pgfpathlineto{\pgfqpoint{3.588138in}{2.803339in}}%
\pgfpathmoveto{\pgfqpoint{3.588138in}{2.803339in}}%
\pgfpathlineto{\pgfqpoint{3.588138in}{2.803339in}}%
\pgfpathlineto{\pgfqpoint{3.588138in}{2.806288in}}%
\pgfpathlineto{\pgfqpoint{3.592679in}{2.806288in}}%
\pgfpathlineto{\pgfqpoint{3.592679in}{2.803339in}}%
\pgfpathmoveto{\pgfqpoint{3.592679in}{2.800390in}}%
\pgfpathlineto{\pgfqpoint{3.592679in}{2.800390in}}%
\pgfpathlineto{\pgfqpoint{3.592679in}{2.803339in}}%
\pgfpathlineto{\pgfqpoint{3.597220in}{2.803339in}}%
\pgfpathlineto{\pgfqpoint{3.597220in}{2.800390in}}%
\pgfpathmoveto{\pgfqpoint{3.592679in}{2.803339in}}%
\pgfpathlineto{\pgfqpoint{3.592679in}{2.803339in}}%
\pgfpathlineto{\pgfqpoint{3.592679in}{2.806288in}}%
\pgfpathlineto{\pgfqpoint{3.597220in}{2.806288in}}%
\pgfpathlineto{\pgfqpoint{3.597220in}{2.803339in}}%
\pgfpathmoveto{\pgfqpoint{3.597220in}{2.800390in}}%
\pgfpathlineto{\pgfqpoint{3.597220in}{2.800390in}}%
\pgfpathlineto{\pgfqpoint{3.597220in}{2.803339in}}%
\pgfpathlineto{\pgfqpoint{3.601761in}{2.803339in}}%
\pgfpathlineto{\pgfqpoint{3.601761in}{2.800390in}}%
\pgfpathmoveto{\pgfqpoint{3.597220in}{2.803339in}}%
\pgfpathlineto{\pgfqpoint{3.597220in}{2.803339in}}%
\pgfpathlineto{\pgfqpoint{3.597220in}{2.806288in}}%
\pgfpathlineto{\pgfqpoint{3.601761in}{2.806288in}}%
\pgfpathlineto{\pgfqpoint{3.601761in}{2.803339in}}%
\pgfpathmoveto{\pgfqpoint{3.606302in}{2.797440in}}%
\pgfpathlineto{\pgfqpoint{3.606302in}{2.797440in}}%
\pgfpathlineto{\pgfqpoint{3.606302in}{2.800390in}}%
\pgfpathlineto{\pgfqpoint{3.610843in}{2.800390in}}%
\pgfpathlineto{\pgfqpoint{3.610843in}{2.797440in}}%
\pgfpathmoveto{\pgfqpoint{3.610843in}{2.797440in}}%
\pgfpathlineto{\pgfqpoint{3.610843in}{2.797440in}}%
\pgfpathlineto{\pgfqpoint{3.610843in}{2.800390in}}%
\pgfpathlineto{\pgfqpoint{3.615384in}{2.800390in}}%
\pgfpathlineto{\pgfqpoint{3.615384in}{2.797440in}}%
\pgfpathmoveto{\pgfqpoint{3.615384in}{2.797440in}}%
\pgfpathlineto{\pgfqpoint{3.615384in}{2.797440in}}%
\pgfpathlineto{\pgfqpoint{3.615384in}{2.800390in}}%
\pgfpathlineto{\pgfqpoint{3.619925in}{2.800390in}}%
\pgfpathlineto{\pgfqpoint{3.619925in}{2.797440in}}%
\pgfpathmoveto{\pgfqpoint{3.601761in}{2.800390in}}%
\pgfpathlineto{\pgfqpoint{3.601761in}{2.800390in}}%
\pgfpathlineto{\pgfqpoint{3.601761in}{2.803339in}}%
\pgfpathlineto{\pgfqpoint{3.606302in}{2.803339in}}%
\pgfpathlineto{\pgfqpoint{3.606302in}{2.800390in}}%
\pgfpathmoveto{\pgfqpoint{3.601761in}{2.803339in}}%
\pgfpathlineto{\pgfqpoint{3.601761in}{2.803339in}}%
\pgfpathlineto{\pgfqpoint{3.601761in}{2.806288in}}%
\pgfpathlineto{\pgfqpoint{3.606302in}{2.806288in}}%
\pgfpathlineto{\pgfqpoint{3.606302in}{2.803339in}}%
\pgfpathmoveto{\pgfqpoint{3.606302in}{2.800390in}}%
\pgfpathlineto{\pgfqpoint{3.606302in}{2.800390in}}%
\pgfpathlineto{\pgfqpoint{3.606302in}{2.803339in}}%
\pgfpathlineto{\pgfqpoint{3.610843in}{2.803339in}}%
\pgfpathlineto{\pgfqpoint{3.610843in}{2.800390in}}%
\pgfpathmoveto{\pgfqpoint{3.606302in}{2.803339in}}%
\pgfpathlineto{\pgfqpoint{3.606302in}{2.803339in}}%
\pgfpathlineto{\pgfqpoint{3.606302in}{2.806288in}}%
\pgfpathlineto{\pgfqpoint{3.610843in}{2.806288in}}%
\pgfpathlineto{\pgfqpoint{3.610843in}{2.803339in}}%
\pgfpathmoveto{\pgfqpoint{3.619925in}{2.794491in}}%
\pgfpathlineto{\pgfqpoint{3.619925in}{2.794491in}}%
\pgfpathlineto{\pgfqpoint{3.619925in}{2.797440in}}%
\pgfpathlineto{\pgfqpoint{3.624466in}{2.797440in}}%
\pgfpathlineto{\pgfqpoint{3.624466in}{2.794491in}}%
\pgfpathmoveto{\pgfqpoint{3.619925in}{2.797440in}}%
\pgfpathlineto{\pgfqpoint{3.619925in}{2.797440in}}%
\pgfpathlineto{\pgfqpoint{3.619925in}{2.800390in}}%
\pgfpathlineto{\pgfqpoint{3.624466in}{2.800390in}}%
\pgfpathlineto{\pgfqpoint{3.624466in}{2.797440in}}%
\pgfpathmoveto{\pgfqpoint{3.624466in}{2.794491in}}%
\pgfpathlineto{\pgfqpoint{3.624466in}{2.794491in}}%
\pgfpathlineto{\pgfqpoint{3.624466in}{2.797440in}}%
\pgfpathlineto{\pgfqpoint{3.629007in}{2.797440in}}%
\pgfpathlineto{\pgfqpoint{3.629007in}{2.794491in}}%
\pgfpathmoveto{\pgfqpoint{3.624466in}{2.797440in}}%
\pgfpathlineto{\pgfqpoint{3.624466in}{2.797440in}}%
\pgfpathlineto{\pgfqpoint{3.624466in}{2.800390in}}%
\pgfpathlineto{\pgfqpoint{3.629007in}{2.800390in}}%
\pgfpathlineto{\pgfqpoint{3.629007in}{2.797440in}}%
\pgfpathmoveto{\pgfqpoint{3.633548in}{2.791542in}}%
\pgfpathlineto{\pgfqpoint{3.633548in}{2.791542in}}%
\pgfpathlineto{\pgfqpoint{3.633548in}{2.794491in}}%
\pgfpathlineto{\pgfqpoint{3.638089in}{2.794491in}}%
\pgfpathlineto{\pgfqpoint{3.638089in}{2.791542in}}%
\pgfpathmoveto{\pgfqpoint{3.629007in}{2.794491in}}%
\pgfpathlineto{\pgfqpoint{3.629007in}{2.794491in}}%
\pgfpathlineto{\pgfqpoint{3.629007in}{2.797440in}}%
\pgfpathlineto{\pgfqpoint{3.633548in}{2.797440in}}%
\pgfpathlineto{\pgfqpoint{3.633548in}{2.794491in}}%
\pgfpathmoveto{\pgfqpoint{3.629007in}{2.797440in}}%
\pgfpathlineto{\pgfqpoint{3.629007in}{2.797440in}}%
\pgfpathlineto{\pgfqpoint{3.629007in}{2.800390in}}%
\pgfpathlineto{\pgfqpoint{3.633548in}{2.800390in}}%
\pgfpathlineto{\pgfqpoint{3.633548in}{2.797440in}}%
\pgfpathmoveto{\pgfqpoint{3.633548in}{2.794491in}}%
\pgfpathlineto{\pgfqpoint{3.633548in}{2.794491in}}%
\pgfpathlineto{\pgfqpoint{3.633548in}{2.797440in}}%
\pgfpathlineto{\pgfqpoint{3.638089in}{2.797440in}}%
\pgfpathlineto{\pgfqpoint{3.638089in}{2.794491in}}%
\pgfpathmoveto{\pgfqpoint{3.633548in}{2.797440in}}%
\pgfpathlineto{\pgfqpoint{3.633548in}{2.797440in}}%
\pgfpathlineto{\pgfqpoint{3.633548in}{2.800390in}}%
\pgfpathlineto{\pgfqpoint{3.638089in}{2.800390in}}%
\pgfpathlineto{\pgfqpoint{3.638089in}{2.797440in}}%
\pgfpathmoveto{\pgfqpoint{3.638089in}{2.791542in}}%
\pgfpathlineto{\pgfqpoint{3.638089in}{2.791542in}}%
\pgfpathlineto{\pgfqpoint{3.638089in}{2.794491in}}%
\pgfpathlineto{\pgfqpoint{3.642630in}{2.794491in}}%
\pgfpathlineto{\pgfqpoint{3.642630in}{2.791542in}}%
\pgfpathmoveto{\pgfqpoint{3.642630in}{2.791542in}}%
\pgfpathlineto{\pgfqpoint{3.642630in}{2.791542in}}%
\pgfpathlineto{\pgfqpoint{3.642630in}{2.794491in}}%
\pgfpathlineto{\pgfqpoint{3.647171in}{2.794491in}}%
\pgfpathlineto{\pgfqpoint{3.647171in}{2.791542in}}%
\pgfpathmoveto{\pgfqpoint{3.647171in}{2.788592in}}%
\pgfpathlineto{\pgfqpoint{3.647171in}{2.788592in}}%
\pgfpathlineto{\pgfqpoint{3.647171in}{2.791542in}}%
\pgfpathlineto{\pgfqpoint{3.651712in}{2.791542in}}%
\pgfpathlineto{\pgfqpoint{3.651712in}{2.788592in}}%
\pgfpathmoveto{\pgfqpoint{3.647171in}{2.791542in}}%
\pgfpathlineto{\pgfqpoint{3.647171in}{2.791542in}}%
\pgfpathlineto{\pgfqpoint{3.647171in}{2.794491in}}%
\pgfpathlineto{\pgfqpoint{3.651712in}{2.794491in}}%
\pgfpathlineto{\pgfqpoint{3.651712in}{2.791542in}}%
\pgfpathmoveto{\pgfqpoint{3.651712in}{2.788592in}}%
\pgfpathlineto{\pgfqpoint{3.651712in}{2.788592in}}%
\pgfpathlineto{\pgfqpoint{3.651712in}{2.791542in}}%
\pgfpathlineto{\pgfqpoint{3.656252in}{2.791542in}}%
\pgfpathlineto{\pgfqpoint{3.656252in}{2.788592in}}%
\pgfpathmoveto{\pgfqpoint{3.651712in}{2.791542in}}%
\pgfpathlineto{\pgfqpoint{3.651712in}{2.791542in}}%
\pgfpathlineto{\pgfqpoint{3.651712in}{2.794491in}}%
\pgfpathlineto{\pgfqpoint{3.656252in}{2.794491in}}%
\pgfpathlineto{\pgfqpoint{3.656252in}{2.791542in}}%
\pgfpathmoveto{\pgfqpoint{3.647171in}{3.136601in}}%
\pgfpathlineto{\pgfqpoint{3.647171in}{3.136601in}}%
\pgfpathlineto{\pgfqpoint{3.647171in}{3.139550in}}%
\pgfpathlineto{\pgfqpoint{3.651712in}{3.139550in}}%
\pgfpathlineto{\pgfqpoint{3.651712in}{3.136601in}}%
\pgfpathmoveto{\pgfqpoint{3.647171in}{3.139550in}}%
\pgfpathlineto{\pgfqpoint{3.647171in}{3.139550in}}%
\pgfpathlineto{\pgfqpoint{3.647171in}{3.142499in}}%
\pgfpathlineto{\pgfqpoint{3.651712in}{3.142499in}}%
\pgfpathlineto{\pgfqpoint{3.651712in}{3.139550in}}%
\pgfpathmoveto{\pgfqpoint{3.651712in}{3.136601in}}%
\pgfpathlineto{\pgfqpoint{3.651712in}{3.136601in}}%
\pgfpathlineto{\pgfqpoint{3.651712in}{3.139550in}}%
\pgfpathlineto{\pgfqpoint{3.656252in}{3.139550in}}%
\pgfpathlineto{\pgfqpoint{3.656252in}{3.136601in}}%
\pgfpathmoveto{\pgfqpoint{3.651712in}{3.139550in}}%
\pgfpathlineto{\pgfqpoint{3.651712in}{3.139550in}}%
\pgfpathlineto{\pgfqpoint{3.651712in}{3.142499in}}%
\pgfpathlineto{\pgfqpoint{3.656252in}{3.142499in}}%
\pgfpathlineto{\pgfqpoint{3.656252in}{3.139550in}}%
\pgfpathmoveto{\pgfqpoint{3.574515in}{3.183788in}}%
\pgfpathlineto{\pgfqpoint{3.574515in}{3.183788in}}%
\pgfpathlineto{\pgfqpoint{3.574515in}{3.186738in}}%
\pgfpathlineto{\pgfqpoint{3.579056in}{3.186738in}}%
\pgfpathlineto{\pgfqpoint{3.579056in}{3.183788in}}%
\pgfpathmoveto{\pgfqpoint{3.574515in}{3.186738in}}%
\pgfpathlineto{\pgfqpoint{3.574515in}{3.186738in}}%
\pgfpathlineto{\pgfqpoint{3.574515in}{3.189687in}}%
\pgfpathlineto{\pgfqpoint{3.579056in}{3.189687in}}%
\pgfpathlineto{\pgfqpoint{3.579056in}{3.186738in}}%
\pgfpathmoveto{\pgfqpoint{3.579056in}{3.183788in}}%
\pgfpathlineto{\pgfqpoint{3.579056in}{3.183788in}}%
\pgfpathlineto{\pgfqpoint{3.579056in}{3.186738in}}%
\pgfpathlineto{\pgfqpoint{3.583597in}{3.186738in}}%
\pgfpathlineto{\pgfqpoint{3.583597in}{3.183788in}}%
\pgfpathmoveto{\pgfqpoint{3.579056in}{3.186738in}}%
\pgfpathlineto{\pgfqpoint{3.579056in}{3.186738in}}%
\pgfpathlineto{\pgfqpoint{3.579056in}{3.189687in}}%
\pgfpathlineto{\pgfqpoint{3.583597in}{3.189687in}}%
\pgfpathlineto{\pgfqpoint{3.583597in}{3.186738in}}%
\pgfpathmoveto{\pgfqpoint{3.538188in}{3.207382in}}%
\pgfpathlineto{\pgfqpoint{3.538188in}{3.207382in}}%
\pgfpathlineto{\pgfqpoint{3.538188in}{3.210331in}}%
\pgfpathlineto{\pgfqpoint{3.542729in}{3.210331in}}%
\pgfpathlineto{\pgfqpoint{3.542729in}{3.207382in}}%
\pgfpathmoveto{\pgfqpoint{3.538188in}{3.210331in}}%
\pgfpathlineto{\pgfqpoint{3.538188in}{3.210331in}}%
\pgfpathlineto{\pgfqpoint{3.538188in}{3.213281in}}%
\pgfpathlineto{\pgfqpoint{3.542729in}{3.213281in}}%
\pgfpathlineto{\pgfqpoint{3.542729in}{3.210331in}}%
\pgfpathmoveto{\pgfqpoint{3.542729in}{3.207382in}}%
\pgfpathlineto{\pgfqpoint{3.542729in}{3.207382in}}%
\pgfpathlineto{\pgfqpoint{3.542729in}{3.210331in}}%
\pgfpathlineto{\pgfqpoint{3.547269in}{3.210331in}}%
\pgfpathlineto{\pgfqpoint{3.547269in}{3.207382in}}%
\pgfpathmoveto{\pgfqpoint{3.542729in}{3.210331in}}%
\pgfpathlineto{\pgfqpoint{3.542729in}{3.210331in}}%
\pgfpathlineto{\pgfqpoint{3.542729in}{3.213281in}}%
\pgfpathlineto{\pgfqpoint{3.547269in}{3.213281in}}%
\pgfpathlineto{\pgfqpoint{3.547269in}{3.210331in}}%
\pgfpathmoveto{\pgfqpoint{3.520024in}{3.219179in}}%
\pgfpathlineto{\pgfqpoint{3.520024in}{3.219179in}}%
\pgfpathlineto{\pgfqpoint{3.520024in}{3.222128in}}%
\pgfpathlineto{\pgfqpoint{3.524565in}{3.222128in}}%
\pgfpathlineto{\pgfqpoint{3.524565in}{3.219179in}}%
\pgfpathmoveto{\pgfqpoint{3.520024in}{3.222128in}}%
\pgfpathlineto{\pgfqpoint{3.520024in}{3.222128in}}%
\pgfpathlineto{\pgfqpoint{3.520024in}{3.225077in}}%
\pgfpathlineto{\pgfqpoint{3.524565in}{3.225077in}}%
\pgfpathlineto{\pgfqpoint{3.524565in}{3.222128in}}%
\pgfpathmoveto{\pgfqpoint{3.524565in}{3.219179in}}%
\pgfpathlineto{\pgfqpoint{3.524565in}{3.219179in}}%
\pgfpathlineto{\pgfqpoint{3.524565in}{3.222128in}}%
\pgfpathlineto{\pgfqpoint{3.529106in}{3.222128in}}%
\pgfpathlineto{\pgfqpoint{3.529106in}{3.219179in}}%
\pgfpathmoveto{\pgfqpoint{3.524565in}{3.222128in}}%
\pgfpathlineto{\pgfqpoint{3.524565in}{3.222128in}}%
\pgfpathlineto{\pgfqpoint{3.524565in}{3.225077in}}%
\pgfpathlineto{\pgfqpoint{3.529106in}{3.225077in}}%
\pgfpathlineto{\pgfqpoint{3.529106in}{3.222128in}}%
\pgfpathmoveto{\pgfqpoint{3.510942in}{3.225077in}}%
\pgfpathlineto{\pgfqpoint{3.510942in}{3.225077in}}%
\pgfpathlineto{\pgfqpoint{3.510942in}{3.228027in}}%
\pgfpathlineto{\pgfqpoint{3.515483in}{3.228027in}}%
\pgfpathlineto{\pgfqpoint{3.515483in}{3.225077in}}%
\pgfpathmoveto{\pgfqpoint{3.510942in}{3.228027in}}%
\pgfpathlineto{\pgfqpoint{3.510942in}{3.228027in}}%
\pgfpathlineto{\pgfqpoint{3.510942in}{3.230976in}}%
\pgfpathlineto{\pgfqpoint{3.515483in}{3.230976in}}%
\pgfpathlineto{\pgfqpoint{3.515483in}{3.228027in}}%
\pgfpathmoveto{\pgfqpoint{3.515483in}{3.225077in}}%
\pgfpathlineto{\pgfqpoint{3.515483in}{3.225077in}}%
\pgfpathlineto{\pgfqpoint{3.515483in}{3.228027in}}%
\pgfpathlineto{\pgfqpoint{3.520024in}{3.228027in}}%
\pgfpathlineto{\pgfqpoint{3.520024in}{3.225077in}}%
\pgfpathmoveto{\pgfqpoint{3.515483in}{3.228027in}}%
\pgfpathlineto{\pgfqpoint{3.515483in}{3.228027in}}%
\pgfpathlineto{\pgfqpoint{3.515483in}{3.230976in}}%
\pgfpathlineto{\pgfqpoint{3.520024in}{3.230976in}}%
\pgfpathlineto{\pgfqpoint{3.520024in}{3.228027in}}%
\pgfpathmoveto{\pgfqpoint{3.510942in}{3.230976in}}%
\pgfpathlineto{\pgfqpoint{3.510942in}{3.230976in}}%
\pgfpathlineto{\pgfqpoint{3.510942in}{3.233925in}}%
\pgfpathlineto{\pgfqpoint{3.515483in}{3.233925in}}%
\pgfpathlineto{\pgfqpoint{3.515483in}{3.230976in}}%
\pgfpathmoveto{\pgfqpoint{3.510942in}{3.233925in}}%
\pgfpathlineto{\pgfqpoint{3.510942in}{3.233925in}}%
\pgfpathlineto{\pgfqpoint{3.510942in}{3.236874in}}%
\pgfpathlineto{\pgfqpoint{3.515483in}{3.236874in}}%
\pgfpathlineto{\pgfqpoint{3.515483in}{3.233925in}}%
\pgfpathmoveto{\pgfqpoint{3.515483in}{3.230976in}}%
\pgfpathlineto{\pgfqpoint{3.515483in}{3.230976in}}%
\pgfpathlineto{\pgfqpoint{3.515483in}{3.233925in}}%
\pgfpathlineto{\pgfqpoint{3.520024in}{3.233925in}}%
\pgfpathlineto{\pgfqpoint{3.520024in}{3.230976in}}%
\pgfpathmoveto{\pgfqpoint{3.520024in}{3.225077in}}%
\pgfpathlineto{\pgfqpoint{3.520024in}{3.225077in}}%
\pgfpathlineto{\pgfqpoint{3.520024in}{3.228027in}}%
\pgfpathlineto{\pgfqpoint{3.524565in}{3.228027in}}%
\pgfpathlineto{\pgfqpoint{3.524565in}{3.225077in}}%
\pgfpathmoveto{\pgfqpoint{3.520024in}{3.228027in}}%
\pgfpathlineto{\pgfqpoint{3.520024in}{3.228027in}}%
\pgfpathlineto{\pgfqpoint{3.520024in}{3.230976in}}%
\pgfpathlineto{\pgfqpoint{3.524565in}{3.230976in}}%
\pgfpathlineto{\pgfqpoint{3.524565in}{3.228027in}}%
\pgfpathmoveto{\pgfqpoint{3.524565in}{3.225077in}}%
\pgfpathlineto{\pgfqpoint{3.524565in}{3.225077in}}%
\pgfpathlineto{\pgfqpoint{3.524565in}{3.228027in}}%
\pgfpathlineto{\pgfqpoint{3.529106in}{3.228027in}}%
\pgfpathlineto{\pgfqpoint{3.529106in}{3.225077in}}%
\pgfpathmoveto{\pgfqpoint{3.529106in}{3.213281in}}%
\pgfpathlineto{\pgfqpoint{3.529106in}{3.213281in}}%
\pgfpathlineto{\pgfqpoint{3.529106in}{3.216230in}}%
\pgfpathlineto{\pgfqpoint{3.533647in}{3.216230in}}%
\pgfpathlineto{\pgfqpoint{3.533647in}{3.213281in}}%
\pgfpathmoveto{\pgfqpoint{3.529106in}{3.216230in}}%
\pgfpathlineto{\pgfqpoint{3.529106in}{3.216230in}}%
\pgfpathlineto{\pgfqpoint{3.529106in}{3.219179in}}%
\pgfpathlineto{\pgfqpoint{3.533647in}{3.219179in}}%
\pgfpathlineto{\pgfqpoint{3.533647in}{3.216230in}}%
\pgfpathmoveto{\pgfqpoint{3.533647in}{3.213281in}}%
\pgfpathlineto{\pgfqpoint{3.533647in}{3.213281in}}%
\pgfpathlineto{\pgfqpoint{3.533647in}{3.216230in}}%
\pgfpathlineto{\pgfqpoint{3.538188in}{3.216230in}}%
\pgfpathlineto{\pgfqpoint{3.538188in}{3.213281in}}%
\pgfpathmoveto{\pgfqpoint{3.533647in}{3.216230in}}%
\pgfpathlineto{\pgfqpoint{3.533647in}{3.216230in}}%
\pgfpathlineto{\pgfqpoint{3.533647in}{3.219179in}}%
\pgfpathlineto{\pgfqpoint{3.538188in}{3.219179in}}%
\pgfpathlineto{\pgfqpoint{3.538188in}{3.216230in}}%
\pgfpathmoveto{\pgfqpoint{3.529106in}{3.219179in}}%
\pgfpathlineto{\pgfqpoint{3.529106in}{3.219179in}}%
\pgfpathlineto{\pgfqpoint{3.529106in}{3.222128in}}%
\pgfpathlineto{\pgfqpoint{3.533647in}{3.222128in}}%
\pgfpathlineto{\pgfqpoint{3.533647in}{3.219179in}}%
\pgfpathmoveto{\pgfqpoint{3.529106in}{3.222128in}}%
\pgfpathlineto{\pgfqpoint{3.529106in}{3.222128in}}%
\pgfpathlineto{\pgfqpoint{3.529106in}{3.225077in}}%
\pgfpathlineto{\pgfqpoint{3.533647in}{3.225077in}}%
\pgfpathlineto{\pgfqpoint{3.533647in}{3.222128in}}%
\pgfpathmoveto{\pgfqpoint{3.533647in}{3.219179in}}%
\pgfpathlineto{\pgfqpoint{3.533647in}{3.219179in}}%
\pgfpathlineto{\pgfqpoint{3.533647in}{3.222128in}}%
\pgfpathlineto{\pgfqpoint{3.538188in}{3.222128in}}%
\pgfpathlineto{\pgfqpoint{3.538188in}{3.219179in}}%
\pgfpathmoveto{\pgfqpoint{3.538188in}{3.213281in}}%
\pgfpathlineto{\pgfqpoint{3.538188in}{3.213281in}}%
\pgfpathlineto{\pgfqpoint{3.538188in}{3.216230in}}%
\pgfpathlineto{\pgfqpoint{3.542729in}{3.216230in}}%
\pgfpathlineto{\pgfqpoint{3.542729in}{3.213281in}}%
\pgfpathmoveto{\pgfqpoint{3.538188in}{3.216230in}}%
\pgfpathlineto{\pgfqpoint{3.538188in}{3.216230in}}%
\pgfpathlineto{\pgfqpoint{3.538188in}{3.219179in}}%
\pgfpathlineto{\pgfqpoint{3.542729in}{3.219179in}}%
\pgfpathlineto{\pgfqpoint{3.542729in}{3.216230in}}%
\pgfpathmoveto{\pgfqpoint{3.542729in}{3.213281in}}%
\pgfpathlineto{\pgfqpoint{3.542729in}{3.213281in}}%
\pgfpathlineto{\pgfqpoint{3.542729in}{3.216230in}}%
\pgfpathlineto{\pgfqpoint{3.547269in}{3.216230in}}%
\pgfpathlineto{\pgfqpoint{3.547269in}{3.213281in}}%
\pgfpathmoveto{\pgfqpoint{3.556351in}{3.195585in}}%
\pgfpathlineto{\pgfqpoint{3.556351in}{3.195585in}}%
\pgfpathlineto{\pgfqpoint{3.556351in}{3.198535in}}%
\pgfpathlineto{\pgfqpoint{3.560892in}{3.198535in}}%
\pgfpathlineto{\pgfqpoint{3.560892in}{3.195585in}}%
\pgfpathmoveto{\pgfqpoint{3.556351in}{3.198535in}}%
\pgfpathlineto{\pgfqpoint{3.556351in}{3.198535in}}%
\pgfpathlineto{\pgfqpoint{3.556351in}{3.201484in}}%
\pgfpathlineto{\pgfqpoint{3.560892in}{3.201484in}}%
\pgfpathlineto{\pgfqpoint{3.560892in}{3.198535in}}%
\pgfpathmoveto{\pgfqpoint{3.560892in}{3.195585in}}%
\pgfpathlineto{\pgfqpoint{3.560892in}{3.195585in}}%
\pgfpathlineto{\pgfqpoint{3.560892in}{3.198535in}}%
\pgfpathlineto{\pgfqpoint{3.565433in}{3.198535in}}%
\pgfpathlineto{\pgfqpoint{3.565433in}{3.195585in}}%
\pgfpathmoveto{\pgfqpoint{3.560892in}{3.198535in}}%
\pgfpathlineto{\pgfqpoint{3.560892in}{3.198535in}}%
\pgfpathlineto{\pgfqpoint{3.560892in}{3.201484in}}%
\pgfpathlineto{\pgfqpoint{3.565433in}{3.201484in}}%
\pgfpathlineto{\pgfqpoint{3.565433in}{3.198535in}}%
\pgfpathmoveto{\pgfqpoint{3.547269in}{3.201484in}}%
\pgfpathlineto{\pgfqpoint{3.547269in}{3.201484in}}%
\pgfpathlineto{\pgfqpoint{3.547269in}{3.204433in}}%
\pgfpathlineto{\pgfqpoint{3.551810in}{3.204433in}}%
\pgfpathlineto{\pgfqpoint{3.551810in}{3.201484in}}%
\pgfpathmoveto{\pgfqpoint{3.547269in}{3.204433in}}%
\pgfpathlineto{\pgfqpoint{3.547269in}{3.204433in}}%
\pgfpathlineto{\pgfqpoint{3.547269in}{3.207382in}}%
\pgfpathlineto{\pgfqpoint{3.551810in}{3.207382in}}%
\pgfpathlineto{\pgfqpoint{3.551810in}{3.204433in}}%
\pgfpathmoveto{\pgfqpoint{3.551810in}{3.201484in}}%
\pgfpathlineto{\pgfqpoint{3.551810in}{3.201484in}}%
\pgfpathlineto{\pgfqpoint{3.551810in}{3.204433in}}%
\pgfpathlineto{\pgfqpoint{3.556351in}{3.204433in}}%
\pgfpathlineto{\pgfqpoint{3.556351in}{3.201484in}}%
\pgfpathmoveto{\pgfqpoint{3.551810in}{3.204433in}}%
\pgfpathlineto{\pgfqpoint{3.551810in}{3.204433in}}%
\pgfpathlineto{\pgfqpoint{3.551810in}{3.207382in}}%
\pgfpathlineto{\pgfqpoint{3.556351in}{3.207382in}}%
\pgfpathlineto{\pgfqpoint{3.556351in}{3.204433in}}%
\pgfpathmoveto{\pgfqpoint{3.547269in}{3.207382in}}%
\pgfpathlineto{\pgfqpoint{3.547269in}{3.207382in}}%
\pgfpathlineto{\pgfqpoint{3.547269in}{3.210331in}}%
\pgfpathlineto{\pgfqpoint{3.551810in}{3.210331in}}%
\pgfpathlineto{\pgfqpoint{3.551810in}{3.207382in}}%
\pgfpathmoveto{\pgfqpoint{3.547269in}{3.210331in}}%
\pgfpathlineto{\pgfqpoint{3.547269in}{3.210331in}}%
\pgfpathlineto{\pgfqpoint{3.547269in}{3.213281in}}%
\pgfpathlineto{\pgfqpoint{3.551810in}{3.213281in}}%
\pgfpathlineto{\pgfqpoint{3.551810in}{3.210331in}}%
\pgfpathmoveto{\pgfqpoint{3.551810in}{3.207382in}}%
\pgfpathlineto{\pgfqpoint{3.551810in}{3.207382in}}%
\pgfpathlineto{\pgfqpoint{3.551810in}{3.210331in}}%
\pgfpathlineto{\pgfqpoint{3.556351in}{3.210331in}}%
\pgfpathlineto{\pgfqpoint{3.556351in}{3.207382in}}%
\pgfpathmoveto{\pgfqpoint{3.556351in}{3.201484in}}%
\pgfpathlineto{\pgfqpoint{3.556351in}{3.201484in}}%
\pgfpathlineto{\pgfqpoint{3.556351in}{3.204433in}}%
\pgfpathlineto{\pgfqpoint{3.560892in}{3.204433in}}%
\pgfpathlineto{\pgfqpoint{3.560892in}{3.201484in}}%
\pgfpathmoveto{\pgfqpoint{3.556351in}{3.204433in}}%
\pgfpathlineto{\pgfqpoint{3.556351in}{3.204433in}}%
\pgfpathlineto{\pgfqpoint{3.556351in}{3.207382in}}%
\pgfpathlineto{\pgfqpoint{3.560892in}{3.207382in}}%
\pgfpathlineto{\pgfqpoint{3.560892in}{3.204433in}}%
\pgfpathmoveto{\pgfqpoint{3.560892in}{3.201484in}}%
\pgfpathlineto{\pgfqpoint{3.560892in}{3.201484in}}%
\pgfpathlineto{\pgfqpoint{3.560892in}{3.204433in}}%
\pgfpathlineto{\pgfqpoint{3.565433in}{3.204433in}}%
\pgfpathlineto{\pgfqpoint{3.565433in}{3.201484in}}%
\pgfpathmoveto{\pgfqpoint{3.565433in}{3.189687in}}%
\pgfpathlineto{\pgfqpoint{3.565433in}{3.189687in}}%
\pgfpathlineto{\pgfqpoint{3.565433in}{3.192636in}}%
\pgfpathlineto{\pgfqpoint{3.569974in}{3.192636in}}%
\pgfpathlineto{\pgfqpoint{3.569974in}{3.189687in}}%
\pgfpathmoveto{\pgfqpoint{3.565433in}{3.192636in}}%
\pgfpathlineto{\pgfqpoint{3.565433in}{3.192636in}}%
\pgfpathlineto{\pgfqpoint{3.565433in}{3.195585in}}%
\pgfpathlineto{\pgfqpoint{3.569974in}{3.195585in}}%
\pgfpathlineto{\pgfqpoint{3.569974in}{3.192636in}}%
\pgfpathmoveto{\pgfqpoint{3.569974in}{3.189687in}}%
\pgfpathlineto{\pgfqpoint{3.569974in}{3.189687in}}%
\pgfpathlineto{\pgfqpoint{3.569974in}{3.192636in}}%
\pgfpathlineto{\pgfqpoint{3.574515in}{3.192636in}}%
\pgfpathlineto{\pgfqpoint{3.574515in}{3.189687in}}%
\pgfpathmoveto{\pgfqpoint{3.569974in}{3.192636in}}%
\pgfpathlineto{\pgfqpoint{3.569974in}{3.192636in}}%
\pgfpathlineto{\pgfqpoint{3.569974in}{3.195585in}}%
\pgfpathlineto{\pgfqpoint{3.574515in}{3.195585in}}%
\pgfpathlineto{\pgfqpoint{3.574515in}{3.192636in}}%
\pgfpathmoveto{\pgfqpoint{3.565433in}{3.195585in}}%
\pgfpathlineto{\pgfqpoint{3.565433in}{3.195585in}}%
\pgfpathlineto{\pgfqpoint{3.565433in}{3.198535in}}%
\pgfpathlineto{\pgfqpoint{3.569974in}{3.198535in}}%
\pgfpathlineto{\pgfqpoint{3.569974in}{3.195585in}}%
\pgfpathmoveto{\pgfqpoint{3.565433in}{3.198535in}}%
\pgfpathlineto{\pgfqpoint{3.565433in}{3.198535in}}%
\pgfpathlineto{\pgfqpoint{3.565433in}{3.201484in}}%
\pgfpathlineto{\pgfqpoint{3.569974in}{3.201484in}}%
\pgfpathlineto{\pgfqpoint{3.569974in}{3.198535in}}%
\pgfpathmoveto{\pgfqpoint{3.569974in}{3.195585in}}%
\pgfpathlineto{\pgfqpoint{3.569974in}{3.195585in}}%
\pgfpathlineto{\pgfqpoint{3.569974in}{3.198535in}}%
\pgfpathlineto{\pgfqpoint{3.574515in}{3.198535in}}%
\pgfpathlineto{\pgfqpoint{3.574515in}{3.195585in}}%
\pgfpathmoveto{\pgfqpoint{3.574515in}{3.189687in}}%
\pgfpathlineto{\pgfqpoint{3.574515in}{3.189687in}}%
\pgfpathlineto{\pgfqpoint{3.574515in}{3.192636in}}%
\pgfpathlineto{\pgfqpoint{3.579056in}{3.192636in}}%
\pgfpathlineto{\pgfqpoint{3.579056in}{3.189687in}}%
\pgfpathmoveto{\pgfqpoint{3.574515in}{3.192636in}}%
\pgfpathlineto{\pgfqpoint{3.574515in}{3.192636in}}%
\pgfpathlineto{\pgfqpoint{3.574515in}{3.195585in}}%
\pgfpathlineto{\pgfqpoint{3.579056in}{3.195585in}}%
\pgfpathlineto{\pgfqpoint{3.579056in}{3.192636in}}%
\pgfpathmoveto{\pgfqpoint{3.579056in}{3.189687in}}%
\pgfpathlineto{\pgfqpoint{3.579056in}{3.189687in}}%
\pgfpathlineto{\pgfqpoint{3.579056in}{3.192636in}}%
\pgfpathlineto{\pgfqpoint{3.583597in}{3.192636in}}%
\pgfpathlineto{\pgfqpoint{3.583597in}{3.189687in}}%
\pgfpathmoveto{\pgfqpoint{3.610843in}{3.160195in}}%
\pgfpathlineto{\pgfqpoint{3.610843in}{3.160195in}}%
\pgfpathlineto{\pgfqpoint{3.610843in}{3.163144in}}%
\pgfpathlineto{\pgfqpoint{3.615384in}{3.163144in}}%
\pgfpathlineto{\pgfqpoint{3.615384in}{3.160195in}}%
\pgfpathmoveto{\pgfqpoint{3.610843in}{3.163144in}}%
\pgfpathlineto{\pgfqpoint{3.610843in}{3.163144in}}%
\pgfpathlineto{\pgfqpoint{3.610843in}{3.166093in}}%
\pgfpathlineto{\pgfqpoint{3.615384in}{3.166093in}}%
\pgfpathlineto{\pgfqpoint{3.615384in}{3.163144in}}%
\pgfpathmoveto{\pgfqpoint{3.615384in}{3.160195in}}%
\pgfpathlineto{\pgfqpoint{3.615384in}{3.160195in}}%
\pgfpathlineto{\pgfqpoint{3.615384in}{3.163144in}}%
\pgfpathlineto{\pgfqpoint{3.619925in}{3.163144in}}%
\pgfpathlineto{\pgfqpoint{3.619925in}{3.160195in}}%
\pgfpathmoveto{\pgfqpoint{3.615384in}{3.163144in}}%
\pgfpathlineto{\pgfqpoint{3.615384in}{3.163144in}}%
\pgfpathlineto{\pgfqpoint{3.615384in}{3.166093in}}%
\pgfpathlineto{\pgfqpoint{3.619925in}{3.166093in}}%
\pgfpathlineto{\pgfqpoint{3.619925in}{3.163144in}}%
\pgfpathmoveto{\pgfqpoint{3.592679in}{3.171992in}}%
\pgfpathlineto{\pgfqpoint{3.592679in}{3.171992in}}%
\pgfpathlineto{\pgfqpoint{3.592679in}{3.174941in}}%
\pgfpathlineto{\pgfqpoint{3.597220in}{3.174941in}}%
\pgfpathlineto{\pgfqpoint{3.597220in}{3.171992in}}%
\pgfpathmoveto{\pgfqpoint{3.592679in}{3.174941in}}%
\pgfpathlineto{\pgfqpoint{3.592679in}{3.174941in}}%
\pgfpathlineto{\pgfqpoint{3.592679in}{3.177890in}}%
\pgfpathlineto{\pgfqpoint{3.597220in}{3.177890in}}%
\pgfpathlineto{\pgfqpoint{3.597220in}{3.174941in}}%
\pgfpathmoveto{\pgfqpoint{3.597220in}{3.171992in}}%
\pgfpathlineto{\pgfqpoint{3.597220in}{3.171992in}}%
\pgfpathlineto{\pgfqpoint{3.597220in}{3.174941in}}%
\pgfpathlineto{\pgfqpoint{3.601761in}{3.174941in}}%
\pgfpathlineto{\pgfqpoint{3.601761in}{3.171992in}}%
\pgfpathmoveto{\pgfqpoint{3.597220in}{3.174941in}}%
\pgfpathlineto{\pgfqpoint{3.597220in}{3.174941in}}%
\pgfpathlineto{\pgfqpoint{3.597220in}{3.177890in}}%
\pgfpathlineto{\pgfqpoint{3.601761in}{3.177890in}}%
\pgfpathlineto{\pgfqpoint{3.601761in}{3.174941in}}%
\pgfpathmoveto{\pgfqpoint{3.583597in}{3.177890in}}%
\pgfpathlineto{\pgfqpoint{3.583597in}{3.177890in}}%
\pgfpathlineto{\pgfqpoint{3.583597in}{3.180839in}}%
\pgfpathlineto{\pgfqpoint{3.588138in}{3.180839in}}%
\pgfpathlineto{\pgfqpoint{3.588138in}{3.177890in}}%
\pgfpathmoveto{\pgfqpoint{3.583597in}{3.180839in}}%
\pgfpathlineto{\pgfqpoint{3.583597in}{3.180839in}}%
\pgfpathlineto{\pgfqpoint{3.583597in}{3.183788in}}%
\pgfpathlineto{\pgfqpoint{3.588138in}{3.183788in}}%
\pgfpathlineto{\pgfqpoint{3.588138in}{3.180839in}}%
\pgfpathmoveto{\pgfqpoint{3.588138in}{3.177890in}}%
\pgfpathlineto{\pgfqpoint{3.588138in}{3.177890in}}%
\pgfpathlineto{\pgfqpoint{3.588138in}{3.180839in}}%
\pgfpathlineto{\pgfqpoint{3.592679in}{3.180839in}}%
\pgfpathlineto{\pgfqpoint{3.592679in}{3.177890in}}%
\pgfpathmoveto{\pgfqpoint{3.588138in}{3.180839in}}%
\pgfpathlineto{\pgfqpoint{3.588138in}{3.180839in}}%
\pgfpathlineto{\pgfqpoint{3.588138in}{3.183788in}}%
\pgfpathlineto{\pgfqpoint{3.592679in}{3.183788in}}%
\pgfpathlineto{\pgfqpoint{3.592679in}{3.180839in}}%
\pgfpathmoveto{\pgfqpoint{3.583597in}{3.183788in}}%
\pgfpathlineto{\pgfqpoint{3.583597in}{3.183788in}}%
\pgfpathlineto{\pgfqpoint{3.583597in}{3.186738in}}%
\pgfpathlineto{\pgfqpoint{3.588138in}{3.186738in}}%
\pgfpathlineto{\pgfqpoint{3.588138in}{3.183788in}}%
\pgfpathmoveto{\pgfqpoint{3.583597in}{3.186738in}}%
\pgfpathlineto{\pgfqpoint{3.583597in}{3.186738in}}%
\pgfpathlineto{\pgfqpoint{3.583597in}{3.189687in}}%
\pgfpathlineto{\pgfqpoint{3.588138in}{3.189687in}}%
\pgfpathlineto{\pgfqpoint{3.588138in}{3.186738in}}%
\pgfpathmoveto{\pgfqpoint{3.588138in}{3.183788in}}%
\pgfpathlineto{\pgfqpoint{3.588138in}{3.183788in}}%
\pgfpathlineto{\pgfqpoint{3.588138in}{3.186738in}}%
\pgfpathlineto{\pgfqpoint{3.592679in}{3.186738in}}%
\pgfpathlineto{\pgfqpoint{3.592679in}{3.183788in}}%
\pgfpathmoveto{\pgfqpoint{3.592679in}{3.177890in}}%
\pgfpathlineto{\pgfqpoint{3.592679in}{3.177890in}}%
\pgfpathlineto{\pgfqpoint{3.592679in}{3.180839in}}%
\pgfpathlineto{\pgfqpoint{3.597220in}{3.180839in}}%
\pgfpathlineto{\pgfqpoint{3.597220in}{3.177890in}}%
\pgfpathmoveto{\pgfqpoint{3.592679in}{3.180839in}}%
\pgfpathlineto{\pgfqpoint{3.592679in}{3.180839in}}%
\pgfpathlineto{\pgfqpoint{3.592679in}{3.183788in}}%
\pgfpathlineto{\pgfqpoint{3.597220in}{3.183788in}}%
\pgfpathlineto{\pgfqpoint{3.597220in}{3.180839in}}%
\pgfpathmoveto{\pgfqpoint{3.597220in}{3.177890in}}%
\pgfpathlineto{\pgfqpoint{3.597220in}{3.177890in}}%
\pgfpathlineto{\pgfqpoint{3.597220in}{3.180839in}}%
\pgfpathlineto{\pgfqpoint{3.601761in}{3.180839in}}%
\pgfpathlineto{\pgfqpoint{3.601761in}{3.177890in}}%
\pgfpathmoveto{\pgfqpoint{3.601761in}{3.166093in}}%
\pgfpathlineto{\pgfqpoint{3.601761in}{3.166093in}}%
\pgfpathlineto{\pgfqpoint{3.601761in}{3.169042in}}%
\pgfpathlineto{\pgfqpoint{3.606302in}{3.169042in}}%
\pgfpathlineto{\pgfqpoint{3.606302in}{3.166093in}}%
\pgfpathmoveto{\pgfqpoint{3.601761in}{3.169042in}}%
\pgfpathlineto{\pgfqpoint{3.601761in}{3.169042in}}%
\pgfpathlineto{\pgfqpoint{3.601761in}{3.171992in}}%
\pgfpathlineto{\pgfqpoint{3.606302in}{3.171992in}}%
\pgfpathlineto{\pgfqpoint{3.606302in}{3.169042in}}%
\pgfpathmoveto{\pgfqpoint{3.606302in}{3.166093in}}%
\pgfpathlineto{\pgfqpoint{3.606302in}{3.166093in}}%
\pgfpathlineto{\pgfqpoint{3.606302in}{3.169042in}}%
\pgfpathlineto{\pgfqpoint{3.610843in}{3.169042in}}%
\pgfpathlineto{\pgfqpoint{3.610843in}{3.166093in}}%
\pgfpathmoveto{\pgfqpoint{3.606302in}{3.169042in}}%
\pgfpathlineto{\pgfqpoint{3.606302in}{3.169042in}}%
\pgfpathlineto{\pgfqpoint{3.606302in}{3.171992in}}%
\pgfpathlineto{\pgfqpoint{3.610843in}{3.171992in}}%
\pgfpathlineto{\pgfqpoint{3.610843in}{3.169042in}}%
\pgfpathmoveto{\pgfqpoint{3.601761in}{3.171992in}}%
\pgfpathlineto{\pgfqpoint{3.601761in}{3.171992in}}%
\pgfpathlineto{\pgfqpoint{3.601761in}{3.174941in}}%
\pgfpathlineto{\pgfqpoint{3.606302in}{3.174941in}}%
\pgfpathlineto{\pgfqpoint{3.606302in}{3.171992in}}%
\pgfpathmoveto{\pgfqpoint{3.601761in}{3.174941in}}%
\pgfpathlineto{\pgfqpoint{3.601761in}{3.174941in}}%
\pgfpathlineto{\pgfqpoint{3.601761in}{3.177890in}}%
\pgfpathlineto{\pgfqpoint{3.606302in}{3.177890in}}%
\pgfpathlineto{\pgfqpoint{3.606302in}{3.174941in}}%
\pgfpathmoveto{\pgfqpoint{3.606302in}{3.171992in}}%
\pgfpathlineto{\pgfqpoint{3.606302in}{3.171992in}}%
\pgfpathlineto{\pgfqpoint{3.606302in}{3.174941in}}%
\pgfpathlineto{\pgfqpoint{3.610843in}{3.174941in}}%
\pgfpathlineto{\pgfqpoint{3.610843in}{3.171992in}}%
\pgfpathmoveto{\pgfqpoint{3.610843in}{3.166093in}}%
\pgfpathlineto{\pgfqpoint{3.610843in}{3.166093in}}%
\pgfpathlineto{\pgfqpoint{3.610843in}{3.169042in}}%
\pgfpathlineto{\pgfqpoint{3.615384in}{3.169042in}}%
\pgfpathlineto{\pgfqpoint{3.615384in}{3.166093in}}%
\pgfpathmoveto{\pgfqpoint{3.610843in}{3.169042in}}%
\pgfpathlineto{\pgfqpoint{3.610843in}{3.169042in}}%
\pgfpathlineto{\pgfqpoint{3.610843in}{3.171992in}}%
\pgfpathlineto{\pgfqpoint{3.615384in}{3.171992in}}%
\pgfpathlineto{\pgfqpoint{3.615384in}{3.169042in}}%
\pgfpathmoveto{\pgfqpoint{3.615384in}{3.166093in}}%
\pgfpathlineto{\pgfqpoint{3.615384in}{3.166093in}}%
\pgfpathlineto{\pgfqpoint{3.615384in}{3.169042in}}%
\pgfpathlineto{\pgfqpoint{3.619925in}{3.169042in}}%
\pgfpathlineto{\pgfqpoint{3.619925in}{3.166093in}}%
\pgfpathmoveto{\pgfqpoint{3.629007in}{3.148398in}}%
\pgfpathlineto{\pgfqpoint{3.629007in}{3.148398in}}%
\pgfpathlineto{\pgfqpoint{3.629007in}{3.151347in}}%
\pgfpathlineto{\pgfqpoint{3.633548in}{3.151347in}}%
\pgfpathlineto{\pgfqpoint{3.633548in}{3.148398in}}%
\pgfpathmoveto{\pgfqpoint{3.629007in}{3.151347in}}%
\pgfpathlineto{\pgfqpoint{3.629007in}{3.151347in}}%
\pgfpathlineto{\pgfqpoint{3.629007in}{3.154296in}}%
\pgfpathlineto{\pgfqpoint{3.633548in}{3.154296in}}%
\pgfpathlineto{\pgfqpoint{3.633548in}{3.151347in}}%
\pgfpathmoveto{\pgfqpoint{3.633548in}{3.148398in}}%
\pgfpathlineto{\pgfqpoint{3.633548in}{3.148398in}}%
\pgfpathlineto{\pgfqpoint{3.633548in}{3.151347in}}%
\pgfpathlineto{\pgfqpoint{3.638089in}{3.151347in}}%
\pgfpathlineto{\pgfqpoint{3.638089in}{3.148398in}}%
\pgfpathmoveto{\pgfqpoint{3.633548in}{3.151347in}}%
\pgfpathlineto{\pgfqpoint{3.633548in}{3.151347in}}%
\pgfpathlineto{\pgfqpoint{3.633548in}{3.154296in}}%
\pgfpathlineto{\pgfqpoint{3.638089in}{3.154296in}}%
\pgfpathlineto{\pgfqpoint{3.638089in}{3.151347in}}%
\pgfpathmoveto{\pgfqpoint{3.619925in}{3.154296in}}%
\pgfpathlineto{\pgfqpoint{3.619925in}{3.154296in}}%
\pgfpathlineto{\pgfqpoint{3.619925in}{3.157246in}}%
\pgfpathlineto{\pgfqpoint{3.624466in}{3.157246in}}%
\pgfpathlineto{\pgfqpoint{3.624466in}{3.154296in}}%
\pgfpathmoveto{\pgfqpoint{3.619925in}{3.157246in}}%
\pgfpathlineto{\pgfqpoint{3.619925in}{3.157246in}}%
\pgfpathlineto{\pgfqpoint{3.619925in}{3.160195in}}%
\pgfpathlineto{\pgfqpoint{3.624466in}{3.160195in}}%
\pgfpathlineto{\pgfqpoint{3.624466in}{3.157246in}}%
\pgfpathmoveto{\pgfqpoint{3.624466in}{3.154296in}}%
\pgfpathlineto{\pgfqpoint{3.624466in}{3.154296in}}%
\pgfpathlineto{\pgfqpoint{3.624466in}{3.157246in}}%
\pgfpathlineto{\pgfqpoint{3.629007in}{3.157246in}}%
\pgfpathlineto{\pgfqpoint{3.629007in}{3.154296in}}%
\pgfpathmoveto{\pgfqpoint{3.624466in}{3.157246in}}%
\pgfpathlineto{\pgfqpoint{3.624466in}{3.157246in}}%
\pgfpathlineto{\pgfqpoint{3.624466in}{3.160195in}}%
\pgfpathlineto{\pgfqpoint{3.629007in}{3.160195in}}%
\pgfpathlineto{\pgfqpoint{3.629007in}{3.157246in}}%
\pgfpathmoveto{\pgfqpoint{3.619925in}{3.160195in}}%
\pgfpathlineto{\pgfqpoint{3.619925in}{3.160195in}}%
\pgfpathlineto{\pgfqpoint{3.619925in}{3.163144in}}%
\pgfpathlineto{\pgfqpoint{3.624466in}{3.163144in}}%
\pgfpathlineto{\pgfqpoint{3.624466in}{3.160195in}}%
\pgfpathmoveto{\pgfqpoint{3.619925in}{3.163144in}}%
\pgfpathlineto{\pgfqpoint{3.619925in}{3.163144in}}%
\pgfpathlineto{\pgfqpoint{3.619925in}{3.166093in}}%
\pgfpathlineto{\pgfqpoint{3.624466in}{3.166093in}}%
\pgfpathlineto{\pgfqpoint{3.624466in}{3.163144in}}%
\pgfpathmoveto{\pgfqpoint{3.624466in}{3.160195in}}%
\pgfpathlineto{\pgfqpoint{3.624466in}{3.160195in}}%
\pgfpathlineto{\pgfqpoint{3.624466in}{3.163144in}}%
\pgfpathlineto{\pgfqpoint{3.629007in}{3.163144in}}%
\pgfpathlineto{\pgfqpoint{3.629007in}{3.160195in}}%
\pgfpathmoveto{\pgfqpoint{3.629007in}{3.154296in}}%
\pgfpathlineto{\pgfqpoint{3.629007in}{3.154296in}}%
\pgfpathlineto{\pgfqpoint{3.629007in}{3.157246in}}%
\pgfpathlineto{\pgfqpoint{3.633548in}{3.157246in}}%
\pgfpathlineto{\pgfqpoint{3.633548in}{3.154296in}}%
\pgfpathmoveto{\pgfqpoint{3.629007in}{3.157246in}}%
\pgfpathlineto{\pgfqpoint{3.629007in}{3.157246in}}%
\pgfpathlineto{\pgfqpoint{3.629007in}{3.160195in}}%
\pgfpathlineto{\pgfqpoint{3.633548in}{3.160195in}}%
\pgfpathlineto{\pgfqpoint{3.633548in}{3.157246in}}%
\pgfpathmoveto{\pgfqpoint{3.633548in}{3.154296in}}%
\pgfpathlineto{\pgfqpoint{3.633548in}{3.154296in}}%
\pgfpathlineto{\pgfqpoint{3.633548in}{3.157246in}}%
\pgfpathlineto{\pgfqpoint{3.638089in}{3.157246in}}%
\pgfpathlineto{\pgfqpoint{3.638089in}{3.154296in}}%
\pgfpathmoveto{\pgfqpoint{3.638089in}{3.142499in}}%
\pgfpathlineto{\pgfqpoint{3.638089in}{3.142499in}}%
\pgfpathlineto{\pgfqpoint{3.638089in}{3.145449in}}%
\pgfpathlineto{\pgfqpoint{3.642630in}{3.145449in}}%
\pgfpathlineto{\pgfqpoint{3.642630in}{3.142499in}}%
\pgfpathmoveto{\pgfqpoint{3.638089in}{3.145449in}}%
\pgfpathlineto{\pgfqpoint{3.638089in}{3.145449in}}%
\pgfpathlineto{\pgfqpoint{3.638089in}{3.148398in}}%
\pgfpathlineto{\pgfqpoint{3.642630in}{3.148398in}}%
\pgfpathlineto{\pgfqpoint{3.642630in}{3.145449in}}%
\pgfpathmoveto{\pgfqpoint{3.642630in}{3.142499in}}%
\pgfpathlineto{\pgfqpoint{3.642630in}{3.142499in}}%
\pgfpathlineto{\pgfqpoint{3.642630in}{3.145449in}}%
\pgfpathlineto{\pgfqpoint{3.647171in}{3.145449in}}%
\pgfpathlineto{\pgfqpoint{3.647171in}{3.142499in}}%
\pgfpathmoveto{\pgfqpoint{3.642630in}{3.145449in}}%
\pgfpathlineto{\pgfqpoint{3.642630in}{3.145449in}}%
\pgfpathlineto{\pgfqpoint{3.642630in}{3.148398in}}%
\pgfpathlineto{\pgfqpoint{3.647171in}{3.148398in}}%
\pgfpathlineto{\pgfqpoint{3.647171in}{3.145449in}}%
\pgfpathmoveto{\pgfqpoint{3.638089in}{3.148398in}}%
\pgfpathlineto{\pgfqpoint{3.638089in}{3.148398in}}%
\pgfpathlineto{\pgfqpoint{3.638089in}{3.151347in}}%
\pgfpathlineto{\pgfqpoint{3.642630in}{3.151347in}}%
\pgfpathlineto{\pgfqpoint{3.642630in}{3.148398in}}%
\pgfpathmoveto{\pgfqpoint{3.638089in}{3.151347in}}%
\pgfpathlineto{\pgfqpoint{3.638089in}{3.151347in}}%
\pgfpathlineto{\pgfqpoint{3.638089in}{3.154296in}}%
\pgfpathlineto{\pgfqpoint{3.642630in}{3.154296in}}%
\pgfpathlineto{\pgfqpoint{3.642630in}{3.151347in}}%
\pgfpathmoveto{\pgfqpoint{3.642630in}{3.148398in}}%
\pgfpathlineto{\pgfqpoint{3.642630in}{3.148398in}}%
\pgfpathlineto{\pgfqpoint{3.642630in}{3.151347in}}%
\pgfpathlineto{\pgfqpoint{3.647171in}{3.151347in}}%
\pgfpathlineto{\pgfqpoint{3.647171in}{3.148398in}}%
\pgfpathmoveto{\pgfqpoint{3.647171in}{3.142499in}}%
\pgfpathlineto{\pgfqpoint{3.647171in}{3.142499in}}%
\pgfpathlineto{\pgfqpoint{3.647171in}{3.145449in}}%
\pgfpathlineto{\pgfqpoint{3.651712in}{3.145449in}}%
\pgfpathlineto{\pgfqpoint{3.651712in}{3.142499in}}%
\pgfpathmoveto{\pgfqpoint{3.647171in}{3.145449in}}%
\pgfpathlineto{\pgfqpoint{3.647171in}{3.145449in}}%
\pgfpathlineto{\pgfqpoint{3.647171in}{3.148398in}}%
\pgfpathlineto{\pgfqpoint{3.651712in}{3.148398in}}%
\pgfpathlineto{\pgfqpoint{3.651712in}{3.145449in}}%
\pgfpathmoveto{\pgfqpoint{3.651712in}{3.142499in}}%
\pgfpathlineto{\pgfqpoint{3.651712in}{3.142499in}}%
\pgfpathlineto{\pgfqpoint{3.651712in}{3.145449in}}%
\pgfpathlineto{\pgfqpoint{3.656252in}{3.145449in}}%
\pgfpathlineto{\pgfqpoint{3.656252in}{3.142499in}}%
\pgfpathmoveto{\pgfqpoint{3.769774in}{2.762049in}}%
\pgfpathlineto{\pgfqpoint{3.769774in}{2.762049in}}%
\pgfpathlineto{\pgfqpoint{3.769774in}{2.764998in}}%
\pgfpathlineto{\pgfqpoint{3.774315in}{2.764998in}}%
\pgfpathlineto{\pgfqpoint{3.774315in}{2.762049in}}%
\pgfpathmoveto{\pgfqpoint{3.774315in}{2.762049in}}%
\pgfpathlineto{\pgfqpoint{3.774315in}{2.762049in}}%
\pgfpathlineto{\pgfqpoint{3.774315in}{2.764998in}}%
\pgfpathlineto{\pgfqpoint{3.778855in}{2.764998in}}%
\pgfpathlineto{\pgfqpoint{3.778855in}{2.762049in}}%
\pgfpathmoveto{\pgfqpoint{3.778855in}{2.762049in}}%
\pgfpathlineto{\pgfqpoint{3.778855in}{2.762049in}}%
\pgfpathlineto{\pgfqpoint{3.778855in}{2.764998in}}%
\pgfpathlineto{\pgfqpoint{3.783396in}{2.764998in}}%
\pgfpathlineto{\pgfqpoint{3.783396in}{2.762049in}}%
\pgfpathmoveto{\pgfqpoint{3.783396in}{2.759100in}}%
\pgfpathlineto{\pgfqpoint{3.783396in}{2.759100in}}%
\pgfpathlineto{\pgfqpoint{3.783396in}{2.762049in}}%
\pgfpathlineto{\pgfqpoint{3.787937in}{2.762049in}}%
\pgfpathlineto{\pgfqpoint{3.787937in}{2.759100in}}%
\pgfpathmoveto{\pgfqpoint{3.783396in}{2.762049in}}%
\pgfpathlineto{\pgfqpoint{3.783396in}{2.762049in}}%
\pgfpathlineto{\pgfqpoint{3.783396in}{2.764998in}}%
\pgfpathlineto{\pgfqpoint{3.787937in}{2.764998in}}%
\pgfpathlineto{\pgfqpoint{3.787937in}{2.762049in}}%
\pgfpathmoveto{\pgfqpoint{3.787937in}{2.759100in}}%
\pgfpathlineto{\pgfqpoint{3.787937in}{2.759100in}}%
\pgfpathlineto{\pgfqpoint{3.787937in}{2.762049in}}%
\pgfpathlineto{\pgfqpoint{3.792478in}{2.762049in}}%
\pgfpathlineto{\pgfqpoint{3.792478in}{2.759100in}}%
\pgfpathmoveto{\pgfqpoint{3.787937in}{2.762049in}}%
\pgfpathlineto{\pgfqpoint{3.787937in}{2.762049in}}%
\pgfpathlineto{\pgfqpoint{3.787937in}{2.764998in}}%
\pgfpathlineto{\pgfqpoint{3.792478in}{2.764998in}}%
\pgfpathlineto{\pgfqpoint{3.792478in}{2.762049in}}%
\pgfpathmoveto{\pgfqpoint{3.797019in}{2.756151in}}%
\pgfpathlineto{\pgfqpoint{3.797019in}{2.756151in}}%
\pgfpathlineto{\pgfqpoint{3.797019in}{2.759100in}}%
\pgfpathlineto{\pgfqpoint{3.801560in}{2.759100in}}%
\pgfpathlineto{\pgfqpoint{3.801560in}{2.756151in}}%
\pgfpathmoveto{\pgfqpoint{3.792478in}{2.759100in}}%
\pgfpathlineto{\pgfqpoint{3.792478in}{2.759100in}}%
\pgfpathlineto{\pgfqpoint{3.792478in}{2.762049in}}%
\pgfpathlineto{\pgfqpoint{3.797019in}{2.762049in}}%
\pgfpathlineto{\pgfqpoint{3.797019in}{2.759100in}}%
\pgfpathmoveto{\pgfqpoint{3.792478in}{2.762049in}}%
\pgfpathlineto{\pgfqpoint{3.792478in}{2.762049in}}%
\pgfpathlineto{\pgfqpoint{3.792478in}{2.764998in}}%
\pgfpathlineto{\pgfqpoint{3.797019in}{2.764998in}}%
\pgfpathlineto{\pgfqpoint{3.797019in}{2.762049in}}%
\pgfpathmoveto{\pgfqpoint{3.797019in}{2.759100in}}%
\pgfpathlineto{\pgfqpoint{3.797019in}{2.759100in}}%
\pgfpathlineto{\pgfqpoint{3.797019in}{2.762049in}}%
\pgfpathlineto{\pgfqpoint{3.801560in}{2.762049in}}%
\pgfpathlineto{\pgfqpoint{3.801560in}{2.759100in}}%
\pgfpathmoveto{\pgfqpoint{3.797019in}{2.762049in}}%
\pgfpathlineto{\pgfqpoint{3.797019in}{2.762049in}}%
\pgfpathlineto{\pgfqpoint{3.797019in}{2.764998in}}%
\pgfpathlineto{\pgfqpoint{3.801560in}{2.764998in}}%
\pgfpathlineto{\pgfqpoint{3.801560in}{2.762049in}}%
\pgfpathmoveto{\pgfqpoint{3.660793in}{2.785643in}}%
\pgfpathlineto{\pgfqpoint{3.660793in}{2.785643in}}%
\pgfpathlineto{\pgfqpoint{3.660793in}{2.788592in}}%
\pgfpathlineto{\pgfqpoint{3.665334in}{2.788592in}}%
\pgfpathlineto{\pgfqpoint{3.665334in}{2.785643in}}%
\pgfpathmoveto{\pgfqpoint{3.665334in}{2.785643in}}%
\pgfpathlineto{\pgfqpoint{3.665334in}{2.785643in}}%
\pgfpathlineto{\pgfqpoint{3.665334in}{2.788592in}}%
\pgfpathlineto{\pgfqpoint{3.669875in}{2.788592in}}%
\pgfpathlineto{\pgfqpoint{3.669875in}{2.785643in}}%
\pgfpathmoveto{\pgfqpoint{3.669875in}{2.785643in}}%
\pgfpathlineto{\pgfqpoint{3.669875in}{2.785643in}}%
\pgfpathlineto{\pgfqpoint{3.669875in}{2.788592in}}%
\pgfpathlineto{\pgfqpoint{3.674416in}{2.788592in}}%
\pgfpathlineto{\pgfqpoint{3.674416in}{2.785643in}}%
\pgfpathmoveto{\pgfqpoint{3.674416in}{2.782694in}}%
\pgfpathlineto{\pgfqpoint{3.674416in}{2.782694in}}%
\pgfpathlineto{\pgfqpoint{3.674416in}{2.785643in}}%
\pgfpathlineto{\pgfqpoint{3.678957in}{2.785643in}}%
\pgfpathlineto{\pgfqpoint{3.678957in}{2.782694in}}%
\pgfpathmoveto{\pgfqpoint{3.674416in}{2.785643in}}%
\pgfpathlineto{\pgfqpoint{3.674416in}{2.785643in}}%
\pgfpathlineto{\pgfqpoint{3.674416in}{2.788592in}}%
\pgfpathlineto{\pgfqpoint{3.678957in}{2.788592in}}%
\pgfpathlineto{\pgfqpoint{3.678957in}{2.785643in}}%
\pgfpathmoveto{\pgfqpoint{3.678957in}{2.782694in}}%
\pgfpathlineto{\pgfqpoint{3.678957in}{2.782694in}}%
\pgfpathlineto{\pgfqpoint{3.678957in}{2.785643in}}%
\pgfpathlineto{\pgfqpoint{3.683498in}{2.785643in}}%
\pgfpathlineto{\pgfqpoint{3.683498in}{2.782694in}}%
\pgfpathmoveto{\pgfqpoint{3.678957in}{2.785643in}}%
\pgfpathlineto{\pgfqpoint{3.678957in}{2.785643in}}%
\pgfpathlineto{\pgfqpoint{3.678957in}{2.788592in}}%
\pgfpathlineto{\pgfqpoint{3.683498in}{2.788592in}}%
\pgfpathlineto{\pgfqpoint{3.683498in}{2.785643in}}%
\pgfpathmoveto{\pgfqpoint{3.688038in}{2.779745in}}%
\pgfpathlineto{\pgfqpoint{3.688038in}{2.779745in}}%
\pgfpathlineto{\pgfqpoint{3.688038in}{2.782694in}}%
\pgfpathlineto{\pgfqpoint{3.692579in}{2.782694in}}%
\pgfpathlineto{\pgfqpoint{3.692579in}{2.779745in}}%
\pgfpathmoveto{\pgfqpoint{3.683498in}{2.782694in}}%
\pgfpathlineto{\pgfqpoint{3.683498in}{2.782694in}}%
\pgfpathlineto{\pgfqpoint{3.683498in}{2.785643in}}%
\pgfpathlineto{\pgfqpoint{3.688038in}{2.785643in}}%
\pgfpathlineto{\pgfqpoint{3.688038in}{2.782694in}}%
\pgfpathmoveto{\pgfqpoint{3.683498in}{2.785643in}}%
\pgfpathlineto{\pgfqpoint{3.683498in}{2.785643in}}%
\pgfpathlineto{\pgfqpoint{3.683498in}{2.788592in}}%
\pgfpathlineto{\pgfqpoint{3.688038in}{2.788592in}}%
\pgfpathlineto{\pgfqpoint{3.688038in}{2.785643in}}%
\pgfpathmoveto{\pgfqpoint{3.688038in}{2.782694in}}%
\pgfpathlineto{\pgfqpoint{3.688038in}{2.782694in}}%
\pgfpathlineto{\pgfqpoint{3.688038in}{2.785643in}}%
\pgfpathlineto{\pgfqpoint{3.692579in}{2.785643in}}%
\pgfpathlineto{\pgfqpoint{3.692579in}{2.782694in}}%
\pgfpathmoveto{\pgfqpoint{3.688038in}{2.785643in}}%
\pgfpathlineto{\pgfqpoint{3.688038in}{2.785643in}}%
\pgfpathlineto{\pgfqpoint{3.688038in}{2.788592in}}%
\pgfpathlineto{\pgfqpoint{3.692579in}{2.788592in}}%
\pgfpathlineto{\pgfqpoint{3.692579in}{2.785643in}}%
\pgfpathmoveto{\pgfqpoint{3.656252in}{2.788592in}}%
\pgfpathlineto{\pgfqpoint{3.656252in}{2.788592in}}%
\pgfpathlineto{\pgfqpoint{3.656252in}{2.791542in}}%
\pgfpathlineto{\pgfqpoint{3.660793in}{2.791542in}}%
\pgfpathlineto{\pgfqpoint{3.660793in}{2.788592in}}%
\pgfpathmoveto{\pgfqpoint{3.656252in}{2.791542in}}%
\pgfpathlineto{\pgfqpoint{3.656252in}{2.791542in}}%
\pgfpathlineto{\pgfqpoint{3.656252in}{2.794491in}}%
\pgfpathlineto{\pgfqpoint{3.660793in}{2.794491in}}%
\pgfpathlineto{\pgfqpoint{3.660793in}{2.791542in}}%
\pgfpathmoveto{\pgfqpoint{3.660793in}{2.788592in}}%
\pgfpathlineto{\pgfqpoint{3.660793in}{2.788592in}}%
\pgfpathlineto{\pgfqpoint{3.660793in}{2.791542in}}%
\pgfpathlineto{\pgfqpoint{3.665334in}{2.791542in}}%
\pgfpathlineto{\pgfqpoint{3.665334in}{2.788592in}}%
\pgfpathmoveto{\pgfqpoint{3.660793in}{2.791542in}}%
\pgfpathlineto{\pgfqpoint{3.660793in}{2.791542in}}%
\pgfpathlineto{\pgfqpoint{3.660793in}{2.794491in}}%
\pgfpathlineto{\pgfqpoint{3.665334in}{2.794491in}}%
\pgfpathlineto{\pgfqpoint{3.665334in}{2.791542in}}%
\pgfpathmoveto{\pgfqpoint{3.692579in}{2.779745in}}%
\pgfpathlineto{\pgfqpoint{3.692579in}{2.779745in}}%
\pgfpathlineto{\pgfqpoint{3.692579in}{2.782694in}}%
\pgfpathlineto{\pgfqpoint{3.697120in}{2.782694in}}%
\pgfpathlineto{\pgfqpoint{3.697120in}{2.779745in}}%
\pgfpathmoveto{\pgfqpoint{3.697120in}{2.779745in}}%
\pgfpathlineto{\pgfqpoint{3.697120in}{2.779745in}}%
\pgfpathlineto{\pgfqpoint{3.697120in}{2.782694in}}%
\pgfpathlineto{\pgfqpoint{3.701661in}{2.782694in}}%
\pgfpathlineto{\pgfqpoint{3.701661in}{2.779745in}}%
\pgfpathmoveto{\pgfqpoint{3.701661in}{2.776795in}}%
\pgfpathlineto{\pgfqpoint{3.701661in}{2.776795in}}%
\pgfpathlineto{\pgfqpoint{3.701661in}{2.779745in}}%
\pgfpathlineto{\pgfqpoint{3.706202in}{2.779745in}}%
\pgfpathlineto{\pgfqpoint{3.706202in}{2.776795in}}%
\pgfpathmoveto{\pgfqpoint{3.701661in}{2.779745in}}%
\pgfpathlineto{\pgfqpoint{3.701661in}{2.779745in}}%
\pgfpathlineto{\pgfqpoint{3.701661in}{2.782694in}}%
\pgfpathlineto{\pgfqpoint{3.706202in}{2.782694in}}%
\pgfpathlineto{\pgfqpoint{3.706202in}{2.779745in}}%
\pgfpathmoveto{\pgfqpoint{3.706202in}{2.776795in}}%
\pgfpathlineto{\pgfqpoint{3.706202in}{2.776795in}}%
\pgfpathlineto{\pgfqpoint{3.706202in}{2.779745in}}%
\pgfpathlineto{\pgfqpoint{3.710743in}{2.779745in}}%
\pgfpathlineto{\pgfqpoint{3.710743in}{2.776795in}}%
\pgfpathmoveto{\pgfqpoint{3.706202in}{2.779745in}}%
\pgfpathlineto{\pgfqpoint{3.706202in}{2.779745in}}%
\pgfpathlineto{\pgfqpoint{3.706202in}{2.782694in}}%
\pgfpathlineto{\pgfqpoint{3.710743in}{2.782694in}}%
\pgfpathlineto{\pgfqpoint{3.710743in}{2.779745in}}%
\pgfpathmoveto{\pgfqpoint{3.715284in}{2.773846in}}%
\pgfpathlineto{\pgfqpoint{3.715284in}{2.773846in}}%
\pgfpathlineto{\pgfqpoint{3.715284in}{2.776795in}}%
\pgfpathlineto{\pgfqpoint{3.719824in}{2.776795in}}%
\pgfpathlineto{\pgfqpoint{3.719824in}{2.773846in}}%
\pgfpathmoveto{\pgfqpoint{3.719824in}{2.773846in}}%
\pgfpathlineto{\pgfqpoint{3.719824in}{2.773846in}}%
\pgfpathlineto{\pgfqpoint{3.719824in}{2.776795in}}%
\pgfpathlineto{\pgfqpoint{3.724365in}{2.776795in}}%
\pgfpathlineto{\pgfqpoint{3.724365in}{2.773846in}}%
\pgfpathmoveto{\pgfqpoint{3.724365in}{2.773846in}}%
\pgfpathlineto{\pgfqpoint{3.724365in}{2.773846in}}%
\pgfpathlineto{\pgfqpoint{3.724365in}{2.776795in}}%
\pgfpathlineto{\pgfqpoint{3.728906in}{2.776795in}}%
\pgfpathlineto{\pgfqpoint{3.728906in}{2.773846in}}%
\pgfpathmoveto{\pgfqpoint{3.710743in}{2.776795in}}%
\pgfpathlineto{\pgfqpoint{3.710743in}{2.776795in}}%
\pgfpathlineto{\pgfqpoint{3.710743in}{2.779745in}}%
\pgfpathlineto{\pgfqpoint{3.715284in}{2.779745in}}%
\pgfpathlineto{\pgfqpoint{3.715284in}{2.776795in}}%
\pgfpathmoveto{\pgfqpoint{3.710743in}{2.779745in}}%
\pgfpathlineto{\pgfqpoint{3.710743in}{2.779745in}}%
\pgfpathlineto{\pgfqpoint{3.710743in}{2.782694in}}%
\pgfpathlineto{\pgfqpoint{3.715284in}{2.782694in}}%
\pgfpathlineto{\pgfqpoint{3.715284in}{2.779745in}}%
\pgfpathmoveto{\pgfqpoint{3.715284in}{2.776795in}}%
\pgfpathlineto{\pgfqpoint{3.715284in}{2.776795in}}%
\pgfpathlineto{\pgfqpoint{3.715284in}{2.779745in}}%
\pgfpathlineto{\pgfqpoint{3.719824in}{2.779745in}}%
\pgfpathlineto{\pgfqpoint{3.719824in}{2.776795in}}%
\pgfpathmoveto{\pgfqpoint{3.715284in}{2.779745in}}%
\pgfpathlineto{\pgfqpoint{3.715284in}{2.779745in}}%
\pgfpathlineto{\pgfqpoint{3.715284in}{2.782694in}}%
\pgfpathlineto{\pgfqpoint{3.719824in}{2.782694in}}%
\pgfpathlineto{\pgfqpoint{3.719824in}{2.779745in}}%
\pgfpathmoveto{\pgfqpoint{3.728906in}{2.770897in}}%
\pgfpathlineto{\pgfqpoint{3.728906in}{2.770897in}}%
\pgfpathlineto{\pgfqpoint{3.728906in}{2.773846in}}%
\pgfpathlineto{\pgfqpoint{3.733447in}{2.773846in}}%
\pgfpathlineto{\pgfqpoint{3.733447in}{2.770897in}}%
\pgfpathmoveto{\pgfqpoint{3.728906in}{2.773846in}}%
\pgfpathlineto{\pgfqpoint{3.728906in}{2.773846in}}%
\pgfpathlineto{\pgfqpoint{3.728906in}{2.776795in}}%
\pgfpathlineto{\pgfqpoint{3.733447in}{2.776795in}}%
\pgfpathlineto{\pgfqpoint{3.733447in}{2.773846in}}%
\pgfpathmoveto{\pgfqpoint{3.733447in}{2.770897in}}%
\pgfpathlineto{\pgfqpoint{3.733447in}{2.770897in}}%
\pgfpathlineto{\pgfqpoint{3.733447in}{2.773846in}}%
\pgfpathlineto{\pgfqpoint{3.737988in}{2.773846in}}%
\pgfpathlineto{\pgfqpoint{3.737988in}{2.770897in}}%
\pgfpathmoveto{\pgfqpoint{3.733447in}{2.773846in}}%
\pgfpathlineto{\pgfqpoint{3.733447in}{2.773846in}}%
\pgfpathlineto{\pgfqpoint{3.733447in}{2.776795in}}%
\pgfpathlineto{\pgfqpoint{3.737988in}{2.776795in}}%
\pgfpathlineto{\pgfqpoint{3.737988in}{2.773846in}}%
\pgfpathmoveto{\pgfqpoint{3.742529in}{2.767947in}}%
\pgfpathlineto{\pgfqpoint{3.742529in}{2.767947in}}%
\pgfpathlineto{\pgfqpoint{3.742529in}{2.770897in}}%
\pgfpathlineto{\pgfqpoint{3.747069in}{2.770897in}}%
\pgfpathlineto{\pgfqpoint{3.747069in}{2.767947in}}%
\pgfpathmoveto{\pgfqpoint{3.737988in}{2.770897in}}%
\pgfpathlineto{\pgfqpoint{3.737988in}{2.770897in}}%
\pgfpathlineto{\pgfqpoint{3.737988in}{2.773846in}}%
\pgfpathlineto{\pgfqpoint{3.742529in}{2.773846in}}%
\pgfpathlineto{\pgfqpoint{3.742529in}{2.770897in}}%
\pgfpathmoveto{\pgfqpoint{3.737988in}{2.773846in}}%
\pgfpathlineto{\pgfqpoint{3.737988in}{2.773846in}}%
\pgfpathlineto{\pgfqpoint{3.737988in}{2.776795in}}%
\pgfpathlineto{\pgfqpoint{3.742529in}{2.776795in}}%
\pgfpathlineto{\pgfqpoint{3.742529in}{2.773846in}}%
\pgfpathmoveto{\pgfqpoint{3.742529in}{2.770897in}}%
\pgfpathlineto{\pgfqpoint{3.742529in}{2.770897in}}%
\pgfpathlineto{\pgfqpoint{3.742529in}{2.773846in}}%
\pgfpathlineto{\pgfqpoint{3.747069in}{2.773846in}}%
\pgfpathlineto{\pgfqpoint{3.747069in}{2.770897in}}%
\pgfpathmoveto{\pgfqpoint{3.742529in}{2.773846in}}%
\pgfpathlineto{\pgfqpoint{3.742529in}{2.773846in}}%
\pgfpathlineto{\pgfqpoint{3.742529in}{2.776795in}}%
\pgfpathlineto{\pgfqpoint{3.747069in}{2.776795in}}%
\pgfpathlineto{\pgfqpoint{3.747069in}{2.773846in}}%
\pgfpathmoveto{\pgfqpoint{3.747069in}{2.767947in}}%
\pgfpathlineto{\pgfqpoint{3.747069in}{2.767947in}}%
\pgfpathlineto{\pgfqpoint{3.747069in}{2.770897in}}%
\pgfpathlineto{\pgfqpoint{3.751610in}{2.770897in}}%
\pgfpathlineto{\pgfqpoint{3.751610in}{2.767947in}}%
\pgfpathmoveto{\pgfqpoint{3.751610in}{2.767947in}}%
\pgfpathlineto{\pgfqpoint{3.751610in}{2.767947in}}%
\pgfpathlineto{\pgfqpoint{3.751610in}{2.770897in}}%
\pgfpathlineto{\pgfqpoint{3.756151in}{2.770897in}}%
\pgfpathlineto{\pgfqpoint{3.756151in}{2.767947in}}%
\pgfpathmoveto{\pgfqpoint{3.756151in}{2.764998in}}%
\pgfpathlineto{\pgfqpoint{3.756151in}{2.764998in}}%
\pgfpathlineto{\pgfqpoint{3.756151in}{2.767947in}}%
\pgfpathlineto{\pgfqpoint{3.760692in}{2.767947in}}%
\pgfpathlineto{\pgfqpoint{3.760692in}{2.764998in}}%
\pgfpathmoveto{\pgfqpoint{3.756151in}{2.767947in}}%
\pgfpathlineto{\pgfqpoint{3.756151in}{2.767947in}}%
\pgfpathlineto{\pgfqpoint{3.756151in}{2.770897in}}%
\pgfpathlineto{\pgfqpoint{3.760692in}{2.770897in}}%
\pgfpathlineto{\pgfqpoint{3.760692in}{2.767947in}}%
\pgfpathmoveto{\pgfqpoint{3.760692in}{2.764998in}}%
\pgfpathlineto{\pgfqpoint{3.760692in}{2.764998in}}%
\pgfpathlineto{\pgfqpoint{3.760692in}{2.767947in}}%
\pgfpathlineto{\pgfqpoint{3.765233in}{2.767947in}}%
\pgfpathlineto{\pgfqpoint{3.765233in}{2.764998in}}%
\pgfpathmoveto{\pgfqpoint{3.760692in}{2.767947in}}%
\pgfpathlineto{\pgfqpoint{3.760692in}{2.767947in}}%
\pgfpathlineto{\pgfqpoint{3.760692in}{2.770897in}}%
\pgfpathlineto{\pgfqpoint{3.765233in}{2.770897in}}%
\pgfpathlineto{\pgfqpoint{3.765233in}{2.767947in}}%
\pgfpathmoveto{\pgfqpoint{3.765233in}{2.764998in}}%
\pgfpathlineto{\pgfqpoint{3.765233in}{2.764998in}}%
\pgfpathlineto{\pgfqpoint{3.765233in}{2.767947in}}%
\pgfpathlineto{\pgfqpoint{3.769774in}{2.767947in}}%
\pgfpathlineto{\pgfqpoint{3.769774in}{2.764998in}}%
\pgfpathmoveto{\pgfqpoint{3.765233in}{2.767947in}}%
\pgfpathlineto{\pgfqpoint{3.765233in}{2.767947in}}%
\pgfpathlineto{\pgfqpoint{3.765233in}{2.770897in}}%
\pgfpathlineto{\pgfqpoint{3.769774in}{2.770897in}}%
\pgfpathlineto{\pgfqpoint{3.769774in}{2.767947in}}%
\pgfpathmoveto{\pgfqpoint{3.769774in}{2.764998in}}%
\pgfpathlineto{\pgfqpoint{3.769774in}{2.764998in}}%
\pgfpathlineto{\pgfqpoint{3.769774in}{2.767947in}}%
\pgfpathlineto{\pgfqpoint{3.774315in}{2.767947in}}%
\pgfpathlineto{\pgfqpoint{3.774315in}{2.764998in}}%
\pgfpathmoveto{\pgfqpoint{3.769774in}{2.767947in}}%
\pgfpathlineto{\pgfqpoint{3.769774in}{2.767947in}}%
\pgfpathlineto{\pgfqpoint{3.769774in}{2.770897in}}%
\pgfpathlineto{\pgfqpoint{3.774315in}{2.770897in}}%
\pgfpathlineto{\pgfqpoint{3.774315in}{2.767947in}}%
\pgfpathmoveto{\pgfqpoint{3.683498in}{3.113007in}}%
\pgfpathlineto{\pgfqpoint{3.683498in}{3.113007in}}%
\pgfpathlineto{\pgfqpoint{3.683498in}{3.115957in}}%
\pgfpathlineto{\pgfqpoint{3.688038in}{3.115957in}}%
\pgfpathlineto{\pgfqpoint{3.688038in}{3.113007in}}%
\pgfpathmoveto{\pgfqpoint{3.683498in}{3.115957in}}%
\pgfpathlineto{\pgfqpoint{3.683498in}{3.115957in}}%
\pgfpathlineto{\pgfqpoint{3.683498in}{3.118906in}}%
\pgfpathlineto{\pgfqpoint{3.688038in}{3.118906in}}%
\pgfpathlineto{\pgfqpoint{3.688038in}{3.115957in}}%
\pgfpathmoveto{\pgfqpoint{3.688038in}{3.113007in}}%
\pgfpathlineto{\pgfqpoint{3.688038in}{3.113007in}}%
\pgfpathlineto{\pgfqpoint{3.688038in}{3.115957in}}%
\pgfpathlineto{\pgfqpoint{3.692579in}{3.115957in}}%
\pgfpathlineto{\pgfqpoint{3.692579in}{3.113007in}}%
\pgfpathmoveto{\pgfqpoint{3.688038in}{3.115957in}}%
\pgfpathlineto{\pgfqpoint{3.688038in}{3.115957in}}%
\pgfpathlineto{\pgfqpoint{3.688038in}{3.118906in}}%
\pgfpathlineto{\pgfqpoint{3.692579in}{3.118906in}}%
\pgfpathlineto{\pgfqpoint{3.692579in}{3.115957in}}%
\pgfpathmoveto{\pgfqpoint{3.665334in}{3.124804in}}%
\pgfpathlineto{\pgfqpoint{3.665334in}{3.124804in}}%
\pgfpathlineto{\pgfqpoint{3.665334in}{3.127753in}}%
\pgfpathlineto{\pgfqpoint{3.669875in}{3.127753in}}%
\pgfpathlineto{\pgfqpoint{3.669875in}{3.124804in}}%
\pgfpathmoveto{\pgfqpoint{3.665334in}{3.127753in}}%
\pgfpathlineto{\pgfqpoint{3.665334in}{3.127753in}}%
\pgfpathlineto{\pgfqpoint{3.665334in}{3.130703in}}%
\pgfpathlineto{\pgfqpoint{3.669875in}{3.130703in}}%
\pgfpathlineto{\pgfqpoint{3.669875in}{3.127753in}}%
\pgfpathmoveto{\pgfqpoint{3.669875in}{3.124804in}}%
\pgfpathlineto{\pgfqpoint{3.669875in}{3.124804in}}%
\pgfpathlineto{\pgfqpoint{3.669875in}{3.127753in}}%
\pgfpathlineto{\pgfqpoint{3.674416in}{3.127753in}}%
\pgfpathlineto{\pgfqpoint{3.674416in}{3.124804in}}%
\pgfpathmoveto{\pgfqpoint{3.669875in}{3.127753in}}%
\pgfpathlineto{\pgfqpoint{3.669875in}{3.127753in}}%
\pgfpathlineto{\pgfqpoint{3.669875in}{3.130703in}}%
\pgfpathlineto{\pgfqpoint{3.674416in}{3.130703in}}%
\pgfpathlineto{\pgfqpoint{3.674416in}{3.127753in}}%
\pgfpathmoveto{\pgfqpoint{3.656252in}{3.130703in}}%
\pgfpathlineto{\pgfqpoint{3.656252in}{3.130703in}}%
\pgfpathlineto{\pgfqpoint{3.656252in}{3.133652in}}%
\pgfpathlineto{\pgfqpoint{3.660793in}{3.133652in}}%
\pgfpathlineto{\pgfqpoint{3.660793in}{3.130703in}}%
\pgfpathmoveto{\pgfqpoint{3.656252in}{3.133652in}}%
\pgfpathlineto{\pgfqpoint{3.656252in}{3.133652in}}%
\pgfpathlineto{\pgfqpoint{3.656252in}{3.136601in}}%
\pgfpathlineto{\pgfqpoint{3.660793in}{3.136601in}}%
\pgfpathlineto{\pgfqpoint{3.660793in}{3.133652in}}%
\pgfpathmoveto{\pgfqpoint{3.660793in}{3.130703in}}%
\pgfpathlineto{\pgfqpoint{3.660793in}{3.130703in}}%
\pgfpathlineto{\pgfqpoint{3.660793in}{3.133652in}}%
\pgfpathlineto{\pgfqpoint{3.665334in}{3.133652in}}%
\pgfpathlineto{\pgfqpoint{3.665334in}{3.130703in}}%
\pgfpathmoveto{\pgfqpoint{3.660793in}{3.133652in}}%
\pgfpathlineto{\pgfqpoint{3.660793in}{3.133652in}}%
\pgfpathlineto{\pgfqpoint{3.660793in}{3.136601in}}%
\pgfpathlineto{\pgfqpoint{3.665334in}{3.136601in}}%
\pgfpathlineto{\pgfqpoint{3.665334in}{3.133652in}}%
\pgfpathmoveto{\pgfqpoint{3.656252in}{3.136601in}}%
\pgfpathlineto{\pgfqpoint{3.656252in}{3.136601in}}%
\pgfpathlineto{\pgfqpoint{3.656252in}{3.139550in}}%
\pgfpathlineto{\pgfqpoint{3.660793in}{3.139550in}}%
\pgfpathlineto{\pgfqpoint{3.660793in}{3.136601in}}%
\pgfpathmoveto{\pgfqpoint{3.656252in}{3.139550in}}%
\pgfpathlineto{\pgfqpoint{3.656252in}{3.139550in}}%
\pgfpathlineto{\pgfqpoint{3.656252in}{3.142499in}}%
\pgfpathlineto{\pgfqpoint{3.660793in}{3.142499in}}%
\pgfpathlineto{\pgfqpoint{3.660793in}{3.139550in}}%
\pgfpathmoveto{\pgfqpoint{3.660793in}{3.136601in}}%
\pgfpathlineto{\pgfqpoint{3.660793in}{3.136601in}}%
\pgfpathlineto{\pgfqpoint{3.660793in}{3.139550in}}%
\pgfpathlineto{\pgfqpoint{3.665334in}{3.139550in}}%
\pgfpathlineto{\pgfqpoint{3.665334in}{3.136601in}}%
\pgfpathmoveto{\pgfqpoint{3.665334in}{3.130703in}}%
\pgfpathlineto{\pgfqpoint{3.665334in}{3.130703in}}%
\pgfpathlineto{\pgfqpoint{3.665334in}{3.133652in}}%
\pgfpathlineto{\pgfqpoint{3.669875in}{3.133652in}}%
\pgfpathlineto{\pgfqpoint{3.669875in}{3.130703in}}%
\pgfpathmoveto{\pgfqpoint{3.665334in}{3.133652in}}%
\pgfpathlineto{\pgfqpoint{3.665334in}{3.133652in}}%
\pgfpathlineto{\pgfqpoint{3.665334in}{3.136601in}}%
\pgfpathlineto{\pgfqpoint{3.669875in}{3.136601in}}%
\pgfpathlineto{\pgfqpoint{3.669875in}{3.133652in}}%
\pgfpathmoveto{\pgfqpoint{3.669875in}{3.130703in}}%
\pgfpathlineto{\pgfqpoint{3.669875in}{3.130703in}}%
\pgfpathlineto{\pgfqpoint{3.669875in}{3.133652in}}%
\pgfpathlineto{\pgfqpoint{3.674416in}{3.133652in}}%
\pgfpathlineto{\pgfqpoint{3.674416in}{3.130703in}}%
\pgfpathmoveto{\pgfqpoint{3.674416in}{3.118906in}}%
\pgfpathlineto{\pgfqpoint{3.674416in}{3.118906in}}%
\pgfpathlineto{\pgfqpoint{3.674416in}{3.121855in}}%
\pgfpathlineto{\pgfqpoint{3.678957in}{3.121855in}}%
\pgfpathlineto{\pgfqpoint{3.678957in}{3.118906in}}%
\pgfpathmoveto{\pgfqpoint{3.674416in}{3.121855in}}%
\pgfpathlineto{\pgfqpoint{3.674416in}{3.121855in}}%
\pgfpathlineto{\pgfqpoint{3.674416in}{3.124804in}}%
\pgfpathlineto{\pgfqpoint{3.678957in}{3.124804in}}%
\pgfpathlineto{\pgfqpoint{3.678957in}{3.121855in}}%
\pgfpathmoveto{\pgfqpoint{3.678957in}{3.118906in}}%
\pgfpathlineto{\pgfqpoint{3.678957in}{3.118906in}}%
\pgfpathlineto{\pgfqpoint{3.678957in}{3.121855in}}%
\pgfpathlineto{\pgfqpoint{3.683498in}{3.121855in}}%
\pgfpathlineto{\pgfqpoint{3.683498in}{3.118906in}}%
\pgfpathmoveto{\pgfqpoint{3.678957in}{3.121855in}}%
\pgfpathlineto{\pgfqpoint{3.678957in}{3.121855in}}%
\pgfpathlineto{\pgfqpoint{3.678957in}{3.124804in}}%
\pgfpathlineto{\pgfqpoint{3.683498in}{3.124804in}}%
\pgfpathlineto{\pgfqpoint{3.683498in}{3.121855in}}%
\pgfpathmoveto{\pgfqpoint{3.674416in}{3.124804in}}%
\pgfpathlineto{\pgfqpoint{3.674416in}{3.124804in}}%
\pgfpathlineto{\pgfqpoint{3.674416in}{3.127753in}}%
\pgfpathlineto{\pgfqpoint{3.678957in}{3.127753in}}%
\pgfpathlineto{\pgfqpoint{3.678957in}{3.124804in}}%
\pgfpathmoveto{\pgfqpoint{3.674416in}{3.127753in}}%
\pgfpathlineto{\pgfqpoint{3.674416in}{3.127753in}}%
\pgfpathlineto{\pgfqpoint{3.674416in}{3.130703in}}%
\pgfpathlineto{\pgfqpoint{3.678957in}{3.130703in}}%
\pgfpathlineto{\pgfqpoint{3.678957in}{3.127753in}}%
\pgfpathmoveto{\pgfqpoint{3.678957in}{3.124804in}}%
\pgfpathlineto{\pgfqpoint{3.678957in}{3.124804in}}%
\pgfpathlineto{\pgfqpoint{3.678957in}{3.127753in}}%
\pgfpathlineto{\pgfqpoint{3.683498in}{3.127753in}}%
\pgfpathlineto{\pgfqpoint{3.683498in}{3.124804in}}%
\pgfpathmoveto{\pgfqpoint{3.683498in}{3.118906in}}%
\pgfpathlineto{\pgfqpoint{3.683498in}{3.118906in}}%
\pgfpathlineto{\pgfqpoint{3.683498in}{3.121855in}}%
\pgfpathlineto{\pgfqpoint{3.688038in}{3.121855in}}%
\pgfpathlineto{\pgfqpoint{3.688038in}{3.118906in}}%
\pgfpathmoveto{\pgfqpoint{3.683498in}{3.121855in}}%
\pgfpathlineto{\pgfqpoint{3.683498in}{3.121855in}}%
\pgfpathlineto{\pgfqpoint{3.683498in}{3.124804in}}%
\pgfpathlineto{\pgfqpoint{3.688038in}{3.124804in}}%
\pgfpathlineto{\pgfqpoint{3.688038in}{3.121855in}}%
\pgfpathmoveto{\pgfqpoint{3.688038in}{3.118906in}}%
\pgfpathlineto{\pgfqpoint{3.688038in}{3.118906in}}%
\pgfpathlineto{\pgfqpoint{3.688038in}{3.121855in}}%
\pgfpathlineto{\pgfqpoint{3.692579in}{3.121855in}}%
\pgfpathlineto{\pgfqpoint{3.692579in}{3.118906in}}%
\pgfpathmoveto{\pgfqpoint{3.692579in}{3.107109in}}%
\pgfpathlineto{\pgfqpoint{3.692579in}{3.107109in}}%
\pgfpathlineto{\pgfqpoint{3.692579in}{3.110058in}}%
\pgfpathlineto{\pgfqpoint{3.697120in}{3.110058in}}%
\pgfpathlineto{\pgfqpoint{3.697120in}{3.107109in}}%
\pgfpathmoveto{\pgfqpoint{3.692579in}{3.110058in}}%
\pgfpathlineto{\pgfqpoint{3.692579in}{3.110058in}}%
\pgfpathlineto{\pgfqpoint{3.692579in}{3.113007in}}%
\pgfpathlineto{\pgfqpoint{3.697120in}{3.113007in}}%
\pgfpathlineto{\pgfqpoint{3.697120in}{3.110058in}}%
\pgfpathmoveto{\pgfqpoint{3.697120in}{3.107109in}}%
\pgfpathlineto{\pgfqpoint{3.697120in}{3.107109in}}%
\pgfpathlineto{\pgfqpoint{3.697120in}{3.110058in}}%
\pgfpathlineto{\pgfqpoint{3.701661in}{3.110058in}}%
\pgfpathlineto{\pgfqpoint{3.701661in}{3.107109in}}%
\pgfpathmoveto{\pgfqpoint{3.697120in}{3.110058in}}%
\pgfpathlineto{\pgfqpoint{3.697120in}{3.110058in}}%
\pgfpathlineto{\pgfqpoint{3.697120in}{3.113007in}}%
\pgfpathlineto{\pgfqpoint{3.701661in}{3.113007in}}%
\pgfpathlineto{\pgfqpoint{3.701661in}{3.110058in}}%
\pgfpathmoveto{\pgfqpoint{3.692579in}{3.113007in}}%
\pgfpathlineto{\pgfqpoint{3.692579in}{3.113007in}}%
\pgfpathlineto{\pgfqpoint{3.692579in}{3.115957in}}%
\pgfpathlineto{\pgfqpoint{3.697120in}{3.115957in}}%
\pgfpathlineto{\pgfqpoint{3.697120in}{3.113007in}}%
\pgfpathmoveto{\pgfqpoint{3.692579in}{3.115957in}}%
\pgfpathlineto{\pgfqpoint{3.692579in}{3.115957in}}%
\pgfpathlineto{\pgfqpoint{3.692579in}{3.118906in}}%
\pgfpathlineto{\pgfqpoint{3.697120in}{3.118906in}}%
\pgfpathlineto{\pgfqpoint{3.697120in}{3.115957in}}%
\pgfpathmoveto{\pgfqpoint{3.697120in}{3.113007in}}%
\pgfpathlineto{\pgfqpoint{3.697120in}{3.113007in}}%
\pgfpathlineto{\pgfqpoint{3.697120in}{3.115957in}}%
\pgfpathlineto{\pgfqpoint{3.701661in}{3.115957in}}%
\pgfpathlineto{\pgfqpoint{3.701661in}{3.113007in}}%
\pgfpathmoveto{\pgfqpoint{3.701661in}{3.107109in}}%
\pgfpathlineto{\pgfqpoint{3.701661in}{3.107109in}}%
\pgfpathlineto{\pgfqpoint{3.701661in}{3.110058in}}%
\pgfpathlineto{\pgfqpoint{3.706202in}{3.110058in}}%
\pgfpathlineto{\pgfqpoint{3.706202in}{3.107109in}}%
\pgfpathmoveto{\pgfqpoint{3.701661in}{3.110058in}}%
\pgfpathlineto{\pgfqpoint{3.701661in}{3.110058in}}%
\pgfpathlineto{\pgfqpoint{3.701661in}{3.113007in}}%
\pgfpathlineto{\pgfqpoint{3.706202in}{3.113007in}}%
\pgfpathlineto{\pgfqpoint{3.706202in}{3.110058in}}%
\pgfpathmoveto{\pgfqpoint{3.706202in}{3.107109in}}%
\pgfpathlineto{\pgfqpoint{3.706202in}{3.107109in}}%
\pgfpathlineto{\pgfqpoint{3.706202in}{3.110058in}}%
\pgfpathlineto{\pgfqpoint{3.710743in}{3.110058in}}%
\pgfpathlineto{\pgfqpoint{3.710743in}{3.107109in}}%
\pgfpathmoveto{\pgfqpoint{3.706202in}{3.110058in}}%
\pgfpathlineto{\pgfqpoint{3.706202in}{3.110058in}}%
\pgfpathlineto{\pgfqpoint{3.706202in}{3.113007in}}%
\pgfpathlineto{\pgfqpoint{3.710743in}{3.113007in}}%
\pgfpathlineto{\pgfqpoint{3.710743in}{3.110058in}}%
\pgfpathmoveto{\pgfqpoint{3.710743in}{3.101210in}}%
\pgfpathlineto{\pgfqpoint{3.710743in}{3.101210in}}%
\pgfpathlineto{\pgfqpoint{3.710743in}{3.104160in}}%
\pgfpathlineto{\pgfqpoint{3.715284in}{3.104160in}}%
\pgfpathlineto{\pgfqpoint{3.715284in}{3.101210in}}%
\pgfpathmoveto{\pgfqpoint{3.710743in}{3.104160in}}%
\pgfpathlineto{\pgfqpoint{3.710743in}{3.104160in}}%
\pgfpathlineto{\pgfqpoint{3.710743in}{3.107109in}}%
\pgfpathlineto{\pgfqpoint{3.715284in}{3.107109in}}%
\pgfpathlineto{\pgfqpoint{3.715284in}{3.104160in}}%
\pgfpathmoveto{\pgfqpoint{3.715284in}{3.101210in}}%
\pgfpathlineto{\pgfqpoint{3.715284in}{3.101210in}}%
\pgfpathlineto{\pgfqpoint{3.715284in}{3.104160in}}%
\pgfpathlineto{\pgfqpoint{3.719824in}{3.104160in}}%
\pgfpathlineto{\pgfqpoint{3.719824in}{3.101210in}}%
\pgfpathmoveto{\pgfqpoint{3.715284in}{3.104160in}}%
\pgfpathlineto{\pgfqpoint{3.715284in}{3.104160in}}%
\pgfpathlineto{\pgfqpoint{3.715284in}{3.107109in}}%
\pgfpathlineto{\pgfqpoint{3.719824in}{3.107109in}}%
\pgfpathlineto{\pgfqpoint{3.719824in}{3.104160in}}%
\pgfpathmoveto{\pgfqpoint{3.719824in}{3.095312in}}%
\pgfpathlineto{\pgfqpoint{3.719824in}{3.095312in}}%
\pgfpathlineto{\pgfqpoint{3.719824in}{3.098261in}}%
\pgfpathlineto{\pgfqpoint{3.724365in}{3.098261in}}%
\pgfpathlineto{\pgfqpoint{3.724365in}{3.095312in}}%
\pgfpathmoveto{\pgfqpoint{3.719824in}{3.098261in}}%
\pgfpathlineto{\pgfqpoint{3.719824in}{3.098261in}}%
\pgfpathlineto{\pgfqpoint{3.719824in}{3.101210in}}%
\pgfpathlineto{\pgfqpoint{3.724365in}{3.101210in}}%
\pgfpathlineto{\pgfqpoint{3.724365in}{3.098261in}}%
\pgfpathmoveto{\pgfqpoint{3.724365in}{3.095312in}}%
\pgfpathlineto{\pgfqpoint{3.724365in}{3.095312in}}%
\pgfpathlineto{\pgfqpoint{3.724365in}{3.098261in}}%
\pgfpathlineto{\pgfqpoint{3.728906in}{3.098261in}}%
\pgfpathlineto{\pgfqpoint{3.728906in}{3.095312in}}%
\pgfpathmoveto{\pgfqpoint{3.724365in}{3.098261in}}%
\pgfpathlineto{\pgfqpoint{3.724365in}{3.098261in}}%
\pgfpathlineto{\pgfqpoint{3.724365in}{3.101210in}}%
\pgfpathlineto{\pgfqpoint{3.728906in}{3.101210in}}%
\pgfpathlineto{\pgfqpoint{3.728906in}{3.098261in}}%
\pgfpathmoveto{\pgfqpoint{3.719824in}{3.101210in}}%
\pgfpathlineto{\pgfqpoint{3.719824in}{3.101210in}}%
\pgfpathlineto{\pgfqpoint{3.719824in}{3.104160in}}%
\pgfpathlineto{\pgfqpoint{3.724365in}{3.104160in}}%
\pgfpathlineto{\pgfqpoint{3.724365in}{3.101210in}}%
\pgfpathmoveto{\pgfqpoint{3.710743in}{3.107109in}}%
\pgfpathlineto{\pgfqpoint{3.710743in}{3.107109in}}%
\pgfpathlineto{\pgfqpoint{3.710743in}{3.110058in}}%
\pgfpathlineto{\pgfqpoint{3.715284in}{3.110058in}}%
\pgfpathlineto{\pgfqpoint{3.715284in}{3.107109in}}%
\pgfpathmoveto{\pgfqpoint{3.728906in}{3.089414in}}%
\pgfpathlineto{\pgfqpoint{3.728906in}{3.089414in}}%
\pgfpathlineto{\pgfqpoint{3.728906in}{3.092363in}}%
\pgfpathlineto{\pgfqpoint{3.733447in}{3.092363in}}%
\pgfpathlineto{\pgfqpoint{3.733447in}{3.089414in}}%
\pgfpathmoveto{\pgfqpoint{3.728906in}{3.092363in}}%
\pgfpathlineto{\pgfqpoint{3.728906in}{3.092363in}}%
\pgfpathlineto{\pgfqpoint{3.728906in}{3.095312in}}%
\pgfpathlineto{\pgfqpoint{3.733447in}{3.095312in}}%
\pgfpathlineto{\pgfqpoint{3.733447in}{3.092363in}}%
\pgfpathmoveto{\pgfqpoint{3.733447in}{3.089414in}}%
\pgfpathlineto{\pgfqpoint{3.733447in}{3.089414in}}%
\pgfpathlineto{\pgfqpoint{3.733447in}{3.092363in}}%
\pgfpathlineto{\pgfqpoint{3.737988in}{3.092363in}}%
\pgfpathlineto{\pgfqpoint{3.737988in}{3.089414in}}%
\pgfpathmoveto{\pgfqpoint{3.733447in}{3.092363in}}%
\pgfpathlineto{\pgfqpoint{3.733447in}{3.092363in}}%
\pgfpathlineto{\pgfqpoint{3.733447in}{3.095312in}}%
\pgfpathlineto{\pgfqpoint{3.737988in}{3.095312in}}%
\pgfpathlineto{\pgfqpoint{3.737988in}{3.092363in}}%
\pgfpathmoveto{\pgfqpoint{3.737988in}{3.083515in}}%
\pgfpathlineto{\pgfqpoint{3.737988in}{3.083515in}}%
\pgfpathlineto{\pgfqpoint{3.737988in}{3.086464in}}%
\pgfpathlineto{\pgfqpoint{3.742529in}{3.086464in}}%
\pgfpathlineto{\pgfqpoint{3.742529in}{3.083515in}}%
\pgfpathmoveto{\pgfqpoint{3.737988in}{3.086464in}}%
\pgfpathlineto{\pgfqpoint{3.737988in}{3.086464in}}%
\pgfpathlineto{\pgfqpoint{3.737988in}{3.089414in}}%
\pgfpathlineto{\pgfqpoint{3.742529in}{3.089414in}}%
\pgfpathlineto{\pgfqpoint{3.742529in}{3.086464in}}%
\pgfpathmoveto{\pgfqpoint{3.742529in}{3.083515in}}%
\pgfpathlineto{\pgfqpoint{3.742529in}{3.083515in}}%
\pgfpathlineto{\pgfqpoint{3.742529in}{3.086464in}}%
\pgfpathlineto{\pgfqpoint{3.747069in}{3.086464in}}%
\pgfpathlineto{\pgfqpoint{3.747069in}{3.083515in}}%
\pgfpathmoveto{\pgfqpoint{3.742529in}{3.086464in}}%
\pgfpathlineto{\pgfqpoint{3.742529in}{3.086464in}}%
\pgfpathlineto{\pgfqpoint{3.742529in}{3.089414in}}%
\pgfpathlineto{\pgfqpoint{3.747069in}{3.089414in}}%
\pgfpathlineto{\pgfqpoint{3.747069in}{3.086464in}}%
\pgfpathmoveto{\pgfqpoint{3.737988in}{3.089414in}}%
\pgfpathlineto{\pgfqpoint{3.737988in}{3.089414in}}%
\pgfpathlineto{\pgfqpoint{3.737988in}{3.092363in}}%
\pgfpathlineto{\pgfqpoint{3.742529in}{3.092363in}}%
\pgfpathlineto{\pgfqpoint{3.742529in}{3.089414in}}%
\pgfpathmoveto{\pgfqpoint{3.747069in}{3.077617in}}%
\pgfpathlineto{\pgfqpoint{3.747069in}{3.077617in}}%
\pgfpathlineto{\pgfqpoint{3.747069in}{3.080566in}}%
\pgfpathlineto{\pgfqpoint{3.751610in}{3.080566in}}%
\pgfpathlineto{\pgfqpoint{3.751610in}{3.077617in}}%
\pgfpathmoveto{\pgfqpoint{3.747069in}{3.080566in}}%
\pgfpathlineto{\pgfqpoint{3.747069in}{3.080566in}}%
\pgfpathlineto{\pgfqpoint{3.747069in}{3.083515in}}%
\pgfpathlineto{\pgfqpoint{3.751610in}{3.083515in}}%
\pgfpathlineto{\pgfqpoint{3.751610in}{3.080566in}}%
\pgfpathmoveto{\pgfqpoint{3.751610in}{3.077617in}}%
\pgfpathlineto{\pgfqpoint{3.751610in}{3.077617in}}%
\pgfpathlineto{\pgfqpoint{3.751610in}{3.080566in}}%
\pgfpathlineto{\pgfqpoint{3.756151in}{3.080566in}}%
\pgfpathlineto{\pgfqpoint{3.756151in}{3.077617in}}%
\pgfpathmoveto{\pgfqpoint{3.751610in}{3.080566in}}%
\pgfpathlineto{\pgfqpoint{3.751610in}{3.080566in}}%
\pgfpathlineto{\pgfqpoint{3.751610in}{3.083515in}}%
\pgfpathlineto{\pgfqpoint{3.756151in}{3.083515in}}%
\pgfpathlineto{\pgfqpoint{3.756151in}{3.080566in}}%
\pgfpathmoveto{\pgfqpoint{3.756151in}{3.071718in}}%
\pgfpathlineto{\pgfqpoint{3.756151in}{3.071718in}}%
\pgfpathlineto{\pgfqpoint{3.756151in}{3.074668in}}%
\pgfpathlineto{\pgfqpoint{3.760692in}{3.074668in}}%
\pgfpathlineto{\pgfqpoint{3.760692in}{3.071718in}}%
\pgfpathmoveto{\pgfqpoint{3.756151in}{3.074668in}}%
\pgfpathlineto{\pgfqpoint{3.756151in}{3.074668in}}%
\pgfpathlineto{\pgfqpoint{3.756151in}{3.077617in}}%
\pgfpathlineto{\pgfqpoint{3.760692in}{3.077617in}}%
\pgfpathlineto{\pgfqpoint{3.760692in}{3.074668in}}%
\pgfpathmoveto{\pgfqpoint{3.760692in}{3.071718in}}%
\pgfpathlineto{\pgfqpoint{3.760692in}{3.071718in}}%
\pgfpathlineto{\pgfqpoint{3.760692in}{3.074668in}}%
\pgfpathlineto{\pgfqpoint{3.765233in}{3.074668in}}%
\pgfpathlineto{\pgfqpoint{3.765233in}{3.071718in}}%
\pgfpathmoveto{\pgfqpoint{3.760692in}{3.074668in}}%
\pgfpathlineto{\pgfqpoint{3.760692in}{3.074668in}}%
\pgfpathlineto{\pgfqpoint{3.760692in}{3.077617in}}%
\pgfpathlineto{\pgfqpoint{3.765233in}{3.077617in}}%
\pgfpathlineto{\pgfqpoint{3.765233in}{3.074668in}}%
\pgfpathmoveto{\pgfqpoint{3.756151in}{3.077617in}}%
\pgfpathlineto{\pgfqpoint{3.756151in}{3.077617in}}%
\pgfpathlineto{\pgfqpoint{3.756151in}{3.080566in}}%
\pgfpathlineto{\pgfqpoint{3.760692in}{3.080566in}}%
\pgfpathlineto{\pgfqpoint{3.760692in}{3.077617in}}%
\pgfpathmoveto{\pgfqpoint{3.747069in}{3.083515in}}%
\pgfpathlineto{\pgfqpoint{3.747069in}{3.083515in}}%
\pgfpathlineto{\pgfqpoint{3.747069in}{3.086464in}}%
\pgfpathlineto{\pgfqpoint{3.751610in}{3.086464in}}%
\pgfpathlineto{\pgfqpoint{3.751610in}{3.083515in}}%
\pgfpathmoveto{\pgfqpoint{3.765233in}{3.065820in}}%
\pgfpathlineto{\pgfqpoint{3.765233in}{3.065820in}}%
\pgfpathlineto{\pgfqpoint{3.765233in}{3.068769in}}%
\pgfpathlineto{\pgfqpoint{3.769774in}{3.068769in}}%
\pgfpathlineto{\pgfqpoint{3.769774in}{3.065820in}}%
\pgfpathmoveto{\pgfqpoint{3.765233in}{3.068769in}}%
\pgfpathlineto{\pgfqpoint{3.765233in}{3.068769in}}%
\pgfpathlineto{\pgfqpoint{3.765233in}{3.071718in}}%
\pgfpathlineto{\pgfqpoint{3.769774in}{3.071718in}}%
\pgfpathlineto{\pgfqpoint{3.769774in}{3.068769in}}%
\pgfpathmoveto{\pgfqpoint{3.769774in}{3.065820in}}%
\pgfpathlineto{\pgfqpoint{3.769774in}{3.065820in}}%
\pgfpathlineto{\pgfqpoint{3.769774in}{3.068769in}}%
\pgfpathlineto{\pgfqpoint{3.774315in}{3.068769in}}%
\pgfpathlineto{\pgfqpoint{3.774315in}{3.065820in}}%
\pgfpathmoveto{\pgfqpoint{3.769774in}{3.068769in}}%
\pgfpathlineto{\pgfqpoint{3.769774in}{3.068769in}}%
\pgfpathlineto{\pgfqpoint{3.769774in}{3.071718in}}%
\pgfpathlineto{\pgfqpoint{3.774315in}{3.071718in}}%
\pgfpathlineto{\pgfqpoint{3.774315in}{3.068769in}}%
\pgfpathmoveto{\pgfqpoint{3.774315in}{3.059922in}}%
\pgfpathlineto{\pgfqpoint{3.774315in}{3.059922in}}%
\pgfpathlineto{\pgfqpoint{3.774315in}{3.062871in}}%
\pgfpathlineto{\pgfqpoint{3.778855in}{3.062871in}}%
\pgfpathlineto{\pgfqpoint{3.778855in}{3.059922in}}%
\pgfpathmoveto{\pgfqpoint{3.774315in}{3.062871in}}%
\pgfpathlineto{\pgfqpoint{3.774315in}{3.062871in}}%
\pgfpathlineto{\pgfqpoint{3.774315in}{3.065820in}}%
\pgfpathlineto{\pgfqpoint{3.778855in}{3.065820in}}%
\pgfpathlineto{\pgfqpoint{3.778855in}{3.062871in}}%
\pgfpathmoveto{\pgfqpoint{3.778855in}{3.059922in}}%
\pgfpathlineto{\pgfqpoint{3.778855in}{3.059922in}}%
\pgfpathlineto{\pgfqpoint{3.778855in}{3.062871in}}%
\pgfpathlineto{\pgfqpoint{3.783396in}{3.062871in}}%
\pgfpathlineto{\pgfqpoint{3.783396in}{3.059922in}}%
\pgfpathmoveto{\pgfqpoint{3.778855in}{3.062871in}}%
\pgfpathlineto{\pgfqpoint{3.778855in}{3.062871in}}%
\pgfpathlineto{\pgfqpoint{3.778855in}{3.065820in}}%
\pgfpathlineto{\pgfqpoint{3.783396in}{3.065820in}}%
\pgfpathlineto{\pgfqpoint{3.783396in}{3.062871in}}%
\pgfpathmoveto{\pgfqpoint{3.774315in}{3.065820in}}%
\pgfpathlineto{\pgfqpoint{3.774315in}{3.065820in}}%
\pgfpathlineto{\pgfqpoint{3.774315in}{3.068769in}}%
\pgfpathlineto{\pgfqpoint{3.778855in}{3.068769in}}%
\pgfpathlineto{\pgfqpoint{3.778855in}{3.065820in}}%
\pgfpathmoveto{\pgfqpoint{3.783396in}{3.054023in}}%
\pgfpathlineto{\pgfqpoint{3.783396in}{3.054023in}}%
\pgfpathlineto{\pgfqpoint{3.783396in}{3.056972in}}%
\pgfpathlineto{\pgfqpoint{3.787937in}{3.056972in}}%
\pgfpathlineto{\pgfqpoint{3.787937in}{3.054023in}}%
\pgfpathmoveto{\pgfqpoint{3.783396in}{3.056972in}}%
\pgfpathlineto{\pgfqpoint{3.783396in}{3.056972in}}%
\pgfpathlineto{\pgfqpoint{3.783396in}{3.059922in}}%
\pgfpathlineto{\pgfqpoint{3.787937in}{3.059922in}}%
\pgfpathlineto{\pgfqpoint{3.787937in}{3.056972in}}%
\pgfpathmoveto{\pgfqpoint{3.787937in}{3.054023in}}%
\pgfpathlineto{\pgfqpoint{3.787937in}{3.054023in}}%
\pgfpathlineto{\pgfqpoint{3.787937in}{3.056972in}}%
\pgfpathlineto{\pgfqpoint{3.792478in}{3.056972in}}%
\pgfpathlineto{\pgfqpoint{3.792478in}{3.054023in}}%
\pgfpathmoveto{\pgfqpoint{3.787937in}{3.056972in}}%
\pgfpathlineto{\pgfqpoint{3.787937in}{3.056972in}}%
\pgfpathlineto{\pgfqpoint{3.787937in}{3.059922in}}%
\pgfpathlineto{\pgfqpoint{3.792478in}{3.059922in}}%
\pgfpathlineto{\pgfqpoint{3.792478in}{3.056972in}}%
\pgfpathmoveto{\pgfqpoint{3.792478in}{3.048125in}}%
\pgfpathlineto{\pgfqpoint{3.792478in}{3.048125in}}%
\pgfpathlineto{\pgfqpoint{3.792478in}{3.051074in}}%
\pgfpathlineto{\pgfqpoint{3.797019in}{3.051074in}}%
\pgfpathlineto{\pgfqpoint{3.797019in}{3.048125in}}%
\pgfpathmoveto{\pgfqpoint{3.792478in}{3.051074in}}%
\pgfpathlineto{\pgfqpoint{3.792478in}{3.051074in}}%
\pgfpathlineto{\pgfqpoint{3.792478in}{3.054023in}}%
\pgfpathlineto{\pgfqpoint{3.797019in}{3.054023in}}%
\pgfpathlineto{\pgfqpoint{3.797019in}{3.051074in}}%
\pgfpathmoveto{\pgfqpoint{3.797019in}{3.048125in}}%
\pgfpathlineto{\pgfqpoint{3.797019in}{3.048125in}}%
\pgfpathlineto{\pgfqpoint{3.797019in}{3.051074in}}%
\pgfpathlineto{\pgfqpoint{3.801560in}{3.051074in}}%
\pgfpathlineto{\pgfqpoint{3.801560in}{3.048125in}}%
\pgfpathmoveto{\pgfqpoint{3.797019in}{3.051074in}}%
\pgfpathlineto{\pgfqpoint{3.797019in}{3.051074in}}%
\pgfpathlineto{\pgfqpoint{3.797019in}{3.054023in}}%
\pgfpathlineto{\pgfqpoint{3.801560in}{3.054023in}}%
\pgfpathlineto{\pgfqpoint{3.801560in}{3.051074in}}%
\pgfpathmoveto{\pgfqpoint{3.792478in}{3.054023in}}%
\pgfpathlineto{\pgfqpoint{3.792478in}{3.054023in}}%
\pgfpathlineto{\pgfqpoint{3.792478in}{3.056972in}}%
\pgfpathlineto{\pgfqpoint{3.797019in}{3.056972in}}%
\pgfpathlineto{\pgfqpoint{3.797019in}{3.054023in}}%
\pgfpathmoveto{\pgfqpoint{3.783396in}{3.059922in}}%
\pgfpathlineto{\pgfqpoint{3.783396in}{3.059922in}}%
\pgfpathlineto{\pgfqpoint{3.783396in}{3.062871in}}%
\pgfpathlineto{\pgfqpoint{3.787937in}{3.062871in}}%
\pgfpathlineto{\pgfqpoint{3.787937in}{3.059922in}}%
\pgfpathmoveto{\pgfqpoint{3.765233in}{3.071718in}}%
\pgfpathlineto{\pgfqpoint{3.765233in}{3.071718in}}%
\pgfpathlineto{\pgfqpoint{3.765233in}{3.074668in}}%
\pgfpathlineto{\pgfqpoint{3.769774in}{3.074668in}}%
\pgfpathlineto{\pgfqpoint{3.769774in}{3.071718in}}%
\pgfpathmoveto{\pgfqpoint{3.728906in}{3.095312in}}%
\pgfpathlineto{\pgfqpoint{3.728906in}{3.095312in}}%
\pgfpathlineto{\pgfqpoint{3.728906in}{3.098261in}}%
\pgfpathlineto{\pgfqpoint{3.733447in}{3.098261in}}%
\pgfpathlineto{\pgfqpoint{3.733447in}{3.095312in}}%
\pgfpathmoveto{\pgfqpoint{3.801560in}{2.756151in}}%
\pgfpathlineto{\pgfqpoint{3.801560in}{2.756151in}}%
\pgfpathlineto{\pgfqpoint{3.801560in}{2.759100in}}%
\pgfpathlineto{\pgfqpoint{3.806101in}{2.759100in}}%
\pgfpathlineto{\pgfqpoint{3.806101in}{2.756151in}}%
\pgfpathmoveto{\pgfqpoint{3.806101in}{2.756151in}}%
\pgfpathlineto{\pgfqpoint{3.806101in}{2.756151in}}%
\pgfpathlineto{\pgfqpoint{3.806101in}{2.759100in}}%
\pgfpathlineto{\pgfqpoint{3.810642in}{2.759100in}}%
\pgfpathlineto{\pgfqpoint{3.810642in}{2.756151in}}%
\pgfpathmoveto{\pgfqpoint{3.810642in}{2.753201in}}%
\pgfpathlineto{\pgfqpoint{3.810642in}{2.753201in}}%
\pgfpathlineto{\pgfqpoint{3.810642in}{2.756151in}}%
\pgfpathlineto{\pgfqpoint{3.815183in}{2.756151in}}%
\pgfpathlineto{\pgfqpoint{3.815183in}{2.753201in}}%
\pgfpathmoveto{\pgfqpoint{3.810642in}{2.756151in}}%
\pgfpathlineto{\pgfqpoint{3.810642in}{2.756151in}}%
\pgfpathlineto{\pgfqpoint{3.810642in}{2.759100in}}%
\pgfpathlineto{\pgfqpoint{3.815183in}{2.759100in}}%
\pgfpathlineto{\pgfqpoint{3.815183in}{2.756151in}}%
\pgfpathmoveto{\pgfqpoint{3.815183in}{2.753201in}}%
\pgfpathlineto{\pgfqpoint{3.815183in}{2.753201in}}%
\pgfpathlineto{\pgfqpoint{3.815183in}{2.756151in}}%
\pgfpathlineto{\pgfqpoint{3.819724in}{2.756151in}}%
\pgfpathlineto{\pgfqpoint{3.819724in}{2.753201in}}%
\pgfpathmoveto{\pgfqpoint{3.815183in}{2.756151in}}%
\pgfpathlineto{\pgfqpoint{3.815183in}{2.756151in}}%
\pgfpathlineto{\pgfqpoint{3.815183in}{2.759100in}}%
\pgfpathlineto{\pgfqpoint{3.819724in}{2.759100in}}%
\pgfpathlineto{\pgfqpoint{3.819724in}{2.756151in}}%
\pgfpathmoveto{\pgfqpoint{3.824265in}{2.750252in}}%
\pgfpathlineto{\pgfqpoint{3.824265in}{2.750252in}}%
\pgfpathlineto{\pgfqpoint{3.824265in}{2.753201in}}%
\pgfpathlineto{\pgfqpoint{3.828806in}{2.753201in}}%
\pgfpathlineto{\pgfqpoint{3.828806in}{2.750252in}}%
\pgfpathmoveto{\pgfqpoint{3.828806in}{2.750252in}}%
\pgfpathlineto{\pgfqpoint{3.828806in}{2.750252in}}%
\pgfpathlineto{\pgfqpoint{3.828806in}{2.753201in}}%
\pgfpathlineto{\pgfqpoint{3.833347in}{2.753201in}}%
\pgfpathlineto{\pgfqpoint{3.833347in}{2.750252in}}%
\pgfpathmoveto{\pgfqpoint{3.833347in}{2.750252in}}%
\pgfpathlineto{\pgfqpoint{3.833347in}{2.750252in}}%
\pgfpathlineto{\pgfqpoint{3.833347in}{2.753201in}}%
\pgfpathlineto{\pgfqpoint{3.837888in}{2.753201in}}%
\pgfpathlineto{\pgfqpoint{3.837888in}{2.750252in}}%
\pgfpathmoveto{\pgfqpoint{3.819724in}{2.753201in}}%
\pgfpathlineto{\pgfqpoint{3.819724in}{2.753201in}}%
\pgfpathlineto{\pgfqpoint{3.819724in}{2.756151in}}%
\pgfpathlineto{\pgfqpoint{3.824265in}{2.756151in}}%
\pgfpathlineto{\pgfqpoint{3.824265in}{2.753201in}}%
\pgfpathmoveto{\pgfqpoint{3.819724in}{2.756151in}}%
\pgfpathlineto{\pgfqpoint{3.819724in}{2.756151in}}%
\pgfpathlineto{\pgfqpoint{3.819724in}{2.759100in}}%
\pgfpathlineto{\pgfqpoint{3.824265in}{2.759100in}}%
\pgfpathlineto{\pgfqpoint{3.824265in}{2.756151in}}%
\pgfpathmoveto{\pgfqpoint{3.824265in}{2.753201in}}%
\pgfpathlineto{\pgfqpoint{3.824265in}{2.753201in}}%
\pgfpathlineto{\pgfqpoint{3.824265in}{2.756151in}}%
\pgfpathlineto{\pgfqpoint{3.828806in}{2.756151in}}%
\pgfpathlineto{\pgfqpoint{3.828806in}{2.753201in}}%
\pgfpathmoveto{\pgfqpoint{3.824265in}{2.756151in}}%
\pgfpathlineto{\pgfqpoint{3.824265in}{2.756151in}}%
\pgfpathlineto{\pgfqpoint{3.824265in}{2.759100in}}%
\pgfpathlineto{\pgfqpoint{3.828806in}{2.759100in}}%
\pgfpathlineto{\pgfqpoint{3.828806in}{2.756151in}}%
\pgfpathmoveto{\pgfqpoint{3.837888in}{2.747303in}}%
\pgfpathlineto{\pgfqpoint{3.837888in}{2.747303in}}%
\pgfpathlineto{\pgfqpoint{3.837888in}{2.750252in}}%
\pgfpathlineto{\pgfqpoint{3.842429in}{2.750252in}}%
\pgfpathlineto{\pgfqpoint{3.842429in}{2.747303in}}%
\pgfpathmoveto{\pgfqpoint{3.837888in}{2.750252in}}%
\pgfpathlineto{\pgfqpoint{3.837888in}{2.750252in}}%
\pgfpathlineto{\pgfqpoint{3.837888in}{2.753201in}}%
\pgfpathlineto{\pgfqpoint{3.842429in}{2.753201in}}%
\pgfpathlineto{\pgfqpoint{3.842429in}{2.750252in}}%
\pgfpathmoveto{\pgfqpoint{3.842429in}{2.747303in}}%
\pgfpathlineto{\pgfqpoint{3.842429in}{2.747303in}}%
\pgfpathlineto{\pgfqpoint{3.842429in}{2.750252in}}%
\pgfpathlineto{\pgfqpoint{3.846970in}{2.750252in}}%
\pgfpathlineto{\pgfqpoint{3.846970in}{2.747303in}}%
\pgfpathmoveto{\pgfqpoint{3.842429in}{2.750252in}}%
\pgfpathlineto{\pgfqpoint{3.842429in}{2.750252in}}%
\pgfpathlineto{\pgfqpoint{3.842429in}{2.753201in}}%
\pgfpathlineto{\pgfqpoint{3.846970in}{2.753201in}}%
\pgfpathlineto{\pgfqpoint{3.846970in}{2.750252in}}%
\pgfpathmoveto{\pgfqpoint{3.851511in}{2.744354in}}%
\pgfpathlineto{\pgfqpoint{3.851511in}{2.744354in}}%
\pgfpathlineto{\pgfqpoint{3.851511in}{2.747303in}}%
\pgfpathlineto{\pgfqpoint{3.856052in}{2.747303in}}%
\pgfpathlineto{\pgfqpoint{3.856052in}{2.744354in}}%
\pgfpathmoveto{\pgfqpoint{3.846970in}{2.747303in}}%
\pgfpathlineto{\pgfqpoint{3.846970in}{2.747303in}}%
\pgfpathlineto{\pgfqpoint{3.846970in}{2.750252in}}%
\pgfpathlineto{\pgfqpoint{3.851511in}{2.750252in}}%
\pgfpathlineto{\pgfqpoint{3.851511in}{2.747303in}}%
\pgfpathmoveto{\pgfqpoint{3.846970in}{2.750252in}}%
\pgfpathlineto{\pgfqpoint{3.846970in}{2.750252in}}%
\pgfpathlineto{\pgfqpoint{3.846970in}{2.753201in}}%
\pgfpathlineto{\pgfqpoint{3.851511in}{2.753201in}}%
\pgfpathlineto{\pgfqpoint{3.851511in}{2.750252in}}%
\pgfpathmoveto{\pgfqpoint{3.851511in}{2.747303in}}%
\pgfpathlineto{\pgfqpoint{3.851511in}{2.747303in}}%
\pgfpathlineto{\pgfqpoint{3.851511in}{2.750252in}}%
\pgfpathlineto{\pgfqpoint{3.856052in}{2.750252in}}%
\pgfpathlineto{\pgfqpoint{3.856052in}{2.747303in}}%
\pgfpathmoveto{\pgfqpoint{3.851511in}{2.750252in}}%
\pgfpathlineto{\pgfqpoint{3.851511in}{2.750252in}}%
\pgfpathlineto{\pgfqpoint{3.851511in}{2.753201in}}%
\pgfpathlineto{\pgfqpoint{3.856052in}{2.753201in}}%
\pgfpathlineto{\pgfqpoint{3.856052in}{2.750252in}}%
\pgfpathmoveto{\pgfqpoint{3.856052in}{2.744354in}}%
\pgfpathlineto{\pgfqpoint{3.856052in}{2.744354in}}%
\pgfpathlineto{\pgfqpoint{3.856052in}{2.747303in}}%
\pgfpathlineto{\pgfqpoint{3.860593in}{2.747303in}}%
\pgfpathlineto{\pgfqpoint{3.860593in}{2.744354in}}%
\pgfpathmoveto{\pgfqpoint{3.860593in}{2.744354in}}%
\pgfpathlineto{\pgfqpoint{3.860593in}{2.744354in}}%
\pgfpathlineto{\pgfqpoint{3.860593in}{2.747303in}}%
\pgfpathlineto{\pgfqpoint{3.865134in}{2.747303in}}%
\pgfpathlineto{\pgfqpoint{3.865134in}{2.744354in}}%
\pgfpathmoveto{\pgfqpoint{3.865134in}{2.741405in}}%
\pgfpathlineto{\pgfqpoint{3.865134in}{2.741405in}}%
\pgfpathlineto{\pgfqpoint{3.865134in}{2.744354in}}%
\pgfpathlineto{\pgfqpoint{3.869675in}{2.744354in}}%
\pgfpathlineto{\pgfqpoint{3.869675in}{2.741405in}}%
\pgfpathmoveto{\pgfqpoint{3.865134in}{2.744354in}}%
\pgfpathlineto{\pgfqpoint{3.865134in}{2.744354in}}%
\pgfpathlineto{\pgfqpoint{3.865134in}{2.747303in}}%
\pgfpathlineto{\pgfqpoint{3.869675in}{2.747303in}}%
\pgfpathlineto{\pgfqpoint{3.869675in}{2.744354in}}%
\pgfpathmoveto{\pgfqpoint{3.869675in}{2.741405in}}%
\pgfpathlineto{\pgfqpoint{3.869675in}{2.741405in}}%
\pgfpathlineto{\pgfqpoint{3.869675in}{2.744354in}}%
\pgfpathlineto{\pgfqpoint{3.874216in}{2.744354in}}%
\pgfpathlineto{\pgfqpoint{3.874216in}{2.741405in}}%
\pgfpathmoveto{\pgfqpoint{3.869675in}{2.744354in}}%
\pgfpathlineto{\pgfqpoint{3.869675in}{2.744354in}}%
\pgfpathlineto{\pgfqpoint{3.869675in}{2.747303in}}%
\pgfpathlineto{\pgfqpoint{3.874216in}{2.747303in}}%
\pgfpathlineto{\pgfqpoint{3.874216in}{2.744354in}}%
\pgfpathmoveto{\pgfqpoint{3.878757in}{2.738455in}}%
\pgfpathlineto{\pgfqpoint{3.878757in}{2.738455in}}%
\pgfpathlineto{\pgfqpoint{3.878757in}{2.741405in}}%
\pgfpathlineto{\pgfqpoint{3.883298in}{2.741405in}}%
\pgfpathlineto{\pgfqpoint{3.883298in}{2.738455in}}%
\pgfpathmoveto{\pgfqpoint{3.883298in}{2.738455in}}%
\pgfpathlineto{\pgfqpoint{3.883298in}{2.738455in}}%
\pgfpathlineto{\pgfqpoint{3.883298in}{2.741405in}}%
\pgfpathlineto{\pgfqpoint{3.887839in}{2.741405in}}%
\pgfpathlineto{\pgfqpoint{3.887839in}{2.738455in}}%
\pgfpathmoveto{\pgfqpoint{3.887839in}{2.738455in}}%
\pgfpathlineto{\pgfqpoint{3.887839in}{2.738455in}}%
\pgfpathlineto{\pgfqpoint{3.887839in}{2.741405in}}%
\pgfpathlineto{\pgfqpoint{3.892380in}{2.741405in}}%
\pgfpathlineto{\pgfqpoint{3.892380in}{2.738455in}}%
\pgfpathmoveto{\pgfqpoint{3.892380in}{2.735506in}}%
\pgfpathlineto{\pgfqpoint{3.892380in}{2.735506in}}%
\pgfpathlineto{\pgfqpoint{3.892380in}{2.738455in}}%
\pgfpathlineto{\pgfqpoint{3.896921in}{2.738455in}}%
\pgfpathlineto{\pgfqpoint{3.896921in}{2.735506in}}%
\pgfpathmoveto{\pgfqpoint{3.892380in}{2.738455in}}%
\pgfpathlineto{\pgfqpoint{3.892380in}{2.738455in}}%
\pgfpathlineto{\pgfqpoint{3.892380in}{2.741405in}}%
\pgfpathlineto{\pgfqpoint{3.896921in}{2.741405in}}%
\pgfpathlineto{\pgfqpoint{3.896921in}{2.738455in}}%
\pgfpathmoveto{\pgfqpoint{3.896921in}{2.735506in}}%
\pgfpathlineto{\pgfqpoint{3.896921in}{2.735506in}}%
\pgfpathlineto{\pgfqpoint{3.896921in}{2.738455in}}%
\pgfpathlineto{\pgfqpoint{3.901462in}{2.738455in}}%
\pgfpathlineto{\pgfqpoint{3.901462in}{2.735506in}}%
\pgfpathmoveto{\pgfqpoint{3.896921in}{2.738455in}}%
\pgfpathlineto{\pgfqpoint{3.896921in}{2.738455in}}%
\pgfpathlineto{\pgfqpoint{3.896921in}{2.741405in}}%
\pgfpathlineto{\pgfqpoint{3.901462in}{2.741405in}}%
\pgfpathlineto{\pgfqpoint{3.901462in}{2.738455in}}%
\pgfpathmoveto{\pgfqpoint{3.906003in}{2.732557in}}%
\pgfpathlineto{\pgfqpoint{3.906003in}{2.732557in}}%
\pgfpathlineto{\pgfqpoint{3.906003in}{2.735506in}}%
\pgfpathlineto{\pgfqpoint{3.910544in}{2.735506in}}%
\pgfpathlineto{\pgfqpoint{3.910544in}{2.732557in}}%
\pgfpathmoveto{\pgfqpoint{3.901462in}{2.735506in}}%
\pgfpathlineto{\pgfqpoint{3.901462in}{2.735506in}}%
\pgfpathlineto{\pgfqpoint{3.901462in}{2.738455in}}%
\pgfpathlineto{\pgfqpoint{3.906003in}{2.738455in}}%
\pgfpathlineto{\pgfqpoint{3.906003in}{2.735506in}}%
\pgfpathmoveto{\pgfqpoint{3.901462in}{2.738455in}}%
\pgfpathlineto{\pgfqpoint{3.901462in}{2.738455in}}%
\pgfpathlineto{\pgfqpoint{3.901462in}{2.741405in}}%
\pgfpathlineto{\pgfqpoint{3.906003in}{2.741405in}}%
\pgfpathlineto{\pgfqpoint{3.906003in}{2.738455in}}%
\pgfpathmoveto{\pgfqpoint{3.906003in}{2.735506in}}%
\pgfpathlineto{\pgfqpoint{3.906003in}{2.735506in}}%
\pgfpathlineto{\pgfqpoint{3.906003in}{2.738455in}}%
\pgfpathlineto{\pgfqpoint{3.910544in}{2.738455in}}%
\pgfpathlineto{\pgfqpoint{3.910544in}{2.735506in}}%
\pgfpathmoveto{\pgfqpoint{3.906003in}{2.738455in}}%
\pgfpathlineto{\pgfqpoint{3.906003in}{2.738455in}}%
\pgfpathlineto{\pgfqpoint{3.906003in}{2.741405in}}%
\pgfpathlineto{\pgfqpoint{3.910544in}{2.741405in}}%
\pgfpathlineto{\pgfqpoint{3.910544in}{2.738455in}}%
\pgfpathmoveto{\pgfqpoint{3.874216in}{2.741405in}}%
\pgfpathlineto{\pgfqpoint{3.874216in}{2.741405in}}%
\pgfpathlineto{\pgfqpoint{3.874216in}{2.744354in}}%
\pgfpathlineto{\pgfqpoint{3.878757in}{2.744354in}}%
\pgfpathlineto{\pgfqpoint{3.878757in}{2.741405in}}%
\pgfpathmoveto{\pgfqpoint{3.874216in}{2.744354in}}%
\pgfpathlineto{\pgfqpoint{3.874216in}{2.744354in}}%
\pgfpathlineto{\pgfqpoint{3.874216in}{2.747303in}}%
\pgfpathlineto{\pgfqpoint{3.878757in}{2.747303in}}%
\pgfpathlineto{\pgfqpoint{3.878757in}{2.744354in}}%
\pgfpathmoveto{\pgfqpoint{3.878757in}{2.741405in}}%
\pgfpathlineto{\pgfqpoint{3.878757in}{2.741405in}}%
\pgfpathlineto{\pgfqpoint{3.878757in}{2.744354in}}%
\pgfpathlineto{\pgfqpoint{3.883298in}{2.744354in}}%
\pgfpathlineto{\pgfqpoint{3.883298in}{2.741405in}}%
\pgfpathmoveto{\pgfqpoint{3.878757in}{2.744354in}}%
\pgfpathlineto{\pgfqpoint{3.878757in}{2.744354in}}%
\pgfpathlineto{\pgfqpoint{3.878757in}{2.747303in}}%
\pgfpathlineto{\pgfqpoint{3.883298in}{2.747303in}}%
\pgfpathlineto{\pgfqpoint{3.883298in}{2.744354in}}%
\pgfpathmoveto{\pgfqpoint{3.910544in}{2.732557in}}%
\pgfpathlineto{\pgfqpoint{3.910544in}{2.732557in}}%
\pgfpathlineto{\pgfqpoint{3.910544in}{2.735506in}}%
\pgfpathlineto{\pgfqpoint{3.915086in}{2.735506in}}%
\pgfpathlineto{\pgfqpoint{3.915086in}{2.732557in}}%
\pgfpathmoveto{\pgfqpoint{3.915086in}{2.732557in}}%
\pgfpathlineto{\pgfqpoint{3.915086in}{2.732557in}}%
\pgfpathlineto{\pgfqpoint{3.915086in}{2.735506in}}%
\pgfpathlineto{\pgfqpoint{3.919627in}{2.735506in}}%
\pgfpathlineto{\pgfqpoint{3.919627in}{2.732557in}}%
\pgfpathmoveto{\pgfqpoint{3.919627in}{2.729608in}}%
\pgfpathlineto{\pgfqpoint{3.919627in}{2.729608in}}%
\pgfpathlineto{\pgfqpoint{3.919627in}{2.732557in}}%
\pgfpathlineto{\pgfqpoint{3.924168in}{2.732557in}}%
\pgfpathlineto{\pgfqpoint{3.924168in}{2.729608in}}%
\pgfpathmoveto{\pgfqpoint{3.919627in}{2.732557in}}%
\pgfpathlineto{\pgfqpoint{3.919627in}{2.732557in}}%
\pgfpathlineto{\pgfqpoint{3.919627in}{2.735506in}}%
\pgfpathlineto{\pgfqpoint{3.924168in}{2.735506in}}%
\pgfpathlineto{\pgfqpoint{3.924168in}{2.732557in}}%
\pgfpathmoveto{\pgfqpoint{3.924168in}{2.729608in}}%
\pgfpathlineto{\pgfqpoint{3.924168in}{2.729608in}}%
\pgfpathlineto{\pgfqpoint{3.924168in}{2.732557in}}%
\pgfpathlineto{\pgfqpoint{3.928709in}{2.732557in}}%
\pgfpathlineto{\pgfqpoint{3.928709in}{2.729608in}}%
\pgfpathmoveto{\pgfqpoint{3.924168in}{2.732557in}}%
\pgfpathlineto{\pgfqpoint{3.924168in}{2.732557in}}%
\pgfpathlineto{\pgfqpoint{3.924168in}{2.735506in}}%
\pgfpathlineto{\pgfqpoint{3.928709in}{2.735506in}}%
\pgfpathlineto{\pgfqpoint{3.928709in}{2.732557in}}%
\pgfpathmoveto{\pgfqpoint{3.933250in}{2.726659in}}%
\pgfpathlineto{\pgfqpoint{3.933250in}{2.726659in}}%
\pgfpathlineto{\pgfqpoint{3.933250in}{2.729608in}}%
\pgfpathlineto{\pgfqpoint{3.937791in}{2.729608in}}%
\pgfpathlineto{\pgfqpoint{3.937791in}{2.726659in}}%
\pgfpathmoveto{\pgfqpoint{3.937791in}{2.726659in}}%
\pgfpathlineto{\pgfqpoint{3.937791in}{2.726659in}}%
\pgfpathlineto{\pgfqpoint{3.937791in}{2.729608in}}%
\pgfpathlineto{\pgfqpoint{3.942332in}{2.729608in}}%
\pgfpathlineto{\pgfqpoint{3.942332in}{2.726659in}}%
\pgfpathmoveto{\pgfqpoint{3.942332in}{2.726659in}}%
\pgfpathlineto{\pgfqpoint{3.942332in}{2.726659in}}%
\pgfpathlineto{\pgfqpoint{3.942332in}{2.729608in}}%
\pgfpathlineto{\pgfqpoint{3.946873in}{2.729608in}}%
\pgfpathlineto{\pgfqpoint{3.946873in}{2.726659in}}%
\pgfpathmoveto{\pgfqpoint{3.928709in}{2.729608in}}%
\pgfpathlineto{\pgfqpoint{3.928709in}{2.729608in}}%
\pgfpathlineto{\pgfqpoint{3.928709in}{2.732557in}}%
\pgfpathlineto{\pgfqpoint{3.933250in}{2.732557in}}%
\pgfpathlineto{\pgfqpoint{3.933250in}{2.729608in}}%
\pgfpathmoveto{\pgfqpoint{3.928709in}{2.732557in}}%
\pgfpathlineto{\pgfqpoint{3.928709in}{2.732557in}}%
\pgfpathlineto{\pgfqpoint{3.928709in}{2.735506in}}%
\pgfpathlineto{\pgfqpoint{3.933250in}{2.735506in}}%
\pgfpathlineto{\pgfqpoint{3.933250in}{2.732557in}}%
\pgfpathmoveto{\pgfqpoint{3.933250in}{2.729608in}}%
\pgfpathlineto{\pgfqpoint{3.933250in}{2.729608in}}%
\pgfpathlineto{\pgfqpoint{3.933250in}{2.732557in}}%
\pgfpathlineto{\pgfqpoint{3.937791in}{2.732557in}}%
\pgfpathlineto{\pgfqpoint{3.937791in}{2.729608in}}%
\pgfpathmoveto{\pgfqpoint{3.933250in}{2.732557in}}%
\pgfpathlineto{\pgfqpoint{3.933250in}{2.732557in}}%
\pgfpathlineto{\pgfqpoint{3.933250in}{2.735506in}}%
\pgfpathlineto{\pgfqpoint{3.937791in}{2.735506in}}%
\pgfpathlineto{\pgfqpoint{3.937791in}{2.732557in}}%
\pgfpathmoveto{\pgfqpoint{3.937791in}{2.947853in}}%
\pgfpathlineto{\pgfqpoint{3.937791in}{2.947853in}}%
\pgfpathlineto{\pgfqpoint{3.937791in}{2.950803in}}%
\pgfpathlineto{\pgfqpoint{3.942332in}{2.950803in}}%
\pgfpathlineto{\pgfqpoint{3.942332in}{2.947853in}}%
\pgfpathmoveto{\pgfqpoint{3.937791in}{2.950803in}}%
\pgfpathlineto{\pgfqpoint{3.937791in}{2.950803in}}%
\pgfpathlineto{\pgfqpoint{3.937791in}{2.953752in}}%
\pgfpathlineto{\pgfqpoint{3.942332in}{2.953752in}}%
\pgfpathlineto{\pgfqpoint{3.942332in}{2.950803in}}%
\pgfpathmoveto{\pgfqpoint{3.942332in}{2.947853in}}%
\pgfpathlineto{\pgfqpoint{3.942332in}{2.947853in}}%
\pgfpathlineto{\pgfqpoint{3.942332in}{2.950803in}}%
\pgfpathlineto{\pgfqpoint{3.946873in}{2.950803in}}%
\pgfpathlineto{\pgfqpoint{3.946873in}{2.947853in}}%
\pgfpathmoveto{\pgfqpoint{3.942332in}{2.950803in}}%
\pgfpathlineto{\pgfqpoint{3.942332in}{2.950803in}}%
\pgfpathlineto{\pgfqpoint{3.942332in}{2.953752in}}%
\pgfpathlineto{\pgfqpoint{3.946873in}{2.953752in}}%
\pgfpathlineto{\pgfqpoint{3.946873in}{2.950803in}}%
\pgfpathmoveto{\pgfqpoint{3.801560in}{3.042226in}}%
\pgfpathlineto{\pgfqpoint{3.801560in}{3.042226in}}%
\pgfpathlineto{\pgfqpoint{3.801560in}{3.045176in}}%
\pgfpathlineto{\pgfqpoint{3.806101in}{3.045176in}}%
\pgfpathlineto{\pgfqpoint{3.806101in}{3.042226in}}%
\pgfpathmoveto{\pgfqpoint{3.801560in}{3.045176in}}%
\pgfpathlineto{\pgfqpoint{3.801560in}{3.045176in}}%
\pgfpathlineto{\pgfqpoint{3.801560in}{3.048125in}}%
\pgfpathlineto{\pgfqpoint{3.806101in}{3.048125in}}%
\pgfpathlineto{\pgfqpoint{3.806101in}{3.045176in}}%
\pgfpathmoveto{\pgfqpoint{3.806101in}{3.042226in}}%
\pgfpathlineto{\pgfqpoint{3.806101in}{3.042226in}}%
\pgfpathlineto{\pgfqpoint{3.806101in}{3.045176in}}%
\pgfpathlineto{\pgfqpoint{3.810642in}{3.045176in}}%
\pgfpathlineto{\pgfqpoint{3.810642in}{3.042226in}}%
\pgfpathmoveto{\pgfqpoint{3.806101in}{3.045176in}}%
\pgfpathlineto{\pgfqpoint{3.806101in}{3.045176in}}%
\pgfpathlineto{\pgfqpoint{3.806101in}{3.048125in}}%
\pgfpathlineto{\pgfqpoint{3.810642in}{3.048125in}}%
\pgfpathlineto{\pgfqpoint{3.810642in}{3.045176in}}%
\pgfpathmoveto{\pgfqpoint{3.810642in}{3.036328in}}%
\pgfpathlineto{\pgfqpoint{3.810642in}{3.036328in}}%
\pgfpathlineto{\pgfqpoint{3.810642in}{3.039277in}}%
\pgfpathlineto{\pgfqpoint{3.815183in}{3.039277in}}%
\pgfpathlineto{\pgfqpoint{3.815183in}{3.036328in}}%
\pgfpathmoveto{\pgfqpoint{3.810642in}{3.039277in}}%
\pgfpathlineto{\pgfqpoint{3.810642in}{3.039277in}}%
\pgfpathlineto{\pgfqpoint{3.810642in}{3.042226in}}%
\pgfpathlineto{\pgfqpoint{3.815183in}{3.042226in}}%
\pgfpathlineto{\pgfqpoint{3.815183in}{3.039277in}}%
\pgfpathmoveto{\pgfqpoint{3.815183in}{3.036328in}}%
\pgfpathlineto{\pgfqpoint{3.815183in}{3.036328in}}%
\pgfpathlineto{\pgfqpoint{3.815183in}{3.039277in}}%
\pgfpathlineto{\pgfqpoint{3.819724in}{3.039277in}}%
\pgfpathlineto{\pgfqpoint{3.819724in}{3.036328in}}%
\pgfpathmoveto{\pgfqpoint{3.815183in}{3.039277in}}%
\pgfpathlineto{\pgfqpoint{3.815183in}{3.039277in}}%
\pgfpathlineto{\pgfqpoint{3.815183in}{3.042226in}}%
\pgfpathlineto{\pgfqpoint{3.819724in}{3.042226in}}%
\pgfpathlineto{\pgfqpoint{3.819724in}{3.039277in}}%
\pgfpathmoveto{\pgfqpoint{3.810642in}{3.042226in}}%
\pgfpathlineto{\pgfqpoint{3.810642in}{3.042226in}}%
\pgfpathlineto{\pgfqpoint{3.810642in}{3.045176in}}%
\pgfpathlineto{\pgfqpoint{3.815183in}{3.045176in}}%
\pgfpathlineto{\pgfqpoint{3.815183in}{3.042226in}}%
\pgfpathmoveto{\pgfqpoint{3.819724in}{3.030430in}}%
\pgfpathlineto{\pgfqpoint{3.819724in}{3.030430in}}%
\pgfpathlineto{\pgfqpoint{3.819724in}{3.033379in}}%
\pgfpathlineto{\pgfqpoint{3.824265in}{3.033379in}}%
\pgfpathlineto{\pgfqpoint{3.824265in}{3.030430in}}%
\pgfpathmoveto{\pgfqpoint{3.819724in}{3.033379in}}%
\pgfpathlineto{\pgfqpoint{3.819724in}{3.033379in}}%
\pgfpathlineto{\pgfqpoint{3.819724in}{3.036328in}}%
\pgfpathlineto{\pgfqpoint{3.824265in}{3.036328in}}%
\pgfpathlineto{\pgfqpoint{3.824265in}{3.033379in}}%
\pgfpathmoveto{\pgfqpoint{3.824265in}{3.030430in}}%
\pgfpathlineto{\pgfqpoint{3.824265in}{3.030430in}}%
\pgfpathlineto{\pgfqpoint{3.824265in}{3.033379in}}%
\pgfpathlineto{\pgfqpoint{3.828806in}{3.033379in}}%
\pgfpathlineto{\pgfqpoint{3.828806in}{3.030430in}}%
\pgfpathmoveto{\pgfqpoint{3.824265in}{3.033379in}}%
\pgfpathlineto{\pgfqpoint{3.824265in}{3.033379in}}%
\pgfpathlineto{\pgfqpoint{3.824265in}{3.036328in}}%
\pgfpathlineto{\pgfqpoint{3.828806in}{3.036328in}}%
\pgfpathlineto{\pgfqpoint{3.828806in}{3.033379in}}%
\pgfpathmoveto{\pgfqpoint{3.828806in}{3.024531in}}%
\pgfpathlineto{\pgfqpoint{3.828806in}{3.024531in}}%
\pgfpathlineto{\pgfqpoint{3.828806in}{3.027481in}}%
\pgfpathlineto{\pgfqpoint{3.833347in}{3.027481in}}%
\pgfpathlineto{\pgfqpoint{3.833347in}{3.024531in}}%
\pgfpathmoveto{\pgfqpoint{3.828806in}{3.027481in}}%
\pgfpathlineto{\pgfqpoint{3.828806in}{3.027481in}}%
\pgfpathlineto{\pgfqpoint{3.828806in}{3.030430in}}%
\pgfpathlineto{\pgfqpoint{3.833347in}{3.030430in}}%
\pgfpathlineto{\pgfqpoint{3.833347in}{3.027481in}}%
\pgfpathmoveto{\pgfqpoint{3.833347in}{3.024531in}}%
\pgfpathlineto{\pgfqpoint{3.833347in}{3.024531in}}%
\pgfpathlineto{\pgfqpoint{3.833347in}{3.027481in}}%
\pgfpathlineto{\pgfqpoint{3.837888in}{3.027481in}}%
\pgfpathlineto{\pgfqpoint{3.837888in}{3.024531in}}%
\pgfpathmoveto{\pgfqpoint{3.833347in}{3.027481in}}%
\pgfpathlineto{\pgfqpoint{3.833347in}{3.027481in}}%
\pgfpathlineto{\pgfqpoint{3.833347in}{3.030430in}}%
\pgfpathlineto{\pgfqpoint{3.837888in}{3.030430in}}%
\pgfpathlineto{\pgfqpoint{3.837888in}{3.027481in}}%
\pgfpathmoveto{\pgfqpoint{3.828806in}{3.030430in}}%
\pgfpathlineto{\pgfqpoint{3.828806in}{3.030430in}}%
\pgfpathlineto{\pgfqpoint{3.828806in}{3.033379in}}%
\pgfpathlineto{\pgfqpoint{3.833347in}{3.033379in}}%
\pgfpathlineto{\pgfqpoint{3.833347in}{3.030430in}}%
\pgfpathmoveto{\pgfqpoint{3.819724in}{3.036328in}}%
\pgfpathlineto{\pgfqpoint{3.819724in}{3.036328in}}%
\pgfpathlineto{\pgfqpoint{3.819724in}{3.039277in}}%
\pgfpathlineto{\pgfqpoint{3.824265in}{3.039277in}}%
\pgfpathlineto{\pgfqpoint{3.824265in}{3.036328in}}%
\pgfpathmoveto{\pgfqpoint{3.837888in}{3.018633in}}%
\pgfpathlineto{\pgfqpoint{3.837888in}{3.018633in}}%
\pgfpathlineto{\pgfqpoint{3.837888in}{3.021582in}}%
\pgfpathlineto{\pgfqpoint{3.842429in}{3.021582in}}%
\pgfpathlineto{\pgfqpoint{3.842429in}{3.018633in}}%
\pgfpathmoveto{\pgfqpoint{3.837888in}{3.021582in}}%
\pgfpathlineto{\pgfqpoint{3.837888in}{3.021582in}}%
\pgfpathlineto{\pgfqpoint{3.837888in}{3.024531in}}%
\pgfpathlineto{\pgfqpoint{3.842429in}{3.024531in}}%
\pgfpathlineto{\pgfqpoint{3.842429in}{3.021582in}}%
\pgfpathmoveto{\pgfqpoint{3.842429in}{3.018633in}}%
\pgfpathlineto{\pgfqpoint{3.842429in}{3.018633in}}%
\pgfpathlineto{\pgfqpoint{3.842429in}{3.021582in}}%
\pgfpathlineto{\pgfqpoint{3.846970in}{3.021582in}}%
\pgfpathlineto{\pgfqpoint{3.846970in}{3.018633in}}%
\pgfpathmoveto{\pgfqpoint{3.842429in}{3.021582in}}%
\pgfpathlineto{\pgfqpoint{3.842429in}{3.021582in}}%
\pgfpathlineto{\pgfqpoint{3.842429in}{3.024531in}}%
\pgfpathlineto{\pgfqpoint{3.846970in}{3.024531in}}%
\pgfpathlineto{\pgfqpoint{3.846970in}{3.021582in}}%
\pgfpathmoveto{\pgfqpoint{3.846970in}{3.012735in}}%
\pgfpathlineto{\pgfqpoint{3.846970in}{3.012735in}}%
\pgfpathlineto{\pgfqpoint{3.846970in}{3.015684in}}%
\pgfpathlineto{\pgfqpoint{3.851511in}{3.015684in}}%
\pgfpathlineto{\pgfqpoint{3.851511in}{3.012735in}}%
\pgfpathmoveto{\pgfqpoint{3.846970in}{3.015684in}}%
\pgfpathlineto{\pgfqpoint{3.846970in}{3.015684in}}%
\pgfpathlineto{\pgfqpoint{3.846970in}{3.018633in}}%
\pgfpathlineto{\pgfqpoint{3.851511in}{3.018633in}}%
\pgfpathlineto{\pgfqpoint{3.851511in}{3.015684in}}%
\pgfpathmoveto{\pgfqpoint{3.851511in}{3.012735in}}%
\pgfpathlineto{\pgfqpoint{3.851511in}{3.012735in}}%
\pgfpathlineto{\pgfqpoint{3.851511in}{3.015684in}}%
\pgfpathlineto{\pgfqpoint{3.856052in}{3.015684in}}%
\pgfpathlineto{\pgfqpoint{3.856052in}{3.012735in}}%
\pgfpathmoveto{\pgfqpoint{3.851511in}{3.015684in}}%
\pgfpathlineto{\pgfqpoint{3.851511in}{3.015684in}}%
\pgfpathlineto{\pgfqpoint{3.851511in}{3.018633in}}%
\pgfpathlineto{\pgfqpoint{3.856052in}{3.018633in}}%
\pgfpathlineto{\pgfqpoint{3.856052in}{3.015684in}}%
\pgfpathmoveto{\pgfqpoint{3.846970in}{3.018633in}}%
\pgfpathlineto{\pgfqpoint{3.846970in}{3.018633in}}%
\pgfpathlineto{\pgfqpoint{3.846970in}{3.021582in}}%
\pgfpathlineto{\pgfqpoint{3.851511in}{3.021582in}}%
\pgfpathlineto{\pgfqpoint{3.851511in}{3.018633in}}%
\pgfpathmoveto{\pgfqpoint{3.856052in}{3.006837in}}%
\pgfpathlineto{\pgfqpoint{3.856052in}{3.006837in}}%
\pgfpathlineto{\pgfqpoint{3.856052in}{3.009786in}}%
\pgfpathlineto{\pgfqpoint{3.860593in}{3.009786in}}%
\pgfpathlineto{\pgfqpoint{3.860593in}{3.006837in}}%
\pgfpathmoveto{\pgfqpoint{3.856052in}{3.009786in}}%
\pgfpathlineto{\pgfqpoint{3.856052in}{3.009786in}}%
\pgfpathlineto{\pgfqpoint{3.856052in}{3.012735in}}%
\pgfpathlineto{\pgfqpoint{3.860593in}{3.012735in}}%
\pgfpathlineto{\pgfqpoint{3.860593in}{3.009786in}}%
\pgfpathmoveto{\pgfqpoint{3.860593in}{3.006837in}}%
\pgfpathlineto{\pgfqpoint{3.860593in}{3.006837in}}%
\pgfpathlineto{\pgfqpoint{3.860593in}{3.009786in}}%
\pgfpathlineto{\pgfqpoint{3.865134in}{3.009786in}}%
\pgfpathlineto{\pgfqpoint{3.865134in}{3.006837in}}%
\pgfpathmoveto{\pgfqpoint{3.860593in}{3.009786in}}%
\pgfpathlineto{\pgfqpoint{3.860593in}{3.009786in}}%
\pgfpathlineto{\pgfqpoint{3.860593in}{3.012735in}}%
\pgfpathlineto{\pgfqpoint{3.865134in}{3.012735in}}%
\pgfpathlineto{\pgfqpoint{3.865134in}{3.009786in}}%
\pgfpathmoveto{\pgfqpoint{3.865134in}{3.000938in}}%
\pgfpathlineto{\pgfqpoint{3.865134in}{3.000938in}}%
\pgfpathlineto{\pgfqpoint{3.865134in}{3.003887in}}%
\pgfpathlineto{\pgfqpoint{3.869675in}{3.003887in}}%
\pgfpathlineto{\pgfqpoint{3.869675in}{3.000938in}}%
\pgfpathmoveto{\pgfqpoint{3.865134in}{3.003887in}}%
\pgfpathlineto{\pgfqpoint{3.865134in}{3.003887in}}%
\pgfpathlineto{\pgfqpoint{3.865134in}{3.006837in}}%
\pgfpathlineto{\pgfqpoint{3.869675in}{3.006837in}}%
\pgfpathlineto{\pgfqpoint{3.869675in}{3.003887in}}%
\pgfpathmoveto{\pgfqpoint{3.869675in}{3.000938in}}%
\pgfpathlineto{\pgfqpoint{3.869675in}{3.000938in}}%
\pgfpathlineto{\pgfqpoint{3.869675in}{3.003887in}}%
\pgfpathlineto{\pgfqpoint{3.874216in}{3.003887in}}%
\pgfpathlineto{\pgfqpoint{3.874216in}{3.000938in}}%
\pgfpathmoveto{\pgfqpoint{3.869675in}{3.003887in}}%
\pgfpathlineto{\pgfqpoint{3.869675in}{3.003887in}}%
\pgfpathlineto{\pgfqpoint{3.869675in}{3.006837in}}%
\pgfpathlineto{\pgfqpoint{3.874216in}{3.006837in}}%
\pgfpathlineto{\pgfqpoint{3.874216in}{3.003887in}}%
\pgfpathmoveto{\pgfqpoint{3.865134in}{3.006837in}}%
\pgfpathlineto{\pgfqpoint{3.865134in}{3.006837in}}%
\pgfpathlineto{\pgfqpoint{3.865134in}{3.009786in}}%
\pgfpathlineto{\pgfqpoint{3.869675in}{3.009786in}}%
\pgfpathlineto{\pgfqpoint{3.869675in}{3.006837in}}%
\pgfpathmoveto{\pgfqpoint{3.856052in}{3.012735in}}%
\pgfpathlineto{\pgfqpoint{3.856052in}{3.012735in}}%
\pgfpathlineto{\pgfqpoint{3.856052in}{3.015684in}}%
\pgfpathlineto{\pgfqpoint{3.860593in}{3.015684in}}%
\pgfpathlineto{\pgfqpoint{3.860593in}{3.012735in}}%
\pgfpathmoveto{\pgfqpoint{3.837888in}{3.024531in}}%
\pgfpathlineto{\pgfqpoint{3.837888in}{3.024531in}}%
\pgfpathlineto{\pgfqpoint{3.837888in}{3.027481in}}%
\pgfpathlineto{\pgfqpoint{3.842429in}{3.027481in}}%
\pgfpathlineto{\pgfqpoint{3.842429in}{3.024531in}}%
\pgfpathmoveto{\pgfqpoint{3.874216in}{2.995040in}}%
\pgfpathlineto{\pgfqpoint{3.874216in}{2.995040in}}%
\pgfpathlineto{\pgfqpoint{3.874216in}{2.997989in}}%
\pgfpathlineto{\pgfqpoint{3.878757in}{2.997989in}}%
\pgfpathlineto{\pgfqpoint{3.878757in}{2.995040in}}%
\pgfpathmoveto{\pgfqpoint{3.874216in}{2.997989in}}%
\pgfpathlineto{\pgfqpoint{3.874216in}{2.997989in}}%
\pgfpathlineto{\pgfqpoint{3.874216in}{3.000938in}}%
\pgfpathlineto{\pgfqpoint{3.878757in}{3.000938in}}%
\pgfpathlineto{\pgfqpoint{3.878757in}{2.997989in}}%
\pgfpathmoveto{\pgfqpoint{3.878757in}{2.995040in}}%
\pgfpathlineto{\pgfqpoint{3.878757in}{2.995040in}}%
\pgfpathlineto{\pgfqpoint{3.878757in}{2.997989in}}%
\pgfpathlineto{\pgfqpoint{3.883298in}{2.997989in}}%
\pgfpathlineto{\pgfqpoint{3.883298in}{2.995040in}}%
\pgfpathmoveto{\pgfqpoint{3.878757in}{2.997989in}}%
\pgfpathlineto{\pgfqpoint{3.878757in}{2.997989in}}%
\pgfpathlineto{\pgfqpoint{3.878757in}{3.000938in}}%
\pgfpathlineto{\pgfqpoint{3.883298in}{3.000938in}}%
\pgfpathlineto{\pgfqpoint{3.883298in}{2.997989in}}%
\pgfpathmoveto{\pgfqpoint{3.883298in}{2.989142in}}%
\pgfpathlineto{\pgfqpoint{3.883298in}{2.989142in}}%
\pgfpathlineto{\pgfqpoint{3.883298in}{2.992091in}}%
\pgfpathlineto{\pgfqpoint{3.887839in}{2.992091in}}%
\pgfpathlineto{\pgfqpoint{3.887839in}{2.989142in}}%
\pgfpathmoveto{\pgfqpoint{3.883298in}{2.992091in}}%
\pgfpathlineto{\pgfqpoint{3.883298in}{2.992091in}}%
\pgfpathlineto{\pgfqpoint{3.883298in}{2.995040in}}%
\pgfpathlineto{\pgfqpoint{3.887839in}{2.995040in}}%
\pgfpathlineto{\pgfqpoint{3.887839in}{2.992091in}}%
\pgfpathmoveto{\pgfqpoint{3.887839in}{2.989142in}}%
\pgfpathlineto{\pgfqpoint{3.887839in}{2.989142in}}%
\pgfpathlineto{\pgfqpoint{3.887839in}{2.992091in}}%
\pgfpathlineto{\pgfqpoint{3.892380in}{2.992091in}}%
\pgfpathlineto{\pgfqpoint{3.892380in}{2.989142in}}%
\pgfpathmoveto{\pgfqpoint{3.887839in}{2.992091in}}%
\pgfpathlineto{\pgfqpoint{3.887839in}{2.992091in}}%
\pgfpathlineto{\pgfqpoint{3.887839in}{2.995040in}}%
\pgfpathlineto{\pgfqpoint{3.892380in}{2.995040in}}%
\pgfpathlineto{\pgfqpoint{3.892380in}{2.992091in}}%
\pgfpathmoveto{\pgfqpoint{3.883298in}{2.995040in}}%
\pgfpathlineto{\pgfqpoint{3.883298in}{2.995040in}}%
\pgfpathlineto{\pgfqpoint{3.883298in}{2.997989in}}%
\pgfpathlineto{\pgfqpoint{3.887839in}{2.997989in}}%
\pgfpathlineto{\pgfqpoint{3.887839in}{2.995040in}}%
\pgfpathmoveto{\pgfqpoint{3.892380in}{2.983243in}}%
\pgfpathlineto{\pgfqpoint{3.892380in}{2.983243in}}%
\pgfpathlineto{\pgfqpoint{3.892380in}{2.986193in}}%
\pgfpathlineto{\pgfqpoint{3.896921in}{2.986193in}}%
\pgfpathlineto{\pgfqpoint{3.896921in}{2.983243in}}%
\pgfpathmoveto{\pgfqpoint{3.892380in}{2.986193in}}%
\pgfpathlineto{\pgfqpoint{3.892380in}{2.986193in}}%
\pgfpathlineto{\pgfqpoint{3.892380in}{2.989142in}}%
\pgfpathlineto{\pgfqpoint{3.896921in}{2.989142in}}%
\pgfpathlineto{\pgfqpoint{3.896921in}{2.986193in}}%
\pgfpathmoveto{\pgfqpoint{3.896921in}{2.983243in}}%
\pgfpathlineto{\pgfqpoint{3.896921in}{2.983243in}}%
\pgfpathlineto{\pgfqpoint{3.896921in}{2.986193in}}%
\pgfpathlineto{\pgfqpoint{3.901462in}{2.986193in}}%
\pgfpathlineto{\pgfqpoint{3.901462in}{2.983243in}}%
\pgfpathmoveto{\pgfqpoint{3.896921in}{2.986193in}}%
\pgfpathlineto{\pgfqpoint{3.896921in}{2.986193in}}%
\pgfpathlineto{\pgfqpoint{3.896921in}{2.989142in}}%
\pgfpathlineto{\pgfqpoint{3.901462in}{2.989142in}}%
\pgfpathlineto{\pgfqpoint{3.901462in}{2.986193in}}%
\pgfpathmoveto{\pgfqpoint{3.901462in}{2.977345in}}%
\pgfpathlineto{\pgfqpoint{3.901462in}{2.977345in}}%
\pgfpathlineto{\pgfqpoint{3.901462in}{2.980294in}}%
\pgfpathlineto{\pgfqpoint{3.906003in}{2.980294in}}%
\pgfpathlineto{\pgfqpoint{3.906003in}{2.977345in}}%
\pgfpathmoveto{\pgfqpoint{3.901462in}{2.980294in}}%
\pgfpathlineto{\pgfqpoint{3.901462in}{2.980294in}}%
\pgfpathlineto{\pgfqpoint{3.901462in}{2.983243in}}%
\pgfpathlineto{\pgfqpoint{3.906003in}{2.983243in}}%
\pgfpathlineto{\pgfqpoint{3.906003in}{2.980294in}}%
\pgfpathmoveto{\pgfqpoint{3.906003in}{2.977345in}}%
\pgfpathlineto{\pgfqpoint{3.906003in}{2.977345in}}%
\pgfpathlineto{\pgfqpoint{3.906003in}{2.980294in}}%
\pgfpathlineto{\pgfqpoint{3.910544in}{2.980294in}}%
\pgfpathlineto{\pgfqpoint{3.910544in}{2.977345in}}%
\pgfpathmoveto{\pgfqpoint{3.906003in}{2.980294in}}%
\pgfpathlineto{\pgfqpoint{3.906003in}{2.980294in}}%
\pgfpathlineto{\pgfqpoint{3.906003in}{2.983243in}}%
\pgfpathlineto{\pgfqpoint{3.910544in}{2.983243in}}%
\pgfpathlineto{\pgfqpoint{3.910544in}{2.980294in}}%
\pgfpathmoveto{\pgfqpoint{3.901462in}{2.983243in}}%
\pgfpathlineto{\pgfqpoint{3.901462in}{2.983243in}}%
\pgfpathlineto{\pgfqpoint{3.901462in}{2.986193in}}%
\pgfpathlineto{\pgfqpoint{3.906003in}{2.986193in}}%
\pgfpathlineto{\pgfqpoint{3.906003in}{2.983243in}}%
\pgfpathmoveto{\pgfqpoint{3.892380in}{2.989142in}}%
\pgfpathlineto{\pgfqpoint{3.892380in}{2.989142in}}%
\pgfpathlineto{\pgfqpoint{3.892380in}{2.992091in}}%
\pgfpathlineto{\pgfqpoint{3.896921in}{2.992091in}}%
\pgfpathlineto{\pgfqpoint{3.896921in}{2.989142in}}%
\pgfpathmoveto{\pgfqpoint{3.919627in}{2.959650in}}%
\pgfpathlineto{\pgfqpoint{3.919627in}{2.959650in}}%
\pgfpathlineto{\pgfqpoint{3.919627in}{2.962599in}}%
\pgfpathlineto{\pgfqpoint{3.924168in}{2.962599in}}%
\pgfpathlineto{\pgfqpoint{3.924168in}{2.959650in}}%
\pgfpathmoveto{\pgfqpoint{3.919627in}{2.962599in}}%
\pgfpathlineto{\pgfqpoint{3.919627in}{2.962599in}}%
\pgfpathlineto{\pgfqpoint{3.919627in}{2.965548in}}%
\pgfpathlineto{\pgfqpoint{3.924168in}{2.965548in}}%
\pgfpathlineto{\pgfqpoint{3.924168in}{2.962599in}}%
\pgfpathmoveto{\pgfqpoint{3.924168in}{2.959650in}}%
\pgfpathlineto{\pgfqpoint{3.924168in}{2.959650in}}%
\pgfpathlineto{\pgfqpoint{3.924168in}{2.962599in}}%
\pgfpathlineto{\pgfqpoint{3.928709in}{2.962599in}}%
\pgfpathlineto{\pgfqpoint{3.928709in}{2.959650in}}%
\pgfpathmoveto{\pgfqpoint{3.924168in}{2.962599in}}%
\pgfpathlineto{\pgfqpoint{3.924168in}{2.962599in}}%
\pgfpathlineto{\pgfqpoint{3.924168in}{2.965548in}}%
\pgfpathlineto{\pgfqpoint{3.928709in}{2.965548in}}%
\pgfpathlineto{\pgfqpoint{3.928709in}{2.962599in}}%
\pgfpathmoveto{\pgfqpoint{3.910544in}{2.971447in}}%
\pgfpathlineto{\pgfqpoint{3.910544in}{2.971447in}}%
\pgfpathlineto{\pgfqpoint{3.910544in}{2.974396in}}%
\pgfpathlineto{\pgfqpoint{3.915086in}{2.974396in}}%
\pgfpathlineto{\pgfqpoint{3.915086in}{2.971447in}}%
\pgfpathmoveto{\pgfqpoint{3.910544in}{2.974396in}}%
\pgfpathlineto{\pgfqpoint{3.910544in}{2.974396in}}%
\pgfpathlineto{\pgfqpoint{3.910544in}{2.977345in}}%
\pgfpathlineto{\pgfqpoint{3.915086in}{2.977345in}}%
\pgfpathlineto{\pgfqpoint{3.915086in}{2.974396in}}%
\pgfpathmoveto{\pgfqpoint{3.915086in}{2.971447in}}%
\pgfpathlineto{\pgfqpoint{3.915086in}{2.971447in}}%
\pgfpathlineto{\pgfqpoint{3.915086in}{2.974396in}}%
\pgfpathlineto{\pgfqpoint{3.919627in}{2.974396in}}%
\pgfpathlineto{\pgfqpoint{3.919627in}{2.971447in}}%
\pgfpathmoveto{\pgfqpoint{3.915086in}{2.974396in}}%
\pgfpathlineto{\pgfqpoint{3.915086in}{2.974396in}}%
\pgfpathlineto{\pgfqpoint{3.915086in}{2.977345in}}%
\pgfpathlineto{\pgfqpoint{3.919627in}{2.977345in}}%
\pgfpathlineto{\pgfqpoint{3.919627in}{2.974396in}}%
\pgfpathmoveto{\pgfqpoint{3.919627in}{2.965548in}}%
\pgfpathlineto{\pgfqpoint{3.919627in}{2.965548in}}%
\pgfpathlineto{\pgfqpoint{3.919627in}{2.968498in}}%
\pgfpathlineto{\pgfqpoint{3.924168in}{2.968498in}}%
\pgfpathlineto{\pgfqpoint{3.924168in}{2.965548in}}%
\pgfpathmoveto{\pgfqpoint{3.919627in}{2.968498in}}%
\pgfpathlineto{\pgfqpoint{3.919627in}{2.968498in}}%
\pgfpathlineto{\pgfqpoint{3.919627in}{2.971447in}}%
\pgfpathlineto{\pgfqpoint{3.924168in}{2.971447in}}%
\pgfpathlineto{\pgfqpoint{3.924168in}{2.968498in}}%
\pgfpathmoveto{\pgfqpoint{3.924168in}{2.965548in}}%
\pgfpathlineto{\pgfqpoint{3.924168in}{2.965548in}}%
\pgfpathlineto{\pgfqpoint{3.924168in}{2.968498in}}%
\pgfpathlineto{\pgfqpoint{3.928709in}{2.968498in}}%
\pgfpathlineto{\pgfqpoint{3.928709in}{2.965548in}}%
\pgfpathmoveto{\pgfqpoint{3.919627in}{2.971447in}}%
\pgfpathlineto{\pgfqpoint{3.919627in}{2.971447in}}%
\pgfpathlineto{\pgfqpoint{3.919627in}{2.974396in}}%
\pgfpathlineto{\pgfqpoint{3.924168in}{2.974396in}}%
\pgfpathlineto{\pgfqpoint{3.924168in}{2.971447in}}%
\pgfpathmoveto{\pgfqpoint{3.928709in}{2.953752in}}%
\pgfpathlineto{\pgfqpoint{3.928709in}{2.953752in}}%
\pgfpathlineto{\pgfqpoint{3.928709in}{2.956701in}}%
\pgfpathlineto{\pgfqpoint{3.933250in}{2.956701in}}%
\pgfpathlineto{\pgfqpoint{3.933250in}{2.953752in}}%
\pgfpathmoveto{\pgfqpoint{3.928709in}{2.956701in}}%
\pgfpathlineto{\pgfqpoint{3.928709in}{2.956701in}}%
\pgfpathlineto{\pgfqpoint{3.928709in}{2.959650in}}%
\pgfpathlineto{\pgfqpoint{3.933250in}{2.959650in}}%
\pgfpathlineto{\pgfqpoint{3.933250in}{2.956701in}}%
\pgfpathmoveto{\pgfqpoint{3.933250in}{2.953752in}}%
\pgfpathlineto{\pgfqpoint{3.933250in}{2.953752in}}%
\pgfpathlineto{\pgfqpoint{3.933250in}{2.956701in}}%
\pgfpathlineto{\pgfqpoint{3.937791in}{2.956701in}}%
\pgfpathlineto{\pgfqpoint{3.937791in}{2.953752in}}%
\pgfpathmoveto{\pgfqpoint{3.933250in}{2.956701in}}%
\pgfpathlineto{\pgfqpoint{3.933250in}{2.956701in}}%
\pgfpathlineto{\pgfqpoint{3.933250in}{2.959650in}}%
\pgfpathlineto{\pgfqpoint{3.937791in}{2.959650in}}%
\pgfpathlineto{\pgfqpoint{3.937791in}{2.956701in}}%
\pgfpathmoveto{\pgfqpoint{3.928709in}{2.959650in}}%
\pgfpathlineto{\pgfqpoint{3.928709in}{2.959650in}}%
\pgfpathlineto{\pgfqpoint{3.928709in}{2.962599in}}%
\pgfpathlineto{\pgfqpoint{3.933250in}{2.962599in}}%
\pgfpathlineto{\pgfqpoint{3.933250in}{2.959650in}}%
\pgfpathmoveto{\pgfqpoint{3.928709in}{2.962599in}}%
\pgfpathlineto{\pgfqpoint{3.928709in}{2.962599in}}%
\pgfpathlineto{\pgfqpoint{3.928709in}{2.965548in}}%
\pgfpathlineto{\pgfqpoint{3.933250in}{2.965548in}}%
\pgfpathlineto{\pgfqpoint{3.933250in}{2.962599in}}%
\pgfpathmoveto{\pgfqpoint{3.933250in}{2.959650in}}%
\pgfpathlineto{\pgfqpoint{3.933250in}{2.959650in}}%
\pgfpathlineto{\pgfqpoint{3.933250in}{2.962599in}}%
\pgfpathlineto{\pgfqpoint{3.937791in}{2.962599in}}%
\pgfpathlineto{\pgfqpoint{3.937791in}{2.959650in}}%
\pgfpathmoveto{\pgfqpoint{3.937791in}{2.953752in}}%
\pgfpathlineto{\pgfqpoint{3.937791in}{2.953752in}}%
\pgfpathlineto{\pgfqpoint{3.937791in}{2.956701in}}%
\pgfpathlineto{\pgfqpoint{3.942332in}{2.956701in}}%
\pgfpathlineto{\pgfqpoint{3.942332in}{2.953752in}}%
\pgfpathmoveto{\pgfqpoint{3.937791in}{2.956701in}}%
\pgfpathlineto{\pgfqpoint{3.937791in}{2.956701in}}%
\pgfpathlineto{\pgfqpoint{3.937791in}{2.959650in}}%
\pgfpathlineto{\pgfqpoint{3.942332in}{2.959650in}}%
\pgfpathlineto{\pgfqpoint{3.942332in}{2.956701in}}%
\pgfpathmoveto{\pgfqpoint{3.942332in}{2.953752in}}%
\pgfpathlineto{\pgfqpoint{3.942332in}{2.953752in}}%
\pgfpathlineto{\pgfqpoint{3.942332in}{2.956701in}}%
\pgfpathlineto{\pgfqpoint{3.946873in}{2.956701in}}%
\pgfpathlineto{\pgfqpoint{3.946873in}{2.953752in}}%
\pgfpathmoveto{\pgfqpoint{3.910544in}{2.977345in}}%
\pgfpathlineto{\pgfqpoint{3.910544in}{2.977345in}}%
\pgfpathlineto{\pgfqpoint{3.910544in}{2.980294in}}%
\pgfpathlineto{\pgfqpoint{3.915086in}{2.980294in}}%
\pgfpathlineto{\pgfqpoint{3.915086in}{2.977345in}}%
\pgfpathmoveto{\pgfqpoint{3.874216in}{3.000938in}}%
\pgfpathlineto{\pgfqpoint{3.874216in}{3.000938in}}%
\pgfpathlineto{\pgfqpoint{3.874216in}{3.003887in}}%
\pgfpathlineto{\pgfqpoint{3.878757in}{3.003887in}}%
\pgfpathlineto{\pgfqpoint{3.878757in}{3.000938in}}%
\pgfpathmoveto{\pgfqpoint{3.801560in}{3.048125in}}%
\pgfpathlineto{\pgfqpoint{3.801560in}{3.048125in}}%
\pgfpathlineto{\pgfqpoint{3.801560in}{3.051074in}}%
\pgfpathlineto{\pgfqpoint{3.806101in}{3.051074in}}%
\pgfpathlineto{\pgfqpoint{3.806101in}{3.048125in}}%
\pgfpathmoveto{\pgfqpoint{3.987743in}{2.714862in}}%
\pgfpathlineto{\pgfqpoint{3.987743in}{2.714862in}}%
\pgfpathlineto{\pgfqpoint{3.987743in}{2.717811in}}%
\pgfpathlineto{\pgfqpoint{3.992284in}{2.717811in}}%
\pgfpathlineto{\pgfqpoint{3.992284in}{2.714862in}}%
\pgfpathmoveto{\pgfqpoint{3.992284in}{2.714862in}}%
\pgfpathlineto{\pgfqpoint{3.992284in}{2.714862in}}%
\pgfpathlineto{\pgfqpoint{3.992284in}{2.717811in}}%
\pgfpathlineto{\pgfqpoint{3.996825in}{2.717811in}}%
\pgfpathlineto{\pgfqpoint{3.996825in}{2.714862in}}%
\pgfpathmoveto{\pgfqpoint{3.996825in}{2.714862in}}%
\pgfpathlineto{\pgfqpoint{3.996825in}{2.714862in}}%
\pgfpathlineto{\pgfqpoint{3.996825in}{2.717811in}}%
\pgfpathlineto{\pgfqpoint{4.001366in}{2.717811in}}%
\pgfpathlineto{\pgfqpoint{4.001366in}{2.714862in}}%
\pgfpathmoveto{\pgfqpoint{4.001366in}{2.711913in}}%
\pgfpathlineto{\pgfqpoint{4.001366in}{2.711913in}}%
\pgfpathlineto{\pgfqpoint{4.001366in}{2.714862in}}%
\pgfpathlineto{\pgfqpoint{4.005907in}{2.714862in}}%
\pgfpathlineto{\pgfqpoint{4.005907in}{2.711913in}}%
\pgfpathmoveto{\pgfqpoint{4.001366in}{2.714862in}}%
\pgfpathlineto{\pgfqpoint{4.001366in}{2.714862in}}%
\pgfpathlineto{\pgfqpoint{4.001366in}{2.717811in}}%
\pgfpathlineto{\pgfqpoint{4.005907in}{2.717811in}}%
\pgfpathlineto{\pgfqpoint{4.005907in}{2.714862in}}%
\pgfpathmoveto{\pgfqpoint{4.005907in}{2.711913in}}%
\pgfpathlineto{\pgfqpoint{4.005907in}{2.711913in}}%
\pgfpathlineto{\pgfqpoint{4.005907in}{2.714862in}}%
\pgfpathlineto{\pgfqpoint{4.010448in}{2.714862in}}%
\pgfpathlineto{\pgfqpoint{4.010448in}{2.711913in}}%
\pgfpathmoveto{\pgfqpoint{4.005907in}{2.714862in}}%
\pgfpathlineto{\pgfqpoint{4.005907in}{2.714862in}}%
\pgfpathlineto{\pgfqpoint{4.005907in}{2.717811in}}%
\pgfpathlineto{\pgfqpoint{4.010448in}{2.717811in}}%
\pgfpathlineto{\pgfqpoint{4.010448in}{2.714862in}}%
\pgfpathmoveto{\pgfqpoint{4.014990in}{2.708963in}}%
\pgfpathlineto{\pgfqpoint{4.014990in}{2.708963in}}%
\pgfpathlineto{\pgfqpoint{4.014990in}{2.711913in}}%
\pgfpathlineto{\pgfqpoint{4.019531in}{2.711913in}}%
\pgfpathlineto{\pgfqpoint{4.019531in}{2.708963in}}%
\pgfpathmoveto{\pgfqpoint{4.010448in}{2.711913in}}%
\pgfpathlineto{\pgfqpoint{4.010448in}{2.711913in}}%
\pgfpathlineto{\pgfqpoint{4.010448in}{2.714862in}}%
\pgfpathlineto{\pgfqpoint{4.014990in}{2.714862in}}%
\pgfpathlineto{\pgfqpoint{4.014990in}{2.711913in}}%
\pgfpathmoveto{\pgfqpoint{4.010448in}{2.714862in}}%
\pgfpathlineto{\pgfqpoint{4.010448in}{2.714862in}}%
\pgfpathlineto{\pgfqpoint{4.010448in}{2.717811in}}%
\pgfpathlineto{\pgfqpoint{4.014990in}{2.717811in}}%
\pgfpathlineto{\pgfqpoint{4.014990in}{2.714862in}}%
\pgfpathmoveto{\pgfqpoint{4.014990in}{2.711913in}}%
\pgfpathlineto{\pgfqpoint{4.014990in}{2.711913in}}%
\pgfpathlineto{\pgfqpoint{4.014990in}{2.714862in}}%
\pgfpathlineto{\pgfqpoint{4.019531in}{2.714862in}}%
\pgfpathlineto{\pgfqpoint{4.019531in}{2.711913in}}%
\pgfpathmoveto{\pgfqpoint{4.014990in}{2.714862in}}%
\pgfpathlineto{\pgfqpoint{4.014990in}{2.714862in}}%
\pgfpathlineto{\pgfqpoint{4.014990in}{2.717811in}}%
\pgfpathlineto{\pgfqpoint{4.019531in}{2.717811in}}%
\pgfpathlineto{\pgfqpoint{4.019531in}{2.714862in}}%
\pgfpathmoveto{\pgfqpoint{3.946873in}{2.723709in}}%
\pgfpathlineto{\pgfqpoint{3.946873in}{2.723709in}}%
\pgfpathlineto{\pgfqpoint{3.946873in}{2.726659in}}%
\pgfpathlineto{\pgfqpoint{3.951414in}{2.726659in}}%
\pgfpathlineto{\pgfqpoint{3.951414in}{2.723709in}}%
\pgfpathmoveto{\pgfqpoint{3.946873in}{2.726659in}}%
\pgfpathlineto{\pgfqpoint{3.946873in}{2.726659in}}%
\pgfpathlineto{\pgfqpoint{3.946873in}{2.729608in}}%
\pgfpathlineto{\pgfqpoint{3.951414in}{2.729608in}}%
\pgfpathlineto{\pgfqpoint{3.951414in}{2.726659in}}%
\pgfpathmoveto{\pgfqpoint{3.951414in}{2.723709in}}%
\pgfpathlineto{\pgfqpoint{3.951414in}{2.723709in}}%
\pgfpathlineto{\pgfqpoint{3.951414in}{2.726659in}}%
\pgfpathlineto{\pgfqpoint{3.955955in}{2.726659in}}%
\pgfpathlineto{\pgfqpoint{3.955955in}{2.723709in}}%
\pgfpathmoveto{\pgfqpoint{3.951414in}{2.726659in}}%
\pgfpathlineto{\pgfqpoint{3.951414in}{2.726659in}}%
\pgfpathlineto{\pgfqpoint{3.951414in}{2.729608in}}%
\pgfpathlineto{\pgfqpoint{3.955955in}{2.729608in}}%
\pgfpathlineto{\pgfqpoint{3.955955in}{2.726659in}}%
\pgfpathmoveto{\pgfqpoint{3.960496in}{2.720760in}}%
\pgfpathlineto{\pgfqpoint{3.960496in}{2.720760in}}%
\pgfpathlineto{\pgfqpoint{3.960496in}{2.723709in}}%
\pgfpathlineto{\pgfqpoint{3.965037in}{2.723709in}}%
\pgfpathlineto{\pgfqpoint{3.965037in}{2.720760in}}%
\pgfpathmoveto{\pgfqpoint{3.955955in}{2.723709in}}%
\pgfpathlineto{\pgfqpoint{3.955955in}{2.723709in}}%
\pgfpathlineto{\pgfqpoint{3.955955in}{2.726659in}}%
\pgfpathlineto{\pgfqpoint{3.960496in}{2.726659in}}%
\pgfpathlineto{\pgfqpoint{3.960496in}{2.723709in}}%
\pgfpathmoveto{\pgfqpoint{3.955955in}{2.726659in}}%
\pgfpathlineto{\pgfqpoint{3.955955in}{2.726659in}}%
\pgfpathlineto{\pgfqpoint{3.955955in}{2.729608in}}%
\pgfpathlineto{\pgfqpoint{3.960496in}{2.729608in}}%
\pgfpathlineto{\pgfqpoint{3.960496in}{2.726659in}}%
\pgfpathmoveto{\pgfqpoint{3.960496in}{2.723709in}}%
\pgfpathlineto{\pgfqpoint{3.960496in}{2.723709in}}%
\pgfpathlineto{\pgfqpoint{3.960496in}{2.726659in}}%
\pgfpathlineto{\pgfqpoint{3.965037in}{2.726659in}}%
\pgfpathlineto{\pgfqpoint{3.965037in}{2.723709in}}%
\pgfpathmoveto{\pgfqpoint{3.960496in}{2.726659in}}%
\pgfpathlineto{\pgfqpoint{3.960496in}{2.726659in}}%
\pgfpathlineto{\pgfqpoint{3.960496in}{2.729608in}}%
\pgfpathlineto{\pgfqpoint{3.965037in}{2.729608in}}%
\pgfpathlineto{\pgfqpoint{3.965037in}{2.726659in}}%
\pgfpathmoveto{\pgfqpoint{3.965037in}{2.720760in}}%
\pgfpathlineto{\pgfqpoint{3.965037in}{2.720760in}}%
\pgfpathlineto{\pgfqpoint{3.965037in}{2.723709in}}%
\pgfpathlineto{\pgfqpoint{3.969578in}{2.723709in}}%
\pgfpathlineto{\pgfqpoint{3.969578in}{2.720760in}}%
\pgfpathmoveto{\pgfqpoint{3.969578in}{2.720760in}}%
\pgfpathlineto{\pgfqpoint{3.969578in}{2.720760in}}%
\pgfpathlineto{\pgfqpoint{3.969578in}{2.723709in}}%
\pgfpathlineto{\pgfqpoint{3.974119in}{2.723709in}}%
\pgfpathlineto{\pgfqpoint{3.974119in}{2.720760in}}%
\pgfpathmoveto{\pgfqpoint{3.974119in}{2.717811in}}%
\pgfpathlineto{\pgfqpoint{3.974119in}{2.717811in}}%
\pgfpathlineto{\pgfqpoint{3.974119in}{2.720760in}}%
\pgfpathlineto{\pgfqpoint{3.978661in}{2.720760in}}%
\pgfpathlineto{\pgfqpoint{3.978661in}{2.717811in}}%
\pgfpathmoveto{\pgfqpoint{3.974119in}{2.720760in}}%
\pgfpathlineto{\pgfqpoint{3.974119in}{2.720760in}}%
\pgfpathlineto{\pgfqpoint{3.974119in}{2.723709in}}%
\pgfpathlineto{\pgfqpoint{3.978661in}{2.723709in}}%
\pgfpathlineto{\pgfqpoint{3.978661in}{2.720760in}}%
\pgfpathmoveto{\pgfqpoint{3.978661in}{2.717811in}}%
\pgfpathlineto{\pgfqpoint{3.978661in}{2.717811in}}%
\pgfpathlineto{\pgfqpoint{3.978661in}{2.720760in}}%
\pgfpathlineto{\pgfqpoint{3.983202in}{2.720760in}}%
\pgfpathlineto{\pgfqpoint{3.983202in}{2.717811in}}%
\pgfpathmoveto{\pgfqpoint{3.978661in}{2.720760in}}%
\pgfpathlineto{\pgfqpoint{3.978661in}{2.720760in}}%
\pgfpathlineto{\pgfqpoint{3.978661in}{2.723709in}}%
\pgfpathlineto{\pgfqpoint{3.983202in}{2.723709in}}%
\pgfpathlineto{\pgfqpoint{3.983202in}{2.720760in}}%
\pgfpathmoveto{\pgfqpoint{3.983202in}{2.717811in}}%
\pgfpathlineto{\pgfqpoint{3.983202in}{2.717811in}}%
\pgfpathlineto{\pgfqpoint{3.983202in}{2.720760in}}%
\pgfpathlineto{\pgfqpoint{3.987743in}{2.720760in}}%
\pgfpathlineto{\pgfqpoint{3.987743in}{2.717811in}}%
\pgfpathmoveto{\pgfqpoint{3.983202in}{2.720760in}}%
\pgfpathlineto{\pgfqpoint{3.983202in}{2.720760in}}%
\pgfpathlineto{\pgfqpoint{3.983202in}{2.723709in}}%
\pgfpathlineto{\pgfqpoint{3.987743in}{2.723709in}}%
\pgfpathlineto{\pgfqpoint{3.987743in}{2.720760in}}%
\pgfpathmoveto{\pgfqpoint{3.987743in}{2.717811in}}%
\pgfpathlineto{\pgfqpoint{3.987743in}{2.717811in}}%
\pgfpathlineto{\pgfqpoint{3.987743in}{2.720760in}}%
\pgfpathlineto{\pgfqpoint{3.992284in}{2.720760in}}%
\pgfpathlineto{\pgfqpoint{3.992284in}{2.717811in}}%
\pgfpathmoveto{\pgfqpoint{3.987743in}{2.720760in}}%
\pgfpathlineto{\pgfqpoint{3.987743in}{2.720760in}}%
\pgfpathlineto{\pgfqpoint{3.987743in}{2.723709in}}%
\pgfpathlineto{\pgfqpoint{3.992284in}{2.723709in}}%
\pgfpathlineto{\pgfqpoint{3.992284in}{2.720760in}}%
\pgfpathmoveto{\pgfqpoint{4.019531in}{2.708963in}}%
\pgfpathlineto{\pgfqpoint{4.019531in}{2.708963in}}%
\pgfpathlineto{\pgfqpoint{4.019531in}{2.711913in}}%
\pgfpathlineto{\pgfqpoint{4.024072in}{2.711913in}}%
\pgfpathlineto{\pgfqpoint{4.024072in}{2.708963in}}%
\pgfpathmoveto{\pgfqpoint{4.024072in}{2.708963in}}%
\pgfpathlineto{\pgfqpoint{4.024072in}{2.708963in}}%
\pgfpathlineto{\pgfqpoint{4.024072in}{2.711913in}}%
\pgfpathlineto{\pgfqpoint{4.028613in}{2.711913in}}%
\pgfpathlineto{\pgfqpoint{4.028613in}{2.708963in}}%
\pgfpathmoveto{\pgfqpoint{4.028613in}{2.706014in}}%
\pgfpathlineto{\pgfqpoint{4.028613in}{2.706014in}}%
\pgfpathlineto{\pgfqpoint{4.028613in}{2.708963in}}%
\pgfpathlineto{\pgfqpoint{4.033154in}{2.708963in}}%
\pgfpathlineto{\pgfqpoint{4.033154in}{2.706014in}}%
\pgfpathmoveto{\pgfqpoint{4.028613in}{2.708963in}}%
\pgfpathlineto{\pgfqpoint{4.028613in}{2.708963in}}%
\pgfpathlineto{\pgfqpoint{4.028613in}{2.711913in}}%
\pgfpathlineto{\pgfqpoint{4.033154in}{2.711913in}}%
\pgfpathlineto{\pgfqpoint{4.033154in}{2.708963in}}%
\pgfpathmoveto{\pgfqpoint{4.033154in}{2.706014in}}%
\pgfpathlineto{\pgfqpoint{4.033154in}{2.706014in}}%
\pgfpathlineto{\pgfqpoint{4.033154in}{2.708963in}}%
\pgfpathlineto{\pgfqpoint{4.037695in}{2.708963in}}%
\pgfpathlineto{\pgfqpoint{4.037695in}{2.706014in}}%
\pgfpathmoveto{\pgfqpoint{4.033154in}{2.708963in}}%
\pgfpathlineto{\pgfqpoint{4.033154in}{2.708963in}}%
\pgfpathlineto{\pgfqpoint{4.033154in}{2.711913in}}%
\pgfpathlineto{\pgfqpoint{4.037695in}{2.711913in}}%
\pgfpathlineto{\pgfqpoint{4.037695in}{2.708963in}}%
\pgfpathmoveto{\pgfqpoint{4.042236in}{2.703065in}}%
\pgfpathlineto{\pgfqpoint{4.042236in}{2.703065in}}%
\pgfpathlineto{\pgfqpoint{4.042236in}{2.706014in}}%
\pgfpathlineto{\pgfqpoint{4.046777in}{2.706014in}}%
\pgfpathlineto{\pgfqpoint{4.046777in}{2.703065in}}%
\pgfpathmoveto{\pgfqpoint{4.046777in}{2.703065in}}%
\pgfpathlineto{\pgfqpoint{4.046777in}{2.703065in}}%
\pgfpathlineto{\pgfqpoint{4.046777in}{2.706014in}}%
\pgfpathlineto{\pgfqpoint{4.051319in}{2.706014in}}%
\pgfpathlineto{\pgfqpoint{4.051319in}{2.703065in}}%
\pgfpathmoveto{\pgfqpoint{4.051319in}{2.703065in}}%
\pgfpathlineto{\pgfqpoint{4.051319in}{2.703065in}}%
\pgfpathlineto{\pgfqpoint{4.051319in}{2.706014in}}%
\pgfpathlineto{\pgfqpoint{4.055860in}{2.706014in}}%
\pgfpathlineto{\pgfqpoint{4.055860in}{2.703065in}}%
\pgfpathmoveto{\pgfqpoint{4.037695in}{2.706014in}}%
\pgfpathlineto{\pgfqpoint{4.037695in}{2.706014in}}%
\pgfpathlineto{\pgfqpoint{4.037695in}{2.708963in}}%
\pgfpathlineto{\pgfqpoint{4.042236in}{2.708963in}}%
\pgfpathlineto{\pgfqpoint{4.042236in}{2.706014in}}%
\pgfpathmoveto{\pgfqpoint{4.037695in}{2.708963in}}%
\pgfpathlineto{\pgfqpoint{4.037695in}{2.708963in}}%
\pgfpathlineto{\pgfqpoint{4.037695in}{2.711913in}}%
\pgfpathlineto{\pgfqpoint{4.042236in}{2.711913in}}%
\pgfpathlineto{\pgfqpoint{4.042236in}{2.708963in}}%
\pgfpathmoveto{\pgfqpoint{4.042236in}{2.706014in}}%
\pgfpathlineto{\pgfqpoint{4.042236in}{2.706014in}}%
\pgfpathlineto{\pgfqpoint{4.042236in}{2.708963in}}%
\pgfpathlineto{\pgfqpoint{4.046777in}{2.708963in}}%
\pgfpathlineto{\pgfqpoint{4.046777in}{2.706014in}}%
\pgfpathmoveto{\pgfqpoint{4.042236in}{2.708963in}}%
\pgfpathlineto{\pgfqpoint{4.042236in}{2.708963in}}%
\pgfpathlineto{\pgfqpoint{4.042236in}{2.711913in}}%
\pgfpathlineto{\pgfqpoint{4.046777in}{2.711913in}}%
\pgfpathlineto{\pgfqpoint{4.046777in}{2.708963in}}%
\pgfpathmoveto{\pgfqpoint{4.055860in}{2.700116in}}%
\pgfpathlineto{\pgfqpoint{4.055860in}{2.700116in}}%
\pgfpathlineto{\pgfqpoint{4.055860in}{2.703065in}}%
\pgfpathlineto{\pgfqpoint{4.060401in}{2.703065in}}%
\pgfpathlineto{\pgfqpoint{4.060401in}{2.700116in}}%
\pgfpathmoveto{\pgfqpoint{4.055860in}{2.703065in}}%
\pgfpathlineto{\pgfqpoint{4.055860in}{2.703065in}}%
\pgfpathlineto{\pgfqpoint{4.055860in}{2.706014in}}%
\pgfpathlineto{\pgfqpoint{4.060401in}{2.706014in}}%
\pgfpathlineto{\pgfqpoint{4.060401in}{2.703065in}}%
\pgfpathmoveto{\pgfqpoint{4.060401in}{2.700116in}}%
\pgfpathlineto{\pgfqpoint{4.060401in}{2.700116in}}%
\pgfpathlineto{\pgfqpoint{4.060401in}{2.703065in}}%
\pgfpathlineto{\pgfqpoint{4.064942in}{2.703065in}}%
\pgfpathlineto{\pgfqpoint{4.064942in}{2.700116in}}%
\pgfpathmoveto{\pgfqpoint{4.060401in}{2.703065in}}%
\pgfpathlineto{\pgfqpoint{4.060401in}{2.703065in}}%
\pgfpathlineto{\pgfqpoint{4.060401in}{2.706014in}}%
\pgfpathlineto{\pgfqpoint{4.064942in}{2.706014in}}%
\pgfpathlineto{\pgfqpoint{4.064942in}{2.703065in}}%
\pgfpathmoveto{\pgfqpoint{4.069483in}{2.697167in}}%
\pgfpathlineto{\pgfqpoint{4.069483in}{2.697167in}}%
\pgfpathlineto{\pgfqpoint{4.069483in}{2.700116in}}%
\pgfpathlineto{\pgfqpoint{4.074024in}{2.700116in}}%
\pgfpathlineto{\pgfqpoint{4.074024in}{2.697167in}}%
\pgfpathmoveto{\pgfqpoint{4.064942in}{2.700116in}}%
\pgfpathlineto{\pgfqpoint{4.064942in}{2.700116in}}%
\pgfpathlineto{\pgfqpoint{4.064942in}{2.703065in}}%
\pgfpathlineto{\pgfqpoint{4.069483in}{2.703065in}}%
\pgfpathlineto{\pgfqpoint{4.069483in}{2.700116in}}%
\pgfpathmoveto{\pgfqpoint{4.064942in}{2.703065in}}%
\pgfpathlineto{\pgfqpoint{4.064942in}{2.703065in}}%
\pgfpathlineto{\pgfqpoint{4.064942in}{2.706014in}}%
\pgfpathlineto{\pgfqpoint{4.069483in}{2.706014in}}%
\pgfpathlineto{\pgfqpoint{4.069483in}{2.703065in}}%
\pgfpathmoveto{\pgfqpoint{4.069483in}{2.700116in}}%
\pgfpathlineto{\pgfqpoint{4.069483in}{2.700116in}}%
\pgfpathlineto{\pgfqpoint{4.069483in}{2.703065in}}%
\pgfpathlineto{\pgfqpoint{4.074024in}{2.703065in}}%
\pgfpathlineto{\pgfqpoint{4.074024in}{2.700116in}}%
\pgfpathmoveto{\pgfqpoint{4.069483in}{2.703065in}}%
\pgfpathlineto{\pgfqpoint{4.069483in}{2.703065in}}%
\pgfpathlineto{\pgfqpoint{4.069483in}{2.706014in}}%
\pgfpathlineto{\pgfqpoint{4.074024in}{2.706014in}}%
\pgfpathlineto{\pgfqpoint{4.074024in}{2.703065in}}%
\pgfpathmoveto{\pgfqpoint{4.074024in}{2.697167in}}%
\pgfpathlineto{\pgfqpoint{4.074024in}{2.697167in}}%
\pgfpathlineto{\pgfqpoint{4.074024in}{2.700116in}}%
\pgfpathlineto{\pgfqpoint{4.078565in}{2.700116in}}%
\pgfpathlineto{\pgfqpoint{4.078565in}{2.697167in}}%
\pgfpathmoveto{\pgfqpoint{4.078565in}{2.697167in}}%
\pgfpathlineto{\pgfqpoint{4.078565in}{2.697167in}}%
\pgfpathlineto{\pgfqpoint{4.078565in}{2.700116in}}%
\pgfpathlineto{\pgfqpoint{4.083106in}{2.700116in}}%
\pgfpathlineto{\pgfqpoint{4.083106in}{2.697167in}}%
\pgfpathmoveto{\pgfqpoint{4.083106in}{2.694217in}}%
\pgfpathlineto{\pgfqpoint{4.083106in}{2.694217in}}%
\pgfpathlineto{\pgfqpoint{4.083106in}{2.697167in}}%
\pgfpathlineto{\pgfqpoint{4.087648in}{2.697167in}}%
\pgfpathlineto{\pgfqpoint{4.087648in}{2.694217in}}%
\pgfpathmoveto{\pgfqpoint{4.083106in}{2.697167in}}%
\pgfpathlineto{\pgfqpoint{4.083106in}{2.697167in}}%
\pgfpathlineto{\pgfqpoint{4.083106in}{2.700116in}}%
\pgfpathlineto{\pgfqpoint{4.087648in}{2.700116in}}%
\pgfpathlineto{\pgfqpoint{4.087648in}{2.697167in}}%
\pgfpathmoveto{\pgfqpoint{4.087648in}{2.694217in}}%
\pgfpathlineto{\pgfqpoint{4.087648in}{2.694217in}}%
\pgfpathlineto{\pgfqpoint{4.087648in}{2.697167in}}%
\pgfpathlineto{\pgfqpoint{4.092189in}{2.697167in}}%
\pgfpathlineto{\pgfqpoint{4.092189in}{2.694217in}}%
\pgfpathmoveto{\pgfqpoint{4.087648in}{2.697167in}}%
\pgfpathlineto{\pgfqpoint{4.087648in}{2.697167in}}%
\pgfpathlineto{\pgfqpoint{4.087648in}{2.700116in}}%
\pgfpathlineto{\pgfqpoint{4.092189in}{2.700116in}}%
\pgfpathlineto{\pgfqpoint{4.092189in}{2.697167in}}%
\pgfpathmoveto{\pgfqpoint{4.083106in}{2.853477in}}%
\pgfpathlineto{\pgfqpoint{4.083106in}{2.853477in}}%
\pgfpathlineto{\pgfqpoint{4.083106in}{2.856426in}}%
\pgfpathlineto{\pgfqpoint{4.087648in}{2.856426in}}%
\pgfpathlineto{\pgfqpoint{4.087648in}{2.853477in}}%
\pgfpathmoveto{\pgfqpoint{4.083106in}{2.856426in}}%
\pgfpathlineto{\pgfqpoint{4.083106in}{2.856426in}}%
\pgfpathlineto{\pgfqpoint{4.083106in}{2.859375in}}%
\pgfpathlineto{\pgfqpoint{4.087648in}{2.859375in}}%
\pgfpathlineto{\pgfqpoint{4.087648in}{2.856426in}}%
\pgfpathmoveto{\pgfqpoint{4.087648in}{2.853477in}}%
\pgfpathlineto{\pgfqpoint{4.087648in}{2.853477in}}%
\pgfpathlineto{\pgfqpoint{4.087648in}{2.856426in}}%
\pgfpathlineto{\pgfqpoint{4.092189in}{2.856426in}}%
\pgfpathlineto{\pgfqpoint{4.092189in}{2.853477in}}%
\pgfpathmoveto{\pgfqpoint{4.087648in}{2.856426in}}%
\pgfpathlineto{\pgfqpoint{4.087648in}{2.856426in}}%
\pgfpathlineto{\pgfqpoint{4.087648in}{2.859375in}}%
\pgfpathlineto{\pgfqpoint{4.092189in}{2.859375in}}%
\pgfpathlineto{\pgfqpoint{4.092189in}{2.856426in}}%
\pgfpathmoveto{\pgfqpoint{4.010448in}{2.900665in}}%
\pgfpathlineto{\pgfqpoint{4.010448in}{2.900665in}}%
\pgfpathlineto{\pgfqpoint{4.010448in}{2.903614in}}%
\pgfpathlineto{\pgfqpoint{4.014990in}{2.903614in}}%
\pgfpathlineto{\pgfqpoint{4.014990in}{2.900665in}}%
\pgfpathmoveto{\pgfqpoint{4.010448in}{2.903614in}}%
\pgfpathlineto{\pgfqpoint{4.010448in}{2.903614in}}%
\pgfpathlineto{\pgfqpoint{4.010448in}{2.906564in}}%
\pgfpathlineto{\pgfqpoint{4.014990in}{2.906564in}}%
\pgfpathlineto{\pgfqpoint{4.014990in}{2.903614in}}%
\pgfpathmoveto{\pgfqpoint{4.014990in}{2.900665in}}%
\pgfpathlineto{\pgfqpoint{4.014990in}{2.900665in}}%
\pgfpathlineto{\pgfqpoint{4.014990in}{2.903614in}}%
\pgfpathlineto{\pgfqpoint{4.019531in}{2.903614in}}%
\pgfpathlineto{\pgfqpoint{4.019531in}{2.900665in}}%
\pgfpathmoveto{\pgfqpoint{4.014990in}{2.903614in}}%
\pgfpathlineto{\pgfqpoint{4.014990in}{2.903614in}}%
\pgfpathlineto{\pgfqpoint{4.014990in}{2.906564in}}%
\pgfpathlineto{\pgfqpoint{4.019531in}{2.906564in}}%
\pgfpathlineto{\pgfqpoint{4.019531in}{2.903614in}}%
\pgfpathmoveto{\pgfqpoint{3.974119in}{2.924259in}}%
\pgfpathlineto{\pgfqpoint{3.974119in}{2.924259in}}%
\pgfpathlineto{\pgfqpoint{3.974119in}{2.927209in}}%
\pgfpathlineto{\pgfqpoint{3.978661in}{2.927209in}}%
\pgfpathlineto{\pgfqpoint{3.978661in}{2.924259in}}%
\pgfpathmoveto{\pgfqpoint{3.974119in}{2.927209in}}%
\pgfpathlineto{\pgfqpoint{3.974119in}{2.927209in}}%
\pgfpathlineto{\pgfqpoint{3.974119in}{2.930158in}}%
\pgfpathlineto{\pgfqpoint{3.978661in}{2.930158in}}%
\pgfpathlineto{\pgfqpoint{3.978661in}{2.927209in}}%
\pgfpathmoveto{\pgfqpoint{3.978661in}{2.924259in}}%
\pgfpathlineto{\pgfqpoint{3.978661in}{2.924259in}}%
\pgfpathlineto{\pgfqpoint{3.978661in}{2.927209in}}%
\pgfpathlineto{\pgfqpoint{3.983202in}{2.927209in}}%
\pgfpathlineto{\pgfqpoint{3.983202in}{2.924259in}}%
\pgfpathmoveto{\pgfqpoint{3.978661in}{2.927209in}}%
\pgfpathlineto{\pgfqpoint{3.978661in}{2.927209in}}%
\pgfpathlineto{\pgfqpoint{3.978661in}{2.930158in}}%
\pgfpathlineto{\pgfqpoint{3.983202in}{2.930158in}}%
\pgfpathlineto{\pgfqpoint{3.983202in}{2.927209in}}%
\pgfpathmoveto{\pgfqpoint{3.955955in}{2.936056in}}%
\pgfpathlineto{\pgfqpoint{3.955955in}{2.936056in}}%
\pgfpathlineto{\pgfqpoint{3.955955in}{2.939006in}}%
\pgfpathlineto{\pgfqpoint{3.960496in}{2.939006in}}%
\pgfpathlineto{\pgfqpoint{3.960496in}{2.936056in}}%
\pgfpathmoveto{\pgfqpoint{3.955955in}{2.939006in}}%
\pgfpathlineto{\pgfqpoint{3.955955in}{2.939006in}}%
\pgfpathlineto{\pgfqpoint{3.955955in}{2.941955in}}%
\pgfpathlineto{\pgfqpoint{3.960496in}{2.941955in}}%
\pgfpathlineto{\pgfqpoint{3.960496in}{2.939006in}}%
\pgfpathmoveto{\pgfqpoint{3.960496in}{2.936056in}}%
\pgfpathlineto{\pgfqpoint{3.960496in}{2.936056in}}%
\pgfpathlineto{\pgfqpoint{3.960496in}{2.939006in}}%
\pgfpathlineto{\pgfqpoint{3.965037in}{2.939006in}}%
\pgfpathlineto{\pgfqpoint{3.965037in}{2.936056in}}%
\pgfpathmoveto{\pgfqpoint{3.960496in}{2.939006in}}%
\pgfpathlineto{\pgfqpoint{3.960496in}{2.939006in}}%
\pgfpathlineto{\pgfqpoint{3.960496in}{2.941955in}}%
\pgfpathlineto{\pgfqpoint{3.965037in}{2.941955in}}%
\pgfpathlineto{\pgfqpoint{3.965037in}{2.939006in}}%
\pgfpathmoveto{\pgfqpoint{3.946873in}{2.941955in}}%
\pgfpathlineto{\pgfqpoint{3.946873in}{2.941955in}}%
\pgfpathlineto{\pgfqpoint{3.946873in}{2.944904in}}%
\pgfpathlineto{\pgfqpoint{3.951414in}{2.944904in}}%
\pgfpathlineto{\pgfqpoint{3.951414in}{2.941955in}}%
\pgfpathmoveto{\pgfqpoint{3.946873in}{2.944904in}}%
\pgfpathlineto{\pgfqpoint{3.946873in}{2.944904in}}%
\pgfpathlineto{\pgfqpoint{3.946873in}{2.947853in}}%
\pgfpathlineto{\pgfqpoint{3.951414in}{2.947853in}}%
\pgfpathlineto{\pgfqpoint{3.951414in}{2.944904in}}%
\pgfpathmoveto{\pgfqpoint{3.951414in}{2.941955in}}%
\pgfpathlineto{\pgfqpoint{3.951414in}{2.941955in}}%
\pgfpathlineto{\pgfqpoint{3.951414in}{2.944904in}}%
\pgfpathlineto{\pgfqpoint{3.955955in}{2.944904in}}%
\pgfpathlineto{\pgfqpoint{3.955955in}{2.941955in}}%
\pgfpathmoveto{\pgfqpoint{3.951414in}{2.944904in}}%
\pgfpathlineto{\pgfqpoint{3.951414in}{2.944904in}}%
\pgfpathlineto{\pgfqpoint{3.951414in}{2.947853in}}%
\pgfpathlineto{\pgfqpoint{3.955955in}{2.947853in}}%
\pgfpathlineto{\pgfqpoint{3.955955in}{2.944904in}}%
\pgfpathmoveto{\pgfqpoint{3.946873in}{2.947853in}}%
\pgfpathlineto{\pgfqpoint{3.946873in}{2.947853in}}%
\pgfpathlineto{\pgfqpoint{3.946873in}{2.950803in}}%
\pgfpathlineto{\pgfqpoint{3.951414in}{2.950803in}}%
\pgfpathlineto{\pgfqpoint{3.951414in}{2.947853in}}%
\pgfpathmoveto{\pgfqpoint{3.946873in}{2.950803in}}%
\pgfpathlineto{\pgfqpoint{3.946873in}{2.950803in}}%
\pgfpathlineto{\pgfqpoint{3.946873in}{2.953752in}}%
\pgfpathlineto{\pgfqpoint{3.951414in}{2.953752in}}%
\pgfpathlineto{\pgfqpoint{3.951414in}{2.950803in}}%
\pgfpathmoveto{\pgfqpoint{3.951414in}{2.947853in}}%
\pgfpathlineto{\pgfqpoint{3.951414in}{2.947853in}}%
\pgfpathlineto{\pgfqpoint{3.951414in}{2.950803in}}%
\pgfpathlineto{\pgfqpoint{3.955955in}{2.950803in}}%
\pgfpathlineto{\pgfqpoint{3.955955in}{2.947853in}}%
\pgfpathmoveto{\pgfqpoint{3.955955in}{2.941955in}}%
\pgfpathlineto{\pgfqpoint{3.955955in}{2.941955in}}%
\pgfpathlineto{\pgfqpoint{3.955955in}{2.944904in}}%
\pgfpathlineto{\pgfqpoint{3.960496in}{2.944904in}}%
\pgfpathlineto{\pgfqpoint{3.960496in}{2.941955in}}%
\pgfpathmoveto{\pgfqpoint{3.955955in}{2.944904in}}%
\pgfpathlineto{\pgfqpoint{3.955955in}{2.944904in}}%
\pgfpathlineto{\pgfqpoint{3.955955in}{2.947853in}}%
\pgfpathlineto{\pgfqpoint{3.960496in}{2.947853in}}%
\pgfpathlineto{\pgfqpoint{3.960496in}{2.944904in}}%
\pgfpathmoveto{\pgfqpoint{3.960496in}{2.941955in}}%
\pgfpathlineto{\pgfqpoint{3.960496in}{2.941955in}}%
\pgfpathlineto{\pgfqpoint{3.960496in}{2.944904in}}%
\pgfpathlineto{\pgfqpoint{3.965037in}{2.944904in}}%
\pgfpathlineto{\pgfqpoint{3.965037in}{2.941955in}}%
\pgfpathmoveto{\pgfqpoint{3.965037in}{2.930158in}}%
\pgfpathlineto{\pgfqpoint{3.965037in}{2.930158in}}%
\pgfpathlineto{\pgfqpoint{3.965037in}{2.933107in}}%
\pgfpathlineto{\pgfqpoint{3.969578in}{2.933107in}}%
\pgfpathlineto{\pgfqpoint{3.969578in}{2.930158in}}%
\pgfpathmoveto{\pgfqpoint{3.965037in}{2.933107in}}%
\pgfpathlineto{\pgfqpoint{3.965037in}{2.933107in}}%
\pgfpathlineto{\pgfqpoint{3.965037in}{2.936056in}}%
\pgfpathlineto{\pgfqpoint{3.969578in}{2.936056in}}%
\pgfpathlineto{\pgfqpoint{3.969578in}{2.933107in}}%
\pgfpathmoveto{\pgfqpoint{3.969578in}{2.930158in}}%
\pgfpathlineto{\pgfqpoint{3.969578in}{2.930158in}}%
\pgfpathlineto{\pgfqpoint{3.969578in}{2.933107in}}%
\pgfpathlineto{\pgfqpoint{3.974119in}{2.933107in}}%
\pgfpathlineto{\pgfqpoint{3.974119in}{2.930158in}}%
\pgfpathmoveto{\pgfqpoint{3.969578in}{2.933107in}}%
\pgfpathlineto{\pgfqpoint{3.969578in}{2.933107in}}%
\pgfpathlineto{\pgfqpoint{3.969578in}{2.936056in}}%
\pgfpathlineto{\pgfqpoint{3.974119in}{2.936056in}}%
\pgfpathlineto{\pgfqpoint{3.974119in}{2.933107in}}%
\pgfpathmoveto{\pgfqpoint{3.965037in}{2.936056in}}%
\pgfpathlineto{\pgfqpoint{3.965037in}{2.936056in}}%
\pgfpathlineto{\pgfqpoint{3.965037in}{2.939006in}}%
\pgfpathlineto{\pgfqpoint{3.969578in}{2.939006in}}%
\pgfpathlineto{\pgfqpoint{3.969578in}{2.936056in}}%
\pgfpathmoveto{\pgfqpoint{3.965037in}{2.939006in}}%
\pgfpathlineto{\pgfqpoint{3.965037in}{2.939006in}}%
\pgfpathlineto{\pgfqpoint{3.965037in}{2.941955in}}%
\pgfpathlineto{\pgfqpoint{3.969578in}{2.941955in}}%
\pgfpathlineto{\pgfqpoint{3.969578in}{2.939006in}}%
\pgfpathmoveto{\pgfqpoint{3.969578in}{2.936056in}}%
\pgfpathlineto{\pgfqpoint{3.969578in}{2.936056in}}%
\pgfpathlineto{\pgfqpoint{3.969578in}{2.939006in}}%
\pgfpathlineto{\pgfqpoint{3.974119in}{2.939006in}}%
\pgfpathlineto{\pgfqpoint{3.974119in}{2.936056in}}%
\pgfpathmoveto{\pgfqpoint{3.974119in}{2.930158in}}%
\pgfpathlineto{\pgfqpoint{3.974119in}{2.930158in}}%
\pgfpathlineto{\pgfqpoint{3.974119in}{2.933107in}}%
\pgfpathlineto{\pgfqpoint{3.978661in}{2.933107in}}%
\pgfpathlineto{\pgfqpoint{3.978661in}{2.930158in}}%
\pgfpathmoveto{\pgfqpoint{3.974119in}{2.933107in}}%
\pgfpathlineto{\pgfqpoint{3.974119in}{2.933107in}}%
\pgfpathlineto{\pgfqpoint{3.974119in}{2.936056in}}%
\pgfpathlineto{\pgfqpoint{3.978661in}{2.936056in}}%
\pgfpathlineto{\pgfqpoint{3.978661in}{2.933107in}}%
\pgfpathmoveto{\pgfqpoint{3.978661in}{2.930158in}}%
\pgfpathlineto{\pgfqpoint{3.978661in}{2.930158in}}%
\pgfpathlineto{\pgfqpoint{3.978661in}{2.933107in}}%
\pgfpathlineto{\pgfqpoint{3.983202in}{2.933107in}}%
\pgfpathlineto{\pgfqpoint{3.983202in}{2.930158in}}%
\pgfpathmoveto{\pgfqpoint{3.992284in}{2.912462in}}%
\pgfpathlineto{\pgfqpoint{3.992284in}{2.912462in}}%
\pgfpathlineto{\pgfqpoint{3.992284in}{2.915411in}}%
\pgfpathlineto{\pgfqpoint{3.996825in}{2.915411in}}%
\pgfpathlineto{\pgfqpoint{3.996825in}{2.912462in}}%
\pgfpathmoveto{\pgfqpoint{3.992284in}{2.915411in}}%
\pgfpathlineto{\pgfqpoint{3.992284in}{2.915411in}}%
\pgfpathlineto{\pgfqpoint{3.992284in}{2.918361in}}%
\pgfpathlineto{\pgfqpoint{3.996825in}{2.918361in}}%
\pgfpathlineto{\pgfqpoint{3.996825in}{2.915411in}}%
\pgfpathmoveto{\pgfqpoint{3.996825in}{2.912462in}}%
\pgfpathlineto{\pgfqpoint{3.996825in}{2.912462in}}%
\pgfpathlineto{\pgfqpoint{3.996825in}{2.915411in}}%
\pgfpathlineto{\pgfqpoint{4.001366in}{2.915411in}}%
\pgfpathlineto{\pgfqpoint{4.001366in}{2.912462in}}%
\pgfpathmoveto{\pgfqpoint{3.996825in}{2.915411in}}%
\pgfpathlineto{\pgfqpoint{3.996825in}{2.915411in}}%
\pgfpathlineto{\pgfqpoint{3.996825in}{2.918361in}}%
\pgfpathlineto{\pgfqpoint{4.001366in}{2.918361in}}%
\pgfpathlineto{\pgfqpoint{4.001366in}{2.915411in}}%
\pgfpathmoveto{\pgfqpoint{3.983202in}{2.918361in}}%
\pgfpathlineto{\pgfqpoint{3.983202in}{2.918361in}}%
\pgfpathlineto{\pgfqpoint{3.983202in}{2.921310in}}%
\pgfpathlineto{\pgfqpoint{3.987743in}{2.921310in}}%
\pgfpathlineto{\pgfqpoint{3.987743in}{2.918361in}}%
\pgfpathmoveto{\pgfqpoint{3.983202in}{2.921310in}}%
\pgfpathlineto{\pgfqpoint{3.983202in}{2.921310in}}%
\pgfpathlineto{\pgfqpoint{3.983202in}{2.924259in}}%
\pgfpathlineto{\pgfqpoint{3.987743in}{2.924259in}}%
\pgfpathlineto{\pgfqpoint{3.987743in}{2.921310in}}%
\pgfpathmoveto{\pgfqpoint{3.987743in}{2.918361in}}%
\pgfpathlineto{\pgfqpoint{3.987743in}{2.918361in}}%
\pgfpathlineto{\pgfqpoint{3.987743in}{2.921310in}}%
\pgfpathlineto{\pgfqpoint{3.992284in}{2.921310in}}%
\pgfpathlineto{\pgfqpoint{3.992284in}{2.918361in}}%
\pgfpathmoveto{\pgfqpoint{3.987743in}{2.921310in}}%
\pgfpathlineto{\pgfqpoint{3.987743in}{2.921310in}}%
\pgfpathlineto{\pgfqpoint{3.987743in}{2.924259in}}%
\pgfpathlineto{\pgfqpoint{3.992284in}{2.924259in}}%
\pgfpathlineto{\pgfqpoint{3.992284in}{2.921310in}}%
\pgfpathmoveto{\pgfqpoint{3.983202in}{2.924259in}}%
\pgfpathlineto{\pgfqpoint{3.983202in}{2.924259in}}%
\pgfpathlineto{\pgfqpoint{3.983202in}{2.927209in}}%
\pgfpathlineto{\pgfqpoint{3.987743in}{2.927209in}}%
\pgfpathlineto{\pgfqpoint{3.987743in}{2.924259in}}%
\pgfpathmoveto{\pgfqpoint{3.983202in}{2.927209in}}%
\pgfpathlineto{\pgfqpoint{3.983202in}{2.927209in}}%
\pgfpathlineto{\pgfqpoint{3.983202in}{2.930158in}}%
\pgfpathlineto{\pgfqpoint{3.987743in}{2.930158in}}%
\pgfpathlineto{\pgfqpoint{3.987743in}{2.927209in}}%
\pgfpathmoveto{\pgfqpoint{3.987743in}{2.924259in}}%
\pgfpathlineto{\pgfqpoint{3.987743in}{2.924259in}}%
\pgfpathlineto{\pgfqpoint{3.987743in}{2.927209in}}%
\pgfpathlineto{\pgfqpoint{3.992284in}{2.927209in}}%
\pgfpathlineto{\pgfqpoint{3.992284in}{2.924259in}}%
\pgfpathmoveto{\pgfqpoint{3.992284in}{2.918361in}}%
\pgfpathlineto{\pgfqpoint{3.992284in}{2.918361in}}%
\pgfpathlineto{\pgfqpoint{3.992284in}{2.921310in}}%
\pgfpathlineto{\pgfqpoint{3.996825in}{2.921310in}}%
\pgfpathlineto{\pgfqpoint{3.996825in}{2.918361in}}%
\pgfpathmoveto{\pgfqpoint{3.992284in}{2.921310in}}%
\pgfpathlineto{\pgfqpoint{3.992284in}{2.921310in}}%
\pgfpathlineto{\pgfqpoint{3.992284in}{2.924259in}}%
\pgfpathlineto{\pgfqpoint{3.996825in}{2.924259in}}%
\pgfpathlineto{\pgfqpoint{3.996825in}{2.921310in}}%
\pgfpathmoveto{\pgfqpoint{3.996825in}{2.918361in}}%
\pgfpathlineto{\pgfqpoint{3.996825in}{2.918361in}}%
\pgfpathlineto{\pgfqpoint{3.996825in}{2.921310in}}%
\pgfpathlineto{\pgfqpoint{4.001366in}{2.921310in}}%
\pgfpathlineto{\pgfqpoint{4.001366in}{2.918361in}}%
\pgfpathmoveto{\pgfqpoint{4.001366in}{2.906564in}}%
\pgfpathlineto{\pgfqpoint{4.001366in}{2.906564in}}%
\pgfpathlineto{\pgfqpoint{4.001366in}{2.909513in}}%
\pgfpathlineto{\pgfqpoint{4.005907in}{2.909513in}}%
\pgfpathlineto{\pgfqpoint{4.005907in}{2.906564in}}%
\pgfpathmoveto{\pgfqpoint{4.001366in}{2.909513in}}%
\pgfpathlineto{\pgfqpoint{4.001366in}{2.909513in}}%
\pgfpathlineto{\pgfqpoint{4.001366in}{2.912462in}}%
\pgfpathlineto{\pgfqpoint{4.005907in}{2.912462in}}%
\pgfpathlineto{\pgfqpoint{4.005907in}{2.909513in}}%
\pgfpathmoveto{\pgfqpoint{4.005907in}{2.906564in}}%
\pgfpathlineto{\pgfqpoint{4.005907in}{2.906564in}}%
\pgfpathlineto{\pgfqpoint{4.005907in}{2.909513in}}%
\pgfpathlineto{\pgfqpoint{4.010448in}{2.909513in}}%
\pgfpathlineto{\pgfqpoint{4.010448in}{2.906564in}}%
\pgfpathmoveto{\pgfqpoint{4.005907in}{2.909513in}}%
\pgfpathlineto{\pgfqpoint{4.005907in}{2.909513in}}%
\pgfpathlineto{\pgfqpoint{4.005907in}{2.912462in}}%
\pgfpathlineto{\pgfqpoint{4.010448in}{2.912462in}}%
\pgfpathlineto{\pgfqpoint{4.010448in}{2.909513in}}%
\pgfpathmoveto{\pgfqpoint{4.001366in}{2.912462in}}%
\pgfpathlineto{\pgfqpoint{4.001366in}{2.912462in}}%
\pgfpathlineto{\pgfqpoint{4.001366in}{2.915411in}}%
\pgfpathlineto{\pgfqpoint{4.005907in}{2.915411in}}%
\pgfpathlineto{\pgfqpoint{4.005907in}{2.912462in}}%
\pgfpathmoveto{\pgfqpoint{4.001366in}{2.915411in}}%
\pgfpathlineto{\pgfqpoint{4.001366in}{2.915411in}}%
\pgfpathlineto{\pgfqpoint{4.001366in}{2.918361in}}%
\pgfpathlineto{\pgfqpoint{4.005907in}{2.918361in}}%
\pgfpathlineto{\pgfqpoint{4.005907in}{2.915411in}}%
\pgfpathmoveto{\pgfqpoint{4.005907in}{2.912462in}}%
\pgfpathlineto{\pgfqpoint{4.005907in}{2.912462in}}%
\pgfpathlineto{\pgfqpoint{4.005907in}{2.915411in}}%
\pgfpathlineto{\pgfqpoint{4.010448in}{2.915411in}}%
\pgfpathlineto{\pgfqpoint{4.010448in}{2.912462in}}%
\pgfpathmoveto{\pgfqpoint{4.010448in}{2.906564in}}%
\pgfpathlineto{\pgfqpoint{4.010448in}{2.906564in}}%
\pgfpathlineto{\pgfqpoint{4.010448in}{2.909513in}}%
\pgfpathlineto{\pgfqpoint{4.014990in}{2.909513in}}%
\pgfpathlineto{\pgfqpoint{4.014990in}{2.906564in}}%
\pgfpathmoveto{\pgfqpoint{4.010448in}{2.909513in}}%
\pgfpathlineto{\pgfqpoint{4.010448in}{2.909513in}}%
\pgfpathlineto{\pgfqpoint{4.010448in}{2.912462in}}%
\pgfpathlineto{\pgfqpoint{4.014990in}{2.912462in}}%
\pgfpathlineto{\pgfqpoint{4.014990in}{2.909513in}}%
\pgfpathmoveto{\pgfqpoint{4.014990in}{2.906564in}}%
\pgfpathlineto{\pgfqpoint{4.014990in}{2.906564in}}%
\pgfpathlineto{\pgfqpoint{4.014990in}{2.909513in}}%
\pgfpathlineto{\pgfqpoint{4.019531in}{2.909513in}}%
\pgfpathlineto{\pgfqpoint{4.019531in}{2.906564in}}%
\pgfpathmoveto{\pgfqpoint{4.046777in}{2.877071in}}%
\pgfpathlineto{\pgfqpoint{4.046777in}{2.877071in}}%
\pgfpathlineto{\pgfqpoint{4.046777in}{2.880020in}}%
\pgfpathlineto{\pgfqpoint{4.051319in}{2.880020in}}%
\pgfpathlineto{\pgfqpoint{4.051319in}{2.877071in}}%
\pgfpathmoveto{\pgfqpoint{4.046777in}{2.880020in}}%
\pgfpathlineto{\pgfqpoint{4.046777in}{2.880020in}}%
\pgfpathlineto{\pgfqpoint{4.046777in}{2.882970in}}%
\pgfpathlineto{\pgfqpoint{4.051319in}{2.882970in}}%
\pgfpathlineto{\pgfqpoint{4.051319in}{2.880020in}}%
\pgfpathmoveto{\pgfqpoint{4.051319in}{2.877071in}}%
\pgfpathlineto{\pgfqpoint{4.051319in}{2.877071in}}%
\pgfpathlineto{\pgfqpoint{4.051319in}{2.880020in}}%
\pgfpathlineto{\pgfqpoint{4.055860in}{2.880020in}}%
\pgfpathlineto{\pgfqpoint{4.055860in}{2.877071in}}%
\pgfpathmoveto{\pgfqpoint{4.051319in}{2.880020in}}%
\pgfpathlineto{\pgfqpoint{4.051319in}{2.880020in}}%
\pgfpathlineto{\pgfqpoint{4.051319in}{2.882970in}}%
\pgfpathlineto{\pgfqpoint{4.055860in}{2.882970in}}%
\pgfpathlineto{\pgfqpoint{4.055860in}{2.880020in}}%
\pgfpathmoveto{\pgfqpoint{4.028613in}{2.888868in}}%
\pgfpathlineto{\pgfqpoint{4.028613in}{2.888868in}}%
\pgfpathlineto{\pgfqpoint{4.028613in}{2.891817in}}%
\pgfpathlineto{\pgfqpoint{4.033154in}{2.891817in}}%
\pgfpathlineto{\pgfqpoint{4.033154in}{2.888868in}}%
\pgfpathmoveto{\pgfqpoint{4.028613in}{2.891817in}}%
\pgfpathlineto{\pgfqpoint{4.028613in}{2.891817in}}%
\pgfpathlineto{\pgfqpoint{4.028613in}{2.894767in}}%
\pgfpathlineto{\pgfqpoint{4.033154in}{2.894767in}}%
\pgfpathlineto{\pgfqpoint{4.033154in}{2.891817in}}%
\pgfpathmoveto{\pgfqpoint{4.033154in}{2.888868in}}%
\pgfpathlineto{\pgfqpoint{4.033154in}{2.888868in}}%
\pgfpathlineto{\pgfqpoint{4.033154in}{2.891817in}}%
\pgfpathlineto{\pgfqpoint{4.037695in}{2.891817in}}%
\pgfpathlineto{\pgfqpoint{4.037695in}{2.888868in}}%
\pgfpathmoveto{\pgfqpoint{4.033154in}{2.891817in}}%
\pgfpathlineto{\pgfqpoint{4.033154in}{2.891817in}}%
\pgfpathlineto{\pgfqpoint{4.033154in}{2.894767in}}%
\pgfpathlineto{\pgfqpoint{4.037695in}{2.894767in}}%
\pgfpathlineto{\pgfqpoint{4.037695in}{2.891817in}}%
\pgfpathmoveto{\pgfqpoint{4.019531in}{2.894767in}}%
\pgfpathlineto{\pgfqpoint{4.019531in}{2.894767in}}%
\pgfpathlineto{\pgfqpoint{4.019531in}{2.897716in}}%
\pgfpathlineto{\pgfqpoint{4.024072in}{2.897716in}}%
\pgfpathlineto{\pgfqpoint{4.024072in}{2.894767in}}%
\pgfpathmoveto{\pgfqpoint{4.019531in}{2.897716in}}%
\pgfpathlineto{\pgfqpoint{4.019531in}{2.897716in}}%
\pgfpathlineto{\pgfqpoint{4.019531in}{2.900665in}}%
\pgfpathlineto{\pgfqpoint{4.024072in}{2.900665in}}%
\pgfpathlineto{\pgfqpoint{4.024072in}{2.897716in}}%
\pgfpathmoveto{\pgfqpoint{4.024072in}{2.894767in}}%
\pgfpathlineto{\pgfqpoint{4.024072in}{2.894767in}}%
\pgfpathlineto{\pgfqpoint{4.024072in}{2.897716in}}%
\pgfpathlineto{\pgfqpoint{4.028613in}{2.897716in}}%
\pgfpathlineto{\pgfqpoint{4.028613in}{2.894767in}}%
\pgfpathmoveto{\pgfqpoint{4.024072in}{2.897716in}}%
\pgfpathlineto{\pgfqpoint{4.024072in}{2.897716in}}%
\pgfpathlineto{\pgfqpoint{4.024072in}{2.900665in}}%
\pgfpathlineto{\pgfqpoint{4.028613in}{2.900665in}}%
\pgfpathlineto{\pgfqpoint{4.028613in}{2.897716in}}%
\pgfpathmoveto{\pgfqpoint{4.019531in}{2.900665in}}%
\pgfpathlineto{\pgfqpoint{4.019531in}{2.900665in}}%
\pgfpathlineto{\pgfqpoint{4.019531in}{2.903614in}}%
\pgfpathlineto{\pgfqpoint{4.024072in}{2.903614in}}%
\pgfpathlineto{\pgfqpoint{4.024072in}{2.900665in}}%
\pgfpathmoveto{\pgfqpoint{4.019531in}{2.903614in}}%
\pgfpathlineto{\pgfqpoint{4.019531in}{2.903614in}}%
\pgfpathlineto{\pgfqpoint{4.019531in}{2.906564in}}%
\pgfpathlineto{\pgfqpoint{4.024072in}{2.906564in}}%
\pgfpathlineto{\pgfqpoint{4.024072in}{2.903614in}}%
\pgfpathmoveto{\pgfqpoint{4.024072in}{2.900665in}}%
\pgfpathlineto{\pgfqpoint{4.024072in}{2.900665in}}%
\pgfpathlineto{\pgfqpoint{4.024072in}{2.903614in}}%
\pgfpathlineto{\pgfqpoint{4.028613in}{2.903614in}}%
\pgfpathlineto{\pgfqpoint{4.028613in}{2.900665in}}%
\pgfpathmoveto{\pgfqpoint{4.028613in}{2.894767in}}%
\pgfpathlineto{\pgfqpoint{4.028613in}{2.894767in}}%
\pgfpathlineto{\pgfqpoint{4.028613in}{2.897716in}}%
\pgfpathlineto{\pgfqpoint{4.033154in}{2.897716in}}%
\pgfpathlineto{\pgfqpoint{4.033154in}{2.894767in}}%
\pgfpathmoveto{\pgfqpoint{4.028613in}{2.897716in}}%
\pgfpathlineto{\pgfqpoint{4.028613in}{2.897716in}}%
\pgfpathlineto{\pgfqpoint{4.028613in}{2.900665in}}%
\pgfpathlineto{\pgfqpoint{4.033154in}{2.900665in}}%
\pgfpathlineto{\pgfqpoint{4.033154in}{2.897716in}}%
\pgfpathmoveto{\pgfqpoint{4.033154in}{2.894767in}}%
\pgfpathlineto{\pgfqpoint{4.033154in}{2.894767in}}%
\pgfpathlineto{\pgfqpoint{4.033154in}{2.897716in}}%
\pgfpathlineto{\pgfqpoint{4.037695in}{2.897716in}}%
\pgfpathlineto{\pgfqpoint{4.037695in}{2.894767in}}%
\pgfpathmoveto{\pgfqpoint{4.037695in}{2.882970in}}%
\pgfpathlineto{\pgfqpoint{4.037695in}{2.882970in}}%
\pgfpathlineto{\pgfqpoint{4.037695in}{2.885919in}}%
\pgfpathlineto{\pgfqpoint{4.042236in}{2.885919in}}%
\pgfpathlineto{\pgfqpoint{4.042236in}{2.882970in}}%
\pgfpathmoveto{\pgfqpoint{4.037695in}{2.885919in}}%
\pgfpathlineto{\pgfqpoint{4.037695in}{2.885919in}}%
\pgfpathlineto{\pgfqpoint{4.037695in}{2.888868in}}%
\pgfpathlineto{\pgfqpoint{4.042236in}{2.888868in}}%
\pgfpathlineto{\pgfqpoint{4.042236in}{2.885919in}}%
\pgfpathmoveto{\pgfqpoint{4.042236in}{2.882970in}}%
\pgfpathlineto{\pgfqpoint{4.042236in}{2.882970in}}%
\pgfpathlineto{\pgfqpoint{4.042236in}{2.885919in}}%
\pgfpathlineto{\pgfqpoint{4.046777in}{2.885919in}}%
\pgfpathlineto{\pgfqpoint{4.046777in}{2.882970in}}%
\pgfpathmoveto{\pgfqpoint{4.042236in}{2.885919in}}%
\pgfpathlineto{\pgfqpoint{4.042236in}{2.885919in}}%
\pgfpathlineto{\pgfqpoint{4.042236in}{2.888868in}}%
\pgfpathlineto{\pgfqpoint{4.046777in}{2.888868in}}%
\pgfpathlineto{\pgfqpoint{4.046777in}{2.885919in}}%
\pgfpathmoveto{\pgfqpoint{4.037695in}{2.888868in}}%
\pgfpathlineto{\pgfqpoint{4.037695in}{2.888868in}}%
\pgfpathlineto{\pgfqpoint{4.037695in}{2.891817in}}%
\pgfpathlineto{\pgfqpoint{4.042236in}{2.891817in}}%
\pgfpathlineto{\pgfqpoint{4.042236in}{2.888868in}}%
\pgfpathmoveto{\pgfqpoint{4.037695in}{2.891817in}}%
\pgfpathlineto{\pgfqpoint{4.037695in}{2.891817in}}%
\pgfpathlineto{\pgfqpoint{4.037695in}{2.894767in}}%
\pgfpathlineto{\pgfqpoint{4.042236in}{2.894767in}}%
\pgfpathlineto{\pgfqpoint{4.042236in}{2.891817in}}%
\pgfpathmoveto{\pgfqpoint{4.042236in}{2.888868in}}%
\pgfpathlineto{\pgfqpoint{4.042236in}{2.888868in}}%
\pgfpathlineto{\pgfqpoint{4.042236in}{2.891817in}}%
\pgfpathlineto{\pgfqpoint{4.046777in}{2.891817in}}%
\pgfpathlineto{\pgfqpoint{4.046777in}{2.888868in}}%
\pgfpathmoveto{\pgfqpoint{4.046777in}{2.882970in}}%
\pgfpathlineto{\pgfqpoint{4.046777in}{2.882970in}}%
\pgfpathlineto{\pgfqpoint{4.046777in}{2.885919in}}%
\pgfpathlineto{\pgfqpoint{4.051319in}{2.885919in}}%
\pgfpathlineto{\pgfqpoint{4.051319in}{2.882970in}}%
\pgfpathmoveto{\pgfqpoint{4.046777in}{2.885919in}}%
\pgfpathlineto{\pgfqpoint{4.046777in}{2.885919in}}%
\pgfpathlineto{\pgfqpoint{4.046777in}{2.888868in}}%
\pgfpathlineto{\pgfqpoint{4.051319in}{2.888868in}}%
\pgfpathlineto{\pgfqpoint{4.051319in}{2.885919in}}%
\pgfpathmoveto{\pgfqpoint{4.051319in}{2.882970in}}%
\pgfpathlineto{\pgfqpoint{4.051319in}{2.882970in}}%
\pgfpathlineto{\pgfqpoint{4.051319in}{2.885919in}}%
\pgfpathlineto{\pgfqpoint{4.055860in}{2.885919in}}%
\pgfpathlineto{\pgfqpoint{4.055860in}{2.882970in}}%
\pgfpathmoveto{\pgfqpoint{4.064942in}{2.865274in}}%
\pgfpathlineto{\pgfqpoint{4.064942in}{2.865274in}}%
\pgfpathlineto{\pgfqpoint{4.064942in}{2.868223in}}%
\pgfpathlineto{\pgfqpoint{4.069483in}{2.868223in}}%
\pgfpathlineto{\pgfqpoint{4.069483in}{2.865274in}}%
\pgfpathmoveto{\pgfqpoint{4.064942in}{2.868223in}}%
\pgfpathlineto{\pgfqpoint{4.064942in}{2.868223in}}%
\pgfpathlineto{\pgfqpoint{4.064942in}{2.871173in}}%
\pgfpathlineto{\pgfqpoint{4.069483in}{2.871173in}}%
\pgfpathlineto{\pgfqpoint{4.069483in}{2.868223in}}%
\pgfpathmoveto{\pgfqpoint{4.069483in}{2.865274in}}%
\pgfpathlineto{\pgfqpoint{4.069483in}{2.865274in}}%
\pgfpathlineto{\pgfqpoint{4.069483in}{2.868223in}}%
\pgfpathlineto{\pgfqpoint{4.074024in}{2.868223in}}%
\pgfpathlineto{\pgfqpoint{4.074024in}{2.865274in}}%
\pgfpathmoveto{\pgfqpoint{4.069483in}{2.868223in}}%
\pgfpathlineto{\pgfqpoint{4.069483in}{2.868223in}}%
\pgfpathlineto{\pgfqpoint{4.069483in}{2.871173in}}%
\pgfpathlineto{\pgfqpoint{4.074024in}{2.871173in}}%
\pgfpathlineto{\pgfqpoint{4.074024in}{2.868223in}}%
\pgfpathmoveto{\pgfqpoint{4.055860in}{2.871173in}}%
\pgfpathlineto{\pgfqpoint{4.055860in}{2.871173in}}%
\pgfpathlineto{\pgfqpoint{4.055860in}{2.874122in}}%
\pgfpathlineto{\pgfqpoint{4.060401in}{2.874122in}}%
\pgfpathlineto{\pgfqpoint{4.060401in}{2.871173in}}%
\pgfpathmoveto{\pgfqpoint{4.055860in}{2.874122in}}%
\pgfpathlineto{\pgfqpoint{4.055860in}{2.874122in}}%
\pgfpathlineto{\pgfqpoint{4.055860in}{2.877071in}}%
\pgfpathlineto{\pgfqpoint{4.060401in}{2.877071in}}%
\pgfpathlineto{\pgfqpoint{4.060401in}{2.874122in}}%
\pgfpathmoveto{\pgfqpoint{4.060401in}{2.871173in}}%
\pgfpathlineto{\pgfqpoint{4.060401in}{2.871173in}}%
\pgfpathlineto{\pgfqpoint{4.060401in}{2.874122in}}%
\pgfpathlineto{\pgfqpoint{4.064942in}{2.874122in}}%
\pgfpathlineto{\pgfqpoint{4.064942in}{2.871173in}}%
\pgfpathmoveto{\pgfqpoint{4.060401in}{2.874122in}}%
\pgfpathlineto{\pgfqpoint{4.060401in}{2.874122in}}%
\pgfpathlineto{\pgfqpoint{4.060401in}{2.877071in}}%
\pgfpathlineto{\pgfqpoint{4.064942in}{2.877071in}}%
\pgfpathlineto{\pgfqpoint{4.064942in}{2.874122in}}%
\pgfpathmoveto{\pgfqpoint{4.055860in}{2.877071in}}%
\pgfpathlineto{\pgfqpoint{4.055860in}{2.877071in}}%
\pgfpathlineto{\pgfqpoint{4.055860in}{2.880020in}}%
\pgfpathlineto{\pgfqpoint{4.060401in}{2.880020in}}%
\pgfpathlineto{\pgfqpoint{4.060401in}{2.877071in}}%
\pgfpathmoveto{\pgfqpoint{4.055860in}{2.880020in}}%
\pgfpathlineto{\pgfqpoint{4.055860in}{2.880020in}}%
\pgfpathlineto{\pgfqpoint{4.055860in}{2.882970in}}%
\pgfpathlineto{\pgfqpoint{4.060401in}{2.882970in}}%
\pgfpathlineto{\pgfqpoint{4.060401in}{2.880020in}}%
\pgfpathmoveto{\pgfqpoint{4.060401in}{2.877071in}}%
\pgfpathlineto{\pgfqpoint{4.060401in}{2.877071in}}%
\pgfpathlineto{\pgfqpoint{4.060401in}{2.880020in}}%
\pgfpathlineto{\pgfqpoint{4.064942in}{2.880020in}}%
\pgfpathlineto{\pgfqpoint{4.064942in}{2.877071in}}%
\pgfpathmoveto{\pgfqpoint{4.064942in}{2.871173in}}%
\pgfpathlineto{\pgfqpoint{4.064942in}{2.871173in}}%
\pgfpathlineto{\pgfqpoint{4.064942in}{2.874122in}}%
\pgfpathlineto{\pgfqpoint{4.069483in}{2.874122in}}%
\pgfpathlineto{\pgfqpoint{4.069483in}{2.871173in}}%
\pgfpathmoveto{\pgfqpoint{4.064942in}{2.874122in}}%
\pgfpathlineto{\pgfqpoint{4.064942in}{2.874122in}}%
\pgfpathlineto{\pgfqpoint{4.064942in}{2.877071in}}%
\pgfpathlineto{\pgfqpoint{4.069483in}{2.877071in}}%
\pgfpathlineto{\pgfqpoint{4.069483in}{2.874122in}}%
\pgfpathmoveto{\pgfqpoint{4.069483in}{2.871173in}}%
\pgfpathlineto{\pgfqpoint{4.069483in}{2.871173in}}%
\pgfpathlineto{\pgfqpoint{4.069483in}{2.874122in}}%
\pgfpathlineto{\pgfqpoint{4.074024in}{2.874122in}}%
\pgfpathlineto{\pgfqpoint{4.074024in}{2.871173in}}%
\pgfpathmoveto{\pgfqpoint{4.074024in}{2.859375in}}%
\pgfpathlineto{\pgfqpoint{4.074024in}{2.859375in}}%
\pgfpathlineto{\pgfqpoint{4.074024in}{2.862325in}}%
\pgfpathlineto{\pgfqpoint{4.078565in}{2.862325in}}%
\pgfpathlineto{\pgfqpoint{4.078565in}{2.859375in}}%
\pgfpathmoveto{\pgfqpoint{4.074024in}{2.862325in}}%
\pgfpathlineto{\pgfqpoint{4.074024in}{2.862325in}}%
\pgfpathlineto{\pgfqpoint{4.074024in}{2.865274in}}%
\pgfpathlineto{\pgfqpoint{4.078565in}{2.865274in}}%
\pgfpathlineto{\pgfqpoint{4.078565in}{2.862325in}}%
\pgfpathmoveto{\pgfqpoint{4.078565in}{2.859375in}}%
\pgfpathlineto{\pgfqpoint{4.078565in}{2.859375in}}%
\pgfpathlineto{\pgfqpoint{4.078565in}{2.862325in}}%
\pgfpathlineto{\pgfqpoint{4.083106in}{2.862325in}}%
\pgfpathlineto{\pgfqpoint{4.083106in}{2.859375in}}%
\pgfpathmoveto{\pgfqpoint{4.078565in}{2.862325in}}%
\pgfpathlineto{\pgfqpoint{4.078565in}{2.862325in}}%
\pgfpathlineto{\pgfqpoint{4.078565in}{2.865274in}}%
\pgfpathlineto{\pgfqpoint{4.083106in}{2.865274in}}%
\pgfpathlineto{\pgfqpoint{4.083106in}{2.862325in}}%
\pgfpathmoveto{\pgfqpoint{4.074024in}{2.865274in}}%
\pgfpathlineto{\pgfqpoint{4.074024in}{2.865274in}}%
\pgfpathlineto{\pgfqpoint{4.074024in}{2.868223in}}%
\pgfpathlineto{\pgfqpoint{4.078565in}{2.868223in}}%
\pgfpathlineto{\pgfqpoint{4.078565in}{2.865274in}}%
\pgfpathmoveto{\pgfqpoint{4.074024in}{2.868223in}}%
\pgfpathlineto{\pgfqpoint{4.074024in}{2.868223in}}%
\pgfpathlineto{\pgfqpoint{4.074024in}{2.871173in}}%
\pgfpathlineto{\pgfqpoint{4.078565in}{2.871173in}}%
\pgfpathlineto{\pgfqpoint{4.078565in}{2.868223in}}%
\pgfpathmoveto{\pgfqpoint{4.078565in}{2.865274in}}%
\pgfpathlineto{\pgfqpoint{4.078565in}{2.865274in}}%
\pgfpathlineto{\pgfqpoint{4.078565in}{2.868223in}}%
\pgfpathlineto{\pgfqpoint{4.083106in}{2.868223in}}%
\pgfpathlineto{\pgfqpoint{4.083106in}{2.865274in}}%
\pgfpathmoveto{\pgfqpoint{4.083106in}{2.859375in}}%
\pgfpathlineto{\pgfqpoint{4.083106in}{2.859375in}}%
\pgfpathlineto{\pgfqpoint{4.083106in}{2.862325in}}%
\pgfpathlineto{\pgfqpoint{4.087648in}{2.862325in}}%
\pgfpathlineto{\pgfqpoint{4.087648in}{2.859375in}}%
\pgfpathmoveto{\pgfqpoint{4.083106in}{2.862325in}}%
\pgfpathlineto{\pgfqpoint{4.083106in}{2.862325in}}%
\pgfpathlineto{\pgfqpoint{4.083106in}{2.865274in}}%
\pgfpathlineto{\pgfqpoint{4.087648in}{2.865274in}}%
\pgfpathlineto{\pgfqpoint{4.087648in}{2.862325in}}%
\pgfpathmoveto{\pgfqpoint{4.087648in}{2.859375in}}%
\pgfpathlineto{\pgfqpoint{4.087648in}{2.859375in}}%
\pgfpathlineto{\pgfqpoint{4.087648in}{2.862325in}}%
\pgfpathlineto{\pgfqpoint{4.092189in}{2.862325in}}%
\pgfpathlineto{\pgfqpoint{4.092189in}{2.859375in}}%
\pgfpathmoveto{\pgfqpoint{4.205716in}{2.667675in}}%
\pgfpathlineto{\pgfqpoint{4.205716in}{2.667675in}}%
\pgfpathlineto{\pgfqpoint{4.205716in}{2.670624in}}%
\pgfpathlineto{\pgfqpoint{4.210257in}{2.670624in}}%
\pgfpathlineto{\pgfqpoint{4.210257in}{2.667675in}}%
\pgfpathmoveto{\pgfqpoint{4.210257in}{2.667675in}}%
\pgfpathlineto{\pgfqpoint{4.210257in}{2.667675in}}%
\pgfpathlineto{\pgfqpoint{4.210257in}{2.670624in}}%
\pgfpathlineto{\pgfqpoint{4.214798in}{2.670624in}}%
\pgfpathlineto{\pgfqpoint{4.214798in}{2.667675in}}%
\pgfpathmoveto{\pgfqpoint{4.214798in}{2.667675in}}%
\pgfpathlineto{\pgfqpoint{4.214798in}{2.667675in}}%
\pgfpathlineto{\pgfqpoint{4.214798in}{2.670624in}}%
\pgfpathlineto{\pgfqpoint{4.219339in}{2.670624in}}%
\pgfpathlineto{\pgfqpoint{4.219339in}{2.667675in}}%
\pgfpathmoveto{\pgfqpoint{4.219339in}{2.664725in}}%
\pgfpathlineto{\pgfqpoint{4.219339in}{2.664725in}}%
\pgfpathlineto{\pgfqpoint{4.219339in}{2.667675in}}%
\pgfpathlineto{\pgfqpoint{4.223880in}{2.667675in}}%
\pgfpathlineto{\pgfqpoint{4.223880in}{2.664725in}}%
\pgfpathmoveto{\pgfqpoint{4.219339in}{2.667675in}}%
\pgfpathlineto{\pgfqpoint{4.219339in}{2.667675in}}%
\pgfpathlineto{\pgfqpoint{4.219339in}{2.670624in}}%
\pgfpathlineto{\pgfqpoint{4.223880in}{2.670624in}}%
\pgfpathlineto{\pgfqpoint{4.223880in}{2.667675in}}%
\pgfpathmoveto{\pgfqpoint{4.223880in}{2.664725in}}%
\pgfpathlineto{\pgfqpoint{4.223880in}{2.664725in}}%
\pgfpathlineto{\pgfqpoint{4.223880in}{2.667675in}}%
\pgfpathlineto{\pgfqpoint{4.228421in}{2.667675in}}%
\pgfpathlineto{\pgfqpoint{4.228421in}{2.664725in}}%
\pgfpathmoveto{\pgfqpoint{4.223880in}{2.667675in}}%
\pgfpathlineto{\pgfqpoint{4.223880in}{2.667675in}}%
\pgfpathlineto{\pgfqpoint{4.223880in}{2.670624in}}%
\pgfpathlineto{\pgfqpoint{4.228421in}{2.670624in}}%
\pgfpathlineto{\pgfqpoint{4.228421in}{2.667675in}}%
\pgfpathmoveto{\pgfqpoint{4.232962in}{2.661776in}}%
\pgfpathlineto{\pgfqpoint{4.232962in}{2.661776in}}%
\pgfpathlineto{\pgfqpoint{4.232962in}{2.664725in}}%
\pgfpathlineto{\pgfqpoint{4.237503in}{2.664725in}}%
\pgfpathlineto{\pgfqpoint{4.237503in}{2.661776in}}%
\pgfpathmoveto{\pgfqpoint{4.228421in}{2.664725in}}%
\pgfpathlineto{\pgfqpoint{4.228421in}{2.664725in}}%
\pgfpathlineto{\pgfqpoint{4.228421in}{2.667675in}}%
\pgfpathlineto{\pgfqpoint{4.232962in}{2.667675in}}%
\pgfpathlineto{\pgfqpoint{4.232962in}{2.664725in}}%
\pgfpathmoveto{\pgfqpoint{4.228421in}{2.667675in}}%
\pgfpathlineto{\pgfqpoint{4.228421in}{2.667675in}}%
\pgfpathlineto{\pgfqpoint{4.228421in}{2.670624in}}%
\pgfpathlineto{\pgfqpoint{4.232962in}{2.670624in}}%
\pgfpathlineto{\pgfqpoint{4.232962in}{2.667675in}}%
\pgfpathmoveto{\pgfqpoint{4.232962in}{2.664725in}}%
\pgfpathlineto{\pgfqpoint{4.232962in}{2.664725in}}%
\pgfpathlineto{\pgfqpoint{4.232962in}{2.667675in}}%
\pgfpathlineto{\pgfqpoint{4.237503in}{2.667675in}}%
\pgfpathlineto{\pgfqpoint{4.237503in}{2.664725in}}%
\pgfpathmoveto{\pgfqpoint{4.232962in}{2.667675in}}%
\pgfpathlineto{\pgfqpoint{4.232962in}{2.667675in}}%
\pgfpathlineto{\pgfqpoint{4.232962in}{2.670624in}}%
\pgfpathlineto{\pgfqpoint{4.237503in}{2.670624in}}%
\pgfpathlineto{\pgfqpoint{4.237503in}{2.667675in}}%
\pgfpathmoveto{\pgfqpoint{4.096730in}{2.691268in}}%
\pgfpathlineto{\pgfqpoint{4.096730in}{2.691268in}}%
\pgfpathlineto{\pgfqpoint{4.096730in}{2.694217in}}%
\pgfpathlineto{\pgfqpoint{4.101271in}{2.694217in}}%
\pgfpathlineto{\pgfqpoint{4.101271in}{2.691268in}}%
\pgfpathmoveto{\pgfqpoint{4.101271in}{2.691268in}}%
\pgfpathlineto{\pgfqpoint{4.101271in}{2.691268in}}%
\pgfpathlineto{\pgfqpoint{4.101271in}{2.694217in}}%
\pgfpathlineto{\pgfqpoint{4.105812in}{2.694217in}}%
\pgfpathlineto{\pgfqpoint{4.105812in}{2.691268in}}%
\pgfpathmoveto{\pgfqpoint{4.105812in}{2.691268in}}%
\pgfpathlineto{\pgfqpoint{4.105812in}{2.691268in}}%
\pgfpathlineto{\pgfqpoint{4.105812in}{2.694217in}}%
\pgfpathlineto{\pgfqpoint{4.110353in}{2.694217in}}%
\pgfpathlineto{\pgfqpoint{4.110353in}{2.691268in}}%
\pgfpathmoveto{\pgfqpoint{4.110353in}{2.688319in}}%
\pgfpathlineto{\pgfqpoint{4.110353in}{2.688319in}}%
\pgfpathlineto{\pgfqpoint{4.110353in}{2.691268in}}%
\pgfpathlineto{\pgfqpoint{4.114894in}{2.691268in}}%
\pgfpathlineto{\pgfqpoint{4.114894in}{2.688319in}}%
\pgfpathmoveto{\pgfqpoint{4.110353in}{2.691268in}}%
\pgfpathlineto{\pgfqpoint{4.110353in}{2.691268in}}%
\pgfpathlineto{\pgfqpoint{4.110353in}{2.694217in}}%
\pgfpathlineto{\pgfqpoint{4.114894in}{2.694217in}}%
\pgfpathlineto{\pgfqpoint{4.114894in}{2.691268in}}%
\pgfpathmoveto{\pgfqpoint{4.114894in}{2.688319in}}%
\pgfpathlineto{\pgfqpoint{4.114894in}{2.688319in}}%
\pgfpathlineto{\pgfqpoint{4.114894in}{2.691268in}}%
\pgfpathlineto{\pgfqpoint{4.119435in}{2.691268in}}%
\pgfpathlineto{\pgfqpoint{4.119435in}{2.688319in}}%
\pgfpathmoveto{\pgfqpoint{4.114894in}{2.691268in}}%
\pgfpathlineto{\pgfqpoint{4.114894in}{2.691268in}}%
\pgfpathlineto{\pgfqpoint{4.114894in}{2.694217in}}%
\pgfpathlineto{\pgfqpoint{4.119435in}{2.694217in}}%
\pgfpathlineto{\pgfqpoint{4.119435in}{2.691268in}}%
\pgfpathmoveto{\pgfqpoint{4.123976in}{2.685370in}}%
\pgfpathlineto{\pgfqpoint{4.123976in}{2.685370in}}%
\pgfpathlineto{\pgfqpoint{4.123976in}{2.688319in}}%
\pgfpathlineto{\pgfqpoint{4.128517in}{2.688319in}}%
\pgfpathlineto{\pgfqpoint{4.128517in}{2.685370in}}%
\pgfpathmoveto{\pgfqpoint{4.119435in}{2.688319in}}%
\pgfpathlineto{\pgfqpoint{4.119435in}{2.688319in}}%
\pgfpathlineto{\pgfqpoint{4.119435in}{2.691268in}}%
\pgfpathlineto{\pgfqpoint{4.123976in}{2.691268in}}%
\pgfpathlineto{\pgfqpoint{4.123976in}{2.688319in}}%
\pgfpathmoveto{\pgfqpoint{4.119435in}{2.691268in}}%
\pgfpathlineto{\pgfqpoint{4.119435in}{2.691268in}}%
\pgfpathlineto{\pgfqpoint{4.119435in}{2.694217in}}%
\pgfpathlineto{\pgfqpoint{4.123976in}{2.694217in}}%
\pgfpathlineto{\pgfqpoint{4.123976in}{2.691268in}}%
\pgfpathmoveto{\pgfqpoint{4.123976in}{2.688319in}}%
\pgfpathlineto{\pgfqpoint{4.123976in}{2.688319in}}%
\pgfpathlineto{\pgfqpoint{4.123976in}{2.691268in}}%
\pgfpathlineto{\pgfqpoint{4.128517in}{2.691268in}}%
\pgfpathlineto{\pgfqpoint{4.128517in}{2.688319in}}%
\pgfpathmoveto{\pgfqpoint{4.123976in}{2.691268in}}%
\pgfpathlineto{\pgfqpoint{4.123976in}{2.691268in}}%
\pgfpathlineto{\pgfqpoint{4.123976in}{2.694217in}}%
\pgfpathlineto{\pgfqpoint{4.128517in}{2.694217in}}%
\pgfpathlineto{\pgfqpoint{4.128517in}{2.691268in}}%
\pgfpathmoveto{\pgfqpoint{4.092189in}{2.694217in}}%
\pgfpathlineto{\pgfqpoint{4.092189in}{2.694217in}}%
\pgfpathlineto{\pgfqpoint{4.092189in}{2.697167in}}%
\pgfpathlineto{\pgfqpoint{4.096730in}{2.697167in}}%
\pgfpathlineto{\pgfqpoint{4.096730in}{2.694217in}}%
\pgfpathmoveto{\pgfqpoint{4.092189in}{2.697167in}}%
\pgfpathlineto{\pgfqpoint{4.092189in}{2.697167in}}%
\pgfpathlineto{\pgfqpoint{4.092189in}{2.700116in}}%
\pgfpathlineto{\pgfqpoint{4.096730in}{2.700116in}}%
\pgfpathlineto{\pgfqpoint{4.096730in}{2.697167in}}%
\pgfpathmoveto{\pgfqpoint{4.096730in}{2.694217in}}%
\pgfpathlineto{\pgfqpoint{4.096730in}{2.694217in}}%
\pgfpathlineto{\pgfqpoint{4.096730in}{2.697167in}}%
\pgfpathlineto{\pgfqpoint{4.101271in}{2.697167in}}%
\pgfpathlineto{\pgfqpoint{4.101271in}{2.694217in}}%
\pgfpathmoveto{\pgfqpoint{4.096730in}{2.697167in}}%
\pgfpathlineto{\pgfqpoint{4.096730in}{2.697167in}}%
\pgfpathlineto{\pgfqpoint{4.096730in}{2.700116in}}%
\pgfpathlineto{\pgfqpoint{4.101271in}{2.700116in}}%
\pgfpathlineto{\pgfqpoint{4.101271in}{2.697167in}}%
\pgfpathmoveto{\pgfqpoint{4.128517in}{2.685370in}}%
\pgfpathlineto{\pgfqpoint{4.128517in}{2.685370in}}%
\pgfpathlineto{\pgfqpoint{4.128517in}{2.688319in}}%
\pgfpathlineto{\pgfqpoint{4.133058in}{2.688319in}}%
\pgfpathlineto{\pgfqpoint{4.133058in}{2.685370in}}%
\pgfpathmoveto{\pgfqpoint{4.133058in}{2.685370in}}%
\pgfpathlineto{\pgfqpoint{4.133058in}{2.685370in}}%
\pgfpathlineto{\pgfqpoint{4.133058in}{2.688319in}}%
\pgfpathlineto{\pgfqpoint{4.137599in}{2.688319in}}%
\pgfpathlineto{\pgfqpoint{4.137599in}{2.685370in}}%
\pgfpathmoveto{\pgfqpoint{4.137599in}{2.682421in}}%
\pgfpathlineto{\pgfqpoint{4.137599in}{2.682421in}}%
\pgfpathlineto{\pgfqpoint{4.137599in}{2.685370in}}%
\pgfpathlineto{\pgfqpoint{4.142141in}{2.685370in}}%
\pgfpathlineto{\pgfqpoint{4.142141in}{2.682421in}}%
\pgfpathmoveto{\pgfqpoint{4.137599in}{2.685370in}}%
\pgfpathlineto{\pgfqpoint{4.137599in}{2.685370in}}%
\pgfpathlineto{\pgfqpoint{4.137599in}{2.688319in}}%
\pgfpathlineto{\pgfqpoint{4.142141in}{2.688319in}}%
\pgfpathlineto{\pgfqpoint{4.142141in}{2.685370in}}%
\pgfpathmoveto{\pgfqpoint{4.142141in}{2.682421in}}%
\pgfpathlineto{\pgfqpoint{4.142141in}{2.682421in}}%
\pgfpathlineto{\pgfqpoint{4.142141in}{2.685370in}}%
\pgfpathlineto{\pgfqpoint{4.146682in}{2.685370in}}%
\pgfpathlineto{\pgfqpoint{4.146682in}{2.682421in}}%
\pgfpathmoveto{\pgfqpoint{4.142141in}{2.685370in}}%
\pgfpathlineto{\pgfqpoint{4.142141in}{2.685370in}}%
\pgfpathlineto{\pgfqpoint{4.142141in}{2.688319in}}%
\pgfpathlineto{\pgfqpoint{4.146682in}{2.688319in}}%
\pgfpathlineto{\pgfqpoint{4.146682in}{2.685370in}}%
\pgfpathmoveto{\pgfqpoint{4.151223in}{2.679471in}}%
\pgfpathlineto{\pgfqpoint{4.151223in}{2.679471in}}%
\pgfpathlineto{\pgfqpoint{4.151223in}{2.682421in}}%
\pgfpathlineto{\pgfqpoint{4.155764in}{2.682421in}}%
\pgfpathlineto{\pgfqpoint{4.155764in}{2.679471in}}%
\pgfpathmoveto{\pgfqpoint{4.155764in}{2.679471in}}%
\pgfpathlineto{\pgfqpoint{4.155764in}{2.679471in}}%
\pgfpathlineto{\pgfqpoint{4.155764in}{2.682421in}}%
\pgfpathlineto{\pgfqpoint{4.160305in}{2.682421in}}%
\pgfpathlineto{\pgfqpoint{4.160305in}{2.679471in}}%
\pgfpathmoveto{\pgfqpoint{4.160305in}{2.679471in}}%
\pgfpathlineto{\pgfqpoint{4.160305in}{2.679471in}}%
\pgfpathlineto{\pgfqpoint{4.160305in}{2.682421in}}%
\pgfpathlineto{\pgfqpoint{4.164846in}{2.682421in}}%
\pgfpathlineto{\pgfqpoint{4.164846in}{2.679471in}}%
\pgfpathmoveto{\pgfqpoint{4.146682in}{2.682421in}}%
\pgfpathlineto{\pgfqpoint{4.146682in}{2.682421in}}%
\pgfpathlineto{\pgfqpoint{4.146682in}{2.685370in}}%
\pgfpathlineto{\pgfqpoint{4.151223in}{2.685370in}}%
\pgfpathlineto{\pgfqpoint{4.151223in}{2.682421in}}%
\pgfpathmoveto{\pgfqpoint{4.146682in}{2.685370in}}%
\pgfpathlineto{\pgfqpoint{4.146682in}{2.685370in}}%
\pgfpathlineto{\pgfqpoint{4.146682in}{2.688319in}}%
\pgfpathlineto{\pgfqpoint{4.151223in}{2.688319in}}%
\pgfpathlineto{\pgfqpoint{4.151223in}{2.685370in}}%
\pgfpathmoveto{\pgfqpoint{4.151223in}{2.682421in}}%
\pgfpathlineto{\pgfqpoint{4.151223in}{2.682421in}}%
\pgfpathlineto{\pgfqpoint{4.151223in}{2.685370in}}%
\pgfpathlineto{\pgfqpoint{4.155764in}{2.685370in}}%
\pgfpathlineto{\pgfqpoint{4.155764in}{2.682421in}}%
\pgfpathmoveto{\pgfqpoint{4.151223in}{2.685370in}}%
\pgfpathlineto{\pgfqpoint{4.151223in}{2.685370in}}%
\pgfpathlineto{\pgfqpoint{4.151223in}{2.688319in}}%
\pgfpathlineto{\pgfqpoint{4.155764in}{2.688319in}}%
\pgfpathlineto{\pgfqpoint{4.155764in}{2.685370in}}%
\pgfpathmoveto{\pgfqpoint{4.164846in}{2.676522in}}%
\pgfpathlineto{\pgfqpoint{4.164846in}{2.676522in}}%
\pgfpathlineto{\pgfqpoint{4.164846in}{2.679471in}}%
\pgfpathlineto{\pgfqpoint{4.169387in}{2.679471in}}%
\pgfpathlineto{\pgfqpoint{4.169387in}{2.676522in}}%
\pgfpathmoveto{\pgfqpoint{4.164846in}{2.679471in}}%
\pgfpathlineto{\pgfqpoint{4.164846in}{2.679471in}}%
\pgfpathlineto{\pgfqpoint{4.164846in}{2.682421in}}%
\pgfpathlineto{\pgfqpoint{4.169387in}{2.682421in}}%
\pgfpathlineto{\pgfqpoint{4.169387in}{2.679471in}}%
\pgfpathmoveto{\pgfqpoint{4.169387in}{2.676522in}}%
\pgfpathlineto{\pgfqpoint{4.169387in}{2.676522in}}%
\pgfpathlineto{\pgfqpoint{4.169387in}{2.679471in}}%
\pgfpathlineto{\pgfqpoint{4.173928in}{2.679471in}}%
\pgfpathlineto{\pgfqpoint{4.173928in}{2.676522in}}%
\pgfpathmoveto{\pgfqpoint{4.169387in}{2.679471in}}%
\pgfpathlineto{\pgfqpoint{4.169387in}{2.679471in}}%
\pgfpathlineto{\pgfqpoint{4.169387in}{2.682421in}}%
\pgfpathlineto{\pgfqpoint{4.173928in}{2.682421in}}%
\pgfpathlineto{\pgfqpoint{4.173928in}{2.679471in}}%
\pgfpathmoveto{\pgfqpoint{4.178469in}{2.673573in}}%
\pgfpathlineto{\pgfqpoint{4.178469in}{2.673573in}}%
\pgfpathlineto{\pgfqpoint{4.178469in}{2.676522in}}%
\pgfpathlineto{\pgfqpoint{4.183010in}{2.676522in}}%
\pgfpathlineto{\pgfqpoint{4.183010in}{2.673573in}}%
\pgfpathmoveto{\pgfqpoint{4.173928in}{2.676522in}}%
\pgfpathlineto{\pgfqpoint{4.173928in}{2.676522in}}%
\pgfpathlineto{\pgfqpoint{4.173928in}{2.679471in}}%
\pgfpathlineto{\pgfqpoint{4.178469in}{2.679471in}}%
\pgfpathlineto{\pgfqpoint{4.178469in}{2.676522in}}%
\pgfpathmoveto{\pgfqpoint{4.173928in}{2.679471in}}%
\pgfpathlineto{\pgfqpoint{4.173928in}{2.679471in}}%
\pgfpathlineto{\pgfqpoint{4.173928in}{2.682421in}}%
\pgfpathlineto{\pgfqpoint{4.178469in}{2.682421in}}%
\pgfpathlineto{\pgfqpoint{4.178469in}{2.679471in}}%
\pgfpathmoveto{\pgfqpoint{4.178469in}{2.676522in}}%
\pgfpathlineto{\pgfqpoint{4.178469in}{2.676522in}}%
\pgfpathlineto{\pgfqpoint{4.178469in}{2.679471in}}%
\pgfpathlineto{\pgfqpoint{4.183010in}{2.679471in}}%
\pgfpathlineto{\pgfqpoint{4.183010in}{2.676522in}}%
\pgfpathmoveto{\pgfqpoint{4.178469in}{2.679471in}}%
\pgfpathlineto{\pgfqpoint{4.178469in}{2.679471in}}%
\pgfpathlineto{\pgfqpoint{4.178469in}{2.682421in}}%
\pgfpathlineto{\pgfqpoint{4.183010in}{2.682421in}}%
\pgfpathlineto{\pgfqpoint{4.183010in}{2.679471in}}%
\pgfpathmoveto{\pgfqpoint{4.183010in}{2.673573in}}%
\pgfpathlineto{\pgfqpoint{4.183010in}{2.673573in}}%
\pgfpathlineto{\pgfqpoint{4.183010in}{2.676522in}}%
\pgfpathlineto{\pgfqpoint{4.187551in}{2.676522in}}%
\pgfpathlineto{\pgfqpoint{4.187551in}{2.673573in}}%
\pgfpathmoveto{\pgfqpoint{4.187551in}{2.673573in}}%
\pgfpathlineto{\pgfqpoint{4.187551in}{2.673573in}}%
\pgfpathlineto{\pgfqpoint{4.187551in}{2.676522in}}%
\pgfpathlineto{\pgfqpoint{4.192092in}{2.676522in}}%
\pgfpathlineto{\pgfqpoint{4.192092in}{2.673573in}}%
\pgfpathmoveto{\pgfqpoint{4.192092in}{2.670624in}}%
\pgfpathlineto{\pgfqpoint{4.192092in}{2.670624in}}%
\pgfpathlineto{\pgfqpoint{4.192092in}{2.673573in}}%
\pgfpathlineto{\pgfqpoint{4.196633in}{2.673573in}}%
\pgfpathlineto{\pgfqpoint{4.196633in}{2.670624in}}%
\pgfpathmoveto{\pgfqpoint{4.192092in}{2.673573in}}%
\pgfpathlineto{\pgfqpoint{4.192092in}{2.673573in}}%
\pgfpathlineto{\pgfqpoint{4.192092in}{2.676522in}}%
\pgfpathlineto{\pgfqpoint{4.196633in}{2.676522in}}%
\pgfpathlineto{\pgfqpoint{4.196633in}{2.673573in}}%
\pgfpathmoveto{\pgfqpoint{4.196633in}{2.670624in}}%
\pgfpathlineto{\pgfqpoint{4.196633in}{2.670624in}}%
\pgfpathlineto{\pgfqpoint{4.196633in}{2.673573in}}%
\pgfpathlineto{\pgfqpoint{4.201175in}{2.673573in}}%
\pgfpathlineto{\pgfqpoint{4.201175in}{2.670624in}}%
\pgfpathmoveto{\pgfqpoint{4.196633in}{2.673573in}}%
\pgfpathlineto{\pgfqpoint{4.196633in}{2.673573in}}%
\pgfpathlineto{\pgfqpoint{4.196633in}{2.676522in}}%
\pgfpathlineto{\pgfqpoint{4.201175in}{2.676522in}}%
\pgfpathlineto{\pgfqpoint{4.201175in}{2.673573in}}%
\pgfpathmoveto{\pgfqpoint{4.201175in}{2.670624in}}%
\pgfpathlineto{\pgfqpoint{4.201175in}{2.670624in}}%
\pgfpathlineto{\pgfqpoint{4.201175in}{2.673573in}}%
\pgfpathlineto{\pgfqpoint{4.205716in}{2.673573in}}%
\pgfpathlineto{\pgfqpoint{4.205716in}{2.670624in}}%
\pgfpathmoveto{\pgfqpoint{4.201175in}{2.673573in}}%
\pgfpathlineto{\pgfqpoint{4.201175in}{2.673573in}}%
\pgfpathlineto{\pgfqpoint{4.201175in}{2.676522in}}%
\pgfpathlineto{\pgfqpoint{4.205716in}{2.676522in}}%
\pgfpathlineto{\pgfqpoint{4.205716in}{2.673573in}}%
\pgfpathmoveto{\pgfqpoint{4.205716in}{2.670624in}}%
\pgfpathlineto{\pgfqpoint{4.205716in}{2.670624in}}%
\pgfpathlineto{\pgfqpoint{4.205716in}{2.673573in}}%
\pgfpathlineto{\pgfqpoint{4.210257in}{2.673573in}}%
\pgfpathlineto{\pgfqpoint{4.210257in}{2.670624in}}%
\pgfpathmoveto{\pgfqpoint{4.205716in}{2.673573in}}%
\pgfpathlineto{\pgfqpoint{4.205716in}{2.673573in}}%
\pgfpathlineto{\pgfqpoint{4.205716in}{2.676522in}}%
\pgfpathlineto{\pgfqpoint{4.210257in}{2.676522in}}%
\pgfpathlineto{\pgfqpoint{4.210257in}{2.673573in}}%
\pgfpathmoveto{\pgfqpoint{4.228421in}{2.759100in}}%
\pgfpathlineto{\pgfqpoint{4.228421in}{2.759100in}}%
\pgfpathlineto{\pgfqpoint{4.228421in}{2.762049in}}%
\pgfpathlineto{\pgfqpoint{4.232962in}{2.762049in}}%
\pgfpathlineto{\pgfqpoint{4.232962in}{2.759100in}}%
\pgfpathmoveto{\pgfqpoint{4.228421in}{2.762049in}}%
\pgfpathlineto{\pgfqpoint{4.228421in}{2.762049in}}%
\pgfpathlineto{\pgfqpoint{4.228421in}{2.764998in}}%
\pgfpathlineto{\pgfqpoint{4.232962in}{2.764998in}}%
\pgfpathlineto{\pgfqpoint{4.232962in}{2.762049in}}%
\pgfpathmoveto{\pgfqpoint{4.232962in}{2.759100in}}%
\pgfpathlineto{\pgfqpoint{4.232962in}{2.759100in}}%
\pgfpathlineto{\pgfqpoint{4.232962in}{2.762049in}}%
\pgfpathlineto{\pgfqpoint{4.237503in}{2.762049in}}%
\pgfpathlineto{\pgfqpoint{4.237503in}{2.759100in}}%
\pgfpathmoveto{\pgfqpoint{4.232962in}{2.762049in}}%
\pgfpathlineto{\pgfqpoint{4.232962in}{2.762049in}}%
\pgfpathlineto{\pgfqpoint{4.232962in}{2.764998in}}%
\pgfpathlineto{\pgfqpoint{4.237503in}{2.764998in}}%
\pgfpathlineto{\pgfqpoint{4.237503in}{2.762049in}}%
\pgfpathmoveto{\pgfqpoint{4.155764in}{2.806288in}}%
\pgfpathlineto{\pgfqpoint{4.155764in}{2.806288in}}%
\pgfpathlineto{\pgfqpoint{4.155764in}{2.809238in}}%
\pgfpathlineto{\pgfqpoint{4.160305in}{2.809238in}}%
\pgfpathlineto{\pgfqpoint{4.160305in}{2.806288in}}%
\pgfpathmoveto{\pgfqpoint{4.155764in}{2.809238in}}%
\pgfpathlineto{\pgfqpoint{4.155764in}{2.809238in}}%
\pgfpathlineto{\pgfqpoint{4.155764in}{2.812187in}}%
\pgfpathlineto{\pgfqpoint{4.160305in}{2.812187in}}%
\pgfpathlineto{\pgfqpoint{4.160305in}{2.809238in}}%
\pgfpathmoveto{\pgfqpoint{4.160305in}{2.806288in}}%
\pgfpathlineto{\pgfqpoint{4.160305in}{2.806288in}}%
\pgfpathlineto{\pgfqpoint{4.160305in}{2.809238in}}%
\pgfpathlineto{\pgfqpoint{4.164846in}{2.809238in}}%
\pgfpathlineto{\pgfqpoint{4.164846in}{2.806288in}}%
\pgfpathmoveto{\pgfqpoint{4.160305in}{2.809238in}}%
\pgfpathlineto{\pgfqpoint{4.160305in}{2.809238in}}%
\pgfpathlineto{\pgfqpoint{4.160305in}{2.812187in}}%
\pgfpathlineto{\pgfqpoint{4.164846in}{2.812187in}}%
\pgfpathlineto{\pgfqpoint{4.164846in}{2.809238in}}%
\pgfpathmoveto{\pgfqpoint{4.119435in}{2.829883in}}%
\pgfpathlineto{\pgfqpoint{4.119435in}{2.829883in}}%
\pgfpathlineto{\pgfqpoint{4.119435in}{2.832832in}}%
\pgfpathlineto{\pgfqpoint{4.123976in}{2.832832in}}%
\pgfpathlineto{\pgfqpoint{4.123976in}{2.829883in}}%
\pgfpathmoveto{\pgfqpoint{4.119435in}{2.832832in}}%
\pgfpathlineto{\pgfqpoint{4.119435in}{2.832832in}}%
\pgfpathlineto{\pgfqpoint{4.119435in}{2.835781in}}%
\pgfpathlineto{\pgfqpoint{4.123976in}{2.835781in}}%
\pgfpathlineto{\pgfqpoint{4.123976in}{2.832832in}}%
\pgfpathmoveto{\pgfqpoint{4.123976in}{2.829883in}}%
\pgfpathlineto{\pgfqpoint{4.123976in}{2.829883in}}%
\pgfpathlineto{\pgfqpoint{4.123976in}{2.832832in}}%
\pgfpathlineto{\pgfqpoint{4.128517in}{2.832832in}}%
\pgfpathlineto{\pgfqpoint{4.128517in}{2.829883in}}%
\pgfpathmoveto{\pgfqpoint{4.123976in}{2.832832in}}%
\pgfpathlineto{\pgfqpoint{4.123976in}{2.832832in}}%
\pgfpathlineto{\pgfqpoint{4.123976in}{2.835781in}}%
\pgfpathlineto{\pgfqpoint{4.128517in}{2.835781in}}%
\pgfpathlineto{\pgfqpoint{4.128517in}{2.832832in}}%
\pgfpathmoveto{\pgfqpoint{4.101271in}{2.841680in}}%
\pgfpathlineto{\pgfqpoint{4.101271in}{2.841680in}}%
\pgfpathlineto{\pgfqpoint{4.101271in}{2.844629in}}%
\pgfpathlineto{\pgfqpoint{4.105812in}{2.844629in}}%
\pgfpathlineto{\pgfqpoint{4.105812in}{2.841680in}}%
\pgfpathmoveto{\pgfqpoint{4.101271in}{2.844629in}}%
\pgfpathlineto{\pgfqpoint{4.101271in}{2.844629in}}%
\pgfpathlineto{\pgfqpoint{4.101271in}{2.847578in}}%
\pgfpathlineto{\pgfqpoint{4.105812in}{2.847578in}}%
\pgfpathlineto{\pgfqpoint{4.105812in}{2.844629in}}%
\pgfpathmoveto{\pgfqpoint{4.105812in}{2.841680in}}%
\pgfpathlineto{\pgfqpoint{4.105812in}{2.841680in}}%
\pgfpathlineto{\pgfqpoint{4.105812in}{2.844629in}}%
\pgfpathlineto{\pgfqpoint{4.110353in}{2.844629in}}%
\pgfpathlineto{\pgfqpoint{4.110353in}{2.841680in}}%
\pgfpathmoveto{\pgfqpoint{4.105812in}{2.844629in}}%
\pgfpathlineto{\pgfqpoint{4.105812in}{2.844629in}}%
\pgfpathlineto{\pgfqpoint{4.105812in}{2.847578in}}%
\pgfpathlineto{\pgfqpoint{4.110353in}{2.847578in}}%
\pgfpathlineto{\pgfqpoint{4.110353in}{2.844629in}}%
\pgfpathmoveto{\pgfqpoint{4.092189in}{2.847578in}}%
\pgfpathlineto{\pgfqpoint{4.092189in}{2.847578in}}%
\pgfpathlineto{\pgfqpoint{4.092189in}{2.850528in}}%
\pgfpathlineto{\pgfqpoint{4.096730in}{2.850528in}}%
\pgfpathlineto{\pgfqpoint{4.096730in}{2.847578in}}%
\pgfpathmoveto{\pgfqpoint{4.092189in}{2.850528in}}%
\pgfpathlineto{\pgfqpoint{4.092189in}{2.850528in}}%
\pgfpathlineto{\pgfqpoint{4.092189in}{2.853477in}}%
\pgfpathlineto{\pgfqpoint{4.096730in}{2.853477in}}%
\pgfpathlineto{\pgfqpoint{4.096730in}{2.850528in}}%
\pgfpathmoveto{\pgfqpoint{4.096730in}{2.847578in}}%
\pgfpathlineto{\pgfqpoint{4.096730in}{2.847578in}}%
\pgfpathlineto{\pgfqpoint{4.096730in}{2.850528in}}%
\pgfpathlineto{\pgfqpoint{4.101271in}{2.850528in}}%
\pgfpathlineto{\pgfqpoint{4.101271in}{2.847578in}}%
\pgfpathmoveto{\pgfqpoint{4.096730in}{2.850528in}}%
\pgfpathlineto{\pgfqpoint{4.096730in}{2.850528in}}%
\pgfpathlineto{\pgfqpoint{4.096730in}{2.853477in}}%
\pgfpathlineto{\pgfqpoint{4.101271in}{2.853477in}}%
\pgfpathlineto{\pgfqpoint{4.101271in}{2.850528in}}%
\pgfpathmoveto{\pgfqpoint{4.092189in}{2.853477in}}%
\pgfpathlineto{\pgfqpoint{4.092189in}{2.853477in}}%
\pgfpathlineto{\pgfqpoint{4.092189in}{2.856426in}}%
\pgfpathlineto{\pgfqpoint{4.096730in}{2.856426in}}%
\pgfpathlineto{\pgfqpoint{4.096730in}{2.853477in}}%
\pgfpathmoveto{\pgfqpoint{4.092189in}{2.856426in}}%
\pgfpathlineto{\pgfqpoint{4.092189in}{2.856426in}}%
\pgfpathlineto{\pgfqpoint{4.092189in}{2.859375in}}%
\pgfpathlineto{\pgfqpoint{4.096730in}{2.859375in}}%
\pgfpathlineto{\pgfqpoint{4.096730in}{2.856426in}}%
\pgfpathmoveto{\pgfqpoint{4.096730in}{2.853477in}}%
\pgfpathlineto{\pgfqpoint{4.096730in}{2.853477in}}%
\pgfpathlineto{\pgfqpoint{4.096730in}{2.856426in}}%
\pgfpathlineto{\pgfqpoint{4.101271in}{2.856426in}}%
\pgfpathlineto{\pgfqpoint{4.101271in}{2.853477in}}%
\pgfpathmoveto{\pgfqpoint{4.101271in}{2.847578in}}%
\pgfpathlineto{\pgfqpoint{4.101271in}{2.847578in}}%
\pgfpathlineto{\pgfqpoint{4.101271in}{2.850528in}}%
\pgfpathlineto{\pgfqpoint{4.105812in}{2.850528in}}%
\pgfpathlineto{\pgfqpoint{4.105812in}{2.847578in}}%
\pgfpathmoveto{\pgfqpoint{4.101271in}{2.850528in}}%
\pgfpathlineto{\pgfqpoint{4.101271in}{2.850528in}}%
\pgfpathlineto{\pgfqpoint{4.101271in}{2.853477in}}%
\pgfpathlineto{\pgfqpoint{4.105812in}{2.853477in}}%
\pgfpathlineto{\pgfqpoint{4.105812in}{2.850528in}}%
\pgfpathmoveto{\pgfqpoint{4.105812in}{2.847578in}}%
\pgfpathlineto{\pgfqpoint{4.105812in}{2.847578in}}%
\pgfpathlineto{\pgfqpoint{4.105812in}{2.850528in}}%
\pgfpathlineto{\pgfqpoint{4.110353in}{2.850528in}}%
\pgfpathlineto{\pgfqpoint{4.110353in}{2.847578in}}%
\pgfpathmoveto{\pgfqpoint{4.110353in}{2.835781in}}%
\pgfpathlineto{\pgfqpoint{4.110353in}{2.835781in}}%
\pgfpathlineto{\pgfqpoint{4.110353in}{2.838730in}}%
\pgfpathlineto{\pgfqpoint{4.114894in}{2.838730in}}%
\pgfpathlineto{\pgfqpoint{4.114894in}{2.835781in}}%
\pgfpathmoveto{\pgfqpoint{4.110353in}{2.838730in}}%
\pgfpathlineto{\pgfqpoint{4.110353in}{2.838730in}}%
\pgfpathlineto{\pgfqpoint{4.110353in}{2.841680in}}%
\pgfpathlineto{\pgfqpoint{4.114894in}{2.841680in}}%
\pgfpathlineto{\pgfqpoint{4.114894in}{2.838730in}}%
\pgfpathmoveto{\pgfqpoint{4.114894in}{2.835781in}}%
\pgfpathlineto{\pgfqpoint{4.114894in}{2.835781in}}%
\pgfpathlineto{\pgfqpoint{4.114894in}{2.838730in}}%
\pgfpathlineto{\pgfqpoint{4.119435in}{2.838730in}}%
\pgfpathlineto{\pgfqpoint{4.119435in}{2.835781in}}%
\pgfpathmoveto{\pgfqpoint{4.114894in}{2.838730in}}%
\pgfpathlineto{\pgfqpoint{4.114894in}{2.838730in}}%
\pgfpathlineto{\pgfqpoint{4.114894in}{2.841680in}}%
\pgfpathlineto{\pgfqpoint{4.119435in}{2.841680in}}%
\pgfpathlineto{\pgfqpoint{4.119435in}{2.838730in}}%
\pgfpathmoveto{\pgfqpoint{4.110353in}{2.841680in}}%
\pgfpathlineto{\pgfqpoint{4.110353in}{2.841680in}}%
\pgfpathlineto{\pgfqpoint{4.110353in}{2.844629in}}%
\pgfpathlineto{\pgfqpoint{4.114894in}{2.844629in}}%
\pgfpathlineto{\pgfqpoint{4.114894in}{2.841680in}}%
\pgfpathmoveto{\pgfqpoint{4.110353in}{2.844629in}}%
\pgfpathlineto{\pgfqpoint{4.110353in}{2.844629in}}%
\pgfpathlineto{\pgfqpoint{4.110353in}{2.847578in}}%
\pgfpathlineto{\pgfqpoint{4.114894in}{2.847578in}}%
\pgfpathlineto{\pgfqpoint{4.114894in}{2.844629in}}%
\pgfpathmoveto{\pgfqpoint{4.114894in}{2.841680in}}%
\pgfpathlineto{\pgfqpoint{4.114894in}{2.841680in}}%
\pgfpathlineto{\pgfqpoint{4.114894in}{2.844629in}}%
\pgfpathlineto{\pgfqpoint{4.119435in}{2.844629in}}%
\pgfpathlineto{\pgfqpoint{4.119435in}{2.841680in}}%
\pgfpathmoveto{\pgfqpoint{4.119435in}{2.835781in}}%
\pgfpathlineto{\pgfqpoint{4.119435in}{2.835781in}}%
\pgfpathlineto{\pgfqpoint{4.119435in}{2.838730in}}%
\pgfpathlineto{\pgfqpoint{4.123976in}{2.838730in}}%
\pgfpathlineto{\pgfqpoint{4.123976in}{2.835781in}}%
\pgfpathmoveto{\pgfqpoint{4.119435in}{2.838730in}}%
\pgfpathlineto{\pgfqpoint{4.119435in}{2.838730in}}%
\pgfpathlineto{\pgfqpoint{4.119435in}{2.841680in}}%
\pgfpathlineto{\pgfqpoint{4.123976in}{2.841680in}}%
\pgfpathlineto{\pgfqpoint{4.123976in}{2.838730in}}%
\pgfpathmoveto{\pgfqpoint{4.123976in}{2.835781in}}%
\pgfpathlineto{\pgfqpoint{4.123976in}{2.835781in}}%
\pgfpathlineto{\pgfqpoint{4.123976in}{2.838730in}}%
\pgfpathlineto{\pgfqpoint{4.128517in}{2.838730in}}%
\pgfpathlineto{\pgfqpoint{4.128517in}{2.835781in}}%
\pgfpathmoveto{\pgfqpoint{4.137599in}{2.818085in}}%
\pgfpathlineto{\pgfqpoint{4.137599in}{2.818085in}}%
\pgfpathlineto{\pgfqpoint{4.137599in}{2.821035in}}%
\pgfpathlineto{\pgfqpoint{4.142141in}{2.821035in}}%
\pgfpathlineto{\pgfqpoint{4.142141in}{2.818085in}}%
\pgfpathmoveto{\pgfqpoint{4.137599in}{2.821035in}}%
\pgfpathlineto{\pgfqpoint{4.137599in}{2.821035in}}%
\pgfpathlineto{\pgfqpoint{4.137599in}{2.823984in}}%
\pgfpathlineto{\pgfqpoint{4.142141in}{2.823984in}}%
\pgfpathlineto{\pgfqpoint{4.142141in}{2.821035in}}%
\pgfpathmoveto{\pgfqpoint{4.142141in}{2.818085in}}%
\pgfpathlineto{\pgfqpoint{4.142141in}{2.818085in}}%
\pgfpathlineto{\pgfqpoint{4.142141in}{2.821035in}}%
\pgfpathlineto{\pgfqpoint{4.146682in}{2.821035in}}%
\pgfpathlineto{\pgfqpoint{4.146682in}{2.818085in}}%
\pgfpathmoveto{\pgfqpoint{4.142141in}{2.821035in}}%
\pgfpathlineto{\pgfqpoint{4.142141in}{2.821035in}}%
\pgfpathlineto{\pgfqpoint{4.142141in}{2.823984in}}%
\pgfpathlineto{\pgfqpoint{4.146682in}{2.823984in}}%
\pgfpathlineto{\pgfqpoint{4.146682in}{2.821035in}}%
\pgfpathmoveto{\pgfqpoint{4.128517in}{2.823984in}}%
\pgfpathlineto{\pgfqpoint{4.128517in}{2.823984in}}%
\pgfpathlineto{\pgfqpoint{4.128517in}{2.826933in}}%
\pgfpathlineto{\pgfqpoint{4.133058in}{2.826933in}}%
\pgfpathlineto{\pgfqpoint{4.133058in}{2.823984in}}%
\pgfpathmoveto{\pgfqpoint{4.128517in}{2.826933in}}%
\pgfpathlineto{\pgfqpoint{4.128517in}{2.826933in}}%
\pgfpathlineto{\pgfqpoint{4.128517in}{2.829883in}}%
\pgfpathlineto{\pgfqpoint{4.133058in}{2.829883in}}%
\pgfpathlineto{\pgfqpoint{4.133058in}{2.826933in}}%
\pgfpathmoveto{\pgfqpoint{4.133058in}{2.823984in}}%
\pgfpathlineto{\pgfqpoint{4.133058in}{2.823984in}}%
\pgfpathlineto{\pgfqpoint{4.133058in}{2.826933in}}%
\pgfpathlineto{\pgfqpoint{4.137599in}{2.826933in}}%
\pgfpathlineto{\pgfqpoint{4.137599in}{2.823984in}}%
\pgfpathmoveto{\pgfqpoint{4.133058in}{2.826933in}}%
\pgfpathlineto{\pgfqpoint{4.133058in}{2.826933in}}%
\pgfpathlineto{\pgfqpoint{4.133058in}{2.829883in}}%
\pgfpathlineto{\pgfqpoint{4.137599in}{2.829883in}}%
\pgfpathlineto{\pgfqpoint{4.137599in}{2.826933in}}%
\pgfpathmoveto{\pgfqpoint{4.128517in}{2.829883in}}%
\pgfpathlineto{\pgfqpoint{4.128517in}{2.829883in}}%
\pgfpathlineto{\pgfqpoint{4.128517in}{2.832832in}}%
\pgfpathlineto{\pgfqpoint{4.133058in}{2.832832in}}%
\pgfpathlineto{\pgfqpoint{4.133058in}{2.829883in}}%
\pgfpathmoveto{\pgfqpoint{4.128517in}{2.832832in}}%
\pgfpathlineto{\pgfqpoint{4.128517in}{2.832832in}}%
\pgfpathlineto{\pgfqpoint{4.128517in}{2.835781in}}%
\pgfpathlineto{\pgfqpoint{4.133058in}{2.835781in}}%
\pgfpathlineto{\pgfqpoint{4.133058in}{2.832832in}}%
\pgfpathmoveto{\pgfqpoint{4.133058in}{2.829883in}}%
\pgfpathlineto{\pgfqpoint{4.133058in}{2.829883in}}%
\pgfpathlineto{\pgfqpoint{4.133058in}{2.832832in}}%
\pgfpathlineto{\pgfqpoint{4.137599in}{2.832832in}}%
\pgfpathlineto{\pgfqpoint{4.137599in}{2.829883in}}%
\pgfpathmoveto{\pgfqpoint{4.137599in}{2.823984in}}%
\pgfpathlineto{\pgfqpoint{4.137599in}{2.823984in}}%
\pgfpathlineto{\pgfqpoint{4.137599in}{2.826933in}}%
\pgfpathlineto{\pgfqpoint{4.142141in}{2.826933in}}%
\pgfpathlineto{\pgfqpoint{4.142141in}{2.823984in}}%
\pgfpathmoveto{\pgfqpoint{4.137599in}{2.826933in}}%
\pgfpathlineto{\pgfqpoint{4.137599in}{2.826933in}}%
\pgfpathlineto{\pgfqpoint{4.137599in}{2.829883in}}%
\pgfpathlineto{\pgfqpoint{4.142141in}{2.829883in}}%
\pgfpathlineto{\pgfqpoint{4.142141in}{2.826933in}}%
\pgfpathmoveto{\pgfqpoint{4.142141in}{2.823984in}}%
\pgfpathlineto{\pgfqpoint{4.142141in}{2.823984in}}%
\pgfpathlineto{\pgfqpoint{4.142141in}{2.826933in}}%
\pgfpathlineto{\pgfqpoint{4.146682in}{2.826933in}}%
\pgfpathlineto{\pgfqpoint{4.146682in}{2.823984in}}%
\pgfpathmoveto{\pgfqpoint{4.146682in}{2.812187in}}%
\pgfpathlineto{\pgfqpoint{4.146682in}{2.812187in}}%
\pgfpathlineto{\pgfqpoint{4.146682in}{2.815136in}}%
\pgfpathlineto{\pgfqpoint{4.151223in}{2.815136in}}%
\pgfpathlineto{\pgfqpoint{4.151223in}{2.812187in}}%
\pgfpathmoveto{\pgfqpoint{4.146682in}{2.815136in}}%
\pgfpathlineto{\pgfqpoint{4.146682in}{2.815136in}}%
\pgfpathlineto{\pgfqpoint{4.146682in}{2.818085in}}%
\pgfpathlineto{\pgfqpoint{4.151223in}{2.818085in}}%
\pgfpathlineto{\pgfqpoint{4.151223in}{2.815136in}}%
\pgfpathmoveto{\pgfqpoint{4.151223in}{2.812187in}}%
\pgfpathlineto{\pgfqpoint{4.151223in}{2.812187in}}%
\pgfpathlineto{\pgfqpoint{4.151223in}{2.815136in}}%
\pgfpathlineto{\pgfqpoint{4.155764in}{2.815136in}}%
\pgfpathlineto{\pgfqpoint{4.155764in}{2.812187in}}%
\pgfpathmoveto{\pgfqpoint{4.151223in}{2.815136in}}%
\pgfpathlineto{\pgfqpoint{4.151223in}{2.815136in}}%
\pgfpathlineto{\pgfqpoint{4.151223in}{2.818085in}}%
\pgfpathlineto{\pgfqpoint{4.155764in}{2.818085in}}%
\pgfpathlineto{\pgfqpoint{4.155764in}{2.815136in}}%
\pgfpathmoveto{\pgfqpoint{4.146682in}{2.818085in}}%
\pgfpathlineto{\pgfqpoint{4.146682in}{2.818085in}}%
\pgfpathlineto{\pgfqpoint{4.146682in}{2.821035in}}%
\pgfpathlineto{\pgfqpoint{4.151223in}{2.821035in}}%
\pgfpathlineto{\pgfqpoint{4.151223in}{2.818085in}}%
\pgfpathmoveto{\pgfqpoint{4.146682in}{2.821035in}}%
\pgfpathlineto{\pgfqpoint{4.146682in}{2.821035in}}%
\pgfpathlineto{\pgfqpoint{4.146682in}{2.823984in}}%
\pgfpathlineto{\pgfqpoint{4.151223in}{2.823984in}}%
\pgfpathlineto{\pgfqpoint{4.151223in}{2.821035in}}%
\pgfpathmoveto{\pgfqpoint{4.151223in}{2.818085in}}%
\pgfpathlineto{\pgfqpoint{4.151223in}{2.818085in}}%
\pgfpathlineto{\pgfqpoint{4.151223in}{2.821035in}}%
\pgfpathlineto{\pgfqpoint{4.155764in}{2.821035in}}%
\pgfpathlineto{\pgfqpoint{4.155764in}{2.818085in}}%
\pgfpathmoveto{\pgfqpoint{4.155764in}{2.812187in}}%
\pgfpathlineto{\pgfqpoint{4.155764in}{2.812187in}}%
\pgfpathlineto{\pgfqpoint{4.155764in}{2.815136in}}%
\pgfpathlineto{\pgfqpoint{4.160305in}{2.815136in}}%
\pgfpathlineto{\pgfqpoint{4.160305in}{2.812187in}}%
\pgfpathmoveto{\pgfqpoint{4.155764in}{2.815136in}}%
\pgfpathlineto{\pgfqpoint{4.155764in}{2.815136in}}%
\pgfpathlineto{\pgfqpoint{4.155764in}{2.818085in}}%
\pgfpathlineto{\pgfqpoint{4.160305in}{2.818085in}}%
\pgfpathlineto{\pgfqpoint{4.160305in}{2.815136in}}%
\pgfpathmoveto{\pgfqpoint{4.160305in}{2.812187in}}%
\pgfpathlineto{\pgfqpoint{4.160305in}{2.812187in}}%
\pgfpathlineto{\pgfqpoint{4.160305in}{2.815136in}}%
\pgfpathlineto{\pgfqpoint{4.164846in}{2.815136in}}%
\pgfpathlineto{\pgfqpoint{4.164846in}{2.812187in}}%
\pgfpathmoveto{\pgfqpoint{4.192092in}{2.782694in}}%
\pgfpathlineto{\pgfqpoint{4.192092in}{2.782694in}}%
\pgfpathlineto{\pgfqpoint{4.192092in}{2.785643in}}%
\pgfpathlineto{\pgfqpoint{4.196633in}{2.785643in}}%
\pgfpathlineto{\pgfqpoint{4.196633in}{2.782694in}}%
\pgfpathmoveto{\pgfqpoint{4.192092in}{2.785643in}}%
\pgfpathlineto{\pgfqpoint{4.192092in}{2.785643in}}%
\pgfpathlineto{\pgfqpoint{4.192092in}{2.788592in}}%
\pgfpathlineto{\pgfqpoint{4.196633in}{2.788592in}}%
\pgfpathlineto{\pgfqpoint{4.196633in}{2.785643in}}%
\pgfpathmoveto{\pgfqpoint{4.196633in}{2.782694in}}%
\pgfpathlineto{\pgfqpoint{4.196633in}{2.782694in}}%
\pgfpathlineto{\pgfqpoint{4.196633in}{2.785643in}}%
\pgfpathlineto{\pgfqpoint{4.201175in}{2.785643in}}%
\pgfpathlineto{\pgfqpoint{4.201175in}{2.782694in}}%
\pgfpathmoveto{\pgfqpoint{4.196633in}{2.785643in}}%
\pgfpathlineto{\pgfqpoint{4.196633in}{2.785643in}}%
\pgfpathlineto{\pgfqpoint{4.196633in}{2.788592in}}%
\pgfpathlineto{\pgfqpoint{4.201175in}{2.788592in}}%
\pgfpathlineto{\pgfqpoint{4.201175in}{2.785643in}}%
\pgfpathmoveto{\pgfqpoint{4.173928in}{2.794491in}}%
\pgfpathlineto{\pgfqpoint{4.173928in}{2.794491in}}%
\pgfpathlineto{\pgfqpoint{4.173928in}{2.797440in}}%
\pgfpathlineto{\pgfqpoint{4.178469in}{2.797440in}}%
\pgfpathlineto{\pgfqpoint{4.178469in}{2.794491in}}%
\pgfpathmoveto{\pgfqpoint{4.173928in}{2.797440in}}%
\pgfpathlineto{\pgfqpoint{4.173928in}{2.797440in}}%
\pgfpathlineto{\pgfqpoint{4.173928in}{2.800390in}}%
\pgfpathlineto{\pgfqpoint{4.178469in}{2.800390in}}%
\pgfpathlineto{\pgfqpoint{4.178469in}{2.797440in}}%
\pgfpathmoveto{\pgfqpoint{4.178469in}{2.794491in}}%
\pgfpathlineto{\pgfqpoint{4.178469in}{2.794491in}}%
\pgfpathlineto{\pgfqpoint{4.178469in}{2.797440in}}%
\pgfpathlineto{\pgfqpoint{4.183010in}{2.797440in}}%
\pgfpathlineto{\pgfqpoint{4.183010in}{2.794491in}}%
\pgfpathmoveto{\pgfqpoint{4.178469in}{2.797440in}}%
\pgfpathlineto{\pgfqpoint{4.178469in}{2.797440in}}%
\pgfpathlineto{\pgfqpoint{4.178469in}{2.800390in}}%
\pgfpathlineto{\pgfqpoint{4.183010in}{2.800390in}}%
\pgfpathlineto{\pgfqpoint{4.183010in}{2.797440in}}%
\pgfpathmoveto{\pgfqpoint{4.164846in}{2.800390in}}%
\pgfpathlineto{\pgfqpoint{4.164846in}{2.800390in}}%
\pgfpathlineto{\pgfqpoint{4.164846in}{2.803339in}}%
\pgfpathlineto{\pgfqpoint{4.169387in}{2.803339in}}%
\pgfpathlineto{\pgfqpoint{4.169387in}{2.800390in}}%
\pgfpathmoveto{\pgfqpoint{4.164846in}{2.803339in}}%
\pgfpathlineto{\pgfqpoint{4.164846in}{2.803339in}}%
\pgfpathlineto{\pgfqpoint{4.164846in}{2.806288in}}%
\pgfpathlineto{\pgfqpoint{4.169387in}{2.806288in}}%
\pgfpathlineto{\pgfqpoint{4.169387in}{2.803339in}}%
\pgfpathmoveto{\pgfqpoint{4.169387in}{2.800390in}}%
\pgfpathlineto{\pgfqpoint{4.169387in}{2.800390in}}%
\pgfpathlineto{\pgfqpoint{4.169387in}{2.803339in}}%
\pgfpathlineto{\pgfqpoint{4.173928in}{2.803339in}}%
\pgfpathlineto{\pgfqpoint{4.173928in}{2.800390in}}%
\pgfpathmoveto{\pgfqpoint{4.169387in}{2.803339in}}%
\pgfpathlineto{\pgfqpoint{4.169387in}{2.803339in}}%
\pgfpathlineto{\pgfqpoint{4.169387in}{2.806288in}}%
\pgfpathlineto{\pgfqpoint{4.173928in}{2.806288in}}%
\pgfpathlineto{\pgfqpoint{4.173928in}{2.803339in}}%
\pgfpathmoveto{\pgfqpoint{4.164846in}{2.806288in}}%
\pgfpathlineto{\pgfqpoint{4.164846in}{2.806288in}}%
\pgfpathlineto{\pgfqpoint{4.164846in}{2.809238in}}%
\pgfpathlineto{\pgfqpoint{4.169387in}{2.809238in}}%
\pgfpathlineto{\pgfqpoint{4.169387in}{2.806288in}}%
\pgfpathmoveto{\pgfqpoint{4.164846in}{2.809238in}}%
\pgfpathlineto{\pgfqpoint{4.164846in}{2.809238in}}%
\pgfpathlineto{\pgfqpoint{4.164846in}{2.812187in}}%
\pgfpathlineto{\pgfqpoint{4.169387in}{2.812187in}}%
\pgfpathlineto{\pgfqpoint{4.169387in}{2.809238in}}%
\pgfpathmoveto{\pgfqpoint{4.169387in}{2.806288in}}%
\pgfpathlineto{\pgfqpoint{4.169387in}{2.806288in}}%
\pgfpathlineto{\pgfqpoint{4.169387in}{2.809238in}}%
\pgfpathlineto{\pgfqpoint{4.173928in}{2.809238in}}%
\pgfpathlineto{\pgfqpoint{4.173928in}{2.806288in}}%
\pgfpathmoveto{\pgfqpoint{4.173928in}{2.800390in}}%
\pgfpathlineto{\pgfqpoint{4.173928in}{2.800390in}}%
\pgfpathlineto{\pgfqpoint{4.173928in}{2.803339in}}%
\pgfpathlineto{\pgfqpoint{4.178469in}{2.803339in}}%
\pgfpathlineto{\pgfqpoint{4.178469in}{2.800390in}}%
\pgfpathmoveto{\pgfqpoint{4.173928in}{2.803339in}}%
\pgfpathlineto{\pgfqpoint{4.173928in}{2.803339in}}%
\pgfpathlineto{\pgfqpoint{4.173928in}{2.806288in}}%
\pgfpathlineto{\pgfqpoint{4.178469in}{2.806288in}}%
\pgfpathlineto{\pgfqpoint{4.178469in}{2.803339in}}%
\pgfpathmoveto{\pgfqpoint{4.178469in}{2.800390in}}%
\pgfpathlineto{\pgfqpoint{4.178469in}{2.800390in}}%
\pgfpathlineto{\pgfqpoint{4.178469in}{2.803339in}}%
\pgfpathlineto{\pgfqpoint{4.183010in}{2.803339in}}%
\pgfpathlineto{\pgfqpoint{4.183010in}{2.800390in}}%
\pgfpathmoveto{\pgfqpoint{4.183010in}{2.788592in}}%
\pgfpathlineto{\pgfqpoint{4.183010in}{2.788592in}}%
\pgfpathlineto{\pgfqpoint{4.183010in}{2.791542in}}%
\pgfpathlineto{\pgfqpoint{4.187551in}{2.791542in}}%
\pgfpathlineto{\pgfqpoint{4.187551in}{2.788592in}}%
\pgfpathmoveto{\pgfqpoint{4.183010in}{2.791542in}}%
\pgfpathlineto{\pgfqpoint{4.183010in}{2.791542in}}%
\pgfpathlineto{\pgfqpoint{4.183010in}{2.794491in}}%
\pgfpathlineto{\pgfqpoint{4.187551in}{2.794491in}}%
\pgfpathlineto{\pgfqpoint{4.187551in}{2.791542in}}%
\pgfpathmoveto{\pgfqpoint{4.187551in}{2.788592in}}%
\pgfpathlineto{\pgfqpoint{4.187551in}{2.788592in}}%
\pgfpathlineto{\pgfqpoint{4.187551in}{2.791542in}}%
\pgfpathlineto{\pgfqpoint{4.192092in}{2.791542in}}%
\pgfpathlineto{\pgfqpoint{4.192092in}{2.788592in}}%
\pgfpathmoveto{\pgfqpoint{4.187551in}{2.791542in}}%
\pgfpathlineto{\pgfqpoint{4.187551in}{2.791542in}}%
\pgfpathlineto{\pgfqpoint{4.187551in}{2.794491in}}%
\pgfpathlineto{\pgfqpoint{4.192092in}{2.794491in}}%
\pgfpathlineto{\pgfqpoint{4.192092in}{2.791542in}}%
\pgfpathmoveto{\pgfqpoint{4.183010in}{2.794491in}}%
\pgfpathlineto{\pgfqpoint{4.183010in}{2.794491in}}%
\pgfpathlineto{\pgfqpoint{4.183010in}{2.797440in}}%
\pgfpathlineto{\pgfqpoint{4.187551in}{2.797440in}}%
\pgfpathlineto{\pgfqpoint{4.187551in}{2.794491in}}%
\pgfpathmoveto{\pgfqpoint{4.183010in}{2.797440in}}%
\pgfpathlineto{\pgfqpoint{4.183010in}{2.797440in}}%
\pgfpathlineto{\pgfqpoint{4.183010in}{2.800390in}}%
\pgfpathlineto{\pgfqpoint{4.187551in}{2.800390in}}%
\pgfpathlineto{\pgfqpoint{4.187551in}{2.797440in}}%
\pgfpathmoveto{\pgfqpoint{4.187551in}{2.794491in}}%
\pgfpathlineto{\pgfqpoint{4.187551in}{2.794491in}}%
\pgfpathlineto{\pgfqpoint{4.187551in}{2.797440in}}%
\pgfpathlineto{\pgfqpoint{4.192092in}{2.797440in}}%
\pgfpathlineto{\pgfqpoint{4.192092in}{2.794491in}}%
\pgfpathmoveto{\pgfqpoint{4.192092in}{2.788592in}}%
\pgfpathlineto{\pgfqpoint{4.192092in}{2.788592in}}%
\pgfpathlineto{\pgfqpoint{4.192092in}{2.791542in}}%
\pgfpathlineto{\pgfqpoint{4.196633in}{2.791542in}}%
\pgfpathlineto{\pgfqpoint{4.196633in}{2.788592in}}%
\pgfpathmoveto{\pgfqpoint{4.192092in}{2.791542in}}%
\pgfpathlineto{\pgfqpoint{4.192092in}{2.791542in}}%
\pgfpathlineto{\pgfqpoint{4.192092in}{2.794491in}}%
\pgfpathlineto{\pgfqpoint{4.196633in}{2.794491in}}%
\pgfpathlineto{\pgfqpoint{4.196633in}{2.791542in}}%
\pgfpathmoveto{\pgfqpoint{4.196633in}{2.788592in}}%
\pgfpathlineto{\pgfqpoint{4.196633in}{2.788592in}}%
\pgfpathlineto{\pgfqpoint{4.196633in}{2.791542in}}%
\pgfpathlineto{\pgfqpoint{4.201175in}{2.791542in}}%
\pgfpathlineto{\pgfqpoint{4.201175in}{2.788592in}}%
\pgfpathmoveto{\pgfqpoint{4.210257in}{2.770897in}}%
\pgfpathlineto{\pgfqpoint{4.210257in}{2.770897in}}%
\pgfpathlineto{\pgfqpoint{4.210257in}{2.773846in}}%
\pgfpathlineto{\pgfqpoint{4.214798in}{2.773846in}}%
\pgfpathlineto{\pgfqpoint{4.214798in}{2.770897in}}%
\pgfpathmoveto{\pgfqpoint{4.210257in}{2.773846in}}%
\pgfpathlineto{\pgfqpoint{4.210257in}{2.773846in}}%
\pgfpathlineto{\pgfqpoint{4.210257in}{2.776795in}}%
\pgfpathlineto{\pgfqpoint{4.214798in}{2.776795in}}%
\pgfpathlineto{\pgfqpoint{4.214798in}{2.773846in}}%
\pgfpathmoveto{\pgfqpoint{4.214798in}{2.770897in}}%
\pgfpathlineto{\pgfqpoint{4.214798in}{2.770897in}}%
\pgfpathlineto{\pgfqpoint{4.214798in}{2.773846in}}%
\pgfpathlineto{\pgfqpoint{4.219339in}{2.773846in}}%
\pgfpathlineto{\pgfqpoint{4.219339in}{2.770897in}}%
\pgfpathmoveto{\pgfqpoint{4.214798in}{2.773846in}}%
\pgfpathlineto{\pgfqpoint{4.214798in}{2.773846in}}%
\pgfpathlineto{\pgfqpoint{4.214798in}{2.776795in}}%
\pgfpathlineto{\pgfqpoint{4.219339in}{2.776795in}}%
\pgfpathlineto{\pgfqpoint{4.219339in}{2.773846in}}%
\pgfpathmoveto{\pgfqpoint{4.201175in}{2.776795in}}%
\pgfpathlineto{\pgfqpoint{4.201175in}{2.776795in}}%
\pgfpathlineto{\pgfqpoint{4.201175in}{2.779745in}}%
\pgfpathlineto{\pgfqpoint{4.205716in}{2.779745in}}%
\pgfpathlineto{\pgfqpoint{4.205716in}{2.776795in}}%
\pgfpathmoveto{\pgfqpoint{4.201175in}{2.779745in}}%
\pgfpathlineto{\pgfqpoint{4.201175in}{2.779745in}}%
\pgfpathlineto{\pgfqpoint{4.201175in}{2.782694in}}%
\pgfpathlineto{\pgfqpoint{4.205716in}{2.782694in}}%
\pgfpathlineto{\pgfqpoint{4.205716in}{2.779745in}}%
\pgfpathmoveto{\pgfqpoint{4.205716in}{2.776795in}}%
\pgfpathlineto{\pgfqpoint{4.205716in}{2.776795in}}%
\pgfpathlineto{\pgfqpoint{4.205716in}{2.779745in}}%
\pgfpathlineto{\pgfqpoint{4.210257in}{2.779745in}}%
\pgfpathlineto{\pgfqpoint{4.210257in}{2.776795in}}%
\pgfpathmoveto{\pgfqpoint{4.205716in}{2.779745in}}%
\pgfpathlineto{\pgfqpoint{4.205716in}{2.779745in}}%
\pgfpathlineto{\pgfqpoint{4.205716in}{2.782694in}}%
\pgfpathlineto{\pgfqpoint{4.210257in}{2.782694in}}%
\pgfpathlineto{\pgfqpoint{4.210257in}{2.779745in}}%
\pgfpathmoveto{\pgfqpoint{4.201175in}{2.782694in}}%
\pgfpathlineto{\pgfqpoint{4.201175in}{2.782694in}}%
\pgfpathlineto{\pgfqpoint{4.201175in}{2.785643in}}%
\pgfpathlineto{\pgfqpoint{4.205716in}{2.785643in}}%
\pgfpathlineto{\pgfqpoint{4.205716in}{2.782694in}}%
\pgfpathmoveto{\pgfqpoint{4.201175in}{2.785643in}}%
\pgfpathlineto{\pgfqpoint{4.201175in}{2.785643in}}%
\pgfpathlineto{\pgfqpoint{4.201175in}{2.788592in}}%
\pgfpathlineto{\pgfqpoint{4.205716in}{2.788592in}}%
\pgfpathlineto{\pgfqpoint{4.205716in}{2.785643in}}%
\pgfpathmoveto{\pgfqpoint{4.205716in}{2.782694in}}%
\pgfpathlineto{\pgfqpoint{4.205716in}{2.782694in}}%
\pgfpathlineto{\pgfqpoint{4.205716in}{2.785643in}}%
\pgfpathlineto{\pgfqpoint{4.210257in}{2.785643in}}%
\pgfpathlineto{\pgfqpoint{4.210257in}{2.782694in}}%
\pgfpathmoveto{\pgfqpoint{4.210257in}{2.776795in}}%
\pgfpathlineto{\pgfqpoint{4.210257in}{2.776795in}}%
\pgfpathlineto{\pgfqpoint{4.210257in}{2.779745in}}%
\pgfpathlineto{\pgfqpoint{4.214798in}{2.779745in}}%
\pgfpathlineto{\pgfqpoint{4.214798in}{2.776795in}}%
\pgfpathmoveto{\pgfqpoint{4.210257in}{2.779745in}}%
\pgfpathlineto{\pgfqpoint{4.210257in}{2.779745in}}%
\pgfpathlineto{\pgfqpoint{4.210257in}{2.782694in}}%
\pgfpathlineto{\pgfqpoint{4.214798in}{2.782694in}}%
\pgfpathlineto{\pgfqpoint{4.214798in}{2.779745in}}%
\pgfpathmoveto{\pgfqpoint{4.214798in}{2.776795in}}%
\pgfpathlineto{\pgfqpoint{4.214798in}{2.776795in}}%
\pgfpathlineto{\pgfqpoint{4.214798in}{2.779745in}}%
\pgfpathlineto{\pgfqpoint{4.219339in}{2.779745in}}%
\pgfpathlineto{\pgfqpoint{4.219339in}{2.776795in}}%
\pgfpathmoveto{\pgfqpoint{4.219339in}{2.764998in}}%
\pgfpathlineto{\pgfqpoint{4.219339in}{2.764998in}}%
\pgfpathlineto{\pgfqpoint{4.219339in}{2.767947in}}%
\pgfpathlineto{\pgfqpoint{4.223880in}{2.767947in}}%
\pgfpathlineto{\pgfqpoint{4.223880in}{2.764998in}}%
\pgfpathmoveto{\pgfqpoint{4.219339in}{2.767947in}}%
\pgfpathlineto{\pgfqpoint{4.219339in}{2.767947in}}%
\pgfpathlineto{\pgfqpoint{4.219339in}{2.770897in}}%
\pgfpathlineto{\pgfqpoint{4.223880in}{2.770897in}}%
\pgfpathlineto{\pgfqpoint{4.223880in}{2.767947in}}%
\pgfpathmoveto{\pgfqpoint{4.223880in}{2.764998in}}%
\pgfpathlineto{\pgfqpoint{4.223880in}{2.764998in}}%
\pgfpathlineto{\pgfqpoint{4.223880in}{2.767947in}}%
\pgfpathlineto{\pgfqpoint{4.228421in}{2.767947in}}%
\pgfpathlineto{\pgfqpoint{4.228421in}{2.764998in}}%
\pgfpathmoveto{\pgfqpoint{4.223880in}{2.767947in}}%
\pgfpathlineto{\pgfqpoint{4.223880in}{2.767947in}}%
\pgfpathlineto{\pgfqpoint{4.223880in}{2.770897in}}%
\pgfpathlineto{\pgfqpoint{4.228421in}{2.770897in}}%
\pgfpathlineto{\pgfqpoint{4.228421in}{2.767947in}}%
\pgfpathmoveto{\pgfqpoint{4.219339in}{2.770897in}}%
\pgfpathlineto{\pgfqpoint{4.219339in}{2.770897in}}%
\pgfpathlineto{\pgfqpoint{4.219339in}{2.773846in}}%
\pgfpathlineto{\pgfqpoint{4.223880in}{2.773846in}}%
\pgfpathlineto{\pgfqpoint{4.223880in}{2.770897in}}%
\pgfpathmoveto{\pgfqpoint{4.219339in}{2.773846in}}%
\pgfpathlineto{\pgfqpoint{4.219339in}{2.773846in}}%
\pgfpathlineto{\pgfqpoint{4.219339in}{2.776795in}}%
\pgfpathlineto{\pgfqpoint{4.223880in}{2.776795in}}%
\pgfpathlineto{\pgfqpoint{4.223880in}{2.773846in}}%
\pgfpathmoveto{\pgfqpoint{4.223880in}{2.770897in}}%
\pgfpathlineto{\pgfqpoint{4.223880in}{2.770897in}}%
\pgfpathlineto{\pgfqpoint{4.223880in}{2.773846in}}%
\pgfpathlineto{\pgfqpoint{4.228421in}{2.773846in}}%
\pgfpathlineto{\pgfqpoint{4.228421in}{2.770897in}}%
\pgfpathmoveto{\pgfqpoint{4.228421in}{2.764998in}}%
\pgfpathlineto{\pgfqpoint{4.228421in}{2.764998in}}%
\pgfpathlineto{\pgfqpoint{4.228421in}{2.767947in}}%
\pgfpathlineto{\pgfqpoint{4.232962in}{2.767947in}}%
\pgfpathlineto{\pgfqpoint{4.232962in}{2.764998in}}%
\pgfpathmoveto{\pgfqpoint{4.228421in}{2.767947in}}%
\pgfpathlineto{\pgfqpoint{4.228421in}{2.767947in}}%
\pgfpathlineto{\pgfqpoint{4.228421in}{2.770897in}}%
\pgfpathlineto{\pgfqpoint{4.232962in}{2.770897in}}%
\pgfpathlineto{\pgfqpoint{4.232962in}{2.767947in}}%
\pgfpathmoveto{\pgfqpoint{4.232962in}{2.764998in}}%
\pgfpathlineto{\pgfqpoint{4.232962in}{2.764998in}}%
\pgfpathlineto{\pgfqpoint{4.232962in}{2.767947in}}%
\pgfpathlineto{\pgfqpoint{4.237503in}{2.767947in}}%
\pgfpathlineto{\pgfqpoint{4.237503in}{2.764998in}}%
\pgfpathmoveto{\pgfqpoint{4.237503in}{2.661776in}}%
\pgfpathlineto{\pgfqpoint{4.237503in}{2.661776in}}%
\pgfpathlineto{\pgfqpoint{4.237503in}{2.664725in}}%
\pgfpathlineto{\pgfqpoint{4.242044in}{2.664725in}}%
\pgfpathlineto{\pgfqpoint{4.242044in}{2.661776in}}%
\pgfpathmoveto{\pgfqpoint{4.242044in}{2.661776in}}%
\pgfpathlineto{\pgfqpoint{4.242044in}{2.661776in}}%
\pgfpathlineto{\pgfqpoint{4.242044in}{2.664725in}}%
\pgfpathlineto{\pgfqpoint{4.246585in}{2.664725in}}%
\pgfpathlineto{\pgfqpoint{4.246585in}{2.661776in}}%
\pgfpathmoveto{\pgfqpoint{4.246585in}{2.658827in}}%
\pgfpathlineto{\pgfqpoint{4.246585in}{2.658827in}}%
\pgfpathlineto{\pgfqpoint{4.246585in}{2.661776in}}%
\pgfpathlineto{\pgfqpoint{4.251126in}{2.661776in}}%
\pgfpathlineto{\pgfqpoint{4.251126in}{2.658827in}}%
\pgfpathmoveto{\pgfqpoint{4.246585in}{2.661776in}}%
\pgfpathlineto{\pgfqpoint{4.246585in}{2.661776in}}%
\pgfpathlineto{\pgfqpoint{4.246585in}{2.664725in}}%
\pgfpathlineto{\pgfqpoint{4.251126in}{2.664725in}}%
\pgfpathlineto{\pgfqpoint{4.251126in}{2.661776in}}%
\pgfpathmoveto{\pgfqpoint{4.251126in}{2.658827in}}%
\pgfpathlineto{\pgfqpoint{4.251126in}{2.658827in}}%
\pgfpathlineto{\pgfqpoint{4.251126in}{2.661776in}}%
\pgfpathlineto{\pgfqpoint{4.255667in}{2.661776in}}%
\pgfpathlineto{\pgfqpoint{4.255667in}{2.658827in}}%
\pgfpathmoveto{\pgfqpoint{4.251126in}{2.661776in}}%
\pgfpathlineto{\pgfqpoint{4.251126in}{2.661776in}}%
\pgfpathlineto{\pgfqpoint{4.251126in}{2.664725in}}%
\pgfpathlineto{\pgfqpoint{4.255667in}{2.664725in}}%
\pgfpathlineto{\pgfqpoint{4.255667in}{2.661776in}}%
\pgfpathmoveto{\pgfqpoint{4.260208in}{2.655877in}}%
\pgfpathlineto{\pgfqpoint{4.260208in}{2.655877in}}%
\pgfpathlineto{\pgfqpoint{4.260208in}{2.658827in}}%
\pgfpathlineto{\pgfqpoint{4.264748in}{2.658827in}}%
\pgfpathlineto{\pgfqpoint{4.264748in}{2.655877in}}%
\pgfpathmoveto{\pgfqpoint{4.264748in}{2.655877in}}%
\pgfpathlineto{\pgfqpoint{4.264748in}{2.655877in}}%
\pgfpathlineto{\pgfqpoint{4.264748in}{2.658827in}}%
\pgfpathlineto{\pgfqpoint{4.269289in}{2.658827in}}%
\pgfpathlineto{\pgfqpoint{4.269289in}{2.655877in}}%
\pgfpathmoveto{\pgfqpoint{4.269289in}{2.655877in}}%
\pgfpathlineto{\pgfqpoint{4.269289in}{2.655877in}}%
\pgfpathlineto{\pgfqpoint{4.269289in}{2.658827in}}%
\pgfpathlineto{\pgfqpoint{4.273830in}{2.658827in}}%
\pgfpathlineto{\pgfqpoint{4.273830in}{2.655877in}}%
\pgfpathmoveto{\pgfqpoint{4.255667in}{2.658827in}}%
\pgfpathlineto{\pgfqpoint{4.255667in}{2.658827in}}%
\pgfpathlineto{\pgfqpoint{4.255667in}{2.661776in}}%
\pgfpathlineto{\pgfqpoint{4.260208in}{2.661776in}}%
\pgfpathlineto{\pgfqpoint{4.260208in}{2.658827in}}%
\pgfpathmoveto{\pgfqpoint{4.255667in}{2.661776in}}%
\pgfpathlineto{\pgfqpoint{4.255667in}{2.661776in}}%
\pgfpathlineto{\pgfqpoint{4.255667in}{2.664725in}}%
\pgfpathlineto{\pgfqpoint{4.260208in}{2.664725in}}%
\pgfpathlineto{\pgfqpoint{4.260208in}{2.661776in}}%
\pgfpathmoveto{\pgfqpoint{4.260208in}{2.658827in}}%
\pgfpathlineto{\pgfqpoint{4.260208in}{2.658827in}}%
\pgfpathlineto{\pgfqpoint{4.260208in}{2.661776in}}%
\pgfpathlineto{\pgfqpoint{4.264748in}{2.661776in}}%
\pgfpathlineto{\pgfqpoint{4.264748in}{2.658827in}}%
\pgfpathmoveto{\pgfqpoint{4.260208in}{2.661776in}}%
\pgfpathlineto{\pgfqpoint{4.260208in}{2.661776in}}%
\pgfpathlineto{\pgfqpoint{4.260208in}{2.664725in}}%
\pgfpathlineto{\pgfqpoint{4.264748in}{2.664725in}}%
\pgfpathlineto{\pgfqpoint{4.264748in}{2.661776in}}%
\pgfpathmoveto{\pgfqpoint{4.273830in}{2.652928in}}%
\pgfpathlineto{\pgfqpoint{4.273830in}{2.652928in}}%
\pgfpathlineto{\pgfqpoint{4.273830in}{2.655877in}}%
\pgfpathlineto{\pgfqpoint{4.278371in}{2.655877in}}%
\pgfpathlineto{\pgfqpoint{4.278371in}{2.652928in}}%
\pgfpathmoveto{\pgfqpoint{4.273830in}{2.655877in}}%
\pgfpathlineto{\pgfqpoint{4.273830in}{2.655877in}}%
\pgfpathlineto{\pgfqpoint{4.273830in}{2.658827in}}%
\pgfpathlineto{\pgfqpoint{4.278371in}{2.658827in}}%
\pgfpathlineto{\pgfqpoint{4.278371in}{2.655877in}}%
\pgfpathmoveto{\pgfqpoint{4.278371in}{2.652928in}}%
\pgfpathlineto{\pgfqpoint{4.278371in}{2.652928in}}%
\pgfpathlineto{\pgfqpoint{4.278371in}{2.655877in}}%
\pgfpathlineto{\pgfqpoint{4.282912in}{2.655877in}}%
\pgfpathlineto{\pgfqpoint{4.282912in}{2.652928in}}%
\pgfpathmoveto{\pgfqpoint{4.278371in}{2.655877in}}%
\pgfpathlineto{\pgfqpoint{4.278371in}{2.655877in}}%
\pgfpathlineto{\pgfqpoint{4.278371in}{2.658827in}}%
\pgfpathlineto{\pgfqpoint{4.282912in}{2.658827in}}%
\pgfpathlineto{\pgfqpoint{4.282912in}{2.655877in}}%
\pgfpathmoveto{\pgfqpoint{4.287453in}{2.649979in}}%
\pgfpathlineto{\pgfqpoint{4.287453in}{2.649979in}}%
\pgfpathlineto{\pgfqpoint{4.287453in}{2.652928in}}%
\pgfpathlineto{\pgfqpoint{4.291994in}{2.652928in}}%
\pgfpathlineto{\pgfqpoint{4.291994in}{2.649979in}}%
\pgfpathmoveto{\pgfqpoint{4.282912in}{2.652928in}}%
\pgfpathlineto{\pgfqpoint{4.282912in}{2.652928in}}%
\pgfpathlineto{\pgfqpoint{4.282912in}{2.655877in}}%
\pgfpathlineto{\pgfqpoint{4.287453in}{2.655877in}}%
\pgfpathlineto{\pgfqpoint{4.287453in}{2.652928in}}%
\pgfpathmoveto{\pgfqpoint{4.282912in}{2.655877in}}%
\pgfpathlineto{\pgfqpoint{4.282912in}{2.655877in}}%
\pgfpathlineto{\pgfqpoint{4.282912in}{2.658827in}}%
\pgfpathlineto{\pgfqpoint{4.287453in}{2.658827in}}%
\pgfpathlineto{\pgfqpoint{4.287453in}{2.655877in}}%
\pgfpathmoveto{\pgfqpoint{4.287453in}{2.652928in}}%
\pgfpathlineto{\pgfqpoint{4.287453in}{2.652928in}}%
\pgfpathlineto{\pgfqpoint{4.287453in}{2.655877in}}%
\pgfpathlineto{\pgfqpoint{4.291994in}{2.655877in}}%
\pgfpathlineto{\pgfqpoint{4.291994in}{2.652928in}}%
\pgfpathmoveto{\pgfqpoint{4.287453in}{2.655877in}}%
\pgfpathlineto{\pgfqpoint{4.287453in}{2.655877in}}%
\pgfpathlineto{\pgfqpoint{4.287453in}{2.658827in}}%
\pgfpathlineto{\pgfqpoint{4.291994in}{2.658827in}}%
\pgfpathlineto{\pgfqpoint{4.291994in}{2.655877in}}%
\pgfpathmoveto{\pgfqpoint{4.291994in}{2.649979in}}%
\pgfpathlineto{\pgfqpoint{4.291994in}{2.649979in}}%
\pgfpathlineto{\pgfqpoint{4.291994in}{2.652928in}}%
\pgfpathlineto{\pgfqpoint{4.296535in}{2.652928in}}%
\pgfpathlineto{\pgfqpoint{4.296535in}{2.649979in}}%
\pgfpathmoveto{\pgfqpoint{4.296535in}{2.649979in}}%
\pgfpathlineto{\pgfqpoint{4.296535in}{2.649979in}}%
\pgfpathlineto{\pgfqpoint{4.296535in}{2.652928in}}%
\pgfpathlineto{\pgfqpoint{4.301075in}{2.652928in}}%
\pgfpathlineto{\pgfqpoint{4.301075in}{2.649979in}}%
\pgfpathmoveto{\pgfqpoint{4.301075in}{2.647030in}}%
\pgfpathlineto{\pgfqpoint{4.301075in}{2.647030in}}%
\pgfpathlineto{\pgfqpoint{4.301075in}{2.649979in}}%
\pgfpathlineto{\pgfqpoint{4.305616in}{2.649979in}}%
\pgfpathlineto{\pgfqpoint{4.305616in}{2.647030in}}%
\pgfpathmoveto{\pgfqpoint{4.301075in}{2.649979in}}%
\pgfpathlineto{\pgfqpoint{4.301075in}{2.649979in}}%
\pgfpathlineto{\pgfqpoint{4.301075in}{2.652928in}}%
\pgfpathlineto{\pgfqpoint{4.305616in}{2.652928in}}%
\pgfpathlineto{\pgfqpoint{4.305616in}{2.649979in}}%
\pgfpathmoveto{\pgfqpoint{4.305616in}{2.647030in}}%
\pgfpathlineto{\pgfqpoint{4.305616in}{2.647030in}}%
\pgfpathlineto{\pgfqpoint{4.305616in}{2.649979in}}%
\pgfpathlineto{\pgfqpoint{4.310157in}{2.649979in}}%
\pgfpathlineto{\pgfqpoint{4.310157in}{2.647030in}}%
\pgfpathmoveto{\pgfqpoint{4.305616in}{2.649979in}}%
\pgfpathlineto{\pgfqpoint{4.305616in}{2.649979in}}%
\pgfpathlineto{\pgfqpoint{4.305616in}{2.652928in}}%
\pgfpathlineto{\pgfqpoint{4.310157in}{2.652928in}}%
\pgfpathlineto{\pgfqpoint{4.310157in}{2.649979in}}%
\pgfpathmoveto{\pgfqpoint{4.314698in}{2.644080in}}%
\pgfpathlineto{\pgfqpoint{4.314698in}{2.644080in}}%
\pgfpathlineto{\pgfqpoint{4.314698in}{2.647030in}}%
\pgfpathlineto{\pgfqpoint{4.319239in}{2.647030in}}%
\pgfpathlineto{\pgfqpoint{4.319239in}{2.644080in}}%
\pgfpathmoveto{\pgfqpoint{4.319239in}{2.644080in}}%
\pgfpathlineto{\pgfqpoint{4.319239in}{2.644080in}}%
\pgfpathlineto{\pgfqpoint{4.319239in}{2.647030in}}%
\pgfpathlineto{\pgfqpoint{4.323780in}{2.647030in}}%
\pgfpathlineto{\pgfqpoint{4.323780in}{2.644080in}}%
\pgfpathmoveto{\pgfqpoint{4.323780in}{2.644080in}}%
\pgfpathlineto{\pgfqpoint{4.323780in}{2.644080in}}%
\pgfpathlineto{\pgfqpoint{4.323780in}{2.647030in}}%
\pgfpathlineto{\pgfqpoint{4.328321in}{2.647030in}}%
\pgfpathlineto{\pgfqpoint{4.328321in}{2.644080in}}%
\pgfpathmoveto{\pgfqpoint{4.328321in}{2.641131in}}%
\pgfpathlineto{\pgfqpoint{4.328321in}{2.641131in}}%
\pgfpathlineto{\pgfqpoint{4.328321in}{2.644080in}}%
\pgfpathlineto{\pgfqpoint{4.332862in}{2.644080in}}%
\pgfpathlineto{\pgfqpoint{4.332862in}{2.641131in}}%
\pgfpathmoveto{\pgfqpoint{4.328321in}{2.644080in}}%
\pgfpathlineto{\pgfqpoint{4.328321in}{2.644080in}}%
\pgfpathlineto{\pgfqpoint{4.328321in}{2.647030in}}%
\pgfpathlineto{\pgfqpoint{4.332862in}{2.647030in}}%
\pgfpathlineto{\pgfqpoint{4.332862in}{2.644080in}}%
\pgfpathmoveto{\pgfqpoint{4.332862in}{2.641131in}}%
\pgfpathlineto{\pgfqpoint{4.332862in}{2.641131in}}%
\pgfpathlineto{\pgfqpoint{4.332862in}{2.644080in}}%
\pgfpathlineto{\pgfqpoint{4.337403in}{2.644080in}}%
\pgfpathlineto{\pgfqpoint{4.337403in}{2.641131in}}%
\pgfpathmoveto{\pgfqpoint{4.332862in}{2.644080in}}%
\pgfpathlineto{\pgfqpoint{4.332862in}{2.644080in}}%
\pgfpathlineto{\pgfqpoint{4.332862in}{2.647030in}}%
\pgfpathlineto{\pgfqpoint{4.337403in}{2.647030in}}%
\pgfpathlineto{\pgfqpoint{4.337403in}{2.644080in}}%
\pgfpathmoveto{\pgfqpoint{4.341943in}{2.638182in}}%
\pgfpathlineto{\pgfqpoint{4.341943in}{2.638182in}}%
\pgfpathlineto{\pgfqpoint{4.341943in}{2.641131in}}%
\pgfpathlineto{\pgfqpoint{4.346484in}{2.641131in}}%
\pgfpathlineto{\pgfqpoint{4.346484in}{2.638182in}}%
\pgfpathmoveto{\pgfqpoint{4.337403in}{2.641131in}}%
\pgfpathlineto{\pgfqpoint{4.337403in}{2.641131in}}%
\pgfpathlineto{\pgfqpoint{4.337403in}{2.644080in}}%
\pgfpathlineto{\pgfqpoint{4.341943in}{2.644080in}}%
\pgfpathlineto{\pgfqpoint{4.341943in}{2.641131in}}%
\pgfpathmoveto{\pgfqpoint{4.337403in}{2.644080in}}%
\pgfpathlineto{\pgfqpoint{4.337403in}{2.644080in}}%
\pgfpathlineto{\pgfqpoint{4.337403in}{2.647030in}}%
\pgfpathlineto{\pgfqpoint{4.341943in}{2.647030in}}%
\pgfpathlineto{\pgfqpoint{4.341943in}{2.644080in}}%
\pgfpathmoveto{\pgfqpoint{4.341943in}{2.641131in}}%
\pgfpathlineto{\pgfqpoint{4.341943in}{2.641131in}}%
\pgfpathlineto{\pgfqpoint{4.341943in}{2.644080in}}%
\pgfpathlineto{\pgfqpoint{4.346484in}{2.644080in}}%
\pgfpathlineto{\pgfqpoint{4.346484in}{2.641131in}}%
\pgfpathmoveto{\pgfqpoint{4.341943in}{2.644080in}}%
\pgfpathlineto{\pgfqpoint{4.341943in}{2.644080in}}%
\pgfpathlineto{\pgfqpoint{4.341943in}{2.647030in}}%
\pgfpathlineto{\pgfqpoint{4.346484in}{2.647030in}}%
\pgfpathlineto{\pgfqpoint{4.346484in}{2.644080in}}%
\pgfpathmoveto{\pgfqpoint{4.310157in}{2.647030in}}%
\pgfpathlineto{\pgfqpoint{4.310157in}{2.647030in}}%
\pgfpathlineto{\pgfqpoint{4.310157in}{2.649979in}}%
\pgfpathlineto{\pgfqpoint{4.314698in}{2.649979in}}%
\pgfpathlineto{\pgfqpoint{4.314698in}{2.647030in}}%
\pgfpathmoveto{\pgfqpoint{4.310157in}{2.649979in}}%
\pgfpathlineto{\pgfqpoint{4.310157in}{2.649979in}}%
\pgfpathlineto{\pgfqpoint{4.310157in}{2.652928in}}%
\pgfpathlineto{\pgfqpoint{4.314698in}{2.652928in}}%
\pgfpathlineto{\pgfqpoint{4.314698in}{2.649979in}}%
\pgfpathmoveto{\pgfqpoint{4.314698in}{2.647030in}}%
\pgfpathlineto{\pgfqpoint{4.314698in}{2.647030in}}%
\pgfpathlineto{\pgfqpoint{4.314698in}{2.649979in}}%
\pgfpathlineto{\pgfqpoint{4.319239in}{2.649979in}}%
\pgfpathlineto{\pgfqpoint{4.319239in}{2.647030in}}%
\pgfpathmoveto{\pgfqpoint{4.314698in}{2.649979in}}%
\pgfpathlineto{\pgfqpoint{4.314698in}{2.649979in}}%
\pgfpathlineto{\pgfqpoint{4.314698in}{2.652928in}}%
\pgfpathlineto{\pgfqpoint{4.319239in}{2.652928in}}%
\pgfpathlineto{\pgfqpoint{4.319239in}{2.649979in}}%
\pgfpathmoveto{\pgfqpoint{4.346484in}{2.638182in}}%
\pgfpathlineto{\pgfqpoint{4.346484in}{2.638182in}}%
\pgfpathlineto{\pgfqpoint{4.346484in}{2.641131in}}%
\pgfpathlineto{\pgfqpoint{4.351025in}{2.641131in}}%
\pgfpathlineto{\pgfqpoint{4.351025in}{2.638182in}}%
\pgfpathmoveto{\pgfqpoint{4.351025in}{2.638182in}}%
\pgfpathlineto{\pgfqpoint{4.351025in}{2.638182in}}%
\pgfpathlineto{\pgfqpoint{4.351025in}{2.641131in}}%
\pgfpathlineto{\pgfqpoint{4.355566in}{2.641131in}}%
\pgfpathlineto{\pgfqpoint{4.355566in}{2.638182in}}%
\pgfpathmoveto{\pgfqpoint{4.355566in}{2.635233in}}%
\pgfpathlineto{\pgfqpoint{4.355566in}{2.635233in}}%
\pgfpathlineto{\pgfqpoint{4.355566in}{2.638182in}}%
\pgfpathlineto{\pgfqpoint{4.360107in}{2.638182in}}%
\pgfpathlineto{\pgfqpoint{4.360107in}{2.635233in}}%
\pgfpathmoveto{\pgfqpoint{4.355566in}{2.638182in}}%
\pgfpathlineto{\pgfqpoint{4.355566in}{2.638182in}}%
\pgfpathlineto{\pgfqpoint{4.355566in}{2.641131in}}%
\pgfpathlineto{\pgfqpoint{4.360107in}{2.641131in}}%
\pgfpathlineto{\pgfqpoint{4.360107in}{2.638182in}}%
\pgfpathmoveto{\pgfqpoint{4.360107in}{2.635233in}}%
\pgfpathlineto{\pgfqpoint{4.360107in}{2.635233in}}%
\pgfpathlineto{\pgfqpoint{4.360107in}{2.638182in}}%
\pgfpathlineto{\pgfqpoint{4.364648in}{2.638182in}}%
\pgfpathlineto{\pgfqpoint{4.364648in}{2.635233in}}%
\pgfpathmoveto{\pgfqpoint{4.360107in}{2.638182in}}%
\pgfpathlineto{\pgfqpoint{4.360107in}{2.638182in}}%
\pgfpathlineto{\pgfqpoint{4.360107in}{2.641131in}}%
\pgfpathlineto{\pgfqpoint{4.364648in}{2.641131in}}%
\pgfpathlineto{\pgfqpoint{4.364648in}{2.638182in}}%
\pgfpathmoveto{\pgfqpoint{4.369189in}{2.632283in}}%
\pgfpathlineto{\pgfqpoint{4.369189in}{2.632283in}}%
\pgfpathlineto{\pgfqpoint{4.369189in}{2.635233in}}%
\pgfpathlineto{\pgfqpoint{4.373730in}{2.635233in}}%
\pgfpathlineto{\pgfqpoint{4.373730in}{2.632283in}}%
\pgfpathmoveto{\pgfqpoint{4.373730in}{2.632283in}}%
\pgfpathlineto{\pgfqpoint{4.373730in}{2.632283in}}%
\pgfpathlineto{\pgfqpoint{4.373730in}{2.635233in}}%
\pgfpathlineto{\pgfqpoint{4.378270in}{2.635233in}}%
\pgfpathlineto{\pgfqpoint{4.378270in}{2.632283in}}%
\pgfpathmoveto{\pgfqpoint{4.378270in}{2.632283in}}%
\pgfpathlineto{\pgfqpoint{4.378270in}{2.632283in}}%
\pgfpathlineto{\pgfqpoint{4.378270in}{2.635233in}}%
\pgfpathlineto{\pgfqpoint{4.382811in}{2.635233in}}%
\pgfpathlineto{\pgfqpoint{4.382811in}{2.632283in}}%
\pgfpathmoveto{\pgfqpoint{4.364648in}{2.635233in}}%
\pgfpathlineto{\pgfqpoint{4.364648in}{2.635233in}}%
\pgfpathlineto{\pgfqpoint{4.364648in}{2.638182in}}%
\pgfpathlineto{\pgfqpoint{4.369189in}{2.638182in}}%
\pgfpathlineto{\pgfqpoint{4.369189in}{2.635233in}}%
\pgfpathmoveto{\pgfqpoint{4.364648in}{2.638182in}}%
\pgfpathlineto{\pgfqpoint{4.364648in}{2.638182in}}%
\pgfpathlineto{\pgfqpoint{4.364648in}{2.641131in}}%
\pgfpathlineto{\pgfqpoint{4.369189in}{2.641131in}}%
\pgfpathlineto{\pgfqpoint{4.369189in}{2.638182in}}%
\pgfpathmoveto{\pgfqpoint{4.369189in}{2.635233in}}%
\pgfpathlineto{\pgfqpoint{4.369189in}{2.635233in}}%
\pgfpathlineto{\pgfqpoint{4.369189in}{2.638182in}}%
\pgfpathlineto{\pgfqpoint{4.373730in}{2.638182in}}%
\pgfpathlineto{\pgfqpoint{4.373730in}{2.635233in}}%
\pgfpathmoveto{\pgfqpoint{4.369189in}{2.638182in}}%
\pgfpathlineto{\pgfqpoint{4.369189in}{2.638182in}}%
\pgfpathlineto{\pgfqpoint{4.369189in}{2.641131in}}%
\pgfpathlineto{\pgfqpoint{4.373730in}{2.641131in}}%
\pgfpathlineto{\pgfqpoint{4.373730in}{2.638182in}}%
\pgfpathmoveto{\pgfqpoint{4.237503in}{2.753201in}}%
\pgfpathlineto{\pgfqpoint{4.237503in}{2.753201in}}%
\pgfpathlineto{\pgfqpoint{4.237503in}{2.756151in}}%
\pgfpathlineto{\pgfqpoint{4.242044in}{2.756151in}}%
\pgfpathlineto{\pgfqpoint{4.242044in}{2.753201in}}%
\pgfpathmoveto{\pgfqpoint{4.237503in}{2.756151in}}%
\pgfpathlineto{\pgfqpoint{4.237503in}{2.756151in}}%
\pgfpathlineto{\pgfqpoint{4.237503in}{2.759100in}}%
\pgfpathlineto{\pgfqpoint{4.242044in}{2.759100in}}%
\pgfpathlineto{\pgfqpoint{4.242044in}{2.756151in}}%
\pgfpathmoveto{\pgfqpoint{4.242044in}{2.753201in}}%
\pgfpathlineto{\pgfqpoint{4.242044in}{2.753201in}}%
\pgfpathlineto{\pgfqpoint{4.242044in}{2.756151in}}%
\pgfpathlineto{\pgfqpoint{4.246585in}{2.756151in}}%
\pgfpathlineto{\pgfqpoint{4.246585in}{2.753201in}}%
\pgfpathmoveto{\pgfqpoint{4.242044in}{2.756151in}}%
\pgfpathlineto{\pgfqpoint{4.242044in}{2.756151in}}%
\pgfpathlineto{\pgfqpoint{4.242044in}{2.759100in}}%
\pgfpathlineto{\pgfqpoint{4.246585in}{2.759100in}}%
\pgfpathlineto{\pgfqpoint{4.246585in}{2.756151in}}%
\pgfpathmoveto{\pgfqpoint{4.237503in}{2.759100in}}%
\pgfpathlineto{\pgfqpoint{4.237503in}{2.759100in}}%
\pgfpathlineto{\pgfqpoint{4.237503in}{2.762049in}}%
\pgfpathlineto{\pgfqpoint{4.242044in}{2.762049in}}%
\pgfpathlineto{\pgfqpoint{4.242044in}{2.759100in}}%
\pgfpathmoveto{\pgfqpoint{4.237503in}{2.762049in}}%
\pgfpathlineto{\pgfqpoint{4.237503in}{2.762049in}}%
\pgfpathlineto{\pgfqpoint{4.237503in}{2.764998in}}%
\pgfpathlineto{\pgfqpoint{4.242044in}{2.764998in}}%
\pgfpathlineto{\pgfqpoint{4.242044in}{2.762049in}}%
\pgfpathmoveto{\pgfqpoint{4.242044in}{2.759100in}}%
\pgfpathlineto{\pgfqpoint{4.242044in}{2.759100in}}%
\pgfpathlineto{\pgfqpoint{4.242044in}{2.762049in}}%
\pgfpathlineto{\pgfqpoint{4.246585in}{2.762049in}}%
\pgfpathlineto{\pgfqpoint{4.246585in}{2.759100in}}%
\pgfpathmoveto{\pgfqpoint{4.246585in}{2.753201in}}%
\pgfpathlineto{\pgfqpoint{4.246585in}{2.753201in}}%
\pgfpathlineto{\pgfqpoint{4.246585in}{2.756151in}}%
\pgfpathlineto{\pgfqpoint{4.251126in}{2.756151in}}%
\pgfpathlineto{\pgfqpoint{4.251126in}{2.753201in}}%
\pgfpathmoveto{\pgfqpoint{4.246585in}{2.756151in}}%
\pgfpathlineto{\pgfqpoint{4.246585in}{2.756151in}}%
\pgfpathlineto{\pgfqpoint{4.246585in}{2.759100in}}%
\pgfpathlineto{\pgfqpoint{4.251126in}{2.759100in}}%
\pgfpathlineto{\pgfqpoint{4.251126in}{2.756151in}}%
\pgfpathmoveto{\pgfqpoint{4.251126in}{2.753201in}}%
\pgfpathlineto{\pgfqpoint{4.251126in}{2.753201in}}%
\pgfpathlineto{\pgfqpoint{4.251126in}{2.756151in}}%
\pgfpathlineto{\pgfqpoint{4.255667in}{2.756151in}}%
\pgfpathlineto{\pgfqpoint{4.255667in}{2.753201in}}%
\pgfpathmoveto{\pgfqpoint{4.251126in}{2.756151in}}%
\pgfpathlineto{\pgfqpoint{4.251126in}{2.756151in}}%
\pgfpathlineto{\pgfqpoint{4.251126in}{2.759100in}}%
\pgfpathlineto{\pgfqpoint{4.255667in}{2.759100in}}%
\pgfpathlineto{\pgfqpoint{4.255667in}{2.756151in}}%
\pgfpathmoveto{\pgfqpoint{4.255667in}{2.747303in}}%
\pgfpathlineto{\pgfqpoint{4.255667in}{2.747303in}}%
\pgfpathlineto{\pgfqpoint{4.255667in}{2.750252in}}%
\pgfpathlineto{\pgfqpoint{4.260208in}{2.750252in}}%
\pgfpathlineto{\pgfqpoint{4.260208in}{2.747303in}}%
\pgfpathmoveto{\pgfqpoint{4.255667in}{2.750252in}}%
\pgfpathlineto{\pgfqpoint{4.255667in}{2.750252in}}%
\pgfpathlineto{\pgfqpoint{4.255667in}{2.753201in}}%
\pgfpathlineto{\pgfqpoint{4.260208in}{2.753201in}}%
\pgfpathlineto{\pgfqpoint{4.260208in}{2.750252in}}%
\pgfpathmoveto{\pgfqpoint{4.260208in}{2.747303in}}%
\pgfpathlineto{\pgfqpoint{4.260208in}{2.747303in}}%
\pgfpathlineto{\pgfqpoint{4.260208in}{2.750252in}}%
\pgfpathlineto{\pgfqpoint{4.264748in}{2.750252in}}%
\pgfpathlineto{\pgfqpoint{4.264748in}{2.747303in}}%
\pgfpathmoveto{\pgfqpoint{4.260208in}{2.750252in}}%
\pgfpathlineto{\pgfqpoint{4.260208in}{2.750252in}}%
\pgfpathlineto{\pgfqpoint{4.260208in}{2.753201in}}%
\pgfpathlineto{\pgfqpoint{4.264748in}{2.753201in}}%
\pgfpathlineto{\pgfqpoint{4.264748in}{2.750252in}}%
\pgfpathmoveto{\pgfqpoint{4.264748in}{2.741405in}}%
\pgfpathlineto{\pgfqpoint{4.264748in}{2.741405in}}%
\pgfpathlineto{\pgfqpoint{4.264748in}{2.744354in}}%
\pgfpathlineto{\pgfqpoint{4.269289in}{2.744354in}}%
\pgfpathlineto{\pgfqpoint{4.269289in}{2.741405in}}%
\pgfpathmoveto{\pgfqpoint{4.264748in}{2.744354in}}%
\pgfpathlineto{\pgfqpoint{4.264748in}{2.744354in}}%
\pgfpathlineto{\pgfqpoint{4.264748in}{2.747303in}}%
\pgfpathlineto{\pgfqpoint{4.269289in}{2.747303in}}%
\pgfpathlineto{\pgfqpoint{4.269289in}{2.744354in}}%
\pgfpathmoveto{\pgfqpoint{4.269289in}{2.741405in}}%
\pgfpathlineto{\pgfqpoint{4.269289in}{2.741405in}}%
\pgfpathlineto{\pgfqpoint{4.269289in}{2.744354in}}%
\pgfpathlineto{\pgfqpoint{4.273830in}{2.744354in}}%
\pgfpathlineto{\pgfqpoint{4.273830in}{2.741405in}}%
\pgfpathmoveto{\pgfqpoint{4.269289in}{2.744354in}}%
\pgfpathlineto{\pgfqpoint{4.269289in}{2.744354in}}%
\pgfpathlineto{\pgfqpoint{4.269289in}{2.747303in}}%
\pgfpathlineto{\pgfqpoint{4.273830in}{2.747303in}}%
\pgfpathlineto{\pgfqpoint{4.273830in}{2.744354in}}%
\pgfpathmoveto{\pgfqpoint{4.264748in}{2.747303in}}%
\pgfpathlineto{\pgfqpoint{4.264748in}{2.747303in}}%
\pgfpathlineto{\pgfqpoint{4.264748in}{2.750252in}}%
\pgfpathlineto{\pgfqpoint{4.269289in}{2.750252in}}%
\pgfpathlineto{\pgfqpoint{4.269289in}{2.747303in}}%
\pgfpathmoveto{\pgfqpoint{4.255667in}{2.753201in}}%
\pgfpathlineto{\pgfqpoint{4.255667in}{2.753201in}}%
\pgfpathlineto{\pgfqpoint{4.255667in}{2.756151in}}%
\pgfpathlineto{\pgfqpoint{4.260208in}{2.756151in}}%
\pgfpathlineto{\pgfqpoint{4.260208in}{2.753201in}}%
\pgfpathmoveto{\pgfqpoint{4.273830in}{2.735506in}}%
\pgfpathlineto{\pgfqpoint{4.273830in}{2.735506in}}%
\pgfpathlineto{\pgfqpoint{4.273830in}{2.738455in}}%
\pgfpathlineto{\pgfqpoint{4.278371in}{2.738455in}}%
\pgfpathlineto{\pgfqpoint{4.278371in}{2.735506in}}%
\pgfpathmoveto{\pgfqpoint{4.273830in}{2.738455in}}%
\pgfpathlineto{\pgfqpoint{4.273830in}{2.738455in}}%
\pgfpathlineto{\pgfqpoint{4.273830in}{2.741405in}}%
\pgfpathlineto{\pgfqpoint{4.278371in}{2.741405in}}%
\pgfpathlineto{\pgfqpoint{4.278371in}{2.738455in}}%
\pgfpathmoveto{\pgfqpoint{4.278371in}{2.735506in}}%
\pgfpathlineto{\pgfqpoint{4.278371in}{2.735506in}}%
\pgfpathlineto{\pgfqpoint{4.278371in}{2.738455in}}%
\pgfpathlineto{\pgfqpoint{4.282912in}{2.738455in}}%
\pgfpathlineto{\pgfqpoint{4.282912in}{2.735506in}}%
\pgfpathmoveto{\pgfqpoint{4.278371in}{2.738455in}}%
\pgfpathlineto{\pgfqpoint{4.278371in}{2.738455in}}%
\pgfpathlineto{\pgfqpoint{4.278371in}{2.741405in}}%
\pgfpathlineto{\pgfqpoint{4.282912in}{2.741405in}}%
\pgfpathlineto{\pgfqpoint{4.282912in}{2.738455in}}%
\pgfpathmoveto{\pgfqpoint{4.282912in}{2.729608in}}%
\pgfpathlineto{\pgfqpoint{4.282912in}{2.729608in}}%
\pgfpathlineto{\pgfqpoint{4.282912in}{2.732557in}}%
\pgfpathlineto{\pgfqpoint{4.287453in}{2.732557in}}%
\pgfpathlineto{\pgfqpoint{4.287453in}{2.729608in}}%
\pgfpathmoveto{\pgfqpoint{4.282912in}{2.732557in}}%
\pgfpathlineto{\pgfqpoint{4.282912in}{2.732557in}}%
\pgfpathlineto{\pgfqpoint{4.282912in}{2.735506in}}%
\pgfpathlineto{\pgfqpoint{4.287453in}{2.735506in}}%
\pgfpathlineto{\pgfqpoint{4.287453in}{2.732557in}}%
\pgfpathmoveto{\pgfqpoint{4.287453in}{2.729608in}}%
\pgfpathlineto{\pgfqpoint{4.287453in}{2.729608in}}%
\pgfpathlineto{\pgfqpoint{4.287453in}{2.732557in}}%
\pgfpathlineto{\pgfqpoint{4.291994in}{2.732557in}}%
\pgfpathlineto{\pgfqpoint{4.291994in}{2.729608in}}%
\pgfpathmoveto{\pgfqpoint{4.287453in}{2.732557in}}%
\pgfpathlineto{\pgfqpoint{4.287453in}{2.732557in}}%
\pgfpathlineto{\pgfqpoint{4.287453in}{2.735506in}}%
\pgfpathlineto{\pgfqpoint{4.291994in}{2.735506in}}%
\pgfpathlineto{\pgfqpoint{4.291994in}{2.732557in}}%
\pgfpathmoveto{\pgfqpoint{4.282912in}{2.735506in}}%
\pgfpathlineto{\pgfqpoint{4.282912in}{2.735506in}}%
\pgfpathlineto{\pgfqpoint{4.282912in}{2.738455in}}%
\pgfpathlineto{\pgfqpoint{4.287453in}{2.738455in}}%
\pgfpathlineto{\pgfqpoint{4.287453in}{2.735506in}}%
\pgfpathmoveto{\pgfqpoint{4.291994in}{2.723709in}}%
\pgfpathlineto{\pgfqpoint{4.291994in}{2.723709in}}%
\pgfpathlineto{\pgfqpoint{4.291994in}{2.726659in}}%
\pgfpathlineto{\pgfqpoint{4.296535in}{2.726659in}}%
\pgfpathlineto{\pgfqpoint{4.296535in}{2.723709in}}%
\pgfpathmoveto{\pgfqpoint{4.291994in}{2.726659in}}%
\pgfpathlineto{\pgfqpoint{4.291994in}{2.726659in}}%
\pgfpathlineto{\pgfqpoint{4.291994in}{2.729608in}}%
\pgfpathlineto{\pgfqpoint{4.296535in}{2.729608in}}%
\pgfpathlineto{\pgfqpoint{4.296535in}{2.726659in}}%
\pgfpathmoveto{\pgfqpoint{4.296535in}{2.723709in}}%
\pgfpathlineto{\pgfqpoint{4.296535in}{2.723709in}}%
\pgfpathlineto{\pgfqpoint{4.296535in}{2.726659in}}%
\pgfpathlineto{\pgfqpoint{4.301075in}{2.726659in}}%
\pgfpathlineto{\pgfqpoint{4.301075in}{2.723709in}}%
\pgfpathmoveto{\pgfqpoint{4.296535in}{2.726659in}}%
\pgfpathlineto{\pgfqpoint{4.296535in}{2.726659in}}%
\pgfpathlineto{\pgfqpoint{4.296535in}{2.729608in}}%
\pgfpathlineto{\pgfqpoint{4.301075in}{2.729608in}}%
\pgfpathlineto{\pgfqpoint{4.301075in}{2.726659in}}%
\pgfpathmoveto{\pgfqpoint{4.301075in}{2.717811in}}%
\pgfpathlineto{\pgfqpoint{4.301075in}{2.717811in}}%
\pgfpathlineto{\pgfqpoint{4.301075in}{2.720760in}}%
\pgfpathlineto{\pgfqpoint{4.305616in}{2.720760in}}%
\pgfpathlineto{\pgfqpoint{4.305616in}{2.717811in}}%
\pgfpathmoveto{\pgfqpoint{4.301075in}{2.720760in}}%
\pgfpathlineto{\pgfqpoint{4.301075in}{2.720760in}}%
\pgfpathlineto{\pgfqpoint{4.301075in}{2.723709in}}%
\pgfpathlineto{\pgfqpoint{4.305616in}{2.723709in}}%
\pgfpathlineto{\pgfqpoint{4.305616in}{2.720760in}}%
\pgfpathmoveto{\pgfqpoint{4.305616in}{2.717811in}}%
\pgfpathlineto{\pgfqpoint{4.305616in}{2.717811in}}%
\pgfpathlineto{\pgfqpoint{4.305616in}{2.720760in}}%
\pgfpathlineto{\pgfqpoint{4.310157in}{2.720760in}}%
\pgfpathlineto{\pgfqpoint{4.310157in}{2.717811in}}%
\pgfpathmoveto{\pgfqpoint{4.305616in}{2.720760in}}%
\pgfpathlineto{\pgfqpoint{4.305616in}{2.720760in}}%
\pgfpathlineto{\pgfqpoint{4.305616in}{2.723709in}}%
\pgfpathlineto{\pgfqpoint{4.310157in}{2.723709in}}%
\pgfpathlineto{\pgfqpoint{4.310157in}{2.720760in}}%
\pgfpathmoveto{\pgfqpoint{4.301075in}{2.723709in}}%
\pgfpathlineto{\pgfqpoint{4.301075in}{2.723709in}}%
\pgfpathlineto{\pgfqpoint{4.301075in}{2.726659in}}%
\pgfpathlineto{\pgfqpoint{4.305616in}{2.726659in}}%
\pgfpathlineto{\pgfqpoint{4.305616in}{2.723709in}}%
\pgfpathmoveto{\pgfqpoint{4.291994in}{2.729608in}}%
\pgfpathlineto{\pgfqpoint{4.291994in}{2.729608in}}%
\pgfpathlineto{\pgfqpoint{4.291994in}{2.732557in}}%
\pgfpathlineto{\pgfqpoint{4.296535in}{2.732557in}}%
\pgfpathlineto{\pgfqpoint{4.296535in}{2.729608in}}%
\pgfpathmoveto{\pgfqpoint{4.273830in}{2.741405in}}%
\pgfpathlineto{\pgfqpoint{4.273830in}{2.741405in}}%
\pgfpathlineto{\pgfqpoint{4.273830in}{2.744354in}}%
\pgfpathlineto{\pgfqpoint{4.278371in}{2.744354in}}%
\pgfpathlineto{\pgfqpoint{4.278371in}{2.741405in}}%
\pgfpathmoveto{\pgfqpoint{4.310157in}{2.711913in}}%
\pgfpathlineto{\pgfqpoint{4.310157in}{2.711913in}}%
\pgfpathlineto{\pgfqpoint{4.310157in}{2.714862in}}%
\pgfpathlineto{\pgfqpoint{4.314698in}{2.714862in}}%
\pgfpathlineto{\pgfqpoint{4.314698in}{2.711913in}}%
\pgfpathmoveto{\pgfqpoint{4.310157in}{2.714862in}}%
\pgfpathlineto{\pgfqpoint{4.310157in}{2.714862in}}%
\pgfpathlineto{\pgfqpoint{4.310157in}{2.717811in}}%
\pgfpathlineto{\pgfqpoint{4.314698in}{2.717811in}}%
\pgfpathlineto{\pgfqpoint{4.314698in}{2.714862in}}%
\pgfpathmoveto{\pgfqpoint{4.314698in}{2.711913in}}%
\pgfpathlineto{\pgfqpoint{4.314698in}{2.711913in}}%
\pgfpathlineto{\pgfqpoint{4.314698in}{2.714862in}}%
\pgfpathlineto{\pgfqpoint{4.319239in}{2.714862in}}%
\pgfpathlineto{\pgfqpoint{4.319239in}{2.711913in}}%
\pgfpathmoveto{\pgfqpoint{4.314698in}{2.714862in}}%
\pgfpathlineto{\pgfqpoint{4.314698in}{2.714862in}}%
\pgfpathlineto{\pgfqpoint{4.314698in}{2.717811in}}%
\pgfpathlineto{\pgfqpoint{4.319239in}{2.717811in}}%
\pgfpathlineto{\pgfqpoint{4.319239in}{2.714862in}}%
\pgfpathmoveto{\pgfqpoint{4.319239in}{2.706014in}}%
\pgfpathlineto{\pgfqpoint{4.319239in}{2.706014in}}%
\pgfpathlineto{\pgfqpoint{4.319239in}{2.708963in}}%
\pgfpathlineto{\pgfqpoint{4.323780in}{2.708963in}}%
\pgfpathlineto{\pgfqpoint{4.323780in}{2.706014in}}%
\pgfpathmoveto{\pgfqpoint{4.319239in}{2.708963in}}%
\pgfpathlineto{\pgfqpoint{4.319239in}{2.708963in}}%
\pgfpathlineto{\pgfqpoint{4.319239in}{2.711913in}}%
\pgfpathlineto{\pgfqpoint{4.323780in}{2.711913in}}%
\pgfpathlineto{\pgfqpoint{4.323780in}{2.708963in}}%
\pgfpathmoveto{\pgfqpoint{4.323780in}{2.706014in}}%
\pgfpathlineto{\pgfqpoint{4.323780in}{2.706014in}}%
\pgfpathlineto{\pgfqpoint{4.323780in}{2.708963in}}%
\pgfpathlineto{\pgfqpoint{4.328321in}{2.708963in}}%
\pgfpathlineto{\pgfqpoint{4.328321in}{2.706014in}}%
\pgfpathmoveto{\pgfqpoint{4.323780in}{2.708963in}}%
\pgfpathlineto{\pgfqpoint{4.323780in}{2.708963in}}%
\pgfpathlineto{\pgfqpoint{4.323780in}{2.711913in}}%
\pgfpathlineto{\pgfqpoint{4.328321in}{2.711913in}}%
\pgfpathlineto{\pgfqpoint{4.328321in}{2.708963in}}%
\pgfpathmoveto{\pgfqpoint{4.319239in}{2.711913in}}%
\pgfpathlineto{\pgfqpoint{4.319239in}{2.711913in}}%
\pgfpathlineto{\pgfqpoint{4.319239in}{2.714862in}}%
\pgfpathlineto{\pgfqpoint{4.323780in}{2.714862in}}%
\pgfpathlineto{\pgfqpoint{4.323780in}{2.711913in}}%
\pgfpathmoveto{\pgfqpoint{4.328321in}{2.700116in}}%
\pgfpathlineto{\pgfqpoint{4.328321in}{2.700116in}}%
\pgfpathlineto{\pgfqpoint{4.328321in}{2.703065in}}%
\pgfpathlineto{\pgfqpoint{4.332862in}{2.703065in}}%
\pgfpathlineto{\pgfqpoint{4.332862in}{2.700116in}}%
\pgfpathmoveto{\pgfqpoint{4.328321in}{2.703065in}}%
\pgfpathlineto{\pgfqpoint{4.328321in}{2.703065in}}%
\pgfpathlineto{\pgfqpoint{4.328321in}{2.706014in}}%
\pgfpathlineto{\pgfqpoint{4.332862in}{2.706014in}}%
\pgfpathlineto{\pgfqpoint{4.332862in}{2.703065in}}%
\pgfpathmoveto{\pgfqpoint{4.332862in}{2.700116in}}%
\pgfpathlineto{\pgfqpoint{4.332862in}{2.700116in}}%
\pgfpathlineto{\pgfqpoint{4.332862in}{2.703065in}}%
\pgfpathlineto{\pgfqpoint{4.337403in}{2.703065in}}%
\pgfpathlineto{\pgfqpoint{4.337403in}{2.700116in}}%
\pgfpathmoveto{\pgfqpoint{4.332862in}{2.703065in}}%
\pgfpathlineto{\pgfqpoint{4.332862in}{2.703065in}}%
\pgfpathlineto{\pgfqpoint{4.332862in}{2.706014in}}%
\pgfpathlineto{\pgfqpoint{4.337403in}{2.706014in}}%
\pgfpathlineto{\pgfqpoint{4.337403in}{2.703065in}}%
\pgfpathmoveto{\pgfqpoint{4.337403in}{2.694217in}}%
\pgfpathlineto{\pgfqpoint{4.337403in}{2.694217in}}%
\pgfpathlineto{\pgfqpoint{4.337403in}{2.697167in}}%
\pgfpathlineto{\pgfqpoint{4.341943in}{2.697167in}}%
\pgfpathlineto{\pgfqpoint{4.341943in}{2.694217in}}%
\pgfpathmoveto{\pgfqpoint{4.337403in}{2.697167in}}%
\pgfpathlineto{\pgfqpoint{4.337403in}{2.697167in}}%
\pgfpathlineto{\pgfqpoint{4.337403in}{2.700116in}}%
\pgfpathlineto{\pgfqpoint{4.341943in}{2.700116in}}%
\pgfpathlineto{\pgfqpoint{4.341943in}{2.697167in}}%
\pgfpathmoveto{\pgfqpoint{4.341943in}{2.694217in}}%
\pgfpathlineto{\pgfqpoint{4.341943in}{2.694217in}}%
\pgfpathlineto{\pgfqpoint{4.341943in}{2.697167in}}%
\pgfpathlineto{\pgfqpoint{4.346484in}{2.697167in}}%
\pgfpathlineto{\pgfqpoint{4.346484in}{2.694217in}}%
\pgfpathmoveto{\pgfqpoint{4.341943in}{2.697167in}}%
\pgfpathlineto{\pgfqpoint{4.341943in}{2.697167in}}%
\pgfpathlineto{\pgfqpoint{4.341943in}{2.700116in}}%
\pgfpathlineto{\pgfqpoint{4.346484in}{2.700116in}}%
\pgfpathlineto{\pgfqpoint{4.346484in}{2.697167in}}%
\pgfpathmoveto{\pgfqpoint{4.337403in}{2.700116in}}%
\pgfpathlineto{\pgfqpoint{4.337403in}{2.700116in}}%
\pgfpathlineto{\pgfqpoint{4.337403in}{2.703065in}}%
\pgfpathlineto{\pgfqpoint{4.341943in}{2.703065in}}%
\pgfpathlineto{\pgfqpoint{4.341943in}{2.700116in}}%
\pgfpathmoveto{\pgfqpoint{4.328321in}{2.706014in}}%
\pgfpathlineto{\pgfqpoint{4.328321in}{2.706014in}}%
\pgfpathlineto{\pgfqpoint{4.328321in}{2.708963in}}%
\pgfpathlineto{\pgfqpoint{4.332862in}{2.708963in}}%
\pgfpathlineto{\pgfqpoint{4.332862in}{2.706014in}}%
\pgfpathmoveto{\pgfqpoint{4.346484in}{2.688319in}}%
\pgfpathlineto{\pgfqpoint{4.346484in}{2.688319in}}%
\pgfpathlineto{\pgfqpoint{4.346484in}{2.691268in}}%
\pgfpathlineto{\pgfqpoint{4.351025in}{2.691268in}}%
\pgfpathlineto{\pgfqpoint{4.351025in}{2.688319in}}%
\pgfpathmoveto{\pgfqpoint{4.346484in}{2.691268in}}%
\pgfpathlineto{\pgfqpoint{4.346484in}{2.691268in}}%
\pgfpathlineto{\pgfqpoint{4.346484in}{2.694217in}}%
\pgfpathlineto{\pgfqpoint{4.351025in}{2.694217in}}%
\pgfpathlineto{\pgfqpoint{4.351025in}{2.691268in}}%
\pgfpathmoveto{\pgfqpoint{4.351025in}{2.688319in}}%
\pgfpathlineto{\pgfqpoint{4.351025in}{2.688319in}}%
\pgfpathlineto{\pgfqpoint{4.351025in}{2.691268in}}%
\pgfpathlineto{\pgfqpoint{4.355566in}{2.691268in}}%
\pgfpathlineto{\pgfqpoint{4.355566in}{2.688319in}}%
\pgfpathmoveto{\pgfqpoint{4.351025in}{2.691268in}}%
\pgfpathlineto{\pgfqpoint{4.351025in}{2.691268in}}%
\pgfpathlineto{\pgfqpoint{4.351025in}{2.694217in}}%
\pgfpathlineto{\pgfqpoint{4.355566in}{2.694217in}}%
\pgfpathlineto{\pgfqpoint{4.355566in}{2.691268in}}%
\pgfpathmoveto{\pgfqpoint{4.355566in}{2.682421in}}%
\pgfpathlineto{\pgfqpoint{4.355566in}{2.682421in}}%
\pgfpathlineto{\pgfqpoint{4.355566in}{2.685370in}}%
\pgfpathlineto{\pgfqpoint{4.360107in}{2.685370in}}%
\pgfpathlineto{\pgfqpoint{4.360107in}{2.682421in}}%
\pgfpathmoveto{\pgfqpoint{4.355566in}{2.685370in}}%
\pgfpathlineto{\pgfqpoint{4.355566in}{2.685370in}}%
\pgfpathlineto{\pgfqpoint{4.355566in}{2.688319in}}%
\pgfpathlineto{\pgfqpoint{4.360107in}{2.688319in}}%
\pgfpathlineto{\pgfqpoint{4.360107in}{2.685370in}}%
\pgfpathmoveto{\pgfqpoint{4.360107in}{2.682421in}}%
\pgfpathlineto{\pgfqpoint{4.360107in}{2.682421in}}%
\pgfpathlineto{\pgfqpoint{4.360107in}{2.685370in}}%
\pgfpathlineto{\pgfqpoint{4.364648in}{2.685370in}}%
\pgfpathlineto{\pgfqpoint{4.364648in}{2.682421in}}%
\pgfpathmoveto{\pgfqpoint{4.360107in}{2.685370in}}%
\pgfpathlineto{\pgfqpoint{4.360107in}{2.685370in}}%
\pgfpathlineto{\pgfqpoint{4.360107in}{2.688319in}}%
\pgfpathlineto{\pgfqpoint{4.364648in}{2.688319in}}%
\pgfpathlineto{\pgfqpoint{4.364648in}{2.685370in}}%
\pgfpathmoveto{\pgfqpoint{4.355566in}{2.688319in}}%
\pgfpathlineto{\pgfqpoint{4.355566in}{2.688319in}}%
\pgfpathlineto{\pgfqpoint{4.355566in}{2.691268in}}%
\pgfpathlineto{\pgfqpoint{4.360107in}{2.691268in}}%
\pgfpathlineto{\pgfqpoint{4.360107in}{2.688319in}}%
\pgfpathmoveto{\pgfqpoint{4.364648in}{2.676522in}}%
\pgfpathlineto{\pgfqpoint{4.364648in}{2.676522in}}%
\pgfpathlineto{\pgfqpoint{4.364648in}{2.679471in}}%
\pgfpathlineto{\pgfqpoint{4.369189in}{2.679471in}}%
\pgfpathlineto{\pgfqpoint{4.369189in}{2.676522in}}%
\pgfpathmoveto{\pgfqpoint{4.364648in}{2.679471in}}%
\pgfpathlineto{\pgfqpoint{4.364648in}{2.679471in}}%
\pgfpathlineto{\pgfqpoint{4.364648in}{2.682421in}}%
\pgfpathlineto{\pgfqpoint{4.369189in}{2.682421in}}%
\pgfpathlineto{\pgfqpoint{4.369189in}{2.679471in}}%
\pgfpathmoveto{\pgfqpoint{4.369189in}{2.676522in}}%
\pgfpathlineto{\pgfqpoint{4.369189in}{2.676522in}}%
\pgfpathlineto{\pgfqpoint{4.369189in}{2.679471in}}%
\pgfpathlineto{\pgfqpoint{4.373730in}{2.679471in}}%
\pgfpathlineto{\pgfqpoint{4.373730in}{2.676522in}}%
\pgfpathmoveto{\pgfqpoint{4.369189in}{2.679471in}}%
\pgfpathlineto{\pgfqpoint{4.369189in}{2.679471in}}%
\pgfpathlineto{\pgfqpoint{4.369189in}{2.682421in}}%
\pgfpathlineto{\pgfqpoint{4.373730in}{2.682421in}}%
\pgfpathlineto{\pgfqpoint{4.373730in}{2.679471in}}%
\pgfpathmoveto{\pgfqpoint{4.373730in}{2.670624in}}%
\pgfpathlineto{\pgfqpoint{4.373730in}{2.670624in}}%
\pgfpathlineto{\pgfqpoint{4.373730in}{2.673573in}}%
\pgfpathlineto{\pgfqpoint{4.378270in}{2.673573in}}%
\pgfpathlineto{\pgfqpoint{4.378270in}{2.670624in}}%
\pgfpathmoveto{\pgfqpoint{4.373730in}{2.673573in}}%
\pgfpathlineto{\pgfqpoint{4.373730in}{2.673573in}}%
\pgfpathlineto{\pgfqpoint{4.373730in}{2.676522in}}%
\pgfpathlineto{\pgfqpoint{4.378270in}{2.676522in}}%
\pgfpathlineto{\pgfqpoint{4.378270in}{2.673573in}}%
\pgfpathmoveto{\pgfqpoint{4.378270in}{2.670624in}}%
\pgfpathlineto{\pgfqpoint{4.378270in}{2.670624in}}%
\pgfpathlineto{\pgfqpoint{4.378270in}{2.673573in}}%
\pgfpathlineto{\pgfqpoint{4.382811in}{2.673573in}}%
\pgfpathlineto{\pgfqpoint{4.382811in}{2.670624in}}%
\pgfpathmoveto{\pgfqpoint{4.378270in}{2.673573in}}%
\pgfpathlineto{\pgfqpoint{4.378270in}{2.673573in}}%
\pgfpathlineto{\pgfqpoint{4.378270in}{2.676522in}}%
\pgfpathlineto{\pgfqpoint{4.382811in}{2.676522in}}%
\pgfpathlineto{\pgfqpoint{4.382811in}{2.673573in}}%
\pgfpathmoveto{\pgfqpoint{4.373730in}{2.676522in}}%
\pgfpathlineto{\pgfqpoint{4.373730in}{2.676522in}}%
\pgfpathlineto{\pgfqpoint{4.373730in}{2.679471in}}%
\pgfpathlineto{\pgfqpoint{4.378270in}{2.679471in}}%
\pgfpathlineto{\pgfqpoint{4.378270in}{2.676522in}}%
\pgfpathmoveto{\pgfqpoint{4.364648in}{2.682421in}}%
\pgfpathlineto{\pgfqpoint{4.364648in}{2.682421in}}%
\pgfpathlineto{\pgfqpoint{4.364648in}{2.685370in}}%
\pgfpathlineto{\pgfqpoint{4.369189in}{2.685370in}}%
\pgfpathlineto{\pgfqpoint{4.369189in}{2.682421in}}%
\pgfpathmoveto{\pgfqpoint{4.346484in}{2.694217in}}%
\pgfpathlineto{\pgfqpoint{4.346484in}{2.694217in}}%
\pgfpathlineto{\pgfqpoint{4.346484in}{2.697167in}}%
\pgfpathlineto{\pgfqpoint{4.351025in}{2.697167in}}%
\pgfpathlineto{\pgfqpoint{4.351025in}{2.694217in}}%
\pgfpathmoveto{\pgfqpoint{4.310157in}{2.717811in}}%
\pgfpathlineto{\pgfqpoint{4.310157in}{2.717811in}}%
\pgfpathlineto{\pgfqpoint{4.310157in}{2.720760in}}%
\pgfpathlineto{\pgfqpoint{4.314698in}{2.720760in}}%
\pgfpathlineto{\pgfqpoint{4.314698in}{2.717811in}}%
\pgfpathmoveto{\pgfqpoint{4.423681in}{2.620486in}}%
\pgfpathlineto{\pgfqpoint{4.423681in}{2.620486in}}%
\pgfpathlineto{\pgfqpoint{4.423681in}{2.623436in}}%
\pgfpathlineto{\pgfqpoint{4.428222in}{2.623436in}}%
\pgfpathlineto{\pgfqpoint{4.428222in}{2.620486in}}%
\pgfpathmoveto{\pgfqpoint{4.428222in}{2.620486in}}%
\pgfpathlineto{\pgfqpoint{4.428222in}{2.620486in}}%
\pgfpathlineto{\pgfqpoint{4.428222in}{2.623436in}}%
\pgfpathlineto{\pgfqpoint{4.432763in}{2.623436in}}%
\pgfpathlineto{\pgfqpoint{4.432763in}{2.620486in}}%
\pgfpathmoveto{\pgfqpoint{4.432763in}{2.620486in}}%
\pgfpathlineto{\pgfqpoint{4.432763in}{2.620486in}}%
\pgfpathlineto{\pgfqpoint{4.432763in}{2.623436in}}%
\pgfpathlineto{\pgfqpoint{4.437304in}{2.623436in}}%
\pgfpathlineto{\pgfqpoint{4.437304in}{2.620486in}}%
\pgfpathmoveto{\pgfqpoint{4.437304in}{2.617537in}}%
\pgfpathlineto{\pgfqpoint{4.437304in}{2.617537in}}%
\pgfpathlineto{\pgfqpoint{4.437304in}{2.620486in}}%
\pgfpathlineto{\pgfqpoint{4.441845in}{2.620486in}}%
\pgfpathlineto{\pgfqpoint{4.441845in}{2.617537in}}%
\pgfpathmoveto{\pgfqpoint{4.437304in}{2.620486in}}%
\pgfpathlineto{\pgfqpoint{4.437304in}{2.620486in}}%
\pgfpathlineto{\pgfqpoint{4.437304in}{2.623436in}}%
\pgfpathlineto{\pgfqpoint{4.441845in}{2.623436in}}%
\pgfpathlineto{\pgfqpoint{4.441845in}{2.620486in}}%
\pgfpathmoveto{\pgfqpoint{4.441845in}{2.617537in}}%
\pgfpathlineto{\pgfqpoint{4.441845in}{2.617537in}}%
\pgfpathlineto{\pgfqpoint{4.441845in}{2.620486in}}%
\pgfpathlineto{\pgfqpoint{4.446386in}{2.620486in}}%
\pgfpathlineto{\pgfqpoint{4.446386in}{2.617537in}}%
\pgfpathmoveto{\pgfqpoint{4.441845in}{2.620486in}}%
\pgfpathlineto{\pgfqpoint{4.441845in}{2.620486in}}%
\pgfpathlineto{\pgfqpoint{4.441845in}{2.623436in}}%
\pgfpathlineto{\pgfqpoint{4.446386in}{2.623436in}}%
\pgfpathlineto{\pgfqpoint{4.446386in}{2.620486in}}%
\pgfpathmoveto{\pgfqpoint{4.450927in}{2.614588in}}%
\pgfpathlineto{\pgfqpoint{4.450927in}{2.614588in}}%
\pgfpathlineto{\pgfqpoint{4.450927in}{2.617537in}}%
\pgfpathlineto{\pgfqpoint{4.455468in}{2.617537in}}%
\pgfpathlineto{\pgfqpoint{4.455468in}{2.614588in}}%
\pgfpathmoveto{\pgfqpoint{4.446386in}{2.617537in}}%
\pgfpathlineto{\pgfqpoint{4.446386in}{2.617537in}}%
\pgfpathlineto{\pgfqpoint{4.446386in}{2.620486in}}%
\pgfpathlineto{\pgfqpoint{4.450927in}{2.620486in}}%
\pgfpathlineto{\pgfqpoint{4.450927in}{2.617537in}}%
\pgfpathmoveto{\pgfqpoint{4.446386in}{2.620486in}}%
\pgfpathlineto{\pgfqpoint{4.446386in}{2.620486in}}%
\pgfpathlineto{\pgfqpoint{4.446386in}{2.623436in}}%
\pgfpathlineto{\pgfqpoint{4.450927in}{2.623436in}}%
\pgfpathlineto{\pgfqpoint{4.450927in}{2.620486in}}%
\pgfpathmoveto{\pgfqpoint{4.450927in}{2.617537in}}%
\pgfpathlineto{\pgfqpoint{4.450927in}{2.617537in}}%
\pgfpathlineto{\pgfqpoint{4.450927in}{2.620486in}}%
\pgfpathlineto{\pgfqpoint{4.455468in}{2.620486in}}%
\pgfpathlineto{\pgfqpoint{4.455468in}{2.617537in}}%
\pgfpathmoveto{\pgfqpoint{4.450927in}{2.620486in}}%
\pgfpathlineto{\pgfqpoint{4.450927in}{2.620486in}}%
\pgfpathlineto{\pgfqpoint{4.450927in}{2.623436in}}%
\pgfpathlineto{\pgfqpoint{4.455468in}{2.623436in}}%
\pgfpathlineto{\pgfqpoint{4.455468in}{2.620486in}}%
\pgfpathmoveto{\pgfqpoint{4.382811in}{2.629334in}}%
\pgfpathlineto{\pgfqpoint{4.382811in}{2.629334in}}%
\pgfpathlineto{\pgfqpoint{4.382811in}{2.632283in}}%
\pgfpathlineto{\pgfqpoint{4.387352in}{2.632283in}}%
\pgfpathlineto{\pgfqpoint{4.387352in}{2.629334in}}%
\pgfpathmoveto{\pgfqpoint{4.382811in}{2.632283in}}%
\pgfpathlineto{\pgfqpoint{4.382811in}{2.632283in}}%
\pgfpathlineto{\pgfqpoint{4.382811in}{2.635233in}}%
\pgfpathlineto{\pgfqpoint{4.387352in}{2.635233in}}%
\pgfpathlineto{\pgfqpoint{4.387352in}{2.632283in}}%
\pgfpathmoveto{\pgfqpoint{4.387352in}{2.629334in}}%
\pgfpathlineto{\pgfqpoint{4.387352in}{2.629334in}}%
\pgfpathlineto{\pgfqpoint{4.387352in}{2.632283in}}%
\pgfpathlineto{\pgfqpoint{4.391893in}{2.632283in}}%
\pgfpathlineto{\pgfqpoint{4.391893in}{2.629334in}}%
\pgfpathmoveto{\pgfqpoint{4.387352in}{2.632283in}}%
\pgfpathlineto{\pgfqpoint{4.387352in}{2.632283in}}%
\pgfpathlineto{\pgfqpoint{4.387352in}{2.635233in}}%
\pgfpathlineto{\pgfqpoint{4.391893in}{2.635233in}}%
\pgfpathlineto{\pgfqpoint{4.391893in}{2.632283in}}%
\pgfpathmoveto{\pgfqpoint{4.396435in}{2.626385in}}%
\pgfpathlineto{\pgfqpoint{4.396435in}{2.626385in}}%
\pgfpathlineto{\pgfqpoint{4.396435in}{2.629334in}}%
\pgfpathlineto{\pgfqpoint{4.400976in}{2.629334in}}%
\pgfpathlineto{\pgfqpoint{4.400976in}{2.626385in}}%
\pgfpathmoveto{\pgfqpoint{4.391893in}{2.629334in}}%
\pgfpathlineto{\pgfqpoint{4.391893in}{2.629334in}}%
\pgfpathlineto{\pgfqpoint{4.391893in}{2.632283in}}%
\pgfpathlineto{\pgfqpoint{4.396435in}{2.632283in}}%
\pgfpathlineto{\pgfqpoint{4.396435in}{2.629334in}}%
\pgfpathmoveto{\pgfqpoint{4.391893in}{2.632283in}}%
\pgfpathlineto{\pgfqpoint{4.391893in}{2.632283in}}%
\pgfpathlineto{\pgfqpoint{4.391893in}{2.635233in}}%
\pgfpathlineto{\pgfqpoint{4.396435in}{2.635233in}}%
\pgfpathlineto{\pgfqpoint{4.396435in}{2.632283in}}%
\pgfpathmoveto{\pgfqpoint{4.396435in}{2.629334in}}%
\pgfpathlineto{\pgfqpoint{4.396435in}{2.629334in}}%
\pgfpathlineto{\pgfqpoint{4.396435in}{2.632283in}}%
\pgfpathlineto{\pgfqpoint{4.400976in}{2.632283in}}%
\pgfpathlineto{\pgfqpoint{4.400976in}{2.629334in}}%
\pgfpathmoveto{\pgfqpoint{4.396435in}{2.632283in}}%
\pgfpathlineto{\pgfqpoint{4.396435in}{2.632283in}}%
\pgfpathlineto{\pgfqpoint{4.396435in}{2.635233in}}%
\pgfpathlineto{\pgfqpoint{4.400976in}{2.635233in}}%
\pgfpathlineto{\pgfqpoint{4.400976in}{2.632283in}}%
\pgfpathmoveto{\pgfqpoint{4.400976in}{2.626385in}}%
\pgfpathlineto{\pgfqpoint{4.400976in}{2.626385in}}%
\pgfpathlineto{\pgfqpoint{4.400976in}{2.629334in}}%
\pgfpathlineto{\pgfqpoint{4.405517in}{2.629334in}}%
\pgfpathlineto{\pgfqpoint{4.405517in}{2.626385in}}%
\pgfpathmoveto{\pgfqpoint{4.405517in}{2.626385in}}%
\pgfpathlineto{\pgfqpoint{4.405517in}{2.626385in}}%
\pgfpathlineto{\pgfqpoint{4.405517in}{2.629334in}}%
\pgfpathlineto{\pgfqpoint{4.410058in}{2.629334in}}%
\pgfpathlineto{\pgfqpoint{4.410058in}{2.626385in}}%
\pgfpathmoveto{\pgfqpoint{4.410058in}{2.623436in}}%
\pgfpathlineto{\pgfqpoint{4.410058in}{2.623436in}}%
\pgfpathlineto{\pgfqpoint{4.410058in}{2.626385in}}%
\pgfpathlineto{\pgfqpoint{4.414599in}{2.626385in}}%
\pgfpathlineto{\pgfqpoint{4.414599in}{2.623436in}}%
\pgfpathmoveto{\pgfqpoint{4.410058in}{2.626385in}}%
\pgfpathlineto{\pgfqpoint{4.410058in}{2.626385in}}%
\pgfpathlineto{\pgfqpoint{4.410058in}{2.629334in}}%
\pgfpathlineto{\pgfqpoint{4.414599in}{2.629334in}}%
\pgfpathlineto{\pgfqpoint{4.414599in}{2.626385in}}%
\pgfpathmoveto{\pgfqpoint{4.414599in}{2.623436in}}%
\pgfpathlineto{\pgfqpoint{4.414599in}{2.623436in}}%
\pgfpathlineto{\pgfqpoint{4.414599in}{2.626385in}}%
\pgfpathlineto{\pgfqpoint{4.419140in}{2.626385in}}%
\pgfpathlineto{\pgfqpoint{4.419140in}{2.623436in}}%
\pgfpathmoveto{\pgfqpoint{4.414599in}{2.626385in}}%
\pgfpathlineto{\pgfqpoint{4.414599in}{2.626385in}}%
\pgfpathlineto{\pgfqpoint{4.414599in}{2.629334in}}%
\pgfpathlineto{\pgfqpoint{4.419140in}{2.629334in}}%
\pgfpathlineto{\pgfqpoint{4.419140in}{2.626385in}}%
\pgfpathmoveto{\pgfqpoint{4.382811in}{2.664725in}}%
\pgfpathlineto{\pgfqpoint{4.382811in}{2.664725in}}%
\pgfpathlineto{\pgfqpoint{4.382811in}{2.667675in}}%
\pgfpathlineto{\pgfqpoint{4.387352in}{2.667675in}}%
\pgfpathlineto{\pgfqpoint{4.387352in}{2.664725in}}%
\pgfpathmoveto{\pgfqpoint{4.382811in}{2.667675in}}%
\pgfpathlineto{\pgfqpoint{4.382811in}{2.667675in}}%
\pgfpathlineto{\pgfqpoint{4.382811in}{2.670624in}}%
\pgfpathlineto{\pgfqpoint{4.387352in}{2.670624in}}%
\pgfpathlineto{\pgfqpoint{4.387352in}{2.667675in}}%
\pgfpathmoveto{\pgfqpoint{4.387352in}{2.664725in}}%
\pgfpathlineto{\pgfqpoint{4.387352in}{2.664725in}}%
\pgfpathlineto{\pgfqpoint{4.387352in}{2.667675in}}%
\pgfpathlineto{\pgfqpoint{4.391893in}{2.667675in}}%
\pgfpathlineto{\pgfqpoint{4.391893in}{2.664725in}}%
\pgfpathmoveto{\pgfqpoint{4.387352in}{2.667675in}}%
\pgfpathlineto{\pgfqpoint{4.387352in}{2.667675in}}%
\pgfpathlineto{\pgfqpoint{4.387352in}{2.670624in}}%
\pgfpathlineto{\pgfqpoint{4.391893in}{2.670624in}}%
\pgfpathlineto{\pgfqpoint{4.391893in}{2.667675in}}%
\pgfpathmoveto{\pgfqpoint{4.391893in}{2.658827in}}%
\pgfpathlineto{\pgfqpoint{4.391893in}{2.658827in}}%
\pgfpathlineto{\pgfqpoint{4.391893in}{2.661776in}}%
\pgfpathlineto{\pgfqpoint{4.396435in}{2.661776in}}%
\pgfpathlineto{\pgfqpoint{4.396435in}{2.658827in}}%
\pgfpathmoveto{\pgfqpoint{4.391893in}{2.661776in}}%
\pgfpathlineto{\pgfqpoint{4.391893in}{2.661776in}}%
\pgfpathlineto{\pgfqpoint{4.391893in}{2.664725in}}%
\pgfpathlineto{\pgfqpoint{4.396435in}{2.664725in}}%
\pgfpathlineto{\pgfqpoint{4.396435in}{2.661776in}}%
\pgfpathmoveto{\pgfqpoint{4.396435in}{2.658827in}}%
\pgfpathlineto{\pgfqpoint{4.396435in}{2.658827in}}%
\pgfpathlineto{\pgfqpoint{4.396435in}{2.661776in}}%
\pgfpathlineto{\pgfqpoint{4.400976in}{2.661776in}}%
\pgfpathlineto{\pgfqpoint{4.400976in}{2.658827in}}%
\pgfpathmoveto{\pgfqpoint{4.396435in}{2.661776in}}%
\pgfpathlineto{\pgfqpoint{4.396435in}{2.661776in}}%
\pgfpathlineto{\pgfqpoint{4.396435in}{2.664725in}}%
\pgfpathlineto{\pgfqpoint{4.400976in}{2.664725in}}%
\pgfpathlineto{\pgfqpoint{4.400976in}{2.661776in}}%
\pgfpathmoveto{\pgfqpoint{4.391893in}{2.664725in}}%
\pgfpathlineto{\pgfqpoint{4.391893in}{2.664725in}}%
\pgfpathlineto{\pgfqpoint{4.391893in}{2.667675in}}%
\pgfpathlineto{\pgfqpoint{4.396435in}{2.667675in}}%
\pgfpathlineto{\pgfqpoint{4.396435in}{2.664725in}}%
\pgfpathmoveto{\pgfqpoint{4.400976in}{2.652928in}}%
\pgfpathlineto{\pgfqpoint{4.400976in}{2.652928in}}%
\pgfpathlineto{\pgfqpoint{4.400976in}{2.655877in}}%
\pgfpathlineto{\pgfqpoint{4.405517in}{2.655877in}}%
\pgfpathlineto{\pgfqpoint{4.405517in}{2.652928in}}%
\pgfpathmoveto{\pgfqpoint{4.400976in}{2.655877in}}%
\pgfpathlineto{\pgfqpoint{4.400976in}{2.655877in}}%
\pgfpathlineto{\pgfqpoint{4.400976in}{2.658827in}}%
\pgfpathlineto{\pgfqpoint{4.405517in}{2.658827in}}%
\pgfpathlineto{\pgfqpoint{4.405517in}{2.655877in}}%
\pgfpathmoveto{\pgfqpoint{4.405517in}{2.652928in}}%
\pgfpathlineto{\pgfqpoint{4.405517in}{2.652928in}}%
\pgfpathlineto{\pgfqpoint{4.405517in}{2.655877in}}%
\pgfpathlineto{\pgfqpoint{4.410058in}{2.655877in}}%
\pgfpathlineto{\pgfqpoint{4.410058in}{2.652928in}}%
\pgfpathmoveto{\pgfqpoint{4.405517in}{2.655877in}}%
\pgfpathlineto{\pgfqpoint{4.405517in}{2.655877in}}%
\pgfpathlineto{\pgfqpoint{4.405517in}{2.658827in}}%
\pgfpathlineto{\pgfqpoint{4.410058in}{2.658827in}}%
\pgfpathlineto{\pgfqpoint{4.410058in}{2.655877in}}%
\pgfpathmoveto{\pgfqpoint{4.410058in}{2.647030in}}%
\pgfpathlineto{\pgfqpoint{4.410058in}{2.647030in}}%
\pgfpathlineto{\pgfqpoint{4.410058in}{2.649979in}}%
\pgfpathlineto{\pgfqpoint{4.414599in}{2.649979in}}%
\pgfpathlineto{\pgfqpoint{4.414599in}{2.647030in}}%
\pgfpathmoveto{\pgfqpoint{4.410058in}{2.649979in}}%
\pgfpathlineto{\pgfqpoint{4.410058in}{2.649979in}}%
\pgfpathlineto{\pgfqpoint{4.410058in}{2.652928in}}%
\pgfpathlineto{\pgfqpoint{4.414599in}{2.652928in}}%
\pgfpathlineto{\pgfqpoint{4.414599in}{2.649979in}}%
\pgfpathmoveto{\pgfqpoint{4.414599in}{2.647030in}}%
\pgfpathlineto{\pgfqpoint{4.414599in}{2.647030in}}%
\pgfpathlineto{\pgfqpoint{4.414599in}{2.649979in}}%
\pgfpathlineto{\pgfqpoint{4.419140in}{2.649979in}}%
\pgfpathlineto{\pgfqpoint{4.419140in}{2.647030in}}%
\pgfpathmoveto{\pgfqpoint{4.414599in}{2.649979in}}%
\pgfpathlineto{\pgfqpoint{4.414599in}{2.649979in}}%
\pgfpathlineto{\pgfqpoint{4.414599in}{2.652928in}}%
\pgfpathlineto{\pgfqpoint{4.419140in}{2.652928in}}%
\pgfpathlineto{\pgfqpoint{4.419140in}{2.649979in}}%
\pgfpathmoveto{\pgfqpoint{4.410058in}{2.652928in}}%
\pgfpathlineto{\pgfqpoint{4.410058in}{2.652928in}}%
\pgfpathlineto{\pgfqpoint{4.410058in}{2.655877in}}%
\pgfpathlineto{\pgfqpoint{4.414599in}{2.655877in}}%
\pgfpathlineto{\pgfqpoint{4.414599in}{2.652928in}}%
\pgfpathmoveto{\pgfqpoint{4.400976in}{2.658827in}}%
\pgfpathlineto{\pgfqpoint{4.400976in}{2.658827in}}%
\pgfpathlineto{\pgfqpoint{4.400976in}{2.661776in}}%
\pgfpathlineto{\pgfqpoint{4.405517in}{2.661776in}}%
\pgfpathlineto{\pgfqpoint{4.405517in}{2.658827in}}%
\pgfpathmoveto{\pgfqpoint{4.419140in}{2.623436in}}%
\pgfpathlineto{\pgfqpoint{4.419140in}{2.623436in}}%
\pgfpathlineto{\pgfqpoint{4.419140in}{2.626385in}}%
\pgfpathlineto{\pgfqpoint{4.423681in}{2.626385in}}%
\pgfpathlineto{\pgfqpoint{4.423681in}{2.623436in}}%
\pgfpathmoveto{\pgfqpoint{4.419140in}{2.626385in}}%
\pgfpathlineto{\pgfqpoint{4.419140in}{2.626385in}}%
\pgfpathlineto{\pgfqpoint{4.419140in}{2.629334in}}%
\pgfpathlineto{\pgfqpoint{4.423681in}{2.629334in}}%
\pgfpathlineto{\pgfqpoint{4.423681in}{2.626385in}}%
\pgfpathmoveto{\pgfqpoint{4.423681in}{2.623436in}}%
\pgfpathlineto{\pgfqpoint{4.423681in}{2.623436in}}%
\pgfpathlineto{\pgfqpoint{4.423681in}{2.626385in}}%
\pgfpathlineto{\pgfqpoint{4.428222in}{2.626385in}}%
\pgfpathlineto{\pgfqpoint{4.428222in}{2.623436in}}%
\pgfpathmoveto{\pgfqpoint{4.423681in}{2.626385in}}%
\pgfpathlineto{\pgfqpoint{4.423681in}{2.626385in}}%
\pgfpathlineto{\pgfqpoint{4.423681in}{2.629334in}}%
\pgfpathlineto{\pgfqpoint{4.428222in}{2.629334in}}%
\pgfpathlineto{\pgfqpoint{4.428222in}{2.626385in}}%
\pgfpathmoveto{\pgfqpoint{4.419140in}{2.641131in}}%
\pgfpathlineto{\pgfqpoint{4.419140in}{2.641131in}}%
\pgfpathlineto{\pgfqpoint{4.419140in}{2.644080in}}%
\pgfpathlineto{\pgfqpoint{4.423681in}{2.644080in}}%
\pgfpathlineto{\pgfqpoint{4.423681in}{2.641131in}}%
\pgfpathmoveto{\pgfqpoint{4.419140in}{2.644080in}}%
\pgfpathlineto{\pgfqpoint{4.419140in}{2.644080in}}%
\pgfpathlineto{\pgfqpoint{4.419140in}{2.647030in}}%
\pgfpathlineto{\pgfqpoint{4.423681in}{2.647030in}}%
\pgfpathlineto{\pgfqpoint{4.423681in}{2.644080in}}%
\pgfpathmoveto{\pgfqpoint{4.423681in}{2.641131in}}%
\pgfpathlineto{\pgfqpoint{4.423681in}{2.641131in}}%
\pgfpathlineto{\pgfqpoint{4.423681in}{2.644080in}}%
\pgfpathlineto{\pgfqpoint{4.428222in}{2.644080in}}%
\pgfpathlineto{\pgfqpoint{4.428222in}{2.641131in}}%
\pgfpathmoveto{\pgfqpoint{4.423681in}{2.644080in}}%
\pgfpathlineto{\pgfqpoint{4.423681in}{2.644080in}}%
\pgfpathlineto{\pgfqpoint{4.423681in}{2.647030in}}%
\pgfpathlineto{\pgfqpoint{4.428222in}{2.647030in}}%
\pgfpathlineto{\pgfqpoint{4.428222in}{2.644080in}}%
\pgfpathmoveto{\pgfqpoint{4.428222in}{2.635233in}}%
\pgfpathlineto{\pgfqpoint{4.428222in}{2.635233in}}%
\pgfpathlineto{\pgfqpoint{4.428222in}{2.638182in}}%
\pgfpathlineto{\pgfqpoint{4.432763in}{2.638182in}}%
\pgfpathlineto{\pgfqpoint{4.432763in}{2.635233in}}%
\pgfpathmoveto{\pgfqpoint{4.428222in}{2.638182in}}%
\pgfpathlineto{\pgfqpoint{4.428222in}{2.638182in}}%
\pgfpathlineto{\pgfqpoint{4.428222in}{2.641131in}}%
\pgfpathlineto{\pgfqpoint{4.432763in}{2.641131in}}%
\pgfpathlineto{\pgfqpoint{4.432763in}{2.638182in}}%
\pgfpathmoveto{\pgfqpoint{4.432763in}{2.635233in}}%
\pgfpathlineto{\pgfqpoint{4.432763in}{2.635233in}}%
\pgfpathlineto{\pgfqpoint{4.432763in}{2.638182in}}%
\pgfpathlineto{\pgfqpoint{4.437304in}{2.638182in}}%
\pgfpathlineto{\pgfqpoint{4.437304in}{2.635233in}}%
\pgfpathmoveto{\pgfqpoint{4.432763in}{2.638182in}}%
\pgfpathlineto{\pgfqpoint{4.432763in}{2.638182in}}%
\pgfpathlineto{\pgfqpoint{4.432763in}{2.641131in}}%
\pgfpathlineto{\pgfqpoint{4.437304in}{2.641131in}}%
\pgfpathlineto{\pgfqpoint{4.437304in}{2.638182in}}%
\pgfpathmoveto{\pgfqpoint{4.428222in}{2.641131in}}%
\pgfpathlineto{\pgfqpoint{4.428222in}{2.641131in}}%
\pgfpathlineto{\pgfqpoint{4.428222in}{2.644080in}}%
\pgfpathlineto{\pgfqpoint{4.432763in}{2.644080in}}%
\pgfpathlineto{\pgfqpoint{4.432763in}{2.641131in}}%
\pgfpathmoveto{\pgfqpoint{4.437304in}{2.629334in}}%
\pgfpathlineto{\pgfqpoint{4.437304in}{2.629334in}}%
\pgfpathlineto{\pgfqpoint{4.437304in}{2.632283in}}%
\pgfpathlineto{\pgfqpoint{4.441845in}{2.632283in}}%
\pgfpathlineto{\pgfqpoint{4.441845in}{2.629334in}}%
\pgfpathmoveto{\pgfqpoint{4.437304in}{2.632283in}}%
\pgfpathlineto{\pgfqpoint{4.437304in}{2.632283in}}%
\pgfpathlineto{\pgfqpoint{4.437304in}{2.635233in}}%
\pgfpathlineto{\pgfqpoint{4.441845in}{2.635233in}}%
\pgfpathlineto{\pgfqpoint{4.441845in}{2.632283in}}%
\pgfpathmoveto{\pgfqpoint{4.441845in}{2.629334in}}%
\pgfpathlineto{\pgfqpoint{4.441845in}{2.629334in}}%
\pgfpathlineto{\pgfqpoint{4.441845in}{2.632283in}}%
\pgfpathlineto{\pgfqpoint{4.446386in}{2.632283in}}%
\pgfpathlineto{\pgfqpoint{4.446386in}{2.629334in}}%
\pgfpathmoveto{\pgfqpoint{4.441845in}{2.632283in}}%
\pgfpathlineto{\pgfqpoint{4.441845in}{2.632283in}}%
\pgfpathlineto{\pgfqpoint{4.441845in}{2.635233in}}%
\pgfpathlineto{\pgfqpoint{4.446386in}{2.635233in}}%
\pgfpathlineto{\pgfqpoint{4.446386in}{2.632283in}}%
\pgfpathmoveto{\pgfqpoint{4.446386in}{2.623436in}}%
\pgfpathlineto{\pgfqpoint{4.446386in}{2.623436in}}%
\pgfpathlineto{\pgfqpoint{4.446386in}{2.626385in}}%
\pgfpathlineto{\pgfqpoint{4.450927in}{2.626385in}}%
\pgfpathlineto{\pgfqpoint{4.450927in}{2.623436in}}%
\pgfpathmoveto{\pgfqpoint{4.446386in}{2.626385in}}%
\pgfpathlineto{\pgfqpoint{4.446386in}{2.626385in}}%
\pgfpathlineto{\pgfqpoint{4.446386in}{2.629334in}}%
\pgfpathlineto{\pgfqpoint{4.450927in}{2.629334in}}%
\pgfpathlineto{\pgfqpoint{4.450927in}{2.626385in}}%
\pgfpathmoveto{\pgfqpoint{4.450927in}{2.623436in}}%
\pgfpathlineto{\pgfqpoint{4.450927in}{2.623436in}}%
\pgfpathlineto{\pgfqpoint{4.450927in}{2.626385in}}%
\pgfpathlineto{\pgfqpoint{4.455468in}{2.626385in}}%
\pgfpathlineto{\pgfqpoint{4.455468in}{2.623436in}}%
\pgfpathmoveto{\pgfqpoint{4.450927in}{2.626385in}}%
\pgfpathlineto{\pgfqpoint{4.450927in}{2.626385in}}%
\pgfpathlineto{\pgfqpoint{4.450927in}{2.629334in}}%
\pgfpathlineto{\pgfqpoint{4.455468in}{2.629334in}}%
\pgfpathlineto{\pgfqpoint{4.455468in}{2.626385in}}%
\pgfpathmoveto{\pgfqpoint{4.446386in}{2.629334in}}%
\pgfpathlineto{\pgfqpoint{4.446386in}{2.629334in}}%
\pgfpathlineto{\pgfqpoint{4.446386in}{2.632283in}}%
\pgfpathlineto{\pgfqpoint{4.450927in}{2.632283in}}%
\pgfpathlineto{\pgfqpoint{4.450927in}{2.629334in}}%
\pgfpathmoveto{\pgfqpoint{4.437304in}{2.635233in}}%
\pgfpathlineto{\pgfqpoint{4.437304in}{2.635233in}}%
\pgfpathlineto{\pgfqpoint{4.437304in}{2.638182in}}%
\pgfpathlineto{\pgfqpoint{4.441845in}{2.638182in}}%
\pgfpathlineto{\pgfqpoint{4.441845in}{2.635233in}}%
\pgfpathmoveto{\pgfqpoint{4.419140in}{2.647030in}}%
\pgfpathlineto{\pgfqpoint{4.419140in}{2.647030in}}%
\pgfpathlineto{\pgfqpoint{4.419140in}{2.649979in}}%
\pgfpathlineto{\pgfqpoint{4.423681in}{2.649979in}}%
\pgfpathlineto{\pgfqpoint{4.423681in}{2.647030in}}%
\pgfpathmoveto{\pgfqpoint{4.455468in}{2.614588in}}%
\pgfpathlineto{\pgfqpoint{4.455468in}{2.614588in}}%
\pgfpathlineto{\pgfqpoint{4.455468in}{2.617537in}}%
\pgfpathlineto{\pgfqpoint{4.460009in}{2.617537in}}%
\pgfpathlineto{\pgfqpoint{4.460009in}{2.614588in}}%
\pgfpathmoveto{\pgfqpoint{4.460009in}{2.614588in}}%
\pgfpathlineto{\pgfqpoint{4.460009in}{2.614588in}}%
\pgfpathlineto{\pgfqpoint{4.460009in}{2.617537in}}%
\pgfpathlineto{\pgfqpoint{4.464550in}{2.617537in}}%
\pgfpathlineto{\pgfqpoint{4.464550in}{2.614588in}}%
\pgfpathmoveto{\pgfqpoint{4.455468in}{2.617537in}}%
\pgfpathlineto{\pgfqpoint{4.455468in}{2.617537in}}%
\pgfpathlineto{\pgfqpoint{4.455468in}{2.620486in}}%
\pgfpathlineto{\pgfqpoint{4.460009in}{2.620486in}}%
\pgfpathlineto{\pgfqpoint{4.460009in}{2.617537in}}%
\pgfpathmoveto{\pgfqpoint{4.455468in}{2.620486in}}%
\pgfpathlineto{\pgfqpoint{4.455468in}{2.620486in}}%
\pgfpathlineto{\pgfqpoint{4.455468in}{2.623436in}}%
\pgfpathlineto{\pgfqpoint{4.460009in}{2.623436in}}%
\pgfpathlineto{\pgfqpoint{4.460009in}{2.620486in}}%
\pgfpathmoveto{\pgfqpoint{4.460009in}{2.617537in}}%
\pgfpathlineto{\pgfqpoint{4.460009in}{2.617537in}}%
\pgfpathlineto{\pgfqpoint{4.460009in}{2.620486in}}%
\pgfpathlineto{\pgfqpoint{4.464550in}{2.620486in}}%
\pgfpathlineto{\pgfqpoint{4.464550in}{2.617537in}}%
\pgfpathmoveto{\pgfqpoint{4.460009in}{2.620486in}}%
\pgfpathlineto{\pgfqpoint{4.460009in}{2.620486in}}%
\pgfpathlineto{\pgfqpoint{4.460009in}{2.623436in}}%
\pgfpathlineto{\pgfqpoint{4.464550in}{2.623436in}}%
\pgfpathlineto{\pgfqpoint{4.464550in}{2.620486in}}%
\pgfpathmoveto{\pgfqpoint{4.464550in}{2.611639in}}%
\pgfpathlineto{\pgfqpoint{4.464550in}{2.611639in}}%
\pgfpathlineto{\pgfqpoint{4.464550in}{2.614588in}}%
\pgfpathlineto{\pgfqpoint{4.469092in}{2.614588in}}%
\pgfpathlineto{\pgfqpoint{4.469092in}{2.611639in}}%
\pgfpathmoveto{\pgfqpoint{4.464550in}{2.614588in}}%
\pgfpathlineto{\pgfqpoint{4.464550in}{2.614588in}}%
\pgfpathlineto{\pgfqpoint{4.464550in}{2.617537in}}%
\pgfpathlineto{\pgfqpoint{4.469092in}{2.617537in}}%
\pgfpathlineto{\pgfqpoint{4.469092in}{2.614588in}}%
\pgfpathmoveto{\pgfqpoint{4.469092in}{2.611639in}}%
\pgfpathlineto{\pgfqpoint{4.469092in}{2.611639in}}%
\pgfpathlineto{\pgfqpoint{4.469092in}{2.614588in}}%
\pgfpathlineto{\pgfqpoint{4.473633in}{2.614588in}}%
\pgfpathlineto{\pgfqpoint{4.473633in}{2.611639in}}%
\pgfpathmoveto{\pgfqpoint{4.469092in}{2.614588in}}%
\pgfpathlineto{\pgfqpoint{4.469092in}{2.614588in}}%
\pgfpathlineto{\pgfqpoint{4.469092in}{2.617537in}}%
\pgfpathlineto{\pgfqpoint{4.473633in}{2.617537in}}%
\pgfpathlineto{\pgfqpoint{4.473633in}{2.614588in}}%
\pgfpathmoveto{\pgfqpoint{4.464550in}{2.617537in}}%
\pgfpathlineto{\pgfqpoint{4.464550in}{2.617537in}}%
\pgfpathlineto{\pgfqpoint{4.464550in}{2.620486in}}%
\pgfpathlineto{\pgfqpoint{4.469092in}{2.620486in}}%
\pgfpathlineto{\pgfqpoint{4.469092in}{2.617537in}}%
\pgfpathmoveto{\pgfqpoint{4.478174in}{2.608689in}}%
\pgfpathlineto{\pgfqpoint{4.478174in}{2.608689in}}%
\pgfpathlineto{\pgfqpoint{4.478174in}{2.611639in}}%
\pgfpathlineto{\pgfqpoint{4.482715in}{2.611639in}}%
\pgfpathlineto{\pgfqpoint{4.482715in}{2.608689in}}%
\pgfpathmoveto{\pgfqpoint{4.473633in}{2.611639in}}%
\pgfpathlineto{\pgfqpoint{4.473633in}{2.611639in}}%
\pgfpathlineto{\pgfqpoint{4.473633in}{2.614588in}}%
\pgfpathlineto{\pgfqpoint{4.478174in}{2.614588in}}%
\pgfpathlineto{\pgfqpoint{4.478174in}{2.611639in}}%
\pgfpathmoveto{\pgfqpoint{4.455468in}{2.623436in}}%
\pgfpathlineto{\pgfqpoint{4.455468in}{2.623436in}}%
\pgfpathlineto{\pgfqpoint{4.455468in}{2.626385in}}%
\pgfpathlineto{\pgfqpoint{4.460009in}{2.626385in}}%
\pgfpathlineto{\pgfqpoint{4.460009in}{2.623436in}}%
\pgfpathmoveto{\pgfqpoint{4.382811in}{2.670624in}}%
\pgfpathlineto{\pgfqpoint{4.382811in}{2.670624in}}%
\pgfpathlineto{\pgfqpoint{4.382811in}{2.673573in}}%
\pgfpathlineto{\pgfqpoint{4.387352in}{2.673573in}}%
\pgfpathlineto{\pgfqpoint{4.387352in}{2.670624in}}%
\pgfpathclose%
\pgfusepath{fill}%
\end{pgfscope}%
\begin{pgfscope}%
\pgfpathrectangle{\pgfqpoint{0.750000in}{0.500000in}}{\pgfqpoint{4.650000in}{3.020000in}}%
\pgfusepath{clip}%
\pgfsetbuttcap%
\pgfsetmiterjoin%
\definecolor{currentfill}{rgb}{1.000000,0.000000,0.000000}%
\pgfsetfillcolor{currentfill}%
\pgfsetlinewidth{0.000000pt}%
\definecolor{currentstroke}{rgb}{0.000000,0.000000,0.000000}%
\pgfsetstrokecolor{currentstroke}%
\pgfsetstrokeopacity{0.000000}%
\pgfsetdash{}{0pt}%
\pgfpathmoveto{\pgfqpoint{3.070462in}{3.517052in}}%
\pgfpathlineto{\pgfqpoint{3.070462in}{3.520002in}}%
\pgfpathlineto{\pgfqpoint{3.075003in}{3.520002in}}%
\pgfpathlineto{\pgfqpoint{3.075003in}{3.517052in}}%
\pgfpathmoveto{\pgfqpoint{3.143116in}{3.469864in}}%
\pgfpathlineto{\pgfqpoint{3.143116in}{3.469864in}}%
\pgfpathlineto{\pgfqpoint{3.143116in}{3.472813in}}%
\pgfpathlineto{\pgfqpoint{3.147657in}{3.472813in}}%
\pgfpathlineto{\pgfqpoint{3.147657in}{3.469864in}}%
\pgfpathmoveto{\pgfqpoint{3.106789in}{3.493458in}}%
\pgfpathlineto{\pgfqpoint{3.106789in}{3.493458in}}%
\pgfpathlineto{\pgfqpoint{3.106789in}{3.496407in}}%
\pgfpathlineto{\pgfqpoint{3.111330in}{3.496407in}}%
\pgfpathlineto{\pgfqpoint{3.111330in}{3.493458in}}%
\pgfpathmoveto{\pgfqpoint{3.088626in}{3.505255in}}%
\pgfpathlineto{\pgfqpoint{3.088626in}{3.505255in}}%
\pgfpathlineto{\pgfqpoint{3.088626in}{3.508205in}}%
\pgfpathlineto{\pgfqpoint{3.093166in}{3.508205in}}%
\pgfpathlineto{\pgfqpoint{3.093166in}{3.505255in}}%
\pgfpathmoveto{\pgfqpoint{3.079544in}{3.511154in}}%
\pgfpathlineto{\pgfqpoint{3.079544in}{3.511154in}}%
\pgfpathlineto{\pgfqpoint{3.079544in}{3.514103in}}%
\pgfpathlineto{\pgfqpoint{3.084085in}{3.514103in}}%
\pgfpathlineto{\pgfqpoint{3.084085in}{3.511154in}}%
\pgfpathmoveto{\pgfqpoint{3.075003in}{3.514103in}}%
\pgfpathlineto{\pgfqpoint{3.075003in}{3.514103in}}%
\pgfpathlineto{\pgfqpoint{3.075003in}{3.517052in}}%
\pgfpathlineto{\pgfqpoint{3.079544in}{3.517052in}}%
\pgfpathlineto{\pgfqpoint{3.079544in}{3.514103in}}%
\pgfpathmoveto{\pgfqpoint{3.075003in}{3.517052in}}%
\pgfpathlineto{\pgfqpoint{3.075003in}{3.517052in}}%
\pgfpathlineto{\pgfqpoint{3.075003in}{3.520002in}}%
\pgfpathlineto{\pgfqpoint{3.079544in}{3.520002in}}%
\pgfpathlineto{\pgfqpoint{3.079544in}{3.517052in}}%
\pgfpathmoveto{\pgfqpoint{3.079544in}{3.514103in}}%
\pgfpathlineto{\pgfqpoint{3.079544in}{3.514103in}}%
\pgfpathlineto{\pgfqpoint{3.079544in}{3.517052in}}%
\pgfpathlineto{\pgfqpoint{3.084085in}{3.517052in}}%
\pgfpathlineto{\pgfqpoint{3.084085in}{3.514103in}}%
\pgfpathmoveto{\pgfqpoint{3.084085in}{3.508205in}}%
\pgfpathlineto{\pgfqpoint{3.084085in}{3.508205in}}%
\pgfpathlineto{\pgfqpoint{3.084085in}{3.511154in}}%
\pgfpathlineto{\pgfqpoint{3.088626in}{3.511154in}}%
\pgfpathlineto{\pgfqpoint{3.088626in}{3.508205in}}%
\pgfpathmoveto{\pgfqpoint{3.084085in}{3.511154in}}%
\pgfpathlineto{\pgfqpoint{3.084085in}{3.511154in}}%
\pgfpathlineto{\pgfqpoint{3.084085in}{3.514103in}}%
\pgfpathlineto{\pgfqpoint{3.088626in}{3.514103in}}%
\pgfpathlineto{\pgfqpoint{3.088626in}{3.511154in}}%
\pgfpathmoveto{\pgfqpoint{3.088626in}{3.508205in}}%
\pgfpathlineto{\pgfqpoint{3.088626in}{3.508205in}}%
\pgfpathlineto{\pgfqpoint{3.088626in}{3.511154in}}%
\pgfpathlineto{\pgfqpoint{3.093166in}{3.511154in}}%
\pgfpathlineto{\pgfqpoint{3.093166in}{3.508205in}}%
\pgfpathmoveto{\pgfqpoint{3.097707in}{3.499357in}}%
\pgfpathlineto{\pgfqpoint{3.097707in}{3.499357in}}%
\pgfpathlineto{\pgfqpoint{3.097707in}{3.502306in}}%
\pgfpathlineto{\pgfqpoint{3.102248in}{3.502306in}}%
\pgfpathlineto{\pgfqpoint{3.102248in}{3.499357in}}%
\pgfpathmoveto{\pgfqpoint{3.093166in}{3.502306in}}%
\pgfpathlineto{\pgfqpoint{3.093166in}{3.502306in}}%
\pgfpathlineto{\pgfqpoint{3.093166in}{3.505255in}}%
\pgfpathlineto{\pgfqpoint{3.097707in}{3.505255in}}%
\pgfpathlineto{\pgfqpoint{3.097707in}{3.502306in}}%
\pgfpathmoveto{\pgfqpoint{3.093166in}{3.505255in}}%
\pgfpathlineto{\pgfqpoint{3.093166in}{3.505255in}}%
\pgfpathlineto{\pgfqpoint{3.093166in}{3.508205in}}%
\pgfpathlineto{\pgfqpoint{3.097707in}{3.508205in}}%
\pgfpathlineto{\pgfqpoint{3.097707in}{3.505255in}}%
\pgfpathmoveto{\pgfqpoint{3.097707in}{3.502306in}}%
\pgfpathlineto{\pgfqpoint{3.097707in}{3.502306in}}%
\pgfpathlineto{\pgfqpoint{3.097707in}{3.505255in}}%
\pgfpathlineto{\pgfqpoint{3.102248in}{3.505255in}}%
\pgfpathlineto{\pgfqpoint{3.102248in}{3.502306in}}%
\pgfpathmoveto{\pgfqpoint{3.102248in}{3.496407in}}%
\pgfpathlineto{\pgfqpoint{3.102248in}{3.496407in}}%
\pgfpathlineto{\pgfqpoint{3.102248in}{3.499357in}}%
\pgfpathlineto{\pgfqpoint{3.106789in}{3.499357in}}%
\pgfpathlineto{\pgfqpoint{3.106789in}{3.496407in}}%
\pgfpathmoveto{\pgfqpoint{3.102248in}{3.499357in}}%
\pgfpathlineto{\pgfqpoint{3.102248in}{3.499357in}}%
\pgfpathlineto{\pgfqpoint{3.102248in}{3.502306in}}%
\pgfpathlineto{\pgfqpoint{3.106789in}{3.502306in}}%
\pgfpathlineto{\pgfqpoint{3.106789in}{3.499357in}}%
\pgfpathmoveto{\pgfqpoint{3.106789in}{3.496407in}}%
\pgfpathlineto{\pgfqpoint{3.106789in}{3.496407in}}%
\pgfpathlineto{\pgfqpoint{3.106789in}{3.499357in}}%
\pgfpathlineto{\pgfqpoint{3.111330in}{3.499357in}}%
\pgfpathlineto{\pgfqpoint{3.111330in}{3.496407in}}%
\pgfpathmoveto{\pgfqpoint{3.124952in}{3.481661in}}%
\pgfpathlineto{\pgfqpoint{3.124952in}{3.481661in}}%
\pgfpathlineto{\pgfqpoint{3.124952in}{3.484610in}}%
\pgfpathlineto{\pgfqpoint{3.129493in}{3.484610in}}%
\pgfpathlineto{\pgfqpoint{3.129493in}{3.481661in}}%
\pgfpathmoveto{\pgfqpoint{3.115871in}{3.487560in}}%
\pgfpathlineto{\pgfqpoint{3.115871in}{3.487560in}}%
\pgfpathlineto{\pgfqpoint{3.115871in}{3.490509in}}%
\pgfpathlineto{\pgfqpoint{3.120412in}{3.490509in}}%
\pgfpathlineto{\pgfqpoint{3.120412in}{3.487560in}}%
\pgfpathmoveto{\pgfqpoint{3.111330in}{3.490509in}}%
\pgfpathlineto{\pgfqpoint{3.111330in}{3.490509in}}%
\pgfpathlineto{\pgfqpoint{3.111330in}{3.493458in}}%
\pgfpathlineto{\pgfqpoint{3.115871in}{3.493458in}}%
\pgfpathlineto{\pgfqpoint{3.115871in}{3.490509in}}%
\pgfpathmoveto{\pgfqpoint{3.111330in}{3.493458in}}%
\pgfpathlineto{\pgfqpoint{3.111330in}{3.493458in}}%
\pgfpathlineto{\pgfqpoint{3.111330in}{3.496407in}}%
\pgfpathlineto{\pgfqpoint{3.115871in}{3.496407in}}%
\pgfpathlineto{\pgfqpoint{3.115871in}{3.493458in}}%
\pgfpathmoveto{\pgfqpoint{3.115871in}{3.490509in}}%
\pgfpathlineto{\pgfqpoint{3.115871in}{3.490509in}}%
\pgfpathlineto{\pgfqpoint{3.115871in}{3.493458in}}%
\pgfpathlineto{\pgfqpoint{3.120412in}{3.493458in}}%
\pgfpathlineto{\pgfqpoint{3.120412in}{3.490509in}}%
\pgfpathmoveto{\pgfqpoint{3.120412in}{3.484610in}}%
\pgfpathlineto{\pgfqpoint{3.120412in}{3.484610in}}%
\pgfpathlineto{\pgfqpoint{3.120412in}{3.487560in}}%
\pgfpathlineto{\pgfqpoint{3.124952in}{3.487560in}}%
\pgfpathlineto{\pgfqpoint{3.124952in}{3.484610in}}%
\pgfpathmoveto{\pgfqpoint{3.120412in}{3.487560in}}%
\pgfpathlineto{\pgfqpoint{3.120412in}{3.487560in}}%
\pgfpathlineto{\pgfqpoint{3.120412in}{3.490509in}}%
\pgfpathlineto{\pgfqpoint{3.124952in}{3.490509in}}%
\pgfpathlineto{\pgfqpoint{3.124952in}{3.487560in}}%
\pgfpathmoveto{\pgfqpoint{3.124952in}{3.484610in}}%
\pgfpathlineto{\pgfqpoint{3.124952in}{3.484610in}}%
\pgfpathlineto{\pgfqpoint{3.124952in}{3.487560in}}%
\pgfpathlineto{\pgfqpoint{3.129493in}{3.487560in}}%
\pgfpathlineto{\pgfqpoint{3.129493in}{3.484610in}}%
\pgfpathmoveto{\pgfqpoint{3.134034in}{3.475763in}}%
\pgfpathlineto{\pgfqpoint{3.134034in}{3.475763in}}%
\pgfpathlineto{\pgfqpoint{3.134034in}{3.478712in}}%
\pgfpathlineto{\pgfqpoint{3.138575in}{3.478712in}}%
\pgfpathlineto{\pgfqpoint{3.138575in}{3.475763in}}%
\pgfpathmoveto{\pgfqpoint{3.129493in}{3.478712in}}%
\pgfpathlineto{\pgfqpoint{3.129493in}{3.478712in}}%
\pgfpathlineto{\pgfqpoint{3.129493in}{3.481661in}}%
\pgfpathlineto{\pgfqpoint{3.134034in}{3.481661in}}%
\pgfpathlineto{\pgfqpoint{3.134034in}{3.478712in}}%
\pgfpathmoveto{\pgfqpoint{3.129493in}{3.481661in}}%
\pgfpathlineto{\pgfqpoint{3.129493in}{3.481661in}}%
\pgfpathlineto{\pgfqpoint{3.129493in}{3.484610in}}%
\pgfpathlineto{\pgfqpoint{3.134034in}{3.484610in}}%
\pgfpathlineto{\pgfqpoint{3.134034in}{3.481661in}}%
\pgfpathmoveto{\pgfqpoint{3.134034in}{3.478712in}}%
\pgfpathlineto{\pgfqpoint{3.134034in}{3.478712in}}%
\pgfpathlineto{\pgfqpoint{3.134034in}{3.481661in}}%
\pgfpathlineto{\pgfqpoint{3.138575in}{3.481661in}}%
\pgfpathlineto{\pgfqpoint{3.138575in}{3.478712in}}%
\pgfpathmoveto{\pgfqpoint{3.138575in}{3.472813in}}%
\pgfpathlineto{\pgfqpoint{3.138575in}{3.472813in}}%
\pgfpathlineto{\pgfqpoint{3.138575in}{3.475763in}}%
\pgfpathlineto{\pgfqpoint{3.143116in}{3.475763in}}%
\pgfpathlineto{\pgfqpoint{3.143116in}{3.472813in}}%
\pgfpathmoveto{\pgfqpoint{3.138575in}{3.475763in}}%
\pgfpathlineto{\pgfqpoint{3.138575in}{3.475763in}}%
\pgfpathlineto{\pgfqpoint{3.138575in}{3.478712in}}%
\pgfpathlineto{\pgfqpoint{3.143116in}{3.478712in}}%
\pgfpathlineto{\pgfqpoint{3.143116in}{3.475763in}}%
\pgfpathmoveto{\pgfqpoint{3.143116in}{3.472813in}}%
\pgfpathlineto{\pgfqpoint{3.143116in}{3.472813in}}%
\pgfpathlineto{\pgfqpoint{3.143116in}{3.475763in}}%
\pgfpathlineto{\pgfqpoint{3.147657in}{3.475763in}}%
\pgfpathlineto{\pgfqpoint{3.147657in}{3.472813in}}%
\pgfpathmoveto{\pgfqpoint{3.161279in}{3.458067in}}%
\pgfpathlineto{\pgfqpoint{3.161279in}{3.458067in}}%
\pgfpathlineto{\pgfqpoint{3.161279in}{3.461016in}}%
\pgfpathlineto{\pgfqpoint{3.165820in}{3.461016in}}%
\pgfpathlineto{\pgfqpoint{3.165820in}{3.458067in}}%
\pgfpathmoveto{\pgfqpoint{3.152197in}{3.463966in}}%
\pgfpathlineto{\pgfqpoint{3.152197in}{3.463966in}}%
\pgfpathlineto{\pgfqpoint{3.152197in}{3.466915in}}%
\pgfpathlineto{\pgfqpoint{3.156738in}{3.466915in}}%
\pgfpathlineto{\pgfqpoint{3.156738in}{3.463966in}}%
\pgfpathmoveto{\pgfqpoint{3.147657in}{3.466915in}}%
\pgfpathlineto{\pgfqpoint{3.147657in}{3.466915in}}%
\pgfpathlineto{\pgfqpoint{3.147657in}{3.469864in}}%
\pgfpathlineto{\pgfqpoint{3.152197in}{3.469864in}}%
\pgfpathlineto{\pgfqpoint{3.152197in}{3.466915in}}%
\pgfpathmoveto{\pgfqpoint{3.147657in}{3.469864in}}%
\pgfpathlineto{\pgfqpoint{3.147657in}{3.469864in}}%
\pgfpathlineto{\pgfqpoint{3.147657in}{3.472813in}}%
\pgfpathlineto{\pgfqpoint{3.152197in}{3.472813in}}%
\pgfpathlineto{\pgfqpoint{3.152197in}{3.469864in}}%
\pgfpathmoveto{\pgfqpoint{3.152197in}{3.466915in}}%
\pgfpathlineto{\pgfqpoint{3.152197in}{3.466915in}}%
\pgfpathlineto{\pgfqpoint{3.152197in}{3.469864in}}%
\pgfpathlineto{\pgfqpoint{3.156738in}{3.469864in}}%
\pgfpathlineto{\pgfqpoint{3.156738in}{3.466915in}}%
\pgfpathmoveto{\pgfqpoint{3.156738in}{3.461016in}}%
\pgfpathlineto{\pgfqpoint{3.156738in}{3.461016in}}%
\pgfpathlineto{\pgfqpoint{3.156738in}{3.463966in}}%
\pgfpathlineto{\pgfqpoint{3.161279in}{3.463966in}}%
\pgfpathlineto{\pgfqpoint{3.161279in}{3.461016in}}%
\pgfpathmoveto{\pgfqpoint{3.156738in}{3.463966in}}%
\pgfpathlineto{\pgfqpoint{3.156738in}{3.463966in}}%
\pgfpathlineto{\pgfqpoint{3.156738in}{3.466915in}}%
\pgfpathlineto{\pgfqpoint{3.161279in}{3.466915in}}%
\pgfpathlineto{\pgfqpoint{3.161279in}{3.463966in}}%
\pgfpathmoveto{\pgfqpoint{3.161279in}{3.461016in}}%
\pgfpathlineto{\pgfqpoint{3.161279in}{3.461016in}}%
\pgfpathlineto{\pgfqpoint{3.161279in}{3.463966in}}%
\pgfpathlineto{\pgfqpoint{3.165820in}{3.463966in}}%
\pgfpathlineto{\pgfqpoint{3.165820in}{3.461016in}}%
\pgfpathmoveto{\pgfqpoint{3.170361in}{3.452169in}}%
\pgfpathlineto{\pgfqpoint{3.170361in}{3.452169in}}%
\pgfpathlineto{\pgfqpoint{3.170361in}{3.455118in}}%
\pgfpathlineto{\pgfqpoint{3.174902in}{3.455118in}}%
\pgfpathlineto{\pgfqpoint{3.174902in}{3.452169in}}%
\pgfpathmoveto{\pgfqpoint{3.165820in}{3.455118in}}%
\pgfpathlineto{\pgfqpoint{3.165820in}{3.455118in}}%
\pgfpathlineto{\pgfqpoint{3.165820in}{3.458067in}}%
\pgfpathlineto{\pgfqpoint{3.170361in}{3.458067in}}%
\pgfpathlineto{\pgfqpoint{3.170361in}{3.455118in}}%
\pgfpathmoveto{\pgfqpoint{3.165820in}{3.458067in}}%
\pgfpathlineto{\pgfqpoint{3.165820in}{3.458067in}}%
\pgfpathlineto{\pgfqpoint{3.165820in}{3.461016in}}%
\pgfpathlineto{\pgfqpoint{3.170361in}{3.461016in}}%
\pgfpathlineto{\pgfqpoint{3.170361in}{3.458067in}}%
\pgfpathmoveto{\pgfqpoint{3.170361in}{3.455118in}}%
\pgfpathlineto{\pgfqpoint{3.170361in}{3.455118in}}%
\pgfpathlineto{\pgfqpoint{3.170361in}{3.458067in}}%
\pgfpathlineto{\pgfqpoint{3.174902in}{3.458067in}}%
\pgfpathlineto{\pgfqpoint{3.174902in}{3.455118in}}%
\pgfpathmoveto{\pgfqpoint{3.174902in}{3.449219in}}%
\pgfpathlineto{\pgfqpoint{3.174902in}{3.449219in}}%
\pgfpathlineto{\pgfqpoint{3.174902in}{3.452169in}}%
\pgfpathlineto{\pgfqpoint{3.179442in}{3.452169in}}%
\pgfpathlineto{\pgfqpoint{3.179442in}{3.449219in}}%
\pgfpathmoveto{\pgfqpoint{3.174902in}{3.452169in}}%
\pgfpathlineto{\pgfqpoint{3.174902in}{3.452169in}}%
\pgfpathlineto{\pgfqpoint{3.174902in}{3.455118in}}%
\pgfpathlineto{\pgfqpoint{3.179442in}{3.455118in}}%
\pgfpathlineto{\pgfqpoint{3.179442in}{3.452169in}}%
\pgfpathmoveto{\pgfqpoint{3.179442in}{3.449219in}}%
\pgfpathlineto{\pgfqpoint{3.179442in}{3.449219in}}%
\pgfpathlineto{\pgfqpoint{3.179442in}{3.452169in}}%
\pgfpathlineto{\pgfqpoint{3.183983in}{3.452169in}}%
\pgfpathlineto{\pgfqpoint{3.183983in}{3.449219in}}%
\pgfpathmoveto{\pgfqpoint{3.183983in}{3.446270in}}%
\pgfpathlineto{\pgfqpoint{3.183983in}{3.446270in}}%
\pgfpathlineto{\pgfqpoint{3.183983in}{3.449219in}}%
\pgfpathlineto{\pgfqpoint{3.188524in}{3.449219in}}%
\pgfpathlineto{\pgfqpoint{3.188524in}{3.446270in}}%
\pgfpathmoveto{\pgfqpoint{3.188524in}{3.443321in}}%
\pgfpathlineto{\pgfqpoint{3.188524in}{3.443321in}}%
\pgfpathlineto{\pgfqpoint{3.188524in}{3.446270in}}%
\pgfpathlineto{\pgfqpoint{3.193065in}{3.446270in}}%
\pgfpathlineto{\pgfqpoint{3.193065in}{3.443321in}}%
\pgfpathmoveto{\pgfqpoint{3.188524in}{3.446270in}}%
\pgfpathlineto{\pgfqpoint{3.188524in}{3.446270in}}%
\pgfpathlineto{\pgfqpoint{3.188524in}{3.449219in}}%
\pgfpathlineto{\pgfqpoint{3.193065in}{3.449219in}}%
\pgfpathlineto{\pgfqpoint{3.193065in}{3.446270in}}%
\pgfpathmoveto{\pgfqpoint{3.193065in}{3.440372in}}%
\pgfpathlineto{\pgfqpoint{3.193065in}{3.440372in}}%
\pgfpathlineto{\pgfqpoint{3.193065in}{3.443321in}}%
\pgfpathlineto{\pgfqpoint{3.197606in}{3.443321in}}%
\pgfpathlineto{\pgfqpoint{3.197606in}{3.440372in}}%
\pgfpathmoveto{\pgfqpoint{3.197606in}{3.437422in}}%
\pgfpathlineto{\pgfqpoint{3.197606in}{3.437422in}}%
\pgfpathlineto{\pgfqpoint{3.197606in}{3.440372in}}%
\pgfpathlineto{\pgfqpoint{3.202147in}{3.440372in}}%
\pgfpathlineto{\pgfqpoint{3.202147in}{3.437422in}}%
\pgfpathmoveto{\pgfqpoint{3.197606in}{3.440372in}}%
\pgfpathlineto{\pgfqpoint{3.197606in}{3.440372in}}%
\pgfpathlineto{\pgfqpoint{3.197606in}{3.443321in}}%
\pgfpathlineto{\pgfqpoint{3.202147in}{3.443321in}}%
\pgfpathlineto{\pgfqpoint{3.202147in}{3.440372in}}%
\pgfpathmoveto{\pgfqpoint{3.193065in}{3.443321in}}%
\pgfpathlineto{\pgfqpoint{3.193065in}{3.443321in}}%
\pgfpathlineto{\pgfqpoint{3.193065in}{3.446270in}}%
\pgfpathlineto{\pgfqpoint{3.197606in}{3.446270in}}%
\pgfpathlineto{\pgfqpoint{3.197606in}{3.443321in}}%
\pgfpathmoveto{\pgfqpoint{3.202147in}{3.434473in}}%
\pgfpathlineto{\pgfqpoint{3.202147in}{3.434473in}}%
\pgfpathlineto{\pgfqpoint{3.202147in}{3.437422in}}%
\pgfpathlineto{\pgfqpoint{3.206687in}{3.437422in}}%
\pgfpathlineto{\pgfqpoint{3.206687in}{3.434473in}}%
\pgfpathmoveto{\pgfqpoint{3.206687in}{3.431524in}}%
\pgfpathlineto{\pgfqpoint{3.206687in}{3.431524in}}%
\pgfpathlineto{\pgfqpoint{3.206687in}{3.434473in}}%
\pgfpathlineto{\pgfqpoint{3.211228in}{3.434473in}}%
\pgfpathlineto{\pgfqpoint{3.211228in}{3.431524in}}%
\pgfpathmoveto{\pgfqpoint{3.206687in}{3.434473in}}%
\pgfpathlineto{\pgfqpoint{3.206687in}{3.434473in}}%
\pgfpathlineto{\pgfqpoint{3.206687in}{3.437422in}}%
\pgfpathlineto{\pgfqpoint{3.211228in}{3.437422in}}%
\pgfpathlineto{\pgfqpoint{3.211228in}{3.434473in}}%
\pgfpathmoveto{\pgfqpoint{3.211228in}{3.428575in}}%
\pgfpathlineto{\pgfqpoint{3.211228in}{3.428575in}}%
\pgfpathlineto{\pgfqpoint{3.211228in}{3.431524in}}%
\pgfpathlineto{\pgfqpoint{3.215769in}{3.431524in}}%
\pgfpathlineto{\pgfqpoint{3.215769in}{3.428575in}}%
\pgfpathmoveto{\pgfqpoint{3.215769in}{3.425625in}}%
\pgfpathlineto{\pgfqpoint{3.215769in}{3.425625in}}%
\pgfpathlineto{\pgfqpoint{3.215769in}{3.428575in}}%
\pgfpathlineto{\pgfqpoint{3.220310in}{3.428575in}}%
\pgfpathlineto{\pgfqpoint{3.220310in}{3.425625in}}%
\pgfpathmoveto{\pgfqpoint{3.215769in}{3.428575in}}%
\pgfpathlineto{\pgfqpoint{3.215769in}{3.428575in}}%
\pgfpathlineto{\pgfqpoint{3.215769in}{3.431524in}}%
\pgfpathlineto{\pgfqpoint{3.220310in}{3.431524in}}%
\pgfpathlineto{\pgfqpoint{3.220310in}{3.428575in}}%
\pgfpathmoveto{\pgfqpoint{3.211228in}{3.431524in}}%
\pgfpathlineto{\pgfqpoint{3.211228in}{3.431524in}}%
\pgfpathlineto{\pgfqpoint{3.211228in}{3.434473in}}%
\pgfpathlineto{\pgfqpoint{3.215769in}{3.434473in}}%
\pgfpathlineto{\pgfqpoint{3.215769in}{3.431524in}}%
\pgfpathmoveto{\pgfqpoint{3.202147in}{3.437422in}}%
\pgfpathlineto{\pgfqpoint{3.202147in}{3.437422in}}%
\pgfpathlineto{\pgfqpoint{3.202147in}{3.440372in}}%
\pgfpathlineto{\pgfqpoint{3.206687in}{3.440372in}}%
\pgfpathlineto{\pgfqpoint{3.206687in}{3.437422in}}%
\pgfpathmoveto{\pgfqpoint{3.183983in}{3.449219in}}%
\pgfpathlineto{\pgfqpoint{3.183983in}{3.449219in}}%
\pgfpathlineto{\pgfqpoint{3.183983in}{3.452169in}}%
\pgfpathlineto{\pgfqpoint{3.188524in}{3.452169in}}%
\pgfpathlineto{\pgfqpoint{3.188524in}{3.449219in}}%
\pgfpathmoveto{\pgfqpoint{3.361088in}{3.328303in}}%
\pgfpathlineto{\pgfqpoint{3.361088in}{3.328303in}}%
\pgfpathlineto{\pgfqpoint{3.361088in}{3.331253in}}%
\pgfpathlineto{\pgfqpoint{3.365629in}{3.331253in}}%
\pgfpathlineto{\pgfqpoint{3.365629in}{3.328303in}}%
\pgfpathmoveto{\pgfqpoint{3.288428in}{3.375490in}}%
\pgfpathlineto{\pgfqpoint{3.288428in}{3.375490in}}%
\pgfpathlineto{\pgfqpoint{3.288428in}{3.378439in}}%
\pgfpathlineto{\pgfqpoint{3.292970in}{3.378439in}}%
\pgfpathlineto{\pgfqpoint{3.292970in}{3.375490in}}%
\pgfpathmoveto{\pgfqpoint{3.252099in}{3.399083in}}%
\pgfpathlineto{\pgfqpoint{3.252099in}{3.399083in}}%
\pgfpathlineto{\pgfqpoint{3.252099in}{3.402032in}}%
\pgfpathlineto{\pgfqpoint{3.256640in}{3.402032in}}%
\pgfpathlineto{\pgfqpoint{3.256640in}{3.399083in}}%
\pgfpathmoveto{\pgfqpoint{3.220310in}{3.422676in}}%
\pgfpathlineto{\pgfqpoint{3.220310in}{3.422676in}}%
\pgfpathlineto{\pgfqpoint{3.220310in}{3.425625in}}%
\pgfpathlineto{\pgfqpoint{3.224851in}{3.425625in}}%
\pgfpathlineto{\pgfqpoint{3.224851in}{3.422676in}}%
\pgfpathmoveto{\pgfqpoint{3.224851in}{3.419727in}}%
\pgfpathlineto{\pgfqpoint{3.224851in}{3.419727in}}%
\pgfpathlineto{\pgfqpoint{3.224851in}{3.422676in}}%
\pgfpathlineto{\pgfqpoint{3.229392in}{3.422676in}}%
\pgfpathlineto{\pgfqpoint{3.229392in}{3.419727in}}%
\pgfpathmoveto{\pgfqpoint{3.224851in}{3.422676in}}%
\pgfpathlineto{\pgfqpoint{3.224851in}{3.422676in}}%
\pgfpathlineto{\pgfqpoint{3.224851in}{3.425625in}}%
\pgfpathlineto{\pgfqpoint{3.229392in}{3.425625in}}%
\pgfpathlineto{\pgfqpoint{3.229392in}{3.422676in}}%
\pgfpathmoveto{\pgfqpoint{3.229392in}{3.416778in}}%
\pgfpathlineto{\pgfqpoint{3.229392in}{3.416778in}}%
\pgfpathlineto{\pgfqpoint{3.229392in}{3.419727in}}%
\pgfpathlineto{\pgfqpoint{3.233934in}{3.419727in}}%
\pgfpathlineto{\pgfqpoint{3.233934in}{3.416778in}}%
\pgfpathmoveto{\pgfqpoint{3.233934in}{3.413829in}}%
\pgfpathlineto{\pgfqpoint{3.233934in}{3.413829in}}%
\pgfpathlineto{\pgfqpoint{3.233934in}{3.416778in}}%
\pgfpathlineto{\pgfqpoint{3.238475in}{3.416778in}}%
\pgfpathlineto{\pgfqpoint{3.238475in}{3.413829in}}%
\pgfpathmoveto{\pgfqpoint{3.233934in}{3.416778in}}%
\pgfpathlineto{\pgfqpoint{3.233934in}{3.416778in}}%
\pgfpathlineto{\pgfqpoint{3.233934in}{3.419727in}}%
\pgfpathlineto{\pgfqpoint{3.238475in}{3.419727in}}%
\pgfpathlineto{\pgfqpoint{3.238475in}{3.416778in}}%
\pgfpathmoveto{\pgfqpoint{3.229392in}{3.419727in}}%
\pgfpathlineto{\pgfqpoint{3.229392in}{3.419727in}}%
\pgfpathlineto{\pgfqpoint{3.229392in}{3.422676in}}%
\pgfpathlineto{\pgfqpoint{3.233934in}{3.422676in}}%
\pgfpathlineto{\pgfqpoint{3.233934in}{3.419727in}}%
\pgfpathmoveto{\pgfqpoint{3.238475in}{3.410880in}}%
\pgfpathlineto{\pgfqpoint{3.238475in}{3.410880in}}%
\pgfpathlineto{\pgfqpoint{3.238475in}{3.413829in}}%
\pgfpathlineto{\pgfqpoint{3.243016in}{3.413829in}}%
\pgfpathlineto{\pgfqpoint{3.243016in}{3.410880in}}%
\pgfpathmoveto{\pgfqpoint{3.243016in}{3.407930in}}%
\pgfpathlineto{\pgfqpoint{3.243016in}{3.407930in}}%
\pgfpathlineto{\pgfqpoint{3.243016in}{3.410880in}}%
\pgfpathlineto{\pgfqpoint{3.247557in}{3.410880in}}%
\pgfpathlineto{\pgfqpoint{3.247557in}{3.407930in}}%
\pgfpathmoveto{\pgfqpoint{3.243016in}{3.410880in}}%
\pgfpathlineto{\pgfqpoint{3.243016in}{3.410880in}}%
\pgfpathlineto{\pgfqpoint{3.243016in}{3.413829in}}%
\pgfpathlineto{\pgfqpoint{3.247557in}{3.413829in}}%
\pgfpathlineto{\pgfqpoint{3.247557in}{3.410880in}}%
\pgfpathmoveto{\pgfqpoint{3.247557in}{3.402032in}}%
\pgfpathlineto{\pgfqpoint{3.247557in}{3.402032in}}%
\pgfpathlineto{\pgfqpoint{3.247557in}{3.404981in}}%
\pgfpathlineto{\pgfqpoint{3.252099in}{3.404981in}}%
\pgfpathlineto{\pgfqpoint{3.252099in}{3.402032in}}%
\pgfpathmoveto{\pgfqpoint{3.247557in}{3.404981in}}%
\pgfpathlineto{\pgfqpoint{3.247557in}{3.404981in}}%
\pgfpathlineto{\pgfqpoint{3.247557in}{3.407930in}}%
\pgfpathlineto{\pgfqpoint{3.252099in}{3.407930in}}%
\pgfpathlineto{\pgfqpoint{3.252099in}{3.404981in}}%
\pgfpathmoveto{\pgfqpoint{3.252099in}{3.402032in}}%
\pgfpathlineto{\pgfqpoint{3.252099in}{3.402032in}}%
\pgfpathlineto{\pgfqpoint{3.252099in}{3.404981in}}%
\pgfpathlineto{\pgfqpoint{3.256640in}{3.404981in}}%
\pgfpathlineto{\pgfqpoint{3.256640in}{3.402032in}}%
\pgfpathmoveto{\pgfqpoint{3.247557in}{3.407930in}}%
\pgfpathlineto{\pgfqpoint{3.247557in}{3.407930in}}%
\pgfpathlineto{\pgfqpoint{3.247557in}{3.410880in}}%
\pgfpathlineto{\pgfqpoint{3.252099in}{3.410880in}}%
\pgfpathlineto{\pgfqpoint{3.252099in}{3.407930in}}%
\pgfpathmoveto{\pgfqpoint{3.238475in}{3.413829in}}%
\pgfpathlineto{\pgfqpoint{3.238475in}{3.413829in}}%
\pgfpathlineto{\pgfqpoint{3.238475in}{3.416778in}}%
\pgfpathlineto{\pgfqpoint{3.243016in}{3.416778in}}%
\pgfpathlineto{\pgfqpoint{3.243016in}{3.413829in}}%
\pgfpathmoveto{\pgfqpoint{3.270264in}{3.387286in}}%
\pgfpathlineto{\pgfqpoint{3.270264in}{3.387286in}}%
\pgfpathlineto{\pgfqpoint{3.270264in}{3.390236in}}%
\pgfpathlineto{\pgfqpoint{3.274805in}{3.390236in}}%
\pgfpathlineto{\pgfqpoint{3.274805in}{3.387286in}}%
\pgfpathmoveto{\pgfqpoint{3.261181in}{3.393185in}}%
\pgfpathlineto{\pgfqpoint{3.261181in}{3.393185in}}%
\pgfpathlineto{\pgfqpoint{3.261181in}{3.396134in}}%
\pgfpathlineto{\pgfqpoint{3.265722in}{3.396134in}}%
\pgfpathlineto{\pgfqpoint{3.265722in}{3.393185in}}%
\pgfpathmoveto{\pgfqpoint{3.256640in}{3.396134in}}%
\pgfpathlineto{\pgfqpoint{3.256640in}{3.396134in}}%
\pgfpathlineto{\pgfqpoint{3.256640in}{3.399083in}}%
\pgfpathlineto{\pgfqpoint{3.261181in}{3.399083in}}%
\pgfpathlineto{\pgfqpoint{3.261181in}{3.396134in}}%
\pgfpathmoveto{\pgfqpoint{3.256640in}{3.399083in}}%
\pgfpathlineto{\pgfqpoint{3.256640in}{3.399083in}}%
\pgfpathlineto{\pgfqpoint{3.256640in}{3.402032in}}%
\pgfpathlineto{\pgfqpoint{3.261181in}{3.402032in}}%
\pgfpathlineto{\pgfqpoint{3.261181in}{3.399083in}}%
\pgfpathmoveto{\pgfqpoint{3.261181in}{3.396134in}}%
\pgfpathlineto{\pgfqpoint{3.261181in}{3.396134in}}%
\pgfpathlineto{\pgfqpoint{3.261181in}{3.399083in}}%
\pgfpathlineto{\pgfqpoint{3.265722in}{3.399083in}}%
\pgfpathlineto{\pgfqpoint{3.265722in}{3.396134in}}%
\pgfpathmoveto{\pgfqpoint{3.265722in}{3.390236in}}%
\pgfpathlineto{\pgfqpoint{3.265722in}{3.390236in}}%
\pgfpathlineto{\pgfqpoint{3.265722in}{3.393185in}}%
\pgfpathlineto{\pgfqpoint{3.270264in}{3.393185in}}%
\pgfpathlineto{\pgfqpoint{3.270264in}{3.390236in}}%
\pgfpathmoveto{\pgfqpoint{3.265722in}{3.393185in}}%
\pgfpathlineto{\pgfqpoint{3.265722in}{3.393185in}}%
\pgfpathlineto{\pgfqpoint{3.265722in}{3.396134in}}%
\pgfpathlineto{\pgfqpoint{3.270264in}{3.396134in}}%
\pgfpathlineto{\pgfqpoint{3.270264in}{3.393185in}}%
\pgfpathmoveto{\pgfqpoint{3.270264in}{3.390236in}}%
\pgfpathlineto{\pgfqpoint{3.270264in}{3.390236in}}%
\pgfpathlineto{\pgfqpoint{3.270264in}{3.393185in}}%
\pgfpathlineto{\pgfqpoint{3.274805in}{3.393185in}}%
\pgfpathlineto{\pgfqpoint{3.274805in}{3.390236in}}%
\pgfpathmoveto{\pgfqpoint{3.279346in}{3.381388in}}%
\pgfpathlineto{\pgfqpoint{3.279346in}{3.381388in}}%
\pgfpathlineto{\pgfqpoint{3.279346in}{3.384337in}}%
\pgfpathlineto{\pgfqpoint{3.283887in}{3.384337in}}%
\pgfpathlineto{\pgfqpoint{3.283887in}{3.381388in}}%
\pgfpathmoveto{\pgfqpoint{3.274805in}{3.384337in}}%
\pgfpathlineto{\pgfqpoint{3.274805in}{3.384337in}}%
\pgfpathlineto{\pgfqpoint{3.274805in}{3.387286in}}%
\pgfpathlineto{\pgfqpoint{3.279346in}{3.387286in}}%
\pgfpathlineto{\pgfqpoint{3.279346in}{3.384337in}}%
\pgfpathmoveto{\pgfqpoint{3.274805in}{3.387286in}}%
\pgfpathlineto{\pgfqpoint{3.274805in}{3.387286in}}%
\pgfpathlineto{\pgfqpoint{3.274805in}{3.390236in}}%
\pgfpathlineto{\pgfqpoint{3.279346in}{3.390236in}}%
\pgfpathlineto{\pgfqpoint{3.279346in}{3.387286in}}%
\pgfpathmoveto{\pgfqpoint{3.279346in}{3.384337in}}%
\pgfpathlineto{\pgfqpoint{3.279346in}{3.384337in}}%
\pgfpathlineto{\pgfqpoint{3.279346in}{3.387286in}}%
\pgfpathlineto{\pgfqpoint{3.283887in}{3.387286in}}%
\pgfpathlineto{\pgfqpoint{3.283887in}{3.384337in}}%
\pgfpathmoveto{\pgfqpoint{3.283887in}{3.378439in}}%
\pgfpathlineto{\pgfqpoint{3.283887in}{3.378439in}}%
\pgfpathlineto{\pgfqpoint{3.283887in}{3.381388in}}%
\pgfpathlineto{\pgfqpoint{3.288428in}{3.381388in}}%
\pgfpathlineto{\pgfqpoint{3.288428in}{3.378439in}}%
\pgfpathmoveto{\pgfqpoint{3.283887in}{3.381388in}}%
\pgfpathlineto{\pgfqpoint{3.283887in}{3.381388in}}%
\pgfpathlineto{\pgfqpoint{3.283887in}{3.384337in}}%
\pgfpathlineto{\pgfqpoint{3.288428in}{3.384337in}}%
\pgfpathlineto{\pgfqpoint{3.288428in}{3.381388in}}%
\pgfpathmoveto{\pgfqpoint{3.288428in}{3.378439in}}%
\pgfpathlineto{\pgfqpoint{3.288428in}{3.378439in}}%
\pgfpathlineto{\pgfqpoint{3.288428in}{3.381388in}}%
\pgfpathlineto{\pgfqpoint{3.292970in}{3.381388in}}%
\pgfpathlineto{\pgfqpoint{3.292970in}{3.378439in}}%
\pgfpathmoveto{\pgfqpoint{3.324758in}{3.351897in}}%
\pgfpathlineto{\pgfqpoint{3.324758in}{3.351897in}}%
\pgfpathlineto{\pgfqpoint{3.324758in}{3.354846in}}%
\pgfpathlineto{\pgfqpoint{3.329300in}{3.354846in}}%
\pgfpathlineto{\pgfqpoint{3.329300in}{3.351897in}}%
\pgfpathmoveto{\pgfqpoint{3.306593in}{3.363693in}}%
\pgfpathlineto{\pgfqpoint{3.306593in}{3.363693in}}%
\pgfpathlineto{\pgfqpoint{3.306593in}{3.366642in}}%
\pgfpathlineto{\pgfqpoint{3.311135in}{3.366642in}}%
\pgfpathlineto{\pgfqpoint{3.311135in}{3.363693in}}%
\pgfpathmoveto{\pgfqpoint{3.297511in}{3.369591in}}%
\pgfpathlineto{\pgfqpoint{3.297511in}{3.369591in}}%
\pgfpathlineto{\pgfqpoint{3.297511in}{3.372541in}}%
\pgfpathlineto{\pgfqpoint{3.302052in}{3.372541in}}%
\pgfpathlineto{\pgfqpoint{3.302052in}{3.369591in}}%
\pgfpathmoveto{\pgfqpoint{3.292970in}{3.372541in}}%
\pgfpathlineto{\pgfqpoint{3.292970in}{3.372541in}}%
\pgfpathlineto{\pgfqpoint{3.292970in}{3.375490in}}%
\pgfpathlineto{\pgfqpoint{3.297511in}{3.375490in}}%
\pgfpathlineto{\pgfqpoint{3.297511in}{3.372541in}}%
\pgfpathmoveto{\pgfqpoint{3.292970in}{3.375490in}}%
\pgfpathlineto{\pgfqpoint{3.292970in}{3.375490in}}%
\pgfpathlineto{\pgfqpoint{3.292970in}{3.378439in}}%
\pgfpathlineto{\pgfqpoint{3.297511in}{3.378439in}}%
\pgfpathlineto{\pgfqpoint{3.297511in}{3.375490in}}%
\pgfpathmoveto{\pgfqpoint{3.297511in}{3.372541in}}%
\pgfpathlineto{\pgfqpoint{3.297511in}{3.372541in}}%
\pgfpathlineto{\pgfqpoint{3.297511in}{3.375490in}}%
\pgfpathlineto{\pgfqpoint{3.302052in}{3.375490in}}%
\pgfpathlineto{\pgfqpoint{3.302052in}{3.372541in}}%
\pgfpathmoveto{\pgfqpoint{3.302052in}{3.366642in}}%
\pgfpathlineto{\pgfqpoint{3.302052in}{3.366642in}}%
\pgfpathlineto{\pgfqpoint{3.302052in}{3.369591in}}%
\pgfpathlineto{\pgfqpoint{3.306593in}{3.369591in}}%
\pgfpathlineto{\pgfqpoint{3.306593in}{3.366642in}}%
\pgfpathmoveto{\pgfqpoint{3.302052in}{3.369591in}}%
\pgfpathlineto{\pgfqpoint{3.302052in}{3.369591in}}%
\pgfpathlineto{\pgfqpoint{3.302052in}{3.372541in}}%
\pgfpathlineto{\pgfqpoint{3.306593in}{3.372541in}}%
\pgfpathlineto{\pgfqpoint{3.306593in}{3.369591in}}%
\pgfpathmoveto{\pgfqpoint{3.306593in}{3.366642in}}%
\pgfpathlineto{\pgfqpoint{3.306593in}{3.366642in}}%
\pgfpathlineto{\pgfqpoint{3.306593in}{3.369591in}}%
\pgfpathlineto{\pgfqpoint{3.311135in}{3.369591in}}%
\pgfpathlineto{\pgfqpoint{3.311135in}{3.366642in}}%
\pgfpathmoveto{\pgfqpoint{3.315676in}{3.357795in}}%
\pgfpathlineto{\pgfqpoint{3.315676in}{3.357795in}}%
\pgfpathlineto{\pgfqpoint{3.315676in}{3.360744in}}%
\pgfpathlineto{\pgfqpoint{3.320217in}{3.360744in}}%
\pgfpathlineto{\pgfqpoint{3.320217in}{3.357795in}}%
\pgfpathmoveto{\pgfqpoint{3.311135in}{3.360744in}}%
\pgfpathlineto{\pgfqpoint{3.311135in}{3.360744in}}%
\pgfpathlineto{\pgfqpoint{3.311135in}{3.363693in}}%
\pgfpathlineto{\pgfqpoint{3.315676in}{3.363693in}}%
\pgfpathlineto{\pgfqpoint{3.315676in}{3.360744in}}%
\pgfpathmoveto{\pgfqpoint{3.311135in}{3.363693in}}%
\pgfpathlineto{\pgfqpoint{3.311135in}{3.363693in}}%
\pgfpathlineto{\pgfqpoint{3.311135in}{3.366642in}}%
\pgfpathlineto{\pgfqpoint{3.315676in}{3.366642in}}%
\pgfpathlineto{\pgfqpoint{3.315676in}{3.363693in}}%
\pgfpathmoveto{\pgfqpoint{3.315676in}{3.360744in}}%
\pgfpathlineto{\pgfqpoint{3.315676in}{3.360744in}}%
\pgfpathlineto{\pgfqpoint{3.315676in}{3.363693in}}%
\pgfpathlineto{\pgfqpoint{3.320217in}{3.363693in}}%
\pgfpathlineto{\pgfqpoint{3.320217in}{3.360744in}}%
\pgfpathmoveto{\pgfqpoint{3.320217in}{3.354846in}}%
\pgfpathlineto{\pgfqpoint{3.320217in}{3.354846in}}%
\pgfpathlineto{\pgfqpoint{3.320217in}{3.357795in}}%
\pgfpathlineto{\pgfqpoint{3.324758in}{3.357795in}}%
\pgfpathlineto{\pgfqpoint{3.324758in}{3.354846in}}%
\pgfpathmoveto{\pgfqpoint{3.320217in}{3.357795in}}%
\pgfpathlineto{\pgfqpoint{3.320217in}{3.357795in}}%
\pgfpathlineto{\pgfqpoint{3.320217in}{3.360744in}}%
\pgfpathlineto{\pgfqpoint{3.324758in}{3.360744in}}%
\pgfpathlineto{\pgfqpoint{3.324758in}{3.357795in}}%
\pgfpathmoveto{\pgfqpoint{3.324758in}{3.354846in}}%
\pgfpathlineto{\pgfqpoint{3.324758in}{3.354846in}}%
\pgfpathlineto{\pgfqpoint{3.324758in}{3.357795in}}%
\pgfpathlineto{\pgfqpoint{3.329300in}{3.357795in}}%
\pgfpathlineto{\pgfqpoint{3.329300in}{3.354846in}}%
\pgfpathmoveto{\pgfqpoint{3.342923in}{3.340100in}}%
\pgfpathlineto{\pgfqpoint{3.342923in}{3.340100in}}%
\pgfpathlineto{\pgfqpoint{3.342923in}{3.343049in}}%
\pgfpathlineto{\pgfqpoint{3.347464in}{3.343049in}}%
\pgfpathlineto{\pgfqpoint{3.347464in}{3.340100in}}%
\pgfpathmoveto{\pgfqpoint{3.333841in}{3.345998in}}%
\pgfpathlineto{\pgfqpoint{3.333841in}{3.345998in}}%
\pgfpathlineto{\pgfqpoint{3.333841in}{3.348947in}}%
\pgfpathlineto{\pgfqpoint{3.338382in}{3.348947in}}%
\pgfpathlineto{\pgfqpoint{3.338382in}{3.345998in}}%
\pgfpathmoveto{\pgfqpoint{3.329300in}{3.348947in}}%
\pgfpathlineto{\pgfqpoint{3.329300in}{3.348947in}}%
\pgfpathlineto{\pgfqpoint{3.329300in}{3.351897in}}%
\pgfpathlineto{\pgfqpoint{3.333841in}{3.351897in}}%
\pgfpathlineto{\pgfqpoint{3.333841in}{3.348947in}}%
\pgfpathmoveto{\pgfqpoint{3.329300in}{3.351897in}}%
\pgfpathlineto{\pgfqpoint{3.329300in}{3.351897in}}%
\pgfpathlineto{\pgfqpoint{3.329300in}{3.354846in}}%
\pgfpathlineto{\pgfqpoint{3.333841in}{3.354846in}}%
\pgfpathlineto{\pgfqpoint{3.333841in}{3.351897in}}%
\pgfpathmoveto{\pgfqpoint{3.333841in}{3.348947in}}%
\pgfpathlineto{\pgfqpoint{3.333841in}{3.348947in}}%
\pgfpathlineto{\pgfqpoint{3.333841in}{3.351897in}}%
\pgfpathlineto{\pgfqpoint{3.338382in}{3.351897in}}%
\pgfpathlineto{\pgfqpoint{3.338382in}{3.348947in}}%
\pgfpathmoveto{\pgfqpoint{3.338382in}{3.343049in}}%
\pgfpathlineto{\pgfqpoint{3.338382in}{3.343049in}}%
\pgfpathlineto{\pgfqpoint{3.338382in}{3.345998in}}%
\pgfpathlineto{\pgfqpoint{3.342923in}{3.345998in}}%
\pgfpathlineto{\pgfqpoint{3.342923in}{3.343049in}}%
\pgfpathmoveto{\pgfqpoint{3.338382in}{3.345998in}}%
\pgfpathlineto{\pgfqpoint{3.338382in}{3.345998in}}%
\pgfpathlineto{\pgfqpoint{3.338382in}{3.348947in}}%
\pgfpathlineto{\pgfqpoint{3.342923in}{3.348947in}}%
\pgfpathlineto{\pgfqpoint{3.342923in}{3.345998in}}%
\pgfpathmoveto{\pgfqpoint{3.342923in}{3.343049in}}%
\pgfpathlineto{\pgfqpoint{3.342923in}{3.343049in}}%
\pgfpathlineto{\pgfqpoint{3.342923in}{3.345998in}}%
\pgfpathlineto{\pgfqpoint{3.347464in}{3.345998in}}%
\pgfpathlineto{\pgfqpoint{3.347464in}{3.343049in}}%
\pgfpathmoveto{\pgfqpoint{3.352006in}{3.334202in}}%
\pgfpathlineto{\pgfqpoint{3.352006in}{3.334202in}}%
\pgfpathlineto{\pgfqpoint{3.352006in}{3.337151in}}%
\pgfpathlineto{\pgfqpoint{3.356547in}{3.337151in}}%
\pgfpathlineto{\pgfqpoint{3.356547in}{3.334202in}}%
\pgfpathmoveto{\pgfqpoint{3.347464in}{3.337151in}}%
\pgfpathlineto{\pgfqpoint{3.347464in}{3.337151in}}%
\pgfpathlineto{\pgfqpoint{3.347464in}{3.340100in}}%
\pgfpathlineto{\pgfqpoint{3.352006in}{3.340100in}}%
\pgfpathlineto{\pgfqpoint{3.352006in}{3.337151in}}%
\pgfpathmoveto{\pgfqpoint{3.347464in}{3.340100in}}%
\pgfpathlineto{\pgfqpoint{3.347464in}{3.340100in}}%
\pgfpathlineto{\pgfqpoint{3.347464in}{3.343049in}}%
\pgfpathlineto{\pgfqpoint{3.352006in}{3.343049in}}%
\pgfpathlineto{\pgfqpoint{3.352006in}{3.340100in}}%
\pgfpathmoveto{\pgfqpoint{3.352006in}{3.337151in}}%
\pgfpathlineto{\pgfqpoint{3.352006in}{3.337151in}}%
\pgfpathlineto{\pgfqpoint{3.352006in}{3.340100in}}%
\pgfpathlineto{\pgfqpoint{3.356547in}{3.340100in}}%
\pgfpathlineto{\pgfqpoint{3.356547in}{3.337151in}}%
\pgfpathmoveto{\pgfqpoint{3.356547in}{3.331253in}}%
\pgfpathlineto{\pgfqpoint{3.356547in}{3.331253in}}%
\pgfpathlineto{\pgfqpoint{3.356547in}{3.334202in}}%
\pgfpathlineto{\pgfqpoint{3.361088in}{3.334202in}}%
\pgfpathlineto{\pgfqpoint{3.361088in}{3.331253in}}%
\pgfpathmoveto{\pgfqpoint{3.356547in}{3.334202in}}%
\pgfpathlineto{\pgfqpoint{3.356547in}{3.334202in}}%
\pgfpathlineto{\pgfqpoint{3.356547in}{3.337151in}}%
\pgfpathlineto{\pgfqpoint{3.361088in}{3.337151in}}%
\pgfpathlineto{\pgfqpoint{3.361088in}{3.334202in}}%
\pgfpathmoveto{\pgfqpoint{3.361088in}{3.331253in}}%
\pgfpathlineto{\pgfqpoint{3.361088in}{3.331253in}}%
\pgfpathlineto{\pgfqpoint{3.361088in}{3.334202in}}%
\pgfpathlineto{\pgfqpoint{3.365629in}{3.334202in}}%
\pgfpathlineto{\pgfqpoint{3.365629in}{3.331253in}}%
\pgfpathmoveto{\pgfqpoint{3.220310in}{3.425625in}}%
\pgfpathlineto{\pgfqpoint{3.220310in}{3.425625in}}%
\pgfpathlineto{\pgfqpoint{3.220310in}{3.428575in}}%
\pgfpathlineto{\pgfqpoint{3.224851in}{3.428575in}}%
\pgfpathlineto{\pgfqpoint{3.224851in}{3.425625in}}%
\pgfpathmoveto{\pgfqpoint{3.433741in}{3.281115in}}%
\pgfpathlineto{\pgfqpoint{3.433741in}{3.281115in}}%
\pgfpathlineto{\pgfqpoint{3.433741in}{3.284064in}}%
\pgfpathlineto{\pgfqpoint{3.438282in}{3.284064in}}%
\pgfpathlineto{\pgfqpoint{3.438282in}{3.281115in}}%
\pgfpathmoveto{\pgfqpoint{3.397415in}{3.304709in}}%
\pgfpathlineto{\pgfqpoint{3.397415in}{3.304709in}}%
\pgfpathlineto{\pgfqpoint{3.397415in}{3.307658in}}%
\pgfpathlineto{\pgfqpoint{3.401956in}{3.307658in}}%
\pgfpathlineto{\pgfqpoint{3.401956in}{3.304709in}}%
\pgfpathmoveto{\pgfqpoint{3.379252in}{3.316506in}}%
\pgfpathlineto{\pgfqpoint{3.379252in}{3.316506in}}%
\pgfpathlineto{\pgfqpoint{3.379252in}{3.319455in}}%
\pgfpathlineto{\pgfqpoint{3.383792in}{3.319455in}}%
\pgfpathlineto{\pgfqpoint{3.383792in}{3.316506in}}%
\pgfpathmoveto{\pgfqpoint{3.370170in}{3.322405in}}%
\pgfpathlineto{\pgfqpoint{3.370170in}{3.322405in}}%
\pgfpathlineto{\pgfqpoint{3.370170in}{3.325354in}}%
\pgfpathlineto{\pgfqpoint{3.374711in}{3.325354in}}%
\pgfpathlineto{\pgfqpoint{3.374711in}{3.322405in}}%
\pgfpathmoveto{\pgfqpoint{3.365629in}{3.325354in}}%
\pgfpathlineto{\pgfqpoint{3.365629in}{3.325354in}}%
\pgfpathlineto{\pgfqpoint{3.365629in}{3.328303in}}%
\pgfpathlineto{\pgfqpoint{3.370170in}{3.328303in}}%
\pgfpathlineto{\pgfqpoint{3.370170in}{3.325354in}}%
\pgfpathmoveto{\pgfqpoint{3.365629in}{3.328303in}}%
\pgfpathlineto{\pgfqpoint{3.365629in}{3.328303in}}%
\pgfpathlineto{\pgfqpoint{3.365629in}{3.331253in}}%
\pgfpathlineto{\pgfqpoint{3.370170in}{3.331253in}}%
\pgfpathlineto{\pgfqpoint{3.370170in}{3.328303in}}%
\pgfpathmoveto{\pgfqpoint{3.370170in}{3.325354in}}%
\pgfpathlineto{\pgfqpoint{3.370170in}{3.325354in}}%
\pgfpathlineto{\pgfqpoint{3.370170in}{3.328303in}}%
\pgfpathlineto{\pgfqpoint{3.374711in}{3.328303in}}%
\pgfpathlineto{\pgfqpoint{3.374711in}{3.325354in}}%
\pgfpathmoveto{\pgfqpoint{3.374711in}{3.319455in}}%
\pgfpathlineto{\pgfqpoint{3.374711in}{3.319455in}}%
\pgfpathlineto{\pgfqpoint{3.374711in}{3.322405in}}%
\pgfpathlineto{\pgfqpoint{3.379252in}{3.322405in}}%
\pgfpathlineto{\pgfqpoint{3.379252in}{3.319455in}}%
\pgfpathmoveto{\pgfqpoint{3.374711in}{3.322405in}}%
\pgfpathlineto{\pgfqpoint{3.374711in}{3.322405in}}%
\pgfpathlineto{\pgfqpoint{3.374711in}{3.325354in}}%
\pgfpathlineto{\pgfqpoint{3.379252in}{3.325354in}}%
\pgfpathlineto{\pgfqpoint{3.379252in}{3.322405in}}%
\pgfpathmoveto{\pgfqpoint{3.379252in}{3.319455in}}%
\pgfpathlineto{\pgfqpoint{3.379252in}{3.319455in}}%
\pgfpathlineto{\pgfqpoint{3.379252in}{3.322405in}}%
\pgfpathlineto{\pgfqpoint{3.383792in}{3.322405in}}%
\pgfpathlineto{\pgfqpoint{3.383792in}{3.319455in}}%
\pgfpathmoveto{\pgfqpoint{3.388333in}{3.310608in}}%
\pgfpathlineto{\pgfqpoint{3.388333in}{3.310608in}}%
\pgfpathlineto{\pgfqpoint{3.388333in}{3.313557in}}%
\pgfpathlineto{\pgfqpoint{3.392874in}{3.313557in}}%
\pgfpathlineto{\pgfqpoint{3.392874in}{3.310608in}}%
\pgfpathmoveto{\pgfqpoint{3.383792in}{3.313557in}}%
\pgfpathlineto{\pgfqpoint{3.383792in}{3.313557in}}%
\pgfpathlineto{\pgfqpoint{3.383792in}{3.316506in}}%
\pgfpathlineto{\pgfqpoint{3.388333in}{3.316506in}}%
\pgfpathlineto{\pgfqpoint{3.388333in}{3.313557in}}%
\pgfpathmoveto{\pgfqpoint{3.383792in}{3.316506in}}%
\pgfpathlineto{\pgfqpoint{3.383792in}{3.316506in}}%
\pgfpathlineto{\pgfqpoint{3.383792in}{3.319455in}}%
\pgfpathlineto{\pgfqpoint{3.388333in}{3.319455in}}%
\pgfpathlineto{\pgfqpoint{3.388333in}{3.316506in}}%
\pgfpathmoveto{\pgfqpoint{3.388333in}{3.313557in}}%
\pgfpathlineto{\pgfqpoint{3.388333in}{3.313557in}}%
\pgfpathlineto{\pgfqpoint{3.388333in}{3.316506in}}%
\pgfpathlineto{\pgfqpoint{3.392874in}{3.316506in}}%
\pgfpathlineto{\pgfqpoint{3.392874in}{3.313557in}}%
\pgfpathmoveto{\pgfqpoint{3.392874in}{3.307658in}}%
\pgfpathlineto{\pgfqpoint{3.392874in}{3.307658in}}%
\pgfpathlineto{\pgfqpoint{3.392874in}{3.310608in}}%
\pgfpathlineto{\pgfqpoint{3.397415in}{3.310608in}}%
\pgfpathlineto{\pgfqpoint{3.397415in}{3.307658in}}%
\pgfpathmoveto{\pgfqpoint{3.392874in}{3.310608in}}%
\pgfpathlineto{\pgfqpoint{3.392874in}{3.310608in}}%
\pgfpathlineto{\pgfqpoint{3.392874in}{3.313557in}}%
\pgfpathlineto{\pgfqpoint{3.397415in}{3.313557in}}%
\pgfpathlineto{\pgfqpoint{3.397415in}{3.310608in}}%
\pgfpathmoveto{\pgfqpoint{3.397415in}{3.307658in}}%
\pgfpathlineto{\pgfqpoint{3.397415in}{3.307658in}}%
\pgfpathlineto{\pgfqpoint{3.397415in}{3.310608in}}%
\pgfpathlineto{\pgfqpoint{3.401956in}{3.310608in}}%
\pgfpathlineto{\pgfqpoint{3.401956in}{3.307658in}}%
\pgfpathmoveto{\pgfqpoint{3.415578in}{3.292912in}}%
\pgfpathlineto{\pgfqpoint{3.415578in}{3.292912in}}%
\pgfpathlineto{\pgfqpoint{3.415578in}{3.295861in}}%
\pgfpathlineto{\pgfqpoint{3.420119in}{3.295861in}}%
\pgfpathlineto{\pgfqpoint{3.420119in}{3.292912in}}%
\pgfpathmoveto{\pgfqpoint{3.406496in}{3.298810in}}%
\pgfpathlineto{\pgfqpoint{3.406496in}{3.298810in}}%
\pgfpathlineto{\pgfqpoint{3.406496in}{3.301760in}}%
\pgfpathlineto{\pgfqpoint{3.411037in}{3.301760in}}%
\pgfpathlineto{\pgfqpoint{3.411037in}{3.298810in}}%
\pgfpathmoveto{\pgfqpoint{3.401956in}{3.301760in}}%
\pgfpathlineto{\pgfqpoint{3.401956in}{3.301760in}}%
\pgfpathlineto{\pgfqpoint{3.401956in}{3.304709in}}%
\pgfpathlineto{\pgfqpoint{3.406496in}{3.304709in}}%
\pgfpathlineto{\pgfqpoint{3.406496in}{3.301760in}}%
\pgfpathmoveto{\pgfqpoint{3.401956in}{3.304709in}}%
\pgfpathlineto{\pgfqpoint{3.401956in}{3.304709in}}%
\pgfpathlineto{\pgfqpoint{3.401956in}{3.307658in}}%
\pgfpathlineto{\pgfqpoint{3.406496in}{3.307658in}}%
\pgfpathlineto{\pgfqpoint{3.406496in}{3.304709in}}%
\pgfpathmoveto{\pgfqpoint{3.406496in}{3.301760in}}%
\pgfpathlineto{\pgfqpoint{3.406496in}{3.301760in}}%
\pgfpathlineto{\pgfqpoint{3.406496in}{3.304709in}}%
\pgfpathlineto{\pgfqpoint{3.411037in}{3.304709in}}%
\pgfpathlineto{\pgfqpoint{3.411037in}{3.301760in}}%
\pgfpathmoveto{\pgfqpoint{3.411037in}{3.295861in}}%
\pgfpathlineto{\pgfqpoint{3.411037in}{3.295861in}}%
\pgfpathlineto{\pgfqpoint{3.411037in}{3.298810in}}%
\pgfpathlineto{\pgfqpoint{3.415578in}{3.298810in}}%
\pgfpathlineto{\pgfqpoint{3.415578in}{3.295861in}}%
\pgfpathmoveto{\pgfqpoint{3.411037in}{3.298810in}}%
\pgfpathlineto{\pgfqpoint{3.411037in}{3.298810in}}%
\pgfpathlineto{\pgfqpoint{3.411037in}{3.301760in}}%
\pgfpathlineto{\pgfqpoint{3.415578in}{3.301760in}}%
\pgfpathlineto{\pgfqpoint{3.415578in}{3.298810in}}%
\pgfpathmoveto{\pgfqpoint{3.415578in}{3.295861in}}%
\pgfpathlineto{\pgfqpoint{3.415578in}{3.295861in}}%
\pgfpathlineto{\pgfqpoint{3.415578in}{3.298810in}}%
\pgfpathlineto{\pgfqpoint{3.420119in}{3.298810in}}%
\pgfpathlineto{\pgfqpoint{3.420119in}{3.295861in}}%
\pgfpathmoveto{\pgfqpoint{3.424659in}{3.287013in}}%
\pgfpathlineto{\pgfqpoint{3.424659in}{3.287013in}}%
\pgfpathlineto{\pgfqpoint{3.424659in}{3.289963in}}%
\pgfpathlineto{\pgfqpoint{3.429200in}{3.289963in}}%
\pgfpathlineto{\pgfqpoint{3.429200in}{3.287013in}}%
\pgfpathmoveto{\pgfqpoint{3.420119in}{3.289963in}}%
\pgfpathlineto{\pgfqpoint{3.420119in}{3.289963in}}%
\pgfpathlineto{\pgfqpoint{3.420119in}{3.292912in}}%
\pgfpathlineto{\pgfqpoint{3.424659in}{3.292912in}}%
\pgfpathlineto{\pgfqpoint{3.424659in}{3.289963in}}%
\pgfpathmoveto{\pgfqpoint{3.420119in}{3.292912in}}%
\pgfpathlineto{\pgfqpoint{3.420119in}{3.292912in}}%
\pgfpathlineto{\pgfqpoint{3.420119in}{3.295861in}}%
\pgfpathlineto{\pgfqpoint{3.424659in}{3.295861in}}%
\pgfpathlineto{\pgfqpoint{3.424659in}{3.292912in}}%
\pgfpathmoveto{\pgfqpoint{3.424659in}{3.289963in}}%
\pgfpathlineto{\pgfqpoint{3.424659in}{3.289963in}}%
\pgfpathlineto{\pgfqpoint{3.424659in}{3.292912in}}%
\pgfpathlineto{\pgfqpoint{3.429200in}{3.292912in}}%
\pgfpathlineto{\pgfqpoint{3.429200in}{3.289963in}}%
\pgfpathmoveto{\pgfqpoint{3.429200in}{3.284064in}}%
\pgfpathlineto{\pgfqpoint{3.429200in}{3.284064in}}%
\pgfpathlineto{\pgfqpoint{3.429200in}{3.287013in}}%
\pgfpathlineto{\pgfqpoint{3.433741in}{3.287013in}}%
\pgfpathlineto{\pgfqpoint{3.433741in}{3.284064in}}%
\pgfpathmoveto{\pgfqpoint{3.429200in}{3.287013in}}%
\pgfpathlineto{\pgfqpoint{3.429200in}{3.287013in}}%
\pgfpathlineto{\pgfqpoint{3.429200in}{3.289963in}}%
\pgfpathlineto{\pgfqpoint{3.433741in}{3.289963in}}%
\pgfpathlineto{\pgfqpoint{3.433741in}{3.287013in}}%
\pgfpathmoveto{\pgfqpoint{3.433741in}{3.284064in}}%
\pgfpathlineto{\pgfqpoint{3.433741in}{3.284064in}}%
\pgfpathlineto{\pgfqpoint{3.433741in}{3.287013in}}%
\pgfpathlineto{\pgfqpoint{3.438282in}{3.287013in}}%
\pgfpathlineto{\pgfqpoint{3.438282in}{3.284064in}}%
\pgfpathmoveto{\pgfqpoint{3.451904in}{3.269318in}}%
\pgfpathlineto{\pgfqpoint{3.451904in}{3.269318in}}%
\pgfpathlineto{\pgfqpoint{3.451904in}{3.272267in}}%
\pgfpathlineto{\pgfqpoint{3.456445in}{3.272267in}}%
\pgfpathlineto{\pgfqpoint{3.456445in}{3.269318in}}%
\pgfpathmoveto{\pgfqpoint{3.442823in}{3.275216in}}%
\pgfpathlineto{\pgfqpoint{3.442823in}{3.275216in}}%
\pgfpathlineto{\pgfqpoint{3.442823in}{3.278165in}}%
\pgfpathlineto{\pgfqpoint{3.447363in}{3.278165in}}%
\pgfpathlineto{\pgfqpoint{3.447363in}{3.275216in}}%
\pgfpathmoveto{\pgfqpoint{3.438282in}{3.278165in}}%
\pgfpathlineto{\pgfqpoint{3.438282in}{3.278165in}}%
\pgfpathlineto{\pgfqpoint{3.438282in}{3.281115in}}%
\pgfpathlineto{\pgfqpoint{3.442823in}{3.281115in}}%
\pgfpathlineto{\pgfqpoint{3.442823in}{3.278165in}}%
\pgfpathmoveto{\pgfqpoint{3.438282in}{3.281115in}}%
\pgfpathlineto{\pgfqpoint{3.438282in}{3.281115in}}%
\pgfpathlineto{\pgfqpoint{3.438282in}{3.284064in}}%
\pgfpathlineto{\pgfqpoint{3.442823in}{3.284064in}}%
\pgfpathlineto{\pgfqpoint{3.442823in}{3.281115in}}%
\pgfpathmoveto{\pgfqpoint{3.442823in}{3.278165in}}%
\pgfpathlineto{\pgfqpoint{3.442823in}{3.278165in}}%
\pgfpathlineto{\pgfqpoint{3.442823in}{3.281115in}}%
\pgfpathlineto{\pgfqpoint{3.447363in}{3.281115in}}%
\pgfpathlineto{\pgfqpoint{3.447363in}{3.278165in}}%
\pgfpathmoveto{\pgfqpoint{3.447363in}{3.272267in}}%
\pgfpathlineto{\pgfqpoint{3.447363in}{3.272267in}}%
\pgfpathlineto{\pgfqpoint{3.447363in}{3.275216in}}%
\pgfpathlineto{\pgfqpoint{3.451904in}{3.275216in}}%
\pgfpathlineto{\pgfqpoint{3.451904in}{3.272267in}}%
\pgfpathmoveto{\pgfqpoint{3.447363in}{3.275216in}}%
\pgfpathlineto{\pgfqpoint{3.447363in}{3.275216in}}%
\pgfpathlineto{\pgfqpoint{3.447363in}{3.278165in}}%
\pgfpathlineto{\pgfqpoint{3.451904in}{3.278165in}}%
\pgfpathlineto{\pgfqpoint{3.451904in}{3.275216in}}%
\pgfpathmoveto{\pgfqpoint{3.451904in}{3.272267in}}%
\pgfpathlineto{\pgfqpoint{3.451904in}{3.272267in}}%
\pgfpathlineto{\pgfqpoint{3.451904in}{3.275216in}}%
\pgfpathlineto{\pgfqpoint{3.456445in}{3.275216in}}%
\pgfpathlineto{\pgfqpoint{3.456445in}{3.272267in}}%
\pgfpathmoveto{\pgfqpoint{3.460986in}{3.263419in}}%
\pgfpathlineto{\pgfqpoint{3.460986in}{3.263419in}}%
\pgfpathlineto{\pgfqpoint{3.460986in}{3.266368in}}%
\pgfpathlineto{\pgfqpoint{3.465526in}{3.266368in}}%
\pgfpathlineto{\pgfqpoint{3.465526in}{3.263419in}}%
\pgfpathmoveto{\pgfqpoint{3.456445in}{3.266368in}}%
\pgfpathlineto{\pgfqpoint{3.456445in}{3.266368in}}%
\pgfpathlineto{\pgfqpoint{3.456445in}{3.269318in}}%
\pgfpathlineto{\pgfqpoint{3.460986in}{3.269318in}}%
\pgfpathlineto{\pgfqpoint{3.460986in}{3.266368in}}%
\pgfpathmoveto{\pgfqpoint{3.456445in}{3.269318in}}%
\pgfpathlineto{\pgfqpoint{3.456445in}{3.269318in}}%
\pgfpathlineto{\pgfqpoint{3.456445in}{3.272267in}}%
\pgfpathlineto{\pgfqpoint{3.460986in}{3.272267in}}%
\pgfpathlineto{\pgfqpoint{3.460986in}{3.269318in}}%
\pgfpathmoveto{\pgfqpoint{3.460986in}{3.266368in}}%
\pgfpathlineto{\pgfqpoint{3.460986in}{3.266368in}}%
\pgfpathlineto{\pgfqpoint{3.460986in}{3.269318in}}%
\pgfpathlineto{\pgfqpoint{3.465526in}{3.269318in}}%
\pgfpathlineto{\pgfqpoint{3.465526in}{3.266368in}}%
\pgfpathmoveto{\pgfqpoint{3.465526in}{3.260470in}}%
\pgfpathlineto{\pgfqpoint{3.465526in}{3.260470in}}%
\pgfpathlineto{\pgfqpoint{3.465526in}{3.263419in}}%
\pgfpathlineto{\pgfqpoint{3.470067in}{3.263419in}}%
\pgfpathlineto{\pgfqpoint{3.470067in}{3.260470in}}%
\pgfpathmoveto{\pgfqpoint{3.465526in}{3.263419in}}%
\pgfpathlineto{\pgfqpoint{3.465526in}{3.263419in}}%
\pgfpathlineto{\pgfqpoint{3.465526in}{3.266368in}}%
\pgfpathlineto{\pgfqpoint{3.470067in}{3.266368in}}%
\pgfpathlineto{\pgfqpoint{3.470067in}{3.263419in}}%
\pgfpathmoveto{\pgfqpoint{3.470067in}{3.260470in}}%
\pgfpathlineto{\pgfqpoint{3.470067in}{3.260470in}}%
\pgfpathlineto{\pgfqpoint{3.470067in}{3.263419in}}%
\pgfpathlineto{\pgfqpoint{3.474608in}{3.263419in}}%
\pgfpathlineto{\pgfqpoint{3.474608in}{3.260470in}}%
\pgfpathmoveto{\pgfqpoint{3.474608in}{3.257520in}}%
\pgfpathlineto{\pgfqpoint{3.474608in}{3.257520in}}%
\pgfpathlineto{\pgfqpoint{3.474608in}{3.260470in}}%
\pgfpathlineto{\pgfqpoint{3.479149in}{3.260470in}}%
\pgfpathlineto{\pgfqpoint{3.479149in}{3.257520in}}%
\pgfpathmoveto{\pgfqpoint{3.479149in}{3.254571in}}%
\pgfpathlineto{\pgfqpoint{3.479149in}{3.254571in}}%
\pgfpathlineto{\pgfqpoint{3.479149in}{3.257520in}}%
\pgfpathlineto{\pgfqpoint{3.483689in}{3.257520in}}%
\pgfpathlineto{\pgfqpoint{3.483689in}{3.254571in}}%
\pgfpathmoveto{\pgfqpoint{3.479149in}{3.257520in}}%
\pgfpathlineto{\pgfqpoint{3.479149in}{3.257520in}}%
\pgfpathlineto{\pgfqpoint{3.479149in}{3.260470in}}%
\pgfpathlineto{\pgfqpoint{3.483689in}{3.260470in}}%
\pgfpathlineto{\pgfqpoint{3.483689in}{3.257520in}}%
\pgfpathmoveto{\pgfqpoint{3.483689in}{3.251622in}}%
\pgfpathlineto{\pgfqpoint{3.483689in}{3.251622in}}%
\pgfpathlineto{\pgfqpoint{3.483689in}{3.254571in}}%
\pgfpathlineto{\pgfqpoint{3.488230in}{3.254571in}}%
\pgfpathlineto{\pgfqpoint{3.488230in}{3.251622in}}%
\pgfpathmoveto{\pgfqpoint{3.488230in}{3.248673in}}%
\pgfpathlineto{\pgfqpoint{3.488230in}{3.248673in}}%
\pgfpathlineto{\pgfqpoint{3.488230in}{3.251622in}}%
\pgfpathlineto{\pgfqpoint{3.492771in}{3.251622in}}%
\pgfpathlineto{\pgfqpoint{3.492771in}{3.248673in}}%
\pgfpathmoveto{\pgfqpoint{3.488230in}{3.251622in}}%
\pgfpathlineto{\pgfqpoint{3.488230in}{3.251622in}}%
\pgfpathlineto{\pgfqpoint{3.488230in}{3.254571in}}%
\pgfpathlineto{\pgfqpoint{3.492771in}{3.254571in}}%
\pgfpathlineto{\pgfqpoint{3.492771in}{3.251622in}}%
\pgfpathmoveto{\pgfqpoint{3.483689in}{3.254571in}}%
\pgfpathlineto{\pgfqpoint{3.483689in}{3.254571in}}%
\pgfpathlineto{\pgfqpoint{3.483689in}{3.257520in}}%
\pgfpathlineto{\pgfqpoint{3.488230in}{3.257520in}}%
\pgfpathlineto{\pgfqpoint{3.488230in}{3.254571in}}%
\pgfpathmoveto{\pgfqpoint{3.492771in}{3.245723in}}%
\pgfpathlineto{\pgfqpoint{3.492771in}{3.245723in}}%
\pgfpathlineto{\pgfqpoint{3.492771in}{3.248673in}}%
\pgfpathlineto{\pgfqpoint{3.497312in}{3.248673in}}%
\pgfpathlineto{\pgfqpoint{3.497312in}{3.245723in}}%
\pgfpathmoveto{\pgfqpoint{3.497312in}{3.242774in}}%
\pgfpathlineto{\pgfqpoint{3.497312in}{3.242774in}}%
\pgfpathlineto{\pgfqpoint{3.497312in}{3.245723in}}%
\pgfpathlineto{\pgfqpoint{3.501853in}{3.245723in}}%
\pgfpathlineto{\pgfqpoint{3.501853in}{3.242774in}}%
\pgfpathmoveto{\pgfqpoint{3.497312in}{3.245723in}}%
\pgfpathlineto{\pgfqpoint{3.497312in}{3.245723in}}%
\pgfpathlineto{\pgfqpoint{3.497312in}{3.248673in}}%
\pgfpathlineto{\pgfqpoint{3.501853in}{3.248673in}}%
\pgfpathlineto{\pgfqpoint{3.501853in}{3.245723in}}%
\pgfpathmoveto{\pgfqpoint{3.501853in}{3.239825in}}%
\pgfpathlineto{\pgfqpoint{3.501853in}{3.239825in}}%
\pgfpathlineto{\pgfqpoint{3.501853in}{3.242774in}}%
\pgfpathlineto{\pgfqpoint{3.506393in}{3.242774in}}%
\pgfpathlineto{\pgfqpoint{3.506393in}{3.239825in}}%
\pgfpathmoveto{\pgfqpoint{3.506393in}{3.236875in}}%
\pgfpathlineto{\pgfqpoint{3.506393in}{3.236875in}}%
\pgfpathlineto{\pgfqpoint{3.506393in}{3.239825in}}%
\pgfpathlineto{\pgfqpoint{3.510934in}{3.239825in}}%
\pgfpathlineto{\pgfqpoint{3.510934in}{3.236875in}}%
\pgfpathmoveto{\pgfqpoint{3.506393in}{3.239825in}}%
\pgfpathlineto{\pgfqpoint{3.506393in}{3.239825in}}%
\pgfpathlineto{\pgfqpoint{3.506393in}{3.242774in}}%
\pgfpathlineto{\pgfqpoint{3.510934in}{3.242774in}}%
\pgfpathlineto{\pgfqpoint{3.510934in}{3.239825in}}%
\pgfpathmoveto{\pgfqpoint{3.501853in}{3.242774in}}%
\pgfpathlineto{\pgfqpoint{3.501853in}{3.242774in}}%
\pgfpathlineto{\pgfqpoint{3.501853in}{3.245723in}}%
\pgfpathlineto{\pgfqpoint{3.506393in}{3.245723in}}%
\pgfpathlineto{\pgfqpoint{3.506393in}{3.242774in}}%
\pgfpathmoveto{\pgfqpoint{3.492771in}{3.248673in}}%
\pgfpathlineto{\pgfqpoint{3.492771in}{3.248673in}}%
\pgfpathlineto{\pgfqpoint{3.492771in}{3.251622in}}%
\pgfpathlineto{\pgfqpoint{3.497312in}{3.251622in}}%
\pgfpathlineto{\pgfqpoint{3.497312in}{3.248673in}}%
\pgfpathmoveto{\pgfqpoint{3.474608in}{3.260470in}}%
\pgfpathlineto{\pgfqpoint{3.474608in}{3.260470in}}%
\pgfpathlineto{\pgfqpoint{3.474608in}{3.263419in}}%
\pgfpathlineto{\pgfqpoint{3.479149in}{3.263419in}}%
\pgfpathlineto{\pgfqpoint{3.479149in}{3.260470in}}%
\pgfpathmoveto{\pgfqpoint{3.510934in}{3.233926in}}%
\pgfpathlineto{\pgfqpoint{3.510934in}{3.233926in}}%
\pgfpathlineto{\pgfqpoint{3.510934in}{3.236875in}}%
\pgfpathlineto{\pgfqpoint{3.515475in}{3.236875in}}%
\pgfpathlineto{\pgfqpoint{3.515475in}{3.233926in}}%
\pgfpathmoveto{\pgfqpoint{3.515475in}{3.230977in}}%
\pgfpathlineto{\pgfqpoint{3.515475in}{3.230977in}}%
\pgfpathlineto{\pgfqpoint{3.515475in}{3.233926in}}%
\pgfpathlineto{\pgfqpoint{3.520016in}{3.233926in}}%
\pgfpathlineto{\pgfqpoint{3.520016in}{3.230977in}}%
\pgfpathmoveto{\pgfqpoint{3.515475in}{3.233926in}}%
\pgfpathlineto{\pgfqpoint{3.515475in}{3.233926in}}%
\pgfpathlineto{\pgfqpoint{3.515475in}{3.236875in}}%
\pgfpathlineto{\pgfqpoint{3.520016in}{3.236875in}}%
\pgfpathlineto{\pgfqpoint{3.520016in}{3.233926in}}%
\pgfpathmoveto{\pgfqpoint{3.520016in}{3.228028in}}%
\pgfpathlineto{\pgfqpoint{3.520016in}{3.228028in}}%
\pgfpathlineto{\pgfqpoint{3.520016in}{3.230977in}}%
\pgfpathlineto{\pgfqpoint{3.524557in}{3.230977in}}%
\pgfpathlineto{\pgfqpoint{3.524557in}{3.228028in}}%
\pgfpathmoveto{\pgfqpoint{3.524557in}{3.225078in}}%
\pgfpathlineto{\pgfqpoint{3.524557in}{3.225078in}}%
\pgfpathlineto{\pgfqpoint{3.524557in}{3.228028in}}%
\pgfpathlineto{\pgfqpoint{3.529098in}{3.228028in}}%
\pgfpathlineto{\pgfqpoint{3.529098in}{3.225078in}}%
\pgfpathmoveto{\pgfqpoint{3.524557in}{3.228028in}}%
\pgfpathlineto{\pgfqpoint{3.524557in}{3.228028in}}%
\pgfpathlineto{\pgfqpoint{3.524557in}{3.230977in}}%
\pgfpathlineto{\pgfqpoint{3.529098in}{3.230977in}}%
\pgfpathlineto{\pgfqpoint{3.529098in}{3.228028in}}%
\pgfpathmoveto{\pgfqpoint{3.520016in}{3.230977in}}%
\pgfpathlineto{\pgfqpoint{3.520016in}{3.230977in}}%
\pgfpathlineto{\pgfqpoint{3.520016in}{3.233926in}}%
\pgfpathlineto{\pgfqpoint{3.524557in}{3.233926in}}%
\pgfpathlineto{\pgfqpoint{3.524557in}{3.230977in}}%
\pgfpathmoveto{\pgfqpoint{3.529098in}{3.222129in}}%
\pgfpathlineto{\pgfqpoint{3.529098in}{3.222129in}}%
\pgfpathlineto{\pgfqpoint{3.529098in}{3.225078in}}%
\pgfpathlineto{\pgfqpoint{3.533639in}{3.225078in}}%
\pgfpathlineto{\pgfqpoint{3.533639in}{3.222129in}}%
\pgfpathmoveto{\pgfqpoint{3.533639in}{3.219180in}}%
\pgfpathlineto{\pgfqpoint{3.533639in}{3.219180in}}%
\pgfpathlineto{\pgfqpoint{3.533639in}{3.222129in}}%
\pgfpathlineto{\pgfqpoint{3.538180in}{3.222129in}}%
\pgfpathlineto{\pgfqpoint{3.538180in}{3.219180in}}%
\pgfpathmoveto{\pgfqpoint{3.533639in}{3.222129in}}%
\pgfpathlineto{\pgfqpoint{3.533639in}{3.222129in}}%
\pgfpathlineto{\pgfqpoint{3.533639in}{3.225078in}}%
\pgfpathlineto{\pgfqpoint{3.538180in}{3.225078in}}%
\pgfpathlineto{\pgfqpoint{3.538180in}{3.222129in}}%
\pgfpathmoveto{\pgfqpoint{3.538180in}{3.216230in}}%
\pgfpathlineto{\pgfqpoint{3.538180in}{3.216230in}}%
\pgfpathlineto{\pgfqpoint{3.538180in}{3.219180in}}%
\pgfpathlineto{\pgfqpoint{3.542721in}{3.219180in}}%
\pgfpathlineto{\pgfqpoint{3.542721in}{3.216230in}}%
\pgfpathmoveto{\pgfqpoint{3.542721in}{3.213281in}}%
\pgfpathlineto{\pgfqpoint{3.542721in}{3.213281in}}%
\pgfpathlineto{\pgfqpoint{3.542721in}{3.216230in}}%
\pgfpathlineto{\pgfqpoint{3.547262in}{3.216230in}}%
\pgfpathlineto{\pgfqpoint{3.547262in}{3.213281in}}%
\pgfpathmoveto{\pgfqpoint{3.542721in}{3.216230in}}%
\pgfpathlineto{\pgfqpoint{3.542721in}{3.216230in}}%
\pgfpathlineto{\pgfqpoint{3.542721in}{3.219180in}}%
\pgfpathlineto{\pgfqpoint{3.547262in}{3.219180in}}%
\pgfpathlineto{\pgfqpoint{3.547262in}{3.216230in}}%
\pgfpathmoveto{\pgfqpoint{3.538180in}{3.219180in}}%
\pgfpathlineto{\pgfqpoint{3.538180in}{3.219180in}}%
\pgfpathlineto{\pgfqpoint{3.538180in}{3.222129in}}%
\pgfpathlineto{\pgfqpoint{3.542721in}{3.222129in}}%
\pgfpathlineto{\pgfqpoint{3.542721in}{3.219180in}}%
\pgfpathmoveto{\pgfqpoint{3.529098in}{3.225078in}}%
\pgfpathlineto{\pgfqpoint{3.529098in}{3.225078in}}%
\pgfpathlineto{\pgfqpoint{3.529098in}{3.228028in}}%
\pgfpathlineto{\pgfqpoint{3.533639in}{3.228028in}}%
\pgfpathlineto{\pgfqpoint{3.533639in}{3.225078in}}%
\pgfpathmoveto{\pgfqpoint{3.547262in}{3.210332in}}%
\pgfpathlineto{\pgfqpoint{3.547262in}{3.210332in}}%
\pgfpathlineto{\pgfqpoint{3.547262in}{3.213281in}}%
\pgfpathlineto{\pgfqpoint{3.551803in}{3.213281in}}%
\pgfpathlineto{\pgfqpoint{3.551803in}{3.210332in}}%
\pgfpathmoveto{\pgfqpoint{3.551803in}{3.207383in}}%
\pgfpathlineto{\pgfqpoint{3.551803in}{3.207383in}}%
\pgfpathlineto{\pgfqpoint{3.551803in}{3.210332in}}%
\pgfpathlineto{\pgfqpoint{3.556344in}{3.210332in}}%
\pgfpathlineto{\pgfqpoint{3.556344in}{3.207383in}}%
\pgfpathmoveto{\pgfqpoint{3.551803in}{3.210332in}}%
\pgfpathlineto{\pgfqpoint{3.551803in}{3.210332in}}%
\pgfpathlineto{\pgfqpoint{3.551803in}{3.213281in}}%
\pgfpathlineto{\pgfqpoint{3.556344in}{3.213281in}}%
\pgfpathlineto{\pgfqpoint{3.556344in}{3.210332in}}%
\pgfpathmoveto{\pgfqpoint{3.556344in}{3.204433in}}%
\pgfpathlineto{\pgfqpoint{3.556344in}{3.204433in}}%
\pgfpathlineto{\pgfqpoint{3.556344in}{3.207383in}}%
\pgfpathlineto{\pgfqpoint{3.560885in}{3.207383in}}%
\pgfpathlineto{\pgfqpoint{3.560885in}{3.204433in}}%
\pgfpathmoveto{\pgfqpoint{3.560885in}{3.201484in}}%
\pgfpathlineto{\pgfqpoint{3.560885in}{3.201484in}}%
\pgfpathlineto{\pgfqpoint{3.560885in}{3.204433in}}%
\pgfpathlineto{\pgfqpoint{3.565426in}{3.204433in}}%
\pgfpathlineto{\pgfqpoint{3.565426in}{3.201484in}}%
\pgfpathmoveto{\pgfqpoint{3.560885in}{3.204433in}}%
\pgfpathlineto{\pgfqpoint{3.560885in}{3.204433in}}%
\pgfpathlineto{\pgfqpoint{3.560885in}{3.207383in}}%
\pgfpathlineto{\pgfqpoint{3.565426in}{3.207383in}}%
\pgfpathlineto{\pgfqpoint{3.565426in}{3.204433in}}%
\pgfpathmoveto{\pgfqpoint{3.556344in}{3.207383in}}%
\pgfpathlineto{\pgfqpoint{3.556344in}{3.207383in}}%
\pgfpathlineto{\pgfqpoint{3.556344in}{3.210332in}}%
\pgfpathlineto{\pgfqpoint{3.560885in}{3.210332in}}%
\pgfpathlineto{\pgfqpoint{3.560885in}{3.207383in}}%
\pgfpathmoveto{\pgfqpoint{3.565426in}{3.198535in}}%
\pgfpathlineto{\pgfqpoint{3.565426in}{3.198535in}}%
\pgfpathlineto{\pgfqpoint{3.565426in}{3.201484in}}%
\pgfpathlineto{\pgfqpoint{3.569967in}{3.201484in}}%
\pgfpathlineto{\pgfqpoint{3.569967in}{3.198535in}}%
\pgfpathmoveto{\pgfqpoint{3.569967in}{3.195586in}}%
\pgfpathlineto{\pgfqpoint{3.569967in}{3.195586in}}%
\pgfpathlineto{\pgfqpoint{3.569967in}{3.198535in}}%
\pgfpathlineto{\pgfqpoint{3.574508in}{3.198535in}}%
\pgfpathlineto{\pgfqpoint{3.574508in}{3.195586in}}%
\pgfpathmoveto{\pgfqpoint{3.569967in}{3.198535in}}%
\pgfpathlineto{\pgfqpoint{3.569967in}{3.198535in}}%
\pgfpathlineto{\pgfqpoint{3.569967in}{3.201484in}}%
\pgfpathlineto{\pgfqpoint{3.574508in}{3.201484in}}%
\pgfpathlineto{\pgfqpoint{3.574508in}{3.198535in}}%
\pgfpathmoveto{\pgfqpoint{3.574508in}{3.192636in}}%
\pgfpathlineto{\pgfqpoint{3.574508in}{3.192636in}}%
\pgfpathlineto{\pgfqpoint{3.574508in}{3.195586in}}%
\pgfpathlineto{\pgfqpoint{3.579050in}{3.195586in}}%
\pgfpathlineto{\pgfqpoint{3.579050in}{3.192636in}}%
\pgfpathmoveto{\pgfqpoint{3.579050in}{3.189687in}}%
\pgfpathlineto{\pgfqpoint{3.579050in}{3.189687in}}%
\pgfpathlineto{\pgfqpoint{3.579050in}{3.192636in}}%
\pgfpathlineto{\pgfqpoint{3.583591in}{3.192636in}}%
\pgfpathlineto{\pgfqpoint{3.583591in}{3.189687in}}%
\pgfpathmoveto{\pgfqpoint{3.579050in}{3.192636in}}%
\pgfpathlineto{\pgfqpoint{3.579050in}{3.192636in}}%
\pgfpathlineto{\pgfqpoint{3.579050in}{3.195586in}}%
\pgfpathlineto{\pgfqpoint{3.583591in}{3.195586in}}%
\pgfpathlineto{\pgfqpoint{3.583591in}{3.192636in}}%
\pgfpathmoveto{\pgfqpoint{3.574508in}{3.195586in}}%
\pgfpathlineto{\pgfqpoint{3.574508in}{3.195586in}}%
\pgfpathlineto{\pgfqpoint{3.574508in}{3.198535in}}%
\pgfpathlineto{\pgfqpoint{3.579050in}{3.198535in}}%
\pgfpathlineto{\pgfqpoint{3.579050in}{3.195586in}}%
\pgfpathmoveto{\pgfqpoint{3.565426in}{3.201484in}}%
\pgfpathlineto{\pgfqpoint{3.565426in}{3.201484in}}%
\pgfpathlineto{\pgfqpoint{3.565426in}{3.204433in}}%
\pgfpathlineto{\pgfqpoint{3.569967in}{3.204433in}}%
\pgfpathlineto{\pgfqpoint{3.569967in}{3.201484in}}%
\pgfpathmoveto{\pgfqpoint{3.547262in}{3.213281in}}%
\pgfpathlineto{\pgfqpoint{3.547262in}{3.213281in}}%
\pgfpathlineto{\pgfqpoint{3.547262in}{3.216230in}}%
\pgfpathlineto{\pgfqpoint{3.551803in}{3.216230in}}%
\pgfpathlineto{\pgfqpoint{3.551803in}{3.213281in}}%
\pgfpathmoveto{\pgfqpoint{3.583591in}{3.186738in}}%
\pgfpathlineto{\pgfqpoint{3.583591in}{3.186738in}}%
\pgfpathlineto{\pgfqpoint{3.583591in}{3.189687in}}%
\pgfpathlineto{\pgfqpoint{3.588132in}{3.189687in}}%
\pgfpathlineto{\pgfqpoint{3.588132in}{3.186738in}}%
\pgfpathmoveto{\pgfqpoint{3.588132in}{3.183789in}}%
\pgfpathlineto{\pgfqpoint{3.588132in}{3.183789in}}%
\pgfpathlineto{\pgfqpoint{3.588132in}{3.186738in}}%
\pgfpathlineto{\pgfqpoint{3.592673in}{3.186738in}}%
\pgfpathlineto{\pgfqpoint{3.592673in}{3.183789in}}%
\pgfpathmoveto{\pgfqpoint{3.588132in}{3.186738in}}%
\pgfpathlineto{\pgfqpoint{3.588132in}{3.186738in}}%
\pgfpathlineto{\pgfqpoint{3.588132in}{3.189687in}}%
\pgfpathlineto{\pgfqpoint{3.592673in}{3.189687in}}%
\pgfpathlineto{\pgfqpoint{3.592673in}{3.186738in}}%
\pgfpathmoveto{\pgfqpoint{3.592673in}{3.180839in}}%
\pgfpathlineto{\pgfqpoint{3.592673in}{3.180839in}}%
\pgfpathlineto{\pgfqpoint{3.592673in}{3.183789in}}%
\pgfpathlineto{\pgfqpoint{3.597214in}{3.183789in}}%
\pgfpathlineto{\pgfqpoint{3.597214in}{3.180839in}}%
\pgfpathmoveto{\pgfqpoint{3.597214in}{3.177890in}}%
\pgfpathlineto{\pgfqpoint{3.597214in}{3.177890in}}%
\pgfpathlineto{\pgfqpoint{3.597214in}{3.180839in}}%
\pgfpathlineto{\pgfqpoint{3.601755in}{3.180839in}}%
\pgfpathlineto{\pgfqpoint{3.601755in}{3.177890in}}%
\pgfpathmoveto{\pgfqpoint{3.597214in}{3.180839in}}%
\pgfpathlineto{\pgfqpoint{3.597214in}{3.180839in}}%
\pgfpathlineto{\pgfqpoint{3.597214in}{3.183789in}}%
\pgfpathlineto{\pgfqpoint{3.601755in}{3.183789in}}%
\pgfpathlineto{\pgfqpoint{3.601755in}{3.180839in}}%
\pgfpathmoveto{\pgfqpoint{3.592673in}{3.183789in}}%
\pgfpathlineto{\pgfqpoint{3.592673in}{3.183789in}}%
\pgfpathlineto{\pgfqpoint{3.592673in}{3.186738in}}%
\pgfpathlineto{\pgfqpoint{3.597214in}{3.186738in}}%
\pgfpathlineto{\pgfqpoint{3.597214in}{3.183789in}}%
\pgfpathmoveto{\pgfqpoint{3.601755in}{3.174941in}}%
\pgfpathlineto{\pgfqpoint{3.601755in}{3.174941in}}%
\pgfpathlineto{\pgfqpoint{3.601755in}{3.177890in}}%
\pgfpathlineto{\pgfqpoint{3.606296in}{3.177890in}}%
\pgfpathlineto{\pgfqpoint{3.606296in}{3.174941in}}%
\pgfpathmoveto{\pgfqpoint{3.606296in}{3.171991in}}%
\pgfpathlineto{\pgfqpoint{3.606296in}{3.171991in}}%
\pgfpathlineto{\pgfqpoint{3.606296in}{3.174941in}}%
\pgfpathlineto{\pgfqpoint{3.610837in}{3.174941in}}%
\pgfpathlineto{\pgfqpoint{3.610837in}{3.171991in}}%
\pgfpathmoveto{\pgfqpoint{3.606296in}{3.174941in}}%
\pgfpathlineto{\pgfqpoint{3.606296in}{3.174941in}}%
\pgfpathlineto{\pgfqpoint{3.606296in}{3.177890in}}%
\pgfpathlineto{\pgfqpoint{3.610837in}{3.177890in}}%
\pgfpathlineto{\pgfqpoint{3.610837in}{3.174941in}}%
\pgfpathmoveto{\pgfqpoint{3.610837in}{3.169042in}}%
\pgfpathlineto{\pgfqpoint{3.610837in}{3.169042in}}%
\pgfpathlineto{\pgfqpoint{3.610837in}{3.171991in}}%
\pgfpathlineto{\pgfqpoint{3.615378in}{3.171991in}}%
\pgfpathlineto{\pgfqpoint{3.615378in}{3.169042in}}%
\pgfpathmoveto{\pgfqpoint{3.615378in}{3.166093in}}%
\pgfpathlineto{\pgfqpoint{3.615378in}{3.166093in}}%
\pgfpathlineto{\pgfqpoint{3.615378in}{3.169042in}}%
\pgfpathlineto{\pgfqpoint{3.619919in}{3.169042in}}%
\pgfpathlineto{\pgfqpoint{3.619919in}{3.166093in}}%
\pgfpathmoveto{\pgfqpoint{3.615378in}{3.169042in}}%
\pgfpathlineto{\pgfqpoint{3.615378in}{3.169042in}}%
\pgfpathlineto{\pgfqpoint{3.615378in}{3.171991in}}%
\pgfpathlineto{\pgfqpoint{3.619919in}{3.171991in}}%
\pgfpathlineto{\pgfqpoint{3.619919in}{3.169042in}}%
\pgfpathmoveto{\pgfqpoint{3.610837in}{3.171991in}}%
\pgfpathlineto{\pgfqpoint{3.610837in}{3.171991in}}%
\pgfpathlineto{\pgfqpoint{3.610837in}{3.174941in}}%
\pgfpathlineto{\pgfqpoint{3.615378in}{3.174941in}}%
\pgfpathlineto{\pgfqpoint{3.615378in}{3.171991in}}%
\pgfpathmoveto{\pgfqpoint{3.601755in}{3.177890in}}%
\pgfpathlineto{\pgfqpoint{3.601755in}{3.177890in}}%
\pgfpathlineto{\pgfqpoint{3.601755in}{3.180839in}}%
\pgfpathlineto{\pgfqpoint{3.606296in}{3.180839in}}%
\pgfpathlineto{\pgfqpoint{3.606296in}{3.177890in}}%
\pgfpathmoveto{\pgfqpoint{3.619919in}{3.163144in}}%
\pgfpathlineto{\pgfqpoint{3.619919in}{3.163144in}}%
\pgfpathlineto{\pgfqpoint{3.619919in}{3.166093in}}%
\pgfpathlineto{\pgfqpoint{3.624460in}{3.166093in}}%
\pgfpathlineto{\pgfqpoint{3.624460in}{3.163144in}}%
\pgfpathmoveto{\pgfqpoint{3.624460in}{3.160194in}}%
\pgfpathlineto{\pgfqpoint{3.624460in}{3.160194in}}%
\pgfpathlineto{\pgfqpoint{3.624460in}{3.163144in}}%
\pgfpathlineto{\pgfqpoint{3.629001in}{3.163144in}}%
\pgfpathlineto{\pgfqpoint{3.629001in}{3.160194in}}%
\pgfpathmoveto{\pgfqpoint{3.624460in}{3.163144in}}%
\pgfpathlineto{\pgfqpoint{3.624460in}{3.163144in}}%
\pgfpathlineto{\pgfqpoint{3.624460in}{3.166093in}}%
\pgfpathlineto{\pgfqpoint{3.629001in}{3.166093in}}%
\pgfpathlineto{\pgfqpoint{3.629001in}{3.163144in}}%
\pgfpathmoveto{\pgfqpoint{3.629001in}{3.157245in}}%
\pgfpathlineto{\pgfqpoint{3.629001in}{3.157245in}}%
\pgfpathlineto{\pgfqpoint{3.629001in}{3.160194in}}%
\pgfpathlineto{\pgfqpoint{3.633542in}{3.160194in}}%
\pgfpathlineto{\pgfqpoint{3.633542in}{3.157245in}}%
\pgfpathmoveto{\pgfqpoint{3.633542in}{3.154296in}}%
\pgfpathlineto{\pgfqpoint{3.633542in}{3.154296in}}%
\pgfpathlineto{\pgfqpoint{3.633542in}{3.157245in}}%
\pgfpathlineto{\pgfqpoint{3.638083in}{3.157245in}}%
\pgfpathlineto{\pgfqpoint{3.638083in}{3.154296in}}%
\pgfpathmoveto{\pgfqpoint{3.633542in}{3.157245in}}%
\pgfpathlineto{\pgfqpoint{3.633542in}{3.157245in}}%
\pgfpathlineto{\pgfqpoint{3.633542in}{3.160194in}}%
\pgfpathlineto{\pgfqpoint{3.638083in}{3.160194in}}%
\pgfpathlineto{\pgfqpoint{3.638083in}{3.157245in}}%
\pgfpathmoveto{\pgfqpoint{3.629001in}{3.160194in}}%
\pgfpathlineto{\pgfqpoint{3.629001in}{3.160194in}}%
\pgfpathlineto{\pgfqpoint{3.629001in}{3.163144in}}%
\pgfpathlineto{\pgfqpoint{3.633542in}{3.163144in}}%
\pgfpathlineto{\pgfqpoint{3.633542in}{3.160194in}}%
\pgfpathmoveto{\pgfqpoint{3.638083in}{3.151347in}}%
\pgfpathlineto{\pgfqpoint{3.638083in}{3.151347in}}%
\pgfpathlineto{\pgfqpoint{3.638083in}{3.154296in}}%
\pgfpathlineto{\pgfqpoint{3.642624in}{3.154296in}}%
\pgfpathlineto{\pgfqpoint{3.642624in}{3.151347in}}%
\pgfpathmoveto{\pgfqpoint{3.642624in}{3.148397in}}%
\pgfpathlineto{\pgfqpoint{3.642624in}{3.148397in}}%
\pgfpathlineto{\pgfqpoint{3.642624in}{3.151347in}}%
\pgfpathlineto{\pgfqpoint{3.647165in}{3.151347in}}%
\pgfpathlineto{\pgfqpoint{3.647165in}{3.148397in}}%
\pgfpathmoveto{\pgfqpoint{3.642624in}{3.151347in}}%
\pgfpathlineto{\pgfqpoint{3.642624in}{3.151347in}}%
\pgfpathlineto{\pgfqpoint{3.642624in}{3.154296in}}%
\pgfpathlineto{\pgfqpoint{3.647165in}{3.154296in}}%
\pgfpathlineto{\pgfqpoint{3.647165in}{3.151347in}}%
\pgfpathmoveto{\pgfqpoint{3.647165in}{3.145448in}}%
\pgfpathlineto{\pgfqpoint{3.647165in}{3.145448in}}%
\pgfpathlineto{\pgfqpoint{3.647165in}{3.148397in}}%
\pgfpathlineto{\pgfqpoint{3.651706in}{3.148397in}}%
\pgfpathlineto{\pgfqpoint{3.651706in}{3.145448in}}%
\pgfpathmoveto{\pgfqpoint{3.651706in}{3.142499in}}%
\pgfpathlineto{\pgfqpoint{3.651706in}{3.142499in}}%
\pgfpathlineto{\pgfqpoint{3.651706in}{3.145448in}}%
\pgfpathlineto{\pgfqpoint{3.656247in}{3.145448in}}%
\pgfpathlineto{\pgfqpoint{3.656247in}{3.142499in}}%
\pgfpathmoveto{\pgfqpoint{3.651706in}{3.145448in}}%
\pgfpathlineto{\pgfqpoint{3.651706in}{3.145448in}}%
\pgfpathlineto{\pgfqpoint{3.651706in}{3.148397in}}%
\pgfpathlineto{\pgfqpoint{3.656247in}{3.148397in}}%
\pgfpathlineto{\pgfqpoint{3.656247in}{3.145448in}}%
\pgfpathmoveto{\pgfqpoint{3.647165in}{3.148397in}}%
\pgfpathlineto{\pgfqpoint{3.647165in}{3.148397in}}%
\pgfpathlineto{\pgfqpoint{3.647165in}{3.151347in}}%
\pgfpathlineto{\pgfqpoint{3.651706in}{3.151347in}}%
\pgfpathlineto{\pgfqpoint{3.651706in}{3.148397in}}%
\pgfpathmoveto{\pgfqpoint{3.638083in}{3.154296in}}%
\pgfpathlineto{\pgfqpoint{3.638083in}{3.154296in}}%
\pgfpathlineto{\pgfqpoint{3.638083in}{3.157245in}}%
\pgfpathlineto{\pgfqpoint{3.642624in}{3.157245in}}%
\pgfpathlineto{\pgfqpoint{3.642624in}{3.154296in}}%
\pgfpathmoveto{\pgfqpoint{3.619919in}{3.166093in}}%
\pgfpathlineto{\pgfqpoint{3.619919in}{3.166093in}}%
\pgfpathlineto{\pgfqpoint{3.619919in}{3.169042in}}%
\pgfpathlineto{\pgfqpoint{3.624460in}{3.169042in}}%
\pgfpathlineto{\pgfqpoint{3.624460in}{3.166093in}}%
\pgfpathmoveto{\pgfqpoint{3.583591in}{3.189687in}}%
\pgfpathlineto{\pgfqpoint{3.583591in}{3.189687in}}%
\pgfpathlineto{\pgfqpoint{3.583591in}{3.192636in}}%
\pgfpathlineto{\pgfqpoint{3.588132in}{3.192636in}}%
\pgfpathlineto{\pgfqpoint{3.588132in}{3.189687in}}%
\pgfpathmoveto{\pgfqpoint{3.510934in}{3.236875in}}%
\pgfpathlineto{\pgfqpoint{3.510934in}{3.236875in}}%
\pgfpathlineto{\pgfqpoint{3.510934in}{3.239825in}}%
\pgfpathlineto{\pgfqpoint{3.515475in}{3.239825in}}%
\pgfpathlineto{\pgfqpoint{3.515475in}{3.236875in}}%
\pgfpathmoveto{\pgfqpoint{3.797025in}{3.045175in}}%
\pgfpathlineto{\pgfqpoint{3.797025in}{3.045175in}}%
\pgfpathlineto{\pgfqpoint{3.797025in}{3.048124in}}%
\pgfpathlineto{\pgfqpoint{3.801567in}{3.048124in}}%
\pgfpathlineto{\pgfqpoint{3.801567in}{3.045175in}}%
\pgfpathmoveto{\pgfqpoint{3.656247in}{3.139550in}}%
\pgfpathlineto{\pgfqpoint{3.656247in}{3.139550in}}%
\pgfpathlineto{\pgfqpoint{3.656247in}{3.142499in}}%
\pgfpathlineto{\pgfqpoint{3.660788in}{3.142499in}}%
\pgfpathlineto{\pgfqpoint{3.660788in}{3.139550in}}%
\pgfpathmoveto{\pgfqpoint{3.660788in}{3.136600in}}%
\pgfpathlineto{\pgfqpoint{3.660788in}{3.136600in}}%
\pgfpathlineto{\pgfqpoint{3.660788in}{3.139550in}}%
\pgfpathlineto{\pgfqpoint{3.665329in}{3.139550in}}%
\pgfpathlineto{\pgfqpoint{3.665329in}{3.136600in}}%
\pgfpathmoveto{\pgfqpoint{3.660788in}{3.139550in}}%
\pgfpathlineto{\pgfqpoint{3.660788in}{3.139550in}}%
\pgfpathlineto{\pgfqpoint{3.660788in}{3.142499in}}%
\pgfpathlineto{\pgfqpoint{3.665329in}{3.142499in}}%
\pgfpathlineto{\pgfqpoint{3.665329in}{3.139550in}}%
\pgfpathmoveto{\pgfqpoint{3.665329in}{3.133651in}}%
\pgfpathlineto{\pgfqpoint{3.665329in}{3.133651in}}%
\pgfpathlineto{\pgfqpoint{3.665329in}{3.136600in}}%
\pgfpathlineto{\pgfqpoint{3.669871in}{3.136600in}}%
\pgfpathlineto{\pgfqpoint{3.669871in}{3.133651in}}%
\pgfpathmoveto{\pgfqpoint{3.669871in}{3.130702in}}%
\pgfpathlineto{\pgfqpoint{3.669871in}{3.130702in}}%
\pgfpathlineto{\pgfqpoint{3.669871in}{3.133651in}}%
\pgfpathlineto{\pgfqpoint{3.674412in}{3.133651in}}%
\pgfpathlineto{\pgfqpoint{3.674412in}{3.130702in}}%
\pgfpathmoveto{\pgfqpoint{3.669871in}{3.133651in}}%
\pgfpathlineto{\pgfqpoint{3.669871in}{3.133651in}}%
\pgfpathlineto{\pgfqpoint{3.669871in}{3.136600in}}%
\pgfpathlineto{\pgfqpoint{3.674412in}{3.136600in}}%
\pgfpathlineto{\pgfqpoint{3.674412in}{3.133651in}}%
\pgfpathmoveto{\pgfqpoint{3.665329in}{3.136600in}}%
\pgfpathlineto{\pgfqpoint{3.665329in}{3.136600in}}%
\pgfpathlineto{\pgfqpoint{3.665329in}{3.139550in}}%
\pgfpathlineto{\pgfqpoint{3.669871in}{3.139550in}}%
\pgfpathlineto{\pgfqpoint{3.669871in}{3.136600in}}%
\pgfpathmoveto{\pgfqpoint{3.674412in}{3.127753in}}%
\pgfpathlineto{\pgfqpoint{3.674412in}{3.127753in}}%
\pgfpathlineto{\pgfqpoint{3.674412in}{3.130702in}}%
\pgfpathlineto{\pgfqpoint{3.678953in}{3.130702in}}%
\pgfpathlineto{\pgfqpoint{3.678953in}{3.127753in}}%
\pgfpathmoveto{\pgfqpoint{3.678953in}{3.124803in}}%
\pgfpathlineto{\pgfqpoint{3.678953in}{3.124803in}}%
\pgfpathlineto{\pgfqpoint{3.678953in}{3.127753in}}%
\pgfpathlineto{\pgfqpoint{3.683494in}{3.127753in}}%
\pgfpathlineto{\pgfqpoint{3.683494in}{3.124803in}}%
\pgfpathmoveto{\pgfqpoint{3.678953in}{3.127753in}}%
\pgfpathlineto{\pgfqpoint{3.678953in}{3.127753in}}%
\pgfpathlineto{\pgfqpoint{3.678953in}{3.130702in}}%
\pgfpathlineto{\pgfqpoint{3.683494in}{3.130702in}}%
\pgfpathlineto{\pgfqpoint{3.683494in}{3.127753in}}%
\pgfpathmoveto{\pgfqpoint{3.683494in}{3.121854in}}%
\pgfpathlineto{\pgfqpoint{3.683494in}{3.121854in}}%
\pgfpathlineto{\pgfqpoint{3.683494in}{3.124803in}}%
\pgfpathlineto{\pgfqpoint{3.688036in}{3.124803in}}%
\pgfpathlineto{\pgfqpoint{3.688036in}{3.121854in}}%
\pgfpathmoveto{\pgfqpoint{3.688036in}{3.118905in}}%
\pgfpathlineto{\pgfqpoint{3.688036in}{3.118905in}}%
\pgfpathlineto{\pgfqpoint{3.688036in}{3.121854in}}%
\pgfpathlineto{\pgfqpoint{3.692577in}{3.121854in}}%
\pgfpathlineto{\pgfqpoint{3.692577in}{3.118905in}}%
\pgfpathmoveto{\pgfqpoint{3.688036in}{3.121854in}}%
\pgfpathlineto{\pgfqpoint{3.688036in}{3.121854in}}%
\pgfpathlineto{\pgfqpoint{3.688036in}{3.124803in}}%
\pgfpathlineto{\pgfqpoint{3.692577in}{3.124803in}}%
\pgfpathlineto{\pgfqpoint{3.692577in}{3.121854in}}%
\pgfpathmoveto{\pgfqpoint{3.683494in}{3.124803in}}%
\pgfpathlineto{\pgfqpoint{3.683494in}{3.124803in}}%
\pgfpathlineto{\pgfqpoint{3.683494in}{3.127753in}}%
\pgfpathlineto{\pgfqpoint{3.688036in}{3.127753in}}%
\pgfpathlineto{\pgfqpoint{3.688036in}{3.124803in}}%
\pgfpathmoveto{\pgfqpoint{3.674412in}{3.130702in}}%
\pgfpathlineto{\pgfqpoint{3.674412in}{3.130702in}}%
\pgfpathlineto{\pgfqpoint{3.674412in}{3.133651in}}%
\pgfpathlineto{\pgfqpoint{3.678953in}{3.133651in}}%
\pgfpathlineto{\pgfqpoint{3.678953in}{3.130702in}}%
\pgfpathmoveto{\pgfqpoint{3.692577in}{3.115956in}}%
\pgfpathlineto{\pgfqpoint{3.692577in}{3.115956in}}%
\pgfpathlineto{\pgfqpoint{3.692577in}{3.118905in}}%
\pgfpathlineto{\pgfqpoint{3.697118in}{3.118905in}}%
\pgfpathlineto{\pgfqpoint{3.697118in}{3.115956in}}%
\pgfpathmoveto{\pgfqpoint{3.697118in}{3.113007in}}%
\pgfpathlineto{\pgfqpoint{3.697118in}{3.113007in}}%
\pgfpathlineto{\pgfqpoint{3.697118in}{3.115956in}}%
\pgfpathlineto{\pgfqpoint{3.701659in}{3.115956in}}%
\pgfpathlineto{\pgfqpoint{3.701659in}{3.113007in}}%
\pgfpathmoveto{\pgfqpoint{3.697118in}{3.115956in}}%
\pgfpathlineto{\pgfqpoint{3.697118in}{3.115956in}}%
\pgfpathlineto{\pgfqpoint{3.697118in}{3.118905in}}%
\pgfpathlineto{\pgfqpoint{3.701659in}{3.118905in}}%
\pgfpathlineto{\pgfqpoint{3.701659in}{3.115956in}}%
\pgfpathmoveto{\pgfqpoint{3.701659in}{3.110057in}}%
\pgfpathlineto{\pgfqpoint{3.701659in}{3.110057in}}%
\pgfpathlineto{\pgfqpoint{3.701659in}{3.113007in}}%
\pgfpathlineto{\pgfqpoint{3.706201in}{3.113007in}}%
\pgfpathlineto{\pgfqpoint{3.706201in}{3.110057in}}%
\pgfpathmoveto{\pgfqpoint{3.706201in}{3.107108in}}%
\pgfpathlineto{\pgfqpoint{3.706201in}{3.107108in}}%
\pgfpathlineto{\pgfqpoint{3.706201in}{3.110057in}}%
\pgfpathlineto{\pgfqpoint{3.710742in}{3.110057in}}%
\pgfpathlineto{\pgfqpoint{3.710742in}{3.107108in}}%
\pgfpathmoveto{\pgfqpoint{3.706201in}{3.110057in}}%
\pgfpathlineto{\pgfqpoint{3.706201in}{3.110057in}}%
\pgfpathlineto{\pgfqpoint{3.706201in}{3.113007in}}%
\pgfpathlineto{\pgfqpoint{3.710742in}{3.113007in}}%
\pgfpathlineto{\pgfqpoint{3.710742in}{3.110057in}}%
\pgfpathmoveto{\pgfqpoint{3.701659in}{3.113007in}}%
\pgfpathlineto{\pgfqpoint{3.701659in}{3.113007in}}%
\pgfpathlineto{\pgfqpoint{3.701659in}{3.115956in}}%
\pgfpathlineto{\pgfqpoint{3.706201in}{3.115956in}}%
\pgfpathlineto{\pgfqpoint{3.706201in}{3.113007in}}%
\pgfpathmoveto{\pgfqpoint{3.710742in}{3.104159in}}%
\pgfpathlineto{\pgfqpoint{3.710742in}{3.104159in}}%
\pgfpathlineto{\pgfqpoint{3.710742in}{3.107108in}}%
\pgfpathlineto{\pgfqpoint{3.715283in}{3.107108in}}%
\pgfpathlineto{\pgfqpoint{3.715283in}{3.104159in}}%
\pgfpathmoveto{\pgfqpoint{3.715283in}{3.101210in}}%
\pgfpathlineto{\pgfqpoint{3.715283in}{3.101210in}}%
\pgfpathlineto{\pgfqpoint{3.715283in}{3.104159in}}%
\pgfpathlineto{\pgfqpoint{3.719824in}{3.104159in}}%
\pgfpathlineto{\pgfqpoint{3.719824in}{3.101210in}}%
\pgfpathmoveto{\pgfqpoint{3.715283in}{3.104159in}}%
\pgfpathlineto{\pgfqpoint{3.715283in}{3.104159in}}%
\pgfpathlineto{\pgfqpoint{3.715283in}{3.107108in}}%
\pgfpathlineto{\pgfqpoint{3.719824in}{3.107108in}}%
\pgfpathlineto{\pgfqpoint{3.719824in}{3.104159in}}%
\pgfpathmoveto{\pgfqpoint{3.719824in}{3.098261in}}%
\pgfpathlineto{\pgfqpoint{3.719824in}{3.098261in}}%
\pgfpathlineto{\pgfqpoint{3.719824in}{3.101210in}}%
\pgfpathlineto{\pgfqpoint{3.724366in}{3.101210in}}%
\pgfpathlineto{\pgfqpoint{3.724366in}{3.098261in}}%
\pgfpathmoveto{\pgfqpoint{3.724366in}{3.095311in}}%
\pgfpathlineto{\pgfqpoint{3.724366in}{3.095311in}}%
\pgfpathlineto{\pgfqpoint{3.724366in}{3.098261in}}%
\pgfpathlineto{\pgfqpoint{3.728907in}{3.098261in}}%
\pgfpathlineto{\pgfqpoint{3.728907in}{3.095311in}}%
\pgfpathmoveto{\pgfqpoint{3.724366in}{3.098261in}}%
\pgfpathlineto{\pgfqpoint{3.724366in}{3.098261in}}%
\pgfpathlineto{\pgfqpoint{3.724366in}{3.101210in}}%
\pgfpathlineto{\pgfqpoint{3.728907in}{3.101210in}}%
\pgfpathlineto{\pgfqpoint{3.728907in}{3.098261in}}%
\pgfpathmoveto{\pgfqpoint{3.719824in}{3.101210in}}%
\pgfpathlineto{\pgfqpoint{3.719824in}{3.101210in}}%
\pgfpathlineto{\pgfqpoint{3.719824in}{3.104159in}}%
\pgfpathlineto{\pgfqpoint{3.724366in}{3.104159in}}%
\pgfpathlineto{\pgfqpoint{3.724366in}{3.101210in}}%
\pgfpathmoveto{\pgfqpoint{3.710742in}{3.107108in}}%
\pgfpathlineto{\pgfqpoint{3.710742in}{3.107108in}}%
\pgfpathlineto{\pgfqpoint{3.710742in}{3.110057in}}%
\pgfpathlineto{\pgfqpoint{3.715283in}{3.110057in}}%
\pgfpathlineto{\pgfqpoint{3.715283in}{3.107108in}}%
\pgfpathmoveto{\pgfqpoint{3.692577in}{3.118905in}}%
\pgfpathlineto{\pgfqpoint{3.692577in}{3.118905in}}%
\pgfpathlineto{\pgfqpoint{3.692577in}{3.121854in}}%
\pgfpathlineto{\pgfqpoint{3.697118in}{3.121854in}}%
\pgfpathlineto{\pgfqpoint{3.697118in}{3.118905in}}%
\pgfpathmoveto{\pgfqpoint{3.760695in}{3.068768in}}%
\pgfpathlineto{\pgfqpoint{3.760695in}{3.068768in}}%
\pgfpathlineto{\pgfqpoint{3.760695in}{3.071718in}}%
\pgfpathlineto{\pgfqpoint{3.765237in}{3.071718in}}%
\pgfpathlineto{\pgfqpoint{3.765237in}{3.068768in}}%
\pgfpathmoveto{\pgfqpoint{3.728907in}{3.092362in}}%
\pgfpathlineto{\pgfqpoint{3.728907in}{3.092362in}}%
\pgfpathlineto{\pgfqpoint{3.728907in}{3.095311in}}%
\pgfpathlineto{\pgfqpoint{3.733448in}{3.095311in}}%
\pgfpathlineto{\pgfqpoint{3.733448in}{3.092362in}}%
\pgfpathmoveto{\pgfqpoint{3.733448in}{3.089413in}}%
\pgfpathlineto{\pgfqpoint{3.733448in}{3.089413in}}%
\pgfpathlineto{\pgfqpoint{3.733448in}{3.092362in}}%
\pgfpathlineto{\pgfqpoint{3.737989in}{3.092362in}}%
\pgfpathlineto{\pgfqpoint{3.737989in}{3.089413in}}%
\pgfpathmoveto{\pgfqpoint{3.733448in}{3.092362in}}%
\pgfpathlineto{\pgfqpoint{3.733448in}{3.092362in}}%
\pgfpathlineto{\pgfqpoint{3.733448in}{3.095311in}}%
\pgfpathlineto{\pgfqpoint{3.737989in}{3.095311in}}%
\pgfpathlineto{\pgfqpoint{3.737989in}{3.092362in}}%
\pgfpathmoveto{\pgfqpoint{3.737989in}{3.086464in}}%
\pgfpathlineto{\pgfqpoint{3.737989in}{3.086464in}}%
\pgfpathlineto{\pgfqpoint{3.737989in}{3.089413in}}%
\pgfpathlineto{\pgfqpoint{3.742531in}{3.089413in}}%
\pgfpathlineto{\pgfqpoint{3.742531in}{3.086464in}}%
\pgfpathmoveto{\pgfqpoint{3.742531in}{3.083515in}}%
\pgfpathlineto{\pgfqpoint{3.742531in}{3.083515in}}%
\pgfpathlineto{\pgfqpoint{3.742531in}{3.086464in}}%
\pgfpathlineto{\pgfqpoint{3.747072in}{3.086464in}}%
\pgfpathlineto{\pgfqpoint{3.747072in}{3.083515in}}%
\pgfpathmoveto{\pgfqpoint{3.742531in}{3.086464in}}%
\pgfpathlineto{\pgfqpoint{3.742531in}{3.086464in}}%
\pgfpathlineto{\pgfqpoint{3.742531in}{3.089413in}}%
\pgfpathlineto{\pgfqpoint{3.747072in}{3.089413in}}%
\pgfpathlineto{\pgfqpoint{3.747072in}{3.086464in}}%
\pgfpathmoveto{\pgfqpoint{3.737989in}{3.089413in}}%
\pgfpathlineto{\pgfqpoint{3.737989in}{3.089413in}}%
\pgfpathlineto{\pgfqpoint{3.737989in}{3.092362in}}%
\pgfpathlineto{\pgfqpoint{3.742531in}{3.092362in}}%
\pgfpathlineto{\pgfqpoint{3.742531in}{3.089413in}}%
\pgfpathmoveto{\pgfqpoint{3.751613in}{3.074667in}}%
\pgfpathlineto{\pgfqpoint{3.751613in}{3.074667in}}%
\pgfpathlineto{\pgfqpoint{3.751613in}{3.077616in}}%
\pgfpathlineto{\pgfqpoint{3.756154in}{3.077616in}}%
\pgfpathlineto{\pgfqpoint{3.756154in}{3.074667in}}%
\pgfpathmoveto{\pgfqpoint{3.747072in}{3.077616in}}%
\pgfpathlineto{\pgfqpoint{3.747072in}{3.077616in}}%
\pgfpathlineto{\pgfqpoint{3.747072in}{3.080565in}}%
\pgfpathlineto{\pgfqpoint{3.751613in}{3.080565in}}%
\pgfpathlineto{\pgfqpoint{3.751613in}{3.077616in}}%
\pgfpathmoveto{\pgfqpoint{3.747072in}{3.080565in}}%
\pgfpathlineto{\pgfqpoint{3.747072in}{3.080565in}}%
\pgfpathlineto{\pgfqpoint{3.747072in}{3.083515in}}%
\pgfpathlineto{\pgfqpoint{3.751613in}{3.083515in}}%
\pgfpathlineto{\pgfqpoint{3.751613in}{3.080565in}}%
\pgfpathmoveto{\pgfqpoint{3.751613in}{3.077616in}}%
\pgfpathlineto{\pgfqpoint{3.751613in}{3.077616in}}%
\pgfpathlineto{\pgfqpoint{3.751613in}{3.080565in}}%
\pgfpathlineto{\pgfqpoint{3.756154in}{3.080565in}}%
\pgfpathlineto{\pgfqpoint{3.756154in}{3.077616in}}%
\pgfpathmoveto{\pgfqpoint{3.756154in}{3.071718in}}%
\pgfpathlineto{\pgfqpoint{3.756154in}{3.071718in}}%
\pgfpathlineto{\pgfqpoint{3.756154in}{3.074667in}}%
\pgfpathlineto{\pgfqpoint{3.760695in}{3.074667in}}%
\pgfpathlineto{\pgfqpoint{3.760695in}{3.071718in}}%
\pgfpathmoveto{\pgfqpoint{3.756154in}{3.074667in}}%
\pgfpathlineto{\pgfqpoint{3.756154in}{3.074667in}}%
\pgfpathlineto{\pgfqpoint{3.756154in}{3.077616in}}%
\pgfpathlineto{\pgfqpoint{3.760695in}{3.077616in}}%
\pgfpathlineto{\pgfqpoint{3.760695in}{3.074667in}}%
\pgfpathmoveto{\pgfqpoint{3.760695in}{3.071718in}}%
\pgfpathlineto{\pgfqpoint{3.760695in}{3.071718in}}%
\pgfpathlineto{\pgfqpoint{3.760695in}{3.074667in}}%
\pgfpathlineto{\pgfqpoint{3.765237in}{3.074667in}}%
\pgfpathlineto{\pgfqpoint{3.765237in}{3.071718in}}%
\pgfpathmoveto{\pgfqpoint{3.747072in}{3.083515in}}%
\pgfpathlineto{\pgfqpoint{3.747072in}{3.083515in}}%
\pgfpathlineto{\pgfqpoint{3.747072in}{3.086464in}}%
\pgfpathlineto{\pgfqpoint{3.751613in}{3.086464in}}%
\pgfpathlineto{\pgfqpoint{3.751613in}{3.083515in}}%
\pgfpathmoveto{\pgfqpoint{3.778860in}{3.056972in}}%
\pgfpathlineto{\pgfqpoint{3.778860in}{3.056972in}}%
\pgfpathlineto{\pgfqpoint{3.778860in}{3.059921in}}%
\pgfpathlineto{\pgfqpoint{3.783402in}{3.059921in}}%
\pgfpathlineto{\pgfqpoint{3.783402in}{3.056972in}}%
\pgfpathmoveto{\pgfqpoint{3.769778in}{3.062870in}}%
\pgfpathlineto{\pgfqpoint{3.769778in}{3.062870in}}%
\pgfpathlineto{\pgfqpoint{3.769778in}{3.065819in}}%
\pgfpathlineto{\pgfqpoint{3.774319in}{3.065819in}}%
\pgfpathlineto{\pgfqpoint{3.774319in}{3.062870in}}%
\pgfpathmoveto{\pgfqpoint{3.765237in}{3.065819in}}%
\pgfpathlineto{\pgfqpoint{3.765237in}{3.065819in}}%
\pgfpathlineto{\pgfqpoint{3.765237in}{3.068768in}}%
\pgfpathlineto{\pgfqpoint{3.769778in}{3.068768in}}%
\pgfpathlineto{\pgfqpoint{3.769778in}{3.065819in}}%
\pgfpathmoveto{\pgfqpoint{3.765237in}{3.068768in}}%
\pgfpathlineto{\pgfqpoint{3.765237in}{3.068768in}}%
\pgfpathlineto{\pgfqpoint{3.765237in}{3.071718in}}%
\pgfpathlineto{\pgfqpoint{3.769778in}{3.071718in}}%
\pgfpathlineto{\pgfqpoint{3.769778in}{3.068768in}}%
\pgfpathmoveto{\pgfqpoint{3.769778in}{3.065819in}}%
\pgfpathlineto{\pgfqpoint{3.769778in}{3.065819in}}%
\pgfpathlineto{\pgfqpoint{3.769778in}{3.068768in}}%
\pgfpathlineto{\pgfqpoint{3.774319in}{3.068768in}}%
\pgfpathlineto{\pgfqpoint{3.774319in}{3.065819in}}%
\pgfpathmoveto{\pgfqpoint{3.774319in}{3.059921in}}%
\pgfpathlineto{\pgfqpoint{3.774319in}{3.059921in}}%
\pgfpathlineto{\pgfqpoint{3.774319in}{3.062870in}}%
\pgfpathlineto{\pgfqpoint{3.778860in}{3.062870in}}%
\pgfpathlineto{\pgfqpoint{3.778860in}{3.059921in}}%
\pgfpathmoveto{\pgfqpoint{3.774319in}{3.062870in}}%
\pgfpathlineto{\pgfqpoint{3.774319in}{3.062870in}}%
\pgfpathlineto{\pgfqpoint{3.774319in}{3.065819in}}%
\pgfpathlineto{\pgfqpoint{3.778860in}{3.065819in}}%
\pgfpathlineto{\pgfqpoint{3.778860in}{3.062870in}}%
\pgfpathmoveto{\pgfqpoint{3.778860in}{3.059921in}}%
\pgfpathlineto{\pgfqpoint{3.778860in}{3.059921in}}%
\pgfpathlineto{\pgfqpoint{3.778860in}{3.062870in}}%
\pgfpathlineto{\pgfqpoint{3.783402in}{3.062870in}}%
\pgfpathlineto{\pgfqpoint{3.783402in}{3.059921in}}%
\pgfpathmoveto{\pgfqpoint{3.787943in}{3.051073in}}%
\pgfpathlineto{\pgfqpoint{3.787943in}{3.051073in}}%
\pgfpathlineto{\pgfqpoint{3.787943in}{3.054022in}}%
\pgfpathlineto{\pgfqpoint{3.792484in}{3.054022in}}%
\pgfpathlineto{\pgfqpoint{3.792484in}{3.051073in}}%
\pgfpathmoveto{\pgfqpoint{3.783402in}{3.054022in}}%
\pgfpathlineto{\pgfqpoint{3.783402in}{3.054022in}}%
\pgfpathlineto{\pgfqpoint{3.783402in}{3.056972in}}%
\pgfpathlineto{\pgfqpoint{3.787943in}{3.056972in}}%
\pgfpathlineto{\pgfqpoint{3.787943in}{3.054022in}}%
\pgfpathmoveto{\pgfqpoint{3.783402in}{3.056972in}}%
\pgfpathlineto{\pgfqpoint{3.783402in}{3.056972in}}%
\pgfpathlineto{\pgfqpoint{3.783402in}{3.059921in}}%
\pgfpathlineto{\pgfqpoint{3.787943in}{3.059921in}}%
\pgfpathlineto{\pgfqpoint{3.787943in}{3.056972in}}%
\pgfpathmoveto{\pgfqpoint{3.787943in}{3.054022in}}%
\pgfpathlineto{\pgfqpoint{3.787943in}{3.054022in}}%
\pgfpathlineto{\pgfqpoint{3.787943in}{3.056972in}}%
\pgfpathlineto{\pgfqpoint{3.792484in}{3.056972in}}%
\pgfpathlineto{\pgfqpoint{3.792484in}{3.054022in}}%
\pgfpathmoveto{\pgfqpoint{3.792484in}{3.048124in}}%
\pgfpathlineto{\pgfqpoint{3.792484in}{3.048124in}}%
\pgfpathlineto{\pgfqpoint{3.792484in}{3.051073in}}%
\pgfpathlineto{\pgfqpoint{3.797025in}{3.051073in}}%
\pgfpathlineto{\pgfqpoint{3.797025in}{3.048124in}}%
\pgfpathmoveto{\pgfqpoint{3.792484in}{3.051073in}}%
\pgfpathlineto{\pgfqpoint{3.792484in}{3.051073in}}%
\pgfpathlineto{\pgfqpoint{3.792484in}{3.054022in}}%
\pgfpathlineto{\pgfqpoint{3.797025in}{3.054022in}}%
\pgfpathlineto{\pgfqpoint{3.797025in}{3.051073in}}%
\pgfpathmoveto{\pgfqpoint{3.797025in}{3.048124in}}%
\pgfpathlineto{\pgfqpoint{3.797025in}{3.048124in}}%
\pgfpathlineto{\pgfqpoint{3.797025in}{3.051073in}}%
\pgfpathlineto{\pgfqpoint{3.801567in}{3.051073in}}%
\pgfpathlineto{\pgfqpoint{3.801567in}{3.048124in}}%
\pgfpathmoveto{\pgfqpoint{3.728907in}{3.095311in}}%
\pgfpathlineto{\pgfqpoint{3.728907in}{3.095311in}}%
\pgfpathlineto{\pgfqpoint{3.728907in}{3.098261in}}%
\pgfpathlineto{\pgfqpoint{3.733448in}{3.098261in}}%
\pgfpathlineto{\pgfqpoint{3.733448in}{3.095311in}}%
\pgfpathmoveto{\pgfqpoint{3.656247in}{3.142499in}}%
\pgfpathlineto{\pgfqpoint{3.656247in}{3.142499in}}%
\pgfpathlineto{\pgfqpoint{3.656247in}{3.145448in}}%
\pgfpathlineto{\pgfqpoint{3.660788in}{3.145448in}}%
\pgfpathlineto{\pgfqpoint{3.660788in}{3.142499in}}%
\pgfpathmoveto{\pgfqpoint{3.942333in}{2.950803in}}%
\pgfpathlineto{\pgfqpoint{3.942333in}{2.950803in}}%
\pgfpathlineto{\pgfqpoint{3.942333in}{2.953752in}}%
\pgfpathlineto{\pgfqpoint{3.946874in}{2.953752in}}%
\pgfpathlineto{\pgfqpoint{3.946874in}{2.950803in}}%
\pgfpathmoveto{\pgfqpoint{3.869679in}{2.997989in}}%
\pgfpathlineto{\pgfqpoint{3.869679in}{2.997989in}}%
\pgfpathlineto{\pgfqpoint{3.869679in}{3.000938in}}%
\pgfpathlineto{\pgfqpoint{3.874220in}{3.000938in}}%
\pgfpathlineto{\pgfqpoint{3.874220in}{2.997989in}}%
\pgfpathmoveto{\pgfqpoint{3.833353in}{3.021582in}}%
\pgfpathlineto{\pgfqpoint{3.833353in}{3.021582in}}%
\pgfpathlineto{\pgfqpoint{3.833353in}{3.024531in}}%
\pgfpathlineto{\pgfqpoint{3.837893in}{3.024531in}}%
\pgfpathlineto{\pgfqpoint{3.837893in}{3.021582in}}%
\pgfpathmoveto{\pgfqpoint{3.815189in}{3.033378in}}%
\pgfpathlineto{\pgfqpoint{3.815189in}{3.033378in}}%
\pgfpathlineto{\pgfqpoint{3.815189in}{3.036328in}}%
\pgfpathlineto{\pgfqpoint{3.819730in}{3.036328in}}%
\pgfpathlineto{\pgfqpoint{3.819730in}{3.033378in}}%
\pgfpathmoveto{\pgfqpoint{3.806107in}{3.039277in}}%
\pgfpathlineto{\pgfqpoint{3.806107in}{3.039277in}}%
\pgfpathlineto{\pgfqpoint{3.806107in}{3.042226in}}%
\pgfpathlineto{\pgfqpoint{3.810648in}{3.042226in}}%
\pgfpathlineto{\pgfqpoint{3.810648in}{3.039277in}}%
\pgfpathmoveto{\pgfqpoint{3.801567in}{3.042226in}}%
\pgfpathlineto{\pgfqpoint{3.801567in}{3.042226in}}%
\pgfpathlineto{\pgfqpoint{3.801567in}{3.045175in}}%
\pgfpathlineto{\pgfqpoint{3.806107in}{3.045175in}}%
\pgfpathlineto{\pgfqpoint{3.806107in}{3.042226in}}%
\pgfpathmoveto{\pgfqpoint{3.801567in}{3.045175in}}%
\pgfpathlineto{\pgfqpoint{3.801567in}{3.045175in}}%
\pgfpathlineto{\pgfqpoint{3.801567in}{3.048124in}}%
\pgfpathlineto{\pgfqpoint{3.806107in}{3.048124in}}%
\pgfpathlineto{\pgfqpoint{3.806107in}{3.045175in}}%
\pgfpathmoveto{\pgfqpoint{3.806107in}{3.042226in}}%
\pgfpathlineto{\pgfqpoint{3.806107in}{3.042226in}}%
\pgfpathlineto{\pgfqpoint{3.806107in}{3.045175in}}%
\pgfpathlineto{\pgfqpoint{3.810648in}{3.045175in}}%
\pgfpathlineto{\pgfqpoint{3.810648in}{3.042226in}}%
\pgfpathmoveto{\pgfqpoint{3.810648in}{3.036328in}}%
\pgfpathlineto{\pgfqpoint{3.810648in}{3.036328in}}%
\pgfpathlineto{\pgfqpoint{3.810648in}{3.039277in}}%
\pgfpathlineto{\pgfqpoint{3.815189in}{3.039277in}}%
\pgfpathlineto{\pgfqpoint{3.815189in}{3.036328in}}%
\pgfpathmoveto{\pgfqpoint{3.810648in}{3.039277in}}%
\pgfpathlineto{\pgfqpoint{3.810648in}{3.039277in}}%
\pgfpathlineto{\pgfqpoint{3.810648in}{3.042226in}}%
\pgfpathlineto{\pgfqpoint{3.815189in}{3.042226in}}%
\pgfpathlineto{\pgfqpoint{3.815189in}{3.039277in}}%
\pgfpathmoveto{\pgfqpoint{3.815189in}{3.036328in}}%
\pgfpathlineto{\pgfqpoint{3.815189in}{3.036328in}}%
\pgfpathlineto{\pgfqpoint{3.815189in}{3.039277in}}%
\pgfpathlineto{\pgfqpoint{3.819730in}{3.039277in}}%
\pgfpathlineto{\pgfqpoint{3.819730in}{3.036328in}}%
\pgfpathmoveto{\pgfqpoint{3.824271in}{3.027480in}}%
\pgfpathlineto{\pgfqpoint{3.824271in}{3.027480in}}%
\pgfpathlineto{\pgfqpoint{3.824271in}{3.030429in}}%
\pgfpathlineto{\pgfqpoint{3.828812in}{3.030429in}}%
\pgfpathlineto{\pgfqpoint{3.828812in}{3.027480in}}%
\pgfpathmoveto{\pgfqpoint{3.819730in}{3.030429in}}%
\pgfpathlineto{\pgfqpoint{3.819730in}{3.030429in}}%
\pgfpathlineto{\pgfqpoint{3.819730in}{3.033378in}}%
\pgfpathlineto{\pgfqpoint{3.824271in}{3.033378in}}%
\pgfpathlineto{\pgfqpoint{3.824271in}{3.030429in}}%
\pgfpathmoveto{\pgfqpoint{3.819730in}{3.033378in}}%
\pgfpathlineto{\pgfqpoint{3.819730in}{3.033378in}}%
\pgfpathlineto{\pgfqpoint{3.819730in}{3.036328in}}%
\pgfpathlineto{\pgfqpoint{3.824271in}{3.036328in}}%
\pgfpathlineto{\pgfqpoint{3.824271in}{3.033378in}}%
\pgfpathmoveto{\pgfqpoint{3.824271in}{3.030429in}}%
\pgfpathlineto{\pgfqpoint{3.824271in}{3.030429in}}%
\pgfpathlineto{\pgfqpoint{3.824271in}{3.033378in}}%
\pgfpathlineto{\pgfqpoint{3.828812in}{3.033378in}}%
\pgfpathlineto{\pgfqpoint{3.828812in}{3.030429in}}%
\pgfpathmoveto{\pgfqpoint{3.828812in}{3.024531in}}%
\pgfpathlineto{\pgfqpoint{3.828812in}{3.024531in}}%
\pgfpathlineto{\pgfqpoint{3.828812in}{3.027480in}}%
\pgfpathlineto{\pgfqpoint{3.833353in}{3.027480in}}%
\pgfpathlineto{\pgfqpoint{3.833353in}{3.024531in}}%
\pgfpathmoveto{\pgfqpoint{3.828812in}{3.027480in}}%
\pgfpathlineto{\pgfqpoint{3.828812in}{3.027480in}}%
\pgfpathlineto{\pgfqpoint{3.828812in}{3.030429in}}%
\pgfpathlineto{\pgfqpoint{3.833353in}{3.030429in}}%
\pgfpathlineto{\pgfqpoint{3.833353in}{3.027480in}}%
\pgfpathmoveto{\pgfqpoint{3.833353in}{3.024531in}}%
\pgfpathlineto{\pgfqpoint{3.833353in}{3.024531in}}%
\pgfpathlineto{\pgfqpoint{3.833353in}{3.027480in}}%
\pgfpathlineto{\pgfqpoint{3.837893in}{3.027480in}}%
\pgfpathlineto{\pgfqpoint{3.837893in}{3.024531in}}%
\pgfpathmoveto{\pgfqpoint{3.851516in}{3.009785in}}%
\pgfpathlineto{\pgfqpoint{3.851516in}{3.009785in}}%
\pgfpathlineto{\pgfqpoint{3.851516in}{3.012734in}}%
\pgfpathlineto{\pgfqpoint{3.856057in}{3.012734in}}%
\pgfpathlineto{\pgfqpoint{3.856057in}{3.009785in}}%
\pgfpathmoveto{\pgfqpoint{3.842434in}{3.015684in}}%
\pgfpathlineto{\pgfqpoint{3.842434in}{3.015684in}}%
\pgfpathlineto{\pgfqpoint{3.842434in}{3.018633in}}%
\pgfpathlineto{\pgfqpoint{3.846975in}{3.018633in}}%
\pgfpathlineto{\pgfqpoint{3.846975in}{3.015684in}}%
\pgfpathmoveto{\pgfqpoint{3.837893in}{3.018633in}}%
\pgfpathlineto{\pgfqpoint{3.837893in}{3.018633in}}%
\pgfpathlineto{\pgfqpoint{3.837893in}{3.021582in}}%
\pgfpathlineto{\pgfqpoint{3.842434in}{3.021582in}}%
\pgfpathlineto{\pgfqpoint{3.842434in}{3.018633in}}%
\pgfpathmoveto{\pgfqpoint{3.837893in}{3.021582in}}%
\pgfpathlineto{\pgfqpoint{3.837893in}{3.021582in}}%
\pgfpathlineto{\pgfqpoint{3.837893in}{3.024531in}}%
\pgfpathlineto{\pgfqpoint{3.842434in}{3.024531in}}%
\pgfpathlineto{\pgfqpoint{3.842434in}{3.021582in}}%
\pgfpathmoveto{\pgfqpoint{3.842434in}{3.018633in}}%
\pgfpathlineto{\pgfqpoint{3.842434in}{3.018633in}}%
\pgfpathlineto{\pgfqpoint{3.842434in}{3.021582in}}%
\pgfpathlineto{\pgfqpoint{3.846975in}{3.021582in}}%
\pgfpathlineto{\pgfqpoint{3.846975in}{3.018633in}}%
\pgfpathmoveto{\pgfqpoint{3.846975in}{3.012734in}}%
\pgfpathlineto{\pgfqpoint{3.846975in}{3.012734in}}%
\pgfpathlineto{\pgfqpoint{3.846975in}{3.015684in}}%
\pgfpathlineto{\pgfqpoint{3.851516in}{3.015684in}}%
\pgfpathlineto{\pgfqpoint{3.851516in}{3.012734in}}%
\pgfpathmoveto{\pgfqpoint{3.846975in}{3.015684in}}%
\pgfpathlineto{\pgfqpoint{3.846975in}{3.015684in}}%
\pgfpathlineto{\pgfqpoint{3.846975in}{3.018633in}}%
\pgfpathlineto{\pgfqpoint{3.851516in}{3.018633in}}%
\pgfpathlineto{\pgfqpoint{3.851516in}{3.015684in}}%
\pgfpathmoveto{\pgfqpoint{3.851516in}{3.012734in}}%
\pgfpathlineto{\pgfqpoint{3.851516in}{3.012734in}}%
\pgfpathlineto{\pgfqpoint{3.851516in}{3.015684in}}%
\pgfpathlineto{\pgfqpoint{3.856057in}{3.015684in}}%
\pgfpathlineto{\pgfqpoint{3.856057in}{3.012734in}}%
\pgfpathmoveto{\pgfqpoint{3.860598in}{3.003887in}}%
\pgfpathlineto{\pgfqpoint{3.860598in}{3.003887in}}%
\pgfpathlineto{\pgfqpoint{3.860598in}{3.006836in}}%
\pgfpathlineto{\pgfqpoint{3.865138in}{3.006836in}}%
\pgfpathlineto{\pgfqpoint{3.865138in}{3.003887in}}%
\pgfpathmoveto{\pgfqpoint{3.856057in}{3.006836in}}%
\pgfpathlineto{\pgfqpoint{3.856057in}{3.006836in}}%
\pgfpathlineto{\pgfqpoint{3.856057in}{3.009785in}}%
\pgfpathlineto{\pgfqpoint{3.860598in}{3.009785in}}%
\pgfpathlineto{\pgfqpoint{3.860598in}{3.006836in}}%
\pgfpathmoveto{\pgfqpoint{3.856057in}{3.009785in}}%
\pgfpathlineto{\pgfqpoint{3.856057in}{3.009785in}}%
\pgfpathlineto{\pgfqpoint{3.856057in}{3.012734in}}%
\pgfpathlineto{\pgfqpoint{3.860598in}{3.012734in}}%
\pgfpathlineto{\pgfqpoint{3.860598in}{3.009785in}}%
\pgfpathmoveto{\pgfqpoint{3.860598in}{3.006836in}}%
\pgfpathlineto{\pgfqpoint{3.860598in}{3.006836in}}%
\pgfpathlineto{\pgfqpoint{3.860598in}{3.009785in}}%
\pgfpathlineto{\pgfqpoint{3.865138in}{3.009785in}}%
\pgfpathlineto{\pgfqpoint{3.865138in}{3.006836in}}%
\pgfpathmoveto{\pgfqpoint{3.865138in}{3.000938in}}%
\pgfpathlineto{\pgfqpoint{3.865138in}{3.000938in}}%
\pgfpathlineto{\pgfqpoint{3.865138in}{3.003887in}}%
\pgfpathlineto{\pgfqpoint{3.869679in}{3.003887in}}%
\pgfpathlineto{\pgfqpoint{3.869679in}{3.000938in}}%
\pgfpathmoveto{\pgfqpoint{3.865138in}{3.003887in}}%
\pgfpathlineto{\pgfqpoint{3.865138in}{3.003887in}}%
\pgfpathlineto{\pgfqpoint{3.865138in}{3.006836in}}%
\pgfpathlineto{\pgfqpoint{3.869679in}{3.006836in}}%
\pgfpathlineto{\pgfqpoint{3.869679in}{3.003887in}}%
\pgfpathmoveto{\pgfqpoint{3.869679in}{3.000938in}}%
\pgfpathlineto{\pgfqpoint{3.869679in}{3.000938in}}%
\pgfpathlineto{\pgfqpoint{3.869679in}{3.003887in}}%
\pgfpathlineto{\pgfqpoint{3.874220in}{3.003887in}}%
\pgfpathlineto{\pgfqpoint{3.874220in}{3.000938in}}%
\pgfpathmoveto{\pgfqpoint{3.906006in}{2.974396in}}%
\pgfpathlineto{\pgfqpoint{3.906006in}{2.974396in}}%
\pgfpathlineto{\pgfqpoint{3.906006in}{2.977345in}}%
\pgfpathlineto{\pgfqpoint{3.910547in}{2.977345in}}%
\pgfpathlineto{\pgfqpoint{3.910547in}{2.974396in}}%
\pgfpathmoveto{\pgfqpoint{3.887843in}{2.986192in}}%
\pgfpathlineto{\pgfqpoint{3.887843in}{2.986192in}}%
\pgfpathlineto{\pgfqpoint{3.887843in}{2.989141in}}%
\pgfpathlineto{\pgfqpoint{3.892384in}{2.989141in}}%
\pgfpathlineto{\pgfqpoint{3.892384in}{2.986192in}}%
\pgfpathmoveto{\pgfqpoint{3.878761in}{2.992091in}}%
\pgfpathlineto{\pgfqpoint{3.878761in}{2.992091in}}%
\pgfpathlineto{\pgfqpoint{3.878761in}{2.995040in}}%
\pgfpathlineto{\pgfqpoint{3.883302in}{2.995040in}}%
\pgfpathlineto{\pgfqpoint{3.883302in}{2.992091in}}%
\pgfpathmoveto{\pgfqpoint{3.874220in}{2.995040in}}%
\pgfpathlineto{\pgfqpoint{3.874220in}{2.995040in}}%
\pgfpathlineto{\pgfqpoint{3.874220in}{2.997989in}}%
\pgfpathlineto{\pgfqpoint{3.878761in}{2.997989in}}%
\pgfpathlineto{\pgfqpoint{3.878761in}{2.995040in}}%
\pgfpathmoveto{\pgfqpoint{3.874220in}{2.997989in}}%
\pgfpathlineto{\pgfqpoint{3.874220in}{2.997989in}}%
\pgfpathlineto{\pgfqpoint{3.874220in}{3.000938in}}%
\pgfpathlineto{\pgfqpoint{3.878761in}{3.000938in}}%
\pgfpathlineto{\pgfqpoint{3.878761in}{2.997989in}}%
\pgfpathmoveto{\pgfqpoint{3.878761in}{2.995040in}}%
\pgfpathlineto{\pgfqpoint{3.878761in}{2.995040in}}%
\pgfpathlineto{\pgfqpoint{3.878761in}{2.997989in}}%
\pgfpathlineto{\pgfqpoint{3.883302in}{2.997989in}}%
\pgfpathlineto{\pgfqpoint{3.883302in}{2.995040in}}%
\pgfpathmoveto{\pgfqpoint{3.883302in}{2.989141in}}%
\pgfpathlineto{\pgfqpoint{3.883302in}{2.989141in}}%
\pgfpathlineto{\pgfqpoint{3.883302in}{2.992091in}}%
\pgfpathlineto{\pgfqpoint{3.887843in}{2.992091in}}%
\pgfpathlineto{\pgfqpoint{3.887843in}{2.989141in}}%
\pgfpathmoveto{\pgfqpoint{3.883302in}{2.992091in}}%
\pgfpathlineto{\pgfqpoint{3.883302in}{2.992091in}}%
\pgfpathlineto{\pgfqpoint{3.883302in}{2.995040in}}%
\pgfpathlineto{\pgfqpoint{3.887843in}{2.995040in}}%
\pgfpathlineto{\pgfqpoint{3.887843in}{2.992091in}}%
\pgfpathmoveto{\pgfqpoint{3.887843in}{2.989141in}}%
\pgfpathlineto{\pgfqpoint{3.887843in}{2.989141in}}%
\pgfpathlineto{\pgfqpoint{3.887843in}{2.992091in}}%
\pgfpathlineto{\pgfqpoint{3.892384in}{2.992091in}}%
\pgfpathlineto{\pgfqpoint{3.892384in}{2.989141in}}%
\pgfpathmoveto{\pgfqpoint{3.896924in}{2.980294in}}%
\pgfpathlineto{\pgfqpoint{3.896924in}{2.980294in}}%
\pgfpathlineto{\pgfqpoint{3.896924in}{2.983243in}}%
\pgfpathlineto{\pgfqpoint{3.901465in}{2.983243in}}%
\pgfpathlineto{\pgfqpoint{3.901465in}{2.980294in}}%
\pgfpathmoveto{\pgfqpoint{3.892384in}{2.983243in}}%
\pgfpathlineto{\pgfqpoint{3.892384in}{2.983243in}}%
\pgfpathlineto{\pgfqpoint{3.892384in}{2.986192in}}%
\pgfpathlineto{\pgfqpoint{3.896924in}{2.986192in}}%
\pgfpathlineto{\pgfqpoint{3.896924in}{2.983243in}}%
\pgfpathmoveto{\pgfqpoint{3.892384in}{2.986192in}}%
\pgfpathlineto{\pgfqpoint{3.892384in}{2.986192in}}%
\pgfpathlineto{\pgfqpoint{3.892384in}{2.989141in}}%
\pgfpathlineto{\pgfqpoint{3.896924in}{2.989141in}}%
\pgfpathlineto{\pgfqpoint{3.896924in}{2.986192in}}%
\pgfpathmoveto{\pgfqpoint{3.896924in}{2.983243in}}%
\pgfpathlineto{\pgfqpoint{3.896924in}{2.983243in}}%
\pgfpathlineto{\pgfqpoint{3.896924in}{2.986192in}}%
\pgfpathlineto{\pgfqpoint{3.901465in}{2.986192in}}%
\pgfpathlineto{\pgfqpoint{3.901465in}{2.983243in}}%
\pgfpathmoveto{\pgfqpoint{3.901465in}{2.977345in}}%
\pgfpathlineto{\pgfqpoint{3.901465in}{2.977345in}}%
\pgfpathlineto{\pgfqpoint{3.901465in}{2.980294in}}%
\pgfpathlineto{\pgfqpoint{3.906006in}{2.980294in}}%
\pgfpathlineto{\pgfqpoint{3.906006in}{2.977345in}}%
\pgfpathmoveto{\pgfqpoint{3.901465in}{2.980294in}}%
\pgfpathlineto{\pgfqpoint{3.901465in}{2.980294in}}%
\pgfpathlineto{\pgfqpoint{3.901465in}{2.983243in}}%
\pgfpathlineto{\pgfqpoint{3.906006in}{2.983243in}}%
\pgfpathlineto{\pgfqpoint{3.906006in}{2.980294in}}%
\pgfpathmoveto{\pgfqpoint{3.906006in}{2.977345in}}%
\pgfpathlineto{\pgfqpoint{3.906006in}{2.977345in}}%
\pgfpathlineto{\pgfqpoint{3.906006in}{2.980294in}}%
\pgfpathlineto{\pgfqpoint{3.910547in}{2.980294in}}%
\pgfpathlineto{\pgfqpoint{3.910547in}{2.977345in}}%
\pgfpathmoveto{\pgfqpoint{3.924169in}{2.962599in}}%
\pgfpathlineto{\pgfqpoint{3.924169in}{2.962599in}}%
\pgfpathlineto{\pgfqpoint{3.924169in}{2.965548in}}%
\pgfpathlineto{\pgfqpoint{3.928710in}{2.965548in}}%
\pgfpathlineto{\pgfqpoint{3.928710in}{2.962599in}}%
\pgfpathmoveto{\pgfqpoint{3.915088in}{2.968498in}}%
\pgfpathlineto{\pgfqpoint{3.915088in}{2.968498in}}%
\pgfpathlineto{\pgfqpoint{3.915088in}{2.971447in}}%
\pgfpathlineto{\pgfqpoint{3.919629in}{2.971447in}}%
\pgfpathlineto{\pgfqpoint{3.919629in}{2.968498in}}%
\pgfpathmoveto{\pgfqpoint{3.910547in}{2.971447in}}%
\pgfpathlineto{\pgfqpoint{3.910547in}{2.971447in}}%
\pgfpathlineto{\pgfqpoint{3.910547in}{2.974396in}}%
\pgfpathlineto{\pgfqpoint{3.915088in}{2.974396in}}%
\pgfpathlineto{\pgfqpoint{3.915088in}{2.971447in}}%
\pgfpathmoveto{\pgfqpoint{3.910547in}{2.974396in}}%
\pgfpathlineto{\pgfqpoint{3.910547in}{2.974396in}}%
\pgfpathlineto{\pgfqpoint{3.910547in}{2.977345in}}%
\pgfpathlineto{\pgfqpoint{3.915088in}{2.977345in}}%
\pgfpathlineto{\pgfqpoint{3.915088in}{2.974396in}}%
\pgfpathmoveto{\pgfqpoint{3.915088in}{2.971447in}}%
\pgfpathlineto{\pgfqpoint{3.915088in}{2.971447in}}%
\pgfpathlineto{\pgfqpoint{3.915088in}{2.974396in}}%
\pgfpathlineto{\pgfqpoint{3.919629in}{2.974396in}}%
\pgfpathlineto{\pgfqpoint{3.919629in}{2.971447in}}%
\pgfpathmoveto{\pgfqpoint{3.919629in}{2.965548in}}%
\pgfpathlineto{\pgfqpoint{3.919629in}{2.965548in}}%
\pgfpathlineto{\pgfqpoint{3.919629in}{2.968498in}}%
\pgfpathlineto{\pgfqpoint{3.924169in}{2.968498in}}%
\pgfpathlineto{\pgfqpoint{3.924169in}{2.965548in}}%
\pgfpathmoveto{\pgfqpoint{3.919629in}{2.968498in}}%
\pgfpathlineto{\pgfqpoint{3.919629in}{2.968498in}}%
\pgfpathlineto{\pgfqpoint{3.919629in}{2.971447in}}%
\pgfpathlineto{\pgfqpoint{3.924169in}{2.971447in}}%
\pgfpathlineto{\pgfqpoint{3.924169in}{2.968498in}}%
\pgfpathmoveto{\pgfqpoint{3.924169in}{2.965548in}}%
\pgfpathlineto{\pgfqpoint{3.924169in}{2.965548in}}%
\pgfpathlineto{\pgfqpoint{3.924169in}{2.968498in}}%
\pgfpathlineto{\pgfqpoint{3.928710in}{2.968498in}}%
\pgfpathlineto{\pgfqpoint{3.928710in}{2.965548in}}%
\pgfpathmoveto{\pgfqpoint{3.933251in}{2.956701in}}%
\pgfpathlineto{\pgfqpoint{3.933251in}{2.956701in}}%
\pgfpathlineto{\pgfqpoint{3.933251in}{2.959650in}}%
\pgfpathlineto{\pgfqpoint{3.937792in}{2.959650in}}%
\pgfpathlineto{\pgfqpoint{3.937792in}{2.956701in}}%
\pgfpathmoveto{\pgfqpoint{3.928710in}{2.959650in}}%
\pgfpathlineto{\pgfqpoint{3.928710in}{2.959650in}}%
\pgfpathlineto{\pgfqpoint{3.928710in}{2.962599in}}%
\pgfpathlineto{\pgfqpoint{3.933251in}{2.962599in}}%
\pgfpathlineto{\pgfqpoint{3.933251in}{2.959650in}}%
\pgfpathmoveto{\pgfqpoint{3.928710in}{2.962599in}}%
\pgfpathlineto{\pgfqpoint{3.928710in}{2.962599in}}%
\pgfpathlineto{\pgfqpoint{3.928710in}{2.965548in}}%
\pgfpathlineto{\pgfqpoint{3.933251in}{2.965548in}}%
\pgfpathlineto{\pgfqpoint{3.933251in}{2.962599in}}%
\pgfpathmoveto{\pgfqpoint{3.933251in}{2.959650in}}%
\pgfpathlineto{\pgfqpoint{3.933251in}{2.959650in}}%
\pgfpathlineto{\pgfqpoint{3.933251in}{2.962599in}}%
\pgfpathlineto{\pgfqpoint{3.937792in}{2.962599in}}%
\pgfpathlineto{\pgfqpoint{3.937792in}{2.959650in}}%
\pgfpathmoveto{\pgfqpoint{3.937792in}{2.953752in}}%
\pgfpathlineto{\pgfqpoint{3.937792in}{2.953752in}}%
\pgfpathlineto{\pgfqpoint{3.937792in}{2.956701in}}%
\pgfpathlineto{\pgfqpoint{3.942333in}{2.956701in}}%
\pgfpathlineto{\pgfqpoint{3.942333in}{2.953752in}}%
\pgfpathmoveto{\pgfqpoint{3.937792in}{2.956701in}}%
\pgfpathlineto{\pgfqpoint{3.937792in}{2.956701in}}%
\pgfpathlineto{\pgfqpoint{3.937792in}{2.959650in}}%
\pgfpathlineto{\pgfqpoint{3.942333in}{2.959650in}}%
\pgfpathlineto{\pgfqpoint{3.942333in}{2.956701in}}%
\pgfpathmoveto{\pgfqpoint{3.942333in}{2.953752in}}%
\pgfpathlineto{\pgfqpoint{3.942333in}{2.953752in}}%
\pgfpathlineto{\pgfqpoint{3.942333in}{2.956701in}}%
\pgfpathlineto{\pgfqpoint{3.946874in}{2.956701in}}%
\pgfpathlineto{\pgfqpoint{3.946874in}{2.953752in}}%
\pgfpathmoveto{\pgfqpoint{3.960497in}{2.939005in}}%
\pgfpathlineto{\pgfqpoint{3.960497in}{2.939005in}}%
\pgfpathlineto{\pgfqpoint{3.960497in}{2.941955in}}%
\pgfpathlineto{\pgfqpoint{3.965038in}{2.941955in}}%
\pgfpathlineto{\pgfqpoint{3.965038in}{2.939005in}}%
\pgfpathmoveto{\pgfqpoint{3.951415in}{2.944904in}}%
\pgfpathlineto{\pgfqpoint{3.951415in}{2.944904in}}%
\pgfpathlineto{\pgfqpoint{3.951415in}{2.947853in}}%
\pgfpathlineto{\pgfqpoint{3.955956in}{2.947853in}}%
\pgfpathlineto{\pgfqpoint{3.955956in}{2.944904in}}%
\pgfpathmoveto{\pgfqpoint{3.946874in}{2.947853in}}%
\pgfpathlineto{\pgfqpoint{3.946874in}{2.947853in}}%
\pgfpathlineto{\pgfqpoint{3.946874in}{2.950803in}}%
\pgfpathlineto{\pgfqpoint{3.951415in}{2.950803in}}%
\pgfpathlineto{\pgfqpoint{3.951415in}{2.947853in}}%
\pgfpathmoveto{\pgfqpoint{3.946874in}{2.950803in}}%
\pgfpathlineto{\pgfqpoint{3.946874in}{2.950803in}}%
\pgfpathlineto{\pgfqpoint{3.946874in}{2.953752in}}%
\pgfpathlineto{\pgfqpoint{3.951415in}{2.953752in}}%
\pgfpathlineto{\pgfqpoint{3.951415in}{2.950803in}}%
\pgfpathmoveto{\pgfqpoint{3.951415in}{2.947853in}}%
\pgfpathlineto{\pgfqpoint{3.951415in}{2.947853in}}%
\pgfpathlineto{\pgfqpoint{3.951415in}{2.950803in}}%
\pgfpathlineto{\pgfqpoint{3.955956in}{2.950803in}}%
\pgfpathlineto{\pgfqpoint{3.955956in}{2.947853in}}%
\pgfpathmoveto{\pgfqpoint{3.955956in}{2.941955in}}%
\pgfpathlineto{\pgfqpoint{3.955956in}{2.941955in}}%
\pgfpathlineto{\pgfqpoint{3.955956in}{2.944904in}}%
\pgfpathlineto{\pgfqpoint{3.960497in}{2.944904in}}%
\pgfpathlineto{\pgfqpoint{3.960497in}{2.941955in}}%
\pgfpathmoveto{\pgfqpoint{3.955956in}{2.944904in}}%
\pgfpathlineto{\pgfqpoint{3.955956in}{2.944904in}}%
\pgfpathlineto{\pgfqpoint{3.955956in}{2.947853in}}%
\pgfpathlineto{\pgfqpoint{3.960497in}{2.947853in}}%
\pgfpathlineto{\pgfqpoint{3.960497in}{2.944904in}}%
\pgfpathmoveto{\pgfqpoint{3.960497in}{2.941955in}}%
\pgfpathlineto{\pgfqpoint{3.960497in}{2.941955in}}%
\pgfpathlineto{\pgfqpoint{3.960497in}{2.944904in}}%
\pgfpathlineto{\pgfqpoint{3.965038in}{2.944904in}}%
\pgfpathlineto{\pgfqpoint{3.965038in}{2.941955in}}%
\pgfpathmoveto{\pgfqpoint{3.969578in}{2.933107in}}%
\pgfpathlineto{\pgfqpoint{3.969578in}{2.933107in}}%
\pgfpathlineto{\pgfqpoint{3.969578in}{2.936056in}}%
\pgfpathlineto{\pgfqpoint{3.974119in}{2.936056in}}%
\pgfpathlineto{\pgfqpoint{3.974119in}{2.933107in}}%
\pgfpathmoveto{\pgfqpoint{3.965038in}{2.936056in}}%
\pgfpathlineto{\pgfqpoint{3.965038in}{2.936056in}}%
\pgfpathlineto{\pgfqpoint{3.965038in}{2.939005in}}%
\pgfpathlineto{\pgfqpoint{3.969578in}{2.939005in}}%
\pgfpathlineto{\pgfqpoint{3.969578in}{2.936056in}}%
\pgfpathmoveto{\pgfqpoint{3.965038in}{2.939005in}}%
\pgfpathlineto{\pgfqpoint{3.965038in}{2.939005in}}%
\pgfpathlineto{\pgfqpoint{3.965038in}{2.941955in}}%
\pgfpathlineto{\pgfqpoint{3.969578in}{2.941955in}}%
\pgfpathlineto{\pgfqpoint{3.969578in}{2.939005in}}%
\pgfpathmoveto{\pgfqpoint{3.969578in}{2.936056in}}%
\pgfpathlineto{\pgfqpoint{3.969578in}{2.936056in}}%
\pgfpathlineto{\pgfqpoint{3.969578in}{2.939005in}}%
\pgfpathlineto{\pgfqpoint{3.974119in}{2.939005in}}%
\pgfpathlineto{\pgfqpoint{3.974119in}{2.936056in}}%
\pgfpathmoveto{\pgfqpoint{3.974119in}{2.930157in}}%
\pgfpathlineto{\pgfqpoint{3.974119in}{2.930157in}}%
\pgfpathlineto{\pgfqpoint{3.974119in}{2.933107in}}%
\pgfpathlineto{\pgfqpoint{3.978660in}{2.933107in}}%
\pgfpathlineto{\pgfqpoint{3.978660in}{2.930157in}}%
\pgfpathmoveto{\pgfqpoint{3.974119in}{2.933107in}}%
\pgfpathlineto{\pgfqpoint{3.974119in}{2.933107in}}%
\pgfpathlineto{\pgfqpoint{3.974119in}{2.936056in}}%
\pgfpathlineto{\pgfqpoint{3.978660in}{2.936056in}}%
\pgfpathlineto{\pgfqpoint{3.978660in}{2.933107in}}%
\pgfpathmoveto{\pgfqpoint{3.978660in}{2.930157in}}%
\pgfpathlineto{\pgfqpoint{3.978660in}{2.930157in}}%
\pgfpathlineto{\pgfqpoint{3.978660in}{2.933107in}}%
\pgfpathlineto{\pgfqpoint{3.983201in}{2.933107in}}%
\pgfpathlineto{\pgfqpoint{3.983201in}{2.930157in}}%
\pgfpathmoveto{\pgfqpoint{3.983201in}{2.927208in}}%
\pgfpathlineto{\pgfqpoint{3.983201in}{2.927208in}}%
\pgfpathlineto{\pgfqpoint{3.983201in}{2.930157in}}%
\pgfpathlineto{\pgfqpoint{3.987742in}{2.930157in}}%
\pgfpathlineto{\pgfqpoint{3.987742in}{2.927208in}}%
\pgfpathmoveto{\pgfqpoint{3.987742in}{2.924259in}}%
\pgfpathlineto{\pgfqpoint{3.987742in}{2.924259in}}%
\pgfpathlineto{\pgfqpoint{3.987742in}{2.927208in}}%
\pgfpathlineto{\pgfqpoint{3.992283in}{2.927208in}}%
\pgfpathlineto{\pgfqpoint{3.992283in}{2.924259in}}%
\pgfpathmoveto{\pgfqpoint{3.987742in}{2.927208in}}%
\pgfpathlineto{\pgfqpoint{3.987742in}{2.927208in}}%
\pgfpathlineto{\pgfqpoint{3.987742in}{2.930157in}}%
\pgfpathlineto{\pgfqpoint{3.992283in}{2.930157in}}%
\pgfpathlineto{\pgfqpoint{3.992283in}{2.927208in}}%
\pgfpathmoveto{\pgfqpoint{3.992283in}{2.921309in}}%
\pgfpathlineto{\pgfqpoint{3.992283in}{2.921309in}}%
\pgfpathlineto{\pgfqpoint{3.992283in}{2.924259in}}%
\pgfpathlineto{\pgfqpoint{3.996824in}{2.924259in}}%
\pgfpathlineto{\pgfqpoint{3.996824in}{2.921309in}}%
\pgfpathmoveto{\pgfqpoint{3.996824in}{2.918360in}}%
\pgfpathlineto{\pgfqpoint{3.996824in}{2.918360in}}%
\pgfpathlineto{\pgfqpoint{3.996824in}{2.921309in}}%
\pgfpathlineto{\pgfqpoint{4.001365in}{2.921309in}}%
\pgfpathlineto{\pgfqpoint{4.001365in}{2.918360in}}%
\pgfpathmoveto{\pgfqpoint{3.996824in}{2.921309in}}%
\pgfpathlineto{\pgfqpoint{3.996824in}{2.921309in}}%
\pgfpathlineto{\pgfqpoint{3.996824in}{2.924259in}}%
\pgfpathlineto{\pgfqpoint{4.001365in}{2.924259in}}%
\pgfpathlineto{\pgfqpoint{4.001365in}{2.921309in}}%
\pgfpathmoveto{\pgfqpoint{3.992283in}{2.924259in}}%
\pgfpathlineto{\pgfqpoint{3.992283in}{2.924259in}}%
\pgfpathlineto{\pgfqpoint{3.992283in}{2.927208in}}%
\pgfpathlineto{\pgfqpoint{3.996824in}{2.927208in}}%
\pgfpathlineto{\pgfqpoint{3.996824in}{2.924259in}}%
\pgfpathmoveto{\pgfqpoint{4.001365in}{2.915411in}}%
\pgfpathlineto{\pgfqpoint{4.001365in}{2.915411in}}%
\pgfpathlineto{\pgfqpoint{4.001365in}{2.918360in}}%
\pgfpathlineto{\pgfqpoint{4.005906in}{2.918360in}}%
\pgfpathlineto{\pgfqpoint{4.005906in}{2.915411in}}%
\pgfpathmoveto{\pgfqpoint{4.005906in}{2.912462in}}%
\pgfpathlineto{\pgfqpoint{4.005906in}{2.912462in}}%
\pgfpathlineto{\pgfqpoint{4.005906in}{2.915411in}}%
\pgfpathlineto{\pgfqpoint{4.010447in}{2.915411in}}%
\pgfpathlineto{\pgfqpoint{4.010447in}{2.912462in}}%
\pgfpathmoveto{\pgfqpoint{4.005906in}{2.915411in}}%
\pgfpathlineto{\pgfqpoint{4.005906in}{2.915411in}}%
\pgfpathlineto{\pgfqpoint{4.005906in}{2.918360in}}%
\pgfpathlineto{\pgfqpoint{4.010447in}{2.918360in}}%
\pgfpathlineto{\pgfqpoint{4.010447in}{2.915411in}}%
\pgfpathmoveto{\pgfqpoint{4.010447in}{2.909512in}}%
\pgfpathlineto{\pgfqpoint{4.010447in}{2.909512in}}%
\pgfpathlineto{\pgfqpoint{4.010447in}{2.912462in}}%
\pgfpathlineto{\pgfqpoint{4.014988in}{2.912462in}}%
\pgfpathlineto{\pgfqpoint{4.014988in}{2.909512in}}%
\pgfpathmoveto{\pgfqpoint{4.014988in}{2.906563in}}%
\pgfpathlineto{\pgfqpoint{4.014988in}{2.906563in}}%
\pgfpathlineto{\pgfqpoint{4.014988in}{2.909512in}}%
\pgfpathlineto{\pgfqpoint{4.019529in}{2.909512in}}%
\pgfpathlineto{\pgfqpoint{4.019529in}{2.906563in}}%
\pgfpathmoveto{\pgfqpoint{4.014988in}{2.909512in}}%
\pgfpathlineto{\pgfqpoint{4.014988in}{2.909512in}}%
\pgfpathlineto{\pgfqpoint{4.014988in}{2.912462in}}%
\pgfpathlineto{\pgfqpoint{4.019529in}{2.912462in}}%
\pgfpathlineto{\pgfqpoint{4.019529in}{2.909512in}}%
\pgfpathmoveto{\pgfqpoint{4.010447in}{2.912462in}}%
\pgfpathlineto{\pgfqpoint{4.010447in}{2.912462in}}%
\pgfpathlineto{\pgfqpoint{4.010447in}{2.915411in}}%
\pgfpathlineto{\pgfqpoint{4.014988in}{2.915411in}}%
\pgfpathlineto{\pgfqpoint{4.014988in}{2.912462in}}%
\pgfpathmoveto{\pgfqpoint{4.001365in}{2.918360in}}%
\pgfpathlineto{\pgfqpoint{4.001365in}{2.918360in}}%
\pgfpathlineto{\pgfqpoint{4.001365in}{2.921309in}}%
\pgfpathlineto{\pgfqpoint{4.005906in}{2.921309in}}%
\pgfpathlineto{\pgfqpoint{4.005906in}{2.918360in}}%
\pgfpathmoveto{\pgfqpoint{3.983201in}{2.930157in}}%
\pgfpathlineto{\pgfqpoint{3.983201in}{2.930157in}}%
\pgfpathlineto{\pgfqpoint{3.983201in}{2.933107in}}%
\pgfpathlineto{\pgfqpoint{3.987742in}{2.933107in}}%
\pgfpathlineto{\pgfqpoint{3.987742in}{2.930157in}}%
\pgfpathmoveto{\pgfqpoint{4.019529in}{2.903614in}}%
\pgfpathlineto{\pgfqpoint{4.019529in}{2.903614in}}%
\pgfpathlineto{\pgfqpoint{4.019529in}{2.906563in}}%
\pgfpathlineto{\pgfqpoint{4.024070in}{2.906563in}}%
\pgfpathlineto{\pgfqpoint{4.024070in}{2.903614in}}%
\pgfpathmoveto{\pgfqpoint{4.024070in}{2.900664in}}%
\pgfpathlineto{\pgfqpoint{4.024070in}{2.900664in}}%
\pgfpathlineto{\pgfqpoint{4.024070in}{2.903614in}}%
\pgfpathlineto{\pgfqpoint{4.028611in}{2.903614in}}%
\pgfpathlineto{\pgfqpoint{4.028611in}{2.900664in}}%
\pgfpathmoveto{\pgfqpoint{4.024070in}{2.903614in}}%
\pgfpathlineto{\pgfqpoint{4.024070in}{2.903614in}}%
\pgfpathlineto{\pgfqpoint{4.024070in}{2.906563in}}%
\pgfpathlineto{\pgfqpoint{4.028611in}{2.906563in}}%
\pgfpathlineto{\pgfqpoint{4.028611in}{2.903614in}}%
\pgfpathmoveto{\pgfqpoint{4.028611in}{2.897715in}}%
\pgfpathlineto{\pgfqpoint{4.028611in}{2.897715in}}%
\pgfpathlineto{\pgfqpoint{4.028611in}{2.900664in}}%
\pgfpathlineto{\pgfqpoint{4.033152in}{2.900664in}}%
\pgfpathlineto{\pgfqpoint{4.033152in}{2.897715in}}%
\pgfpathmoveto{\pgfqpoint{4.033152in}{2.894766in}}%
\pgfpathlineto{\pgfqpoint{4.033152in}{2.894766in}}%
\pgfpathlineto{\pgfqpoint{4.033152in}{2.897715in}}%
\pgfpathlineto{\pgfqpoint{4.037693in}{2.897715in}}%
\pgfpathlineto{\pgfqpoint{4.037693in}{2.894766in}}%
\pgfpathmoveto{\pgfqpoint{4.033152in}{2.897715in}}%
\pgfpathlineto{\pgfqpoint{4.033152in}{2.897715in}}%
\pgfpathlineto{\pgfqpoint{4.033152in}{2.900664in}}%
\pgfpathlineto{\pgfqpoint{4.037693in}{2.900664in}}%
\pgfpathlineto{\pgfqpoint{4.037693in}{2.897715in}}%
\pgfpathmoveto{\pgfqpoint{4.028611in}{2.900664in}}%
\pgfpathlineto{\pgfqpoint{4.028611in}{2.900664in}}%
\pgfpathlineto{\pgfqpoint{4.028611in}{2.903614in}}%
\pgfpathlineto{\pgfqpoint{4.033152in}{2.903614in}}%
\pgfpathlineto{\pgfqpoint{4.033152in}{2.900664in}}%
\pgfpathmoveto{\pgfqpoint{4.037693in}{2.891816in}}%
\pgfpathlineto{\pgfqpoint{4.037693in}{2.891816in}}%
\pgfpathlineto{\pgfqpoint{4.037693in}{2.894766in}}%
\pgfpathlineto{\pgfqpoint{4.042234in}{2.894766in}}%
\pgfpathlineto{\pgfqpoint{4.042234in}{2.891816in}}%
\pgfpathmoveto{\pgfqpoint{4.042234in}{2.888867in}}%
\pgfpathlineto{\pgfqpoint{4.042234in}{2.888867in}}%
\pgfpathlineto{\pgfqpoint{4.042234in}{2.891816in}}%
\pgfpathlineto{\pgfqpoint{4.046775in}{2.891816in}}%
\pgfpathlineto{\pgfqpoint{4.046775in}{2.888867in}}%
\pgfpathmoveto{\pgfqpoint{4.042234in}{2.891816in}}%
\pgfpathlineto{\pgfqpoint{4.042234in}{2.891816in}}%
\pgfpathlineto{\pgfqpoint{4.042234in}{2.894766in}}%
\pgfpathlineto{\pgfqpoint{4.046775in}{2.894766in}}%
\pgfpathlineto{\pgfqpoint{4.046775in}{2.891816in}}%
\pgfpathmoveto{\pgfqpoint{4.046775in}{2.885918in}}%
\pgfpathlineto{\pgfqpoint{4.046775in}{2.885918in}}%
\pgfpathlineto{\pgfqpoint{4.046775in}{2.888867in}}%
\pgfpathlineto{\pgfqpoint{4.051316in}{2.888867in}}%
\pgfpathlineto{\pgfqpoint{4.051316in}{2.885918in}}%
\pgfpathmoveto{\pgfqpoint{4.051316in}{2.882968in}}%
\pgfpathlineto{\pgfqpoint{4.051316in}{2.882968in}}%
\pgfpathlineto{\pgfqpoint{4.051316in}{2.885918in}}%
\pgfpathlineto{\pgfqpoint{4.055857in}{2.885918in}}%
\pgfpathlineto{\pgfqpoint{4.055857in}{2.882968in}}%
\pgfpathmoveto{\pgfqpoint{4.051316in}{2.885918in}}%
\pgfpathlineto{\pgfqpoint{4.051316in}{2.885918in}}%
\pgfpathlineto{\pgfqpoint{4.051316in}{2.888867in}}%
\pgfpathlineto{\pgfqpoint{4.055857in}{2.888867in}}%
\pgfpathlineto{\pgfqpoint{4.055857in}{2.885918in}}%
\pgfpathmoveto{\pgfqpoint{4.046775in}{2.888867in}}%
\pgfpathlineto{\pgfqpoint{4.046775in}{2.888867in}}%
\pgfpathlineto{\pgfqpoint{4.046775in}{2.891816in}}%
\pgfpathlineto{\pgfqpoint{4.051316in}{2.891816in}}%
\pgfpathlineto{\pgfqpoint{4.051316in}{2.888867in}}%
\pgfpathmoveto{\pgfqpoint{4.037693in}{2.894766in}}%
\pgfpathlineto{\pgfqpoint{4.037693in}{2.894766in}}%
\pgfpathlineto{\pgfqpoint{4.037693in}{2.897715in}}%
\pgfpathlineto{\pgfqpoint{4.042234in}{2.897715in}}%
\pgfpathlineto{\pgfqpoint{4.042234in}{2.894766in}}%
\pgfpathmoveto{\pgfqpoint{4.055857in}{2.880019in}}%
\pgfpathlineto{\pgfqpoint{4.055857in}{2.880019in}}%
\pgfpathlineto{\pgfqpoint{4.055857in}{2.882968in}}%
\pgfpathlineto{\pgfqpoint{4.060398in}{2.882968in}}%
\pgfpathlineto{\pgfqpoint{4.060398in}{2.880019in}}%
\pgfpathmoveto{\pgfqpoint{4.060398in}{2.877070in}}%
\pgfpathlineto{\pgfqpoint{4.060398in}{2.877070in}}%
\pgfpathlineto{\pgfqpoint{4.060398in}{2.880019in}}%
\pgfpathlineto{\pgfqpoint{4.064938in}{2.880019in}}%
\pgfpathlineto{\pgfqpoint{4.064938in}{2.877070in}}%
\pgfpathmoveto{\pgfqpoint{4.060398in}{2.880019in}}%
\pgfpathlineto{\pgfqpoint{4.060398in}{2.880019in}}%
\pgfpathlineto{\pgfqpoint{4.060398in}{2.882968in}}%
\pgfpathlineto{\pgfqpoint{4.064938in}{2.882968in}}%
\pgfpathlineto{\pgfqpoint{4.064938in}{2.880019in}}%
\pgfpathmoveto{\pgfqpoint{4.064938in}{2.874120in}}%
\pgfpathlineto{\pgfqpoint{4.064938in}{2.874120in}}%
\pgfpathlineto{\pgfqpoint{4.064938in}{2.877070in}}%
\pgfpathlineto{\pgfqpoint{4.069479in}{2.877070in}}%
\pgfpathlineto{\pgfqpoint{4.069479in}{2.874120in}}%
\pgfpathmoveto{\pgfqpoint{4.069479in}{2.871171in}}%
\pgfpathlineto{\pgfqpoint{4.069479in}{2.871171in}}%
\pgfpathlineto{\pgfqpoint{4.069479in}{2.874120in}}%
\pgfpathlineto{\pgfqpoint{4.074020in}{2.874120in}}%
\pgfpathlineto{\pgfqpoint{4.074020in}{2.871171in}}%
\pgfpathmoveto{\pgfqpoint{4.069479in}{2.874120in}}%
\pgfpathlineto{\pgfqpoint{4.069479in}{2.874120in}}%
\pgfpathlineto{\pgfqpoint{4.069479in}{2.877070in}}%
\pgfpathlineto{\pgfqpoint{4.074020in}{2.877070in}}%
\pgfpathlineto{\pgfqpoint{4.074020in}{2.874120in}}%
\pgfpathmoveto{\pgfqpoint{4.064938in}{2.877070in}}%
\pgfpathlineto{\pgfqpoint{4.064938in}{2.877070in}}%
\pgfpathlineto{\pgfqpoint{4.064938in}{2.880019in}}%
\pgfpathlineto{\pgfqpoint{4.069479in}{2.880019in}}%
\pgfpathlineto{\pgfqpoint{4.069479in}{2.877070in}}%
\pgfpathmoveto{\pgfqpoint{4.074020in}{2.868222in}}%
\pgfpathlineto{\pgfqpoint{4.074020in}{2.868222in}}%
\pgfpathlineto{\pgfqpoint{4.074020in}{2.871171in}}%
\pgfpathlineto{\pgfqpoint{4.078561in}{2.871171in}}%
\pgfpathlineto{\pgfqpoint{4.078561in}{2.868222in}}%
\pgfpathmoveto{\pgfqpoint{4.078561in}{2.865273in}}%
\pgfpathlineto{\pgfqpoint{4.078561in}{2.865273in}}%
\pgfpathlineto{\pgfqpoint{4.078561in}{2.868222in}}%
\pgfpathlineto{\pgfqpoint{4.083102in}{2.868222in}}%
\pgfpathlineto{\pgfqpoint{4.083102in}{2.865273in}}%
\pgfpathmoveto{\pgfqpoint{4.078561in}{2.868222in}}%
\pgfpathlineto{\pgfqpoint{4.078561in}{2.868222in}}%
\pgfpathlineto{\pgfqpoint{4.078561in}{2.871171in}}%
\pgfpathlineto{\pgfqpoint{4.083102in}{2.871171in}}%
\pgfpathlineto{\pgfqpoint{4.083102in}{2.868222in}}%
\pgfpathmoveto{\pgfqpoint{4.083102in}{2.862323in}}%
\pgfpathlineto{\pgfqpoint{4.083102in}{2.862323in}}%
\pgfpathlineto{\pgfqpoint{4.083102in}{2.865273in}}%
\pgfpathlineto{\pgfqpoint{4.087643in}{2.865273in}}%
\pgfpathlineto{\pgfqpoint{4.087643in}{2.862323in}}%
\pgfpathmoveto{\pgfqpoint{4.087643in}{2.859374in}}%
\pgfpathlineto{\pgfqpoint{4.087643in}{2.859374in}}%
\pgfpathlineto{\pgfqpoint{4.087643in}{2.862323in}}%
\pgfpathlineto{\pgfqpoint{4.092184in}{2.862323in}}%
\pgfpathlineto{\pgfqpoint{4.092184in}{2.859374in}}%
\pgfpathmoveto{\pgfqpoint{4.087643in}{2.862323in}}%
\pgfpathlineto{\pgfqpoint{4.087643in}{2.862323in}}%
\pgfpathlineto{\pgfqpoint{4.087643in}{2.865273in}}%
\pgfpathlineto{\pgfqpoint{4.092184in}{2.865273in}}%
\pgfpathlineto{\pgfqpoint{4.092184in}{2.862323in}}%
\pgfpathmoveto{\pgfqpoint{4.083102in}{2.865273in}}%
\pgfpathlineto{\pgfqpoint{4.083102in}{2.865273in}}%
\pgfpathlineto{\pgfqpoint{4.083102in}{2.868222in}}%
\pgfpathlineto{\pgfqpoint{4.087643in}{2.868222in}}%
\pgfpathlineto{\pgfqpoint{4.087643in}{2.865273in}}%
\pgfpathmoveto{\pgfqpoint{4.074020in}{2.871171in}}%
\pgfpathlineto{\pgfqpoint{4.074020in}{2.871171in}}%
\pgfpathlineto{\pgfqpoint{4.074020in}{2.874120in}}%
\pgfpathlineto{\pgfqpoint{4.078561in}{2.874120in}}%
\pgfpathlineto{\pgfqpoint{4.078561in}{2.871171in}}%
\pgfpathmoveto{\pgfqpoint{4.055857in}{2.882968in}}%
\pgfpathlineto{\pgfqpoint{4.055857in}{2.882968in}}%
\pgfpathlineto{\pgfqpoint{4.055857in}{2.885918in}}%
\pgfpathlineto{\pgfqpoint{4.060398in}{2.885918in}}%
\pgfpathlineto{\pgfqpoint{4.060398in}{2.882968in}}%
\pgfpathmoveto{\pgfqpoint{4.019529in}{2.906563in}}%
\pgfpathlineto{\pgfqpoint{4.019529in}{2.906563in}}%
\pgfpathlineto{\pgfqpoint{4.019529in}{2.909512in}}%
\pgfpathlineto{\pgfqpoint{4.024070in}{2.909512in}}%
\pgfpathlineto{\pgfqpoint{4.024070in}{2.906563in}}%
\pgfpathmoveto{\pgfqpoint{4.232961in}{2.762050in}}%
\pgfpathlineto{\pgfqpoint{4.232961in}{2.762050in}}%
\pgfpathlineto{\pgfqpoint{4.232961in}{2.764999in}}%
\pgfpathlineto{\pgfqpoint{4.237502in}{2.764999in}}%
\pgfpathlineto{\pgfqpoint{4.237502in}{2.762050in}}%
\pgfpathmoveto{\pgfqpoint{4.092184in}{2.856425in}}%
\pgfpathlineto{\pgfqpoint{4.092184in}{2.856425in}}%
\pgfpathlineto{\pgfqpoint{4.092184in}{2.859374in}}%
\pgfpathlineto{\pgfqpoint{4.096725in}{2.859374in}}%
\pgfpathlineto{\pgfqpoint{4.096725in}{2.856425in}}%
\pgfpathmoveto{\pgfqpoint{4.096725in}{2.853476in}}%
\pgfpathlineto{\pgfqpoint{4.096725in}{2.853476in}}%
\pgfpathlineto{\pgfqpoint{4.096725in}{2.856425in}}%
\pgfpathlineto{\pgfqpoint{4.101267in}{2.856425in}}%
\pgfpathlineto{\pgfqpoint{4.101267in}{2.853476in}}%
\pgfpathmoveto{\pgfqpoint{4.096725in}{2.856425in}}%
\pgfpathlineto{\pgfqpoint{4.096725in}{2.856425in}}%
\pgfpathlineto{\pgfqpoint{4.096725in}{2.859374in}}%
\pgfpathlineto{\pgfqpoint{4.101267in}{2.859374in}}%
\pgfpathlineto{\pgfqpoint{4.101267in}{2.856425in}}%
\pgfpathmoveto{\pgfqpoint{4.101267in}{2.850526in}}%
\pgfpathlineto{\pgfqpoint{4.101267in}{2.850526in}}%
\pgfpathlineto{\pgfqpoint{4.101267in}{2.853476in}}%
\pgfpathlineto{\pgfqpoint{4.105808in}{2.853476in}}%
\pgfpathlineto{\pgfqpoint{4.105808in}{2.850526in}}%
\pgfpathmoveto{\pgfqpoint{4.105808in}{2.847577in}}%
\pgfpathlineto{\pgfqpoint{4.105808in}{2.847577in}}%
\pgfpathlineto{\pgfqpoint{4.105808in}{2.850526in}}%
\pgfpathlineto{\pgfqpoint{4.110349in}{2.850526in}}%
\pgfpathlineto{\pgfqpoint{4.110349in}{2.847577in}}%
\pgfpathmoveto{\pgfqpoint{4.105808in}{2.850526in}}%
\pgfpathlineto{\pgfqpoint{4.105808in}{2.850526in}}%
\pgfpathlineto{\pgfqpoint{4.105808in}{2.853476in}}%
\pgfpathlineto{\pgfqpoint{4.110349in}{2.853476in}}%
\pgfpathlineto{\pgfqpoint{4.110349in}{2.850526in}}%
\pgfpathmoveto{\pgfqpoint{4.101267in}{2.853476in}}%
\pgfpathlineto{\pgfqpoint{4.101267in}{2.853476in}}%
\pgfpathlineto{\pgfqpoint{4.101267in}{2.856425in}}%
\pgfpathlineto{\pgfqpoint{4.105808in}{2.856425in}}%
\pgfpathlineto{\pgfqpoint{4.105808in}{2.853476in}}%
\pgfpathmoveto{\pgfqpoint{4.110349in}{2.844628in}}%
\pgfpathlineto{\pgfqpoint{4.110349in}{2.844628in}}%
\pgfpathlineto{\pgfqpoint{4.110349in}{2.847577in}}%
\pgfpathlineto{\pgfqpoint{4.114890in}{2.847577in}}%
\pgfpathlineto{\pgfqpoint{4.114890in}{2.844628in}}%
\pgfpathmoveto{\pgfqpoint{4.114890in}{2.841679in}}%
\pgfpathlineto{\pgfqpoint{4.114890in}{2.841679in}}%
\pgfpathlineto{\pgfqpoint{4.114890in}{2.844628in}}%
\pgfpathlineto{\pgfqpoint{4.119431in}{2.844628in}}%
\pgfpathlineto{\pgfqpoint{4.119431in}{2.841679in}}%
\pgfpathmoveto{\pgfqpoint{4.114890in}{2.844628in}}%
\pgfpathlineto{\pgfqpoint{4.114890in}{2.844628in}}%
\pgfpathlineto{\pgfqpoint{4.114890in}{2.847577in}}%
\pgfpathlineto{\pgfqpoint{4.119431in}{2.847577in}}%
\pgfpathlineto{\pgfqpoint{4.119431in}{2.844628in}}%
\pgfpathmoveto{\pgfqpoint{4.119431in}{2.838729in}}%
\pgfpathlineto{\pgfqpoint{4.119431in}{2.838729in}}%
\pgfpathlineto{\pgfqpoint{4.119431in}{2.841679in}}%
\pgfpathlineto{\pgfqpoint{4.123973in}{2.841679in}}%
\pgfpathlineto{\pgfqpoint{4.123973in}{2.838729in}}%
\pgfpathmoveto{\pgfqpoint{4.123973in}{2.835780in}}%
\pgfpathlineto{\pgfqpoint{4.123973in}{2.835780in}}%
\pgfpathlineto{\pgfqpoint{4.123973in}{2.838729in}}%
\pgfpathlineto{\pgfqpoint{4.128514in}{2.838729in}}%
\pgfpathlineto{\pgfqpoint{4.128514in}{2.835780in}}%
\pgfpathmoveto{\pgfqpoint{4.123973in}{2.838729in}}%
\pgfpathlineto{\pgfqpoint{4.123973in}{2.838729in}}%
\pgfpathlineto{\pgfqpoint{4.123973in}{2.841679in}}%
\pgfpathlineto{\pgfqpoint{4.128514in}{2.841679in}}%
\pgfpathlineto{\pgfqpoint{4.128514in}{2.838729in}}%
\pgfpathmoveto{\pgfqpoint{4.119431in}{2.841679in}}%
\pgfpathlineto{\pgfqpoint{4.119431in}{2.841679in}}%
\pgfpathlineto{\pgfqpoint{4.119431in}{2.844628in}}%
\pgfpathlineto{\pgfqpoint{4.123973in}{2.844628in}}%
\pgfpathlineto{\pgfqpoint{4.123973in}{2.841679in}}%
\pgfpathmoveto{\pgfqpoint{4.110349in}{2.847577in}}%
\pgfpathlineto{\pgfqpoint{4.110349in}{2.847577in}}%
\pgfpathlineto{\pgfqpoint{4.110349in}{2.850526in}}%
\pgfpathlineto{\pgfqpoint{4.114890in}{2.850526in}}%
\pgfpathlineto{\pgfqpoint{4.114890in}{2.847577in}}%
\pgfpathmoveto{\pgfqpoint{4.128514in}{2.832831in}}%
\pgfpathlineto{\pgfqpoint{4.128514in}{2.832831in}}%
\pgfpathlineto{\pgfqpoint{4.128514in}{2.835780in}}%
\pgfpathlineto{\pgfqpoint{4.133055in}{2.835780in}}%
\pgfpathlineto{\pgfqpoint{4.133055in}{2.832831in}}%
\pgfpathmoveto{\pgfqpoint{4.133055in}{2.829882in}}%
\pgfpathlineto{\pgfqpoint{4.133055in}{2.829882in}}%
\pgfpathlineto{\pgfqpoint{4.133055in}{2.832831in}}%
\pgfpathlineto{\pgfqpoint{4.137596in}{2.832831in}}%
\pgfpathlineto{\pgfqpoint{4.137596in}{2.829882in}}%
\pgfpathmoveto{\pgfqpoint{4.133055in}{2.832831in}}%
\pgfpathlineto{\pgfqpoint{4.133055in}{2.832831in}}%
\pgfpathlineto{\pgfqpoint{4.133055in}{2.835780in}}%
\pgfpathlineto{\pgfqpoint{4.137596in}{2.835780in}}%
\pgfpathlineto{\pgfqpoint{4.137596in}{2.832831in}}%
\pgfpathmoveto{\pgfqpoint{4.137596in}{2.826933in}}%
\pgfpathlineto{\pgfqpoint{4.137596in}{2.826933in}}%
\pgfpathlineto{\pgfqpoint{4.137596in}{2.829882in}}%
\pgfpathlineto{\pgfqpoint{4.142137in}{2.829882in}}%
\pgfpathlineto{\pgfqpoint{4.142137in}{2.826933in}}%
\pgfpathmoveto{\pgfqpoint{4.142137in}{2.823983in}}%
\pgfpathlineto{\pgfqpoint{4.142137in}{2.823983in}}%
\pgfpathlineto{\pgfqpoint{4.142137in}{2.826933in}}%
\pgfpathlineto{\pgfqpoint{4.146679in}{2.826933in}}%
\pgfpathlineto{\pgfqpoint{4.146679in}{2.823983in}}%
\pgfpathmoveto{\pgfqpoint{4.142137in}{2.826933in}}%
\pgfpathlineto{\pgfqpoint{4.142137in}{2.826933in}}%
\pgfpathlineto{\pgfqpoint{4.142137in}{2.829882in}}%
\pgfpathlineto{\pgfqpoint{4.146679in}{2.829882in}}%
\pgfpathlineto{\pgfqpoint{4.146679in}{2.826933in}}%
\pgfpathmoveto{\pgfqpoint{4.137596in}{2.829882in}}%
\pgfpathlineto{\pgfqpoint{4.137596in}{2.829882in}}%
\pgfpathlineto{\pgfqpoint{4.137596in}{2.832831in}}%
\pgfpathlineto{\pgfqpoint{4.142137in}{2.832831in}}%
\pgfpathlineto{\pgfqpoint{4.142137in}{2.829882in}}%
\pgfpathmoveto{\pgfqpoint{4.146679in}{2.821034in}}%
\pgfpathlineto{\pgfqpoint{4.146679in}{2.821034in}}%
\pgfpathlineto{\pgfqpoint{4.146679in}{2.823983in}}%
\pgfpathlineto{\pgfqpoint{4.151220in}{2.823983in}}%
\pgfpathlineto{\pgfqpoint{4.151220in}{2.821034in}}%
\pgfpathmoveto{\pgfqpoint{4.151220in}{2.818085in}}%
\pgfpathlineto{\pgfqpoint{4.151220in}{2.818085in}}%
\pgfpathlineto{\pgfqpoint{4.151220in}{2.821034in}}%
\pgfpathlineto{\pgfqpoint{4.155761in}{2.821034in}}%
\pgfpathlineto{\pgfqpoint{4.155761in}{2.818085in}}%
\pgfpathmoveto{\pgfqpoint{4.151220in}{2.821034in}}%
\pgfpathlineto{\pgfqpoint{4.151220in}{2.821034in}}%
\pgfpathlineto{\pgfqpoint{4.151220in}{2.823983in}}%
\pgfpathlineto{\pgfqpoint{4.155761in}{2.823983in}}%
\pgfpathlineto{\pgfqpoint{4.155761in}{2.821034in}}%
\pgfpathmoveto{\pgfqpoint{4.155761in}{2.815136in}}%
\pgfpathlineto{\pgfqpoint{4.155761in}{2.815136in}}%
\pgfpathlineto{\pgfqpoint{4.155761in}{2.818085in}}%
\pgfpathlineto{\pgfqpoint{4.160302in}{2.818085in}}%
\pgfpathlineto{\pgfqpoint{4.160302in}{2.815136in}}%
\pgfpathmoveto{\pgfqpoint{4.160302in}{2.812187in}}%
\pgfpathlineto{\pgfqpoint{4.160302in}{2.812187in}}%
\pgfpathlineto{\pgfqpoint{4.160302in}{2.815136in}}%
\pgfpathlineto{\pgfqpoint{4.164843in}{2.815136in}}%
\pgfpathlineto{\pgfqpoint{4.164843in}{2.812187in}}%
\pgfpathmoveto{\pgfqpoint{4.160302in}{2.815136in}}%
\pgfpathlineto{\pgfqpoint{4.160302in}{2.815136in}}%
\pgfpathlineto{\pgfqpoint{4.160302in}{2.818085in}}%
\pgfpathlineto{\pgfqpoint{4.164843in}{2.818085in}}%
\pgfpathlineto{\pgfqpoint{4.164843in}{2.815136in}}%
\pgfpathmoveto{\pgfqpoint{4.155761in}{2.818085in}}%
\pgfpathlineto{\pgfqpoint{4.155761in}{2.818085in}}%
\pgfpathlineto{\pgfqpoint{4.155761in}{2.821034in}}%
\pgfpathlineto{\pgfqpoint{4.160302in}{2.821034in}}%
\pgfpathlineto{\pgfqpoint{4.160302in}{2.818085in}}%
\pgfpathmoveto{\pgfqpoint{4.146679in}{2.823983in}}%
\pgfpathlineto{\pgfqpoint{4.146679in}{2.823983in}}%
\pgfpathlineto{\pgfqpoint{4.146679in}{2.826933in}}%
\pgfpathlineto{\pgfqpoint{4.151220in}{2.826933in}}%
\pgfpathlineto{\pgfqpoint{4.151220in}{2.823983in}}%
\pgfpathmoveto{\pgfqpoint{4.128514in}{2.835780in}}%
\pgfpathlineto{\pgfqpoint{4.128514in}{2.835780in}}%
\pgfpathlineto{\pgfqpoint{4.128514in}{2.838729in}}%
\pgfpathlineto{\pgfqpoint{4.133055in}{2.838729in}}%
\pgfpathlineto{\pgfqpoint{4.133055in}{2.835780in}}%
\pgfpathmoveto{\pgfqpoint{4.164843in}{2.809237in}}%
\pgfpathlineto{\pgfqpoint{4.164843in}{2.809237in}}%
\pgfpathlineto{\pgfqpoint{4.164843in}{2.812187in}}%
\pgfpathlineto{\pgfqpoint{4.169384in}{2.812187in}}%
\pgfpathlineto{\pgfqpoint{4.169384in}{2.809237in}}%
\pgfpathmoveto{\pgfqpoint{4.169384in}{2.806288in}}%
\pgfpathlineto{\pgfqpoint{4.169384in}{2.806288in}}%
\pgfpathlineto{\pgfqpoint{4.169384in}{2.809237in}}%
\pgfpathlineto{\pgfqpoint{4.173926in}{2.809237in}}%
\pgfpathlineto{\pgfqpoint{4.173926in}{2.806288in}}%
\pgfpathmoveto{\pgfqpoint{4.169384in}{2.809237in}}%
\pgfpathlineto{\pgfqpoint{4.169384in}{2.809237in}}%
\pgfpathlineto{\pgfqpoint{4.169384in}{2.812187in}}%
\pgfpathlineto{\pgfqpoint{4.173926in}{2.812187in}}%
\pgfpathlineto{\pgfqpoint{4.173926in}{2.809237in}}%
\pgfpathmoveto{\pgfqpoint{4.173926in}{2.803339in}}%
\pgfpathlineto{\pgfqpoint{4.173926in}{2.803339in}}%
\pgfpathlineto{\pgfqpoint{4.173926in}{2.806288in}}%
\pgfpathlineto{\pgfqpoint{4.178467in}{2.806288in}}%
\pgfpathlineto{\pgfqpoint{4.178467in}{2.803339in}}%
\pgfpathmoveto{\pgfqpoint{4.178467in}{2.800390in}}%
\pgfpathlineto{\pgfqpoint{4.178467in}{2.800390in}}%
\pgfpathlineto{\pgfqpoint{4.178467in}{2.803339in}}%
\pgfpathlineto{\pgfqpoint{4.183008in}{2.803339in}}%
\pgfpathlineto{\pgfqpoint{4.183008in}{2.800390in}}%
\pgfpathmoveto{\pgfqpoint{4.178467in}{2.803339in}}%
\pgfpathlineto{\pgfqpoint{4.178467in}{2.803339in}}%
\pgfpathlineto{\pgfqpoint{4.178467in}{2.806288in}}%
\pgfpathlineto{\pgfqpoint{4.183008in}{2.806288in}}%
\pgfpathlineto{\pgfqpoint{4.183008in}{2.803339in}}%
\pgfpathmoveto{\pgfqpoint{4.173926in}{2.806288in}}%
\pgfpathlineto{\pgfqpoint{4.173926in}{2.806288in}}%
\pgfpathlineto{\pgfqpoint{4.173926in}{2.809237in}}%
\pgfpathlineto{\pgfqpoint{4.178467in}{2.809237in}}%
\pgfpathlineto{\pgfqpoint{4.178467in}{2.806288in}}%
\pgfpathmoveto{\pgfqpoint{4.183008in}{2.797441in}}%
\pgfpathlineto{\pgfqpoint{4.183008in}{2.797441in}}%
\pgfpathlineto{\pgfqpoint{4.183008in}{2.800390in}}%
\pgfpathlineto{\pgfqpoint{4.187549in}{2.800390in}}%
\pgfpathlineto{\pgfqpoint{4.187549in}{2.797441in}}%
\pgfpathmoveto{\pgfqpoint{4.187549in}{2.794491in}}%
\pgfpathlineto{\pgfqpoint{4.187549in}{2.794491in}}%
\pgfpathlineto{\pgfqpoint{4.187549in}{2.797441in}}%
\pgfpathlineto{\pgfqpoint{4.192090in}{2.797441in}}%
\pgfpathlineto{\pgfqpoint{4.192090in}{2.794491in}}%
\pgfpathmoveto{\pgfqpoint{4.187549in}{2.797441in}}%
\pgfpathlineto{\pgfqpoint{4.187549in}{2.797441in}}%
\pgfpathlineto{\pgfqpoint{4.187549in}{2.800390in}}%
\pgfpathlineto{\pgfqpoint{4.192090in}{2.800390in}}%
\pgfpathlineto{\pgfqpoint{4.192090in}{2.797441in}}%
\pgfpathmoveto{\pgfqpoint{4.192090in}{2.791542in}}%
\pgfpathlineto{\pgfqpoint{4.192090in}{2.791542in}}%
\pgfpathlineto{\pgfqpoint{4.192090in}{2.794491in}}%
\pgfpathlineto{\pgfqpoint{4.196632in}{2.794491in}}%
\pgfpathlineto{\pgfqpoint{4.196632in}{2.791542in}}%
\pgfpathmoveto{\pgfqpoint{4.196632in}{2.788593in}}%
\pgfpathlineto{\pgfqpoint{4.196632in}{2.788593in}}%
\pgfpathlineto{\pgfqpoint{4.196632in}{2.791542in}}%
\pgfpathlineto{\pgfqpoint{4.201173in}{2.791542in}}%
\pgfpathlineto{\pgfqpoint{4.201173in}{2.788593in}}%
\pgfpathmoveto{\pgfqpoint{4.196632in}{2.791542in}}%
\pgfpathlineto{\pgfqpoint{4.196632in}{2.791542in}}%
\pgfpathlineto{\pgfqpoint{4.196632in}{2.794491in}}%
\pgfpathlineto{\pgfqpoint{4.201173in}{2.794491in}}%
\pgfpathlineto{\pgfqpoint{4.201173in}{2.791542in}}%
\pgfpathmoveto{\pgfqpoint{4.192090in}{2.794491in}}%
\pgfpathlineto{\pgfqpoint{4.192090in}{2.794491in}}%
\pgfpathlineto{\pgfqpoint{4.192090in}{2.797441in}}%
\pgfpathlineto{\pgfqpoint{4.196632in}{2.797441in}}%
\pgfpathlineto{\pgfqpoint{4.196632in}{2.794491in}}%
\pgfpathmoveto{\pgfqpoint{4.183008in}{2.800390in}}%
\pgfpathlineto{\pgfqpoint{4.183008in}{2.800390in}}%
\pgfpathlineto{\pgfqpoint{4.183008in}{2.803339in}}%
\pgfpathlineto{\pgfqpoint{4.187549in}{2.803339in}}%
\pgfpathlineto{\pgfqpoint{4.187549in}{2.800390in}}%
\pgfpathmoveto{\pgfqpoint{4.214796in}{2.773847in}}%
\pgfpathlineto{\pgfqpoint{4.214796in}{2.773847in}}%
\pgfpathlineto{\pgfqpoint{4.214796in}{2.776796in}}%
\pgfpathlineto{\pgfqpoint{4.219338in}{2.776796in}}%
\pgfpathlineto{\pgfqpoint{4.219338in}{2.773847in}}%
\pgfpathmoveto{\pgfqpoint{4.205714in}{2.779745in}}%
\pgfpathlineto{\pgfqpoint{4.205714in}{2.779745in}}%
\pgfpathlineto{\pgfqpoint{4.205714in}{2.782694in}}%
\pgfpathlineto{\pgfqpoint{4.210255in}{2.782694in}}%
\pgfpathlineto{\pgfqpoint{4.210255in}{2.779745in}}%
\pgfpathmoveto{\pgfqpoint{4.201173in}{2.785644in}}%
\pgfpathlineto{\pgfqpoint{4.201173in}{2.785644in}}%
\pgfpathlineto{\pgfqpoint{4.201173in}{2.788593in}}%
\pgfpathlineto{\pgfqpoint{4.205714in}{2.788593in}}%
\pgfpathlineto{\pgfqpoint{4.205714in}{2.785644in}}%
\pgfpathmoveto{\pgfqpoint{4.205714in}{2.782694in}}%
\pgfpathlineto{\pgfqpoint{4.205714in}{2.782694in}}%
\pgfpathlineto{\pgfqpoint{4.205714in}{2.785644in}}%
\pgfpathlineto{\pgfqpoint{4.210255in}{2.785644in}}%
\pgfpathlineto{\pgfqpoint{4.210255in}{2.782694in}}%
\pgfpathmoveto{\pgfqpoint{4.205714in}{2.785644in}}%
\pgfpathlineto{\pgfqpoint{4.205714in}{2.785644in}}%
\pgfpathlineto{\pgfqpoint{4.205714in}{2.788593in}}%
\pgfpathlineto{\pgfqpoint{4.210255in}{2.788593in}}%
\pgfpathlineto{\pgfqpoint{4.210255in}{2.785644in}}%
\pgfpathmoveto{\pgfqpoint{4.210255in}{2.776796in}}%
\pgfpathlineto{\pgfqpoint{4.210255in}{2.776796in}}%
\pgfpathlineto{\pgfqpoint{4.210255in}{2.779745in}}%
\pgfpathlineto{\pgfqpoint{4.214796in}{2.779745in}}%
\pgfpathlineto{\pgfqpoint{4.214796in}{2.776796in}}%
\pgfpathmoveto{\pgfqpoint{4.210255in}{2.779745in}}%
\pgfpathlineto{\pgfqpoint{4.210255in}{2.779745in}}%
\pgfpathlineto{\pgfqpoint{4.210255in}{2.782694in}}%
\pgfpathlineto{\pgfqpoint{4.214796in}{2.782694in}}%
\pgfpathlineto{\pgfqpoint{4.214796in}{2.779745in}}%
\pgfpathmoveto{\pgfqpoint{4.214796in}{2.776796in}}%
\pgfpathlineto{\pgfqpoint{4.214796in}{2.776796in}}%
\pgfpathlineto{\pgfqpoint{4.214796in}{2.779745in}}%
\pgfpathlineto{\pgfqpoint{4.219338in}{2.779745in}}%
\pgfpathlineto{\pgfqpoint{4.219338in}{2.776796in}}%
\pgfpathmoveto{\pgfqpoint{4.223879in}{2.767948in}}%
\pgfpathlineto{\pgfqpoint{4.223879in}{2.767948in}}%
\pgfpathlineto{\pgfqpoint{4.223879in}{2.770898in}}%
\pgfpathlineto{\pgfqpoint{4.228420in}{2.770898in}}%
\pgfpathlineto{\pgfqpoint{4.228420in}{2.767948in}}%
\pgfpathmoveto{\pgfqpoint{4.219338in}{2.770898in}}%
\pgfpathlineto{\pgfqpoint{4.219338in}{2.770898in}}%
\pgfpathlineto{\pgfqpoint{4.219338in}{2.773847in}}%
\pgfpathlineto{\pgfqpoint{4.223879in}{2.773847in}}%
\pgfpathlineto{\pgfqpoint{4.223879in}{2.770898in}}%
\pgfpathmoveto{\pgfqpoint{4.219338in}{2.773847in}}%
\pgfpathlineto{\pgfqpoint{4.219338in}{2.773847in}}%
\pgfpathlineto{\pgfqpoint{4.219338in}{2.776796in}}%
\pgfpathlineto{\pgfqpoint{4.223879in}{2.776796in}}%
\pgfpathlineto{\pgfqpoint{4.223879in}{2.773847in}}%
\pgfpathmoveto{\pgfqpoint{4.223879in}{2.770898in}}%
\pgfpathlineto{\pgfqpoint{4.223879in}{2.770898in}}%
\pgfpathlineto{\pgfqpoint{4.223879in}{2.773847in}}%
\pgfpathlineto{\pgfqpoint{4.228420in}{2.773847in}}%
\pgfpathlineto{\pgfqpoint{4.228420in}{2.770898in}}%
\pgfpathmoveto{\pgfqpoint{4.228420in}{2.764999in}}%
\pgfpathlineto{\pgfqpoint{4.228420in}{2.764999in}}%
\pgfpathlineto{\pgfqpoint{4.228420in}{2.767948in}}%
\pgfpathlineto{\pgfqpoint{4.232961in}{2.767948in}}%
\pgfpathlineto{\pgfqpoint{4.232961in}{2.764999in}}%
\pgfpathmoveto{\pgfqpoint{4.228420in}{2.767948in}}%
\pgfpathlineto{\pgfqpoint{4.228420in}{2.767948in}}%
\pgfpathlineto{\pgfqpoint{4.228420in}{2.770898in}}%
\pgfpathlineto{\pgfqpoint{4.232961in}{2.770898in}}%
\pgfpathlineto{\pgfqpoint{4.232961in}{2.767948in}}%
\pgfpathmoveto{\pgfqpoint{4.232961in}{2.764999in}}%
\pgfpathlineto{\pgfqpoint{4.232961in}{2.764999in}}%
\pgfpathlineto{\pgfqpoint{4.232961in}{2.767948in}}%
\pgfpathlineto{\pgfqpoint{4.237502in}{2.767948in}}%
\pgfpathlineto{\pgfqpoint{4.237502in}{2.764999in}}%
\pgfpathmoveto{\pgfqpoint{4.201173in}{2.788593in}}%
\pgfpathlineto{\pgfqpoint{4.201173in}{2.788593in}}%
\pgfpathlineto{\pgfqpoint{4.201173in}{2.791542in}}%
\pgfpathlineto{\pgfqpoint{4.205714in}{2.791542in}}%
\pgfpathlineto{\pgfqpoint{4.205714in}{2.788593in}}%
\pgfpathmoveto{\pgfqpoint{4.164843in}{2.812187in}}%
\pgfpathlineto{\pgfqpoint{4.164843in}{2.812187in}}%
\pgfpathlineto{\pgfqpoint{4.164843in}{2.815136in}}%
\pgfpathlineto{\pgfqpoint{4.169384in}{2.815136in}}%
\pgfpathlineto{\pgfqpoint{4.169384in}{2.812187in}}%
\pgfpathmoveto{\pgfqpoint{4.092184in}{2.859374in}}%
\pgfpathlineto{\pgfqpoint{4.092184in}{2.859374in}}%
\pgfpathlineto{\pgfqpoint{4.092184in}{2.862323in}}%
\pgfpathlineto{\pgfqpoint{4.096725in}{2.862323in}}%
\pgfpathlineto{\pgfqpoint{4.096725in}{2.859374in}}%
\pgfpathmoveto{\pgfqpoint{4.269289in}{2.738456in}}%
\pgfpathlineto{\pgfqpoint{4.269289in}{2.738456in}}%
\pgfpathlineto{\pgfqpoint{4.269289in}{2.741405in}}%
\pgfpathlineto{\pgfqpoint{4.273830in}{2.741405in}}%
\pgfpathlineto{\pgfqpoint{4.273830in}{2.738456in}}%
\pgfpathmoveto{\pgfqpoint{4.251125in}{2.750253in}}%
\pgfpathlineto{\pgfqpoint{4.251125in}{2.750253in}}%
\pgfpathlineto{\pgfqpoint{4.251125in}{2.753202in}}%
\pgfpathlineto{\pgfqpoint{4.255666in}{2.753202in}}%
\pgfpathlineto{\pgfqpoint{4.255666in}{2.750253in}}%
\pgfpathmoveto{\pgfqpoint{4.242043in}{2.756152in}}%
\pgfpathlineto{\pgfqpoint{4.242043in}{2.756152in}}%
\pgfpathlineto{\pgfqpoint{4.242043in}{2.759101in}}%
\pgfpathlineto{\pgfqpoint{4.246584in}{2.759101in}}%
\pgfpathlineto{\pgfqpoint{4.246584in}{2.756152in}}%
\pgfpathmoveto{\pgfqpoint{4.237502in}{2.759101in}}%
\pgfpathlineto{\pgfqpoint{4.237502in}{2.759101in}}%
\pgfpathlineto{\pgfqpoint{4.237502in}{2.762050in}}%
\pgfpathlineto{\pgfqpoint{4.242043in}{2.762050in}}%
\pgfpathlineto{\pgfqpoint{4.242043in}{2.759101in}}%
\pgfpathmoveto{\pgfqpoint{4.237502in}{2.762050in}}%
\pgfpathlineto{\pgfqpoint{4.237502in}{2.762050in}}%
\pgfpathlineto{\pgfqpoint{4.237502in}{2.764999in}}%
\pgfpathlineto{\pgfqpoint{4.242043in}{2.764999in}}%
\pgfpathlineto{\pgfqpoint{4.242043in}{2.762050in}}%
\pgfpathmoveto{\pgfqpoint{4.242043in}{2.759101in}}%
\pgfpathlineto{\pgfqpoint{4.242043in}{2.759101in}}%
\pgfpathlineto{\pgfqpoint{4.242043in}{2.762050in}}%
\pgfpathlineto{\pgfqpoint{4.246584in}{2.762050in}}%
\pgfpathlineto{\pgfqpoint{4.246584in}{2.759101in}}%
\pgfpathmoveto{\pgfqpoint{4.246584in}{2.753202in}}%
\pgfpathlineto{\pgfqpoint{4.246584in}{2.753202in}}%
\pgfpathlineto{\pgfqpoint{4.246584in}{2.756152in}}%
\pgfpathlineto{\pgfqpoint{4.251125in}{2.756152in}}%
\pgfpathlineto{\pgfqpoint{4.251125in}{2.753202in}}%
\pgfpathmoveto{\pgfqpoint{4.246584in}{2.756152in}}%
\pgfpathlineto{\pgfqpoint{4.246584in}{2.756152in}}%
\pgfpathlineto{\pgfqpoint{4.246584in}{2.759101in}}%
\pgfpathlineto{\pgfqpoint{4.251125in}{2.759101in}}%
\pgfpathlineto{\pgfqpoint{4.251125in}{2.756152in}}%
\pgfpathmoveto{\pgfqpoint{4.251125in}{2.753202in}}%
\pgfpathlineto{\pgfqpoint{4.251125in}{2.753202in}}%
\pgfpathlineto{\pgfqpoint{4.251125in}{2.756152in}}%
\pgfpathlineto{\pgfqpoint{4.255666in}{2.756152in}}%
\pgfpathlineto{\pgfqpoint{4.255666in}{2.753202in}}%
\pgfpathmoveto{\pgfqpoint{4.260207in}{2.744355in}}%
\pgfpathlineto{\pgfqpoint{4.260207in}{2.744355in}}%
\pgfpathlineto{\pgfqpoint{4.260207in}{2.747304in}}%
\pgfpathlineto{\pgfqpoint{4.264748in}{2.747304in}}%
\pgfpathlineto{\pgfqpoint{4.264748in}{2.744355in}}%
\pgfpathmoveto{\pgfqpoint{4.255666in}{2.747304in}}%
\pgfpathlineto{\pgfqpoint{4.255666in}{2.747304in}}%
\pgfpathlineto{\pgfqpoint{4.255666in}{2.750253in}}%
\pgfpathlineto{\pgfqpoint{4.260207in}{2.750253in}}%
\pgfpathlineto{\pgfqpoint{4.260207in}{2.747304in}}%
\pgfpathmoveto{\pgfqpoint{4.255666in}{2.750253in}}%
\pgfpathlineto{\pgfqpoint{4.255666in}{2.750253in}}%
\pgfpathlineto{\pgfqpoint{4.255666in}{2.753202in}}%
\pgfpathlineto{\pgfqpoint{4.260207in}{2.753202in}}%
\pgfpathlineto{\pgfqpoint{4.260207in}{2.750253in}}%
\pgfpathmoveto{\pgfqpoint{4.260207in}{2.747304in}}%
\pgfpathlineto{\pgfqpoint{4.260207in}{2.747304in}}%
\pgfpathlineto{\pgfqpoint{4.260207in}{2.750253in}}%
\pgfpathlineto{\pgfqpoint{4.264748in}{2.750253in}}%
\pgfpathlineto{\pgfqpoint{4.264748in}{2.747304in}}%
\pgfpathmoveto{\pgfqpoint{4.264748in}{2.741405in}}%
\pgfpathlineto{\pgfqpoint{4.264748in}{2.741405in}}%
\pgfpathlineto{\pgfqpoint{4.264748in}{2.744355in}}%
\pgfpathlineto{\pgfqpoint{4.269289in}{2.744355in}}%
\pgfpathlineto{\pgfqpoint{4.269289in}{2.741405in}}%
\pgfpathmoveto{\pgfqpoint{4.264748in}{2.744355in}}%
\pgfpathlineto{\pgfqpoint{4.264748in}{2.744355in}}%
\pgfpathlineto{\pgfqpoint{4.264748in}{2.747304in}}%
\pgfpathlineto{\pgfqpoint{4.269289in}{2.747304in}}%
\pgfpathlineto{\pgfqpoint{4.269289in}{2.744355in}}%
\pgfpathmoveto{\pgfqpoint{4.269289in}{2.741405in}}%
\pgfpathlineto{\pgfqpoint{4.269289in}{2.741405in}}%
\pgfpathlineto{\pgfqpoint{4.269289in}{2.744355in}}%
\pgfpathlineto{\pgfqpoint{4.273830in}{2.744355in}}%
\pgfpathlineto{\pgfqpoint{4.273830in}{2.741405in}}%
\pgfpathmoveto{\pgfqpoint{4.287453in}{2.726659in}}%
\pgfpathlineto{\pgfqpoint{4.287453in}{2.726659in}}%
\pgfpathlineto{\pgfqpoint{4.287453in}{2.729609in}}%
\pgfpathlineto{\pgfqpoint{4.291994in}{2.729609in}}%
\pgfpathlineto{\pgfqpoint{4.291994in}{2.726659in}}%
\pgfpathmoveto{\pgfqpoint{4.278371in}{2.732558in}}%
\pgfpathlineto{\pgfqpoint{4.278371in}{2.732558in}}%
\pgfpathlineto{\pgfqpoint{4.278371in}{2.735507in}}%
\pgfpathlineto{\pgfqpoint{4.282912in}{2.735507in}}%
\pgfpathlineto{\pgfqpoint{4.282912in}{2.732558in}}%
\pgfpathmoveto{\pgfqpoint{4.273830in}{2.735507in}}%
\pgfpathlineto{\pgfqpoint{4.273830in}{2.735507in}}%
\pgfpathlineto{\pgfqpoint{4.273830in}{2.738456in}}%
\pgfpathlineto{\pgfqpoint{4.278371in}{2.738456in}}%
\pgfpathlineto{\pgfqpoint{4.278371in}{2.735507in}}%
\pgfpathmoveto{\pgfqpoint{4.273830in}{2.738456in}}%
\pgfpathlineto{\pgfqpoint{4.273830in}{2.738456in}}%
\pgfpathlineto{\pgfqpoint{4.273830in}{2.741405in}}%
\pgfpathlineto{\pgfqpoint{4.278371in}{2.741405in}}%
\pgfpathlineto{\pgfqpoint{4.278371in}{2.738456in}}%
\pgfpathmoveto{\pgfqpoint{4.278371in}{2.735507in}}%
\pgfpathlineto{\pgfqpoint{4.278371in}{2.735507in}}%
\pgfpathlineto{\pgfqpoint{4.278371in}{2.738456in}}%
\pgfpathlineto{\pgfqpoint{4.282912in}{2.738456in}}%
\pgfpathlineto{\pgfqpoint{4.282912in}{2.735507in}}%
\pgfpathmoveto{\pgfqpoint{4.282912in}{2.729609in}}%
\pgfpathlineto{\pgfqpoint{4.282912in}{2.729609in}}%
\pgfpathlineto{\pgfqpoint{4.282912in}{2.732558in}}%
\pgfpathlineto{\pgfqpoint{4.287453in}{2.732558in}}%
\pgfpathlineto{\pgfqpoint{4.287453in}{2.729609in}}%
\pgfpathmoveto{\pgfqpoint{4.282912in}{2.732558in}}%
\pgfpathlineto{\pgfqpoint{4.282912in}{2.732558in}}%
\pgfpathlineto{\pgfqpoint{4.282912in}{2.735507in}}%
\pgfpathlineto{\pgfqpoint{4.287453in}{2.735507in}}%
\pgfpathlineto{\pgfqpoint{4.287453in}{2.732558in}}%
\pgfpathmoveto{\pgfqpoint{4.287453in}{2.729609in}}%
\pgfpathlineto{\pgfqpoint{4.287453in}{2.729609in}}%
\pgfpathlineto{\pgfqpoint{4.287453in}{2.732558in}}%
\pgfpathlineto{\pgfqpoint{4.291994in}{2.732558in}}%
\pgfpathlineto{\pgfqpoint{4.291994in}{2.729609in}}%
\pgfpathmoveto{\pgfqpoint{4.296535in}{2.720761in}}%
\pgfpathlineto{\pgfqpoint{4.296535in}{2.720761in}}%
\pgfpathlineto{\pgfqpoint{4.296535in}{2.723710in}}%
\pgfpathlineto{\pgfqpoint{4.301075in}{2.723710in}}%
\pgfpathlineto{\pgfqpoint{4.301075in}{2.720761in}}%
\pgfpathmoveto{\pgfqpoint{4.291994in}{2.723710in}}%
\pgfpathlineto{\pgfqpoint{4.291994in}{2.723710in}}%
\pgfpathlineto{\pgfqpoint{4.291994in}{2.726659in}}%
\pgfpathlineto{\pgfqpoint{4.296535in}{2.726659in}}%
\pgfpathlineto{\pgfqpoint{4.296535in}{2.723710in}}%
\pgfpathmoveto{\pgfqpoint{4.291994in}{2.726659in}}%
\pgfpathlineto{\pgfqpoint{4.291994in}{2.726659in}}%
\pgfpathlineto{\pgfqpoint{4.291994in}{2.729609in}}%
\pgfpathlineto{\pgfqpoint{4.296535in}{2.729609in}}%
\pgfpathlineto{\pgfqpoint{4.296535in}{2.726659in}}%
\pgfpathmoveto{\pgfqpoint{4.296535in}{2.723710in}}%
\pgfpathlineto{\pgfqpoint{4.296535in}{2.723710in}}%
\pgfpathlineto{\pgfqpoint{4.296535in}{2.726659in}}%
\pgfpathlineto{\pgfqpoint{4.301075in}{2.726659in}}%
\pgfpathlineto{\pgfqpoint{4.301075in}{2.723710in}}%
\pgfpathmoveto{\pgfqpoint{4.301075in}{2.720761in}}%
\pgfpathlineto{\pgfqpoint{4.301075in}{2.720761in}}%
\pgfpathlineto{\pgfqpoint{4.301075in}{2.723710in}}%
\pgfpathlineto{\pgfqpoint{4.305616in}{2.723710in}}%
\pgfpathlineto{\pgfqpoint{4.305616in}{2.720761in}}%
\pgfpathmoveto{\pgfqpoint{4.305616in}{2.717812in}}%
\pgfpathlineto{\pgfqpoint{4.305616in}{2.717812in}}%
\pgfpathlineto{\pgfqpoint{4.305616in}{2.720761in}}%
\pgfpathlineto{\pgfqpoint{4.310157in}{2.720761in}}%
\pgfpathlineto{\pgfqpoint{4.310157in}{2.717812in}}%
\pgfpathmoveto{\pgfqpoint{4.305616in}{2.720761in}}%
\pgfpathlineto{\pgfqpoint{4.305616in}{2.720761in}}%
\pgfpathlineto{\pgfqpoint{4.305616in}{2.723710in}}%
\pgfpathlineto{\pgfqpoint{4.310157in}{2.723710in}}%
\pgfpathlineto{\pgfqpoint{4.310157in}{2.720761in}}%
\pgfpathmoveto{\pgfqpoint{4.310157in}{2.714863in}}%
\pgfpathlineto{\pgfqpoint{4.310157in}{2.714863in}}%
\pgfpathlineto{\pgfqpoint{4.310157in}{2.717812in}}%
\pgfpathlineto{\pgfqpoint{4.314698in}{2.717812in}}%
\pgfpathlineto{\pgfqpoint{4.314698in}{2.714863in}}%
\pgfpathmoveto{\pgfqpoint{4.314698in}{2.711913in}}%
\pgfpathlineto{\pgfqpoint{4.314698in}{2.711913in}}%
\pgfpathlineto{\pgfqpoint{4.314698in}{2.714863in}}%
\pgfpathlineto{\pgfqpoint{4.319239in}{2.714863in}}%
\pgfpathlineto{\pgfqpoint{4.319239in}{2.711913in}}%
\pgfpathmoveto{\pgfqpoint{4.314698in}{2.714863in}}%
\pgfpathlineto{\pgfqpoint{4.314698in}{2.714863in}}%
\pgfpathlineto{\pgfqpoint{4.314698in}{2.717812in}}%
\pgfpathlineto{\pgfqpoint{4.319239in}{2.717812in}}%
\pgfpathlineto{\pgfqpoint{4.319239in}{2.714863in}}%
\pgfpathmoveto{\pgfqpoint{4.319239in}{2.708964in}}%
\pgfpathlineto{\pgfqpoint{4.319239in}{2.708964in}}%
\pgfpathlineto{\pgfqpoint{4.319239in}{2.711913in}}%
\pgfpathlineto{\pgfqpoint{4.323780in}{2.711913in}}%
\pgfpathlineto{\pgfqpoint{4.323780in}{2.708964in}}%
\pgfpathmoveto{\pgfqpoint{4.323780in}{2.706015in}}%
\pgfpathlineto{\pgfqpoint{4.323780in}{2.706015in}}%
\pgfpathlineto{\pgfqpoint{4.323780in}{2.708964in}}%
\pgfpathlineto{\pgfqpoint{4.328321in}{2.708964in}}%
\pgfpathlineto{\pgfqpoint{4.328321in}{2.706015in}}%
\pgfpathmoveto{\pgfqpoint{4.323780in}{2.708964in}}%
\pgfpathlineto{\pgfqpoint{4.323780in}{2.708964in}}%
\pgfpathlineto{\pgfqpoint{4.323780in}{2.711913in}}%
\pgfpathlineto{\pgfqpoint{4.328321in}{2.711913in}}%
\pgfpathlineto{\pgfqpoint{4.328321in}{2.708964in}}%
\pgfpathmoveto{\pgfqpoint{4.319239in}{2.711913in}}%
\pgfpathlineto{\pgfqpoint{4.319239in}{2.711913in}}%
\pgfpathlineto{\pgfqpoint{4.319239in}{2.714863in}}%
\pgfpathlineto{\pgfqpoint{4.323780in}{2.714863in}}%
\pgfpathlineto{\pgfqpoint{4.323780in}{2.711913in}}%
\pgfpathmoveto{\pgfqpoint{4.328321in}{2.703066in}}%
\pgfpathlineto{\pgfqpoint{4.328321in}{2.703066in}}%
\pgfpathlineto{\pgfqpoint{4.328321in}{2.706015in}}%
\pgfpathlineto{\pgfqpoint{4.332862in}{2.706015in}}%
\pgfpathlineto{\pgfqpoint{4.332862in}{2.703066in}}%
\pgfpathmoveto{\pgfqpoint{4.332862in}{2.700116in}}%
\pgfpathlineto{\pgfqpoint{4.332862in}{2.700116in}}%
\pgfpathlineto{\pgfqpoint{4.332862in}{2.703066in}}%
\pgfpathlineto{\pgfqpoint{4.337403in}{2.703066in}}%
\pgfpathlineto{\pgfqpoint{4.337403in}{2.700116in}}%
\pgfpathmoveto{\pgfqpoint{4.332862in}{2.703066in}}%
\pgfpathlineto{\pgfqpoint{4.332862in}{2.703066in}}%
\pgfpathlineto{\pgfqpoint{4.332862in}{2.706015in}}%
\pgfpathlineto{\pgfqpoint{4.337403in}{2.706015in}}%
\pgfpathlineto{\pgfqpoint{4.337403in}{2.703066in}}%
\pgfpathmoveto{\pgfqpoint{4.337403in}{2.697167in}}%
\pgfpathlineto{\pgfqpoint{4.337403in}{2.697167in}}%
\pgfpathlineto{\pgfqpoint{4.337403in}{2.700116in}}%
\pgfpathlineto{\pgfqpoint{4.341944in}{2.700116in}}%
\pgfpathlineto{\pgfqpoint{4.341944in}{2.697167in}}%
\pgfpathmoveto{\pgfqpoint{4.341944in}{2.694218in}}%
\pgfpathlineto{\pgfqpoint{4.341944in}{2.694218in}}%
\pgfpathlineto{\pgfqpoint{4.341944in}{2.697167in}}%
\pgfpathlineto{\pgfqpoint{4.346485in}{2.697167in}}%
\pgfpathlineto{\pgfqpoint{4.346485in}{2.694218in}}%
\pgfpathmoveto{\pgfqpoint{4.341944in}{2.697167in}}%
\pgfpathlineto{\pgfqpoint{4.341944in}{2.697167in}}%
\pgfpathlineto{\pgfqpoint{4.341944in}{2.700116in}}%
\pgfpathlineto{\pgfqpoint{4.346485in}{2.700116in}}%
\pgfpathlineto{\pgfqpoint{4.346485in}{2.697167in}}%
\pgfpathmoveto{\pgfqpoint{4.337403in}{2.700116in}}%
\pgfpathlineto{\pgfqpoint{4.337403in}{2.700116in}}%
\pgfpathlineto{\pgfqpoint{4.337403in}{2.703066in}}%
\pgfpathlineto{\pgfqpoint{4.341944in}{2.703066in}}%
\pgfpathlineto{\pgfqpoint{4.341944in}{2.700116in}}%
\pgfpathmoveto{\pgfqpoint{4.328321in}{2.706015in}}%
\pgfpathlineto{\pgfqpoint{4.328321in}{2.706015in}}%
\pgfpathlineto{\pgfqpoint{4.328321in}{2.708964in}}%
\pgfpathlineto{\pgfqpoint{4.332862in}{2.708964in}}%
\pgfpathlineto{\pgfqpoint{4.332862in}{2.706015in}}%
\pgfpathmoveto{\pgfqpoint{4.346485in}{2.691269in}}%
\pgfpathlineto{\pgfqpoint{4.346485in}{2.691269in}}%
\pgfpathlineto{\pgfqpoint{4.346485in}{2.694218in}}%
\pgfpathlineto{\pgfqpoint{4.351026in}{2.694218in}}%
\pgfpathlineto{\pgfqpoint{4.351026in}{2.691269in}}%
\pgfpathmoveto{\pgfqpoint{4.351026in}{2.688320in}}%
\pgfpathlineto{\pgfqpoint{4.351026in}{2.688320in}}%
\pgfpathlineto{\pgfqpoint{4.351026in}{2.691269in}}%
\pgfpathlineto{\pgfqpoint{4.355567in}{2.691269in}}%
\pgfpathlineto{\pgfqpoint{4.355567in}{2.688320in}}%
\pgfpathmoveto{\pgfqpoint{4.351026in}{2.691269in}}%
\pgfpathlineto{\pgfqpoint{4.351026in}{2.691269in}}%
\pgfpathlineto{\pgfqpoint{4.351026in}{2.694218in}}%
\pgfpathlineto{\pgfqpoint{4.355567in}{2.694218in}}%
\pgfpathlineto{\pgfqpoint{4.355567in}{2.691269in}}%
\pgfpathmoveto{\pgfqpoint{4.355567in}{2.685370in}}%
\pgfpathlineto{\pgfqpoint{4.355567in}{2.685370in}}%
\pgfpathlineto{\pgfqpoint{4.355567in}{2.688320in}}%
\pgfpathlineto{\pgfqpoint{4.360108in}{2.688320in}}%
\pgfpathlineto{\pgfqpoint{4.360108in}{2.685370in}}%
\pgfpathmoveto{\pgfqpoint{4.360108in}{2.682421in}}%
\pgfpathlineto{\pgfqpoint{4.360108in}{2.682421in}}%
\pgfpathlineto{\pgfqpoint{4.360108in}{2.685370in}}%
\pgfpathlineto{\pgfqpoint{4.364649in}{2.685370in}}%
\pgfpathlineto{\pgfqpoint{4.364649in}{2.682421in}}%
\pgfpathmoveto{\pgfqpoint{4.360108in}{2.685370in}}%
\pgfpathlineto{\pgfqpoint{4.360108in}{2.685370in}}%
\pgfpathlineto{\pgfqpoint{4.360108in}{2.688320in}}%
\pgfpathlineto{\pgfqpoint{4.364649in}{2.688320in}}%
\pgfpathlineto{\pgfqpoint{4.364649in}{2.685370in}}%
\pgfpathmoveto{\pgfqpoint{4.355567in}{2.688320in}}%
\pgfpathlineto{\pgfqpoint{4.355567in}{2.688320in}}%
\pgfpathlineto{\pgfqpoint{4.355567in}{2.691269in}}%
\pgfpathlineto{\pgfqpoint{4.360108in}{2.691269in}}%
\pgfpathlineto{\pgfqpoint{4.360108in}{2.688320in}}%
\pgfpathmoveto{\pgfqpoint{4.364649in}{2.679472in}}%
\pgfpathlineto{\pgfqpoint{4.364649in}{2.679472in}}%
\pgfpathlineto{\pgfqpoint{4.364649in}{2.682421in}}%
\pgfpathlineto{\pgfqpoint{4.369189in}{2.682421in}}%
\pgfpathlineto{\pgfqpoint{4.369189in}{2.679472in}}%
\pgfpathmoveto{\pgfqpoint{4.369189in}{2.676523in}}%
\pgfpathlineto{\pgfqpoint{4.369189in}{2.676523in}}%
\pgfpathlineto{\pgfqpoint{4.369189in}{2.679472in}}%
\pgfpathlineto{\pgfqpoint{4.373730in}{2.679472in}}%
\pgfpathlineto{\pgfqpoint{4.373730in}{2.676523in}}%
\pgfpathmoveto{\pgfqpoint{4.369189in}{2.679472in}}%
\pgfpathlineto{\pgfqpoint{4.369189in}{2.679472in}}%
\pgfpathlineto{\pgfqpoint{4.369189in}{2.682421in}}%
\pgfpathlineto{\pgfqpoint{4.373730in}{2.682421in}}%
\pgfpathlineto{\pgfqpoint{4.373730in}{2.679472in}}%
\pgfpathmoveto{\pgfqpoint{4.373730in}{2.673573in}}%
\pgfpathlineto{\pgfqpoint{4.373730in}{2.673573in}}%
\pgfpathlineto{\pgfqpoint{4.373730in}{2.676523in}}%
\pgfpathlineto{\pgfqpoint{4.378271in}{2.676523in}}%
\pgfpathlineto{\pgfqpoint{4.378271in}{2.673573in}}%
\pgfpathmoveto{\pgfqpoint{4.378271in}{2.670624in}}%
\pgfpathlineto{\pgfqpoint{4.378271in}{2.670624in}}%
\pgfpathlineto{\pgfqpoint{4.378271in}{2.673573in}}%
\pgfpathlineto{\pgfqpoint{4.382812in}{2.673573in}}%
\pgfpathlineto{\pgfqpoint{4.382812in}{2.670624in}}%
\pgfpathmoveto{\pgfqpoint{4.378271in}{2.673573in}}%
\pgfpathlineto{\pgfqpoint{4.378271in}{2.673573in}}%
\pgfpathlineto{\pgfqpoint{4.378271in}{2.676523in}}%
\pgfpathlineto{\pgfqpoint{4.382812in}{2.676523in}}%
\pgfpathlineto{\pgfqpoint{4.382812in}{2.673573in}}%
\pgfpathmoveto{\pgfqpoint{4.373730in}{2.676523in}}%
\pgfpathlineto{\pgfqpoint{4.373730in}{2.676523in}}%
\pgfpathlineto{\pgfqpoint{4.373730in}{2.679472in}}%
\pgfpathlineto{\pgfqpoint{4.378271in}{2.679472in}}%
\pgfpathlineto{\pgfqpoint{4.378271in}{2.676523in}}%
\pgfpathmoveto{\pgfqpoint{4.364649in}{2.682421in}}%
\pgfpathlineto{\pgfqpoint{4.364649in}{2.682421in}}%
\pgfpathlineto{\pgfqpoint{4.364649in}{2.685370in}}%
\pgfpathlineto{\pgfqpoint{4.369189in}{2.685370in}}%
\pgfpathlineto{\pgfqpoint{4.369189in}{2.682421in}}%
\pgfpathmoveto{\pgfqpoint{4.346485in}{2.694218in}}%
\pgfpathlineto{\pgfqpoint{4.346485in}{2.694218in}}%
\pgfpathlineto{\pgfqpoint{4.346485in}{2.697167in}}%
\pgfpathlineto{\pgfqpoint{4.351026in}{2.697167in}}%
\pgfpathlineto{\pgfqpoint{4.351026in}{2.694218in}}%
\pgfpathmoveto{\pgfqpoint{4.310157in}{2.717812in}}%
\pgfpathlineto{\pgfqpoint{4.310157in}{2.717812in}}%
\pgfpathlineto{\pgfqpoint{4.310157in}{2.720761in}}%
\pgfpathlineto{\pgfqpoint{4.314698in}{2.720761in}}%
\pgfpathlineto{\pgfqpoint{4.314698in}{2.717812in}}%
\pgfpathmoveto{\pgfqpoint{4.523588in}{2.573302in}}%
\pgfpathlineto{\pgfqpoint{4.523588in}{2.573302in}}%
\pgfpathlineto{\pgfqpoint{4.523588in}{2.576251in}}%
\pgfpathlineto{\pgfqpoint{4.528129in}{2.576251in}}%
\pgfpathlineto{\pgfqpoint{4.528129in}{2.573302in}}%
\pgfpathmoveto{\pgfqpoint{4.450929in}{2.620489in}}%
\pgfpathlineto{\pgfqpoint{4.450929in}{2.620489in}}%
\pgfpathlineto{\pgfqpoint{4.450929in}{2.623438in}}%
\pgfpathlineto{\pgfqpoint{4.455471in}{2.623438in}}%
\pgfpathlineto{\pgfqpoint{4.455471in}{2.620489in}}%
\pgfpathmoveto{\pgfqpoint{4.414600in}{2.644082in}}%
\pgfpathlineto{\pgfqpoint{4.414600in}{2.644082in}}%
\pgfpathlineto{\pgfqpoint{4.414600in}{2.647031in}}%
\pgfpathlineto{\pgfqpoint{4.419141in}{2.647031in}}%
\pgfpathlineto{\pgfqpoint{4.419141in}{2.644082in}}%
\pgfpathmoveto{\pgfqpoint{4.382812in}{2.667675in}}%
\pgfpathlineto{\pgfqpoint{4.382812in}{2.667675in}}%
\pgfpathlineto{\pgfqpoint{4.382812in}{2.670624in}}%
\pgfpathlineto{\pgfqpoint{4.387353in}{2.670624in}}%
\pgfpathlineto{\pgfqpoint{4.387353in}{2.667675in}}%
\pgfpathmoveto{\pgfqpoint{4.387353in}{2.664726in}}%
\pgfpathlineto{\pgfqpoint{4.387353in}{2.664726in}}%
\pgfpathlineto{\pgfqpoint{4.387353in}{2.667675in}}%
\pgfpathlineto{\pgfqpoint{4.391895in}{2.667675in}}%
\pgfpathlineto{\pgfqpoint{4.391895in}{2.664726in}}%
\pgfpathmoveto{\pgfqpoint{4.387353in}{2.667675in}}%
\pgfpathlineto{\pgfqpoint{4.387353in}{2.667675in}}%
\pgfpathlineto{\pgfqpoint{4.387353in}{2.670624in}}%
\pgfpathlineto{\pgfqpoint{4.391895in}{2.670624in}}%
\pgfpathlineto{\pgfqpoint{4.391895in}{2.667675in}}%
\pgfpathmoveto{\pgfqpoint{4.391895in}{2.661777in}}%
\pgfpathlineto{\pgfqpoint{4.391895in}{2.661777in}}%
\pgfpathlineto{\pgfqpoint{4.391895in}{2.664726in}}%
\pgfpathlineto{\pgfqpoint{4.396436in}{2.664726in}}%
\pgfpathlineto{\pgfqpoint{4.396436in}{2.661777in}}%
\pgfpathmoveto{\pgfqpoint{4.396436in}{2.658828in}}%
\pgfpathlineto{\pgfqpoint{4.396436in}{2.658828in}}%
\pgfpathlineto{\pgfqpoint{4.396436in}{2.661777in}}%
\pgfpathlineto{\pgfqpoint{4.400977in}{2.661777in}}%
\pgfpathlineto{\pgfqpoint{4.400977in}{2.658828in}}%
\pgfpathmoveto{\pgfqpoint{4.396436in}{2.661777in}}%
\pgfpathlineto{\pgfqpoint{4.396436in}{2.661777in}}%
\pgfpathlineto{\pgfqpoint{4.396436in}{2.664726in}}%
\pgfpathlineto{\pgfqpoint{4.400977in}{2.664726in}}%
\pgfpathlineto{\pgfqpoint{4.400977in}{2.661777in}}%
\pgfpathmoveto{\pgfqpoint{4.391895in}{2.664726in}}%
\pgfpathlineto{\pgfqpoint{4.391895in}{2.664726in}}%
\pgfpathlineto{\pgfqpoint{4.391895in}{2.667675in}}%
\pgfpathlineto{\pgfqpoint{4.396436in}{2.667675in}}%
\pgfpathlineto{\pgfqpoint{4.396436in}{2.664726in}}%
\pgfpathmoveto{\pgfqpoint{4.405518in}{2.649980in}}%
\pgfpathlineto{\pgfqpoint{4.405518in}{2.649980in}}%
\pgfpathlineto{\pgfqpoint{4.405518in}{2.652929in}}%
\pgfpathlineto{\pgfqpoint{4.410059in}{2.652929in}}%
\pgfpathlineto{\pgfqpoint{4.410059in}{2.649980in}}%
\pgfpathmoveto{\pgfqpoint{4.400977in}{2.655879in}}%
\pgfpathlineto{\pgfqpoint{4.400977in}{2.655879in}}%
\pgfpathlineto{\pgfqpoint{4.400977in}{2.658828in}}%
\pgfpathlineto{\pgfqpoint{4.405518in}{2.658828in}}%
\pgfpathlineto{\pgfqpoint{4.405518in}{2.655879in}}%
\pgfpathmoveto{\pgfqpoint{4.405518in}{2.652929in}}%
\pgfpathlineto{\pgfqpoint{4.405518in}{2.652929in}}%
\pgfpathlineto{\pgfqpoint{4.405518in}{2.655879in}}%
\pgfpathlineto{\pgfqpoint{4.410059in}{2.655879in}}%
\pgfpathlineto{\pgfqpoint{4.410059in}{2.652929in}}%
\pgfpathmoveto{\pgfqpoint{4.405518in}{2.655879in}}%
\pgfpathlineto{\pgfqpoint{4.405518in}{2.655879in}}%
\pgfpathlineto{\pgfqpoint{4.405518in}{2.658828in}}%
\pgfpathlineto{\pgfqpoint{4.410059in}{2.658828in}}%
\pgfpathlineto{\pgfqpoint{4.410059in}{2.655879in}}%
\pgfpathmoveto{\pgfqpoint{4.410059in}{2.647031in}}%
\pgfpathlineto{\pgfqpoint{4.410059in}{2.647031in}}%
\pgfpathlineto{\pgfqpoint{4.410059in}{2.649980in}}%
\pgfpathlineto{\pgfqpoint{4.414600in}{2.649980in}}%
\pgfpathlineto{\pgfqpoint{4.414600in}{2.647031in}}%
\pgfpathmoveto{\pgfqpoint{4.410059in}{2.649980in}}%
\pgfpathlineto{\pgfqpoint{4.410059in}{2.649980in}}%
\pgfpathlineto{\pgfqpoint{4.410059in}{2.652929in}}%
\pgfpathlineto{\pgfqpoint{4.414600in}{2.652929in}}%
\pgfpathlineto{\pgfqpoint{4.414600in}{2.649980in}}%
\pgfpathmoveto{\pgfqpoint{4.414600in}{2.647031in}}%
\pgfpathlineto{\pgfqpoint{4.414600in}{2.647031in}}%
\pgfpathlineto{\pgfqpoint{4.414600in}{2.649980in}}%
\pgfpathlineto{\pgfqpoint{4.419141in}{2.649980in}}%
\pgfpathlineto{\pgfqpoint{4.419141in}{2.647031in}}%
\pgfpathmoveto{\pgfqpoint{4.400977in}{2.658828in}}%
\pgfpathlineto{\pgfqpoint{4.400977in}{2.658828in}}%
\pgfpathlineto{\pgfqpoint{4.400977in}{2.661777in}}%
\pgfpathlineto{\pgfqpoint{4.405518in}{2.661777in}}%
\pgfpathlineto{\pgfqpoint{4.405518in}{2.658828in}}%
\pgfpathmoveto{\pgfqpoint{4.432765in}{2.632285in}}%
\pgfpathlineto{\pgfqpoint{4.432765in}{2.632285in}}%
\pgfpathlineto{\pgfqpoint{4.432765in}{2.635234in}}%
\pgfpathlineto{\pgfqpoint{4.437306in}{2.635234in}}%
\pgfpathlineto{\pgfqpoint{4.437306in}{2.632285in}}%
\pgfpathmoveto{\pgfqpoint{4.423683in}{2.638184in}}%
\pgfpathlineto{\pgfqpoint{4.423683in}{2.638184in}}%
\pgfpathlineto{\pgfqpoint{4.423683in}{2.641133in}}%
\pgfpathlineto{\pgfqpoint{4.428224in}{2.641133in}}%
\pgfpathlineto{\pgfqpoint{4.428224in}{2.638184in}}%
\pgfpathmoveto{\pgfqpoint{4.419141in}{2.641133in}}%
\pgfpathlineto{\pgfqpoint{4.419141in}{2.641133in}}%
\pgfpathlineto{\pgfqpoint{4.419141in}{2.644082in}}%
\pgfpathlineto{\pgfqpoint{4.423683in}{2.644082in}}%
\pgfpathlineto{\pgfqpoint{4.423683in}{2.641133in}}%
\pgfpathmoveto{\pgfqpoint{4.419141in}{2.644082in}}%
\pgfpathlineto{\pgfqpoint{4.419141in}{2.644082in}}%
\pgfpathlineto{\pgfqpoint{4.419141in}{2.647031in}}%
\pgfpathlineto{\pgfqpoint{4.423683in}{2.647031in}}%
\pgfpathlineto{\pgfqpoint{4.423683in}{2.644082in}}%
\pgfpathmoveto{\pgfqpoint{4.423683in}{2.641133in}}%
\pgfpathlineto{\pgfqpoint{4.423683in}{2.641133in}}%
\pgfpathlineto{\pgfqpoint{4.423683in}{2.644082in}}%
\pgfpathlineto{\pgfqpoint{4.428224in}{2.644082in}}%
\pgfpathlineto{\pgfqpoint{4.428224in}{2.641133in}}%
\pgfpathmoveto{\pgfqpoint{4.428224in}{2.635234in}}%
\pgfpathlineto{\pgfqpoint{4.428224in}{2.635234in}}%
\pgfpathlineto{\pgfqpoint{4.428224in}{2.638184in}}%
\pgfpathlineto{\pgfqpoint{4.432765in}{2.638184in}}%
\pgfpathlineto{\pgfqpoint{4.432765in}{2.635234in}}%
\pgfpathmoveto{\pgfqpoint{4.428224in}{2.638184in}}%
\pgfpathlineto{\pgfqpoint{4.428224in}{2.638184in}}%
\pgfpathlineto{\pgfqpoint{4.428224in}{2.641133in}}%
\pgfpathlineto{\pgfqpoint{4.432765in}{2.641133in}}%
\pgfpathlineto{\pgfqpoint{4.432765in}{2.638184in}}%
\pgfpathmoveto{\pgfqpoint{4.432765in}{2.635234in}}%
\pgfpathlineto{\pgfqpoint{4.432765in}{2.635234in}}%
\pgfpathlineto{\pgfqpoint{4.432765in}{2.638184in}}%
\pgfpathlineto{\pgfqpoint{4.437306in}{2.638184in}}%
\pgfpathlineto{\pgfqpoint{4.437306in}{2.635234in}}%
\pgfpathmoveto{\pgfqpoint{4.441847in}{2.626387in}}%
\pgfpathlineto{\pgfqpoint{4.441847in}{2.626387in}}%
\pgfpathlineto{\pgfqpoint{4.441847in}{2.629336in}}%
\pgfpathlineto{\pgfqpoint{4.446388in}{2.629336in}}%
\pgfpathlineto{\pgfqpoint{4.446388in}{2.626387in}}%
\pgfpathmoveto{\pgfqpoint{4.437306in}{2.629336in}}%
\pgfpathlineto{\pgfqpoint{4.437306in}{2.629336in}}%
\pgfpathlineto{\pgfqpoint{4.437306in}{2.632285in}}%
\pgfpathlineto{\pgfqpoint{4.441847in}{2.632285in}}%
\pgfpathlineto{\pgfqpoint{4.441847in}{2.629336in}}%
\pgfpathmoveto{\pgfqpoint{4.437306in}{2.632285in}}%
\pgfpathlineto{\pgfqpoint{4.437306in}{2.632285in}}%
\pgfpathlineto{\pgfqpoint{4.437306in}{2.635234in}}%
\pgfpathlineto{\pgfqpoint{4.441847in}{2.635234in}}%
\pgfpathlineto{\pgfqpoint{4.441847in}{2.632285in}}%
\pgfpathmoveto{\pgfqpoint{4.441847in}{2.629336in}}%
\pgfpathlineto{\pgfqpoint{4.441847in}{2.629336in}}%
\pgfpathlineto{\pgfqpoint{4.441847in}{2.632285in}}%
\pgfpathlineto{\pgfqpoint{4.446388in}{2.632285in}}%
\pgfpathlineto{\pgfqpoint{4.446388in}{2.629336in}}%
\pgfpathmoveto{\pgfqpoint{4.446388in}{2.623438in}}%
\pgfpathlineto{\pgfqpoint{4.446388in}{2.623438in}}%
\pgfpathlineto{\pgfqpoint{4.446388in}{2.626387in}}%
\pgfpathlineto{\pgfqpoint{4.450929in}{2.626387in}}%
\pgfpathlineto{\pgfqpoint{4.450929in}{2.623438in}}%
\pgfpathmoveto{\pgfqpoint{4.446388in}{2.626387in}}%
\pgfpathlineto{\pgfqpoint{4.446388in}{2.626387in}}%
\pgfpathlineto{\pgfqpoint{4.446388in}{2.629336in}}%
\pgfpathlineto{\pgfqpoint{4.450929in}{2.629336in}}%
\pgfpathlineto{\pgfqpoint{4.450929in}{2.626387in}}%
\pgfpathmoveto{\pgfqpoint{4.450929in}{2.623438in}}%
\pgfpathlineto{\pgfqpoint{4.450929in}{2.623438in}}%
\pgfpathlineto{\pgfqpoint{4.450929in}{2.626387in}}%
\pgfpathlineto{\pgfqpoint{4.455471in}{2.626387in}}%
\pgfpathlineto{\pgfqpoint{4.455471in}{2.623438in}}%
\pgfpathmoveto{\pgfqpoint{4.487259in}{2.596896in}}%
\pgfpathlineto{\pgfqpoint{4.487259in}{2.596896in}}%
\pgfpathlineto{\pgfqpoint{4.487259in}{2.599845in}}%
\pgfpathlineto{\pgfqpoint{4.491800in}{2.599845in}}%
\pgfpathlineto{\pgfqpoint{4.491800in}{2.596896in}}%
\pgfpathmoveto{\pgfqpoint{4.469094in}{2.608692in}}%
\pgfpathlineto{\pgfqpoint{4.469094in}{2.608692in}}%
\pgfpathlineto{\pgfqpoint{4.469094in}{2.611641in}}%
\pgfpathlineto{\pgfqpoint{4.473635in}{2.611641in}}%
\pgfpathlineto{\pgfqpoint{4.473635in}{2.608692in}}%
\pgfpathmoveto{\pgfqpoint{4.460012in}{2.614590in}}%
\pgfpathlineto{\pgfqpoint{4.460012in}{2.614590in}}%
\pgfpathlineto{\pgfqpoint{4.460012in}{2.617540in}}%
\pgfpathlineto{\pgfqpoint{4.464553in}{2.617540in}}%
\pgfpathlineto{\pgfqpoint{4.464553in}{2.614590in}}%
\pgfpathmoveto{\pgfqpoint{4.455471in}{2.617540in}}%
\pgfpathlineto{\pgfqpoint{4.455471in}{2.617540in}}%
\pgfpathlineto{\pgfqpoint{4.455471in}{2.620489in}}%
\pgfpathlineto{\pgfqpoint{4.460012in}{2.620489in}}%
\pgfpathlineto{\pgfqpoint{4.460012in}{2.617540in}}%
\pgfpathmoveto{\pgfqpoint{4.455471in}{2.620489in}}%
\pgfpathlineto{\pgfqpoint{4.455471in}{2.620489in}}%
\pgfpathlineto{\pgfqpoint{4.455471in}{2.623438in}}%
\pgfpathlineto{\pgfqpoint{4.460012in}{2.623438in}}%
\pgfpathlineto{\pgfqpoint{4.460012in}{2.620489in}}%
\pgfpathmoveto{\pgfqpoint{4.460012in}{2.617540in}}%
\pgfpathlineto{\pgfqpoint{4.460012in}{2.617540in}}%
\pgfpathlineto{\pgfqpoint{4.460012in}{2.620489in}}%
\pgfpathlineto{\pgfqpoint{4.464553in}{2.620489in}}%
\pgfpathlineto{\pgfqpoint{4.464553in}{2.617540in}}%
\pgfpathmoveto{\pgfqpoint{4.464553in}{2.611641in}}%
\pgfpathlineto{\pgfqpoint{4.464553in}{2.611641in}}%
\pgfpathlineto{\pgfqpoint{4.464553in}{2.614590in}}%
\pgfpathlineto{\pgfqpoint{4.469094in}{2.614590in}}%
\pgfpathlineto{\pgfqpoint{4.469094in}{2.611641in}}%
\pgfpathmoveto{\pgfqpoint{4.464553in}{2.614590in}}%
\pgfpathlineto{\pgfqpoint{4.464553in}{2.614590in}}%
\pgfpathlineto{\pgfqpoint{4.464553in}{2.617540in}}%
\pgfpathlineto{\pgfqpoint{4.469094in}{2.617540in}}%
\pgfpathlineto{\pgfqpoint{4.469094in}{2.614590in}}%
\pgfpathmoveto{\pgfqpoint{4.469094in}{2.611641in}}%
\pgfpathlineto{\pgfqpoint{4.469094in}{2.611641in}}%
\pgfpathlineto{\pgfqpoint{4.469094in}{2.614590in}}%
\pgfpathlineto{\pgfqpoint{4.473635in}{2.614590in}}%
\pgfpathlineto{\pgfqpoint{4.473635in}{2.611641in}}%
\pgfpathmoveto{\pgfqpoint{4.478176in}{2.602794in}}%
\pgfpathlineto{\pgfqpoint{4.478176in}{2.602794in}}%
\pgfpathlineto{\pgfqpoint{4.478176in}{2.605743in}}%
\pgfpathlineto{\pgfqpoint{4.482717in}{2.605743in}}%
\pgfpathlineto{\pgfqpoint{4.482717in}{2.602794in}}%
\pgfpathmoveto{\pgfqpoint{4.473635in}{2.605743in}}%
\pgfpathlineto{\pgfqpoint{4.473635in}{2.605743in}}%
\pgfpathlineto{\pgfqpoint{4.473635in}{2.608692in}}%
\pgfpathlineto{\pgfqpoint{4.478176in}{2.608692in}}%
\pgfpathlineto{\pgfqpoint{4.478176in}{2.605743in}}%
\pgfpathmoveto{\pgfqpoint{4.473635in}{2.608692in}}%
\pgfpathlineto{\pgfqpoint{4.473635in}{2.608692in}}%
\pgfpathlineto{\pgfqpoint{4.473635in}{2.611641in}}%
\pgfpathlineto{\pgfqpoint{4.478176in}{2.611641in}}%
\pgfpathlineto{\pgfqpoint{4.478176in}{2.608692in}}%
\pgfpathmoveto{\pgfqpoint{4.478176in}{2.605743in}}%
\pgfpathlineto{\pgfqpoint{4.478176in}{2.605743in}}%
\pgfpathlineto{\pgfqpoint{4.478176in}{2.608692in}}%
\pgfpathlineto{\pgfqpoint{4.482717in}{2.608692in}}%
\pgfpathlineto{\pgfqpoint{4.482717in}{2.605743in}}%
\pgfpathmoveto{\pgfqpoint{4.482717in}{2.599845in}}%
\pgfpathlineto{\pgfqpoint{4.482717in}{2.599845in}}%
\pgfpathlineto{\pgfqpoint{4.482717in}{2.602794in}}%
\pgfpathlineto{\pgfqpoint{4.487259in}{2.602794in}}%
\pgfpathlineto{\pgfqpoint{4.487259in}{2.599845in}}%
\pgfpathmoveto{\pgfqpoint{4.482717in}{2.602794in}}%
\pgfpathlineto{\pgfqpoint{4.482717in}{2.602794in}}%
\pgfpathlineto{\pgfqpoint{4.482717in}{2.605743in}}%
\pgfpathlineto{\pgfqpoint{4.487259in}{2.605743in}}%
\pgfpathlineto{\pgfqpoint{4.487259in}{2.602794in}}%
\pgfpathmoveto{\pgfqpoint{4.487259in}{2.599845in}}%
\pgfpathlineto{\pgfqpoint{4.487259in}{2.599845in}}%
\pgfpathlineto{\pgfqpoint{4.487259in}{2.602794in}}%
\pgfpathlineto{\pgfqpoint{4.491800in}{2.602794in}}%
\pgfpathlineto{\pgfqpoint{4.491800in}{2.599845in}}%
\pgfpathmoveto{\pgfqpoint{4.505423in}{2.585099in}}%
\pgfpathlineto{\pgfqpoint{4.505423in}{2.585099in}}%
\pgfpathlineto{\pgfqpoint{4.505423in}{2.588048in}}%
\pgfpathlineto{\pgfqpoint{4.509964in}{2.588048in}}%
\pgfpathlineto{\pgfqpoint{4.509964in}{2.585099in}}%
\pgfpathmoveto{\pgfqpoint{4.496341in}{2.590997in}}%
\pgfpathlineto{\pgfqpoint{4.496341in}{2.590997in}}%
\pgfpathlineto{\pgfqpoint{4.496341in}{2.593946in}}%
\pgfpathlineto{\pgfqpoint{4.500882in}{2.593946in}}%
\pgfpathlineto{\pgfqpoint{4.500882in}{2.590997in}}%
\pgfpathmoveto{\pgfqpoint{4.491800in}{2.593946in}}%
\pgfpathlineto{\pgfqpoint{4.491800in}{2.593946in}}%
\pgfpathlineto{\pgfqpoint{4.491800in}{2.596896in}}%
\pgfpathlineto{\pgfqpoint{4.496341in}{2.596896in}}%
\pgfpathlineto{\pgfqpoint{4.496341in}{2.593946in}}%
\pgfpathmoveto{\pgfqpoint{4.491800in}{2.596896in}}%
\pgfpathlineto{\pgfqpoint{4.491800in}{2.596896in}}%
\pgfpathlineto{\pgfqpoint{4.491800in}{2.599845in}}%
\pgfpathlineto{\pgfqpoint{4.496341in}{2.599845in}}%
\pgfpathlineto{\pgfqpoint{4.496341in}{2.596896in}}%
\pgfpathmoveto{\pgfqpoint{4.496341in}{2.593946in}}%
\pgfpathlineto{\pgfqpoint{4.496341in}{2.593946in}}%
\pgfpathlineto{\pgfqpoint{4.496341in}{2.596896in}}%
\pgfpathlineto{\pgfqpoint{4.500882in}{2.596896in}}%
\pgfpathlineto{\pgfqpoint{4.500882in}{2.593946in}}%
\pgfpathmoveto{\pgfqpoint{4.500882in}{2.588048in}}%
\pgfpathlineto{\pgfqpoint{4.500882in}{2.588048in}}%
\pgfpathlineto{\pgfqpoint{4.500882in}{2.590997in}}%
\pgfpathlineto{\pgfqpoint{4.505423in}{2.590997in}}%
\pgfpathlineto{\pgfqpoint{4.505423in}{2.588048in}}%
\pgfpathmoveto{\pgfqpoint{4.500882in}{2.590997in}}%
\pgfpathlineto{\pgfqpoint{4.500882in}{2.590997in}}%
\pgfpathlineto{\pgfqpoint{4.500882in}{2.593946in}}%
\pgfpathlineto{\pgfqpoint{4.505423in}{2.593946in}}%
\pgfpathlineto{\pgfqpoint{4.505423in}{2.590997in}}%
\pgfpathmoveto{\pgfqpoint{4.505423in}{2.588048in}}%
\pgfpathlineto{\pgfqpoint{4.505423in}{2.588048in}}%
\pgfpathlineto{\pgfqpoint{4.505423in}{2.590997in}}%
\pgfpathlineto{\pgfqpoint{4.509964in}{2.590997in}}%
\pgfpathlineto{\pgfqpoint{4.509964in}{2.588048in}}%
\pgfpathmoveto{\pgfqpoint{4.514505in}{2.579201in}}%
\pgfpathlineto{\pgfqpoint{4.514505in}{2.579201in}}%
\pgfpathlineto{\pgfqpoint{4.514505in}{2.582150in}}%
\pgfpathlineto{\pgfqpoint{4.519047in}{2.582150in}}%
\pgfpathlineto{\pgfqpoint{4.519047in}{2.579201in}}%
\pgfpathmoveto{\pgfqpoint{4.509964in}{2.582150in}}%
\pgfpathlineto{\pgfqpoint{4.509964in}{2.582150in}}%
\pgfpathlineto{\pgfqpoint{4.509964in}{2.585099in}}%
\pgfpathlineto{\pgfqpoint{4.514505in}{2.585099in}}%
\pgfpathlineto{\pgfqpoint{4.514505in}{2.582150in}}%
\pgfpathmoveto{\pgfqpoint{4.509964in}{2.585099in}}%
\pgfpathlineto{\pgfqpoint{4.509964in}{2.585099in}}%
\pgfpathlineto{\pgfqpoint{4.509964in}{2.588048in}}%
\pgfpathlineto{\pgfqpoint{4.514505in}{2.588048in}}%
\pgfpathlineto{\pgfqpoint{4.514505in}{2.585099in}}%
\pgfpathmoveto{\pgfqpoint{4.514505in}{2.582150in}}%
\pgfpathlineto{\pgfqpoint{4.514505in}{2.582150in}}%
\pgfpathlineto{\pgfqpoint{4.514505in}{2.585099in}}%
\pgfpathlineto{\pgfqpoint{4.519047in}{2.585099in}}%
\pgfpathlineto{\pgfqpoint{4.519047in}{2.582150in}}%
\pgfpathmoveto{\pgfqpoint{4.519047in}{2.576251in}}%
\pgfpathlineto{\pgfqpoint{4.519047in}{2.576251in}}%
\pgfpathlineto{\pgfqpoint{4.519047in}{2.579201in}}%
\pgfpathlineto{\pgfqpoint{4.523588in}{2.579201in}}%
\pgfpathlineto{\pgfqpoint{4.523588in}{2.576251in}}%
\pgfpathmoveto{\pgfqpoint{4.519047in}{2.579201in}}%
\pgfpathlineto{\pgfqpoint{4.519047in}{2.579201in}}%
\pgfpathlineto{\pgfqpoint{4.519047in}{2.582150in}}%
\pgfpathlineto{\pgfqpoint{4.523588in}{2.582150in}}%
\pgfpathlineto{\pgfqpoint{4.523588in}{2.579201in}}%
\pgfpathmoveto{\pgfqpoint{4.523588in}{2.576251in}}%
\pgfpathlineto{\pgfqpoint{4.523588in}{2.576251in}}%
\pgfpathlineto{\pgfqpoint{4.523588in}{2.579201in}}%
\pgfpathlineto{\pgfqpoint{4.528129in}{2.579201in}}%
\pgfpathlineto{\pgfqpoint{4.528129in}{2.576251in}}%
\pgfpathmoveto{\pgfqpoint{4.382812in}{2.670624in}}%
\pgfpathlineto{\pgfqpoint{4.382812in}{2.670624in}}%
\pgfpathlineto{\pgfqpoint{4.382812in}{2.673573in}}%
\pgfpathlineto{\pgfqpoint{4.387353in}{2.673573in}}%
\pgfpathlineto{\pgfqpoint{4.387353in}{2.670624in}}%
\pgfpathmoveto{\pgfqpoint{4.596241in}{2.526113in}}%
\pgfpathlineto{\pgfqpoint{4.596241in}{2.526113in}}%
\pgfpathlineto{\pgfqpoint{4.596241in}{2.529063in}}%
\pgfpathlineto{\pgfqpoint{4.600782in}{2.529063in}}%
\pgfpathlineto{\pgfqpoint{4.600782in}{2.526113in}}%
\pgfpathmoveto{\pgfqpoint{4.559914in}{2.549708in}}%
\pgfpathlineto{\pgfqpoint{4.559914in}{2.549708in}}%
\pgfpathlineto{\pgfqpoint{4.559914in}{2.552657in}}%
\pgfpathlineto{\pgfqpoint{4.564455in}{2.552657in}}%
\pgfpathlineto{\pgfqpoint{4.564455in}{2.549708in}}%
\pgfpathmoveto{\pgfqpoint{4.541751in}{2.561505in}}%
\pgfpathlineto{\pgfqpoint{4.541751in}{2.561505in}}%
\pgfpathlineto{\pgfqpoint{4.541751in}{2.564454in}}%
\pgfpathlineto{\pgfqpoint{4.546292in}{2.564454in}}%
\pgfpathlineto{\pgfqpoint{4.546292in}{2.561505in}}%
\pgfpathmoveto{\pgfqpoint{4.532670in}{2.567404in}}%
\pgfpathlineto{\pgfqpoint{4.532670in}{2.567404in}}%
\pgfpathlineto{\pgfqpoint{4.532670in}{2.570353in}}%
\pgfpathlineto{\pgfqpoint{4.537210in}{2.570353in}}%
\pgfpathlineto{\pgfqpoint{4.537210in}{2.567404in}}%
\pgfpathmoveto{\pgfqpoint{4.528129in}{2.570353in}}%
\pgfpathlineto{\pgfqpoint{4.528129in}{2.570353in}}%
\pgfpathlineto{\pgfqpoint{4.528129in}{2.573302in}}%
\pgfpathlineto{\pgfqpoint{4.532670in}{2.573302in}}%
\pgfpathlineto{\pgfqpoint{4.532670in}{2.570353in}}%
\pgfpathmoveto{\pgfqpoint{4.528129in}{2.573302in}}%
\pgfpathlineto{\pgfqpoint{4.528129in}{2.573302in}}%
\pgfpathlineto{\pgfqpoint{4.528129in}{2.576251in}}%
\pgfpathlineto{\pgfqpoint{4.532670in}{2.576251in}}%
\pgfpathlineto{\pgfqpoint{4.532670in}{2.573302in}}%
\pgfpathmoveto{\pgfqpoint{4.532670in}{2.570353in}}%
\pgfpathlineto{\pgfqpoint{4.532670in}{2.570353in}}%
\pgfpathlineto{\pgfqpoint{4.532670in}{2.573302in}}%
\pgfpathlineto{\pgfqpoint{4.537210in}{2.573302in}}%
\pgfpathlineto{\pgfqpoint{4.537210in}{2.570353in}}%
\pgfpathmoveto{\pgfqpoint{4.537210in}{2.564454in}}%
\pgfpathlineto{\pgfqpoint{4.537210in}{2.564454in}}%
\pgfpathlineto{\pgfqpoint{4.537210in}{2.567404in}}%
\pgfpathlineto{\pgfqpoint{4.541751in}{2.567404in}}%
\pgfpathlineto{\pgfqpoint{4.541751in}{2.564454in}}%
\pgfpathmoveto{\pgfqpoint{4.537210in}{2.567404in}}%
\pgfpathlineto{\pgfqpoint{4.537210in}{2.567404in}}%
\pgfpathlineto{\pgfqpoint{4.537210in}{2.570353in}}%
\pgfpathlineto{\pgfqpoint{4.541751in}{2.570353in}}%
\pgfpathlineto{\pgfqpoint{4.541751in}{2.567404in}}%
\pgfpathmoveto{\pgfqpoint{4.541751in}{2.564454in}}%
\pgfpathlineto{\pgfqpoint{4.541751in}{2.564454in}}%
\pgfpathlineto{\pgfqpoint{4.541751in}{2.567404in}}%
\pgfpathlineto{\pgfqpoint{4.546292in}{2.567404in}}%
\pgfpathlineto{\pgfqpoint{4.546292in}{2.564454in}}%
\pgfpathmoveto{\pgfqpoint{4.550833in}{2.555606in}}%
\pgfpathlineto{\pgfqpoint{4.550833in}{2.555606in}}%
\pgfpathlineto{\pgfqpoint{4.550833in}{2.558556in}}%
\pgfpathlineto{\pgfqpoint{4.555374in}{2.558556in}}%
\pgfpathlineto{\pgfqpoint{4.555374in}{2.555606in}}%
\pgfpathmoveto{\pgfqpoint{4.546292in}{2.558556in}}%
\pgfpathlineto{\pgfqpoint{4.546292in}{2.558556in}}%
\pgfpathlineto{\pgfqpoint{4.546292in}{2.561505in}}%
\pgfpathlineto{\pgfqpoint{4.550833in}{2.561505in}}%
\pgfpathlineto{\pgfqpoint{4.550833in}{2.558556in}}%
\pgfpathmoveto{\pgfqpoint{4.546292in}{2.561505in}}%
\pgfpathlineto{\pgfqpoint{4.546292in}{2.561505in}}%
\pgfpathlineto{\pgfqpoint{4.546292in}{2.564454in}}%
\pgfpathlineto{\pgfqpoint{4.550833in}{2.564454in}}%
\pgfpathlineto{\pgfqpoint{4.550833in}{2.561505in}}%
\pgfpathmoveto{\pgfqpoint{4.550833in}{2.558556in}}%
\pgfpathlineto{\pgfqpoint{4.550833in}{2.558556in}}%
\pgfpathlineto{\pgfqpoint{4.550833in}{2.561505in}}%
\pgfpathlineto{\pgfqpoint{4.555374in}{2.561505in}}%
\pgfpathlineto{\pgfqpoint{4.555374in}{2.558556in}}%
\pgfpathmoveto{\pgfqpoint{4.555374in}{2.552657in}}%
\pgfpathlineto{\pgfqpoint{4.555374in}{2.552657in}}%
\pgfpathlineto{\pgfqpoint{4.555374in}{2.555606in}}%
\pgfpathlineto{\pgfqpoint{4.559914in}{2.555606in}}%
\pgfpathlineto{\pgfqpoint{4.559914in}{2.552657in}}%
\pgfpathmoveto{\pgfqpoint{4.555374in}{2.555606in}}%
\pgfpathlineto{\pgfqpoint{4.555374in}{2.555606in}}%
\pgfpathlineto{\pgfqpoint{4.555374in}{2.558556in}}%
\pgfpathlineto{\pgfqpoint{4.559914in}{2.558556in}}%
\pgfpathlineto{\pgfqpoint{4.559914in}{2.555606in}}%
\pgfpathmoveto{\pgfqpoint{4.559914in}{2.552657in}}%
\pgfpathlineto{\pgfqpoint{4.559914in}{2.552657in}}%
\pgfpathlineto{\pgfqpoint{4.559914in}{2.555606in}}%
\pgfpathlineto{\pgfqpoint{4.564455in}{2.555606in}}%
\pgfpathlineto{\pgfqpoint{4.564455in}{2.552657in}}%
\pgfpathmoveto{\pgfqpoint{4.578078in}{2.537911in}}%
\pgfpathlineto{\pgfqpoint{4.578078in}{2.537911in}}%
\pgfpathlineto{\pgfqpoint{4.578078in}{2.540860in}}%
\pgfpathlineto{\pgfqpoint{4.582618in}{2.540860in}}%
\pgfpathlineto{\pgfqpoint{4.582618in}{2.537911in}}%
\pgfpathmoveto{\pgfqpoint{4.568996in}{2.543809in}}%
\pgfpathlineto{\pgfqpoint{4.568996in}{2.543809in}}%
\pgfpathlineto{\pgfqpoint{4.568996in}{2.546758in}}%
\pgfpathlineto{\pgfqpoint{4.573537in}{2.546758in}}%
\pgfpathlineto{\pgfqpoint{4.573537in}{2.543809in}}%
\pgfpathmoveto{\pgfqpoint{4.564455in}{2.546758in}}%
\pgfpathlineto{\pgfqpoint{4.564455in}{2.546758in}}%
\pgfpathlineto{\pgfqpoint{4.564455in}{2.549708in}}%
\pgfpathlineto{\pgfqpoint{4.568996in}{2.549708in}}%
\pgfpathlineto{\pgfqpoint{4.568996in}{2.546758in}}%
\pgfpathmoveto{\pgfqpoint{4.564455in}{2.549708in}}%
\pgfpathlineto{\pgfqpoint{4.564455in}{2.549708in}}%
\pgfpathlineto{\pgfqpoint{4.564455in}{2.552657in}}%
\pgfpathlineto{\pgfqpoint{4.568996in}{2.552657in}}%
\pgfpathlineto{\pgfqpoint{4.568996in}{2.549708in}}%
\pgfpathmoveto{\pgfqpoint{4.568996in}{2.546758in}}%
\pgfpathlineto{\pgfqpoint{4.568996in}{2.546758in}}%
\pgfpathlineto{\pgfqpoint{4.568996in}{2.549708in}}%
\pgfpathlineto{\pgfqpoint{4.573537in}{2.549708in}}%
\pgfpathlineto{\pgfqpoint{4.573537in}{2.546758in}}%
\pgfpathmoveto{\pgfqpoint{4.573537in}{2.540860in}}%
\pgfpathlineto{\pgfqpoint{4.573537in}{2.540860in}}%
\pgfpathlineto{\pgfqpoint{4.573537in}{2.543809in}}%
\pgfpathlineto{\pgfqpoint{4.578078in}{2.543809in}}%
\pgfpathlineto{\pgfqpoint{4.578078in}{2.540860in}}%
\pgfpathmoveto{\pgfqpoint{4.573537in}{2.543809in}}%
\pgfpathlineto{\pgfqpoint{4.573537in}{2.543809in}}%
\pgfpathlineto{\pgfqpoint{4.573537in}{2.546758in}}%
\pgfpathlineto{\pgfqpoint{4.578078in}{2.546758in}}%
\pgfpathlineto{\pgfqpoint{4.578078in}{2.543809in}}%
\pgfpathmoveto{\pgfqpoint{4.578078in}{2.540860in}}%
\pgfpathlineto{\pgfqpoint{4.578078in}{2.540860in}}%
\pgfpathlineto{\pgfqpoint{4.578078in}{2.543809in}}%
\pgfpathlineto{\pgfqpoint{4.582618in}{2.543809in}}%
\pgfpathlineto{\pgfqpoint{4.582618in}{2.540860in}}%
\pgfpathmoveto{\pgfqpoint{4.587159in}{2.532012in}}%
\pgfpathlineto{\pgfqpoint{4.587159in}{2.532012in}}%
\pgfpathlineto{\pgfqpoint{4.587159in}{2.534961in}}%
\pgfpathlineto{\pgfqpoint{4.591700in}{2.534961in}}%
\pgfpathlineto{\pgfqpoint{4.591700in}{2.532012in}}%
\pgfpathmoveto{\pgfqpoint{4.582618in}{2.534961in}}%
\pgfpathlineto{\pgfqpoint{4.582618in}{2.534961in}}%
\pgfpathlineto{\pgfqpoint{4.582618in}{2.537911in}}%
\pgfpathlineto{\pgfqpoint{4.587159in}{2.537911in}}%
\pgfpathlineto{\pgfqpoint{4.587159in}{2.534961in}}%
\pgfpathmoveto{\pgfqpoint{4.582618in}{2.537911in}}%
\pgfpathlineto{\pgfqpoint{4.582618in}{2.537911in}}%
\pgfpathlineto{\pgfqpoint{4.582618in}{2.540860in}}%
\pgfpathlineto{\pgfqpoint{4.587159in}{2.540860in}}%
\pgfpathlineto{\pgfqpoint{4.587159in}{2.537911in}}%
\pgfpathmoveto{\pgfqpoint{4.587159in}{2.534961in}}%
\pgfpathlineto{\pgfqpoint{4.587159in}{2.534961in}}%
\pgfpathlineto{\pgfqpoint{4.587159in}{2.537911in}}%
\pgfpathlineto{\pgfqpoint{4.591700in}{2.537911in}}%
\pgfpathlineto{\pgfqpoint{4.591700in}{2.534961in}}%
\pgfpathmoveto{\pgfqpoint{4.591700in}{2.529063in}}%
\pgfpathlineto{\pgfqpoint{4.591700in}{2.529063in}}%
\pgfpathlineto{\pgfqpoint{4.591700in}{2.532012in}}%
\pgfpathlineto{\pgfqpoint{4.596241in}{2.532012in}}%
\pgfpathlineto{\pgfqpoint{4.596241in}{2.529063in}}%
\pgfpathmoveto{\pgfqpoint{4.591700in}{2.532012in}}%
\pgfpathlineto{\pgfqpoint{4.591700in}{2.532012in}}%
\pgfpathlineto{\pgfqpoint{4.591700in}{2.534961in}}%
\pgfpathlineto{\pgfqpoint{4.596241in}{2.534961in}}%
\pgfpathlineto{\pgfqpoint{4.596241in}{2.532012in}}%
\pgfpathmoveto{\pgfqpoint{4.596241in}{2.529063in}}%
\pgfpathlineto{\pgfqpoint{4.596241in}{2.529063in}}%
\pgfpathlineto{\pgfqpoint{4.596241in}{2.532012in}}%
\pgfpathlineto{\pgfqpoint{4.600782in}{2.532012in}}%
\pgfpathlineto{\pgfqpoint{4.600782in}{2.529063in}}%
\pgfpathmoveto{\pgfqpoint{4.600782in}{2.523164in}}%
\pgfpathlineto{\pgfqpoint{4.600782in}{2.523164in}}%
\pgfpathlineto{\pgfqpoint{4.600782in}{2.526113in}}%
\pgfpathlineto{\pgfqpoint{4.605322in}{2.526113in}}%
\pgfpathlineto{\pgfqpoint{4.605322in}{2.523164in}}%
\pgfpathmoveto{\pgfqpoint{4.600782in}{2.526113in}}%
\pgfpathlineto{\pgfqpoint{4.600782in}{2.526113in}}%
\pgfpathlineto{\pgfqpoint{4.600782in}{2.529063in}}%
\pgfpathlineto{\pgfqpoint{4.605322in}{2.529063in}}%
\pgfpathlineto{\pgfqpoint{4.605322in}{2.526113in}}%
\pgfpathmoveto{\pgfqpoint{4.605322in}{2.523164in}}%
\pgfpathlineto{\pgfqpoint{4.605322in}{2.523164in}}%
\pgfpathlineto{\pgfqpoint{4.605322in}{2.526113in}}%
\pgfpathlineto{\pgfqpoint{4.609863in}{2.526113in}}%
\pgfpathlineto{\pgfqpoint{4.609863in}{2.523164in}}%
\pgfpathmoveto{\pgfqpoint{4.609863in}{2.520215in}}%
\pgfpathlineto{\pgfqpoint{4.609863in}{2.520215in}}%
\pgfpathlineto{\pgfqpoint{4.609863in}{2.523164in}}%
\pgfpathlineto{\pgfqpoint{4.614404in}{2.523164in}}%
\pgfpathlineto{\pgfqpoint{4.614404in}{2.520215in}}%
\pgfpathmoveto{\pgfqpoint{4.614404in}{2.517265in}}%
\pgfpathlineto{\pgfqpoint{4.614404in}{2.517265in}}%
\pgfpathlineto{\pgfqpoint{4.614404in}{2.520215in}}%
\pgfpathlineto{\pgfqpoint{4.618945in}{2.520215in}}%
\pgfpathlineto{\pgfqpoint{4.618945in}{2.517265in}}%
\pgfpathmoveto{\pgfqpoint{4.614404in}{2.520215in}}%
\pgfpathlineto{\pgfqpoint{4.614404in}{2.520215in}}%
\pgfpathlineto{\pgfqpoint{4.614404in}{2.523164in}}%
\pgfpathlineto{\pgfqpoint{4.618945in}{2.523164in}}%
\pgfpathlineto{\pgfqpoint{4.618945in}{2.520215in}}%
\pgfpathmoveto{\pgfqpoint{4.609863in}{2.523164in}}%
\pgfpathlineto{\pgfqpoint{4.609863in}{2.523164in}}%
\pgfpathlineto{\pgfqpoint{4.609863in}{2.526113in}}%
\pgfpathlineto{\pgfqpoint{4.614404in}{2.526113in}}%
\pgfpathlineto{\pgfqpoint{4.614404in}{2.523164in}}%
\pgfpathmoveto{\pgfqpoint{4.618945in}{2.514316in}}%
\pgfpathlineto{\pgfqpoint{4.618945in}{2.514316in}}%
\pgfpathlineto{\pgfqpoint{4.618945in}{2.517265in}}%
\pgfpathlineto{\pgfqpoint{4.623486in}{2.517265in}}%
\pgfpathlineto{\pgfqpoint{4.623486in}{2.514316in}}%
\pgfpathmoveto{\pgfqpoint{4.623486in}{2.511367in}}%
\pgfpathlineto{\pgfqpoint{4.623486in}{2.511367in}}%
\pgfpathlineto{\pgfqpoint{4.623486in}{2.514316in}}%
\pgfpathlineto{\pgfqpoint{4.628026in}{2.514316in}}%
\pgfpathlineto{\pgfqpoint{4.628026in}{2.511367in}}%
\pgfpathmoveto{\pgfqpoint{4.623486in}{2.514316in}}%
\pgfpathlineto{\pgfqpoint{4.623486in}{2.514316in}}%
\pgfpathlineto{\pgfqpoint{4.623486in}{2.517265in}}%
\pgfpathlineto{\pgfqpoint{4.628026in}{2.517265in}}%
\pgfpathlineto{\pgfqpoint{4.628026in}{2.514316in}}%
\pgfpathmoveto{\pgfqpoint{4.628026in}{2.508418in}}%
\pgfpathlineto{\pgfqpoint{4.628026in}{2.508418in}}%
\pgfpathlineto{\pgfqpoint{4.628026in}{2.511367in}}%
\pgfpathlineto{\pgfqpoint{4.632567in}{2.511367in}}%
\pgfpathlineto{\pgfqpoint{4.632567in}{2.508418in}}%
\pgfpathmoveto{\pgfqpoint{4.632567in}{2.505468in}}%
\pgfpathlineto{\pgfqpoint{4.632567in}{2.505468in}}%
\pgfpathlineto{\pgfqpoint{4.632567in}{2.508418in}}%
\pgfpathlineto{\pgfqpoint{4.637108in}{2.508418in}}%
\pgfpathlineto{\pgfqpoint{4.637108in}{2.505468in}}%
\pgfpathmoveto{\pgfqpoint{4.632567in}{2.508418in}}%
\pgfpathlineto{\pgfqpoint{4.632567in}{2.508418in}}%
\pgfpathlineto{\pgfqpoint{4.632567in}{2.511367in}}%
\pgfpathlineto{\pgfqpoint{4.637108in}{2.511367in}}%
\pgfpathlineto{\pgfqpoint{4.637108in}{2.508418in}}%
\pgfpathmoveto{\pgfqpoint{4.628026in}{2.511367in}}%
\pgfpathlineto{\pgfqpoint{4.628026in}{2.511367in}}%
\pgfpathlineto{\pgfqpoint{4.628026in}{2.514316in}}%
\pgfpathlineto{\pgfqpoint{4.632567in}{2.514316in}}%
\pgfpathlineto{\pgfqpoint{4.632567in}{2.511367in}}%
\pgfpathmoveto{\pgfqpoint{4.618945in}{2.517265in}}%
\pgfpathlineto{\pgfqpoint{4.618945in}{2.517265in}}%
\pgfpathlineto{\pgfqpoint{4.618945in}{2.520215in}}%
\pgfpathlineto{\pgfqpoint{4.623486in}{2.520215in}}%
\pgfpathlineto{\pgfqpoint{4.623486in}{2.517265in}}%
\pgfpathmoveto{\pgfqpoint{4.637108in}{2.502519in}}%
\pgfpathlineto{\pgfqpoint{4.637108in}{2.502519in}}%
\pgfpathlineto{\pgfqpoint{4.637108in}{2.505468in}}%
\pgfpathlineto{\pgfqpoint{4.641649in}{2.505468in}}%
\pgfpathlineto{\pgfqpoint{4.641649in}{2.502519in}}%
\pgfpathmoveto{\pgfqpoint{4.641649in}{2.499570in}}%
\pgfpathlineto{\pgfqpoint{4.641649in}{2.499570in}}%
\pgfpathlineto{\pgfqpoint{4.641649in}{2.502519in}}%
\pgfpathlineto{\pgfqpoint{4.646190in}{2.502519in}}%
\pgfpathlineto{\pgfqpoint{4.646190in}{2.499570in}}%
\pgfpathmoveto{\pgfqpoint{4.641649in}{2.502519in}}%
\pgfpathlineto{\pgfqpoint{4.641649in}{2.502519in}}%
\pgfpathlineto{\pgfqpoint{4.641649in}{2.505468in}}%
\pgfpathlineto{\pgfqpoint{4.646190in}{2.505468in}}%
\pgfpathlineto{\pgfqpoint{4.646190in}{2.502519in}}%
\pgfpathmoveto{\pgfqpoint{4.646190in}{2.496620in}}%
\pgfpathlineto{\pgfqpoint{4.646190in}{2.496620in}}%
\pgfpathlineto{\pgfqpoint{4.646190in}{2.499570in}}%
\pgfpathlineto{\pgfqpoint{4.650730in}{2.499570in}}%
\pgfpathlineto{\pgfqpoint{4.650730in}{2.496620in}}%
\pgfpathmoveto{\pgfqpoint{4.650730in}{2.493671in}}%
\pgfpathlineto{\pgfqpoint{4.650730in}{2.493671in}}%
\pgfpathlineto{\pgfqpoint{4.650730in}{2.496620in}}%
\pgfpathlineto{\pgfqpoint{4.655271in}{2.496620in}}%
\pgfpathlineto{\pgfqpoint{4.655271in}{2.493671in}}%
\pgfpathmoveto{\pgfqpoint{4.650730in}{2.496620in}}%
\pgfpathlineto{\pgfqpoint{4.650730in}{2.496620in}}%
\pgfpathlineto{\pgfqpoint{4.650730in}{2.499570in}}%
\pgfpathlineto{\pgfqpoint{4.655271in}{2.499570in}}%
\pgfpathlineto{\pgfqpoint{4.655271in}{2.496620in}}%
\pgfpathmoveto{\pgfqpoint{4.646190in}{2.499570in}}%
\pgfpathlineto{\pgfqpoint{4.646190in}{2.499570in}}%
\pgfpathlineto{\pgfqpoint{4.646190in}{2.502519in}}%
\pgfpathlineto{\pgfqpoint{4.650730in}{2.502519in}}%
\pgfpathlineto{\pgfqpoint{4.650730in}{2.499570in}}%
\pgfpathmoveto{\pgfqpoint{4.655271in}{2.490722in}}%
\pgfpathlineto{\pgfqpoint{4.655271in}{2.490722in}}%
\pgfpathlineto{\pgfqpoint{4.655271in}{2.493671in}}%
\pgfpathlineto{\pgfqpoint{4.659812in}{2.493671in}}%
\pgfpathlineto{\pgfqpoint{4.659812in}{2.490722in}}%
\pgfpathmoveto{\pgfqpoint{4.659812in}{2.487772in}}%
\pgfpathlineto{\pgfqpoint{4.659812in}{2.487772in}}%
\pgfpathlineto{\pgfqpoint{4.659812in}{2.490722in}}%
\pgfpathlineto{\pgfqpoint{4.664353in}{2.490722in}}%
\pgfpathlineto{\pgfqpoint{4.664353in}{2.487772in}}%
\pgfpathmoveto{\pgfqpoint{4.659812in}{2.490722in}}%
\pgfpathlineto{\pgfqpoint{4.659812in}{2.490722in}}%
\pgfpathlineto{\pgfqpoint{4.659812in}{2.493671in}}%
\pgfpathlineto{\pgfqpoint{4.664353in}{2.493671in}}%
\pgfpathlineto{\pgfqpoint{4.664353in}{2.490722in}}%
\pgfpathmoveto{\pgfqpoint{4.664353in}{2.484823in}}%
\pgfpathlineto{\pgfqpoint{4.664353in}{2.484823in}}%
\pgfpathlineto{\pgfqpoint{4.664353in}{2.487772in}}%
\pgfpathlineto{\pgfqpoint{4.668894in}{2.487772in}}%
\pgfpathlineto{\pgfqpoint{4.668894in}{2.484823in}}%
\pgfpathmoveto{\pgfqpoint{4.668894in}{2.481874in}}%
\pgfpathlineto{\pgfqpoint{4.668894in}{2.481874in}}%
\pgfpathlineto{\pgfqpoint{4.668894in}{2.484823in}}%
\pgfpathlineto{\pgfqpoint{4.673434in}{2.484823in}}%
\pgfpathlineto{\pgfqpoint{4.673434in}{2.481874in}}%
\pgfpathmoveto{\pgfqpoint{4.668894in}{2.484823in}}%
\pgfpathlineto{\pgfqpoint{4.668894in}{2.484823in}}%
\pgfpathlineto{\pgfqpoint{4.668894in}{2.487772in}}%
\pgfpathlineto{\pgfqpoint{4.673434in}{2.487772in}}%
\pgfpathlineto{\pgfqpoint{4.673434in}{2.484823in}}%
\pgfpathmoveto{\pgfqpoint{4.664353in}{2.487772in}}%
\pgfpathlineto{\pgfqpoint{4.664353in}{2.487772in}}%
\pgfpathlineto{\pgfqpoint{4.664353in}{2.490722in}}%
\pgfpathlineto{\pgfqpoint{4.668894in}{2.490722in}}%
\pgfpathlineto{\pgfqpoint{4.668894in}{2.487772in}}%
\pgfpathmoveto{\pgfqpoint{4.655271in}{2.493671in}}%
\pgfpathlineto{\pgfqpoint{4.655271in}{2.493671in}}%
\pgfpathlineto{\pgfqpoint{4.655271in}{2.496620in}}%
\pgfpathlineto{\pgfqpoint{4.659812in}{2.496620in}}%
\pgfpathlineto{\pgfqpoint{4.659812in}{2.493671in}}%
\pgfpathmoveto{\pgfqpoint{4.637108in}{2.505468in}}%
\pgfpathlineto{\pgfqpoint{4.637108in}{2.505468in}}%
\pgfpathlineto{\pgfqpoint{4.637108in}{2.508418in}}%
\pgfpathlineto{\pgfqpoint{4.641649in}{2.508418in}}%
\pgfpathlineto{\pgfqpoint{4.641649in}{2.505468in}}%
\pgfpathmoveto{\pgfqpoint{4.673434in}{2.478925in}}%
\pgfpathlineto{\pgfqpoint{4.673434in}{2.478925in}}%
\pgfpathlineto{\pgfqpoint{4.673434in}{2.481874in}}%
\pgfpathlineto{\pgfqpoint{4.677976in}{2.481874in}}%
\pgfpathlineto{\pgfqpoint{4.677976in}{2.478925in}}%
\pgfpathmoveto{\pgfqpoint{4.677976in}{2.475975in}}%
\pgfpathlineto{\pgfqpoint{4.677976in}{2.475975in}}%
\pgfpathlineto{\pgfqpoint{4.677976in}{2.478925in}}%
\pgfpathlineto{\pgfqpoint{4.682517in}{2.478925in}}%
\pgfpathlineto{\pgfqpoint{4.682517in}{2.475975in}}%
\pgfpathmoveto{\pgfqpoint{4.677976in}{2.478925in}}%
\pgfpathlineto{\pgfqpoint{4.677976in}{2.478925in}}%
\pgfpathlineto{\pgfqpoint{4.677976in}{2.481874in}}%
\pgfpathlineto{\pgfqpoint{4.682517in}{2.481874in}}%
\pgfpathlineto{\pgfqpoint{4.682517in}{2.478925in}}%
\pgfpathmoveto{\pgfqpoint{4.682517in}{2.473026in}}%
\pgfpathlineto{\pgfqpoint{4.682517in}{2.473026in}}%
\pgfpathlineto{\pgfqpoint{4.682517in}{2.475975in}}%
\pgfpathlineto{\pgfqpoint{4.687058in}{2.475975in}}%
\pgfpathlineto{\pgfqpoint{4.687058in}{2.473026in}}%
\pgfpathmoveto{\pgfqpoint{4.687058in}{2.470077in}}%
\pgfpathlineto{\pgfqpoint{4.687058in}{2.470077in}}%
\pgfpathlineto{\pgfqpoint{4.687058in}{2.473026in}}%
\pgfpathlineto{\pgfqpoint{4.691599in}{2.473026in}}%
\pgfpathlineto{\pgfqpoint{4.691599in}{2.470077in}}%
\pgfpathmoveto{\pgfqpoint{4.687058in}{2.473026in}}%
\pgfpathlineto{\pgfqpoint{4.687058in}{2.473026in}}%
\pgfpathlineto{\pgfqpoint{4.687058in}{2.475975in}}%
\pgfpathlineto{\pgfqpoint{4.691599in}{2.475975in}}%
\pgfpathlineto{\pgfqpoint{4.691599in}{2.473026in}}%
\pgfpathmoveto{\pgfqpoint{4.682517in}{2.475975in}}%
\pgfpathlineto{\pgfqpoint{4.682517in}{2.475975in}}%
\pgfpathlineto{\pgfqpoint{4.682517in}{2.478925in}}%
\pgfpathlineto{\pgfqpoint{4.687058in}{2.478925in}}%
\pgfpathlineto{\pgfqpoint{4.687058in}{2.475975in}}%
\pgfpathmoveto{\pgfqpoint{4.691599in}{2.467128in}}%
\pgfpathlineto{\pgfqpoint{4.691599in}{2.467128in}}%
\pgfpathlineto{\pgfqpoint{4.691599in}{2.470077in}}%
\pgfpathlineto{\pgfqpoint{4.696140in}{2.470077in}}%
\pgfpathlineto{\pgfqpoint{4.696140in}{2.467128in}}%
\pgfpathmoveto{\pgfqpoint{4.696140in}{2.464179in}}%
\pgfpathlineto{\pgfqpoint{4.696140in}{2.464179in}}%
\pgfpathlineto{\pgfqpoint{4.696140in}{2.467128in}}%
\pgfpathlineto{\pgfqpoint{4.700681in}{2.467128in}}%
\pgfpathlineto{\pgfqpoint{4.700681in}{2.464179in}}%
\pgfpathmoveto{\pgfqpoint{4.696140in}{2.467128in}}%
\pgfpathlineto{\pgfqpoint{4.696140in}{2.467128in}}%
\pgfpathlineto{\pgfqpoint{4.696140in}{2.470077in}}%
\pgfpathlineto{\pgfqpoint{4.700681in}{2.470077in}}%
\pgfpathlineto{\pgfqpoint{4.700681in}{2.467128in}}%
\pgfpathmoveto{\pgfqpoint{4.700681in}{2.461229in}}%
\pgfpathlineto{\pgfqpoint{4.700681in}{2.461229in}}%
\pgfpathlineto{\pgfqpoint{4.700681in}{2.464179in}}%
\pgfpathlineto{\pgfqpoint{4.705222in}{2.464179in}}%
\pgfpathlineto{\pgfqpoint{4.705222in}{2.461229in}}%
\pgfpathmoveto{\pgfqpoint{4.705222in}{2.458280in}}%
\pgfpathlineto{\pgfqpoint{4.705222in}{2.458280in}}%
\pgfpathlineto{\pgfqpoint{4.705222in}{2.461229in}}%
\pgfpathlineto{\pgfqpoint{4.709763in}{2.461229in}}%
\pgfpathlineto{\pgfqpoint{4.709763in}{2.458280in}}%
\pgfpathmoveto{\pgfqpoint{4.705222in}{2.461229in}}%
\pgfpathlineto{\pgfqpoint{4.705222in}{2.461229in}}%
\pgfpathlineto{\pgfqpoint{4.705222in}{2.464179in}}%
\pgfpathlineto{\pgfqpoint{4.709763in}{2.464179in}}%
\pgfpathlineto{\pgfqpoint{4.709763in}{2.461229in}}%
\pgfpathmoveto{\pgfqpoint{4.700681in}{2.464179in}}%
\pgfpathlineto{\pgfqpoint{4.700681in}{2.464179in}}%
\pgfpathlineto{\pgfqpoint{4.700681in}{2.467128in}}%
\pgfpathlineto{\pgfqpoint{4.705222in}{2.467128in}}%
\pgfpathlineto{\pgfqpoint{4.705222in}{2.464179in}}%
\pgfpathmoveto{\pgfqpoint{4.691599in}{2.470077in}}%
\pgfpathlineto{\pgfqpoint{4.691599in}{2.470077in}}%
\pgfpathlineto{\pgfqpoint{4.691599in}{2.473026in}}%
\pgfpathlineto{\pgfqpoint{4.696140in}{2.473026in}}%
\pgfpathlineto{\pgfqpoint{4.696140in}{2.470077in}}%
\pgfpathmoveto{\pgfqpoint{4.709763in}{2.455331in}}%
\pgfpathlineto{\pgfqpoint{4.709763in}{2.455331in}}%
\pgfpathlineto{\pgfqpoint{4.709763in}{2.458280in}}%
\pgfpathlineto{\pgfqpoint{4.714304in}{2.458280in}}%
\pgfpathlineto{\pgfqpoint{4.714304in}{2.455331in}}%
\pgfpathmoveto{\pgfqpoint{4.714304in}{2.452382in}}%
\pgfpathlineto{\pgfqpoint{4.714304in}{2.452382in}}%
\pgfpathlineto{\pgfqpoint{4.714304in}{2.455331in}}%
\pgfpathlineto{\pgfqpoint{4.718845in}{2.455331in}}%
\pgfpathlineto{\pgfqpoint{4.718845in}{2.452382in}}%
\pgfpathmoveto{\pgfqpoint{4.714304in}{2.455331in}}%
\pgfpathlineto{\pgfqpoint{4.714304in}{2.455331in}}%
\pgfpathlineto{\pgfqpoint{4.714304in}{2.458280in}}%
\pgfpathlineto{\pgfqpoint{4.718845in}{2.458280in}}%
\pgfpathlineto{\pgfqpoint{4.718845in}{2.455331in}}%
\pgfpathmoveto{\pgfqpoint{4.718845in}{2.449432in}}%
\pgfpathlineto{\pgfqpoint{4.718845in}{2.449432in}}%
\pgfpathlineto{\pgfqpoint{4.718845in}{2.452382in}}%
\pgfpathlineto{\pgfqpoint{4.723386in}{2.452382in}}%
\pgfpathlineto{\pgfqpoint{4.723386in}{2.449432in}}%
\pgfpathmoveto{\pgfqpoint{4.723386in}{2.446483in}}%
\pgfpathlineto{\pgfqpoint{4.723386in}{2.446483in}}%
\pgfpathlineto{\pgfqpoint{4.723386in}{2.449432in}}%
\pgfpathlineto{\pgfqpoint{4.727927in}{2.449432in}}%
\pgfpathlineto{\pgfqpoint{4.727927in}{2.446483in}}%
\pgfpathmoveto{\pgfqpoint{4.723386in}{2.449432in}}%
\pgfpathlineto{\pgfqpoint{4.723386in}{2.449432in}}%
\pgfpathlineto{\pgfqpoint{4.723386in}{2.452382in}}%
\pgfpathlineto{\pgfqpoint{4.727927in}{2.452382in}}%
\pgfpathlineto{\pgfqpoint{4.727927in}{2.449432in}}%
\pgfpathmoveto{\pgfqpoint{4.718845in}{2.452382in}}%
\pgfpathlineto{\pgfqpoint{4.718845in}{2.452382in}}%
\pgfpathlineto{\pgfqpoint{4.718845in}{2.455331in}}%
\pgfpathlineto{\pgfqpoint{4.723386in}{2.455331in}}%
\pgfpathlineto{\pgfqpoint{4.723386in}{2.452382in}}%
\pgfpathmoveto{\pgfqpoint{4.727927in}{2.443534in}}%
\pgfpathlineto{\pgfqpoint{4.727927in}{2.443534in}}%
\pgfpathlineto{\pgfqpoint{4.727927in}{2.446483in}}%
\pgfpathlineto{\pgfqpoint{4.732468in}{2.446483in}}%
\pgfpathlineto{\pgfqpoint{4.732468in}{2.443534in}}%
\pgfpathmoveto{\pgfqpoint{4.732468in}{2.440585in}}%
\pgfpathlineto{\pgfqpoint{4.732468in}{2.440585in}}%
\pgfpathlineto{\pgfqpoint{4.732468in}{2.443534in}}%
\pgfpathlineto{\pgfqpoint{4.737009in}{2.443534in}}%
\pgfpathlineto{\pgfqpoint{4.737009in}{2.440585in}}%
\pgfpathmoveto{\pgfqpoint{4.732468in}{2.443534in}}%
\pgfpathlineto{\pgfqpoint{4.732468in}{2.443534in}}%
\pgfpathlineto{\pgfqpoint{4.732468in}{2.446483in}}%
\pgfpathlineto{\pgfqpoint{4.737009in}{2.446483in}}%
\pgfpathlineto{\pgfqpoint{4.737009in}{2.443534in}}%
\pgfpathmoveto{\pgfqpoint{4.737009in}{2.437636in}}%
\pgfpathlineto{\pgfqpoint{4.737009in}{2.437636in}}%
\pgfpathlineto{\pgfqpoint{4.737009in}{2.440585in}}%
\pgfpathlineto{\pgfqpoint{4.741550in}{2.440585in}}%
\pgfpathlineto{\pgfqpoint{4.741550in}{2.437636in}}%
\pgfpathmoveto{\pgfqpoint{4.741550in}{2.434686in}}%
\pgfpathlineto{\pgfqpoint{4.741550in}{2.434686in}}%
\pgfpathlineto{\pgfqpoint{4.741550in}{2.437636in}}%
\pgfpathlineto{\pgfqpoint{4.746091in}{2.437636in}}%
\pgfpathlineto{\pgfqpoint{4.746091in}{2.434686in}}%
\pgfpathmoveto{\pgfqpoint{4.741550in}{2.437636in}}%
\pgfpathlineto{\pgfqpoint{4.741550in}{2.437636in}}%
\pgfpathlineto{\pgfqpoint{4.741550in}{2.440585in}}%
\pgfpathlineto{\pgfqpoint{4.746091in}{2.440585in}}%
\pgfpathlineto{\pgfqpoint{4.746091in}{2.437636in}}%
\pgfpathmoveto{\pgfqpoint{4.737009in}{2.440585in}}%
\pgfpathlineto{\pgfqpoint{4.737009in}{2.440585in}}%
\pgfpathlineto{\pgfqpoint{4.737009in}{2.443534in}}%
\pgfpathlineto{\pgfqpoint{4.741550in}{2.443534in}}%
\pgfpathlineto{\pgfqpoint{4.741550in}{2.440585in}}%
\pgfpathmoveto{\pgfqpoint{4.727927in}{2.446483in}}%
\pgfpathlineto{\pgfqpoint{4.727927in}{2.446483in}}%
\pgfpathlineto{\pgfqpoint{4.727927in}{2.449432in}}%
\pgfpathlineto{\pgfqpoint{4.732468in}{2.449432in}}%
\pgfpathlineto{\pgfqpoint{4.732468in}{2.446483in}}%
\pgfpathmoveto{\pgfqpoint{4.709763in}{2.458280in}}%
\pgfpathlineto{\pgfqpoint{4.709763in}{2.458280in}}%
\pgfpathlineto{\pgfqpoint{4.709763in}{2.461229in}}%
\pgfpathlineto{\pgfqpoint{4.714304in}{2.461229in}}%
\pgfpathlineto{\pgfqpoint{4.714304in}{2.458280in}}%
\pgfpathmoveto{\pgfqpoint{4.746091in}{2.431737in}}%
\pgfpathlineto{\pgfqpoint{4.746091in}{2.431737in}}%
\pgfpathlineto{\pgfqpoint{4.746091in}{2.434686in}}%
\pgfpathlineto{\pgfqpoint{4.750632in}{2.434686in}}%
\pgfpathlineto{\pgfqpoint{4.750632in}{2.431737in}}%
\pgfpathmoveto{\pgfqpoint{4.750632in}{2.428788in}}%
\pgfpathlineto{\pgfqpoint{4.750632in}{2.428788in}}%
\pgfpathlineto{\pgfqpoint{4.750632in}{2.431737in}}%
\pgfpathlineto{\pgfqpoint{4.755173in}{2.431737in}}%
\pgfpathlineto{\pgfqpoint{4.755173in}{2.428788in}}%
\pgfpathmoveto{\pgfqpoint{4.750632in}{2.431737in}}%
\pgfpathlineto{\pgfqpoint{4.750632in}{2.431737in}}%
\pgfpathlineto{\pgfqpoint{4.750632in}{2.434686in}}%
\pgfpathlineto{\pgfqpoint{4.755173in}{2.434686in}}%
\pgfpathlineto{\pgfqpoint{4.755173in}{2.431737in}}%
\pgfpathmoveto{\pgfqpoint{4.755173in}{2.425839in}}%
\pgfpathlineto{\pgfqpoint{4.755173in}{2.425839in}}%
\pgfpathlineto{\pgfqpoint{4.755173in}{2.428788in}}%
\pgfpathlineto{\pgfqpoint{4.759714in}{2.428788in}}%
\pgfpathlineto{\pgfqpoint{4.759714in}{2.425839in}}%
\pgfpathmoveto{\pgfqpoint{4.759714in}{2.422890in}}%
\pgfpathlineto{\pgfqpoint{4.759714in}{2.422890in}}%
\pgfpathlineto{\pgfqpoint{4.759714in}{2.425839in}}%
\pgfpathlineto{\pgfqpoint{4.764255in}{2.425839in}}%
\pgfpathlineto{\pgfqpoint{4.764255in}{2.422890in}}%
\pgfpathmoveto{\pgfqpoint{4.759714in}{2.425839in}}%
\pgfpathlineto{\pgfqpoint{4.759714in}{2.425839in}}%
\pgfpathlineto{\pgfqpoint{4.759714in}{2.428788in}}%
\pgfpathlineto{\pgfqpoint{4.764255in}{2.428788in}}%
\pgfpathlineto{\pgfqpoint{4.764255in}{2.425839in}}%
\pgfpathmoveto{\pgfqpoint{4.755173in}{2.428788in}}%
\pgfpathlineto{\pgfqpoint{4.755173in}{2.428788in}}%
\pgfpathlineto{\pgfqpoint{4.755173in}{2.431737in}}%
\pgfpathlineto{\pgfqpoint{4.759714in}{2.431737in}}%
\pgfpathlineto{\pgfqpoint{4.759714in}{2.428788in}}%
\pgfpathmoveto{\pgfqpoint{4.764255in}{2.419940in}}%
\pgfpathlineto{\pgfqpoint{4.764255in}{2.419940in}}%
\pgfpathlineto{\pgfqpoint{4.764255in}{2.422890in}}%
\pgfpathlineto{\pgfqpoint{4.768796in}{2.422890in}}%
\pgfpathlineto{\pgfqpoint{4.768796in}{2.419940in}}%
\pgfpathmoveto{\pgfqpoint{4.768796in}{2.416991in}}%
\pgfpathlineto{\pgfqpoint{4.768796in}{2.416991in}}%
\pgfpathlineto{\pgfqpoint{4.768796in}{2.419940in}}%
\pgfpathlineto{\pgfqpoint{4.773337in}{2.419940in}}%
\pgfpathlineto{\pgfqpoint{4.773337in}{2.416991in}}%
\pgfpathmoveto{\pgfqpoint{4.768796in}{2.419940in}}%
\pgfpathlineto{\pgfqpoint{4.768796in}{2.419940in}}%
\pgfpathlineto{\pgfqpoint{4.768796in}{2.422890in}}%
\pgfpathlineto{\pgfqpoint{4.773337in}{2.422890in}}%
\pgfpathlineto{\pgfqpoint{4.773337in}{2.419940in}}%
\pgfpathmoveto{\pgfqpoint{4.773337in}{2.414042in}}%
\pgfpathlineto{\pgfqpoint{4.773337in}{2.414042in}}%
\pgfpathlineto{\pgfqpoint{4.773337in}{2.416991in}}%
\pgfpathlineto{\pgfqpoint{4.777878in}{2.416991in}}%
\pgfpathlineto{\pgfqpoint{4.777878in}{2.414042in}}%
\pgfpathmoveto{\pgfqpoint{4.777878in}{2.411093in}}%
\pgfpathlineto{\pgfqpoint{4.777878in}{2.411093in}}%
\pgfpathlineto{\pgfqpoint{4.777878in}{2.414042in}}%
\pgfpathlineto{\pgfqpoint{4.782419in}{2.414042in}}%
\pgfpathlineto{\pgfqpoint{4.782419in}{2.411093in}}%
\pgfpathmoveto{\pgfqpoint{4.777878in}{2.414042in}}%
\pgfpathlineto{\pgfqpoint{4.777878in}{2.414042in}}%
\pgfpathlineto{\pgfqpoint{4.777878in}{2.416991in}}%
\pgfpathlineto{\pgfqpoint{4.782419in}{2.416991in}}%
\pgfpathlineto{\pgfqpoint{4.782419in}{2.414042in}}%
\pgfpathmoveto{\pgfqpoint{4.773337in}{2.416991in}}%
\pgfpathlineto{\pgfqpoint{4.773337in}{2.416991in}}%
\pgfpathlineto{\pgfqpoint{4.773337in}{2.419940in}}%
\pgfpathlineto{\pgfqpoint{4.777878in}{2.419940in}}%
\pgfpathlineto{\pgfqpoint{4.777878in}{2.416991in}}%
\pgfpathmoveto{\pgfqpoint{4.764255in}{2.422890in}}%
\pgfpathlineto{\pgfqpoint{4.764255in}{2.422890in}}%
\pgfpathlineto{\pgfqpoint{4.764255in}{2.425839in}}%
\pgfpathlineto{\pgfqpoint{4.768796in}{2.425839in}}%
\pgfpathlineto{\pgfqpoint{4.768796in}{2.422890in}}%
\pgfpathmoveto{\pgfqpoint{4.782419in}{2.408144in}}%
\pgfpathlineto{\pgfqpoint{4.782419in}{2.408144in}}%
\pgfpathlineto{\pgfqpoint{4.782419in}{2.411093in}}%
\pgfpathlineto{\pgfqpoint{4.786960in}{2.411093in}}%
\pgfpathlineto{\pgfqpoint{4.786960in}{2.408144in}}%
\pgfpathmoveto{\pgfqpoint{4.786960in}{2.405194in}}%
\pgfpathlineto{\pgfqpoint{4.786960in}{2.405194in}}%
\pgfpathlineto{\pgfqpoint{4.786960in}{2.408144in}}%
\pgfpathlineto{\pgfqpoint{4.791501in}{2.408144in}}%
\pgfpathlineto{\pgfqpoint{4.791501in}{2.405194in}}%
\pgfpathmoveto{\pgfqpoint{4.786960in}{2.408144in}}%
\pgfpathlineto{\pgfqpoint{4.786960in}{2.408144in}}%
\pgfpathlineto{\pgfqpoint{4.786960in}{2.411093in}}%
\pgfpathlineto{\pgfqpoint{4.791501in}{2.411093in}}%
\pgfpathlineto{\pgfqpoint{4.791501in}{2.408144in}}%
\pgfpathmoveto{\pgfqpoint{4.791501in}{2.402245in}}%
\pgfpathlineto{\pgfqpoint{4.791501in}{2.402245in}}%
\pgfpathlineto{\pgfqpoint{4.791501in}{2.405194in}}%
\pgfpathlineto{\pgfqpoint{4.796042in}{2.405194in}}%
\pgfpathlineto{\pgfqpoint{4.796042in}{2.402245in}}%
\pgfpathmoveto{\pgfqpoint{4.796042in}{2.399296in}}%
\pgfpathlineto{\pgfqpoint{4.796042in}{2.399296in}}%
\pgfpathlineto{\pgfqpoint{4.796042in}{2.402245in}}%
\pgfpathlineto{\pgfqpoint{4.800583in}{2.402245in}}%
\pgfpathlineto{\pgfqpoint{4.800583in}{2.399296in}}%
\pgfpathmoveto{\pgfqpoint{4.796042in}{2.402245in}}%
\pgfpathlineto{\pgfqpoint{4.796042in}{2.402245in}}%
\pgfpathlineto{\pgfqpoint{4.796042in}{2.405194in}}%
\pgfpathlineto{\pgfqpoint{4.800583in}{2.405194in}}%
\pgfpathlineto{\pgfqpoint{4.800583in}{2.402245in}}%
\pgfpathmoveto{\pgfqpoint{4.791501in}{2.405194in}}%
\pgfpathlineto{\pgfqpoint{4.791501in}{2.405194in}}%
\pgfpathlineto{\pgfqpoint{4.791501in}{2.408144in}}%
\pgfpathlineto{\pgfqpoint{4.796042in}{2.408144in}}%
\pgfpathlineto{\pgfqpoint{4.796042in}{2.405194in}}%
\pgfpathmoveto{\pgfqpoint{4.800583in}{2.396347in}}%
\pgfpathlineto{\pgfqpoint{4.800583in}{2.396347in}}%
\pgfpathlineto{\pgfqpoint{4.800583in}{2.399296in}}%
\pgfpathlineto{\pgfqpoint{4.805124in}{2.399296in}}%
\pgfpathlineto{\pgfqpoint{4.805124in}{2.396347in}}%
\pgfpathmoveto{\pgfqpoint{4.805124in}{2.393397in}}%
\pgfpathlineto{\pgfqpoint{4.805124in}{2.393397in}}%
\pgfpathlineto{\pgfqpoint{4.805124in}{2.396347in}}%
\pgfpathlineto{\pgfqpoint{4.809665in}{2.396347in}}%
\pgfpathlineto{\pgfqpoint{4.809665in}{2.393397in}}%
\pgfpathmoveto{\pgfqpoint{4.805124in}{2.396347in}}%
\pgfpathlineto{\pgfqpoint{4.805124in}{2.396347in}}%
\pgfpathlineto{\pgfqpoint{4.805124in}{2.399296in}}%
\pgfpathlineto{\pgfqpoint{4.809665in}{2.399296in}}%
\pgfpathlineto{\pgfqpoint{4.809665in}{2.396347in}}%
\pgfpathmoveto{\pgfqpoint{4.809665in}{2.390448in}}%
\pgfpathlineto{\pgfqpoint{4.809665in}{2.390448in}}%
\pgfpathlineto{\pgfqpoint{4.809665in}{2.393397in}}%
\pgfpathlineto{\pgfqpoint{4.814206in}{2.393397in}}%
\pgfpathlineto{\pgfqpoint{4.814206in}{2.390448in}}%
\pgfpathmoveto{\pgfqpoint{4.814206in}{2.387499in}}%
\pgfpathlineto{\pgfqpoint{4.814206in}{2.387499in}}%
\pgfpathlineto{\pgfqpoint{4.814206in}{2.390448in}}%
\pgfpathlineto{\pgfqpoint{4.818747in}{2.390448in}}%
\pgfpathlineto{\pgfqpoint{4.818747in}{2.387499in}}%
\pgfpathmoveto{\pgfqpoint{4.814206in}{2.390448in}}%
\pgfpathlineto{\pgfqpoint{4.814206in}{2.390448in}}%
\pgfpathlineto{\pgfqpoint{4.814206in}{2.393397in}}%
\pgfpathlineto{\pgfqpoint{4.818747in}{2.393397in}}%
\pgfpathlineto{\pgfqpoint{4.818747in}{2.390448in}}%
\pgfpathmoveto{\pgfqpoint{4.809665in}{2.393397in}}%
\pgfpathlineto{\pgfqpoint{4.809665in}{2.393397in}}%
\pgfpathlineto{\pgfqpoint{4.809665in}{2.396347in}}%
\pgfpathlineto{\pgfqpoint{4.814206in}{2.396347in}}%
\pgfpathlineto{\pgfqpoint{4.814206in}{2.393397in}}%
\pgfpathmoveto{\pgfqpoint{4.800583in}{2.399296in}}%
\pgfpathlineto{\pgfqpoint{4.800583in}{2.399296in}}%
\pgfpathlineto{\pgfqpoint{4.800583in}{2.402245in}}%
\pgfpathlineto{\pgfqpoint{4.805124in}{2.402245in}}%
\pgfpathlineto{\pgfqpoint{4.805124in}{2.399296in}}%
\pgfpathmoveto{\pgfqpoint{4.782419in}{2.411093in}}%
\pgfpathlineto{\pgfqpoint{4.782419in}{2.411093in}}%
\pgfpathlineto{\pgfqpoint{4.782419in}{2.414042in}}%
\pgfpathlineto{\pgfqpoint{4.786960in}{2.414042in}}%
\pgfpathlineto{\pgfqpoint{4.786960in}{2.411093in}}%
\pgfpathmoveto{\pgfqpoint{4.746091in}{2.434686in}}%
\pgfpathlineto{\pgfqpoint{4.746091in}{2.434686in}}%
\pgfpathlineto{\pgfqpoint{4.746091in}{2.437636in}}%
\pgfpathlineto{\pgfqpoint{4.750632in}{2.437636in}}%
\pgfpathlineto{\pgfqpoint{4.750632in}{2.434686in}}%
\pgfpathmoveto{\pgfqpoint{4.673434in}{2.481874in}}%
\pgfpathlineto{\pgfqpoint{4.673434in}{2.481874in}}%
\pgfpathlineto{\pgfqpoint{4.673434in}{2.484823in}}%
\pgfpathlineto{\pgfqpoint{4.677976in}{2.484823in}}%
\pgfpathlineto{\pgfqpoint{4.677976in}{2.481874in}}%
\pgfpathmoveto{\pgfqpoint{4.818747in}{2.384550in}}%
\pgfpathlineto{\pgfqpoint{4.818747in}{2.384550in}}%
\pgfpathlineto{\pgfqpoint{4.818747in}{2.387499in}}%
\pgfpathlineto{\pgfqpoint{4.823288in}{2.387499in}}%
\pgfpathlineto{\pgfqpoint{4.823288in}{2.384550in}}%
\pgfpathmoveto{\pgfqpoint{4.823288in}{2.381601in}}%
\pgfpathlineto{\pgfqpoint{4.823288in}{2.381601in}}%
\pgfpathlineto{\pgfqpoint{4.823288in}{2.384550in}}%
\pgfpathlineto{\pgfqpoint{4.827830in}{2.384550in}}%
\pgfpathlineto{\pgfqpoint{4.827830in}{2.381601in}}%
\pgfpathmoveto{\pgfqpoint{4.823288in}{2.384550in}}%
\pgfpathlineto{\pgfqpoint{4.823288in}{2.384550in}}%
\pgfpathlineto{\pgfqpoint{4.823288in}{2.387499in}}%
\pgfpathlineto{\pgfqpoint{4.827830in}{2.387499in}}%
\pgfpathlineto{\pgfqpoint{4.827830in}{2.384550in}}%
\pgfpathmoveto{\pgfqpoint{4.827830in}{2.378651in}}%
\pgfpathlineto{\pgfqpoint{4.827830in}{2.378651in}}%
\pgfpathlineto{\pgfqpoint{4.827830in}{2.381601in}}%
\pgfpathlineto{\pgfqpoint{4.832371in}{2.381601in}}%
\pgfpathlineto{\pgfqpoint{4.832371in}{2.378651in}}%
\pgfpathmoveto{\pgfqpoint{4.832371in}{2.375702in}}%
\pgfpathlineto{\pgfqpoint{4.832371in}{2.375702in}}%
\pgfpathlineto{\pgfqpoint{4.832371in}{2.378651in}}%
\pgfpathlineto{\pgfqpoint{4.836912in}{2.378651in}}%
\pgfpathlineto{\pgfqpoint{4.836912in}{2.375702in}}%
\pgfpathmoveto{\pgfqpoint{4.832371in}{2.378651in}}%
\pgfpathlineto{\pgfqpoint{4.832371in}{2.378651in}}%
\pgfpathlineto{\pgfqpoint{4.832371in}{2.381601in}}%
\pgfpathlineto{\pgfqpoint{4.836912in}{2.381601in}}%
\pgfpathlineto{\pgfqpoint{4.836912in}{2.378651in}}%
\pgfpathmoveto{\pgfqpoint{4.827830in}{2.381601in}}%
\pgfpathlineto{\pgfqpoint{4.827830in}{2.381601in}}%
\pgfpathlineto{\pgfqpoint{4.827830in}{2.384550in}}%
\pgfpathlineto{\pgfqpoint{4.832371in}{2.384550in}}%
\pgfpathlineto{\pgfqpoint{4.832371in}{2.381601in}}%
\pgfpathmoveto{\pgfqpoint{4.836912in}{2.372753in}}%
\pgfpathlineto{\pgfqpoint{4.836912in}{2.372753in}}%
\pgfpathlineto{\pgfqpoint{4.836912in}{2.375702in}}%
\pgfpathlineto{\pgfqpoint{4.841453in}{2.375702in}}%
\pgfpathlineto{\pgfqpoint{4.841453in}{2.372753in}}%
\pgfpathmoveto{\pgfqpoint{4.841453in}{2.369804in}}%
\pgfpathlineto{\pgfqpoint{4.841453in}{2.369804in}}%
\pgfpathlineto{\pgfqpoint{4.841453in}{2.372753in}}%
\pgfpathlineto{\pgfqpoint{4.845994in}{2.372753in}}%
\pgfpathlineto{\pgfqpoint{4.845994in}{2.369804in}}%
\pgfpathmoveto{\pgfqpoint{4.841453in}{2.372753in}}%
\pgfpathlineto{\pgfqpoint{4.841453in}{2.372753in}}%
\pgfpathlineto{\pgfqpoint{4.841453in}{2.375702in}}%
\pgfpathlineto{\pgfqpoint{4.845994in}{2.375702in}}%
\pgfpathlineto{\pgfqpoint{4.845994in}{2.372753in}}%
\pgfpathmoveto{\pgfqpoint{4.845994in}{2.366854in}}%
\pgfpathlineto{\pgfqpoint{4.845994in}{2.366854in}}%
\pgfpathlineto{\pgfqpoint{4.845994in}{2.369804in}}%
\pgfpathlineto{\pgfqpoint{4.850535in}{2.369804in}}%
\pgfpathlineto{\pgfqpoint{4.850535in}{2.366854in}}%
\pgfpathmoveto{\pgfqpoint{4.850535in}{2.363905in}}%
\pgfpathlineto{\pgfqpoint{4.850535in}{2.363905in}}%
\pgfpathlineto{\pgfqpoint{4.850535in}{2.366854in}}%
\pgfpathlineto{\pgfqpoint{4.855076in}{2.366854in}}%
\pgfpathlineto{\pgfqpoint{4.855076in}{2.363905in}}%
\pgfpathmoveto{\pgfqpoint{4.850535in}{2.366854in}}%
\pgfpathlineto{\pgfqpoint{4.850535in}{2.366854in}}%
\pgfpathlineto{\pgfqpoint{4.850535in}{2.369804in}}%
\pgfpathlineto{\pgfqpoint{4.855076in}{2.369804in}}%
\pgfpathlineto{\pgfqpoint{4.855076in}{2.366854in}}%
\pgfpathmoveto{\pgfqpoint{4.845994in}{2.369804in}}%
\pgfpathlineto{\pgfqpoint{4.845994in}{2.369804in}}%
\pgfpathlineto{\pgfqpoint{4.845994in}{2.372753in}}%
\pgfpathlineto{\pgfqpoint{4.850535in}{2.372753in}}%
\pgfpathlineto{\pgfqpoint{4.850535in}{2.369804in}}%
\pgfpathmoveto{\pgfqpoint{4.836912in}{2.375702in}}%
\pgfpathlineto{\pgfqpoint{4.836912in}{2.375702in}}%
\pgfpathlineto{\pgfqpoint{4.836912in}{2.378651in}}%
\pgfpathlineto{\pgfqpoint{4.841453in}{2.378651in}}%
\pgfpathlineto{\pgfqpoint{4.841453in}{2.375702in}}%
\pgfpathmoveto{\pgfqpoint{4.855076in}{2.360956in}}%
\pgfpathlineto{\pgfqpoint{4.855076in}{2.360956in}}%
\pgfpathlineto{\pgfqpoint{4.855076in}{2.363905in}}%
\pgfpathlineto{\pgfqpoint{4.859617in}{2.363905in}}%
\pgfpathlineto{\pgfqpoint{4.859617in}{2.360956in}}%
\pgfpathmoveto{\pgfqpoint{4.859617in}{2.358007in}}%
\pgfpathlineto{\pgfqpoint{4.859617in}{2.358007in}}%
\pgfpathlineto{\pgfqpoint{4.859617in}{2.360956in}}%
\pgfpathlineto{\pgfqpoint{4.864159in}{2.360956in}}%
\pgfpathlineto{\pgfqpoint{4.864159in}{2.358007in}}%
\pgfpathmoveto{\pgfqpoint{4.859617in}{2.360956in}}%
\pgfpathlineto{\pgfqpoint{4.859617in}{2.360956in}}%
\pgfpathlineto{\pgfqpoint{4.859617in}{2.363905in}}%
\pgfpathlineto{\pgfqpoint{4.864159in}{2.363905in}}%
\pgfpathlineto{\pgfqpoint{4.864159in}{2.360956in}}%
\pgfpathmoveto{\pgfqpoint{4.864159in}{2.355057in}}%
\pgfpathlineto{\pgfqpoint{4.864159in}{2.355057in}}%
\pgfpathlineto{\pgfqpoint{4.864159in}{2.358007in}}%
\pgfpathlineto{\pgfqpoint{4.868700in}{2.358007in}}%
\pgfpathlineto{\pgfqpoint{4.868700in}{2.355057in}}%
\pgfpathmoveto{\pgfqpoint{4.868700in}{2.352108in}}%
\pgfpathlineto{\pgfqpoint{4.868700in}{2.352108in}}%
\pgfpathlineto{\pgfqpoint{4.868700in}{2.355057in}}%
\pgfpathlineto{\pgfqpoint{4.873241in}{2.355057in}}%
\pgfpathlineto{\pgfqpoint{4.873241in}{2.352108in}}%
\pgfpathmoveto{\pgfqpoint{4.868700in}{2.355057in}}%
\pgfpathlineto{\pgfqpoint{4.868700in}{2.355057in}}%
\pgfpathlineto{\pgfqpoint{4.868700in}{2.358007in}}%
\pgfpathlineto{\pgfqpoint{4.873241in}{2.358007in}}%
\pgfpathlineto{\pgfqpoint{4.873241in}{2.355057in}}%
\pgfpathmoveto{\pgfqpoint{4.864159in}{2.358007in}}%
\pgfpathlineto{\pgfqpoint{4.864159in}{2.358007in}}%
\pgfpathlineto{\pgfqpoint{4.864159in}{2.360956in}}%
\pgfpathlineto{\pgfqpoint{4.868700in}{2.360956in}}%
\pgfpathlineto{\pgfqpoint{4.868700in}{2.358007in}}%
\pgfpathmoveto{\pgfqpoint{4.873241in}{2.349159in}}%
\pgfpathlineto{\pgfqpoint{4.873241in}{2.349159in}}%
\pgfpathlineto{\pgfqpoint{4.873241in}{2.352108in}}%
\pgfpathlineto{\pgfqpoint{4.877782in}{2.352108in}}%
\pgfpathlineto{\pgfqpoint{4.877782in}{2.349159in}}%
\pgfpathmoveto{\pgfqpoint{4.877782in}{2.346210in}}%
\pgfpathlineto{\pgfqpoint{4.877782in}{2.346210in}}%
\pgfpathlineto{\pgfqpoint{4.877782in}{2.349159in}}%
\pgfpathlineto{\pgfqpoint{4.882323in}{2.349159in}}%
\pgfpathlineto{\pgfqpoint{4.882323in}{2.346210in}}%
\pgfpathmoveto{\pgfqpoint{4.877782in}{2.349159in}}%
\pgfpathlineto{\pgfqpoint{4.877782in}{2.349159in}}%
\pgfpathlineto{\pgfqpoint{4.877782in}{2.352108in}}%
\pgfpathlineto{\pgfqpoint{4.882323in}{2.352108in}}%
\pgfpathlineto{\pgfqpoint{4.882323in}{2.349159in}}%
\pgfpathmoveto{\pgfqpoint{4.882323in}{2.343260in}}%
\pgfpathlineto{\pgfqpoint{4.882323in}{2.343260in}}%
\pgfpathlineto{\pgfqpoint{4.882323in}{2.346210in}}%
\pgfpathlineto{\pgfqpoint{4.886864in}{2.346210in}}%
\pgfpathlineto{\pgfqpoint{4.886864in}{2.343260in}}%
\pgfpathmoveto{\pgfqpoint{4.886864in}{2.340311in}}%
\pgfpathlineto{\pgfqpoint{4.886864in}{2.340311in}}%
\pgfpathlineto{\pgfqpoint{4.886864in}{2.343260in}}%
\pgfpathlineto{\pgfqpoint{4.891405in}{2.343260in}}%
\pgfpathlineto{\pgfqpoint{4.891405in}{2.340311in}}%
\pgfpathmoveto{\pgfqpoint{4.886864in}{2.343260in}}%
\pgfpathlineto{\pgfqpoint{4.886864in}{2.343260in}}%
\pgfpathlineto{\pgfqpoint{4.886864in}{2.346210in}}%
\pgfpathlineto{\pgfqpoint{4.891405in}{2.346210in}}%
\pgfpathlineto{\pgfqpoint{4.891405in}{2.343260in}}%
\pgfpathmoveto{\pgfqpoint{4.882323in}{2.346210in}}%
\pgfpathlineto{\pgfqpoint{4.882323in}{2.346210in}}%
\pgfpathlineto{\pgfqpoint{4.882323in}{2.349159in}}%
\pgfpathlineto{\pgfqpoint{4.886864in}{2.349159in}}%
\pgfpathlineto{\pgfqpoint{4.886864in}{2.346210in}}%
\pgfpathmoveto{\pgfqpoint{4.873241in}{2.352108in}}%
\pgfpathlineto{\pgfqpoint{4.873241in}{2.352108in}}%
\pgfpathlineto{\pgfqpoint{4.873241in}{2.355057in}}%
\pgfpathlineto{\pgfqpoint{4.877782in}{2.355057in}}%
\pgfpathlineto{\pgfqpoint{4.877782in}{2.352108in}}%
\pgfpathmoveto{\pgfqpoint{4.855076in}{2.363905in}}%
\pgfpathlineto{\pgfqpoint{4.855076in}{2.363905in}}%
\pgfpathlineto{\pgfqpoint{4.855076in}{2.366854in}}%
\pgfpathlineto{\pgfqpoint{4.859617in}{2.366854in}}%
\pgfpathlineto{\pgfqpoint{4.859617in}{2.363905in}}%
\pgfpathmoveto{\pgfqpoint{4.891405in}{2.337362in}}%
\pgfpathlineto{\pgfqpoint{4.891405in}{2.337362in}}%
\pgfpathlineto{\pgfqpoint{4.891405in}{2.340311in}}%
\pgfpathlineto{\pgfqpoint{4.895946in}{2.340311in}}%
\pgfpathlineto{\pgfqpoint{4.895946in}{2.337362in}}%
\pgfpathmoveto{\pgfqpoint{4.895946in}{2.334413in}}%
\pgfpathlineto{\pgfqpoint{4.895946in}{2.334413in}}%
\pgfpathlineto{\pgfqpoint{4.895946in}{2.337362in}}%
\pgfpathlineto{\pgfqpoint{4.900488in}{2.337362in}}%
\pgfpathlineto{\pgfqpoint{4.900488in}{2.334413in}}%
\pgfpathmoveto{\pgfqpoint{4.895946in}{2.337362in}}%
\pgfpathlineto{\pgfqpoint{4.895946in}{2.337362in}}%
\pgfpathlineto{\pgfqpoint{4.895946in}{2.340311in}}%
\pgfpathlineto{\pgfqpoint{4.900488in}{2.340311in}}%
\pgfpathlineto{\pgfqpoint{4.900488in}{2.337362in}}%
\pgfpathmoveto{\pgfqpoint{4.900488in}{2.331463in}}%
\pgfpathlineto{\pgfqpoint{4.900488in}{2.331463in}}%
\pgfpathlineto{\pgfqpoint{4.900488in}{2.334413in}}%
\pgfpathlineto{\pgfqpoint{4.905029in}{2.334413in}}%
\pgfpathlineto{\pgfqpoint{4.905029in}{2.331463in}}%
\pgfpathmoveto{\pgfqpoint{4.905029in}{2.328514in}}%
\pgfpathlineto{\pgfqpoint{4.905029in}{2.328514in}}%
\pgfpathlineto{\pgfqpoint{4.905029in}{2.331463in}}%
\pgfpathlineto{\pgfqpoint{4.909570in}{2.331463in}}%
\pgfpathlineto{\pgfqpoint{4.909570in}{2.328514in}}%
\pgfpathmoveto{\pgfqpoint{4.905029in}{2.331463in}}%
\pgfpathlineto{\pgfqpoint{4.905029in}{2.331463in}}%
\pgfpathlineto{\pgfqpoint{4.905029in}{2.334413in}}%
\pgfpathlineto{\pgfqpoint{4.909570in}{2.334413in}}%
\pgfpathlineto{\pgfqpoint{4.909570in}{2.331463in}}%
\pgfpathmoveto{\pgfqpoint{4.900488in}{2.334413in}}%
\pgfpathlineto{\pgfqpoint{4.900488in}{2.334413in}}%
\pgfpathlineto{\pgfqpoint{4.900488in}{2.337362in}}%
\pgfpathlineto{\pgfqpoint{4.905029in}{2.337362in}}%
\pgfpathlineto{\pgfqpoint{4.905029in}{2.334413in}}%
\pgfpathmoveto{\pgfqpoint{4.909570in}{2.325565in}}%
\pgfpathlineto{\pgfqpoint{4.909570in}{2.325565in}}%
\pgfpathlineto{\pgfqpoint{4.909570in}{2.328514in}}%
\pgfpathlineto{\pgfqpoint{4.914111in}{2.328514in}}%
\pgfpathlineto{\pgfqpoint{4.914111in}{2.325565in}}%
\pgfpathmoveto{\pgfqpoint{4.914111in}{2.322616in}}%
\pgfpathlineto{\pgfqpoint{4.914111in}{2.322616in}}%
\pgfpathlineto{\pgfqpoint{4.914111in}{2.325565in}}%
\pgfpathlineto{\pgfqpoint{4.918652in}{2.325565in}}%
\pgfpathlineto{\pgfqpoint{4.918652in}{2.322616in}}%
\pgfpathmoveto{\pgfqpoint{4.914111in}{2.325565in}}%
\pgfpathlineto{\pgfqpoint{4.914111in}{2.325565in}}%
\pgfpathlineto{\pgfqpoint{4.914111in}{2.328514in}}%
\pgfpathlineto{\pgfqpoint{4.918652in}{2.328514in}}%
\pgfpathlineto{\pgfqpoint{4.918652in}{2.325565in}}%
\pgfpathmoveto{\pgfqpoint{4.918652in}{2.319667in}}%
\pgfpathlineto{\pgfqpoint{4.918652in}{2.319667in}}%
\pgfpathlineto{\pgfqpoint{4.918652in}{2.322616in}}%
\pgfpathlineto{\pgfqpoint{4.923193in}{2.322616in}}%
\pgfpathlineto{\pgfqpoint{4.923193in}{2.319667in}}%
\pgfpathmoveto{\pgfqpoint{4.923193in}{2.316717in}}%
\pgfpathlineto{\pgfqpoint{4.923193in}{2.316717in}}%
\pgfpathlineto{\pgfqpoint{4.923193in}{2.319667in}}%
\pgfpathlineto{\pgfqpoint{4.927734in}{2.319667in}}%
\pgfpathlineto{\pgfqpoint{4.927734in}{2.316717in}}%
\pgfpathmoveto{\pgfqpoint{4.923193in}{2.319667in}}%
\pgfpathlineto{\pgfqpoint{4.923193in}{2.319667in}}%
\pgfpathlineto{\pgfqpoint{4.923193in}{2.322616in}}%
\pgfpathlineto{\pgfqpoint{4.927734in}{2.322616in}}%
\pgfpathlineto{\pgfqpoint{4.927734in}{2.319667in}}%
\pgfpathmoveto{\pgfqpoint{4.918652in}{2.322616in}}%
\pgfpathlineto{\pgfqpoint{4.918652in}{2.322616in}}%
\pgfpathlineto{\pgfqpoint{4.918652in}{2.325565in}}%
\pgfpathlineto{\pgfqpoint{4.923193in}{2.325565in}}%
\pgfpathlineto{\pgfqpoint{4.923193in}{2.322616in}}%
\pgfpathmoveto{\pgfqpoint{4.909570in}{2.328514in}}%
\pgfpathlineto{\pgfqpoint{4.909570in}{2.328514in}}%
\pgfpathlineto{\pgfqpoint{4.909570in}{2.331463in}}%
\pgfpathlineto{\pgfqpoint{4.914111in}{2.331463in}}%
\pgfpathlineto{\pgfqpoint{4.914111in}{2.328514in}}%
\pgfpathmoveto{\pgfqpoint{4.927734in}{2.313768in}}%
\pgfpathlineto{\pgfqpoint{4.927734in}{2.313768in}}%
\pgfpathlineto{\pgfqpoint{4.927734in}{2.316717in}}%
\pgfpathlineto{\pgfqpoint{4.932275in}{2.316717in}}%
\pgfpathlineto{\pgfqpoint{4.932275in}{2.313768in}}%
\pgfpathmoveto{\pgfqpoint{4.932275in}{2.310819in}}%
\pgfpathlineto{\pgfqpoint{4.932275in}{2.310819in}}%
\pgfpathlineto{\pgfqpoint{4.932275in}{2.313768in}}%
\pgfpathlineto{\pgfqpoint{4.936817in}{2.313768in}}%
\pgfpathlineto{\pgfqpoint{4.936817in}{2.310819in}}%
\pgfpathmoveto{\pgfqpoint{4.932275in}{2.313768in}}%
\pgfpathlineto{\pgfqpoint{4.932275in}{2.313768in}}%
\pgfpathlineto{\pgfqpoint{4.932275in}{2.316717in}}%
\pgfpathlineto{\pgfqpoint{4.936817in}{2.316717in}}%
\pgfpathlineto{\pgfqpoint{4.936817in}{2.313768in}}%
\pgfpathmoveto{\pgfqpoint{4.936817in}{2.307870in}}%
\pgfpathlineto{\pgfqpoint{4.936817in}{2.307870in}}%
\pgfpathlineto{\pgfqpoint{4.936817in}{2.310819in}}%
\pgfpathlineto{\pgfqpoint{4.941358in}{2.310819in}}%
\pgfpathlineto{\pgfqpoint{4.941358in}{2.307870in}}%
\pgfpathmoveto{\pgfqpoint{4.941358in}{2.304920in}}%
\pgfpathlineto{\pgfqpoint{4.941358in}{2.304920in}}%
\pgfpathlineto{\pgfqpoint{4.941358in}{2.307870in}}%
\pgfpathlineto{\pgfqpoint{4.945899in}{2.307870in}}%
\pgfpathlineto{\pgfqpoint{4.945899in}{2.304920in}}%
\pgfpathmoveto{\pgfqpoint{4.941358in}{2.307870in}}%
\pgfpathlineto{\pgfqpoint{4.941358in}{2.307870in}}%
\pgfpathlineto{\pgfqpoint{4.941358in}{2.310819in}}%
\pgfpathlineto{\pgfqpoint{4.945899in}{2.310819in}}%
\pgfpathlineto{\pgfqpoint{4.945899in}{2.307870in}}%
\pgfpathmoveto{\pgfqpoint{4.936817in}{2.310819in}}%
\pgfpathlineto{\pgfqpoint{4.936817in}{2.310819in}}%
\pgfpathlineto{\pgfqpoint{4.936817in}{2.313768in}}%
\pgfpathlineto{\pgfqpoint{4.941358in}{2.313768in}}%
\pgfpathlineto{\pgfqpoint{4.941358in}{2.310819in}}%
\pgfpathmoveto{\pgfqpoint{4.945899in}{2.301971in}}%
\pgfpathlineto{\pgfqpoint{4.945899in}{2.301971in}}%
\pgfpathlineto{\pgfqpoint{4.945899in}{2.304920in}}%
\pgfpathlineto{\pgfqpoint{4.950440in}{2.304920in}}%
\pgfpathlineto{\pgfqpoint{4.950440in}{2.301971in}}%
\pgfpathmoveto{\pgfqpoint{4.950440in}{2.299022in}}%
\pgfpathlineto{\pgfqpoint{4.950440in}{2.299022in}}%
\pgfpathlineto{\pgfqpoint{4.950440in}{2.301971in}}%
\pgfpathlineto{\pgfqpoint{4.954981in}{2.301971in}}%
\pgfpathlineto{\pgfqpoint{4.954981in}{2.299022in}}%
\pgfpathmoveto{\pgfqpoint{4.950440in}{2.301971in}}%
\pgfpathlineto{\pgfqpoint{4.950440in}{2.301971in}}%
\pgfpathlineto{\pgfqpoint{4.950440in}{2.304920in}}%
\pgfpathlineto{\pgfqpoint{4.954981in}{2.304920in}}%
\pgfpathlineto{\pgfqpoint{4.954981in}{2.301971in}}%
\pgfpathmoveto{\pgfqpoint{4.954981in}{2.296073in}}%
\pgfpathlineto{\pgfqpoint{4.954981in}{2.296073in}}%
\pgfpathlineto{\pgfqpoint{4.954981in}{2.299022in}}%
\pgfpathlineto{\pgfqpoint{4.959522in}{2.299022in}}%
\pgfpathlineto{\pgfqpoint{4.959522in}{2.296073in}}%
\pgfpathmoveto{\pgfqpoint{4.959522in}{2.293123in}}%
\pgfpathlineto{\pgfqpoint{4.959522in}{2.293123in}}%
\pgfpathlineto{\pgfqpoint{4.959522in}{2.296073in}}%
\pgfpathlineto{\pgfqpoint{4.964063in}{2.296073in}}%
\pgfpathlineto{\pgfqpoint{4.964063in}{2.293123in}}%
\pgfpathmoveto{\pgfqpoint{4.959522in}{2.296073in}}%
\pgfpathlineto{\pgfqpoint{4.959522in}{2.296073in}}%
\pgfpathlineto{\pgfqpoint{4.959522in}{2.299022in}}%
\pgfpathlineto{\pgfqpoint{4.964063in}{2.299022in}}%
\pgfpathlineto{\pgfqpoint{4.964063in}{2.296073in}}%
\pgfpathmoveto{\pgfqpoint{4.954981in}{2.299022in}}%
\pgfpathlineto{\pgfqpoint{4.954981in}{2.299022in}}%
\pgfpathlineto{\pgfqpoint{4.954981in}{2.301971in}}%
\pgfpathlineto{\pgfqpoint{4.959522in}{2.301971in}}%
\pgfpathlineto{\pgfqpoint{4.959522in}{2.299022in}}%
\pgfpathmoveto{\pgfqpoint{4.945899in}{2.304920in}}%
\pgfpathlineto{\pgfqpoint{4.945899in}{2.304920in}}%
\pgfpathlineto{\pgfqpoint{4.945899in}{2.307870in}}%
\pgfpathlineto{\pgfqpoint{4.950440in}{2.307870in}}%
\pgfpathlineto{\pgfqpoint{4.950440in}{2.304920in}}%
\pgfpathmoveto{\pgfqpoint{4.927734in}{2.316717in}}%
\pgfpathlineto{\pgfqpoint{4.927734in}{2.316717in}}%
\pgfpathlineto{\pgfqpoint{4.927734in}{2.319667in}}%
\pgfpathlineto{\pgfqpoint{4.932275in}{2.319667in}}%
\pgfpathlineto{\pgfqpoint{4.932275in}{2.316717in}}%
\pgfpathmoveto{\pgfqpoint{4.891405in}{2.340311in}}%
\pgfpathlineto{\pgfqpoint{4.891405in}{2.340311in}}%
\pgfpathlineto{\pgfqpoint{4.891405in}{2.343260in}}%
\pgfpathlineto{\pgfqpoint{4.895946in}{2.343260in}}%
\pgfpathlineto{\pgfqpoint{4.895946in}{2.340311in}}%
\pgfpathmoveto{\pgfqpoint{4.818747in}{2.387499in}}%
\pgfpathlineto{\pgfqpoint{4.818747in}{2.387499in}}%
\pgfpathlineto{\pgfqpoint{4.818747in}{2.390448in}}%
\pgfpathlineto{\pgfqpoint{4.823288in}{2.390448in}}%
\pgfpathlineto{\pgfqpoint{4.823288in}{2.387499in}}%
\pgfpathmoveto{\pgfqpoint{5.104838in}{2.195801in}}%
\pgfpathlineto{\pgfqpoint{5.104838in}{2.195801in}}%
\pgfpathlineto{\pgfqpoint{5.104838in}{2.198750in}}%
\pgfpathlineto{\pgfqpoint{5.109379in}{2.198750in}}%
\pgfpathlineto{\pgfqpoint{5.109379in}{2.195801in}}%
\pgfpathmoveto{\pgfqpoint{5.032180in}{2.242987in}}%
\pgfpathlineto{\pgfqpoint{5.032180in}{2.242987in}}%
\pgfpathlineto{\pgfqpoint{5.032180in}{2.245937in}}%
\pgfpathlineto{\pgfqpoint{5.036721in}{2.245937in}}%
\pgfpathlineto{\pgfqpoint{5.036721in}{2.242987in}}%
\pgfpathmoveto{\pgfqpoint{4.964063in}{2.290174in}}%
\pgfpathlineto{\pgfqpoint{4.964063in}{2.290174in}}%
\pgfpathlineto{\pgfqpoint{4.964063in}{2.293123in}}%
\pgfpathlineto{\pgfqpoint{4.968604in}{2.293123in}}%
\pgfpathlineto{\pgfqpoint{4.968604in}{2.290174in}}%
\pgfpathmoveto{\pgfqpoint{4.968604in}{2.287225in}}%
\pgfpathlineto{\pgfqpoint{4.968604in}{2.287225in}}%
\pgfpathlineto{\pgfqpoint{4.968604in}{2.290174in}}%
\pgfpathlineto{\pgfqpoint{4.973146in}{2.290174in}}%
\pgfpathlineto{\pgfqpoint{4.973146in}{2.287225in}}%
\pgfpathmoveto{\pgfqpoint{4.968604in}{2.290174in}}%
\pgfpathlineto{\pgfqpoint{4.968604in}{2.290174in}}%
\pgfpathlineto{\pgfqpoint{4.968604in}{2.293123in}}%
\pgfpathlineto{\pgfqpoint{4.973146in}{2.293123in}}%
\pgfpathlineto{\pgfqpoint{4.973146in}{2.290174in}}%
\pgfpathmoveto{\pgfqpoint{4.973146in}{2.284276in}}%
\pgfpathlineto{\pgfqpoint{4.973146in}{2.284276in}}%
\pgfpathlineto{\pgfqpoint{4.973146in}{2.287225in}}%
\pgfpathlineto{\pgfqpoint{4.977687in}{2.287225in}}%
\pgfpathlineto{\pgfqpoint{4.977687in}{2.284276in}}%
\pgfpathmoveto{\pgfqpoint{4.977687in}{2.281327in}}%
\pgfpathlineto{\pgfqpoint{4.977687in}{2.281327in}}%
\pgfpathlineto{\pgfqpoint{4.977687in}{2.284276in}}%
\pgfpathlineto{\pgfqpoint{4.982228in}{2.284276in}}%
\pgfpathlineto{\pgfqpoint{4.982228in}{2.281327in}}%
\pgfpathmoveto{\pgfqpoint{4.977687in}{2.284276in}}%
\pgfpathlineto{\pgfqpoint{4.977687in}{2.284276in}}%
\pgfpathlineto{\pgfqpoint{4.977687in}{2.287225in}}%
\pgfpathlineto{\pgfqpoint{4.982228in}{2.287225in}}%
\pgfpathlineto{\pgfqpoint{4.982228in}{2.284276in}}%
\pgfpathmoveto{\pgfqpoint{4.973146in}{2.287225in}}%
\pgfpathlineto{\pgfqpoint{4.973146in}{2.287225in}}%
\pgfpathlineto{\pgfqpoint{4.973146in}{2.290174in}}%
\pgfpathlineto{\pgfqpoint{4.977687in}{2.290174in}}%
\pgfpathlineto{\pgfqpoint{4.977687in}{2.287225in}}%
\pgfpathmoveto{\pgfqpoint{4.982228in}{2.278377in}}%
\pgfpathlineto{\pgfqpoint{4.982228in}{2.278377in}}%
\pgfpathlineto{\pgfqpoint{4.982228in}{2.281327in}}%
\pgfpathlineto{\pgfqpoint{4.986769in}{2.281327in}}%
\pgfpathlineto{\pgfqpoint{4.986769in}{2.278377in}}%
\pgfpathmoveto{\pgfqpoint{4.986769in}{2.275428in}}%
\pgfpathlineto{\pgfqpoint{4.986769in}{2.275428in}}%
\pgfpathlineto{\pgfqpoint{4.986769in}{2.278377in}}%
\pgfpathlineto{\pgfqpoint{4.991310in}{2.278377in}}%
\pgfpathlineto{\pgfqpoint{4.991310in}{2.275428in}}%
\pgfpathmoveto{\pgfqpoint{4.986769in}{2.278377in}}%
\pgfpathlineto{\pgfqpoint{4.986769in}{2.278377in}}%
\pgfpathlineto{\pgfqpoint{4.986769in}{2.281327in}}%
\pgfpathlineto{\pgfqpoint{4.991310in}{2.281327in}}%
\pgfpathlineto{\pgfqpoint{4.991310in}{2.278377in}}%
\pgfpathmoveto{\pgfqpoint{4.991310in}{2.272479in}}%
\pgfpathlineto{\pgfqpoint{4.991310in}{2.272479in}}%
\pgfpathlineto{\pgfqpoint{4.991310in}{2.275428in}}%
\pgfpathlineto{\pgfqpoint{4.995851in}{2.275428in}}%
\pgfpathlineto{\pgfqpoint{4.995851in}{2.272479in}}%
\pgfpathmoveto{\pgfqpoint{4.995851in}{2.269530in}}%
\pgfpathlineto{\pgfqpoint{4.995851in}{2.269530in}}%
\pgfpathlineto{\pgfqpoint{4.995851in}{2.272479in}}%
\pgfpathlineto{\pgfqpoint{5.000392in}{2.272479in}}%
\pgfpathlineto{\pgfqpoint{5.000392in}{2.269530in}}%
\pgfpathmoveto{\pgfqpoint{4.995851in}{2.272479in}}%
\pgfpathlineto{\pgfqpoint{4.995851in}{2.272479in}}%
\pgfpathlineto{\pgfqpoint{4.995851in}{2.275428in}}%
\pgfpathlineto{\pgfqpoint{5.000392in}{2.275428in}}%
\pgfpathlineto{\pgfqpoint{5.000392in}{2.272479in}}%
\pgfpathmoveto{\pgfqpoint{4.991310in}{2.275428in}}%
\pgfpathlineto{\pgfqpoint{4.991310in}{2.275428in}}%
\pgfpathlineto{\pgfqpoint{4.991310in}{2.278377in}}%
\pgfpathlineto{\pgfqpoint{4.995851in}{2.278377in}}%
\pgfpathlineto{\pgfqpoint{4.995851in}{2.275428in}}%
\pgfpathmoveto{\pgfqpoint{4.982228in}{2.281327in}}%
\pgfpathlineto{\pgfqpoint{4.982228in}{2.281327in}}%
\pgfpathlineto{\pgfqpoint{4.982228in}{2.284276in}}%
\pgfpathlineto{\pgfqpoint{4.986769in}{2.284276in}}%
\pgfpathlineto{\pgfqpoint{4.986769in}{2.281327in}}%
\pgfpathmoveto{\pgfqpoint{5.014016in}{2.254784in}}%
\pgfpathlineto{\pgfqpoint{5.014016in}{2.254784in}}%
\pgfpathlineto{\pgfqpoint{5.014016in}{2.257733in}}%
\pgfpathlineto{\pgfqpoint{5.018557in}{2.257733in}}%
\pgfpathlineto{\pgfqpoint{5.018557in}{2.254784in}}%
\pgfpathmoveto{\pgfqpoint{5.000392in}{2.266581in}}%
\pgfpathlineto{\pgfqpoint{5.000392in}{2.266581in}}%
\pgfpathlineto{\pgfqpoint{5.000392in}{2.269530in}}%
\pgfpathlineto{\pgfqpoint{5.004933in}{2.269530in}}%
\pgfpathlineto{\pgfqpoint{5.004933in}{2.266581in}}%
\pgfpathmoveto{\pgfqpoint{5.004933in}{2.263632in}}%
\pgfpathlineto{\pgfqpoint{5.004933in}{2.263632in}}%
\pgfpathlineto{\pgfqpoint{5.004933in}{2.266581in}}%
\pgfpathlineto{\pgfqpoint{5.009474in}{2.266581in}}%
\pgfpathlineto{\pgfqpoint{5.009474in}{2.263632in}}%
\pgfpathmoveto{\pgfqpoint{5.004933in}{2.266581in}}%
\pgfpathlineto{\pgfqpoint{5.004933in}{2.266581in}}%
\pgfpathlineto{\pgfqpoint{5.004933in}{2.269530in}}%
\pgfpathlineto{\pgfqpoint{5.009474in}{2.269530in}}%
\pgfpathlineto{\pgfqpoint{5.009474in}{2.266581in}}%
\pgfpathmoveto{\pgfqpoint{5.009474in}{2.257733in}}%
\pgfpathlineto{\pgfqpoint{5.009474in}{2.257733in}}%
\pgfpathlineto{\pgfqpoint{5.009474in}{2.260682in}}%
\pgfpathlineto{\pgfqpoint{5.014016in}{2.260682in}}%
\pgfpathlineto{\pgfqpoint{5.014016in}{2.257733in}}%
\pgfpathmoveto{\pgfqpoint{5.009474in}{2.260682in}}%
\pgfpathlineto{\pgfqpoint{5.009474in}{2.260682in}}%
\pgfpathlineto{\pgfqpoint{5.009474in}{2.263632in}}%
\pgfpathlineto{\pgfqpoint{5.014016in}{2.263632in}}%
\pgfpathlineto{\pgfqpoint{5.014016in}{2.260682in}}%
\pgfpathmoveto{\pgfqpoint{5.014016in}{2.257733in}}%
\pgfpathlineto{\pgfqpoint{5.014016in}{2.257733in}}%
\pgfpathlineto{\pgfqpoint{5.014016in}{2.260682in}}%
\pgfpathlineto{\pgfqpoint{5.018557in}{2.260682in}}%
\pgfpathlineto{\pgfqpoint{5.018557in}{2.257733in}}%
\pgfpathmoveto{\pgfqpoint{5.009474in}{2.263632in}}%
\pgfpathlineto{\pgfqpoint{5.009474in}{2.263632in}}%
\pgfpathlineto{\pgfqpoint{5.009474in}{2.266581in}}%
\pgfpathlineto{\pgfqpoint{5.014016in}{2.266581in}}%
\pgfpathlineto{\pgfqpoint{5.014016in}{2.263632in}}%
\pgfpathmoveto{\pgfqpoint{5.023098in}{2.248886in}}%
\pgfpathlineto{\pgfqpoint{5.023098in}{2.248886in}}%
\pgfpathlineto{\pgfqpoint{5.023098in}{2.251835in}}%
\pgfpathlineto{\pgfqpoint{5.027639in}{2.251835in}}%
\pgfpathlineto{\pgfqpoint{5.027639in}{2.248886in}}%
\pgfpathmoveto{\pgfqpoint{5.018557in}{2.251835in}}%
\pgfpathlineto{\pgfqpoint{5.018557in}{2.251835in}}%
\pgfpathlineto{\pgfqpoint{5.018557in}{2.254784in}}%
\pgfpathlineto{\pgfqpoint{5.023098in}{2.254784in}}%
\pgfpathlineto{\pgfqpoint{5.023098in}{2.251835in}}%
\pgfpathmoveto{\pgfqpoint{5.018557in}{2.254784in}}%
\pgfpathlineto{\pgfqpoint{5.018557in}{2.254784in}}%
\pgfpathlineto{\pgfqpoint{5.018557in}{2.257733in}}%
\pgfpathlineto{\pgfqpoint{5.023098in}{2.257733in}}%
\pgfpathlineto{\pgfqpoint{5.023098in}{2.254784in}}%
\pgfpathmoveto{\pgfqpoint{5.023098in}{2.251835in}}%
\pgfpathlineto{\pgfqpoint{5.023098in}{2.251835in}}%
\pgfpathlineto{\pgfqpoint{5.023098in}{2.254784in}}%
\pgfpathlineto{\pgfqpoint{5.027639in}{2.254784in}}%
\pgfpathlineto{\pgfqpoint{5.027639in}{2.251835in}}%
\pgfpathmoveto{\pgfqpoint{5.027639in}{2.245937in}}%
\pgfpathlineto{\pgfqpoint{5.027639in}{2.245937in}}%
\pgfpathlineto{\pgfqpoint{5.027639in}{2.248886in}}%
\pgfpathlineto{\pgfqpoint{5.032180in}{2.248886in}}%
\pgfpathlineto{\pgfqpoint{5.032180in}{2.245937in}}%
\pgfpathmoveto{\pgfqpoint{5.027639in}{2.248886in}}%
\pgfpathlineto{\pgfqpoint{5.027639in}{2.248886in}}%
\pgfpathlineto{\pgfqpoint{5.027639in}{2.251835in}}%
\pgfpathlineto{\pgfqpoint{5.032180in}{2.251835in}}%
\pgfpathlineto{\pgfqpoint{5.032180in}{2.248886in}}%
\pgfpathmoveto{\pgfqpoint{5.032180in}{2.245937in}}%
\pgfpathlineto{\pgfqpoint{5.032180in}{2.245937in}}%
\pgfpathlineto{\pgfqpoint{5.032180in}{2.248886in}}%
\pgfpathlineto{\pgfqpoint{5.036721in}{2.248886in}}%
\pgfpathlineto{\pgfqpoint{5.036721in}{2.245937in}}%
\pgfpathmoveto{\pgfqpoint{5.000392in}{2.269530in}}%
\pgfpathlineto{\pgfqpoint{5.000392in}{2.269530in}}%
\pgfpathlineto{\pgfqpoint{5.000392in}{2.272479in}}%
\pgfpathlineto{\pgfqpoint{5.004933in}{2.272479in}}%
\pgfpathlineto{\pgfqpoint{5.004933in}{2.269530in}}%
\pgfpathmoveto{\pgfqpoint{5.068509in}{2.219394in}}%
\pgfpathlineto{\pgfqpoint{5.068509in}{2.219394in}}%
\pgfpathlineto{\pgfqpoint{5.068509in}{2.222343in}}%
\pgfpathlineto{\pgfqpoint{5.073050in}{2.222343in}}%
\pgfpathlineto{\pgfqpoint{5.073050in}{2.219394in}}%
\pgfpathmoveto{\pgfqpoint{5.050344in}{2.231191in}}%
\pgfpathlineto{\pgfqpoint{5.050344in}{2.231191in}}%
\pgfpathlineto{\pgfqpoint{5.050344in}{2.234140in}}%
\pgfpathlineto{\pgfqpoint{5.054885in}{2.234140in}}%
\pgfpathlineto{\pgfqpoint{5.054885in}{2.231191in}}%
\pgfpathmoveto{\pgfqpoint{5.041262in}{2.237089in}}%
\pgfpathlineto{\pgfqpoint{5.041262in}{2.237089in}}%
\pgfpathlineto{\pgfqpoint{5.041262in}{2.240038in}}%
\pgfpathlineto{\pgfqpoint{5.045803in}{2.240038in}}%
\pgfpathlineto{\pgfqpoint{5.045803in}{2.237089in}}%
\pgfpathmoveto{\pgfqpoint{5.036721in}{2.240038in}}%
\pgfpathlineto{\pgfqpoint{5.036721in}{2.240038in}}%
\pgfpathlineto{\pgfqpoint{5.036721in}{2.242987in}}%
\pgfpathlineto{\pgfqpoint{5.041262in}{2.242987in}}%
\pgfpathlineto{\pgfqpoint{5.041262in}{2.240038in}}%
\pgfpathmoveto{\pgfqpoint{5.036721in}{2.242987in}}%
\pgfpathlineto{\pgfqpoint{5.036721in}{2.242987in}}%
\pgfpathlineto{\pgfqpoint{5.036721in}{2.245937in}}%
\pgfpathlineto{\pgfqpoint{5.041262in}{2.245937in}}%
\pgfpathlineto{\pgfqpoint{5.041262in}{2.242987in}}%
\pgfpathmoveto{\pgfqpoint{5.041262in}{2.240038in}}%
\pgfpathlineto{\pgfqpoint{5.041262in}{2.240038in}}%
\pgfpathlineto{\pgfqpoint{5.041262in}{2.242987in}}%
\pgfpathlineto{\pgfqpoint{5.045803in}{2.242987in}}%
\pgfpathlineto{\pgfqpoint{5.045803in}{2.240038in}}%
\pgfpathmoveto{\pgfqpoint{5.045803in}{2.234140in}}%
\pgfpathlineto{\pgfqpoint{5.045803in}{2.234140in}}%
\pgfpathlineto{\pgfqpoint{5.045803in}{2.237089in}}%
\pgfpathlineto{\pgfqpoint{5.050344in}{2.237089in}}%
\pgfpathlineto{\pgfqpoint{5.050344in}{2.234140in}}%
\pgfpathmoveto{\pgfqpoint{5.045803in}{2.237089in}}%
\pgfpathlineto{\pgfqpoint{5.045803in}{2.237089in}}%
\pgfpathlineto{\pgfqpoint{5.045803in}{2.240038in}}%
\pgfpathlineto{\pgfqpoint{5.050344in}{2.240038in}}%
\pgfpathlineto{\pgfqpoint{5.050344in}{2.237089in}}%
\pgfpathmoveto{\pgfqpoint{5.050344in}{2.234140in}}%
\pgfpathlineto{\pgfqpoint{5.050344in}{2.234140in}}%
\pgfpathlineto{\pgfqpoint{5.050344in}{2.237089in}}%
\pgfpathlineto{\pgfqpoint{5.054885in}{2.237089in}}%
\pgfpathlineto{\pgfqpoint{5.054885in}{2.234140in}}%
\pgfpathmoveto{\pgfqpoint{5.059427in}{2.225292in}}%
\pgfpathlineto{\pgfqpoint{5.059427in}{2.225292in}}%
\pgfpathlineto{\pgfqpoint{5.059427in}{2.228242in}}%
\pgfpathlineto{\pgfqpoint{5.063968in}{2.228242in}}%
\pgfpathlineto{\pgfqpoint{5.063968in}{2.225292in}}%
\pgfpathmoveto{\pgfqpoint{5.054885in}{2.228242in}}%
\pgfpathlineto{\pgfqpoint{5.054885in}{2.228242in}}%
\pgfpathlineto{\pgfqpoint{5.054885in}{2.231191in}}%
\pgfpathlineto{\pgfqpoint{5.059427in}{2.231191in}}%
\pgfpathlineto{\pgfqpoint{5.059427in}{2.228242in}}%
\pgfpathmoveto{\pgfqpoint{5.054885in}{2.231191in}}%
\pgfpathlineto{\pgfqpoint{5.054885in}{2.231191in}}%
\pgfpathlineto{\pgfqpoint{5.054885in}{2.234140in}}%
\pgfpathlineto{\pgfqpoint{5.059427in}{2.234140in}}%
\pgfpathlineto{\pgfqpoint{5.059427in}{2.231191in}}%
\pgfpathmoveto{\pgfqpoint{5.059427in}{2.228242in}}%
\pgfpathlineto{\pgfqpoint{5.059427in}{2.228242in}}%
\pgfpathlineto{\pgfqpoint{5.059427in}{2.231191in}}%
\pgfpathlineto{\pgfqpoint{5.063968in}{2.231191in}}%
\pgfpathlineto{\pgfqpoint{5.063968in}{2.228242in}}%
\pgfpathmoveto{\pgfqpoint{5.063968in}{2.222343in}}%
\pgfpathlineto{\pgfqpoint{5.063968in}{2.222343in}}%
\pgfpathlineto{\pgfqpoint{5.063968in}{2.225292in}}%
\pgfpathlineto{\pgfqpoint{5.068509in}{2.225292in}}%
\pgfpathlineto{\pgfqpoint{5.068509in}{2.222343in}}%
\pgfpathmoveto{\pgfqpoint{5.063968in}{2.225292in}}%
\pgfpathlineto{\pgfqpoint{5.063968in}{2.225292in}}%
\pgfpathlineto{\pgfqpoint{5.063968in}{2.228242in}}%
\pgfpathlineto{\pgfqpoint{5.068509in}{2.228242in}}%
\pgfpathlineto{\pgfqpoint{5.068509in}{2.225292in}}%
\pgfpathmoveto{\pgfqpoint{5.068509in}{2.222343in}}%
\pgfpathlineto{\pgfqpoint{5.068509in}{2.222343in}}%
\pgfpathlineto{\pgfqpoint{5.068509in}{2.225292in}}%
\pgfpathlineto{\pgfqpoint{5.073050in}{2.225292in}}%
\pgfpathlineto{\pgfqpoint{5.073050in}{2.222343in}}%
\pgfpathmoveto{\pgfqpoint{5.086673in}{2.207597in}}%
\pgfpathlineto{\pgfqpoint{5.086673in}{2.207597in}}%
\pgfpathlineto{\pgfqpoint{5.086673in}{2.210547in}}%
\pgfpathlineto{\pgfqpoint{5.091214in}{2.210547in}}%
\pgfpathlineto{\pgfqpoint{5.091214in}{2.207597in}}%
\pgfpathmoveto{\pgfqpoint{5.077591in}{2.213496in}}%
\pgfpathlineto{\pgfqpoint{5.077591in}{2.213496in}}%
\pgfpathlineto{\pgfqpoint{5.077591in}{2.216445in}}%
\pgfpathlineto{\pgfqpoint{5.082132in}{2.216445in}}%
\pgfpathlineto{\pgfqpoint{5.082132in}{2.213496in}}%
\pgfpathmoveto{\pgfqpoint{5.073050in}{2.216445in}}%
\pgfpathlineto{\pgfqpoint{5.073050in}{2.216445in}}%
\pgfpathlineto{\pgfqpoint{5.073050in}{2.219394in}}%
\pgfpathlineto{\pgfqpoint{5.077591in}{2.219394in}}%
\pgfpathlineto{\pgfqpoint{5.077591in}{2.216445in}}%
\pgfpathmoveto{\pgfqpoint{5.073050in}{2.219394in}}%
\pgfpathlineto{\pgfqpoint{5.073050in}{2.219394in}}%
\pgfpathlineto{\pgfqpoint{5.073050in}{2.222343in}}%
\pgfpathlineto{\pgfqpoint{5.077591in}{2.222343in}}%
\pgfpathlineto{\pgfqpoint{5.077591in}{2.219394in}}%
\pgfpathmoveto{\pgfqpoint{5.077591in}{2.216445in}}%
\pgfpathlineto{\pgfqpoint{5.077591in}{2.216445in}}%
\pgfpathlineto{\pgfqpoint{5.077591in}{2.219394in}}%
\pgfpathlineto{\pgfqpoint{5.082132in}{2.219394in}}%
\pgfpathlineto{\pgfqpoint{5.082132in}{2.216445in}}%
\pgfpathmoveto{\pgfqpoint{5.082132in}{2.210547in}}%
\pgfpathlineto{\pgfqpoint{5.082132in}{2.210547in}}%
\pgfpathlineto{\pgfqpoint{5.082132in}{2.213496in}}%
\pgfpathlineto{\pgfqpoint{5.086673in}{2.213496in}}%
\pgfpathlineto{\pgfqpoint{5.086673in}{2.210547in}}%
\pgfpathmoveto{\pgfqpoint{5.082132in}{2.213496in}}%
\pgfpathlineto{\pgfqpoint{5.082132in}{2.213496in}}%
\pgfpathlineto{\pgfqpoint{5.082132in}{2.216445in}}%
\pgfpathlineto{\pgfqpoint{5.086673in}{2.216445in}}%
\pgfpathlineto{\pgfqpoint{5.086673in}{2.213496in}}%
\pgfpathmoveto{\pgfqpoint{5.086673in}{2.210547in}}%
\pgfpathlineto{\pgfqpoint{5.086673in}{2.210547in}}%
\pgfpathlineto{\pgfqpoint{5.086673in}{2.213496in}}%
\pgfpathlineto{\pgfqpoint{5.091214in}{2.213496in}}%
\pgfpathlineto{\pgfqpoint{5.091214in}{2.210547in}}%
\pgfpathmoveto{\pgfqpoint{5.095755in}{2.201699in}}%
\pgfpathlineto{\pgfqpoint{5.095755in}{2.201699in}}%
\pgfpathlineto{\pgfqpoint{5.095755in}{2.204648in}}%
\pgfpathlineto{\pgfqpoint{5.100296in}{2.204648in}}%
\pgfpathlineto{\pgfqpoint{5.100296in}{2.201699in}}%
\pgfpathmoveto{\pgfqpoint{5.091214in}{2.204648in}}%
\pgfpathlineto{\pgfqpoint{5.091214in}{2.204648in}}%
\pgfpathlineto{\pgfqpoint{5.091214in}{2.207597in}}%
\pgfpathlineto{\pgfqpoint{5.095755in}{2.207597in}}%
\pgfpathlineto{\pgfqpoint{5.095755in}{2.204648in}}%
\pgfpathmoveto{\pgfqpoint{5.091214in}{2.207597in}}%
\pgfpathlineto{\pgfqpoint{5.091214in}{2.207597in}}%
\pgfpathlineto{\pgfqpoint{5.091214in}{2.210547in}}%
\pgfpathlineto{\pgfqpoint{5.095755in}{2.210547in}}%
\pgfpathlineto{\pgfqpoint{5.095755in}{2.207597in}}%
\pgfpathmoveto{\pgfqpoint{5.095755in}{2.204648in}}%
\pgfpathlineto{\pgfqpoint{5.095755in}{2.204648in}}%
\pgfpathlineto{\pgfqpoint{5.095755in}{2.207597in}}%
\pgfpathlineto{\pgfqpoint{5.100296in}{2.207597in}}%
\pgfpathlineto{\pgfqpoint{5.100296in}{2.204648in}}%
\pgfpathmoveto{\pgfqpoint{5.100296in}{2.198750in}}%
\pgfpathlineto{\pgfqpoint{5.100296in}{2.198750in}}%
\pgfpathlineto{\pgfqpoint{5.100296in}{2.201699in}}%
\pgfpathlineto{\pgfqpoint{5.104838in}{2.201699in}}%
\pgfpathlineto{\pgfqpoint{5.104838in}{2.198750in}}%
\pgfpathmoveto{\pgfqpoint{5.100296in}{2.201699in}}%
\pgfpathlineto{\pgfqpoint{5.100296in}{2.201699in}}%
\pgfpathlineto{\pgfqpoint{5.100296in}{2.204648in}}%
\pgfpathlineto{\pgfqpoint{5.104838in}{2.204648in}}%
\pgfpathlineto{\pgfqpoint{5.104838in}{2.201699in}}%
\pgfpathmoveto{\pgfqpoint{5.104838in}{2.198750in}}%
\pgfpathlineto{\pgfqpoint{5.104838in}{2.198750in}}%
\pgfpathlineto{\pgfqpoint{5.104838in}{2.201699in}}%
\pgfpathlineto{\pgfqpoint{5.109379in}{2.201699in}}%
\pgfpathlineto{\pgfqpoint{5.109379in}{2.198750in}}%
\pgfpathmoveto{\pgfqpoint{4.964063in}{2.293123in}}%
\pgfpathlineto{\pgfqpoint{4.964063in}{2.293123in}}%
\pgfpathlineto{\pgfqpoint{4.964063in}{2.296073in}}%
\pgfpathlineto{\pgfqpoint{4.968604in}{2.296073in}}%
\pgfpathlineto{\pgfqpoint{4.968604in}{2.293123in}}%
\pgfpathmoveto{\pgfqpoint{5.250144in}{2.101428in}}%
\pgfpathlineto{\pgfqpoint{5.250144in}{2.101428in}}%
\pgfpathlineto{\pgfqpoint{5.250144in}{2.104377in}}%
\pgfpathlineto{\pgfqpoint{5.254685in}{2.104377in}}%
\pgfpathlineto{\pgfqpoint{5.254685in}{2.101428in}}%
\pgfpathmoveto{\pgfqpoint{5.177491in}{2.148614in}}%
\pgfpathlineto{\pgfqpoint{5.177491in}{2.148614in}}%
\pgfpathlineto{\pgfqpoint{5.177491in}{2.151563in}}%
\pgfpathlineto{\pgfqpoint{5.182032in}{2.151563in}}%
\pgfpathlineto{\pgfqpoint{5.182032in}{2.148614in}}%
\pgfpathmoveto{\pgfqpoint{5.141164in}{2.172207in}}%
\pgfpathlineto{\pgfqpoint{5.141164in}{2.172207in}}%
\pgfpathlineto{\pgfqpoint{5.141164in}{2.175157in}}%
\pgfpathlineto{\pgfqpoint{5.145705in}{2.175157in}}%
\pgfpathlineto{\pgfqpoint{5.145705in}{2.172207in}}%
\pgfpathmoveto{\pgfqpoint{5.123001in}{2.184004in}}%
\pgfpathlineto{\pgfqpoint{5.123001in}{2.184004in}}%
\pgfpathlineto{\pgfqpoint{5.123001in}{2.186953in}}%
\pgfpathlineto{\pgfqpoint{5.127542in}{2.186953in}}%
\pgfpathlineto{\pgfqpoint{5.127542in}{2.184004in}}%
\pgfpathmoveto{\pgfqpoint{5.113920in}{2.189902in}}%
\pgfpathlineto{\pgfqpoint{5.113920in}{2.189902in}}%
\pgfpathlineto{\pgfqpoint{5.113920in}{2.192852in}}%
\pgfpathlineto{\pgfqpoint{5.118460in}{2.192852in}}%
\pgfpathlineto{\pgfqpoint{5.118460in}{2.189902in}}%
\pgfpathmoveto{\pgfqpoint{5.109379in}{2.192852in}}%
\pgfpathlineto{\pgfqpoint{5.109379in}{2.192852in}}%
\pgfpathlineto{\pgfqpoint{5.109379in}{2.195801in}}%
\pgfpathlineto{\pgfqpoint{5.113920in}{2.195801in}}%
\pgfpathlineto{\pgfqpoint{5.113920in}{2.192852in}}%
\pgfpathmoveto{\pgfqpoint{5.109379in}{2.195801in}}%
\pgfpathlineto{\pgfqpoint{5.109379in}{2.195801in}}%
\pgfpathlineto{\pgfqpoint{5.109379in}{2.198750in}}%
\pgfpathlineto{\pgfqpoint{5.113920in}{2.198750in}}%
\pgfpathlineto{\pgfqpoint{5.113920in}{2.195801in}}%
\pgfpathmoveto{\pgfqpoint{5.113920in}{2.192852in}}%
\pgfpathlineto{\pgfqpoint{5.113920in}{2.192852in}}%
\pgfpathlineto{\pgfqpoint{5.113920in}{2.195801in}}%
\pgfpathlineto{\pgfqpoint{5.118460in}{2.195801in}}%
\pgfpathlineto{\pgfqpoint{5.118460in}{2.192852in}}%
\pgfpathmoveto{\pgfqpoint{5.118460in}{2.186953in}}%
\pgfpathlineto{\pgfqpoint{5.118460in}{2.186953in}}%
\pgfpathlineto{\pgfqpoint{5.118460in}{2.189902in}}%
\pgfpathlineto{\pgfqpoint{5.123001in}{2.189902in}}%
\pgfpathlineto{\pgfqpoint{5.123001in}{2.186953in}}%
\pgfpathmoveto{\pgfqpoint{5.118460in}{2.189902in}}%
\pgfpathlineto{\pgfqpoint{5.118460in}{2.189902in}}%
\pgfpathlineto{\pgfqpoint{5.118460in}{2.192852in}}%
\pgfpathlineto{\pgfqpoint{5.123001in}{2.192852in}}%
\pgfpathlineto{\pgfqpoint{5.123001in}{2.189902in}}%
\pgfpathmoveto{\pgfqpoint{5.123001in}{2.186953in}}%
\pgfpathlineto{\pgfqpoint{5.123001in}{2.186953in}}%
\pgfpathlineto{\pgfqpoint{5.123001in}{2.189902in}}%
\pgfpathlineto{\pgfqpoint{5.127542in}{2.189902in}}%
\pgfpathlineto{\pgfqpoint{5.127542in}{2.186953in}}%
\pgfpathmoveto{\pgfqpoint{5.132083in}{2.178106in}}%
\pgfpathlineto{\pgfqpoint{5.132083in}{2.178106in}}%
\pgfpathlineto{\pgfqpoint{5.132083in}{2.181055in}}%
\pgfpathlineto{\pgfqpoint{5.136624in}{2.181055in}}%
\pgfpathlineto{\pgfqpoint{5.136624in}{2.178106in}}%
\pgfpathmoveto{\pgfqpoint{5.127542in}{2.181055in}}%
\pgfpathlineto{\pgfqpoint{5.127542in}{2.181055in}}%
\pgfpathlineto{\pgfqpoint{5.127542in}{2.184004in}}%
\pgfpathlineto{\pgfqpoint{5.132083in}{2.184004in}}%
\pgfpathlineto{\pgfqpoint{5.132083in}{2.181055in}}%
\pgfpathmoveto{\pgfqpoint{5.127542in}{2.184004in}}%
\pgfpathlineto{\pgfqpoint{5.127542in}{2.184004in}}%
\pgfpathlineto{\pgfqpoint{5.127542in}{2.186953in}}%
\pgfpathlineto{\pgfqpoint{5.132083in}{2.186953in}}%
\pgfpathlineto{\pgfqpoint{5.132083in}{2.184004in}}%
\pgfpathmoveto{\pgfqpoint{5.132083in}{2.181055in}}%
\pgfpathlineto{\pgfqpoint{5.132083in}{2.181055in}}%
\pgfpathlineto{\pgfqpoint{5.132083in}{2.184004in}}%
\pgfpathlineto{\pgfqpoint{5.136624in}{2.184004in}}%
\pgfpathlineto{\pgfqpoint{5.136624in}{2.181055in}}%
\pgfpathmoveto{\pgfqpoint{5.136624in}{2.175157in}}%
\pgfpathlineto{\pgfqpoint{5.136624in}{2.175157in}}%
\pgfpathlineto{\pgfqpoint{5.136624in}{2.178106in}}%
\pgfpathlineto{\pgfqpoint{5.141164in}{2.178106in}}%
\pgfpathlineto{\pgfqpoint{5.141164in}{2.175157in}}%
\pgfpathmoveto{\pgfqpoint{5.136624in}{2.178106in}}%
\pgfpathlineto{\pgfqpoint{5.136624in}{2.178106in}}%
\pgfpathlineto{\pgfqpoint{5.136624in}{2.181055in}}%
\pgfpathlineto{\pgfqpoint{5.141164in}{2.181055in}}%
\pgfpathlineto{\pgfqpoint{5.141164in}{2.178106in}}%
\pgfpathmoveto{\pgfqpoint{5.141164in}{2.175157in}}%
\pgfpathlineto{\pgfqpoint{5.141164in}{2.175157in}}%
\pgfpathlineto{\pgfqpoint{5.141164in}{2.178106in}}%
\pgfpathlineto{\pgfqpoint{5.145705in}{2.178106in}}%
\pgfpathlineto{\pgfqpoint{5.145705in}{2.175157in}}%
\pgfpathmoveto{\pgfqpoint{5.159328in}{2.160411in}}%
\pgfpathlineto{\pgfqpoint{5.159328in}{2.160411in}}%
\pgfpathlineto{\pgfqpoint{5.159328in}{2.163360in}}%
\pgfpathlineto{\pgfqpoint{5.163869in}{2.163360in}}%
\pgfpathlineto{\pgfqpoint{5.163869in}{2.160411in}}%
\pgfpathmoveto{\pgfqpoint{5.150246in}{2.166309in}}%
\pgfpathlineto{\pgfqpoint{5.150246in}{2.166309in}}%
\pgfpathlineto{\pgfqpoint{5.150246in}{2.169258in}}%
\pgfpathlineto{\pgfqpoint{5.154787in}{2.169258in}}%
\pgfpathlineto{\pgfqpoint{5.154787in}{2.166309in}}%
\pgfpathmoveto{\pgfqpoint{5.145705in}{2.169258in}}%
\pgfpathlineto{\pgfqpoint{5.145705in}{2.169258in}}%
\pgfpathlineto{\pgfqpoint{5.145705in}{2.172207in}}%
\pgfpathlineto{\pgfqpoint{5.150246in}{2.172207in}}%
\pgfpathlineto{\pgfqpoint{5.150246in}{2.169258in}}%
\pgfpathmoveto{\pgfqpoint{5.145705in}{2.172207in}}%
\pgfpathlineto{\pgfqpoint{5.145705in}{2.172207in}}%
\pgfpathlineto{\pgfqpoint{5.145705in}{2.175157in}}%
\pgfpathlineto{\pgfqpoint{5.150246in}{2.175157in}}%
\pgfpathlineto{\pgfqpoint{5.150246in}{2.172207in}}%
\pgfpathmoveto{\pgfqpoint{5.150246in}{2.169258in}}%
\pgfpathlineto{\pgfqpoint{5.150246in}{2.169258in}}%
\pgfpathlineto{\pgfqpoint{5.150246in}{2.172207in}}%
\pgfpathlineto{\pgfqpoint{5.154787in}{2.172207in}}%
\pgfpathlineto{\pgfqpoint{5.154787in}{2.169258in}}%
\pgfpathmoveto{\pgfqpoint{5.154787in}{2.163360in}}%
\pgfpathlineto{\pgfqpoint{5.154787in}{2.163360in}}%
\pgfpathlineto{\pgfqpoint{5.154787in}{2.166309in}}%
\pgfpathlineto{\pgfqpoint{5.159328in}{2.166309in}}%
\pgfpathlineto{\pgfqpoint{5.159328in}{2.163360in}}%
\pgfpathmoveto{\pgfqpoint{5.154787in}{2.166309in}}%
\pgfpathlineto{\pgfqpoint{5.154787in}{2.166309in}}%
\pgfpathlineto{\pgfqpoint{5.154787in}{2.169258in}}%
\pgfpathlineto{\pgfqpoint{5.159328in}{2.169258in}}%
\pgfpathlineto{\pgfqpoint{5.159328in}{2.166309in}}%
\pgfpathmoveto{\pgfqpoint{5.159328in}{2.163360in}}%
\pgfpathlineto{\pgfqpoint{5.159328in}{2.163360in}}%
\pgfpathlineto{\pgfqpoint{5.159328in}{2.166309in}}%
\pgfpathlineto{\pgfqpoint{5.163869in}{2.166309in}}%
\pgfpathlineto{\pgfqpoint{5.163869in}{2.163360in}}%
\pgfpathmoveto{\pgfqpoint{5.168409in}{2.154513in}}%
\pgfpathlineto{\pgfqpoint{5.168409in}{2.154513in}}%
\pgfpathlineto{\pgfqpoint{5.168409in}{2.157462in}}%
\pgfpathlineto{\pgfqpoint{5.172950in}{2.157462in}}%
\pgfpathlineto{\pgfqpoint{5.172950in}{2.154513in}}%
\pgfpathmoveto{\pgfqpoint{5.163869in}{2.157462in}}%
\pgfpathlineto{\pgfqpoint{5.163869in}{2.157462in}}%
\pgfpathlineto{\pgfqpoint{5.163869in}{2.160411in}}%
\pgfpathlineto{\pgfqpoint{5.168409in}{2.160411in}}%
\pgfpathlineto{\pgfqpoint{5.168409in}{2.157462in}}%
\pgfpathmoveto{\pgfqpoint{5.163869in}{2.160411in}}%
\pgfpathlineto{\pgfqpoint{5.163869in}{2.160411in}}%
\pgfpathlineto{\pgfqpoint{5.163869in}{2.163360in}}%
\pgfpathlineto{\pgfqpoint{5.168409in}{2.163360in}}%
\pgfpathlineto{\pgfqpoint{5.168409in}{2.160411in}}%
\pgfpathmoveto{\pgfqpoint{5.168409in}{2.157462in}}%
\pgfpathlineto{\pgfqpoint{5.168409in}{2.157462in}}%
\pgfpathlineto{\pgfqpoint{5.168409in}{2.160411in}}%
\pgfpathlineto{\pgfqpoint{5.172950in}{2.160411in}}%
\pgfpathlineto{\pgfqpoint{5.172950in}{2.157462in}}%
\pgfpathmoveto{\pgfqpoint{5.172950in}{2.151563in}}%
\pgfpathlineto{\pgfqpoint{5.172950in}{2.151563in}}%
\pgfpathlineto{\pgfqpoint{5.172950in}{2.154513in}}%
\pgfpathlineto{\pgfqpoint{5.177491in}{2.154513in}}%
\pgfpathlineto{\pgfqpoint{5.177491in}{2.151563in}}%
\pgfpathmoveto{\pgfqpoint{5.172950in}{2.154513in}}%
\pgfpathlineto{\pgfqpoint{5.172950in}{2.154513in}}%
\pgfpathlineto{\pgfqpoint{5.172950in}{2.157462in}}%
\pgfpathlineto{\pgfqpoint{5.177491in}{2.157462in}}%
\pgfpathlineto{\pgfqpoint{5.177491in}{2.154513in}}%
\pgfpathmoveto{\pgfqpoint{5.177491in}{2.151563in}}%
\pgfpathlineto{\pgfqpoint{5.177491in}{2.151563in}}%
\pgfpathlineto{\pgfqpoint{5.177491in}{2.154513in}}%
\pgfpathlineto{\pgfqpoint{5.182032in}{2.154513in}}%
\pgfpathlineto{\pgfqpoint{5.182032in}{2.151563in}}%
\pgfpathmoveto{\pgfqpoint{5.213818in}{2.125021in}}%
\pgfpathlineto{\pgfqpoint{5.213818in}{2.125021in}}%
\pgfpathlineto{\pgfqpoint{5.213818in}{2.127970in}}%
\pgfpathlineto{\pgfqpoint{5.218358in}{2.127970in}}%
\pgfpathlineto{\pgfqpoint{5.218358in}{2.125021in}}%
\pgfpathmoveto{\pgfqpoint{5.195654in}{2.136818in}}%
\pgfpathlineto{\pgfqpoint{5.195654in}{2.136818in}}%
\pgfpathlineto{\pgfqpoint{5.195654in}{2.139767in}}%
\pgfpathlineto{\pgfqpoint{5.200195in}{2.139767in}}%
\pgfpathlineto{\pgfqpoint{5.200195in}{2.136818in}}%
\pgfpathmoveto{\pgfqpoint{5.186573in}{2.142716in}}%
\pgfpathlineto{\pgfqpoint{5.186573in}{2.142716in}}%
\pgfpathlineto{\pgfqpoint{5.186573in}{2.145665in}}%
\pgfpathlineto{\pgfqpoint{5.191113in}{2.145665in}}%
\pgfpathlineto{\pgfqpoint{5.191113in}{2.142716in}}%
\pgfpathmoveto{\pgfqpoint{5.182032in}{2.145665in}}%
\pgfpathlineto{\pgfqpoint{5.182032in}{2.145665in}}%
\pgfpathlineto{\pgfqpoint{5.182032in}{2.148614in}}%
\pgfpathlineto{\pgfqpoint{5.186573in}{2.148614in}}%
\pgfpathlineto{\pgfqpoint{5.186573in}{2.145665in}}%
\pgfpathmoveto{\pgfqpoint{5.182032in}{2.148614in}}%
\pgfpathlineto{\pgfqpoint{5.182032in}{2.148614in}}%
\pgfpathlineto{\pgfqpoint{5.182032in}{2.151563in}}%
\pgfpathlineto{\pgfqpoint{5.186573in}{2.151563in}}%
\pgfpathlineto{\pgfqpoint{5.186573in}{2.148614in}}%
\pgfpathmoveto{\pgfqpoint{5.186573in}{2.145665in}}%
\pgfpathlineto{\pgfqpoint{5.186573in}{2.145665in}}%
\pgfpathlineto{\pgfqpoint{5.186573in}{2.148614in}}%
\pgfpathlineto{\pgfqpoint{5.191113in}{2.148614in}}%
\pgfpathlineto{\pgfqpoint{5.191113in}{2.145665in}}%
\pgfpathmoveto{\pgfqpoint{5.191113in}{2.139767in}}%
\pgfpathlineto{\pgfqpoint{5.191113in}{2.139767in}}%
\pgfpathlineto{\pgfqpoint{5.191113in}{2.142716in}}%
\pgfpathlineto{\pgfqpoint{5.195654in}{2.142716in}}%
\pgfpathlineto{\pgfqpoint{5.195654in}{2.139767in}}%
\pgfpathmoveto{\pgfqpoint{5.191113in}{2.142716in}}%
\pgfpathlineto{\pgfqpoint{5.191113in}{2.142716in}}%
\pgfpathlineto{\pgfqpoint{5.191113in}{2.145665in}}%
\pgfpathlineto{\pgfqpoint{5.195654in}{2.145665in}}%
\pgfpathlineto{\pgfqpoint{5.195654in}{2.142716in}}%
\pgfpathmoveto{\pgfqpoint{5.195654in}{2.139767in}}%
\pgfpathlineto{\pgfqpoint{5.195654in}{2.139767in}}%
\pgfpathlineto{\pgfqpoint{5.195654in}{2.142716in}}%
\pgfpathlineto{\pgfqpoint{5.200195in}{2.142716in}}%
\pgfpathlineto{\pgfqpoint{5.200195in}{2.139767in}}%
\pgfpathmoveto{\pgfqpoint{5.204736in}{2.130919in}}%
\pgfpathlineto{\pgfqpoint{5.204736in}{2.130919in}}%
\pgfpathlineto{\pgfqpoint{5.204736in}{2.133868in}}%
\pgfpathlineto{\pgfqpoint{5.209277in}{2.133868in}}%
\pgfpathlineto{\pgfqpoint{5.209277in}{2.130919in}}%
\pgfpathmoveto{\pgfqpoint{5.200195in}{2.133868in}}%
\pgfpathlineto{\pgfqpoint{5.200195in}{2.133868in}}%
\pgfpathlineto{\pgfqpoint{5.200195in}{2.136818in}}%
\pgfpathlineto{\pgfqpoint{5.204736in}{2.136818in}}%
\pgfpathlineto{\pgfqpoint{5.204736in}{2.133868in}}%
\pgfpathmoveto{\pgfqpoint{5.200195in}{2.136818in}}%
\pgfpathlineto{\pgfqpoint{5.200195in}{2.136818in}}%
\pgfpathlineto{\pgfqpoint{5.200195in}{2.139767in}}%
\pgfpathlineto{\pgfqpoint{5.204736in}{2.139767in}}%
\pgfpathlineto{\pgfqpoint{5.204736in}{2.136818in}}%
\pgfpathmoveto{\pgfqpoint{5.204736in}{2.133868in}}%
\pgfpathlineto{\pgfqpoint{5.204736in}{2.133868in}}%
\pgfpathlineto{\pgfqpoint{5.204736in}{2.136818in}}%
\pgfpathlineto{\pgfqpoint{5.209277in}{2.136818in}}%
\pgfpathlineto{\pgfqpoint{5.209277in}{2.133868in}}%
\pgfpathmoveto{\pgfqpoint{5.209277in}{2.127970in}}%
\pgfpathlineto{\pgfqpoint{5.209277in}{2.127970in}}%
\pgfpathlineto{\pgfqpoint{5.209277in}{2.130919in}}%
\pgfpathlineto{\pgfqpoint{5.213818in}{2.130919in}}%
\pgfpathlineto{\pgfqpoint{5.213818in}{2.127970in}}%
\pgfpathmoveto{\pgfqpoint{5.209277in}{2.130919in}}%
\pgfpathlineto{\pgfqpoint{5.209277in}{2.130919in}}%
\pgfpathlineto{\pgfqpoint{5.209277in}{2.133868in}}%
\pgfpathlineto{\pgfqpoint{5.213818in}{2.133868in}}%
\pgfpathlineto{\pgfqpoint{5.213818in}{2.130919in}}%
\pgfpathmoveto{\pgfqpoint{5.213818in}{2.127970in}}%
\pgfpathlineto{\pgfqpoint{5.213818in}{2.127970in}}%
\pgfpathlineto{\pgfqpoint{5.213818in}{2.130919in}}%
\pgfpathlineto{\pgfqpoint{5.218358in}{2.130919in}}%
\pgfpathlineto{\pgfqpoint{5.218358in}{2.127970in}}%
\pgfpathmoveto{\pgfqpoint{5.231981in}{2.113224in}}%
\pgfpathlineto{\pgfqpoint{5.231981in}{2.113224in}}%
\pgfpathlineto{\pgfqpoint{5.231981in}{2.116173in}}%
\pgfpathlineto{\pgfqpoint{5.236522in}{2.116173in}}%
\pgfpathlineto{\pgfqpoint{5.236522in}{2.113224in}}%
\pgfpathmoveto{\pgfqpoint{5.222899in}{2.119123in}}%
\pgfpathlineto{\pgfqpoint{5.222899in}{2.119123in}}%
\pgfpathlineto{\pgfqpoint{5.222899in}{2.122072in}}%
\pgfpathlineto{\pgfqpoint{5.227440in}{2.122072in}}%
\pgfpathlineto{\pgfqpoint{5.227440in}{2.119123in}}%
\pgfpathmoveto{\pgfqpoint{5.218358in}{2.122072in}}%
\pgfpathlineto{\pgfqpoint{5.218358in}{2.122072in}}%
\pgfpathlineto{\pgfqpoint{5.218358in}{2.125021in}}%
\pgfpathlineto{\pgfqpoint{5.222899in}{2.125021in}}%
\pgfpathlineto{\pgfqpoint{5.222899in}{2.122072in}}%
\pgfpathmoveto{\pgfqpoint{5.218358in}{2.125021in}}%
\pgfpathlineto{\pgfqpoint{5.218358in}{2.125021in}}%
\pgfpathlineto{\pgfqpoint{5.218358in}{2.127970in}}%
\pgfpathlineto{\pgfqpoint{5.222899in}{2.127970in}}%
\pgfpathlineto{\pgfqpoint{5.222899in}{2.125021in}}%
\pgfpathmoveto{\pgfqpoint{5.222899in}{2.122072in}}%
\pgfpathlineto{\pgfqpoint{5.222899in}{2.122072in}}%
\pgfpathlineto{\pgfqpoint{5.222899in}{2.125021in}}%
\pgfpathlineto{\pgfqpoint{5.227440in}{2.125021in}}%
\pgfpathlineto{\pgfqpoint{5.227440in}{2.122072in}}%
\pgfpathmoveto{\pgfqpoint{5.227440in}{2.116173in}}%
\pgfpathlineto{\pgfqpoint{5.227440in}{2.116173in}}%
\pgfpathlineto{\pgfqpoint{5.227440in}{2.119123in}}%
\pgfpathlineto{\pgfqpoint{5.231981in}{2.119123in}}%
\pgfpathlineto{\pgfqpoint{5.231981in}{2.116173in}}%
\pgfpathmoveto{\pgfqpoint{5.227440in}{2.119123in}}%
\pgfpathlineto{\pgfqpoint{5.227440in}{2.119123in}}%
\pgfpathlineto{\pgfqpoint{5.227440in}{2.122072in}}%
\pgfpathlineto{\pgfqpoint{5.231981in}{2.122072in}}%
\pgfpathlineto{\pgfqpoint{5.231981in}{2.119123in}}%
\pgfpathmoveto{\pgfqpoint{5.231981in}{2.116173in}}%
\pgfpathlineto{\pgfqpoint{5.231981in}{2.116173in}}%
\pgfpathlineto{\pgfqpoint{5.231981in}{2.119123in}}%
\pgfpathlineto{\pgfqpoint{5.236522in}{2.119123in}}%
\pgfpathlineto{\pgfqpoint{5.236522in}{2.116173in}}%
\pgfpathmoveto{\pgfqpoint{5.241063in}{2.107326in}}%
\pgfpathlineto{\pgfqpoint{5.241063in}{2.107326in}}%
\pgfpathlineto{\pgfqpoint{5.241063in}{2.110275in}}%
\pgfpathlineto{\pgfqpoint{5.245603in}{2.110275in}}%
\pgfpathlineto{\pgfqpoint{5.245603in}{2.107326in}}%
\pgfpathmoveto{\pgfqpoint{5.236522in}{2.110275in}}%
\pgfpathlineto{\pgfqpoint{5.236522in}{2.110275in}}%
\pgfpathlineto{\pgfqpoint{5.236522in}{2.113224in}}%
\pgfpathlineto{\pgfqpoint{5.241063in}{2.113224in}}%
\pgfpathlineto{\pgfqpoint{5.241063in}{2.110275in}}%
\pgfpathmoveto{\pgfqpoint{5.236522in}{2.113224in}}%
\pgfpathlineto{\pgfqpoint{5.236522in}{2.113224in}}%
\pgfpathlineto{\pgfqpoint{5.236522in}{2.116173in}}%
\pgfpathlineto{\pgfqpoint{5.241063in}{2.116173in}}%
\pgfpathlineto{\pgfqpoint{5.241063in}{2.113224in}}%
\pgfpathmoveto{\pgfqpoint{5.241063in}{2.110275in}}%
\pgfpathlineto{\pgfqpoint{5.241063in}{2.110275in}}%
\pgfpathlineto{\pgfqpoint{5.241063in}{2.113224in}}%
\pgfpathlineto{\pgfqpoint{5.245603in}{2.113224in}}%
\pgfpathlineto{\pgfqpoint{5.245603in}{2.110275in}}%
\pgfpathmoveto{\pgfqpoint{5.245603in}{2.104377in}}%
\pgfpathlineto{\pgfqpoint{5.245603in}{2.104377in}}%
\pgfpathlineto{\pgfqpoint{5.245603in}{2.107326in}}%
\pgfpathlineto{\pgfqpoint{5.250144in}{2.107326in}}%
\pgfpathlineto{\pgfqpoint{5.250144in}{2.104377in}}%
\pgfpathmoveto{\pgfqpoint{5.245603in}{2.107326in}}%
\pgfpathlineto{\pgfqpoint{5.245603in}{2.107326in}}%
\pgfpathlineto{\pgfqpoint{5.245603in}{2.110275in}}%
\pgfpathlineto{\pgfqpoint{5.250144in}{2.110275in}}%
\pgfpathlineto{\pgfqpoint{5.250144in}{2.107326in}}%
\pgfpathmoveto{\pgfqpoint{5.250144in}{2.104377in}}%
\pgfpathlineto{\pgfqpoint{5.250144in}{2.104377in}}%
\pgfpathlineto{\pgfqpoint{5.250144in}{2.107326in}}%
\pgfpathlineto{\pgfqpoint{5.254685in}{2.107326in}}%
\pgfpathlineto{\pgfqpoint{5.254685in}{2.104377in}}%
\pgfpathmoveto{\pgfqpoint{5.268308in}{2.089631in}}%
\pgfpathlineto{\pgfqpoint{5.268308in}{2.089631in}}%
\pgfpathlineto{\pgfqpoint{5.268308in}{2.092580in}}%
\pgfpathlineto{\pgfqpoint{5.272849in}{2.092580in}}%
\pgfpathlineto{\pgfqpoint{5.272849in}{2.089631in}}%
\pgfpathmoveto{\pgfqpoint{5.259226in}{2.095529in}}%
\pgfpathlineto{\pgfqpoint{5.259226in}{2.095529in}}%
\pgfpathlineto{\pgfqpoint{5.259226in}{2.098478in}}%
\pgfpathlineto{\pgfqpoint{5.263767in}{2.098478in}}%
\pgfpathlineto{\pgfqpoint{5.263767in}{2.095529in}}%
\pgfpathmoveto{\pgfqpoint{5.254685in}{2.098478in}}%
\pgfpathlineto{\pgfqpoint{5.254685in}{2.098478in}}%
\pgfpathlineto{\pgfqpoint{5.254685in}{2.101428in}}%
\pgfpathlineto{\pgfqpoint{5.259226in}{2.101428in}}%
\pgfpathlineto{\pgfqpoint{5.259226in}{2.098478in}}%
\pgfpathmoveto{\pgfqpoint{5.254685in}{2.101428in}}%
\pgfpathlineto{\pgfqpoint{5.254685in}{2.101428in}}%
\pgfpathlineto{\pgfqpoint{5.254685in}{2.104377in}}%
\pgfpathlineto{\pgfqpoint{5.259226in}{2.104377in}}%
\pgfpathlineto{\pgfqpoint{5.259226in}{2.101428in}}%
\pgfpathmoveto{\pgfqpoint{5.259226in}{2.098478in}}%
\pgfpathlineto{\pgfqpoint{5.259226in}{2.098478in}}%
\pgfpathlineto{\pgfqpoint{5.259226in}{2.101428in}}%
\pgfpathlineto{\pgfqpoint{5.263767in}{2.101428in}}%
\pgfpathlineto{\pgfqpoint{5.263767in}{2.098478in}}%
\pgfpathmoveto{\pgfqpoint{5.263767in}{2.092580in}}%
\pgfpathlineto{\pgfqpoint{5.263767in}{2.092580in}}%
\pgfpathlineto{\pgfqpoint{5.263767in}{2.095529in}}%
\pgfpathlineto{\pgfqpoint{5.268308in}{2.095529in}}%
\pgfpathlineto{\pgfqpoint{5.268308in}{2.092580in}}%
\pgfpathmoveto{\pgfqpoint{5.263767in}{2.095529in}}%
\pgfpathlineto{\pgfqpoint{5.263767in}{2.095529in}}%
\pgfpathlineto{\pgfqpoint{5.263767in}{2.098478in}}%
\pgfpathlineto{\pgfqpoint{5.268308in}{2.098478in}}%
\pgfpathlineto{\pgfqpoint{5.268308in}{2.095529in}}%
\pgfpathmoveto{\pgfqpoint{5.268308in}{2.092580in}}%
\pgfpathlineto{\pgfqpoint{5.268308in}{2.092580in}}%
\pgfpathlineto{\pgfqpoint{5.268308in}{2.095529in}}%
\pgfpathlineto{\pgfqpoint{5.272849in}{2.095529in}}%
\pgfpathlineto{\pgfqpoint{5.272849in}{2.092580in}}%
\pgfpathmoveto{\pgfqpoint{5.272849in}{2.086681in}}%
\pgfpathlineto{\pgfqpoint{5.272849in}{2.086681in}}%
\pgfpathlineto{\pgfqpoint{5.272849in}{2.089631in}}%
\pgfpathlineto{\pgfqpoint{5.277390in}{2.089631in}}%
\pgfpathlineto{\pgfqpoint{5.277390in}{2.086681in}}%
\pgfpathmoveto{\pgfqpoint{5.272849in}{2.089631in}}%
\pgfpathlineto{\pgfqpoint{5.272849in}{2.089631in}}%
\pgfpathlineto{\pgfqpoint{5.272849in}{2.092580in}}%
\pgfpathlineto{\pgfqpoint{5.277390in}{2.092580in}}%
\pgfpathlineto{\pgfqpoint{5.277390in}{2.089631in}}%
\pgfpathmoveto{\pgfqpoint{5.277390in}{2.086681in}}%
\pgfpathlineto{\pgfqpoint{5.277390in}{2.086681in}}%
\pgfpathlineto{\pgfqpoint{5.277390in}{2.089631in}}%
\pgfpathlineto{\pgfqpoint{5.281931in}{2.089631in}}%
\pgfpathlineto{\pgfqpoint{5.281931in}{2.086681in}}%
\pgfpathmoveto{\pgfqpoint{5.281931in}{2.083732in}}%
\pgfpathlineto{\pgfqpoint{5.281931in}{2.083732in}}%
\pgfpathlineto{\pgfqpoint{5.281931in}{2.086681in}}%
\pgfpathlineto{\pgfqpoint{5.286472in}{2.086681in}}%
\pgfpathlineto{\pgfqpoint{5.286472in}{2.083732in}}%
\pgfpathmoveto{\pgfqpoint{5.286472in}{2.080783in}}%
\pgfpathlineto{\pgfqpoint{5.286472in}{2.080783in}}%
\pgfpathlineto{\pgfqpoint{5.286472in}{2.083732in}}%
\pgfpathlineto{\pgfqpoint{5.291013in}{2.083732in}}%
\pgfpathlineto{\pgfqpoint{5.291013in}{2.080783in}}%
\pgfpathmoveto{\pgfqpoint{5.286472in}{2.083732in}}%
\pgfpathlineto{\pgfqpoint{5.286472in}{2.083732in}}%
\pgfpathlineto{\pgfqpoint{5.286472in}{2.086681in}}%
\pgfpathlineto{\pgfqpoint{5.291013in}{2.086681in}}%
\pgfpathlineto{\pgfqpoint{5.291013in}{2.083732in}}%
\pgfpathmoveto{\pgfqpoint{5.281931in}{2.086681in}}%
\pgfpathlineto{\pgfqpoint{5.281931in}{2.086681in}}%
\pgfpathlineto{\pgfqpoint{5.281931in}{2.089631in}}%
\pgfpathlineto{\pgfqpoint{5.286472in}{2.089631in}}%
\pgfpathlineto{\pgfqpoint{5.286472in}{2.086681in}}%
\pgfpathmoveto{\pgfqpoint{5.291013in}{2.077834in}}%
\pgfpathlineto{\pgfqpoint{5.291013in}{2.077834in}}%
\pgfpathlineto{\pgfqpoint{5.291013in}{2.080783in}}%
\pgfpathlineto{\pgfqpoint{5.295554in}{2.080783in}}%
\pgfpathlineto{\pgfqpoint{5.295554in}{2.077834in}}%
\pgfpathmoveto{\pgfqpoint{5.295554in}{2.074884in}}%
\pgfpathlineto{\pgfqpoint{5.295554in}{2.074884in}}%
\pgfpathlineto{\pgfqpoint{5.295554in}{2.077834in}}%
\pgfpathlineto{\pgfqpoint{5.300095in}{2.077834in}}%
\pgfpathlineto{\pgfqpoint{5.300095in}{2.074884in}}%
\pgfpathmoveto{\pgfqpoint{5.295554in}{2.077834in}}%
\pgfpathlineto{\pgfqpoint{5.295554in}{2.077834in}}%
\pgfpathlineto{\pgfqpoint{5.295554in}{2.080783in}}%
\pgfpathlineto{\pgfqpoint{5.300095in}{2.080783in}}%
\pgfpathlineto{\pgfqpoint{5.300095in}{2.077834in}}%
\pgfpathmoveto{\pgfqpoint{5.300095in}{2.071935in}}%
\pgfpathlineto{\pgfqpoint{5.300095in}{2.071935in}}%
\pgfpathlineto{\pgfqpoint{5.300095in}{2.074884in}}%
\pgfpathlineto{\pgfqpoint{5.304636in}{2.074884in}}%
\pgfpathlineto{\pgfqpoint{5.304636in}{2.071935in}}%
\pgfpathmoveto{\pgfqpoint{5.304636in}{2.068986in}}%
\pgfpathlineto{\pgfqpoint{5.304636in}{2.068986in}}%
\pgfpathlineto{\pgfqpoint{5.304636in}{2.071935in}}%
\pgfpathlineto{\pgfqpoint{5.309177in}{2.071935in}}%
\pgfpathlineto{\pgfqpoint{5.309177in}{2.068986in}}%
\pgfpathmoveto{\pgfqpoint{5.304636in}{2.071935in}}%
\pgfpathlineto{\pgfqpoint{5.304636in}{2.071935in}}%
\pgfpathlineto{\pgfqpoint{5.304636in}{2.074884in}}%
\pgfpathlineto{\pgfqpoint{5.309177in}{2.074884in}}%
\pgfpathlineto{\pgfqpoint{5.309177in}{2.071935in}}%
\pgfpathmoveto{\pgfqpoint{5.300095in}{2.074884in}}%
\pgfpathlineto{\pgfqpoint{5.300095in}{2.074884in}}%
\pgfpathlineto{\pgfqpoint{5.300095in}{2.077834in}}%
\pgfpathlineto{\pgfqpoint{5.304636in}{2.077834in}}%
\pgfpathlineto{\pgfqpoint{5.304636in}{2.074884in}}%
\pgfpathmoveto{\pgfqpoint{5.309177in}{2.066036in}}%
\pgfpathlineto{\pgfqpoint{5.309177in}{2.066036in}}%
\pgfpathlineto{\pgfqpoint{5.309177in}{2.068986in}}%
\pgfpathlineto{\pgfqpoint{5.313718in}{2.068986in}}%
\pgfpathlineto{\pgfqpoint{5.313718in}{2.066036in}}%
\pgfpathmoveto{\pgfqpoint{5.313718in}{2.063087in}}%
\pgfpathlineto{\pgfqpoint{5.313718in}{2.063087in}}%
\pgfpathlineto{\pgfqpoint{5.313718in}{2.066036in}}%
\pgfpathlineto{\pgfqpoint{5.318259in}{2.066036in}}%
\pgfpathlineto{\pgfqpoint{5.318259in}{2.063087in}}%
\pgfpathmoveto{\pgfqpoint{5.313718in}{2.066036in}}%
\pgfpathlineto{\pgfqpoint{5.313718in}{2.066036in}}%
\pgfpathlineto{\pgfqpoint{5.313718in}{2.068986in}}%
\pgfpathlineto{\pgfqpoint{5.318259in}{2.068986in}}%
\pgfpathlineto{\pgfqpoint{5.318259in}{2.066036in}}%
\pgfpathmoveto{\pgfqpoint{5.318259in}{2.060138in}}%
\pgfpathlineto{\pgfqpoint{5.318259in}{2.060138in}}%
\pgfpathlineto{\pgfqpoint{5.318259in}{2.063087in}}%
\pgfpathlineto{\pgfqpoint{5.322800in}{2.063087in}}%
\pgfpathlineto{\pgfqpoint{5.322800in}{2.060138in}}%
\pgfpathmoveto{\pgfqpoint{5.322800in}{2.057189in}}%
\pgfpathlineto{\pgfqpoint{5.322800in}{2.057189in}}%
\pgfpathlineto{\pgfqpoint{5.322800in}{2.060138in}}%
\pgfpathlineto{\pgfqpoint{5.327341in}{2.060138in}}%
\pgfpathlineto{\pgfqpoint{5.327341in}{2.057189in}}%
\pgfpathmoveto{\pgfqpoint{5.322800in}{2.060138in}}%
\pgfpathlineto{\pgfqpoint{5.322800in}{2.060138in}}%
\pgfpathlineto{\pgfqpoint{5.322800in}{2.063087in}}%
\pgfpathlineto{\pgfqpoint{5.327341in}{2.063087in}}%
\pgfpathlineto{\pgfqpoint{5.327341in}{2.060138in}}%
\pgfpathmoveto{\pgfqpoint{5.318259in}{2.063087in}}%
\pgfpathlineto{\pgfqpoint{5.318259in}{2.063087in}}%
\pgfpathlineto{\pgfqpoint{5.318259in}{2.066036in}}%
\pgfpathlineto{\pgfqpoint{5.322800in}{2.066036in}}%
\pgfpathlineto{\pgfqpoint{5.322800in}{2.063087in}}%
\pgfpathmoveto{\pgfqpoint{5.309177in}{2.068986in}}%
\pgfpathlineto{\pgfqpoint{5.309177in}{2.068986in}}%
\pgfpathlineto{\pgfqpoint{5.309177in}{2.071935in}}%
\pgfpathlineto{\pgfqpoint{5.313718in}{2.071935in}}%
\pgfpathlineto{\pgfqpoint{5.313718in}{2.068986in}}%
\pgfpathmoveto{\pgfqpoint{5.291013in}{2.080783in}}%
\pgfpathlineto{\pgfqpoint{5.291013in}{2.080783in}}%
\pgfpathlineto{\pgfqpoint{5.291013in}{2.083732in}}%
\pgfpathlineto{\pgfqpoint{5.295554in}{2.083732in}}%
\pgfpathlineto{\pgfqpoint{5.295554in}{2.080783in}}%
\pgfpathmoveto{\pgfqpoint{5.327341in}{2.054239in}}%
\pgfpathlineto{\pgfqpoint{5.327341in}{2.054239in}}%
\pgfpathlineto{\pgfqpoint{5.327341in}{2.057189in}}%
\pgfpathlineto{\pgfqpoint{5.331882in}{2.057189in}}%
\pgfpathlineto{\pgfqpoint{5.331882in}{2.054239in}}%
\pgfpathmoveto{\pgfqpoint{5.331882in}{2.051290in}}%
\pgfpathlineto{\pgfqpoint{5.331882in}{2.051290in}}%
\pgfpathlineto{\pgfqpoint{5.331882in}{2.054239in}}%
\pgfpathlineto{\pgfqpoint{5.336423in}{2.054239in}}%
\pgfpathlineto{\pgfqpoint{5.336423in}{2.051290in}}%
\pgfpathmoveto{\pgfqpoint{5.331882in}{2.054239in}}%
\pgfpathlineto{\pgfqpoint{5.331882in}{2.054239in}}%
\pgfpathlineto{\pgfqpoint{5.331882in}{2.057189in}}%
\pgfpathlineto{\pgfqpoint{5.336423in}{2.057189in}}%
\pgfpathlineto{\pgfqpoint{5.336423in}{2.054239in}}%
\pgfpathmoveto{\pgfqpoint{5.336423in}{2.048341in}}%
\pgfpathlineto{\pgfqpoint{5.336423in}{2.048341in}}%
\pgfpathlineto{\pgfqpoint{5.336423in}{2.051290in}}%
\pgfpathlineto{\pgfqpoint{5.340964in}{2.051290in}}%
\pgfpathlineto{\pgfqpoint{5.340964in}{2.048341in}}%
\pgfpathmoveto{\pgfqpoint{5.340964in}{2.045392in}}%
\pgfpathlineto{\pgfqpoint{5.340964in}{2.045392in}}%
\pgfpathlineto{\pgfqpoint{5.340964in}{2.048341in}}%
\pgfpathlineto{\pgfqpoint{5.345505in}{2.048341in}}%
\pgfpathlineto{\pgfqpoint{5.345505in}{2.045392in}}%
\pgfpathmoveto{\pgfqpoint{5.340964in}{2.048341in}}%
\pgfpathlineto{\pgfqpoint{5.340964in}{2.048341in}}%
\pgfpathlineto{\pgfqpoint{5.340964in}{2.051290in}}%
\pgfpathlineto{\pgfqpoint{5.345505in}{2.051290in}}%
\pgfpathlineto{\pgfqpoint{5.345505in}{2.048341in}}%
\pgfpathmoveto{\pgfqpoint{5.336423in}{2.051290in}}%
\pgfpathlineto{\pgfqpoint{5.336423in}{2.051290in}}%
\pgfpathlineto{\pgfqpoint{5.336423in}{2.054239in}}%
\pgfpathlineto{\pgfqpoint{5.340964in}{2.054239in}}%
\pgfpathlineto{\pgfqpoint{5.340964in}{2.051290in}}%
\pgfpathmoveto{\pgfqpoint{5.345505in}{2.042442in}}%
\pgfpathlineto{\pgfqpoint{5.345505in}{2.042442in}}%
\pgfpathlineto{\pgfqpoint{5.345505in}{2.045392in}}%
\pgfpathlineto{\pgfqpoint{5.350046in}{2.045392in}}%
\pgfpathlineto{\pgfqpoint{5.350046in}{2.042442in}}%
\pgfpathmoveto{\pgfqpoint{5.350046in}{2.039493in}}%
\pgfpathlineto{\pgfqpoint{5.350046in}{2.039493in}}%
\pgfpathlineto{\pgfqpoint{5.350046in}{2.042442in}}%
\pgfpathlineto{\pgfqpoint{5.354587in}{2.042442in}}%
\pgfpathlineto{\pgfqpoint{5.354587in}{2.039493in}}%
\pgfpathmoveto{\pgfqpoint{5.350046in}{2.042442in}}%
\pgfpathlineto{\pgfqpoint{5.350046in}{2.042442in}}%
\pgfpathlineto{\pgfqpoint{5.350046in}{2.045392in}}%
\pgfpathlineto{\pgfqpoint{5.354587in}{2.045392in}}%
\pgfpathlineto{\pgfqpoint{5.354587in}{2.042442in}}%
\pgfpathmoveto{\pgfqpoint{5.354587in}{2.036544in}}%
\pgfpathlineto{\pgfqpoint{5.354587in}{2.036544in}}%
\pgfpathlineto{\pgfqpoint{5.354587in}{2.039493in}}%
\pgfpathlineto{\pgfqpoint{5.359128in}{2.039493in}}%
\pgfpathlineto{\pgfqpoint{5.359128in}{2.036544in}}%
\pgfpathmoveto{\pgfqpoint{5.359128in}{2.033595in}}%
\pgfpathlineto{\pgfqpoint{5.359128in}{2.033595in}}%
\pgfpathlineto{\pgfqpoint{5.359128in}{2.036544in}}%
\pgfpathlineto{\pgfqpoint{5.363670in}{2.036544in}}%
\pgfpathlineto{\pgfqpoint{5.363670in}{2.033595in}}%
\pgfpathmoveto{\pgfqpoint{5.359128in}{2.036544in}}%
\pgfpathlineto{\pgfqpoint{5.359128in}{2.036544in}}%
\pgfpathlineto{\pgfqpoint{5.359128in}{2.039493in}}%
\pgfpathlineto{\pgfqpoint{5.363670in}{2.039493in}}%
\pgfpathlineto{\pgfqpoint{5.363670in}{2.036544in}}%
\pgfpathmoveto{\pgfqpoint{5.354587in}{2.039493in}}%
\pgfpathlineto{\pgfqpoint{5.354587in}{2.039493in}}%
\pgfpathlineto{\pgfqpoint{5.354587in}{2.042442in}}%
\pgfpathlineto{\pgfqpoint{5.359128in}{2.042442in}}%
\pgfpathlineto{\pgfqpoint{5.359128in}{2.039493in}}%
\pgfpathmoveto{\pgfqpoint{5.345505in}{2.045392in}}%
\pgfpathlineto{\pgfqpoint{5.345505in}{2.045392in}}%
\pgfpathlineto{\pgfqpoint{5.345505in}{2.048341in}}%
\pgfpathlineto{\pgfqpoint{5.350046in}{2.048341in}}%
\pgfpathlineto{\pgfqpoint{5.350046in}{2.045392in}}%
\pgfpathmoveto{\pgfqpoint{5.363670in}{2.030645in}}%
\pgfpathlineto{\pgfqpoint{5.363670in}{2.030645in}}%
\pgfpathlineto{\pgfqpoint{5.363670in}{2.033595in}}%
\pgfpathlineto{\pgfqpoint{5.368211in}{2.033595in}}%
\pgfpathlineto{\pgfqpoint{5.368211in}{2.030645in}}%
\pgfpathmoveto{\pgfqpoint{5.368211in}{2.027696in}}%
\pgfpathlineto{\pgfqpoint{5.368211in}{2.027696in}}%
\pgfpathlineto{\pgfqpoint{5.368211in}{2.030645in}}%
\pgfpathlineto{\pgfqpoint{5.372752in}{2.030645in}}%
\pgfpathlineto{\pgfqpoint{5.372752in}{2.027696in}}%
\pgfpathmoveto{\pgfqpoint{5.368211in}{2.030645in}}%
\pgfpathlineto{\pgfqpoint{5.368211in}{2.030645in}}%
\pgfpathlineto{\pgfqpoint{5.368211in}{2.033595in}}%
\pgfpathlineto{\pgfqpoint{5.372752in}{2.033595in}}%
\pgfpathlineto{\pgfqpoint{5.372752in}{2.030645in}}%
\pgfpathmoveto{\pgfqpoint{5.372752in}{2.024747in}}%
\pgfpathlineto{\pgfqpoint{5.372752in}{2.024747in}}%
\pgfpathlineto{\pgfqpoint{5.372752in}{2.027696in}}%
\pgfpathlineto{\pgfqpoint{5.377293in}{2.027696in}}%
\pgfpathlineto{\pgfqpoint{5.377293in}{2.024747in}}%
\pgfpathmoveto{\pgfqpoint{5.377293in}{2.021798in}}%
\pgfpathlineto{\pgfqpoint{5.377293in}{2.021798in}}%
\pgfpathlineto{\pgfqpoint{5.377293in}{2.024747in}}%
\pgfpathlineto{\pgfqpoint{5.381834in}{2.024747in}}%
\pgfpathlineto{\pgfqpoint{5.381834in}{2.021798in}}%
\pgfpathmoveto{\pgfqpoint{5.377293in}{2.024747in}}%
\pgfpathlineto{\pgfqpoint{5.377293in}{2.024747in}}%
\pgfpathlineto{\pgfqpoint{5.377293in}{2.027696in}}%
\pgfpathlineto{\pgfqpoint{5.381834in}{2.027696in}}%
\pgfpathlineto{\pgfqpoint{5.381834in}{2.024747in}}%
\pgfpathmoveto{\pgfqpoint{5.372752in}{2.027696in}}%
\pgfpathlineto{\pgfqpoint{5.372752in}{2.027696in}}%
\pgfpathlineto{\pgfqpoint{5.372752in}{2.030645in}}%
\pgfpathlineto{\pgfqpoint{5.377293in}{2.030645in}}%
\pgfpathlineto{\pgfqpoint{5.377293in}{2.027696in}}%
\pgfpathmoveto{\pgfqpoint{5.381834in}{2.018848in}}%
\pgfpathlineto{\pgfqpoint{5.381834in}{2.018848in}}%
\pgfpathlineto{\pgfqpoint{5.381834in}{2.021798in}}%
\pgfpathlineto{\pgfqpoint{5.386375in}{2.021798in}}%
\pgfpathlineto{\pgfqpoint{5.386375in}{2.018848in}}%
\pgfpathmoveto{\pgfqpoint{5.386375in}{2.015899in}}%
\pgfpathlineto{\pgfqpoint{5.386375in}{2.015899in}}%
\pgfpathlineto{\pgfqpoint{5.386375in}{2.018848in}}%
\pgfpathlineto{\pgfqpoint{5.390916in}{2.018848in}}%
\pgfpathlineto{\pgfqpoint{5.390916in}{2.015899in}}%
\pgfpathmoveto{\pgfqpoint{5.386375in}{2.018848in}}%
\pgfpathlineto{\pgfqpoint{5.386375in}{2.018848in}}%
\pgfpathlineto{\pgfqpoint{5.386375in}{2.021798in}}%
\pgfpathlineto{\pgfqpoint{5.390916in}{2.021798in}}%
\pgfpathlineto{\pgfqpoint{5.390916in}{2.018848in}}%
\pgfpathmoveto{\pgfqpoint{5.390916in}{2.012950in}}%
\pgfpathlineto{\pgfqpoint{5.390916in}{2.012950in}}%
\pgfpathlineto{\pgfqpoint{5.390916in}{2.015899in}}%
\pgfpathlineto{\pgfqpoint{5.395457in}{2.015899in}}%
\pgfpathlineto{\pgfqpoint{5.395457in}{2.012950in}}%
\pgfpathmoveto{\pgfqpoint{5.395457in}{2.010001in}}%
\pgfpathlineto{\pgfqpoint{5.395457in}{2.010001in}}%
\pgfpathlineto{\pgfqpoint{5.395457in}{2.012950in}}%
\pgfpathlineto{\pgfqpoint{5.399998in}{2.012950in}}%
\pgfpathlineto{\pgfqpoint{5.399998in}{2.010001in}}%
\pgfpathmoveto{\pgfqpoint{5.395457in}{2.012950in}}%
\pgfpathlineto{\pgfqpoint{5.395457in}{2.012950in}}%
\pgfpathlineto{\pgfqpoint{5.395457in}{2.015899in}}%
\pgfpathlineto{\pgfqpoint{5.399998in}{2.015899in}}%
\pgfpathlineto{\pgfqpoint{5.399998in}{2.012950in}}%
\pgfpathmoveto{\pgfqpoint{5.390916in}{2.015899in}}%
\pgfpathlineto{\pgfqpoint{5.390916in}{2.015899in}}%
\pgfpathlineto{\pgfqpoint{5.390916in}{2.018848in}}%
\pgfpathlineto{\pgfqpoint{5.395457in}{2.018848in}}%
\pgfpathlineto{\pgfqpoint{5.395457in}{2.015899in}}%
\pgfpathmoveto{\pgfqpoint{5.381834in}{2.021798in}}%
\pgfpathlineto{\pgfqpoint{5.381834in}{2.021798in}}%
\pgfpathlineto{\pgfqpoint{5.381834in}{2.024747in}}%
\pgfpathlineto{\pgfqpoint{5.386375in}{2.024747in}}%
\pgfpathlineto{\pgfqpoint{5.386375in}{2.021798in}}%
\pgfpathmoveto{\pgfqpoint{5.363670in}{2.033595in}}%
\pgfpathlineto{\pgfqpoint{5.363670in}{2.033595in}}%
\pgfpathlineto{\pgfqpoint{5.363670in}{2.036544in}}%
\pgfpathlineto{\pgfqpoint{5.368211in}{2.036544in}}%
\pgfpathlineto{\pgfqpoint{5.368211in}{2.033595in}}%
\pgfpathmoveto{\pgfqpoint{5.327341in}{2.057189in}}%
\pgfpathlineto{\pgfqpoint{5.327341in}{2.057189in}}%
\pgfpathlineto{\pgfqpoint{5.327341in}{2.060138in}}%
\pgfpathlineto{\pgfqpoint{5.331882in}{2.060138in}}%
\pgfpathlineto{\pgfqpoint{5.331882in}{2.057189in}}%
\pgfpathclose%
\pgfusepath{fill}%
\end{pgfscope}%
\begin{pgfscope}%
\pgfpathrectangle{\pgfqpoint{0.750000in}{0.500000in}}{\pgfqpoint{4.650000in}{3.020000in}}%
\pgfusepath{clip}%
\pgfsetbuttcap%
\pgfsetmiterjoin%
\definecolor{currentfill}{rgb}{1.000000,0.000000,0.000000}%
\pgfsetfillcolor{currentfill}%
\pgfsetlinewidth{0.000000pt}%
\definecolor{currentstroke}{rgb}{0.000000,0.000000,0.000000}%
\pgfsetstrokecolor{currentstroke}%
\pgfsetstrokeopacity{0.000000}%
\pgfsetdash{}{0pt}%
\pgfpathmoveto{\pgfqpoint{0.749999in}{3.416779in}}%
\pgfpathlineto{\pgfqpoint{0.749999in}{3.419728in}}%
\pgfpathlineto{\pgfqpoint{0.754540in}{3.419728in}}%
\pgfpathlineto{\pgfqpoint{0.754540in}{3.416779in}}%
\pgfpathmoveto{\pgfqpoint{0.754540in}{3.416779in}}%
\pgfpathlineto{\pgfqpoint{0.754540in}{3.416779in}}%
\pgfpathlineto{\pgfqpoint{0.754540in}{3.419728in}}%
\pgfpathlineto{\pgfqpoint{0.759081in}{3.419728in}}%
\pgfpathlineto{\pgfqpoint{0.759081in}{3.416779in}}%
\pgfpathmoveto{\pgfqpoint{0.759081in}{3.413830in}}%
\pgfpathlineto{\pgfqpoint{0.759081in}{3.413830in}}%
\pgfpathlineto{\pgfqpoint{0.759081in}{3.416779in}}%
\pgfpathlineto{\pgfqpoint{0.763623in}{3.416779in}}%
\pgfpathlineto{\pgfqpoint{0.763623in}{3.413830in}}%
\pgfpathmoveto{\pgfqpoint{0.759081in}{3.416779in}}%
\pgfpathlineto{\pgfqpoint{0.759081in}{3.416779in}}%
\pgfpathlineto{\pgfqpoint{0.759081in}{3.419728in}}%
\pgfpathlineto{\pgfqpoint{0.763623in}{3.419728in}}%
\pgfpathlineto{\pgfqpoint{0.763623in}{3.416779in}}%
\pgfpathmoveto{\pgfqpoint{0.763623in}{3.413830in}}%
\pgfpathlineto{\pgfqpoint{0.763623in}{3.413830in}}%
\pgfpathlineto{\pgfqpoint{0.763623in}{3.416779in}}%
\pgfpathlineto{\pgfqpoint{0.768164in}{3.416779in}}%
\pgfpathlineto{\pgfqpoint{0.768164in}{3.413830in}}%
\pgfpathmoveto{\pgfqpoint{0.772705in}{3.410881in}}%
\pgfpathlineto{\pgfqpoint{0.772705in}{3.410881in}}%
\pgfpathlineto{\pgfqpoint{0.772705in}{3.413830in}}%
\pgfpathlineto{\pgfqpoint{0.777246in}{3.413830in}}%
\pgfpathlineto{\pgfqpoint{0.777246in}{3.410881in}}%
\pgfpathmoveto{\pgfqpoint{0.777246in}{3.410881in}}%
\pgfpathlineto{\pgfqpoint{0.777246in}{3.410881in}}%
\pgfpathlineto{\pgfqpoint{0.777246in}{3.413830in}}%
\pgfpathlineto{\pgfqpoint{0.781787in}{3.413830in}}%
\pgfpathlineto{\pgfqpoint{0.781787in}{3.410881in}}%
\pgfpathmoveto{\pgfqpoint{0.781787in}{3.410881in}}%
\pgfpathlineto{\pgfqpoint{0.781787in}{3.410881in}}%
\pgfpathlineto{\pgfqpoint{0.781787in}{3.413830in}}%
\pgfpathlineto{\pgfqpoint{0.786328in}{3.413830in}}%
\pgfpathlineto{\pgfqpoint{0.786328in}{3.410881in}}%
\pgfpathmoveto{\pgfqpoint{0.768164in}{3.413830in}}%
\pgfpathlineto{\pgfqpoint{0.768164in}{3.413830in}}%
\pgfpathlineto{\pgfqpoint{0.768164in}{3.416779in}}%
\pgfpathlineto{\pgfqpoint{0.772705in}{3.416779in}}%
\pgfpathlineto{\pgfqpoint{0.772705in}{3.413830in}}%
\pgfpathmoveto{\pgfqpoint{0.772705in}{3.413830in}}%
\pgfpathlineto{\pgfqpoint{0.772705in}{3.413830in}}%
\pgfpathlineto{\pgfqpoint{0.772705in}{3.416779in}}%
\pgfpathlineto{\pgfqpoint{0.777246in}{3.416779in}}%
\pgfpathlineto{\pgfqpoint{0.777246in}{3.413830in}}%
\pgfpathmoveto{\pgfqpoint{0.786328in}{3.407932in}}%
\pgfpathlineto{\pgfqpoint{0.786328in}{3.407932in}}%
\pgfpathlineto{\pgfqpoint{0.786328in}{3.410881in}}%
\pgfpathlineto{\pgfqpoint{0.790869in}{3.410881in}}%
\pgfpathlineto{\pgfqpoint{0.790869in}{3.407932in}}%
\pgfpathmoveto{\pgfqpoint{0.786328in}{3.410881in}}%
\pgfpathlineto{\pgfqpoint{0.786328in}{3.410881in}}%
\pgfpathlineto{\pgfqpoint{0.786328in}{3.413830in}}%
\pgfpathlineto{\pgfqpoint{0.790869in}{3.413830in}}%
\pgfpathlineto{\pgfqpoint{0.790869in}{3.410881in}}%
\pgfpathmoveto{\pgfqpoint{0.790869in}{3.407932in}}%
\pgfpathlineto{\pgfqpoint{0.790869in}{3.407932in}}%
\pgfpathlineto{\pgfqpoint{0.790869in}{3.410881in}}%
\pgfpathlineto{\pgfqpoint{0.795411in}{3.410881in}}%
\pgfpathlineto{\pgfqpoint{0.795411in}{3.407932in}}%
\pgfpathmoveto{\pgfqpoint{0.799952in}{3.404982in}}%
\pgfpathlineto{\pgfqpoint{0.799952in}{3.404982in}}%
\pgfpathlineto{\pgfqpoint{0.799952in}{3.407932in}}%
\pgfpathlineto{\pgfqpoint{0.804493in}{3.407932in}}%
\pgfpathlineto{\pgfqpoint{0.804493in}{3.404982in}}%
\pgfpathmoveto{\pgfqpoint{0.795411in}{3.407932in}}%
\pgfpathlineto{\pgfqpoint{0.795411in}{3.407932in}}%
\pgfpathlineto{\pgfqpoint{0.795411in}{3.410881in}}%
\pgfpathlineto{\pgfqpoint{0.799952in}{3.410881in}}%
\pgfpathlineto{\pgfqpoint{0.799952in}{3.407932in}}%
\pgfpathmoveto{\pgfqpoint{0.799952in}{3.407932in}}%
\pgfpathlineto{\pgfqpoint{0.799952in}{3.407932in}}%
\pgfpathlineto{\pgfqpoint{0.799952in}{3.410881in}}%
\pgfpathlineto{\pgfqpoint{0.804493in}{3.410881in}}%
\pgfpathlineto{\pgfqpoint{0.804493in}{3.407932in}}%
\pgfpathmoveto{\pgfqpoint{0.804493in}{3.404982in}}%
\pgfpathlineto{\pgfqpoint{0.804493in}{3.404982in}}%
\pgfpathlineto{\pgfqpoint{0.804493in}{3.407932in}}%
\pgfpathlineto{\pgfqpoint{0.809034in}{3.407932in}}%
\pgfpathlineto{\pgfqpoint{0.809034in}{3.404982in}}%
\pgfpathmoveto{\pgfqpoint{0.809034in}{3.404982in}}%
\pgfpathlineto{\pgfqpoint{0.809034in}{3.404982in}}%
\pgfpathlineto{\pgfqpoint{0.809034in}{3.407932in}}%
\pgfpathlineto{\pgfqpoint{0.813575in}{3.407932in}}%
\pgfpathlineto{\pgfqpoint{0.813575in}{3.404982in}}%
\pgfpathmoveto{\pgfqpoint{0.813575in}{3.402033in}}%
\pgfpathlineto{\pgfqpoint{0.813575in}{3.402033in}}%
\pgfpathlineto{\pgfqpoint{0.813575in}{3.404982in}}%
\pgfpathlineto{\pgfqpoint{0.818116in}{3.404982in}}%
\pgfpathlineto{\pgfqpoint{0.818116in}{3.402033in}}%
\pgfpathmoveto{\pgfqpoint{0.813575in}{3.404982in}}%
\pgfpathlineto{\pgfqpoint{0.813575in}{3.404982in}}%
\pgfpathlineto{\pgfqpoint{0.813575in}{3.407932in}}%
\pgfpathlineto{\pgfqpoint{0.818116in}{3.407932in}}%
\pgfpathlineto{\pgfqpoint{0.818116in}{3.404982in}}%
\pgfpathmoveto{\pgfqpoint{0.818116in}{3.402033in}}%
\pgfpathlineto{\pgfqpoint{0.818116in}{3.402033in}}%
\pgfpathlineto{\pgfqpoint{0.818116in}{3.404982in}}%
\pgfpathlineto{\pgfqpoint{0.822657in}{3.404982in}}%
\pgfpathlineto{\pgfqpoint{0.822657in}{3.402033in}}%
\pgfpathmoveto{\pgfqpoint{0.827198in}{3.399084in}}%
\pgfpathlineto{\pgfqpoint{0.827198in}{3.399084in}}%
\pgfpathlineto{\pgfqpoint{0.827198in}{3.402033in}}%
\pgfpathlineto{\pgfqpoint{0.831740in}{3.402033in}}%
\pgfpathlineto{\pgfqpoint{0.831740in}{3.399084in}}%
\pgfpathmoveto{\pgfqpoint{0.831740in}{3.399084in}}%
\pgfpathlineto{\pgfqpoint{0.831740in}{3.399084in}}%
\pgfpathlineto{\pgfqpoint{0.831740in}{3.402033in}}%
\pgfpathlineto{\pgfqpoint{0.836281in}{3.402033in}}%
\pgfpathlineto{\pgfqpoint{0.836281in}{3.399084in}}%
\pgfpathmoveto{\pgfqpoint{0.836281in}{3.399084in}}%
\pgfpathlineto{\pgfqpoint{0.836281in}{3.399084in}}%
\pgfpathlineto{\pgfqpoint{0.836281in}{3.402033in}}%
\pgfpathlineto{\pgfqpoint{0.840822in}{3.402033in}}%
\pgfpathlineto{\pgfqpoint{0.840822in}{3.399084in}}%
\pgfpathmoveto{\pgfqpoint{0.840822in}{3.396135in}}%
\pgfpathlineto{\pgfqpoint{0.840822in}{3.396135in}}%
\pgfpathlineto{\pgfqpoint{0.840822in}{3.399084in}}%
\pgfpathlineto{\pgfqpoint{0.845363in}{3.399084in}}%
\pgfpathlineto{\pgfqpoint{0.845363in}{3.396135in}}%
\pgfpathmoveto{\pgfqpoint{0.840822in}{3.399084in}}%
\pgfpathlineto{\pgfqpoint{0.840822in}{3.399084in}}%
\pgfpathlineto{\pgfqpoint{0.840822in}{3.402033in}}%
\pgfpathlineto{\pgfqpoint{0.845363in}{3.402033in}}%
\pgfpathlineto{\pgfqpoint{0.845363in}{3.399084in}}%
\pgfpathmoveto{\pgfqpoint{0.845363in}{3.396135in}}%
\pgfpathlineto{\pgfqpoint{0.845363in}{3.396135in}}%
\pgfpathlineto{\pgfqpoint{0.845363in}{3.399084in}}%
\pgfpathlineto{\pgfqpoint{0.849904in}{3.399084in}}%
\pgfpathlineto{\pgfqpoint{0.849904in}{3.396135in}}%
\pgfpathmoveto{\pgfqpoint{0.854445in}{3.393186in}}%
\pgfpathlineto{\pgfqpoint{0.854445in}{3.393186in}}%
\pgfpathlineto{\pgfqpoint{0.854445in}{3.396135in}}%
\pgfpathlineto{\pgfqpoint{0.858986in}{3.396135in}}%
\pgfpathlineto{\pgfqpoint{0.858986in}{3.393186in}}%
\pgfpathmoveto{\pgfqpoint{0.849904in}{3.396135in}}%
\pgfpathlineto{\pgfqpoint{0.849904in}{3.396135in}}%
\pgfpathlineto{\pgfqpoint{0.849904in}{3.399084in}}%
\pgfpathlineto{\pgfqpoint{0.854445in}{3.399084in}}%
\pgfpathlineto{\pgfqpoint{0.854445in}{3.396135in}}%
\pgfpathmoveto{\pgfqpoint{0.854445in}{3.396135in}}%
\pgfpathlineto{\pgfqpoint{0.854445in}{3.396135in}}%
\pgfpathlineto{\pgfqpoint{0.854445in}{3.399084in}}%
\pgfpathlineto{\pgfqpoint{0.858986in}{3.399084in}}%
\pgfpathlineto{\pgfqpoint{0.858986in}{3.396135in}}%
\pgfpathmoveto{\pgfqpoint{0.822657in}{3.402033in}}%
\pgfpathlineto{\pgfqpoint{0.822657in}{3.402033in}}%
\pgfpathlineto{\pgfqpoint{0.822657in}{3.404982in}}%
\pgfpathlineto{\pgfqpoint{0.827198in}{3.404982in}}%
\pgfpathlineto{\pgfqpoint{0.827198in}{3.402033in}}%
\pgfpathmoveto{\pgfqpoint{0.827198in}{3.402033in}}%
\pgfpathlineto{\pgfqpoint{0.827198in}{3.402033in}}%
\pgfpathlineto{\pgfqpoint{0.827198in}{3.404982in}}%
\pgfpathlineto{\pgfqpoint{0.831740in}{3.404982in}}%
\pgfpathlineto{\pgfqpoint{0.831740in}{3.402033in}}%
\pgfpathmoveto{\pgfqpoint{0.858986in}{3.393186in}}%
\pgfpathlineto{\pgfqpoint{0.858986in}{3.393186in}}%
\pgfpathlineto{\pgfqpoint{0.858986in}{3.396135in}}%
\pgfpathlineto{\pgfqpoint{0.863528in}{3.396135in}}%
\pgfpathlineto{\pgfqpoint{0.863528in}{3.393186in}}%
\pgfpathmoveto{\pgfqpoint{0.863528in}{3.393186in}}%
\pgfpathlineto{\pgfqpoint{0.863528in}{3.393186in}}%
\pgfpathlineto{\pgfqpoint{0.863528in}{3.396135in}}%
\pgfpathlineto{\pgfqpoint{0.868069in}{3.396135in}}%
\pgfpathlineto{\pgfqpoint{0.868069in}{3.393186in}}%
\pgfpathmoveto{\pgfqpoint{0.868069in}{3.390236in}}%
\pgfpathlineto{\pgfqpoint{0.868069in}{3.390236in}}%
\pgfpathlineto{\pgfqpoint{0.868069in}{3.393186in}}%
\pgfpathlineto{\pgfqpoint{0.872610in}{3.393186in}}%
\pgfpathlineto{\pgfqpoint{0.872610in}{3.390236in}}%
\pgfpathmoveto{\pgfqpoint{0.868069in}{3.393186in}}%
\pgfpathlineto{\pgfqpoint{0.868069in}{3.393186in}}%
\pgfpathlineto{\pgfqpoint{0.868069in}{3.396135in}}%
\pgfpathlineto{\pgfqpoint{0.872610in}{3.396135in}}%
\pgfpathlineto{\pgfqpoint{0.872610in}{3.393186in}}%
\pgfpathmoveto{\pgfqpoint{0.872610in}{3.390236in}}%
\pgfpathlineto{\pgfqpoint{0.872610in}{3.390236in}}%
\pgfpathlineto{\pgfqpoint{0.872610in}{3.393186in}}%
\pgfpathlineto{\pgfqpoint{0.877151in}{3.393186in}}%
\pgfpathlineto{\pgfqpoint{0.877151in}{3.390236in}}%
\pgfpathmoveto{\pgfqpoint{0.881692in}{3.387287in}}%
\pgfpathlineto{\pgfqpoint{0.881692in}{3.387287in}}%
\pgfpathlineto{\pgfqpoint{0.881692in}{3.390236in}}%
\pgfpathlineto{\pgfqpoint{0.886233in}{3.390236in}}%
\pgfpathlineto{\pgfqpoint{0.886233in}{3.387287in}}%
\pgfpathmoveto{\pgfqpoint{0.886233in}{3.387287in}}%
\pgfpathlineto{\pgfqpoint{0.886233in}{3.387287in}}%
\pgfpathlineto{\pgfqpoint{0.886233in}{3.390236in}}%
\pgfpathlineto{\pgfqpoint{0.890774in}{3.390236in}}%
\pgfpathlineto{\pgfqpoint{0.890774in}{3.387287in}}%
\pgfpathmoveto{\pgfqpoint{0.890774in}{3.387287in}}%
\pgfpathlineto{\pgfqpoint{0.890774in}{3.387287in}}%
\pgfpathlineto{\pgfqpoint{0.890774in}{3.390236in}}%
\pgfpathlineto{\pgfqpoint{0.895316in}{3.390236in}}%
\pgfpathlineto{\pgfqpoint{0.895316in}{3.387287in}}%
\pgfpathmoveto{\pgfqpoint{0.877151in}{3.390236in}}%
\pgfpathlineto{\pgfqpoint{0.877151in}{3.390236in}}%
\pgfpathlineto{\pgfqpoint{0.877151in}{3.393186in}}%
\pgfpathlineto{\pgfqpoint{0.881692in}{3.393186in}}%
\pgfpathlineto{\pgfqpoint{0.881692in}{3.390236in}}%
\pgfpathmoveto{\pgfqpoint{0.881692in}{3.390236in}}%
\pgfpathlineto{\pgfqpoint{0.881692in}{3.390236in}}%
\pgfpathlineto{\pgfqpoint{0.881692in}{3.393186in}}%
\pgfpathlineto{\pgfqpoint{0.886233in}{3.393186in}}%
\pgfpathlineto{\pgfqpoint{0.886233in}{3.390236in}}%
\pgfpathmoveto{\pgfqpoint{0.936185in}{3.375490in}}%
\pgfpathlineto{\pgfqpoint{0.936185in}{3.375490in}}%
\pgfpathlineto{\pgfqpoint{0.936185in}{3.378440in}}%
\pgfpathlineto{\pgfqpoint{0.940726in}{3.378440in}}%
\pgfpathlineto{\pgfqpoint{0.940726in}{3.375490in}}%
\pgfpathmoveto{\pgfqpoint{0.940726in}{3.375490in}}%
\pgfpathlineto{\pgfqpoint{0.940726in}{3.375490in}}%
\pgfpathlineto{\pgfqpoint{0.940726in}{3.378440in}}%
\pgfpathlineto{\pgfqpoint{0.945267in}{3.378440in}}%
\pgfpathlineto{\pgfqpoint{0.945267in}{3.375490in}}%
\pgfpathmoveto{\pgfqpoint{0.945267in}{3.375490in}}%
\pgfpathlineto{\pgfqpoint{0.945267in}{3.375490in}}%
\pgfpathlineto{\pgfqpoint{0.945267in}{3.378440in}}%
\pgfpathlineto{\pgfqpoint{0.949808in}{3.378440in}}%
\pgfpathlineto{\pgfqpoint{0.949808in}{3.375490in}}%
\pgfpathmoveto{\pgfqpoint{0.949808in}{3.372541in}}%
\pgfpathlineto{\pgfqpoint{0.949808in}{3.372541in}}%
\pgfpathlineto{\pgfqpoint{0.949808in}{3.375490in}}%
\pgfpathlineto{\pgfqpoint{0.954349in}{3.375490in}}%
\pgfpathlineto{\pgfqpoint{0.954349in}{3.372541in}}%
\pgfpathmoveto{\pgfqpoint{0.949808in}{3.375490in}}%
\pgfpathlineto{\pgfqpoint{0.949808in}{3.375490in}}%
\pgfpathlineto{\pgfqpoint{0.949808in}{3.378440in}}%
\pgfpathlineto{\pgfqpoint{0.954349in}{3.378440in}}%
\pgfpathlineto{\pgfqpoint{0.954349in}{3.375490in}}%
\pgfpathmoveto{\pgfqpoint{0.954349in}{3.372541in}}%
\pgfpathlineto{\pgfqpoint{0.954349in}{3.372541in}}%
\pgfpathlineto{\pgfqpoint{0.954349in}{3.375490in}}%
\pgfpathlineto{\pgfqpoint{0.958890in}{3.375490in}}%
\pgfpathlineto{\pgfqpoint{0.958890in}{3.372541in}}%
\pgfpathmoveto{\pgfqpoint{0.963431in}{3.369592in}}%
\pgfpathlineto{\pgfqpoint{0.963431in}{3.369592in}}%
\pgfpathlineto{\pgfqpoint{0.963431in}{3.372541in}}%
\pgfpathlineto{\pgfqpoint{0.967972in}{3.372541in}}%
\pgfpathlineto{\pgfqpoint{0.967972in}{3.369592in}}%
\pgfpathmoveto{\pgfqpoint{0.958890in}{3.372541in}}%
\pgfpathlineto{\pgfqpoint{0.958890in}{3.372541in}}%
\pgfpathlineto{\pgfqpoint{0.958890in}{3.375490in}}%
\pgfpathlineto{\pgfqpoint{0.963431in}{3.375490in}}%
\pgfpathlineto{\pgfqpoint{0.963431in}{3.372541in}}%
\pgfpathmoveto{\pgfqpoint{0.963431in}{3.372541in}}%
\pgfpathlineto{\pgfqpoint{0.963431in}{3.372541in}}%
\pgfpathlineto{\pgfqpoint{0.963431in}{3.375490in}}%
\pgfpathlineto{\pgfqpoint{0.967972in}{3.375490in}}%
\pgfpathlineto{\pgfqpoint{0.967972in}{3.372541in}}%
\pgfpathmoveto{\pgfqpoint{0.895316in}{3.384338in}}%
\pgfpathlineto{\pgfqpoint{0.895316in}{3.384338in}}%
\pgfpathlineto{\pgfqpoint{0.895316in}{3.387287in}}%
\pgfpathlineto{\pgfqpoint{0.899857in}{3.387287in}}%
\pgfpathlineto{\pgfqpoint{0.899857in}{3.384338in}}%
\pgfpathmoveto{\pgfqpoint{0.895316in}{3.387287in}}%
\pgfpathlineto{\pgfqpoint{0.895316in}{3.387287in}}%
\pgfpathlineto{\pgfqpoint{0.895316in}{3.390236in}}%
\pgfpathlineto{\pgfqpoint{0.899857in}{3.390236in}}%
\pgfpathlineto{\pgfqpoint{0.899857in}{3.387287in}}%
\pgfpathmoveto{\pgfqpoint{0.899857in}{3.384338in}}%
\pgfpathlineto{\pgfqpoint{0.899857in}{3.384338in}}%
\pgfpathlineto{\pgfqpoint{0.899857in}{3.387287in}}%
\pgfpathlineto{\pgfqpoint{0.904398in}{3.387287in}}%
\pgfpathlineto{\pgfqpoint{0.904398in}{3.384338in}}%
\pgfpathmoveto{\pgfqpoint{0.908939in}{3.381389in}}%
\pgfpathlineto{\pgfqpoint{0.908939in}{3.381389in}}%
\pgfpathlineto{\pgfqpoint{0.908939in}{3.384338in}}%
\pgfpathlineto{\pgfqpoint{0.913480in}{3.384338in}}%
\pgfpathlineto{\pgfqpoint{0.913480in}{3.381389in}}%
\pgfpathmoveto{\pgfqpoint{0.904398in}{3.384338in}}%
\pgfpathlineto{\pgfqpoint{0.904398in}{3.384338in}}%
\pgfpathlineto{\pgfqpoint{0.904398in}{3.387287in}}%
\pgfpathlineto{\pgfqpoint{0.908939in}{3.387287in}}%
\pgfpathlineto{\pgfqpoint{0.908939in}{3.384338in}}%
\pgfpathmoveto{\pgfqpoint{0.908939in}{3.384338in}}%
\pgfpathlineto{\pgfqpoint{0.908939in}{3.384338in}}%
\pgfpathlineto{\pgfqpoint{0.908939in}{3.387287in}}%
\pgfpathlineto{\pgfqpoint{0.913480in}{3.387287in}}%
\pgfpathlineto{\pgfqpoint{0.913480in}{3.384338in}}%
\pgfpathmoveto{\pgfqpoint{0.913480in}{3.381389in}}%
\pgfpathlineto{\pgfqpoint{0.913480in}{3.381389in}}%
\pgfpathlineto{\pgfqpoint{0.913480in}{3.384338in}}%
\pgfpathlineto{\pgfqpoint{0.918021in}{3.384338in}}%
\pgfpathlineto{\pgfqpoint{0.918021in}{3.381389in}}%
\pgfpathmoveto{\pgfqpoint{0.918021in}{3.381389in}}%
\pgfpathlineto{\pgfqpoint{0.918021in}{3.381389in}}%
\pgfpathlineto{\pgfqpoint{0.918021in}{3.384338in}}%
\pgfpathlineto{\pgfqpoint{0.922562in}{3.384338in}}%
\pgfpathlineto{\pgfqpoint{0.922562in}{3.381389in}}%
\pgfpathmoveto{\pgfqpoint{0.922562in}{3.378440in}}%
\pgfpathlineto{\pgfqpoint{0.922562in}{3.378440in}}%
\pgfpathlineto{\pgfqpoint{0.922562in}{3.381389in}}%
\pgfpathlineto{\pgfqpoint{0.927103in}{3.381389in}}%
\pgfpathlineto{\pgfqpoint{0.927103in}{3.378440in}}%
\pgfpathmoveto{\pgfqpoint{0.922562in}{3.381389in}}%
\pgfpathlineto{\pgfqpoint{0.922562in}{3.381389in}}%
\pgfpathlineto{\pgfqpoint{0.922562in}{3.384338in}}%
\pgfpathlineto{\pgfqpoint{0.927103in}{3.384338in}}%
\pgfpathlineto{\pgfqpoint{0.927103in}{3.381389in}}%
\pgfpathmoveto{\pgfqpoint{0.927103in}{3.378440in}}%
\pgfpathlineto{\pgfqpoint{0.927103in}{3.378440in}}%
\pgfpathlineto{\pgfqpoint{0.927103in}{3.381389in}}%
\pgfpathlineto{\pgfqpoint{0.931644in}{3.381389in}}%
\pgfpathlineto{\pgfqpoint{0.931644in}{3.378440in}}%
\pgfpathmoveto{\pgfqpoint{0.931644in}{3.378440in}}%
\pgfpathlineto{\pgfqpoint{0.931644in}{3.378440in}}%
\pgfpathlineto{\pgfqpoint{0.931644in}{3.381389in}}%
\pgfpathlineto{\pgfqpoint{0.936185in}{3.381389in}}%
\pgfpathlineto{\pgfqpoint{0.936185in}{3.378440in}}%
\pgfpathmoveto{\pgfqpoint{0.936185in}{3.378440in}}%
\pgfpathlineto{\pgfqpoint{0.936185in}{3.378440in}}%
\pgfpathlineto{\pgfqpoint{0.936185in}{3.381389in}}%
\pgfpathlineto{\pgfqpoint{0.940726in}{3.381389in}}%
\pgfpathlineto{\pgfqpoint{0.940726in}{3.378440in}}%
\pgfpathmoveto{\pgfqpoint{0.967972in}{3.369592in}}%
\pgfpathlineto{\pgfqpoint{0.967972in}{3.369592in}}%
\pgfpathlineto{\pgfqpoint{0.967972in}{3.372541in}}%
\pgfpathlineto{\pgfqpoint{0.972513in}{3.372541in}}%
\pgfpathlineto{\pgfqpoint{0.972513in}{3.369592in}}%
\pgfpathmoveto{\pgfqpoint{0.972513in}{3.369592in}}%
\pgfpathlineto{\pgfqpoint{0.972513in}{3.369592in}}%
\pgfpathlineto{\pgfqpoint{0.972513in}{3.372541in}}%
\pgfpathlineto{\pgfqpoint{0.977054in}{3.372541in}}%
\pgfpathlineto{\pgfqpoint{0.977054in}{3.369592in}}%
\pgfpathmoveto{\pgfqpoint{0.977054in}{3.366643in}}%
\pgfpathlineto{\pgfqpoint{0.977054in}{3.366643in}}%
\pgfpathlineto{\pgfqpoint{0.977054in}{3.369592in}}%
\pgfpathlineto{\pgfqpoint{0.981595in}{3.369592in}}%
\pgfpathlineto{\pgfqpoint{0.981595in}{3.366643in}}%
\pgfpathmoveto{\pgfqpoint{0.977054in}{3.369592in}}%
\pgfpathlineto{\pgfqpoint{0.977054in}{3.369592in}}%
\pgfpathlineto{\pgfqpoint{0.977054in}{3.372541in}}%
\pgfpathlineto{\pgfqpoint{0.981595in}{3.372541in}}%
\pgfpathlineto{\pgfqpoint{0.981595in}{3.369592in}}%
\pgfpathmoveto{\pgfqpoint{0.981595in}{3.366643in}}%
\pgfpathlineto{\pgfqpoint{0.981595in}{3.366643in}}%
\pgfpathlineto{\pgfqpoint{0.981595in}{3.369592in}}%
\pgfpathlineto{\pgfqpoint{0.986136in}{3.369592in}}%
\pgfpathlineto{\pgfqpoint{0.986136in}{3.366643in}}%
\pgfpathmoveto{\pgfqpoint{0.990677in}{3.363694in}}%
\pgfpathlineto{\pgfqpoint{0.990677in}{3.363694in}}%
\pgfpathlineto{\pgfqpoint{0.990677in}{3.366643in}}%
\pgfpathlineto{\pgfqpoint{0.995218in}{3.366643in}}%
\pgfpathlineto{\pgfqpoint{0.995218in}{3.363694in}}%
\pgfpathmoveto{\pgfqpoint{0.995218in}{3.363694in}}%
\pgfpathlineto{\pgfqpoint{0.995218in}{3.363694in}}%
\pgfpathlineto{\pgfqpoint{0.995218in}{3.366643in}}%
\pgfpathlineto{\pgfqpoint{0.999759in}{3.366643in}}%
\pgfpathlineto{\pgfqpoint{0.999759in}{3.363694in}}%
\pgfpathmoveto{\pgfqpoint{0.999759in}{3.363694in}}%
\pgfpathlineto{\pgfqpoint{0.999759in}{3.363694in}}%
\pgfpathlineto{\pgfqpoint{0.999759in}{3.366643in}}%
\pgfpathlineto{\pgfqpoint{1.004300in}{3.366643in}}%
\pgfpathlineto{\pgfqpoint{1.004300in}{3.363694in}}%
\pgfpathmoveto{\pgfqpoint{0.986136in}{3.366643in}}%
\pgfpathlineto{\pgfqpoint{0.986136in}{3.366643in}}%
\pgfpathlineto{\pgfqpoint{0.986136in}{3.369592in}}%
\pgfpathlineto{\pgfqpoint{0.990677in}{3.369592in}}%
\pgfpathlineto{\pgfqpoint{0.990677in}{3.366643in}}%
\pgfpathmoveto{\pgfqpoint{0.990677in}{3.366643in}}%
\pgfpathlineto{\pgfqpoint{0.990677in}{3.366643in}}%
\pgfpathlineto{\pgfqpoint{0.990677in}{3.369592in}}%
\pgfpathlineto{\pgfqpoint{0.995218in}{3.369592in}}%
\pgfpathlineto{\pgfqpoint{0.995218in}{3.366643in}}%
\pgfpathmoveto{\pgfqpoint{1.004300in}{3.360744in}}%
\pgfpathlineto{\pgfqpoint{1.004300in}{3.360744in}}%
\pgfpathlineto{\pgfqpoint{1.004300in}{3.363694in}}%
\pgfpathlineto{\pgfqpoint{1.008841in}{3.363694in}}%
\pgfpathlineto{\pgfqpoint{1.008841in}{3.360744in}}%
\pgfpathmoveto{\pgfqpoint{1.004300in}{3.363694in}}%
\pgfpathlineto{\pgfqpoint{1.004300in}{3.363694in}}%
\pgfpathlineto{\pgfqpoint{1.004300in}{3.366643in}}%
\pgfpathlineto{\pgfqpoint{1.008841in}{3.366643in}}%
\pgfpathlineto{\pgfqpoint{1.008841in}{3.363694in}}%
\pgfpathmoveto{\pgfqpoint{1.008841in}{3.360744in}}%
\pgfpathlineto{\pgfqpoint{1.008841in}{3.360744in}}%
\pgfpathlineto{\pgfqpoint{1.008841in}{3.363694in}}%
\pgfpathlineto{\pgfqpoint{1.013382in}{3.363694in}}%
\pgfpathlineto{\pgfqpoint{1.013382in}{3.360744in}}%
\pgfpathmoveto{\pgfqpoint{1.017923in}{3.357795in}}%
\pgfpathlineto{\pgfqpoint{1.017923in}{3.357795in}}%
\pgfpathlineto{\pgfqpoint{1.017923in}{3.360744in}}%
\pgfpathlineto{\pgfqpoint{1.022464in}{3.360744in}}%
\pgfpathlineto{\pgfqpoint{1.022464in}{3.357795in}}%
\pgfpathmoveto{\pgfqpoint{1.013382in}{3.360744in}}%
\pgfpathlineto{\pgfqpoint{1.013382in}{3.360744in}}%
\pgfpathlineto{\pgfqpoint{1.013382in}{3.363694in}}%
\pgfpathlineto{\pgfqpoint{1.017923in}{3.363694in}}%
\pgfpathlineto{\pgfqpoint{1.017923in}{3.360744in}}%
\pgfpathmoveto{\pgfqpoint{1.017923in}{3.360744in}}%
\pgfpathlineto{\pgfqpoint{1.017923in}{3.360744in}}%
\pgfpathlineto{\pgfqpoint{1.017923in}{3.363694in}}%
\pgfpathlineto{\pgfqpoint{1.022464in}{3.363694in}}%
\pgfpathlineto{\pgfqpoint{1.022464in}{3.360744in}}%
\pgfpathmoveto{\pgfqpoint{1.022464in}{3.357795in}}%
\pgfpathlineto{\pgfqpoint{1.022464in}{3.357795in}}%
\pgfpathlineto{\pgfqpoint{1.022464in}{3.360744in}}%
\pgfpathlineto{\pgfqpoint{1.027005in}{3.360744in}}%
\pgfpathlineto{\pgfqpoint{1.027005in}{3.357795in}}%
\pgfpathmoveto{\pgfqpoint{1.027005in}{3.357795in}}%
\pgfpathlineto{\pgfqpoint{1.027005in}{3.357795in}}%
\pgfpathlineto{\pgfqpoint{1.027005in}{3.360744in}}%
\pgfpathlineto{\pgfqpoint{1.031546in}{3.360744in}}%
\pgfpathlineto{\pgfqpoint{1.031546in}{3.357795in}}%
\pgfpathmoveto{\pgfqpoint{1.031546in}{3.354846in}}%
\pgfpathlineto{\pgfqpoint{1.031546in}{3.354846in}}%
\pgfpathlineto{\pgfqpoint{1.031546in}{3.357795in}}%
\pgfpathlineto{\pgfqpoint{1.036087in}{3.357795in}}%
\pgfpathlineto{\pgfqpoint{1.036087in}{3.354846in}}%
\pgfpathmoveto{\pgfqpoint{1.031546in}{3.357795in}}%
\pgfpathlineto{\pgfqpoint{1.031546in}{3.357795in}}%
\pgfpathlineto{\pgfqpoint{1.031546in}{3.360744in}}%
\pgfpathlineto{\pgfqpoint{1.036087in}{3.360744in}}%
\pgfpathlineto{\pgfqpoint{1.036087in}{3.357795in}}%
\pgfpathmoveto{\pgfqpoint{1.036087in}{3.354846in}}%
\pgfpathlineto{\pgfqpoint{1.036087in}{3.354846in}}%
\pgfpathlineto{\pgfqpoint{1.036087in}{3.357795in}}%
\pgfpathlineto{\pgfqpoint{1.040628in}{3.357795in}}%
\pgfpathlineto{\pgfqpoint{1.040628in}{3.354846in}}%
\pgfpathmoveto{\pgfqpoint{1.154151in}{3.328303in}}%
\pgfpathlineto{\pgfqpoint{1.154151in}{3.328303in}}%
\pgfpathlineto{\pgfqpoint{1.154151in}{3.331252in}}%
\pgfpathlineto{\pgfqpoint{1.158692in}{3.331252in}}%
\pgfpathlineto{\pgfqpoint{1.158692in}{3.328303in}}%
\pgfpathmoveto{\pgfqpoint{1.158692in}{3.328303in}}%
\pgfpathlineto{\pgfqpoint{1.158692in}{3.328303in}}%
\pgfpathlineto{\pgfqpoint{1.158692in}{3.331252in}}%
\pgfpathlineto{\pgfqpoint{1.163233in}{3.331252in}}%
\pgfpathlineto{\pgfqpoint{1.163233in}{3.328303in}}%
\pgfpathmoveto{\pgfqpoint{1.163233in}{3.328303in}}%
\pgfpathlineto{\pgfqpoint{1.163233in}{3.328303in}}%
\pgfpathlineto{\pgfqpoint{1.163233in}{3.331252in}}%
\pgfpathlineto{\pgfqpoint{1.167774in}{3.331252in}}%
\pgfpathlineto{\pgfqpoint{1.167774in}{3.328303in}}%
\pgfpathmoveto{\pgfqpoint{1.167774in}{3.325354in}}%
\pgfpathlineto{\pgfqpoint{1.167774in}{3.325354in}}%
\pgfpathlineto{\pgfqpoint{1.167774in}{3.328303in}}%
\pgfpathlineto{\pgfqpoint{1.172315in}{3.328303in}}%
\pgfpathlineto{\pgfqpoint{1.172315in}{3.325354in}}%
\pgfpathmoveto{\pgfqpoint{1.167774in}{3.328303in}}%
\pgfpathlineto{\pgfqpoint{1.167774in}{3.328303in}}%
\pgfpathlineto{\pgfqpoint{1.167774in}{3.331252in}}%
\pgfpathlineto{\pgfqpoint{1.172315in}{3.331252in}}%
\pgfpathlineto{\pgfqpoint{1.172315in}{3.328303in}}%
\pgfpathmoveto{\pgfqpoint{1.172315in}{3.325354in}}%
\pgfpathlineto{\pgfqpoint{1.172315in}{3.325354in}}%
\pgfpathlineto{\pgfqpoint{1.172315in}{3.328303in}}%
\pgfpathlineto{\pgfqpoint{1.176856in}{3.328303in}}%
\pgfpathlineto{\pgfqpoint{1.176856in}{3.325354in}}%
\pgfpathmoveto{\pgfqpoint{1.181397in}{3.322404in}}%
\pgfpathlineto{\pgfqpoint{1.181397in}{3.322404in}}%
\pgfpathlineto{\pgfqpoint{1.181397in}{3.325354in}}%
\pgfpathlineto{\pgfqpoint{1.185938in}{3.325354in}}%
\pgfpathlineto{\pgfqpoint{1.185938in}{3.322404in}}%
\pgfpathmoveto{\pgfqpoint{1.176856in}{3.325354in}}%
\pgfpathlineto{\pgfqpoint{1.176856in}{3.325354in}}%
\pgfpathlineto{\pgfqpoint{1.176856in}{3.328303in}}%
\pgfpathlineto{\pgfqpoint{1.181397in}{3.328303in}}%
\pgfpathlineto{\pgfqpoint{1.181397in}{3.325354in}}%
\pgfpathmoveto{\pgfqpoint{1.181397in}{3.325354in}}%
\pgfpathlineto{\pgfqpoint{1.181397in}{3.325354in}}%
\pgfpathlineto{\pgfqpoint{1.181397in}{3.328303in}}%
\pgfpathlineto{\pgfqpoint{1.185938in}{3.328303in}}%
\pgfpathlineto{\pgfqpoint{1.185938in}{3.325354in}}%
\pgfpathmoveto{\pgfqpoint{1.045169in}{3.351897in}}%
\pgfpathlineto{\pgfqpoint{1.045169in}{3.351897in}}%
\pgfpathlineto{\pgfqpoint{1.045169in}{3.354846in}}%
\pgfpathlineto{\pgfqpoint{1.049710in}{3.354846in}}%
\pgfpathlineto{\pgfqpoint{1.049710in}{3.351897in}}%
\pgfpathmoveto{\pgfqpoint{1.049710in}{3.351897in}}%
\pgfpathlineto{\pgfqpoint{1.049710in}{3.351897in}}%
\pgfpathlineto{\pgfqpoint{1.049710in}{3.354846in}}%
\pgfpathlineto{\pgfqpoint{1.054251in}{3.354846in}}%
\pgfpathlineto{\pgfqpoint{1.054251in}{3.351897in}}%
\pgfpathmoveto{\pgfqpoint{1.054251in}{3.351897in}}%
\pgfpathlineto{\pgfqpoint{1.054251in}{3.351897in}}%
\pgfpathlineto{\pgfqpoint{1.054251in}{3.354846in}}%
\pgfpathlineto{\pgfqpoint{1.058792in}{3.354846in}}%
\pgfpathlineto{\pgfqpoint{1.058792in}{3.351897in}}%
\pgfpathmoveto{\pgfqpoint{1.058792in}{3.348948in}}%
\pgfpathlineto{\pgfqpoint{1.058792in}{3.348948in}}%
\pgfpathlineto{\pgfqpoint{1.058792in}{3.351897in}}%
\pgfpathlineto{\pgfqpoint{1.063333in}{3.351897in}}%
\pgfpathlineto{\pgfqpoint{1.063333in}{3.348948in}}%
\pgfpathmoveto{\pgfqpoint{1.058792in}{3.351897in}}%
\pgfpathlineto{\pgfqpoint{1.058792in}{3.351897in}}%
\pgfpathlineto{\pgfqpoint{1.058792in}{3.354846in}}%
\pgfpathlineto{\pgfqpoint{1.063333in}{3.354846in}}%
\pgfpathlineto{\pgfqpoint{1.063333in}{3.351897in}}%
\pgfpathmoveto{\pgfqpoint{1.063333in}{3.348948in}}%
\pgfpathlineto{\pgfqpoint{1.063333in}{3.348948in}}%
\pgfpathlineto{\pgfqpoint{1.063333in}{3.351897in}}%
\pgfpathlineto{\pgfqpoint{1.067874in}{3.351897in}}%
\pgfpathlineto{\pgfqpoint{1.067874in}{3.348948in}}%
\pgfpathmoveto{\pgfqpoint{1.072415in}{3.345998in}}%
\pgfpathlineto{\pgfqpoint{1.072415in}{3.345998in}}%
\pgfpathlineto{\pgfqpoint{1.072415in}{3.348948in}}%
\pgfpathlineto{\pgfqpoint{1.076956in}{3.348948in}}%
\pgfpathlineto{\pgfqpoint{1.076956in}{3.345998in}}%
\pgfpathmoveto{\pgfqpoint{1.067874in}{3.348948in}}%
\pgfpathlineto{\pgfqpoint{1.067874in}{3.348948in}}%
\pgfpathlineto{\pgfqpoint{1.067874in}{3.351897in}}%
\pgfpathlineto{\pgfqpoint{1.072415in}{3.351897in}}%
\pgfpathlineto{\pgfqpoint{1.072415in}{3.348948in}}%
\pgfpathmoveto{\pgfqpoint{1.072415in}{3.348948in}}%
\pgfpathlineto{\pgfqpoint{1.072415in}{3.348948in}}%
\pgfpathlineto{\pgfqpoint{1.072415in}{3.351897in}}%
\pgfpathlineto{\pgfqpoint{1.076956in}{3.351897in}}%
\pgfpathlineto{\pgfqpoint{1.076956in}{3.348948in}}%
\pgfpathmoveto{\pgfqpoint{1.040628in}{3.354846in}}%
\pgfpathlineto{\pgfqpoint{1.040628in}{3.354846in}}%
\pgfpathlineto{\pgfqpoint{1.040628in}{3.357795in}}%
\pgfpathlineto{\pgfqpoint{1.045169in}{3.357795in}}%
\pgfpathlineto{\pgfqpoint{1.045169in}{3.354846in}}%
\pgfpathmoveto{\pgfqpoint{1.045169in}{3.354846in}}%
\pgfpathlineto{\pgfqpoint{1.045169in}{3.354846in}}%
\pgfpathlineto{\pgfqpoint{1.045169in}{3.357795in}}%
\pgfpathlineto{\pgfqpoint{1.049710in}{3.357795in}}%
\pgfpathlineto{\pgfqpoint{1.049710in}{3.354846in}}%
\pgfpathmoveto{\pgfqpoint{1.076956in}{3.345998in}}%
\pgfpathlineto{\pgfqpoint{1.076956in}{3.345998in}}%
\pgfpathlineto{\pgfqpoint{1.076956in}{3.348948in}}%
\pgfpathlineto{\pgfqpoint{1.081497in}{3.348948in}}%
\pgfpathlineto{\pgfqpoint{1.081497in}{3.345998in}}%
\pgfpathmoveto{\pgfqpoint{1.081497in}{3.345998in}}%
\pgfpathlineto{\pgfqpoint{1.081497in}{3.345998in}}%
\pgfpathlineto{\pgfqpoint{1.081497in}{3.348948in}}%
\pgfpathlineto{\pgfqpoint{1.086038in}{3.348948in}}%
\pgfpathlineto{\pgfqpoint{1.086038in}{3.345998in}}%
\pgfpathmoveto{\pgfqpoint{1.086038in}{3.343049in}}%
\pgfpathlineto{\pgfqpoint{1.086038in}{3.343049in}}%
\pgfpathlineto{\pgfqpoint{1.086038in}{3.345998in}}%
\pgfpathlineto{\pgfqpoint{1.090578in}{3.345998in}}%
\pgfpathlineto{\pgfqpoint{1.090578in}{3.343049in}}%
\pgfpathmoveto{\pgfqpoint{1.086038in}{3.345998in}}%
\pgfpathlineto{\pgfqpoint{1.086038in}{3.345998in}}%
\pgfpathlineto{\pgfqpoint{1.086038in}{3.348948in}}%
\pgfpathlineto{\pgfqpoint{1.090578in}{3.348948in}}%
\pgfpathlineto{\pgfqpoint{1.090578in}{3.345998in}}%
\pgfpathmoveto{\pgfqpoint{1.090578in}{3.343049in}}%
\pgfpathlineto{\pgfqpoint{1.090578in}{3.343049in}}%
\pgfpathlineto{\pgfqpoint{1.090578in}{3.345998in}}%
\pgfpathlineto{\pgfqpoint{1.095119in}{3.345998in}}%
\pgfpathlineto{\pgfqpoint{1.095119in}{3.343049in}}%
\pgfpathmoveto{\pgfqpoint{1.099660in}{3.340100in}}%
\pgfpathlineto{\pgfqpoint{1.099660in}{3.340100in}}%
\pgfpathlineto{\pgfqpoint{1.099660in}{3.343049in}}%
\pgfpathlineto{\pgfqpoint{1.104201in}{3.343049in}}%
\pgfpathlineto{\pgfqpoint{1.104201in}{3.340100in}}%
\pgfpathmoveto{\pgfqpoint{1.104201in}{3.340100in}}%
\pgfpathlineto{\pgfqpoint{1.104201in}{3.340100in}}%
\pgfpathlineto{\pgfqpoint{1.104201in}{3.343049in}}%
\pgfpathlineto{\pgfqpoint{1.108742in}{3.343049in}}%
\pgfpathlineto{\pgfqpoint{1.108742in}{3.340100in}}%
\pgfpathmoveto{\pgfqpoint{1.108742in}{3.340100in}}%
\pgfpathlineto{\pgfqpoint{1.108742in}{3.340100in}}%
\pgfpathlineto{\pgfqpoint{1.108742in}{3.343049in}}%
\pgfpathlineto{\pgfqpoint{1.113283in}{3.343049in}}%
\pgfpathlineto{\pgfqpoint{1.113283in}{3.340100in}}%
\pgfpathmoveto{\pgfqpoint{1.095119in}{3.343049in}}%
\pgfpathlineto{\pgfqpoint{1.095119in}{3.343049in}}%
\pgfpathlineto{\pgfqpoint{1.095119in}{3.345998in}}%
\pgfpathlineto{\pgfqpoint{1.099660in}{3.345998in}}%
\pgfpathlineto{\pgfqpoint{1.099660in}{3.343049in}}%
\pgfpathmoveto{\pgfqpoint{1.099660in}{3.343049in}}%
\pgfpathlineto{\pgfqpoint{1.099660in}{3.343049in}}%
\pgfpathlineto{\pgfqpoint{1.099660in}{3.345998in}}%
\pgfpathlineto{\pgfqpoint{1.104201in}{3.345998in}}%
\pgfpathlineto{\pgfqpoint{1.104201in}{3.343049in}}%
\pgfpathmoveto{\pgfqpoint{1.113283in}{3.337151in}}%
\pgfpathlineto{\pgfqpoint{1.113283in}{3.337151in}}%
\pgfpathlineto{\pgfqpoint{1.113283in}{3.340100in}}%
\pgfpathlineto{\pgfqpoint{1.117824in}{3.340100in}}%
\pgfpathlineto{\pgfqpoint{1.117824in}{3.337151in}}%
\pgfpathmoveto{\pgfqpoint{1.113283in}{3.340100in}}%
\pgfpathlineto{\pgfqpoint{1.113283in}{3.340100in}}%
\pgfpathlineto{\pgfqpoint{1.113283in}{3.343049in}}%
\pgfpathlineto{\pgfqpoint{1.117824in}{3.343049in}}%
\pgfpathlineto{\pgfqpoint{1.117824in}{3.340100in}}%
\pgfpathmoveto{\pgfqpoint{1.117824in}{3.337151in}}%
\pgfpathlineto{\pgfqpoint{1.117824in}{3.337151in}}%
\pgfpathlineto{\pgfqpoint{1.117824in}{3.340100in}}%
\pgfpathlineto{\pgfqpoint{1.122365in}{3.340100in}}%
\pgfpathlineto{\pgfqpoint{1.122365in}{3.337151in}}%
\pgfpathmoveto{\pgfqpoint{1.126906in}{3.334201in}}%
\pgfpathlineto{\pgfqpoint{1.126906in}{3.334201in}}%
\pgfpathlineto{\pgfqpoint{1.126906in}{3.337151in}}%
\pgfpathlineto{\pgfqpoint{1.131447in}{3.337151in}}%
\pgfpathlineto{\pgfqpoint{1.131447in}{3.334201in}}%
\pgfpathmoveto{\pgfqpoint{1.122365in}{3.337151in}}%
\pgfpathlineto{\pgfqpoint{1.122365in}{3.337151in}}%
\pgfpathlineto{\pgfqpoint{1.122365in}{3.340100in}}%
\pgfpathlineto{\pgfqpoint{1.126906in}{3.340100in}}%
\pgfpathlineto{\pgfqpoint{1.126906in}{3.337151in}}%
\pgfpathmoveto{\pgfqpoint{1.126906in}{3.337151in}}%
\pgfpathlineto{\pgfqpoint{1.126906in}{3.337151in}}%
\pgfpathlineto{\pgfqpoint{1.126906in}{3.340100in}}%
\pgfpathlineto{\pgfqpoint{1.131447in}{3.340100in}}%
\pgfpathlineto{\pgfqpoint{1.131447in}{3.337151in}}%
\pgfpathmoveto{\pgfqpoint{1.131447in}{3.334201in}}%
\pgfpathlineto{\pgfqpoint{1.131447in}{3.334201in}}%
\pgfpathlineto{\pgfqpoint{1.131447in}{3.337151in}}%
\pgfpathlineto{\pgfqpoint{1.135988in}{3.337151in}}%
\pgfpathlineto{\pgfqpoint{1.135988in}{3.334201in}}%
\pgfpathmoveto{\pgfqpoint{1.135988in}{3.334201in}}%
\pgfpathlineto{\pgfqpoint{1.135988in}{3.334201in}}%
\pgfpathlineto{\pgfqpoint{1.135988in}{3.337151in}}%
\pgfpathlineto{\pgfqpoint{1.140528in}{3.337151in}}%
\pgfpathlineto{\pgfqpoint{1.140528in}{3.334201in}}%
\pgfpathmoveto{\pgfqpoint{1.140528in}{3.331252in}}%
\pgfpathlineto{\pgfqpoint{1.140528in}{3.331252in}}%
\pgfpathlineto{\pgfqpoint{1.140528in}{3.334201in}}%
\pgfpathlineto{\pgfqpoint{1.145069in}{3.334201in}}%
\pgfpathlineto{\pgfqpoint{1.145069in}{3.331252in}}%
\pgfpathmoveto{\pgfqpoint{1.140528in}{3.334201in}}%
\pgfpathlineto{\pgfqpoint{1.140528in}{3.334201in}}%
\pgfpathlineto{\pgfqpoint{1.140528in}{3.337151in}}%
\pgfpathlineto{\pgfqpoint{1.145069in}{3.337151in}}%
\pgfpathlineto{\pgfqpoint{1.145069in}{3.334201in}}%
\pgfpathmoveto{\pgfqpoint{1.145069in}{3.331252in}}%
\pgfpathlineto{\pgfqpoint{1.145069in}{3.331252in}}%
\pgfpathlineto{\pgfqpoint{1.145069in}{3.334201in}}%
\pgfpathlineto{\pgfqpoint{1.149610in}{3.334201in}}%
\pgfpathlineto{\pgfqpoint{1.149610in}{3.331252in}}%
\pgfpathmoveto{\pgfqpoint{1.149610in}{3.331252in}}%
\pgfpathlineto{\pgfqpoint{1.149610in}{3.331252in}}%
\pgfpathlineto{\pgfqpoint{1.149610in}{3.334201in}}%
\pgfpathlineto{\pgfqpoint{1.154151in}{3.334201in}}%
\pgfpathlineto{\pgfqpoint{1.154151in}{3.331252in}}%
\pgfpathmoveto{\pgfqpoint{1.154151in}{3.331252in}}%
\pgfpathlineto{\pgfqpoint{1.154151in}{3.331252in}}%
\pgfpathlineto{\pgfqpoint{1.154151in}{3.334201in}}%
\pgfpathlineto{\pgfqpoint{1.158692in}{3.334201in}}%
\pgfpathlineto{\pgfqpoint{1.158692in}{3.331252in}}%
\pgfpathmoveto{\pgfqpoint{1.185938in}{3.322404in}}%
\pgfpathlineto{\pgfqpoint{1.185938in}{3.322404in}}%
\pgfpathlineto{\pgfqpoint{1.185938in}{3.325354in}}%
\pgfpathlineto{\pgfqpoint{1.190478in}{3.325354in}}%
\pgfpathlineto{\pgfqpoint{1.190478in}{3.322404in}}%
\pgfpathmoveto{\pgfqpoint{1.190478in}{3.322404in}}%
\pgfpathlineto{\pgfqpoint{1.190478in}{3.322404in}}%
\pgfpathlineto{\pgfqpoint{1.190478in}{3.325354in}}%
\pgfpathlineto{\pgfqpoint{1.195019in}{3.325354in}}%
\pgfpathlineto{\pgfqpoint{1.195019in}{3.322404in}}%
\pgfpathmoveto{\pgfqpoint{1.195019in}{3.319455in}}%
\pgfpathlineto{\pgfqpoint{1.195019in}{3.319455in}}%
\pgfpathlineto{\pgfqpoint{1.195019in}{3.322404in}}%
\pgfpathlineto{\pgfqpoint{1.199560in}{3.322404in}}%
\pgfpathlineto{\pgfqpoint{1.199560in}{3.319455in}}%
\pgfpathmoveto{\pgfqpoint{1.195019in}{3.322404in}}%
\pgfpathlineto{\pgfqpoint{1.195019in}{3.322404in}}%
\pgfpathlineto{\pgfqpoint{1.195019in}{3.325354in}}%
\pgfpathlineto{\pgfqpoint{1.199560in}{3.325354in}}%
\pgfpathlineto{\pgfqpoint{1.199560in}{3.322404in}}%
\pgfpathmoveto{\pgfqpoint{1.199560in}{3.319455in}}%
\pgfpathlineto{\pgfqpoint{1.199560in}{3.319455in}}%
\pgfpathlineto{\pgfqpoint{1.199560in}{3.322404in}}%
\pgfpathlineto{\pgfqpoint{1.204101in}{3.322404in}}%
\pgfpathlineto{\pgfqpoint{1.204101in}{3.319455in}}%
\pgfpathmoveto{\pgfqpoint{1.208642in}{3.316506in}}%
\pgfpathlineto{\pgfqpoint{1.208642in}{3.316506in}}%
\pgfpathlineto{\pgfqpoint{1.208642in}{3.319455in}}%
\pgfpathlineto{\pgfqpoint{1.213183in}{3.319455in}}%
\pgfpathlineto{\pgfqpoint{1.213183in}{3.316506in}}%
\pgfpathmoveto{\pgfqpoint{1.213183in}{3.316506in}}%
\pgfpathlineto{\pgfqpoint{1.213183in}{3.316506in}}%
\pgfpathlineto{\pgfqpoint{1.213183in}{3.319455in}}%
\pgfpathlineto{\pgfqpoint{1.217724in}{3.319455in}}%
\pgfpathlineto{\pgfqpoint{1.217724in}{3.316506in}}%
\pgfpathmoveto{\pgfqpoint{1.217724in}{3.316506in}}%
\pgfpathlineto{\pgfqpoint{1.217724in}{3.316506in}}%
\pgfpathlineto{\pgfqpoint{1.217724in}{3.319455in}}%
\pgfpathlineto{\pgfqpoint{1.222265in}{3.319455in}}%
\pgfpathlineto{\pgfqpoint{1.222265in}{3.316506in}}%
\pgfpathmoveto{\pgfqpoint{1.204101in}{3.319455in}}%
\pgfpathlineto{\pgfqpoint{1.204101in}{3.319455in}}%
\pgfpathlineto{\pgfqpoint{1.204101in}{3.322404in}}%
\pgfpathlineto{\pgfqpoint{1.208642in}{3.322404in}}%
\pgfpathlineto{\pgfqpoint{1.208642in}{3.319455in}}%
\pgfpathmoveto{\pgfqpoint{1.208642in}{3.319455in}}%
\pgfpathlineto{\pgfqpoint{1.208642in}{3.319455in}}%
\pgfpathlineto{\pgfqpoint{1.208642in}{3.322404in}}%
\pgfpathlineto{\pgfqpoint{1.213183in}{3.322404in}}%
\pgfpathlineto{\pgfqpoint{1.213183in}{3.319455in}}%
\pgfpathmoveto{\pgfqpoint{1.222265in}{3.313557in}}%
\pgfpathlineto{\pgfqpoint{1.222265in}{3.313557in}}%
\pgfpathlineto{\pgfqpoint{1.222265in}{3.316506in}}%
\pgfpathlineto{\pgfqpoint{1.226806in}{3.316506in}}%
\pgfpathlineto{\pgfqpoint{1.226806in}{3.313557in}}%
\pgfpathmoveto{\pgfqpoint{1.222265in}{3.316506in}}%
\pgfpathlineto{\pgfqpoint{1.222265in}{3.316506in}}%
\pgfpathlineto{\pgfqpoint{1.222265in}{3.319455in}}%
\pgfpathlineto{\pgfqpoint{1.226806in}{3.319455in}}%
\pgfpathlineto{\pgfqpoint{1.226806in}{3.316506in}}%
\pgfpathmoveto{\pgfqpoint{1.226806in}{3.313557in}}%
\pgfpathlineto{\pgfqpoint{1.226806in}{3.313557in}}%
\pgfpathlineto{\pgfqpoint{1.226806in}{3.316506in}}%
\pgfpathlineto{\pgfqpoint{1.231347in}{3.316506in}}%
\pgfpathlineto{\pgfqpoint{1.231347in}{3.313557in}}%
\pgfpathmoveto{\pgfqpoint{1.235887in}{3.310607in}}%
\pgfpathlineto{\pgfqpoint{1.235887in}{3.310607in}}%
\pgfpathlineto{\pgfqpoint{1.235887in}{3.313557in}}%
\pgfpathlineto{\pgfqpoint{1.240428in}{3.313557in}}%
\pgfpathlineto{\pgfqpoint{1.240428in}{3.310607in}}%
\pgfpathmoveto{\pgfqpoint{1.231347in}{3.313557in}}%
\pgfpathlineto{\pgfqpoint{1.231347in}{3.313557in}}%
\pgfpathlineto{\pgfqpoint{1.231347in}{3.316506in}}%
\pgfpathlineto{\pgfqpoint{1.235887in}{3.316506in}}%
\pgfpathlineto{\pgfqpoint{1.235887in}{3.313557in}}%
\pgfpathmoveto{\pgfqpoint{1.235887in}{3.313557in}}%
\pgfpathlineto{\pgfqpoint{1.235887in}{3.313557in}}%
\pgfpathlineto{\pgfqpoint{1.235887in}{3.316506in}}%
\pgfpathlineto{\pgfqpoint{1.240428in}{3.316506in}}%
\pgfpathlineto{\pgfqpoint{1.240428in}{3.313557in}}%
\pgfpathmoveto{\pgfqpoint{1.240428in}{3.310607in}}%
\pgfpathlineto{\pgfqpoint{1.240428in}{3.310607in}}%
\pgfpathlineto{\pgfqpoint{1.240428in}{3.313557in}}%
\pgfpathlineto{\pgfqpoint{1.244969in}{3.313557in}}%
\pgfpathlineto{\pgfqpoint{1.244969in}{3.310607in}}%
\pgfpathmoveto{\pgfqpoint{1.244969in}{3.310607in}}%
\pgfpathlineto{\pgfqpoint{1.244969in}{3.310607in}}%
\pgfpathlineto{\pgfqpoint{1.244969in}{3.313557in}}%
\pgfpathlineto{\pgfqpoint{1.249510in}{3.313557in}}%
\pgfpathlineto{\pgfqpoint{1.249510in}{3.310607in}}%
\pgfpathmoveto{\pgfqpoint{1.249510in}{3.307658in}}%
\pgfpathlineto{\pgfqpoint{1.249510in}{3.307658in}}%
\pgfpathlineto{\pgfqpoint{1.249510in}{3.310607in}}%
\pgfpathlineto{\pgfqpoint{1.254051in}{3.310607in}}%
\pgfpathlineto{\pgfqpoint{1.254051in}{3.307658in}}%
\pgfpathmoveto{\pgfqpoint{1.249510in}{3.310607in}}%
\pgfpathlineto{\pgfqpoint{1.249510in}{3.310607in}}%
\pgfpathlineto{\pgfqpoint{1.249510in}{3.313557in}}%
\pgfpathlineto{\pgfqpoint{1.254051in}{3.313557in}}%
\pgfpathlineto{\pgfqpoint{1.254051in}{3.310607in}}%
\pgfpathmoveto{\pgfqpoint{1.254051in}{3.307658in}}%
\pgfpathlineto{\pgfqpoint{1.254051in}{3.307658in}}%
\pgfpathlineto{\pgfqpoint{1.254051in}{3.310607in}}%
\pgfpathlineto{\pgfqpoint{1.258592in}{3.310607in}}%
\pgfpathlineto{\pgfqpoint{1.258592in}{3.307658in}}%
\pgfpathmoveto{\pgfqpoint{1.263133in}{3.304709in}}%
\pgfpathlineto{\pgfqpoint{1.263133in}{3.304709in}}%
\pgfpathlineto{\pgfqpoint{1.263133in}{3.307658in}}%
\pgfpathlineto{\pgfqpoint{1.267674in}{3.307658in}}%
\pgfpathlineto{\pgfqpoint{1.267674in}{3.304709in}}%
\pgfpathmoveto{\pgfqpoint{1.267674in}{3.304709in}}%
\pgfpathlineto{\pgfqpoint{1.267674in}{3.304709in}}%
\pgfpathlineto{\pgfqpoint{1.267674in}{3.307658in}}%
\pgfpathlineto{\pgfqpoint{1.272215in}{3.307658in}}%
\pgfpathlineto{\pgfqpoint{1.272215in}{3.304709in}}%
\pgfpathmoveto{\pgfqpoint{1.272215in}{3.304709in}}%
\pgfpathlineto{\pgfqpoint{1.272215in}{3.304709in}}%
\pgfpathlineto{\pgfqpoint{1.272215in}{3.307658in}}%
\pgfpathlineto{\pgfqpoint{1.276756in}{3.307658in}}%
\pgfpathlineto{\pgfqpoint{1.276756in}{3.304709in}}%
\pgfpathmoveto{\pgfqpoint{1.276756in}{3.301759in}}%
\pgfpathlineto{\pgfqpoint{1.276756in}{3.301759in}}%
\pgfpathlineto{\pgfqpoint{1.276756in}{3.304709in}}%
\pgfpathlineto{\pgfqpoint{1.281296in}{3.304709in}}%
\pgfpathlineto{\pgfqpoint{1.281296in}{3.301759in}}%
\pgfpathmoveto{\pgfqpoint{1.276756in}{3.304709in}}%
\pgfpathlineto{\pgfqpoint{1.276756in}{3.304709in}}%
\pgfpathlineto{\pgfqpoint{1.276756in}{3.307658in}}%
\pgfpathlineto{\pgfqpoint{1.281296in}{3.307658in}}%
\pgfpathlineto{\pgfqpoint{1.281296in}{3.304709in}}%
\pgfpathmoveto{\pgfqpoint{1.281296in}{3.301759in}}%
\pgfpathlineto{\pgfqpoint{1.281296in}{3.301759in}}%
\pgfpathlineto{\pgfqpoint{1.281296in}{3.304709in}}%
\pgfpathlineto{\pgfqpoint{1.285837in}{3.304709in}}%
\pgfpathlineto{\pgfqpoint{1.285837in}{3.301759in}}%
\pgfpathmoveto{\pgfqpoint{1.290378in}{3.298810in}}%
\pgfpathlineto{\pgfqpoint{1.290378in}{3.298810in}}%
\pgfpathlineto{\pgfqpoint{1.290378in}{3.301759in}}%
\pgfpathlineto{\pgfqpoint{1.294919in}{3.301759in}}%
\pgfpathlineto{\pgfqpoint{1.294919in}{3.298810in}}%
\pgfpathmoveto{\pgfqpoint{1.285837in}{3.301759in}}%
\pgfpathlineto{\pgfqpoint{1.285837in}{3.301759in}}%
\pgfpathlineto{\pgfqpoint{1.285837in}{3.304709in}}%
\pgfpathlineto{\pgfqpoint{1.290378in}{3.304709in}}%
\pgfpathlineto{\pgfqpoint{1.290378in}{3.301759in}}%
\pgfpathmoveto{\pgfqpoint{1.290378in}{3.301759in}}%
\pgfpathlineto{\pgfqpoint{1.290378in}{3.301759in}}%
\pgfpathlineto{\pgfqpoint{1.290378in}{3.304709in}}%
\pgfpathlineto{\pgfqpoint{1.294919in}{3.304709in}}%
\pgfpathlineto{\pgfqpoint{1.294919in}{3.301759in}}%
\pgfpathmoveto{\pgfqpoint{1.258592in}{3.307658in}}%
\pgfpathlineto{\pgfqpoint{1.258592in}{3.307658in}}%
\pgfpathlineto{\pgfqpoint{1.258592in}{3.310607in}}%
\pgfpathlineto{\pgfqpoint{1.263133in}{3.310607in}}%
\pgfpathlineto{\pgfqpoint{1.263133in}{3.307658in}}%
\pgfpathmoveto{\pgfqpoint{1.263133in}{3.307658in}}%
\pgfpathlineto{\pgfqpoint{1.263133in}{3.307658in}}%
\pgfpathlineto{\pgfqpoint{1.263133in}{3.310607in}}%
\pgfpathlineto{\pgfqpoint{1.267674in}{3.310607in}}%
\pgfpathlineto{\pgfqpoint{1.267674in}{3.307658in}}%
\pgfpathmoveto{\pgfqpoint{1.294919in}{3.298810in}}%
\pgfpathlineto{\pgfqpoint{1.294919in}{3.298810in}}%
\pgfpathlineto{\pgfqpoint{1.294919in}{3.301759in}}%
\pgfpathlineto{\pgfqpoint{1.299460in}{3.301759in}}%
\pgfpathlineto{\pgfqpoint{1.299460in}{3.298810in}}%
\pgfpathmoveto{\pgfqpoint{1.299460in}{3.298810in}}%
\pgfpathlineto{\pgfqpoint{1.299460in}{3.298810in}}%
\pgfpathlineto{\pgfqpoint{1.299460in}{3.301759in}}%
\pgfpathlineto{\pgfqpoint{1.304001in}{3.301759in}}%
\pgfpathlineto{\pgfqpoint{1.304001in}{3.298810in}}%
\pgfpathmoveto{\pgfqpoint{1.304001in}{3.295861in}}%
\pgfpathlineto{\pgfqpoint{1.304001in}{3.295861in}}%
\pgfpathlineto{\pgfqpoint{1.304001in}{3.298810in}}%
\pgfpathlineto{\pgfqpoint{1.308542in}{3.298810in}}%
\pgfpathlineto{\pgfqpoint{1.308542in}{3.295861in}}%
\pgfpathmoveto{\pgfqpoint{1.304001in}{3.298810in}}%
\pgfpathlineto{\pgfqpoint{1.304001in}{3.298810in}}%
\pgfpathlineto{\pgfqpoint{1.304001in}{3.301759in}}%
\pgfpathlineto{\pgfqpoint{1.308542in}{3.301759in}}%
\pgfpathlineto{\pgfqpoint{1.308542in}{3.298810in}}%
\pgfpathmoveto{\pgfqpoint{1.308542in}{3.295861in}}%
\pgfpathlineto{\pgfqpoint{1.308542in}{3.295861in}}%
\pgfpathlineto{\pgfqpoint{1.308542in}{3.298810in}}%
\pgfpathlineto{\pgfqpoint{1.313083in}{3.298810in}}%
\pgfpathlineto{\pgfqpoint{1.313083in}{3.295861in}}%
\pgfpathmoveto{\pgfqpoint{1.317624in}{3.292911in}}%
\pgfpathlineto{\pgfqpoint{1.317624in}{3.292911in}}%
\pgfpathlineto{\pgfqpoint{1.317624in}{3.295861in}}%
\pgfpathlineto{\pgfqpoint{1.322165in}{3.295861in}}%
\pgfpathlineto{\pgfqpoint{1.322165in}{3.292911in}}%
\pgfpathmoveto{\pgfqpoint{1.322165in}{3.292911in}}%
\pgfpathlineto{\pgfqpoint{1.322165in}{3.292911in}}%
\pgfpathlineto{\pgfqpoint{1.322165in}{3.295861in}}%
\pgfpathlineto{\pgfqpoint{1.326705in}{3.295861in}}%
\pgfpathlineto{\pgfqpoint{1.326705in}{3.292911in}}%
\pgfpathmoveto{\pgfqpoint{1.326705in}{3.292911in}}%
\pgfpathlineto{\pgfqpoint{1.326705in}{3.292911in}}%
\pgfpathlineto{\pgfqpoint{1.326705in}{3.295861in}}%
\pgfpathlineto{\pgfqpoint{1.331246in}{3.295861in}}%
\pgfpathlineto{\pgfqpoint{1.331246in}{3.292911in}}%
\pgfpathmoveto{\pgfqpoint{1.313083in}{3.295861in}}%
\pgfpathlineto{\pgfqpoint{1.313083in}{3.295861in}}%
\pgfpathlineto{\pgfqpoint{1.313083in}{3.298810in}}%
\pgfpathlineto{\pgfqpoint{1.317624in}{3.298810in}}%
\pgfpathlineto{\pgfqpoint{1.317624in}{3.295861in}}%
\pgfpathmoveto{\pgfqpoint{1.317624in}{3.295861in}}%
\pgfpathlineto{\pgfqpoint{1.317624in}{3.295861in}}%
\pgfpathlineto{\pgfqpoint{1.317624in}{3.298810in}}%
\pgfpathlineto{\pgfqpoint{1.322165in}{3.298810in}}%
\pgfpathlineto{\pgfqpoint{1.322165in}{3.295861in}}%
\pgfpathmoveto{\pgfqpoint{1.372116in}{3.281114in}}%
\pgfpathlineto{\pgfqpoint{1.372116in}{3.281114in}}%
\pgfpathlineto{\pgfqpoint{1.372116in}{3.284064in}}%
\pgfpathlineto{\pgfqpoint{1.376657in}{3.284064in}}%
\pgfpathlineto{\pgfqpoint{1.376657in}{3.281114in}}%
\pgfpathmoveto{\pgfqpoint{1.376657in}{3.281114in}}%
\pgfpathlineto{\pgfqpoint{1.376657in}{3.281114in}}%
\pgfpathlineto{\pgfqpoint{1.376657in}{3.284064in}}%
\pgfpathlineto{\pgfqpoint{1.381198in}{3.284064in}}%
\pgfpathlineto{\pgfqpoint{1.381198in}{3.281114in}}%
\pgfpathmoveto{\pgfqpoint{1.381198in}{3.281114in}}%
\pgfpathlineto{\pgfqpoint{1.381198in}{3.281114in}}%
\pgfpathlineto{\pgfqpoint{1.381198in}{3.284064in}}%
\pgfpathlineto{\pgfqpoint{1.385739in}{3.284064in}}%
\pgfpathlineto{\pgfqpoint{1.385739in}{3.281114in}}%
\pgfpathmoveto{\pgfqpoint{1.385739in}{3.278165in}}%
\pgfpathlineto{\pgfqpoint{1.385739in}{3.278165in}}%
\pgfpathlineto{\pgfqpoint{1.385739in}{3.281114in}}%
\pgfpathlineto{\pgfqpoint{1.390280in}{3.281114in}}%
\pgfpathlineto{\pgfqpoint{1.390280in}{3.278165in}}%
\pgfpathmoveto{\pgfqpoint{1.385739in}{3.281114in}}%
\pgfpathlineto{\pgfqpoint{1.385739in}{3.281114in}}%
\pgfpathlineto{\pgfqpoint{1.385739in}{3.284064in}}%
\pgfpathlineto{\pgfqpoint{1.390280in}{3.284064in}}%
\pgfpathlineto{\pgfqpoint{1.390280in}{3.281114in}}%
\pgfpathmoveto{\pgfqpoint{1.390280in}{3.278165in}}%
\pgfpathlineto{\pgfqpoint{1.390280in}{3.278165in}}%
\pgfpathlineto{\pgfqpoint{1.390280in}{3.281114in}}%
\pgfpathlineto{\pgfqpoint{1.394821in}{3.281114in}}%
\pgfpathlineto{\pgfqpoint{1.394821in}{3.278165in}}%
\pgfpathmoveto{\pgfqpoint{1.399362in}{3.275216in}}%
\pgfpathlineto{\pgfqpoint{1.399362in}{3.275216in}}%
\pgfpathlineto{\pgfqpoint{1.399362in}{3.278165in}}%
\pgfpathlineto{\pgfqpoint{1.403903in}{3.278165in}}%
\pgfpathlineto{\pgfqpoint{1.403903in}{3.275216in}}%
\pgfpathmoveto{\pgfqpoint{1.394821in}{3.278165in}}%
\pgfpathlineto{\pgfqpoint{1.394821in}{3.278165in}}%
\pgfpathlineto{\pgfqpoint{1.394821in}{3.281114in}}%
\pgfpathlineto{\pgfqpoint{1.399362in}{3.281114in}}%
\pgfpathlineto{\pgfqpoint{1.399362in}{3.278165in}}%
\pgfpathmoveto{\pgfqpoint{1.399362in}{3.278165in}}%
\pgfpathlineto{\pgfqpoint{1.399362in}{3.278165in}}%
\pgfpathlineto{\pgfqpoint{1.399362in}{3.281114in}}%
\pgfpathlineto{\pgfqpoint{1.403903in}{3.281114in}}%
\pgfpathlineto{\pgfqpoint{1.403903in}{3.278165in}}%
\pgfpathmoveto{\pgfqpoint{1.331246in}{3.289962in}}%
\pgfpathlineto{\pgfqpoint{1.331246in}{3.289962in}}%
\pgfpathlineto{\pgfqpoint{1.331246in}{3.292911in}}%
\pgfpathlineto{\pgfqpoint{1.335787in}{3.292911in}}%
\pgfpathlineto{\pgfqpoint{1.335787in}{3.289962in}}%
\pgfpathmoveto{\pgfqpoint{1.331246in}{3.292911in}}%
\pgfpathlineto{\pgfqpoint{1.331246in}{3.292911in}}%
\pgfpathlineto{\pgfqpoint{1.331246in}{3.295861in}}%
\pgfpathlineto{\pgfqpoint{1.335787in}{3.295861in}}%
\pgfpathlineto{\pgfqpoint{1.335787in}{3.292911in}}%
\pgfpathmoveto{\pgfqpoint{1.335787in}{3.289962in}}%
\pgfpathlineto{\pgfqpoint{1.335787in}{3.289962in}}%
\pgfpathlineto{\pgfqpoint{1.335787in}{3.292911in}}%
\pgfpathlineto{\pgfqpoint{1.340328in}{3.292911in}}%
\pgfpathlineto{\pgfqpoint{1.340328in}{3.289962in}}%
\pgfpathmoveto{\pgfqpoint{1.344869in}{3.287013in}}%
\pgfpathlineto{\pgfqpoint{1.344869in}{3.287013in}}%
\pgfpathlineto{\pgfqpoint{1.344869in}{3.289962in}}%
\pgfpathlineto{\pgfqpoint{1.349410in}{3.289962in}}%
\pgfpathlineto{\pgfqpoint{1.349410in}{3.287013in}}%
\pgfpathmoveto{\pgfqpoint{1.340328in}{3.289962in}}%
\pgfpathlineto{\pgfqpoint{1.340328in}{3.289962in}}%
\pgfpathlineto{\pgfqpoint{1.340328in}{3.292911in}}%
\pgfpathlineto{\pgfqpoint{1.344869in}{3.292911in}}%
\pgfpathlineto{\pgfqpoint{1.344869in}{3.289962in}}%
\pgfpathmoveto{\pgfqpoint{1.344869in}{3.289962in}}%
\pgfpathlineto{\pgfqpoint{1.344869in}{3.289962in}}%
\pgfpathlineto{\pgfqpoint{1.344869in}{3.292911in}}%
\pgfpathlineto{\pgfqpoint{1.349410in}{3.292911in}}%
\pgfpathlineto{\pgfqpoint{1.349410in}{3.289962in}}%
\pgfpathmoveto{\pgfqpoint{1.349410in}{3.287013in}}%
\pgfpathlineto{\pgfqpoint{1.349410in}{3.287013in}}%
\pgfpathlineto{\pgfqpoint{1.349410in}{3.289962in}}%
\pgfpathlineto{\pgfqpoint{1.353951in}{3.289962in}}%
\pgfpathlineto{\pgfqpoint{1.353951in}{3.287013in}}%
\pgfpathmoveto{\pgfqpoint{1.353951in}{3.287013in}}%
\pgfpathlineto{\pgfqpoint{1.353951in}{3.287013in}}%
\pgfpathlineto{\pgfqpoint{1.353951in}{3.289962in}}%
\pgfpathlineto{\pgfqpoint{1.358493in}{3.289962in}}%
\pgfpathlineto{\pgfqpoint{1.358493in}{3.287013in}}%
\pgfpathmoveto{\pgfqpoint{1.358493in}{3.284064in}}%
\pgfpathlineto{\pgfqpoint{1.358493in}{3.284064in}}%
\pgfpathlineto{\pgfqpoint{1.358493in}{3.287013in}}%
\pgfpathlineto{\pgfqpoint{1.363034in}{3.287013in}}%
\pgfpathlineto{\pgfqpoint{1.363034in}{3.284064in}}%
\pgfpathmoveto{\pgfqpoint{1.358493in}{3.287013in}}%
\pgfpathlineto{\pgfqpoint{1.358493in}{3.287013in}}%
\pgfpathlineto{\pgfqpoint{1.358493in}{3.289962in}}%
\pgfpathlineto{\pgfqpoint{1.363034in}{3.289962in}}%
\pgfpathlineto{\pgfqpoint{1.363034in}{3.287013in}}%
\pgfpathmoveto{\pgfqpoint{1.363034in}{3.284064in}}%
\pgfpathlineto{\pgfqpoint{1.363034in}{3.284064in}}%
\pgfpathlineto{\pgfqpoint{1.363034in}{3.287013in}}%
\pgfpathlineto{\pgfqpoint{1.367575in}{3.287013in}}%
\pgfpathlineto{\pgfqpoint{1.367575in}{3.284064in}}%
\pgfpathmoveto{\pgfqpoint{1.367575in}{3.284064in}}%
\pgfpathlineto{\pgfqpoint{1.367575in}{3.284064in}}%
\pgfpathlineto{\pgfqpoint{1.367575in}{3.287013in}}%
\pgfpathlineto{\pgfqpoint{1.372116in}{3.287013in}}%
\pgfpathlineto{\pgfqpoint{1.372116in}{3.284064in}}%
\pgfpathmoveto{\pgfqpoint{1.372116in}{3.284064in}}%
\pgfpathlineto{\pgfqpoint{1.372116in}{3.284064in}}%
\pgfpathlineto{\pgfqpoint{1.372116in}{3.287013in}}%
\pgfpathlineto{\pgfqpoint{1.376657in}{3.287013in}}%
\pgfpathlineto{\pgfqpoint{1.376657in}{3.284064in}}%
\pgfpathmoveto{\pgfqpoint{1.403903in}{3.275216in}}%
\pgfpathlineto{\pgfqpoint{1.403903in}{3.275216in}}%
\pgfpathlineto{\pgfqpoint{1.403903in}{3.278165in}}%
\pgfpathlineto{\pgfqpoint{1.408444in}{3.278165in}}%
\pgfpathlineto{\pgfqpoint{1.408444in}{3.275216in}}%
\pgfpathmoveto{\pgfqpoint{1.408444in}{3.275216in}}%
\pgfpathlineto{\pgfqpoint{1.408444in}{3.275216in}}%
\pgfpathlineto{\pgfqpoint{1.408444in}{3.278165in}}%
\pgfpathlineto{\pgfqpoint{1.412985in}{3.278165in}}%
\pgfpathlineto{\pgfqpoint{1.412985in}{3.275216in}}%
\pgfpathmoveto{\pgfqpoint{1.412985in}{3.272266in}}%
\pgfpathlineto{\pgfqpoint{1.412985in}{3.272266in}}%
\pgfpathlineto{\pgfqpoint{1.412985in}{3.275216in}}%
\pgfpathlineto{\pgfqpoint{1.417526in}{3.275216in}}%
\pgfpathlineto{\pgfqpoint{1.417526in}{3.272266in}}%
\pgfpathmoveto{\pgfqpoint{1.412985in}{3.275216in}}%
\pgfpathlineto{\pgfqpoint{1.412985in}{3.275216in}}%
\pgfpathlineto{\pgfqpoint{1.412985in}{3.278165in}}%
\pgfpathlineto{\pgfqpoint{1.417526in}{3.278165in}}%
\pgfpathlineto{\pgfqpoint{1.417526in}{3.275216in}}%
\pgfpathmoveto{\pgfqpoint{1.417526in}{3.272266in}}%
\pgfpathlineto{\pgfqpoint{1.417526in}{3.272266in}}%
\pgfpathlineto{\pgfqpoint{1.417526in}{3.275216in}}%
\pgfpathlineto{\pgfqpoint{1.422067in}{3.275216in}}%
\pgfpathlineto{\pgfqpoint{1.422067in}{3.272266in}}%
\pgfpathmoveto{\pgfqpoint{1.426608in}{3.269317in}}%
\pgfpathlineto{\pgfqpoint{1.426608in}{3.269317in}}%
\pgfpathlineto{\pgfqpoint{1.426608in}{3.272266in}}%
\pgfpathlineto{\pgfqpoint{1.431149in}{3.272266in}}%
\pgfpathlineto{\pgfqpoint{1.431149in}{3.269317in}}%
\pgfpathmoveto{\pgfqpoint{1.431149in}{3.269317in}}%
\pgfpathlineto{\pgfqpoint{1.431149in}{3.269317in}}%
\pgfpathlineto{\pgfqpoint{1.431149in}{3.272266in}}%
\pgfpathlineto{\pgfqpoint{1.435690in}{3.272266in}}%
\pgfpathlineto{\pgfqpoint{1.435690in}{3.269317in}}%
\pgfpathmoveto{\pgfqpoint{1.435690in}{3.269317in}}%
\pgfpathlineto{\pgfqpoint{1.435690in}{3.269317in}}%
\pgfpathlineto{\pgfqpoint{1.435690in}{3.272266in}}%
\pgfpathlineto{\pgfqpoint{1.440231in}{3.272266in}}%
\pgfpathlineto{\pgfqpoint{1.440231in}{3.269317in}}%
\pgfpathmoveto{\pgfqpoint{1.422067in}{3.272266in}}%
\pgfpathlineto{\pgfqpoint{1.422067in}{3.272266in}}%
\pgfpathlineto{\pgfqpoint{1.422067in}{3.275216in}}%
\pgfpathlineto{\pgfqpoint{1.426608in}{3.275216in}}%
\pgfpathlineto{\pgfqpoint{1.426608in}{3.272266in}}%
\pgfpathmoveto{\pgfqpoint{1.426608in}{3.272266in}}%
\pgfpathlineto{\pgfqpoint{1.426608in}{3.272266in}}%
\pgfpathlineto{\pgfqpoint{1.426608in}{3.275216in}}%
\pgfpathlineto{\pgfqpoint{1.431149in}{3.275216in}}%
\pgfpathlineto{\pgfqpoint{1.431149in}{3.272266in}}%
\pgfpathmoveto{\pgfqpoint{1.440231in}{3.266368in}}%
\pgfpathlineto{\pgfqpoint{1.440231in}{3.266368in}}%
\pgfpathlineto{\pgfqpoint{1.440231in}{3.269317in}}%
\pgfpathlineto{\pgfqpoint{1.444772in}{3.269317in}}%
\pgfpathlineto{\pgfqpoint{1.444772in}{3.266368in}}%
\pgfpathmoveto{\pgfqpoint{1.440231in}{3.269317in}}%
\pgfpathlineto{\pgfqpoint{1.440231in}{3.269317in}}%
\pgfpathlineto{\pgfqpoint{1.440231in}{3.272266in}}%
\pgfpathlineto{\pgfqpoint{1.444772in}{3.272266in}}%
\pgfpathlineto{\pgfqpoint{1.444772in}{3.269317in}}%
\pgfpathmoveto{\pgfqpoint{1.444772in}{3.266368in}}%
\pgfpathlineto{\pgfqpoint{1.444772in}{3.266368in}}%
\pgfpathlineto{\pgfqpoint{1.444772in}{3.269317in}}%
\pgfpathlineto{\pgfqpoint{1.449313in}{3.269317in}}%
\pgfpathlineto{\pgfqpoint{1.449313in}{3.266368in}}%
\pgfpathmoveto{\pgfqpoint{1.453854in}{3.263419in}}%
\pgfpathlineto{\pgfqpoint{1.453854in}{3.263419in}}%
\pgfpathlineto{\pgfqpoint{1.453854in}{3.266368in}}%
\pgfpathlineto{\pgfqpoint{1.458395in}{3.266368in}}%
\pgfpathlineto{\pgfqpoint{1.458395in}{3.263419in}}%
\pgfpathmoveto{\pgfqpoint{1.449313in}{3.266368in}}%
\pgfpathlineto{\pgfqpoint{1.449313in}{3.266368in}}%
\pgfpathlineto{\pgfqpoint{1.449313in}{3.269317in}}%
\pgfpathlineto{\pgfqpoint{1.453854in}{3.269317in}}%
\pgfpathlineto{\pgfqpoint{1.453854in}{3.266368in}}%
\pgfpathmoveto{\pgfqpoint{1.453854in}{3.266368in}}%
\pgfpathlineto{\pgfqpoint{1.453854in}{3.266368in}}%
\pgfpathlineto{\pgfqpoint{1.453854in}{3.269317in}}%
\pgfpathlineto{\pgfqpoint{1.458395in}{3.269317in}}%
\pgfpathlineto{\pgfqpoint{1.458395in}{3.266368in}}%
\pgfpathmoveto{\pgfqpoint{1.458395in}{3.263419in}}%
\pgfpathlineto{\pgfqpoint{1.458395in}{3.263419in}}%
\pgfpathlineto{\pgfqpoint{1.458395in}{3.266368in}}%
\pgfpathlineto{\pgfqpoint{1.462936in}{3.266368in}}%
\pgfpathlineto{\pgfqpoint{1.462936in}{3.263419in}}%
\pgfpathmoveto{\pgfqpoint{1.462936in}{3.263419in}}%
\pgfpathlineto{\pgfqpoint{1.462936in}{3.263419in}}%
\pgfpathlineto{\pgfqpoint{1.462936in}{3.266368in}}%
\pgfpathlineto{\pgfqpoint{1.467477in}{3.266368in}}%
\pgfpathlineto{\pgfqpoint{1.467477in}{3.263419in}}%
\pgfpathmoveto{\pgfqpoint{1.467477in}{3.260469in}}%
\pgfpathlineto{\pgfqpoint{1.467477in}{3.260469in}}%
\pgfpathlineto{\pgfqpoint{1.467477in}{3.263419in}}%
\pgfpathlineto{\pgfqpoint{1.472018in}{3.263419in}}%
\pgfpathlineto{\pgfqpoint{1.472018in}{3.260469in}}%
\pgfpathmoveto{\pgfqpoint{1.467477in}{3.263419in}}%
\pgfpathlineto{\pgfqpoint{1.467477in}{3.263419in}}%
\pgfpathlineto{\pgfqpoint{1.467477in}{3.266368in}}%
\pgfpathlineto{\pgfqpoint{1.472018in}{3.266368in}}%
\pgfpathlineto{\pgfqpoint{1.472018in}{3.263419in}}%
\pgfpathmoveto{\pgfqpoint{1.472018in}{3.260469in}}%
\pgfpathlineto{\pgfqpoint{1.472018in}{3.260469in}}%
\pgfpathlineto{\pgfqpoint{1.472018in}{3.263419in}}%
\pgfpathlineto{\pgfqpoint{1.476559in}{3.263419in}}%
\pgfpathlineto{\pgfqpoint{1.476559in}{3.260469in}}%
\pgfpathmoveto{\pgfqpoint{1.590086in}{3.233926in}}%
\pgfpathlineto{\pgfqpoint{1.590086in}{3.233926in}}%
\pgfpathlineto{\pgfqpoint{1.590086in}{3.236875in}}%
\pgfpathlineto{\pgfqpoint{1.594627in}{3.236875in}}%
\pgfpathlineto{\pgfqpoint{1.594627in}{3.233926in}}%
\pgfpathmoveto{\pgfqpoint{1.594627in}{3.233926in}}%
\pgfpathlineto{\pgfqpoint{1.594627in}{3.233926in}}%
\pgfpathlineto{\pgfqpoint{1.594627in}{3.236875in}}%
\pgfpathlineto{\pgfqpoint{1.599168in}{3.236875in}}%
\pgfpathlineto{\pgfqpoint{1.599168in}{3.233926in}}%
\pgfpathmoveto{\pgfqpoint{1.599168in}{3.233926in}}%
\pgfpathlineto{\pgfqpoint{1.599168in}{3.233926in}}%
\pgfpathlineto{\pgfqpoint{1.599168in}{3.236875in}}%
\pgfpathlineto{\pgfqpoint{1.603709in}{3.236875in}}%
\pgfpathlineto{\pgfqpoint{1.603709in}{3.233926in}}%
\pgfpathmoveto{\pgfqpoint{1.603709in}{3.230976in}}%
\pgfpathlineto{\pgfqpoint{1.603709in}{3.230976in}}%
\pgfpathlineto{\pgfqpoint{1.603709in}{3.233926in}}%
\pgfpathlineto{\pgfqpoint{1.608251in}{3.233926in}}%
\pgfpathlineto{\pgfqpoint{1.608251in}{3.230976in}}%
\pgfpathmoveto{\pgfqpoint{1.603709in}{3.233926in}}%
\pgfpathlineto{\pgfqpoint{1.603709in}{3.233926in}}%
\pgfpathlineto{\pgfqpoint{1.603709in}{3.236875in}}%
\pgfpathlineto{\pgfqpoint{1.608251in}{3.236875in}}%
\pgfpathlineto{\pgfqpoint{1.608251in}{3.233926in}}%
\pgfpathmoveto{\pgfqpoint{1.608251in}{3.230976in}}%
\pgfpathlineto{\pgfqpoint{1.608251in}{3.230976in}}%
\pgfpathlineto{\pgfqpoint{1.608251in}{3.233926in}}%
\pgfpathlineto{\pgfqpoint{1.612792in}{3.233926in}}%
\pgfpathlineto{\pgfqpoint{1.612792in}{3.230976in}}%
\pgfpathmoveto{\pgfqpoint{1.617333in}{3.228027in}}%
\pgfpathlineto{\pgfqpoint{1.617333in}{3.228027in}}%
\pgfpathlineto{\pgfqpoint{1.617333in}{3.230976in}}%
\pgfpathlineto{\pgfqpoint{1.621874in}{3.230976in}}%
\pgfpathlineto{\pgfqpoint{1.621874in}{3.228027in}}%
\pgfpathmoveto{\pgfqpoint{1.612792in}{3.230976in}}%
\pgfpathlineto{\pgfqpoint{1.612792in}{3.230976in}}%
\pgfpathlineto{\pgfqpoint{1.612792in}{3.233926in}}%
\pgfpathlineto{\pgfqpoint{1.617333in}{3.233926in}}%
\pgfpathlineto{\pgfqpoint{1.617333in}{3.230976in}}%
\pgfpathmoveto{\pgfqpoint{1.617333in}{3.230976in}}%
\pgfpathlineto{\pgfqpoint{1.617333in}{3.230976in}}%
\pgfpathlineto{\pgfqpoint{1.617333in}{3.233926in}}%
\pgfpathlineto{\pgfqpoint{1.621874in}{3.233926in}}%
\pgfpathlineto{\pgfqpoint{1.621874in}{3.230976in}}%
\pgfpathmoveto{\pgfqpoint{1.481100in}{3.257520in}}%
\pgfpathlineto{\pgfqpoint{1.481100in}{3.257520in}}%
\pgfpathlineto{\pgfqpoint{1.481100in}{3.260469in}}%
\pgfpathlineto{\pgfqpoint{1.485641in}{3.260469in}}%
\pgfpathlineto{\pgfqpoint{1.485641in}{3.257520in}}%
\pgfpathmoveto{\pgfqpoint{1.485641in}{3.257520in}}%
\pgfpathlineto{\pgfqpoint{1.485641in}{3.257520in}}%
\pgfpathlineto{\pgfqpoint{1.485641in}{3.260469in}}%
\pgfpathlineto{\pgfqpoint{1.490182in}{3.260469in}}%
\pgfpathlineto{\pgfqpoint{1.490182in}{3.257520in}}%
\pgfpathmoveto{\pgfqpoint{1.490182in}{3.257520in}}%
\pgfpathlineto{\pgfqpoint{1.490182in}{3.257520in}}%
\pgfpathlineto{\pgfqpoint{1.490182in}{3.260469in}}%
\pgfpathlineto{\pgfqpoint{1.494723in}{3.260469in}}%
\pgfpathlineto{\pgfqpoint{1.494723in}{3.257520in}}%
\pgfpathmoveto{\pgfqpoint{1.494723in}{3.254571in}}%
\pgfpathlineto{\pgfqpoint{1.494723in}{3.254571in}}%
\pgfpathlineto{\pgfqpoint{1.494723in}{3.257520in}}%
\pgfpathlineto{\pgfqpoint{1.499265in}{3.257520in}}%
\pgfpathlineto{\pgfqpoint{1.499265in}{3.254571in}}%
\pgfpathmoveto{\pgfqpoint{1.494723in}{3.257520in}}%
\pgfpathlineto{\pgfqpoint{1.494723in}{3.257520in}}%
\pgfpathlineto{\pgfqpoint{1.494723in}{3.260469in}}%
\pgfpathlineto{\pgfqpoint{1.499265in}{3.260469in}}%
\pgfpathlineto{\pgfqpoint{1.499265in}{3.257520in}}%
\pgfpathmoveto{\pgfqpoint{1.499265in}{3.254571in}}%
\pgfpathlineto{\pgfqpoint{1.499265in}{3.254571in}}%
\pgfpathlineto{\pgfqpoint{1.499265in}{3.257520in}}%
\pgfpathlineto{\pgfqpoint{1.503806in}{3.257520in}}%
\pgfpathlineto{\pgfqpoint{1.503806in}{3.254571in}}%
\pgfpathmoveto{\pgfqpoint{1.508347in}{3.251621in}}%
\pgfpathlineto{\pgfqpoint{1.508347in}{3.251621in}}%
\pgfpathlineto{\pgfqpoint{1.508347in}{3.254571in}}%
\pgfpathlineto{\pgfqpoint{1.512888in}{3.254571in}}%
\pgfpathlineto{\pgfqpoint{1.512888in}{3.251621in}}%
\pgfpathmoveto{\pgfqpoint{1.503806in}{3.254571in}}%
\pgfpathlineto{\pgfqpoint{1.503806in}{3.254571in}}%
\pgfpathlineto{\pgfqpoint{1.503806in}{3.257520in}}%
\pgfpathlineto{\pgfqpoint{1.508347in}{3.257520in}}%
\pgfpathlineto{\pgfqpoint{1.508347in}{3.254571in}}%
\pgfpathmoveto{\pgfqpoint{1.508347in}{3.254571in}}%
\pgfpathlineto{\pgfqpoint{1.508347in}{3.254571in}}%
\pgfpathlineto{\pgfqpoint{1.508347in}{3.257520in}}%
\pgfpathlineto{\pgfqpoint{1.512888in}{3.257520in}}%
\pgfpathlineto{\pgfqpoint{1.512888in}{3.254571in}}%
\pgfpathmoveto{\pgfqpoint{1.476559in}{3.260469in}}%
\pgfpathlineto{\pgfqpoint{1.476559in}{3.260469in}}%
\pgfpathlineto{\pgfqpoint{1.476559in}{3.263419in}}%
\pgfpathlineto{\pgfqpoint{1.481100in}{3.263419in}}%
\pgfpathlineto{\pgfqpoint{1.481100in}{3.260469in}}%
\pgfpathmoveto{\pgfqpoint{1.481100in}{3.260469in}}%
\pgfpathlineto{\pgfqpoint{1.481100in}{3.260469in}}%
\pgfpathlineto{\pgfqpoint{1.481100in}{3.263419in}}%
\pgfpathlineto{\pgfqpoint{1.485641in}{3.263419in}}%
\pgfpathlineto{\pgfqpoint{1.485641in}{3.260469in}}%
\pgfpathmoveto{\pgfqpoint{1.512888in}{3.251621in}}%
\pgfpathlineto{\pgfqpoint{1.512888in}{3.251621in}}%
\pgfpathlineto{\pgfqpoint{1.512888in}{3.254571in}}%
\pgfpathlineto{\pgfqpoint{1.517429in}{3.254571in}}%
\pgfpathlineto{\pgfqpoint{1.517429in}{3.251621in}}%
\pgfpathmoveto{\pgfqpoint{1.517429in}{3.251621in}}%
\pgfpathlineto{\pgfqpoint{1.517429in}{3.251621in}}%
\pgfpathlineto{\pgfqpoint{1.517429in}{3.254571in}}%
\pgfpathlineto{\pgfqpoint{1.521970in}{3.254571in}}%
\pgfpathlineto{\pgfqpoint{1.521970in}{3.251621in}}%
\pgfpathmoveto{\pgfqpoint{1.521970in}{3.248672in}}%
\pgfpathlineto{\pgfqpoint{1.521970in}{3.248672in}}%
\pgfpathlineto{\pgfqpoint{1.521970in}{3.251621in}}%
\pgfpathlineto{\pgfqpoint{1.526511in}{3.251621in}}%
\pgfpathlineto{\pgfqpoint{1.526511in}{3.248672in}}%
\pgfpathmoveto{\pgfqpoint{1.521970in}{3.251621in}}%
\pgfpathlineto{\pgfqpoint{1.521970in}{3.251621in}}%
\pgfpathlineto{\pgfqpoint{1.521970in}{3.254571in}}%
\pgfpathlineto{\pgfqpoint{1.526511in}{3.254571in}}%
\pgfpathlineto{\pgfqpoint{1.526511in}{3.251621in}}%
\pgfpathmoveto{\pgfqpoint{1.526511in}{3.248672in}}%
\pgfpathlineto{\pgfqpoint{1.526511in}{3.248672in}}%
\pgfpathlineto{\pgfqpoint{1.526511in}{3.251621in}}%
\pgfpathlineto{\pgfqpoint{1.531052in}{3.251621in}}%
\pgfpathlineto{\pgfqpoint{1.531052in}{3.248672in}}%
\pgfpathmoveto{\pgfqpoint{1.535593in}{3.245723in}}%
\pgfpathlineto{\pgfqpoint{1.535593in}{3.245723in}}%
\pgfpathlineto{\pgfqpoint{1.535593in}{3.248672in}}%
\pgfpathlineto{\pgfqpoint{1.540134in}{3.248672in}}%
\pgfpathlineto{\pgfqpoint{1.540134in}{3.245723in}}%
\pgfpathmoveto{\pgfqpoint{1.540134in}{3.245723in}}%
\pgfpathlineto{\pgfqpoint{1.540134in}{3.245723in}}%
\pgfpathlineto{\pgfqpoint{1.540134in}{3.248672in}}%
\pgfpathlineto{\pgfqpoint{1.544675in}{3.248672in}}%
\pgfpathlineto{\pgfqpoint{1.544675in}{3.245723in}}%
\pgfpathmoveto{\pgfqpoint{1.544675in}{3.245723in}}%
\pgfpathlineto{\pgfqpoint{1.544675in}{3.245723in}}%
\pgfpathlineto{\pgfqpoint{1.544675in}{3.248672in}}%
\pgfpathlineto{\pgfqpoint{1.549216in}{3.248672in}}%
\pgfpathlineto{\pgfqpoint{1.549216in}{3.245723in}}%
\pgfpathmoveto{\pgfqpoint{1.531052in}{3.248672in}}%
\pgfpathlineto{\pgfqpoint{1.531052in}{3.248672in}}%
\pgfpathlineto{\pgfqpoint{1.531052in}{3.251621in}}%
\pgfpathlineto{\pgfqpoint{1.535593in}{3.251621in}}%
\pgfpathlineto{\pgfqpoint{1.535593in}{3.248672in}}%
\pgfpathmoveto{\pgfqpoint{1.535593in}{3.248672in}}%
\pgfpathlineto{\pgfqpoint{1.535593in}{3.248672in}}%
\pgfpathlineto{\pgfqpoint{1.535593in}{3.251621in}}%
\pgfpathlineto{\pgfqpoint{1.540134in}{3.251621in}}%
\pgfpathlineto{\pgfqpoint{1.540134in}{3.248672in}}%
\pgfpathmoveto{\pgfqpoint{1.549216in}{3.242773in}}%
\pgfpathlineto{\pgfqpoint{1.549216in}{3.242773in}}%
\pgfpathlineto{\pgfqpoint{1.549216in}{3.245723in}}%
\pgfpathlineto{\pgfqpoint{1.553758in}{3.245723in}}%
\pgfpathlineto{\pgfqpoint{1.553758in}{3.242773in}}%
\pgfpathmoveto{\pgfqpoint{1.549216in}{3.245723in}}%
\pgfpathlineto{\pgfqpoint{1.549216in}{3.245723in}}%
\pgfpathlineto{\pgfqpoint{1.549216in}{3.248672in}}%
\pgfpathlineto{\pgfqpoint{1.553758in}{3.248672in}}%
\pgfpathlineto{\pgfqpoint{1.553758in}{3.245723in}}%
\pgfpathmoveto{\pgfqpoint{1.553758in}{3.242773in}}%
\pgfpathlineto{\pgfqpoint{1.553758in}{3.242773in}}%
\pgfpathlineto{\pgfqpoint{1.553758in}{3.245723in}}%
\pgfpathlineto{\pgfqpoint{1.558299in}{3.245723in}}%
\pgfpathlineto{\pgfqpoint{1.558299in}{3.242773in}}%
\pgfpathmoveto{\pgfqpoint{1.562840in}{3.239824in}}%
\pgfpathlineto{\pgfqpoint{1.562840in}{3.239824in}}%
\pgfpathlineto{\pgfqpoint{1.562840in}{3.242773in}}%
\pgfpathlineto{\pgfqpoint{1.567381in}{3.242773in}}%
\pgfpathlineto{\pgfqpoint{1.567381in}{3.239824in}}%
\pgfpathmoveto{\pgfqpoint{1.558299in}{3.242773in}}%
\pgfpathlineto{\pgfqpoint{1.558299in}{3.242773in}}%
\pgfpathlineto{\pgfqpoint{1.558299in}{3.245723in}}%
\pgfpathlineto{\pgfqpoint{1.562840in}{3.245723in}}%
\pgfpathlineto{\pgfqpoint{1.562840in}{3.242773in}}%
\pgfpathmoveto{\pgfqpoint{1.562840in}{3.242773in}}%
\pgfpathlineto{\pgfqpoint{1.562840in}{3.242773in}}%
\pgfpathlineto{\pgfqpoint{1.562840in}{3.245723in}}%
\pgfpathlineto{\pgfqpoint{1.567381in}{3.245723in}}%
\pgfpathlineto{\pgfqpoint{1.567381in}{3.242773in}}%
\pgfpathmoveto{\pgfqpoint{1.567381in}{3.239824in}}%
\pgfpathlineto{\pgfqpoint{1.567381in}{3.239824in}}%
\pgfpathlineto{\pgfqpoint{1.567381in}{3.242773in}}%
\pgfpathlineto{\pgfqpoint{1.571922in}{3.242773in}}%
\pgfpathlineto{\pgfqpoint{1.571922in}{3.239824in}}%
\pgfpathmoveto{\pgfqpoint{1.571922in}{3.239824in}}%
\pgfpathlineto{\pgfqpoint{1.571922in}{3.239824in}}%
\pgfpathlineto{\pgfqpoint{1.571922in}{3.242773in}}%
\pgfpathlineto{\pgfqpoint{1.576463in}{3.242773in}}%
\pgfpathlineto{\pgfqpoint{1.576463in}{3.239824in}}%
\pgfpathmoveto{\pgfqpoint{1.576463in}{3.236875in}}%
\pgfpathlineto{\pgfqpoint{1.576463in}{3.236875in}}%
\pgfpathlineto{\pgfqpoint{1.576463in}{3.239824in}}%
\pgfpathlineto{\pgfqpoint{1.581004in}{3.239824in}}%
\pgfpathlineto{\pgfqpoint{1.581004in}{3.236875in}}%
\pgfpathmoveto{\pgfqpoint{1.576463in}{3.239824in}}%
\pgfpathlineto{\pgfqpoint{1.576463in}{3.239824in}}%
\pgfpathlineto{\pgfqpoint{1.576463in}{3.242773in}}%
\pgfpathlineto{\pgfqpoint{1.581004in}{3.242773in}}%
\pgfpathlineto{\pgfqpoint{1.581004in}{3.239824in}}%
\pgfpathmoveto{\pgfqpoint{1.581004in}{3.236875in}}%
\pgfpathlineto{\pgfqpoint{1.581004in}{3.236875in}}%
\pgfpathlineto{\pgfqpoint{1.581004in}{3.239824in}}%
\pgfpathlineto{\pgfqpoint{1.585545in}{3.239824in}}%
\pgfpathlineto{\pgfqpoint{1.585545in}{3.236875in}}%
\pgfpathmoveto{\pgfqpoint{1.585545in}{3.236875in}}%
\pgfpathlineto{\pgfqpoint{1.585545in}{3.236875in}}%
\pgfpathlineto{\pgfqpoint{1.585545in}{3.239824in}}%
\pgfpathlineto{\pgfqpoint{1.590086in}{3.239824in}}%
\pgfpathlineto{\pgfqpoint{1.590086in}{3.236875in}}%
\pgfpathmoveto{\pgfqpoint{1.590086in}{3.236875in}}%
\pgfpathlineto{\pgfqpoint{1.590086in}{3.236875in}}%
\pgfpathlineto{\pgfqpoint{1.590086in}{3.239824in}}%
\pgfpathlineto{\pgfqpoint{1.594627in}{3.239824in}}%
\pgfpathlineto{\pgfqpoint{1.594627in}{3.236875in}}%
\pgfpathmoveto{\pgfqpoint{1.621874in}{3.228027in}}%
\pgfpathlineto{\pgfqpoint{1.621874in}{3.228027in}}%
\pgfpathlineto{\pgfqpoint{1.621874in}{3.230976in}}%
\pgfpathlineto{\pgfqpoint{1.626415in}{3.230976in}}%
\pgfpathlineto{\pgfqpoint{1.626415in}{3.228027in}}%
\pgfpathmoveto{\pgfqpoint{1.626415in}{3.228027in}}%
\pgfpathlineto{\pgfqpoint{1.626415in}{3.228027in}}%
\pgfpathlineto{\pgfqpoint{1.626415in}{3.230976in}}%
\pgfpathlineto{\pgfqpoint{1.630956in}{3.230976in}}%
\pgfpathlineto{\pgfqpoint{1.630956in}{3.228027in}}%
\pgfpathmoveto{\pgfqpoint{1.630956in}{3.225078in}}%
\pgfpathlineto{\pgfqpoint{1.630956in}{3.225078in}}%
\pgfpathlineto{\pgfqpoint{1.630956in}{3.228027in}}%
\pgfpathlineto{\pgfqpoint{1.635497in}{3.228027in}}%
\pgfpathlineto{\pgfqpoint{1.635497in}{3.225078in}}%
\pgfpathmoveto{\pgfqpoint{1.630956in}{3.228027in}}%
\pgfpathlineto{\pgfqpoint{1.630956in}{3.228027in}}%
\pgfpathlineto{\pgfqpoint{1.630956in}{3.230976in}}%
\pgfpathlineto{\pgfqpoint{1.635497in}{3.230976in}}%
\pgfpathlineto{\pgfqpoint{1.635497in}{3.228027in}}%
\pgfpathmoveto{\pgfqpoint{1.635497in}{3.225078in}}%
\pgfpathlineto{\pgfqpoint{1.635497in}{3.225078in}}%
\pgfpathlineto{\pgfqpoint{1.635497in}{3.228027in}}%
\pgfpathlineto{\pgfqpoint{1.640038in}{3.228027in}}%
\pgfpathlineto{\pgfqpoint{1.640038in}{3.225078in}}%
\pgfpathmoveto{\pgfqpoint{1.644579in}{3.222129in}}%
\pgfpathlineto{\pgfqpoint{1.644579in}{3.222129in}}%
\pgfpathlineto{\pgfqpoint{1.644579in}{3.225078in}}%
\pgfpathlineto{\pgfqpoint{1.649120in}{3.225078in}}%
\pgfpathlineto{\pgfqpoint{1.649120in}{3.222129in}}%
\pgfpathmoveto{\pgfqpoint{1.649120in}{3.222129in}}%
\pgfpathlineto{\pgfqpoint{1.649120in}{3.222129in}}%
\pgfpathlineto{\pgfqpoint{1.649120in}{3.225078in}}%
\pgfpathlineto{\pgfqpoint{1.653661in}{3.225078in}}%
\pgfpathlineto{\pgfqpoint{1.653661in}{3.222129in}}%
\pgfpathmoveto{\pgfqpoint{1.653661in}{3.222129in}}%
\pgfpathlineto{\pgfqpoint{1.653661in}{3.222129in}}%
\pgfpathlineto{\pgfqpoint{1.653661in}{3.225078in}}%
\pgfpathlineto{\pgfqpoint{1.658202in}{3.225078in}}%
\pgfpathlineto{\pgfqpoint{1.658202in}{3.222129in}}%
\pgfpathmoveto{\pgfqpoint{1.640038in}{3.225078in}}%
\pgfpathlineto{\pgfqpoint{1.640038in}{3.225078in}}%
\pgfpathlineto{\pgfqpoint{1.640038in}{3.228027in}}%
\pgfpathlineto{\pgfqpoint{1.644579in}{3.228027in}}%
\pgfpathlineto{\pgfqpoint{1.644579in}{3.225078in}}%
\pgfpathmoveto{\pgfqpoint{1.644579in}{3.225078in}}%
\pgfpathlineto{\pgfqpoint{1.644579in}{3.225078in}}%
\pgfpathlineto{\pgfqpoint{1.644579in}{3.228027in}}%
\pgfpathlineto{\pgfqpoint{1.649120in}{3.228027in}}%
\pgfpathlineto{\pgfqpoint{1.649120in}{3.225078in}}%
\pgfpathmoveto{\pgfqpoint{1.658202in}{3.219179in}}%
\pgfpathlineto{\pgfqpoint{1.658202in}{3.219179in}}%
\pgfpathlineto{\pgfqpoint{1.658202in}{3.222129in}}%
\pgfpathlineto{\pgfqpoint{1.662743in}{3.222129in}}%
\pgfpathlineto{\pgfqpoint{1.662743in}{3.219179in}}%
\pgfpathmoveto{\pgfqpoint{1.658202in}{3.222129in}}%
\pgfpathlineto{\pgfqpoint{1.658202in}{3.222129in}}%
\pgfpathlineto{\pgfqpoint{1.658202in}{3.225078in}}%
\pgfpathlineto{\pgfqpoint{1.662743in}{3.225078in}}%
\pgfpathlineto{\pgfqpoint{1.662743in}{3.222129in}}%
\pgfpathmoveto{\pgfqpoint{1.662743in}{3.219179in}}%
\pgfpathlineto{\pgfqpoint{1.662743in}{3.219179in}}%
\pgfpathlineto{\pgfqpoint{1.662743in}{3.222129in}}%
\pgfpathlineto{\pgfqpoint{1.667285in}{3.222129in}}%
\pgfpathlineto{\pgfqpoint{1.667285in}{3.219179in}}%
\pgfpathmoveto{\pgfqpoint{1.671826in}{3.216230in}}%
\pgfpathlineto{\pgfqpoint{1.671826in}{3.216230in}}%
\pgfpathlineto{\pgfqpoint{1.671826in}{3.219179in}}%
\pgfpathlineto{\pgfqpoint{1.676367in}{3.219179in}}%
\pgfpathlineto{\pgfqpoint{1.676367in}{3.216230in}}%
\pgfpathmoveto{\pgfqpoint{1.667285in}{3.219179in}}%
\pgfpathlineto{\pgfqpoint{1.667285in}{3.219179in}}%
\pgfpathlineto{\pgfqpoint{1.667285in}{3.222129in}}%
\pgfpathlineto{\pgfqpoint{1.671826in}{3.222129in}}%
\pgfpathlineto{\pgfqpoint{1.671826in}{3.219179in}}%
\pgfpathmoveto{\pgfqpoint{1.671826in}{3.219179in}}%
\pgfpathlineto{\pgfqpoint{1.671826in}{3.219179in}}%
\pgfpathlineto{\pgfqpoint{1.671826in}{3.222129in}}%
\pgfpathlineto{\pgfqpoint{1.676367in}{3.222129in}}%
\pgfpathlineto{\pgfqpoint{1.676367in}{3.219179in}}%
\pgfpathmoveto{\pgfqpoint{1.676367in}{3.216230in}}%
\pgfpathlineto{\pgfqpoint{1.676367in}{3.216230in}}%
\pgfpathlineto{\pgfqpoint{1.676367in}{3.219179in}}%
\pgfpathlineto{\pgfqpoint{1.680908in}{3.219179in}}%
\pgfpathlineto{\pgfqpoint{1.680908in}{3.216230in}}%
\pgfpathmoveto{\pgfqpoint{1.680908in}{3.216230in}}%
\pgfpathlineto{\pgfqpoint{1.680908in}{3.216230in}}%
\pgfpathlineto{\pgfqpoint{1.680908in}{3.219179in}}%
\pgfpathlineto{\pgfqpoint{1.685449in}{3.219179in}}%
\pgfpathlineto{\pgfqpoint{1.685449in}{3.216230in}}%
\pgfpathmoveto{\pgfqpoint{1.685449in}{3.213281in}}%
\pgfpathlineto{\pgfqpoint{1.685449in}{3.213281in}}%
\pgfpathlineto{\pgfqpoint{1.685449in}{3.216230in}}%
\pgfpathlineto{\pgfqpoint{1.689990in}{3.216230in}}%
\pgfpathlineto{\pgfqpoint{1.689990in}{3.213281in}}%
\pgfpathmoveto{\pgfqpoint{1.685449in}{3.216230in}}%
\pgfpathlineto{\pgfqpoint{1.685449in}{3.216230in}}%
\pgfpathlineto{\pgfqpoint{1.685449in}{3.219179in}}%
\pgfpathlineto{\pgfqpoint{1.689990in}{3.219179in}}%
\pgfpathlineto{\pgfqpoint{1.689990in}{3.216230in}}%
\pgfpathmoveto{\pgfqpoint{1.689990in}{3.213281in}}%
\pgfpathlineto{\pgfqpoint{1.689990in}{3.213281in}}%
\pgfpathlineto{\pgfqpoint{1.689990in}{3.216230in}}%
\pgfpathlineto{\pgfqpoint{1.694531in}{3.216230in}}%
\pgfpathlineto{\pgfqpoint{1.694531in}{3.213281in}}%
\pgfpathmoveto{\pgfqpoint{1.699072in}{3.210332in}}%
\pgfpathlineto{\pgfqpoint{1.699072in}{3.210332in}}%
\pgfpathlineto{\pgfqpoint{1.699072in}{3.213281in}}%
\pgfpathlineto{\pgfqpoint{1.703613in}{3.213281in}}%
\pgfpathlineto{\pgfqpoint{1.703613in}{3.210332in}}%
\pgfpathmoveto{\pgfqpoint{1.703613in}{3.210332in}}%
\pgfpathlineto{\pgfqpoint{1.703613in}{3.210332in}}%
\pgfpathlineto{\pgfqpoint{1.703613in}{3.213281in}}%
\pgfpathlineto{\pgfqpoint{1.708154in}{3.213281in}}%
\pgfpathlineto{\pgfqpoint{1.708154in}{3.210332in}}%
\pgfpathmoveto{\pgfqpoint{1.708154in}{3.210332in}}%
\pgfpathlineto{\pgfqpoint{1.708154in}{3.210332in}}%
\pgfpathlineto{\pgfqpoint{1.708154in}{3.213281in}}%
\pgfpathlineto{\pgfqpoint{1.712695in}{3.213281in}}%
\pgfpathlineto{\pgfqpoint{1.712695in}{3.210332in}}%
\pgfpathmoveto{\pgfqpoint{1.712695in}{3.207382in}}%
\pgfpathlineto{\pgfqpoint{1.712695in}{3.207382in}}%
\pgfpathlineto{\pgfqpoint{1.712695in}{3.210332in}}%
\pgfpathlineto{\pgfqpoint{1.717236in}{3.210332in}}%
\pgfpathlineto{\pgfqpoint{1.717236in}{3.207382in}}%
\pgfpathmoveto{\pgfqpoint{1.712695in}{3.210332in}}%
\pgfpathlineto{\pgfqpoint{1.712695in}{3.210332in}}%
\pgfpathlineto{\pgfqpoint{1.712695in}{3.213281in}}%
\pgfpathlineto{\pgfqpoint{1.717236in}{3.213281in}}%
\pgfpathlineto{\pgfqpoint{1.717236in}{3.210332in}}%
\pgfpathmoveto{\pgfqpoint{1.717236in}{3.207382in}}%
\pgfpathlineto{\pgfqpoint{1.717236in}{3.207382in}}%
\pgfpathlineto{\pgfqpoint{1.717236in}{3.210332in}}%
\pgfpathlineto{\pgfqpoint{1.721778in}{3.210332in}}%
\pgfpathlineto{\pgfqpoint{1.721778in}{3.207382in}}%
\pgfpathmoveto{\pgfqpoint{1.726319in}{3.204433in}}%
\pgfpathlineto{\pgfqpoint{1.726319in}{3.204433in}}%
\pgfpathlineto{\pgfqpoint{1.726319in}{3.207382in}}%
\pgfpathlineto{\pgfqpoint{1.730860in}{3.207382in}}%
\pgfpathlineto{\pgfqpoint{1.730860in}{3.204433in}}%
\pgfpathmoveto{\pgfqpoint{1.721778in}{3.207382in}}%
\pgfpathlineto{\pgfqpoint{1.721778in}{3.207382in}}%
\pgfpathlineto{\pgfqpoint{1.721778in}{3.210332in}}%
\pgfpathlineto{\pgfqpoint{1.726319in}{3.210332in}}%
\pgfpathlineto{\pgfqpoint{1.726319in}{3.207382in}}%
\pgfpathmoveto{\pgfqpoint{1.726319in}{3.207382in}}%
\pgfpathlineto{\pgfqpoint{1.726319in}{3.207382in}}%
\pgfpathlineto{\pgfqpoint{1.726319in}{3.210332in}}%
\pgfpathlineto{\pgfqpoint{1.730860in}{3.210332in}}%
\pgfpathlineto{\pgfqpoint{1.730860in}{3.207382in}}%
\pgfpathmoveto{\pgfqpoint{1.694531in}{3.213281in}}%
\pgfpathlineto{\pgfqpoint{1.694531in}{3.213281in}}%
\pgfpathlineto{\pgfqpoint{1.694531in}{3.216230in}}%
\pgfpathlineto{\pgfqpoint{1.699072in}{3.216230in}}%
\pgfpathlineto{\pgfqpoint{1.699072in}{3.213281in}}%
\pgfpathmoveto{\pgfqpoint{1.699072in}{3.213281in}}%
\pgfpathlineto{\pgfqpoint{1.699072in}{3.213281in}}%
\pgfpathlineto{\pgfqpoint{1.699072in}{3.216230in}}%
\pgfpathlineto{\pgfqpoint{1.703613in}{3.216230in}}%
\pgfpathlineto{\pgfqpoint{1.703613in}{3.213281in}}%
\pgfpathmoveto{\pgfqpoint{1.730860in}{3.204433in}}%
\pgfpathlineto{\pgfqpoint{1.730860in}{3.204433in}}%
\pgfpathlineto{\pgfqpoint{1.730860in}{3.207382in}}%
\pgfpathlineto{\pgfqpoint{1.735401in}{3.207382in}}%
\pgfpathlineto{\pgfqpoint{1.735401in}{3.204433in}}%
\pgfpathmoveto{\pgfqpoint{1.735401in}{3.204433in}}%
\pgfpathlineto{\pgfqpoint{1.735401in}{3.204433in}}%
\pgfpathlineto{\pgfqpoint{1.735401in}{3.207382in}}%
\pgfpathlineto{\pgfqpoint{1.739942in}{3.207382in}}%
\pgfpathlineto{\pgfqpoint{1.739942in}{3.204433in}}%
\pgfpathmoveto{\pgfqpoint{1.739942in}{3.201484in}}%
\pgfpathlineto{\pgfqpoint{1.739942in}{3.201484in}}%
\pgfpathlineto{\pgfqpoint{1.739942in}{3.204433in}}%
\pgfpathlineto{\pgfqpoint{1.744483in}{3.204433in}}%
\pgfpathlineto{\pgfqpoint{1.744483in}{3.201484in}}%
\pgfpathmoveto{\pgfqpoint{1.739942in}{3.204433in}}%
\pgfpathlineto{\pgfqpoint{1.739942in}{3.204433in}}%
\pgfpathlineto{\pgfqpoint{1.739942in}{3.207382in}}%
\pgfpathlineto{\pgfqpoint{1.744483in}{3.207382in}}%
\pgfpathlineto{\pgfqpoint{1.744483in}{3.204433in}}%
\pgfpathmoveto{\pgfqpoint{1.744483in}{3.201484in}}%
\pgfpathlineto{\pgfqpoint{1.744483in}{3.201484in}}%
\pgfpathlineto{\pgfqpoint{1.744483in}{3.204433in}}%
\pgfpathlineto{\pgfqpoint{1.749024in}{3.204433in}}%
\pgfpathlineto{\pgfqpoint{1.749024in}{3.201484in}}%
\pgfpathmoveto{\pgfqpoint{1.753565in}{3.198534in}}%
\pgfpathlineto{\pgfqpoint{1.753565in}{3.198534in}}%
\pgfpathlineto{\pgfqpoint{1.753565in}{3.201484in}}%
\pgfpathlineto{\pgfqpoint{1.758106in}{3.201484in}}%
\pgfpathlineto{\pgfqpoint{1.758106in}{3.198534in}}%
\pgfpathmoveto{\pgfqpoint{1.758106in}{3.198534in}}%
\pgfpathlineto{\pgfqpoint{1.758106in}{3.198534in}}%
\pgfpathlineto{\pgfqpoint{1.758106in}{3.201484in}}%
\pgfpathlineto{\pgfqpoint{1.762647in}{3.201484in}}%
\pgfpathlineto{\pgfqpoint{1.762647in}{3.198534in}}%
\pgfpathmoveto{\pgfqpoint{1.762647in}{3.198534in}}%
\pgfpathlineto{\pgfqpoint{1.762647in}{3.198534in}}%
\pgfpathlineto{\pgfqpoint{1.762647in}{3.201484in}}%
\pgfpathlineto{\pgfqpoint{1.767188in}{3.201484in}}%
\pgfpathlineto{\pgfqpoint{1.767188in}{3.198534in}}%
\pgfpathmoveto{\pgfqpoint{1.749024in}{3.201484in}}%
\pgfpathlineto{\pgfqpoint{1.749024in}{3.201484in}}%
\pgfpathlineto{\pgfqpoint{1.749024in}{3.204433in}}%
\pgfpathlineto{\pgfqpoint{1.753565in}{3.204433in}}%
\pgfpathlineto{\pgfqpoint{1.753565in}{3.201484in}}%
\pgfpathmoveto{\pgfqpoint{1.753565in}{3.201484in}}%
\pgfpathlineto{\pgfqpoint{1.753565in}{3.201484in}}%
\pgfpathlineto{\pgfqpoint{1.753565in}{3.204433in}}%
\pgfpathlineto{\pgfqpoint{1.758106in}{3.204433in}}%
\pgfpathlineto{\pgfqpoint{1.758106in}{3.201484in}}%
\pgfpathmoveto{\pgfqpoint{1.808058in}{3.186737in}}%
\pgfpathlineto{\pgfqpoint{1.808058in}{3.186737in}}%
\pgfpathlineto{\pgfqpoint{1.808058in}{3.189687in}}%
\pgfpathlineto{\pgfqpoint{1.812599in}{3.189687in}}%
\pgfpathlineto{\pgfqpoint{1.812599in}{3.186737in}}%
\pgfpathmoveto{\pgfqpoint{1.812599in}{3.186737in}}%
\pgfpathlineto{\pgfqpoint{1.812599in}{3.186737in}}%
\pgfpathlineto{\pgfqpoint{1.812599in}{3.189687in}}%
\pgfpathlineto{\pgfqpoint{1.817140in}{3.189687in}}%
\pgfpathlineto{\pgfqpoint{1.817140in}{3.186737in}}%
\pgfpathmoveto{\pgfqpoint{1.817140in}{3.186737in}}%
\pgfpathlineto{\pgfqpoint{1.817140in}{3.186737in}}%
\pgfpathlineto{\pgfqpoint{1.817140in}{3.189687in}}%
\pgfpathlineto{\pgfqpoint{1.821681in}{3.189687in}}%
\pgfpathlineto{\pgfqpoint{1.821681in}{3.186737in}}%
\pgfpathmoveto{\pgfqpoint{1.821681in}{3.183788in}}%
\pgfpathlineto{\pgfqpoint{1.821681in}{3.183788in}}%
\pgfpathlineto{\pgfqpoint{1.821681in}{3.186737in}}%
\pgfpathlineto{\pgfqpoint{1.826222in}{3.186737in}}%
\pgfpathlineto{\pgfqpoint{1.826222in}{3.183788in}}%
\pgfpathmoveto{\pgfqpoint{1.821681in}{3.186737in}}%
\pgfpathlineto{\pgfqpoint{1.821681in}{3.186737in}}%
\pgfpathlineto{\pgfqpoint{1.821681in}{3.189687in}}%
\pgfpathlineto{\pgfqpoint{1.826222in}{3.189687in}}%
\pgfpathlineto{\pgfqpoint{1.826222in}{3.186737in}}%
\pgfpathmoveto{\pgfqpoint{1.826222in}{3.183788in}}%
\pgfpathlineto{\pgfqpoint{1.826222in}{3.183788in}}%
\pgfpathlineto{\pgfqpoint{1.826222in}{3.186737in}}%
\pgfpathlineto{\pgfqpoint{1.830763in}{3.186737in}}%
\pgfpathlineto{\pgfqpoint{1.830763in}{3.183788in}}%
\pgfpathmoveto{\pgfqpoint{1.835304in}{3.180839in}}%
\pgfpathlineto{\pgfqpoint{1.835304in}{3.180839in}}%
\pgfpathlineto{\pgfqpoint{1.835304in}{3.183788in}}%
\pgfpathlineto{\pgfqpoint{1.839846in}{3.183788in}}%
\pgfpathlineto{\pgfqpoint{1.839846in}{3.180839in}}%
\pgfpathmoveto{\pgfqpoint{1.830763in}{3.183788in}}%
\pgfpathlineto{\pgfqpoint{1.830763in}{3.183788in}}%
\pgfpathlineto{\pgfqpoint{1.830763in}{3.186737in}}%
\pgfpathlineto{\pgfqpoint{1.835304in}{3.186737in}}%
\pgfpathlineto{\pgfqpoint{1.835304in}{3.183788in}}%
\pgfpathmoveto{\pgfqpoint{1.835304in}{3.183788in}}%
\pgfpathlineto{\pgfqpoint{1.835304in}{3.183788in}}%
\pgfpathlineto{\pgfqpoint{1.835304in}{3.186737in}}%
\pgfpathlineto{\pgfqpoint{1.839846in}{3.186737in}}%
\pgfpathlineto{\pgfqpoint{1.839846in}{3.183788in}}%
\pgfpathmoveto{\pgfqpoint{1.767188in}{3.195585in}}%
\pgfpathlineto{\pgfqpoint{1.767188in}{3.195585in}}%
\pgfpathlineto{\pgfqpoint{1.767188in}{3.198534in}}%
\pgfpathlineto{\pgfqpoint{1.771729in}{3.198534in}}%
\pgfpathlineto{\pgfqpoint{1.771729in}{3.195585in}}%
\pgfpathmoveto{\pgfqpoint{1.767188in}{3.198534in}}%
\pgfpathlineto{\pgfqpoint{1.767188in}{3.198534in}}%
\pgfpathlineto{\pgfqpoint{1.767188in}{3.201484in}}%
\pgfpathlineto{\pgfqpoint{1.771729in}{3.201484in}}%
\pgfpathlineto{\pgfqpoint{1.771729in}{3.198534in}}%
\pgfpathmoveto{\pgfqpoint{1.771729in}{3.195585in}}%
\pgfpathlineto{\pgfqpoint{1.771729in}{3.195585in}}%
\pgfpathlineto{\pgfqpoint{1.771729in}{3.198534in}}%
\pgfpathlineto{\pgfqpoint{1.776271in}{3.198534in}}%
\pgfpathlineto{\pgfqpoint{1.776271in}{3.195585in}}%
\pgfpathmoveto{\pgfqpoint{1.780812in}{3.192636in}}%
\pgfpathlineto{\pgfqpoint{1.780812in}{3.192636in}}%
\pgfpathlineto{\pgfqpoint{1.780812in}{3.195585in}}%
\pgfpathlineto{\pgfqpoint{1.785353in}{3.195585in}}%
\pgfpathlineto{\pgfqpoint{1.785353in}{3.192636in}}%
\pgfpathmoveto{\pgfqpoint{1.776271in}{3.195585in}}%
\pgfpathlineto{\pgfqpoint{1.776271in}{3.195585in}}%
\pgfpathlineto{\pgfqpoint{1.776271in}{3.198534in}}%
\pgfpathlineto{\pgfqpoint{1.780812in}{3.198534in}}%
\pgfpathlineto{\pgfqpoint{1.780812in}{3.195585in}}%
\pgfpathmoveto{\pgfqpoint{1.780812in}{3.195585in}}%
\pgfpathlineto{\pgfqpoint{1.780812in}{3.195585in}}%
\pgfpathlineto{\pgfqpoint{1.780812in}{3.198534in}}%
\pgfpathlineto{\pgfqpoint{1.785353in}{3.198534in}}%
\pgfpathlineto{\pgfqpoint{1.785353in}{3.195585in}}%
\pgfpathmoveto{\pgfqpoint{1.785353in}{3.192636in}}%
\pgfpathlineto{\pgfqpoint{1.785353in}{3.192636in}}%
\pgfpathlineto{\pgfqpoint{1.785353in}{3.195585in}}%
\pgfpathlineto{\pgfqpoint{1.789894in}{3.195585in}}%
\pgfpathlineto{\pgfqpoint{1.789894in}{3.192636in}}%
\pgfpathmoveto{\pgfqpoint{1.789894in}{3.192636in}}%
\pgfpathlineto{\pgfqpoint{1.789894in}{3.192636in}}%
\pgfpathlineto{\pgfqpoint{1.789894in}{3.195585in}}%
\pgfpathlineto{\pgfqpoint{1.794435in}{3.195585in}}%
\pgfpathlineto{\pgfqpoint{1.794435in}{3.192636in}}%
\pgfpathmoveto{\pgfqpoint{1.794435in}{3.189687in}}%
\pgfpathlineto{\pgfqpoint{1.794435in}{3.189687in}}%
\pgfpathlineto{\pgfqpoint{1.794435in}{3.192636in}}%
\pgfpathlineto{\pgfqpoint{1.798976in}{3.192636in}}%
\pgfpathlineto{\pgfqpoint{1.798976in}{3.189687in}}%
\pgfpathmoveto{\pgfqpoint{1.794435in}{3.192636in}}%
\pgfpathlineto{\pgfqpoint{1.794435in}{3.192636in}}%
\pgfpathlineto{\pgfqpoint{1.794435in}{3.195585in}}%
\pgfpathlineto{\pgfqpoint{1.798976in}{3.195585in}}%
\pgfpathlineto{\pgfqpoint{1.798976in}{3.192636in}}%
\pgfpathmoveto{\pgfqpoint{1.798976in}{3.189687in}}%
\pgfpathlineto{\pgfqpoint{1.798976in}{3.189687in}}%
\pgfpathlineto{\pgfqpoint{1.798976in}{3.192636in}}%
\pgfpathlineto{\pgfqpoint{1.803517in}{3.192636in}}%
\pgfpathlineto{\pgfqpoint{1.803517in}{3.189687in}}%
\pgfpathmoveto{\pgfqpoint{1.803517in}{3.189687in}}%
\pgfpathlineto{\pgfqpoint{1.803517in}{3.189687in}}%
\pgfpathlineto{\pgfqpoint{1.803517in}{3.192636in}}%
\pgfpathlineto{\pgfqpoint{1.808058in}{3.192636in}}%
\pgfpathlineto{\pgfqpoint{1.808058in}{3.189687in}}%
\pgfpathmoveto{\pgfqpoint{1.808058in}{3.189687in}}%
\pgfpathlineto{\pgfqpoint{1.808058in}{3.189687in}}%
\pgfpathlineto{\pgfqpoint{1.808058in}{3.192636in}}%
\pgfpathlineto{\pgfqpoint{1.812599in}{3.192636in}}%
\pgfpathlineto{\pgfqpoint{1.812599in}{3.189687in}}%
\pgfpathmoveto{\pgfqpoint{1.839846in}{3.180839in}}%
\pgfpathlineto{\pgfqpoint{1.839846in}{3.180839in}}%
\pgfpathlineto{\pgfqpoint{1.839846in}{3.183788in}}%
\pgfpathlineto{\pgfqpoint{1.844387in}{3.183788in}}%
\pgfpathlineto{\pgfqpoint{1.844387in}{3.180839in}}%
\pgfpathmoveto{\pgfqpoint{1.844387in}{3.180839in}}%
\pgfpathlineto{\pgfqpoint{1.844387in}{3.180839in}}%
\pgfpathlineto{\pgfqpoint{1.844387in}{3.183788in}}%
\pgfpathlineto{\pgfqpoint{1.848928in}{3.183788in}}%
\pgfpathlineto{\pgfqpoint{1.848928in}{3.180839in}}%
\pgfpathmoveto{\pgfqpoint{1.848928in}{3.177890in}}%
\pgfpathlineto{\pgfqpoint{1.848928in}{3.177890in}}%
\pgfpathlineto{\pgfqpoint{1.848928in}{3.180839in}}%
\pgfpathlineto{\pgfqpoint{1.853469in}{3.180839in}}%
\pgfpathlineto{\pgfqpoint{1.853469in}{3.177890in}}%
\pgfpathmoveto{\pgfqpoint{1.848928in}{3.180839in}}%
\pgfpathlineto{\pgfqpoint{1.848928in}{3.180839in}}%
\pgfpathlineto{\pgfqpoint{1.848928in}{3.183788in}}%
\pgfpathlineto{\pgfqpoint{1.853469in}{3.183788in}}%
\pgfpathlineto{\pgfqpoint{1.853469in}{3.180839in}}%
\pgfpathmoveto{\pgfqpoint{1.853469in}{3.177890in}}%
\pgfpathlineto{\pgfqpoint{1.853469in}{3.177890in}}%
\pgfpathlineto{\pgfqpoint{1.853469in}{3.180839in}}%
\pgfpathlineto{\pgfqpoint{1.858010in}{3.180839in}}%
\pgfpathlineto{\pgfqpoint{1.858010in}{3.177890in}}%
\pgfpathmoveto{\pgfqpoint{1.862551in}{3.174940in}}%
\pgfpathlineto{\pgfqpoint{1.862551in}{3.174940in}}%
\pgfpathlineto{\pgfqpoint{1.862551in}{3.177890in}}%
\pgfpathlineto{\pgfqpoint{1.867092in}{3.177890in}}%
\pgfpathlineto{\pgfqpoint{1.867092in}{3.174940in}}%
\pgfpathmoveto{\pgfqpoint{1.867092in}{3.174940in}}%
\pgfpathlineto{\pgfqpoint{1.867092in}{3.174940in}}%
\pgfpathlineto{\pgfqpoint{1.867092in}{3.177890in}}%
\pgfpathlineto{\pgfqpoint{1.871633in}{3.177890in}}%
\pgfpathlineto{\pgfqpoint{1.871633in}{3.174940in}}%
\pgfpathmoveto{\pgfqpoint{1.871633in}{3.174940in}}%
\pgfpathlineto{\pgfqpoint{1.871633in}{3.174940in}}%
\pgfpathlineto{\pgfqpoint{1.871633in}{3.177890in}}%
\pgfpathlineto{\pgfqpoint{1.876174in}{3.177890in}}%
\pgfpathlineto{\pgfqpoint{1.876174in}{3.174940in}}%
\pgfpathmoveto{\pgfqpoint{1.858010in}{3.177890in}}%
\pgfpathlineto{\pgfqpoint{1.858010in}{3.177890in}}%
\pgfpathlineto{\pgfqpoint{1.858010in}{3.180839in}}%
\pgfpathlineto{\pgfqpoint{1.862551in}{3.180839in}}%
\pgfpathlineto{\pgfqpoint{1.862551in}{3.177890in}}%
\pgfpathmoveto{\pgfqpoint{1.862551in}{3.177890in}}%
\pgfpathlineto{\pgfqpoint{1.862551in}{3.177890in}}%
\pgfpathlineto{\pgfqpoint{1.862551in}{3.180839in}}%
\pgfpathlineto{\pgfqpoint{1.867092in}{3.180839in}}%
\pgfpathlineto{\pgfqpoint{1.867092in}{3.177890in}}%
\pgfpathmoveto{\pgfqpoint{1.876174in}{3.171991in}}%
\pgfpathlineto{\pgfqpoint{1.876174in}{3.171991in}}%
\pgfpathlineto{\pgfqpoint{1.876174in}{3.174940in}}%
\pgfpathlineto{\pgfqpoint{1.880715in}{3.174940in}}%
\pgfpathlineto{\pgfqpoint{1.880715in}{3.171991in}}%
\pgfpathmoveto{\pgfqpoint{1.876174in}{3.174940in}}%
\pgfpathlineto{\pgfqpoint{1.876174in}{3.174940in}}%
\pgfpathlineto{\pgfqpoint{1.876174in}{3.177890in}}%
\pgfpathlineto{\pgfqpoint{1.880715in}{3.177890in}}%
\pgfpathlineto{\pgfqpoint{1.880715in}{3.174940in}}%
\pgfpathmoveto{\pgfqpoint{1.880715in}{3.171991in}}%
\pgfpathlineto{\pgfqpoint{1.880715in}{3.171991in}}%
\pgfpathlineto{\pgfqpoint{1.880715in}{3.174940in}}%
\pgfpathlineto{\pgfqpoint{1.885256in}{3.174940in}}%
\pgfpathlineto{\pgfqpoint{1.885256in}{3.171991in}}%
\pgfpathmoveto{\pgfqpoint{1.889797in}{3.169042in}}%
\pgfpathlineto{\pgfqpoint{1.889797in}{3.169042in}}%
\pgfpathlineto{\pgfqpoint{1.889797in}{3.171991in}}%
\pgfpathlineto{\pgfqpoint{1.894338in}{3.171991in}}%
\pgfpathlineto{\pgfqpoint{1.894338in}{3.169042in}}%
\pgfpathmoveto{\pgfqpoint{1.885256in}{3.171991in}}%
\pgfpathlineto{\pgfqpoint{1.885256in}{3.171991in}}%
\pgfpathlineto{\pgfqpoint{1.885256in}{3.174940in}}%
\pgfpathlineto{\pgfqpoint{1.889797in}{3.174940in}}%
\pgfpathlineto{\pgfqpoint{1.889797in}{3.171991in}}%
\pgfpathmoveto{\pgfqpoint{1.889797in}{3.171991in}}%
\pgfpathlineto{\pgfqpoint{1.889797in}{3.171991in}}%
\pgfpathlineto{\pgfqpoint{1.889797in}{3.174940in}}%
\pgfpathlineto{\pgfqpoint{1.894338in}{3.174940in}}%
\pgfpathlineto{\pgfqpoint{1.894338in}{3.171991in}}%
\pgfpathmoveto{\pgfqpoint{1.894338in}{3.169042in}}%
\pgfpathlineto{\pgfqpoint{1.894338in}{3.169042in}}%
\pgfpathlineto{\pgfqpoint{1.894338in}{3.171991in}}%
\pgfpathlineto{\pgfqpoint{1.898879in}{3.171991in}}%
\pgfpathlineto{\pgfqpoint{1.898879in}{3.169042in}}%
\pgfpathmoveto{\pgfqpoint{1.898879in}{3.169042in}}%
\pgfpathlineto{\pgfqpoint{1.898879in}{3.169042in}}%
\pgfpathlineto{\pgfqpoint{1.898879in}{3.171991in}}%
\pgfpathlineto{\pgfqpoint{1.903421in}{3.171991in}}%
\pgfpathlineto{\pgfqpoint{1.903421in}{3.169042in}}%
\pgfpathmoveto{\pgfqpoint{1.903421in}{3.166093in}}%
\pgfpathlineto{\pgfqpoint{1.903421in}{3.166093in}}%
\pgfpathlineto{\pgfqpoint{1.903421in}{3.169042in}}%
\pgfpathlineto{\pgfqpoint{1.907962in}{3.169042in}}%
\pgfpathlineto{\pgfqpoint{1.907962in}{3.166093in}}%
\pgfpathmoveto{\pgfqpoint{1.903421in}{3.169042in}}%
\pgfpathlineto{\pgfqpoint{1.903421in}{3.169042in}}%
\pgfpathlineto{\pgfqpoint{1.903421in}{3.171991in}}%
\pgfpathlineto{\pgfqpoint{1.907962in}{3.171991in}}%
\pgfpathlineto{\pgfqpoint{1.907962in}{3.169042in}}%
\pgfpathmoveto{\pgfqpoint{1.907962in}{3.166093in}}%
\pgfpathlineto{\pgfqpoint{1.907962in}{3.166093in}}%
\pgfpathlineto{\pgfqpoint{1.907962in}{3.169042in}}%
\pgfpathlineto{\pgfqpoint{1.912503in}{3.169042in}}%
\pgfpathlineto{\pgfqpoint{1.912503in}{3.166093in}}%
\pgfpathmoveto{\pgfqpoint{2.026024in}{3.139549in}}%
\pgfpathlineto{\pgfqpoint{2.026024in}{3.139549in}}%
\pgfpathlineto{\pgfqpoint{2.026024in}{3.142499in}}%
\pgfpathlineto{\pgfqpoint{2.030565in}{3.142499in}}%
\pgfpathlineto{\pgfqpoint{2.030565in}{3.139549in}}%
\pgfpathmoveto{\pgfqpoint{2.030565in}{3.139549in}}%
\pgfpathlineto{\pgfqpoint{2.030565in}{3.139549in}}%
\pgfpathlineto{\pgfqpoint{2.030565in}{3.142499in}}%
\pgfpathlineto{\pgfqpoint{2.035106in}{3.142499in}}%
\pgfpathlineto{\pgfqpoint{2.035106in}{3.139549in}}%
\pgfpathmoveto{\pgfqpoint{2.035106in}{3.139549in}}%
\pgfpathlineto{\pgfqpoint{2.035106in}{3.139549in}}%
\pgfpathlineto{\pgfqpoint{2.035106in}{3.142499in}}%
\pgfpathlineto{\pgfqpoint{2.039646in}{3.142499in}}%
\pgfpathlineto{\pgfqpoint{2.039646in}{3.139549in}}%
\pgfpathmoveto{\pgfqpoint{2.039646in}{3.136600in}}%
\pgfpathlineto{\pgfqpoint{2.039646in}{3.136600in}}%
\pgfpathlineto{\pgfqpoint{2.039646in}{3.139549in}}%
\pgfpathlineto{\pgfqpoint{2.044187in}{3.139549in}}%
\pgfpathlineto{\pgfqpoint{2.044187in}{3.136600in}}%
\pgfpathmoveto{\pgfqpoint{2.039646in}{3.139549in}}%
\pgfpathlineto{\pgfqpoint{2.039646in}{3.139549in}}%
\pgfpathlineto{\pgfqpoint{2.039646in}{3.142499in}}%
\pgfpathlineto{\pgfqpoint{2.044187in}{3.142499in}}%
\pgfpathlineto{\pgfqpoint{2.044187in}{3.139549in}}%
\pgfpathmoveto{\pgfqpoint{2.044187in}{3.136600in}}%
\pgfpathlineto{\pgfqpoint{2.044187in}{3.136600in}}%
\pgfpathlineto{\pgfqpoint{2.044187in}{3.139549in}}%
\pgfpathlineto{\pgfqpoint{2.048728in}{3.139549in}}%
\pgfpathlineto{\pgfqpoint{2.048728in}{3.136600in}}%
\pgfpathmoveto{\pgfqpoint{2.053269in}{3.133651in}}%
\pgfpathlineto{\pgfqpoint{2.053269in}{3.133651in}}%
\pgfpathlineto{\pgfqpoint{2.053269in}{3.136600in}}%
\pgfpathlineto{\pgfqpoint{2.057810in}{3.136600in}}%
\pgfpathlineto{\pgfqpoint{2.057810in}{3.133651in}}%
\pgfpathmoveto{\pgfqpoint{2.048728in}{3.136600in}}%
\pgfpathlineto{\pgfqpoint{2.048728in}{3.136600in}}%
\pgfpathlineto{\pgfqpoint{2.048728in}{3.139549in}}%
\pgfpathlineto{\pgfqpoint{2.053269in}{3.139549in}}%
\pgfpathlineto{\pgfqpoint{2.053269in}{3.136600in}}%
\pgfpathmoveto{\pgfqpoint{2.053269in}{3.136600in}}%
\pgfpathlineto{\pgfqpoint{2.053269in}{3.136600in}}%
\pgfpathlineto{\pgfqpoint{2.053269in}{3.139549in}}%
\pgfpathlineto{\pgfqpoint{2.057810in}{3.139549in}}%
\pgfpathlineto{\pgfqpoint{2.057810in}{3.136600in}}%
\pgfpathmoveto{\pgfqpoint{1.917043in}{3.163143in}}%
\pgfpathlineto{\pgfqpoint{1.917043in}{3.163143in}}%
\pgfpathlineto{\pgfqpoint{1.917043in}{3.166093in}}%
\pgfpathlineto{\pgfqpoint{1.921584in}{3.166093in}}%
\pgfpathlineto{\pgfqpoint{1.921584in}{3.163143in}}%
\pgfpathmoveto{\pgfqpoint{1.921584in}{3.163143in}}%
\pgfpathlineto{\pgfqpoint{1.921584in}{3.163143in}}%
\pgfpathlineto{\pgfqpoint{1.921584in}{3.166093in}}%
\pgfpathlineto{\pgfqpoint{1.926125in}{3.166093in}}%
\pgfpathlineto{\pgfqpoint{1.926125in}{3.163143in}}%
\pgfpathmoveto{\pgfqpoint{1.926125in}{3.163143in}}%
\pgfpathlineto{\pgfqpoint{1.926125in}{3.163143in}}%
\pgfpathlineto{\pgfqpoint{1.926125in}{3.166093in}}%
\pgfpathlineto{\pgfqpoint{1.930666in}{3.166093in}}%
\pgfpathlineto{\pgfqpoint{1.930666in}{3.163143in}}%
\pgfpathmoveto{\pgfqpoint{1.930666in}{3.160194in}}%
\pgfpathlineto{\pgfqpoint{1.930666in}{3.160194in}}%
\pgfpathlineto{\pgfqpoint{1.930666in}{3.163143in}}%
\pgfpathlineto{\pgfqpoint{1.935207in}{3.163143in}}%
\pgfpathlineto{\pgfqpoint{1.935207in}{3.160194in}}%
\pgfpathmoveto{\pgfqpoint{1.930666in}{3.163143in}}%
\pgfpathlineto{\pgfqpoint{1.930666in}{3.163143in}}%
\pgfpathlineto{\pgfqpoint{1.930666in}{3.166093in}}%
\pgfpathlineto{\pgfqpoint{1.935207in}{3.166093in}}%
\pgfpathlineto{\pgfqpoint{1.935207in}{3.163143in}}%
\pgfpathmoveto{\pgfqpoint{1.935207in}{3.160194in}}%
\pgfpathlineto{\pgfqpoint{1.935207in}{3.160194in}}%
\pgfpathlineto{\pgfqpoint{1.935207in}{3.163143in}}%
\pgfpathlineto{\pgfqpoint{1.939748in}{3.163143in}}%
\pgfpathlineto{\pgfqpoint{1.939748in}{3.160194in}}%
\pgfpathmoveto{\pgfqpoint{1.944289in}{3.157245in}}%
\pgfpathlineto{\pgfqpoint{1.944289in}{3.157245in}}%
\pgfpathlineto{\pgfqpoint{1.944289in}{3.160194in}}%
\pgfpathlineto{\pgfqpoint{1.948829in}{3.160194in}}%
\pgfpathlineto{\pgfqpoint{1.948829in}{3.157245in}}%
\pgfpathmoveto{\pgfqpoint{1.939748in}{3.160194in}}%
\pgfpathlineto{\pgfqpoint{1.939748in}{3.160194in}}%
\pgfpathlineto{\pgfqpoint{1.939748in}{3.163143in}}%
\pgfpathlineto{\pgfqpoint{1.944289in}{3.163143in}}%
\pgfpathlineto{\pgfqpoint{1.944289in}{3.160194in}}%
\pgfpathmoveto{\pgfqpoint{1.944289in}{3.160194in}}%
\pgfpathlineto{\pgfqpoint{1.944289in}{3.160194in}}%
\pgfpathlineto{\pgfqpoint{1.944289in}{3.163143in}}%
\pgfpathlineto{\pgfqpoint{1.948829in}{3.163143in}}%
\pgfpathlineto{\pgfqpoint{1.948829in}{3.160194in}}%
\pgfpathmoveto{\pgfqpoint{1.912503in}{3.166093in}}%
\pgfpathlineto{\pgfqpoint{1.912503in}{3.166093in}}%
\pgfpathlineto{\pgfqpoint{1.912503in}{3.169042in}}%
\pgfpathlineto{\pgfqpoint{1.917043in}{3.169042in}}%
\pgfpathlineto{\pgfqpoint{1.917043in}{3.166093in}}%
\pgfpathmoveto{\pgfqpoint{1.917043in}{3.166093in}}%
\pgfpathlineto{\pgfqpoint{1.917043in}{3.166093in}}%
\pgfpathlineto{\pgfqpoint{1.917043in}{3.169042in}}%
\pgfpathlineto{\pgfqpoint{1.921584in}{3.169042in}}%
\pgfpathlineto{\pgfqpoint{1.921584in}{3.166093in}}%
\pgfpathmoveto{\pgfqpoint{1.948829in}{3.157245in}}%
\pgfpathlineto{\pgfqpoint{1.948829in}{3.157245in}}%
\pgfpathlineto{\pgfqpoint{1.948829in}{3.160194in}}%
\pgfpathlineto{\pgfqpoint{1.953370in}{3.160194in}}%
\pgfpathlineto{\pgfqpoint{1.953370in}{3.157245in}}%
\pgfpathmoveto{\pgfqpoint{1.953370in}{3.157245in}}%
\pgfpathlineto{\pgfqpoint{1.953370in}{3.157245in}}%
\pgfpathlineto{\pgfqpoint{1.953370in}{3.160194in}}%
\pgfpathlineto{\pgfqpoint{1.957911in}{3.160194in}}%
\pgfpathlineto{\pgfqpoint{1.957911in}{3.157245in}}%
\pgfpathmoveto{\pgfqpoint{1.957911in}{3.154296in}}%
\pgfpathlineto{\pgfqpoint{1.957911in}{3.154296in}}%
\pgfpathlineto{\pgfqpoint{1.957911in}{3.157245in}}%
\pgfpathlineto{\pgfqpoint{1.962452in}{3.157245in}}%
\pgfpathlineto{\pgfqpoint{1.962452in}{3.154296in}}%
\pgfpathmoveto{\pgfqpoint{1.957911in}{3.157245in}}%
\pgfpathlineto{\pgfqpoint{1.957911in}{3.157245in}}%
\pgfpathlineto{\pgfqpoint{1.957911in}{3.160194in}}%
\pgfpathlineto{\pgfqpoint{1.962452in}{3.160194in}}%
\pgfpathlineto{\pgfqpoint{1.962452in}{3.157245in}}%
\pgfpathmoveto{\pgfqpoint{1.962452in}{3.154296in}}%
\pgfpathlineto{\pgfqpoint{1.962452in}{3.154296in}}%
\pgfpathlineto{\pgfqpoint{1.962452in}{3.157245in}}%
\pgfpathlineto{\pgfqpoint{1.966993in}{3.157245in}}%
\pgfpathlineto{\pgfqpoint{1.966993in}{3.154296in}}%
\pgfpathmoveto{\pgfqpoint{1.971534in}{3.151346in}}%
\pgfpathlineto{\pgfqpoint{1.971534in}{3.151346in}}%
\pgfpathlineto{\pgfqpoint{1.971534in}{3.154296in}}%
\pgfpathlineto{\pgfqpoint{1.976075in}{3.154296in}}%
\pgfpathlineto{\pgfqpoint{1.976075in}{3.151346in}}%
\pgfpathmoveto{\pgfqpoint{1.976075in}{3.151346in}}%
\pgfpathlineto{\pgfqpoint{1.976075in}{3.151346in}}%
\pgfpathlineto{\pgfqpoint{1.976075in}{3.154296in}}%
\pgfpathlineto{\pgfqpoint{1.980615in}{3.154296in}}%
\pgfpathlineto{\pgfqpoint{1.980615in}{3.151346in}}%
\pgfpathmoveto{\pgfqpoint{1.980615in}{3.151346in}}%
\pgfpathlineto{\pgfqpoint{1.980615in}{3.151346in}}%
\pgfpathlineto{\pgfqpoint{1.980615in}{3.154296in}}%
\pgfpathlineto{\pgfqpoint{1.985156in}{3.154296in}}%
\pgfpathlineto{\pgfqpoint{1.985156in}{3.151346in}}%
\pgfpathmoveto{\pgfqpoint{1.966993in}{3.154296in}}%
\pgfpathlineto{\pgfqpoint{1.966993in}{3.154296in}}%
\pgfpathlineto{\pgfqpoint{1.966993in}{3.157245in}}%
\pgfpathlineto{\pgfqpoint{1.971534in}{3.157245in}}%
\pgfpathlineto{\pgfqpoint{1.971534in}{3.154296in}}%
\pgfpathmoveto{\pgfqpoint{1.971534in}{3.154296in}}%
\pgfpathlineto{\pgfqpoint{1.971534in}{3.154296in}}%
\pgfpathlineto{\pgfqpoint{1.971534in}{3.157245in}}%
\pgfpathlineto{\pgfqpoint{1.976075in}{3.157245in}}%
\pgfpathlineto{\pgfqpoint{1.976075in}{3.154296in}}%
\pgfpathmoveto{\pgfqpoint{1.985156in}{3.148397in}}%
\pgfpathlineto{\pgfqpoint{1.985156in}{3.148397in}}%
\pgfpathlineto{\pgfqpoint{1.985156in}{3.151346in}}%
\pgfpathlineto{\pgfqpoint{1.989697in}{3.151346in}}%
\pgfpathlineto{\pgfqpoint{1.989697in}{3.148397in}}%
\pgfpathmoveto{\pgfqpoint{1.985156in}{3.151346in}}%
\pgfpathlineto{\pgfqpoint{1.985156in}{3.151346in}}%
\pgfpathlineto{\pgfqpoint{1.985156in}{3.154296in}}%
\pgfpathlineto{\pgfqpoint{1.989697in}{3.154296in}}%
\pgfpathlineto{\pgfqpoint{1.989697in}{3.151346in}}%
\pgfpathmoveto{\pgfqpoint{1.989697in}{3.148397in}}%
\pgfpathlineto{\pgfqpoint{1.989697in}{3.148397in}}%
\pgfpathlineto{\pgfqpoint{1.989697in}{3.151346in}}%
\pgfpathlineto{\pgfqpoint{1.994238in}{3.151346in}}%
\pgfpathlineto{\pgfqpoint{1.994238in}{3.148397in}}%
\pgfpathmoveto{\pgfqpoint{1.998779in}{3.145448in}}%
\pgfpathlineto{\pgfqpoint{1.998779in}{3.145448in}}%
\pgfpathlineto{\pgfqpoint{1.998779in}{3.148397in}}%
\pgfpathlineto{\pgfqpoint{2.003320in}{3.148397in}}%
\pgfpathlineto{\pgfqpoint{2.003320in}{3.145448in}}%
\pgfpathmoveto{\pgfqpoint{1.994238in}{3.148397in}}%
\pgfpathlineto{\pgfqpoint{1.994238in}{3.148397in}}%
\pgfpathlineto{\pgfqpoint{1.994238in}{3.151346in}}%
\pgfpathlineto{\pgfqpoint{1.998779in}{3.151346in}}%
\pgfpathlineto{\pgfqpoint{1.998779in}{3.148397in}}%
\pgfpathmoveto{\pgfqpoint{1.998779in}{3.148397in}}%
\pgfpathlineto{\pgfqpoint{1.998779in}{3.148397in}}%
\pgfpathlineto{\pgfqpoint{1.998779in}{3.151346in}}%
\pgfpathlineto{\pgfqpoint{2.003320in}{3.151346in}}%
\pgfpathlineto{\pgfqpoint{2.003320in}{3.148397in}}%
\pgfpathmoveto{\pgfqpoint{2.003320in}{3.145448in}}%
\pgfpathlineto{\pgfqpoint{2.003320in}{3.145448in}}%
\pgfpathlineto{\pgfqpoint{2.003320in}{3.148397in}}%
\pgfpathlineto{\pgfqpoint{2.007860in}{3.148397in}}%
\pgfpathlineto{\pgfqpoint{2.007860in}{3.145448in}}%
\pgfpathmoveto{\pgfqpoint{2.007860in}{3.145448in}}%
\pgfpathlineto{\pgfqpoint{2.007860in}{3.145448in}}%
\pgfpathlineto{\pgfqpoint{2.007860in}{3.148397in}}%
\pgfpathlineto{\pgfqpoint{2.012401in}{3.148397in}}%
\pgfpathlineto{\pgfqpoint{2.012401in}{3.145448in}}%
\pgfpathmoveto{\pgfqpoint{2.012401in}{3.142499in}}%
\pgfpathlineto{\pgfqpoint{2.012401in}{3.142499in}}%
\pgfpathlineto{\pgfqpoint{2.012401in}{3.145448in}}%
\pgfpathlineto{\pgfqpoint{2.016942in}{3.145448in}}%
\pgfpathlineto{\pgfqpoint{2.016942in}{3.142499in}}%
\pgfpathmoveto{\pgfqpoint{2.012401in}{3.145448in}}%
\pgfpathlineto{\pgfqpoint{2.012401in}{3.145448in}}%
\pgfpathlineto{\pgfqpoint{2.012401in}{3.148397in}}%
\pgfpathlineto{\pgfqpoint{2.016942in}{3.148397in}}%
\pgfpathlineto{\pgfqpoint{2.016942in}{3.145448in}}%
\pgfpathmoveto{\pgfqpoint{2.016942in}{3.142499in}}%
\pgfpathlineto{\pgfqpoint{2.016942in}{3.142499in}}%
\pgfpathlineto{\pgfqpoint{2.016942in}{3.145448in}}%
\pgfpathlineto{\pgfqpoint{2.021483in}{3.145448in}}%
\pgfpathlineto{\pgfqpoint{2.021483in}{3.142499in}}%
\pgfpathmoveto{\pgfqpoint{2.021483in}{3.142499in}}%
\pgfpathlineto{\pgfqpoint{2.021483in}{3.142499in}}%
\pgfpathlineto{\pgfqpoint{2.021483in}{3.145448in}}%
\pgfpathlineto{\pgfqpoint{2.026024in}{3.145448in}}%
\pgfpathlineto{\pgfqpoint{2.026024in}{3.142499in}}%
\pgfpathmoveto{\pgfqpoint{2.026024in}{3.142499in}}%
\pgfpathlineto{\pgfqpoint{2.026024in}{3.142499in}}%
\pgfpathlineto{\pgfqpoint{2.026024in}{3.145448in}}%
\pgfpathlineto{\pgfqpoint{2.030565in}{3.145448in}}%
\pgfpathlineto{\pgfqpoint{2.030565in}{3.142499in}}%
\pgfpathmoveto{\pgfqpoint{2.057810in}{3.133651in}}%
\pgfpathlineto{\pgfqpoint{2.057810in}{3.133651in}}%
\pgfpathlineto{\pgfqpoint{2.057810in}{3.136600in}}%
\pgfpathlineto{\pgfqpoint{2.062351in}{3.136600in}}%
\pgfpathlineto{\pgfqpoint{2.062351in}{3.133651in}}%
\pgfpathmoveto{\pgfqpoint{2.062351in}{3.133651in}}%
\pgfpathlineto{\pgfqpoint{2.062351in}{3.133651in}}%
\pgfpathlineto{\pgfqpoint{2.062351in}{3.136600in}}%
\pgfpathlineto{\pgfqpoint{2.066892in}{3.136600in}}%
\pgfpathlineto{\pgfqpoint{2.066892in}{3.133651in}}%
\pgfpathmoveto{\pgfqpoint{2.066892in}{3.130702in}}%
\pgfpathlineto{\pgfqpoint{2.066892in}{3.130702in}}%
\pgfpathlineto{\pgfqpoint{2.066892in}{3.133651in}}%
\pgfpathlineto{\pgfqpoint{2.071433in}{3.133651in}}%
\pgfpathlineto{\pgfqpoint{2.071433in}{3.130702in}}%
\pgfpathmoveto{\pgfqpoint{2.066892in}{3.133651in}}%
\pgfpathlineto{\pgfqpoint{2.066892in}{3.133651in}}%
\pgfpathlineto{\pgfqpoint{2.066892in}{3.136600in}}%
\pgfpathlineto{\pgfqpoint{2.071433in}{3.136600in}}%
\pgfpathlineto{\pgfqpoint{2.071433in}{3.133651in}}%
\pgfpathmoveto{\pgfqpoint{2.071433in}{3.130702in}}%
\pgfpathlineto{\pgfqpoint{2.071433in}{3.130702in}}%
\pgfpathlineto{\pgfqpoint{2.071433in}{3.133651in}}%
\pgfpathlineto{\pgfqpoint{2.075974in}{3.133651in}}%
\pgfpathlineto{\pgfqpoint{2.075974in}{3.130702in}}%
\pgfpathmoveto{\pgfqpoint{2.080515in}{3.127752in}}%
\pgfpathlineto{\pgfqpoint{2.080515in}{3.127752in}}%
\pgfpathlineto{\pgfqpoint{2.080515in}{3.130702in}}%
\pgfpathlineto{\pgfqpoint{2.085056in}{3.130702in}}%
\pgfpathlineto{\pgfqpoint{2.085056in}{3.127752in}}%
\pgfpathmoveto{\pgfqpoint{2.085056in}{3.127752in}}%
\pgfpathlineto{\pgfqpoint{2.085056in}{3.127752in}}%
\pgfpathlineto{\pgfqpoint{2.085056in}{3.130702in}}%
\pgfpathlineto{\pgfqpoint{2.089597in}{3.130702in}}%
\pgfpathlineto{\pgfqpoint{2.089597in}{3.127752in}}%
\pgfpathmoveto{\pgfqpoint{2.089597in}{3.127752in}}%
\pgfpathlineto{\pgfqpoint{2.089597in}{3.127752in}}%
\pgfpathlineto{\pgfqpoint{2.089597in}{3.130702in}}%
\pgfpathlineto{\pgfqpoint{2.094138in}{3.130702in}}%
\pgfpathlineto{\pgfqpoint{2.094138in}{3.127752in}}%
\pgfpathmoveto{\pgfqpoint{2.075974in}{3.130702in}}%
\pgfpathlineto{\pgfqpoint{2.075974in}{3.130702in}}%
\pgfpathlineto{\pgfqpoint{2.075974in}{3.133651in}}%
\pgfpathlineto{\pgfqpoint{2.080515in}{3.133651in}}%
\pgfpathlineto{\pgfqpoint{2.080515in}{3.130702in}}%
\pgfpathmoveto{\pgfqpoint{2.080515in}{3.130702in}}%
\pgfpathlineto{\pgfqpoint{2.080515in}{3.130702in}}%
\pgfpathlineto{\pgfqpoint{2.080515in}{3.133651in}}%
\pgfpathlineto{\pgfqpoint{2.085056in}{3.133651in}}%
\pgfpathlineto{\pgfqpoint{2.085056in}{3.130702in}}%
\pgfpathmoveto{\pgfqpoint{2.094138in}{3.124803in}}%
\pgfpathlineto{\pgfqpoint{2.094138in}{3.124803in}}%
\pgfpathlineto{\pgfqpoint{2.094138in}{3.127752in}}%
\pgfpathlineto{\pgfqpoint{2.098679in}{3.127752in}}%
\pgfpathlineto{\pgfqpoint{2.098679in}{3.124803in}}%
\pgfpathmoveto{\pgfqpoint{2.094138in}{3.127752in}}%
\pgfpathlineto{\pgfqpoint{2.094138in}{3.127752in}}%
\pgfpathlineto{\pgfqpoint{2.094138in}{3.130702in}}%
\pgfpathlineto{\pgfqpoint{2.098679in}{3.130702in}}%
\pgfpathlineto{\pgfqpoint{2.098679in}{3.127752in}}%
\pgfpathmoveto{\pgfqpoint{2.098679in}{3.124803in}}%
\pgfpathlineto{\pgfqpoint{2.098679in}{3.124803in}}%
\pgfpathlineto{\pgfqpoint{2.098679in}{3.127752in}}%
\pgfpathlineto{\pgfqpoint{2.103220in}{3.127752in}}%
\pgfpathlineto{\pgfqpoint{2.103220in}{3.124803in}}%
\pgfpathmoveto{\pgfqpoint{2.107761in}{3.121854in}}%
\pgfpathlineto{\pgfqpoint{2.107761in}{3.121854in}}%
\pgfpathlineto{\pgfqpoint{2.107761in}{3.124803in}}%
\pgfpathlineto{\pgfqpoint{2.112302in}{3.124803in}}%
\pgfpathlineto{\pgfqpoint{2.112302in}{3.121854in}}%
\pgfpathmoveto{\pgfqpoint{2.103220in}{3.124803in}}%
\pgfpathlineto{\pgfqpoint{2.103220in}{3.124803in}}%
\pgfpathlineto{\pgfqpoint{2.103220in}{3.127752in}}%
\pgfpathlineto{\pgfqpoint{2.107761in}{3.127752in}}%
\pgfpathlineto{\pgfqpoint{2.107761in}{3.124803in}}%
\pgfpathmoveto{\pgfqpoint{2.107761in}{3.124803in}}%
\pgfpathlineto{\pgfqpoint{2.107761in}{3.124803in}}%
\pgfpathlineto{\pgfqpoint{2.107761in}{3.127752in}}%
\pgfpathlineto{\pgfqpoint{2.112302in}{3.127752in}}%
\pgfpathlineto{\pgfqpoint{2.112302in}{3.124803in}}%
\pgfpathmoveto{\pgfqpoint{2.112302in}{3.121854in}}%
\pgfpathlineto{\pgfqpoint{2.112302in}{3.121854in}}%
\pgfpathlineto{\pgfqpoint{2.112302in}{3.124803in}}%
\pgfpathlineto{\pgfqpoint{2.116843in}{3.124803in}}%
\pgfpathlineto{\pgfqpoint{2.116843in}{3.121854in}}%
\pgfpathmoveto{\pgfqpoint{2.116843in}{3.121854in}}%
\pgfpathlineto{\pgfqpoint{2.116843in}{3.121854in}}%
\pgfpathlineto{\pgfqpoint{2.116843in}{3.124803in}}%
\pgfpathlineto{\pgfqpoint{2.121384in}{3.124803in}}%
\pgfpathlineto{\pgfqpoint{2.121384in}{3.121854in}}%
\pgfpathmoveto{\pgfqpoint{2.121384in}{3.118905in}}%
\pgfpathlineto{\pgfqpoint{2.121384in}{3.118905in}}%
\pgfpathlineto{\pgfqpoint{2.121384in}{3.121854in}}%
\pgfpathlineto{\pgfqpoint{2.125925in}{3.121854in}}%
\pgfpathlineto{\pgfqpoint{2.125925in}{3.118905in}}%
\pgfpathmoveto{\pgfqpoint{2.121384in}{3.121854in}}%
\pgfpathlineto{\pgfqpoint{2.121384in}{3.121854in}}%
\pgfpathlineto{\pgfqpoint{2.121384in}{3.124803in}}%
\pgfpathlineto{\pgfqpoint{2.125925in}{3.124803in}}%
\pgfpathlineto{\pgfqpoint{2.125925in}{3.121854in}}%
\pgfpathmoveto{\pgfqpoint{2.125925in}{3.118905in}}%
\pgfpathlineto{\pgfqpoint{2.125925in}{3.118905in}}%
\pgfpathlineto{\pgfqpoint{2.125925in}{3.121854in}}%
\pgfpathlineto{\pgfqpoint{2.130466in}{3.121854in}}%
\pgfpathlineto{\pgfqpoint{2.130466in}{3.118905in}}%
\pgfpathmoveto{\pgfqpoint{2.135008in}{3.115956in}}%
\pgfpathlineto{\pgfqpoint{2.135008in}{3.115956in}}%
\pgfpathlineto{\pgfqpoint{2.135008in}{3.118905in}}%
\pgfpathlineto{\pgfqpoint{2.139549in}{3.118905in}}%
\pgfpathlineto{\pgfqpoint{2.139549in}{3.115956in}}%
\pgfpathmoveto{\pgfqpoint{2.139549in}{3.115956in}}%
\pgfpathlineto{\pgfqpoint{2.139549in}{3.115956in}}%
\pgfpathlineto{\pgfqpoint{2.139549in}{3.118905in}}%
\pgfpathlineto{\pgfqpoint{2.144090in}{3.118905in}}%
\pgfpathlineto{\pgfqpoint{2.144090in}{3.115956in}}%
\pgfpathmoveto{\pgfqpoint{2.144090in}{3.115956in}}%
\pgfpathlineto{\pgfqpoint{2.144090in}{3.115956in}}%
\pgfpathlineto{\pgfqpoint{2.144090in}{3.118905in}}%
\pgfpathlineto{\pgfqpoint{2.148631in}{3.118905in}}%
\pgfpathlineto{\pgfqpoint{2.148631in}{3.115956in}}%
\pgfpathmoveto{\pgfqpoint{2.148631in}{3.113006in}}%
\pgfpathlineto{\pgfqpoint{2.148631in}{3.113006in}}%
\pgfpathlineto{\pgfqpoint{2.148631in}{3.115956in}}%
\pgfpathlineto{\pgfqpoint{2.153172in}{3.115956in}}%
\pgfpathlineto{\pgfqpoint{2.153172in}{3.113006in}}%
\pgfpathmoveto{\pgfqpoint{2.148631in}{3.115956in}}%
\pgfpathlineto{\pgfqpoint{2.148631in}{3.115956in}}%
\pgfpathlineto{\pgfqpoint{2.148631in}{3.118905in}}%
\pgfpathlineto{\pgfqpoint{2.153172in}{3.118905in}}%
\pgfpathlineto{\pgfqpoint{2.153172in}{3.115956in}}%
\pgfpathmoveto{\pgfqpoint{2.153172in}{3.113006in}}%
\pgfpathlineto{\pgfqpoint{2.153172in}{3.113006in}}%
\pgfpathlineto{\pgfqpoint{2.153172in}{3.115956in}}%
\pgfpathlineto{\pgfqpoint{2.157713in}{3.115956in}}%
\pgfpathlineto{\pgfqpoint{2.157713in}{3.113006in}}%
\pgfpathmoveto{\pgfqpoint{2.162254in}{3.110057in}}%
\pgfpathlineto{\pgfqpoint{2.162254in}{3.110057in}}%
\pgfpathlineto{\pgfqpoint{2.162254in}{3.113006in}}%
\pgfpathlineto{\pgfqpoint{2.166795in}{3.113006in}}%
\pgfpathlineto{\pgfqpoint{2.166795in}{3.110057in}}%
\pgfpathmoveto{\pgfqpoint{2.157713in}{3.113006in}}%
\pgfpathlineto{\pgfqpoint{2.157713in}{3.113006in}}%
\pgfpathlineto{\pgfqpoint{2.157713in}{3.115956in}}%
\pgfpathlineto{\pgfqpoint{2.162254in}{3.115956in}}%
\pgfpathlineto{\pgfqpoint{2.162254in}{3.113006in}}%
\pgfpathmoveto{\pgfqpoint{2.162254in}{3.113006in}}%
\pgfpathlineto{\pgfqpoint{2.162254in}{3.113006in}}%
\pgfpathlineto{\pgfqpoint{2.162254in}{3.115956in}}%
\pgfpathlineto{\pgfqpoint{2.166795in}{3.115956in}}%
\pgfpathlineto{\pgfqpoint{2.166795in}{3.113006in}}%
\pgfpathmoveto{\pgfqpoint{2.130466in}{3.118905in}}%
\pgfpathlineto{\pgfqpoint{2.130466in}{3.118905in}}%
\pgfpathlineto{\pgfqpoint{2.130466in}{3.121854in}}%
\pgfpathlineto{\pgfqpoint{2.135008in}{3.121854in}}%
\pgfpathlineto{\pgfqpoint{2.135008in}{3.118905in}}%
\pgfpathmoveto{\pgfqpoint{2.135008in}{3.118905in}}%
\pgfpathlineto{\pgfqpoint{2.135008in}{3.118905in}}%
\pgfpathlineto{\pgfqpoint{2.135008in}{3.121854in}}%
\pgfpathlineto{\pgfqpoint{2.139549in}{3.121854in}}%
\pgfpathlineto{\pgfqpoint{2.139549in}{3.118905in}}%
\pgfpathmoveto{\pgfqpoint{2.166795in}{3.110057in}}%
\pgfpathlineto{\pgfqpoint{2.166795in}{3.110057in}}%
\pgfpathlineto{\pgfqpoint{2.166795in}{3.113006in}}%
\pgfpathlineto{\pgfqpoint{2.171336in}{3.113006in}}%
\pgfpathlineto{\pgfqpoint{2.171336in}{3.110057in}}%
\pgfpathmoveto{\pgfqpoint{2.171336in}{3.110057in}}%
\pgfpathlineto{\pgfqpoint{2.171336in}{3.110057in}}%
\pgfpathlineto{\pgfqpoint{2.171336in}{3.113006in}}%
\pgfpathlineto{\pgfqpoint{2.175877in}{3.113006in}}%
\pgfpathlineto{\pgfqpoint{2.175877in}{3.110057in}}%
\pgfpathmoveto{\pgfqpoint{2.175877in}{3.107108in}}%
\pgfpathlineto{\pgfqpoint{2.175877in}{3.107108in}}%
\pgfpathlineto{\pgfqpoint{2.175877in}{3.110057in}}%
\pgfpathlineto{\pgfqpoint{2.180418in}{3.110057in}}%
\pgfpathlineto{\pgfqpoint{2.180418in}{3.107108in}}%
\pgfpathmoveto{\pgfqpoint{2.175877in}{3.110057in}}%
\pgfpathlineto{\pgfqpoint{2.175877in}{3.110057in}}%
\pgfpathlineto{\pgfqpoint{2.175877in}{3.113006in}}%
\pgfpathlineto{\pgfqpoint{2.180418in}{3.113006in}}%
\pgfpathlineto{\pgfqpoint{2.180418in}{3.110057in}}%
\pgfpathmoveto{\pgfqpoint{2.180418in}{3.107108in}}%
\pgfpathlineto{\pgfqpoint{2.180418in}{3.107108in}}%
\pgfpathlineto{\pgfqpoint{2.180418in}{3.110057in}}%
\pgfpathlineto{\pgfqpoint{2.184959in}{3.110057in}}%
\pgfpathlineto{\pgfqpoint{2.184959in}{3.107108in}}%
\pgfpathmoveto{\pgfqpoint{2.189500in}{3.104159in}}%
\pgfpathlineto{\pgfqpoint{2.189500in}{3.104159in}}%
\pgfpathlineto{\pgfqpoint{2.189500in}{3.107108in}}%
\pgfpathlineto{\pgfqpoint{2.194041in}{3.107108in}}%
\pgfpathlineto{\pgfqpoint{2.194041in}{3.104159in}}%
\pgfpathmoveto{\pgfqpoint{2.194041in}{3.104159in}}%
\pgfpathlineto{\pgfqpoint{2.194041in}{3.104159in}}%
\pgfpathlineto{\pgfqpoint{2.194041in}{3.107108in}}%
\pgfpathlineto{\pgfqpoint{2.198582in}{3.107108in}}%
\pgfpathlineto{\pgfqpoint{2.198582in}{3.104159in}}%
\pgfpathmoveto{\pgfqpoint{2.198582in}{3.104159in}}%
\pgfpathlineto{\pgfqpoint{2.198582in}{3.104159in}}%
\pgfpathlineto{\pgfqpoint{2.198582in}{3.107108in}}%
\pgfpathlineto{\pgfqpoint{2.203123in}{3.107108in}}%
\pgfpathlineto{\pgfqpoint{2.203123in}{3.104159in}}%
\pgfpathmoveto{\pgfqpoint{2.184959in}{3.107108in}}%
\pgfpathlineto{\pgfqpoint{2.184959in}{3.107108in}}%
\pgfpathlineto{\pgfqpoint{2.184959in}{3.110057in}}%
\pgfpathlineto{\pgfqpoint{2.189500in}{3.110057in}}%
\pgfpathlineto{\pgfqpoint{2.189500in}{3.107108in}}%
\pgfpathmoveto{\pgfqpoint{2.189500in}{3.107108in}}%
\pgfpathlineto{\pgfqpoint{2.189500in}{3.107108in}}%
\pgfpathlineto{\pgfqpoint{2.189500in}{3.110057in}}%
\pgfpathlineto{\pgfqpoint{2.194041in}{3.110057in}}%
\pgfpathlineto{\pgfqpoint{2.194041in}{3.107108in}}%
\pgfpathmoveto{\pgfqpoint{2.243994in}{3.092362in}}%
\pgfpathlineto{\pgfqpoint{2.243994in}{3.092362in}}%
\pgfpathlineto{\pgfqpoint{2.243994in}{3.095311in}}%
\pgfpathlineto{\pgfqpoint{2.248535in}{3.095311in}}%
\pgfpathlineto{\pgfqpoint{2.248535in}{3.092362in}}%
\pgfpathmoveto{\pgfqpoint{2.248535in}{3.092362in}}%
\pgfpathlineto{\pgfqpoint{2.248535in}{3.092362in}}%
\pgfpathlineto{\pgfqpoint{2.248535in}{3.095311in}}%
\pgfpathlineto{\pgfqpoint{2.253076in}{3.095311in}}%
\pgfpathlineto{\pgfqpoint{2.253076in}{3.092362in}}%
\pgfpathmoveto{\pgfqpoint{2.253076in}{3.092362in}}%
\pgfpathlineto{\pgfqpoint{2.253076in}{3.092362in}}%
\pgfpathlineto{\pgfqpoint{2.253076in}{3.095311in}}%
\pgfpathlineto{\pgfqpoint{2.257618in}{3.095311in}}%
\pgfpathlineto{\pgfqpoint{2.257618in}{3.092362in}}%
\pgfpathmoveto{\pgfqpoint{2.257618in}{3.089413in}}%
\pgfpathlineto{\pgfqpoint{2.257618in}{3.089413in}}%
\pgfpathlineto{\pgfqpoint{2.257618in}{3.092362in}}%
\pgfpathlineto{\pgfqpoint{2.262159in}{3.092362in}}%
\pgfpathlineto{\pgfqpoint{2.262159in}{3.089413in}}%
\pgfpathmoveto{\pgfqpoint{2.257618in}{3.092362in}}%
\pgfpathlineto{\pgfqpoint{2.257618in}{3.092362in}}%
\pgfpathlineto{\pgfqpoint{2.257618in}{3.095311in}}%
\pgfpathlineto{\pgfqpoint{2.262159in}{3.095311in}}%
\pgfpathlineto{\pgfqpoint{2.262159in}{3.092362in}}%
\pgfpathmoveto{\pgfqpoint{2.262159in}{3.089413in}}%
\pgfpathlineto{\pgfqpoint{2.262159in}{3.089413in}}%
\pgfpathlineto{\pgfqpoint{2.262159in}{3.092362in}}%
\pgfpathlineto{\pgfqpoint{2.266700in}{3.092362in}}%
\pgfpathlineto{\pgfqpoint{2.266700in}{3.089413in}}%
\pgfpathmoveto{\pgfqpoint{2.271241in}{3.086463in}}%
\pgfpathlineto{\pgfqpoint{2.271241in}{3.086463in}}%
\pgfpathlineto{\pgfqpoint{2.271241in}{3.089413in}}%
\pgfpathlineto{\pgfqpoint{2.275783in}{3.089413in}}%
\pgfpathlineto{\pgfqpoint{2.275783in}{3.086463in}}%
\pgfpathmoveto{\pgfqpoint{2.266700in}{3.089413in}}%
\pgfpathlineto{\pgfqpoint{2.266700in}{3.089413in}}%
\pgfpathlineto{\pgfqpoint{2.266700in}{3.092362in}}%
\pgfpathlineto{\pgfqpoint{2.271241in}{3.092362in}}%
\pgfpathlineto{\pgfqpoint{2.271241in}{3.089413in}}%
\pgfpathmoveto{\pgfqpoint{2.271241in}{3.089413in}}%
\pgfpathlineto{\pgfqpoint{2.271241in}{3.089413in}}%
\pgfpathlineto{\pgfqpoint{2.271241in}{3.092362in}}%
\pgfpathlineto{\pgfqpoint{2.275783in}{3.092362in}}%
\pgfpathlineto{\pgfqpoint{2.275783in}{3.089413in}}%
\pgfpathmoveto{\pgfqpoint{2.203123in}{3.101209in}}%
\pgfpathlineto{\pgfqpoint{2.203123in}{3.101209in}}%
\pgfpathlineto{\pgfqpoint{2.203123in}{3.104159in}}%
\pgfpathlineto{\pgfqpoint{2.207664in}{3.104159in}}%
\pgfpathlineto{\pgfqpoint{2.207664in}{3.101209in}}%
\pgfpathmoveto{\pgfqpoint{2.203123in}{3.104159in}}%
\pgfpathlineto{\pgfqpoint{2.203123in}{3.104159in}}%
\pgfpathlineto{\pgfqpoint{2.203123in}{3.107108in}}%
\pgfpathlineto{\pgfqpoint{2.207664in}{3.107108in}}%
\pgfpathlineto{\pgfqpoint{2.207664in}{3.104159in}}%
\pgfpathmoveto{\pgfqpoint{2.207664in}{3.101209in}}%
\pgfpathlineto{\pgfqpoint{2.207664in}{3.101209in}}%
\pgfpathlineto{\pgfqpoint{2.207664in}{3.104159in}}%
\pgfpathlineto{\pgfqpoint{2.212206in}{3.104159in}}%
\pgfpathlineto{\pgfqpoint{2.212206in}{3.101209in}}%
\pgfpathmoveto{\pgfqpoint{2.216747in}{3.098260in}}%
\pgfpathlineto{\pgfqpoint{2.216747in}{3.098260in}}%
\pgfpathlineto{\pgfqpoint{2.216747in}{3.101209in}}%
\pgfpathlineto{\pgfqpoint{2.221288in}{3.101209in}}%
\pgfpathlineto{\pgfqpoint{2.221288in}{3.098260in}}%
\pgfpathmoveto{\pgfqpoint{2.212206in}{3.101209in}}%
\pgfpathlineto{\pgfqpoint{2.212206in}{3.101209in}}%
\pgfpathlineto{\pgfqpoint{2.212206in}{3.104159in}}%
\pgfpathlineto{\pgfqpoint{2.216747in}{3.104159in}}%
\pgfpathlineto{\pgfqpoint{2.216747in}{3.101209in}}%
\pgfpathmoveto{\pgfqpoint{2.216747in}{3.101209in}}%
\pgfpathlineto{\pgfqpoint{2.216747in}{3.101209in}}%
\pgfpathlineto{\pgfqpoint{2.216747in}{3.104159in}}%
\pgfpathlineto{\pgfqpoint{2.221288in}{3.104159in}}%
\pgfpathlineto{\pgfqpoint{2.221288in}{3.101209in}}%
\pgfpathmoveto{\pgfqpoint{2.221288in}{3.098260in}}%
\pgfpathlineto{\pgfqpoint{2.221288in}{3.098260in}}%
\pgfpathlineto{\pgfqpoint{2.221288in}{3.101209in}}%
\pgfpathlineto{\pgfqpoint{2.225829in}{3.101209in}}%
\pgfpathlineto{\pgfqpoint{2.225829in}{3.098260in}}%
\pgfpathmoveto{\pgfqpoint{2.225829in}{3.098260in}}%
\pgfpathlineto{\pgfqpoint{2.225829in}{3.098260in}}%
\pgfpathlineto{\pgfqpoint{2.225829in}{3.101209in}}%
\pgfpathlineto{\pgfqpoint{2.230370in}{3.101209in}}%
\pgfpathlineto{\pgfqpoint{2.230370in}{3.098260in}}%
\pgfpathmoveto{\pgfqpoint{2.230370in}{3.095311in}}%
\pgfpathlineto{\pgfqpoint{2.230370in}{3.095311in}}%
\pgfpathlineto{\pgfqpoint{2.230370in}{3.098260in}}%
\pgfpathlineto{\pgfqpoint{2.234912in}{3.098260in}}%
\pgfpathlineto{\pgfqpoint{2.234912in}{3.095311in}}%
\pgfpathmoveto{\pgfqpoint{2.230370in}{3.098260in}}%
\pgfpathlineto{\pgfqpoint{2.230370in}{3.098260in}}%
\pgfpathlineto{\pgfqpoint{2.230370in}{3.101209in}}%
\pgfpathlineto{\pgfqpoint{2.234912in}{3.101209in}}%
\pgfpathlineto{\pgfqpoint{2.234912in}{3.098260in}}%
\pgfpathmoveto{\pgfqpoint{2.234912in}{3.095311in}}%
\pgfpathlineto{\pgfqpoint{2.234912in}{3.095311in}}%
\pgfpathlineto{\pgfqpoint{2.234912in}{3.098260in}}%
\pgfpathlineto{\pgfqpoint{2.239453in}{3.098260in}}%
\pgfpathlineto{\pgfqpoint{2.239453in}{3.095311in}}%
\pgfpathmoveto{\pgfqpoint{2.239453in}{3.095311in}}%
\pgfpathlineto{\pgfqpoint{2.239453in}{3.095311in}}%
\pgfpathlineto{\pgfqpoint{2.239453in}{3.098260in}}%
\pgfpathlineto{\pgfqpoint{2.243994in}{3.098260in}}%
\pgfpathlineto{\pgfqpoint{2.243994in}{3.095311in}}%
\pgfpathmoveto{\pgfqpoint{2.243994in}{3.095311in}}%
\pgfpathlineto{\pgfqpoint{2.243994in}{3.095311in}}%
\pgfpathlineto{\pgfqpoint{2.243994in}{3.098260in}}%
\pgfpathlineto{\pgfqpoint{2.248535in}{3.098260in}}%
\pgfpathlineto{\pgfqpoint{2.248535in}{3.095311in}}%
\pgfpathmoveto{\pgfqpoint{2.275783in}{3.086463in}}%
\pgfpathlineto{\pgfqpoint{2.275783in}{3.086463in}}%
\pgfpathlineto{\pgfqpoint{2.275783in}{3.089413in}}%
\pgfpathlineto{\pgfqpoint{2.280324in}{3.089413in}}%
\pgfpathlineto{\pgfqpoint{2.280324in}{3.086463in}}%
\pgfpathmoveto{\pgfqpoint{2.280324in}{3.086463in}}%
\pgfpathlineto{\pgfqpoint{2.280324in}{3.086463in}}%
\pgfpathlineto{\pgfqpoint{2.280324in}{3.089413in}}%
\pgfpathlineto{\pgfqpoint{2.284865in}{3.089413in}}%
\pgfpathlineto{\pgfqpoint{2.284865in}{3.086463in}}%
\pgfpathmoveto{\pgfqpoint{2.284865in}{3.083514in}}%
\pgfpathlineto{\pgfqpoint{2.284865in}{3.083514in}}%
\pgfpathlineto{\pgfqpoint{2.284865in}{3.086463in}}%
\pgfpathlineto{\pgfqpoint{2.289406in}{3.086463in}}%
\pgfpathlineto{\pgfqpoint{2.289406in}{3.083514in}}%
\pgfpathmoveto{\pgfqpoint{2.284865in}{3.086463in}}%
\pgfpathlineto{\pgfqpoint{2.284865in}{3.086463in}}%
\pgfpathlineto{\pgfqpoint{2.284865in}{3.089413in}}%
\pgfpathlineto{\pgfqpoint{2.289406in}{3.089413in}}%
\pgfpathlineto{\pgfqpoint{2.289406in}{3.086463in}}%
\pgfpathmoveto{\pgfqpoint{2.289406in}{3.083514in}}%
\pgfpathlineto{\pgfqpoint{2.289406in}{3.083514in}}%
\pgfpathlineto{\pgfqpoint{2.289406in}{3.086463in}}%
\pgfpathlineto{\pgfqpoint{2.293947in}{3.086463in}}%
\pgfpathlineto{\pgfqpoint{2.293947in}{3.083514in}}%
\pgfpathmoveto{\pgfqpoint{2.298489in}{3.080565in}}%
\pgfpathlineto{\pgfqpoint{2.298489in}{3.080565in}}%
\pgfpathlineto{\pgfqpoint{2.298489in}{3.083514in}}%
\pgfpathlineto{\pgfqpoint{2.303030in}{3.083514in}}%
\pgfpathlineto{\pgfqpoint{2.303030in}{3.080565in}}%
\pgfpathmoveto{\pgfqpoint{2.303030in}{3.080565in}}%
\pgfpathlineto{\pgfqpoint{2.303030in}{3.080565in}}%
\pgfpathlineto{\pgfqpoint{2.303030in}{3.083514in}}%
\pgfpathlineto{\pgfqpoint{2.307571in}{3.083514in}}%
\pgfpathlineto{\pgfqpoint{2.307571in}{3.080565in}}%
\pgfpathmoveto{\pgfqpoint{2.307571in}{3.080565in}}%
\pgfpathlineto{\pgfqpoint{2.307571in}{3.080565in}}%
\pgfpathlineto{\pgfqpoint{2.307571in}{3.083514in}}%
\pgfpathlineto{\pgfqpoint{2.312112in}{3.083514in}}%
\pgfpathlineto{\pgfqpoint{2.312112in}{3.080565in}}%
\pgfpathmoveto{\pgfqpoint{2.293947in}{3.083514in}}%
\pgfpathlineto{\pgfqpoint{2.293947in}{3.083514in}}%
\pgfpathlineto{\pgfqpoint{2.293947in}{3.086463in}}%
\pgfpathlineto{\pgfqpoint{2.298489in}{3.086463in}}%
\pgfpathlineto{\pgfqpoint{2.298489in}{3.083514in}}%
\pgfpathmoveto{\pgfqpoint{2.298489in}{3.083514in}}%
\pgfpathlineto{\pgfqpoint{2.298489in}{3.083514in}}%
\pgfpathlineto{\pgfqpoint{2.298489in}{3.086463in}}%
\pgfpathlineto{\pgfqpoint{2.303030in}{3.086463in}}%
\pgfpathlineto{\pgfqpoint{2.303030in}{3.083514in}}%
\pgfpathmoveto{\pgfqpoint{2.312112in}{3.077616in}}%
\pgfpathlineto{\pgfqpoint{2.312112in}{3.077616in}}%
\pgfpathlineto{\pgfqpoint{2.312112in}{3.080565in}}%
\pgfpathlineto{\pgfqpoint{2.316653in}{3.080565in}}%
\pgfpathlineto{\pgfqpoint{2.316653in}{3.077616in}}%
\pgfpathmoveto{\pgfqpoint{2.312112in}{3.080565in}}%
\pgfpathlineto{\pgfqpoint{2.312112in}{3.080565in}}%
\pgfpathlineto{\pgfqpoint{2.312112in}{3.083514in}}%
\pgfpathlineto{\pgfqpoint{2.316653in}{3.083514in}}%
\pgfpathlineto{\pgfqpoint{2.316653in}{3.080565in}}%
\pgfpathmoveto{\pgfqpoint{2.316653in}{3.077616in}}%
\pgfpathlineto{\pgfqpoint{2.316653in}{3.077616in}}%
\pgfpathlineto{\pgfqpoint{2.316653in}{3.080565in}}%
\pgfpathlineto{\pgfqpoint{2.321195in}{3.080565in}}%
\pgfpathlineto{\pgfqpoint{2.321195in}{3.077616in}}%
\pgfpathmoveto{\pgfqpoint{2.325736in}{3.074666in}}%
\pgfpathlineto{\pgfqpoint{2.325736in}{3.074666in}}%
\pgfpathlineto{\pgfqpoint{2.325736in}{3.077616in}}%
\pgfpathlineto{\pgfqpoint{2.330277in}{3.077616in}}%
\pgfpathlineto{\pgfqpoint{2.330277in}{3.074666in}}%
\pgfpathmoveto{\pgfqpoint{2.321195in}{3.077616in}}%
\pgfpathlineto{\pgfqpoint{2.321195in}{3.077616in}}%
\pgfpathlineto{\pgfqpoint{2.321195in}{3.080565in}}%
\pgfpathlineto{\pgfqpoint{2.325736in}{3.080565in}}%
\pgfpathlineto{\pgfqpoint{2.325736in}{3.077616in}}%
\pgfpathmoveto{\pgfqpoint{2.325736in}{3.077616in}}%
\pgfpathlineto{\pgfqpoint{2.325736in}{3.077616in}}%
\pgfpathlineto{\pgfqpoint{2.325736in}{3.080565in}}%
\pgfpathlineto{\pgfqpoint{2.330277in}{3.080565in}}%
\pgfpathlineto{\pgfqpoint{2.330277in}{3.077616in}}%
\pgfpathmoveto{\pgfqpoint{2.330277in}{3.074666in}}%
\pgfpathlineto{\pgfqpoint{2.330277in}{3.074666in}}%
\pgfpathlineto{\pgfqpoint{2.330277in}{3.077616in}}%
\pgfpathlineto{\pgfqpoint{2.334818in}{3.077616in}}%
\pgfpathlineto{\pgfqpoint{2.334818in}{3.074666in}}%
\pgfpathmoveto{\pgfqpoint{2.334818in}{3.074666in}}%
\pgfpathlineto{\pgfqpoint{2.334818in}{3.074666in}}%
\pgfpathlineto{\pgfqpoint{2.334818in}{3.077616in}}%
\pgfpathlineto{\pgfqpoint{2.339359in}{3.077616in}}%
\pgfpathlineto{\pgfqpoint{2.339359in}{3.074666in}}%
\pgfpathmoveto{\pgfqpoint{2.339359in}{3.071717in}}%
\pgfpathlineto{\pgfqpoint{2.339359in}{3.071717in}}%
\pgfpathlineto{\pgfqpoint{2.339359in}{3.074666in}}%
\pgfpathlineto{\pgfqpoint{2.343901in}{3.074666in}}%
\pgfpathlineto{\pgfqpoint{2.343901in}{3.071717in}}%
\pgfpathmoveto{\pgfqpoint{2.339359in}{3.074666in}}%
\pgfpathlineto{\pgfqpoint{2.339359in}{3.074666in}}%
\pgfpathlineto{\pgfqpoint{2.339359in}{3.077616in}}%
\pgfpathlineto{\pgfqpoint{2.343901in}{3.077616in}}%
\pgfpathlineto{\pgfqpoint{2.343901in}{3.074666in}}%
\pgfpathmoveto{\pgfqpoint{2.343901in}{3.071717in}}%
\pgfpathlineto{\pgfqpoint{2.343901in}{3.071717in}}%
\pgfpathlineto{\pgfqpoint{2.343901in}{3.074666in}}%
\pgfpathlineto{\pgfqpoint{2.348442in}{3.074666in}}%
\pgfpathlineto{\pgfqpoint{2.348442in}{3.071717in}}%
\pgfpathmoveto{\pgfqpoint{2.461966in}{3.045174in}}%
\pgfpathlineto{\pgfqpoint{2.461966in}{3.045174in}}%
\pgfpathlineto{\pgfqpoint{2.461966in}{3.048123in}}%
\pgfpathlineto{\pgfqpoint{2.466507in}{3.048123in}}%
\pgfpathlineto{\pgfqpoint{2.466507in}{3.045174in}}%
\pgfpathmoveto{\pgfqpoint{2.466507in}{3.045174in}}%
\pgfpathlineto{\pgfqpoint{2.466507in}{3.045174in}}%
\pgfpathlineto{\pgfqpoint{2.466507in}{3.048123in}}%
\pgfpathlineto{\pgfqpoint{2.471048in}{3.048123in}}%
\pgfpathlineto{\pgfqpoint{2.471048in}{3.045174in}}%
\pgfpathmoveto{\pgfqpoint{2.471048in}{3.045174in}}%
\pgfpathlineto{\pgfqpoint{2.471048in}{3.045174in}}%
\pgfpathlineto{\pgfqpoint{2.471048in}{3.048123in}}%
\pgfpathlineto{\pgfqpoint{2.475589in}{3.048123in}}%
\pgfpathlineto{\pgfqpoint{2.475589in}{3.045174in}}%
\pgfpathmoveto{\pgfqpoint{2.475589in}{3.042225in}}%
\pgfpathlineto{\pgfqpoint{2.475589in}{3.042225in}}%
\pgfpathlineto{\pgfqpoint{2.475589in}{3.045174in}}%
\pgfpathlineto{\pgfqpoint{2.480130in}{3.045174in}}%
\pgfpathlineto{\pgfqpoint{2.480130in}{3.042225in}}%
\pgfpathmoveto{\pgfqpoint{2.475589in}{3.045174in}}%
\pgfpathlineto{\pgfqpoint{2.475589in}{3.045174in}}%
\pgfpathlineto{\pgfqpoint{2.475589in}{3.048123in}}%
\pgfpathlineto{\pgfqpoint{2.480130in}{3.048123in}}%
\pgfpathlineto{\pgfqpoint{2.480130in}{3.045174in}}%
\pgfpathmoveto{\pgfqpoint{2.480130in}{3.042225in}}%
\pgfpathlineto{\pgfqpoint{2.480130in}{3.042225in}}%
\pgfpathlineto{\pgfqpoint{2.480130in}{3.045174in}}%
\pgfpathlineto{\pgfqpoint{2.484671in}{3.045174in}}%
\pgfpathlineto{\pgfqpoint{2.484671in}{3.042225in}}%
\pgfpathmoveto{\pgfqpoint{2.489212in}{3.039276in}}%
\pgfpathlineto{\pgfqpoint{2.489212in}{3.039276in}}%
\pgfpathlineto{\pgfqpoint{2.489212in}{3.042225in}}%
\pgfpathlineto{\pgfqpoint{2.493753in}{3.042225in}}%
\pgfpathlineto{\pgfqpoint{2.493753in}{3.039276in}}%
\pgfpathmoveto{\pgfqpoint{2.484671in}{3.042225in}}%
\pgfpathlineto{\pgfqpoint{2.484671in}{3.042225in}}%
\pgfpathlineto{\pgfqpoint{2.484671in}{3.045174in}}%
\pgfpathlineto{\pgfqpoint{2.489212in}{3.045174in}}%
\pgfpathlineto{\pgfqpoint{2.489212in}{3.042225in}}%
\pgfpathmoveto{\pgfqpoint{2.489212in}{3.042225in}}%
\pgfpathlineto{\pgfqpoint{2.489212in}{3.042225in}}%
\pgfpathlineto{\pgfqpoint{2.489212in}{3.045174in}}%
\pgfpathlineto{\pgfqpoint{2.493753in}{3.045174in}}%
\pgfpathlineto{\pgfqpoint{2.493753in}{3.042225in}}%
\pgfpathmoveto{\pgfqpoint{2.352983in}{3.068768in}}%
\pgfpathlineto{\pgfqpoint{2.352983in}{3.068768in}}%
\pgfpathlineto{\pgfqpoint{2.352983in}{3.071717in}}%
\pgfpathlineto{\pgfqpoint{2.357524in}{3.071717in}}%
\pgfpathlineto{\pgfqpoint{2.357524in}{3.068768in}}%
\pgfpathmoveto{\pgfqpoint{2.357524in}{3.068768in}}%
\pgfpathlineto{\pgfqpoint{2.357524in}{3.068768in}}%
\pgfpathlineto{\pgfqpoint{2.357524in}{3.071717in}}%
\pgfpathlineto{\pgfqpoint{2.362065in}{3.071717in}}%
\pgfpathlineto{\pgfqpoint{2.362065in}{3.068768in}}%
\pgfpathmoveto{\pgfqpoint{2.362065in}{3.068768in}}%
\pgfpathlineto{\pgfqpoint{2.362065in}{3.068768in}}%
\pgfpathlineto{\pgfqpoint{2.362065in}{3.071717in}}%
\pgfpathlineto{\pgfqpoint{2.366606in}{3.071717in}}%
\pgfpathlineto{\pgfqpoint{2.366606in}{3.068768in}}%
\pgfpathmoveto{\pgfqpoint{2.366606in}{3.065819in}}%
\pgfpathlineto{\pgfqpoint{2.366606in}{3.065819in}}%
\pgfpathlineto{\pgfqpoint{2.366606in}{3.068768in}}%
\pgfpathlineto{\pgfqpoint{2.371147in}{3.068768in}}%
\pgfpathlineto{\pgfqpoint{2.371147in}{3.065819in}}%
\pgfpathmoveto{\pgfqpoint{2.366606in}{3.068768in}}%
\pgfpathlineto{\pgfqpoint{2.366606in}{3.068768in}}%
\pgfpathlineto{\pgfqpoint{2.366606in}{3.071717in}}%
\pgfpathlineto{\pgfqpoint{2.371147in}{3.071717in}}%
\pgfpathlineto{\pgfqpoint{2.371147in}{3.068768in}}%
\pgfpathmoveto{\pgfqpoint{2.371147in}{3.065819in}}%
\pgfpathlineto{\pgfqpoint{2.371147in}{3.065819in}}%
\pgfpathlineto{\pgfqpoint{2.371147in}{3.068768in}}%
\pgfpathlineto{\pgfqpoint{2.375688in}{3.068768in}}%
\pgfpathlineto{\pgfqpoint{2.375688in}{3.065819in}}%
\pgfpathmoveto{\pgfqpoint{2.380229in}{3.062870in}}%
\pgfpathlineto{\pgfqpoint{2.380229in}{3.062870in}}%
\pgfpathlineto{\pgfqpoint{2.380229in}{3.065819in}}%
\pgfpathlineto{\pgfqpoint{2.384770in}{3.065819in}}%
\pgfpathlineto{\pgfqpoint{2.384770in}{3.062870in}}%
\pgfpathmoveto{\pgfqpoint{2.375688in}{3.065819in}}%
\pgfpathlineto{\pgfqpoint{2.375688in}{3.065819in}}%
\pgfpathlineto{\pgfqpoint{2.375688in}{3.068768in}}%
\pgfpathlineto{\pgfqpoint{2.380229in}{3.068768in}}%
\pgfpathlineto{\pgfqpoint{2.380229in}{3.065819in}}%
\pgfpathmoveto{\pgfqpoint{2.380229in}{3.065819in}}%
\pgfpathlineto{\pgfqpoint{2.380229in}{3.065819in}}%
\pgfpathlineto{\pgfqpoint{2.380229in}{3.068768in}}%
\pgfpathlineto{\pgfqpoint{2.384770in}{3.068768in}}%
\pgfpathlineto{\pgfqpoint{2.384770in}{3.065819in}}%
\pgfpathmoveto{\pgfqpoint{2.348442in}{3.071717in}}%
\pgfpathlineto{\pgfqpoint{2.348442in}{3.071717in}}%
\pgfpathlineto{\pgfqpoint{2.348442in}{3.074666in}}%
\pgfpathlineto{\pgfqpoint{2.352983in}{3.074666in}}%
\pgfpathlineto{\pgfqpoint{2.352983in}{3.071717in}}%
\pgfpathmoveto{\pgfqpoint{2.352983in}{3.071717in}}%
\pgfpathlineto{\pgfqpoint{2.352983in}{3.071717in}}%
\pgfpathlineto{\pgfqpoint{2.352983in}{3.074666in}}%
\pgfpathlineto{\pgfqpoint{2.357524in}{3.074666in}}%
\pgfpathlineto{\pgfqpoint{2.357524in}{3.071717in}}%
\pgfpathmoveto{\pgfqpoint{2.384770in}{3.062870in}}%
\pgfpathlineto{\pgfqpoint{2.384770in}{3.062870in}}%
\pgfpathlineto{\pgfqpoint{2.384770in}{3.065819in}}%
\pgfpathlineto{\pgfqpoint{2.389311in}{3.065819in}}%
\pgfpathlineto{\pgfqpoint{2.389311in}{3.062870in}}%
\pgfpathmoveto{\pgfqpoint{2.389311in}{3.062870in}}%
\pgfpathlineto{\pgfqpoint{2.389311in}{3.062870in}}%
\pgfpathlineto{\pgfqpoint{2.389311in}{3.065819in}}%
\pgfpathlineto{\pgfqpoint{2.393852in}{3.065819in}}%
\pgfpathlineto{\pgfqpoint{2.393852in}{3.062870in}}%
\pgfpathmoveto{\pgfqpoint{2.393852in}{3.059920in}}%
\pgfpathlineto{\pgfqpoint{2.393852in}{3.059920in}}%
\pgfpathlineto{\pgfqpoint{2.393852in}{3.062870in}}%
\pgfpathlineto{\pgfqpoint{2.398393in}{3.062870in}}%
\pgfpathlineto{\pgfqpoint{2.398393in}{3.059920in}}%
\pgfpathmoveto{\pgfqpoint{2.393852in}{3.062870in}}%
\pgfpathlineto{\pgfqpoint{2.393852in}{3.062870in}}%
\pgfpathlineto{\pgfqpoint{2.393852in}{3.065819in}}%
\pgfpathlineto{\pgfqpoint{2.398393in}{3.065819in}}%
\pgfpathlineto{\pgfqpoint{2.398393in}{3.062870in}}%
\pgfpathmoveto{\pgfqpoint{2.398393in}{3.059920in}}%
\pgfpathlineto{\pgfqpoint{2.398393in}{3.059920in}}%
\pgfpathlineto{\pgfqpoint{2.398393in}{3.062870in}}%
\pgfpathlineto{\pgfqpoint{2.402934in}{3.062870in}}%
\pgfpathlineto{\pgfqpoint{2.402934in}{3.059920in}}%
\pgfpathmoveto{\pgfqpoint{2.407475in}{3.056971in}}%
\pgfpathlineto{\pgfqpoint{2.407475in}{3.056971in}}%
\pgfpathlineto{\pgfqpoint{2.407475in}{3.059920in}}%
\pgfpathlineto{\pgfqpoint{2.412015in}{3.059920in}}%
\pgfpathlineto{\pgfqpoint{2.412015in}{3.056971in}}%
\pgfpathmoveto{\pgfqpoint{2.412015in}{3.056971in}}%
\pgfpathlineto{\pgfqpoint{2.412015in}{3.056971in}}%
\pgfpathlineto{\pgfqpoint{2.412015in}{3.059920in}}%
\pgfpathlineto{\pgfqpoint{2.416556in}{3.059920in}}%
\pgfpathlineto{\pgfqpoint{2.416556in}{3.056971in}}%
\pgfpathmoveto{\pgfqpoint{2.416556in}{3.056971in}}%
\pgfpathlineto{\pgfqpoint{2.416556in}{3.056971in}}%
\pgfpathlineto{\pgfqpoint{2.416556in}{3.059920in}}%
\pgfpathlineto{\pgfqpoint{2.421097in}{3.059920in}}%
\pgfpathlineto{\pgfqpoint{2.421097in}{3.056971in}}%
\pgfpathmoveto{\pgfqpoint{2.402934in}{3.059920in}}%
\pgfpathlineto{\pgfqpoint{2.402934in}{3.059920in}}%
\pgfpathlineto{\pgfqpoint{2.402934in}{3.062870in}}%
\pgfpathlineto{\pgfqpoint{2.407475in}{3.062870in}}%
\pgfpathlineto{\pgfqpoint{2.407475in}{3.059920in}}%
\pgfpathmoveto{\pgfqpoint{2.407475in}{3.059920in}}%
\pgfpathlineto{\pgfqpoint{2.407475in}{3.059920in}}%
\pgfpathlineto{\pgfqpoint{2.407475in}{3.062870in}}%
\pgfpathlineto{\pgfqpoint{2.412015in}{3.062870in}}%
\pgfpathlineto{\pgfqpoint{2.412015in}{3.059920in}}%
\pgfpathmoveto{\pgfqpoint{2.421097in}{3.054022in}}%
\pgfpathlineto{\pgfqpoint{2.421097in}{3.054022in}}%
\pgfpathlineto{\pgfqpoint{2.421097in}{3.056971in}}%
\pgfpathlineto{\pgfqpoint{2.425638in}{3.056971in}}%
\pgfpathlineto{\pgfqpoint{2.425638in}{3.054022in}}%
\pgfpathmoveto{\pgfqpoint{2.421097in}{3.056971in}}%
\pgfpathlineto{\pgfqpoint{2.421097in}{3.056971in}}%
\pgfpathlineto{\pgfqpoint{2.421097in}{3.059920in}}%
\pgfpathlineto{\pgfqpoint{2.425638in}{3.059920in}}%
\pgfpathlineto{\pgfqpoint{2.425638in}{3.056971in}}%
\pgfpathmoveto{\pgfqpoint{2.425638in}{3.054022in}}%
\pgfpathlineto{\pgfqpoint{2.425638in}{3.054022in}}%
\pgfpathlineto{\pgfqpoint{2.425638in}{3.056971in}}%
\pgfpathlineto{\pgfqpoint{2.430179in}{3.056971in}}%
\pgfpathlineto{\pgfqpoint{2.430179in}{3.054022in}}%
\pgfpathmoveto{\pgfqpoint{2.434720in}{3.051073in}}%
\pgfpathlineto{\pgfqpoint{2.434720in}{3.051073in}}%
\pgfpathlineto{\pgfqpoint{2.434720in}{3.054022in}}%
\pgfpathlineto{\pgfqpoint{2.439261in}{3.054022in}}%
\pgfpathlineto{\pgfqpoint{2.439261in}{3.051073in}}%
\pgfpathmoveto{\pgfqpoint{2.430179in}{3.054022in}}%
\pgfpathlineto{\pgfqpoint{2.430179in}{3.054022in}}%
\pgfpathlineto{\pgfqpoint{2.430179in}{3.056971in}}%
\pgfpathlineto{\pgfqpoint{2.434720in}{3.056971in}}%
\pgfpathlineto{\pgfqpoint{2.434720in}{3.054022in}}%
\pgfpathmoveto{\pgfqpoint{2.434720in}{3.054022in}}%
\pgfpathlineto{\pgfqpoint{2.434720in}{3.054022in}}%
\pgfpathlineto{\pgfqpoint{2.434720in}{3.056971in}}%
\pgfpathlineto{\pgfqpoint{2.439261in}{3.056971in}}%
\pgfpathlineto{\pgfqpoint{2.439261in}{3.054022in}}%
\pgfpathmoveto{\pgfqpoint{2.439261in}{3.051073in}}%
\pgfpathlineto{\pgfqpoint{2.439261in}{3.051073in}}%
\pgfpathlineto{\pgfqpoint{2.439261in}{3.054022in}}%
\pgfpathlineto{\pgfqpoint{2.443802in}{3.054022in}}%
\pgfpathlineto{\pgfqpoint{2.443802in}{3.051073in}}%
\pgfpathmoveto{\pgfqpoint{2.443802in}{3.051073in}}%
\pgfpathlineto{\pgfqpoint{2.443802in}{3.051073in}}%
\pgfpathlineto{\pgfqpoint{2.443802in}{3.054022in}}%
\pgfpathlineto{\pgfqpoint{2.448343in}{3.054022in}}%
\pgfpathlineto{\pgfqpoint{2.448343in}{3.051073in}}%
\pgfpathmoveto{\pgfqpoint{2.448343in}{3.048123in}}%
\pgfpathlineto{\pgfqpoint{2.448343in}{3.048123in}}%
\pgfpathlineto{\pgfqpoint{2.448343in}{3.051073in}}%
\pgfpathlineto{\pgfqpoint{2.452884in}{3.051073in}}%
\pgfpathlineto{\pgfqpoint{2.452884in}{3.048123in}}%
\pgfpathmoveto{\pgfqpoint{2.448343in}{3.051073in}}%
\pgfpathlineto{\pgfqpoint{2.448343in}{3.051073in}}%
\pgfpathlineto{\pgfqpoint{2.448343in}{3.054022in}}%
\pgfpathlineto{\pgfqpoint{2.452884in}{3.054022in}}%
\pgfpathlineto{\pgfqpoint{2.452884in}{3.051073in}}%
\pgfpathmoveto{\pgfqpoint{2.452884in}{3.048123in}}%
\pgfpathlineto{\pgfqpoint{2.452884in}{3.048123in}}%
\pgfpathlineto{\pgfqpoint{2.452884in}{3.051073in}}%
\pgfpathlineto{\pgfqpoint{2.457425in}{3.051073in}}%
\pgfpathlineto{\pgfqpoint{2.457425in}{3.048123in}}%
\pgfpathmoveto{\pgfqpoint{2.457425in}{3.048123in}}%
\pgfpathlineto{\pgfqpoint{2.457425in}{3.048123in}}%
\pgfpathlineto{\pgfqpoint{2.457425in}{3.051073in}}%
\pgfpathlineto{\pgfqpoint{2.461966in}{3.051073in}}%
\pgfpathlineto{\pgfqpoint{2.461966in}{3.048123in}}%
\pgfpathmoveto{\pgfqpoint{2.461966in}{3.048123in}}%
\pgfpathlineto{\pgfqpoint{2.461966in}{3.048123in}}%
\pgfpathlineto{\pgfqpoint{2.461966in}{3.051073in}}%
\pgfpathlineto{\pgfqpoint{2.466507in}{3.051073in}}%
\pgfpathlineto{\pgfqpoint{2.466507in}{3.048123in}}%
\pgfpathmoveto{\pgfqpoint{2.493753in}{3.039276in}}%
\pgfpathlineto{\pgfqpoint{2.493753in}{3.039276in}}%
\pgfpathlineto{\pgfqpoint{2.493753in}{3.042225in}}%
\pgfpathlineto{\pgfqpoint{2.498294in}{3.042225in}}%
\pgfpathlineto{\pgfqpoint{2.498294in}{3.039276in}}%
\pgfpathmoveto{\pgfqpoint{2.498294in}{3.039276in}}%
\pgfpathlineto{\pgfqpoint{2.498294in}{3.039276in}}%
\pgfpathlineto{\pgfqpoint{2.498294in}{3.042225in}}%
\pgfpathlineto{\pgfqpoint{2.502835in}{3.042225in}}%
\pgfpathlineto{\pgfqpoint{2.502835in}{3.039276in}}%
\pgfpathmoveto{\pgfqpoint{2.502835in}{3.036327in}}%
\pgfpathlineto{\pgfqpoint{2.502835in}{3.036327in}}%
\pgfpathlineto{\pgfqpoint{2.502835in}{3.039276in}}%
\pgfpathlineto{\pgfqpoint{2.507376in}{3.039276in}}%
\pgfpathlineto{\pgfqpoint{2.507376in}{3.036327in}}%
\pgfpathmoveto{\pgfqpoint{2.502835in}{3.039276in}}%
\pgfpathlineto{\pgfqpoint{2.502835in}{3.039276in}}%
\pgfpathlineto{\pgfqpoint{2.502835in}{3.042225in}}%
\pgfpathlineto{\pgfqpoint{2.507376in}{3.042225in}}%
\pgfpathlineto{\pgfqpoint{2.507376in}{3.039276in}}%
\pgfpathmoveto{\pgfqpoint{2.507376in}{3.036327in}}%
\pgfpathlineto{\pgfqpoint{2.507376in}{3.036327in}}%
\pgfpathlineto{\pgfqpoint{2.507376in}{3.039276in}}%
\pgfpathlineto{\pgfqpoint{2.511917in}{3.039276in}}%
\pgfpathlineto{\pgfqpoint{2.511917in}{3.036327in}}%
\pgfpathmoveto{\pgfqpoint{2.516457in}{3.033378in}}%
\pgfpathlineto{\pgfqpoint{2.516457in}{3.033378in}}%
\pgfpathlineto{\pgfqpoint{2.516457in}{3.036327in}}%
\pgfpathlineto{\pgfqpoint{2.520998in}{3.036327in}}%
\pgfpathlineto{\pgfqpoint{2.520998in}{3.033378in}}%
\pgfpathmoveto{\pgfqpoint{2.520998in}{3.033378in}}%
\pgfpathlineto{\pgfqpoint{2.520998in}{3.033378in}}%
\pgfpathlineto{\pgfqpoint{2.520998in}{3.036327in}}%
\pgfpathlineto{\pgfqpoint{2.525539in}{3.036327in}}%
\pgfpathlineto{\pgfqpoint{2.525539in}{3.033378in}}%
\pgfpathmoveto{\pgfqpoint{2.525539in}{3.033378in}}%
\pgfpathlineto{\pgfqpoint{2.525539in}{3.033378in}}%
\pgfpathlineto{\pgfqpoint{2.525539in}{3.036327in}}%
\pgfpathlineto{\pgfqpoint{2.530080in}{3.036327in}}%
\pgfpathlineto{\pgfqpoint{2.530080in}{3.033378in}}%
\pgfpathmoveto{\pgfqpoint{2.511917in}{3.036327in}}%
\pgfpathlineto{\pgfqpoint{2.511917in}{3.036327in}}%
\pgfpathlineto{\pgfqpoint{2.511917in}{3.039276in}}%
\pgfpathlineto{\pgfqpoint{2.516457in}{3.039276in}}%
\pgfpathlineto{\pgfqpoint{2.516457in}{3.036327in}}%
\pgfpathmoveto{\pgfqpoint{2.516457in}{3.036327in}}%
\pgfpathlineto{\pgfqpoint{2.516457in}{3.036327in}}%
\pgfpathlineto{\pgfqpoint{2.516457in}{3.039276in}}%
\pgfpathlineto{\pgfqpoint{2.520998in}{3.039276in}}%
\pgfpathlineto{\pgfqpoint{2.520998in}{3.036327in}}%
\pgfpathmoveto{\pgfqpoint{2.530080in}{3.030429in}}%
\pgfpathlineto{\pgfqpoint{2.530080in}{3.030429in}}%
\pgfpathlineto{\pgfqpoint{2.530080in}{3.033378in}}%
\pgfpathlineto{\pgfqpoint{2.534621in}{3.033378in}}%
\pgfpathlineto{\pgfqpoint{2.534621in}{3.030429in}}%
\pgfpathmoveto{\pgfqpoint{2.530080in}{3.033378in}}%
\pgfpathlineto{\pgfqpoint{2.530080in}{3.033378in}}%
\pgfpathlineto{\pgfqpoint{2.530080in}{3.036327in}}%
\pgfpathlineto{\pgfqpoint{2.534621in}{3.036327in}}%
\pgfpathlineto{\pgfqpoint{2.534621in}{3.033378in}}%
\pgfpathmoveto{\pgfqpoint{2.534621in}{3.030429in}}%
\pgfpathlineto{\pgfqpoint{2.534621in}{3.030429in}}%
\pgfpathlineto{\pgfqpoint{2.534621in}{3.033378in}}%
\pgfpathlineto{\pgfqpoint{2.539162in}{3.033378in}}%
\pgfpathlineto{\pgfqpoint{2.539162in}{3.030429in}}%
\pgfpathmoveto{\pgfqpoint{2.543703in}{3.027479in}}%
\pgfpathlineto{\pgfqpoint{2.543703in}{3.027479in}}%
\pgfpathlineto{\pgfqpoint{2.543703in}{3.030429in}}%
\pgfpathlineto{\pgfqpoint{2.548243in}{3.030429in}}%
\pgfpathlineto{\pgfqpoint{2.548243in}{3.027479in}}%
\pgfpathmoveto{\pgfqpoint{2.539162in}{3.030429in}}%
\pgfpathlineto{\pgfqpoint{2.539162in}{3.030429in}}%
\pgfpathlineto{\pgfqpoint{2.539162in}{3.033378in}}%
\pgfpathlineto{\pgfqpoint{2.543703in}{3.033378in}}%
\pgfpathlineto{\pgfqpoint{2.543703in}{3.030429in}}%
\pgfpathmoveto{\pgfqpoint{2.543703in}{3.030429in}}%
\pgfpathlineto{\pgfqpoint{2.543703in}{3.030429in}}%
\pgfpathlineto{\pgfqpoint{2.543703in}{3.033378in}}%
\pgfpathlineto{\pgfqpoint{2.548243in}{3.033378in}}%
\pgfpathlineto{\pgfqpoint{2.548243in}{3.030429in}}%
\pgfpathmoveto{\pgfqpoint{2.548243in}{3.027479in}}%
\pgfpathlineto{\pgfqpoint{2.548243in}{3.027479in}}%
\pgfpathlineto{\pgfqpoint{2.548243in}{3.030429in}}%
\pgfpathlineto{\pgfqpoint{2.552784in}{3.030429in}}%
\pgfpathlineto{\pgfqpoint{2.552784in}{3.027479in}}%
\pgfpathmoveto{\pgfqpoint{2.552784in}{3.027479in}}%
\pgfpathlineto{\pgfqpoint{2.552784in}{3.027479in}}%
\pgfpathlineto{\pgfqpoint{2.552784in}{3.030429in}}%
\pgfpathlineto{\pgfqpoint{2.557325in}{3.030429in}}%
\pgfpathlineto{\pgfqpoint{2.557325in}{3.027479in}}%
\pgfpathmoveto{\pgfqpoint{2.557325in}{3.024530in}}%
\pgfpathlineto{\pgfqpoint{2.557325in}{3.024530in}}%
\pgfpathlineto{\pgfqpoint{2.557325in}{3.027479in}}%
\pgfpathlineto{\pgfqpoint{2.561866in}{3.027479in}}%
\pgfpathlineto{\pgfqpoint{2.561866in}{3.024530in}}%
\pgfpathmoveto{\pgfqpoint{2.557325in}{3.027479in}}%
\pgfpathlineto{\pgfqpoint{2.557325in}{3.027479in}}%
\pgfpathlineto{\pgfqpoint{2.557325in}{3.030429in}}%
\pgfpathlineto{\pgfqpoint{2.561866in}{3.030429in}}%
\pgfpathlineto{\pgfqpoint{2.561866in}{3.027479in}}%
\pgfpathmoveto{\pgfqpoint{2.561866in}{3.024530in}}%
\pgfpathlineto{\pgfqpoint{2.561866in}{3.024530in}}%
\pgfpathlineto{\pgfqpoint{2.561866in}{3.027479in}}%
\pgfpathlineto{\pgfqpoint{2.566407in}{3.027479in}}%
\pgfpathlineto{\pgfqpoint{2.566407in}{3.024530in}}%
\pgfpathmoveto{\pgfqpoint{2.570948in}{3.021581in}}%
\pgfpathlineto{\pgfqpoint{2.570948in}{3.021581in}}%
\pgfpathlineto{\pgfqpoint{2.570948in}{3.024530in}}%
\pgfpathlineto{\pgfqpoint{2.575489in}{3.024530in}}%
\pgfpathlineto{\pgfqpoint{2.575489in}{3.021581in}}%
\pgfpathmoveto{\pgfqpoint{2.575489in}{3.021581in}}%
\pgfpathlineto{\pgfqpoint{2.575489in}{3.021581in}}%
\pgfpathlineto{\pgfqpoint{2.575489in}{3.024530in}}%
\pgfpathlineto{\pgfqpoint{2.580030in}{3.024530in}}%
\pgfpathlineto{\pgfqpoint{2.580030in}{3.021581in}}%
\pgfpathmoveto{\pgfqpoint{2.580030in}{3.021581in}}%
\pgfpathlineto{\pgfqpoint{2.580030in}{3.021581in}}%
\pgfpathlineto{\pgfqpoint{2.580030in}{3.024530in}}%
\pgfpathlineto{\pgfqpoint{2.584570in}{3.024530in}}%
\pgfpathlineto{\pgfqpoint{2.584570in}{3.021581in}}%
\pgfpathmoveto{\pgfqpoint{2.584570in}{3.018632in}}%
\pgfpathlineto{\pgfqpoint{2.584570in}{3.018632in}}%
\pgfpathlineto{\pgfqpoint{2.584570in}{3.021581in}}%
\pgfpathlineto{\pgfqpoint{2.589111in}{3.021581in}}%
\pgfpathlineto{\pgfqpoint{2.589111in}{3.018632in}}%
\pgfpathmoveto{\pgfqpoint{2.584570in}{3.021581in}}%
\pgfpathlineto{\pgfqpoint{2.584570in}{3.021581in}}%
\pgfpathlineto{\pgfqpoint{2.584570in}{3.024530in}}%
\pgfpathlineto{\pgfqpoint{2.589111in}{3.024530in}}%
\pgfpathlineto{\pgfqpoint{2.589111in}{3.021581in}}%
\pgfpathmoveto{\pgfqpoint{2.589111in}{3.018632in}}%
\pgfpathlineto{\pgfqpoint{2.589111in}{3.018632in}}%
\pgfpathlineto{\pgfqpoint{2.589111in}{3.021581in}}%
\pgfpathlineto{\pgfqpoint{2.593652in}{3.021581in}}%
\pgfpathlineto{\pgfqpoint{2.593652in}{3.018632in}}%
\pgfpathmoveto{\pgfqpoint{2.598193in}{3.015683in}}%
\pgfpathlineto{\pgfqpoint{2.598193in}{3.015683in}}%
\pgfpathlineto{\pgfqpoint{2.598193in}{3.018632in}}%
\pgfpathlineto{\pgfqpoint{2.602734in}{3.018632in}}%
\pgfpathlineto{\pgfqpoint{2.602734in}{3.015683in}}%
\pgfpathmoveto{\pgfqpoint{2.593652in}{3.018632in}}%
\pgfpathlineto{\pgfqpoint{2.593652in}{3.018632in}}%
\pgfpathlineto{\pgfqpoint{2.593652in}{3.021581in}}%
\pgfpathlineto{\pgfqpoint{2.598193in}{3.021581in}}%
\pgfpathlineto{\pgfqpoint{2.598193in}{3.018632in}}%
\pgfpathmoveto{\pgfqpoint{2.598193in}{3.018632in}}%
\pgfpathlineto{\pgfqpoint{2.598193in}{3.018632in}}%
\pgfpathlineto{\pgfqpoint{2.598193in}{3.021581in}}%
\pgfpathlineto{\pgfqpoint{2.602734in}{3.021581in}}%
\pgfpathlineto{\pgfqpoint{2.602734in}{3.018632in}}%
\pgfpathmoveto{\pgfqpoint{2.566407in}{3.024530in}}%
\pgfpathlineto{\pgfqpoint{2.566407in}{3.024530in}}%
\pgfpathlineto{\pgfqpoint{2.566407in}{3.027479in}}%
\pgfpathlineto{\pgfqpoint{2.570948in}{3.027479in}}%
\pgfpathlineto{\pgfqpoint{2.570948in}{3.024530in}}%
\pgfpathmoveto{\pgfqpoint{2.570948in}{3.024530in}}%
\pgfpathlineto{\pgfqpoint{2.570948in}{3.024530in}}%
\pgfpathlineto{\pgfqpoint{2.570948in}{3.027479in}}%
\pgfpathlineto{\pgfqpoint{2.575489in}{3.027479in}}%
\pgfpathlineto{\pgfqpoint{2.575489in}{3.024530in}}%
\pgfpathmoveto{\pgfqpoint{2.602734in}{3.015683in}}%
\pgfpathlineto{\pgfqpoint{2.602734in}{3.015683in}}%
\pgfpathlineto{\pgfqpoint{2.602734in}{3.018632in}}%
\pgfpathlineto{\pgfqpoint{2.607275in}{3.018632in}}%
\pgfpathlineto{\pgfqpoint{2.607275in}{3.015683in}}%
\pgfpathmoveto{\pgfqpoint{2.607275in}{3.015683in}}%
\pgfpathlineto{\pgfqpoint{2.607275in}{3.015683in}}%
\pgfpathlineto{\pgfqpoint{2.607275in}{3.018632in}}%
\pgfpathlineto{\pgfqpoint{2.611816in}{3.018632in}}%
\pgfpathlineto{\pgfqpoint{2.611816in}{3.015683in}}%
\pgfpathmoveto{\pgfqpoint{2.611816in}{3.012734in}}%
\pgfpathlineto{\pgfqpoint{2.611816in}{3.012734in}}%
\pgfpathlineto{\pgfqpoint{2.611816in}{3.015683in}}%
\pgfpathlineto{\pgfqpoint{2.616357in}{3.015683in}}%
\pgfpathlineto{\pgfqpoint{2.616357in}{3.012734in}}%
\pgfpathmoveto{\pgfqpoint{2.611816in}{3.015683in}}%
\pgfpathlineto{\pgfqpoint{2.611816in}{3.015683in}}%
\pgfpathlineto{\pgfqpoint{2.611816in}{3.018632in}}%
\pgfpathlineto{\pgfqpoint{2.616357in}{3.018632in}}%
\pgfpathlineto{\pgfqpoint{2.616357in}{3.015683in}}%
\pgfpathmoveto{\pgfqpoint{2.616357in}{3.012734in}}%
\pgfpathlineto{\pgfqpoint{2.616357in}{3.012734in}}%
\pgfpathlineto{\pgfqpoint{2.616357in}{3.015683in}}%
\pgfpathlineto{\pgfqpoint{2.620897in}{3.015683in}}%
\pgfpathlineto{\pgfqpoint{2.620897in}{3.012734in}}%
\pgfpathmoveto{\pgfqpoint{2.625438in}{3.009785in}}%
\pgfpathlineto{\pgfqpoint{2.625438in}{3.009785in}}%
\pgfpathlineto{\pgfqpoint{2.625438in}{3.012734in}}%
\pgfpathlineto{\pgfqpoint{2.629979in}{3.012734in}}%
\pgfpathlineto{\pgfqpoint{2.629979in}{3.009785in}}%
\pgfpathmoveto{\pgfqpoint{2.629979in}{3.009785in}}%
\pgfpathlineto{\pgfqpoint{2.629979in}{3.009785in}}%
\pgfpathlineto{\pgfqpoint{2.629979in}{3.012734in}}%
\pgfpathlineto{\pgfqpoint{2.634520in}{3.012734in}}%
\pgfpathlineto{\pgfqpoint{2.634520in}{3.009785in}}%
\pgfpathmoveto{\pgfqpoint{2.634520in}{3.009785in}}%
\pgfpathlineto{\pgfqpoint{2.634520in}{3.009785in}}%
\pgfpathlineto{\pgfqpoint{2.634520in}{3.012734in}}%
\pgfpathlineto{\pgfqpoint{2.639061in}{3.012734in}}%
\pgfpathlineto{\pgfqpoint{2.639061in}{3.009785in}}%
\pgfpathmoveto{\pgfqpoint{2.620897in}{3.012734in}}%
\pgfpathlineto{\pgfqpoint{2.620897in}{3.012734in}}%
\pgfpathlineto{\pgfqpoint{2.620897in}{3.015683in}}%
\pgfpathlineto{\pgfqpoint{2.625438in}{3.015683in}}%
\pgfpathlineto{\pgfqpoint{2.625438in}{3.012734in}}%
\pgfpathmoveto{\pgfqpoint{2.625438in}{3.012734in}}%
\pgfpathlineto{\pgfqpoint{2.625438in}{3.012734in}}%
\pgfpathlineto{\pgfqpoint{2.625438in}{3.015683in}}%
\pgfpathlineto{\pgfqpoint{2.629979in}{3.015683in}}%
\pgfpathlineto{\pgfqpoint{2.629979in}{3.012734in}}%
\pgfpathmoveto{\pgfqpoint{2.679931in}{2.997988in}}%
\pgfpathlineto{\pgfqpoint{2.679931in}{2.997988in}}%
\pgfpathlineto{\pgfqpoint{2.679931in}{3.000937in}}%
\pgfpathlineto{\pgfqpoint{2.684473in}{3.000937in}}%
\pgfpathlineto{\pgfqpoint{2.684473in}{2.997988in}}%
\pgfpathmoveto{\pgfqpoint{2.684473in}{2.997988in}}%
\pgfpathlineto{\pgfqpoint{2.684473in}{2.997988in}}%
\pgfpathlineto{\pgfqpoint{2.684473in}{3.000937in}}%
\pgfpathlineto{\pgfqpoint{2.689014in}{3.000937in}}%
\pgfpathlineto{\pgfqpoint{2.689014in}{2.997988in}}%
\pgfpathmoveto{\pgfqpoint{2.689014in}{2.997988in}}%
\pgfpathlineto{\pgfqpoint{2.689014in}{2.997988in}}%
\pgfpathlineto{\pgfqpoint{2.689014in}{3.000937in}}%
\pgfpathlineto{\pgfqpoint{2.693555in}{3.000937in}}%
\pgfpathlineto{\pgfqpoint{2.693555in}{2.997988in}}%
\pgfpathmoveto{\pgfqpoint{2.693555in}{2.995039in}}%
\pgfpathlineto{\pgfqpoint{2.693555in}{2.995039in}}%
\pgfpathlineto{\pgfqpoint{2.693555in}{2.997988in}}%
\pgfpathlineto{\pgfqpoint{2.698096in}{2.997988in}}%
\pgfpathlineto{\pgfqpoint{2.698096in}{2.995039in}}%
\pgfpathmoveto{\pgfqpoint{2.693555in}{2.997988in}}%
\pgfpathlineto{\pgfqpoint{2.693555in}{2.997988in}}%
\pgfpathlineto{\pgfqpoint{2.693555in}{3.000937in}}%
\pgfpathlineto{\pgfqpoint{2.698096in}{3.000937in}}%
\pgfpathlineto{\pgfqpoint{2.698096in}{2.997988in}}%
\pgfpathmoveto{\pgfqpoint{2.698096in}{2.995039in}}%
\pgfpathlineto{\pgfqpoint{2.698096in}{2.995039in}}%
\pgfpathlineto{\pgfqpoint{2.698096in}{2.997988in}}%
\pgfpathlineto{\pgfqpoint{2.702637in}{2.997988in}}%
\pgfpathlineto{\pgfqpoint{2.702637in}{2.995039in}}%
\pgfpathmoveto{\pgfqpoint{2.707178in}{2.992090in}}%
\pgfpathlineto{\pgfqpoint{2.707178in}{2.992090in}}%
\pgfpathlineto{\pgfqpoint{2.707178in}{2.995039in}}%
\pgfpathlineto{\pgfqpoint{2.711720in}{2.995039in}}%
\pgfpathlineto{\pgfqpoint{2.711720in}{2.992090in}}%
\pgfpathmoveto{\pgfqpoint{2.702637in}{2.995039in}}%
\pgfpathlineto{\pgfqpoint{2.702637in}{2.995039in}}%
\pgfpathlineto{\pgfqpoint{2.702637in}{2.997988in}}%
\pgfpathlineto{\pgfqpoint{2.707178in}{2.997988in}}%
\pgfpathlineto{\pgfqpoint{2.707178in}{2.995039in}}%
\pgfpathmoveto{\pgfqpoint{2.707178in}{2.995039in}}%
\pgfpathlineto{\pgfqpoint{2.707178in}{2.995039in}}%
\pgfpathlineto{\pgfqpoint{2.707178in}{2.997988in}}%
\pgfpathlineto{\pgfqpoint{2.711720in}{2.997988in}}%
\pgfpathlineto{\pgfqpoint{2.711720in}{2.995039in}}%
\pgfpathmoveto{\pgfqpoint{2.639061in}{3.006835in}}%
\pgfpathlineto{\pgfqpoint{2.639061in}{3.006835in}}%
\pgfpathlineto{\pgfqpoint{2.639061in}{3.009785in}}%
\pgfpathlineto{\pgfqpoint{2.643602in}{3.009785in}}%
\pgfpathlineto{\pgfqpoint{2.643602in}{3.006835in}}%
\pgfpathmoveto{\pgfqpoint{2.639061in}{3.009785in}}%
\pgfpathlineto{\pgfqpoint{2.639061in}{3.009785in}}%
\pgfpathlineto{\pgfqpoint{2.639061in}{3.012734in}}%
\pgfpathlineto{\pgfqpoint{2.643602in}{3.012734in}}%
\pgfpathlineto{\pgfqpoint{2.643602in}{3.009785in}}%
\pgfpathmoveto{\pgfqpoint{2.643602in}{3.006835in}}%
\pgfpathlineto{\pgfqpoint{2.643602in}{3.006835in}}%
\pgfpathlineto{\pgfqpoint{2.643602in}{3.009785in}}%
\pgfpathlineto{\pgfqpoint{2.648143in}{3.009785in}}%
\pgfpathlineto{\pgfqpoint{2.648143in}{3.006835in}}%
\pgfpathmoveto{\pgfqpoint{2.652684in}{3.003886in}}%
\pgfpathlineto{\pgfqpoint{2.652684in}{3.003886in}}%
\pgfpathlineto{\pgfqpoint{2.652684in}{3.006835in}}%
\pgfpathlineto{\pgfqpoint{2.657226in}{3.006835in}}%
\pgfpathlineto{\pgfqpoint{2.657226in}{3.003886in}}%
\pgfpathmoveto{\pgfqpoint{2.648143in}{3.006835in}}%
\pgfpathlineto{\pgfqpoint{2.648143in}{3.006835in}}%
\pgfpathlineto{\pgfqpoint{2.648143in}{3.009785in}}%
\pgfpathlineto{\pgfqpoint{2.652684in}{3.009785in}}%
\pgfpathlineto{\pgfqpoint{2.652684in}{3.006835in}}%
\pgfpathmoveto{\pgfqpoint{2.652684in}{3.006835in}}%
\pgfpathlineto{\pgfqpoint{2.652684in}{3.006835in}}%
\pgfpathlineto{\pgfqpoint{2.652684in}{3.009785in}}%
\pgfpathlineto{\pgfqpoint{2.657226in}{3.009785in}}%
\pgfpathlineto{\pgfqpoint{2.657226in}{3.006835in}}%
\pgfpathmoveto{\pgfqpoint{2.657226in}{3.003886in}}%
\pgfpathlineto{\pgfqpoint{2.657226in}{3.003886in}}%
\pgfpathlineto{\pgfqpoint{2.657226in}{3.006835in}}%
\pgfpathlineto{\pgfqpoint{2.661767in}{3.006835in}}%
\pgfpathlineto{\pgfqpoint{2.661767in}{3.003886in}}%
\pgfpathmoveto{\pgfqpoint{2.661767in}{3.003886in}}%
\pgfpathlineto{\pgfqpoint{2.661767in}{3.003886in}}%
\pgfpathlineto{\pgfqpoint{2.661767in}{3.006835in}}%
\pgfpathlineto{\pgfqpoint{2.666308in}{3.006835in}}%
\pgfpathlineto{\pgfqpoint{2.666308in}{3.003886in}}%
\pgfpathmoveto{\pgfqpoint{2.666308in}{3.000937in}}%
\pgfpathlineto{\pgfqpoint{2.666308in}{3.000937in}}%
\pgfpathlineto{\pgfqpoint{2.666308in}{3.003886in}}%
\pgfpathlineto{\pgfqpoint{2.670849in}{3.003886in}}%
\pgfpathlineto{\pgfqpoint{2.670849in}{3.000937in}}%
\pgfpathmoveto{\pgfqpoint{2.666308in}{3.003886in}}%
\pgfpathlineto{\pgfqpoint{2.666308in}{3.003886in}}%
\pgfpathlineto{\pgfqpoint{2.666308in}{3.006835in}}%
\pgfpathlineto{\pgfqpoint{2.670849in}{3.006835in}}%
\pgfpathlineto{\pgfqpoint{2.670849in}{3.003886in}}%
\pgfpathmoveto{\pgfqpoint{2.670849in}{3.000937in}}%
\pgfpathlineto{\pgfqpoint{2.670849in}{3.000937in}}%
\pgfpathlineto{\pgfqpoint{2.670849in}{3.003886in}}%
\pgfpathlineto{\pgfqpoint{2.675390in}{3.003886in}}%
\pgfpathlineto{\pgfqpoint{2.675390in}{3.000937in}}%
\pgfpathmoveto{\pgfqpoint{2.675390in}{3.000937in}}%
\pgfpathlineto{\pgfqpoint{2.675390in}{3.000937in}}%
\pgfpathlineto{\pgfqpoint{2.675390in}{3.003886in}}%
\pgfpathlineto{\pgfqpoint{2.679931in}{3.003886in}}%
\pgfpathlineto{\pgfqpoint{2.679931in}{3.000937in}}%
\pgfpathmoveto{\pgfqpoint{2.679931in}{3.000937in}}%
\pgfpathlineto{\pgfqpoint{2.679931in}{3.000937in}}%
\pgfpathlineto{\pgfqpoint{2.679931in}{3.003886in}}%
\pgfpathlineto{\pgfqpoint{2.684473in}{3.003886in}}%
\pgfpathlineto{\pgfqpoint{2.684473in}{3.000937in}}%
\pgfpathmoveto{\pgfqpoint{2.711720in}{2.992090in}}%
\pgfpathlineto{\pgfqpoint{2.711720in}{2.992090in}}%
\pgfpathlineto{\pgfqpoint{2.711720in}{2.995039in}}%
\pgfpathlineto{\pgfqpoint{2.716261in}{2.995039in}}%
\pgfpathlineto{\pgfqpoint{2.716261in}{2.992090in}}%
\pgfpathmoveto{\pgfqpoint{2.716261in}{2.992090in}}%
\pgfpathlineto{\pgfqpoint{2.716261in}{2.992090in}}%
\pgfpathlineto{\pgfqpoint{2.716261in}{2.995039in}}%
\pgfpathlineto{\pgfqpoint{2.720802in}{2.995039in}}%
\pgfpathlineto{\pgfqpoint{2.720802in}{2.992090in}}%
\pgfpathmoveto{\pgfqpoint{2.720802in}{2.989141in}}%
\pgfpathlineto{\pgfqpoint{2.720802in}{2.989141in}}%
\pgfpathlineto{\pgfqpoint{2.720802in}{2.992090in}}%
\pgfpathlineto{\pgfqpoint{2.725343in}{2.992090in}}%
\pgfpathlineto{\pgfqpoint{2.725343in}{2.989141in}}%
\pgfpathmoveto{\pgfqpoint{2.720802in}{2.992090in}}%
\pgfpathlineto{\pgfqpoint{2.720802in}{2.992090in}}%
\pgfpathlineto{\pgfqpoint{2.720802in}{2.995039in}}%
\pgfpathlineto{\pgfqpoint{2.725343in}{2.995039in}}%
\pgfpathlineto{\pgfqpoint{2.725343in}{2.992090in}}%
\pgfpathmoveto{\pgfqpoint{2.725343in}{2.989141in}}%
\pgfpathlineto{\pgfqpoint{2.725343in}{2.989141in}}%
\pgfpathlineto{\pgfqpoint{2.725343in}{2.992090in}}%
\pgfpathlineto{\pgfqpoint{2.729884in}{2.992090in}}%
\pgfpathlineto{\pgfqpoint{2.729884in}{2.989141in}}%
\pgfpathmoveto{\pgfqpoint{2.734425in}{2.986191in}}%
\pgfpathlineto{\pgfqpoint{2.734425in}{2.986191in}}%
\pgfpathlineto{\pgfqpoint{2.734425in}{2.989141in}}%
\pgfpathlineto{\pgfqpoint{2.738967in}{2.989141in}}%
\pgfpathlineto{\pgfqpoint{2.738967in}{2.986191in}}%
\pgfpathmoveto{\pgfqpoint{2.738967in}{2.986191in}}%
\pgfpathlineto{\pgfqpoint{2.738967in}{2.986191in}}%
\pgfpathlineto{\pgfqpoint{2.738967in}{2.989141in}}%
\pgfpathlineto{\pgfqpoint{2.743508in}{2.989141in}}%
\pgfpathlineto{\pgfqpoint{2.743508in}{2.986191in}}%
\pgfpathmoveto{\pgfqpoint{2.743508in}{2.986191in}}%
\pgfpathlineto{\pgfqpoint{2.743508in}{2.986191in}}%
\pgfpathlineto{\pgfqpoint{2.743508in}{2.989141in}}%
\pgfpathlineto{\pgfqpoint{2.748049in}{2.989141in}}%
\pgfpathlineto{\pgfqpoint{2.748049in}{2.986191in}}%
\pgfpathmoveto{\pgfqpoint{2.729884in}{2.989141in}}%
\pgfpathlineto{\pgfqpoint{2.729884in}{2.989141in}}%
\pgfpathlineto{\pgfqpoint{2.729884in}{2.992090in}}%
\pgfpathlineto{\pgfqpoint{2.734425in}{2.992090in}}%
\pgfpathlineto{\pgfqpoint{2.734425in}{2.989141in}}%
\pgfpathmoveto{\pgfqpoint{2.734425in}{2.989141in}}%
\pgfpathlineto{\pgfqpoint{2.734425in}{2.989141in}}%
\pgfpathlineto{\pgfqpoint{2.734425in}{2.992090in}}%
\pgfpathlineto{\pgfqpoint{2.738967in}{2.992090in}}%
\pgfpathlineto{\pgfqpoint{2.738967in}{2.989141in}}%
\pgfpathmoveto{\pgfqpoint{2.748049in}{2.983242in}}%
\pgfpathlineto{\pgfqpoint{2.748049in}{2.983242in}}%
\pgfpathlineto{\pgfqpoint{2.748049in}{2.986191in}}%
\pgfpathlineto{\pgfqpoint{2.752590in}{2.986191in}}%
\pgfpathlineto{\pgfqpoint{2.752590in}{2.983242in}}%
\pgfpathmoveto{\pgfqpoint{2.748049in}{2.986191in}}%
\pgfpathlineto{\pgfqpoint{2.748049in}{2.986191in}}%
\pgfpathlineto{\pgfqpoint{2.748049in}{2.989141in}}%
\pgfpathlineto{\pgfqpoint{2.752590in}{2.989141in}}%
\pgfpathlineto{\pgfqpoint{2.752590in}{2.986191in}}%
\pgfpathmoveto{\pgfqpoint{2.752590in}{2.983242in}}%
\pgfpathlineto{\pgfqpoint{2.752590in}{2.983242in}}%
\pgfpathlineto{\pgfqpoint{2.752590in}{2.986191in}}%
\pgfpathlineto{\pgfqpoint{2.757131in}{2.986191in}}%
\pgfpathlineto{\pgfqpoint{2.757131in}{2.983242in}}%
\pgfpathmoveto{\pgfqpoint{2.761672in}{2.980293in}}%
\pgfpathlineto{\pgfqpoint{2.761672in}{2.980293in}}%
\pgfpathlineto{\pgfqpoint{2.761672in}{2.983242in}}%
\pgfpathlineto{\pgfqpoint{2.766214in}{2.983242in}}%
\pgfpathlineto{\pgfqpoint{2.766214in}{2.980293in}}%
\pgfpathmoveto{\pgfqpoint{2.757131in}{2.983242in}}%
\pgfpathlineto{\pgfqpoint{2.757131in}{2.983242in}}%
\pgfpathlineto{\pgfqpoint{2.757131in}{2.986191in}}%
\pgfpathlineto{\pgfqpoint{2.761672in}{2.986191in}}%
\pgfpathlineto{\pgfqpoint{2.761672in}{2.983242in}}%
\pgfpathmoveto{\pgfqpoint{2.761672in}{2.983242in}}%
\pgfpathlineto{\pgfqpoint{2.761672in}{2.983242in}}%
\pgfpathlineto{\pgfqpoint{2.761672in}{2.986191in}}%
\pgfpathlineto{\pgfqpoint{2.766214in}{2.986191in}}%
\pgfpathlineto{\pgfqpoint{2.766214in}{2.983242in}}%
\pgfpathmoveto{\pgfqpoint{2.766214in}{2.980293in}}%
\pgfpathlineto{\pgfqpoint{2.766214in}{2.980293in}}%
\pgfpathlineto{\pgfqpoint{2.766214in}{2.983242in}}%
\pgfpathlineto{\pgfqpoint{2.770755in}{2.983242in}}%
\pgfpathlineto{\pgfqpoint{2.770755in}{2.980293in}}%
\pgfpathmoveto{\pgfqpoint{2.770755in}{2.980293in}}%
\pgfpathlineto{\pgfqpoint{2.770755in}{2.980293in}}%
\pgfpathlineto{\pgfqpoint{2.770755in}{2.983242in}}%
\pgfpathlineto{\pgfqpoint{2.775296in}{2.983242in}}%
\pgfpathlineto{\pgfqpoint{2.775296in}{2.980293in}}%
\pgfpathmoveto{\pgfqpoint{2.775296in}{2.977344in}}%
\pgfpathlineto{\pgfqpoint{2.775296in}{2.977344in}}%
\pgfpathlineto{\pgfqpoint{2.775296in}{2.980293in}}%
\pgfpathlineto{\pgfqpoint{2.779837in}{2.980293in}}%
\pgfpathlineto{\pgfqpoint{2.779837in}{2.977344in}}%
\pgfpathmoveto{\pgfqpoint{2.775296in}{2.980293in}}%
\pgfpathlineto{\pgfqpoint{2.775296in}{2.980293in}}%
\pgfpathlineto{\pgfqpoint{2.775296in}{2.983242in}}%
\pgfpathlineto{\pgfqpoint{2.779837in}{2.983242in}}%
\pgfpathlineto{\pgfqpoint{2.779837in}{2.980293in}}%
\pgfpathmoveto{\pgfqpoint{2.779837in}{2.977344in}}%
\pgfpathlineto{\pgfqpoint{2.779837in}{2.977344in}}%
\pgfpathlineto{\pgfqpoint{2.779837in}{2.980293in}}%
\pgfpathlineto{\pgfqpoint{2.784378in}{2.980293in}}%
\pgfpathlineto{\pgfqpoint{2.784378in}{2.977344in}}%
\pgfpathmoveto{\pgfqpoint{2.897898in}{2.950802in}}%
\pgfpathlineto{\pgfqpoint{2.897898in}{2.950802in}}%
\pgfpathlineto{\pgfqpoint{2.897898in}{2.953751in}}%
\pgfpathlineto{\pgfqpoint{2.902439in}{2.953751in}}%
\pgfpathlineto{\pgfqpoint{2.902439in}{2.950802in}}%
\pgfpathmoveto{\pgfqpoint{2.902439in}{2.950802in}}%
\pgfpathlineto{\pgfqpoint{2.902439in}{2.950802in}}%
\pgfpathlineto{\pgfqpoint{2.902439in}{2.953751in}}%
\pgfpathlineto{\pgfqpoint{2.906980in}{2.953751in}}%
\pgfpathlineto{\pgfqpoint{2.906980in}{2.950802in}}%
\pgfpathmoveto{\pgfqpoint{2.906980in}{2.950802in}}%
\pgfpathlineto{\pgfqpoint{2.906980in}{2.950802in}}%
\pgfpathlineto{\pgfqpoint{2.906980in}{2.953751in}}%
\pgfpathlineto{\pgfqpoint{2.911520in}{2.953751in}}%
\pgfpathlineto{\pgfqpoint{2.911520in}{2.950802in}}%
\pgfpathmoveto{\pgfqpoint{2.911520in}{2.947852in}}%
\pgfpathlineto{\pgfqpoint{2.911520in}{2.947852in}}%
\pgfpathlineto{\pgfqpoint{2.911520in}{2.950802in}}%
\pgfpathlineto{\pgfqpoint{2.916061in}{2.950802in}}%
\pgfpathlineto{\pgfqpoint{2.916061in}{2.947852in}}%
\pgfpathmoveto{\pgfqpoint{2.911520in}{2.950802in}}%
\pgfpathlineto{\pgfqpoint{2.911520in}{2.950802in}}%
\pgfpathlineto{\pgfqpoint{2.911520in}{2.953751in}}%
\pgfpathlineto{\pgfqpoint{2.916061in}{2.953751in}}%
\pgfpathlineto{\pgfqpoint{2.916061in}{2.950802in}}%
\pgfpathmoveto{\pgfqpoint{2.916061in}{2.947852in}}%
\pgfpathlineto{\pgfqpoint{2.916061in}{2.947852in}}%
\pgfpathlineto{\pgfqpoint{2.916061in}{2.950802in}}%
\pgfpathlineto{\pgfqpoint{2.920602in}{2.950802in}}%
\pgfpathlineto{\pgfqpoint{2.920602in}{2.947852in}}%
\pgfpathmoveto{\pgfqpoint{2.925143in}{2.944903in}}%
\pgfpathlineto{\pgfqpoint{2.925143in}{2.944903in}}%
\pgfpathlineto{\pgfqpoint{2.925143in}{2.947852in}}%
\pgfpathlineto{\pgfqpoint{2.929683in}{2.947852in}}%
\pgfpathlineto{\pgfqpoint{2.929683in}{2.944903in}}%
\pgfpathmoveto{\pgfqpoint{2.920602in}{2.947852in}}%
\pgfpathlineto{\pgfqpoint{2.920602in}{2.947852in}}%
\pgfpathlineto{\pgfqpoint{2.920602in}{2.950802in}}%
\pgfpathlineto{\pgfqpoint{2.925143in}{2.950802in}}%
\pgfpathlineto{\pgfqpoint{2.925143in}{2.947852in}}%
\pgfpathmoveto{\pgfqpoint{2.925143in}{2.947852in}}%
\pgfpathlineto{\pgfqpoint{2.925143in}{2.947852in}}%
\pgfpathlineto{\pgfqpoint{2.925143in}{2.950802in}}%
\pgfpathlineto{\pgfqpoint{2.929683in}{2.950802in}}%
\pgfpathlineto{\pgfqpoint{2.929683in}{2.947852in}}%
\pgfpathmoveto{\pgfqpoint{2.788919in}{2.974395in}}%
\pgfpathlineto{\pgfqpoint{2.788919in}{2.974395in}}%
\pgfpathlineto{\pgfqpoint{2.788919in}{2.977344in}}%
\pgfpathlineto{\pgfqpoint{2.793460in}{2.977344in}}%
\pgfpathlineto{\pgfqpoint{2.793460in}{2.974395in}}%
\pgfpathmoveto{\pgfqpoint{2.793460in}{2.974395in}}%
\pgfpathlineto{\pgfqpoint{2.793460in}{2.974395in}}%
\pgfpathlineto{\pgfqpoint{2.793460in}{2.977344in}}%
\pgfpathlineto{\pgfqpoint{2.798001in}{2.977344in}}%
\pgfpathlineto{\pgfqpoint{2.798001in}{2.974395in}}%
\pgfpathmoveto{\pgfqpoint{2.798001in}{2.974395in}}%
\pgfpathlineto{\pgfqpoint{2.798001in}{2.974395in}}%
\pgfpathlineto{\pgfqpoint{2.798001in}{2.977344in}}%
\pgfpathlineto{\pgfqpoint{2.802541in}{2.977344in}}%
\pgfpathlineto{\pgfqpoint{2.802541in}{2.974395in}}%
\pgfpathmoveto{\pgfqpoint{2.802541in}{2.971446in}}%
\pgfpathlineto{\pgfqpoint{2.802541in}{2.971446in}}%
\pgfpathlineto{\pgfqpoint{2.802541in}{2.974395in}}%
\pgfpathlineto{\pgfqpoint{2.807082in}{2.974395in}}%
\pgfpathlineto{\pgfqpoint{2.807082in}{2.971446in}}%
\pgfpathmoveto{\pgfqpoint{2.802541in}{2.974395in}}%
\pgfpathlineto{\pgfqpoint{2.802541in}{2.974395in}}%
\pgfpathlineto{\pgfqpoint{2.802541in}{2.977344in}}%
\pgfpathlineto{\pgfqpoint{2.807082in}{2.977344in}}%
\pgfpathlineto{\pgfqpoint{2.807082in}{2.974395in}}%
\pgfpathmoveto{\pgfqpoint{2.807082in}{2.971446in}}%
\pgfpathlineto{\pgfqpoint{2.807082in}{2.971446in}}%
\pgfpathlineto{\pgfqpoint{2.807082in}{2.974395in}}%
\pgfpathlineto{\pgfqpoint{2.811623in}{2.974395in}}%
\pgfpathlineto{\pgfqpoint{2.811623in}{2.971446in}}%
\pgfpathmoveto{\pgfqpoint{2.816164in}{2.968497in}}%
\pgfpathlineto{\pgfqpoint{2.816164in}{2.968497in}}%
\pgfpathlineto{\pgfqpoint{2.816164in}{2.971446in}}%
\pgfpathlineto{\pgfqpoint{2.820705in}{2.971446in}}%
\pgfpathlineto{\pgfqpoint{2.820705in}{2.968497in}}%
\pgfpathmoveto{\pgfqpoint{2.811623in}{2.971446in}}%
\pgfpathlineto{\pgfqpoint{2.811623in}{2.971446in}}%
\pgfpathlineto{\pgfqpoint{2.811623in}{2.974395in}}%
\pgfpathlineto{\pgfqpoint{2.816164in}{2.974395in}}%
\pgfpathlineto{\pgfqpoint{2.816164in}{2.971446in}}%
\pgfpathmoveto{\pgfqpoint{2.816164in}{2.971446in}}%
\pgfpathlineto{\pgfqpoint{2.816164in}{2.971446in}}%
\pgfpathlineto{\pgfqpoint{2.816164in}{2.974395in}}%
\pgfpathlineto{\pgfqpoint{2.820705in}{2.974395in}}%
\pgfpathlineto{\pgfqpoint{2.820705in}{2.971446in}}%
\pgfpathmoveto{\pgfqpoint{2.784378in}{2.977344in}}%
\pgfpathlineto{\pgfqpoint{2.784378in}{2.977344in}}%
\pgfpathlineto{\pgfqpoint{2.784378in}{2.980293in}}%
\pgfpathlineto{\pgfqpoint{2.788919in}{2.980293in}}%
\pgfpathlineto{\pgfqpoint{2.788919in}{2.977344in}}%
\pgfpathmoveto{\pgfqpoint{2.788919in}{2.977344in}}%
\pgfpathlineto{\pgfqpoint{2.788919in}{2.977344in}}%
\pgfpathlineto{\pgfqpoint{2.788919in}{2.980293in}}%
\pgfpathlineto{\pgfqpoint{2.793460in}{2.980293in}}%
\pgfpathlineto{\pgfqpoint{2.793460in}{2.977344in}}%
\pgfpathmoveto{\pgfqpoint{2.820705in}{2.968497in}}%
\pgfpathlineto{\pgfqpoint{2.820705in}{2.968497in}}%
\pgfpathlineto{\pgfqpoint{2.820705in}{2.971446in}}%
\pgfpathlineto{\pgfqpoint{2.825245in}{2.971446in}}%
\pgfpathlineto{\pgfqpoint{2.825245in}{2.968497in}}%
\pgfpathmoveto{\pgfqpoint{2.825245in}{2.968497in}}%
\pgfpathlineto{\pgfqpoint{2.825245in}{2.968497in}}%
\pgfpathlineto{\pgfqpoint{2.825245in}{2.971446in}}%
\pgfpathlineto{\pgfqpoint{2.829786in}{2.971446in}}%
\pgfpathlineto{\pgfqpoint{2.829786in}{2.968497in}}%
\pgfpathmoveto{\pgfqpoint{2.829786in}{2.965547in}}%
\pgfpathlineto{\pgfqpoint{2.829786in}{2.965547in}}%
\pgfpathlineto{\pgfqpoint{2.829786in}{2.968497in}}%
\pgfpathlineto{\pgfqpoint{2.834327in}{2.968497in}}%
\pgfpathlineto{\pgfqpoint{2.834327in}{2.965547in}}%
\pgfpathmoveto{\pgfqpoint{2.829786in}{2.968497in}}%
\pgfpathlineto{\pgfqpoint{2.829786in}{2.968497in}}%
\pgfpathlineto{\pgfqpoint{2.829786in}{2.971446in}}%
\pgfpathlineto{\pgfqpoint{2.834327in}{2.971446in}}%
\pgfpathlineto{\pgfqpoint{2.834327in}{2.968497in}}%
\pgfpathmoveto{\pgfqpoint{2.834327in}{2.965547in}}%
\pgfpathlineto{\pgfqpoint{2.834327in}{2.965547in}}%
\pgfpathlineto{\pgfqpoint{2.834327in}{2.968497in}}%
\pgfpathlineto{\pgfqpoint{2.838868in}{2.968497in}}%
\pgfpathlineto{\pgfqpoint{2.838868in}{2.965547in}}%
\pgfpathmoveto{\pgfqpoint{2.843408in}{2.962598in}}%
\pgfpathlineto{\pgfqpoint{2.843408in}{2.962598in}}%
\pgfpathlineto{\pgfqpoint{2.843408in}{2.965547in}}%
\pgfpathlineto{\pgfqpoint{2.847949in}{2.965547in}}%
\pgfpathlineto{\pgfqpoint{2.847949in}{2.962598in}}%
\pgfpathmoveto{\pgfqpoint{2.847949in}{2.962598in}}%
\pgfpathlineto{\pgfqpoint{2.847949in}{2.962598in}}%
\pgfpathlineto{\pgfqpoint{2.847949in}{2.965547in}}%
\pgfpathlineto{\pgfqpoint{2.852490in}{2.965547in}}%
\pgfpathlineto{\pgfqpoint{2.852490in}{2.962598in}}%
\pgfpathmoveto{\pgfqpoint{2.852490in}{2.962598in}}%
\pgfpathlineto{\pgfqpoint{2.852490in}{2.962598in}}%
\pgfpathlineto{\pgfqpoint{2.852490in}{2.965547in}}%
\pgfpathlineto{\pgfqpoint{2.857031in}{2.965547in}}%
\pgfpathlineto{\pgfqpoint{2.857031in}{2.962598in}}%
\pgfpathmoveto{\pgfqpoint{2.838868in}{2.965547in}}%
\pgfpathlineto{\pgfqpoint{2.838868in}{2.965547in}}%
\pgfpathlineto{\pgfqpoint{2.838868in}{2.968497in}}%
\pgfpathlineto{\pgfqpoint{2.843408in}{2.968497in}}%
\pgfpathlineto{\pgfqpoint{2.843408in}{2.965547in}}%
\pgfpathmoveto{\pgfqpoint{2.843408in}{2.965547in}}%
\pgfpathlineto{\pgfqpoint{2.843408in}{2.965547in}}%
\pgfpathlineto{\pgfqpoint{2.843408in}{2.968497in}}%
\pgfpathlineto{\pgfqpoint{2.847949in}{2.968497in}}%
\pgfpathlineto{\pgfqpoint{2.847949in}{2.965547in}}%
\pgfpathmoveto{\pgfqpoint{2.857031in}{2.959649in}}%
\pgfpathlineto{\pgfqpoint{2.857031in}{2.959649in}}%
\pgfpathlineto{\pgfqpoint{2.857031in}{2.962598in}}%
\pgfpathlineto{\pgfqpoint{2.861572in}{2.962598in}}%
\pgfpathlineto{\pgfqpoint{2.861572in}{2.959649in}}%
\pgfpathmoveto{\pgfqpoint{2.857031in}{2.962598in}}%
\pgfpathlineto{\pgfqpoint{2.857031in}{2.962598in}}%
\pgfpathlineto{\pgfqpoint{2.857031in}{2.965547in}}%
\pgfpathlineto{\pgfqpoint{2.861572in}{2.965547in}}%
\pgfpathlineto{\pgfqpoint{2.861572in}{2.962598in}}%
\pgfpathmoveto{\pgfqpoint{2.861572in}{2.959649in}}%
\pgfpathlineto{\pgfqpoint{2.861572in}{2.959649in}}%
\pgfpathlineto{\pgfqpoint{2.861572in}{2.962598in}}%
\pgfpathlineto{\pgfqpoint{2.866112in}{2.962598in}}%
\pgfpathlineto{\pgfqpoint{2.866112in}{2.959649in}}%
\pgfpathmoveto{\pgfqpoint{2.870653in}{2.956700in}}%
\pgfpathlineto{\pgfqpoint{2.870653in}{2.956700in}}%
\pgfpathlineto{\pgfqpoint{2.870653in}{2.959649in}}%
\pgfpathlineto{\pgfqpoint{2.875194in}{2.959649in}}%
\pgfpathlineto{\pgfqpoint{2.875194in}{2.956700in}}%
\pgfpathmoveto{\pgfqpoint{2.866112in}{2.959649in}}%
\pgfpathlineto{\pgfqpoint{2.866112in}{2.959649in}}%
\pgfpathlineto{\pgfqpoint{2.866112in}{2.962598in}}%
\pgfpathlineto{\pgfqpoint{2.870653in}{2.962598in}}%
\pgfpathlineto{\pgfqpoint{2.870653in}{2.959649in}}%
\pgfpathmoveto{\pgfqpoint{2.870653in}{2.959649in}}%
\pgfpathlineto{\pgfqpoint{2.870653in}{2.959649in}}%
\pgfpathlineto{\pgfqpoint{2.870653in}{2.962598in}}%
\pgfpathlineto{\pgfqpoint{2.875194in}{2.962598in}}%
\pgfpathlineto{\pgfqpoint{2.875194in}{2.959649in}}%
\pgfpathmoveto{\pgfqpoint{2.875194in}{2.956700in}}%
\pgfpathlineto{\pgfqpoint{2.875194in}{2.956700in}}%
\pgfpathlineto{\pgfqpoint{2.875194in}{2.959649in}}%
\pgfpathlineto{\pgfqpoint{2.879735in}{2.959649in}}%
\pgfpathlineto{\pgfqpoint{2.879735in}{2.956700in}}%
\pgfpathmoveto{\pgfqpoint{2.879735in}{2.956700in}}%
\pgfpathlineto{\pgfqpoint{2.879735in}{2.956700in}}%
\pgfpathlineto{\pgfqpoint{2.879735in}{2.959649in}}%
\pgfpathlineto{\pgfqpoint{2.884276in}{2.959649in}}%
\pgfpathlineto{\pgfqpoint{2.884276in}{2.956700in}}%
\pgfpathmoveto{\pgfqpoint{2.884276in}{2.953751in}}%
\pgfpathlineto{\pgfqpoint{2.884276in}{2.953751in}}%
\pgfpathlineto{\pgfqpoint{2.884276in}{2.956700in}}%
\pgfpathlineto{\pgfqpoint{2.888816in}{2.956700in}}%
\pgfpathlineto{\pgfqpoint{2.888816in}{2.953751in}}%
\pgfpathmoveto{\pgfqpoint{2.884276in}{2.956700in}}%
\pgfpathlineto{\pgfqpoint{2.884276in}{2.956700in}}%
\pgfpathlineto{\pgfqpoint{2.884276in}{2.959649in}}%
\pgfpathlineto{\pgfqpoint{2.888816in}{2.959649in}}%
\pgfpathlineto{\pgfqpoint{2.888816in}{2.956700in}}%
\pgfpathmoveto{\pgfqpoint{2.888816in}{2.953751in}}%
\pgfpathlineto{\pgfqpoint{2.888816in}{2.953751in}}%
\pgfpathlineto{\pgfqpoint{2.888816in}{2.956700in}}%
\pgfpathlineto{\pgfqpoint{2.893357in}{2.956700in}}%
\pgfpathlineto{\pgfqpoint{2.893357in}{2.953751in}}%
\pgfpathmoveto{\pgfqpoint{2.893357in}{2.953751in}}%
\pgfpathlineto{\pgfqpoint{2.893357in}{2.953751in}}%
\pgfpathlineto{\pgfqpoint{2.893357in}{2.956700in}}%
\pgfpathlineto{\pgfqpoint{2.897898in}{2.956700in}}%
\pgfpathlineto{\pgfqpoint{2.897898in}{2.953751in}}%
\pgfpathmoveto{\pgfqpoint{2.897898in}{2.953751in}}%
\pgfpathlineto{\pgfqpoint{2.897898in}{2.953751in}}%
\pgfpathlineto{\pgfqpoint{2.897898in}{2.956700in}}%
\pgfpathlineto{\pgfqpoint{2.902439in}{2.956700in}}%
\pgfpathlineto{\pgfqpoint{2.902439in}{2.953751in}}%
\pgfpathmoveto{\pgfqpoint{2.929683in}{2.944903in}}%
\pgfpathlineto{\pgfqpoint{2.929683in}{2.944903in}}%
\pgfpathlineto{\pgfqpoint{2.929683in}{2.947852in}}%
\pgfpathlineto{\pgfqpoint{2.934225in}{2.947852in}}%
\pgfpathlineto{\pgfqpoint{2.934225in}{2.944903in}}%
\pgfpathmoveto{\pgfqpoint{2.934225in}{2.944903in}}%
\pgfpathlineto{\pgfqpoint{2.934225in}{2.944903in}}%
\pgfpathlineto{\pgfqpoint{2.934225in}{2.947852in}}%
\pgfpathlineto{\pgfqpoint{2.938766in}{2.947852in}}%
\pgfpathlineto{\pgfqpoint{2.938766in}{2.944903in}}%
\pgfpathmoveto{\pgfqpoint{2.938766in}{2.941954in}}%
\pgfpathlineto{\pgfqpoint{2.938766in}{2.941954in}}%
\pgfpathlineto{\pgfqpoint{2.938766in}{2.944903in}}%
\pgfpathlineto{\pgfqpoint{2.943307in}{2.944903in}}%
\pgfpathlineto{\pgfqpoint{2.943307in}{2.941954in}}%
\pgfpathmoveto{\pgfqpoint{2.938766in}{2.944903in}}%
\pgfpathlineto{\pgfqpoint{2.938766in}{2.944903in}}%
\pgfpathlineto{\pgfqpoint{2.938766in}{2.947852in}}%
\pgfpathlineto{\pgfqpoint{2.943307in}{2.947852in}}%
\pgfpathlineto{\pgfqpoint{2.943307in}{2.944903in}}%
\pgfpathmoveto{\pgfqpoint{2.943307in}{2.941954in}}%
\pgfpathlineto{\pgfqpoint{2.943307in}{2.941954in}}%
\pgfpathlineto{\pgfqpoint{2.943307in}{2.944903in}}%
\pgfpathlineto{\pgfqpoint{2.947849in}{2.944903in}}%
\pgfpathlineto{\pgfqpoint{2.947849in}{2.941954in}}%
\pgfpathmoveto{\pgfqpoint{2.952390in}{2.939005in}}%
\pgfpathlineto{\pgfqpoint{2.952390in}{2.939005in}}%
\pgfpathlineto{\pgfqpoint{2.952390in}{2.941954in}}%
\pgfpathlineto{\pgfqpoint{2.956931in}{2.941954in}}%
\pgfpathlineto{\pgfqpoint{2.956931in}{2.939005in}}%
\pgfpathmoveto{\pgfqpoint{2.956931in}{2.939005in}}%
\pgfpathlineto{\pgfqpoint{2.956931in}{2.939005in}}%
\pgfpathlineto{\pgfqpoint{2.956931in}{2.941954in}}%
\pgfpathlineto{\pgfqpoint{2.961472in}{2.941954in}}%
\pgfpathlineto{\pgfqpoint{2.961472in}{2.939005in}}%
\pgfpathmoveto{\pgfqpoint{2.961472in}{2.939005in}}%
\pgfpathlineto{\pgfqpoint{2.961472in}{2.939005in}}%
\pgfpathlineto{\pgfqpoint{2.961472in}{2.941954in}}%
\pgfpathlineto{\pgfqpoint{2.966014in}{2.941954in}}%
\pgfpathlineto{\pgfqpoint{2.966014in}{2.939005in}}%
\pgfpathmoveto{\pgfqpoint{2.947849in}{2.941954in}}%
\pgfpathlineto{\pgfqpoint{2.947849in}{2.941954in}}%
\pgfpathlineto{\pgfqpoint{2.947849in}{2.944903in}}%
\pgfpathlineto{\pgfqpoint{2.952390in}{2.944903in}}%
\pgfpathlineto{\pgfqpoint{2.952390in}{2.941954in}}%
\pgfpathmoveto{\pgfqpoint{2.952390in}{2.941954in}}%
\pgfpathlineto{\pgfqpoint{2.952390in}{2.941954in}}%
\pgfpathlineto{\pgfqpoint{2.952390in}{2.944903in}}%
\pgfpathlineto{\pgfqpoint{2.956931in}{2.944903in}}%
\pgfpathlineto{\pgfqpoint{2.956931in}{2.941954in}}%
\pgfpathmoveto{\pgfqpoint{2.966014in}{2.936055in}}%
\pgfpathlineto{\pgfqpoint{2.966014in}{2.936055in}}%
\pgfpathlineto{\pgfqpoint{2.966014in}{2.939005in}}%
\pgfpathlineto{\pgfqpoint{2.970555in}{2.939005in}}%
\pgfpathlineto{\pgfqpoint{2.970555in}{2.936055in}}%
\pgfpathmoveto{\pgfqpoint{2.966014in}{2.939005in}}%
\pgfpathlineto{\pgfqpoint{2.966014in}{2.939005in}}%
\pgfpathlineto{\pgfqpoint{2.966014in}{2.941954in}}%
\pgfpathlineto{\pgfqpoint{2.970555in}{2.941954in}}%
\pgfpathlineto{\pgfqpoint{2.970555in}{2.939005in}}%
\pgfpathmoveto{\pgfqpoint{2.970555in}{2.936055in}}%
\pgfpathlineto{\pgfqpoint{2.970555in}{2.936055in}}%
\pgfpathlineto{\pgfqpoint{2.970555in}{2.939005in}}%
\pgfpathlineto{\pgfqpoint{2.975096in}{2.939005in}}%
\pgfpathlineto{\pgfqpoint{2.975096in}{2.936055in}}%
\pgfpathmoveto{\pgfqpoint{2.979637in}{2.933106in}}%
\pgfpathlineto{\pgfqpoint{2.979637in}{2.933106in}}%
\pgfpathlineto{\pgfqpoint{2.979637in}{2.936055in}}%
\pgfpathlineto{\pgfqpoint{2.984179in}{2.936055in}}%
\pgfpathlineto{\pgfqpoint{2.984179in}{2.933106in}}%
\pgfpathmoveto{\pgfqpoint{2.975096in}{2.936055in}}%
\pgfpathlineto{\pgfqpoint{2.975096in}{2.936055in}}%
\pgfpathlineto{\pgfqpoint{2.975096in}{2.939005in}}%
\pgfpathlineto{\pgfqpoint{2.979637in}{2.939005in}}%
\pgfpathlineto{\pgfqpoint{2.979637in}{2.936055in}}%
\pgfpathmoveto{\pgfqpoint{2.979637in}{2.936055in}}%
\pgfpathlineto{\pgfqpoint{2.979637in}{2.936055in}}%
\pgfpathlineto{\pgfqpoint{2.979637in}{2.939005in}}%
\pgfpathlineto{\pgfqpoint{2.984179in}{2.939005in}}%
\pgfpathlineto{\pgfqpoint{2.984179in}{2.936055in}}%
\pgfpathmoveto{\pgfqpoint{2.984179in}{2.933106in}}%
\pgfpathlineto{\pgfqpoint{2.984179in}{2.933106in}}%
\pgfpathlineto{\pgfqpoint{2.984179in}{2.936055in}}%
\pgfpathlineto{\pgfqpoint{2.988720in}{2.936055in}}%
\pgfpathlineto{\pgfqpoint{2.988720in}{2.933106in}}%
\pgfpathmoveto{\pgfqpoint{2.988720in}{2.933106in}}%
\pgfpathlineto{\pgfqpoint{2.988720in}{2.933106in}}%
\pgfpathlineto{\pgfqpoint{2.988720in}{2.936055in}}%
\pgfpathlineto{\pgfqpoint{2.993261in}{2.936055in}}%
\pgfpathlineto{\pgfqpoint{2.993261in}{2.933106in}}%
\pgfpathmoveto{\pgfqpoint{2.993261in}{2.930157in}}%
\pgfpathlineto{\pgfqpoint{2.993261in}{2.930157in}}%
\pgfpathlineto{\pgfqpoint{2.993261in}{2.933106in}}%
\pgfpathlineto{\pgfqpoint{2.997802in}{2.933106in}}%
\pgfpathlineto{\pgfqpoint{2.997802in}{2.930157in}}%
\pgfpathmoveto{\pgfqpoint{2.993261in}{2.933106in}}%
\pgfpathlineto{\pgfqpoint{2.993261in}{2.933106in}}%
\pgfpathlineto{\pgfqpoint{2.993261in}{2.936055in}}%
\pgfpathlineto{\pgfqpoint{2.997802in}{2.936055in}}%
\pgfpathlineto{\pgfqpoint{2.997802in}{2.933106in}}%
\pgfpathmoveto{\pgfqpoint{2.997802in}{2.930157in}}%
\pgfpathlineto{\pgfqpoint{2.997802in}{2.930157in}}%
\pgfpathlineto{\pgfqpoint{2.997802in}{2.933106in}}%
\pgfpathlineto{\pgfqpoint{3.002344in}{2.933106in}}%
\pgfpathlineto{\pgfqpoint{3.002344in}{2.930157in}}%
\pgfpathmoveto{\pgfqpoint{3.006885in}{2.927207in}}%
\pgfpathlineto{\pgfqpoint{3.006885in}{2.927207in}}%
\pgfpathlineto{\pgfqpoint{3.006885in}{2.930157in}}%
\pgfpathlineto{\pgfqpoint{3.011426in}{2.930157in}}%
\pgfpathlineto{\pgfqpoint{3.011426in}{2.927207in}}%
\pgfpathmoveto{\pgfqpoint{3.011426in}{2.927207in}}%
\pgfpathlineto{\pgfqpoint{3.011426in}{2.927207in}}%
\pgfpathlineto{\pgfqpoint{3.011426in}{2.930157in}}%
\pgfpathlineto{\pgfqpoint{3.015967in}{2.930157in}}%
\pgfpathlineto{\pgfqpoint{3.015967in}{2.927207in}}%
\pgfpathmoveto{\pgfqpoint{3.015967in}{2.927207in}}%
\pgfpathlineto{\pgfqpoint{3.015967in}{2.927207in}}%
\pgfpathlineto{\pgfqpoint{3.015967in}{2.930157in}}%
\pgfpathlineto{\pgfqpoint{3.020509in}{2.930157in}}%
\pgfpathlineto{\pgfqpoint{3.020509in}{2.927207in}}%
\pgfpathmoveto{\pgfqpoint{3.020509in}{2.924258in}}%
\pgfpathlineto{\pgfqpoint{3.020509in}{2.924258in}}%
\pgfpathlineto{\pgfqpoint{3.020509in}{2.927207in}}%
\pgfpathlineto{\pgfqpoint{3.025050in}{2.927207in}}%
\pgfpathlineto{\pgfqpoint{3.025050in}{2.924258in}}%
\pgfpathmoveto{\pgfqpoint{3.020509in}{2.927207in}}%
\pgfpathlineto{\pgfqpoint{3.020509in}{2.927207in}}%
\pgfpathlineto{\pgfqpoint{3.020509in}{2.930157in}}%
\pgfpathlineto{\pgfqpoint{3.025050in}{2.930157in}}%
\pgfpathlineto{\pgfqpoint{3.025050in}{2.927207in}}%
\pgfpathmoveto{\pgfqpoint{3.025050in}{2.924258in}}%
\pgfpathlineto{\pgfqpoint{3.025050in}{2.924258in}}%
\pgfpathlineto{\pgfqpoint{3.025050in}{2.927207in}}%
\pgfpathlineto{\pgfqpoint{3.029591in}{2.927207in}}%
\pgfpathlineto{\pgfqpoint{3.029591in}{2.924258in}}%
\pgfpathmoveto{\pgfqpoint{3.034132in}{2.921309in}}%
\pgfpathlineto{\pgfqpoint{3.034132in}{2.921309in}}%
\pgfpathlineto{\pgfqpoint{3.034132in}{2.924258in}}%
\pgfpathlineto{\pgfqpoint{3.038674in}{2.924258in}}%
\pgfpathlineto{\pgfqpoint{3.038674in}{2.921309in}}%
\pgfpathmoveto{\pgfqpoint{3.029591in}{2.924258in}}%
\pgfpathlineto{\pgfqpoint{3.029591in}{2.924258in}}%
\pgfpathlineto{\pgfqpoint{3.029591in}{2.927207in}}%
\pgfpathlineto{\pgfqpoint{3.034132in}{2.927207in}}%
\pgfpathlineto{\pgfqpoint{3.034132in}{2.924258in}}%
\pgfpathmoveto{\pgfqpoint{3.034132in}{2.924258in}}%
\pgfpathlineto{\pgfqpoint{3.034132in}{2.924258in}}%
\pgfpathlineto{\pgfqpoint{3.034132in}{2.927207in}}%
\pgfpathlineto{\pgfqpoint{3.038674in}{2.927207in}}%
\pgfpathlineto{\pgfqpoint{3.038674in}{2.924258in}}%
\pgfpathmoveto{\pgfqpoint{3.002344in}{2.930157in}}%
\pgfpathlineto{\pgfqpoint{3.002344in}{2.930157in}}%
\pgfpathlineto{\pgfqpoint{3.002344in}{2.933106in}}%
\pgfpathlineto{\pgfqpoint{3.006885in}{2.933106in}}%
\pgfpathlineto{\pgfqpoint{3.006885in}{2.930157in}}%
\pgfpathmoveto{\pgfqpoint{3.006885in}{2.930157in}}%
\pgfpathlineto{\pgfqpoint{3.006885in}{2.930157in}}%
\pgfpathlineto{\pgfqpoint{3.006885in}{2.933106in}}%
\pgfpathlineto{\pgfqpoint{3.011426in}{2.933106in}}%
\pgfpathlineto{\pgfqpoint{3.011426in}{2.930157in}}%
\pgfpathmoveto{\pgfqpoint{3.038674in}{2.921309in}}%
\pgfpathlineto{\pgfqpoint{3.038674in}{2.921309in}}%
\pgfpathlineto{\pgfqpoint{3.038674in}{2.924258in}}%
\pgfpathlineto{\pgfqpoint{3.043215in}{2.924258in}}%
\pgfpathlineto{\pgfqpoint{3.043215in}{2.921309in}}%
\pgfpathmoveto{\pgfqpoint{3.043215in}{2.921309in}}%
\pgfpathlineto{\pgfqpoint{3.043215in}{2.921309in}}%
\pgfpathlineto{\pgfqpoint{3.043215in}{2.924258in}}%
\pgfpathlineto{\pgfqpoint{3.047756in}{2.924258in}}%
\pgfpathlineto{\pgfqpoint{3.047756in}{2.921309in}}%
\pgfpathmoveto{\pgfqpoint{3.047756in}{2.918360in}}%
\pgfpathlineto{\pgfqpoint{3.047756in}{2.918360in}}%
\pgfpathlineto{\pgfqpoint{3.047756in}{2.921309in}}%
\pgfpathlineto{\pgfqpoint{3.052297in}{2.921309in}}%
\pgfpathlineto{\pgfqpoint{3.052297in}{2.918360in}}%
\pgfpathmoveto{\pgfqpoint{3.047756in}{2.921309in}}%
\pgfpathlineto{\pgfqpoint{3.047756in}{2.921309in}}%
\pgfpathlineto{\pgfqpoint{3.047756in}{2.924258in}}%
\pgfpathlineto{\pgfqpoint{3.052297in}{2.924258in}}%
\pgfpathlineto{\pgfqpoint{3.052297in}{2.921309in}}%
\pgfpathmoveto{\pgfqpoint{3.052297in}{2.918360in}}%
\pgfpathlineto{\pgfqpoint{3.052297in}{2.918360in}}%
\pgfpathlineto{\pgfqpoint{3.052297in}{2.921309in}}%
\pgfpathlineto{\pgfqpoint{3.056839in}{2.921309in}}%
\pgfpathlineto{\pgfqpoint{3.056839in}{2.918360in}}%
\pgfpathmoveto{\pgfqpoint{3.061380in}{2.915410in}}%
\pgfpathlineto{\pgfqpoint{3.061380in}{2.915410in}}%
\pgfpathlineto{\pgfqpoint{3.061380in}{2.918360in}}%
\pgfpathlineto{\pgfqpoint{3.065921in}{2.918360in}}%
\pgfpathlineto{\pgfqpoint{3.065921in}{2.915410in}}%
\pgfpathmoveto{\pgfqpoint{3.065921in}{2.915410in}}%
\pgfpathlineto{\pgfqpoint{3.065921in}{2.915410in}}%
\pgfpathlineto{\pgfqpoint{3.065921in}{2.918360in}}%
\pgfpathlineto{\pgfqpoint{3.070463in}{2.918360in}}%
\pgfpathlineto{\pgfqpoint{3.070463in}{2.915410in}}%
\pgfpathmoveto{\pgfqpoint{3.070463in}{2.915410in}}%
\pgfpathlineto{\pgfqpoint{3.070463in}{2.915410in}}%
\pgfpathlineto{\pgfqpoint{3.070463in}{2.918360in}}%
\pgfpathlineto{\pgfqpoint{3.075004in}{2.918360in}}%
\pgfpathlineto{\pgfqpoint{3.075004in}{2.915410in}}%
\pgfpathmoveto{\pgfqpoint{3.056839in}{2.918360in}}%
\pgfpathlineto{\pgfqpoint{3.056839in}{2.918360in}}%
\pgfpathlineto{\pgfqpoint{3.056839in}{2.921309in}}%
\pgfpathlineto{\pgfqpoint{3.061380in}{2.921309in}}%
\pgfpathlineto{\pgfqpoint{3.061380in}{2.918360in}}%
\pgfpathmoveto{\pgfqpoint{3.061380in}{2.918360in}}%
\pgfpathlineto{\pgfqpoint{3.061380in}{2.918360in}}%
\pgfpathlineto{\pgfqpoint{3.061380in}{2.921309in}}%
\pgfpathlineto{\pgfqpoint{3.065921in}{2.921309in}}%
\pgfpathlineto{\pgfqpoint{3.065921in}{2.918360in}}%
\pgfpathmoveto{\pgfqpoint{3.115871in}{2.903613in}}%
\pgfpathlineto{\pgfqpoint{3.115871in}{2.903613in}}%
\pgfpathlineto{\pgfqpoint{3.115871in}{2.906563in}}%
\pgfpathlineto{\pgfqpoint{3.120412in}{2.906563in}}%
\pgfpathlineto{\pgfqpoint{3.120412in}{2.903613in}}%
\pgfpathmoveto{\pgfqpoint{3.120412in}{2.903613in}}%
\pgfpathlineto{\pgfqpoint{3.120412in}{2.903613in}}%
\pgfpathlineto{\pgfqpoint{3.120412in}{2.906563in}}%
\pgfpathlineto{\pgfqpoint{3.124953in}{2.906563in}}%
\pgfpathlineto{\pgfqpoint{3.124953in}{2.903613in}}%
\pgfpathmoveto{\pgfqpoint{3.124953in}{2.903613in}}%
\pgfpathlineto{\pgfqpoint{3.124953in}{2.903613in}}%
\pgfpathlineto{\pgfqpoint{3.124953in}{2.906563in}}%
\pgfpathlineto{\pgfqpoint{3.129494in}{2.906563in}}%
\pgfpathlineto{\pgfqpoint{3.129494in}{2.903613in}}%
\pgfpathmoveto{\pgfqpoint{3.129494in}{2.900664in}}%
\pgfpathlineto{\pgfqpoint{3.129494in}{2.900664in}}%
\pgfpathlineto{\pgfqpoint{3.129494in}{2.903613in}}%
\pgfpathlineto{\pgfqpoint{3.134035in}{2.903613in}}%
\pgfpathlineto{\pgfqpoint{3.134035in}{2.900664in}}%
\pgfpathmoveto{\pgfqpoint{3.129494in}{2.903613in}}%
\pgfpathlineto{\pgfqpoint{3.129494in}{2.903613in}}%
\pgfpathlineto{\pgfqpoint{3.129494in}{2.906563in}}%
\pgfpathlineto{\pgfqpoint{3.134035in}{2.906563in}}%
\pgfpathlineto{\pgfqpoint{3.134035in}{2.903613in}}%
\pgfpathmoveto{\pgfqpoint{3.134035in}{2.900664in}}%
\pgfpathlineto{\pgfqpoint{3.134035in}{2.900664in}}%
\pgfpathlineto{\pgfqpoint{3.134035in}{2.903613in}}%
\pgfpathlineto{\pgfqpoint{3.138576in}{2.903613in}}%
\pgfpathlineto{\pgfqpoint{3.138576in}{2.900664in}}%
\pgfpathmoveto{\pgfqpoint{3.143117in}{2.897715in}}%
\pgfpathlineto{\pgfqpoint{3.143117in}{2.897715in}}%
\pgfpathlineto{\pgfqpoint{3.143117in}{2.900664in}}%
\pgfpathlineto{\pgfqpoint{3.147657in}{2.900664in}}%
\pgfpathlineto{\pgfqpoint{3.147657in}{2.897715in}}%
\pgfpathmoveto{\pgfqpoint{3.138576in}{2.900664in}}%
\pgfpathlineto{\pgfqpoint{3.138576in}{2.900664in}}%
\pgfpathlineto{\pgfqpoint{3.138576in}{2.903613in}}%
\pgfpathlineto{\pgfqpoint{3.143117in}{2.903613in}}%
\pgfpathlineto{\pgfqpoint{3.143117in}{2.900664in}}%
\pgfpathmoveto{\pgfqpoint{3.143117in}{2.900664in}}%
\pgfpathlineto{\pgfqpoint{3.143117in}{2.900664in}}%
\pgfpathlineto{\pgfqpoint{3.143117in}{2.903613in}}%
\pgfpathlineto{\pgfqpoint{3.147657in}{2.903613in}}%
\pgfpathlineto{\pgfqpoint{3.147657in}{2.900664in}}%
\pgfpathmoveto{\pgfqpoint{3.075004in}{2.912461in}}%
\pgfpathlineto{\pgfqpoint{3.075004in}{2.912461in}}%
\pgfpathlineto{\pgfqpoint{3.075004in}{2.915410in}}%
\pgfpathlineto{\pgfqpoint{3.079545in}{2.915410in}}%
\pgfpathlineto{\pgfqpoint{3.079545in}{2.912461in}}%
\pgfpathmoveto{\pgfqpoint{3.075004in}{2.915410in}}%
\pgfpathlineto{\pgfqpoint{3.075004in}{2.915410in}}%
\pgfpathlineto{\pgfqpoint{3.075004in}{2.918360in}}%
\pgfpathlineto{\pgfqpoint{3.079545in}{2.918360in}}%
\pgfpathlineto{\pgfqpoint{3.079545in}{2.915410in}}%
\pgfpathmoveto{\pgfqpoint{3.079545in}{2.912461in}}%
\pgfpathlineto{\pgfqpoint{3.079545in}{2.912461in}}%
\pgfpathlineto{\pgfqpoint{3.079545in}{2.915410in}}%
\pgfpathlineto{\pgfqpoint{3.084085in}{2.915410in}}%
\pgfpathlineto{\pgfqpoint{3.084085in}{2.912461in}}%
\pgfpathmoveto{\pgfqpoint{3.088626in}{2.909512in}}%
\pgfpathlineto{\pgfqpoint{3.088626in}{2.909512in}}%
\pgfpathlineto{\pgfqpoint{3.088626in}{2.912461in}}%
\pgfpathlineto{\pgfqpoint{3.093167in}{2.912461in}}%
\pgfpathlineto{\pgfqpoint{3.093167in}{2.909512in}}%
\pgfpathmoveto{\pgfqpoint{3.084085in}{2.912461in}}%
\pgfpathlineto{\pgfqpoint{3.084085in}{2.912461in}}%
\pgfpathlineto{\pgfqpoint{3.084085in}{2.915410in}}%
\pgfpathlineto{\pgfqpoint{3.088626in}{2.915410in}}%
\pgfpathlineto{\pgfqpoint{3.088626in}{2.912461in}}%
\pgfpathmoveto{\pgfqpoint{3.088626in}{2.912461in}}%
\pgfpathlineto{\pgfqpoint{3.088626in}{2.912461in}}%
\pgfpathlineto{\pgfqpoint{3.088626in}{2.915410in}}%
\pgfpathlineto{\pgfqpoint{3.093167in}{2.915410in}}%
\pgfpathlineto{\pgfqpoint{3.093167in}{2.912461in}}%
\pgfpathmoveto{\pgfqpoint{3.093167in}{2.909512in}}%
\pgfpathlineto{\pgfqpoint{3.093167in}{2.909512in}}%
\pgfpathlineto{\pgfqpoint{3.093167in}{2.912461in}}%
\pgfpathlineto{\pgfqpoint{3.097708in}{2.912461in}}%
\pgfpathlineto{\pgfqpoint{3.097708in}{2.909512in}}%
\pgfpathmoveto{\pgfqpoint{3.097708in}{2.909512in}}%
\pgfpathlineto{\pgfqpoint{3.097708in}{2.909512in}}%
\pgfpathlineto{\pgfqpoint{3.097708in}{2.912461in}}%
\pgfpathlineto{\pgfqpoint{3.102249in}{2.912461in}}%
\pgfpathlineto{\pgfqpoint{3.102249in}{2.909512in}}%
\pgfpathmoveto{\pgfqpoint{3.102249in}{2.906563in}}%
\pgfpathlineto{\pgfqpoint{3.102249in}{2.906563in}}%
\pgfpathlineto{\pgfqpoint{3.102249in}{2.909512in}}%
\pgfpathlineto{\pgfqpoint{3.106790in}{2.909512in}}%
\pgfpathlineto{\pgfqpoint{3.106790in}{2.906563in}}%
\pgfpathmoveto{\pgfqpoint{3.102249in}{2.909512in}}%
\pgfpathlineto{\pgfqpoint{3.102249in}{2.909512in}}%
\pgfpathlineto{\pgfqpoint{3.102249in}{2.912461in}}%
\pgfpathlineto{\pgfqpoint{3.106790in}{2.912461in}}%
\pgfpathlineto{\pgfqpoint{3.106790in}{2.909512in}}%
\pgfpathmoveto{\pgfqpoint{3.106790in}{2.906563in}}%
\pgfpathlineto{\pgfqpoint{3.106790in}{2.906563in}}%
\pgfpathlineto{\pgfqpoint{3.106790in}{2.909512in}}%
\pgfpathlineto{\pgfqpoint{3.111331in}{2.909512in}}%
\pgfpathlineto{\pgfqpoint{3.111331in}{2.906563in}}%
\pgfpathmoveto{\pgfqpoint{3.111331in}{2.906563in}}%
\pgfpathlineto{\pgfqpoint{3.111331in}{2.906563in}}%
\pgfpathlineto{\pgfqpoint{3.111331in}{2.909512in}}%
\pgfpathlineto{\pgfqpoint{3.115871in}{2.909512in}}%
\pgfpathlineto{\pgfqpoint{3.115871in}{2.906563in}}%
\pgfpathmoveto{\pgfqpoint{3.115871in}{2.906563in}}%
\pgfpathlineto{\pgfqpoint{3.115871in}{2.906563in}}%
\pgfpathlineto{\pgfqpoint{3.115871in}{2.909512in}}%
\pgfpathlineto{\pgfqpoint{3.120412in}{2.909512in}}%
\pgfpathlineto{\pgfqpoint{3.120412in}{2.906563in}}%
\pgfpathmoveto{\pgfqpoint{3.147657in}{2.897715in}}%
\pgfpathlineto{\pgfqpoint{3.147657in}{2.897715in}}%
\pgfpathlineto{\pgfqpoint{3.147657in}{2.900664in}}%
\pgfpathlineto{\pgfqpoint{3.152198in}{2.900664in}}%
\pgfpathlineto{\pgfqpoint{3.152198in}{2.897715in}}%
\pgfpathmoveto{\pgfqpoint{3.152198in}{2.897715in}}%
\pgfpathlineto{\pgfqpoint{3.152198in}{2.897715in}}%
\pgfpathlineto{\pgfqpoint{3.152198in}{2.900664in}}%
\pgfpathlineto{\pgfqpoint{3.156739in}{2.900664in}}%
\pgfpathlineto{\pgfqpoint{3.156739in}{2.897715in}}%
\pgfpathmoveto{\pgfqpoint{3.156739in}{2.894766in}}%
\pgfpathlineto{\pgfqpoint{3.156739in}{2.894766in}}%
\pgfpathlineto{\pgfqpoint{3.156739in}{2.897715in}}%
\pgfpathlineto{\pgfqpoint{3.161280in}{2.897715in}}%
\pgfpathlineto{\pgfqpoint{3.161280in}{2.894766in}}%
\pgfpathmoveto{\pgfqpoint{3.156739in}{2.897715in}}%
\pgfpathlineto{\pgfqpoint{3.156739in}{2.897715in}}%
\pgfpathlineto{\pgfqpoint{3.156739in}{2.900664in}}%
\pgfpathlineto{\pgfqpoint{3.161280in}{2.900664in}}%
\pgfpathlineto{\pgfqpoint{3.161280in}{2.897715in}}%
\pgfpathmoveto{\pgfqpoint{3.161280in}{2.894766in}}%
\pgfpathlineto{\pgfqpoint{3.161280in}{2.894766in}}%
\pgfpathlineto{\pgfqpoint{3.161280in}{2.897715in}}%
\pgfpathlineto{\pgfqpoint{3.165821in}{2.897715in}}%
\pgfpathlineto{\pgfqpoint{3.165821in}{2.894766in}}%
\pgfpathmoveto{\pgfqpoint{3.170362in}{2.891816in}}%
\pgfpathlineto{\pgfqpoint{3.170362in}{2.891816in}}%
\pgfpathlineto{\pgfqpoint{3.170362in}{2.894766in}}%
\pgfpathlineto{\pgfqpoint{3.174903in}{2.894766in}}%
\pgfpathlineto{\pgfqpoint{3.174903in}{2.891816in}}%
\pgfpathmoveto{\pgfqpoint{3.174903in}{2.891816in}}%
\pgfpathlineto{\pgfqpoint{3.174903in}{2.891816in}}%
\pgfpathlineto{\pgfqpoint{3.174903in}{2.894766in}}%
\pgfpathlineto{\pgfqpoint{3.179443in}{2.894766in}}%
\pgfpathlineto{\pgfqpoint{3.179443in}{2.891816in}}%
\pgfpathmoveto{\pgfqpoint{3.179443in}{2.891816in}}%
\pgfpathlineto{\pgfqpoint{3.179443in}{2.891816in}}%
\pgfpathlineto{\pgfqpoint{3.179443in}{2.894766in}}%
\pgfpathlineto{\pgfqpoint{3.183984in}{2.894766in}}%
\pgfpathlineto{\pgfqpoint{3.183984in}{2.891816in}}%
\pgfpathmoveto{\pgfqpoint{3.165821in}{2.894766in}}%
\pgfpathlineto{\pgfqpoint{3.165821in}{2.894766in}}%
\pgfpathlineto{\pgfqpoint{3.165821in}{2.897715in}}%
\pgfpathlineto{\pgfqpoint{3.170362in}{2.897715in}}%
\pgfpathlineto{\pgfqpoint{3.170362in}{2.894766in}}%
\pgfpathmoveto{\pgfqpoint{3.170362in}{2.894766in}}%
\pgfpathlineto{\pgfqpoint{3.170362in}{2.894766in}}%
\pgfpathlineto{\pgfqpoint{3.170362in}{2.897715in}}%
\pgfpathlineto{\pgfqpoint{3.174903in}{2.897715in}}%
\pgfpathlineto{\pgfqpoint{3.174903in}{2.894766in}}%
\pgfpathmoveto{\pgfqpoint{3.183984in}{2.888867in}}%
\pgfpathlineto{\pgfqpoint{3.183984in}{2.888867in}}%
\pgfpathlineto{\pgfqpoint{3.183984in}{2.891816in}}%
\pgfpathlineto{\pgfqpoint{3.188525in}{2.891816in}}%
\pgfpathlineto{\pgfqpoint{3.188525in}{2.888867in}}%
\pgfpathmoveto{\pgfqpoint{3.183984in}{2.891816in}}%
\pgfpathlineto{\pgfqpoint{3.183984in}{2.891816in}}%
\pgfpathlineto{\pgfqpoint{3.183984in}{2.894766in}}%
\pgfpathlineto{\pgfqpoint{3.188525in}{2.894766in}}%
\pgfpathlineto{\pgfqpoint{3.188525in}{2.891816in}}%
\pgfpathmoveto{\pgfqpoint{3.188525in}{2.888867in}}%
\pgfpathlineto{\pgfqpoint{3.188525in}{2.888867in}}%
\pgfpathlineto{\pgfqpoint{3.188525in}{2.891816in}}%
\pgfpathlineto{\pgfqpoint{3.193066in}{2.891816in}}%
\pgfpathlineto{\pgfqpoint{3.193066in}{2.888867in}}%
\pgfpathmoveto{\pgfqpoint{3.197607in}{2.885918in}}%
\pgfpathlineto{\pgfqpoint{3.197607in}{2.885918in}}%
\pgfpathlineto{\pgfqpoint{3.197607in}{2.888867in}}%
\pgfpathlineto{\pgfqpoint{3.202148in}{2.888867in}}%
\pgfpathlineto{\pgfqpoint{3.202148in}{2.885918in}}%
\pgfpathmoveto{\pgfqpoint{3.193066in}{2.888867in}}%
\pgfpathlineto{\pgfqpoint{3.193066in}{2.888867in}}%
\pgfpathlineto{\pgfqpoint{3.193066in}{2.891816in}}%
\pgfpathlineto{\pgfqpoint{3.197607in}{2.891816in}}%
\pgfpathlineto{\pgfqpoint{3.197607in}{2.888867in}}%
\pgfpathmoveto{\pgfqpoint{3.197607in}{2.888867in}}%
\pgfpathlineto{\pgfqpoint{3.197607in}{2.888867in}}%
\pgfpathlineto{\pgfqpoint{3.197607in}{2.891816in}}%
\pgfpathlineto{\pgfqpoint{3.202148in}{2.891816in}}%
\pgfpathlineto{\pgfqpoint{3.202148in}{2.888867in}}%
\pgfpathmoveto{\pgfqpoint{3.202148in}{2.885918in}}%
\pgfpathlineto{\pgfqpoint{3.202148in}{2.885918in}}%
\pgfpathlineto{\pgfqpoint{3.202148in}{2.888867in}}%
\pgfpathlineto{\pgfqpoint{3.206688in}{2.888867in}}%
\pgfpathlineto{\pgfqpoint{3.206688in}{2.885918in}}%
\pgfpathmoveto{\pgfqpoint{3.206688in}{2.885918in}}%
\pgfpathlineto{\pgfqpoint{3.206688in}{2.885918in}}%
\pgfpathlineto{\pgfqpoint{3.206688in}{2.888867in}}%
\pgfpathlineto{\pgfqpoint{3.211229in}{2.888867in}}%
\pgfpathlineto{\pgfqpoint{3.211229in}{2.885918in}}%
\pgfpathmoveto{\pgfqpoint{3.211229in}{2.882969in}}%
\pgfpathlineto{\pgfqpoint{3.211229in}{2.882969in}}%
\pgfpathlineto{\pgfqpoint{3.211229in}{2.885918in}}%
\pgfpathlineto{\pgfqpoint{3.215770in}{2.885918in}}%
\pgfpathlineto{\pgfqpoint{3.215770in}{2.882969in}}%
\pgfpathmoveto{\pgfqpoint{3.211229in}{2.885918in}}%
\pgfpathlineto{\pgfqpoint{3.211229in}{2.885918in}}%
\pgfpathlineto{\pgfqpoint{3.211229in}{2.888867in}}%
\pgfpathlineto{\pgfqpoint{3.215770in}{2.888867in}}%
\pgfpathlineto{\pgfqpoint{3.215770in}{2.885918in}}%
\pgfpathmoveto{\pgfqpoint{3.215770in}{2.882969in}}%
\pgfpathlineto{\pgfqpoint{3.215770in}{2.882969in}}%
\pgfpathlineto{\pgfqpoint{3.215770in}{2.885918in}}%
\pgfpathlineto{\pgfqpoint{3.220311in}{2.885918in}}%
\pgfpathlineto{\pgfqpoint{3.220311in}{2.882969in}}%
\pgfpathmoveto{\pgfqpoint{3.333836in}{2.856425in}}%
\pgfpathlineto{\pgfqpoint{3.333836in}{2.856425in}}%
\pgfpathlineto{\pgfqpoint{3.333836in}{2.859374in}}%
\pgfpathlineto{\pgfqpoint{3.338377in}{2.859374in}}%
\pgfpathlineto{\pgfqpoint{3.338377in}{2.856425in}}%
\pgfpathmoveto{\pgfqpoint{3.338377in}{2.856425in}}%
\pgfpathlineto{\pgfqpoint{3.338377in}{2.856425in}}%
\pgfpathlineto{\pgfqpoint{3.338377in}{2.859374in}}%
\pgfpathlineto{\pgfqpoint{3.342918in}{2.859374in}}%
\pgfpathlineto{\pgfqpoint{3.342918in}{2.856425in}}%
\pgfpathmoveto{\pgfqpoint{3.342918in}{2.856425in}}%
\pgfpathlineto{\pgfqpoint{3.342918in}{2.856425in}}%
\pgfpathlineto{\pgfqpoint{3.342918in}{2.859374in}}%
\pgfpathlineto{\pgfqpoint{3.347459in}{2.859374in}}%
\pgfpathlineto{\pgfqpoint{3.347459in}{2.856425in}}%
\pgfpathmoveto{\pgfqpoint{3.347459in}{2.853476in}}%
\pgfpathlineto{\pgfqpoint{3.347459in}{2.853476in}}%
\pgfpathlineto{\pgfqpoint{3.347459in}{2.856425in}}%
\pgfpathlineto{\pgfqpoint{3.352000in}{2.856425in}}%
\pgfpathlineto{\pgfqpoint{3.352000in}{2.853476in}}%
\pgfpathmoveto{\pgfqpoint{3.347459in}{2.856425in}}%
\pgfpathlineto{\pgfqpoint{3.347459in}{2.856425in}}%
\pgfpathlineto{\pgfqpoint{3.347459in}{2.859374in}}%
\pgfpathlineto{\pgfqpoint{3.352000in}{2.859374in}}%
\pgfpathlineto{\pgfqpoint{3.352000in}{2.856425in}}%
\pgfpathmoveto{\pgfqpoint{3.352000in}{2.853476in}}%
\pgfpathlineto{\pgfqpoint{3.352000in}{2.853476in}}%
\pgfpathlineto{\pgfqpoint{3.352000in}{2.856425in}}%
\pgfpathlineto{\pgfqpoint{3.356541in}{2.856425in}}%
\pgfpathlineto{\pgfqpoint{3.356541in}{2.853476in}}%
\pgfpathmoveto{\pgfqpoint{3.361082in}{2.850527in}}%
\pgfpathlineto{\pgfqpoint{3.361082in}{2.850527in}}%
\pgfpathlineto{\pgfqpoint{3.361082in}{2.853476in}}%
\pgfpathlineto{\pgfqpoint{3.365623in}{2.853476in}}%
\pgfpathlineto{\pgfqpoint{3.365623in}{2.850527in}}%
\pgfpathmoveto{\pgfqpoint{3.356541in}{2.853476in}}%
\pgfpathlineto{\pgfqpoint{3.356541in}{2.853476in}}%
\pgfpathlineto{\pgfqpoint{3.356541in}{2.856425in}}%
\pgfpathlineto{\pgfqpoint{3.361082in}{2.856425in}}%
\pgfpathlineto{\pgfqpoint{3.361082in}{2.853476in}}%
\pgfpathmoveto{\pgfqpoint{3.361082in}{2.853476in}}%
\pgfpathlineto{\pgfqpoint{3.361082in}{2.853476in}}%
\pgfpathlineto{\pgfqpoint{3.361082in}{2.856425in}}%
\pgfpathlineto{\pgfqpoint{3.365623in}{2.856425in}}%
\pgfpathlineto{\pgfqpoint{3.365623in}{2.853476in}}%
\pgfpathmoveto{\pgfqpoint{3.224852in}{2.880019in}}%
\pgfpathlineto{\pgfqpoint{3.224852in}{2.880019in}}%
\pgfpathlineto{\pgfqpoint{3.224852in}{2.882969in}}%
\pgfpathlineto{\pgfqpoint{3.229393in}{2.882969in}}%
\pgfpathlineto{\pgfqpoint{3.229393in}{2.880019in}}%
\pgfpathmoveto{\pgfqpoint{3.229393in}{2.880019in}}%
\pgfpathlineto{\pgfqpoint{3.229393in}{2.880019in}}%
\pgfpathlineto{\pgfqpoint{3.229393in}{2.882969in}}%
\pgfpathlineto{\pgfqpoint{3.233934in}{2.882969in}}%
\pgfpathlineto{\pgfqpoint{3.233934in}{2.880019in}}%
\pgfpathmoveto{\pgfqpoint{3.233934in}{2.880019in}}%
\pgfpathlineto{\pgfqpoint{3.233934in}{2.880019in}}%
\pgfpathlineto{\pgfqpoint{3.233934in}{2.882969in}}%
\pgfpathlineto{\pgfqpoint{3.238475in}{2.882969in}}%
\pgfpathlineto{\pgfqpoint{3.238475in}{2.880019in}}%
\pgfpathmoveto{\pgfqpoint{3.238475in}{2.877070in}}%
\pgfpathlineto{\pgfqpoint{3.238475in}{2.877070in}}%
\pgfpathlineto{\pgfqpoint{3.238475in}{2.880019in}}%
\pgfpathlineto{\pgfqpoint{3.243016in}{2.880019in}}%
\pgfpathlineto{\pgfqpoint{3.243016in}{2.877070in}}%
\pgfpathmoveto{\pgfqpoint{3.238475in}{2.880019in}}%
\pgfpathlineto{\pgfqpoint{3.238475in}{2.880019in}}%
\pgfpathlineto{\pgfqpoint{3.238475in}{2.882969in}}%
\pgfpathlineto{\pgfqpoint{3.243016in}{2.882969in}}%
\pgfpathlineto{\pgfqpoint{3.243016in}{2.880019in}}%
\pgfpathmoveto{\pgfqpoint{3.243016in}{2.877070in}}%
\pgfpathlineto{\pgfqpoint{3.243016in}{2.877070in}}%
\pgfpathlineto{\pgfqpoint{3.243016in}{2.880019in}}%
\pgfpathlineto{\pgfqpoint{3.247557in}{2.880019in}}%
\pgfpathlineto{\pgfqpoint{3.247557in}{2.877070in}}%
\pgfpathmoveto{\pgfqpoint{3.252098in}{2.874121in}}%
\pgfpathlineto{\pgfqpoint{3.252098in}{2.874121in}}%
\pgfpathlineto{\pgfqpoint{3.252098in}{2.877070in}}%
\pgfpathlineto{\pgfqpoint{3.256639in}{2.877070in}}%
\pgfpathlineto{\pgfqpoint{3.256639in}{2.874121in}}%
\pgfpathmoveto{\pgfqpoint{3.247557in}{2.877070in}}%
\pgfpathlineto{\pgfqpoint{3.247557in}{2.877070in}}%
\pgfpathlineto{\pgfqpoint{3.247557in}{2.880019in}}%
\pgfpathlineto{\pgfqpoint{3.252098in}{2.880019in}}%
\pgfpathlineto{\pgfqpoint{3.252098in}{2.877070in}}%
\pgfpathmoveto{\pgfqpoint{3.252098in}{2.877070in}}%
\pgfpathlineto{\pgfqpoint{3.252098in}{2.877070in}}%
\pgfpathlineto{\pgfqpoint{3.252098in}{2.880019in}}%
\pgfpathlineto{\pgfqpoint{3.256639in}{2.880019in}}%
\pgfpathlineto{\pgfqpoint{3.256639in}{2.877070in}}%
\pgfpathmoveto{\pgfqpoint{3.220311in}{2.882969in}}%
\pgfpathlineto{\pgfqpoint{3.220311in}{2.882969in}}%
\pgfpathlineto{\pgfqpoint{3.220311in}{2.885918in}}%
\pgfpathlineto{\pgfqpoint{3.224852in}{2.885918in}}%
\pgfpathlineto{\pgfqpoint{3.224852in}{2.882969in}}%
\pgfpathmoveto{\pgfqpoint{3.224852in}{2.882969in}}%
\pgfpathlineto{\pgfqpoint{3.224852in}{2.882969in}}%
\pgfpathlineto{\pgfqpoint{3.224852in}{2.885918in}}%
\pgfpathlineto{\pgfqpoint{3.229393in}{2.885918in}}%
\pgfpathlineto{\pgfqpoint{3.229393in}{2.882969in}}%
\pgfpathmoveto{\pgfqpoint{3.256639in}{2.874121in}}%
\pgfpathlineto{\pgfqpoint{3.256639in}{2.874121in}}%
\pgfpathlineto{\pgfqpoint{3.256639in}{2.877070in}}%
\pgfpathlineto{\pgfqpoint{3.261180in}{2.877070in}}%
\pgfpathlineto{\pgfqpoint{3.261180in}{2.874121in}}%
\pgfpathmoveto{\pgfqpoint{3.261180in}{2.874121in}}%
\pgfpathlineto{\pgfqpoint{3.261180in}{2.874121in}}%
\pgfpathlineto{\pgfqpoint{3.261180in}{2.877070in}}%
\pgfpathlineto{\pgfqpoint{3.265721in}{2.877070in}}%
\pgfpathlineto{\pgfqpoint{3.265721in}{2.874121in}}%
\pgfpathmoveto{\pgfqpoint{3.265721in}{2.871171in}}%
\pgfpathlineto{\pgfqpoint{3.265721in}{2.871171in}}%
\pgfpathlineto{\pgfqpoint{3.265721in}{2.874121in}}%
\pgfpathlineto{\pgfqpoint{3.270262in}{2.874121in}}%
\pgfpathlineto{\pgfqpoint{3.270262in}{2.871171in}}%
\pgfpathmoveto{\pgfqpoint{3.265721in}{2.874121in}}%
\pgfpathlineto{\pgfqpoint{3.265721in}{2.874121in}}%
\pgfpathlineto{\pgfqpoint{3.265721in}{2.877070in}}%
\pgfpathlineto{\pgfqpoint{3.270262in}{2.877070in}}%
\pgfpathlineto{\pgfqpoint{3.270262in}{2.874121in}}%
\pgfpathmoveto{\pgfqpoint{3.270262in}{2.871171in}}%
\pgfpathlineto{\pgfqpoint{3.270262in}{2.871171in}}%
\pgfpathlineto{\pgfqpoint{3.270262in}{2.874121in}}%
\pgfpathlineto{\pgfqpoint{3.274803in}{2.874121in}}%
\pgfpathlineto{\pgfqpoint{3.274803in}{2.871171in}}%
\pgfpathmoveto{\pgfqpoint{3.279344in}{2.868222in}}%
\pgfpathlineto{\pgfqpoint{3.279344in}{2.868222in}}%
\pgfpathlineto{\pgfqpoint{3.279344in}{2.871171in}}%
\pgfpathlineto{\pgfqpoint{3.283885in}{2.871171in}}%
\pgfpathlineto{\pgfqpoint{3.283885in}{2.868222in}}%
\pgfpathmoveto{\pgfqpoint{3.283885in}{2.868222in}}%
\pgfpathlineto{\pgfqpoint{3.283885in}{2.868222in}}%
\pgfpathlineto{\pgfqpoint{3.283885in}{2.871171in}}%
\pgfpathlineto{\pgfqpoint{3.288426in}{2.871171in}}%
\pgfpathlineto{\pgfqpoint{3.288426in}{2.868222in}}%
\pgfpathmoveto{\pgfqpoint{3.288426in}{2.868222in}}%
\pgfpathlineto{\pgfqpoint{3.288426in}{2.868222in}}%
\pgfpathlineto{\pgfqpoint{3.288426in}{2.871171in}}%
\pgfpathlineto{\pgfqpoint{3.292967in}{2.871171in}}%
\pgfpathlineto{\pgfqpoint{3.292967in}{2.868222in}}%
\pgfpathmoveto{\pgfqpoint{3.274803in}{2.871171in}}%
\pgfpathlineto{\pgfqpoint{3.274803in}{2.871171in}}%
\pgfpathlineto{\pgfqpoint{3.274803in}{2.874121in}}%
\pgfpathlineto{\pgfqpoint{3.279344in}{2.874121in}}%
\pgfpathlineto{\pgfqpoint{3.279344in}{2.871171in}}%
\pgfpathmoveto{\pgfqpoint{3.279344in}{2.871171in}}%
\pgfpathlineto{\pgfqpoint{3.279344in}{2.871171in}}%
\pgfpathlineto{\pgfqpoint{3.279344in}{2.874121in}}%
\pgfpathlineto{\pgfqpoint{3.283885in}{2.874121in}}%
\pgfpathlineto{\pgfqpoint{3.283885in}{2.871171in}}%
\pgfpathmoveto{\pgfqpoint{3.292967in}{2.865273in}}%
\pgfpathlineto{\pgfqpoint{3.292967in}{2.865273in}}%
\pgfpathlineto{\pgfqpoint{3.292967in}{2.868222in}}%
\pgfpathlineto{\pgfqpoint{3.297508in}{2.868222in}}%
\pgfpathlineto{\pgfqpoint{3.297508in}{2.865273in}}%
\pgfpathmoveto{\pgfqpoint{3.292967in}{2.868222in}}%
\pgfpathlineto{\pgfqpoint{3.292967in}{2.868222in}}%
\pgfpathlineto{\pgfqpoint{3.292967in}{2.871171in}}%
\pgfpathlineto{\pgfqpoint{3.297508in}{2.871171in}}%
\pgfpathlineto{\pgfqpoint{3.297508in}{2.868222in}}%
\pgfpathmoveto{\pgfqpoint{3.297508in}{2.865273in}}%
\pgfpathlineto{\pgfqpoint{3.297508in}{2.865273in}}%
\pgfpathlineto{\pgfqpoint{3.297508in}{2.868222in}}%
\pgfpathlineto{\pgfqpoint{3.302049in}{2.868222in}}%
\pgfpathlineto{\pgfqpoint{3.302049in}{2.865273in}}%
\pgfpathmoveto{\pgfqpoint{3.306590in}{2.862324in}}%
\pgfpathlineto{\pgfqpoint{3.306590in}{2.862324in}}%
\pgfpathlineto{\pgfqpoint{3.306590in}{2.865273in}}%
\pgfpathlineto{\pgfqpoint{3.311131in}{2.865273in}}%
\pgfpathlineto{\pgfqpoint{3.311131in}{2.862324in}}%
\pgfpathmoveto{\pgfqpoint{3.302049in}{2.865273in}}%
\pgfpathlineto{\pgfqpoint{3.302049in}{2.865273in}}%
\pgfpathlineto{\pgfqpoint{3.302049in}{2.868222in}}%
\pgfpathlineto{\pgfqpoint{3.306590in}{2.868222in}}%
\pgfpathlineto{\pgfqpoint{3.306590in}{2.865273in}}%
\pgfpathmoveto{\pgfqpoint{3.306590in}{2.865273in}}%
\pgfpathlineto{\pgfqpoint{3.306590in}{2.865273in}}%
\pgfpathlineto{\pgfqpoint{3.306590in}{2.868222in}}%
\pgfpathlineto{\pgfqpoint{3.311131in}{2.868222in}}%
\pgfpathlineto{\pgfqpoint{3.311131in}{2.865273in}}%
\pgfpathmoveto{\pgfqpoint{3.311131in}{2.862324in}}%
\pgfpathlineto{\pgfqpoint{3.311131in}{2.862324in}}%
\pgfpathlineto{\pgfqpoint{3.311131in}{2.865273in}}%
\pgfpathlineto{\pgfqpoint{3.315672in}{2.865273in}}%
\pgfpathlineto{\pgfqpoint{3.315672in}{2.862324in}}%
\pgfpathmoveto{\pgfqpoint{3.315672in}{2.862324in}}%
\pgfpathlineto{\pgfqpoint{3.315672in}{2.862324in}}%
\pgfpathlineto{\pgfqpoint{3.315672in}{2.865273in}}%
\pgfpathlineto{\pgfqpoint{3.320213in}{2.865273in}}%
\pgfpathlineto{\pgfqpoint{3.320213in}{2.862324in}}%
\pgfpathmoveto{\pgfqpoint{3.320213in}{2.859374in}}%
\pgfpathlineto{\pgfqpoint{3.320213in}{2.859374in}}%
\pgfpathlineto{\pgfqpoint{3.320213in}{2.862324in}}%
\pgfpathlineto{\pgfqpoint{3.324754in}{2.862324in}}%
\pgfpathlineto{\pgfqpoint{3.324754in}{2.859374in}}%
\pgfpathmoveto{\pgfqpoint{3.320213in}{2.862324in}}%
\pgfpathlineto{\pgfqpoint{3.320213in}{2.862324in}}%
\pgfpathlineto{\pgfqpoint{3.320213in}{2.865273in}}%
\pgfpathlineto{\pgfqpoint{3.324754in}{2.865273in}}%
\pgfpathlineto{\pgfqpoint{3.324754in}{2.862324in}}%
\pgfpathmoveto{\pgfqpoint{3.324754in}{2.859374in}}%
\pgfpathlineto{\pgfqpoint{3.324754in}{2.859374in}}%
\pgfpathlineto{\pgfqpoint{3.324754in}{2.862324in}}%
\pgfpathlineto{\pgfqpoint{3.329295in}{2.862324in}}%
\pgfpathlineto{\pgfqpoint{3.329295in}{2.859374in}}%
\pgfpathmoveto{\pgfqpoint{3.329295in}{2.859374in}}%
\pgfpathlineto{\pgfqpoint{3.329295in}{2.859374in}}%
\pgfpathlineto{\pgfqpoint{3.329295in}{2.862324in}}%
\pgfpathlineto{\pgfqpoint{3.333836in}{2.862324in}}%
\pgfpathlineto{\pgfqpoint{3.333836in}{2.859374in}}%
\pgfpathmoveto{\pgfqpoint{3.333836in}{2.859374in}}%
\pgfpathlineto{\pgfqpoint{3.333836in}{2.859374in}}%
\pgfpathlineto{\pgfqpoint{3.333836in}{2.862324in}}%
\pgfpathlineto{\pgfqpoint{3.338377in}{2.862324in}}%
\pgfpathlineto{\pgfqpoint{3.338377in}{2.859374in}}%
\pgfpathmoveto{\pgfqpoint{3.365623in}{2.850527in}}%
\pgfpathlineto{\pgfqpoint{3.365623in}{2.850527in}}%
\pgfpathlineto{\pgfqpoint{3.365623in}{2.853476in}}%
\pgfpathlineto{\pgfqpoint{3.370164in}{2.853476in}}%
\pgfpathlineto{\pgfqpoint{3.370164in}{2.850527in}}%
\pgfpathmoveto{\pgfqpoint{3.370164in}{2.850527in}}%
\pgfpathlineto{\pgfqpoint{3.370164in}{2.850527in}}%
\pgfpathlineto{\pgfqpoint{3.370164in}{2.853476in}}%
\pgfpathlineto{\pgfqpoint{3.374705in}{2.853476in}}%
\pgfpathlineto{\pgfqpoint{3.374705in}{2.850527in}}%
\pgfpathmoveto{\pgfqpoint{3.374705in}{2.847577in}}%
\pgfpathlineto{\pgfqpoint{3.374705in}{2.847577in}}%
\pgfpathlineto{\pgfqpoint{3.374705in}{2.850527in}}%
\pgfpathlineto{\pgfqpoint{3.379246in}{2.850527in}}%
\pgfpathlineto{\pgfqpoint{3.379246in}{2.847577in}}%
\pgfpathmoveto{\pgfqpoint{3.374705in}{2.850527in}}%
\pgfpathlineto{\pgfqpoint{3.374705in}{2.850527in}}%
\pgfpathlineto{\pgfqpoint{3.374705in}{2.853476in}}%
\pgfpathlineto{\pgfqpoint{3.379246in}{2.853476in}}%
\pgfpathlineto{\pgfqpoint{3.379246in}{2.850527in}}%
\pgfpathmoveto{\pgfqpoint{3.379246in}{2.847577in}}%
\pgfpathlineto{\pgfqpoint{3.379246in}{2.847577in}}%
\pgfpathlineto{\pgfqpoint{3.379246in}{2.850527in}}%
\pgfpathlineto{\pgfqpoint{3.383787in}{2.850527in}}%
\pgfpathlineto{\pgfqpoint{3.383787in}{2.847577in}}%
\pgfpathmoveto{\pgfqpoint{3.388328in}{2.844628in}}%
\pgfpathlineto{\pgfqpoint{3.388328in}{2.844628in}}%
\pgfpathlineto{\pgfqpoint{3.388328in}{2.847577in}}%
\pgfpathlineto{\pgfqpoint{3.392869in}{2.847577in}}%
\pgfpathlineto{\pgfqpoint{3.392869in}{2.844628in}}%
\pgfpathmoveto{\pgfqpoint{3.392869in}{2.844628in}}%
\pgfpathlineto{\pgfqpoint{3.392869in}{2.844628in}}%
\pgfpathlineto{\pgfqpoint{3.392869in}{2.847577in}}%
\pgfpathlineto{\pgfqpoint{3.397410in}{2.847577in}}%
\pgfpathlineto{\pgfqpoint{3.397410in}{2.844628in}}%
\pgfpathmoveto{\pgfqpoint{3.397410in}{2.844628in}}%
\pgfpathlineto{\pgfqpoint{3.397410in}{2.844628in}}%
\pgfpathlineto{\pgfqpoint{3.397410in}{2.847577in}}%
\pgfpathlineto{\pgfqpoint{3.401951in}{2.847577in}}%
\pgfpathlineto{\pgfqpoint{3.401951in}{2.844628in}}%
\pgfpathmoveto{\pgfqpoint{3.383787in}{2.847577in}}%
\pgfpathlineto{\pgfqpoint{3.383787in}{2.847577in}}%
\pgfpathlineto{\pgfqpoint{3.383787in}{2.850527in}}%
\pgfpathlineto{\pgfqpoint{3.388328in}{2.850527in}}%
\pgfpathlineto{\pgfqpoint{3.388328in}{2.847577in}}%
\pgfpathmoveto{\pgfqpoint{3.388328in}{2.847577in}}%
\pgfpathlineto{\pgfqpoint{3.388328in}{2.847577in}}%
\pgfpathlineto{\pgfqpoint{3.388328in}{2.850527in}}%
\pgfpathlineto{\pgfqpoint{3.392869in}{2.850527in}}%
\pgfpathlineto{\pgfqpoint{3.392869in}{2.847577in}}%
\pgfpathmoveto{\pgfqpoint{3.401951in}{2.841679in}}%
\pgfpathlineto{\pgfqpoint{3.401951in}{2.841679in}}%
\pgfpathlineto{\pgfqpoint{3.401951in}{2.844628in}}%
\pgfpathlineto{\pgfqpoint{3.406492in}{2.844628in}}%
\pgfpathlineto{\pgfqpoint{3.406492in}{2.841679in}}%
\pgfpathmoveto{\pgfqpoint{3.401951in}{2.844628in}}%
\pgfpathlineto{\pgfqpoint{3.401951in}{2.844628in}}%
\pgfpathlineto{\pgfqpoint{3.401951in}{2.847577in}}%
\pgfpathlineto{\pgfqpoint{3.406492in}{2.847577in}}%
\pgfpathlineto{\pgfqpoint{3.406492in}{2.844628in}}%
\pgfpathmoveto{\pgfqpoint{3.406492in}{2.841679in}}%
\pgfpathlineto{\pgfqpoint{3.406492in}{2.841679in}}%
\pgfpathlineto{\pgfqpoint{3.406492in}{2.844628in}}%
\pgfpathlineto{\pgfqpoint{3.411033in}{2.844628in}}%
\pgfpathlineto{\pgfqpoint{3.411033in}{2.841679in}}%
\pgfpathmoveto{\pgfqpoint{3.415574in}{2.838730in}}%
\pgfpathlineto{\pgfqpoint{3.415574in}{2.838730in}}%
\pgfpathlineto{\pgfqpoint{3.415574in}{2.841679in}}%
\pgfpathlineto{\pgfqpoint{3.420116in}{2.841679in}}%
\pgfpathlineto{\pgfqpoint{3.420116in}{2.838730in}}%
\pgfpathmoveto{\pgfqpoint{3.411033in}{2.841679in}}%
\pgfpathlineto{\pgfqpoint{3.411033in}{2.841679in}}%
\pgfpathlineto{\pgfqpoint{3.411033in}{2.844628in}}%
\pgfpathlineto{\pgfqpoint{3.415574in}{2.844628in}}%
\pgfpathlineto{\pgfqpoint{3.415574in}{2.841679in}}%
\pgfpathmoveto{\pgfqpoint{3.415574in}{2.841679in}}%
\pgfpathlineto{\pgfqpoint{3.415574in}{2.841679in}}%
\pgfpathlineto{\pgfqpoint{3.415574in}{2.844628in}}%
\pgfpathlineto{\pgfqpoint{3.420116in}{2.844628in}}%
\pgfpathlineto{\pgfqpoint{3.420116in}{2.841679in}}%
\pgfpathmoveto{\pgfqpoint{3.420116in}{2.838730in}}%
\pgfpathlineto{\pgfqpoint{3.420116in}{2.838730in}}%
\pgfpathlineto{\pgfqpoint{3.420116in}{2.841679in}}%
\pgfpathlineto{\pgfqpoint{3.424657in}{2.841679in}}%
\pgfpathlineto{\pgfqpoint{3.424657in}{2.838730in}}%
\pgfpathmoveto{\pgfqpoint{3.424657in}{2.838730in}}%
\pgfpathlineto{\pgfqpoint{3.424657in}{2.838730in}}%
\pgfpathlineto{\pgfqpoint{3.424657in}{2.841679in}}%
\pgfpathlineto{\pgfqpoint{3.429198in}{2.841679in}}%
\pgfpathlineto{\pgfqpoint{3.429198in}{2.838730in}}%
\pgfpathmoveto{\pgfqpoint{3.429198in}{2.835780in}}%
\pgfpathlineto{\pgfqpoint{3.429198in}{2.835780in}}%
\pgfpathlineto{\pgfqpoint{3.429198in}{2.838730in}}%
\pgfpathlineto{\pgfqpoint{3.433739in}{2.838730in}}%
\pgfpathlineto{\pgfqpoint{3.433739in}{2.835780in}}%
\pgfpathmoveto{\pgfqpoint{3.429198in}{2.838730in}}%
\pgfpathlineto{\pgfqpoint{3.429198in}{2.838730in}}%
\pgfpathlineto{\pgfqpoint{3.429198in}{2.841679in}}%
\pgfpathlineto{\pgfqpoint{3.433739in}{2.841679in}}%
\pgfpathlineto{\pgfqpoint{3.433739in}{2.838730in}}%
\pgfpathmoveto{\pgfqpoint{3.433739in}{2.835780in}}%
\pgfpathlineto{\pgfqpoint{3.433739in}{2.835780in}}%
\pgfpathlineto{\pgfqpoint{3.433739in}{2.838730in}}%
\pgfpathlineto{\pgfqpoint{3.438280in}{2.838730in}}%
\pgfpathlineto{\pgfqpoint{3.438280in}{2.835780in}}%
\pgfpathmoveto{\pgfqpoint{3.442821in}{2.832831in}}%
\pgfpathlineto{\pgfqpoint{3.442821in}{2.832831in}}%
\pgfpathlineto{\pgfqpoint{3.442821in}{2.835780in}}%
\pgfpathlineto{\pgfqpoint{3.447362in}{2.835780in}}%
\pgfpathlineto{\pgfqpoint{3.447362in}{2.832831in}}%
\pgfpathmoveto{\pgfqpoint{3.447362in}{2.832831in}}%
\pgfpathlineto{\pgfqpoint{3.447362in}{2.832831in}}%
\pgfpathlineto{\pgfqpoint{3.447362in}{2.835780in}}%
\pgfpathlineto{\pgfqpoint{3.451903in}{2.835780in}}%
\pgfpathlineto{\pgfqpoint{3.451903in}{2.832831in}}%
\pgfpathmoveto{\pgfqpoint{3.451903in}{2.832831in}}%
\pgfpathlineto{\pgfqpoint{3.451903in}{2.832831in}}%
\pgfpathlineto{\pgfqpoint{3.451903in}{2.835780in}}%
\pgfpathlineto{\pgfqpoint{3.456444in}{2.835780in}}%
\pgfpathlineto{\pgfqpoint{3.456444in}{2.832831in}}%
\pgfpathmoveto{\pgfqpoint{3.456444in}{2.829882in}}%
\pgfpathlineto{\pgfqpoint{3.456444in}{2.829882in}}%
\pgfpathlineto{\pgfqpoint{3.456444in}{2.832831in}}%
\pgfpathlineto{\pgfqpoint{3.460985in}{2.832831in}}%
\pgfpathlineto{\pgfqpoint{3.460985in}{2.829882in}}%
\pgfpathmoveto{\pgfqpoint{3.456444in}{2.832831in}}%
\pgfpathlineto{\pgfqpoint{3.456444in}{2.832831in}}%
\pgfpathlineto{\pgfqpoint{3.456444in}{2.835780in}}%
\pgfpathlineto{\pgfqpoint{3.460985in}{2.835780in}}%
\pgfpathlineto{\pgfqpoint{3.460985in}{2.832831in}}%
\pgfpathmoveto{\pgfqpoint{3.460985in}{2.829882in}}%
\pgfpathlineto{\pgfqpoint{3.460985in}{2.829882in}}%
\pgfpathlineto{\pgfqpoint{3.460985in}{2.832831in}}%
\pgfpathlineto{\pgfqpoint{3.465526in}{2.832831in}}%
\pgfpathlineto{\pgfqpoint{3.465526in}{2.829882in}}%
\pgfpathmoveto{\pgfqpoint{3.470067in}{2.826932in}}%
\pgfpathlineto{\pgfqpoint{3.470067in}{2.826932in}}%
\pgfpathlineto{\pgfqpoint{3.470067in}{2.829882in}}%
\pgfpathlineto{\pgfqpoint{3.474608in}{2.829882in}}%
\pgfpathlineto{\pgfqpoint{3.474608in}{2.826932in}}%
\pgfpathmoveto{\pgfqpoint{3.465526in}{2.829882in}}%
\pgfpathlineto{\pgfqpoint{3.465526in}{2.829882in}}%
\pgfpathlineto{\pgfqpoint{3.465526in}{2.832831in}}%
\pgfpathlineto{\pgfqpoint{3.470067in}{2.832831in}}%
\pgfpathlineto{\pgfqpoint{3.470067in}{2.829882in}}%
\pgfpathmoveto{\pgfqpoint{3.470067in}{2.829882in}}%
\pgfpathlineto{\pgfqpoint{3.470067in}{2.829882in}}%
\pgfpathlineto{\pgfqpoint{3.470067in}{2.832831in}}%
\pgfpathlineto{\pgfqpoint{3.474608in}{2.832831in}}%
\pgfpathlineto{\pgfqpoint{3.474608in}{2.829882in}}%
\pgfpathmoveto{\pgfqpoint{3.438280in}{2.835780in}}%
\pgfpathlineto{\pgfqpoint{3.438280in}{2.835780in}}%
\pgfpathlineto{\pgfqpoint{3.438280in}{2.838730in}}%
\pgfpathlineto{\pgfqpoint{3.442821in}{2.838730in}}%
\pgfpathlineto{\pgfqpoint{3.442821in}{2.835780in}}%
\pgfpathmoveto{\pgfqpoint{3.442821in}{2.835780in}}%
\pgfpathlineto{\pgfqpoint{3.442821in}{2.835780in}}%
\pgfpathlineto{\pgfqpoint{3.442821in}{2.838730in}}%
\pgfpathlineto{\pgfqpoint{3.447362in}{2.838730in}}%
\pgfpathlineto{\pgfqpoint{3.447362in}{2.835780in}}%
\pgfpathmoveto{\pgfqpoint{3.474608in}{2.826932in}}%
\pgfpathlineto{\pgfqpoint{3.474608in}{2.826932in}}%
\pgfpathlineto{\pgfqpoint{3.474608in}{2.829882in}}%
\pgfpathlineto{\pgfqpoint{3.479149in}{2.829882in}}%
\pgfpathlineto{\pgfqpoint{3.479149in}{2.826932in}}%
\pgfpathmoveto{\pgfqpoint{3.479149in}{2.826932in}}%
\pgfpathlineto{\pgfqpoint{3.479149in}{2.826932in}}%
\pgfpathlineto{\pgfqpoint{3.479149in}{2.829882in}}%
\pgfpathlineto{\pgfqpoint{3.483690in}{2.829882in}}%
\pgfpathlineto{\pgfqpoint{3.483690in}{2.826932in}}%
\pgfpathmoveto{\pgfqpoint{3.483690in}{2.823983in}}%
\pgfpathlineto{\pgfqpoint{3.483690in}{2.823983in}}%
\pgfpathlineto{\pgfqpoint{3.483690in}{2.826932in}}%
\pgfpathlineto{\pgfqpoint{3.488231in}{2.826932in}}%
\pgfpathlineto{\pgfqpoint{3.488231in}{2.823983in}}%
\pgfpathmoveto{\pgfqpoint{3.483690in}{2.826932in}}%
\pgfpathlineto{\pgfqpoint{3.483690in}{2.826932in}}%
\pgfpathlineto{\pgfqpoint{3.483690in}{2.829882in}}%
\pgfpathlineto{\pgfqpoint{3.488231in}{2.829882in}}%
\pgfpathlineto{\pgfqpoint{3.488231in}{2.826932in}}%
\pgfpathmoveto{\pgfqpoint{3.488231in}{2.823983in}}%
\pgfpathlineto{\pgfqpoint{3.488231in}{2.823983in}}%
\pgfpathlineto{\pgfqpoint{3.488231in}{2.826932in}}%
\pgfpathlineto{\pgfqpoint{3.492773in}{2.826932in}}%
\pgfpathlineto{\pgfqpoint{3.492773in}{2.823983in}}%
\pgfpathmoveto{\pgfqpoint{3.497314in}{2.821034in}}%
\pgfpathlineto{\pgfqpoint{3.497314in}{2.821034in}}%
\pgfpathlineto{\pgfqpoint{3.497314in}{2.823983in}}%
\pgfpathlineto{\pgfqpoint{3.501855in}{2.823983in}}%
\pgfpathlineto{\pgfqpoint{3.501855in}{2.821034in}}%
\pgfpathmoveto{\pgfqpoint{3.501855in}{2.821034in}}%
\pgfpathlineto{\pgfqpoint{3.501855in}{2.821034in}}%
\pgfpathlineto{\pgfqpoint{3.501855in}{2.823983in}}%
\pgfpathlineto{\pgfqpoint{3.506396in}{2.823983in}}%
\pgfpathlineto{\pgfqpoint{3.506396in}{2.821034in}}%
\pgfpathmoveto{\pgfqpoint{3.506396in}{2.821034in}}%
\pgfpathlineto{\pgfqpoint{3.506396in}{2.821034in}}%
\pgfpathlineto{\pgfqpoint{3.506396in}{2.823983in}}%
\pgfpathlineto{\pgfqpoint{3.510937in}{2.823983in}}%
\pgfpathlineto{\pgfqpoint{3.510937in}{2.821034in}}%
\pgfpathmoveto{\pgfqpoint{3.492773in}{2.823983in}}%
\pgfpathlineto{\pgfqpoint{3.492773in}{2.823983in}}%
\pgfpathlineto{\pgfqpoint{3.492773in}{2.826932in}}%
\pgfpathlineto{\pgfqpoint{3.497314in}{2.826932in}}%
\pgfpathlineto{\pgfqpoint{3.497314in}{2.823983in}}%
\pgfpathmoveto{\pgfqpoint{3.497314in}{2.823983in}}%
\pgfpathlineto{\pgfqpoint{3.497314in}{2.823983in}}%
\pgfpathlineto{\pgfqpoint{3.497314in}{2.826932in}}%
\pgfpathlineto{\pgfqpoint{3.501855in}{2.826932in}}%
\pgfpathlineto{\pgfqpoint{3.501855in}{2.823983in}}%
\pgfpathmoveto{\pgfqpoint{3.551807in}{2.809237in}}%
\pgfpathlineto{\pgfqpoint{3.551807in}{2.809237in}}%
\pgfpathlineto{\pgfqpoint{3.551807in}{2.812186in}}%
\pgfpathlineto{\pgfqpoint{3.556348in}{2.812186in}}%
\pgfpathlineto{\pgfqpoint{3.556348in}{2.809237in}}%
\pgfpathmoveto{\pgfqpoint{3.556348in}{2.809237in}}%
\pgfpathlineto{\pgfqpoint{3.556348in}{2.809237in}}%
\pgfpathlineto{\pgfqpoint{3.556348in}{2.812186in}}%
\pgfpathlineto{\pgfqpoint{3.560890in}{2.812186in}}%
\pgfpathlineto{\pgfqpoint{3.560890in}{2.809237in}}%
\pgfpathmoveto{\pgfqpoint{3.560890in}{2.809237in}}%
\pgfpathlineto{\pgfqpoint{3.560890in}{2.809237in}}%
\pgfpathlineto{\pgfqpoint{3.560890in}{2.812186in}}%
\pgfpathlineto{\pgfqpoint{3.565431in}{2.812186in}}%
\pgfpathlineto{\pgfqpoint{3.565431in}{2.809237in}}%
\pgfpathmoveto{\pgfqpoint{3.565431in}{2.806288in}}%
\pgfpathlineto{\pgfqpoint{3.565431in}{2.806288in}}%
\pgfpathlineto{\pgfqpoint{3.565431in}{2.809237in}}%
\pgfpathlineto{\pgfqpoint{3.569972in}{2.809237in}}%
\pgfpathlineto{\pgfqpoint{3.569972in}{2.806288in}}%
\pgfpathmoveto{\pgfqpoint{3.565431in}{2.809237in}}%
\pgfpathlineto{\pgfqpoint{3.565431in}{2.809237in}}%
\pgfpathlineto{\pgfqpoint{3.565431in}{2.812186in}}%
\pgfpathlineto{\pgfqpoint{3.569972in}{2.812186in}}%
\pgfpathlineto{\pgfqpoint{3.569972in}{2.809237in}}%
\pgfpathmoveto{\pgfqpoint{3.569972in}{2.806288in}}%
\pgfpathlineto{\pgfqpoint{3.569972in}{2.806288in}}%
\pgfpathlineto{\pgfqpoint{3.569972in}{2.809237in}}%
\pgfpathlineto{\pgfqpoint{3.574513in}{2.809237in}}%
\pgfpathlineto{\pgfqpoint{3.574513in}{2.806288in}}%
\pgfpathmoveto{\pgfqpoint{3.579054in}{2.803338in}}%
\pgfpathlineto{\pgfqpoint{3.579054in}{2.803338in}}%
\pgfpathlineto{\pgfqpoint{3.579054in}{2.806288in}}%
\pgfpathlineto{\pgfqpoint{3.583596in}{2.806288in}}%
\pgfpathlineto{\pgfqpoint{3.583596in}{2.803338in}}%
\pgfpathmoveto{\pgfqpoint{3.574513in}{2.806288in}}%
\pgfpathlineto{\pgfqpoint{3.574513in}{2.806288in}}%
\pgfpathlineto{\pgfqpoint{3.574513in}{2.809237in}}%
\pgfpathlineto{\pgfqpoint{3.579054in}{2.809237in}}%
\pgfpathlineto{\pgfqpoint{3.579054in}{2.806288in}}%
\pgfpathmoveto{\pgfqpoint{3.579054in}{2.806288in}}%
\pgfpathlineto{\pgfqpoint{3.579054in}{2.806288in}}%
\pgfpathlineto{\pgfqpoint{3.579054in}{2.809237in}}%
\pgfpathlineto{\pgfqpoint{3.583596in}{2.809237in}}%
\pgfpathlineto{\pgfqpoint{3.583596in}{2.806288in}}%
\pgfpathmoveto{\pgfqpoint{3.510937in}{2.818085in}}%
\pgfpathlineto{\pgfqpoint{3.510937in}{2.818085in}}%
\pgfpathlineto{\pgfqpoint{3.510937in}{2.821034in}}%
\pgfpathlineto{\pgfqpoint{3.515478in}{2.821034in}}%
\pgfpathlineto{\pgfqpoint{3.515478in}{2.818085in}}%
\pgfpathmoveto{\pgfqpoint{3.510937in}{2.821034in}}%
\pgfpathlineto{\pgfqpoint{3.510937in}{2.821034in}}%
\pgfpathlineto{\pgfqpoint{3.510937in}{2.823983in}}%
\pgfpathlineto{\pgfqpoint{3.515478in}{2.823983in}}%
\pgfpathlineto{\pgfqpoint{3.515478in}{2.821034in}}%
\pgfpathmoveto{\pgfqpoint{3.515478in}{2.818085in}}%
\pgfpathlineto{\pgfqpoint{3.515478in}{2.818085in}}%
\pgfpathlineto{\pgfqpoint{3.515478in}{2.821034in}}%
\pgfpathlineto{\pgfqpoint{3.520019in}{2.821034in}}%
\pgfpathlineto{\pgfqpoint{3.520019in}{2.818085in}}%
\pgfpathmoveto{\pgfqpoint{3.524560in}{2.815135in}}%
\pgfpathlineto{\pgfqpoint{3.524560in}{2.815135in}}%
\pgfpathlineto{\pgfqpoint{3.524560in}{2.818085in}}%
\pgfpathlineto{\pgfqpoint{3.529101in}{2.818085in}}%
\pgfpathlineto{\pgfqpoint{3.529101in}{2.815135in}}%
\pgfpathmoveto{\pgfqpoint{3.520019in}{2.818085in}}%
\pgfpathlineto{\pgfqpoint{3.520019in}{2.818085in}}%
\pgfpathlineto{\pgfqpoint{3.520019in}{2.821034in}}%
\pgfpathlineto{\pgfqpoint{3.524560in}{2.821034in}}%
\pgfpathlineto{\pgfqpoint{3.524560in}{2.818085in}}%
\pgfpathmoveto{\pgfqpoint{3.524560in}{2.818085in}}%
\pgfpathlineto{\pgfqpoint{3.524560in}{2.818085in}}%
\pgfpathlineto{\pgfqpoint{3.524560in}{2.821034in}}%
\pgfpathlineto{\pgfqpoint{3.529101in}{2.821034in}}%
\pgfpathlineto{\pgfqpoint{3.529101in}{2.818085in}}%
\pgfpathmoveto{\pgfqpoint{3.529101in}{2.815135in}}%
\pgfpathlineto{\pgfqpoint{3.529101in}{2.815135in}}%
\pgfpathlineto{\pgfqpoint{3.529101in}{2.818085in}}%
\pgfpathlineto{\pgfqpoint{3.533643in}{2.818085in}}%
\pgfpathlineto{\pgfqpoint{3.533643in}{2.815135in}}%
\pgfpathmoveto{\pgfqpoint{3.533643in}{2.815135in}}%
\pgfpathlineto{\pgfqpoint{3.533643in}{2.815135in}}%
\pgfpathlineto{\pgfqpoint{3.533643in}{2.818085in}}%
\pgfpathlineto{\pgfqpoint{3.538184in}{2.818085in}}%
\pgfpathlineto{\pgfqpoint{3.538184in}{2.815135in}}%
\pgfpathmoveto{\pgfqpoint{3.538184in}{2.812186in}}%
\pgfpathlineto{\pgfqpoint{3.538184in}{2.812186in}}%
\pgfpathlineto{\pgfqpoint{3.538184in}{2.815135in}}%
\pgfpathlineto{\pgfqpoint{3.542725in}{2.815135in}}%
\pgfpathlineto{\pgfqpoint{3.542725in}{2.812186in}}%
\pgfpathmoveto{\pgfqpoint{3.538184in}{2.815135in}}%
\pgfpathlineto{\pgfqpoint{3.538184in}{2.815135in}}%
\pgfpathlineto{\pgfqpoint{3.538184in}{2.818085in}}%
\pgfpathlineto{\pgfqpoint{3.542725in}{2.818085in}}%
\pgfpathlineto{\pgfqpoint{3.542725in}{2.815135in}}%
\pgfpathmoveto{\pgfqpoint{3.542725in}{2.812186in}}%
\pgfpathlineto{\pgfqpoint{3.542725in}{2.812186in}}%
\pgfpathlineto{\pgfqpoint{3.542725in}{2.815135in}}%
\pgfpathlineto{\pgfqpoint{3.547266in}{2.815135in}}%
\pgfpathlineto{\pgfqpoint{3.547266in}{2.812186in}}%
\pgfpathmoveto{\pgfqpoint{3.547266in}{2.812186in}}%
\pgfpathlineto{\pgfqpoint{3.547266in}{2.812186in}}%
\pgfpathlineto{\pgfqpoint{3.547266in}{2.815135in}}%
\pgfpathlineto{\pgfqpoint{3.551807in}{2.815135in}}%
\pgfpathlineto{\pgfqpoint{3.551807in}{2.812186in}}%
\pgfpathmoveto{\pgfqpoint{3.551807in}{2.812186in}}%
\pgfpathlineto{\pgfqpoint{3.551807in}{2.812186in}}%
\pgfpathlineto{\pgfqpoint{3.551807in}{2.815135in}}%
\pgfpathlineto{\pgfqpoint{3.556348in}{2.815135in}}%
\pgfpathlineto{\pgfqpoint{3.556348in}{2.812186in}}%
\pgfpathmoveto{\pgfqpoint{3.583596in}{2.803338in}}%
\pgfpathlineto{\pgfqpoint{3.583596in}{2.803338in}}%
\pgfpathlineto{\pgfqpoint{3.583596in}{2.806288in}}%
\pgfpathlineto{\pgfqpoint{3.588137in}{2.806288in}}%
\pgfpathlineto{\pgfqpoint{3.588137in}{2.803338in}}%
\pgfpathmoveto{\pgfqpoint{3.588137in}{2.803338in}}%
\pgfpathlineto{\pgfqpoint{3.588137in}{2.803338in}}%
\pgfpathlineto{\pgfqpoint{3.588137in}{2.806288in}}%
\pgfpathlineto{\pgfqpoint{3.592678in}{2.806288in}}%
\pgfpathlineto{\pgfqpoint{3.592678in}{2.803338in}}%
\pgfpathmoveto{\pgfqpoint{3.592678in}{2.800389in}}%
\pgfpathlineto{\pgfqpoint{3.592678in}{2.800389in}}%
\pgfpathlineto{\pgfqpoint{3.592678in}{2.803338in}}%
\pgfpathlineto{\pgfqpoint{3.597219in}{2.803338in}}%
\pgfpathlineto{\pgfqpoint{3.597219in}{2.800389in}}%
\pgfpathmoveto{\pgfqpoint{3.592678in}{2.803338in}}%
\pgfpathlineto{\pgfqpoint{3.592678in}{2.803338in}}%
\pgfpathlineto{\pgfqpoint{3.592678in}{2.806288in}}%
\pgfpathlineto{\pgfqpoint{3.597219in}{2.806288in}}%
\pgfpathlineto{\pgfqpoint{3.597219in}{2.803338in}}%
\pgfpathmoveto{\pgfqpoint{3.597219in}{2.800389in}}%
\pgfpathlineto{\pgfqpoint{3.597219in}{2.800389in}}%
\pgfpathlineto{\pgfqpoint{3.597219in}{2.803338in}}%
\pgfpathlineto{\pgfqpoint{3.601760in}{2.803338in}}%
\pgfpathlineto{\pgfqpoint{3.601760in}{2.800389in}}%
\pgfpathmoveto{\pgfqpoint{3.606301in}{2.797440in}}%
\pgfpathlineto{\pgfqpoint{3.606301in}{2.797440in}}%
\pgfpathlineto{\pgfqpoint{3.606301in}{2.800389in}}%
\pgfpathlineto{\pgfqpoint{3.610843in}{2.800389in}}%
\pgfpathlineto{\pgfqpoint{3.610843in}{2.797440in}}%
\pgfpathmoveto{\pgfqpoint{3.610843in}{2.797440in}}%
\pgfpathlineto{\pgfqpoint{3.610843in}{2.797440in}}%
\pgfpathlineto{\pgfqpoint{3.610843in}{2.800389in}}%
\pgfpathlineto{\pgfqpoint{3.615384in}{2.800389in}}%
\pgfpathlineto{\pgfqpoint{3.615384in}{2.797440in}}%
\pgfpathmoveto{\pgfqpoint{3.615384in}{2.797440in}}%
\pgfpathlineto{\pgfqpoint{3.615384in}{2.797440in}}%
\pgfpathlineto{\pgfqpoint{3.615384in}{2.800389in}}%
\pgfpathlineto{\pgfqpoint{3.619925in}{2.800389in}}%
\pgfpathlineto{\pgfqpoint{3.619925in}{2.797440in}}%
\pgfpathmoveto{\pgfqpoint{3.601760in}{2.800389in}}%
\pgfpathlineto{\pgfqpoint{3.601760in}{2.800389in}}%
\pgfpathlineto{\pgfqpoint{3.601760in}{2.803338in}}%
\pgfpathlineto{\pgfqpoint{3.606301in}{2.803338in}}%
\pgfpathlineto{\pgfqpoint{3.606301in}{2.800389in}}%
\pgfpathmoveto{\pgfqpoint{3.606301in}{2.800389in}}%
\pgfpathlineto{\pgfqpoint{3.606301in}{2.800389in}}%
\pgfpathlineto{\pgfqpoint{3.606301in}{2.803338in}}%
\pgfpathlineto{\pgfqpoint{3.610843in}{2.803338in}}%
\pgfpathlineto{\pgfqpoint{3.610843in}{2.800389in}}%
\pgfpathmoveto{\pgfqpoint{3.619925in}{2.794491in}}%
\pgfpathlineto{\pgfqpoint{3.619925in}{2.794491in}}%
\pgfpathlineto{\pgfqpoint{3.619925in}{2.797440in}}%
\pgfpathlineto{\pgfqpoint{3.624466in}{2.797440in}}%
\pgfpathlineto{\pgfqpoint{3.624466in}{2.794491in}}%
\pgfpathmoveto{\pgfqpoint{3.619925in}{2.797440in}}%
\pgfpathlineto{\pgfqpoint{3.619925in}{2.797440in}}%
\pgfpathlineto{\pgfqpoint{3.619925in}{2.800389in}}%
\pgfpathlineto{\pgfqpoint{3.624466in}{2.800389in}}%
\pgfpathlineto{\pgfqpoint{3.624466in}{2.797440in}}%
\pgfpathmoveto{\pgfqpoint{3.624466in}{2.794491in}}%
\pgfpathlineto{\pgfqpoint{3.624466in}{2.794491in}}%
\pgfpathlineto{\pgfqpoint{3.624466in}{2.797440in}}%
\pgfpathlineto{\pgfqpoint{3.629007in}{2.797440in}}%
\pgfpathlineto{\pgfqpoint{3.629007in}{2.794491in}}%
\pgfpathmoveto{\pgfqpoint{3.633548in}{2.791541in}}%
\pgfpathlineto{\pgfqpoint{3.633548in}{2.791541in}}%
\pgfpathlineto{\pgfqpoint{3.633548in}{2.794491in}}%
\pgfpathlineto{\pgfqpoint{3.638090in}{2.794491in}}%
\pgfpathlineto{\pgfqpoint{3.638090in}{2.791541in}}%
\pgfpathmoveto{\pgfqpoint{3.629007in}{2.794491in}}%
\pgfpathlineto{\pgfqpoint{3.629007in}{2.794491in}}%
\pgfpathlineto{\pgfqpoint{3.629007in}{2.797440in}}%
\pgfpathlineto{\pgfqpoint{3.633548in}{2.797440in}}%
\pgfpathlineto{\pgfqpoint{3.633548in}{2.794491in}}%
\pgfpathmoveto{\pgfqpoint{3.633548in}{2.794491in}}%
\pgfpathlineto{\pgfqpoint{3.633548in}{2.794491in}}%
\pgfpathlineto{\pgfqpoint{3.633548in}{2.797440in}}%
\pgfpathlineto{\pgfqpoint{3.638090in}{2.797440in}}%
\pgfpathlineto{\pgfqpoint{3.638090in}{2.794491in}}%
\pgfpathmoveto{\pgfqpoint{3.638090in}{2.791541in}}%
\pgfpathlineto{\pgfqpoint{3.638090in}{2.791541in}}%
\pgfpathlineto{\pgfqpoint{3.638090in}{2.794491in}}%
\pgfpathlineto{\pgfqpoint{3.642631in}{2.794491in}}%
\pgfpathlineto{\pgfqpoint{3.642631in}{2.791541in}}%
\pgfpathmoveto{\pgfqpoint{3.642631in}{2.791541in}}%
\pgfpathlineto{\pgfqpoint{3.642631in}{2.791541in}}%
\pgfpathlineto{\pgfqpoint{3.642631in}{2.794491in}}%
\pgfpathlineto{\pgfqpoint{3.647172in}{2.794491in}}%
\pgfpathlineto{\pgfqpoint{3.647172in}{2.791541in}}%
\pgfpathmoveto{\pgfqpoint{3.647172in}{2.788592in}}%
\pgfpathlineto{\pgfqpoint{3.647172in}{2.788592in}}%
\pgfpathlineto{\pgfqpoint{3.647172in}{2.791541in}}%
\pgfpathlineto{\pgfqpoint{3.651713in}{2.791541in}}%
\pgfpathlineto{\pgfqpoint{3.651713in}{2.788592in}}%
\pgfpathmoveto{\pgfqpoint{3.647172in}{2.791541in}}%
\pgfpathlineto{\pgfqpoint{3.647172in}{2.791541in}}%
\pgfpathlineto{\pgfqpoint{3.647172in}{2.794491in}}%
\pgfpathlineto{\pgfqpoint{3.651713in}{2.794491in}}%
\pgfpathlineto{\pgfqpoint{3.651713in}{2.791541in}}%
\pgfpathmoveto{\pgfqpoint{3.651713in}{2.788592in}}%
\pgfpathlineto{\pgfqpoint{3.651713in}{2.788592in}}%
\pgfpathlineto{\pgfqpoint{3.651713in}{2.791541in}}%
\pgfpathlineto{\pgfqpoint{3.656254in}{2.791541in}}%
\pgfpathlineto{\pgfqpoint{3.656254in}{2.788592in}}%
\pgfpathmoveto{\pgfqpoint{3.769777in}{2.762049in}}%
\pgfpathlineto{\pgfqpoint{3.769777in}{2.762049in}}%
\pgfpathlineto{\pgfqpoint{3.769777in}{2.764998in}}%
\pgfpathlineto{\pgfqpoint{3.774318in}{2.764998in}}%
\pgfpathlineto{\pgfqpoint{3.774318in}{2.762049in}}%
\pgfpathmoveto{\pgfqpoint{3.774318in}{2.762049in}}%
\pgfpathlineto{\pgfqpoint{3.774318in}{2.762049in}}%
\pgfpathlineto{\pgfqpoint{3.774318in}{2.764998in}}%
\pgfpathlineto{\pgfqpoint{3.778859in}{2.764998in}}%
\pgfpathlineto{\pgfqpoint{3.778859in}{2.762049in}}%
\pgfpathmoveto{\pgfqpoint{3.778859in}{2.762049in}}%
\pgfpathlineto{\pgfqpoint{3.778859in}{2.762049in}}%
\pgfpathlineto{\pgfqpoint{3.778859in}{2.764998in}}%
\pgfpathlineto{\pgfqpoint{3.783400in}{2.764998in}}%
\pgfpathlineto{\pgfqpoint{3.783400in}{2.762049in}}%
\pgfpathmoveto{\pgfqpoint{3.783400in}{2.759100in}}%
\pgfpathlineto{\pgfqpoint{3.783400in}{2.759100in}}%
\pgfpathlineto{\pgfqpoint{3.783400in}{2.762049in}}%
\pgfpathlineto{\pgfqpoint{3.787941in}{2.762049in}}%
\pgfpathlineto{\pgfqpoint{3.787941in}{2.759100in}}%
\pgfpathmoveto{\pgfqpoint{3.783400in}{2.762049in}}%
\pgfpathlineto{\pgfqpoint{3.783400in}{2.762049in}}%
\pgfpathlineto{\pgfqpoint{3.783400in}{2.764998in}}%
\pgfpathlineto{\pgfqpoint{3.787941in}{2.764998in}}%
\pgfpathlineto{\pgfqpoint{3.787941in}{2.762049in}}%
\pgfpathmoveto{\pgfqpoint{3.787941in}{2.759100in}}%
\pgfpathlineto{\pgfqpoint{3.787941in}{2.759100in}}%
\pgfpathlineto{\pgfqpoint{3.787941in}{2.762049in}}%
\pgfpathlineto{\pgfqpoint{3.792481in}{2.762049in}}%
\pgfpathlineto{\pgfqpoint{3.792481in}{2.759100in}}%
\pgfpathmoveto{\pgfqpoint{3.797022in}{2.756151in}}%
\pgfpathlineto{\pgfqpoint{3.797022in}{2.756151in}}%
\pgfpathlineto{\pgfqpoint{3.797022in}{2.759100in}}%
\pgfpathlineto{\pgfqpoint{3.801563in}{2.759100in}}%
\pgfpathlineto{\pgfqpoint{3.801563in}{2.756151in}}%
\pgfpathmoveto{\pgfqpoint{3.792481in}{2.759100in}}%
\pgfpathlineto{\pgfqpoint{3.792481in}{2.759100in}}%
\pgfpathlineto{\pgfqpoint{3.792481in}{2.762049in}}%
\pgfpathlineto{\pgfqpoint{3.797022in}{2.762049in}}%
\pgfpathlineto{\pgfqpoint{3.797022in}{2.759100in}}%
\pgfpathmoveto{\pgfqpoint{3.797022in}{2.759100in}}%
\pgfpathlineto{\pgfqpoint{3.797022in}{2.759100in}}%
\pgfpathlineto{\pgfqpoint{3.797022in}{2.762049in}}%
\pgfpathlineto{\pgfqpoint{3.801563in}{2.762049in}}%
\pgfpathlineto{\pgfqpoint{3.801563in}{2.759100in}}%
\pgfpathmoveto{\pgfqpoint{3.660795in}{2.785643in}}%
\pgfpathlineto{\pgfqpoint{3.660795in}{2.785643in}}%
\pgfpathlineto{\pgfqpoint{3.660795in}{2.788592in}}%
\pgfpathlineto{\pgfqpoint{3.665336in}{2.788592in}}%
\pgfpathlineto{\pgfqpoint{3.665336in}{2.785643in}}%
\pgfpathmoveto{\pgfqpoint{3.665336in}{2.785643in}}%
\pgfpathlineto{\pgfqpoint{3.665336in}{2.785643in}}%
\pgfpathlineto{\pgfqpoint{3.665336in}{2.788592in}}%
\pgfpathlineto{\pgfqpoint{3.669877in}{2.788592in}}%
\pgfpathlineto{\pgfqpoint{3.669877in}{2.785643in}}%
\pgfpathmoveto{\pgfqpoint{3.669877in}{2.785643in}}%
\pgfpathlineto{\pgfqpoint{3.669877in}{2.785643in}}%
\pgfpathlineto{\pgfqpoint{3.669877in}{2.788592in}}%
\pgfpathlineto{\pgfqpoint{3.674418in}{2.788592in}}%
\pgfpathlineto{\pgfqpoint{3.674418in}{2.785643in}}%
\pgfpathmoveto{\pgfqpoint{3.674418in}{2.782694in}}%
\pgfpathlineto{\pgfqpoint{3.674418in}{2.782694in}}%
\pgfpathlineto{\pgfqpoint{3.674418in}{2.785643in}}%
\pgfpathlineto{\pgfqpoint{3.678959in}{2.785643in}}%
\pgfpathlineto{\pgfqpoint{3.678959in}{2.782694in}}%
\pgfpathmoveto{\pgfqpoint{3.674418in}{2.785643in}}%
\pgfpathlineto{\pgfqpoint{3.674418in}{2.785643in}}%
\pgfpathlineto{\pgfqpoint{3.674418in}{2.788592in}}%
\pgfpathlineto{\pgfqpoint{3.678959in}{2.788592in}}%
\pgfpathlineto{\pgfqpoint{3.678959in}{2.785643in}}%
\pgfpathmoveto{\pgfqpoint{3.678959in}{2.782694in}}%
\pgfpathlineto{\pgfqpoint{3.678959in}{2.782694in}}%
\pgfpathlineto{\pgfqpoint{3.678959in}{2.785643in}}%
\pgfpathlineto{\pgfqpoint{3.683500in}{2.785643in}}%
\pgfpathlineto{\pgfqpoint{3.683500in}{2.782694in}}%
\pgfpathmoveto{\pgfqpoint{3.688041in}{2.779744in}}%
\pgfpathlineto{\pgfqpoint{3.688041in}{2.779744in}}%
\pgfpathlineto{\pgfqpoint{3.688041in}{2.782694in}}%
\pgfpathlineto{\pgfqpoint{3.692581in}{2.782694in}}%
\pgfpathlineto{\pgfqpoint{3.692581in}{2.779744in}}%
\pgfpathmoveto{\pgfqpoint{3.683500in}{2.782694in}}%
\pgfpathlineto{\pgfqpoint{3.683500in}{2.782694in}}%
\pgfpathlineto{\pgfqpoint{3.683500in}{2.785643in}}%
\pgfpathlineto{\pgfqpoint{3.688041in}{2.785643in}}%
\pgfpathlineto{\pgfqpoint{3.688041in}{2.782694in}}%
\pgfpathmoveto{\pgfqpoint{3.688041in}{2.782694in}}%
\pgfpathlineto{\pgfqpoint{3.688041in}{2.782694in}}%
\pgfpathlineto{\pgfqpoint{3.688041in}{2.785643in}}%
\pgfpathlineto{\pgfqpoint{3.692581in}{2.785643in}}%
\pgfpathlineto{\pgfqpoint{3.692581in}{2.782694in}}%
\pgfpathmoveto{\pgfqpoint{3.656254in}{2.788592in}}%
\pgfpathlineto{\pgfqpoint{3.656254in}{2.788592in}}%
\pgfpathlineto{\pgfqpoint{3.656254in}{2.791541in}}%
\pgfpathlineto{\pgfqpoint{3.660795in}{2.791541in}}%
\pgfpathlineto{\pgfqpoint{3.660795in}{2.788592in}}%
\pgfpathmoveto{\pgfqpoint{3.660795in}{2.788592in}}%
\pgfpathlineto{\pgfqpoint{3.660795in}{2.788592in}}%
\pgfpathlineto{\pgfqpoint{3.660795in}{2.791541in}}%
\pgfpathlineto{\pgfqpoint{3.665336in}{2.791541in}}%
\pgfpathlineto{\pgfqpoint{3.665336in}{2.788592in}}%
\pgfpathmoveto{\pgfqpoint{3.692581in}{2.779744in}}%
\pgfpathlineto{\pgfqpoint{3.692581in}{2.779744in}}%
\pgfpathlineto{\pgfqpoint{3.692581in}{2.782694in}}%
\pgfpathlineto{\pgfqpoint{3.697122in}{2.782694in}}%
\pgfpathlineto{\pgfqpoint{3.697122in}{2.779744in}}%
\pgfpathmoveto{\pgfqpoint{3.697122in}{2.779744in}}%
\pgfpathlineto{\pgfqpoint{3.697122in}{2.779744in}}%
\pgfpathlineto{\pgfqpoint{3.697122in}{2.782694in}}%
\pgfpathlineto{\pgfqpoint{3.701663in}{2.782694in}}%
\pgfpathlineto{\pgfqpoint{3.701663in}{2.779744in}}%
\pgfpathmoveto{\pgfqpoint{3.701663in}{2.776795in}}%
\pgfpathlineto{\pgfqpoint{3.701663in}{2.776795in}}%
\pgfpathlineto{\pgfqpoint{3.701663in}{2.779744in}}%
\pgfpathlineto{\pgfqpoint{3.706204in}{2.779744in}}%
\pgfpathlineto{\pgfqpoint{3.706204in}{2.776795in}}%
\pgfpathmoveto{\pgfqpoint{3.701663in}{2.779744in}}%
\pgfpathlineto{\pgfqpoint{3.701663in}{2.779744in}}%
\pgfpathlineto{\pgfqpoint{3.701663in}{2.782694in}}%
\pgfpathlineto{\pgfqpoint{3.706204in}{2.782694in}}%
\pgfpathlineto{\pgfqpoint{3.706204in}{2.779744in}}%
\pgfpathmoveto{\pgfqpoint{3.706204in}{2.776795in}}%
\pgfpathlineto{\pgfqpoint{3.706204in}{2.776795in}}%
\pgfpathlineto{\pgfqpoint{3.706204in}{2.779744in}}%
\pgfpathlineto{\pgfqpoint{3.710745in}{2.779744in}}%
\pgfpathlineto{\pgfqpoint{3.710745in}{2.776795in}}%
\pgfpathmoveto{\pgfqpoint{3.715286in}{2.773846in}}%
\pgfpathlineto{\pgfqpoint{3.715286in}{2.773846in}}%
\pgfpathlineto{\pgfqpoint{3.715286in}{2.776795in}}%
\pgfpathlineto{\pgfqpoint{3.719827in}{2.776795in}}%
\pgfpathlineto{\pgfqpoint{3.719827in}{2.773846in}}%
\pgfpathmoveto{\pgfqpoint{3.719827in}{2.773846in}}%
\pgfpathlineto{\pgfqpoint{3.719827in}{2.773846in}}%
\pgfpathlineto{\pgfqpoint{3.719827in}{2.776795in}}%
\pgfpathlineto{\pgfqpoint{3.724368in}{2.776795in}}%
\pgfpathlineto{\pgfqpoint{3.724368in}{2.773846in}}%
\pgfpathmoveto{\pgfqpoint{3.724368in}{2.773846in}}%
\pgfpathlineto{\pgfqpoint{3.724368in}{2.773846in}}%
\pgfpathlineto{\pgfqpoint{3.724368in}{2.776795in}}%
\pgfpathlineto{\pgfqpoint{3.728909in}{2.776795in}}%
\pgfpathlineto{\pgfqpoint{3.728909in}{2.773846in}}%
\pgfpathmoveto{\pgfqpoint{3.710745in}{2.776795in}}%
\pgfpathlineto{\pgfqpoint{3.710745in}{2.776795in}}%
\pgfpathlineto{\pgfqpoint{3.710745in}{2.779744in}}%
\pgfpathlineto{\pgfqpoint{3.715286in}{2.779744in}}%
\pgfpathlineto{\pgfqpoint{3.715286in}{2.776795in}}%
\pgfpathmoveto{\pgfqpoint{3.715286in}{2.776795in}}%
\pgfpathlineto{\pgfqpoint{3.715286in}{2.776795in}}%
\pgfpathlineto{\pgfqpoint{3.715286in}{2.779744in}}%
\pgfpathlineto{\pgfqpoint{3.719827in}{2.779744in}}%
\pgfpathlineto{\pgfqpoint{3.719827in}{2.776795in}}%
\pgfpathmoveto{\pgfqpoint{3.728909in}{2.770896in}}%
\pgfpathlineto{\pgfqpoint{3.728909in}{2.770896in}}%
\pgfpathlineto{\pgfqpoint{3.728909in}{2.773846in}}%
\pgfpathlineto{\pgfqpoint{3.733450in}{2.773846in}}%
\pgfpathlineto{\pgfqpoint{3.733450in}{2.770896in}}%
\pgfpathmoveto{\pgfqpoint{3.728909in}{2.773846in}}%
\pgfpathlineto{\pgfqpoint{3.728909in}{2.773846in}}%
\pgfpathlineto{\pgfqpoint{3.728909in}{2.776795in}}%
\pgfpathlineto{\pgfqpoint{3.733450in}{2.776795in}}%
\pgfpathlineto{\pgfqpoint{3.733450in}{2.773846in}}%
\pgfpathmoveto{\pgfqpoint{3.733450in}{2.770896in}}%
\pgfpathlineto{\pgfqpoint{3.733450in}{2.770896in}}%
\pgfpathlineto{\pgfqpoint{3.733450in}{2.773846in}}%
\pgfpathlineto{\pgfqpoint{3.737991in}{2.773846in}}%
\pgfpathlineto{\pgfqpoint{3.737991in}{2.770896in}}%
\pgfpathmoveto{\pgfqpoint{3.742531in}{2.767947in}}%
\pgfpathlineto{\pgfqpoint{3.742531in}{2.767947in}}%
\pgfpathlineto{\pgfqpoint{3.742531in}{2.770896in}}%
\pgfpathlineto{\pgfqpoint{3.747072in}{2.770896in}}%
\pgfpathlineto{\pgfqpoint{3.747072in}{2.767947in}}%
\pgfpathmoveto{\pgfqpoint{3.737991in}{2.770896in}}%
\pgfpathlineto{\pgfqpoint{3.737991in}{2.770896in}}%
\pgfpathlineto{\pgfqpoint{3.737991in}{2.773846in}}%
\pgfpathlineto{\pgfqpoint{3.742531in}{2.773846in}}%
\pgfpathlineto{\pgfqpoint{3.742531in}{2.770896in}}%
\pgfpathmoveto{\pgfqpoint{3.742531in}{2.770896in}}%
\pgfpathlineto{\pgfqpoint{3.742531in}{2.770896in}}%
\pgfpathlineto{\pgfqpoint{3.742531in}{2.773846in}}%
\pgfpathlineto{\pgfqpoint{3.747072in}{2.773846in}}%
\pgfpathlineto{\pgfqpoint{3.747072in}{2.770896in}}%
\pgfpathmoveto{\pgfqpoint{3.747072in}{2.767947in}}%
\pgfpathlineto{\pgfqpoint{3.747072in}{2.767947in}}%
\pgfpathlineto{\pgfqpoint{3.747072in}{2.770896in}}%
\pgfpathlineto{\pgfqpoint{3.751613in}{2.770896in}}%
\pgfpathlineto{\pgfqpoint{3.751613in}{2.767947in}}%
\pgfpathmoveto{\pgfqpoint{3.751613in}{2.767947in}}%
\pgfpathlineto{\pgfqpoint{3.751613in}{2.767947in}}%
\pgfpathlineto{\pgfqpoint{3.751613in}{2.770896in}}%
\pgfpathlineto{\pgfqpoint{3.756154in}{2.770896in}}%
\pgfpathlineto{\pgfqpoint{3.756154in}{2.767947in}}%
\pgfpathmoveto{\pgfqpoint{3.756154in}{2.764998in}}%
\pgfpathlineto{\pgfqpoint{3.756154in}{2.764998in}}%
\pgfpathlineto{\pgfqpoint{3.756154in}{2.767947in}}%
\pgfpathlineto{\pgfqpoint{3.760695in}{2.767947in}}%
\pgfpathlineto{\pgfqpoint{3.760695in}{2.764998in}}%
\pgfpathmoveto{\pgfqpoint{3.756154in}{2.767947in}}%
\pgfpathlineto{\pgfqpoint{3.756154in}{2.767947in}}%
\pgfpathlineto{\pgfqpoint{3.756154in}{2.770896in}}%
\pgfpathlineto{\pgfqpoint{3.760695in}{2.770896in}}%
\pgfpathlineto{\pgfqpoint{3.760695in}{2.767947in}}%
\pgfpathmoveto{\pgfqpoint{3.760695in}{2.764998in}}%
\pgfpathlineto{\pgfqpoint{3.760695in}{2.764998in}}%
\pgfpathlineto{\pgfqpoint{3.760695in}{2.767947in}}%
\pgfpathlineto{\pgfqpoint{3.765236in}{2.767947in}}%
\pgfpathlineto{\pgfqpoint{3.765236in}{2.764998in}}%
\pgfpathmoveto{\pgfqpoint{3.765236in}{2.764998in}}%
\pgfpathlineto{\pgfqpoint{3.765236in}{2.764998in}}%
\pgfpathlineto{\pgfqpoint{3.765236in}{2.767947in}}%
\pgfpathlineto{\pgfqpoint{3.769777in}{2.767947in}}%
\pgfpathlineto{\pgfqpoint{3.769777in}{2.764998in}}%
\pgfpathmoveto{\pgfqpoint{3.769777in}{2.764998in}}%
\pgfpathlineto{\pgfqpoint{3.769777in}{2.764998in}}%
\pgfpathlineto{\pgfqpoint{3.769777in}{2.767947in}}%
\pgfpathlineto{\pgfqpoint{3.774318in}{2.767947in}}%
\pgfpathlineto{\pgfqpoint{3.774318in}{2.764998in}}%
\pgfpathmoveto{\pgfqpoint{3.801563in}{2.756151in}}%
\pgfpathlineto{\pgfqpoint{3.801563in}{2.756151in}}%
\pgfpathlineto{\pgfqpoint{3.801563in}{2.759100in}}%
\pgfpathlineto{\pgfqpoint{3.806104in}{2.759100in}}%
\pgfpathlineto{\pgfqpoint{3.806104in}{2.756151in}}%
\pgfpathmoveto{\pgfqpoint{3.806104in}{2.756151in}}%
\pgfpathlineto{\pgfqpoint{3.806104in}{2.756151in}}%
\pgfpathlineto{\pgfqpoint{3.806104in}{2.759100in}}%
\pgfpathlineto{\pgfqpoint{3.810645in}{2.759100in}}%
\pgfpathlineto{\pgfqpoint{3.810645in}{2.756151in}}%
\pgfpathmoveto{\pgfqpoint{3.810645in}{2.753201in}}%
\pgfpathlineto{\pgfqpoint{3.810645in}{2.753201in}}%
\pgfpathlineto{\pgfqpoint{3.810645in}{2.756151in}}%
\pgfpathlineto{\pgfqpoint{3.815186in}{2.756151in}}%
\pgfpathlineto{\pgfqpoint{3.815186in}{2.753201in}}%
\pgfpathmoveto{\pgfqpoint{3.810645in}{2.756151in}}%
\pgfpathlineto{\pgfqpoint{3.810645in}{2.756151in}}%
\pgfpathlineto{\pgfqpoint{3.810645in}{2.759100in}}%
\pgfpathlineto{\pgfqpoint{3.815186in}{2.759100in}}%
\pgfpathlineto{\pgfqpoint{3.815186in}{2.756151in}}%
\pgfpathmoveto{\pgfqpoint{3.815186in}{2.753201in}}%
\pgfpathlineto{\pgfqpoint{3.815186in}{2.753201in}}%
\pgfpathlineto{\pgfqpoint{3.815186in}{2.756151in}}%
\pgfpathlineto{\pgfqpoint{3.819727in}{2.756151in}}%
\pgfpathlineto{\pgfqpoint{3.819727in}{2.753201in}}%
\pgfpathmoveto{\pgfqpoint{3.824268in}{2.750252in}}%
\pgfpathlineto{\pgfqpoint{3.824268in}{2.750252in}}%
\pgfpathlineto{\pgfqpoint{3.824268in}{2.753201in}}%
\pgfpathlineto{\pgfqpoint{3.828809in}{2.753201in}}%
\pgfpathlineto{\pgfqpoint{3.828809in}{2.750252in}}%
\pgfpathmoveto{\pgfqpoint{3.828809in}{2.750252in}}%
\pgfpathlineto{\pgfqpoint{3.828809in}{2.750252in}}%
\pgfpathlineto{\pgfqpoint{3.828809in}{2.753201in}}%
\pgfpathlineto{\pgfqpoint{3.833350in}{2.753201in}}%
\pgfpathlineto{\pgfqpoint{3.833350in}{2.750252in}}%
\pgfpathmoveto{\pgfqpoint{3.833350in}{2.750252in}}%
\pgfpathlineto{\pgfqpoint{3.833350in}{2.750252in}}%
\pgfpathlineto{\pgfqpoint{3.833350in}{2.753201in}}%
\pgfpathlineto{\pgfqpoint{3.837891in}{2.753201in}}%
\pgfpathlineto{\pgfqpoint{3.837891in}{2.750252in}}%
\pgfpathmoveto{\pgfqpoint{3.819727in}{2.753201in}}%
\pgfpathlineto{\pgfqpoint{3.819727in}{2.753201in}}%
\pgfpathlineto{\pgfqpoint{3.819727in}{2.756151in}}%
\pgfpathlineto{\pgfqpoint{3.824268in}{2.756151in}}%
\pgfpathlineto{\pgfqpoint{3.824268in}{2.753201in}}%
\pgfpathmoveto{\pgfqpoint{3.824268in}{2.753201in}}%
\pgfpathlineto{\pgfqpoint{3.824268in}{2.753201in}}%
\pgfpathlineto{\pgfqpoint{3.824268in}{2.756151in}}%
\pgfpathlineto{\pgfqpoint{3.828809in}{2.756151in}}%
\pgfpathlineto{\pgfqpoint{3.828809in}{2.753201in}}%
\pgfpathmoveto{\pgfqpoint{3.837891in}{2.747303in}}%
\pgfpathlineto{\pgfqpoint{3.837891in}{2.747303in}}%
\pgfpathlineto{\pgfqpoint{3.837891in}{2.750252in}}%
\pgfpathlineto{\pgfqpoint{3.842432in}{2.750252in}}%
\pgfpathlineto{\pgfqpoint{3.842432in}{2.747303in}}%
\pgfpathmoveto{\pgfqpoint{3.837891in}{2.750252in}}%
\pgfpathlineto{\pgfqpoint{3.837891in}{2.750252in}}%
\pgfpathlineto{\pgfqpoint{3.837891in}{2.753201in}}%
\pgfpathlineto{\pgfqpoint{3.842432in}{2.753201in}}%
\pgfpathlineto{\pgfqpoint{3.842432in}{2.750252in}}%
\pgfpathmoveto{\pgfqpoint{3.842432in}{2.747303in}}%
\pgfpathlineto{\pgfqpoint{3.842432in}{2.747303in}}%
\pgfpathlineto{\pgfqpoint{3.842432in}{2.750252in}}%
\pgfpathlineto{\pgfqpoint{3.846973in}{2.750252in}}%
\pgfpathlineto{\pgfqpoint{3.846973in}{2.747303in}}%
\pgfpathmoveto{\pgfqpoint{3.851515in}{2.744354in}}%
\pgfpathlineto{\pgfqpoint{3.851515in}{2.744354in}}%
\pgfpathlineto{\pgfqpoint{3.851515in}{2.747303in}}%
\pgfpathlineto{\pgfqpoint{3.856056in}{2.747303in}}%
\pgfpathlineto{\pgfqpoint{3.856056in}{2.744354in}}%
\pgfpathmoveto{\pgfqpoint{3.846973in}{2.747303in}}%
\pgfpathlineto{\pgfqpoint{3.846973in}{2.747303in}}%
\pgfpathlineto{\pgfqpoint{3.846973in}{2.750252in}}%
\pgfpathlineto{\pgfqpoint{3.851515in}{2.750252in}}%
\pgfpathlineto{\pgfqpoint{3.851515in}{2.747303in}}%
\pgfpathmoveto{\pgfqpoint{3.851515in}{2.747303in}}%
\pgfpathlineto{\pgfqpoint{3.851515in}{2.747303in}}%
\pgfpathlineto{\pgfqpoint{3.851515in}{2.750252in}}%
\pgfpathlineto{\pgfqpoint{3.856056in}{2.750252in}}%
\pgfpathlineto{\pgfqpoint{3.856056in}{2.747303in}}%
\pgfpathmoveto{\pgfqpoint{3.856056in}{2.744354in}}%
\pgfpathlineto{\pgfqpoint{3.856056in}{2.744354in}}%
\pgfpathlineto{\pgfqpoint{3.856056in}{2.747303in}}%
\pgfpathlineto{\pgfqpoint{3.860597in}{2.747303in}}%
\pgfpathlineto{\pgfqpoint{3.860597in}{2.744354in}}%
\pgfpathmoveto{\pgfqpoint{3.860597in}{2.744354in}}%
\pgfpathlineto{\pgfqpoint{3.860597in}{2.744354in}}%
\pgfpathlineto{\pgfqpoint{3.860597in}{2.747303in}}%
\pgfpathlineto{\pgfqpoint{3.865138in}{2.747303in}}%
\pgfpathlineto{\pgfqpoint{3.865138in}{2.744354in}}%
\pgfpathmoveto{\pgfqpoint{3.865138in}{2.741405in}}%
\pgfpathlineto{\pgfqpoint{3.865138in}{2.741405in}}%
\pgfpathlineto{\pgfqpoint{3.865138in}{2.744354in}}%
\pgfpathlineto{\pgfqpoint{3.869679in}{2.744354in}}%
\pgfpathlineto{\pgfqpoint{3.869679in}{2.741405in}}%
\pgfpathmoveto{\pgfqpoint{3.865138in}{2.744354in}}%
\pgfpathlineto{\pgfqpoint{3.865138in}{2.744354in}}%
\pgfpathlineto{\pgfqpoint{3.865138in}{2.747303in}}%
\pgfpathlineto{\pgfqpoint{3.869679in}{2.747303in}}%
\pgfpathlineto{\pgfqpoint{3.869679in}{2.744354in}}%
\pgfpathmoveto{\pgfqpoint{3.869679in}{2.741405in}}%
\pgfpathlineto{\pgfqpoint{3.869679in}{2.741405in}}%
\pgfpathlineto{\pgfqpoint{3.869679in}{2.744354in}}%
\pgfpathlineto{\pgfqpoint{3.874220in}{2.744354in}}%
\pgfpathlineto{\pgfqpoint{3.874220in}{2.741405in}}%
\pgfpathmoveto{\pgfqpoint{3.878761in}{2.738456in}}%
\pgfpathlineto{\pgfqpoint{3.878761in}{2.738456in}}%
\pgfpathlineto{\pgfqpoint{3.878761in}{2.741405in}}%
\pgfpathlineto{\pgfqpoint{3.883302in}{2.741405in}}%
\pgfpathlineto{\pgfqpoint{3.883302in}{2.738456in}}%
\pgfpathmoveto{\pgfqpoint{3.883302in}{2.738456in}}%
\pgfpathlineto{\pgfqpoint{3.883302in}{2.738456in}}%
\pgfpathlineto{\pgfqpoint{3.883302in}{2.741405in}}%
\pgfpathlineto{\pgfqpoint{3.887843in}{2.741405in}}%
\pgfpathlineto{\pgfqpoint{3.887843in}{2.738456in}}%
\pgfpathmoveto{\pgfqpoint{3.887843in}{2.738456in}}%
\pgfpathlineto{\pgfqpoint{3.887843in}{2.738456in}}%
\pgfpathlineto{\pgfqpoint{3.887843in}{2.741405in}}%
\pgfpathlineto{\pgfqpoint{3.892384in}{2.741405in}}%
\pgfpathlineto{\pgfqpoint{3.892384in}{2.738456in}}%
\pgfpathmoveto{\pgfqpoint{3.892384in}{2.735507in}}%
\pgfpathlineto{\pgfqpoint{3.892384in}{2.735507in}}%
\pgfpathlineto{\pgfqpoint{3.892384in}{2.738456in}}%
\pgfpathlineto{\pgfqpoint{3.896925in}{2.738456in}}%
\pgfpathlineto{\pgfqpoint{3.896925in}{2.735507in}}%
\pgfpathmoveto{\pgfqpoint{3.892384in}{2.738456in}}%
\pgfpathlineto{\pgfqpoint{3.892384in}{2.738456in}}%
\pgfpathlineto{\pgfqpoint{3.892384in}{2.741405in}}%
\pgfpathlineto{\pgfqpoint{3.896925in}{2.741405in}}%
\pgfpathlineto{\pgfqpoint{3.896925in}{2.738456in}}%
\pgfpathmoveto{\pgfqpoint{3.896925in}{2.735507in}}%
\pgfpathlineto{\pgfqpoint{3.896925in}{2.735507in}}%
\pgfpathlineto{\pgfqpoint{3.896925in}{2.738456in}}%
\pgfpathlineto{\pgfqpoint{3.901466in}{2.738456in}}%
\pgfpathlineto{\pgfqpoint{3.901466in}{2.735507in}}%
\pgfpathmoveto{\pgfqpoint{3.906007in}{2.732557in}}%
\pgfpathlineto{\pgfqpoint{3.906007in}{2.732557in}}%
\pgfpathlineto{\pgfqpoint{3.906007in}{2.735507in}}%
\pgfpathlineto{\pgfqpoint{3.910548in}{2.735507in}}%
\pgfpathlineto{\pgfqpoint{3.910548in}{2.732557in}}%
\pgfpathmoveto{\pgfqpoint{3.901466in}{2.735507in}}%
\pgfpathlineto{\pgfqpoint{3.901466in}{2.735507in}}%
\pgfpathlineto{\pgfqpoint{3.901466in}{2.738456in}}%
\pgfpathlineto{\pgfqpoint{3.906007in}{2.738456in}}%
\pgfpathlineto{\pgfqpoint{3.906007in}{2.735507in}}%
\pgfpathmoveto{\pgfqpoint{3.906007in}{2.735507in}}%
\pgfpathlineto{\pgfqpoint{3.906007in}{2.735507in}}%
\pgfpathlineto{\pgfqpoint{3.906007in}{2.738456in}}%
\pgfpathlineto{\pgfqpoint{3.910548in}{2.738456in}}%
\pgfpathlineto{\pgfqpoint{3.910548in}{2.735507in}}%
\pgfpathmoveto{\pgfqpoint{3.874220in}{2.741405in}}%
\pgfpathlineto{\pgfqpoint{3.874220in}{2.741405in}}%
\pgfpathlineto{\pgfqpoint{3.874220in}{2.744354in}}%
\pgfpathlineto{\pgfqpoint{3.878761in}{2.744354in}}%
\pgfpathlineto{\pgfqpoint{3.878761in}{2.741405in}}%
\pgfpathmoveto{\pgfqpoint{3.878761in}{2.741405in}}%
\pgfpathlineto{\pgfqpoint{3.878761in}{2.741405in}}%
\pgfpathlineto{\pgfqpoint{3.878761in}{2.744354in}}%
\pgfpathlineto{\pgfqpoint{3.883302in}{2.744354in}}%
\pgfpathlineto{\pgfqpoint{3.883302in}{2.741405in}}%
\pgfpathmoveto{\pgfqpoint{3.910548in}{2.732557in}}%
\pgfpathlineto{\pgfqpoint{3.910548in}{2.732557in}}%
\pgfpathlineto{\pgfqpoint{3.910548in}{2.735507in}}%
\pgfpathlineto{\pgfqpoint{3.915089in}{2.735507in}}%
\pgfpathlineto{\pgfqpoint{3.915089in}{2.732557in}}%
\pgfpathmoveto{\pgfqpoint{3.915089in}{2.732557in}}%
\pgfpathlineto{\pgfqpoint{3.915089in}{2.732557in}}%
\pgfpathlineto{\pgfqpoint{3.915089in}{2.735507in}}%
\pgfpathlineto{\pgfqpoint{3.919630in}{2.735507in}}%
\pgfpathlineto{\pgfqpoint{3.919630in}{2.732557in}}%
\pgfpathmoveto{\pgfqpoint{3.919630in}{2.729608in}}%
\pgfpathlineto{\pgfqpoint{3.919630in}{2.729608in}}%
\pgfpathlineto{\pgfqpoint{3.919630in}{2.732557in}}%
\pgfpathlineto{\pgfqpoint{3.924171in}{2.732557in}}%
\pgfpathlineto{\pgfqpoint{3.924171in}{2.729608in}}%
\pgfpathmoveto{\pgfqpoint{3.919630in}{2.732557in}}%
\pgfpathlineto{\pgfqpoint{3.919630in}{2.732557in}}%
\pgfpathlineto{\pgfqpoint{3.919630in}{2.735507in}}%
\pgfpathlineto{\pgfqpoint{3.924171in}{2.735507in}}%
\pgfpathlineto{\pgfqpoint{3.924171in}{2.732557in}}%
\pgfpathmoveto{\pgfqpoint{3.924171in}{2.729608in}}%
\pgfpathlineto{\pgfqpoint{3.924171in}{2.729608in}}%
\pgfpathlineto{\pgfqpoint{3.924171in}{2.732557in}}%
\pgfpathlineto{\pgfqpoint{3.928712in}{2.732557in}}%
\pgfpathlineto{\pgfqpoint{3.928712in}{2.729608in}}%
\pgfpathmoveto{\pgfqpoint{3.933253in}{2.726659in}}%
\pgfpathlineto{\pgfqpoint{3.933253in}{2.726659in}}%
\pgfpathlineto{\pgfqpoint{3.933253in}{2.729608in}}%
\pgfpathlineto{\pgfqpoint{3.937794in}{2.729608in}}%
\pgfpathlineto{\pgfqpoint{3.937794in}{2.726659in}}%
\pgfpathmoveto{\pgfqpoint{3.937794in}{2.726659in}}%
\pgfpathlineto{\pgfqpoint{3.937794in}{2.726659in}}%
\pgfpathlineto{\pgfqpoint{3.937794in}{2.729608in}}%
\pgfpathlineto{\pgfqpoint{3.942335in}{2.729608in}}%
\pgfpathlineto{\pgfqpoint{3.942335in}{2.726659in}}%
\pgfpathmoveto{\pgfqpoint{3.942335in}{2.726659in}}%
\pgfpathlineto{\pgfqpoint{3.942335in}{2.726659in}}%
\pgfpathlineto{\pgfqpoint{3.942335in}{2.729608in}}%
\pgfpathlineto{\pgfqpoint{3.946876in}{2.729608in}}%
\pgfpathlineto{\pgfqpoint{3.946876in}{2.726659in}}%
\pgfpathmoveto{\pgfqpoint{3.928712in}{2.729608in}}%
\pgfpathlineto{\pgfqpoint{3.928712in}{2.729608in}}%
\pgfpathlineto{\pgfqpoint{3.928712in}{2.732557in}}%
\pgfpathlineto{\pgfqpoint{3.933253in}{2.732557in}}%
\pgfpathlineto{\pgfqpoint{3.933253in}{2.729608in}}%
\pgfpathmoveto{\pgfqpoint{3.933253in}{2.729608in}}%
\pgfpathlineto{\pgfqpoint{3.933253in}{2.729608in}}%
\pgfpathlineto{\pgfqpoint{3.933253in}{2.732557in}}%
\pgfpathlineto{\pgfqpoint{3.937794in}{2.732557in}}%
\pgfpathlineto{\pgfqpoint{3.937794in}{2.729608in}}%
\pgfpathmoveto{\pgfqpoint{3.987745in}{2.714863in}}%
\pgfpathlineto{\pgfqpoint{3.987745in}{2.714863in}}%
\pgfpathlineto{\pgfqpoint{3.987745in}{2.717812in}}%
\pgfpathlineto{\pgfqpoint{3.992286in}{2.717812in}}%
\pgfpathlineto{\pgfqpoint{3.992286in}{2.714863in}}%
\pgfpathmoveto{\pgfqpoint{3.992286in}{2.714863in}}%
\pgfpathlineto{\pgfqpoint{3.992286in}{2.714863in}}%
\pgfpathlineto{\pgfqpoint{3.992286in}{2.717812in}}%
\pgfpathlineto{\pgfqpoint{3.996827in}{2.717812in}}%
\pgfpathlineto{\pgfqpoint{3.996827in}{2.714863in}}%
\pgfpathmoveto{\pgfqpoint{3.996827in}{2.714863in}}%
\pgfpathlineto{\pgfqpoint{3.996827in}{2.714863in}}%
\pgfpathlineto{\pgfqpoint{3.996827in}{2.717812in}}%
\pgfpathlineto{\pgfqpoint{4.001368in}{2.717812in}}%
\pgfpathlineto{\pgfqpoint{4.001368in}{2.714863in}}%
\pgfpathmoveto{\pgfqpoint{4.001368in}{2.711914in}}%
\pgfpathlineto{\pgfqpoint{4.001368in}{2.711914in}}%
\pgfpathlineto{\pgfqpoint{4.001368in}{2.714863in}}%
\pgfpathlineto{\pgfqpoint{4.005908in}{2.714863in}}%
\pgfpathlineto{\pgfqpoint{4.005908in}{2.711914in}}%
\pgfpathmoveto{\pgfqpoint{4.001368in}{2.714863in}}%
\pgfpathlineto{\pgfqpoint{4.001368in}{2.714863in}}%
\pgfpathlineto{\pgfqpoint{4.001368in}{2.717812in}}%
\pgfpathlineto{\pgfqpoint{4.005908in}{2.717812in}}%
\pgfpathlineto{\pgfqpoint{4.005908in}{2.714863in}}%
\pgfpathmoveto{\pgfqpoint{4.005908in}{2.711914in}}%
\pgfpathlineto{\pgfqpoint{4.005908in}{2.711914in}}%
\pgfpathlineto{\pgfqpoint{4.005908in}{2.714863in}}%
\pgfpathlineto{\pgfqpoint{4.010449in}{2.714863in}}%
\pgfpathlineto{\pgfqpoint{4.010449in}{2.711914in}}%
\pgfpathmoveto{\pgfqpoint{4.014990in}{2.708964in}}%
\pgfpathlineto{\pgfqpoint{4.014990in}{2.708964in}}%
\pgfpathlineto{\pgfqpoint{4.014990in}{2.711914in}}%
\pgfpathlineto{\pgfqpoint{4.019531in}{2.711914in}}%
\pgfpathlineto{\pgfqpoint{4.019531in}{2.708964in}}%
\pgfpathmoveto{\pgfqpoint{4.010449in}{2.711914in}}%
\pgfpathlineto{\pgfqpoint{4.010449in}{2.711914in}}%
\pgfpathlineto{\pgfqpoint{4.010449in}{2.714863in}}%
\pgfpathlineto{\pgfqpoint{4.014990in}{2.714863in}}%
\pgfpathlineto{\pgfqpoint{4.014990in}{2.711914in}}%
\pgfpathmoveto{\pgfqpoint{4.014990in}{2.711914in}}%
\pgfpathlineto{\pgfqpoint{4.014990in}{2.711914in}}%
\pgfpathlineto{\pgfqpoint{4.014990in}{2.714863in}}%
\pgfpathlineto{\pgfqpoint{4.019531in}{2.714863in}}%
\pgfpathlineto{\pgfqpoint{4.019531in}{2.711914in}}%
\pgfpathmoveto{\pgfqpoint{3.946876in}{2.723710in}}%
\pgfpathlineto{\pgfqpoint{3.946876in}{2.723710in}}%
\pgfpathlineto{\pgfqpoint{3.946876in}{2.726659in}}%
\pgfpathlineto{\pgfqpoint{3.951417in}{2.726659in}}%
\pgfpathlineto{\pgfqpoint{3.951417in}{2.723710in}}%
\pgfpathmoveto{\pgfqpoint{3.946876in}{2.726659in}}%
\pgfpathlineto{\pgfqpoint{3.946876in}{2.726659in}}%
\pgfpathlineto{\pgfqpoint{3.946876in}{2.729608in}}%
\pgfpathlineto{\pgfqpoint{3.951417in}{2.729608in}}%
\pgfpathlineto{\pgfqpoint{3.951417in}{2.726659in}}%
\pgfpathmoveto{\pgfqpoint{3.951417in}{2.723710in}}%
\pgfpathlineto{\pgfqpoint{3.951417in}{2.723710in}}%
\pgfpathlineto{\pgfqpoint{3.951417in}{2.726659in}}%
\pgfpathlineto{\pgfqpoint{3.955958in}{2.726659in}}%
\pgfpathlineto{\pgfqpoint{3.955958in}{2.723710in}}%
\pgfpathmoveto{\pgfqpoint{3.960499in}{2.720761in}}%
\pgfpathlineto{\pgfqpoint{3.960499in}{2.720761in}}%
\pgfpathlineto{\pgfqpoint{3.960499in}{2.723710in}}%
\pgfpathlineto{\pgfqpoint{3.965040in}{2.723710in}}%
\pgfpathlineto{\pgfqpoint{3.965040in}{2.720761in}}%
\pgfpathmoveto{\pgfqpoint{3.955958in}{2.723710in}}%
\pgfpathlineto{\pgfqpoint{3.955958in}{2.723710in}}%
\pgfpathlineto{\pgfqpoint{3.955958in}{2.726659in}}%
\pgfpathlineto{\pgfqpoint{3.960499in}{2.726659in}}%
\pgfpathlineto{\pgfqpoint{3.960499in}{2.723710in}}%
\pgfpathmoveto{\pgfqpoint{3.960499in}{2.723710in}}%
\pgfpathlineto{\pgfqpoint{3.960499in}{2.723710in}}%
\pgfpathlineto{\pgfqpoint{3.960499in}{2.726659in}}%
\pgfpathlineto{\pgfqpoint{3.965040in}{2.726659in}}%
\pgfpathlineto{\pgfqpoint{3.965040in}{2.723710in}}%
\pgfpathmoveto{\pgfqpoint{3.965040in}{2.720761in}}%
\pgfpathlineto{\pgfqpoint{3.965040in}{2.720761in}}%
\pgfpathlineto{\pgfqpoint{3.965040in}{2.723710in}}%
\pgfpathlineto{\pgfqpoint{3.969581in}{2.723710in}}%
\pgfpathlineto{\pgfqpoint{3.969581in}{2.720761in}}%
\pgfpathmoveto{\pgfqpoint{3.969581in}{2.720761in}}%
\pgfpathlineto{\pgfqpoint{3.969581in}{2.720761in}}%
\pgfpathlineto{\pgfqpoint{3.969581in}{2.723710in}}%
\pgfpathlineto{\pgfqpoint{3.974122in}{2.723710in}}%
\pgfpathlineto{\pgfqpoint{3.974122in}{2.720761in}}%
\pgfpathmoveto{\pgfqpoint{3.974122in}{2.717812in}}%
\pgfpathlineto{\pgfqpoint{3.974122in}{2.717812in}}%
\pgfpathlineto{\pgfqpoint{3.974122in}{2.720761in}}%
\pgfpathlineto{\pgfqpoint{3.978663in}{2.720761in}}%
\pgfpathlineto{\pgfqpoint{3.978663in}{2.717812in}}%
\pgfpathmoveto{\pgfqpoint{3.974122in}{2.720761in}}%
\pgfpathlineto{\pgfqpoint{3.974122in}{2.720761in}}%
\pgfpathlineto{\pgfqpoint{3.974122in}{2.723710in}}%
\pgfpathlineto{\pgfqpoint{3.978663in}{2.723710in}}%
\pgfpathlineto{\pgfqpoint{3.978663in}{2.720761in}}%
\pgfpathmoveto{\pgfqpoint{3.978663in}{2.717812in}}%
\pgfpathlineto{\pgfqpoint{3.978663in}{2.717812in}}%
\pgfpathlineto{\pgfqpoint{3.978663in}{2.720761in}}%
\pgfpathlineto{\pgfqpoint{3.983204in}{2.720761in}}%
\pgfpathlineto{\pgfqpoint{3.983204in}{2.717812in}}%
\pgfpathmoveto{\pgfqpoint{3.983204in}{2.717812in}}%
\pgfpathlineto{\pgfqpoint{3.983204in}{2.717812in}}%
\pgfpathlineto{\pgfqpoint{3.983204in}{2.720761in}}%
\pgfpathlineto{\pgfqpoint{3.987745in}{2.720761in}}%
\pgfpathlineto{\pgfqpoint{3.987745in}{2.717812in}}%
\pgfpathmoveto{\pgfqpoint{3.987745in}{2.717812in}}%
\pgfpathlineto{\pgfqpoint{3.987745in}{2.717812in}}%
\pgfpathlineto{\pgfqpoint{3.987745in}{2.720761in}}%
\pgfpathlineto{\pgfqpoint{3.992286in}{2.720761in}}%
\pgfpathlineto{\pgfqpoint{3.992286in}{2.717812in}}%
\pgfpathmoveto{\pgfqpoint{4.019531in}{2.708964in}}%
\pgfpathlineto{\pgfqpoint{4.019531in}{2.708964in}}%
\pgfpathlineto{\pgfqpoint{4.019531in}{2.711914in}}%
\pgfpathlineto{\pgfqpoint{4.024072in}{2.711914in}}%
\pgfpathlineto{\pgfqpoint{4.024072in}{2.708964in}}%
\pgfpathmoveto{\pgfqpoint{4.024072in}{2.708964in}}%
\pgfpathlineto{\pgfqpoint{4.024072in}{2.708964in}}%
\pgfpathlineto{\pgfqpoint{4.024072in}{2.711914in}}%
\pgfpathlineto{\pgfqpoint{4.028613in}{2.711914in}}%
\pgfpathlineto{\pgfqpoint{4.028613in}{2.708964in}}%
\pgfpathmoveto{\pgfqpoint{4.028613in}{2.706015in}}%
\pgfpathlineto{\pgfqpoint{4.028613in}{2.706015in}}%
\pgfpathlineto{\pgfqpoint{4.028613in}{2.708964in}}%
\pgfpathlineto{\pgfqpoint{4.033154in}{2.708964in}}%
\pgfpathlineto{\pgfqpoint{4.033154in}{2.706015in}}%
\pgfpathmoveto{\pgfqpoint{4.028613in}{2.708964in}}%
\pgfpathlineto{\pgfqpoint{4.028613in}{2.708964in}}%
\pgfpathlineto{\pgfqpoint{4.028613in}{2.711914in}}%
\pgfpathlineto{\pgfqpoint{4.033154in}{2.711914in}}%
\pgfpathlineto{\pgfqpoint{4.033154in}{2.708964in}}%
\pgfpathmoveto{\pgfqpoint{4.033154in}{2.706015in}}%
\pgfpathlineto{\pgfqpoint{4.033154in}{2.706015in}}%
\pgfpathlineto{\pgfqpoint{4.033154in}{2.708964in}}%
\pgfpathlineto{\pgfqpoint{4.037695in}{2.708964in}}%
\pgfpathlineto{\pgfqpoint{4.037695in}{2.706015in}}%
\pgfpathmoveto{\pgfqpoint{4.042236in}{2.703066in}}%
\pgfpathlineto{\pgfqpoint{4.042236in}{2.703066in}}%
\pgfpathlineto{\pgfqpoint{4.042236in}{2.706015in}}%
\pgfpathlineto{\pgfqpoint{4.046777in}{2.706015in}}%
\pgfpathlineto{\pgfqpoint{4.046777in}{2.703066in}}%
\pgfpathmoveto{\pgfqpoint{4.046777in}{2.703066in}}%
\pgfpathlineto{\pgfqpoint{4.046777in}{2.703066in}}%
\pgfpathlineto{\pgfqpoint{4.046777in}{2.706015in}}%
\pgfpathlineto{\pgfqpoint{4.051318in}{2.706015in}}%
\pgfpathlineto{\pgfqpoint{4.051318in}{2.703066in}}%
\pgfpathmoveto{\pgfqpoint{4.051318in}{2.703066in}}%
\pgfpathlineto{\pgfqpoint{4.051318in}{2.703066in}}%
\pgfpathlineto{\pgfqpoint{4.051318in}{2.706015in}}%
\pgfpathlineto{\pgfqpoint{4.055859in}{2.706015in}}%
\pgfpathlineto{\pgfqpoint{4.055859in}{2.703066in}}%
\pgfpathmoveto{\pgfqpoint{4.037695in}{2.706015in}}%
\pgfpathlineto{\pgfqpoint{4.037695in}{2.706015in}}%
\pgfpathlineto{\pgfqpoint{4.037695in}{2.708964in}}%
\pgfpathlineto{\pgfqpoint{4.042236in}{2.708964in}}%
\pgfpathlineto{\pgfqpoint{4.042236in}{2.706015in}}%
\pgfpathmoveto{\pgfqpoint{4.042236in}{2.706015in}}%
\pgfpathlineto{\pgfqpoint{4.042236in}{2.706015in}}%
\pgfpathlineto{\pgfqpoint{4.042236in}{2.708964in}}%
\pgfpathlineto{\pgfqpoint{4.046777in}{2.708964in}}%
\pgfpathlineto{\pgfqpoint{4.046777in}{2.706015in}}%
\pgfpathmoveto{\pgfqpoint{4.055859in}{2.700117in}}%
\pgfpathlineto{\pgfqpoint{4.055859in}{2.700117in}}%
\pgfpathlineto{\pgfqpoint{4.055859in}{2.703066in}}%
\pgfpathlineto{\pgfqpoint{4.060400in}{2.703066in}}%
\pgfpathlineto{\pgfqpoint{4.060400in}{2.700117in}}%
\pgfpathmoveto{\pgfqpoint{4.055859in}{2.703066in}}%
\pgfpathlineto{\pgfqpoint{4.055859in}{2.703066in}}%
\pgfpathlineto{\pgfqpoint{4.055859in}{2.706015in}}%
\pgfpathlineto{\pgfqpoint{4.060400in}{2.706015in}}%
\pgfpathlineto{\pgfqpoint{4.060400in}{2.703066in}}%
\pgfpathmoveto{\pgfqpoint{4.060400in}{2.700117in}}%
\pgfpathlineto{\pgfqpoint{4.060400in}{2.700117in}}%
\pgfpathlineto{\pgfqpoint{4.060400in}{2.703066in}}%
\pgfpathlineto{\pgfqpoint{4.064941in}{2.703066in}}%
\pgfpathlineto{\pgfqpoint{4.064941in}{2.700117in}}%
\pgfpathmoveto{\pgfqpoint{4.069482in}{2.697168in}}%
\pgfpathlineto{\pgfqpoint{4.069482in}{2.697168in}}%
\pgfpathlineto{\pgfqpoint{4.069482in}{2.700117in}}%
\pgfpathlineto{\pgfqpoint{4.074023in}{2.700117in}}%
\pgfpathlineto{\pgfqpoint{4.074023in}{2.697168in}}%
\pgfpathmoveto{\pgfqpoint{4.064941in}{2.700117in}}%
\pgfpathlineto{\pgfqpoint{4.064941in}{2.700117in}}%
\pgfpathlineto{\pgfqpoint{4.064941in}{2.703066in}}%
\pgfpathlineto{\pgfqpoint{4.069482in}{2.703066in}}%
\pgfpathlineto{\pgfqpoint{4.069482in}{2.700117in}}%
\pgfpathmoveto{\pgfqpoint{4.069482in}{2.700117in}}%
\pgfpathlineto{\pgfqpoint{4.069482in}{2.700117in}}%
\pgfpathlineto{\pgfqpoint{4.069482in}{2.703066in}}%
\pgfpathlineto{\pgfqpoint{4.074023in}{2.703066in}}%
\pgfpathlineto{\pgfqpoint{4.074023in}{2.700117in}}%
\pgfpathmoveto{\pgfqpoint{4.074023in}{2.697168in}}%
\pgfpathlineto{\pgfqpoint{4.074023in}{2.697168in}}%
\pgfpathlineto{\pgfqpoint{4.074023in}{2.700117in}}%
\pgfpathlineto{\pgfqpoint{4.078564in}{2.700117in}}%
\pgfpathlineto{\pgfqpoint{4.078564in}{2.697168in}}%
\pgfpathmoveto{\pgfqpoint{4.078564in}{2.697168in}}%
\pgfpathlineto{\pgfqpoint{4.078564in}{2.697168in}}%
\pgfpathlineto{\pgfqpoint{4.078564in}{2.700117in}}%
\pgfpathlineto{\pgfqpoint{4.083105in}{2.700117in}}%
\pgfpathlineto{\pgfqpoint{4.083105in}{2.697168in}}%
\pgfpathmoveto{\pgfqpoint{4.083105in}{2.694219in}}%
\pgfpathlineto{\pgfqpoint{4.083105in}{2.694219in}}%
\pgfpathlineto{\pgfqpoint{4.083105in}{2.697168in}}%
\pgfpathlineto{\pgfqpoint{4.087646in}{2.697168in}}%
\pgfpathlineto{\pgfqpoint{4.087646in}{2.694219in}}%
\pgfpathmoveto{\pgfqpoint{4.083105in}{2.697168in}}%
\pgfpathlineto{\pgfqpoint{4.083105in}{2.697168in}}%
\pgfpathlineto{\pgfqpoint{4.083105in}{2.700117in}}%
\pgfpathlineto{\pgfqpoint{4.087646in}{2.700117in}}%
\pgfpathlineto{\pgfqpoint{4.087646in}{2.697168in}}%
\pgfpathmoveto{\pgfqpoint{4.087646in}{2.694219in}}%
\pgfpathlineto{\pgfqpoint{4.087646in}{2.694219in}}%
\pgfpathlineto{\pgfqpoint{4.087646in}{2.697168in}}%
\pgfpathlineto{\pgfqpoint{4.092187in}{2.697168in}}%
\pgfpathlineto{\pgfqpoint{4.092187in}{2.694219in}}%
\pgfpathmoveto{\pgfqpoint{4.205715in}{2.667676in}}%
\pgfpathlineto{\pgfqpoint{4.205715in}{2.667676in}}%
\pgfpathlineto{\pgfqpoint{4.205715in}{2.670626in}}%
\pgfpathlineto{\pgfqpoint{4.210256in}{2.670626in}}%
\pgfpathlineto{\pgfqpoint{4.210256in}{2.667676in}}%
\pgfpathmoveto{\pgfqpoint{4.210256in}{2.667676in}}%
\pgfpathlineto{\pgfqpoint{4.210256in}{2.667676in}}%
\pgfpathlineto{\pgfqpoint{4.210256in}{2.670626in}}%
\pgfpathlineto{\pgfqpoint{4.214797in}{2.670626in}}%
\pgfpathlineto{\pgfqpoint{4.214797in}{2.667676in}}%
\pgfpathmoveto{\pgfqpoint{4.214797in}{2.667676in}}%
\pgfpathlineto{\pgfqpoint{4.214797in}{2.667676in}}%
\pgfpathlineto{\pgfqpoint{4.214797in}{2.670626in}}%
\pgfpathlineto{\pgfqpoint{4.219338in}{2.670626in}}%
\pgfpathlineto{\pgfqpoint{4.219338in}{2.667676in}}%
\pgfpathmoveto{\pgfqpoint{4.219338in}{2.664727in}}%
\pgfpathlineto{\pgfqpoint{4.219338in}{2.664727in}}%
\pgfpathlineto{\pgfqpoint{4.219338in}{2.667676in}}%
\pgfpathlineto{\pgfqpoint{4.223879in}{2.667676in}}%
\pgfpathlineto{\pgfqpoint{4.223879in}{2.664727in}}%
\pgfpathmoveto{\pgfqpoint{4.219338in}{2.667676in}}%
\pgfpathlineto{\pgfqpoint{4.219338in}{2.667676in}}%
\pgfpathlineto{\pgfqpoint{4.219338in}{2.670626in}}%
\pgfpathlineto{\pgfqpoint{4.223879in}{2.670626in}}%
\pgfpathlineto{\pgfqpoint{4.223879in}{2.667676in}}%
\pgfpathmoveto{\pgfqpoint{4.223879in}{2.664727in}}%
\pgfpathlineto{\pgfqpoint{4.223879in}{2.664727in}}%
\pgfpathlineto{\pgfqpoint{4.223879in}{2.667676in}}%
\pgfpathlineto{\pgfqpoint{4.228421in}{2.667676in}}%
\pgfpathlineto{\pgfqpoint{4.228421in}{2.664727in}}%
\pgfpathmoveto{\pgfqpoint{4.232962in}{2.661778in}}%
\pgfpathlineto{\pgfqpoint{4.232962in}{2.661778in}}%
\pgfpathlineto{\pgfqpoint{4.232962in}{2.664727in}}%
\pgfpathlineto{\pgfqpoint{4.237503in}{2.664727in}}%
\pgfpathlineto{\pgfqpoint{4.237503in}{2.661778in}}%
\pgfpathmoveto{\pgfqpoint{4.228421in}{2.664727in}}%
\pgfpathlineto{\pgfqpoint{4.228421in}{2.664727in}}%
\pgfpathlineto{\pgfqpoint{4.228421in}{2.667676in}}%
\pgfpathlineto{\pgfqpoint{4.232962in}{2.667676in}}%
\pgfpathlineto{\pgfqpoint{4.232962in}{2.664727in}}%
\pgfpathmoveto{\pgfqpoint{4.232962in}{2.664727in}}%
\pgfpathlineto{\pgfqpoint{4.232962in}{2.664727in}}%
\pgfpathlineto{\pgfqpoint{4.232962in}{2.667676in}}%
\pgfpathlineto{\pgfqpoint{4.237503in}{2.667676in}}%
\pgfpathlineto{\pgfqpoint{4.237503in}{2.664727in}}%
\pgfpathmoveto{\pgfqpoint{4.096728in}{2.691270in}}%
\pgfpathlineto{\pgfqpoint{4.096728in}{2.691270in}}%
\pgfpathlineto{\pgfqpoint{4.096728in}{2.694219in}}%
\pgfpathlineto{\pgfqpoint{4.101269in}{2.694219in}}%
\pgfpathlineto{\pgfqpoint{4.101269in}{2.691270in}}%
\pgfpathmoveto{\pgfqpoint{4.101269in}{2.691270in}}%
\pgfpathlineto{\pgfqpoint{4.101269in}{2.691270in}}%
\pgfpathlineto{\pgfqpoint{4.101269in}{2.694219in}}%
\pgfpathlineto{\pgfqpoint{4.105810in}{2.694219in}}%
\pgfpathlineto{\pgfqpoint{4.105810in}{2.691270in}}%
\pgfpathmoveto{\pgfqpoint{4.105810in}{2.691270in}}%
\pgfpathlineto{\pgfqpoint{4.105810in}{2.691270in}}%
\pgfpathlineto{\pgfqpoint{4.105810in}{2.694219in}}%
\pgfpathlineto{\pgfqpoint{4.110351in}{2.694219in}}%
\pgfpathlineto{\pgfqpoint{4.110351in}{2.691270in}}%
\pgfpathmoveto{\pgfqpoint{4.110351in}{2.688320in}}%
\pgfpathlineto{\pgfqpoint{4.110351in}{2.688320in}}%
\pgfpathlineto{\pgfqpoint{4.110351in}{2.691270in}}%
\pgfpathlineto{\pgfqpoint{4.114892in}{2.691270in}}%
\pgfpathlineto{\pgfqpoint{4.114892in}{2.688320in}}%
\pgfpathmoveto{\pgfqpoint{4.110351in}{2.691270in}}%
\pgfpathlineto{\pgfqpoint{4.110351in}{2.691270in}}%
\pgfpathlineto{\pgfqpoint{4.110351in}{2.694219in}}%
\pgfpathlineto{\pgfqpoint{4.114892in}{2.694219in}}%
\pgfpathlineto{\pgfqpoint{4.114892in}{2.691270in}}%
\pgfpathmoveto{\pgfqpoint{4.114892in}{2.688320in}}%
\pgfpathlineto{\pgfqpoint{4.114892in}{2.688320in}}%
\pgfpathlineto{\pgfqpoint{4.114892in}{2.691270in}}%
\pgfpathlineto{\pgfqpoint{4.119434in}{2.691270in}}%
\pgfpathlineto{\pgfqpoint{4.119434in}{2.688320in}}%
\pgfpathmoveto{\pgfqpoint{4.123975in}{2.685371in}}%
\pgfpathlineto{\pgfqpoint{4.123975in}{2.685371in}}%
\pgfpathlineto{\pgfqpoint{4.123975in}{2.688320in}}%
\pgfpathlineto{\pgfqpoint{4.128516in}{2.688320in}}%
\pgfpathlineto{\pgfqpoint{4.128516in}{2.685371in}}%
\pgfpathmoveto{\pgfqpoint{4.119434in}{2.688320in}}%
\pgfpathlineto{\pgfqpoint{4.119434in}{2.688320in}}%
\pgfpathlineto{\pgfqpoint{4.119434in}{2.691270in}}%
\pgfpathlineto{\pgfqpoint{4.123975in}{2.691270in}}%
\pgfpathlineto{\pgfqpoint{4.123975in}{2.688320in}}%
\pgfpathmoveto{\pgfqpoint{4.123975in}{2.688320in}}%
\pgfpathlineto{\pgfqpoint{4.123975in}{2.688320in}}%
\pgfpathlineto{\pgfqpoint{4.123975in}{2.691270in}}%
\pgfpathlineto{\pgfqpoint{4.128516in}{2.691270in}}%
\pgfpathlineto{\pgfqpoint{4.128516in}{2.688320in}}%
\pgfpathmoveto{\pgfqpoint{4.092187in}{2.694219in}}%
\pgfpathlineto{\pgfqpoint{4.092187in}{2.694219in}}%
\pgfpathlineto{\pgfqpoint{4.092187in}{2.697168in}}%
\pgfpathlineto{\pgfqpoint{4.096728in}{2.697168in}}%
\pgfpathlineto{\pgfqpoint{4.096728in}{2.694219in}}%
\pgfpathmoveto{\pgfqpoint{4.096728in}{2.694219in}}%
\pgfpathlineto{\pgfqpoint{4.096728in}{2.694219in}}%
\pgfpathlineto{\pgfqpoint{4.096728in}{2.697168in}}%
\pgfpathlineto{\pgfqpoint{4.101269in}{2.697168in}}%
\pgfpathlineto{\pgfqpoint{4.101269in}{2.694219in}}%
\pgfpathmoveto{\pgfqpoint{4.128516in}{2.685371in}}%
\pgfpathlineto{\pgfqpoint{4.128516in}{2.685371in}}%
\pgfpathlineto{\pgfqpoint{4.128516in}{2.688320in}}%
\pgfpathlineto{\pgfqpoint{4.133057in}{2.688320in}}%
\pgfpathlineto{\pgfqpoint{4.133057in}{2.685371in}}%
\pgfpathmoveto{\pgfqpoint{4.133057in}{2.685371in}}%
\pgfpathlineto{\pgfqpoint{4.133057in}{2.685371in}}%
\pgfpathlineto{\pgfqpoint{4.133057in}{2.688320in}}%
\pgfpathlineto{\pgfqpoint{4.137598in}{2.688320in}}%
\pgfpathlineto{\pgfqpoint{4.137598in}{2.685371in}}%
\pgfpathmoveto{\pgfqpoint{4.137598in}{2.682422in}}%
\pgfpathlineto{\pgfqpoint{4.137598in}{2.682422in}}%
\pgfpathlineto{\pgfqpoint{4.137598in}{2.685371in}}%
\pgfpathlineto{\pgfqpoint{4.142139in}{2.685371in}}%
\pgfpathlineto{\pgfqpoint{4.142139in}{2.682422in}}%
\pgfpathmoveto{\pgfqpoint{4.137598in}{2.685371in}}%
\pgfpathlineto{\pgfqpoint{4.137598in}{2.685371in}}%
\pgfpathlineto{\pgfqpoint{4.137598in}{2.688320in}}%
\pgfpathlineto{\pgfqpoint{4.142139in}{2.688320in}}%
\pgfpathlineto{\pgfqpoint{4.142139in}{2.685371in}}%
\pgfpathmoveto{\pgfqpoint{4.142139in}{2.682422in}}%
\pgfpathlineto{\pgfqpoint{4.142139in}{2.682422in}}%
\pgfpathlineto{\pgfqpoint{4.142139in}{2.685371in}}%
\pgfpathlineto{\pgfqpoint{4.146680in}{2.685371in}}%
\pgfpathlineto{\pgfqpoint{4.146680in}{2.682422in}}%
\pgfpathmoveto{\pgfqpoint{4.151221in}{2.679473in}}%
\pgfpathlineto{\pgfqpoint{4.151221in}{2.679473in}}%
\pgfpathlineto{\pgfqpoint{4.151221in}{2.682422in}}%
\pgfpathlineto{\pgfqpoint{4.155763in}{2.682422in}}%
\pgfpathlineto{\pgfqpoint{4.155763in}{2.679473in}}%
\pgfpathmoveto{\pgfqpoint{4.155763in}{2.679473in}}%
\pgfpathlineto{\pgfqpoint{4.155763in}{2.679473in}}%
\pgfpathlineto{\pgfqpoint{4.155763in}{2.682422in}}%
\pgfpathlineto{\pgfqpoint{4.160304in}{2.682422in}}%
\pgfpathlineto{\pgfqpoint{4.160304in}{2.679473in}}%
\pgfpathmoveto{\pgfqpoint{4.160304in}{2.679473in}}%
\pgfpathlineto{\pgfqpoint{4.160304in}{2.679473in}}%
\pgfpathlineto{\pgfqpoint{4.160304in}{2.682422in}}%
\pgfpathlineto{\pgfqpoint{4.164845in}{2.682422in}}%
\pgfpathlineto{\pgfqpoint{4.164845in}{2.679473in}}%
\pgfpathmoveto{\pgfqpoint{4.146680in}{2.682422in}}%
\pgfpathlineto{\pgfqpoint{4.146680in}{2.682422in}}%
\pgfpathlineto{\pgfqpoint{4.146680in}{2.685371in}}%
\pgfpathlineto{\pgfqpoint{4.151221in}{2.685371in}}%
\pgfpathlineto{\pgfqpoint{4.151221in}{2.682422in}}%
\pgfpathmoveto{\pgfqpoint{4.151221in}{2.682422in}}%
\pgfpathlineto{\pgfqpoint{4.151221in}{2.682422in}}%
\pgfpathlineto{\pgfqpoint{4.151221in}{2.685371in}}%
\pgfpathlineto{\pgfqpoint{4.155763in}{2.685371in}}%
\pgfpathlineto{\pgfqpoint{4.155763in}{2.682422in}}%
\pgfpathmoveto{\pgfqpoint{4.164845in}{2.676524in}}%
\pgfpathlineto{\pgfqpoint{4.164845in}{2.676524in}}%
\pgfpathlineto{\pgfqpoint{4.164845in}{2.679473in}}%
\pgfpathlineto{\pgfqpoint{4.169386in}{2.679473in}}%
\pgfpathlineto{\pgfqpoint{4.169386in}{2.676524in}}%
\pgfpathmoveto{\pgfqpoint{4.164845in}{2.679473in}}%
\pgfpathlineto{\pgfqpoint{4.164845in}{2.679473in}}%
\pgfpathlineto{\pgfqpoint{4.164845in}{2.682422in}}%
\pgfpathlineto{\pgfqpoint{4.169386in}{2.682422in}}%
\pgfpathlineto{\pgfqpoint{4.169386in}{2.679473in}}%
\pgfpathmoveto{\pgfqpoint{4.169386in}{2.676524in}}%
\pgfpathlineto{\pgfqpoint{4.169386in}{2.676524in}}%
\pgfpathlineto{\pgfqpoint{4.169386in}{2.679473in}}%
\pgfpathlineto{\pgfqpoint{4.173927in}{2.679473in}}%
\pgfpathlineto{\pgfqpoint{4.173927in}{2.676524in}}%
\pgfpathmoveto{\pgfqpoint{4.178468in}{2.673575in}}%
\pgfpathlineto{\pgfqpoint{4.178468in}{2.673575in}}%
\pgfpathlineto{\pgfqpoint{4.178468in}{2.676524in}}%
\pgfpathlineto{\pgfqpoint{4.183009in}{2.676524in}}%
\pgfpathlineto{\pgfqpoint{4.183009in}{2.673575in}}%
\pgfpathmoveto{\pgfqpoint{4.173927in}{2.676524in}}%
\pgfpathlineto{\pgfqpoint{4.173927in}{2.676524in}}%
\pgfpathlineto{\pgfqpoint{4.173927in}{2.679473in}}%
\pgfpathlineto{\pgfqpoint{4.178468in}{2.679473in}}%
\pgfpathlineto{\pgfqpoint{4.178468in}{2.676524in}}%
\pgfpathmoveto{\pgfqpoint{4.178468in}{2.676524in}}%
\pgfpathlineto{\pgfqpoint{4.178468in}{2.676524in}}%
\pgfpathlineto{\pgfqpoint{4.178468in}{2.679473in}}%
\pgfpathlineto{\pgfqpoint{4.183009in}{2.679473in}}%
\pgfpathlineto{\pgfqpoint{4.183009in}{2.676524in}}%
\pgfpathmoveto{\pgfqpoint{4.183009in}{2.673575in}}%
\pgfpathlineto{\pgfqpoint{4.183009in}{2.673575in}}%
\pgfpathlineto{\pgfqpoint{4.183009in}{2.676524in}}%
\pgfpathlineto{\pgfqpoint{4.187550in}{2.676524in}}%
\pgfpathlineto{\pgfqpoint{4.187550in}{2.673575in}}%
\pgfpathmoveto{\pgfqpoint{4.187550in}{2.673575in}}%
\pgfpathlineto{\pgfqpoint{4.187550in}{2.673575in}}%
\pgfpathlineto{\pgfqpoint{4.187550in}{2.676524in}}%
\pgfpathlineto{\pgfqpoint{4.192092in}{2.676524in}}%
\pgfpathlineto{\pgfqpoint{4.192092in}{2.673575in}}%
\pgfpathmoveto{\pgfqpoint{4.192092in}{2.670626in}}%
\pgfpathlineto{\pgfqpoint{4.192092in}{2.670626in}}%
\pgfpathlineto{\pgfqpoint{4.192092in}{2.673575in}}%
\pgfpathlineto{\pgfqpoint{4.196633in}{2.673575in}}%
\pgfpathlineto{\pgfqpoint{4.196633in}{2.670626in}}%
\pgfpathmoveto{\pgfqpoint{4.192092in}{2.673575in}}%
\pgfpathlineto{\pgfqpoint{4.192092in}{2.673575in}}%
\pgfpathlineto{\pgfqpoint{4.192092in}{2.676524in}}%
\pgfpathlineto{\pgfqpoint{4.196633in}{2.676524in}}%
\pgfpathlineto{\pgfqpoint{4.196633in}{2.673575in}}%
\pgfpathmoveto{\pgfqpoint{4.196633in}{2.670626in}}%
\pgfpathlineto{\pgfqpoint{4.196633in}{2.670626in}}%
\pgfpathlineto{\pgfqpoint{4.196633in}{2.673575in}}%
\pgfpathlineto{\pgfqpoint{4.201174in}{2.673575in}}%
\pgfpathlineto{\pgfqpoint{4.201174in}{2.670626in}}%
\pgfpathmoveto{\pgfqpoint{4.201174in}{2.670626in}}%
\pgfpathlineto{\pgfqpoint{4.201174in}{2.670626in}}%
\pgfpathlineto{\pgfqpoint{4.201174in}{2.673575in}}%
\pgfpathlineto{\pgfqpoint{4.205715in}{2.673575in}}%
\pgfpathlineto{\pgfqpoint{4.205715in}{2.670626in}}%
\pgfpathmoveto{\pgfqpoint{4.205715in}{2.670626in}}%
\pgfpathlineto{\pgfqpoint{4.205715in}{2.670626in}}%
\pgfpathlineto{\pgfqpoint{4.205715in}{2.673575in}}%
\pgfpathlineto{\pgfqpoint{4.210256in}{2.673575in}}%
\pgfpathlineto{\pgfqpoint{4.210256in}{2.670626in}}%
\pgfpathmoveto{\pgfqpoint{4.237503in}{2.661778in}}%
\pgfpathlineto{\pgfqpoint{4.237503in}{2.661778in}}%
\pgfpathlineto{\pgfqpoint{4.237503in}{2.664727in}}%
\pgfpathlineto{\pgfqpoint{4.242044in}{2.664727in}}%
\pgfpathlineto{\pgfqpoint{4.242044in}{2.661778in}}%
\pgfpathmoveto{\pgfqpoint{4.242044in}{2.661778in}}%
\pgfpathlineto{\pgfqpoint{4.242044in}{2.661778in}}%
\pgfpathlineto{\pgfqpoint{4.242044in}{2.664727in}}%
\pgfpathlineto{\pgfqpoint{4.246585in}{2.664727in}}%
\pgfpathlineto{\pgfqpoint{4.246585in}{2.661778in}}%
\pgfpathmoveto{\pgfqpoint{4.246585in}{2.658828in}}%
\pgfpathlineto{\pgfqpoint{4.246585in}{2.658828in}}%
\pgfpathlineto{\pgfqpoint{4.246585in}{2.661778in}}%
\pgfpathlineto{\pgfqpoint{4.251126in}{2.661778in}}%
\pgfpathlineto{\pgfqpoint{4.251126in}{2.658828in}}%
\pgfpathmoveto{\pgfqpoint{4.246585in}{2.661778in}}%
\pgfpathlineto{\pgfqpoint{4.246585in}{2.661778in}}%
\pgfpathlineto{\pgfqpoint{4.246585in}{2.664727in}}%
\pgfpathlineto{\pgfqpoint{4.251126in}{2.664727in}}%
\pgfpathlineto{\pgfqpoint{4.251126in}{2.661778in}}%
\pgfpathmoveto{\pgfqpoint{4.251126in}{2.658828in}}%
\pgfpathlineto{\pgfqpoint{4.251126in}{2.658828in}}%
\pgfpathlineto{\pgfqpoint{4.251126in}{2.661778in}}%
\pgfpathlineto{\pgfqpoint{4.255667in}{2.661778in}}%
\pgfpathlineto{\pgfqpoint{4.255667in}{2.658828in}}%
\pgfpathmoveto{\pgfqpoint{4.260208in}{2.655879in}}%
\pgfpathlineto{\pgfqpoint{4.260208in}{2.655879in}}%
\pgfpathlineto{\pgfqpoint{4.260208in}{2.658828in}}%
\pgfpathlineto{\pgfqpoint{4.264748in}{2.658828in}}%
\pgfpathlineto{\pgfqpoint{4.264748in}{2.655879in}}%
\pgfpathmoveto{\pgfqpoint{4.264748in}{2.655879in}}%
\pgfpathlineto{\pgfqpoint{4.264748in}{2.655879in}}%
\pgfpathlineto{\pgfqpoint{4.264748in}{2.658828in}}%
\pgfpathlineto{\pgfqpoint{4.269289in}{2.658828in}}%
\pgfpathlineto{\pgfqpoint{4.269289in}{2.655879in}}%
\pgfpathmoveto{\pgfqpoint{4.269289in}{2.655879in}}%
\pgfpathlineto{\pgfqpoint{4.269289in}{2.655879in}}%
\pgfpathlineto{\pgfqpoint{4.269289in}{2.658828in}}%
\pgfpathlineto{\pgfqpoint{4.273830in}{2.658828in}}%
\pgfpathlineto{\pgfqpoint{4.273830in}{2.655879in}}%
\pgfpathmoveto{\pgfqpoint{4.255667in}{2.658828in}}%
\pgfpathlineto{\pgfqpoint{4.255667in}{2.658828in}}%
\pgfpathlineto{\pgfqpoint{4.255667in}{2.661778in}}%
\pgfpathlineto{\pgfqpoint{4.260208in}{2.661778in}}%
\pgfpathlineto{\pgfqpoint{4.260208in}{2.658828in}}%
\pgfpathmoveto{\pgfqpoint{4.260208in}{2.658828in}}%
\pgfpathlineto{\pgfqpoint{4.260208in}{2.658828in}}%
\pgfpathlineto{\pgfqpoint{4.260208in}{2.661778in}}%
\pgfpathlineto{\pgfqpoint{4.264748in}{2.661778in}}%
\pgfpathlineto{\pgfqpoint{4.264748in}{2.658828in}}%
\pgfpathmoveto{\pgfqpoint{4.273830in}{2.652930in}}%
\pgfpathlineto{\pgfqpoint{4.273830in}{2.652930in}}%
\pgfpathlineto{\pgfqpoint{4.273830in}{2.655879in}}%
\pgfpathlineto{\pgfqpoint{4.278371in}{2.655879in}}%
\pgfpathlineto{\pgfqpoint{4.278371in}{2.652930in}}%
\pgfpathmoveto{\pgfqpoint{4.273830in}{2.655879in}}%
\pgfpathlineto{\pgfqpoint{4.273830in}{2.655879in}}%
\pgfpathlineto{\pgfqpoint{4.273830in}{2.658828in}}%
\pgfpathlineto{\pgfqpoint{4.278371in}{2.658828in}}%
\pgfpathlineto{\pgfqpoint{4.278371in}{2.655879in}}%
\pgfpathmoveto{\pgfqpoint{4.278371in}{2.652930in}}%
\pgfpathlineto{\pgfqpoint{4.278371in}{2.652930in}}%
\pgfpathlineto{\pgfqpoint{4.278371in}{2.655879in}}%
\pgfpathlineto{\pgfqpoint{4.282912in}{2.655879in}}%
\pgfpathlineto{\pgfqpoint{4.282912in}{2.652930in}}%
\pgfpathmoveto{\pgfqpoint{4.287453in}{2.649980in}}%
\pgfpathlineto{\pgfqpoint{4.287453in}{2.649980in}}%
\pgfpathlineto{\pgfqpoint{4.287453in}{2.652930in}}%
\pgfpathlineto{\pgfqpoint{4.291994in}{2.652930in}}%
\pgfpathlineto{\pgfqpoint{4.291994in}{2.649980in}}%
\pgfpathmoveto{\pgfqpoint{4.282912in}{2.652930in}}%
\pgfpathlineto{\pgfqpoint{4.282912in}{2.652930in}}%
\pgfpathlineto{\pgfqpoint{4.282912in}{2.655879in}}%
\pgfpathlineto{\pgfqpoint{4.287453in}{2.655879in}}%
\pgfpathlineto{\pgfqpoint{4.287453in}{2.652930in}}%
\pgfpathmoveto{\pgfqpoint{4.287453in}{2.652930in}}%
\pgfpathlineto{\pgfqpoint{4.287453in}{2.652930in}}%
\pgfpathlineto{\pgfqpoint{4.287453in}{2.655879in}}%
\pgfpathlineto{\pgfqpoint{4.291994in}{2.655879in}}%
\pgfpathlineto{\pgfqpoint{4.291994in}{2.652930in}}%
\pgfpathmoveto{\pgfqpoint{4.291994in}{2.649980in}}%
\pgfpathlineto{\pgfqpoint{4.291994in}{2.649980in}}%
\pgfpathlineto{\pgfqpoint{4.291994in}{2.652930in}}%
\pgfpathlineto{\pgfqpoint{4.296535in}{2.652930in}}%
\pgfpathlineto{\pgfqpoint{4.296535in}{2.649980in}}%
\pgfpathmoveto{\pgfqpoint{4.296535in}{2.649980in}}%
\pgfpathlineto{\pgfqpoint{4.296535in}{2.649980in}}%
\pgfpathlineto{\pgfqpoint{4.296535in}{2.652930in}}%
\pgfpathlineto{\pgfqpoint{4.301076in}{2.652930in}}%
\pgfpathlineto{\pgfqpoint{4.301076in}{2.649980in}}%
\pgfpathmoveto{\pgfqpoint{4.301076in}{2.647031in}}%
\pgfpathlineto{\pgfqpoint{4.301076in}{2.647031in}}%
\pgfpathlineto{\pgfqpoint{4.301076in}{2.649980in}}%
\pgfpathlineto{\pgfqpoint{4.305617in}{2.649980in}}%
\pgfpathlineto{\pgfqpoint{4.305617in}{2.647031in}}%
\pgfpathmoveto{\pgfqpoint{4.301076in}{2.649980in}}%
\pgfpathlineto{\pgfqpoint{4.301076in}{2.649980in}}%
\pgfpathlineto{\pgfqpoint{4.301076in}{2.652930in}}%
\pgfpathlineto{\pgfqpoint{4.305617in}{2.652930in}}%
\pgfpathlineto{\pgfqpoint{4.305617in}{2.649980in}}%
\pgfpathmoveto{\pgfqpoint{4.305617in}{2.647031in}}%
\pgfpathlineto{\pgfqpoint{4.305617in}{2.647031in}}%
\pgfpathlineto{\pgfqpoint{4.305617in}{2.649980in}}%
\pgfpathlineto{\pgfqpoint{4.310158in}{2.649980in}}%
\pgfpathlineto{\pgfqpoint{4.310158in}{2.647031in}}%
\pgfpathmoveto{\pgfqpoint{4.314699in}{2.644082in}}%
\pgfpathlineto{\pgfqpoint{4.314699in}{2.644082in}}%
\pgfpathlineto{\pgfqpoint{4.314699in}{2.647031in}}%
\pgfpathlineto{\pgfqpoint{4.319240in}{2.647031in}}%
\pgfpathlineto{\pgfqpoint{4.319240in}{2.644082in}}%
\pgfpathmoveto{\pgfqpoint{4.319240in}{2.644082in}}%
\pgfpathlineto{\pgfqpoint{4.319240in}{2.644082in}}%
\pgfpathlineto{\pgfqpoint{4.319240in}{2.647031in}}%
\pgfpathlineto{\pgfqpoint{4.323781in}{2.647031in}}%
\pgfpathlineto{\pgfqpoint{4.323781in}{2.644082in}}%
\pgfpathmoveto{\pgfqpoint{4.323781in}{2.644082in}}%
\pgfpathlineto{\pgfqpoint{4.323781in}{2.644082in}}%
\pgfpathlineto{\pgfqpoint{4.323781in}{2.647031in}}%
\pgfpathlineto{\pgfqpoint{4.328322in}{2.647031in}}%
\pgfpathlineto{\pgfqpoint{4.328322in}{2.644082in}}%
\pgfpathmoveto{\pgfqpoint{4.328322in}{2.641133in}}%
\pgfpathlineto{\pgfqpoint{4.328322in}{2.641133in}}%
\pgfpathlineto{\pgfqpoint{4.328322in}{2.644082in}}%
\pgfpathlineto{\pgfqpoint{4.332862in}{2.644082in}}%
\pgfpathlineto{\pgfqpoint{4.332862in}{2.641133in}}%
\pgfpathmoveto{\pgfqpoint{4.328322in}{2.644082in}}%
\pgfpathlineto{\pgfqpoint{4.328322in}{2.644082in}}%
\pgfpathlineto{\pgfqpoint{4.328322in}{2.647031in}}%
\pgfpathlineto{\pgfqpoint{4.332862in}{2.647031in}}%
\pgfpathlineto{\pgfqpoint{4.332862in}{2.644082in}}%
\pgfpathmoveto{\pgfqpoint{4.332862in}{2.641133in}}%
\pgfpathlineto{\pgfqpoint{4.332862in}{2.641133in}}%
\pgfpathlineto{\pgfqpoint{4.332862in}{2.644082in}}%
\pgfpathlineto{\pgfqpoint{4.337403in}{2.644082in}}%
\pgfpathlineto{\pgfqpoint{4.337403in}{2.641133in}}%
\pgfpathmoveto{\pgfqpoint{4.341944in}{2.638183in}}%
\pgfpathlineto{\pgfqpoint{4.341944in}{2.638183in}}%
\pgfpathlineto{\pgfqpoint{4.341944in}{2.641133in}}%
\pgfpathlineto{\pgfqpoint{4.346485in}{2.641133in}}%
\pgfpathlineto{\pgfqpoint{4.346485in}{2.638183in}}%
\pgfpathmoveto{\pgfqpoint{4.337403in}{2.641133in}}%
\pgfpathlineto{\pgfqpoint{4.337403in}{2.641133in}}%
\pgfpathlineto{\pgfqpoint{4.337403in}{2.644082in}}%
\pgfpathlineto{\pgfqpoint{4.341944in}{2.644082in}}%
\pgfpathlineto{\pgfqpoint{4.341944in}{2.641133in}}%
\pgfpathmoveto{\pgfqpoint{4.341944in}{2.641133in}}%
\pgfpathlineto{\pgfqpoint{4.341944in}{2.641133in}}%
\pgfpathlineto{\pgfqpoint{4.341944in}{2.644082in}}%
\pgfpathlineto{\pgfqpoint{4.346485in}{2.644082in}}%
\pgfpathlineto{\pgfqpoint{4.346485in}{2.641133in}}%
\pgfpathmoveto{\pgfqpoint{4.310158in}{2.647031in}}%
\pgfpathlineto{\pgfqpoint{4.310158in}{2.647031in}}%
\pgfpathlineto{\pgfqpoint{4.310158in}{2.649980in}}%
\pgfpathlineto{\pgfqpoint{4.314699in}{2.649980in}}%
\pgfpathlineto{\pgfqpoint{4.314699in}{2.647031in}}%
\pgfpathmoveto{\pgfqpoint{4.314699in}{2.647031in}}%
\pgfpathlineto{\pgfqpoint{4.314699in}{2.647031in}}%
\pgfpathlineto{\pgfqpoint{4.314699in}{2.649980in}}%
\pgfpathlineto{\pgfqpoint{4.319240in}{2.649980in}}%
\pgfpathlineto{\pgfqpoint{4.319240in}{2.647031in}}%
\pgfpathmoveto{\pgfqpoint{4.346485in}{2.638183in}}%
\pgfpathlineto{\pgfqpoint{4.346485in}{2.638183in}}%
\pgfpathlineto{\pgfqpoint{4.346485in}{2.641133in}}%
\pgfpathlineto{\pgfqpoint{4.351026in}{2.641133in}}%
\pgfpathlineto{\pgfqpoint{4.351026in}{2.638183in}}%
\pgfpathmoveto{\pgfqpoint{4.351026in}{2.638183in}}%
\pgfpathlineto{\pgfqpoint{4.351026in}{2.638183in}}%
\pgfpathlineto{\pgfqpoint{4.351026in}{2.641133in}}%
\pgfpathlineto{\pgfqpoint{4.355567in}{2.641133in}}%
\pgfpathlineto{\pgfqpoint{4.355567in}{2.638183in}}%
\pgfpathmoveto{\pgfqpoint{4.355567in}{2.635234in}}%
\pgfpathlineto{\pgfqpoint{4.355567in}{2.635234in}}%
\pgfpathlineto{\pgfqpoint{4.355567in}{2.638183in}}%
\pgfpathlineto{\pgfqpoint{4.360108in}{2.638183in}}%
\pgfpathlineto{\pgfqpoint{4.360108in}{2.635234in}}%
\pgfpathmoveto{\pgfqpoint{4.355567in}{2.638183in}}%
\pgfpathlineto{\pgfqpoint{4.355567in}{2.638183in}}%
\pgfpathlineto{\pgfqpoint{4.355567in}{2.641133in}}%
\pgfpathlineto{\pgfqpoint{4.360108in}{2.641133in}}%
\pgfpathlineto{\pgfqpoint{4.360108in}{2.638183in}}%
\pgfpathmoveto{\pgfqpoint{4.360108in}{2.635234in}}%
\pgfpathlineto{\pgfqpoint{4.360108in}{2.635234in}}%
\pgfpathlineto{\pgfqpoint{4.360108in}{2.638183in}}%
\pgfpathlineto{\pgfqpoint{4.364649in}{2.638183in}}%
\pgfpathlineto{\pgfqpoint{4.364649in}{2.635234in}}%
\pgfpathmoveto{\pgfqpoint{4.369190in}{2.632285in}}%
\pgfpathlineto{\pgfqpoint{4.369190in}{2.632285in}}%
\pgfpathlineto{\pgfqpoint{4.369190in}{2.635234in}}%
\pgfpathlineto{\pgfqpoint{4.373731in}{2.635234in}}%
\pgfpathlineto{\pgfqpoint{4.373731in}{2.632285in}}%
\pgfpathmoveto{\pgfqpoint{4.373731in}{2.632285in}}%
\pgfpathlineto{\pgfqpoint{4.373731in}{2.632285in}}%
\pgfpathlineto{\pgfqpoint{4.373731in}{2.635234in}}%
\pgfpathlineto{\pgfqpoint{4.378272in}{2.635234in}}%
\pgfpathlineto{\pgfqpoint{4.378272in}{2.632285in}}%
\pgfpathmoveto{\pgfqpoint{4.378272in}{2.632285in}}%
\pgfpathlineto{\pgfqpoint{4.378272in}{2.632285in}}%
\pgfpathlineto{\pgfqpoint{4.378272in}{2.635234in}}%
\pgfpathlineto{\pgfqpoint{4.382813in}{2.635234in}}%
\pgfpathlineto{\pgfqpoint{4.382813in}{2.632285in}}%
\pgfpathmoveto{\pgfqpoint{4.364649in}{2.635234in}}%
\pgfpathlineto{\pgfqpoint{4.364649in}{2.635234in}}%
\pgfpathlineto{\pgfqpoint{4.364649in}{2.638183in}}%
\pgfpathlineto{\pgfqpoint{4.369190in}{2.638183in}}%
\pgfpathlineto{\pgfqpoint{4.369190in}{2.635234in}}%
\pgfpathmoveto{\pgfqpoint{4.369190in}{2.635234in}}%
\pgfpathlineto{\pgfqpoint{4.369190in}{2.635234in}}%
\pgfpathlineto{\pgfqpoint{4.369190in}{2.638183in}}%
\pgfpathlineto{\pgfqpoint{4.373731in}{2.638183in}}%
\pgfpathlineto{\pgfqpoint{4.373731in}{2.635234in}}%
\pgfpathmoveto{\pgfqpoint{4.423682in}{2.620487in}}%
\pgfpathlineto{\pgfqpoint{4.423682in}{2.620487in}}%
\pgfpathlineto{\pgfqpoint{4.423682in}{2.623437in}}%
\pgfpathlineto{\pgfqpoint{4.428223in}{2.623437in}}%
\pgfpathlineto{\pgfqpoint{4.428223in}{2.620487in}}%
\pgfpathmoveto{\pgfqpoint{4.428223in}{2.620487in}}%
\pgfpathlineto{\pgfqpoint{4.428223in}{2.620487in}}%
\pgfpathlineto{\pgfqpoint{4.428223in}{2.623437in}}%
\pgfpathlineto{\pgfqpoint{4.432764in}{2.623437in}}%
\pgfpathlineto{\pgfqpoint{4.432764in}{2.620487in}}%
\pgfpathmoveto{\pgfqpoint{4.432764in}{2.620487in}}%
\pgfpathlineto{\pgfqpoint{4.432764in}{2.620487in}}%
\pgfpathlineto{\pgfqpoint{4.432764in}{2.623437in}}%
\pgfpathlineto{\pgfqpoint{4.437305in}{2.623437in}}%
\pgfpathlineto{\pgfqpoint{4.437305in}{2.620487in}}%
\pgfpathmoveto{\pgfqpoint{4.437305in}{2.617538in}}%
\pgfpathlineto{\pgfqpoint{4.437305in}{2.617538in}}%
\pgfpathlineto{\pgfqpoint{4.437305in}{2.620487in}}%
\pgfpathlineto{\pgfqpoint{4.441846in}{2.620487in}}%
\pgfpathlineto{\pgfqpoint{4.441846in}{2.617538in}}%
\pgfpathmoveto{\pgfqpoint{4.437305in}{2.620487in}}%
\pgfpathlineto{\pgfqpoint{4.437305in}{2.620487in}}%
\pgfpathlineto{\pgfqpoint{4.437305in}{2.623437in}}%
\pgfpathlineto{\pgfqpoint{4.441846in}{2.623437in}}%
\pgfpathlineto{\pgfqpoint{4.441846in}{2.620487in}}%
\pgfpathmoveto{\pgfqpoint{4.441846in}{2.617538in}}%
\pgfpathlineto{\pgfqpoint{4.441846in}{2.617538in}}%
\pgfpathlineto{\pgfqpoint{4.441846in}{2.620487in}}%
\pgfpathlineto{\pgfqpoint{4.446388in}{2.620487in}}%
\pgfpathlineto{\pgfqpoint{4.446388in}{2.617538in}}%
\pgfpathmoveto{\pgfqpoint{4.450929in}{2.614589in}}%
\pgfpathlineto{\pgfqpoint{4.450929in}{2.614589in}}%
\pgfpathlineto{\pgfqpoint{4.450929in}{2.617538in}}%
\pgfpathlineto{\pgfqpoint{4.455470in}{2.617538in}}%
\pgfpathlineto{\pgfqpoint{4.455470in}{2.614589in}}%
\pgfpathmoveto{\pgfqpoint{4.446388in}{2.617538in}}%
\pgfpathlineto{\pgfqpoint{4.446388in}{2.617538in}}%
\pgfpathlineto{\pgfqpoint{4.446388in}{2.620487in}}%
\pgfpathlineto{\pgfqpoint{4.450929in}{2.620487in}}%
\pgfpathlineto{\pgfqpoint{4.450929in}{2.617538in}}%
\pgfpathmoveto{\pgfqpoint{4.450929in}{2.617538in}}%
\pgfpathlineto{\pgfqpoint{4.450929in}{2.617538in}}%
\pgfpathlineto{\pgfqpoint{4.450929in}{2.620487in}}%
\pgfpathlineto{\pgfqpoint{4.455470in}{2.620487in}}%
\pgfpathlineto{\pgfqpoint{4.455470in}{2.617538in}}%
\pgfpathmoveto{\pgfqpoint{4.382813in}{2.629335in}}%
\pgfpathlineto{\pgfqpoint{4.382813in}{2.629335in}}%
\pgfpathlineto{\pgfqpoint{4.382813in}{2.632285in}}%
\pgfpathlineto{\pgfqpoint{4.387354in}{2.632285in}}%
\pgfpathlineto{\pgfqpoint{4.387354in}{2.629335in}}%
\pgfpathmoveto{\pgfqpoint{4.382813in}{2.632285in}}%
\pgfpathlineto{\pgfqpoint{4.382813in}{2.632285in}}%
\pgfpathlineto{\pgfqpoint{4.382813in}{2.635234in}}%
\pgfpathlineto{\pgfqpoint{4.387354in}{2.635234in}}%
\pgfpathlineto{\pgfqpoint{4.387354in}{2.632285in}}%
\pgfpathmoveto{\pgfqpoint{4.387354in}{2.629335in}}%
\pgfpathlineto{\pgfqpoint{4.387354in}{2.629335in}}%
\pgfpathlineto{\pgfqpoint{4.387354in}{2.632285in}}%
\pgfpathlineto{\pgfqpoint{4.391895in}{2.632285in}}%
\pgfpathlineto{\pgfqpoint{4.391895in}{2.629335in}}%
\pgfpathmoveto{\pgfqpoint{4.396436in}{2.626386in}}%
\pgfpathlineto{\pgfqpoint{4.396436in}{2.626386in}}%
\pgfpathlineto{\pgfqpoint{4.396436in}{2.629335in}}%
\pgfpathlineto{\pgfqpoint{4.400977in}{2.629335in}}%
\pgfpathlineto{\pgfqpoint{4.400977in}{2.626386in}}%
\pgfpathmoveto{\pgfqpoint{4.391895in}{2.629335in}}%
\pgfpathlineto{\pgfqpoint{4.391895in}{2.629335in}}%
\pgfpathlineto{\pgfqpoint{4.391895in}{2.632285in}}%
\pgfpathlineto{\pgfqpoint{4.396436in}{2.632285in}}%
\pgfpathlineto{\pgfqpoint{4.396436in}{2.629335in}}%
\pgfpathmoveto{\pgfqpoint{4.396436in}{2.629335in}}%
\pgfpathlineto{\pgfqpoint{4.396436in}{2.629335in}}%
\pgfpathlineto{\pgfqpoint{4.396436in}{2.632285in}}%
\pgfpathlineto{\pgfqpoint{4.400977in}{2.632285in}}%
\pgfpathlineto{\pgfqpoint{4.400977in}{2.629335in}}%
\pgfpathmoveto{\pgfqpoint{4.400977in}{2.626386in}}%
\pgfpathlineto{\pgfqpoint{4.400977in}{2.626386in}}%
\pgfpathlineto{\pgfqpoint{4.400977in}{2.629335in}}%
\pgfpathlineto{\pgfqpoint{4.405518in}{2.629335in}}%
\pgfpathlineto{\pgfqpoint{4.405518in}{2.626386in}}%
\pgfpathmoveto{\pgfqpoint{4.405518in}{2.626386in}}%
\pgfpathlineto{\pgfqpoint{4.405518in}{2.626386in}}%
\pgfpathlineto{\pgfqpoint{4.405518in}{2.629335in}}%
\pgfpathlineto{\pgfqpoint{4.410059in}{2.629335in}}%
\pgfpathlineto{\pgfqpoint{4.410059in}{2.626386in}}%
\pgfpathmoveto{\pgfqpoint{4.410059in}{2.623437in}}%
\pgfpathlineto{\pgfqpoint{4.410059in}{2.623437in}}%
\pgfpathlineto{\pgfqpoint{4.410059in}{2.626386in}}%
\pgfpathlineto{\pgfqpoint{4.414600in}{2.626386in}}%
\pgfpathlineto{\pgfqpoint{4.414600in}{2.623437in}}%
\pgfpathmoveto{\pgfqpoint{4.410059in}{2.626386in}}%
\pgfpathlineto{\pgfqpoint{4.410059in}{2.626386in}}%
\pgfpathlineto{\pgfqpoint{4.410059in}{2.629335in}}%
\pgfpathlineto{\pgfqpoint{4.414600in}{2.629335in}}%
\pgfpathlineto{\pgfqpoint{4.414600in}{2.626386in}}%
\pgfpathmoveto{\pgfqpoint{4.414600in}{2.623437in}}%
\pgfpathlineto{\pgfqpoint{4.414600in}{2.623437in}}%
\pgfpathlineto{\pgfqpoint{4.414600in}{2.626386in}}%
\pgfpathlineto{\pgfqpoint{4.419141in}{2.626386in}}%
\pgfpathlineto{\pgfqpoint{4.419141in}{2.623437in}}%
\pgfpathmoveto{\pgfqpoint{4.419141in}{2.623437in}}%
\pgfpathlineto{\pgfqpoint{4.419141in}{2.623437in}}%
\pgfpathlineto{\pgfqpoint{4.419141in}{2.626386in}}%
\pgfpathlineto{\pgfqpoint{4.423682in}{2.626386in}}%
\pgfpathlineto{\pgfqpoint{4.423682in}{2.623437in}}%
\pgfpathmoveto{\pgfqpoint{4.423682in}{2.623437in}}%
\pgfpathlineto{\pgfqpoint{4.423682in}{2.623437in}}%
\pgfpathlineto{\pgfqpoint{4.423682in}{2.626386in}}%
\pgfpathlineto{\pgfqpoint{4.428223in}{2.626386in}}%
\pgfpathlineto{\pgfqpoint{4.428223in}{2.623437in}}%
\pgfpathmoveto{\pgfqpoint{4.455470in}{2.614589in}}%
\pgfpathlineto{\pgfqpoint{4.455470in}{2.614589in}}%
\pgfpathlineto{\pgfqpoint{4.455470in}{2.617538in}}%
\pgfpathlineto{\pgfqpoint{4.460011in}{2.617538in}}%
\pgfpathlineto{\pgfqpoint{4.460011in}{2.614589in}}%
\pgfpathmoveto{\pgfqpoint{4.460011in}{2.614589in}}%
\pgfpathlineto{\pgfqpoint{4.460011in}{2.614589in}}%
\pgfpathlineto{\pgfqpoint{4.460011in}{2.617538in}}%
\pgfpathlineto{\pgfqpoint{4.464552in}{2.617538in}}%
\pgfpathlineto{\pgfqpoint{4.464552in}{2.614589in}}%
\pgfpathmoveto{\pgfqpoint{4.464552in}{2.611639in}}%
\pgfpathlineto{\pgfqpoint{4.464552in}{2.611639in}}%
\pgfpathlineto{\pgfqpoint{4.464552in}{2.614589in}}%
\pgfpathlineto{\pgfqpoint{4.469093in}{2.614589in}}%
\pgfpathlineto{\pgfqpoint{4.469093in}{2.611639in}}%
\pgfpathmoveto{\pgfqpoint{4.464552in}{2.614589in}}%
\pgfpathlineto{\pgfqpoint{4.464552in}{2.614589in}}%
\pgfpathlineto{\pgfqpoint{4.464552in}{2.617538in}}%
\pgfpathlineto{\pgfqpoint{4.469093in}{2.617538in}}%
\pgfpathlineto{\pgfqpoint{4.469093in}{2.614589in}}%
\pgfpathmoveto{\pgfqpoint{4.469093in}{2.611639in}}%
\pgfpathlineto{\pgfqpoint{4.469093in}{2.611639in}}%
\pgfpathlineto{\pgfqpoint{4.469093in}{2.614589in}}%
\pgfpathlineto{\pgfqpoint{4.473634in}{2.614589in}}%
\pgfpathlineto{\pgfqpoint{4.473634in}{2.611639in}}%
\pgfpathmoveto{\pgfqpoint{4.478175in}{2.608690in}}%
\pgfpathlineto{\pgfqpoint{4.478175in}{2.608690in}}%
\pgfpathlineto{\pgfqpoint{4.478175in}{2.611639in}}%
\pgfpathlineto{\pgfqpoint{4.482716in}{2.611639in}}%
\pgfpathlineto{\pgfqpoint{4.482716in}{2.608690in}}%
\pgfpathmoveto{\pgfqpoint{4.482716in}{2.608690in}}%
\pgfpathlineto{\pgfqpoint{4.482716in}{2.608690in}}%
\pgfpathlineto{\pgfqpoint{4.482716in}{2.611639in}}%
\pgfpathlineto{\pgfqpoint{4.487257in}{2.611639in}}%
\pgfpathlineto{\pgfqpoint{4.487257in}{2.608690in}}%
\pgfpathmoveto{\pgfqpoint{4.487257in}{2.608690in}}%
\pgfpathlineto{\pgfqpoint{4.487257in}{2.608690in}}%
\pgfpathlineto{\pgfqpoint{4.487257in}{2.611639in}}%
\pgfpathlineto{\pgfqpoint{4.491798in}{2.611639in}}%
\pgfpathlineto{\pgfqpoint{4.491798in}{2.608690in}}%
\pgfpathmoveto{\pgfqpoint{4.473634in}{2.611639in}}%
\pgfpathlineto{\pgfqpoint{4.473634in}{2.611639in}}%
\pgfpathlineto{\pgfqpoint{4.473634in}{2.614589in}}%
\pgfpathlineto{\pgfqpoint{4.478175in}{2.614589in}}%
\pgfpathlineto{\pgfqpoint{4.478175in}{2.611639in}}%
\pgfpathmoveto{\pgfqpoint{4.478175in}{2.611639in}}%
\pgfpathlineto{\pgfqpoint{4.478175in}{2.611639in}}%
\pgfpathlineto{\pgfqpoint{4.478175in}{2.614589in}}%
\pgfpathlineto{\pgfqpoint{4.482716in}{2.614589in}}%
\pgfpathlineto{\pgfqpoint{4.482716in}{2.611639in}}%
\pgfpathmoveto{\pgfqpoint{4.491798in}{2.605741in}}%
\pgfpathlineto{\pgfqpoint{4.491798in}{2.605741in}}%
\pgfpathlineto{\pgfqpoint{4.491798in}{2.608690in}}%
\pgfpathlineto{\pgfqpoint{4.496339in}{2.608690in}}%
\pgfpathlineto{\pgfqpoint{4.496339in}{2.605741in}}%
\pgfpathmoveto{\pgfqpoint{4.491798in}{2.608690in}}%
\pgfpathlineto{\pgfqpoint{4.491798in}{2.608690in}}%
\pgfpathlineto{\pgfqpoint{4.491798in}{2.611639in}}%
\pgfpathlineto{\pgfqpoint{4.496339in}{2.611639in}}%
\pgfpathlineto{\pgfqpoint{4.496339in}{2.608690in}}%
\pgfpathmoveto{\pgfqpoint{4.496339in}{2.605741in}}%
\pgfpathlineto{\pgfqpoint{4.496339in}{2.605741in}}%
\pgfpathlineto{\pgfqpoint{4.496339in}{2.608690in}}%
\pgfpathlineto{\pgfqpoint{4.500880in}{2.608690in}}%
\pgfpathlineto{\pgfqpoint{4.500880in}{2.605741in}}%
\pgfpathmoveto{\pgfqpoint{4.505421in}{2.602792in}}%
\pgfpathlineto{\pgfqpoint{4.505421in}{2.602792in}}%
\pgfpathlineto{\pgfqpoint{4.505421in}{2.605741in}}%
\pgfpathlineto{\pgfqpoint{4.509962in}{2.605741in}}%
\pgfpathlineto{\pgfqpoint{4.509962in}{2.602792in}}%
\pgfpathmoveto{\pgfqpoint{4.500880in}{2.605741in}}%
\pgfpathlineto{\pgfqpoint{4.500880in}{2.605741in}}%
\pgfpathlineto{\pgfqpoint{4.500880in}{2.608690in}}%
\pgfpathlineto{\pgfqpoint{4.505421in}{2.608690in}}%
\pgfpathlineto{\pgfqpoint{4.505421in}{2.605741in}}%
\pgfpathmoveto{\pgfqpoint{4.505421in}{2.605741in}}%
\pgfpathlineto{\pgfqpoint{4.505421in}{2.605741in}}%
\pgfpathlineto{\pgfqpoint{4.505421in}{2.608690in}}%
\pgfpathlineto{\pgfqpoint{4.509962in}{2.608690in}}%
\pgfpathlineto{\pgfqpoint{4.509962in}{2.605741in}}%
\pgfpathmoveto{\pgfqpoint{4.509962in}{2.602792in}}%
\pgfpathlineto{\pgfqpoint{4.509962in}{2.602792in}}%
\pgfpathlineto{\pgfqpoint{4.509962in}{2.605741in}}%
\pgfpathlineto{\pgfqpoint{4.514503in}{2.605741in}}%
\pgfpathlineto{\pgfqpoint{4.514503in}{2.602792in}}%
\pgfpathmoveto{\pgfqpoint{4.514503in}{2.602792in}}%
\pgfpathlineto{\pgfqpoint{4.514503in}{2.602792in}}%
\pgfpathlineto{\pgfqpoint{4.514503in}{2.605741in}}%
\pgfpathlineto{\pgfqpoint{4.519044in}{2.605741in}}%
\pgfpathlineto{\pgfqpoint{4.519044in}{2.602792in}}%
\pgfpathmoveto{\pgfqpoint{4.519044in}{2.599842in}}%
\pgfpathlineto{\pgfqpoint{4.519044in}{2.599842in}}%
\pgfpathlineto{\pgfqpoint{4.519044in}{2.602792in}}%
\pgfpathlineto{\pgfqpoint{4.523585in}{2.602792in}}%
\pgfpathlineto{\pgfqpoint{4.523585in}{2.599842in}}%
\pgfpathmoveto{\pgfqpoint{4.519044in}{2.602792in}}%
\pgfpathlineto{\pgfqpoint{4.519044in}{2.602792in}}%
\pgfpathlineto{\pgfqpoint{4.519044in}{2.605741in}}%
\pgfpathlineto{\pgfqpoint{4.523585in}{2.605741in}}%
\pgfpathlineto{\pgfqpoint{4.523585in}{2.602792in}}%
\pgfpathmoveto{\pgfqpoint{4.523585in}{2.599842in}}%
\pgfpathlineto{\pgfqpoint{4.523585in}{2.599842in}}%
\pgfpathlineto{\pgfqpoint{4.523585in}{2.602792in}}%
\pgfpathlineto{\pgfqpoint{4.528126in}{2.602792in}}%
\pgfpathlineto{\pgfqpoint{4.528126in}{2.599842in}}%
\pgfpathmoveto{\pgfqpoint{4.641653in}{2.573299in}}%
\pgfpathlineto{\pgfqpoint{4.641653in}{2.573299in}}%
\pgfpathlineto{\pgfqpoint{4.641653in}{2.576248in}}%
\pgfpathlineto{\pgfqpoint{4.646194in}{2.576248in}}%
\pgfpathlineto{\pgfqpoint{4.646194in}{2.573299in}}%
\pgfpathmoveto{\pgfqpoint{4.646194in}{2.573299in}}%
\pgfpathlineto{\pgfqpoint{4.646194in}{2.573299in}}%
\pgfpathlineto{\pgfqpoint{4.646194in}{2.576248in}}%
\pgfpathlineto{\pgfqpoint{4.650735in}{2.576248in}}%
\pgfpathlineto{\pgfqpoint{4.650735in}{2.573299in}}%
\pgfpathmoveto{\pgfqpoint{4.650735in}{2.573299in}}%
\pgfpathlineto{\pgfqpoint{4.650735in}{2.573299in}}%
\pgfpathlineto{\pgfqpoint{4.650735in}{2.576248in}}%
\pgfpathlineto{\pgfqpoint{4.655276in}{2.576248in}}%
\pgfpathlineto{\pgfqpoint{4.655276in}{2.573299in}}%
\pgfpathmoveto{\pgfqpoint{4.655276in}{2.570349in}}%
\pgfpathlineto{\pgfqpoint{4.655276in}{2.570349in}}%
\pgfpathlineto{\pgfqpoint{4.655276in}{2.573299in}}%
\pgfpathlineto{\pgfqpoint{4.659817in}{2.573299in}}%
\pgfpathlineto{\pgfqpoint{4.659817in}{2.570349in}}%
\pgfpathmoveto{\pgfqpoint{4.655276in}{2.573299in}}%
\pgfpathlineto{\pgfqpoint{4.655276in}{2.573299in}}%
\pgfpathlineto{\pgfqpoint{4.655276in}{2.576248in}}%
\pgfpathlineto{\pgfqpoint{4.659817in}{2.576248in}}%
\pgfpathlineto{\pgfqpoint{4.659817in}{2.573299in}}%
\pgfpathmoveto{\pgfqpoint{4.659817in}{2.570349in}}%
\pgfpathlineto{\pgfqpoint{4.659817in}{2.570349in}}%
\pgfpathlineto{\pgfqpoint{4.659817in}{2.573299in}}%
\pgfpathlineto{\pgfqpoint{4.664359in}{2.573299in}}%
\pgfpathlineto{\pgfqpoint{4.664359in}{2.570349in}}%
\pgfpathmoveto{\pgfqpoint{4.668900in}{2.567400in}}%
\pgfpathlineto{\pgfqpoint{4.668900in}{2.567400in}}%
\pgfpathlineto{\pgfqpoint{4.668900in}{2.570349in}}%
\pgfpathlineto{\pgfqpoint{4.673441in}{2.570349in}}%
\pgfpathlineto{\pgfqpoint{4.673441in}{2.567400in}}%
\pgfpathmoveto{\pgfqpoint{4.664359in}{2.570349in}}%
\pgfpathlineto{\pgfqpoint{4.664359in}{2.570349in}}%
\pgfpathlineto{\pgfqpoint{4.664359in}{2.573299in}}%
\pgfpathlineto{\pgfqpoint{4.668900in}{2.573299in}}%
\pgfpathlineto{\pgfqpoint{4.668900in}{2.570349in}}%
\pgfpathmoveto{\pgfqpoint{4.668900in}{2.570349in}}%
\pgfpathlineto{\pgfqpoint{4.668900in}{2.570349in}}%
\pgfpathlineto{\pgfqpoint{4.668900in}{2.573299in}}%
\pgfpathlineto{\pgfqpoint{4.673441in}{2.573299in}}%
\pgfpathlineto{\pgfqpoint{4.673441in}{2.570349in}}%
\pgfpathmoveto{\pgfqpoint{4.532668in}{2.596893in}}%
\pgfpathlineto{\pgfqpoint{4.532668in}{2.596893in}}%
\pgfpathlineto{\pgfqpoint{4.532668in}{2.599842in}}%
\pgfpathlineto{\pgfqpoint{4.537209in}{2.599842in}}%
\pgfpathlineto{\pgfqpoint{4.537209in}{2.596893in}}%
\pgfpathmoveto{\pgfqpoint{4.537209in}{2.596893in}}%
\pgfpathlineto{\pgfqpoint{4.537209in}{2.596893in}}%
\pgfpathlineto{\pgfqpoint{4.537209in}{2.599842in}}%
\pgfpathlineto{\pgfqpoint{4.541750in}{2.599842in}}%
\pgfpathlineto{\pgfqpoint{4.541750in}{2.596893in}}%
\pgfpathmoveto{\pgfqpoint{4.541750in}{2.596893in}}%
\pgfpathlineto{\pgfqpoint{4.541750in}{2.596893in}}%
\pgfpathlineto{\pgfqpoint{4.541750in}{2.599842in}}%
\pgfpathlineto{\pgfqpoint{4.546291in}{2.599842in}}%
\pgfpathlineto{\pgfqpoint{4.546291in}{2.596893in}}%
\pgfpathmoveto{\pgfqpoint{4.546291in}{2.593944in}}%
\pgfpathlineto{\pgfqpoint{4.546291in}{2.593944in}}%
\pgfpathlineto{\pgfqpoint{4.546291in}{2.596893in}}%
\pgfpathlineto{\pgfqpoint{4.550832in}{2.596893in}}%
\pgfpathlineto{\pgfqpoint{4.550832in}{2.593944in}}%
\pgfpathmoveto{\pgfqpoint{4.546291in}{2.596893in}}%
\pgfpathlineto{\pgfqpoint{4.546291in}{2.596893in}}%
\pgfpathlineto{\pgfqpoint{4.546291in}{2.599842in}}%
\pgfpathlineto{\pgfqpoint{4.550832in}{2.599842in}}%
\pgfpathlineto{\pgfqpoint{4.550832in}{2.596893in}}%
\pgfpathmoveto{\pgfqpoint{4.550832in}{2.593944in}}%
\pgfpathlineto{\pgfqpoint{4.550832in}{2.593944in}}%
\pgfpathlineto{\pgfqpoint{4.550832in}{2.596893in}}%
\pgfpathlineto{\pgfqpoint{4.555373in}{2.596893in}}%
\pgfpathlineto{\pgfqpoint{4.555373in}{2.593944in}}%
\pgfpathmoveto{\pgfqpoint{4.559914in}{2.590994in}}%
\pgfpathlineto{\pgfqpoint{4.559914in}{2.590994in}}%
\pgfpathlineto{\pgfqpoint{4.559914in}{2.593944in}}%
\pgfpathlineto{\pgfqpoint{4.564455in}{2.593944in}}%
\pgfpathlineto{\pgfqpoint{4.564455in}{2.590994in}}%
\pgfpathmoveto{\pgfqpoint{4.555373in}{2.593944in}}%
\pgfpathlineto{\pgfqpoint{4.555373in}{2.593944in}}%
\pgfpathlineto{\pgfqpoint{4.555373in}{2.596893in}}%
\pgfpathlineto{\pgfqpoint{4.559914in}{2.596893in}}%
\pgfpathlineto{\pgfqpoint{4.559914in}{2.593944in}}%
\pgfpathmoveto{\pgfqpoint{4.559914in}{2.593944in}}%
\pgfpathlineto{\pgfqpoint{4.559914in}{2.593944in}}%
\pgfpathlineto{\pgfqpoint{4.559914in}{2.596893in}}%
\pgfpathlineto{\pgfqpoint{4.564455in}{2.596893in}}%
\pgfpathlineto{\pgfqpoint{4.564455in}{2.593944in}}%
\pgfpathmoveto{\pgfqpoint{4.528126in}{2.599842in}}%
\pgfpathlineto{\pgfqpoint{4.528126in}{2.599842in}}%
\pgfpathlineto{\pgfqpoint{4.528126in}{2.602792in}}%
\pgfpathlineto{\pgfqpoint{4.532668in}{2.602792in}}%
\pgfpathlineto{\pgfqpoint{4.532668in}{2.599842in}}%
\pgfpathmoveto{\pgfqpoint{4.532668in}{2.599842in}}%
\pgfpathlineto{\pgfqpoint{4.532668in}{2.599842in}}%
\pgfpathlineto{\pgfqpoint{4.532668in}{2.602792in}}%
\pgfpathlineto{\pgfqpoint{4.537209in}{2.602792in}}%
\pgfpathlineto{\pgfqpoint{4.537209in}{2.599842in}}%
\pgfpathmoveto{\pgfqpoint{4.564455in}{2.590994in}}%
\pgfpathlineto{\pgfqpoint{4.564455in}{2.590994in}}%
\pgfpathlineto{\pgfqpoint{4.564455in}{2.593944in}}%
\pgfpathlineto{\pgfqpoint{4.568996in}{2.593944in}}%
\pgfpathlineto{\pgfqpoint{4.568996in}{2.590994in}}%
\pgfpathmoveto{\pgfqpoint{4.568996in}{2.590994in}}%
\pgfpathlineto{\pgfqpoint{4.568996in}{2.590994in}}%
\pgfpathlineto{\pgfqpoint{4.568996in}{2.593944in}}%
\pgfpathlineto{\pgfqpoint{4.573537in}{2.593944in}}%
\pgfpathlineto{\pgfqpoint{4.573537in}{2.590994in}}%
\pgfpathmoveto{\pgfqpoint{4.573537in}{2.588045in}}%
\pgfpathlineto{\pgfqpoint{4.573537in}{2.588045in}}%
\pgfpathlineto{\pgfqpoint{4.573537in}{2.590994in}}%
\pgfpathlineto{\pgfqpoint{4.578078in}{2.590994in}}%
\pgfpathlineto{\pgfqpoint{4.578078in}{2.588045in}}%
\pgfpathmoveto{\pgfqpoint{4.573537in}{2.590994in}}%
\pgfpathlineto{\pgfqpoint{4.573537in}{2.590994in}}%
\pgfpathlineto{\pgfqpoint{4.573537in}{2.593944in}}%
\pgfpathlineto{\pgfqpoint{4.578078in}{2.593944in}}%
\pgfpathlineto{\pgfqpoint{4.578078in}{2.590994in}}%
\pgfpathmoveto{\pgfqpoint{4.578078in}{2.588045in}}%
\pgfpathlineto{\pgfqpoint{4.578078in}{2.588045in}}%
\pgfpathlineto{\pgfqpoint{4.578078in}{2.590994in}}%
\pgfpathlineto{\pgfqpoint{4.582619in}{2.590994in}}%
\pgfpathlineto{\pgfqpoint{4.582619in}{2.588045in}}%
\pgfpathmoveto{\pgfqpoint{4.587160in}{2.585096in}}%
\pgfpathlineto{\pgfqpoint{4.587160in}{2.585096in}}%
\pgfpathlineto{\pgfqpoint{4.587160in}{2.588045in}}%
\pgfpathlineto{\pgfqpoint{4.591701in}{2.588045in}}%
\pgfpathlineto{\pgfqpoint{4.591701in}{2.585096in}}%
\pgfpathmoveto{\pgfqpoint{4.591701in}{2.585096in}}%
\pgfpathlineto{\pgfqpoint{4.591701in}{2.585096in}}%
\pgfpathlineto{\pgfqpoint{4.591701in}{2.588045in}}%
\pgfpathlineto{\pgfqpoint{4.596243in}{2.588045in}}%
\pgfpathlineto{\pgfqpoint{4.596243in}{2.585096in}}%
\pgfpathmoveto{\pgfqpoint{4.596243in}{2.585096in}}%
\pgfpathlineto{\pgfqpoint{4.596243in}{2.585096in}}%
\pgfpathlineto{\pgfqpoint{4.596243in}{2.588045in}}%
\pgfpathlineto{\pgfqpoint{4.600784in}{2.588045in}}%
\pgfpathlineto{\pgfqpoint{4.600784in}{2.585096in}}%
\pgfpathmoveto{\pgfqpoint{4.582619in}{2.588045in}}%
\pgfpathlineto{\pgfqpoint{4.582619in}{2.588045in}}%
\pgfpathlineto{\pgfqpoint{4.582619in}{2.590994in}}%
\pgfpathlineto{\pgfqpoint{4.587160in}{2.590994in}}%
\pgfpathlineto{\pgfqpoint{4.587160in}{2.588045in}}%
\pgfpathmoveto{\pgfqpoint{4.587160in}{2.588045in}}%
\pgfpathlineto{\pgfqpoint{4.587160in}{2.588045in}}%
\pgfpathlineto{\pgfqpoint{4.587160in}{2.590994in}}%
\pgfpathlineto{\pgfqpoint{4.591701in}{2.590994in}}%
\pgfpathlineto{\pgfqpoint{4.591701in}{2.588045in}}%
\pgfpathmoveto{\pgfqpoint{4.600784in}{2.582146in}}%
\pgfpathlineto{\pgfqpoint{4.600784in}{2.582146in}}%
\pgfpathlineto{\pgfqpoint{4.600784in}{2.585096in}}%
\pgfpathlineto{\pgfqpoint{4.605325in}{2.585096in}}%
\pgfpathlineto{\pgfqpoint{4.605325in}{2.582146in}}%
\pgfpathmoveto{\pgfqpoint{4.600784in}{2.585096in}}%
\pgfpathlineto{\pgfqpoint{4.600784in}{2.585096in}}%
\pgfpathlineto{\pgfqpoint{4.600784in}{2.588045in}}%
\pgfpathlineto{\pgfqpoint{4.605325in}{2.588045in}}%
\pgfpathlineto{\pgfqpoint{4.605325in}{2.585096in}}%
\pgfpathmoveto{\pgfqpoint{4.605325in}{2.582146in}}%
\pgfpathlineto{\pgfqpoint{4.605325in}{2.582146in}}%
\pgfpathlineto{\pgfqpoint{4.605325in}{2.585096in}}%
\pgfpathlineto{\pgfqpoint{4.609866in}{2.585096in}}%
\pgfpathlineto{\pgfqpoint{4.609866in}{2.582146in}}%
\pgfpathmoveto{\pgfqpoint{4.614407in}{2.579197in}}%
\pgfpathlineto{\pgfqpoint{4.614407in}{2.579197in}}%
\pgfpathlineto{\pgfqpoint{4.614407in}{2.582146in}}%
\pgfpathlineto{\pgfqpoint{4.618948in}{2.582146in}}%
\pgfpathlineto{\pgfqpoint{4.618948in}{2.579197in}}%
\pgfpathmoveto{\pgfqpoint{4.609866in}{2.582146in}}%
\pgfpathlineto{\pgfqpoint{4.609866in}{2.582146in}}%
\pgfpathlineto{\pgfqpoint{4.609866in}{2.585096in}}%
\pgfpathlineto{\pgfqpoint{4.614407in}{2.585096in}}%
\pgfpathlineto{\pgfqpoint{4.614407in}{2.582146in}}%
\pgfpathmoveto{\pgfqpoint{4.614407in}{2.582146in}}%
\pgfpathlineto{\pgfqpoint{4.614407in}{2.582146in}}%
\pgfpathlineto{\pgfqpoint{4.614407in}{2.585096in}}%
\pgfpathlineto{\pgfqpoint{4.618948in}{2.585096in}}%
\pgfpathlineto{\pgfqpoint{4.618948in}{2.582146in}}%
\pgfpathmoveto{\pgfqpoint{4.618948in}{2.579197in}}%
\pgfpathlineto{\pgfqpoint{4.618948in}{2.579197in}}%
\pgfpathlineto{\pgfqpoint{4.618948in}{2.582146in}}%
\pgfpathlineto{\pgfqpoint{4.623489in}{2.582146in}}%
\pgfpathlineto{\pgfqpoint{4.623489in}{2.579197in}}%
\pgfpathmoveto{\pgfqpoint{4.623489in}{2.579197in}}%
\pgfpathlineto{\pgfqpoint{4.623489in}{2.579197in}}%
\pgfpathlineto{\pgfqpoint{4.623489in}{2.582146in}}%
\pgfpathlineto{\pgfqpoint{4.628030in}{2.582146in}}%
\pgfpathlineto{\pgfqpoint{4.628030in}{2.579197in}}%
\pgfpathmoveto{\pgfqpoint{4.628030in}{2.576248in}}%
\pgfpathlineto{\pgfqpoint{4.628030in}{2.576248in}}%
\pgfpathlineto{\pgfqpoint{4.628030in}{2.579197in}}%
\pgfpathlineto{\pgfqpoint{4.632571in}{2.579197in}}%
\pgfpathlineto{\pgfqpoint{4.632571in}{2.576248in}}%
\pgfpathmoveto{\pgfqpoint{4.628030in}{2.579197in}}%
\pgfpathlineto{\pgfqpoint{4.628030in}{2.579197in}}%
\pgfpathlineto{\pgfqpoint{4.628030in}{2.582146in}}%
\pgfpathlineto{\pgfqpoint{4.632571in}{2.582146in}}%
\pgfpathlineto{\pgfqpoint{4.632571in}{2.579197in}}%
\pgfpathmoveto{\pgfqpoint{4.632571in}{2.576248in}}%
\pgfpathlineto{\pgfqpoint{4.632571in}{2.576248in}}%
\pgfpathlineto{\pgfqpoint{4.632571in}{2.579197in}}%
\pgfpathlineto{\pgfqpoint{4.637112in}{2.579197in}}%
\pgfpathlineto{\pgfqpoint{4.637112in}{2.576248in}}%
\pgfpathmoveto{\pgfqpoint{4.637112in}{2.576248in}}%
\pgfpathlineto{\pgfqpoint{4.637112in}{2.576248in}}%
\pgfpathlineto{\pgfqpoint{4.637112in}{2.579197in}}%
\pgfpathlineto{\pgfqpoint{4.641653in}{2.579197in}}%
\pgfpathlineto{\pgfqpoint{4.641653in}{2.576248in}}%
\pgfpathmoveto{\pgfqpoint{4.641653in}{2.576248in}}%
\pgfpathlineto{\pgfqpoint{4.641653in}{2.576248in}}%
\pgfpathlineto{\pgfqpoint{4.641653in}{2.579197in}}%
\pgfpathlineto{\pgfqpoint{4.646194in}{2.579197in}}%
\pgfpathlineto{\pgfqpoint{4.646194in}{2.576248in}}%
\pgfpathmoveto{\pgfqpoint{4.673441in}{2.567400in}}%
\pgfpathlineto{\pgfqpoint{4.673441in}{2.567400in}}%
\pgfpathlineto{\pgfqpoint{4.673441in}{2.570349in}}%
\pgfpathlineto{\pgfqpoint{4.677982in}{2.570349in}}%
\pgfpathlineto{\pgfqpoint{4.677982in}{2.567400in}}%
\pgfpathmoveto{\pgfqpoint{4.677982in}{2.567400in}}%
\pgfpathlineto{\pgfqpoint{4.677982in}{2.567400in}}%
\pgfpathlineto{\pgfqpoint{4.677982in}{2.570349in}}%
\pgfpathlineto{\pgfqpoint{4.682523in}{2.570349in}}%
\pgfpathlineto{\pgfqpoint{4.682523in}{2.567400in}}%
\pgfpathmoveto{\pgfqpoint{4.682523in}{2.564451in}}%
\pgfpathlineto{\pgfqpoint{4.682523in}{2.564451in}}%
\pgfpathlineto{\pgfqpoint{4.682523in}{2.567400in}}%
\pgfpathlineto{\pgfqpoint{4.687063in}{2.567400in}}%
\pgfpathlineto{\pgfqpoint{4.687063in}{2.564451in}}%
\pgfpathmoveto{\pgfqpoint{4.682523in}{2.567400in}}%
\pgfpathlineto{\pgfqpoint{4.682523in}{2.567400in}}%
\pgfpathlineto{\pgfqpoint{4.682523in}{2.570349in}}%
\pgfpathlineto{\pgfqpoint{4.687063in}{2.570349in}}%
\pgfpathlineto{\pgfqpoint{4.687063in}{2.567400in}}%
\pgfpathmoveto{\pgfqpoint{4.687063in}{2.564451in}}%
\pgfpathlineto{\pgfqpoint{4.687063in}{2.564451in}}%
\pgfpathlineto{\pgfqpoint{4.687063in}{2.567400in}}%
\pgfpathlineto{\pgfqpoint{4.691604in}{2.567400in}}%
\pgfpathlineto{\pgfqpoint{4.691604in}{2.564451in}}%
\pgfpathmoveto{\pgfqpoint{4.696145in}{2.561502in}}%
\pgfpathlineto{\pgfqpoint{4.696145in}{2.561502in}}%
\pgfpathlineto{\pgfqpoint{4.696145in}{2.564451in}}%
\pgfpathlineto{\pgfqpoint{4.700686in}{2.564451in}}%
\pgfpathlineto{\pgfqpoint{4.700686in}{2.561502in}}%
\pgfpathmoveto{\pgfqpoint{4.700686in}{2.561502in}}%
\pgfpathlineto{\pgfqpoint{4.700686in}{2.561502in}}%
\pgfpathlineto{\pgfqpoint{4.700686in}{2.564451in}}%
\pgfpathlineto{\pgfqpoint{4.705227in}{2.564451in}}%
\pgfpathlineto{\pgfqpoint{4.705227in}{2.561502in}}%
\pgfpathmoveto{\pgfqpoint{4.705227in}{2.561502in}}%
\pgfpathlineto{\pgfqpoint{4.705227in}{2.561502in}}%
\pgfpathlineto{\pgfqpoint{4.705227in}{2.564451in}}%
\pgfpathlineto{\pgfqpoint{4.709768in}{2.564451in}}%
\pgfpathlineto{\pgfqpoint{4.709768in}{2.561502in}}%
\pgfpathmoveto{\pgfqpoint{4.691604in}{2.564451in}}%
\pgfpathlineto{\pgfqpoint{4.691604in}{2.564451in}}%
\pgfpathlineto{\pgfqpoint{4.691604in}{2.567400in}}%
\pgfpathlineto{\pgfqpoint{4.696145in}{2.567400in}}%
\pgfpathlineto{\pgfqpoint{4.696145in}{2.564451in}}%
\pgfpathmoveto{\pgfqpoint{4.696145in}{2.564451in}}%
\pgfpathlineto{\pgfqpoint{4.696145in}{2.564451in}}%
\pgfpathlineto{\pgfqpoint{4.696145in}{2.567400in}}%
\pgfpathlineto{\pgfqpoint{4.700686in}{2.567400in}}%
\pgfpathlineto{\pgfqpoint{4.700686in}{2.564451in}}%
\pgfpathmoveto{\pgfqpoint{4.709768in}{2.558553in}}%
\pgfpathlineto{\pgfqpoint{4.709768in}{2.558553in}}%
\pgfpathlineto{\pgfqpoint{4.709768in}{2.561502in}}%
\pgfpathlineto{\pgfqpoint{4.714309in}{2.561502in}}%
\pgfpathlineto{\pgfqpoint{4.714309in}{2.558553in}}%
\pgfpathmoveto{\pgfqpoint{4.709768in}{2.561502in}}%
\pgfpathlineto{\pgfqpoint{4.709768in}{2.561502in}}%
\pgfpathlineto{\pgfqpoint{4.709768in}{2.564451in}}%
\pgfpathlineto{\pgfqpoint{4.714309in}{2.564451in}}%
\pgfpathlineto{\pgfqpoint{4.714309in}{2.561502in}}%
\pgfpathmoveto{\pgfqpoint{4.714309in}{2.558553in}}%
\pgfpathlineto{\pgfqpoint{4.714309in}{2.558553in}}%
\pgfpathlineto{\pgfqpoint{4.714309in}{2.561502in}}%
\pgfpathlineto{\pgfqpoint{4.718850in}{2.561502in}}%
\pgfpathlineto{\pgfqpoint{4.718850in}{2.558553in}}%
\pgfpathmoveto{\pgfqpoint{4.723391in}{2.555604in}}%
\pgfpathlineto{\pgfqpoint{4.723391in}{2.555604in}}%
\pgfpathlineto{\pgfqpoint{4.723391in}{2.558553in}}%
\pgfpathlineto{\pgfqpoint{4.727932in}{2.558553in}}%
\pgfpathlineto{\pgfqpoint{4.727932in}{2.555604in}}%
\pgfpathmoveto{\pgfqpoint{4.718850in}{2.558553in}}%
\pgfpathlineto{\pgfqpoint{4.718850in}{2.558553in}}%
\pgfpathlineto{\pgfqpoint{4.718850in}{2.561502in}}%
\pgfpathlineto{\pgfqpoint{4.723391in}{2.561502in}}%
\pgfpathlineto{\pgfqpoint{4.723391in}{2.558553in}}%
\pgfpathmoveto{\pgfqpoint{4.723391in}{2.558553in}}%
\pgfpathlineto{\pgfqpoint{4.723391in}{2.558553in}}%
\pgfpathlineto{\pgfqpoint{4.723391in}{2.561502in}}%
\pgfpathlineto{\pgfqpoint{4.727932in}{2.561502in}}%
\pgfpathlineto{\pgfqpoint{4.727932in}{2.558553in}}%
\pgfpathmoveto{\pgfqpoint{4.727932in}{2.555604in}}%
\pgfpathlineto{\pgfqpoint{4.727932in}{2.555604in}}%
\pgfpathlineto{\pgfqpoint{4.727932in}{2.558553in}}%
\pgfpathlineto{\pgfqpoint{4.732473in}{2.558553in}}%
\pgfpathlineto{\pgfqpoint{4.732473in}{2.555604in}}%
\pgfpathmoveto{\pgfqpoint{4.732473in}{2.555604in}}%
\pgfpathlineto{\pgfqpoint{4.732473in}{2.555604in}}%
\pgfpathlineto{\pgfqpoint{4.732473in}{2.558553in}}%
\pgfpathlineto{\pgfqpoint{4.737014in}{2.558553in}}%
\pgfpathlineto{\pgfqpoint{4.737014in}{2.555604in}}%
\pgfpathmoveto{\pgfqpoint{4.737014in}{2.552655in}}%
\pgfpathlineto{\pgfqpoint{4.737014in}{2.552655in}}%
\pgfpathlineto{\pgfqpoint{4.737014in}{2.555604in}}%
\pgfpathlineto{\pgfqpoint{4.741555in}{2.555604in}}%
\pgfpathlineto{\pgfqpoint{4.741555in}{2.552655in}}%
\pgfpathmoveto{\pgfqpoint{4.737014in}{2.555604in}}%
\pgfpathlineto{\pgfqpoint{4.737014in}{2.555604in}}%
\pgfpathlineto{\pgfqpoint{4.737014in}{2.558553in}}%
\pgfpathlineto{\pgfqpoint{4.741555in}{2.558553in}}%
\pgfpathlineto{\pgfqpoint{4.741555in}{2.555604in}}%
\pgfpathmoveto{\pgfqpoint{4.741555in}{2.552655in}}%
\pgfpathlineto{\pgfqpoint{4.741555in}{2.552655in}}%
\pgfpathlineto{\pgfqpoint{4.741555in}{2.555604in}}%
\pgfpathlineto{\pgfqpoint{4.746096in}{2.555604in}}%
\pgfpathlineto{\pgfqpoint{4.746096in}{2.552655in}}%
\pgfpathmoveto{\pgfqpoint{4.750636in}{2.549706in}}%
\pgfpathlineto{\pgfqpoint{4.750636in}{2.549706in}}%
\pgfpathlineto{\pgfqpoint{4.750636in}{2.552655in}}%
\pgfpathlineto{\pgfqpoint{4.755177in}{2.552655in}}%
\pgfpathlineto{\pgfqpoint{4.755177in}{2.549706in}}%
\pgfpathmoveto{\pgfqpoint{4.755177in}{2.549706in}}%
\pgfpathlineto{\pgfqpoint{4.755177in}{2.549706in}}%
\pgfpathlineto{\pgfqpoint{4.755177in}{2.552655in}}%
\pgfpathlineto{\pgfqpoint{4.759718in}{2.552655in}}%
\pgfpathlineto{\pgfqpoint{4.759718in}{2.549706in}}%
\pgfpathmoveto{\pgfqpoint{4.759718in}{2.549706in}}%
\pgfpathlineto{\pgfqpoint{4.759718in}{2.549706in}}%
\pgfpathlineto{\pgfqpoint{4.759718in}{2.552655in}}%
\pgfpathlineto{\pgfqpoint{4.764259in}{2.552655in}}%
\pgfpathlineto{\pgfqpoint{4.764259in}{2.549706in}}%
\pgfpathmoveto{\pgfqpoint{4.764259in}{2.546757in}}%
\pgfpathlineto{\pgfqpoint{4.764259in}{2.546757in}}%
\pgfpathlineto{\pgfqpoint{4.764259in}{2.549706in}}%
\pgfpathlineto{\pgfqpoint{4.768800in}{2.549706in}}%
\pgfpathlineto{\pgfqpoint{4.768800in}{2.546757in}}%
\pgfpathmoveto{\pgfqpoint{4.764259in}{2.549706in}}%
\pgfpathlineto{\pgfqpoint{4.764259in}{2.549706in}}%
\pgfpathlineto{\pgfqpoint{4.764259in}{2.552655in}}%
\pgfpathlineto{\pgfqpoint{4.768800in}{2.552655in}}%
\pgfpathlineto{\pgfqpoint{4.768800in}{2.549706in}}%
\pgfpathmoveto{\pgfqpoint{4.768800in}{2.546757in}}%
\pgfpathlineto{\pgfqpoint{4.768800in}{2.546757in}}%
\pgfpathlineto{\pgfqpoint{4.768800in}{2.549706in}}%
\pgfpathlineto{\pgfqpoint{4.773341in}{2.549706in}}%
\pgfpathlineto{\pgfqpoint{4.773341in}{2.546757in}}%
\pgfpathmoveto{\pgfqpoint{4.777882in}{2.543807in}}%
\pgfpathlineto{\pgfqpoint{4.777882in}{2.543807in}}%
\pgfpathlineto{\pgfqpoint{4.777882in}{2.546757in}}%
\pgfpathlineto{\pgfqpoint{4.782423in}{2.546757in}}%
\pgfpathlineto{\pgfqpoint{4.782423in}{2.543807in}}%
\pgfpathmoveto{\pgfqpoint{4.773341in}{2.546757in}}%
\pgfpathlineto{\pgfqpoint{4.773341in}{2.546757in}}%
\pgfpathlineto{\pgfqpoint{4.773341in}{2.549706in}}%
\pgfpathlineto{\pgfqpoint{4.777882in}{2.549706in}}%
\pgfpathlineto{\pgfqpoint{4.777882in}{2.546757in}}%
\pgfpathmoveto{\pgfqpoint{4.777882in}{2.546757in}}%
\pgfpathlineto{\pgfqpoint{4.777882in}{2.546757in}}%
\pgfpathlineto{\pgfqpoint{4.777882in}{2.549706in}}%
\pgfpathlineto{\pgfqpoint{4.782423in}{2.549706in}}%
\pgfpathlineto{\pgfqpoint{4.782423in}{2.546757in}}%
\pgfpathmoveto{\pgfqpoint{4.746096in}{2.552655in}}%
\pgfpathlineto{\pgfqpoint{4.746096in}{2.552655in}}%
\pgfpathlineto{\pgfqpoint{4.746096in}{2.555604in}}%
\pgfpathlineto{\pgfqpoint{4.750636in}{2.555604in}}%
\pgfpathlineto{\pgfqpoint{4.750636in}{2.552655in}}%
\pgfpathmoveto{\pgfqpoint{4.750636in}{2.552655in}}%
\pgfpathlineto{\pgfqpoint{4.750636in}{2.552655in}}%
\pgfpathlineto{\pgfqpoint{4.750636in}{2.555604in}}%
\pgfpathlineto{\pgfqpoint{4.755177in}{2.555604in}}%
\pgfpathlineto{\pgfqpoint{4.755177in}{2.552655in}}%
\pgfpathmoveto{\pgfqpoint{4.782423in}{2.543807in}}%
\pgfpathlineto{\pgfqpoint{4.782423in}{2.543807in}}%
\pgfpathlineto{\pgfqpoint{4.782423in}{2.546757in}}%
\pgfpathlineto{\pgfqpoint{4.786964in}{2.546757in}}%
\pgfpathlineto{\pgfqpoint{4.786964in}{2.543807in}}%
\pgfpathmoveto{\pgfqpoint{4.786964in}{2.543807in}}%
\pgfpathlineto{\pgfqpoint{4.786964in}{2.543807in}}%
\pgfpathlineto{\pgfqpoint{4.786964in}{2.546757in}}%
\pgfpathlineto{\pgfqpoint{4.791505in}{2.546757in}}%
\pgfpathlineto{\pgfqpoint{4.791505in}{2.543807in}}%
\pgfpathmoveto{\pgfqpoint{4.791505in}{2.540858in}}%
\pgfpathlineto{\pgfqpoint{4.791505in}{2.540858in}}%
\pgfpathlineto{\pgfqpoint{4.791505in}{2.543807in}}%
\pgfpathlineto{\pgfqpoint{4.796046in}{2.543807in}}%
\pgfpathlineto{\pgfqpoint{4.796046in}{2.540858in}}%
\pgfpathmoveto{\pgfqpoint{4.791505in}{2.543807in}}%
\pgfpathlineto{\pgfqpoint{4.791505in}{2.543807in}}%
\pgfpathlineto{\pgfqpoint{4.791505in}{2.546757in}}%
\pgfpathlineto{\pgfqpoint{4.796046in}{2.546757in}}%
\pgfpathlineto{\pgfqpoint{4.796046in}{2.543807in}}%
\pgfpathmoveto{\pgfqpoint{4.796046in}{2.540858in}}%
\pgfpathlineto{\pgfqpoint{4.796046in}{2.540858in}}%
\pgfpathlineto{\pgfqpoint{4.796046in}{2.543807in}}%
\pgfpathlineto{\pgfqpoint{4.800587in}{2.543807in}}%
\pgfpathlineto{\pgfqpoint{4.800587in}{2.540858in}}%
\pgfpathmoveto{\pgfqpoint{4.805128in}{2.537909in}}%
\pgfpathlineto{\pgfqpoint{4.805128in}{2.537909in}}%
\pgfpathlineto{\pgfqpoint{4.805128in}{2.540858in}}%
\pgfpathlineto{\pgfqpoint{4.809669in}{2.540858in}}%
\pgfpathlineto{\pgfqpoint{4.809669in}{2.537909in}}%
\pgfpathmoveto{\pgfqpoint{4.809669in}{2.537909in}}%
\pgfpathlineto{\pgfqpoint{4.809669in}{2.537909in}}%
\pgfpathlineto{\pgfqpoint{4.809669in}{2.540858in}}%
\pgfpathlineto{\pgfqpoint{4.814209in}{2.540858in}}%
\pgfpathlineto{\pgfqpoint{4.814209in}{2.537909in}}%
\pgfpathmoveto{\pgfqpoint{4.814209in}{2.537909in}}%
\pgfpathlineto{\pgfqpoint{4.814209in}{2.537909in}}%
\pgfpathlineto{\pgfqpoint{4.814209in}{2.540858in}}%
\pgfpathlineto{\pgfqpoint{4.818750in}{2.540858in}}%
\pgfpathlineto{\pgfqpoint{4.818750in}{2.537909in}}%
\pgfpathmoveto{\pgfqpoint{4.800587in}{2.540858in}}%
\pgfpathlineto{\pgfqpoint{4.800587in}{2.540858in}}%
\pgfpathlineto{\pgfqpoint{4.800587in}{2.543807in}}%
\pgfpathlineto{\pgfqpoint{4.805128in}{2.543807in}}%
\pgfpathlineto{\pgfqpoint{4.805128in}{2.540858in}}%
\pgfpathmoveto{\pgfqpoint{4.805128in}{2.540858in}}%
\pgfpathlineto{\pgfqpoint{4.805128in}{2.540858in}}%
\pgfpathlineto{\pgfqpoint{4.805128in}{2.543807in}}%
\pgfpathlineto{\pgfqpoint{4.809669in}{2.543807in}}%
\pgfpathlineto{\pgfqpoint{4.809669in}{2.540858in}}%
\pgfpathmoveto{\pgfqpoint{4.859620in}{2.526113in}}%
\pgfpathlineto{\pgfqpoint{4.859620in}{2.526113in}}%
\pgfpathlineto{\pgfqpoint{4.859620in}{2.529062in}}%
\pgfpathlineto{\pgfqpoint{4.864161in}{2.529062in}}%
\pgfpathlineto{\pgfqpoint{4.864161in}{2.526113in}}%
\pgfpathmoveto{\pgfqpoint{4.864161in}{2.526113in}}%
\pgfpathlineto{\pgfqpoint{4.864161in}{2.526113in}}%
\pgfpathlineto{\pgfqpoint{4.864161in}{2.529062in}}%
\pgfpathlineto{\pgfqpoint{4.868702in}{2.529062in}}%
\pgfpathlineto{\pgfqpoint{4.868702in}{2.526113in}}%
\pgfpathmoveto{\pgfqpoint{4.868702in}{2.526113in}}%
\pgfpathlineto{\pgfqpoint{4.868702in}{2.526113in}}%
\pgfpathlineto{\pgfqpoint{4.868702in}{2.529062in}}%
\pgfpathlineto{\pgfqpoint{4.873243in}{2.529062in}}%
\pgfpathlineto{\pgfqpoint{4.873243in}{2.526113in}}%
\pgfpathmoveto{\pgfqpoint{4.873243in}{2.523164in}}%
\pgfpathlineto{\pgfqpoint{4.873243in}{2.523164in}}%
\pgfpathlineto{\pgfqpoint{4.873243in}{2.526113in}}%
\pgfpathlineto{\pgfqpoint{4.877785in}{2.526113in}}%
\pgfpathlineto{\pgfqpoint{4.877785in}{2.523164in}}%
\pgfpathmoveto{\pgfqpoint{4.873243in}{2.526113in}}%
\pgfpathlineto{\pgfqpoint{4.873243in}{2.526113in}}%
\pgfpathlineto{\pgfqpoint{4.873243in}{2.529062in}}%
\pgfpathlineto{\pgfqpoint{4.877785in}{2.529062in}}%
\pgfpathlineto{\pgfqpoint{4.877785in}{2.526113in}}%
\pgfpathmoveto{\pgfqpoint{4.877785in}{2.523164in}}%
\pgfpathlineto{\pgfqpoint{4.877785in}{2.523164in}}%
\pgfpathlineto{\pgfqpoint{4.877785in}{2.526113in}}%
\pgfpathlineto{\pgfqpoint{4.882326in}{2.526113in}}%
\pgfpathlineto{\pgfqpoint{4.882326in}{2.523164in}}%
\pgfpathmoveto{\pgfqpoint{4.886867in}{2.520215in}}%
\pgfpathlineto{\pgfqpoint{4.886867in}{2.520215in}}%
\pgfpathlineto{\pgfqpoint{4.886867in}{2.523164in}}%
\pgfpathlineto{\pgfqpoint{4.891408in}{2.523164in}}%
\pgfpathlineto{\pgfqpoint{4.891408in}{2.520215in}}%
\pgfpathmoveto{\pgfqpoint{4.882326in}{2.523164in}}%
\pgfpathlineto{\pgfqpoint{4.882326in}{2.523164in}}%
\pgfpathlineto{\pgfqpoint{4.882326in}{2.526113in}}%
\pgfpathlineto{\pgfqpoint{4.886867in}{2.526113in}}%
\pgfpathlineto{\pgfqpoint{4.886867in}{2.523164in}}%
\pgfpathmoveto{\pgfqpoint{4.886867in}{2.523164in}}%
\pgfpathlineto{\pgfqpoint{4.886867in}{2.523164in}}%
\pgfpathlineto{\pgfqpoint{4.886867in}{2.526113in}}%
\pgfpathlineto{\pgfqpoint{4.891408in}{2.526113in}}%
\pgfpathlineto{\pgfqpoint{4.891408in}{2.523164in}}%
\pgfpathmoveto{\pgfqpoint{4.818750in}{2.534960in}}%
\pgfpathlineto{\pgfqpoint{4.818750in}{2.534960in}}%
\pgfpathlineto{\pgfqpoint{4.818750in}{2.537909in}}%
\pgfpathlineto{\pgfqpoint{4.823291in}{2.537909in}}%
\pgfpathlineto{\pgfqpoint{4.823291in}{2.534960in}}%
\pgfpathmoveto{\pgfqpoint{4.818750in}{2.537909in}}%
\pgfpathlineto{\pgfqpoint{4.818750in}{2.537909in}}%
\pgfpathlineto{\pgfqpoint{4.818750in}{2.540858in}}%
\pgfpathlineto{\pgfqpoint{4.823291in}{2.540858in}}%
\pgfpathlineto{\pgfqpoint{4.823291in}{2.537909in}}%
\pgfpathmoveto{\pgfqpoint{4.823291in}{2.534960in}}%
\pgfpathlineto{\pgfqpoint{4.823291in}{2.534960in}}%
\pgfpathlineto{\pgfqpoint{4.823291in}{2.537909in}}%
\pgfpathlineto{\pgfqpoint{4.827833in}{2.537909in}}%
\pgfpathlineto{\pgfqpoint{4.827833in}{2.534960in}}%
\pgfpathmoveto{\pgfqpoint{4.832374in}{2.532011in}}%
\pgfpathlineto{\pgfqpoint{4.832374in}{2.532011in}}%
\pgfpathlineto{\pgfqpoint{4.832374in}{2.534960in}}%
\pgfpathlineto{\pgfqpoint{4.836915in}{2.534960in}}%
\pgfpathlineto{\pgfqpoint{4.836915in}{2.532011in}}%
\pgfpathmoveto{\pgfqpoint{4.827833in}{2.534960in}}%
\pgfpathlineto{\pgfqpoint{4.827833in}{2.534960in}}%
\pgfpathlineto{\pgfqpoint{4.827833in}{2.537909in}}%
\pgfpathlineto{\pgfqpoint{4.832374in}{2.537909in}}%
\pgfpathlineto{\pgfqpoint{4.832374in}{2.534960in}}%
\pgfpathmoveto{\pgfqpoint{4.832374in}{2.534960in}}%
\pgfpathlineto{\pgfqpoint{4.832374in}{2.534960in}}%
\pgfpathlineto{\pgfqpoint{4.832374in}{2.537909in}}%
\pgfpathlineto{\pgfqpoint{4.836915in}{2.537909in}}%
\pgfpathlineto{\pgfqpoint{4.836915in}{2.534960in}}%
\pgfpathmoveto{\pgfqpoint{4.836915in}{2.532011in}}%
\pgfpathlineto{\pgfqpoint{4.836915in}{2.532011in}}%
\pgfpathlineto{\pgfqpoint{4.836915in}{2.534960in}}%
\pgfpathlineto{\pgfqpoint{4.841456in}{2.534960in}}%
\pgfpathlineto{\pgfqpoint{4.841456in}{2.532011in}}%
\pgfpathmoveto{\pgfqpoint{4.841456in}{2.532011in}}%
\pgfpathlineto{\pgfqpoint{4.841456in}{2.532011in}}%
\pgfpathlineto{\pgfqpoint{4.841456in}{2.534960in}}%
\pgfpathlineto{\pgfqpoint{4.845997in}{2.534960in}}%
\pgfpathlineto{\pgfqpoint{4.845997in}{2.532011in}}%
\pgfpathmoveto{\pgfqpoint{4.845997in}{2.529062in}}%
\pgfpathlineto{\pgfqpoint{4.845997in}{2.529062in}}%
\pgfpathlineto{\pgfqpoint{4.845997in}{2.532011in}}%
\pgfpathlineto{\pgfqpoint{4.850538in}{2.532011in}}%
\pgfpathlineto{\pgfqpoint{4.850538in}{2.529062in}}%
\pgfpathmoveto{\pgfqpoint{4.845997in}{2.532011in}}%
\pgfpathlineto{\pgfqpoint{4.845997in}{2.532011in}}%
\pgfpathlineto{\pgfqpoint{4.845997in}{2.534960in}}%
\pgfpathlineto{\pgfqpoint{4.850538in}{2.534960in}}%
\pgfpathlineto{\pgfqpoint{4.850538in}{2.532011in}}%
\pgfpathmoveto{\pgfqpoint{4.850538in}{2.529062in}}%
\pgfpathlineto{\pgfqpoint{4.850538in}{2.529062in}}%
\pgfpathlineto{\pgfqpoint{4.850538in}{2.532011in}}%
\pgfpathlineto{\pgfqpoint{4.855079in}{2.532011in}}%
\pgfpathlineto{\pgfqpoint{4.855079in}{2.529062in}}%
\pgfpathmoveto{\pgfqpoint{4.855079in}{2.529062in}}%
\pgfpathlineto{\pgfqpoint{4.855079in}{2.529062in}}%
\pgfpathlineto{\pgfqpoint{4.855079in}{2.532011in}}%
\pgfpathlineto{\pgfqpoint{4.859620in}{2.532011in}}%
\pgfpathlineto{\pgfqpoint{4.859620in}{2.529062in}}%
\pgfpathmoveto{\pgfqpoint{4.859620in}{2.529062in}}%
\pgfpathlineto{\pgfqpoint{4.859620in}{2.529062in}}%
\pgfpathlineto{\pgfqpoint{4.859620in}{2.532011in}}%
\pgfpathlineto{\pgfqpoint{4.864161in}{2.532011in}}%
\pgfpathlineto{\pgfqpoint{4.864161in}{2.529062in}}%
\pgfpathmoveto{\pgfqpoint{4.891408in}{2.520215in}}%
\pgfpathlineto{\pgfqpoint{4.891408in}{2.520215in}}%
\pgfpathlineto{\pgfqpoint{4.891408in}{2.523164in}}%
\pgfpathlineto{\pgfqpoint{4.895949in}{2.523164in}}%
\pgfpathlineto{\pgfqpoint{4.895949in}{2.520215in}}%
\pgfpathmoveto{\pgfqpoint{4.895949in}{2.520215in}}%
\pgfpathlineto{\pgfqpoint{4.895949in}{2.520215in}}%
\pgfpathlineto{\pgfqpoint{4.895949in}{2.523164in}}%
\pgfpathlineto{\pgfqpoint{4.900490in}{2.523164in}}%
\pgfpathlineto{\pgfqpoint{4.900490in}{2.520215in}}%
\pgfpathmoveto{\pgfqpoint{4.900490in}{2.517265in}}%
\pgfpathlineto{\pgfqpoint{4.900490in}{2.517265in}}%
\pgfpathlineto{\pgfqpoint{4.900490in}{2.520215in}}%
\pgfpathlineto{\pgfqpoint{4.905031in}{2.520215in}}%
\pgfpathlineto{\pgfqpoint{4.905031in}{2.517265in}}%
\pgfpathmoveto{\pgfqpoint{4.900490in}{2.520215in}}%
\pgfpathlineto{\pgfqpoint{4.900490in}{2.520215in}}%
\pgfpathlineto{\pgfqpoint{4.900490in}{2.523164in}}%
\pgfpathlineto{\pgfqpoint{4.905031in}{2.523164in}}%
\pgfpathlineto{\pgfqpoint{4.905031in}{2.520215in}}%
\pgfpathmoveto{\pgfqpoint{4.905031in}{2.517265in}}%
\pgfpathlineto{\pgfqpoint{4.905031in}{2.517265in}}%
\pgfpathlineto{\pgfqpoint{4.905031in}{2.520215in}}%
\pgfpathlineto{\pgfqpoint{4.909572in}{2.520215in}}%
\pgfpathlineto{\pgfqpoint{4.909572in}{2.517265in}}%
\pgfpathmoveto{\pgfqpoint{4.914113in}{2.514316in}}%
\pgfpathlineto{\pgfqpoint{4.914113in}{2.514316in}}%
\pgfpathlineto{\pgfqpoint{4.914113in}{2.517265in}}%
\pgfpathlineto{\pgfqpoint{4.918654in}{2.517265in}}%
\pgfpathlineto{\pgfqpoint{4.918654in}{2.514316in}}%
\pgfpathmoveto{\pgfqpoint{4.918654in}{2.514316in}}%
\pgfpathlineto{\pgfqpoint{4.918654in}{2.514316in}}%
\pgfpathlineto{\pgfqpoint{4.918654in}{2.517265in}}%
\pgfpathlineto{\pgfqpoint{4.923195in}{2.517265in}}%
\pgfpathlineto{\pgfqpoint{4.923195in}{2.514316in}}%
\pgfpathmoveto{\pgfqpoint{4.923195in}{2.514316in}}%
\pgfpathlineto{\pgfqpoint{4.923195in}{2.514316in}}%
\pgfpathlineto{\pgfqpoint{4.923195in}{2.517265in}}%
\pgfpathlineto{\pgfqpoint{4.927737in}{2.517265in}}%
\pgfpathlineto{\pgfqpoint{4.927737in}{2.514316in}}%
\pgfpathmoveto{\pgfqpoint{4.909572in}{2.517265in}}%
\pgfpathlineto{\pgfqpoint{4.909572in}{2.517265in}}%
\pgfpathlineto{\pgfqpoint{4.909572in}{2.520215in}}%
\pgfpathlineto{\pgfqpoint{4.914113in}{2.520215in}}%
\pgfpathlineto{\pgfqpoint{4.914113in}{2.517265in}}%
\pgfpathmoveto{\pgfqpoint{4.914113in}{2.517265in}}%
\pgfpathlineto{\pgfqpoint{4.914113in}{2.517265in}}%
\pgfpathlineto{\pgfqpoint{4.914113in}{2.520215in}}%
\pgfpathlineto{\pgfqpoint{4.918654in}{2.520215in}}%
\pgfpathlineto{\pgfqpoint{4.918654in}{2.517265in}}%
\pgfpathmoveto{\pgfqpoint{4.927737in}{2.511367in}}%
\pgfpathlineto{\pgfqpoint{4.927737in}{2.511367in}}%
\pgfpathlineto{\pgfqpoint{4.927737in}{2.514316in}}%
\pgfpathlineto{\pgfqpoint{4.932278in}{2.514316in}}%
\pgfpathlineto{\pgfqpoint{4.932278in}{2.511367in}}%
\pgfpathmoveto{\pgfqpoint{4.927737in}{2.514316in}}%
\pgfpathlineto{\pgfqpoint{4.927737in}{2.514316in}}%
\pgfpathlineto{\pgfqpoint{4.927737in}{2.517265in}}%
\pgfpathlineto{\pgfqpoint{4.932278in}{2.517265in}}%
\pgfpathlineto{\pgfqpoint{4.932278in}{2.514316in}}%
\pgfpathmoveto{\pgfqpoint{4.932278in}{2.511367in}}%
\pgfpathlineto{\pgfqpoint{4.932278in}{2.511367in}}%
\pgfpathlineto{\pgfqpoint{4.932278in}{2.514316in}}%
\pgfpathlineto{\pgfqpoint{4.936819in}{2.514316in}}%
\pgfpathlineto{\pgfqpoint{4.936819in}{2.511367in}}%
\pgfpathmoveto{\pgfqpoint{4.941360in}{2.508418in}}%
\pgfpathlineto{\pgfqpoint{4.941360in}{2.508418in}}%
\pgfpathlineto{\pgfqpoint{4.941360in}{2.511367in}}%
\pgfpathlineto{\pgfqpoint{4.945901in}{2.511367in}}%
\pgfpathlineto{\pgfqpoint{4.945901in}{2.508418in}}%
\pgfpathmoveto{\pgfqpoint{4.936819in}{2.511367in}}%
\pgfpathlineto{\pgfqpoint{4.936819in}{2.511367in}}%
\pgfpathlineto{\pgfqpoint{4.936819in}{2.514316in}}%
\pgfpathlineto{\pgfqpoint{4.941360in}{2.514316in}}%
\pgfpathlineto{\pgfqpoint{4.941360in}{2.511367in}}%
\pgfpathmoveto{\pgfqpoint{4.941360in}{2.511367in}}%
\pgfpathlineto{\pgfqpoint{4.941360in}{2.511367in}}%
\pgfpathlineto{\pgfqpoint{4.941360in}{2.514316in}}%
\pgfpathlineto{\pgfqpoint{4.945901in}{2.514316in}}%
\pgfpathlineto{\pgfqpoint{4.945901in}{2.511367in}}%
\pgfpathmoveto{\pgfqpoint{4.945901in}{2.508418in}}%
\pgfpathlineto{\pgfqpoint{4.945901in}{2.508418in}}%
\pgfpathlineto{\pgfqpoint{4.945901in}{2.511367in}}%
\pgfpathlineto{\pgfqpoint{4.950442in}{2.511367in}}%
\pgfpathlineto{\pgfqpoint{4.950442in}{2.508418in}}%
\pgfpathmoveto{\pgfqpoint{4.950442in}{2.508418in}}%
\pgfpathlineto{\pgfqpoint{4.950442in}{2.508418in}}%
\pgfpathlineto{\pgfqpoint{4.950442in}{2.511367in}}%
\pgfpathlineto{\pgfqpoint{4.954983in}{2.511367in}}%
\pgfpathlineto{\pgfqpoint{4.954983in}{2.508418in}}%
\pgfpathmoveto{\pgfqpoint{4.954983in}{2.505469in}}%
\pgfpathlineto{\pgfqpoint{4.954983in}{2.505469in}}%
\pgfpathlineto{\pgfqpoint{4.954983in}{2.508418in}}%
\pgfpathlineto{\pgfqpoint{4.959524in}{2.508418in}}%
\pgfpathlineto{\pgfqpoint{4.959524in}{2.505469in}}%
\pgfpathmoveto{\pgfqpoint{4.954983in}{2.508418in}}%
\pgfpathlineto{\pgfqpoint{4.954983in}{2.508418in}}%
\pgfpathlineto{\pgfqpoint{4.954983in}{2.511367in}}%
\pgfpathlineto{\pgfqpoint{4.959524in}{2.511367in}}%
\pgfpathlineto{\pgfqpoint{4.959524in}{2.508418in}}%
\pgfpathmoveto{\pgfqpoint{4.959524in}{2.505469in}}%
\pgfpathlineto{\pgfqpoint{4.959524in}{2.505469in}}%
\pgfpathlineto{\pgfqpoint{4.959524in}{2.508418in}}%
\pgfpathlineto{\pgfqpoint{4.964065in}{2.508418in}}%
\pgfpathlineto{\pgfqpoint{4.964065in}{2.505469in}}%
\pgfpathmoveto{\pgfqpoint{5.077586in}{2.478927in}}%
\pgfpathlineto{\pgfqpoint{5.077586in}{2.478927in}}%
\pgfpathlineto{\pgfqpoint{5.077586in}{2.481876in}}%
\pgfpathlineto{\pgfqpoint{5.082126in}{2.481876in}}%
\pgfpathlineto{\pgfqpoint{5.082126in}{2.478927in}}%
\pgfpathmoveto{\pgfqpoint{5.082126in}{2.478927in}}%
\pgfpathlineto{\pgfqpoint{5.082126in}{2.478927in}}%
\pgfpathlineto{\pgfqpoint{5.082126in}{2.481876in}}%
\pgfpathlineto{\pgfqpoint{5.086667in}{2.481876in}}%
\pgfpathlineto{\pgfqpoint{5.086667in}{2.478927in}}%
\pgfpathmoveto{\pgfqpoint{5.086667in}{2.478927in}}%
\pgfpathlineto{\pgfqpoint{5.086667in}{2.478927in}}%
\pgfpathlineto{\pgfqpoint{5.086667in}{2.481876in}}%
\pgfpathlineto{\pgfqpoint{5.091208in}{2.481876in}}%
\pgfpathlineto{\pgfqpoint{5.091208in}{2.478927in}}%
\pgfpathmoveto{\pgfqpoint{5.091208in}{2.475978in}}%
\pgfpathlineto{\pgfqpoint{5.091208in}{2.475978in}}%
\pgfpathlineto{\pgfqpoint{5.091208in}{2.478927in}}%
\pgfpathlineto{\pgfqpoint{5.095749in}{2.478927in}}%
\pgfpathlineto{\pgfqpoint{5.095749in}{2.475978in}}%
\pgfpathmoveto{\pgfqpoint{5.091208in}{2.478927in}}%
\pgfpathlineto{\pgfqpoint{5.091208in}{2.478927in}}%
\pgfpathlineto{\pgfqpoint{5.091208in}{2.481876in}}%
\pgfpathlineto{\pgfqpoint{5.095749in}{2.481876in}}%
\pgfpathlineto{\pgfqpoint{5.095749in}{2.478927in}}%
\pgfpathmoveto{\pgfqpoint{5.095749in}{2.475978in}}%
\pgfpathlineto{\pgfqpoint{5.095749in}{2.475978in}}%
\pgfpathlineto{\pgfqpoint{5.095749in}{2.478927in}}%
\pgfpathlineto{\pgfqpoint{5.100290in}{2.478927in}}%
\pgfpathlineto{\pgfqpoint{5.100290in}{2.475978in}}%
\pgfpathmoveto{\pgfqpoint{5.104830in}{2.473028in}}%
\pgfpathlineto{\pgfqpoint{5.104830in}{2.473028in}}%
\pgfpathlineto{\pgfqpoint{5.104830in}{2.475978in}}%
\pgfpathlineto{\pgfqpoint{5.109371in}{2.475978in}}%
\pgfpathlineto{\pgfqpoint{5.109371in}{2.473028in}}%
\pgfpathmoveto{\pgfqpoint{5.100290in}{2.475978in}}%
\pgfpathlineto{\pgfqpoint{5.100290in}{2.475978in}}%
\pgfpathlineto{\pgfqpoint{5.100290in}{2.478927in}}%
\pgfpathlineto{\pgfqpoint{5.104830in}{2.478927in}}%
\pgfpathlineto{\pgfqpoint{5.104830in}{2.475978in}}%
\pgfpathmoveto{\pgfqpoint{5.104830in}{2.475978in}}%
\pgfpathlineto{\pgfqpoint{5.104830in}{2.475978in}}%
\pgfpathlineto{\pgfqpoint{5.104830in}{2.478927in}}%
\pgfpathlineto{\pgfqpoint{5.109371in}{2.478927in}}%
\pgfpathlineto{\pgfqpoint{5.109371in}{2.475978in}}%
\pgfpathmoveto{\pgfqpoint{4.968606in}{2.502520in}}%
\pgfpathlineto{\pgfqpoint{4.968606in}{2.502520in}}%
\pgfpathlineto{\pgfqpoint{4.968606in}{2.505469in}}%
\pgfpathlineto{\pgfqpoint{4.973147in}{2.505469in}}%
\pgfpathlineto{\pgfqpoint{4.973147in}{2.502520in}}%
\pgfpathmoveto{\pgfqpoint{4.973147in}{2.502520in}}%
\pgfpathlineto{\pgfqpoint{4.973147in}{2.502520in}}%
\pgfpathlineto{\pgfqpoint{4.973147in}{2.505469in}}%
\pgfpathlineto{\pgfqpoint{4.977688in}{2.505469in}}%
\pgfpathlineto{\pgfqpoint{4.977688in}{2.502520in}}%
\pgfpathmoveto{\pgfqpoint{4.977688in}{2.502520in}}%
\pgfpathlineto{\pgfqpoint{4.977688in}{2.502520in}}%
\pgfpathlineto{\pgfqpoint{4.977688in}{2.505469in}}%
\pgfpathlineto{\pgfqpoint{4.982228in}{2.505469in}}%
\pgfpathlineto{\pgfqpoint{4.982228in}{2.502520in}}%
\pgfpathmoveto{\pgfqpoint{4.982228in}{2.499571in}}%
\pgfpathlineto{\pgfqpoint{4.982228in}{2.499571in}}%
\pgfpathlineto{\pgfqpoint{4.982228in}{2.502520in}}%
\pgfpathlineto{\pgfqpoint{4.986769in}{2.502520in}}%
\pgfpathlineto{\pgfqpoint{4.986769in}{2.499571in}}%
\pgfpathmoveto{\pgfqpoint{4.982228in}{2.502520in}}%
\pgfpathlineto{\pgfqpoint{4.982228in}{2.502520in}}%
\pgfpathlineto{\pgfqpoint{4.982228in}{2.505469in}}%
\pgfpathlineto{\pgfqpoint{4.986769in}{2.505469in}}%
\pgfpathlineto{\pgfqpoint{4.986769in}{2.502520in}}%
\pgfpathmoveto{\pgfqpoint{4.986769in}{2.499571in}}%
\pgfpathlineto{\pgfqpoint{4.986769in}{2.499571in}}%
\pgfpathlineto{\pgfqpoint{4.986769in}{2.502520in}}%
\pgfpathlineto{\pgfqpoint{4.991310in}{2.502520in}}%
\pgfpathlineto{\pgfqpoint{4.991310in}{2.499571in}}%
\pgfpathmoveto{\pgfqpoint{4.995851in}{2.496622in}}%
\pgfpathlineto{\pgfqpoint{4.995851in}{2.496622in}}%
\pgfpathlineto{\pgfqpoint{4.995851in}{2.499571in}}%
\pgfpathlineto{\pgfqpoint{5.000392in}{2.499571in}}%
\pgfpathlineto{\pgfqpoint{5.000392in}{2.496622in}}%
\pgfpathmoveto{\pgfqpoint{4.991310in}{2.499571in}}%
\pgfpathlineto{\pgfqpoint{4.991310in}{2.499571in}}%
\pgfpathlineto{\pgfqpoint{4.991310in}{2.502520in}}%
\pgfpathlineto{\pgfqpoint{4.995851in}{2.502520in}}%
\pgfpathlineto{\pgfqpoint{4.995851in}{2.499571in}}%
\pgfpathmoveto{\pgfqpoint{4.995851in}{2.499571in}}%
\pgfpathlineto{\pgfqpoint{4.995851in}{2.499571in}}%
\pgfpathlineto{\pgfqpoint{4.995851in}{2.502520in}}%
\pgfpathlineto{\pgfqpoint{5.000392in}{2.502520in}}%
\pgfpathlineto{\pgfqpoint{5.000392in}{2.499571in}}%
\pgfpathmoveto{\pgfqpoint{4.964065in}{2.505469in}}%
\pgfpathlineto{\pgfqpoint{4.964065in}{2.505469in}}%
\pgfpathlineto{\pgfqpoint{4.964065in}{2.508418in}}%
\pgfpathlineto{\pgfqpoint{4.968606in}{2.508418in}}%
\pgfpathlineto{\pgfqpoint{4.968606in}{2.505469in}}%
\pgfpathmoveto{\pgfqpoint{4.968606in}{2.505469in}}%
\pgfpathlineto{\pgfqpoint{4.968606in}{2.505469in}}%
\pgfpathlineto{\pgfqpoint{4.968606in}{2.508418in}}%
\pgfpathlineto{\pgfqpoint{4.973147in}{2.508418in}}%
\pgfpathlineto{\pgfqpoint{4.973147in}{2.505469in}}%
\pgfpathmoveto{\pgfqpoint{5.000392in}{2.496622in}}%
\pgfpathlineto{\pgfqpoint{5.000392in}{2.496622in}}%
\pgfpathlineto{\pgfqpoint{5.000392in}{2.499571in}}%
\pgfpathlineto{\pgfqpoint{5.004933in}{2.499571in}}%
\pgfpathlineto{\pgfqpoint{5.004933in}{2.496622in}}%
\pgfpathmoveto{\pgfqpoint{5.004933in}{2.496622in}}%
\pgfpathlineto{\pgfqpoint{5.004933in}{2.496622in}}%
\pgfpathlineto{\pgfqpoint{5.004933in}{2.499571in}}%
\pgfpathlineto{\pgfqpoint{5.009473in}{2.499571in}}%
\pgfpathlineto{\pgfqpoint{5.009473in}{2.496622in}}%
\pgfpathmoveto{\pgfqpoint{5.009473in}{2.493673in}}%
\pgfpathlineto{\pgfqpoint{5.009473in}{2.493673in}}%
\pgfpathlineto{\pgfqpoint{5.009473in}{2.496622in}}%
\pgfpathlineto{\pgfqpoint{5.014014in}{2.496622in}}%
\pgfpathlineto{\pgfqpoint{5.014014in}{2.493673in}}%
\pgfpathmoveto{\pgfqpoint{5.009473in}{2.496622in}}%
\pgfpathlineto{\pgfqpoint{5.009473in}{2.496622in}}%
\pgfpathlineto{\pgfqpoint{5.009473in}{2.499571in}}%
\pgfpathlineto{\pgfqpoint{5.014014in}{2.499571in}}%
\pgfpathlineto{\pgfqpoint{5.014014in}{2.496622in}}%
\pgfpathmoveto{\pgfqpoint{5.014014in}{2.493673in}}%
\pgfpathlineto{\pgfqpoint{5.014014in}{2.493673in}}%
\pgfpathlineto{\pgfqpoint{5.014014in}{2.496622in}}%
\pgfpathlineto{\pgfqpoint{5.018555in}{2.496622in}}%
\pgfpathlineto{\pgfqpoint{5.018555in}{2.493673in}}%
\pgfpathmoveto{\pgfqpoint{5.023096in}{2.490723in}}%
\pgfpathlineto{\pgfqpoint{5.023096in}{2.490723in}}%
\pgfpathlineto{\pgfqpoint{5.023096in}{2.493673in}}%
\pgfpathlineto{\pgfqpoint{5.027637in}{2.493673in}}%
\pgfpathlineto{\pgfqpoint{5.027637in}{2.490723in}}%
\pgfpathmoveto{\pgfqpoint{5.027637in}{2.490723in}}%
\pgfpathlineto{\pgfqpoint{5.027637in}{2.490723in}}%
\pgfpathlineto{\pgfqpoint{5.027637in}{2.493673in}}%
\pgfpathlineto{\pgfqpoint{5.032177in}{2.493673in}}%
\pgfpathlineto{\pgfqpoint{5.032177in}{2.490723in}}%
\pgfpathmoveto{\pgfqpoint{5.032177in}{2.490723in}}%
\pgfpathlineto{\pgfqpoint{5.032177in}{2.490723in}}%
\pgfpathlineto{\pgfqpoint{5.032177in}{2.493673in}}%
\pgfpathlineto{\pgfqpoint{5.036718in}{2.493673in}}%
\pgfpathlineto{\pgfqpoint{5.036718in}{2.490723in}}%
\pgfpathmoveto{\pgfqpoint{5.018555in}{2.493673in}}%
\pgfpathlineto{\pgfqpoint{5.018555in}{2.493673in}}%
\pgfpathlineto{\pgfqpoint{5.018555in}{2.496622in}}%
\pgfpathlineto{\pgfqpoint{5.023096in}{2.496622in}}%
\pgfpathlineto{\pgfqpoint{5.023096in}{2.493673in}}%
\pgfpathmoveto{\pgfqpoint{5.023096in}{2.493673in}}%
\pgfpathlineto{\pgfqpoint{5.023096in}{2.493673in}}%
\pgfpathlineto{\pgfqpoint{5.023096in}{2.496622in}}%
\pgfpathlineto{\pgfqpoint{5.027637in}{2.496622in}}%
\pgfpathlineto{\pgfqpoint{5.027637in}{2.493673in}}%
\pgfpathmoveto{\pgfqpoint{5.036718in}{2.487774in}}%
\pgfpathlineto{\pgfqpoint{5.036718in}{2.487774in}}%
\pgfpathlineto{\pgfqpoint{5.036718in}{2.490723in}}%
\pgfpathlineto{\pgfqpoint{5.041259in}{2.490723in}}%
\pgfpathlineto{\pgfqpoint{5.041259in}{2.487774in}}%
\pgfpathmoveto{\pgfqpoint{5.036718in}{2.490723in}}%
\pgfpathlineto{\pgfqpoint{5.036718in}{2.490723in}}%
\pgfpathlineto{\pgfqpoint{5.036718in}{2.493673in}}%
\pgfpathlineto{\pgfqpoint{5.041259in}{2.493673in}}%
\pgfpathlineto{\pgfqpoint{5.041259in}{2.490723in}}%
\pgfpathmoveto{\pgfqpoint{5.041259in}{2.487774in}}%
\pgfpathlineto{\pgfqpoint{5.041259in}{2.487774in}}%
\pgfpathlineto{\pgfqpoint{5.041259in}{2.490723in}}%
\pgfpathlineto{\pgfqpoint{5.045800in}{2.490723in}}%
\pgfpathlineto{\pgfqpoint{5.045800in}{2.487774in}}%
\pgfpathmoveto{\pgfqpoint{5.050341in}{2.484825in}}%
\pgfpathlineto{\pgfqpoint{5.050341in}{2.484825in}}%
\pgfpathlineto{\pgfqpoint{5.050341in}{2.487774in}}%
\pgfpathlineto{\pgfqpoint{5.054882in}{2.487774in}}%
\pgfpathlineto{\pgfqpoint{5.054882in}{2.484825in}}%
\pgfpathmoveto{\pgfqpoint{5.045800in}{2.487774in}}%
\pgfpathlineto{\pgfqpoint{5.045800in}{2.487774in}}%
\pgfpathlineto{\pgfqpoint{5.045800in}{2.490723in}}%
\pgfpathlineto{\pgfqpoint{5.050341in}{2.490723in}}%
\pgfpathlineto{\pgfqpoint{5.050341in}{2.487774in}}%
\pgfpathmoveto{\pgfqpoint{5.050341in}{2.487774in}}%
\pgfpathlineto{\pgfqpoint{5.050341in}{2.487774in}}%
\pgfpathlineto{\pgfqpoint{5.050341in}{2.490723in}}%
\pgfpathlineto{\pgfqpoint{5.054882in}{2.490723in}}%
\pgfpathlineto{\pgfqpoint{5.054882in}{2.487774in}}%
\pgfpathmoveto{\pgfqpoint{5.054882in}{2.484825in}}%
\pgfpathlineto{\pgfqpoint{5.054882in}{2.484825in}}%
\pgfpathlineto{\pgfqpoint{5.054882in}{2.487774in}}%
\pgfpathlineto{\pgfqpoint{5.059422in}{2.487774in}}%
\pgfpathlineto{\pgfqpoint{5.059422in}{2.484825in}}%
\pgfpathmoveto{\pgfqpoint{5.059422in}{2.484825in}}%
\pgfpathlineto{\pgfqpoint{5.059422in}{2.484825in}}%
\pgfpathlineto{\pgfqpoint{5.059422in}{2.487774in}}%
\pgfpathlineto{\pgfqpoint{5.063963in}{2.487774in}}%
\pgfpathlineto{\pgfqpoint{5.063963in}{2.484825in}}%
\pgfpathmoveto{\pgfqpoint{5.063963in}{2.481876in}}%
\pgfpathlineto{\pgfqpoint{5.063963in}{2.481876in}}%
\pgfpathlineto{\pgfqpoint{5.063963in}{2.484825in}}%
\pgfpathlineto{\pgfqpoint{5.068504in}{2.484825in}}%
\pgfpathlineto{\pgfqpoint{5.068504in}{2.481876in}}%
\pgfpathmoveto{\pgfqpoint{5.063963in}{2.484825in}}%
\pgfpathlineto{\pgfqpoint{5.063963in}{2.484825in}}%
\pgfpathlineto{\pgfqpoint{5.063963in}{2.487774in}}%
\pgfpathlineto{\pgfqpoint{5.068504in}{2.487774in}}%
\pgfpathlineto{\pgfqpoint{5.068504in}{2.484825in}}%
\pgfpathmoveto{\pgfqpoint{5.068504in}{2.481876in}}%
\pgfpathlineto{\pgfqpoint{5.068504in}{2.481876in}}%
\pgfpathlineto{\pgfqpoint{5.068504in}{2.484825in}}%
\pgfpathlineto{\pgfqpoint{5.073045in}{2.484825in}}%
\pgfpathlineto{\pgfqpoint{5.073045in}{2.481876in}}%
\pgfpathmoveto{\pgfqpoint{5.073045in}{2.481876in}}%
\pgfpathlineto{\pgfqpoint{5.073045in}{2.481876in}}%
\pgfpathlineto{\pgfqpoint{5.073045in}{2.484825in}}%
\pgfpathlineto{\pgfqpoint{5.077586in}{2.484825in}}%
\pgfpathlineto{\pgfqpoint{5.077586in}{2.481876in}}%
\pgfpathmoveto{\pgfqpoint{5.077586in}{2.481876in}}%
\pgfpathlineto{\pgfqpoint{5.077586in}{2.481876in}}%
\pgfpathlineto{\pgfqpoint{5.077586in}{2.484825in}}%
\pgfpathlineto{\pgfqpoint{5.082126in}{2.484825in}}%
\pgfpathlineto{\pgfqpoint{5.082126in}{2.481876in}}%
\pgfpathmoveto{\pgfqpoint{5.109371in}{2.473028in}}%
\pgfpathlineto{\pgfqpoint{5.109371in}{2.473028in}}%
\pgfpathlineto{\pgfqpoint{5.109371in}{2.475978in}}%
\pgfpathlineto{\pgfqpoint{5.113912in}{2.475978in}}%
\pgfpathlineto{\pgfqpoint{5.113912in}{2.473028in}}%
\pgfpathmoveto{\pgfqpoint{5.113912in}{2.473028in}}%
\pgfpathlineto{\pgfqpoint{5.113912in}{2.473028in}}%
\pgfpathlineto{\pgfqpoint{5.113912in}{2.475978in}}%
\pgfpathlineto{\pgfqpoint{5.118453in}{2.475978in}}%
\pgfpathlineto{\pgfqpoint{5.118453in}{2.473028in}}%
\pgfpathmoveto{\pgfqpoint{5.118453in}{2.470079in}}%
\pgfpathlineto{\pgfqpoint{5.118453in}{2.470079in}}%
\pgfpathlineto{\pgfqpoint{5.118453in}{2.473028in}}%
\pgfpathlineto{\pgfqpoint{5.122994in}{2.473028in}}%
\pgfpathlineto{\pgfqpoint{5.122994in}{2.470079in}}%
\pgfpathmoveto{\pgfqpoint{5.118453in}{2.473028in}}%
\pgfpathlineto{\pgfqpoint{5.118453in}{2.473028in}}%
\pgfpathlineto{\pgfqpoint{5.118453in}{2.475978in}}%
\pgfpathlineto{\pgfqpoint{5.122994in}{2.475978in}}%
\pgfpathlineto{\pgfqpoint{5.122994in}{2.473028in}}%
\pgfpathmoveto{\pgfqpoint{5.122994in}{2.470079in}}%
\pgfpathlineto{\pgfqpoint{5.122994in}{2.470079in}}%
\pgfpathlineto{\pgfqpoint{5.122994in}{2.473028in}}%
\pgfpathlineto{\pgfqpoint{5.127535in}{2.473028in}}%
\pgfpathlineto{\pgfqpoint{5.127535in}{2.470079in}}%
\pgfpathmoveto{\pgfqpoint{5.132076in}{2.467130in}}%
\pgfpathlineto{\pgfqpoint{5.132076in}{2.467130in}}%
\pgfpathlineto{\pgfqpoint{5.132076in}{2.470079in}}%
\pgfpathlineto{\pgfqpoint{5.136617in}{2.470079in}}%
\pgfpathlineto{\pgfqpoint{5.136617in}{2.467130in}}%
\pgfpathmoveto{\pgfqpoint{5.136617in}{2.467130in}}%
\pgfpathlineto{\pgfqpoint{5.136617in}{2.467130in}}%
\pgfpathlineto{\pgfqpoint{5.136617in}{2.470079in}}%
\pgfpathlineto{\pgfqpoint{5.141158in}{2.470079in}}%
\pgfpathlineto{\pgfqpoint{5.141158in}{2.467130in}}%
\pgfpathmoveto{\pgfqpoint{5.141158in}{2.467130in}}%
\pgfpathlineto{\pgfqpoint{5.141158in}{2.467130in}}%
\pgfpathlineto{\pgfqpoint{5.141158in}{2.470079in}}%
\pgfpathlineto{\pgfqpoint{5.145699in}{2.470079in}}%
\pgfpathlineto{\pgfqpoint{5.145699in}{2.467130in}}%
\pgfpathmoveto{\pgfqpoint{5.127535in}{2.470079in}}%
\pgfpathlineto{\pgfqpoint{5.127535in}{2.470079in}}%
\pgfpathlineto{\pgfqpoint{5.127535in}{2.473028in}}%
\pgfpathlineto{\pgfqpoint{5.132076in}{2.473028in}}%
\pgfpathlineto{\pgfqpoint{5.132076in}{2.470079in}}%
\pgfpathmoveto{\pgfqpoint{5.132076in}{2.470079in}}%
\pgfpathlineto{\pgfqpoint{5.132076in}{2.470079in}}%
\pgfpathlineto{\pgfqpoint{5.132076in}{2.473028in}}%
\pgfpathlineto{\pgfqpoint{5.136617in}{2.473028in}}%
\pgfpathlineto{\pgfqpoint{5.136617in}{2.470079in}}%
\pgfpathmoveto{\pgfqpoint{5.145699in}{2.464181in}}%
\pgfpathlineto{\pgfqpoint{5.145699in}{2.464181in}}%
\pgfpathlineto{\pgfqpoint{5.145699in}{2.467130in}}%
\pgfpathlineto{\pgfqpoint{5.150240in}{2.467130in}}%
\pgfpathlineto{\pgfqpoint{5.150240in}{2.464181in}}%
\pgfpathmoveto{\pgfqpoint{5.145699in}{2.467130in}}%
\pgfpathlineto{\pgfqpoint{5.145699in}{2.467130in}}%
\pgfpathlineto{\pgfqpoint{5.145699in}{2.470079in}}%
\pgfpathlineto{\pgfqpoint{5.150240in}{2.470079in}}%
\pgfpathlineto{\pgfqpoint{5.150240in}{2.467130in}}%
\pgfpathmoveto{\pgfqpoint{5.150240in}{2.464181in}}%
\pgfpathlineto{\pgfqpoint{5.150240in}{2.464181in}}%
\pgfpathlineto{\pgfqpoint{5.150240in}{2.467130in}}%
\pgfpathlineto{\pgfqpoint{5.154782in}{2.467130in}}%
\pgfpathlineto{\pgfqpoint{5.154782in}{2.464181in}}%
\pgfpathmoveto{\pgfqpoint{5.159323in}{2.461231in}}%
\pgfpathlineto{\pgfqpoint{5.159323in}{2.461231in}}%
\pgfpathlineto{\pgfqpoint{5.159323in}{2.464181in}}%
\pgfpathlineto{\pgfqpoint{5.163864in}{2.464181in}}%
\pgfpathlineto{\pgfqpoint{5.163864in}{2.461231in}}%
\pgfpathmoveto{\pgfqpoint{5.154782in}{2.464181in}}%
\pgfpathlineto{\pgfqpoint{5.154782in}{2.464181in}}%
\pgfpathlineto{\pgfqpoint{5.154782in}{2.467130in}}%
\pgfpathlineto{\pgfqpoint{5.159323in}{2.467130in}}%
\pgfpathlineto{\pgfqpoint{5.159323in}{2.464181in}}%
\pgfpathmoveto{\pgfqpoint{5.159323in}{2.464181in}}%
\pgfpathlineto{\pgfqpoint{5.159323in}{2.464181in}}%
\pgfpathlineto{\pgfqpoint{5.159323in}{2.467130in}}%
\pgfpathlineto{\pgfqpoint{5.163864in}{2.467130in}}%
\pgfpathlineto{\pgfqpoint{5.163864in}{2.464181in}}%
\pgfpathmoveto{\pgfqpoint{5.163864in}{2.461231in}}%
\pgfpathlineto{\pgfqpoint{5.163864in}{2.461231in}}%
\pgfpathlineto{\pgfqpoint{5.163864in}{2.464181in}}%
\pgfpathlineto{\pgfqpoint{5.168405in}{2.464181in}}%
\pgfpathlineto{\pgfqpoint{5.168405in}{2.461231in}}%
\pgfpathmoveto{\pgfqpoint{5.168405in}{2.461231in}}%
\pgfpathlineto{\pgfqpoint{5.168405in}{2.461231in}}%
\pgfpathlineto{\pgfqpoint{5.168405in}{2.464181in}}%
\pgfpathlineto{\pgfqpoint{5.172946in}{2.464181in}}%
\pgfpathlineto{\pgfqpoint{5.172946in}{2.461231in}}%
\pgfpathmoveto{\pgfqpoint{5.172946in}{2.458282in}}%
\pgfpathlineto{\pgfqpoint{5.172946in}{2.458282in}}%
\pgfpathlineto{\pgfqpoint{5.172946in}{2.461231in}}%
\pgfpathlineto{\pgfqpoint{5.177487in}{2.461231in}}%
\pgfpathlineto{\pgfqpoint{5.177487in}{2.458282in}}%
\pgfpathmoveto{\pgfqpoint{5.172946in}{2.461231in}}%
\pgfpathlineto{\pgfqpoint{5.172946in}{2.461231in}}%
\pgfpathlineto{\pgfqpoint{5.172946in}{2.464181in}}%
\pgfpathlineto{\pgfqpoint{5.177487in}{2.464181in}}%
\pgfpathlineto{\pgfqpoint{5.177487in}{2.461231in}}%
\pgfpathmoveto{\pgfqpoint{5.177487in}{2.458282in}}%
\pgfpathlineto{\pgfqpoint{5.177487in}{2.458282in}}%
\pgfpathlineto{\pgfqpoint{5.177487in}{2.461231in}}%
\pgfpathlineto{\pgfqpoint{5.182028in}{2.461231in}}%
\pgfpathlineto{\pgfqpoint{5.182028in}{2.458282in}}%
\pgfpathmoveto{\pgfqpoint{5.186569in}{2.455333in}}%
\pgfpathlineto{\pgfqpoint{5.186569in}{2.455333in}}%
\pgfpathlineto{\pgfqpoint{5.186569in}{2.458282in}}%
\pgfpathlineto{\pgfqpoint{5.191110in}{2.458282in}}%
\pgfpathlineto{\pgfqpoint{5.191110in}{2.455333in}}%
\pgfpathmoveto{\pgfqpoint{5.191110in}{2.455333in}}%
\pgfpathlineto{\pgfqpoint{5.191110in}{2.455333in}}%
\pgfpathlineto{\pgfqpoint{5.191110in}{2.458282in}}%
\pgfpathlineto{\pgfqpoint{5.195651in}{2.458282in}}%
\pgfpathlineto{\pgfqpoint{5.195651in}{2.455333in}}%
\pgfpathmoveto{\pgfqpoint{5.195651in}{2.455333in}}%
\pgfpathlineto{\pgfqpoint{5.195651in}{2.455333in}}%
\pgfpathlineto{\pgfqpoint{5.195651in}{2.458282in}}%
\pgfpathlineto{\pgfqpoint{5.200192in}{2.458282in}}%
\pgfpathlineto{\pgfqpoint{5.200192in}{2.455333in}}%
\pgfpathmoveto{\pgfqpoint{5.200192in}{2.452384in}}%
\pgfpathlineto{\pgfqpoint{5.200192in}{2.452384in}}%
\pgfpathlineto{\pgfqpoint{5.200192in}{2.455333in}}%
\pgfpathlineto{\pgfqpoint{5.204733in}{2.455333in}}%
\pgfpathlineto{\pgfqpoint{5.204733in}{2.452384in}}%
\pgfpathmoveto{\pgfqpoint{5.200192in}{2.455333in}}%
\pgfpathlineto{\pgfqpoint{5.200192in}{2.455333in}}%
\pgfpathlineto{\pgfqpoint{5.200192in}{2.458282in}}%
\pgfpathlineto{\pgfqpoint{5.204733in}{2.458282in}}%
\pgfpathlineto{\pgfqpoint{5.204733in}{2.455333in}}%
\pgfpathmoveto{\pgfqpoint{5.204733in}{2.452384in}}%
\pgfpathlineto{\pgfqpoint{5.204733in}{2.452384in}}%
\pgfpathlineto{\pgfqpoint{5.204733in}{2.455333in}}%
\pgfpathlineto{\pgfqpoint{5.209274in}{2.455333in}}%
\pgfpathlineto{\pgfqpoint{5.209274in}{2.452384in}}%
\pgfpathmoveto{\pgfqpoint{5.213815in}{2.449435in}}%
\pgfpathlineto{\pgfqpoint{5.213815in}{2.449435in}}%
\pgfpathlineto{\pgfqpoint{5.213815in}{2.452384in}}%
\pgfpathlineto{\pgfqpoint{5.218356in}{2.452384in}}%
\pgfpathlineto{\pgfqpoint{5.218356in}{2.449435in}}%
\pgfpathmoveto{\pgfqpoint{5.209274in}{2.452384in}}%
\pgfpathlineto{\pgfqpoint{5.209274in}{2.452384in}}%
\pgfpathlineto{\pgfqpoint{5.209274in}{2.455333in}}%
\pgfpathlineto{\pgfqpoint{5.213815in}{2.455333in}}%
\pgfpathlineto{\pgfqpoint{5.213815in}{2.452384in}}%
\pgfpathmoveto{\pgfqpoint{5.213815in}{2.452384in}}%
\pgfpathlineto{\pgfqpoint{5.213815in}{2.452384in}}%
\pgfpathlineto{\pgfqpoint{5.213815in}{2.455333in}}%
\pgfpathlineto{\pgfqpoint{5.218356in}{2.455333in}}%
\pgfpathlineto{\pgfqpoint{5.218356in}{2.452384in}}%
\pgfpathmoveto{\pgfqpoint{5.182028in}{2.458282in}}%
\pgfpathlineto{\pgfqpoint{5.182028in}{2.458282in}}%
\pgfpathlineto{\pgfqpoint{5.182028in}{2.461231in}}%
\pgfpathlineto{\pgfqpoint{5.186569in}{2.461231in}}%
\pgfpathlineto{\pgfqpoint{5.186569in}{2.458282in}}%
\pgfpathmoveto{\pgfqpoint{5.186569in}{2.458282in}}%
\pgfpathlineto{\pgfqpoint{5.186569in}{2.458282in}}%
\pgfpathlineto{\pgfqpoint{5.186569in}{2.461231in}}%
\pgfpathlineto{\pgfqpoint{5.191110in}{2.461231in}}%
\pgfpathlineto{\pgfqpoint{5.191110in}{2.458282in}}%
\pgfpathmoveto{\pgfqpoint{5.218356in}{2.449435in}}%
\pgfpathlineto{\pgfqpoint{5.218356in}{2.449435in}}%
\pgfpathlineto{\pgfqpoint{5.218356in}{2.452384in}}%
\pgfpathlineto{\pgfqpoint{5.222897in}{2.452384in}}%
\pgfpathlineto{\pgfqpoint{5.222897in}{2.449435in}}%
\pgfpathmoveto{\pgfqpoint{5.222897in}{2.449435in}}%
\pgfpathlineto{\pgfqpoint{5.222897in}{2.449435in}}%
\pgfpathlineto{\pgfqpoint{5.222897in}{2.452384in}}%
\pgfpathlineto{\pgfqpoint{5.227438in}{2.452384in}}%
\pgfpathlineto{\pgfqpoint{5.227438in}{2.449435in}}%
\pgfpathmoveto{\pgfqpoint{5.227438in}{2.446485in}}%
\pgfpathlineto{\pgfqpoint{5.227438in}{2.446485in}}%
\pgfpathlineto{\pgfqpoint{5.227438in}{2.449435in}}%
\pgfpathlineto{\pgfqpoint{5.231979in}{2.449435in}}%
\pgfpathlineto{\pgfqpoint{5.231979in}{2.446485in}}%
\pgfpathmoveto{\pgfqpoint{5.227438in}{2.449435in}}%
\pgfpathlineto{\pgfqpoint{5.227438in}{2.449435in}}%
\pgfpathlineto{\pgfqpoint{5.227438in}{2.452384in}}%
\pgfpathlineto{\pgfqpoint{5.231979in}{2.452384in}}%
\pgfpathlineto{\pgfqpoint{5.231979in}{2.449435in}}%
\pgfpathmoveto{\pgfqpoint{5.231979in}{2.446485in}}%
\pgfpathlineto{\pgfqpoint{5.231979in}{2.446485in}}%
\pgfpathlineto{\pgfqpoint{5.231979in}{2.449435in}}%
\pgfpathlineto{\pgfqpoint{5.236520in}{2.449435in}}%
\pgfpathlineto{\pgfqpoint{5.236520in}{2.446485in}}%
\pgfpathmoveto{\pgfqpoint{5.241061in}{2.443536in}}%
\pgfpathlineto{\pgfqpoint{5.241061in}{2.443536in}}%
\pgfpathlineto{\pgfqpoint{5.241061in}{2.446485in}}%
\pgfpathlineto{\pgfqpoint{5.245602in}{2.446485in}}%
\pgfpathlineto{\pgfqpoint{5.245602in}{2.443536in}}%
\pgfpathmoveto{\pgfqpoint{5.245602in}{2.443536in}}%
\pgfpathlineto{\pgfqpoint{5.245602in}{2.443536in}}%
\pgfpathlineto{\pgfqpoint{5.245602in}{2.446485in}}%
\pgfpathlineto{\pgfqpoint{5.250143in}{2.446485in}}%
\pgfpathlineto{\pgfqpoint{5.250143in}{2.443536in}}%
\pgfpathmoveto{\pgfqpoint{5.250143in}{2.443536in}}%
\pgfpathlineto{\pgfqpoint{5.250143in}{2.443536in}}%
\pgfpathlineto{\pgfqpoint{5.250143in}{2.446485in}}%
\pgfpathlineto{\pgfqpoint{5.254684in}{2.446485in}}%
\pgfpathlineto{\pgfqpoint{5.254684in}{2.443536in}}%
\pgfpathmoveto{\pgfqpoint{5.236520in}{2.446485in}}%
\pgfpathlineto{\pgfqpoint{5.236520in}{2.446485in}}%
\pgfpathlineto{\pgfqpoint{5.236520in}{2.449435in}}%
\pgfpathlineto{\pgfqpoint{5.241061in}{2.449435in}}%
\pgfpathlineto{\pgfqpoint{5.241061in}{2.446485in}}%
\pgfpathmoveto{\pgfqpoint{5.241061in}{2.446485in}}%
\pgfpathlineto{\pgfqpoint{5.241061in}{2.446485in}}%
\pgfpathlineto{\pgfqpoint{5.241061in}{2.449435in}}%
\pgfpathlineto{\pgfqpoint{5.245602in}{2.449435in}}%
\pgfpathlineto{\pgfqpoint{5.245602in}{2.446485in}}%
\pgfpathmoveto{\pgfqpoint{5.295555in}{2.431739in}}%
\pgfpathlineto{\pgfqpoint{5.295555in}{2.431739in}}%
\pgfpathlineto{\pgfqpoint{5.295555in}{2.434688in}}%
\pgfpathlineto{\pgfqpoint{5.300096in}{2.434688in}}%
\pgfpathlineto{\pgfqpoint{5.300096in}{2.431739in}}%
\pgfpathmoveto{\pgfqpoint{5.300096in}{2.431739in}}%
\pgfpathlineto{\pgfqpoint{5.300096in}{2.431739in}}%
\pgfpathlineto{\pgfqpoint{5.300096in}{2.434688in}}%
\pgfpathlineto{\pgfqpoint{5.304637in}{2.434688in}}%
\pgfpathlineto{\pgfqpoint{5.304637in}{2.431739in}}%
\pgfpathmoveto{\pgfqpoint{5.304637in}{2.431739in}}%
\pgfpathlineto{\pgfqpoint{5.304637in}{2.431739in}}%
\pgfpathlineto{\pgfqpoint{5.304637in}{2.434688in}}%
\pgfpathlineto{\pgfqpoint{5.309179in}{2.434688in}}%
\pgfpathlineto{\pgfqpoint{5.309179in}{2.431739in}}%
\pgfpathmoveto{\pgfqpoint{5.309179in}{2.428790in}}%
\pgfpathlineto{\pgfqpoint{5.309179in}{2.428790in}}%
\pgfpathlineto{\pgfqpoint{5.309179in}{2.431739in}}%
\pgfpathlineto{\pgfqpoint{5.313720in}{2.431739in}}%
\pgfpathlineto{\pgfqpoint{5.313720in}{2.428790in}}%
\pgfpathmoveto{\pgfqpoint{5.309179in}{2.431739in}}%
\pgfpathlineto{\pgfqpoint{5.309179in}{2.431739in}}%
\pgfpathlineto{\pgfqpoint{5.309179in}{2.434688in}}%
\pgfpathlineto{\pgfqpoint{5.313720in}{2.434688in}}%
\pgfpathlineto{\pgfqpoint{5.313720in}{2.431739in}}%
\pgfpathmoveto{\pgfqpoint{5.313720in}{2.428790in}}%
\pgfpathlineto{\pgfqpoint{5.313720in}{2.428790in}}%
\pgfpathlineto{\pgfqpoint{5.313720in}{2.431739in}}%
\pgfpathlineto{\pgfqpoint{5.318261in}{2.431739in}}%
\pgfpathlineto{\pgfqpoint{5.318261in}{2.428790in}}%
\pgfpathmoveto{\pgfqpoint{5.322802in}{2.425841in}}%
\pgfpathlineto{\pgfqpoint{5.322802in}{2.425841in}}%
\pgfpathlineto{\pgfqpoint{5.322802in}{2.428790in}}%
\pgfpathlineto{\pgfqpoint{5.327344in}{2.428790in}}%
\pgfpathlineto{\pgfqpoint{5.327344in}{2.425841in}}%
\pgfpathmoveto{\pgfqpoint{5.318261in}{2.428790in}}%
\pgfpathlineto{\pgfqpoint{5.318261in}{2.428790in}}%
\pgfpathlineto{\pgfqpoint{5.318261in}{2.431739in}}%
\pgfpathlineto{\pgfqpoint{5.322802in}{2.431739in}}%
\pgfpathlineto{\pgfqpoint{5.322802in}{2.428790in}}%
\pgfpathmoveto{\pgfqpoint{5.322802in}{2.428790in}}%
\pgfpathlineto{\pgfqpoint{5.322802in}{2.428790in}}%
\pgfpathlineto{\pgfqpoint{5.322802in}{2.431739in}}%
\pgfpathlineto{\pgfqpoint{5.327344in}{2.431739in}}%
\pgfpathlineto{\pgfqpoint{5.327344in}{2.428790in}}%
\pgfpathmoveto{\pgfqpoint{5.254684in}{2.440587in}}%
\pgfpathlineto{\pgfqpoint{5.254684in}{2.440587in}}%
\pgfpathlineto{\pgfqpoint{5.254684in}{2.443536in}}%
\pgfpathlineto{\pgfqpoint{5.259225in}{2.443536in}}%
\pgfpathlineto{\pgfqpoint{5.259225in}{2.440587in}}%
\pgfpathmoveto{\pgfqpoint{5.254684in}{2.443536in}}%
\pgfpathlineto{\pgfqpoint{5.254684in}{2.443536in}}%
\pgfpathlineto{\pgfqpoint{5.254684in}{2.446485in}}%
\pgfpathlineto{\pgfqpoint{5.259225in}{2.446485in}}%
\pgfpathlineto{\pgfqpoint{5.259225in}{2.443536in}}%
\pgfpathmoveto{\pgfqpoint{5.259225in}{2.440587in}}%
\pgfpathlineto{\pgfqpoint{5.259225in}{2.440587in}}%
\pgfpathlineto{\pgfqpoint{5.259225in}{2.443536in}}%
\pgfpathlineto{\pgfqpoint{5.263766in}{2.443536in}}%
\pgfpathlineto{\pgfqpoint{5.263766in}{2.440587in}}%
\pgfpathmoveto{\pgfqpoint{5.268308in}{2.437638in}}%
\pgfpathlineto{\pgfqpoint{5.268308in}{2.437638in}}%
\pgfpathlineto{\pgfqpoint{5.268308in}{2.440587in}}%
\pgfpathlineto{\pgfqpoint{5.272849in}{2.440587in}}%
\pgfpathlineto{\pgfqpoint{5.272849in}{2.437638in}}%
\pgfpathmoveto{\pgfqpoint{5.263766in}{2.440587in}}%
\pgfpathlineto{\pgfqpoint{5.263766in}{2.440587in}}%
\pgfpathlineto{\pgfqpoint{5.263766in}{2.443536in}}%
\pgfpathlineto{\pgfqpoint{5.268308in}{2.443536in}}%
\pgfpathlineto{\pgfqpoint{5.268308in}{2.440587in}}%
\pgfpathmoveto{\pgfqpoint{5.268308in}{2.440587in}}%
\pgfpathlineto{\pgfqpoint{5.268308in}{2.440587in}}%
\pgfpathlineto{\pgfqpoint{5.268308in}{2.443536in}}%
\pgfpathlineto{\pgfqpoint{5.272849in}{2.443536in}}%
\pgfpathlineto{\pgfqpoint{5.272849in}{2.440587in}}%
\pgfpathmoveto{\pgfqpoint{5.272849in}{2.437638in}}%
\pgfpathlineto{\pgfqpoint{5.272849in}{2.437638in}}%
\pgfpathlineto{\pgfqpoint{5.272849in}{2.440587in}}%
\pgfpathlineto{\pgfqpoint{5.277390in}{2.440587in}}%
\pgfpathlineto{\pgfqpoint{5.277390in}{2.437638in}}%
\pgfpathmoveto{\pgfqpoint{5.277390in}{2.437638in}}%
\pgfpathlineto{\pgfqpoint{5.277390in}{2.437638in}}%
\pgfpathlineto{\pgfqpoint{5.277390in}{2.440587in}}%
\pgfpathlineto{\pgfqpoint{5.281931in}{2.440587in}}%
\pgfpathlineto{\pgfqpoint{5.281931in}{2.437638in}}%
\pgfpathmoveto{\pgfqpoint{5.281931in}{2.434688in}}%
\pgfpathlineto{\pgfqpoint{5.281931in}{2.434688in}}%
\pgfpathlineto{\pgfqpoint{5.281931in}{2.437638in}}%
\pgfpathlineto{\pgfqpoint{5.286473in}{2.437638in}}%
\pgfpathlineto{\pgfqpoint{5.286473in}{2.434688in}}%
\pgfpathmoveto{\pgfqpoint{5.281931in}{2.437638in}}%
\pgfpathlineto{\pgfqpoint{5.281931in}{2.437638in}}%
\pgfpathlineto{\pgfqpoint{5.281931in}{2.440587in}}%
\pgfpathlineto{\pgfqpoint{5.286473in}{2.440587in}}%
\pgfpathlineto{\pgfqpoint{5.286473in}{2.437638in}}%
\pgfpathmoveto{\pgfqpoint{5.286473in}{2.434688in}}%
\pgfpathlineto{\pgfqpoint{5.286473in}{2.434688in}}%
\pgfpathlineto{\pgfqpoint{5.286473in}{2.437638in}}%
\pgfpathlineto{\pgfqpoint{5.291014in}{2.437638in}}%
\pgfpathlineto{\pgfqpoint{5.291014in}{2.434688in}}%
\pgfpathmoveto{\pgfqpoint{5.291014in}{2.434688in}}%
\pgfpathlineto{\pgfqpoint{5.291014in}{2.434688in}}%
\pgfpathlineto{\pgfqpoint{5.291014in}{2.437638in}}%
\pgfpathlineto{\pgfqpoint{5.295555in}{2.437638in}}%
\pgfpathlineto{\pgfqpoint{5.295555in}{2.434688in}}%
\pgfpathmoveto{\pgfqpoint{5.295555in}{2.434688in}}%
\pgfpathlineto{\pgfqpoint{5.295555in}{2.434688in}}%
\pgfpathlineto{\pgfqpoint{5.295555in}{2.437638in}}%
\pgfpathlineto{\pgfqpoint{5.300096in}{2.437638in}}%
\pgfpathlineto{\pgfqpoint{5.300096in}{2.434688in}}%
\pgfpathmoveto{\pgfqpoint{5.327344in}{2.425841in}}%
\pgfpathlineto{\pgfqpoint{5.327344in}{2.425841in}}%
\pgfpathlineto{\pgfqpoint{5.327344in}{2.428790in}}%
\pgfpathlineto{\pgfqpoint{5.331885in}{2.428790in}}%
\pgfpathlineto{\pgfqpoint{5.331885in}{2.425841in}}%
\pgfpathmoveto{\pgfqpoint{5.331885in}{2.425841in}}%
\pgfpathlineto{\pgfqpoint{5.331885in}{2.425841in}}%
\pgfpathlineto{\pgfqpoint{5.331885in}{2.428790in}}%
\pgfpathlineto{\pgfqpoint{5.336426in}{2.428790in}}%
\pgfpathlineto{\pgfqpoint{5.336426in}{2.425841in}}%
\pgfpathmoveto{\pgfqpoint{5.336426in}{2.422892in}}%
\pgfpathlineto{\pgfqpoint{5.336426in}{2.422892in}}%
\pgfpathlineto{\pgfqpoint{5.336426in}{2.425841in}}%
\pgfpathlineto{\pgfqpoint{5.340967in}{2.425841in}}%
\pgfpathlineto{\pgfqpoint{5.340967in}{2.422892in}}%
\pgfpathmoveto{\pgfqpoint{5.336426in}{2.425841in}}%
\pgfpathlineto{\pgfqpoint{5.336426in}{2.425841in}}%
\pgfpathlineto{\pgfqpoint{5.336426in}{2.428790in}}%
\pgfpathlineto{\pgfqpoint{5.340967in}{2.428790in}}%
\pgfpathlineto{\pgfqpoint{5.340967in}{2.425841in}}%
\pgfpathmoveto{\pgfqpoint{5.340967in}{2.422892in}}%
\pgfpathlineto{\pgfqpoint{5.340967in}{2.422892in}}%
\pgfpathlineto{\pgfqpoint{5.340967in}{2.425841in}}%
\pgfpathlineto{\pgfqpoint{5.345508in}{2.425841in}}%
\pgfpathlineto{\pgfqpoint{5.345508in}{2.422892in}}%
\pgfpathmoveto{\pgfqpoint{5.350050in}{2.419942in}}%
\pgfpathlineto{\pgfqpoint{5.350050in}{2.419942in}}%
\pgfpathlineto{\pgfqpoint{5.350050in}{2.422892in}}%
\pgfpathlineto{\pgfqpoint{5.354591in}{2.422892in}}%
\pgfpathlineto{\pgfqpoint{5.354591in}{2.419942in}}%
\pgfpathmoveto{\pgfqpoint{5.354591in}{2.419942in}}%
\pgfpathlineto{\pgfqpoint{5.354591in}{2.419942in}}%
\pgfpathlineto{\pgfqpoint{5.354591in}{2.422892in}}%
\pgfpathlineto{\pgfqpoint{5.359132in}{2.422892in}}%
\pgfpathlineto{\pgfqpoint{5.359132in}{2.419942in}}%
\pgfpathmoveto{\pgfqpoint{5.359132in}{2.419942in}}%
\pgfpathlineto{\pgfqpoint{5.359132in}{2.419942in}}%
\pgfpathlineto{\pgfqpoint{5.359132in}{2.422892in}}%
\pgfpathlineto{\pgfqpoint{5.363673in}{2.422892in}}%
\pgfpathlineto{\pgfqpoint{5.363673in}{2.419942in}}%
\pgfpathmoveto{\pgfqpoint{5.345508in}{2.422892in}}%
\pgfpathlineto{\pgfqpoint{5.345508in}{2.422892in}}%
\pgfpathlineto{\pgfqpoint{5.345508in}{2.425841in}}%
\pgfpathlineto{\pgfqpoint{5.350050in}{2.425841in}}%
\pgfpathlineto{\pgfqpoint{5.350050in}{2.422892in}}%
\pgfpathmoveto{\pgfqpoint{5.350050in}{2.422892in}}%
\pgfpathlineto{\pgfqpoint{5.350050in}{2.422892in}}%
\pgfpathlineto{\pgfqpoint{5.350050in}{2.425841in}}%
\pgfpathlineto{\pgfqpoint{5.354591in}{2.425841in}}%
\pgfpathlineto{\pgfqpoint{5.354591in}{2.422892in}}%
\pgfpathmoveto{\pgfqpoint{5.363673in}{2.416993in}}%
\pgfpathlineto{\pgfqpoint{5.363673in}{2.416993in}}%
\pgfpathlineto{\pgfqpoint{5.363673in}{2.419942in}}%
\pgfpathlineto{\pgfqpoint{5.368215in}{2.419942in}}%
\pgfpathlineto{\pgfqpoint{5.368215in}{2.416993in}}%
\pgfpathmoveto{\pgfqpoint{5.363673in}{2.419942in}}%
\pgfpathlineto{\pgfqpoint{5.363673in}{2.419942in}}%
\pgfpathlineto{\pgfqpoint{5.363673in}{2.422892in}}%
\pgfpathlineto{\pgfqpoint{5.368215in}{2.422892in}}%
\pgfpathlineto{\pgfqpoint{5.368215in}{2.419942in}}%
\pgfpathmoveto{\pgfqpoint{5.368215in}{2.416993in}}%
\pgfpathlineto{\pgfqpoint{5.368215in}{2.416993in}}%
\pgfpathlineto{\pgfqpoint{5.368215in}{2.419942in}}%
\pgfpathlineto{\pgfqpoint{5.372756in}{2.419942in}}%
\pgfpathlineto{\pgfqpoint{5.372756in}{2.416993in}}%
\pgfpathmoveto{\pgfqpoint{5.377297in}{2.414044in}}%
\pgfpathlineto{\pgfqpoint{5.377297in}{2.414044in}}%
\pgfpathlineto{\pgfqpoint{5.377297in}{2.416993in}}%
\pgfpathlineto{\pgfqpoint{5.381838in}{2.416993in}}%
\pgfpathlineto{\pgfqpoint{5.381838in}{2.414044in}}%
\pgfpathmoveto{\pgfqpoint{5.372756in}{2.416993in}}%
\pgfpathlineto{\pgfqpoint{5.372756in}{2.416993in}}%
\pgfpathlineto{\pgfqpoint{5.372756in}{2.419942in}}%
\pgfpathlineto{\pgfqpoint{5.377297in}{2.419942in}}%
\pgfpathlineto{\pgfqpoint{5.377297in}{2.416993in}}%
\pgfpathmoveto{\pgfqpoint{5.377297in}{2.416993in}}%
\pgfpathlineto{\pgfqpoint{5.377297in}{2.416993in}}%
\pgfpathlineto{\pgfqpoint{5.377297in}{2.419942in}}%
\pgfpathlineto{\pgfqpoint{5.381838in}{2.419942in}}%
\pgfpathlineto{\pgfqpoint{5.381838in}{2.416993in}}%
\pgfpathmoveto{\pgfqpoint{5.381838in}{2.414044in}}%
\pgfpathlineto{\pgfqpoint{5.381838in}{2.414044in}}%
\pgfpathlineto{\pgfqpoint{5.381838in}{2.416993in}}%
\pgfpathlineto{\pgfqpoint{5.386379in}{2.416993in}}%
\pgfpathlineto{\pgfqpoint{5.386379in}{2.414044in}}%
\pgfpathmoveto{\pgfqpoint{5.386379in}{2.414044in}}%
\pgfpathlineto{\pgfqpoint{5.386379in}{2.414044in}}%
\pgfpathlineto{\pgfqpoint{5.386379in}{2.416993in}}%
\pgfpathlineto{\pgfqpoint{5.390921in}{2.416993in}}%
\pgfpathlineto{\pgfqpoint{5.390921in}{2.414044in}}%
\pgfpathmoveto{\pgfqpoint{5.390921in}{2.411095in}}%
\pgfpathlineto{\pgfqpoint{5.390921in}{2.411095in}}%
\pgfpathlineto{\pgfqpoint{5.390921in}{2.414044in}}%
\pgfpathlineto{\pgfqpoint{5.395462in}{2.414044in}}%
\pgfpathlineto{\pgfqpoint{5.395462in}{2.411095in}}%
\pgfpathmoveto{\pgfqpoint{5.390921in}{2.414044in}}%
\pgfpathlineto{\pgfqpoint{5.390921in}{2.414044in}}%
\pgfpathlineto{\pgfqpoint{5.390921in}{2.416993in}}%
\pgfpathlineto{\pgfqpoint{5.395462in}{2.416993in}}%
\pgfpathlineto{\pgfqpoint{5.395462in}{2.414044in}}%
\pgfpathmoveto{\pgfqpoint{5.395462in}{2.411095in}}%
\pgfpathlineto{\pgfqpoint{5.395462in}{2.411095in}}%
\pgfpathlineto{\pgfqpoint{5.395462in}{2.414044in}}%
\pgfpathlineto{\pgfqpoint{5.400003in}{2.414044in}}%
\pgfpathlineto{\pgfqpoint{5.400003in}{2.411095in}}%
\pgfpathclose%
\pgfusepath{fill}%
\end{pgfscope}%
\begin{pgfscope}%
\pgfpathrectangle{\pgfqpoint{0.750000in}{0.500000in}}{\pgfqpoint{4.650000in}{3.020000in}}%
\pgfusepath{clip}%
\pgfsetbuttcap%
\pgfsetmiterjoin%
\definecolor{currentfill}{rgb}{1.000000,0.000000,0.000000}%
\pgfsetfillcolor{currentfill}%
\pgfsetlinewidth{0.000000pt}%
\definecolor{currentstroke}{rgb}{0.000000,0.000000,0.000000}%
\pgfsetstrokecolor{currentstroke}%
\pgfsetstrokeopacity{0.000000}%
\pgfsetdash{}{0pt}%
\pgfpathmoveto{\pgfqpoint{3.070459in}{0.500002in}}%
\pgfpathlineto{\pgfqpoint{3.070459in}{0.502951in}}%
\pgfpathlineto{\pgfqpoint{3.075000in}{0.502951in}}%
\pgfpathlineto{\pgfqpoint{3.075000in}{0.500002in}}%
\pgfpathmoveto{\pgfqpoint{3.070459in}{0.502951in}}%
\pgfpathlineto{\pgfqpoint{3.070459in}{0.502951in}}%
\pgfpathlineto{\pgfqpoint{3.070459in}{0.505901in}}%
\pgfpathlineto{\pgfqpoint{3.075000in}{0.505901in}}%
\pgfpathlineto{\pgfqpoint{3.075000in}{0.502951in}}%
\pgfpathmoveto{\pgfqpoint{3.070459in}{0.505901in}}%
\pgfpathlineto{\pgfqpoint{3.070459in}{0.505901in}}%
\pgfpathlineto{\pgfqpoint{3.070459in}{0.508850in}}%
\pgfpathlineto{\pgfqpoint{3.075000in}{0.508850in}}%
\pgfpathlineto{\pgfqpoint{3.075000in}{0.505901in}}%
\pgfpathmoveto{\pgfqpoint{3.070459in}{0.508850in}}%
\pgfpathlineto{\pgfqpoint{3.070459in}{0.508850in}}%
\pgfpathlineto{\pgfqpoint{3.070459in}{0.511799in}}%
\pgfpathlineto{\pgfqpoint{3.075000in}{0.511799in}}%
\pgfpathlineto{\pgfqpoint{3.075000in}{0.508850in}}%
\pgfpathmoveto{\pgfqpoint{3.070459in}{0.511799in}}%
\pgfpathlineto{\pgfqpoint{3.070459in}{0.511799in}}%
\pgfpathlineto{\pgfqpoint{3.070459in}{0.514748in}}%
\pgfpathlineto{\pgfqpoint{3.075000in}{0.514748in}}%
\pgfpathlineto{\pgfqpoint{3.075000in}{0.511799in}}%
\pgfpathmoveto{\pgfqpoint{3.070459in}{0.514748in}}%
\pgfpathlineto{\pgfqpoint{3.070459in}{0.514748in}}%
\pgfpathlineto{\pgfqpoint{3.070459in}{0.517697in}}%
\pgfpathlineto{\pgfqpoint{3.075000in}{0.517697in}}%
\pgfpathlineto{\pgfqpoint{3.075000in}{0.514748in}}%
\pgfpathmoveto{\pgfqpoint{3.070459in}{0.517697in}}%
\pgfpathlineto{\pgfqpoint{3.070459in}{0.517697in}}%
\pgfpathlineto{\pgfqpoint{3.070459in}{0.520646in}}%
\pgfpathlineto{\pgfqpoint{3.075000in}{0.520646in}}%
\pgfpathlineto{\pgfqpoint{3.075000in}{0.517697in}}%
\pgfpathmoveto{\pgfqpoint{3.070459in}{0.520646in}}%
\pgfpathlineto{\pgfqpoint{3.070459in}{0.520646in}}%
\pgfpathlineto{\pgfqpoint{3.070459in}{0.523595in}}%
\pgfpathlineto{\pgfqpoint{3.075000in}{0.523595in}}%
\pgfpathlineto{\pgfqpoint{3.075000in}{0.520646in}}%
\pgfpathmoveto{\pgfqpoint{3.070459in}{0.523595in}}%
\pgfpathlineto{\pgfqpoint{3.070459in}{0.523595in}}%
\pgfpathlineto{\pgfqpoint{3.070459in}{0.526545in}}%
\pgfpathlineto{\pgfqpoint{3.075000in}{0.526545in}}%
\pgfpathlineto{\pgfqpoint{3.075000in}{0.523595in}}%
\pgfpathmoveto{\pgfqpoint{3.070459in}{0.526545in}}%
\pgfpathlineto{\pgfqpoint{3.070459in}{0.526545in}}%
\pgfpathlineto{\pgfqpoint{3.070459in}{0.529494in}}%
\pgfpathlineto{\pgfqpoint{3.075000in}{0.529494in}}%
\pgfpathlineto{\pgfqpoint{3.075000in}{0.526545in}}%
\pgfpathmoveto{\pgfqpoint{3.070459in}{0.529494in}}%
\pgfpathlineto{\pgfqpoint{3.070459in}{0.529494in}}%
\pgfpathlineto{\pgfqpoint{3.070459in}{0.532443in}}%
\pgfpathlineto{\pgfqpoint{3.075000in}{0.532443in}}%
\pgfpathlineto{\pgfqpoint{3.075000in}{0.529494in}}%
\pgfpathmoveto{\pgfqpoint{3.070459in}{0.532443in}}%
\pgfpathlineto{\pgfqpoint{3.070459in}{0.532443in}}%
\pgfpathlineto{\pgfqpoint{3.070459in}{0.535392in}}%
\pgfpathlineto{\pgfqpoint{3.075000in}{0.535392in}}%
\pgfpathlineto{\pgfqpoint{3.075000in}{0.532443in}}%
\pgfpathmoveto{\pgfqpoint{3.070459in}{0.535392in}}%
\pgfpathlineto{\pgfqpoint{3.070459in}{0.535392in}}%
\pgfpathlineto{\pgfqpoint{3.070459in}{0.538341in}}%
\pgfpathlineto{\pgfqpoint{3.075000in}{0.538341in}}%
\pgfpathlineto{\pgfqpoint{3.075000in}{0.535392in}}%
\pgfpathmoveto{\pgfqpoint{3.070459in}{0.538341in}}%
\pgfpathlineto{\pgfqpoint{3.070459in}{0.538341in}}%
\pgfpathlineto{\pgfqpoint{3.070459in}{0.541290in}}%
\pgfpathlineto{\pgfqpoint{3.075000in}{0.541290in}}%
\pgfpathlineto{\pgfqpoint{3.075000in}{0.538341in}}%
\pgfpathmoveto{\pgfqpoint{3.070459in}{0.541290in}}%
\pgfpathlineto{\pgfqpoint{3.070459in}{0.541290in}}%
\pgfpathlineto{\pgfqpoint{3.070459in}{0.544239in}}%
\pgfpathlineto{\pgfqpoint{3.075000in}{0.544239in}}%
\pgfpathlineto{\pgfqpoint{3.075000in}{0.541290in}}%
\pgfpathmoveto{\pgfqpoint{3.070459in}{0.544239in}}%
\pgfpathlineto{\pgfqpoint{3.070459in}{0.544239in}}%
\pgfpathlineto{\pgfqpoint{3.070459in}{0.547188in}}%
\pgfpathlineto{\pgfqpoint{3.075000in}{0.547188in}}%
\pgfpathlineto{\pgfqpoint{3.075000in}{0.544239in}}%
\pgfpathmoveto{\pgfqpoint{3.070459in}{0.547188in}}%
\pgfpathlineto{\pgfqpoint{3.070459in}{0.547188in}}%
\pgfpathlineto{\pgfqpoint{3.070459in}{0.550138in}}%
\pgfpathlineto{\pgfqpoint{3.075000in}{0.550138in}}%
\pgfpathlineto{\pgfqpoint{3.075000in}{0.547188in}}%
\pgfpathmoveto{\pgfqpoint{3.070459in}{0.550138in}}%
\pgfpathlineto{\pgfqpoint{3.070459in}{0.550138in}}%
\pgfpathlineto{\pgfqpoint{3.070459in}{0.553087in}}%
\pgfpathlineto{\pgfqpoint{3.075000in}{0.553087in}}%
\pgfpathlineto{\pgfqpoint{3.075000in}{0.550138in}}%
\pgfpathmoveto{\pgfqpoint{3.070459in}{0.553087in}}%
\pgfpathlineto{\pgfqpoint{3.070459in}{0.553087in}}%
\pgfpathlineto{\pgfqpoint{3.070459in}{0.556036in}}%
\pgfpathlineto{\pgfqpoint{3.075000in}{0.556036in}}%
\pgfpathlineto{\pgfqpoint{3.075000in}{0.553087in}}%
\pgfpathmoveto{\pgfqpoint{3.070459in}{0.556036in}}%
\pgfpathlineto{\pgfqpoint{3.070459in}{0.556036in}}%
\pgfpathlineto{\pgfqpoint{3.070459in}{0.558985in}}%
\pgfpathlineto{\pgfqpoint{3.075000in}{0.558985in}}%
\pgfpathlineto{\pgfqpoint{3.075000in}{0.556036in}}%
\pgfpathmoveto{\pgfqpoint{3.070459in}{0.558985in}}%
\pgfpathlineto{\pgfqpoint{3.070459in}{0.558985in}}%
\pgfpathlineto{\pgfqpoint{3.070459in}{0.561934in}}%
\pgfpathlineto{\pgfqpoint{3.075000in}{0.561934in}}%
\pgfpathlineto{\pgfqpoint{3.075000in}{0.558985in}}%
\pgfpathmoveto{\pgfqpoint{3.070459in}{0.561934in}}%
\pgfpathlineto{\pgfqpoint{3.070459in}{0.561934in}}%
\pgfpathlineto{\pgfqpoint{3.070459in}{0.564883in}}%
\pgfpathlineto{\pgfqpoint{3.075000in}{0.564883in}}%
\pgfpathlineto{\pgfqpoint{3.075000in}{0.561934in}}%
\pgfpathmoveto{\pgfqpoint{3.070459in}{0.564883in}}%
\pgfpathlineto{\pgfqpoint{3.070459in}{0.564883in}}%
\pgfpathlineto{\pgfqpoint{3.070459in}{0.567832in}}%
\pgfpathlineto{\pgfqpoint{3.075000in}{0.567832in}}%
\pgfpathlineto{\pgfqpoint{3.075000in}{0.564883in}}%
\pgfpathmoveto{\pgfqpoint{3.070459in}{0.567832in}}%
\pgfpathlineto{\pgfqpoint{3.070459in}{0.567832in}}%
\pgfpathlineto{\pgfqpoint{3.070459in}{0.570782in}}%
\pgfpathlineto{\pgfqpoint{3.075000in}{0.570782in}}%
\pgfpathlineto{\pgfqpoint{3.075000in}{0.567832in}}%
\pgfpathmoveto{\pgfqpoint{3.070459in}{0.570782in}}%
\pgfpathlineto{\pgfqpoint{3.070459in}{0.570782in}}%
\pgfpathlineto{\pgfqpoint{3.070459in}{0.573731in}}%
\pgfpathlineto{\pgfqpoint{3.075000in}{0.573731in}}%
\pgfpathlineto{\pgfqpoint{3.075000in}{0.570782in}}%
\pgfpathmoveto{\pgfqpoint{3.070459in}{0.573731in}}%
\pgfpathlineto{\pgfqpoint{3.070459in}{0.573731in}}%
\pgfpathlineto{\pgfqpoint{3.070459in}{0.576680in}}%
\pgfpathlineto{\pgfqpoint{3.075000in}{0.576680in}}%
\pgfpathlineto{\pgfqpoint{3.075000in}{0.573731in}}%
\pgfpathmoveto{\pgfqpoint{3.070459in}{0.576680in}}%
\pgfpathlineto{\pgfqpoint{3.070459in}{0.576680in}}%
\pgfpathlineto{\pgfqpoint{3.070459in}{0.579629in}}%
\pgfpathlineto{\pgfqpoint{3.075000in}{0.579629in}}%
\pgfpathlineto{\pgfqpoint{3.075000in}{0.576680in}}%
\pgfpathmoveto{\pgfqpoint{3.070459in}{0.579629in}}%
\pgfpathlineto{\pgfqpoint{3.070459in}{0.579629in}}%
\pgfpathlineto{\pgfqpoint{3.070459in}{0.582578in}}%
\pgfpathlineto{\pgfqpoint{3.075000in}{0.582578in}}%
\pgfpathlineto{\pgfqpoint{3.075000in}{0.579629in}}%
\pgfpathmoveto{\pgfqpoint{3.070459in}{0.582578in}}%
\pgfpathlineto{\pgfqpoint{3.070459in}{0.582578in}}%
\pgfpathlineto{\pgfqpoint{3.070459in}{0.585527in}}%
\pgfpathlineto{\pgfqpoint{3.075000in}{0.585527in}}%
\pgfpathlineto{\pgfqpoint{3.075000in}{0.582578in}}%
\pgfpathmoveto{\pgfqpoint{3.070459in}{0.585527in}}%
\pgfpathlineto{\pgfqpoint{3.070459in}{0.585527in}}%
\pgfpathlineto{\pgfqpoint{3.070459in}{0.588476in}}%
\pgfpathlineto{\pgfqpoint{3.075000in}{0.588476in}}%
\pgfpathlineto{\pgfqpoint{3.075000in}{0.585527in}}%
\pgfpathmoveto{\pgfqpoint{3.070459in}{0.588476in}}%
\pgfpathlineto{\pgfqpoint{3.070459in}{0.588476in}}%
\pgfpathlineto{\pgfqpoint{3.070459in}{0.591425in}}%
\pgfpathlineto{\pgfqpoint{3.075000in}{0.591425in}}%
\pgfpathlineto{\pgfqpoint{3.075000in}{0.588476in}}%
\pgfpathmoveto{\pgfqpoint{3.070459in}{0.591425in}}%
\pgfpathlineto{\pgfqpoint{3.070459in}{0.591425in}}%
\pgfpathlineto{\pgfqpoint{3.070459in}{0.594375in}}%
\pgfpathlineto{\pgfqpoint{3.075000in}{0.594375in}}%
\pgfpathlineto{\pgfqpoint{3.075000in}{0.591425in}}%
\pgfpathmoveto{\pgfqpoint{3.070459in}{0.594375in}}%
\pgfpathlineto{\pgfqpoint{3.070459in}{0.594375in}}%
\pgfpathlineto{\pgfqpoint{3.070459in}{0.597324in}}%
\pgfpathlineto{\pgfqpoint{3.075000in}{0.597324in}}%
\pgfpathlineto{\pgfqpoint{3.075000in}{0.594375in}}%
\pgfpathmoveto{\pgfqpoint{3.070459in}{0.597324in}}%
\pgfpathlineto{\pgfqpoint{3.070459in}{0.597324in}}%
\pgfpathlineto{\pgfqpoint{3.070459in}{0.600273in}}%
\pgfpathlineto{\pgfqpoint{3.075000in}{0.600273in}}%
\pgfpathlineto{\pgfqpoint{3.075000in}{0.597324in}}%
\pgfpathmoveto{\pgfqpoint{3.070459in}{0.600273in}}%
\pgfpathlineto{\pgfqpoint{3.070459in}{0.600273in}}%
\pgfpathlineto{\pgfqpoint{3.070459in}{0.603222in}}%
\pgfpathlineto{\pgfqpoint{3.075000in}{0.603222in}}%
\pgfpathlineto{\pgfqpoint{3.075000in}{0.600273in}}%
\pgfpathmoveto{\pgfqpoint{3.070459in}{0.603222in}}%
\pgfpathlineto{\pgfqpoint{3.070459in}{0.603222in}}%
\pgfpathlineto{\pgfqpoint{3.070459in}{0.606172in}}%
\pgfpathlineto{\pgfqpoint{3.075000in}{0.606172in}}%
\pgfpathlineto{\pgfqpoint{3.075000in}{0.603222in}}%
\pgfpathmoveto{\pgfqpoint{3.070459in}{0.606172in}}%
\pgfpathlineto{\pgfqpoint{3.070459in}{0.606172in}}%
\pgfpathlineto{\pgfqpoint{3.070459in}{0.609121in}}%
\pgfpathlineto{\pgfqpoint{3.075000in}{0.609121in}}%
\pgfpathlineto{\pgfqpoint{3.075000in}{0.606172in}}%
\pgfpathmoveto{\pgfqpoint{3.070459in}{0.609121in}}%
\pgfpathlineto{\pgfqpoint{3.070459in}{0.609121in}}%
\pgfpathlineto{\pgfqpoint{3.070459in}{0.612070in}}%
\pgfpathlineto{\pgfqpoint{3.075000in}{0.612070in}}%
\pgfpathlineto{\pgfqpoint{3.075000in}{0.609121in}}%
\pgfpathmoveto{\pgfqpoint{3.070459in}{0.612070in}}%
\pgfpathlineto{\pgfqpoint{3.070459in}{0.612070in}}%
\pgfpathlineto{\pgfqpoint{3.070459in}{0.615019in}}%
\pgfpathlineto{\pgfqpoint{3.075000in}{0.615019in}}%
\pgfpathlineto{\pgfqpoint{3.075000in}{0.612070in}}%
\pgfpathmoveto{\pgfqpoint{3.070459in}{0.615019in}}%
\pgfpathlineto{\pgfqpoint{3.070459in}{0.615019in}}%
\pgfpathlineto{\pgfqpoint{3.070459in}{0.617968in}}%
\pgfpathlineto{\pgfqpoint{3.075000in}{0.617968in}}%
\pgfpathlineto{\pgfqpoint{3.075000in}{0.615019in}}%
\pgfpathmoveto{\pgfqpoint{3.070459in}{0.617968in}}%
\pgfpathlineto{\pgfqpoint{3.070459in}{0.617968in}}%
\pgfpathlineto{\pgfqpoint{3.070459in}{0.620918in}}%
\pgfpathlineto{\pgfqpoint{3.075000in}{0.620918in}}%
\pgfpathlineto{\pgfqpoint{3.075000in}{0.617968in}}%
\pgfpathmoveto{\pgfqpoint{3.070459in}{0.620918in}}%
\pgfpathlineto{\pgfqpoint{3.070459in}{0.620918in}}%
\pgfpathlineto{\pgfqpoint{3.070459in}{0.623867in}}%
\pgfpathlineto{\pgfqpoint{3.075000in}{0.623867in}}%
\pgfpathlineto{\pgfqpoint{3.075000in}{0.620918in}}%
\pgfpathmoveto{\pgfqpoint{3.070459in}{0.623867in}}%
\pgfpathlineto{\pgfqpoint{3.070459in}{0.623867in}}%
\pgfpathlineto{\pgfqpoint{3.070459in}{0.626816in}}%
\pgfpathlineto{\pgfqpoint{3.075000in}{0.626816in}}%
\pgfpathlineto{\pgfqpoint{3.075000in}{0.623867in}}%
\pgfpathmoveto{\pgfqpoint{3.070459in}{0.626816in}}%
\pgfpathlineto{\pgfqpoint{3.070459in}{0.626816in}}%
\pgfpathlineto{\pgfqpoint{3.070459in}{0.629765in}}%
\pgfpathlineto{\pgfqpoint{3.075000in}{0.629765in}}%
\pgfpathlineto{\pgfqpoint{3.075000in}{0.626816in}}%
\pgfpathmoveto{\pgfqpoint{3.070459in}{0.629765in}}%
\pgfpathlineto{\pgfqpoint{3.070459in}{0.629765in}}%
\pgfpathlineto{\pgfqpoint{3.070459in}{0.632715in}}%
\pgfpathlineto{\pgfqpoint{3.075000in}{0.632715in}}%
\pgfpathlineto{\pgfqpoint{3.075000in}{0.629765in}}%
\pgfpathmoveto{\pgfqpoint{3.070459in}{0.632715in}}%
\pgfpathlineto{\pgfqpoint{3.070459in}{0.632715in}}%
\pgfpathlineto{\pgfqpoint{3.070459in}{0.635664in}}%
\pgfpathlineto{\pgfqpoint{3.075000in}{0.635664in}}%
\pgfpathlineto{\pgfqpoint{3.075000in}{0.632715in}}%
\pgfpathmoveto{\pgfqpoint{3.070459in}{0.635664in}}%
\pgfpathlineto{\pgfqpoint{3.070459in}{0.635664in}}%
\pgfpathlineto{\pgfqpoint{3.070459in}{0.638613in}}%
\pgfpathlineto{\pgfqpoint{3.075000in}{0.638613in}}%
\pgfpathlineto{\pgfqpoint{3.075000in}{0.635664in}}%
\pgfpathmoveto{\pgfqpoint{3.070459in}{0.638613in}}%
\pgfpathlineto{\pgfqpoint{3.070459in}{0.638613in}}%
\pgfpathlineto{\pgfqpoint{3.070459in}{0.641562in}}%
\pgfpathlineto{\pgfqpoint{3.075000in}{0.641562in}}%
\pgfpathlineto{\pgfqpoint{3.075000in}{0.638613in}}%
\pgfpathmoveto{\pgfqpoint{3.070459in}{0.641562in}}%
\pgfpathlineto{\pgfqpoint{3.070459in}{0.641562in}}%
\pgfpathlineto{\pgfqpoint{3.070459in}{0.644511in}}%
\pgfpathlineto{\pgfqpoint{3.075000in}{0.644511in}}%
\pgfpathlineto{\pgfqpoint{3.075000in}{0.641562in}}%
\pgfpathmoveto{\pgfqpoint{3.070459in}{0.644511in}}%
\pgfpathlineto{\pgfqpoint{3.070459in}{0.644511in}}%
\pgfpathlineto{\pgfqpoint{3.070459in}{0.647461in}}%
\pgfpathlineto{\pgfqpoint{3.075000in}{0.647461in}}%
\pgfpathlineto{\pgfqpoint{3.075000in}{0.644511in}}%
\pgfpathmoveto{\pgfqpoint{3.070459in}{0.647461in}}%
\pgfpathlineto{\pgfqpoint{3.070459in}{0.647461in}}%
\pgfpathlineto{\pgfqpoint{3.070459in}{0.650410in}}%
\pgfpathlineto{\pgfqpoint{3.075000in}{0.650410in}}%
\pgfpathlineto{\pgfqpoint{3.075000in}{0.647461in}}%
\pgfpathmoveto{\pgfqpoint{3.070459in}{0.650410in}}%
\pgfpathlineto{\pgfqpoint{3.070459in}{0.650410in}}%
\pgfpathlineto{\pgfqpoint{3.070459in}{0.653359in}}%
\pgfpathlineto{\pgfqpoint{3.075000in}{0.653359in}}%
\pgfpathlineto{\pgfqpoint{3.075000in}{0.650410in}}%
\pgfpathmoveto{\pgfqpoint{3.070459in}{0.653359in}}%
\pgfpathlineto{\pgfqpoint{3.070459in}{0.653359in}}%
\pgfpathlineto{\pgfqpoint{3.070459in}{0.656308in}}%
\pgfpathlineto{\pgfqpoint{3.075000in}{0.656308in}}%
\pgfpathlineto{\pgfqpoint{3.075000in}{0.653359in}}%
\pgfpathmoveto{\pgfqpoint{3.070459in}{0.656308in}}%
\pgfpathlineto{\pgfqpoint{3.070459in}{0.656308in}}%
\pgfpathlineto{\pgfqpoint{3.070459in}{0.659258in}}%
\pgfpathlineto{\pgfqpoint{3.075000in}{0.659258in}}%
\pgfpathlineto{\pgfqpoint{3.075000in}{0.656308in}}%
\pgfpathmoveto{\pgfqpoint{3.070459in}{0.659258in}}%
\pgfpathlineto{\pgfqpoint{3.070459in}{0.659258in}}%
\pgfpathlineto{\pgfqpoint{3.070459in}{0.662207in}}%
\pgfpathlineto{\pgfqpoint{3.075000in}{0.662207in}}%
\pgfpathlineto{\pgfqpoint{3.075000in}{0.659258in}}%
\pgfpathmoveto{\pgfqpoint{3.070459in}{0.662207in}}%
\pgfpathlineto{\pgfqpoint{3.070459in}{0.662207in}}%
\pgfpathlineto{\pgfqpoint{3.070459in}{0.665156in}}%
\pgfpathlineto{\pgfqpoint{3.075000in}{0.665156in}}%
\pgfpathlineto{\pgfqpoint{3.075000in}{0.662207in}}%
\pgfpathmoveto{\pgfqpoint{3.070459in}{0.665156in}}%
\pgfpathlineto{\pgfqpoint{3.070459in}{0.665156in}}%
\pgfpathlineto{\pgfqpoint{3.070459in}{0.668105in}}%
\pgfpathlineto{\pgfqpoint{3.075000in}{0.668105in}}%
\pgfpathlineto{\pgfqpoint{3.075000in}{0.665156in}}%
\pgfpathmoveto{\pgfqpoint{3.070459in}{0.668105in}}%
\pgfpathlineto{\pgfqpoint{3.070459in}{0.668105in}}%
\pgfpathlineto{\pgfqpoint{3.070459in}{0.671054in}}%
\pgfpathlineto{\pgfqpoint{3.075000in}{0.671054in}}%
\pgfpathlineto{\pgfqpoint{3.075000in}{0.668105in}}%
\pgfpathmoveto{\pgfqpoint{3.070459in}{0.671054in}}%
\pgfpathlineto{\pgfqpoint{3.070459in}{0.671054in}}%
\pgfpathlineto{\pgfqpoint{3.070459in}{0.674004in}}%
\pgfpathlineto{\pgfqpoint{3.075000in}{0.674004in}}%
\pgfpathlineto{\pgfqpoint{3.075000in}{0.671054in}}%
\pgfpathmoveto{\pgfqpoint{3.070459in}{0.674004in}}%
\pgfpathlineto{\pgfqpoint{3.070459in}{0.674004in}}%
\pgfpathlineto{\pgfqpoint{3.070459in}{0.676953in}}%
\pgfpathlineto{\pgfqpoint{3.075000in}{0.676953in}}%
\pgfpathlineto{\pgfqpoint{3.075000in}{0.674004in}}%
\pgfpathmoveto{\pgfqpoint{3.070459in}{0.676953in}}%
\pgfpathlineto{\pgfqpoint{3.070459in}{0.676953in}}%
\pgfpathlineto{\pgfqpoint{3.070459in}{0.679902in}}%
\pgfpathlineto{\pgfqpoint{3.075000in}{0.679902in}}%
\pgfpathlineto{\pgfqpoint{3.075000in}{0.676953in}}%
\pgfpathmoveto{\pgfqpoint{3.070459in}{0.679902in}}%
\pgfpathlineto{\pgfqpoint{3.070459in}{0.679902in}}%
\pgfpathlineto{\pgfqpoint{3.070459in}{0.682851in}}%
\pgfpathlineto{\pgfqpoint{3.075000in}{0.682851in}}%
\pgfpathlineto{\pgfqpoint{3.075000in}{0.679902in}}%
\pgfpathmoveto{\pgfqpoint{3.070459in}{0.682851in}}%
\pgfpathlineto{\pgfqpoint{3.070459in}{0.682851in}}%
\pgfpathlineto{\pgfqpoint{3.070459in}{0.685801in}}%
\pgfpathlineto{\pgfqpoint{3.075000in}{0.685801in}}%
\pgfpathlineto{\pgfqpoint{3.075000in}{0.682851in}}%
\pgfpathmoveto{\pgfqpoint{3.070459in}{0.685801in}}%
\pgfpathlineto{\pgfqpoint{3.070459in}{0.685801in}}%
\pgfpathlineto{\pgfqpoint{3.070459in}{0.688750in}}%
\pgfpathlineto{\pgfqpoint{3.075000in}{0.688750in}}%
\pgfpathlineto{\pgfqpoint{3.075000in}{0.685801in}}%
\pgfpathmoveto{\pgfqpoint{3.070459in}{0.688750in}}%
\pgfpathlineto{\pgfqpoint{3.070459in}{0.688750in}}%
\pgfpathlineto{\pgfqpoint{3.070459in}{0.691699in}}%
\pgfpathlineto{\pgfqpoint{3.075000in}{0.691699in}}%
\pgfpathlineto{\pgfqpoint{3.075000in}{0.688750in}}%
\pgfpathmoveto{\pgfqpoint{3.070459in}{0.691699in}}%
\pgfpathlineto{\pgfqpoint{3.070459in}{0.691699in}}%
\pgfpathlineto{\pgfqpoint{3.070459in}{0.694648in}}%
\pgfpathlineto{\pgfqpoint{3.075000in}{0.694648in}}%
\pgfpathlineto{\pgfqpoint{3.075000in}{0.691699in}}%
\pgfpathmoveto{\pgfqpoint{3.070459in}{0.694648in}}%
\pgfpathlineto{\pgfqpoint{3.070459in}{0.694648in}}%
\pgfpathlineto{\pgfqpoint{3.070459in}{0.697597in}}%
\pgfpathlineto{\pgfqpoint{3.075000in}{0.697597in}}%
\pgfpathlineto{\pgfqpoint{3.075000in}{0.694648in}}%
\pgfpathmoveto{\pgfqpoint{3.070459in}{0.697597in}}%
\pgfpathlineto{\pgfqpoint{3.070459in}{0.697597in}}%
\pgfpathlineto{\pgfqpoint{3.070459in}{0.700546in}}%
\pgfpathlineto{\pgfqpoint{3.075000in}{0.700546in}}%
\pgfpathlineto{\pgfqpoint{3.075000in}{0.697597in}}%
\pgfpathmoveto{\pgfqpoint{3.070459in}{0.700546in}}%
\pgfpathlineto{\pgfqpoint{3.070459in}{0.700546in}}%
\pgfpathlineto{\pgfqpoint{3.070459in}{0.703496in}}%
\pgfpathlineto{\pgfqpoint{3.075000in}{0.703496in}}%
\pgfpathlineto{\pgfqpoint{3.075000in}{0.700546in}}%
\pgfpathmoveto{\pgfqpoint{3.070459in}{0.703496in}}%
\pgfpathlineto{\pgfqpoint{3.070459in}{0.703496in}}%
\pgfpathlineto{\pgfqpoint{3.070459in}{0.706445in}}%
\pgfpathlineto{\pgfqpoint{3.075000in}{0.706445in}}%
\pgfpathlineto{\pgfqpoint{3.075000in}{0.703496in}}%
\pgfpathmoveto{\pgfqpoint{3.070459in}{0.706445in}}%
\pgfpathlineto{\pgfqpoint{3.070459in}{0.706445in}}%
\pgfpathlineto{\pgfqpoint{3.070459in}{0.709394in}}%
\pgfpathlineto{\pgfqpoint{3.075000in}{0.709394in}}%
\pgfpathlineto{\pgfqpoint{3.075000in}{0.706445in}}%
\pgfpathmoveto{\pgfqpoint{3.070459in}{0.709394in}}%
\pgfpathlineto{\pgfqpoint{3.070459in}{0.709394in}}%
\pgfpathlineto{\pgfqpoint{3.070459in}{0.712343in}}%
\pgfpathlineto{\pgfqpoint{3.075000in}{0.712343in}}%
\pgfpathlineto{\pgfqpoint{3.075000in}{0.709394in}}%
\pgfpathmoveto{\pgfqpoint{3.070459in}{0.712343in}}%
\pgfpathlineto{\pgfqpoint{3.070459in}{0.712343in}}%
\pgfpathlineto{\pgfqpoint{3.070459in}{0.715292in}}%
\pgfpathlineto{\pgfqpoint{3.075000in}{0.715292in}}%
\pgfpathlineto{\pgfqpoint{3.075000in}{0.712343in}}%
\pgfpathmoveto{\pgfqpoint{3.070459in}{0.715292in}}%
\pgfpathlineto{\pgfqpoint{3.070459in}{0.715292in}}%
\pgfpathlineto{\pgfqpoint{3.070459in}{0.718242in}}%
\pgfpathlineto{\pgfqpoint{3.075000in}{0.718242in}}%
\pgfpathlineto{\pgfqpoint{3.075000in}{0.715292in}}%
\pgfpathmoveto{\pgfqpoint{3.070459in}{0.718242in}}%
\pgfpathlineto{\pgfqpoint{3.070459in}{0.718242in}}%
\pgfpathlineto{\pgfqpoint{3.070459in}{0.721191in}}%
\pgfpathlineto{\pgfqpoint{3.075000in}{0.721191in}}%
\pgfpathlineto{\pgfqpoint{3.075000in}{0.718242in}}%
\pgfpathmoveto{\pgfqpoint{3.070459in}{0.721191in}}%
\pgfpathlineto{\pgfqpoint{3.070459in}{0.721191in}}%
\pgfpathlineto{\pgfqpoint{3.070459in}{0.724140in}}%
\pgfpathlineto{\pgfqpoint{3.075000in}{0.724140in}}%
\pgfpathlineto{\pgfqpoint{3.075000in}{0.721191in}}%
\pgfpathmoveto{\pgfqpoint{3.070459in}{0.724140in}}%
\pgfpathlineto{\pgfqpoint{3.070459in}{0.724140in}}%
\pgfpathlineto{\pgfqpoint{3.070459in}{0.727089in}}%
\pgfpathlineto{\pgfqpoint{3.075000in}{0.727089in}}%
\pgfpathlineto{\pgfqpoint{3.075000in}{0.724140in}}%
\pgfpathmoveto{\pgfqpoint{3.070459in}{0.727089in}}%
\pgfpathlineto{\pgfqpoint{3.070459in}{0.727089in}}%
\pgfpathlineto{\pgfqpoint{3.070459in}{0.730038in}}%
\pgfpathlineto{\pgfqpoint{3.075000in}{0.730038in}}%
\pgfpathlineto{\pgfqpoint{3.075000in}{0.727089in}}%
\pgfpathmoveto{\pgfqpoint{3.070459in}{0.730038in}}%
\pgfpathlineto{\pgfqpoint{3.070459in}{0.730038in}}%
\pgfpathlineto{\pgfqpoint{3.070459in}{0.732987in}}%
\pgfpathlineto{\pgfqpoint{3.075000in}{0.732987in}}%
\pgfpathlineto{\pgfqpoint{3.075000in}{0.730038in}}%
\pgfpathmoveto{\pgfqpoint{3.070459in}{0.732987in}}%
\pgfpathlineto{\pgfqpoint{3.070459in}{0.732987in}}%
\pgfpathlineto{\pgfqpoint{3.070459in}{0.735937in}}%
\pgfpathlineto{\pgfqpoint{3.075000in}{0.735937in}}%
\pgfpathlineto{\pgfqpoint{3.075000in}{0.732987in}}%
\pgfpathmoveto{\pgfqpoint{3.070459in}{0.735937in}}%
\pgfpathlineto{\pgfqpoint{3.070459in}{0.735937in}}%
\pgfpathlineto{\pgfqpoint{3.070459in}{0.738886in}}%
\pgfpathlineto{\pgfqpoint{3.075000in}{0.738886in}}%
\pgfpathlineto{\pgfqpoint{3.075000in}{0.735937in}}%
\pgfpathmoveto{\pgfqpoint{3.070459in}{0.738886in}}%
\pgfpathlineto{\pgfqpoint{3.070459in}{0.738886in}}%
\pgfpathlineto{\pgfqpoint{3.070459in}{0.741835in}}%
\pgfpathlineto{\pgfqpoint{3.075000in}{0.741835in}}%
\pgfpathlineto{\pgfqpoint{3.075000in}{0.738886in}}%
\pgfpathmoveto{\pgfqpoint{3.070459in}{0.741835in}}%
\pgfpathlineto{\pgfqpoint{3.070459in}{0.741835in}}%
\pgfpathlineto{\pgfqpoint{3.070459in}{0.744784in}}%
\pgfpathlineto{\pgfqpoint{3.075000in}{0.744784in}}%
\pgfpathlineto{\pgfqpoint{3.075000in}{0.741835in}}%
\pgfpathmoveto{\pgfqpoint{3.070459in}{0.744784in}}%
\pgfpathlineto{\pgfqpoint{3.070459in}{0.744784in}}%
\pgfpathlineto{\pgfqpoint{3.070459in}{0.747733in}}%
\pgfpathlineto{\pgfqpoint{3.075000in}{0.747733in}}%
\pgfpathlineto{\pgfqpoint{3.075000in}{0.744784in}}%
\pgfpathmoveto{\pgfqpoint{3.070459in}{0.747733in}}%
\pgfpathlineto{\pgfqpoint{3.070459in}{0.747733in}}%
\pgfpathlineto{\pgfqpoint{3.070459in}{0.750683in}}%
\pgfpathlineto{\pgfqpoint{3.075000in}{0.750683in}}%
\pgfpathlineto{\pgfqpoint{3.075000in}{0.747733in}}%
\pgfpathmoveto{\pgfqpoint{3.070459in}{0.750683in}}%
\pgfpathlineto{\pgfqpoint{3.070459in}{0.750683in}}%
\pgfpathlineto{\pgfqpoint{3.070459in}{0.753632in}}%
\pgfpathlineto{\pgfqpoint{3.075000in}{0.753632in}}%
\pgfpathlineto{\pgfqpoint{3.075000in}{0.750683in}}%
\pgfpathmoveto{\pgfqpoint{3.070459in}{0.753632in}}%
\pgfpathlineto{\pgfqpoint{3.070459in}{0.753632in}}%
\pgfpathlineto{\pgfqpoint{3.070459in}{0.756581in}}%
\pgfpathlineto{\pgfqpoint{3.075000in}{0.756581in}}%
\pgfpathlineto{\pgfqpoint{3.075000in}{0.753632in}}%
\pgfpathmoveto{\pgfqpoint{3.070459in}{0.756581in}}%
\pgfpathlineto{\pgfqpoint{3.070459in}{0.756581in}}%
\pgfpathlineto{\pgfqpoint{3.070459in}{0.759530in}}%
\pgfpathlineto{\pgfqpoint{3.075000in}{0.759530in}}%
\pgfpathlineto{\pgfqpoint{3.075000in}{0.756581in}}%
\pgfpathmoveto{\pgfqpoint{3.070459in}{0.759530in}}%
\pgfpathlineto{\pgfqpoint{3.070459in}{0.759530in}}%
\pgfpathlineto{\pgfqpoint{3.070459in}{0.762479in}}%
\pgfpathlineto{\pgfqpoint{3.075000in}{0.762479in}}%
\pgfpathlineto{\pgfqpoint{3.075000in}{0.759530in}}%
\pgfpathmoveto{\pgfqpoint{3.070459in}{0.762479in}}%
\pgfpathlineto{\pgfqpoint{3.070459in}{0.762479in}}%
\pgfpathlineto{\pgfqpoint{3.070459in}{0.765428in}}%
\pgfpathlineto{\pgfqpoint{3.075000in}{0.765428in}}%
\pgfpathlineto{\pgfqpoint{3.075000in}{0.762479in}}%
\pgfpathmoveto{\pgfqpoint{3.070459in}{0.765428in}}%
\pgfpathlineto{\pgfqpoint{3.070459in}{0.765428in}}%
\pgfpathlineto{\pgfqpoint{3.070459in}{0.768378in}}%
\pgfpathlineto{\pgfqpoint{3.075000in}{0.768378in}}%
\pgfpathlineto{\pgfqpoint{3.075000in}{0.765428in}}%
\pgfpathmoveto{\pgfqpoint{3.070459in}{0.768378in}}%
\pgfpathlineto{\pgfqpoint{3.070459in}{0.768378in}}%
\pgfpathlineto{\pgfqpoint{3.070459in}{0.771327in}}%
\pgfpathlineto{\pgfqpoint{3.075000in}{0.771327in}}%
\pgfpathlineto{\pgfqpoint{3.075000in}{0.768378in}}%
\pgfpathmoveto{\pgfqpoint{3.070459in}{0.771327in}}%
\pgfpathlineto{\pgfqpoint{3.070459in}{0.771327in}}%
\pgfpathlineto{\pgfqpoint{3.070459in}{0.774276in}}%
\pgfpathlineto{\pgfqpoint{3.075000in}{0.774276in}}%
\pgfpathlineto{\pgfqpoint{3.075000in}{0.771327in}}%
\pgfpathmoveto{\pgfqpoint{3.070459in}{0.774276in}}%
\pgfpathlineto{\pgfqpoint{3.070459in}{0.774276in}}%
\pgfpathlineto{\pgfqpoint{3.070459in}{0.777225in}}%
\pgfpathlineto{\pgfqpoint{3.075000in}{0.777225in}}%
\pgfpathlineto{\pgfqpoint{3.075000in}{0.774276in}}%
\pgfpathmoveto{\pgfqpoint{3.070459in}{0.777225in}}%
\pgfpathlineto{\pgfqpoint{3.070459in}{0.777225in}}%
\pgfpathlineto{\pgfqpoint{3.070459in}{0.780174in}}%
\pgfpathlineto{\pgfqpoint{3.075000in}{0.780174in}}%
\pgfpathlineto{\pgfqpoint{3.075000in}{0.777225in}}%
\pgfpathmoveto{\pgfqpoint{3.070459in}{0.780174in}}%
\pgfpathlineto{\pgfqpoint{3.070459in}{0.780174in}}%
\pgfpathlineto{\pgfqpoint{3.070459in}{0.783124in}}%
\pgfpathlineto{\pgfqpoint{3.075000in}{0.783124in}}%
\pgfpathlineto{\pgfqpoint{3.075000in}{0.780174in}}%
\pgfpathmoveto{\pgfqpoint{3.070459in}{0.783124in}}%
\pgfpathlineto{\pgfqpoint{3.070459in}{0.783124in}}%
\pgfpathlineto{\pgfqpoint{3.070459in}{0.786073in}}%
\pgfpathlineto{\pgfqpoint{3.075000in}{0.786073in}}%
\pgfpathlineto{\pgfqpoint{3.075000in}{0.783124in}}%
\pgfpathmoveto{\pgfqpoint{3.070459in}{0.786073in}}%
\pgfpathlineto{\pgfqpoint{3.070459in}{0.786073in}}%
\pgfpathlineto{\pgfqpoint{3.070459in}{0.789022in}}%
\pgfpathlineto{\pgfqpoint{3.075000in}{0.789022in}}%
\pgfpathlineto{\pgfqpoint{3.075000in}{0.786073in}}%
\pgfpathmoveto{\pgfqpoint{3.070459in}{0.789022in}}%
\pgfpathlineto{\pgfqpoint{3.070459in}{0.789022in}}%
\pgfpathlineto{\pgfqpoint{3.070459in}{0.791972in}}%
\pgfpathlineto{\pgfqpoint{3.075000in}{0.791972in}}%
\pgfpathlineto{\pgfqpoint{3.075000in}{0.789022in}}%
\pgfpathmoveto{\pgfqpoint{3.070459in}{0.791972in}}%
\pgfpathlineto{\pgfqpoint{3.070459in}{0.791972in}}%
\pgfpathlineto{\pgfqpoint{3.070459in}{0.794921in}}%
\pgfpathlineto{\pgfqpoint{3.075000in}{0.794921in}}%
\pgfpathlineto{\pgfqpoint{3.075000in}{0.791972in}}%
\pgfpathmoveto{\pgfqpoint{3.070459in}{0.794921in}}%
\pgfpathlineto{\pgfqpoint{3.070459in}{0.794921in}}%
\pgfpathlineto{\pgfqpoint{3.070459in}{0.797870in}}%
\pgfpathlineto{\pgfqpoint{3.075000in}{0.797870in}}%
\pgfpathlineto{\pgfqpoint{3.075000in}{0.794921in}}%
\pgfpathmoveto{\pgfqpoint{3.070459in}{0.797870in}}%
\pgfpathlineto{\pgfqpoint{3.070459in}{0.797870in}}%
\pgfpathlineto{\pgfqpoint{3.070459in}{0.800820in}}%
\pgfpathlineto{\pgfqpoint{3.075000in}{0.800820in}}%
\pgfpathlineto{\pgfqpoint{3.075000in}{0.797870in}}%
\pgfpathmoveto{\pgfqpoint{3.070459in}{0.800820in}}%
\pgfpathlineto{\pgfqpoint{3.070459in}{0.800820in}}%
\pgfpathlineto{\pgfqpoint{3.070459in}{0.803769in}}%
\pgfpathlineto{\pgfqpoint{3.075000in}{0.803769in}}%
\pgfpathlineto{\pgfqpoint{3.075000in}{0.800820in}}%
\pgfpathmoveto{\pgfqpoint{3.070459in}{0.803769in}}%
\pgfpathlineto{\pgfqpoint{3.070459in}{0.803769in}}%
\pgfpathlineto{\pgfqpoint{3.070459in}{0.806718in}}%
\pgfpathlineto{\pgfqpoint{3.075000in}{0.806718in}}%
\pgfpathlineto{\pgfqpoint{3.075000in}{0.803769in}}%
\pgfpathmoveto{\pgfqpoint{3.070459in}{0.806718in}}%
\pgfpathlineto{\pgfqpoint{3.070459in}{0.806718in}}%
\pgfpathlineto{\pgfqpoint{3.070459in}{0.809668in}}%
\pgfpathlineto{\pgfqpoint{3.075000in}{0.809668in}}%
\pgfpathlineto{\pgfqpoint{3.075000in}{0.806718in}}%
\pgfpathmoveto{\pgfqpoint{3.070459in}{0.809668in}}%
\pgfpathlineto{\pgfqpoint{3.070459in}{0.809668in}}%
\pgfpathlineto{\pgfqpoint{3.070459in}{0.812617in}}%
\pgfpathlineto{\pgfqpoint{3.075000in}{0.812617in}}%
\pgfpathlineto{\pgfqpoint{3.075000in}{0.809668in}}%
\pgfpathmoveto{\pgfqpoint{3.070459in}{0.812617in}}%
\pgfpathlineto{\pgfqpoint{3.070459in}{0.812617in}}%
\pgfpathlineto{\pgfqpoint{3.070459in}{0.815566in}}%
\pgfpathlineto{\pgfqpoint{3.075000in}{0.815566in}}%
\pgfpathlineto{\pgfqpoint{3.075000in}{0.812617in}}%
\pgfpathmoveto{\pgfqpoint{3.070459in}{0.815566in}}%
\pgfpathlineto{\pgfqpoint{3.070459in}{0.815566in}}%
\pgfpathlineto{\pgfqpoint{3.070459in}{0.818516in}}%
\pgfpathlineto{\pgfqpoint{3.075000in}{0.818516in}}%
\pgfpathlineto{\pgfqpoint{3.075000in}{0.815566in}}%
\pgfpathmoveto{\pgfqpoint{3.070459in}{0.818516in}}%
\pgfpathlineto{\pgfqpoint{3.070459in}{0.818516in}}%
\pgfpathlineto{\pgfqpoint{3.070459in}{0.821465in}}%
\pgfpathlineto{\pgfqpoint{3.075000in}{0.821465in}}%
\pgfpathlineto{\pgfqpoint{3.075000in}{0.818516in}}%
\pgfpathmoveto{\pgfqpoint{3.070459in}{0.821465in}}%
\pgfpathlineto{\pgfqpoint{3.070459in}{0.821465in}}%
\pgfpathlineto{\pgfqpoint{3.070459in}{0.824414in}}%
\pgfpathlineto{\pgfqpoint{3.075000in}{0.824414in}}%
\pgfpathlineto{\pgfqpoint{3.075000in}{0.821465in}}%
\pgfpathmoveto{\pgfqpoint{3.070459in}{0.824414in}}%
\pgfpathlineto{\pgfqpoint{3.070459in}{0.824414in}}%
\pgfpathlineto{\pgfqpoint{3.070459in}{0.827364in}}%
\pgfpathlineto{\pgfqpoint{3.075000in}{0.827364in}}%
\pgfpathlineto{\pgfqpoint{3.075000in}{0.824414in}}%
\pgfpathmoveto{\pgfqpoint{3.070459in}{0.827364in}}%
\pgfpathlineto{\pgfqpoint{3.070459in}{0.827364in}}%
\pgfpathlineto{\pgfqpoint{3.070459in}{0.830313in}}%
\pgfpathlineto{\pgfqpoint{3.075000in}{0.830313in}}%
\pgfpathlineto{\pgfqpoint{3.075000in}{0.827364in}}%
\pgfpathmoveto{\pgfqpoint{3.070459in}{0.830313in}}%
\pgfpathlineto{\pgfqpoint{3.070459in}{0.830313in}}%
\pgfpathlineto{\pgfqpoint{3.070459in}{0.833262in}}%
\pgfpathlineto{\pgfqpoint{3.075000in}{0.833262in}}%
\pgfpathlineto{\pgfqpoint{3.075000in}{0.830313in}}%
\pgfpathmoveto{\pgfqpoint{3.070459in}{0.833262in}}%
\pgfpathlineto{\pgfqpoint{3.070459in}{0.833262in}}%
\pgfpathlineto{\pgfqpoint{3.070459in}{0.836212in}}%
\pgfpathlineto{\pgfqpoint{3.075000in}{0.836212in}}%
\pgfpathlineto{\pgfqpoint{3.075000in}{0.833262in}}%
\pgfpathmoveto{\pgfqpoint{3.070459in}{0.836212in}}%
\pgfpathlineto{\pgfqpoint{3.070459in}{0.836212in}}%
\pgfpathlineto{\pgfqpoint{3.070459in}{0.839161in}}%
\pgfpathlineto{\pgfqpoint{3.075000in}{0.839161in}}%
\pgfpathlineto{\pgfqpoint{3.075000in}{0.836212in}}%
\pgfpathmoveto{\pgfqpoint{3.070459in}{0.839161in}}%
\pgfpathlineto{\pgfqpoint{3.070459in}{0.839161in}}%
\pgfpathlineto{\pgfqpoint{3.070459in}{0.842110in}}%
\pgfpathlineto{\pgfqpoint{3.075000in}{0.842110in}}%
\pgfpathlineto{\pgfqpoint{3.075000in}{0.839161in}}%
\pgfpathmoveto{\pgfqpoint{3.070459in}{0.842110in}}%
\pgfpathlineto{\pgfqpoint{3.070459in}{0.842110in}}%
\pgfpathlineto{\pgfqpoint{3.070459in}{0.845060in}}%
\pgfpathlineto{\pgfqpoint{3.075000in}{0.845060in}}%
\pgfpathlineto{\pgfqpoint{3.075000in}{0.842110in}}%
\pgfpathmoveto{\pgfqpoint{3.070459in}{0.845060in}}%
\pgfpathlineto{\pgfqpoint{3.070459in}{0.845060in}}%
\pgfpathlineto{\pgfqpoint{3.070459in}{0.848009in}}%
\pgfpathlineto{\pgfqpoint{3.075000in}{0.848009in}}%
\pgfpathlineto{\pgfqpoint{3.075000in}{0.845060in}}%
\pgfpathmoveto{\pgfqpoint{3.070459in}{0.848009in}}%
\pgfpathlineto{\pgfqpoint{3.070459in}{0.848009in}}%
\pgfpathlineto{\pgfqpoint{3.070459in}{0.850958in}}%
\pgfpathlineto{\pgfqpoint{3.075000in}{0.850958in}}%
\pgfpathlineto{\pgfqpoint{3.075000in}{0.848009in}}%
\pgfpathmoveto{\pgfqpoint{3.070459in}{0.850958in}}%
\pgfpathlineto{\pgfqpoint{3.070459in}{0.850958in}}%
\pgfpathlineto{\pgfqpoint{3.070459in}{0.853908in}}%
\pgfpathlineto{\pgfqpoint{3.075000in}{0.853908in}}%
\pgfpathlineto{\pgfqpoint{3.075000in}{0.850958in}}%
\pgfpathmoveto{\pgfqpoint{3.070459in}{0.853908in}}%
\pgfpathlineto{\pgfqpoint{3.070459in}{0.853908in}}%
\pgfpathlineto{\pgfqpoint{3.070459in}{0.856857in}}%
\pgfpathlineto{\pgfqpoint{3.075000in}{0.856857in}}%
\pgfpathlineto{\pgfqpoint{3.075000in}{0.853908in}}%
\pgfpathmoveto{\pgfqpoint{3.070459in}{0.856857in}}%
\pgfpathlineto{\pgfqpoint{3.070459in}{0.856857in}}%
\pgfpathlineto{\pgfqpoint{3.070459in}{0.859806in}}%
\pgfpathlineto{\pgfqpoint{3.075000in}{0.859806in}}%
\pgfpathlineto{\pgfqpoint{3.075000in}{0.856857in}}%
\pgfpathmoveto{\pgfqpoint{3.070459in}{0.859806in}}%
\pgfpathlineto{\pgfqpoint{3.070459in}{0.859806in}}%
\pgfpathlineto{\pgfqpoint{3.070459in}{0.862756in}}%
\pgfpathlineto{\pgfqpoint{3.075000in}{0.862756in}}%
\pgfpathlineto{\pgfqpoint{3.075000in}{0.859806in}}%
\pgfpathmoveto{\pgfqpoint{3.070459in}{0.862756in}}%
\pgfpathlineto{\pgfqpoint{3.070459in}{0.862756in}}%
\pgfpathlineto{\pgfqpoint{3.070459in}{0.865705in}}%
\pgfpathlineto{\pgfqpoint{3.075000in}{0.865705in}}%
\pgfpathlineto{\pgfqpoint{3.075000in}{0.862756in}}%
\pgfpathmoveto{\pgfqpoint{3.070459in}{0.865705in}}%
\pgfpathlineto{\pgfqpoint{3.070459in}{0.865705in}}%
\pgfpathlineto{\pgfqpoint{3.070459in}{0.868654in}}%
\pgfpathlineto{\pgfqpoint{3.075000in}{0.868654in}}%
\pgfpathlineto{\pgfqpoint{3.075000in}{0.865705in}}%
\pgfpathmoveto{\pgfqpoint{3.070459in}{0.868654in}}%
\pgfpathlineto{\pgfqpoint{3.070459in}{0.868654in}}%
\pgfpathlineto{\pgfqpoint{3.070459in}{0.871604in}}%
\pgfpathlineto{\pgfqpoint{3.075000in}{0.871604in}}%
\pgfpathlineto{\pgfqpoint{3.075000in}{0.868654in}}%
\pgfpathmoveto{\pgfqpoint{3.070459in}{0.871604in}}%
\pgfpathlineto{\pgfqpoint{3.070459in}{0.871604in}}%
\pgfpathlineto{\pgfqpoint{3.070459in}{0.874553in}}%
\pgfpathlineto{\pgfqpoint{3.075000in}{0.874553in}}%
\pgfpathlineto{\pgfqpoint{3.075000in}{0.871604in}}%
\pgfpathmoveto{\pgfqpoint{3.070459in}{0.874553in}}%
\pgfpathlineto{\pgfqpoint{3.070459in}{0.874553in}}%
\pgfpathlineto{\pgfqpoint{3.070459in}{0.877502in}}%
\pgfpathlineto{\pgfqpoint{3.075000in}{0.877502in}}%
\pgfpathlineto{\pgfqpoint{3.075000in}{0.874553in}}%
\pgfpathmoveto{\pgfqpoint{3.070459in}{0.877502in}}%
\pgfpathlineto{\pgfqpoint{3.070459in}{0.877502in}}%
\pgfpathlineto{\pgfqpoint{3.070459in}{0.880451in}}%
\pgfpathlineto{\pgfqpoint{3.075000in}{0.880451in}}%
\pgfpathlineto{\pgfqpoint{3.075000in}{0.877502in}}%
\pgfpathmoveto{\pgfqpoint{3.070459in}{0.880451in}}%
\pgfpathlineto{\pgfqpoint{3.070459in}{0.880451in}}%
\pgfpathlineto{\pgfqpoint{3.070459in}{0.883401in}}%
\pgfpathlineto{\pgfqpoint{3.075000in}{0.883401in}}%
\pgfpathlineto{\pgfqpoint{3.075000in}{0.880451in}}%
\pgfpathmoveto{\pgfqpoint{3.070459in}{0.883401in}}%
\pgfpathlineto{\pgfqpoint{3.070459in}{0.883401in}}%
\pgfpathlineto{\pgfqpoint{3.070459in}{0.886350in}}%
\pgfpathlineto{\pgfqpoint{3.075000in}{0.886350in}}%
\pgfpathlineto{\pgfqpoint{3.075000in}{0.883401in}}%
\pgfpathmoveto{\pgfqpoint{3.070459in}{0.886350in}}%
\pgfpathlineto{\pgfqpoint{3.070459in}{0.886350in}}%
\pgfpathlineto{\pgfqpoint{3.070459in}{0.889299in}}%
\pgfpathlineto{\pgfqpoint{3.075000in}{0.889299in}}%
\pgfpathlineto{\pgfqpoint{3.075000in}{0.886350in}}%
\pgfpathmoveto{\pgfqpoint{3.070459in}{0.889299in}}%
\pgfpathlineto{\pgfqpoint{3.070459in}{0.889299in}}%
\pgfpathlineto{\pgfqpoint{3.070459in}{0.892248in}}%
\pgfpathlineto{\pgfqpoint{3.075000in}{0.892248in}}%
\pgfpathlineto{\pgfqpoint{3.075000in}{0.889299in}}%
\pgfpathmoveto{\pgfqpoint{3.070459in}{0.892248in}}%
\pgfpathlineto{\pgfqpoint{3.070459in}{0.892248in}}%
\pgfpathlineto{\pgfqpoint{3.070459in}{0.895197in}}%
\pgfpathlineto{\pgfqpoint{3.075000in}{0.895197in}}%
\pgfpathlineto{\pgfqpoint{3.075000in}{0.892248in}}%
\pgfpathmoveto{\pgfqpoint{3.070459in}{0.895197in}}%
\pgfpathlineto{\pgfqpoint{3.070459in}{0.895197in}}%
\pgfpathlineto{\pgfqpoint{3.070459in}{0.898147in}}%
\pgfpathlineto{\pgfqpoint{3.075000in}{0.898147in}}%
\pgfpathlineto{\pgfqpoint{3.075000in}{0.895197in}}%
\pgfpathmoveto{\pgfqpoint{3.070459in}{0.898147in}}%
\pgfpathlineto{\pgfqpoint{3.070459in}{0.898147in}}%
\pgfpathlineto{\pgfqpoint{3.070459in}{0.901096in}}%
\pgfpathlineto{\pgfqpoint{3.075000in}{0.901096in}}%
\pgfpathlineto{\pgfqpoint{3.075000in}{0.898147in}}%
\pgfpathmoveto{\pgfqpoint{3.070459in}{0.901096in}}%
\pgfpathlineto{\pgfqpoint{3.070459in}{0.901096in}}%
\pgfpathlineto{\pgfqpoint{3.070459in}{0.904045in}}%
\pgfpathlineto{\pgfqpoint{3.075000in}{0.904045in}}%
\pgfpathlineto{\pgfqpoint{3.075000in}{0.901096in}}%
\pgfpathmoveto{\pgfqpoint{3.070459in}{0.904045in}}%
\pgfpathlineto{\pgfqpoint{3.070459in}{0.904045in}}%
\pgfpathlineto{\pgfqpoint{3.070459in}{0.906994in}}%
\pgfpathlineto{\pgfqpoint{3.075000in}{0.906994in}}%
\pgfpathlineto{\pgfqpoint{3.075000in}{0.904045in}}%
\pgfpathmoveto{\pgfqpoint{3.070459in}{0.906994in}}%
\pgfpathlineto{\pgfqpoint{3.070459in}{0.906994in}}%
\pgfpathlineto{\pgfqpoint{3.070459in}{0.909943in}}%
\pgfpathlineto{\pgfqpoint{3.075000in}{0.909943in}}%
\pgfpathlineto{\pgfqpoint{3.075000in}{0.906994in}}%
\pgfpathmoveto{\pgfqpoint{3.070459in}{0.909943in}}%
\pgfpathlineto{\pgfqpoint{3.070459in}{0.909943in}}%
\pgfpathlineto{\pgfqpoint{3.070459in}{0.912893in}}%
\pgfpathlineto{\pgfqpoint{3.075000in}{0.912893in}}%
\pgfpathlineto{\pgfqpoint{3.075000in}{0.909943in}}%
\pgfpathmoveto{\pgfqpoint{3.070459in}{0.912893in}}%
\pgfpathlineto{\pgfqpoint{3.070459in}{0.912893in}}%
\pgfpathlineto{\pgfqpoint{3.070459in}{0.915842in}}%
\pgfpathlineto{\pgfqpoint{3.075000in}{0.915842in}}%
\pgfpathlineto{\pgfqpoint{3.075000in}{0.912893in}}%
\pgfpathmoveto{\pgfqpoint{3.070459in}{0.915842in}}%
\pgfpathlineto{\pgfqpoint{3.070459in}{0.915842in}}%
\pgfpathlineto{\pgfqpoint{3.070459in}{0.918791in}}%
\pgfpathlineto{\pgfqpoint{3.075000in}{0.918791in}}%
\pgfpathlineto{\pgfqpoint{3.075000in}{0.915842in}}%
\pgfpathmoveto{\pgfqpoint{3.070459in}{0.918791in}}%
\pgfpathlineto{\pgfqpoint{3.070459in}{0.918791in}}%
\pgfpathlineto{\pgfqpoint{3.070459in}{0.921740in}}%
\pgfpathlineto{\pgfqpoint{3.075000in}{0.921740in}}%
\pgfpathlineto{\pgfqpoint{3.075000in}{0.918791in}}%
\pgfpathmoveto{\pgfqpoint{3.070459in}{0.921740in}}%
\pgfpathlineto{\pgfqpoint{3.070459in}{0.921740in}}%
\pgfpathlineto{\pgfqpoint{3.070459in}{0.924689in}}%
\pgfpathlineto{\pgfqpoint{3.075000in}{0.924689in}}%
\pgfpathlineto{\pgfqpoint{3.075000in}{0.921740in}}%
\pgfpathmoveto{\pgfqpoint{3.070459in}{0.924689in}}%
\pgfpathlineto{\pgfqpoint{3.070459in}{0.924689in}}%
\pgfpathlineto{\pgfqpoint{3.070459in}{0.927639in}}%
\pgfpathlineto{\pgfqpoint{3.075000in}{0.927639in}}%
\pgfpathlineto{\pgfqpoint{3.075000in}{0.924689in}}%
\pgfpathmoveto{\pgfqpoint{3.070459in}{0.927639in}}%
\pgfpathlineto{\pgfqpoint{3.070459in}{0.927639in}}%
\pgfpathlineto{\pgfqpoint{3.070459in}{0.930588in}}%
\pgfpathlineto{\pgfqpoint{3.075000in}{0.930588in}}%
\pgfpathlineto{\pgfqpoint{3.075000in}{0.927639in}}%
\pgfpathmoveto{\pgfqpoint{3.070459in}{0.930588in}}%
\pgfpathlineto{\pgfqpoint{3.070459in}{0.930588in}}%
\pgfpathlineto{\pgfqpoint{3.070459in}{0.933537in}}%
\pgfpathlineto{\pgfqpoint{3.075000in}{0.933537in}}%
\pgfpathlineto{\pgfqpoint{3.075000in}{0.930588in}}%
\pgfpathmoveto{\pgfqpoint{3.070459in}{0.933537in}}%
\pgfpathlineto{\pgfqpoint{3.070459in}{0.933537in}}%
\pgfpathlineto{\pgfqpoint{3.070459in}{0.936486in}}%
\pgfpathlineto{\pgfqpoint{3.075000in}{0.936486in}}%
\pgfpathlineto{\pgfqpoint{3.075000in}{0.933537in}}%
\pgfpathmoveto{\pgfqpoint{3.070459in}{0.936486in}}%
\pgfpathlineto{\pgfqpoint{3.070459in}{0.936486in}}%
\pgfpathlineto{\pgfqpoint{3.070459in}{0.939435in}}%
\pgfpathlineto{\pgfqpoint{3.075000in}{0.939435in}}%
\pgfpathlineto{\pgfqpoint{3.075000in}{0.936486in}}%
\pgfpathmoveto{\pgfqpoint{3.070459in}{0.939435in}}%
\pgfpathlineto{\pgfqpoint{3.070459in}{0.939435in}}%
\pgfpathlineto{\pgfqpoint{3.070459in}{0.942385in}}%
\pgfpathlineto{\pgfqpoint{3.075000in}{0.942385in}}%
\pgfpathlineto{\pgfqpoint{3.075000in}{0.939435in}}%
\pgfpathmoveto{\pgfqpoint{3.070459in}{0.942385in}}%
\pgfpathlineto{\pgfqpoint{3.070459in}{0.942385in}}%
\pgfpathlineto{\pgfqpoint{3.070459in}{0.945334in}}%
\pgfpathlineto{\pgfqpoint{3.075000in}{0.945334in}}%
\pgfpathlineto{\pgfqpoint{3.075000in}{0.942385in}}%
\pgfpathmoveto{\pgfqpoint{3.070459in}{0.945334in}}%
\pgfpathlineto{\pgfqpoint{3.070459in}{0.945334in}}%
\pgfpathlineto{\pgfqpoint{3.070459in}{0.948283in}}%
\pgfpathlineto{\pgfqpoint{3.075000in}{0.948283in}}%
\pgfpathlineto{\pgfqpoint{3.075000in}{0.945334in}}%
\pgfpathmoveto{\pgfqpoint{3.070459in}{0.948283in}}%
\pgfpathlineto{\pgfqpoint{3.070459in}{0.948283in}}%
\pgfpathlineto{\pgfqpoint{3.070459in}{0.951232in}}%
\pgfpathlineto{\pgfqpoint{3.075000in}{0.951232in}}%
\pgfpathlineto{\pgfqpoint{3.075000in}{0.948283in}}%
\pgfpathmoveto{\pgfqpoint{3.070459in}{0.951232in}}%
\pgfpathlineto{\pgfqpoint{3.070459in}{0.951232in}}%
\pgfpathlineto{\pgfqpoint{3.070459in}{0.954181in}}%
\pgfpathlineto{\pgfqpoint{3.075000in}{0.954181in}}%
\pgfpathlineto{\pgfqpoint{3.075000in}{0.951232in}}%
\pgfpathmoveto{\pgfqpoint{3.070459in}{0.954181in}}%
\pgfpathlineto{\pgfqpoint{3.070459in}{0.954181in}}%
\pgfpathlineto{\pgfqpoint{3.070459in}{0.957131in}}%
\pgfpathlineto{\pgfqpoint{3.075000in}{0.957131in}}%
\pgfpathlineto{\pgfqpoint{3.075000in}{0.954181in}}%
\pgfpathmoveto{\pgfqpoint{3.070459in}{0.957131in}}%
\pgfpathlineto{\pgfqpoint{3.070459in}{0.957131in}}%
\pgfpathlineto{\pgfqpoint{3.070459in}{0.960080in}}%
\pgfpathlineto{\pgfqpoint{3.075000in}{0.960080in}}%
\pgfpathlineto{\pgfqpoint{3.075000in}{0.957131in}}%
\pgfpathmoveto{\pgfqpoint{3.070459in}{0.960080in}}%
\pgfpathlineto{\pgfqpoint{3.070459in}{0.960080in}}%
\pgfpathlineto{\pgfqpoint{3.070459in}{0.963029in}}%
\pgfpathlineto{\pgfqpoint{3.075000in}{0.963029in}}%
\pgfpathlineto{\pgfqpoint{3.075000in}{0.960080in}}%
\pgfpathmoveto{\pgfqpoint{3.070459in}{0.963029in}}%
\pgfpathlineto{\pgfqpoint{3.070459in}{0.963029in}}%
\pgfpathlineto{\pgfqpoint{3.070459in}{0.965978in}}%
\pgfpathlineto{\pgfqpoint{3.075000in}{0.965978in}}%
\pgfpathlineto{\pgfqpoint{3.075000in}{0.963029in}}%
\pgfpathmoveto{\pgfqpoint{3.070459in}{0.965978in}}%
\pgfpathlineto{\pgfqpoint{3.070459in}{0.965978in}}%
\pgfpathlineto{\pgfqpoint{3.070459in}{0.968927in}}%
\pgfpathlineto{\pgfqpoint{3.075000in}{0.968927in}}%
\pgfpathlineto{\pgfqpoint{3.075000in}{0.965978in}}%
\pgfpathmoveto{\pgfqpoint{3.070459in}{0.968927in}}%
\pgfpathlineto{\pgfqpoint{3.070459in}{0.968927in}}%
\pgfpathlineto{\pgfqpoint{3.070459in}{0.971877in}}%
\pgfpathlineto{\pgfqpoint{3.075000in}{0.971877in}}%
\pgfpathlineto{\pgfqpoint{3.075000in}{0.968927in}}%
\pgfpathmoveto{\pgfqpoint{3.070459in}{0.971877in}}%
\pgfpathlineto{\pgfqpoint{3.070459in}{0.971877in}}%
\pgfpathlineto{\pgfqpoint{3.070459in}{0.974826in}}%
\pgfpathlineto{\pgfqpoint{3.075000in}{0.974826in}}%
\pgfpathlineto{\pgfqpoint{3.075000in}{0.971877in}}%
\pgfpathmoveto{\pgfqpoint{3.070459in}{0.974826in}}%
\pgfpathlineto{\pgfqpoint{3.070459in}{0.974826in}}%
\pgfpathlineto{\pgfqpoint{3.070459in}{0.977775in}}%
\pgfpathlineto{\pgfqpoint{3.075000in}{0.977775in}}%
\pgfpathlineto{\pgfqpoint{3.075000in}{0.974826in}}%
\pgfpathmoveto{\pgfqpoint{3.070459in}{0.977775in}}%
\pgfpathlineto{\pgfqpoint{3.070459in}{0.977775in}}%
\pgfpathlineto{\pgfqpoint{3.070459in}{0.980724in}}%
\pgfpathlineto{\pgfqpoint{3.075000in}{0.980724in}}%
\pgfpathlineto{\pgfqpoint{3.075000in}{0.977775in}}%
\pgfpathmoveto{\pgfqpoint{3.070459in}{0.980724in}}%
\pgfpathlineto{\pgfqpoint{3.070459in}{0.980724in}}%
\pgfpathlineto{\pgfqpoint{3.070459in}{0.983673in}}%
\pgfpathlineto{\pgfqpoint{3.075000in}{0.983673in}}%
\pgfpathlineto{\pgfqpoint{3.075000in}{0.980724in}}%
\pgfpathmoveto{\pgfqpoint{3.070459in}{0.983673in}}%
\pgfpathlineto{\pgfqpoint{3.070459in}{0.983673in}}%
\pgfpathlineto{\pgfqpoint{3.070459in}{0.986622in}}%
\pgfpathlineto{\pgfqpoint{3.075000in}{0.986622in}}%
\pgfpathlineto{\pgfqpoint{3.075000in}{0.983673in}}%
\pgfpathmoveto{\pgfqpoint{3.070459in}{0.986622in}}%
\pgfpathlineto{\pgfqpoint{3.070459in}{0.986622in}}%
\pgfpathlineto{\pgfqpoint{3.070459in}{0.989572in}}%
\pgfpathlineto{\pgfqpoint{3.075000in}{0.989572in}}%
\pgfpathlineto{\pgfqpoint{3.075000in}{0.986622in}}%
\pgfpathmoveto{\pgfqpoint{3.070459in}{0.989572in}}%
\pgfpathlineto{\pgfqpoint{3.070459in}{0.989572in}}%
\pgfpathlineto{\pgfqpoint{3.070459in}{0.992521in}}%
\pgfpathlineto{\pgfqpoint{3.075000in}{0.992521in}}%
\pgfpathlineto{\pgfqpoint{3.075000in}{0.989572in}}%
\pgfpathmoveto{\pgfqpoint{3.070459in}{0.992521in}}%
\pgfpathlineto{\pgfqpoint{3.070459in}{0.992521in}}%
\pgfpathlineto{\pgfqpoint{3.070459in}{0.995470in}}%
\pgfpathlineto{\pgfqpoint{3.075000in}{0.995470in}}%
\pgfpathlineto{\pgfqpoint{3.075000in}{0.992521in}}%
\pgfpathmoveto{\pgfqpoint{3.070459in}{0.995470in}}%
\pgfpathlineto{\pgfqpoint{3.070459in}{0.995470in}}%
\pgfpathlineto{\pgfqpoint{3.070459in}{0.998419in}}%
\pgfpathlineto{\pgfqpoint{3.075000in}{0.998419in}}%
\pgfpathlineto{\pgfqpoint{3.075000in}{0.995470in}}%
\pgfpathmoveto{\pgfqpoint{3.070459in}{0.998419in}}%
\pgfpathlineto{\pgfqpoint{3.070459in}{0.998419in}}%
\pgfpathlineto{\pgfqpoint{3.070459in}{1.001368in}}%
\pgfpathlineto{\pgfqpoint{3.075000in}{1.001368in}}%
\pgfpathlineto{\pgfqpoint{3.075000in}{0.998419in}}%
\pgfpathmoveto{\pgfqpoint{3.070459in}{1.001368in}}%
\pgfpathlineto{\pgfqpoint{3.070459in}{1.001368in}}%
\pgfpathlineto{\pgfqpoint{3.070459in}{1.004318in}}%
\pgfpathlineto{\pgfqpoint{3.075000in}{1.004318in}}%
\pgfpathlineto{\pgfqpoint{3.075000in}{1.001368in}}%
\pgfpathmoveto{\pgfqpoint{3.070459in}{1.004318in}}%
\pgfpathlineto{\pgfqpoint{3.070459in}{1.004318in}}%
\pgfpathlineto{\pgfqpoint{3.070459in}{1.007267in}}%
\pgfpathlineto{\pgfqpoint{3.075000in}{1.007267in}}%
\pgfpathlineto{\pgfqpoint{3.075000in}{1.004318in}}%
\pgfpathmoveto{\pgfqpoint{3.070459in}{1.007267in}}%
\pgfpathlineto{\pgfqpoint{3.070459in}{1.007267in}}%
\pgfpathlineto{\pgfqpoint{3.070459in}{1.010216in}}%
\pgfpathlineto{\pgfqpoint{3.075000in}{1.010216in}}%
\pgfpathlineto{\pgfqpoint{3.075000in}{1.007267in}}%
\pgfpathmoveto{\pgfqpoint{3.070459in}{1.010216in}}%
\pgfpathlineto{\pgfqpoint{3.070459in}{1.010216in}}%
\pgfpathlineto{\pgfqpoint{3.070459in}{1.013165in}}%
\pgfpathlineto{\pgfqpoint{3.075000in}{1.013165in}}%
\pgfpathlineto{\pgfqpoint{3.075000in}{1.010216in}}%
\pgfpathmoveto{\pgfqpoint{3.070459in}{1.013165in}}%
\pgfpathlineto{\pgfqpoint{3.070459in}{1.013165in}}%
\pgfpathlineto{\pgfqpoint{3.070459in}{1.016114in}}%
\pgfpathlineto{\pgfqpoint{3.075000in}{1.016114in}}%
\pgfpathlineto{\pgfqpoint{3.075000in}{1.013165in}}%
\pgfpathmoveto{\pgfqpoint{3.070459in}{1.016114in}}%
\pgfpathlineto{\pgfqpoint{3.070459in}{1.016114in}}%
\pgfpathlineto{\pgfqpoint{3.070459in}{1.019063in}}%
\pgfpathlineto{\pgfqpoint{3.075000in}{1.019063in}}%
\pgfpathlineto{\pgfqpoint{3.075000in}{1.016114in}}%
\pgfpathmoveto{\pgfqpoint{3.070459in}{1.019063in}}%
\pgfpathlineto{\pgfqpoint{3.070459in}{1.019063in}}%
\pgfpathlineto{\pgfqpoint{3.070459in}{1.022013in}}%
\pgfpathlineto{\pgfqpoint{3.075000in}{1.022013in}}%
\pgfpathlineto{\pgfqpoint{3.075000in}{1.019063in}}%
\pgfpathmoveto{\pgfqpoint{3.070459in}{1.022013in}}%
\pgfpathlineto{\pgfqpoint{3.070459in}{1.022013in}}%
\pgfpathlineto{\pgfqpoint{3.070459in}{1.024962in}}%
\pgfpathlineto{\pgfqpoint{3.075000in}{1.024962in}}%
\pgfpathlineto{\pgfqpoint{3.075000in}{1.022013in}}%
\pgfpathmoveto{\pgfqpoint{3.070459in}{1.024962in}}%
\pgfpathlineto{\pgfqpoint{3.070459in}{1.024962in}}%
\pgfpathlineto{\pgfqpoint{3.070459in}{1.027911in}}%
\pgfpathlineto{\pgfqpoint{3.075000in}{1.027911in}}%
\pgfpathlineto{\pgfqpoint{3.075000in}{1.024962in}}%
\pgfpathmoveto{\pgfqpoint{3.070459in}{1.027911in}}%
\pgfpathlineto{\pgfqpoint{3.070459in}{1.027911in}}%
\pgfpathlineto{\pgfqpoint{3.070459in}{1.030860in}}%
\pgfpathlineto{\pgfqpoint{3.075000in}{1.030860in}}%
\pgfpathlineto{\pgfqpoint{3.075000in}{1.027911in}}%
\pgfpathmoveto{\pgfqpoint{3.070459in}{1.030860in}}%
\pgfpathlineto{\pgfqpoint{3.070459in}{1.030860in}}%
\pgfpathlineto{\pgfqpoint{3.070459in}{1.033809in}}%
\pgfpathlineto{\pgfqpoint{3.075000in}{1.033809in}}%
\pgfpathlineto{\pgfqpoint{3.075000in}{1.030860in}}%
\pgfpathmoveto{\pgfqpoint{3.070459in}{1.033809in}}%
\pgfpathlineto{\pgfqpoint{3.070459in}{1.033809in}}%
\pgfpathlineto{\pgfqpoint{3.070459in}{1.036759in}}%
\pgfpathlineto{\pgfqpoint{3.075000in}{1.036759in}}%
\pgfpathlineto{\pgfqpoint{3.075000in}{1.033809in}}%
\pgfpathmoveto{\pgfqpoint{3.070459in}{1.036759in}}%
\pgfpathlineto{\pgfqpoint{3.070459in}{1.036759in}}%
\pgfpathlineto{\pgfqpoint{3.070459in}{1.039708in}}%
\pgfpathlineto{\pgfqpoint{3.075000in}{1.039708in}}%
\pgfpathlineto{\pgfqpoint{3.075000in}{1.036759in}}%
\pgfpathmoveto{\pgfqpoint{3.070459in}{1.039708in}}%
\pgfpathlineto{\pgfqpoint{3.070459in}{1.039708in}}%
\pgfpathlineto{\pgfqpoint{3.070459in}{1.042657in}}%
\pgfpathlineto{\pgfqpoint{3.075000in}{1.042657in}}%
\pgfpathlineto{\pgfqpoint{3.075000in}{1.039708in}}%
\pgfpathmoveto{\pgfqpoint{3.070459in}{1.042657in}}%
\pgfpathlineto{\pgfqpoint{3.070459in}{1.042657in}}%
\pgfpathlineto{\pgfqpoint{3.070459in}{1.045606in}}%
\pgfpathlineto{\pgfqpoint{3.075000in}{1.045606in}}%
\pgfpathlineto{\pgfqpoint{3.075000in}{1.042657in}}%
\pgfpathmoveto{\pgfqpoint{3.070459in}{1.045606in}}%
\pgfpathlineto{\pgfqpoint{3.070459in}{1.045606in}}%
\pgfpathlineto{\pgfqpoint{3.070459in}{1.048555in}}%
\pgfpathlineto{\pgfqpoint{3.075000in}{1.048555in}}%
\pgfpathlineto{\pgfqpoint{3.075000in}{1.045606in}}%
\pgfpathmoveto{\pgfqpoint{3.070459in}{1.048555in}}%
\pgfpathlineto{\pgfqpoint{3.070459in}{1.048555in}}%
\pgfpathlineto{\pgfqpoint{3.070459in}{1.051504in}}%
\pgfpathlineto{\pgfqpoint{3.075000in}{1.051504in}}%
\pgfpathlineto{\pgfqpoint{3.075000in}{1.048555in}}%
\pgfpathmoveto{\pgfqpoint{3.070459in}{1.051504in}}%
\pgfpathlineto{\pgfqpoint{3.070459in}{1.051504in}}%
\pgfpathlineto{\pgfqpoint{3.070459in}{1.054454in}}%
\pgfpathlineto{\pgfqpoint{3.075000in}{1.054454in}}%
\pgfpathlineto{\pgfqpoint{3.075000in}{1.051504in}}%
\pgfpathmoveto{\pgfqpoint{3.070459in}{1.054454in}}%
\pgfpathlineto{\pgfqpoint{3.070459in}{1.054454in}}%
\pgfpathlineto{\pgfqpoint{3.070459in}{1.057403in}}%
\pgfpathlineto{\pgfqpoint{3.075000in}{1.057403in}}%
\pgfpathlineto{\pgfqpoint{3.075000in}{1.054454in}}%
\pgfpathmoveto{\pgfqpoint{3.070459in}{1.057403in}}%
\pgfpathlineto{\pgfqpoint{3.070459in}{1.057403in}}%
\pgfpathlineto{\pgfqpoint{3.070459in}{1.060352in}}%
\pgfpathlineto{\pgfqpoint{3.075000in}{1.060352in}}%
\pgfpathlineto{\pgfqpoint{3.075000in}{1.057403in}}%
\pgfpathmoveto{\pgfqpoint{3.070459in}{1.060352in}}%
\pgfpathlineto{\pgfqpoint{3.070459in}{1.060352in}}%
\pgfpathlineto{\pgfqpoint{3.070459in}{1.063301in}}%
\pgfpathlineto{\pgfqpoint{3.075000in}{1.063301in}}%
\pgfpathlineto{\pgfqpoint{3.075000in}{1.060352in}}%
\pgfpathmoveto{\pgfqpoint{3.070459in}{1.063301in}}%
\pgfpathlineto{\pgfqpoint{3.070459in}{1.063301in}}%
\pgfpathlineto{\pgfqpoint{3.070459in}{1.066250in}}%
\pgfpathlineto{\pgfqpoint{3.075000in}{1.066250in}}%
\pgfpathlineto{\pgfqpoint{3.075000in}{1.063301in}}%
\pgfpathmoveto{\pgfqpoint{3.070459in}{1.066250in}}%
\pgfpathlineto{\pgfqpoint{3.070459in}{1.066250in}}%
\pgfpathlineto{\pgfqpoint{3.070459in}{1.069200in}}%
\pgfpathlineto{\pgfqpoint{3.075000in}{1.069200in}}%
\pgfpathlineto{\pgfqpoint{3.075000in}{1.066250in}}%
\pgfpathmoveto{\pgfqpoint{3.070459in}{1.069200in}}%
\pgfpathlineto{\pgfqpoint{3.070459in}{1.069200in}}%
\pgfpathlineto{\pgfqpoint{3.070459in}{1.072149in}}%
\pgfpathlineto{\pgfqpoint{3.075000in}{1.072149in}}%
\pgfpathlineto{\pgfqpoint{3.075000in}{1.069200in}}%
\pgfpathmoveto{\pgfqpoint{3.070459in}{1.072149in}}%
\pgfpathlineto{\pgfqpoint{3.070459in}{1.072149in}}%
\pgfpathlineto{\pgfqpoint{3.070459in}{1.075098in}}%
\pgfpathlineto{\pgfqpoint{3.075000in}{1.075098in}}%
\pgfpathlineto{\pgfqpoint{3.075000in}{1.072149in}}%
\pgfpathmoveto{\pgfqpoint{3.070459in}{1.075098in}}%
\pgfpathlineto{\pgfqpoint{3.070459in}{1.075098in}}%
\pgfpathlineto{\pgfqpoint{3.070459in}{1.078047in}}%
\pgfpathlineto{\pgfqpoint{3.075000in}{1.078047in}}%
\pgfpathlineto{\pgfqpoint{3.075000in}{1.075098in}}%
\pgfpathmoveto{\pgfqpoint{3.070459in}{1.078047in}}%
\pgfpathlineto{\pgfqpoint{3.070459in}{1.078047in}}%
\pgfpathlineto{\pgfqpoint{3.070459in}{1.080996in}}%
\pgfpathlineto{\pgfqpoint{3.075000in}{1.080996in}}%
\pgfpathlineto{\pgfqpoint{3.075000in}{1.078047in}}%
\pgfpathmoveto{\pgfqpoint{3.070459in}{1.080996in}}%
\pgfpathlineto{\pgfqpoint{3.070459in}{1.080996in}}%
\pgfpathlineto{\pgfqpoint{3.070459in}{1.083945in}}%
\pgfpathlineto{\pgfqpoint{3.075000in}{1.083945in}}%
\pgfpathlineto{\pgfqpoint{3.075000in}{1.080996in}}%
\pgfpathmoveto{\pgfqpoint{3.070459in}{1.083945in}}%
\pgfpathlineto{\pgfqpoint{3.070459in}{1.083945in}}%
\pgfpathlineto{\pgfqpoint{3.070459in}{1.086895in}}%
\pgfpathlineto{\pgfqpoint{3.075000in}{1.086895in}}%
\pgfpathlineto{\pgfqpoint{3.075000in}{1.083945in}}%
\pgfpathmoveto{\pgfqpoint{3.070459in}{1.086895in}}%
\pgfpathlineto{\pgfqpoint{3.070459in}{1.086895in}}%
\pgfpathlineto{\pgfqpoint{3.070459in}{1.089844in}}%
\pgfpathlineto{\pgfqpoint{3.075000in}{1.089844in}}%
\pgfpathlineto{\pgfqpoint{3.075000in}{1.086895in}}%
\pgfpathmoveto{\pgfqpoint{3.070459in}{1.089844in}}%
\pgfpathlineto{\pgfqpoint{3.070459in}{1.089844in}}%
\pgfpathlineto{\pgfqpoint{3.070459in}{1.092793in}}%
\pgfpathlineto{\pgfqpoint{3.075000in}{1.092793in}}%
\pgfpathlineto{\pgfqpoint{3.075000in}{1.089844in}}%
\pgfpathmoveto{\pgfqpoint{3.070459in}{1.092793in}}%
\pgfpathlineto{\pgfqpoint{3.070459in}{1.092793in}}%
\pgfpathlineto{\pgfqpoint{3.070459in}{1.095742in}}%
\pgfpathlineto{\pgfqpoint{3.075000in}{1.095742in}}%
\pgfpathlineto{\pgfqpoint{3.075000in}{1.092793in}}%
\pgfpathmoveto{\pgfqpoint{3.070459in}{1.095742in}}%
\pgfpathlineto{\pgfqpoint{3.070459in}{1.095742in}}%
\pgfpathlineto{\pgfqpoint{3.070459in}{1.098691in}}%
\pgfpathlineto{\pgfqpoint{3.075000in}{1.098691in}}%
\pgfpathlineto{\pgfqpoint{3.075000in}{1.095742in}}%
\pgfpathmoveto{\pgfqpoint{3.070459in}{1.098691in}}%
\pgfpathlineto{\pgfqpoint{3.070459in}{1.098691in}}%
\pgfpathlineto{\pgfqpoint{3.070459in}{1.101640in}}%
\pgfpathlineto{\pgfqpoint{3.075000in}{1.101640in}}%
\pgfpathlineto{\pgfqpoint{3.075000in}{1.098691in}}%
\pgfpathmoveto{\pgfqpoint{3.070459in}{1.101640in}}%
\pgfpathlineto{\pgfqpoint{3.070459in}{1.101640in}}%
\pgfpathlineto{\pgfqpoint{3.070459in}{1.104590in}}%
\pgfpathlineto{\pgfqpoint{3.075000in}{1.104590in}}%
\pgfpathlineto{\pgfqpoint{3.075000in}{1.101640in}}%
\pgfpathmoveto{\pgfqpoint{3.070459in}{1.104590in}}%
\pgfpathlineto{\pgfqpoint{3.070459in}{1.104590in}}%
\pgfpathlineto{\pgfqpoint{3.070459in}{1.107539in}}%
\pgfpathlineto{\pgfqpoint{3.075000in}{1.107539in}}%
\pgfpathlineto{\pgfqpoint{3.075000in}{1.104590in}}%
\pgfpathmoveto{\pgfqpoint{3.070459in}{1.107539in}}%
\pgfpathlineto{\pgfqpoint{3.070459in}{1.107539in}}%
\pgfpathlineto{\pgfqpoint{3.070459in}{1.110488in}}%
\pgfpathlineto{\pgfqpoint{3.075000in}{1.110488in}}%
\pgfpathlineto{\pgfqpoint{3.075000in}{1.107539in}}%
\pgfpathmoveto{\pgfqpoint{3.070459in}{1.110488in}}%
\pgfpathlineto{\pgfqpoint{3.070459in}{1.110488in}}%
\pgfpathlineto{\pgfqpoint{3.070459in}{1.113437in}}%
\pgfpathlineto{\pgfqpoint{3.075000in}{1.113437in}}%
\pgfpathlineto{\pgfqpoint{3.075000in}{1.110488in}}%
\pgfpathmoveto{\pgfqpoint{3.070459in}{1.113437in}}%
\pgfpathlineto{\pgfqpoint{3.070459in}{1.113437in}}%
\pgfpathlineto{\pgfqpoint{3.070459in}{1.116386in}}%
\pgfpathlineto{\pgfqpoint{3.075000in}{1.116386in}}%
\pgfpathlineto{\pgfqpoint{3.075000in}{1.113437in}}%
\pgfpathmoveto{\pgfqpoint{3.070459in}{1.116386in}}%
\pgfpathlineto{\pgfqpoint{3.070459in}{1.116386in}}%
\pgfpathlineto{\pgfqpoint{3.070459in}{1.119335in}}%
\pgfpathlineto{\pgfqpoint{3.075000in}{1.119335in}}%
\pgfpathlineto{\pgfqpoint{3.075000in}{1.116386in}}%
\pgfpathmoveto{\pgfqpoint{3.070459in}{1.119335in}}%
\pgfpathlineto{\pgfqpoint{3.070459in}{1.119335in}}%
\pgfpathlineto{\pgfqpoint{3.070459in}{1.122285in}}%
\pgfpathlineto{\pgfqpoint{3.075000in}{1.122285in}}%
\pgfpathlineto{\pgfqpoint{3.075000in}{1.119335in}}%
\pgfpathmoveto{\pgfqpoint{3.070459in}{1.122285in}}%
\pgfpathlineto{\pgfqpoint{3.070459in}{1.122285in}}%
\pgfpathlineto{\pgfqpoint{3.070459in}{1.125234in}}%
\pgfpathlineto{\pgfqpoint{3.075000in}{1.125234in}}%
\pgfpathlineto{\pgfqpoint{3.075000in}{1.122285in}}%
\pgfpathmoveto{\pgfqpoint{3.070459in}{1.125234in}}%
\pgfpathlineto{\pgfqpoint{3.070459in}{1.125234in}}%
\pgfpathlineto{\pgfqpoint{3.070459in}{1.128183in}}%
\pgfpathlineto{\pgfqpoint{3.075000in}{1.128183in}}%
\pgfpathlineto{\pgfqpoint{3.075000in}{1.125234in}}%
\pgfpathmoveto{\pgfqpoint{3.070459in}{1.128183in}}%
\pgfpathlineto{\pgfqpoint{3.070459in}{1.128183in}}%
\pgfpathlineto{\pgfqpoint{3.070459in}{1.131132in}}%
\pgfpathlineto{\pgfqpoint{3.075000in}{1.131132in}}%
\pgfpathlineto{\pgfqpoint{3.075000in}{1.128183in}}%
\pgfpathmoveto{\pgfqpoint{3.070459in}{1.131132in}}%
\pgfpathlineto{\pgfqpoint{3.070459in}{1.131132in}}%
\pgfpathlineto{\pgfqpoint{3.070459in}{1.134081in}}%
\pgfpathlineto{\pgfqpoint{3.075000in}{1.134081in}}%
\pgfpathlineto{\pgfqpoint{3.075000in}{1.131132in}}%
\pgfpathmoveto{\pgfqpoint{3.070459in}{1.134081in}}%
\pgfpathlineto{\pgfqpoint{3.070459in}{1.134081in}}%
\pgfpathlineto{\pgfqpoint{3.070459in}{1.137030in}}%
\pgfpathlineto{\pgfqpoint{3.075000in}{1.137030in}}%
\pgfpathlineto{\pgfqpoint{3.075000in}{1.134081in}}%
\pgfpathmoveto{\pgfqpoint{3.070459in}{1.137030in}}%
\pgfpathlineto{\pgfqpoint{3.070459in}{1.137030in}}%
\pgfpathlineto{\pgfqpoint{3.070459in}{1.139979in}}%
\pgfpathlineto{\pgfqpoint{3.075000in}{1.139979in}}%
\pgfpathlineto{\pgfqpoint{3.075000in}{1.137030in}}%
\pgfpathmoveto{\pgfqpoint{3.070459in}{1.139979in}}%
\pgfpathlineto{\pgfqpoint{3.070459in}{1.139979in}}%
\pgfpathlineto{\pgfqpoint{3.070459in}{1.142929in}}%
\pgfpathlineto{\pgfqpoint{3.075000in}{1.142929in}}%
\pgfpathlineto{\pgfqpoint{3.075000in}{1.139979in}}%
\pgfpathmoveto{\pgfqpoint{3.070459in}{1.142929in}}%
\pgfpathlineto{\pgfqpoint{3.070459in}{1.142929in}}%
\pgfpathlineto{\pgfqpoint{3.070459in}{1.145878in}}%
\pgfpathlineto{\pgfqpoint{3.075000in}{1.145878in}}%
\pgfpathlineto{\pgfqpoint{3.075000in}{1.142929in}}%
\pgfpathmoveto{\pgfqpoint{3.070459in}{1.145878in}}%
\pgfpathlineto{\pgfqpoint{3.070459in}{1.145878in}}%
\pgfpathlineto{\pgfqpoint{3.070459in}{1.148827in}}%
\pgfpathlineto{\pgfqpoint{3.075000in}{1.148827in}}%
\pgfpathlineto{\pgfqpoint{3.075000in}{1.145878in}}%
\pgfpathmoveto{\pgfqpoint{3.070459in}{1.148827in}}%
\pgfpathlineto{\pgfqpoint{3.070459in}{1.148827in}}%
\pgfpathlineto{\pgfqpoint{3.070459in}{1.151776in}}%
\pgfpathlineto{\pgfqpoint{3.075000in}{1.151776in}}%
\pgfpathlineto{\pgfqpoint{3.075000in}{1.148827in}}%
\pgfpathmoveto{\pgfqpoint{3.070459in}{1.151776in}}%
\pgfpathlineto{\pgfqpoint{3.070459in}{1.151776in}}%
\pgfpathlineto{\pgfqpoint{3.070459in}{1.154725in}}%
\pgfpathlineto{\pgfqpoint{3.075000in}{1.154725in}}%
\pgfpathlineto{\pgfqpoint{3.075000in}{1.151776in}}%
\pgfpathmoveto{\pgfqpoint{3.070459in}{1.154725in}}%
\pgfpathlineto{\pgfqpoint{3.070459in}{1.154725in}}%
\pgfpathlineto{\pgfqpoint{3.070459in}{1.157674in}}%
\pgfpathlineto{\pgfqpoint{3.075000in}{1.157674in}}%
\pgfpathlineto{\pgfqpoint{3.075000in}{1.154725in}}%
\pgfpathmoveto{\pgfqpoint{3.070459in}{1.157674in}}%
\pgfpathlineto{\pgfqpoint{3.070459in}{1.157674in}}%
\pgfpathlineto{\pgfqpoint{3.070459in}{1.160624in}}%
\pgfpathlineto{\pgfqpoint{3.075000in}{1.160624in}}%
\pgfpathlineto{\pgfqpoint{3.075000in}{1.157674in}}%
\pgfpathmoveto{\pgfqpoint{3.070459in}{1.160624in}}%
\pgfpathlineto{\pgfqpoint{3.070459in}{1.160624in}}%
\pgfpathlineto{\pgfqpoint{3.070459in}{1.163573in}}%
\pgfpathlineto{\pgfqpoint{3.075000in}{1.163573in}}%
\pgfpathlineto{\pgfqpoint{3.075000in}{1.160624in}}%
\pgfpathmoveto{\pgfqpoint{3.070459in}{1.163573in}}%
\pgfpathlineto{\pgfqpoint{3.070459in}{1.163573in}}%
\pgfpathlineto{\pgfqpoint{3.070459in}{1.166522in}}%
\pgfpathlineto{\pgfqpoint{3.075000in}{1.166522in}}%
\pgfpathlineto{\pgfqpoint{3.075000in}{1.163573in}}%
\pgfpathmoveto{\pgfqpoint{3.070459in}{1.166522in}}%
\pgfpathlineto{\pgfqpoint{3.070459in}{1.166522in}}%
\pgfpathlineto{\pgfqpoint{3.070459in}{1.169472in}}%
\pgfpathlineto{\pgfqpoint{3.075000in}{1.169472in}}%
\pgfpathlineto{\pgfqpoint{3.075000in}{1.166522in}}%
\pgfpathmoveto{\pgfqpoint{3.070459in}{1.169472in}}%
\pgfpathlineto{\pgfqpoint{3.070459in}{1.169472in}}%
\pgfpathlineto{\pgfqpoint{3.070459in}{1.172421in}}%
\pgfpathlineto{\pgfqpoint{3.075000in}{1.172421in}}%
\pgfpathlineto{\pgfqpoint{3.075000in}{1.169472in}}%
\pgfpathmoveto{\pgfqpoint{3.070459in}{1.172421in}}%
\pgfpathlineto{\pgfqpoint{3.070459in}{1.172421in}}%
\pgfpathlineto{\pgfqpoint{3.070459in}{1.175370in}}%
\pgfpathlineto{\pgfqpoint{3.075000in}{1.175370in}}%
\pgfpathlineto{\pgfqpoint{3.075000in}{1.172421in}}%
\pgfpathmoveto{\pgfqpoint{3.070459in}{1.175370in}}%
\pgfpathlineto{\pgfqpoint{3.070459in}{1.175370in}}%
\pgfpathlineto{\pgfqpoint{3.070459in}{1.178320in}}%
\pgfpathlineto{\pgfqpoint{3.075000in}{1.178320in}}%
\pgfpathlineto{\pgfqpoint{3.075000in}{1.175370in}}%
\pgfpathmoveto{\pgfqpoint{3.070459in}{1.178320in}}%
\pgfpathlineto{\pgfqpoint{3.070459in}{1.178320in}}%
\pgfpathlineto{\pgfqpoint{3.070459in}{1.181269in}}%
\pgfpathlineto{\pgfqpoint{3.075000in}{1.181269in}}%
\pgfpathlineto{\pgfqpoint{3.075000in}{1.178320in}}%
\pgfpathmoveto{\pgfqpoint{3.070459in}{1.181269in}}%
\pgfpathlineto{\pgfqpoint{3.070459in}{1.181269in}}%
\pgfpathlineto{\pgfqpoint{3.070459in}{1.184218in}}%
\pgfpathlineto{\pgfqpoint{3.075000in}{1.184218in}}%
\pgfpathlineto{\pgfqpoint{3.075000in}{1.181269in}}%
\pgfpathmoveto{\pgfqpoint{3.070459in}{1.184218in}}%
\pgfpathlineto{\pgfqpoint{3.070459in}{1.184218in}}%
\pgfpathlineto{\pgfqpoint{3.070459in}{1.187168in}}%
\pgfpathlineto{\pgfqpoint{3.075000in}{1.187168in}}%
\pgfpathlineto{\pgfqpoint{3.075000in}{1.184218in}}%
\pgfpathmoveto{\pgfqpoint{3.070459in}{1.187168in}}%
\pgfpathlineto{\pgfqpoint{3.070459in}{1.187168in}}%
\pgfpathlineto{\pgfqpoint{3.070459in}{1.190117in}}%
\pgfpathlineto{\pgfqpoint{3.075000in}{1.190117in}}%
\pgfpathlineto{\pgfqpoint{3.075000in}{1.187168in}}%
\pgfpathmoveto{\pgfqpoint{3.070459in}{1.190117in}}%
\pgfpathlineto{\pgfqpoint{3.070459in}{1.190117in}}%
\pgfpathlineto{\pgfqpoint{3.070459in}{1.193066in}}%
\pgfpathlineto{\pgfqpoint{3.075000in}{1.193066in}}%
\pgfpathlineto{\pgfqpoint{3.075000in}{1.190117in}}%
\pgfpathmoveto{\pgfqpoint{3.070459in}{1.193066in}}%
\pgfpathlineto{\pgfqpoint{3.070459in}{1.193066in}}%
\pgfpathlineto{\pgfqpoint{3.070459in}{1.196016in}}%
\pgfpathlineto{\pgfqpoint{3.075000in}{1.196016in}}%
\pgfpathlineto{\pgfqpoint{3.075000in}{1.193066in}}%
\pgfpathmoveto{\pgfqpoint{3.070459in}{1.196016in}}%
\pgfpathlineto{\pgfqpoint{3.070459in}{1.196016in}}%
\pgfpathlineto{\pgfqpoint{3.070459in}{1.198965in}}%
\pgfpathlineto{\pgfqpoint{3.075000in}{1.198965in}}%
\pgfpathlineto{\pgfqpoint{3.075000in}{1.196016in}}%
\pgfpathmoveto{\pgfqpoint{3.070459in}{1.198965in}}%
\pgfpathlineto{\pgfqpoint{3.070459in}{1.198965in}}%
\pgfpathlineto{\pgfqpoint{3.070459in}{1.201914in}}%
\pgfpathlineto{\pgfqpoint{3.075000in}{1.201914in}}%
\pgfpathlineto{\pgfqpoint{3.075000in}{1.198965in}}%
\pgfpathmoveto{\pgfqpoint{3.070459in}{1.201914in}}%
\pgfpathlineto{\pgfqpoint{3.070459in}{1.201914in}}%
\pgfpathlineto{\pgfqpoint{3.070459in}{1.204864in}}%
\pgfpathlineto{\pgfqpoint{3.075000in}{1.204864in}}%
\pgfpathlineto{\pgfqpoint{3.075000in}{1.201914in}}%
\pgfpathmoveto{\pgfqpoint{3.070459in}{1.204864in}}%
\pgfpathlineto{\pgfqpoint{3.070459in}{1.204864in}}%
\pgfpathlineto{\pgfqpoint{3.070459in}{1.207813in}}%
\pgfpathlineto{\pgfqpoint{3.075000in}{1.207813in}}%
\pgfpathlineto{\pgfqpoint{3.075000in}{1.204864in}}%
\pgfpathmoveto{\pgfqpoint{3.070459in}{1.207813in}}%
\pgfpathlineto{\pgfqpoint{3.070459in}{1.207813in}}%
\pgfpathlineto{\pgfqpoint{3.070459in}{1.210762in}}%
\pgfpathlineto{\pgfqpoint{3.075000in}{1.210762in}}%
\pgfpathlineto{\pgfqpoint{3.075000in}{1.207813in}}%
\pgfpathmoveto{\pgfqpoint{3.070459in}{1.210762in}}%
\pgfpathlineto{\pgfqpoint{3.070459in}{1.210762in}}%
\pgfpathlineto{\pgfqpoint{3.070459in}{1.213712in}}%
\pgfpathlineto{\pgfqpoint{3.075000in}{1.213712in}}%
\pgfpathlineto{\pgfqpoint{3.075000in}{1.210762in}}%
\pgfpathmoveto{\pgfqpoint{3.070459in}{1.213712in}}%
\pgfpathlineto{\pgfqpoint{3.070459in}{1.213712in}}%
\pgfpathlineto{\pgfqpoint{3.070459in}{1.216661in}}%
\pgfpathlineto{\pgfqpoint{3.075000in}{1.216661in}}%
\pgfpathlineto{\pgfqpoint{3.075000in}{1.213712in}}%
\pgfpathmoveto{\pgfqpoint{3.070459in}{1.216661in}}%
\pgfpathlineto{\pgfqpoint{3.070459in}{1.216661in}}%
\pgfpathlineto{\pgfqpoint{3.070459in}{1.219610in}}%
\pgfpathlineto{\pgfqpoint{3.075000in}{1.219610in}}%
\pgfpathlineto{\pgfqpoint{3.075000in}{1.216661in}}%
\pgfpathmoveto{\pgfqpoint{3.070459in}{1.219610in}}%
\pgfpathlineto{\pgfqpoint{3.070459in}{1.219610in}}%
\pgfpathlineto{\pgfqpoint{3.070459in}{1.222560in}}%
\pgfpathlineto{\pgfqpoint{3.075000in}{1.222560in}}%
\pgfpathlineto{\pgfqpoint{3.075000in}{1.219610in}}%
\pgfpathmoveto{\pgfqpoint{3.070459in}{1.222560in}}%
\pgfpathlineto{\pgfqpoint{3.070459in}{1.222560in}}%
\pgfpathlineto{\pgfqpoint{3.070459in}{1.225509in}}%
\pgfpathlineto{\pgfqpoint{3.075000in}{1.225509in}}%
\pgfpathlineto{\pgfqpoint{3.075000in}{1.222560in}}%
\pgfpathmoveto{\pgfqpoint{3.070459in}{1.225509in}}%
\pgfpathlineto{\pgfqpoint{3.070459in}{1.225509in}}%
\pgfpathlineto{\pgfqpoint{3.070459in}{1.228458in}}%
\pgfpathlineto{\pgfqpoint{3.075000in}{1.228458in}}%
\pgfpathlineto{\pgfqpoint{3.075000in}{1.225509in}}%
\pgfpathmoveto{\pgfqpoint{3.070459in}{1.228458in}}%
\pgfpathlineto{\pgfqpoint{3.070459in}{1.228458in}}%
\pgfpathlineto{\pgfqpoint{3.070459in}{1.231407in}}%
\pgfpathlineto{\pgfqpoint{3.075000in}{1.231407in}}%
\pgfpathlineto{\pgfqpoint{3.075000in}{1.228458in}}%
\pgfpathmoveto{\pgfqpoint{3.070459in}{1.231407in}}%
\pgfpathlineto{\pgfqpoint{3.070459in}{1.231407in}}%
\pgfpathlineto{\pgfqpoint{3.070459in}{1.234357in}}%
\pgfpathlineto{\pgfqpoint{3.075000in}{1.234357in}}%
\pgfpathlineto{\pgfqpoint{3.075000in}{1.231407in}}%
\pgfpathmoveto{\pgfqpoint{3.070459in}{1.234357in}}%
\pgfpathlineto{\pgfqpoint{3.070459in}{1.234357in}}%
\pgfpathlineto{\pgfqpoint{3.070459in}{1.237306in}}%
\pgfpathlineto{\pgfqpoint{3.075000in}{1.237306in}}%
\pgfpathlineto{\pgfqpoint{3.075000in}{1.234357in}}%
\pgfpathmoveto{\pgfqpoint{3.070459in}{1.237306in}}%
\pgfpathlineto{\pgfqpoint{3.070459in}{1.237306in}}%
\pgfpathlineto{\pgfqpoint{3.070459in}{1.240255in}}%
\pgfpathlineto{\pgfqpoint{3.075000in}{1.240255in}}%
\pgfpathlineto{\pgfqpoint{3.075000in}{1.237306in}}%
\pgfpathmoveto{\pgfqpoint{3.070459in}{1.240255in}}%
\pgfpathlineto{\pgfqpoint{3.070459in}{1.240255in}}%
\pgfpathlineto{\pgfqpoint{3.070459in}{1.243205in}}%
\pgfpathlineto{\pgfqpoint{3.075000in}{1.243205in}}%
\pgfpathlineto{\pgfqpoint{3.075000in}{1.240255in}}%
\pgfpathmoveto{\pgfqpoint{3.070459in}{1.243205in}}%
\pgfpathlineto{\pgfqpoint{3.070459in}{1.243205in}}%
\pgfpathlineto{\pgfqpoint{3.070459in}{1.246154in}}%
\pgfpathlineto{\pgfqpoint{3.075000in}{1.246154in}}%
\pgfpathlineto{\pgfqpoint{3.075000in}{1.243205in}}%
\pgfpathmoveto{\pgfqpoint{3.070459in}{1.246154in}}%
\pgfpathlineto{\pgfqpoint{3.070459in}{1.246154in}}%
\pgfpathlineto{\pgfqpoint{3.070459in}{1.249103in}}%
\pgfpathlineto{\pgfqpoint{3.075000in}{1.249103in}}%
\pgfpathlineto{\pgfqpoint{3.075000in}{1.246154in}}%
\pgfpathmoveto{\pgfqpoint{3.070459in}{1.249103in}}%
\pgfpathlineto{\pgfqpoint{3.070459in}{1.249103in}}%
\pgfpathlineto{\pgfqpoint{3.070459in}{1.252053in}}%
\pgfpathlineto{\pgfqpoint{3.075000in}{1.252053in}}%
\pgfpathlineto{\pgfqpoint{3.075000in}{1.249103in}}%
\pgfpathmoveto{\pgfqpoint{3.070459in}{1.252053in}}%
\pgfpathlineto{\pgfqpoint{3.070459in}{1.252053in}}%
\pgfpathlineto{\pgfqpoint{3.070459in}{1.255002in}}%
\pgfpathlineto{\pgfqpoint{3.075000in}{1.255002in}}%
\pgfpathlineto{\pgfqpoint{3.075000in}{1.252053in}}%
\pgfpathmoveto{\pgfqpoint{3.070459in}{1.255002in}}%
\pgfpathlineto{\pgfqpoint{3.070459in}{1.255002in}}%
\pgfpathlineto{\pgfqpoint{3.070459in}{1.257951in}}%
\pgfpathlineto{\pgfqpoint{3.075000in}{1.257951in}}%
\pgfpathlineto{\pgfqpoint{3.075000in}{1.255002in}}%
\pgfpathmoveto{\pgfqpoint{3.070459in}{1.257951in}}%
\pgfpathlineto{\pgfqpoint{3.070459in}{1.257951in}}%
\pgfpathlineto{\pgfqpoint{3.070459in}{1.260900in}}%
\pgfpathlineto{\pgfqpoint{3.075000in}{1.260900in}}%
\pgfpathlineto{\pgfqpoint{3.075000in}{1.257951in}}%
\pgfpathmoveto{\pgfqpoint{3.070459in}{1.260900in}}%
\pgfpathlineto{\pgfqpoint{3.070459in}{1.260900in}}%
\pgfpathlineto{\pgfqpoint{3.070459in}{1.263849in}}%
\pgfpathlineto{\pgfqpoint{3.075000in}{1.263849in}}%
\pgfpathlineto{\pgfqpoint{3.075000in}{1.260900in}}%
\pgfpathmoveto{\pgfqpoint{3.070459in}{1.263849in}}%
\pgfpathlineto{\pgfqpoint{3.070459in}{1.263849in}}%
\pgfpathlineto{\pgfqpoint{3.070459in}{1.266799in}}%
\pgfpathlineto{\pgfqpoint{3.075000in}{1.266799in}}%
\pgfpathlineto{\pgfqpoint{3.075000in}{1.263849in}}%
\pgfpathmoveto{\pgfqpoint{3.070459in}{1.266799in}}%
\pgfpathlineto{\pgfqpoint{3.070459in}{1.266799in}}%
\pgfpathlineto{\pgfqpoint{3.070459in}{1.269748in}}%
\pgfpathlineto{\pgfqpoint{3.075000in}{1.269748in}}%
\pgfpathlineto{\pgfqpoint{3.075000in}{1.266799in}}%
\pgfpathmoveto{\pgfqpoint{3.070459in}{1.269748in}}%
\pgfpathlineto{\pgfqpoint{3.070459in}{1.269748in}}%
\pgfpathlineto{\pgfqpoint{3.070459in}{1.272697in}}%
\pgfpathlineto{\pgfqpoint{3.075000in}{1.272697in}}%
\pgfpathlineto{\pgfqpoint{3.075000in}{1.269748in}}%
\pgfpathmoveto{\pgfqpoint{3.070459in}{1.272697in}}%
\pgfpathlineto{\pgfqpoint{3.070459in}{1.272697in}}%
\pgfpathlineto{\pgfqpoint{3.070459in}{1.275646in}}%
\pgfpathlineto{\pgfqpoint{3.075000in}{1.275646in}}%
\pgfpathlineto{\pgfqpoint{3.075000in}{1.272697in}}%
\pgfpathmoveto{\pgfqpoint{3.070459in}{1.275646in}}%
\pgfpathlineto{\pgfqpoint{3.070459in}{1.275646in}}%
\pgfpathlineto{\pgfqpoint{3.070459in}{1.278595in}}%
\pgfpathlineto{\pgfqpoint{3.075000in}{1.278595in}}%
\pgfpathlineto{\pgfqpoint{3.075000in}{1.275646in}}%
\pgfpathmoveto{\pgfqpoint{3.070459in}{1.278595in}}%
\pgfpathlineto{\pgfqpoint{3.070459in}{1.278595in}}%
\pgfpathlineto{\pgfqpoint{3.070459in}{1.281544in}}%
\pgfpathlineto{\pgfqpoint{3.075000in}{1.281544in}}%
\pgfpathlineto{\pgfqpoint{3.075000in}{1.278595in}}%
\pgfpathmoveto{\pgfqpoint{3.070459in}{1.281544in}}%
\pgfpathlineto{\pgfqpoint{3.070459in}{1.281544in}}%
\pgfpathlineto{\pgfqpoint{3.070459in}{1.284493in}}%
\pgfpathlineto{\pgfqpoint{3.075000in}{1.284493in}}%
\pgfpathlineto{\pgfqpoint{3.075000in}{1.281544in}}%
\pgfpathmoveto{\pgfqpoint{3.070459in}{1.284493in}}%
\pgfpathlineto{\pgfqpoint{3.070459in}{1.284493in}}%
\pgfpathlineto{\pgfqpoint{3.070459in}{1.287442in}}%
\pgfpathlineto{\pgfqpoint{3.075000in}{1.287442in}}%
\pgfpathlineto{\pgfqpoint{3.075000in}{1.284493in}}%
\pgfpathmoveto{\pgfqpoint{3.070459in}{1.287442in}}%
\pgfpathlineto{\pgfqpoint{3.070459in}{1.287442in}}%
\pgfpathlineto{\pgfqpoint{3.070459in}{1.290392in}}%
\pgfpathlineto{\pgfqpoint{3.075000in}{1.290392in}}%
\pgfpathlineto{\pgfqpoint{3.075000in}{1.287442in}}%
\pgfpathmoveto{\pgfqpoint{3.070459in}{1.290392in}}%
\pgfpathlineto{\pgfqpoint{3.070459in}{1.290392in}}%
\pgfpathlineto{\pgfqpoint{3.070459in}{1.293341in}}%
\pgfpathlineto{\pgfqpoint{3.075000in}{1.293341in}}%
\pgfpathlineto{\pgfqpoint{3.075000in}{1.290392in}}%
\pgfpathmoveto{\pgfqpoint{3.070459in}{1.293341in}}%
\pgfpathlineto{\pgfqpoint{3.070459in}{1.293341in}}%
\pgfpathlineto{\pgfqpoint{3.070459in}{1.296290in}}%
\pgfpathlineto{\pgfqpoint{3.075000in}{1.296290in}}%
\pgfpathlineto{\pgfqpoint{3.075000in}{1.293341in}}%
\pgfpathmoveto{\pgfqpoint{3.070459in}{1.296290in}}%
\pgfpathlineto{\pgfqpoint{3.070459in}{1.296290in}}%
\pgfpathlineto{\pgfqpoint{3.070459in}{1.299239in}}%
\pgfpathlineto{\pgfqpoint{3.075000in}{1.299239in}}%
\pgfpathlineto{\pgfqpoint{3.075000in}{1.296290in}}%
\pgfpathmoveto{\pgfqpoint{3.070459in}{1.299239in}}%
\pgfpathlineto{\pgfqpoint{3.070459in}{1.299239in}}%
\pgfpathlineto{\pgfqpoint{3.070459in}{1.302188in}}%
\pgfpathlineto{\pgfqpoint{3.075000in}{1.302188in}}%
\pgfpathlineto{\pgfqpoint{3.075000in}{1.299239in}}%
\pgfpathmoveto{\pgfqpoint{3.070459in}{1.302188in}}%
\pgfpathlineto{\pgfqpoint{3.070459in}{1.302188in}}%
\pgfpathlineto{\pgfqpoint{3.070459in}{1.305137in}}%
\pgfpathlineto{\pgfqpoint{3.075000in}{1.305137in}}%
\pgfpathlineto{\pgfqpoint{3.075000in}{1.302188in}}%
\pgfpathmoveto{\pgfqpoint{3.070459in}{1.305137in}}%
\pgfpathlineto{\pgfqpoint{3.070459in}{1.305137in}}%
\pgfpathlineto{\pgfqpoint{3.070459in}{1.308086in}}%
\pgfpathlineto{\pgfqpoint{3.075000in}{1.308086in}}%
\pgfpathlineto{\pgfqpoint{3.075000in}{1.305137in}}%
\pgfpathmoveto{\pgfqpoint{3.070459in}{1.308086in}}%
\pgfpathlineto{\pgfqpoint{3.070459in}{1.308086in}}%
\pgfpathlineto{\pgfqpoint{3.070459in}{1.311035in}}%
\pgfpathlineto{\pgfqpoint{3.075000in}{1.311035in}}%
\pgfpathlineto{\pgfqpoint{3.075000in}{1.308086in}}%
\pgfpathmoveto{\pgfqpoint{3.070459in}{1.311035in}}%
\pgfpathlineto{\pgfqpoint{3.070459in}{1.311035in}}%
\pgfpathlineto{\pgfqpoint{3.070459in}{1.313985in}}%
\pgfpathlineto{\pgfqpoint{3.075000in}{1.313985in}}%
\pgfpathlineto{\pgfqpoint{3.075000in}{1.311035in}}%
\pgfpathmoveto{\pgfqpoint{3.070459in}{1.313985in}}%
\pgfpathlineto{\pgfqpoint{3.070459in}{1.313985in}}%
\pgfpathlineto{\pgfqpoint{3.070459in}{1.316934in}}%
\pgfpathlineto{\pgfqpoint{3.075000in}{1.316934in}}%
\pgfpathlineto{\pgfqpoint{3.075000in}{1.313985in}}%
\pgfpathmoveto{\pgfqpoint{3.070459in}{1.316934in}}%
\pgfpathlineto{\pgfqpoint{3.070459in}{1.316934in}}%
\pgfpathlineto{\pgfqpoint{3.070459in}{1.319883in}}%
\pgfpathlineto{\pgfqpoint{3.075000in}{1.319883in}}%
\pgfpathlineto{\pgfqpoint{3.075000in}{1.316934in}}%
\pgfpathmoveto{\pgfqpoint{3.070459in}{1.319883in}}%
\pgfpathlineto{\pgfqpoint{3.070459in}{1.319883in}}%
\pgfpathlineto{\pgfqpoint{3.070459in}{1.322832in}}%
\pgfpathlineto{\pgfqpoint{3.075000in}{1.322832in}}%
\pgfpathlineto{\pgfqpoint{3.075000in}{1.319883in}}%
\pgfpathmoveto{\pgfqpoint{3.070459in}{1.322832in}}%
\pgfpathlineto{\pgfqpoint{3.070459in}{1.322832in}}%
\pgfpathlineto{\pgfqpoint{3.070459in}{1.325781in}}%
\pgfpathlineto{\pgfqpoint{3.075000in}{1.325781in}}%
\pgfpathlineto{\pgfqpoint{3.075000in}{1.322832in}}%
\pgfpathmoveto{\pgfqpoint{3.070459in}{1.325781in}}%
\pgfpathlineto{\pgfqpoint{3.070459in}{1.325781in}}%
\pgfpathlineto{\pgfqpoint{3.070459in}{1.328730in}}%
\pgfpathlineto{\pgfqpoint{3.075000in}{1.328730in}}%
\pgfpathlineto{\pgfqpoint{3.075000in}{1.325781in}}%
\pgfpathmoveto{\pgfqpoint{3.070459in}{1.328730in}}%
\pgfpathlineto{\pgfqpoint{3.070459in}{1.328730in}}%
\pgfpathlineto{\pgfqpoint{3.070459in}{1.331679in}}%
\pgfpathlineto{\pgfqpoint{3.075000in}{1.331679in}}%
\pgfpathlineto{\pgfqpoint{3.075000in}{1.328730in}}%
\pgfpathmoveto{\pgfqpoint{3.070459in}{1.331679in}}%
\pgfpathlineto{\pgfqpoint{3.070459in}{1.331679in}}%
\pgfpathlineto{\pgfqpoint{3.070459in}{1.334628in}}%
\pgfpathlineto{\pgfqpoint{3.075000in}{1.334628in}}%
\pgfpathlineto{\pgfqpoint{3.075000in}{1.331679in}}%
\pgfpathmoveto{\pgfqpoint{3.070459in}{1.334628in}}%
\pgfpathlineto{\pgfqpoint{3.070459in}{1.334628in}}%
\pgfpathlineto{\pgfqpoint{3.070459in}{1.337577in}}%
\pgfpathlineto{\pgfqpoint{3.075000in}{1.337577in}}%
\pgfpathlineto{\pgfqpoint{3.075000in}{1.334628in}}%
\pgfpathmoveto{\pgfqpoint{3.070459in}{1.337577in}}%
\pgfpathlineto{\pgfqpoint{3.070459in}{1.337577in}}%
\pgfpathlineto{\pgfqpoint{3.070459in}{1.340527in}}%
\pgfpathlineto{\pgfqpoint{3.075000in}{1.340527in}}%
\pgfpathlineto{\pgfqpoint{3.075000in}{1.337577in}}%
\pgfpathmoveto{\pgfqpoint{3.070459in}{1.340527in}}%
\pgfpathlineto{\pgfqpoint{3.070459in}{1.340527in}}%
\pgfpathlineto{\pgfqpoint{3.070459in}{1.343476in}}%
\pgfpathlineto{\pgfqpoint{3.075000in}{1.343476in}}%
\pgfpathlineto{\pgfqpoint{3.075000in}{1.340527in}}%
\pgfpathmoveto{\pgfqpoint{3.070459in}{1.343476in}}%
\pgfpathlineto{\pgfqpoint{3.070459in}{1.343476in}}%
\pgfpathlineto{\pgfqpoint{3.070459in}{1.346425in}}%
\pgfpathlineto{\pgfqpoint{3.075000in}{1.346425in}}%
\pgfpathlineto{\pgfqpoint{3.075000in}{1.343476in}}%
\pgfpathmoveto{\pgfqpoint{3.070459in}{1.346425in}}%
\pgfpathlineto{\pgfqpoint{3.070459in}{1.346425in}}%
\pgfpathlineto{\pgfqpoint{3.070459in}{1.349374in}}%
\pgfpathlineto{\pgfqpoint{3.075000in}{1.349374in}}%
\pgfpathlineto{\pgfqpoint{3.075000in}{1.346425in}}%
\pgfpathmoveto{\pgfqpoint{3.070459in}{1.349374in}}%
\pgfpathlineto{\pgfqpoint{3.070459in}{1.349374in}}%
\pgfpathlineto{\pgfqpoint{3.070459in}{1.352323in}}%
\pgfpathlineto{\pgfqpoint{3.075000in}{1.352323in}}%
\pgfpathlineto{\pgfqpoint{3.075000in}{1.349374in}}%
\pgfpathmoveto{\pgfqpoint{3.070459in}{1.352323in}}%
\pgfpathlineto{\pgfqpoint{3.070459in}{1.352323in}}%
\pgfpathlineto{\pgfqpoint{3.070459in}{1.355272in}}%
\pgfpathlineto{\pgfqpoint{3.075000in}{1.355272in}}%
\pgfpathlineto{\pgfqpoint{3.075000in}{1.352323in}}%
\pgfpathmoveto{\pgfqpoint{3.070459in}{1.355272in}}%
\pgfpathlineto{\pgfqpoint{3.070459in}{1.355272in}}%
\pgfpathlineto{\pgfqpoint{3.070459in}{1.358222in}}%
\pgfpathlineto{\pgfqpoint{3.075000in}{1.358222in}}%
\pgfpathlineto{\pgfqpoint{3.075000in}{1.355272in}}%
\pgfpathmoveto{\pgfqpoint{3.070459in}{1.358222in}}%
\pgfpathlineto{\pgfqpoint{3.070459in}{1.358222in}}%
\pgfpathlineto{\pgfqpoint{3.070459in}{1.361171in}}%
\pgfpathlineto{\pgfqpoint{3.075000in}{1.361171in}}%
\pgfpathlineto{\pgfqpoint{3.075000in}{1.358222in}}%
\pgfpathmoveto{\pgfqpoint{3.070459in}{1.361171in}}%
\pgfpathlineto{\pgfqpoint{3.070459in}{1.361171in}}%
\pgfpathlineto{\pgfqpoint{3.070459in}{1.364120in}}%
\pgfpathlineto{\pgfqpoint{3.075000in}{1.364120in}}%
\pgfpathlineto{\pgfqpoint{3.075000in}{1.361171in}}%
\pgfpathmoveto{\pgfqpoint{3.070459in}{1.364120in}}%
\pgfpathlineto{\pgfqpoint{3.070459in}{1.364120in}}%
\pgfpathlineto{\pgfqpoint{3.070459in}{1.367069in}}%
\pgfpathlineto{\pgfqpoint{3.075000in}{1.367069in}}%
\pgfpathlineto{\pgfqpoint{3.075000in}{1.364120in}}%
\pgfpathmoveto{\pgfqpoint{3.070459in}{1.367069in}}%
\pgfpathlineto{\pgfqpoint{3.070459in}{1.367069in}}%
\pgfpathlineto{\pgfqpoint{3.070459in}{1.370019in}}%
\pgfpathlineto{\pgfqpoint{3.075000in}{1.370019in}}%
\pgfpathlineto{\pgfqpoint{3.075000in}{1.367069in}}%
\pgfpathmoveto{\pgfqpoint{3.070459in}{1.370019in}}%
\pgfpathlineto{\pgfqpoint{3.070459in}{1.370019in}}%
\pgfpathlineto{\pgfqpoint{3.070459in}{1.372968in}}%
\pgfpathlineto{\pgfqpoint{3.075000in}{1.372968in}}%
\pgfpathlineto{\pgfqpoint{3.075000in}{1.370019in}}%
\pgfpathmoveto{\pgfqpoint{3.070459in}{1.372968in}}%
\pgfpathlineto{\pgfqpoint{3.070459in}{1.372968in}}%
\pgfpathlineto{\pgfqpoint{3.070459in}{1.375917in}}%
\pgfpathlineto{\pgfqpoint{3.075000in}{1.375917in}}%
\pgfpathlineto{\pgfqpoint{3.075000in}{1.372968in}}%
\pgfpathmoveto{\pgfqpoint{3.070459in}{1.375917in}}%
\pgfpathlineto{\pgfqpoint{3.070459in}{1.375917in}}%
\pgfpathlineto{\pgfqpoint{3.070459in}{1.378866in}}%
\pgfpathlineto{\pgfqpoint{3.075000in}{1.378866in}}%
\pgfpathlineto{\pgfqpoint{3.075000in}{1.375917in}}%
\pgfpathmoveto{\pgfqpoint{3.070459in}{1.378866in}}%
\pgfpathlineto{\pgfqpoint{3.070459in}{1.378866in}}%
\pgfpathlineto{\pgfqpoint{3.070459in}{1.381816in}}%
\pgfpathlineto{\pgfqpoint{3.075000in}{1.381816in}}%
\pgfpathlineto{\pgfqpoint{3.075000in}{1.378866in}}%
\pgfpathmoveto{\pgfqpoint{3.070459in}{1.381816in}}%
\pgfpathlineto{\pgfqpoint{3.070459in}{1.381816in}}%
\pgfpathlineto{\pgfqpoint{3.070459in}{1.384765in}}%
\pgfpathlineto{\pgfqpoint{3.075000in}{1.384765in}}%
\pgfpathlineto{\pgfqpoint{3.075000in}{1.381816in}}%
\pgfpathmoveto{\pgfqpoint{3.070459in}{1.384765in}}%
\pgfpathlineto{\pgfqpoint{3.070459in}{1.384765in}}%
\pgfpathlineto{\pgfqpoint{3.070459in}{1.387714in}}%
\pgfpathlineto{\pgfqpoint{3.075000in}{1.387714in}}%
\pgfpathlineto{\pgfqpoint{3.075000in}{1.384765in}}%
\pgfpathmoveto{\pgfqpoint{3.070459in}{1.387714in}}%
\pgfpathlineto{\pgfqpoint{3.070459in}{1.387714in}}%
\pgfpathlineto{\pgfqpoint{3.070459in}{1.390663in}}%
\pgfpathlineto{\pgfqpoint{3.075000in}{1.390663in}}%
\pgfpathlineto{\pgfqpoint{3.075000in}{1.387714in}}%
\pgfpathmoveto{\pgfqpoint{3.070459in}{1.390663in}}%
\pgfpathlineto{\pgfqpoint{3.070459in}{1.390663in}}%
\pgfpathlineto{\pgfqpoint{3.070459in}{1.393613in}}%
\pgfpathlineto{\pgfqpoint{3.075000in}{1.393613in}}%
\pgfpathlineto{\pgfqpoint{3.075000in}{1.390663in}}%
\pgfpathmoveto{\pgfqpoint{3.070459in}{1.393613in}}%
\pgfpathlineto{\pgfqpoint{3.070459in}{1.393613in}}%
\pgfpathlineto{\pgfqpoint{3.070459in}{1.396562in}}%
\pgfpathlineto{\pgfqpoint{3.075000in}{1.396562in}}%
\pgfpathlineto{\pgfqpoint{3.075000in}{1.393613in}}%
\pgfpathmoveto{\pgfqpoint{3.070459in}{1.396562in}}%
\pgfpathlineto{\pgfqpoint{3.070459in}{1.396562in}}%
\pgfpathlineto{\pgfqpoint{3.070459in}{1.399511in}}%
\pgfpathlineto{\pgfqpoint{3.075000in}{1.399511in}}%
\pgfpathlineto{\pgfqpoint{3.075000in}{1.396562in}}%
\pgfpathmoveto{\pgfqpoint{3.070459in}{1.399511in}}%
\pgfpathlineto{\pgfqpoint{3.070459in}{1.399511in}}%
\pgfpathlineto{\pgfqpoint{3.070459in}{1.402460in}}%
\pgfpathlineto{\pgfqpoint{3.075000in}{1.402460in}}%
\pgfpathlineto{\pgfqpoint{3.075000in}{1.399511in}}%
\pgfpathmoveto{\pgfqpoint{3.070459in}{1.402460in}}%
\pgfpathlineto{\pgfqpoint{3.070459in}{1.402460in}}%
\pgfpathlineto{\pgfqpoint{3.070459in}{1.405410in}}%
\pgfpathlineto{\pgfqpoint{3.075000in}{1.405410in}}%
\pgfpathlineto{\pgfqpoint{3.075000in}{1.402460in}}%
\pgfpathmoveto{\pgfqpoint{3.070459in}{1.405410in}}%
\pgfpathlineto{\pgfqpoint{3.070459in}{1.405410in}}%
\pgfpathlineto{\pgfqpoint{3.070459in}{1.408359in}}%
\pgfpathlineto{\pgfqpoint{3.075000in}{1.408359in}}%
\pgfpathlineto{\pgfqpoint{3.075000in}{1.405410in}}%
\pgfpathmoveto{\pgfqpoint{3.070459in}{1.408359in}}%
\pgfpathlineto{\pgfqpoint{3.070459in}{1.408359in}}%
\pgfpathlineto{\pgfqpoint{3.070459in}{1.411308in}}%
\pgfpathlineto{\pgfqpoint{3.075000in}{1.411308in}}%
\pgfpathlineto{\pgfqpoint{3.075000in}{1.408359in}}%
\pgfpathmoveto{\pgfqpoint{3.070459in}{1.411308in}}%
\pgfpathlineto{\pgfqpoint{3.070459in}{1.411308in}}%
\pgfpathlineto{\pgfqpoint{3.070459in}{1.414257in}}%
\pgfpathlineto{\pgfqpoint{3.075000in}{1.414257in}}%
\pgfpathlineto{\pgfqpoint{3.075000in}{1.411308in}}%
\pgfpathmoveto{\pgfqpoint{3.070459in}{1.414257in}}%
\pgfpathlineto{\pgfqpoint{3.070459in}{1.414257in}}%
\pgfpathlineto{\pgfqpoint{3.070459in}{1.417206in}}%
\pgfpathlineto{\pgfqpoint{3.075000in}{1.417206in}}%
\pgfpathlineto{\pgfqpoint{3.075000in}{1.414257in}}%
\pgfpathmoveto{\pgfqpoint{3.070459in}{1.417206in}}%
\pgfpathlineto{\pgfqpoint{3.070459in}{1.417206in}}%
\pgfpathlineto{\pgfqpoint{3.070459in}{1.420156in}}%
\pgfpathlineto{\pgfqpoint{3.075000in}{1.420156in}}%
\pgfpathlineto{\pgfqpoint{3.075000in}{1.417206in}}%
\pgfpathmoveto{\pgfqpoint{3.070459in}{1.420156in}}%
\pgfpathlineto{\pgfqpoint{3.070459in}{1.420156in}}%
\pgfpathlineto{\pgfqpoint{3.070459in}{1.423105in}}%
\pgfpathlineto{\pgfqpoint{3.075000in}{1.423105in}}%
\pgfpathlineto{\pgfqpoint{3.075000in}{1.420156in}}%
\pgfpathmoveto{\pgfqpoint{3.070459in}{1.423105in}}%
\pgfpathlineto{\pgfqpoint{3.070459in}{1.423105in}}%
\pgfpathlineto{\pgfqpoint{3.070459in}{1.426054in}}%
\pgfpathlineto{\pgfqpoint{3.075000in}{1.426054in}}%
\pgfpathlineto{\pgfqpoint{3.075000in}{1.423105in}}%
\pgfpathmoveto{\pgfqpoint{3.070459in}{1.426054in}}%
\pgfpathlineto{\pgfqpoint{3.070459in}{1.426054in}}%
\pgfpathlineto{\pgfqpoint{3.070459in}{1.429003in}}%
\pgfpathlineto{\pgfqpoint{3.075000in}{1.429003in}}%
\pgfpathlineto{\pgfqpoint{3.075000in}{1.426054in}}%
\pgfpathmoveto{\pgfqpoint{3.070459in}{1.429003in}}%
\pgfpathlineto{\pgfqpoint{3.070459in}{1.429003in}}%
\pgfpathlineto{\pgfqpoint{3.070459in}{1.431953in}}%
\pgfpathlineto{\pgfqpoint{3.075000in}{1.431953in}}%
\pgfpathlineto{\pgfqpoint{3.075000in}{1.429003in}}%
\pgfpathmoveto{\pgfqpoint{3.070459in}{1.431953in}}%
\pgfpathlineto{\pgfqpoint{3.070459in}{1.431953in}}%
\pgfpathlineto{\pgfqpoint{3.070459in}{1.434902in}}%
\pgfpathlineto{\pgfqpoint{3.075000in}{1.434902in}}%
\pgfpathlineto{\pgfqpoint{3.075000in}{1.431953in}}%
\pgfpathmoveto{\pgfqpoint{3.070459in}{1.434902in}}%
\pgfpathlineto{\pgfqpoint{3.070459in}{1.434902in}}%
\pgfpathlineto{\pgfqpoint{3.070459in}{1.437851in}}%
\pgfpathlineto{\pgfqpoint{3.075000in}{1.437851in}}%
\pgfpathlineto{\pgfqpoint{3.075000in}{1.434902in}}%
\pgfpathmoveto{\pgfqpoint{3.070459in}{1.437851in}}%
\pgfpathlineto{\pgfqpoint{3.070459in}{1.437851in}}%
\pgfpathlineto{\pgfqpoint{3.070459in}{1.440800in}}%
\pgfpathlineto{\pgfqpoint{3.075000in}{1.440800in}}%
\pgfpathlineto{\pgfqpoint{3.075000in}{1.437851in}}%
\pgfpathmoveto{\pgfqpoint{3.070459in}{1.440800in}}%
\pgfpathlineto{\pgfqpoint{3.070459in}{1.440800in}}%
\pgfpathlineto{\pgfqpoint{3.070459in}{1.443750in}}%
\pgfpathlineto{\pgfqpoint{3.075000in}{1.443750in}}%
\pgfpathlineto{\pgfqpoint{3.075000in}{1.440800in}}%
\pgfpathmoveto{\pgfqpoint{3.070459in}{1.443750in}}%
\pgfpathlineto{\pgfqpoint{3.070459in}{1.443750in}}%
\pgfpathlineto{\pgfqpoint{3.070459in}{1.446699in}}%
\pgfpathlineto{\pgfqpoint{3.075000in}{1.446699in}}%
\pgfpathlineto{\pgfqpoint{3.075000in}{1.443750in}}%
\pgfpathmoveto{\pgfqpoint{3.070459in}{1.446699in}}%
\pgfpathlineto{\pgfqpoint{3.070459in}{1.446699in}}%
\pgfpathlineto{\pgfqpoint{3.070459in}{1.449648in}}%
\pgfpathlineto{\pgfqpoint{3.075000in}{1.449648in}}%
\pgfpathlineto{\pgfqpoint{3.075000in}{1.446699in}}%
\pgfpathmoveto{\pgfqpoint{3.070459in}{1.449648in}}%
\pgfpathlineto{\pgfqpoint{3.070459in}{1.449648in}}%
\pgfpathlineto{\pgfqpoint{3.070459in}{1.452597in}}%
\pgfpathlineto{\pgfqpoint{3.075000in}{1.452597in}}%
\pgfpathlineto{\pgfqpoint{3.075000in}{1.449648in}}%
\pgfpathmoveto{\pgfqpoint{3.070459in}{1.452597in}}%
\pgfpathlineto{\pgfqpoint{3.070459in}{1.452597in}}%
\pgfpathlineto{\pgfqpoint{3.070459in}{1.455547in}}%
\pgfpathlineto{\pgfqpoint{3.075000in}{1.455547in}}%
\pgfpathlineto{\pgfqpoint{3.075000in}{1.452597in}}%
\pgfpathmoveto{\pgfqpoint{3.070459in}{1.455547in}}%
\pgfpathlineto{\pgfqpoint{3.070459in}{1.455547in}}%
\pgfpathlineto{\pgfqpoint{3.070459in}{1.458496in}}%
\pgfpathlineto{\pgfqpoint{3.075000in}{1.458496in}}%
\pgfpathlineto{\pgfqpoint{3.075000in}{1.455547in}}%
\pgfpathmoveto{\pgfqpoint{3.070459in}{1.458496in}}%
\pgfpathlineto{\pgfqpoint{3.070459in}{1.458496in}}%
\pgfpathlineto{\pgfqpoint{3.070459in}{1.461445in}}%
\pgfpathlineto{\pgfqpoint{3.075000in}{1.461445in}}%
\pgfpathlineto{\pgfqpoint{3.075000in}{1.458496in}}%
\pgfpathmoveto{\pgfqpoint{3.070459in}{1.461445in}}%
\pgfpathlineto{\pgfqpoint{3.070459in}{1.461445in}}%
\pgfpathlineto{\pgfqpoint{3.070459in}{1.464394in}}%
\pgfpathlineto{\pgfqpoint{3.075000in}{1.464394in}}%
\pgfpathlineto{\pgfqpoint{3.075000in}{1.461445in}}%
\pgfpathmoveto{\pgfqpoint{3.070459in}{1.464394in}}%
\pgfpathlineto{\pgfqpoint{3.070459in}{1.464394in}}%
\pgfpathlineto{\pgfqpoint{3.070459in}{1.467344in}}%
\pgfpathlineto{\pgfqpoint{3.075000in}{1.467344in}}%
\pgfpathlineto{\pgfqpoint{3.075000in}{1.464394in}}%
\pgfpathmoveto{\pgfqpoint{3.070459in}{1.467344in}}%
\pgfpathlineto{\pgfqpoint{3.070459in}{1.467344in}}%
\pgfpathlineto{\pgfqpoint{3.070459in}{1.470293in}}%
\pgfpathlineto{\pgfqpoint{3.075000in}{1.470293in}}%
\pgfpathlineto{\pgfqpoint{3.075000in}{1.467344in}}%
\pgfpathmoveto{\pgfqpoint{3.070459in}{1.470293in}}%
\pgfpathlineto{\pgfqpoint{3.070459in}{1.470293in}}%
\pgfpathlineto{\pgfqpoint{3.070459in}{1.473242in}}%
\pgfpathlineto{\pgfqpoint{3.075000in}{1.473242in}}%
\pgfpathlineto{\pgfqpoint{3.075000in}{1.470293in}}%
\pgfpathmoveto{\pgfqpoint{3.070459in}{1.473242in}}%
\pgfpathlineto{\pgfqpoint{3.070459in}{1.473242in}}%
\pgfpathlineto{\pgfqpoint{3.070459in}{1.476191in}}%
\pgfpathlineto{\pgfqpoint{3.075000in}{1.476191in}}%
\pgfpathlineto{\pgfqpoint{3.075000in}{1.473242in}}%
\pgfpathmoveto{\pgfqpoint{3.070459in}{1.476191in}}%
\pgfpathlineto{\pgfqpoint{3.070459in}{1.476191in}}%
\pgfpathlineto{\pgfqpoint{3.070459in}{1.479140in}}%
\pgfpathlineto{\pgfqpoint{3.075000in}{1.479140in}}%
\pgfpathlineto{\pgfqpoint{3.075000in}{1.476191in}}%
\pgfpathmoveto{\pgfqpoint{3.070459in}{1.479140in}}%
\pgfpathlineto{\pgfqpoint{3.070459in}{1.479140in}}%
\pgfpathlineto{\pgfqpoint{3.070459in}{1.482090in}}%
\pgfpathlineto{\pgfqpoint{3.075000in}{1.482090in}}%
\pgfpathlineto{\pgfqpoint{3.075000in}{1.479140in}}%
\pgfpathmoveto{\pgfqpoint{3.070459in}{1.482090in}}%
\pgfpathlineto{\pgfqpoint{3.070459in}{1.482090in}}%
\pgfpathlineto{\pgfqpoint{3.070459in}{1.485039in}}%
\pgfpathlineto{\pgfqpoint{3.075000in}{1.485039in}}%
\pgfpathlineto{\pgfqpoint{3.075000in}{1.482090in}}%
\pgfpathmoveto{\pgfqpoint{3.070459in}{1.485039in}}%
\pgfpathlineto{\pgfqpoint{3.070459in}{1.485039in}}%
\pgfpathlineto{\pgfqpoint{3.070459in}{1.487988in}}%
\pgfpathlineto{\pgfqpoint{3.075000in}{1.487988in}}%
\pgfpathlineto{\pgfqpoint{3.075000in}{1.485039in}}%
\pgfpathmoveto{\pgfqpoint{3.070459in}{1.487988in}}%
\pgfpathlineto{\pgfqpoint{3.070459in}{1.487988in}}%
\pgfpathlineto{\pgfqpoint{3.070459in}{1.490937in}}%
\pgfpathlineto{\pgfqpoint{3.075000in}{1.490937in}}%
\pgfpathlineto{\pgfqpoint{3.075000in}{1.487988in}}%
\pgfpathmoveto{\pgfqpoint{3.070459in}{1.490937in}}%
\pgfpathlineto{\pgfqpoint{3.070459in}{1.490937in}}%
\pgfpathlineto{\pgfqpoint{3.070459in}{1.493887in}}%
\pgfpathlineto{\pgfqpoint{3.075000in}{1.493887in}}%
\pgfpathlineto{\pgfqpoint{3.075000in}{1.490937in}}%
\pgfpathmoveto{\pgfqpoint{3.070459in}{1.493887in}}%
\pgfpathlineto{\pgfqpoint{3.070459in}{1.493887in}}%
\pgfpathlineto{\pgfqpoint{3.070459in}{1.496836in}}%
\pgfpathlineto{\pgfqpoint{3.075000in}{1.496836in}}%
\pgfpathlineto{\pgfqpoint{3.075000in}{1.493887in}}%
\pgfpathmoveto{\pgfqpoint{3.070459in}{1.496836in}}%
\pgfpathlineto{\pgfqpoint{3.070459in}{1.496836in}}%
\pgfpathlineto{\pgfqpoint{3.070459in}{1.499785in}}%
\pgfpathlineto{\pgfqpoint{3.075000in}{1.499785in}}%
\pgfpathlineto{\pgfqpoint{3.075000in}{1.496836in}}%
\pgfpathmoveto{\pgfqpoint{3.070459in}{1.499785in}}%
\pgfpathlineto{\pgfqpoint{3.070459in}{1.499785in}}%
\pgfpathlineto{\pgfqpoint{3.070459in}{1.502734in}}%
\pgfpathlineto{\pgfqpoint{3.075000in}{1.502734in}}%
\pgfpathlineto{\pgfqpoint{3.075000in}{1.499785in}}%
\pgfpathmoveto{\pgfqpoint{3.070459in}{1.502734in}}%
\pgfpathlineto{\pgfqpoint{3.070459in}{1.502734in}}%
\pgfpathlineto{\pgfqpoint{3.070459in}{1.505684in}}%
\pgfpathlineto{\pgfqpoint{3.075000in}{1.505684in}}%
\pgfpathlineto{\pgfqpoint{3.075000in}{1.502734in}}%
\pgfpathmoveto{\pgfqpoint{3.070459in}{1.505684in}}%
\pgfpathlineto{\pgfqpoint{3.070459in}{1.505684in}}%
\pgfpathlineto{\pgfqpoint{3.070459in}{1.508633in}}%
\pgfpathlineto{\pgfqpoint{3.075000in}{1.508633in}}%
\pgfpathlineto{\pgfqpoint{3.075000in}{1.505684in}}%
\pgfpathmoveto{\pgfqpoint{3.070459in}{1.508633in}}%
\pgfpathlineto{\pgfqpoint{3.070459in}{1.508633in}}%
\pgfpathlineto{\pgfqpoint{3.070459in}{1.511582in}}%
\pgfpathlineto{\pgfqpoint{3.075000in}{1.511582in}}%
\pgfpathlineto{\pgfqpoint{3.075000in}{1.508633in}}%
\pgfpathmoveto{\pgfqpoint{3.070459in}{1.511582in}}%
\pgfpathlineto{\pgfqpoint{3.070459in}{1.511582in}}%
\pgfpathlineto{\pgfqpoint{3.070459in}{1.514531in}}%
\pgfpathlineto{\pgfqpoint{3.075000in}{1.514531in}}%
\pgfpathlineto{\pgfqpoint{3.075000in}{1.511582in}}%
\pgfpathmoveto{\pgfqpoint{3.070459in}{1.514531in}}%
\pgfpathlineto{\pgfqpoint{3.070459in}{1.514531in}}%
\pgfpathlineto{\pgfqpoint{3.070459in}{1.517480in}}%
\pgfpathlineto{\pgfqpoint{3.075000in}{1.517480in}}%
\pgfpathlineto{\pgfqpoint{3.075000in}{1.514531in}}%
\pgfpathmoveto{\pgfqpoint{3.070459in}{1.517480in}}%
\pgfpathlineto{\pgfqpoint{3.070459in}{1.517480in}}%
\pgfpathlineto{\pgfqpoint{3.070459in}{1.520430in}}%
\pgfpathlineto{\pgfqpoint{3.075000in}{1.520430in}}%
\pgfpathlineto{\pgfqpoint{3.075000in}{1.517480in}}%
\pgfpathmoveto{\pgfqpoint{3.070459in}{1.520430in}}%
\pgfpathlineto{\pgfqpoint{3.070459in}{1.520430in}}%
\pgfpathlineto{\pgfqpoint{3.070459in}{1.523379in}}%
\pgfpathlineto{\pgfqpoint{3.075000in}{1.523379in}}%
\pgfpathlineto{\pgfqpoint{3.075000in}{1.520430in}}%
\pgfpathmoveto{\pgfqpoint{3.070459in}{1.523379in}}%
\pgfpathlineto{\pgfqpoint{3.070459in}{1.523379in}}%
\pgfpathlineto{\pgfqpoint{3.070459in}{1.526328in}}%
\pgfpathlineto{\pgfqpoint{3.075000in}{1.526328in}}%
\pgfpathlineto{\pgfqpoint{3.075000in}{1.523379in}}%
\pgfpathmoveto{\pgfqpoint{3.070459in}{1.526328in}}%
\pgfpathlineto{\pgfqpoint{3.070459in}{1.526328in}}%
\pgfpathlineto{\pgfqpoint{3.070459in}{1.529277in}}%
\pgfpathlineto{\pgfqpoint{3.075000in}{1.529277in}}%
\pgfpathlineto{\pgfqpoint{3.075000in}{1.526328in}}%
\pgfpathmoveto{\pgfqpoint{3.070459in}{1.529277in}}%
\pgfpathlineto{\pgfqpoint{3.070459in}{1.529277in}}%
\pgfpathlineto{\pgfqpoint{3.070459in}{1.532227in}}%
\pgfpathlineto{\pgfqpoint{3.075000in}{1.532227in}}%
\pgfpathlineto{\pgfqpoint{3.075000in}{1.529277in}}%
\pgfpathmoveto{\pgfqpoint{3.070459in}{1.532227in}}%
\pgfpathlineto{\pgfqpoint{3.070459in}{1.532227in}}%
\pgfpathlineto{\pgfqpoint{3.070459in}{1.535176in}}%
\pgfpathlineto{\pgfqpoint{3.075000in}{1.535176in}}%
\pgfpathlineto{\pgfqpoint{3.075000in}{1.532227in}}%
\pgfpathmoveto{\pgfqpoint{3.070459in}{1.535176in}}%
\pgfpathlineto{\pgfqpoint{3.070459in}{1.535176in}}%
\pgfpathlineto{\pgfqpoint{3.070459in}{1.538125in}}%
\pgfpathlineto{\pgfqpoint{3.075000in}{1.538125in}}%
\pgfpathlineto{\pgfqpoint{3.075000in}{1.535176in}}%
\pgfpathmoveto{\pgfqpoint{3.070459in}{1.538125in}}%
\pgfpathlineto{\pgfqpoint{3.070459in}{1.538125in}}%
\pgfpathlineto{\pgfqpoint{3.070459in}{1.541074in}}%
\pgfpathlineto{\pgfqpoint{3.075000in}{1.541074in}}%
\pgfpathlineto{\pgfqpoint{3.075000in}{1.538125in}}%
\pgfpathmoveto{\pgfqpoint{3.070459in}{1.541074in}}%
\pgfpathlineto{\pgfqpoint{3.070459in}{1.541074in}}%
\pgfpathlineto{\pgfqpoint{3.070459in}{1.544024in}}%
\pgfpathlineto{\pgfqpoint{3.075000in}{1.544024in}}%
\pgfpathlineto{\pgfqpoint{3.075000in}{1.541074in}}%
\pgfpathmoveto{\pgfqpoint{3.070459in}{1.544024in}}%
\pgfpathlineto{\pgfqpoint{3.070459in}{1.544024in}}%
\pgfpathlineto{\pgfqpoint{3.070459in}{1.546973in}}%
\pgfpathlineto{\pgfqpoint{3.075000in}{1.546973in}}%
\pgfpathlineto{\pgfqpoint{3.075000in}{1.544024in}}%
\pgfpathmoveto{\pgfqpoint{3.070459in}{1.546973in}}%
\pgfpathlineto{\pgfqpoint{3.070459in}{1.546973in}}%
\pgfpathlineto{\pgfqpoint{3.070459in}{1.549922in}}%
\pgfpathlineto{\pgfqpoint{3.075000in}{1.549922in}}%
\pgfpathlineto{\pgfqpoint{3.075000in}{1.546973in}}%
\pgfpathmoveto{\pgfqpoint{3.070459in}{1.549922in}}%
\pgfpathlineto{\pgfqpoint{3.070459in}{1.549922in}}%
\pgfpathlineto{\pgfqpoint{3.070459in}{1.552871in}}%
\pgfpathlineto{\pgfqpoint{3.075000in}{1.552871in}}%
\pgfpathlineto{\pgfqpoint{3.075000in}{1.549922in}}%
\pgfpathmoveto{\pgfqpoint{3.070459in}{1.552871in}}%
\pgfpathlineto{\pgfqpoint{3.070459in}{1.552871in}}%
\pgfpathlineto{\pgfqpoint{3.070459in}{1.555821in}}%
\pgfpathlineto{\pgfqpoint{3.075000in}{1.555821in}}%
\pgfpathlineto{\pgfqpoint{3.075000in}{1.552871in}}%
\pgfpathmoveto{\pgfqpoint{3.070459in}{1.555821in}}%
\pgfpathlineto{\pgfqpoint{3.070459in}{1.555821in}}%
\pgfpathlineto{\pgfqpoint{3.070459in}{1.558770in}}%
\pgfpathlineto{\pgfqpoint{3.075000in}{1.558770in}}%
\pgfpathlineto{\pgfqpoint{3.075000in}{1.555821in}}%
\pgfpathmoveto{\pgfqpoint{3.070459in}{1.558770in}}%
\pgfpathlineto{\pgfqpoint{3.070459in}{1.558770in}}%
\pgfpathlineto{\pgfqpoint{3.070459in}{1.561719in}}%
\pgfpathlineto{\pgfqpoint{3.075000in}{1.561719in}}%
\pgfpathlineto{\pgfqpoint{3.075000in}{1.558770in}}%
\pgfpathmoveto{\pgfqpoint{3.070459in}{1.561719in}}%
\pgfpathlineto{\pgfqpoint{3.070459in}{1.561719in}}%
\pgfpathlineto{\pgfqpoint{3.070459in}{1.564668in}}%
\pgfpathlineto{\pgfqpoint{3.075000in}{1.564668in}}%
\pgfpathlineto{\pgfqpoint{3.075000in}{1.561719in}}%
\pgfpathmoveto{\pgfqpoint{3.070459in}{1.564668in}}%
\pgfpathlineto{\pgfqpoint{3.070459in}{1.564668in}}%
\pgfpathlineto{\pgfqpoint{3.070459in}{1.567618in}}%
\pgfpathlineto{\pgfqpoint{3.075000in}{1.567618in}}%
\pgfpathlineto{\pgfqpoint{3.075000in}{1.564668in}}%
\pgfpathmoveto{\pgfqpoint{3.070459in}{1.567618in}}%
\pgfpathlineto{\pgfqpoint{3.070459in}{1.567618in}}%
\pgfpathlineto{\pgfqpoint{3.070459in}{1.570567in}}%
\pgfpathlineto{\pgfqpoint{3.075000in}{1.570567in}}%
\pgfpathlineto{\pgfqpoint{3.075000in}{1.567618in}}%
\pgfpathmoveto{\pgfqpoint{3.070459in}{1.570567in}}%
\pgfpathlineto{\pgfqpoint{3.070459in}{1.570567in}}%
\pgfpathlineto{\pgfqpoint{3.070459in}{1.573516in}}%
\pgfpathlineto{\pgfqpoint{3.075000in}{1.573516in}}%
\pgfpathlineto{\pgfqpoint{3.075000in}{1.570567in}}%
\pgfpathmoveto{\pgfqpoint{3.070459in}{1.573516in}}%
\pgfpathlineto{\pgfqpoint{3.070459in}{1.573516in}}%
\pgfpathlineto{\pgfqpoint{3.070459in}{1.576465in}}%
\pgfpathlineto{\pgfqpoint{3.075000in}{1.576465in}}%
\pgfpathlineto{\pgfqpoint{3.075000in}{1.573516in}}%
\pgfpathmoveto{\pgfqpoint{3.070459in}{1.576465in}}%
\pgfpathlineto{\pgfqpoint{3.070459in}{1.576465in}}%
\pgfpathlineto{\pgfqpoint{3.070459in}{1.579415in}}%
\pgfpathlineto{\pgfqpoint{3.075000in}{1.579415in}}%
\pgfpathlineto{\pgfqpoint{3.075000in}{1.576465in}}%
\pgfpathmoveto{\pgfqpoint{3.070459in}{1.579415in}}%
\pgfpathlineto{\pgfqpoint{3.070459in}{1.579415in}}%
\pgfpathlineto{\pgfqpoint{3.070459in}{1.582364in}}%
\pgfpathlineto{\pgfqpoint{3.075000in}{1.582364in}}%
\pgfpathlineto{\pgfqpoint{3.075000in}{1.579415in}}%
\pgfpathmoveto{\pgfqpoint{3.070459in}{1.582364in}}%
\pgfpathlineto{\pgfqpoint{3.070459in}{1.582364in}}%
\pgfpathlineto{\pgfqpoint{3.070459in}{1.585313in}}%
\pgfpathlineto{\pgfqpoint{3.075000in}{1.585313in}}%
\pgfpathlineto{\pgfqpoint{3.075000in}{1.582364in}}%
\pgfpathmoveto{\pgfqpoint{3.070459in}{1.585313in}}%
\pgfpathlineto{\pgfqpoint{3.070459in}{1.585313in}}%
\pgfpathlineto{\pgfqpoint{3.070459in}{1.588262in}}%
\pgfpathlineto{\pgfqpoint{3.075000in}{1.588262in}}%
\pgfpathlineto{\pgfqpoint{3.075000in}{1.585313in}}%
\pgfpathmoveto{\pgfqpoint{3.070459in}{1.588262in}}%
\pgfpathlineto{\pgfqpoint{3.070459in}{1.588262in}}%
\pgfpathlineto{\pgfqpoint{3.070459in}{1.591212in}}%
\pgfpathlineto{\pgfqpoint{3.075000in}{1.591212in}}%
\pgfpathlineto{\pgfqpoint{3.075000in}{1.588262in}}%
\pgfpathmoveto{\pgfqpoint{3.070459in}{1.591212in}}%
\pgfpathlineto{\pgfqpoint{3.070459in}{1.591212in}}%
\pgfpathlineto{\pgfqpoint{3.070459in}{1.594161in}}%
\pgfpathlineto{\pgfqpoint{3.075000in}{1.594161in}}%
\pgfpathlineto{\pgfqpoint{3.075000in}{1.591212in}}%
\pgfpathmoveto{\pgfqpoint{3.070459in}{1.594161in}}%
\pgfpathlineto{\pgfqpoint{3.070459in}{1.594161in}}%
\pgfpathlineto{\pgfqpoint{3.070459in}{1.597110in}}%
\pgfpathlineto{\pgfqpoint{3.075000in}{1.597110in}}%
\pgfpathlineto{\pgfqpoint{3.075000in}{1.594161in}}%
\pgfpathmoveto{\pgfqpoint{3.070459in}{1.597110in}}%
\pgfpathlineto{\pgfqpoint{3.070459in}{1.597110in}}%
\pgfpathlineto{\pgfqpoint{3.070459in}{1.600060in}}%
\pgfpathlineto{\pgfqpoint{3.075000in}{1.600060in}}%
\pgfpathlineto{\pgfqpoint{3.075000in}{1.597110in}}%
\pgfpathmoveto{\pgfqpoint{3.070459in}{1.600060in}}%
\pgfpathlineto{\pgfqpoint{3.070459in}{1.600060in}}%
\pgfpathlineto{\pgfqpoint{3.070459in}{1.603009in}}%
\pgfpathlineto{\pgfqpoint{3.075000in}{1.603009in}}%
\pgfpathlineto{\pgfqpoint{3.075000in}{1.600060in}}%
\pgfpathmoveto{\pgfqpoint{3.070459in}{1.603009in}}%
\pgfpathlineto{\pgfqpoint{3.070459in}{1.603009in}}%
\pgfpathlineto{\pgfqpoint{3.070459in}{1.605958in}}%
\pgfpathlineto{\pgfqpoint{3.075000in}{1.605958in}}%
\pgfpathlineto{\pgfqpoint{3.075000in}{1.603009in}}%
\pgfpathmoveto{\pgfqpoint{3.070459in}{1.605958in}}%
\pgfpathlineto{\pgfqpoint{3.070459in}{1.605958in}}%
\pgfpathlineto{\pgfqpoint{3.070459in}{1.608907in}}%
\pgfpathlineto{\pgfqpoint{3.075000in}{1.608907in}}%
\pgfpathlineto{\pgfqpoint{3.075000in}{1.605958in}}%
\pgfpathmoveto{\pgfqpoint{3.070459in}{1.608907in}}%
\pgfpathlineto{\pgfqpoint{3.070459in}{1.608907in}}%
\pgfpathlineto{\pgfqpoint{3.070459in}{1.611857in}}%
\pgfpathlineto{\pgfqpoint{3.075000in}{1.611857in}}%
\pgfpathlineto{\pgfqpoint{3.075000in}{1.608907in}}%
\pgfpathmoveto{\pgfqpoint{3.070459in}{1.611857in}}%
\pgfpathlineto{\pgfqpoint{3.070459in}{1.611857in}}%
\pgfpathlineto{\pgfqpoint{3.070459in}{1.614806in}}%
\pgfpathlineto{\pgfqpoint{3.075000in}{1.614806in}}%
\pgfpathlineto{\pgfqpoint{3.075000in}{1.611857in}}%
\pgfpathmoveto{\pgfqpoint{3.070459in}{1.614806in}}%
\pgfpathlineto{\pgfqpoint{3.070459in}{1.614806in}}%
\pgfpathlineto{\pgfqpoint{3.070459in}{1.617755in}}%
\pgfpathlineto{\pgfqpoint{3.075000in}{1.617755in}}%
\pgfpathlineto{\pgfqpoint{3.075000in}{1.614806in}}%
\pgfpathmoveto{\pgfqpoint{3.070459in}{1.617755in}}%
\pgfpathlineto{\pgfqpoint{3.070459in}{1.617755in}}%
\pgfpathlineto{\pgfqpoint{3.070459in}{1.620704in}}%
\pgfpathlineto{\pgfqpoint{3.075000in}{1.620704in}}%
\pgfpathlineto{\pgfqpoint{3.075000in}{1.617755in}}%
\pgfpathmoveto{\pgfqpoint{3.070459in}{1.620704in}}%
\pgfpathlineto{\pgfqpoint{3.070459in}{1.620704in}}%
\pgfpathlineto{\pgfqpoint{3.070459in}{1.623654in}}%
\pgfpathlineto{\pgfqpoint{3.075000in}{1.623654in}}%
\pgfpathlineto{\pgfqpoint{3.075000in}{1.620704in}}%
\pgfpathmoveto{\pgfqpoint{3.070459in}{1.623654in}}%
\pgfpathlineto{\pgfqpoint{3.070459in}{1.623654in}}%
\pgfpathlineto{\pgfqpoint{3.070459in}{1.626603in}}%
\pgfpathlineto{\pgfqpoint{3.075000in}{1.626603in}}%
\pgfpathlineto{\pgfqpoint{3.075000in}{1.623654in}}%
\pgfpathmoveto{\pgfqpoint{3.070459in}{1.626603in}}%
\pgfpathlineto{\pgfqpoint{3.070459in}{1.626603in}}%
\pgfpathlineto{\pgfqpoint{3.070459in}{1.629552in}}%
\pgfpathlineto{\pgfqpoint{3.075000in}{1.629552in}}%
\pgfpathlineto{\pgfqpoint{3.075000in}{1.626603in}}%
\pgfpathmoveto{\pgfqpoint{3.070459in}{1.629552in}}%
\pgfpathlineto{\pgfqpoint{3.070459in}{1.629552in}}%
\pgfpathlineto{\pgfqpoint{3.070459in}{1.632501in}}%
\pgfpathlineto{\pgfqpoint{3.075000in}{1.632501in}}%
\pgfpathlineto{\pgfqpoint{3.075000in}{1.629552in}}%
\pgfpathmoveto{\pgfqpoint{3.070459in}{1.632501in}}%
\pgfpathlineto{\pgfqpoint{3.070459in}{1.632501in}}%
\pgfpathlineto{\pgfqpoint{3.070459in}{1.635451in}}%
\pgfpathlineto{\pgfqpoint{3.075000in}{1.635451in}}%
\pgfpathlineto{\pgfqpoint{3.075000in}{1.632501in}}%
\pgfpathmoveto{\pgfqpoint{3.070459in}{1.635451in}}%
\pgfpathlineto{\pgfqpoint{3.070459in}{1.635451in}}%
\pgfpathlineto{\pgfqpoint{3.070459in}{1.638400in}}%
\pgfpathlineto{\pgfqpoint{3.075000in}{1.638400in}}%
\pgfpathlineto{\pgfqpoint{3.075000in}{1.635451in}}%
\pgfpathmoveto{\pgfqpoint{3.070459in}{1.638400in}}%
\pgfpathlineto{\pgfqpoint{3.070459in}{1.638400in}}%
\pgfpathlineto{\pgfqpoint{3.070459in}{1.641349in}}%
\pgfpathlineto{\pgfqpoint{3.075000in}{1.641349in}}%
\pgfpathlineto{\pgfqpoint{3.075000in}{1.638400in}}%
\pgfpathmoveto{\pgfqpoint{3.070459in}{1.641349in}}%
\pgfpathlineto{\pgfqpoint{3.070459in}{1.641349in}}%
\pgfpathlineto{\pgfqpoint{3.070459in}{1.644298in}}%
\pgfpathlineto{\pgfqpoint{3.075000in}{1.644298in}}%
\pgfpathlineto{\pgfqpoint{3.075000in}{1.641349in}}%
\pgfpathmoveto{\pgfqpoint{3.070459in}{1.644298in}}%
\pgfpathlineto{\pgfqpoint{3.070459in}{1.644298in}}%
\pgfpathlineto{\pgfqpoint{3.070459in}{1.647247in}}%
\pgfpathlineto{\pgfqpoint{3.075000in}{1.647247in}}%
\pgfpathlineto{\pgfqpoint{3.075000in}{1.644298in}}%
\pgfpathmoveto{\pgfqpoint{3.070459in}{1.647247in}}%
\pgfpathlineto{\pgfqpoint{3.070459in}{1.647247in}}%
\pgfpathlineto{\pgfqpoint{3.070459in}{1.650197in}}%
\pgfpathlineto{\pgfqpoint{3.075000in}{1.650197in}}%
\pgfpathlineto{\pgfqpoint{3.075000in}{1.647247in}}%
\pgfpathmoveto{\pgfqpoint{3.070459in}{1.650197in}}%
\pgfpathlineto{\pgfqpoint{3.070459in}{1.650197in}}%
\pgfpathlineto{\pgfqpoint{3.070459in}{1.653146in}}%
\pgfpathlineto{\pgfqpoint{3.075000in}{1.653146in}}%
\pgfpathlineto{\pgfqpoint{3.075000in}{1.650197in}}%
\pgfpathmoveto{\pgfqpoint{3.070459in}{1.653146in}}%
\pgfpathlineto{\pgfqpoint{3.070459in}{1.653146in}}%
\pgfpathlineto{\pgfqpoint{3.070459in}{1.656095in}}%
\pgfpathlineto{\pgfqpoint{3.075000in}{1.656095in}}%
\pgfpathlineto{\pgfqpoint{3.075000in}{1.653146in}}%
\pgfpathmoveto{\pgfqpoint{3.070459in}{1.656095in}}%
\pgfpathlineto{\pgfqpoint{3.070459in}{1.656095in}}%
\pgfpathlineto{\pgfqpoint{3.070459in}{1.659044in}}%
\pgfpathlineto{\pgfqpoint{3.075000in}{1.659044in}}%
\pgfpathlineto{\pgfqpoint{3.075000in}{1.656095in}}%
\pgfpathmoveto{\pgfqpoint{3.070459in}{1.659044in}}%
\pgfpathlineto{\pgfqpoint{3.070459in}{1.659044in}}%
\pgfpathlineto{\pgfqpoint{3.070459in}{1.661994in}}%
\pgfpathlineto{\pgfqpoint{3.075000in}{1.661994in}}%
\pgfpathlineto{\pgfqpoint{3.075000in}{1.659044in}}%
\pgfpathmoveto{\pgfqpoint{3.070459in}{1.661994in}}%
\pgfpathlineto{\pgfqpoint{3.070459in}{1.661994in}}%
\pgfpathlineto{\pgfqpoint{3.070459in}{1.664943in}}%
\pgfpathlineto{\pgfqpoint{3.075000in}{1.664943in}}%
\pgfpathlineto{\pgfqpoint{3.075000in}{1.661994in}}%
\pgfpathmoveto{\pgfqpoint{3.070459in}{1.664943in}}%
\pgfpathlineto{\pgfqpoint{3.070459in}{1.664943in}}%
\pgfpathlineto{\pgfqpoint{3.070459in}{1.667892in}}%
\pgfpathlineto{\pgfqpoint{3.075000in}{1.667892in}}%
\pgfpathlineto{\pgfqpoint{3.075000in}{1.664943in}}%
\pgfpathmoveto{\pgfqpoint{3.070459in}{1.667892in}}%
\pgfpathlineto{\pgfqpoint{3.070459in}{1.667892in}}%
\pgfpathlineto{\pgfqpoint{3.070459in}{1.670841in}}%
\pgfpathlineto{\pgfqpoint{3.075000in}{1.670841in}}%
\pgfpathlineto{\pgfqpoint{3.075000in}{1.667892in}}%
\pgfpathmoveto{\pgfqpoint{3.070459in}{1.670841in}}%
\pgfpathlineto{\pgfqpoint{3.070459in}{1.670841in}}%
\pgfpathlineto{\pgfqpoint{3.070459in}{1.673790in}}%
\pgfpathlineto{\pgfqpoint{3.075000in}{1.673790in}}%
\pgfpathlineto{\pgfqpoint{3.075000in}{1.670841in}}%
\pgfpathmoveto{\pgfqpoint{3.070459in}{1.673790in}}%
\pgfpathlineto{\pgfqpoint{3.070459in}{1.673790in}}%
\pgfpathlineto{\pgfqpoint{3.070459in}{1.676740in}}%
\pgfpathlineto{\pgfqpoint{3.075000in}{1.676740in}}%
\pgfpathlineto{\pgfqpoint{3.075000in}{1.673790in}}%
\pgfpathmoveto{\pgfqpoint{3.070459in}{1.676740in}}%
\pgfpathlineto{\pgfqpoint{3.070459in}{1.676740in}}%
\pgfpathlineto{\pgfqpoint{3.070459in}{1.679689in}}%
\pgfpathlineto{\pgfqpoint{3.075000in}{1.679689in}}%
\pgfpathlineto{\pgfqpoint{3.075000in}{1.676740in}}%
\pgfpathmoveto{\pgfqpoint{3.070459in}{1.679689in}}%
\pgfpathlineto{\pgfqpoint{3.070459in}{1.679689in}}%
\pgfpathlineto{\pgfqpoint{3.070459in}{1.682638in}}%
\pgfpathlineto{\pgfqpoint{3.075000in}{1.682638in}}%
\pgfpathlineto{\pgfqpoint{3.075000in}{1.679689in}}%
\pgfpathmoveto{\pgfqpoint{3.070459in}{1.682638in}}%
\pgfpathlineto{\pgfqpoint{3.070459in}{1.682638in}}%
\pgfpathlineto{\pgfqpoint{3.070459in}{1.685587in}}%
\pgfpathlineto{\pgfqpoint{3.075000in}{1.685587in}}%
\pgfpathlineto{\pgfqpoint{3.075000in}{1.682638in}}%
\pgfpathmoveto{\pgfqpoint{3.070459in}{1.685587in}}%
\pgfpathlineto{\pgfqpoint{3.070459in}{1.685587in}}%
\pgfpathlineto{\pgfqpoint{3.070459in}{1.688536in}}%
\pgfpathlineto{\pgfqpoint{3.075000in}{1.688536in}}%
\pgfpathlineto{\pgfqpoint{3.075000in}{1.685587in}}%
\pgfpathmoveto{\pgfqpoint{3.070459in}{1.688536in}}%
\pgfpathlineto{\pgfqpoint{3.070459in}{1.688536in}}%
\pgfpathlineto{\pgfqpoint{3.070459in}{1.691486in}}%
\pgfpathlineto{\pgfqpoint{3.075000in}{1.691486in}}%
\pgfpathlineto{\pgfqpoint{3.075000in}{1.688536in}}%
\pgfpathmoveto{\pgfqpoint{3.070459in}{1.691486in}}%
\pgfpathlineto{\pgfqpoint{3.070459in}{1.691486in}}%
\pgfpathlineto{\pgfqpoint{3.070459in}{1.694435in}}%
\pgfpathlineto{\pgfqpoint{3.075000in}{1.694435in}}%
\pgfpathlineto{\pgfqpoint{3.075000in}{1.691486in}}%
\pgfpathmoveto{\pgfqpoint{3.070459in}{1.694435in}}%
\pgfpathlineto{\pgfqpoint{3.070459in}{1.694435in}}%
\pgfpathlineto{\pgfqpoint{3.070459in}{1.697384in}}%
\pgfpathlineto{\pgfqpoint{3.075000in}{1.697384in}}%
\pgfpathlineto{\pgfqpoint{3.075000in}{1.694435in}}%
\pgfpathmoveto{\pgfqpoint{3.070459in}{1.697384in}}%
\pgfpathlineto{\pgfqpoint{3.070459in}{1.697384in}}%
\pgfpathlineto{\pgfqpoint{3.070459in}{1.700333in}}%
\pgfpathlineto{\pgfqpoint{3.075000in}{1.700333in}}%
\pgfpathlineto{\pgfqpoint{3.075000in}{1.697384in}}%
\pgfpathmoveto{\pgfqpoint{3.070459in}{1.700333in}}%
\pgfpathlineto{\pgfqpoint{3.070459in}{1.700333in}}%
\pgfpathlineto{\pgfqpoint{3.070459in}{1.703283in}}%
\pgfpathlineto{\pgfqpoint{3.075000in}{1.703283in}}%
\pgfpathlineto{\pgfqpoint{3.075000in}{1.700333in}}%
\pgfpathmoveto{\pgfqpoint{3.070459in}{1.703283in}}%
\pgfpathlineto{\pgfqpoint{3.070459in}{1.703283in}}%
\pgfpathlineto{\pgfqpoint{3.070459in}{1.706232in}}%
\pgfpathlineto{\pgfqpoint{3.075000in}{1.706232in}}%
\pgfpathlineto{\pgfqpoint{3.075000in}{1.703283in}}%
\pgfpathmoveto{\pgfqpoint{3.070459in}{1.706232in}}%
\pgfpathlineto{\pgfqpoint{3.070459in}{1.706232in}}%
\pgfpathlineto{\pgfqpoint{3.070459in}{1.709181in}}%
\pgfpathlineto{\pgfqpoint{3.075000in}{1.709181in}}%
\pgfpathlineto{\pgfqpoint{3.075000in}{1.706232in}}%
\pgfpathmoveto{\pgfqpoint{3.070459in}{1.709181in}}%
\pgfpathlineto{\pgfqpoint{3.070459in}{1.709181in}}%
\pgfpathlineto{\pgfqpoint{3.070459in}{1.712130in}}%
\pgfpathlineto{\pgfqpoint{3.075000in}{1.712130in}}%
\pgfpathlineto{\pgfqpoint{3.075000in}{1.709181in}}%
\pgfpathmoveto{\pgfqpoint{3.070459in}{1.712130in}}%
\pgfpathlineto{\pgfqpoint{3.070459in}{1.712130in}}%
\pgfpathlineto{\pgfqpoint{3.070459in}{1.715079in}}%
\pgfpathlineto{\pgfqpoint{3.075000in}{1.715079in}}%
\pgfpathlineto{\pgfqpoint{3.075000in}{1.712130in}}%
\pgfpathmoveto{\pgfqpoint{3.070459in}{1.715079in}}%
\pgfpathlineto{\pgfqpoint{3.070459in}{1.715079in}}%
\pgfpathlineto{\pgfqpoint{3.070459in}{1.718029in}}%
\pgfpathlineto{\pgfqpoint{3.075000in}{1.718029in}}%
\pgfpathlineto{\pgfqpoint{3.075000in}{1.715079in}}%
\pgfpathmoveto{\pgfqpoint{3.070459in}{1.718029in}}%
\pgfpathlineto{\pgfqpoint{3.070459in}{1.718029in}}%
\pgfpathlineto{\pgfqpoint{3.070459in}{1.720978in}}%
\pgfpathlineto{\pgfqpoint{3.075000in}{1.720978in}}%
\pgfpathlineto{\pgfqpoint{3.075000in}{1.718029in}}%
\pgfpathmoveto{\pgfqpoint{3.070459in}{1.720978in}}%
\pgfpathlineto{\pgfqpoint{3.070459in}{1.720978in}}%
\pgfpathlineto{\pgfqpoint{3.070459in}{1.723927in}}%
\pgfpathlineto{\pgfqpoint{3.075000in}{1.723927in}}%
\pgfpathlineto{\pgfqpoint{3.075000in}{1.720978in}}%
\pgfpathmoveto{\pgfqpoint{3.070459in}{1.723927in}}%
\pgfpathlineto{\pgfqpoint{3.070459in}{1.723927in}}%
\pgfpathlineto{\pgfqpoint{3.070459in}{1.726876in}}%
\pgfpathlineto{\pgfqpoint{3.075000in}{1.726876in}}%
\pgfpathlineto{\pgfqpoint{3.075000in}{1.723927in}}%
\pgfpathmoveto{\pgfqpoint{3.070459in}{1.726876in}}%
\pgfpathlineto{\pgfqpoint{3.070459in}{1.726876in}}%
\pgfpathlineto{\pgfqpoint{3.070459in}{1.729825in}}%
\pgfpathlineto{\pgfqpoint{3.075000in}{1.729825in}}%
\pgfpathlineto{\pgfqpoint{3.075000in}{1.726876in}}%
\pgfpathmoveto{\pgfqpoint{3.070459in}{1.729825in}}%
\pgfpathlineto{\pgfqpoint{3.070459in}{1.729825in}}%
\pgfpathlineto{\pgfqpoint{3.070459in}{1.732775in}}%
\pgfpathlineto{\pgfqpoint{3.075000in}{1.732775in}}%
\pgfpathlineto{\pgfqpoint{3.075000in}{1.729825in}}%
\pgfpathmoveto{\pgfqpoint{3.070459in}{1.732775in}}%
\pgfpathlineto{\pgfqpoint{3.070459in}{1.732775in}}%
\pgfpathlineto{\pgfqpoint{3.070459in}{1.735724in}}%
\pgfpathlineto{\pgfqpoint{3.075000in}{1.735724in}}%
\pgfpathlineto{\pgfqpoint{3.075000in}{1.732775in}}%
\pgfpathmoveto{\pgfqpoint{3.070459in}{1.735724in}}%
\pgfpathlineto{\pgfqpoint{3.070459in}{1.735724in}}%
\pgfpathlineto{\pgfqpoint{3.070459in}{1.738673in}}%
\pgfpathlineto{\pgfqpoint{3.075000in}{1.738673in}}%
\pgfpathlineto{\pgfqpoint{3.075000in}{1.735724in}}%
\pgfpathmoveto{\pgfqpoint{3.070459in}{1.738673in}}%
\pgfpathlineto{\pgfqpoint{3.070459in}{1.738673in}}%
\pgfpathlineto{\pgfqpoint{3.070459in}{1.741622in}}%
\pgfpathlineto{\pgfqpoint{3.075000in}{1.741622in}}%
\pgfpathlineto{\pgfqpoint{3.075000in}{1.738673in}}%
\pgfpathmoveto{\pgfqpoint{3.070459in}{1.741622in}}%
\pgfpathlineto{\pgfqpoint{3.070459in}{1.741622in}}%
\pgfpathlineto{\pgfqpoint{3.070459in}{1.744572in}}%
\pgfpathlineto{\pgfqpoint{3.075000in}{1.744572in}}%
\pgfpathlineto{\pgfqpoint{3.075000in}{1.741622in}}%
\pgfpathmoveto{\pgfqpoint{3.070459in}{1.744572in}}%
\pgfpathlineto{\pgfqpoint{3.070459in}{1.744572in}}%
\pgfpathlineto{\pgfqpoint{3.070459in}{1.747521in}}%
\pgfpathlineto{\pgfqpoint{3.075000in}{1.747521in}}%
\pgfpathlineto{\pgfqpoint{3.075000in}{1.744572in}}%
\pgfpathmoveto{\pgfqpoint{3.070459in}{1.747521in}}%
\pgfpathlineto{\pgfqpoint{3.070459in}{1.747521in}}%
\pgfpathlineto{\pgfqpoint{3.070459in}{1.750470in}}%
\pgfpathlineto{\pgfqpoint{3.075000in}{1.750470in}}%
\pgfpathlineto{\pgfqpoint{3.075000in}{1.747521in}}%
\pgfpathmoveto{\pgfqpoint{3.070459in}{1.750470in}}%
\pgfpathlineto{\pgfqpoint{3.070459in}{1.750470in}}%
\pgfpathlineto{\pgfqpoint{3.070459in}{1.753419in}}%
\pgfpathlineto{\pgfqpoint{3.075000in}{1.753419in}}%
\pgfpathlineto{\pgfqpoint{3.075000in}{1.750470in}}%
\pgfpathmoveto{\pgfqpoint{3.070459in}{1.753419in}}%
\pgfpathlineto{\pgfqpoint{3.070459in}{1.753419in}}%
\pgfpathlineto{\pgfqpoint{3.070459in}{1.756369in}}%
\pgfpathlineto{\pgfqpoint{3.075000in}{1.756369in}}%
\pgfpathlineto{\pgfqpoint{3.075000in}{1.753419in}}%
\pgfpathmoveto{\pgfqpoint{3.070459in}{1.756369in}}%
\pgfpathlineto{\pgfqpoint{3.070459in}{1.756369in}}%
\pgfpathlineto{\pgfqpoint{3.070459in}{1.759318in}}%
\pgfpathlineto{\pgfqpoint{3.075000in}{1.759318in}}%
\pgfpathlineto{\pgfqpoint{3.075000in}{1.756369in}}%
\pgfpathmoveto{\pgfqpoint{3.070459in}{1.759318in}}%
\pgfpathlineto{\pgfqpoint{3.070459in}{1.759318in}}%
\pgfpathlineto{\pgfqpoint{3.070459in}{1.762267in}}%
\pgfpathlineto{\pgfqpoint{3.075000in}{1.762267in}}%
\pgfpathlineto{\pgfqpoint{3.075000in}{1.759318in}}%
\pgfpathmoveto{\pgfqpoint{3.070459in}{1.762267in}}%
\pgfpathlineto{\pgfqpoint{3.070459in}{1.762267in}}%
\pgfpathlineto{\pgfqpoint{3.070459in}{1.765216in}}%
\pgfpathlineto{\pgfqpoint{3.075000in}{1.765216in}}%
\pgfpathlineto{\pgfqpoint{3.075000in}{1.762267in}}%
\pgfpathmoveto{\pgfqpoint{3.070459in}{1.765216in}}%
\pgfpathlineto{\pgfqpoint{3.070459in}{1.765216in}}%
\pgfpathlineto{\pgfqpoint{3.070459in}{1.768165in}}%
\pgfpathlineto{\pgfqpoint{3.075000in}{1.768165in}}%
\pgfpathlineto{\pgfqpoint{3.075000in}{1.765216in}}%
\pgfpathmoveto{\pgfqpoint{3.070459in}{1.768165in}}%
\pgfpathlineto{\pgfqpoint{3.070459in}{1.768165in}}%
\pgfpathlineto{\pgfqpoint{3.070459in}{1.771115in}}%
\pgfpathlineto{\pgfqpoint{3.075000in}{1.771115in}}%
\pgfpathlineto{\pgfqpoint{3.075000in}{1.768165in}}%
\pgfpathmoveto{\pgfqpoint{3.070459in}{1.771115in}}%
\pgfpathlineto{\pgfqpoint{3.070459in}{1.771115in}}%
\pgfpathlineto{\pgfqpoint{3.070459in}{1.774064in}}%
\pgfpathlineto{\pgfqpoint{3.075000in}{1.774064in}}%
\pgfpathlineto{\pgfqpoint{3.075000in}{1.771115in}}%
\pgfpathmoveto{\pgfqpoint{3.070459in}{1.774064in}}%
\pgfpathlineto{\pgfqpoint{3.070459in}{1.774064in}}%
\pgfpathlineto{\pgfqpoint{3.070459in}{1.777013in}}%
\pgfpathlineto{\pgfqpoint{3.075000in}{1.777013in}}%
\pgfpathlineto{\pgfqpoint{3.075000in}{1.774064in}}%
\pgfpathmoveto{\pgfqpoint{3.070459in}{1.777013in}}%
\pgfpathlineto{\pgfqpoint{3.070459in}{1.777013in}}%
\pgfpathlineto{\pgfqpoint{3.070459in}{1.779962in}}%
\pgfpathlineto{\pgfqpoint{3.075000in}{1.779962in}}%
\pgfpathlineto{\pgfqpoint{3.075000in}{1.777013in}}%
\pgfpathmoveto{\pgfqpoint{3.070459in}{1.779962in}}%
\pgfpathlineto{\pgfqpoint{3.070459in}{1.779962in}}%
\pgfpathlineto{\pgfqpoint{3.070459in}{1.782912in}}%
\pgfpathlineto{\pgfqpoint{3.075000in}{1.782912in}}%
\pgfpathlineto{\pgfqpoint{3.075000in}{1.779962in}}%
\pgfpathmoveto{\pgfqpoint{3.070459in}{1.782912in}}%
\pgfpathlineto{\pgfqpoint{3.070459in}{1.782912in}}%
\pgfpathlineto{\pgfqpoint{3.070459in}{1.785861in}}%
\pgfpathlineto{\pgfqpoint{3.075000in}{1.785861in}}%
\pgfpathlineto{\pgfqpoint{3.075000in}{1.782912in}}%
\pgfpathmoveto{\pgfqpoint{3.070459in}{1.785861in}}%
\pgfpathlineto{\pgfqpoint{3.070459in}{1.785861in}}%
\pgfpathlineto{\pgfqpoint{3.070459in}{1.788810in}}%
\pgfpathlineto{\pgfqpoint{3.075000in}{1.788810in}}%
\pgfpathlineto{\pgfqpoint{3.075000in}{1.785861in}}%
\pgfpathmoveto{\pgfqpoint{3.070459in}{1.788810in}}%
\pgfpathlineto{\pgfqpoint{3.070459in}{1.788810in}}%
\pgfpathlineto{\pgfqpoint{3.070459in}{1.791759in}}%
\pgfpathlineto{\pgfqpoint{3.075000in}{1.791759in}}%
\pgfpathlineto{\pgfqpoint{3.075000in}{1.788810in}}%
\pgfpathmoveto{\pgfqpoint{3.070459in}{1.791759in}}%
\pgfpathlineto{\pgfqpoint{3.070459in}{1.791759in}}%
\pgfpathlineto{\pgfqpoint{3.070459in}{1.794709in}}%
\pgfpathlineto{\pgfqpoint{3.075000in}{1.794709in}}%
\pgfpathlineto{\pgfqpoint{3.075000in}{1.791759in}}%
\pgfpathmoveto{\pgfqpoint{3.070459in}{1.794709in}}%
\pgfpathlineto{\pgfqpoint{3.070459in}{1.794709in}}%
\pgfpathlineto{\pgfqpoint{3.070459in}{1.797658in}}%
\pgfpathlineto{\pgfqpoint{3.075000in}{1.797658in}}%
\pgfpathlineto{\pgfqpoint{3.075000in}{1.794709in}}%
\pgfpathmoveto{\pgfqpoint{3.070459in}{1.797658in}}%
\pgfpathlineto{\pgfqpoint{3.070459in}{1.797658in}}%
\pgfpathlineto{\pgfqpoint{3.070459in}{1.800607in}}%
\pgfpathlineto{\pgfqpoint{3.075000in}{1.800607in}}%
\pgfpathlineto{\pgfqpoint{3.075000in}{1.797658in}}%
\pgfpathmoveto{\pgfqpoint{3.070459in}{1.800607in}}%
\pgfpathlineto{\pgfqpoint{3.070459in}{1.800607in}}%
\pgfpathlineto{\pgfqpoint{3.070459in}{1.803556in}}%
\pgfpathlineto{\pgfqpoint{3.075000in}{1.803556in}}%
\pgfpathlineto{\pgfqpoint{3.075000in}{1.800607in}}%
\pgfpathmoveto{\pgfqpoint{3.070459in}{1.803556in}}%
\pgfpathlineto{\pgfqpoint{3.070459in}{1.803556in}}%
\pgfpathlineto{\pgfqpoint{3.070459in}{1.806505in}}%
\pgfpathlineto{\pgfqpoint{3.075000in}{1.806505in}}%
\pgfpathlineto{\pgfqpoint{3.075000in}{1.803556in}}%
\pgfpathmoveto{\pgfqpoint{3.070459in}{1.806505in}}%
\pgfpathlineto{\pgfqpoint{3.070459in}{1.806505in}}%
\pgfpathlineto{\pgfqpoint{3.070459in}{1.809455in}}%
\pgfpathlineto{\pgfqpoint{3.075000in}{1.809455in}}%
\pgfpathlineto{\pgfqpoint{3.075000in}{1.806505in}}%
\pgfpathmoveto{\pgfqpoint{3.070459in}{1.809455in}}%
\pgfpathlineto{\pgfqpoint{3.070459in}{1.809455in}}%
\pgfpathlineto{\pgfqpoint{3.070459in}{1.812404in}}%
\pgfpathlineto{\pgfqpoint{3.075000in}{1.812404in}}%
\pgfpathlineto{\pgfqpoint{3.075000in}{1.809455in}}%
\pgfpathmoveto{\pgfqpoint{3.070459in}{1.812404in}}%
\pgfpathlineto{\pgfqpoint{3.070459in}{1.812404in}}%
\pgfpathlineto{\pgfqpoint{3.070459in}{1.815353in}}%
\pgfpathlineto{\pgfqpoint{3.075000in}{1.815353in}}%
\pgfpathlineto{\pgfqpoint{3.075000in}{1.812404in}}%
\pgfpathmoveto{\pgfqpoint{3.070459in}{1.815353in}}%
\pgfpathlineto{\pgfqpoint{3.070459in}{1.815353in}}%
\pgfpathlineto{\pgfqpoint{3.070459in}{1.818302in}}%
\pgfpathlineto{\pgfqpoint{3.075000in}{1.818302in}}%
\pgfpathlineto{\pgfqpoint{3.075000in}{1.815353in}}%
\pgfpathmoveto{\pgfqpoint{3.070459in}{1.818302in}}%
\pgfpathlineto{\pgfqpoint{3.070459in}{1.818302in}}%
\pgfpathlineto{\pgfqpoint{3.070459in}{1.821252in}}%
\pgfpathlineto{\pgfqpoint{3.075000in}{1.821252in}}%
\pgfpathlineto{\pgfqpoint{3.075000in}{1.818302in}}%
\pgfpathmoveto{\pgfqpoint{3.070459in}{1.821252in}}%
\pgfpathlineto{\pgfqpoint{3.070459in}{1.821252in}}%
\pgfpathlineto{\pgfqpoint{3.070459in}{1.824201in}}%
\pgfpathlineto{\pgfqpoint{3.075000in}{1.824201in}}%
\pgfpathlineto{\pgfqpoint{3.075000in}{1.821252in}}%
\pgfpathmoveto{\pgfqpoint{3.070459in}{1.824201in}}%
\pgfpathlineto{\pgfqpoint{3.070459in}{1.824201in}}%
\pgfpathlineto{\pgfqpoint{3.070459in}{1.827150in}}%
\pgfpathlineto{\pgfqpoint{3.075000in}{1.827150in}}%
\pgfpathlineto{\pgfqpoint{3.075000in}{1.824201in}}%
\pgfpathmoveto{\pgfqpoint{3.070459in}{1.827150in}}%
\pgfpathlineto{\pgfqpoint{3.070459in}{1.827150in}}%
\pgfpathlineto{\pgfqpoint{3.070459in}{1.830099in}}%
\pgfpathlineto{\pgfqpoint{3.075000in}{1.830099in}}%
\pgfpathlineto{\pgfqpoint{3.075000in}{1.827150in}}%
\pgfpathmoveto{\pgfqpoint{3.070459in}{1.830099in}}%
\pgfpathlineto{\pgfqpoint{3.070459in}{1.830099in}}%
\pgfpathlineto{\pgfqpoint{3.070459in}{1.833048in}}%
\pgfpathlineto{\pgfqpoint{3.075000in}{1.833048in}}%
\pgfpathlineto{\pgfqpoint{3.075000in}{1.830099in}}%
\pgfpathmoveto{\pgfqpoint{3.070459in}{1.833048in}}%
\pgfpathlineto{\pgfqpoint{3.070459in}{1.833048in}}%
\pgfpathlineto{\pgfqpoint{3.070459in}{1.835997in}}%
\pgfpathlineto{\pgfqpoint{3.075000in}{1.835997in}}%
\pgfpathlineto{\pgfqpoint{3.075000in}{1.833048in}}%
\pgfpathmoveto{\pgfqpoint{3.070459in}{1.835997in}}%
\pgfpathlineto{\pgfqpoint{3.070459in}{1.835997in}}%
\pgfpathlineto{\pgfqpoint{3.070459in}{1.838946in}}%
\pgfpathlineto{\pgfqpoint{3.075000in}{1.838946in}}%
\pgfpathlineto{\pgfqpoint{3.075000in}{1.835997in}}%
\pgfpathmoveto{\pgfqpoint{3.070459in}{1.838946in}}%
\pgfpathlineto{\pgfqpoint{3.070459in}{1.838946in}}%
\pgfpathlineto{\pgfqpoint{3.070459in}{1.841895in}}%
\pgfpathlineto{\pgfqpoint{3.075000in}{1.841895in}}%
\pgfpathlineto{\pgfqpoint{3.075000in}{1.838946in}}%
\pgfpathmoveto{\pgfqpoint{3.070459in}{1.841895in}}%
\pgfpathlineto{\pgfqpoint{3.070459in}{1.841895in}}%
\pgfpathlineto{\pgfqpoint{3.070459in}{1.844844in}}%
\pgfpathlineto{\pgfqpoint{3.075000in}{1.844844in}}%
\pgfpathlineto{\pgfqpoint{3.075000in}{1.841895in}}%
\pgfpathmoveto{\pgfqpoint{3.070459in}{1.844844in}}%
\pgfpathlineto{\pgfqpoint{3.070459in}{1.844844in}}%
\pgfpathlineto{\pgfqpoint{3.070459in}{1.847793in}}%
\pgfpathlineto{\pgfqpoint{3.075000in}{1.847793in}}%
\pgfpathlineto{\pgfqpoint{3.075000in}{1.844844in}}%
\pgfpathmoveto{\pgfqpoint{3.070459in}{1.847793in}}%
\pgfpathlineto{\pgfqpoint{3.070459in}{1.847793in}}%
\pgfpathlineto{\pgfqpoint{3.070459in}{1.850743in}}%
\pgfpathlineto{\pgfqpoint{3.075000in}{1.850743in}}%
\pgfpathlineto{\pgfqpoint{3.075000in}{1.847793in}}%
\pgfpathmoveto{\pgfqpoint{3.070459in}{1.850743in}}%
\pgfpathlineto{\pgfqpoint{3.070459in}{1.850743in}}%
\pgfpathlineto{\pgfqpoint{3.070459in}{1.853692in}}%
\pgfpathlineto{\pgfqpoint{3.075000in}{1.853692in}}%
\pgfpathlineto{\pgfqpoint{3.075000in}{1.850743in}}%
\pgfpathmoveto{\pgfqpoint{3.070459in}{1.853692in}}%
\pgfpathlineto{\pgfqpoint{3.070459in}{1.853692in}}%
\pgfpathlineto{\pgfqpoint{3.070459in}{1.856641in}}%
\pgfpathlineto{\pgfqpoint{3.075000in}{1.856641in}}%
\pgfpathlineto{\pgfqpoint{3.075000in}{1.853692in}}%
\pgfpathmoveto{\pgfqpoint{3.070459in}{1.856641in}}%
\pgfpathlineto{\pgfqpoint{3.070459in}{1.856641in}}%
\pgfpathlineto{\pgfqpoint{3.070459in}{1.859590in}}%
\pgfpathlineto{\pgfqpoint{3.075000in}{1.859590in}}%
\pgfpathlineto{\pgfqpoint{3.075000in}{1.856641in}}%
\pgfpathmoveto{\pgfqpoint{3.070459in}{1.859590in}}%
\pgfpathlineto{\pgfqpoint{3.070459in}{1.859590in}}%
\pgfpathlineto{\pgfqpoint{3.070459in}{1.862539in}}%
\pgfpathlineto{\pgfqpoint{3.075000in}{1.862539in}}%
\pgfpathlineto{\pgfqpoint{3.075000in}{1.859590in}}%
\pgfpathmoveto{\pgfqpoint{3.070459in}{1.862539in}}%
\pgfpathlineto{\pgfqpoint{3.070459in}{1.862539in}}%
\pgfpathlineto{\pgfqpoint{3.070459in}{1.865488in}}%
\pgfpathlineto{\pgfqpoint{3.075000in}{1.865488in}}%
\pgfpathlineto{\pgfqpoint{3.075000in}{1.862539in}}%
\pgfpathmoveto{\pgfqpoint{3.070459in}{1.865488in}}%
\pgfpathlineto{\pgfqpoint{3.070459in}{1.865488in}}%
\pgfpathlineto{\pgfqpoint{3.070459in}{1.868437in}}%
\pgfpathlineto{\pgfqpoint{3.075000in}{1.868437in}}%
\pgfpathlineto{\pgfqpoint{3.075000in}{1.865488in}}%
\pgfpathmoveto{\pgfqpoint{3.070459in}{1.868437in}}%
\pgfpathlineto{\pgfqpoint{3.070459in}{1.868437in}}%
\pgfpathlineto{\pgfqpoint{3.070459in}{1.871386in}}%
\pgfpathlineto{\pgfqpoint{3.075000in}{1.871386in}}%
\pgfpathlineto{\pgfqpoint{3.075000in}{1.868437in}}%
\pgfpathmoveto{\pgfqpoint{3.070459in}{1.871386in}}%
\pgfpathlineto{\pgfqpoint{3.070459in}{1.871386in}}%
\pgfpathlineto{\pgfqpoint{3.070459in}{1.874335in}}%
\pgfpathlineto{\pgfqpoint{3.075000in}{1.874335in}}%
\pgfpathlineto{\pgfqpoint{3.075000in}{1.871386in}}%
\pgfpathmoveto{\pgfqpoint{3.070459in}{1.874335in}}%
\pgfpathlineto{\pgfqpoint{3.070459in}{1.874335in}}%
\pgfpathlineto{\pgfqpoint{3.070459in}{1.877284in}}%
\pgfpathlineto{\pgfqpoint{3.075000in}{1.877284in}}%
\pgfpathlineto{\pgfqpoint{3.075000in}{1.874335in}}%
\pgfpathmoveto{\pgfqpoint{3.070459in}{1.877284in}}%
\pgfpathlineto{\pgfqpoint{3.070459in}{1.877284in}}%
\pgfpathlineto{\pgfqpoint{3.070459in}{1.880234in}}%
\pgfpathlineto{\pgfqpoint{3.075000in}{1.880234in}}%
\pgfpathlineto{\pgfqpoint{3.075000in}{1.877284in}}%
\pgfpathmoveto{\pgfqpoint{3.070459in}{1.880234in}}%
\pgfpathlineto{\pgfqpoint{3.070459in}{1.880234in}}%
\pgfpathlineto{\pgfqpoint{3.070459in}{1.883183in}}%
\pgfpathlineto{\pgfqpoint{3.075000in}{1.883183in}}%
\pgfpathlineto{\pgfqpoint{3.075000in}{1.880234in}}%
\pgfpathmoveto{\pgfqpoint{3.070459in}{1.883183in}}%
\pgfpathlineto{\pgfqpoint{3.070459in}{1.883183in}}%
\pgfpathlineto{\pgfqpoint{3.070459in}{1.886132in}}%
\pgfpathlineto{\pgfqpoint{3.075000in}{1.886132in}}%
\pgfpathlineto{\pgfqpoint{3.075000in}{1.883183in}}%
\pgfpathmoveto{\pgfqpoint{3.070459in}{1.886132in}}%
\pgfpathlineto{\pgfqpoint{3.070459in}{1.886132in}}%
\pgfpathlineto{\pgfqpoint{3.070459in}{1.889081in}}%
\pgfpathlineto{\pgfqpoint{3.075000in}{1.889081in}}%
\pgfpathlineto{\pgfqpoint{3.075000in}{1.886132in}}%
\pgfpathmoveto{\pgfqpoint{3.070459in}{1.889081in}}%
\pgfpathlineto{\pgfqpoint{3.070459in}{1.889081in}}%
\pgfpathlineto{\pgfqpoint{3.070459in}{1.892030in}}%
\pgfpathlineto{\pgfqpoint{3.075000in}{1.892030in}}%
\pgfpathlineto{\pgfqpoint{3.075000in}{1.889081in}}%
\pgfpathmoveto{\pgfqpoint{3.070459in}{1.892030in}}%
\pgfpathlineto{\pgfqpoint{3.070459in}{1.892030in}}%
\pgfpathlineto{\pgfqpoint{3.070459in}{1.894979in}}%
\pgfpathlineto{\pgfqpoint{3.075000in}{1.894979in}}%
\pgfpathlineto{\pgfqpoint{3.075000in}{1.892030in}}%
\pgfpathmoveto{\pgfqpoint{3.070459in}{1.894979in}}%
\pgfpathlineto{\pgfqpoint{3.070459in}{1.894979in}}%
\pgfpathlineto{\pgfqpoint{3.070459in}{1.897928in}}%
\pgfpathlineto{\pgfqpoint{3.075000in}{1.897928in}}%
\pgfpathlineto{\pgfqpoint{3.075000in}{1.894979in}}%
\pgfpathmoveto{\pgfqpoint{3.070459in}{1.897928in}}%
\pgfpathlineto{\pgfqpoint{3.070459in}{1.897928in}}%
\pgfpathlineto{\pgfqpoint{3.070459in}{1.900877in}}%
\pgfpathlineto{\pgfqpoint{3.075000in}{1.900877in}}%
\pgfpathlineto{\pgfqpoint{3.075000in}{1.897928in}}%
\pgfpathmoveto{\pgfqpoint{3.070459in}{1.900877in}}%
\pgfpathlineto{\pgfqpoint{3.070459in}{1.900877in}}%
\pgfpathlineto{\pgfqpoint{3.070459in}{1.903826in}}%
\pgfpathlineto{\pgfqpoint{3.075000in}{1.903826in}}%
\pgfpathlineto{\pgfqpoint{3.075000in}{1.900877in}}%
\pgfpathmoveto{\pgfqpoint{3.070459in}{1.903826in}}%
\pgfpathlineto{\pgfqpoint{3.070459in}{1.903826in}}%
\pgfpathlineto{\pgfqpoint{3.070459in}{1.906775in}}%
\pgfpathlineto{\pgfqpoint{3.075000in}{1.906775in}}%
\pgfpathlineto{\pgfqpoint{3.075000in}{1.903826in}}%
\pgfpathmoveto{\pgfqpoint{3.070459in}{1.906775in}}%
\pgfpathlineto{\pgfqpoint{3.070459in}{1.906775in}}%
\pgfpathlineto{\pgfqpoint{3.070459in}{1.909724in}}%
\pgfpathlineto{\pgfqpoint{3.075000in}{1.909724in}}%
\pgfpathlineto{\pgfqpoint{3.075000in}{1.906775in}}%
\pgfpathmoveto{\pgfqpoint{3.070459in}{1.909724in}}%
\pgfpathlineto{\pgfqpoint{3.070459in}{1.909724in}}%
\pgfpathlineto{\pgfqpoint{3.070459in}{1.912674in}}%
\pgfpathlineto{\pgfqpoint{3.075000in}{1.912674in}}%
\pgfpathlineto{\pgfqpoint{3.075000in}{1.909724in}}%
\pgfpathmoveto{\pgfqpoint{3.070459in}{1.912674in}}%
\pgfpathlineto{\pgfqpoint{3.070459in}{1.912674in}}%
\pgfpathlineto{\pgfqpoint{3.070459in}{1.915623in}}%
\pgfpathlineto{\pgfqpoint{3.075000in}{1.915623in}}%
\pgfpathlineto{\pgfqpoint{3.075000in}{1.912674in}}%
\pgfpathmoveto{\pgfqpoint{3.070459in}{1.915623in}}%
\pgfpathlineto{\pgfqpoint{3.070459in}{1.915623in}}%
\pgfpathlineto{\pgfqpoint{3.070459in}{1.918572in}}%
\pgfpathlineto{\pgfqpoint{3.075000in}{1.918572in}}%
\pgfpathlineto{\pgfqpoint{3.075000in}{1.915623in}}%
\pgfpathmoveto{\pgfqpoint{3.070459in}{1.918572in}}%
\pgfpathlineto{\pgfqpoint{3.070459in}{1.918572in}}%
\pgfpathlineto{\pgfqpoint{3.070459in}{1.921521in}}%
\pgfpathlineto{\pgfqpoint{3.075000in}{1.921521in}}%
\pgfpathlineto{\pgfqpoint{3.075000in}{1.918572in}}%
\pgfpathmoveto{\pgfqpoint{3.070459in}{1.921521in}}%
\pgfpathlineto{\pgfqpoint{3.070459in}{1.921521in}}%
\pgfpathlineto{\pgfqpoint{3.070459in}{1.924470in}}%
\pgfpathlineto{\pgfqpoint{3.075000in}{1.924470in}}%
\pgfpathlineto{\pgfqpoint{3.075000in}{1.921521in}}%
\pgfpathmoveto{\pgfqpoint{3.070459in}{1.924470in}}%
\pgfpathlineto{\pgfqpoint{3.070459in}{1.924470in}}%
\pgfpathlineto{\pgfqpoint{3.070459in}{1.927420in}}%
\pgfpathlineto{\pgfqpoint{3.075000in}{1.927420in}}%
\pgfpathlineto{\pgfqpoint{3.075000in}{1.924470in}}%
\pgfpathmoveto{\pgfqpoint{3.070459in}{1.927420in}}%
\pgfpathlineto{\pgfqpoint{3.070459in}{1.927420in}}%
\pgfpathlineto{\pgfqpoint{3.070459in}{1.930369in}}%
\pgfpathlineto{\pgfqpoint{3.075000in}{1.930369in}}%
\pgfpathlineto{\pgfqpoint{3.075000in}{1.927420in}}%
\pgfpathmoveto{\pgfqpoint{3.070459in}{1.930369in}}%
\pgfpathlineto{\pgfqpoint{3.070459in}{1.930369in}}%
\pgfpathlineto{\pgfqpoint{3.070459in}{1.933318in}}%
\pgfpathlineto{\pgfqpoint{3.075000in}{1.933318in}}%
\pgfpathlineto{\pgfqpoint{3.075000in}{1.930369in}}%
\pgfpathmoveto{\pgfqpoint{3.070459in}{1.933318in}}%
\pgfpathlineto{\pgfqpoint{3.070459in}{1.933318in}}%
\pgfpathlineto{\pgfqpoint{3.070459in}{1.936267in}}%
\pgfpathlineto{\pgfqpoint{3.075000in}{1.936267in}}%
\pgfpathlineto{\pgfqpoint{3.075000in}{1.933318in}}%
\pgfpathmoveto{\pgfqpoint{3.070459in}{1.936267in}}%
\pgfpathlineto{\pgfqpoint{3.070459in}{1.936267in}}%
\pgfpathlineto{\pgfqpoint{3.070459in}{1.939217in}}%
\pgfpathlineto{\pgfqpoint{3.075000in}{1.939217in}}%
\pgfpathlineto{\pgfqpoint{3.075000in}{1.936267in}}%
\pgfpathmoveto{\pgfqpoint{3.070459in}{1.939217in}}%
\pgfpathlineto{\pgfqpoint{3.070459in}{1.939217in}}%
\pgfpathlineto{\pgfqpoint{3.070459in}{1.942166in}}%
\pgfpathlineto{\pgfqpoint{3.075000in}{1.942166in}}%
\pgfpathlineto{\pgfqpoint{3.075000in}{1.939217in}}%
\pgfpathmoveto{\pgfqpoint{3.070459in}{1.942166in}}%
\pgfpathlineto{\pgfqpoint{3.070459in}{1.942166in}}%
\pgfpathlineto{\pgfqpoint{3.070459in}{1.945115in}}%
\pgfpathlineto{\pgfqpoint{3.075000in}{1.945115in}}%
\pgfpathlineto{\pgfqpoint{3.075000in}{1.942166in}}%
\pgfpathmoveto{\pgfqpoint{3.070459in}{1.945115in}}%
\pgfpathlineto{\pgfqpoint{3.070459in}{1.945115in}}%
\pgfpathlineto{\pgfqpoint{3.070459in}{1.948064in}}%
\pgfpathlineto{\pgfqpoint{3.075000in}{1.948064in}}%
\pgfpathlineto{\pgfqpoint{3.075000in}{1.945115in}}%
\pgfpathmoveto{\pgfqpoint{3.070459in}{1.948064in}}%
\pgfpathlineto{\pgfqpoint{3.070459in}{1.948064in}}%
\pgfpathlineto{\pgfqpoint{3.070459in}{1.951014in}}%
\pgfpathlineto{\pgfqpoint{3.075000in}{1.951014in}}%
\pgfpathlineto{\pgfqpoint{3.075000in}{1.948064in}}%
\pgfpathmoveto{\pgfqpoint{3.070459in}{1.951014in}}%
\pgfpathlineto{\pgfqpoint{3.070459in}{1.951014in}}%
\pgfpathlineto{\pgfqpoint{3.070459in}{1.953963in}}%
\pgfpathlineto{\pgfqpoint{3.075000in}{1.953963in}}%
\pgfpathlineto{\pgfqpoint{3.075000in}{1.951014in}}%
\pgfpathmoveto{\pgfqpoint{3.070459in}{1.953963in}}%
\pgfpathlineto{\pgfqpoint{3.070459in}{1.953963in}}%
\pgfpathlineto{\pgfqpoint{3.070459in}{1.956912in}}%
\pgfpathlineto{\pgfqpoint{3.075000in}{1.956912in}}%
\pgfpathlineto{\pgfqpoint{3.075000in}{1.953963in}}%
\pgfpathmoveto{\pgfqpoint{3.070459in}{1.956912in}}%
\pgfpathlineto{\pgfqpoint{3.070459in}{1.956912in}}%
\pgfpathlineto{\pgfqpoint{3.070459in}{1.959861in}}%
\pgfpathlineto{\pgfqpoint{3.075000in}{1.959861in}}%
\pgfpathlineto{\pgfqpoint{3.075000in}{1.956912in}}%
\pgfpathmoveto{\pgfqpoint{3.070459in}{1.959861in}}%
\pgfpathlineto{\pgfqpoint{3.070459in}{1.959861in}}%
\pgfpathlineto{\pgfqpoint{3.070459in}{1.962811in}}%
\pgfpathlineto{\pgfqpoint{3.075000in}{1.962811in}}%
\pgfpathlineto{\pgfqpoint{3.075000in}{1.959861in}}%
\pgfpathmoveto{\pgfqpoint{3.070459in}{1.962811in}}%
\pgfpathlineto{\pgfqpoint{3.070459in}{1.962811in}}%
\pgfpathlineto{\pgfqpoint{3.070459in}{1.965760in}}%
\pgfpathlineto{\pgfqpoint{3.075000in}{1.965760in}}%
\pgfpathlineto{\pgfqpoint{3.075000in}{1.962811in}}%
\pgfpathmoveto{\pgfqpoint{3.070459in}{1.965760in}}%
\pgfpathlineto{\pgfqpoint{3.070459in}{1.965760in}}%
\pgfpathlineto{\pgfqpoint{3.070459in}{1.968709in}}%
\pgfpathlineto{\pgfqpoint{3.075000in}{1.968709in}}%
\pgfpathlineto{\pgfqpoint{3.075000in}{1.965760in}}%
\pgfpathmoveto{\pgfqpoint{3.070459in}{1.968709in}}%
\pgfpathlineto{\pgfqpoint{3.070459in}{1.968709in}}%
\pgfpathlineto{\pgfqpoint{3.070459in}{1.971658in}}%
\pgfpathlineto{\pgfqpoint{3.075000in}{1.971658in}}%
\pgfpathlineto{\pgfqpoint{3.075000in}{1.968709in}}%
\pgfpathmoveto{\pgfqpoint{3.070459in}{1.971658in}}%
\pgfpathlineto{\pgfqpoint{3.070459in}{1.971658in}}%
\pgfpathlineto{\pgfqpoint{3.070459in}{1.974608in}}%
\pgfpathlineto{\pgfqpoint{3.075000in}{1.974608in}}%
\pgfpathlineto{\pgfqpoint{3.075000in}{1.971658in}}%
\pgfpathmoveto{\pgfqpoint{3.070459in}{1.974608in}}%
\pgfpathlineto{\pgfqpoint{3.070459in}{1.974608in}}%
\pgfpathlineto{\pgfqpoint{3.070459in}{1.977557in}}%
\pgfpathlineto{\pgfqpoint{3.075000in}{1.977557in}}%
\pgfpathlineto{\pgfqpoint{3.075000in}{1.974608in}}%
\pgfpathmoveto{\pgfqpoint{3.070459in}{1.977557in}}%
\pgfpathlineto{\pgfqpoint{3.070459in}{1.977557in}}%
\pgfpathlineto{\pgfqpoint{3.070459in}{1.980506in}}%
\pgfpathlineto{\pgfqpoint{3.075000in}{1.980506in}}%
\pgfpathlineto{\pgfqpoint{3.075000in}{1.977557in}}%
\pgfpathmoveto{\pgfqpoint{3.070459in}{1.980506in}}%
\pgfpathlineto{\pgfqpoint{3.070459in}{1.980506in}}%
\pgfpathlineto{\pgfqpoint{3.070459in}{1.983455in}}%
\pgfpathlineto{\pgfqpoint{3.075000in}{1.983455in}}%
\pgfpathlineto{\pgfqpoint{3.075000in}{1.980506in}}%
\pgfpathmoveto{\pgfqpoint{3.070459in}{1.983455in}}%
\pgfpathlineto{\pgfqpoint{3.070459in}{1.983455in}}%
\pgfpathlineto{\pgfqpoint{3.070459in}{1.986405in}}%
\pgfpathlineto{\pgfqpoint{3.075000in}{1.986405in}}%
\pgfpathlineto{\pgfqpoint{3.075000in}{1.983455in}}%
\pgfpathmoveto{\pgfqpoint{3.070459in}{1.986405in}}%
\pgfpathlineto{\pgfqpoint{3.070459in}{1.986405in}}%
\pgfpathlineto{\pgfqpoint{3.070459in}{1.989354in}}%
\pgfpathlineto{\pgfqpoint{3.075000in}{1.989354in}}%
\pgfpathlineto{\pgfqpoint{3.075000in}{1.986405in}}%
\pgfpathmoveto{\pgfqpoint{3.070459in}{1.989354in}}%
\pgfpathlineto{\pgfqpoint{3.070459in}{1.989354in}}%
\pgfpathlineto{\pgfqpoint{3.070459in}{1.992303in}}%
\pgfpathlineto{\pgfqpoint{3.075000in}{1.992303in}}%
\pgfpathlineto{\pgfqpoint{3.075000in}{1.989354in}}%
\pgfpathmoveto{\pgfqpoint{3.070459in}{1.992303in}}%
\pgfpathlineto{\pgfqpoint{3.070459in}{1.992303in}}%
\pgfpathlineto{\pgfqpoint{3.070459in}{1.995252in}}%
\pgfpathlineto{\pgfqpoint{3.075000in}{1.995252in}}%
\pgfpathlineto{\pgfqpoint{3.075000in}{1.992303in}}%
\pgfpathmoveto{\pgfqpoint{3.070459in}{1.995252in}}%
\pgfpathlineto{\pgfqpoint{3.070459in}{1.995252in}}%
\pgfpathlineto{\pgfqpoint{3.070459in}{1.998202in}}%
\pgfpathlineto{\pgfqpoint{3.075000in}{1.998202in}}%
\pgfpathlineto{\pgfqpoint{3.075000in}{1.995252in}}%
\pgfpathmoveto{\pgfqpoint{3.070459in}{1.998202in}}%
\pgfpathlineto{\pgfqpoint{3.070459in}{1.998202in}}%
\pgfpathlineto{\pgfqpoint{3.070459in}{2.001151in}}%
\pgfpathlineto{\pgfqpoint{3.075000in}{2.001151in}}%
\pgfpathlineto{\pgfqpoint{3.075000in}{1.998202in}}%
\pgfpathmoveto{\pgfqpoint{3.070459in}{2.001151in}}%
\pgfpathlineto{\pgfqpoint{3.070459in}{2.001151in}}%
\pgfpathlineto{\pgfqpoint{3.070459in}{2.004100in}}%
\pgfpathlineto{\pgfqpoint{3.075000in}{2.004100in}}%
\pgfpathlineto{\pgfqpoint{3.075000in}{2.001151in}}%
\pgfpathmoveto{\pgfqpoint{3.070459in}{2.004100in}}%
\pgfpathlineto{\pgfqpoint{3.070459in}{2.004100in}}%
\pgfpathlineto{\pgfqpoint{3.070459in}{2.007049in}}%
\pgfpathlineto{\pgfqpoint{3.075000in}{2.007049in}}%
\pgfpathlineto{\pgfqpoint{3.075000in}{2.004100in}}%
\pgfpathmoveto{\pgfqpoint{3.070459in}{2.007049in}}%
\pgfpathlineto{\pgfqpoint{3.070459in}{2.007049in}}%
\pgfpathlineto{\pgfqpoint{3.070459in}{2.009999in}}%
\pgfpathlineto{\pgfqpoint{3.075000in}{2.009999in}}%
\pgfpathlineto{\pgfqpoint{3.075000in}{2.007049in}}%
\pgfpathmoveto{\pgfqpoint{3.070459in}{2.009999in}}%
\pgfpathlineto{\pgfqpoint{3.070459in}{2.009999in}}%
\pgfpathlineto{\pgfqpoint{3.070459in}{2.012948in}}%
\pgfpathlineto{\pgfqpoint{3.075000in}{2.012948in}}%
\pgfpathlineto{\pgfqpoint{3.075000in}{2.009999in}}%
\pgfpathmoveto{\pgfqpoint{3.070459in}{2.012948in}}%
\pgfpathlineto{\pgfqpoint{3.070459in}{2.012948in}}%
\pgfpathlineto{\pgfqpoint{3.070459in}{2.015897in}}%
\pgfpathlineto{\pgfqpoint{3.075000in}{2.015897in}}%
\pgfpathlineto{\pgfqpoint{3.075000in}{2.012948in}}%
\pgfpathmoveto{\pgfqpoint{3.070459in}{2.015897in}}%
\pgfpathlineto{\pgfqpoint{3.070459in}{2.015897in}}%
\pgfpathlineto{\pgfqpoint{3.070459in}{2.018846in}}%
\pgfpathlineto{\pgfqpoint{3.075000in}{2.018846in}}%
\pgfpathlineto{\pgfqpoint{3.075000in}{2.015897in}}%
\pgfpathmoveto{\pgfqpoint{3.070459in}{2.018846in}}%
\pgfpathlineto{\pgfqpoint{3.070459in}{2.018846in}}%
\pgfpathlineto{\pgfqpoint{3.070459in}{2.021795in}}%
\pgfpathlineto{\pgfqpoint{3.075000in}{2.021795in}}%
\pgfpathlineto{\pgfqpoint{3.075000in}{2.018846in}}%
\pgfpathmoveto{\pgfqpoint{3.070459in}{2.021795in}}%
\pgfpathlineto{\pgfqpoint{3.070459in}{2.021795in}}%
\pgfpathlineto{\pgfqpoint{3.070459in}{2.024745in}}%
\pgfpathlineto{\pgfqpoint{3.075000in}{2.024745in}}%
\pgfpathlineto{\pgfqpoint{3.075000in}{2.021795in}}%
\pgfpathmoveto{\pgfqpoint{3.070459in}{2.024745in}}%
\pgfpathlineto{\pgfqpoint{3.070459in}{2.024745in}}%
\pgfpathlineto{\pgfqpoint{3.070459in}{2.027694in}}%
\pgfpathlineto{\pgfqpoint{3.075000in}{2.027694in}}%
\pgfpathlineto{\pgfqpoint{3.075000in}{2.024745in}}%
\pgfpathmoveto{\pgfqpoint{3.070459in}{2.027694in}}%
\pgfpathlineto{\pgfqpoint{3.070459in}{2.027694in}}%
\pgfpathlineto{\pgfqpoint{3.070459in}{2.030643in}}%
\pgfpathlineto{\pgfqpoint{3.075000in}{2.030643in}}%
\pgfpathlineto{\pgfqpoint{3.075000in}{2.027694in}}%
\pgfpathmoveto{\pgfqpoint{3.070459in}{2.030643in}}%
\pgfpathlineto{\pgfqpoint{3.070459in}{2.030643in}}%
\pgfpathlineto{\pgfqpoint{3.070459in}{2.033592in}}%
\pgfpathlineto{\pgfqpoint{3.075000in}{2.033592in}}%
\pgfpathlineto{\pgfqpoint{3.075000in}{2.030643in}}%
\pgfpathmoveto{\pgfqpoint{3.070459in}{2.033592in}}%
\pgfpathlineto{\pgfqpoint{3.070459in}{2.033592in}}%
\pgfpathlineto{\pgfqpoint{3.070459in}{2.036542in}}%
\pgfpathlineto{\pgfqpoint{3.075000in}{2.036542in}}%
\pgfpathlineto{\pgfqpoint{3.075000in}{2.033592in}}%
\pgfpathmoveto{\pgfqpoint{3.070459in}{2.036542in}}%
\pgfpathlineto{\pgfqpoint{3.070459in}{2.036542in}}%
\pgfpathlineto{\pgfqpoint{3.070459in}{2.039491in}}%
\pgfpathlineto{\pgfqpoint{3.075000in}{2.039491in}}%
\pgfpathlineto{\pgfqpoint{3.075000in}{2.036542in}}%
\pgfpathmoveto{\pgfqpoint{3.070459in}{2.039491in}}%
\pgfpathlineto{\pgfqpoint{3.070459in}{2.039491in}}%
\pgfpathlineto{\pgfqpoint{3.070459in}{2.042440in}}%
\pgfpathlineto{\pgfqpoint{3.075000in}{2.042440in}}%
\pgfpathlineto{\pgfqpoint{3.075000in}{2.039491in}}%
\pgfpathmoveto{\pgfqpoint{3.070459in}{2.042440in}}%
\pgfpathlineto{\pgfqpoint{3.070459in}{2.042440in}}%
\pgfpathlineto{\pgfqpoint{3.070459in}{2.045389in}}%
\pgfpathlineto{\pgfqpoint{3.075000in}{2.045389in}}%
\pgfpathlineto{\pgfqpoint{3.075000in}{2.042440in}}%
\pgfpathmoveto{\pgfqpoint{3.070459in}{2.045389in}}%
\pgfpathlineto{\pgfqpoint{3.070459in}{2.045389in}}%
\pgfpathlineto{\pgfqpoint{3.070459in}{2.048338in}}%
\pgfpathlineto{\pgfqpoint{3.075000in}{2.048338in}}%
\pgfpathlineto{\pgfqpoint{3.075000in}{2.045389in}}%
\pgfpathmoveto{\pgfqpoint{3.070459in}{2.048338in}}%
\pgfpathlineto{\pgfqpoint{3.070459in}{2.048338in}}%
\pgfpathlineto{\pgfqpoint{3.070459in}{2.051288in}}%
\pgfpathlineto{\pgfqpoint{3.075000in}{2.051288in}}%
\pgfpathlineto{\pgfqpoint{3.075000in}{2.048338in}}%
\pgfpathmoveto{\pgfqpoint{3.070459in}{2.051288in}}%
\pgfpathlineto{\pgfqpoint{3.070459in}{2.051288in}}%
\pgfpathlineto{\pgfqpoint{3.070459in}{2.054237in}}%
\pgfpathlineto{\pgfqpoint{3.075000in}{2.054237in}}%
\pgfpathlineto{\pgfqpoint{3.075000in}{2.051288in}}%
\pgfpathmoveto{\pgfqpoint{3.070459in}{2.054237in}}%
\pgfpathlineto{\pgfqpoint{3.070459in}{2.054237in}}%
\pgfpathlineto{\pgfqpoint{3.070459in}{2.057186in}}%
\pgfpathlineto{\pgfqpoint{3.075000in}{2.057186in}}%
\pgfpathlineto{\pgfqpoint{3.075000in}{2.054237in}}%
\pgfpathmoveto{\pgfqpoint{3.070459in}{2.057186in}}%
\pgfpathlineto{\pgfqpoint{3.070459in}{2.057186in}}%
\pgfpathlineto{\pgfqpoint{3.070459in}{2.060135in}}%
\pgfpathlineto{\pgfqpoint{3.075000in}{2.060135in}}%
\pgfpathlineto{\pgfqpoint{3.075000in}{2.057186in}}%
\pgfpathmoveto{\pgfqpoint{3.070459in}{2.060135in}}%
\pgfpathlineto{\pgfqpoint{3.070459in}{2.060135in}}%
\pgfpathlineto{\pgfqpoint{3.070459in}{2.063085in}}%
\pgfpathlineto{\pgfqpoint{3.075000in}{2.063085in}}%
\pgfpathlineto{\pgfqpoint{3.075000in}{2.060135in}}%
\pgfpathmoveto{\pgfqpoint{3.070459in}{2.063085in}}%
\pgfpathlineto{\pgfqpoint{3.070459in}{2.063085in}}%
\pgfpathlineto{\pgfqpoint{3.070459in}{2.066034in}}%
\pgfpathlineto{\pgfqpoint{3.075000in}{2.066034in}}%
\pgfpathlineto{\pgfqpoint{3.075000in}{2.063085in}}%
\pgfpathmoveto{\pgfqpoint{3.070459in}{2.066034in}}%
\pgfpathlineto{\pgfqpoint{3.070459in}{2.066034in}}%
\pgfpathlineto{\pgfqpoint{3.070459in}{2.068983in}}%
\pgfpathlineto{\pgfqpoint{3.075000in}{2.068983in}}%
\pgfpathlineto{\pgfqpoint{3.075000in}{2.066034in}}%
\pgfpathmoveto{\pgfqpoint{3.070459in}{2.068983in}}%
\pgfpathlineto{\pgfqpoint{3.070459in}{2.068983in}}%
\pgfpathlineto{\pgfqpoint{3.070459in}{2.071932in}}%
\pgfpathlineto{\pgfqpoint{3.075000in}{2.071932in}}%
\pgfpathlineto{\pgfqpoint{3.075000in}{2.068983in}}%
\pgfpathmoveto{\pgfqpoint{3.070459in}{2.071932in}}%
\pgfpathlineto{\pgfqpoint{3.070459in}{2.071932in}}%
\pgfpathlineto{\pgfqpoint{3.070459in}{2.074882in}}%
\pgfpathlineto{\pgfqpoint{3.075000in}{2.074882in}}%
\pgfpathlineto{\pgfqpoint{3.075000in}{2.071932in}}%
\pgfpathmoveto{\pgfqpoint{3.070459in}{2.074882in}}%
\pgfpathlineto{\pgfqpoint{3.070459in}{2.074882in}}%
\pgfpathlineto{\pgfqpoint{3.070459in}{2.077831in}}%
\pgfpathlineto{\pgfqpoint{3.075000in}{2.077831in}}%
\pgfpathlineto{\pgfqpoint{3.075000in}{2.074882in}}%
\pgfpathmoveto{\pgfqpoint{3.070459in}{2.077831in}}%
\pgfpathlineto{\pgfqpoint{3.070459in}{2.077831in}}%
\pgfpathlineto{\pgfqpoint{3.070459in}{2.080780in}}%
\pgfpathlineto{\pgfqpoint{3.075000in}{2.080780in}}%
\pgfpathlineto{\pgfqpoint{3.075000in}{2.077831in}}%
\pgfpathmoveto{\pgfqpoint{3.070459in}{2.080780in}}%
\pgfpathlineto{\pgfqpoint{3.070459in}{2.080780in}}%
\pgfpathlineto{\pgfqpoint{3.070459in}{2.083729in}}%
\pgfpathlineto{\pgfqpoint{3.075000in}{2.083729in}}%
\pgfpathlineto{\pgfqpoint{3.075000in}{2.080780in}}%
\pgfpathmoveto{\pgfqpoint{3.070459in}{2.083729in}}%
\pgfpathlineto{\pgfqpoint{3.070459in}{2.083729in}}%
\pgfpathlineto{\pgfqpoint{3.070459in}{2.086678in}}%
\pgfpathlineto{\pgfqpoint{3.075000in}{2.086678in}}%
\pgfpathlineto{\pgfqpoint{3.075000in}{2.083729in}}%
\pgfpathmoveto{\pgfqpoint{3.070459in}{2.086678in}}%
\pgfpathlineto{\pgfqpoint{3.070459in}{2.086678in}}%
\pgfpathlineto{\pgfqpoint{3.070459in}{2.089628in}}%
\pgfpathlineto{\pgfqpoint{3.075000in}{2.089628in}}%
\pgfpathlineto{\pgfqpoint{3.075000in}{2.086678in}}%
\pgfpathmoveto{\pgfqpoint{3.070459in}{2.089628in}}%
\pgfpathlineto{\pgfqpoint{3.070459in}{2.089628in}}%
\pgfpathlineto{\pgfqpoint{3.070459in}{2.092577in}}%
\pgfpathlineto{\pgfqpoint{3.075000in}{2.092577in}}%
\pgfpathlineto{\pgfqpoint{3.075000in}{2.089628in}}%
\pgfpathmoveto{\pgfqpoint{3.070459in}{2.092577in}}%
\pgfpathlineto{\pgfqpoint{3.070459in}{2.092577in}}%
\pgfpathlineto{\pgfqpoint{3.070459in}{2.095526in}}%
\pgfpathlineto{\pgfqpoint{3.075000in}{2.095526in}}%
\pgfpathlineto{\pgfqpoint{3.075000in}{2.092577in}}%
\pgfpathmoveto{\pgfqpoint{3.070459in}{2.095526in}}%
\pgfpathlineto{\pgfqpoint{3.070459in}{2.095526in}}%
\pgfpathlineto{\pgfqpoint{3.070459in}{2.098475in}}%
\pgfpathlineto{\pgfqpoint{3.075000in}{2.098475in}}%
\pgfpathlineto{\pgfqpoint{3.075000in}{2.095526in}}%
\pgfpathmoveto{\pgfqpoint{3.070459in}{2.098475in}}%
\pgfpathlineto{\pgfqpoint{3.070459in}{2.098475in}}%
\pgfpathlineto{\pgfqpoint{3.070459in}{2.101425in}}%
\pgfpathlineto{\pgfqpoint{3.075000in}{2.101425in}}%
\pgfpathlineto{\pgfqpoint{3.075000in}{2.098475in}}%
\pgfpathmoveto{\pgfqpoint{3.070459in}{2.101425in}}%
\pgfpathlineto{\pgfqpoint{3.070459in}{2.101425in}}%
\pgfpathlineto{\pgfqpoint{3.070459in}{2.104374in}}%
\pgfpathlineto{\pgfqpoint{3.075000in}{2.104374in}}%
\pgfpathlineto{\pgfqpoint{3.075000in}{2.101425in}}%
\pgfpathmoveto{\pgfqpoint{3.070459in}{2.104374in}}%
\pgfpathlineto{\pgfqpoint{3.070459in}{2.104374in}}%
\pgfpathlineto{\pgfqpoint{3.070459in}{2.107323in}}%
\pgfpathlineto{\pgfqpoint{3.075000in}{2.107323in}}%
\pgfpathlineto{\pgfqpoint{3.075000in}{2.104374in}}%
\pgfpathmoveto{\pgfqpoint{3.070459in}{2.107323in}}%
\pgfpathlineto{\pgfqpoint{3.070459in}{2.107323in}}%
\pgfpathlineto{\pgfqpoint{3.070459in}{2.110272in}}%
\pgfpathlineto{\pgfqpoint{3.075000in}{2.110272in}}%
\pgfpathlineto{\pgfqpoint{3.075000in}{2.107323in}}%
\pgfpathmoveto{\pgfqpoint{3.070459in}{2.110272in}}%
\pgfpathlineto{\pgfqpoint{3.070459in}{2.110272in}}%
\pgfpathlineto{\pgfqpoint{3.070459in}{2.113222in}}%
\pgfpathlineto{\pgfqpoint{3.075000in}{2.113222in}}%
\pgfpathlineto{\pgfqpoint{3.075000in}{2.110272in}}%
\pgfpathmoveto{\pgfqpoint{3.070459in}{2.113222in}}%
\pgfpathlineto{\pgfqpoint{3.070459in}{2.113222in}}%
\pgfpathlineto{\pgfqpoint{3.070459in}{2.116171in}}%
\pgfpathlineto{\pgfqpoint{3.075000in}{2.116171in}}%
\pgfpathlineto{\pgfqpoint{3.075000in}{2.113222in}}%
\pgfpathmoveto{\pgfqpoint{3.070459in}{2.116171in}}%
\pgfpathlineto{\pgfqpoint{3.070459in}{2.116171in}}%
\pgfpathlineto{\pgfqpoint{3.070459in}{2.119120in}}%
\pgfpathlineto{\pgfqpoint{3.075000in}{2.119120in}}%
\pgfpathlineto{\pgfqpoint{3.075000in}{2.116171in}}%
\pgfpathmoveto{\pgfqpoint{3.070459in}{2.119120in}}%
\pgfpathlineto{\pgfqpoint{3.070459in}{2.119120in}}%
\pgfpathlineto{\pgfqpoint{3.070459in}{2.122070in}}%
\pgfpathlineto{\pgfqpoint{3.075000in}{2.122070in}}%
\pgfpathlineto{\pgfqpoint{3.075000in}{2.119120in}}%
\pgfpathmoveto{\pgfqpoint{3.070459in}{2.122070in}}%
\pgfpathlineto{\pgfqpoint{3.070459in}{2.122070in}}%
\pgfpathlineto{\pgfqpoint{3.070459in}{2.125019in}}%
\pgfpathlineto{\pgfqpoint{3.075000in}{2.125019in}}%
\pgfpathlineto{\pgfqpoint{3.075000in}{2.122070in}}%
\pgfpathmoveto{\pgfqpoint{3.070459in}{2.125019in}}%
\pgfpathlineto{\pgfqpoint{3.070459in}{2.125019in}}%
\pgfpathlineto{\pgfqpoint{3.070459in}{2.127968in}}%
\pgfpathlineto{\pgfqpoint{3.075000in}{2.127968in}}%
\pgfpathlineto{\pgfqpoint{3.075000in}{2.125019in}}%
\pgfpathmoveto{\pgfqpoint{3.070459in}{2.127968in}}%
\pgfpathlineto{\pgfqpoint{3.070459in}{2.127968in}}%
\pgfpathlineto{\pgfqpoint{3.070459in}{2.130918in}}%
\pgfpathlineto{\pgfqpoint{3.075000in}{2.130918in}}%
\pgfpathlineto{\pgfqpoint{3.075000in}{2.127968in}}%
\pgfpathmoveto{\pgfqpoint{3.070459in}{2.130918in}}%
\pgfpathlineto{\pgfqpoint{3.070459in}{2.130918in}}%
\pgfpathlineto{\pgfqpoint{3.070459in}{2.133867in}}%
\pgfpathlineto{\pgfqpoint{3.075000in}{2.133867in}}%
\pgfpathlineto{\pgfqpoint{3.075000in}{2.130918in}}%
\pgfpathmoveto{\pgfqpoint{3.070459in}{2.133867in}}%
\pgfpathlineto{\pgfqpoint{3.070459in}{2.133867in}}%
\pgfpathlineto{\pgfqpoint{3.070459in}{2.136816in}}%
\pgfpathlineto{\pgfqpoint{3.075000in}{2.136816in}}%
\pgfpathlineto{\pgfqpoint{3.075000in}{2.133867in}}%
\pgfpathmoveto{\pgfqpoint{3.070459in}{2.136816in}}%
\pgfpathlineto{\pgfqpoint{3.070459in}{2.136816in}}%
\pgfpathlineto{\pgfqpoint{3.070459in}{2.139766in}}%
\pgfpathlineto{\pgfqpoint{3.075000in}{2.139766in}}%
\pgfpathlineto{\pgfqpoint{3.075000in}{2.136816in}}%
\pgfpathmoveto{\pgfqpoint{3.070459in}{2.139766in}}%
\pgfpathlineto{\pgfqpoint{3.070459in}{2.139766in}}%
\pgfpathlineto{\pgfqpoint{3.070459in}{2.142715in}}%
\pgfpathlineto{\pgfqpoint{3.075000in}{2.142715in}}%
\pgfpathlineto{\pgfqpoint{3.075000in}{2.139766in}}%
\pgfpathmoveto{\pgfqpoint{3.070459in}{2.142715in}}%
\pgfpathlineto{\pgfqpoint{3.070459in}{2.142715in}}%
\pgfpathlineto{\pgfqpoint{3.070459in}{2.145664in}}%
\pgfpathlineto{\pgfqpoint{3.075000in}{2.145664in}}%
\pgfpathlineto{\pgfqpoint{3.075000in}{2.142715in}}%
\pgfpathmoveto{\pgfqpoint{3.070459in}{2.145664in}}%
\pgfpathlineto{\pgfqpoint{3.070459in}{2.145664in}}%
\pgfpathlineto{\pgfqpoint{3.070459in}{2.148614in}}%
\pgfpathlineto{\pgfqpoint{3.075000in}{2.148614in}}%
\pgfpathlineto{\pgfqpoint{3.075000in}{2.145664in}}%
\pgfpathmoveto{\pgfqpoint{3.070459in}{2.148614in}}%
\pgfpathlineto{\pgfqpoint{3.070459in}{2.148614in}}%
\pgfpathlineto{\pgfqpoint{3.070459in}{2.151563in}}%
\pgfpathlineto{\pgfqpoint{3.075000in}{2.151563in}}%
\pgfpathlineto{\pgfqpoint{3.075000in}{2.148614in}}%
\pgfpathmoveto{\pgfqpoint{3.070459in}{2.151563in}}%
\pgfpathlineto{\pgfqpoint{3.070459in}{2.151563in}}%
\pgfpathlineto{\pgfqpoint{3.070459in}{2.154512in}}%
\pgfpathlineto{\pgfqpoint{3.075000in}{2.154512in}}%
\pgfpathlineto{\pgfqpoint{3.075000in}{2.151563in}}%
\pgfpathmoveto{\pgfqpoint{3.070459in}{2.154512in}}%
\pgfpathlineto{\pgfqpoint{3.070459in}{2.154512in}}%
\pgfpathlineto{\pgfqpoint{3.070459in}{2.157462in}}%
\pgfpathlineto{\pgfqpoint{3.075000in}{2.157462in}}%
\pgfpathlineto{\pgfqpoint{3.075000in}{2.154512in}}%
\pgfpathmoveto{\pgfqpoint{3.070459in}{2.157462in}}%
\pgfpathlineto{\pgfqpoint{3.070459in}{2.157462in}}%
\pgfpathlineto{\pgfqpoint{3.070459in}{2.160411in}}%
\pgfpathlineto{\pgfqpoint{3.075000in}{2.160411in}}%
\pgfpathlineto{\pgfqpoint{3.075000in}{2.157462in}}%
\pgfpathmoveto{\pgfqpoint{3.070459in}{2.160411in}}%
\pgfpathlineto{\pgfqpoint{3.070459in}{2.160411in}}%
\pgfpathlineto{\pgfqpoint{3.070459in}{2.163360in}}%
\pgfpathlineto{\pgfqpoint{3.075000in}{2.163360in}}%
\pgfpathlineto{\pgfqpoint{3.075000in}{2.160411in}}%
\pgfpathmoveto{\pgfqpoint{3.070459in}{2.163360in}}%
\pgfpathlineto{\pgfqpoint{3.070459in}{2.163360in}}%
\pgfpathlineto{\pgfqpoint{3.070459in}{2.166309in}}%
\pgfpathlineto{\pgfqpoint{3.075000in}{2.166309in}}%
\pgfpathlineto{\pgfqpoint{3.075000in}{2.163360in}}%
\pgfpathmoveto{\pgfqpoint{3.070459in}{2.166309in}}%
\pgfpathlineto{\pgfqpoint{3.070459in}{2.166309in}}%
\pgfpathlineto{\pgfqpoint{3.070459in}{2.169259in}}%
\pgfpathlineto{\pgfqpoint{3.075000in}{2.169259in}}%
\pgfpathlineto{\pgfqpoint{3.075000in}{2.166309in}}%
\pgfpathmoveto{\pgfqpoint{3.070459in}{2.169259in}}%
\pgfpathlineto{\pgfqpoint{3.070459in}{2.169259in}}%
\pgfpathlineto{\pgfqpoint{3.070459in}{2.172208in}}%
\pgfpathlineto{\pgfqpoint{3.075000in}{2.172208in}}%
\pgfpathlineto{\pgfqpoint{3.075000in}{2.169259in}}%
\pgfpathmoveto{\pgfqpoint{3.070459in}{2.172208in}}%
\pgfpathlineto{\pgfqpoint{3.070459in}{2.172208in}}%
\pgfpathlineto{\pgfqpoint{3.070459in}{2.175157in}}%
\pgfpathlineto{\pgfqpoint{3.075000in}{2.175157in}}%
\pgfpathlineto{\pgfqpoint{3.075000in}{2.172208in}}%
\pgfpathmoveto{\pgfqpoint{3.070459in}{2.175157in}}%
\pgfpathlineto{\pgfqpoint{3.070459in}{2.175157in}}%
\pgfpathlineto{\pgfqpoint{3.070459in}{2.178107in}}%
\pgfpathlineto{\pgfqpoint{3.075000in}{2.178107in}}%
\pgfpathlineto{\pgfqpoint{3.075000in}{2.175157in}}%
\pgfpathmoveto{\pgfqpoint{3.070459in}{2.178107in}}%
\pgfpathlineto{\pgfqpoint{3.070459in}{2.178107in}}%
\pgfpathlineto{\pgfqpoint{3.070459in}{2.181056in}}%
\pgfpathlineto{\pgfqpoint{3.075000in}{2.181056in}}%
\pgfpathlineto{\pgfqpoint{3.075000in}{2.178107in}}%
\pgfpathmoveto{\pgfqpoint{3.070459in}{2.181056in}}%
\pgfpathlineto{\pgfqpoint{3.070459in}{2.181056in}}%
\pgfpathlineto{\pgfqpoint{3.070459in}{2.184005in}}%
\pgfpathlineto{\pgfqpoint{3.075000in}{2.184005in}}%
\pgfpathlineto{\pgfqpoint{3.075000in}{2.181056in}}%
\pgfpathmoveto{\pgfqpoint{3.070459in}{2.184005in}}%
\pgfpathlineto{\pgfqpoint{3.070459in}{2.184005in}}%
\pgfpathlineto{\pgfqpoint{3.070459in}{2.186955in}}%
\pgfpathlineto{\pgfqpoint{3.075000in}{2.186955in}}%
\pgfpathlineto{\pgfqpoint{3.075000in}{2.184005in}}%
\pgfpathmoveto{\pgfqpoint{3.070459in}{2.186955in}}%
\pgfpathlineto{\pgfqpoint{3.070459in}{2.186955in}}%
\pgfpathlineto{\pgfqpoint{3.070459in}{2.189904in}}%
\pgfpathlineto{\pgfqpoint{3.075000in}{2.189904in}}%
\pgfpathlineto{\pgfqpoint{3.075000in}{2.186955in}}%
\pgfpathmoveto{\pgfqpoint{3.070459in}{2.189904in}}%
\pgfpathlineto{\pgfqpoint{3.070459in}{2.189904in}}%
\pgfpathlineto{\pgfqpoint{3.070459in}{2.192853in}}%
\pgfpathlineto{\pgfqpoint{3.075000in}{2.192853in}}%
\pgfpathlineto{\pgfqpoint{3.075000in}{2.189904in}}%
\pgfpathmoveto{\pgfqpoint{3.070459in}{2.192853in}}%
\pgfpathlineto{\pgfqpoint{3.070459in}{2.192853in}}%
\pgfpathlineto{\pgfqpoint{3.070459in}{2.195803in}}%
\pgfpathlineto{\pgfqpoint{3.075000in}{2.195803in}}%
\pgfpathlineto{\pgfqpoint{3.075000in}{2.192853in}}%
\pgfpathmoveto{\pgfqpoint{3.070459in}{2.195803in}}%
\pgfpathlineto{\pgfqpoint{3.070459in}{2.195803in}}%
\pgfpathlineto{\pgfqpoint{3.070459in}{2.198752in}}%
\pgfpathlineto{\pgfqpoint{3.075000in}{2.198752in}}%
\pgfpathlineto{\pgfqpoint{3.075000in}{2.195803in}}%
\pgfpathmoveto{\pgfqpoint{3.070459in}{2.198752in}}%
\pgfpathlineto{\pgfqpoint{3.070459in}{2.198752in}}%
\pgfpathlineto{\pgfqpoint{3.070459in}{2.201701in}}%
\pgfpathlineto{\pgfqpoint{3.075000in}{2.201701in}}%
\pgfpathlineto{\pgfqpoint{3.075000in}{2.198752in}}%
\pgfpathmoveto{\pgfqpoint{3.070459in}{2.201701in}}%
\pgfpathlineto{\pgfqpoint{3.070459in}{2.201701in}}%
\pgfpathlineto{\pgfqpoint{3.070459in}{2.204650in}}%
\pgfpathlineto{\pgfqpoint{3.075000in}{2.204650in}}%
\pgfpathlineto{\pgfqpoint{3.075000in}{2.201701in}}%
\pgfpathmoveto{\pgfqpoint{3.070459in}{2.204650in}}%
\pgfpathlineto{\pgfqpoint{3.070459in}{2.204650in}}%
\pgfpathlineto{\pgfqpoint{3.070459in}{2.207600in}}%
\pgfpathlineto{\pgfqpoint{3.075000in}{2.207600in}}%
\pgfpathlineto{\pgfqpoint{3.075000in}{2.204650in}}%
\pgfpathmoveto{\pgfqpoint{3.070459in}{2.207600in}}%
\pgfpathlineto{\pgfqpoint{3.070459in}{2.207600in}}%
\pgfpathlineto{\pgfqpoint{3.070459in}{2.210549in}}%
\pgfpathlineto{\pgfqpoint{3.075000in}{2.210549in}}%
\pgfpathlineto{\pgfqpoint{3.075000in}{2.207600in}}%
\pgfpathmoveto{\pgfqpoint{3.070459in}{2.210549in}}%
\pgfpathlineto{\pgfqpoint{3.070459in}{2.210549in}}%
\pgfpathlineto{\pgfqpoint{3.070459in}{2.213498in}}%
\pgfpathlineto{\pgfqpoint{3.075000in}{2.213498in}}%
\pgfpathlineto{\pgfqpoint{3.075000in}{2.210549in}}%
\pgfpathmoveto{\pgfqpoint{3.070459in}{2.213498in}}%
\pgfpathlineto{\pgfqpoint{3.070459in}{2.213498in}}%
\pgfpathlineto{\pgfqpoint{3.070459in}{2.216447in}}%
\pgfpathlineto{\pgfqpoint{3.075000in}{2.216447in}}%
\pgfpathlineto{\pgfqpoint{3.075000in}{2.213498in}}%
\pgfpathmoveto{\pgfqpoint{3.070459in}{2.216447in}}%
\pgfpathlineto{\pgfqpoint{3.070459in}{2.216447in}}%
\pgfpathlineto{\pgfqpoint{3.070459in}{2.219397in}}%
\pgfpathlineto{\pgfqpoint{3.075000in}{2.219397in}}%
\pgfpathlineto{\pgfqpoint{3.075000in}{2.216447in}}%
\pgfpathmoveto{\pgfqpoint{3.070459in}{2.219397in}}%
\pgfpathlineto{\pgfqpoint{3.070459in}{2.219397in}}%
\pgfpathlineto{\pgfqpoint{3.070459in}{2.222346in}}%
\pgfpathlineto{\pgfqpoint{3.075000in}{2.222346in}}%
\pgfpathlineto{\pgfqpoint{3.075000in}{2.219397in}}%
\pgfpathmoveto{\pgfqpoint{3.070459in}{2.222346in}}%
\pgfpathlineto{\pgfqpoint{3.070459in}{2.222346in}}%
\pgfpathlineto{\pgfqpoint{3.070459in}{2.225295in}}%
\pgfpathlineto{\pgfqpoint{3.075000in}{2.225295in}}%
\pgfpathlineto{\pgfqpoint{3.075000in}{2.222346in}}%
\pgfpathmoveto{\pgfqpoint{3.070459in}{2.225295in}}%
\pgfpathlineto{\pgfqpoint{3.070459in}{2.225295in}}%
\pgfpathlineto{\pgfqpoint{3.070459in}{2.228244in}}%
\pgfpathlineto{\pgfqpoint{3.075000in}{2.228244in}}%
\pgfpathlineto{\pgfqpoint{3.075000in}{2.225295in}}%
\pgfpathmoveto{\pgfqpoint{3.070459in}{2.228244in}}%
\pgfpathlineto{\pgfqpoint{3.070459in}{2.228244in}}%
\pgfpathlineto{\pgfqpoint{3.070459in}{2.231194in}}%
\pgfpathlineto{\pgfqpoint{3.075000in}{2.231194in}}%
\pgfpathlineto{\pgfqpoint{3.075000in}{2.228244in}}%
\pgfpathmoveto{\pgfqpoint{3.070459in}{2.231194in}}%
\pgfpathlineto{\pgfqpoint{3.070459in}{2.231194in}}%
\pgfpathlineto{\pgfqpoint{3.070459in}{2.234143in}}%
\pgfpathlineto{\pgfqpoint{3.075000in}{2.234143in}}%
\pgfpathlineto{\pgfqpoint{3.075000in}{2.231194in}}%
\pgfpathmoveto{\pgfqpoint{3.070459in}{2.234143in}}%
\pgfpathlineto{\pgfqpoint{3.070459in}{2.234143in}}%
\pgfpathlineto{\pgfqpoint{3.070459in}{2.237092in}}%
\pgfpathlineto{\pgfqpoint{3.075000in}{2.237092in}}%
\pgfpathlineto{\pgfqpoint{3.075000in}{2.234143in}}%
\pgfpathmoveto{\pgfqpoint{3.070459in}{2.237092in}}%
\pgfpathlineto{\pgfqpoint{3.070459in}{2.237092in}}%
\pgfpathlineto{\pgfqpoint{3.070459in}{2.240041in}}%
\pgfpathlineto{\pgfqpoint{3.075000in}{2.240041in}}%
\pgfpathlineto{\pgfqpoint{3.075000in}{2.237092in}}%
\pgfpathmoveto{\pgfqpoint{3.070459in}{2.240041in}}%
\pgfpathlineto{\pgfqpoint{3.070459in}{2.240041in}}%
\pgfpathlineto{\pgfqpoint{3.070459in}{2.242991in}}%
\pgfpathlineto{\pgfqpoint{3.075000in}{2.242991in}}%
\pgfpathlineto{\pgfqpoint{3.075000in}{2.240041in}}%
\pgfpathmoveto{\pgfqpoint{3.070459in}{2.242991in}}%
\pgfpathlineto{\pgfqpoint{3.070459in}{2.242991in}}%
\pgfpathlineto{\pgfqpoint{3.070459in}{2.245940in}}%
\pgfpathlineto{\pgfqpoint{3.075000in}{2.245940in}}%
\pgfpathlineto{\pgfqpoint{3.075000in}{2.242991in}}%
\pgfpathmoveto{\pgfqpoint{3.070459in}{2.245940in}}%
\pgfpathlineto{\pgfqpoint{3.070459in}{2.245940in}}%
\pgfpathlineto{\pgfqpoint{3.070459in}{2.248889in}}%
\pgfpathlineto{\pgfqpoint{3.075000in}{2.248889in}}%
\pgfpathlineto{\pgfqpoint{3.075000in}{2.245940in}}%
\pgfpathmoveto{\pgfqpoint{3.070459in}{2.248889in}}%
\pgfpathlineto{\pgfqpoint{3.070459in}{2.248889in}}%
\pgfpathlineto{\pgfqpoint{3.070459in}{2.251838in}}%
\pgfpathlineto{\pgfqpoint{3.075000in}{2.251838in}}%
\pgfpathlineto{\pgfqpoint{3.075000in}{2.248889in}}%
\pgfpathmoveto{\pgfqpoint{3.070459in}{2.251838in}}%
\pgfpathlineto{\pgfqpoint{3.070459in}{2.251838in}}%
\pgfpathlineto{\pgfqpoint{3.070459in}{2.254787in}}%
\pgfpathlineto{\pgfqpoint{3.075000in}{2.254787in}}%
\pgfpathlineto{\pgfqpoint{3.075000in}{2.251838in}}%
\pgfpathmoveto{\pgfqpoint{3.070459in}{2.254787in}}%
\pgfpathlineto{\pgfqpoint{3.070459in}{2.254787in}}%
\pgfpathlineto{\pgfqpoint{3.070459in}{2.257737in}}%
\pgfpathlineto{\pgfqpoint{3.075000in}{2.257737in}}%
\pgfpathlineto{\pgfqpoint{3.075000in}{2.254787in}}%
\pgfpathmoveto{\pgfqpoint{3.070459in}{2.257737in}}%
\pgfpathlineto{\pgfqpoint{3.070459in}{2.257737in}}%
\pgfpathlineto{\pgfqpoint{3.070459in}{2.260686in}}%
\pgfpathlineto{\pgfqpoint{3.075000in}{2.260686in}}%
\pgfpathlineto{\pgfqpoint{3.075000in}{2.257737in}}%
\pgfpathmoveto{\pgfqpoint{3.070459in}{2.260686in}}%
\pgfpathlineto{\pgfqpoint{3.070459in}{2.260686in}}%
\pgfpathlineto{\pgfqpoint{3.070459in}{2.263635in}}%
\pgfpathlineto{\pgfqpoint{3.075000in}{2.263635in}}%
\pgfpathlineto{\pgfqpoint{3.075000in}{2.260686in}}%
\pgfpathmoveto{\pgfqpoint{3.070459in}{2.263635in}}%
\pgfpathlineto{\pgfqpoint{3.070459in}{2.263635in}}%
\pgfpathlineto{\pgfqpoint{3.070459in}{2.266584in}}%
\pgfpathlineto{\pgfqpoint{3.075000in}{2.266584in}}%
\pgfpathlineto{\pgfqpoint{3.075000in}{2.263635in}}%
\pgfpathmoveto{\pgfqpoint{3.070459in}{2.266584in}}%
\pgfpathlineto{\pgfqpoint{3.070459in}{2.266584in}}%
\pgfpathlineto{\pgfqpoint{3.070459in}{2.269534in}}%
\pgfpathlineto{\pgfqpoint{3.075000in}{2.269534in}}%
\pgfpathlineto{\pgfqpoint{3.075000in}{2.266584in}}%
\pgfpathmoveto{\pgfqpoint{3.070459in}{2.269534in}}%
\pgfpathlineto{\pgfqpoint{3.070459in}{2.269534in}}%
\pgfpathlineto{\pgfqpoint{3.070459in}{2.272483in}}%
\pgfpathlineto{\pgfqpoint{3.075000in}{2.272483in}}%
\pgfpathlineto{\pgfqpoint{3.075000in}{2.269534in}}%
\pgfpathmoveto{\pgfqpoint{3.070459in}{2.272483in}}%
\pgfpathlineto{\pgfqpoint{3.070459in}{2.272483in}}%
\pgfpathlineto{\pgfqpoint{3.070459in}{2.275432in}}%
\pgfpathlineto{\pgfqpoint{3.075000in}{2.275432in}}%
\pgfpathlineto{\pgfqpoint{3.075000in}{2.272483in}}%
\pgfpathmoveto{\pgfqpoint{3.070459in}{2.275432in}}%
\pgfpathlineto{\pgfqpoint{3.070459in}{2.275432in}}%
\pgfpathlineto{\pgfqpoint{3.070459in}{2.278381in}}%
\pgfpathlineto{\pgfqpoint{3.075000in}{2.278381in}}%
\pgfpathlineto{\pgfqpoint{3.075000in}{2.275432in}}%
\pgfpathmoveto{\pgfqpoint{3.070459in}{2.278381in}}%
\pgfpathlineto{\pgfqpoint{3.070459in}{2.278381in}}%
\pgfpathlineto{\pgfqpoint{3.070459in}{2.281331in}}%
\pgfpathlineto{\pgfqpoint{3.075000in}{2.281331in}}%
\pgfpathlineto{\pgfqpoint{3.075000in}{2.278381in}}%
\pgfpathmoveto{\pgfqpoint{3.070459in}{2.281331in}}%
\pgfpathlineto{\pgfqpoint{3.070459in}{2.281331in}}%
\pgfpathlineto{\pgfqpoint{3.070459in}{2.284280in}}%
\pgfpathlineto{\pgfqpoint{3.075000in}{2.284280in}}%
\pgfpathlineto{\pgfqpoint{3.075000in}{2.281331in}}%
\pgfpathmoveto{\pgfqpoint{3.070459in}{2.284280in}}%
\pgfpathlineto{\pgfqpoint{3.070459in}{2.284280in}}%
\pgfpathlineto{\pgfqpoint{3.070459in}{2.287229in}}%
\pgfpathlineto{\pgfqpoint{3.075000in}{2.287229in}}%
\pgfpathlineto{\pgfqpoint{3.075000in}{2.284280in}}%
\pgfpathmoveto{\pgfqpoint{3.070459in}{2.287229in}}%
\pgfpathlineto{\pgfqpoint{3.070459in}{2.287229in}}%
\pgfpathlineto{\pgfqpoint{3.070459in}{2.290178in}}%
\pgfpathlineto{\pgfqpoint{3.075000in}{2.290178in}}%
\pgfpathlineto{\pgfqpoint{3.075000in}{2.287229in}}%
\pgfpathmoveto{\pgfqpoint{3.070459in}{2.290178in}}%
\pgfpathlineto{\pgfqpoint{3.070459in}{2.290178in}}%
\pgfpathlineto{\pgfqpoint{3.070459in}{2.293128in}}%
\pgfpathlineto{\pgfqpoint{3.075000in}{2.293128in}}%
\pgfpathlineto{\pgfqpoint{3.075000in}{2.290178in}}%
\pgfpathmoveto{\pgfqpoint{3.070459in}{2.293128in}}%
\pgfpathlineto{\pgfqpoint{3.070459in}{2.293128in}}%
\pgfpathlineto{\pgfqpoint{3.070459in}{2.296077in}}%
\pgfpathlineto{\pgfqpoint{3.075000in}{2.296077in}}%
\pgfpathlineto{\pgfqpoint{3.075000in}{2.293128in}}%
\pgfpathmoveto{\pgfqpoint{3.070459in}{2.296077in}}%
\pgfpathlineto{\pgfqpoint{3.070459in}{2.296077in}}%
\pgfpathlineto{\pgfqpoint{3.070459in}{2.299026in}}%
\pgfpathlineto{\pgfqpoint{3.075000in}{2.299026in}}%
\pgfpathlineto{\pgfqpoint{3.075000in}{2.296077in}}%
\pgfpathmoveto{\pgfqpoint{3.070459in}{2.299026in}}%
\pgfpathlineto{\pgfqpoint{3.070459in}{2.299026in}}%
\pgfpathlineto{\pgfqpoint{3.070459in}{2.301975in}}%
\pgfpathlineto{\pgfqpoint{3.075000in}{2.301975in}}%
\pgfpathlineto{\pgfqpoint{3.075000in}{2.299026in}}%
\pgfpathmoveto{\pgfqpoint{3.070459in}{2.301975in}}%
\pgfpathlineto{\pgfqpoint{3.070459in}{2.301975in}}%
\pgfpathlineto{\pgfqpoint{3.070459in}{2.304924in}}%
\pgfpathlineto{\pgfqpoint{3.075000in}{2.304924in}}%
\pgfpathlineto{\pgfqpoint{3.075000in}{2.301975in}}%
\pgfpathmoveto{\pgfqpoint{3.070459in}{2.304924in}}%
\pgfpathlineto{\pgfqpoint{3.070459in}{2.304924in}}%
\pgfpathlineto{\pgfqpoint{3.070459in}{2.307874in}}%
\pgfpathlineto{\pgfqpoint{3.075000in}{2.307874in}}%
\pgfpathlineto{\pgfqpoint{3.075000in}{2.304924in}}%
\pgfpathmoveto{\pgfqpoint{3.070459in}{2.307874in}}%
\pgfpathlineto{\pgfqpoint{3.070459in}{2.307874in}}%
\pgfpathlineto{\pgfqpoint{3.070459in}{2.310823in}}%
\pgfpathlineto{\pgfqpoint{3.075000in}{2.310823in}}%
\pgfpathlineto{\pgfqpoint{3.075000in}{2.307874in}}%
\pgfpathmoveto{\pgfqpoint{3.070459in}{2.310823in}}%
\pgfpathlineto{\pgfqpoint{3.070459in}{2.310823in}}%
\pgfpathlineto{\pgfqpoint{3.070459in}{2.313772in}}%
\pgfpathlineto{\pgfqpoint{3.075000in}{2.313772in}}%
\pgfpathlineto{\pgfqpoint{3.075000in}{2.310823in}}%
\pgfpathmoveto{\pgfqpoint{3.070459in}{2.313772in}}%
\pgfpathlineto{\pgfqpoint{3.070459in}{2.313772in}}%
\pgfpathlineto{\pgfqpoint{3.070459in}{2.316721in}}%
\pgfpathlineto{\pgfqpoint{3.075000in}{2.316721in}}%
\pgfpathlineto{\pgfqpoint{3.075000in}{2.313772in}}%
\pgfpathmoveto{\pgfqpoint{3.070459in}{2.316721in}}%
\pgfpathlineto{\pgfqpoint{3.070459in}{2.316721in}}%
\pgfpathlineto{\pgfqpoint{3.070459in}{2.319670in}}%
\pgfpathlineto{\pgfqpoint{3.075000in}{2.319670in}}%
\pgfpathlineto{\pgfqpoint{3.075000in}{2.316721in}}%
\pgfpathmoveto{\pgfqpoint{3.070459in}{2.319670in}}%
\pgfpathlineto{\pgfqpoint{3.070459in}{2.319670in}}%
\pgfpathlineto{\pgfqpoint{3.070459in}{2.322620in}}%
\pgfpathlineto{\pgfqpoint{3.075000in}{2.322620in}}%
\pgfpathlineto{\pgfqpoint{3.075000in}{2.319670in}}%
\pgfpathmoveto{\pgfqpoint{3.070459in}{2.322620in}}%
\pgfpathlineto{\pgfqpoint{3.070459in}{2.322620in}}%
\pgfpathlineto{\pgfqpoint{3.070459in}{2.325569in}}%
\pgfpathlineto{\pgfqpoint{3.075000in}{2.325569in}}%
\pgfpathlineto{\pgfqpoint{3.075000in}{2.322620in}}%
\pgfpathmoveto{\pgfqpoint{3.070459in}{2.325569in}}%
\pgfpathlineto{\pgfqpoint{3.070459in}{2.325569in}}%
\pgfpathlineto{\pgfqpoint{3.070459in}{2.328518in}}%
\pgfpathlineto{\pgfqpoint{3.075000in}{2.328518in}}%
\pgfpathlineto{\pgfqpoint{3.075000in}{2.325569in}}%
\pgfpathmoveto{\pgfqpoint{3.070459in}{2.328518in}}%
\pgfpathlineto{\pgfqpoint{3.070459in}{2.328518in}}%
\pgfpathlineto{\pgfqpoint{3.070459in}{2.331467in}}%
\pgfpathlineto{\pgfqpoint{3.075000in}{2.331467in}}%
\pgfpathlineto{\pgfqpoint{3.075000in}{2.328518in}}%
\pgfpathmoveto{\pgfqpoint{3.070459in}{2.331467in}}%
\pgfpathlineto{\pgfqpoint{3.070459in}{2.331467in}}%
\pgfpathlineto{\pgfqpoint{3.070459in}{2.334416in}}%
\pgfpathlineto{\pgfqpoint{3.075000in}{2.334416in}}%
\pgfpathlineto{\pgfqpoint{3.075000in}{2.331467in}}%
\pgfpathmoveto{\pgfqpoint{3.070459in}{2.334416in}}%
\pgfpathlineto{\pgfqpoint{3.070459in}{2.334416in}}%
\pgfpathlineto{\pgfqpoint{3.070459in}{2.337366in}}%
\pgfpathlineto{\pgfqpoint{3.075000in}{2.337366in}}%
\pgfpathlineto{\pgfqpoint{3.075000in}{2.334416in}}%
\pgfpathmoveto{\pgfqpoint{3.070459in}{2.337366in}}%
\pgfpathlineto{\pgfqpoint{3.070459in}{2.337366in}}%
\pgfpathlineto{\pgfqpoint{3.070459in}{2.340315in}}%
\pgfpathlineto{\pgfqpoint{3.075000in}{2.340315in}}%
\pgfpathlineto{\pgfqpoint{3.075000in}{2.337366in}}%
\pgfpathmoveto{\pgfqpoint{3.070459in}{2.340315in}}%
\pgfpathlineto{\pgfqpoint{3.070459in}{2.340315in}}%
\pgfpathlineto{\pgfqpoint{3.070459in}{2.343264in}}%
\pgfpathlineto{\pgfqpoint{3.075000in}{2.343264in}}%
\pgfpathlineto{\pgfqpoint{3.075000in}{2.340315in}}%
\pgfpathmoveto{\pgfqpoint{3.070459in}{2.343264in}}%
\pgfpathlineto{\pgfqpoint{3.070459in}{2.343264in}}%
\pgfpathlineto{\pgfqpoint{3.070459in}{2.346213in}}%
\pgfpathlineto{\pgfqpoint{3.075000in}{2.346213in}}%
\pgfpathlineto{\pgfqpoint{3.075000in}{2.343264in}}%
\pgfpathmoveto{\pgfqpoint{3.070459in}{2.346213in}}%
\pgfpathlineto{\pgfqpoint{3.070459in}{2.346213in}}%
\pgfpathlineto{\pgfqpoint{3.070459in}{2.349162in}}%
\pgfpathlineto{\pgfqpoint{3.075000in}{2.349162in}}%
\pgfpathlineto{\pgfqpoint{3.075000in}{2.346213in}}%
\pgfpathmoveto{\pgfqpoint{3.070459in}{2.349162in}}%
\pgfpathlineto{\pgfqpoint{3.070459in}{2.349162in}}%
\pgfpathlineto{\pgfqpoint{3.070459in}{2.352112in}}%
\pgfpathlineto{\pgfqpoint{3.075000in}{2.352112in}}%
\pgfpathlineto{\pgfqpoint{3.075000in}{2.349162in}}%
\pgfpathmoveto{\pgfqpoint{3.070459in}{2.352112in}}%
\pgfpathlineto{\pgfqpoint{3.070459in}{2.352112in}}%
\pgfpathlineto{\pgfqpoint{3.070459in}{2.355061in}}%
\pgfpathlineto{\pgfqpoint{3.075000in}{2.355061in}}%
\pgfpathlineto{\pgfqpoint{3.075000in}{2.352112in}}%
\pgfpathmoveto{\pgfqpoint{3.070459in}{2.355061in}}%
\pgfpathlineto{\pgfqpoint{3.070459in}{2.355061in}}%
\pgfpathlineto{\pgfqpoint{3.070459in}{2.358010in}}%
\pgfpathlineto{\pgfqpoint{3.075000in}{2.358010in}}%
\pgfpathlineto{\pgfqpoint{3.075000in}{2.355061in}}%
\pgfpathmoveto{\pgfqpoint{3.070459in}{2.358010in}}%
\pgfpathlineto{\pgfqpoint{3.070459in}{2.358010in}}%
\pgfpathlineto{\pgfqpoint{3.070459in}{2.360959in}}%
\pgfpathlineto{\pgfqpoint{3.075000in}{2.360959in}}%
\pgfpathlineto{\pgfqpoint{3.075000in}{2.358010in}}%
\pgfpathmoveto{\pgfqpoint{3.070459in}{2.360959in}}%
\pgfpathlineto{\pgfqpoint{3.070459in}{2.360959in}}%
\pgfpathlineto{\pgfqpoint{3.070459in}{2.363908in}}%
\pgfpathlineto{\pgfqpoint{3.075000in}{2.363908in}}%
\pgfpathlineto{\pgfqpoint{3.075000in}{2.360959in}}%
\pgfpathmoveto{\pgfqpoint{3.070459in}{2.363908in}}%
\pgfpathlineto{\pgfqpoint{3.070459in}{2.363908in}}%
\pgfpathlineto{\pgfqpoint{3.070459in}{2.366858in}}%
\pgfpathlineto{\pgfqpoint{3.075000in}{2.366858in}}%
\pgfpathlineto{\pgfqpoint{3.075000in}{2.363908in}}%
\pgfpathmoveto{\pgfqpoint{3.070459in}{2.366858in}}%
\pgfpathlineto{\pgfqpoint{3.070459in}{2.366858in}}%
\pgfpathlineto{\pgfqpoint{3.070459in}{2.369807in}}%
\pgfpathlineto{\pgfqpoint{3.075000in}{2.369807in}}%
\pgfpathlineto{\pgfqpoint{3.075000in}{2.366858in}}%
\pgfpathmoveto{\pgfqpoint{3.070459in}{2.369807in}}%
\pgfpathlineto{\pgfqpoint{3.070459in}{2.369807in}}%
\pgfpathlineto{\pgfqpoint{3.070459in}{2.372756in}}%
\pgfpathlineto{\pgfqpoint{3.075000in}{2.372756in}}%
\pgfpathlineto{\pgfqpoint{3.075000in}{2.369807in}}%
\pgfpathmoveto{\pgfqpoint{3.070459in}{2.372756in}}%
\pgfpathlineto{\pgfqpoint{3.070459in}{2.372756in}}%
\pgfpathlineto{\pgfqpoint{3.070459in}{2.375705in}}%
\pgfpathlineto{\pgfqpoint{3.075000in}{2.375705in}}%
\pgfpathlineto{\pgfqpoint{3.075000in}{2.372756in}}%
\pgfpathmoveto{\pgfqpoint{3.070459in}{2.375705in}}%
\pgfpathlineto{\pgfqpoint{3.070459in}{2.375705in}}%
\pgfpathlineto{\pgfqpoint{3.070459in}{2.378654in}}%
\pgfpathlineto{\pgfqpoint{3.075000in}{2.378654in}}%
\pgfpathlineto{\pgfqpoint{3.075000in}{2.375705in}}%
\pgfpathmoveto{\pgfqpoint{3.070459in}{2.378654in}}%
\pgfpathlineto{\pgfqpoint{3.070459in}{2.378654in}}%
\pgfpathlineto{\pgfqpoint{3.070459in}{2.381604in}}%
\pgfpathlineto{\pgfqpoint{3.075000in}{2.381604in}}%
\pgfpathlineto{\pgfqpoint{3.075000in}{2.378654in}}%
\pgfpathmoveto{\pgfqpoint{3.070459in}{2.381604in}}%
\pgfpathlineto{\pgfqpoint{3.070459in}{2.381604in}}%
\pgfpathlineto{\pgfqpoint{3.070459in}{2.384553in}}%
\pgfpathlineto{\pgfqpoint{3.075000in}{2.384553in}}%
\pgfpathlineto{\pgfqpoint{3.075000in}{2.381604in}}%
\pgfpathmoveto{\pgfqpoint{3.070459in}{2.384553in}}%
\pgfpathlineto{\pgfqpoint{3.070459in}{2.384553in}}%
\pgfpathlineto{\pgfqpoint{3.070459in}{2.387502in}}%
\pgfpathlineto{\pgfqpoint{3.075000in}{2.387502in}}%
\pgfpathlineto{\pgfqpoint{3.075000in}{2.384553in}}%
\pgfpathmoveto{\pgfqpoint{3.070459in}{2.387502in}}%
\pgfpathlineto{\pgfqpoint{3.070459in}{2.387502in}}%
\pgfpathlineto{\pgfqpoint{3.070459in}{2.390451in}}%
\pgfpathlineto{\pgfqpoint{3.075000in}{2.390451in}}%
\pgfpathlineto{\pgfqpoint{3.075000in}{2.387502in}}%
\pgfpathmoveto{\pgfqpoint{3.070459in}{2.390451in}}%
\pgfpathlineto{\pgfqpoint{3.070459in}{2.390451in}}%
\pgfpathlineto{\pgfqpoint{3.070459in}{2.393400in}}%
\pgfpathlineto{\pgfqpoint{3.075000in}{2.393400in}}%
\pgfpathlineto{\pgfqpoint{3.075000in}{2.390451in}}%
\pgfpathmoveto{\pgfqpoint{3.070459in}{2.393400in}}%
\pgfpathlineto{\pgfqpoint{3.070459in}{2.393400in}}%
\pgfpathlineto{\pgfqpoint{3.070459in}{2.396349in}}%
\pgfpathlineto{\pgfqpoint{3.075000in}{2.396349in}}%
\pgfpathlineto{\pgfqpoint{3.075000in}{2.393400in}}%
\pgfpathmoveto{\pgfqpoint{3.070459in}{2.396349in}}%
\pgfpathlineto{\pgfqpoint{3.070459in}{2.396349in}}%
\pgfpathlineto{\pgfqpoint{3.070459in}{2.399299in}}%
\pgfpathlineto{\pgfqpoint{3.075000in}{2.399299in}}%
\pgfpathlineto{\pgfqpoint{3.075000in}{2.396349in}}%
\pgfpathmoveto{\pgfqpoint{3.070459in}{2.399299in}}%
\pgfpathlineto{\pgfqpoint{3.070459in}{2.399299in}}%
\pgfpathlineto{\pgfqpoint{3.070459in}{2.402248in}}%
\pgfpathlineto{\pgfqpoint{3.075000in}{2.402248in}}%
\pgfpathlineto{\pgfqpoint{3.075000in}{2.399299in}}%
\pgfpathmoveto{\pgfqpoint{3.070459in}{2.402248in}}%
\pgfpathlineto{\pgfqpoint{3.070459in}{2.402248in}}%
\pgfpathlineto{\pgfqpoint{3.070459in}{2.405197in}}%
\pgfpathlineto{\pgfqpoint{3.075000in}{2.405197in}}%
\pgfpathlineto{\pgfqpoint{3.075000in}{2.402248in}}%
\pgfpathmoveto{\pgfqpoint{3.070459in}{2.405197in}}%
\pgfpathlineto{\pgfqpoint{3.070459in}{2.405197in}}%
\pgfpathlineto{\pgfqpoint{3.070459in}{2.408146in}}%
\pgfpathlineto{\pgfqpoint{3.075000in}{2.408146in}}%
\pgfpathlineto{\pgfqpoint{3.075000in}{2.405197in}}%
\pgfpathmoveto{\pgfqpoint{3.070459in}{2.408146in}}%
\pgfpathlineto{\pgfqpoint{3.070459in}{2.408146in}}%
\pgfpathlineto{\pgfqpoint{3.070459in}{2.411095in}}%
\pgfpathlineto{\pgfqpoint{3.075000in}{2.411095in}}%
\pgfpathlineto{\pgfqpoint{3.075000in}{2.408146in}}%
\pgfpathmoveto{\pgfqpoint{3.070459in}{2.411095in}}%
\pgfpathlineto{\pgfqpoint{3.070459in}{2.411095in}}%
\pgfpathlineto{\pgfqpoint{3.070459in}{2.414044in}}%
\pgfpathlineto{\pgfqpoint{3.075000in}{2.414044in}}%
\pgfpathlineto{\pgfqpoint{3.075000in}{2.411095in}}%
\pgfpathmoveto{\pgfqpoint{3.070459in}{2.414044in}}%
\pgfpathlineto{\pgfqpoint{3.070459in}{2.414044in}}%
\pgfpathlineto{\pgfqpoint{3.070459in}{2.416993in}}%
\pgfpathlineto{\pgfqpoint{3.075000in}{2.416993in}}%
\pgfpathlineto{\pgfqpoint{3.075000in}{2.414044in}}%
\pgfpathmoveto{\pgfqpoint{3.070459in}{2.416993in}}%
\pgfpathlineto{\pgfqpoint{3.070459in}{2.416993in}}%
\pgfpathlineto{\pgfqpoint{3.070459in}{2.419943in}}%
\pgfpathlineto{\pgfqpoint{3.075000in}{2.419943in}}%
\pgfpathlineto{\pgfqpoint{3.075000in}{2.416993in}}%
\pgfpathmoveto{\pgfqpoint{3.070459in}{2.419943in}}%
\pgfpathlineto{\pgfqpoint{3.070459in}{2.419943in}}%
\pgfpathlineto{\pgfqpoint{3.070459in}{2.422892in}}%
\pgfpathlineto{\pgfqpoint{3.075000in}{2.422892in}}%
\pgfpathlineto{\pgfqpoint{3.075000in}{2.419943in}}%
\pgfpathmoveto{\pgfqpoint{3.070459in}{2.422892in}}%
\pgfpathlineto{\pgfqpoint{3.070459in}{2.422892in}}%
\pgfpathlineto{\pgfqpoint{3.070459in}{2.425841in}}%
\pgfpathlineto{\pgfqpoint{3.075000in}{2.425841in}}%
\pgfpathlineto{\pgfqpoint{3.075000in}{2.422892in}}%
\pgfpathmoveto{\pgfqpoint{3.070459in}{2.425841in}}%
\pgfpathlineto{\pgfqpoint{3.070459in}{2.425841in}}%
\pgfpathlineto{\pgfqpoint{3.070459in}{2.428790in}}%
\pgfpathlineto{\pgfqpoint{3.075000in}{2.428790in}}%
\pgfpathlineto{\pgfqpoint{3.075000in}{2.425841in}}%
\pgfpathmoveto{\pgfqpoint{3.070459in}{2.428790in}}%
\pgfpathlineto{\pgfqpoint{3.070459in}{2.428790in}}%
\pgfpathlineto{\pgfqpoint{3.070459in}{2.431739in}}%
\pgfpathlineto{\pgfqpoint{3.075000in}{2.431739in}}%
\pgfpathlineto{\pgfqpoint{3.075000in}{2.428790in}}%
\pgfpathmoveto{\pgfqpoint{3.070459in}{2.431739in}}%
\pgfpathlineto{\pgfqpoint{3.070459in}{2.431739in}}%
\pgfpathlineto{\pgfqpoint{3.070459in}{2.434688in}}%
\pgfpathlineto{\pgfqpoint{3.075000in}{2.434688in}}%
\pgfpathlineto{\pgfqpoint{3.075000in}{2.431739in}}%
\pgfpathmoveto{\pgfqpoint{3.070459in}{2.434688in}}%
\pgfpathlineto{\pgfqpoint{3.070459in}{2.434688in}}%
\pgfpathlineto{\pgfqpoint{3.070459in}{2.437637in}}%
\pgfpathlineto{\pgfqpoint{3.075000in}{2.437637in}}%
\pgfpathlineto{\pgfqpoint{3.075000in}{2.434688in}}%
\pgfpathmoveto{\pgfqpoint{3.070459in}{2.437637in}}%
\pgfpathlineto{\pgfqpoint{3.070459in}{2.437637in}}%
\pgfpathlineto{\pgfqpoint{3.070459in}{2.440586in}}%
\pgfpathlineto{\pgfqpoint{3.075000in}{2.440586in}}%
\pgfpathlineto{\pgfqpoint{3.075000in}{2.437637in}}%
\pgfpathmoveto{\pgfqpoint{3.070459in}{2.440586in}}%
\pgfpathlineto{\pgfqpoint{3.070459in}{2.440586in}}%
\pgfpathlineto{\pgfqpoint{3.070459in}{2.443536in}}%
\pgfpathlineto{\pgfqpoint{3.075000in}{2.443536in}}%
\pgfpathlineto{\pgfqpoint{3.075000in}{2.440586in}}%
\pgfpathmoveto{\pgfqpoint{3.070459in}{2.443536in}}%
\pgfpathlineto{\pgfqpoint{3.070459in}{2.443536in}}%
\pgfpathlineto{\pgfqpoint{3.070459in}{2.446485in}}%
\pgfpathlineto{\pgfqpoint{3.075000in}{2.446485in}}%
\pgfpathlineto{\pgfqpoint{3.075000in}{2.443536in}}%
\pgfpathmoveto{\pgfqpoint{3.070459in}{2.446485in}}%
\pgfpathlineto{\pgfqpoint{3.070459in}{2.446485in}}%
\pgfpathlineto{\pgfqpoint{3.070459in}{2.449434in}}%
\pgfpathlineto{\pgfqpoint{3.075000in}{2.449434in}}%
\pgfpathlineto{\pgfqpoint{3.075000in}{2.446485in}}%
\pgfpathmoveto{\pgfqpoint{3.070459in}{2.449434in}}%
\pgfpathlineto{\pgfqpoint{3.070459in}{2.449434in}}%
\pgfpathlineto{\pgfqpoint{3.070459in}{2.452383in}}%
\pgfpathlineto{\pgfqpoint{3.075000in}{2.452383in}}%
\pgfpathlineto{\pgfqpoint{3.075000in}{2.449434in}}%
\pgfpathmoveto{\pgfqpoint{3.070459in}{2.452383in}}%
\pgfpathlineto{\pgfqpoint{3.070459in}{2.452383in}}%
\pgfpathlineto{\pgfqpoint{3.070459in}{2.455332in}}%
\pgfpathlineto{\pgfqpoint{3.075000in}{2.455332in}}%
\pgfpathlineto{\pgfqpoint{3.075000in}{2.452383in}}%
\pgfpathmoveto{\pgfqpoint{3.070459in}{2.455332in}}%
\pgfpathlineto{\pgfqpoint{3.070459in}{2.455332in}}%
\pgfpathlineto{\pgfqpoint{3.070459in}{2.458281in}}%
\pgfpathlineto{\pgfqpoint{3.075000in}{2.458281in}}%
\pgfpathlineto{\pgfqpoint{3.075000in}{2.455332in}}%
\pgfpathmoveto{\pgfqpoint{3.070459in}{2.458281in}}%
\pgfpathlineto{\pgfqpoint{3.070459in}{2.458281in}}%
\pgfpathlineto{\pgfqpoint{3.070459in}{2.461230in}}%
\pgfpathlineto{\pgfqpoint{3.075000in}{2.461230in}}%
\pgfpathlineto{\pgfqpoint{3.075000in}{2.458281in}}%
\pgfpathmoveto{\pgfqpoint{3.070459in}{2.461230in}}%
\pgfpathlineto{\pgfqpoint{3.070459in}{2.461230in}}%
\pgfpathlineto{\pgfqpoint{3.070459in}{2.464180in}}%
\pgfpathlineto{\pgfqpoint{3.075000in}{2.464180in}}%
\pgfpathlineto{\pgfqpoint{3.075000in}{2.461230in}}%
\pgfpathmoveto{\pgfqpoint{3.070459in}{2.464180in}}%
\pgfpathlineto{\pgfqpoint{3.070459in}{2.464180in}}%
\pgfpathlineto{\pgfqpoint{3.070459in}{2.467129in}}%
\pgfpathlineto{\pgfqpoint{3.075000in}{2.467129in}}%
\pgfpathlineto{\pgfqpoint{3.075000in}{2.464180in}}%
\pgfpathmoveto{\pgfqpoint{3.070459in}{2.467129in}}%
\pgfpathlineto{\pgfqpoint{3.070459in}{2.467129in}}%
\pgfpathlineto{\pgfqpoint{3.070459in}{2.470078in}}%
\pgfpathlineto{\pgfqpoint{3.075000in}{2.470078in}}%
\pgfpathlineto{\pgfqpoint{3.075000in}{2.467129in}}%
\pgfpathmoveto{\pgfqpoint{3.070459in}{2.470078in}}%
\pgfpathlineto{\pgfqpoint{3.070459in}{2.470078in}}%
\pgfpathlineto{\pgfqpoint{3.070459in}{2.473027in}}%
\pgfpathlineto{\pgfqpoint{3.075000in}{2.473027in}}%
\pgfpathlineto{\pgfqpoint{3.075000in}{2.470078in}}%
\pgfpathmoveto{\pgfqpoint{3.070459in}{2.473027in}}%
\pgfpathlineto{\pgfqpoint{3.070459in}{2.473027in}}%
\pgfpathlineto{\pgfqpoint{3.070459in}{2.475976in}}%
\pgfpathlineto{\pgfqpoint{3.075000in}{2.475976in}}%
\pgfpathlineto{\pgfqpoint{3.075000in}{2.473027in}}%
\pgfpathmoveto{\pgfqpoint{3.070459in}{2.475976in}}%
\pgfpathlineto{\pgfqpoint{3.070459in}{2.475976in}}%
\pgfpathlineto{\pgfqpoint{3.070459in}{2.478925in}}%
\pgfpathlineto{\pgfqpoint{3.075000in}{2.478925in}}%
\pgfpathlineto{\pgfqpoint{3.075000in}{2.475976in}}%
\pgfpathmoveto{\pgfqpoint{3.070459in}{2.478925in}}%
\pgfpathlineto{\pgfqpoint{3.070459in}{2.478925in}}%
\pgfpathlineto{\pgfqpoint{3.070459in}{2.481874in}}%
\pgfpathlineto{\pgfqpoint{3.075000in}{2.481874in}}%
\pgfpathlineto{\pgfqpoint{3.075000in}{2.478925in}}%
\pgfpathmoveto{\pgfqpoint{3.070459in}{2.481874in}}%
\pgfpathlineto{\pgfqpoint{3.070459in}{2.481874in}}%
\pgfpathlineto{\pgfqpoint{3.070459in}{2.484824in}}%
\pgfpathlineto{\pgfqpoint{3.075000in}{2.484824in}}%
\pgfpathlineto{\pgfqpoint{3.075000in}{2.481874in}}%
\pgfpathmoveto{\pgfqpoint{3.070459in}{2.484824in}}%
\pgfpathlineto{\pgfqpoint{3.070459in}{2.484824in}}%
\pgfpathlineto{\pgfqpoint{3.070459in}{2.487773in}}%
\pgfpathlineto{\pgfqpoint{3.075000in}{2.487773in}}%
\pgfpathlineto{\pgfqpoint{3.075000in}{2.484824in}}%
\pgfpathmoveto{\pgfqpoint{3.070459in}{2.487773in}}%
\pgfpathlineto{\pgfqpoint{3.070459in}{2.487773in}}%
\pgfpathlineto{\pgfqpoint{3.070459in}{2.490722in}}%
\pgfpathlineto{\pgfqpoint{3.075000in}{2.490722in}}%
\pgfpathlineto{\pgfqpoint{3.075000in}{2.487773in}}%
\pgfpathmoveto{\pgfqpoint{3.070459in}{2.490722in}}%
\pgfpathlineto{\pgfqpoint{3.070459in}{2.490722in}}%
\pgfpathlineto{\pgfqpoint{3.070459in}{2.493672in}}%
\pgfpathlineto{\pgfqpoint{3.075000in}{2.493672in}}%
\pgfpathlineto{\pgfqpoint{3.075000in}{2.490722in}}%
\pgfpathmoveto{\pgfqpoint{3.070459in}{2.493672in}}%
\pgfpathlineto{\pgfqpoint{3.070459in}{2.493672in}}%
\pgfpathlineto{\pgfqpoint{3.070459in}{2.496621in}}%
\pgfpathlineto{\pgfqpoint{3.075000in}{2.496621in}}%
\pgfpathlineto{\pgfqpoint{3.075000in}{2.493672in}}%
\pgfpathmoveto{\pgfqpoint{3.070459in}{2.496621in}}%
\pgfpathlineto{\pgfqpoint{3.070459in}{2.496621in}}%
\pgfpathlineto{\pgfqpoint{3.070459in}{2.499570in}}%
\pgfpathlineto{\pgfqpoint{3.075000in}{2.499570in}}%
\pgfpathlineto{\pgfqpoint{3.075000in}{2.496621in}}%
\pgfpathmoveto{\pgfqpoint{3.070459in}{2.499570in}}%
\pgfpathlineto{\pgfqpoint{3.070459in}{2.499570in}}%
\pgfpathlineto{\pgfqpoint{3.070459in}{2.502520in}}%
\pgfpathlineto{\pgfqpoint{3.075000in}{2.502520in}}%
\pgfpathlineto{\pgfqpoint{3.075000in}{2.499570in}}%
\pgfpathmoveto{\pgfqpoint{3.070459in}{2.502520in}}%
\pgfpathlineto{\pgfqpoint{3.070459in}{2.502520in}}%
\pgfpathlineto{\pgfqpoint{3.070459in}{2.505469in}}%
\pgfpathlineto{\pgfqpoint{3.075000in}{2.505469in}}%
\pgfpathlineto{\pgfqpoint{3.075000in}{2.502520in}}%
\pgfpathmoveto{\pgfqpoint{3.070459in}{2.505469in}}%
\pgfpathlineto{\pgfqpoint{3.070459in}{2.505469in}}%
\pgfpathlineto{\pgfqpoint{3.070459in}{2.508418in}}%
\pgfpathlineto{\pgfqpoint{3.075000in}{2.508418in}}%
\pgfpathlineto{\pgfqpoint{3.075000in}{2.505469in}}%
\pgfpathmoveto{\pgfqpoint{3.070459in}{2.508418in}}%
\pgfpathlineto{\pgfqpoint{3.070459in}{2.508418in}}%
\pgfpathlineto{\pgfqpoint{3.070459in}{2.511367in}}%
\pgfpathlineto{\pgfqpoint{3.075000in}{2.511367in}}%
\pgfpathlineto{\pgfqpoint{3.075000in}{2.508418in}}%
\pgfpathmoveto{\pgfqpoint{3.070459in}{2.511367in}}%
\pgfpathlineto{\pgfqpoint{3.070459in}{2.511367in}}%
\pgfpathlineto{\pgfqpoint{3.070459in}{2.514317in}}%
\pgfpathlineto{\pgfqpoint{3.075000in}{2.514317in}}%
\pgfpathlineto{\pgfqpoint{3.075000in}{2.511367in}}%
\pgfpathmoveto{\pgfqpoint{3.070459in}{2.514317in}}%
\pgfpathlineto{\pgfqpoint{3.070459in}{2.514317in}}%
\pgfpathlineto{\pgfqpoint{3.070459in}{2.517266in}}%
\pgfpathlineto{\pgfqpoint{3.075000in}{2.517266in}}%
\pgfpathlineto{\pgfqpoint{3.075000in}{2.514317in}}%
\pgfpathmoveto{\pgfqpoint{3.070459in}{2.517266in}}%
\pgfpathlineto{\pgfqpoint{3.070459in}{2.517266in}}%
\pgfpathlineto{\pgfqpoint{3.070459in}{2.520215in}}%
\pgfpathlineto{\pgfqpoint{3.075000in}{2.520215in}}%
\pgfpathlineto{\pgfqpoint{3.075000in}{2.517266in}}%
\pgfpathmoveto{\pgfqpoint{3.070459in}{2.520215in}}%
\pgfpathlineto{\pgfqpoint{3.070459in}{2.520215in}}%
\pgfpathlineto{\pgfqpoint{3.070459in}{2.523165in}}%
\pgfpathlineto{\pgfqpoint{3.075000in}{2.523165in}}%
\pgfpathlineto{\pgfqpoint{3.075000in}{2.520215in}}%
\pgfpathmoveto{\pgfqpoint{3.070459in}{2.523165in}}%
\pgfpathlineto{\pgfqpoint{3.070459in}{2.523165in}}%
\pgfpathlineto{\pgfqpoint{3.070459in}{2.526114in}}%
\pgfpathlineto{\pgfqpoint{3.075000in}{2.526114in}}%
\pgfpathlineto{\pgfqpoint{3.075000in}{2.523165in}}%
\pgfpathmoveto{\pgfqpoint{3.070459in}{2.526114in}}%
\pgfpathlineto{\pgfqpoint{3.070459in}{2.526114in}}%
\pgfpathlineto{\pgfqpoint{3.070459in}{2.529063in}}%
\pgfpathlineto{\pgfqpoint{3.075000in}{2.529063in}}%
\pgfpathlineto{\pgfqpoint{3.075000in}{2.526114in}}%
\pgfpathmoveto{\pgfqpoint{3.070459in}{2.529063in}}%
\pgfpathlineto{\pgfqpoint{3.070459in}{2.529063in}}%
\pgfpathlineto{\pgfqpoint{3.070459in}{2.532013in}}%
\pgfpathlineto{\pgfqpoint{3.075000in}{2.532013in}}%
\pgfpathlineto{\pgfqpoint{3.075000in}{2.529063in}}%
\pgfpathmoveto{\pgfqpoint{3.070459in}{2.532013in}}%
\pgfpathlineto{\pgfqpoint{3.070459in}{2.532013in}}%
\pgfpathlineto{\pgfqpoint{3.070459in}{2.534962in}}%
\pgfpathlineto{\pgfqpoint{3.075000in}{2.534962in}}%
\pgfpathlineto{\pgfqpoint{3.075000in}{2.532013in}}%
\pgfpathmoveto{\pgfqpoint{3.070459in}{2.534962in}}%
\pgfpathlineto{\pgfqpoint{3.070459in}{2.534962in}}%
\pgfpathlineto{\pgfqpoint{3.070459in}{2.537911in}}%
\pgfpathlineto{\pgfqpoint{3.075000in}{2.537911in}}%
\pgfpathlineto{\pgfqpoint{3.075000in}{2.534962in}}%
\pgfpathmoveto{\pgfqpoint{3.070459in}{2.537911in}}%
\pgfpathlineto{\pgfqpoint{3.070459in}{2.537911in}}%
\pgfpathlineto{\pgfqpoint{3.070459in}{2.540860in}}%
\pgfpathlineto{\pgfqpoint{3.075000in}{2.540860in}}%
\pgfpathlineto{\pgfqpoint{3.075000in}{2.537911in}}%
\pgfpathmoveto{\pgfqpoint{3.070459in}{2.540860in}}%
\pgfpathlineto{\pgfqpoint{3.070459in}{2.540860in}}%
\pgfpathlineto{\pgfqpoint{3.070459in}{2.543810in}}%
\pgfpathlineto{\pgfqpoint{3.075000in}{2.543810in}}%
\pgfpathlineto{\pgfqpoint{3.075000in}{2.540860in}}%
\pgfpathmoveto{\pgfqpoint{3.070459in}{2.543810in}}%
\pgfpathlineto{\pgfqpoint{3.070459in}{2.543810in}}%
\pgfpathlineto{\pgfqpoint{3.070459in}{2.546759in}}%
\pgfpathlineto{\pgfqpoint{3.075000in}{2.546759in}}%
\pgfpathlineto{\pgfqpoint{3.075000in}{2.543810in}}%
\pgfpathmoveto{\pgfqpoint{3.070459in}{2.546759in}}%
\pgfpathlineto{\pgfqpoint{3.070459in}{2.546759in}}%
\pgfpathlineto{\pgfqpoint{3.070459in}{2.549708in}}%
\pgfpathlineto{\pgfqpoint{3.075000in}{2.549708in}}%
\pgfpathlineto{\pgfqpoint{3.075000in}{2.546759in}}%
\pgfpathmoveto{\pgfqpoint{3.070459in}{2.549708in}}%
\pgfpathlineto{\pgfqpoint{3.070459in}{2.549708in}}%
\pgfpathlineto{\pgfqpoint{3.070459in}{2.552658in}}%
\pgfpathlineto{\pgfqpoint{3.075000in}{2.552658in}}%
\pgfpathlineto{\pgfqpoint{3.075000in}{2.549708in}}%
\pgfpathmoveto{\pgfqpoint{3.070459in}{2.552658in}}%
\pgfpathlineto{\pgfqpoint{3.070459in}{2.552658in}}%
\pgfpathlineto{\pgfqpoint{3.070459in}{2.555607in}}%
\pgfpathlineto{\pgfqpoint{3.075000in}{2.555607in}}%
\pgfpathlineto{\pgfqpoint{3.075000in}{2.552658in}}%
\pgfpathmoveto{\pgfqpoint{3.070459in}{2.555607in}}%
\pgfpathlineto{\pgfqpoint{3.070459in}{2.555607in}}%
\pgfpathlineto{\pgfqpoint{3.070459in}{2.558556in}}%
\pgfpathlineto{\pgfqpoint{3.075000in}{2.558556in}}%
\pgfpathlineto{\pgfqpoint{3.075000in}{2.555607in}}%
\pgfpathmoveto{\pgfqpoint{3.070459in}{2.558556in}}%
\pgfpathlineto{\pgfqpoint{3.070459in}{2.558556in}}%
\pgfpathlineto{\pgfqpoint{3.070459in}{2.561506in}}%
\pgfpathlineto{\pgfqpoint{3.075000in}{2.561506in}}%
\pgfpathlineto{\pgfqpoint{3.075000in}{2.558556in}}%
\pgfpathmoveto{\pgfqpoint{3.070459in}{2.561506in}}%
\pgfpathlineto{\pgfqpoint{3.070459in}{2.561506in}}%
\pgfpathlineto{\pgfqpoint{3.070459in}{2.564455in}}%
\pgfpathlineto{\pgfqpoint{3.075000in}{2.564455in}}%
\pgfpathlineto{\pgfqpoint{3.075000in}{2.561506in}}%
\pgfpathmoveto{\pgfqpoint{3.070459in}{2.564455in}}%
\pgfpathlineto{\pgfqpoint{3.070459in}{2.564455in}}%
\pgfpathlineto{\pgfqpoint{3.070459in}{2.567404in}}%
\pgfpathlineto{\pgfqpoint{3.075000in}{2.567404in}}%
\pgfpathlineto{\pgfqpoint{3.075000in}{2.564455in}}%
\pgfpathmoveto{\pgfqpoint{3.070459in}{2.567404in}}%
\pgfpathlineto{\pgfqpoint{3.070459in}{2.567404in}}%
\pgfpathlineto{\pgfqpoint{3.070459in}{2.570353in}}%
\pgfpathlineto{\pgfqpoint{3.075000in}{2.570353in}}%
\pgfpathlineto{\pgfqpoint{3.075000in}{2.567404in}}%
\pgfpathmoveto{\pgfqpoint{3.070459in}{2.570353in}}%
\pgfpathlineto{\pgfqpoint{3.070459in}{2.570353in}}%
\pgfpathlineto{\pgfqpoint{3.070459in}{2.573303in}}%
\pgfpathlineto{\pgfqpoint{3.075000in}{2.573303in}}%
\pgfpathlineto{\pgfqpoint{3.075000in}{2.570353in}}%
\pgfpathmoveto{\pgfqpoint{3.070459in}{2.573303in}}%
\pgfpathlineto{\pgfqpoint{3.070459in}{2.573303in}}%
\pgfpathlineto{\pgfqpoint{3.070459in}{2.576252in}}%
\pgfpathlineto{\pgfqpoint{3.075000in}{2.576252in}}%
\pgfpathlineto{\pgfqpoint{3.075000in}{2.573303in}}%
\pgfpathmoveto{\pgfqpoint{3.070459in}{2.576252in}}%
\pgfpathlineto{\pgfqpoint{3.070459in}{2.576252in}}%
\pgfpathlineto{\pgfqpoint{3.070459in}{2.579201in}}%
\pgfpathlineto{\pgfqpoint{3.075000in}{2.579201in}}%
\pgfpathlineto{\pgfqpoint{3.075000in}{2.576252in}}%
\pgfpathmoveto{\pgfqpoint{3.070459in}{2.579201in}}%
\pgfpathlineto{\pgfqpoint{3.070459in}{2.579201in}}%
\pgfpathlineto{\pgfqpoint{3.070459in}{2.582150in}}%
\pgfpathlineto{\pgfqpoint{3.075000in}{2.582150in}}%
\pgfpathlineto{\pgfqpoint{3.075000in}{2.579201in}}%
\pgfpathmoveto{\pgfqpoint{3.070459in}{2.582150in}}%
\pgfpathlineto{\pgfqpoint{3.070459in}{2.582150in}}%
\pgfpathlineto{\pgfqpoint{3.070459in}{2.585100in}}%
\pgfpathlineto{\pgfqpoint{3.075000in}{2.585100in}}%
\pgfpathlineto{\pgfqpoint{3.075000in}{2.582150in}}%
\pgfpathmoveto{\pgfqpoint{3.070459in}{2.585100in}}%
\pgfpathlineto{\pgfqpoint{3.070459in}{2.585100in}}%
\pgfpathlineto{\pgfqpoint{3.070459in}{2.588049in}}%
\pgfpathlineto{\pgfqpoint{3.075000in}{2.588049in}}%
\pgfpathlineto{\pgfqpoint{3.075000in}{2.585100in}}%
\pgfpathmoveto{\pgfqpoint{3.070459in}{2.588049in}}%
\pgfpathlineto{\pgfqpoint{3.070459in}{2.588049in}}%
\pgfpathlineto{\pgfqpoint{3.070459in}{2.590998in}}%
\pgfpathlineto{\pgfqpoint{3.075000in}{2.590998in}}%
\pgfpathlineto{\pgfqpoint{3.075000in}{2.588049in}}%
\pgfpathmoveto{\pgfqpoint{3.070459in}{2.590998in}}%
\pgfpathlineto{\pgfqpoint{3.070459in}{2.590998in}}%
\pgfpathlineto{\pgfqpoint{3.070459in}{2.593947in}}%
\pgfpathlineto{\pgfqpoint{3.075000in}{2.593947in}}%
\pgfpathlineto{\pgfqpoint{3.075000in}{2.590998in}}%
\pgfpathmoveto{\pgfqpoint{3.070459in}{2.593947in}}%
\pgfpathlineto{\pgfqpoint{3.070459in}{2.593947in}}%
\pgfpathlineto{\pgfqpoint{3.070459in}{2.596896in}}%
\pgfpathlineto{\pgfqpoint{3.075000in}{2.596896in}}%
\pgfpathlineto{\pgfqpoint{3.075000in}{2.593947in}}%
\pgfpathmoveto{\pgfqpoint{3.070459in}{2.596896in}}%
\pgfpathlineto{\pgfqpoint{3.070459in}{2.596896in}}%
\pgfpathlineto{\pgfqpoint{3.070459in}{2.599845in}}%
\pgfpathlineto{\pgfqpoint{3.075000in}{2.599845in}}%
\pgfpathlineto{\pgfqpoint{3.075000in}{2.596896in}}%
\pgfpathmoveto{\pgfqpoint{3.070459in}{2.599845in}}%
\pgfpathlineto{\pgfqpoint{3.070459in}{2.599845in}}%
\pgfpathlineto{\pgfqpoint{3.070459in}{2.602795in}}%
\pgfpathlineto{\pgfqpoint{3.075000in}{2.602795in}}%
\pgfpathlineto{\pgfqpoint{3.075000in}{2.599845in}}%
\pgfpathmoveto{\pgfqpoint{3.070459in}{2.602795in}}%
\pgfpathlineto{\pgfqpoint{3.070459in}{2.602795in}}%
\pgfpathlineto{\pgfqpoint{3.070459in}{2.605744in}}%
\pgfpathlineto{\pgfqpoint{3.075000in}{2.605744in}}%
\pgfpathlineto{\pgfqpoint{3.075000in}{2.602795in}}%
\pgfpathmoveto{\pgfqpoint{3.070459in}{2.605744in}}%
\pgfpathlineto{\pgfqpoint{3.070459in}{2.605744in}}%
\pgfpathlineto{\pgfqpoint{3.070459in}{2.608693in}}%
\pgfpathlineto{\pgfqpoint{3.075000in}{2.608693in}}%
\pgfpathlineto{\pgfqpoint{3.075000in}{2.605744in}}%
\pgfpathmoveto{\pgfqpoint{3.070459in}{2.608693in}}%
\pgfpathlineto{\pgfqpoint{3.070459in}{2.608693in}}%
\pgfpathlineto{\pgfqpoint{3.070459in}{2.611642in}}%
\pgfpathlineto{\pgfqpoint{3.075000in}{2.611642in}}%
\pgfpathlineto{\pgfqpoint{3.075000in}{2.608693in}}%
\pgfpathmoveto{\pgfqpoint{3.070459in}{2.611642in}}%
\pgfpathlineto{\pgfqpoint{3.070459in}{2.611642in}}%
\pgfpathlineto{\pgfqpoint{3.070459in}{2.614591in}}%
\pgfpathlineto{\pgfqpoint{3.075000in}{2.614591in}}%
\pgfpathlineto{\pgfqpoint{3.075000in}{2.611642in}}%
\pgfpathmoveto{\pgfqpoint{3.070459in}{2.614591in}}%
\pgfpathlineto{\pgfqpoint{3.070459in}{2.614591in}}%
\pgfpathlineto{\pgfqpoint{3.070459in}{2.617540in}}%
\pgfpathlineto{\pgfqpoint{3.075000in}{2.617540in}}%
\pgfpathlineto{\pgfqpoint{3.075000in}{2.614591in}}%
\pgfpathmoveto{\pgfqpoint{3.070459in}{2.617540in}}%
\pgfpathlineto{\pgfqpoint{3.070459in}{2.617540in}}%
\pgfpathlineto{\pgfqpoint{3.070459in}{2.620490in}}%
\pgfpathlineto{\pgfqpoint{3.075000in}{2.620490in}}%
\pgfpathlineto{\pgfqpoint{3.075000in}{2.617540in}}%
\pgfpathmoveto{\pgfqpoint{3.070459in}{2.620490in}}%
\pgfpathlineto{\pgfqpoint{3.070459in}{2.620490in}}%
\pgfpathlineto{\pgfqpoint{3.070459in}{2.623439in}}%
\pgfpathlineto{\pgfqpoint{3.075000in}{2.623439in}}%
\pgfpathlineto{\pgfqpoint{3.075000in}{2.620490in}}%
\pgfpathmoveto{\pgfqpoint{3.070459in}{2.623439in}}%
\pgfpathlineto{\pgfqpoint{3.070459in}{2.623439in}}%
\pgfpathlineto{\pgfqpoint{3.070459in}{2.626388in}}%
\pgfpathlineto{\pgfqpoint{3.075000in}{2.626388in}}%
\pgfpathlineto{\pgfqpoint{3.075000in}{2.623439in}}%
\pgfpathmoveto{\pgfqpoint{3.070459in}{2.626388in}}%
\pgfpathlineto{\pgfqpoint{3.070459in}{2.626388in}}%
\pgfpathlineto{\pgfqpoint{3.070459in}{2.629337in}}%
\pgfpathlineto{\pgfqpoint{3.075000in}{2.629337in}}%
\pgfpathlineto{\pgfqpoint{3.075000in}{2.626388in}}%
\pgfpathmoveto{\pgfqpoint{3.070459in}{2.629337in}}%
\pgfpathlineto{\pgfqpoint{3.070459in}{2.629337in}}%
\pgfpathlineto{\pgfqpoint{3.070459in}{2.632286in}}%
\pgfpathlineto{\pgfqpoint{3.075000in}{2.632286in}}%
\pgfpathlineto{\pgfqpoint{3.075000in}{2.629337in}}%
\pgfpathmoveto{\pgfqpoint{3.070459in}{2.632286in}}%
\pgfpathlineto{\pgfqpoint{3.070459in}{2.632286in}}%
\pgfpathlineto{\pgfqpoint{3.070459in}{2.635236in}}%
\pgfpathlineto{\pgfqpoint{3.075000in}{2.635236in}}%
\pgfpathlineto{\pgfqpoint{3.075000in}{2.632286in}}%
\pgfpathmoveto{\pgfqpoint{3.070459in}{2.635236in}}%
\pgfpathlineto{\pgfqpoint{3.070459in}{2.635236in}}%
\pgfpathlineto{\pgfqpoint{3.070459in}{2.638185in}}%
\pgfpathlineto{\pgfqpoint{3.075000in}{2.638185in}}%
\pgfpathlineto{\pgfqpoint{3.075000in}{2.635236in}}%
\pgfpathmoveto{\pgfqpoint{3.070459in}{2.638185in}}%
\pgfpathlineto{\pgfqpoint{3.070459in}{2.638185in}}%
\pgfpathlineto{\pgfqpoint{3.070459in}{2.641134in}}%
\pgfpathlineto{\pgfqpoint{3.075000in}{2.641134in}}%
\pgfpathlineto{\pgfqpoint{3.075000in}{2.638185in}}%
\pgfpathmoveto{\pgfqpoint{3.070459in}{2.641134in}}%
\pgfpathlineto{\pgfqpoint{3.070459in}{2.641134in}}%
\pgfpathlineto{\pgfqpoint{3.070459in}{2.644083in}}%
\pgfpathlineto{\pgfqpoint{3.075000in}{2.644083in}}%
\pgfpathlineto{\pgfqpoint{3.075000in}{2.641134in}}%
\pgfpathmoveto{\pgfqpoint{3.070459in}{2.644083in}}%
\pgfpathlineto{\pgfqpoint{3.070459in}{2.644083in}}%
\pgfpathlineto{\pgfqpoint{3.070459in}{2.647032in}}%
\pgfpathlineto{\pgfqpoint{3.075000in}{2.647032in}}%
\pgfpathlineto{\pgfqpoint{3.075000in}{2.644083in}}%
\pgfpathmoveto{\pgfqpoint{3.070459in}{2.647032in}}%
\pgfpathlineto{\pgfqpoint{3.070459in}{2.647032in}}%
\pgfpathlineto{\pgfqpoint{3.070459in}{2.649981in}}%
\pgfpathlineto{\pgfqpoint{3.075000in}{2.649981in}}%
\pgfpathlineto{\pgfqpoint{3.075000in}{2.647032in}}%
\pgfpathmoveto{\pgfqpoint{3.070459in}{2.649981in}}%
\pgfpathlineto{\pgfqpoint{3.070459in}{2.649981in}}%
\pgfpathlineto{\pgfqpoint{3.070459in}{2.652931in}}%
\pgfpathlineto{\pgfqpoint{3.075000in}{2.652931in}}%
\pgfpathlineto{\pgfqpoint{3.075000in}{2.649981in}}%
\pgfpathmoveto{\pgfqpoint{3.070459in}{2.652931in}}%
\pgfpathlineto{\pgfqpoint{3.070459in}{2.652931in}}%
\pgfpathlineto{\pgfqpoint{3.070459in}{2.655880in}}%
\pgfpathlineto{\pgfqpoint{3.075000in}{2.655880in}}%
\pgfpathlineto{\pgfqpoint{3.075000in}{2.652931in}}%
\pgfpathmoveto{\pgfqpoint{3.070459in}{2.655880in}}%
\pgfpathlineto{\pgfqpoint{3.070459in}{2.655880in}}%
\pgfpathlineto{\pgfqpoint{3.070459in}{2.658829in}}%
\pgfpathlineto{\pgfqpoint{3.075000in}{2.658829in}}%
\pgfpathlineto{\pgfqpoint{3.075000in}{2.655880in}}%
\pgfpathmoveto{\pgfqpoint{3.070459in}{2.658829in}}%
\pgfpathlineto{\pgfqpoint{3.070459in}{2.658829in}}%
\pgfpathlineto{\pgfqpoint{3.070459in}{2.661778in}}%
\pgfpathlineto{\pgfqpoint{3.075000in}{2.661778in}}%
\pgfpathlineto{\pgfqpoint{3.075000in}{2.658829in}}%
\pgfpathmoveto{\pgfqpoint{3.070459in}{2.661778in}}%
\pgfpathlineto{\pgfqpoint{3.070459in}{2.661778in}}%
\pgfpathlineto{\pgfqpoint{3.070459in}{2.664727in}}%
\pgfpathlineto{\pgfqpoint{3.075000in}{2.664727in}}%
\pgfpathlineto{\pgfqpoint{3.075000in}{2.661778in}}%
\pgfpathmoveto{\pgfqpoint{3.070459in}{2.664727in}}%
\pgfpathlineto{\pgfqpoint{3.070459in}{2.664727in}}%
\pgfpathlineto{\pgfqpoint{3.070459in}{2.667676in}}%
\pgfpathlineto{\pgfqpoint{3.075000in}{2.667676in}}%
\pgfpathlineto{\pgfqpoint{3.075000in}{2.664727in}}%
\pgfpathmoveto{\pgfqpoint{3.070459in}{2.667676in}}%
\pgfpathlineto{\pgfqpoint{3.070459in}{2.667676in}}%
\pgfpathlineto{\pgfqpoint{3.070459in}{2.670626in}}%
\pgfpathlineto{\pgfqpoint{3.075000in}{2.670626in}}%
\pgfpathlineto{\pgfqpoint{3.075000in}{2.667676in}}%
\pgfpathmoveto{\pgfqpoint{3.070459in}{2.670626in}}%
\pgfpathlineto{\pgfqpoint{3.070459in}{2.670626in}}%
\pgfpathlineto{\pgfqpoint{3.070459in}{2.673575in}}%
\pgfpathlineto{\pgfqpoint{3.075000in}{2.673575in}}%
\pgfpathlineto{\pgfqpoint{3.075000in}{2.670626in}}%
\pgfpathmoveto{\pgfqpoint{3.070459in}{2.673575in}}%
\pgfpathlineto{\pgfqpoint{3.070459in}{2.673575in}}%
\pgfpathlineto{\pgfqpoint{3.070459in}{2.676524in}}%
\pgfpathlineto{\pgfqpoint{3.075000in}{2.676524in}}%
\pgfpathlineto{\pgfqpoint{3.075000in}{2.673575in}}%
\pgfpathmoveto{\pgfqpoint{3.070459in}{2.676524in}}%
\pgfpathlineto{\pgfqpoint{3.070459in}{2.676524in}}%
\pgfpathlineto{\pgfqpoint{3.070459in}{2.679473in}}%
\pgfpathlineto{\pgfqpoint{3.075000in}{2.679473in}}%
\pgfpathlineto{\pgfqpoint{3.075000in}{2.676524in}}%
\pgfpathmoveto{\pgfqpoint{3.070459in}{2.679473in}}%
\pgfpathlineto{\pgfqpoint{3.070459in}{2.679473in}}%
\pgfpathlineto{\pgfqpoint{3.070459in}{2.682422in}}%
\pgfpathlineto{\pgfqpoint{3.075000in}{2.682422in}}%
\pgfpathlineto{\pgfqpoint{3.075000in}{2.679473in}}%
\pgfpathmoveto{\pgfqpoint{3.070459in}{2.682422in}}%
\pgfpathlineto{\pgfqpoint{3.070459in}{2.682422in}}%
\pgfpathlineto{\pgfqpoint{3.070459in}{2.685371in}}%
\pgfpathlineto{\pgfqpoint{3.075000in}{2.685371in}}%
\pgfpathlineto{\pgfqpoint{3.075000in}{2.682422in}}%
\pgfpathmoveto{\pgfqpoint{3.070459in}{2.685371in}}%
\pgfpathlineto{\pgfqpoint{3.070459in}{2.685371in}}%
\pgfpathlineto{\pgfqpoint{3.070459in}{2.688321in}}%
\pgfpathlineto{\pgfqpoint{3.075000in}{2.688321in}}%
\pgfpathlineto{\pgfqpoint{3.075000in}{2.685371in}}%
\pgfpathmoveto{\pgfqpoint{3.070459in}{2.688321in}}%
\pgfpathlineto{\pgfqpoint{3.070459in}{2.688321in}}%
\pgfpathlineto{\pgfqpoint{3.070459in}{2.691270in}}%
\pgfpathlineto{\pgfqpoint{3.075000in}{2.691270in}}%
\pgfpathlineto{\pgfqpoint{3.075000in}{2.688321in}}%
\pgfpathmoveto{\pgfqpoint{3.070459in}{2.691270in}}%
\pgfpathlineto{\pgfqpoint{3.070459in}{2.691270in}}%
\pgfpathlineto{\pgfqpoint{3.070459in}{2.694219in}}%
\pgfpathlineto{\pgfqpoint{3.075000in}{2.694219in}}%
\pgfpathlineto{\pgfqpoint{3.075000in}{2.691270in}}%
\pgfpathmoveto{\pgfqpoint{3.070459in}{2.694219in}}%
\pgfpathlineto{\pgfqpoint{3.070459in}{2.694219in}}%
\pgfpathlineto{\pgfqpoint{3.070459in}{2.697168in}}%
\pgfpathlineto{\pgfqpoint{3.075000in}{2.697168in}}%
\pgfpathlineto{\pgfqpoint{3.075000in}{2.694219in}}%
\pgfpathmoveto{\pgfqpoint{3.070459in}{2.697168in}}%
\pgfpathlineto{\pgfqpoint{3.070459in}{2.697168in}}%
\pgfpathlineto{\pgfqpoint{3.070459in}{2.700117in}}%
\pgfpathlineto{\pgfqpoint{3.075000in}{2.700117in}}%
\pgfpathlineto{\pgfqpoint{3.075000in}{2.697168in}}%
\pgfpathmoveto{\pgfqpoint{3.070459in}{2.700117in}}%
\pgfpathlineto{\pgfqpoint{3.070459in}{2.700117in}}%
\pgfpathlineto{\pgfqpoint{3.070459in}{2.703066in}}%
\pgfpathlineto{\pgfqpoint{3.075000in}{2.703066in}}%
\pgfpathlineto{\pgfqpoint{3.075000in}{2.700117in}}%
\pgfpathmoveto{\pgfqpoint{3.070459in}{2.703066in}}%
\pgfpathlineto{\pgfqpoint{3.070459in}{2.703066in}}%
\pgfpathlineto{\pgfqpoint{3.070459in}{2.706016in}}%
\pgfpathlineto{\pgfqpoint{3.075000in}{2.706016in}}%
\pgfpathlineto{\pgfqpoint{3.075000in}{2.703066in}}%
\pgfpathmoveto{\pgfqpoint{3.070459in}{2.706016in}}%
\pgfpathlineto{\pgfqpoint{3.070459in}{2.706016in}}%
\pgfpathlineto{\pgfqpoint{3.070459in}{2.708965in}}%
\pgfpathlineto{\pgfqpoint{3.075000in}{2.708965in}}%
\pgfpathlineto{\pgfqpoint{3.075000in}{2.706016in}}%
\pgfpathmoveto{\pgfqpoint{3.070459in}{2.708965in}}%
\pgfpathlineto{\pgfqpoint{3.070459in}{2.708965in}}%
\pgfpathlineto{\pgfqpoint{3.070459in}{2.711914in}}%
\pgfpathlineto{\pgfqpoint{3.075000in}{2.711914in}}%
\pgfpathlineto{\pgfqpoint{3.075000in}{2.708965in}}%
\pgfpathmoveto{\pgfqpoint{3.070459in}{2.711914in}}%
\pgfpathlineto{\pgfqpoint{3.070459in}{2.711914in}}%
\pgfpathlineto{\pgfqpoint{3.070459in}{2.714863in}}%
\pgfpathlineto{\pgfqpoint{3.075000in}{2.714863in}}%
\pgfpathlineto{\pgfqpoint{3.075000in}{2.711914in}}%
\pgfpathmoveto{\pgfqpoint{3.070459in}{2.714863in}}%
\pgfpathlineto{\pgfqpoint{3.070459in}{2.714863in}}%
\pgfpathlineto{\pgfqpoint{3.070459in}{2.717812in}}%
\pgfpathlineto{\pgfqpoint{3.075000in}{2.717812in}}%
\pgfpathlineto{\pgfqpoint{3.075000in}{2.714863in}}%
\pgfpathmoveto{\pgfqpoint{3.070459in}{2.717812in}}%
\pgfpathlineto{\pgfqpoint{3.070459in}{2.717812in}}%
\pgfpathlineto{\pgfqpoint{3.070459in}{2.720761in}}%
\pgfpathlineto{\pgfqpoint{3.075000in}{2.720761in}}%
\pgfpathlineto{\pgfqpoint{3.075000in}{2.717812in}}%
\pgfpathmoveto{\pgfqpoint{3.070459in}{2.720761in}}%
\pgfpathlineto{\pgfqpoint{3.070459in}{2.720761in}}%
\pgfpathlineto{\pgfqpoint{3.070459in}{2.723711in}}%
\pgfpathlineto{\pgfqpoint{3.075000in}{2.723711in}}%
\pgfpathlineto{\pgfqpoint{3.075000in}{2.720761in}}%
\pgfpathmoveto{\pgfqpoint{3.070459in}{2.723711in}}%
\pgfpathlineto{\pgfqpoint{3.070459in}{2.723711in}}%
\pgfpathlineto{\pgfqpoint{3.070459in}{2.726660in}}%
\pgfpathlineto{\pgfqpoint{3.075000in}{2.726660in}}%
\pgfpathlineto{\pgfqpoint{3.075000in}{2.723711in}}%
\pgfpathmoveto{\pgfqpoint{3.070459in}{2.726660in}}%
\pgfpathlineto{\pgfqpoint{3.070459in}{2.726660in}}%
\pgfpathlineto{\pgfqpoint{3.070459in}{2.729609in}}%
\pgfpathlineto{\pgfqpoint{3.075000in}{2.729609in}}%
\pgfpathlineto{\pgfqpoint{3.075000in}{2.726660in}}%
\pgfpathmoveto{\pgfqpoint{3.070459in}{2.729609in}}%
\pgfpathlineto{\pgfqpoint{3.070459in}{2.729609in}}%
\pgfpathlineto{\pgfqpoint{3.070459in}{2.732558in}}%
\pgfpathlineto{\pgfqpoint{3.075000in}{2.732558in}}%
\pgfpathlineto{\pgfqpoint{3.075000in}{2.729609in}}%
\pgfpathmoveto{\pgfqpoint{3.070459in}{2.732558in}}%
\pgfpathlineto{\pgfqpoint{3.070459in}{2.732558in}}%
\pgfpathlineto{\pgfqpoint{3.070459in}{2.735507in}}%
\pgfpathlineto{\pgfqpoint{3.075000in}{2.735507in}}%
\pgfpathlineto{\pgfqpoint{3.075000in}{2.732558in}}%
\pgfpathmoveto{\pgfqpoint{3.070459in}{2.735507in}}%
\pgfpathlineto{\pgfqpoint{3.070459in}{2.735507in}}%
\pgfpathlineto{\pgfqpoint{3.070459in}{2.738456in}}%
\pgfpathlineto{\pgfqpoint{3.075000in}{2.738456in}}%
\pgfpathlineto{\pgfqpoint{3.075000in}{2.735507in}}%
\pgfpathmoveto{\pgfqpoint{3.070459in}{2.738456in}}%
\pgfpathlineto{\pgfqpoint{3.070459in}{2.738456in}}%
\pgfpathlineto{\pgfqpoint{3.070459in}{2.741406in}}%
\pgfpathlineto{\pgfqpoint{3.075000in}{2.741406in}}%
\pgfpathlineto{\pgfqpoint{3.075000in}{2.738456in}}%
\pgfpathmoveto{\pgfqpoint{3.070459in}{2.741406in}}%
\pgfpathlineto{\pgfqpoint{3.070459in}{2.741406in}}%
\pgfpathlineto{\pgfqpoint{3.070459in}{2.744355in}}%
\pgfpathlineto{\pgfqpoint{3.075000in}{2.744355in}}%
\pgfpathlineto{\pgfqpoint{3.075000in}{2.741406in}}%
\pgfpathmoveto{\pgfqpoint{3.070459in}{2.744355in}}%
\pgfpathlineto{\pgfqpoint{3.070459in}{2.744355in}}%
\pgfpathlineto{\pgfqpoint{3.070459in}{2.747304in}}%
\pgfpathlineto{\pgfqpoint{3.075000in}{2.747304in}}%
\pgfpathlineto{\pgfqpoint{3.075000in}{2.744355in}}%
\pgfpathmoveto{\pgfqpoint{3.070459in}{2.747304in}}%
\pgfpathlineto{\pgfqpoint{3.070459in}{2.747304in}}%
\pgfpathlineto{\pgfqpoint{3.070459in}{2.750253in}}%
\pgfpathlineto{\pgfqpoint{3.075000in}{2.750253in}}%
\pgfpathlineto{\pgfqpoint{3.075000in}{2.747304in}}%
\pgfpathmoveto{\pgfqpoint{3.070459in}{2.750253in}}%
\pgfpathlineto{\pgfqpoint{3.070459in}{2.750253in}}%
\pgfpathlineto{\pgfqpoint{3.070459in}{2.753202in}}%
\pgfpathlineto{\pgfqpoint{3.075000in}{2.753202in}}%
\pgfpathlineto{\pgfqpoint{3.075000in}{2.750253in}}%
\pgfpathmoveto{\pgfqpoint{3.070459in}{2.753202in}}%
\pgfpathlineto{\pgfqpoint{3.070459in}{2.753202in}}%
\pgfpathlineto{\pgfqpoint{3.070459in}{2.756151in}}%
\pgfpathlineto{\pgfqpoint{3.075000in}{2.756151in}}%
\pgfpathlineto{\pgfqpoint{3.075000in}{2.753202in}}%
\pgfpathmoveto{\pgfqpoint{3.070459in}{2.756151in}}%
\pgfpathlineto{\pgfqpoint{3.070459in}{2.756151in}}%
\pgfpathlineto{\pgfqpoint{3.070459in}{2.759101in}}%
\pgfpathlineto{\pgfqpoint{3.075000in}{2.759101in}}%
\pgfpathlineto{\pgfqpoint{3.075000in}{2.756151in}}%
\pgfpathmoveto{\pgfqpoint{3.070459in}{2.759101in}}%
\pgfpathlineto{\pgfqpoint{3.070459in}{2.759101in}}%
\pgfpathlineto{\pgfqpoint{3.070459in}{2.762050in}}%
\pgfpathlineto{\pgfqpoint{3.075000in}{2.762050in}}%
\pgfpathlineto{\pgfqpoint{3.075000in}{2.759101in}}%
\pgfpathmoveto{\pgfqpoint{3.070459in}{2.762050in}}%
\pgfpathlineto{\pgfqpoint{3.070459in}{2.762050in}}%
\pgfpathlineto{\pgfqpoint{3.070459in}{2.764999in}}%
\pgfpathlineto{\pgfqpoint{3.075000in}{2.764999in}}%
\pgfpathlineto{\pgfqpoint{3.075000in}{2.762050in}}%
\pgfpathmoveto{\pgfqpoint{3.070459in}{2.764999in}}%
\pgfpathlineto{\pgfqpoint{3.070459in}{2.764999in}}%
\pgfpathlineto{\pgfqpoint{3.070459in}{2.767948in}}%
\pgfpathlineto{\pgfqpoint{3.075000in}{2.767948in}}%
\pgfpathlineto{\pgfqpoint{3.075000in}{2.764999in}}%
\pgfpathmoveto{\pgfqpoint{3.070459in}{2.767948in}}%
\pgfpathlineto{\pgfqpoint{3.070459in}{2.767948in}}%
\pgfpathlineto{\pgfqpoint{3.070459in}{2.770897in}}%
\pgfpathlineto{\pgfqpoint{3.075000in}{2.770897in}}%
\pgfpathlineto{\pgfqpoint{3.075000in}{2.767948in}}%
\pgfpathmoveto{\pgfqpoint{3.070459in}{2.770897in}}%
\pgfpathlineto{\pgfqpoint{3.070459in}{2.770897in}}%
\pgfpathlineto{\pgfqpoint{3.070459in}{2.773847in}}%
\pgfpathlineto{\pgfqpoint{3.075000in}{2.773847in}}%
\pgfpathlineto{\pgfqpoint{3.075000in}{2.770897in}}%
\pgfpathmoveto{\pgfqpoint{3.070459in}{2.773847in}}%
\pgfpathlineto{\pgfqpoint{3.070459in}{2.773847in}}%
\pgfpathlineto{\pgfqpoint{3.070459in}{2.776796in}}%
\pgfpathlineto{\pgfqpoint{3.075000in}{2.776796in}}%
\pgfpathlineto{\pgfqpoint{3.075000in}{2.773847in}}%
\pgfpathmoveto{\pgfqpoint{3.070459in}{2.776796in}}%
\pgfpathlineto{\pgfqpoint{3.070459in}{2.776796in}}%
\pgfpathlineto{\pgfqpoint{3.070459in}{2.779745in}}%
\pgfpathlineto{\pgfqpoint{3.075000in}{2.779745in}}%
\pgfpathlineto{\pgfqpoint{3.075000in}{2.776796in}}%
\pgfpathmoveto{\pgfqpoint{3.070459in}{2.779745in}}%
\pgfpathlineto{\pgfqpoint{3.070459in}{2.779745in}}%
\pgfpathlineto{\pgfqpoint{3.070459in}{2.782694in}}%
\pgfpathlineto{\pgfqpoint{3.075000in}{2.782694in}}%
\pgfpathlineto{\pgfqpoint{3.075000in}{2.779745in}}%
\pgfpathmoveto{\pgfqpoint{3.070459in}{2.782694in}}%
\pgfpathlineto{\pgfqpoint{3.070459in}{2.782694in}}%
\pgfpathlineto{\pgfqpoint{3.070459in}{2.785644in}}%
\pgfpathlineto{\pgfqpoint{3.075000in}{2.785644in}}%
\pgfpathlineto{\pgfqpoint{3.075000in}{2.782694in}}%
\pgfpathmoveto{\pgfqpoint{3.070459in}{2.785644in}}%
\pgfpathlineto{\pgfqpoint{3.070459in}{2.785644in}}%
\pgfpathlineto{\pgfqpoint{3.070459in}{2.788593in}}%
\pgfpathlineto{\pgfqpoint{3.075000in}{2.788593in}}%
\pgfpathlineto{\pgfqpoint{3.075000in}{2.785644in}}%
\pgfpathmoveto{\pgfqpoint{3.070459in}{2.788593in}}%
\pgfpathlineto{\pgfqpoint{3.070459in}{2.788593in}}%
\pgfpathlineto{\pgfqpoint{3.070459in}{2.791542in}}%
\pgfpathlineto{\pgfqpoint{3.075000in}{2.791542in}}%
\pgfpathlineto{\pgfqpoint{3.075000in}{2.788593in}}%
\pgfpathmoveto{\pgfqpoint{3.070459in}{2.791542in}}%
\pgfpathlineto{\pgfqpoint{3.070459in}{2.791542in}}%
\pgfpathlineto{\pgfqpoint{3.070459in}{2.794492in}}%
\pgfpathlineto{\pgfqpoint{3.075000in}{2.794492in}}%
\pgfpathlineto{\pgfqpoint{3.075000in}{2.791542in}}%
\pgfpathmoveto{\pgfqpoint{3.070459in}{2.794492in}}%
\pgfpathlineto{\pgfqpoint{3.070459in}{2.794492in}}%
\pgfpathlineto{\pgfqpoint{3.070459in}{2.797441in}}%
\pgfpathlineto{\pgfqpoint{3.075000in}{2.797441in}}%
\pgfpathlineto{\pgfqpoint{3.075000in}{2.794492in}}%
\pgfpathmoveto{\pgfqpoint{3.070459in}{2.797441in}}%
\pgfpathlineto{\pgfqpoint{3.070459in}{2.797441in}}%
\pgfpathlineto{\pgfqpoint{3.070459in}{2.800390in}}%
\pgfpathlineto{\pgfqpoint{3.075000in}{2.800390in}}%
\pgfpathlineto{\pgfqpoint{3.075000in}{2.797441in}}%
\pgfpathmoveto{\pgfqpoint{3.070459in}{2.800390in}}%
\pgfpathlineto{\pgfqpoint{3.070459in}{2.800390in}}%
\pgfpathlineto{\pgfqpoint{3.070459in}{2.803339in}}%
\pgfpathlineto{\pgfqpoint{3.075000in}{2.803339in}}%
\pgfpathlineto{\pgfqpoint{3.075000in}{2.800390in}}%
\pgfpathmoveto{\pgfqpoint{3.070459in}{2.803339in}}%
\pgfpathlineto{\pgfqpoint{3.070459in}{2.803339in}}%
\pgfpathlineto{\pgfqpoint{3.070459in}{2.806289in}}%
\pgfpathlineto{\pgfqpoint{3.075000in}{2.806289in}}%
\pgfpathlineto{\pgfqpoint{3.075000in}{2.803339in}}%
\pgfpathmoveto{\pgfqpoint{3.070459in}{2.806289in}}%
\pgfpathlineto{\pgfqpoint{3.070459in}{2.806289in}}%
\pgfpathlineto{\pgfqpoint{3.070459in}{2.809238in}}%
\pgfpathlineto{\pgfqpoint{3.075000in}{2.809238in}}%
\pgfpathlineto{\pgfqpoint{3.075000in}{2.806289in}}%
\pgfpathmoveto{\pgfqpoint{3.070459in}{2.809238in}}%
\pgfpathlineto{\pgfqpoint{3.070459in}{2.809238in}}%
\pgfpathlineto{\pgfqpoint{3.070459in}{2.812187in}}%
\pgfpathlineto{\pgfqpoint{3.075000in}{2.812187in}}%
\pgfpathlineto{\pgfqpoint{3.075000in}{2.809238in}}%
\pgfpathmoveto{\pgfqpoint{3.070459in}{2.812187in}}%
\pgfpathlineto{\pgfqpoint{3.070459in}{2.812187in}}%
\pgfpathlineto{\pgfqpoint{3.070459in}{2.815136in}}%
\pgfpathlineto{\pgfqpoint{3.075000in}{2.815136in}}%
\pgfpathlineto{\pgfqpoint{3.075000in}{2.812187in}}%
\pgfpathmoveto{\pgfqpoint{3.070459in}{2.815136in}}%
\pgfpathlineto{\pgfqpoint{3.070459in}{2.815136in}}%
\pgfpathlineto{\pgfqpoint{3.070459in}{2.818086in}}%
\pgfpathlineto{\pgfqpoint{3.075000in}{2.818086in}}%
\pgfpathlineto{\pgfqpoint{3.075000in}{2.815136in}}%
\pgfpathmoveto{\pgfqpoint{3.070459in}{2.818086in}}%
\pgfpathlineto{\pgfqpoint{3.070459in}{2.818086in}}%
\pgfpathlineto{\pgfqpoint{3.070459in}{2.821035in}}%
\pgfpathlineto{\pgfqpoint{3.075000in}{2.821035in}}%
\pgfpathlineto{\pgfqpoint{3.075000in}{2.818086in}}%
\pgfpathmoveto{\pgfqpoint{3.070459in}{2.821035in}}%
\pgfpathlineto{\pgfqpoint{3.070459in}{2.821035in}}%
\pgfpathlineto{\pgfqpoint{3.070459in}{2.823984in}}%
\pgfpathlineto{\pgfqpoint{3.075000in}{2.823984in}}%
\pgfpathlineto{\pgfqpoint{3.075000in}{2.821035in}}%
\pgfpathmoveto{\pgfqpoint{3.070459in}{2.823984in}}%
\pgfpathlineto{\pgfqpoint{3.070459in}{2.823984in}}%
\pgfpathlineto{\pgfqpoint{3.070459in}{2.826934in}}%
\pgfpathlineto{\pgfqpoint{3.075000in}{2.826934in}}%
\pgfpathlineto{\pgfqpoint{3.075000in}{2.823984in}}%
\pgfpathmoveto{\pgfqpoint{3.070459in}{2.826934in}}%
\pgfpathlineto{\pgfqpoint{3.070459in}{2.826934in}}%
\pgfpathlineto{\pgfqpoint{3.070459in}{2.829883in}}%
\pgfpathlineto{\pgfqpoint{3.075000in}{2.829883in}}%
\pgfpathlineto{\pgfqpoint{3.075000in}{2.826934in}}%
\pgfpathmoveto{\pgfqpoint{3.070459in}{2.829883in}}%
\pgfpathlineto{\pgfqpoint{3.070459in}{2.829883in}}%
\pgfpathlineto{\pgfqpoint{3.070459in}{2.832832in}}%
\pgfpathlineto{\pgfqpoint{3.075000in}{2.832832in}}%
\pgfpathlineto{\pgfqpoint{3.075000in}{2.829883in}}%
\pgfpathmoveto{\pgfqpoint{3.070459in}{2.832832in}}%
\pgfpathlineto{\pgfqpoint{3.070459in}{2.832832in}}%
\pgfpathlineto{\pgfqpoint{3.070459in}{2.835781in}}%
\pgfpathlineto{\pgfqpoint{3.075000in}{2.835781in}}%
\pgfpathlineto{\pgfqpoint{3.075000in}{2.832832in}}%
\pgfpathmoveto{\pgfqpoint{3.070459in}{2.835781in}}%
\pgfpathlineto{\pgfqpoint{3.070459in}{2.835781in}}%
\pgfpathlineto{\pgfqpoint{3.070459in}{2.838731in}}%
\pgfpathlineto{\pgfqpoint{3.075000in}{2.838731in}}%
\pgfpathlineto{\pgfqpoint{3.075000in}{2.835781in}}%
\pgfpathmoveto{\pgfqpoint{3.070459in}{2.838731in}}%
\pgfpathlineto{\pgfqpoint{3.070459in}{2.838731in}}%
\pgfpathlineto{\pgfqpoint{3.070459in}{2.841680in}}%
\pgfpathlineto{\pgfqpoint{3.075000in}{2.841680in}}%
\pgfpathlineto{\pgfqpoint{3.075000in}{2.838731in}}%
\pgfpathmoveto{\pgfqpoint{3.070459in}{2.841680in}}%
\pgfpathlineto{\pgfqpoint{3.070459in}{2.841680in}}%
\pgfpathlineto{\pgfqpoint{3.070459in}{2.844629in}}%
\pgfpathlineto{\pgfqpoint{3.075000in}{2.844629in}}%
\pgfpathlineto{\pgfqpoint{3.075000in}{2.841680in}}%
\pgfpathmoveto{\pgfqpoint{3.070459in}{2.844629in}}%
\pgfpathlineto{\pgfqpoint{3.070459in}{2.844629in}}%
\pgfpathlineto{\pgfqpoint{3.070459in}{2.847578in}}%
\pgfpathlineto{\pgfqpoint{3.075000in}{2.847578in}}%
\pgfpathlineto{\pgfqpoint{3.075000in}{2.844629in}}%
\pgfpathmoveto{\pgfqpoint{3.070459in}{2.847578in}}%
\pgfpathlineto{\pgfqpoint{3.070459in}{2.847578in}}%
\pgfpathlineto{\pgfqpoint{3.070459in}{2.850528in}}%
\pgfpathlineto{\pgfqpoint{3.075000in}{2.850528in}}%
\pgfpathlineto{\pgfqpoint{3.075000in}{2.847578in}}%
\pgfpathmoveto{\pgfqpoint{3.070459in}{2.850528in}}%
\pgfpathlineto{\pgfqpoint{3.070459in}{2.850528in}}%
\pgfpathlineto{\pgfqpoint{3.070459in}{2.853477in}}%
\pgfpathlineto{\pgfqpoint{3.075000in}{2.853477in}}%
\pgfpathlineto{\pgfqpoint{3.075000in}{2.850528in}}%
\pgfpathmoveto{\pgfqpoint{3.070459in}{2.853477in}}%
\pgfpathlineto{\pgfqpoint{3.070459in}{2.853477in}}%
\pgfpathlineto{\pgfqpoint{3.070459in}{2.856426in}}%
\pgfpathlineto{\pgfqpoint{3.075000in}{2.856426in}}%
\pgfpathlineto{\pgfqpoint{3.075000in}{2.853477in}}%
\pgfpathmoveto{\pgfqpoint{3.070459in}{2.856426in}}%
\pgfpathlineto{\pgfqpoint{3.070459in}{2.856426in}}%
\pgfpathlineto{\pgfqpoint{3.070459in}{2.859376in}}%
\pgfpathlineto{\pgfqpoint{3.075000in}{2.859376in}}%
\pgfpathlineto{\pgfqpoint{3.075000in}{2.856426in}}%
\pgfpathmoveto{\pgfqpoint{3.070459in}{2.859376in}}%
\pgfpathlineto{\pgfqpoint{3.070459in}{2.859376in}}%
\pgfpathlineto{\pgfqpoint{3.070459in}{2.862325in}}%
\pgfpathlineto{\pgfqpoint{3.075000in}{2.862325in}}%
\pgfpathlineto{\pgfqpoint{3.075000in}{2.859376in}}%
\pgfpathmoveto{\pgfqpoint{3.070459in}{2.862325in}}%
\pgfpathlineto{\pgfqpoint{3.070459in}{2.862325in}}%
\pgfpathlineto{\pgfqpoint{3.070459in}{2.865274in}}%
\pgfpathlineto{\pgfqpoint{3.075000in}{2.865274in}}%
\pgfpathlineto{\pgfqpoint{3.075000in}{2.862325in}}%
\pgfpathmoveto{\pgfqpoint{3.070459in}{2.865274in}}%
\pgfpathlineto{\pgfqpoint{3.070459in}{2.865274in}}%
\pgfpathlineto{\pgfqpoint{3.070459in}{2.868223in}}%
\pgfpathlineto{\pgfqpoint{3.075000in}{2.868223in}}%
\pgfpathlineto{\pgfqpoint{3.075000in}{2.865274in}}%
\pgfpathmoveto{\pgfqpoint{3.070459in}{2.868223in}}%
\pgfpathlineto{\pgfqpoint{3.070459in}{2.868223in}}%
\pgfpathlineto{\pgfqpoint{3.070459in}{2.871172in}}%
\pgfpathlineto{\pgfqpoint{3.075000in}{2.871172in}}%
\pgfpathlineto{\pgfqpoint{3.075000in}{2.868223in}}%
\pgfpathmoveto{\pgfqpoint{3.070459in}{2.871172in}}%
\pgfpathlineto{\pgfqpoint{3.070459in}{2.871172in}}%
\pgfpathlineto{\pgfqpoint{3.070459in}{2.874121in}}%
\pgfpathlineto{\pgfqpoint{3.075000in}{2.874121in}}%
\pgfpathlineto{\pgfqpoint{3.075000in}{2.871172in}}%
\pgfpathmoveto{\pgfqpoint{3.070459in}{2.874121in}}%
\pgfpathlineto{\pgfqpoint{3.070459in}{2.874121in}}%
\pgfpathlineto{\pgfqpoint{3.070459in}{2.877071in}}%
\pgfpathlineto{\pgfqpoint{3.075000in}{2.877071in}}%
\pgfpathlineto{\pgfqpoint{3.075000in}{2.874121in}}%
\pgfpathmoveto{\pgfqpoint{3.070459in}{2.877071in}}%
\pgfpathlineto{\pgfqpoint{3.070459in}{2.877071in}}%
\pgfpathlineto{\pgfqpoint{3.070459in}{2.880020in}}%
\pgfpathlineto{\pgfqpoint{3.075000in}{2.880020in}}%
\pgfpathlineto{\pgfqpoint{3.075000in}{2.877071in}}%
\pgfpathmoveto{\pgfqpoint{3.070459in}{2.880020in}}%
\pgfpathlineto{\pgfqpoint{3.070459in}{2.880020in}}%
\pgfpathlineto{\pgfqpoint{3.070459in}{2.882969in}}%
\pgfpathlineto{\pgfqpoint{3.075000in}{2.882969in}}%
\pgfpathlineto{\pgfqpoint{3.075000in}{2.880020in}}%
\pgfpathmoveto{\pgfqpoint{3.070459in}{2.882969in}}%
\pgfpathlineto{\pgfqpoint{3.070459in}{2.882969in}}%
\pgfpathlineto{\pgfqpoint{3.070459in}{2.885918in}}%
\pgfpathlineto{\pgfqpoint{3.075000in}{2.885918in}}%
\pgfpathlineto{\pgfqpoint{3.075000in}{2.882969in}}%
\pgfpathmoveto{\pgfqpoint{3.070459in}{2.885918in}}%
\pgfpathlineto{\pgfqpoint{3.070459in}{2.885918in}}%
\pgfpathlineto{\pgfqpoint{3.070459in}{2.888867in}}%
\pgfpathlineto{\pgfqpoint{3.075000in}{2.888867in}}%
\pgfpathlineto{\pgfqpoint{3.075000in}{2.885918in}}%
\pgfpathmoveto{\pgfqpoint{3.070459in}{2.888867in}}%
\pgfpathlineto{\pgfqpoint{3.070459in}{2.888867in}}%
\pgfpathlineto{\pgfqpoint{3.070459in}{2.891816in}}%
\pgfpathlineto{\pgfqpoint{3.075000in}{2.891816in}}%
\pgfpathlineto{\pgfqpoint{3.075000in}{2.888867in}}%
\pgfpathmoveto{\pgfqpoint{3.070459in}{2.891816in}}%
\pgfpathlineto{\pgfqpoint{3.070459in}{2.891816in}}%
\pgfpathlineto{\pgfqpoint{3.070459in}{2.894766in}}%
\pgfpathlineto{\pgfqpoint{3.075000in}{2.894766in}}%
\pgfpathlineto{\pgfqpoint{3.075000in}{2.891816in}}%
\pgfpathmoveto{\pgfqpoint{3.070459in}{2.894766in}}%
\pgfpathlineto{\pgfqpoint{3.070459in}{2.894766in}}%
\pgfpathlineto{\pgfqpoint{3.070459in}{2.897715in}}%
\pgfpathlineto{\pgfqpoint{3.075000in}{2.897715in}}%
\pgfpathlineto{\pgfqpoint{3.075000in}{2.894766in}}%
\pgfpathmoveto{\pgfqpoint{3.070459in}{2.897715in}}%
\pgfpathlineto{\pgfqpoint{3.070459in}{2.897715in}}%
\pgfpathlineto{\pgfqpoint{3.070459in}{2.900664in}}%
\pgfpathlineto{\pgfqpoint{3.075000in}{2.900664in}}%
\pgfpathlineto{\pgfqpoint{3.075000in}{2.897715in}}%
\pgfpathmoveto{\pgfqpoint{3.070459in}{2.900664in}}%
\pgfpathlineto{\pgfqpoint{3.070459in}{2.900664in}}%
\pgfpathlineto{\pgfqpoint{3.070459in}{2.903613in}}%
\pgfpathlineto{\pgfqpoint{3.075000in}{2.903613in}}%
\pgfpathlineto{\pgfqpoint{3.075000in}{2.900664in}}%
\pgfpathmoveto{\pgfqpoint{3.070459in}{2.903613in}}%
\pgfpathlineto{\pgfqpoint{3.070459in}{2.903613in}}%
\pgfpathlineto{\pgfqpoint{3.070459in}{2.906562in}}%
\pgfpathlineto{\pgfqpoint{3.075000in}{2.906562in}}%
\pgfpathlineto{\pgfqpoint{3.075000in}{2.903613in}}%
\pgfpathmoveto{\pgfqpoint{3.070459in}{2.906562in}}%
\pgfpathlineto{\pgfqpoint{3.070459in}{2.906562in}}%
\pgfpathlineto{\pgfqpoint{3.070459in}{2.909511in}}%
\pgfpathlineto{\pgfqpoint{3.075000in}{2.909511in}}%
\pgfpathlineto{\pgfqpoint{3.075000in}{2.906562in}}%
\pgfpathmoveto{\pgfqpoint{3.070459in}{2.909511in}}%
\pgfpathlineto{\pgfqpoint{3.070459in}{2.909511in}}%
\pgfpathlineto{\pgfqpoint{3.070459in}{2.912461in}}%
\pgfpathlineto{\pgfqpoint{3.075000in}{2.912461in}}%
\pgfpathlineto{\pgfqpoint{3.075000in}{2.909511in}}%
\pgfpathmoveto{\pgfqpoint{3.070459in}{2.912461in}}%
\pgfpathlineto{\pgfqpoint{3.070459in}{2.912461in}}%
\pgfpathlineto{\pgfqpoint{3.070459in}{2.915410in}}%
\pgfpathlineto{\pgfqpoint{3.075000in}{2.915410in}}%
\pgfpathlineto{\pgfqpoint{3.075000in}{2.912461in}}%
\pgfpathmoveto{\pgfqpoint{3.070459in}{2.915410in}}%
\pgfpathlineto{\pgfqpoint{3.070459in}{2.915410in}}%
\pgfpathlineto{\pgfqpoint{3.070459in}{2.918359in}}%
\pgfpathlineto{\pgfqpoint{3.075000in}{2.918359in}}%
\pgfpathlineto{\pgfqpoint{3.075000in}{2.915410in}}%
\pgfpathmoveto{\pgfqpoint{3.070459in}{2.918359in}}%
\pgfpathlineto{\pgfqpoint{3.070459in}{2.918359in}}%
\pgfpathlineto{\pgfqpoint{3.070459in}{2.921308in}}%
\pgfpathlineto{\pgfqpoint{3.075000in}{2.921308in}}%
\pgfpathlineto{\pgfqpoint{3.075000in}{2.918359in}}%
\pgfpathmoveto{\pgfqpoint{3.070459in}{2.921308in}}%
\pgfpathlineto{\pgfqpoint{3.070459in}{2.921308in}}%
\pgfpathlineto{\pgfqpoint{3.070459in}{2.924257in}}%
\pgfpathlineto{\pgfqpoint{3.075000in}{2.924257in}}%
\pgfpathlineto{\pgfqpoint{3.075000in}{2.921308in}}%
\pgfpathmoveto{\pgfqpoint{3.070459in}{2.924257in}}%
\pgfpathlineto{\pgfqpoint{3.070459in}{2.924257in}}%
\pgfpathlineto{\pgfqpoint{3.070459in}{2.927206in}}%
\pgfpathlineto{\pgfqpoint{3.075000in}{2.927206in}}%
\pgfpathlineto{\pgfqpoint{3.075000in}{2.924257in}}%
\pgfpathmoveto{\pgfqpoint{3.070459in}{2.927206in}}%
\pgfpathlineto{\pgfqpoint{3.070459in}{2.927206in}}%
\pgfpathlineto{\pgfqpoint{3.070459in}{2.930156in}}%
\pgfpathlineto{\pgfqpoint{3.075000in}{2.930156in}}%
\pgfpathlineto{\pgfqpoint{3.075000in}{2.927206in}}%
\pgfpathmoveto{\pgfqpoint{3.070459in}{2.930156in}}%
\pgfpathlineto{\pgfqpoint{3.070459in}{2.930156in}}%
\pgfpathlineto{\pgfqpoint{3.070459in}{2.933105in}}%
\pgfpathlineto{\pgfqpoint{3.075000in}{2.933105in}}%
\pgfpathlineto{\pgfqpoint{3.075000in}{2.930156in}}%
\pgfpathmoveto{\pgfqpoint{3.070459in}{2.933105in}}%
\pgfpathlineto{\pgfqpoint{3.070459in}{2.933105in}}%
\pgfpathlineto{\pgfqpoint{3.070459in}{2.936054in}}%
\pgfpathlineto{\pgfqpoint{3.075000in}{2.936054in}}%
\pgfpathlineto{\pgfqpoint{3.075000in}{2.933105in}}%
\pgfpathmoveto{\pgfqpoint{3.070459in}{2.936054in}}%
\pgfpathlineto{\pgfqpoint{3.070459in}{2.936054in}}%
\pgfpathlineto{\pgfqpoint{3.070459in}{2.939003in}}%
\pgfpathlineto{\pgfqpoint{3.075000in}{2.939003in}}%
\pgfpathlineto{\pgfqpoint{3.075000in}{2.936054in}}%
\pgfpathmoveto{\pgfqpoint{3.070459in}{2.939003in}}%
\pgfpathlineto{\pgfqpoint{3.070459in}{2.939003in}}%
\pgfpathlineto{\pgfqpoint{3.070459in}{2.941952in}}%
\pgfpathlineto{\pgfqpoint{3.075000in}{2.941952in}}%
\pgfpathlineto{\pgfqpoint{3.075000in}{2.939003in}}%
\pgfpathmoveto{\pgfqpoint{3.070459in}{2.941952in}}%
\pgfpathlineto{\pgfqpoint{3.070459in}{2.941952in}}%
\pgfpathlineto{\pgfqpoint{3.070459in}{2.944901in}}%
\pgfpathlineto{\pgfqpoint{3.075000in}{2.944901in}}%
\pgfpathlineto{\pgfqpoint{3.075000in}{2.941952in}}%
\pgfpathmoveto{\pgfqpoint{3.070459in}{2.944901in}}%
\pgfpathlineto{\pgfqpoint{3.070459in}{2.944901in}}%
\pgfpathlineto{\pgfqpoint{3.070459in}{2.947851in}}%
\pgfpathlineto{\pgfqpoint{3.075000in}{2.947851in}}%
\pgfpathlineto{\pgfqpoint{3.075000in}{2.944901in}}%
\pgfpathmoveto{\pgfqpoint{3.070459in}{2.947851in}}%
\pgfpathlineto{\pgfqpoint{3.070459in}{2.947851in}}%
\pgfpathlineto{\pgfqpoint{3.070459in}{2.950800in}}%
\pgfpathlineto{\pgfqpoint{3.075000in}{2.950800in}}%
\pgfpathlineto{\pgfqpoint{3.075000in}{2.947851in}}%
\pgfpathmoveto{\pgfqpoint{3.070459in}{2.950800in}}%
\pgfpathlineto{\pgfqpoint{3.070459in}{2.950800in}}%
\pgfpathlineto{\pgfqpoint{3.070459in}{2.953749in}}%
\pgfpathlineto{\pgfqpoint{3.075000in}{2.953749in}}%
\pgfpathlineto{\pgfqpoint{3.075000in}{2.950800in}}%
\pgfpathmoveto{\pgfqpoint{3.070459in}{2.953749in}}%
\pgfpathlineto{\pgfqpoint{3.070459in}{2.953749in}}%
\pgfpathlineto{\pgfqpoint{3.070459in}{2.956698in}}%
\pgfpathlineto{\pgfqpoint{3.075000in}{2.956698in}}%
\pgfpathlineto{\pgfqpoint{3.075000in}{2.953749in}}%
\pgfpathmoveto{\pgfqpoint{3.070459in}{2.956698in}}%
\pgfpathlineto{\pgfqpoint{3.070459in}{2.956698in}}%
\pgfpathlineto{\pgfqpoint{3.070459in}{2.959647in}}%
\pgfpathlineto{\pgfqpoint{3.075000in}{2.959647in}}%
\pgfpathlineto{\pgfqpoint{3.075000in}{2.956698in}}%
\pgfpathmoveto{\pgfqpoint{3.070459in}{2.959647in}}%
\pgfpathlineto{\pgfqpoint{3.070459in}{2.959647in}}%
\pgfpathlineto{\pgfqpoint{3.070459in}{2.962597in}}%
\pgfpathlineto{\pgfqpoint{3.075000in}{2.962597in}}%
\pgfpathlineto{\pgfqpoint{3.075000in}{2.959647in}}%
\pgfpathmoveto{\pgfqpoint{3.070459in}{2.962597in}}%
\pgfpathlineto{\pgfqpoint{3.070459in}{2.962597in}}%
\pgfpathlineto{\pgfqpoint{3.070459in}{2.965546in}}%
\pgfpathlineto{\pgfqpoint{3.075000in}{2.965546in}}%
\pgfpathlineto{\pgfqpoint{3.075000in}{2.962597in}}%
\pgfpathmoveto{\pgfqpoint{3.070459in}{2.965546in}}%
\pgfpathlineto{\pgfqpoint{3.070459in}{2.965546in}}%
\pgfpathlineto{\pgfqpoint{3.070459in}{2.968495in}}%
\pgfpathlineto{\pgfqpoint{3.075000in}{2.968495in}}%
\pgfpathlineto{\pgfqpoint{3.075000in}{2.965546in}}%
\pgfpathmoveto{\pgfqpoint{3.070459in}{2.968495in}}%
\pgfpathlineto{\pgfqpoint{3.070459in}{2.968495in}}%
\pgfpathlineto{\pgfqpoint{3.070459in}{2.971444in}}%
\pgfpathlineto{\pgfqpoint{3.075000in}{2.971444in}}%
\pgfpathlineto{\pgfqpoint{3.075000in}{2.968495in}}%
\pgfpathmoveto{\pgfqpoint{3.070459in}{2.971444in}}%
\pgfpathlineto{\pgfqpoint{3.070459in}{2.971444in}}%
\pgfpathlineto{\pgfqpoint{3.070459in}{2.974394in}}%
\pgfpathlineto{\pgfqpoint{3.075000in}{2.974394in}}%
\pgfpathlineto{\pgfqpoint{3.075000in}{2.971444in}}%
\pgfpathmoveto{\pgfqpoint{3.070459in}{2.974394in}}%
\pgfpathlineto{\pgfqpoint{3.070459in}{2.974394in}}%
\pgfpathlineto{\pgfqpoint{3.070459in}{2.977343in}}%
\pgfpathlineto{\pgfqpoint{3.075000in}{2.977343in}}%
\pgfpathlineto{\pgfqpoint{3.075000in}{2.974394in}}%
\pgfpathmoveto{\pgfqpoint{3.070459in}{2.977343in}}%
\pgfpathlineto{\pgfqpoint{3.070459in}{2.977343in}}%
\pgfpathlineto{\pgfqpoint{3.070459in}{2.980292in}}%
\pgfpathlineto{\pgfqpoint{3.075000in}{2.980292in}}%
\pgfpathlineto{\pgfqpoint{3.075000in}{2.977343in}}%
\pgfpathmoveto{\pgfqpoint{3.070459in}{2.980292in}}%
\pgfpathlineto{\pgfqpoint{3.070459in}{2.980292in}}%
\pgfpathlineto{\pgfqpoint{3.070459in}{2.983242in}}%
\pgfpathlineto{\pgfqpoint{3.075000in}{2.983242in}}%
\pgfpathlineto{\pgfqpoint{3.075000in}{2.980292in}}%
\pgfpathmoveto{\pgfqpoint{3.070459in}{2.983242in}}%
\pgfpathlineto{\pgfqpoint{3.070459in}{2.983242in}}%
\pgfpathlineto{\pgfqpoint{3.070459in}{2.986191in}}%
\pgfpathlineto{\pgfqpoint{3.075000in}{2.986191in}}%
\pgfpathlineto{\pgfqpoint{3.075000in}{2.983242in}}%
\pgfpathmoveto{\pgfqpoint{3.070459in}{2.986191in}}%
\pgfpathlineto{\pgfqpoint{3.070459in}{2.986191in}}%
\pgfpathlineto{\pgfqpoint{3.070459in}{2.989140in}}%
\pgfpathlineto{\pgfqpoint{3.075000in}{2.989140in}}%
\pgfpathlineto{\pgfqpoint{3.075000in}{2.986191in}}%
\pgfpathmoveto{\pgfqpoint{3.070459in}{2.989140in}}%
\pgfpathlineto{\pgfqpoint{3.070459in}{2.989140in}}%
\pgfpathlineto{\pgfqpoint{3.070459in}{2.992089in}}%
\pgfpathlineto{\pgfqpoint{3.075000in}{2.992089in}}%
\pgfpathlineto{\pgfqpoint{3.075000in}{2.989140in}}%
\pgfpathmoveto{\pgfqpoint{3.070459in}{2.992089in}}%
\pgfpathlineto{\pgfqpoint{3.070459in}{2.992089in}}%
\pgfpathlineto{\pgfqpoint{3.070459in}{2.995039in}}%
\pgfpathlineto{\pgfqpoint{3.075000in}{2.995039in}}%
\pgfpathlineto{\pgfqpoint{3.075000in}{2.992089in}}%
\pgfpathmoveto{\pgfqpoint{3.070459in}{2.995039in}}%
\pgfpathlineto{\pgfqpoint{3.070459in}{2.995039in}}%
\pgfpathlineto{\pgfqpoint{3.070459in}{2.997988in}}%
\pgfpathlineto{\pgfqpoint{3.075000in}{2.997988in}}%
\pgfpathlineto{\pgfqpoint{3.075000in}{2.995039in}}%
\pgfpathmoveto{\pgfqpoint{3.070459in}{2.997988in}}%
\pgfpathlineto{\pgfqpoint{3.070459in}{2.997988in}}%
\pgfpathlineto{\pgfqpoint{3.070459in}{3.000937in}}%
\pgfpathlineto{\pgfqpoint{3.075000in}{3.000937in}}%
\pgfpathlineto{\pgfqpoint{3.075000in}{2.997988in}}%
\pgfpathmoveto{\pgfqpoint{3.070459in}{3.000937in}}%
\pgfpathlineto{\pgfqpoint{3.070459in}{3.000937in}}%
\pgfpathlineto{\pgfqpoint{3.070459in}{3.003886in}}%
\pgfpathlineto{\pgfqpoint{3.075000in}{3.003886in}}%
\pgfpathlineto{\pgfqpoint{3.075000in}{3.000937in}}%
\pgfpathmoveto{\pgfqpoint{3.070459in}{3.003886in}}%
\pgfpathlineto{\pgfqpoint{3.070459in}{3.003886in}}%
\pgfpathlineto{\pgfqpoint{3.070459in}{3.006836in}}%
\pgfpathlineto{\pgfqpoint{3.075000in}{3.006836in}}%
\pgfpathlineto{\pgfqpoint{3.075000in}{3.003886in}}%
\pgfpathmoveto{\pgfqpoint{3.070459in}{3.006836in}}%
\pgfpathlineto{\pgfqpoint{3.070459in}{3.006836in}}%
\pgfpathlineto{\pgfqpoint{3.070459in}{3.009785in}}%
\pgfpathlineto{\pgfqpoint{3.075000in}{3.009785in}}%
\pgfpathlineto{\pgfqpoint{3.075000in}{3.006836in}}%
\pgfpathmoveto{\pgfqpoint{3.070459in}{3.009785in}}%
\pgfpathlineto{\pgfqpoint{3.070459in}{3.009785in}}%
\pgfpathlineto{\pgfqpoint{3.070459in}{3.012734in}}%
\pgfpathlineto{\pgfqpoint{3.075000in}{3.012734in}}%
\pgfpathlineto{\pgfqpoint{3.075000in}{3.009785in}}%
\pgfpathmoveto{\pgfqpoint{3.070459in}{3.012734in}}%
\pgfpathlineto{\pgfqpoint{3.070459in}{3.012734in}}%
\pgfpathlineto{\pgfqpoint{3.070459in}{3.015683in}}%
\pgfpathlineto{\pgfqpoint{3.075000in}{3.015683in}}%
\pgfpathlineto{\pgfqpoint{3.075000in}{3.012734in}}%
\pgfpathmoveto{\pgfqpoint{3.070459in}{3.015683in}}%
\pgfpathlineto{\pgfqpoint{3.070459in}{3.015683in}}%
\pgfpathlineto{\pgfqpoint{3.070459in}{3.018633in}}%
\pgfpathlineto{\pgfqpoint{3.075000in}{3.018633in}}%
\pgfpathlineto{\pgfqpoint{3.075000in}{3.015683in}}%
\pgfpathmoveto{\pgfqpoint{3.070459in}{3.018633in}}%
\pgfpathlineto{\pgfqpoint{3.070459in}{3.018633in}}%
\pgfpathlineto{\pgfqpoint{3.070459in}{3.021582in}}%
\pgfpathlineto{\pgfqpoint{3.075000in}{3.021582in}}%
\pgfpathlineto{\pgfqpoint{3.075000in}{3.018633in}}%
\pgfpathmoveto{\pgfqpoint{3.070459in}{3.021582in}}%
\pgfpathlineto{\pgfqpoint{3.070459in}{3.021582in}}%
\pgfpathlineto{\pgfqpoint{3.070459in}{3.024531in}}%
\pgfpathlineto{\pgfqpoint{3.075000in}{3.024531in}}%
\pgfpathlineto{\pgfqpoint{3.075000in}{3.021582in}}%
\pgfpathmoveto{\pgfqpoint{3.070459in}{3.024531in}}%
\pgfpathlineto{\pgfqpoint{3.070459in}{3.024531in}}%
\pgfpathlineto{\pgfqpoint{3.070459in}{3.027481in}}%
\pgfpathlineto{\pgfqpoint{3.075000in}{3.027481in}}%
\pgfpathlineto{\pgfqpoint{3.075000in}{3.024531in}}%
\pgfpathmoveto{\pgfqpoint{3.070459in}{3.027481in}}%
\pgfpathlineto{\pgfqpoint{3.070459in}{3.027481in}}%
\pgfpathlineto{\pgfqpoint{3.070459in}{3.030430in}}%
\pgfpathlineto{\pgfqpoint{3.075000in}{3.030430in}}%
\pgfpathlineto{\pgfqpoint{3.075000in}{3.027481in}}%
\pgfpathmoveto{\pgfqpoint{3.070459in}{3.030430in}}%
\pgfpathlineto{\pgfqpoint{3.070459in}{3.030430in}}%
\pgfpathlineto{\pgfqpoint{3.070459in}{3.033379in}}%
\pgfpathlineto{\pgfqpoint{3.075000in}{3.033379in}}%
\pgfpathlineto{\pgfqpoint{3.075000in}{3.030430in}}%
\pgfpathmoveto{\pgfqpoint{3.070459in}{3.033379in}}%
\pgfpathlineto{\pgfqpoint{3.070459in}{3.033379in}}%
\pgfpathlineto{\pgfqpoint{3.070459in}{3.036328in}}%
\pgfpathlineto{\pgfqpoint{3.075000in}{3.036328in}}%
\pgfpathlineto{\pgfqpoint{3.075000in}{3.033379in}}%
\pgfpathmoveto{\pgfqpoint{3.070459in}{3.036328in}}%
\pgfpathlineto{\pgfqpoint{3.070459in}{3.036328in}}%
\pgfpathlineto{\pgfqpoint{3.070459in}{3.039278in}}%
\pgfpathlineto{\pgfqpoint{3.075000in}{3.039278in}}%
\pgfpathlineto{\pgfqpoint{3.075000in}{3.036328in}}%
\pgfpathmoveto{\pgfqpoint{3.070459in}{3.039278in}}%
\pgfpathlineto{\pgfqpoint{3.070459in}{3.039278in}}%
\pgfpathlineto{\pgfqpoint{3.070459in}{3.042227in}}%
\pgfpathlineto{\pgfqpoint{3.075000in}{3.042227in}}%
\pgfpathlineto{\pgfqpoint{3.075000in}{3.039278in}}%
\pgfpathmoveto{\pgfqpoint{3.070459in}{3.042227in}}%
\pgfpathlineto{\pgfqpoint{3.070459in}{3.042227in}}%
\pgfpathlineto{\pgfqpoint{3.070459in}{3.045176in}}%
\pgfpathlineto{\pgfqpoint{3.075000in}{3.045176in}}%
\pgfpathlineto{\pgfqpoint{3.075000in}{3.042227in}}%
\pgfpathmoveto{\pgfqpoint{3.070459in}{3.045176in}}%
\pgfpathlineto{\pgfqpoint{3.070459in}{3.045176in}}%
\pgfpathlineto{\pgfqpoint{3.070459in}{3.048125in}}%
\pgfpathlineto{\pgfqpoint{3.075000in}{3.048125in}}%
\pgfpathlineto{\pgfqpoint{3.075000in}{3.045176in}}%
\pgfpathmoveto{\pgfqpoint{3.070459in}{3.048125in}}%
\pgfpathlineto{\pgfqpoint{3.070459in}{3.048125in}}%
\pgfpathlineto{\pgfqpoint{3.070459in}{3.051075in}}%
\pgfpathlineto{\pgfqpoint{3.075000in}{3.051075in}}%
\pgfpathlineto{\pgfqpoint{3.075000in}{3.048125in}}%
\pgfpathmoveto{\pgfqpoint{3.070459in}{3.051075in}}%
\pgfpathlineto{\pgfqpoint{3.070459in}{3.051075in}}%
\pgfpathlineto{\pgfqpoint{3.070459in}{3.054024in}}%
\pgfpathlineto{\pgfqpoint{3.075000in}{3.054024in}}%
\pgfpathlineto{\pgfqpoint{3.075000in}{3.051075in}}%
\pgfpathmoveto{\pgfqpoint{3.070459in}{3.054024in}}%
\pgfpathlineto{\pgfqpoint{3.070459in}{3.054024in}}%
\pgfpathlineto{\pgfqpoint{3.070459in}{3.056973in}}%
\pgfpathlineto{\pgfqpoint{3.075000in}{3.056973in}}%
\pgfpathlineto{\pgfqpoint{3.075000in}{3.054024in}}%
\pgfpathmoveto{\pgfqpoint{3.070459in}{3.056973in}}%
\pgfpathlineto{\pgfqpoint{3.070459in}{3.056973in}}%
\pgfpathlineto{\pgfqpoint{3.070459in}{3.059922in}}%
\pgfpathlineto{\pgfqpoint{3.075000in}{3.059922in}}%
\pgfpathlineto{\pgfqpoint{3.075000in}{3.056973in}}%
\pgfpathmoveto{\pgfqpoint{3.070459in}{3.059922in}}%
\pgfpathlineto{\pgfqpoint{3.070459in}{3.059922in}}%
\pgfpathlineto{\pgfqpoint{3.070459in}{3.062871in}}%
\pgfpathlineto{\pgfqpoint{3.075000in}{3.062871in}}%
\pgfpathlineto{\pgfqpoint{3.075000in}{3.059922in}}%
\pgfpathmoveto{\pgfqpoint{3.070459in}{3.062871in}}%
\pgfpathlineto{\pgfqpoint{3.070459in}{3.062871in}}%
\pgfpathlineto{\pgfqpoint{3.070459in}{3.065820in}}%
\pgfpathlineto{\pgfqpoint{3.075000in}{3.065820in}}%
\pgfpathlineto{\pgfqpoint{3.075000in}{3.062871in}}%
\pgfpathmoveto{\pgfqpoint{3.070459in}{3.065820in}}%
\pgfpathlineto{\pgfqpoint{3.070459in}{3.065820in}}%
\pgfpathlineto{\pgfqpoint{3.070459in}{3.068770in}}%
\pgfpathlineto{\pgfqpoint{3.075000in}{3.068770in}}%
\pgfpathlineto{\pgfqpoint{3.075000in}{3.065820in}}%
\pgfpathmoveto{\pgfqpoint{3.070459in}{3.068770in}}%
\pgfpathlineto{\pgfqpoint{3.070459in}{3.068770in}}%
\pgfpathlineto{\pgfqpoint{3.070459in}{3.071719in}}%
\pgfpathlineto{\pgfqpoint{3.075000in}{3.071719in}}%
\pgfpathlineto{\pgfqpoint{3.075000in}{3.068770in}}%
\pgfpathmoveto{\pgfqpoint{3.070459in}{3.071719in}}%
\pgfpathlineto{\pgfqpoint{3.070459in}{3.071719in}}%
\pgfpathlineto{\pgfqpoint{3.070459in}{3.074668in}}%
\pgfpathlineto{\pgfqpoint{3.075000in}{3.074668in}}%
\pgfpathlineto{\pgfqpoint{3.075000in}{3.071719in}}%
\pgfpathmoveto{\pgfqpoint{3.070459in}{3.074668in}}%
\pgfpathlineto{\pgfqpoint{3.070459in}{3.074668in}}%
\pgfpathlineto{\pgfqpoint{3.070459in}{3.077617in}}%
\pgfpathlineto{\pgfqpoint{3.075000in}{3.077617in}}%
\pgfpathlineto{\pgfqpoint{3.075000in}{3.074668in}}%
\pgfpathmoveto{\pgfqpoint{3.070459in}{3.077617in}}%
\pgfpathlineto{\pgfqpoint{3.070459in}{3.077617in}}%
\pgfpathlineto{\pgfqpoint{3.070459in}{3.080566in}}%
\pgfpathlineto{\pgfqpoint{3.075000in}{3.080566in}}%
\pgfpathlineto{\pgfqpoint{3.075000in}{3.077617in}}%
\pgfpathmoveto{\pgfqpoint{3.070459in}{3.080566in}}%
\pgfpathlineto{\pgfqpoint{3.070459in}{3.080566in}}%
\pgfpathlineto{\pgfqpoint{3.070459in}{3.083515in}}%
\pgfpathlineto{\pgfqpoint{3.075000in}{3.083515in}}%
\pgfpathlineto{\pgfqpoint{3.075000in}{3.080566in}}%
\pgfpathmoveto{\pgfqpoint{3.070459in}{3.083515in}}%
\pgfpathlineto{\pgfqpoint{3.070459in}{3.083515in}}%
\pgfpathlineto{\pgfqpoint{3.070459in}{3.086465in}}%
\pgfpathlineto{\pgfqpoint{3.075000in}{3.086465in}}%
\pgfpathlineto{\pgfqpoint{3.075000in}{3.083515in}}%
\pgfpathmoveto{\pgfqpoint{3.070459in}{3.086465in}}%
\pgfpathlineto{\pgfqpoint{3.070459in}{3.086465in}}%
\pgfpathlineto{\pgfqpoint{3.070459in}{3.089414in}}%
\pgfpathlineto{\pgfqpoint{3.075000in}{3.089414in}}%
\pgfpathlineto{\pgfqpoint{3.075000in}{3.086465in}}%
\pgfpathmoveto{\pgfqpoint{3.070459in}{3.089414in}}%
\pgfpathlineto{\pgfqpoint{3.070459in}{3.089414in}}%
\pgfpathlineto{\pgfqpoint{3.070459in}{3.092363in}}%
\pgfpathlineto{\pgfqpoint{3.075000in}{3.092363in}}%
\pgfpathlineto{\pgfqpoint{3.075000in}{3.089414in}}%
\pgfpathmoveto{\pgfqpoint{3.070459in}{3.092363in}}%
\pgfpathlineto{\pgfqpoint{3.070459in}{3.092363in}}%
\pgfpathlineto{\pgfqpoint{3.070459in}{3.095312in}}%
\pgfpathlineto{\pgfqpoint{3.075000in}{3.095312in}}%
\pgfpathlineto{\pgfqpoint{3.075000in}{3.092363in}}%
\pgfpathmoveto{\pgfqpoint{3.070459in}{3.095312in}}%
\pgfpathlineto{\pgfqpoint{3.070459in}{3.095312in}}%
\pgfpathlineto{\pgfqpoint{3.070459in}{3.098261in}}%
\pgfpathlineto{\pgfqpoint{3.075000in}{3.098261in}}%
\pgfpathlineto{\pgfqpoint{3.075000in}{3.095312in}}%
\pgfpathmoveto{\pgfqpoint{3.070459in}{3.098261in}}%
\pgfpathlineto{\pgfqpoint{3.070459in}{3.098261in}}%
\pgfpathlineto{\pgfqpoint{3.070459in}{3.101210in}}%
\pgfpathlineto{\pgfqpoint{3.075000in}{3.101210in}}%
\pgfpathlineto{\pgfqpoint{3.075000in}{3.098261in}}%
\pgfpathmoveto{\pgfqpoint{3.070459in}{3.101210in}}%
\pgfpathlineto{\pgfqpoint{3.070459in}{3.101210in}}%
\pgfpathlineto{\pgfqpoint{3.070459in}{3.104160in}}%
\pgfpathlineto{\pgfqpoint{3.075000in}{3.104160in}}%
\pgfpathlineto{\pgfqpoint{3.075000in}{3.101210in}}%
\pgfpathmoveto{\pgfqpoint{3.070459in}{3.104160in}}%
\pgfpathlineto{\pgfqpoint{3.070459in}{3.104160in}}%
\pgfpathlineto{\pgfqpoint{3.070459in}{3.107109in}}%
\pgfpathlineto{\pgfqpoint{3.075000in}{3.107109in}}%
\pgfpathlineto{\pgfqpoint{3.075000in}{3.104160in}}%
\pgfpathmoveto{\pgfqpoint{3.070459in}{3.107109in}}%
\pgfpathlineto{\pgfqpoint{3.070459in}{3.107109in}}%
\pgfpathlineto{\pgfqpoint{3.070459in}{3.110058in}}%
\pgfpathlineto{\pgfqpoint{3.075000in}{3.110058in}}%
\pgfpathlineto{\pgfqpoint{3.075000in}{3.107109in}}%
\pgfpathmoveto{\pgfqpoint{3.070459in}{3.110058in}}%
\pgfpathlineto{\pgfqpoint{3.070459in}{3.110058in}}%
\pgfpathlineto{\pgfqpoint{3.070459in}{3.113007in}}%
\pgfpathlineto{\pgfqpoint{3.075000in}{3.113007in}}%
\pgfpathlineto{\pgfqpoint{3.075000in}{3.110058in}}%
\pgfpathmoveto{\pgfqpoint{3.070459in}{3.113007in}}%
\pgfpathlineto{\pgfqpoint{3.070459in}{3.113007in}}%
\pgfpathlineto{\pgfqpoint{3.070459in}{3.115956in}}%
\pgfpathlineto{\pgfqpoint{3.075000in}{3.115956in}}%
\pgfpathlineto{\pgfqpoint{3.075000in}{3.113007in}}%
\pgfpathmoveto{\pgfqpoint{3.070459in}{3.115956in}}%
\pgfpathlineto{\pgfqpoint{3.070459in}{3.115956in}}%
\pgfpathlineto{\pgfqpoint{3.070459in}{3.118905in}}%
\pgfpathlineto{\pgfqpoint{3.075000in}{3.118905in}}%
\pgfpathlineto{\pgfqpoint{3.075000in}{3.115956in}}%
\pgfpathmoveto{\pgfqpoint{3.070459in}{3.118905in}}%
\pgfpathlineto{\pgfqpoint{3.070459in}{3.118905in}}%
\pgfpathlineto{\pgfqpoint{3.070459in}{3.121855in}}%
\pgfpathlineto{\pgfqpoint{3.075000in}{3.121855in}}%
\pgfpathlineto{\pgfqpoint{3.075000in}{3.118905in}}%
\pgfpathmoveto{\pgfqpoint{3.070459in}{3.121855in}}%
\pgfpathlineto{\pgfqpoint{3.070459in}{3.121855in}}%
\pgfpathlineto{\pgfqpoint{3.070459in}{3.124804in}}%
\pgfpathlineto{\pgfqpoint{3.075000in}{3.124804in}}%
\pgfpathlineto{\pgfqpoint{3.075000in}{3.121855in}}%
\pgfpathmoveto{\pgfqpoint{3.070459in}{3.124804in}}%
\pgfpathlineto{\pgfqpoint{3.070459in}{3.124804in}}%
\pgfpathlineto{\pgfqpoint{3.070459in}{3.127753in}}%
\pgfpathlineto{\pgfqpoint{3.075000in}{3.127753in}}%
\pgfpathlineto{\pgfqpoint{3.075000in}{3.124804in}}%
\pgfpathmoveto{\pgfqpoint{3.070459in}{3.127753in}}%
\pgfpathlineto{\pgfqpoint{3.070459in}{3.127753in}}%
\pgfpathlineto{\pgfqpoint{3.070459in}{3.130702in}}%
\pgfpathlineto{\pgfqpoint{3.075000in}{3.130702in}}%
\pgfpathlineto{\pgfqpoint{3.075000in}{3.127753in}}%
\pgfpathmoveto{\pgfqpoint{3.070459in}{3.130702in}}%
\pgfpathlineto{\pgfqpoint{3.070459in}{3.130702in}}%
\pgfpathlineto{\pgfqpoint{3.070459in}{3.133651in}}%
\pgfpathlineto{\pgfqpoint{3.075000in}{3.133651in}}%
\pgfpathlineto{\pgfqpoint{3.075000in}{3.130702in}}%
\pgfpathmoveto{\pgfqpoint{3.070459in}{3.133651in}}%
\pgfpathlineto{\pgfqpoint{3.070459in}{3.133651in}}%
\pgfpathlineto{\pgfqpoint{3.070459in}{3.136600in}}%
\pgfpathlineto{\pgfqpoint{3.075000in}{3.136600in}}%
\pgfpathlineto{\pgfqpoint{3.075000in}{3.133651in}}%
\pgfpathmoveto{\pgfqpoint{3.070459in}{3.136600in}}%
\pgfpathlineto{\pgfqpoint{3.070459in}{3.136600in}}%
\pgfpathlineto{\pgfqpoint{3.070459in}{3.139550in}}%
\pgfpathlineto{\pgfqpoint{3.075000in}{3.139550in}}%
\pgfpathlineto{\pgfqpoint{3.075000in}{3.136600in}}%
\pgfpathmoveto{\pgfqpoint{3.070459in}{3.139550in}}%
\pgfpathlineto{\pgfqpoint{3.070459in}{3.139550in}}%
\pgfpathlineto{\pgfqpoint{3.070459in}{3.142499in}}%
\pgfpathlineto{\pgfqpoint{3.075000in}{3.142499in}}%
\pgfpathlineto{\pgfqpoint{3.075000in}{3.139550in}}%
\pgfpathmoveto{\pgfqpoint{3.070459in}{3.142499in}}%
\pgfpathlineto{\pgfqpoint{3.070459in}{3.142499in}}%
\pgfpathlineto{\pgfqpoint{3.070459in}{3.145448in}}%
\pgfpathlineto{\pgfqpoint{3.075000in}{3.145448in}}%
\pgfpathlineto{\pgfqpoint{3.075000in}{3.142499in}}%
\pgfpathmoveto{\pgfqpoint{3.070459in}{3.145448in}}%
\pgfpathlineto{\pgfqpoint{3.070459in}{3.145448in}}%
\pgfpathlineto{\pgfqpoint{3.070459in}{3.148397in}}%
\pgfpathlineto{\pgfqpoint{3.075000in}{3.148397in}}%
\pgfpathlineto{\pgfqpoint{3.075000in}{3.145448in}}%
\pgfpathmoveto{\pgfqpoint{3.070459in}{3.148397in}}%
\pgfpathlineto{\pgfqpoint{3.070459in}{3.148397in}}%
\pgfpathlineto{\pgfqpoint{3.070459in}{3.151347in}}%
\pgfpathlineto{\pgfqpoint{3.075000in}{3.151347in}}%
\pgfpathlineto{\pgfqpoint{3.075000in}{3.148397in}}%
\pgfpathmoveto{\pgfqpoint{3.070459in}{3.151347in}}%
\pgfpathlineto{\pgfqpoint{3.070459in}{3.151347in}}%
\pgfpathlineto{\pgfqpoint{3.070459in}{3.154296in}}%
\pgfpathlineto{\pgfqpoint{3.075000in}{3.154296in}}%
\pgfpathlineto{\pgfqpoint{3.075000in}{3.151347in}}%
\pgfpathmoveto{\pgfqpoint{3.070459in}{3.154296in}}%
\pgfpathlineto{\pgfqpoint{3.070459in}{3.154296in}}%
\pgfpathlineto{\pgfqpoint{3.070459in}{3.157245in}}%
\pgfpathlineto{\pgfqpoint{3.075000in}{3.157245in}}%
\pgfpathlineto{\pgfqpoint{3.075000in}{3.154296in}}%
\pgfpathmoveto{\pgfqpoint{3.070459in}{3.157245in}}%
\pgfpathlineto{\pgfqpoint{3.070459in}{3.157245in}}%
\pgfpathlineto{\pgfqpoint{3.070459in}{3.160194in}}%
\pgfpathlineto{\pgfqpoint{3.075000in}{3.160194in}}%
\pgfpathlineto{\pgfqpoint{3.075000in}{3.157245in}}%
\pgfpathmoveto{\pgfqpoint{3.070459in}{3.160194in}}%
\pgfpathlineto{\pgfqpoint{3.070459in}{3.160194in}}%
\pgfpathlineto{\pgfqpoint{3.070459in}{3.163144in}}%
\pgfpathlineto{\pgfqpoint{3.075000in}{3.163144in}}%
\pgfpathlineto{\pgfqpoint{3.075000in}{3.160194in}}%
\pgfpathmoveto{\pgfqpoint{3.070459in}{3.163144in}}%
\pgfpathlineto{\pgfqpoint{3.070459in}{3.163144in}}%
\pgfpathlineto{\pgfqpoint{3.070459in}{3.166093in}}%
\pgfpathlineto{\pgfqpoint{3.075000in}{3.166093in}}%
\pgfpathlineto{\pgfqpoint{3.075000in}{3.163144in}}%
\pgfpathmoveto{\pgfqpoint{3.070459in}{3.166093in}}%
\pgfpathlineto{\pgfqpoint{3.070459in}{3.166093in}}%
\pgfpathlineto{\pgfqpoint{3.070459in}{3.169042in}}%
\pgfpathlineto{\pgfqpoint{3.075000in}{3.169042in}}%
\pgfpathlineto{\pgfqpoint{3.075000in}{3.166093in}}%
\pgfpathmoveto{\pgfqpoint{3.070459in}{3.169042in}}%
\pgfpathlineto{\pgfqpoint{3.070459in}{3.169042in}}%
\pgfpathlineto{\pgfqpoint{3.070459in}{3.171991in}}%
\pgfpathlineto{\pgfqpoint{3.075000in}{3.171991in}}%
\pgfpathlineto{\pgfqpoint{3.075000in}{3.169042in}}%
\pgfpathmoveto{\pgfqpoint{3.070459in}{3.171991in}}%
\pgfpathlineto{\pgfqpoint{3.070459in}{3.171991in}}%
\pgfpathlineto{\pgfqpoint{3.070459in}{3.174941in}}%
\pgfpathlineto{\pgfqpoint{3.075000in}{3.174941in}}%
\pgfpathlineto{\pgfqpoint{3.075000in}{3.171991in}}%
\pgfpathmoveto{\pgfqpoint{3.070459in}{3.174941in}}%
\pgfpathlineto{\pgfqpoint{3.070459in}{3.174941in}}%
\pgfpathlineto{\pgfqpoint{3.070459in}{3.177890in}}%
\pgfpathlineto{\pgfqpoint{3.075000in}{3.177890in}}%
\pgfpathlineto{\pgfqpoint{3.075000in}{3.174941in}}%
\pgfpathmoveto{\pgfqpoint{3.070459in}{3.177890in}}%
\pgfpathlineto{\pgfqpoint{3.070459in}{3.177890in}}%
\pgfpathlineto{\pgfqpoint{3.070459in}{3.180839in}}%
\pgfpathlineto{\pgfqpoint{3.075000in}{3.180839in}}%
\pgfpathlineto{\pgfqpoint{3.075000in}{3.177890in}}%
\pgfpathmoveto{\pgfqpoint{3.070459in}{3.180839in}}%
\pgfpathlineto{\pgfqpoint{3.070459in}{3.180839in}}%
\pgfpathlineto{\pgfqpoint{3.070459in}{3.183789in}}%
\pgfpathlineto{\pgfqpoint{3.075000in}{3.183789in}}%
\pgfpathlineto{\pgfqpoint{3.075000in}{3.180839in}}%
\pgfpathmoveto{\pgfqpoint{3.070459in}{3.183789in}}%
\pgfpathlineto{\pgfqpoint{3.070459in}{3.183789in}}%
\pgfpathlineto{\pgfqpoint{3.070459in}{3.186738in}}%
\pgfpathlineto{\pgfqpoint{3.075000in}{3.186738in}}%
\pgfpathlineto{\pgfqpoint{3.075000in}{3.183789in}}%
\pgfpathmoveto{\pgfqpoint{3.070459in}{3.186738in}}%
\pgfpathlineto{\pgfqpoint{3.070459in}{3.186738in}}%
\pgfpathlineto{\pgfqpoint{3.070459in}{3.189687in}}%
\pgfpathlineto{\pgfqpoint{3.075000in}{3.189687in}}%
\pgfpathlineto{\pgfqpoint{3.075000in}{3.186738in}}%
\pgfpathmoveto{\pgfqpoint{3.070459in}{3.189687in}}%
\pgfpathlineto{\pgfqpoint{3.070459in}{3.189687in}}%
\pgfpathlineto{\pgfqpoint{3.070459in}{3.192636in}}%
\pgfpathlineto{\pgfqpoint{3.075000in}{3.192636in}}%
\pgfpathlineto{\pgfqpoint{3.075000in}{3.189687in}}%
\pgfpathmoveto{\pgfqpoint{3.070459in}{3.192636in}}%
\pgfpathlineto{\pgfqpoint{3.070459in}{3.192636in}}%
\pgfpathlineto{\pgfqpoint{3.070459in}{3.195586in}}%
\pgfpathlineto{\pgfqpoint{3.075000in}{3.195586in}}%
\pgfpathlineto{\pgfqpoint{3.075000in}{3.192636in}}%
\pgfpathmoveto{\pgfqpoint{3.070459in}{3.195586in}}%
\pgfpathlineto{\pgfqpoint{3.070459in}{3.195586in}}%
\pgfpathlineto{\pgfqpoint{3.070459in}{3.198535in}}%
\pgfpathlineto{\pgfqpoint{3.075000in}{3.198535in}}%
\pgfpathlineto{\pgfqpoint{3.075000in}{3.195586in}}%
\pgfpathmoveto{\pgfqpoint{3.070459in}{3.198535in}}%
\pgfpathlineto{\pgfqpoint{3.070459in}{3.198535in}}%
\pgfpathlineto{\pgfqpoint{3.070459in}{3.201484in}}%
\pgfpathlineto{\pgfqpoint{3.075000in}{3.201484in}}%
\pgfpathlineto{\pgfqpoint{3.075000in}{3.198535in}}%
\pgfpathmoveto{\pgfqpoint{3.070459in}{3.201484in}}%
\pgfpathlineto{\pgfqpoint{3.070459in}{3.201484in}}%
\pgfpathlineto{\pgfqpoint{3.070459in}{3.204433in}}%
\pgfpathlineto{\pgfqpoint{3.075000in}{3.204433in}}%
\pgfpathlineto{\pgfqpoint{3.075000in}{3.201484in}}%
\pgfpathmoveto{\pgfqpoint{3.070459in}{3.204433in}}%
\pgfpathlineto{\pgfqpoint{3.070459in}{3.204433in}}%
\pgfpathlineto{\pgfqpoint{3.070459in}{3.207383in}}%
\pgfpathlineto{\pgfqpoint{3.075000in}{3.207383in}}%
\pgfpathlineto{\pgfqpoint{3.075000in}{3.204433in}}%
\pgfpathmoveto{\pgfqpoint{3.070459in}{3.207383in}}%
\pgfpathlineto{\pgfqpoint{3.070459in}{3.207383in}}%
\pgfpathlineto{\pgfqpoint{3.070459in}{3.210332in}}%
\pgfpathlineto{\pgfqpoint{3.075000in}{3.210332in}}%
\pgfpathlineto{\pgfqpoint{3.075000in}{3.207383in}}%
\pgfpathmoveto{\pgfqpoint{3.070459in}{3.210332in}}%
\pgfpathlineto{\pgfqpoint{3.070459in}{3.210332in}}%
\pgfpathlineto{\pgfqpoint{3.070459in}{3.213281in}}%
\pgfpathlineto{\pgfqpoint{3.075000in}{3.213281in}}%
\pgfpathlineto{\pgfqpoint{3.075000in}{3.210332in}}%
\pgfpathmoveto{\pgfqpoint{3.070459in}{3.213281in}}%
\pgfpathlineto{\pgfqpoint{3.070459in}{3.213281in}}%
\pgfpathlineto{\pgfqpoint{3.070459in}{3.216230in}}%
\pgfpathlineto{\pgfqpoint{3.075000in}{3.216230in}}%
\pgfpathlineto{\pgfqpoint{3.075000in}{3.213281in}}%
\pgfpathmoveto{\pgfqpoint{3.070459in}{3.216230in}}%
\pgfpathlineto{\pgfqpoint{3.070459in}{3.216230in}}%
\pgfpathlineto{\pgfqpoint{3.070459in}{3.219180in}}%
\pgfpathlineto{\pgfqpoint{3.075000in}{3.219180in}}%
\pgfpathlineto{\pgfqpoint{3.075000in}{3.216230in}}%
\pgfpathmoveto{\pgfqpoint{3.070459in}{3.219180in}}%
\pgfpathlineto{\pgfqpoint{3.070459in}{3.219180in}}%
\pgfpathlineto{\pgfqpoint{3.070459in}{3.222129in}}%
\pgfpathlineto{\pgfqpoint{3.075000in}{3.222129in}}%
\pgfpathlineto{\pgfqpoint{3.075000in}{3.219180in}}%
\pgfpathmoveto{\pgfqpoint{3.070459in}{3.222129in}}%
\pgfpathlineto{\pgfqpoint{3.070459in}{3.222129in}}%
\pgfpathlineto{\pgfqpoint{3.070459in}{3.225078in}}%
\pgfpathlineto{\pgfqpoint{3.075000in}{3.225078in}}%
\pgfpathlineto{\pgfqpoint{3.075000in}{3.222129in}}%
\pgfpathmoveto{\pgfqpoint{3.070459in}{3.225078in}}%
\pgfpathlineto{\pgfqpoint{3.070459in}{3.225078in}}%
\pgfpathlineto{\pgfqpoint{3.070459in}{3.228028in}}%
\pgfpathlineto{\pgfqpoint{3.075000in}{3.228028in}}%
\pgfpathlineto{\pgfqpoint{3.075000in}{3.225078in}}%
\pgfpathmoveto{\pgfqpoint{3.070459in}{3.228028in}}%
\pgfpathlineto{\pgfqpoint{3.070459in}{3.228028in}}%
\pgfpathlineto{\pgfqpoint{3.070459in}{3.230977in}}%
\pgfpathlineto{\pgfqpoint{3.075000in}{3.230977in}}%
\pgfpathlineto{\pgfqpoint{3.075000in}{3.228028in}}%
\pgfpathmoveto{\pgfqpoint{3.070459in}{3.230977in}}%
\pgfpathlineto{\pgfqpoint{3.070459in}{3.230977in}}%
\pgfpathlineto{\pgfqpoint{3.070459in}{3.233926in}}%
\pgfpathlineto{\pgfqpoint{3.075000in}{3.233926in}}%
\pgfpathlineto{\pgfqpoint{3.075000in}{3.230977in}}%
\pgfpathmoveto{\pgfqpoint{3.070459in}{3.233926in}}%
\pgfpathlineto{\pgfqpoint{3.070459in}{3.233926in}}%
\pgfpathlineto{\pgfqpoint{3.070459in}{3.236875in}}%
\pgfpathlineto{\pgfqpoint{3.075000in}{3.236875in}}%
\pgfpathlineto{\pgfqpoint{3.075000in}{3.233926in}}%
\pgfpathmoveto{\pgfqpoint{3.070459in}{3.236875in}}%
\pgfpathlineto{\pgfqpoint{3.070459in}{3.236875in}}%
\pgfpathlineto{\pgfqpoint{3.070459in}{3.239824in}}%
\pgfpathlineto{\pgfqpoint{3.075000in}{3.239824in}}%
\pgfpathlineto{\pgfqpoint{3.075000in}{3.236875in}}%
\pgfpathmoveto{\pgfqpoint{3.070459in}{3.239824in}}%
\pgfpathlineto{\pgfqpoint{3.070459in}{3.239824in}}%
\pgfpathlineto{\pgfqpoint{3.070459in}{3.242774in}}%
\pgfpathlineto{\pgfqpoint{3.075000in}{3.242774in}}%
\pgfpathlineto{\pgfqpoint{3.075000in}{3.239824in}}%
\pgfpathmoveto{\pgfqpoint{3.070459in}{3.242774in}}%
\pgfpathlineto{\pgfqpoint{3.070459in}{3.242774in}}%
\pgfpathlineto{\pgfqpoint{3.070459in}{3.245723in}}%
\pgfpathlineto{\pgfqpoint{3.075000in}{3.245723in}}%
\pgfpathlineto{\pgfqpoint{3.075000in}{3.242774in}}%
\pgfpathmoveto{\pgfqpoint{3.070459in}{3.245723in}}%
\pgfpathlineto{\pgfqpoint{3.070459in}{3.245723in}}%
\pgfpathlineto{\pgfqpoint{3.070459in}{3.248672in}}%
\pgfpathlineto{\pgfqpoint{3.075000in}{3.248672in}}%
\pgfpathlineto{\pgfqpoint{3.075000in}{3.245723in}}%
\pgfpathmoveto{\pgfqpoint{3.070459in}{3.248672in}}%
\pgfpathlineto{\pgfqpoint{3.070459in}{3.248672in}}%
\pgfpathlineto{\pgfqpoint{3.070459in}{3.251621in}}%
\pgfpathlineto{\pgfqpoint{3.075000in}{3.251621in}}%
\pgfpathlineto{\pgfqpoint{3.075000in}{3.248672in}}%
\pgfpathmoveto{\pgfqpoint{3.070459in}{3.251621in}}%
\pgfpathlineto{\pgfqpoint{3.070459in}{3.251621in}}%
\pgfpathlineto{\pgfqpoint{3.070459in}{3.254570in}}%
\pgfpathlineto{\pgfqpoint{3.075000in}{3.254570in}}%
\pgfpathlineto{\pgfqpoint{3.075000in}{3.251621in}}%
\pgfpathmoveto{\pgfqpoint{3.070459in}{3.254570in}}%
\pgfpathlineto{\pgfqpoint{3.070459in}{3.254570in}}%
\pgfpathlineto{\pgfqpoint{3.070459in}{3.257519in}}%
\pgfpathlineto{\pgfqpoint{3.075000in}{3.257519in}}%
\pgfpathlineto{\pgfqpoint{3.075000in}{3.254570in}}%
\pgfpathmoveto{\pgfqpoint{3.070459in}{3.257519in}}%
\pgfpathlineto{\pgfqpoint{3.070459in}{3.257519in}}%
\pgfpathlineto{\pgfqpoint{3.070459in}{3.260469in}}%
\pgfpathlineto{\pgfqpoint{3.075000in}{3.260469in}}%
\pgfpathlineto{\pgfqpoint{3.075000in}{3.257519in}}%
\pgfpathmoveto{\pgfqpoint{3.070459in}{3.260469in}}%
\pgfpathlineto{\pgfqpoint{3.070459in}{3.260469in}}%
\pgfpathlineto{\pgfqpoint{3.070459in}{3.263418in}}%
\pgfpathlineto{\pgfqpoint{3.075000in}{3.263418in}}%
\pgfpathlineto{\pgfqpoint{3.075000in}{3.260469in}}%
\pgfpathmoveto{\pgfqpoint{3.070459in}{3.263418in}}%
\pgfpathlineto{\pgfqpoint{3.070459in}{3.263418in}}%
\pgfpathlineto{\pgfqpoint{3.070459in}{3.266367in}}%
\pgfpathlineto{\pgfqpoint{3.075000in}{3.266367in}}%
\pgfpathlineto{\pgfqpoint{3.075000in}{3.263418in}}%
\pgfpathmoveto{\pgfqpoint{3.070459in}{3.266367in}}%
\pgfpathlineto{\pgfqpoint{3.070459in}{3.266367in}}%
\pgfpathlineto{\pgfqpoint{3.070459in}{3.269316in}}%
\pgfpathlineto{\pgfqpoint{3.075000in}{3.269316in}}%
\pgfpathlineto{\pgfqpoint{3.075000in}{3.266367in}}%
\pgfpathmoveto{\pgfqpoint{3.070459in}{3.269316in}}%
\pgfpathlineto{\pgfqpoint{3.070459in}{3.269316in}}%
\pgfpathlineto{\pgfqpoint{3.070459in}{3.272265in}}%
\pgfpathlineto{\pgfqpoint{3.075000in}{3.272265in}}%
\pgfpathlineto{\pgfqpoint{3.075000in}{3.269316in}}%
\pgfpathmoveto{\pgfqpoint{3.070459in}{3.272265in}}%
\pgfpathlineto{\pgfqpoint{3.070459in}{3.272265in}}%
\pgfpathlineto{\pgfqpoint{3.070459in}{3.275214in}}%
\pgfpathlineto{\pgfqpoint{3.075000in}{3.275214in}}%
\pgfpathlineto{\pgfqpoint{3.075000in}{3.272265in}}%
\pgfpathmoveto{\pgfqpoint{3.070459in}{3.275214in}}%
\pgfpathlineto{\pgfqpoint{3.070459in}{3.275214in}}%
\pgfpathlineto{\pgfqpoint{3.070459in}{3.278164in}}%
\pgfpathlineto{\pgfqpoint{3.075000in}{3.278164in}}%
\pgfpathlineto{\pgfqpoint{3.075000in}{3.275214in}}%
\pgfpathmoveto{\pgfqpoint{3.070459in}{3.278164in}}%
\pgfpathlineto{\pgfqpoint{3.070459in}{3.278164in}}%
\pgfpathlineto{\pgfqpoint{3.070459in}{3.281113in}}%
\pgfpathlineto{\pgfqpoint{3.075000in}{3.281113in}}%
\pgfpathlineto{\pgfqpoint{3.075000in}{3.278164in}}%
\pgfpathmoveto{\pgfqpoint{3.070459in}{3.281113in}}%
\pgfpathlineto{\pgfqpoint{3.070459in}{3.281113in}}%
\pgfpathlineto{\pgfqpoint{3.070459in}{3.284062in}}%
\pgfpathlineto{\pgfqpoint{3.075000in}{3.284062in}}%
\pgfpathlineto{\pgfqpoint{3.075000in}{3.281113in}}%
\pgfpathmoveto{\pgfqpoint{3.070459in}{3.284062in}}%
\pgfpathlineto{\pgfqpoint{3.070459in}{3.284062in}}%
\pgfpathlineto{\pgfqpoint{3.070459in}{3.287011in}}%
\pgfpathlineto{\pgfqpoint{3.075000in}{3.287011in}}%
\pgfpathlineto{\pgfqpoint{3.075000in}{3.284062in}}%
\pgfpathmoveto{\pgfqpoint{3.070459in}{3.287011in}}%
\pgfpathlineto{\pgfqpoint{3.070459in}{3.287011in}}%
\pgfpathlineto{\pgfqpoint{3.070459in}{3.289960in}}%
\pgfpathlineto{\pgfqpoint{3.075000in}{3.289960in}}%
\pgfpathlineto{\pgfqpoint{3.075000in}{3.287011in}}%
\pgfpathmoveto{\pgfqpoint{3.070459in}{3.289960in}}%
\pgfpathlineto{\pgfqpoint{3.070459in}{3.289960in}}%
\pgfpathlineto{\pgfqpoint{3.070459in}{3.292909in}}%
\pgfpathlineto{\pgfqpoint{3.075000in}{3.292909in}}%
\pgfpathlineto{\pgfqpoint{3.075000in}{3.289960in}}%
\pgfpathmoveto{\pgfqpoint{3.070459in}{3.292909in}}%
\pgfpathlineto{\pgfqpoint{3.070459in}{3.292909in}}%
\pgfpathlineto{\pgfqpoint{3.070459in}{3.295859in}}%
\pgfpathlineto{\pgfqpoint{3.075000in}{3.295859in}}%
\pgfpathlineto{\pgfqpoint{3.075000in}{3.292909in}}%
\pgfpathmoveto{\pgfqpoint{3.070459in}{3.295859in}}%
\pgfpathlineto{\pgfqpoint{3.070459in}{3.295859in}}%
\pgfpathlineto{\pgfqpoint{3.070459in}{3.298808in}}%
\pgfpathlineto{\pgfqpoint{3.075000in}{3.298808in}}%
\pgfpathlineto{\pgfqpoint{3.075000in}{3.295859in}}%
\pgfpathmoveto{\pgfqpoint{3.070459in}{3.298808in}}%
\pgfpathlineto{\pgfqpoint{3.070459in}{3.298808in}}%
\pgfpathlineto{\pgfqpoint{3.070459in}{3.301757in}}%
\pgfpathlineto{\pgfqpoint{3.075000in}{3.301757in}}%
\pgfpathlineto{\pgfqpoint{3.075000in}{3.298808in}}%
\pgfpathmoveto{\pgfqpoint{3.070459in}{3.301757in}}%
\pgfpathlineto{\pgfqpoint{3.070459in}{3.301757in}}%
\pgfpathlineto{\pgfqpoint{3.070459in}{3.304706in}}%
\pgfpathlineto{\pgfqpoint{3.075000in}{3.304706in}}%
\pgfpathlineto{\pgfqpoint{3.075000in}{3.301757in}}%
\pgfpathmoveto{\pgfqpoint{3.070459in}{3.304706in}}%
\pgfpathlineto{\pgfqpoint{3.070459in}{3.304706in}}%
\pgfpathlineto{\pgfqpoint{3.070459in}{3.307655in}}%
\pgfpathlineto{\pgfqpoint{3.075000in}{3.307655in}}%
\pgfpathlineto{\pgfqpoint{3.075000in}{3.304706in}}%
\pgfpathmoveto{\pgfqpoint{3.070459in}{3.307655in}}%
\pgfpathlineto{\pgfqpoint{3.070459in}{3.307655in}}%
\pgfpathlineto{\pgfqpoint{3.070459in}{3.310604in}}%
\pgfpathlineto{\pgfqpoint{3.075000in}{3.310604in}}%
\pgfpathlineto{\pgfqpoint{3.075000in}{3.307655in}}%
\pgfpathmoveto{\pgfqpoint{3.070459in}{3.310604in}}%
\pgfpathlineto{\pgfqpoint{3.070459in}{3.310604in}}%
\pgfpathlineto{\pgfqpoint{3.070459in}{3.313554in}}%
\pgfpathlineto{\pgfqpoint{3.075000in}{3.313554in}}%
\pgfpathlineto{\pgfqpoint{3.075000in}{3.310604in}}%
\pgfpathmoveto{\pgfqpoint{3.070459in}{3.313554in}}%
\pgfpathlineto{\pgfqpoint{3.070459in}{3.313554in}}%
\pgfpathlineto{\pgfqpoint{3.070459in}{3.316503in}}%
\pgfpathlineto{\pgfqpoint{3.075000in}{3.316503in}}%
\pgfpathlineto{\pgfqpoint{3.075000in}{3.313554in}}%
\pgfpathmoveto{\pgfqpoint{3.070459in}{3.316503in}}%
\pgfpathlineto{\pgfqpoint{3.070459in}{3.316503in}}%
\pgfpathlineto{\pgfqpoint{3.070459in}{3.319452in}}%
\pgfpathlineto{\pgfqpoint{3.075000in}{3.319452in}}%
\pgfpathlineto{\pgfqpoint{3.075000in}{3.316503in}}%
\pgfpathmoveto{\pgfqpoint{3.070459in}{3.319452in}}%
\pgfpathlineto{\pgfqpoint{3.070459in}{3.319452in}}%
\pgfpathlineto{\pgfqpoint{3.070459in}{3.322401in}}%
\pgfpathlineto{\pgfqpoint{3.075000in}{3.322401in}}%
\pgfpathlineto{\pgfqpoint{3.075000in}{3.319452in}}%
\pgfpathmoveto{\pgfqpoint{3.070459in}{3.322401in}}%
\pgfpathlineto{\pgfqpoint{3.070459in}{3.322401in}}%
\pgfpathlineto{\pgfqpoint{3.070459in}{3.325350in}}%
\pgfpathlineto{\pgfqpoint{3.075000in}{3.325350in}}%
\pgfpathlineto{\pgfqpoint{3.075000in}{3.322401in}}%
\pgfpathmoveto{\pgfqpoint{3.070459in}{3.325350in}}%
\pgfpathlineto{\pgfqpoint{3.070459in}{3.325350in}}%
\pgfpathlineto{\pgfqpoint{3.070459in}{3.328299in}}%
\pgfpathlineto{\pgfqpoint{3.075000in}{3.328299in}}%
\pgfpathlineto{\pgfqpoint{3.075000in}{3.325350in}}%
\pgfpathmoveto{\pgfqpoint{3.070459in}{3.328299in}}%
\pgfpathlineto{\pgfqpoint{3.070459in}{3.328299in}}%
\pgfpathlineto{\pgfqpoint{3.070459in}{3.331249in}}%
\pgfpathlineto{\pgfqpoint{3.075000in}{3.331249in}}%
\pgfpathlineto{\pgfqpoint{3.075000in}{3.328299in}}%
\pgfpathmoveto{\pgfqpoint{3.070459in}{3.331249in}}%
\pgfpathlineto{\pgfqpoint{3.070459in}{3.331249in}}%
\pgfpathlineto{\pgfqpoint{3.070459in}{3.334198in}}%
\pgfpathlineto{\pgfqpoint{3.075000in}{3.334198in}}%
\pgfpathlineto{\pgfqpoint{3.075000in}{3.331249in}}%
\pgfpathmoveto{\pgfqpoint{3.070459in}{3.334198in}}%
\pgfpathlineto{\pgfqpoint{3.070459in}{3.334198in}}%
\pgfpathlineto{\pgfqpoint{3.070459in}{3.337147in}}%
\pgfpathlineto{\pgfqpoint{3.075000in}{3.337147in}}%
\pgfpathlineto{\pgfqpoint{3.075000in}{3.334198in}}%
\pgfpathmoveto{\pgfqpoint{3.070459in}{3.337147in}}%
\pgfpathlineto{\pgfqpoint{3.070459in}{3.337147in}}%
\pgfpathlineto{\pgfqpoint{3.070459in}{3.340096in}}%
\pgfpathlineto{\pgfqpoint{3.075000in}{3.340096in}}%
\pgfpathlineto{\pgfqpoint{3.075000in}{3.337147in}}%
\pgfpathmoveto{\pgfqpoint{3.070459in}{3.340096in}}%
\pgfpathlineto{\pgfqpoint{3.070459in}{3.340096in}}%
\pgfpathlineto{\pgfqpoint{3.070459in}{3.343045in}}%
\pgfpathlineto{\pgfqpoint{3.075000in}{3.343045in}}%
\pgfpathlineto{\pgfqpoint{3.075000in}{3.340096in}}%
\pgfpathmoveto{\pgfqpoint{3.070459in}{3.343045in}}%
\pgfpathlineto{\pgfqpoint{3.070459in}{3.343045in}}%
\pgfpathlineto{\pgfqpoint{3.070459in}{3.345995in}}%
\pgfpathlineto{\pgfqpoint{3.075000in}{3.345995in}}%
\pgfpathlineto{\pgfqpoint{3.075000in}{3.343045in}}%
\pgfpathmoveto{\pgfqpoint{3.070459in}{3.345995in}}%
\pgfpathlineto{\pgfqpoint{3.070459in}{3.345995in}}%
\pgfpathlineto{\pgfqpoint{3.070459in}{3.348944in}}%
\pgfpathlineto{\pgfqpoint{3.075000in}{3.348944in}}%
\pgfpathlineto{\pgfqpoint{3.075000in}{3.345995in}}%
\pgfpathmoveto{\pgfqpoint{3.070459in}{3.348944in}}%
\pgfpathlineto{\pgfqpoint{3.070459in}{3.348944in}}%
\pgfpathlineto{\pgfqpoint{3.070459in}{3.351893in}}%
\pgfpathlineto{\pgfqpoint{3.075000in}{3.351893in}}%
\pgfpathlineto{\pgfqpoint{3.075000in}{3.348944in}}%
\pgfpathmoveto{\pgfqpoint{3.070459in}{3.351893in}}%
\pgfpathlineto{\pgfqpoint{3.070459in}{3.351893in}}%
\pgfpathlineto{\pgfqpoint{3.070459in}{3.354842in}}%
\pgfpathlineto{\pgfqpoint{3.075000in}{3.354842in}}%
\pgfpathlineto{\pgfqpoint{3.075000in}{3.351893in}}%
\pgfpathmoveto{\pgfqpoint{3.070459in}{3.354842in}}%
\pgfpathlineto{\pgfqpoint{3.070459in}{3.354842in}}%
\pgfpathlineto{\pgfqpoint{3.070459in}{3.357791in}}%
\pgfpathlineto{\pgfqpoint{3.075000in}{3.357791in}}%
\pgfpathlineto{\pgfqpoint{3.075000in}{3.354842in}}%
\pgfpathmoveto{\pgfqpoint{3.070459in}{3.357791in}}%
\pgfpathlineto{\pgfqpoint{3.070459in}{3.357791in}}%
\pgfpathlineto{\pgfqpoint{3.070459in}{3.360741in}}%
\pgfpathlineto{\pgfqpoint{3.075000in}{3.360741in}}%
\pgfpathlineto{\pgfqpoint{3.075000in}{3.357791in}}%
\pgfpathmoveto{\pgfqpoint{3.070459in}{3.360741in}}%
\pgfpathlineto{\pgfqpoint{3.070459in}{3.360741in}}%
\pgfpathlineto{\pgfqpoint{3.070459in}{3.363690in}}%
\pgfpathlineto{\pgfqpoint{3.075000in}{3.363690in}}%
\pgfpathlineto{\pgfqpoint{3.075000in}{3.360741in}}%
\pgfpathmoveto{\pgfqpoint{3.070459in}{3.363690in}}%
\pgfpathlineto{\pgfqpoint{3.070459in}{3.363690in}}%
\pgfpathlineto{\pgfqpoint{3.070459in}{3.366639in}}%
\pgfpathlineto{\pgfqpoint{3.075000in}{3.366639in}}%
\pgfpathlineto{\pgfqpoint{3.075000in}{3.363690in}}%
\pgfpathmoveto{\pgfqpoint{3.070459in}{3.366639in}}%
\pgfpathlineto{\pgfqpoint{3.070459in}{3.366639in}}%
\pgfpathlineto{\pgfqpoint{3.070459in}{3.369588in}}%
\pgfpathlineto{\pgfqpoint{3.075000in}{3.369588in}}%
\pgfpathlineto{\pgfqpoint{3.075000in}{3.366639in}}%
\pgfpathmoveto{\pgfqpoint{3.070459in}{3.369588in}}%
\pgfpathlineto{\pgfqpoint{3.070459in}{3.369588in}}%
\pgfpathlineto{\pgfqpoint{3.070459in}{3.372537in}}%
\pgfpathlineto{\pgfqpoint{3.075000in}{3.372537in}}%
\pgfpathlineto{\pgfqpoint{3.075000in}{3.369588in}}%
\pgfpathmoveto{\pgfqpoint{3.070459in}{3.372537in}}%
\pgfpathlineto{\pgfqpoint{3.070459in}{3.372537in}}%
\pgfpathlineto{\pgfqpoint{3.070459in}{3.375487in}}%
\pgfpathlineto{\pgfqpoint{3.075000in}{3.375487in}}%
\pgfpathlineto{\pgfqpoint{3.075000in}{3.372537in}}%
\pgfpathmoveto{\pgfqpoint{3.070459in}{3.375487in}}%
\pgfpathlineto{\pgfqpoint{3.070459in}{3.375487in}}%
\pgfpathlineto{\pgfqpoint{3.070459in}{3.378436in}}%
\pgfpathlineto{\pgfqpoint{3.075000in}{3.378436in}}%
\pgfpathlineto{\pgfqpoint{3.075000in}{3.375487in}}%
\pgfpathmoveto{\pgfqpoint{3.070459in}{3.378436in}}%
\pgfpathlineto{\pgfqpoint{3.070459in}{3.378436in}}%
\pgfpathlineto{\pgfqpoint{3.070459in}{3.381385in}}%
\pgfpathlineto{\pgfqpoint{3.075000in}{3.381385in}}%
\pgfpathlineto{\pgfqpoint{3.075000in}{3.378436in}}%
\pgfpathmoveto{\pgfqpoint{3.070459in}{3.381385in}}%
\pgfpathlineto{\pgfqpoint{3.070459in}{3.381385in}}%
\pgfpathlineto{\pgfqpoint{3.070459in}{3.384334in}}%
\pgfpathlineto{\pgfqpoint{3.075000in}{3.384334in}}%
\pgfpathlineto{\pgfqpoint{3.075000in}{3.381385in}}%
\pgfpathmoveto{\pgfqpoint{3.070459in}{3.384334in}}%
\pgfpathlineto{\pgfqpoint{3.070459in}{3.384334in}}%
\pgfpathlineto{\pgfqpoint{3.070459in}{3.387283in}}%
\pgfpathlineto{\pgfqpoint{3.075000in}{3.387283in}}%
\pgfpathlineto{\pgfqpoint{3.075000in}{3.384334in}}%
\pgfpathmoveto{\pgfqpoint{3.070459in}{3.387283in}}%
\pgfpathlineto{\pgfqpoint{3.070459in}{3.387283in}}%
\pgfpathlineto{\pgfqpoint{3.070459in}{3.390233in}}%
\pgfpathlineto{\pgfqpoint{3.075000in}{3.390233in}}%
\pgfpathlineto{\pgfqpoint{3.075000in}{3.387283in}}%
\pgfpathmoveto{\pgfqpoint{3.070459in}{3.390233in}}%
\pgfpathlineto{\pgfqpoint{3.070459in}{3.390233in}}%
\pgfpathlineto{\pgfqpoint{3.070459in}{3.393182in}}%
\pgfpathlineto{\pgfqpoint{3.075000in}{3.393182in}}%
\pgfpathlineto{\pgfqpoint{3.075000in}{3.390233in}}%
\pgfpathmoveto{\pgfqpoint{3.070459in}{3.393182in}}%
\pgfpathlineto{\pgfqpoint{3.070459in}{3.393182in}}%
\pgfpathlineto{\pgfqpoint{3.070459in}{3.396131in}}%
\pgfpathlineto{\pgfqpoint{3.075000in}{3.396131in}}%
\pgfpathlineto{\pgfqpoint{3.075000in}{3.393182in}}%
\pgfpathmoveto{\pgfqpoint{3.070459in}{3.396131in}}%
\pgfpathlineto{\pgfqpoint{3.070459in}{3.396131in}}%
\pgfpathlineto{\pgfqpoint{3.070459in}{3.399080in}}%
\pgfpathlineto{\pgfqpoint{3.075000in}{3.399080in}}%
\pgfpathlineto{\pgfqpoint{3.075000in}{3.396131in}}%
\pgfpathmoveto{\pgfqpoint{3.070459in}{3.399080in}}%
\pgfpathlineto{\pgfqpoint{3.070459in}{3.399080in}}%
\pgfpathlineto{\pgfqpoint{3.070459in}{3.402029in}}%
\pgfpathlineto{\pgfqpoint{3.075000in}{3.402029in}}%
\pgfpathlineto{\pgfqpoint{3.075000in}{3.399080in}}%
\pgfpathmoveto{\pgfqpoint{3.070459in}{3.402029in}}%
\pgfpathlineto{\pgfqpoint{3.070459in}{3.402029in}}%
\pgfpathlineto{\pgfqpoint{3.070459in}{3.404978in}}%
\pgfpathlineto{\pgfqpoint{3.075000in}{3.404978in}}%
\pgfpathlineto{\pgfqpoint{3.075000in}{3.402029in}}%
\pgfpathmoveto{\pgfqpoint{3.070459in}{3.404978in}}%
\pgfpathlineto{\pgfqpoint{3.070459in}{3.404978in}}%
\pgfpathlineto{\pgfqpoint{3.070459in}{3.407928in}}%
\pgfpathlineto{\pgfqpoint{3.075000in}{3.407928in}}%
\pgfpathlineto{\pgfqpoint{3.075000in}{3.404978in}}%
\pgfpathmoveto{\pgfqpoint{3.070459in}{3.407928in}}%
\pgfpathlineto{\pgfqpoint{3.070459in}{3.407928in}}%
\pgfpathlineto{\pgfqpoint{3.070459in}{3.410877in}}%
\pgfpathlineto{\pgfqpoint{3.075000in}{3.410877in}}%
\pgfpathlineto{\pgfqpoint{3.075000in}{3.407928in}}%
\pgfpathmoveto{\pgfqpoint{3.070459in}{3.410877in}}%
\pgfpathlineto{\pgfqpoint{3.070459in}{3.410877in}}%
\pgfpathlineto{\pgfqpoint{3.070459in}{3.413826in}}%
\pgfpathlineto{\pgfqpoint{3.075000in}{3.413826in}}%
\pgfpathlineto{\pgfqpoint{3.075000in}{3.410877in}}%
\pgfpathmoveto{\pgfqpoint{3.070459in}{3.413826in}}%
\pgfpathlineto{\pgfqpoint{3.070459in}{3.413826in}}%
\pgfpathlineto{\pgfqpoint{3.070459in}{3.416775in}}%
\pgfpathlineto{\pgfqpoint{3.075000in}{3.416775in}}%
\pgfpathlineto{\pgfqpoint{3.075000in}{3.413826in}}%
\pgfpathmoveto{\pgfqpoint{3.070459in}{3.416775in}}%
\pgfpathlineto{\pgfqpoint{3.070459in}{3.416775in}}%
\pgfpathlineto{\pgfqpoint{3.070459in}{3.419724in}}%
\pgfpathlineto{\pgfqpoint{3.075000in}{3.419724in}}%
\pgfpathlineto{\pgfqpoint{3.075000in}{3.416775in}}%
\pgfpathmoveto{\pgfqpoint{3.070459in}{3.419724in}}%
\pgfpathlineto{\pgfqpoint{3.070459in}{3.419724in}}%
\pgfpathlineto{\pgfqpoint{3.070459in}{3.422674in}}%
\pgfpathlineto{\pgfqpoint{3.075000in}{3.422674in}}%
\pgfpathlineto{\pgfqpoint{3.075000in}{3.419724in}}%
\pgfpathmoveto{\pgfqpoint{3.070459in}{3.422674in}}%
\pgfpathlineto{\pgfqpoint{3.070459in}{3.422674in}}%
\pgfpathlineto{\pgfqpoint{3.070459in}{3.425623in}}%
\pgfpathlineto{\pgfqpoint{3.075000in}{3.425623in}}%
\pgfpathlineto{\pgfqpoint{3.075000in}{3.422674in}}%
\pgfpathmoveto{\pgfqpoint{3.070459in}{3.425623in}}%
\pgfpathlineto{\pgfqpoint{3.070459in}{3.425623in}}%
\pgfpathlineto{\pgfqpoint{3.070459in}{3.428572in}}%
\pgfpathlineto{\pgfqpoint{3.075000in}{3.428572in}}%
\pgfpathlineto{\pgfqpoint{3.075000in}{3.425623in}}%
\pgfpathmoveto{\pgfqpoint{3.070459in}{3.428572in}}%
\pgfpathlineto{\pgfqpoint{3.070459in}{3.428572in}}%
\pgfpathlineto{\pgfqpoint{3.070459in}{3.431522in}}%
\pgfpathlineto{\pgfqpoint{3.075000in}{3.431522in}}%
\pgfpathlineto{\pgfqpoint{3.075000in}{3.428572in}}%
\pgfpathmoveto{\pgfqpoint{3.070459in}{3.431522in}}%
\pgfpathlineto{\pgfqpoint{3.070459in}{3.431522in}}%
\pgfpathlineto{\pgfqpoint{3.070459in}{3.434471in}}%
\pgfpathlineto{\pgfqpoint{3.075000in}{3.434471in}}%
\pgfpathlineto{\pgfqpoint{3.075000in}{3.431522in}}%
\pgfpathmoveto{\pgfqpoint{3.070459in}{3.434471in}}%
\pgfpathlineto{\pgfqpoint{3.070459in}{3.434471in}}%
\pgfpathlineto{\pgfqpoint{3.070459in}{3.437420in}}%
\pgfpathlineto{\pgfqpoint{3.075000in}{3.437420in}}%
\pgfpathlineto{\pgfqpoint{3.075000in}{3.434471in}}%
\pgfpathmoveto{\pgfqpoint{3.070459in}{3.437420in}}%
\pgfpathlineto{\pgfqpoint{3.070459in}{3.437420in}}%
\pgfpathlineto{\pgfqpoint{3.070459in}{3.440370in}}%
\pgfpathlineto{\pgfqpoint{3.075000in}{3.440370in}}%
\pgfpathlineto{\pgfqpoint{3.075000in}{3.437420in}}%
\pgfpathmoveto{\pgfqpoint{3.070459in}{3.440370in}}%
\pgfpathlineto{\pgfqpoint{3.070459in}{3.440370in}}%
\pgfpathlineto{\pgfqpoint{3.070459in}{3.443319in}}%
\pgfpathlineto{\pgfqpoint{3.075000in}{3.443319in}}%
\pgfpathlineto{\pgfqpoint{3.075000in}{3.440370in}}%
\pgfpathmoveto{\pgfqpoint{3.070459in}{3.443319in}}%
\pgfpathlineto{\pgfqpoint{3.070459in}{3.443319in}}%
\pgfpathlineto{\pgfqpoint{3.070459in}{3.446268in}}%
\pgfpathlineto{\pgfqpoint{3.075000in}{3.446268in}}%
\pgfpathlineto{\pgfqpoint{3.075000in}{3.443319in}}%
\pgfpathmoveto{\pgfqpoint{3.070459in}{3.446268in}}%
\pgfpathlineto{\pgfqpoint{3.070459in}{3.446268in}}%
\pgfpathlineto{\pgfqpoint{3.070459in}{3.449218in}}%
\pgfpathlineto{\pgfqpoint{3.075000in}{3.449218in}}%
\pgfpathlineto{\pgfqpoint{3.075000in}{3.446268in}}%
\pgfpathmoveto{\pgfqpoint{3.070459in}{3.449218in}}%
\pgfpathlineto{\pgfqpoint{3.070459in}{3.449218in}}%
\pgfpathlineto{\pgfqpoint{3.070459in}{3.452167in}}%
\pgfpathlineto{\pgfqpoint{3.075000in}{3.452167in}}%
\pgfpathlineto{\pgfqpoint{3.075000in}{3.449218in}}%
\pgfpathmoveto{\pgfqpoint{3.070459in}{3.452167in}}%
\pgfpathlineto{\pgfqpoint{3.070459in}{3.452167in}}%
\pgfpathlineto{\pgfqpoint{3.070459in}{3.455117in}}%
\pgfpathlineto{\pgfqpoint{3.075000in}{3.455117in}}%
\pgfpathlineto{\pgfqpoint{3.075000in}{3.452167in}}%
\pgfpathmoveto{\pgfqpoint{3.070459in}{3.455117in}}%
\pgfpathlineto{\pgfqpoint{3.070459in}{3.455117in}}%
\pgfpathlineto{\pgfqpoint{3.070459in}{3.458066in}}%
\pgfpathlineto{\pgfqpoint{3.075000in}{3.458066in}}%
\pgfpathlineto{\pgfqpoint{3.075000in}{3.455117in}}%
\pgfpathmoveto{\pgfqpoint{3.070459in}{3.458066in}}%
\pgfpathlineto{\pgfqpoint{3.070459in}{3.458066in}}%
\pgfpathlineto{\pgfqpoint{3.070459in}{3.461015in}}%
\pgfpathlineto{\pgfqpoint{3.075000in}{3.461015in}}%
\pgfpathlineto{\pgfqpoint{3.075000in}{3.458066in}}%
\pgfpathmoveto{\pgfqpoint{3.070459in}{3.461015in}}%
\pgfpathlineto{\pgfqpoint{3.070459in}{3.461015in}}%
\pgfpathlineto{\pgfqpoint{3.070459in}{3.463965in}}%
\pgfpathlineto{\pgfqpoint{3.075000in}{3.463965in}}%
\pgfpathlineto{\pgfqpoint{3.075000in}{3.461015in}}%
\pgfpathmoveto{\pgfqpoint{3.070459in}{3.463965in}}%
\pgfpathlineto{\pgfqpoint{3.070459in}{3.463965in}}%
\pgfpathlineto{\pgfqpoint{3.070459in}{3.466914in}}%
\pgfpathlineto{\pgfqpoint{3.075000in}{3.466914in}}%
\pgfpathlineto{\pgfqpoint{3.075000in}{3.463965in}}%
\pgfpathmoveto{\pgfqpoint{3.070459in}{3.466914in}}%
\pgfpathlineto{\pgfqpoint{3.070459in}{3.466914in}}%
\pgfpathlineto{\pgfqpoint{3.070459in}{3.469863in}}%
\pgfpathlineto{\pgfqpoint{3.075000in}{3.469863in}}%
\pgfpathlineto{\pgfqpoint{3.075000in}{3.466914in}}%
\pgfpathmoveto{\pgfqpoint{3.070459in}{3.469863in}}%
\pgfpathlineto{\pgfqpoint{3.070459in}{3.469863in}}%
\pgfpathlineto{\pgfqpoint{3.070459in}{3.472813in}}%
\pgfpathlineto{\pgfqpoint{3.075000in}{3.472813in}}%
\pgfpathlineto{\pgfqpoint{3.075000in}{3.469863in}}%
\pgfpathmoveto{\pgfqpoint{3.070459in}{3.472813in}}%
\pgfpathlineto{\pgfqpoint{3.070459in}{3.472813in}}%
\pgfpathlineto{\pgfqpoint{3.070459in}{3.475762in}}%
\pgfpathlineto{\pgfqpoint{3.075000in}{3.475762in}}%
\pgfpathlineto{\pgfqpoint{3.075000in}{3.472813in}}%
\pgfpathmoveto{\pgfqpoint{3.070459in}{3.475762in}}%
\pgfpathlineto{\pgfqpoint{3.070459in}{3.475762in}}%
\pgfpathlineto{\pgfqpoint{3.070459in}{3.478712in}}%
\pgfpathlineto{\pgfqpoint{3.075000in}{3.478712in}}%
\pgfpathlineto{\pgfqpoint{3.075000in}{3.475762in}}%
\pgfpathmoveto{\pgfqpoint{3.070459in}{3.478712in}}%
\pgfpathlineto{\pgfqpoint{3.070459in}{3.478712in}}%
\pgfpathlineto{\pgfqpoint{3.070459in}{3.481661in}}%
\pgfpathlineto{\pgfqpoint{3.075000in}{3.481661in}}%
\pgfpathlineto{\pgfqpoint{3.075000in}{3.478712in}}%
\pgfpathmoveto{\pgfqpoint{3.070459in}{3.481661in}}%
\pgfpathlineto{\pgfqpoint{3.070459in}{3.481661in}}%
\pgfpathlineto{\pgfqpoint{3.070459in}{3.484610in}}%
\pgfpathlineto{\pgfqpoint{3.075000in}{3.484610in}}%
\pgfpathlineto{\pgfqpoint{3.075000in}{3.481661in}}%
\pgfpathmoveto{\pgfqpoint{3.070459in}{3.484610in}}%
\pgfpathlineto{\pgfqpoint{3.070459in}{3.484610in}}%
\pgfpathlineto{\pgfqpoint{3.070459in}{3.487560in}}%
\pgfpathlineto{\pgfqpoint{3.075000in}{3.487560in}}%
\pgfpathlineto{\pgfqpoint{3.075000in}{3.484610in}}%
\pgfpathmoveto{\pgfqpoint{3.070459in}{3.487560in}}%
\pgfpathlineto{\pgfqpoint{3.070459in}{3.487560in}}%
\pgfpathlineto{\pgfqpoint{3.070459in}{3.490509in}}%
\pgfpathlineto{\pgfqpoint{3.075000in}{3.490509in}}%
\pgfpathlineto{\pgfqpoint{3.075000in}{3.487560in}}%
\pgfpathmoveto{\pgfqpoint{3.070459in}{3.490509in}}%
\pgfpathlineto{\pgfqpoint{3.070459in}{3.490509in}}%
\pgfpathlineto{\pgfqpoint{3.070459in}{3.493458in}}%
\pgfpathlineto{\pgfqpoint{3.075000in}{3.493458in}}%
\pgfpathlineto{\pgfqpoint{3.075000in}{3.490509in}}%
\pgfpathmoveto{\pgfqpoint{3.070459in}{3.493458in}}%
\pgfpathlineto{\pgfqpoint{3.070459in}{3.493458in}}%
\pgfpathlineto{\pgfqpoint{3.070459in}{3.496408in}}%
\pgfpathlineto{\pgfqpoint{3.075000in}{3.496408in}}%
\pgfpathlineto{\pgfqpoint{3.075000in}{3.493458in}}%
\pgfpathmoveto{\pgfqpoint{3.070459in}{3.496408in}}%
\pgfpathlineto{\pgfqpoint{3.070459in}{3.496408in}}%
\pgfpathlineto{\pgfqpoint{3.070459in}{3.499357in}}%
\pgfpathlineto{\pgfqpoint{3.075000in}{3.499357in}}%
\pgfpathlineto{\pgfqpoint{3.075000in}{3.496408in}}%
\pgfpathmoveto{\pgfqpoint{3.070459in}{3.499357in}}%
\pgfpathlineto{\pgfqpoint{3.070459in}{3.499357in}}%
\pgfpathlineto{\pgfqpoint{3.070459in}{3.502307in}}%
\pgfpathlineto{\pgfqpoint{3.075000in}{3.502307in}}%
\pgfpathlineto{\pgfqpoint{3.075000in}{3.499357in}}%
\pgfpathmoveto{\pgfqpoint{3.070459in}{3.502307in}}%
\pgfpathlineto{\pgfqpoint{3.070459in}{3.502307in}}%
\pgfpathlineto{\pgfqpoint{3.070459in}{3.505256in}}%
\pgfpathlineto{\pgfqpoint{3.075000in}{3.505256in}}%
\pgfpathlineto{\pgfqpoint{3.075000in}{3.502307in}}%
\pgfpathmoveto{\pgfqpoint{3.070459in}{3.505256in}}%
\pgfpathlineto{\pgfqpoint{3.070459in}{3.505256in}}%
\pgfpathlineto{\pgfqpoint{3.070459in}{3.508205in}}%
\pgfpathlineto{\pgfqpoint{3.075000in}{3.508205in}}%
\pgfpathlineto{\pgfqpoint{3.075000in}{3.505256in}}%
\pgfpathmoveto{\pgfqpoint{3.070459in}{3.508205in}}%
\pgfpathlineto{\pgfqpoint{3.070459in}{3.508205in}}%
\pgfpathlineto{\pgfqpoint{3.070459in}{3.511155in}}%
\pgfpathlineto{\pgfqpoint{3.075000in}{3.511155in}}%
\pgfpathlineto{\pgfqpoint{3.075000in}{3.508205in}}%
\pgfpathmoveto{\pgfqpoint{3.070459in}{3.511155in}}%
\pgfpathlineto{\pgfqpoint{3.070459in}{3.511155in}}%
\pgfpathlineto{\pgfqpoint{3.070459in}{3.514104in}}%
\pgfpathlineto{\pgfqpoint{3.075000in}{3.514104in}}%
\pgfpathlineto{\pgfqpoint{3.075000in}{3.511155in}}%
\pgfpathmoveto{\pgfqpoint{3.070459in}{3.514104in}}%
\pgfpathlineto{\pgfqpoint{3.070459in}{3.514104in}}%
\pgfpathlineto{\pgfqpoint{3.070459in}{3.517053in}}%
\pgfpathlineto{\pgfqpoint{3.075000in}{3.517053in}}%
\pgfpathlineto{\pgfqpoint{3.075000in}{3.514104in}}%
\pgfpathmoveto{\pgfqpoint{3.070459in}{3.517053in}}%
\pgfpathlineto{\pgfqpoint{3.070459in}{3.517053in}}%
\pgfpathlineto{\pgfqpoint{3.070459in}{3.520003in}}%
\pgfpathlineto{\pgfqpoint{3.075000in}{3.520003in}}%
\pgfpathlineto{\pgfqpoint{3.075000in}{3.517053in}}%
\pgfpathclose%
\pgfusepath{fill}%
\end{pgfscope}%
\begin{pgfscope}%
\pgfpathrectangle{\pgfqpoint{0.750000in}{0.500000in}}{\pgfqpoint{4.650000in}{3.020000in}}%
\pgfusepath{clip}%
\pgfsetbuttcap%
\pgfsetmiterjoin%
\definecolor{currentfill}{rgb}{1.000000,0.000000,0.000000}%
\pgfsetfillcolor{currentfill}%
\pgfsetlinewidth{0.000000pt}%
\definecolor{currentstroke}{rgb}{0.000000,0.000000,0.000000}%
\pgfsetstrokecolor{currentstroke}%
\pgfsetstrokeopacity{0.000000}%
\pgfsetdash{}{0pt}%
\pgfpathmoveto{\pgfqpoint{0.750001in}{2.007051in}}%
\pgfpathlineto{\pgfqpoint{0.750001in}{2.010001in}}%
\pgfpathlineto{\pgfqpoint{0.754542in}{2.010001in}}%
\pgfpathlineto{\pgfqpoint{0.754542in}{2.007051in}}%
\pgfpathmoveto{\pgfqpoint{0.754542in}{2.007051in}}%
\pgfpathlineto{\pgfqpoint{0.754542in}{2.007051in}}%
\pgfpathlineto{\pgfqpoint{0.754542in}{2.010001in}}%
\pgfpathlineto{\pgfqpoint{0.759083in}{2.010001in}}%
\pgfpathlineto{\pgfqpoint{0.759083in}{2.007051in}}%
\pgfpathmoveto{\pgfqpoint{0.759083in}{2.007051in}}%
\pgfpathlineto{\pgfqpoint{0.759083in}{2.007051in}}%
\pgfpathlineto{\pgfqpoint{0.759083in}{2.010001in}}%
\pgfpathlineto{\pgfqpoint{0.763624in}{2.010001in}}%
\pgfpathlineto{\pgfqpoint{0.763624in}{2.007051in}}%
\pgfpathmoveto{\pgfqpoint{0.763624in}{2.007051in}}%
\pgfpathlineto{\pgfqpoint{0.763624in}{2.007051in}}%
\pgfpathlineto{\pgfqpoint{0.763624in}{2.010001in}}%
\pgfpathlineto{\pgfqpoint{0.768165in}{2.010001in}}%
\pgfpathlineto{\pgfqpoint{0.768165in}{2.007051in}}%
\pgfpathmoveto{\pgfqpoint{0.768165in}{2.007051in}}%
\pgfpathlineto{\pgfqpoint{0.768165in}{2.007051in}}%
\pgfpathlineto{\pgfqpoint{0.768165in}{2.010001in}}%
\pgfpathlineto{\pgfqpoint{0.772706in}{2.010001in}}%
\pgfpathlineto{\pgfqpoint{0.772706in}{2.007051in}}%
\pgfpathmoveto{\pgfqpoint{0.772706in}{2.007051in}}%
\pgfpathlineto{\pgfqpoint{0.772706in}{2.007051in}}%
\pgfpathlineto{\pgfqpoint{0.772706in}{2.010001in}}%
\pgfpathlineto{\pgfqpoint{0.777247in}{2.010001in}}%
\pgfpathlineto{\pgfqpoint{0.777247in}{2.007051in}}%
\pgfpathmoveto{\pgfqpoint{0.777247in}{2.007051in}}%
\pgfpathlineto{\pgfqpoint{0.777247in}{2.007051in}}%
\pgfpathlineto{\pgfqpoint{0.777247in}{2.010001in}}%
\pgfpathlineto{\pgfqpoint{0.781788in}{2.010001in}}%
\pgfpathlineto{\pgfqpoint{0.781788in}{2.007051in}}%
\pgfpathmoveto{\pgfqpoint{0.781788in}{2.007051in}}%
\pgfpathlineto{\pgfqpoint{0.781788in}{2.007051in}}%
\pgfpathlineto{\pgfqpoint{0.781788in}{2.010001in}}%
\pgfpathlineto{\pgfqpoint{0.786329in}{2.010001in}}%
\pgfpathlineto{\pgfqpoint{0.786329in}{2.007051in}}%
\pgfpathmoveto{\pgfqpoint{0.786329in}{2.007051in}}%
\pgfpathlineto{\pgfqpoint{0.786329in}{2.007051in}}%
\pgfpathlineto{\pgfqpoint{0.786329in}{2.010001in}}%
\pgfpathlineto{\pgfqpoint{0.790871in}{2.010001in}}%
\pgfpathlineto{\pgfqpoint{0.790871in}{2.007051in}}%
\pgfpathmoveto{\pgfqpoint{0.790871in}{2.007051in}}%
\pgfpathlineto{\pgfqpoint{0.790871in}{2.007051in}}%
\pgfpathlineto{\pgfqpoint{0.790871in}{2.010001in}}%
\pgfpathlineto{\pgfqpoint{0.795412in}{2.010001in}}%
\pgfpathlineto{\pgfqpoint{0.795412in}{2.007051in}}%
\pgfpathmoveto{\pgfqpoint{0.795412in}{2.007051in}}%
\pgfpathlineto{\pgfqpoint{0.795412in}{2.007051in}}%
\pgfpathlineto{\pgfqpoint{0.795412in}{2.010001in}}%
\pgfpathlineto{\pgfqpoint{0.799953in}{2.010001in}}%
\pgfpathlineto{\pgfqpoint{0.799953in}{2.007051in}}%
\pgfpathmoveto{\pgfqpoint{0.799953in}{2.007051in}}%
\pgfpathlineto{\pgfqpoint{0.799953in}{2.007051in}}%
\pgfpathlineto{\pgfqpoint{0.799953in}{2.010001in}}%
\pgfpathlineto{\pgfqpoint{0.804494in}{2.010001in}}%
\pgfpathlineto{\pgfqpoint{0.804494in}{2.007051in}}%
\pgfpathmoveto{\pgfqpoint{0.804494in}{2.007051in}}%
\pgfpathlineto{\pgfqpoint{0.804494in}{2.007051in}}%
\pgfpathlineto{\pgfqpoint{0.804494in}{2.010001in}}%
\pgfpathlineto{\pgfqpoint{0.809035in}{2.010001in}}%
\pgfpathlineto{\pgfqpoint{0.809035in}{2.007051in}}%
\pgfpathmoveto{\pgfqpoint{0.809035in}{2.007051in}}%
\pgfpathlineto{\pgfqpoint{0.809035in}{2.007051in}}%
\pgfpathlineto{\pgfqpoint{0.809035in}{2.010001in}}%
\pgfpathlineto{\pgfqpoint{0.813576in}{2.010001in}}%
\pgfpathlineto{\pgfqpoint{0.813576in}{2.007051in}}%
\pgfpathmoveto{\pgfqpoint{0.813576in}{2.007051in}}%
\pgfpathlineto{\pgfqpoint{0.813576in}{2.007051in}}%
\pgfpathlineto{\pgfqpoint{0.813576in}{2.010001in}}%
\pgfpathlineto{\pgfqpoint{0.818117in}{2.010001in}}%
\pgfpathlineto{\pgfqpoint{0.818117in}{2.007051in}}%
\pgfpathmoveto{\pgfqpoint{0.818117in}{2.007051in}}%
\pgfpathlineto{\pgfqpoint{0.818117in}{2.007051in}}%
\pgfpathlineto{\pgfqpoint{0.818117in}{2.010001in}}%
\pgfpathlineto{\pgfqpoint{0.822658in}{2.010001in}}%
\pgfpathlineto{\pgfqpoint{0.822658in}{2.007051in}}%
\pgfpathmoveto{\pgfqpoint{0.822658in}{2.007051in}}%
\pgfpathlineto{\pgfqpoint{0.822658in}{2.007051in}}%
\pgfpathlineto{\pgfqpoint{0.822658in}{2.010001in}}%
\pgfpathlineto{\pgfqpoint{0.827199in}{2.010001in}}%
\pgfpathlineto{\pgfqpoint{0.827199in}{2.007051in}}%
\pgfpathmoveto{\pgfqpoint{0.827199in}{2.007051in}}%
\pgfpathlineto{\pgfqpoint{0.827199in}{2.007051in}}%
\pgfpathlineto{\pgfqpoint{0.827199in}{2.010001in}}%
\pgfpathlineto{\pgfqpoint{0.831740in}{2.010001in}}%
\pgfpathlineto{\pgfqpoint{0.831740in}{2.007051in}}%
\pgfpathmoveto{\pgfqpoint{0.831740in}{2.007051in}}%
\pgfpathlineto{\pgfqpoint{0.831740in}{2.007051in}}%
\pgfpathlineto{\pgfqpoint{0.831740in}{2.010001in}}%
\pgfpathlineto{\pgfqpoint{0.836281in}{2.010001in}}%
\pgfpathlineto{\pgfqpoint{0.836281in}{2.007051in}}%
\pgfpathmoveto{\pgfqpoint{0.836281in}{2.007051in}}%
\pgfpathlineto{\pgfqpoint{0.836281in}{2.007051in}}%
\pgfpathlineto{\pgfqpoint{0.836281in}{2.010001in}}%
\pgfpathlineto{\pgfqpoint{0.840822in}{2.010001in}}%
\pgfpathlineto{\pgfqpoint{0.840822in}{2.007051in}}%
\pgfpathmoveto{\pgfqpoint{0.840822in}{2.007051in}}%
\pgfpathlineto{\pgfqpoint{0.840822in}{2.007051in}}%
\pgfpathlineto{\pgfqpoint{0.840822in}{2.010001in}}%
\pgfpathlineto{\pgfqpoint{0.845363in}{2.010001in}}%
\pgfpathlineto{\pgfqpoint{0.845363in}{2.007051in}}%
\pgfpathmoveto{\pgfqpoint{0.845363in}{2.007051in}}%
\pgfpathlineto{\pgfqpoint{0.845363in}{2.007051in}}%
\pgfpathlineto{\pgfqpoint{0.845363in}{2.010001in}}%
\pgfpathlineto{\pgfqpoint{0.849904in}{2.010001in}}%
\pgfpathlineto{\pgfqpoint{0.849904in}{2.007051in}}%
\pgfpathmoveto{\pgfqpoint{0.849904in}{2.007051in}}%
\pgfpathlineto{\pgfqpoint{0.849904in}{2.007051in}}%
\pgfpathlineto{\pgfqpoint{0.849904in}{2.010001in}}%
\pgfpathlineto{\pgfqpoint{0.854446in}{2.010001in}}%
\pgfpathlineto{\pgfqpoint{0.854446in}{2.007051in}}%
\pgfpathmoveto{\pgfqpoint{0.854446in}{2.007051in}}%
\pgfpathlineto{\pgfqpoint{0.854446in}{2.007051in}}%
\pgfpathlineto{\pgfqpoint{0.854446in}{2.010001in}}%
\pgfpathlineto{\pgfqpoint{0.858987in}{2.010001in}}%
\pgfpathlineto{\pgfqpoint{0.858987in}{2.007051in}}%
\pgfpathmoveto{\pgfqpoint{0.858987in}{2.007051in}}%
\pgfpathlineto{\pgfqpoint{0.858987in}{2.007051in}}%
\pgfpathlineto{\pgfqpoint{0.858987in}{2.010001in}}%
\pgfpathlineto{\pgfqpoint{0.863528in}{2.010001in}}%
\pgfpathlineto{\pgfqpoint{0.863528in}{2.007051in}}%
\pgfpathmoveto{\pgfqpoint{0.863528in}{2.007051in}}%
\pgfpathlineto{\pgfqpoint{0.863528in}{2.007051in}}%
\pgfpathlineto{\pgfqpoint{0.863528in}{2.010001in}}%
\pgfpathlineto{\pgfqpoint{0.868069in}{2.010001in}}%
\pgfpathlineto{\pgfqpoint{0.868069in}{2.007051in}}%
\pgfpathmoveto{\pgfqpoint{0.868069in}{2.007051in}}%
\pgfpathlineto{\pgfqpoint{0.868069in}{2.007051in}}%
\pgfpathlineto{\pgfqpoint{0.868069in}{2.010001in}}%
\pgfpathlineto{\pgfqpoint{0.872610in}{2.010001in}}%
\pgfpathlineto{\pgfqpoint{0.872610in}{2.007051in}}%
\pgfpathmoveto{\pgfqpoint{0.872610in}{2.007051in}}%
\pgfpathlineto{\pgfqpoint{0.872610in}{2.007051in}}%
\pgfpathlineto{\pgfqpoint{0.872610in}{2.010001in}}%
\pgfpathlineto{\pgfqpoint{0.877151in}{2.010001in}}%
\pgfpathlineto{\pgfqpoint{0.877151in}{2.007051in}}%
\pgfpathmoveto{\pgfqpoint{0.877151in}{2.007051in}}%
\pgfpathlineto{\pgfqpoint{0.877151in}{2.007051in}}%
\pgfpathlineto{\pgfqpoint{0.877151in}{2.010001in}}%
\pgfpathlineto{\pgfqpoint{0.881692in}{2.010001in}}%
\pgfpathlineto{\pgfqpoint{0.881692in}{2.007051in}}%
\pgfpathmoveto{\pgfqpoint{0.881692in}{2.007051in}}%
\pgfpathlineto{\pgfqpoint{0.881692in}{2.007051in}}%
\pgfpathlineto{\pgfqpoint{0.881692in}{2.010001in}}%
\pgfpathlineto{\pgfqpoint{0.886233in}{2.010001in}}%
\pgfpathlineto{\pgfqpoint{0.886233in}{2.007051in}}%
\pgfpathmoveto{\pgfqpoint{0.886233in}{2.007051in}}%
\pgfpathlineto{\pgfqpoint{0.886233in}{2.007051in}}%
\pgfpathlineto{\pgfqpoint{0.886233in}{2.010001in}}%
\pgfpathlineto{\pgfqpoint{0.890774in}{2.010001in}}%
\pgfpathlineto{\pgfqpoint{0.890774in}{2.007051in}}%
\pgfpathmoveto{\pgfqpoint{0.890774in}{2.007051in}}%
\pgfpathlineto{\pgfqpoint{0.890774in}{2.007051in}}%
\pgfpathlineto{\pgfqpoint{0.890774in}{2.010001in}}%
\pgfpathlineto{\pgfqpoint{0.895315in}{2.010001in}}%
\pgfpathlineto{\pgfqpoint{0.895315in}{2.007051in}}%
\pgfpathmoveto{\pgfqpoint{0.895315in}{2.007051in}}%
\pgfpathlineto{\pgfqpoint{0.895315in}{2.007051in}}%
\pgfpathlineto{\pgfqpoint{0.895315in}{2.010001in}}%
\pgfpathlineto{\pgfqpoint{0.899856in}{2.010001in}}%
\pgfpathlineto{\pgfqpoint{0.899856in}{2.007051in}}%
\pgfpathmoveto{\pgfqpoint{0.899856in}{2.007051in}}%
\pgfpathlineto{\pgfqpoint{0.899856in}{2.007051in}}%
\pgfpathlineto{\pgfqpoint{0.899856in}{2.010001in}}%
\pgfpathlineto{\pgfqpoint{0.904397in}{2.010001in}}%
\pgfpathlineto{\pgfqpoint{0.904397in}{2.007051in}}%
\pgfpathmoveto{\pgfqpoint{0.904397in}{2.007051in}}%
\pgfpathlineto{\pgfqpoint{0.904397in}{2.007051in}}%
\pgfpathlineto{\pgfqpoint{0.904397in}{2.010001in}}%
\pgfpathlineto{\pgfqpoint{0.908938in}{2.010001in}}%
\pgfpathlineto{\pgfqpoint{0.908938in}{2.007051in}}%
\pgfpathmoveto{\pgfqpoint{0.908938in}{2.007051in}}%
\pgfpathlineto{\pgfqpoint{0.908938in}{2.007051in}}%
\pgfpathlineto{\pgfqpoint{0.908938in}{2.010001in}}%
\pgfpathlineto{\pgfqpoint{0.913478in}{2.010001in}}%
\pgfpathlineto{\pgfqpoint{0.913478in}{2.007051in}}%
\pgfpathmoveto{\pgfqpoint{0.913478in}{2.007051in}}%
\pgfpathlineto{\pgfqpoint{0.913478in}{2.007051in}}%
\pgfpathlineto{\pgfqpoint{0.913478in}{2.010001in}}%
\pgfpathlineto{\pgfqpoint{0.918019in}{2.010001in}}%
\pgfpathlineto{\pgfqpoint{0.918019in}{2.007051in}}%
\pgfpathmoveto{\pgfqpoint{0.918019in}{2.007051in}}%
\pgfpathlineto{\pgfqpoint{0.918019in}{2.007051in}}%
\pgfpathlineto{\pgfqpoint{0.918019in}{2.010001in}}%
\pgfpathlineto{\pgfqpoint{0.922560in}{2.010001in}}%
\pgfpathlineto{\pgfqpoint{0.922560in}{2.007051in}}%
\pgfpathmoveto{\pgfqpoint{0.922560in}{2.007051in}}%
\pgfpathlineto{\pgfqpoint{0.922560in}{2.007051in}}%
\pgfpathlineto{\pgfqpoint{0.922560in}{2.010001in}}%
\pgfpathlineto{\pgfqpoint{0.927101in}{2.010001in}}%
\pgfpathlineto{\pgfqpoint{0.927101in}{2.007051in}}%
\pgfpathmoveto{\pgfqpoint{0.927101in}{2.007051in}}%
\pgfpathlineto{\pgfqpoint{0.927101in}{2.007051in}}%
\pgfpathlineto{\pgfqpoint{0.927101in}{2.010001in}}%
\pgfpathlineto{\pgfqpoint{0.931642in}{2.010001in}}%
\pgfpathlineto{\pgfqpoint{0.931642in}{2.007051in}}%
\pgfpathmoveto{\pgfqpoint{0.931642in}{2.007051in}}%
\pgfpathlineto{\pgfqpoint{0.931642in}{2.007051in}}%
\pgfpathlineto{\pgfqpoint{0.931642in}{2.010001in}}%
\pgfpathlineto{\pgfqpoint{0.936183in}{2.010001in}}%
\pgfpathlineto{\pgfqpoint{0.936183in}{2.007051in}}%
\pgfpathmoveto{\pgfqpoint{0.936183in}{2.007051in}}%
\pgfpathlineto{\pgfqpoint{0.936183in}{2.007051in}}%
\pgfpathlineto{\pgfqpoint{0.936183in}{2.010001in}}%
\pgfpathlineto{\pgfqpoint{0.940723in}{2.010001in}}%
\pgfpathlineto{\pgfqpoint{0.940723in}{2.007051in}}%
\pgfpathmoveto{\pgfqpoint{0.940723in}{2.007051in}}%
\pgfpathlineto{\pgfqpoint{0.940723in}{2.007051in}}%
\pgfpathlineto{\pgfqpoint{0.940723in}{2.010001in}}%
\pgfpathlineto{\pgfqpoint{0.945264in}{2.010001in}}%
\pgfpathlineto{\pgfqpoint{0.945264in}{2.007051in}}%
\pgfpathmoveto{\pgfqpoint{0.945264in}{2.007051in}}%
\pgfpathlineto{\pgfqpoint{0.945264in}{2.007051in}}%
\pgfpathlineto{\pgfqpoint{0.945264in}{2.010001in}}%
\pgfpathlineto{\pgfqpoint{0.949805in}{2.010001in}}%
\pgfpathlineto{\pgfqpoint{0.949805in}{2.007051in}}%
\pgfpathmoveto{\pgfqpoint{0.949805in}{2.007051in}}%
\pgfpathlineto{\pgfqpoint{0.949805in}{2.007051in}}%
\pgfpathlineto{\pgfqpoint{0.949805in}{2.010001in}}%
\pgfpathlineto{\pgfqpoint{0.954346in}{2.010001in}}%
\pgfpathlineto{\pgfqpoint{0.954346in}{2.007051in}}%
\pgfpathmoveto{\pgfqpoint{0.954346in}{2.007051in}}%
\pgfpathlineto{\pgfqpoint{0.954346in}{2.007051in}}%
\pgfpathlineto{\pgfqpoint{0.954346in}{2.010001in}}%
\pgfpathlineto{\pgfqpoint{0.958887in}{2.010001in}}%
\pgfpathlineto{\pgfqpoint{0.958887in}{2.007051in}}%
\pgfpathmoveto{\pgfqpoint{0.958887in}{2.007051in}}%
\pgfpathlineto{\pgfqpoint{0.958887in}{2.007051in}}%
\pgfpathlineto{\pgfqpoint{0.958887in}{2.010001in}}%
\pgfpathlineto{\pgfqpoint{0.963428in}{2.010001in}}%
\pgfpathlineto{\pgfqpoint{0.963428in}{2.007051in}}%
\pgfpathmoveto{\pgfqpoint{0.963428in}{2.007051in}}%
\pgfpathlineto{\pgfqpoint{0.963428in}{2.007051in}}%
\pgfpathlineto{\pgfqpoint{0.963428in}{2.010001in}}%
\pgfpathlineto{\pgfqpoint{0.967968in}{2.010001in}}%
\pgfpathlineto{\pgfqpoint{0.967968in}{2.007051in}}%
\pgfpathmoveto{\pgfqpoint{0.967968in}{2.007051in}}%
\pgfpathlineto{\pgfqpoint{0.967968in}{2.007051in}}%
\pgfpathlineto{\pgfqpoint{0.967968in}{2.010001in}}%
\pgfpathlineto{\pgfqpoint{0.972509in}{2.010001in}}%
\pgfpathlineto{\pgfqpoint{0.972509in}{2.007051in}}%
\pgfpathmoveto{\pgfqpoint{0.972509in}{2.007051in}}%
\pgfpathlineto{\pgfqpoint{0.972509in}{2.007051in}}%
\pgfpathlineto{\pgfqpoint{0.972509in}{2.010001in}}%
\pgfpathlineto{\pgfqpoint{0.977050in}{2.010001in}}%
\pgfpathlineto{\pgfqpoint{0.977050in}{2.007051in}}%
\pgfpathmoveto{\pgfqpoint{0.977050in}{2.007051in}}%
\pgfpathlineto{\pgfqpoint{0.977050in}{2.007051in}}%
\pgfpathlineto{\pgfqpoint{0.977050in}{2.010001in}}%
\pgfpathlineto{\pgfqpoint{0.981591in}{2.010001in}}%
\pgfpathlineto{\pgfqpoint{0.981591in}{2.007051in}}%
\pgfpathmoveto{\pgfqpoint{0.981591in}{2.007051in}}%
\pgfpathlineto{\pgfqpoint{0.981591in}{2.007051in}}%
\pgfpathlineto{\pgfqpoint{0.981591in}{2.010001in}}%
\pgfpathlineto{\pgfqpoint{0.986132in}{2.010001in}}%
\pgfpathlineto{\pgfqpoint{0.986132in}{2.007051in}}%
\pgfpathmoveto{\pgfqpoint{0.986132in}{2.007051in}}%
\pgfpathlineto{\pgfqpoint{0.986132in}{2.007051in}}%
\pgfpathlineto{\pgfqpoint{0.986132in}{2.010001in}}%
\pgfpathlineto{\pgfqpoint{0.990672in}{2.010001in}}%
\pgfpathlineto{\pgfqpoint{0.990672in}{2.007051in}}%
\pgfpathmoveto{\pgfqpoint{0.990672in}{2.007051in}}%
\pgfpathlineto{\pgfqpoint{0.990672in}{2.007051in}}%
\pgfpathlineto{\pgfqpoint{0.990672in}{2.010001in}}%
\pgfpathlineto{\pgfqpoint{0.995213in}{2.010001in}}%
\pgfpathlineto{\pgfqpoint{0.995213in}{2.007051in}}%
\pgfpathmoveto{\pgfqpoint{0.995213in}{2.007051in}}%
\pgfpathlineto{\pgfqpoint{0.995213in}{2.007051in}}%
\pgfpathlineto{\pgfqpoint{0.995213in}{2.010001in}}%
\pgfpathlineto{\pgfqpoint{0.999754in}{2.010001in}}%
\pgfpathlineto{\pgfqpoint{0.999754in}{2.007051in}}%
\pgfpathmoveto{\pgfqpoint{0.999754in}{2.007051in}}%
\pgfpathlineto{\pgfqpoint{0.999754in}{2.007051in}}%
\pgfpathlineto{\pgfqpoint{0.999754in}{2.010001in}}%
\pgfpathlineto{\pgfqpoint{1.004295in}{2.010001in}}%
\pgfpathlineto{\pgfqpoint{1.004295in}{2.007051in}}%
\pgfpathmoveto{\pgfqpoint{1.004295in}{2.007051in}}%
\pgfpathlineto{\pgfqpoint{1.004295in}{2.007051in}}%
\pgfpathlineto{\pgfqpoint{1.004295in}{2.010001in}}%
\pgfpathlineto{\pgfqpoint{1.008836in}{2.010001in}}%
\pgfpathlineto{\pgfqpoint{1.008836in}{2.007051in}}%
\pgfpathmoveto{\pgfqpoint{1.008836in}{2.007051in}}%
\pgfpathlineto{\pgfqpoint{1.008836in}{2.007051in}}%
\pgfpathlineto{\pgfqpoint{1.008836in}{2.010001in}}%
\pgfpathlineto{\pgfqpoint{1.013377in}{2.010001in}}%
\pgfpathlineto{\pgfqpoint{1.013377in}{2.007051in}}%
\pgfpathmoveto{\pgfqpoint{1.013377in}{2.007051in}}%
\pgfpathlineto{\pgfqpoint{1.013377in}{2.007051in}}%
\pgfpathlineto{\pgfqpoint{1.013377in}{2.010001in}}%
\pgfpathlineto{\pgfqpoint{1.017917in}{2.010001in}}%
\pgfpathlineto{\pgfqpoint{1.017917in}{2.007051in}}%
\pgfpathmoveto{\pgfqpoint{1.017917in}{2.007051in}}%
\pgfpathlineto{\pgfqpoint{1.017917in}{2.007051in}}%
\pgfpathlineto{\pgfqpoint{1.017917in}{2.010001in}}%
\pgfpathlineto{\pgfqpoint{1.022458in}{2.010001in}}%
\pgfpathlineto{\pgfqpoint{1.022458in}{2.007051in}}%
\pgfpathmoveto{\pgfqpoint{1.022458in}{2.007051in}}%
\pgfpathlineto{\pgfqpoint{1.022458in}{2.007051in}}%
\pgfpathlineto{\pgfqpoint{1.022458in}{2.010001in}}%
\pgfpathlineto{\pgfqpoint{1.026999in}{2.010001in}}%
\pgfpathlineto{\pgfqpoint{1.026999in}{2.007051in}}%
\pgfpathmoveto{\pgfqpoint{1.026999in}{2.007051in}}%
\pgfpathlineto{\pgfqpoint{1.026999in}{2.007051in}}%
\pgfpathlineto{\pgfqpoint{1.026999in}{2.010001in}}%
\pgfpathlineto{\pgfqpoint{1.031540in}{2.010001in}}%
\pgfpathlineto{\pgfqpoint{1.031540in}{2.007051in}}%
\pgfpathmoveto{\pgfqpoint{1.031540in}{2.007051in}}%
\pgfpathlineto{\pgfqpoint{1.031540in}{2.007051in}}%
\pgfpathlineto{\pgfqpoint{1.031540in}{2.010001in}}%
\pgfpathlineto{\pgfqpoint{1.036081in}{2.010001in}}%
\pgfpathlineto{\pgfqpoint{1.036081in}{2.007051in}}%
\pgfpathmoveto{\pgfqpoint{1.036081in}{2.007051in}}%
\pgfpathlineto{\pgfqpoint{1.036081in}{2.007051in}}%
\pgfpathlineto{\pgfqpoint{1.036081in}{2.010001in}}%
\pgfpathlineto{\pgfqpoint{1.040621in}{2.010001in}}%
\pgfpathlineto{\pgfqpoint{1.040621in}{2.007051in}}%
\pgfpathmoveto{\pgfqpoint{1.040621in}{2.007051in}}%
\pgfpathlineto{\pgfqpoint{1.040621in}{2.007051in}}%
\pgfpathlineto{\pgfqpoint{1.040621in}{2.010001in}}%
\pgfpathlineto{\pgfqpoint{1.045163in}{2.010001in}}%
\pgfpathlineto{\pgfqpoint{1.045163in}{2.007051in}}%
\pgfpathmoveto{\pgfqpoint{1.045163in}{2.007051in}}%
\pgfpathlineto{\pgfqpoint{1.045163in}{2.007051in}}%
\pgfpathlineto{\pgfqpoint{1.045163in}{2.010001in}}%
\pgfpathlineto{\pgfqpoint{1.049704in}{2.010001in}}%
\pgfpathlineto{\pgfqpoint{1.049704in}{2.007051in}}%
\pgfpathmoveto{\pgfqpoint{1.049704in}{2.007051in}}%
\pgfpathlineto{\pgfqpoint{1.049704in}{2.007051in}}%
\pgfpathlineto{\pgfqpoint{1.049704in}{2.010001in}}%
\pgfpathlineto{\pgfqpoint{1.054245in}{2.010001in}}%
\pgfpathlineto{\pgfqpoint{1.054245in}{2.007051in}}%
\pgfpathmoveto{\pgfqpoint{1.054245in}{2.007051in}}%
\pgfpathlineto{\pgfqpoint{1.054245in}{2.007051in}}%
\pgfpathlineto{\pgfqpoint{1.054245in}{2.010001in}}%
\pgfpathlineto{\pgfqpoint{1.058786in}{2.010001in}}%
\pgfpathlineto{\pgfqpoint{1.058786in}{2.007051in}}%
\pgfpathmoveto{\pgfqpoint{1.058786in}{2.007051in}}%
\pgfpathlineto{\pgfqpoint{1.058786in}{2.007051in}}%
\pgfpathlineto{\pgfqpoint{1.058786in}{2.010001in}}%
\pgfpathlineto{\pgfqpoint{1.063327in}{2.010001in}}%
\pgfpathlineto{\pgfqpoint{1.063327in}{2.007051in}}%
\pgfpathmoveto{\pgfqpoint{1.063327in}{2.007051in}}%
\pgfpathlineto{\pgfqpoint{1.063327in}{2.007051in}}%
\pgfpathlineto{\pgfqpoint{1.063327in}{2.010001in}}%
\pgfpathlineto{\pgfqpoint{1.067869in}{2.010001in}}%
\pgfpathlineto{\pgfqpoint{1.067869in}{2.007051in}}%
\pgfpathmoveto{\pgfqpoint{1.067869in}{2.007051in}}%
\pgfpathlineto{\pgfqpoint{1.067869in}{2.007051in}}%
\pgfpathlineto{\pgfqpoint{1.067869in}{2.010001in}}%
\pgfpathlineto{\pgfqpoint{1.072410in}{2.010001in}}%
\pgfpathlineto{\pgfqpoint{1.072410in}{2.007051in}}%
\pgfpathmoveto{\pgfqpoint{1.072410in}{2.007051in}}%
\pgfpathlineto{\pgfqpoint{1.072410in}{2.007051in}}%
\pgfpathlineto{\pgfqpoint{1.072410in}{2.010001in}}%
\pgfpathlineto{\pgfqpoint{1.076951in}{2.010001in}}%
\pgfpathlineto{\pgfqpoint{1.076951in}{2.007051in}}%
\pgfpathmoveto{\pgfqpoint{1.076951in}{2.007051in}}%
\pgfpathlineto{\pgfqpoint{1.076951in}{2.007051in}}%
\pgfpathlineto{\pgfqpoint{1.076951in}{2.010001in}}%
\pgfpathlineto{\pgfqpoint{1.081492in}{2.010001in}}%
\pgfpathlineto{\pgfqpoint{1.081492in}{2.007051in}}%
\pgfpathmoveto{\pgfqpoint{1.081492in}{2.007051in}}%
\pgfpathlineto{\pgfqpoint{1.081492in}{2.007051in}}%
\pgfpathlineto{\pgfqpoint{1.081492in}{2.010001in}}%
\pgfpathlineto{\pgfqpoint{1.086033in}{2.010001in}}%
\pgfpathlineto{\pgfqpoint{1.086033in}{2.007051in}}%
\pgfpathmoveto{\pgfqpoint{1.086033in}{2.007051in}}%
\pgfpathlineto{\pgfqpoint{1.086033in}{2.007051in}}%
\pgfpathlineto{\pgfqpoint{1.086033in}{2.010001in}}%
\pgfpathlineto{\pgfqpoint{1.090574in}{2.010001in}}%
\pgfpathlineto{\pgfqpoint{1.090574in}{2.007051in}}%
\pgfpathmoveto{\pgfqpoint{1.090574in}{2.007051in}}%
\pgfpathlineto{\pgfqpoint{1.090574in}{2.007051in}}%
\pgfpathlineto{\pgfqpoint{1.090574in}{2.010001in}}%
\pgfpathlineto{\pgfqpoint{1.095116in}{2.010001in}}%
\pgfpathlineto{\pgfqpoint{1.095116in}{2.007051in}}%
\pgfpathmoveto{\pgfqpoint{1.095116in}{2.007051in}}%
\pgfpathlineto{\pgfqpoint{1.095116in}{2.007051in}}%
\pgfpathlineto{\pgfqpoint{1.095116in}{2.010001in}}%
\pgfpathlineto{\pgfqpoint{1.099657in}{2.010001in}}%
\pgfpathlineto{\pgfqpoint{1.099657in}{2.007051in}}%
\pgfpathmoveto{\pgfqpoint{1.099657in}{2.007051in}}%
\pgfpathlineto{\pgfqpoint{1.099657in}{2.007051in}}%
\pgfpathlineto{\pgfqpoint{1.099657in}{2.010001in}}%
\pgfpathlineto{\pgfqpoint{1.104198in}{2.010001in}}%
\pgfpathlineto{\pgfqpoint{1.104198in}{2.007051in}}%
\pgfpathmoveto{\pgfqpoint{1.104198in}{2.007051in}}%
\pgfpathlineto{\pgfqpoint{1.104198in}{2.007051in}}%
\pgfpathlineto{\pgfqpoint{1.104198in}{2.010001in}}%
\pgfpathlineto{\pgfqpoint{1.108739in}{2.010001in}}%
\pgfpathlineto{\pgfqpoint{1.108739in}{2.007051in}}%
\pgfpathmoveto{\pgfqpoint{1.108739in}{2.007051in}}%
\pgfpathlineto{\pgfqpoint{1.108739in}{2.007051in}}%
\pgfpathlineto{\pgfqpoint{1.108739in}{2.010001in}}%
\pgfpathlineto{\pgfqpoint{1.113280in}{2.010001in}}%
\pgfpathlineto{\pgfqpoint{1.113280in}{2.007051in}}%
\pgfpathmoveto{\pgfqpoint{1.113280in}{2.007051in}}%
\pgfpathlineto{\pgfqpoint{1.113280in}{2.007051in}}%
\pgfpathlineto{\pgfqpoint{1.113280in}{2.010001in}}%
\pgfpathlineto{\pgfqpoint{1.117822in}{2.010001in}}%
\pgfpathlineto{\pgfqpoint{1.117822in}{2.007051in}}%
\pgfpathmoveto{\pgfqpoint{1.117822in}{2.007051in}}%
\pgfpathlineto{\pgfqpoint{1.117822in}{2.007051in}}%
\pgfpathlineto{\pgfqpoint{1.117822in}{2.010001in}}%
\pgfpathlineto{\pgfqpoint{1.122363in}{2.010001in}}%
\pgfpathlineto{\pgfqpoint{1.122363in}{2.007051in}}%
\pgfpathmoveto{\pgfqpoint{1.122363in}{2.007051in}}%
\pgfpathlineto{\pgfqpoint{1.122363in}{2.007051in}}%
\pgfpathlineto{\pgfqpoint{1.122363in}{2.010001in}}%
\pgfpathlineto{\pgfqpoint{1.126904in}{2.010001in}}%
\pgfpathlineto{\pgfqpoint{1.126904in}{2.007051in}}%
\pgfpathmoveto{\pgfqpoint{1.126904in}{2.007051in}}%
\pgfpathlineto{\pgfqpoint{1.126904in}{2.007051in}}%
\pgfpathlineto{\pgfqpoint{1.126904in}{2.010001in}}%
\pgfpathlineto{\pgfqpoint{1.131445in}{2.010001in}}%
\pgfpathlineto{\pgfqpoint{1.131445in}{2.007051in}}%
\pgfpathmoveto{\pgfqpoint{1.131445in}{2.007051in}}%
\pgfpathlineto{\pgfqpoint{1.131445in}{2.007051in}}%
\pgfpathlineto{\pgfqpoint{1.131445in}{2.010001in}}%
\pgfpathlineto{\pgfqpoint{1.135986in}{2.010001in}}%
\pgfpathlineto{\pgfqpoint{1.135986in}{2.007051in}}%
\pgfpathmoveto{\pgfqpoint{1.135986in}{2.007051in}}%
\pgfpathlineto{\pgfqpoint{1.135986in}{2.007051in}}%
\pgfpathlineto{\pgfqpoint{1.135986in}{2.010001in}}%
\pgfpathlineto{\pgfqpoint{1.140527in}{2.010001in}}%
\pgfpathlineto{\pgfqpoint{1.140527in}{2.007051in}}%
\pgfpathmoveto{\pgfqpoint{1.140527in}{2.007051in}}%
\pgfpathlineto{\pgfqpoint{1.140527in}{2.007051in}}%
\pgfpathlineto{\pgfqpoint{1.140527in}{2.010001in}}%
\pgfpathlineto{\pgfqpoint{1.145069in}{2.010001in}}%
\pgfpathlineto{\pgfqpoint{1.145069in}{2.007051in}}%
\pgfpathmoveto{\pgfqpoint{1.145069in}{2.007051in}}%
\pgfpathlineto{\pgfqpoint{1.145069in}{2.007051in}}%
\pgfpathlineto{\pgfqpoint{1.145069in}{2.010001in}}%
\pgfpathlineto{\pgfqpoint{1.149610in}{2.010001in}}%
\pgfpathlineto{\pgfqpoint{1.149610in}{2.007051in}}%
\pgfpathmoveto{\pgfqpoint{1.149610in}{2.007051in}}%
\pgfpathlineto{\pgfqpoint{1.149610in}{2.007051in}}%
\pgfpathlineto{\pgfqpoint{1.149610in}{2.010001in}}%
\pgfpathlineto{\pgfqpoint{1.154151in}{2.010001in}}%
\pgfpathlineto{\pgfqpoint{1.154151in}{2.007051in}}%
\pgfpathmoveto{\pgfqpoint{1.154151in}{2.007051in}}%
\pgfpathlineto{\pgfqpoint{1.154151in}{2.007051in}}%
\pgfpathlineto{\pgfqpoint{1.154151in}{2.010001in}}%
\pgfpathlineto{\pgfqpoint{1.158692in}{2.010001in}}%
\pgfpathlineto{\pgfqpoint{1.158692in}{2.007051in}}%
\pgfpathmoveto{\pgfqpoint{1.158692in}{2.007051in}}%
\pgfpathlineto{\pgfqpoint{1.158692in}{2.007051in}}%
\pgfpathlineto{\pgfqpoint{1.158692in}{2.010001in}}%
\pgfpathlineto{\pgfqpoint{1.163233in}{2.010001in}}%
\pgfpathlineto{\pgfqpoint{1.163233in}{2.007051in}}%
\pgfpathmoveto{\pgfqpoint{1.163233in}{2.007051in}}%
\pgfpathlineto{\pgfqpoint{1.163233in}{2.007051in}}%
\pgfpathlineto{\pgfqpoint{1.163233in}{2.010001in}}%
\pgfpathlineto{\pgfqpoint{1.167774in}{2.010001in}}%
\pgfpathlineto{\pgfqpoint{1.167774in}{2.007051in}}%
\pgfpathmoveto{\pgfqpoint{1.167774in}{2.007051in}}%
\pgfpathlineto{\pgfqpoint{1.167774in}{2.007051in}}%
\pgfpathlineto{\pgfqpoint{1.167774in}{2.010001in}}%
\pgfpathlineto{\pgfqpoint{1.172316in}{2.010001in}}%
\pgfpathlineto{\pgfqpoint{1.172316in}{2.007051in}}%
\pgfpathmoveto{\pgfqpoint{1.172316in}{2.007051in}}%
\pgfpathlineto{\pgfqpoint{1.172316in}{2.007051in}}%
\pgfpathlineto{\pgfqpoint{1.172316in}{2.010001in}}%
\pgfpathlineto{\pgfqpoint{1.176857in}{2.010001in}}%
\pgfpathlineto{\pgfqpoint{1.176857in}{2.007051in}}%
\pgfpathmoveto{\pgfqpoint{1.176857in}{2.007051in}}%
\pgfpathlineto{\pgfqpoint{1.176857in}{2.007051in}}%
\pgfpathlineto{\pgfqpoint{1.176857in}{2.010001in}}%
\pgfpathlineto{\pgfqpoint{1.181398in}{2.010001in}}%
\pgfpathlineto{\pgfqpoint{1.181398in}{2.007051in}}%
\pgfpathmoveto{\pgfqpoint{1.181398in}{2.007051in}}%
\pgfpathlineto{\pgfqpoint{1.181398in}{2.007051in}}%
\pgfpathlineto{\pgfqpoint{1.181398in}{2.010001in}}%
\pgfpathlineto{\pgfqpoint{1.185939in}{2.010001in}}%
\pgfpathlineto{\pgfqpoint{1.185939in}{2.007051in}}%
\pgfpathmoveto{\pgfqpoint{1.185939in}{2.007051in}}%
\pgfpathlineto{\pgfqpoint{1.185939in}{2.007051in}}%
\pgfpathlineto{\pgfqpoint{1.185939in}{2.010001in}}%
\pgfpathlineto{\pgfqpoint{1.190480in}{2.010001in}}%
\pgfpathlineto{\pgfqpoint{1.190480in}{2.007051in}}%
\pgfpathmoveto{\pgfqpoint{1.190480in}{2.007051in}}%
\pgfpathlineto{\pgfqpoint{1.190480in}{2.007051in}}%
\pgfpathlineto{\pgfqpoint{1.190480in}{2.010001in}}%
\pgfpathlineto{\pgfqpoint{1.195021in}{2.010001in}}%
\pgfpathlineto{\pgfqpoint{1.195021in}{2.007051in}}%
\pgfpathmoveto{\pgfqpoint{1.195021in}{2.007051in}}%
\pgfpathlineto{\pgfqpoint{1.195021in}{2.007051in}}%
\pgfpathlineto{\pgfqpoint{1.195021in}{2.010001in}}%
\pgfpathlineto{\pgfqpoint{1.199562in}{2.010001in}}%
\pgfpathlineto{\pgfqpoint{1.199562in}{2.007051in}}%
\pgfpathmoveto{\pgfqpoint{1.199562in}{2.007051in}}%
\pgfpathlineto{\pgfqpoint{1.199562in}{2.007051in}}%
\pgfpathlineto{\pgfqpoint{1.199562in}{2.010001in}}%
\pgfpathlineto{\pgfqpoint{1.204103in}{2.010001in}}%
\pgfpathlineto{\pgfqpoint{1.204103in}{2.007051in}}%
\pgfpathmoveto{\pgfqpoint{1.204103in}{2.007051in}}%
\pgfpathlineto{\pgfqpoint{1.204103in}{2.007051in}}%
\pgfpathlineto{\pgfqpoint{1.204103in}{2.010001in}}%
\pgfpathlineto{\pgfqpoint{1.208644in}{2.010001in}}%
\pgfpathlineto{\pgfqpoint{1.208644in}{2.007051in}}%
\pgfpathmoveto{\pgfqpoint{1.208644in}{2.007051in}}%
\pgfpathlineto{\pgfqpoint{1.208644in}{2.007051in}}%
\pgfpathlineto{\pgfqpoint{1.208644in}{2.010001in}}%
\pgfpathlineto{\pgfqpoint{1.213186in}{2.010001in}}%
\pgfpathlineto{\pgfqpoint{1.213186in}{2.007051in}}%
\pgfpathmoveto{\pgfqpoint{1.213186in}{2.007051in}}%
\pgfpathlineto{\pgfqpoint{1.213186in}{2.007051in}}%
\pgfpathlineto{\pgfqpoint{1.213186in}{2.010001in}}%
\pgfpathlineto{\pgfqpoint{1.217727in}{2.010001in}}%
\pgfpathlineto{\pgfqpoint{1.217727in}{2.007051in}}%
\pgfpathmoveto{\pgfqpoint{1.217727in}{2.007051in}}%
\pgfpathlineto{\pgfqpoint{1.217727in}{2.007051in}}%
\pgfpathlineto{\pgfqpoint{1.217727in}{2.010001in}}%
\pgfpathlineto{\pgfqpoint{1.222268in}{2.010001in}}%
\pgfpathlineto{\pgfqpoint{1.222268in}{2.007051in}}%
\pgfpathmoveto{\pgfqpoint{1.222268in}{2.007051in}}%
\pgfpathlineto{\pgfqpoint{1.222268in}{2.007051in}}%
\pgfpathlineto{\pgfqpoint{1.222268in}{2.010001in}}%
\pgfpathlineto{\pgfqpoint{1.226809in}{2.010001in}}%
\pgfpathlineto{\pgfqpoint{1.226809in}{2.007051in}}%
\pgfpathmoveto{\pgfqpoint{1.226809in}{2.007051in}}%
\pgfpathlineto{\pgfqpoint{1.226809in}{2.007051in}}%
\pgfpathlineto{\pgfqpoint{1.226809in}{2.010001in}}%
\pgfpathlineto{\pgfqpoint{1.231350in}{2.010001in}}%
\pgfpathlineto{\pgfqpoint{1.231350in}{2.007051in}}%
\pgfpathmoveto{\pgfqpoint{1.231350in}{2.007051in}}%
\pgfpathlineto{\pgfqpoint{1.231350in}{2.007051in}}%
\pgfpathlineto{\pgfqpoint{1.231350in}{2.010001in}}%
\pgfpathlineto{\pgfqpoint{1.235891in}{2.010001in}}%
\pgfpathlineto{\pgfqpoint{1.235891in}{2.007051in}}%
\pgfpathmoveto{\pgfqpoint{1.235891in}{2.007051in}}%
\pgfpathlineto{\pgfqpoint{1.235891in}{2.007051in}}%
\pgfpathlineto{\pgfqpoint{1.235891in}{2.010001in}}%
\pgfpathlineto{\pgfqpoint{1.240432in}{2.010001in}}%
\pgfpathlineto{\pgfqpoint{1.240432in}{2.007051in}}%
\pgfpathmoveto{\pgfqpoint{1.240432in}{2.007051in}}%
\pgfpathlineto{\pgfqpoint{1.240432in}{2.007051in}}%
\pgfpathlineto{\pgfqpoint{1.240432in}{2.010001in}}%
\pgfpathlineto{\pgfqpoint{1.244973in}{2.010001in}}%
\pgfpathlineto{\pgfqpoint{1.244973in}{2.007051in}}%
\pgfpathmoveto{\pgfqpoint{1.244973in}{2.007051in}}%
\pgfpathlineto{\pgfqpoint{1.244973in}{2.007051in}}%
\pgfpathlineto{\pgfqpoint{1.244973in}{2.010001in}}%
\pgfpathlineto{\pgfqpoint{1.249514in}{2.010001in}}%
\pgfpathlineto{\pgfqpoint{1.249514in}{2.007051in}}%
\pgfpathmoveto{\pgfqpoint{1.249514in}{2.007051in}}%
\pgfpathlineto{\pgfqpoint{1.249514in}{2.007051in}}%
\pgfpathlineto{\pgfqpoint{1.249514in}{2.010001in}}%
\pgfpathlineto{\pgfqpoint{1.254055in}{2.010001in}}%
\pgfpathlineto{\pgfqpoint{1.254055in}{2.007051in}}%
\pgfpathmoveto{\pgfqpoint{1.254055in}{2.007051in}}%
\pgfpathlineto{\pgfqpoint{1.254055in}{2.007051in}}%
\pgfpathlineto{\pgfqpoint{1.254055in}{2.010001in}}%
\pgfpathlineto{\pgfqpoint{1.258596in}{2.010001in}}%
\pgfpathlineto{\pgfqpoint{1.258596in}{2.007051in}}%
\pgfpathmoveto{\pgfqpoint{1.258596in}{2.007051in}}%
\pgfpathlineto{\pgfqpoint{1.258596in}{2.007051in}}%
\pgfpathlineto{\pgfqpoint{1.258596in}{2.010001in}}%
\pgfpathlineto{\pgfqpoint{1.263137in}{2.010001in}}%
\pgfpathlineto{\pgfqpoint{1.263137in}{2.007051in}}%
\pgfpathmoveto{\pgfqpoint{1.263137in}{2.007051in}}%
\pgfpathlineto{\pgfqpoint{1.263137in}{2.007051in}}%
\pgfpathlineto{\pgfqpoint{1.263137in}{2.010001in}}%
\pgfpathlineto{\pgfqpoint{1.267678in}{2.010001in}}%
\pgfpathlineto{\pgfqpoint{1.267678in}{2.007051in}}%
\pgfpathmoveto{\pgfqpoint{1.267678in}{2.007051in}}%
\pgfpathlineto{\pgfqpoint{1.267678in}{2.007051in}}%
\pgfpathlineto{\pgfqpoint{1.267678in}{2.010001in}}%
\pgfpathlineto{\pgfqpoint{1.272219in}{2.010001in}}%
\pgfpathlineto{\pgfqpoint{1.272219in}{2.007051in}}%
\pgfpathmoveto{\pgfqpoint{1.272219in}{2.007051in}}%
\pgfpathlineto{\pgfqpoint{1.272219in}{2.007051in}}%
\pgfpathlineto{\pgfqpoint{1.272219in}{2.010001in}}%
\pgfpathlineto{\pgfqpoint{1.276760in}{2.010001in}}%
\pgfpathlineto{\pgfqpoint{1.276760in}{2.007051in}}%
\pgfpathmoveto{\pgfqpoint{1.276760in}{2.007051in}}%
\pgfpathlineto{\pgfqpoint{1.276760in}{2.007051in}}%
\pgfpathlineto{\pgfqpoint{1.276760in}{2.010001in}}%
\pgfpathlineto{\pgfqpoint{1.281301in}{2.010001in}}%
\pgfpathlineto{\pgfqpoint{1.281301in}{2.007051in}}%
\pgfpathmoveto{\pgfqpoint{1.281301in}{2.007051in}}%
\pgfpathlineto{\pgfqpoint{1.281301in}{2.007051in}}%
\pgfpathlineto{\pgfqpoint{1.281301in}{2.010001in}}%
\pgfpathlineto{\pgfqpoint{1.285842in}{2.010001in}}%
\pgfpathlineto{\pgfqpoint{1.285842in}{2.007051in}}%
\pgfpathmoveto{\pgfqpoint{1.285842in}{2.007051in}}%
\pgfpathlineto{\pgfqpoint{1.285842in}{2.007051in}}%
\pgfpathlineto{\pgfqpoint{1.285842in}{2.010001in}}%
\pgfpathlineto{\pgfqpoint{1.290383in}{2.010001in}}%
\pgfpathlineto{\pgfqpoint{1.290383in}{2.007051in}}%
\pgfpathmoveto{\pgfqpoint{1.290383in}{2.007051in}}%
\pgfpathlineto{\pgfqpoint{1.290383in}{2.007051in}}%
\pgfpathlineto{\pgfqpoint{1.290383in}{2.010001in}}%
\pgfpathlineto{\pgfqpoint{1.294925in}{2.010001in}}%
\pgfpathlineto{\pgfqpoint{1.294925in}{2.007051in}}%
\pgfpathmoveto{\pgfqpoint{1.294925in}{2.007051in}}%
\pgfpathlineto{\pgfqpoint{1.294925in}{2.007051in}}%
\pgfpathlineto{\pgfqpoint{1.294925in}{2.010001in}}%
\pgfpathlineto{\pgfqpoint{1.299466in}{2.010001in}}%
\pgfpathlineto{\pgfqpoint{1.299466in}{2.007051in}}%
\pgfpathmoveto{\pgfqpoint{1.299466in}{2.007051in}}%
\pgfpathlineto{\pgfqpoint{1.299466in}{2.007051in}}%
\pgfpathlineto{\pgfqpoint{1.299466in}{2.010001in}}%
\pgfpathlineto{\pgfqpoint{1.304007in}{2.010001in}}%
\pgfpathlineto{\pgfqpoint{1.304007in}{2.007051in}}%
\pgfpathmoveto{\pgfqpoint{1.304007in}{2.007051in}}%
\pgfpathlineto{\pgfqpoint{1.304007in}{2.007051in}}%
\pgfpathlineto{\pgfqpoint{1.304007in}{2.010001in}}%
\pgfpathlineto{\pgfqpoint{1.308548in}{2.010001in}}%
\pgfpathlineto{\pgfqpoint{1.308548in}{2.007051in}}%
\pgfpathmoveto{\pgfqpoint{1.308548in}{2.007051in}}%
\pgfpathlineto{\pgfqpoint{1.308548in}{2.007051in}}%
\pgfpathlineto{\pgfqpoint{1.308548in}{2.010001in}}%
\pgfpathlineto{\pgfqpoint{1.313089in}{2.010001in}}%
\pgfpathlineto{\pgfqpoint{1.313089in}{2.007051in}}%
\pgfpathmoveto{\pgfqpoint{1.313089in}{2.007051in}}%
\pgfpathlineto{\pgfqpoint{1.313089in}{2.007051in}}%
\pgfpathlineto{\pgfqpoint{1.313089in}{2.010001in}}%
\pgfpathlineto{\pgfqpoint{1.317630in}{2.010001in}}%
\pgfpathlineto{\pgfqpoint{1.317630in}{2.007051in}}%
\pgfpathmoveto{\pgfqpoint{1.317630in}{2.007051in}}%
\pgfpathlineto{\pgfqpoint{1.317630in}{2.007051in}}%
\pgfpathlineto{\pgfqpoint{1.317630in}{2.010001in}}%
\pgfpathlineto{\pgfqpoint{1.322171in}{2.010001in}}%
\pgfpathlineto{\pgfqpoint{1.322171in}{2.007051in}}%
\pgfpathmoveto{\pgfqpoint{1.322171in}{2.007051in}}%
\pgfpathlineto{\pgfqpoint{1.322171in}{2.007051in}}%
\pgfpathlineto{\pgfqpoint{1.322171in}{2.010001in}}%
\pgfpathlineto{\pgfqpoint{1.326712in}{2.010001in}}%
\pgfpathlineto{\pgfqpoint{1.326712in}{2.007051in}}%
\pgfpathmoveto{\pgfqpoint{1.326712in}{2.007051in}}%
\pgfpathlineto{\pgfqpoint{1.326712in}{2.007051in}}%
\pgfpathlineto{\pgfqpoint{1.326712in}{2.010001in}}%
\pgfpathlineto{\pgfqpoint{1.331253in}{2.010001in}}%
\pgfpathlineto{\pgfqpoint{1.331253in}{2.007051in}}%
\pgfpathmoveto{\pgfqpoint{1.331253in}{2.007051in}}%
\pgfpathlineto{\pgfqpoint{1.331253in}{2.007051in}}%
\pgfpathlineto{\pgfqpoint{1.331253in}{2.010001in}}%
\pgfpathlineto{\pgfqpoint{1.335794in}{2.010001in}}%
\pgfpathlineto{\pgfqpoint{1.335794in}{2.007051in}}%
\pgfpathmoveto{\pgfqpoint{1.335794in}{2.007051in}}%
\pgfpathlineto{\pgfqpoint{1.335794in}{2.007051in}}%
\pgfpathlineto{\pgfqpoint{1.335794in}{2.010001in}}%
\pgfpathlineto{\pgfqpoint{1.340335in}{2.010001in}}%
\pgfpathlineto{\pgfqpoint{1.340335in}{2.007051in}}%
\pgfpathmoveto{\pgfqpoint{1.340335in}{2.007051in}}%
\pgfpathlineto{\pgfqpoint{1.340335in}{2.007051in}}%
\pgfpathlineto{\pgfqpoint{1.340335in}{2.010001in}}%
\pgfpathlineto{\pgfqpoint{1.344876in}{2.010001in}}%
\pgfpathlineto{\pgfqpoint{1.344876in}{2.007051in}}%
\pgfpathmoveto{\pgfqpoint{1.344876in}{2.007051in}}%
\pgfpathlineto{\pgfqpoint{1.344876in}{2.007051in}}%
\pgfpathlineto{\pgfqpoint{1.344876in}{2.010001in}}%
\pgfpathlineto{\pgfqpoint{1.349417in}{2.010001in}}%
\pgfpathlineto{\pgfqpoint{1.349417in}{2.007051in}}%
\pgfpathmoveto{\pgfqpoint{1.349417in}{2.007051in}}%
\pgfpathlineto{\pgfqpoint{1.349417in}{2.007051in}}%
\pgfpathlineto{\pgfqpoint{1.349417in}{2.010001in}}%
\pgfpathlineto{\pgfqpoint{1.353958in}{2.010001in}}%
\pgfpathlineto{\pgfqpoint{1.353958in}{2.007051in}}%
\pgfpathmoveto{\pgfqpoint{1.353958in}{2.007051in}}%
\pgfpathlineto{\pgfqpoint{1.353958in}{2.007051in}}%
\pgfpathlineto{\pgfqpoint{1.353958in}{2.010001in}}%
\pgfpathlineto{\pgfqpoint{1.358499in}{2.010001in}}%
\pgfpathlineto{\pgfqpoint{1.358499in}{2.007051in}}%
\pgfpathmoveto{\pgfqpoint{1.358499in}{2.007051in}}%
\pgfpathlineto{\pgfqpoint{1.358499in}{2.007051in}}%
\pgfpathlineto{\pgfqpoint{1.358499in}{2.010001in}}%
\pgfpathlineto{\pgfqpoint{1.363040in}{2.010001in}}%
\pgfpathlineto{\pgfqpoint{1.363040in}{2.007051in}}%
\pgfpathmoveto{\pgfqpoint{1.363040in}{2.007051in}}%
\pgfpathlineto{\pgfqpoint{1.363040in}{2.007051in}}%
\pgfpathlineto{\pgfqpoint{1.363040in}{2.010001in}}%
\pgfpathlineto{\pgfqpoint{1.367580in}{2.010001in}}%
\pgfpathlineto{\pgfqpoint{1.367580in}{2.007051in}}%
\pgfpathmoveto{\pgfqpoint{1.367580in}{2.007051in}}%
\pgfpathlineto{\pgfqpoint{1.367580in}{2.007051in}}%
\pgfpathlineto{\pgfqpoint{1.367580in}{2.010001in}}%
\pgfpathlineto{\pgfqpoint{1.372121in}{2.010001in}}%
\pgfpathlineto{\pgfqpoint{1.372121in}{2.007051in}}%
\pgfpathmoveto{\pgfqpoint{1.372121in}{2.007051in}}%
\pgfpathlineto{\pgfqpoint{1.372121in}{2.007051in}}%
\pgfpathlineto{\pgfqpoint{1.372121in}{2.010001in}}%
\pgfpathlineto{\pgfqpoint{1.376662in}{2.010001in}}%
\pgfpathlineto{\pgfqpoint{1.376662in}{2.007051in}}%
\pgfpathmoveto{\pgfqpoint{1.376662in}{2.007051in}}%
\pgfpathlineto{\pgfqpoint{1.376662in}{2.007051in}}%
\pgfpathlineto{\pgfqpoint{1.376662in}{2.010001in}}%
\pgfpathlineto{\pgfqpoint{1.381203in}{2.010001in}}%
\pgfpathlineto{\pgfqpoint{1.381203in}{2.007051in}}%
\pgfpathmoveto{\pgfqpoint{1.381203in}{2.007051in}}%
\pgfpathlineto{\pgfqpoint{1.381203in}{2.007051in}}%
\pgfpathlineto{\pgfqpoint{1.381203in}{2.010001in}}%
\pgfpathlineto{\pgfqpoint{1.385744in}{2.010001in}}%
\pgfpathlineto{\pgfqpoint{1.385744in}{2.007051in}}%
\pgfpathmoveto{\pgfqpoint{1.385744in}{2.007051in}}%
\pgfpathlineto{\pgfqpoint{1.385744in}{2.007051in}}%
\pgfpathlineto{\pgfqpoint{1.385744in}{2.010001in}}%
\pgfpathlineto{\pgfqpoint{1.390285in}{2.010001in}}%
\pgfpathlineto{\pgfqpoint{1.390285in}{2.007051in}}%
\pgfpathmoveto{\pgfqpoint{1.390285in}{2.007051in}}%
\pgfpathlineto{\pgfqpoint{1.390285in}{2.007051in}}%
\pgfpathlineto{\pgfqpoint{1.390285in}{2.010001in}}%
\pgfpathlineto{\pgfqpoint{1.394826in}{2.010001in}}%
\pgfpathlineto{\pgfqpoint{1.394826in}{2.007051in}}%
\pgfpathmoveto{\pgfqpoint{1.394826in}{2.007051in}}%
\pgfpathlineto{\pgfqpoint{1.394826in}{2.007051in}}%
\pgfpathlineto{\pgfqpoint{1.394826in}{2.010001in}}%
\pgfpathlineto{\pgfqpoint{1.399367in}{2.010001in}}%
\pgfpathlineto{\pgfqpoint{1.399367in}{2.007051in}}%
\pgfpathmoveto{\pgfqpoint{1.399367in}{2.007051in}}%
\pgfpathlineto{\pgfqpoint{1.399367in}{2.007051in}}%
\pgfpathlineto{\pgfqpoint{1.399367in}{2.010001in}}%
\pgfpathlineto{\pgfqpoint{1.403908in}{2.010001in}}%
\pgfpathlineto{\pgfqpoint{1.403908in}{2.007051in}}%
\pgfpathmoveto{\pgfqpoint{1.403908in}{2.007051in}}%
\pgfpathlineto{\pgfqpoint{1.403908in}{2.007051in}}%
\pgfpathlineto{\pgfqpoint{1.403908in}{2.010001in}}%
\pgfpathlineto{\pgfqpoint{1.408449in}{2.010001in}}%
\pgfpathlineto{\pgfqpoint{1.408449in}{2.007051in}}%
\pgfpathmoveto{\pgfqpoint{1.408449in}{2.007051in}}%
\pgfpathlineto{\pgfqpoint{1.408449in}{2.007051in}}%
\pgfpathlineto{\pgfqpoint{1.408449in}{2.010001in}}%
\pgfpathlineto{\pgfqpoint{1.412990in}{2.010001in}}%
\pgfpathlineto{\pgfqpoint{1.412990in}{2.007051in}}%
\pgfpathmoveto{\pgfqpoint{1.412990in}{2.007051in}}%
\pgfpathlineto{\pgfqpoint{1.412990in}{2.007051in}}%
\pgfpathlineto{\pgfqpoint{1.412990in}{2.010001in}}%
\pgfpathlineto{\pgfqpoint{1.417531in}{2.010001in}}%
\pgfpathlineto{\pgfqpoint{1.417531in}{2.007051in}}%
\pgfpathmoveto{\pgfqpoint{1.417531in}{2.007051in}}%
\pgfpathlineto{\pgfqpoint{1.417531in}{2.007051in}}%
\pgfpathlineto{\pgfqpoint{1.417531in}{2.010001in}}%
\pgfpathlineto{\pgfqpoint{1.422072in}{2.010001in}}%
\pgfpathlineto{\pgfqpoint{1.422072in}{2.007051in}}%
\pgfpathmoveto{\pgfqpoint{1.422072in}{2.007051in}}%
\pgfpathlineto{\pgfqpoint{1.422072in}{2.007051in}}%
\pgfpathlineto{\pgfqpoint{1.422072in}{2.010001in}}%
\pgfpathlineto{\pgfqpoint{1.426613in}{2.010001in}}%
\pgfpathlineto{\pgfqpoint{1.426613in}{2.007051in}}%
\pgfpathmoveto{\pgfqpoint{1.426613in}{2.007051in}}%
\pgfpathlineto{\pgfqpoint{1.426613in}{2.007051in}}%
\pgfpathlineto{\pgfqpoint{1.426613in}{2.010001in}}%
\pgfpathlineto{\pgfqpoint{1.431154in}{2.010001in}}%
\pgfpathlineto{\pgfqpoint{1.431154in}{2.007051in}}%
\pgfpathmoveto{\pgfqpoint{1.431154in}{2.007051in}}%
\pgfpathlineto{\pgfqpoint{1.431154in}{2.007051in}}%
\pgfpathlineto{\pgfqpoint{1.431154in}{2.010001in}}%
\pgfpathlineto{\pgfqpoint{1.435695in}{2.010001in}}%
\pgfpathlineto{\pgfqpoint{1.435695in}{2.007051in}}%
\pgfpathmoveto{\pgfqpoint{1.435695in}{2.007051in}}%
\pgfpathlineto{\pgfqpoint{1.435695in}{2.007051in}}%
\pgfpathlineto{\pgfqpoint{1.435695in}{2.010001in}}%
\pgfpathlineto{\pgfqpoint{1.440236in}{2.010001in}}%
\pgfpathlineto{\pgfqpoint{1.440236in}{2.007051in}}%
\pgfpathmoveto{\pgfqpoint{1.440236in}{2.007051in}}%
\pgfpathlineto{\pgfqpoint{1.440236in}{2.007051in}}%
\pgfpathlineto{\pgfqpoint{1.440236in}{2.010001in}}%
\pgfpathlineto{\pgfqpoint{1.444776in}{2.010001in}}%
\pgfpathlineto{\pgfqpoint{1.444776in}{2.007051in}}%
\pgfpathmoveto{\pgfqpoint{1.444776in}{2.007051in}}%
\pgfpathlineto{\pgfqpoint{1.444776in}{2.007051in}}%
\pgfpathlineto{\pgfqpoint{1.444776in}{2.010001in}}%
\pgfpathlineto{\pgfqpoint{1.449317in}{2.010001in}}%
\pgfpathlineto{\pgfqpoint{1.449317in}{2.007051in}}%
\pgfpathmoveto{\pgfqpoint{1.449317in}{2.007051in}}%
\pgfpathlineto{\pgfqpoint{1.449317in}{2.007051in}}%
\pgfpathlineto{\pgfqpoint{1.449317in}{2.010001in}}%
\pgfpathlineto{\pgfqpoint{1.453858in}{2.010001in}}%
\pgfpathlineto{\pgfqpoint{1.453858in}{2.007051in}}%
\pgfpathmoveto{\pgfqpoint{1.453858in}{2.007051in}}%
\pgfpathlineto{\pgfqpoint{1.453858in}{2.007051in}}%
\pgfpathlineto{\pgfqpoint{1.453858in}{2.010001in}}%
\pgfpathlineto{\pgfqpoint{1.458399in}{2.010001in}}%
\pgfpathlineto{\pgfqpoint{1.458399in}{2.007051in}}%
\pgfpathmoveto{\pgfqpoint{1.458399in}{2.007051in}}%
\pgfpathlineto{\pgfqpoint{1.458399in}{2.007051in}}%
\pgfpathlineto{\pgfqpoint{1.458399in}{2.010001in}}%
\pgfpathlineto{\pgfqpoint{1.462940in}{2.010001in}}%
\pgfpathlineto{\pgfqpoint{1.462940in}{2.007051in}}%
\pgfpathmoveto{\pgfqpoint{1.462940in}{2.007051in}}%
\pgfpathlineto{\pgfqpoint{1.462940in}{2.007051in}}%
\pgfpathlineto{\pgfqpoint{1.462940in}{2.010001in}}%
\pgfpathlineto{\pgfqpoint{1.467481in}{2.010001in}}%
\pgfpathlineto{\pgfqpoint{1.467481in}{2.007051in}}%
\pgfpathmoveto{\pgfqpoint{1.467481in}{2.007051in}}%
\pgfpathlineto{\pgfqpoint{1.467481in}{2.007051in}}%
\pgfpathlineto{\pgfqpoint{1.467481in}{2.010001in}}%
\pgfpathlineto{\pgfqpoint{1.472022in}{2.010001in}}%
\pgfpathlineto{\pgfqpoint{1.472022in}{2.007051in}}%
\pgfpathmoveto{\pgfqpoint{1.472022in}{2.007051in}}%
\pgfpathlineto{\pgfqpoint{1.472022in}{2.007051in}}%
\pgfpathlineto{\pgfqpoint{1.472022in}{2.010001in}}%
\pgfpathlineto{\pgfqpoint{1.476563in}{2.010001in}}%
\pgfpathlineto{\pgfqpoint{1.476563in}{2.007051in}}%
\pgfpathmoveto{\pgfqpoint{1.476563in}{2.007051in}}%
\pgfpathlineto{\pgfqpoint{1.476563in}{2.007051in}}%
\pgfpathlineto{\pgfqpoint{1.476563in}{2.010001in}}%
\pgfpathlineto{\pgfqpoint{1.481104in}{2.010001in}}%
\pgfpathlineto{\pgfqpoint{1.481104in}{2.007051in}}%
\pgfpathmoveto{\pgfqpoint{1.481104in}{2.007051in}}%
\pgfpathlineto{\pgfqpoint{1.481104in}{2.007051in}}%
\pgfpathlineto{\pgfqpoint{1.481104in}{2.010001in}}%
\pgfpathlineto{\pgfqpoint{1.485645in}{2.010001in}}%
\pgfpathlineto{\pgfqpoint{1.485645in}{2.007051in}}%
\pgfpathmoveto{\pgfqpoint{1.485645in}{2.007051in}}%
\pgfpathlineto{\pgfqpoint{1.485645in}{2.007051in}}%
\pgfpathlineto{\pgfqpoint{1.485645in}{2.010001in}}%
\pgfpathlineto{\pgfqpoint{1.490186in}{2.010001in}}%
\pgfpathlineto{\pgfqpoint{1.490186in}{2.007051in}}%
\pgfpathmoveto{\pgfqpoint{1.490186in}{2.007051in}}%
\pgfpathlineto{\pgfqpoint{1.490186in}{2.007051in}}%
\pgfpathlineto{\pgfqpoint{1.490186in}{2.010001in}}%
\pgfpathlineto{\pgfqpoint{1.494727in}{2.010001in}}%
\pgfpathlineto{\pgfqpoint{1.494727in}{2.007051in}}%
\pgfpathmoveto{\pgfqpoint{1.494727in}{2.007051in}}%
\pgfpathlineto{\pgfqpoint{1.494727in}{2.007051in}}%
\pgfpathlineto{\pgfqpoint{1.494727in}{2.010001in}}%
\pgfpathlineto{\pgfqpoint{1.499268in}{2.010001in}}%
\pgfpathlineto{\pgfqpoint{1.499268in}{2.007051in}}%
\pgfpathmoveto{\pgfqpoint{1.499268in}{2.007051in}}%
\pgfpathlineto{\pgfqpoint{1.499268in}{2.007051in}}%
\pgfpathlineto{\pgfqpoint{1.499268in}{2.010001in}}%
\pgfpathlineto{\pgfqpoint{1.503810in}{2.010001in}}%
\pgfpathlineto{\pgfqpoint{1.503810in}{2.007051in}}%
\pgfpathmoveto{\pgfqpoint{1.503810in}{2.007051in}}%
\pgfpathlineto{\pgfqpoint{1.503810in}{2.007051in}}%
\pgfpathlineto{\pgfqpoint{1.503810in}{2.010001in}}%
\pgfpathlineto{\pgfqpoint{1.508351in}{2.010001in}}%
\pgfpathlineto{\pgfqpoint{1.508351in}{2.007051in}}%
\pgfpathmoveto{\pgfqpoint{1.508351in}{2.007051in}}%
\pgfpathlineto{\pgfqpoint{1.508351in}{2.007051in}}%
\pgfpathlineto{\pgfqpoint{1.508351in}{2.010001in}}%
\pgfpathlineto{\pgfqpoint{1.512892in}{2.010001in}}%
\pgfpathlineto{\pgfqpoint{1.512892in}{2.007051in}}%
\pgfpathmoveto{\pgfqpoint{1.512892in}{2.007051in}}%
\pgfpathlineto{\pgfqpoint{1.512892in}{2.007051in}}%
\pgfpathlineto{\pgfqpoint{1.512892in}{2.010001in}}%
\pgfpathlineto{\pgfqpoint{1.517433in}{2.010001in}}%
\pgfpathlineto{\pgfqpoint{1.517433in}{2.007051in}}%
\pgfpathmoveto{\pgfqpoint{1.517433in}{2.007051in}}%
\pgfpathlineto{\pgfqpoint{1.517433in}{2.007051in}}%
\pgfpathlineto{\pgfqpoint{1.517433in}{2.010001in}}%
\pgfpathlineto{\pgfqpoint{1.521974in}{2.010001in}}%
\pgfpathlineto{\pgfqpoint{1.521974in}{2.007051in}}%
\pgfpathmoveto{\pgfqpoint{1.521974in}{2.007051in}}%
\pgfpathlineto{\pgfqpoint{1.521974in}{2.007051in}}%
\pgfpathlineto{\pgfqpoint{1.521974in}{2.010001in}}%
\pgfpathlineto{\pgfqpoint{1.526515in}{2.010001in}}%
\pgfpathlineto{\pgfqpoint{1.526515in}{2.007051in}}%
\pgfpathmoveto{\pgfqpoint{1.526515in}{2.007051in}}%
\pgfpathlineto{\pgfqpoint{1.526515in}{2.007051in}}%
\pgfpathlineto{\pgfqpoint{1.526515in}{2.010001in}}%
\pgfpathlineto{\pgfqpoint{1.531056in}{2.010001in}}%
\pgfpathlineto{\pgfqpoint{1.531056in}{2.007051in}}%
\pgfpathmoveto{\pgfqpoint{1.531056in}{2.007051in}}%
\pgfpathlineto{\pgfqpoint{1.531056in}{2.007051in}}%
\pgfpathlineto{\pgfqpoint{1.531056in}{2.010001in}}%
\pgfpathlineto{\pgfqpoint{1.535597in}{2.010001in}}%
\pgfpathlineto{\pgfqpoint{1.535597in}{2.007051in}}%
\pgfpathmoveto{\pgfqpoint{1.535597in}{2.007051in}}%
\pgfpathlineto{\pgfqpoint{1.535597in}{2.007051in}}%
\pgfpathlineto{\pgfqpoint{1.535597in}{2.010001in}}%
\pgfpathlineto{\pgfqpoint{1.540138in}{2.010001in}}%
\pgfpathlineto{\pgfqpoint{1.540138in}{2.007051in}}%
\pgfpathmoveto{\pgfqpoint{1.540138in}{2.007051in}}%
\pgfpathlineto{\pgfqpoint{1.540138in}{2.007051in}}%
\pgfpathlineto{\pgfqpoint{1.540138in}{2.010001in}}%
\pgfpathlineto{\pgfqpoint{1.544679in}{2.010001in}}%
\pgfpathlineto{\pgfqpoint{1.544679in}{2.007051in}}%
\pgfpathmoveto{\pgfqpoint{1.544679in}{2.007051in}}%
\pgfpathlineto{\pgfqpoint{1.544679in}{2.007051in}}%
\pgfpathlineto{\pgfqpoint{1.544679in}{2.010001in}}%
\pgfpathlineto{\pgfqpoint{1.549220in}{2.010001in}}%
\pgfpathlineto{\pgfqpoint{1.549220in}{2.007051in}}%
\pgfpathmoveto{\pgfqpoint{1.549220in}{2.007051in}}%
\pgfpathlineto{\pgfqpoint{1.549220in}{2.007051in}}%
\pgfpathlineto{\pgfqpoint{1.549220in}{2.010001in}}%
\pgfpathlineto{\pgfqpoint{1.553761in}{2.010001in}}%
\pgfpathlineto{\pgfqpoint{1.553761in}{2.007051in}}%
\pgfpathmoveto{\pgfqpoint{1.553761in}{2.007051in}}%
\pgfpathlineto{\pgfqpoint{1.553761in}{2.007051in}}%
\pgfpathlineto{\pgfqpoint{1.553761in}{2.010001in}}%
\pgfpathlineto{\pgfqpoint{1.558302in}{2.010001in}}%
\pgfpathlineto{\pgfqpoint{1.558302in}{2.007051in}}%
\pgfpathmoveto{\pgfqpoint{1.558302in}{2.007051in}}%
\pgfpathlineto{\pgfqpoint{1.558302in}{2.007051in}}%
\pgfpathlineto{\pgfqpoint{1.558302in}{2.010001in}}%
\pgfpathlineto{\pgfqpoint{1.562844in}{2.010001in}}%
\pgfpathlineto{\pgfqpoint{1.562844in}{2.007051in}}%
\pgfpathmoveto{\pgfqpoint{1.562844in}{2.007051in}}%
\pgfpathlineto{\pgfqpoint{1.562844in}{2.007051in}}%
\pgfpathlineto{\pgfqpoint{1.562844in}{2.010001in}}%
\pgfpathlineto{\pgfqpoint{1.567385in}{2.010001in}}%
\pgfpathlineto{\pgfqpoint{1.567385in}{2.007051in}}%
\pgfpathmoveto{\pgfqpoint{1.567385in}{2.007051in}}%
\pgfpathlineto{\pgfqpoint{1.567385in}{2.007051in}}%
\pgfpathlineto{\pgfqpoint{1.567385in}{2.010001in}}%
\pgfpathlineto{\pgfqpoint{1.571926in}{2.010001in}}%
\pgfpathlineto{\pgfqpoint{1.571926in}{2.007051in}}%
\pgfpathmoveto{\pgfqpoint{1.571926in}{2.007051in}}%
\pgfpathlineto{\pgfqpoint{1.571926in}{2.007051in}}%
\pgfpathlineto{\pgfqpoint{1.571926in}{2.010001in}}%
\pgfpathlineto{\pgfqpoint{1.576467in}{2.010001in}}%
\pgfpathlineto{\pgfqpoint{1.576467in}{2.007051in}}%
\pgfpathmoveto{\pgfqpoint{1.576467in}{2.007051in}}%
\pgfpathlineto{\pgfqpoint{1.576467in}{2.007051in}}%
\pgfpathlineto{\pgfqpoint{1.576467in}{2.010001in}}%
\pgfpathlineto{\pgfqpoint{1.581008in}{2.010001in}}%
\pgfpathlineto{\pgfqpoint{1.581008in}{2.007051in}}%
\pgfpathmoveto{\pgfqpoint{1.581008in}{2.007051in}}%
\pgfpathlineto{\pgfqpoint{1.581008in}{2.007051in}}%
\pgfpathlineto{\pgfqpoint{1.581008in}{2.010001in}}%
\pgfpathlineto{\pgfqpoint{1.585549in}{2.010001in}}%
\pgfpathlineto{\pgfqpoint{1.585549in}{2.007051in}}%
\pgfpathmoveto{\pgfqpoint{1.585549in}{2.007051in}}%
\pgfpathlineto{\pgfqpoint{1.585549in}{2.007051in}}%
\pgfpathlineto{\pgfqpoint{1.585549in}{2.010001in}}%
\pgfpathlineto{\pgfqpoint{1.590090in}{2.010001in}}%
\pgfpathlineto{\pgfqpoint{1.590090in}{2.007051in}}%
\pgfpathmoveto{\pgfqpoint{1.590090in}{2.007051in}}%
\pgfpathlineto{\pgfqpoint{1.590090in}{2.007051in}}%
\pgfpathlineto{\pgfqpoint{1.590090in}{2.010001in}}%
\pgfpathlineto{\pgfqpoint{1.594631in}{2.010001in}}%
\pgfpathlineto{\pgfqpoint{1.594631in}{2.007051in}}%
\pgfpathmoveto{\pgfqpoint{1.594631in}{2.007051in}}%
\pgfpathlineto{\pgfqpoint{1.594631in}{2.007051in}}%
\pgfpathlineto{\pgfqpoint{1.594631in}{2.010001in}}%
\pgfpathlineto{\pgfqpoint{1.599172in}{2.010001in}}%
\pgfpathlineto{\pgfqpoint{1.599172in}{2.007051in}}%
\pgfpathmoveto{\pgfqpoint{1.599172in}{2.007051in}}%
\pgfpathlineto{\pgfqpoint{1.599172in}{2.007051in}}%
\pgfpathlineto{\pgfqpoint{1.599172in}{2.010001in}}%
\pgfpathlineto{\pgfqpoint{1.603713in}{2.010001in}}%
\pgfpathlineto{\pgfqpoint{1.603713in}{2.007051in}}%
\pgfpathmoveto{\pgfqpoint{1.603713in}{2.007051in}}%
\pgfpathlineto{\pgfqpoint{1.603713in}{2.007051in}}%
\pgfpathlineto{\pgfqpoint{1.603713in}{2.010001in}}%
\pgfpathlineto{\pgfqpoint{1.608254in}{2.010001in}}%
\pgfpathlineto{\pgfqpoint{1.608254in}{2.007051in}}%
\pgfpathmoveto{\pgfqpoint{1.608254in}{2.007051in}}%
\pgfpathlineto{\pgfqpoint{1.608254in}{2.007051in}}%
\pgfpathlineto{\pgfqpoint{1.608254in}{2.010001in}}%
\pgfpathlineto{\pgfqpoint{1.612795in}{2.010001in}}%
\pgfpathlineto{\pgfqpoint{1.612795in}{2.007051in}}%
\pgfpathmoveto{\pgfqpoint{1.612795in}{2.007051in}}%
\pgfpathlineto{\pgfqpoint{1.612795in}{2.007051in}}%
\pgfpathlineto{\pgfqpoint{1.612795in}{2.010001in}}%
\pgfpathlineto{\pgfqpoint{1.617336in}{2.010001in}}%
\pgfpathlineto{\pgfqpoint{1.617336in}{2.007051in}}%
\pgfpathmoveto{\pgfqpoint{1.617336in}{2.007051in}}%
\pgfpathlineto{\pgfqpoint{1.617336in}{2.007051in}}%
\pgfpathlineto{\pgfqpoint{1.617336in}{2.010001in}}%
\pgfpathlineto{\pgfqpoint{1.621878in}{2.010001in}}%
\pgfpathlineto{\pgfqpoint{1.621878in}{2.007051in}}%
\pgfpathmoveto{\pgfqpoint{1.621878in}{2.007051in}}%
\pgfpathlineto{\pgfqpoint{1.621878in}{2.007051in}}%
\pgfpathlineto{\pgfqpoint{1.621878in}{2.010001in}}%
\pgfpathlineto{\pgfqpoint{1.626419in}{2.010001in}}%
\pgfpathlineto{\pgfqpoint{1.626419in}{2.007051in}}%
\pgfpathmoveto{\pgfqpoint{1.626419in}{2.007051in}}%
\pgfpathlineto{\pgfqpoint{1.626419in}{2.007051in}}%
\pgfpathlineto{\pgfqpoint{1.626419in}{2.010001in}}%
\pgfpathlineto{\pgfqpoint{1.630959in}{2.010001in}}%
\pgfpathlineto{\pgfqpoint{1.630959in}{2.007051in}}%
\pgfpathmoveto{\pgfqpoint{1.630959in}{2.007051in}}%
\pgfpathlineto{\pgfqpoint{1.630959in}{2.007051in}}%
\pgfpathlineto{\pgfqpoint{1.630959in}{2.010001in}}%
\pgfpathlineto{\pgfqpoint{1.635500in}{2.010001in}}%
\pgfpathlineto{\pgfqpoint{1.635500in}{2.007051in}}%
\pgfpathmoveto{\pgfqpoint{1.635500in}{2.007051in}}%
\pgfpathlineto{\pgfqpoint{1.635500in}{2.007051in}}%
\pgfpathlineto{\pgfqpoint{1.635500in}{2.010001in}}%
\pgfpathlineto{\pgfqpoint{1.640041in}{2.010001in}}%
\pgfpathlineto{\pgfqpoint{1.640041in}{2.007051in}}%
\pgfpathmoveto{\pgfqpoint{1.640041in}{2.007051in}}%
\pgfpathlineto{\pgfqpoint{1.640041in}{2.007051in}}%
\pgfpathlineto{\pgfqpoint{1.640041in}{2.010001in}}%
\pgfpathlineto{\pgfqpoint{1.644582in}{2.010001in}}%
\pgfpathlineto{\pgfqpoint{1.644582in}{2.007051in}}%
\pgfpathmoveto{\pgfqpoint{1.644582in}{2.007051in}}%
\pgfpathlineto{\pgfqpoint{1.644582in}{2.007051in}}%
\pgfpathlineto{\pgfqpoint{1.644582in}{2.010001in}}%
\pgfpathlineto{\pgfqpoint{1.649123in}{2.010001in}}%
\pgfpathlineto{\pgfqpoint{1.649123in}{2.007051in}}%
\pgfpathmoveto{\pgfqpoint{1.649123in}{2.007051in}}%
\pgfpathlineto{\pgfqpoint{1.649123in}{2.007051in}}%
\pgfpathlineto{\pgfqpoint{1.649123in}{2.010001in}}%
\pgfpathlineto{\pgfqpoint{1.653664in}{2.010001in}}%
\pgfpathlineto{\pgfqpoint{1.653664in}{2.007051in}}%
\pgfpathmoveto{\pgfqpoint{1.653664in}{2.007051in}}%
\pgfpathlineto{\pgfqpoint{1.653664in}{2.007051in}}%
\pgfpathlineto{\pgfqpoint{1.653664in}{2.010001in}}%
\pgfpathlineto{\pgfqpoint{1.658205in}{2.010001in}}%
\pgfpathlineto{\pgfqpoint{1.658205in}{2.007051in}}%
\pgfpathmoveto{\pgfqpoint{1.658205in}{2.007051in}}%
\pgfpathlineto{\pgfqpoint{1.658205in}{2.007051in}}%
\pgfpathlineto{\pgfqpoint{1.658205in}{2.010001in}}%
\pgfpathlineto{\pgfqpoint{1.662746in}{2.010001in}}%
\pgfpathlineto{\pgfqpoint{1.662746in}{2.007051in}}%
\pgfpathmoveto{\pgfqpoint{1.662746in}{2.007051in}}%
\pgfpathlineto{\pgfqpoint{1.662746in}{2.007051in}}%
\pgfpathlineto{\pgfqpoint{1.662746in}{2.010001in}}%
\pgfpathlineto{\pgfqpoint{1.667287in}{2.010001in}}%
\pgfpathlineto{\pgfqpoint{1.667287in}{2.007051in}}%
\pgfpathmoveto{\pgfqpoint{1.667287in}{2.007051in}}%
\pgfpathlineto{\pgfqpoint{1.667287in}{2.007051in}}%
\pgfpathlineto{\pgfqpoint{1.667287in}{2.010001in}}%
\pgfpathlineto{\pgfqpoint{1.671828in}{2.010001in}}%
\pgfpathlineto{\pgfqpoint{1.671828in}{2.007051in}}%
\pgfpathmoveto{\pgfqpoint{1.671828in}{2.007051in}}%
\pgfpathlineto{\pgfqpoint{1.671828in}{2.007051in}}%
\pgfpathlineto{\pgfqpoint{1.671828in}{2.010001in}}%
\pgfpathlineto{\pgfqpoint{1.676369in}{2.010001in}}%
\pgfpathlineto{\pgfqpoint{1.676369in}{2.007051in}}%
\pgfpathmoveto{\pgfqpoint{1.676369in}{2.007051in}}%
\pgfpathlineto{\pgfqpoint{1.676369in}{2.007051in}}%
\pgfpathlineto{\pgfqpoint{1.676369in}{2.010001in}}%
\pgfpathlineto{\pgfqpoint{1.680910in}{2.010001in}}%
\pgfpathlineto{\pgfqpoint{1.680910in}{2.007051in}}%
\pgfpathmoveto{\pgfqpoint{1.680910in}{2.007051in}}%
\pgfpathlineto{\pgfqpoint{1.680910in}{2.007051in}}%
\pgfpathlineto{\pgfqpoint{1.680910in}{2.010001in}}%
\pgfpathlineto{\pgfqpoint{1.685451in}{2.010001in}}%
\pgfpathlineto{\pgfqpoint{1.685451in}{2.007051in}}%
\pgfpathmoveto{\pgfqpoint{1.685451in}{2.007051in}}%
\pgfpathlineto{\pgfqpoint{1.685451in}{2.007051in}}%
\pgfpathlineto{\pgfqpoint{1.685451in}{2.010001in}}%
\pgfpathlineto{\pgfqpoint{1.689992in}{2.010001in}}%
\pgfpathlineto{\pgfqpoint{1.689992in}{2.007051in}}%
\pgfpathmoveto{\pgfqpoint{1.689992in}{2.007051in}}%
\pgfpathlineto{\pgfqpoint{1.689992in}{2.007051in}}%
\pgfpathlineto{\pgfqpoint{1.689992in}{2.010001in}}%
\pgfpathlineto{\pgfqpoint{1.694533in}{2.010001in}}%
\pgfpathlineto{\pgfqpoint{1.694533in}{2.007051in}}%
\pgfpathmoveto{\pgfqpoint{1.694533in}{2.007051in}}%
\pgfpathlineto{\pgfqpoint{1.694533in}{2.007051in}}%
\pgfpathlineto{\pgfqpoint{1.694533in}{2.010001in}}%
\pgfpathlineto{\pgfqpoint{1.699074in}{2.010001in}}%
\pgfpathlineto{\pgfqpoint{1.699074in}{2.007051in}}%
\pgfpathmoveto{\pgfqpoint{1.699074in}{2.007051in}}%
\pgfpathlineto{\pgfqpoint{1.699074in}{2.007051in}}%
\pgfpathlineto{\pgfqpoint{1.699074in}{2.010001in}}%
\pgfpathlineto{\pgfqpoint{1.703615in}{2.010001in}}%
\pgfpathlineto{\pgfqpoint{1.703615in}{2.007051in}}%
\pgfpathmoveto{\pgfqpoint{1.703615in}{2.007051in}}%
\pgfpathlineto{\pgfqpoint{1.703615in}{2.007051in}}%
\pgfpathlineto{\pgfqpoint{1.703615in}{2.010001in}}%
\pgfpathlineto{\pgfqpoint{1.708156in}{2.010001in}}%
\pgfpathlineto{\pgfqpoint{1.708156in}{2.007051in}}%
\pgfpathmoveto{\pgfqpoint{1.708156in}{2.007051in}}%
\pgfpathlineto{\pgfqpoint{1.708156in}{2.007051in}}%
\pgfpathlineto{\pgfqpoint{1.708156in}{2.010001in}}%
\pgfpathlineto{\pgfqpoint{1.712697in}{2.010001in}}%
\pgfpathlineto{\pgfqpoint{1.712697in}{2.007051in}}%
\pgfpathmoveto{\pgfqpoint{1.712697in}{2.007051in}}%
\pgfpathlineto{\pgfqpoint{1.712697in}{2.007051in}}%
\pgfpathlineto{\pgfqpoint{1.712697in}{2.010001in}}%
\pgfpathlineto{\pgfqpoint{1.717238in}{2.010001in}}%
\pgfpathlineto{\pgfqpoint{1.717238in}{2.007051in}}%
\pgfpathmoveto{\pgfqpoint{1.717238in}{2.007051in}}%
\pgfpathlineto{\pgfqpoint{1.717238in}{2.007051in}}%
\pgfpathlineto{\pgfqpoint{1.717238in}{2.010001in}}%
\pgfpathlineto{\pgfqpoint{1.721779in}{2.010001in}}%
\pgfpathlineto{\pgfqpoint{1.721779in}{2.007051in}}%
\pgfpathmoveto{\pgfqpoint{1.721779in}{2.007051in}}%
\pgfpathlineto{\pgfqpoint{1.721779in}{2.007051in}}%
\pgfpathlineto{\pgfqpoint{1.721779in}{2.010001in}}%
\pgfpathlineto{\pgfqpoint{1.726320in}{2.010001in}}%
\pgfpathlineto{\pgfqpoint{1.726320in}{2.007051in}}%
\pgfpathmoveto{\pgfqpoint{1.726320in}{2.007051in}}%
\pgfpathlineto{\pgfqpoint{1.726320in}{2.007051in}}%
\pgfpathlineto{\pgfqpoint{1.726320in}{2.010001in}}%
\pgfpathlineto{\pgfqpoint{1.730861in}{2.010001in}}%
\pgfpathlineto{\pgfqpoint{1.730861in}{2.007051in}}%
\pgfpathmoveto{\pgfqpoint{1.730861in}{2.007051in}}%
\pgfpathlineto{\pgfqpoint{1.730861in}{2.007051in}}%
\pgfpathlineto{\pgfqpoint{1.730861in}{2.010001in}}%
\pgfpathlineto{\pgfqpoint{1.735402in}{2.010001in}}%
\pgfpathlineto{\pgfqpoint{1.735402in}{2.007051in}}%
\pgfpathmoveto{\pgfqpoint{1.735402in}{2.007051in}}%
\pgfpathlineto{\pgfqpoint{1.735402in}{2.007051in}}%
\pgfpathlineto{\pgfqpoint{1.735402in}{2.010001in}}%
\pgfpathlineto{\pgfqpoint{1.739943in}{2.010001in}}%
\pgfpathlineto{\pgfqpoint{1.739943in}{2.007051in}}%
\pgfpathmoveto{\pgfqpoint{1.739943in}{2.007051in}}%
\pgfpathlineto{\pgfqpoint{1.739943in}{2.007051in}}%
\pgfpathlineto{\pgfqpoint{1.739943in}{2.010001in}}%
\pgfpathlineto{\pgfqpoint{1.744484in}{2.010001in}}%
\pgfpathlineto{\pgfqpoint{1.744484in}{2.007051in}}%
\pgfpathmoveto{\pgfqpoint{1.744484in}{2.007051in}}%
\pgfpathlineto{\pgfqpoint{1.744484in}{2.007051in}}%
\pgfpathlineto{\pgfqpoint{1.744484in}{2.010001in}}%
\pgfpathlineto{\pgfqpoint{1.749025in}{2.010001in}}%
\pgfpathlineto{\pgfqpoint{1.749025in}{2.007051in}}%
\pgfpathmoveto{\pgfqpoint{1.749025in}{2.007051in}}%
\pgfpathlineto{\pgfqpoint{1.749025in}{2.007051in}}%
\pgfpathlineto{\pgfqpoint{1.749025in}{2.010001in}}%
\pgfpathlineto{\pgfqpoint{1.753566in}{2.010001in}}%
\pgfpathlineto{\pgfqpoint{1.753566in}{2.007051in}}%
\pgfpathmoveto{\pgfqpoint{1.753566in}{2.007051in}}%
\pgfpathlineto{\pgfqpoint{1.753566in}{2.007051in}}%
\pgfpathlineto{\pgfqpoint{1.753566in}{2.010001in}}%
\pgfpathlineto{\pgfqpoint{1.758106in}{2.010001in}}%
\pgfpathlineto{\pgfqpoint{1.758106in}{2.007051in}}%
\pgfpathmoveto{\pgfqpoint{1.758106in}{2.007051in}}%
\pgfpathlineto{\pgfqpoint{1.758106in}{2.007051in}}%
\pgfpathlineto{\pgfqpoint{1.758106in}{2.010001in}}%
\pgfpathlineto{\pgfqpoint{1.762647in}{2.010001in}}%
\pgfpathlineto{\pgfqpoint{1.762647in}{2.007051in}}%
\pgfpathmoveto{\pgfqpoint{1.762647in}{2.007051in}}%
\pgfpathlineto{\pgfqpoint{1.762647in}{2.007051in}}%
\pgfpathlineto{\pgfqpoint{1.762647in}{2.010001in}}%
\pgfpathlineto{\pgfqpoint{1.767188in}{2.010001in}}%
\pgfpathlineto{\pgfqpoint{1.767188in}{2.007051in}}%
\pgfpathmoveto{\pgfqpoint{1.767188in}{2.007051in}}%
\pgfpathlineto{\pgfqpoint{1.767188in}{2.007051in}}%
\pgfpathlineto{\pgfqpoint{1.767188in}{2.010001in}}%
\pgfpathlineto{\pgfqpoint{1.771729in}{2.010001in}}%
\pgfpathlineto{\pgfqpoint{1.771729in}{2.007051in}}%
\pgfpathmoveto{\pgfqpoint{1.771729in}{2.007051in}}%
\pgfpathlineto{\pgfqpoint{1.771729in}{2.007051in}}%
\pgfpathlineto{\pgfqpoint{1.771729in}{2.010001in}}%
\pgfpathlineto{\pgfqpoint{1.776271in}{2.010001in}}%
\pgfpathlineto{\pgfqpoint{1.776271in}{2.007051in}}%
\pgfpathmoveto{\pgfqpoint{1.776271in}{2.007051in}}%
\pgfpathlineto{\pgfqpoint{1.776271in}{2.007051in}}%
\pgfpathlineto{\pgfqpoint{1.776271in}{2.010001in}}%
\pgfpathlineto{\pgfqpoint{1.780812in}{2.010001in}}%
\pgfpathlineto{\pgfqpoint{1.780812in}{2.007051in}}%
\pgfpathmoveto{\pgfqpoint{1.780812in}{2.007051in}}%
\pgfpathlineto{\pgfqpoint{1.780812in}{2.007051in}}%
\pgfpathlineto{\pgfqpoint{1.780812in}{2.010001in}}%
\pgfpathlineto{\pgfqpoint{1.785353in}{2.010001in}}%
\pgfpathlineto{\pgfqpoint{1.785353in}{2.007051in}}%
\pgfpathmoveto{\pgfqpoint{1.785353in}{2.007051in}}%
\pgfpathlineto{\pgfqpoint{1.785353in}{2.007051in}}%
\pgfpathlineto{\pgfqpoint{1.785353in}{2.010001in}}%
\pgfpathlineto{\pgfqpoint{1.789894in}{2.010001in}}%
\pgfpathlineto{\pgfqpoint{1.789894in}{2.007051in}}%
\pgfpathmoveto{\pgfqpoint{1.789894in}{2.007051in}}%
\pgfpathlineto{\pgfqpoint{1.789894in}{2.007051in}}%
\pgfpathlineto{\pgfqpoint{1.789894in}{2.010001in}}%
\pgfpathlineto{\pgfqpoint{1.794435in}{2.010001in}}%
\pgfpathlineto{\pgfqpoint{1.794435in}{2.007051in}}%
\pgfpathmoveto{\pgfqpoint{1.794435in}{2.007051in}}%
\pgfpathlineto{\pgfqpoint{1.794435in}{2.007051in}}%
\pgfpathlineto{\pgfqpoint{1.794435in}{2.010001in}}%
\pgfpathlineto{\pgfqpoint{1.798976in}{2.010001in}}%
\pgfpathlineto{\pgfqpoint{1.798976in}{2.007051in}}%
\pgfpathmoveto{\pgfqpoint{1.798976in}{2.007051in}}%
\pgfpathlineto{\pgfqpoint{1.798976in}{2.007051in}}%
\pgfpathlineto{\pgfqpoint{1.798976in}{2.010001in}}%
\pgfpathlineto{\pgfqpoint{1.803517in}{2.010001in}}%
\pgfpathlineto{\pgfqpoint{1.803517in}{2.007051in}}%
\pgfpathmoveto{\pgfqpoint{1.803517in}{2.007051in}}%
\pgfpathlineto{\pgfqpoint{1.803517in}{2.007051in}}%
\pgfpathlineto{\pgfqpoint{1.803517in}{2.010001in}}%
\pgfpathlineto{\pgfqpoint{1.808058in}{2.010001in}}%
\pgfpathlineto{\pgfqpoint{1.808058in}{2.007051in}}%
\pgfpathmoveto{\pgfqpoint{1.808058in}{2.007051in}}%
\pgfpathlineto{\pgfqpoint{1.808058in}{2.007051in}}%
\pgfpathlineto{\pgfqpoint{1.808058in}{2.010001in}}%
\pgfpathlineto{\pgfqpoint{1.812599in}{2.010001in}}%
\pgfpathlineto{\pgfqpoint{1.812599in}{2.007051in}}%
\pgfpathmoveto{\pgfqpoint{1.812599in}{2.007051in}}%
\pgfpathlineto{\pgfqpoint{1.812599in}{2.007051in}}%
\pgfpathlineto{\pgfqpoint{1.812599in}{2.010001in}}%
\pgfpathlineto{\pgfqpoint{1.817140in}{2.010001in}}%
\pgfpathlineto{\pgfqpoint{1.817140in}{2.007051in}}%
\pgfpathmoveto{\pgfqpoint{1.817140in}{2.007051in}}%
\pgfpathlineto{\pgfqpoint{1.817140in}{2.007051in}}%
\pgfpathlineto{\pgfqpoint{1.817140in}{2.010001in}}%
\pgfpathlineto{\pgfqpoint{1.821681in}{2.010001in}}%
\pgfpathlineto{\pgfqpoint{1.821681in}{2.007051in}}%
\pgfpathmoveto{\pgfqpoint{1.821681in}{2.007051in}}%
\pgfpathlineto{\pgfqpoint{1.821681in}{2.007051in}}%
\pgfpathlineto{\pgfqpoint{1.821681in}{2.010001in}}%
\pgfpathlineto{\pgfqpoint{1.826222in}{2.010001in}}%
\pgfpathlineto{\pgfqpoint{1.826222in}{2.007051in}}%
\pgfpathmoveto{\pgfqpoint{1.826222in}{2.007051in}}%
\pgfpathlineto{\pgfqpoint{1.826222in}{2.007051in}}%
\pgfpathlineto{\pgfqpoint{1.826222in}{2.010001in}}%
\pgfpathlineto{\pgfqpoint{1.830764in}{2.010001in}}%
\pgfpathlineto{\pgfqpoint{1.830764in}{2.007051in}}%
\pgfpathmoveto{\pgfqpoint{1.830764in}{2.007051in}}%
\pgfpathlineto{\pgfqpoint{1.830764in}{2.007051in}}%
\pgfpathlineto{\pgfqpoint{1.830764in}{2.010001in}}%
\pgfpathlineto{\pgfqpoint{1.835305in}{2.010001in}}%
\pgfpathlineto{\pgfqpoint{1.835305in}{2.007051in}}%
\pgfpathmoveto{\pgfqpoint{1.835305in}{2.007051in}}%
\pgfpathlineto{\pgfqpoint{1.835305in}{2.007051in}}%
\pgfpathlineto{\pgfqpoint{1.835305in}{2.010001in}}%
\pgfpathlineto{\pgfqpoint{1.839846in}{2.010001in}}%
\pgfpathlineto{\pgfqpoint{1.839846in}{2.007051in}}%
\pgfpathmoveto{\pgfqpoint{1.839846in}{2.007051in}}%
\pgfpathlineto{\pgfqpoint{1.839846in}{2.007051in}}%
\pgfpathlineto{\pgfqpoint{1.839846in}{2.010001in}}%
\pgfpathlineto{\pgfqpoint{1.844387in}{2.010001in}}%
\pgfpathlineto{\pgfqpoint{1.844387in}{2.007051in}}%
\pgfpathmoveto{\pgfqpoint{1.844387in}{2.007051in}}%
\pgfpathlineto{\pgfqpoint{1.844387in}{2.007051in}}%
\pgfpathlineto{\pgfqpoint{1.844387in}{2.010001in}}%
\pgfpathlineto{\pgfqpoint{1.848928in}{2.010001in}}%
\pgfpathlineto{\pgfqpoint{1.848928in}{2.007051in}}%
\pgfpathmoveto{\pgfqpoint{1.848928in}{2.007051in}}%
\pgfpathlineto{\pgfqpoint{1.848928in}{2.007051in}}%
\pgfpathlineto{\pgfqpoint{1.848928in}{2.010001in}}%
\pgfpathlineto{\pgfqpoint{1.853469in}{2.010001in}}%
\pgfpathlineto{\pgfqpoint{1.853469in}{2.007051in}}%
\pgfpathmoveto{\pgfqpoint{1.853469in}{2.007051in}}%
\pgfpathlineto{\pgfqpoint{1.853469in}{2.007051in}}%
\pgfpathlineto{\pgfqpoint{1.853469in}{2.010001in}}%
\pgfpathlineto{\pgfqpoint{1.858010in}{2.010001in}}%
\pgfpathlineto{\pgfqpoint{1.858010in}{2.007051in}}%
\pgfpathmoveto{\pgfqpoint{1.858010in}{2.007051in}}%
\pgfpathlineto{\pgfqpoint{1.858010in}{2.007051in}}%
\pgfpathlineto{\pgfqpoint{1.858010in}{2.010001in}}%
\pgfpathlineto{\pgfqpoint{1.862551in}{2.010001in}}%
\pgfpathlineto{\pgfqpoint{1.862551in}{2.007051in}}%
\pgfpathmoveto{\pgfqpoint{1.862551in}{2.007051in}}%
\pgfpathlineto{\pgfqpoint{1.862551in}{2.007051in}}%
\pgfpathlineto{\pgfqpoint{1.862551in}{2.010001in}}%
\pgfpathlineto{\pgfqpoint{1.867092in}{2.010001in}}%
\pgfpathlineto{\pgfqpoint{1.867092in}{2.007051in}}%
\pgfpathmoveto{\pgfqpoint{1.867092in}{2.007051in}}%
\pgfpathlineto{\pgfqpoint{1.867092in}{2.007051in}}%
\pgfpathlineto{\pgfqpoint{1.867092in}{2.010001in}}%
\pgfpathlineto{\pgfqpoint{1.871633in}{2.010001in}}%
\pgfpathlineto{\pgfqpoint{1.871633in}{2.007051in}}%
\pgfpathmoveto{\pgfqpoint{1.871633in}{2.007051in}}%
\pgfpathlineto{\pgfqpoint{1.871633in}{2.007051in}}%
\pgfpathlineto{\pgfqpoint{1.871633in}{2.010001in}}%
\pgfpathlineto{\pgfqpoint{1.876174in}{2.010001in}}%
\pgfpathlineto{\pgfqpoint{1.876174in}{2.007051in}}%
\pgfpathmoveto{\pgfqpoint{1.876174in}{2.007051in}}%
\pgfpathlineto{\pgfqpoint{1.876174in}{2.007051in}}%
\pgfpathlineto{\pgfqpoint{1.876174in}{2.010001in}}%
\pgfpathlineto{\pgfqpoint{1.880715in}{2.010001in}}%
\pgfpathlineto{\pgfqpoint{1.880715in}{2.007051in}}%
\pgfpathmoveto{\pgfqpoint{1.880715in}{2.007051in}}%
\pgfpathlineto{\pgfqpoint{1.880715in}{2.007051in}}%
\pgfpathlineto{\pgfqpoint{1.880715in}{2.010001in}}%
\pgfpathlineto{\pgfqpoint{1.885257in}{2.010001in}}%
\pgfpathlineto{\pgfqpoint{1.885257in}{2.007051in}}%
\pgfpathmoveto{\pgfqpoint{1.885257in}{2.007051in}}%
\pgfpathlineto{\pgfqpoint{1.885257in}{2.007051in}}%
\pgfpathlineto{\pgfqpoint{1.885257in}{2.010001in}}%
\pgfpathlineto{\pgfqpoint{1.889798in}{2.010001in}}%
\pgfpathlineto{\pgfqpoint{1.889798in}{2.007051in}}%
\pgfpathmoveto{\pgfqpoint{1.889798in}{2.007051in}}%
\pgfpathlineto{\pgfqpoint{1.889798in}{2.007051in}}%
\pgfpathlineto{\pgfqpoint{1.889798in}{2.010001in}}%
\pgfpathlineto{\pgfqpoint{1.894339in}{2.010001in}}%
\pgfpathlineto{\pgfqpoint{1.894339in}{2.007051in}}%
\pgfpathmoveto{\pgfqpoint{1.894339in}{2.007051in}}%
\pgfpathlineto{\pgfqpoint{1.894339in}{2.007051in}}%
\pgfpathlineto{\pgfqpoint{1.894339in}{2.010001in}}%
\pgfpathlineto{\pgfqpoint{1.898880in}{2.010001in}}%
\pgfpathlineto{\pgfqpoint{1.898880in}{2.007051in}}%
\pgfpathmoveto{\pgfqpoint{1.898880in}{2.007051in}}%
\pgfpathlineto{\pgfqpoint{1.898880in}{2.007051in}}%
\pgfpathlineto{\pgfqpoint{1.898880in}{2.010001in}}%
\pgfpathlineto{\pgfqpoint{1.903421in}{2.010001in}}%
\pgfpathlineto{\pgfqpoint{1.903421in}{2.007051in}}%
\pgfpathmoveto{\pgfqpoint{1.903421in}{2.007051in}}%
\pgfpathlineto{\pgfqpoint{1.903421in}{2.007051in}}%
\pgfpathlineto{\pgfqpoint{1.903421in}{2.010001in}}%
\pgfpathlineto{\pgfqpoint{1.907962in}{2.010001in}}%
\pgfpathlineto{\pgfqpoint{1.907962in}{2.007051in}}%
\pgfpathmoveto{\pgfqpoint{1.907962in}{2.007051in}}%
\pgfpathlineto{\pgfqpoint{1.907962in}{2.007051in}}%
\pgfpathlineto{\pgfqpoint{1.907962in}{2.010001in}}%
\pgfpathlineto{\pgfqpoint{1.912503in}{2.010001in}}%
\pgfpathlineto{\pgfqpoint{1.912503in}{2.007051in}}%
\pgfpathmoveto{\pgfqpoint{1.912503in}{2.007051in}}%
\pgfpathlineto{\pgfqpoint{1.912503in}{2.007051in}}%
\pgfpathlineto{\pgfqpoint{1.912503in}{2.010001in}}%
\pgfpathlineto{\pgfqpoint{1.917044in}{2.010001in}}%
\pgfpathlineto{\pgfqpoint{1.917044in}{2.007051in}}%
\pgfpathmoveto{\pgfqpoint{1.917044in}{2.007051in}}%
\pgfpathlineto{\pgfqpoint{1.917044in}{2.007051in}}%
\pgfpathlineto{\pgfqpoint{1.917044in}{2.010001in}}%
\pgfpathlineto{\pgfqpoint{1.921585in}{2.010001in}}%
\pgfpathlineto{\pgfqpoint{1.921585in}{2.007051in}}%
\pgfpathmoveto{\pgfqpoint{1.921585in}{2.007051in}}%
\pgfpathlineto{\pgfqpoint{1.921585in}{2.007051in}}%
\pgfpathlineto{\pgfqpoint{1.921585in}{2.010001in}}%
\pgfpathlineto{\pgfqpoint{1.926126in}{2.010001in}}%
\pgfpathlineto{\pgfqpoint{1.926126in}{2.007051in}}%
\pgfpathmoveto{\pgfqpoint{1.926126in}{2.007051in}}%
\pgfpathlineto{\pgfqpoint{1.926126in}{2.007051in}}%
\pgfpathlineto{\pgfqpoint{1.926126in}{2.010001in}}%
\pgfpathlineto{\pgfqpoint{1.930667in}{2.010001in}}%
\pgfpathlineto{\pgfqpoint{1.930667in}{2.007051in}}%
\pgfpathmoveto{\pgfqpoint{1.930667in}{2.007051in}}%
\pgfpathlineto{\pgfqpoint{1.930667in}{2.007051in}}%
\pgfpathlineto{\pgfqpoint{1.930667in}{2.010001in}}%
\pgfpathlineto{\pgfqpoint{1.935207in}{2.010001in}}%
\pgfpathlineto{\pgfqpoint{1.935207in}{2.007051in}}%
\pgfpathmoveto{\pgfqpoint{1.935207in}{2.007051in}}%
\pgfpathlineto{\pgfqpoint{1.935207in}{2.007051in}}%
\pgfpathlineto{\pgfqpoint{1.935207in}{2.010001in}}%
\pgfpathlineto{\pgfqpoint{1.939748in}{2.010001in}}%
\pgfpathlineto{\pgfqpoint{1.939748in}{2.007051in}}%
\pgfpathmoveto{\pgfqpoint{1.939748in}{2.007051in}}%
\pgfpathlineto{\pgfqpoint{1.939748in}{2.007051in}}%
\pgfpathlineto{\pgfqpoint{1.939748in}{2.010001in}}%
\pgfpathlineto{\pgfqpoint{1.944289in}{2.010001in}}%
\pgfpathlineto{\pgfqpoint{1.944289in}{2.007051in}}%
\pgfpathmoveto{\pgfqpoint{1.944289in}{2.007051in}}%
\pgfpathlineto{\pgfqpoint{1.944289in}{2.007051in}}%
\pgfpathlineto{\pgfqpoint{1.944289in}{2.010001in}}%
\pgfpathlineto{\pgfqpoint{1.948830in}{2.010001in}}%
\pgfpathlineto{\pgfqpoint{1.948830in}{2.007051in}}%
\pgfpathmoveto{\pgfqpoint{1.948830in}{2.007051in}}%
\pgfpathlineto{\pgfqpoint{1.948830in}{2.007051in}}%
\pgfpathlineto{\pgfqpoint{1.948830in}{2.010001in}}%
\pgfpathlineto{\pgfqpoint{1.953371in}{2.010001in}}%
\pgfpathlineto{\pgfqpoint{1.953371in}{2.007051in}}%
\pgfpathmoveto{\pgfqpoint{1.953371in}{2.007051in}}%
\pgfpathlineto{\pgfqpoint{1.953371in}{2.007051in}}%
\pgfpathlineto{\pgfqpoint{1.953371in}{2.010001in}}%
\pgfpathlineto{\pgfqpoint{1.957912in}{2.010001in}}%
\pgfpathlineto{\pgfqpoint{1.957912in}{2.007051in}}%
\pgfpathmoveto{\pgfqpoint{1.957912in}{2.007051in}}%
\pgfpathlineto{\pgfqpoint{1.957912in}{2.007051in}}%
\pgfpathlineto{\pgfqpoint{1.957912in}{2.010001in}}%
\pgfpathlineto{\pgfqpoint{1.962453in}{2.010001in}}%
\pgfpathlineto{\pgfqpoint{1.962453in}{2.007051in}}%
\pgfpathmoveto{\pgfqpoint{1.962453in}{2.007051in}}%
\pgfpathlineto{\pgfqpoint{1.962453in}{2.007051in}}%
\pgfpathlineto{\pgfqpoint{1.962453in}{2.010001in}}%
\pgfpathlineto{\pgfqpoint{1.966994in}{2.010001in}}%
\pgfpathlineto{\pgfqpoint{1.966994in}{2.007051in}}%
\pgfpathmoveto{\pgfqpoint{1.966994in}{2.007051in}}%
\pgfpathlineto{\pgfqpoint{1.966994in}{2.007051in}}%
\pgfpathlineto{\pgfqpoint{1.966994in}{2.010001in}}%
\pgfpathlineto{\pgfqpoint{1.971534in}{2.010001in}}%
\pgfpathlineto{\pgfqpoint{1.971534in}{2.007051in}}%
\pgfpathmoveto{\pgfqpoint{1.971534in}{2.007051in}}%
\pgfpathlineto{\pgfqpoint{1.971534in}{2.007051in}}%
\pgfpathlineto{\pgfqpoint{1.971534in}{2.010001in}}%
\pgfpathlineto{\pgfqpoint{1.976075in}{2.010001in}}%
\pgfpathlineto{\pgfqpoint{1.976075in}{2.007051in}}%
\pgfpathmoveto{\pgfqpoint{1.976075in}{2.007051in}}%
\pgfpathlineto{\pgfqpoint{1.976075in}{2.007051in}}%
\pgfpathlineto{\pgfqpoint{1.976075in}{2.010001in}}%
\pgfpathlineto{\pgfqpoint{1.980616in}{2.010001in}}%
\pgfpathlineto{\pgfqpoint{1.980616in}{2.007051in}}%
\pgfpathmoveto{\pgfqpoint{1.980616in}{2.007051in}}%
\pgfpathlineto{\pgfqpoint{1.980616in}{2.007051in}}%
\pgfpathlineto{\pgfqpoint{1.980616in}{2.010001in}}%
\pgfpathlineto{\pgfqpoint{1.985157in}{2.010001in}}%
\pgfpathlineto{\pgfqpoint{1.985157in}{2.007051in}}%
\pgfpathmoveto{\pgfqpoint{1.985157in}{2.007051in}}%
\pgfpathlineto{\pgfqpoint{1.985157in}{2.007051in}}%
\pgfpathlineto{\pgfqpoint{1.985157in}{2.010001in}}%
\pgfpathlineto{\pgfqpoint{1.989698in}{2.010001in}}%
\pgfpathlineto{\pgfqpoint{1.989698in}{2.007051in}}%
\pgfpathmoveto{\pgfqpoint{1.989698in}{2.007051in}}%
\pgfpathlineto{\pgfqpoint{1.989698in}{2.007051in}}%
\pgfpathlineto{\pgfqpoint{1.989698in}{2.010001in}}%
\pgfpathlineto{\pgfqpoint{1.994239in}{2.010001in}}%
\pgfpathlineto{\pgfqpoint{1.994239in}{2.007051in}}%
\pgfpathmoveto{\pgfqpoint{1.994239in}{2.007051in}}%
\pgfpathlineto{\pgfqpoint{1.994239in}{2.007051in}}%
\pgfpathlineto{\pgfqpoint{1.994239in}{2.010001in}}%
\pgfpathlineto{\pgfqpoint{1.998780in}{2.010001in}}%
\pgfpathlineto{\pgfqpoint{1.998780in}{2.007051in}}%
\pgfpathmoveto{\pgfqpoint{1.998780in}{2.007051in}}%
\pgfpathlineto{\pgfqpoint{1.998780in}{2.007051in}}%
\pgfpathlineto{\pgfqpoint{1.998780in}{2.010001in}}%
\pgfpathlineto{\pgfqpoint{2.003321in}{2.010001in}}%
\pgfpathlineto{\pgfqpoint{2.003321in}{2.007051in}}%
\pgfpathmoveto{\pgfqpoint{2.003321in}{2.007051in}}%
\pgfpathlineto{\pgfqpoint{2.003321in}{2.007051in}}%
\pgfpathlineto{\pgfqpoint{2.003321in}{2.010001in}}%
\pgfpathlineto{\pgfqpoint{2.007861in}{2.010001in}}%
\pgfpathlineto{\pgfqpoint{2.007861in}{2.007051in}}%
\pgfpathmoveto{\pgfqpoint{2.007861in}{2.007051in}}%
\pgfpathlineto{\pgfqpoint{2.007861in}{2.007051in}}%
\pgfpathlineto{\pgfqpoint{2.007861in}{2.010001in}}%
\pgfpathlineto{\pgfqpoint{2.012402in}{2.010001in}}%
\pgfpathlineto{\pgfqpoint{2.012402in}{2.007051in}}%
\pgfpathmoveto{\pgfqpoint{2.012402in}{2.007051in}}%
\pgfpathlineto{\pgfqpoint{2.012402in}{2.007051in}}%
\pgfpathlineto{\pgfqpoint{2.012402in}{2.010001in}}%
\pgfpathlineto{\pgfqpoint{2.016943in}{2.010001in}}%
\pgfpathlineto{\pgfqpoint{2.016943in}{2.007051in}}%
\pgfpathmoveto{\pgfqpoint{2.016943in}{2.007051in}}%
\pgfpathlineto{\pgfqpoint{2.016943in}{2.007051in}}%
\pgfpathlineto{\pgfqpoint{2.016943in}{2.010001in}}%
\pgfpathlineto{\pgfqpoint{2.021484in}{2.010001in}}%
\pgfpathlineto{\pgfqpoint{2.021484in}{2.007051in}}%
\pgfpathmoveto{\pgfqpoint{2.021484in}{2.007051in}}%
\pgfpathlineto{\pgfqpoint{2.021484in}{2.007051in}}%
\pgfpathlineto{\pgfqpoint{2.021484in}{2.010001in}}%
\pgfpathlineto{\pgfqpoint{2.026025in}{2.010001in}}%
\pgfpathlineto{\pgfqpoint{2.026025in}{2.007051in}}%
\pgfpathmoveto{\pgfqpoint{2.026025in}{2.007051in}}%
\pgfpathlineto{\pgfqpoint{2.026025in}{2.007051in}}%
\pgfpathlineto{\pgfqpoint{2.026025in}{2.010001in}}%
\pgfpathlineto{\pgfqpoint{2.030566in}{2.010001in}}%
\pgfpathlineto{\pgfqpoint{2.030566in}{2.007051in}}%
\pgfpathmoveto{\pgfqpoint{2.030566in}{2.007051in}}%
\pgfpathlineto{\pgfqpoint{2.030566in}{2.007051in}}%
\pgfpathlineto{\pgfqpoint{2.030566in}{2.010001in}}%
\pgfpathlineto{\pgfqpoint{2.035107in}{2.010001in}}%
\pgfpathlineto{\pgfqpoint{2.035107in}{2.007051in}}%
\pgfpathmoveto{\pgfqpoint{2.035107in}{2.007051in}}%
\pgfpathlineto{\pgfqpoint{2.035107in}{2.007051in}}%
\pgfpathlineto{\pgfqpoint{2.035107in}{2.010001in}}%
\pgfpathlineto{\pgfqpoint{2.039648in}{2.010001in}}%
\pgfpathlineto{\pgfqpoint{2.039648in}{2.007051in}}%
\pgfpathmoveto{\pgfqpoint{2.039648in}{2.007051in}}%
\pgfpathlineto{\pgfqpoint{2.039648in}{2.007051in}}%
\pgfpathlineto{\pgfqpoint{2.039648in}{2.010001in}}%
\pgfpathlineto{\pgfqpoint{2.044188in}{2.010001in}}%
\pgfpathlineto{\pgfqpoint{2.044188in}{2.007051in}}%
\pgfpathmoveto{\pgfqpoint{2.044188in}{2.007051in}}%
\pgfpathlineto{\pgfqpoint{2.044188in}{2.007051in}}%
\pgfpathlineto{\pgfqpoint{2.044188in}{2.010001in}}%
\pgfpathlineto{\pgfqpoint{2.048729in}{2.010001in}}%
\pgfpathlineto{\pgfqpoint{2.048729in}{2.007051in}}%
\pgfpathmoveto{\pgfqpoint{2.048729in}{2.007051in}}%
\pgfpathlineto{\pgfqpoint{2.048729in}{2.007051in}}%
\pgfpathlineto{\pgfqpoint{2.048729in}{2.010001in}}%
\pgfpathlineto{\pgfqpoint{2.053270in}{2.010001in}}%
\pgfpathlineto{\pgfqpoint{2.053270in}{2.007051in}}%
\pgfpathmoveto{\pgfqpoint{2.053270in}{2.007051in}}%
\pgfpathlineto{\pgfqpoint{2.053270in}{2.007051in}}%
\pgfpathlineto{\pgfqpoint{2.053270in}{2.010001in}}%
\pgfpathlineto{\pgfqpoint{2.057811in}{2.010001in}}%
\pgfpathlineto{\pgfqpoint{2.057811in}{2.007051in}}%
\pgfpathmoveto{\pgfqpoint{2.057811in}{2.007051in}}%
\pgfpathlineto{\pgfqpoint{2.057811in}{2.007051in}}%
\pgfpathlineto{\pgfqpoint{2.057811in}{2.010001in}}%
\pgfpathlineto{\pgfqpoint{2.062352in}{2.010001in}}%
\pgfpathlineto{\pgfqpoint{2.062352in}{2.007051in}}%
\pgfpathmoveto{\pgfqpoint{2.062352in}{2.007051in}}%
\pgfpathlineto{\pgfqpoint{2.062352in}{2.007051in}}%
\pgfpathlineto{\pgfqpoint{2.062352in}{2.010001in}}%
\pgfpathlineto{\pgfqpoint{2.066893in}{2.010001in}}%
\pgfpathlineto{\pgfqpoint{2.066893in}{2.007051in}}%
\pgfpathmoveto{\pgfqpoint{2.066893in}{2.007051in}}%
\pgfpathlineto{\pgfqpoint{2.066893in}{2.007051in}}%
\pgfpathlineto{\pgfqpoint{2.066893in}{2.010001in}}%
\pgfpathlineto{\pgfqpoint{2.071434in}{2.010001in}}%
\pgfpathlineto{\pgfqpoint{2.071434in}{2.007051in}}%
\pgfpathmoveto{\pgfqpoint{2.071434in}{2.007051in}}%
\pgfpathlineto{\pgfqpoint{2.071434in}{2.007051in}}%
\pgfpathlineto{\pgfqpoint{2.071434in}{2.010001in}}%
\pgfpathlineto{\pgfqpoint{2.075975in}{2.010001in}}%
\pgfpathlineto{\pgfqpoint{2.075975in}{2.007051in}}%
\pgfpathmoveto{\pgfqpoint{2.075975in}{2.007051in}}%
\pgfpathlineto{\pgfqpoint{2.075975in}{2.007051in}}%
\pgfpathlineto{\pgfqpoint{2.075975in}{2.010001in}}%
\pgfpathlineto{\pgfqpoint{2.080516in}{2.010001in}}%
\pgfpathlineto{\pgfqpoint{2.080516in}{2.007051in}}%
\pgfpathmoveto{\pgfqpoint{2.080516in}{2.007051in}}%
\pgfpathlineto{\pgfqpoint{2.080516in}{2.007051in}}%
\pgfpathlineto{\pgfqpoint{2.080516in}{2.010001in}}%
\pgfpathlineto{\pgfqpoint{2.085057in}{2.010001in}}%
\pgfpathlineto{\pgfqpoint{2.085057in}{2.007051in}}%
\pgfpathmoveto{\pgfqpoint{2.085057in}{2.007051in}}%
\pgfpathlineto{\pgfqpoint{2.085057in}{2.007051in}}%
\pgfpathlineto{\pgfqpoint{2.085057in}{2.010001in}}%
\pgfpathlineto{\pgfqpoint{2.089598in}{2.010001in}}%
\pgfpathlineto{\pgfqpoint{2.089598in}{2.007051in}}%
\pgfpathmoveto{\pgfqpoint{2.089598in}{2.007051in}}%
\pgfpathlineto{\pgfqpoint{2.089598in}{2.007051in}}%
\pgfpathlineto{\pgfqpoint{2.089598in}{2.010001in}}%
\pgfpathlineto{\pgfqpoint{2.094139in}{2.010001in}}%
\pgfpathlineto{\pgfqpoint{2.094139in}{2.007051in}}%
\pgfpathmoveto{\pgfqpoint{2.094139in}{2.007051in}}%
\pgfpathlineto{\pgfqpoint{2.094139in}{2.007051in}}%
\pgfpathlineto{\pgfqpoint{2.094139in}{2.010001in}}%
\pgfpathlineto{\pgfqpoint{2.098680in}{2.010001in}}%
\pgfpathlineto{\pgfqpoint{2.098680in}{2.007051in}}%
\pgfpathmoveto{\pgfqpoint{2.098680in}{2.007051in}}%
\pgfpathlineto{\pgfqpoint{2.098680in}{2.007051in}}%
\pgfpathlineto{\pgfqpoint{2.098680in}{2.010001in}}%
\pgfpathlineto{\pgfqpoint{2.103221in}{2.010001in}}%
\pgfpathlineto{\pgfqpoint{2.103221in}{2.007051in}}%
\pgfpathmoveto{\pgfqpoint{2.103221in}{2.007051in}}%
\pgfpathlineto{\pgfqpoint{2.103221in}{2.007051in}}%
\pgfpathlineto{\pgfqpoint{2.103221in}{2.010001in}}%
\pgfpathlineto{\pgfqpoint{2.107762in}{2.010001in}}%
\pgfpathlineto{\pgfqpoint{2.107762in}{2.007051in}}%
\pgfpathmoveto{\pgfqpoint{2.107762in}{2.007051in}}%
\pgfpathlineto{\pgfqpoint{2.107762in}{2.007051in}}%
\pgfpathlineto{\pgfqpoint{2.107762in}{2.010001in}}%
\pgfpathlineto{\pgfqpoint{2.112303in}{2.010001in}}%
\pgfpathlineto{\pgfqpoint{2.112303in}{2.007051in}}%
\pgfpathmoveto{\pgfqpoint{2.112303in}{2.007051in}}%
\pgfpathlineto{\pgfqpoint{2.112303in}{2.007051in}}%
\pgfpathlineto{\pgfqpoint{2.112303in}{2.010001in}}%
\pgfpathlineto{\pgfqpoint{2.116844in}{2.010001in}}%
\pgfpathlineto{\pgfqpoint{2.116844in}{2.007051in}}%
\pgfpathmoveto{\pgfqpoint{2.116844in}{2.007051in}}%
\pgfpathlineto{\pgfqpoint{2.116844in}{2.007051in}}%
\pgfpathlineto{\pgfqpoint{2.116844in}{2.010001in}}%
\pgfpathlineto{\pgfqpoint{2.121385in}{2.010001in}}%
\pgfpathlineto{\pgfqpoint{2.121385in}{2.007051in}}%
\pgfpathmoveto{\pgfqpoint{2.121385in}{2.007051in}}%
\pgfpathlineto{\pgfqpoint{2.121385in}{2.007051in}}%
\pgfpathlineto{\pgfqpoint{2.121385in}{2.010001in}}%
\pgfpathlineto{\pgfqpoint{2.125926in}{2.010001in}}%
\pgfpathlineto{\pgfqpoint{2.125926in}{2.007051in}}%
\pgfpathmoveto{\pgfqpoint{2.125926in}{2.007051in}}%
\pgfpathlineto{\pgfqpoint{2.125926in}{2.007051in}}%
\pgfpathlineto{\pgfqpoint{2.125926in}{2.010001in}}%
\pgfpathlineto{\pgfqpoint{2.130467in}{2.010001in}}%
\pgfpathlineto{\pgfqpoint{2.130467in}{2.007051in}}%
\pgfpathmoveto{\pgfqpoint{2.130467in}{2.007051in}}%
\pgfpathlineto{\pgfqpoint{2.130467in}{2.007051in}}%
\pgfpathlineto{\pgfqpoint{2.130467in}{2.010001in}}%
\pgfpathlineto{\pgfqpoint{2.135008in}{2.010001in}}%
\pgfpathlineto{\pgfqpoint{2.135008in}{2.007051in}}%
\pgfpathmoveto{\pgfqpoint{2.135008in}{2.007051in}}%
\pgfpathlineto{\pgfqpoint{2.135008in}{2.007051in}}%
\pgfpathlineto{\pgfqpoint{2.135008in}{2.010001in}}%
\pgfpathlineto{\pgfqpoint{2.139549in}{2.010001in}}%
\pgfpathlineto{\pgfqpoint{2.139549in}{2.007051in}}%
\pgfpathmoveto{\pgfqpoint{2.139549in}{2.007051in}}%
\pgfpathlineto{\pgfqpoint{2.139549in}{2.007051in}}%
\pgfpathlineto{\pgfqpoint{2.139549in}{2.010001in}}%
\pgfpathlineto{\pgfqpoint{2.144090in}{2.010001in}}%
\pgfpathlineto{\pgfqpoint{2.144090in}{2.007051in}}%
\pgfpathmoveto{\pgfqpoint{2.144090in}{2.007051in}}%
\pgfpathlineto{\pgfqpoint{2.144090in}{2.007051in}}%
\pgfpathlineto{\pgfqpoint{2.144090in}{2.010001in}}%
\pgfpathlineto{\pgfqpoint{2.148631in}{2.010001in}}%
\pgfpathlineto{\pgfqpoint{2.148631in}{2.007051in}}%
\pgfpathmoveto{\pgfqpoint{2.148631in}{2.007051in}}%
\pgfpathlineto{\pgfqpoint{2.148631in}{2.007051in}}%
\pgfpathlineto{\pgfqpoint{2.148631in}{2.010001in}}%
\pgfpathlineto{\pgfqpoint{2.153172in}{2.010001in}}%
\pgfpathlineto{\pgfqpoint{2.153172in}{2.007051in}}%
\pgfpathmoveto{\pgfqpoint{2.153172in}{2.007051in}}%
\pgfpathlineto{\pgfqpoint{2.153172in}{2.007051in}}%
\pgfpathlineto{\pgfqpoint{2.153172in}{2.010001in}}%
\pgfpathlineto{\pgfqpoint{2.157713in}{2.010001in}}%
\pgfpathlineto{\pgfqpoint{2.157713in}{2.007051in}}%
\pgfpathmoveto{\pgfqpoint{2.157713in}{2.007051in}}%
\pgfpathlineto{\pgfqpoint{2.157713in}{2.007051in}}%
\pgfpathlineto{\pgfqpoint{2.157713in}{2.010001in}}%
\pgfpathlineto{\pgfqpoint{2.162254in}{2.010001in}}%
\pgfpathlineto{\pgfqpoint{2.162254in}{2.007051in}}%
\pgfpathmoveto{\pgfqpoint{2.162254in}{2.007051in}}%
\pgfpathlineto{\pgfqpoint{2.162254in}{2.007051in}}%
\pgfpathlineto{\pgfqpoint{2.162254in}{2.010001in}}%
\pgfpathlineto{\pgfqpoint{2.166795in}{2.010001in}}%
\pgfpathlineto{\pgfqpoint{2.166795in}{2.007051in}}%
\pgfpathmoveto{\pgfqpoint{2.166795in}{2.007051in}}%
\pgfpathlineto{\pgfqpoint{2.166795in}{2.007051in}}%
\pgfpathlineto{\pgfqpoint{2.166795in}{2.010001in}}%
\pgfpathlineto{\pgfqpoint{2.171336in}{2.010001in}}%
\pgfpathlineto{\pgfqpoint{2.171336in}{2.007051in}}%
\pgfpathmoveto{\pgfqpoint{2.171336in}{2.007051in}}%
\pgfpathlineto{\pgfqpoint{2.171336in}{2.007051in}}%
\pgfpathlineto{\pgfqpoint{2.171336in}{2.010001in}}%
\pgfpathlineto{\pgfqpoint{2.175877in}{2.010001in}}%
\pgfpathlineto{\pgfqpoint{2.175877in}{2.007051in}}%
\pgfpathmoveto{\pgfqpoint{2.175877in}{2.007051in}}%
\pgfpathlineto{\pgfqpoint{2.175877in}{2.007051in}}%
\pgfpathlineto{\pgfqpoint{2.175877in}{2.010001in}}%
\pgfpathlineto{\pgfqpoint{2.180418in}{2.010001in}}%
\pgfpathlineto{\pgfqpoint{2.180418in}{2.007051in}}%
\pgfpathmoveto{\pgfqpoint{2.180418in}{2.007051in}}%
\pgfpathlineto{\pgfqpoint{2.180418in}{2.007051in}}%
\pgfpathlineto{\pgfqpoint{2.180418in}{2.010001in}}%
\pgfpathlineto{\pgfqpoint{2.184959in}{2.010001in}}%
\pgfpathlineto{\pgfqpoint{2.184959in}{2.007051in}}%
\pgfpathmoveto{\pgfqpoint{2.184959in}{2.007051in}}%
\pgfpathlineto{\pgfqpoint{2.184959in}{2.007051in}}%
\pgfpathlineto{\pgfqpoint{2.184959in}{2.010001in}}%
\pgfpathlineto{\pgfqpoint{2.189500in}{2.010001in}}%
\pgfpathlineto{\pgfqpoint{2.189500in}{2.007051in}}%
\pgfpathmoveto{\pgfqpoint{2.189500in}{2.007051in}}%
\pgfpathlineto{\pgfqpoint{2.189500in}{2.007051in}}%
\pgfpathlineto{\pgfqpoint{2.189500in}{2.010001in}}%
\pgfpathlineto{\pgfqpoint{2.194041in}{2.010001in}}%
\pgfpathlineto{\pgfqpoint{2.194041in}{2.007051in}}%
\pgfpathmoveto{\pgfqpoint{2.194041in}{2.007051in}}%
\pgfpathlineto{\pgfqpoint{2.194041in}{2.007051in}}%
\pgfpathlineto{\pgfqpoint{2.194041in}{2.010001in}}%
\pgfpathlineto{\pgfqpoint{2.198582in}{2.010001in}}%
\pgfpathlineto{\pgfqpoint{2.198582in}{2.007051in}}%
\pgfpathmoveto{\pgfqpoint{2.198582in}{2.007051in}}%
\pgfpathlineto{\pgfqpoint{2.198582in}{2.007051in}}%
\pgfpathlineto{\pgfqpoint{2.198582in}{2.010001in}}%
\pgfpathlineto{\pgfqpoint{2.203124in}{2.010001in}}%
\pgfpathlineto{\pgfqpoint{2.203124in}{2.007051in}}%
\pgfpathmoveto{\pgfqpoint{2.203124in}{2.007051in}}%
\pgfpathlineto{\pgfqpoint{2.203124in}{2.007051in}}%
\pgfpathlineto{\pgfqpoint{2.203124in}{2.010001in}}%
\pgfpathlineto{\pgfqpoint{2.207665in}{2.010001in}}%
\pgfpathlineto{\pgfqpoint{2.207665in}{2.007051in}}%
\pgfpathmoveto{\pgfqpoint{2.207665in}{2.007051in}}%
\pgfpathlineto{\pgfqpoint{2.207665in}{2.007051in}}%
\pgfpathlineto{\pgfqpoint{2.207665in}{2.010001in}}%
\pgfpathlineto{\pgfqpoint{2.212206in}{2.010001in}}%
\pgfpathlineto{\pgfqpoint{2.212206in}{2.007051in}}%
\pgfpathmoveto{\pgfqpoint{2.212206in}{2.007051in}}%
\pgfpathlineto{\pgfqpoint{2.212206in}{2.007051in}}%
\pgfpathlineto{\pgfqpoint{2.212206in}{2.010001in}}%
\pgfpathlineto{\pgfqpoint{2.216747in}{2.010001in}}%
\pgfpathlineto{\pgfqpoint{2.216747in}{2.007051in}}%
\pgfpathmoveto{\pgfqpoint{2.216747in}{2.007051in}}%
\pgfpathlineto{\pgfqpoint{2.216747in}{2.007051in}}%
\pgfpathlineto{\pgfqpoint{2.216747in}{2.010001in}}%
\pgfpathlineto{\pgfqpoint{2.221288in}{2.010001in}}%
\pgfpathlineto{\pgfqpoint{2.221288in}{2.007051in}}%
\pgfpathmoveto{\pgfqpoint{2.221288in}{2.007051in}}%
\pgfpathlineto{\pgfqpoint{2.221288in}{2.007051in}}%
\pgfpathlineto{\pgfqpoint{2.221288in}{2.010001in}}%
\pgfpathlineto{\pgfqpoint{2.225829in}{2.010001in}}%
\pgfpathlineto{\pgfqpoint{2.225829in}{2.007051in}}%
\pgfpathmoveto{\pgfqpoint{2.225829in}{2.007051in}}%
\pgfpathlineto{\pgfqpoint{2.225829in}{2.007051in}}%
\pgfpathlineto{\pgfqpoint{2.225829in}{2.010001in}}%
\pgfpathlineto{\pgfqpoint{2.230370in}{2.010001in}}%
\pgfpathlineto{\pgfqpoint{2.230370in}{2.007051in}}%
\pgfpathmoveto{\pgfqpoint{2.230370in}{2.007051in}}%
\pgfpathlineto{\pgfqpoint{2.230370in}{2.007051in}}%
\pgfpathlineto{\pgfqpoint{2.230370in}{2.010001in}}%
\pgfpathlineto{\pgfqpoint{2.234911in}{2.010001in}}%
\pgfpathlineto{\pgfqpoint{2.234911in}{2.007051in}}%
\pgfpathmoveto{\pgfqpoint{2.234911in}{2.007051in}}%
\pgfpathlineto{\pgfqpoint{2.234911in}{2.007051in}}%
\pgfpathlineto{\pgfqpoint{2.234911in}{2.010001in}}%
\pgfpathlineto{\pgfqpoint{2.239453in}{2.010001in}}%
\pgfpathlineto{\pgfqpoint{2.239453in}{2.007051in}}%
\pgfpathmoveto{\pgfqpoint{2.239453in}{2.007051in}}%
\pgfpathlineto{\pgfqpoint{2.239453in}{2.007051in}}%
\pgfpathlineto{\pgfqpoint{2.239453in}{2.010001in}}%
\pgfpathlineto{\pgfqpoint{2.243994in}{2.010001in}}%
\pgfpathlineto{\pgfqpoint{2.243994in}{2.007051in}}%
\pgfpathmoveto{\pgfqpoint{2.243994in}{2.007051in}}%
\pgfpathlineto{\pgfqpoint{2.243994in}{2.007051in}}%
\pgfpathlineto{\pgfqpoint{2.243994in}{2.010001in}}%
\pgfpathlineto{\pgfqpoint{2.248535in}{2.010001in}}%
\pgfpathlineto{\pgfqpoint{2.248535in}{2.007051in}}%
\pgfpathmoveto{\pgfqpoint{2.248535in}{2.007051in}}%
\pgfpathlineto{\pgfqpoint{2.248535in}{2.007051in}}%
\pgfpathlineto{\pgfqpoint{2.248535in}{2.010001in}}%
\pgfpathlineto{\pgfqpoint{2.253076in}{2.010001in}}%
\pgfpathlineto{\pgfqpoint{2.253076in}{2.007051in}}%
\pgfpathmoveto{\pgfqpoint{2.253076in}{2.007051in}}%
\pgfpathlineto{\pgfqpoint{2.253076in}{2.007051in}}%
\pgfpathlineto{\pgfqpoint{2.253076in}{2.010001in}}%
\pgfpathlineto{\pgfqpoint{2.257617in}{2.010001in}}%
\pgfpathlineto{\pgfqpoint{2.257617in}{2.007051in}}%
\pgfpathmoveto{\pgfqpoint{2.257617in}{2.007051in}}%
\pgfpathlineto{\pgfqpoint{2.257617in}{2.007051in}}%
\pgfpathlineto{\pgfqpoint{2.257617in}{2.010001in}}%
\pgfpathlineto{\pgfqpoint{2.262158in}{2.010001in}}%
\pgfpathlineto{\pgfqpoint{2.262158in}{2.007051in}}%
\pgfpathmoveto{\pgfqpoint{2.262158in}{2.007051in}}%
\pgfpathlineto{\pgfqpoint{2.262158in}{2.007051in}}%
\pgfpathlineto{\pgfqpoint{2.262158in}{2.010001in}}%
\pgfpathlineto{\pgfqpoint{2.266699in}{2.010001in}}%
\pgfpathlineto{\pgfqpoint{2.266699in}{2.007051in}}%
\pgfpathmoveto{\pgfqpoint{2.266699in}{2.007051in}}%
\pgfpathlineto{\pgfqpoint{2.266699in}{2.007051in}}%
\pgfpathlineto{\pgfqpoint{2.266699in}{2.010001in}}%
\pgfpathlineto{\pgfqpoint{2.271241in}{2.010001in}}%
\pgfpathlineto{\pgfqpoint{2.271241in}{2.007051in}}%
\pgfpathmoveto{\pgfqpoint{2.271241in}{2.007051in}}%
\pgfpathlineto{\pgfqpoint{2.271241in}{2.007051in}}%
\pgfpathlineto{\pgfqpoint{2.271241in}{2.010001in}}%
\pgfpathlineto{\pgfqpoint{2.275782in}{2.010001in}}%
\pgfpathlineto{\pgfqpoint{2.275782in}{2.007051in}}%
\pgfpathmoveto{\pgfqpoint{2.275782in}{2.007051in}}%
\pgfpathlineto{\pgfqpoint{2.275782in}{2.007051in}}%
\pgfpathlineto{\pgfqpoint{2.275782in}{2.010001in}}%
\pgfpathlineto{\pgfqpoint{2.280323in}{2.010001in}}%
\pgfpathlineto{\pgfqpoint{2.280323in}{2.007051in}}%
\pgfpathmoveto{\pgfqpoint{2.280323in}{2.007051in}}%
\pgfpathlineto{\pgfqpoint{2.280323in}{2.007051in}}%
\pgfpathlineto{\pgfqpoint{2.280323in}{2.010001in}}%
\pgfpathlineto{\pgfqpoint{2.284864in}{2.010001in}}%
\pgfpathlineto{\pgfqpoint{2.284864in}{2.007051in}}%
\pgfpathmoveto{\pgfqpoint{2.284864in}{2.007051in}}%
\pgfpathlineto{\pgfqpoint{2.284864in}{2.007051in}}%
\pgfpathlineto{\pgfqpoint{2.284864in}{2.010001in}}%
\pgfpathlineto{\pgfqpoint{2.289405in}{2.010001in}}%
\pgfpathlineto{\pgfqpoint{2.289405in}{2.007051in}}%
\pgfpathmoveto{\pgfqpoint{2.289405in}{2.007051in}}%
\pgfpathlineto{\pgfqpoint{2.289405in}{2.007051in}}%
\pgfpathlineto{\pgfqpoint{2.289405in}{2.010001in}}%
\pgfpathlineto{\pgfqpoint{2.293946in}{2.010001in}}%
\pgfpathlineto{\pgfqpoint{2.293946in}{2.007051in}}%
\pgfpathmoveto{\pgfqpoint{2.293946in}{2.007051in}}%
\pgfpathlineto{\pgfqpoint{2.293946in}{2.007051in}}%
\pgfpathlineto{\pgfqpoint{2.293946in}{2.010001in}}%
\pgfpathlineto{\pgfqpoint{2.298487in}{2.010001in}}%
\pgfpathlineto{\pgfqpoint{2.298487in}{2.007051in}}%
\pgfpathmoveto{\pgfqpoint{2.298487in}{2.007051in}}%
\pgfpathlineto{\pgfqpoint{2.298487in}{2.007051in}}%
\pgfpathlineto{\pgfqpoint{2.298487in}{2.010001in}}%
\pgfpathlineto{\pgfqpoint{2.303029in}{2.010001in}}%
\pgfpathlineto{\pgfqpoint{2.303029in}{2.007051in}}%
\pgfpathmoveto{\pgfqpoint{2.303029in}{2.007051in}}%
\pgfpathlineto{\pgfqpoint{2.303029in}{2.007051in}}%
\pgfpathlineto{\pgfqpoint{2.303029in}{2.010001in}}%
\pgfpathlineto{\pgfqpoint{2.307570in}{2.010001in}}%
\pgfpathlineto{\pgfqpoint{2.307570in}{2.007051in}}%
\pgfpathmoveto{\pgfqpoint{2.307570in}{2.007051in}}%
\pgfpathlineto{\pgfqpoint{2.307570in}{2.007051in}}%
\pgfpathlineto{\pgfqpoint{2.307570in}{2.010001in}}%
\pgfpathlineto{\pgfqpoint{2.312111in}{2.010001in}}%
\pgfpathlineto{\pgfqpoint{2.312111in}{2.007051in}}%
\pgfpathmoveto{\pgfqpoint{2.312111in}{2.007051in}}%
\pgfpathlineto{\pgfqpoint{2.312111in}{2.007051in}}%
\pgfpathlineto{\pgfqpoint{2.312111in}{2.010001in}}%
\pgfpathlineto{\pgfqpoint{2.316652in}{2.010001in}}%
\pgfpathlineto{\pgfqpoint{2.316652in}{2.007051in}}%
\pgfpathmoveto{\pgfqpoint{2.316652in}{2.007051in}}%
\pgfpathlineto{\pgfqpoint{2.316652in}{2.007051in}}%
\pgfpathlineto{\pgfqpoint{2.316652in}{2.010001in}}%
\pgfpathlineto{\pgfqpoint{2.321193in}{2.010001in}}%
\pgfpathlineto{\pgfqpoint{2.321193in}{2.007051in}}%
\pgfpathmoveto{\pgfqpoint{2.321193in}{2.007051in}}%
\pgfpathlineto{\pgfqpoint{2.321193in}{2.007051in}}%
\pgfpathlineto{\pgfqpoint{2.321193in}{2.010001in}}%
\pgfpathlineto{\pgfqpoint{2.325734in}{2.010001in}}%
\pgfpathlineto{\pgfqpoint{2.325734in}{2.007051in}}%
\pgfpathmoveto{\pgfqpoint{2.325734in}{2.007051in}}%
\pgfpathlineto{\pgfqpoint{2.325734in}{2.007051in}}%
\pgfpathlineto{\pgfqpoint{2.325734in}{2.010001in}}%
\pgfpathlineto{\pgfqpoint{2.330275in}{2.010001in}}%
\pgfpathlineto{\pgfqpoint{2.330275in}{2.007051in}}%
\pgfpathmoveto{\pgfqpoint{2.330275in}{2.007051in}}%
\pgfpathlineto{\pgfqpoint{2.330275in}{2.007051in}}%
\pgfpathlineto{\pgfqpoint{2.330275in}{2.010001in}}%
\pgfpathlineto{\pgfqpoint{2.334817in}{2.010001in}}%
\pgfpathlineto{\pgfqpoint{2.334817in}{2.007051in}}%
\pgfpathmoveto{\pgfqpoint{2.334817in}{2.007051in}}%
\pgfpathlineto{\pgfqpoint{2.334817in}{2.007051in}}%
\pgfpathlineto{\pgfqpoint{2.334817in}{2.010001in}}%
\pgfpathlineto{\pgfqpoint{2.339358in}{2.010001in}}%
\pgfpathlineto{\pgfqpoint{2.339358in}{2.007051in}}%
\pgfpathmoveto{\pgfqpoint{2.339358in}{2.007051in}}%
\pgfpathlineto{\pgfqpoint{2.339358in}{2.007051in}}%
\pgfpathlineto{\pgfqpoint{2.339358in}{2.010001in}}%
\pgfpathlineto{\pgfqpoint{2.343899in}{2.010001in}}%
\pgfpathlineto{\pgfqpoint{2.343899in}{2.007051in}}%
\pgfpathmoveto{\pgfqpoint{2.343899in}{2.007051in}}%
\pgfpathlineto{\pgfqpoint{2.343899in}{2.007051in}}%
\pgfpathlineto{\pgfqpoint{2.343899in}{2.010001in}}%
\pgfpathlineto{\pgfqpoint{2.348440in}{2.010001in}}%
\pgfpathlineto{\pgfqpoint{2.348440in}{2.007051in}}%
\pgfpathmoveto{\pgfqpoint{2.348440in}{2.007051in}}%
\pgfpathlineto{\pgfqpoint{2.348440in}{2.007051in}}%
\pgfpathlineto{\pgfqpoint{2.348440in}{2.010001in}}%
\pgfpathlineto{\pgfqpoint{2.352981in}{2.010001in}}%
\pgfpathlineto{\pgfqpoint{2.352981in}{2.007051in}}%
\pgfpathmoveto{\pgfqpoint{2.352981in}{2.007051in}}%
\pgfpathlineto{\pgfqpoint{2.352981in}{2.007051in}}%
\pgfpathlineto{\pgfqpoint{2.352981in}{2.010001in}}%
\pgfpathlineto{\pgfqpoint{2.357522in}{2.010001in}}%
\pgfpathlineto{\pgfqpoint{2.357522in}{2.007051in}}%
\pgfpathmoveto{\pgfqpoint{2.357522in}{2.007051in}}%
\pgfpathlineto{\pgfqpoint{2.357522in}{2.007051in}}%
\pgfpathlineto{\pgfqpoint{2.357522in}{2.010001in}}%
\pgfpathlineto{\pgfqpoint{2.362063in}{2.010001in}}%
\pgfpathlineto{\pgfqpoint{2.362063in}{2.007051in}}%
\pgfpathmoveto{\pgfqpoint{2.362063in}{2.007051in}}%
\pgfpathlineto{\pgfqpoint{2.362063in}{2.007051in}}%
\pgfpathlineto{\pgfqpoint{2.362063in}{2.010001in}}%
\pgfpathlineto{\pgfqpoint{2.366604in}{2.010001in}}%
\pgfpathlineto{\pgfqpoint{2.366604in}{2.007051in}}%
\pgfpathmoveto{\pgfqpoint{2.366604in}{2.007051in}}%
\pgfpathlineto{\pgfqpoint{2.366604in}{2.007051in}}%
\pgfpathlineto{\pgfqpoint{2.366604in}{2.010001in}}%
\pgfpathlineto{\pgfqpoint{2.371145in}{2.010001in}}%
\pgfpathlineto{\pgfqpoint{2.371145in}{2.007051in}}%
\pgfpathmoveto{\pgfqpoint{2.371145in}{2.007051in}}%
\pgfpathlineto{\pgfqpoint{2.371145in}{2.007051in}}%
\pgfpathlineto{\pgfqpoint{2.371145in}{2.010001in}}%
\pgfpathlineto{\pgfqpoint{2.375686in}{2.010001in}}%
\pgfpathlineto{\pgfqpoint{2.375686in}{2.007051in}}%
\pgfpathmoveto{\pgfqpoint{2.375686in}{2.007051in}}%
\pgfpathlineto{\pgfqpoint{2.375686in}{2.007051in}}%
\pgfpathlineto{\pgfqpoint{2.375686in}{2.010001in}}%
\pgfpathlineto{\pgfqpoint{2.380227in}{2.010001in}}%
\pgfpathlineto{\pgfqpoint{2.380227in}{2.007051in}}%
\pgfpathmoveto{\pgfqpoint{2.380227in}{2.007051in}}%
\pgfpathlineto{\pgfqpoint{2.380227in}{2.007051in}}%
\pgfpathlineto{\pgfqpoint{2.380227in}{2.010001in}}%
\pgfpathlineto{\pgfqpoint{2.384768in}{2.010001in}}%
\pgfpathlineto{\pgfqpoint{2.384768in}{2.007051in}}%
\pgfpathmoveto{\pgfqpoint{2.384768in}{2.007051in}}%
\pgfpathlineto{\pgfqpoint{2.384768in}{2.007051in}}%
\pgfpathlineto{\pgfqpoint{2.384768in}{2.010001in}}%
\pgfpathlineto{\pgfqpoint{2.389309in}{2.010001in}}%
\pgfpathlineto{\pgfqpoint{2.389309in}{2.007051in}}%
\pgfpathmoveto{\pgfqpoint{2.389309in}{2.007051in}}%
\pgfpathlineto{\pgfqpoint{2.389309in}{2.007051in}}%
\pgfpathlineto{\pgfqpoint{2.389309in}{2.010001in}}%
\pgfpathlineto{\pgfqpoint{2.393850in}{2.010001in}}%
\pgfpathlineto{\pgfqpoint{2.393850in}{2.007051in}}%
\pgfpathmoveto{\pgfqpoint{2.393850in}{2.007051in}}%
\pgfpathlineto{\pgfqpoint{2.393850in}{2.007051in}}%
\pgfpathlineto{\pgfqpoint{2.393850in}{2.010001in}}%
\pgfpathlineto{\pgfqpoint{2.398391in}{2.010001in}}%
\pgfpathlineto{\pgfqpoint{2.398391in}{2.007051in}}%
\pgfpathmoveto{\pgfqpoint{2.398391in}{2.007051in}}%
\pgfpathlineto{\pgfqpoint{2.398391in}{2.007051in}}%
\pgfpathlineto{\pgfqpoint{2.398391in}{2.010001in}}%
\pgfpathlineto{\pgfqpoint{2.402932in}{2.010001in}}%
\pgfpathlineto{\pgfqpoint{2.402932in}{2.007051in}}%
\pgfpathmoveto{\pgfqpoint{2.402932in}{2.007051in}}%
\pgfpathlineto{\pgfqpoint{2.402932in}{2.007051in}}%
\pgfpathlineto{\pgfqpoint{2.402932in}{2.010001in}}%
\pgfpathlineto{\pgfqpoint{2.407473in}{2.010001in}}%
\pgfpathlineto{\pgfqpoint{2.407473in}{2.007051in}}%
\pgfpathmoveto{\pgfqpoint{2.407473in}{2.007051in}}%
\pgfpathlineto{\pgfqpoint{2.407473in}{2.007051in}}%
\pgfpathlineto{\pgfqpoint{2.407473in}{2.010001in}}%
\pgfpathlineto{\pgfqpoint{2.412014in}{2.010001in}}%
\pgfpathlineto{\pgfqpoint{2.412014in}{2.007051in}}%
\pgfpathmoveto{\pgfqpoint{2.412014in}{2.007051in}}%
\pgfpathlineto{\pgfqpoint{2.412014in}{2.007051in}}%
\pgfpathlineto{\pgfqpoint{2.412014in}{2.010001in}}%
\pgfpathlineto{\pgfqpoint{2.416555in}{2.010001in}}%
\pgfpathlineto{\pgfqpoint{2.416555in}{2.007051in}}%
\pgfpathmoveto{\pgfqpoint{2.416555in}{2.007051in}}%
\pgfpathlineto{\pgfqpoint{2.416555in}{2.007051in}}%
\pgfpathlineto{\pgfqpoint{2.416555in}{2.010001in}}%
\pgfpathlineto{\pgfqpoint{2.421096in}{2.010001in}}%
\pgfpathlineto{\pgfqpoint{2.421096in}{2.007051in}}%
\pgfpathmoveto{\pgfqpoint{2.421096in}{2.007051in}}%
\pgfpathlineto{\pgfqpoint{2.421096in}{2.007051in}}%
\pgfpathlineto{\pgfqpoint{2.421096in}{2.010001in}}%
\pgfpathlineto{\pgfqpoint{2.425637in}{2.010001in}}%
\pgfpathlineto{\pgfqpoint{2.425637in}{2.007051in}}%
\pgfpathmoveto{\pgfqpoint{2.425637in}{2.007051in}}%
\pgfpathlineto{\pgfqpoint{2.425637in}{2.007051in}}%
\pgfpathlineto{\pgfqpoint{2.425637in}{2.010001in}}%
\pgfpathlineto{\pgfqpoint{2.430178in}{2.010001in}}%
\pgfpathlineto{\pgfqpoint{2.430178in}{2.007051in}}%
\pgfpathmoveto{\pgfqpoint{2.430178in}{2.007051in}}%
\pgfpathlineto{\pgfqpoint{2.430178in}{2.007051in}}%
\pgfpathlineto{\pgfqpoint{2.430178in}{2.010001in}}%
\pgfpathlineto{\pgfqpoint{2.434719in}{2.010001in}}%
\pgfpathlineto{\pgfqpoint{2.434719in}{2.007051in}}%
\pgfpathmoveto{\pgfqpoint{2.434719in}{2.007051in}}%
\pgfpathlineto{\pgfqpoint{2.434719in}{2.007051in}}%
\pgfpathlineto{\pgfqpoint{2.434719in}{2.010001in}}%
\pgfpathlineto{\pgfqpoint{2.439260in}{2.010001in}}%
\pgfpathlineto{\pgfqpoint{2.439260in}{2.007051in}}%
\pgfpathmoveto{\pgfqpoint{2.439260in}{2.007051in}}%
\pgfpathlineto{\pgfqpoint{2.439260in}{2.007051in}}%
\pgfpathlineto{\pgfqpoint{2.439260in}{2.010001in}}%
\pgfpathlineto{\pgfqpoint{2.443801in}{2.010001in}}%
\pgfpathlineto{\pgfqpoint{2.443801in}{2.007051in}}%
\pgfpathmoveto{\pgfqpoint{2.443801in}{2.007051in}}%
\pgfpathlineto{\pgfqpoint{2.443801in}{2.007051in}}%
\pgfpathlineto{\pgfqpoint{2.443801in}{2.010001in}}%
\pgfpathlineto{\pgfqpoint{2.448342in}{2.010001in}}%
\pgfpathlineto{\pgfqpoint{2.448342in}{2.007051in}}%
\pgfpathmoveto{\pgfqpoint{2.448342in}{2.007051in}}%
\pgfpathlineto{\pgfqpoint{2.448342in}{2.007051in}}%
\pgfpathlineto{\pgfqpoint{2.448342in}{2.010001in}}%
\pgfpathlineto{\pgfqpoint{2.452884in}{2.010001in}}%
\pgfpathlineto{\pgfqpoint{2.452884in}{2.007051in}}%
\pgfpathmoveto{\pgfqpoint{2.452884in}{2.007051in}}%
\pgfpathlineto{\pgfqpoint{2.452884in}{2.007051in}}%
\pgfpathlineto{\pgfqpoint{2.452884in}{2.010001in}}%
\pgfpathlineto{\pgfqpoint{2.457425in}{2.010001in}}%
\pgfpathlineto{\pgfqpoint{2.457425in}{2.007051in}}%
\pgfpathmoveto{\pgfqpoint{2.457425in}{2.007051in}}%
\pgfpathlineto{\pgfqpoint{2.457425in}{2.007051in}}%
\pgfpathlineto{\pgfqpoint{2.457425in}{2.010001in}}%
\pgfpathlineto{\pgfqpoint{2.461966in}{2.010001in}}%
\pgfpathlineto{\pgfqpoint{2.461966in}{2.007051in}}%
\pgfpathmoveto{\pgfqpoint{2.461966in}{2.007051in}}%
\pgfpathlineto{\pgfqpoint{2.461966in}{2.007051in}}%
\pgfpathlineto{\pgfqpoint{2.461966in}{2.010001in}}%
\pgfpathlineto{\pgfqpoint{2.466507in}{2.010001in}}%
\pgfpathlineto{\pgfqpoint{2.466507in}{2.007051in}}%
\pgfpathmoveto{\pgfqpoint{2.466507in}{2.007051in}}%
\pgfpathlineto{\pgfqpoint{2.466507in}{2.007051in}}%
\pgfpathlineto{\pgfqpoint{2.466507in}{2.010001in}}%
\pgfpathlineto{\pgfqpoint{2.471048in}{2.010001in}}%
\pgfpathlineto{\pgfqpoint{2.471048in}{2.007051in}}%
\pgfpathmoveto{\pgfqpoint{2.471048in}{2.007051in}}%
\pgfpathlineto{\pgfqpoint{2.471048in}{2.007051in}}%
\pgfpathlineto{\pgfqpoint{2.471048in}{2.010001in}}%
\pgfpathlineto{\pgfqpoint{2.475589in}{2.010001in}}%
\pgfpathlineto{\pgfqpoint{2.475589in}{2.007051in}}%
\pgfpathmoveto{\pgfqpoint{2.475589in}{2.007051in}}%
\pgfpathlineto{\pgfqpoint{2.475589in}{2.007051in}}%
\pgfpathlineto{\pgfqpoint{2.475589in}{2.010001in}}%
\pgfpathlineto{\pgfqpoint{2.480130in}{2.010001in}}%
\pgfpathlineto{\pgfqpoint{2.480130in}{2.007051in}}%
\pgfpathmoveto{\pgfqpoint{2.480130in}{2.007051in}}%
\pgfpathlineto{\pgfqpoint{2.480130in}{2.007051in}}%
\pgfpathlineto{\pgfqpoint{2.480130in}{2.010001in}}%
\pgfpathlineto{\pgfqpoint{2.484671in}{2.010001in}}%
\pgfpathlineto{\pgfqpoint{2.484671in}{2.007051in}}%
\pgfpathmoveto{\pgfqpoint{2.484671in}{2.007051in}}%
\pgfpathlineto{\pgfqpoint{2.484671in}{2.007051in}}%
\pgfpathlineto{\pgfqpoint{2.484671in}{2.010001in}}%
\pgfpathlineto{\pgfqpoint{2.489212in}{2.010001in}}%
\pgfpathlineto{\pgfqpoint{2.489212in}{2.007051in}}%
\pgfpathmoveto{\pgfqpoint{2.489212in}{2.007051in}}%
\pgfpathlineto{\pgfqpoint{2.489212in}{2.007051in}}%
\pgfpathlineto{\pgfqpoint{2.489212in}{2.010001in}}%
\pgfpathlineto{\pgfqpoint{2.493753in}{2.010001in}}%
\pgfpathlineto{\pgfqpoint{2.493753in}{2.007051in}}%
\pgfpathmoveto{\pgfqpoint{2.493753in}{2.007051in}}%
\pgfpathlineto{\pgfqpoint{2.493753in}{2.007051in}}%
\pgfpathlineto{\pgfqpoint{2.493753in}{2.010001in}}%
\pgfpathlineto{\pgfqpoint{2.498294in}{2.010001in}}%
\pgfpathlineto{\pgfqpoint{2.498294in}{2.007051in}}%
\pgfpathmoveto{\pgfqpoint{2.498294in}{2.007051in}}%
\pgfpathlineto{\pgfqpoint{2.498294in}{2.007051in}}%
\pgfpathlineto{\pgfqpoint{2.498294in}{2.010001in}}%
\pgfpathlineto{\pgfqpoint{2.502835in}{2.010001in}}%
\pgfpathlineto{\pgfqpoint{2.502835in}{2.007051in}}%
\pgfpathmoveto{\pgfqpoint{2.502835in}{2.007051in}}%
\pgfpathlineto{\pgfqpoint{2.502835in}{2.007051in}}%
\pgfpathlineto{\pgfqpoint{2.502835in}{2.010001in}}%
\pgfpathlineto{\pgfqpoint{2.507376in}{2.010001in}}%
\pgfpathlineto{\pgfqpoint{2.507376in}{2.007051in}}%
\pgfpathmoveto{\pgfqpoint{2.507376in}{2.007051in}}%
\pgfpathlineto{\pgfqpoint{2.507376in}{2.007051in}}%
\pgfpathlineto{\pgfqpoint{2.507376in}{2.010001in}}%
\pgfpathlineto{\pgfqpoint{2.511917in}{2.010001in}}%
\pgfpathlineto{\pgfqpoint{2.511917in}{2.007051in}}%
\pgfpathmoveto{\pgfqpoint{2.511917in}{2.007051in}}%
\pgfpathlineto{\pgfqpoint{2.511917in}{2.007051in}}%
\pgfpathlineto{\pgfqpoint{2.511917in}{2.010001in}}%
\pgfpathlineto{\pgfqpoint{2.516458in}{2.010001in}}%
\pgfpathlineto{\pgfqpoint{2.516458in}{2.007051in}}%
\pgfpathmoveto{\pgfqpoint{2.516458in}{2.007051in}}%
\pgfpathlineto{\pgfqpoint{2.516458in}{2.007051in}}%
\pgfpathlineto{\pgfqpoint{2.516458in}{2.010001in}}%
\pgfpathlineto{\pgfqpoint{2.520999in}{2.010001in}}%
\pgfpathlineto{\pgfqpoint{2.520999in}{2.007051in}}%
\pgfpathmoveto{\pgfqpoint{2.520999in}{2.007051in}}%
\pgfpathlineto{\pgfqpoint{2.520999in}{2.007051in}}%
\pgfpathlineto{\pgfqpoint{2.520999in}{2.010001in}}%
\pgfpathlineto{\pgfqpoint{2.525540in}{2.010001in}}%
\pgfpathlineto{\pgfqpoint{2.525540in}{2.007051in}}%
\pgfpathmoveto{\pgfqpoint{2.525540in}{2.007051in}}%
\pgfpathlineto{\pgfqpoint{2.525540in}{2.007051in}}%
\pgfpathlineto{\pgfqpoint{2.525540in}{2.010001in}}%
\pgfpathlineto{\pgfqpoint{2.530081in}{2.010001in}}%
\pgfpathlineto{\pgfqpoint{2.530081in}{2.007051in}}%
\pgfpathmoveto{\pgfqpoint{2.530081in}{2.007051in}}%
\pgfpathlineto{\pgfqpoint{2.530081in}{2.007051in}}%
\pgfpathlineto{\pgfqpoint{2.530081in}{2.010001in}}%
\pgfpathlineto{\pgfqpoint{2.534622in}{2.010001in}}%
\pgfpathlineto{\pgfqpoint{2.534622in}{2.007051in}}%
\pgfpathmoveto{\pgfqpoint{2.534622in}{2.007051in}}%
\pgfpathlineto{\pgfqpoint{2.534622in}{2.007051in}}%
\pgfpathlineto{\pgfqpoint{2.534622in}{2.010001in}}%
\pgfpathlineto{\pgfqpoint{2.539163in}{2.010001in}}%
\pgfpathlineto{\pgfqpoint{2.539163in}{2.007051in}}%
\pgfpathmoveto{\pgfqpoint{2.539163in}{2.007051in}}%
\pgfpathlineto{\pgfqpoint{2.539163in}{2.007051in}}%
\pgfpathlineto{\pgfqpoint{2.539163in}{2.010001in}}%
\pgfpathlineto{\pgfqpoint{2.543704in}{2.010001in}}%
\pgfpathlineto{\pgfqpoint{2.543704in}{2.007051in}}%
\pgfpathmoveto{\pgfqpoint{2.543704in}{2.007051in}}%
\pgfpathlineto{\pgfqpoint{2.543704in}{2.007051in}}%
\pgfpathlineto{\pgfqpoint{2.543704in}{2.010001in}}%
\pgfpathlineto{\pgfqpoint{2.548245in}{2.010001in}}%
\pgfpathlineto{\pgfqpoint{2.548245in}{2.007051in}}%
\pgfpathmoveto{\pgfqpoint{2.548245in}{2.007051in}}%
\pgfpathlineto{\pgfqpoint{2.548245in}{2.007051in}}%
\pgfpathlineto{\pgfqpoint{2.548245in}{2.010001in}}%
\pgfpathlineto{\pgfqpoint{2.552786in}{2.010001in}}%
\pgfpathlineto{\pgfqpoint{2.552786in}{2.007051in}}%
\pgfpathmoveto{\pgfqpoint{2.552786in}{2.007051in}}%
\pgfpathlineto{\pgfqpoint{2.552786in}{2.007051in}}%
\pgfpathlineto{\pgfqpoint{2.552786in}{2.010001in}}%
\pgfpathlineto{\pgfqpoint{2.557327in}{2.010001in}}%
\pgfpathlineto{\pgfqpoint{2.557327in}{2.007051in}}%
\pgfpathmoveto{\pgfqpoint{2.557327in}{2.007051in}}%
\pgfpathlineto{\pgfqpoint{2.557327in}{2.007051in}}%
\pgfpathlineto{\pgfqpoint{2.557327in}{2.010001in}}%
\pgfpathlineto{\pgfqpoint{2.561868in}{2.010001in}}%
\pgfpathlineto{\pgfqpoint{2.561868in}{2.007051in}}%
\pgfpathmoveto{\pgfqpoint{2.561868in}{2.007051in}}%
\pgfpathlineto{\pgfqpoint{2.561868in}{2.007051in}}%
\pgfpathlineto{\pgfqpoint{2.561868in}{2.010001in}}%
\pgfpathlineto{\pgfqpoint{2.566410in}{2.010001in}}%
\pgfpathlineto{\pgfqpoint{2.566410in}{2.007051in}}%
\pgfpathmoveto{\pgfqpoint{2.566410in}{2.007051in}}%
\pgfpathlineto{\pgfqpoint{2.566410in}{2.007051in}}%
\pgfpathlineto{\pgfqpoint{2.566410in}{2.010001in}}%
\pgfpathlineto{\pgfqpoint{2.570951in}{2.010001in}}%
\pgfpathlineto{\pgfqpoint{2.570951in}{2.007051in}}%
\pgfpathmoveto{\pgfqpoint{2.570951in}{2.007051in}}%
\pgfpathlineto{\pgfqpoint{2.570951in}{2.007051in}}%
\pgfpathlineto{\pgfqpoint{2.570951in}{2.010001in}}%
\pgfpathlineto{\pgfqpoint{2.575492in}{2.010001in}}%
\pgfpathlineto{\pgfqpoint{2.575492in}{2.007051in}}%
\pgfpathmoveto{\pgfqpoint{2.575492in}{2.007051in}}%
\pgfpathlineto{\pgfqpoint{2.575492in}{2.007051in}}%
\pgfpathlineto{\pgfqpoint{2.575492in}{2.010001in}}%
\pgfpathlineto{\pgfqpoint{2.580033in}{2.010001in}}%
\pgfpathlineto{\pgfqpoint{2.580033in}{2.007051in}}%
\pgfpathmoveto{\pgfqpoint{2.580033in}{2.007051in}}%
\pgfpathlineto{\pgfqpoint{2.580033in}{2.007051in}}%
\pgfpathlineto{\pgfqpoint{2.580033in}{2.010001in}}%
\pgfpathlineto{\pgfqpoint{2.584574in}{2.010001in}}%
\pgfpathlineto{\pgfqpoint{2.584574in}{2.007051in}}%
\pgfpathmoveto{\pgfqpoint{2.584574in}{2.007051in}}%
\pgfpathlineto{\pgfqpoint{2.584574in}{2.007051in}}%
\pgfpathlineto{\pgfqpoint{2.584574in}{2.010001in}}%
\pgfpathlineto{\pgfqpoint{2.589115in}{2.010001in}}%
\pgfpathlineto{\pgfqpoint{2.589115in}{2.007051in}}%
\pgfpathmoveto{\pgfqpoint{2.589115in}{2.007051in}}%
\pgfpathlineto{\pgfqpoint{2.589115in}{2.007051in}}%
\pgfpathlineto{\pgfqpoint{2.589115in}{2.010001in}}%
\pgfpathlineto{\pgfqpoint{2.593656in}{2.010001in}}%
\pgfpathlineto{\pgfqpoint{2.593656in}{2.007051in}}%
\pgfpathmoveto{\pgfqpoint{2.593656in}{2.007051in}}%
\pgfpathlineto{\pgfqpoint{2.593656in}{2.007051in}}%
\pgfpathlineto{\pgfqpoint{2.593656in}{2.010001in}}%
\pgfpathlineto{\pgfqpoint{2.598197in}{2.010001in}}%
\pgfpathlineto{\pgfqpoint{2.598197in}{2.007051in}}%
\pgfpathmoveto{\pgfqpoint{2.598197in}{2.007051in}}%
\pgfpathlineto{\pgfqpoint{2.598197in}{2.007051in}}%
\pgfpathlineto{\pgfqpoint{2.598197in}{2.010001in}}%
\pgfpathlineto{\pgfqpoint{2.602738in}{2.010001in}}%
\pgfpathlineto{\pgfqpoint{2.602738in}{2.007051in}}%
\pgfpathmoveto{\pgfqpoint{2.602738in}{2.007051in}}%
\pgfpathlineto{\pgfqpoint{2.602738in}{2.007051in}}%
\pgfpathlineto{\pgfqpoint{2.602738in}{2.010001in}}%
\pgfpathlineto{\pgfqpoint{2.607279in}{2.010001in}}%
\pgfpathlineto{\pgfqpoint{2.607279in}{2.007051in}}%
\pgfpathmoveto{\pgfqpoint{2.607279in}{2.007051in}}%
\pgfpathlineto{\pgfqpoint{2.607279in}{2.007051in}}%
\pgfpathlineto{\pgfqpoint{2.607279in}{2.010001in}}%
\pgfpathlineto{\pgfqpoint{2.611820in}{2.010001in}}%
\pgfpathlineto{\pgfqpoint{2.611820in}{2.007051in}}%
\pgfpathmoveto{\pgfqpoint{2.611820in}{2.007051in}}%
\pgfpathlineto{\pgfqpoint{2.611820in}{2.007051in}}%
\pgfpathlineto{\pgfqpoint{2.611820in}{2.010001in}}%
\pgfpathlineto{\pgfqpoint{2.616361in}{2.010001in}}%
\pgfpathlineto{\pgfqpoint{2.616361in}{2.007051in}}%
\pgfpathmoveto{\pgfqpoint{2.616361in}{2.007051in}}%
\pgfpathlineto{\pgfqpoint{2.616361in}{2.007051in}}%
\pgfpathlineto{\pgfqpoint{2.616361in}{2.010001in}}%
\pgfpathlineto{\pgfqpoint{2.620902in}{2.010001in}}%
\pgfpathlineto{\pgfqpoint{2.620902in}{2.007051in}}%
\pgfpathmoveto{\pgfqpoint{2.620902in}{2.007051in}}%
\pgfpathlineto{\pgfqpoint{2.620902in}{2.007051in}}%
\pgfpathlineto{\pgfqpoint{2.620902in}{2.010001in}}%
\pgfpathlineto{\pgfqpoint{2.625443in}{2.010001in}}%
\pgfpathlineto{\pgfqpoint{2.625443in}{2.007051in}}%
\pgfpathmoveto{\pgfqpoint{2.625443in}{2.007051in}}%
\pgfpathlineto{\pgfqpoint{2.625443in}{2.007051in}}%
\pgfpathlineto{\pgfqpoint{2.625443in}{2.010001in}}%
\pgfpathlineto{\pgfqpoint{2.629984in}{2.010001in}}%
\pgfpathlineto{\pgfqpoint{2.629984in}{2.007051in}}%
\pgfpathmoveto{\pgfqpoint{2.629984in}{2.007051in}}%
\pgfpathlineto{\pgfqpoint{2.629984in}{2.007051in}}%
\pgfpathlineto{\pgfqpoint{2.629984in}{2.010001in}}%
\pgfpathlineto{\pgfqpoint{2.634525in}{2.010001in}}%
\pgfpathlineto{\pgfqpoint{2.634525in}{2.007051in}}%
\pgfpathmoveto{\pgfqpoint{2.634525in}{2.007051in}}%
\pgfpathlineto{\pgfqpoint{2.634525in}{2.007051in}}%
\pgfpathlineto{\pgfqpoint{2.634525in}{2.010001in}}%
\pgfpathlineto{\pgfqpoint{2.639066in}{2.010001in}}%
\pgfpathlineto{\pgfqpoint{2.639066in}{2.007051in}}%
\pgfpathmoveto{\pgfqpoint{2.639066in}{2.007051in}}%
\pgfpathlineto{\pgfqpoint{2.639066in}{2.007051in}}%
\pgfpathlineto{\pgfqpoint{2.639066in}{2.010001in}}%
\pgfpathlineto{\pgfqpoint{2.643607in}{2.010001in}}%
\pgfpathlineto{\pgfqpoint{2.643607in}{2.007051in}}%
\pgfpathmoveto{\pgfqpoint{2.643607in}{2.007051in}}%
\pgfpathlineto{\pgfqpoint{2.643607in}{2.007051in}}%
\pgfpathlineto{\pgfqpoint{2.643607in}{2.010001in}}%
\pgfpathlineto{\pgfqpoint{2.648148in}{2.010001in}}%
\pgfpathlineto{\pgfqpoint{2.648148in}{2.007051in}}%
\pgfpathmoveto{\pgfqpoint{2.648148in}{2.007051in}}%
\pgfpathlineto{\pgfqpoint{2.648148in}{2.007051in}}%
\pgfpathlineto{\pgfqpoint{2.648148in}{2.010001in}}%
\pgfpathlineto{\pgfqpoint{2.652689in}{2.010001in}}%
\pgfpathlineto{\pgfqpoint{2.652689in}{2.007051in}}%
\pgfpathmoveto{\pgfqpoint{2.652689in}{2.007051in}}%
\pgfpathlineto{\pgfqpoint{2.652689in}{2.007051in}}%
\pgfpathlineto{\pgfqpoint{2.652689in}{2.010001in}}%
\pgfpathlineto{\pgfqpoint{2.657229in}{2.010001in}}%
\pgfpathlineto{\pgfqpoint{2.657229in}{2.007051in}}%
\pgfpathmoveto{\pgfqpoint{2.657229in}{2.007051in}}%
\pgfpathlineto{\pgfqpoint{2.657229in}{2.007051in}}%
\pgfpathlineto{\pgfqpoint{2.657229in}{2.010001in}}%
\pgfpathlineto{\pgfqpoint{2.661770in}{2.010001in}}%
\pgfpathlineto{\pgfqpoint{2.661770in}{2.007051in}}%
\pgfpathmoveto{\pgfqpoint{2.661770in}{2.007051in}}%
\pgfpathlineto{\pgfqpoint{2.661770in}{2.007051in}}%
\pgfpathlineto{\pgfqpoint{2.661770in}{2.010001in}}%
\pgfpathlineto{\pgfqpoint{2.666311in}{2.010001in}}%
\pgfpathlineto{\pgfqpoint{2.666311in}{2.007051in}}%
\pgfpathmoveto{\pgfqpoint{2.666311in}{2.007051in}}%
\pgfpathlineto{\pgfqpoint{2.666311in}{2.007051in}}%
\pgfpathlineto{\pgfqpoint{2.666311in}{2.010001in}}%
\pgfpathlineto{\pgfqpoint{2.670852in}{2.010001in}}%
\pgfpathlineto{\pgfqpoint{2.670852in}{2.007051in}}%
\pgfpathmoveto{\pgfqpoint{2.670852in}{2.007051in}}%
\pgfpathlineto{\pgfqpoint{2.670852in}{2.007051in}}%
\pgfpathlineto{\pgfqpoint{2.670852in}{2.010001in}}%
\pgfpathlineto{\pgfqpoint{2.675392in}{2.010001in}}%
\pgfpathlineto{\pgfqpoint{2.675392in}{2.007051in}}%
\pgfpathmoveto{\pgfqpoint{2.675392in}{2.007051in}}%
\pgfpathlineto{\pgfqpoint{2.675392in}{2.007051in}}%
\pgfpathlineto{\pgfqpoint{2.675392in}{2.010001in}}%
\pgfpathlineto{\pgfqpoint{2.679933in}{2.010001in}}%
\pgfpathlineto{\pgfqpoint{2.679933in}{2.007051in}}%
\pgfpathmoveto{\pgfqpoint{2.679933in}{2.007051in}}%
\pgfpathlineto{\pgfqpoint{2.679933in}{2.007051in}}%
\pgfpathlineto{\pgfqpoint{2.679933in}{2.010001in}}%
\pgfpathlineto{\pgfqpoint{2.684474in}{2.010001in}}%
\pgfpathlineto{\pgfqpoint{2.684474in}{2.007051in}}%
\pgfpathmoveto{\pgfqpoint{2.684474in}{2.007051in}}%
\pgfpathlineto{\pgfqpoint{2.684474in}{2.007051in}}%
\pgfpathlineto{\pgfqpoint{2.684474in}{2.010001in}}%
\pgfpathlineto{\pgfqpoint{2.689015in}{2.010001in}}%
\pgfpathlineto{\pgfqpoint{2.689015in}{2.007051in}}%
\pgfpathmoveto{\pgfqpoint{2.689015in}{2.007051in}}%
\pgfpathlineto{\pgfqpoint{2.689015in}{2.007051in}}%
\pgfpathlineto{\pgfqpoint{2.689015in}{2.010001in}}%
\pgfpathlineto{\pgfqpoint{2.693556in}{2.010001in}}%
\pgfpathlineto{\pgfqpoint{2.693556in}{2.007051in}}%
\pgfpathmoveto{\pgfqpoint{2.693556in}{2.007051in}}%
\pgfpathlineto{\pgfqpoint{2.693556in}{2.007051in}}%
\pgfpathlineto{\pgfqpoint{2.693556in}{2.010001in}}%
\pgfpathlineto{\pgfqpoint{2.698096in}{2.010001in}}%
\pgfpathlineto{\pgfqpoint{2.698096in}{2.007051in}}%
\pgfpathmoveto{\pgfqpoint{2.698096in}{2.007051in}}%
\pgfpathlineto{\pgfqpoint{2.698096in}{2.007051in}}%
\pgfpathlineto{\pgfqpoint{2.698096in}{2.010001in}}%
\pgfpathlineto{\pgfqpoint{2.702637in}{2.010001in}}%
\pgfpathlineto{\pgfqpoint{2.702637in}{2.007051in}}%
\pgfpathmoveto{\pgfqpoint{2.702637in}{2.007051in}}%
\pgfpathlineto{\pgfqpoint{2.702637in}{2.007051in}}%
\pgfpathlineto{\pgfqpoint{2.702637in}{2.010001in}}%
\pgfpathlineto{\pgfqpoint{2.707178in}{2.010001in}}%
\pgfpathlineto{\pgfqpoint{2.707178in}{2.007051in}}%
\pgfpathmoveto{\pgfqpoint{2.707178in}{2.007051in}}%
\pgfpathlineto{\pgfqpoint{2.707178in}{2.007051in}}%
\pgfpathlineto{\pgfqpoint{2.707178in}{2.010001in}}%
\pgfpathlineto{\pgfqpoint{2.711719in}{2.010001in}}%
\pgfpathlineto{\pgfqpoint{2.711719in}{2.007051in}}%
\pgfpathmoveto{\pgfqpoint{2.711719in}{2.007051in}}%
\pgfpathlineto{\pgfqpoint{2.711719in}{2.007051in}}%
\pgfpathlineto{\pgfqpoint{2.711719in}{2.010001in}}%
\pgfpathlineto{\pgfqpoint{2.716259in}{2.010001in}}%
\pgfpathlineto{\pgfqpoint{2.716259in}{2.007051in}}%
\pgfpathmoveto{\pgfqpoint{2.716259in}{2.007051in}}%
\pgfpathlineto{\pgfqpoint{2.716259in}{2.007051in}}%
\pgfpathlineto{\pgfqpoint{2.716259in}{2.010001in}}%
\pgfpathlineto{\pgfqpoint{2.720800in}{2.010001in}}%
\pgfpathlineto{\pgfqpoint{2.720800in}{2.007051in}}%
\pgfpathmoveto{\pgfqpoint{2.720800in}{2.007051in}}%
\pgfpathlineto{\pgfqpoint{2.720800in}{2.007051in}}%
\pgfpathlineto{\pgfqpoint{2.720800in}{2.010001in}}%
\pgfpathlineto{\pgfqpoint{2.725341in}{2.010001in}}%
\pgfpathlineto{\pgfqpoint{2.725341in}{2.007051in}}%
\pgfpathmoveto{\pgfqpoint{2.725341in}{2.007051in}}%
\pgfpathlineto{\pgfqpoint{2.725341in}{2.007051in}}%
\pgfpathlineto{\pgfqpoint{2.725341in}{2.010001in}}%
\pgfpathlineto{\pgfqpoint{2.729882in}{2.010001in}}%
\pgfpathlineto{\pgfqpoint{2.729882in}{2.007051in}}%
\pgfpathmoveto{\pgfqpoint{2.729882in}{2.007051in}}%
\pgfpathlineto{\pgfqpoint{2.729882in}{2.007051in}}%
\pgfpathlineto{\pgfqpoint{2.729882in}{2.010001in}}%
\pgfpathlineto{\pgfqpoint{2.734422in}{2.010001in}}%
\pgfpathlineto{\pgfqpoint{2.734422in}{2.007051in}}%
\pgfpathmoveto{\pgfqpoint{2.734422in}{2.007051in}}%
\pgfpathlineto{\pgfqpoint{2.734422in}{2.007051in}}%
\pgfpathlineto{\pgfqpoint{2.734422in}{2.010001in}}%
\pgfpathlineto{\pgfqpoint{2.738963in}{2.010001in}}%
\pgfpathlineto{\pgfqpoint{2.738963in}{2.007051in}}%
\pgfpathmoveto{\pgfqpoint{2.738963in}{2.007051in}}%
\pgfpathlineto{\pgfqpoint{2.738963in}{2.007051in}}%
\pgfpathlineto{\pgfqpoint{2.738963in}{2.010001in}}%
\pgfpathlineto{\pgfqpoint{2.743504in}{2.010001in}}%
\pgfpathlineto{\pgfqpoint{2.743504in}{2.007051in}}%
\pgfpathmoveto{\pgfqpoint{2.743504in}{2.007051in}}%
\pgfpathlineto{\pgfqpoint{2.743504in}{2.007051in}}%
\pgfpathlineto{\pgfqpoint{2.743504in}{2.010001in}}%
\pgfpathlineto{\pgfqpoint{2.748045in}{2.010001in}}%
\pgfpathlineto{\pgfqpoint{2.748045in}{2.007051in}}%
\pgfpathmoveto{\pgfqpoint{2.748045in}{2.007051in}}%
\pgfpathlineto{\pgfqpoint{2.748045in}{2.007051in}}%
\pgfpathlineto{\pgfqpoint{2.748045in}{2.010001in}}%
\pgfpathlineto{\pgfqpoint{2.752585in}{2.010001in}}%
\pgfpathlineto{\pgfqpoint{2.752585in}{2.007051in}}%
\pgfpathmoveto{\pgfqpoint{2.752585in}{2.007051in}}%
\pgfpathlineto{\pgfqpoint{2.752585in}{2.007051in}}%
\pgfpathlineto{\pgfqpoint{2.752585in}{2.010001in}}%
\pgfpathlineto{\pgfqpoint{2.757126in}{2.010001in}}%
\pgfpathlineto{\pgfqpoint{2.757126in}{2.007051in}}%
\pgfpathmoveto{\pgfqpoint{2.757126in}{2.007051in}}%
\pgfpathlineto{\pgfqpoint{2.757126in}{2.007051in}}%
\pgfpathlineto{\pgfqpoint{2.757126in}{2.010001in}}%
\pgfpathlineto{\pgfqpoint{2.761667in}{2.010001in}}%
\pgfpathlineto{\pgfqpoint{2.761667in}{2.007051in}}%
\pgfpathmoveto{\pgfqpoint{2.761667in}{2.007051in}}%
\pgfpathlineto{\pgfqpoint{2.761667in}{2.007051in}}%
\pgfpathlineto{\pgfqpoint{2.761667in}{2.010001in}}%
\pgfpathlineto{\pgfqpoint{2.766208in}{2.010001in}}%
\pgfpathlineto{\pgfqpoint{2.766208in}{2.007051in}}%
\pgfpathmoveto{\pgfqpoint{2.766208in}{2.007051in}}%
\pgfpathlineto{\pgfqpoint{2.766208in}{2.007051in}}%
\pgfpathlineto{\pgfqpoint{2.766208in}{2.010001in}}%
\pgfpathlineto{\pgfqpoint{2.770748in}{2.010001in}}%
\pgfpathlineto{\pgfqpoint{2.770748in}{2.007051in}}%
\pgfpathmoveto{\pgfqpoint{2.770748in}{2.007051in}}%
\pgfpathlineto{\pgfqpoint{2.770748in}{2.007051in}}%
\pgfpathlineto{\pgfqpoint{2.770748in}{2.010001in}}%
\pgfpathlineto{\pgfqpoint{2.775289in}{2.010001in}}%
\pgfpathlineto{\pgfqpoint{2.775289in}{2.007051in}}%
\pgfpathmoveto{\pgfqpoint{2.775289in}{2.007051in}}%
\pgfpathlineto{\pgfqpoint{2.775289in}{2.007051in}}%
\pgfpathlineto{\pgfqpoint{2.775289in}{2.010001in}}%
\pgfpathlineto{\pgfqpoint{2.779830in}{2.010001in}}%
\pgfpathlineto{\pgfqpoint{2.779830in}{2.007051in}}%
\pgfpathmoveto{\pgfqpoint{2.779830in}{2.007051in}}%
\pgfpathlineto{\pgfqpoint{2.779830in}{2.007051in}}%
\pgfpathlineto{\pgfqpoint{2.779830in}{2.010001in}}%
\pgfpathlineto{\pgfqpoint{2.784371in}{2.010001in}}%
\pgfpathlineto{\pgfqpoint{2.784371in}{2.007051in}}%
\pgfpathmoveto{\pgfqpoint{2.784371in}{2.007051in}}%
\pgfpathlineto{\pgfqpoint{2.784371in}{2.007051in}}%
\pgfpathlineto{\pgfqpoint{2.784371in}{2.010001in}}%
\pgfpathlineto{\pgfqpoint{2.788912in}{2.010001in}}%
\pgfpathlineto{\pgfqpoint{2.788912in}{2.007051in}}%
\pgfpathmoveto{\pgfqpoint{2.788912in}{2.007051in}}%
\pgfpathlineto{\pgfqpoint{2.788912in}{2.007051in}}%
\pgfpathlineto{\pgfqpoint{2.788912in}{2.010001in}}%
\pgfpathlineto{\pgfqpoint{2.793453in}{2.010001in}}%
\pgfpathlineto{\pgfqpoint{2.793453in}{2.007051in}}%
\pgfpathmoveto{\pgfqpoint{2.793453in}{2.007051in}}%
\pgfpathlineto{\pgfqpoint{2.793453in}{2.007051in}}%
\pgfpathlineto{\pgfqpoint{2.793453in}{2.010001in}}%
\pgfpathlineto{\pgfqpoint{2.797994in}{2.010001in}}%
\pgfpathlineto{\pgfqpoint{2.797994in}{2.007051in}}%
\pgfpathmoveto{\pgfqpoint{2.797994in}{2.007051in}}%
\pgfpathlineto{\pgfqpoint{2.797994in}{2.007051in}}%
\pgfpathlineto{\pgfqpoint{2.797994in}{2.010001in}}%
\pgfpathlineto{\pgfqpoint{2.802536in}{2.010001in}}%
\pgfpathlineto{\pgfqpoint{2.802536in}{2.007051in}}%
\pgfpathmoveto{\pgfqpoint{2.802536in}{2.007051in}}%
\pgfpathlineto{\pgfqpoint{2.802536in}{2.007051in}}%
\pgfpathlineto{\pgfqpoint{2.802536in}{2.010001in}}%
\pgfpathlineto{\pgfqpoint{2.807077in}{2.010001in}}%
\pgfpathlineto{\pgfqpoint{2.807077in}{2.007051in}}%
\pgfpathmoveto{\pgfqpoint{2.807077in}{2.007051in}}%
\pgfpathlineto{\pgfqpoint{2.807077in}{2.007051in}}%
\pgfpathlineto{\pgfqpoint{2.807077in}{2.010001in}}%
\pgfpathlineto{\pgfqpoint{2.811618in}{2.010001in}}%
\pgfpathlineto{\pgfqpoint{2.811618in}{2.007051in}}%
\pgfpathmoveto{\pgfqpoint{2.811618in}{2.007051in}}%
\pgfpathlineto{\pgfqpoint{2.811618in}{2.007051in}}%
\pgfpathlineto{\pgfqpoint{2.811618in}{2.010001in}}%
\pgfpathlineto{\pgfqpoint{2.816159in}{2.010001in}}%
\pgfpathlineto{\pgfqpoint{2.816159in}{2.007051in}}%
\pgfpathmoveto{\pgfqpoint{2.816159in}{2.007051in}}%
\pgfpathlineto{\pgfqpoint{2.816159in}{2.007051in}}%
\pgfpathlineto{\pgfqpoint{2.816159in}{2.010001in}}%
\pgfpathlineto{\pgfqpoint{2.820700in}{2.010001in}}%
\pgfpathlineto{\pgfqpoint{2.820700in}{2.007051in}}%
\pgfpathmoveto{\pgfqpoint{2.820700in}{2.007051in}}%
\pgfpathlineto{\pgfqpoint{2.820700in}{2.007051in}}%
\pgfpathlineto{\pgfqpoint{2.820700in}{2.010001in}}%
\pgfpathlineto{\pgfqpoint{2.825242in}{2.010001in}}%
\pgfpathlineto{\pgfqpoint{2.825242in}{2.007051in}}%
\pgfpathmoveto{\pgfqpoint{2.825242in}{2.007051in}}%
\pgfpathlineto{\pgfqpoint{2.825242in}{2.007051in}}%
\pgfpathlineto{\pgfqpoint{2.825242in}{2.010001in}}%
\pgfpathlineto{\pgfqpoint{2.829783in}{2.010001in}}%
\pgfpathlineto{\pgfqpoint{2.829783in}{2.007051in}}%
\pgfpathmoveto{\pgfqpoint{2.829783in}{2.007051in}}%
\pgfpathlineto{\pgfqpoint{2.829783in}{2.007051in}}%
\pgfpathlineto{\pgfqpoint{2.829783in}{2.010001in}}%
\pgfpathlineto{\pgfqpoint{2.834324in}{2.010001in}}%
\pgfpathlineto{\pgfqpoint{2.834324in}{2.007051in}}%
\pgfpathmoveto{\pgfqpoint{2.834324in}{2.007051in}}%
\pgfpathlineto{\pgfqpoint{2.834324in}{2.007051in}}%
\pgfpathlineto{\pgfqpoint{2.834324in}{2.010001in}}%
\pgfpathlineto{\pgfqpoint{2.838865in}{2.010001in}}%
\pgfpathlineto{\pgfqpoint{2.838865in}{2.007051in}}%
\pgfpathmoveto{\pgfqpoint{2.838865in}{2.007051in}}%
\pgfpathlineto{\pgfqpoint{2.838865in}{2.007051in}}%
\pgfpathlineto{\pgfqpoint{2.838865in}{2.010001in}}%
\pgfpathlineto{\pgfqpoint{2.843406in}{2.010001in}}%
\pgfpathlineto{\pgfqpoint{2.843406in}{2.007051in}}%
\pgfpathmoveto{\pgfqpoint{2.843406in}{2.007051in}}%
\pgfpathlineto{\pgfqpoint{2.843406in}{2.007051in}}%
\pgfpathlineto{\pgfqpoint{2.843406in}{2.010001in}}%
\pgfpathlineto{\pgfqpoint{2.847948in}{2.010001in}}%
\pgfpathlineto{\pgfqpoint{2.847948in}{2.007051in}}%
\pgfpathmoveto{\pgfqpoint{2.847948in}{2.007051in}}%
\pgfpathlineto{\pgfqpoint{2.847948in}{2.007051in}}%
\pgfpathlineto{\pgfqpoint{2.847948in}{2.010001in}}%
\pgfpathlineto{\pgfqpoint{2.852489in}{2.010001in}}%
\pgfpathlineto{\pgfqpoint{2.852489in}{2.007051in}}%
\pgfpathmoveto{\pgfqpoint{2.852489in}{2.007051in}}%
\pgfpathlineto{\pgfqpoint{2.852489in}{2.007051in}}%
\pgfpathlineto{\pgfqpoint{2.852489in}{2.010001in}}%
\pgfpathlineto{\pgfqpoint{2.857030in}{2.010001in}}%
\pgfpathlineto{\pgfqpoint{2.857030in}{2.007051in}}%
\pgfpathmoveto{\pgfqpoint{2.857030in}{2.007051in}}%
\pgfpathlineto{\pgfqpoint{2.857030in}{2.007051in}}%
\pgfpathlineto{\pgfqpoint{2.857030in}{2.010001in}}%
\pgfpathlineto{\pgfqpoint{2.861571in}{2.010001in}}%
\pgfpathlineto{\pgfqpoint{2.861571in}{2.007051in}}%
\pgfpathmoveto{\pgfqpoint{2.861571in}{2.007051in}}%
\pgfpathlineto{\pgfqpoint{2.861571in}{2.007051in}}%
\pgfpathlineto{\pgfqpoint{2.861571in}{2.010001in}}%
\pgfpathlineto{\pgfqpoint{2.866112in}{2.010001in}}%
\pgfpathlineto{\pgfqpoint{2.866112in}{2.007051in}}%
\pgfpathmoveto{\pgfqpoint{2.866112in}{2.007051in}}%
\pgfpathlineto{\pgfqpoint{2.866112in}{2.007051in}}%
\pgfpathlineto{\pgfqpoint{2.866112in}{2.010001in}}%
\pgfpathlineto{\pgfqpoint{2.870654in}{2.010001in}}%
\pgfpathlineto{\pgfqpoint{2.870654in}{2.007051in}}%
\pgfpathmoveto{\pgfqpoint{2.870654in}{2.007051in}}%
\pgfpathlineto{\pgfqpoint{2.870654in}{2.007051in}}%
\pgfpathlineto{\pgfqpoint{2.870654in}{2.010001in}}%
\pgfpathlineto{\pgfqpoint{2.875195in}{2.010001in}}%
\pgfpathlineto{\pgfqpoint{2.875195in}{2.007051in}}%
\pgfpathmoveto{\pgfqpoint{2.875195in}{2.007051in}}%
\pgfpathlineto{\pgfqpoint{2.875195in}{2.007051in}}%
\pgfpathlineto{\pgfqpoint{2.875195in}{2.010001in}}%
\pgfpathlineto{\pgfqpoint{2.879736in}{2.010001in}}%
\pgfpathlineto{\pgfqpoint{2.879736in}{2.007051in}}%
\pgfpathmoveto{\pgfqpoint{2.879736in}{2.007051in}}%
\pgfpathlineto{\pgfqpoint{2.879736in}{2.007051in}}%
\pgfpathlineto{\pgfqpoint{2.879736in}{2.010001in}}%
\pgfpathlineto{\pgfqpoint{2.884277in}{2.010001in}}%
\pgfpathlineto{\pgfqpoint{2.884277in}{2.007051in}}%
\pgfpathmoveto{\pgfqpoint{2.884277in}{2.007051in}}%
\pgfpathlineto{\pgfqpoint{2.884277in}{2.007051in}}%
\pgfpathlineto{\pgfqpoint{2.884277in}{2.010001in}}%
\pgfpathlineto{\pgfqpoint{2.888818in}{2.010001in}}%
\pgfpathlineto{\pgfqpoint{2.888818in}{2.007051in}}%
\pgfpathmoveto{\pgfqpoint{2.888818in}{2.007051in}}%
\pgfpathlineto{\pgfqpoint{2.888818in}{2.007051in}}%
\pgfpathlineto{\pgfqpoint{2.888818in}{2.010001in}}%
\pgfpathlineto{\pgfqpoint{2.893360in}{2.010001in}}%
\pgfpathlineto{\pgfqpoint{2.893360in}{2.007051in}}%
\pgfpathmoveto{\pgfqpoint{2.893360in}{2.007051in}}%
\pgfpathlineto{\pgfqpoint{2.893360in}{2.007051in}}%
\pgfpathlineto{\pgfqpoint{2.893360in}{2.010001in}}%
\pgfpathlineto{\pgfqpoint{2.897901in}{2.010001in}}%
\pgfpathlineto{\pgfqpoint{2.897901in}{2.007051in}}%
\pgfpathmoveto{\pgfqpoint{2.897901in}{2.007051in}}%
\pgfpathlineto{\pgfqpoint{2.897901in}{2.007051in}}%
\pgfpathlineto{\pgfqpoint{2.897901in}{2.010001in}}%
\pgfpathlineto{\pgfqpoint{2.902442in}{2.010001in}}%
\pgfpathlineto{\pgfqpoint{2.902442in}{2.007051in}}%
\pgfpathmoveto{\pgfqpoint{2.902442in}{2.007051in}}%
\pgfpathlineto{\pgfqpoint{2.902442in}{2.007051in}}%
\pgfpathlineto{\pgfqpoint{2.902442in}{2.010001in}}%
\pgfpathlineto{\pgfqpoint{2.906983in}{2.010001in}}%
\pgfpathlineto{\pgfqpoint{2.906983in}{2.007051in}}%
\pgfpathmoveto{\pgfqpoint{2.906983in}{2.007051in}}%
\pgfpathlineto{\pgfqpoint{2.906983in}{2.007051in}}%
\pgfpathlineto{\pgfqpoint{2.906983in}{2.010001in}}%
\pgfpathlineto{\pgfqpoint{2.911524in}{2.010001in}}%
\pgfpathlineto{\pgfqpoint{2.911524in}{2.007051in}}%
\pgfpathmoveto{\pgfqpoint{2.911524in}{2.007051in}}%
\pgfpathlineto{\pgfqpoint{2.911524in}{2.007051in}}%
\pgfpathlineto{\pgfqpoint{2.911524in}{2.010001in}}%
\pgfpathlineto{\pgfqpoint{2.916066in}{2.010001in}}%
\pgfpathlineto{\pgfqpoint{2.916066in}{2.007051in}}%
\pgfpathmoveto{\pgfqpoint{2.916066in}{2.007051in}}%
\pgfpathlineto{\pgfqpoint{2.916066in}{2.007051in}}%
\pgfpathlineto{\pgfqpoint{2.916066in}{2.010001in}}%
\pgfpathlineto{\pgfqpoint{2.920607in}{2.010001in}}%
\pgfpathlineto{\pgfqpoint{2.920607in}{2.007051in}}%
\pgfpathmoveto{\pgfqpoint{2.920607in}{2.007051in}}%
\pgfpathlineto{\pgfqpoint{2.920607in}{2.007051in}}%
\pgfpathlineto{\pgfqpoint{2.920607in}{2.010001in}}%
\pgfpathlineto{\pgfqpoint{2.925148in}{2.010001in}}%
\pgfpathlineto{\pgfqpoint{2.925148in}{2.007051in}}%
\pgfpathmoveto{\pgfqpoint{2.925148in}{2.007051in}}%
\pgfpathlineto{\pgfqpoint{2.925148in}{2.007051in}}%
\pgfpathlineto{\pgfqpoint{2.925148in}{2.010001in}}%
\pgfpathlineto{\pgfqpoint{2.929689in}{2.010001in}}%
\pgfpathlineto{\pgfqpoint{2.929689in}{2.007051in}}%
\pgfpathmoveto{\pgfqpoint{2.929689in}{2.007051in}}%
\pgfpathlineto{\pgfqpoint{2.929689in}{2.007051in}}%
\pgfpathlineto{\pgfqpoint{2.929689in}{2.010001in}}%
\pgfpathlineto{\pgfqpoint{2.934230in}{2.010001in}}%
\pgfpathlineto{\pgfqpoint{2.934230in}{2.007051in}}%
\pgfpathmoveto{\pgfqpoint{2.934230in}{2.007051in}}%
\pgfpathlineto{\pgfqpoint{2.934230in}{2.007051in}}%
\pgfpathlineto{\pgfqpoint{2.934230in}{2.010001in}}%
\pgfpathlineto{\pgfqpoint{2.938771in}{2.010001in}}%
\pgfpathlineto{\pgfqpoint{2.938771in}{2.007051in}}%
\pgfpathmoveto{\pgfqpoint{2.938771in}{2.007051in}}%
\pgfpathlineto{\pgfqpoint{2.938771in}{2.007051in}}%
\pgfpathlineto{\pgfqpoint{2.938771in}{2.010001in}}%
\pgfpathlineto{\pgfqpoint{2.943312in}{2.010001in}}%
\pgfpathlineto{\pgfqpoint{2.943312in}{2.007051in}}%
\pgfpathmoveto{\pgfqpoint{2.943312in}{2.007051in}}%
\pgfpathlineto{\pgfqpoint{2.943312in}{2.007051in}}%
\pgfpathlineto{\pgfqpoint{2.943312in}{2.010001in}}%
\pgfpathlineto{\pgfqpoint{2.947853in}{2.010001in}}%
\pgfpathlineto{\pgfqpoint{2.947853in}{2.007051in}}%
\pgfpathmoveto{\pgfqpoint{2.947853in}{2.007051in}}%
\pgfpathlineto{\pgfqpoint{2.947853in}{2.007051in}}%
\pgfpathlineto{\pgfqpoint{2.947853in}{2.010001in}}%
\pgfpathlineto{\pgfqpoint{2.952394in}{2.010001in}}%
\pgfpathlineto{\pgfqpoint{2.952394in}{2.007051in}}%
\pgfpathmoveto{\pgfqpoint{2.952394in}{2.007051in}}%
\pgfpathlineto{\pgfqpoint{2.952394in}{2.007051in}}%
\pgfpathlineto{\pgfqpoint{2.952394in}{2.010001in}}%
\pgfpathlineto{\pgfqpoint{2.956935in}{2.010001in}}%
\pgfpathlineto{\pgfqpoint{2.956935in}{2.007051in}}%
\pgfpathmoveto{\pgfqpoint{2.956935in}{2.007051in}}%
\pgfpathlineto{\pgfqpoint{2.956935in}{2.007051in}}%
\pgfpathlineto{\pgfqpoint{2.956935in}{2.010001in}}%
\pgfpathlineto{\pgfqpoint{2.961476in}{2.010001in}}%
\pgfpathlineto{\pgfqpoint{2.961476in}{2.007051in}}%
\pgfpathmoveto{\pgfqpoint{2.961476in}{2.007051in}}%
\pgfpathlineto{\pgfqpoint{2.961476in}{2.007051in}}%
\pgfpathlineto{\pgfqpoint{2.961476in}{2.010001in}}%
\pgfpathlineto{\pgfqpoint{2.966017in}{2.010001in}}%
\pgfpathlineto{\pgfqpoint{2.966017in}{2.007051in}}%
\pgfpathmoveto{\pgfqpoint{2.966017in}{2.007051in}}%
\pgfpathlineto{\pgfqpoint{2.966017in}{2.007051in}}%
\pgfpathlineto{\pgfqpoint{2.966017in}{2.010001in}}%
\pgfpathlineto{\pgfqpoint{2.970558in}{2.010001in}}%
\pgfpathlineto{\pgfqpoint{2.970558in}{2.007051in}}%
\pgfpathmoveto{\pgfqpoint{2.970558in}{2.007051in}}%
\pgfpathlineto{\pgfqpoint{2.970558in}{2.007051in}}%
\pgfpathlineto{\pgfqpoint{2.970558in}{2.010001in}}%
\pgfpathlineto{\pgfqpoint{2.975099in}{2.010001in}}%
\pgfpathlineto{\pgfqpoint{2.975099in}{2.007051in}}%
\pgfpathmoveto{\pgfqpoint{2.975099in}{2.007051in}}%
\pgfpathlineto{\pgfqpoint{2.975099in}{2.007051in}}%
\pgfpathlineto{\pgfqpoint{2.975099in}{2.010001in}}%
\pgfpathlineto{\pgfqpoint{2.979640in}{2.010001in}}%
\pgfpathlineto{\pgfqpoint{2.979640in}{2.007051in}}%
\pgfpathmoveto{\pgfqpoint{2.979640in}{2.007051in}}%
\pgfpathlineto{\pgfqpoint{2.979640in}{2.007051in}}%
\pgfpathlineto{\pgfqpoint{2.979640in}{2.010001in}}%
\pgfpathlineto{\pgfqpoint{2.984181in}{2.010001in}}%
\pgfpathlineto{\pgfqpoint{2.984181in}{2.007051in}}%
\pgfpathmoveto{\pgfqpoint{2.984181in}{2.007051in}}%
\pgfpathlineto{\pgfqpoint{2.984181in}{2.007051in}}%
\pgfpathlineto{\pgfqpoint{2.984181in}{2.010001in}}%
\pgfpathlineto{\pgfqpoint{2.988722in}{2.010001in}}%
\pgfpathlineto{\pgfqpoint{2.988722in}{2.007051in}}%
\pgfpathmoveto{\pgfqpoint{2.988722in}{2.007051in}}%
\pgfpathlineto{\pgfqpoint{2.988722in}{2.007051in}}%
\pgfpathlineto{\pgfqpoint{2.988722in}{2.010001in}}%
\pgfpathlineto{\pgfqpoint{2.993263in}{2.010001in}}%
\pgfpathlineto{\pgfqpoint{2.993263in}{2.007051in}}%
\pgfpathmoveto{\pgfqpoint{2.993263in}{2.007051in}}%
\pgfpathlineto{\pgfqpoint{2.993263in}{2.007051in}}%
\pgfpathlineto{\pgfqpoint{2.993263in}{2.010001in}}%
\pgfpathlineto{\pgfqpoint{2.997804in}{2.010001in}}%
\pgfpathlineto{\pgfqpoint{2.997804in}{2.007051in}}%
\pgfpathmoveto{\pgfqpoint{2.997804in}{2.007051in}}%
\pgfpathlineto{\pgfqpoint{2.997804in}{2.007051in}}%
\pgfpathlineto{\pgfqpoint{2.997804in}{2.010001in}}%
\pgfpathlineto{\pgfqpoint{3.002345in}{2.010001in}}%
\pgfpathlineto{\pgfqpoint{3.002345in}{2.007051in}}%
\pgfpathmoveto{\pgfqpoint{3.002345in}{2.007051in}}%
\pgfpathlineto{\pgfqpoint{3.002345in}{2.007051in}}%
\pgfpathlineto{\pgfqpoint{3.002345in}{2.010001in}}%
\pgfpathlineto{\pgfqpoint{3.006886in}{2.010001in}}%
\pgfpathlineto{\pgfqpoint{3.006886in}{2.007051in}}%
\pgfpathmoveto{\pgfqpoint{3.006886in}{2.007051in}}%
\pgfpathlineto{\pgfqpoint{3.006886in}{2.007051in}}%
\pgfpathlineto{\pgfqpoint{3.006886in}{2.010001in}}%
\pgfpathlineto{\pgfqpoint{3.011427in}{2.010001in}}%
\pgfpathlineto{\pgfqpoint{3.011427in}{2.007051in}}%
\pgfpathmoveto{\pgfqpoint{3.011427in}{2.007051in}}%
\pgfpathlineto{\pgfqpoint{3.011427in}{2.007051in}}%
\pgfpathlineto{\pgfqpoint{3.011427in}{2.010001in}}%
\pgfpathlineto{\pgfqpoint{3.015968in}{2.010001in}}%
\pgfpathlineto{\pgfqpoint{3.015968in}{2.007051in}}%
\pgfpathmoveto{\pgfqpoint{3.015968in}{2.007051in}}%
\pgfpathlineto{\pgfqpoint{3.015968in}{2.007051in}}%
\pgfpathlineto{\pgfqpoint{3.015968in}{2.010001in}}%
\pgfpathlineto{\pgfqpoint{3.020509in}{2.010001in}}%
\pgfpathlineto{\pgfqpoint{3.020509in}{2.007051in}}%
\pgfpathmoveto{\pgfqpoint{3.020509in}{2.007051in}}%
\pgfpathlineto{\pgfqpoint{3.020509in}{2.007051in}}%
\pgfpathlineto{\pgfqpoint{3.020509in}{2.010001in}}%
\pgfpathlineto{\pgfqpoint{3.025050in}{2.010001in}}%
\pgfpathlineto{\pgfqpoint{3.025050in}{2.007051in}}%
\pgfpathmoveto{\pgfqpoint{3.025050in}{2.007051in}}%
\pgfpathlineto{\pgfqpoint{3.025050in}{2.007051in}}%
\pgfpathlineto{\pgfqpoint{3.025050in}{2.010001in}}%
\pgfpathlineto{\pgfqpoint{3.029591in}{2.010001in}}%
\pgfpathlineto{\pgfqpoint{3.029591in}{2.007051in}}%
\pgfpathmoveto{\pgfqpoint{3.029591in}{2.007051in}}%
\pgfpathlineto{\pgfqpoint{3.029591in}{2.007051in}}%
\pgfpathlineto{\pgfqpoint{3.029591in}{2.010001in}}%
\pgfpathlineto{\pgfqpoint{3.034132in}{2.010001in}}%
\pgfpathlineto{\pgfqpoint{3.034132in}{2.007051in}}%
\pgfpathmoveto{\pgfqpoint{3.034132in}{2.007051in}}%
\pgfpathlineto{\pgfqpoint{3.034132in}{2.007051in}}%
\pgfpathlineto{\pgfqpoint{3.034132in}{2.010001in}}%
\pgfpathlineto{\pgfqpoint{3.038672in}{2.010001in}}%
\pgfpathlineto{\pgfqpoint{3.038672in}{2.007051in}}%
\pgfpathmoveto{\pgfqpoint{3.038672in}{2.007051in}}%
\pgfpathlineto{\pgfqpoint{3.038672in}{2.007051in}}%
\pgfpathlineto{\pgfqpoint{3.038672in}{2.010001in}}%
\pgfpathlineto{\pgfqpoint{3.043213in}{2.010001in}}%
\pgfpathlineto{\pgfqpoint{3.043213in}{2.007051in}}%
\pgfpathmoveto{\pgfqpoint{3.043213in}{2.007051in}}%
\pgfpathlineto{\pgfqpoint{3.043213in}{2.007051in}}%
\pgfpathlineto{\pgfqpoint{3.043213in}{2.010001in}}%
\pgfpathlineto{\pgfqpoint{3.047754in}{2.010001in}}%
\pgfpathlineto{\pgfqpoint{3.047754in}{2.007051in}}%
\pgfpathmoveto{\pgfqpoint{3.047754in}{2.007051in}}%
\pgfpathlineto{\pgfqpoint{3.047754in}{2.007051in}}%
\pgfpathlineto{\pgfqpoint{3.047754in}{2.010001in}}%
\pgfpathlineto{\pgfqpoint{3.052295in}{2.010001in}}%
\pgfpathlineto{\pgfqpoint{3.052295in}{2.007051in}}%
\pgfpathmoveto{\pgfqpoint{3.052295in}{2.007051in}}%
\pgfpathlineto{\pgfqpoint{3.052295in}{2.007051in}}%
\pgfpathlineto{\pgfqpoint{3.052295in}{2.010001in}}%
\pgfpathlineto{\pgfqpoint{3.056836in}{2.010001in}}%
\pgfpathlineto{\pgfqpoint{3.056836in}{2.007051in}}%
\pgfpathmoveto{\pgfqpoint{3.056836in}{2.007051in}}%
\pgfpathlineto{\pgfqpoint{3.056836in}{2.007051in}}%
\pgfpathlineto{\pgfqpoint{3.056836in}{2.010001in}}%
\pgfpathlineto{\pgfqpoint{3.061377in}{2.010001in}}%
\pgfpathlineto{\pgfqpoint{3.061377in}{2.007051in}}%
\pgfpathmoveto{\pgfqpoint{3.061377in}{2.007051in}}%
\pgfpathlineto{\pgfqpoint{3.061377in}{2.007051in}}%
\pgfpathlineto{\pgfqpoint{3.061377in}{2.010001in}}%
\pgfpathlineto{\pgfqpoint{3.065918in}{2.010001in}}%
\pgfpathlineto{\pgfqpoint{3.065918in}{2.007051in}}%
\pgfpathmoveto{\pgfqpoint{3.065918in}{2.007051in}}%
\pgfpathlineto{\pgfqpoint{3.065918in}{2.007051in}}%
\pgfpathlineto{\pgfqpoint{3.065918in}{2.010001in}}%
\pgfpathlineto{\pgfqpoint{3.070459in}{2.010001in}}%
\pgfpathlineto{\pgfqpoint{3.070459in}{2.007051in}}%
\pgfpathmoveto{\pgfqpoint{3.070459in}{2.007051in}}%
\pgfpathlineto{\pgfqpoint{3.070459in}{2.007051in}}%
\pgfpathlineto{\pgfqpoint{3.070459in}{2.010001in}}%
\pgfpathlineto{\pgfqpoint{3.075000in}{2.010001in}}%
\pgfpathlineto{\pgfqpoint{3.075000in}{2.007051in}}%
\pgfpathmoveto{\pgfqpoint{3.075000in}{2.007051in}}%
\pgfpathlineto{\pgfqpoint{3.075000in}{2.007051in}}%
\pgfpathlineto{\pgfqpoint{3.075000in}{2.010001in}}%
\pgfpathlineto{\pgfqpoint{3.079541in}{2.010001in}}%
\pgfpathlineto{\pgfqpoint{3.079541in}{2.007051in}}%
\pgfpathmoveto{\pgfqpoint{3.079541in}{2.007051in}}%
\pgfpathlineto{\pgfqpoint{3.079541in}{2.007051in}}%
\pgfpathlineto{\pgfqpoint{3.079541in}{2.010001in}}%
\pgfpathlineto{\pgfqpoint{3.084082in}{2.010001in}}%
\pgfpathlineto{\pgfqpoint{3.084082in}{2.007051in}}%
\pgfpathmoveto{\pgfqpoint{3.084082in}{2.007051in}}%
\pgfpathlineto{\pgfqpoint{3.084082in}{2.007051in}}%
\pgfpathlineto{\pgfqpoint{3.084082in}{2.010001in}}%
\pgfpathlineto{\pgfqpoint{3.088623in}{2.010001in}}%
\pgfpathlineto{\pgfqpoint{3.088623in}{2.007051in}}%
\pgfpathmoveto{\pgfqpoint{3.088623in}{2.007051in}}%
\pgfpathlineto{\pgfqpoint{3.088623in}{2.007051in}}%
\pgfpathlineto{\pgfqpoint{3.088623in}{2.010001in}}%
\pgfpathlineto{\pgfqpoint{3.093164in}{2.010001in}}%
\pgfpathlineto{\pgfqpoint{3.093164in}{2.007051in}}%
\pgfpathmoveto{\pgfqpoint{3.093164in}{2.007051in}}%
\pgfpathlineto{\pgfqpoint{3.093164in}{2.007051in}}%
\pgfpathlineto{\pgfqpoint{3.093164in}{2.010001in}}%
\pgfpathlineto{\pgfqpoint{3.097705in}{2.010001in}}%
\pgfpathlineto{\pgfqpoint{3.097705in}{2.007051in}}%
\pgfpathmoveto{\pgfqpoint{3.097705in}{2.007051in}}%
\pgfpathlineto{\pgfqpoint{3.097705in}{2.007051in}}%
\pgfpathlineto{\pgfqpoint{3.097705in}{2.010001in}}%
\pgfpathlineto{\pgfqpoint{3.102246in}{2.010001in}}%
\pgfpathlineto{\pgfqpoint{3.102246in}{2.007051in}}%
\pgfpathmoveto{\pgfqpoint{3.102246in}{2.007051in}}%
\pgfpathlineto{\pgfqpoint{3.102246in}{2.007051in}}%
\pgfpathlineto{\pgfqpoint{3.102246in}{2.010001in}}%
\pgfpathlineto{\pgfqpoint{3.106787in}{2.010001in}}%
\pgfpathlineto{\pgfqpoint{3.106787in}{2.007051in}}%
\pgfpathmoveto{\pgfqpoint{3.106787in}{2.007051in}}%
\pgfpathlineto{\pgfqpoint{3.106787in}{2.007051in}}%
\pgfpathlineto{\pgfqpoint{3.106787in}{2.010001in}}%
\pgfpathlineto{\pgfqpoint{3.111328in}{2.010001in}}%
\pgfpathlineto{\pgfqpoint{3.111328in}{2.007051in}}%
\pgfpathmoveto{\pgfqpoint{3.111328in}{2.007051in}}%
\pgfpathlineto{\pgfqpoint{3.111328in}{2.007051in}}%
\pgfpathlineto{\pgfqpoint{3.111328in}{2.010001in}}%
\pgfpathlineto{\pgfqpoint{3.115869in}{2.010001in}}%
\pgfpathlineto{\pgfqpoint{3.115869in}{2.007051in}}%
\pgfpathmoveto{\pgfqpoint{3.115869in}{2.007051in}}%
\pgfpathlineto{\pgfqpoint{3.115869in}{2.007051in}}%
\pgfpathlineto{\pgfqpoint{3.115869in}{2.010001in}}%
\pgfpathlineto{\pgfqpoint{3.120410in}{2.010001in}}%
\pgfpathlineto{\pgfqpoint{3.120410in}{2.007051in}}%
\pgfpathmoveto{\pgfqpoint{3.120410in}{2.007051in}}%
\pgfpathlineto{\pgfqpoint{3.120410in}{2.007051in}}%
\pgfpathlineto{\pgfqpoint{3.120410in}{2.010001in}}%
\pgfpathlineto{\pgfqpoint{3.124951in}{2.010001in}}%
\pgfpathlineto{\pgfqpoint{3.124951in}{2.007051in}}%
\pgfpathmoveto{\pgfqpoint{3.124951in}{2.007051in}}%
\pgfpathlineto{\pgfqpoint{3.124951in}{2.007051in}}%
\pgfpathlineto{\pgfqpoint{3.124951in}{2.010001in}}%
\pgfpathlineto{\pgfqpoint{3.129492in}{2.010001in}}%
\pgfpathlineto{\pgfqpoint{3.129492in}{2.007051in}}%
\pgfpathmoveto{\pgfqpoint{3.129492in}{2.007051in}}%
\pgfpathlineto{\pgfqpoint{3.129492in}{2.007051in}}%
\pgfpathlineto{\pgfqpoint{3.129492in}{2.010001in}}%
\pgfpathlineto{\pgfqpoint{3.134033in}{2.010001in}}%
\pgfpathlineto{\pgfqpoint{3.134033in}{2.007051in}}%
\pgfpathmoveto{\pgfqpoint{3.134033in}{2.007051in}}%
\pgfpathlineto{\pgfqpoint{3.134033in}{2.007051in}}%
\pgfpathlineto{\pgfqpoint{3.134033in}{2.010001in}}%
\pgfpathlineto{\pgfqpoint{3.138574in}{2.010001in}}%
\pgfpathlineto{\pgfqpoint{3.138574in}{2.007051in}}%
\pgfpathmoveto{\pgfqpoint{3.138574in}{2.007051in}}%
\pgfpathlineto{\pgfqpoint{3.138574in}{2.007051in}}%
\pgfpathlineto{\pgfqpoint{3.138574in}{2.010001in}}%
\pgfpathlineto{\pgfqpoint{3.143115in}{2.010001in}}%
\pgfpathlineto{\pgfqpoint{3.143115in}{2.007051in}}%
\pgfpathmoveto{\pgfqpoint{3.143115in}{2.007051in}}%
\pgfpathlineto{\pgfqpoint{3.143115in}{2.007051in}}%
\pgfpathlineto{\pgfqpoint{3.143115in}{2.010001in}}%
\pgfpathlineto{\pgfqpoint{3.147655in}{2.010001in}}%
\pgfpathlineto{\pgfqpoint{3.147655in}{2.007051in}}%
\pgfpathmoveto{\pgfqpoint{3.147655in}{2.007051in}}%
\pgfpathlineto{\pgfqpoint{3.147655in}{2.007051in}}%
\pgfpathlineto{\pgfqpoint{3.147655in}{2.010001in}}%
\pgfpathlineto{\pgfqpoint{3.152196in}{2.010001in}}%
\pgfpathlineto{\pgfqpoint{3.152196in}{2.007051in}}%
\pgfpathmoveto{\pgfqpoint{3.152196in}{2.007051in}}%
\pgfpathlineto{\pgfqpoint{3.152196in}{2.007051in}}%
\pgfpathlineto{\pgfqpoint{3.152196in}{2.010001in}}%
\pgfpathlineto{\pgfqpoint{3.156737in}{2.010001in}}%
\pgfpathlineto{\pgfqpoint{3.156737in}{2.007051in}}%
\pgfpathmoveto{\pgfqpoint{3.156737in}{2.007051in}}%
\pgfpathlineto{\pgfqpoint{3.156737in}{2.007051in}}%
\pgfpathlineto{\pgfqpoint{3.156737in}{2.010001in}}%
\pgfpathlineto{\pgfqpoint{3.161278in}{2.010001in}}%
\pgfpathlineto{\pgfqpoint{3.161278in}{2.007051in}}%
\pgfpathmoveto{\pgfqpoint{3.161278in}{2.007051in}}%
\pgfpathlineto{\pgfqpoint{3.161278in}{2.007051in}}%
\pgfpathlineto{\pgfqpoint{3.161278in}{2.010001in}}%
\pgfpathlineto{\pgfqpoint{3.165819in}{2.010001in}}%
\pgfpathlineto{\pgfqpoint{3.165819in}{2.007051in}}%
\pgfpathmoveto{\pgfqpoint{3.165819in}{2.007051in}}%
\pgfpathlineto{\pgfqpoint{3.165819in}{2.007051in}}%
\pgfpathlineto{\pgfqpoint{3.165819in}{2.010001in}}%
\pgfpathlineto{\pgfqpoint{3.170360in}{2.010001in}}%
\pgfpathlineto{\pgfqpoint{3.170360in}{2.007051in}}%
\pgfpathmoveto{\pgfqpoint{3.170360in}{2.007051in}}%
\pgfpathlineto{\pgfqpoint{3.170360in}{2.007051in}}%
\pgfpathlineto{\pgfqpoint{3.170360in}{2.010001in}}%
\pgfpathlineto{\pgfqpoint{3.174901in}{2.010001in}}%
\pgfpathlineto{\pgfqpoint{3.174901in}{2.007051in}}%
\pgfpathmoveto{\pgfqpoint{3.174901in}{2.007051in}}%
\pgfpathlineto{\pgfqpoint{3.174901in}{2.007051in}}%
\pgfpathlineto{\pgfqpoint{3.174901in}{2.010001in}}%
\pgfpathlineto{\pgfqpoint{3.179442in}{2.010001in}}%
\pgfpathlineto{\pgfqpoint{3.179442in}{2.007051in}}%
\pgfpathmoveto{\pgfqpoint{3.179442in}{2.007051in}}%
\pgfpathlineto{\pgfqpoint{3.179442in}{2.007051in}}%
\pgfpathlineto{\pgfqpoint{3.179442in}{2.010001in}}%
\pgfpathlineto{\pgfqpoint{3.183983in}{2.010001in}}%
\pgfpathlineto{\pgfqpoint{3.183983in}{2.007051in}}%
\pgfpathmoveto{\pgfqpoint{3.183983in}{2.007051in}}%
\pgfpathlineto{\pgfqpoint{3.183983in}{2.007051in}}%
\pgfpathlineto{\pgfqpoint{3.183983in}{2.010001in}}%
\pgfpathlineto{\pgfqpoint{3.188524in}{2.010001in}}%
\pgfpathlineto{\pgfqpoint{3.188524in}{2.007051in}}%
\pgfpathmoveto{\pgfqpoint{3.188524in}{2.007051in}}%
\pgfpathlineto{\pgfqpoint{3.188524in}{2.007051in}}%
\pgfpathlineto{\pgfqpoint{3.188524in}{2.010001in}}%
\pgfpathlineto{\pgfqpoint{3.193065in}{2.010001in}}%
\pgfpathlineto{\pgfqpoint{3.193065in}{2.007051in}}%
\pgfpathmoveto{\pgfqpoint{3.193065in}{2.007051in}}%
\pgfpathlineto{\pgfqpoint{3.193065in}{2.007051in}}%
\pgfpathlineto{\pgfqpoint{3.193065in}{2.010001in}}%
\pgfpathlineto{\pgfqpoint{3.197606in}{2.010001in}}%
\pgfpathlineto{\pgfqpoint{3.197606in}{2.007051in}}%
\pgfpathmoveto{\pgfqpoint{3.197606in}{2.007051in}}%
\pgfpathlineto{\pgfqpoint{3.197606in}{2.007051in}}%
\pgfpathlineto{\pgfqpoint{3.197606in}{2.010001in}}%
\pgfpathlineto{\pgfqpoint{3.202147in}{2.010001in}}%
\pgfpathlineto{\pgfqpoint{3.202147in}{2.007051in}}%
\pgfpathmoveto{\pgfqpoint{3.202147in}{2.007051in}}%
\pgfpathlineto{\pgfqpoint{3.202147in}{2.007051in}}%
\pgfpathlineto{\pgfqpoint{3.202147in}{2.010001in}}%
\pgfpathlineto{\pgfqpoint{3.206688in}{2.010001in}}%
\pgfpathlineto{\pgfqpoint{3.206688in}{2.007051in}}%
\pgfpathmoveto{\pgfqpoint{3.206688in}{2.007051in}}%
\pgfpathlineto{\pgfqpoint{3.206688in}{2.007051in}}%
\pgfpathlineto{\pgfqpoint{3.206688in}{2.010001in}}%
\pgfpathlineto{\pgfqpoint{3.211229in}{2.010001in}}%
\pgfpathlineto{\pgfqpoint{3.211229in}{2.007051in}}%
\pgfpathmoveto{\pgfqpoint{3.211229in}{2.007051in}}%
\pgfpathlineto{\pgfqpoint{3.211229in}{2.007051in}}%
\pgfpathlineto{\pgfqpoint{3.211229in}{2.010001in}}%
\pgfpathlineto{\pgfqpoint{3.215770in}{2.010001in}}%
\pgfpathlineto{\pgfqpoint{3.215770in}{2.007051in}}%
\pgfpathmoveto{\pgfqpoint{3.215770in}{2.007051in}}%
\pgfpathlineto{\pgfqpoint{3.215770in}{2.007051in}}%
\pgfpathlineto{\pgfqpoint{3.215770in}{2.010001in}}%
\pgfpathlineto{\pgfqpoint{3.220311in}{2.010001in}}%
\pgfpathlineto{\pgfqpoint{3.220311in}{2.007051in}}%
\pgfpathmoveto{\pgfqpoint{3.220311in}{2.007051in}}%
\pgfpathlineto{\pgfqpoint{3.220311in}{2.007051in}}%
\pgfpathlineto{\pgfqpoint{3.220311in}{2.010001in}}%
\pgfpathlineto{\pgfqpoint{3.224852in}{2.010001in}}%
\pgfpathlineto{\pgfqpoint{3.224852in}{2.007051in}}%
\pgfpathmoveto{\pgfqpoint{3.224852in}{2.007051in}}%
\pgfpathlineto{\pgfqpoint{3.224852in}{2.007051in}}%
\pgfpathlineto{\pgfqpoint{3.224852in}{2.010001in}}%
\pgfpathlineto{\pgfqpoint{3.229393in}{2.010001in}}%
\pgfpathlineto{\pgfqpoint{3.229393in}{2.007051in}}%
\pgfpathmoveto{\pgfqpoint{3.229393in}{2.007051in}}%
\pgfpathlineto{\pgfqpoint{3.229393in}{2.007051in}}%
\pgfpathlineto{\pgfqpoint{3.229393in}{2.010001in}}%
\pgfpathlineto{\pgfqpoint{3.233934in}{2.010001in}}%
\pgfpathlineto{\pgfqpoint{3.233934in}{2.007051in}}%
\pgfpathmoveto{\pgfqpoint{3.233934in}{2.007051in}}%
\pgfpathlineto{\pgfqpoint{3.233934in}{2.007051in}}%
\pgfpathlineto{\pgfqpoint{3.233934in}{2.010001in}}%
\pgfpathlineto{\pgfqpoint{3.238475in}{2.010001in}}%
\pgfpathlineto{\pgfqpoint{3.238475in}{2.007051in}}%
\pgfpathmoveto{\pgfqpoint{3.238475in}{2.007051in}}%
\pgfpathlineto{\pgfqpoint{3.238475in}{2.007051in}}%
\pgfpathlineto{\pgfqpoint{3.238475in}{2.010001in}}%
\pgfpathlineto{\pgfqpoint{3.243016in}{2.010001in}}%
\pgfpathlineto{\pgfqpoint{3.243016in}{2.007051in}}%
\pgfpathmoveto{\pgfqpoint{3.243016in}{2.007051in}}%
\pgfpathlineto{\pgfqpoint{3.243016in}{2.007051in}}%
\pgfpathlineto{\pgfqpoint{3.243016in}{2.010001in}}%
\pgfpathlineto{\pgfqpoint{3.247557in}{2.010001in}}%
\pgfpathlineto{\pgfqpoint{3.247557in}{2.007051in}}%
\pgfpathmoveto{\pgfqpoint{3.247557in}{2.007051in}}%
\pgfpathlineto{\pgfqpoint{3.247557in}{2.007051in}}%
\pgfpathlineto{\pgfqpoint{3.247557in}{2.010001in}}%
\pgfpathlineto{\pgfqpoint{3.252098in}{2.010001in}}%
\pgfpathlineto{\pgfqpoint{3.252098in}{2.007051in}}%
\pgfpathmoveto{\pgfqpoint{3.252098in}{2.007051in}}%
\pgfpathlineto{\pgfqpoint{3.252098in}{2.007051in}}%
\pgfpathlineto{\pgfqpoint{3.252098in}{2.010001in}}%
\pgfpathlineto{\pgfqpoint{3.256639in}{2.010001in}}%
\pgfpathlineto{\pgfqpoint{3.256639in}{2.007051in}}%
\pgfpathmoveto{\pgfqpoint{3.256639in}{2.007051in}}%
\pgfpathlineto{\pgfqpoint{3.256639in}{2.007051in}}%
\pgfpathlineto{\pgfqpoint{3.256639in}{2.010001in}}%
\pgfpathlineto{\pgfqpoint{3.261180in}{2.010001in}}%
\pgfpathlineto{\pgfqpoint{3.261180in}{2.007051in}}%
\pgfpathmoveto{\pgfqpoint{3.261180in}{2.007051in}}%
\pgfpathlineto{\pgfqpoint{3.261180in}{2.007051in}}%
\pgfpathlineto{\pgfqpoint{3.261180in}{2.010001in}}%
\pgfpathlineto{\pgfqpoint{3.265721in}{2.010001in}}%
\pgfpathlineto{\pgfqpoint{3.265721in}{2.007051in}}%
\pgfpathmoveto{\pgfqpoint{3.265721in}{2.007051in}}%
\pgfpathlineto{\pgfqpoint{3.265721in}{2.007051in}}%
\pgfpathlineto{\pgfqpoint{3.265721in}{2.010001in}}%
\pgfpathlineto{\pgfqpoint{3.270262in}{2.010001in}}%
\pgfpathlineto{\pgfqpoint{3.270262in}{2.007051in}}%
\pgfpathmoveto{\pgfqpoint{3.270262in}{2.007051in}}%
\pgfpathlineto{\pgfqpoint{3.270262in}{2.007051in}}%
\pgfpathlineto{\pgfqpoint{3.270262in}{2.010001in}}%
\pgfpathlineto{\pgfqpoint{3.274803in}{2.010001in}}%
\pgfpathlineto{\pgfqpoint{3.274803in}{2.007051in}}%
\pgfpathmoveto{\pgfqpoint{3.274803in}{2.007051in}}%
\pgfpathlineto{\pgfqpoint{3.274803in}{2.007051in}}%
\pgfpathlineto{\pgfqpoint{3.274803in}{2.010001in}}%
\pgfpathlineto{\pgfqpoint{3.279344in}{2.010001in}}%
\pgfpathlineto{\pgfqpoint{3.279344in}{2.007051in}}%
\pgfpathmoveto{\pgfqpoint{3.279344in}{2.007051in}}%
\pgfpathlineto{\pgfqpoint{3.279344in}{2.007051in}}%
\pgfpathlineto{\pgfqpoint{3.279344in}{2.010001in}}%
\pgfpathlineto{\pgfqpoint{3.283885in}{2.010001in}}%
\pgfpathlineto{\pgfqpoint{3.283885in}{2.007051in}}%
\pgfpathmoveto{\pgfqpoint{3.283885in}{2.007051in}}%
\pgfpathlineto{\pgfqpoint{3.283885in}{2.007051in}}%
\pgfpathlineto{\pgfqpoint{3.283885in}{2.010001in}}%
\pgfpathlineto{\pgfqpoint{3.288426in}{2.010001in}}%
\pgfpathlineto{\pgfqpoint{3.288426in}{2.007051in}}%
\pgfpathmoveto{\pgfqpoint{3.288426in}{2.007051in}}%
\pgfpathlineto{\pgfqpoint{3.288426in}{2.007051in}}%
\pgfpathlineto{\pgfqpoint{3.288426in}{2.010001in}}%
\pgfpathlineto{\pgfqpoint{3.292968in}{2.010001in}}%
\pgfpathlineto{\pgfqpoint{3.292968in}{2.007051in}}%
\pgfpathmoveto{\pgfqpoint{3.292968in}{2.007051in}}%
\pgfpathlineto{\pgfqpoint{3.292968in}{2.007051in}}%
\pgfpathlineto{\pgfqpoint{3.292968in}{2.010001in}}%
\pgfpathlineto{\pgfqpoint{3.297509in}{2.010001in}}%
\pgfpathlineto{\pgfqpoint{3.297509in}{2.007051in}}%
\pgfpathmoveto{\pgfqpoint{3.297509in}{2.007051in}}%
\pgfpathlineto{\pgfqpoint{3.297509in}{2.007051in}}%
\pgfpathlineto{\pgfqpoint{3.297509in}{2.010001in}}%
\pgfpathlineto{\pgfqpoint{3.302050in}{2.010001in}}%
\pgfpathlineto{\pgfqpoint{3.302050in}{2.007051in}}%
\pgfpathmoveto{\pgfqpoint{3.302050in}{2.007051in}}%
\pgfpathlineto{\pgfqpoint{3.302050in}{2.007051in}}%
\pgfpathlineto{\pgfqpoint{3.302050in}{2.010001in}}%
\pgfpathlineto{\pgfqpoint{3.306591in}{2.010001in}}%
\pgfpathlineto{\pgfqpoint{3.306591in}{2.007051in}}%
\pgfpathmoveto{\pgfqpoint{3.306591in}{2.007051in}}%
\pgfpathlineto{\pgfqpoint{3.306591in}{2.007051in}}%
\pgfpathlineto{\pgfqpoint{3.306591in}{2.010001in}}%
\pgfpathlineto{\pgfqpoint{3.311132in}{2.010001in}}%
\pgfpathlineto{\pgfqpoint{3.311132in}{2.007051in}}%
\pgfpathmoveto{\pgfqpoint{3.311132in}{2.007051in}}%
\pgfpathlineto{\pgfqpoint{3.311132in}{2.007051in}}%
\pgfpathlineto{\pgfqpoint{3.311132in}{2.010001in}}%
\pgfpathlineto{\pgfqpoint{3.315673in}{2.010001in}}%
\pgfpathlineto{\pgfqpoint{3.315673in}{2.007051in}}%
\pgfpathmoveto{\pgfqpoint{3.315673in}{2.007051in}}%
\pgfpathlineto{\pgfqpoint{3.315673in}{2.007051in}}%
\pgfpathlineto{\pgfqpoint{3.315673in}{2.010001in}}%
\pgfpathlineto{\pgfqpoint{3.320214in}{2.010001in}}%
\pgfpathlineto{\pgfqpoint{3.320214in}{2.007051in}}%
\pgfpathmoveto{\pgfqpoint{3.320214in}{2.007051in}}%
\pgfpathlineto{\pgfqpoint{3.320214in}{2.007051in}}%
\pgfpathlineto{\pgfqpoint{3.320214in}{2.010001in}}%
\pgfpathlineto{\pgfqpoint{3.324755in}{2.010001in}}%
\pgfpathlineto{\pgfqpoint{3.324755in}{2.007051in}}%
\pgfpathmoveto{\pgfqpoint{3.324755in}{2.007051in}}%
\pgfpathlineto{\pgfqpoint{3.324755in}{2.007051in}}%
\pgfpathlineto{\pgfqpoint{3.324755in}{2.010001in}}%
\pgfpathlineto{\pgfqpoint{3.329296in}{2.010001in}}%
\pgfpathlineto{\pgfqpoint{3.329296in}{2.007051in}}%
\pgfpathmoveto{\pgfqpoint{3.329296in}{2.007051in}}%
\pgfpathlineto{\pgfqpoint{3.329296in}{2.007051in}}%
\pgfpathlineto{\pgfqpoint{3.329296in}{2.010001in}}%
\pgfpathlineto{\pgfqpoint{3.333837in}{2.010001in}}%
\pgfpathlineto{\pgfqpoint{3.333837in}{2.007051in}}%
\pgfpathmoveto{\pgfqpoint{3.333837in}{2.007051in}}%
\pgfpathlineto{\pgfqpoint{3.333837in}{2.007051in}}%
\pgfpathlineto{\pgfqpoint{3.333837in}{2.010001in}}%
\pgfpathlineto{\pgfqpoint{3.338378in}{2.010001in}}%
\pgfpathlineto{\pgfqpoint{3.338378in}{2.007051in}}%
\pgfpathmoveto{\pgfqpoint{3.338378in}{2.007051in}}%
\pgfpathlineto{\pgfqpoint{3.338378in}{2.007051in}}%
\pgfpathlineto{\pgfqpoint{3.338378in}{2.010001in}}%
\pgfpathlineto{\pgfqpoint{3.342919in}{2.010001in}}%
\pgfpathlineto{\pgfqpoint{3.342919in}{2.007051in}}%
\pgfpathmoveto{\pgfqpoint{3.342919in}{2.007051in}}%
\pgfpathlineto{\pgfqpoint{3.342919in}{2.007051in}}%
\pgfpathlineto{\pgfqpoint{3.342919in}{2.010001in}}%
\pgfpathlineto{\pgfqpoint{3.347460in}{2.010001in}}%
\pgfpathlineto{\pgfqpoint{3.347460in}{2.007051in}}%
\pgfpathmoveto{\pgfqpoint{3.347460in}{2.007051in}}%
\pgfpathlineto{\pgfqpoint{3.347460in}{2.007051in}}%
\pgfpathlineto{\pgfqpoint{3.347460in}{2.010001in}}%
\pgfpathlineto{\pgfqpoint{3.352001in}{2.010001in}}%
\pgfpathlineto{\pgfqpoint{3.352001in}{2.007051in}}%
\pgfpathmoveto{\pgfqpoint{3.352001in}{2.007051in}}%
\pgfpathlineto{\pgfqpoint{3.352001in}{2.007051in}}%
\pgfpathlineto{\pgfqpoint{3.352001in}{2.010001in}}%
\pgfpathlineto{\pgfqpoint{3.356542in}{2.010001in}}%
\pgfpathlineto{\pgfqpoint{3.356542in}{2.007051in}}%
\pgfpathmoveto{\pgfqpoint{3.356542in}{2.007051in}}%
\pgfpathlineto{\pgfqpoint{3.356542in}{2.007051in}}%
\pgfpathlineto{\pgfqpoint{3.356542in}{2.010001in}}%
\pgfpathlineto{\pgfqpoint{3.361083in}{2.010001in}}%
\pgfpathlineto{\pgfqpoint{3.361083in}{2.007051in}}%
\pgfpathmoveto{\pgfqpoint{3.361083in}{2.007051in}}%
\pgfpathlineto{\pgfqpoint{3.361083in}{2.007051in}}%
\pgfpathlineto{\pgfqpoint{3.361083in}{2.010001in}}%
\pgfpathlineto{\pgfqpoint{3.365624in}{2.010001in}}%
\pgfpathlineto{\pgfqpoint{3.365624in}{2.007051in}}%
\pgfpathmoveto{\pgfqpoint{3.365624in}{2.007051in}}%
\pgfpathlineto{\pgfqpoint{3.365624in}{2.007051in}}%
\pgfpathlineto{\pgfqpoint{3.365624in}{2.010001in}}%
\pgfpathlineto{\pgfqpoint{3.370165in}{2.010001in}}%
\pgfpathlineto{\pgfqpoint{3.370165in}{2.007051in}}%
\pgfpathmoveto{\pgfqpoint{3.370165in}{2.007051in}}%
\pgfpathlineto{\pgfqpoint{3.370165in}{2.007051in}}%
\pgfpathlineto{\pgfqpoint{3.370165in}{2.010001in}}%
\pgfpathlineto{\pgfqpoint{3.374706in}{2.010001in}}%
\pgfpathlineto{\pgfqpoint{3.374706in}{2.007051in}}%
\pgfpathmoveto{\pgfqpoint{3.374706in}{2.007051in}}%
\pgfpathlineto{\pgfqpoint{3.374706in}{2.007051in}}%
\pgfpathlineto{\pgfqpoint{3.374706in}{2.010001in}}%
\pgfpathlineto{\pgfqpoint{3.379248in}{2.010001in}}%
\pgfpathlineto{\pgfqpoint{3.379248in}{2.007051in}}%
\pgfpathmoveto{\pgfqpoint{3.379248in}{2.007051in}}%
\pgfpathlineto{\pgfqpoint{3.379248in}{2.007051in}}%
\pgfpathlineto{\pgfqpoint{3.379248in}{2.010001in}}%
\pgfpathlineto{\pgfqpoint{3.383789in}{2.010001in}}%
\pgfpathlineto{\pgfqpoint{3.383789in}{2.007051in}}%
\pgfpathmoveto{\pgfqpoint{3.383789in}{2.007051in}}%
\pgfpathlineto{\pgfqpoint{3.383789in}{2.007051in}}%
\pgfpathlineto{\pgfqpoint{3.383789in}{2.010001in}}%
\pgfpathlineto{\pgfqpoint{3.388330in}{2.010001in}}%
\pgfpathlineto{\pgfqpoint{3.388330in}{2.007051in}}%
\pgfpathmoveto{\pgfqpoint{3.388330in}{2.007051in}}%
\pgfpathlineto{\pgfqpoint{3.388330in}{2.007051in}}%
\pgfpathlineto{\pgfqpoint{3.388330in}{2.010001in}}%
\pgfpathlineto{\pgfqpoint{3.392871in}{2.010001in}}%
\pgfpathlineto{\pgfqpoint{3.392871in}{2.007051in}}%
\pgfpathmoveto{\pgfqpoint{3.392871in}{2.007051in}}%
\pgfpathlineto{\pgfqpoint{3.392871in}{2.007051in}}%
\pgfpathlineto{\pgfqpoint{3.392871in}{2.010001in}}%
\pgfpathlineto{\pgfqpoint{3.397412in}{2.010001in}}%
\pgfpathlineto{\pgfqpoint{3.397412in}{2.007051in}}%
\pgfpathmoveto{\pgfqpoint{3.397412in}{2.007051in}}%
\pgfpathlineto{\pgfqpoint{3.397412in}{2.007051in}}%
\pgfpathlineto{\pgfqpoint{3.397412in}{2.010001in}}%
\pgfpathlineto{\pgfqpoint{3.401953in}{2.010001in}}%
\pgfpathlineto{\pgfqpoint{3.401953in}{2.007051in}}%
\pgfpathmoveto{\pgfqpoint{3.401953in}{2.007051in}}%
\pgfpathlineto{\pgfqpoint{3.401953in}{2.007051in}}%
\pgfpathlineto{\pgfqpoint{3.401953in}{2.010001in}}%
\pgfpathlineto{\pgfqpoint{3.406494in}{2.010001in}}%
\pgfpathlineto{\pgfqpoint{3.406494in}{2.007051in}}%
\pgfpathmoveto{\pgfqpoint{3.406494in}{2.007051in}}%
\pgfpathlineto{\pgfqpoint{3.406494in}{2.007051in}}%
\pgfpathlineto{\pgfqpoint{3.406494in}{2.010001in}}%
\pgfpathlineto{\pgfqpoint{3.411035in}{2.010001in}}%
\pgfpathlineto{\pgfqpoint{3.411035in}{2.007051in}}%
\pgfpathmoveto{\pgfqpoint{3.411035in}{2.007051in}}%
\pgfpathlineto{\pgfqpoint{3.411035in}{2.007051in}}%
\pgfpathlineto{\pgfqpoint{3.411035in}{2.010001in}}%
\pgfpathlineto{\pgfqpoint{3.415576in}{2.010001in}}%
\pgfpathlineto{\pgfqpoint{3.415576in}{2.007051in}}%
\pgfpathmoveto{\pgfqpoint{3.415576in}{2.007051in}}%
\pgfpathlineto{\pgfqpoint{3.415576in}{2.007051in}}%
\pgfpathlineto{\pgfqpoint{3.415576in}{2.010001in}}%
\pgfpathlineto{\pgfqpoint{3.420117in}{2.010001in}}%
\pgfpathlineto{\pgfqpoint{3.420117in}{2.007051in}}%
\pgfpathmoveto{\pgfqpoint{3.420117in}{2.007051in}}%
\pgfpathlineto{\pgfqpoint{3.420117in}{2.007051in}}%
\pgfpathlineto{\pgfqpoint{3.420117in}{2.010001in}}%
\pgfpathlineto{\pgfqpoint{3.424658in}{2.010001in}}%
\pgfpathlineto{\pgfqpoint{3.424658in}{2.007051in}}%
\pgfpathmoveto{\pgfqpoint{3.424658in}{2.007051in}}%
\pgfpathlineto{\pgfqpoint{3.424658in}{2.007051in}}%
\pgfpathlineto{\pgfqpoint{3.424658in}{2.010001in}}%
\pgfpathlineto{\pgfqpoint{3.429199in}{2.010001in}}%
\pgfpathlineto{\pgfqpoint{3.429199in}{2.007051in}}%
\pgfpathmoveto{\pgfqpoint{3.429199in}{2.007051in}}%
\pgfpathlineto{\pgfqpoint{3.429199in}{2.007051in}}%
\pgfpathlineto{\pgfqpoint{3.429199in}{2.010001in}}%
\pgfpathlineto{\pgfqpoint{3.433740in}{2.010001in}}%
\pgfpathlineto{\pgfqpoint{3.433740in}{2.007051in}}%
\pgfpathmoveto{\pgfqpoint{3.433740in}{2.007051in}}%
\pgfpathlineto{\pgfqpoint{3.433740in}{2.007051in}}%
\pgfpathlineto{\pgfqpoint{3.433740in}{2.010001in}}%
\pgfpathlineto{\pgfqpoint{3.438281in}{2.010001in}}%
\pgfpathlineto{\pgfqpoint{3.438281in}{2.007051in}}%
\pgfpathmoveto{\pgfqpoint{3.438281in}{2.007051in}}%
\pgfpathlineto{\pgfqpoint{3.438281in}{2.007051in}}%
\pgfpathlineto{\pgfqpoint{3.438281in}{2.010001in}}%
\pgfpathlineto{\pgfqpoint{3.442822in}{2.010001in}}%
\pgfpathlineto{\pgfqpoint{3.442822in}{2.007051in}}%
\pgfpathmoveto{\pgfqpoint{3.442822in}{2.007051in}}%
\pgfpathlineto{\pgfqpoint{3.442822in}{2.007051in}}%
\pgfpathlineto{\pgfqpoint{3.442822in}{2.010001in}}%
\pgfpathlineto{\pgfqpoint{3.447363in}{2.010001in}}%
\pgfpathlineto{\pgfqpoint{3.447363in}{2.007051in}}%
\pgfpathmoveto{\pgfqpoint{3.447363in}{2.007051in}}%
\pgfpathlineto{\pgfqpoint{3.447363in}{2.007051in}}%
\pgfpathlineto{\pgfqpoint{3.447363in}{2.010001in}}%
\pgfpathlineto{\pgfqpoint{3.451905in}{2.010001in}}%
\pgfpathlineto{\pgfqpoint{3.451905in}{2.007051in}}%
\pgfpathmoveto{\pgfqpoint{3.451905in}{2.007051in}}%
\pgfpathlineto{\pgfqpoint{3.451905in}{2.007051in}}%
\pgfpathlineto{\pgfqpoint{3.451905in}{2.010001in}}%
\pgfpathlineto{\pgfqpoint{3.456446in}{2.010001in}}%
\pgfpathlineto{\pgfqpoint{3.456446in}{2.007051in}}%
\pgfpathmoveto{\pgfqpoint{3.456446in}{2.007051in}}%
\pgfpathlineto{\pgfqpoint{3.456446in}{2.007051in}}%
\pgfpathlineto{\pgfqpoint{3.456446in}{2.010001in}}%
\pgfpathlineto{\pgfqpoint{3.460987in}{2.010001in}}%
\pgfpathlineto{\pgfqpoint{3.460987in}{2.007051in}}%
\pgfpathmoveto{\pgfqpoint{3.460987in}{2.007051in}}%
\pgfpathlineto{\pgfqpoint{3.460987in}{2.007051in}}%
\pgfpathlineto{\pgfqpoint{3.460987in}{2.010001in}}%
\pgfpathlineto{\pgfqpoint{3.465528in}{2.010001in}}%
\pgfpathlineto{\pgfqpoint{3.465528in}{2.007051in}}%
\pgfpathmoveto{\pgfqpoint{3.465528in}{2.007051in}}%
\pgfpathlineto{\pgfqpoint{3.465528in}{2.007051in}}%
\pgfpathlineto{\pgfqpoint{3.465528in}{2.010001in}}%
\pgfpathlineto{\pgfqpoint{3.470069in}{2.010001in}}%
\pgfpathlineto{\pgfqpoint{3.470069in}{2.007051in}}%
\pgfpathmoveto{\pgfqpoint{3.470069in}{2.007051in}}%
\pgfpathlineto{\pgfqpoint{3.470069in}{2.007051in}}%
\pgfpathlineto{\pgfqpoint{3.470069in}{2.010001in}}%
\pgfpathlineto{\pgfqpoint{3.474610in}{2.010001in}}%
\pgfpathlineto{\pgfqpoint{3.474610in}{2.007051in}}%
\pgfpathmoveto{\pgfqpoint{3.474610in}{2.007051in}}%
\pgfpathlineto{\pgfqpoint{3.474610in}{2.007051in}}%
\pgfpathlineto{\pgfqpoint{3.474610in}{2.010001in}}%
\pgfpathlineto{\pgfqpoint{3.479151in}{2.010001in}}%
\pgfpathlineto{\pgfqpoint{3.479151in}{2.007051in}}%
\pgfpathmoveto{\pgfqpoint{3.479151in}{2.007051in}}%
\pgfpathlineto{\pgfqpoint{3.479151in}{2.007051in}}%
\pgfpathlineto{\pgfqpoint{3.479151in}{2.010001in}}%
\pgfpathlineto{\pgfqpoint{3.483692in}{2.010001in}}%
\pgfpathlineto{\pgfqpoint{3.483692in}{2.007051in}}%
\pgfpathmoveto{\pgfqpoint{3.483692in}{2.007051in}}%
\pgfpathlineto{\pgfqpoint{3.483692in}{2.007051in}}%
\pgfpathlineto{\pgfqpoint{3.483692in}{2.010001in}}%
\pgfpathlineto{\pgfqpoint{3.488233in}{2.010001in}}%
\pgfpathlineto{\pgfqpoint{3.488233in}{2.007051in}}%
\pgfpathmoveto{\pgfqpoint{3.488233in}{2.007051in}}%
\pgfpathlineto{\pgfqpoint{3.488233in}{2.007051in}}%
\pgfpathlineto{\pgfqpoint{3.488233in}{2.010001in}}%
\pgfpathlineto{\pgfqpoint{3.492774in}{2.010001in}}%
\pgfpathlineto{\pgfqpoint{3.492774in}{2.007051in}}%
\pgfpathmoveto{\pgfqpoint{3.492774in}{2.007051in}}%
\pgfpathlineto{\pgfqpoint{3.492774in}{2.007051in}}%
\pgfpathlineto{\pgfqpoint{3.492774in}{2.010001in}}%
\pgfpathlineto{\pgfqpoint{3.497315in}{2.010001in}}%
\pgfpathlineto{\pgfqpoint{3.497315in}{2.007051in}}%
\pgfpathmoveto{\pgfqpoint{3.497315in}{2.007051in}}%
\pgfpathlineto{\pgfqpoint{3.497315in}{2.007051in}}%
\pgfpathlineto{\pgfqpoint{3.497315in}{2.010001in}}%
\pgfpathlineto{\pgfqpoint{3.501856in}{2.010001in}}%
\pgfpathlineto{\pgfqpoint{3.501856in}{2.007051in}}%
\pgfpathmoveto{\pgfqpoint{3.501856in}{2.007051in}}%
\pgfpathlineto{\pgfqpoint{3.501856in}{2.007051in}}%
\pgfpathlineto{\pgfqpoint{3.501856in}{2.010001in}}%
\pgfpathlineto{\pgfqpoint{3.506397in}{2.010001in}}%
\pgfpathlineto{\pgfqpoint{3.506397in}{2.007051in}}%
\pgfpathmoveto{\pgfqpoint{3.506397in}{2.007051in}}%
\pgfpathlineto{\pgfqpoint{3.506397in}{2.007051in}}%
\pgfpathlineto{\pgfqpoint{3.506397in}{2.010001in}}%
\pgfpathlineto{\pgfqpoint{3.510938in}{2.010001in}}%
\pgfpathlineto{\pgfqpoint{3.510938in}{2.007051in}}%
\pgfpathmoveto{\pgfqpoint{3.510938in}{2.007051in}}%
\pgfpathlineto{\pgfqpoint{3.510938in}{2.007051in}}%
\pgfpathlineto{\pgfqpoint{3.510938in}{2.010001in}}%
\pgfpathlineto{\pgfqpoint{3.515479in}{2.010001in}}%
\pgfpathlineto{\pgfqpoint{3.515479in}{2.007051in}}%
\pgfpathmoveto{\pgfqpoint{3.515479in}{2.007051in}}%
\pgfpathlineto{\pgfqpoint{3.515479in}{2.007051in}}%
\pgfpathlineto{\pgfqpoint{3.515479in}{2.010001in}}%
\pgfpathlineto{\pgfqpoint{3.520020in}{2.010001in}}%
\pgfpathlineto{\pgfqpoint{3.520020in}{2.007051in}}%
\pgfpathmoveto{\pgfqpoint{3.520020in}{2.007051in}}%
\pgfpathlineto{\pgfqpoint{3.520020in}{2.007051in}}%
\pgfpathlineto{\pgfqpoint{3.520020in}{2.010001in}}%
\pgfpathlineto{\pgfqpoint{3.524561in}{2.010001in}}%
\pgfpathlineto{\pgfqpoint{3.524561in}{2.007051in}}%
\pgfpathmoveto{\pgfqpoint{3.524561in}{2.007051in}}%
\pgfpathlineto{\pgfqpoint{3.524561in}{2.007051in}}%
\pgfpathlineto{\pgfqpoint{3.524561in}{2.010001in}}%
\pgfpathlineto{\pgfqpoint{3.529102in}{2.010001in}}%
\pgfpathlineto{\pgfqpoint{3.529102in}{2.007051in}}%
\pgfpathmoveto{\pgfqpoint{3.529102in}{2.007051in}}%
\pgfpathlineto{\pgfqpoint{3.529102in}{2.007051in}}%
\pgfpathlineto{\pgfqpoint{3.529102in}{2.010001in}}%
\pgfpathlineto{\pgfqpoint{3.533643in}{2.010001in}}%
\pgfpathlineto{\pgfqpoint{3.533643in}{2.007051in}}%
\pgfpathmoveto{\pgfqpoint{3.533643in}{2.007051in}}%
\pgfpathlineto{\pgfqpoint{3.533643in}{2.007051in}}%
\pgfpathlineto{\pgfqpoint{3.533643in}{2.010001in}}%
\pgfpathlineto{\pgfqpoint{3.538184in}{2.010001in}}%
\pgfpathlineto{\pgfqpoint{3.538184in}{2.007051in}}%
\pgfpathmoveto{\pgfqpoint{3.538184in}{2.007051in}}%
\pgfpathlineto{\pgfqpoint{3.538184in}{2.007051in}}%
\pgfpathlineto{\pgfqpoint{3.538184in}{2.010001in}}%
\pgfpathlineto{\pgfqpoint{3.542725in}{2.010001in}}%
\pgfpathlineto{\pgfqpoint{3.542725in}{2.007051in}}%
\pgfpathmoveto{\pgfqpoint{3.542725in}{2.007051in}}%
\pgfpathlineto{\pgfqpoint{3.542725in}{2.007051in}}%
\pgfpathlineto{\pgfqpoint{3.542725in}{2.010001in}}%
\pgfpathlineto{\pgfqpoint{3.547266in}{2.010001in}}%
\pgfpathlineto{\pgfqpoint{3.547266in}{2.007051in}}%
\pgfpathmoveto{\pgfqpoint{3.547266in}{2.007051in}}%
\pgfpathlineto{\pgfqpoint{3.547266in}{2.007051in}}%
\pgfpathlineto{\pgfqpoint{3.547266in}{2.010001in}}%
\pgfpathlineto{\pgfqpoint{3.551806in}{2.010001in}}%
\pgfpathlineto{\pgfqpoint{3.551806in}{2.007051in}}%
\pgfpathmoveto{\pgfqpoint{3.551806in}{2.007051in}}%
\pgfpathlineto{\pgfqpoint{3.551806in}{2.007051in}}%
\pgfpathlineto{\pgfqpoint{3.551806in}{2.010001in}}%
\pgfpathlineto{\pgfqpoint{3.556347in}{2.010001in}}%
\pgfpathlineto{\pgfqpoint{3.556347in}{2.007051in}}%
\pgfpathmoveto{\pgfqpoint{3.556347in}{2.007051in}}%
\pgfpathlineto{\pgfqpoint{3.556347in}{2.007051in}}%
\pgfpathlineto{\pgfqpoint{3.556347in}{2.010001in}}%
\pgfpathlineto{\pgfqpoint{3.560888in}{2.010001in}}%
\pgfpathlineto{\pgfqpoint{3.560888in}{2.007051in}}%
\pgfpathmoveto{\pgfqpoint{3.560888in}{2.007051in}}%
\pgfpathlineto{\pgfqpoint{3.560888in}{2.007051in}}%
\pgfpathlineto{\pgfqpoint{3.560888in}{2.010001in}}%
\pgfpathlineto{\pgfqpoint{3.565429in}{2.010001in}}%
\pgfpathlineto{\pgfqpoint{3.565429in}{2.007051in}}%
\pgfpathmoveto{\pgfqpoint{3.565429in}{2.007051in}}%
\pgfpathlineto{\pgfqpoint{3.565429in}{2.007051in}}%
\pgfpathlineto{\pgfqpoint{3.565429in}{2.010001in}}%
\pgfpathlineto{\pgfqpoint{3.569970in}{2.010001in}}%
\pgfpathlineto{\pgfqpoint{3.569970in}{2.007051in}}%
\pgfpathmoveto{\pgfqpoint{3.569970in}{2.007051in}}%
\pgfpathlineto{\pgfqpoint{3.569970in}{2.007051in}}%
\pgfpathlineto{\pgfqpoint{3.569970in}{2.010001in}}%
\pgfpathlineto{\pgfqpoint{3.574511in}{2.010001in}}%
\pgfpathlineto{\pgfqpoint{3.574511in}{2.007051in}}%
\pgfpathmoveto{\pgfqpoint{3.574511in}{2.007051in}}%
\pgfpathlineto{\pgfqpoint{3.574511in}{2.007051in}}%
\pgfpathlineto{\pgfqpoint{3.574511in}{2.010001in}}%
\pgfpathlineto{\pgfqpoint{3.579052in}{2.010001in}}%
\pgfpathlineto{\pgfqpoint{3.579052in}{2.007051in}}%
\pgfpathmoveto{\pgfqpoint{3.579052in}{2.007051in}}%
\pgfpathlineto{\pgfqpoint{3.579052in}{2.007051in}}%
\pgfpathlineto{\pgfqpoint{3.579052in}{2.010001in}}%
\pgfpathlineto{\pgfqpoint{3.583593in}{2.010001in}}%
\pgfpathlineto{\pgfqpoint{3.583593in}{2.007051in}}%
\pgfpathmoveto{\pgfqpoint{3.583593in}{2.007051in}}%
\pgfpathlineto{\pgfqpoint{3.583593in}{2.007051in}}%
\pgfpathlineto{\pgfqpoint{3.583593in}{2.010001in}}%
\pgfpathlineto{\pgfqpoint{3.588134in}{2.010001in}}%
\pgfpathlineto{\pgfqpoint{3.588134in}{2.007051in}}%
\pgfpathmoveto{\pgfqpoint{3.588134in}{2.007051in}}%
\pgfpathlineto{\pgfqpoint{3.588134in}{2.007051in}}%
\pgfpathlineto{\pgfqpoint{3.588134in}{2.010001in}}%
\pgfpathlineto{\pgfqpoint{3.592674in}{2.010001in}}%
\pgfpathlineto{\pgfqpoint{3.592674in}{2.007051in}}%
\pgfpathmoveto{\pgfqpoint{3.592674in}{2.007051in}}%
\pgfpathlineto{\pgfqpoint{3.592674in}{2.007051in}}%
\pgfpathlineto{\pgfqpoint{3.592674in}{2.010001in}}%
\pgfpathlineto{\pgfqpoint{3.597215in}{2.010001in}}%
\pgfpathlineto{\pgfqpoint{3.597215in}{2.007051in}}%
\pgfpathmoveto{\pgfqpoint{3.597215in}{2.007051in}}%
\pgfpathlineto{\pgfqpoint{3.597215in}{2.007051in}}%
\pgfpathlineto{\pgfqpoint{3.597215in}{2.010001in}}%
\pgfpathlineto{\pgfqpoint{3.601756in}{2.010001in}}%
\pgfpathlineto{\pgfqpoint{3.601756in}{2.007051in}}%
\pgfpathmoveto{\pgfqpoint{3.601756in}{2.007051in}}%
\pgfpathlineto{\pgfqpoint{3.601756in}{2.007051in}}%
\pgfpathlineto{\pgfqpoint{3.601756in}{2.010001in}}%
\pgfpathlineto{\pgfqpoint{3.606297in}{2.010001in}}%
\pgfpathlineto{\pgfqpoint{3.606297in}{2.007051in}}%
\pgfpathmoveto{\pgfqpoint{3.606297in}{2.007051in}}%
\pgfpathlineto{\pgfqpoint{3.606297in}{2.007051in}}%
\pgfpathlineto{\pgfqpoint{3.606297in}{2.010001in}}%
\pgfpathlineto{\pgfqpoint{3.610838in}{2.010001in}}%
\pgfpathlineto{\pgfqpoint{3.610838in}{2.007051in}}%
\pgfpathmoveto{\pgfqpoint{3.610838in}{2.007051in}}%
\pgfpathlineto{\pgfqpoint{3.610838in}{2.007051in}}%
\pgfpathlineto{\pgfqpoint{3.610838in}{2.010001in}}%
\pgfpathlineto{\pgfqpoint{3.615379in}{2.010001in}}%
\pgfpathlineto{\pgfqpoint{3.615379in}{2.007051in}}%
\pgfpathmoveto{\pgfqpoint{3.615379in}{2.007051in}}%
\pgfpathlineto{\pgfqpoint{3.615379in}{2.007051in}}%
\pgfpathlineto{\pgfqpoint{3.615379in}{2.010001in}}%
\pgfpathlineto{\pgfqpoint{3.619920in}{2.010001in}}%
\pgfpathlineto{\pgfqpoint{3.619920in}{2.007051in}}%
\pgfpathmoveto{\pgfqpoint{3.619920in}{2.007051in}}%
\pgfpathlineto{\pgfqpoint{3.619920in}{2.007051in}}%
\pgfpathlineto{\pgfqpoint{3.619920in}{2.010001in}}%
\pgfpathlineto{\pgfqpoint{3.624461in}{2.010001in}}%
\pgfpathlineto{\pgfqpoint{3.624461in}{2.007051in}}%
\pgfpathmoveto{\pgfqpoint{3.624461in}{2.007051in}}%
\pgfpathlineto{\pgfqpoint{3.624461in}{2.007051in}}%
\pgfpathlineto{\pgfqpoint{3.624461in}{2.010001in}}%
\pgfpathlineto{\pgfqpoint{3.629002in}{2.010001in}}%
\pgfpathlineto{\pgfqpoint{3.629002in}{2.007051in}}%
\pgfpathmoveto{\pgfqpoint{3.629002in}{2.007051in}}%
\pgfpathlineto{\pgfqpoint{3.629002in}{2.007051in}}%
\pgfpathlineto{\pgfqpoint{3.629002in}{2.010001in}}%
\pgfpathlineto{\pgfqpoint{3.633542in}{2.010001in}}%
\pgfpathlineto{\pgfqpoint{3.633542in}{2.007051in}}%
\pgfpathmoveto{\pgfqpoint{3.633542in}{2.007051in}}%
\pgfpathlineto{\pgfqpoint{3.633542in}{2.007051in}}%
\pgfpathlineto{\pgfqpoint{3.633542in}{2.010001in}}%
\pgfpathlineto{\pgfqpoint{3.638083in}{2.010001in}}%
\pgfpathlineto{\pgfqpoint{3.638083in}{2.007051in}}%
\pgfpathmoveto{\pgfqpoint{3.638083in}{2.007051in}}%
\pgfpathlineto{\pgfqpoint{3.638083in}{2.007051in}}%
\pgfpathlineto{\pgfqpoint{3.638083in}{2.010001in}}%
\pgfpathlineto{\pgfqpoint{3.642624in}{2.010001in}}%
\pgfpathlineto{\pgfqpoint{3.642624in}{2.007051in}}%
\pgfpathmoveto{\pgfqpoint{3.642624in}{2.007051in}}%
\pgfpathlineto{\pgfqpoint{3.642624in}{2.007051in}}%
\pgfpathlineto{\pgfqpoint{3.642624in}{2.010001in}}%
\pgfpathlineto{\pgfqpoint{3.647165in}{2.010001in}}%
\pgfpathlineto{\pgfqpoint{3.647165in}{2.007051in}}%
\pgfpathmoveto{\pgfqpoint{3.647165in}{2.007051in}}%
\pgfpathlineto{\pgfqpoint{3.647165in}{2.007051in}}%
\pgfpathlineto{\pgfqpoint{3.647165in}{2.010001in}}%
\pgfpathlineto{\pgfqpoint{3.651706in}{2.010001in}}%
\pgfpathlineto{\pgfqpoint{3.651706in}{2.007051in}}%
\pgfpathmoveto{\pgfqpoint{3.651706in}{2.007051in}}%
\pgfpathlineto{\pgfqpoint{3.651706in}{2.007051in}}%
\pgfpathlineto{\pgfqpoint{3.651706in}{2.010001in}}%
\pgfpathlineto{\pgfqpoint{3.656247in}{2.010001in}}%
\pgfpathlineto{\pgfqpoint{3.656247in}{2.007051in}}%
\pgfpathmoveto{\pgfqpoint{3.656247in}{2.007051in}}%
\pgfpathlineto{\pgfqpoint{3.656247in}{2.007051in}}%
\pgfpathlineto{\pgfqpoint{3.656247in}{2.010001in}}%
\pgfpathlineto{\pgfqpoint{3.660788in}{2.010001in}}%
\pgfpathlineto{\pgfqpoint{3.660788in}{2.007051in}}%
\pgfpathmoveto{\pgfqpoint{3.660788in}{2.007051in}}%
\pgfpathlineto{\pgfqpoint{3.660788in}{2.007051in}}%
\pgfpathlineto{\pgfqpoint{3.660788in}{2.010001in}}%
\pgfpathlineto{\pgfqpoint{3.665329in}{2.010001in}}%
\pgfpathlineto{\pgfqpoint{3.665329in}{2.007051in}}%
\pgfpathmoveto{\pgfqpoint{3.665329in}{2.007051in}}%
\pgfpathlineto{\pgfqpoint{3.665329in}{2.007051in}}%
\pgfpathlineto{\pgfqpoint{3.665329in}{2.010001in}}%
\pgfpathlineto{\pgfqpoint{3.669870in}{2.010001in}}%
\pgfpathlineto{\pgfqpoint{3.669870in}{2.007051in}}%
\pgfpathmoveto{\pgfqpoint{3.669870in}{2.007051in}}%
\pgfpathlineto{\pgfqpoint{3.669870in}{2.007051in}}%
\pgfpathlineto{\pgfqpoint{3.669870in}{2.010001in}}%
\pgfpathlineto{\pgfqpoint{3.674412in}{2.010001in}}%
\pgfpathlineto{\pgfqpoint{3.674412in}{2.007051in}}%
\pgfpathmoveto{\pgfqpoint{3.674412in}{2.007051in}}%
\pgfpathlineto{\pgfqpoint{3.674412in}{2.007051in}}%
\pgfpathlineto{\pgfqpoint{3.674412in}{2.010001in}}%
\pgfpathlineto{\pgfqpoint{3.678953in}{2.010001in}}%
\pgfpathlineto{\pgfqpoint{3.678953in}{2.007051in}}%
\pgfpathmoveto{\pgfqpoint{3.678953in}{2.007051in}}%
\pgfpathlineto{\pgfqpoint{3.678953in}{2.007051in}}%
\pgfpathlineto{\pgfqpoint{3.678953in}{2.010001in}}%
\pgfpathlineto{\pgfqpoint{3.683494in}{2.010001in}}%
\pgfpathlineto{\pgfqpoint{3.683494in}{2.007051in}}%
\pgfpathmoveto{\pgfqpoint{3.683494in}{2.007051in}}%
\pgfpathlineto{\pgfqpoint{3.683494in}{2.007051in}}%
\pgfpathlineto{\pgfqpoint{3.683494in}{2.010001in}}%
\pgfpathlineto{\pgfqpoint{3.688035in}{2.010001in}}%
\pgfpathlineto{\pgfqpoint{3.688035in}{2.007051in}}%
\pgfpathmoveto{\pgfqpoint{3.688035in}{2.007051in}}%
\pgfpathlineto{\pgfqpoint{3.688035in}{2.007051in}}%
\pgfpathlineto{\pgfqpoint{3.688035in}{2.010001in}}%
\pgfpathlineto{\pgfqpoint{3.692576in}{2.010001in}}%
\pgfpathlineto{\pgfqpoint{3.692576in}{2.007051in}}%
\pgfpathmoveto{\pgfqpoint{3.692576in}{2.007051in}}%
\pgfpathlineto{\pgfqpoint{3.692576in}{2.007051in}}%
\pgfpathlineto{\pgfqpoint{3.692576in}{2.010001in}}%
\pgfpathlineto{\pgfqpoint{3.697118in}{2.010001in}}%
\pgfpathlineto{\pgfqpoint{3.697118in}{2.007051in}}%
\pgfpathmoveto{\pgfqpoint{3.697118in}{2.007051in}}%
\pgfpathlineto{\pgfqpoint{3.697118in}{2.007051in}}%
\pgfpathlineto{\pgfqpoint{3.697118in}{2.010001in}}%
\pgfpathlineto{\pgfqpoint{3.701659in}{2.010001in}}%
\pgfpathlineto{\pgfqpoint{3.701659in}{2.007051in}}%
\pgfpathmoveto{\pgfqpoint{3.701659in}{2.007051in}}%
\pgfpathlineto{\pgfqpoint{3.701659in}{2.007051in}}%
\pgfpathlineto{\pgfqpoint{3.701659in}{2.010001in}}%
\pgfpathlineto{\pgfqpoint{3.706200in}{2.010001in}}%
\pgfpathlineto{\pgfqpoint{3.706200in}{2.007051in}}%
\pgfpathmoveto{\pgfqpoint{3.706200in}{2.007051in}}%
\pgfpathlineto{\pgfqpoint{3.706200in}{2.007051in}}%
\pgfpathlineto{\pgfqpoint{3.706200in}{2.010001in}}%
\pgfpathlineto{\pgfqpoint{3.710741in}{2.010001in}}%
\pgfpathlineto{\pgfqpoint{3.710741in}{2.007051in}}%
\pgfpathmoveto{\pgfqpoint{3.710741in}{2.007051in}}%
\pgfpathlineto{\pgfqpoint{3.710741in}{2.007051in}}%
\pgfpathlineto{\pgfqpoint{3.710741in}{2.010001in}}%
\pgfpathlineto{\pgfqpoint{3.715282in}{2.010001in}}%
\pgfpathlineto{\pgfqpoint{3.715282in}{2.007051in}}%
\pgfpathmoveto{\pgfqpoint{3.715282in}{2.007051in}}%
\pgfpathlineto{\pgfqpoint{3.715282in}{2.007051in}}%
\pgfpathlineto{\pgfqpoint{3.715282in}{2.010001in}}%
\pgfpathlineto{\pgfqpoint{3.719824in}{2.010001in}}%
\pgfpathlineto{\pgfqpoint{3.719824in}{2.007051in}}%
\pgfpathmoveto{\pgfqpoint{3.719824in}{2.007051in}}%
\pgfpathlineto{\pgfqpoint{3.719824in}{2.007051in}}%
\pgfpathlineto{\pgfqpoint{3.719824in}{2.010001in}}%
\pgfpathlineto{\pgfqpoint{3.724365in}{2.010001in}}%
\pgfpathlineto{\pgfqpoint{3.724365in}{2.007051in}}%
\pgfpathmoveto{\pgfqpoint{3.724365in}{2.007051in}}%
\pgfpathlineto{\pgfqpoint{3.724365in}{2.007051in}}%
\pgfpathlineto{\pgfqpoint{3.724365in}{2.010001in}}%
\pgfpathlineto{\pgfqpoint{3.728906in}{2.010001in}}%
\pgfpathlineto{\pgfqpoint{3.728906in}{2.007051in}}%
\pgfpathmoveto{\pgfqpoint{3.728906in}{2.007051in}}%
\pgfpathlineto{\pgfqpoint{3.728906in}{2.007051in}}%
\pgfpathlineto{\pgfqpoint{3.728906in}{2.010001in}}%
\pgfpathlineto{\pgfqpoint{3.733447in}{2.010001in}}%
\pgfpathlineto{\pgfqpoint{3.733447in}{2.007051in}}%
\pgfpathmoveto{\pgfqpoint{3.733447in}{2.007051in}}%
\pgfpathlineto{\pgfqpoint{3.733447in}{2.007051in}}%
\pgfpathlineto{\pgfqpoint{3.733447in}{2.010001in}}%
\pgfpathlineto{\pgfqpoint{3.737988in}{2.010001in}}%
\pgfpathlineto{\pgfqpoint{3.737988in}{2.007051in}}%
\pgfpathmoveto{\pgfqpoint{3.737988in}{2.007051in}}%
\pgfpathlineto{\pgfqpoint{3.737988in}{2.007051in}}%
\pgfpathlineto{\pgfqpoint{3.737988in}{2.010001in}}%
\pgfpathlineto{\pgfqpoint{3.742529in}{2.010001in}}%
\pgfpathlineto{\pgfqpoint{3.742529in}{2.007051in}}%
\pgfpathmoveto{\pgfqpoint{3.742529in}{2.007051in}}%
\pgfpathlineto{\pgfqpoint{3.742529in}{2.007051in}}%
\pgfpathlineto{\pgfqpoint{3.742529in}{2.010001in}}%
\pgfpathlineto{\pgfqpoint{3.747071in}{2.010001in}}%
\pgfpathlineto{\pgfqpoint{3.747071in}{2.007051in}}%
\pgfpathmoveto{\pgfqpoint{3.747071in}{2.007051in}}%
\pgfpathlineto{\pgfqpoint{3.747071in}{2.007051in}}%
\pgfpathlineto{\pgfqpoint{3.747071in}{2.010001in}}%
\pgfpathlineto{\pgfqpoint{3.751612in}{2.010001in}}%
\pgfpathlineto{\pgfqpoint{3.751612in}{2.007051in}}%
\pgfpathmoveto{\pgfqpoint{3.751612in}{2.007051in}}%
\pgfpathlineto{\pgfqpoint{3.751612in}{2.007051in}}%
\pgfpathlineto{\pgfqpoint{3.751612in}{2.010001in}}%
\pgfpathlineto{\pgfqpoint{3.756153in}{2.010001in}}%
\pgfpathlineto{\pgfqpoint{3.756153in}{2.007051in}}%
\pgfpathmoveto{\pgfqpoint{3.756153in}{2.007051in}}%
\pgfpathlineto{\pgfqpoint{3.756153in}{2.007051in}}%
\pgfpathlineto{\pgfqpoint{3.756153in}{2.010001in}}%
\pgfpathlineto{\pgfqpoint{3.760694in}{2.010001in}}%
\pgfpathlineto{\pgfqpoint{3.760694in}{2.007051in}}%
\pgfpathmoveto{\pgfqpoint{3.760694in}{2.007051in}}%
\pgfpathlineto{\pgfqpoint{3.760694in}{2.007051in}}%
\pgfpathlineto{\pgfqpoint{3.760694in}{2.010001in}}%
\pgfpathlineto{\pgfqpoint{3.765235in}{2.010001in}}%
\pgfpathlineto{\pgfqpoint{3.765235in}{2.007051in}}%
\pgfpathmoveto{\pgfqpoint{3.765235in}{2.007051in}}%
\pgfpathlineto{\pgfqpoint{3.765235in}{2.007051in}}%
\pgfpathlineto{\pgfqpoint{3.765235in}{2.010001in}}%
\pgfpathlineto{\pgfqpoint{3.769777in}{2.010001in}}%
\pgfpathlineto{\pgfqpoint{3.769777in}{2.007051in}}%
\pgfpathmoveto{\pgfqpoint{3.769777in}{2.007051in}}%
\pgfpathlineto{\pgfqpoint{3.769777in}{2.007051in}}%
\pgfpathlineto{\pgfqpoint{3.769777in}{2.010001in}}%
\pgfpathlineto{\pgfqpoint{3.774318in}{2.010001in}}%
\pgfpathlineto{\pgfqpoint{3.774318in}{2.007051in}}%
\pgfpathmoveto{\pgfqpoint{3.774318in}{2.007051in}}%
\pgfpathlineto{\pgfqpoint{3.774318in}{2.007051in}}%
\pgfpathlineto{\pgfqpoint{3.774318in}{2.010001in}}%
\pgfpathlineto{\pgfqpoint{3.778859in}{2.010001in}}%
\pgfpathlineto{\pgfqpoint{3.778859in}{2.007051in}}%
\pgfpathmoveto{\pgfqpoint{3.778859in}{2.007051in}}%
\pgfpathlineto{\pgfqpoint{3.778859in}{2.007051in}}%
\pgfpathlineto{\pgfqpoint{3.778859in}{2.010001in}}%
\pgfpathlineto{\pgfqpoint{3.783400in}{2.010001in}}%
\pgfpathlineto{\pgfqpoint{3.783400in}{2.007051in}}%
\pgfpathmoveto{\pgfqpoint{3.783400in}{2.007051in}}%
\pgfpathlineto{\pgfqpoint{3.783400in}{2.007051in}}%
\pgfpathlineto{\pgfqpoint{3.783400in}{2.010001in}}%
\pgfpathlineto{\pgfqpoint{3.787941in}{2.010001in}}%
\pgfpathlineto{\pgfqpoint{3.787941in}{2.007051in}}%
\pgfpathmoveto{\pgfqpoint{3.787941in}{2.007051in}}%
\pgfpathlineto{\pgfqpoint{3.787941in}{2.007051in}}%
\pgfpathlineto{\pgfqpoint{3.787941in}{2.010001in}}%
\pgfpathlineto{\pgfqpoint{3.792482in}{2.010001in}}%
\pgfpathlineto{\pgfqpoint{3.792482in}{2.007051in}}%
\pgfpathmoveto{\pgfqpoint{3.792482in}{2.007051in}}%
\pgfpathlineto{\pgfqpoint{3.792482in}{2.007051in}}%
\pgfpathlineto{\pgfqpoint{3.792482in}{2.010001in}}%
\pgfpathlineto{\pgfqpoint{3.797024in}{2.010001in}}%
\pgfpathlineto{\pgfqpoint{3.797024in}{2.007051in}}%
\pgfpathmoveto{\pgfqpoint{3.797024in}{2.007051in}}%
\pgfpathlineto{\pgfqpoint{3.797024in}{2.007051in}}%
\pgfpathlineto{\pgfqpoint{3.797024in}{2.010001in}}%
\pgfpathlineto{\pgfqpoint{3.801565in}{2.010001in}}%
\pgfpathlineto{\pgfqpoint{3.801565in}{2.007051in}}%
\pgfpathmoveto{\pgfqpoint{3.801565in}{2.007051in}}%
\pgfpathlineto{\pgfqpoint{3.801565in}{2.007051in}}%
\pgfpathlineto{\pgfqpoint{3.801565in}{2.010001in}}%
\pgfpathlineto{\pgfqpoint{3.806106in}{2.010001in}}%
\pgfpathlineto{\pgfqpoint{3.806106in}{2.007051in}}%
\pgfpathmoveto{\pgfqpoint{3.806106in}{2.007051in}}%
\pgfpathlineto{\pgfqpoint{3.806106in}{2.007051in}}%
\pgfpathlineto{\pgfqpoint{3.806106in}{2.010001in}}%
\pgfpathlineto{\pgfqpoint{3.810647in}{2.010001in}}%
\pgfpathlineto{\pgfqpoint{3.810647in}{2.007051in}}%
\pgfpathmoveto{\pgfqpoint{3.810647in}{2.007051in}}%
\pgfpathlineto{\pgfqpoint{3.810647in}{2.007051in}}%
\pgfpathlineto{\pgfqpoint{3.810647in}{2.010001in}}%
\pgfpathlineto{\pgfqpoint{3.815187in}{2.010001in}}%
\pgfpathlineto{\pgfqpoint{3.815187in}{2.007051in}}%
\pgfpathmoveto{\pgfqpoint{3.815187in}{2.007051in}}%
\pgfpathlineto{\pgfqpoint{3.815187in}{2.007051in}}%
\pgfpathlineto{\pgfqpoint{3.815187in}{2.010001in}}%
\pgfpathlineto{\pgfqpoint{3.819728in}{2.010001in}}%
\pgfpathlineto{\pgfqpoint{3.819728in}{2.007051in}}%
\pgfpathmoveto{\pgfqpoint{3.819728in}{2.007051in}}%
\pgfpathlineto{\pgfqpoint{3.819728in}{2.007051in}}%
\pgfpathlineto{\pgfqpoint{3.819728in}{2.010001in}}%
\pgfpathlineto{\pgfqpoint{3.824269in}{2.010001in}}%
\pgfpathlineto{\pgfqpoint{3.824269in}{2.007051in}}%
\pgfpathmoveto{\pgfqpoint{3.824269in}{2.007051in}}%
\pgfpathlineto{\pgfqpoint{3.824269in}{2.007051in}}%
\pgfpathlineto{\pgfqpoint{3.824269in}{2.010001in}}%
\pgfpathlineto{\pgfqpoint{3.828810in}{2.010001in}}%
\pgfpathlineto{\pgfqpoint{3.828810in}{2.007051in}}%
\pgfpathmoveto{\pgfqpoint{3.828810in}{2.007051in}}%
\pgfpathlineto{\pgfqpoint{3.828810in}{2.007051in}}%
\pgfpathlineto{\pgfqpoint{3.828810in}{2.010001in}}%
\pgfpathlineto{\pgfqpoint{3.833351in}{2.010001in}}%
\pgfpathlineto{\pgfqpoint{3.833351in}{2.007051in}}%
\pgfpathmoveto{\pgfqpoint{3.833351in}{2.007051in}}%
\pgfpathlineto{\pgfqpoint{3.833351in}{2.007051in}}%
\pgfpathlineto{\pgfqpoint{3.833351in}{2.010001in}}%
\pgfpathlineto{\pgfqpoint{3.837892in}{2.010001in}}%
\pgfpathlineto{\pgfqpoint{3.837892in}{2.007051in}}%
\pgfpathmoveto{\pgfqpoint{3.837892in}{2.007051in}}%
\pgfpathlineto{\pgfqpoint{3.837892in}{2.007051in}}%
\pgfpathlineto{\pgfqpoint{3.837892in}{2.010001in}}%
\pgfpathlineto{\pgfqpoint{3.842432in}{2.010001in}}%
\pgfpathlineto{\pgfqpoint{3.842432in}{2.007051in}}%
\pgfpathmoveto{\pgfqpoint{3.842432in}{2.007051in}}%
\pgfpathlineto{\pgfqpoint{3.842432in}{2.007051in}}%
\pgfpathlineto{\pgfqpoint{3.842432in}{2.010001in}}%
\pgfpathlineto{\pgfqpoint{3.846973in}{2.010001in}}%
\pgfpathlineto{\pgfqpoint{3.846973in}{2.007051in}}%
\pgfpathmoveto{\pgfqpoint{3.846973in}{2.007051in}}%
\pgfpathlineto{\pgfqpoint{3.846973in}{2.007051in}}%
\pgfpathlineto{\pgfqpoint{3.846973in}{2.010001in}}%
\pgfpathlineto{\pgfqpoint{3.851514in}{2.010001in}}%
\pgfpathlineto{\pgfqpoint{3.851514in}{2.007051in}}%
\pgfpathmoveto{\pgfqpoint{3.851514in}{2.007051in}}%
\pgfpathlineto{\pgfqpoint{3.851514in}{2.007051in}}%
\pgfpathlineto{\pgfqpoint{3.851514in}{2.010001in}}%
\pgfpathlineto{\pgfqpoint{3.856055in}{2.010001in}}%
\pgfpathlineto{\pgfqpoint{3.856055in}{2.007051in}}%
\pgfpathmoveto{\pgfqpoint{3.856055in}{2.007051in}}%
\pgfpathlineto{\pgfqpoint{3.856055in}{2.007051in}}%
\pgfpathlineto{\pgfqpoint{3.856055in}{2.010001in}}%
\pgfpathlineto{\pgfqpoint{3.860596in}{2.010001in}}%
\pgfpathlineto{\pgfqpoint{3.860596in}{2.007051in}}%
\pgfpathmoveto{\pgfqpoint{3.860596in}{2.007051in}}%
\pgfpathlineto{\pgfqpoint{3.860596in}{2.007051in}}%
\pgfpathlineto{\pgfqpoint{3.860596in}{2.010001in}}%
\pgfpathlineto{\pgfqpoint{3.865137in}{2.010001in}}%
\pgfpathlineto{\pgfqpoint{3.865137in}{2.007051in}}%
\pgfpathmoveto{\pgfqpoint{3.865137in}{2.007051in}}%
\pgfpathlineto{\pgfqpoint{3.865137in}{2.007051in}}%
\pgfpathlineto{\pgfqpoint{3.865137in}{2.010001in}}%
\pgfpathlineto{\pgfqpoint{3.869678in}{2.010001in}}%
\pgfpathlineto{\pgfqpoint{3.869678in}{2.007051in}}%
\pgfpathmoveto{\pgfqpoint{3.869678in}{2.007051in}}%
\pgfpathlineto{\pgfqpoint{3.869678in}{2.007051in}}%
\pgfpathlineto{\pgfqpoint{3.869678in}{2.010001in}}%
\pgfpathlineto{\pgfqpoint{3.874218in}{2.010001in}}%
\pgfpathlineto{\pgfqpoint{3.874218in}{2.007051in}}%
\pgfpathmoveto{\pgfqpoint{3.874218in}{2.007051in}}%
\pgfpathlineto{\pgfqpoint{3.874218in}{2.007051in}}%
\pgfpathlineto{\pgfqpoint{3.874218in}{2.010001in}}%
\pgfpathlineto{\pgfqpoint{3.878759in}{2.010001in}}%
\pgfpathlineto{\pgfqpoint{3.878759in}{2.007051in}}%
\pgfpathmoveto{\pgfqpoint{3.878759in}{2.007051in}}%
\pgfpathlineto{\pgfqpoint{3.878759in}{2.007051in}}%
\pgfpathlineto{\pgfqpoint{3.878759in}{2.010001in}}%
\pgfpathlineto{\pgfqpoint{3.883300in}{2.010001in}}%
\pgfpathlineto{\pgfqpoint{3.883300in}{2.007051in}}%
\pgfpathmoveto{\pgfqpoint{3.883300in}{2.007051in}}%
\pgfpathlineto{\pgfqpoint{3.883300in}{2.007051in}}%
\pgfpathlineto{\pgfqpoint{3.883300in}{2.010001in}}%
\pgfpathlineto{\pgfqpoint{3.887841in}{2.010001in}}%
\pgfpathlineto{\pgfqpoint{3.887841in}{2.007051in}}%
\pgfpathmoveto{\pgfqpoint{3.887841in}{2.007051in}}%
\pgfpathlineto{\pgfqpoint{3.887841in}{2.007051in}}%
\pgfpathlineto{\pgfqpoint{3.887841in}{2.010001in}}%
\pgfpathlineto{\pgfqpoint{3.892382in}{2.010001in}}%
\pgfpathlineto{\pgfqpoint{3.892382in}{2.007051in}}%
\pgfpathmoveto{\pgfqpoint{3.892382in}{2.007051in}}%
\pgfpathlineto{\pgfqpoint{3.892382in}{2.007051in}}%
\pgfpathlineto{\pgfqpoint{3.892382in}{2.010001in}}%
\pgfpathlineto{\pgfqpoint{3.896923in}{2.010001in}}%
\pgfpathlineto{\pgfqpoint{3.896923in}{2.007051in}}%
\pgfpathmoveto{\pgfqpoint{3.896923in}{2.007051in}}%
\pgfpathlineto{\pgfqpoint{3.896923in}{2.007051in}}%
\pgfpathlineto{\pgfqpoint{3.896923in}{2.010001in}}%
\pgfpathlineto{\pgfqpoint{3.901463in}{2.010001in}}%
\pgfpathlineto{\pgfqpoint{3.901463in}{2.007051in}}%
\pgfpathmoveto{\pgfqpoint{3.901463in}{2.007051in}}%
\pgfpathlineto{\pgfqpoint{3.901463in}{2.007051in}}%
\pgfpathlineto{\pgfqpoint{3.901463in}{2.010001in}}%
\pgfpathlineto{\pgfqpoint{3.906004in}{2.010001in}}%
\pgfpathlineto{\pgfqpoint{3.906004in}{2.007051in}}%
\pgfpathmoveto{\pgfqpoint{3.906004in}{2.007051in}}%
\pgfpathlineto{\pgfqpoint{3.906004in}{2.007051in}}%
\pgfpathlineto{\pgfqpoint{3.906004in}{2.010001in}}%
\pgfpathlineto{\pgfqpoint{3.910545in}{2.010001in}}%
\pgfpathlineto{\pgfqpoint{3.910545in}{2.007051in}}%
\pgfpathmoveto{\pgfqpoint{3.910545in}{2.007051in}}%
\pgfpathlineto{\pgfqpoint{3.910545in}{2.007051in}}%
\pgfpathlineto{\pgfqpoint{3.910545in}{2.010001in}}%
\pgfpathlineto{\pgfqpoint{3.915086in}{2.010001in}}%
\pgfpathlineto{\pgfqpoint{3.915086in}{2.007051in}}%
\pgfpathmoveto{\pgfqpoint{3.915086in}{2.007051in}}%
\pgfpathlineto{\pgfqpoint{3.915086in}{2.007051in}}%
\pgfpathlineto{\pgfqpoint{3.915086in}{2.010001in}}%
\pgfpathlineto{\pgfqpoint{3.919627in}{2.010001in}}%
\pgfpathlineto{\pgfqpoint{3.919627in}{2.007051in}}%
\pgfpathmoveto{\pgfqpoint{3.919627in}{2.007051in}}%
\pgfpathlineto{\pgfqpoint{3.919627in}{2.007051in}}%
\pgfpathlineto{\pgfqpoint{3.919627in}{2.010001in}}%
\pgfpathlineto{\pgfqpoint{3.924168in}{2.010001in}}%
\pgfpathlineto{\pgfqpoint{3.924168in}{2.007051in}}%
\pgfpathmoveto{\pgfqpoint{3.924168in}{2.007051in}}%
\pgfpathlineto{\pgfqpoint{3.924168in}{2.007051in}}%
\pgfpathlineto{\pgfqpoint{3.924168in}{2.010001in}}%
\pgfpathlineto{\pgfqpoint{3.928708in}{2.010001in}}%
\pgfpathlineto{\pgfqpoint{3.928708in}{2.007051in}}%
\pgfpathmoveto{\pgfqpoint{3.928708in}{2.007051in}}%
\pgfpathlineto{\pgfqpoint{3.928708in}{2.007051in}}%
\pgfpathlineto{\pgfqpoint{3.928708in}{2.010001in}}%
\pgfpathlineto{\pgfqpoint{3.933249in}{2.010001in}}%
\pgfpathlineto{\pgfqpoint{3.933249in}{2.007051in}}%
\pgfpathmoveto{\pgfqpoint{3.933249in}{2.007051in}}%
\pgfpathlineto{\pgfqpoint{3.933249in}{2.007051in}}%
\pgfpathlineto{\pgfqpoint{3.933249in}{2.010001in}}%
\pgfpathlineto{\pgfqpoint{3.937790in}{2.010001in}}%
\pgfpathlineto{\pgfqpoint{3.937790in}{2.007051in}}%
\pgfpathmoveto{\pgfqpoint{3.937790in}{2.007051in}}%
\pgfpathlineto{\pgfqpoint{3.937790in}{2.007051in}}%
\pgfpathlineto{\pgfqpoint{3.937790in}{2.010001in}}%
\pgfpathlineto{\pgfqpoint{3.942331in}{2.010001in}}%
\pgfpathlineto{\pgfqpoint{3.942331in}{2.007051in}}%
\pgfpathmoveto{\pgfqpoint{3.942331in}{2.007051in}}%
\pgfpathlineto{\pgfqpoint{3.942331in}{2.007051in}}%
\pgfpathlineto{\pgfqpoint{3.942331in}{2.010001in}}%
\pgfpathlineto{\pgfqpoint{3.946872in}{2.010001in}}%
\pgfpathlineto{\pgfqpoint{3.946872in}{2.007051in}}%
\pgfpathmoveto{\pgfqpoint{3.946872in}{2.007051in}}%
\pgfpathlineto{\pgfqpoint{3.946872in}{2.007051in}}%
\pgfpathlineto{\pgfqpoint{3.946872in}{2.010001in}}%
\pgfpathlineto{\pgfqpoint{3.951413in}{2.010001in}}%
\pgfpathlineto{\pgfqpoint{3.951413in}{2.007051in}}%
\pgfpathmoveto{\pgfqpoint{3.951413in}{2.007051in}}%
\pgfpathlineto{\pgfqpoint{3.951413in}{2.007051in}}%
\pgfpathlineto{\pgfqpoint{3.951413in}{2.010001in}}%
\pgfpathlineto{\pgfqpoint{3.955954in}{2.010001in}}%
\pgfpathlineto{\pgfqpoint{3.955954in}{2.007051in}}%
\pgfpathmoveto{\pgfqpoint{3.955954in}{2.007051in}}%
\pgfpathlineto{\pgfqpoint{3.955954in}{2.007051in}}%
\pgfpathlineto{\pgfqpoint{3.955954in}{2.010001in}}%
\pgfpathlineto{\pgfqpoint{3.960495in}{2.010001in}}%
\pgfpathlineto{\pgfqpoint{3.960495in}{2.007051in}}%
\pgfpathmoveto{\pgfqpoint{3.960495in}{2.007051in}}%
\pgfpathlineto{\pgfqpoint{3.960495in}{2.007051in}}%
\pgfpathlineto{\pgfqpoint{3.960495in}{2.010001in}}%
\pgfpathlineto{\pgfqpoint{3.965037in}{2.010001in}}%
\pgfpathlineto{\pgfqpoint{3.965037in}{2.007051in}}%
\pgfpathmoveto{\pgfqpoint{3.965037in}{2.007051in}}%
\pgfpathlineto{\pgfqpoint{3.965037in}{2.007051in}}%
\pgfpathlineto{\pgfqpoint{3.965037in}{2.010001in}}%
\pgfpathlineto{\pgfqpoint{3.969578in}{2.010001in}}%
\pgfpathlineto{\pgfqpoint{3.969578in}{2.007051in}}%
\pgfpathmoveto{\pgfqpoint{3.969578in}{2.007051in}}%
\pgfpathlineto{\pgfqpoint{3.969578in}{2.007051in}}%
\pgfpathlineto{\pgfqpoint{3.969578in}{2.010001in}}%
\pgfpathlineto{\pgfqpoint{3.974119in}{2.010001in}}%
\pgfpathlineto{\pgfqpoint{3.974119in}{2.007051in}}%
\pgfpathmoveto{\pgfqpoint{3.974119in}{2.007051in}}%
\pgfpathlineto{\pgfqpoint{3.974119in}{2.007051in}}%
\pgfpathlineto{\pgfqpoint{3.974119in}{2.010001in}}%
\pgfpathlineto{\pgfqpoint{3.978660in}{2.010001in}}%
\pgfpathlineto{\pgfqpoint{3.978660in}{2.007051in}}%
\pgfpathmoveto{\pgfqpoint{3.978660in}{2.007051in}}%
\pgfpathlineto{\pgfqpoint{3.978660in}{2.007051in}}%
\pgfpathlineto{\pgfqpoint{3.978660in}{2.010001in}}%
\pgfpathlineto{\pgfqpoint{3.983201in}{2.010001in}}%
\pgfpathlineto{\pgfqpoint{3.983201in}{2.007051in}}%
\pgfpathmoveto{\pgfqpoint{3.983201in}{2.007051in}}%
\pgfpathlineto{\pgfqpoint{3.983201in}{2.007051in}}%
\pgfpathlineto{\pgfqpoint{3.983201in}{2.010001in}}%
\pgfpathlineto{\pgfqpoint{3.987743in}{2.010001in}}%
\pgfpathlineto{\pgfqpoint{3.987743in}{2.007051in}}%
\pgfpathmoveto{\pgfqpoint{3.987743in}{2.007051in}}%
\pgfpathlineto{\pgfqpoint{3.987743in}{2.007051in}}%
\pgfpathlineto{\pgfqpoint{3.987743in}{2.010001in}}%
\pgfpathlineto{\pgfqpoint{3.992284in}{2.010001in}}%
\pgfpathlineto{\pgfqpoint{3.992284in}{2.007051in}}%
\pgfpathmoveto{\pgfqpoint{3.992284in}{2.007051in}}%
\pgfpathlineto{\pgfqpoint{3.992284in}{2.007051in}}%
\pgfpathlineto{\pgfqpoint{3.992284in}{2.010001in}}%
\pgfpathlineto{\pgfqpoint{3.996825in}{2.010001in}}%
\pgfpathlineto{\pgfqpoint{3.996825in}{2.007051in}}%
\pgfpathmoveto{\pgfqpoint{3.996825in}{2.007051in}}%
\pgfpathlineto{\pgfqpoint{3.996825in}{2.007051in}}%
\pgfpathlineto{\pgfqpoint{3.996825in}{2.010001in}}%
\pgfpathlineto{\pgfqpoint{4.001366in}{2.010001in}}%
\pgfpathlineto{\pgfqpoint{4.001366in}{2.007051in}}%
\pgfpathmoveto{\pgfqpoint{4.001366in}{2.007051in}}%
\pgfpathlineto{\pgfqpoint{4.001366in}{2.007051in}}%
\pgfpathlineto{\pgfqpoint{4.001366in}{2.010001in}}%
\pgfpathlineto{\pgfqpoint{4.005907in}{2.010001in}}%
\pgfpathlineto{\pgfqpoint{4.005907in}{2.007051in}}%
\pgfpathmoveto{\pgfqpoint{4.005907in}{2.007051in}}%
\pgfpathlineto{\pgfqpoint{4.005907in}{2.007051in}}%
\pgfpathlineto{\pgfqpoint{4.005907in}{2.010001in}}%
\pgfpathlineto{\pgfqpoint{4.010449in}{2.010001in}}%
\pgfpathlineto{\pgfqpoint{4.010449in}{2.007051in}}%
\pgfpathmoveto{\pgfqpoint{4.010449in}{2.007051in}}%
\pgfpathlineto{\pgfqpoint{4.010449in}{2.007051in}}%
\pgfpathlineto{\pgfqpoint{4.010449in}{2.010001in}}%
\pgfpathlineto{\pgfqpoint{4.014990in}{2.010001in}}%
\pgfpathlineto{\pgfqpoint{4.014990in}{2.007051in}}%
\pgfpathmoveto{\pgfqpoint{4.014990in}{2.007051in}}%
\pgfpathlineto{\pgfqpoint{4.014990in}{2.007051in}}%
\pgfpathlineto{\pgfqpoint{4.014990in}{2.010001in}}%
\pgfpathlineto{\pgfqpoint{4.019531in}{2.010001in}}%
\pgfpathlineto{\pgfqpoint{4.019531in}{2.007051in}}%
\pgfpathmoveto{\pgfqpoint{4.019531in}{2.007051in}}%
\pgfpathlineto{\pgfqpoint{4.019531in}{2.007051in}}%
\pgfpathlineto{\pgfqpoint{4.019531in}{2.010001in}}%
\pgfpathlineto{\pgfqpoint{4.024072in}{2.010001in}}%
\pgfpathlineto{\pgfqpoint{4.024072in}{2.007051in}}%
\pgfpathmoveto{\pgfqpoint{4.024072in}{2.007051in}}%
\pgfpathlineto{\pgfqpoint{4.024072in}{2.007051in}}%
\pgfpathlineto{\pgfqpoint{4.024072in}{2.010001in}}%
\pgfpathlineto{\pgfqpoint{4.028613in}{2.010001in}}%
\pgfpathlineto{\pgfqpoint{4.028613in}{2.007051in}}%
\pgfpathmoveto{\pgfqpoint{4.028613in}{2.007051in}}%
\pgfpathlineto{\pgfqpoint{4.028613in}{2.007051in}}%
\pgfpathlineto{\pgfqpoint{4.028613in}{2.010001in}}%
\pgfpathlineto{\pgfqpoint{4.033154in}{2.010001in}}%
\pgfpathlineto{\pgfqpoint{4.033154in}{2.007051in}}%
\pgfpathmoveto{\pgfqpoint{4.033154in}{2.007051in}}%
\pgfpathlineto{\pgfqpoint{4.033154in}{2.007051in}}%
\pgfpathlineto{\pgfqpoint{4.033154in}{2.010001in}}%
\pgfpathlineto{\pgfqpoint{4.037696in}{2.010001in}}%
\pgfpathlineto{\pgfqpoint{4.037696in}{2.007051in}}%
\pgfpathmoveto{\pgfqpoint{4.037696in}{2.007051in}}%
\pgfpathlineto{\pgfqpoint{4.037696in}{2.007051in}}%
\pgfpathlineto{\pgfqpoint{4.037696in}{2.010001in}}%
\pgfpathlineto{\pgfqpoint{4.042237in}{2.010001in}}%
\pgfpathlineto{\pgfqpoint{4.042237in}{2.007051in}}%
\pgfpathmoveto{\pgfqpoint{4.042237in}{2.007051in}}%
\pgfpathlineto{\pgfqpoint{4.042237in}{2.007051in}}%
\pgfpathlineto{\pgfqpoint{4.042237in}{2.010001in}}%
\pgfpathlineto{\pgfqpoint{4.046778in}{2.010001in}}%
\pgfpathlineto{\pgfqpoint{4.046778in}{2.007051in}}%
\pgfpathmoveto{\pgfqpoint{4.046778in}{2.007051in}}%
\pgfpathlineto{\pgfqpoint{4.046778in}{2.007051in}}%
\pgfpathlineto{\pgfqpoint{4.046778in}{2.010001in}}%
\pgfpathlineto{\pgfqpoint{4.051319in}{2.010001in}}%
\pgfpathlineto{\pgfqpoint{4.051319in}{2.007051in}}%
\pgfpathmoveto{\pgfqpoint{4.051319in}{2.007051in}}%
\pgfpathlineto{\pgfqpoint{4.051319in}{2.007051in}}%
\pgfpathlineto{\pgfqpoint{4.051319in}{2.010001in}}%
\pgfpathlineto{\pgfqpoint{4.055860in}{2.010001in}}%
\pgfpathlineto{\pgfqpoint{4.055860in}{2.007051in}}%
\pgfpathmoveto{\pgfqpoint{4.055860in}{2.007051in}}%
\pgfpathlineto{\pgfqpoint{4.055860in}{2.007051in}}%
\pgfpathlineto{\pgfqpoint{4.055860in}{2.010001in}}%
\pgfpathlineto{\pgfqpoint{4.060402in}{2.010001in}}%
\pgfpathlineto{\pgfqpoint{4.060402in}{2.007051in}}%
\pgfpathmoveto{\pgfqpoint{4.060402in}{2.007051in}}%
\pgfpathlineto{\pgfqpoint{4.060402in}{2.007051in}}%
\pgfpathlineto{\pgfqpoint{4.060402in}{2.010001in}}%
\pgfpathlineto{\pgfqpoint{4.064943in}{2.010001in}}%
\pgfpathlineto{\pgfqpoint{4.064943in}{2.007051in}}%
\pgfpathmoveto{\pgfqpoint{4.064943in}{2.007051in}}%
\pgfpathlineto{\pgfqpoint{4.064943in}{2.007051in}}%
\pgfpathlineto{\pgfqpoint{4.064943in}{2.010001in}}%
\pgfpathlineto{\pgfqpoint{4.069484in}{2.010001in}}%
\pgfpathlineto{\pgfqpoint{4.069484in}{2.007051in}}%
\pgfpathmoveto{\pgfqpoint{4.069484in}{2.007051in}}%
\pgfpathlineto{\pgfqpoint{4.069484in}{2.007051in}}%
\pgfpathlineto{\pgfqpoint{4.069484in}{2.010001in}}%
\pgfpathlineto{\pgfqpoint{4.074025in}{2.010001in}}%
\pgfpathlineto{\pgfqpoint{4.074025in}{2.007051in}}%
\pgfpathmoveto{\pgfqpoint{4.074025in}{2.007051in}}%
\pgfpathlineto{\pgfqpoint{4.074025in}{2.007051in}}%
\pgfpathlineto{\pgfqpoint{4.074025in}{2.010001in}}%
\pgfpathlineto{\pgfqpoint{4.078566in}{2.010001in}}%
\pgfpathlineto{\pgfqpoint{4.078566in}{2.007051in}}%
\pgfpathmoveto{\pgfqpoint{4.078566in}{2.007051in}}%
\pgfpathlineto{\pgfqpoint{4.078566in}{2.007051in}}%
\pgfpathlineto{\pgfqpoint{4.078566in}{2.010001in}}%
\pgfpathlineto{\pgfqpoint{4.083108in}{2.010001in}}%
\pgfpathlineto{\pgfqpoint{4.083108in}{2.007051in}}%
\pgfpathmoveto{\pgfqpoint{4.083108in}{2.007051in}}%
\pgfpathlineto{\pgfqpoint{4.083108in}{2.007051in}}%
\pgfpathlineto{\pgfqpoint{4.083108in}{2.010001in}}%
\pgfpathlineto{\pgfqpoint{4.087649in}{2.010001in}}%
\pgfpathlineto{\pgfqpoint{4.087649in}{2.007051in}}%
\pgfpathmoveto{\pgfqpoint{4.087649in}{2.007051in}}%
\pgfpathlineto{\pgfqpoint{4.087649in}{2.007051in}}%
\pgfpathlineto{\pgfqpoint{4.087649in}{2.010001in}}%
\pgfpathlineto{\pgfqpoint{4.092190in}{2.010001in}}%
\pgfpathlineto{\pgfqpoint{4.092190in}{2.007051in}}%
\pgfpathmoveto{\pgfqpoint{4.092190in}{2.007051in}}%
\pgfpathlineto{\pgfqpoint{4.092190in}{2.007051in}}%
\pgfpathlineto{\pgfqpoint{4.092190in}{2.010001in}}%
\pgfpathlineto{\pgfqpoint{4.096731in}{2.010001in}}%
\pgfpathlineto{\pgfqpoint{4.096731in}{2.007051in}}%
\pgfpathmoveto{\pgfqpoint{4.096731in}{2.007051in}}%
\pgfpathlineto{\pgfqpoint{4.096731in}{2.007051in}}%
\pgfpathlineto{\pgfqpoint{4.096731in}{2.010001in}}%
\pgfpathlineto{\pgfqpoint{4.101272in}{2.010001in}}%
\pgfpathlineto{\pgfqpoint{4.101272in}{2.007051in}}%
\pgfpathmoveto{\pgfqpoint{4.101272in}{2.007051in}}%
\pgfpathlineto{\pgfqpoint{4.101272in}{2.007051in}}%
\pgfpathlineto{\pgfqpoint{4.101272in}{2.010001in}}%
\pgfpathlineto{\pgfqpoint{4.105812in}{2.010001in}}%
\pgfpathlineto{\pgfqpoint{4.105812in}{2.007051in}}%
\pgfpathmoveto{\pgfqpoint{4.105812in}{2.007051in}}%
\pgfpathlineto{\pgfqpoint{4.105812in}{2.007051in}}%
\pgfpathlineto{\pgfqpoint{4.105812in}{2.010001in}}%
\pgfpathlineto{\pgfqpoint{4.110353in}{2.010001in}}%
\pgfpathlineto{\pgfqpoint{4.110353in}{2.007051in}}%
\pgfpathmoveto{\pgfqpoint{4.110353in}{2.007051in}}%
\pgfpathlineto{\pgfqpoint{4.110353in}{2.007051in}}%
\pgfpathlineto{\pgfqpoint{4.110353in}{2.010001in}}%
\pgfpathlineto{\pgfqpoint{4.114894in}{2.010001in}}%
\pgfpathlineto{\pgfqpoint{4.114894in}{2.007051in}}%
\pgfpathmoveto{\pgfqpoint{4.114894in}{2.007051in}}%
\pgfpathlineto{\pgfqpoint{4.114894in}{2.007051in}}%
\pgfpathlineto{\pgfqpoint{4.114894in}{2.010001in}}%
\pgfpathlineto{\pgfqpoint{4.119435in}{2.010001in}}%
\pgfpathlineto{\pgfqpoint{4.119435in}{2.007051in}}%
\pgfpathmoveto{\pgfqpoint{4.119435in}{2.007051in}}%
\pgfpathlineto{\pgfqpoint{4.119435in}{2.007051in}}%
\pgfpathlineto{\pgfqpoint{4.119435in}{2.010001in}}%
\pgfpathlineto{\pgfqpoint{4.123976in}{2.010001in}}%
\pgfpathlineto{\pgfqpoint{4.123976in}{2.007051in}}%
\pgfpathmoveto{\pgfqpoint{4.123976in}{2.007051in}}%
\pgfpathlineto{\pgfqpoint{4.123976in}{2.007051in}}%
\pgfpathlineto{\pgfqpoint{4.123976in}{2.010001in}}%
\pgfpathlineto{\pgfqpoint{4.128517in}{2.010001in}}%
\pgfpathlineto{\pgfqpoint{4.128517in}{2.007051in}}%
\pgfpathmoveto{\pgfqpoint{4.128517in}{2.007051in}}%
\pgfpathlineto{\pgfqpoint{4.128517in}{2.007051in}}%
\pgfpathlineto{\pgfqpoint{4.128517in}{2.010001in}}%
\pgfpathlineto{\pgfqpoint{4.133058in}{2.010001in}}%
\pgfpathlineto{\pgfqpoint{4.133058in}{2.007051in}}%
\pgfpathmoveto{\pgfqpoint{4.133058in}{2.007051in}}%
\pgfpathlineto{\pgfqpoint{4.133058in}{2.007051in}}%
\pgfpathlineto{\pgfqpoint{4.133058in}{2.010001in}}%
\pgfpathlineto{\pgfqpoint{4.137599in}{2.010001in}}%
\pgfpathlineto{\pgfqpoint{4.137599in}{2.007051in}}%
\pgfpathmoveto{\pgfqpoint{4.137599in}{2.007051in}}%
\pgfpathlineto{\pgfqpoint{4.137599in}{2.007051in}}%
\pgfpathlineto{\pgfqpoint{4.137599in}{2.010001in}}%
\pgfpathlineto{\pgfqpoint{4.142139in}{2.010001in}}%
\pgfpathlineto{\pgfqpoint{4.142139in}{2.007051in}}%
\pgfpathmoveto{\pgfqpoint{4.142139in}{2.007051in}}%
\pgfpathlineto{\pgfqpoint{4.142139in}{2.007051in}}%
\pgfpathlineto{\pgfqpoint{4.142139in}{2.010001in}}%
\pgfpathlineto{\pgfqpoint{4.146680in}{2.010001in}}%
\pgfpathlineto{\pgfqpoint{4.146680in}{2.007051in}}%
\pgfpathmoveto{\pgfqpoint{4.146680in}{2.007051in}}%
\pgfpathlineto{\pgfqpoint{4.146680in}{2.007051in}}%
\pgfpathlineto{\pgfqpoint{4.146680in}{2.010001in}}%
\pgfpathlineto{\pgfqpoint{4.151221in}{2.010001in}}%
\pgfpathlineto{\pgfqpoint{4.151221in}{2.007051in}}%
\pgfpathmoveto{\pgfqpoint{4.151221in}{2.007051in}}%
\pgfpathlineto{\pgfqpoint{4.151221in}{2.007051in}}%
\pgfpathlineto{\pgfqpoint{4.151221in}{2.010001in}}%
\pgfpathlineto{\pgfqpoint{4.155762in}{2.010001in}}%
\pgfpathlineto{\pgfqpoint{4.155762in}{2.007051in}}%
\pgfpathmoveto{\pgfqpoint{4.155762in}{2.007051in}}%
\pgfpathlineto{\pgfqpoint{4.155762in}{2.007051in}}%
\pgfpathlineto{\pgfqpoint{4.155762in}{2.010001in}}%
\pgfpathlineto{\pgfqpoint{4.160303in}{2.010001in}}%
\pgfpathlineto{\pgfqpoint{4.160303in}{2.007051in}}%
\pgfpathmoveto{\pgfqpoint{4.160303in}{2.007051in}}%
\pgfpathlineto{\pgfqpoint{4.160303in}{2.007051in}}%
\pgfpathlineto{\pgfqpoint{4.160303in}{2.010001in}}%
\pgfpathlineto{\pgfqpoint{4.164844in}{2.010001in}}%
\pgfpathlineto{\pgfqpoint{4.164844in}{2.007051in}}%
\pgfpathmoveto{\pgfqpoint{4.164844in}{2.007051in}}%
\pgfpathlineto{\pgfqpoint{4.164844in}{2.007051in}}%
\pgfpathlineto{\pgfqpoint{4.164844in}{2.010001in}}%
\pgfpathlineto{\pgfqpoint{4.169385in}{2.010001in}}%
\pgfpathlineto{\pgfqpoint{4.169385in}{2.007051in}}%
\pgfpathmoveto{\pgfqpoint{4.169385in}{2.007051in}}%
\pgfpathlineto{\pgfqpoint{4.169385in}{2.007051in}}%
\pgfpathlineto{\pgfqpoint{4.169385in}{2.010001in}}%
\pgfpathlineto{\pgfqpoint{4.173926in}{2.010001in}}%
\pgfpathlineto{\pgfqpoint{4.173926in}{2.007051in}}%
\pgfpathmoveto{\pgfqpoint{4.173926in}{2.007051in}}%
\pgfpathlineto{\pgfqpoint{4.173926in}{2.007051in}}%
\pgfpathlineto{\pgfqpoint{4.173926in}{2.010001in}}%
\pgfpathlineto{\pgfqpoint{4.178466in}{2.010001in}}%
\pgfpathlineto{\pgfqpoint{4.178466in}{2.007051in}}%
\pgfpathmoveto{\pgfqpoint{4.178466in}{2.007051in}}%
\pgfpathlineto{\pgfqpoint{4.178466in}{2.007051in}}%
\pgfpathlineto{\pgfqpoint{4.178466in}{2.010001in}}%
\pgfpathlineto{\pgfqpoint{4.183007in}{2.010001in}}%
\pgfpathlineto{\pgfqpoint{4.183007in}{2.007051in}}%
\pgfpathmoveto{\pgfqpoint{4.183007in}{2.007051in}}%
\pgfpathlineto{\pgfqpoint{4.183007in}{2.007051in}}%
\pgfpathlineto{\pgfqpoint{4.183007in}{2.010001in}}%
\pgfpathlineto{\pgfqpoint{4.187548in}{2.010001in}}%
\pgfpathlineto{\pgfqpoint{4.187548in}{2.007051in}}%
\pgfpathmoveto{\pgfqpoint{4.187548in}{2.007051in}}%
\pgfpathlineto{\pgfqpoint{4.187548in}{2.007051in}}%
\pgfpathlineto{\pgfqpoint{4.187548in}{2.010001in}}%
\pgfpathlineto{\pgfqpoint{4.192089in}{2.010001in}}%
\pgfpathlineto{\pgfqpoint{4.192089in}{2.007051in}}%
\pgfpathmoveto{\pgfqpoint{4.192089in}{2.007051in}}%
\pgfpathlineto{\pgfqpoint{4.192089in}{2.007051in}}%
\pgfpathlineto{\pgfqpoint{4.192089in}{2.010001in}}%
\pgfpathlineto{\pgfqpoint{4.196630in}{2.010001in}}%
\pgfpathlineto{\pgfqpoint{4.196630in}{2.007051in}}%
\pgfpathmoveto{\pgfqpoint{4.196630in}{2.007051in}}%
\pgfpathlineto{\pgfqpoint{4.196630in}{2.007051in}}%
\pgfpathlineto{\pgfqpoint{4.196630in}{2.010001in}}%
\pgfpathlineto{\pgfqpoint{4.201171in}{2.010001in}}%
\pgfpathlineto{\pgfqpoint{4.201171in}{2.007051in}}%
\pgfpathmoveto{\pgfqpoint{4.201171in}{2.007051in}}%
\pgfpathlineto{\pgfqpoint{4.201171in}{2.007051in}}%
\pgfpathlineto{\pgfqpoint{4.201171in}{2.010001in}}%
\pgfpathlineto{\pgfqpoint{4.205712in}{2.010001in}}%
\pgfpathlineto{\pgfqpoint{4.205712in}{2.007051in}}%
\pgfpathmoveto{\pgfqpoint{4.205712in}{2.007051in}}%
\pgfpathlineto{\pgfqpoint{4.205712in}{2.007051in}}%
\pgfpathlineto{\pgfqpoint{4.205712in}{2.010001in}}%
\pgfpathlineto{\pgfqpoint{4.210252in}{2.010001in}}%
\pgfpathlineto{\pgfqpoint{4.210252in}{2.007051in}}%
\pgfpathmoveto{\pgfqpoint{4.210252in}{2.007051in}}%
\pgfpathlineto{\pgfqpoint{4.210252in}{2.007051in}}%
\pgfpathlineto{\pgfqpoint{4.210252in}{2.010001in}}%
\pgfpathlineto{\pgfqpoint{4.214793in}{2.010001in}}%
\pgfpathlineto{\pgfqpoint{4.214793in}{2.007051in}}%
\pgfpathmoveto{\pgfqpoint{4.214793in}{2.007051in}}%
\pgfpathlineto{\pgfqpoint{4.214793in}{2.007051in}}%
\pgfpathlineto{\pgfqpoint{4.214793in}{2.010001in}}%
\pgfpathlineto{\pgfqpoint{4.219334in}{2.010001in}}%
\pgfpathlineto{\pgfqpoint{4.219334in}{2.007051in}}%
\pgfpathmoveto{\pgfqpoint{4.219334in}{2.007051in}}%
\pgfpathlineto{\pgfqpoint{4.219334in}{2.007051in}}%
\pgfpathlineto{\pgfqpoint{4.219334in}{2.010001in}}%
\pgfpathlineto{\pgfqpoint{4.223875in}{2.010001in}}%
\pgfpathlineto{\pgfqpoint{4.223875in}{2.007051in}}%
\pgfpathmoveto{\pgfqpoint{4.223875in}{2.007051in}}%
\pgfpathlineto{\pgfqpoint{4.223875in}{2.007051in}}%
\pgfpathlineto{\pgfqpoint{4.223875in}{2.010001in}}%
\pgfpathlineto{\pgfqpoint{4.228416in}{2.010001in}}%
\pgfpathlineto{\pgfqpoint{4.228416in}{2.007051in}}%
\pgfpathmoveto{\pgfqpoint{4.228416in}{2.007051in}}%
\pgfpathlineto{\pgfqpoint{4.228416in}{2.007051in}}%
\pgfpathlineto{\pgfqpoint{4.228416in}{2.010001in}}%
\pgfpathlineto{\pgfqpoint{4.232957in}{2.010001in}}%
\pgfpathlineto{\pgfqpoint{4.232957in}{2.007051in}}%
\pgfpathmoveto{\pgfqpoint{4.232957in}{2.007051in}}%
\pgfpathlineto{\pgfqpoint{4.232957in}{2.007051in}}%
\pgfpathlineto{\pgfqpoint{4.232957in}{2.010001in}}%
\pgfpathlineto{\pgfqpoint{4.237498in}{2.010001in}}%
\pgfpathlineto{\pgfqpoint{4.237498in}{2.007051in}}%
\pgfpathmoveto{\pgfqpoint{4.237498in}{2.007051in}}%
\pgfpathlineto{\pgfqpoint{4.237498in}{2.007051in}}%
\pgfpathlineto{\pgfqpoint{4.237498in}{2.010001in}}%
\pgfpathlineto{\pgfqpoint{4.242039in}{2.010001in}}%
\pgfpathlineto{\pgfqpoint{4.242039in}{2.007051in}}%
\pgfpathmoveto{\pgfqpoint{4.242039in}{2.007051in}}%
\pgfpathlineto{\pgfqpoint{4.242039in}{2.007051in}}%
\pgfpathlineto{\pgfqpoint{4.242039in}{2.010001in}}%
\pgfpathlineto{\pgfqpoint{4.246580in}{2.010001in}}%
\pgfpathlineto{\pgfqpoint{4.246580in}{2.007051in}}%
\pgfpathmoveto{\pgfqpoint{4.246580in}{2.007051in}}%
\pgfpathlineto{\pgfqpoint{4.246580in}{2.007051in}}%
\pgfpathlineto{\pgfqpoint{4.246580in}{2.010001in}}%
\pgfpathlineto{\pgfqpoint{4.251121in}{2.010001in}}%
\pgfpathlineto{\pgfqpoint{4.251121in}{2.007051in}}%
\pgfpathmoveto{\pgfqpoint{4.251121in}{2.007051in}}%
\pgfpathlineto{\pgfqpoint{4.251121in}{2.007051in}}%
\pgfpathlineto{\pgfqpoint{4.251121in}{2.010001in}}%
\pgfpathlineto{\pgfqpoint{4.255662in}{2.010001in}}%
\pgfpathlineto{\pgfqpoint{4.255662in}{2.007051in}}%
\pgfpathmoveto{\pgfqpoint{4.255662in}{2.007051in}}%
\pgfpathlineto{\pgfqpoint{4.255662in}{2.007051in}}%
\pgfpathlineto{\pgfqpoint{4.255662in}{2.010001in}}%
\pgfpathlineto{\pgfqpoint{4.260203in}{2.010001in}}%
\pgfpathlineto{\pgfqpoint{4.260203in}{2.007051in}}%
\pgfpathmoveto{\pgfqpoint{4.260203in}{2.007051in}}%
\pgfpathlineto{\pgfqpoint{4.260203in}{2.007051in}}%
\pgfpathlineto{\pgfqpoint{4.260203in}{2.010001in}}%
\pgfpathlineto{\pgfqpoint{4.264744in}{2.010001in}}%
\pgfpathlineto{\pgfqpoint{4.264744in}{2.007051in}}%
\pgfpathmoveto{\pgfqpoint{4.264744in}{2.007051in}}%
\pgfpathlineto{\pgfqpoint{4.264744in}{2.007051in}}%
\pgfpathlineto{\pgfqpoint{4.264744in}{2.010001in}}%
\pgfpathlineto{\pgfqpoint{4.269285in}{2.010001in}}%
\pgfpathlineto{\pgfqpoint{4.269285in}{2.007051in}}%
\pgfpathmoveto{\pgfqpoint{4.269285in}{2.007051in}}%
\pgfpathlineto{\pgfqpoint{4.269285in}{2.007051in}}%
\pgfpathlineto{\pgfqpoint{4.269285in}{2.010001in}}%
\pgfpathlineto{\pgfqpoint{4.273826in}{2.010001in}}%
\pgfpathlineto{\pgfqpoint{4.273826in}{2.007051in}}%
\pgfpathmoveto{\pgfqpoint{4.273826in}{2.007051in}}%
\pgfpathlineto{\pgfqpoint{4.273826in}{2.007051in}}%
\pgfpathlineto{\pgfqpoint{4.273826in}{2.010001in}}%
\pgfpathlineto{\pgfqpoint{4.278367in}{2.010001in}}%
\pgfpathlineto{\pgfqpoint{4.278367in}{2.007051in}}%
\pgfpathmoveto{\pgfqpoint{4.278367in}{2.007051in}}%
\pgfpathlineto{\pgfqpoint{4.278367in}{2.007051in}}%
\pgfpathlineto{\pgfqpoint{4.278367in}{2.010001in}}%
\pgfpathlineto{\pgfqpoint{4.282908in}{2.010001in}}%
\pgfpathlineto{\pgfqpoint{4.282908in}{2.007051in}}%
\pgfpathmoveto{\pgfqpoint{4.282908in}{2.007051in}}%
\pgfpathlineto{\pgfqpoint{4.282908in}{2.007051in}}%
\pgfpathlineto{\pgfqpoint{4.282908in}{2.010001in}}%
\pgfpathlineto{\pgfqpoint{4.287449in}{2.010001in}}%
\pgfpathlineto{\pgfqpoint{4.287449in}{2.007051in}}%
\pgfpathmoveto{\pgfqpoint{4.287449in}{2.007051in}}%
\pgfpathlineto{\pgfqpoint{4.287449in}{2.007051in}}%
\pgfpathlineto{\pgfqpoint{4.287449in}{2.010001in}}%
\pgfpathlineto{\pgfqpoint{4.291990in}{2.010001in}}%
\pgfpathlineto{\pgfqpoint{4.291990in}{2.007051in}}%
\pgfpathmoveto{\pgfqpoint{4.291990in}{2.007051in}}%
\pgfpathlineto{\pgfqpoint{4.291990in}{2.007051in}}%
\pgfpathlineto{\pgfqpoint{4.291990in}{2.010001in}}%
\pgfpathlineto{\pgfqpoint{4.296531in}{2.010001in}}%
\pgfpathlineto{\pgfqpoint{4.296531in}{2.007051in}}%
\pgfpathmoveto{\pgfqpoint{4.296531in}{2.007051in}}%
\pgfpathlineto{\pgfqpoint{4.296531in}{2.007051in}}%
\pgfpathlineto{\pgfqpoint{4.296531in}{2.010001in}}%
\pgfpathlineto{\pgfqpoint{4.301072in}{2.010001in}}%
\pgfpathlineto{\pgfqpoint{4.301072in}{2.007051in}}%
\pgfpathmoveto{\pgfqpoint{4.301072in}{2.007051in}}%
\pgfpathlineto{\pgfqpoint{4.301072in}{2.007051in}}%
\pgfpathlineto{\pgfqpoint{4.301072in}{2.010001in}}%
\pgfpathlineto{\pgfqpoint{4.305613in}{2.010001in}}%
\pgfpathlineto{\pgfqpoint{4.305613in}{2.007051in}}%
\pgfpathmoveto{\pgfqpoint{4.305613in}{2.007051in}}%
\pgfpathlineto{\pgfqpoint{4.305613in}{2.007051in}}%
\pgfpathlineto{\pgfqpoint{4.305613in}{2.010001in}}%
\pgfpathlineto{\pgfqpoint{4.310154in}{2.010001in}}%
\pgfpathlineto{\pgfqpoint{4.310154in}{2.007051in}}%
\pgfpathmoveto{\pgfqpoint{4.310154in}{2.007051in}}%
\pgfpathlineto{\pgfqpoint{4.310154in}{2.007051in}}%
\pgfpathlineto{\pgfqpoint{4.310154in}{2.010001in}}%
\pgfpathlineto{\pgfqpoint{4.314695in}{2.010001in}}%
\pgfpathlineto{\pgfqpoint{4.314695in}{2.007051in}}%
\pgfpathmoveto{\pgfqpoint{4.314695in}{2.007051in}}%
\pgfpathlineto{\pgfqpoint{4.314695in}{2.007051in}}%
\pgfpathlineto{\pgfqpoint{4.314695in}{2.010001in}}%
\pgfpathlineto{\pgfqpoint{4.319236in}{2.010001in}}%
\pgfpathlineto{\pgfqpoint{4.319236in}{2.007051in}}%
\pgfpathmoveto{\pgfqpoint{4.319236in}{2.007051in}}%
\pgfpathlineto{\pgfqpoint{4.319236in}{2.007051in}}%
\pgfpathlineto{\pgfqpoint{4.319236in}{2.010001in}}%
\pgfpathlineto{\pgfqpoint{4.323777in}{2.010001in}}%
\pgfpathlineto{\pgfqpoint{4.323777in}{2.007051in}}%
\pgfpathmoveto{\pgfqpoint{4.323777in}{2.007051in}}%
\pgfpathlineto{\pgfqpoint{4.323777in}{2.007051in}}%
\pgfpathlineto{\pgfqpoint{4.323777in}{2.010001in}}%
\pgfpathlineto{\pgfqpoint{4.328318in}{2.010001in}}%
\pgfpathlineto{\pgfqpoint{4.328318in}{2.007051in}}%
\pgfpathmoveto{\pgfqpoint{4.328318in}{2.007051in}}%
\pgfpathlineto{\pgfqpoint{4.328318in}{2.007051in}}%
\pgfpathlineto{\pgfqpoint{4.328318in}{2.010001in}}%
\pgfpathlineto{\pgfqpoint{4.332859in}{2.010001in}}%
\pgfpathlineto{\pgfqpoint{4.332859in}{2.007051in}}%
\pgfpathmoveto{\pgfqpoint{4.332859in}{2.007051in}}%
\pgfpathlineto{\pgfqpoint{4.332859in}{2.007051in}}%
\pgfpathlineto{\pgfqpoint{4.332859in}{2.010001in}}%
\pgfpathlineto{\pgfqpoint{4.337400in}{2.010001in}}%
\pgfpathlineto{\pgfqpoint{4.337400in}{2.007051in}}%
\pgfpathmoveto{\pgfqpoint{4.337400in}{2.007051in}}%
\pgfpathlineto{\pgfqpoint{4.337400in}{2.007051in}}%
\pgfpathlineto{\pgfqpoint{4.337400in}{2.010001in}}%
\pgfpathlineto{\pgfqpoint{4.341941in}{2.010001in}}%
\pgfpathlineto{\pgfqpoint{4.341941in}{2.007051in}}%
\pgfpathmoveto{\pgfqpoint{4.341941in}{2.007051in}}%
\pgfpathlineto{\pgfqpoint{4.341941in}{2.007051in}}%
\pgfpathlineto{\pgfqpoint{4.341941in}{2.010001in}}%
\pgfpathlineto{\pgfqpoint{4.346482in}{2.010001in}}%
\pgfpathlineto{\pgfqpoint{4.346482in}{2.007051in}}%
\pgfpathmoveto{\pgfqpoint{4.346482in}{2.007051in}}%
\pgfpathlineto{\pgfqpoint{4.346482in}{2.007051in}}%
\pgfpathlineto{\pgfqpoint{4.346482in}{2.010001in}}%
\pgfpathlineto{\pgfqpoint{4.351023in}{2.010001in}}%
\pgfpathlineto{\pgfqpoint{4.351023in}{2.007051in}}%
\pgfpathmoveto{\pgfqpoint{4.351023in}{2.007051in}}%
\pgfpathlineto{\pgfqpoint{4.351023in}{2.007051in}}%
\pgfpathlineto{\pgfqpoint{4.351023in}{2.010001in}}%
\pgfpathlineto{\pgfqpoint{4.355564in}{2.010001in}}%
\pgfpathlineto{\pgfqpoint{4.355564in}{2.007051in}}%
\pgfpathmoveto{\pgfqpoint{4.355564in}{2.007051in}}%
\pgfpathlineto{\pgfqpoint{4.355564in}{2.007051in}}%
\pgfpathlineto{\pgfqpoint{4.355564in}{2.010001in}}%
\pgfpathlineto{\pgfqpoint{4.360105in}{2.010001in}}%
\pgfpathlineto{\pgfqpoint{4.360105in}{2.007051in}}%
\pgfpathmoveto{\pgfqpoint{4.360105in}{2.007051in}}%
\pgfpathlineto{\pgfqpoint{4.360105in}{2.007051in}}%
\pgfpathlineto{\pgfqpoint{4.360105in}{2.010001in}}%
\pgfpathlineto{\pgfqpoint{4.364646in}{2.010001in}}%
\pgfpathlineto{\pgfqpoint{4.364646in}{2.007051in}}%
\pgfpathmoveto{\pgfqpoint{4.364646in}{2.007051in}}%
\pgfpathlineto{\pgfqpoint{4.364646in}{2.007051in}}%
\pgfpathlineto{\pgfqpoint{4.364646in}{2.010001in}}%
\pgfpathlineto{\pgfqpoint{4.369187in}{2.010001in}}%
\pgfpathlineto{\pgfqpoint{4.369187in}{2.007051in}}%
\pgfpathmoveto{\pgfqpoint{4.369187in}{2.007051in}}%
\pgfpathlineto{\pgfqpoint{4.369187in}{2.007051in}}%
\pgfpathlineto{\pgfqpoint{4.369187in}{2.010001in}}%
\pgfpathlineto{\pgfqpoint{4.373728in}{2.010001in}}%
\pgfpathlineto{\pgfqpoint{4.373728in}{2.007051in}}%
\pgfpathmoveto{\pgfqpoint{4.373728in}{2.007051in}}%
\pgfpathlineto{\pgfqpoint{4.373728in}{2.007051in}}%
\pgfpathlineto{\pgfqpoint{4.373728in}{2.010001in}}%
\pgfpathlineto{\pgfqpoint{4.378269in}{2.010001in}}%
\pgfpathlineto{\pgfqpoint{4.378269in}{2.007051in}}%
\pgfpathmoveto{\pgfqpoint{4.378269in}{2.007051in}}%
\pgfpathlineto{\pgfqpoint{4.378269in}{2.007051in}}%
\pgfpathlineto{\pgfqpoint{4.378269in}{2.010001in}}%
\pgfpathlineto{\pgfqpoint{4.382810in}{2.010001in}}%
\pgfpathlineto{\pgfqpoint{4.382810in}{2.007051in}}%
\pgfpathmoveto{\pgfqpoint{4.382810in}{2.007051in}}%
\pgfpathlineto{\pgfqpoint{4.382810in}{2.007051in}}%
\pgfpathlineto{\pgfqpoint{4.382810in}{2.010001in}}%
\pgfpathlineto{\pgfqpoint{4.387351in}{2.010001in}}%
\pgfpathlineto{\pgfqpoint{4.387351in}{2.007051in}}%
\pgfpathmoveto{\pgfqpoint{4.387351in}{2.007051in}}%
\pgfpathlineto{\pgfqpoint{4.387351in}{2.007051in}}%
\pgfpathlineto{\pgfqpoint{4.387351in}{2.010001in}}%
\pgfpathlineto{\pgfqpoint{4.391892in}{2.010001in}}%
\pgfpathlineto{\pgfqpoint{4.391892in}{2.007051in}}%
\pgfpathmoveto{\pgfqpoint{4.391892in}{2.007051in}}%
\pgfpathlineto{\pgfqpoint{4.391892in}{2.007051in}}%
\pgfpathlineto{\pgfqpoint{4.391892in}{2.010001in}}%
\pgfpathlineto{\pgfqpoint{4.396433in}{2.010001in}}%
\pgfpathlineto{\pgfqpoint{4.396433in}{2.007051in}}%
\pgfpathmoveto{\pgfqpoint{4.396433in}{2.007051in}}%
\pgfpathlineto{\pgfqpoint{4.396433in}{2.007051in}}%
\pgfpathlineto{\pgfqpoint{4.396433in}{2.010001in}}%
\pgfpathlineto{\pgfqpoint{4.400974in}{2.010001in}}%
\pgfpathlineto{\pgfqpoint{4.400974in}{2.007051in}}%
\pgfpathmoveto{\pgfqpoint{4.400974in}{2.007051in}}%
\pgfpathlineto{\pgfqpoint{4.400974in}{2.007051in}}%
\pgfpathlineto{\pgfqpoint{4.400974in}{2.010001in}}%
\pgfpathlineto{\pgfqpoint{4.405515in}{2.010001in}}%
\pgfpathlineto{\pgfqpoint{4.405515in}{2.007051in}}%
\pgfpathmoveto{\pgfqpoint{4.405515in}{2.007051in}}%
\pgfpathlineto{\pgfqpoint{4.405515in}{2.007051in}}%
\pgfpathlineto{\pgfqpoint{4.405515in}{2.010001in}}%
\pgfpathlineto{\pgfqpoint{4.410057in}{2.010001in}}%
\pgfpathlineto{\pgfqpoint{4.410057in}{2.007051in}}%
\pgfpathmoveto{\pgfqpoint{4.410057in}{2.007051in}}%
\pgfpathlineto{\pgfqpoint{4.410057in}{2.007051in}}%
\pgfpathlineto{\pgfqpoint{4.410057in}{2.010001in}}%
\pgfpathlineto{\pgfqpoint{4.414598in}{2.010001in}}%
\pgfpathlineto{\pgfqpoint{4.414598in}{2.007051in}}%
\pgfpathmoveto{\pgfqpoint{4.414598in}{2.007051in}}%
\pgfpathlineto{\pgfqpoint{4.414598in}{2.007051in}}%
\pgfpathlineto{\pgfqpoint{4.414598in}{2.010001in}}%
\pgfpathlineto{\pgfqpoint{4.419139in}{2.010001in}}%
\pgfpathlineto{\pgfqpoint{4.419139in}{2.007051in}}%
\pgfpathmoveto{\pgfqpoint{4.419139in}{2.007051in}}%
\pgfpathlineto{\pgfqpoint{4.419139in}{2.007051in}}%
\pgfpathlineto{\pgfqpoint{4.419139in}{2.010001in}}%
\pgfpathlineto{\pgfqpoint{4.423680in}{2.010001in}}%
\pgfpathlineto{\pgfqpoint{4.423680in}{2.007051in}}%
\pgfpathmoveto{\pgfqpoint{4.423680in}{2.007051in}}%
\pgfpathlineto{\pgfqpoint{4.423680in}{2.007051in}}%
\pgfpathlineto{\pgfqpoint{4.423680in}{2.010001in}}%
\pgfpathlineto{\pgfqpoint{4.428221in}{2.010001in}}%
\pgfpathlineto{\pgfqpoint{4.428221in}{2.007051in}}%
\pgfpathmoveto{\pgfqpoint{4.428221in}{2.007051in}}%
\pgfpathlineto{\pgfqpoint{4.428221in}{2.007051in}}%
\pgfpathlineto{\pgfqpoint{4.428221in}{2.010001in}}%
\pgfpathlineto{\pgfqpoint{4.432762in}{2.010001in}}%
\pgfpathlineto{\pgfqpoint{4.432762in}{2.007051in}}%
\pgfpathmoveto{\pgfqpoint{4.432762in}{2.007051in}}%
\pgfpathlineto{\pgfqpoint{4.432762in}{2.007051in}}%
\pgfpathlineto{\pgfqpoint{4.432762in}{2.010001in}}%
\pgfpathlineto{\pgfqpoint{4.437304in}{2.010001in}}%
\pgfpathlineto{\pgfqpoint{4.437304in}{2.007051in}}%
\pgfpathmoveto{\pgfqpoint{4.437304in}{2.007051in}}%
\pgfpathlineto{\pgfqpoint{4.437304in}{2.007051in}}%
\pgfpathlineto{\pgfqpoint{4.437304in}{2.010001in}}%
\pgfpathlineto{\pgfqpoint{4.441845in}{2.010001in}}%
\pgfpathlineto{\pgfqpoint{4.441845in}{2.007051in}}%
\pgfpathmoveto{\pgfqpoint{4.441845in}{2.007051in}}%
\pgfpathlineto{\pgfqpoint{4.441845in}{2.007051in}}%
\pgfpathlineto{\pgfqpoint{4.441845in}{2.010001in}}%
\pgfpathlineto{\pgfqpoint{4.446386in}{2.010001in}}%
\pgfpathlineto{\pgfqpoint{4.446386in}{2.007051in}}%
\pgfpathmoveto{\pgfqpoint{4.446386in}{2.007051in}}%
\pgfpathlineto{\pgfqpoint{4.446386in}{2.007051in}}%
\pgfpathlineto{\pgfqpoint{4.446386in}{2.010001in}}%
\pgfpathlineto{\pgfqpoint{4.450927in}{2.010001in}}%
\pgfpathlineto{\pgfqpoint{4.450927in}{2.007051in}}%
\pgfpathmoveto{\pgfqpoint{4.450927in}{2.007051in}}%
\pgfpathlineto{\pgfqpoint{4.450927in}{2.007051in}}%
\pgfpathlineto{\pgfqpoint{4.450927in}{2.010001in}}%
\pgfpathlineto{\pgfqpoint{4.455468in}{2.010001in}}%
\pgfpathlineto{\pgfqpoint{4.455468in}{2.007051in}}%
\pgfpathmoveto{\pgfqpoint{4.455468in}{2.007051in}}%
\pgfpathlineto{\pgfqpoint{4.455468in}{2.007051in}}%
\pgfpathlineto{\pgfqpoint{4.455468in}{2.010001in}}%
\pgfpathlineto{\pgfqpoint{4.460009in}{2.010001in}}%
\pgfpathlineto{\pgfqpoint{4.460009in}{2.007051in}}%
\pgfpathmoveto{\pgfqpoint{4.460009in}{2.007051in}}%
\pgfpathlineto{\pgfqpoint{4.460009in}{2.007051in}}%
\pgfpathlineto{\pgfqpoint{4.460009in}{2.010001in}}%
\pgfpathlineto{\pgfqpoint{4.464550in}{2.010001in}}%
\pgfpathlineto{\pgfqpoint{4.464550in}{2.007051in}}%
\pgfpathmoveto{\pgfqpoint{4.464550in}{2.007051in}}%
\pgfpathlineto{\pgfqpoint{4.464550in}{2.007051in}}%
\pgfpathlineto{\pgfqpoint{4.464550in}{2.010001in}}%
\pgfpathlineto{\pgfqpoint{4.469092in}{2.010001in}}%
\pgfpathlineto{\pgfqpoint{4.469092in}{2.007051in}}%
\pgfpathmoveto{\pgfqpoint{4.469092in}{2.007051in}}%
\pgfpathlineto{\pgfqpoint{4.469092in}{2.007051in}}%
\pgfpathlineto{\pgfqpoint{4.469092in}{2.010001in}}%
\pgfpathlineto{\pgfqpoint{4.473633in}{2.010001in}}%
\pgfpathlineto{\pgfqpoint{4.473633in}{2.007051in}}%
\pgfpathmoveto{\pgfqpoint{4.473633in}{2.007051in}}%
\pgfpathlineto{\pgfqpoint{4.473633in}{2.007051in}}%
\pgfpathlineto{\pgfqpoint{4.473633in}{2.010001in}}%
\pgfpathlineto{\pgfqpoint{4.478174in}{2.010001in}}%
\pgfpathlineto{\pgfqpoint{4.478174in}{2.007051in}}%
\pgfpathmoveto{\pgfqpoint{4.478174in}{2.007051in}}%
\pgfpathlineto{\pgfqpoint{4.478174in}{2.007051in}}%
\pgfpathlineto{\pgfqpoint{4.478174in}{2.010001in}}%
\pgfpathlineto{\pgfqpoint{4.482715in}{2.010001in}}%
\pgfpathlineto{\pgfqpoint{4.482715in}{2.007051in}}%
\pgfpathmoveto{\pgfqpoint{4.482715in}{2.007051in}}%
\pgfpathlineto{\pgfqpoint{4.482715in}{2.007051in}}%
\pgfpathlineto{\pgfqpoint{4.482715in}{2.010001in}}%
\pgfpathlineto{\pgfqpoint{4.487256in}{2.010001in}}%
\pgfpathlineto{\pgfqpoint{4.487256in}{2.007051in}}%
\pgfpathmoveto{\pgfqpoint{4.487256in}{2.007051in}}%
\pgfpathlineto{\pgfqpoint{4.487256in}{2.007051in}}%
\pgfpathlineto{\pgfqpoint{4.487256in}{2.010001in}}%
\pgfpathlineto{\pgfqpoint{4.491797in}{2.010001in}}%
\pgfpathlineto{\pgfqpoint{4.491797in}{2.007051in}}%
\pgfpathmoveto{\pgfqpoint{4.491797in}{2.007051in}}%
\pgfpathlineto{\pgfqpoint{4.491797in}{2.007051in}}%
\pgfpathlineto{\pgfqpoint{4.491797in}{2.010001in}}%
\pgfpathlineto{\pgfqpoint{4.496339in}{2.010001in}}%
\pgfpathlineto{\pgfqpoint{4.496339in}{2.007051in}}%
\pgfpathmoveto{\pgfqpoint{4.496339in}{2.007051in}}%
\pgfpathlineto{\pgfqpoint{4.496339in}{2.007051in}}%
\pgfpathlineto{\pgfqpoint{4.496339in}{2.010001in}}%
\pgfpathlineto{\pgfqpoint{4.500880in}{2.010001in}}%
\pgfpathlineto{\pgfqpoint{4.500880in}{2.007051in}}%
\pgfpathmoveto{\pgfqpoint{4.500880in}{2.007051in}}%
\pgfpathlineto{\pgfqpoint{4.500880in}{2.007051in}}%
\pgfpathlineto{\pgfqpoint{4.500880in}{2.010001in}}%
\pgfpathlineto{\pgfqpoint{4.505421in}{2.010001in}}%
\pgfpathlineto{\pgfqpoint{4.505421in}{2.007051in}}%
\pgfpathmoveto{\pgfqpoint{4.505421in}{2.007051in}}%
\pgfpathlineto{\pgfqpoint{4.505421in}{2.007051in}}%
\pgfpathlineto{\pgfqpoint{4.505421in}{2.010001in}}%
\pgfpathlineto{\pgfqpoint{4.509962in}{2.010001in}}%
\pgfpathlineto{\pgfqpoint{4.509962in}{2.007051in}}%
\pgfpathmoveto{\pgfqpoint{4.509962in}{2.007051in}}%
\pgfpathlineto{\pgfqpoint{4.509962in}{2.007051in}}%
\pgfpathlineto{\pgfqpoint{4.509962in}{2.010001in}}%
\pgfpathlineto{\pgfqpoint{4.514503in}{2.010001in}}%
\pgfpathlineto{\pgfqpoint{4.514503in}{2.007051in}}%
\pgfpathmoveto{\pgfqpoint{4.514503in}{2.007051in}}%
\pgfpathlineto{\pgfqpoint{4.514503in}{2.007051in}}%
\pgfpathlineto{\pgfqpoint{4.514503in}{2.010001in}}%
\pgfpathlineto{\pgfqpoint{4.519044in}{2.010001in}}%
\pgfpathlineto{\pgfqpoint{4.519044in}{2.007051in}}%
\pgfpathmoveto{\pgfqpoint{4.519044in}{2.007051in}}%
\pgfpathlineto{\pgfqpoint{4.519044in}{2.007051in}}%
\pgfpathlineto{\pgfqpoint{4.519044in}{2.010001in}}%
\pgfpathlineto{\pgfqpoint{4.523585in}{2.010001in}}%
\pgfpathlineto{\pgfqpoint{4.523585in}{2.007051in}}%
\pgfpathmoveto{\pgfqpoint{4.523585in}{2.007051in}}%
\pgfpathlineto{\pgfqpoint{4.523585in}{2.007051in}}%
\pgfpathlineto{\pgfqpoint{4.523585in}{2.010001in}}%
\pgfpathlineto{\pgfqpoint{4.528127in}{2.010001in}}%
\pgfpathlineto{\pgfqpoint{4.528127in}{2.007051in}}%
\pgfpathmoveto{\pgfqpoint{4.528127in}{2.007051in}}%
\pgfpathlineto{\pgfqpoint{4.528127in}{2.007051in}}%
\pgfpathlineto{\pgfqpoint{4.528127in}{2.010001in}}%
\pgfpathlineto{\pgfqpoint{4.532668in}{2.010001in}}%
\pgfpathlineto{\pgfqpoint{4.532668in}{2.007051in}}%
\pgfpathmoveto{\pgfqpoint{4.532668in}{2.007051in}}%
\pgfpathlineto{\pgfqpoint{4.532668in}{2.007051in}}%
\pgfpathlineto{\pgfqpoint{4.532668in}{2.010001in}}%
\pgfpathlineto{\pgfqpoint{4.537208in}{2.010001in}}%
\pgfpathlineto{\pgfqpoint{4.537208in}{2.007051in}}%
\pgfpathmoveto{\pgfqpoint{4.537208in}{2.007051in}}%
\pgfpathlineto{\pgfqpoint{4.537208in}{2.007051in}}%
\pgfpathlineto{\pgfqpoint{4.537208in}{2.010001in}}%
\pgfpathlineto{\pgfqpoint{4.541749in}{2.010001in}}%
\pgfpathlineto{\pgfqpoint{4.541749in}{2.007051in}}%
\pgfpathmoveto{\pgfqpoint{4.541749in}{2.007051in}}%
\pgfpathlineto{\pgfqpoint{4.541749in}{2.007051in}}%
\pgfpathlineto{\pgfqpoint{4.541749in}{2.010001in}}%
\pgfpathlineto{\pgfqpoint{4.546290in}{2.010001in}}%
\pgfpathlineto{\pgfqpoint{4.546290in}{2.007051in}}%
\pgfpathmoveto{\pgfqpoint{4.546290in}{2.007051in}}%
\pgfpathlineto{\pgfqpoint{4.546290in}{2.007051in}}%
\pgfpathlineto{\pgfqpoint{4.546290in}{2.010001in}}%
\pgfpathlineto{\pgfqpoint{4.550831in}{2.010001in}}%
\pgfpathlineto{\pgfqpoint{4.550831in}{2.007051in}}%
\pgfpathmoveto{\pgfqpoint{4.550831in}{2.007051in}}%
\pgfpathlineto{\pgfqpoint{4.550831in}{2.007051in}}%
\pgfpathlineto{\pgfqpoint{4.550831in}{2.010001in}}%
\pgfpathlineto{\pgfqpoint{4.555372in}{2.010001in}}%
\pgfpathlineto{\pgfqpoint{4.555372in}{2.007051in}}%
\pgfpathmoveto{\pgfqpoint{4.555372in}{2.007051in}}%
\pgfpathlineto{\pgfqpoint{4.555372in}{2.007051in}}%
\pgfpathlineto{\pgfqpoint{4.555372in}{2.010001in}}%
\pgfpathlineto{\pgfqpoint{4.559913in}{2.010001in}}%
\pgfpathlineto{\pgfqpoint{4.559913in}{2.007051in}}%
\pgfpathmoveto{\pgfqpoint{4.559913in}{2.007051in}}%
\pgfpathlineto{\pgfqpoint{4.559913in}{2.007051in}}%
\pgfpathlineto{\pgfqpoint{4.559913in}{2.010001in}}%
\pgfpathlineto{\pgfqpoint{4.564454in}{2.010001in}}%
\pgfpathlineto{\pgfqpoint{4.564454in}{2.007051in}}%
\pgfpathmoveto{\pgfqpoint{4.564454in}{2.007051in}}%
\pgfpathlineto{\pgfqpoint{4.564454in}{2.007051in}}%
\pgfpathlineto{\pgfqpoint{4.564454in}{2.010001in}}%
\pgfpathlineto{\pgfqpoint{4.568995in}{2.010001in}}%
\pgfpathlineto{\pgfqpoint{4.568995in}{2.007051in}}%
\pgfpathmoveto{\pgfqpoint{4.568995in}{2.007051in}}%
\pgfpathlineto{\pgfqpoint{4.568995in}{2.007051in}}%
\pgfpathlineto{\pgfqpoint{4.568995in}{2.010001in}}%
\pgfpathlineto{\pgfqpoint{4.573536in}{2.010001in}}%
\pgfpathlineto{\pgfqpoint{4.573536in}{2.007051in}}%
\pgfpathmoveto{\pgfqpoint{4.573536in}{2.007051in}}%
\pgfpathlineto{\pgfqpoint{4.573536in}{2.007051in}}%
\pgfpathlineto{\pgfqpoint{4.573536in}{2.010001in}}%
\pgfpathlineto{\pgfqpoint{4.578076in}{2.010001in}}%
\pgfpathlineto{\pgfqpoint{4.578076in}{2.007051in}}%
\pgfpathmoveto{\pgfqpoint{4.578076in}{2.007051in}}%
\pgfpathlineto{\pgfqpoint{4.578076in}{2.007051in}}%
\pgfpathlineto{\pgfqpoint{4.578076in}{2.010001in}}%
\pgfpathlineto{\pgfqpoint{4.582617in}{2.010001in}}%
\pgfpathlineto{\pgfqpoint{4.582617in}{2.007051in}}%
\pgfpathmoveto{\pgfqpoint{4.582617in}{2.007051in}}%
\pgfpathlineto{\pgfqpoint{4.582617in}{2.007051in}}%
\pgfpathlineto{\pgfqpoint{4.582617in}{2.010001in}}%
\pgfpathlineto{\pgfqpoint{4.587158in}{2.010001in}}%
\pgfpathlineto{\pgfqpoint{4.587158in}{2.007051in}}%
\pgfpathmoveto{\pgfqpoint{4.587158in}{2.007051in}}%
\pgfpathlineto{\pgfqpoint{4.587158in}{2.007051in}}%
\pgfpathlineto{\pgfqpoint{4.587158in}{2.010001in}}%
\pgfpathlineto{\pgfqpoint{4.591699in}{2.010001in}}%
\pgfpathlineto{\pgfqpoint{4.591699in}{2.007051in}}%
\pgfpathmoveto{\pgfqpoint{4.591699in}{2.007051in}}%
\pgfpathlineto{\pgfqpoint{4.591699in}{2.007051in}}%
\pgfpathlineto{\pgfqpoint{4.591699in}{2.010001in}}%
\pgfpathlineto{\pgfqpoint{4.596240in}{2.010001in}}%
\pgfpathlineto{\pgfqpoint{4.596240in}{2.007051in}}%
\pgfpathmoveto{\pgfqpoint{4.596240in}{2.007051in}}%
\pgfpathlineto{\pgfqpoint{4.596240in}{2.007051in}}%
\pgfpathlineto{\pgfqpoint{4.596240in}{2.010001in}}%
\pgfpathlineto{\pgfqpoint{4.600781in}{2.010001in}}%
\pgfpathlineto{\pgfqpoint{4.600781in}{2.007051in}}%
\pgfpathmoveto{\pgfqpoint{4.600781in}{2.007051in}}%
\pgfpathlineto{\pgfqpoint{4.600781in}{2.007051in}}%
\pgfpathlineto{\pgfqpoint{4.600781in}{2.010001in}}%
\pgfpathlineto{\pgfqpoint{4.605322in}{2.010001in}}%
\pgfpathlineto{\pgfqpoint{4.605322in}{2.007051in}}%
\pgfpathmoveto{\pgfqpoint{4.605322in}{2.007051in}}%
\pgfpathlineto{\pgfqpoint{4.605322in}{2.007051in}}%
\pgfpathlineto{\pgfqpoint{4.605322in}{2.010001in}}%
\pgfpathlineto{\pgfqpoint{4.609863in}{2.010001in}}%
\pgfpathlineto{\pgfqpoint{4.609863in}{2.007051in}}%
\pgfpathmoveto{\pgfqpoint{4.609863in}{2.007051in}}%
\pgfpathlineto{\pgfqpoint{4.609863in}{2.007051in}}%
\pgfpathlineto{\pgfqpoint{4.609863in}{2.010001in}}%
\pgfpathlineto{\pgfqpoint{4.614404in}{2.010001in}}%
\pgfpathlineto{\pgfqpoint{4.614404in}{2.007051in}}%
\pgfpathmoveto{\pgfqpoint{4.614404in}{2.007051in}}%
\pgfpathlineto{\pgfqpoint{4.614404in}{2.007051in}}%
\pgfpathlineto{\pgfqpoint{4.614404in}{2.010001in}}%
\pgfpathlineto{\pgfqpoint{4.618944in}{2.010001in}}%
\pgfpathlineto{\pgfqpoint{4.618944in}{2.007051in}}%
\pgfpathmoveto{\pgfqpoint{4.618944in}{2.007051in}}%
\pgfpathlineto{\pgfqpoint{4.618944in}{2.007051in}}%
\pgfpathlineto{\pgfqpoint{4.618944in}{2.010001in}}%
\pgfpathlineto{\pgfqpoint{4.623485in}{2.010001in}}%
\pgfpathlineto{\pgfqpoint{4.623485in}{2.007051in}}%
\pgfpathmoveto{\pgfqpoint{4.623485in}{2.007051in}}%
\pgfpathlineto{\pgfqpoint{4.623485in}{2.007051in}}%
\pgfpathlineto{\pgfqpoint{4.623485in}{2.010001in}}%
\pgfpathlineto{\pgfqpoint{4.628026in}{2.010001in}}%
\pgfpathlineto{\pgfqpoint{4.628026in}{2.007051in}}%
\pgfpathmoveto{\pgfqpoint{4.628026in}{2.007051in}}%
\pgfpathlineto{\pgfqpoint{4.628026in}{2.007051in}}%
\pgfpathlineto{\pgfqpoint{4.628026in}{2.010001in}}%
\pgfpathlineto{\pgfqpoint{4.632567in}{2.010001in}}%
\pgfpathlineto{\pgfqpoint{4.632567in}{2.007051in}}%
\pgfpathmoveto{\pgfqpoint{4.632567in}{2.007051in}}%
\pgfpathlineto{\pgfqpoint{4.632567in}{2.007051in}}%
\pgfpathlineto{\pgfqpoint{4.632567in}{2.010001in}}%
\pgfpathlineto{\pgfqpoint{4.637108in}{2.010001in}}%
\pgfpathlineto{\pgfqpoint{4.637108in}{2.007051in}}%
\pgfpathmoveto{\pgfqpoint{4.637108in}{2.007051in}}%
\pgfpathlineto{\pgfqpoint{4.637108in}{2.007051in}}%
\pgfpathlineto{\pgfqpoint{4.637108in}{2.010001in}}%
\pgfpathlineto{\pgfqpoint{4.641649in}{2.010001in}}%
\pgfpathlineto{\pgfqpoint{4.641649in}{2.007051in}}%
\pgfpathmoveto{\pgfqpoint{4.641649in}{2.007051in}}%
\pgfpathlineto{\pgfqpoint{4.641649in}{2.007051in}}%
\pgfpathlineto{\pgfqpoint{4.641649in}{2.010001in}}%
\pgfpathlineto{\pgfqpoint{4.646190in}{2.010001in}}%
\pgfpathlineto{\pgfqpoint{4.646190in}{2.007051in}}%
\pgfpathmoveto{\pgfqpoint{4.646190in}{2.007051in}}%
\pgfpathlineto{\pgfqpoint{4.646190in}{2.007051in}}%
\pgfpathlineto{\pgfqpoint{4.646190in}{2.010001in}}%
\pgfpathlineto{\pgfqpoint{4.650731in}{2.010001in}}%
\pgfpathlineto{\pgfqpoint{4.650731in}{2.007051in}}%
\pgfpathmoveto{\pgfqpoint{4.650731in}{2.007051in}}%
\pgfpathlineto{\pgfqpoint{4.650731in}{2.007051in}}%
\pgfpathlineto{\pgfqpoint{4.650731in}{2.010001in}}%
\pgfpathlineto{\pgfqpoint{4.655272in}{2.010001in}}%
\pgfpathlineto{\pgfqpoint{4.655272in}{2.007051in}}%
\pgfpathmoveto{\pgfqpoint{4.655272in}{2.007051in}}%
\pgfpathlineto{\pgfqpoint{4.655272in}{2.007051in}}%
\pgfpathlineto{\pgfqpoint{4.655272in}{2.010001in}}%
\pgfpathlineto{\pgfqpoint{4.659812in}{2.010001in}}%
\pgfpathlineto{\pgfqpoint{4.659812in}{2.007051in}}%
\pgfpathmoveto{\pgfqpoint{4.659812in}{2.007051in}}%
\pgfpathlineto{\pgfqpoint{4.659812in}{2.007051in}}%
\pgfpathlineto{\pgfqpoint{4.659812in}{2.010001in}}%
\pgfpathlineto{\pgfqpoint{4.664353in}{2.010001in}}%
\pgfpathlineto{\pgfqpoint{4.664353in}{2.007051in}}%
\pgfpathmoveto{\pgfqpoint{4.664353in}{2.007051in}}%
\pgfpathlineto{\pgfqpoint{4.664353in}{2.007051in}}%
\pgfpathlineto{\pgfqpoint{4.664353in}{2.010001in}}%
\pgfpathlineto{\pgfqpoint{4.668894in}{2.010001in}}%
\pgfpathlineto{\pgfqpoint{4.668894in}{2.007051in}}%
\pgfpathmoveto{\pgfqpoint{4.668894in}{2.007051in}}%
\pgfpathlineto{\pgfqpoint{4.668894in}{2.007051in}}%
\pgfpathlineto{\pgfqpoint{4.668894in}{2.010001in}}%
\pgfpathlineto{\pgfqpoint{4.673435in}{2.010001in}}%
\pgfpathlineto{\pgfqpoint{4.673435in}{2.007051in}}%
\pgfpathmoveto{\pgfqpoint{4.673435in}{2.007051in}}%
\pgfpathlineto{\pgfqpoint{4.673435in}{2.007051in}}%
\pgfpathlineto{\pgfqpoint{4.673435in}{2.010001in}}%
\pgfpathlineto{\pgfqpoint{4.677976in}{2.010001in}}%
\pgfpathlineto{\pgfqpoint{4.677976in}{2.007051in}}%
\pgfpathmoveto{\pgfqpoint{4.677976in}{2.007051in}}%
\pgfpathlineto{\pgfqpoint{4.677976in}{2.007051in}}%
\pgfpathlineto{\pgfqpoint{4.677976in}{2.010001in}}%
\pgfpathlineto{\pgfqpoint{4.682517in}{2.010001in}}%
\pgfpathlineto{\pgfqpoint{4.682517in}{2.007051in}}%
\pgfpathmoveto{\pgfqpoint{4.682517in}{2.007051in}}%
\pgfpathlineto{\pgfqpoint{4.682517in}{2.007051in}}%
\pgfpathlineto{\pgfqpoint{4.682517in}{2.010001in}}%
\pgfpathlineto{\pgfqpoint{4.687059in}{2.010001in}}%
\pgfpathlineto{\pgfqpoint{4.687059in}{2.007051in}}%
\pgfpathmoveto{\pgfqpoint{4.687059in}{2.007051in}}%
\pgfpathlineto{\pgfqpoint{4.687059in}{2.007051in}}%
\pgfpathlineto{\pgfqpoint{4.687059in}{2.010001in}}%
\pgfpathlineto{\pgfqpoint{4.691600in}{2.010001in}}%
\pgfpathlineto{\pgfqpoint{4.691600in}{2.007051in}}%
\pgfpathmoveto{\pgfqpoint{4.691600in}{2.007051in}}%
\pgfpathlineto{\pgfqpoint{4.691600in}{2.007051in}}%
\pgfpathlineto{\pgfqpoint{4.691600in}{2.010001in}}%
\pgfpathlineto{\pgfqpoint{4.696141in}{2.010001in}}%
\pgfpathlineto{\pgfqpoint{4.696141in}{2.007051in}}%
\pgfpathmoveto{\pgfqpoint{4.696141in}{2.007051in}}%
\pgfpathlineto{\pgfqpoint{4.696141in}{2.007051in}}%
\pgfpathlineto{\pgfqpoint{4.696141in}{2.010001in}}%
\pgfpathlineto{\pgfqpoint{4.700682in}{2.010001in}}%
\pgfpathlineto{\pgfqpoint{4.700682in}{2.007051in}}%
\pgfpathmoveto{\pgfqpoint{4.700682in}{2.007051in}}%
\pgfpathlineto{\pgfqpoint{4.700682in}{2.007051in}}%
\pgfpathlineto{\pgfqpoint{4.700682in}{2.010001in}}%
\pgfpathlineto{\pgfqpoint{4.705223in}{2.010001in}}%
\pgfpathlineto{\pgfqpoint{4.705223in}{2.007051in}}%
\pgfpathmoveto{\pgfqpoint{4.705223in}{2.007051in}}%
\pgfpathlineto{\pgfqpoint{4.705223in}{2.007051in}}%
\pgfpathlineto{\pgfqpoint{4.705223in}{2.010001in}}%
\pgfpathlineto{\pgfqpoint{4.709765in}{2.010001in}}%
\pgfpathlineto{\pgfqpoint{4.709765in}{2.007051in}}%
\pgfpathmoveto{\pgfqpoint{4.709765in}{2.007051in}}%
\pgfpathlineto{\pgfqpoint{4.709765in}{2.007051in}}%
\pgfpathlineto{\pgfqpoint{4.709765in}{2.010001in}}%
\pgfpathlineto{\pgfqpoint{4.714306in}{2.010001in}}%
\pgfpathlineto{\pgfqpoint{4.714306in}{2.007051in}}%
\pgfpathmoveto{\pgfqpoint{4.714306in}{2.007051in}}%
\pgfpathlineto{\pgfqpoint{4.714306in}{2.007051in}}%
\pgfpathlineto{\pgfqpoint{4.714306in}{2.010001in}}%
\pgfpathlineto{\pgfqpoint{4.718847in}{2.010001in}}%
\pgfpathlineto{\pgfqpoint{4.718847in}{2.007051in}}%
\pgfpathmoveto{\pgfqpoint{4.718847in}{2.007051in}}%
\pgfpathlineto{\pgfqpoint{4.718847in}{2.007051in}}%
\pgfpathlineto{\pgfqpoint{4.718847in}{2.010001in}}%
\pgfpathlineto{\pgfqpoint{4.723388in}{2.010001in}}%
\pgfpathlineto{\pgfqpoint{4.723388in}{2.007051in}}%
\pgfpathmoveto{\pgfqpoint{4.723388in}{2.007051in}}%
\pgfpathlineto{\pgfqpoint{4.723388in}{2.007051in}}%
\pgfpathlineto{\pgfqpoint{4.723388in}{2.010001in}}%
\pgfpathlineto{\pgfqpoint{4.727929in}{2.010001in}}%
\pgfpathlineto{\pgfqpoint{4.727929in}{2.007051in}}%
\pgfpathmoveto{\pgfqpoint{4.727929in}{2.007051in}}%
\pgfpathlineto{\pgfqpoint{4.727929in}{2.007051in}}%
\pgfpathlineto{\pgfqpoint{4.727929in}{2.010001in}}%
\pgfpathlineto{\pgfqpoint{4.732471in}{2.010001in}}%
\pgfpathlineto{\pgfqpoint{4.732471in}{2.007051in}}%
\pgfpathmoveto{\pgfqpoint{4.732471in}{2.007051in}}%
\pgfpathlineto{\pgfqpoint{4.732471in}{2.007051in}}%
\pgfpathlineto{\pgfqpoint{4.732471in}{2.010001in}}%
\pgfpathlineto{\pgfqpoint{4.737012in}{2.010001in}}%
\pgfpathlineto{\pgfqpoint{4.737012in}{2.007051in}}%
\pgfpathmoveto{\pgfqpoint{4.737012in}{2.007051in}}%
\pgfpathlineto{\pgfqpoint{4.737012in}{2.007051in}}%
\pgfpathlineto{\pgfqpoint{4.737012in}{2.010001in}}%
\pgfpathlineto{\pgfqpoint{4.741553in}{2.010001in}}%
\pgfpathlineto{\pgfqpoint{4.741553in}{2.007051in}}%
\pgfpathmoveto{\pgfqpoint{4.741553in}{2.007051in}}%
\pgfpathlineto{\pgfqpoint{4.741553in}{2.007051in}}%
\pgfpathlineto{\pgfqpoint{4.741553in}{2.010001in}}%
\pgfpathlineto{\pgfqpoint{4.746094in}{2.010001in}}%
\pgfpathlineto{\pgfqpoint{4.746094in}{2.007051in}}%
\pgfpathmoveto{\pgfqpoint{4.746094in}{2.007051in}}%
\pgfpathlineto{\pgfqpoint{4.746094in}{2.007051in}}%
\pgfpathlineto{\pgfqpoint{4.746094in}{2.010001in}}%
\pgfpathlineto{\pgfqpoint{4.750635in}{2.010001in}}%
\pgfpathlineto{\pgfqpoint{4.750635in}{2.007051in}}%
\pgfpathmoveto{\pgfqpoint{4.750635in}{2.007051in}}%
\pgfpathlineto{\pgfqpoint{4.750635in}{2.007051in}}%
\pgfpathlineto{\pgfqpoint{4.750635in}{2.010001in}}%
\pgfpathlineto{\pgfqpoint{4.755177in}{2.010001in}}%
\pgfpathlineto{\pgfqpoint{4.755177in}{2.007051in}}%
\pgfpathmoveto{\pgfqpoint{4.755177in}{2.007051in}}%
\pgfpathlineto{\pgfqpoint{4.755177in}{2.007051in}}%
\pgfpathlineto{\pgfqpoint{4.755177in}{2.010001in}}%
\pgfpathlineto{\pgfqpoint{4.759718in}{2.010001in}}%
\pgfpathlineto{\pgfqpoint{4.759718in}{2.007051in}}%
\pgfpathmoveto{\pgfqpoint{4.759718in}{2.007051in}}%
\pgfpathlineto{\pgfqpoint{4.759718in}{2.007051in}}%
\pgfpathlineto{\pgfqpoint{4.759718in}{2.010001in}}%
\pgfpathlineto{\pgfqpoint{4.764259in}{2.010001in}}%
\pgfpathlineto{\pgfqpoint{4.764259in}{2.007051in}}%
\pgfpathmoveto{\pgfqpoint{4.764259in}{2.007051in}}%
\pgfpathlineto{\pgfqpoint{4.764259in}{2.007051in}}%
\pgfpathlineto{\pgfqpoint{4.764259in}{2.010001in}}%
\pgfpathlineto{\pgfqpoint{4.768800in}{2.010001in}}%
\pgfpathlineto{\pgfqpoint{4.768800in}{2.007051in}}%
\pgfpathmoveto{\pgfqpoint{4.768800in}{2.007051in}}%
\pgfpathlineto{\pgfqpoint{4.768800in}{2.007051in}}%
\pgfpathlineto{\pgfqpoint{4.768800in}{2.010001in}}%
\pgfpathlineto{\pgfqpoint{4.773341in}{2.010001in}}%
\pgfpathlineto{\pgfqpoint{4.773341in}{2.007051in}}%
\pgfpathmoveto{\pgfqpoint{4.773341in}{2.007051in}}%
\pgfpathlineto{\pgfqpoint{4.773341in}{2.007051in}}%
\pgfpathlineto{\pgfqpoint{4.773341in}{2.010001in}}%
\pgfpathlineto{\pgfqpoint{4.777883in}{2.010001in}}%
\pgfpathlineto{\pgfqpoint{4.777883in}{2.007051in}}%
\pgfpathmoveto{\pgfqpoint{4.777883in}{2.007051in}}%
\pgfpathlineto{\pgfqpoint{4.777883in}{2.007051in}}%
\pgfpathlineto{\pgfqpoint{4.777883in}{2.010001in}}%
\pgfpathlineto{\pgfqpoint{4.782424in}{2.010001in}}%
\pgfpathlineto{\pgfqpoint{4.782424in}{2.007051in}}%
\pgfpathmoveto{\pgfqpoint{4.782424in}{2.007051in}}%
\pgfpathlineto{\pgfqpoint{4.782424in}{2.007051in}}%
\pgfpathlineto{\pgfqpoint{4.782424in}{2.010001in}}%
\pgfpathlineto{\pgfqpoint{4.786965in}{2.010001in}}%
\pgfpathlineto{\pgfqpoint{4.786965in}{2.007051in}}%
\pgfpathmoveto{\pgfqpoint{4.786965in}{2.007051in}}%
\pgfpathlineto{\pgfqpoint{4.786965in}{2.007051in}}%
\pgfpathlineto{\pgfqpoint{4.786965in}{2.010001in}}%
\pgfpathlineto{\pgfqpoint{4.791506in}{2.010001in}}%
\pgfpathlineto{\pgfqpoint{4.791506in}{2.007051in}}%
\pgfpathmoveto{\pgfqpoint{4.791506in}{2.007051in}}%
\pgfpathlineto{\pgfqpoint{4.791506in}{2.007051in}}%
\pgfpathlineto{\pgfqpoint{4.791506in}{2.010001in}}%
\pgfpathlineto{\pgfqpoint{4.796047in}{2.010001in}}%
\pgfpathlineto{\pgfqpoint{4.796047in}{2.007051in}}%
\pgfpathmoveto{\pgfqpoint{4.796047in}{2.007051in}}%
\pgfpathlineto{\pgfqpoint{4.796047in}{2.007051in}}%
\pgfpathlineto{\pgfqpoint{4.796047in}{2.010001in}}%
\pgfpathlineto{\pgfqpoint{4.800589in}{2.010001in}}%
\pgfpathlineto{\pgfqpoint{4.800589in}{2.007051in}}%
\pgfpathmoveto{\pgfqpoint{4.800589in}{2.007051in}}%
\pgfpathlineto{\pgfqpoint{4.800589in}{2.007051in}}%
\pgfpathlineto{\pgfqpoint{4.800589in}{2.010001in}}%
\pgfpathlineto{\pgfqpoint{4.805130in}{2.010001in}}%
\pgfpathlineto{\pgfqpoint{4.805130in}{2.007051in}}%
\pgfpathmoveto{\pgfqpoint{4.805130in}{2.007051in}}%
\pgfpathlineto{\pgfqpoint{4.805130in}{2.007051in}}%
\pgfpathlineto{\pgfqpoint{4.805130in}{2.010001in}}%
\pgfpathlineto{\pgfqpoint{4.809671in}{2.010001in}}%
\pgfpathlineto{\pgfqpoint{4.809671in}{2.007051in}}%
\pgfpathmoveto{\pgfqpoint{4.809671in}{2.007051in}}%
\pgfpathlineto{\pgfqpoint{4.809671in}{2.007051in}}%
\pgfpathlineto{\pgfqpoint{4.809671in}{2.010001in}}%
\pgfpathlineto{\pgfqpoint{4.814212in}{2.010001in}}%
\pgfpathlineto{\pgfqpoint{4.814212in}{2.007051in}}%
\pgfpathmoveto{\pgfqpoint{4.814212in}{2.007051in}}%
\pgfpathlineto{\pgfqpoint{4.814212in}{2.007051in}}%
\pgfpathlineto{\pgfqpoint{4.814212in}{2.010001in}}%
\pgfpathlineto{\pgfqpoint{4.818753in}{2.010001in}}%
\pgfpathlineto{\pgfqpoint{4.818753in}{2.007051in}}%
\pgfpathmoveto{\pgfqpoint{4.818753in}{2.007051in}}%
\pgfpathlineto{\pgfqpoint{4.818753in}{2.007051in}}%
\pgfpathlineto{\pgfqpoint{4.818753in}{2.010001in}}%
\pgfpathlineto{\pgfqpoint{4.823294in}{2.010001in}}%
\pgfpathlineto{\pgfqpoint{4.823294in}{2.007051in}}%
\pgfpathmoveto{\pgfqpoint{4.823294in}{2.007051in}}%
\pgfpathlineto{\pgfqpoint{4.823294in}{2.007051in}}%
\pgfpathlineto{\pgfqpoint{4.823294in}{2.010001in}}%
\pgfpathlineto{\pgfqpoint{4.827835in}{2.010001in}}%
\pgfpathlineto{\pgfqpoint{4.827835in}{2.007051in}}%
\pgfpathmoveto{\pgfqpoint{4.827835in}{2.007051in}}%
\pgfpathlineto{\pgfqpoint{4.827835in}{2.007051in}}%
\pgfpathlineto{\pgfqpoint{4.827835in}{2.010001in}}%
\pgfpathlineto{\pgfqpoint{4.832376in}{2.010001in}}%
\pgfpathlineto{\pgfqpoint{4.832376in}{2.007051in}}%
\pgfpathmoveto{\pgfqpoint{4.832376in}{2.007051in}}%
\pgfpathlineto{\pgfqpoint{4.832376in}{2.007051in}}%
\pgfpathlineto{\pgfqpoint{4.832376in}{2.010001in}}%
\pgfpathlineto{\pgfqpoint{4.836917in}{2.010001in}}%
\pgfpathlineto{\pgfqpoint{4.836917in}{2.007051in}}%
\pgfpathmoveto{\pgfqpoint{4.836917in}{2.007051in}}%
\pgfpathlineto{\pgfqpoint{4.836917in}{2.007051in}}%
\pgfpathlineto{\pgfqpoint{4.836917in}{2.010001in}}%
\pgfpathlineto{\pgfqpoint{4.841458in}{2.010001in}}%
\pgfpathlineto{\pgfqpoint{4.841458in}{2.007051in}}%
\pgfpathmoveto{\pgfqpoint{4.841458in}{2.007051in}}%
\pgfpathlineto{\pgfqpoint{4.841458in}{2.007051in}}%
\pgfpathlineto{\pgfqpoint{4.841458in}{2.010001in}}%
\pgfpathlineto{\pgfqpoint{4.845999in}{2.010001in}}%
\pgfpathlineto{\pgfqpoint{4.845999in}{2.007051in}}%
\pgfpathmoveto{\pgfqpoint{4.845999in}{2.007051in}}%
\pgfpathlineto{\pgfqpoint{4.845999in}{2.007051in}}%
\pgfpathlineto{\pgfqpoint{4.845999in}{2.010001in}}%
\pgfpathlineto{\pgfqpoint{4.850540in}{2.010001in}}%
\pgfpathlineto{\pgfqpoint{4.850540in}{2.007051in}}%
\pgfpathmoveto{\pgfqpoint{4.850540in}{2.007051in}}%
\pgfpathlineto{\pgfqpoint{4.850540in}{2.007051in}}%
\pgfpathlineto{\pgfqpoint{4.850540in}{2.010001in}}%
\pgfpathlineto{\pgfqpoint{4.855081in}{2.010001in}}%
\pgfpathlineto{\pgfqpoint{4.855081in}{2.007051in}}%
\pgfpathmoveto{\pgfqpoint{4.855081in}{2.007051in}}%
\pgfpathlineto{\pgfqpoint{4.855081in}{2.007051in}}%
\pgfpathlineto{\pgfqpoint{4.855081in}{2.010001in}}%
\pgfpathlineto{\pgfqpoint{4.859622in}{2.010001in}}%
\pgfpathlineto{\pgfqpoint{4.859622in}{2.007051in}}%
\pgfpathmoveto{\pgfqpoint{4.859622in}{2.007051in}}%
\pgfpathlineto{\pgfqpoint{4.859622in}{2.007051in}}%
\pgfpathlineto{\pgfqpoint{4.859622in}{2.010001in}}%
\pgfpathlineto{\pgfqpoint{4.864163in}{2.010001in}}%
\pgfpathlineto{\pgfqpoint{4.864163in}{2.007051in}}%
\pgfpathmoveto{\pgfqpoint{4.864163in}{2.007051in}}%
\pgfpathlineto{\pgfqpoint{4.864163in}{2.007051in}}%
\pgfpathlineto{\pgfqpoint{4.864163in}{2.010001in}}%
\pgfpathlineto{\pgfqpoint{4.868704in}{2.010001in}}%
\pgfpathlineto{\pgfqpoint{4.868704in}{2.007051in}}%
\pgfpathmoveto{\pgfqpoint{4.868704in}{2.007051in}}%
\pgfpathlineto{\pgfqpoint{4.868704in}{2.007051in}}%
\pgfpathlineto{\pgfqpoint{4.868704in}{2.010001in}}%
\pgfpathlineto{\pgfqpoint{4.873245in}{2.010001in}}%
\pgfpathlineto{\pgfqpoint{4.873245in}{2.007051in}}%
\pgfpathmoveto{\pgfqpoint{4.873245in}{2.007051in}}%
\pgfpathlineto{\pgfqpoint{4.873245in}{2.007051in}}%
\pgfpathlineto{\pgfqpoint{4.873245in}{2.010001in}}%
\pgfpathlineto{\pgfqpoint{4.877786in}{2.010001in}}%
\pgfpathlineto{\pgfqpoint{4.877786in}{2.007051in}}%
\pgfpathmoveto{\pgfqpoint{4.877786in}{2.007051in}}%
\pgfpathlineto{\pgfqpoint{4.877786in}{2.007051in}}%
\pgfpathlineto{\pgfqpoint{4.877786in}{2.010001in}}%
\pgfpathlineto{\pgfqpoint{4.882327in}{2.010001in}}%
\pgfpathlineto{\pgfqpoint{4.882327in}{2.007051in}}%
\pgfpathmoveto{\pgfqpoint{4.882327in}{2.007051in}}%
\pgfpathlineto{\pgfqpoint{4.882327in}{2.007051in}}%
\pgfpathlineto{\pgfqpoint{4.882327in}{2.010001in}}%
\pgfpathlineto{\pgfqpoint{4.886868in}{2.010001in}}%
\pgfpathlineto{\pgfqpoint{4.886868in}{2.007051in}}%
\pgfpathmoveto{\pgfqpoint{4.886868in}{2.007051in}}%
\pgfpathlineto{\pgfqpoint{4.886868in}{2.007051in}}%
\pgfpathlineto{\pgfqpoint{4.886868in}{2.010001in}}%
\pgfpathlineto{\pgfqpoint{4.891409in}{2.010001in}}%
\pgfpathlineto{\pgfqpoint{4.891409in}{2.007051in}}%
\pgfpathmoveto{\pgfqpoint{4.891409in}{2.007051in}}%
\pgfpathlineto{\pgfqpoint{4.891409in}{2.007051in}}%
\pgfpathlineto{\pgfqpoint{4.891409in}{2.010001in}}%
\pgfpathlineto{\pgfqpoint{4.895950in}{2.010001in}}%
\pgfpathlineto{\pgfqpoint{4.895950in}{2.007051in}}%
\pgfpathmoveto{\pgfqpoint{4.895950in}{2.007051in}}%
\pgfpathlineto{\pgfqpoint{4.895950in}{2.007051in}}%
\pgfpathlineto{\pgfqpoint{4.895950in}{2.010001in}}%
\pgfpathlineto{\pgfqpoint{4.900491in}{2.010001in}}%
\pgfpathlineto{\pgfqpoint{4.900491in}{2.007051in}}%
\pgfpathmoveto{\pgfqpoint{4.900491in}{2.007051in}}%
\pgfpathlineto{\pgfqpoint{4.900491in}{2.007051in}}%
\pgfpathlineto{\pgfqpoint{4.900491in}{2.010001in}}%
\pgfpathlineto{\pgfqpoint{4.905031in}{2.010001in}}%
\pgfpathlineto{\pgfqpoint{4.905031in}{2.007051in}}%
\pgfpathmoveto{\pgfqpoint{4.905031in}{2.007051in}}%
\pgfpathlineto{\pgfqpoint{4.905031in}{2.007051in}}%
\pgfpathlineto{\pgfqpoint{4.905031in}{2.010001in}}%
\pgfpathlineto{\pgfqpoint{4.909572in}{2.010001in}}%
\pgfpathlineto{\pgfqpoint{4.909572in}{2.007051in}}%
\pgfpathmoveto{\pgfqpoint{4.909572in}{2.007051in}}%
\pgfpathlineto{\pgfqpoint{4.909572in}{2.007051in}}%
\pgfpathlineto{\pgfqpoint{4.909572in}{2.010001in}}%
\pgfpathlineto{\pgfqpoint{4.914113in}{2.010001in}}%
\pgfpathlineto{\pgfqpoint{4.914113in}{2.007051in}}%
\pgfpathmoveto{\pgfqpoint{4.914113in}{2.007051in}}%
\pgfpathlineto{\pgfqpoint{4.914113in}{2.007051in}}%
\pgfpathlineto{\pgfqpoint{4.914113in}{2.010001in}}%
\pgfpathlineto{\pgfqpoint{4.918654in}{2.010001in}}%
\pgfpathlineto{\pgfqpoint{4.918654in}{2.007051in}}%
\pgfpathmoveto{\pgfqpoint{4.918654in}{2.007051in}}%
\pgfpathlineto{\pgfqpoint{4.918654in}{2.007051in}}%
\pgfpathlineto{\pgfqpoint{4.918654in}{2.010001in}}%
\pgfpathlineto{\pgfqpoint{4.923195in}{2.010001in}}%
\pgfpathlineto{\pgfqpoint{4.923195in}{2.007051in}}%
\pgfpathmoveto{\pgfqpoint{4.923195in}{2.007051in}}%
\pgfpathlineto{\pgfqpoint{4.923195in}{2.007051in}}%
\pgfpathlineto{\pgfqpoint{4.923195in}{2.010001in}}%
\pgfpathlineto{\pgfqpoint{4.927736in}{2.010001in}}%
\pgfpathlineto{\pgfqpoint{4.927736in}{2.007051in}}%
\pgfpathmoveto{\pgfqpoint{4.927736in}{2.007051in}}%
\pgfpathlineto{\pgfqpoint{4.927736in}{2.007051in}}%
\pgfpathlineto{\pgfqpoint{4.927736in}{2.010001in}}%
\pgfpathlineto{\pgfqpoint{4.932277in}{2.010001in}}%
\pgfpathlineto{\pgfqpoint{4.932277in}{2.007051in}}%
\pgfpathmoveto{\pgfqpoint{4.932277in}{2.007051in}}%
\pgfpathlineto{\pgfqpoint{4.932277in}{2.007051in}}%
\pgfpathlineto{\pgfqpoint{4.932277in}{2.010001in}}%
\pgfpathlineto{\pgfqpoint{4.936818in}{2.010001in}}%
\pgfpathlineto{\pgfqpoint{4.936818in}{2.007051in}}%
\pgfpathmoveto{\pgfqpoint{4.936818in}{2.007051in}}%
\pgfpathlineto{\pgfqpoint{4.936818in}{2.007051in}}%
\pgfpathlineto{\pgfqpoint{4.936818in}{2.010001in}}%
\pgfpathlineto{\pgfqpoint{4.941359in}{2.010001in}}%
\pgfpathlineto{\pgfqpoint{4.941359in}{2.007051in}}%
\pgfpathmoveto{\pgfqpoint{4.941359in}{2.007051in}}%
\pgfpathlineto{\pgfqpoint{4.941359in}{2.007051in}}%
\pgfpathlineto{\pgfqpoint{4.941359in}{2.010001in}}%
\pgfpathlineto{\pgfqpoint{4.945900in}{2.010001in}}%
\pgfpathlineto{\pgfqpoint{4.945900in}{2.007051in}}%
\pgfpathmoveto{\pgfqpoint{4.945900in}{2.007051in}}%
\pgfpathlineto{\pgfqpoint{4.945900in}{2.007051in}}%
\pgfpathlineto{\pgfqpoint{4.945900in}{2.010001in}}%
\pgfpathlineto{\pgfqpoint{4.950441in}{2.010001in}}%
\pgfpathlineto{\pgfqpoint{4.950441in}{2.007051in}}%
\pgfpathmoveto{\pgfqpoint{4.950441in}{2.007051in}}%
\pgfpathlineto{\pgfqpoint{4.950441in}{2.007051in}}%
\pgfpathlineto{\pgfqpoint{4.950441in}{2.010001in}}%
\pgfpathlineto{\pgfqpoint{4.954982in}{2.010001in}}%
\pgfpathlineto{\pgfqpoint{4.954982in}{2.007051in}}%
\pgfpathmoveto{\pgfqpoint{4.954982in}{2.007051in}}%
\pgfpathlineto{\pgfqpoint{4.954982in}{2.007051in}}%
\pgfpathlineto{\pgfqpoint{4.954982in}{2.010001in}}%
\pgfpathlineto{\pgfqpoint{4.959523in}{2.010001in}}%
\pgfpathlineto{\pgfqpoint{4.959523in}{2.007051in}}%
\pgfpathmoveto{\pgfqpoint{4.959523in}{2.007051in}}%
\pgfpathlineto{\pgfqpoint{4.959523in}{2.007051in}}%
\pgfpathlineto{\pgfqpoint{4.959523in}{2.010001in}}%
\pgfpathlineto{\pgfqpoint{4.964064in}{2.010001in}}%
\pgfpathlineto{\pgfqpoint{4.964064in}{2.007051in}}%
\pgfpathmoveto{\pgfqpoint{4.964064in}{2.007051in}}%
\pgfpathlineto{\pgfqpoint{4.964064in}{2.007051in}}%
\pgfpathlineto{\pgfqpoint{4.964064in}{2.010001in}}%
\pgfpathlineto{\pgfqpoint{4.968605in}{2.010001in}}%
\pgfpathlineto{\pgfqpoint{4.968605in}{2.007051in}}%
\pgfpathmoveto{\pgfqpoint{4.968605in}{2.007051in}}%
\pgfpathlineto{\pgfqpoint{4.968605in}{2.007051in}}%
\pgfpathlineto{\pgfqpoint{4.968605in}{2.010001in}}%
\pgfpathlineto{\pgfqpoint{4.973146in}{2.010001in}}%
\pgfpathlineto{\pgfqpoint{4.973146in}{2.007051in}}%
\pgfpathmoveto{\pgfqpoint{4.973146in}{2.007051in}}%
\pgfpathlineto{\pgfqpoint{4.973146in}{2.007051in}}%
\pgfpathlineto{\pgfqpoint{4.973146in}{2.010001in}}%
\pgfpathlineto{\pgfqpoint{4.977686in}{2.010001in}}%
\pgfpathlineto{\pgfqpoint{4.977686in}{2.007051in}}%
\pgfpathmoveto{\pgfqpoint{4.977686in}{2.007051in}}%
\pgfpathlineto{\pgfqpoint{4.977686in}{2.007051in}}%
\pgfpathlineto{\pgfqpoint{4.977686in}{2.010001in}}%
\pgfpathlineto{\pgfqpoint{4.982227in}{2.010001in}}%
\pgfpathlineto{\pgfqpoint{4.982227in}{2.007051in}}%
\pgfpathmoveto{\pgfqpoint{4.982227in}{2.007051in}}%
\pgfpathlineto{\pgfqpoint{4.982227in}{2.007051in}}%
\pgfpathlineto{\pgfqpoint{4.982227in}{2.010001in}}%
\pgfpathlineto{\pgfqpoint{4.986768in}{2.010001in}}%
\pgfpathlineto{\pgfqpoint{4.986768in}{2.007051in}}%
\pgfpathmoveto{\pgfqpoint{4.986768in}{2.007051in}}%
\pgfpathlineto{\pgfqpoint{4.986768in}{2.007051in}}%
\pgfpathlineto{\pgfqpoint{4.986768in}{2.010001in}}%
\pgfpathlineto{\pgfqpoint{4.991309in}{2.010001in}}%
\pgfpathlineto{\pgfqpoint{4.991309in}{2.007051in}}%
\pgfpathmoveto{\pgfqpoint{4.991309in}{2.007051in}}%
\pgfpathlineto{\pgfqpoint{4.991309in}{2.007051in}}%
\pgfpathlineto{\pgfqpoint{4.991309in}{2.010001in}}%
\pgfpathlineto{\pgfqpoint{4.995850in}{2.010001in}}%
\pgfpathlineto{\pgfqpoint{4.995850in}{2.007051in}}%
\pgfpathmoveto{\pgfqpoint{4.995850in}{2.007051in}}%
\pgfpathlineto{\pgfqpoint{4.995850in}{2.007051in}}%
\pgfpathlineto{\pgfqpoint{4.995850in}{2.010001in}}%
\pgfpathlineto{\pgfqpoint{5.000391in}{2.010001in}}%
\pgfpathlineto{\pgfqpoint{5.000391in}{2.007051in}}%
\pgfpathmoveto{\pgfqpoint{5.000391in}{2.007051in}}%
\pgfpathlineto{\pgfqpoint{5.000391in}{2.007051in}}%
\pgfpathlineto{\pgfqpoint{5.000391in}{2.010001in}}%
\pgfpathlineto{\pgfqpoint{5.004932in}{2.010001in}}%
\pgfpathlineto{\pgfqpoint{5.004932in}{2.007051in}}%
\pgfpathmoveto{\pgfqpoint{5.004932in}{2.007051in}}%
\pgfpathlineto{\pgfqpoint{5.004932in}{2.007051in}}%
\pgfpathlineto{\pgfqpoint{5.004932in}{2.010001in}}%
\pgfpathlineto{\pgfqpoint{5.009472in}{2.010001in}}%
\pgfpathlineto{\pgfqpoint{5.009472in}{2.007051in}}%
\pgfpathmoveto{\pgfqpoint{5.009472in}{2.007051in}}%
\pgfpathlineto{\pgfqpoint{5.009472in}{2.007051in}}%
\pgfpathlineto{\pgfqpoint{5.009472in}{2.010001in}}%
\pgfpathlineto{\pgfqpoint{5.014013in}{2.010001in}}%
\pgfpathlineto{\pgfqpoint{5.014013in}{2.007051in}}%
\pgfpathmoveto{\pgfqpoint{5.014013in}{2.007051in}}%
\pgfpathlineto{\pgfqpoint{5.014013in}{2.007051in}}%
\pgfpathlineto{\pgfqpoint{5.014013in}{2.010001in}}%
\pgfpathlineto{\pgfqpoint{5.018554in}{2.010001in}}%
\pgfpathlineto{\pgfqpoint{5.018554in}{2.007051in}}%
\pgfpathmoveto{\pgfqpoint{5.018554in}{2.007051in}}%
\pgfpathlineto{\pgfqpoint{5.018554in}{2.007051in}}%
\pgfpathlineto{\pgfqpoint{5.018554in}{2.010001in}}%
\pgfpathlineto{\pgfqpoint{5.023095in}{2.010001in}}%
\pgfpathlineto{\pgfqpoint{5.023095in}{2.007051in}}%
\pgfpathmoveto{\pgfqpoint{5.023095in}{2.007051in}}%
\pgfpathlineto{\pgfqpoint{5.023095in}{2.007051in}}%
\pgfpathlineto{\pgfqpoint{5.023095in}{2.010001in}}%
\pgfpathlineto{\pgfqpoint{5.027636in}{2.010001in}}%
\pgfpathlineto{\pgfqpoint{5.027636in}{2.007051in}}%
\pgfpathmoveto{\pgfqpoint{5.027636in}{2.007051in}}%
\pgfpathlineto{\pgfqpoint{5.027636in}{2.007051in}}%
\pgfpathlineto{\pgfqpoint{5.027636in}{2.010001in}}%
\pgfpathlineto{\pgfqpoint{5.032177in}{2.010001in}}%
\pgfpathlineto{\pgfqpoint{5.032177in}{2.007051in}}%
\pgfpathmoveto{\pgfqpoint{5.032177in}{2.007051in}}%
\pgfpathlineto{\pgfqpoint{5.032177in}{2.007051in}}%
\pgfpathlineto{\pgfqpoint{5.032177in}{2.010001in}}%
\pgfpathlineto{\pgfqpoint{5.036718in}{2.010001in}}%
\pgfpathlineto{\pgfqpoint{5.036718in}{2.007051in}}%
\pgfpathmoveto{\pgfqpoint{5.036718in}{2.007051in}}%
\pgfpathlineto{\pgfqpoint{5.036718in}{2.007051in}}%
\pgfpathlineto{\pgfqpoint{5.036718in}{2.010001in}}%
\pgfpathlineto{\pgfqpoint{5.041259in}{2.010001in}}%
\pgfpathlineto{\pgfqpoint{5.041259in}{2.007051in}}%
\pgfpathmoveto{\pgfqpoint{5.041259in}{2.007051in}}%
\pgfpathlineto{\pgfqpoint{5.041259in}{2.007051in}}%
\pgfpathlineto{\pgfqpoint{5.041259in}{2.010001in}}%
\pgfpathlineto{\pgfqpoint{5.045799in}{2.010001in}}%
\pgfpathlineto{\pgfqpoint{5.045799in}{2.007051in}}%
\pgfpathmoveto{\pgfqpoint{5.045799in}{2.007051in}}%
\pgfpathlineto{\pgfqpoint{5.045799in}{2.007051in}}%
\pgfpathlineto{\pgfqpoint{5.045799in}{2.010001in}}%
\pgfpathlineto{\pgfqpoint{5.050340in}{2.010001in}}%
\pgfpathlineto{\pgfqpoint{5.050340in}{2.007051in}}%
\pgfpathmoveto{\pgfqpoint{5.050340in}{2.007051in}}%
\pgfpathlineto{\pgfqpoint{5.050340in}{2.007051in}}%
\pgfpathlineto{\pgfqpoint{5.050340in}{2.010001in}}%
\pgfpathlineto{\pgfqpoint{5.054881in}{2.010001in}}%
\pgfpathlineto{\pgfqpoint{5.054881in}{2.007051in}}%
\pgfpathmoveto{\pgfqpoint{5.054881in}{2.007051in}}%
\pgfpathlineto{\pgfqpoint{5.054881in}{2.007051in}}%
\pgfpathlineto{\pgfqpoint{5.054881in}{2.010001in}}%
\pgfpathlineto{\pgfqpoint{5.059422in}{2.010001in}}%
\pgfpathlineto{\pgfqpoint{5.059422in}{2.007051in}}%
\pgfpathmoveto{\pgfqpoint{5.059422in}{2.007051in}}%
\pgfpathlineto{\pgfqpoint{5.059422in}{2.007051in}}%
\pgfpathlineto{\pgfqpoint{5.059422in}{2.010001in}}%
\pgfpathlineto{\pgfqpoint{5.063963in}{2.010001in}}%
\pgfpathlineto{\pgfqpoint{5.063963in}{2.007051in}}%
\pgfpathmoveto{\pgfqpoint{5.063963in}{2.007051in}}%
\pgfpathlineto{\pgfqpoint{5.063963in}{2.007051in}}%
\pgfpathlineto{\pgfqpoint{5.063963in}{2.010001in}}%
\pgfpathlineto{\pgfqpoint{5.068504in}{2.010001in}}%
\pgfpathlineto{\pgfqpoint{5.068504in}{2.007051in}}%
\pgfpathmoveto{\pgfqpoint{5.068504in}{2.007051in}}%
\pgfpathlineto{\pgfqpoint{5.068504in}{2.007051in}}%
\pgfpathlineto{\pgfqpoint{5.068504in}{2.010001in}}%
\pgfpathlineto{\pgfqpoint{5.073045in}{2.010001in}}%
\pgfpathlineto{\pgfqpoint{5.073045in}{2.007051in}}%
\pgfpathmoveto{\pgfqpoint{5.073045in}{2.007051in}}%
\pgfpathlineto{\pgfqpoint{5.073045in}{2.007051in}}%
\pgfpathlineto{\pgfqpoint{5.073045in}{2.010001in}}%
\pgfpathlineto{\pgfqpoint{5.077585in}{2.010001in}}%
\pgfpathlineto{\pgfqpoint{5.077585in}{2.007051in}}%
\pgfpathmoveto{\pgfqpoint{5.077585in}{2.007051in}}%
\pgfpathlineto{\pgfqpoint{5.077585in}{2.007051in}}%
\pgfpathlineto{\pgfqpoint{5.077585in}{2.010001in}}%
\pgfpathlineto{\pgfqpoint{5.082126in}{2.010001in}}%
\pgfpathlineto{\pgfqpoint{5.082126in}{2.007051in}}%
\pgfpathmoveto{\pgfqpoint{5.082126in}{2.007051in}}%
\pgfpathlineto{\pgfqpoint{5.082126in}{2.007051in}}%
\pgfpathlineto{\pgfqpoint{5.082126in}{2.010001in}}%
\pgfpathlineto{\pgfqpoint{5.086667in}{2.010001in}}%
\pgfpathlineto{\pgfqpoint{5.086667in}{2.007051in}}%
\pgfpathmoveto{\pgfqpoint{5.086667in}{2.007051in}}%
\pgfpathlineto{\pgfqpoint{5.086667in}{2.007051in}}%
\pgfpathlineto{\pgfqpoint{5.086667in}{2.010001in}}%
\pgfpathlineto{\pgfqpoint{5.091208in}{2.010001in}}%
\pgfpathlineto{\pgfqpoint{5.091208in}{2.007051in}}%
\pgfpathmoveto{\pgfqpoint{5.091208in}{2.007051in}}%
\pgfpathlineto{\pgfqpoint{5.091208in}{2.007051in}}%
\pgfpathlineto{\pgfqpoint{5.091208in}{2.010001in}}%
\pgfpathlineto{\pgfqpoint{5.095749in}{2.010001in}}%
\pgfpathlineto{\pgfqpoint{5.095749in}{2.007051in}}%
\pgfpathmoveto{\pgfqpoint{5.095749in}{2.007051in}}%
\pgfpathlineto{\pgfqpoint{5.095749in}{2.007051in}}%
\pgfpathlineto{\pgfqpoint{5.095749in}{2.010001in}}%
\pgfpathlineto{\pgfqpoint{5.100290in}{2.010001in}}%
\pgfpathlineto{\pgfqpoint{5.100290in}{2.007051in}}%
\pgfpathmoveto{\pgfqpoint{5.100290in}{2.007051in}}%
\pgfpathlineto{\pgfqpoint{5.100290in}{2.007051in}}%
\pgfpathlineto{\pgfqpoint{5.100290in}{2.010001in}}%
\pgfpathlineto{\pgfqpoint{5.104831in}{2.010001in}}%
\pgfpathlineto{\pgfqpoint{5.104831in}{2.007051in}}%
\pgfpathmoveto{\pgfqpoint{5.104831in}{2.007051in}}%
\pgfpathlineto{\pgfqpoint{5.104831in}{2.007051in}}%
\pgfpathlineto{\pgfqpoint{5.104831in}{2.010001in}}%
\pgfpathlineto{\pgfqpoint{5.109371in}{2.010001in}}%
\pgfpathlineto{\pgfqpoint{5.109371in}{2.007051in}}%
\pgfpathmoveto{\pgfqpoint{5.109371in}{2.007051in}}%
\pgfpathlineto{\pgfqpoint{5.109371in}{2.007051in}}%
\pgfpathlineto{\pgfqpoint{5.109371in}{2.010001in}}%
\pgfpathlineto{\pgfqpoint{5.113913in}{2.010001in}}%
\pgfpathlineto{\pgfqpoint{5.113913in}{2.007051in}}%
\pgfpathmoveto{\pgfqpoint{5.113913in}{2.007051in}}%
\pgfpathlineto{\pgfqpoint{5.113913in}{2.007051in}}%
\pgfpathlineto{\pgfqpoint{5.113913in}{2.010001in}}%
\pgfpathlineto{\pgfqpoint{5.118454in}{2.010001in}}%
\pgfpathlineto{\pgfqpoint{5.118454in}{2.007051in}}%
\pgfpathmoveto{\pgfqpoint{5.118454in}{2.007051in}}%
\pgfpathlineto{\pgfqpoint{5.118454in}{2.007051in}}%
\pgfpathlineto{\pgfqpoint{5.118454in}{2.010001in}}%
\pgfpathlineto{\pgfqpoint{5.122995in}{2.010001in}}%
\pgfpathlineto{\pgfqpoint{5.122995in}{2.007051in}}%
\pgfpathmoveto{\pgfqpoint{5.122995in}{2.007051in}}%
\pgfpathlineto{\pgfqpoint{5.122995in}{2.007051in}}%
\pgfpathlineto{\pgfqpoint{5.122995in}{2.010001in}}%
\pgfpathlineto{\pgfqpoint{5.127536in}{2.010001in}}%
\pgfpathlineto{\pgfqpoint{5.127536in}{2.007051in}}%
\pgfpathmoveto{\pgfqpoint{5.127536in}{2.007051in}}%
\pgfpathlineto{\pgfqpoint{5.127536in}{2.007051in}}%
\pgfpathlineto{\pgfqpoint{5.127536in}{2.010001in}}%
\pgfpathlineto{\pgfqpoint{5.132077in}{2.010001in}}%
\pgfpathlineto{\pgfqpoint{5.132077in}{2.007051in}}%
\pgfpathmoveto{\pgfqpoint{5.132077in}{2.007051in}}%
\pgfpathlineto{\pgfqpoint{5.132077in}{2.007051in}}%
\pgfpathlineto{\pgfqpoint{5.132077in}{2.010001in}}%
\pgfpathlineto{\pgfqpoint{5.136619in}{2.010001in}}%
\pgfpathlineto{\pgfqpoint{5.136619in}{2.007051in}}%
\pgfpathmoveto{\pgfqpoint{5.136619in}{2.007051in}}%
\pgfpathlineto{\pgfqpoint{5.136619in}{2.007051in}}%
\pgfpathlineto{\pgfqpoint{5.136619in}{2.010001in}}%
\pgfpathlineto{\pgfqpoint{5.141160in}{2.010001in}}%
\pgfpathlineto{\pgfqpoint{5.141160in}{2.007051in}}%
\pgfpathmoveto{\pgfqpoint{5.141160in}{2.007051in}}%
\pgfpathlineto{\pgfqpoint{5.141160in}{2.007051in}}%
\pgfpathlineto{\pgfqpoint{5.141160in}{2.010001in}}%
\pgfpathlineto{\pgfqpoint{5.145701in}{2.010001in}}%
\pgfpathlineto{\pgfqpoint{5.145701in}{2.007051in}}%
\pgfpathmoveto{\pgfqpoint{5.145701in}{2.007051in}}%
\pgfpathlineto{\pgfqpoint{5.145701in}{2.007051in}}%
\pgfpathlineto{\pgfqpoint{5.145701in}{2.010001in}}%
\pgfpathlineto{\pgfqpoint{5.150242in}{2.010001in}}%
\pgfpathlineto{\pgfqpoint{5.150242in}{2.007051in}}%
\pgfpathmoveto{\pgfqpoint{5.150242in}{2.007051in}}%
\pgfpathlineto{\pgfqpoint{5.150242in}{2.007051in}}%
\pgfpathlineto{\pgfqpoint{5.150242in}{2.010001in}}%
\pgfpathlineto{\pgfqpoint{5.154783in}{2.010001in}}%
\pgfpathlineto{\pgfqpoint{5.154783in}{2.007051in}}%
\pgfpathmoveto{\pgfqpoint{5.154783in}{2.007051in}}%
\pgfpathlineto{\pgfqpoint{5.154783in}{2.007051in}}%
\pgfpathlineto{\pgfqpoint{5.154783in}{2.010001in}}%
\pgfpathlineto{\pgfqpoint{5.159325in}{2.010001in}}%
\pgfpathlineto{\pgfqpoint{5.159325in}{2.007051in}}%
\pgfpathmoveto{\pgfqpoint{5.159325in}{2.007051in}}%
\pgfpathlineto{\pgfqpoint{5.159325in}{2.007051in}}%
\pgfpathlineto{\pgfqpoint{5.159325in}{2.010001in}}%
\pgfpathlineto{\pgfqpoint{5.163866in}{2.010001in}}%
\pgfpathlineto{\pgfqpoint{5.163866in}{2.007051in}}%
\pgfpathmoveto{\pgfqpoint{5.163866in}{2.007051in}}%
\pgfpathlineto{\pgfqpoint{5.163866in}{2.007051in}}%
\pgfpathlineto{\pgfqpoint{5.163866in}{2.010001in}}%
\pgfpathlineto{\pgfqpoint{5.168407in}{2.010001in}}%
\pgfpathlineto{\pgfqpoint{5.168407in}{2.007051in}}%
\pgfpathmoveto{\pgfqpoint{5.168407in}{2.007051in}}%
\pgfpathlineto{\pgfqpoint{5.168407in}{2.007051in}}%
\pgfpathlineto{\pgfqpoint{5.168407in}{2.010001in}}%
\pgfpathlineto{\pgfqpoint{5.172948in}{2.010001in}}%
\pgfpathlineto{\pgfqpoint{5.172948in}{2.007051in}}%
\pgfpathmoveto{\pgfqpoint{5.172948in}{2.007051in}}%
\pgfpathlineto{\pgfqpoint{5.172948in}{2.007051in}}%
\pgfpathlineto{\pgfqpoint{5.172948in}{2.010001in}}%
\pgfpathlineto{\pgfqpoint{5.177489in}{2.010001in}}%
\pgfpathlineto{\pgfqpoint{5.177489in}{2.007051in}}%
\pgfpathmoveto{\pgfqpoint{5.177489in}{2.007051in}}%
\pgfpathlineto{\pgfqpoint{5.177489in}{2.007051in}}%
\pgfpathlineto{\pgfqpoint{5.177489in}{2.010001in}}%
\pgfpathlineto{\pgfqpoint{5.182031in}{2.010001in}}%
\pgfpathlineto{\pgfqpoint{5.182031in}{2.007051in}}%
\pgfpathmoveto{\pgfqpoint{5.182031in}{2.007051in}}%
\pgfpathlineto{\pgfqpoint{5.182031in}{2.007051in}}%
\pgfpathlineto{\pgfqpoint{5.182031in}{2.010001in}}%
\pgfpathlineto{\pgfqpoint{5.186572in}{2.010001in}}%
\pgfpathlineto{\pgfqpoint{5.186572in}{2.007051in}}%
\pgfpathmoveto{\pgfqpoint{5.186572in}{2.007051in}}%
\pgfpathlineto{\pgfqpoint{5.186572in}{2.007051in}}%
\pgfpathlineto{\pgfqpoint{5.186572in}{2.010001in}}%
\pgfpathlineto{\pgfqpoint{5.191113in}{2.010001in}}%
\pgfpathlineto{\pgfqpoint{5.191113in}{2.007051in}}%
\pgfpathmoveto{\pgfqpoint{5.191113in}{2.007051in}}%
\pgfpathlineto{\pgfqpoint{5.191113in}{2.007051in}}%
\pgfpathlineto{\pgfqpoint{5.191113in}{2.010001in}}%
\pgfpathlineto{\pgfqpoint{5.195654in}{2.010001in}}%
\pgfpathlineto{\pgfqpoint{5.195654in}{2.007051in}}%
\pgfpathmoveto{\pgfqpoint{5.195654in}{2.007051in}}%
\pgfpathlineto{\pgfqpoint{5.195654in}{2.007051in}}%
\pgfpathlineto{\pgfqpoint{5.195654in}{2.010001in}}%
\pgfpathlineto{\pgfqpoint{5.200195in}{2.010001in}}%
\pgfpathlineto{\pgfqpoint{5.200195in}{2.007051in}}%
\pgfpathmoveto{\pgfqpoint{5.200195in}{2.007051in}}%
\pgfpathlineto{\pgfqpoint{5.200195in}{2.007051in}}%
\pgfpathlineto{\pgfqpoint{5.200195in}{2.010001in}}%
\pgfpathlineto{\pgfqpoint{5.204737in}{2.010001in}}%
\pgfpathlineto{\pgfqpoint{5.204737in}{2.007051in}}%
\pgfpathmoveto{\pgfqpoint{5.204737in}{2.007051in}}%
\pgfpathlineto{\pgfqpoint{5.204737in}{2.007051in}}%
\pgfpathlineto{\pgfqpoint{5.204737in}{2.010001in}}%
\pgfpathlineto{\pgfqpoint{5.209278in}{2.010001in}}%
\pgfpathlineto{\pgfqpoint{5.209278in}{2.007051in}}%
\pgfpathmoveto{\pgfqpoint{5.209278in}{2.007051in}}%
\pgfpathlineto{\pgfqpoint{5.209278in}{2.007051in}}%
\pgfpathlineto{\pgfqpoint{5.209278in}{2.010001in}}%
\pgfpathlineto{\pgfqpoint{5.213819in}{2.010001in}}%
\pgfpathlineto{\pgfqpoint{5.213819in}{2.007051in}}%
\pgfpathmoveto{\pgfqpoint{5.213819in}{2.007051in}}%
\pgfpathlineto{\pgfqpoint{5.213819in}{2.007051in}}%
\pgfpathlineto{\pgfqpoint{5.213819in}{2.010001in}}%
\pgfpathlineto{\pgfqpoint{5.218360in}{2.010001in}}%
\pgfpathlineto{\pgfqpoint{5.218360in}{2.007051in}}%
\pgfpathmoveto{\pgfqpoint{5.218360in}{2.007051in}}%
\pgfpathlineto{\pgfqpoint{5.218360in}{2.007051in}}%
\pgfpathlineto{\pgfqpoint{5.218360in}{2.010001in}}%
\pgfpathlineto{\pgfqpoint{5.222901in}{2.010001in}}%
\pgfpathlineto{\pgfqpoint{5.222901in}{2.007051in}}%
\pgfpathmoveto{\pgfqpoint{5.222901in}{2.007051in}}%
\pgfpathlineto{\pgfqpoint{5.222901in}{2.007051in}}%
\pgfpathlineto{\pgfqpoint{5.222901in}{2.010001in}}%
\pgfpathlineto{\pgfqpoint{5.227442in}{2.010001in}}%
\pgfpathlineto{\pgfqpoint{5.227442in}{2.007051in}}%
\pgfpathmoveto{\pgfqpoint{5.227442in}{2.007051in}}%
\pgfpathlineto{\pgfqpoint{5.227442in}{2.007051in}}%
\pgfpathlineto{\pgfqpoint{5.227442in}{2.010001in}}%
\pgfpathlineto{\pgfqpoint{5.231984in}{2.010001in}}%
\pgfpathlineto{\pgfqpoint{5.231984in}{2.007051in}}%
\pgfpathmoveto{\pgfqpoint{5.231984in}{2.007051in}}%
\pgfpathlineto{\pgfqpoint{5.231984in}{2.007051in}}%
\pgfpathlineto{\pgfqpoint{5.231984in}{2.010001in}}%
\pgfpathlineto{\pgfqpoint{5.236525in}{2.010001in}}%
\pgfpathlineto{\pgfqpoint{5.236525in}{2.007051in}}%
\pgfpathmoveto{\pgfqpoint{5.236525in}{2.007051in}}%
\pgfpathlineto{\pgfqpoint{5.236525in}{2.007051in}}%
\pgfpathlineto{\pgfqpoint{5.236525in}{2.010001in}}%
\pgfpathlineto{\pgfqpoint{5.241066in}{2.010001in}}%
\pgfpathlineto{\pgfqpoint{5.241066in}{2.007051in}}%
\pgfpathmoveto{\pgfqpoint{5.241066in}{2.007051in}}%
\pgfpathlineto{\pgfqpoint{5.241066in}{2.007051in}}%
\pgfpathlineto{\pgfqpoint{5.241066in}{2.010001in}}%
\pgfpathlineto{\pgfqpoint{5.245607in}{2.010001in}}%
\pgfpathlineto{\pgfqpoint{5.245607in}{2.007051in}}%
\pgfpathmoveto{\pgfqpoint{5.245607in}{2.007051in}}%
\pgfpathlineto{\pgfqpoint{5.245607in}{2.007051in}}%
\pgfpathlineto{\pgfqpoint{5.245607in}{2.010001in}}%
\pgfpathlineto{\pgfqpoint{5.250148in}{2.010001in}}%
\pgfpathlineto{\pgfqpoint{5.250148in}{2.007051in}}%
\pgfpathmoveto{\pgfqpoint{5.250148in}{2.007051in}}%
\pgfpathlineto{\pgfqpoint{5.250148in}{2.007051in}}%
\pgfpathlineto{\pgfqpoint{5.250148in}{2.010001in}}%
\pgfpathlineto{\pgfqpoint{5.254690in}{2.010001in}}%
\pgfpathlineto{\pgfqpoint{5.254690in}{2.007051in}}%
\pgfpathmoveto{\pgfqpoint{5.254690in}{2.007051in}}%
\pgfpathlineto{\pgfqpoint{5.254690in}{2.007051in}}%
\pgfpathlineto{\pgfqpoint{5.254690in}{2.010001in}}%
\pgfpathlineto{\pgfqpoint{5.259231in}{2.010001in}}%
\pgfpathlineto{\pgfqpoint{5.259231in}{2.007051in}}%
\pgfpathmoveto{\pgfqpoint{5.259231in}{2.007051in}}%
\pgfpathlineto{\pgfqpoint{5.259231in}{2.007051in}}%
\pgfpathlineto{\pgfqpoint{5.259231in}{2.010001in}}%
\pgfpathlineto{\pgfqpoint{5.263771in}{2.010001in}}%
\pgfpathlineto{\pgfqpoint{5.263771in}{2.007051in}}%
\pgfpathmoveto{\pgfqpoint{5.263771in}{2.007051in}}%
\pgfpathlineto{\pgfqpoint{5.263771in}{2.007051in}}%
\pgfpathlineto{\pgfqpoint{5.263771in}{2.010001in}}%
\pgfpathlineto{\pgfqpoint{5.268312in}{2.010001in}}%
\pgfpathlineto{\pgfqpoint{5.268312in}{2.007051in}}%
\pgfpathmoveto{\pgfqpoint{5.268312in}{2.007051in}}%
\pgfpathlineto{\pgfqpoint{5.268312in}{2.007051in}}%
\pgfpathlineto{\pgfqpoint{5.268312in}{2.010001in}}%
\pgfpathlineto{\pgfqpoint{5.272853in}{2.010001in}}%
\pgfpathlineto{\pgfqpoint{5.272853in}{2.007051in}}%
\pgfpathmoveto{\pgfqpoint{5.272853in}{2.007051in}}%
\pgfpathlineto{\pgfqpoint{5.272853in}{2.007051in}}%
\pgfpathlineto{\pgfqpoint{5.272853in}{2.010001in}}%
\pgfpathlineto{\pgfqpoint{5.277394in}{2.010001in}}%
\pgfpathlineto{\pgfqpoint{5.277394in}{2.007051in}}%
\pgfpathmoveto{\pgfqpoint{5.277394in}{2.007051in}}%
\pgfpathlineto{\pgfqpoint{5.277394in}{2.007051in}}%
\pgfpathlineto{\pgfqpoint{5.277394in}{2.010001in}}%
\pgfpathlineto{\pgfqpoint{5.281935in}{2.010001in}}%
\pgfpathlineto{\pgfqpoint{5.281935in}{2.007051in}}%
\pgfpathmoveto{\pgfqpoint{5.281935in}{2.007051in}}%
\pgfpathlineto{\pgfqpoint{5.281935in}{2.007051in}}%
\pgfpathlineto{\pgfqpoint{5.281935in}{2.010001in}}%
\pgfpathlineto{\pgfqpoint{5.286476in}{2.010001in}}%
\pgfpathlineto{\pgfqpoint{5.286476in}{2.007051in}}%
\pgfpathmoveto{\pgfqpoint{5.286476in}{2.007051in}}%
\pgfpathlineto{\pgfqpoint{5.286476in}{2.007051in}}%
\pgfpathlineto{\pgfqpoint{5.286476in}{2.010001in}}%
\pgfpathlineto{\pgfqpoint{5.291017in}{2.010001in}}%
\pgfpathlineto{\pgfqpoint{5.291017in}{2.007051in}}%
\pgfpathmoveto{\pgfqpoint{5.291017in}{2.007051in}}%
\pgfpathlineto{\pgfqpoint{5.291017in}{2.007051in}}%
\pgfpathlineto{\pgfqpoint{5.291017in}{2.010001in}}%
\pgfpathlineto{\pgfqpoint{5.295558in}{2.010001in}}%
\pgfpathlineto{\pgfqpoint{5.295558in}{2.007051in}}%
\pgfpathmoveto{\pgfqpoint{5.295558in}{2.007051in}}%
\pgfpathlineto{\pgfqpoint{5.295558in}{2.007051in}}%
\pgfpathlineto{\pgfqpoint{5.295558in}{2.010001in}}%
\pgfpathlineto{\pgfqpoint{5.300098in}{2.010001in}}%
\pgfpathlineto{\pgfqpoint{5.300098in}{2.007051in}}%
\pgfpathmoveto{\pgfqpoint{5.300098in}{2.007051in}}%
\pgfpathlineto{\pgfqpoint{5.300098in}{2.007051in}}%
\pgfpathlineto{\pgfqpoint{5.300098in}{2.010001in}}%
\pgfpathlineto{\pgfqpoint{5.304639in}{2.010001in}}%
\pgfpathlineto{\pgfqpoint{5.304639in}{2.007051in}}%
\pgfpathmoveto{\pgfqpoint{5.304639in}{2.007051in}}%
\pgfpathlineto{\pgfqpoint{5.304639in}{2.007051in}}%
\pgfpathlineto{\pgfqpoint{5.304639in}{2.010001in}}%
\pgfpathlineto{\pgfqpoint{5.309180in}{2.010001in}}%
\pgfpathlineto{\pgfqpoint{5.309180in}{2.007051in}}%
\pgfpathmoveto{\pgfqpoint{5.309180in}{2.007051in}}%
\pgfpathlineto{\pgfqpoint{5.309180in}{2.007051in}}%
\pgfpathlineto{\pgfqpoint{5.309180in}{2.010001in}}%
\pgfpathlineto{\pgfqpoint{5.313721in}{2.010001in}}%
\pgfpathlineto{\pgfqpoint{5.313721in}{2.007051in}}%
\pgfpathmoveto{\pgfqpoint{5.313721in}{2.007051in}}%
\pgfpathlineto{\pgfqpoint{5.313721in}{2.007051in}}%
\pgfpathlineto{\pgfqpoint{5.313721in}{2.010001in}}%
\pgfpathlineto{\pgfqpoint{5.318262in}{2.010001in}}%
\pgfpathlineto{\pgfqpoint{5.318262in}{2.007051in}}%
\pgfpathmoveto{\pgfqpoint{5.318262in}{2.007051in}}%
\pgfpathlineto{\pgfqpoint{5.318262in}{2.007051in}}%
\pgfpathlineto{\pgfqpoint{5.318262in}{2.010001in}}%
\pgfpathlineto{\pgfqpoint{5.322803in}{2.010001in}}%
\pgfpathlineto{\pgfqpoint{5.322803in}{2.007051in}}%
\pgfpathmoveto{\pgfqpoint{5.322803in}{2.007051in}}%
\pgfpathlineto{\pgfqpoint{5.322803in}{2.007051in}}%
\pgfpathlineto{\pgfqpoint{5.322803in}{2.010001in}}%
\pgfpathlineto{\pgfqpoint{5.327344in}{2.010001in}}%
\pgfpathlineto{\pgfqpoint{5.327344in}{2.007051in}}%
\pgfpathmoveto{\pgfqpoint{5.327344in}{2.007051in}}%
\pgfpathlineto{\pgfqpoint{5.327344in}{2.007051in}}%
\pgfpathlineto{\pgfqpoint{5.327344in}{2.010001in}}%
\pgfpathlineto{\pgfqpoint{5.331885in}{2.010001in}}%
\pgfpathlineto{\pgfqpoint{5.331885in}{2.007051in}}%
\pgfpathmoveto{\pgfqpoint{5.331885in}{2.007051in}}%
\pgfpathlineto{\pgfqpoint{5.331885in}{2.007051in}}%
\pgfpathlineto{\pgfqpoint{5.331885in}{2.010001in}}%
\pgfpathlineto{\pgfqpoint{5.336425in}{2.010001in}}%
\pgfpathlineto{\pgfqpoint{5.336425in}{2.007051in}}%
\pgfpathmoveto{\pgfqpoint{5.336425in}{2.007051in}}%
\pgfpathlineto{\pgfqpoint{5.336425in}{2.007051in}}%
\pgfpathlineto{\pgfqpoint{5.336425in}{2.010001in}}%
\pgfpathlineto{\pgfqpoint{5.340966in}{2.010001in}}%
\pgfpathlineto{\pgfqpoint{5.340966in}{2.007051in}}%
\pgfpathmoveto{\pgfqpoint{5.340966in}{2.007051in}}%
\pgfpathlineto{\pgfqpoint{5.340966in}{2.007051in}}%
\pgfpathlineto{\pgfqpoint{5.340966in}{2.010001in}}%
\pgfpathlineto{\pgfqpoint{5.345507in}{2.010001in}}%
\pgfpathlineto{\pgfqpoint{5.345507in}{2.007051in}}%
\pgfpathmoveto{\pgfqpoint{5.345507in}{2.007051in}}%
\pgfpathlineto{\pgfqpoint{5.345507in}{2.007051in}}%
\pgfpathlineto{\pgfqpoint{5.345507in}{2.010001in}}%
\pgfpathlineto{\pgfqpoint{5.350048in}{2.010001in}}%
\pgfpathlineto{\pgfqpoint{5.350048in}{2.007051in}}%
\pgfpathmoveto{\pgfqpoint{5.350048in}{2.007051in}}%
\pgfpathlineto{\pgfqpoint{5.350048in}{2.007051in}}%
\pgfpathlineto{\pgfqpoint{5.350048in}{2.010001in}}%
\pgfpathlineto{\pgfqpoint{5.354589in}{2.010001in}}%
\pgfpathlineto{\pgfqpoint{5.354589in}{2.007051in}}%
\pgfpathmoveto{\pgfqpoint{5.354589in}{2.007051in}}%
\pgfpathlineto{\pgfqpoint{5.354589in}{2.007051in}}%
\pgfpathlineto{\pgfqpoint{5.354589in}{2.010001in}}%
\pgfpathlineto{\pgfqpoint{5.359130in}{2.010001in}}%
\pgfpathlineto{\pgfqpoint{5.359130in}{2.007051in}}%
\pgfpathmoveto{\pgfqpoint{5.359130in}{2.007051in}}%
\pgfpathlineto{\pgfqpoint{5.359130in}{2.007051in}}%
\pgfpathlineto{\pgfqpoint{5.359130in}{2.010001in}}%
\pgfpathlineto{\pgfqpoint{5.363671in}{2.010001in}}%
\pgfpathlineto{\pgfqpoint{5.363671in}{2.007051in}}%
\pgfpathmoveto{\pgfqpoint{5.363671in}{2.007051in}}%
\pgfpathlineto{\pgfqpoint{5.363671in}{2.007051in}}%
\pgfpathlineto{\pgfqpoint{5.363671in}{2.010001in}}%
\pgfpathlineto{\pgfqpoint{5.368212in}{2.010001in}}%
\pgfpathlineto{\pgfqpoint{5.368212in}{2.007051in}}%
\pgfpathmoveto{\pgfqpoint{5.368212in}{2.007051in}}%
\pgfpathlineto{\pgfqpoint{5.368212in}{2.007051in}}%
\pgfpathlineto{\pgfqpoint{5.368212in}{2.010001in}}%
\pgfpathlineto{\pgfqpoint{5.372752in}{2.010001in}}%
\pgfpathlineto{\pgfqpoint{5.372752in}{2.007051in}}%
\pgfpathmoveto{\pgfqpoint{5.372752in}{2.007051in}}%
\pgfpathlineto{\pgfqpoint{5.372752in}{2.007051in}}%
\pgfpathlineto{\pgfqpoint{5.372752in}{2.010001in}}%
\pgfpathlineto{\pgfqpoint{5.377293in}{2.010001in}}%
\pgfpathlineto{\pgfqpoint{5.377293in}{2.007051in}}%
\pgfpathmoveto{\pgfqpoint{5.377293in}{2.007051in}}%
\pgfpathlineto{\pgfqpoint{5.377293in}{2.007051in}}%
\pgfpathlineto{\pgfqpoint{5.377293in}{2.010001in}}%
\pgfpathlineto{\pgfqpoint{5.381834in}{2.010001in}}%
\pgfpathlineto{\pgfqpoint{5.381834in}{2.007051in}}%
\pgfpathmoveto{\pgfqpoint{5.381834in}{2.007051in}}%
\pgfpathlineto{\pgfqpoint{5.381834in}{2.007051in}}%
\pgfpathlineto{\pgfqpoint{5.381834in}{2.010001in}}%
\pgfpathlineto{\pgfqpoint{5.386375in}{2.010001in}}%
\pgfpathlineto{\pgfqpoint{5.386375in}{2.007051in}}%
\pgfpathmoveto{\pgfqpoint{5.386375in}{2.007051in}}%
\pgfpathlineto{\pgfqpoint{5.386375in}{2.007051in}}%
\pgfpathlineto{\pgfqpoint{5.386375in}{2.010001in}}%
\pgfpathlineto{\pgfqpoint{5.390916in}{2.010001in}}%
\pgfpathlineto{\pgfqpoint{5.390916in}{2.007051in}}%
\pgfpathmoveto{\pgfqpoint{5.390916in}{2.007051in}}%
\pgfpathlineto{\pgfqpoint{5.390916in}{2.007051in}}%
\pgfpathlineto{\pgfqpoint{5.390916in}{2.010001in}}%
\pgfpathlineto{\pgfqpoint{5.395457in}{2.010001in}}%
\pgfpathlineto{\pgfqpoint{5.395457in}{2.007051in}}%
\pgfpathmoveto{\pgfqpoint{5.395457in}{2.007051in}}%
\pgfpathlineto{\pgfqpoint{5.395457in}{2.007051in}}%
\pgfpathlineto{\pgfqpoint{5.395457in}{2.010001in}}%
\pgfpathlineto{\pgfqpoint{5.399998in}{2.010001in}}%
\pgfpathlineto{\pgfqpoint{5.399998in}{2.007051in}}%
\pgfpathclose%
\pgfusepath{fill}%
\end{pgfscope}%
\begin{pgfscope}%
\pgfsetbuttcap%
\pgfsetroundjoin%
\definecolor{currentfill}{rgb}{0.000000,0.000000,0.000000}%
\pgfsetfillcolor{currentfill}%
\pgfsetlinewidth{0.803000pt}%
\definecolor{currentstroke}{rgb}{0.000000,0.000000,0.000000}%
\pgfsetstrokecolor{currentstroke}%
\pgfsetdash{}{0pt}%
\pgfsys@defobject{currentmarker}{\pgfqpoint{0.000000in}{-0.048611in}}{\pgfqpoint{0.000000in}{0.000000in}}{%
\pgfpathmoveto{\pgfqpoint{0.000000in}{0.000000in}}%
\pgfpathlineto{\pgfqpoint{0.000000in}{-0.048611in}}%
\pgfusepath{stroke,fill}%
}%
\begin{pgfscope}%
\pgfsys@transformshift{1.215000in}{2.010000in}%
\pgfsys@useobject{currentmarker}{}%
\end{pgfscope}%
\end{pgfscope}%
\begin{pgfscope}%
\definecolor{textcolor}{rgb}{0.000000,0.000000,0.000000}%
\pgfsetstrokecolor{textcolor}%
\pgfsetfillcolor{textcolor}%
\pgftext[x=1.215000in,y=1.912778in,,top]{\color{textcolor}\sffamily\fontsize{10.000000}{12.000000}\selectfont −4}%
\end{pgfscope}%
\begin{pgfscope}%
\pgfsetbuttcap%
\pgfsetroundjoin%
\definecolor{currentfill}{rgb}{0.000000,0.000000,0.000000}%
\pgfsetfillcolor{currentfill}%
\pgfsetlinewidth{0.803000pt}%
\definecolor{currentstroke}{rgb}{0.000000,0.000000,0.000000}%
\pgfsetstrokecolor{currentstroke}%
\pgfsetdash{}{0pt}%
\pgfsys@defobject{currentmarker}{\pgfqpoint{0.000000in}{-0.048611in}}{\pgfqpoint{0.000000in}{0.000000in}}{%
\pgfpathmoveto{\pgfqpoint{0.000000in}{0.000000in}}%
\pgfpathlineto{\pgfqpoint{0.000000in}{-0.048611in}}%
\pgfusepath{stroke,fill}%
}%
\begin{pgfscope}%
\pgfsys@transformshift{2.145000in}{2.010000in}%
\pgfsys@useobject{currentmarker}{}%
\end{pgfscope}%
\end{pgfscope}%
\begin{pgfscope}%
\definecolor{textcolor}{rgb}{0.000000,0.000000,0.000000}%
\pgfsetstrokecolor{textcolor}%
\pgfsetfillcolor{textcolor}%
\pgftext[x=2.145000in,y=1.912778in,,top]{\color{textcolor}\sffamily\fontsize{10.000000}{12.000000}\selectfont −2}%
\end{pgfscope}%
\begin{pgfscope}%
\pgfsetbuttcap%
\pgfsetroundjoin%
\definecolor{currentfill}{rgb}{0.000000,0.000000,0.000000}%
\pgfsetfillcolor{currentfill}%
\pgfsetlinewidth{0.803000pt}%
\definecolor{currentstroke}{rgb}{0.000000,0.000000,0.000000}%
\pgfsetstrokecolor{currentstroke}%
\pgfsetdash{}{0pt}%
\pgfsys@defobject{currentmarker}{\pgfqpoint{0.000000in}{-0.048611in}}{\pgfqpoint{0.000000in}{0.000000in}}{%
\pgfpathmoveto{\pgfqpoint{0.000000in}{0.000000in}}%
\pgfpathlineto{\pgfqpoint{0.000000in}{-0.048611in}}%
\pgfusepath{stroke,fill}%
}%
\begin{pgfscope}%
\pgfsys@transformshift{3.075000in}{2.010000in}%
\pgfsys@useobject{currentmarker}{}%
\end{pgfscope}%
\end{pgfscope}%
\begin{pgfscope}%
\definecolor{textcolor}{rgb}{0.000000,0.000000,0.000000}%
\pgfsetstrokecolor{textcolor}%
\pgfsetfillcolor{textcolor}%
\pgftext[x=3.075000in,y=1.912778in,,top]{\color{textcolor}\sffamily\fontsize{10.000000}{12.000000}\selectfont 0}%
\end{pgfscope}%
\begin{pgfscope}%
\pgfsetbuttcap%
\pgfsetroundjoin%
\definecolor{currentfill}{rgb}{0.000000,0.000000,0.000000}%
\pgfsetfillcolor{currentfill}%
\pgfsetlinewidth{0.803000pt}%
\definecolor{currentstroke}{rgb}{0.000000,0.000000,0.000000}%
\pgfsetstrokecolor{currentstroke}%
\pgfsetdash{}{0pt}%
\pgfsys@defobject{currentmarker}{\pgfqpoint{0.000000in}{-0.048611in}}{\pgfqpoint{0.000000in}{0.000000in}}{%
\pgfpathmoveto{\pgfqpoint{0.000000in}{0.000000in}}%
\pgfpathlineto{\pgfqpoint{0.000000in}{-0.048611in}}%
\pgfusepath{stroke,fill}%
}%
\begin{pgfscope}%
\pgfsys@transformshift{4.005000in}{2.010000in}%
\pgfsys@useobject{currentmarker}{}%
\end{pgfscope}%
\end{pgfscope}%
\begin{pgfscope}%
\definecolor{textcolor}{rgb}{0.000000,0.000000,0.000000}%
\pgfsetstrokecolor{textcolor}%
\pgfsetfillcolor{textcolor}%
\pgftext[x=4.005000in,y=1.912778in,,top]{\color{textcolor}\sffamily\fontsize{10.000000}{12.000000}\selectfont 2}%
\end{pgfscope}%
\begin{pgfscope}%
\pgfsetbuttcap%
\pgfsetroundjoin%
\definecolor{currentfill}{rgb}{0.000000,0.000000,0.000000}%
\pgfsetfillcolor{currentfill}%
\pgfsetlinewidth{0.803000pt}%
\definecolor{currentstroke}{rgb}{0.000000,0.000000,0.000000}%
\pgfsetstrokecolor{currentstroke}%
\pgfsetdash{}{0pt}%
\pgfsys@defobject{currentmarker}{\pgfqpoint{0.000000in}{-0.048611in}}{\pgfqpoint{0.000000in}{0.000000in}}{%
\pgfpathmoveto{\pgfqpoint{0.000000in}{0.000000in}}%
\pgfpathlineto{\pgfqpoint{0.000000in}{-0.048611in}}%
\pgfusepath{stroke,fill}%
}%
\begin{pgfscope}%
\pgfsys@transformshift{4.935000in}{2.010000in}%
\pgfsys@useobject{currentmarker}{}%
\end{pgfscope}%
\end{pgfscope}%
\begin{pgfscope}%
\definecolor{textcolor}{rgb}{0.000000,0.000000,0.000000}%
\pgfsetstrokecolor{textcolor}%
\pgfsetfillcolor{textcolor}%
\pgftext[x=4.935000in,y=1.912778in,,top]{\color{textcolor}\sffamily\fontsize{10.000000}{12.000000}\selectfont 4}%
\end{pgfscope}%
\begin{pgfscope}%
\definecolor{textcolor}{rgb}{0.000000,0.000000,0.000000}%
\pgfsetstrokecolor{textcolor}%
\pgfsetfillcolor{textcolor}%
\pgftext[x=5.400000in,y=1.722809in,,top]{\color{textcolor}\sffamily\fontsize{10.000000}{12.000000}\selectfont x}%
\end{pgfscope}%
\begin{pgfscope}%
\pgfsetbuttcap%
\pgfsetroundjoin%
\definecolor{currentfill}{rgb}{0.000000,0.000000,0.000000}%
\pgfsetfillcolor{currentfill}%
\pgfsetlinewidth{0.803000pt}%
\definecolor{currentstroke}{rgb}{0.000000,0.000000,0.000000}%
\pgfsetstrokecolor{currentstroke}%
\pgfsetdash{}{0pt}%
\pgfsys@defobject{currentmarker}{\pgfqpoint{-0.048611in}{0.000000in}}{\pgfqpoint{-0.000000in}{0.000000in}}{%
\pgfpathmoveto{\pgfqpoint{-0.000000in}{0.000000in}}%
\pgfpathlineto{\pgfqpoint{-0.048611in}{0.000000in}}%
\pgfusepath{stroke,fill}%
}%
\begin{pgfscope}%
\pgfsys@transformshift{3.075000in}{0.802000in}%
\pgfsys@useobject{currentmarker}{}%
\end{pgfscope}%
\end{pgfscope}%
\begin{pgfscope}%
\definecolor{textcolor}{rgb}{0.000000,0.000000,0.000000}%
\pgfsetstrokecolor{textcolor}%
\pgfsetfillcolor{textcolor}%
\pgftext[x=2.773039in, y=0.749238in, left, base]{\color{textcolor}\sffamily\fontsize{10.000000}{12.000000}\selectfont −4}%
\end{pgfscope}%
\begin{pgfscope}%
\pgfsetbuttcap%
\pgfsetroundjoin%
\definecolor{currentfill}{rgb}{0.000000,0.000000,0.000000}%
\pgfsetfillcolor{currentfill}%
\pgfsetlinewidth{0.803000pt}%
\definecolor{currentstroke}{rgb}{0.000000,0.000000,0.000000}%
\pgfsetstrokecolor{currentstroke}%
\pgfsetdash{}{0pt}%
\pgfsys@defobject{currentmarker}{\pgfqpoint{-0.048611in}{0.000000in}}{\pgfqpoint{-0.000000in}{0.000000in}}{%
\pgfpathmoveto{\pgfqpoint{-0.000000in}{0.000000in}}%
\pgfpathlineto{\pgfqpoint{-0.048611in}{0.000000in}}%
\pgfusepath{stroke,fill}%
}%
\begin{pgfscope}%
\pgfsys@transformshift{3.075000in}{1.406000in}%
\pgfsys@useobject{currentmarker}{}%
\end{pgfscope}%
\end{pgfscope}%
\begin{pgfscope}%
\definecolor{textcolor}{rgb}{0.000000,0.000000,0.000000}%
\pgfsetstrokecolor{textcolor}%
\pgfsetfillcolor{textcolor}%
\pgftext[x=2.773039in, y=1.353238in, left, base]{\color{textcolor}\sffamily\fontsize{10.000000}{12.000000}\selectfont −2}%
\end{pgfscope}%
\begin{pgfscope}%
\pgfsetbuttcap%
\pgfsetroundjoin%
\definecolor{currentfill}{rgb}{0.000000,0.000000,0.000000}%
\pgfsetfillcolor{currentfill}%
\pgfsetlinewidth{0.803000pt}%
\definecolor{currentstroke}{rgb}{0.000000,0.000000,0.000000}%
\pgfsetstrokecolor{currentstroke}%
\pgfsetdash{}{0pt}%
\pgfsys@defobject{currentmarker}{\pgfqpoint{-0.048611in}{0.000000in}}{\pgfqpoint{-0.000000in}{0.000000in}}{%
\pgfpathmoveto{\pgfqpoint{-0.000000in}{0.000000in}}%
\pgfpathlineto{\pgfqpoint{-0.048611in}{0.000000in}}%
\pgfusepath{stroke,fill}%
}%
\begin{pgfscope}%
\pgfsys@transformshift{3.075000in}{2.010000in}%
\pgfsys@useobject{currentmarker}{}%
\end{pgfscope}%
\end{pgfscope}%
\begin{pgfscope}%
\definecolor{textcolor}{rgb}{0.000000,0.000000,0.000000}%
\pgfsetstrokecolor{textcolor}%
\pgfsetfillcolor{textcolor}%
\pgftext[x=2.889413in, y=1.957238in, left, base]{\color{textcolor}\sffamily\fontsize{10.000000}{12.000000}\selectfont 0}%
\end{pgfscope}%
\begin{pgfscope}%
\pgfsetbuttcap%
\pgfsetroundjoin%
\definecolor{currentfill}{rgb}{0.000000,0.000000,0.000000}%
\pgfsetfillcolor{currentfill}%
\pgfsetlinewidth{0.803000pt}%
\definecolor{currentstroke}{rgb}{0.000000,0.000000,0.000000}%
\pgfsetstrokecolor{currentstroke}%
\pgfsetdash{}{0pt}%
\pgfsys@defobject{currentmarker}{\pgfqpoint{-0.048611in}{0.000000in}}{\pgfqpoint{-0.000000in}{0.000000in}}{%
\pgfpathmoveto{\pgfqpoint{-0.000000in}{0.000000in}}%
\pgfpathlineto{\pgfqpoint{-0.048611in}{0.000000in}}%
\pgfusepath{stroke,fill}%
}%
\begin{pgfscope}%
\pgfsys@transformshift{3.075000in}{2.614000in}%
\pgfsys@useobject{currentmarker}{}%
\end{pgfscope}%
\end{pgfscope}%
\begin{pgfscope}%
\definecolor{textcolor}{rgb}{0.000000,0.000000,0.000000}%
\pgfsetstrokecolor{textcolor}%
\pgfsetfillcolor{textcolor}%
\pgftext[x=2.889413in, y=2.561238in, left, base]{\color{textcolor}\sffamily\fontsize{10.000000}{12.000000}\selectfont 2}%
\end{pgfscope}%
\begin{pgfscope}%
\pgfsetbuttcap%
\pgfsetroundjoin%
\definecolor{currentfill}{rgb}{0.000000,0.000000,0.000000}%
\pgfsetfillcolor{currentfill}%
\pgfsetlinewidth{0.803000pt}%
\definecolor{currentstroke}{rgb}{0.000000,0.000000,0.000000}%
\pgfsetstrokecolor{currentstroke}%
\pgfsetdash{}{0pt}%
\pgfsys@defobject{currentmarker}{\pgfqpoint{-0.048611in}{0.000000in}}{\pgfqpoint{-0.000000in}{0.000000in}}{%
\pgfpathmoveto{\pgfqpoint{-0.000000in}{0.000000in}}%
\pgfpathlineto{\pgfqpoint{-0.048611in}{0.000000in}}%
\pgfusepath{stroke,fill}%
}%
\begin{pgfscope}%
\pgfsys@transformshift{3.075000in}{3.218000in}%
\pgfsys@useobject{currentmarker}{}%
\end{pgfscope}%
\end{pgfscope}%
\begin{pgfscope}%
\definecolor{textcolor}{rgb}{0.000000,0.000000,0.000000}%
\pgfsetstrokecolor{textcolor}%
\pgfsetfillcolor{textcolor}%
\pgftext[x=2.889413in, y=3.165238in, left, base]{\color{textcolor}\sffamily\fontsize{10.000000}{12.000000}\selectfont 4}%
\end{pgfscope}%
\begin{pgfscope}%
\definecolor{textcolor}{rgb}{0.000000,0.000000,0.000000}%
\pgfsetstrokecolor{textcolor}%
\pgfsetfillcolor{textcolor}%
\pgftext[x=2.717483in,y=3.520000in,,bottom,rotate=90.000000]{\color{textcolor}\sffamily\fontsize{10.000000}{12.000000}\selectfont y}%
\end{pgfscope}%
\begin{pgfscope}%
\pgfsetrectcap%
\pgfsetmiterjoin%
\pgfsetlinewidth{0.803000pt}%
\definecolor{currentstroke}{rgb}{0.000000,0.000000,0.000000}%
\pgfsetstrokecolor{currentstroke}%
\pgfsetdash{}{0pt}%
\pgfpathmoveto{\pgfqpoint{3.075000in}{0.500000in}}%
\pgfpathlineto{\pgfqpoint{3.075000in}{3.520000in}}%
\pgfusepath{stroke}%
\end{pgfscope}%
\begin{pgfscope}%
\pgfsetrectcap%
\pgfsetmiterjoin%
\pgfsetlinewidth{0.000000pt}%
\definecolor{currentstroke}{rgb}{0.000000,0.000000,0.000000}%
\pgfsetstrokecolor{currentstroke}%
\pgfsetstrokeopacity{0.000000}%
\pgfsetdash{}{0pt}%
\pgfpathmoveto{\pgfqpoint{5.400000in}{0.500000in}}%
\pgfpathlineto{\pgfqpoint{5.400000in}{3.520000in}}%
\pgfusepath{}%
\end{pgfscope}%
\begin{pgfscope}%
\pgfsetrectcap%
\pgfsetmiterjoin%
\pgfsetlinewidth{0.803000pt}%
\definecolor{currentstroke}{rgb}{0.000000,0.000000,0.000000}%
\pgfsetstrokecolor{currentstroke}%
\pgfsetdash{}{0pt}%
\pgfpathmoveto{\pgfqpoint{0.750000in}{2.010000in}}%
\pgfpathlineto{\pgfqpoint{5.400000in}{2.010000in}}%
\pgfusepath{stroke}%
\end{pgfscope}%
\begin{pgfscope}%
\pgfsetrectcap%
\pgfsetmiterjoin%
\pgfsetlinewidth{0.000000pt}%
\definecolor{currentstroke}{rgb}{0.000000,0.000000,0.000000}%
\pgfsetstrokecolor{currentstroke}%
\pgfsetstrokeopacity{0.000000}%
\pgfsetdash{}{0pt}%
\pgfpathmoveto{\pgfqpoint{0.750000in}{3.520000in}}%
\pgfpathlineto{\pgfqpoint{5.400000in}{3.520000in}}%
\pgfusepath{}%
\end{pgfscope}%
\end{pgfpicture}%
\makeatother%
\endgroup%
} \end{solution}


\end{parts}
%\end{multicols}

\question Resuelve los siguientes problemas
\begin{multicols}{2}
\begin{parts}
\part[] La  base  de  un  rectángulo  excede  en 3 cm a  la altura. Si  elevamos  al  cuadrado  su  diagonal  el  resultado  es menor que 29. Determina las posibles dimensiones del rectángulo. 
\begin{solution} $\left\{ {\begin{matrix}
   {x=y+3}  \\ 
   {x^2+y^2<29}  \\ 
 \end{matrix} } \right. \to x\in \left(-2, 5\right) \land y=x-3 \to y \in \left(0,2\right) \land x=y+3$\end{solution}
\part[]  Una fábrica A paga a sus viajantes 10 euros por artículo vendido más una cantidad fija de 400 euros. Otra fábrica B paga  15  euros  por  artículo  vendido  más  una  cantidad  fija  de  300  euros.  ¿Cuántos artículos  debe vender el viajante de fábrica B para ganar más dinero que el viajante de la fábrica A?
\begin{solution}$300+15x>400+10x\to\left(20, +\infty\right)\to$ más de 20  \end{solution}
\part[]  Se ha medido un campo rectangular con un error menor que un metro. Los valores de los lados, $x$ e $y$, cumplen $70 < x < 71$ y $47 < y < 48$. Determina entre qué valores está el perímetro y el área.
\begin{solution}%234<p<236 y 3290<a<3408 
$ 234<p<238$  y  $3290<a<3408$\end{solution}
\part[]  Un alumno ha realizado dos exámenes de Matemáticas obteniendo calificaciones respectivas de 4,5 y 5,7 puntos. ¿Cuánto ha de sacar como mínimo en el tercero para aprobar, si la nota final es la media aritmética de las tres notas? ¿Y si el primer examen cuenta un 15\%, el segundo un 35\% y el tercero un 50\%?
\begin{solution}%latex(reduce_rational_inequalities([[(4.5+5.7+x)/3>5]], x,relational=0)) y latex(reduce_rational_inequalities([[4.5*0.15+5.7*0.35+0.5*x>5]], x,relational=0))
$\frac{4,5+5,7+x}{3}>\to\left(4.8, \infty\right)$ y  $\left(4.66, \infty\right) $
\end{solution}
\part[]  Entre los triángulos isósceles de lado desigual igual a 20 cm., ¿cuáles tienen perímetro inferior a 200 cm.?
\begin{solution} $ 20+2x<200 \to x<90 \to$ lados iguales menores que 90\end{solution}
\part[]  Un padre tiene 33 años más que su hijo y el abuelo 33 años más que el padre. Hace tres años, sus edades sumaban menos de un siglo. ¿Qué edad puede tener cada uno?
\begin{solution} $x+\left(x+33\right)+\left(x+66\right)-3\cdot3<100 \to x<\frac{10}{3} \to$ Menos de 3 años y 4 meses el hijo\end{solution}

\end{parts}
\end{multicols}

\begin{comment}
\question 
\begin{multicols}{3}
\begin{parts}
\part[]  
\begin{solution} \end{solution}
\end{parts}
\end{multicols}
\end{comment}



\end{questions}

\end{document}


