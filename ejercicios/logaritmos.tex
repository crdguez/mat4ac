\documentclass[spanish, 12pt]{exam}

%These tell TeX which packages to use.
\usepackage{array,epsfig}
\usepackage{amsmath}
\usepackage{amsfonts}
\usepackage{amssymb}
\usepackage{amsxtra}
\usepackage{amsthm}
\usepackage{mathrsfs}
\usepackage{color}
\usepackage{multicol}

\usepackage[utf8]{inputenc}
\usepackage[spanish]{babel}
\usepackage{eurosym}

\usepackage{graphicx}
\graphicspath{{../img/}}

%\printanswers
\nopointsinmargin
\pointformat{}

%\spanishdecimal{.}

%Pagination stuff.
%\setlength{\topmargin}{-.3 in}
%\setlength{\oddsidemargin}{0in}
%\setlength{\evensidemargin}{0in}
%\setlength{\textheight}{9.in}
%\setlength{\textwidth}{6.5in}
%\pagestyle{empty}

\renewcommand{\solutiontitle}{\noindent\textbf{Sol:}\enspace}

\newcommand{\class}{4º Académicas}
\newcommand{\examdate}{\today}
\newcommand{\examnum}{Logaritmos}
\newcommand{\tipo}{A}


\newcommand{\timelimit}{50 minutos}



\pagestyle{head}
\firstpageheader{\includegraphics[width=0.2\columnwidth]{header_left}}{\textbf{Departamento de Matemáticas\linebreak \class}\linebreak \examnum}{\includegraphics[width=0.1\columnwidth]{header_right}}
\runningheader{\class}{\examnum}{Página \thepage\ of \numpages}
\runningheadrule

\begin{document}



\begin{questions}

\question Calcular el valor del número x:
\begin{multicols}{3}
\begin{parts}
\part[] ${x^{ - 1}} = 3$
\begin{solution} ${x^{ - 1}} = 3 \to \frac{1}{x} = 3 \to 1 = 3x \to x = \frac{1}{3}$ \end{solution}
\part[] ${3^{{\raise0.7ex\hbox{$1$} \!\mathord{\left/
 {\vphantom {1 4}}\right.\kern-\nulldelimiterspace}
\!\lower0.7ex\hbox{$4$}}}} = x$
\begin{solution} ${3^{{\raise0.7ex\hbox{$1$} \!\mathord{\left/
 {\vphantom {1 4}}\right.\kern-\nulldelimiterspace}
\!\lower0.7ex\hbox{$4$}}}} = x \to x = \sqrt[4]{3}$ \end{solution}
\part[] ${\left( {2x} \right)^{ - 2}} = 4$
\begin{solution} ${\left( {2x} \right)^{ - 2}} = 4 \to \frac{1}{{{{\left( {2x} \right)}^2}}} = 4 \to \frac{1}{{4{x^2}}} = 4 \to 16{x^2} = 1 \to {x^2} = \frac{1}{{16}} \to x =  \pm \frac{1}{4} \to x = \frac{1}{4}$ \end{solution}
\part[] ${\left( {\frac{1}{4}} \right)^{\frac{{ - 2}}{3}}} = x$

\begin{solution} $ {\left( {\frac{1}{4}} \right)^{\frac{{ - 2}}{3}}} = x \to {4^{{\raise0.7ex\hbox{$2$} \!\mathord{\left/
 {\vphantom {2 3}}\right.\kern-\nulldelimiterspace}
\!\lower0.7ex\hbox{$3$}}}} = x \to x = \sqrt[3]{{{2^4}}} \to x = 2\sqrt[3]{2}  $ \end{solution}
\part[] $ {\left( 0,2 \right)^{{\raise0.7ex\hbox{${ - 1}$} \!\mathord{\left/
 {\vphantom {{ - 1} 2}}\right.\kern-\nulldelimiterspace}
\!\lower0.7ex\hbox{$2$}}}} = x$
\begin{solution} ${\left( 0.2 \right)^{{\raise0.7ex\hbox{${ - 1}$} \!\mathord{\left/
 {\vphantom {{ - 1} 2}}\right.\kern-\nulldelimiterspace}
\!\lower0.7ex\hbox{$2$}}}} = x \to {\left( {\frac{2}{{10}}} \right)^{{\raise0.7ex\hbox{${ - 1}$} \!\mathord{\left/
 {\vphantom {{ - 1} 2}}\right.\kern-\nulldelimiterspace}
\!\lower0.7ex\hbox{$2$}}}} = x \to {\left( {\frac{1}{5}} \right)^{{\raise0.7ex\hbox{${ - 1}$} \!\mathord{\left/
 {\vphantom {{ - 1} 2}}\right.\kern-\nulldelimiterspace}
\!\lower0.7ex\hbox{$2$}}}} = x \to {5^{{\raise0.7ex\hbox{$1$} \!\mathord{\left/
 {\vphantom {1 2}}\right.\kern-\nulldelimiterspace}
\!\lower0.7ex\hbox{$2$}}}} = x \to x = \sqrt 5 $ \end{solution}
\part[] ${x^{ - 1}} = 3$
\begin{solution} ${x^{ - 1}} = 3 \to \frac{1}{x} = 3 \to 1 = 3x \to x = \frac{1}{3}$ \end{solution}
\end{parts}
\end{multicols}


\question Calcular los siguientes logaritmos aplicando la definición:
\begin{multicols}{3}
\begin{parts}
\part[] ${\log_{\sqrt 2 }}\frac{1}{{64}}$ 
\begin{solution} ${\log_{\sqrt 2 }}\frac{1}{{64}} = x \to {\left( {\sqrt 2 } \right)^x} = {2^{ - 6}} \to {2^{{\raise0.7ex\hbox{$x$} \!\mathord{\left/
 {\vphantom {x 2}}\right.\kern-\nulldelimiterspace}
\!\lower0.7ex\hbox{$2$}}}} = {2^{ - 6}} \to \frac{x}{2} =  - 6 \to x =  - 12$ \end{solution}
\part[] ${\log _2} 2\sqrt{2}$ \begin{solution} ${\log _2}2\sqrt 2  = x \to {2^x} = {2\cdot2^{{\raise0.7ex\hbox{$1$} \!\mathord{\left/
 {\vphantom {1 2}}\right.\kern-\nulldelimiterspace}
\!\lower0.7ex\hbox{$2$}}}} \to {2^x} = {2^{{\raise0.7ex\hbox{$3$} \!\mathord{\left/
 {\vphantom {3 2}}\right.\kern-\nulldelimiterspace}
\!\lower0.7ex\hbox{$2$}}}} \to x = \frac{3}{2}$ \end{solution}
\part[] $\left( {0,125} \right)^x = 16$ 
\begin{solution} ${\left( {0,125} \right)^x} = 16 \to {\left( {\frac{{125}}{{1000}}} \right)^x} = {2^4} \to {\left( {\frac{{{5^3}}}{{{2^3}{{\cdot5}^3}}}} \right)^x} = {2^4} \to {\left( {{2^{ - 3}}} \right)^x} = {2^4} \to {2^{ - 3x}} = {2^4} \to  - 3x = 4 \to x =  - \frac{4}{3}$ \end{solution}
\part[] ${\left( {\frac{1}{{\sqrt 3 }}} \right)^x} = {3^3}$ 
\begin{solution} ${\left( {\frac{1}{{\sqrt 3 }}} \right)^x} = {3^3} \to {\left( {{3^{ - {\raise0.7ex\hbox{$1$} \!\mathord{\left/
 {\vphantom {1 2}}\right.\kern-\nulldelimiterspace}
\!\lower0.7ex\hbox{$2$}}}}} \right)^x} = {3^3} \to {3^{ - {\raise0.7ex\hbox{$x$} \!\mathord{\left/
 {\vphantom {x 2}}\right.\kern-\nulldelimiterspace}
\!\lower0.7ex\hbox{$2$}}}} = {3^3} \to  - \frac{x}{2} = 3 \to x =  - 6$ \end{solution}
\part[] ${\left( 0,2 \right)^x} = {5^4}$ 
\begin{solution} ${\left( {0,2} \right)^x} = {5^4} \to {\left( {\frac{2}{{10}}} \right)^x} = {5^4} \to {\left( {\frac{1}{5}} \right)^x} = {5^4} \to {5^{ - x}} = {5^4} \to  - x = 4 \to x =  - 4$ \end{solution}
\end{parts}
\end{multicols}




\question Calcular:
\begin{multicols}{3}
\begin{parts}
\part[] ${\log _2}{2^3 \cdot 2^5}$ \begin{solution} $ {\log _2}{2^3\cdot2^5} = {\log _2}{2^8} = 8 \cdot \log _2 2 = 8$ \end{solution}
\part[] ${\log _5}{5^2\cdot5^3} $ \begin{solution} ${\log _5}{5^2\cdot5^3} = \log _5 {5^5} = 5\cdot \log _5 5 = 5$ \end{solution}
\part[] ${\log _{{\raise0.7ex\hbox{$1$} \!\mathord{\left/
 {\vphantom {1 5}}\right.\kern-\nulldelimiterspace}
\!\lower0.7ex\hbox{$5$}}}}{5^4}$ \begin{solution} ${\log _{{\raise0.7ex\hbox{$1$} \!\mathord{\left/
 {\vphantom {1 5}}\right.\kern-\nulldelimiterspace}
\!\lower0.7ex\hbox{$5$}}}}{5^4} = 4{\log _{{\raise0.7ex\hbox{$1$} \!\mathord{\left/
 {\vphantom {1 5}}\right.\kern-\nulldelimiterspace}
\!\lower0.7ex\hbox{$5$}}}}5 = 4\frac{{{{\log }_5}5}}{{{{\log }_5}\frac{1}{5}}} = 4\frac{1}{{{{\log }_5}1 - {{\log }_5}5}} = \frac{4}{{0 - 1}} =  - 4$ \end{solution}

\end{parts}
\end{multicols}


\question Sabiendo que $\log 2$ = 0,30103 y que $\log 3$ = 0,47712 calcular los siguientes logaritmos:
\begin{multicols}{4}
\begin{parts}
\part[] ${\log _2}3$ \begin{solution} ${\log _2}3 = \frac{{\log 3}}{{\log 2}} = \frac{{0,47712}}{{0.30103}} = 1,58$ \end{solution}
\part[] $\log 5$ \begin{solution} $\log 5 = \log \frac{{10}}{2} = \log 10 - \log 2 = 1 - \log 2 = 1 - 0.30103 = 0,69897$ \end{solution}
\part[] $\log 0.6$ \begin{solution} $\log 0.6 = \log \frac{6}{{10}} = \log 6 - \log 10 = \log \left( {2\cdot 3} \right) - 1 = \log 2 + \log 3 - 1 = 0.30103 + 0.47712 - 1 =  - 0,22$ \end{solution}
\part[] $\log 0.0125$ \begin{solution} $\log 0.0125 = \log \frac{{125}}{{10000}} = \log 125 - \log 10000 = \log {5^3} - \log {10^4} = 3\log 5 - 4\log 10 =  \hfill \\
  3\log \frac{{10}}{2} - 4 = 3\left( {\log 10 - \log 2} \right) - 4 = 3 - \log 2 - 4 =  - 1 - \log 2 =  - 1 - 0.30103 =  - 1,30103$ \end{solution}
\end{parts}
\end{multicols}

	
\question Sin utilizar la calculadora, resuelve los siguientes logaritmos:
\begin{multicols}{3}
\begin{parts}
\part[] ${\log _3}27$ \begin{solution} ${\log _3}27 = {\log _3}{3^3} = 3$ \end{solution}
\part[] ${\log _3}\frac{1}{{81}}$ \begin{solution} ${\log _3}\frac{1}{{81}} = {\log _3}{3^{ - 4}} =  - 4$ \end{solution}
\part[] ${\log _{{\raise0.7ex\hbox{$1$} \!\mathord{\left/
 {\vphantom {1 3}}\right.\kern-\nulldelimiterspace}
\!\lower0.7ex\hbox{$3$}}}}27$ \begin{solution} ${\log _{{\raise0.7ex\hbox{$1$} \!\mathord{\left/
 {\vphantom {1 3}}\right.\kern-\nulldelimiterspace}
\!\lower0.7ex\hbox{$3$}}}}27 = {\log _{{\raise0.7ex\hbox{$1$} \!\mathord{\left/ {\vphantom {1 3}}\right.\kern-\nulldelimiterspace}
\!\lower0.7ex\hbox{$3$}}}}{3^3} = 3\frac{{{{\log }_3}3}}{{{{\log }_3}\frac{1}{3}}} = 3\frac{1}{{0 - 1}} =  - 3$ \end{solution}
\part[] ${\log _{{\raise0.7ex\hbox{$1$} \!\mathord{\left/
 {\vphantom {1 3}}\right.\kern-\nulldelimiterspace}
\!\lower0.7ex\hbox{$3$}}}}\frac{1}{{81}}$ \begin{solution} ${\log _{{\raise0.7ex\hbox{$1$} \!\mathord{\left/
 {\vphantom {1 3}}\right.\kern-\nulldelimiterspace}
\!\lower0.7ex\hbox{$3$}}}}\frac{1}{{81}} = {\log _{{\raise0.7ex\hbox{$1$} \!\mathord{\left/
 {\vphantom {1 3}}\right.\kern-\nulldelimiterspace}
\!\lower0.7ex\hbox{$3$}}}}{\left( {\frac{1}{3}} \right)^4} = 4$\end{solution}
\part[] ${\log _5}\sqrt {125} $ \begin{solution} ${\log _5}\sqrt {125}  = {\log _5}{5^{{\raise0.7ex\hbox{$3$} \!\mathord{\left/
 {\vphantom {3 2}}\right.\kern-\nulldelimiterspace}
\!\lower0.7ex\hbox{$2$}}}} = \frac{3}{2}$ \end{solution}
\part[] ${\log _{{\raise0.7ex\hbox{$1$} \!\mathord{\left/
 {\vphantom {1 5}}\right.\kern-\nulldelimiterspace}
\!\lower0.7ex\hbox{$5$}}}}625 $ \begin{solution} ${\log _{{\raise0.7ex\hbox{$1$} \!\mathord{\left/
 {\vphantom {1 5}}\right.\kern-\nulldelimiterspace}
\!\lower0.7ex\hbox{$5$}}}}625 = {\log _{{\raise0.7ex\hbox{$1$} \!\mathord{\left/
 {\vphantom {1 5}}\right.\kern-\nulldelimiterspace}
\!\lower0.7ex\hbox{$5$}}}}{5^4} = 4\frac{{{{\log }_5}5}}{{{{\log }_5}\frac{1}{5}}} = 4\frac{1}{{0 - 1}} =  - 4$ \end{solution}
\end{parts}
\end{multicols}


\question Escribe las siguientes expresiones como el log de una sola expresión, lo más simple posible:
\begin{parts}
\part[] $\log a + 2\log b - \frac{3}{2}\log c + \frac{5}{2}\log d $ \begin{solution} $\log a + 2\log b - \frac{3}{2}\log c + \frac{5}{2}\log d = \log {a^3} + \log {b^2} - \log {c^{{\raise0.7ex\hbox{$3$} \!\mathord{\left/
 {\vphantom {3 2}}\right.\kern-\nulldelimiterspace}
\!\lower0.7ex\hbox{$2$}}}} + \log {d^{\frac{5}{2}}} = \log \frac{{{a^3}{b^2}}}{{{c^{{\raise0.7ex\hbox{$3$} \!\mathord{\left/
 {\vphantom {3 2}}\right.\kern-\nulldelimiterspace}
\!\lower0.7ex\hbox{$2$}}}}}} + \log {d^{\frac{5}{2}}} =  \hfill \\
   = \log (\frac{{{a^3}{b^2}}}{{c\sqrt c }} \cdot \sqrt[2]{{{d^5}}}) = \log \frac{{{a^2}{b^2}{d^2}\sqrt d }}{{c\sqrt c }}$ \end{solution}
\part[] $\frac{1}{2}\log \left( {{x^2} + 4} \right) + \frac{1}{2}\log \left( {x + 3} \right) + \frac{1}{2}\log \left( {x - 3} \right)$ \begin{solution} $\frac{1}{2}\log \left( {{x^2} + 4} \right) + \frac{1}{2}\log \left( {x + 3} \right) + \frac{1}{2}\log \left( {x - 3} \right) = \log \sqrt {{x^2} + 4}  + \log \sqrt {x + 3}  + \log \sqrt {x - 3}  =  \hfill \\
   = \log \sqrt {\left( {{x^2} + 4} \right)\left( {x + 3} \right)\left( {x - 3} \right)}  = \log \sqrt {\left( {{x^2} + 4} \right)\left( {{x^2} - 9} \right)}  = \log \sqrt {{x^4} - 5{x^2} - 36}$ \end{solution}
\end{parts}


\question Resolver las siguientes ecuaciones logarítmicas:
\begin{multicols}{2}
\begin{parts}
\part[] $\log x = \log 2 + \log \left( {x - 3} \right)$ \begin{solution} $\log x = \log 2 + \log \left( {x - 3} \right) \to x = 2\left( {x - 3} \right) \to x = 6$ \end{solution}
\part[] $log \left( {3x + 1} \right) - \log \left( {2x - 3} \right) = 1 - \log 5$ \begin{solution} $log \left( {3x + 1} \right) - \log \left( {2x - 3} \right) = 1 - \log 5 \to \frac{{3x + 1}}{{2x - 3}} = \frac{{10}}{5} \to 3x + 1 = 4x - 6 \to x = 7$ \end{solution}
\part[] $\log \left( {20x} \right) + \log \left( {2x} \right) = 3$ \begin{solution} $\log \left( {20x} \right) + \log \left( {2x} \right) = 3 \to 20x.2x = 1000 \to 40x = 1000 \to x = 25$ \end{solution}
\part[] $\log \left( {x + 2} \right) + \log \left( {10x + 20} \right) = 3$ \begin{solution} $\log \left( {x + 2} \right) + \log \left( {10x + 20} \right) = 3 \to \left( {x + 2} \right)\left( {10x + 20} \right) = 1000 \to \left( {x + 2} \right)\left( {x + 2} \right)10 = 1000 \hfill \\
  {x^2} + 4x + 4 = 100 \to {x^2} + 4x - 96 = 0$ \end{solution}
\end{parts}
\end{multicols}



\end{questions}
\end{document}


