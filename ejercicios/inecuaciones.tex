\documentclass[spanish, 12pt]{exam}

%These tell TeX which packages to use.
\usepackage{array,epsfig}
\usepackage{amsmath}
\usepackage{amsfonts}
\usepackage{amssymb}
\usepackage{amsxtra}
\usepackage{amsthm}
\usepackage{mathrsfs}
\usepackage{color}
\usepackage{multicol}
\usepackage{verbatim}

\usepackage[utf8]{inputenc}
\usepackage[spanish]{babel}
\usepackage{eurosym}

\usepackage{graphicx}
\graphicspath{{../img/}}

%\printanswers
\nopointsinmargin
\pointformat{}

%Pagination stuff.
%\setlength{\topmargin}{-.3 in}
%\setlength{\oddsidemargin}{0in}
%\setlength{\evensidemargin}{0in}
%\setlength{\textheight}{9.in}
%\setlength{\textwidth}{6.5in}
%\pagestyle{empty}

\renewcommand{\solutiontitle}{\noindent\textbf{Sol:}\enspace}

\newcommand{\class}{4º Académicas}
\newcommand{\examdate}{\today}
\newcommand{\examnum}{Inecuaciones}
\newcommand{\tipo}{A}


\newcommand{\timelimit}{50 minutos}



\pagestyle{head}
\firstpageheader{\includegraphics[width=0.2\columnwidth]{header_left}}{\textbf{Departamento de Matemáticas\linebreak \class}\linebreak \examnum}{\includegraphics[width=0.1\columnwidth]{header_right}}
\runningheader{\class}{\examnum}{Página \thepage\ of \numpages}
\runningheadrule

\begin{document}



\begin{questions}

\question Demuestra que son equivalentes las siguientes inecuaciones:
\begin{multicols}{3}
\begin{parts}
\part[] $\dfrac{x}{3} - 3x + 4 > x - 2$   
\part[]  $7x < 12$ 
\begin{solution} \end{solution}
\end{parts}
\end{multicols}

\question ¿Son equivalentes las siguientes inecuaciones?
\begin{multicols}{3}
\begin{parts}
\part[]  $x - 2 \geqslant \frac{x}{3}$ 
\part[]  $3x - \frac{x}{6} + 2 \geqslant 3 + \frac{{5x}}{2}$
\begin{solution} \end{solution}
\end{parts}
\end{multicols}

\question Resuelve las siguientes inecuaciones lineales con una incógnita:
\begin{multicols}{3}
\begin{parts}
\part[]  $5x + 6 - \frac{x}{2} >  - 1$
\begin{solution} \end{solution}
\part[]  $3 \cdot (x - 3) > 1 - 5x$
\begin{solution} \end{solution}
\part[]  $2 - \frac{x}{3} + 2x > 1 - x$
\begin{solution} \end{solution}
\part[]  $8x - 6 \geqslant \frac{{2x}}{5}$
\begin{solution} \end{solution}
\part[]  $\frac{{2x}}{3} - \frac{x}{2} + 5 < x - \frac{1}{3}$
\begin{solution} \end{solution}
\part[]  $\frac{{4 - 3x}}{2} < x + 3$
\begin{solution} \end{solution}
\part[]  $\frac{{x + 1}}{3} - \frac{{x - 2}}{5} > 1 + \frac{{x - 1}}{{15}}$
\begin{solution} \end{solution}
\part[]  $\frac{{x - 2}}{5} - \frac{{3x + 1}}{2} < \frac{x}{2} - 3x$
\begin{solution} \end{solution}
\part[]  $3x - \frac{{1 - 2x}}{4} < \frac{{x - 1}}{2} + 1$
\begin{solution} \end{solution}
\end{parts}
\end{multicols}

\question Resuelve las siguientes inecuaciones de segundo grado:
\begin{multicols}{3}
\begin{parts}
\part[] $2{x^2} - 12x + 16 \leqslant 0$ 
\begin{solution} \end{solution}
\part[]  $ - {x^2} + 4 \leqslant 0$
\begin{solution} \end{solution}
\part[] ${x^2} + 6x + 9 \leqslant 0$  
\begin{solution} \end{solution}
\part[] ${x^2} - x - 6 > 0$
\begin{solution} \end{solution}
\part[]  ${x^2} - 6x + 8 > 0$
\begin{solution} \end{solution}
\part[]  ${x^2} + 6x + 24 > 0$
\begin{solution} \end{solution}
\part[]  ${x^2} - 3x > 0$
\begin{solution} \end{solution}
\part[]  $2{x^2} - 10x - 12 \leqslant 0$
\begin{solution} \end{solution}
\part[]  $ - 3{x^2} + 6x + 9 \leqslant 0$
\begin{solution} \end{solution}
\part[]   $x \cdot \left( {x + 1} \right) + 3x > 5x + 6$
\begin{solution} \end{solution}
\end{parts}
\end{multicols}

\question Resuelve los siguientes sistemas de inecuaciones lineales con una incógnita:
\begin{multicols}{3}
\begin{parts}
\part[]  $\left\{ \begin{gathered}
  2x + 6 < 0 \hfill \\
  x + \frac{1}{3} \geqslant \frac{x}{2} \hfill \\ 
\end{gathered}  \right.$

\begin{solution} \end{solution}
\part[]  $\left\{ \begin{gathered}
  x + 4 < 0 \hfill \\
  x + 1 < \frac{x}{2} \hfill \\ 
\end{gathered}  \right.$

\begin{solution} \end{solution}
\part[]  $\left\{ \begin{gathered}
  x \leqslant 0 \hfill \\
  \frac{x}{2} + \frac{x}{3} - 6 \leqslant 5 \hfill \\ 
\end{gathered}  \right.$

\begin{solution} \end{solution}
\part[]  $\left\{ \begin{gathered}
  x - 2 \leqslant 2x + 1 \hfill \\
  3 - x < 1 - 2x \hfill \\ 
\end{gathered}  \right.$

\begin{solution} \end{solution}
\part[]  $\left\{ \begin{gathered}
  \frac{x}{3} + x < 4 \hfill \\
  \frac{x}{2} - \frac{x}{3} > 0 \hfill \\ 
\end{gathered}  \right.\begin{gathered}
   \hfill \\
   \hfill \\ 
\end{gathered} $ 
\begin{solution} \end{solution}
\part[]  $\left\{ \begin{gathered}
  \frac{{x - 4}}{2} + \frac{{x + 2}}{3} \leqslant 2 \hfill \\
  \frac{x}{3} - \frac{x}{2} \leqslant 1 \hfill \\ 
\end{gathered}  \right.$  
\begin{solution} \end{solution}
\part[] $\left\{ \begin{gathered}
  \frac{x}{3} - \frac{x}{2} \geqslant 1 \hfill \\
  {\left( {x + 1} \right)^2} - {x^2} \leqslant 1 \hfill \\ 
\end{gathered}  \right.$
 
\begin{solution} \end{solution}
\part[]  $\left\{ \begin{gathered}
  5x + 8 \leqslant 14 + 3x \hfill \\
  14x - 34 > 12x - 36 \hfill \\ 
\end{gathered}  \right.$

\begin{solution} \end{solution}


\end{parts}
\end{multicols}

\question Resuelve los siguientes inecuaciones de expresión racional:
\begin{multicols}{3}
\begin{parts}
\part[]  $\dfrac{{\left( {x + 3} \right)}}{{x - 4}} \geqslant 0$ 
\begin{solution} \end{solution}
\part[]  $\dfrac{{2x + 6}}{{x - 5}} \geqslant 0$
\begin{solution} \end{solution}
\part[]  $\dfrac{{x - 8}}{{3x - 6}} < 0$

\begin{solution} \end{solution}
\part[]  $\dfrac{{x - 2}}{{x + 2}} < 0$
\begin{solution} \end{solution}
\part[]  $\dfrac{{x + 3}}{{4x}} < 0$  
\begin{solution} \end{solution}
\part[]  $\dfrac{{\left( {x - 1} \right) \cdot \left( {x - 2} \right)}}{{3{x^2} + 4}} \geqslant 0$
\begin{solution} \end{solution}

\end{parts}
\end{multicols}

\begin{comment}
\question 
\begin{multicols}{3}
\begin{parts}
\part[]  
\begin{solution} \end{solution}

\end{parts}
\end{multicols}
\end{comment}



\end{questions}

\end{document}


