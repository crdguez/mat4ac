\documentclass[spanish, 11pt]{exam}

%These tell TeX which packages to use.
\usepackage{array,epsfig}
\usepackage{amsmath}
\usepackage{amsfonts}
\usepackage{amssymb}
\usepackage{amsxtra}
\usepackage{amsthm}
\usepackage{mathrsfs}
\usepackage{color}
\usepackage{multicol}
\usepackage{verbatim}

\usepackage[utf8]{inputenc}
\usepackage[spanish]{babel}
\usepackage{eurosym}

\usepackage{graphicx}
\graphicspath{{../img/}}


\printanswers
\nopointsinmargin
\pointformat{}

%Pagination stuff.
%\setlength{\topmargin}{-.3 in}
%\setlength{\oddsidemargin}{0in}
%\setlength{\evensidemargin}{0in}
%\setlength{\textheight}{9.in}
%\setlength{\textwidth}{6.5in}
%\pagestyle{empty}

\renewcommand{\solutiontitle}{\noindent\textbf{Sol:}\enspace}

\newcommand{\class}{4º Académicas}
\newcommand{\examdate}{\today}
\newcommand{\examnum}{Trigonometría}
\newcommand{\tipo}{A}


\newcommand{\timelimit}{50 minutos}



\pagestyle{head}
\firstpageheader{\includegraphics[width=0.2\columnwidth]{header_left}}{\textbf{Departamento de Matemáticas\linebreak \class}\linebreak \examnum}{\includegraphics[width=0.1\columnwidth]{header_right}}
\runningheader{\class}{\examnum}{Página \thepage\ of \numpages}
\runningheadrule

\begin{document}



\begin{questions}

\question Teorema del cateto y altura:
\begin{multicols}{2}
\begin{parts}
\part[]Calcula la hipotenusa de un triángulo rectángulo, sabiendo que sus catetos miden 156 cm y 65 cm.
\begin{solution} $=\sqrt{156^2-65^2}=169$ \end{solution}
\part[]Halla las longitudes de las proyecciones sobre la hipotenusa de los catetos del triángulo del ejercicio anterior.  
\begin{solution}
%solve(156**2-169*x) 
144 y 25 \end{solution}

\part[]En un triángulo rectángulo, las proyecciones de los catetos sobre la hipotenusa miden 64 m y 225 m
respectivamente. Halla la longitud de los tres lados del triángulo.
\begin{solution} $h=64+225=289 \\ c_1^2=289\cdot64 \to c_1=12 \\
c_2^2=289\cdot 225 \to c_2=255$\end{solution}

\part[]Halla la altura de un trapecio isósceles, sabiendo que sus bases miden 6 m y 16 m y los lados oblicuos 13 m
cada uno de ellos.
\begin{solution} $\to \sqrt{13^2-5^2}=12$ \end{solution}

\part[] En un triángulo rectángulo se conoce un cateto, ($7\sqrt 2 $), y la proyección del otro cateto sobre la
hipotenusa, ($2\sqrt 2 $). Halla la hipotenusa y el otro cateto.
\begin{solution} %solve([(7*sqrt(2))**2-(x+2*sqrt(2))*x, x>0], x) 
$hipotenusa \to 10-\sqrt{2}+2\sqrt{2}=10+\sqrt{2}\\
cateto \to \sqrt{-98 + \left(\sqrt{2} + 10\right)^{2}}= 2 \sqrt{1 + 5 \sqrt{2}}$ \end{solution}

\end{parts}
\end{multicols}






\question \textbf{Razones trigonométricas en un triángulo rectángulo}. En los siguientes ejercicios los lados de un triángulo
rectángulo se representan con las letras a, b y c, siendo siempre a la hipotenusa. Los lados del triángulo se
representan con las letras A, B y C, siendo siempre A el ángulo recto, B el ángulo opuesto a b y C el ángulo
opuesto a c. Usando exclusivamente la definición de las razones trigonométricas involucradas en cada caso,
calcula el lado que se pide:
\begin{multicols}{2}
\begin{parts}
\part[] a = 40 m B = 30º. Hallar b.  
 
\begin{solution} $b=40\cdot \sen30$\end{solution}

\part[]  $\left\{ {\begin{matrix}
   {2x + y \leqslant 4}  \\ 
   {x \geqslant 0}  \\ 
   {y \geqslant 1}  \\ 

 \end{matrix} } \right.$

\end{parts}
\end{multicols}

\question Resuelve los siguientes problemas:
\begin{multicols}{2}
\begin{parts}
\part[] Se tienen dos cuadrados distintos. La suma de dos lados, uno de cada cuadrado, es de 62 centímetros, y la suma de sus áreas, de 1954 centímetros cuadrados. ¿Cuáles son sus medidas?
\begin{solution} $\left\{ {\begin{matrix}
   {x+y=62}  \\ 
   {x^2+y^2=1954}  \\ 
 \end{matrix} } \right. \to \\ s\left [ \left ( 27, \quad 35\right ), \quad \left ( 35, \quad 27\right )\right ]$ \end{solution}
\part[] En una clase hay 5 chicos más que chicas. Sabemos que en total son algo más de 20 alumnos, pero no llegan a 25. ¿Cuál puede ser la composición de la clase?
\begin{solution} $\left\{ {\begin{matrix}
   {y=x+5}  \\ 
   {20<x+y<25}  \\ 
 \end{matrix} } \right. \to$ 8 chicas y 13 chicos o 9 chicas y 14 chicos\end{solution}
\part[]  ¿Cuántos litros de vino de 5\euro/l se deben mezclar con 20 l de otro de 3,50\euro/l para que el
precio de la mezcla sea inferior a 4\euro /l ?
\begin{solution}$5x + 70 < 4\cdot(20 + x)\to\left(20, +\infty\right)\to x<10 \to$ Se deben mezclar menos de 10 l del vino caro  \end{solution}


\end{parts}
\end{multicols}

\begin{comment}
\question 
\begin{multicols}{3}
\begin{parts}
\part[]  
\begin{solution} \end{solution}
\end{parts}
\end{multicols}
\end{comment}



\end{questions}

\end{document}


