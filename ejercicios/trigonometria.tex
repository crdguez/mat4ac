\documentclass[spanish, 11pt]{exam}

%These tell TeX which packages to use.
\usepackage{array,epsfig}
\usepackage{amsmath}
\usepackage{amsfonts}
\usepackage{amssymb}
\usepackage{amsxtra}
\usepackage{amsthm}
\usepackage{mathrsfs}
\usepackage{color}
\usepackage{multicol}
\usepackage{verbatim}

\usepackage[utf8]{inputenc}
\usepackage[spanish]{babel}
\usepackage{eurosym}

\usepackage{graphicx}
\graphicspath{{../img/}}


\printanswers
\nopointsinmargin
\pointformat{}

%Pagination stuff.
%\setlength{\topmargin}{-.3 in}
%\setlength{\oddsidemargin}{0in}
%\setlength{\evensidemargin}{0in}
%\setlength{\textheight}{9.in}
%\setlength{\textwidth}{6.5in}
%\pagestyle{empty}

\renewcommand{\solutiontitle}{\noindent\textbf{Sol:}\enspace}

\newcommand{\class}{4º Académicas}
\newcommand{\examdate}{\today}
\newcommand{\examnum}{Semejanza y Trigonometría 1}
\newcommand{\tipo}{A}


\newcommand{\timelimit}{50 minutos}



\pagestyle{head}
\firstpageheader{\includegraphics[width=0.2\columnwidth]{header_left}}{\textbf{Departamento de Matemáticas\linebreak \class}\linebreak \examnum}{\includegraphics[width=0.1\columnwidth]{header_right}}
\runningheader{\class}{\examnum}{Página \thepage\ of \numpages}
\runningheadrule

\begin{document}



\begin{questions}
\question Teorema del cateto y altura:
\begin{multicols}{2}
\begin{parts}
\part[]Calcula la hipotenusa de un triángulo rectángulo, sabiendo que sus catetos miden 156 cm y 65 cm.
\begin{solution} $=\sqrt{156^2-65^2}=169$ \end{solution}
\part[]Halla las longitudes de las proyecciones sobre la hipotenusa de los catetos del triángulo del ejercicio anterior.  
\begin{solution}
%solve(156**2-169*x) 
144 y 25 \end{solution}

\part[]En un triángulo rectángulo, las proyecciones de los catetos sobre la hipotenusa miden 64 m y 225 m
respectivamente. Halla la longitud de los tres lados del triángulo.
\begin{solution} $h=64+225=289 \\ c_1^2=289\cdot64 \to c_1=136 \\
c_2^2=289\cdot 225 \to c_2=255$\end{solution}

\part[]Halla la altura de un trapecio isósceles, sabiendo que sus bases miden 6 m y 16 m y los lados oblicuos 13 m
cada uno de ellos.
\begin{solution} $\to \sqrt{13^2-5^2}=12$ \end{solution}

\part[] En un triángulo rectángulo se conoce un cateto, ($7\sqrt 2 $), y la proyección del otro cateto sobre la
hipotenusa, ($2\sqrt 2 $). Halla la hipotenusa y el otro cateto.
\begin{solution} %solve([(7*sqrt(2))**2-(x+2*sqrt(2))*x, x>0], x) 
$hipotenusa \to 10-\sqrt{2}+2\sqrt{2}=10+\sqrt{2}\\
cateto \to \sqrt{-98 + \left(\sqrt{2} + 10\right)^{2}}= 2 \sqrt{1 + 5 \sqrt{2}}$ \end{solution}

\end{parts}
\end{multicols}







\question Razones trigonométricas en un triángulo rectángulo. En los siguientes ejercicios los lados de un triángulo
rectángulo se representan con las letras a, b y c, siendo siempre a la hipotenusa. Los lados del triángulo se
representan con las letras A, B y C, siendo siempre A el ángulo recto, B el ángulo opuesto a b y C el ángulo
opuesto a c. Usando exclusivamente la definición de las razones trigonométricas involucradas en cada caso,
calcula el lado que se pide:
\begin{multicols}{2}
\begin{parts}
\part[] a = 40 m B = 30º. Hallar b.   
%sin(rad(30))
\begin{solution} $b=40\cdot \sen30=20\ m$\end{solution}
\part[] a = 40 cm B = 30º. Hallar c.  
\begin{solution} $c=40\cdot \cos 30\approx 20 \sqrt{3} \ cm$\end{solution}
\part[] a = 12 dm C = 60º. Hallar b.  
\begin{solution} $b=12\cdot \cos60 = 6 \ dm$\end{solution}
\part[] a = 12 Hm C = 60º. Hallar c. 
\begin{solution} $c=12\cdot \sen60=6 \sqrt{3}\approx10.3923048454133 \ Hm$\end{solution}
\part[] b = 20 mm B = 45º. Hallar c.  
\begin{solution} $c=\dfrac{20}{\tg45} = 20 \ mm $\end{solution}


\end{parts}
\end{multicols}

\question Halla sin calculadora el valor de las siguientes expresiones:
\begin{multicols}{2}
\begin{parts}
\part[] $\dfrac{\sen 60 - \sen 30}{\sen 60 + \sen 30}$   
%(sin(rad(60))-sin(rad(30)))/(sin(rad(60))+sin(rad(30)))
\begin{solution} $- \sqrt{3} + 2\approx0.267949192431123$\end{solution}
\part[] $\dfrac{\cos 45 - \sen 30}{\sen 45 + \cos 60}$     
\begin{solution} $- 2 \sqrt{2} + 3\approx0.17157287525381$\end{solution}
\part[] $\dfrac{\cos 30 - \cos 60}{\tg 60 - \tg 30}$      
\begin{solution} $- \frac{\sqrt{3}}{4} + \frac{3}{4}\approx0.316987298107781$\end{solution}
\part[] $\dfrac{\sen 45 \cdot \cos 30 - \sen 30 \cdot \cos 45}{\cos 45 + \cos 30}$ 
\begin{solution} $\frac{\sqrt{2} + \sqrt{6}}{2 \sqrt{2} + 2 \sqrt{3}}\approx0.614014407382354$\end{solution}


\end{parts}
\end{multicols}


\question Resuelve los siguientes problemas:
\begin{multicols}{2}
\begin{parts}
\part[] Halla la altura de una antena de radio si su sombra mide 100 m cuando los rayos del Sol forman un ángulo de 30º con la horizontal
\begin{solution} $\tg30=\frac{a}{100}\to a= 100\cdot\tg30 = \frac{100 \sqrt{3}}{3}\approx57.7350269189626 \ m$ \end{solution}
\part[] Averigua la distancia a la que se encuentra un castillo que está situado en la orilla opuesta de un río, sabiendo
que la torre más alta del mismo se ve desde nuestra orilla bajo un ángulo de 40º y alejándonos 100 m del río el
ángulo es de 25º.
% solve([tan(rad(40))-y/x,tan(rad(25))-y/(x+100)],[x,y])
\begin{solution} $\left. \begin{gathered}
	  \tg 40 = \frac{y}{x} \hfill \\
	  \tg 25 = \frac{y}{x+100} \hfill \\ 
	\end{gathered}  \right\} \to \\ \begin{Bmatrix}x : \frac{100 \tan{\left (\frac{5 \pi}{36} \right )}}{- \tan{\left (\frac{5 \pi}{36} \right )} + \tan{\left (\frac{2 \pi}{9} \right )}}, & y : \frac{100 \tan{\left (\frac{5 \pi}{36} \right )}}{- \frac{\tan{\left (\frac{5 \pi}{36} \right )}}{\tan{\left (\frac{2 \pi}{9} \right )}} + 1}\end{Bmatrix}$ \end{solution}
\part[]  ¿Calcula el área de un decágono regular de 5 cm de lado.
\begin{solution}$ apotema = \frac{2.5}{\tg18}\approx7.69420884293813 cm \\ area=\dfrac{10\cdot5\cdot\frac{2.5}{\tg18}}{2}\approx192.355221073453 cm^2 $ \end{solution}

\part[]  En una circunferencia de 7 cm de radio trazamos una cuerda de 9 cm. ¿Cuánto mide el ángulo central que
abarca dicha cuerda?
%deg (2 *asin (4.5 /7 ))
\begin{solution}$2\cdot\arcsen(\frac{4.5}{7})\approx80.0104017697205 º$ \end{solution}

\part[]  Halla los ángulos de un triángulo isósceles cuya base mide 50 cm y los lados iguales 40 cm cada uno.
\begin{solution} $ \alpha = \arccos \frac{5}{8}\approx $ y \\ $ \beta = 2\cdot (90-\arccos \frac{5}{8}) $\end{solution}

\part[]  Si vemos una chimenea bajo un ángulo de 30º, ¿bajo qué ángulo la veríamos si la distancia a la que nos
encontramos de la misma fuese el doble? ¿Y si fuese el triple?
\begin{solution} \end{solution}


\end{parts}
\end{multicols}

\begin{comment}
\question 
\begin{multicols}{3}
\begin{parts}
\part[]  
\begin{solution} \end{solution}
\end{parts}
\end{multicols}
\end{comment}



\end{questions}

\end{document}


